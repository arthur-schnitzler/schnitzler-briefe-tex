%% latex-leseansicht-vorspann.tex
%% Vorspann für die Leseansicht.
%% Lädt die gemeinsame Datei latex-vorspann.tex mit nicht gesetztem Schalter.

\newif\ifkorrekturansicht
\korrekturansichtfalse

\input{../tex-inputs/latex-vorspann}

\begin{center}
            \textcolor{red}{ENTWURF, NICHT FERTIG KORRIGIERT}
                      \end{center}
            
         
         \renewcommand{\erwaehntePersonen}{Personen: Else Berger, M. Laurent, Julius Schlesinger, Emil Schlesinger}
         \renewcommand{\erwaehnteOrte}{Orte: Bad Ischl, Badehotel, Dänemark, Skodsborg}
         \renewcommand{\erwaehnteWerke}{}
               \section[Felix Salten u. a. an Arthur Schnitzler, 6. 8. 1896]{ Felix Salten u. a. an Arthur Schnitzler, 6. 8. 1896}\nopagebreak\mylabel{v}\rehead{ }\begin{ledgroupsized}[t]{13cm}\normalsize\beginnumbering \toendnotes[C]{\smallbreak\pagebreak[2]} \Standort{CUL, Schnitzler, B 89, A 1.}
\physDesc{Postkarte, 861 Zeichen
\newline{}Handschrift Felix Salten: schwarze Tinte, lateinische Kurrent\newline{}Handschrift Margherita Schlesinger: schwarze Tinte\newline{}Handschrift Leonhard Fanto: schwarze Tinte\newline{}Handschrift Richard Eisler: schwarze Tinte\newline{}Handschrift Franziska Schlesinger: schwarze Tinte\newline{}Handschrift Alfred Schlesinger: schwarze Tinte, lateinische Kurrent\newline{}Handschrift Else Berger: schwarze Tinte, lateinische Kurrent\newline{}Handschrift Julius Schlesinger: schwarze Tinte\newline{}Handschrift M. Laurent: schwarze Tinte
\newline{}Versand: Stempel: »\nobreak{}\oindex{Bad Ischl@\textbf{Bad Ischl}|pwk}Ischl, 7. 8. 96, 10–11V\nobreak{}«.  
\newline{}Ordnung: mit Bleistift von unbekannter Hand nummeriert:
                                    »76« }\pstart{}{\pb}Herrn D\textsuperscript{r} Arthur Schnitzler\pend{}\pstart{}Skodsborg\oindex{Skodsborg@\textbf{Skodsborg}|pw}\pend{}\pstart{}Dänemark\oindex{Daenemark@\textbf{Dänemark}|pw}\pend{}\pstart{}Badehôtel\oindex{Badehotel@\textbf{Badehotel}|pw}\pend{}{\bigskip}\pstart
           \raggedleft{}{\pb}Ischl\oindex{Bad Ischl@\textbf{Bad Ischl}|pw}, 6. August 96. \pend
           \pstart
           Man soupiert nämlich heute Abend bei Schlesinger\pwindex{Schlesinger, Franziska 17.08.1851 – 11.08.1932@\textsc{Schlesinger, Franziska} (17.08.1851 – 11.08.1932)|pw}\pwindex{Schlesinger, Emil 10.05.1844 – 31.05.1899@\textsc{Schlesinger, Emil} (10.05.1844 – 31.05.1899), \emph{Bankdirektor}|pw}. Es war Kalbsbraten da, und über den Weg »Schnitzl« kam ein
               Toast auf Sie zustande. Die Consequenz dieses lobenden Gefühlsausbruches ist
               »vorliegende« Karte, welche Ihnen Grüße von nachstehenden Persönlichkeiten
               übermittelt:\pend
           \pstart \spacefill\mbox{{[}hs. Margherita Schlesinger:{]} Therese Schlesinger}{ }\spacefill\mbox{{[}hs. Julius Schlesinger:{]} Julius Schlesinger}\spacefill\mbox{{[}hs. Fanto:{]} Fanto}, \spacefill\mbox{{[}hs. Eisler:{]} D\textsuperscript{r} R. Eisler}, \spacefill\mbox{{[}hs. Laurent:{]} M. Laurent}\spacefill\mbox{Gretl Schlesinger,}\spacefill\mbox{{[}hs. Franziska Schlesinger:{]} Fanny Schlesinger}\pend{}\pstart
           {[}hs. Berger:{]} Trotzdem Herr \introOben{}will\introOben{} Salten mir
               absolut nicht erlauben mehr als meinen Namen zu schreiben benütze ich die gute
               Gelegenheit Ihnen viele herzliche Grüße zu senden.\pend
           \pstart
           Herzlich und freundschaftlich, Ihre \spacefill\mbox{Else.}\pend
           \pstart
           {[}hs. Alfred Schlesinger:{]} Med. Dr. \introOben{}in spe\introOben{} Alfred
               Schlesinger grüßt den zukünftigen Herrn Collegen bestens nachdem er seine Matura
               glücklich überstanden.\pend
           \pstart
           \noindent{}{[}hs. Berger:{]} Fanny fragt warum »nachstehend« nicht unter
                  Anführungszeichen steht. Bitte, erklären \uuline{Sie} ihr
                  das!! \spacefill\mbox{E}\pend
           
         
         \endnumbering\mylabel{h}\end{ledgroupsized}\begin{anhang}\end{anhang}\newcommand{\dateiname}{L03177}\newcommand{\titel}{Felix Salten u. a. an Arthur Schnitzler, 6. 8. 1896}\newcommand{\editorInnen}{Martin Anton Müller und Laura Untner}%% latex-leseansicht-abspann.tex
%% Abspann für die Leseansicht.
%% Der Schalter \ifkorrekturansicht ist bereits durch den Vorspann gesetzt.

%% latex-abspann.tex
%% Gemeinsamer Abspann für Korrekturansicht und Leseansicht.
%% Setzt den Schalter \ifkorrekturansicht voraus (gesetzt in den
%% einbindenden Dateien latex-korrekturansicht-abspann.tex bzw.
%% latex-leseansicht-abspann.tex).
%% ---------------------------------------------------------------

\normalsize

% Das esempio-Environment wird nur in der Leseansicht benötigt
\ifkorrekturansicht\else
\newenvironment{esempio}[3]%
{
    \vspace{1.5ex}
    \rlap{\underline{#1}}
    \par
    \setlength{\parindent}{0cm}
    \nopagebreak
    \leftskip=#2cm
    \rightskip=#3cm
}
{
    \par
}
\fi

\doendnotes{C}
\bigskip
\vfill

\clearpage

\footnotesize

\ifkorrekturansicht
  \lohead{\textsc{register}}
\fi

% theindex-Environment neu definieren ohne reledmac
\makeatletter
\renewenvironment{theindex}{%
  \ifkorrekturansicht
    \section*{\indexname}%
  \else
    \subsubsection*{Index der erwähnten Entitäten}%
  \fi
  \setlength{\parindent}{0pt}%
  \setlength{\parskip}{0pt plus 0.3pt}%
  \let\item\@idxitem
}{%
  \ifkorrekturansicht\clearpage\fi
}
\makeatother

\IfFileExists{\jobname-pw.ind}{\input{\jobname-pw.ind}}{}

% Quellenangabe nur in der Leseansicht
\ifkorrekturansicht\else
% Fallback-Definitionen, falls die .tex-Datei \titel etc. nicht gesetzt hat
\providecommand{\titel}{}
\providecommand{\editorInnen}{}
\providecommand{\dateiname}{\jobname}

\vspace{3cm}

\vfill

\footnotesize
\textsc{Quelle}: \titel. Herausgegeben von {\editorInnen}. In: \emph{Arthur Schnitzler: Briefwechsel mit Autorinnen und Autoren}.
 Digitale Edition, https://schnitzler-briefe.acdh.oeaw.ac.at/{\dateiname}.html (Stand \today)
\fi

\end{document}


      