%% latex-korrekturansicht-vorspann.tex
%% Vorspann für die Korrekturansicht.
%% Lädt die gemeinsame Datei latex-vorspann.tex mit gesetztem Schalter.

\newif\ifkorrekturansicht
\korrekturansichttrue

\input{../tex-inputs/latex-vorspann}


\section[Arthur Schnitzler an Richard Beer-Hofmann, 28. 11. 1904]{L01472 Arthur Schnitzler an Richard Beer-Hofmann, 28. 11. 1904}
\nopagebreak\mylabel{L01472v}
\rehead{ }\normalsize\beginnumbering\briefempfaengerindex{Beer-Hofmann, Richard@\textsc{Beer-Hofmann, Richard}!zzzSchnitzler, Arthur@\emph{von Arthur Schnitzler}!1904-11-281@{28. 11. 1904}|(be}
\toendnotes[C]{\smallbreak\pagebreak[2]}\Standort{YCGL, MSS 31.}
\physDesc{Brief, 1 Blatt, 3 Seiten, Umschlag, 810 Zeichen
\newline{}Handschrift: 1) Bleistift, deutsche Kurrent\hspace{1em}2) Bleistift, lateinische Kurrent (\noindent{}Adresse)\hspace{1em}
\newline{}Versand: 1) Stempel: »\nobreak{}5\nobreak{}«.   2) Stempel: »\nobreak{}{\pb}Bestellt vom
                                          {[}Po{]}stamte 6\nobreak{}«. }
\buchAbdrucke{\weitereDrucke{Arthur Schnitzler, Richard Beer-Hofmann: \emph{Briefwechsel 1891–1931}. Wien, Zürich: \emph{Europaverlag} 1992, S. 170–171.} }\toendnotes[C]{\smallbreak}\pstart{}{\pb}Herrn Dr. Richard\pend{}\pstart{}Beer-Hofmann\pend{}\pstart{}Berlin\oindex{Berlin@\textbf{Berlin}, \emph{P.PPLC}|pw}\pend{}\pstart{}Hotel Bristol\oindex{Hotel Bristol Berlin@\textbf{Hotel Bristol Berlin}, \emph{Hotel (K.HTL)}|pw}\pend{}{\bigskip}\vspace{1em}
\pstart
           \raggedleft{}{\pb}Wien\oindex{Wien@\textbf{Wien}, \emph{A.ADM2}|pw}, 28. 11. 904\pend
           
\pstart{}lieber Richard,\pend\vspace{0.5em}
\pstart
           ich bitte Sie ſehr Reinhardt\pwindex{Reinhardt, Max 09.09.1873 – 30.10.1943@\textsc{Reinhardt, Max} (09.09.1873 – 30.10.1943), \emph{Theaterleiter/Theaterleiterin, Regisseur/Regisseurin, Schauspieler/Schauspielerin}|pw} nochmals in
               meinem Namen dringend zu erſuchen, er möge, ob nun \textsc{Delorme}\pwindex{Haus Delorme. Eine Familienszene@\emph{Das Haus Delorme. Eine Familienszene}|pw} freigegeben oder ob es definitiv verboten wird, \uline{abſolut nichts} in die Zeitung geben und überhaupt \uline{nichts verfügen}, ohne ſich vorher mit mir in Verbin{\pb}dung zu ſetzen. –\pend
           
\pstart
           Gern würde ich Ihre Meinung wiſſen, ob Sie es nicht auch für opportun hielten, ſelbſt
               im Fall eines Erlaubtwerdens, die \strikeout{\textcolor{gray}{Geſchichte}} ev. Aufführung\pwindex{Haus Delorme. Eine Familienszene@\emph{Das Haus Delorme. Eine Familienszene}|pwv}
               hinauszuſchieben. An dieſer Überfracht von unfreiwilliger Reclame und geſpannten
               Erwartungen müsste meiner {\pb}Empfindung nach auch ein
               ſtärkeres Stück zu Grunde gehen.\pend
           
\pstart
           Theilen Sie mir mit wie es Ihnen und Ihren Proben\pwindex{Graf von Charolais. Ein Trauerspiel@\emph{Der Graf von Charolais. Ein Trauerspiel}|pwv} geht, grüßen Sie mit mehrerem oder minderem \textsc{Empressement}.\pend
           
\pstart
           Alles gute an \textsc{Reinhardt}\pwindex{Reinhardt, Max 09.09.1873 – 30.10.1943@\textsc{Reinhardt, Max} (09.09.1873 – 30.10.1943), \emph{Theaterleiter/Theaterleiterin, Regisseur/Regisseurin, Schauspieler/Schauspielerin}|pw} u noch etwas mehr an Sie.\pend
           
\pstart
           Herzlichst Ihr{\\[\baselineskip]}\spacefill\mbox{A.}\pend
           \leftskip=0em{}\selectlanguage{ngerman}\endnumbering\briefempfaengerindex{Beer-Hofmann, Richard@\textsc{Beer-Hofmann, Richard}!zzzSchnitzler, Arthur@\emph{von Arthur Schnitzler}!1904-11-281@{28. 11. 1904}|)be}\mylabel{L01472h}  \normalsize

\doendnotes{C}
\bigskip
\vfill

\clearpage

\footnotesize

\lohead{\textsc{register}}

% Definiere theindex-Environment komplett neu ohne reledmac
\makeatletter
\renewenvironment{theindex}{%
  \section*{\indexname}%
  \setlength{\parindent}{0pt}%
  \setlength{\parskip}{0pt plus 0.3pt}%
  \let\item\@idxitem
}{%
  \clearpage
}
\makeatother

\IfFileExists{\jobname-pw.ind}{\input{\jobname-pw.ind}}{}

\end{document}

      