%% latex-leseansicht-vorspann.tex
%% Vorspann für die Leseansicht.
%% Lädt die gemeinsame Datei latex-vorspann.tex mit nicht gesetztem Schalter.

\newif\ifkorrekturansicht
\korrekturansichtfalse

\input{../tex-inputs/latex-vorspann}


\section[Arthur Schnitzler an Richard Beer-Hofmann, 28. 11. 1904]{L01472 Arthur Schnitzler an Richard Beer-Hofmann, 28. 11. 1904}
\nopagebreak\mylabel{L01472v}
\rehead{ }\normalsize\beginnumbering\briefempfaengerindex{Beer-Hofmann, Richard@\textsc{Beer-Hofmann, Richard}!zzzSchnitzler, Arthur@\emph{von Arthur Schnitzler}!1904-11-281@{28. 11. 1904}|(be}
\toendnotes[C]{\smallbreak\pagebreak[2]}
\correspDesc{Versand  durch Arthur Schnitzler am 28. 11. 1904 in Wien
\newline{}Erhalt  durch Richard Beer-Hofmann im Zeitraum [29. 11. 1904 – 3. 12. 1904?] in Berlin}\toendnotes[C]{\smallbreak}
\Standort{YCGL, MSS 31.}
\physDesc{Brief, 1 Blatt, 3 Seiten, Kuvert, 810 Zeichen
\newline{}Handschrift: Bleistift, deutsche Kurrent
\newline{}Versand: 1) Stempel: »\nobreak{}5\nobreak{}«.   2) Stempel: »\nobreak{}{\pb}Bestellt vom
                                          {[}Po{]}stamte 6\nobreak{}«. }
\buchAbdrucke{\weitereDrucke{Arthur Schnitzler, Richard Beer-Hofmann: \emph{Briefwechsel 1891–1931}. Herausgegeben von Konstanze Fliedl. Wien, Zürich: \emph{Europaverlag} 1992, S. 170–171.} }\toendnotes[C]{\smallbreak}\pstart{}\textsc{{\pb}Herrn Dr. Richard}\pend{}\pstart{}\textsc{Beer-Hofmann}\pend{}\pstart{}\textsc{Berlin\oindex{Berlin@\textbf{Berlin}, \emph{Hauptstadt}|pw}}\pend{}\pstart{}\textsc{Hotel Bristol\oindex{Hotel Bristol Berlin@\textbf{Hotel Bristol Berlin}, \emph{Hotel}|pw}}\pend{}{\bigskip}\vspace{1em}
\pstart
           \raggedleft{}{\pb}Wien\oindex{Wien@\textbf{Wien}, \emph{Verwaltungsgebiet}|pw}, 28. 11. 904\pend
           
\pstart{}lieber Richard,\pend\vspace{0.5em}
\pstart
           ich bitte Sie{ }ſehr Reinhardt\pwindex{Reinhardt, Max 9.\,9.\,1873 Baden bei Wien – 30.\,10.\,1943 New York City@\textsc{Reinhardt, Max} (9.\,9.\,1873 Baden bei Wien – 30.\,10.\,1943 New York City), \emph{Theaterleiter, Regisseur, Schauspieler}|pw} nochmals in
               meinem Namen dringend zu erſuchen, er möge, ob nun \textsc{Delorme}\pwindex{Schnitzler, Arthur 15.\,5.\,1862 Wien – 21.\,10.\,1931 ebd.@\textsc{Schnitzler, Arthur} (15.\,5.\,1862 Wien – 21.\,10.\,1931 ebd.), \emph{Schriftsteller, Mediziner}!Haus Delorme. Eine Familienszene@\strich\emph{Das Haus Delorme. Eine Familienszene}|pw} freigegeben oder ob es definitiv verboten wird, \uline{abſolut nichts} in die Zeitung geben und überhaupt \uline{nichts verfügen}, ohne{ }ſich vorher mit mir in Verbin{\pb}dung zu{ }ſetzen. –\pend
           
\pstart
           Gern würde ich Ihre Meinung wiſſen, ob Sie es nicht auch für opportun hielten,{ }ſelbſt
               im Fall eines Erlaubtwerdens, die \strikeout{\textcolor{gray}{Geſchichte}} ev. Aufführung\pwindex{Schnitzler, Arthur 15.\,5.\,1862 Wien – 21.\,10.\,1931 ebd.@\textsc{Schnitzler, Arthur} (15.\,5.\,1862 Wien – 21.\,10.\,1931 ebd.), \emph{Schriftsteller, Mediziner}!Haus Delorme. Eine Familienszene@\strich\emph{Das Haus Delorme. Eine Familienszene}|pwv}
               hinauszuſchieben. An dieſer Überfracht von unfreiwilliger Reclame und geſpannten
               Erwartungen müsste meiner {\pb}Empfindung nach auch ein{ }ſtärkeres Stück zu Grunde gehen.\pend
           
\pstart
           Theilen Sie mir mit wie es Ihnen und Ihren Proben\pwindex{Beer-Hofmann, Richard 11.\,7.\,1866 Wien – 26.\,9.\,1945 New York City@\textsc{Beer-Hofmann, Richard} (11.\,7.\,1866 Wien – 26.\,9.\,1945 New York City), \emph{Schriftsteller}!Graf von Charolais. Ein Trauerspiel@\strich\emph{Der Graf von Charolais. Ein Trauerspiel}|pwv} geht, grüßen Sie mit mehrerem oder minderem \textsc{Empressement}.\pend
           
\pstart
           Alles gute an \textsc{Reinhardt}\pwindex{Reinhardt, Max 9.\,9.\,1873 Baden bei Wien – 30.\,10.\,1943 New York City@\textsc{Reinhardt, Max} (9.\,9.\,1873 Baden bei Wien – 30.\,10.\,1943 New York City), \emph{Theaterleiter, Regisseur, Schauspieler}|pw} u noch etwas mehr an Sie.\pend
           
\pstart
           Herzlichst Ihr{\\[\baselineskip]}\spacefill\mbox{A.}\pend
           \leftskip=0em{}\selectlanguage{ngerman}\endnumbering\briefempfaengerindex{Beer-Hofmann, Richard@\textsc{Beer-Hofmann, Richard}!zzzSchnitzler, Arthur@\emph{von Arthur Schnitzler}!1904-11-281@{28. 11. 1904}|)be}\mylabel{L01472h}  \newcommand{\dateiname}{L01472}\newcommand{\titel}{Arthur Schnitzler an Richard Beer-Hofmann, 28. 11. 1904}\newcommand{\editorInnen}{Martin Anton Müller und Gerd-Hermann Susen}%% latex-leseansicht-abspann.tex
%% Abspann für die Leseansicht.
%% Der Schalter \ifkorrekturansicht ist bereits durch den Vorspann gesetzt.

%% latex-abspann.tex
%% Gemeinsamer Abspann für Korrekturansicht und Leseansicht.
%% Setzt den Schalter \ifkorrekturansicht voraus (gesetzt in den
%% einbindenden Dateien latex-korrekturansicht-abspann.tex bzw.
%% latex-leseansicht-abspann.tex).
%% ---------------------------------------------------------------

\normalsize

% Das esempio-Environment wird nur in der Leseansicht benötigt
\ifkorrekturansicht\else
\newenvironment{esempio}[3]%
{
    \vspace{1.5ex}
    \rlap{\underline{#1}}
    \par
    \setlength{\parindent}{0cm}
    \nopagebreak
    \leftskip=#2cm
    \rightskip=#3cm
}
{
    \par
}
\fi

\doendnotes{C}
\bigskip
\vfill

\clearpage

\footnotesize

\ifkorrekturansicht
  \lohead{\textsc{register}}
\fi

% theindex-Environment neu definieren ohne reledmac
\makeatletter
\renewenvironment{theindex}{%
  \ifkorrekturansicht
    \section*{\indexname}%
  \else
    \subsubsection*{Index der erwähnten Entitäten}%
  \fi
  \setlength{\parindent}{0pt}%
  \setlength{\parskip}{0pt plus 0.3pt}%
  \let\item\@idxitem
}{%
  \ifkorrekturansicht\clearpage\fi
}
\makeatother

\IfFileExists{\jobname-pw.ind}{\input{\jobname-pw.ind}}{}

% Quellenangabe nur in der Leseansicht
\ifkorrekturansicht\else
% Fallback-Definitionen, falls die .tex-Datei \titel etc. nicht gesetzt hat
\providecommand{\titel}{}
\providecommand{\editorInnen}{}
\providecommand{\dateiname}{\jobname}

\vspace{3cm}

\vfill

\footnotesize
\textsc{Quelle}: \titel. Herausgegeben von {\editorInnen}. In: \emph{Arthur Schnitzler: Briefwechsel mit Autorinnen und Autoren}.
 Digitale Edition, https://schnitzler-briefe.acdh.oeaw.ac.at/{\dateiname}.html (Stand \today)
\fi

\end{document}


