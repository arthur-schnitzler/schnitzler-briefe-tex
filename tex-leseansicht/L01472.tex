\input{../tex-inputs/latex-pdf-vorspann}
\begin{center}
            \textcolor{red}{ENTWURF. ENTZIFFERUNG NOCH NICHT KORREKTURGELESEN}
                      \end{center}
            
               \section[Arthur Schnitzler an Richard Beer-Hofmann, 28. 11. 1904]{ Arthur Schnitzler an Richard Beer-Hofmann, 28. 11. 1904}\nopagebreak\mylabel{v}\rehead{ }\begin{ledgroupsized}[t]{13cm}\normalsize\beginnumbering\briefempfaengerindex{Beer-Hofmann, Richard@\textsc{Beer-Hofmann, Richard}!zzzSchnitzler, Arthur@\emph{von Arthur Schnitzler}!1904-11-281@{28. 11. 1904}|(be} \toendnotes[C]{\smallbreak\pagebreak[2]} \Standort{YCGL, MSS 31.}
\physDesc{Brief, 1 Blatt, 3 Seiten, Umschlag
\newline{}Handschrift: Bleistift, deutsche Kurrent\newline{}Versand: 1) Stempel: »\nobreak{}5\nobreak{}«.  2) Stempel: »\nobreak{}{\pb}Bestellt vom
                                          {[}Po{]}stamte 6\nobreak{}«. }\buchAbdrucke{\weitereDrucke{Arthur Schnitzler, Richard Beer-Hofmann: \emph{Briefwechsel 1891–1931}. Hg. Konstanze Fliedl. Wien, Zürich: \emph{Europaverlag} 1992, S. 170–171.} }\toendnotes[C]{\smallbreak}\pstart{}{\pb}\textsc{Herrn Dr. Richard}\pend{}\pstart{}\textsc{Beer-Hofmann}\pend{}\pstart{}\textsc{Berlin\oindex{Berlin@\textbf{Berlin}|pw}}\pend{}\pstart{}\textsc{Hotel Bristol\oindex{Hotel Bristol@\textbf{Hotel Bristol}|pw}}\pend{}{\bigskip}\pstart
           \raggedleft{}{\pb}Wien\oindex{Wien@\textbf{Wien}|pw}, 28. 11. 904\pend
           \pstart{}lieber Richard,\pend\pstart
           ich bitte Sie ſehr Reinhardt\pwindex{Reinhardt, Max 09.09.1873 – 30.10.1943@\textsc{Reinhardt, Max} (09.09.1873 – 30.10.1943), \emph{Theaterleiter, Regisseur, Schauspieler}|pw} nochmals in meinem
               Namen dringend zu erſuchen, er möge, ob nun \textsc{Delorme}\pwindex{Schnitzler, Arthur 15.05.1862 – 21.10.1931@\textsc{Schnitzler, Arthur} (15.05.1862 – 21.10.1931), \emph{Schriftsteller, Mediziner}!Haus Delorme. Eine Familienszene1977@\strich\emph{Das Haus Delorme. Eine Familienszene} {[}1977{]}|pw} freigegeben oder ob es definitiv verboten wird, \uline{abſolut nichts} in die Zeitung geben und überhaupt \uline{nichts verfügen}, ohne ſich vorher mit mir in Verbin{\pb}dung zu ſetzen. –\pend
           \pstart
           Gern würde ich Ihre Meinung wiſſen, ob Sie es nicht auch für opportun hielten, ſelbſt
               im Fall eines Erlaubtwerdens, die \strikeout{\textcolor{gray}{Geſchichte}} ev. Aufführung\pwindex{Schnitzler, Arthur 15.05.1862 – 21.10.1931@\textsc{Schnitzler, Arthur} (15.05.1862 – 21.10.1931), \emph{Schriftsteller, Mediziner}!Haus Delorme. Eine Familienszene1977@\strich\emph{Das Haus Delorme. Eine Familienszene} {[}1977{]}|pwv}
               hinauszuſchieben. An dieſer Überfracht von unfreiwilliger Reclame und geſpannten
               Erwartungen müsste meiner {\pb}Empfindung nach auch ein
               ſtärkeres Stück zu Grunde gehen.\pend
           \pstart
           Theilen Sie mir mit wie es Ihnen und Ihren Proben\pwindex{Beer-Hofmann, Richard 11.07.1866 – 26.09.1945@\textsc{Beer-Hofmann, Richard} (11.07.1866 – 26.09.1945), \emph{Schriftsteller}!Graf von Charolais. Ein Trauerspiel1904-12-23 – 1904-12-23@\strich\emph{Der Graf von Charolais. Ein Trauerspiel} {[}1904-12-23 – 1904-12-23{]}|pwv} geht, grüßen Sie mit mehrerem oder minderem \textsc{Empressement}.\pend
           \pstart
           Alles gute an \textsc{Reinhardt}\pwindex{Reinhardt, Max 09.09.1873 – 30.10.1943@\textsc{Reinhardt, Max} (09.09.1873 – 30.10.1943), \emph{Theaterleiter, Regisseur, Schauspieler}|pw} u noch etwas mehr an Sie.\pend
           \pstart
           Herzlichst Ihr{\\[\baselineskip]}\spacefill\mbox{A.}\pend
           \leftskip=0em{}\endnumbering\briefempfaengerindex{Beer-Hofmann, Richard@\textsc{Beer-Hofmann, Richard}!zzzSchnitzler, Arthur@\emph{von Arthur Schnitzler}!1904-11-281@{28. 11. 1904}|)be}\mylabel{h}\end{ledgroupsized}  \newcommand{\dateiname}{L01472}\newcommand{\titel}{Arthur Schnitzler an Richard Beer-Hofmann, 28. 11. 1904}\newcommand{\editorInnen}{Martin Anton Müller und Gerd-Hermann Susen}\input{../tex-inputs/latex-pdf-abspann}
      