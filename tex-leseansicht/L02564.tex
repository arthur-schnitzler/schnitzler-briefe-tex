%% latex-leseansicht-vorspann.tex
%% Vorspann für die Leseansicht.
%% Lädt die gemeinsame Datei latex-vorspann.tex mit nicht gesetztem Schalter.

\newif\ifkorrekturansicht
\korrekturansichtfalse

\input{../tex-inputs/latex-vorspann}


         
         \newcommand{\erwaehntePersonen}{Personen: Richard Beer-Hofmann, Paula Beer-Hofmann, Artur Fleischer, Erna Fleischer}
         \newcommand{\erwaehnteInstitutionen}{}
         \newcommand{\erwaehnteOrte}{Orte: Türkenschanzpark, Wien}
         \newcommand{\erwaehnteWerke}{
               \section[Olga Schnitzler an Richard Beer-Hofmann, {[}17. 1. 1909?{]}]{ Olga Schnitzler an Richard Beer-Hofmann, {[}17. 1. 1909?{]}}\nopagebreak\mylabel{v}\rehead{ }\begin{ledgroupsized}[t]{13cm}\normalsize\beginnumbering \toendnotes[C]{\smallbreak\pagebreak[2]} \Standort{YCGL, MSS 31.}
\physDesc{Briefkarte, Umschlag
\newline{}Handschrift: schwarze Tinte, lateinische Kurrent\newline{}Versand: ohne postalischen Übermittlungsvermerk }\toendnotes[C]{\smallbreak}\pstart{}{\pb}Herrn D\textsuperscript{r} Richard
                  Beer-Hofmann\pend{}{\bigskip}\pstart
           \noindent{}{\pb}Lieber Herr Doctor, ich danke sehr für
               die schönen \label{K_L02564-1v}\edtext{Blumen}{\lemma{\textnormal{\emph{Blumen}}}\Cendnote{\textnormal{Die Karte ist undatiert, aber unter den
                  Korrespondenzstücken des Jahres 1909 überliefert. Olga Schnitzler\pwindex{Schnitzler, Olga 17.01.1882 – 13.01.1970@\textsc{Schnitzler, Olga} (17.01.1882 – 13.01.1970), \emph{Schauspielerin, Sängerin}|pwk} feierte am 17. 1. 1909 ihren 27. Geburtstag. Auch der in
                  Folge angesprochene, mögliche Besuch von Artur\pwindex{Fleischer, Artur 14.12.1881 – 11.4.1948@\textsc{Fleischer, Artur} (14.12.1881 – 11.4.1948), \emph{Sänger}|pwk}
                  und Erna Fleischer\pwindex{Fleischer, Erna 23.01.1885 – 11.04.1982@\textsc{Fleischer, Erna} (23.01.1885 – 11.04.1982)|pwk} stützt die zeitliche
                  Datierung, insofern das Ehepaar Schnitzler\pwindex{Schnitzler, Arthur 15.05.1862 – 21.10.1931@\textsc{Schnitzler, Arthur} (15.05.1862 – 21.10.1931), \emph{Schriftsteller, Mediziner}|pwk}\pwindex{Schnitzler, Olga 17.01.1882 – 13.01.1970@\textsc{Schnitzler, Olga} (17.01.1882 – 13.01.1970), \emph{Schauspielerin, Sängerin}|pwk} nur wenige Tage zuvor, am 12. 1. 1909, bei diesen zuhause gewesen war. 
                  Ein Gegenbesuch ist also wahrscheinlicher als in den anderen Monaten, in denen der
                  Kontakt unterbrochen gewesen sein dürfte.}}}\label{K_L02564-1h}, ich habe mich sehr damit
               gefreut.\pend
           \pstart
           Fleischer\pwindex{Fleischer, Artur 14.12.1881 – 11.4.1948@\textsc{Fleischer, Artur} (14.12.1881 – 11.4.1948), \emph{Sänger}|pw}\pwindex{Fleischer, Erna 23.01.1885 – 11.04.1982@\textsc{Fleischer, Erna} (23.01.1885 – 11.04.1982)|pw}s kommen heute nicht, wir bleiben
               zuhause und würden uns sehr freuen, wenn Ihr Beide\pwindex{Beer-Hofmann, Paula 25.02.1879 – 30.10.1939@\textsc{Beer-Hofmann, Paula} (25.02.1879 – 30.10.1939)|pwv} mit uns nachtmahlen wolltet. Backhaendel sind {\pb}auf jeden Fall versorgt. (Mit Fleischer\pwindex{Fleischer, Artur 14.12.1881 – 11.4.1948@\textsc{Fleischer, Artur} (14.12.1881 – 11.4.1948), \emph{Sänger}|pw}\pwindex{Fleischer, Erna 23.01.1885 – 11.04.1982@\textsc{Fleischer, Erna} (23.01.1885 – 11.04.1982)|pw}s wollten wir im Türkenschanzpark\oindex{Tuerkenschanzpark@\textbf{Türkenschanzpark}|pw} essen – ma soll sehen, die Backhaendel sind Euch zuliebe
               da.)\pend
           \pstart
           Herzliche Grüsse!{\\[\baselineskip]}\spacefill\mbox{OlgaS.}\pend
           \leftskip=0em{}
         
         \endnumbering\mylabel{h}\end{ledgroupsized}  \newcommand{\dateiname}{L02564}\newcommand{\titel}{Olga Schnitzler an Richard Beer-Hofmann, [17. 1. 1909?]}\newcommand{\editorInnen}{Martin Anton Müller und Gerd-Hermann Susen}%% latex-leseansicht-abspann.tex
%% Abspann für die Leseansicht.
%% Der Schalter \ifkorrekturansicht ist bereits durch den Vorspann gesetzt.

%% latex-abspann.tex
%% Gemeinsamer Abspann für Korrekturansicht und Leseansicht.
%% Setzt den Schalter \ifkorrekturansicht voraus (gesetzt in den
%% einbindenden Dateien latex-korrekturansicht-abspann.tex bzw.
%% latex-leseansicht-abspann.tex).
%% ---------------------------------------------------------------

\normalsize

% Das esempio-Environment wird nur in der Leseansicht benötigt
\ifkorrekturansicht\else
\newenvironment{esempio}[3]%
{
    \vspace{1.5ex}
    \rlap{\underline{#1}}
    \par
    \setlength{\parindent}{0cm}
    \nopagebreak
    \leftskip=#2cm
    \rightskip=#3cm
}
{
    \par
}
\fi

\doendnotes{C}
\bigskip
\vfill

\clearpage

\footnotesize

\ifkorrekturansicht
  \lohead{\textsc{register}}
\fi

% theindex-Environment neu definieren ohne reledmac
\makeatletter
\renewenvironment{theindex}{%
  \ifkorrekturansicht
    \section*{\indexname}%
  \else
    \subsubsection*{Index der erwähnten Entitäten}%
  \fi
  \setlength{\parindent}{0pt}%
  \setlength{\parskip}{0pt plus 0.3pt}%
  \let\item\@idxitem
}{%
  \ifkorrekturansicht\clearpage\fi
}
\makeatother

\IfFileExists{\jobname-pw.ind}{\input{\jobname-pw.ind}}{}

% Quellenangabe nur in der Leseansicht
\ifkorrekturansicht\else
% Fallback-Definitionen, falls die .tex-Datei \titel etc. nicht gesetzt hat
\providecommand{\titel}{}
\providecommand{\editorInnen}{}
\providecommand{\dateiname}{\jobname}

\vspace{3cm}

\vfill

\footnotesize
\textsc{Quelle}: \titel. Herausgegeben von {\editorInnen}. In: \emph{Arthur Schnitzler: Briefwechsel mit Autorinnen und Autoren}.
 Digitale Edition, https://schnitzler-briefe.acdh.oeaw.ac.at/{\dateiname}.html (Stand \today)
\fi

\end{document}


      