%% latex-leseansicht-vorspann.tex
%% Vorspann für die Leseansicht.
%% Lädt die gemeinsame Datei latex-vorspann.tex mit nicht gesetztem Schalter.

\newif\ifkorrekturansicht
\korrekturansichtfalse

\input{../tex-inputs/latex-vorspann}


\section[Olga Schnitzler an Richard Beer-Hofmann, {[}17. 1. 1909?{]}]{L02564 Olga Schnitzler an Richard Beer-Hofmann, {[}17. 1. 1909?{]}}
\nopagebreak\mylabel{L02564v}
\rehead{ }\normalsize\beginnumbering\briefempfaengerindex{Beer-Hofmann, Richard@\textsc{Beer-Hofmann, Richard}!zzzSchnitzler, Olga@\emph{von Olga Schnitzler}!1909-01-171@{{[}17. 1. 1909?{]}}|(be}
\toendnotes[C]{\smallbreak\pagebreak[2]}
\correspDesc{Versand  durch Olga Schnitzler am [17. 1. 1909?] in Wien
\newline{}Erhalt  durch Richard Beer-Hofmann am [17. 1. 1909?] in Wien}\toendnotes[C]{\smallbreak}
\Standort{YCGL, MSS 31.}
\physDesc{Briefkarte, , Kuvert, 409 Zeichen
\newline{}Handschrift: schwarze Tinte, lateinische Kurrent
\newline{}Versand: ohne postalischen Übermittlungsvermerk }\toendnotes[C]{\smallbreak}\pstart{}{\pb}Herrn D\textsuperscript{r} Richard
                  Beer-Hofmann\pend{}{\bigskip}\vspace{1em}
\pstart
           \noindent{}{\pb}Lieber Herr Doctor, ich danke sehr für
               die schönen \label{K_L02564-1v}\edtext{Blumen}{\lemma{\textnormal{\emph{Blumen}}}\Cendnote{\textnormal{Die Karte ist undatiert, aber unter den
                  Korrespondenzstücken des Jahres 1909 überliefert. Olga Schnitzler\pwindex{Schnitzler, Olga 17.\,1.\,1882 Wien – 13.\,1.\,1970 Lugano@\textsc{Schnitzler, Olga} (17.\,1.\,1882 Wien – 13.\,1.\,1970 Lugano), \emph{Schauspielerin, Sängerin}|pwk} feierte am 17. 1. 1909 ihren 27. Geburtstag. Auch der
                  in Folge angesprochene mögliche Besuch von Artur\pwindex{Fleischer, Artur 14.\,12.\,1881 Wien – 11.\,4.\,1948 San Francisco@\textsc{Fleischer, Artur} (14.\,12.\,1881 Wien – 11.\,4.\,1948 San Francisco), \emph{Sänger}|pwk} und Erna Fleischer\pwindex{Fleischer, Erna 23.\,1.\,1885 Wien – 11.\,4.\,1982 New York City@\textsc{Fleischer, Erna} (23.\,1.\,1885 Wien – 11.\,4.\,1982 New York City)|pwk} stützt
                  die zeitliche Datierung, insofern das Ehepaar Schnitzler\pwindex{Schnitzler, Olga 17.\,1.\,1882 Wien – 13.\,1.\,1970 Lugano@\textsc{Schnitzler, Olga} (17.\,1.\,1882 Wien – 13.\,1.\,1970 Lugano), \emph{Schauspielerin, Sängerin}|pwk} nur wenige Tage zuvor, am 12. 1. 1909, bei
                  diesen zu Hause gewesen war. Ein Gegenbesuch ist also wahrscheinlicher als in den
                  anderen Monaten, in denen der Kontakt unterbrochen gewesen sein dürfte.}}}\label{K_L02564-1},
               ich habe mich sehr damit gefreut.\pend
           
\pstart
           Fleischers\pwindex{Fleischer, Artur 14.\,12.\,1881 Wien – 11.\,4.\,1948 San Francisco@\textsc{Fleischer, Artur} (14.\,12.\,1881 Wien – 11.\,4.\,1948 San Francisco), \emph{Sänger}|pw}\pwindex{Fleischer, Erna 23.\,1.\,1885 Wien – 11.\,4.\,1982 New York City@\textsc{Fleischer, Erna} (23.\,1.\,1885 Wien – 11.\,4.\,1982 New York City)|pw} kommen heute nicht, wir
               bleiben zuhause und würden uns sehr freuen, wenn Ihr Beide\pwindex{Beer-Hofmann, Paula 25.\,2.\,1879 Wien – 30.\,10.\,1939 Zürich@\textsc{Beer-Hofmann, Paula} (25.\,2.\,1879 Wien – 30.\,10.\,1939 Zürich)|pwv} mit uns nachtmahlen wolltet.
               Backhaendel sind {\pb}auf jeden Fall versorgt. (Mit Fleischers\pwindex{Fleischer, Artur 14.\,12.\,1881 Wien – 11.\,4.\,1948 San Francisco@\textsc{Fleischer, Artur} (14.\,12.\,1881 Wien – 11.\,4.\,1948 San Francisco), \emph{Sänger}|pw}\pwindex{Fleischer, Erna 23.\,1.\,1885 Wien – 11.\,4.\,1982 New York City@\textsc{Fleischer, Erna} (23.\,1.\,1885 Wien – 11.\,4.\,1982 New York City)|pw} wollten wir im Türkenschanzpark\oindex{Wien@\textbf{Wien}!XVIII., Währing@\textbf{XVIII., Währing}!Türkenschanzpark@\textbf{Türkenschanzpark}, \emph{Park}|pw} essen – ma soll sehen, die
               Backhaendel sind Euch zuliebe da.)\pend
           
\pstart
           Herzliche Grüsse!{\\[\baselineskip]}\spacefill\mbox{OlgaS.}\pend
           \leftskip=0em{}\selectlanguage{ngerman}\endnumbering\briefempfaengerindex{Beer-Hofmann, Richard@\textsc{Beer-Hofmann, Richard}!zzzSchnitzler, Olga@\emph{von Olga Schnitzler}!1909-01-171@{{[}17. 1. 1909?{]}}|)be}\mylabel{L02564h}  \newcommand{\dateiname}{L02564}\newcommand{\titel}{Olga Schnitzler an Richard Beer-Hofmann, [17. 1. 1909?]}\newcommand{\editorInnen}{Martin Anton Müller und Gerd-Hermann Susen}%% latex-leseansicht-abspann.tex
%% Abspann für die Leseansicht.
%% Der Schalter \ifkorrekturansicht ist bereits durch den Vorspann gesetzt.

%% latex-abspann.tex
%% Gemeinsamer Abspann für Korrekturansicht und Leseansicht.
%% Setzt den Schalter \ifkorrekturansicht voraus (gesetzt in den
%% einbindenden Dateien latex-korrekturansicht-abspann.tex bzw.
%% latex-leseansicht-abspann.tex).
%% ---------------------------------------------------------------

\normalsize

% Das esempio-Environment wird nur in der Leseansicht benötigt
\ifkorrekturansicht\else
\newenvironment{esempio}[3]%
{
    \vspace{1.5ex}
    \rlap{\underline{#1}}
    \par
    \setlength{\parindent}{0cm}
    \nopagebreak
    \leftskip=#2cm
    \rightskip=#3cm
}
{
    \par
}
\fi

\doendnotes{C}
\bigskip
\vfill

\clearpage

\footnotesize

\ifkorrekturansicht
  \lohead{\textsc{register}}
\fi

% theindex-Environment neu definieren ohne reledmac
\makeatletter
\renewenvironment{theindex}{%
  \ifkorrekturansicht
    \section*{\indexname}%
  \else
    \subsubsection*{Index der erwähnten Entitäten}%
  \fi
  \setlength{\parindent}{0pt}%
  \setlength{\parskip}{0pt plus 0.3pt}%
  \let\item\@idxitem
}{%
  \ifkorrekturansicht\clearpage\fi
}
\makeatother

\IfFileExists{\jobname-pw.ind}{\input{\jobname-pw.ind}}{}

% Quellenangabe nur in der Leseansicht
\ifkorrekturansicht\else
% Fallback-Definitionen, falls die .tex-Datei \titel etc. nicht gesetzt hat
\providecommand{\titel}{}
\providecommand{\editorInnen}{}
\providecommand{\dateiname}{\jobname}

\vspace{3cm}

\vfill

\footnotesize
\textsc{Quelle}: \titel. Herausgegeben von {\editorInnen}. In: \emph{Arthur Schnitzler: Briefwechsel mit Autorinnen und Autoren}.
 Digitale Edition, https://schnitzler-briefe.acdh.oeaw.ac.at/{\dateiname}.html (Stand \today)
\fi

\end{document}


