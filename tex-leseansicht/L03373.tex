%% latex-leseansicht-vorspann.tex
%% Vorspann für die Leseansicht.
%% Lädt die gemeinsame Datei latex-vorspann.tex mit nicht gesetztem Schalter.

\newif\ifkorrekturansicht
\korrekturansichtfalse

\input{../tex-inputs/latex-vorspann}


\section[ Paul Goldmann an Arthur Schnitzler, 2[2?]. 5. [1903]]{L03373 Paul Goldmann an Arthur Schnitzler,  2[2?]. 5. [1903]}
\nopagebreak\mylabel{L03373v}
\rehead{ }\normalsize\beginnumbering\briefempfaengerindex{Schnitzler, Arthur@\textsc{Schnitzler, Arthur}!zzzGoldmann, Paul@\emph{von Paul Goldmann}!1903-05-221@{2[2?]. 5. [1903]}|(be}
\toendnotes[C]{\smallbreak\pagebreak[2]}
\correspDesc{Versand  durch Paul Goldmann am 2[2?]. 5. [1903] in Berlin
\newline{}Erhalt  durch Arthur Schnitzler im Zeitraum [23. 5. 1903
                  – 27. 5. 1903?] in Wien}\toendnotes[C]{\smallbreak}
\Standort{DLA, A:Schnitzler, HS.NZ85.1.3173.}
\physDesc{Brief, 1 Blatt, 4 Seiten, 1500 Zeichen
\newline{}Handschrift: blaue Tinte, deutsche Kurrent
\newline{}Schnitzler: 1) mit Bleistift das Jahr »903« vermerkt  2) mit rotem Buntstift drei Unterstreichungen}\toendnotes[C]{\smallbreak}
\pstart
           \raggedleft{}{\pb}\textcolor{gray}{\textbf{DESSAUERSTRASSE 19\oindex{Dessauer Straße@\textbf{Dessauer Straße}, \emph{Straße}|pw}}}\pend
           
\pstart
           Berlin\oindex{Berlin@\textbf{Berlin}, \emph{Hauptstadt}|pw}, 2\textcolor{gray}{2}. Mai.\pend
           
\pstart{}Mein lieber Freund,\pend\vspace{0.5em}
\pstart
           Dein lieber Brief hat mich{ }ſehr erfreut. Ich war über das Ausbleiben Deiner
               Nachrichten bereits in Sorge. Auch \textsc{Olgas\pwindex{Schnitzler, Olga 17.\,1.\,1882 Wien – 13.\,1.\,1970 Lugano@\textsc{Schnitzler, Olga} (17.\,1.\,1882 Wien – 13.\,1.\,1970 Lugano), \emph{Schauspielerin, Sängerin}|pw}} Brief war{ }ſehr charmant; und ich
               bitte Dich (bis ich Zeit finde, ihn zu beantworten), ihr einſtweilen in meinem Namen
               zu danken.\pend
           
\pstart
           Heut nur in aller Eile: Ich war geſtern{ }Abend bei \textsc{Dr. Elias\pwindex{Elias, Julius 12.\,7.\,1861 Hoya – 2.\,7.\,1927 Berlin@\textsc{Elias, Julius} (12.\,7.\,1861 Hoya – 2.\,7.\,1927 Berlin), \emph{Übersetzer, Publizist}|pw}}. {\pb}Sonſt anweſend \textsc{Brahm\pwindex{Brahm, Otto 5.\,2.\,1856 Hamburg – 28.\,11.\,1912 Berlin@\textsc{Brahm, Otto} (5.\,2.\,1856 Hamburg – 28.\,11.\,1912 Berlin), \emph{Theaterleiter, Regisseur}|pw}} (der mir immer{ }ſympathiſcher wird), \textsc{Georg Hirschfeld\pwindex{Hirschfeld, Georg 11.\,2.\,1873 Berlin – 17.\,1.\,1942 München@\textsc{Hirschfeld, Georg} (11.\,2.\,1873 Berlin – 17.\,1.\,1942 München), \emph{Schriftsteller}|pw}} und Frau\pwindex{Petersen, Elly 26.\,2.\,1874 Berlin – 29.\,12.\,1965 München@\textsc{Petersen, Elly} (26.\,2.\,1874 Berlin – 29.\,12.\,1965 München), \emph{Schriftstellerin}|pwv}, \textsc{Dr. Jonas\pwindex{Jonas, Paul 28.\,9.\,1850 Berlin – 11.\,2.\,1916 ebd.@\textsc{Jonas, Paul} (28.\,9.\,1850 Berlin – 11.\,2.\,1916 ebd.), \emph{Rechtsanwalt}|pw} etc}.
               Allgemeines Fragen nach Dir. Ich konnte keine Auskunft erteilen. \textsc{Brahm\pwindex{Brahm, Otto 5.\,2.\,1856 Hamburg – 28.\,11.\,1912 Berlin@\textsc{Brahm, Otto} (5.\,2.\,1856 Hamburg – 28.\,11.\,1912 Berlin), \emph{Theaterleiter, Regisseur}|pw}}{ }ſagte: Du habeſt ihm \label{K_L03373-1v}\edtext{mitgeteilt}{\lemma{\textnormal{\emph{mitgeteilt}}}\Cendnote{\textnormal{Am 29. 4. 1903 hatte Schnitzler den Plan des »\emph{Journalistenstück}\pwindex{Schnitzler, Arthur 15.\,5.\,1862 Wien – 21.\,10.\,1931 ebd.@\textsc{Schnitzler, Arthur} (15.\,5.\,1862 Wien – 21.\,10.\,1931 ebd.), \emph{Schriftsteller, Mediziner}!Fink und Fliederbusch. Komödie in drei Akten@\strich\emph{Fink und Fliederbusch. Komödie in drei Akten}|pwk}[s]« (\emph{Flink und Fiederbusch}\pwindex{Schnitzler, Arthur 15.\,5.\,1862 Wien – 21.\,10.\,1931 ebd.@\textsc{Schnitzler, Arthur} (15.\,5.\,1862 Wien – 21.\,10.\,1931 ebd.), \emph{Schriftsteller, Mediziner}!Fink und Fliederbusch. Komödie in drei Akten@\strich\emph{Fink und Fliederbusch. Komödie in drei Akten}|pwk}) notiert, zwei Tage später, am 1. 5. 1903, hatte er
                     Brahm\pwindex{Brahm, Otto 5.\,2.\,1856 Hamburg – 28.\,11.\,1912 Berlin@\textsc{Brahm, Otto} (5.\,2.\,1856 Hamburg – 28.\,11.\,1912 Berlin), \emph{Theaterleiter, Regisseur}|pwk} davon erzählt. Später beschrieb er
                  die Beschäftigung mit dem Stoff gegenüber Hugo
                     von Hofmannsthal\pwindex{Hofmannsthal, Hugo von 1.\,2.\,1874 Wien – 15.\,7.\,1929 Rodaun@\textsc{Hofmannsthal, Hugo von} (1.\,2.\,1874 Wien – 15.\,7.\,1929 Rodaun), \emph{Schriftsteller}|pwk} mehr als Gedankenspiel denn als tatsächliches Schreiben,
                     vgl. XXXX Auszeichnungsfehler: Dokument L01300 nicht gefunden.}}}\label{K_L03373-1}, es{ }ſei
               Dir ein Luſtſpiel\pwindex{Schnitzler, Arthur 15.\,5.\,1862 Wien – 21.\,10.\,1931 ebd.@\textsc{Schnitzler, Arthur} (15.\,5.\,1862 Wien – 21.\,10.\,1931 ebd.), \emph{Schriftsteller, Mediziner}!Fink und Fliederbusch. Komödie in drei Akten@\strich\emph{Fink und Fliederbusch. Komödie in drei Akten}|pwv} eingefallen.
               Darüber freuten{ }ſich Alle (\label{K_L03373-2v}\edtext{ich
                  beſonders}{\lemma{\textnormal{\emph{ich
                  besonders}}}\Cendnote{\textnormal{Goldmann\pwindex{Goldmann, Paul 31.\,1.\,1865 Breslau – 25.\,9.\,1935 Wien@\textsc{Goldmann, Paul} (31.\,1.\,1865 Breslau – 25.\,9.\,1935 Wien), \emph{Schriftsteller, Journalist}|pwk} hatte Schnitzler bereits mehrmals dazu aufgefordert, ein
                  Lustspiel zu schreiben, siehe XXXX Auszeichnungsfehler: Dokument L02914 nicht gefunden. Auch in seinem \emph{Feuilleton}\pwindex{Goldmann, Paul 31.\,1.\,1865 Breslau – 25.\,9.\,1935 Wien@\textsc{Goldmann, Paul} (31.\,1.\,1865 Breslau – 25.\,9.\,1935 Wien), \emph{Schriftsteller, Journalist}!Berliner Theater. (»Lebendige Stunden« von Arthur Schnitzler.)@\strich\emph{Berliner Theater. (»Lebendige Stunden« von Arthur Schnitzler.)}|pwk} zu \emph{Lebendige Stunden}\pwindex{Schnitzler, Arthur 15.\,5.\,1862 Wien – 21.\,10.\,1931 ebd.@\textsc{Schnitzler, Arthur} (15.\,5.\,1862 Wien – 21.\,10.\,1931 ebd.), \emph{Schriftsteller, Mediziner}!Lebendige Stunden. Vier Einakter@\strich\emph{Lebendige Stunden. Vier Einakter}|pwk}
                  nannte er neben dem historischen Stück die Gattung des Lustspiels als Schnitzlers eigentliche Spezialität.}}}\label{K_L03373-2}),
               und Alle (ich beſonders) hoffen, daß Du den Plan ausführen wirſt.\pend
           
\pstart
           Wenn ich Deine und \textsc{Olgas\pwindex{Schnitzler, Olga 17.\,1.\,1882 Wien – 13.\,1.\,1970 Lugano@\textsc{Schnitzler, Olga} (17.\,1.\,1882 Wien – 13.\,1.\,1970 Lugano), \emph{Schauspielerin, Sängerin}|pw}}{ }{\pb}Andeutungen recht verſtehe, wollt Ihr im Herbſt
                  \label{K_L03373-3v}\edtext{heirathen}{\lemma{\textnormal{\emph{heirathen}}}\Cendnote{\textnormal{Schnitzler und Olga Gussmann\pwindex{Schnitzler, Olga 17.\,1.\,1882 Wien – 13.\,1.\,1970 Lugano@\textsc{Schnitzler, Olga} (17.\,1.\,1882 Wien – 13.\,1.\,1970 Lugano), \emph{Schauspielerin, Sängerin}|pwk} heirateten am 26. 8. 1903.}}}\label{K_L03373-3}. Das iſt{ }ſehr geſcheit, und
               ich denke, die Legaliſirung des Zuſtandes wird in jeder Beziehung von{ }ſegensreichen
               Folgen{ }ſein.\pend
           
\pstart
           Auch von Euren \label{K_L03373-4v}\edtext{Reiſeplänen}{\lemma{\textnormal{\emph{Reiseplänen}}}\Cendnote{\textnormal{Schnitzler und Olga Gussmann\pwindex{Schnitzler, Olga 17.\,1.\,1882 Wien – 13.\,1.\,1970 Lugano@\textsc{Schnitzler, Olga} (17.\,1.\,1882 Wien – 13.\,1.\,1970 Lugano), \emph{Schauspielerin, Sängerin}|pwk} reisten zwischen 28. 5. 1903 und 15. 6. 1903 nach Italien\oindex{Italien@\textbf{Italien}|pwk} und Südtirol\oindex{Südtirol@\textbf{Südtirol}, \emph{Verwaltungsgebiet}|pwk}.}}}\label{K_L03373-4} habe ich mit Vergnügen vernommen; meine beſten Wünſche
               begleiten Euch nach dem{ }ſchönen Süden.\pend
           
\pstart
           \strikeout{D\textcolor{gray}{a}} Da Du{ }ſicherlich Luſt bekommen wirſt, \label{K_L03373-5v}\edtext{mehr von \textsc{Wilde\pwindex{Wilde, Oscar 16.\,10.\,1854 Dublin – 30.\,11.\,1900 Paris@\textsc{Wilde, Oscar} (16.\,10.\,1854 Dublin – 30.\,11.\,1900 Paris), \emph{Schriftsteller}|pw}}}{\lemma{\textnormal{\emph{mehr von Wilde}}}\Cendnote{\textnormal{Siehe XXXX Auszeichnungsfehler: Dokument L03372 nicht gefunden.
               }}}\label{K_L03373-5} zu \strikeout{leſen,} leſen,{ }ſo lies \label{K_L03373-6v}\edtext{»\textsc{Dorian {\pb}Grays}
                  Bildniß\pwindex{Wilde, Oscar 16.\,10.\,1854 Dublin – 30.\,11.\,1900 Paris@\textsc{Wilde, Oscar} (16.\,10.\,1854 Dublin – 30.\,11.\,1900 Paris), \emph{Schriftsteller}!Dorian Grays Bildnis@\strich\emph{Dorian Grays Bildnis}|pw}« (in der Überſetzung von \textsc{Greve\pwindex{Greve, Felix Paul 14.\,2.\,1879 Radomno – 19.\,8.\,1948 Simcoe@\textsc{Greve, Felix Paul} (14.\,2.\,1879 Radomno – 19.\,8.\,1948 Simcoe), \emph{Schriftsteller, Übersetzer}|pw}})}{\lemma{\textnormal{\emph{»Dorian … Greve)}}}\Cendnote{\textnormal{Oscar Wilde\pwindex{Wilde, Oscar 16.\,10.\,1854 Dublin – 30.\,11.\,1900 Paris@\textsc{Wilde, Oscar} (16.\,10.\,1854 Dublin – 30.\,11.\,1900 Paris), \emph{Schriftsteller}|pwk}: \emph{Dorian Grays Bildnis}\pwindex{Wilde, Oscar 16.\,10.\,1854 Dublin – 30.\,11.\,1900 Paris@\textsc{Wilde, Oscar} (16.\,10.\,1854 Dublin – 30.\,11.\,1900 Paris), \emph{Schriftsteller}!Dorian Grays Bildnis@\strich\emph{Dorian Grays Bildnis}|pwk}. Übersetzt von Felix Paul Greve\pwindex{Greve, Felix Paul 14.\,2.\,1879 Radomno – 19.\,8.\,1948 Simcoe@\textsc{Greve, Felix Paul} (14.\,2.\,1879 Radomno – 19.\,8.\,1948 Simcoe), \emph{Schriftsteller, Übersetzer}|pwk}. Minden\oindex{Minden@\textbf{Minden}, \emph{Hauptstadt}|pwk}: \emph{J. C. C. Bruns’ Verlag}\orgindex{J. C. C. Bruns@J. C. C. Bruns|pwk}
                        [1902]. Schnitzler las \emph{The picture of Dorian
                     Gray}\pwindex{Wilde, Oscar 16.\,10.\,1854 Dublin – 30.\,11.\,1900 Paris@\textsc{Wilde, Oscar} (16.\,10.\,1854 Dublin – 30.\,11.\,1900 Paris), \emph{Schriftsteller}!picture of Dorian Gray@\strich\emph{The picture of Dorian Gray}|pwk} (1890) am 30. 6. 1904. Welche Übersetzung er benutzte, jene
                  von Greve\pwindex{Greve, Felix Paul 14.\,2.\,1879 Radomno – 19.\,8.\,1948 Simcoe@\textsc{Greve, Felix Paul} (14.\,2.\,1879 Radomno – 19.\,8.\,1948 Simcoe), \emph{Schriftsteller, Übersetzer}|pwk} oder jene von Johannes Gaulke\pwindex{Gaulke, Johannes 25.\,7.\,1869 Kolberg – 17.\,11.\,1938 Berlin@\textsc{Gaulke, Johannes} (25.\,7.\,1869 Kolberg – 17.\,11.\,1938 Berlin), \emph{Schriftsteller, Übersetzer}|pwk} (Oscar Wilde\pwindex{Wilde, Oscar 16.\,10.\,1854 Dublin – 30.\,11.\,1900 Paris@\textsc{Wilde, Oscar} (16.\,10.\,1854 Dublin – 30.\,11.\,1900 Paris), \emph{Schriftsteller}|pwk}: \emph{Dorian Gray}\pwindex{Wilde, Oscar 16.\,10.\,1854 Dublin – 30.\,11.\,1900 Paris@\textsc{Wilde, Oscar} (16.\,10.\,1854 Dublin – 30.\,11.\,1900 Paris), \emph{Schriftsteller}!Dorian Gray@\strich\emph{Dorian Gray}|pwk}. Übersetzt von Johannes Gaulke\pwindex{Gaulke, Johannes 25.\,7.\,1869 Kolberg – 17.\,11.\,1938 Berlin@\textsc{Gaulke, Johannes} (25.\,7.\,1869 Kolberg – 17.\,11.\,1938 Berlin), \emph{Schriftsteller, Übersetzer}|pwk}. Leipzig\oindex{Leipzig@\textbf{Leipzig}, \emph{Hauptstadt}|pwk}: \emph{Verlag von Max Spohr}\orgindex{Verlag Max Spohr@Verlag Max Spohr|pwk}{ }{[}1901{]}), ist offen. Vgl. A. S.: \emph{Lektüren}, England.}}}\label{K_L03373-6}.\pend
           
\pstart
           Ich habe nichts vergeſſen, nichts überwunden; habe nach meiner \label{K_L03373-7v}\edtext{Rückkehr aus Wien\oindex{Wien@\textbf{Wien}, \emph{Verwaltungsgebiet}|pw}}{\lemma{\textnormal{\emph{Rückkehr aus Wien}}}\Cendnote{\textnormal{Siehe XXXX Auszeichnungsfehler: Dokument L03372 nicht gefunden.
               }}}\label{K_L03373-7} wieder eine{ }ſchreckliche Kriſis durchgemacht; und verbringe mein Leben in
                  \label{K_L03373-8v}\edtext{Reue und Sehnſucht}{\lemma{\textnormal{\emph{Reue und Sehnsucht}}}\Cendnote{\textnormal{Er konnte es nicht verwinden, dass ihn Theodore Rottenberg\pwindex{Rottenberg, Theodore 7.\,9.\,1875 – 5.\,4.\,1945 Limburg an der Lahn@\textsc{Rottenberg, Theodore} (7.\,9.\,1875 – 5.\,4.\,1945 Limburg an der Lahn)|pwk}{ }Anfang 1903 verlassen hatte, vgl. XXXX Auszeichnungsfehler: Dokument L03360 nicht gefunden.}}}\label{K_L03373-8}, hoffnungsloſer
                  Sehnſucht{\dotsfour}\pend
           
\pstart
           Nächſtens mehr! Viele herzliche Grüße Dir und \textsc{Olga\pwindex{Schnitzler, Olga 17.\,1.\,1882 Wien – 13.\,1.\,1970 Lugano@\textsc{Schnitzler, Olga} (17.\,1.\,1882 Wien – 13.\,1.\,1970 Lugano), \emph{Schauspielerin, Sängerin}|pw}}! {\\[\baselineskip]}Dein {\\[\baselineskip]}\spacefill\mbox{Paul Goldmn}\pend
           \leftskip=0em{}
\pstart
           \noindent{}\label{T_L03373-1v}\edtext{{\pb}Die \textsc{Triesch\pwindex{Triesch, Irene 13.\,4.\,1877 Wien – 24.\,11.\,1964 Basel@\textsc{Triesch, Irene} (13.\,4.\,1877 Wien – 24.\,11.\,1964 Basel), \emph{Schauspielerin}|pw}}, die geſtern{ }Abend auch da war,{ }ſagte, daß{ }ſie nach \label{K_L03373-9v}\edtext{Leipzig\oindex{Leipzig@\textbf{Leipzig}, \emph{Hauptstadt}|pw}}{\lemma{\textnormal{\emph{Leipzig}}}\Cendnote{\textnormal{Das \emph{Deutsche Theater Berlin}\orgindex{Deutsches Theater Berlin@Deutsches Theater Berlin|pwk} hatte ein Gastspiel am Schauspielhaus Leipzig\oindex{Schauspielhaus Leipzig@\textbf{Schauspielhaus Leipzig}, \emph{Theater}|pwk}. Die Aufführung von \emph{Der Schleier der Beatrice}\pwindex{Schnitzler, Arthur 15.\,5.\,1862 Wien – 21.\,10.\,1931 ebd.@\textsc{Schnitzler, Arthur} (15.\,5.\,1862 Wien – 21.\,10.\,1931 ebd.), \emph{Schriftsteller, Mediziner}!Schleier der Beatrice. Schauspiel in fünf Akten@\strich\emph{Der Schleier der Beatrice. Schauspiel in fünf Akten}|pwk} mit Irene Triesch\pwindex{Triesch, Irene 13.\,4.\,1877 Wien – 24.\,11.\,1964 Basel@\textsc{Triesch, Irene} (13.\,4.\,1877 Wien – 24.\,11.\,1964 Basel), \emph{Schauspielerin}|pwk} in der Hauptrolle fand am
                        24. 5. 1903 statt.}}}\label{K_L03373-9} geht, um dort den
                     »Schleier der \textsc{Beatrice}\pwindex{Schnitzler, Arthur 15.\,5.\,1862 Wien – 21.\,10.\,1931 ebd.@\textsc{Schnitzler, Arthur} (15.\,5.\,1862 Wien – 21.\,10.\,1931 ebd.), \emph{Schriftsteller, Mediziner}!Schleier der Beatrice. Schauspiel in fünf Akten@\strich\emph{Der Schleier der Beatrice. Schauspiel in fünf Akten}|pw}« zu{ }ſpielen.}{\lemma{\textnormal{\emph{Die … spielen.}}}\Cendnote{\textnormal{kopfüber im oberen
                     rechten Eck der ersten Seite}}}\label{T_L03373-1}\pend
           \selectlanguage{ngerman}\endnumbering\briefempfaengerindex{Schnitzler, Arthur@\textsc{Schnitzler, Arthur}!zzzGoldmann, Paul@\emph{von Paul Goldmann}!1903-05-221@{2[2?]. 5. [1903]}|)be}\mylabel{L03373h}  \newcommand{\dateiname}{L03373}\newcommand{\titel}{Paul Goldmann an Arthur Schnitzler, 2[2?]. 5. [1903]}\newcommand{\editorInnen}{Martin Anton Müller und Laura Untner}%% latex-leseansicht-abspann.tex
%% Abspann für die Leseansicht.
%% Der Schalter \ifkorrekturansicht ist bereits durch den Vorspann gesetzt.

%% latex-abspann.tex
%% Gemeinsamer Abspann für Korrekturansicht und Leseansicht.
%% Setzt den Schalter \ifkorrekturansicht voraus (gesetzt in den
%% einbindenden Dateien latex-korrekturansicht-abspann.tex bzw.
%% latex-leseansicht-abspann.tex).
%% ---------------------------------------------------------------

\normalsize

% Das esempio-Environment wird nur in der Leseansicht benötigt
\ifkorrekturansicht\else
\newenvironment{esempio}[3]%
{
    \vspace{1.5ex}
    \rlap{\underline{#1}}
    \par
    \setlength{\parindent}{0cm}
    \nopagebreak
    \leftskip=#2cm
    \rightskip=#3cm
}
{
    \par
}
\fi

\doendnotes{C}
\bigskip
\vfill

\clearpage

\footnotesize

\ifkorrekturansicht
  \lohead{\textsc{register}}
\fi

% theindex-Environment neu definieren ohne reledmac
\makeatletter
\renewenvironment{theindex}{%
  \ifkorrekturansicht
    \section*{\indexname}%
  \else
    \subsubsection*{Index der erwähnten Entitäten}%
  \fi
  \setlength{\parindent}{0pt}%
  \setlength{\parskip}{0pt plus 0.3pt}%
  \let\item\@idxitem
}{%
  \ifkorrekturansicht\clearpage\fi
}
\makeatother

\IfFileExists{\jobname-pw.ind}{\input{\jobname-pw.ind}}{}

% Quellenangabe nur in der Leseansicht
\ifkorrekturansicht\else
% Fallback-Definitionen, falls die .tex-Datei \titel etc. nicht gesetzt hat
\providecommand{\titel}{}
\providecommand{\editorInnen}{}
\providecommand{\dateiname}{\jobname}

\vspace{3cm}

\vfill

\footnotesize
\textsc{Quelle}: \titel. Herausgegeben von {\editorInnen}. In: \emph{Arthur Schnitzler: Briefwechsel mit Autorinnen und Autoren}.
 Digitale Edition, https://schnitzler-briefe.acdh.oeaw.ac.at/{\dateiname}.html (Stand \today)
\fi

\end{document}


