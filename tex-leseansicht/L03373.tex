%% latex-korrekturansicht-vorspann.tex
%% Vorspann für die Korrekturansicht.
%% Lädt die gemeinsame Datei latex-vorspann.tex mit gesetztem Schalter.

\newif\ifkorrekturansicht
\korrekturansichttrue

\input{../tex-inputs/latex-vorspann}


\section[ Paul Goldmann an Arthur Schnitzler, 2{[}2?{]}. 5. {[}1903{]}]{L03373 Paul Goldmann an Arthur Schnitzler, 2{[}2?{]}. 5. {[}1903{]}}
\nopagebreak\mylabel{L03373v}
\rehead{ }\normalsize\beginnumbering\briefempfaengerindex{Schnitzler, Arthur@\textsc{Schnitzler, Arthur}!zzzGoldmann, Paul@\emph{von Paul Goldmann}!1903-05-221@{2{[}2?{]}. 5. {[}1903{]}}|(be}
\toendnotes[C]{\smallbreak\pagebreak[2]}\Standort{DLA, A:Schnitzler, HS.NZ85.1.3173.}
\physDesc{Brief, 1 Blatt, 4 Seiten, 1500 Zeichen
\newline{}Handschrift: blaue Tinte, deutsche Kurrent
\newline{}Schnitzler: 1) mit Bleistift das Jahr »903« vermerkt  2) mit rotem Buntstift drei Unterstreichungen}\toendnotes[C]{\smallbreak}
\pstart
           \raggedleft{}{\pb}\textcolor{gray}{\textbf{DESSAUERSTRASSE 19\oindex{Dessauer Strasse@\textbf{Dessauer Straße}, \emph{Straße (K.STR)}|pw}}}\pend
           
\pstart
           Berlin\oindex{Berlin@\textbf{Berlin}, \emph{P.PPLC}|pw}, 2\textcolor{gray}{2}. Mai.\pend
           
\pstart{}Mein lieber Freund,\pend\vspace{0.5em}
\pstart
           Dein lieber Brief hat mich ſehr erfreut. Ich war über das Ausbleiben Deiner
               Nachrichten bereits in Sorge. Auch \textsc{Olgas\pwindex{Schnitzler, Olga 17.01.1882 – 13.01.1970@\textsc{Schnitzler, Olga} (17.01.1882 – 13.01.1970), \emph{Schauspieler/Schauspielerin, Sänger/Sängerin}|pw}} Brief war ſehr charmant; und ich
               bitte Dich (bis ich Zeit finde, ihn zu beantworten), ihr einſtweilen in meinem Namen
               zu danken.\pend
           
\pstart
           Heut nur in aller Eile: Ich war geſtern{ }Abend bei \textsc{Dr. Elias\pwindex{Elias, Julius 12.07.1861 – 02.07.1927@\textsc{Elias, Julius} (12.07.1861 – 02.07.1927), \emph{Übersetzer/Übersetzerin, Publizist/Publizistin}|pw}}. {\pb}Sonſt anweſend \textsc{Brahm\pwindex{Brahm, Otto 05.02.1856 – 28.11.1912@\textsc{Brahm, Otto} (05.02.1856 – 28.11.1912), \emph{Theaterleiter/Theaterleiterin, Regisseur/Regisseurin}|pw}} (der mir immer ſympathiſcher wird), \textsc{Georg Hirschfeld\pwindex{Hirschfeld, Georg 11.02.1873 – 17.01.1942@\textsc{Hirschfeld, Georg} (11.02.1873 – 17.01.1942), \emph{Schriftsteller/Schriftstellerin}|pw}} und Frau\pwindex{Petersen, Elly 26.02.1874 – 29.12.1965@\textsc{Petersen, Elly} (26.02.1874 – 29.12.1965), \emph{Schriftsteller/Schriftstellerin}|pwv}, \textsc{Dr. Jonas\pwindex{Jonas, Paul 28.09.1850 – 11.02.1916@\textsc{Jonas, Paul} (28.09.1850 – 11.02.1916), \emph{Rechtsanwalt/Rechtsanwältin}|pw} etc}.
               Allgemeines Fragen nach Dir. Ich konnte keine Auskunft erteilen. \textsc{Brahm\pwindex{Brahm, Otto 05.02.1856 – 28.11.1912@\textsc{Brahm, Otto} (05.02.1856 – 28.11.1912), \emph{Theaterleiter/Theaterleiterin, Regisseur/Regisseurin}|pw}} ſagte: Du habeſt ihm \label{K_L03373-1v}\edtext{mitgeteilt}{\lemma{\textnormal{\emph{mitgeteilt}}}\Cendnote{\textnormal{Am 29. 4. 1903 hatte Schnitzler den Plan des »\emph{Journalistenstück}\pwindex{Fink und Fliederbusch. Komoedie in drei Akten@\emph{Fink und Fliederbusch. Komödie in drei Akten}|pwk}[s]« (\emph{Flink und Fiederbusch}\pwindex{Fink und Fliederbusch. Komoedie in drei Akten@\emph{Fink und Fliederbusch. Komödie in drei Akten}|pwk}) notiert, zwei Tage später, am 1. 5. 1903, hatte er
                     Brahm\pwindex{Brahm, Otto 05.02.1856 – 28.11.1912@\textsc{Brahm, Otto} (05.02.1856 – 28.11.1912), \emph{Theaterleiter/Theaterleiterin, Regisseur/Regisseurin}|pwk} davon erzählt. Später beschrieb er
                  die Beschäftigung mit dem Stoff gegenüber Hugo
                     von Hofmannsthal\pwindex{Hofmannsthal, Hugo von 1874-02-01 – 1929-07-15@\textsc{Hofmannsthal, Hugo von} (1874-02-01 – 1929-07-15), \emph{Schriftsteller/Schriftstellerin}|pwk} mehr als Gedankenspiel denn als tatsächliches Schreiben,
                     vgl. Arthur Schnitzler an Hugo von Hofmannsthal, 26. 6. 1903.}}}\label{K_L03373-1}, es ſei
               Dir ein Luſtſpiel\pwindex{Fink und Fliederbusch. Komoedie in drei Akten@\emph{Fink und Fliederbusch. Komödie in drei Akten}|pwv} eingefallen.
               Darüber freuten ſich Alle (\label{K_L03373-2v}\edtext{ich
                  beſonders}{\lemma{\textnormal{\emph{ich
                  beſonders}}}\Cendnote{\textnormal{Goldmann\pwindex{Goldmann, Paul 31.01.1865 – 25.09.1935@\textsc{Goldmann, Paul} (31.01.1865 – 25.09.1935), \emph{Schriftsteller/Schriftstellerin, Journalist/Journalistin}|pwk} hatte Schnitzler bereits mehrmals dazu aufgefordert, ein
                  Lustspiel zu schreiben, siehe Paul Goldmann an Arthur Schnitzler, 2. 5. [1900]. Auch in seinem \emph{Feuilleton}\pwindex{Berliner Theater. (»Lebendige Stunden« von Arthur Schnitzler.)@\emph{Berliner Theater. (»Lebendige Stunden« von Arthur Schnitzler.)}|pwk} zu \emph{Lebendige Stunden}\pwindex{Lebendige Stunden. Vier Einakter@\emph{Lebendige Stunden. Vier Einakter}|pwk}
                  nannte er neben dem historischen Stück die Gattung des Lustspiels als Schnitzlers eigentliche Spezialität.}}}\label{K_L03373-2}),
               und Alle (ich beſonders) hoffen, daß Du den Plan ausführen wirſt.\pend
           
\pstart
           Wenn ich Deine und \textsc{Olgas\pwindex{Schnitzler, Olga 17.01.1882 – 13.01.1970@\textsc{Schnitzler, Olga} (17.01.1882 – 13.01.1970), \emph{Schauspieler/Schauspielerin, Sänger/Sängerin}|pw}}{ }{\pb}Andeutungen recht verſtehe, wollt Ihr im Herbſt
                  \label{K_L03373-3v}\edtext{heirathen}{\lemma{\textnormal{\emph{heirathen}}}\Cendnote{\textnormal{Schnitzler und Olga Gussmann\pwindex{Schnitzler, Olga 17.01.1882 – 13.01.1970@\textsc{Schnitzler, Olga} (17.01.1882 – 13.01.1970), \emph{Schauspieler/Schauspielerin, Sänger/Sängerin}|pwk} heirateten am 26. 8. 1903.}}}\label{K_L03373-3}. Das iſt ſehr geſcheit, und
               ich denke, die Legaliſirung des Zuſtandes wird in jeder Beziehung von ſegensreichen
               Folgen ſein.\pend
           
\pstart
           Auch von Euren \label{K_L03373-4v}\edtext{Reiſeplänen}{\lemma{\textnormal{\emph{Reiſeplänen}}}\Cendnote{\textnormal{Schnitzler und Olga Gussmann\pwindex{Schnitzler, Olga 17.01.1882 – 13.01.1970@\textsc{Schnitzler, Olga} (17.01.1882 – 13.01.1970), \emph{Schauspieler/Schauspielerin, Sänger/Sängerin}|pwk} reisten zwischen 28. 5. 1903 und 15. 6. 1903 nach Italien\oindex{Italien@\textbf{Italien}, \emph{A.PCLI}|pwk} und Südtirol\oindex{Suedtirol@\textbf{Südtirol}, \emph{A.ADM2}|pwk}.}}}\label{K_L03373-4} habe ich mit Vergnügen vernommen; meine beſten Wünſche
               begleiten Euch nach dem ſchönen Süden.\pend
           
\pstart
           \strikeout{D\textcolor{gray}{a}} Da Du ſicherlich Luſt bekommen wirſt, \label{K_L03373-5v}\edtext{mehr von \textsc{Wilde\pwindex{Wilde, Oscar 16.10.1854 – 30.11.1900@\textsc{Wilde, Oscar} (16.10.1854 – 30.11.1900), \emph{Schriftsteller/Schriftstellerin}|pw}}}{\lemma{\textnormal{\emph{mehr von Wilde}}}\Cendnote{\textnormal{Siehe Paul Goldmann an Arthur Schnitzler, 13. 5. [1903].
               }}}\label{K_L03373-5} zu \strikeout{leſen,} leſen, ſo lies \label{K_L03373-6v}\edtext{»\textsc{Dorian {\pb}Grays}
                  Bildniß\pwindex{Dorian Grays Bildnis@\emph{Dorian Grays Bildnis}|pw}« (in der Überſetzung von \textsc{Greve\pwindex{Greve, Felix Paul 1879-02-14 – 1948-08-19@\textsc{Greve, Felix Paul} (1879-02-14 – 1948-08-19), \emph{Schriftsteller/Schriftstellerin, Übersetzer/Übersetzerin}|pw}})}{\lemma{\textnormal{\emph{»Dorian … Greve)}}}\Cendnote{\textnormal{Oscar Wilde\pwindex{Wilde, Oscar 16.10.1854 – 30.11.1900@\textsc{Wilde, Oscar} (16.10.1854 – 30.11.1900), \emph{Schriftsteller/Schriftstellerin}|pwk}: \emph{Dorian Grays Bildnis}\pwindex{Dorian Grays Bildnis@\emph{Dorian Grays Bildnis}|pwk}. Übersetzt von Felix Paul Greve\pwindex{Greve, Felix Paul 1879-02-14 – 1948-08-19@\textsc{Greve, Felix Paul} (1879-02-14 – 1948-08-19), \emph{Schriftsteller/Schriftstellerin, Übersetzer/Übersetzerin}|pwk}. Minden\oindex{Minden@\textbf{Minden}, \emph{P.PPLA3}|pwk}: \emph{J. C. C. Bruns’ Verlag}\orgindex{J. C. C. Bruns@J. C. C. Bruns|pwk}
                        [1902]. Schnitzler las \emph{The picture of Dorian
                     Gray}\pwindex{picture of Dorian Gray@\emph{The picture of Dorian Gray}|pwk} (1890) am 30. 6. 1904. Welche Übersetzung er benutzte, jene
                  von Greve\pwindex{Greve, Felix Paul 1879-02-14 – 1948-08-19@\textsc{Greve, Felix Paul} (1879-02-14 – 1948-08-19), \emph{Schriftsteller/Schriftstellerin, Übersetzer/Übersetzerin}|pwk} oder jene von Johannes Gaulke\pwindex{Gaulke, Johannes 1869-07-25 – 1938-11-17@\textsc{Gaulke, Johannes} (1869-07-25 – 1938-11-17), \emph{Schriftsteller/Schriftstellerin, Übersetzer/Übersetzerin}|pwk} (Oscar Wilde\pwindex{Wilde, Oscar 16.10.1854 – 30.11.1900@\textsc{Wilde, Oscar} (16.10.1854 – 30.11.1900), \emph{Schriftsteller/Schriftstellerin}|pwk}: \emph{Dorian Gray}\pwindex{Dorian Gray@\emph{Dorian Gray}|pwk}. Übersetzt von Johannes Gaulke\pwindex{Gaulke, Johannes 1869-07-25 – 1938-11-17@\textsc{Gaulke, Johannes} (1869-07-25 – 1938-11-17), \emph{Schriftsteller/Schriftstellerin, Übersetzer/Übersetzerin}|pwk}. Leipzig\oindex{Leipzig@\textbf{Leipzig}, \emph{P.PPLA3}|pwk}: \emph{Verlag von Max Spohr}\orgindex{Verlag Max Spohr@Verlag Max Spohr|pwk}{ }{[}1901{]}), ist offen. Vgl. A. S.: \emph{Lektüren}, England.}}}\label{K_L03373-6}.\pend
           
\pstart
           Ich habe nichts vergeſſen, nichts überwunden; habe nach meiner \label{K_L03373-7v}\edtext{Rückkehr aus Wien\oindex{Wien@\textbf{Wien}, \emph{A.ADM2}|pw}}{\lemma{\textnormal{\emph{Rückkehr aus Wien}}}\Cendnote{\textnormal{Siehe Paul Goldmann an Arthur Schnitzler, 13. 5. [1903].
               }}}\label{K_L03373-7} wieder eine ſchreckliche Kriſis durchgemacht; und verbringe mein Leben in
                  \label{K_L03373-8v}\edtext{Reue und Sehnſucht}{\lemma{\textnormal{\emph{Reue und Sehnſucht}}}\Cendnote{\textnormal{Er konnte es nicht verwinden, dass ihn Theodore Rottenberg\pwindex{Rottenberg, Theodore 1875-09-07 – 1945-04-05@\textsc{Rottenberg, Theodore} (1875-09-07 – 1945-04-05)|pwk}{ }Anfang 1903 verlassen hatte, vgl. Paul Goldmann an Arthur Schnitzler, 3. 1. [1903].}}}\label{K_L03373-8}, hoffnungsloſer
                  Sehnſucht{\dotsfour}\pend
           
\pstart
           Nächſtens mehr! Viele herzliche Grüße Dir und \textsc{Olga\pwindex{Schnitzler, Olga 17.01.1882 – 13.01.1970@\textsc{Schnitzler, Olga} (17.01.1882 – 13.01.1970), \emph{Schauspieler/Schauspielerin, Sänger/Sängerin}|pw}}! {\\[\baselineskip]}Dein {\\[\baselineskip]}\spacefill\mbox{Paul Goldmn}\pend
           \leftskip=0em{}
\pstart
           \noindent{}\label{T_L03373-1v}\edtext{{\pb}Die \textsc{Triesch\pwindex{Triesch, Irene 13.04.1877 – 24.11.1964@\textsc{Triesch, Irene} (13.04.1877 – 24.11.1964), \emph{Schauspieler/Schauspielerin}|pw}}, die geſtern{ }Abend auch da war, ſagte, daß ſie nach \label{K_L03373-9v}\edtext{Leipzig\oindex{Leipzig@\textbf{Leipzig}, \emph{P.PPLA3}|pw}}{\lemma{\textnormal{\emph{Leipzig}}}\Cendnote{\textnormal{Das \emph{Deutsche Theater Berlin}\orgindex{Deutsches Theater Berlin@Deutsches Theater Berlin|pwk} hatte ein Gastspiel am Schauspielhaus Leipzig\oindex{Schauspielhaus Leipzig@\textbf{Schauspielhaus Leipzig}, \emph{Theater (K.THE)}|pwk}. Die Aufführung von \emph{Der Schleier der Beatrice}\pwindex{Schleier der Beatrice. Schauspiel in fuenf Akten@\emph{Der Schleier der Beatrice. Schauspiel in fünf Akten}|pwk} mit Irene Triesch\pwindex{Triesch, Irene 13.04.1877 – 24.11.1964@\textsc{Triesch, Irene} (13.04.1877 – 24.11.1964), \emph{Schauspieler/Schauspielerin}|pwk} in der Hauptrolle fand am
                        24. 5. 1903 statt.}}}\label{K_L03373-9} geht, um dort den
                     »Schleier der \textsc{Beatrice}\pwindex{Schleier der Beatrice. Schauspiel in fuenf Akten@\emph{Der Schleier der Beatrice. Schauspiel in fünf Akten}|pw}« zu ſpielen.}{\lemma{\textnormal{\emph{Die … ſpielen.}}}\Cendnote{\textnormal{kopfüber im oberen
                     rechten Eck der ersten Seite}}}\label{T_L03373-1}\pend
           \selectlanguage{ngerman}\endnumbering\briefempfaengerindex{Schnitzler, Arthur@\textsc{Schnitzler, Arthur}!zzzGoldmann, Paul@\emph{von Paul Goldmann}!1903-05-221@{2{[}2?{]}. 5. {[}1903{]}}|)be}\mylabel{L03373h}  \normalsize

\doendnotes{C}
\bigskip
\vfill

\clearpage

\footnotesize

\lohead{\textsc{register}}

% Definiere theindex-Environment komplett neu ohne reledmac
\makeatletter
\renewenvironment{theindex}{%
  \section*{\indexname}%
  \setlength{\parindent}{0pt}%
  \setlength{\parskip}{0pt plus 0.3pt}%
  \let\item\@idxitem
}{%
  \clearpage
}
\makeatother

\IfFileExists{\jobname-pw.ind}{\input{\jobname-pw.ind}}{}

\end{document}

      