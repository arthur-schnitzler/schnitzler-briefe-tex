%% latex-leseansicht-vorspann.tex
%% Vorspann für die Leseansicht.
%% Lädt die gemeinsame Datei latex-vorspann.tex mit nicht gesetztem Schalter.

\newif\ifkorrekturansicht
\korrekturansichtfalse

\input{../tex-inputs/latex-vorspann}

\begin{center}
            \textcolor{red}{ENTWURF, NICHT FERTIG KORRIGIERT}
                      \end{center}
            
         
         \renewcommand{\erwaehntePersonen}{Personen: Otto Brahm, Julius Elias, Johannes Gaulke, Felix Paul Greve, Georg Hirschfeld, Paul Jonas, Elly Petersen, Theodore Rottenberg, Olga Schnitzler, Irene Triesch, Oscar Wilde}
         \renewcommand{\erwaehnteInstitutionen}{Institutionen: Deutsches Theater Berlin, J. C. C. Bruns, Verlag Max Spohr}
         \renewcommand{\erwaehnteOrte}{Orte: Berlin, Dessauer Straße, Italien, Leipzig, Minden, Schauspielhaus Leipzig, Südtirol, Wien}
         \renewcommand{\erwaehnteWerke}{Werke: Berliner Theater. (»Lebendige Stunden« von Arthur Schnitzler.), Der Schleier der Beatrice. Schauspiel in fünf Akten, Dorian Gray, Dorian Grays Bildnis, Fink und Fliederbusch. Komödie in drei Akten, Lebendige Stunden. Vier Einakter, Professor Bernhardi. Komödie in fünf Akten, The picture of Dorian Gray}
               \section[ Paul Goldmann an Arthur Schnitzler, {[}22.?{]} 5. {[}1903{]}]{ Paul Goldmann an Arthur Schnitzler, {[}22.?{]} 5. {[}1903{]}}\nopagebreak\mylabel{v}\rehead{ }\begin{ledgroupsized}[t]{13cm}\normalsize\beginnumbering \toendnotes[C]{\smallbreak\pagebreak[2]} \Standort{DLA, A:Schnitzler, HS.NZ85.1.3173.}
\physDesc{Brief, 1 Blatt, 4 Seiten
\newline{}Handschrift: blaue Tinte, deutsche Kurrent
\newline{}Schnitzler: 1) mit Bleistift das Jahr »{[}1{]}903« vermerkt  2) mit rotem Buntstift drei Unterstreichungen}\toendnotes[C]{\smallbreak}\pstart
           \noindent{}\raggedleft{}{\pb}\textcolor{gray}{\textbf{DESSAUERSTRASSE 19\oindex{Dessauer Strasse@\textbf{Dessauer Straße}|pw}}}\pend
           \pstart
           Berlin\oindex{Berlin@\textbf{Berlin}|pw}, 2\textcolor{gray}{2}. Mai.\pend
           \pstart\center{}Mein lieber Freund,\pend\pstart
           Dein lieber Brief hat mich ſehr erfreut. Ich war über das Ausbleiben Deiner
               Nachrichten bereits in Sorge. Auch \textsc{Olga\pwindex{Schnitzler, Olga 17.01.1882 – 13.01.1970@\textsc{Schnitzler, Olga} (17.01.1882 – 13.01.1970), \emph{Schauspielerin, Sängerin}|pw}s} Brief war ſehr charmant; und ich
               bitte Dich (bis ich Zeit finde, ihn zu beantworten), ihr einſtweilen in meinem Namen
               zu danken.\pend
           \pstart
           Heut nur in aller Eile: Ich war geſtern{ }Abend bei \textsc{Dr. Elias\pwindex{Elias, Julius 12.07.1861 – 02.07.1927@\textsc{Elias, Julius} (12.07.1861 – 02.07.1927), \emph{Übersetzer, Publizist}|pw}}.\pend
           \pstart
           {\pb}Sonſt anweſend \textsc{Brahm\pwindex{Brahm, Otto 05.02.1856 – 28.11.1912@\textsc{Brahm, Otto} (05.02.1856 – 28.11.1912), \emph{Theaterleiter, Regisseur}|pw}} (der mir immer ſympathiſcher wird), \textsc{Georg Hirschfeld\pwindex{Hirschfeld, Georg 11.02.1873 – 17.01.1942@\textsc{Hirschfeld, Georg} (11.02.1873 – 17.01.1942), \emph{Schriftsteller}|pw}} und Frau\pwindex{Petersen, Elly 26.02.1874 – 29.12.1965@\textsc{Petersen, Elly} (26.02.1874 – 29.12.1965), \emph{Schriftstellerin}|pwv}, \textsc{Dr. Jonas\pwindex{Jonas, Paul 28.09.1850 – 11.02.1916@\textsc{Jonas, Paul} (28.09.1850 – 11.02.1916), \emph{Rechtsanwalt}|pw} etc}.
               Allgemeines Fragen nach Dir. Ich konnte keine Auskunft erteilen. \textsc{Brahm\pwindex{Brahm, Otto 05.02.1856 – 28.11.1912@\textsc{Brahm, Otto} (05.02.1856 – 28.11.1912), \emph{Theaterleiter, Regisseur}|pw}} ſagte: Du habeſt ihm \label{K_L03373-3v}\edtext{mitgeteilt}{\lemma{\textnormal{\emph{mitgeteilt}}}\Cendnote{\textnormal{Vgl. A. S.: \emph{Tagebuch}, 1. 5. 1903. Otto Seidlin
                  vermutete, es könnte sich um \emph{Professor
                     Bernhardi}\pwindex{Schnitzler, Arthur 15.05.1862 – 21.10.1931@\textsc{Schnitzler, Arthur} (15.05.1862 – 21.10.1931), \emph{Schriftsteller, Mediziner}!Professor Bernhardi. Komoedie in fuenf Akten1912@\strich\emph{Professor Bernhardi. Komödie in fünf Akten} {[}1912{]}|pwk} gehandelt haben, vgl. \emph{Der Briefwechsel
                        Arthur Schnitzler — Otto Brahm}. Vollständige Ausgabe. Herausgegeben,
                     eingeleitet und erläutert von Oskar Seidlin. Tübingen:
                        \emph{Niemeyer}{ }1975, S. 141. Vermutlich ging es jedoch um \emph{Flink und Fiederbusch}\pwindex{Schnitzler, Arthur 15.05.1862 – 21.10.1931@\textsc{Schnitzler, Arthur} (15.05.1862 – 21.10.1931), \emph{Schriftsteller, Mediziner}!Fink und Fliederbusch. Komoedie in drei Akten1917@\strich\emph{Fink und Fliederbusch. Komödie in drei Akten} {[}1917{]}|pwk}.}}}\label{K_L03373-3h}, es ſei Dir ein
                  Luſtſpiel\pwindex{Schnitzler, Arthur 15.05.1862 – 21.10.1931@\textsc{Schnitzler, Arthur} (15.05.1862 – 21.10.1931), \emph{Schriftsteller, Mediziner}!Fink und Fliederbusch. Komoedie in drei Akten1917@\strich\emph{Fink und Fliederbusch. Komödie in drei Akten} {[}1917{]}|pwv} eingefallen.
               Darüber freuten ſich Alle (\label{K_L03373-4v}\edtext{ich
                  beſonders}{\lemma{\textnormal{\emph{ich
                  beſonders}}}\Cendnote{\textnormal{Goldmann\pwindex{Goldmann, Paul 31.01.1865 – 25.09.1935@\textsc{Goldmann, Paul} (31.01.1865 – 25.09.1935), \emph{Schriftsteller, Journalist}|pwk} hatte Schnitzler\pwindex{Schnitzler, Arthur 15.05.1862 – 21.10.1931@\textsc{Schnitzler, Arthur} (15.05.1862 – 21.10.1931), \emph{Schriftsteller, Mediziner}|pwk} bereits mehrmals dazu aufgefordert, ein
                  Lustspiel zu schreiben, siehe Paul Goldmann an Arthur Schnitzler, 2. 5. [1900]. Auch in seinem \emph{Feuilleton}\pwindex{Goldmann, Paul 31.01.1865 – 25.09.1935@\textsc{Goldmann, Paul} (31.01.1865 – 25.09.1935), \emph{Schriftsteller, Journalist}!Berliner Theater. (»Lebendige Stunden« von Arthur Schnitzler.)1902-01-22@\strich\emph{Berliner Theater. (»Lebendige Stunden« von Arthur Schnitzler.)} {[}1902-01-22{]}|pwk} zu \emph{Lebendige Stunden}\pwindex{Schnitzler, Arthur 15.05.1862 – 21.10.1931@\textsc{Schnitzler, Arthur} (15.05.1862 – 21.10.1931), \emph{Schriftsteller, Mediziner}!Lebendige Stunden. Vier Einakter1901-12-23@\strich\emph{Lebendige Stunden. Vier Einakter} {[}1901-12-23{]}|pwk}
                  nannte er neben dem historischen Stück die Gattung des Lustspiels als Schnitzler\pwindex{Schnitzler, Arthur 15.05.1862 – 21.10.1931@\textsc{Schnitzler, Arthur} (15.05.1862 – 21.10.1931), \emph{Schriftsteller, Mediziner}|pwk}s eigentliche Spezialität.}}}\label{K_L03373-4h}),
               und Alle (ich beſonders) hoffen, daß Du den Plan ausführen wirſt.\pend
           \pstart
           Wenn ich Deine und \textsc{Olga\pwindex{Schnitzler, Olga 17.01.1882 – 13.01.1970@\textsc{Schnitzler, Olga} (17.01.1882 – 13.01.1970), \emph{Schauspielerin, Sängerin}|pw}s}{\pb}Andeutungen recht verſtehe, wollt Ihr im Herbſt
                  \label{K_L03373-5v}\edtext{heirathen}{\lemma{\textnormal{\emph{heirathen}}}\Cendnote{\textnormal{Schnitzler\pwindex{Schnitzler, Arthur 15.05.1862 – 21.10.1931@\textsc{Schnitzler, Arthur} (15.05.1862 – 21.10.1931), \emph{Schriftsteller, Mediziner}|pwk} und Olga Gussmann\pwindex{Schnitzler, Olga 17.01.1882 – 13.01.1970@\textsc{Schnitzler, Olga} (17.01.1882 – 13.01.1970), \emph{Schauspielerin, Sängerin}|pwk} heirateten am 26. 8. 1903.}}}\label{K_L03373-5h}. Das iſt ſehr geſcheit, und
               ich denke, die Legaliſirung des Zuſtandes wird in jeder Beziehung von ſagensreichen
               Folgen ſein.\pend
           \pstart
           Auch von Euren \label{K_L03373-11v}\edtext{Reiſeplänen}{\lemma{\textnormal{\emph{Reiſeplänen}}}\Cendnote{\textnormal{Schnitzler\pwindex{Schnitzler, Arthur 15.05.1862 – 21.10.1931@\textsc{Schnitzler, Arthur} (15.05.1862 – 21.10.1931), \emph{Schriftsteller, Mediziner}|pwk} und Olga Gussmann\pwindex{Schnitzler, Olga 17.01.1882 – 13.01.1970@\textsc{Schnitzler, Olga} (17.01.1882 – 13.01.1970), \emph{Schauspielerin, Sängerin}|pwk} reisten zwischen 28. 5. 1903 und 15. 6. 1903 nach Italien\oindex{Italien@\textbf{Italien}|pwk} und Südtirol\oindex{Suedtirol@\textbf{Südtirol}|pwk}.}}}\label{K_L03373-11h} habe ich mit Vergnügen vernommen; meine beſten Wünſche
               begleiten Euch nach dem ſchönen Süden.\pend
           \pstart
           \strikeout{D\textcolor{gray}{a}} Da Du ſicherlich Luſt bekommen wirſt, \label{K_L03373-23v}\edtext{mehr von\textsc{Wilde\pwindex{Wilde, Oscar 16.10.1854 – 30.11.1900@\textsc{Wilde, Oscar} (16.10.1854 – 30.11.1900), \emph{Schriftsteller}|pw}}}{\lemma{\textnormal{\emph{mehr vonWilde}}}\Cendnote{\textnormal{siehe Paul Goldmann an Arthur Schnitzler, 13. 5. [1903]}}}\label{K_L03373-23h} zu \strikeout{leſen,} leſen, ſo lies \label{K_L03373-32v}\edtext{»\textsc{Dorian {\pb}Grays}
                  Bildniß\pwindex{Wilde, Oscar 16.10.1854 – 30.11.1900@\textsc{Wilde, Oscar} (16.10.1854 – 30.11.1900), \emph{Schriftsteller}!Dorian Grays Bildnis1902@\strich\emph{Dorian Grays Bildnis} {[}1902{]}|pw}« (in der Überſetzung von \textsc{Greve\pwindex{Greve, Felix Paul 1879-02-14 – 1948-08-19@\textsc{Greve, Felix Paul} (1879-02-14 – 1948-08-19), \emph{Schriftsteller, Übersetzer}|pw}})}{\lemma{\textnormal{\emph{»Dorian … Greve)}}}\Cendnote{\textnormal{Oscar Wilde\pwindex{Wilde, Oscar 16.10.1854 – 30.11.1900@\textsc{Wilde, Oscar} (16.10.1854 – 30.11.1900), \emph{Schriftsteller}|pwk}: \emph{Dorian Grays Bildnis}\pwindex{Wilde, Oscar 16.10.1854 – 30.11.1900@\textsc{Wilde, Oscar} (16.10.1854 – 30.11.1900), \emph{Schriftsteller}!Dorian Grays Bildnis1902@\strich\emph{Dorian Grays Bildnis} {[}1902{]}|pwk}. Übers. v. Felix Paul Greve\pwindex{Greve, Felix Paul 1879-02-14 – 1948-08-19@\textsc{Greve, Felix Paul} (1879-02-14 – 1948-08-19), \emph{Schriftsteller, Übersetzer}|pwk}. Minden\oindex{Minden@\textbf{Minden}|pwk}: \emph{J. C. C. Bruns’ Verlag}\orgindex{J. C. C. Bruns@J. C. C. Bruns|pwk}
                        [1902]. Schnitzler\pwindex{Schnitzler, Arthur 15.05.1862 – 21.10.1931@\textsc{Schnitzler, Arthur} (15.05.1862 – 21.10.1931), \emph{Schriftsteller, Mediziner}|pwk} las \emph{The picture of Dorian
                     Gray}\pwindex{Wilde, Oscar 16.10.1854 – 30.11.1900@\textsc{Wilde, Oscar} (16.10.1854 – 30.11.1900), \emph{Schriftsteller}!picture of Dorian Gray1890@\strich\emph{The picture of Dorian Gray} {[}1890{]}|pwk} (1890) am 30. 6. 1904 in der Übersetzung\pwindex{Wilde, Oscar 16.10.1854 – 30.11.1900@\textsc{Wilde, Oscar} (16.10.1854 – 30.11.1900), \emph{Schriftsteller}!Dorian Gray1901@\strich\emph{Dorian Gray} {[}1901{]}|pwkv} von Johannes Gaulke\pwindex{Gaulke, Johannes 1869-07-25 – 1938-11-17@\textsc{Gaulke, Johannes} (1869-07-25 – 1938-11-17), \emph{Schriftsteller, Übersetzer}|pwk}: Oscar Wilde\pwindex{Wilde, Oscar 16.10.1854 – 30.11.1900@\textsc{Wilde, Oscar} (16.10.1854 – 30.11.1900), \emph{Schriftsteller}|pwk}: \emph{Dorian Gray}\pwindex{Wilde, Oscar 16.10.1854 – 30.11.1900@\textsc{Wilde, Oscar} (16.10.1854 – 30.11.1900), \emph{Schriftsteller}!Dorian Gray1901@\strich\emph{Dorian Gray} {[}1901{]}|pwk}. Übers. v. Johannes Gaulke\pwindex{Gaulke, Johannes 1869-07-25 – 1938-11-17@\textsc{Gaulke, Johannes} (1869-07-25 – 1938-11-17), \emph{Schriftsteller, Übersetzer}|pwk}. Leipzig\oindex{Leipzig@\textbf{Leipzig}|pwk}: \emph{Verlag von Max Spohr}\orgindex{Verlag Max Spohr@Verlag Max Spohr|pwk}
                        [1901] (vgl. A. S.: \emph{Lektüren}, England).}}}\label{K_L03373-32h}.\pend
           \pstart
           Ich habe nichts vergeſſen, nichts überwunden; habe nach meiner \label{K_L03373-56v}\edtext{Rückkehr aus Wien\oindex{Wien@\textbf{Wien}|pw}}{\lemma{\textnormal{\emph{Rückkehr aus Wien}}}\Cendnote{\textnormal{siehe Paul Goldmann an Arthur Schnitzler, 13. 5. [1903]}}}\label{K_L03373-56h} wieder eine ſchreckliche Kriſis durchgemacht; und verbringe mein Leben in
                  \label{K_L03373-77v}\edtext{Reue und Sehnſucht}{\lemma{\textnormal{\emph{Reue und Sehnſucht}}}\Cendnote{\textnormal{vermutlich Bezug auf Goldmann\pwindex{Goldmann, Paul 31.01.1865 – 25.09.1935@\textsc{Goldmann, Paul} (31.01.1865 – 25.09.1935), \emph{Schriftsteller, Journalist}|pwk}s Liebeskummer wegen Theodore Rottenberg\pwindex{Rottenberg, Theodore 1875-09-07 – 1945-04-05@\textsc{Rottenberg, Theodore} (1875-09-07 – 1945-04-05)|pwk}, die ihn Anfang 1903 verlassen hatte (vgl. Paul Goldmann an Arthur Schnitzler, 3. 1. [1903])}}}\label{K_L03373-77h}, hoffnungsloſer Sehnſucht{\dots}\pend
           \pstart
           Nächſtens mehr! Viele herzliche Grüße Dir und \textsc{Olga\pwindex{Schnitzler, Olga 17.01.1882 – 13.01.1970@\textsc{Schnitzler, Olga} (17.01.1882 – 13.01.1970), \emph{Schauspielerin, Sängerin}|pw}}! {\\[\baselineskip]}Dein \spacefill\mbox{Paul Goldmn}\pend
           \leftskip=0em{}\pstart
           \noindent{}\label{T_L03373-1v}\edtext{{\pb}Die \textsc{Triesch\pwindex{Triesch, Irene 13.04.1877 – 24.11.1964@\textsc{Triesch, Irene} (13.04.1877 – 24.11.1964), \emph{Schauspielerin}|pw}}, die geſtern{ }Abend auch da war, ſagte, daß ſie nach \label{K_L03373-7v}\edtext{Leipzig\oindex{Leipzig@\textbf{Leipzig}|pw}}{\lemma{\textnormal{\emph{Leipzig}}}\Cendnote{\textnormal{Das \emph{Deutsche Theater Berlin}\orgindex{Deutsches Theater Berlin@Deutsches Theater Berlin|pwk} hatte ein Gastspiel am Schauspielhaus Leipzig\oindex{Schauspielhaus Leipzig@\textbf{Schauspielhaus Leipzig}|pwk}. Die Premiere von \emph{Der Schleier der Beatrice}\pwindex{Schnitzler, Arthur 15.05.1862 – 21.10.1931@\textsc{Schnitzler, Arthur} (15.05.1862 – 21.10.1931), \emph{Schriftsteller, Mediziner}!Schleier der Beatrice. Schauspiel in fuenf Akten1900-12-01@\strich\emph{Der Schleier der Beatrice. Schauspiel in fünf Akten} {[}1900-12-01{]}|pwk}, mit Irene Triesch\pwindex{Triesch, Irene 13.04.1877 – 24.11.1964@\textsc{Triesch, Irene} (13.04.1877 – 24.11.1964), \emph{Schauspielerin}|pwk} in der Hauptrolle, fand am
                        24. 5. 1903 statt.}}}\label{K_L03373-7h} geht, um dort den
                     »Schleier der \textsc{Beatrice}\pwindex{Schnitzler, Arthur 15.05.1862 – 21.10.1931@\textsc{Schnitzler, Arthur} (15.05.1862 – 21.10.1931), \emph{Schriftsteller, Mediziner}!Schleier der Beatrice. Schauspiel in fuenf Akten1900-12-01@\strich\emph{Der Schleier der Beatrice. Schauspiel in fünf Akten} {[}1900-12-01{]}|pw}« zu ſpielen.}{\lemma{\textnormal{\emph{Die … ſpielen.}}}\Cendnote{\textnormal{kopfüber im oberen
                     rechten Eck der ersten Seite}}}\label{T_L03373-1h}\pend
           
         
         \endnumbering\mylabel{h}\end{ledgroupsized}\begin{anhang}\end{anhang}\newcommand{\dateiname}{L03373}\newcommand{\titel}{Paul Goldmann an Arthur Schnitzler, [22.?] 5. [1903]}\newcommand{\editorInnen}{Martin Anton Müller und Laura Untner}%% latex-leseansicht-abspann.tex
%% Abspann für die Leseansicht.
%% Der Schalter \ifkorrekturansicht ist bereits durch den Vorspann gesetzt.

%% latex-abspann.tex
%% Gemeinsamer Abspann für Korrekturansicht und Leseansicht.
%% Setzt den Schalter \ifkorrekturansicht voraus (gesetzt in den
%% einbindenden Dateien latex-korrekturansicht-abspann.tex bzw.
%% latex-leseansicht-abspann.tex).
%% ---------------------------------------------------------------

\normalsize

% Das esempio-Environment wird nur in der Leseansicht benötigt
\ifkorrekturansicht\else
\newenvironment{esempio}[3]%
{
    \vspace{1.5ex}
    \rlap{\underline{#1}}
    \par
    \setlength{\parindent}{0cm}
    \nopagebreak
    \leftskip=#2cm
    \rightskip=#3cm
}
{
    \par
}
\fi

\doendnotes{C}
\bigskip
\vfill

\clearpage

\footnotesize

\ifkorrekturansicht
  \lohead{\textsc{register}}
\fi

% theindex-Environment neu definieren ohne reledmac
\makeatletter
\renewenvironment{theindex}{%
  \ifkorrekturansicht
    \section*{\indexname}%
  \else
    \subsubsection*{Index der erwähnten Entitäten}%
  \fi
  \setlength{\parindent}{0pt}%
  \setlength{\parskip}{0pt plus 0.3pt}%
  \let\item\@idxitem
}{%
  \ifkorrekturansicht\clearpage\fi
}
\makeatother

\IfFileExists{\jobname-pw.ind}{\input{\jobname-pw.ind}}{}

% Quellenangabe nur in der Leseansicht
\ifkorrekturansicht\else
% Fallback-Definitionen, falls die .tex-Datei \titel etc. nicht gesetzt hat
\providecommand{\titel}{}
\providecommand{\editorInnen}{}
\providecommand{\dateiname}{\jobname}

\vspace{3cm}

\vfill

\footnotesize
\textsc{Quelle}: \titel. Herausgegeben von {\editorInnen}. In: \emph{Arthur Schnitzler: Briefwechsel mit Autorinnen und Autoren}.
 Digitale Edition, https://schnitzler-briefe.acdh.oeaw.ac.at/{\dateiname}.html (Stand \today)
\fi

\end{document}


      