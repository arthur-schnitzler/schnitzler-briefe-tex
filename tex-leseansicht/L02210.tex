%% latex-korrekturansicht-vorspann.tex
%% Vorspann für die Korrekturansicht.
%% Lädt die gemeinsame Datei latex-vorspann.tex mit gesetztem Schalter.

\newif\ifkorrekturansicht
\korrekturansichttrue

\input{../tex-inputs/latex-vorspann}


\section[Arthur Schnitzler an Robert Adam, 29. 6. 1915]{L02210 Arthur Schnitzler an Robert Adam, 29. 6. 1915}
\nopagebreak\mylabel{L02210v}
\rehead{ }\normalsize\beginnumbering\briefempfaengerindex{Adam, Robert@\textsc{Adam, Robert}!zzzSchnitzler, Arthur@\emph{von Arthur Schnitzler}!1915-06-291@{29. 6. 1915}|(be}
\toendnotes[C]{\smallbreak\pagebreak[2]}\Standort{DLA, 96.34.1/13.}
\physDesc{Briefkarte, , Umschlag, 906 Zeichen
\newline{}Handschrift: schwarze Tinte, lateinische Kurrent
\newline{}Versand: Stempel: »\nobreak{}Wien\nobreak{}«.  }\toendnotes[C]{\smallbreak}\pstart{}{\pb}\textcolor{gray}{\textbf{Dr. Arthur Schnitzler}}\pend{}\pstart{}\textcolor{gray}{\textbf{Wien XVIII. Sternwartestrasse 71\oindex{Sternwartestrasse 71@\textbf{Sternwartestraße 71}, \emph{Wohngebäude (K.WHS)}|pw}}}\pend{}{\bigskip}\pstart{}{\pb}Hrn Dr. Robert Adam Pollak,\pend{}\pstart{}Wien XII\oindex{XII., Meidling@\textbf{XII., Meidling}, \emph{A.ADM3}|pw}\pend{}\pstart{}Meidlinger Hauptstr 56\oindex{Meidlinger Hauptstrasse@\textbf{Meidlinger Hauptstraße}, \emph{Straße (K.STR)}|pw}\pend{}{\bigskip}\vspace{1em}
\pstart
           {\pb}\textcolor{gray}{\textbf{Dr. Arthur Schnitzler}}\hfill 29. 6. 191\textcolor{gray}{5}\pend
           
\pstart
           \textcolor{gray}{\textbf{Wien XVIII. Sternwartestrasse 71\oindex{Sternwartestrasse 71@\textbf{Sternwartestraße 71}, \emph{Wohngebäude (K.WHS)}|pw}}}\pend
           \vspace{0.5em}
\pstart
           verehrter Herr Doctor, es hat sich in all diesen Tagen nicht gefügt,
               daß ich den Leiter\pwindex{Thimig, Hugo 16.06.1854 – 24.09.1944@\textsc{Thimig, Hugo} (16.06.1854 – 24.09.1944), \emph{Theaterleiter/Theaterleiterin, Schauspieler/Schauspielerin}|pwv} des Burgtheaters\oindex{Burgtheater@\textbf{Burgtheater}, \emph{S.THTR}|pw} sprach; – doch hab ich mir erlaubt,
               den Regisseur und Schauspieler des Münchner
                  Hoftheaters\orgindex{Nationaltheater Muenchen@Nationaltheater München|pw}, meinen Schwager Albert
                  Steinrück\pwindex{Steinrueck, Albert 20.05.1872 – 11.02.1929@\textsc{Steinrück, Albert} (20.05.1872 – 11.02.1929), \emph{Schauspieler/Schauspielerin}|pw}, der über den Mangel an neuen Stücken klagte, auf Sie und Ihre drei
               Dramen (Abû Bekkr\pwindex{Geschichte des Alî ibn Bekkâr mit Schams an-Nahâr@\emph{Die Geschichte des Alî ibn Bekkâr mit Schams an-Nahâr}|pw}, Fremdling\pwindex{Fremde@\emph{Der Fremde}|pw} und das dritte\pwindex{Fatme@\emph{Fatme}|pwv}, dessen Name mir eben entfiel –) in gebührender
               Weise aufmerksam zu machen, {\pb}und ich würde Ihnen rathen,
               all das, unter Berufung auf mich an St.\pwindex{Steinrueck, Albert 20.05.1872 – 11.02.1929@\textsc{Steinrück, Albert} (20.05.1872 – 11.02.1929), \emph{Schauspieler/Schauspielerin}|pw}, d. h.
                  Partenkirchen, Villa Zufriedenheit\oindex{Villa Zufriedenheit@\textbf{Villa Zufriedenheit}, \emph{Gebäude (K.GBD)}|pw}
               abzusenden. – Die anderen Chancen verlier ich damit nicht aus dem Auge; aber wie
               schon gesagt, ich warte ein persönliches Zusammentreffen mit den betreffenden
               Partnern ab.\pend
           
\pstart
           Übermorgen fahr ich nach Altaussee (Villa
                  Annerl\oindex{Villa Annerl@\textbf{Villa Annerl}, \emph{Gebäude (K.GBD)}|pw}), denke im September daheim zu sein und hoffe Sie bald
               wiederzusehn.\pend
           
\pstart
           herzlichlich grüßend Ihr ergebner{\\[\baselineskip]}\spacefill\mbox{A. S.}\pend
           \leftskip=0em{}\selectlanguage{ngerman}\endnumbering\briefempfaengerindex{Adam, Robert@\textsc{Adam, Robert}!zzzSchnitzler, Arthur@\emph{von Arthur Schnitzler}!1915-06-291@{29. 6. 1915}|)be}\mylabel{L02210h}  \normalsize

\doendnotes{C}
\bigskip
\vfill

\clearpage

\footnotesize

\lohead{\textsc{register}}

% Definiere theindex-Environment komplett neu ohne reledmac
\makeatletter
\renewenvironment{theindex}{%
  \section*{\indexname}%
  \setlength{\parindent}{0pt}%
  \setlength{\parskip}{0pt plus 0.3pt}%
  \let\item\@idxitem
}{%
  \clearpage
}
\makeatother

\IfFileExists{\jobname-pw.ind}{\input{\jobname-pw.ind}}{}

\end{document}

      