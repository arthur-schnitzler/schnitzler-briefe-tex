%% latex-leseansicht-vorspann.tex
%% Vorspann für die Leseansicht.
%% Lädt die gemeinsame Datei latex-vorspann.tex mit nicht gesetztem Schalter.

\newif\ifkorrekturansicht
\korrekturansichtfalse

\input{../tex-inputs/latex-vorspann}


\section[ Arthur Schnitzler an Felix Salten, [12. 10. 1903?]]{L02986 Arthur Schnitzler an Felix Salten,  [12. 10. 1903?]}
\nopagebreak\mylabel{L02986v}
\rehead{ }\normalsize\beginnumbering\briefempfaengerindex{Salten, Felix@\textsc{Salten, Felix}!zzzSchnitzler, Arthur@\emph{von Arthur Schnitzler}!1903-10-121@{{[}12. 10. 1903?{]}}|(be}
\toendnotes[C]{\smallbreak\pagebreak[2]}
\correspDesc{Versand  durch Arthur Schnitzler am [12. 10. 1903?] in Wien
\newline{}Erhalt  durch Felix Salten am [12. 10. 1903?] in Wien}\toendnotes[C]{\smallbreak}
\Standort{Wienbibliothek im Rathaus, ZPH 1681, 2.1.516.}
\physDesc{Brief, 1 Blatt, 3 Seiten, 382 Zeichen
\newline{}Handschrift: Bleistift, deutsche Kurrent
\newline{}Ordnung: mit Bleistift von unbekannter Hand Nummerierung der Doppelseiten des Konvoluts:
                                    »3«–»4« }\toendnotes[C]{\smallbreak}
\pstart
           \raggedleft{}{\pb}\uline{Montag.}\pend
           \vspace{0.5em}
\pstart
           lieber,{ }\label{K_L02986-11v}\edtext{Hofmth.\pwindex{Hofmannsthal, Hugo von 1.\,2.\,1874 Wien – 15.\,7.\,1929 Rodaun@\textsc{Hofmannsthal, Hugo von} (1.\,2.\,1874 Wien – 15.\,7.\,1929 Rodaun), \emph{Schriftsteller}|pw}{ }ſagte}{\lemma{\textnormal{\emph{Hofmth. sagte}}}\Cendnote{\textnormal{Vermutlich hatte
                  Hofmannsthal\pwindex{Hofmannsthal, Hugo von 1.\,2.\,1874 Wien – 15.\,7.\,1929 Rodaun@\textsc{Hofmannsthal, Hugo von} (1.\,2.\,1874 Wien – 15.\,7.\,1929 Rodaun), \emph{Schriftsteller}|pwk} das zwei Tage zuvor, am 10. 10. 1903 gesagt.}}}\label{K_L02986-11} mir, dſs Sie \label{K_L02986-1v}\edtext{morgen Dinſtag den Schrei\pwindex{Salten, Felix 6.\,9.\,1869 Budapest – 8.\,10.\,1945 Zürich@\textsc{Salten, Felix} (6.\,9.\,1869 Budapest – 8.\,10.\,1945 Zürich), \emph{Schriftsteller, Journalist, Chefredakteur}!Schrei der Liebe. Novelle@\strich\emph{Der Schrei der Liebe. Novelle}|pw} vorleſen}{\lemma{\textnormal{\emph{morgen … vorlesen}}}\Cendnote{\textnormal{Da Saltens\pwindex{Salten, Felix 6.\,9.\,1869 Budapest – 8.\,10.\,1945 Zürich@\textsc{Salten, Felix} (6.\,9.\,1869 Budapest – 8.\,10.\,1945 Zürich), \emph{Schriftsteller, Journalist, Chefredakteur}|pwk} Antwortschreiben (XXXX Auszeichnungsfehler: Dokument L03347 nicht gefunden) von Schnitzler datiert wurde, kann auch dieser Brief auf den
                     XXXX Auszeichnungsfehler: Dokument L02986 nicht gefunden datiert
                  werden.}}}\label{K_L02986-1} werden – ich habe bisher von Ihnen keine Nachricht erhalten u denke
               an die Möglichkeit, dſs ein {\pb}Brief verloren
               gegangen wäre?\pend
           
\pstart
           Könnten Sie nicht an irgend einem Abend mit Otti\pwindex{Salten, Ottilie 7.\,3.\,1868 Prag – 22.\,6.\,1942 Zürich@\textsc{Salten, Ottilie} (7.\,3.\,1868 Prag – 22.\,6.\,1942 Zürich), \emph{Schauspielerin}|pw} bei uns \label{K_L02986-2v}\edtext{nachtmahlen}{\lemma{\textnormal{\emph{nachtmahlen}}}\Cendnote{\textnormal{Nicht zum Abendessen,
                  aber nachmittags sahen sie sich kurz darauf, am 18. 10. 1903.}}}\label{K_L02986-2}?
               Eſſen müſſen Sie ja doch irgendwo, und ich finde es mehr als aergerlich, {\pb}daſs man einander{ }ſo entſchwindet.\pend
           
\pstart
           Herzlichſt Ihr {\\[\baselineskip]}\spacefill\mbox{A.}\pend
           \leftskip=0em{}\selectlanguage{ngerman}\endnumbering\briefempfaengerindex{Salten, Felix@\textsc{Salten, Felix}!zzzSchnitzler, Arthur@\emph{von Arthur Schnitzler}!1903-10-121@{{[}12. 10. 1903?{]}}|)be}\mylabel{L02986h}  \newcommand{\dateiname}{L02986}\newcommand{\titel}{Arthur Schnitzler an Felix Salten, [12. 10. 1903?]}\newcommand{\editorInnen}{Martin Anton Müller und Laura Untner}%% latex-leseansicht-abspann.tex
%% Abspann für die Leseansicht.
%% Der Schalter \ifkorrekturansicht ist bereits durch den Vorspann gesetzt.

%% latex-abspann.tex
%% Gemeinsamer Abspann für Korrekturansicht und Leseansicht.
%% Setzt den Schalter \ifkorrekturansicht voraus (gesetzt in den
%% einbindenden Dateien latex-korrekturansicht-abspann.tex bzw.
%% latex-leseansicht-abspann.tex).
%% ---------------------------------------------------------------

\normalsize

% Das esempio-Environment wird nur in der Leseansicht benötigt
\ifkorrekturansicht\else
\newenvironment{esempio}[3]%
{
    \vspace{1.5ex}
    \rlap{\underline{#1}}
    \par
    \setlength{\parindent}{0cm}
    \nopagebreak
    \leftskip=#2cm
    \rightskip=#3cm
}
{
    \par
}
\fi

\doendnotes{C}
\bigskip
\vfill

\clearpage

\footnotesize

\ifkorrekturansicht
  \lohead{\textsc{register}}
\fi

% theindex-Environment neu definieren ohne reledmac
\makeatletter
\renewenvironment{theindex}{%
  \ifkorrekturansicht
    \section*{\indexname}%
  \else
    \subsubsection*{Index der erwähnten Entitäten}%
  \fi
  \setlength{\parindent}{0pt}%
  \setlength{\parskip}{0pt plus 0.3pt}%
  \let\item\@idxitem
}{%
  \ifkorrekturansicht\clearpage\fi
}
\makeatother

\IfFileExists{\jobname-pw.ind}{\input{\jobname-pw.ind}}{}

% Quellenangabe nur in der Leseansicht
\ifkorrekturansicht\else
% Fallback-Definitionen, falls die .tex-Datei \titel etc. nicht gesetzt hat
\providecommand{\titel}{}
\providecommand{\editorInnen}{}
\providecommand{\dateiname}{\jobname}

\vspace{3cm}

\vfill

\footnotesize
\textsc{Quelle}: \titel. Herausgegeben von {\editorInnen}. In: \emph{Arthur Schnitzler: Briefwechsel mit Autorinnen und Autoren}.
 Digitale Edition, https://schnitzler-briefe.acdh.oeaw.ac.at/{\dateiname}.html (Stand \today)
\fi

\end{document}


