%% latex-leseansicht-vorspann.tex
%% Vorspann für die Leseansicht.
%% Lädt die gemeinsame Datei latex-vorspann.tex mit nicht gesetztem Schalter.

\newif\ifkorrekturansicht
\korrekturansichtfalse

\input{../tex-inputs/latex-vorspann}


         
         \renewcommand{\erwaehntePersonen}{Personen: Hugo von Hofmannsthal, Felix Salten, Ottilie Salten}
         \renewcommand{\erwaehnteOrte}{Orte: Wien}
         \renewcommand{\erwaehnteWerke}{Werke: Der Schrei der Liebe. Novelle}
               \section[ Arthur Schnitzler an Felix Salten, {[}12. 10. 1903?{]}]{ Arthur Schnitzler an Felix Salten, {[}12. 10. 1903?{]}}\nopagebreak\mylabel{v}\rehead{ }\begin{ledgroupsized}[t]{13cm}\normalsize\beginnumbering\briefempfaengerindex{Salten, Felix@\textsc{Salten, Felix}!zzzSchnitzler, Arthur@\emph{von Arthur Schnitzler}!1903-10-121@{{[}12. 10. 1903?{]}}|(be} \toendnotes[C]{\smallbreak\pagebreak[2]} \Standort{Wienbibliothek im Rathaus, ZPH 1681, 2.1.516.}
\physDesc{Brief, 1 Blatt, 3 Seiten, 382 Zeichen
\newline{}Handschrift: Bleistift, deutsche Kurrent
\newline{}Ordnung: mit Bleistift von unbekannter Hand Nummerierung der Doppelseiten des Konvoluts:
                                    »3«–»4« }\toendnotes[C]{\smallbreak}\pstart
           \raggedleft{}{\pb}\uline{Montag.}\pend
           \pstart
           lieber,{ }\label{K_L02986-11v}\edtext{Hofmth.\pwindex{Hofmannsthal, Hugo von 1874-02-01 – 1929-07-15@\textsc{Hofmannsthal, Hugo von} (1874-02-01 – 1929-07-15), \emph{Schriftsteller}|pw} ſagte}{\lemma{\textnormal{\emph{Hofmth. ſagte}}}\Cendnote{\textnormal{Vermutlich
                  bereits zwei Tage zuvor, am A. S.: \emph{Tagebuch}, 10. 10. 1903.}}}\label{K_L02986-11h} mir, dſs Sie \label{K_L02986-1v}\edtext{morgen Dinſtag den Schrei\pwindex{Salten, Felix 06.09.1869 – 08.10.1945@\textsc{Salten, Felix} (06.09.1869 – 08.10.1945), \emph{Schriftsteller, Journalist}!Schrei der Liebe. Novelle1904-10-22@\strich\emph{Der Schrei der Liebe. Novelle} {[}1904-10-22{]}|pw} vorleſen}{\lemma{\textnormal{\emph{morgen … vorleſen}}}\Cendnote{\textnormal{Da Salten\pwindex{Salten, Felix 06.09.1869 – 08.10.1945@\textsc{Salten, Felix} (06.09.1869 – 08.10.1945), \emph{Schriftsteller, Journalist}|pwk}s Antwortschreiben (Felix Salten an Arthur Schnitzler, [12. 10. 1903]) von Schnitzler\pwindex{Schnitzler, Arthur 15.05.1862 – 21.10.1931@\textsc{Schnitzler, Arthur} (15.05.1862 – 21.10.1931), \emph{Schriftsteller, Mediziner}|pwk} datiert wurde, kann auch dieser Brief auf den
                     [12. 10. 1903?] datiert
                  werden.}}}\label{K_L02986-1h} werden – ich habe bisher von Ihnen keine Nachricht erhalten u denke
               an die Möglichkeit, dſs ein {\pb}Brief verloren
               gegangen wäre?\pend
           \pstart
           Könnten Sie nicht an irgend einem Abend mit Otti\pwindex{Salten, Ottilie 07.03.1868 – 22.06.1942@\textsc{Salten, Ottilie} (07.03.1868 – 22.06.1942), \emph{Schauspielerin}|pw} bei uns \label{K_L02986-2v}\edtext{nachtmahlen}{\lemma{\textnormal{\emph{nachtmahlen}}}\Cendnote{\textnormal{Nicht zum Abendessen,
                  aber nachmittags sahen sie sich kurz darauf, am 18. 10. 1903.}}}\label{K_L02986-2h}?
               Eſſen müſſen Sie ja doch irgendwo, und ich finde es mehr als aergerlich, {\pb}daſs man einander ſo entſchwindet.\pend
           \pstart
           Herzlichſt Ihr {\\[\baselineskip]}\spacefill\mbox{A.}\pend
           \leftskip=0em{}
         
         \endnumbering\mylabel{h}\end{ledgroupsized}  \newcommand{\dateiname}{L02986}\newcommand{\titel}{Arthur Schnitzler an Felix Salten, [12. 10. 1903?]}\newcommand{\editorInnen}{Martin Anton Müller und Laura Untner}%% latex-leseansicht-abspann.tex
%% Abspann für die Leseansicht.
%% Der Schalter \ifkorrekturansicht ist bereits durch den Vorspann gesetzt.

%% latex-abspann.tex
%% Gemeinsamer Abspann für Korrekturansicht und Leseansicht.
%% Setzt den Schalter \ifkorrekturansicht voraus (gesetzt in den
%% einbindenden Dateien latex-korrekturansicht-abspann.tex bzw.
%% latex-leseansicht-abspann.tex).
%% ---------------------------------------------------------------

\normalsize

% Das esempio-Environment wird nur in der Leseansicht benötigt
\ifkorrekturansicht\else
\newenvironment{esempio}[3]%
{
    \vspace{1.5ex}
    \rlap{\underline{#1}}
    \par
    \setlength{\parindent}{0cm}
    \nopagebreak
    \leftskip=#2cm
    \rightskip=#3cm
}
{
    \par
}
\fi

\doendnotes{C}
\bigskip
\vfill

\clearpage

\footnotesize

\ifkorrekturansicht
  \lohead{\textsc{register}}
\fi

% theindex-Environment neu definieren ohne reledmac
\makeatletter
\renewenvironment{theindex}{%
  \ifkorrekturansicht
    \section*{\indexname}%
  \else
    \subsubsection*{Index der erwähnten Entitäten}%
  \fi
  \setlength{\parindent}{0pt}%
  \setlength{\parskip}{0pt plus 0.3pt}%
  \let\item\@idxitem
}{%
  \ifkorrekturansicht\clearpage\fi
}
\makeatother

\IfFileExists{\jobname-pw.ind}{\input{\jobname-pw.ind}}{}

% Quellenangabe nur in der Leseansicht
\ifkorrekturansicht\else
% Fallback-Definitionen, falls die .tex-Datei \titel etc. nicht gesetzt hat
\providecommand{\titel}{}
\providecommand{\editorInnen}{}
\providecommand{\dateiname}{\jobname}

\vspace{3cm}

\vfill

\footnotesize
\textsc{Quelle}: \titel. Herausgegeben von {\editorInnen}. In: \emph{Arthur Schnitzler: Briefwechsel mit Autorinnen und Autoren}.
 Digitale Edition, https://schnitzler-briefe.acdh.oeaw.ac.at/{\dateiname}.html (Stand \today)
\fi

\end{document}


      