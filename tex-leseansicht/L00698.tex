%% latex-leseansicht-vorspann.tex
%% Vorspann für die Leseansicht.
%% Lädt die gemeinsame Datei latex-vorspann.tex mit nicht gesetztem Schalter.

\newif\ifkorrekturansicht
\korrekturansichtfalse

\input{../tex-inputs/latex-vorspann}


\section[Arthur Schnitzler an Hugo von Hofmannsthal, 9. 7. 1897]{L00698 Arthur Schnitzler an Hugo von Hofmannsthal, 9. 7. 1897}
\nopagebreak\mylabel{L00698v}
\rehead{ }\normalsize\beginnumbering\briefempfaengerindex{Hofmannsthal, Hugo von@\textsc{Hofmannsthal, Hugo von}!zzzSchnitzler, Arthur@\emph{von Arthur Schnitzler}!1897-07-093@{9. 7. 1897}|(be}
\toendnotes[C]{\smallbreak\pagebreak[2]}
\correspDesc{Versand  durch Arthur Schnitzler am 9. 7. 1897 in Bad Ischl
\newline{}Erhalt  durch Hugo von Hofmannsthal im Zeitraum [10. 7. 1897
                  – 14. 7. 1897?] in Wien}\toendnotes[C]{\smallbreak}
\Standort{FDH, Hs-30885,60.}
\physDesc{Brief, 1 Blatt, 4 Seiten, 1356 Zeichen
\newline{}Handschrift: schwarze Tinte, deutsche Kurrent}
\buchAbdrucke{\weitereDrucke{Hugo von Hofmannsthal, Arthur Schnitzler: \emph{Briefwechsel}. Herausgegeben von Therese Nickl und Heinrich Schnitzler. Frankfurt am Main: \emph{S. Fischer} 1964, S. 90–91.} }
\pstart
           \raggedleft{}{\pb}\textsc{Ischl}\oindex{Bad Ischl@\textbf{Bad Ischl}|pw}, 9. 7. 97.\pend
           \vspace{0.5em}
\pstart
           Mein lieber Hugo, überallher ko{\geminationm}en nur
               ärgerliche Nachrichten, insbeſonders dieſe Schwierigkeiten mit der Wien\oindex{Wien@\textbf{Wien}, \emph{Verwaltungsgebiet}|pw}er Wohnung{ }ſtören mich{ }ſehr. Ich werde wohl früher nach Wien\oindex{Wien@\textbf{Wien}, \emph{Verwaltungsgebiet}|pw} fahren u gleich definitiv in Wien\oindex{Wien@\textbf{Wien}, \emph{Verwaltungsgebiet}|pw} bleiben.\pend
           
\pstart
           Jetzt ka{\geminationn} ich nicht weg von hier, es wäre auch eine
               wahrſcheinlich nutzloſe Hin u Herhetzerei. {\pb}Bitte lieber
               Hugo, ginge das, daſs wir unſer Salzburg\oindex{Salzburg@\textbf{Salzburg}, \emph{Verwaltungsgebiet}|pw}er Zuſa{\geminationm}enſein um ein paar Tage früher hätten? Daſs Sie{ }ſtatt
               am 23.{ }ſchon am 22. oder noch lieber am
                  21. in S.\oindex{Salzburg@\textbf{Salzburg}, \emph{Verwaltungsgebiet}|pw} wären, \textsc{resp.} ich Sie in \textsc{Bruck}\oindex{Bruck an der Großglocknerstraße@\textbf{Bruck an der Großglocknerstraße}, \emph{Hauptstadt}|pw}-\textsc{Fusch}\oindex{Bad Fusch@\textbf{Bad Fusch}|pw} abholte? –\pend
           
\pstart
           Mit Poldi Andrian\pwindex{Andrian-Werburg, Leopold von 9.\,5.\,1875 Berlin – 19.\,11.\,1951 Fribourg@\textsc{Andrian-Werburg, Leopold von} (9.\,5.\,1875 Berlin – 19.\,11.\,1951 Fribourg), \emph{Schriftsteller, Diplomat}|pw} wirds hoffentlich (dieſes
               »hoffentlich« kommt nicht nur aus Bequemlichkeit{ }ſondern auch aus »ärztlicher
               Einſicht« her) bald {\pb}wieder beſſer{ }ſein. Jetzt gleich nach
                  Wien\oindex{Wien@\textbf{Wien}, \emph{Verwaltungsgebiet}|pw} zu fahren wäre mir eine rechte
               Unannehmlichkeit, und wirklich nöthig iſt’s ja gewiſs nicht. Schreiben Sie mir aber
               doch, wenn Sie können, näheres! –\pend
           
\pstart
           – Könnten Einem doch nur alle äußeren Sachen abgenommen werden. Es gibt ja{ }ſoviel
               Leute, denen das{ }ſo viel Freude macht und die \textcolor{gray}{nur} dadurch, daß ich
               es äußere, ich {\pb}meine{[},{]} adminiſtrative
               Sachen gibt, die{ }ſie zu beſorgen haben, zum Bewußtſein ihrer Exiſtenz kommen; – ließe{ }ſich das nicht irgendwie vertheilen? Ich{ }ſtelle mir ein Secretariat, eine Agentur im
               großen Stile vor, wo man alles findet, we{\geminationn} man nur in
               zehn Worten mittheilt: dieſe oder jene Schwierigkeit habe ich.\pend
           \pstart – Auf Wiederſehen. Herzliche Grüße! Ihr \spacefill\mbox{Arthur.}\pend{}\selectlanguage{ngerman}\endnumbering\briefempfaengerindex{Hofmannsthal, Hugo von@\textsc{Hofmannsthal, Hugo von}!zzzSchnitzler, Arthur@\emph{von Arthur Schnitzler}!1897-07-093@{9. 7. 1897}|)be}\mylabel{L00698h}  \newcommand{\dateiname}{L00698}\newcommand{\titel}{Arthur Schnitzler an Hugo von Hofmannsthal, 9. 7. 1897}\newcommand{\editorInnen}{Martin Anton Müller und Gerd-Hermann Susen}%% latex-leseansicht-abspann.tex
%% Abspann für die Leseansicht.
%% Der Schalter \ifkorrekturansicht ist bereits durch den Vorspann gesetzt.

%% latex-abspann.tex
%% Gemeinsamer Abspann für Korrekturansicht und Leseansicht.
%% Setzt den Schalter \ifkorrekturansicht voraus (gesetzt in den
%% einbindenden Dateien latex-korrekturansicht-abspann.tex bzw.
%% latex-leseansicht-abspann.tex).
%% ---------------------------------------------------------------

\normalsize

% Das esempio-Environment wird nur in der Leseansicht benötigt
\ifkorrekturansicht\else
\newenvironment{esempio}[3]%
{
    \vspace{1.5ex}
    \rlap{\underline{#1}}
    \par
    \setlength{\parindent}{0cm}
    \nopagebreak
    \leftskip=#2cm
    \rightskip=#3cm
}
{
    \par
}
\fi

\doendnotes{C}
\bigskip
\vfill

\clearpage

\footnotesize

\ifkorrekturansicht
  \lohead{\textsc{register}}
\fi

% theindex-Environment neu definieren ohne reledmac
\makeatletter
\renewenvironment{theindex}{%
  \ifkorrekturansicht
    \section*{\indexname}%
  \else
    \subsubsection*{Index der erwähnten Entitäten}%
  \fi
  \setlength{\parindent}{0pt}%
  \setlength{\parskip}{0pt plus 0.3pt}%
  \let\item\@idxitem
}{%
  \ifkorrekturansicht\clearpage\fi
}
\makeatother

\IfFileExists{\jobname-pw.ind}{\input{\jobname-pw.ind}}{}

% Quellenangabe nur in der Leseansicht
\ifkorrekturansicht\else
% Fallback-Definitionen, falls die .tex-Datei \titel etc. nicht gesetzt hat
\providecommand{\titel}{}
\providecommand{\editorInnen}{}
\providecommand{\dateiname}{\jobname}

\vspace{3cm}

\vfill

\footnotesize
\textsc{Quelle}: \titel. Herausgegeben von {\editorInnen}. In: \emph{Arthur Schnitzler: Briefwechsel mit Autorinnen und Autoren}.
 Digitale Edition, https://schnitzler-briefe.acdh.oeaw.ac.at/{\dateiname}.html (Stand \today)
\fi

\end{document}


