%% latex-korrekturansicht-vorspann.tex
%% Vorspann für die Korrekturansicht.
%% Lädt die gemeinsame Datei latex-vorspann.tex mit gesetztem Schalter.

\newif\ifkorrekturansicht
\korrekturansichttrue

\input{../tex-inputs/latex-vorspann}


\section[ Paul Goldmann an Arthur Schnitzler, 28. 2. 1903]{L03367 Paul Goldmann an Arthur Schnitzler, 28. 2. 1903}
\nopagebreak\mylabel{L03367v}
\rehead{ }\normalsize\beginnumbering\briefempfaengerindex{Schnitzler, Arthur@\textsc{Schnitzler, Arthur}!zzzGoldmann, Paul@\emph{von Paul Goldmann}!1903-02-281@{28. 2. 1903}|(be}
\toendnotes[C]{\smallbreak\pagebreak[2]}\Standort{DLA, A:Schnitzler, HS.NZ85.1.3173.}
\physDesc{Postkarte, 311 Zeichen
\newline{}Handschrift: 1) blaue Tinte, deutsche Kurrent\hspace{1em}2) blaue Tinte, lateinische Kurrent (\noindent{}Adresse)\hspace{1em}
\newline{}Versand: Stempel: »\nobreak{}\oindex{Berlin@\textbf{Berlin}, \emph{P.PPLC}|pwk}Berlin, S. W. 11, 28. 2. 03., 11\textsuperscript{20} V.\nobreak{}«. Stempel: »\nobreak{}\oindex{Berlin@\textbf{Berlin}, \emph{P.PPLC}|pwk}Berlin, S. W. 11 b, 28. 2. 03., 11–12 V.\nobreak{}«. Stempel: »\nobreak{}\oindex{Berlin@\textbf{Berlin}, \emph{P.PPLC}|pwk}Berlin, W. P9 (R6), 28 II 03, 11\textsuperscript{30} V.\nobreak{}«.  
\newline{}Schnitzler: mit Bleistift datiert: »28/2 {[}1{]}90\textcolor{gray}{3}.« }\toendnotes[C]{\smallbreak}\pstart{}{\pb}Herrn\pend{}\pstart{}Dr. Arthur Schnitzler\pend{}\pstart{}Palasthotel\oindex{Palasthotel Berlin@\textbf{Palasthotel Berlin}, \emph{Hotel (K.HTL)}|pw}\pend{}{\bigskip}\vspace{1em}
\pstart
           {\pb}Samſtag.\pend
           
\pstart{}Liebſter Freund,\pend\vspace{0.5em}
\pstart
           Ich werde heut{ }Abend zwiſchen 10 u. 10 ½ Uhr bei \textsc{Josty\oindex{Cafe Josty@\textbf{Café Josty}, \emph{Kaffeehaus (K.KAF)}|pw}}, \textsc{Potsdamer Platz\oindex{Potsdamer Platz@\textbf{Potsdamer Platz}, \emph{Platz (K.PLT)}|pw}}, nachſchauen, ob Du \label{K_L03367-1v}\edtext{dort
                  biſt}{\lemma{\textnormal{\emph{dort
                  biſt}}}\Cendnote{\textnormal{Goldmann\pwindex{Goldmann, Paul 31.01.1865 – 25.09.1935@\textsc{Goldmann, Paul} (31.01.1865 – 25.09.1935), \emph{Schriftsteller/Schriftstellerin, Journalist/Journalistin}|pwk} und Schnitzler waren – womöglich in Folge dieser Verabredung –
                  am 28. 2. 1902 bei
                     Elisabeth Gussmann\pwindex{Steinrueck, Elisabeth 19.11.1885 – 07.04.1920@\textsc{Steinrück, Elisabeth} (19.11.1885 – 07.04.1920)|pwk}.}}}\label{K_L03367-1}. \uline{Du biſt}{ }\strikeout{aber}{ }\uline{aber nicht im Mindeſten gebunden.} Treffen wir uns
                  heut nicht, ſo erwarte ich morgen{ }Vormittag bis 11 ½ Uhr eine Verſtändigung\pend
           
\pstart
           Herzlichſt Dein {\\[\baselineskip]}\spacefill\mbox{P. G.}\pend
           \leftskip=0em{}\selectlanguage{ngerman}\endnumbering\briefempfaengerindex{Schnitzler, Arthur@\textsc{Schnitzler, Arthur}!zzzGoldmann, Paul@\emph{von Paul Goldmann}!1903-02-281@{28. 2. 1903}|)be}\mylabel{L03367h}  \normalsize

\doendnotes{C}
\bigskip
\vfill

\clearpage

\footnotesize

\lohead{\textsc{register}}

% Definiere theindex-Environment komplett neu ohne reledmac
\makeatletter
\renewenvironment{theindex}{%
  \section*{\indexname}%
  \setlength{\parindent}{0pt}%
  \setlength{\parskip}{0pt plus 0.3pt}%
  \let\item\@idxitem
}{%
  \clearpage
}
\makeatother

\IfFileExists{\jobname-pw.ind}{\input{\jobname-pw.ind}}{}

\end{document}

      