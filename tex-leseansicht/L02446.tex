%% latex-leseansicht-vorspann.tex
%% Vorspann für die Leseansicht.
%% Lädt die gemeinsame Datei latex-vorspann.tex mit nicht gesetztem Schalter.

\newif\ifkorrekturansicht
\korrekturansichtfalse

\input{../tex-inputs/latex-vorspann}


\section[Arthur Schnitzler an Georg Brandes, 2[7?]. 7. 1925]{L02446 Arthur Schnitzler an Georg Brandes, 2[7?]. 7. 1925}
\nopagebreak\mylabel{L02446v}
\rehead{ }\normalsize\beginnumbering\briefempfaengerindex{Brandes, Georg@\textsc{Brandes, Georg}!zzzSchnitzler, Arthur@\emph{von Arthur Schnitzler}!1925-07-271@{2[7?]. 7. 1925}|(be}
\toendnotes[C]{\smallbreak\pagebreak[2]}
\correspDesc{Versand  durch Arthur Schnitzler am 2[7?]. 7. 1925 in Wien
\newline{}Erhalt  durch Georg Brandes am 29. 7. [1925] in Kopenhagen}\toendnotes[C]{\smallbreak}
\Standort{Kopenhagen, Det Kongelige Bibliotek, Georg Brandes Arkiv, box 125.}
\physDesc{Bildpostkarte, 471 Zeichen
\newline{}Handschrift: schwarze Tinte, lateinische Kurrent
\newline{}Versand: 1) Stempel: »\nobreak{}\oindex{Wien@\textbf{Wien}, \emph{Verwaltungsgebiet}|pwk}Wien\nobreak{}«.   2) Stempel: »\nobreak{}\oindex{Kopenhagen@\textbf{Kopenhagen}, \emph{Hauptstadt}|pwk}\textcolor{gray}{Kjobenhav}n, 29. 7. {[}1925{]}, 20M\nobreak{}«. 
\newline{}Ordnung: mit Bleistift von unbekannter Hand nummeriert:
                                    »53« und datiert: »29-7-25 (?)« }
\buchAbdrucke{\weitereDrucke{1) Georg Brandes, Arthur Schnitzler: \emph{Ein Briefwechsel}. Herausgegeben von Kurt Bergel. Bern: \emph{Francke} 1956, S. 150.} \weitereDrucke{2) Arthur Schnitzler: \emph{Briefe 1913–1931}. Herausgegeben von Peter Michael Braunwarth, Richard Miklin, Susanne Pertlik und Heinrich Schnitzler. Frankfurt am Main: \emph{S. Fischer} 1984, S. 417.} }\toendnotes[C]{\smallbreak}\pstart{}{\pb}\label{T_L02446-1v}\edtext{\textcolor{gray}{\textbf{A. S.}}}{\lemma{\textnormal{\emph{A. S.}}}\Cendnote{\textnormal{ovaler Absenderkleber über die
                     Kartenkante, teilweise über den Text}}}\label{T_L02446-1}\pend{}\pstart{}\textcolor{gray}{\textbf{WIEN, XVIII.}}\oindex{XVIII., Währing@\textbf{XVIII., Währing}, \emph{Verwaltungsgebiet}|pw}\pend{}\pstart{}\textcolor{gray}{\textbf{STERNWARTESTR. 71}}\oindex{Wien@\textbf{Wien}!XVIII., Währing@\textbf{XVIII., Währing}!Sternwartestraße 71@\textbf{Sternwartestraße 71}, \emph{Wohngebäude}|pw}\pend{}{\bigskip}\pstart{}{\pb}Herrn\pend{}\pstart{}Georg Brandes\pend{}\pstart{}Kopenhagen\oindex{Kopenhagen@\textbf{Kopenhagen}, \emph{Hauptstadt}|pw}\pend{}\pstart{}Daenemark\oindex{Dänemark@\textbf{Dänemark}|pw}\pend{}{\bigskip}
\pstart
           \noindent{}\centering{}{\pb}{[}Sternwartestraße 71\oindex{Wien@\textbf{Wien}!XVIII., Währing@\textbf{XVIII., Währing}!Sternwartestraße 71@\textbf{Sternwartestraße 71}, \emph{Wohngebäude}|pw}{]}\pend
           \vspace{1em}
\pstart
           \noindent{}{\pb}Herzlichen \damage{Dank für} Ihre liebe Karte. Ihre Bitte es niemandem zu sagen, daß die Menschheit eine
               abscheuliche Bande, ko{\geminationm}t leider verspätet. Weiſs der
               Teufel durch welche Indiscretion – die Sache hat sich herumgesprochen!\pend
           
\pstart
           – Ich bin noch in Wien\oindex{Wien@\textbf{Wien}, \emph{Verwaltungsgebiet}|pw}, arbeite allerlei, denke
               Ihrer in alter inniger Freundschaft und bitte Sie, mich und {\pb}dieses Haus in gütiger Erinnerung zu behalten\pend
           
\pstart
           Mit tausend Grüßen{\\[\baselineskip]}Ihr getreuer\spacefill\mbox{Arthur Schnitzler}\pend
           \leftskip=0em{}\selectlanguage{ngerman}\endnumbering\briefempfaengerindex{Brandes, Georg@\textsc{Brandes, Georg}!zzzSchnitzler, Arthur@\emph{von Arthur Schnitzler}!1925-07-271@{2[7?]. 7. 1925}|)be}\mylabel{L02446h}  \newcommand{\dateiname}{L02446}\newcommand{\titel}{Arthur Schnitzler an Georg Brandes, 2[7?]. 7. 1925}\newcommand{\editorInnen}{Martin Anton Müller und Gerd-Hermann Susen}%% latex-leseansicht-abspann.tex
%% Abspann für die Leseansicht.
%% Der Schalter \ifkorrekturansicht ist bereits durch den Vorspann gesetzt.

%% latex-abspann.tex
%% Gemeinsamer Abspann für Korrekturansicht und Leseansicht.
%% Setzt den Schalter \ifkorrekturansicht voraus (gesetzt in den
%% einbindenden Dateien latex-korrekturansicht-abspann.tex bzw.
%% latex-leseansicht-abspann.tex).
%% ---------------------------------------------------------------

\normalsize

% Das esempio-Environment wird nur in der Leseansicht benötigt
\ifkorrekturansicht\else
\newenvironment{esempio}[3]%
{
    \vspace{1.5ex}
    \rlap{\underline{#1}}
    \par
    \setlength{\parindent}{0cm}
    \nopagebreak
    \leftskip=#2cm
    \rightskip=#3cm
}
{
    \par
}
\fi

\doendnotes{C}
\bigskip
\vfill

\clearpage

\footnotesize

\ifkorrekturansicht
  \lohead{\textsc{register}}
\fi

% theindex-Environment neu definieren ohne reledmac
\makeatletter
\renewenvironment{theindex}{%
  \ifkorrekturansicht
    \section*{\indexname}%
  \else
    \subsubsection*{Index der erwähnten Entitäten}%
  \fi
  \setlength{\parindent}{0pt}%
  \setlength{\parskip}{0pt plus 0.3pt}%
  \let\item\@idxitem
}{%
  \ifkorrekturansicht\clearpage\fi
}
\makeatother

\IfFileExists{\jobname-pw.ind}{\input{\jobname-pw.ind}}{}

% Quellenangabe nur in der Leseansicht
\ifkorrekturansicht\else
% Fallback-Definitionen, falls die .tex-Datei \titel etc. nicht gesetzt hat
\providecommand{\titel}{}
\providecommand{\editorInnen}{}
\providecommand{\dateiname}{\jobname}

\vspace{3cm}

\vfill

\footnotesize
\textsc{Quelle}: \titel. Herausgegeben von {\editorInnen}. In: \emph{Arthur Schnitzler: Briefwechsel mit Autorinnen und Autoren}.
 Digitale Edition, https://schnitzler-briefe.acdh.oeaw.ac.at/{\dateiname}.html (Stand \today)
\fi

\end{document}


