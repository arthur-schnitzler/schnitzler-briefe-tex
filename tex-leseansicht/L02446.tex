%% latex-korrekturansicht-vorspann.tex
%% Vorspann für die Korrekturansicht.
%% Lädt die gemeinsame Datei latex-vorspann.tex mit gesetztem Schalter.

\newif\ifkorrekturansicht
\korrekturansichttrue

\input{../tex-inputs/latex-vorspann}


\section[Arthur Schnitzler an Georg Brandes, 2{[}7?{]}. 7. 1925]{L02446 Arthur Schnitzler an Georg Brandes, 2{[}7?{]}. 7. 1925}
\nopagebreak\mylabel{L02446v}
\rehead{ }\normalsize\beginnumbering\briefempfaengerindex{Brandes, Georg@\textsc{Brandes, Georg}!zzzSchnitzler, Arthur@\emph{von Arthur Schnitzler}!1925-07-271@{2{[}7?{]}. 7. 1925}|(be}
\toendnotes[C]{\smallbreak\pagebreak[2]}\Standort{Kopenhagen, Det Kongelige Bibliotek, Georg Brandes Arkiv, box 125.}
\physDesc{Bildpostkarte, 471 Zeichen
\newline{}Handschrift: schwarze Tinte, lateinische Kurrent
\newline{}Versand: 1) Stempel: »\nobreak{}Wien\nobreak{}«.   2) Stempel: »\nobreak{}\oindex{Kopenhagen@\textbf{Kopenhagen}, \emph{P.PPLC}|pwk}\textcolor{gray}{Kjobenhav}n, 29. 7. {[}1925{]}, 20M\nobreak{}«. 
\newline{}Ordnung: mit Bleistift von unbekannter Hand nummeriert:
                                    »53« und datiert: »29-7-25 (?)« }
\buchAbdrucke{\weitereDrucke{1) Georg Brandes, Arthur Schnitzler: \emph{Ein Briefwechsel}. Bern: \emph{Francke} 1956, S. 150.} \weitereDrucke{2) Arthur Schnitzler: \emph{Briefe 1913–1931}. Frankfurt am Main: \emph{S. Fischer} 1984, S. 417.} }\toendnotes[C]{\smallbreak}\pstart{}{\pb}\label{T_L02446-1v}\edtext{\textcolor{gray}{\textbf{A. S.}}}{\lemma{\textnormal{\emph{A. S.}}}\Cendnote{\textnormal{ovaler Absenderkleber über die
                     Kartenkante, teilweise über den Text}}}\label{T_L02446-1}\pend{}\pstart{}\textcolor{gray}{\textbf{WIEN, XVIII.}}\oindex{XVIII., Waehring@\textbf{XVIII., Währing}, \emph{A.ADM3}|pw}\pend{}\pstart{}\textcolor{gray}{\textbf{STERNWARTESTR. 71}}\oindex{Sternwartestrasse 71@\textbf{Sternwartestraße 71}, \emph{Wohngebäude (K.WHS)}|pw}\pend{}{\bigskip}\pstart{}{\pb}Herrn\pend{}\pstart{}Georg Brandes\pend{}\pstart{}Kopenhagen\oindex{Kopenhagen@\textbf{Kopenhagen}, \emph{P.PPLC}|pw}\pend{}\pstart{}Daenemark\oindex{Daenemark@\textbf{Dänemark}, \emph{A.PCLI}|pw}\pend{}{\bigskip}
\pstart
           \noindent{}\centering{}{\pb}{[}Sternwartestraße 71\oindex{Sternwartestrasse 71@\textbf{Sternwartestraße 71}, \emph{Wohngebäude (K.WHS)}|pw}{]}\pend
           \vspace{1em}
\pstart
           \noindent{}{\pb}Herzlichen \damage{Dank für} Ihre liebe Karte. Ihre Bitte es niemandem zu sagen, daß die Menschheit eine
               abscheuliche Bande, ko{\geminationm}t leider verspätet. Weiſs der
               Teufel durch welche Indiscretion – die Sache hat sich herumgesprochen!\pend
           
\pstart
           – Ich bin noch in Wien\oindex{Wien@\textbf{Wien}, \emph{A.ADM2}|pw}, arbeite allerlei, denke
               Ihrer in alter inniger Freundschaft und bitte Sie, mich und {\pb}dieses Haus in gütiger Erinnerung zu behalten\pend
           
\pstart
           Mit tausend Grüßen{\\[\baselineskip]}Ihr getreuer\spacefill\mbox{Arthur Schnitzler}\pend
           \leftskip=0em{}\selectlanguage{ngerman}\endnumbering\briefempfaengerindex{Brandes, Georg@\textsc{Brandes, Georg}!zzzSchnitzler, Arthur@\emph{von Arthur Schnitzler}!1925-07-271@{2{[}7?{]}. 7. 1925}|)be}\mylabel{L02446h}  \normalsize

\doendnotes{C}
\bigskip
\vfill

\clearpage

\footnotesize

\lohead{\textsc{register}}

% Definiere theindex-Environment komplett neu ohne reledmac
\makeatletter
\renewenvironment{theindex}{%
  \section*{\indexname}%
  \setlength{\parindent}{0pt}%
  \setlength{\parskip}{0pt plus 0.3pt}%
  \let\item\@idxitem
}{%
  \clearpage
}
\makeatother

\IfFileExists{\jobname-pw.ind}{\input{\jobname-pw.ind}}{}

\end{document}

      