%% latex-leseansicht-vorspann.tex
%% Vorspann für die Leseansicht.
%% Lädt die gemeinsame Datei latex-vorspann.tex mit nicht gesetztem Schalter.

\newif\ifkorrekturansicht
\korrekturansichtfalse

\input{../tex-inputs/latex-vorspann}


               \section[Arthur Schnitzler an Richard Beer-Hofmann, {[}18. 7. 1894?{]}]{ Arthur Schnitzler an Richard Beer-Hofmann, {[}18. 7. 1894?{]}}\nopagebreak\mylabel{v}\rehead{ }\begin{ledgroupsized}[t]{13cm}\normalsize\beginnumbering\briefempfaengerindex{Beer-Hofmann, Richard@\textsc{Beer-Hofmann, Richard}!zzzSchnitzler, Arthur@\emph{von Arthur Schnitzler}!1894-07-183@{{[}18. 7. 1894?{]}}|(be} \toendnotes[C]{\smallbreak\pagebreak[2]} \Standort{YCGL, MSS 31.}
\physDesc{Brief, 1 Blatt, 3 Seiten, Umschlag
\newline{}Handschrift: Bleistift, deutsche Kurrent\newline{}Versand: ohne postalischen Übermittlungsvermerk }\pstart{}{\pb}Herrn \textsc{Dr. Richard Beer
                     Hofmann}\pend{}\pstart{}\textsc{Ischl\oindex{Bad Ischl@\textbf{Bad Ischl}|pw}}\pend{}\pstart{}\textsc{Egelmoos 22}\oindex{Eglmoosgasse@\textbf{Eglmoosgasse}|pw}.\pend{}{\bigskip}\pstart
           \noindent{}{\pb}Lieber Richard! –Ich wüßt nicht, warum Salzburg\oindex{Salzburg@\textbf{Salzburg}|pw} ganz ins Waſſer fallen ſoll, weil Bahr\pwindex{Bahr, Hermann 19.07.1863 – 15.01.1934@\textsc{Bahr, Hermann} (19.07.1863 – 15.01.1934), \emph{Schriftsteller, Kritiker}|pw} keine Zeit hat. Auch hat Hugo\pwindex{Hofmannsthal, Hugo von 01.02.1874 – 15.07.1929@\textsc{Hofmannsthal, Hugo von} (01.02.1874 – 15.07.1929), \emph{Schriftsteller}|pw}
               ziemlich ſicher zugeſagt. – Ich fahr jedenfalls über Salzburg\oindex{Salzburg@\textbf{Salzburg}|pw} zurück. – Ich antworte dem {\pb}Bahr\pwindex{Bahr, Hermann 19.07.1863 – 15.01.1934@\textsc{Bahr, Hermann} (19.07.1863 – 15.01.1934), \emph{Schriftsteller, Kritiker}|pw} natürlich, daſs ich Samſtag
               noch hier bin. Ich werd wohl Sonntag wegfahren. –\pend
           \pstart
           Heut geh ich zwiſchen 5 u 6 zu Ornſtein\pwindex{Ornstein, Sophie 26.9.1844 – 18.11.1906@\textsc{Ornstein, Sophie} (26.9.1844 – 18.11.1906)|pw}\pwindex{Ornstein, Wilhelm 22.2.1831 – 7.11.1908@\textsc{Ornstein, Wilhelm} (22.2.1831 – 7.11.1908), \emph{Kaufmann}|pw}{ }\introOben{}(Gina Z.\pwindex{Zeisler, Regine Januar 1864 – 1940-09-03@\textsc{Zeisler, Regine} (Januar 1864 – 1940-09-03), \emph{Pianistin}|pw})\introOben{}. Ich glaube, dſs ich dann
               zwiſchen 7 u ½ 8 auf die \textsc{Esplan.}\oindex{Esplanade@\textbf{Esplanade}|pw} wi{\geminationm}eln werde. Nett {\pb}wärs we{\geminationn}
               Sie mit mir bei Leopold\oindex{Hotel und Pension Rudolfshoehe (Leopold Petter)@\textbf{Hotel und Pension Rudolfshöhe (Leopold Petter)}|pw}{ }\introOben{}zu Nacht,\introOben{}{ }ſpeiſten.\pend
           \pstart Herzlich Ihr \spacefill\mbox{Arthur}\pend{}          \endnumbering\briefempfaengerindex{Beer-Hofmann, Richard@\textsc{Beer-Hofmann, Richard}!zzzSchnitzler, Arthur@\emph{von Arthur Schnitzler}!1894-07-183@{{[}18. 7. 1894?{]}}|)be}\mylabel{h}\end{ledgroupsized}  \newcommand{\dateiname}{L00356}\newcommand{\titel}{Arthur Schnitzler an Richard Beer-Hofmann, [18. 7. 1894?]}\newcommand{\editorInnen}{Martin Anton Müller und Gerd-Hermann Susen}%% latex-leseansicht-abspann.tex
%% Abspann für die Leseansicht.
%% Der Schalter \ifkorrekturansicht ist bereits durch den Vorspann gesetzt.

%% latex-abspann.tex
%% Gemeinsamer Abspann für Korrekturansicht und Leseansicht.
%% Setzt den Schalter \ifkorrekturansicht voraus (gesetzt in den
%% einbindenden Dateien latex-korrekturansicht-abspann.tex bzw.
%% latex-leseansicht-abspann.tex).
%% ---------------------------------------------------------------

\normalsize

% Das esempio-Environment wird nur in der Leseansicht benötigt
\ifkorrekturansicht\else
\newenvironment{esempio}[3]%
{
    \vspace{1.5ex}
    \rlap{\underline{#1}}
    \par
    \setlength{\parindent}{0cm}
    \nopagebreak
    \leftskip=#2cm
    \rightskip=#3cm
}
{
    \par
}
\fi

\doendnotes{C}
\bigskip
\vfill

\clearpage

\footnotesize

\ifkorrekturansicht
  \lohead{\textsc{register}}
\fi

% theindex-Environment neu definieren ohne reledmac
\makeatletter
\renewenvironment{theindex}{%
  \ifkorrekturansicht
    \section*{\indexname}%
  \else
    \subsubsection*{Index der erwähnten Entitäten}%
  \fi
  \setlength{\parindent}{0pt}%
  \setlength{\parskip}{0pt plus 0.3pt}%
  \let\item\@idxitem
}{%
  \ifkorrekturansicht\clearpage\fi
}
\makeatother

\IfFileExists{\jobname-pw.ind}{\input{\jobname-pw.ind}}{}

% Quellenangabe nur in der Leseansicht
\ifkorrekturansicht\else
% Fallback-Definitionen, falls die .tex-Datei \titel etc. nicht gesetzt hat
\providecommand{\titel}{}
\providecommand{\editorInnen}{}
\providecommand{\dateiname}{\jobname}

\vspace{3cm}

\vfill

\footnotesize
\textsc{Quelle}: \titel. Herausgegeben von {\editorInnen}. In: \emph{Arthur Schnitzler: Briefwechsel mit Autorinnen und Autoren}.
 Digitale Edition, https://schnitzler-briefe.acdh.oeaw.ac.at/{\dateiname}.html (Stand \today)
\fi

\end{document}


      