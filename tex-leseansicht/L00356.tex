%% latex-korrekturansicht-vorspann.tex
%% Vorspann für die Korrekturansicht.
%% Lädt die gemeinsame Datei latex-vorspann.tex mit gesetztem Schalter.

\newif\ifkorrekturansicht
\korrekturansichttrue

\input{../tex-inputs/latex-vorspann}


\section[Arthur Schnitzler an Richard Beer-Hofmann, {[}18. 7. 1894?{]}]{L00356 Arthur Schnitzler an Richard Beer-Hofmann, {[}18. 7. 1894?{]}}
\nopagebreak\mylabel{L00356v}
\rehead{ }\normalsize\beginnumbering\briefempfaengerindex{Beer-Hofmann, Richard@\textsc{Beer-Hofmann, Richard}!zzzSchnitzler, Arthur@\emph{von Arthur Schnitzler}!1894-07-183@{{[}18. 7. 1894?{]}}|(be}
\toendnotes[C]{\smallbreak\pagebreak[2]}\Standort{YCGL, MSS 31.}
\physDesc{Brief, 1 Blatt, 3 Seiten, Umschlag, 504 Zeichen
\newline{}Handschrift: Bleistift, deutsche Kurrent
\newline{}Versand: ohne postalischen Übermittlungsvermerk }\pstart{}{\pb}Herrn \textsc{Dr. Richard Beer
                     Hofmann}\pend{}\pstart{}\textsc{Ischl\oindex{Bad Ischl@\textbf{Bad Ischl}, \emph{P.PPL}|pw}}\pend{}\pstart{}\textsc{Egelmoos 22}\oindex{Eglmoosgasse@\textbf{Eglmoosgasse}, \emph{Bezirk (A.BZK)}|pw}.\pend{}{\bigskip}\vspace{1em}
\pstart
           \noindent{}{\pb}Lieber Richard! –Ich wüßt nicht, warum Salzburg\oindex{Salzburg@\textbf{Salzburg}, \emph{A.ADM2}|pw} ganz ins Waſſer fallen ſoll, weil Bahr\pwindex{Bahr, Hermann 19.07.1863 – 15.01.1934@\textsc{Bahr, Hermann} (19.07.1863 – 15.01.1934), \emph{Schriftsteller/Schriftstellerin, Kritiker/Kritikerin}|pw} keine Zeit hat. Auch hat Hugo\pwindex{Hofmannsthal, Hugo von 1874-02-01 – 1929-07-15@\textsc{Hofmannsthal, Hugo von} (1874-02-01 – 1929-07-15), \emph{Schriftsteller/Schriftstellerin}|pw}
               ziemlich ſicher zugeſagt. – Ich fahr jedenfalls über Salzburg\oindex{Salzburg@\textbf{Salzburg}, \emph{A.ADM2}|pw} zurück. – Ich antworte dem {\pb}Bahr\pwindex{Bahr, Hermann 19.07.1863 – 15.01.1934@\textsc{Bahr, Hermann} (19.07.1863 – 15.01.1934), \emph{Schriftsteller/Schriftstellerin, Kritiker/Kritikerin}|pw} natürlich, daſs ich Samſtag
               noch hier bin. Ich werd wohl Sonntag wegfahren. –\pend
           
\pstart
           Heut geh ich zwiſchen 5 u 6 zu Ornſtein\pwindex{Ornstein, Sophie 26.9.1844 – 18.11.1906@\textsc{Ornstein, Sophie} (26.9.1844 – 18.11.1906)|pw}\pwindex{Ornstein, Wilhelm 22.2.1831 – 7.11.1908@\textsc{Ornstein, Wilhelm} (22.2.1831 – 7.11.1908), \emph{Kaufmann/Kauffrau}|pw}{ }\introOben{}(Gina Z.\pwindex{Zeisler, Regine Januar 1864 – 1940-09-03@\textsc{Zeisler, Regine} (Januar 1864 – 1940-09-03), \emph{Pianist/Pianistin}|pw})\introOben{}. Ich
               glaube, dſs ich dann zwiſchen 7 u ½ 8 auf die \textsc{Esplan.}\oindex{Esplanade [Bad Ischl]@\textbf{Esplanade [Bad Ischl]}, \emph{Straße (K.STR)}|pw} wi{\geminationm}eln werde. Nett {\pb}wärs we{\geminationn} Sie mit mir bei Leopold\oindex{Hotel und Pension Rudolfshoehe (Leopold Petter)@\textbf{Hotel und Pension Rudolfshöhe (Leopold Petter)}, \emph{Hotel (K.HTL)}|pw}{ }\introOben{}zu Nacht,\introOben{}{ }ſpeiſten.\pend
           \pstart Herzlich Ihr \spacefill\mbox{Arthur}\pend{}\selectlanguage{ngerman}\endnumbering\briefempfaengerindex{Beer-Hofmann, Richard@\textsc{Beer-Hofmann, Richard}!zzzSchnitzler, Arthur@\emph{von Arthur Schnitzler}!1894-07-183@{{[}18. 7. 1894?{]}}|)be}\mylabel{L00356h}  \normalsize

\doendnotes{C}
\bigskip
\vfill

\clearpage

\footnotesize

\lohead{\textsc{register}}

% Definiere theindex-Environment komplett neu ohne reledmac
\makeatletter
\renewenvironment{theindex}{%
  \section*{\indexname}%
  \setlength{\parindent}{0pt}%
  \setlength{\parskip}{0pt plus 0.3pt}%
  \let\item\@idxitem
}{%
  \clearpage
}
\makeatother

\IfFileExists{\jobname-pw.ind}{\input{\jobname-pw.ind}}{}

\end{document}

      