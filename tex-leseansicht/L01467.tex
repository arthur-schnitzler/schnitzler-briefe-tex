%% latex-korrekturansicht-vorspann.tex
%% Vorspann für die Korrekturansicht.
%% Lädt die gemeinsame Datei latex-vorspann.tex mit gesetztem Schalter.

\newif\ifkorrekturansicht
\korrekturansichttrue

\input{../tex-inputs/latex-vorspann}


\section[Arthur Schnitzler an Richard Beer-Hofmann, 8. 11. 1904]{L01467 Arthur Schnitzler an Richard Beer-Hofmann, 8. 11. 1904}
\nopagebreak\mylabel{L01467v}
\rehead{ }\normalsize\beginnumbering\briefempfaengerindex{Beer-Hofmann, Richard@\textsc{Beer-Hofmann, Richard}!zzzSchnitzler, Arthur@\emph{von Arthur Schnitzler}!1904-11-081@{8. 11. 1904}|(be}
\toendnotes[C]{\smallbreak\pagebreak[2]}\Standort{YCGL, MSS 31.}
\physDesc{Kartenbrief, 483 Zeichen
\newline{}Handschrift: schwarze Tinte, deutsche Kurrent
\newline{}Versand: 1) Stempel: »\nobreak{}\textcolor{gray}{Wien}, 8. XI. 04, 6\nobreak{}«.   2) Stempel: »\nobreak{}\oindex{Rodaun@\textbf{Rodaun}, \emph{A.ADM4}|pwk}\textcolor{gray}{Ro}daun\nobreak{}«. 
\newline{}Beer-Hofmann: mit schwarzer Tinte das Datum der Beantwortung notiert: »9/XI\strikeout{I} b.« }
\buchAbdrucke{\weitereDrucke{Arthur Schnitzler, Richard Beer-Hofmann: \emph{Briefwechsel 1891–1931}. Wien, Zürich: \emph{Europaverlag} 1992, S. 169.} }\toendnotes[C]{\smallbreak}\pstart{}{\pb}\textsc{Herrn Dr Richard Beer-Hofmann}\pend{}\pstart{}\textsc{Rodaun\oindex{Rodaun@\textbf{Rodaun}, \emph{A.ADM4}|pw}}\pend{}\pstart{}\textsc{Liesingerstraße} 2\oindex{Liesingerstrasse@\textbf{Liesingerstraße}, \emph{Straße (K.STR)}|pw}\pend{}{\bigskip}\vspace{1em}
\pstart
           \raggedleft{}{\pb}XVIII \textsc{Spoettel} 7\oindex{Edmund-Weiss-Gasse 7@\textbf{Edmund-Weiß-Gasse 7}, \emph{Wohngebäude (K.WHS)}|pw}.\pend
           
\pstart
           \raggedleft{}8. 11. 904.\pend
           \vspace{0.5em}
\pstart
           lieber Richard, ich fahre vorausſichtlich Samſtag nach
                  Berlin\oindex{Berlin@\textbf{Berlin}, \emph{P.PPLC}|pw}. Soll ich Ihnen dort irgendwas
               beſorgen, ſo ſchreiben Sie mir ein Wort.\pend
           
\pstart
           Meine »\textsc{Première}\pwindex{gruene Kakadu. Groteske in einem Akt@\emph{Der grüne Kakadu. Groteske in einem Akt}|pwv}\pwindex{tapfere Cassian. Puppenspiel in einem Akt@\emph{Der tapfere Cassian. Puppenspiel in einem Akt}|pwv}« ſoll am 19. ſein. –\pend
           
\pstart
           – Hörte von dem echt jüdiſchen Vorgehen Ihres Hausherrn\pwindex{Berger, Rudolf *~10.9.1858@\textsc{Berger, Rudolf} (*~10.9.1858), \emph{Vermieter/Vermieterin, Metzger/Metzgerin}|pwv}. Immerhin wäre es eine »fertige Sach« –.\pend
           
\pstart
           Wie gehts Ihnen denn? Ich kann die Bemerkung nicht unterdrücken, daſs es mir lieb wär
                  we{\geminationn} wir nicht ſo weit von einander wohnten. –
               Herzlichſt Ihr \spacefill\mbox{A.}\pend
           \selectlanguage{ngerman}\endnumbering\briefempfaengerindex{Beer-Hofmann, Richard@\textsc{Beer-Hofmann, Richard}!zzzSchnitzler, Arthur@\emph{von Arthur Schnitzler}!1904-11-081@{8. 11. 1904}|)be}\mylabel{L01467h}  \normalsize

\doendnotes{C}
\bigskip
\vfill

\clearpage

\footnotesize

\lohead{\textsc{register}}

% Definiere theindex-Environment komplett neu ohne reledmac
\makeatletter
\renewenvironment{theindex}{%
  \section*{\indexname}%
  \setlength{\parindent}{0pt}%
  \setlength{\parskip}{0pt plus 0.3pt}%
  \let\item\@idxitem
}{%
  \clearpage
}
\makeatother

\IfFileExists{\jobname-pw.ind}{\input{\jobname-pw.ind}}{}

\end{document}

      