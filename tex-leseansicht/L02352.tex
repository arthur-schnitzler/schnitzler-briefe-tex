%% latex-leseansicht-vorspann.tex
%% Vorspann für die Leseansicht.
%% Lädt die gemeinsame Datei latex-vorspann.tex mit nicht gesetztem Schalter.

\newif\ifkorrekturansicht
\korrekturansichtfalse

\input{../tex-inputs/latex-vorspann}


         
         \renewcommand{\erwaehntePersonen}{Personen: Albert Heine, Hugo von Hofmannsthal, Frieda Pollak}
         \renewcommand{\erwaehnteInstitutionen}{Institutionen: Residenztheater München, Salzburger Festspiele}
         \renewcommand{\erwaehnteOrte}{Orte: Salzburg, Schlosstheater Schönbrunn, Sternwartestraße, Wien}
         \renewcommand{\erwaehnteWerke}{Werke: Der Schwierige. Lustspiel in drei Akten}
               \section[Hugo Hofmannsthal an Arthur Schnitzler, 30. 7. 1920]{ Hugo Hofmannsthal an Arthur Schnitzler, 30. 7. 1920}\nopagebreak\mylabel{v}\rehead{ }\begin{ledgroupsized}[t]{13cm}\normalsize\beginnumbering\briefempfaengerindex{Schnitzler, Arthur@\textsc{Schnitzler, Arthur}!zzzHofmannsthal, Hugo von@\emph{von Hugo von Hofmannsthal}!1920-07-301@{30. 7. 1920}|(be} \toendnotes[C]{\smallbreak\pagebreak[2]} \Standort{CUL, Schnitzler, B 43.}
\physDesc{Postkarte, 504 Zeichen
\newline{}Handschrift: 1) schwarze Tinte, deutsche Kurrent\hspace{1em}2) schwarze Tinte, lateinische Kurrent (\noindent{}Adresse)\hspace{1em}
\newline{}Versand: Stempel: »\nobreak{}\oindex{Salzburg@\textbf{Salzburg}|pwk}Salzburg 2, 31. VII. \textcolor{gray}{20}\nobreak{}«.  
\newline{}Ordnung: 1) mit Bleistift von Frieda
                                    Pollak\pwindex{Pollak, Frieda 08.12.1881 – 13.07.1937@\textsc{Pollak, Frieda} (08.12.1881 – 13.07.1937), \emph{Sekretärin}|pw} (?) mit dem Buchstaben »A«
                                 (Abgeschrieben/Abschrift) gekennzeichnet  2) mit Bleistift von unbekannter Hand nummeriert: »\strikeout{387}« 3) mit Bleistift von unbekannter Hand nummeriert:
                                    »369«}\buchAbdrucke{\weitereDrucke{Hugo von Hofmannsthal, Arthur Schnitzler: \emph{Briefwechsel}. Hg. Therese Nickl und Heinrich Schnitzler. Frankfurt am Main: \emph{S. Fischer} 1964, S. 294.} }\toendnotes[C]{\smallbreak}\pstart{}{\pb}Herrn D\textsuperscript{r} Arthur Schnitzler\pend{}\pstart{}Wien\oindex{Wien@\textbf{Wien}|pw}\pend{}\pstart{}XVIII Sternwartestrasse 71\oindex{XXXX Ortsangabe fehlt|pw}\pend{}{\bigskip}\pstart
           \raggedleft{}{\pb}Salzburg\oindex{Salzburg@\textbf{Salzburg}|pw}{ }30 VII 20.\pend
           \pstart{}mein lieber Arthur\pend\pstart
           \label{K_L02352-1v}\edtext{hier}{\lemma{\textnormal{\emph{hier}}}\Cendnote{\textnormal{Er war anlässlich der 1. \emph{Festspiele}\orgindex{Salzburger Festspiele@Salzburger Festspiele|pwk} in Salzburg\oindex{Salzburg@\textbf{Salzburg}|pwk}.}}}\label{K_L02352-1h} kann
               ich nie ſein, ohne Ihrer und ſchöner weit entſchwundener Begegnungen, leichter und
               tiefer Geſpräche und {\pb}unſerer
               Lebensfreundſchaft mit dem undefinierbaren Gefühl, das man mit »Wehmut« oft aber
               nicht richtig benennt, zu gedenken.\pend
           \pstart
           Ihr Rat war, wie immer, ſehr gut; Heine\pwindex{Heine, Albert 16.11.1867 – 13.4.1949@\textsc{Heine, Albert} (16.11.1867 – 13.4.1949), \emph{Theaterleiter, Schauspieler}|pw} hat das
                  \label{K_L02352-2v}\edtext{Stück\pwindex{Hofmannsthal, Hugo von 1874-02-01 – 1929-07-15@\textsc{Hofmannsthal, Hugo von} (1874-02-01 – 1929-07-15), \emph{Schriftsteller}!Schwierige. Lustspiel in drei Akten1921@\strich\emph{Der Schwierige. Lustspiel in drei Akten} {[}1921{]}|pwv}}{\lemma{\textnormal{\emph{Stück}}}\Cendnote{\textnormal{Zu der angedachten Inszenierung von \emph{Der Schwierige}\pwindex{Hofmannsthal, Hugo von 1874-02-01 – 1929-07-15@\textsc{Hofmannsthal, Hugo von} (1874-02-01 – 1929-07-15), \emph{Schriftsteller}!Schwierige. Lustspiel in drei Akten1921@\strich\emph{Der Schwierige. Lustspiel in drei Akten} {[}1921{]}|pwk} kam es nicht. Stattdessen
                  erlebte dies am 7. 11. 1921 am \emph{Münchner Residenztheater}\orgindex{Residenztheater Muenchen@Residenztheater München|pwk} seine Uraufführung.}}}\label{K_L02352-2h}, als ich es ihm
               anbot, ohne weiteres angeno{\geminationm}en, er will es als erſte
               Frühjahrsnovität in Schönbrunn\oindex{Schlosstheater Schoenbrunn@\textbf{Schlosstheater Schönbrunn}|pw} ſpielen.\pend
           \pstart
           Von Herzen Ihr{\\[\baselineskip]}\spacefill\mbox{Hugo.}\pend
           \leftskip=0em{}
         
         \endnumbering\mylabel{h}\end{ledgroupsized}  \newcommand{\dateiname}{L02352}\newcommand{\titel}{Hugo Hofmannsthal an Arthur Schnitzler, 30. 7. 1920}\newcommand{\editorInnen}{Martin Anton Müller und Gerd-Hermann Susen}%% latex-leseansicht-abspann.tex
%% Abspann für die Leseansicht.
%% Der Schalter \ifkorrekturansicht ist bereits durch den Vorspann gesetzt.

%% latex-abspann.tex
%% Gemeinsamer Abspann für Korrekturansicht und Leseansicht.
%% Setzt den Schalter \ifkorrekturansicht voraus (gesetzt in den
%% einbindenden Dateien latex-korrekturansicht-abspann.tex bzw.
%% latex-leseansicht-abspann.tex).
%% ---------------------------------------------------------------

\normalsize

% Das esempio-Environment wird nur in der Leseansicht benötigt
\ifkorrekturansicht\else
\newenvironment{esempio}[3]%
{
    \vspace{1.5ex}
    \rlap{\underline{#1}}
    \par
    \setlength{\parindent}{0cm}
    \nopagebreak
    \leftskip=#2cm
    \rightskip=#3cm
}
{
    \par
}
\fi

\doendnotes{C}
\bigskip
\vfill

\clearpage

\footnotesize

\ifkorrekturansicht
  \lohead{\textsc{register}}
\fi

% theindex-Environment neu definieren ohne reledmac
\makeatletter
\renewenvironment{theindex}{%
  \ifkorrekturansicht
    \section*{\indexname}%
  \else
    \subsubsection*{Index der erwähnten Entitäten}%
  \fi
  \setlength{\parindent}{0pt}%
  \setlength{\parskip}{0pt plus 0.3pt}%
  \let\item\@idxitem
}{%
  \ifkorrekturansicht\clearpage\fi
}
\makeatother

\IfFileExists{\jobname-pw.ind}{\input{\jobname-pw.ind}}{}

% Quellenangabe nur in der Leseansicht
\ifkorrekturansicht\else
% Fallback-Definitionen, falls die .tex-Datei \titel etc. nicht gesetzt hat
\providecommand{\titel}{}
\providecommand{\editorInnen}{}
\providecommand{\dateiname}{\jobname}

\vspace{3cm}

\vfill

\footnotesize
\textsc{Quelle}: \titel. Herausgegeben von {\editorInnen}. In: \emph{Arthur Schnitzler: Briefwechsel mit Autorinnen und Autoren}.
 Digitale Edition, https://schnitzler-briefe.acdh.oeaw.ac.at/{\dateiname}.html (Stand \today)
\fi

\end{document}


      