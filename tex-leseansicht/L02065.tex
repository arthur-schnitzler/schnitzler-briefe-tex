%% latex-leseansicht-vorspann.tex
%% Vorspann für die Leseansicht.
%% Lädt die gemeinsame Datei latex-vorspann.tex mit nicht gesetztem Schalter.

\newif\ifkorrekturansicht
\korrekturansichtfalse

\input{../tex-inputs/latex-vorspann}


         
         \renewcommand{\erwaehntePersonen}{Personen: Max Mell}
         \renewcommand{\erwaehnteOrte}{Orte: Wien, Wittelsbachstraße}
         \renewcommand{\erwaehnteWerke}{}
               \section[Max Mell an Arthur Schnitzler, 14. 5. 1912]{ Max Mell an Arthur Schnitzler, 14. 5. 1912}\nopagebreak\mylabel{v}\rehead{ }\begin{ledgroupsized}[t]{13cm}\normalsize\beginnumbering\briefempfaengerindex{Schnitzler, Arthur@\textsc{Schnitzler, Arthur}!zzzMell, Max@\emph{von Max Mell}!1912-05-142@{14. 5. 1912}|(be} \toendnotes[C]{\smallbreak\pagebreak[2]} \Standort{DLA, A:Schnitzler, HS.NZ85.1.5556.}
\physDesc{Brief, 1 Blatt, 1 Seite, 681 Zeichen (Briefpapier mit Trauerrand)
\newline{}Handschrift: schwarze Tinte, deutsche Kurrent
\newline{}Schnitzler: 1) mit rotem Buntstift ein Strich etwas versetzt zur
                                 Datumsangabe  2) mit Bleistift die Absenderadresse unterhalb des Brieftexts: »\textsc{II.
                                       Wittelsbachg. 5\oindex{Wittelsbachstrasse@\textbf{Wittelsbachstraße}|pw}.}«}\pstart
           \raggedleft{}{\pb}Wien\oindex{Wien@\textbf{Wien}|pw}, 14. Mai 1912.\pend
           \pstart{}Sehr verehrter Herr Doktor!\pend\pstart
           Das ſchöne Feſt, das Sie heute begehn, ſcheint mir eine ſchickliche Gelegenheit,
               Ihnen dankbar zu bekennen, daß ich mich vor dem Phänomen Ihres Werkes immer berührt,
               forſchend, ſtudierend, erkennend, bewundernd ſtehen fühle. Ich ſage das, weil ich
               meine, geiſtigen Beſitz zu geben, das iſt ja das, weshalb man ſchafft, und was die
               Freude an dem erledigten, innerlich abgelöſten Werk noch immer weiter fortzuſsetzen
               vermag. Ich fühle mich Ihnen tief verpflichtet und darf, in Erinnerung vieler
               Freundlichkeit, die Sie mir erwieſen, zu dieſen Worten vielleicht noch meine
               herzlichen Wünſche für heute und immer hinzufügen:\pend
           \pstart
           als Ihr{\\[\baselineskip]}\spacefill\mbox{Max Mell.}\pend
           \leftskip=0em{}
         
         \endnumbering\mylabel{h}\end{ledgroupsized}  \newcommand{\dateiname}{L02065}\newcommand{\titel}{Max Mell an Arthur Schnitzler, 14. 5. 1912}\newcommand{\editorInnen}{Martin Anton Müller und Gerd-Hermann Susen}%% latex-leseansicht-abspann.tex
%% Abspann für die Leseansicht.
%% Der Schalter \ifkorrekturansicht ist bereits durch den Vorspann gesetzt.

%% latex-abspann.tex
%% Gemeinsamer Abspann für Korrekturansicht und Leseansicht.
%% Setzt den Schalter \ifkorrekturansicht voraus (gesetzt in den
%% einbindenden Dateien latex-korrekturansicht-abspann.tex bzw.
%% latex-leseansicht-abspann.tex).
%% ---------------------------------------------------------------

\normalsize

% Das esempio-Environment wird nur in der Leseansicht benötigt
\ifkorrekturansicht\else
\newenvironment{esempio}[3]%
{
    \vspace{1.5ex}
    \rlap{\underline{#1}}
    \par
    \setlength{\parindent}{0cm}
    \nopagebreak
    \leftskip=#2cm
    \rightskip=#3cm
}
{
    \par
}
\fi

\doendnotes{C}
\bigskip
\vfill

\clearpage

\footnotesize

\ifkorrekturansicht
  \lohead{\textsc{register}}
\fi

% theindex-Environment neu definieren ohne reledmac
\makeatletter
\renewenvironment{theindex}{%
  \ifkorrekturansicht
    \section*{\indexname}%
  \else
    \subsubsection*{Index der erwähnten Entitäten}%
  \fi
  \setlength{\parindent}{0pt}%
  \setlength{\parskip}{0pt plus 0.3pt}%
  \let\item\@idxitem
}{%
  \ifkorrekturansicht\clearpage\fi
}
\makeatother

\IfFileExists{\jobname-pw.ind}{\input{\jobname-pw.ind}}{}

% Quellenangabe nur in der Leseansicht
\ifkorrekturansicht\else
% Fallback-Definitionen, falls die .tex-Datei \titel etc. nicht gesetzt hat
\providecommand{\titel}{}
\providecommand{\editorInnen}{}
\providecommand{\dateiname}{\jobname}

\vspace{3cm}

\vfill

\footnotesize
\textsc{Quelle}: \titel. Herausgegeben von {\editorInnen}. In: \emph{Arthur Schnitzler: Briefwechsel mit Autorinnen und Autoren}.
 Digitale Edition, https://schnitzler-briefe.acdh.oeaw.ac.at/{\dateiname}.html (Stand \today)
\fi

\end{document}


      