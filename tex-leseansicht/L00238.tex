%% latex-korrekturansicht-vorspann.tex
%% Vorspann für die Korrekturansicht.
%% Lädt die gemeinsame Datei latex-vorspann.tex mit gesetztem Schalter.

\newif\ifkorrekturansicht
\korrekturansichttrue

\input{../tex-inputs/latex-vorspann}


\section[Hugo von Hofmannsthal an Arthur Schnitzler, 19. 7. 1893]{L00238 Hugo von Hofmannsthal an Arthur Schnitzler, 19. 7. 1893}
\nopagebreak\mylabel{L00238v}
\rehead{ }\normalsize\beginnumbering\briefempfaengerindex{Schnitzler, Arthur@\textsc{Schnitzler, Arthur}!zzzHofmannsthal, Hugo von@\emph{von Hugo von Hofmannsthal}!1893-07-191@{19. 7. 1893}|(be}
\toendnotes[C]{\smallbreak\pagebreak[2]}\Standort{CUL, Schnitzler, B 43.}
\physDesc{Brief, 1 Blatt, 4 Seiten, 2037 Zeichen
\newline{}Handschrift: schwarze Tinte, deutsche Kurrent
\newline{}Ordnung: mit Bleistift von unbekannter Hand nummeriert:
                                    »54« }
\buchAbdrucke{\weitereDrucke{1) Hugo von Hofmannsthal, Arthur Schnitzler: \emph{Briefwechsel}. Frankfurt am Main: \emph{S. Fischer} 1964, S. 40–41.} \weitereDrucke{2) Hermann Bahr, Arthur Schnitzler: \emph{Briefwechsel, Aufzeichnungen, Dokumente (1891–1931)}. Göttingen: \emph{Wallstein} 2018, S. 35.} }\toendnotes[C]{\smallbreak}
\pstart
           
\pstart
           {\pb}Salzburg Bad-Fuſch\oindex{Bad Fusch@\textbf{Bad Fusch}, \emph{A.ADM3}|pw},\pend
           
\pstart
           \raggedleft{}19. VII. 93\pend
           \pend
           
\pstart{}lieber Arthur!\pend\vspace{0.5em}
\pstart
           Richards\pwindex{Beer-Hofmann, Richard 1866-07-11 – 1945-09-26@\textsc{Beer-Hofmann, Richard} (1866-07-11 – 1945-09-26), \emph{Schriftsteller/Schriftstellerin}|pw} Bericht von dem »Abschiedsſouper\pwindex{Abschiedssouper@\emph{Abschiedssouper}|pw}« war recht unerfreulich; er ſcheint mit der
               gewiſſen Hellſichtigkeit der Autoren jede Mücke als Elefanten empfunden zu haben; wie
               es wirklich war, weiß ich natürlich nicht, jedenfalls iſt die überaus freundliche,
               gewiſſermaßen reſpectvolle Notiz in
                  der »Neuen Freien Preſſe\orgindex{Neue Freie Presse@Neue Freie Presse|pw}«\pwindex{Aus Ischl, 14. Juli, schreibt man uns: …@\emph{Aus Ischl, 14. Juli, schreibt man uns: …}|pwv}{ }ſehr erfreulich und nützt 10mal mehr als die
               Aufführung ſelbſt. So wird im ganzen dieſer Einbruch von äußerem Leben in Ihr inneres
               keine ſchlechte Laune zurückgelaſſen haben.\pend
           
\pstart
           {\pb}Ich freue mich ſchon recht ſehr
               auf die Parallel-novelle\pwindex{kleine Komoedie@\emph{Die kleine Komödie}|pwv}.\pend
           
\pstart
           Mein Leben verſtreicht ziemlich nichtsſagend, mit \introOben{}langſam\introOben{}{ }ſteigendem inneren Wohlbefinden. Von Strobl\oindex{Strobl@\textbf{Strobl}, \emph{A.ADM3}|pw} hoffe ich manches Schöne: Sonne und Mond am
               Waſſer, Segeln, kindlich-lärmende Vergnügungen, Richard\pwindex{Beer-Hofmann, Richard 1866-07-11 – 1945-09-26@\textsc{Beer-Hofmann, Richard} (1866-07-11 – 1945-09-26), \emph{Schriftsteller/Schriftstellerin}|pw}, auch Schwarzkopf\pwindex{Schwarzkopf, Gustav 07.11.1853 – 13.11.1939@\textsc{Schwarzkopf, Gustav} (07.11.1853 – 13.11.1939), \emph{Schriftsteller/Schriftstellerin}|pw}; nur Sie
               gar nicht?\pend
           
\pstart
           Ich leſe mit lebhafteſtem Intereſſe die »Hauptſtrömungen\pwindex{Hauptstroemungen der Literatur des neunzehnten Jahrhunderts@\emph{Hauptströmungen der Literatur des neunzehnten Jahrhunderts}|pw}« von Brandes\pwindex{Brandes, Georg 04.02.1842 – 19.02.1927@\textsc{Brandes, Georg} (04.02.1842 – 19.02.1927)|pw},
               unendlich vieles aus der 1\textsuperscript{ten} Hälfte des Säculums besitzt
               im zweiten ein Gegenbild, manches eine Carricatur; namentlich ſehe ich mit halb
               ſchauerndem Staunen, {\pb}wie völlig
               ſich die \introOben{}Producte der\introOben{} jüngſten Strömungen, in denen ich ja
               auch mit einer Fußſpitze ſtehe, der Romantik als Kugelſpiegelbild, halb verſchrumpft,
               halb aufgedunſen, gegenüberſtellen.\pend
           
\pstart
           Ich habe mir ſehr viel abzugewöhnen, aber es ſind wenigſtens lauter echte
               Dichterkrankheiten.\pend
           
\pstart
           Mir ſcheint, der Satz klingt maßlos arrogant; leſen Sie ihn nicht ſo.\pend
           
\pstart
           Sie müſſen mir einen handgreiflichen Gefallen thuen: ich bin mit Bahr\pwindex{Bahr, Hermann 19.07.1863 – 15.01.1934@\textsc{Bahr, Hermann} (19.07.1863 – 15.01.1934), \emph{Schriftsteller/Schriftstellerin, Kritiker/Kritikerin}|pw} verabredet, Ende Juli nach München\oindex{Muenchen@\textbf{München}, \emph{P.PPLA}|pw} zu gehen; mir paſst 24.
               (eventuell 25.) bis 1. Auguſt; ſeit 14 Tagen beantwortet
                  Bahr\pwindex{Bahr, Hermann 19.07.1863 – 15.01.1934@\textsc{Bahr, Hermann} (19.07.1863 – 15.01.1934), \emph{Schriftsteller/Schriftstellerin, Kritiker/Kritikerin}|pw} keinen Brief. Ich muſs aber doch
               endlich wiſſen, {\pb}woran ich bin.
               Alſo bitte, telefonieren Sie in meinem Namen an die Redaction der »Deutſchen Zeitung«\orgindex{Deutsche Zeitung@Deutsche Zeitung|pw}, man möge entweder Bahr\pwindex{Bahr, Hermann 19.07.1863 – 15.01.1934@\textsc{Bahr, Hermann} (19.07.1863 – 15.01.1934), \emph{Schriftsteller/Schriftstellerin, Kritiker/Kritikerin}|pw} meine dringende Aufforderung endlich zukommen laſſen, oder
               ſeine Adreſſe angeben, oder wenn man das nicht darf, wenigſtens ſagen, wie lang er
               beiläufig \textsc{incognito} oder verſchollen bleiben dürfte. Und
               bitte, ſchreiben Sie mir \uuline{ſofort} den Beſcheid.\pend
           
\pstart
           Herzlichst{\\[\baselineskip]}Ihr \spacefill\mbox{Loris.}\pend
           \leftskip=0em{}
\pstart
           \noindent{}Warum antwortet Salten\pwindex{Salten, Felix 06.09.1869 – 08.10.1945@\textsc{Salten, Felix} (06.09.1869 – 08.10.1945), \emph{Schriftsteller/Schriftstellerin, Journalist/Journalistin, Chefredakteur/Chefredakteurin}|pw} nicht?\pend
           \selectlanguage{ngerman}\endnumbering\briefempfaengerindex{Schnitzler, Arthur@\textsc{Schnitzler, Arthur}!zzzHofmannsthal, Hugo von@\emph{von Hugo von Hofmannsthal}!1893-07-191@{19. 7. 1893}|)be}\mylabel{L00238h}  \normalsize

\doendnotes{C}
\bigskip
\vfill

\clearpage

\footnotesize

\lohead{\textsc{register}}

% Definiere theindex-Environment komplett neu ohne reledmac
\makeatletter
\renewenvironment{theindex}{%
  \section*{\indexname}%
  \setlength{\parindent}{0pt}%
  \setlength{\parskip}{0pt plus 0.3pt}%
  \let\item\@idxitem
}{%
  \clearpage
}
\makeatother

\IfFileExists{\jobname-pw.ind}{\input{\jobname-pw.ind}}{}

\end{document}

      