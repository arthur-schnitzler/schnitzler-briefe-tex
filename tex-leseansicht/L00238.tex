\input{../tex-inputs/latex-pdf-vorspann}
\begin{center}
            \textcolor{red}{ENTWURF. ENTZIFFERUNG NOCH NICHT KORREKTURGELESEN}
                      \end{center}
            
               \section[Hugo von Hofmannsthal an Arthur Schnitzler, 19. 7. 1893]{ Hugo von Hofmannsthal an Arthur Schnitzler, 19. 7. 1893}\nopagebreak\mylabel{v}\rehead{ }\begin{ledgroupsized}[t]{13cm}\normalsize\beginnumbering\briefempfaengerindex{Schnitzler, Arthur@\textsc{Schnitzler, Arthur}!zzzHofmannsthal, Hugo von@\emph{von Hugo von Hofmannsthal}!1893-07-191@{19. 7. 1893}|(be} \toendnotes[C]{\smallbreak\pagebreak[2]} \Standort{CUL, Schnitzler, B 43.}
\physDesc{Brief, 1 Blatt, 4 Seiten
\newline{}Handschrift: schwarze Tinte, deutsche Kurrent\newline{}Ordnung: mit Bleistift von unbekannter Hand nummeriert: »54« }\buchAbdrucke{\weitereDrucke{1) Hugo von Hofmannsthal, Arthur Schnitzler: \emph{Briefwechsel}. Hg. Therese Nickl und Heinrich Schnitzler. Frankfurt am Main: \emph{S. Fischer} 1964, S. 40–41.} \weitereDrucke{2) Hermann Bahr, Arthur Schnitzler: \emph{Briefwechsel, Aufzeichnungen, Dokumente
                                (1891–1931)}. Hg. Kurt Ifkovits und Martin Anton Müller. Göttingen: \emph{Wallstein} 2018, S. 35.} }\toendnotes[C]{\smallbreak}\pstart
           {\pb}Salzburg Bad-Fuſch\oindex{Bad Fusch@\textbf{Bad Fusch}|pw},\hfill 19. VII. 93\pend
           \pstart{}lieber Arthur!\pend\pstart
           Richard\pwindex{Beer-Hofmann, Richard 11.07.1866 – 26.09.1945@\textsc{Beer-Hofmann, Richard} (11.07.1866 – 26.09.1945), \emph{Schriftsteller}|pw}s Bericht von dem »Abschiedsſouper\pwindex{Schnitzler, Arthur 15.05.1862 – 21.10.1931@\textsc{Schnitzler, Arthur} (15.05.1862 – 21.10.1931), \emph{Schriftsteller, Mediziner}!Abschiedssouper1892@\strich\emph{Abschiedssouper} {[}1892{]}|pw}« war recht unerfreulich; er ſcheint mit der
                    gewiſſen Hellſichtigkeit der Autoren jede Mücke als Elefanten empfunden zu
                    haben; wie es wirklich war, weiß ich natürlich nicht, jedenfalls iſt die überaus
                    freundliche, gewiſſermaßen reſpectvolle Notiz in der »Neuen Freien
                            Preſſe\orgindex{Neue Freie Presse@Neue Freie Presse|pw}«\pwindex{?? Werk@Nicht ermittelte Verfasserinnen und Verfasser!Aus Ischl, 14. Juli, schreibt man uns: …18.07.1893 – 18.07.1893@\emph{Aus Ischl, 14. Juli, schreibt man uns: …} {[}18.07.1893 – 18.07.1893{]}|pwv}{ }ſehr erfreulich und nützt 10mal mehr als die
                    Aufführung ſelbſt. So wird im ganzen dieſer Einbruch von äußerem Leben in Ihr
                    inneres keine ſchlechte Laune zurückgelaſſen haben.\pend
           \pstart
           {\pb}Ich freue mich ſchon recht
                    ſehr auf die Parallel-novelle\pwindex{Schnitzler, Arthur 15.05.1862 – 21.10.1931@\textsc{Schnitzler, Arthur} (15.05.1862 – 21.10.1931), \emph{Schriftsteller, Mediziner}!kleine Komoedie01.08.1895 – 01.08.1895@\strich\emph{Die kleine Komödie} {[}01.08.1895 – 01.08.1895{]}|pwv}.\pend
           \pstart
           Mein Leben verſtreicht ziemlich nichtsſagend, mit \introOben{}langſam\introOben{}{ }ſteigendem inneren Wohlbefinden. Von Strobl\oindex{Strobl@\textbf{Strobl}|pw} hoffe ich manches Schöne: Sonne und Mond
                    am Waſſer, Segeln, kindlich-lärmende Vergnügungen, Richard\pwindex{Beer-Hofmann, Richard 11.07.1866 – 26.09.1945@\textsc{Beer-Hofmann, Richard} (11.07.1866 – 26.09.1945), \emph{Schriftsteller}|pw}, auch Schwarzkopf\pwindex{Schwarzkopf, Gustav 07.11.1853 – 13.11.1939@\textsc{Schwarzkopf, Gustav} (07.11.1853 – 13.11.1939), \emph{Schriftsteller}|pw}; nur Sie gar nicht?\pend
           \pstart
           Ich leſe mit lebhafteſtem Intereſſe die »Hauptſtrömungen\pwindex{Brandes, Georg 04.02.1842 – 19.02.1927@\textsc{Brandes, Georg} (04.02.1842 – 19.02.1927)!Hauptstroemungen der Literatur des neunzehnten Jahrhunderts1872@\strich\emph{Hauptströmungen der Literatur des neunzehnten Jahrhunderts} {[}1872{]}|pw}« von Brandes\pwindex{Brandes, Georg 04.02.1842 – 19.02.1927@\textsc{Brandes, Georg} (04.02.1842 – 19.02.1927)|pw},
                    unendlich vieles aus der 1\textsuperscript{ten} Hälfte des Säculums
                    besitzt im zweiten ein Gegenbild, manches eine Carricatur; namentlich ſehe ich
                    mit halb ſchauerndem Staunen, {\pb}wie völlig ſich die \introOben{}Producte der\introOben{} jüngſten Strömungen,
                    in denen ich ja auch mit einer Fußſpitze ſtehe, der Romantik als
                    Kugelſpiegelbild, halb verſchrumpft, halb aufgedunſen, gegenüberſtellen.\pend
           \pstart
           Ich habe mir ſehr viel abzugewöhnen, aber es ſind wenigſtens lauter echte
                    Dichterkrankheiten.\pend
           \pstart
           Mir ſcheint, der Satz klingt maßlos arrogant; leſen Sie ihn nicht ſo.\pend
           \pstart
           Sie müſſen mir einen handgreiflichen Gefallen thuen: ich bin mit Bahr\pwindex{Bahr, Hermann 19.07.1863 – 15.01.1934@\textsc{Bahr, Hermann} (19.07.1863 – 15.01.1934), \emph{Schriftsteller, Kritiker}|pw} verabredet, Ende Juli nach München\oindex{Muenchen@\textbf{München}|pw} zu gehen; mir paſst 24.
                    (eventuell 25.) bis 1. Auguſt; ſeit 14 Tagen
                    beantwortet Bahr\pwindex{Bahr, Hermann 19.07.1863 – 15.01.1934@\textsc{Bahr, Hermann} (19.07.1863 – 15.01.1934), \emph{Schriftsteller, Kritiker}|pw} keinen Brief. Ich muſs aber
                    doch endlich wiſſen, {\pb}woran
                    ich bin. Alſo bitte, telefonieren Sie in meinem Namen an die Redaction der »Deutſchen Zeitung«\orgindex{Deutsche Zeitung@Deutsche Zeitung|pw}, man möge entweder Bahr\pwindex{Bahr, Hermann 19.07.1863 – 15.01.1934@\textsc{Bahr, Hermann} (19.07.1863 – 15.01.1934), \emph{Schriftsteller, Kritiker}|pw} meine dringende Aufforderung endlich
                    zukommen laſſen, oder ſeine Adreſſe angeben, oder wenn man das nicht darf,
                    wenigſtens ſagen, wie lang er beiläufig \textsc{incognito} oder
                    verſchollen bleiben dürfte. Und bitte, ſchreiben Sie mir \uuline{ſofort} den Beſcheid.\pend
           \pstart
           Herzlichst{\\[\baselineskip]}Ihr \spacefill\mbox{Loris.}\pend
           \leftskip=0em{}\pstart
           \noindent{}Warum antwortet Salten\pwindex{Salten, Felix 06.09.1869 – 08.10.1945@\textsc{Salten, Felix} (06.09.1869 – 08.10.1945), \emph{Schriftsteller, Journalist}|pw} nicht?\pend
           \endnumbering\briefempfaengerindex{Schnitzler, Arthur@\textsc{Schnitzler, Arthur}!zzzHofmannsthal, Hugo von@\emph{von Hugo von Hofmannsthal}!1893-07-191@{19. 7. 1893}|)be}\mylabel{h}\end{ledgroupsized}  \newcommand{\dateiname}{L00238}\newcommand{\titel}{Hugo von Hofmannsthal an Arthur Schnitzler, 19. 7. 1893}\newcommand{\editorInnen}{ Martin Anton Müller und Gerd-Hermann Susen}\input{../tex-inputs/latex-pdf-abspann}
      