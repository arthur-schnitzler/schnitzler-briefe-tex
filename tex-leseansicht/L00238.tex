%% latex-leseansicht-vorspann.tex
%% Vorspann für die Leseansicht.
%% Lädt die gemeinsame Datei latex-vorspann.tex mit nicht gesetztem Schalter.

\newif\ifkorrekturansicht
\korrekturansichtfalse

\input{../tex-inputs/latex-vorspann}


\section[Hugo von Hofmannsthal an Arthur Schnitzler, 19. 7. 1893]{L00238 Hugo von Hofmannsthal an Arthur Schnitzler, 19. 7. 1893}
\nopagebreak\mylabel{L00238v}
\rehead{ }\normalsize\beginnumbering\briefempfaengerindex{Schnitzler, Arthur@\textsc{Schnitzler, Arthur}!zzzHofmannsthal, Hugo von@\emph{von Hugo von Hofmannsthal}!1893-07-191@{19. 7. 1893}|(be}
\toendnotes[C]{\smallbreak\pagebreak[2]}
\correspDesc{Versand  durch Hugo von Hofmannsthal am 19. 7. 1893 in Bad Fusch
\newline{}Erhalt  durch Arthur Schnitzler im Zeitraum [20. 7. 1893
                  – 24. 7. 1893?] in Wien}\toendnotes[C]{\smallbreak}
\Standort{CUL, Schnitzler, B 43.}
\physDesc{Brief, 1 Blatt, 4 Seiten, 2037 Zeichen
\newline{}Handschrift: schwarze Tinte, deutsche Kurrent
\newline{}Ordnung: mit Bleistift von unbekannter Hand nummeriert:
                                    »54« }
\buchAbdrucke{\weitereDrucke{1) Hugo von Hofmannsthal, Arthur Schnitzler: \emph{Briefwechsel}. Herausgegeben von Therese Nickl und Heinrich Schnitzler. Frankfurt am Main: \emph{S. Fischer} 1964, S. 40–41.} \weitereDrucke{2) Hermann Bahr, Arthur Schnitzler: \emph{Briefwechsel, Aufzeichnungen, Dokumente (1891–1931)}. Herausgegeben von Kurt Ifkovits und Martin Anton Müller. Göttingen: \emph{Wallstein} 2018, S. 35.} }\toendnotes[C]{\smallbreak}
\pstart
           
\pstart
           {\pb}Salzburg Bad-Fuſch\oindex{Bad Fusch@\textbf{Bad Fusch}|pw},\pend
           
\pstart
           \raggedleft{}19. VII. 93\pend
           \pend
           
\pstart{}lieber Arthur!\pend\vspace{0.5em}
\pstart
           Richards\pwindex{Beer-Hofmann, Richard 11.\,7.\,1866 Wien – 26.\,9.\,1945 New York City@\textsc{Beer-Hofmann, Richard} (11.\,7.\,1866 Wien – 26.\,9.\,1945 New York City), \emph{Schriftsteller}|pw} Bericht von dem »Abschiedsſouper\pwindex{Schnitzler, Arthur 15.\,5.\,1862 Wien – 21.\,10.\,1931 ebd.@\textsc{Schnitzler, Arthur} (15.\,5.\,1862 Wien – 21.\,10.\,1931 ebd.), \emph{Schriftsteller, Mediziner}!Abschiedssouper@\strich\emph{Abschiedssouper}|pw}« war recht unerfreulich; er{ }ſcheint mit der
               gewiſſen Hellſichtigkeit der Autoren jede Mücke als Elefanten empfunden zu haben; wie
               es wirklich war, weiß ich natürlich nicht, jedenfalls iſt die überaus freundliche,
               gewiſſermaßen reſpectvolle Notiz in
                  der »Neuen Freien Preſſe\orgindex{Neue Freie Presse@Neue Freie Presse|pw}«\pwindex{Aus Ischl, 14. Juli, schreibt man uns: …@\emph{Aus Ischl, 14. Juli, schreibt man uns: …}|pwv}{ }ſehr erfreulich und nützt 10mal mehr als die
               Aufführung{ }ſelbſt. So wird im ganzen dieſer Einbruch von äußerem Leben in Ihr inneres
               keine{ }ſchlechte Laune zurückgelaſſen haben.\pend
           
\pstart
           {\pb}Ich freue mich{ }ſchon recht{ }ſehr
               auf die Parallel-novelle\pwindex{Schnitzler, Arthur 15.\,5.\,1862 Wien – 21.\,10.\,1931 ebd.@\textsc{Schnitzler, Arthur} (15.\,5.\,1862 Wien – 21.\,10.\,1931 ebd.), \emph{Schriftsteller, Mediziner}!kleine Komödie@\strich\emph{Die kleine Komödie}|pwv}.\pend
           
\pstart
           Mein Leben verſtreicht ziemlich nichtsſagend, mit \introOben{}langſam\introOben{}{ }ſteigendem inneren Wohlbefinden. Von Strobl\oindex{Strobl@\textbf{Strobl}, \emph{Verwaltungsgebiet}|pw} hoffe ich manches Schöne: Sonne und Mond am
               Waſſer, Segeln, kindlich-lärmende Vergnügungen, Richard\pwindex{Beer-Hofmann, Richard 11.\,7.\,1866 Wien – 26.\,9.\,1945 New York City@\textsc{Beer-Hofmann, Richard} (11.\,7.\,1866 Wien – 26.\,9.\,1945 New York City), \emph{Schriftsteller}|pw}, auch Schwarzkopf\pwindex{Schwarzkopf, Gustav 7.\,11.\,1853 Wien – 13.\,11.\,1939 ebd.@\textsc{Schwarzkopf, Gustav} (7.\,11.\,1853 Wien – 13.\,11.\,1939 ebd.), \emph{Schriftsteller}|pw}; nur Sie
               gar nicht?\pend
           
\pstart
           Ich leſe mit lebhafteſtem Intereſſe die »Hauptſtrömungen\pwindex{Brandes, Georg 4.\,2.\,1842 Kopenhagen – 19.\,2.\,1927 ebd.@\textsc{Brandes, Georg} (4.\,2.\,1842 Kopenhagen – 19.\,2.\,1927 ebd.)!Hauptströmungen der Literatur des neunzehnten Jahrhunderts@\strich\emph{Hauptströmungen der Literatur des neunzehnten Jahrhunderts}|pw}« von Brandes\pwindex{Brandes, Georg 4.\,2.\,1842 Kopenhagen – 19.\,2.\,1927 ebd.@\textsc{Brandes, Georg} (4.\,2.\,1842 Kopenhagen – 19.\,2.\,1927 ebd.)|pw},
               unendlich vieles aus der 1\textsuperscript{ten} Hälfte des Säculums besitzt
               im zweiten ein Gegenbild, manches eine Carricatur; namentlich{ }ſehe ich mit halb{ }ſchauerndem Staunen, {\pb}wie völlig{ }ſich die \introOben{}Producte der\introOben{} jüngſten Strömungen, in denen ich ja
               auch mit einer Fußſpitze{ }ſtehe, der Romantik als Kugelſpiegelbild, halb verſchrumpft,
               halb aufgedunſen, gegenüberſtellen.\pend
           
\pstart
           Ich habe mir{ }ſehr viel abzugewöhnen, aber es{ }ſind wenigſtens lauter echte
               Dichterkrankheiten.\pend
           
\pstart
           Mir{ }ſcheint, der Satz klingt maßlos arrogant; leſen Sie ihn nicht{ }ſo.\pend
           
\pstart
           Sie müſſen mir einen handgreiflichen Gefallen thuen: ich bin mit Bahr\pwindex{Bahr, Hermann 19.\,7.\,1863 Linz – 15.\,1.\,1934 München@\textsc{Bahr, Hermann} (19.\,7.\,1863 Linz – 15.\,1.\,1934 München), \emph{Schriftsteller, Kritiker}|pw} verabredet, Ende Juli nach München\oindex{München@\textbf{München}|pw} zu gehen; mir paſst 24.
               (eventuell 25.) bis 1. Auguſt;{ }ſeit 14 Tagen beantwortet
                  Bahr\pwindex{Bahr, Hermann 19.\,7.\,1863 Linz – 15.\,1.\,1934 München@\textsc{Bahr, Hermann} (19.\,7.\,1863 Linz – 15.\,1.\,1934 München), \emph{Schriftsteller, Kritiker}|pw} keinen Brief. Ich muſs aber doch
               endlich wiſſen, {\pb}woran ich bin.
               Alſo bitte, telefonieren Sie in meinem Namen an die Redaction der »Deutſchen Zeitung«\orgindex{Deutsche Zeitung@Deutsche Zeitung|pw}, man möge entweder Bahr\pwindex{Bahr, Hermann 19.\,7.\,1863 Linz – 15.\,1.\,1934 München@\textsc{Bahr, Hermann} (19.\,7.\,1863 Linz – 15.\,1.\,1934 München), \emph{Schriftsteller, Kritiker}|pw} meine dringende Aufforderung endlich zukommen laſſen, oder{ }ſeine Adreſſe angeben, oder wenn man das nicht darf, wenigſtens{ }ſagen, wie lang er
               beiläufig \textsc{incognito} oder verſchollen bleiben dürfte. Und
               bitte,{ }ſchreiben Sie mir \uuline{ſofort} den Beſcheid.\pend
           
\pstart
           Herzlichst{\\[\baselineskip]}Ihr \spacefill\mbox{Loris.}\pend
           \leftskip=0em{}
\pstart
           \noindent{}Warum antwortet Salten\pwindex{Salten, Felix 6.\,9.\,1869 Budapest – 8.\,10.\,1945 Zürich@\textsc{Salten, Felix} (6.\,9.\,1869 Budapest – 8.\,10.\,1945 Zürich), \emph{Schriftsteller, Journalist, Chefredakteur}|pw} nicht?\pend
           \selectlanguage{ngerman}\endnumbering\briefempfaengerindex{Schnitzler, Arthur@\textsc{Schnitzler, Arthur}!zzzHofmannsthal, Hugo von@\emph{von Hugo von Hofmannsthal}!1893-07-191@{19. 7. 1893}|)be}\mylabel{L00238h}  \newcommand{\dateiname}{L00238}\newcommand{\titel}{Hugo von Hofmannsthal an Arthur Schnitzler, 19. 7. 1893}\newcommand{\editorInnen}{Herausgegeben von Martin Anton Müller}%% latex-leseansicht-abspann.tex
%% Abspann für die Leseansicht.
%% Der Schalter \ifkorrekturansicht ist bereits durch den Vorspann gesetzt.

%% latex-abspann.tex
%% Gemeinsamer Abspann für Korrekturansicht und Leseansicht.
%% Setzt den Schalter \ifkorrekturansicht voraus (gesetzt in den
%% einbindenden Dateien latex-korrekturansicht-abspann.tex bzw.
%% latex-leseansicht-abspann.tex).
%% ---------------------------------------------------------------

\normalsize

% Das esempio-Environment wird nur in der Leseansicht benötigt
\ifkorrekturansicht\else
\newenvironment{esempio}[3]%
{
    \vspace{1.5ex}
    \rlap{\underline{#1}}
    \par
    \setlength{\parindent}{0cm}
    \nopagebreak
    \leftskip=#2cm
    \rightskip=#3cm
}
{
    \par
}
\fi

\doendnotes{C}
\bigskip
\vfill

\clearpage

\footnotesize

\ifkorrekturansicht
  \lohead{\textsc{register}}
\fi

% theindex-Environment neu definieren ohne reledmac
\makeatletter
\renewenvironment{theindex}{%
  \ifkorrekturansicht
    \section*{\indexname}%
  \else
    \subsubsection*{Index der erwähnten Entitäten}%
  \fi
  \setlength{\parindent}{0pt}%
  \setlength{\parskip}{0pt plus 0.3pt}%
  \let\item\@idxitem
}{%
  \ifkorrekturansicht\clearpage\fi
}
\makeatother

\IfFileExists{\jobname-pw.ind}{\input{\jobname-pw.ind}}{}

% Quellenangabe nur in der Leseansicht
\ifkorrekturansicht\else
% Fallback-Definitionen, falls die .tex-Datei \titel etc. nicht gesetzt hat
\providecommand{\titel}{}
\providecommand{\editorInnen}{}
\providecommand{\dateiname}{\jobname}

\vspace{3cm}

\vfill

\footnotesize
\textsc{Quelle}: \titel. Herausgegeben von {\editorInnen}. In: \emph{Arthur Schnitzler: Briefwechsel mit Autorinnen und Autoren}.
 Digitale Edition, https://schnitzler-briefe.acdh.oeaw.ac.at/{\dateiname}.html (Stand \today)
\fi

\end{document}


