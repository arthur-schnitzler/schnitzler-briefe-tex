%% latex-korrekturansicht-vorspann.tex
%% Vorspann für die Korrekturansicht.
%% Lädt die gemeinsame Datei latex-vorspann.tex mit gesetztem Schalter.

\newif\ifkorrekturansicht
\korrekturansichttrue

\input{../tex-inputs/latex-vorspann}


\section[Arthur Schnitzler an Gerty Hofmannsthal, 2. 8. 1929]{L02516 Arthur Schnitzler an Gerty Hofmannsthal, 2. 8. 1929}
\nopagebreak\mylabel{L02516v}
\rehead{ }\normalsize\beginnumbering\briefempfaengerindex{Hofmannsthal, Gertrude von@\textsc{Hofmannsthal, Gertrude von}!zzzSchnitzler, Arthur@\emph{von Arthur Schnitzler}!1929-08-021@{2. 8. 1929}|(be}
\toendnotes[C]{\smallbreak\pagebreak[2]}\Standort{FDH, Hs-31346,2.}
\physDesc{Brief, 1 Blatt, 2 Seiten, 903 Zeichen (Briefpapier mit Trauerrand)
\newline{}Handschrift: schwarze Tinte, lateinische Kurrent
\newline{}Hofmannsthal: mit schwarzer Tinte beschriftet: »\textsc{erledigt}« }
\pstart
           \raggedleft{}{\pb}Wien\oindex{Wien@\textbf{Wien}, \emph{A.ADM2}|pw}, 2/8 929\pend
           \vspace{0.5em}
\pstart
           liebe Gerty, die Briefe sind angelangt, es sind auch einige wenige
               von Gustav Schwarzkopf\pwindex{Schwarzkopf, Gustav 07.11.1853 – 13.11.1939@\textsc{Schwarzkopf, Gustav} (07.11.1853 – 13.11.1939), \emph{Schriftsteller/Schriftstellerin}|pw} und Felix Salten\pwindex{Salten, Felix 06.09.1869 – 08.10.1945@\textsc{Salten, Felix} (06.09.1869 – 08.10.1945), \emph{Schriftsteller/Schriftstellerin, Journalist/Journalistin, Chefredakteur/Chefredakteurin}|pw} aus der gleichen Zeit dabei.
                  Indeß habe ich mir die \uline{Briefe Hugos\pwindex{Hofmannsthal, Hugo von 1874-02-01 – 1929-07-15@\textsc{Hofmannsthal, Hugo von} (1874-02-01 – 1929-07-15), \emph{Schriftsteller/Schriftstellerin}|pw} an G. Schw.\pwindex{Schwarzkopf, Gustav 07.11.1853 – 13.11.1939@\textsc{Schwarzkopf, Gustav} (07.11.1853 – 13.11.1939), \emph{Schriftsteller/Schriftstellerin}|pw}} von diesem geben lassen, dabei waren auch etliche ungedruckte Gedichte – ich
               habe, speciell in die Briefe vorläufg nur flüchtig hineingeblickt – es sind besondere
               Briefe aus der \textcolor{gray}{früherlieg} Zeit, – ganz wunderbares. Vor allem würd
               ich \introOben{}an Ihrer Stelle\introOben{} dies alles (es ist nicht übermäßg viel)
               abschreiben lassen, eventuell gleich in 2 Exemplaren – Soll ich dieses Paket (gleich
               mit den Briefen Hugos\pwindex{Hofmannsthal, Hugo von 1874-02-01 – 1929-07-15@\textsc{Hofmannsthal, Hugo von} (1874-02-01 – 1929-07-15), \emph{Schriftsteller/Schriftstellerin}|pw} an mich) {\pb}(vielfach undatiert) nach Aussee\oindex{Bad Aussee@\textbf{Bad Aussee}, \emph{P.PPLA3}|pw} schicken, oder möchten Sie, dſs \introOben{}ich\introOben{} die
               Abschriften \substVorne{}\textsuperscript{aus}\substDazwischen{}der Briefe von\substHinten{}{ }Schwarzkopf\pwindex{Schwarzkopf, Gustav 07.11.1853 – 13.11.1939@\textsc{Schwarzkopf, Gustav} (07.11.1853 – 13.11.1939), \emph{Schriftsteller/Schriftstellerin}|pw}{ }\uline{hier} besorgen lasse, (was erst im
                  September möglich wäre.)\pend
           
\pstart
           Ich hoffe liebe Gerty die Tage in Aussee\oindex{Bad Aussee@\textbf{Bad Aussee}, \emph{P.PPLA3}|pw} sind für
               Sie und die \textcolor{gray}{I}hren so gut und ruhig wie sie eben sein können. In
               Freundschaft mit Grüßen an Alle\pend
           
\pstart
           Ihr{\\[\baselineskip]}\spacefill\mbox{Arthur}\pend
           \leftskip=0em{}\selectlanguage{ngerman}\endnumbering\briefempfaengerindex{Hofmannsthal, Gertrude von@\textsc{Hofmannsthal, Gertrude von}!zzzSchnitzler, Arthur@\emph{von Arthur Schnitzler}!1929-08-021@{2. 8. 1929}|)be}\mylabel{L02516h}  \normalsize

\doendnotes{C}
\bigskip
\vfill

\clearpage

\footnotesize

\lohead{\textsc{register}}

% Definiere theindex-Environment komplett neu ohne reledmac
\makeatletter
\renewenvironment{theindex}{%
  \section*{\indexname}%
  \setlength{\parindent}{0pt}%
  \setlength{\parskip}{0pt plus 0.3pt}%
  \let\item\@idxitem
}{%
  \clearpage
}
\makeatother

\IfFileExists{\jobname-pw.ind}{\input{\jobname-pw.ind}}{}

\end{document}

      