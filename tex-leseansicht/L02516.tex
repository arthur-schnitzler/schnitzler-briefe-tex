%% latex-leseansicht-vorspann.tex
%% Vorspann für die Leseansicht.
%% Lädt die gemeinsame Datei latex-vorspann.tex mit nicht gesetztem Schalter.

\newif\ifkorrekturansicht
\korrekturansichtfalse

\input{../tex-inputs/latex-vorspann}

\begin{center}
            \textcolor{red}{ENTWURF. ENTZIFFERUNG NOCH NICHT KORREKTURGELESEN}
                      \end{center}
            
               \section[Arthur Schnitzler an Gerty von Hofmannsthal, 2. 8. 1929]{ Arthur Schnitzler an Gerty von Hofmannsthal, 2. 8. 1929}\nopagebreak\mylabel{v}\rehead{ }\begin{ledgroupsized}[t]{13cm}\normalsize\beginnumbering\briefempfaengerindex{Hofmannsthal, Gertrude von@\textsc{Hofmannsthal, Gertrude von}!zzzSchnitzler, Arthur@\emph{von Arthur Schnitzler}!1929-08-021@{2. 8. 1929}|(be} \toendnotes[C]{\smallbreak\pagebreak[2]} \Standort{FDH, Hs-31346,2.}
\physDesc{Brief, 1 Blatt (Briefpapier mit Trauerrand), 2 Seiten
\newline{}Handschrift: schwarze Tinte, lateinische Kurrent
\newline{}Hofmannsthal: mit schwarzer Tinte beschriftet: »\textsc{erledigt}« }\pstart
           \raggedleft{}{\pb}Wien\oindex{Wien@\textbf{Wien}|pw}, 2/8 929\pend
           \pstart
           liebe Gerty, die Briefe sind angelangt, es sind auch einige wenige
               von Gustav Schwarzkopf\pwindex{Schwarzkopf, Gustav 07.11.1853 – 13.11.1939@\textsc{Schwarzkopf, Gustav} (07.11.1853 – 13.11.1939), \emph{Schriftsteller}|pw} und Felix Salten\pwindex{Salten, Felix 06.09.1869 – 08.10.1945@\textsc{Salten, Felix} (06.09.1869 – 08.10.1945), \emph{Schriftsteller, Journalist}|pw} aus der gleichen Zeit dabei. \textcolor{gray}{Indß}
               habe ich mir die \uline{Briefe Hugo\pwindex{Hofmannsthal, Hugo von 01.02.1874 – 15.07.1929@\textsc{Hofmannsthal, Hugo von} (01.02.1874 – 15.07.1929), \emph{Schriftsteller}|pw}s an G. Schw.\pwindex{Schwarzkopf, Gustav 07.11.1853 – 13.11.1939@\textsc{Schwarzkopf, Gustav} (07.11.1853 – 13.11.1939), \emph{Schriftsteller}|pw}} von diesem geben lassen, dabei waren auch etliche ungedruckte Gedichte – ich
               habe, speciell in die Briefe vorläufg nur flüchtig hineingeblickt – es sind besondere
               Briefe aus der \textcolor{gray}{früherlieg} Zeit, – ganz wunderbares. Vor allem würd
               ich \introOben{}an Ihrer Stelle\introOben{} dies alles (es ist nicht übermäßg viel)
               abschreiben lassen, eventuell gleich in 2 Exemplaren – Soll ich dieses Paket (gleich
               mit den Briefen Hugo\pwindex{Hofmannsthal, Hugo von 01.02.1874 – 15.07.1929@\textsc{Hofmannsthal, Hugo von} (01.02.1874 – 15.07.1929), \emph{Schriftsteller}|pw}s an mich) {\pb}(vielfach undatiert) nach Aussee\oindex{Bad Aussee@\textbf{Bad Aussee}|pw} schicken, oder möchten Sie, dſs \introOben{}ich\introOben{} die
               Abschriften \substVorne{}\textsuperscript{aus}\substDazwischen{}der Briefe von\substHinten{}{ }Schwarzkopf\pwindex{Schwarzkopf, Gustav 07.11.1853 – 13.11.1939@\textsc{Schwarzkopf, Gustav} (07.11.1853 – 13.11.1939), \emph{Schriftsteller}|pw}{ }\uline{hier} besorgen lasse, (was erst im
                  September möglich wäre.)\pend
           \pstart
           Ich hoffe liebe Gerty die Tage in Aussee\oindex{Bad Aussee@\textbf{Bad Aussee}|pw} sind für
               Sie und die \textcolor{gray}{I}hren so gut und ruhig wie sie eben sein können. In
               Freundschaft mit Grüßen an Alle\pend
           \pstart
           Ihr{\\[\baselineskip]}\spacefill\mbox{Arthur}\pend
           \leftskip=0em{}\endnumbering\briefempfaengerindex{Hofmannsthal, Gertrude von@\textsc{Hofmannsthal, Gertrude von}!zzzSchnitzler, Arthur@\emph{von Arthur Schnitzler}!1929-08-021@{2. 8. 1929}|)be}\mylabel{h}\end{ledgroupsized}  \newcommand{\dateiname}{L02516}\newcommand{\titel}{Arthur Schnitzler an Gerty von Hofmannsthal, 2. 8. 1929}\newcommand{\editorInnen}{Martin Anton Müller und Gerd-Hermann Susen}%% latex-leseansicht-abspann.tex
%% Abspann für die Leseansicht.
%% Der Schalter \ifkorrekturansicht ist bereits durch den Vorspann gesetzt.

%% latex-abspann.tex
%% Gemeinsamer Abspann für Korrekturansicht und Leseansicht.
%% Setzt den Schalter \ifkorrekturansicht voraus (gesetzt in den
%% einbindenden Dateien latex-korrekturansicht-abspann.tex bzw.
%% latex-leseansicht-abspann.tex).
%% ---------------------------------------------------------------

\normalsize

% Das esempio-Environment wird nur in der Leseansicht benötigt
\ifkorrekturansicht\else
\newenvironment{esempio}[3]%
{
    \vspace{1.5ex}
    \rlap{\underline{#1}}
    \par
    \setlength{\parindent}{0cm}
    \nopagebreak
    \leftskip=#2cm
    \rightskip=#3cm
}
{
    \par
}
\fi

\doendnotes{C}
\bigskip
\vfill

\clearpage

\footnotesize

\ifkorrekturansicht
  \lohead{\textsc{register}}
\fi

% theindex-Environment neu definieren ohne reledmac
\makeatletter
\renewenvironment{theindex}{%
  \ifkorrekturansicht
    \section*{\indexname}%
  \else
    \subsubsection*{Index der erwähnten Entitäten}%
  \fi
  \setlength{\parindent}{0pt}%
  \setlength{\parskip}{0pt plus 0.3pt}%
  \let\item\@idxitem
}{%
  \ifkorrekturansicht\clearpage\fi
}
\makeatother

\IfFileExists{\jobname-pw.ind}{\input{\jobname-pw.ind}}{}

% Quellenangabe nur in der Leseansicht
\ifkorrekturansicht\else
% Fallback-Definitionen, falls die .tex-Datei \titel etc. nicht gesetzt hat
\providecommand{\titel}{}
\providecommand{\editorInnen}{}
\providecommand{\dateiname}{\jobname}

\vspace{3cm}

\vfill

\footnotesize
\textsc{Quelle}: \titel. Herausgegeben von {\editorInnen}. In: \emph{Arthur Schnitzler: Briefwechsel mit Autorinnen und Autoren}.
 Digitale Edition, https://schnitzler-briefe.acdh.oeaw.ac.at/{\dateiname}.html (Stand \today)
\fi

\end{document}


      