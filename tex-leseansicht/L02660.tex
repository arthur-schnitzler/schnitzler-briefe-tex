%% latex-leseansicht-vorspann.tex
%% Vorspann für die Leseansicht.
%% Lädt die gemeinsame Datei latex-vorspann.tex mit nicht gesetztem Schalter.

\newif\ifkorrekturansicht
\korrekturansichtfalse

\input{../tex-inputs/latex-vorspann}

\begin{center}
            \textcolor{red}{ENTWURF. ENTZIFFERUNG NOCH NICHT KORREKTURGELESEN}
                      \end{center}
            
               \section[Paul Goldmann an Arthur Schnitzler, 6. 4. 1891]{ Paul Goldmann an Arthur Schnitzler, 6. 4. 1891}\nopagebreak\mylabel{v}\rehead{ }\begin{ledgroupsized}[t]{13cm}\normalsize\beginnumbering\briefempfaengerindex{Schnitzler, Arthur@\textsc{Schnitzler, Arthur}!zzzGoldmann, Paul@\emph{von Paul Goldmann}!1891-04-061@{6. 4. 1891}|(be} \toendnotes[C]{\smallbreak\pagebreak[2]} \Standort{DLA, A:Schnitzler, HS.NZ85.1.3162.}
\physDesc{Brief, 1 Blatt, 4 Seiten
\newline{}Handschrift: schwarze Tinte, deutsche Kurrent
\newline{}Schnitzler: mit rotem Buntstift eine Unterstreichung }\toendnotes[C]{\smallbreak}\pstart
           \noindent{}{\pb}\textcolor{gray}{\textbf{\textsc{Frankfurter Zeitung}}}\orgindex{Frankfurter Zeitung@Frankfurter Zeitung|pw}\pend
           \pstart
           \textcolor{gray}{\textbf{\textsc{und}}}\pend
           \pstart
           \textcolor{gray}{\textbf{\textsc{Handelsblatt.}}}\hfill \textcolor{gray}{\textbf{Frankfurt a. M.\oindex{Frankfurt am Main@\textbf{Frankfurt am Main}|pw}, }}6. April \textcolor{gray}{\textbf{189}}1.\pend
           \pstart
           \textcolor{gray}{\textbf{\textbf{\textsc{Redaction.}}}}\pend
           \pstart
           \textcolor{gray}{\textbf{\textbf{\textsc{Telegramm-Adresse:}}}}\pend
           \pstart
           \textcolor{gray}{\textbf{\textbf{\textsc{Zeitung Frankfurt
                              Main\oindex{Frankfurt am Main@\textbf{Frankfurt am Main}|pw}.}}}}\pend
           \pstart\center{}Mein lieber Arthur!\pend\pstart
           Die Geſchichte von den Grenzen der menſchlichen Empfindungsfähigkeit iſt wohl
               richtig; aber es bleibt Einem doch nicht erſpart, die ganze Größe des Schmerzes zu
               empfinden, nicht auf einmal zwar, aber ratenweis, in einzelnen Attaquen. Ich habe
                  heut{ }Nacht wieder ſo ein wildes \label{K_L02660-22v}\edtext{Heimwehfieber}{\lemma{\textnormal{\emph{Heimwehfieber}}}\Cendnote{\textnormal{Im \emph{Tagebuch}\pwindex{Schnitzler, Arthur 15.05.1862 – 21.10.1931@\textsc{Schnitzler, Arthur} (15.05.1862 – 21.10.1931), \emph{Schriftsteller, Mediziner}!Tagebuch1981 – 2000@\strich\emph{Tagebuch} {[}1981 – 2000{]}|pwk} fasste Schnitzler\pwindex{Schnitzler, Arthur 15.05.1862 – 21.10.1931@\textsc{Schnitzler, Arthur} (15.05.1862 – 21.10.1931), \emph{Schriftsteller, Mediziner}|pwk} den Brief zusammen: »Heute von Goldmann\pwindex{Goldmann, Paul 31.01.1865 – 25.09.1935@\textsc{Goldmann, Paul} (31.01.1865 – 25.09.1935), \emph{Schriftsteller, Journalist}|pw} der erste Brief, fühlt sich in
                        Frankfurt\oindex{Frankfurt am Main@\textbf{Frankfurt am Main}|pw} sehr unglücklich.«
                     (8. 4. 1891)}}}\label{K_L02660-22h}
               durchgemacht; und wenn ich feig wäre, möchte ich den nächſten Zug benutzen und in der
               geliebten Stadt\oindex{Wien@\textbf{Wien}|pwv} mich in irgend
               einen Winkel verkriechen und nimmer daraus hervorkommen. Weiß der Himmmel – es kommt
               mir vor, als hätte ich die größte Dummheit gemacht, da ich von Wien\oindex{Wien@\textbf{Wien}|pw} wegging. Hier iſt es öde und troſtlos: die kleine Stadt\oindex{Frankfurt am Main@\textbf{Frankfurt am Main}|pwv}, die unſympathiſchen
               Menſchen und Langweile an allen Ecken und Enden; man kommt ſich vor wie im Gefängniß,
               und der Ruck, mit dem {\pb}die ſchwere Thür hinter Einem
               in’s Schloß gefallen, zittert in allen Nerven nach. Meinen Onkel\pwindex{Mamroth, Fedor 21.02.1851 – 25.06.1907@\textsc{Mamroth, Fedor} (21.02.1851 – 25.06.1907), \emph{Journalist, Kritiker}|pwv} finde ich stumpf, gedrückt, reſignirt
               wieder, halb erſtickt von der Kleinſtadtatmoſphäre, mit einer tollen Sehnſucht nach
               der Welt draußen und, ich glaube auch, nach Wien\oindex{Wien@\textbf{Wien}|pw}
               im Herzen. Meine Mutter\pwindex{Goldmann, Clementine 1842-05-15 – 1924-02-24@\textsc{Goldmann, Clementine} (1842-05-15 – 1924-02-24)|pwv}
               krank, gealtert, ſorgenvoll, tief unglücklich. Was ich von den Verhältniſſen in der
               deutſchen Journaliſtik bisher gehört habe, lautet höchſt unerquicklich und läßt die
                  Wien\oindex{Wien@\textbf{Wien}|pw}er Zuſtände eher günſtiger erſcheinen. Die
               hieſigen Collegen empfingen mich freundlich aber kühl, wie es ſchon in Preußen\oindex{Preussen@\textbf{Preußen}|pw} Brauch iſt. Zum Chefredacteur\pwindex{Sonnemann, Leopold 1831-10-29 – 1909-10-30@\textsc{Sonnemann, Leopold} (1831-10-29 – 1909-10-30), \emph{Journalist, Herausgeber}|pwv} vorzudringen iſt mir noch
               nicht gelungen. Vorläufig heißt es, daß ich bis 1. Juni hierbleiben ſoll; Näheres iſt noch nicht verfügt. Was daraus
               werden ſoll, weiß ich nicht. Mir ſcheint, ich hätte beſſer gethan, als {\pb}Stiefelputzer bei irgendwem in Wien\oindex{Wien@\textbf{Wien}|pw} zu bleiben. Hier draußen iſt das Sibirien\oindex{Sibirien@\textbf{Sibirien}|pw} und die Verbannung.\pend
           \pstart
           Dir und allen Freunden danke ich noch von ganzem Herzen für alles Liebe, das Ihr \strikeout{mich} mir bis zum Schluß gethan. Beim \label{K_L02660-2v}\edtext{Abſchied}{\lemma{\textnormal{\emph{Abſchied}}}\Cendnote{\textnormal{Goldmann\pwindex{Goldmann, Paul 31.01.1865 – 25.09.1935@\textsc{Goldmann, Paul} (31.01.1865 – 25.09.1935), \emph{Schriftsteller, Journalist}|pwk} war am 1. 4. 1891 abgereist.}}}\label{K_L02660-2h} hätte ich Euch
               gern noch ein Paar innige Worte geſagt, ſand aber nur – wie gewöhnlich – ein Paar
               dumme Witze. Auch jetzt finde ich den rechten Ausdruck nicht; ich mag auch nach
               keiner ſtylvollen Redewendung ſuchen. Mir brennt im Herzen die Trauer um Euch Alle, –
               die Überzeugung, daß ich es nie mehr wieder ſo gut haben werde wie bei Euch – und der
               eitle Schmerz, daß ich jetzt schon ganz erſetzt und halb vergeſſen bin.\pend
           \pstart
           Schreib’ mir bald, grüß’ mir Alle – beſonders \textsc{Richard\pwindex{Beer-Hofmann, Richard 11.07.1866 – 26.09.1945@\textsc{Beer-Hofmann, Richard} (11.07.1866 – 26.09.1945), \emph{Schriftsteller}|pw}}, \textsc{Loris\pwindex{Hofmannsthal, Hugo von 01.02.1874 – 15.07.1929@\textsc{Hofmannsthal, Hugo von} (01.02.1874 – 15.07.1929), \emph{Schriftsteller}|pw}} und die \textsc{Fanjungs\pwindex{Van-Jung, Leo 15.10.1866 – 02.07.1939@\textsc{Van-Jung, Leo} (15.10.1866 – 02.07.1939), \emph{Gesangspädagoge, Mathematiker}|pwv}\pwindex{Van-Jung, Boris 15.10.1872 – 03.10.1899@\textsc{Van-Jung, Boris} (15.10.1872 – 03.10.1899), \emph{Mediziner}|pwv}} – und wenn Du Dich {\pb}ſelbſt erwiſcheſt, ſo
               grüß’ Dich, ſo oft Du kannſt (Briefkaſtenwitz!).\pend
           \pstart
           Dein treuer {\\[\baselineskip]}\spacefill\mbox{Paul Goldmann.}\pend
           \leftskip=0em{}\pstart
           \noindent{}Zeige dieſen Brief, wenn Du willſt, dem kleinen \textsc{Richard\pwindex{Beer-Hofmann, Richard 11.07.1866 – 26.09.1945@\textsc{Beer-Hofmann, Richard} (11.07.1866 – 26.09.1945), \emph{Schriftsteller}|pw}}, ſonſt aber Niemandem.\pend
           \pstart
           Empfehlungen an Deine Familie.\pend
           \endnumbering\briefempfaengerindex{Schnitzler, Arthur@\textsc{Schnitzler, Arthur}!zzzGoldmann, Paul@\emph{von Paul Goldmann}!1891-04-061@{6. 4. 1891}|)be}\mylabel{h}\end{ledgroupsized}  \newcommand{\dateiname}{L02660}\newcommand{\titel}{Paul Goldmann an Arthur Schnitzler, 6. 4. 1891}\newcommand{\editorInnen}{Martin Anton Müller und Laura Untner}%% latex-leseansicht-abspann.tex
%% Abspann für die Leseansicht.
%% Der Schalter \ifkorrekturansicht ist bereits durch den Vorspann gesetzt.

%% latex-abspann.tex
%% Gemeinsamer Abspann für Korrekturansicht und Leseansicht.
%% Setzt den Schalter \ifkorrekturansicht voraus (gesetzt in den
%% einbindenden Dateien latex-korrekturansicht-abspann.tex bzw.
%% latex-leseansicht-abspann.tex).
%% ---------------------------------------------------------------

\normalsize

% Das esempio-Environment wird nur in der Leseansicht benötigt
\ifkorrekturansicht\else
\newenvironment{esempio}[3]%
{
    \vspace{1.5ex}
    \rlap{\underline{#1}}
    \par
    \setlength{\parindent}{0cm}
    \nopagebreak
    \leftskip=#2cm
    \rightskip=#3cm
}
{
    \par
}
\fi

\doendnotes{C}
\bigskip
\vfill

\clearpage

\footnotesize

\ifkorrekturansicht
  \lohead{\textsc{register}}
\fi

% theindex-Environment neu definieren ohne reledmac
\makeatletter
\renewenvironment{theindex}{%
  \ifkorrekturansicht
    \section*{\indexname}%
  \else
    \subsubsection*{Index der erwähnten Entitäten}%
  \fi
  \setlength{\parindent}{0pt}%
  \setlength{\parskip}{0pt plus 0.3pt}%
  \let\item\@idxitem
}{%
  \ifkorrekturansicht\clearpage\fi
}
\makeatother

\IfFileExists{\jobname-pw.ind}{\input{\jobname-pw.ind}}{}

% Quellenangabe nur in der Leseansicht
\ifkorrekturansicht\else
% Fallback-Definitionen, falls die .tex-Datei \titel etc. nicht gesetzt hat
\providecommand{\titel}{}
\providecommand{\editorInnen}{}
\providecommand{\dateiname}{\jobname}

\vspace{3cm}

\vfill

\footnotesize
\textsc{Quelle}: \titel. Herausgegeben von {\editorInnen}. In: \emph{Arthur Schnitzler: Briefwechsel mit Autorinnen und Autoren}.
 Digitale Edition, https://schnitzler-briefe.acdh.oeaw.ac.at/{\dateiname}.html (Stand \today)
\fi

\end{document}


      