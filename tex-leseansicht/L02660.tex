%% latex-korrekturansicht-vorspann.tex
%% Vorspann für die Korrekturansicht.
%% Lädt die gemeinsame Datei latex-vorspann.tex mit gesetztem Schalter.

\newif\ifkorrekturansicht
\korrekturansichttrue

\input{../tex-inputs/latex-vorspann}


\section[Paul Goldmann an Arthur Schnitzler, 6. 4. 1891]{L02660 Paul Goldmann an Arthur Schnitzler, 6. 4. 1891}
\nopagebreak\mylabel{L02660v}
\rehead{ }\normalsize\beginnumbering\briefempfaengerindex{Schnitzler, Arthur@\textsc{Schnitzler, Arthur}!zzzGoldmann, Paul@\emph{von Paul Goldmann}!1891-04-061@{6. 4. 1891}|(be}
\toendnotes[C]{\smallbreak\pagebreak[2]}\Standort{DLA, A:Schnitzler, HS.NZ85.1.3162.}
\physDesc{Brief, 1 Blatt, 4 Seiten, 2506 Zeichen
\newline{}Handschrift: schwarze Tinte, deutsche Kurrent
\newline{}Schnitzler: mit rotem Buntstift eine Unterstreichung }\toendnotes[C]{\smallbreak}
\pstart
           {\pb}\textcolor{gray}{\textbf{\textsc{Frankfurter Zeitung}}}\orgindex{Frankfurter Zeitung@Frankfurter Zeitung|pw}\pend
           
\pstart
           \textcolor{gray}{\textbf{\textsc{und}}}\pend
           
\pstart
           \textcolor{gray}{\textbf{\textsc{Handelsblatt.}}}\hfill \textcolor{gray}{\textbf{Frankfurt a. M.\oindex{Frankfurt am Main@\textbf{Frankfurt am Main}, \emph{P.PPLA3}|pw}, }}6. April \textcolor{gray}{\textbf{189}}1.\pend
           
\pstart
           \textcolor{gray}{\textbf{\textbf{\textsc{Redaction.}}}}\pend
           
\pstart
           \textcolor{gray}{\textbf{\textbf{\textsc{Telegramm-Adresse:}}}}\pend
           
\pstart
           \textcolor{gray}{\textbf{\textbf{\textsc{Zeitung Frankfurt
                              Main\oindex{Frankfurt am Main@\textbf{Frankfurt am Main}, \emph{P.PPLA3}|pw}.}}}}\pend
           
\pstart\center{}Mein lieber Arthur!\pend\vspace{0.5em}
\pstart
           Die Geſchichte von den Grenzen der menſchlichen Empfindungsfähigkeit iſt wohl
               richtig; aber es bleibt Einem doch nicht erſpart, die ganze Größe des Schmerzes zu
               empfinden, nicht auf einmal zwar, aber ratenweis, in einzelnen Attaquen. Ich habe
                  heut{ }Nacht wieder ſo ein wildes \label{K_L02660-1v}\edtext{Heimwehfieber}{\lemma{\textnormal{\emph{Heimwehfieber}}}\Cendnote{\textnormal{Im \emph{Tagebuch}\pwindex{Tagebuch@\emph{Tagebuch}|pwk} fasste Schnitzler den Brief zusammen: »Heute von Goldmann\pwindex{Goldmann, Paul 31.01.1865 – 25.09.1935@\textsc{Goldmann, Paul} (31.01.1865 – 25.09.1935), \emph{Schriftsteller/Schriftstellerin, Journalist/Journalistin}|pw} der erste Brief, fühlt sich in
                        Frankfurt\oindex{Frankfurt am Main@\textbf{Frankfurt am Main}, \emph{P.PPLA3}|pw} sehr unglücklich.«
                     (8. 4. 1891.)}}}\label{K_L02660-1} durchgemacht; und wenn ich feig wäre, möchte ich den nächſten Zug
               benutzen und in der geliebten Stadt\oindex{Wien@\textbf{Wien}, \emph{A.ADM2}|pwv} mich in irgend einen Winkel verkriechen und nimmer daraus
               hervorkommen. Weiß der Himmmel – es kommt mir vor, als hätte ich die größte Dummheit
               gemacht, da ich von Wien\oindex{Wien@\textbf{Wien}, \emph{A.ADM2}|pw} wegging. Hier iſt es öde
               und troſtlos: die kleine Stadt\oindex{Frankfurt am Main@\textbf{Frankfurt am Main}, \emph{P.PPLA3}|pwv},
               die unſympathiſchen Menſchen und Langweile an allen Ecken und Enden; man kommt ſich
               vor wie im Gefängniß, und der Ruck, mit dem {\pb}die
               ſchwere Thür hinter Einem in’s Schloß gefallen, zittert in allen Nerven nach. Meinen
                  Onkel\pwindex{Mamroth, Fedor 21.02.1851 – 25.06.1907@\textsc{Mamroth, Fedor} (21.02.1851 – 25.06.1907), \emph{Journalist/Journalistin, Kritiker/Kritikerin}|pwv} finde ich stumpf,
               gedrückt, reſignirt wieder, halb erſtickt von der Kleinſtadtatmoſphäre, mit einer
               tollen Sehnſucht nach der Welt draußen und, ich glaube auch, nach Wien\oindex{Wien@\textbf{Wien}, \emph{A.ADM2}|pw} im Herzen. Meine Mutter\pwindex{Goldmann, Clementine 1842-05-15 – 1924-02-24@\textsc{Goldmann, Clementine} (1842-05-15 – 1924-02-24)|pwv} krank, gealtert, ſorgenvoll, tief unglücklich. Was ich
               von den Verhältniſſen in der deutſchen Journaliſtik bisher gehört habe, lautet höchſt
               unerquicklich und läßt die Wien\oindex{Wien@\textbf{Wien}, \emph{A.ADM2}|pw}er Zuſtände eher
               günſtiger erſcheinen. Die hieſigen Collegen empfingen mich freundlich aber kühl, wie
               es ſchon in Preußen\oindex{Preussen@\textbf{Preußen}, \emph{Land (A.LND)}|pw} Brauch iſt. Zum Chefredacteur\pwindex{Sonnemann, Leopold 1831-10-29 – 1909-10-30@\textsc{Sonnemann, Leopold} (1831-10-29 – 1909-10-30), \emph{Journalist/Journalistin, Herausgeber/Herausgeberin}|pwv} vorzudringen
               iſt mir noch nicht gelungen. Vorläufig heißt es, daß ich bis 1. Juni hierbleiben ſoll; Näheres iſt noch nicht verfügt. Was daraus
               werden ſoll, weiß ich nicht. Mir ſcheint, ich hätte beſſer gethan, als {\pb}Stiefelputzer bei irgendwem in Wien\oindex{Wien@\textbf{Wien}, \emph{A.ADM2}|pw} zu bleiben. Hier draußen iſt das Sibirien\oindex{Sibirien@\textbf{Sibirien}, \emph{L.RGN}|pw} und die Verbannung.\pend
           
\pstart
           Dir und allen Freunden danke ich noch von ganzem Herzen für alles Liebe, das Ihr \strikeout{mich} mir bis zum Schluß gethan. Beim \label{K_L02660-2v}\edtext{Abſchied}{\lemma{\textnormal{\emph{Abſchied}}}\Cendnote{\textnormal{Goldmann\pwindex{Goldmann, Paul 31.01.1865 – 25.09.1935@\textsc{Goldmann, Paul} (31.01.1865 – 25.09.1935), \emph{Schriftsteller/Schriftstellerin, Journalist/Journalistin}|pwk} war am 1. 4. 1891 abgereist.}}}\label{K_L02660-2} hätte ich
               Euch gern noch ein Paar innige Worte geſagt, ſand aber nur – wie gewöhnlich – ein
               Paar dumme Witze. Auch jetzt finde ich den rechten Ausdruck nicht; ich mag auch nach
               keiner ſtylvollen Redewendung ſuchen. Mir brennt im Herzen die Trauer um Euch Alle, –
               die Überzeugung, daß ich es nie mehr wieder ſo gut haben werde wie bei Euch – und der
               eitle Schmerz, daß ich jetzt schon ganz erſetzt und halb vergeſſen bin.\pend
           
\pstart
           Schreib’ mir bald, grüß’ mir Alle – beſonders \textsc{Richard\pwindex{Beer-Hofmann, Richard 1866-07-11 – 1945-09-26@\textsc{Beer-Hofmann, Richard} (1866-07-11 – 1945-09-26), \emph{Schriftsteller/Schriftstellerin}|pw}}, \textsc{Loris\pwindex{Hofmannsthal, Hugo von 1874-02-01 – 1929-07-15@\textsc{Hofmannsthal, Hugo von} (1874-02-01 – 1929-07-15), \emph{Schriftsteller/Schriftstellerin}|pw}} und die \textsc{Fanjungs\pwindex{Van-Jung, Leo 15.10.1866 – 02.07.1939@\textsc{Van-Jung, Leo} (15.10.1866 – 02.07.1939), \emph{Gesangspädagoge/Gesangspädagogin, Mathematiker/Mathematikerin}|pwv}\pwindex{Van-Jung, Boris 15.10.1872 – 03.10.1899@\textsc{Van-Jung, Boris} (15.10.1872 – 03.10.1899), \emph{Mediziner/Medizinerin}|pwv}} – und wenn Du Dich {\pb}ſelbſt erwiſcheſt, ſo
               grüß’ Dich, ſo oft Du kannſt (Briefkaſtenwitz!).\pend
           
\pstart
           Dein treuer {\\[\baselineskip]}\spacefill\mbox{Paul Goldmann.}\pend
           \leftskip=0em{}
\pstart
           \noindent{}Zeige dieſen Brief, wenn Du willſt, dem kleinen \textsc{Richard\pwindex{Beer-Hofmann, Richard 1866-07-11 – 1945-09-26@\textsc{Beer-Hofmann, Richard} (1866-07-11 – 1945-09-26), \emph{Schriftsteller/Schriftstellerin}|pw}}, ſonſt aber Niemandem.\pend
           
\pstart
           Empfehlungen an Deine Familie.\pend
           \selectlanguage{ngerman}\endnumbering\briefempfaengerindex{Schnitzler, Arthur@\textsc{Schnitzler, Arthur}!zzzGoldmann, Paul@\emph{von Paul Goldmann}!1891-04-061@{6. 4. 1891}|)be}\mylabel{L02660h}  \normalsize

\doendnotes{C}
\bigskip
\vfill

\clearpage

\footnotesize

\lohead{\textsc{register}}

% Definiere theindex-Environment komplett neu ohne reledmac
\makeatletter
\renewenvironment{theindex}{%
  \section*{\indexname}%
  \setlength{\parindent}{0pt}%
  \setlength{\parskip}{0pt plus 0.3pt}%
  \let\item\@idxitem
}{%
  \clearpage
}
\makeatother

\IfFileExists{\jobname-pw.ind}{\input{\jobname-pw.ind}}{}

\end{document}

      