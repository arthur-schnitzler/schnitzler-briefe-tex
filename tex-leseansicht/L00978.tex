%% latex-leseansicht-vorspann.tex
%% Vorspann für die Leseansicht.
%% Lädt die gemeinsame Datei latex-vorspann.tex mit nicht gesetztem Schalter.

\newif\ifkorrekturansicht
\korrekturansichtfalse

\input{../tex-inputs/latex-vorspann}


         
         \newcommand{\erwaehntePersonen}{Personen: Eduard Klingebeil, Adolf Pichler}
         \newcommand{\erwaehnteOrte}{Orte: Frankfurt am Main, Vahrn}
         \newcommand{\erwaehnteWerke}{Werke: Zweierlei Pegasus}
               \section[Richard Beer-Hofmann und Hugo von Hofmannsthal an Arthur Schnitzler, 19. 9. 1899]{ Richard Beer-Hofmann und Hugo von Hofmannsthal an Arthur Schnitzler,
               19. 9. 1899}\nopagebreak\mylabel{v}\rehead{ }\begin{ledgroupsized}[t]{13cm}\normalsize\beginnumbering \toendnotes[C]{\smallbreak\pagebreak[2]} \Standort{CUL, Schnitzler, B 8.}
\physDesc{Bildpostkarte
\newline{}Handschrift Hugo von Hofmannsthal: schwarze Tinte\newline{}Handschrift Richard Beer-Hofmann: schwarze Tinte, lateinische Kurrent\newline{}Versand: 1) Stempel: »\nobreak{}20. 9. 99\nobreak{}«.   2) Stempel: »\nobreak{}\oindex{Frankfurt am Main@\textbf{Frankfurt am Main}|pwk}Frankfurt (Main), 22. 9. 99, 7–8V\nobreak{}«. \newline{}Ordnung: mit Bleistift von unbekannter Hand nummeriert:
                                    »145« }\buchAbdrucke{\weitereDrucke{Arthur Schnitzler, Richard Beer-Hofmann: \emph{Briefwechsel 1891–1931}. Hg. Konstanze Fliedl. Wien, Zürich: \emph{Europaverlag} 1992, S. 137.} }\toendnotes[C]{\smallbreak}\pstart{}{\pb}D\textsuperscript{r}
                  Arthur Schnitzler\pend{}\pstart{}Frankfurt a. Main\oindex{Frankfurt am Main@\textbf{Frankfurt am Main}|pw}\pend{}\pstart{}Poste restante\pend{}{\bigskip}\pstart
           \noindent{}\centering{}\textcolor{gray}{\textbf{{\pb}Künstler-Postkarte.}}\pend
           \pstart
           \noindent{}\centering{}\textcolor{gray}{\textbf{E. Klingebeil\pwindex{Klingebeil, Eduard *~1863-03-23@\textsc{Klingebeil, Eduard} (*~1863-03-23), \emph{Maler}|pw}: Zweierlei Pegasus\pwindex{Klingebeil, Eduard *~1863-03-23@\textsc{Klingebeil, Eduard} (*~1863-03-23), \emph{Maler}!Zweierlei Pegasus1898@\strich\emph{Zweierlei Pegasus} {[}1898{]}|pw}.}}\pend
           \pstart
           \raggedleft{}Vahrn\oindex{Vahrn@\textbf{Vahrn}|pw}{ }19/IX 1899\pend
           \pstart
           \label{K_L00978_1v}\edtext{Adolf Pichler\pwindex{Pichler, Adolf 04.09.1819 – 15.11.1900@\textsc{Pichler, Adolf} (04.09.1819 – 15.11.1900), \emph{Schriftsteller, Naturforscher}|pw}}{\lemma{\textnormal{\emph{Adolf Pichler}}}\Cendnote{\textnormal{Die Gegenüberstellung der beiden
                  Schriftsteller Schnitzler\pwindex{Schnitzler, Arthur 15.05.1862 – 21.10.1931@\textsc{Schnitzler, Arthur} (15.05.1862 – 21.10.1931), \emph{Schriftsteller, Mediziner}|pwk} und Adolf Pichler\pwindex{Pichler, Adolf 04.09.1819 – 15.11.1900@\textsc{Pichler, Adolf} (04.09.1819 – 15.11.1900), \emph{Schriftsteller, Naturforscher}|pwk} möchte nicht nur durch die Zuordnung zu den zwei auf
                  der Karte dargestellten Poeten – der eine reitet mit einer Lyra auf einem Pegasus zum
                  Himmel, der andere mit einem Leierkasten und einer Tänzerin auf einem Schwein
                  durch den Dreck – witzig sein, sondern zieht den Humor auch aus dem
                  Altersunterschied: Pichler\pwindex{Pichler, Adolf 04.09.1819 – 15.11.1900@\textsc{Pichler, Adolf} (04.09.1819 – 15.11.1900), \emph{Schriftsteller, Naturforscher}|pwk} wurde am
                     4. 9. 1899 achtzig.}}}\label{K_L00978_1h}\hspace*{2.5em} Arthur S.\pend
           \pstart
           Dies wünschen Ihnen\pend
           \pstart
           \spacefill\mbox{Richard}{\\[\baselineskip]}\spacefill\mbox{{[}hs. Hofmannsthal:{]} Hugo}\pend
           \leftskip=0em{}
         
         \endnumbering\mylabel{h}\end{ledgroupsized}  \newcommand{\dateiname}{L00978}\newcommand{\titel}{Richard Beer-Hofmann und Hugo von Hofmannsthal an Arthur Schnitzler, 19. 9. 1899}\newcommand{\editorInnen}{Martin Anton Müller und Gerd-Hermann Susen}%% latex-leseansicht-abspann.tex
%% Abspann für die Leseansicht.
%% Der Schalter \ifkorrekturansicht ist bereits durch den Vorspann gesetzt.

%% latex-abspann.tex
%% Gemeinsamer Abspann für Korrekturansicht und Leseansicht.
%% Setzt den Schalter \ifkorrekturansicht voraus (gesetzt in den
%% einbindenden Dateien latex-korrekturansicht-abspann.tex bzw.
%% latex-leseansicht-abspann.tex).
%% ---------------------------------------------------------------

\normalsize

% Das esempio-Environment wird nur in der Leseansicht benötigt
\ifkorrekturansicht\else
\newenvironment{esempio}[3]%
{
    \vspace{1.5ex}
    \rlap{\underline{#1}}
    \par
    \setlength{\parindent}{0cm}
    \nopagebreak
    \leftskip=#2cm
    \rightskip=#3cm
}
{
    \par
}
\fi

\doendnotes{C}
\bigskip
\vfill

\clearpage

\footnotesize

\ifkorrekturansicht
  \lohead{\textsc{register}}
\fi

% theindex-Environment neu definieren ohne reledmac
\makeatletter
\renewenvironment{theindex}{%
  \ifkorrekturansicht
    \section*{\indexname}%
  \else
    \subsubsection*{Index der erwähnten Entitäten}%
  \fi
  \setlength{\parindent}{0pt}%
  \setlength{\parskip}{0pt plus 0.3pt}%
  \let\item\@idxitem
}{%
  \ifkorrekturansicht\clearpage\fi
}
\makeatother

\IfFileExists{\jobname-pw.ind}{\input{\jobname-pw.ind}}{}

% Quellenangabe nur in der Leseansicht
\ifkorrekturansicht\else
% Fallback-Definitionen, falls die .tex-Datei \titel etc. nicht gesetzt hat
\providecommand{\titel}{}
\providecommand{\editorInnen}{}
\providecommand{\dateiname}{\jobname}

\vspace{3cm}

\vfill

\footnotesize
\textsc{Quelle}: \titel. Herausgegeben von {\editorInnen}. In: \emph{Arthur Schnitzler: Briefwechsel mit Autorinnen und Autoren}.
 Digitale Edition, https://schnitzler-briefe.acdh.oeaw.ac.at/{\dateiname}.html (Stand \today)
\fi

\end{document}


      