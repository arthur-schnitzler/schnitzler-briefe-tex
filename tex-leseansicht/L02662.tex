%% latex-leseansicht-vorspann.tex
%% Vorspann für die Leseansicht.
%% Lädt die gemeinsame Datei latex-vorspann.tex mit nicht gesetztem Schalter.

\newif\ifkorrekturansicht
\korrekturansichtfalse

\input{../tex-inputs/latex-vorspann}


\section[Paul Goldmann an Arthur Schnitzler, 11. 5. 1891]{L02662 Paul Goldmann an Arthur Schnitzler, 11. 5. 1891}
\nopagebreak\mylabel{L02662v}
\rehead{ }\normalsize\beginnumbering\briefempfaengerindex{Schnitzler, Arthur@\textsc{Schnitzler, Arthur}!zzzGoldmann, Paul@\emph{von Paul Goldmann}!1891-05-111@{11. 5. 1891}|(be}
\toendnotes[C]{\smallbreak\pagebreak[2]}
\correspDesc{Versand  durch Paul Goldmann am 11. 5. 1891 in Lüttich
\newline{}Erhalt  durch Arthur Schnitzler am 14. 5. 1891 in Wien}\toendnotes[C]{\smallbreak}
\Standort{DLA, A:Schnitzler, HS.NZ85.1.3162.}
\physDesc{Postkarte, 998 Zeichen
\newline{}Handschrift: schwarze Tinte, deutsche Kurrent
\newline{}Versand: 1) Stempel: »\nobreak{}\oindex{Lüttich@\textbf{Lüttich}|pwk}Liege, 11 Mai {[}1891{]}, 11–S\nobreak{}«.   2) Stempel: »\nobreak{}\oindex{Wien@\textbf{Wien}, \emph{Verwaltungsgebiet}|pwk}Wien 1/1, Bestellt, 14{[}.{]} 5. 91, VIII–IX½\nobreak{}«. 
\newline{}Schnitzler: mit Bleistift das Datum »11/ 5. 91« vermerkt }\toendnotes[C]{\smallbreak}\pstart{}\textsc{{\pb}\begin{otherlanguage}{french}Autriche\end{otherlanguage}\oindex{Österreich@\textbf{Österreich}|pw}!}\pend{}\pstart{}\textsc{\begin{otherlanguage}{french}\textcolor{gray}{\textbf{M}}onsieur le docteur\end{otherlanguage} Arthur
                  Schnitzler}\pend{}\pstart{}\textsc{\begin{otherlanguage}{french}Vienne\end{otherlanguage}\oindex{Wien@\textbf{Wien}, \emph{Verwaltungsgebiet}|pw}}\pend{}\pstart{}\textsc{I. Giselastraſse 11\oindex{Wien@\textbf{Wien}!I., Innere Stadt@\textbf{I., Innere Stadt}!Ordination Arthur Schnitzler [Bösendorferstraße 11]@\textbf{Ordination Arthur Schnitzler [Bösendorferstraße 11]}, \emph{Ordination}|pw}.}\pend{}{\bigskip}\vspace{1em}
\pstart
           \noindent{}{\pb}Lüttich\oindex{Lüttich@\textbf{Lüttich}|pw}{ }11. Mai. Lieber alter Freund! Einen
               kurzen Gruß einſtweilen. Ich habe über Nacht Marſchbefehl erhalten und bin{ }ſeit heut im belgiſchen\oindex{Belgien@\textbf{Belgien}|pw}{ }\label{K_L02662-1v}\edtext{Strikerevier}{\lemma{\textnormal{\emph{Strikerevier}}}\Cendnote{\textnormal{Bergarbeiterinnen und Bergarbeiter hatten am
                     2. 5. 1891 einen Streik begonnen, der sich in Folge auch auf
                  andere Berufsgruppen ausweitete und zu einem massiven Einsatz von staatlicher
                  Gewalt führte.}}}\label{K_L02662-1}. Fürchterliche Arbeit – aber eine neue, herrliche Welt. Ich{ }ſtecke voll neuer Eindrücke bis unter’s Dach. Soeben habe ich einen Apoſtel der Heilsarmee\orgindex{Heilsarmee@Heilsarmee|pw}, der mich bekehren wollte,
               hinausgeſchmiſſen. Zwei Königreiche dafür, Dich mitzuhaben! Eine neue Zeit beginnt
               für mich – Gott gebe, daß die neuen Vorſätze anhalten. Eine neue Zeit auf dem Boden
               der alten, der ganz alten Moral. Kein Künſtler mehr – ein{ }ſachlicher Philiſter{ }ſtattdeſſen; kein Genußmenſch –{ }ſondern \strikeout{\textcolor{gray}{nur}} Pflichtenmenſch; nicht mehr ich –{ }ſondern ein Sohn meiner Mutter\pwindex{Goldmann, Clementine 15.\,5.\,1842 Breslau – 24.\,2.\,1924 Frankfurt am Main@\textsc{Goldmann, Clementine} (15.\,5.\,1842 Breslau – 24.\,2.\,1924 Frankfurt am Main)|pwv} und ein Bruder meiner Schweſter\pwindex{Rosengart, Vally 29.\,12.\,1866 Breslau – nach 1926@\textsc{Rosengart, Vally} (29.\,12.\,1866 Breslau – nach 1926)|pwv}. \label{K_L02662-2v}\edtext{\textsc{\begin{otherlanguage}{french}Tu tarderas de me comprendre\end{otherlanguage}}.}{\lemma{\textnormal{\emph{Tu … comprendre.}}}\Cendnote{\textnormal{französisch, etwa: Du wirst es
                  noch verstehen.}}}\label{K_L02662-2} Dank einſtweilen für Deinen lieben, lieben Brief! Zwei
               Zeilen nach Brüſſel\oindex{Brüssel@\textbf{Brüssel}, \emph{Hauptstadt}|pw}{ }\introOben{}\textsc{\uline{Poste restante}}\introOben{}{ }{\dotstwo} bitte, bitte! Ich grüße Dich von ganzem Herzen. Dein
                  \spacefill\mbox{Paul.}\pend
           
\pstart
           \noindent{}\label{T_L02662-1v}\edtext{Lüttich\oindex{Lüttich@\textbf{Lüttich}|pw} – nein, das läßt{ }ſich nicht{ }ſagen.}{\lemma{\textnormal{\emph{Lüttich … sagen.}}}\Cendnote{\textnormal{seitlich am rechten
                     Rand}}}\label{T_L02662-1}\pend
           \selectlanguage{ngerman}\endnumbering\briefempfaengerindex{Schnitzler, Arthur@\textsc{Schnitzler, Arthur}!zzzGoldmann, Paul@\emph{von Paul Goldmann}!1891-05-111@{11. 5. 1891}|)be}\mylabel{L02662h}  \newcommand{\dateiname}{L02662}\newcommand{\titel}{Paul Goldmann an Arthur Schnitzler, 11. 5. 1891}\newcommand{\editorInnen}{Martin Anton Müller und Laura Untner}%% latex-leseansicht-abspann.tex
%% Abspann für die Leseansicht.
%% Der Schalter \ifkorrekturansicht ist bereits durch den Vorspann gesetzt.

%% latex-abspann.tex
%% Gemeinsamer Abspann für Korrekturansicht und Leseansicht.
%% Setzt den Schalter \ifkorrekturansicht voraus (gesetzt in den
%% einbindenden Dateien latex-korrekturansicht-abspann.tex bzw.
%% latex-leseansicht-abspann.tex).
%% ---------------------------------------------------------------

\normalsize

% Das esempio-Environment wird nur in der Leseansicht benötigt
\ifkorrekturansicht\else
\newenvironment{esempio}[3]%
{
    \vspace{1.5ex}
    \rlap{\underline{#1}}
    \par
    \setlength{\parindent}{0cm}
    \nopagebreak
    \leftskip=#2cm
    \rightskip=#3cm
}
{
    \par
}
\fi

\doendnotes{C}
\bigskip
\vfill

\clearpage

\footnotesize

\ifkorrekturansicht
  \lohead{\textsc{register}}
\fi

% theindex-Environment neu definieren ohne reledmac
\makeatletter
\renewenvironment{theindex}{%
  \ifkorrekturansicht
    \section*{\indexname}%
  \else
    \subsubsection*{Index der erwähnten Entitäten}%
  \fi
  \setlength{\parindent}{0pt}%
  \setlength{\parskip}{0pt plus 0.3pt}%
  \let\item\@idxitem
}{%
  \ifkorrekturansicht\clearpage\fi
}
\makeatother

\IfFileExists{\jobname-pw.ind}{\input{\jobname-pw.ind}}{}

% Quellenangabe nur in der Leseansicht
\ifkorrekturansicht\else
% Fallback-Definitionen, falls die .tex-Datei \titel etc. nicht gesetzt hat
\providecommand{\titel}{}
\providecommand{\editorInnen}{}
\providecommand{\dateiname}{\jobname}

\vspace{3cm}

\vfill

\footnotesize
\textsc{Quelle}: \titel. Herausgegeben von {\editorInnen}. In: \emph{Arthur Schnitzler: Briefwechsel mit Autorinnen und Autoren}.
 Digitale Edition, https://schnitzler-briefe.acdh.oeaw.ac.at/{\dateiname}.html (Stand \today)
\fi

\end{document}


