%% latex-korrekturansicht-vorspann.tex
%% Vorspann für die Korrekturansicht.
%% Lädt die gemeinsame Datei latex-vorspann.tex mit gesetztem Schalter.

\newif\ifkorrekturansicht
\korrekturansichttrue

\input{../tex-inputs/latex-vorspann}


\section[Paul Goldmann an Arthur Schnitzler, 11. 5. 1891]{L02662 Paul Goldmann an Arthur Schnitzler, 11. 5. 1891}
\nopagebreak\mylabel{L02662v}
\rehead{ }\normalsize\beginnumbering\briefempfaengerindex{Schnitzler, Arthur@\textsc{Schnitzler, Arthur}!zzzGoldmann, Paul@\emph{von Paul Goldmann}!1891-05-111@{11. 5. 1891}|(be}
\toendnotes[C]{\smallbreak\pagebreak[2]}\Standort{DLA, A:Schnitzler, HS.NZ85.1.3162.}
\physDesc{Postkarte, 998 Zeichen
\newline{}Handschrift: 1) schwarze Tinte, deutsche Kurrent\hspace{1em}2) schwarze Tinte, lateinische Kurrent (\noindent{}Adresse)\hspace{1em}
\newline{}Versand: 1) Stempel: »\nobreak{}\oindex{Luettich@\textbf{Lüttich}, \emph{P.PPL}|pwk}Liege, 11 Mai {[}1891{]}, 11–S\nobreak{}«.   2) Stempel: »\nobreak{}Wien 1/1, Bestellt, 14{[}.{]} 5. 91, VIII–IX½\nobreak{}«. 
\newline{}Schnitzler: mit Bleistift das Datum »11/ 5. 91« vermerkt }\toendnotes[C]{\smallbreak}\pstart{}{\pb}\begin{otherlanguage}{french}Autriche\end{otherlanguage}\oindex{Oesterreich@\textbf{Österreich}, \emph{A.PCLI}|pw}! \pend{}\pstart{}\begin{otherlanguage}{french}\textcolor{gray}{\textbf{M}}onsieur le docteur\end{otherlanguage} Arthur
                  Schnitzler\pend{}\pstart{}\begin{otherlanguage}{french}Vienne\end{otherlanguage}\oindex{Wien@\textbf{Wien}, \emph{A.ADM2}|pw}\pend{}\pstart{}I. Giselastraſse 11\oindex{Ordination Arthur Schnitzler [Boesendorferstrasse 11]@\textbf{Ordination Arthur Schnitzler [Bösendorferstraße 11]}, \emph{Ordination}|pw}. \pend{}{\bigskip}\vspace{1em}
\pstart
           \noindent{}{\pb}Lüttich\oindex{Luettich@\textbf{Lüttich}, \emph{P.PPL}|pw}{ }11. Mai. Lieber alter Freund! Einen
               kurzen Gruß einſtweilen. Ich habe über Nacht Marſchbefehl erhalten und bin ſeit heut im belgiſchen\oindex{Belgien@\textbf{Belgien}, \emph{A.PCLI}|pw}{ }\label{K_L02662-1v}\edtext{Strikerevier}{\lemma{\textnormal{\emph{Strikerevier}}}\Cendnote{\textnormal{Bergarbeiterinnen und Bergarbeiter hatten am
                     2. 5. 1891 einen Streik begonnen, der sich in Folge auch auf
                  andere Berufsgruppen ausweitete und zu einem massiven Einsatz von staatlicher
                  Gewalt führte.}}}\label{K_L02662-1}. Fürchterliche Arbeit – aber eine neue, herrliche Welt. Ich
               ſtecke voll neuer Eindrücke bis unter’s Dach. Soeben habe ich einen Apoſtel der Heilsarmee\orgindex{Heilsarmee@Heilsarmee|pw}, der mich bekehren wollte,
               hinausgeſchmiſſen. Zwei Königreiche dafür, Dich mitzuhaben! Eine neue Zeit beginnt
               für mich – Gott gebe, daß die neuen Vorſätze anhalten. Eine neue Zeit auf dem Boden
               der alten, der ganz alten Moral. Kein Künſtler mehr – ein ſachlicher Philiſter
               ſtattdeſſen; kein Genußmenſch – ſondern \strikeout{\textcolor{gray}{nur}} Pflichtenmenſch; nicht mehr ich – ſondern ein Sohn meiner Mutter\pwindex{Goldmann, Clementine 1842-05-15 – 1924-02-24@\textsc{Goldmann, Clementine} (1842-05-15 – 1924-02-24)|pwv} und ein Bruder meiner Schweſter\pwindex{Rosengart, Vally 1866-12-29 – nach 1926@\textsc{Rosengart, Vally} (1866-12-29 – nach 1926)|pwv}. \label{K_L02662-2v}\edtext{\textsc{\begin{otherlanguage}{french}Tu tarderas de me comprendre\end{otherlanguage}}.}{\lemma{\textnormal{\emph{Tu … comprendre.}}}\Cendnote{\textnormal{französisch, etwa: Du wirst es
                  noch verstehen.}}}\label{K_L02662-2} Dank einſtweilen für Deinen lieben, lieben Brief! Zwei
               Zeilen nach Brüſſel\oindex{Bruessel@\textbf{Brüssel}, \emph{P.PPLC}|pw}{ }\introOben{}\textsc{\uline{Poste restante}}\introOben{}{ }{\dotstwo} bitte, bitte! Ich grüße Dich von ganzem Herzen. Dein
                  \spacefill\mbox{Paul.}\pend
           
\pstart
           \noindent{}\label{T_L02662-1v}\edtext{Lüttich\oindex{Luettich@\textbf{Lüttich}, \emph{P.PPL}|pw} – nein, das läßt ſich nicht
                     ſagen.}{\lemma{\textnormal{\emph{Lüttich … ſagen.}}}\Cendnote{\textnormal{seitlich am rechten
                     Rand}}}\label{T_L02662-1}\pend
           \selectlanguage{ngerman}\endnumbering\briefempfaengerindex{Schnitzler, Arthur@\textsc{Schnitzler, Arthur}!zzzGoldmann, Paul@\emph{von Paul Goldmann}!1891-05-111@{11. 5. 1891}|)be}\mylabel{L02662h}  \normalsize

\doendnotes{C}
\bigskip
\vfill

\clearpage

\footnotesize

\lohead{\textsc{register}}

% Definiere theindex-Environment komplett neu ohne reledmac
\makeatletter
\renewenvironment{theindex}{%
  \section*{\indexname}%
  \setlength{\parindent}{0pt}%
  \setlength{\parskip}{0pt plus 0.3pt}%
  \let\item\@idxitem
}{%
  \clearpage
}
\makeatother

\IfFileExists{\jobname-pw.ind}{\input{\jobname-pw.ind}}{}

\end{document}

      