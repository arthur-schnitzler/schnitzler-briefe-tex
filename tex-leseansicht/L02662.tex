%% latex-leseansicht-vorspann.tex
%% Vorspann für die Leseansicht.
%% Lädt die gemeinsame Datei latex-vorspann.tex mit nicht gesetztem Schalter.

\newif\ifkorrekturansicht
\korrekturansichtfalse

\input{../tex-inputs/latex-vorspann}


         
         \renewcommand{\erwaehntePersonen}{Personen: Paul Goldmann, Clementine Goldmann, Vally Rosengart}
         \renewcommand{\erwaehnteInstitutionen}{Institutionen: Heilsarmee}
         \renewcommand{\erwaehnteOrte}{Orte: Belgien, Brüssel, Lüttich, Ordination Dr. Arthur Schnitzler Giselastraße 11, Wien, Österreich}
         \renewcommand{\erwaehnteWerke}{}
               \section[Paul Goldmann an Arthur Schnitzler, 11. 5. 1891]{ Paul Goldmann an Arthur Schnitzler, 11. 5. 1891}\nopagebreak\mylabel{v}\rehead{ }\begin{ledgroupsized}[t]{13cm}\normalsize\beginnumbering\briefempfaengerindex{Schnitzler, Arthur@\textsc{Schnitzler, Arthur}!zzzGoldmann, Paul@\emph{von Paul Goldmann}!1891-05-111@{11. 5. 1891}|(be} \toendnotes[C]{\smallbreak\pagebreak[2]} \Standort{DLA, A:Schnitzler, HS.NZ85.1.3162.}
\physDesc{Postkarte, 998 Zeichen
\newline{}Handschrift: 1) schwarze Tinte, deutsche Kurrent\hspace{1em}2) schwarze Tinte, lateinische Kurrent (\noindent{}Adresse)\hspace{1em}
\newline{}Versand: 1) Stempel: »\nobreak{}\oindex{Luettich@\textbf{Lüttich}|pwk}Liege, 11 Mai {[}1891{]}, 11–S\nobreak{}«.   2) Stempel: »\nobreak{}Wien 1/1, Bestellt, 14{[}.{]} 5. 91, VIII–IX½\nobreak{}«. 
\newline{}Schnitzler: mit Bleistift das Datum »11/ 5. 91« vermerkt }\toendnotes[C]{\smallbreak}\pstart{}{\pb}\begin{otherlanguage}{french}Autriche\end{otherlanguage}\oindex{Oesterreich@\textbf{Österreich}|pw}! \pend{}\pstart{}\begin{otherlanguage}{french}\textcolor{gray}{\textbf{M}}onsieur le docteur\end{otherlanguage} Arthur
                  Schnitzler\pend{}\pstart{}\begin{otherlanguage}{french}Vienne\end{otherlanguage}\oindex{Wien@\textbf{Wien}|pw}\pend{}\pstart{}I. Giselastraſse 11\oindex{Ordination Dr. Arthur Schnitzler Giselastrasse 11@\textbf{Ordination Dr. Arthur Schnitzler Giselastraße 11}|pw}. \pend{}{\bigskip}\pstart
           \noindent{}{\pb}Lüttich\oindex{Luettich@\textbf{Lüttich}|pw}{ }11. Mai. Lieber alter Freund! Einen
               kurzen Gruß einſtweilen. Ich habe über Nacht Marſchbefehl erhalten und bin ſeit heut im belgiſchen\oindex{Belgien@\textbf{Belgien}|pw}{ }\label{K_L02662-1v}\edtext{Strikerevier}{\lemma{\textnormal{\emph{Strikerevier}}}\Cendnote{\textnormal{Bergarbeiterinnen und Bergarbeiter hatten am
                     2. 5. 1891 einen Streik begonnen, der sich in Folge auch auf
                  andere Berufsgruppen ausweitete und zu einem massiven Einsatz von staatlicher
                  Gewalt führte.}}}\label{K_L02662-1h}. Fürchterliche Arbeit – aber eine neue, herrliche Welt. Ich
               ſtecke voll neuer Eindrücke bis unter’s Dach. Soeben habe ich einen Apoſtel der Heilsarmee\orgindex{Heilsarmee@Heilsarmee|pw}, der mich bekehren wollte,
               hinausgeſchmiſſen. Zwei Königreiche dafür, Dich mitzuhaben! Eine neue Zeit beginnt
               für mich – Gott gebe, daß die neuen Vorſätze anhalten. Eine neue Zeit auf dem Boden
               der alten, der ganz alten Moral. Kein Künſtler mehr – ein ſachlicher Philiſter
               ſtattdeſſen; kein Genußmenſch – ſondern \strikeout{\textcolor{gray}{nur}} Pflichtenmenſch; nicht mehr ich – ſondern ein Sohn meiner Mutter\pwindex{Goldmann, Clementine 1842-05-15 – 1924-02-24@\textsc{Goldmann, Clementine} (1842-05-15 – 1924-02-24)|pwv} und ein Bruder meiner Schweſter\pwindex{Rosengart, Vally 1866-12-29 – nach 1926@\textsc{Rosengart, Vally} (1866-12-29 – nach 1926)|pwv}. \label{K_L02662-2v}\edtext{\textsc{\begin{otherlanguage}{french}Tu tarderas de me comprendre\end{otherlanguage}}.}{\lemma{\textnormal{\emph{Tu … comprendre.}}}\Cendnote{\textnormal{französisch, etwa: Du wirst es
                  noch verstehen.}}}\label{K_L02662-2h} Dank einſtweilen für Deinen lieben, lieben Brief! Zwei
               Zeilen nach Brüſſel\oindex{Bruessel@\textbf{Brüssel}|pw}{ }\introOben{}\textsc{\uline{Poste restante}}\introOben{}{ }{\dotstwo} bitte, bitte! Ich grüße Dich von ganzem Herzen. Dein
                  \spacefill\mbox{Paul.}\pend
           \pstart
           \noindent{}\label{T_L02662-1v}\edtext{Lüttich\oindex{Luettich@\textbf{Lüttich}|pw} – nein, das läßt ſich nicht
                     ſagen.}{\lemma{\textnormal{\emph{Lüttich … ſagen.}}}\Cendnote{\textnormal{seitlich am rechten
                     Rand}}}\label{T_L02662-1h}\pend
           
         
         \endnumbering\mylabel{h}\end{ledgroupsized}  \newcommand{\dateiname}{L02662}\newcommand{\titel}{Paul Goldmann an Arthur Schnitzler, 11. 5. 1891}\newcommand{\editorInnen}{Martin Anton Müller und Laura Untner}%% latex-leseansicht-abspann.tex
%% Abspann für die Leseansicht.
%% Der Schalter \ifkorrekturansicht ist bereits durch den Vorspann gesetzt.

%% latex-abspann.tex
%% Gemeinsamer Abspann für Korrekturansicht und Leseansicht.
%% Setzt den Schalter \ifkorrekturansicht voraus (gesetzt in den
%% einbindenden Dateien latex-korrekturansicht-abspann.tex bzw.
%% latex-leseansicht-abspann.tex).
%% ---------------------------------------------------------------

\normalsize

% Das esempio-Environment wird nur in der Leseansicht benötigt
\ifkorrekturansicht\else
\newenvironment{esempio}[3]%
{
    \vspace{1.5ex}
    \rlap{\underline{#1}}
    \par
    \setlength{\parindent}{0cm}
    \nopagebreak
    \leftskip=#2cm
    \rightskip=#3cm
}
{
    \par
}
\fi

\doendnotes{C}
\bigskip
\vfill

\clearpage

\footnotesize

\ifkorrekturansicht
  \lohead{\textsc{register}}
\fi

% theindex-Environment neu definieren ohne reledmac
\makeatletter
\renewenvironment{theindex}{%
  \ifkorrekturansicht
    \section*{\indexname}%
  \else
    \subsubsection*{Index der erwähnten Entitäten}%
  \fi
  \setlength{\parindent}{0pt}%
  \setlength{\parskip}{0pt plus 0.3pt}%
  \let\item\@idxitem
}{%
  \ifkorrekturansicht\clearpage\fi
}
\makeatother

\IfFileExists{\jobname-pw.ind}{\input{\jobname-pw.ind}}{}

% Quellenangabe nur in der Leseansicht
\ifkorrekturansicht\else
% Fallback-Definitionen, falls die .tex-Datei \titel etc. nicht gesetzt hat
\providecommand{\titel}{}
\providecommand{\editorInnen}{}
\providecommand{\dateiname}{\jobname}

\vspace{3cm}

\vfill

\footnotesize
\textsc{Quelle}: \titel. Herausgegeben von {\editorInnen}. In: \emph{Arthur Schnitzler: Briefwechsel mit Autorinnen und Autoren}.
 Digitale Edition, https://schnitzler-briefe.acdh.oeaw.ac.at/{\dateiname}.html (Stand \today)
\fi

\end{document}


      