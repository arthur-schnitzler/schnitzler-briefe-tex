%% latex-korrekturansicht-vorspann.tex
%% Vorspann für die Korrekturansicht.
%% Lädt die gemeinsame Datei latex-vorspann.tex mit gesetztem Schalter.

\newif\ifkorrekturansicht
\korrekturansichttrue

\input{../tex-inputs/latex-vorspann}


\section[ Felix Salten an Arthur Schnitzler, 26. 3. 1924]{L03583 Felix Salten an Arthur Schnitzler, 26. 3. 1924}
\nopagebreak\mylabel{L03583v}
\rehead{ }\normalsize\beginnumbering\briefempfaengerindex{Schnitzler, Arthur@\textsc{Schnitzler, Arthur}!zzzSalten, Felix@\emph{von Felix Salten}!1924-03-261@{26. 3. 1924}|(be}
\toendnotes[C]{\smallbreak\pagebreak[2]}\Standort{CUL, Schnitzler, B 89, B 2.}
\physDesc{Bildpostkarte, 132 Zeichen
\newline{}Handschrift: schwarze Tinte, lateinische Kurrent
\newline{}Versand: Stempel: »\nobreak{}\oindex{Ramla@\textbf{Ramla}, \emph{P.PPLA}|pwk}{[}R{]}\textcolor{gray}{a}mleh
                                       {[}Pa{]}lestine, 27 MR 24\nobreak{}«.  
\newline{}Ordnung: mit Bleistift von unbekannter Hand nummeriert: »294« }\pstart{}{\pb}Austria\oindex{Oesterreich@\textbf{Österreich}, \emph{A.PCLI}|pw}\pend{}\pstart{}Herrn D\textsuperscript{r} Arthur Schnitzler\pend{}\pstart{}Wien\oindex{Wien@\textbf{Wien}, \emph{A.ADM2}|pw}\pend{}\pstart{}XVIII. Sternwartestrasse 71\oindex{Sternwartestrasse 71@\textbf{Sternwartestraße 71}, \emph{Wohngebäude (K.WHS)}|pw}\pend{}{\bigskip}
\pstart
           \noindent{}{\pb}\textcolor{gray}{\textbf{\begin{otherlanguage}{french}Caifa et Mt. Carmel\end{otherlanguage} – \begin{otherlanguage}{english}Haifa and Mt. Carmel\end{otherlanguage} – Caifa y Monte Carmelo – \begin{otherlanguage}{italian}Caifa ed il
                        Monte Carmelo\end{otherlanguage}}}\oindex{Mount Carmel@\textbf{Mount Carmel}, \emph{T.MTS}|pw}\pend
           \vspace{1em}
\pstart
           \raggedleft{}{\pb}Jaffa\oindex{Jaffa@\textbf{Jaffa}, \emph{P.PPLX}|pw}, 26. 3. 24\pend
           \vspace{0.5em}
\pstart
           Viele herzlichste Grüße Ihnen {\kaufmannsund} den Ihrigen. {\\}\spacefill\mbox{Felix Salten}\pend
           \selectlanguage{ngerman}\endnumbering\briefempfaengerindex{Schnitzler, Arthur@\textsc{Schnitzler, Arthur}!zzzSalten, Felix@\emph{von Felix Salten}!1924-03-261@{26. 3. 1924}|)be}\mylabel{L03583h}  \normalsize

\doendnotes{C}
\bigskip
\vfill

\clearpage

\footnotesize

\lohead{\textsc{register}}

% Definiere theindex-Environment komplett neu ohne reledmac
\makeatletter
\renewenvironment{theindex}{%
  \section*{\indexname}%
  \setlength{\parindent}{0pt}%
  \setlength{\parskip}{0pt plus 0.3pt}%
  \let\item\@idxitem
}{%
  \clearpage
}
\makeatother

\IfFileExists{\jobname-pw.ind}{\input{\jobname-pw.ind}}{}

\end{document}

      