%% latex-leseansicht-vorspann.tex
%% Vorspann für die Leseansicht.
%% Lädt die gemeinsame Datei latex-vorspann.tex mit nicht gesetztem Schalter.

\newif\ifkorrekturansicht
\korrekturansichtfalse

\input{../tex-inputs/latex-vorspann}


\section[Arthur Schnitzler an Theodor Herzl, 25. 7. 1892]{L03900 Arthur Schnitzler an Theodor Herzl, 25. 7. 1892}
\nopagebreak\mylabel{L03900v}
\rehead{ }\normalsize\beginnumbering\briefempfaengerindex{Herzl, Theodor@\textsc{Herzl, Theodor}!zzzSchnitzler, Arthur@\emph{von Arthur Schnitzler}!1892-07-251@{25. 7. 1892}|(be}
\toendnotes[C]{\smallbreak\pagebreak[2]}
\correspDesc{Versand  durch Arthur Schnitzler am 25. 7. 1892 in Wien
\newline{}Erhalt  durch Theodor Herzl in Wien}\toendnotes[C]{\smallbreak}
\Standort{Jerusalem, Central Zionist Archives, H1:1924-1.}
\physDesc{Brief, 1 Blatt, 3 Seiten, 721 Zeichen
\newline{}Handschrift: schwarze Tinte, deutsche Kurrent
\newline{}Ordnung: mit Bleistift von unbekannter Hand nummeriert: »1« und innerhalb das Konvoluts paginiert: »1«–»3« }
\buchAbdrucke{\weitereDrucke{Arthur Schnitzler: \emph{Briefe 1875–1912}. Herausgegeben von Therese Nickl und Heinrich Schnitzler. Frankfurt am Main: \emph{S. Fischer} 1981, S. 123.} }\toendnotes[C]{\smallbreak}
\pstart
           \raggedleft{}{\pb}25. 7. 92.\pend
           
\pstart
           \raggedleft{}Wien\oindex{Wien@\textbf{Wien}, \emph{Verwaltungsgebiet}|pw}.\pend
           
\pstart{}Verehrteſter Herr Doktor,\pend\vspace{0.5em}
\pstart
           es hat mich ganz beſonders gefreut, mit dem Mährchen\pwindex{Schnitzler, Arthur 15.\,5.\,1862 Wien – 21.\,10.\,1931 ebd.@\textsc{Schnitzler, Arthur} (15.\,5.\,1862 Wien – 21.\,10.\,1931 ebd.), \emph{Schriftsteller, Mediziner}!Märchen. Schauspiel in drei Aufzügen@\strich\emph{Das Märchen. Schauspiel in drei Aufzügen}|pw} – wie mir \label{K_L03900-1v}\edtext{Goldma{\geminationn}\pwindex{Goldmann, Paul 31.\,1.\,1865 Breslau – 25.\,9.\,1935 Wien@\textsc{Goldmann, Paul} (31.\,1.\,1865 Breslau – 25.\,9.\,1935 Wien), \emph{Schriftsteller, Journalist}|pw}
                  mittheilte}{\lemma{\textnormal{\emph{Goldmann
                  mittheilte}}}\Cendnote{\textnormal{XXXX Auszeichnungsfehler: Dokument L02698 nicht gefunden. }}}\label{K_L03900-1} – vor Ihrem
               kunſtverſtändigen Urtheile Beifall gefunden zu haben. Was nun meine andre Sachen
               betrifft, die Sie zu leſen wünſchen,{ }ſo thut es mir recht leid, Ihrem liebens{\pb}würdigen Intereſſe nur zum geringen Theil entſprechen zu
               können. Das meiſte iſt in Zeitungen minderer Verbreitung erſchienen; und nur weniges\pwindex{Schnitzler, Arthur 15.\,5.\,1862 Wien – 21.\,10.\,1931 ebd.@\textsc{Schnitzler, Arthur} (15.\,5.\,1862 Wien – 21.\,10.\,1931 ebd.), \emph{Schriftsteller, Mediziner}!Episode@\strich\emph{Episode}|pwv}\pwindex{Schnitzler, Arthur 15.\,5.\,1862 Wien – 21.\,10.\,1931 ebd.@\textsc{Schnitzler, Arthur} (15.\,5.\,1862 Wien – 21.\,10.\,1931 ebd.), \emph{Schriftsteller, Mediziner}!Reichtum. Erzählung@\strich\emph{Reichtum. Erzählung}|pwv}{ }ſteht mir in
               Separatabdrucken zur Verfügung. Dieſes wenige\pwindex{Schnitzler, Arthur 15.\,5.\,1862 Wien – 21.\,10.\,1931 ebd.@\textsc{Schnitzler, Arthur} (15.\,5.\,1862 Wien – 21.\,10.\,1931 ebd.), \emph{Schriftsteller, Mediziner}!Episode@\strich\emph{Episode}|pwv}\pwindex{Schnitzler, Arthur 15.\,5.\,1862 Wien – 21.\,10.\,1931 ebd.@\textsc{Schnitzler, Arthur} (15.\,5.\,1862 Wien – 21.\,10.\,1931 ebd.), \emph{Schriftsteller, Mediziner}!Reichtum. Erzählung@\strich\emph{Reichtum. Erzählung}|pwv}{ }ſchicke ich Ihnen hiemit; will Ihnen aber nicht
               verſchweigen, daſs im Herbſt {\pb}eine ganze Sa{\geminationm}lung von \textsc{Anatol}ſtückeln\pwindex{Schnitzler, Arthur 15.\,5.\,1862 Wien – 21.\,10.\,1931 ebd.@\textsc{Schnitzler, Arthur} (15.\,5.\,1862 Wien – 21.\,10.\,1931 ebd.), \emph{Schriftsteller, Mediziner}!Anatol@\strich\emph{Anatol}|pw}\footnote{\noindent{}\textsc{Episode\pwindex{Schnitzler, Arthur 15.\,5.\,1862 Wien – 21.\,10.\,1931 ebd.@\textsc{Schnitzler, Arthur} (15.\,5.\,1862 Wien – 21.\,10.\,1931 ebd.), \emph{Schriftsteller, Mediziner}!Episode@\strich\emph{Episode}|pw}}} – u hoffentlich bald drauf noch was andres\pwindex{Schnitzler, Arthur 15.\,5.\,1862 Wien – 21.\,10.\,1931 ebd.@\textsc{Schnitzler, Arthur} (15.\,5.\,1862 Wien – 21.\,10.\,1931 ebd.), \emph{Schriftsteller, Mediziner}!Sterben. Novelle@\strich\emph{Sterben. Novelle}|pwuv} von mir erſcheinen wird. –\pend
           
\pstart
           In aufrichtiger Verehrung{\\[\baselineskip]}Ihr{ }ſehr ergebner{\\[\baselineskip]}\spacefill\mbox{D\textsuperscript{r}ArthSchnitzler}\pend
           \leftskip=0em{}\selectlanguage{ngerman}\endnumbering\briefempfaengerindex{Herzl, Theodor@\textsc{Herzl, Theodor}!zzzSchnitzler, Arthur@\emph{von Arthur Schnitzler}!1892-07-251@{25. 7. 1892}|)be}\mylabel{L03900h}
\begin{anhang}
\end{anhang}\newcommand{\dateiname}{L03900}\newcommand{\titel}{Arthur Schnitzler an Theodor Herzl, 25. 7. 1892}\newcommand{\editorInnen}{Herausgegeben von Jahnke, SelmaMüller, Martin Anton}%% latex-leseansicht-abspann.tex
%% Abspann für die Leseansicht.
%% Der Schalter \ifkorrekturansicht ist bereits durch den Vorspann gesetzt.

%% latex-abspann.tex
%% Gemeinsamer Abspann für Korrekturansicht und Leseansicht.
%% Setzt den Schalter \ifkorrekturansicht voraus (gesetzt in den
%% einbindenden Dateien latex-korrekturansicht-abspann.tex bzw.
%% latex-leseansicht-abspann.tex).
%% ---------------------------------------------------------------

\normalsize

% Das esempio-Environment wird nur in der Leseansicht benötigt
\ifkorrekturansicht\else
\newenvironment{esempio}[3]%
{
    \vspace{1.5ex}
    \rlap{\underline{#1}}
    \par
    \setlength{\parindent}{0cm}
    \nopagebreak
    \leftskip=#2cm
    \rightskip=#3cm
}
{
    \par
}
\fi

\doendnotes{C}
\bigskip
\vfill

\clearpage

\footnotesize

\ifkorrekturansicht
  \lohead{\textsc{register}}
\fi

% theindex-Environment neu definieren ohne reledmac
\makeatletter
\renewenvironment{theindex}{%
  \ifkorrekturansicht
    \section*{\indexname}%
  \else
    \subsubsection*{Index der erwähnten Entitäten}%
  \fi
  \setlength{\parindent}{0pt}%
  \setlength{\parskip}{0pt plus 0.3pt}%
  \let\item\@idxitem
}{%
  \ifkorrekturansicht\clearpage\fi
}
\makeatother

\IfFileExists{\jobname-pw.ind}{\input{\jobname-pw.ind}}{}

% Quellenangabe nur in der Leseansicht
\ifkorrekturansicht\else
% Fallback-Definitionen, falls die .tex-Datei \titel etc. nicht gesetzt hat
\providecommand{\titel}{}
\providecommand{\editorInnen}{}
\providecommand{\dateiname}{\jobname}

\vspace{3cm}

\vfill

\footnotesize
\textsc{Quelle}: \titel. Herausgegeben von {\editorInnen}. In: \emph{Arthur Schnitzler: Briefwechsel mit Autorinnen und Autoren}.
 Digitale Edition, https://schnitzler-briefe.acdh.oeaw.ac.at/{\dateiname}.html (Stand \today)
\fi

\end{document}


