%% latex-korrekturansicht-vorspann.tex
%% Vorspann für die Korrekturansicht.
%% Lädt die gemeinsame Datei latex-vorspann.tex mit gesetztem Schalter.

\newif\ifkorrekturansicht
\korrekturansichttrue

\input{../tex-inputs/latex-vorspann}


\section[Max Burckhard an Arthur Schnitzler, {[}1905?{]}]{L01489 Max Burckhard an Arthur Schnitzler, {[}1905?{]}}
\nopagebreak\mylabel{L01489v}
\rehead{ }\normalsize\beginnumbering\briefempfaengerindex{Schnitzler, Arthur@\textsc{Schnitzler, Arthur}!zzzBurckhard, Max Eugen@\emph{von Max Eugen Burckhard}!1905-12-311@{{[}1905?{]}}|(be}
\toendnotes[C]{\smallbreak\pagebreak[2]}\Standort{CUL, Schnitzler, B 20.}
\physDesc{Telegramm, 36 Zeichen
\newline{}Handschrift einer Schreibkraft: blaue Tinte, deutsche Kurrent
\newline{}Versand: am oberen Rand noch zu erkennen: »\textcolor{gray}{\textbf{Aufgabe-Nr.}}
                                    1\textcolor{gray}{6}0« 
\newline{}Ordnung: beschnitten }\toendnotes[C]{\smallbreak}
\pstart
           \noindent{}{\pb}\label{K_L01489-1v}\edtext{Herzlichen Dank.}{\lemma{\textnormal{\emph{Herzlichen Dank.}}}\Cendnote{\textnormal{Die Ordnung der Mappe gibt keinen
                  Aufschluss über die chronologische Reihung. In der Abschrift ist dieses Telegramm
                  nach dem 24. 12. 1904 und vor dem 30. 11. 1905
                  eingereiht, sodass es hier 1905 zugeordnet wird – auch wenn es dafür
                  keinen Anhaltspunkt gibt. Ein späterer Zeitpunkt dürfte allerdings ausgeschlossen
                  sein, weil die Post ab 1906 andere Telegrammformulare genutzt
                  hat.}}}\label{K_L01489-1} Mit Freuden\pend
           \pstart \spacefill\mbox{Burckhard}\pend{}\selectlanguage{ngerman}\endnumbering\briefempfaengerindex{Schnitzler, Arthur@\textsc{Schnitzler, Arthur}!zzzBurckhard, Max Eugen@\emph{von Max Eugen Burckhard}!1905-01-011@{{[}1905?{]}}|)be}\mylabel{L01489h}  \normalsize

\doendnotes{C}
\bigskip
\vfill

\clearpage

\footnotesize

\lohead{\textsc{register}}

% Definiere theindex-Environment komplett neu ohne reledmac
\makeatletter
\renewenvironment{theindex}{%
  \section*{\indexname}%
  \setlength{\parindent}{0pt}%
  \setlength{\parskip}{0pt plus 0.3pt}%
  \let\item\@idxitem
}{%
  \clearpage
}
\makeatother

\IfFileExists{\jobname-pw.ind}{\input{\jobname-pw.ind}}{}

\end{document}

      