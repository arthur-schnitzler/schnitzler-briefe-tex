%% latex-korrekturansicht-vorspann.tex
%% Vorspann für die Korrekturansicht.
%% Lädt die gemeinsame Datei latex-vorspann.tex mit gesetztem Schalter.

\newif\ifkorrekturansicht
\korrekturansichttrue

\input{../tex-inputs/latex-vorspann}


\section[Paul Goldmann an Arthur Schnitzler, 21. 10. 1889]{L02644 Paul Goldmann an Arthur Schnitzler, 21. 10. 1889}
\nopagebreak\mylabel{L02644v}
\rehead{ }\normalsize\beginnumbering\briefempfaengerindex{Schnitzler, Arthur@\textsc{Schnitzler, Arthur}!zzzGoldmann, Paul@\emph{von Paul Goldmann}!1889-10-211@{21. 10. 1889}|(be}
\toendnotes[C]{\smallbreak\pagebreak[2]}\Standort{DLA, A:Schnitzler, HS.NZ85.1.3162.}
\physDesc{Brief, 1 Blatt, 2 Seiten, 802 Zeichen
\newline{}Handschrift: blaue Tinte, deutsche Kurrent
\newline{}Schnitzler: mit rotem Buntstift eine Unterstreichung }\toendnotes[C]{\smallbreak}
\pstart
           \centering{}{\pb}\textcolor{gray}{\textbf{\textbf{Adminiſtration: VII.
                           Seidengaſſe 7\oindex{Seidengasse@\textbf{Seidengasse}, \emph{Straße (K.STR)}|pw}} (Jos. Eberle {\kaufmannsund} Co.\orgindex{Josef Eberle Stein-, Buch und Musikaliendruckerei@Josef Eberle Stein-, Buch und Musikaliendruckerei|pw})}}\pend
           
\pstart
           \centering{}\textcolor{gray}{\textbf{An der Schönen Blauen Donau\orgindex{der schoenen blauen Donau@An der schönen blauen Donau|pw}}}\pend
           
\pstart
           \centering{}\textcolor{gray}{\textbf{Chef-Redacteur: Dr. F.
                        Mamroth\pwindex{Mamroth, Fedor 21.02.1851 – 25.06.1907@\textsc{Mamroth, Fedor} (21.02.1851 – 25.06.1907), \emph{Journalist/Journalistin, Kritiker/Kritikerin}|pw}. – Redaction: IX.,
                        Berggaſſe 31\oindex{Berggasse@\textbf{Berggasse}, \emph{Straße (K.STR)}|pw}.}}\pend
           
\pstart
           \raggedleft{}\textcolor{gray}{\textbf{Wien\oindex{Wien@\textbf{Wien}, \emph{A.ADM2}|pw}, den}}{ }21. October \textcolor{gray}{\textbf{18}}89.\pend
           
\pstart\center{}Lieber Herr Doctor!\pend\vspace{0.5em}
\pstart
           Ich habe den \label{K_L02644-1v}\edtext{Beitrag\pwindex{?? [Abgelehnte Erzaehlung fuer An der schoenen blauen Donau]@\emph{?? [Abgelehnte Erzählung für An der schönen blauen Donau]}|pwv}}{\lemma{\textnormal{\emph{Beitrag}}}\Cendnote{\textnormal{nicht ermittelt}}}\label{K_L02644-1} Ihres unbekannten
                  \label{K_L02644-2v}\edtext{Freund\pwindex{?? [Verfasser einer abgelehnten Erzaehlung] [1889] @\textsc{?? [Verfasser einer abgelehnten Erzählung] [1889]}|pwv}es}{\lemma{\textnormal{\emph{Freundes}}}\Cendnote{\textnormal{nicht identifiziert}}}\label{K_L02644-2} mit lebhaftem
               Intereſſe geleſen. Es ſteckt viel Talent in der kleinen Arbeit – ſie\strikeout{\textcolor{gray}{×}} iſt warm und poetiſch empfunden und nicht ohne
               Gewand{[}t{]}heit dargeſtellt. Ich hätte ſie gern in unſerem Allerfeelen-Heft\pwindex{der schoenen blauen Donau@\emph{An der schönen blauen Donau}|pwv}
               veröffentlicht. Aber leider füllt die Erzählung\pwindex{?? [Abgelehnte Erzaehlung fuer An der schoenen blauen Donau]@\emph{?? [Abgelehnte Erzählung für An der schönen blauen Donau]}|pwv} nicht den vierten Theil des räumlichen Ausmaßes
               aus, das – nach den techniſchen Principien unferes {\pb}Blattes\pwindex{der schoenen blauen Donau@\emph{An der schönen blauen Donau}|pwv} – ein Feuilleton
               aufweiſen muß. Mit einem Worte: Die hübſche Arbeit\pwindex{?? [Abgelehnte Erzaehlung fuer An der schoenen blauen Donau]@\emph{?? [Abgelehnte Erzählung für An der schönen blauen Donau]}|pwv} iſt zu klein für uns. Vielleicht wächſt ſie ſich bis
               zum nächſten Allerſeelen ein wenig aus. Inzwiſchen
               aber wäre ich Ihnen dankbar, wenn Sie mir bei Gelegenheit eine andere Arbeit von
               Ihrem Schützling\pwindex{?? [Verfasser einer abgelehnten Erzaehlung] [1889] @\textsc{?? [Verfasser einer abgelehnten Erzählung] [1889]}|pwv} verſchaffen
               wollten. Der junge Mann\pwindex{?? [Verfasser einer abgelehnten Erzaehlung] [1889] @\textsc{?? [Verfasser einer abgelehnten Erzählung] [1889]}|pwv}
               intereſſirt mich{\dots}\pend
           
\pstart
           Ich begrüße Sie herzlichſt! {\\[\baselineskip]}Ihr {\\[\baselineskip]}ergebener {\\[\baselineskip]}\spacefill\mbox{Dr. Paul Goldmann.}\pend
           \leftskip=0em{}\selectlanguage{ngerman}\endnumbering\briefempfaengerindex{Schnitzler, Arthur@\textsc{Schnitzler, Arthur}!zzzGoldmann, Paul@\emph{von Paul Goldmann}!1889-10-211@{21. 10. 1889}|)be}\mylabel{L02644h}  \normalsize

\doendnotes{C}
\bigskip
\vfill

\clearpage

\footnotesize

\lohead{\textsc{register}}

% Definiere theindex-Environment komplett neu ohne reledmac
\makeatletter
\renewenvironment{theindex}{%
  \section*{\indexname}%
  \setlength{\parindent}{0pt}%
  \setlength{\parskip}{0pt plus 0.3pt}%
  \let\item\@idxitem
}{%
  \clearpage
}
\makeatother

\IfFileExists{\jobname-pw.ind}{\input{\jobname-pw.ind}}{}

\end{document}

      