%% latex-leseansicht-vorspann.tex
%% Vorspann für die Leseansicht.
%% Lädt die gemeinsame Datei latex-vorspann.tex mit nicht gesetztem Schalter.

\newif\ifkorrekturansicht
\korrekturansichtfalse

\input{../tex-inputs/latex-vorspann}


         
         \renewcommand{\erwaehntePersonen}{Personen:  ?? [Verfasser einer abgelehnten Erzählung] [1889], Fedor Mamroth}
         \renewcommand{\erwaehnteInstitutionen}{Institutionen: An der schönen blauen Donau, Josef Eberle Stein-, Buch und Musikaliendruckerei}
         \renewcommand{\erwaehnteOrte}{Orte: Berggasse, Seidengasse, Wien}
         \renewcommand{\erwaehnteWerke}{Werke: ?? [Abgelehnte Erzählung für An der schönen blauen Donau], An der schönen blauen Donau}
               \section[Paul Goldmann an Arthur Schnitzler, 21. 10. 1889]{ Paul Goldmann an Arthur Schnitzler, 21. 10. 1889}\nopagebreak\mylabel{v}\rehead{ }\begin{ledgroupsized}[t]{13cm}\normalsize\beginnumbering \toendnotes[C]{\smallbreak\pagebreak[2]} \Standort{DLA, A:Schnitzler, HS.NZ85.1.3162.}
\physDesc{Brief, 1 Blatt, 2 Seiten, 802 Zeichen
\newline{}Handschrift: blaue Tinte, deutsche Kurrent
\newline{}Schnitzler: mit rotem Buntstift eine Unterstreichung }\toendnotes[C]{\smallbreak}\pstart
           \noindent{}\centering{}{\pb}\textcolor{gray}{\textbf{\textbf{Adminiſtration: VII.
                           Seidengaſſe 7\oindex{Seidengasse@\textbf{Seidengasse}|pw}} (Jos. Eberle {\kaufmannsund} Co.\orgindex{Josef Eberle Stein-, Buch und Musikaliendruckerei@Josef Eberle Stein-, Buch und Musikaliendruckerei|pw})}}\pend
           \pstart
           \noindent{}\centering{}\textcolor{gray}{\textbf{An der Schönen Blauen Donau\orgindex{der schoenen blauen Donau@An der schönen blauen Donau|pw}}}\pend
           \pstart
           \noindent{}\centering{}\textcolor{gray}{\textbf{Chef-Redacteur: Dr. F.
                        Mamroth\pwindex{Mamroth, Fedor 21.02.1851 – 25.06.1907@\textsc{Mamroth, Fedor} (21.02.1851 – 25.06.1907), \emph{Journalist, Kritiker}|pw}. – Redaction: IX.,
                        Berggaſſe 31\oindex{Berggasse@\textbf{Berggasse}|pw}.}}\pend
           \pstart
           \raggedleft{}\textcolor{gray}{\textbf{Wien\oindex{Wien@\textbf{Wien}|pw}, den}}{ }21. October \textcolor{gray}{\textbf{18}}89.\pend
           \pstart\center{}Lieber Herr Doctor!\pend\pstart
           Ich habe den \label{K_L02644-1v}\edtext{Beitrag\pwindex{?? [Verfasser einer abgelehnten Erzaehlung] [1889] @\textsc{?? [Verfasser einer abgelehnten Erzählung] [1889]}!?? [Abgelehnte Erzaehlung fuer An der schoenen blauen Donau]1889@\strich\emph{?? [Abgelehnte Erzählung für An der schönen blauen Donau]} {[}1889{]}|pwv}}{\lemma{\textnormal{\emph{Beitrag}}}\Cendnote{\textnormal{nicht ermittelt}}}\label{K_L02644-1h} Ihres unbekannten
                  \label{K_L02644-2v}\edtext{Freund\pwindex{?? [Verfasser einer abgelehnten Erzaehlung] [1889] @\textsc{?? [Verfasser einer abgelehnten Erzählung] [1889]}|pwv}es }{\lemma{\textnormal{\emph{Freundes }}}\Cendnote{\textnormal{nicht identifiziert}}}\label{K_L02644-2h} mit lebhaftem
               Intereſſe geleſen. Es ſteckt viel Talent in der kleinen Arbeit – ſie\strikeout{\textcolor{gray}{×}} iſt warm und poetiſch empfunden und nicht ohne
               Gewand{[}t{]}heit dargeſtellt. Ich hätte ſie gern in unſerem Allerfeelen-Heft\pwindex{?? Werk@Nicht ermittelte Verfasserinnen und Verfasser!der schoenen blauen Donau1886 – 1896@\emph{An der schönen blauen Donau} {[}1886 – 1896{]}|pwv}
               veröffentlicht. Aber leider füllt die Erzählung\pwindex{?? [Verfasser einer abgelehnten Erzaehlung] [1889] @\textsc{?? [Verfasser einer abgelehnten Erzählung] [1889]}!?? [Abgelehnte Erzaehlung fuer An der schoenen blauen Donau]1889@\strich\emph{?? [Abgelehnte Erzählung für An der schönen blauen Donau]} {[}1889{]}|pwv} nicht den vierten Theil des räumlichen Ausmaßes
               aus, das – nach den techniſchen Principien unferes {\pb}Blattes\pwindex{?? Werk@Nicht ermittelte Verfasserinnen und Verfasser!der schoenen blauen Donau1886 – 1896@\emph{An der schönen blauen Donau} {[}1886 – 1896{]}|pwv} – ein Feuilleton
               aufweiſen muß. Mit einem Worte: Die hübſche Arbeit\pwindex{?? [Verfasser einer abgelehnten Erzaehlung] [1889] @\textsc{?? [Verfasser einer abgelehnten Erzählung] [1889]}!?? [Abgelehnte Erzaehlung fuer An der schoenen blauen Donau]1889@\strich\emph{?? [Abgelehnte Erzählung für An der schönen blauen Donau]} {[}1889{]}|pwv} iſt zu klein für uns. Vielleicht wächſt ſie ſich bis
               zum nächſten Allerſeelen ein wenig aus. Inzwiſchen
               aber wäre ich Ihnen dankbar, wenn Sie mir bei Gelegenheit eine andere Arbeit von
               Ihrem Schützling\pwindex{?? [Verfasser einer abgelehnten Erzaehlung] [1889] @\textsc{?? [Verfasser einer abgelehnten Erzählung] [1889]}|pwv} verſchaffen
               wollten. Der junge Mann\pwindex{?? [Verfasser einer abgelehnten Erzaehlung] [1889] @\textsc{?? [Verfasser einer abgelehnten Erzählung] [1889]}|pwv}
               intereſſirt mich{\dots}\pend
           \pstart
           Ich begrüße Sie herzlichſt! {\\[\baselineskip]}Ihr {\\[\baselineskip]}ergebener {\\[\baselineskip]}\spacefill\mbox{Dr. Paul Goldmann.}\pend
           \leftskip=0em{}
         
         \endnumbering\mylabel{h}\end{ledgroupsized}  \newcommand{\dateiname}{L02644}\newcommand{\titel}{Paul Goldmann an Arthur Schnitzler, 21. 10. 1889}\newcommand{\editorInnen}{Martin Anton Müller und Laura Untner}%% latex-leseansicht-abspann.tex
%% Abspann für die Leseansicht.
%% Der Schalter \ifkorrekturansicht ist bereits durch den Vorspann gesetzt.

%% latex-abspann.tex
%% Gemeinsamer Abspann für Korrekturansicht und Leseansicht.
%% Setzt den Schalter \ifkorrekturansicht voraus (gesetzt in den
%% einbindenden Dateien latex-korrekturansicht-abspann.tex bzw.
%% latex-leseansicht-abspann.tex).
%% ---------------------------------------------------------------

\normalsize

% Das esempio-Environment wird nur in der Leseansicht benötigt
\ifkorrekturansicht\else
\newenvironment{esempio}[3]%
{
    \vspace{1.5ex}
    \rlap{\underline{#1}}
    \par
    \setlength{\parindent}{0cm}
    \nopagebreak
    \leftskip=#2cm
    \rightskip=#3cm
}
{
    \par
}
\fi

\doendnotes{C}
\bigskip
\vfill

\clearpage

\footnotesize

\ifkorrekturansicht
  \lohead{\textsc{register}}
\fi

% theindex-Environment neu definieren ohne reledmac
\makeatletter
\renewenvironment{theindex}{%
  \ifkorrekturansicht
    \section*{\indexname}%
  \else
    \subsubsection*{Index der erwähnten Entitäten}%
  \fi
  \setlength{\parindent}{0pt}%
  \setlength{\parskip}{0pt plus 0.3pt}%
  \let\item\@idxitem
}{%
  \ifkorrekturansicht\clearpage\fi
}
\makeatother

\IfFileExists{\jobname-pw.ind}{\input{\jobname-pw.ind}}{}

% Quellenangabe nur in der Leseansicht
\ifkorrekturansicht\else
% Fallback-Definitionen, falls die .tex-Datei \titel etc. nicht gesetzt hat
\providecommand{\titel}{}
\providecommand{\editorInnen}{}
\providecommand{\dateiname}{\jobname}

\vspace{3cm}

\vfill

\footnotesize
\textsc{Quelle}: \titel. Herausgegeben von {\editorInnen}. In: \emph{Arthur Schnitzler: Briefwechsel mit Autorinnen und Autoren}.
 Digitale Edition, https://schnitzler-briefe.acdh.oeaw.ac.at/{\dateiname}.html (Stand \today)
\fi

\end{document}


      