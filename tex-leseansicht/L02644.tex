%% latex-leseansicht-vorspann.tex
%% Vorspann für die Leseansicht.
%% Lädt die gemeinsame Datei latex-vorspann.tex mit nicht gesetztem Schalter.

\newif\ifkorrekturansicht
\korrekturansichtfalse

\input{../tex-inputs/latex-vorspann}


\section[Paul Goldmann an Arthur Schnitzler, 21. 10. 1889]{L02644 Paul Goldmann an Arthur Schnitzler, 21. 10. 1889}
\nopagebreak\mylabel{L02644v}
\rehead{ }\normalsize\beginnumbering\briefempfaengerindex{Schnitzler, Arthur@\textsc{Schnitzler, Arthur}!zzzGoldmann, Paul@\emph{von Paul Goldmann}!1889-10-211@{21. 10. 1889}|(be}
\toendnotes[C]{\smallbreak\pagebreak[2]}
\correspDesc{Versand  durch Paul Goldmann am 21. 10. 1889 in Wien
\newline{}Erhalt  durch Arthur Schnitzler im Zeitraum [21. 10. 1889 – 25. 10. 1889?] in Wien}\toendnotes[C]{\smallbreak}
\Standort{DLA, A:Schnitzler, HS.NZ85.1.3162.}
\physDesc{Brief, 1 Blatt, 2 Seiten, 802 Zeichen
\newline{}Handschrift: blaue Tinte, deutsche Kurrent
\newline{}Schnitzler: mit rotem Buntstift eine Unterstreichung }\toendnotes[C]{\smallbreak}
\pstart
           \centering{}{\pb}\textcolor{gray}{\textbf{\textbf{Adminiſtration: VII.
                           Seidengaſſe 7\oindex{Wien@\textbf{Wien}!VII., Neubau@\textbf{VII., Neubau}!Seidengasse@\textbf{Seidengasse}, \emph{Straße}|pw}} (Jos. Eberle {\kaufmannsund} Co.\orgindex{Josef Eberle Stein-, Buch und Musikaliendruckerei@Josef Eberle Stein-, Buch und Musikaliendruckerei|pw})}}\pend
           
\pstart
           \centering{}\textcolor{gray}{\textbf{An der Schönen Blauen Donau\orgindex{der schönen blauen Donau@An der schönen blauen Donau|pw}}}\pend
           
\pstart
           \centering{}\textcolor{gray}{\textbf{Chef-Redacteur: Dr. F.
                        Mamroth\pwindex{Mamroth, Fedor 21.\,2.\,1851 Breslau – 25.\,6.\,1907 Frankfurt am Main@\textsc{Mamroth, Fedor} (21.\,2.\,1851 Breslau – 25.\,6.\,1907 Frankfurt am Main), \emph{Journalist, Kritiker}|pw}. – Redaction: IX.,
                        Berggaſſe 31\oindex{Wien@\textbf{Wien}!IX., Alsergrund@\textbf{IX., Alsergrund}!Berggasse@\textbf{Berggasse}, \emph{Straße}|pw}.}}\pend
           
\pstart
           \raggedleft{}\textcolor{gray}{\textbf{Wien\oindex{Wien@\textbf{Wien}, \emph{Verwaltungsgebiet}|pw}, den}}{ }21. October \textcolor{gray}{\textbf{18}}89.\pend
           
\pstart\center{}Lieber Herr Doctor!\pend\vspace{0.5em}
\pstart
           Ich habe den \label{K_L02644-1v}\edtext{Beitrag\pwindex{?? [Verfasser einer abgelehnten Erzählung] [1889] @\textsc{?? [Verfasser einer abgelehnten Erzählung] [1889]}!?? [Abgelehnte Erzählung für An der schönen blauen Donau]@\strich\emph{?? [Abgelehnte Erzählung für An der schönen blauen Donau]}|pwv}}{\lemma{\textnormal{\emph{Beitrag}}}\Cendnote{\textnormal{nicht ermittelt}}}\label{K_L02644-1} Ihres unbekannten
                  \label{K_L02644-2v}\edtext{Freund\pwindex{?? [Verfasser einer abgelehnten Erzählung] [1889] @\textsc{?? [Verfasser einer abgelehnten Erzählung] [1889]}|pwv}es}{\lemma{\textnormal{\emph{Freundes}}}\Cendnote{\textnormal{nicht identifiziert}}}\label{K_L02644-2} mit lebhaftem
               Intereſſe geleſen. Es{ }ſteckt viel Talent in der kleinen Arbeit –{ }ſie\strikeout{\textcolor{gray}{×}} iſt warm und poetiſch empfunden und nicht ohne
               Gewand{[}t{]}heit dargeſtellt. Ich hätte{ }ſie gern in unſerem Allerfeelen-Heft\pwindex{der schönen blauen Donau@\emph{An der schönen blauen Donau}|pwv}
               veröffentlicht. Aber leider füllt die Erzählung\pwindex{?? [Verfasser einer abgelehnten Erzählung] [1889] @\textsc{?? [Verfasser einer abgelehnten Erzählung] [1889]}!?? [Abgelehnte Erzählung für An der schönen blauen Donau]@\strich\emph{?? [Abgelehnte Erzählung für An der schönen blauen Donau]}|pwv} nicht den vierten Theil des räumlichen Ausmaßes
               aus, das – nach den techniſchen Principien unferes {\pb}Blattes\pwindex{der schönen blauen Donau@\emph{An der schönen blauen Donau}|pwv} – ein Feuilleton
               aufweiſen muß. Mit einem Worte: Die hübſche Arbeit\pwindex{?? [Verfasser einer abgelehnten Erzählung] [1889] @\textsc{?? [Verfasser einer abgelehnten Erzählung] [1889]}!?? [Abgelehnte Erzählung für An der schönen blauen Donau]@\strich\emph{?? [Abgelehnte Erzählung für An der schönen blauen Donau]}|pwv} iſt zu klein für uns. Vielleicht wächſt{ }ſie{ }ſich bis
               zum nächſten Allerſeelen ein wenig aus. Inzwiſchen
               aber wäre ich Ihnen dankbar, wenn Sie mir bei Gelegenheit eine andere Arbeit von
               Ihrem Schützling\pwindex{?? [Verfasser einer abgelehnten Erzählung] [1889] @\textsc{?? [Verfasser einer abgelehnten Erzählung] [1889]}|pwv} verſchaffen
               wollten. Der junge Mann\pwindex{?? [Verfasser einer abgelehnten Erzählung] [1889] @\textsc{?? [Verfasser einer abgelehnten Erzählung] [1889]}|pwv}
               intereſſirt mich{\dots}\pend
           
\pstart
           Ich begrüße Sie herzlichſt! {\\[\baselineskip]}Ihr {\\[\baselineskip]}ergebener {\\[\baselineskip]}\spacefill\mbox{Dr. Paul Goldmann.}\pend
           \leftskip=0em{}\selectlanguage{ngerman}\endnumbering\briefempfaengerindex{Schnitzler, Arthur@\textsc{Schnitzler, Arthur}!zzzGoldmann, Paul@\emph{von Paul Goldmann}!1889-10-211@{21. 10. 1889}|)be}\mylabel{L02644h}  \newcommand{\dateiname}{L02644}\newcommand{\titel}{Paul Goldmann an Arthur Schnitzler, 21. 10. 1889}\newcommand{\editorInnen}{Martin Anton Müller und Laura Untner}%% latex-leseansicht-abspann.tex
%% Abspann für die Leseansicht.
%% Der Schalter \ifkorrekturansicht ist bereits durch den Vorspann gesetzt.

%% latex-abspann.tex
%% Gemeinsamer Abspann für Korrekturansicht und Leseansicht.
%% Setzt den Schalter \ifkorrekturansicht voraus (gesetzt in den
%% einbindenden Dateien latex-korrekturansicht-abspann.tex bzw.
%% latex-leseansicht-abspann.tex).
%% ---------------------------------------------------------------

\normalsize

% Das esempio-Environment wird nur in der Leseansicht benötigt
\ifkorrekturansicht\else
\newenvironment{esempio}[3]%
{
    \vspace{1.5ex}
    \rlap{\underline{#1}}
    \par
    \setlength{\parindent}{0cm}
    \nopagebreak
    \leftskip=#2cm
    \rightskip=#3cm
}
{
    \par
}
\fi

\doendnotes{C}
\bigskip
\vfill

\clearpage

\footnotesize

\ifkorrekturansicht
  \lohead{\textsc{register}}
\fi

% theindex-Environment neu definieren ohne reledmac
\makeatletter
\renewenvironment{theindex}{%
  \ifkorrekturansicht
    \section*{\indexname}%
  \else
    \subsubsection*{Index der erwähnten Entitäten}%
  \fi
  \setlength{\parindent}{0pt}%
  \setlength{\parskip}{0pt plus 0.3pt}%
  \let\item\@idxitem
}{%
  \ifkorrekturansicht\clearpage\fi
}
\makeatother

\IfFileExists{\jobname-pw.ind}{\input{\jobname-pw.ind}}{}

% Quellenangabe nur in der Leseansicht
\ifkorrekturansicht\else
% Fallback-Definitionen, falls die .tex-Datei \titel etc. nicht gesetzt hat
\providecommand{\titel}{}
\providecommand{\editorInnen}{}
\providecommand{\dateiname}{\jobname}

\vspace{3cm}

\vfill

\footnotesize
\textsc{Quelle}: \titel. Herausgegeben von {\editorInnen}. In: \emph{Arthur Schnitzler: Briefwechsel mit Autorinnen und Autoren}.
 Digitale Edition, https://schnitzler-briefe.acdh.oeaw.ac.at/{\dateiname}.html (Stand \today)
\fi

\end{document}


