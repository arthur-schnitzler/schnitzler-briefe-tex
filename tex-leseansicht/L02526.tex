%% latex-leseansicht-vorspann.tex
%% Vorspann für die Leseansicht.
%% Lädt die gemeinsame Datei latex-vorspann.tex mit nicht gesetztem Schalter.

\newif\ifkorrekturansicht
\korrekturansichtfalse

\input{../tex-inputs/latex-vorspann}


         
         \renewcommand{\erwaehntePersonen}{Personen: Gerhart Hauptmann, Margarete Hauptmann}
         \renewcommand{\erwaehnteOrte}{Orte: Wien}
         \renewcommand{\erwaehnteWerke}{Werke: Spuk}
               \section[Arthur Schnitzler an Gerhart Hauptmann, 27. 11. 1929]{ Arthur Schnitzler an Gerhart Hauptmann, 27. 11. 1929}\nopagebreak\mylabel{v}\rehead{ }\begin{ledgroupsized}[t]{13cm}\normalsize\beginnumbering\briefempfaengerindex{Hauptmann, Gerhart@\textsc{Hauptmann, Gerhart}!zzzSchnitzler, Arthur@\emph{von Arthur Schnitzler}!1929-11-271@{27. 11. 1929}|(be} \toendnotes[C]{\smallbreak\pagebreak[2]} \Standort{Staatsbibliothek Berlin – Preußischer Kulturbesitz, GH Br NL (ehem. AdK) B 1324.}
\physDesc{Brief, 1 Blatt, 2 Seiten, 920 Zeichen
\newline{}Handschrift: schwarze Tinte, lateinische Kurrent}\Standort{DLA, A:Schnitzler, HS.NZ85.1.5684.}
\physDesc{Brief, Fotokopie, 1 Blatt, 2 Seiten, 920 Zeichen
\newline{}Handschrift: schwarze Tinte, lateinische Kurrent}\buchAbdrucke{\weitereDrucke{Arthur Schnitzler: \emph{Briefe 1913–1931}. Hg. Peter Michael Braunwarth, Richard Miklin, Susanne Pertlik und Heinrich Schnitzler. Frankfurt am Main: \emph{S. Fischer} 1984, S. 637.} }\toendnotes[C]{\smallbreak}\pstart
           \raggedleft{}{\pb}Wien\oindex{Wien@\textbf{Wien}|pw}, 27. 11. 29.\pend
           \pstart{}Verehrter Herr Gerhart Hauptmann\pend\pstart
           Sie nehmen mir gewiſs nicht übel daſs ich an dem \label{K_L02526-1v}\edtext{Bankett}{\lemma{\textnormal{\emph{Bankett}}}\Cendnote{\textnormal{Dieses
                  fand am 28. 11. 1929 statt.}}}\label{K_L02526-1h} Ihnen zu Ehren nicht theilnehme,
               seit längerer Zeit halte ich mich (nicht aus Princip, sondern aus einer vorläufg
               nicht zu überwindenden Abneigung) von großen Gesellschaften, insbesondre aber von
               Feierlichkeiten fern, mag ich im Herzen auch so begeistert mitfeiern, wie ich es
               z. B. bei einem Hauptmann Bankett thue. Ich muß Ihnen ja nicht erst von meiner
               Bewunderung und Liebe sprechen, – Sie haben immer gewußt, was Sie {\pb}mir bedeuten. \pend
           \pstart
           In jedem Falle aber werde ich Sie während Ihres Wien\oindex{Wien@\textbf{Wien}|pw}er Aufenthaltes sehen, ich melde mich, sobald Sie nicht mehr
               allzugeplagt sind und bin jedenfalls bei Ihrer \label{K_L02526-2v}\edtext{Generalprobe\pwindex{Hauptmann, Gerhart 15.11.1862 – 06.06.1946@\textsc{Hauptmann, Gerhart} (15.11.1862 – 06.06.1946), \emph{Schriftsteller}!Spuk3. 12. 1929@\strich\emph{Spuk} {[}3. 12. 1929{]}|pwv}}{\lemma{\textnormal{\emph{Generalprobe}}}\Cendnote{\textnormal{Die Generalprobe von \emph{Spuk}\pwindex{Hauptmann, Gerhart 15.11.1862 – 06.06.1946@\textsc{Hauptmann, Gerhart} (15.11.1862 – 06.06.1946), \emph{Schriftsteller}!Spuk3. 12. 1929@\strich\emph{Spuk} {[}3. 12. 1929{]}|pwk} fand am 2. 12. 1929 statt, dem Tag vor der Uraufführung.}}}\label{K_L02526-2h}. Doch hoff
               ich Ihnen noch vorher persönlich zu begegnen.\pend
           \pstart
           Empfehlen Sie mich Ihrer sehr verehrten Gattin\pwindex{Hauptmann, Margarete 07.01.1875 – 17.01.1957@\textsc{Hauptmann, Margarete} (07.01.1875 – 17.01.1957)|pwv} mein lieber und verehrter Gerhard
               Hauptmann und seien Sie in herzlicher Ergebenheit gegrüßt.\pend
           \pstart
           Ihr{\\[\baselineskip]}\spacefill\mbox{Arthur Schnitzler}\pend
           \leftskip=0em{}
         
         \endnumbering\mylabel{h}\end{ledgroupsized}  \newcommand{\dateiname}{L02526}\newcommand{\titel}{Arthur Schnitzler an Gerhart Hauptmann, 27. 11. 1929}\newcommand{\editorInnen}{ Martin Anton Müller und Gerd-Hermann Susen}%% latex-leseansicht-abspann.tex
%% Abspann für die Leseansicht.
%% Der Schalter \ifkorrekturansicht ist bereits durch den Vorspann gesetzt.

%% latex-abspann.tex
%% Gemeinsamer Abspann für Korrekturansicht und Leseansicht.
%% Setzt den Schalter \ifkorrekturansicht voraus (gesetzt in den
%% einbindenden Dateien latex-korrekturansicht-abspann.tex bzw.
%% latex-leseansicht-abspann.tex).
%% ---------------------------------------------------------------

\normalsize

% Das esempio-Environment wird nur in der Leseansicht benötigt
\ifkorrekturansicht\else
\newenvironment{esempio}[3]%
{
    \vspace{1.5ex}
    \rlap{\underline{#1}}
    \par
    \setlength{\parindent}{0cm}
    \nopagebreak
    \leftskip=#2cm
    \rightskip=#3cm
}
{
    \par
}
\fi

\doendnotes{C}
\bigskip
\vfill

\clearpage

\footnotesize

\ifkorrekturansicht
  \lohead{\textsc{register}}
\fi

% theindex-Environment neu definieren ohne reledmac
\makeatletter
\renewenvironment{theindex}{%
  \ifkorrekturansicht
    \section*{\indexname}%
  \else
    \subsubsection*{Index der erwähnten Entitäten}%
  \fi
  \setlength{\parindent}{0pt}%
  \setlength{\parskip}{0pt plus 0.3pt}%
  \let\item\@idxitem
}{%
  \ifkorrekturansicht\clearpage\fi
}
\makeatother

\IfFileExists{\jobname-pw.ind}{\input{\jobname-pw.ind}}{}

% Quellenangabe nur in der Leseansicht
\ifkorrekturansicht\else
% Fallback-Definitionen, falls die .tex-Datei \titel etc. nicht gesetzt hat
\providecommand{\titel}{}
\providecommand{\editorInnen}{}
\providecommand{\dateiname}{\jobname}

\vspace{3cm}

\vfill

\footnotesize
\textsc{Quelle}: \titel. Herausgegeben von {\editorInnen}. In: \emph{Arthur Schnitzler: Briefwechsel mit Autorinnen und Autoren}.
 Digitale Edition, https://schnitzler-briefe.acdh.oeaw.ac.at/{\dateiname}.html (Stand \today)
\fi

\end{document}


      