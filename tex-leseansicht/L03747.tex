%% latex-korrekturansicht-vorspann.tex
%% Vorspann für die Korrekturansicht.
%% Lädt die gemeinsame Datei latex-vorspann.tex mit gesetztem Schalter.

\newif\ifkorrekturansicht
\korrekturansichttrue

\input{../tex-inputs/latex-vorspann}


\section[Arthur Schnitzler an Stefan Zweig, 2. 10. 1926]{L03747 Arthur Schnitzler an Stefan Zweig, 2. 10. 1926}
\nopagebreak\mylabel{L03747v}
\rehead{ }\normalsize\beginnumbering\briefempfaengerindex{Zweig, Stefan@\textsc{Zweig, Stefan}!zzzSchnitzler, Arthur@\emph{von Arthur Schnitzler}!1926-10-021@{2. 10. 1926}|(be}
\toendnotes[C]{\smallbreak\pagebreak[2]}\Standort{Jerusalem, National Library of Israel, ARC. Ms. Var. 305 1 58 Stefan Zweig Collection.}
\physDesc{Brief, 1 Blatt, 2 Seiten, 2520 Zeichen
\newline{}Schreibmaschine
\newline{}Handschrift: Bleistift, lateinische Kurrent (\noindent{}drei Ergänzungen, zwei Streichungen, Unterschrift)}\toendnotes[C]{\smallbreak}
\pstart
           {\pb}\textcolor{gray}{\textbf{D\textsuperscript{R} ARTHUR SCHNITZLER}}\hfill 2. 10. 1926. \pend
           
\pstart
           \textcolor{gray}{\textbf{WIEN, XVIII.
                           STERNWARTESTRASSE 71\oindex{Sternwartestrasse 71@\textbf{Sternwartestraße 71}, \emph{Wohngebäude (K.WHS)}|pw}.}}\pend
           
\pstart{}Lieber und verehrter Herr Doktor.\pend\vspace{0.5em}
\pstart
           Ich danke Ihnen sehr für Ihr neues Novellenbuch\pwindex{Verwirrung der Gefuehle. Drei Novellen@\emph{Verwirrung der Gefühle. Drei Novellen}|pwv}, das ich bei meiner Heimkehr vorgefunden habe. Mit stärkster
               innerer und äusserer Anteilnahme habe ich es gelesen. Die \label{K_L03747-1v}\edtext{erste Novelle\pwindex{Vierundzwanzig Stunden aus dem Leben einer Frau. Novelle@\emph{Vierundzwanzig Stunden aus dem Leben einer Frau. Novelle}|pwv} kannte ich}{\lemma{\textnormal{\emph{erste Novelle kannte ich}}}\Cendnote{\textnormal{Stefan Zweig\pwindex{Zweig, Stefan 28.11.1881 – 23.02.1942@\textsc{Zweig, Stefan} (28.11.1881 – 23.02.1942), \emph{Schriftsteller/Schriftstellerin}|pwk}: \emph{Vierundzwanzig Stunden aus dem Leben einer Frau.
                        Novelle}\pwindex{Vierundzwanzig Stunden aus dem Leben einer Frau. Novelle@\emph{Vierundzwanzig Stunden aus dem Leben einer Frau. Novelle}|pwk}. In: \emph{Neue Freie Presse}\pwindex{Neue Freie Presse@\emph{Neue Freie Presse}|pwk},
                     Nr. 22.013, 25. 12. 1925, Morgenblatt,
                     Weihnachtsbeilage, S. 31–44.}}}\label{K_L03747-1} schon, die Wirkung ist bei der
               zweiten Lektüre die gleiche ausserordentliche geblieben. Am merkwürdigsten ist wohl
               die dritte\pwindex{Verwirrung der Gefuehle@\emph{Verwirrung der Gefühle}|pwv}. Die Steigerung
               des epischen Stils an einzelnen in gewissem Sinn gefährlichen Stellen \introOben{}ins
                  hymnische\introOben{} erkannte ich nach anfänglichem leisen Widerstand als die
               wahrscheinlich einzige künstlerische Möglichkeit das kühne Problem zu meistern.
               Beinahe noch fesselnder, unmittelbarer ans Herz greifend hebt die zweite Novelle\pwindex{Untergang eines Herzens@\emph{Untergang eines Herzens}|pwv} an, aber mir ist, so
               glänzend auch diese Erzählung geführt ist, als hätte der Stoff – von einem gewissen
               Moment an, vielleicht schon von der Stelle, wo der Vater seine Tochter in dem fremden
               Hotelzimmer verschwinden sieht, noch ergiebigere Entwicklungsmöglichkeiten geboten
               als Sie ihm abgewonnen oder als Sie mit Absicht gewählt haben. Für mein Gefühl
               erklingt der Abgesang dieses väterlichen Schicksals zu früh. Aber das kommt
               vielleicht nur daher, weil von dem starken und originalen Anfang an die
               Ideenassoziationen des Lesers (und gar eines Lesers, in dem die Phantasie angeborener
               Weise und berufsmässig sozusagen auch Kunstwerken gegenüber, noch ehe er sie geduldig
                  \strikeout{aufgenommen} vom Beginn bis zum Ende in sich aufgenommen\substVorne{}\textsuperscript{)}\substDazwischen{},\substHinten{} frei und auf eigene Verantwortung zu schwingen anhebt) nach so vielen und
               verschiedenartigen {\pb}Richtungen gehen, und er nicht in die
               Notwendigkeit versetzt ist eine Entscheidung zu treffen, wäre es auch nur, um
               endgiltig seinem Stoff und seinen Gestalten zu entgehen. (Ich für meinen Teil war
               schon manchmal in der Versuchung einer oder der anderen meiner Novellen Varianten
               beizufügen. Ich glaube, dass so\introOben{}lch\introOben{} ein Versuch auch künstlerisch sehr diskutabel
               wäre. Die inneren Notwendigkeiten eines Schicksals sind ja natürlich immer gegeben,
               aber die äusseren Notwendigkeiten (zu denen für die Hauptgestalt ja auch wieder die
               inneren Notwendigkeiten der Gegenspieler und sogar der Episodenfiguren gehören)
               stehen von vornherein keineswegs fest. »\label{K_L03747-2v}\edtext{In unserer Brust sind unseres Schicksals
                  Sterne\pwindex{Wallensteins Tod@\emph{Wallensteins Tod}|pwv}}{\lemma{\textnormal{\emph{In … Sterne}}}\Cendnote{\textnormal{In \emph{Wallensteins Tod}\pwindex{Wallensteins Tod@\emph{Wallensteins Tod}|pwk} von Friedrich Schiller\pwindex{Schiller, Friedrich von 10.11.1759 – 09.05.1805@\textsc{Schiller, Friedrich von} (10.11.1759 – 09.05.1805), \emph{Schriftsteller/Schriftstellerin, Historiker/Historikerin, Philosoph/Philosophin}|pwk}
                     heißt es: »In deiner Brust sind deines Schicksals Sterne.«}}}\label{K_L03747-2}«. Zweifellos. Aber es sind nicht diese Sterne allein, die unser Schicksal
               regieren. \strikeout{(}Das hat, wie Sie hoffentlich merken,nur eine Spitze gegen den lieben Gott
               und nicht gegen den ausgezeichneten Dichter der »Verwirrung der Gefühle\pwindex{Verwirrung der Gefuehle. Drei Novellen@\emph{Verwirrung der Gefühle. Drei Novellen}|pw}«, den ich herzlich grüsse als sein aufrichtig
               ergebener\pend
           \pstart \spacefill\mbox{{[}hs.:{]} ArthurSchitzler}\pend{}
\pstart
           \noindent{}{[}ms.:{]} Herrn Dr. Stefan Zweig,\pend
           
\pstart
           Salzburg\oindex{Salzburg@\textbf{Salzburg}, \emph{A.ADM2}|pw}.\pend
           \selectlanguage{ngerman}\endnumbering\briefempfaengerindex{Zweig, Stefan@\textsc{Zweig, Stefan}!zzzSchnitzler, Arthur@\emph{von Arthur Schnitzler}!1926-10-021@{2. 10. 1926}|)be}\mylabel{L03747h}
\begin{anhang}
\end{anhang}\normalsize

\doendnotes{C}
\bigskip
\vfill

\clearpage

\footnotesize

\lohead{\textsc{register}}

% Definiere theindex-Environment komplett neu ohne reledmac
\makeatletter
\renewenvironment{theindex}{%
  \section*{\indexname}%
  \setlength{\parindent}{0pt}%
  \setlength{\parskip}{0pt plus 0.3pt}%
  \let\item\@idxitem
}{%
  \clearpage
}
\makeatother

\IfFileExists{\jobname-pw.ind}{\input{\jobname-pw.ind}}{}

\end{document}

      