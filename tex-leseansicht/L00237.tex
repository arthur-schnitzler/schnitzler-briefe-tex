%% latex-leseansicht-vorspann.tex
%% Vorspann für die Leseansicht.
%% Lädt die gemeinsame Datei latex-vorspann.tex mit nicht gesetztem Schalter.

\newif\ifkorrekturansicht
\korrekturansichtfalse

\input{../tex-inputs/latex-vorspann}


\section[Richard Beer-Hofmann an Arthur Schnitzler, 18. 7. 1893]{L00237 Richard Beer-Hofmann an Arthur Schnitzler, 18. 7. 1893}
\nopagebreak\mylabel{L00237v}
\rehead{ }\normalsize\beginnumbering\briefempfaengerindex{Schnitzler, Arthur@\textsc{Schnitzler, Arthur}!zzzBeer-Hofmann, Richard@\emph{von Richard Beer-Hofmann}!1893-07-181@{18. 7. 1893}|(be}
\toendnotes[C]{\smallbreak\pagebreak[2]}
\correspDesc{Versand  durch Richard Beer-Hofmann am 18. 7. 1893 in Bad Ischl
\newline{}Erhalt  durch Arthur Schnitzler im Zeitraum [19. 7. 1893
                  – 23. 7. 1893?] in Wien}\toendnotes[C]{\smallbreak}
\Standort{CUL, Schnitzler, B 8.}
\physDesc{Brief, 1 Blatt, 2 Seiten, 715 Zeichen
\newline{}Handschrift: Bleistift, lateinische Kurrent
\newline{}Ordnung: mit Bleistift von unbekannter Hand nummeriert:
                                    »20« }
\buchAbdrucke{\weitereDrucke{Arthur Schnitzler, Richard Beer-Hofmann: \emph{Briefwechsel 1891–1931}. Herausgegeben von Konstanze Fliedl. Wien, Zürich: \emph{Europaverlag} 1992, S. 46.} }\toendnotes[C]{\smallbreak}
\pstart
           \noindent{}{\pb}Lieber Arthur! Hier
               die Novelle\pwindex{Beer-Hofmann, Richard 11.\,7.\,1866 Wien – 26.\,9.\,1945 New York City@\textsc{Beer-Hofmann, Richard} (11.\,7.\,1866 Wien – 26.\,9.\,1945 New York City), \emph{Schriftsteller}!Kind@\strich\emph{Das Kind}|pwv} – bis auf das
               letzte Capitel das ich noch ändere. Bitte tun Sie was Sie können um die Abschrift zu
               beschleunigen, \uline{und schreiben Sie mir \introOben{}für\introOben{} wann er es verspricht}; geben Sie ihm eventuell eine Prämie für
               Beschleunigung. Vielleicht schicke ich auch das letzte Capitel\pwindex{Beer-Hofmann, Richard 11.\,7.\,1866 Wien – 26.\,9.\,1945 New York City@\textsc{Beer-Hofmann, Richard} (11.\,7.\,1866 Wien – 26.\,9.\,1945 New York City), \emph{Schriftsteller}!Kind@\strich\emph{Das Kind}|pwv} ein, aber warten Sie keinesfalls
               darauf.\pend
           
\pstart
           Devrient\pwindex{Devrient, Max 12.\,12.\,1857 Hannover – 13.\,6.\,1929 Chur@\textsc{Devrient, Max} (12.\,12.\,1857 Hannover – 13.\,6.\,1929 Chur), \emph{Regisseur, Schauspieler}|pw} wollte gestern Gedichte\pwindex{Schnitzler, Arthur 15.\,5.\,1862 Wien – 21.\,10.\,1931 ebd.@\textsc{Schnitzler, Arthur} (15.\,5.\,1862 Wien – 21.\,10.\,1931 ebd.), \emph{Schriftsteller, Mediziner}!Gedichte]@\strich\emph{[Gedichte]}|pwv} von Ihnen als Zugabe lesen, man
               schickte zu mir, – ich hatte begreiflicherweise keine. Schade! \label{K_L00237-1v}\edtext{Bauers\pwindex{Bauer, Ludwig 5.\,9.\,1876 Wien – 1.\,2.\,1935 Lugano@\textsc{Bauer, Ludwig} (5.\,9.\,1876 Wien – 1.\,2.\,1935 Lugano), \emph{Schriftsteller, Journalist}|pw}{ }Notiz\pwindex{Abschiedsouper in Ischl]@\emph{[Abschiedsouper in Ischl]}|pwv}}{\lemma{\textnormal{\emph{Bauers Notiz}}}\Cendnote{\textnormal{\emph{Illustrirtes Wiener Extrablatt}\pwindex{Illustrirtes Wiener Extrablatt@\emph{Illustrirtes Wiener Extrablatt}|pwk}, Jg. 22,
                     Nr. 196, 18. 7. 1893, S. 5.}}}\label{K_L00237-1} – er sagte mir gestern
               den Wortlaut {[}–{]} ist gut. Mit Paul Horn\pwindex{Horn, Paul 13.\,2.\,1867 Wien – 18.\,1.\,1936 Menton@\textsc{Horn, Paul} (13.\,2.\,1867 Wien – 18.\,1.\,1936 Menton), \emph{Fabrikant}|pw} habe ich wegen »Börsencourir\pwindex{Berliner Börsen-Courier@\emph{Berliner Börsen-Courier}|pw}\pwindex{Man schreibt uns aus Ischl]@\emph{[Man schreibt uns aus Ischl]}|pwv}« gesprochen. Lautenburg\pwindex{Lautenburg, Sigmund 11.\,9.\,1851 Budapest – 21.\,7.\,1918 Marienbad@\textsc{Lautenburg, Sigmund} (11.\,9.\,1851 Budapest – 21.\,7.\,1918 Marienbad), \emph{Theaterleiter, Schauspieler}|pw} ist {\pb}\strikeout{heut} gestern geko{\geminationm}en.\pend
           
\pstart
           Bitte also nochmals tun Sie was Sie können.\pend
           
\pstart
           Herzlichst{\\[\baselineskip]}\spacefill\mbox{Richard}\pend
           \leftskip=0em{}
\pstart
           Schwarzkopf\pwindex{Schwarzkopf, Gustav 7.\,11.\,1853 Wien – 13.\,11.\,1939 ebd.@\textsc{Schwarzkopf, Gustav} (7.\,11.\,1853 Wien – 13.\,11.\,1939 ebd.), \emph{Schriftsteller}|pw}, Salten\pwindex{Salten, Felix 6.\,9.\,1869 Budapest – 8.\,10.\,1945 Zürich@\textsc{Salten, Felix} (6.\,9.\,1869 Budapest – 8.\,10.\,1945 Zürich), \emph{Schriftsteller, Journalist, Chefredakteur}|pw}, herzlichst gegrüßt.\pend
           
\pstart
           Dienstag 18 Juli 93.\pend
           \selectlanguage{ngerman}\endnumbering\briefempfaengerindex{Schnitzler, Arthur@\textsc{Schnitzler, Arthur}!zzzBeer-Hofmann, Richard@\emph{von Richard Beer-Hofmann}!1893-07-181@{18. 7. 1893}|)be}\mylabel{L00237h}  \newcommand{\dateiname}{L00237}\newcommand{\titel}{Richard Beer-Hofmann an Arthur Schnitzler, 18. 7. 1893}\newcommand{\editorInnen}{Martin Anton Müller und Gerd-Hermann Susen}%% latex-leseansicht-abspann.tex
%% Abspann für die Leseansicht.
%% Der Schalter \ifkorrekturansicht ist bereits durch den Vorspann gesetzt.

%% latex-abspann.tex
%% Gemeinsamer Abspann für Korrekturansicht und Leseansicht.
%% Setzt den Schalter \ifkorrekturansicht voraus (gesetzt in den
%% einbindenden Dateien latex-korrekturansicht-abspann.tex bzw.
%% latex-leseansicht-abspann.tex).
%% ---------------------------------------------------------------

\normalsize

% Das esempio-Environment wird nur in der Leseansicht benötigt
\ifkorrekturansicht\else
\newenvironment{esempio}[3]%
{
    \vspace{1.5ex}
    \rlap{\underline{#1}}
    \par
    \setlength{\parindent}{0cm}
    \nopagebreak
    \leftskip=#2cm
    \rightskip=#3cm
}
{
    \par
}
\fi

\doendnotes{C}
\bigskip
\vfill

\clearpage

\footnotesize

\ifkorrekturansicht
  \lohead{\textsc{register}}
\fi

% theindex-Environment neu definieren ohne reledmac
\makeatletter
\renewenvironment{theindex}{%
  \ifkorrekturansicht
    \section*{\indexname}%
  \else
    \subsubsection*{Index der erwähnten Entitäten}%
  \fi
  \setlength{\parindent}{0pt}%
  \setlength{\parskip}{0pt plus 0.3pt}%
  \let\item\@idxitem
}{%
  \ifkorrekturansicht\clearpage\fi
}
\makeatother

\IfFileExists{\jobname-pw.ind}{\input{\jobname-pw.ind}}{}

% Quellenangabe nur in der Leseansicht
\ifkorrekturansicht\else
% Fallback-Definitionen, falls die .tex-Datei \titel etc. nicht gesetzt hat
\providecommand{\titel}{}
\providecommand{\editorInnen}{}
\providecommand{\dateiname}{\jobname}

\vspace{3cm}

\vfill

\footnotesize
\textsc{Quelle}: \titel. Herausgegeben von {\editorInnen}. In: \emph{Arthur Schnitzler: Briefwechsel mit Autorinnen und Autoren}.
 Digitale Edition, https://schnitzler-briefe.acdh.oeaw.ac.at/{\dateiname}.html (Stand \today)
\fi

\end{document}


