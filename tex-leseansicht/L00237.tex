%% latex-korrekturansicht-vorspann.tex
%% Vorspann für die Korrekturansicht.
%% Lädt die gemeinsame Datei latex-vorspann.tex mit gesetztem Schalter.

\newif\ifkorrekturansicht
\korrekturansichttrue

\input{../tex-inputs/latex-vorspann}


\section[Richard Beer-Hofmann an Arthur Schnitzler, 18. 7. 1893]{L00237 Richard Beer-Hofmann an Arthur Schnitzler, 18. 7. 1893}
\nopagebreak\mylabel{L00237v}
\rehead{ }\normalsize\beginnumbering\briefempfaengerindex{Schnitzler, Arthur@\textsc{Schnitzler, Arthur}!zzzBeer-Hofmann, Richard@\emph{von Richard Beer-Hofmann}!1893-07-181@{18. 7. 1893}|(be}
\toendnotes[C]{\smallbreak\pagebreak[2]}\Standort{CUL, Schnitzler, B 8.}
\physDesc{Brief, 1 Blatt, 2 Seiten, 715 Zeichen
\newline{}Handschrift: Bleistift, lateinische Kurrent
\newline{}Ordnung: mit Bleistift von unbekannter Hand nummeriert:
                                    »20« }
\buchAbdrucke{\weitereDrucke{Arthur Schnitzler, Richard Beer-Hofmann: \emph{Briefwechsel 1891–1931}. Wien, Zürich: \emph{Europaverlag} 1992, S. 46.} }\toendnotes[C]{\smallbreak}
\pstart
           \noindent{}{\pb}Lieber Arthur! Hier
               die Novelle\pwindex{Kind@\emph{Das Kind}|pwv} – bis auf das
               letzte Capitel das ich noch ändere. Bitte tun Sie was Sie können um die Abschrift zu
               beschleunigen, \uline{und schreiben Sie mir \introOben{}für\introOben{} wann er es verspricht}; geben Sie ihm eventuell eine Prämie für
               Beschleunigung. Vielleicht schicke ich auch das letzte Capitel\pwindex{Kind@\emph{Das Kind}|pwv} ein, aber warten Sie keinesfalls
               darauf.\pend
           
\pstart
           Devrient\pwindex{Devrient, Max 12.12.1857 – 13.06.1929@\textsc{Devrient, Max} (12.12.1857 – 13.06.1929), \emph{Regisseur/Regisseurin, Schauspieler/Schauspielerin}|pw} wollte gestern Gedichte\pwindex{Gedichte]@\emph{[Gedichte]}|pwv} von Ihnen als Zugabe lesen, man
               schickte zu mir, – ich hatte begreiflicherweise keine. Schade! \label{K_L00237-1v}\edtext{Bauers\pwindex{Bauer, Ludwig 05.09.1876 – 01.02.1935@\textsc{Bauer, Ludwig} (05.09.1876 – 01.02.1935), \emph{Schriftsteller/Schriftstellerin, Journalist/Journalistin}|pw}{ }Notiz\pwindex{Abschiedsouper in Ischl]@\emph{[Abschiedsouper in Ischl]}|pwv}}{\lemma{\textnormal{\emph{Bauers Notiz}}}\Cendnote{\textnormal{\emph{Illustrirtes Wiener Extrablatt}\pwindex{Illustrirtes Wiener Extrablatt@\emph{Illustrirtes Wiener Extrablatt}|pwk}, Jg. 22,
                     Nr. 196, 18. 7. 1893, S. 5.}}}\label{K_L00237-1} – er sagte mir gestern
               den Wortlaut {[}–{]} ist gut. Mit Paul Horn\pwindex{Horn, Paul 13.02.1867 – 18.01.1936@\textsc{Horn, Paul} (13.02.1867 – 18.01.1936), \emph{Fabrikant/Fabrikantin}|pw} habe ich wegen »Börsencourir\pwindex{Berliner Boersen-Courier@\emph{Berliner Börsen-Courier}|pw}\pwindex{Man schreibt uns aus Ischl]@\emph{[Man schreibt uns aus Ischl]}|pwv}« gesprochen. Lautenburg\pwindex{Lautenburg, Sigmund 11.09.1851 – 21.07.1918@\textsc{Lautenburg, Sigmund} (11.09.1851 – 21.07.1918), \emph{Theaterleiter/Theaterleiterin, Schauspieler/Schauspielerin}|pw} ist {\pb}\strikeout{heut} gestern geko{\geminationm}en.\pend
           
\pstart
           Bitte also nochmals tun Sie was Sie können.\pend
           
\pstart
           Herzlichst{\\[\baselineskip]}\spacefill\mbox{Richard}\pend
           \leftskip=0em{}
\pstart
           Schwarzkopf\pwindex{Schwarzkopf, Gustav 07.11.1853 – 13.11.1939@\textsc{Schwarzkopf, Gustav} (07.11.1853 – 13.11.1939), \emph{Schriftsteller/Schriftstellerin}|pw}, Salten\pwindex{Salten, Felix 06.09.1869 – 08.10.1945@\textsc{Salten, Felix} (06.09.1869 – 08.10.1945), \emph{Schriftsteller/Schriftstellerin, Journalist/Journalistin, Chefredakteur/Chefredakteurin}|pw}, herzlichst gegrüßt.\pend
           
\pstart
           Dienstag 18 Juli 93.\pend
           \selectlanguage{ngerman}\endnumbering\briefempfaengerindex{Schnitzler, Arthur@\textsc{Schnitzler, Arthur}!zzzBeer-Hofmann, Richard@\emph{von Richard Beer-Hofmann}!1893-07-181@{18. 7. 1893}|)be}\mylabel{L00237h}  \normalsize

\doendnotes{C}
\bigskip
\vfill

\clearpage

\footnotesize

\lohead{\textsc{register}}

% Definiere theindex-Environment komplett neu ohne reledmac
\makeatletter
\renewenvironment{theindex}{%
  \section*{\indexname}%
  \setlength{\parindent}{0pt}%
  \setlength{\parskip}{0pt plus 0.3pt}%
  \let\item\@idxitem
}{%
  \clearpage
}
\makeatother

\IfFileExists{\jobname-pw.ind}{\input{\jobname-pw.ind}}{}

\end{document}

      