%% latex-leseansicht-vorspann.tex
%% Vorspann für die Leseansicht.
%% Lädt die gemeinsame Datei latex-vorspann.tex mit nicht gesetztem Schalter.

\newif\ifkorrekturansicht
\korrekturansichtfalse

\input{../tex-inputs/latex-vorspann}


         
         \newcommand{\erwaehntePersonen}{Personen: Ludwig Bauer, Max Devrient, Paul Horn, Sigmund Lautenburg, Felix Salten, Gustav Schwarzkopf}
         \newcommand{\erwaehnteOrte}{Orte: Bad Ischl, Wien}
         \newcommand{\erwaehnteWerke}{Werke: Berliner Börsen-Courier, Das Kind, Illustrirtes Wiener Extrablatt, [Abschiedsouper in Ischl], [Gedichte], [Man schreibt uns aus Ischl]}
               \section[Richard Beer-Hofmann an Arthur Schnitzler, 18. 7. 1893]{ Richard Beer-Hofmann an Arthur Schnitzler, 18. 7. 1893}\nopagebreak\mylabel{v}\rehead{ }\begin{ledgroupsized}[t]{13cm}\normalsize\beginnumbering \toendnotes[C]{\smallbreak\pagebreak[2]} \Standort{CUL, Schnitzler, B 8.}
\physDesc{Brief, 1 Blatt, 2 Seiten
\newline{}Handschrift: Bleistift, lateinische Kurrent\newline{}Ordnung: mit Bleistift von unbekannter Hand nummeriert:
                                    »20« }\buchAbdrucke{\weitereDrucke{Arthur Schnitzler, Richard Beer-Hofmann: \emph{Briefwechsel 1891–1931}. Hg. Konstanze Fliedl. Wien, Zürich: \emph{Europaverlag} 1992, S. 46.} }\toendnotes[C]{\smallbreak}\pstart
           \noindent{}{\pb}Lieber Arthur! Hier
               die Novelle\pwindex{Beer-Hofmann, Richard 1866-07-11 – 1945-09-26@\textsc{Beer-Hofmann, Richard} (1866-07-11 – 1945-09-26), \emph{Schriftsteller}!Kind1893@\strich\emph{Das Kind} {[}1893{]}|pwv} – bis auf das
               letzte Capitel das ich noch ändere. Bitte tun Sie was Sie können um die Abschrift zu
               beschleunigen, \uline{und schreiben Sie mir \introOben{}für\introOben{} wann er es verspricht}; geben Sie ihm eventuell eine Prämie für
               Beschleunigung. Vielleicht schicke ich auch das letzte Capitel\pwindex{Beer-Hofmann, Richard 1866-07-11 – 1945-09-26@\textsc{Beer-Hofmann, Richard} (1866-07-11 – 1945-09-26), \emph{Schriftsteller}!Kind1893@\strich\emph{Das Kind} {[}1893{]}|pwv} ein, aber warten Sie keinesfalls
               darauf.\pend
           \pstart
           Devrient\pwindex{Devrient, Max 12.12.1857 – 13.06.1929@\textsc{Devrient, Max} (12.12.1857 – 13.06.1929), \emph{Regisseur, Schauspieler}|pw} wollte gestern Gedichte\pwindex{Schnitzler, Arthur 15.05.1862 – 21.10.1931@\textsc{Schnitzler, Arthur} (15.05.1862 – 21.10.1931), \emph{Schriftsteller, Mediziner}!Gedichte]None@\strich\emph{[Gedichte]} {[}None{]}|pwv} von Ihnen als Zugabe lesen, man
               schickte zu mir, – ich hatte begreiflicherweise keine. Schade! \label{K_L00237-55v}\edtext{Bauer\pwindex{Bauer, Ludwig 05.09.1876 – 01.02.1935@\textsc{Bauer, Ludwig} (05.09.1876 – 01.02.1935), \emph{Schriftsteller, Journalist}|pw}s Notiz\pwindex{Abschiedsouper in Ischl]18. 07. 1893@\emph{[Abschiedsouper in Ischl]} {[}18. 07. 1893{]}|pwv}}{\lemma{\textnormal{\emph{Bauers Notiz}}}\Cendnote{\textnormal{\emph{Illustriertes Wiener Extrablatt}\pwindex{Illustrirtes Wiener Extrablatt1872 – 1928@\emph{Illustrirtes Wiener Extrablatt} {[}1872 – 1928{]}|pwk}, Jg. 22, Nr. 196,
                        18. 7. 1893, S. 5.}}}\label{K_L00237-55h} – er sagte mir
               gestern den Wortlaut {[}–{]} ist gut. Mit Paul Horn\pwindex{Horn, Paul 13.02.1867 – 18.01.1936@\textsc{Horn, Paul} (13.02.1867 – 18.01.1936), \emph{Fabrikant}|pw} habe ich wegen »Börsencourir\pwindex{?? Werk@Nicht ermittelte Verfasserinnen und Verfasser!Berliner Boersen-Courier1868 – 1933@\emph{Berliner Börsen-Courier} {[}1868 – 1933{]}|pw}\pwindex{?? Werk@Nicht ermittelte Verfasserinnen und Verfasser!Man schreibt uns aus Ischl]25. 07. 1893@\emph{[Man schreibt uns aus Ischl]} {[}25. 07. 1893{]}|pwv}« gesprochen. Lautenburg\pwindex{Lautenburg, Sigmund 11.09.1851 – 21.07.1918@\textsc{Lautenburg, Sigmund} (11.09.1851 – 21.07.1918), \emph{Theaterleiter, Schauspieler}|pw} ist {\pb}\strikeout{heut} gestern geko{\geminationm}en.\pend
           \pstart
           Bitte also nochmals tun Sie was Sie können.\pend
           \pstart
           Herzlichst{\\[\baselineskip]}\spacefill\mbox{Richard}\pend
           \leftskip=0em{}\pstart
           Schwarzkopf\pwindex{Schwarzkopf, Gustav 07.11.1853 – 13.11.1939@\textsc{Schwarzkopf, Gustav} (07.11.1853 – 13.11.1939), \emph{Schriftsteller}|pw}, Salten\pwindex{Salten, Felix 06.09.1869 – 08.10.1945@\textsc{Salten, Felix} (06.09.1869 – 08.10.1945), \emph{Schriftsteller, Journalist}|pw}, herzlichst gegrüßt.\pend
           \pstart
           Dienstag 18 Juli 93.\pend
           
         
         \endnumbering\mylabel{h}\end{ledgroupsized}  \newcommand{\dateiname}{L00237}\newcommand{\titel}{Richard Beer-Hofmann an Arthur Schnitzler, 18. 7. 1893}\newcommand{\editorInnen}{Martin Anton Müller und Gerd-Hermann Susen}%% latex-leseansicht-abspann.tex
%% Abspann für die Leseansicht.
%% Der Schalter \ifkorrekturansicht ist bereits durch den Vorspann gesetzt.

%% latex-abspann.tex
%% Gemeinsamer Abspann für Korrekturansicht und Leseansicht.
%% Setzt den Schalter \ifkorrekturansicht voraus (gesetzt in den
%% einbindenden Dateien latex-korrekturansicht-abspann.tex bzw.
%% latex-leseansicht-abspann.tex).
%% ---------------------------------------------------------------

\normalsize

% Das esempio-Environment wird nur in der Leseansicht benötigt
\ifkorrekturansicht\else
\newenvironment{esempio}[3]%
{
    \vspace{1.5ex}
    \rlap{\underline{#1}}
    \par
    \setlength{\parindent}{0cm}
    \nopagebreak
    \leftskip=#2cm
    \rightskip=#3cm
}
{
    \par
}
\fi

\doendnotes{C}
\bigskip
\vfill

\clearpage

\footnotesize

\ifkorrekturansicht
  \lohead{\textsc{register}}
\fi

% theindex-Environment neu definieren ohne reledmac
\makeatletter
\renewenvironment{theindex}{%
  \ifkorrekturansicht
    \section*{\indexname}%
  \else
    \subsubsection*{Index der erwähnten Entitäten}%
  \fi
  \setlength{\parindent}{0pt}%
  \setlength{\parskip}{0pt plus 0.3pt}%
  \let\item\@idxitem
}{%
  \ifkorrekturansicht\clearpage\fi
}
\makeatother

\IfFileExists{\jobname-pw.ind}{\input{\jobname-pw.ind}}{}

% Quellenangabe nur in der Leseansicht
\ifkorrekturansicht\else
% Fallback-Definitionen, falls die .tex-Datei \titel etc. nicht gesetzt hat
\providecommand{\titel}{}
\providecommand{\editorInnen}{}
\providecommand{\dateiname}{\jobname}

\vspace{3cm}

\vfill

\footnotesize
\textsc{Quelle}: \titel. Herausgegeben von {\editorInnen}. In: \emph{Arthur Schnitzler: Briefwechsel mit Autorinnen und Autoren}.
 Digitale Edition, https://schnitzler-briefe.acdh.oeaw.ac.at/{\dateiname}.html (Stand \today)
\fi

\end{document}


      