%% latex-korrekturansicht-vorspann.tex
%% Vorspann für die Korrekturansicht.
%% Lädt die gemeinsame Datei latex-vorspann.tex mit gesetztem Schalter.

\newif\ifkorrekturansicht
\korrekturansichttrue

\input{../tex-inputs/latex-vorspann}


\section[Felix Salten an Arthur Schnitzler, {[}1. oder 3.? 8. 1893{]}]{L03124 Felix Salten an Arthur Schnitzler, {[}1. oder 3.? 8. 1893{]}}
\nopagebreak\mylabel{L03124v}
\rehead{ }\normalsize\beginnumbering\briefempfaengerindex{Schnitzler, Arthur@\textsc{Schnitzler, Arthur}!zzzSalten, Felix@\emph{von Felix Salten}!1893-08-031@{{[}1. oder
                  3.? 8. 1893{]}}|(be}
\toendnotes[C]{\smallbreak\pagebreak[2]}\Standort{CUL, Schnitzler, B 89, A 1.}
\physDesc{Brief, 1 Blatt, 2 Seiten, 338 Zeichen
\newline{}Handschrift: blauer Buntstift, lateinische Kurrent
\newline{}Schnitzler: mit Bleistift datiert: »Anf{[}ang{]}
                                       Aug{[}ust{]} 93« 
\newline{}Ordnung: mit Bleistift von unbekannter Hand nummeriert: »27« }\toendnotes[C]{\smallbreak}
\pstart
           \noindent{}{\pb}lieber Freund! Ich habe die herzliche Bitte an Sie,
               mir, wenn es Ihnen möglich ist 5 f zu senden. \label{K_L03124-1v}\edtext{Dülberg\pwindex{Duelberg, Karl 29.01.1865 – 31.12.1933@\textsc{Dülberg, Karl} (29.01.1865 – 31.12.1933)|pwu}}{\lemma{\textnormal{\emph{Dülberg}}}\Cendnote{\textnormal{Möglicherweise hatte Salten\pwindex{Salten, Felix 06.09.1869 – 08.10.1945@\textsc{Salten, Felix} (06.09.1869 – 08.10.1945), \emph{Schriftsteller/Schriftstellerin, Journalist/Journalistin, Chefredakteur/Chefredakteurin}|pwk} einen Text in der von Karl Dülberg\pwindex{Duelberg, Karl 29.01.1865 – 31.12.1933@\textsc{Dülberg, Karl} (29.01.1865 – 31.12.1933)|pwk} herausgegebenen \emph{Österreichischen Feuilleton-Korrespondenz}\pwindex{Oesterreichische Feuilleton-Korrespondenz@\emph{Österreichische Feuilleton-Korrespondenz}|pwk} veröffentlicht
                  und dafür (noch) kein Honorar erhalten.}}}\label{K_L03124-1} hat mir wider Erwarten Nichts
               gegeben, u. will mir das Geld möglicherweise nachschicken. Mein \label{K_L03124-2v}\edtext{Rad muss ich Nachmittag}{\lemma{\textnormal{\emph{Rad muss ich Nachmittag}}}\Cendnote{\textnormal{An mehreren Tagen Anfang August 1893 unternahmen Salten\pwindex{Salten, Felix 06.09.1869 – 08.10.1945@\textsc{Salten, Felix} (06.09.1869 – 08.10.1945), \emph{Schriftsteller/Schriftstellerin, Journalist/Journalistin, Chefredakteur/Chefredakteurin}|pwk} und Schnitzler gemeinsame
                  Radausflüge, doch nur die am 1. 8. 1893 und am 3. 8. 1893 scheinen am Abend
                  stattgefunden zu haben.}}}\label{K_L03124-2} aus der Re{\pb}paratur holen, und habe gar kein
               Geld. Wenns geht hole ich Sie um ½ 6 Uhr aus Ihrer Wohnung\oindex{Kaerntnerring 12/Boesendorferstrasse 11@\textbf{Kärntnerring 12/Bösendorferstraße 11}, \emph{Wohngebäude (K.WHS)}|pwv} ab.\pend
           
\pstart
           Herzlichst {\\[\baselineskip]}Ihr {\\[\baselineskip]}\spacefill\mbox{Salten}\pend
           \leftskip=0em{}\selectlanguage{ngerman}\endnumbering\briefempfaengerindex{Schnitzler, Arthur@\textsc{Schnitzler, Arthur}!zzzSalten, Felix@\emph{von Felix Salten}!1893-08-011@{{[}1. oder
                  3.? 8. 1893{]}}|)be}\mylabel{L03124h}  \normalsize

\doendnotes{C}
\bigskip
\vfill

\clearpage

\footnotesize

\lohead{\textsc{register}}

% Definiere theindex-Environment komplett neu ohne reledmac
\makeatletter
\renewenvironment{theindex}{%
  \section*{\indexname}%
  \setlength{\parindent}{0pt}%
  \setlength{\parskip}{0pt plus 0.3pt}%
  \let\item\@idxitem
}{%
  \clearpage
}
\makeatother

\IfFileExists{\jobname-pw.ind}{\input{\jobname-pw.ind}}{}

\end{document}

      