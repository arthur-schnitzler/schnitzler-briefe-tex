%% latex-leseansicht-vorspann.tex
%% Vorspann für die Leseansicht.
%% Lädt die gemeinsame Datei latex-vorspann.tex mit nicht gesetztem Schalter.

\newif\ifkorrekturansicht
\korrekturansichtfalse

\input{../tex-inputs/latex-vorspann}

\begin{center}
            \textcolor{red}{ENTWURF, NICHT FERTIG KORRIGIERT}
                      \end{center}
            
         
         \renewcommand{\erwaehntePersonen}{Personen: Karl Dülberg}
         \renewcommand{\erwaehnteOrte}{Orte: Wien}
         \renewcommand{\erwaehnteWerke}{}
               \section[Felix Salten an Arthur Schnitzler, {[}1. oder 3.? 8. 1893{]}]{ Felix Salten an Arthur Schnitzler, {[}1. oder 3.? 8. 1893{]}}\nopagebreak\mylabel{v}\rehead{ }\begin{ledgroupsized}[t]{13cm}\normalsize\beginnumbering \toendnotes[C]{\smallbreak\pagebreak[2]} \Standort{CUL, Schnitzler, B 89, A 1.}
\physDesc{Brief, 1 Blatt, 2 Seiten, 345 Zeichen
\newline{}Handschrift: blauer Buntstift, lateinische Kurrent
\newline{}Schnitzler: mit Bleistift datiert: »Anf{[}ang{]} Aug{[}ust{]}
                                       93« 
\newline{}Ordnung: mit Bleistift von unbekannter Hand nummeriert:
                                    »27« }\toendnotes[C]{\smallbreak}\pstart
           \noindent{}{\pb}lieber Freund! Ich habe die herzliche Bitte an Sie, nur, wenn es
               Ihnen möglich ist: 5 f zu senden. Dülberg\pwindex{Duelberg, Karl 29.01.1865 – 31.12.1933@\textsc{Dülberg, Karl} (29.01.1865 – 31.12.1933)|pwu} hat mir wider Erwarten Nichts gegeben, u. will mir das Geld
               möglicherweise nachschicken. Mein \label{K_L03124-1v}\edtext{Rad
               muss ich Nachmittag}{\lemma{\textnormal{\emph{Rad
               muss ich Nachmittag}}}\Cendnote{\textnormal{An mehreren Tagen
                  Anfang August 1893 unternehmen Salten\pwindex{Salten, Felix 06.09.1869 – 08.10.1945@\textsc{Salten, Felix} (06.09.1869 – 08.10.1945), \emph{Schriftsteller, Journalist}|pwk} und
                     Schnitzler\pwindex{Schnitzler, Arthur 15.05.1862 – 21.10.1931@\textsc{Schnitzler, Arthur} (15.05.1862 – 21.10.1931), \emph{Schriftsteller, Mediziner}|pwk} gemeinsame Radausflüge, doch
                  nur die am 1. 8. 1893 und am 3. 8. 1893 scheinen am Abend stattgefunden zu haben.}}}\label{K_L03124-1h} aus der Re{\pb}paratur holen, und habe gar kein
               Geld. Wenn's geht hole ich Sie um ½ 6 Uhr aus Ihrer Wohnung ab.\pend
           \pstart
           Herzlichst {\\[\baselineskip]}Ihr \spacefill\mbox{Salten}\pend
           \leftskip=0em{}
         
         \endnumbering\mylabel{h}\end{ledgroupsized}\begin{anhang}\end{anhang}\newcommand{\dateiname}{L03124}\newcommand{\titel}{Felix Salten an Arthur Schnitzler, [1. oder 3.? 8. 1893]}\newcommand{\editorInnen}{Martin Anton Müller und Laura Untner}%% latex-leseansicht-abspann.tex
%% Abspann für die Leseansicht.
%% Der Schalter \ifkorrekturansicht ist bereits durch den Vorspann gesetzt.

%% latex-abspann.tex
%% Gemeinsamer Abspann für Korrekturansicht und Leseansicht.
%% Setzt den Schalter \ifkorrekturansicht voraus (gesetzt in den
%% einbindenden Dateien latex-korrekturansicht-abspann.tex bzw.
%% latex-leseansicht-abspann.tex).
%% ---------------------------------------------------------------

\normalsize

% Das esempio-Environment wird nur in der Leseansicht benötigt
\ifkorrekturansicht\else
\newenvironment{esempio}[3]%
{
    \vspace{1.5ex}
    \rlap{\underline{#1}}
    \par
    \setlength{\parindent}{0cm}
    \nopagebreak
    \leftskip=#2cm
    \rightskip=#3cm
}
{
    \par
}
\fi

\doendnotes{C}
\bigskip
\vfill

\clearpage

\footnotesize

\ifkorrekturansicht
  \lohead{\textsc{register}}
\fi

% theindex-Environment neu definieren ohne reledmac
\makeatletter
\renewenvironment{theindex}{%
  \ifkorrekturansicht
    \section*{\indexname}%
  \else
    \subsubsection*{Index der erwähnten Entitäten}%
  \fi
  \setlength{\parindent}{0pt}%
  \setlength{\parskip}{0pt plus 0.3pt}%
  \let\item\@idxitem
}{%
  \ifkorrekturansicht\clearpage\fi
}
\makeatother

\IfFileExists{\jobname-pw.ind}{\input{\jobname-pw.ind}}{}

% Quellenangabe nur in der Leseansicht
\ifkorrekturansicht\else
% Fallback-Definitionen, falls die .tex-Datei \titel etc. nicht gesetzt hat
\providecommand{\titel}{}
\providecommand{\editorInnen}{}
\providecommand{\dateiname}{\jobname}

\vspace{3cm}

\vfill

\footnotesize
\textsc{Quelle}: \titel. Herausgegeben von {\editorInnen}. In: \emph{Arthur Schnitzler: Briefwechsel mit Autorinnen und Autoren}.
 Digitale Edition, https://schnitzler-briefe.acdh.oeaw.ac.at/{\dateiname}.html (Stand \today)
\fi

\end{document}


      