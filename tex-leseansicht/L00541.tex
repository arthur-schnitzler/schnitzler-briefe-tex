%% latex-leseansicht-vorspann.tex
%% Vorspann für die Leseansicht.
%% Lädt die gemeinsame Datei latex-vorspann.tex mit nicht gesetztem Schalter.

\newif\ifkorrekturansicht
\korrekturansichtfalse

\input{../tex-inputs/latex-vorspann}


\section[Georg Brandes an Arthur Schnitzler, 22. 4. 1896]{L00541 Georg Brandes an Arthur Schnitzler, 22. 4. 1896}
\nopagebreak\mylabel{L00541v}
\rehead{ }\normalsize\beginnumbering\briefempfaengerindex{Schnitzler, Arthur@\textsc{Schnitzler, Arthur}!zzzBrandes, Georg@\emph{von Georg Brandes}!1896-04-221@{22. 4. 1896}|(be}
\toendnotes[C]{\smallbreak\pagebreak[2]}
\correspDesc{Versand  durch Georg Brandes am 22. 4. 1896 in Kopenhagen
\newline{}Erhalt  durch Arthur Schnitzler im Zeitraum [23. 4. 1896
                  – 27. 4. 1896?] in Wien}\toendnotes[C]{\smallbreak}
\Standort{CUL, Schnitzler, B 17.}
\physDesc{Brief, 1 Blatt, 1 Seite, 624 Zeichen
\newline{}Handschrift: blaue Tinte, deutsche Kurrent
\newline{}Ordnung: von unbekannter Hand nummeriert: »2« }
\buchAbdrucke{\weitereDrucke{Georg Brandes, Arthur Schnitzler: \emph{Ein Briefwechsel}. Herausgegeben von Kurt Bergel. Bern: \emph{Francke} 1956, S. 56.} }
\pstart
           \raggedleft{}{\pb}Kopenhagen\oindex{Kopenhagen@\textbf{Kopenhagen}, \emph{Hauptstadt}|pw}{ }22 April 96\pend
           
\pstart{}Verehrter Herr\pend\vspace{0.5em}
\pstart
           Kürzlich sah ich in Berlin\oindex{Österreich@\textbf{Österreich}|pw} im Deutschen Theater\oindex{Deutsches Theater Berlin@\textbf{Deutsches Theater Berlin}, \emph{Theater}|pw} das Schauspiel Liebelei\pwindex{Schnitzler, Arthur 15.\,5.\,1862 Wien – 21.\,10.\,1931 ebd.@\textsc{Schnitzler, Arthur} (15.\,5.\,1862 Wien – 21.\,10.\,1931 ebd.), \emph{Schriftsteller, Mediziner}!Liebelei. Schauspiel in drei Akten@\strich\emph{Liebelei. Schauspiel in drei Akten}|pw} und es drängt mich, Ihnen zu sagen, welch starken Eindruck es auf
               mich gemacht hat. Seit langer Zeit habe ich bei einem deutschen Stück nicht so viel
               gefühlt. Es hat mich ganz ergriffen. Die Aufführung, die Sie vermutlich kennen, war
               ganz auf der Höhe der dramatischen Arbeit, Reicher\pwindex{Reicher, Emanuel 18.\,6.\,1849 Bochnia – 15.\,5.\,1924 Berlin@\textsc{Reicher, Emanuel} (18.\,6.\,1849 Bochnia – 15.\,5.\,1924 Berlin), \emph{Schauspieler}|pw}, Rittner\pwindex{Rittner, Rudolf 30.\,6.\,1869 Bílý Potok – 4.\,2.\,1943 ebd.@\textsc{Rittner, Rudolf} (30.\,6.\,1869 Bílý Potok – 4.\,2.\,1943 ebd.), \emph{Theaterleiter, Schauspieler}|pw}, Jarno\pwindex{Jarno, Josef 24.\,8.\,1865 Budapest – 11.\,1.\,1932 Wien@\textsc{Jarno, Josef} (24.\,8.\,1865 Budapest – 11.\,1.\,1932 Wien), \emph{Theaterleiter, Schauspieler}|pw}, besonders die Sorma\pwindex{Sorma, Agnes 17.\,5.\,1862 Breslau – 10.\,2.\,1927 Crown King@\textsc{Sorma, Agnes} (17.\,5.\,1862 Breslau – 10.\,2.\,1927 Crown King), \emph{Schauspielerin}|pw} alle vorzüglich.\pend
           
\pstart
           Es ist nicht hübsch von Ihnen, dass Sie mich vergessen haben in Ihrem Erfolg und
                  mi\strikeout{c}r\strikeout{h} das Ding nicht
               sandten, da ich doch alles Andere von Ihnen habe\pend
           \pstart Mit herzlichstem Gruss\hspace*{3.5em}Ihr\hspace*{3.5em}\spacefill\mbox{Georg Brandes}\pend{}\selectlanguage{ngerman}\endnumbering\briefempfaengerindex{Schnitzler, Arthur@\textsc{Schnitzler, Arthur}!zzzBrandes, Georg@\emph{von Georg Brandes}!1896-04-221@{22. 4. 1896}|)be}\mylabel{L00541h}  \newcommand{\dateiname}{L00541}\newcommand{\titel}{Georg Brandes an Arthur Schnitzler, 22. 4. 1896}\newcommand{\editorInnen}{Martin Anton Müller und Gerd-Hermann Susen}%% latex-leseansicht-abspann.tex
%% Abspann für die Leseansicht.
%% Der Schalter \ifkorrekturansicht ist bereits durch den Vorspann gesetzt.

%% latex-abspann.tex
%% Gemeinsamer Abspann für Korrekturansicht und Leseansicht.
%% Setzt den Schalter \ifkorrekturansicht voraus (gesetzt in den
%% einbindenden Dateien latex-korrekturansicht-abspann.tex bzw.
%% latex-leseansicht-abspann.tex).
%% ---------------------------------------------------------------

\normalsize

% Das esempio-Environment wird nur in der Leseansicht benötigt
\ifkorrekturansicht\else
\newenvironment{esempio}[3]%
{
    \vspace{1.5ex}
    \rlap{\underline{#1}}
    \par
    \setlength{\parindent}{0cm}
    \nopagebreak
    \leftskip=#2cm
    \rightskip=#3cm
}
{
    \par
}
\fi

\doendnotes{C}
\bigskip
\vfill

\clearpage

\footnotesize

\ifkorrekturansicht
  \lohead{\textsc{register}}
\fi

% theindex-Environment neu definieren ohne reledmac
\makeatletter
\renewenvironment{theindex}{%
  \ifkorrekturansicht
    \section*{\indexname}%
  \else
    \subsubsection*{Index der erwähnten Entitäten}%
  \fi
  \setlength{\parindent}{0pt}%
  \setlength{\parskip}{0pt plus 0.3pt}%
  \let\item\@idxitem
}{%
  \ifkorrekturansicht\clearpage\fi
}
\makeatother

\IfFileExists{\jobname-pw.ind}{\input{\jobname-pw.ind}}{}

% Quellenangabe nur in der Leseansicht
\ifkorrekturansicht\else
% Fallback-Definitionen, falls die .tex-Datei \titel etc. nicht gesetzt hat
\providecommand{\titel}{}
\providecommand{\editorInnen}{}
\providecommand{\dateiname}{\jobname}

\vspace{3cm}

\vfill

\footnotesize
\textsc{Quelle}: \titel. Herausgegeben von {\editorInnen}. In: \emph{Arthur Schnitzler: Briefwechsel mit Autorinnen und Autoren}.
 Digitale Edition, https://schnitzler-briefe.acdh.oeaw.ac.at/{\dateiname}.html (Stand \today)
\fi

\end{document}


