%% latex-leseansicht-vorspann.tex
%% Vorspann für die Leseansicht.
%% Lädt die gemeinsame Datei latex-vorspann.tex mit nicht gesetztem Schalter.

\newif\ifkorrekturansicht
\korrekturansichtfalse

\input{../tex-inputs/latex-vorspann}

\begin{center}
            \textcolor{red}{ENTWURF, NICHT FERTIG KORRIGIERT}
                      \end{center}
            
         
         \renewcommand{\erwaehntePersonen}{Personen: Josef Kainz, Margarethe Kainz, Anna Katharina Rehmann, Paul Salten, Olga Schnitzler, Heinrich Schnitzler}
         \renewcommand{\erwaehnteInstitutionen}{Institutionen: Burgtheater}
         \renewcommand{\erwaehnteOrte}{Orte: Grado, Lido, Venedig, Villa Bauer, Wien}
         \renewcommand{\erwaehnteWerke}{Werke: Der neue Vertrag von Josef Kainz’, Neue Freie Presse}
               \section[Felix Salten an Arthur Schnitzler, 29. 6. 1909]{ Felix Salten an Arthur Schnitzler, 29. 6. 1909}\nopagebreak\mylabel{v}\rehead{ }\begin{ledgroupsized}[t]{13cm}\normalsize\beginnumbering \toendnotes[C]{\smallbreak\pagebreak[2]} \Standort{CUL, Schnitzler, B 89, B 1.}
\physDesc{Brief, 1 Blatt, 1 Seite
\newline{}Handschrift: schwarze Tinte, lateinische Kurrent
\newline{}Schnitzler: mit Bleistift Vermerk: »\textsc{Salten}« \newline{}Ordnung: mit Bleistift von unbekannter Hand nummeriert:
                                    »251« }\toendnotes[C]{\smallbreak}\pstart
           \noindent{}{\pb}\textcolor{gray}{\textbf{Villa Bauer}}\oindex{Villa Bauer@\textbf{Villa Bauer}|pw}\hfill \textcolor{gray}{\textbf{Grado}}\oindex{Grado@\textbf{Grado}|pw}\pend
           \pstart
           \raggedleft{}\textcolor{gray}{\textbf{Küstenland}}\pend
           \pstart
           \raggedleft{}29. VI. 09\pend
           \pstart{}Lieber,\pend\pstart
            wir sind heute aus Venedig\oindex{Venedig@\textbf{Venedig}|pw} zurückgekommen, und
               ich finde Ihren Brief vom 22. Das letzte, was mir Kainz\pwindex{Kainz, Josef 02.01.1858 – 20.09.1910@\textsc{Kainz, Josef} (02.01.1858 – 20.09.1910), \emph{Schauspieler}|pw} sagte, war etwa zwei Tage vor meiner Abreise, und da meinte er, er
               wolle es seiner Frau\pwindex{Kainz, Margarethe 13.12.1858 – 12.02.1950@\textsc{Kainz, Margarethe} (13.12.1858 – 12.02.1950), \emph{Schauspielerin}|pwv}
               überlaſsen, im Dezember darüber zu verfügen. Deshalb glaube ich, die \label{K_L03501-1v}\edtext{Zeitungsnotiz\pwindex{?? Werk@Nicht ermittelte Verfasserinnen und Verfasser!neue Vertrag von Josef Kainz 1909-06-23@\emph{Der neue Vertrag von Josef Kainz’} {[}1909-06-23{]}|pwv}}{\lemma{\textnormal{\emph{Zeitungsnotiz}}}\Cendnote{\textnormal{[O. V.:] \emph{Der neue Vertrag von
                        Josef Kainz’}\pwindex{?? Werk@Nicht ermittelte Verfasserinnen und Verfasser!neue Vertrag von Josef Kainz 1909-06-23@\emph{Der neue Vertrag von Josef Kainz’} {[}1909-06-23{]}|pwk}. In: \emph{Neue Freie
                        Presse}\pwindex{Neue Freie Presse1864 – 1939@\emph{Neue Freie Presse} {[}1864 – 1939{]}|pwk}, Nr. 16.105, 23. 6. 1909,
                        Abendblatt, S. 5. Darin wird vom neuen Vertrag von Josef Kainz\pwindex{Kainz, Josef 02.01.1858 – 20.09.1910@\textsc{Kainz, Josef} (02.01.1858 – 20.09.1910), \emph{Schauspieler}|pwk} mit
                     dem \emph{Burgtheater}\orgindex{Burgtheater@Burgtheater|pwk} berichtet und kolportiert, Kainz\pwindex{Kainz, Josef 02.01.1858 – 20.09.1910@\textsc{Kainz, Josef} (02.01.1858 – 20.09.1910), \emph{Schauspieler}|pwk} wäre
                  nurmehr zwei Monate im Semester in Wien\oindex{Wien@\textbf{Wien}|pwk} und löse deshalb seinen Haushalt auf.}}}\label{K_L03501-1h} dürfte nicht ganz stimmen. Auch scheint es
               mir nicht wahrscheinlich, dass Kainz\pwindex{Kainz, Josef 02.01.1858 – 20.09.1910@\textsc{Kainz, Josef} (02.01.1858 – 20.09.1910), \emph{Schauspieler}|pw}, müde wie
               er jetzt ist, sich vor den Ferien mit der »Auflösung des Hausstandes« befaſsen wird.
               Es müſste denn inzwischen seine Frau\pwindex{Kainz, Margarethe 13.12.1858 – 12.02.1950@\textsc{Kainz, Margarethe} (13.12.1858 – 12.02.1950), \emph{Schauspielerin}|pwv} irgend etwas veranlaſst haben. Aber auch das halte ich nicht für
               wahrscheinlich. Sollte es dennoch der Fall sein, dann bezieht es sich wol nur auf den
               Termin, wann die Wohnung geräumt wird. Wenn Sie wollen, frage ich direkt bei Frau Kainz\pwindex{Kainz, Margarethe 13.12.1858 – 12.02.1950@\textsc{Kainz, Margarethe} (13.12.1858 – 12.02.1950), \emph{Schauspielerin}|pw} an. Sicherlich wird sie mir dann gegen den
               7. od. 8. Juli Bescheid geben, sobald sie mit ihm zusammentrifft. Oder Sie schreiben
               ihm ein paar Zeilen. Ich bleibe jetzt voraussichtlich bis 15. Juli ununterbrochen
               hier. In Venedig\oindex{Venedig@\textbf{Venedig}|pw} war es sehr schön, und den Lido\oindex{Lido@\textbf{Lido}|pw} fanden wir in allen Verhältnissen, Strand,
               Bad, Capanne, ec. um so viel komfortabler, dass wir nächstes Jahr wol hingehen
               werden, falls wir wieder ans Meer wollen. Den Kindern\pwindex{Salten, Paul 11.08.1903 – 08.05.1937@\textsc{Salten, Paul} (11.08.1903 – 08.05.1937), \emph{Filmcutter}|pwv}\pwindex{Rehmann, Anna Katharina 18.08.1904 – 27.03.1977@\textsc{Rehmann, Anna Katharina} (18.08.1904 – 27.03.1977), \emph{Schauspielerin}|pwv} ist hier bis jetzt und
               unberufen sehr wol. Sie haben nichts von den kleinen Übeln bekommen, die für
               gefährlich profezeiht werden. Ich hatte den Sonnenbrand und Fieber, aber das Fieber
               war von der Erkältung, die ich mit her brachte. Und jetzt ist auch das längst vorbei.
               Ich häute mich nur an Nase, Armen und Beinen wie ein Molch. Alles Gute für Sie, Frau
                  Olga\pwindex{Schnitzler, Olga 17.01.1882 – 13.01.1970@\textsc{Schnitzler, Olga} (17.01.1882 – 13.01.1970), \emph{Schauspielerin, Sängerin}|pw} u. Heini\pwindex{Schnitzler, Heinrich 09.08.1902 – 12.07.1982@\textsc{Schnitzler, Heinrich} (09.08.1902 – 12.07.1982), \emph{Regisseur, Schauspieler}|pw}! {\\}Viele herzliche Grüße von uns zu Ihnen\pend
           \pstart Ihr \spacefill\mbox{Salten}\pend{}
         
         \endnumbering\mylabel{h}\end{ledgroupsized}\begin{anhang}\end{anhang}\newcommand{\dateiname}{L03501}\newcommand{\titel}{Felix Salten an Arthur Schnitzler, 29. 6. 1909}\newcommand{\editorInnen}{Martin Anton Müller und Laura Untner}%% latex-leseansicht-abspann.tex
%% Abspann für die Leseansicht.
%% Der Schalter \ifkorrekturansicht ist bereits durch den Vorspann gesetzt.

%% latex-abspann.tex
%% Gemeinsamer Abspann für Korrekturansicht und Leseansicht.
%% Setzt den Schalter \ifkorrekturansicht voraus (gesetzt in den
%% einbindenden Dateien latex-korrekturansicht-abspann.tex bzw.
%% latex-leseansicht-abspann.tex).
%% ---------------------------------------------------------------

\normalsize

% Das esempio-Environment wird nur in der Leseansicht benötigt
\ifkorrekturansicht\else
\newenvironment{esempio}[3]%
{
    \vspace{1.5ex}
    \rlap{\underline{#1}}
    \par
    \setlength{\parindent}{0cm}
    \nopagebreak
    \leftskip=#2cm
    \rightskip=#3cm
}
{
    \par
}
\fi

\doendnotes{C}
\bigskip
\vfill

\clearpage

\footnotesize

\ifkorrekturansicht
  \lohead{\textsc{register}}
\fi

% theindex-Environment neu definieren ohne reledmac
\makeatletter
\renewenvironment{theindex}{%
  \ifkorrekturansicht
    \section*{\indexname}%
  \else
    \subsubsection*{Index der erwähnten Entitäten}%
  \fi
  \setlength{\parindent}{0pt}%
  \setlength{\parskip}{0pt plus 0.3pt}%
  \let\item\@idxitem
}{%
  \ifkorrekturansicht\clearpage\fi
}
\makeatother

\IfFileExists{\jobname-pw.ind}{\input{\jobname-pw.ind}}{}

% Quellenangabe nur in der Leseansicht
\ifkorrekturansicht\else
% Fallback-Definitionen, falls die .tex-Datei \titel etc. nicht gesetzt hat
\providecommand{\titel}{}
\providecommand{\editorInnen}{}
\providecommand{\dateiname}{\jobname}

\vspace{3cm}

\vfill

\footnotesize
\textsc{Quelle}: \titel. Herausgegeben von {\editorInnen}. In: \emph{Arthur Schnitzler: Briefwechsel mit Autorinnen und Autoren}.
 Digitale Edition, https://schnitzler-briefe.acdh.oeaw.ac.at/{\dateiname}.html (Stand \today)
\fi

\end{document}


      