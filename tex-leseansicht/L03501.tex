%% latex-korrekturansicht-vorspann.tex
%% Vorspann für die Korrekturansicht.
%% Lädt die gemeinsame Datei latex-vorspann.tex mit gesetztem Schalter.

\newif\ifkorrekturansicht
\korrekturansichttrue

\input{../tex-inputs/latex-vorspann}


\section[ Felix Salten an Arthur Schnitzler, 29. 6. 1909]{L03501 Felix Salten an Arthur Schnitzler, 29. 6. 1909}
\nopagebreak\mylabel{L03501v}
\rehead{ }\normalsize\beginnumbering\briefempfaengerindex{Schnitzler, Arthur@\textsc{Schnitzler, Arthur}!zzzSalten, Felix@\emph{von Felix Salten}!1909-06-291@{29. 6. 1909}|(be}
\toendnotes[C]{\smallbreak\pagebreak[2]}\Standort{CUL, Schnitzler, B 89, B 1.}
\physDesc{Brief, 1 Blatt, 1 Seite, 1555 Zeichen
\newline{}Handschrift: schwarze Tinte, lateinische Kurrent
\newline{}Schnitzler: mit Bleistift Vermerk: »\textsc{Salten}« 
\newline{}Ordnung: mit Bleistift von unbekannter Hand nummeriert: »251« }\toendnotes[C]{\smallbreak}
\pstart
           {\pb}\textcolor{gray}{\textbf{\textsc{Villa Bauer}}}\oindex{Villa Bauer@\textbf{Villa Bauer}, \emph{Wohngebäude (K.WHS)}|pw}\hfill \textcolor{gray}{\textbf{\textsc{Grado}}}\oindex{Grado@\textbf{Grado}, \emph{P.PPLA3}|pw}\pend
           
\pstart
           \raggedleft{}\textcolor{gray}{\textbf{\textsc{Küstenland}}}\pend
           
\pstart
           \raggedleft{}29. VI. 09\pend
           
\pstart{}Lieber,\pend\vspace{0.5em}
\pstart
           wir sind heute aus Venedig\oindex{Venedig@\textbf{Venedig}, \emph{P.PPLA}|pw} zurückgekommen, und ich finde Ihren Brief vom 22. Das letzte, was mir Kainz\pwindex{Kainz, Josef 02.01.1858 – 20.09.1910@\textsc{Kainz, Josef} (02.01.1858 – 20.09.1910), \emph{Schauspieler/Schauspielerin}|pw} sagte, war etwa zwei Tage vor meiner Abreise; und da meinte er, er
               wolle es seiner Frau\pwindex{Kainz, Margarethe 13.12.1858 – 12.02.1950@\textsc{Kainz, Margarethe} (13.12.1858 – 12.02.1950), \emph{Schauspieler/Schauspielerin}|pwv}
               überlaßen, im Dezember darüber zu verfügen. Deshalb
               glaube ich, die \label{K_L03501-1v}\edtext{Zeitungsnotiz\pwindex{neue Vertrag Josef Kainz @\emph{Der neue Vertrag Josef Kainz’}|pwv}}{\lemma{\textnormal{\emph{Zeitungsnotiz}}}\Cendnote{\textnormal{In den Tagen rund um Schnitzlers (nicht überlieferten) Brief an Salten\pwindex{Salten, Felix 06.09.1869 – 08.10.1945@\textsc{Salten, Felix} (06.09.1869 – 08.10.1945), \emph{Schriftsteller/Schriftstellerin, Journalist/Journalistin, Chefredakteur/Chefredakteurin}|pwk} vom 22. 6. 1909 waren die
                  Vertragsverhandlungen des \emph{Burgtheaters}\orgindex{Burgtheater@Burgtheater|pwk} mit Josef Kainz\pwindex{Kainz, Josef 02.01.1858 – 20.09.1910@\textsc{Kainz, Josef} (02.01.1858 – 20.09.1910), \emph{Schauspieler/Schauspielerin}|pwk} in den Wien\oindex{Wien@\textbf{Wien}, \emph{A.ADM2}|pwk}er Tageszeitungen ein großes Thema. Wenngleich unklar
                  ist, auf welche Meldung sich Schnitzler
                  genau bezieht, dürfte es inhaltlich dieser entsprechen, die aber erst nach dem
                  Brief an Salten\pwindex{Salten, Felix 06.09.1869 – 08.10.1945@\textsc{Salten, Felix} (06.09.1869 – 08.10.1945), \emph{Schriftsteller/Schriftstellerin, Journalist/Journalistin, Chefredakteur/Chefredakteurin}|pwk} erschienen ist: [O. V.]: \emph{Der neue Vertrag Josef Kainz’}\pwindex{neue Vertrag Josef Kainz @\emph{Der neue Vertrag Josef Kainz’}|pwk}. In: \emph{Neue Freie Presse}\pwindex{Neue Freie Presse@\emph{Neue Freie Presse}|pwk}, Nr. 16.105, 23. 6. 1909, Abendblatt, S. 5. Darin wird
                  vom neuen Vertrag von Josef Kainz\pwindex{Kainz, Josef 02.01.1858 – 20.09.1910@\textsc{Kainz, Josef} (02.01.1858 – 20.09.1910), \emph{Schauspieler/Schauspielerin}|pwk} mit dem
                     \emph{Burgtheater}\orgindex{Burgtheater@Burgtheater|pwk} berichtet und kolportiert, Kainz\pwindex{Kainz, Josef 02.01.1858 – 20.09.1910@\textsc{Kainz, Josef} (02.01.1858 – 20.09.1910), \emph{Schauspieler/Schauspielerin}|pwk} sei nur noch zwei Monate im Semester
                  in Wien\oindex{Wien@\textbf{Wien}, \emph{A.ADM2}|pwk} und löse deshalb seinen Haushalt auf.
                     Schnitzler hatte Interesse an dieser
                  Wohnung, siehe A. S.: \emph{Tagebuch}, 23. 6. 1909.}}}\label{K_L03501-1} dürfte nicht ganz stimmen. Auch scheint es mir nicht wahrscheinlich,
               dass Kainz\pwindex{Kainz, Josef 02.01.1858 – 20.09.1910@\textsc{Kainz, Josef} (02.01.1858 – 20.09.1910), \emph{Schauspieler/Schauspielerin}|pw}, müde wie er jetzt ist, sich vor
               den Ferien mit der »Auflösung des Hausstandes« befaßen wird. Es müßte denn inzwischen
               seine Frau\pwindex{Kainz, Margarethe 13.12.1858 – 12.02.1950@\textsc{Kainz, Margarethe} (13.12.1858 – 12.02.1950), \emph{Schauspieler/Schauspielerin}|pwv} irgend etwas
               veranlaßt haben. Aber auch das halte ich nicht für wahrscheinlich. Sollte es dennoch
               der Fall sein, dann bezieht es sich wol nur auf den Termin, wann die Wohnung geräumt
               wird. Wenn Sie wollen, frage ich direkt bei Frau Kainz\pwindex{Kainz, Margarethe 13.12.1858 – 12.02.1950@\textsc{Kainz, Margarethe} (13.12.1858 – 12.02.1950), \emph{Schauspieler/Schauspielerin}|pw} an. Sicherlich wird sie mir dann gegen den 7. od. 8. Juli Bescheid geben, sobald sie
               mit ihm zusammentrifft. Oder Sie schreiben ihm ein paar Zeilen. Ich bleibe jetzt
               voraussichtlich bis 15. Juli ununterbrochen hier\oindex{Grado@\textbf{Grado}, \emph{P.PPLA3}|pwv}. In Venedig\oindex{Venedig@\textbf{Venedig}, \emph{P.PPLA}|pw} war es sehr schön, und den Lido\oindex{Lido@\textbf{Lido}, \emph{P.PPL}|pw} fanden wir in allen Verhältnissen, Strand, Bad, Capanne,
               ec. um so viel komfortabler, dass wir nächstes Jahr wol hingehen werden, falls wir
               wieder ans Meer wollen. Den Kinder\pwindex{Salten, Paul 11.08.1903 – 08.05.1937@\textsc{Salten, Paul} (11.08.1903 – 08.05.1937), \emph{Filmcutter/Filmcutterin}|pwv}\pwindex{Rehmann, Anna Katharina 18.08.1904 – 27.03.1977@\textsc{Rehmann, Anna Katharina} (18.08.1904 – 27.03.1977), \emph{Schauspieler/Schauspielerin, Übersetzer/Übersetzerin}|pwv}n ist hier bis jetzt und unberufen sehr wol. Sie
               haben nichts von den kleinen Übeln bekommen, die für gefährlich profezeiht werden.
               Ich hatte den Sonnenbrand und Fieber, aber das Fieber war von der Erkältung, die ich
               mit her brachte. Und jetzt ist auch das längst vorbei. Ich häute mich nur an Nase,
               Armen und Beinen wie ein Molch. Alles Gute für Sie, Frau Olga\pwindex{Schnitzler, Olga 17.01.1882 – 13.01.1970@\textsc{Schnitzler, Olga} (17.01.1882 – 13.01.1970), \emph{Schauspieler/Schauspielerin, Sänger/Sängerin}|pw} u. Heini\pwindex{Schnitzler, Heinrich 09.08.1902 – 12.07.1982@\textsc{Schnitzler, Heinrich} (09.08.1902 – 12.07.1982), \emph{Regisseur/Regisseurin, Schauspieler/Schauspielerin}|pw}!\pend
           
\pstart
           Viele herzliche Grüße von uns zu Ihnen {\\[\baselineskip]}Ihr \spacefill\mbox{Salten}\pend
           \leftskip=0em{}\selectlanguage{ngerman}\endnumbering\briefempfaengerindex{Schnitzler, Arthur@\textsc{Schnitzler, Arthur}!zzzSalten, Felix@\emph{von Felix Salten}!1909-06-291@{29. 6. 1909}|)be}\mylabel{L03501h}  \normalsize

\doendnotes{C}
\bigskip
\vfill

\clearpage

\footnotesize

\lohead{\textsc{register}}

% Definiere theindex-Environment komplett neu ohne reledmac
\makeatletter
\renewenvironment{theindex}{%
  \section*{\indexname}%
  \setlength{\parindent}{0pt}%
  \setlength{\parskip}{0pt plus 0.3pt}%
  \let\item\@idxitem
}{%
  \clearpage
}
\makeatother

\IfFileExists{\jobname-pw.ind}{\input{\jobname-pw.ind}}{}

\end{document}

      