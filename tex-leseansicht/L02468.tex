%% latex-leseansicht-vorspann.tex
%% Vorspann für die Leseansicht.
%% Lädt die gemeinsame Datei latex-vorspann.tex mit nicht gesetztem Schalter.

\newif\ifkorrekturansicht
\korrekturansichtfalse

\input{../tex-inputs/latex-vorspann}


         
         \renewcommand{\erwaehntePersonen}{Personen: Johann Wolfgang von Goethe}
         \renewcommand{\erwaehnteOrte}{Orte: Badisches Staatstheater, Karlsruhe, Wien}
         \renewcommand{\erwaehnteWerke}{Werke: Der Gang zum Weiher. Dramatische Dichtung, Der einsame Weg. Schauspiel in fünf Akten, Komödie der Verführung. In drei Akten, Tantalos}
               \section[Felix Braun an Arthur Schnitzler, 15. 3. 1926]{ Felix Braun an Arthur Schnitzler, 15. 3. 1926}\nopagebreak\mylabel{v}\rehead{ }\begin{ledgroupsized}[t]{13cm}\normalsize\beginnumbering \toendnotes[C]{\smallbreak\pagebreak[2]} \Standort{DLA, A:Schnitzler, HS.NZ85.1.2604,7.}
\physDesc{Brief, 2 Blätter, 4 Seiten
\newline{}Handschrift: schwarze Tinte, deutsche Kurrent
\newline{}Schnitzler: 1) mit Bleistift beschriftet: »\strikeout{\textcolor{gray}{×}\-\textcolor{gray}{×}}{ }\textsc{Braun}«  2) mit rotem Buntstift mehrere Unterstreichungen}\toendnotes[C]{\smallbreak}\pstart
           \centering{}{\pb}Wien\oindex{Wien@\textbf{Wien}|pw}, den 15. III. 26\pend
           \pstart{}Verehrter Herr Doktor!\pend\pstart
           Tief ergriffen und bewegt hat mich Ihr »Gang zum
                        Weiher\pwindex{Schnitzler, Arthur 15.05.1862 – 21.10.1931@\textsc{Schnitzler, Arthur} (15.05.1862 – 21.10.1931), \emph{Schriftsteller, Mediziner}!Gang zum Weiher. Dramatische Dichtung1926@\strich\emph{Der Gang zum Weiher. Dramatische Dichtung} {[}1926{]}|pw}« und nicht nur dieſe Wirkung, zu der die ſchönſte äſthetiſche
                    tritt, auch eine innerſt-perſönliche fühle ich auf mich ausgeübt, Antwort auf
                    manche Frage, Qual und Furcht gegeben, und ſo kann ich nur ſagen, daß ich Ihnen
                    für dieſe Dichtung als Leſer, als Schriftſteller und nicht zuletzt als Menſch
                    aufs Innigſte verbunden bin.\pend
           \pstart
           Es iſt eine Dichtung der Weisheit und der ſpäten Einſamkeit, von der die Jugend,
                    die Einſamkeit ſo leidenſchaftlich ſucht, nichts {\pb}ahnt. Wie ſchon im »Einſamen Weg\pwindex{Schnitzler, Arthur 15.05.1862 – 21.10.1931@\textsc{Schnitzler, Arthur} (15.05.1862 – 21.10.1931), \emph{Schriftsteller, Mediziner}!einsame Weg. Schauspiel in fuenf Akten1904@\strich\emph{Der einsame Weg. Schauspiel in fünf Akten} {[}1904{]}|pw}« und
                    neuerdings in der »Komoedie der Verführung\pwindex{Schnitzler, Arthur 15.05.1862 – 21.10.1931@\textsc{Schnitzler, Arthur} (15.05.1862 – 21.10.1931), \emph{Schriftsteller, Mediziner}!Komoedie der Verfuehrung. In drei Akten1924@\strich\emph{Komödie der Verführung. In drei Akten} {[}1924{]}|pw}«
                    iſt hier Einſamkeitsluft um die Geſtalten von Männern, die aus der Jugend
                    getreten ſind. Das iſt ſehr erregend und ergreifend. \uline{Dieſe} Tragoedie des Mannes haben Sie wohl als Erſter gedichtet. Und
                    dies Älterwerden beginnt vielleicht weit früher, als es ſich Jugend träumen
                    läßt. Das Erbarmungsloſe, das in ſolchem Kampf jeden, aber auch jeden Vorzug zu
                    nichte macht, iſt noch nie ſo erkannt, ſo gewieſen worden.\pend
           \pstart
           Schön ſind die Verſe, Ihre ſchönſten bisher. Dieſelbe hohe, klare Luft ſchwebt
                    über ihnen. Ein Goethe\pwindex{Goethe, Johann Wolfgang von 1749-08-28 – 1832-03-22@\textsc{Goethe, Johann Wolfgang von} (1749-08-28 – 1832-03-22), \emph{Schriftsteller}|pw}’ſcher Hauch,
                    überhaupt Atem unſerer klaſſiſchen Dramendichtung {\pb}beglückt darin mit. Daß Sie durch das neue Werk\pwindex{Schnitzler, Arthur 15.05.1862 – 21.10.1931@\textsc{Schnitzler, Arthur} (15.05.1862 – 21.10.1931), \emph{Schriftsteller, Mediziner}!Gang zum Weiher. Dramatische Dichtung1926@\strich\emph{Der Gang zum Weiher. Dramatische Dichtung} {[}1926{]}|pwv} an unſere große Tradition anſchließen, iſt mir
                    beſonders, der ich ich mich immer darum bemüht habe, erwünſcht und wertvoll.\pend
           \pstart
           Nicht ganz überzeugend finde ich die Geſtalt des Mädchens\pwindex{Schnitzler, Arthur 15.05.1862 – 21.10.1931@\textsc{Schnitzler, Arthur} (15.05.1862 – 21.10.1931), \emph{Schriftsteller, Mediziner}!Gang zum Weiher. Dramatische Dichtung1926@\strich\emph{Der Gang zum Weiher. Dramatische Dichtung} {[}1926{]}|pwv}. Soll ſie nur eine Idee ſein? Die der
                    Jugend? Die des weiblichen Naturweſens? Sie verſagt nach meinem Gefühl ſowohl
                    gegen Konrad\pwindex{Schnitzler, Arthur 15.05.1862 – 21.10.1931@\textsc{Schnitzler, Arthur} (15.05.1862 – 21.10.1931), \emph{Schriftsteller, Mediziner}!Gang zum Weiher. Dramatische Dichtung1926@\strich\emph{Der Gang zum Weiher. Dramatische Dichtung} {[}1926{]}|pwv} wie gegen Sylveſter\pwindex{Schnitzler, Arthur 15.05.1862 – 21.10.1931@\textsc{Schnitzler, Arthur} (15.05.1862 – 21.10.1931), \emph{Schriftsteller, Mediziner}!Gang zum Weiher. Dramatische Dichtung1926@\strich\emph{Der Gang zum Weiher. Dramatische Dichtung} {[}1926{]}|pwv}. Sie iſt nicht
                    weiblich und nicht menſchlich genug. Andererſeits wüßte ich freilich ſelbſt
                    keine beſſere Löſung.\pend
           \pstart
           Ich ſchreibe in Eile, denn ich bin vor der \label{K_L02468_1v}\edtext{Abreiſe}{\lemma{\textnormal{\emph{Abreiſe}}}\Cendnote{\textnormal{Die Uraufführung fand am 27. 3. 1926 im
                            Badischen Landestheater Karlsruhe\oindex{Badisches Staatstheater@\textbf{Badisches Staatstheater}|pwk}
                        statt.}}}\label{K_L02468_1h}: in Karlsruhe\oindex{Karlsruhe@\textbf{Karlsruhe}|pw} wird mein »\textsc{Tantalos}\pwindex{Braun, Felix 04.11.1885 – 29.11.1973@\textsc{Braun, Felix} (04.11.1885 – 29.11.1973), \emph{Schriftsteller}!Tantalos1917@\strich\emph{Tantalos} {[}1917{]}|pw}« geſpielt und ich will bei den {\pb}Proben
                    dabei ſein. Es iſt zum erſten Mal, daß ich das erlebe.\pend
           \pstart
           Seien Sie von Herzen bedankt, verehrter Arthur Schnitzler! Wie glücklich müſſen
                    Sie beim Schreiben dieſes Werks\pwindex{Schnitzler, Arthur 15.05.1862 – 21.10.1931@\textsc{Schnitzler, Arthur} (15.05.1862 – 21.10.1931), \emph{Schriftsteller, Mediziner}!Gang zum Weiher. Dramatische Dichtung1926@\strich\emph{Der Gang zum Weiher. Dramatische Dichtung} {[}1926{]}|pwv} geweſen ſein! Ich halte es für Ihr größtes!\pend
           \pstart
           Wie immer verharrend\hspace*{1.5em}Ihr{\\[\baselineskip]}\spacefill\mbox{Felix Braun.}\pend
           \leftskip=0em{}
         
         \endnumbering\mylabel{h}\end{ledgroupsized}  \newcommand{\dateiname}{L02468}\newcommand{\titel}{Felix Braun an Arthur Schnitzler, 15. 3. 1926}\newcommand{\editorInnen}{Martin Anton Müller und Gerd-Hermann Susen}%% latex-leseansicht-abspann.tex
%% Abspann für die Leseansicht.
%% Der Schalter \ifkorrekturansicht ist bereits durch den Vorspann gesetzt.

%% latex-abspann.tex
%% Gemeinsamer Abspann für Korrekturansicht und Leseansicht.
%% Setzt den Schalter \ifkorrekturansicht voraus (gesetzt in den
%% einbindenden Dateien latex-korrekturansicht-abspann.tex bzw.
%% latex-leseansicht-abspann.tex).
%% ---------------------------------------------------------------

\normalsize

% Das esempio-Environment wird nur in der Leseansicht benötigt
\ifkorrekturansicht\else
\newenvironment{esempio}[3]%
{
    \vspace{1.5ex}
    \rlap{\underline{#1}}
    \par
    \setlength{\parindent}{0cm}
    \nopagebreak
    \leftskip=#2cm
    \rightskip=#3cm
}
{
    \par
}
\fi

\doendnotes{C}
\bigskip
\vfill

\clearpage

\footnotesize

\ifkorrekturansicht
  \lohead{\textsc{register}}
\fi

% theindex-Environment neu definieren ohne reledmac
\makeatletter
\renewenvironment{theindex}{%
  \ifkorrekturansicht
    \section*{\indexname}%
  \else
    \subsubsection*{Index der erwähnten Entitäten}%
  \fi
  \setlength{\parindent}{0pt}%
  \setlength{\parskip}{0pt plus 0.3pt}%
  \let\item\@idxitem
}{%
  \ifkorrekturansicht\clearpage\fi
}
\makeatother

\IfFileExists{\jobname-pw.ind}{\input{\jobname-pw.ind}}{}

% Quellenangabe nur in der Leseansicht
\ifkorrekturansicht\else
% Fallback-Definitionen, falls die .tex-Datei \titel etc. nicht gesetzt hat
\providecommand{\titel}{}
\providecommand{\editorInnen}{}
\providecommand{\dateiname}{\jobname}

\vspace{3cm}

\vfill

\footnotesize
\textsc{Quelle}: \titel. Herausgegeben von {\editorInnen}. In: \emph{Arthur Schnitzler: Briefwechsel mit Autorinnen und Autoren}.
 Digitale Edition, https://schnitzler-briefe.acdh.oeaw.ac.at/{\dateiname}.html (Stand \today)
\fi

\end{document}


      