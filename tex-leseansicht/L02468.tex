%% latex-leseansicht-vorspann.tex
%% Vorspann für die Leseansicht.
%% Lädt die gemeinsame Datei latex-vorspann.tex mit nicht gesetztem Schalter.

\newif\ifkorrekturansicht
\korrekturansichtfalse

\input{../tex-inputs/latex-vorspann}


\section[Felix Braun an Arthur Schnitzler, 15. 3. 1926]{L02468 Felix Braun an Arthur Schnitzler, 15. 3. 1926}
\nopagebreak\mylabel{L02468v}
\rehead{ }\normalsize\beginnumbering\briefempfaengerindex{Schnitzler, Arthur@\textsc{Schnitzler, Arthur}!zzzBraun, Felix@\emph{von Felix Braun}!1926-03-151@{15. 3. 1926}|(be}
\toendnotes[C]{\smallbreak\pagebreak[2]}
\correspDesc{Versand  durch Felix Braun am 15. 3. 1926 in Wien
\newline{}Erhalt  durch Arthur Schnitzler im Zeitraum [15. 3. 1926
                  – 19. 3. 1926?] in Wien}\toendnotes[C]{\smallbreak}
\Standort{DLA, A:Schnitzler, HS.NZ85.1.2604,7.}
\physDesc{Brief, 2 Blätter, 4 Seiten, 2051 Zeichen
\newline{}Handschrift: schwarze Tinte, deutsche Kurrent
\newline{}Schnitzler: 1) mit Bleistift beschriftet: »\strikeout{\textcolor{gray}{×}\-\textcolor{gray}{×}}{ }\textsc{Braun}«  2) mit rotem Buntstift mehrere Unterstreichungen}\toendnotes[C]{\smallbreak}
\pstart
           \centering{}{\pb}Wien\oindex{Wien@\textbf{Wien}, \emph{Verwaltungsgebiet}|pw}, den 15. III. 26\pend
           
\pstart{}Verehrter Herr Doktor!\pend\vspace{0.5em}
\pstart
           Tief ergriffen und bewegt hat mich Ihr »Gang zum
                  Weiher\pwindex{Schnitzler, Arthur 15.\,5.\,1862 Wien – 21.\,10.\,1931 ebd.@\textsc{Schnitzler, Arthur} (15.\,5.\,1862 Wien – 21.\,10.\,1931 ebd.), \emph{Schriftsteller, Mediziner}!Gang zum Weiher. Dramatische Dichtung@\strich\emph{Der Gang zum Weiher. Dramatische Dichtung}|pw}« und nicht nur dieſe Wirkung, zu der die{ }ſchönſte äſthetiſche tritt,
               auch eine innerſt-perſönliche fühle ich auf mich ausgeübt, Antwort auf manche Frage,
               Qual und Furcht gegeben, und{ }ſo kann ich nur{ }ſagen, daß ich Ihnen für dieſe Dichtung
               als Leſer, als Schriftſteller und nicht zuletzt als Menſch aufs Innigſte verbunden
               bin.\pend
           
\pstart
           Es iſt eine Dichtung der Weisheit und der{ }ſpäten Einſamkeit, von der die Jugend, die
               Einſamkeit{ }ſo leidenſchaftlich{ }ſucht, nichts {\pb}ahnt.
               Wie{ }ſchon im »Einſamen Weg\pwindex{Schnitzler, Arthur 15.\,5.\,1862 Wien – 21.\,10.\,1931 ebd.@\textsc{Schnitzler, Arthur} (15.\,5.\,1862 Wien – 21.\,10.\,1931 ebd.), \emph{Schriftsteller, Mediziner}!einsame Weg. Schauspiel in fünf Akten@\strich\emph{Der einsame Weg. Schauspiel in fünf Akten}|pw}« und neuerdings in
               der »Komoedie der Verführung\pwindex{Schnitzler, Arthur 15.\,5.\,1862 Wien – 21.\,10.\,1931 ebd.@\textsc{Schnitzler, Arthur} (15.\,5.\,1862 Wien – 21.\,10.\,1931 ebd.), \emph{Schriftsteller, Mediziner}!Komödie der Verführung. In drei Akten@\strich\emph{Komödie der Verführung. In drei Akten}|pw}« iſt hier
               Einſamkeitsluft um die Geſtalten von Männern, die aus der Jugend getreten{ }ſind. Das
               iſt{ }ſehr erregend und ergreifend. \uline{Dieſe} Tragoedie des
               Mannes haben Sie wohl als Erſter gedichtet. Und dies Älterwerden beginnt vielleicht
               weit früher, als es{ }ſich Jugend träumen läßt. Das Erbarmungsloſe, das in{ }ſolchem
               Kampf jeden, aber auch jeden Vorzug zu nichte macht, iſt noch nie{ }ſo erkannt,{ }ſo
               gewieſen worden.\pend
           
\pstart
           Schön{ }ſind die Verſe, Ihre{ }ſchönſten bisher. Dieſelbe hohe, klare Luft{ }ſchwebt über
               ihnen. Ein Goethe\pwindex{Goethe, Johann Wolfgang von 28.\,8.\,1749 Frankfurt am Main – 22.\,3.\,1832 Weimar@\textsc{Goethe, Johann Wolfgang von} (28.\,8.\,1749 Frankfurt am Main – 22.\,3.\,1832 Weimar), \emph{Schriftsteller}|pw}’ſcher Hauch, überhaupt Atem
               unſerer klaſſiſchen Dramendichtung {\pb}beglückt darin
               mit. Daß Sie durch das neue Werk\pwindex{Schnitzler, Arthur 15.\,5.\,1862 Wien – 21.\,10.\,1931 ebd.@\textsc{Schnitzler, Arthur} (15.\,5.\,1862 Wien – 21.\,10.\,1931 ebd.), \emph{Schriftsteller, Mediziner}!Gang zum Weiher. Dramatische Dichtung@\strich\emph{Der Gang zum Weiher. Dramatische Dichtung}|pwv} an unſere große Tradition anſchließen, iſt mir beſonders, der ich ich
               mich immer darum bemüht habe, erwünſcht und wertvoll.\pend
           
\pstart
           Nicht ganz überzeugend finde ich die Geſtalt des Mädchens\pwindex{Schnitzler, Arthur 15.\,5.\,1862 Wien – 21.\,10.\,1931 ebd.@\textsc{Schnitzler, Arthur} (15.\,5.\,1862 Wien – 21.\,10.\,1931 ebd.), \emph{Schriftsteller, Mediziner}!Gang zum Weiher. Dramatische Dichtung@\strich\emph{Der Gang zum Weiher. Dramatische Dichtung}|pwv}. Soll{ }ſie nur eine Idee{ }ſein? Die der Jugend? Die
               des weiblichen Naturweſens? Sie verſagt nach meinem Gefühl{ }ſowohl gegen Konrad\pwindex{Schnitzler, Arthur 15.\,5.\,1862 Wien – 21.\,10.\,1931 ebd.@\textsc{Schnitzler, Arthur} (15.\,5.\,1862 Wien – 21.\,10.\,1931 ebd.), \emph{Schriftsteller, Mediziner}!Gang zum Weiher. Dramatische Dichtung@\strich\emph{Der Gang zum Weiher. Dramatische Dichtung}|pwv} wie gegen Sylveſter\pwindex{Schnitzler, Arthur 15.\,5.\,1862 Wien – 21.\,10.\,1931 ebd.@\textsc{Schnitzler, Arthur} (15.\,5.\,1862 Wien – 21.\,10.\,1931 ebd.), \emph{Schriftsteller, Mediziner}!Gang zum Weiher. Dramatische Dichtung@\strich\emph{Der Gang zum Weiher. Dramatische Dichtung}|pwv}. Sie iſt nicht
               weiblich und nicht menſchlich genug. Andererſeits wüßte ich freilich{ }ſelbſt keine
               beſſere Löſung.\pend
           
\pstart
           Ich{ }ſchreibe in Eile, denn ich bin vor der \label{K_L02468-1v}\edtext{Abreiſe}{\lemma{\textnormal{\emph{Abreise}}}\Cendnote{\textnormal{Die
                Uraufführung\eventindex{Badisches Staatstheater@\textbf{Badisches Staatstheater}!Uraufführung von Tantalos@Uraufführung von Tantalos|pwkv} fand am 27. 3. 1926 im Badischen
                     Landestheater Karlsruhe\oindex{Badisches Staatstheater@\textbf{Badisches Staatstheater}, \emph{Theater}|pwk} statt.}}}\label{K_L02468-1}: in Karlsruhe\oindex{Karlsruhe@\textbf{Karlsruhe}, \emph{Hauptstadt}|pw} wird mein »\textsc{Tantalos}\pwindex{Braun, Felix 4.\,11.\,1885 Wien – 29.\,11.\,1973 Klosterneuburg@\textsc{Braun, Felix} (4.\,11.\,1885 Wien – 29.\,11.\,1973 Klosterneuburg), \emph{Schriftsteller}!Tantalos@\strich\emph{Tantalos}|pw}« geſpielt und ich will bei den {\pb}Proben dabei{ }ſein. Es iſt zum erſten Mal, daß ich das erlebe.\pend
           
\pstart
           Seien Sie von Herzen bedankt, verehrter Arthur Schnitzler! Wie glücklich müſſen Sie
               beim Schreiben dieſes Werks\pwindex{Schnitzler, Arthur 15.\,5.\,1862 Wien – 21.\,10.\,1931 ebd.@\textsc{Schnitzler, Arthur} (15.\,5.\,1862 Wien – 21.\,10.\,1931 ebd.), \emph{Schriftsteller, Mediziner}!Gang zum Weiher. Dramatische Dichtung@\strich\emph{Der Gang zum Weiher. Dramatische Dichtung}|pwv}
               geweſen{ }ſein! Ich halte es für Ihr größtes!\pend
           
\pstart
           Wie immer verharrend\hspace*{1.5em}Ihr{\\[\baselineskip]}\spacefill\mbox{Felix Braun.}\pend
           \leftskip=0em{}\selectlanguage{ngerman}\endnumbering\briefempfaengerindex{Schnitzler, Arthur@\textsc{Schnitzler, Arthur}!zzzBraun, Felix@\emph{von Felix Braun}!1926-03-151@{15. 3. 1926}|)be}\mylabel{L02468h}  \newcommand{\dateiname}{L02468}\newcommand{\titel}{Felix Braun an Arthur Schnitzler, 15. 3. 1926}\newcommand{\editorInnen}{Martin Anton Müller und Gerd-Hermann Susen}%% latex-leseansicht-abspann.tex
%% Abspann für die Leseansicht.
%% Der Schalter \ifkorrekturansicht ist bereits durch den Vorspann gesetzt.

%% latex-abspann.tex
%% Gemeinsamer Abspann für Korrekturansicht und Leseansicht.
%% Setzt den Schalter \ifkorrekturansicht voraus (gesetzt in den
%% einbindenden Dateien latex-korrekturansicht-abspann.tex bzw.
%% latex-leseansicht-abspann.tex).
%% ---------------------------------------------------------------

\normalsize

% Das esempio-Environment wird nur in der Leseansicht benötigt
\ifkorrekturansicht\else
\newenvironment{esempio}[3]%
{
    \vspace{1.5ex}
    \rlap{\underline{#1}}
    \par
    \setlength{\parindent}{0cm}
    \nopagebreak
    \leftskip=#2cm
    \rightskip=#3cm
}
{
    \par
}
\fi

\doendnotes{C}
\bigskip
\vfill

\clearpage

\footnotesize

\ifkorrekturansicht
  \lohead{\textsc{register}}
\fi

% theindex-Environment neu definieren ohne reledmac
\makeatletter
\renewenvironment{theindex}{%
  \ifkorrekturansicht
    \section*{\indexname}%
  \else
    \subsubsection*{Index der erwähnten Entitäten}%
  \fi
  \setlength{\parindent}{0pt}%
  \setlength{\parskip}{0pt plus 0.3pt}%
  \let\item\@idxitem
}{%
  \ifkorrekturansicht\clearpage\fi
}
\makeatother

\IfFileExists{\jobname-pw.ind}{\input{\jobname-pw.ind}}{}

% Quellenangabe nur in der Leseansicht
\ifkorrekturansicht\else
% Fallback-Definitionen, falls die .tex-Datei \titel etc. nicht gesetzt hat
\providecommand{\titel}{}
\providecommand{\editorInnen}{}
\providecommand{\dateiname}{\jobname}

\vspace{3cm}

\vfill

\footnotesize
\textsc{Quelle}: \titel. Herausgegeben von {\editorInnen}. In: \emph{Arthur Schnitzler: Briefwechsel mit Autorinnen und Autoren}.
 Digitale Edition, https://schnitzler-briefe.acdh.oeaw.ac.at/{\dateiname}.html (Stand \today)
\fi

\end{document}


