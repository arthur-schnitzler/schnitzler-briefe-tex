%% latex-korrekturansicht-vorspann.tex
%% Vorspann für die Korrekturansicht.
%% Lädt die gemeinsame Datei latex-vorspann.tex mit gesetztem Schalter.

\newif\ifkorrekturansicht
\korrekturansichttrue

\input{../tex-inputs/latex-vorspann}


\section[Robert Adam an Arthur Schnitzler, 9. 2. 1911]{L02009 Robert Adam an Arthur Schnitzler, 9. 2. 1911}
\nopagebreak\mylabel{L02009v}
\rehead{ }\normalsize\beginnumbering\briefempfaengerindex{Schnitzler, Arthur@\textsc{Schnitzler, Arthur}!zzzAdam, Robert@\emph{von Robert Adam}!1911-02-092@{9. 2. 1911}|(be}
\toendnotes[C]{\smallbreak\pagebreak[2]}\Standort{DLA, A:Schnitzler, HS.NZ85.1.4230,4.}
\physDesc{Brief, 1 Blatt, 3 Seiten, 1331 Zeichen
\newline{}Handschrift: schwarze Tinte, deutsche Kurrent
\newline{}Schnitzler: 1) mit Bleistift beschriftet: »\textsc{Adam}«  2) mit rotem Buntstift eine Unterstreichung}\Standort{Wien, Österreichische Nationalbibliothek, Cod.ser. 52.266, 82.}
\physDesc{handschriftliche Abschrift1 Blatt, 1 Seite, 1331 Zeichen
\newline{}Handschrift: schwarze Tinte, Gabelsberger Kurzschrift}\Standort{Wien, Österreichische Nationalbibliothek, Cod.ser. 52.266, 82.}
\physDesc{maschinenschriftliche Abschrift1 Blatt, 1 Seite, 1331 Zeichen
\newline{}Schreibmaschine}\toendnotes[C]{\smallbreak}
\pstart
           \raggedleft{}{\pb}Wien\oindex{Wien@\textbf{Wien}, \emph{A.ADM2}|pw}, am 9. Februar 1911\pend
           
\pstart{}Hochverehrter Herr Doktor!\pend\vspace{0.5em}
\pstart
           Um dieſen Brief zu entſchuldigen, möchte ich zwei Verſe aus »Neidhard\pwindex{Neidhard@\emph{Neidhard}|pw}« an die Spitze ſetzen: »Kein gröberes Geſchäft auf
               Erden, – als einen Poeten loszuwerden.« Daß ich Ihnen wieder, und gar ſo raſch wieder
               ſchreibe, iſt nämlich, ſcheint es mir, ſchon ein Akt der Zudringlichkeit; und doch
               wollte ich nur alles in der Welt nicht, daß Sie, hochverehrter Herr Doktor, mich für
               zudringlich hielten. Ich weiß ſehr wohl, daß Sie Wichtigeres zu tun haben, als ſich
               bloß um das Schickſal meiner Komödie\pwindex{Neidhard@\emph{Neidhard}|pwv} zu beküm{\pb}mern (bei mir ſelber iſt’s
               leider damit auch nicht viel anders beſtellt.)\pend
           
\pstart
           Wenn ich Ihnen ſchreibe, geſchieht es nur, weil ich jetzt abſolut nicht weiß, was ich
               mit dieſem »Neidhard\pwindex{Neidhard@\emph{Neidhard}|pw}« anfangen ſoll. Soll ich
               ihn einem andern Verlag zuſenden? und welchem? oder ſoll ich nun den verzweifelten
               Verſuch unternehmen, einzelne Zeitſchriften mit meinem Helden bekanntzumachen?\pend
           
\pstart
           Sie waren ſo gütig, hochverehrter Herr Doktor, mir nach Fehlſchlagen des Fiſcher\orgindex{S. Fischer Verlag@S. Fischer Verlag|pw}’ſchen Verſuchs die Erteilung weiterer
               Ratſchläge in Ausſicht zu ſtellen. Verzeihen Sie mir nun, daß ich Sie neuerlich
               quäle: aber wahrhaftig, ich weiß mir nicht zu raten noch zu helfen.\pend
           
\pstart
           Bitte, helfen Sie mir den Karren noch ein bischen weiter ſchleppen! und ſeien Sie
               meiner Dankbarkeit und {\pb}Verehrung verſichert!\pend
           
\pstart
           Ihr ergebener{\\[\baselineskip]}\spacefill\mbox{Robert Adam}\pend
           \leftskip=0em{}\selectlanguage{ngerman}\endnumbering\briefempfaengerindex{Schnitzler, Arthur@\textsc{Schnitzler, Arthur}!zzzAdam, Robert@\emph{von Robert Adam}!1911-02-092@{9. 2. 1911}|)be}\mylabel{L02009h}  \normalsize

\doendnotes{C}
\bigskip
\vfill

\clearpage

\footnotesize

\lohead{\textsc{register}}

% Definiere theindex-Environment komplett neu ohne reledmac
\makeatletter
\renewenvironment{theindex}{%
  \section*{\indexname}%
  \setlength{\parindent}{0pt}%
  \setlength{\parskip}{0pt plus 0.3pt}%
  \let\item\@idxitem
}{%
  \clearpage
}
\makeatother

\IfFileExists{\jobname-pw.ind}{\input{\jobname-pw.ind}}{}

\end{document}

      