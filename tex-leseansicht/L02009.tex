%% latex-leseansicht-vorspann.tex
%% Vorspann für die Leseansicht.
%% Lädt die gemeinsame Datei latex-vorspann.tex mit nicht gesetztem Schalter.

\newif\ifkorrekturansicht
\korrekturansichtfalse

\input{../tex-inputs/latex-vorspann}


         
         \newcommand{\erwaehntePersonen}{Personen: }
         \newcommand{\erwaehnteInstitutionen}{Institutionen: S. Fischer Verlag}
         \newcommand{\erwaehnteOrte}{Orte: Wien}
         \newcommand{\erwaehnteWerke}{Werke: Neidhard}
               \section[Robert Adam an Arthur Schnitzler, 9. 2. 1911]{ Robert Adam an Arthur Schnitzler, 9. 2. 1911}\nopagebreak\mylabel{v}\rehead{ }\begin{ledgroupsized}[t]{13cm}\normalsize\beginnumbering \toendnotes[C]{\smallbreak\pagebreak[2]} \Standort{DLA, A:Schnitzler, HS.NZ85.1.4230,4.}
\physDesc{Brief, 1 Blatt, 3 Seiten
\newline{}Handschrift: schwarze Tinte, deutsche Kurrent
\newline{}Schnitzler: 1) mit Bleistift beschriftet: »\textsc{Adam}«  2) mit rotem Buntstift eine Unterstreichung}\Standort{Wien, Österreichische Nationalbibliothek, Cod.ser. 52.266, 82.}
\physDesc{handschriftliche Abschrift
\newline{}Handschrift: schwarze Tinte, Gabelsberger Kurzschrift}\Standort{Wien, Österreichische Nationalbibliothek, Cod.ser. 52.266, 82.}
\physDesc{maschinelle Abschrift
\newline{}Schreibmaschine}\toendnotes[C]{\smallbreak}\pstart
           \raggedleft{}{\pb}Wien\oindex{Wien@\textbf{Wien}|pw}, am 9. Februar 1911\pend
           \pstart{}Hochverehrter Herr Doktor!\pend\pstart
           Um dieſen Brief zu entſchuldigen, möchte ich zwei Verſe aus »Neidhard\pwindex{Adam, Robert 20.04.1877 – 16.10.1961@\textsc{Adam, Robert} (20.04.1877 – 16.10.1961), \emph{Schriftsteller, Richter}!NeidhardNone@\strich\emph{Neidhard} {[}None{]}|pw}« an die Spitze ſetzen: »Kein gröberes Geſchäft auf
               Erden, – als einen Poeten loszuwerden.« Daß ich Ihnen wieder, und gar ſo raſch wieder
               ſchreibe, iſt nämlich, ſcheint es mir, ſchon ein Akt der Zudringlichkeit; und doch
               wollte ich nur alles in der Welt nicht, daß Sie, hochverehrter Herr Doktor, mich für
               zudringlich hielten. Ich weiß ſehr wohl, daß Sie Wichtigeres zu tun haben, als ſich
               bloß um das Schickſal meiner Komödie\pwindex{Adam, Robert 20.04.1877 – 16.10.1961@\textsc{Adam, Robert} (20.04.1877 – 16.10.1961), \emph{Schriftsteller, Richter}!NeidhardNone@\strich\emph{Neidhard} {[}None{]}|pwv} zu beküm{\pb}mern (bei mir ſelber iſt’s
               leider damit auch nicht viel anders beſtellt.)\pend
           \pstart
           Wenn ich Ihnen ſchreibe, geſchieht es nur, weil ich jetzt abſolut nicht weiß, was ich
               mit dieſem »Neidhard\pwindex{Adam, Robert 20.04.1877 – 16.10.1961@\textsc{Adam, Robert} (20.04.1877 – 16.10.1961), \emph{Schriftsteller, Richter}!NeidhardNone@\strich\emph{Neidhard} {[}None{]}|pw}« anfangen ſoll. Soll ich ihn
               einem andern Verlag zuſenden? und welchem? oder ſoll ich nun den verzweifelten
               Verſuch unternehmen, einzelne Zeitſchriften mit meinem Helden bekanntzumachen?\pend
           \pstart
           Sie waren ſo gütig, hochverehrter Herr Doktor, mir nach Fehlſchlagen des Fiſcher\orgindex{S. Fischer Verlag@S. Fischer Verlag|pw}’ſchen Verſuchs die Erteilung weiterer
               Ratſchläge in Ausſicht zu ſtellen. Verzeihen Sie mir nun, daß ich Sie neuerlich
               quäle: aber wahrhaftig, ich weiß mir nicht zu raten noch zu helfen.\pend
           \pstart
           Bitte, helfen Sie mir den Karren noch ein bischen weiter ſchleppen! und ſeien Sie
               meiner Dankbarkeit und {\pb}Verehrung verſichert!\pend
           \pstart
           Ihr ergebener{\\[\baselineskip]}\spacefill\mbox{Robert Adam}\pend
           \leftskip=0em{}
         
         \endnumbering\mylabel{h}\end{ledgroupsized}  \newcommand{\dateiname}{L02009}\newcommand{\titel}{Robert Adam an Arthur Schnitzler, 9. 2. 1911}\newcommand{\editorInnen}{Martin Anton Müller und Gerd-Hermann Susen}%% latex-leseansicht-abspann.tex
%% Abspann für die Leseansicht.
%% Der Schalter \ifkorrekturansicht ist bereits durch den Vorspann gesetzt.

%% latex-abspann.tex
%% Gemeinsamer Abspann für Korrekturansicht und Leseansicht.
%% Setzt den Schalter \ifkorrekturansicht voraus (gesetzt in den
%% einbindenden Dateien latex-korrekturansicht-abspann.tex bzw.
%% latex-leseansicht-abspann.tex).
%% ---------------------------------------------------------------

\normalsize

% Das esempio-Environment wird nur in der Leseansicht benötigt
\ifkorrekturansicht\else
\newenvironment{esempio}[3]%
{
    \vspace{1.5ex}
    \rlap{\underline{#1}}
    \par
    \setlength{\parindent}{0cm}
    \nopagebreak
    \leftskip=#2cm
    \rightskip=#3cm
}
{
    \par
}
\fi

\doendnotes{C}
\bigskip
\vfill

\clearpage

\footnotesize

\ifkorrekturansicht
  \lohead{\textsc{register}}
\fi

% theindex-Environment neu definieren ohne reledmac
\makeatletter
\renewenvironment{theindex}{%
  \ifkorrekturansicht
    \section*{\indexname}%
  \else
    \subsubsection*{Index der erwähnten Entitäten}%
  \fi
  \setlength{\parindent}{0pt}%
  \setlength{\parskip}{0pt plus 0.3pt}%
  \let\item\@idxitem
}{%
  \ifkorrekturansicht\clearpage\fi
}
\makeatother

\IfFileExists{\jobname-pw.ind}{\input{\jobname-pw.ind}}{}

% Quellenangabe nur in der Leseansicht
\ifkorrekturansicht\else
% Fallback-Definitionen, falls die .tex-Datei \titel etc. nicht gesetzt hat
\providecommand{\titel}{}
\providecommand{\editorInnen}{}
\providecommand{\dateiname}{\jobname}

\vspace{3cm}

\vfill

\footnotesize
\textsc{Quelle}: \titel. Herausgegeben von {\editorInnen}. In: \emph{Arthur Schnitzler: Briefwechsel mit Autorinnen und Autoren}.
 Digitale Edition, https://schnitzler-briefe.acdh.oeaw.ac.at/{\dateiname}.html (Stand \today)
\fi

\end{document}


      