%% latex-korrekturansicht-vorspann.tex
%% Vorspann für die Korrekturansicht.
%% Lädt die gemeinsame Datei latex-vorspann.tex mit gesetztem Schalter.

\newif\ifkorrekturansicht
\korrekturansichttrue

\input{../tex-inputs/latex-vorspann}


\section[Hugo von Hofmannsthal an Arthur Schnitzler, 25. 8. 1898]{L00838 Hugo von Hofmannsthal an Arthur Schnitzler, 25. 8. 1898}
\nopagebreak\mylabel{L00838v}
\rehead{ }\normalsize\beginnumbering\briefempfaengerindex{Schnitzler, Arthur@\textsc{Schnitzler, Arthur}!zzzHofmannsthal, Hugo von@\emph{von Hugo von Hofmannsthal}!1898-08-251@{25. 8. 1898}|(be}
\toendnotes[C]{\smallbreak\pagebreak[2]}\Standort{CUL, Schnitzler, B 43.}
\physDesc{Postkarte, 393 Zeichen
\newline{}Handschrift: 1) Bleistift, deutsche Kurrent\hspace{1em}2) Bleistift, lateinische Kurrent (\noindent{}Adresse)\hspace{1em}
\newline{}Versand: 1) Stempel: »\nobreak{}\oindex{Lugano@\textbf{Lugano}, \emph{P.PPLA2}|pwk}Lugano, 25. VIII. 98, XII\nobreak{}«.   2) Stempel: »\nobreak{}\oindex{Luzern@\textbf{Luzern}, \emph{P.PPLA}|pwk}Luzern Brf. Dist, 25. VIII. 98, 7\nobreak{}«. 
\newline{}Schnitzler: mit Bleistift datiert: »25/8 98« 
\newline{}Ordnung: 1) mit Bleistift von unbekannter Hand nummeriert: »\strikeout{121}«  2) mit Bleistift von unbekannter Hand nummeriert:
                                    »122«}
\buchAbdrucke{\weitereDrucke{Hugo von Hofmannsthal, Arthur Schnitzler: \emph{Briefwechsel}. Frankfurt am Main: \emph{S. Fischer} 1964, S. 110–111.} }\toendnotes[C]{\smallbreak}\pstart{}{\pb}Herrn D\textsuperscript{r} Arthur Schnitzler\pend{}\pstart{}Luzerne\oindex{Luzern@\textbf{Luzern}, \emph{P.PPLA}|pw}\pend{}\pstart{}post. rest.\pend{}{\bigskip}\vspace{1em}
\pstart
           \raggedleft{}{\pb}Lugano\oindex{Hôtel du Parc@\textbf{Hôtel du Parc}, \emph{Hotel (K.HTL)}|pw}, Do{\geminationn}erstg.\pend
           \vspace{0.5em}
\pstart
           Ich arbeite nicht, war darüber in den erſten Tagen unſinnig verſti{\geminationm}t und niedergeſchlagen, jetzt hab ich mich
               dreingefunden und leb ſtill und angenehm, beſonders ſeit die furchtbare Schwüle
               aufgehört hat.\pend
           
\pstart
           Richard\pwindex{Beer-Hofmann, Richard 1866-07-11 – 1945-09-26@\textsc{Beer-Hofmann, Richard} (1866-07-11 – 1945-09-26), \emph{Schriftsteller/Schriftstellerin}|pw} arbeitet »\label{K_L00838-1v}\edtext{mehr und leichter als je}{\lemma{\textnormal{\emph{mehr und leichter als je}}}\Cendnote{\textnormal{Im Brief vom 22. 8. 1898{ }schreibt Beer-Hofmann\pwindex{Beer-Hofmann, Richard 1866-07-11 – 1945-09-26@\textsc{Beer-Hofmann, Richard} (1866-07-11 – 1945-09-26), \emph{Schriftsteller/Schriftstellerin}|pwk} an Hofmannsthal\pwindex{Hofmannsthal, Hugo von 1874-02-01 – 1929-07-15@\textsc{Hofmannsthal, Hugo von} (1874-02-01 – 1929-07-15), \emph{Schriftsteller/Schriftstellerin}|pwk}:
                     »ich bin mitten in der Arbeit, arbeite leicht, und mehr als
                        sonst.« (Hugo von Hofmannsthal, Richard Beer-Hofmann: \emph{Briefwechsel}. Herausgegeben von  Eugene Weber. Frankfurt am Main:
                        \emph{S. Fischer} 1972, S. 83.)}}}\label{K_L00838-1}« und dürfte den
                     31\textsuperscript{ten} hierher zu mir ko{\geminationm}en. Bitte \uline{bald} wieder Nachricht. Von Herzen Ihr \spacefill\mbox{Hugo.}\pend
           \selectlanguage{ngerman}\endnumbering\briefempfaengerindex{Schnitzler, Arthur@\textsc{Schnitzler, Arthur}!zzzHofmannsthal, Hugo von@\emph{von Hugo von Hofmannsthal}!1898-08-251@{25. 8. 1898}|)be}\mylabel{L00838h}  \normalsize

\doendnotes{C}
\bigskip
\vfill

\clearpage

\footnotesize

\lohead{\textsc{register}}

% Definiere theindex-Environment komplett neu ohne reledmac
\makeatletter
\renewenvironment{theindex}{%
  \section*{\indexname}%
  \setlength{\parindent}{0pt}%
  \setlength{\parskip}{0pt plus 0.3pt}%
  \let\item\@idxitem
}{%
  \clearpage
}
\makeatother

\IfFileExists{\jobname-pw.ind}{\input{\jobname-pw.ind}}{}

\end{document}

      