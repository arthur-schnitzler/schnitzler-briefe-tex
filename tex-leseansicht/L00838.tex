%% latex-leseansicht-vorspann.tex
%% Vorspann für die Leseansicht.
%% Lädt die gemeinsame Datei latex-vorspann.tex mit nicht gesetztem Schalter.

\newif\ifkorrekturansicht
\korrekturansichtfalse

\input{../tex-inputs/latex-vorspann}


\section[Hugo von Hofmannsthal an Arthur Schnitzler, 25. 8. 1898]{L00838 Hugo von Hofmannsthal an Arthur Schnitzler, 25. 8. 1898}
\nopagebreak\mylabel{L00838v}
\rehead{ }\normalsize\beginnumbering\briefempfaengerindex{Schnitzler, Arthur@\textsc{Schnitzler, Arthur}!zzzHofmannsthal, Hugo von@\emph{von Hugo von Hofmannsthal}!1898-08-251@{25. 8. 1898}|(be}
\toendnotes[C]{\smallbreak\pagebreak[2]}
\correspDesc{Versand  durch Hugo von Hofmannsthal am 25. 8. 1898 in Lugano
\newline{}Erhalt  durch Arthur Schnitzler am 25. 8. 1898 in Luzern}\toendnotes[C]{\smallbreak}
\Standort{CUL, Schnitzler, B 43.}
\physDesc{Postkarte, 393 Zeichen
\newline{}Handschrift: Bleistift, deutsche Kurrent
\newline{}Versand: 1) Stempel: »\nobreak{}\oindex{Lugano@\textbf{Lugano}, \emph{Hauptstadt}|pwk}Lugano, 25. VIII. 98, XII\nobreak{}«.   2) Stempel: »\nobreak{}\oindex{Luzern@\textbf{Luzern}|pwk}Luzern Brf. Dist, 25. VIII. 98, 7\nobreak{}«. 
\newline{}Schnitzler: mit Bleistift datiert: »25/8 98« 
\newline{}Ordnung: 1) mit Bleistift von unbekannter Hand nummeriert: »\strikeout{121}«  2) mit Bleistift von unbekannter Hand nummeriert:
                                    »122«}
\buchAbdrucke{\weitereDrucke{Hugo von Hofmannsthal, Arthur Schnitzler: \emph{Briefwechsel}. Herausgegeben von Therese Nickl und Heinrich Schnitzler. Frankfurt am Main: \emph{S. Fischer} 1964, S. 110–111.} }\toendnotes[C]{\smallbreak}\pstart{}\textsc{{\pb}Herrn D\textsuperscript{r} Arthur Schnitzler}\pend{}\pstart{}\textsc{Luzerne\oindex{Luzern@\textbf{Luzern}|pw}}\pend{}\pstart{}\textsc{post. rest.}\pend{}{\bigskip}\vspace{1em}
\pstart
           \raggedleft{}{\pb}Lugano\oindex{Hôtel du Parc@\textbf{Hôtel du Parc}, \emph{Hotel}|pw}, Do{\geminationn}erstg.\pend
           \vspace{0.5em}
\pstart
           Ich arbeite nicht, war darüber in den erſten Tagen unſinnig verſti{\geminationm}t und niedergeſchlagen, jetzt hab ich mich
               dreingefunden und leb{ }ſtill und angenehm, beſonders{ }ſeit die furchtbare Schwüle
               aufgehört hat.\pend
           
\pstart
           Richard\pwindex{Beer-Hofmann, Richard 11.\,7.\,1866 Wien – 26.\,9.\,1945 New York City@\textsc{Beer-Hofmann, Richard} (11.\,7.\,1866 Wien – 26.\,9.\,1945 New York City), \emph{Schriftsteller}|pw} arbeitet »\label{K_L00838-1v}\edtext{mehr und leichter als je}{\lemma{\textnormal{\emph{mehr und leichter als je}}}\Cendnote{\textnormal{Im Brief vom 22. 8. 1898{ }schreibt Beer-Hofmann\pwindex{Beer-Hofmann, Richard 11.\,7.\,1866 Wien – 26.\,9.\,1945 New York City@\textsc{Beer-Hofmann, Richard} (11.\,7.\,1866 Wien – 26.\,9.\,1945 New York City), \emph{Schriftsteller}|pwk} an Hofmannsthal\pwindex{Hofmannsthal, Hugo von 1.\,2.\,1874 Wien – 15.\,7.\,1929 Rodaun@\textsc{Hofmannsthal, Hugo von} (1.\,2.\,1874 Wien – 15.\,7.\,1929 Rodaun), \emph{Schriftsteller}|pwk}:
                     »ich bin mitten in der Arbeit, arbeite leicht, und mehr als
                        sonst.« (Hugo von Hofmannsthal, Richard Beer-Hofmann: \emph{Briefwechsel}. Herausgegeben von  Eugene Weber. Frankfurt am Main:
                        \emph{S. Fischer} 1972, S. 83.)}}}\label{K_L00838-1}« und dürfte den
                     31\textsuperscript{ten} hierher zu mir ko{\geminationm}en. Bitte \uline{bald} wieder Nachricht. Von Herzen Ihr \spacefill\mbox{Hugo.}\pend
           \selectlanguage{ngerman}\endnumbering\briefempfaengerindex{Schnitzler, Arthur@\textsc{Schnitzler, Arthur}!zzzHofmannsthal, Hugo von@\emph{von Hugo von Hofmannsthal}!1898-08-251@{25. 8. 1898}|)be}\mylabel{L00838h}  \newcommand{\dateiname}{L00838}\newcommand{\titel}{Hugo von Hofmannsthal an Arthur Schnitzler, 25. 8. 1898}\newcommand{\editorInnen}{Martin Anton Müller und Gerd-Hermann Susen}%% latex-leseansicht-abspann.tex
%% Abspann für die Leseansicht.
%% Der Schalter \ifkorrekturansicht ist bereits durch den Vorspann gesetzt.

%% latex-abspann.tex
%% Gemeinsamer Abspann für Korrekturansicht und Leseansicht.
%% Setzt den Schalter \ifkorrekturansicht voraus (gesetzt in den
%% einbindenden Dateien latex-korrekturansicht-abspann.tex bzw.
%% latex-leseansicht-abspann.tex).
%% ---------------------------------------------------------------

\normalsize

% Das esempio-Environment wird nur in der Leseansicht benötigt
\ifkorrekturansicht\else
\newenvironment{esempio}[3]%
{
    \vspace{1.5ex}
    \rlap{\underline{#1}}
    \par
    \setlength{\parindent}{0cm}
    \nopagebreak
    \leftskip=#2cm
    \rightskip=#3cm
}
{
    \par
}
\fi

\doendnotes{C}
\bigskip
\vfill

\clearpage

\footnotesize

\ifkorrekturansicht
  \lohead{\textsc{register}}
\fi

% theindex-Environment neu definieren ohne reledmac
\makeatletter
\renewenvironment{theindex}{%
  \ifkorrekturansicht
    \section*{\indexname}%
  \else
    \subsubsection*{Index der erwähnten Entitäten}%
  \fi
  \setlength{\parindent}{0pt}%
  \setlength{\parskip}{0pt plus 0.3pt}%
  \let\item\@idxitem
}{%
  \ifkorrekturansicht\clearpage\fi
}
\makeatother

\IfFileExists{\jobname-pw.ind}{\input{\jobname-pw.ind}}{}

% Quellenangabe nur in der Leseansicht
\ifkorrekturansicht\else
% Fallback-Definitionen, falls die .tex-Datei \titel etc. nicht gesetzt hat
\providecommand{\titel}{}
\providecommand{\editorInnen}{}
\providecommand{\dateiname}{\jobname}

\vspace{3cm}

\vfill

\footnotesize
\textsc{Quelle}: \titel. Herausgegeben von {\editorInnen}. In: \emph{Arthur Schnitzler: Briefwechsel mit Autorinnen und Autoren}.
 Digitale Edition, https://schnitzler-briefe.acdh.oeaw.ac.at/{\dateiname}.html (Stand \today)
\fi

\end{document}


