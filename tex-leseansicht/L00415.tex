%% latex-korrekturansicht-vorspann.tex
%% Vorspann für die Korrekturansicht.
%% Lädt die gemeinsame Datei latex-vorspann.tex mit gesetztem Schalter.

\newif\ifkorrekturansicht
\korrekturansichttrue

\input{../tex-inputs/latex-vorspann}


\section[Arthur Schnitzler an Richard Beer-Hofmann, {[}19. 1. 1895?{]}]{L00415 Arthur Schnitzler an Richard Beer-Hofmann, {[}19. 1. 1895?{]}}
\nopagebreak\mylabel{L00415v}
\rehead{ }\normalsize\beginnumbering\briefempfaengerindex{Beer-Hofmann, Richard@\textsc{Beer-Hofmann, Richard}!zzzSchnitzler, Arthur@\emph{von Arthur Schnitzler}!1895-01-191@{{[}19. 1. 1895?{]}}|(be}
\toendnotes[C]{\smallbreak\pagebreak[2]}\Standort{YCGL, MSS 31.}
\physDesc{Brief, 1 Blatt, 3 Seiten, 327 Zeichen
\newline{}Handschrift: Bleistift, deutsche Kurrent}
\buchAbdrucke{\weitereDrucke{Arthur Schnitzler, Richard Beer-Hofmann: \emph{Briefwechsel 1891–1931}. Wien, Zürich: \emph{Europaverlag} 1992, S. 71.} }\toendnotes[C]{\smallbreak}
\pstart
           \noindent{}{\pb}Lieber Richard. Ko{\geminationm}en Sie in die
               Loge\pend
           
\pstart
           \centering{}\textsc{Nr. eilf}, I. Stock links.\pend
           
\pstart
           War nichts {\pb}andres zu beko{\geminationm}en. –\pend
           
\pstart
           Hoffe, zur \label{K_L00415-1v}\edtext{Repartirung}{\lemma{\textnormal{\emph{Repartirung}}}\Cendnote{\textnormal{Aufteilung (der Kosten)}}}\label{K_L00415-1}, daſs mein
                  Bruder\pwindex{Schnitzler, Julius 13.07.1865 – 29.06.1939@\textsc{Schnitzler, Julius} (13.07.1865 – 29.06.1939), \emph{Chirurg/Chirurgin}|pwv} u Schwägerin\pwindex{Schnitzler, Helene 16.07.1871 – September 1941@\textsc{Schnitzler, Helene} (16.07.1871 – September 1941)|pwv} mitko{\geminationm}en.\pend
           
\pstart
           Die Loge hab ich. –\pend
           
\pstart
           Nachher sind wir, dh. Sie, Qualle\pwindex{Sandrock, Adele 1863-08-19 – 1937-08-30@\textsc{Sandrock, Adele} (1863-08-19 – 1937-08-30), \emph{Schauspieler/Schauspielerin}|pwv}, {\pb}Schweſter\pwindex{Sandrock, Wilhelmine 05.02.1861 – 29.11.1948@\textsc{Sandrock, Wilhelmine} (05.02.1861 – 29.11.1948), \emph{Schauspieler/Schauspielerin}|pwv} u Salten\pwindex{Salten, Felix 06.09.1869 – 08.10.1945@\textsc{Salten, Felix} (06.09.1869 – 08.10.1945), \emph{Schriftsteller/Schriftstellerin, Journalist/Journalistin, Chefredakteur/Chefredakteurin}|pw}{ }\introOben{}u ich\introOben{} zusa{\geminationm}en. Bitte \uuline{dringend}{ }\uuline{keine}{ }\label{K_L00415-2v}\edtext{Elegance}{\lemma{\textnormal{\emph{Elegance}}}\Cendnote{\textnormal{Das Korrespondenzstück ist undatiert, die Hinweise sind
                  spärlich. Der Umstand, dass Schnitzler das
                  Reglement zur Kleidungswahl bestimmt, deutet auf eine von ihm organisierte
                  Festlichkeit. Naheliegend ist dafür der 19. 1. 1895, jener Tag, an
                  dem in der Zeitung steht, dass \emph{Liebelei}\pwindex{Liebelei. Schauspiel in drei Akten@\emph{Liebelei. Schauspiel in drei Akten}|pwk}
                  zur Aufführung am \emph{Burgtheater}\orgindex{Burgtheater@Burgtheater|pwk} angenommen
                  worden ist. An diesem Abend trafen sich die Genannten – ohne Willy Sandrock\pwindex{Sandrock, Wilhelmine 05.02.1861 – 29.11.1948@\textsc{Sandrock, Wilhelmine} (05.02.1861 – 29.11.1948), \emph{Schauspieler/Schauspielerin}|pwk}, dafür aber mit Robert Nhil\pwindex{Nhil, Robert 18.07.1858 – 31.10.1938@\textsc{Nhil, Robert} (18.07.1858 – 31.10.1938), \emph{Schauspieler/Schauspielerin}|pwk}. Grund für die Loge im Theater wäre dann
                  wiederum, dass am selben Abend Josef
                     Giampietro\pwindex{Giampietro, Josef 21.06.1866 – 29.12.1913@\textsc{Giampietro, Josef} (21.06.1866 – 29.12.1913), \emph{Schauspieler/Schauspielerin, Filmschauspieler/Filmschauspielerin, Komiker/Komikerin}|pwk} in der Premiere von \emph{Die
                     Kameraden}\pwindex{Kameraden. Lustspiel in drei Aufzuegen@\emph{Die Kameraden. Lustspiel in drei Aufzügen}|pwk} seine Rolle als Nachahmung von Schnitzler anlegte.}}}\label{K_L00415-2}.\pend
           
\pstart
           Herzlich Ihr{\\[\baselineskip]}\spacefill\mbox{Arthur}\pend
           \leftskip=0em{}
\pstart
           \noindent{}(Ich gehe ſchwarzes \textsc{Sacco}.)\pend
           
\pstart
           Vielleicht doch \textsc{smoking}\pend
           \selectlanguage{ngerman}\endnumbering\briefempfaengerindex{Beer-Hofmann, Richard@\textsc{Beer-Hofmann, Richard}!zzzSchnitzler, Arthur@\emph{von Arthur Schnitzler}!1895-01-191@{{[}19. 1. 1895?{]}}|)be}\mylabel{L00415h}  \normalsize

\doendnotes{C}
\bigskip
\vfill

\clearpage

\footnotesize

\lohead{\textsc{register}}

% Definiere theindex-Environment komplett neu ohne reledmac
\makeatletter
\renewenvironment{theindex}{%
  \section*{\indexname}%
  \setlength{\parindent}{0pt}%
  \setlength{\parskip}{0pt plus 0.3pt}%
  \let\item\@idxitem
}{%
  \clearpage
}
\makeatother

\IfFileExists{\jobname-pw.ind}{\input{\jobname-pw.ind}}{}

\end{document}

      