%% latex-leseansicht-vorspann.tex
%% Vorspann für die Leseansicht.
%% Lädt die gemeinsame Datei latex-vorspann.tex mit nicht gesetztem Schalter.

\newif\ifkorrekturansicht
\korrekturansichtfalse

\input{../tex-inputs/latex-vorspann}


         
         \newcommand{\erwaehntePersonen}{Personen: Richard Beer-Hofmann, Josef Giampietro, Robert Nhil, Felix Salten, Adele Sandrock, Wilhelmine Sandrock, Julius Schnitzler, Helene Schnitzler}
         \newcommand{\erwaehnteInstitutionen}{Institutionen: Burgtheater}
         \newcommand{\erwaehnteOrte}{Orte: Wien}
         \newcommand{\erwaehnteWerke}{Werke: Die Kameraden. Lustspiel in drei Aufzügen, Liebelei. Schauspiel in drei Akten}
               \section[Arthur Schnitzler an Richard Beer-Hofmann, {[}19. 1. 1895?{]}]{ Arthur Schnitzler an Richard Beer-Hofmann, {[}19. 1. 1895?{]}}\nopagebreak\mylabel{v}\rehead{ }\begin{ledgroupsized}[t]{13cm}\normalsize\beginnumbering \toendnotes[C]{\smallbreak\pagebreak[2]} \Standort{YCGL, MSS 31.}
\physDesc{Brief, 1 Blatt, 3 Seiten
\newline{}Handschrift: Bleistift, deutsche Kurrent}\buchAbdrucke{\weitereDrucke{Arthur Schnitzler, Richard Beer-Hofmann: \emph{Briefwechsel 1891–1931}. Hg. Konstanze Fliedl. Wien, Zürich: \emph{Europaverlag} 1992, S. 71.} }\toendnotes[C]{\smallbreak}\pstart
           \noindent{}{\pb}Lieber Richard. Ko{\geminationm}en Sie in die
               Loge\pend
           \pstart
           \centering{}\textsc{Nr. eilf}, I. Stock links.\pend
           \pstart
           \noindent{}War nichts {\pb}andres zu beko{\geminationm}en. –\pend
           \pstart
           Hoffe, zur \label{K_L00415_1v}\edtext{Repartirung}{\lemma{\textnormal{\emph{Repartirung}}}\Cendnote{\textnormal{Aufteilung (der Kosten)}}}\label{K_L00415_1h}, daſs mein
                  Bruder\pwindex{Schnitzler, Julius 13.07.1865 – 29.06.1939@\textsc{Schnitzler, Julius} (13.07.1865 – 29.06.1939), \emph{Chirurg}|pwv} u Schwägerin\pwindex{Schnitzler, Helene 16.07.1871 – September 1941@\textsc{Schnitzler, Helene} (16.07.1871 – September 1941)|pwv} mitko{\geminationm}en.\pend
           \pstart
           Die Loge hab ich. –\pend
           \pstart
           Nachher sind wir, dh. Sie, Qualle\pwindex{Sandrock, Adele 1863-08-19 – 1937-08-30@\textsc{Sandrock, Adele} (1863-08-19 – 1937-08-30), \emph{Schauspielerin}|pwv}, {\pb}Schweſter\pwindex{Sandrock, Wilhelmine 05.02.1861 – 29.11.1948@\textsc{Sandrock, Wilhelmine} (05.02.1861 – 29.11.1948), \emph{Schauspielerin}|pwv} u Salten\pwindex{Salten, Felix 06.09.1869 – 08.10.1945@\textsc{Salten, Felix} (06.09.1869 – 08.10.1945), \emph{Schriftsteller, Journalist}|pw}{ }\introOben{}u ich\introOben{} zusa{\geminationm}en. Bitte \uuline{dringend}{ }\uuline{keine}{ }\label{K_L00415_2v}\edtext{Elegance}{\lemma{\textnormal{\emph{Elegance}}}\Cendnote{\textnormal{Das Korrespondenzstück ist undatiert, die Hinweise sind
                  spärlich. Der Umstand, dass Schnitzler\pwindex{Schnitzler, Arthur 15.05.1862 – 21.10.1931@\textsc{Schnitzler, Arthur} (15.05.1862 – 21.10.1931), \emph{Schriftsteller, Mediziner}|pwk} das
                  Reglement zur Kleidungswahl bestimmt, deutet auf eine von ihm organisierte
                  Festlichkeit. Naheliegend ist dafür der 19. 1. 1895, jener Tag, an
                  dem in der Zeitung steht, dass die \emph{Liebelei}\pwindex{Schnitzler, Arthur 15.05.1862 – 21.10.1931@\textsc{Schnitzler, Arthur} (15.05.1862 – 21.10.1931), \emph{Schriftsteller, Mediziner}!Liebelei. Schauspiel in drei Akten1895-10-09@\strich\emph{Liebelei. Schauspiel in drei Akten} {[}1895-10-09{]}|pwk} zur
                  Aufführung am \emph{Burgtheater}\orgindex{Burgtheater@Burgtheater|pwk} angenommen worden ist.
                  An diesem Abend trafen sich die Genannten – ohne Willy Sandrock\pwindex{Sandrock, Wilhelmine 05.02.1861 – 29.11.1948@\textsc{Sandrock, Wilhelmine} (05.02.1861 – 29.11.1948), \emph{Schauspielerin}|pwk}, dafür aber mit Robert
                     Nhil\pwindex{Nhil, Robert 18.07.1858 – 31.10.1938@\textsc{Nhil, Robert} (18.07.1858 – 31.10.1938), \emph{Schauspieler}|pwk}. Grund für die Loge im Theater wäre dann wiederum, dass am selben
                  Abend Josef Giampietro\pwindex{Giampietro, Josef 21.06.1866 – 29.12.1913@\textsc{Giampietro, Josef} (21.06.1866 – 29.12.1913), \emph{Schauspieler, Filmschauspieler, Komiker}|pwk} in der Premiere von \emph{Die Kameraden}\pwindex{\textcolor{red}{\textsuperscript{XXXX1 indx}}!Kameraden. Lustspiel in drei Aufzuegen1894@\strich\emph{Die Kameraden. Lustspiel in drei Aufzügen} {[}1894{]}|pwk} seine Rolle offensichtlich Schnitzler\pwindex{Schnitzler, Arthur 15.05.1862 – 21.10.1931@\textsc{Schnitzler, Arthur} (15.05.1862 – 21.10.1931), \emph{Schriftsteller, Mediziner}|pwk} nachahmend anlegte.}}}\label{K_L00415_2h}.\pend
           \pstart
           Herzlich Ihr{\\[\baselineskip]}\spacefill\mbox{Arthur}\pend
           \leftskip=0em{}\pstart
           \noindent{}(Ich gehe ſchwarzes \textsc{Sacco}.)\pend
           \pstart
           Vielleicht doch \textsc{smoking}\pend
           
         
         \endnumbering\mylabel{h}\end{ledgroupsized}  \newcommand{\dateiname}{L00415}\newcommand{\titel}{Arthur Schnitzler an Richard Beer-Hofmann, [19. 1. 1895?]}\newcommand{\editorInnen}{Martin Anton Müller und Gerd-Hermann Susen}%% latex-leseansicht-abspann.tex
%% Abspann für die Leseansicht.
%% Der Schalter \ifkorrekturansicht ist bereits durch den Vorspann gesetzt.

%% latex-abspann.tex
%% Gemeinsamer Abspann für Korrekturansicht und Leseansicht.
%% Setzt den Schalter \ifkorrekturansicht voraus (gesetzt in den
%% einbindenden Dateien latex-korrekturansicht-abspann.tex bzw.
%% latex-leseansicht-abspann.tex).
%% ---------------------------------------------------------------

\normalsize

% Das esempio-Environment wird nur in der Leseansicht benötigt
\ifkorrekturansicht\else
\newenvironment{esempio}[3]%
{
    \vspace{1.5ex}
    \rlap{\underline{#1}}
    \par
    \setlength{\parindent}{0cm}
    \nopagebreak
    \leftskip=#2cm
    \rightskip=#3cm
}
{
    \par
}
\fi

\doendnotes{C}
\bigskip
\vfill

\clearpage

\footnotesize

\ifkorrekturansicht
  \lohead{\textsc{register}}
\fi

% theindex-Environment neu definieren ohne reledmac
\makeatletter
\renewenvironment{theindex}{%
  \ifkorrekturansicht
    \section*{\indexname}%
  \else
    \subsubsection*{Index der erwähnten Entitäten}%
  \fi
  \setlength{\parindent}{0pt}%
  \setlength{\parskip}{0pt plus 0.3pt}%
  \let\item\@idxitem
}{%
  \ifkorrekturansicht\clearpage\fi
}
\makeatother

\IfFileExists{\jobname-pw.ind}{\input{\jobname-pw.ind}}{}

% Quellenangabe nur in der Leseansicht
\ifkorrekturansicht\else
% Fallback-Definitionen, falls die .tex-Datei \titel etc. nicht gesetzt hat
\providecommand{\titel}{}
\providecommand{\editorInnen}{}
\providecommand{\dateiname}{\jobname}

\vspace{3cm}

\vfill

\footnotesize
\textsc{Quelle}: \titel. Herausgegeben von {\editorInnen}. In: \emph{Arthur Schnitzler: Briefwechsel mit Autorinnen und Autoren}.
 Digitale Edition, https://schnitzler-briefe.acdh.oeaw.ac.at/{\dateiname}.html (Stand \today)
\fi

\end{document}


      