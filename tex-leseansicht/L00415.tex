%% latex-leseansicht-vorspann.tex
%% Vorspann für die Leseansicht.
%% Lädt die gemeinsame Datei latex-vorspann.tex mit nicht gesetztem Schalter.

\newif\ifkorrekturansicht
\korrekturansichtfalse

\input{../tex-inputs/latex-vorspann}


\section[Arthur Schnitzler an Richard Beer-Hofmann, {{[}}19. 1. 1895?{{]}}]{L00415 Arthur Schnitzler an Richard Beer-Hofmann, {[}19. 1. 1895?{]}}
\nopagebreak\mylabel{L00415v}
\rehead{ }\normalsize\beginnumbering\briefempfaengerindex{Beer-Hofmann, Richard@\textsc{Beer-Hofmann, Richard}!zzzSchnitzler, Arthur@\emph{von Arthur Schnitzler}!1895-01-191@{{[}19. 1. 1895?{]}}|(be}
\toendnotes[C]{\smallbreak\pagebreak[2]}
\correspDesc{Versand  durch Arthur Schnitzler am [19. 1. 1895?] in Wien
\newline{}Erhalt  durch Richard Beer-Hofmann im Zeitraum [19. 1. 1895
                  – 23. 1. 1895?] \textbf{Ort fehlend} }\toendnotes[C]{\smallbreak}
\Standort{YCGL, MSS 31.}
\physDesc{Brief, 1 Blatt, 3 Seiten, 327 Zeichen
\newline{}Handschrift: Bleistift, deutsche Kurrent}
\buchAbdrucke{\weitereDrucke{Arthur Schnitzler, Richard Beer-Hofmann: \emph{Briefwechsel 1891–1931}. Herausgegeben von Konstanze Fliedl. Wien, Zürich: \emph{Europaverlag} 1992, S. 71.} }\toendnotes[C]{\smallbreak}
\pstart
           \noindent{}{\pb}Lieber Richard. Ko{\geminationm}en Sie in die
               Loge\pend
           
\pstart
           \centering{}\textsc{Nr. eilf}, I. Stock links.\pend
           
\pstart
           War nichts {\pb}andres zu beko{\geminationm}en. –\pend
           
\pstart
           Hoffe, zur \label{K_L00415-1v}\edtext{Repartirung}{\lemma{\textnormal{\emph{Repartirung}}}\Cendnote{\textnormal{Aufteilung (der Kosten)}}}\label{K_L00415-1}, daſs mein
                  Bruder\pwindex{Schnitzler, Julius 13.\,7.\,1865 Wien – 29.\,6.\,1939 ebd.@\textsc{Schnitzler, Julius} (13.\,7.\,1865 Wien – 29.\,6.\,1939 ebd.), \emph{Chirurg}|pwv} u Schwägerin\pwindex{Schnitzler, Helene 16.\,7.\,1871 Budapest – September 1941 Atlantischer Ozean@\textsc{Schnitzler, Helene} (16.\,7.\,1871 Budapest – September 1941 Atlantischer Ozean)|pwv} mitko{\geminationm}en.\pend
           
\pstart
           Die Loge hab ich. –\pend
           
\pstart
           Nachher sind wir, dh. Sie, Qualle\pwindex{Sandrock, Adele 19.\,8.\,1863 Rotterdam – 30.\,8.\,1937 Berlin@\textsc{Sandrock, Adele} (19.\,8.\,1863 Rotterdam – 30.\,8.\,1937 Berlin), \emph{Schauspielerin}|pwv}, {\pb}Schweſter\pwindex{Sandrock, Wilhelmine 5.\,2.\,1861 Rotterdam – 29.\,11.\,1948 Berlin@\textsc{Sandrock, Wilhelmine} (5.\,2.\,1861 Rotterdam – 29.\,11.\,1948 Berlin), \emph{Schauspielerin}|pwv} u Salten\pwindex{Salten, Felix 6.\,9.\,1869 Budapest – 8.\,10.\,1945 Zürich@\textsc{Salten, Felix} (6.\,9.\,1869 Budapest – 8.\,10.\,1945 Zürich), \emph{Schriftsteller, Journalist, Chefredakteur}|pw}{ }\introOben{}u ich\introOben{} zusa{\geminationm}en. Bitte \uuline{dringend}{ }\uuline{keine}{ }\label{K_L00415-2v}\edtext{Elegance}{\lemma{\textnormal{\emph{Elegance}}}\Cendnote{\textnormal{Das Korrespondenzstück ist undatiert, die Hinweise sind
                  spärlich. Der Umstand, dass Schnitzler das
                  Reglement zur Kleidungswahl bestimmt, deutet auf eine von ihm organisierte
                  Festlichkeit. Naheliegend ist dafür der 19. 1. 1895, jener Tag, an
                  dem in der Zeitung steht, dass \emph{Liebelei}\pwindex{Schnitzler, Arthur 15.\,5.\,1862 Wien – 21.\,10.\,1931 ebd.@\textsc{Schnitzler, Arthur} (15.\,5.\,1862 Wien – 21.\,10.\,1931 ebd.), \emph{Schriftsteller, Mediziner}!Liebelei. Schauspiel in drei Akten@\strich\emph{Liebelei. Schauspiel in drei Akten}|pwk}
                  zur Aufführung am \emph{Burgtheater}\orgindex{Burgtheater@Burgtheater|pwk} angenommen
                  worden ist. An diesem Abend trafen sich die Genannten – ohne Willy Sandrock\pwindex{Sandrock, Wilhelmine 5.\,2.\,1861 Rotterdam – 29.\,11.\,1948 Berlin@\textsc{Sandrock, Wilhelmine} (5.\,2.\,1861 Rotterdam – 29.\,11.\,1948 Berlin), \emph{Schauspielerin}|pwk}, dafür aber mit Robert Nhil\pwindex{Nhil, Robert 18.\,7.\,1858 Hamburg – 31.\,10.\,1938 ebd.@\textsc{Nhil, Robert} (18.\,7.\,1858 Hamburg – 31.\,10.\,1938 ebd.), \emph{Schauspieler}|pwk}. Grund für die Loge im Theater wäre dann
                  wiederum, dass am selben Abend Josef
                     Giampietro\pwindex{Giampietro, Josef 21.\,6.\,1866 Wien – 29.\,12.\,1913 Berlin@\textsc{Giampietro, Josef} (21.\,6.\,1866 Wien – 29.\,12.\,1913 Berlin), \emph{Schauspieler, Filmschauspieler, Komiker}|pwk} in der Premiere von \emph{Die
                     Kameraden}\pwindex{\textcolor{red}{\textsuperscript{XXXX indx1}}!Kameraden. Lustspiel in drei Aufzügen@\strich\emph{Die Kameraden. Lustspiel in drei Aufzügen}|pwk} seine Rolle als Nachahmung von Schnitzler anlegte.}}}\label{K_L00415-2}.\pend
           
\pstart
           Herzlich Ihr{\\[\baselineskip]}\spacefill\mbox{Arthur}\pend
           \leftskip=0em{}
\pstart
           \noindent{}(Ich gehe{ }ſchwarzes \textsc{Sacco}.)\pend
           
\pstart
           Vielleicht doch \textsc{smoking}\pend
           \selectlanguage{ngerman}\endnumbering\briefempfaengerindex{Beer-Hofmann, Richard@\textsc{Beer-Hofmann, Richard}!zzzSchnitzler, Arthur@\emph{von Arthur Schnitzler}!1895-01-191@{{[}19. 1. 1895?{]}}|)be}\mylabel{L00415h}  \newcommand{\dateiname}{L00415}\newcommand{\titel}{Arthur Schnitzler an Richard Beer-Hofmann, [19. 1. 1895?]}\newcommand{\editorInnen}{Martin Anton Müller und Gerd-Hermann Susen}%% latex-leseansicht-abspann.tex
%% Abspann für die Leseansicht.
%% Der Schalter \ifkorrekturansicht ist bereits durch den Vorspann gesetzt.

%% latex-abspann.tex
%% Gemeinsamer Abspann für Korrekturansicht und Leseansicht.
%% Setzt den Schalter \ifkorrekturansicht voraus (gesetzt in den
%% einbindenden Dateien latex-korrekturansicht-abspann.tex bzw.
%% latex-leseansicht-abspann.tex).
%% ---------------------------------------------------------------

\normalsize

% Das esempio-Environment wird nur in der Leseansicht benötigt
\ifkorrekturansicht\else
\newenvironment{esempio}[3]%
{
    \vspace{1.5ex}
    \rlap{\underline{#1}}
    \par
    \setlength{\parindent}{0cm}
    \nopagebreak
    \leftskip=#2cm
    \rightskip=#3cm
}
{
    \par
}
\fi

\doendnotes{C}
\bigskip
\vfill

\clearpage

\footnotesize

\ifkorrekturansicht
  \lohead{\textsc{register}}
\fi

% theindex-Environment neu definieren ohne reledmac
\makeatletter
\renewenvironment{theindex}{%
  \ifkorrekturansicht
    \section*{\indexname}%
  \else
    \subsubsection*{Index der erwähnten Entitäten}%
  \fi
  \setlength{\parindent}{0pt}%
  \setlength{\parskip}{0pt plus 0.3pt}%
  \let\item\@idxitem
}{%
  \ifkorrekturansicht\clearpage\fi
}
\makeatother

\IfFileExists{\jobname-pw.ind}{\input{\jobname-pw.ind}}{}

% Quellenangabe nur in der Leseansicht
\ifkorrekturansicht\else
% Fallback-Definitionen, falls die .tex-Datei \titel etc. nicht gesetzt hat
\providecommand{\titel}{}
\providecommand{\editorInnen}{}
\providecommand{\dateiname}{\jobname}

\vspace{3cm}

\vfill

\footnotesize
\textsc{Quelle}: \titel. Herausgegeben von {\editorInnen}. In: \emph{Arthur Schnitzler: Briefwechsel mit Autorinnen und Autoren}.
 Digitale Edition, https://schnitzler-briefe.acdh.oeaw.ac.at/{\dateiname}.html (Stand \today)
\fi

\end{document}


