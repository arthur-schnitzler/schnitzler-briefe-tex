%% latex-korrekturansicht-vorspann.tex
%% Vorspann für die Korrekturansicht.
%% Lädt die gemeinsame Datei latex-vorspann.tex mit gesetztem Schalter.

\newif\ifkorrekturansicht
\korrekturansichttrue

\input{../tex-inputs/latex-vorspann}


\section[Arthur Schnitzler an Richard Beer-Hofmann, 15. 5. 1922]{L02382 Arthur Schnitzler an Richard Beer-Hofmann, 15. 5. 1922}
\nopagebreak\mylabel{L02382v}
\rehead{ }\normalsize\beginnumbering\briefempfaengerindex{Beer-Hofmann, Richard@\textsc{Beer-Hofmann, Richard}!zzzSchnitzler, Arthur@\emph{von Arthur Schnitzler}!1922-05-151@{15. 5. 1921}|(be}
\toendnotes[C]{\smallbreak\pagebreak[2]}\Standort{YCGL, MSS 31.}
\physDesc{Bildpostkarte, 97 Zeichen
\newline{}Handschrift: Bleistift, lateinische Kurrent
\newline{}Versand: Stempel: »\nobreak{}\oindex{Nuernberg@\textbf{Nürnberg}, \emph{P.PPL}|pwk}Nürnberg, 15. 5. 22, N3–4\nobreak{}«.  }\pstart{}{\pb}Hrn Dr. Richard Beer-Hofmann\pend{}\pstart{}Wien XVIII\oindex{XVIII., Waehring@\textbf{XVIII., Währing}, \emph{A.ADM3}|pw}\pend{}\pstart{}Hasenauerstr 59\oindex{Hasenauerstrasse 59@\textbf{Hasenauerstraße 59}, \emph{Wohngebäude (K.WHS)}|pw}.\pend{}{\bigskip}
\pstart
           \noindent{}\centering{}{\pb}\textcolor{gray}{\textbf{Nürnberg\oindex{Nuernberg@\textbf{Nürnberg}, \emph{P.PPL}|pw}}}\pend
           
\pstart
           \centering{}\textcolor{gray}{\textbf{Blick von der Museumsbrücke\oindex{Museumsbruecke Nuernberg@\textbf{Museumsbrücke Nürnberg}, \emph{Brücke (K.BRK)}|pw} zum Heiliggeist-Spital\oindex{Heilig-Geist-Spital@\textbf{Heilig-Geist-Spital}, \emph{Krankenhaus (K.KKH)}|pw}.}}\pend
           \vspace{1em}
\pstart
           \raggedleft{}{\pb}Nürnberg\oindex{Nuernberg@\textbf{Nürnberg}, \emph{P.PPL}|pw}{ }15. 5. 22\pend
           \vspace{0.5em}
\pstart
           Herzliche Grüße!\pend
           
\pstart
           Ihr{\\[\baselineskip]}\spacefill\mbox{Arthur}\pend
           \leftskip=0em{}\selectlanguage{ngerman}\endnumbering\briefempfaengerindex{Beer-Hofmann, Richard@\textsc{Beer-Hofmann, Richard}!zzzSchnitzler, Arthur@\emph{von Arthur Schnitzler}!1922-05-151@{15. 5. 1921}|)be}\mylabel{L02382h}  \normalsize

\doendnotes{C}
\bigskip
\vfill

\clearpage

\footnotesize

\lohead{\textsc{register}}

% Definiere theindex-Environment komplett neu ohne reledmac
\makeatletter
\renewenvironment{theindex}{%
  \section*{\indexname}%
  \setlength{\parindent}{0pt}%
  \setlength{\parskip}{0pt plus 0.3pt}%
  \let\item\@idxitem
}{%
  \clearpage
}
\makeatother

\IfFileExists{\jobname-pw.ind}{\input{\jobname-pw.ind}}{}

\end{document}

      