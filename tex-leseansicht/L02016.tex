%% latex-leseansicht-vorspann.tex
%% Vorspann für die Leseansicht.
%% Lädt die gemeinsame Datei latex-vorspann.tex mit nicht gesetztem Schalter.

\newif\ifkorrekturansicht
\korrekturansichtfalse

\input{../tex-inputs/latex-vorspann}


         \renewcommand{\erwaehnteOrte}{Orte: Boulevard Raspail, Deutschland, Dänemark, Havnegade, Hôtel Lutetia, Kopenhagen, Menton, Sternwartestraße, XVIII., Währing}
         \renewcommand{\erwaehnteWerke}{Werke: Der Schleier der Pierrette, Det kgl. Teater [Der Schleier der Pierrette], Før og nu. To tragiske Skaebner}
               \section[Georg Brandes an Arthur Schnitzler, 11. 4. 1911]{ Georg Brandes an Arthur Schnitzler, 11. 4. 1911}\nopagebreak\mylabel{v}\rehead{ }\begin{ledgroupsized}[t]{13cm}\normalsize\beginnumbering \toendnotes[C]{\smallbreak\pagebreak[2]} \Standort{CUL, Schnitzler, B 17.}
\physDesc{Postkarte, 822 Zeichen
\newline{}Handschrift: schwarze Tinte, lateinische Kurrent
\newline{}Versand: Stempel: »\nobreak{}\oindex{Kopenhagen@\textbf{Kopenhagen}|pwk}Kjøbenhavn, 11. 4. 11., 10–11¾E\nobreak{}«.  
\newline{}Ordnung: mit Bleistift von unbekannter Hand nummeriert:
                                    »35« }\buchAbdrucke{\weitereDrucke{Georg Brandes, Arthur Schnitzler: \emph{Ein Briefwechsel}. Hg. Kurt Bergel. Bern: \emph{Francke} 1956, S. 101.} }\toendnotes[C]{\smallbreak}\pstart{}{\pb}Herrn Dr. Arthur
                  Schnitzler\pend{}\pstart{}Sternwartestrasse 71\oindex{XXXX Ortsangabe fehlt|pw}\pend{}\pstart{}Wien XVIII\oindex{XVIII., Waehring@\textbf{XVIII., Währing}|pw}\pend{}{\bigskip}\pstart
           \raggedleft{}{\pb}Kopenhagen\oindex{Kopenhagen@\textbf{Kopenhagen}|pw} (\uline{nicht}{ }Havnegade\oindex{Havnegade@\textbf{Havnegade}|pw})\pend
           \pstart{}Verehrter Herr und Freund.\pend\pstart
           Heute schickte ich Ihnen eine Bagatelle\pwindex{Det kgl. Teater [Der Schleier der Pierrette]20. 03. 1911@\emph{Det kgl. Teater [Der Schleier der Pierrette]} {[}20. 03. 1911{]}|pwv} die ich über Ihr hier aufgeführtes Ballet\pwindex{Schnitzler, Arthur 15.05.1862 – 21.10.1931@\textsc{Schnitzler, Arthur} (15.05.1862 – 21.10.1931), \emph{Schriftsteller, Mediziner}!Schleier der Pierrette1910-01-22@\strich\emph{Der Schleier der Pierrette} {[}1910-01-22{]}|pwv} geschrieben habe und legte eine \label{K_L02016-1v}\edtext{andere Bagatelle}{\lemma{\textnormal{\emph{andere Bagatelle}}}\Cendnote{\textnormal{nicht ermittelt}}}\label{K_L02016-1h} anbei. In deutscher Sprache habe ich
               sonst Nichts. In Deutschland\oindex{Deutschland@\textbf{Deutschland}|pw} habe ich nicht
               einmal mehr einen Verleger. Ich gab in diesen Tagen eine Broschüre\pwindex{Brandes, Georg 04.02.1842 – 19.02.1927@\textsc{Brandes, Georg} (04.02.1842 – 19.02.1927)!Før og nu. To tragiske Skaebner1911@\strich\emph{Før og nu. To tragiske Skaebner} {[}1911{]}|pwv} heraus, aber Sie lesen ja leider
               nicht Dänisch\oindex{Daenemark@\textbf{Dänemark}|pw}.\pend
           \pstart
           Ihr grosser Brief machte mir Freude. Wie schön dass es Ihnen endlich gut geht. Nur
               die Schwerhörigkeit gefällt mir gar nicht. Es ist lumpig von den höheren Mächten, mit
               Solchem sich schadlos zu halten.\pend
           \pstart
           Mir geht es nicht eben strahlend, aber ich bin nicht krank. Adresse von jetzt bis
               weiter Hotel Lutetia\oindex{Hôtel Lutetia@\textbf{Hôtel Lutetia}|pw}, Boulevard Raspail, Paris\oindex{Boulevard Raspail@\textbf{Boulevard Raspail}|pw}.\pend
           \pstart
           Ich drücke Ihre Hand in alter Freundschaft.\pend
           \pstart
           Ihr ergebener{\\[\baselineskip]}\spacefill\mbox{Georg Brandes}\pend
           \leftskip=0em{}
         
         \endnumbering\mylabel{h}\end{ledgroupsized}  \newcommand{\dateiname}{L02016}\newcommand{\titel}{Georg Brandes an Arthur Schnitzler, 11. 4. 1911}\newcommand{\editorInnen}{Martin Anton Müller und Gerd-Hermann Susen}%% latex-leseansicht-abspann.tex
%% Abspann für die Leseansicht.
%% Der Schalter \ifkorrekturansicht ist bereits durch den Vorspann gesetzt.

%% latex-abspann.tex
%% Gemeinsamer Abspann für Korrekturansicht und Leseansicht.
%% Setzt den Schalter \ifkorrekturansicht voraus (gesetzt in den
%% einbindenden Dateien latex-korrekturansicht-abspann.tex bzw.
%% latex-leseansicht-abspann.tex).
%% ---------------------------------------------------------------

\normalsize

% Das esempio-Environment wird nur in der Leseansicht benötigt
\ifkorrekturansicht\else
\newenvironment{esempio}[3]%
{
    \vspace{1.5ex}
    \rlap{\underline{#1}}
    \par
    \setlength{\parindent}{0cm}
    \nopagebreak
    \leftskip=#2cm
    \rightskip=#3cm
}
{
    \par
}
\fi

\doendnotes{C}
\bigskip
\vfill

\clearpage

\footnotesize

\ifkorrekturansicht
  \lohead{\textsc{register}}
\fi

% theindex-Environment neu definieren ohne reledmac
\makeatletter
\renewenvironment{theindex}{%
  \ifkorrekturansicht
    \section*{\indexname}%
  \else
    \subsubsection*{Index der erwähnten Entitäten}%
  \fi
  \setlength{\parindent}{0pt}%
  \setlength{\parskip}{0pt plus 0.3pt}%
  \let\item\@idxitem
}{%
  \ifkorrekturansicht\clearpage\fi
}
\makeatother

\IfFileExists{\jobname-pw.ind}{\input{\jobname-pw.ind}}{}

% Quellenangabe nur in der Leseansicht
\ifkorrekturansicht\else
% Fallback-Definitionen, falls die .tex-Datei \titel etc. nicht gesetzt hat
\providecommand{\titel}{}
\providecommand{\editorInnen}{}
\providecommand{\dateiname}{\jobname}

\vspace{3cm}

\vfill

\footnotesize
\textsc{Quelle}: \titel. Herausgegeben von {\editorInnen}. In: \emph{Arthur Schnitzler: Briefwechsel mit Autorinnen und Autoren}.
 Digitale Edition, https://schnitzler-briefe.acdh.oeaw.ac.at/{\dateiname}.html (Stand \today)
\fi

\end{document}


      