%% latex-leseansicht-vorspann.tex
%% Vorspann für die Leseansicht.
%% Lädt die gemeinsame Datei latex-vorspann.tex mit nicht gesetztem Schalter.

\newif\ifkorrekturansicht
\korrekturansichtfalse

\input{../tex-inputs/latex-vorspann}


\section[Stefan Großmann an Arthur Schnitzler, 30. 8. 1900]{L01069 Stefan Großmann an Arthur Schnitzler, 30. 8. 1900}
\nopagebreak\mylabel{L01069v}
\rehead{ }\normalsize\beginnumbering\briefempfaengerindex{Schnitzler, Arthur@\textsc{Schnitzler, Arthur}!zzzGroßmann, Stefan@\emph{von Stefan Großmann}!1900-08-301@{30. 8. 1900}|(be}
\toendnotes[C]{\smallbreak\pagebreak[2]}
\correspDesc{Versand  durch Stefan Großmann am 30. 8. 1900 in Wien
\newline{}Erhalt  durch Arthur Schnitzler im Zeitraum [30. 8. 1900
                  – 3. 9. 1900?] in Wien}\toendnotes[C]{\smallbreak}
\Standort{CUL, Schnitzler, B 34.}
\physDesc{Brief, 1 Blatt, 2 Seiten, 582 Zeichen
\newline{}Handschrift: schwarze Tinte, deutsche Kurrent
\newline{}Schnitzler: 1) mit Bleistift beschriftet: »leihen«  2) mit rotem Buntstift eine Unterstreichung
\newline{}Ordnung: mit Bleistift von unbekannter Hand nummeriert:
                                 »2« }
\pstart
           \raggedleft{}{\pb}Wien\oindex{Wien@\textbf{Wien}, \emph{Verwaltungsgebiet}|pw}, den 30. Auguſt 1900\pend
           
\pstart{}ſehr geehrter Herr Doctor,\pend\vspace{0.5em}
\pstart
           Schon seit einiger Zeit möchte ich Sie, verehrter Herr, bitten, mir – wenn es Ihnen
               möglich iſt – ein \introOben{}etwa\introOben{} überflüſſiges Exemplar des »\textsc{Reigen}\pwindex{Schnitzler, Arthur 15.\,5.\,1862 Wien – 21.\,10.\,1931 ebd.@\textsc{Schnitzler, Arthur} (15.\,5.\,1862 Wien – 21.\,10.\,1931 ebd.), \emph{Schriftsteller, Mediziner}!Reigen. Zehn Dialoge@\strich\emph{Reigen. Zehn Dialoge}|pw}« gütigſt leihen oder{ }ſchenken zu wollen.\pend
           
\pstart
           Ich fürchte, daſs es mir im Moment nicht möglich{ }ſein wird Ihren Glauben an meinen
               einſeitigen aeſthetischen Doctrinarismus zu erſchüttern und beſchränke mich daher
               Ihnen zu{ }ſagen, daſs ich Ihnen für die Zuſendung des Buches, {\pb}auf deſſen Lecture ich{ }ſchon{ }ſehr geſpannt
               bin, \uline{aufrichtig} und \uline{herzlichſt} danke.\pend
           
\pstart
           Sehr ergeben:{\\[\baselineskip]}\spacefill\mbox{StefanGroßmann}\pend
           \leftskip=0em{}
\pstart
           \noindent{}\raggedleft{}VIII. \textsc{Langega}ſſ\textsc{e} 52{\\}Th. 12\oindex{Wien@\textbf{Wien}!VIII., Josefstadt@\textbf{VIII., Josefstadt}!Lange Gasse@\textbf{Lange Gasse}, \emph{Straße}|pw}\pend
           \selectlanguage{ngerman}\endnumbering\briefempfaengerindex{Schnitzler, Arthur@\textsc{Schnitzler, Arthur}!zzzGroßmann, Stefan@\emph{von Stefan Großmann}!1900-08-301@{30. 8. 1900}|)be}\mylabel{L01069h}  \newcommand{\dateiname}{L01069}\newcommand{\titel}{Stefan Großmann an Arthur Schnitzler, 30. 8. 1900}\newcommand{\editorInnen}{Herausgegeben von Martin Anton Müller}%% latex-leseansicht-abspann.tex
%% Abspann für die Leseansicht.
%% Der Schalter \ifkorrekturansicht ist bereits durch den Vorspann gesetzt.

%% latex-abspann.tex
%% Gemeinsamer Abspann für Korrekturansicht und Leseansicht.
%% Setzt den Schalter \ifkorrekturansicht voraus (gesetzt in den
%% einbindenden Dateien latex-korrekturansicht-abspann.tex bzw.
%% latex-leseansicht-abspann.tex).
%% ---------------------------------------------------------------

\normalsize

% Das esempio-Environment wird nur in der Leseansicht benötigt
\ifkorrekturansicht\else
\newenvironment{esempio}[3]%
{
    \vspace{1.5ex}
    \rlap{\underline{#1}}
    \par
    \setlength{\parindent}{0cm}
    \nopagebreak
    \leftskip=#2cm
    \rightskip=#3cm
}
{
    \par
}
\fi

\doendnotes{C}
\bigskip
\vfill

\clearpage

\footnotesize

\ifkorrekturansicht
  \lohead{\textsc{register}}
\fi

% theindex-Environment neu definieren ohne reledmac
\makeatletter
\renewenvironment{theindex}{%
  \ifkorrekturansicht
    \section*{\indexname}%
  \else
    \subsubsection*{Index der erwähnten Entitäten}%
  \fi
  \setlength{\parindent}{0pt}%
  \setlength{\parskip}{0pt plus 0.3pt}%
  \let\item\@idxitem
}{%
  \ifkorrekturansicht\clearpage\fi
}
\makeatother

\IfFileExists{\jobname-pw.ind}{\input{\jobname-pw.ind}}{}

% Quellenangabe nur in der Leseansicht
\ifkorrekturansicht\else
% Fallback-Definitionen, falls die .tex-Datei \titel etc. nicht gesetzt hat
\providecommand{\titel}{}
\providecommand{\editorInnen}{}
\providecommand{\dateiname}{\jobname}

\vspace{3cm}

\vfill

\footnotesize
\textsc{Quelle}: \titel. Herausgegeben von {\editorInnen}. In: \emph{Arthur Schnitzler: Briefwechsel mit Autorinnen und Autoren}.
 Digitale Edition, https://schnitzler-briefe.acdh.oeaw.ac.at/{\dateiname}.html (Stand \today)
\fi

\end{document}


