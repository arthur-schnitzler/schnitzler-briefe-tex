%% latex-korrekturansicht-vorspann.tex
%% Vorspann für die Korrekturansicht.
%% Lädt die gemeinsame Datei latex-vorspann.tex mit gesetztem Schalter.

\newif\ifkorrekturansicht
\korrekturansichttrue

\input{../tex-inputs/latex-vorspann}


\section[Hermann Bahr an Arthur Schnitzler, 15. 11. 1903]{L01345 Hermann Bahr an Arthur Schnitzler, 15. 11. 1903}
\nopagebreak\mylabel{L01345v}
\rehead{ }\normalsize\beginnumbering\briefempfaengerindex{Schnitzler, Arthur@\textsc{Schnitzler, Arthur}!zzzBahr, Hermann@\emph{von Hermann Bahr}!1903-11-151@{15. 11. 1903}|(be}
\toendnotes[C]{\smallbreak\pagebreak[2]}\Standort{CUL, Schnitzler, B 5b.}
\physDesc{Postkarte, 413 Zeichen
\newline{}Handschrift: schwarze Tinte, deutsche Kurrent
\newline{}Versand: 1) Stempel: »\nobreak{}\oindex{XIII., Hietzing@\textbf{XIII., Hietzing}, \emph{A.ADM3}|pwk}Wien 13/7, 15. 11{[}.{]} 03, 12–1M\nobreak{}«.   2) Stempel: »\nobreak{}\oindex{XVIII., Waehring@\textbf{XVIII., Währing}, \emph{A.ADM3}|pwk}18/1 Wien, 16. 11. 03, 8.V, Bestellt\nobreak{}«. 
\newline{}Ordnung: mit Bleistift von unbekannter Hand nummeriert:
                                    »104« }
\buchAbdrucke{\weitereDrucke{Hermann Bahr, Arthur Schnitzler: \emph{Briefwechsel, Aufzeichnungen, Dokumente (1891–1931)}. Göttingen: \emph{Wallstein} 2018, S. 282.} }\toendnotes[C]{\smallbreak}\pstart{}{\pb}Herrn \textsc{D\textsuperscript{r} Arthur Schnitzler}\pend{}\pstart{}Wien XVIII\oindex{XVIII., Waehring@\textbf{XVIII., Währing}, \emph{A.ADM3}|pw}\pend{}\pstart{}Spöttelgaſſe 7\oindex{Edmund-Weiss-Gasse 7@\textbf{Edmund-Weiß-Gasse 7}, \emph{Wohngebäude (K.WHS)}|pw}\pend{}{\bigskip}\vspace{1em}
\pstart
           \raggedleft{}{\pb}15. 11. 03\pend
           \vspace{0.5em}
\pstart
           Danke ſehr, lieber Arthur.  Der Berliner Börſen Courier\orgindex{Berliner Boersen-Courier@Berliner Börsen-Courier|pw} hat ſchon abgelehnt u. ich habe wenig Hoffnung.
               Dieſe Bande!\pend
           
\pstart
           Hugo\pwindex{Hofmannsthal, Hugo von 1874-02-01 – 1929-07-15@\textsc{Hofmannsthal, Hugo von} (1874-02-01 – 1929-07-15), \emph{Schriftsteller/Schriftstellerin}|pw}{ }ſchreibt mir, Dein neues Stück\pwindex{einsame Weg. Schauspiel in fuenf Akten@\emph{Der einsame Weg. Schauspiel in fünf Akten}|pwv}{ }ſei »prachtvoll«. Ich freu mich ſehr u. wünſch Dir
               herzlichſt Glück.\pend
           
\pstart
           Brahm\pwindex{Brahm, Otto 05.02.1856 – 28.11.1912@\textsc{Brahm, Otto} (05.02.1856 – 28.11.1912), \emph{Theaterleiter/Theaterleiterin, Regisseur/Regisseurin}|pw} hat meine \label{K_L01345-1v}\edtext{Première\eventindex{Deutsches Theater Berlin@\textbf{Deutsches Theater Berlin}!Urauffuehrung von Der Meister, 12.12.1903@Uraufführung von Der Meister, 12.12.1903|pwv}}{\lemma{\textnormal{\emph{Première}}}\Cendnote{\textnormal{Am 12. 12. 1903 fand am Berlin\oindex{Berlin@\textbf{Berlin}, \emph{P.PPLC}|pwk}er
                   \emph{Deutschen Theater}\orgindex{Deutsches Theater Berlin@Deutsches Theater Berlin|pwk} die Uraufführung\eventindex{Deutsches Theater Berlin@\textbf{Deutsches Theater Berlin}!Urauffuehrung von Der Meister, 12.12.1903@Uraufführung von Der Meister, 12.12.1903|pwkv} von 
                  von \emph{Der
                     Meister}\pwindex{Meister. Komoedie in drei Akten@\emph{Der Meister. Komödie in drei Akten}|pwk} statt.}}}\label{K_L01345-1} auf den 12. Dezember angeſetzt. Warum plötzlich dieſe Eile, weiß
               ich nicht. Er kommt Montag im \textsc{Imperial}\oindex{Hotel Imperial@\textbf{Hotel Imperial}, \emph{Hotel (K.HTL)}|pw} an.\pend
           
\pstart
           Herzlichſt{\\[\baselineskip]}Dein{\\[\baselineskip]}\spacefill\mbox{H.}\pend
           \leftskip=0em{}\selectlanguage{ngerman}\endnumbering\briefempfaengerindex{Schnitzler, Arthur@\textsc{Schnitzler, Arthur}!zzzBahr, Hermann@\emph{von Hermann Bahr}!1903-11-151@{15. 11. 1903}|)be}\mylabel{L01345h}  \normalsize

\doendnotes{C}
\bigskip
\vfill

\clearpage

\footnotesize

\lohead{\textsc{register}}

% Definiere theindex-Environment komplett neu ohne reledmac
\makeatletter
\renewenvironment{theindex}{%
  \section*{\indexname}%
  \setlength{\parindent}{0pt}%
  \setlength{\parskip}{0pt plus 0.3pt}%
  \let\item\@idxitem
}{%
  \clearpage
}
\makeatother

\IfFileExists{\jobname-pw.ind}{\input{\jobname-pw.ind}}{}

\end{document}

      