%% latex-leseansicht-vorspann.tex
%% Vorspann für die Leseansicht.
%% Lädt die gemeinsame Datei latex-vorspann.tex mit nicht gesetztem Schalter.

\newif\ifkorrekturansicht
\korrekturansichtfalse

\input{../tex-inputs/latex-vorspann}


         
         \newcommand{\erwaehntePersonen}{Personen: Wilhelm Bölsche, Samuel Fischer}
         \newcommand{\erwaehnteInstitutionen}{Institutionen: Lessing-Theater}
         \newcommand{\erwaehnteOrte}{Orte: Berlin, Grillparzerstraße, Wien}
         \newcommand{\erwaehnteWerke}{Werke: Das Märchen. Schauspiel in drei Aufzügen, Freie Bühne für den Entwickelungskampf der Zeit}
               \section[Arthur Schnitzler an Samuel Fischer, 25. 7. 1893]{ Arthur Schnitzler an Samuel Fischer, 25. 7. 1893}\nopagebreak\mylabel{v}\rehead{ }\begin{ledgroupsized}[t]{13cm}\normalsize\beginnumbering \toendnotes[C]{\smallbreak\pagebreak[2]} \Standort{Wrocław, Biblioteka Uniwersytecka, Böl.Nau 417.}
\physDesc{Brief, 1 Blatt (Briefpapier mit Trauerrand), 4 Seiten
\newline{}Handschrift: schwarze Tinte, deutsche Kurrent}\buchAbdrucke{\weitereDrucke{Wilhelm Bölsche: \emph{Briefwechsel. Mit Autoren der Freien Bühne}. Hg. Gerd-Hermann Susen. Berlin: \emph{Weidler} 2010, S. 693 (Werke und Briefe. Wissenschaftliche Ausgabe, Briefe I).} }\toendnotes[C]{\smallbreak}\pstart{}{\pb}Sehr geehrter Herr,\pend\pstart
           über \label{K_L00242_1v}\edtext{Aufforderung}{\lemma{\textnormal{\emph{Aufforderung}}}\Cendnote{\textnormal{Dieser Brief ist im Nachlass Bölsche\pwindex{Boelsche, Wilhelm 02.01.1861 – 31.08.1939@\textsc{Bölsche, Wilhelm} (02.01.1861 – 31.08.1939), \emph{Schriftsteller, Publizist}|pwk}s überliefert, S. Fischer\pwindex{Fischer, Samuel 24.12.1859 – 15.10.1934@\textsc{Fischer, Samuel} (24.12.1859 – 15.10.1934), \emph{Verleger}|pwk} hat ihn also an diesen
                        weitergegeben.}}}\label{K_L00242_1h} des Herrn \textsc{Dr. W. Bölsche}\pwindex{Boelsche, Wilhelm 02.01.1861 – 31.08.1939@\textsc{Bölsche, Wilhelm} (02.01.1861 – 31.08.1939), \emph{Schriftsteller, Publizist}|pw}{ }ſende ich Ihnen \uline{Das Märchen}\pwindex{Schnitzler, Arthur 15.05.1862 – 21.10.1931@\textsc{Schnitzler, Arthur} (15.05.1862 – 21.10.1931), \emph{Schriftsteller, Mediziner}!Maerchen. Schauspiel in drei Aufzuegen1893-12-01@\strich\emph{Das Märchen. Schauspiel in drei Aufzügen} {[}1893-12-01{]}|pw} zu. Wollen Sie mir gütigſt bald mittheilen, wann eine eventuelle
                    Veröffentlichung in der »\textsc{Freien Bühne}\pwindex{Freie Buehne fuer den Entwickelungskampf der Zeit1892 – 1893@\emph{Freie Bühne für den Entwickelungskampf der Zeit} {[}1892 – 1893{]}|pw}« {\pb}beginnen kann. Ich ſende Ihnen das Manuscript\pwindex{Schnitzler, Arthur 15.05.1862 – 21.10.1931@\textsc{Schnitzler, Arthur} (15.05.1862 – 21.10.1931), \emph{Schriftsteller, Mediziner}!Maerchen. Schauspiel in drei Aufzuegen1893-12-01@\strich\emph{Das Märchen. Schauspiel in drei Aufzügen} {[}1893-12-01{]}|pwv}, ſa{\geminationm}t den
                    Zuſätzen und Anmerkungen, wie ich ſie für eine bevorſtehende Aufführg am Leſſing Theater\orgindex{Lessing-Theater@Lessing-Theater|pw} angebracht habe. Nur wünſchte
                    ich, daſs die Schilderungen der Perſonen, wie ſie ſich auf den erſten 2
                    beigefügten Blättern befinden, im Druck wegbleiben.\pend
           \pstart
           {\pb}Um Correcturen erſuche ich dringend.\pend
           \pstart
           Ich ſehe Ihrer werthen Entſcheidung ſowie der Angabe der Bedingungen, unter
                    welchen Sie das Stück\pwindex{Schnitzler, Arthur 15.05.1862 – 21.10.1931@\textsc{Schnitzler, Arthur} (15.05.1862 – 21.10.1931), \emph{Schriftsteller, Mediziner}!Maerchen. Schauspiel in drei Aufzuegen1893-12-01@\strich\emph{Das Märchen. Schauspiel in drei Aufzügen} {[}1893-12-01{]}|pw} nehmen wollen, mit
                    lebhaftem Intereſſe entgegen, und möchte auch gern Ihre Äußerung über eine
                    event. Buchausgabe vernehmen.\pend
           \pstart
           – In der Hoffnung, daſs {\pb}Sie mich nicht zu lange auf
                    Antwort warten laſſen, bin ich in beſonderer Hochachtg\pend
           \pstart
           Ihr erg\textcolor{gray}{ebener}{\\[\baselineskip]}\spacefill\mbox{Dr. Arthur Schnitzler}\pend
           \leftskip=0em{}\pstart
           \textsc{Wien\oindex{Wien@\textbf{Wien}|pw}}, 25. Juli 93{\\}\textsc{I. Grillparzerstraße 7}\oindex{Grillparzerstrasse@\textbf{Grillparzerstraße}|pw}\pend
           
         
         \endnumbering\mylabel{h}\end{ledgroupsized}  \newcommand{\dateiname}{L00242}\newcommand{\titel}{Arthur Schnitzler an Samuel Fischer, 25. 7. 1893}\newcommand{\editorInnen}{Martin Anton Müller und Gerd-Hermann Susen}%% latex-leseansicht-abspann.tex
%% Abspann für die Leseansicht.
%% Der Schalter \ifkorrekturansicht ist bereits durch den Vorspann gesetzt.

%% latex-abspann.tex
%% Gemeinsamer Abspann für Korrekturansicht und Leseansicht.
%% Setzt den Schalter \ifkorrekturansicht voraus (gesetzt in den
%% einbindenden Dateien latex-korrekturansicht-abspann.tex bzw.
%% latex-leseansicht-abspann.tex).
%% ---------------------------------------------------------------

\normalsize

% Das esempio-Environment wird nur in der Leseansicht benötigt
\ifkorrekturansicht\else
\newenvironment{esempio}[3]%
{
    \vspace{1.5ex}
    \rlap{\underline{#1}}
    \par
    \setlength{\parindent}{0cm}
    \nopagebreak
    \leftskip=#2cm
    \rightskip=#3cm
}
{
    \par
}
\fi

\doendnotes{C}
\bigskip
\vfill

\clearpage

\footnotesize

\ifkorrekturansicht
  \lohead{\textsc{register}}
\fi

% theindex-Environment neu definieren ohne reledmac
\makeatletter
\renewenvironment{theindex}{%
  \ifkorrekturansicht
    \section*{\indexname}%
  \else
    \subsubsection*{Index der erwähnten Entitäten}%
  \fi
  \setlength{\parindent}{0pt}%
  \setlength{\parskip}{0pt plus 0.3pt}%
  \let\item\@idxitem
}{%
  \ifkorrekturansicht\clearpage\fi
}
\makeatother

\IfFileExists{\jobname-pw.ind}{\input{\jobname-pw.ind}}{}

% Quellenangabe nur in der Leseansicht
\ifkorrekturansicht\else
% Fallback-Definitionen, falls die .tex-Datei \titel etc. nicht gesetzt hat
\providecommand{\titel}{}
\providecommand{\editorInnen}{}
\providecommand{\dateiname}{\jobname}

\vspace{3cm}

\vfill

\footnotesize
\textsc{Quelle}: \titel. Herausgegeben von {\editorInnen}. In: \emph{Arthur Schnitzler: Briefwechsel mit Autorinnen und Autoren}.
 Digitale Edition, https://schnitzler-briefe.acdh.oeaw.ac.at/{\dateiname}.html (Stand \today)
\fi

\end{document}


      