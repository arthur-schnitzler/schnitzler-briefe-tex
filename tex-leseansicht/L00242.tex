%% latex-korrekturansicht-vorspann.tex
%% Vorspann für die Korrekturansicht.
%% Lädt die gemeinsame Datei latex-vorspann.tex mit gesetztem Schalter.

\newif\ifkorrekturansicht
\korrekturansichttrue

\input{../tex-inputs/latex-vorspann}


\section[Arthur Schnitzler an Samuel Fischer, 25. 7. 1893]{L00242 Arthur Schnitzler an Samuel Fischer, 25. 7. 1893}
\nopagebreak\mylabel{L00242v}
\rehead{ }\normalsize\beginnumbering\briefempfaengerindex{Fischer, Samuel@\textsc{Fischer, Samuel}!zzzSchnitzler, Arthur@\emph{von Arthur Schnitzler}!1893-07-251@{25. 7. 1893}|(be}
\toendnotes[C]{\smallbreak\pagebreak[2]}\Standort{Wrocław, Biblioteka Uniwersytecka, Böl.Nau 417.}
\physDesc{Brief, 1 Blatt, 4 Seiten, 914 Zeichen (Briefpapier mit Trauerrand)
\newline{}Handschrift: schwarze Tinte, deutsche Kurrent}
\buchAbdrucke{\weitereDrucke{Wilhelm Bölsche: \emph{Briefwechsel. Mit Autoren der Freien Bühne}. Berlin: \emph{Weidler} 2010, S. 693.} }\toendnotes[C]{\smallbreak}
\pstart{}{\pb}Sehr geehrter Herr,\pend\vspace{0.5em}
\pstart
           über \label{K_L00242-1v}\edtext{Aufforderung}{\lemma{\textnormal{\emph{Aufforderung}}}\Cendnote{\textnormal{Dieser Brief ist im Nachlass Bölsches\pwindex{Boelsche, Wilhelm 02.01.1861 – 31.08.1939@\textsc{Bölsche, Wilhelm} (02.01.1861 – 31.08.1939), \emph{Schriftsteller/Schriftstellerin, Publizist/Publizistin}|pwk} überliefert, S. Fischer\pwindex{Fischer, Samuel 24.12.1859 – 15.10.1934@\textsc{Fischer, Samuel} (24.12.1859 – 15.10.1934), \emph{Verleger/Verlegerin}|pwk} hat ihn also an diesen weitergegeben.}}}\label{K_L00242-1} des
               Herrn \textsc{Dr. W. Bölsche}\pwindex{Boelsche, Wilhelm 02.01.1861 – 31.08.1939@\textsc{Bölsche, Wilhelm} (02.01.1861 – 31.08.1939), \emph{Schriftsteller/Schriftstellerin, Publizist/Publizistin}|pw}{ }ſende ich Ihnen \uline{Das Märchen}\pwindex{Maerchen. Schauspiel in drei Aufzuegen@\emph{Das Märchen. Schauspiel in drei Aufzügen}|pw} zu. Wollen Sie mir gütigſt bald mittheilen, wann eine eventuelle
               Veröffentlichung in der »\textsc{Freien Bühne}\pwindex{Freie Buehne fuer den Entwickelungskampf der Zeit@\emph{Freie Bühne für den Entwickelungskampf der Zeit}|pw}« {\pb}beginnen kann. Ich ſende Ihnen das Manuscript\pwindex{Maerchen. Schauspiel in drei Aufzuegen@\emph{Das Märchen. Schauspiel in drei Aufzügen}|pwv}, ſa{\geminationm}t den Zuſätzen und Anmerkungen, wie ich ſie für eine bevorſtehende Aufführg am Leſſing Theater\orgindex{Lessing-Theater@Lessing-Theater|pw} angebracht habe. Nur wünſchte ich,
               daſs die Schilderungen der Perſonen, wie ſie ſich auf den erſten 2 beigefügten
               Blättern befinden, im Druck wegbleiben.\pend
           
\pstart
           {\pb}Um Correcturen erſuche ich dringend.\pend
           
\pstart
           Ich ſehe Ihrer werthen Entſcheidung ſowie der Angabe der Bedingungen, unter welchen
               Sie das Stück\pwindex{Maerchen. Schauspiel in drei Aufzuegen@\emph{Das Märchen. Schauspiel in drei Aufzügen}|pw} nehmen wollen, mit lebhaftem
               Intereſſe entgegen, und möchte auch gern Ihre Äußerung über eine event. Buchausgabe
               vernehmen.\pend
           
\pstart
           – In der Hoffnung, daſs {\pb}Sie mich nicht zu lange auf Antwort
               warten laſſen, bin ich in beſonderer Hochachtg\pend
           
\pstart
           Ihr erg\textcolor{gray}{ebener}{\\[\baselineskip]}\spacefill\mbox{Dr. Arthur Schnitzler}\pend
           \leftskip=0em{}
\pstart
           \textsc{Wien\oindex{Wien@\textbf{Wien}, \emph{A.ADM2}|pw}}, 25. Juli 93{\\}\textsc{I. Grillparzerstraße 7}\oindex{Grillparzerstrasse@\textbf{Grillparzerstraße}, \emph{R.ST}|pw}\pend
           \selectlanguage{ngerman}\endnumbering\briefempfaengerindex{Fischer, Samuel@\textsc{Fischer, Samuel}!zzzSchnitzler, Arthur@\emph{von Arthur Schnitzler}!1893-07-251@{25. 7. 1893}|)be}\mylabel{L00242h}  \normalsize

\doendnotes{C}
\bigskip
\vfill

\clearpage

\footnotesize

\lohead{\textsc{register}}

% Definiere theindex-Environment komplett neu ohne reledmac
\makeatletter
\renewenvironment{theindex}{%
  \section*{\indexname}%
  \setlength{\parindent}{0pt}%
  \setlength{\parskip}{0pt plus 0.3pt}%
  \let\item\@idxitem
}{%
  \clearpage
}
\makeatother

\IfFileExists{\jobname-pw.ind}{\input{\jobname-pw.ind}}{}

\end{document}

      