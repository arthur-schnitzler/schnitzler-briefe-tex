%% latex-leseansicht-vorspann.tex
%% Vorspann für die Leseansicht.
%% Lädt die gemeinsame Datei latex-vorspann.tex mit nicht gesetztem Schalter.

\newif\ifkorrekturansicht
\korrekturansichtfalse

\input{../tex-inputs/latex-vorspann}


\section[Arthur Schnitzler an Samuel Fischer, 25. 7. 1893]{L00242 Arthur Schnitzler an Samuel Fischer, 25. 7. 1893}
\nopagebreak\mylabel{L00242v}
\rehead{ }\normalsize\beginnumbering\briefempfaengerindex{Fischer, Samuel@\textsc{Fischer, Samuel}!zzzSchnitzler, Arthur@\emph{von Arthur Schnitzler}!1893-07-251@{25. 7. 1893}|(be}
\toendnotes[C]{\smallbreak\pagebreak[2]}
\correspDesc{Versand  durch Arthur Schnitzler am 25. 7. 1893 in Wien
\newline{}Erhalt  durch Samuel Fischer im Zeitraum [26. 7. 1893
                  – 30. 7. 1893?] in Berlin}\toendnotes[C]{\smallbreak}
\Standort{Wrocław, Biblioteka Uniwersytecka, Böl.Nau 417.}
\physDesc{Brief, 1 Blatt, 4 Seiten, 914 Zeichen (Briefpapier mit Trauerrand)
\newline{}Handschrift: schwarze Tinte, deutsche Kurrent}
\buchAbdrucke{\weitereDrucke{Wilhelm Bölsche: \emph{Briefwechsel. Mit Autoren der Freien Bühne}. Herausgegeben von Gerd-Hermann Susen. Berlin: \emph{Weidler} 2010, S. 693 (Werke und Briefe. Wissenschaftliche Ausgabe, Briefe I).} }\toendnotes[C]{\smallbreak}
\pstart{}{\pb}Sehr geehrter Herr,\pend\vspace{0.5em}
\pstart
           über \label{K_L00242-1v}\edtext{Aufforderung}{\lemma{\textnormal{\emph{Aufforderung}}}\Cendnote{\textnormal{Dieser Brief ist im Nachlass Bölsches\pwindex{Bölsche, Wilhelm 2.\,1.\,1861 Köln – 31.\,8.\,1939 Szklarska Poręba@\textsc{Bölsche, Wilhelm} (2.\,1.\,1861 Köln – 31.\,8.\,1939 Szklarska Poręba), \emph{Schriftsteller, Publizist}|pwk} überliefert, S. Fischer\pwindex{Fischer, Samuel 24.\,12.\,1859 Liptovský Mikuláš – 15.\,10.\,1934 Berlin@\textsc{Fischer, Samuel} (24.\,12.\,1859 Liptovský Mikuláš – 15.\,10.\,1934 Berlin), \emph{Verleger}|pwk} hat ihn also an diesen weitergegeben.}}}\label{K_L00242-1} des
               Herrn \textsc{Dr. W. Bölsche}\pwindex{Bölsche, Wilhelm 2.\,1.\,1861 Köln – 31.\,8.\,1939 Szklarska Poręba@\textsc{Bölsche, Wilhelm} (2.\,1.\,1861 Köln – 31.\,8.\,1939 Szklarska Poręba), \emph{Schriftsteller, Publizist}|pw}{ }ſende ich Ihnen \uline{Das Märchen}\pwindex{Schnitzler, Arthur 15.\,5.\,1862 Wien – 21.\,10.\,1931 ebd.@\textsc{Schnitzler, Arthur} (15.\,5.\,1862 Wien – 21.\,10.\,1931 ebd.), \emph{Schriftsteller, Mediziner}!Märchen. Schauspiel in drei Aufzügen@\strich\emph{Das Märchen. Schauspiel in drei Aufzügen}|pw} zu. Wollen Sie mir gütigſt bald mittheilen, wann eine eventuelle
               Veröffentlichung in der »\textsc{Freien Bühne}\pwindex{Freie Bühne für den Entwickelungskampf der Zeit@\emph{Freie Bühne für den Entwickelungskampf der Zeit}|pw}« {\pb}beginnen kann. Ich{ }ſende Ihnen das Manuscript\pwindex{Schnitzler, Arthur 15.\,5.\,1862 Wien – 21.\,10.\,1931 ebd.@\textsc{Schnitzler, Arthur} (15.\,5.\,1862 Wien – 21.\,10.\,1931 ebd.), \emph{Schriftsteller, Mediziner}!Märchen. Schauspiel in drei Aufzügen@\strich\emph{Das Märchen. Schauspiel in drei Aufzügen}|pwv},{ }ſa{\geminationm}t den Zuſätzen und Anmerkungen, wie ich{ }ſie für eine bevorſtehende Aufführg am Leſſing Theater\orgindex{Lessing-Theater@Lessing-Theater|pw} angebracht habe. Nur wünſchte ich,
               daſs die Schilderungen der Perſonen, wie{ }ſie{ }ſich auf den erſten 2 beigefügten
               Blättern befinden, im Druck wegbleiben.\pend
           
\pstart
           {\pb}Um Correcturen erſuche ich dringend.\pend
           
\pstart
           Ich{ }ſehe Ihrer werthen Entſcheidung{ }ſowie der Angabe der Bedingungen, unter welchen
               Sie das Stück\pwindex{Schnitzler, Arthur 15.\,5.\,1862 Wien – 21.\,10.\,1931 ebd.@\textsc{Schnitzler, Arthur} (15.\,5.\,1862 Wien – 21.\,10.\,1931 ebd.), \emph{Schriftsteller, Mediziner}!Märchen. Schauspiel in drei Aufzügen@\strich\emph{Das Märchen. Schauspiel in drei Aufzügen}|pw} nehmen wollen, mit lebhaftem
               Intereſſe entgegen, und möchte auch gern Ihre Äußerung über eine event. Buchausgabe
               vernehmen.\pend
           
\pstart
           – In der Hoffnung, daſs {\pb}Sie mich nicht zu lange auf Antwort
               warten laſſen, bin ich in beſonderer Hochachtg\pend
           
\pstart
           Ihr erg\textcolor{gray}{ebener}{\\[\baselineskip]}\spacefill\mbox{Dr. Arthur Schnitzler}\pend
           \leftskip=0em{}
\pstart
           \textsc{Wien\oindex{Wien@\textbf{Wien}, \emph{Verwaltungsgebiet}|pw}}, 25. Juli 93{\\}\textsc{I. Grillparzerstraße 7}\oindex{Wien@\textbf{Wien}!I., Innere Stadt@\textbf{I., Innere Stadt}!Grillparzerstraße@\textbf{Grillparzerstraße}, \emph{Straße}|pw}\pend
           \selectlanguage{ngerman}\endnumbering\briefempfaengerindex{Fischer, Samuel@\textsc{Fischer, Samuel}!zzzSchnitzler, Arthur@\emph{von Arthur Schnitzler}!1893-07-251@{25. 7. 1893}|)be}\mylabel{L00242h}  \newcommand{\dateiname}{L00242}\newcommand{\titel}{Arthur Schnitzler an Samuel Fischer, 25. 7. 1893}\newcommand{\editorInnen}{Martin Anton Müller und Gerd-Hermann Susen}%% latex-leseansicht-abspann.tex
%% Abspann für die Leseansicht.
%% Der Schalter \ifkorrekturansicht ist bereits durch den Vorspann gesetzt.

%% latex-abspann.tex
%% Gemeinsamer Abspann für Korrekturansicht und Leseansicht.
%% Setzt den Schalter \ifkorrekturansicht voraus (gesetzt in den
%% einbindenden Dateien latex-korrekturansicht-abspann.tex bzw.
%% latex-leseansicht-abspann.tex).
%% ---------------------------------------------------------------

\normalsize

% Das esempio-Environment wird nur in der Leseansicht benötigt
\ifkorrekturansicht\else
\newenvironment{esempio}[3]%
{
    \vspace{1.5ex}
    \rlap{\underline{#1}}
    \par
    \setlength{\parindent}{0cm}
    \nopagebreak
    \leftskip=#2cm
    \rightskip=#3cm
}
{
    \par
}
\fi

\doendnotes{C}
\bigskip
\vfill

\clearpage

\footnotesize

\ifkorrekturansicht
  \lohead{\textsc{register}}
\fi

% theindex-Environment neu definieren ohne reledmac
\makeatletter
\renewenvironment{theindex}{%
  \ifkorrekturansicht
    \section*{\indexname}%
  \else
    \subsubsection*{Index der erwähnten Entitäten}%
  \fi
  \setlength{\parindent}{0pt}%
  \setlength{\parskip}{0pt plus 0.3pt}%
  \let\item\@idxitem
}{%
  \ifkorrekturansicht\clearpage\fi
}
\makeatother

\IfFileExists{\jobname-pw.ind}{\input{\jobname-pw.ind}}{}

% Quellenangabe nur in der Leseansicht
\ifkorrekturansicht\else
% Fallback-Definitionen, falls die .tex-Datei \titel etc. nicht gesetzt hat
\providecommand{\titel}{}
\providecommand{\editorInnen}{}
\providecommand{\dateiname}{\jobname}

\vspace{3cm}

\vfill

\footnotesize
\textsc{Quelle}: \titel. Herausgegeben von {\editorInnen}. In: \emph{Arthur Schnitzler: Briefwechsel mit Autorinnen und Autoren}.
 Digitale Edition, https://schnitzler-briefe.acdh.oeaw.ac.at/{\dateiname}.html (Stand \today)
\fi

\end{document}


