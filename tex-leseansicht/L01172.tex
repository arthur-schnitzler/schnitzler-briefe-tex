%% latex-korrekturansicht-vorspann.tex
%% Vorspann für die Korrekturansicht.
%% Lädt die gemeinsame Datei latex-vorspann.tex mit gesetztem Schalter.

\newif\ifkorrekturansicht
\korrekturansichttrue

\input{../tex-inputs/latex-vorspann}


\section[Hermann Bahr an Arthur Schnitzler, 12. 9. 1901]{L01172 Hermann Bahr an Arthur Schnitzler, 12. 9. 1901}
\nopagebreak\mylabel{L01172v}
\rehead{ }\normalsize\beginnumbering\briefempfaengerindex{Schnitzler, Arthur@\textsc{Schnitzler, Arthur}!zzzBahr, Hermann@\emph{von Hermann Bahr}!1901-09-122@{12. 9. 1901}|(be}
\toendnotes[C]{\smallbreak\pagebreak[2]}\Standort{CUL, Schnitzler, B 5b.}
\physDesc{Brief, 1 Blatt, 2 Seiten, 690 Zeichen
\newline{}Handschrift: schwarze Tinte, deutsche Kurrent
\newline{}Schnitzler: mit Bleistift die Jahreszahl »901« ergänzt 
\newline{}Ordnung: mit Bleistift von unbekannter Hand nummeriert:
                                    »79« }
\buchAbdrucke{\weitereDrucke{Hermann Bahr, Arthur Schnitzler: \emph{Briefwechsel, Aufzeichnungen, Dokumente (1891–1931)}. Göttingen: \emph{Wallstein} 2018, S. 214.} }\toendnotes[C]{\smallbreak}
\pstart
           \centering{}{\pb}\textcolor{gray}{\textbf{Redaktion des Neuen Wiener Tagblatt\orgindex{Neues Wiener Tagblatt@Neues Wiener Tagblatt|pw}}}\pend
           
\pstart
           \centering{}\textcolor{gray}{\textbf{\textsc{Wien, I., Rothenturmstrasse,
                        Steyrerhof\oindex{Steyrerhof@\textbf{Steyrerhof}, \emph{Gebäude (K.GBD)}|pw}.}}}\pend
           
\pstart
           \centering{}\textcolor{gray}{\textbf{Telegramm-Adresse: Tagblatt\orgindex{Neues Wiener Tagblatt@Neues Wiener Tagblatt|pw}, Steyrerhof, Wien\oindex{Steyrerhof@\textbf{Steyrerhof}, \emph{Gebäude (K.GBD)}|pw}. –
                     Telephon Nr. 384. Staats-Telephon Nr. 36.}}\pend
           
\pstart
           \raggedleft{}12. 9.\pend
           
\pstart\center{}Lieber Arthur!\pend\vspace{0.5em}
\pstart
           Ich habe Deine Stücke\pwindex{Lebendige Stunden. Vier Einakter@\emph{Lebendige Stunden. Vier Einakter}|pwv}\pwindex{Frau mit dem Dolche@\emph{Die Frau mit dem Dolche}|pwv} geſtern abends bekommen, nachts geleſen und heute früh dem \textsc{Bukovics}\pwindex{Bukovics, Emerich von 28.02.1844 – 04.07.1905@\textsc{Bukovics, Emerich von} (28.02.1844 – 04.07.1905), \emph{Journalist/Journalistin, Theaterleiter/Theaterleiterin}|pw} gegeben. Die Idee, die Du in ihnen mit Deiner wunderbaren, ja ganz einzigen
               Technik ausführſt, geht mir ſehr nahe und berührt mich ſehr; in \label{K_L01172-1v}\edtext{einer\pwindex{schoene Maedchen. Pantomime@\emph{Das schöne Mädchen. Pantomime}|pwv} der »Exiſtenzen\pwindex{schoene Maedchen. Pantomime@\emph{Das schöne Mädchen. Pantomime}|pwv}«, für Salten\pwindex{Salten, Felix 06.09.1869 – 08.10.1945@\textsc{Salten, Felix} (06.09.1869 – 08.10.1945), \emph{Schriftsteller/Schriftstellerin, Journalist/Journalistin, Chefredakteur/Chefredakteurin}|pw}}{\lemma{\textnormal{\emph{einer … Salten}}}\Cendnote{\textnormal{\emph{Das schöne Mädchen}\pwindex{schoene Maedchen. Pantomime@\emph{Das schöne Mädchen. Pantomime}|pwk}, verfasst für das von Salten\pwindex{Salten, Felix 06.09.1869 – 08.10.1945@\textsc{Salten, Felix} (06.09.1869 – 08.10.1945), \emph{Schriftsteller/Schriftstellerin, Journalist/Journalistin, Chefredakteur/Chefredakteurin}|pwk} geleitete Kabarett \emph{Zum lieben Augustin}\orgindex{Jung-Wiener Theater zum Lieben Augustin@Jung-Wiener Theater zum Lieben Augustin|pwk} (veröffentlicht in:
                        \emph{Schwarz auf Weiss}\pwindex{Schwarz auf Weiss. Wiener Autoren den Wiener Kunstgewerbeschuelern zu ihrem Feste am 6. Februar 1902@\emph{Schwarz auf Weiss. Wiener Autoren den Wiener Kunstgewerbeschülern zu ihrem Feste am 6. Februar 1902}|pwk}. Wien:
                        \emph{Comité für das Fest der Kunstgewerbeschüler}{ }1902, S. 23–32).}}}\label{K_L01172-1}, iſt was ähnliches gemeint, nur
               pantomimiſch und ſchon deshalb roher dargeſtellt. In den »Lebendigen Stunden\pwindex{Lebendige Stunden. Vier Einakter@\emph{Lebendige Stunden. Vier Einakter}|pw}« möchte ich die Verſtorbene deutlicher {\pb}zu ſehen kriegen. Im »Dolch\pwindex{Frau mit dem Dolche@\emph{Die Frau mit dem Dolche}|pw}« fürchte ich die Dummheit unſerer Premièren-Idioten; auch
               macht mir Sorge, ob die zweite Verwandlung rapid genug geſchehen kann. Aber von
               alledem mündlich und \introOben{}in\introOben{} Ruhe, wenn ich nicht gerade auf dem
               Sprung zur \label{K_L01172-2v}\edtext{Stuart\pwindex{Maria Stuart@\emph{Maria Stuart}|pw}}{\lemma{\textnormal{\emph{Stuart}}}\Cendnote{\textnormal{\emph{Maria Stuart}\pwindex{Maria Stuart@\emph{Maria Stuart}|pwk} wurde im Deutschen Volkstheater\oindex{Volkstheater@\textbf{Volkstheater}, \emph{Theater (K.THE)}|pwk} gespielt. Es war keine Premiere.}}}\label{K_L01172-2} bin.\pend
           
\pstart
           Herzlichſt{\\[\baselineskip]}Dein{\\[\baselineskip]}\spacefill\mbox{Hermann}\pend
           \leftskip=0em{}\selectlanguage{ngerman}\endnumbering\briefempfaengerindex{Schnitzler, Arthur@\textsc{Schnitzler, Arthur}!zzzBahr, Hermann@\emph{von Hermann Bahr}!1901-09-122@{12. 9. 1901}|)be}\mylabel{L01172h}  \normalsize

\doendnotes{C}
\bigskip
\vfill

\clearpage

\footnotesize

\lohead{\textsc{register}}

% Definiere theindex-Environment komplett neu ohne reledmac
\makeatletter
\renewenvironment{theindex}{%
  \section*{\indexname}%
  \setlength{\parindent}{0pt}%
  \setlength{\parskip}{0pt plus 0.3pt}%
  \let\item\@idxitem
}{%
  \clearpage
}
\makeatother

\IfFileExists{\jobname-pw.ind}{\input{\jobname-pw.ind}}{}

\end{document}

      