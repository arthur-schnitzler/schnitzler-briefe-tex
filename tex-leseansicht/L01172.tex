%% latex-leseansicht-vorspann.tex
%% Vorspann für die Leseansicht.
%% Lädt die gemeinsame Datei latex-vorspann.tex mit nicht gesetztem Schalter.

\newif\ifkorrekturansicht
\korrekturansichtfalse

\input{../tex-inputs/latex-vorspann}


         
         \renewcommand{\erwaehntePersonen}{Personen: Emerich von Bukovics, Felix Salten}
         \renewcommand{\erwaehnteInstitutionen}{Institutionen: Jung-Wiener Theater zum Lieben Augustin, Neues Wiener Tagblatt}
         \renewcommand{\erwaehnteOrte}{Orte: Steyrerhof, Volkstheater, Wien}
         \renewcommand{\erwaehnteWerke}{Werke: Das schöne Mädchen. Pantomime, Die Frau mit dem Dolche, Lebendige Stunden. Vier Einakter, Maria Stuart, Schwarz auf Weiss. Wiener Autoren den Wiener Kunstgewerbeschülern zu ihrem Feste am 6. Februar 1902}
               \section[Hermann Bahr an Arthur Schnitzler, 12. 9. 1901]{ Hermann Bahr an Arthur Schnitzler, 12. 9. 1901}\nopagebreak\mylabel{v}\rehead{ }\begin{ledgroupsized}[t]{13cm}\normalsize\beginnumbering \toendnotes[C]{\smallbreak\pagebreak[2]} \Standort{CUL, Schnitzler, B 5b.}
\physDesc{Brief, 1 Blatt, 2 Seiten, 690 Zeichen
\newline{}Handschrift: schwarze Tinte, deutsche Kurrent
\newline{}Schnitzler: mit Bleistift die Jahreszahl »901« ergänzt 
\newline{}Ordnung: mit Bleistift von unbekannter Hand nummeriert:
                                    »79« }\buchAbdrucke{\weitereDrucke{Hermann Bahr, Arthur Schnitzler: \emph{Briefwechsel, Aufzeichnungen, Dokumente (1891–1931)}. Hg. Kurt Ifkovits und Martin Anton Müller. Göttingen: \emph{Wallstein} 2018, S. 214.} }\toendnotes[C]{\smallbreak}\pstart
           \noindent{}\centering{}{\pb}\textcolor{gray}{\textbf{Redaktion des Neuen Wiener Tagblatt\orgindex{Neues Wiener Tagblatt@Neues Wiener Tagblatt|pw}}}\pend
           \pstart
           \noindent{}\centering{}\textcolor{gray}{\textbf{\textsc{Wien, I., Rothenturmstrasse,
                        Steyrerhof\oindex{Steyrerhof@\textbf{Steyrerhof}|pw}.}}}\pend
           \pstart
           \noindent{}\centering{}\textcolor{gray}{\textbf{Telegramm-Adresse: Tagblatt\orgindex{Neues Wiener Tagblatt@Neues Wiener Tagblatt|pw}, Steyrerhof, Wien\oindex{Steyrerhof@\textbf{Steyrerhof}|pw}. –
                     Telephon Nr. 384. Staats-Telephon Nr. 36.}}\pend
           \pstart
           \raggedleft{}12. 9.\pend
           \pstart\center{}Lieber Arthur!\pend\pstart
           Ich habe Deine Stücke\pwindex{Schnitzler, Arthur 15.05.1862 – 21.10.1931@\textsc{Schnitzler, Arthur} (15.05.1862 – 21.10.1931), \emph{Schriftsteller, Mediziner}!Lebendige Stunden. Vier Einakter1901-12-23@\strich\emph{Lebendige Stunden. Vier Einakter} {[}1901-12-23{]}|pwv}\pwindex{Schnitzler, Arthur 15.05.1862 – 21.10.1931@\textsc{Schnitzler, Arthur} (15.05.1862 – 21.10.1931), \emph{Schriftsteller, Mediziner}!Frau mit dem Dolche1901@\strich\emph{Die Frau mit dem Dolche} {[}1901{]}|pwv} geſtern abends bekommen, nachts geleſen und heute früh dem \textsc{Bukovics}\pwindex{Bukovics, Emerich von 28.02.1844 – 04.07.1905@\textsc{Bukovics, Emerich von} (28.02.1844 – 04.07.1905), \emph{Journalist, Theaterleiter}|pw} gegeben. Die Idee, die Du in ihnen mit Deiner wunderbaren, ja ganz einzigen
               Technik ausführſt, geht mir ſehr nahe und berührt mich ſehr; in \label{K_L01172_1v}\edtext{einer\pwindex{Bahr, Hermann 19.07.1863 – 15.01.1934@\textsc{Bahr, Hermann} (19.07.1863 – 15.01.1934), \emph{Schriftsteller, Kritiker}!schoene Maedchen. Pantomime1902@\strich\emph{Das schöne Mädchen. Pantomime} {[}1902{]}|pwv} der »Exiſtenzen\pwindex{Bahr, Hermann 19.07.1863 – 15.01.1934@\textsc{Bahr, Hermann} (19.07.1863 – 15.01.1934), \emph{Schriftsteller, Kritiker}!schoene Maedchen. Pantomime1902@\strich\emph{Das schöne Mädchen. Pantomime} {[}1902{]}|pwv}«, für Salten\pwindex{Salten, Felix 06.09.1869 – 08.10.1945@\textsc{Salten, Felix} (06.09.1869 – 08.10.1945), \emph{Schriftsteller, Journalist}|pw}}{\lemma{\textnormal{\emph{einer … Salten}}}\Cendnote{\textnormal{\emph{Das schöne Mädchen}\pwindex{Bahr, Hermann 19.07.1863 – 15.01.1934@\textsc{Bahr, Hermann} (19.07.1863 – 15.01.1934), \emph{Schriftsteller, Kritiker}!schoene Maedchen. Pantomime1902@\strich\emph{Das schöne Mädchen. Pantomime} {[}1902{]}|pwk}, verfasst für das von Salten\pwindex{Salten, Felix 06.09.1869 – 08.10.1945@\textsc{Salten, Felix} (06.09.1869 – 08.10.1945), \emph{Schriftsteller, Journalist}|pwk} geleitete Kabarett \emph{Zum lieben Augustin}\orgindex{Jung-Wiener Theater zum Lieben Augustin@Jung-Wiener Theater zum Lieben Augustin|pwk} (veröffentlicht in:
                        \emph{Schwarz auf Weiss}\pwindex{?? Werk@Nicht ermittelte Verfasserinnen und Verfasser!Schwarz auf Weiss. Wiener Autoren den Wiener Kunstgewerbeschuelern zu ihrem
                  Feste am 6. Februar 19021902@\emph{Schwarz auf Weiss. Wiener Autoren den Wiener Kunstgewerbeschülern zu ihrem Feste am 6. Februar 1902} {[}1902{]}|pwk}. Wien:
                        \emph{Comité für das Fest der Kunstgewerbeschüler}{ }1902, S. 23–32).}}}\label{K_L01172_1h}, iſt was ähnliches gemeint, nur
               pantomimiſch und ſchon deshalb roher dargeſtellt. In den »Lebendigen Stunden\pwindex{Schnitzler, Arthur 15.05.1862 – 21.10.1931@\textsc{Schnitzler, Arthur} (15.05.1862 – 21.10.1931), \emph{Schriftsteller, Mediziner}!Lebendige Stunden. Vier Einakter1901-12-23@\strich\emph{Lebendige Stunden. Vier Einakter} {[}1901-12-23{]}|pw}« möchte ich die Verſtorbene deutlicher {\pb}zu ſehen kriegen. Im »Dolch\pwindex{Schnitzler, Arthur 15.05.1862 – 21.10.1931@\textsc{Schnitzler, Arthur} (15.05.1862 – 21.10.1931), \emph{Schriftsteller, Mediziner}!Frau mit dem Dolche1901@\strich\emph{Die Frau mit dem Dolche} {[}1901{]}|pw}« fürchte ich die Dummheit unſerer Premièren-Idioten; auch
               macht mir Sorge, ob die zweite Verwandlung rapid genug geſchehen kann. Aber von
               alledem mündlich und \introOben{}in\introOben{} Ruhe, wenn ich nicht gerade auf dem
               Sprung zur \label{K_L01172_2v}\edtext{Stuart\pwindex{\textcolor{red}{\textsuperscript{XXXX1 indx}}!Maria Stuart1800@\strich\emph{Maria Stuart} {[}1800{]}|pw}}{\lemma{\textnormal{\emph{Stuart}}}\Cendnote{\textnormal{im Deutschen Volkstheater\oindex{Volkstheater@\textbf{Volkstheater}|pwk}; keine Premiere.}}}\label{K_L01172_2h} bin.\pend
           \pstart
           Herzlichſt{\\[\baselineskip]}Dein{\\[\baselineskip]}\spacefill\mbox{Hermann}\pend
           \leftskip=0em{}
         
         \endnumbering\mylabel{h}\end{ledgroupsized}  \newcommand{\dateiname}{L01172}\newcommand{\titel}{Hermann Bahr an Arthur Schnitzler, 12. 9. 1901}\newcommand{\editorInnen}{ Kurt Ifkovits,  Martin Anton Müller}%% latex-leseansicht-abspann.tex
%% Abspann für die Leseansicht.
%% Der Schalter \ifkorrekturansicht ist bereits durch den Vorspann gesetzt.

%% latex-abspann.tex
%% Gemeinsamer Abspann für Korrekturansicht und Leseansicht.
%% Setzt den Schalter \ifkorrekturansicht voraus (gesetzt in den
%% einbindenden Dateien latex-korrekturansicht-abspann.tex bzw.
%% latex-leseansicht-abspann.tex).
%% ---------------------------------------------------------------

\normalsize

% Das esempio-Environment wird nur in der Leseansicht benötigt
\ifkorrekturansicht\else
\newenvironment{esempio}[3]%
{
    \vspace{1.5ex}
    \rlap{\underline{#1}}
    \par
    \setlength{\parindent}{0cm}
    \nopagebreak
    \leftskip=#2cm
    \rightskip=#3cm
}
{
    \par
}
\fi

\doendnotes{C}
\bigskip
\vfill

\clearpage

\footnotesize

\ifkorrekturansicht
  \lohead{\textsc{register}}
\fi

% theindex-Environment neu definieren ohne reledmac
\makeatletter
\renewenvironment{theindex}{%
  \ifkorrekturansicht
    \section*{\indexname}%
  \else
    \subsubsection*{Index der erwähnten Entitäten}%
  \fi
  \setlength{\parindent}{0pt}%
  \setlength{\parskip}{0pt plus 0.3pt}%
  \let\item\@idxitem
}{%
  \ifkorrekturansicht\clearpage\fi
}
\makeatother

\IfFileExists{\jobname-pw.ind}{\input{\jobname-pw.ind}}{}

% Quellenangabe nur in der Leseansicht
\ifkorrekturansicht\else
% Fallback-Definitionen, falls die .tex-Datei \titel etc. nicht gesetzt hat
\providecommand{\titel}{}
\providecommand{\editorInnen}{}
\providecommand{\dateiname}{\jobname}

\vspace{3cm}

\vfill

\footnotesize
\textsc{Quelle}: \titel. Herausgegeben von {\editorInnen}. In: \emph{Arthur Schnitzler: Briefwechsel mit Autorinnen und Autoren}.
 Digitale Edition, https://schnitzler-briefe.acdh.oeaw.ac.at/{\dateiname}.html (Stand \today)
\fi

\end{document}


      