%% latex-korrekturansicht-vorspann.tex
%% Vorspann für die Korrekturansicht.
%% Lädt die gemeinsame Datei latex-vorspann.tex mit gesetztem Schalter.

\newif\ifkorrekturansicht
\korrekturansichttrue

\input{../tex-inputs/latex-vorspann}


\section[ Paul Goldmann an Arthur Schnitzler, 13. 12. {[}1901{]}]{L03095 Paul Goldmann an Arthur Schnitzler, 13. 12. {[}1901{]}}
\nopagebreak\mylabel{L03095v}
\rehead{ }\normalsize\beginnumbering\briefempfaengerindex{Schnitzler, Arthur@\textsc{Schnitzler, Arthur}!zzzGoldmann, Paul@\emph{von Paul Goldmann}!1901-12-131@{13. 12. {[}1901{]}}|(be}
\toendnotes[C]{\smallbreak\pagebreak[2]}\Standort{DLA, A:Schnitzler, HS.NZ85.1.3171.}
\physDesc{Brief, 1 Blatt, 4 Seiten, 1746 Zeichen
\newline{}Handschrift: blaue Tinte, deutsche Kurrent
\newline{}Schnitzler: 1) mit Bleistift das Jahr »901« vermerkt  2) mit rotem Buntstift eine Unterstreichung}\toendnotes[C]{\smallbreak}
\pstart
           \raggedleft{}{\pb}\textcolor{gray}{\textbf{DESSAUERSTRASSE 19}}\oindex{Dessauer Strasse@\textbf{Dessauer Straße}, \emph{Straße (K.STR)}|pw}\pend
           
\pstart
           Berlin\oindex{Berlin@\textbf{Berlin}, \emph{P.PPLC}|pw}, 13. Dezember.\pend
           
\pstart\center{}Mein lieber Freund,\pend\vspace{0.5em}
\pstart
           Das \label{K_L03095-1v}\edtext{Verhalten des Volkstheaters\orgindex{Volkstheater@Volkstheater|pw}}{\lemma{\textnormal{\emph{Verhalten des Volkstheaters}}}\Cendnote{\textnormal{hinsichtlich einer möglichen Aufführung der \emph{Lebendigen Stunden}\pwindex{Lebendige Stunden. Vier Einakter@\emph{Lebendige Stunden. Vier Einakter}|pwk} am \emph{Volkstheater}\orgindex{Volkstheater@Volkstheater|pwk};
                     siehe A. S.: \emph{Tagebuch}, 6. 12. 1901 und 10. 12. 1901 sowie Hermann Bahr an Arthur Schnitzler, 27. 10. [1901], Arthur Schnitzler an Hermann Bahr, 28. 10. 1901 und 11. 12. 1901.}}}\label{K_L03095-1} iſt ſkandalös, und Dein \label{K_L03095-2v}\edtext{Brief}{\lemma{\textnormal{\emph{Brief}}}\Cendnote{\textnormal{Siehe Hermann Bahr, Arthur Schnitzler: \emph{Briefwechsel, Aufzeichnungen, Dokumente (1891–1931)}, Arthur Schnitzler an Emerich von Bukovics, 11. 12. 1901.
               }}}\label{K_L03095-2} iſt unter dieſen Umſtänden nur der Ausdruck legitimer Entrüſtung. Ob es aber
                  \uline{klug} war, die Beziehungen ganz abzulehnen, kann
               ich von hier aus nicht beurtheilen. Dazu bedarf ich Deiner mündlichen Aufklärungen.
               Herr \textsc{Bahr\pwindex{Bahr, Hermann 19.07.1863 – 15.01.1934@\textsc{Bahr, Hermann} (19.07.1863 – 15.01.1934), \emph{Schriftsteller/Schriftstellerin, Kritiker/Kritikerin}|pw}} ſcheint da wieder eine feine Rolle geſpielt zu haben. Wie aber wird die Zukunft
               ſein? Wenn Du \label{K_L03095-3v}\edtext{in Wien\oindex{Wien@\textbf{Wien}, \emph{A.ADM2}|pw} kein Theater mehr}{\lemma{\textnormal{\emph{in … mehr}}}\Cendnote{\textnormal{Anspielung auf die vorjährige verzögerte Ablehnung Paul Schlenthers\pwindex{Schlenther, Paul 20.08.1854 – 30.04.1916@\textsc{Schlenther, Paul} (20.08.1854 – 30.04.1916), \emph{Schriftsteller/Schriftstellerin, Kritiker/Kritikerin, Theaterleiter/Theaterleiterin}|pwk}, den \emph{Schleier der Beatrice}\pwindex{Schleier der Beatrice. Schauspiel in fuenf Akten@\emph{Der Schleier der Beatrice. Schauspiel in fünf Akten}|pwk} am \emph{Burgtheater}\orgindex{Burgtheater@Burgtheater|pwk} zu inszenieren.}}}\label{K_L03095-3} haſt, wirſt Du, ſo denke ich mir, nach
                  Berlin\oindex{Berlin@\textbf{Berlin}, \emph{P.PPLC}|pw} überſiedeln. Hier wirſt Du die Stellung
               finden, die man Dir in Wien\oindex{Wien@\textbf{Wien}, \emph{A.ADM2}|pw} verſagt. Und {\pb}auch \strikeout{\textcolor{gray}{h}\textcolor{gray}{×}} Deine Weiterentwickelung könnte nur günſtig \strikeout{beein} beeinflußt werden, wenn Du die engen Wien\oindex{Wien@\textbf{Wien}, \emph{A.ADM2}|pw}er Verhältniſſe verließeſt und in die große Welt hinauszögeſt.\pend
           
\pstart
           Die \label{K_L03095-4v}\edtext{Karte, die wir Dir ſandten}{\lemma{\textnormal{\emph{Karte, … ſandten}}}\Cendnote{\textnormal{nicht ermittelt}}}\label{K_L03095-4}, war in der That
               bei \label{K_L03095-5v}\edtext{Dr. \textsc{Friedmann\pwindex{Friedmann, Alfred 26.10.1845 – 13.02.1923@\textsc{Friedmann, Alfred} (26.10.1845 – 13.02.1923), \emph{Schriftsteller/Schriftstellerin}|pwu}}}{\lemma{\textnormal{\emph{Dr. Friedmann}}}\Cendnote{\textnormal{möglicherweise der Schriftsteller Alfred Friedmann\pwindex{Friedmann, Alfred 26.10.1845 – 13.02.1923@\textsc{Friedmann, Alfred} (26.10.1845 – 13.02.1923), \emph{Schriftsteller/Schriftstellerin}|pwk}, der in Berlin\oindex{Berlin@\textbf{Berlin}, \emph{P.PPLC}|pwk} wohnte}}}\label{K_L03095-5} geſchrieben.\pend
           
\pstart
           Warum führt der Akademiſch-Literariſche Verein\orgindex{Akademischer Verein fuer Kunst und Literatur@Akademischer Verein für Kunst und Literatur|pw},
               der ſich in Wien\oindex{Wien@\textbf{Wien}, \emph{A.ADM2}|pw} begründet hat, nicht den »Schleier der \textsc{Beatrice}\pwindex{Schleier der Beatrice. Schauspiel in fuenf Akten@\emph{Der Schleier der Beatrice. Schauspiel in fünf Akten}|pw}« auf?\pend
           
\pstart
           Ich hoffe um Weihnachten herum etwa 14 Tage in Frankfurt\oindex{Frankfurt am Main@\textbf{Frankfurt am Main}, \emph{P.PPLA3}|pw} bleiben zu können bis zur Wiedererö\textcolor{gray}{f}fnung des
               Reichstags\orgindex{Reichstag@Reichstag|pw} (\label{K_L03095-6v}\edtext{8. Jänner}{\lemma{\textnormal{\emph{8. Jänner}}}\Cendnote{\textnormal{Goldmann\pwindex{Goldmann, Paul 31.01.1865 – 25.09.1935@\textsc{Goldmann, Paul} (31.01.1865 – 25.09.1935), \emph{Schriftsteller/Schriftstellerin, Journalist/Journalistin}|pwk} war ab dem 4. 1. 1902 wieder in Berlin\oindex{Berlin@\textbf{Berlin}, \emph{P.PPLC}|pwk}, vgl. Paul Goldmann an Arthur Schnitzler, 29. 12. [1901].}}}\label{K_L03095-6}). Ich bin {\pb}unbeſchreiblich heruntergearbeitet und bedarf der
               Ruhe und Erholung. Daß Deine \label{K_L03095-7v}\edtext{\textsc{Première\pwindex{Lebendige Stunden. Vier Einakter@\emph{Lebendige Stunden. Vier Einakter}|pwv}}}{\lemma{\textnormal{\emph{Première}}}\Cendnote{\textnormal{Am 4. 1. 1902 fand am Deutschen Theater\oindex{Deutsches Theater Berlin@\textbf{Deutsches Theater Berlin}, \emph{Theater (K.THE)}|pwk} in Berlin\oindex{Berlin@\textbf{Berlin}, \emph{P.PPLC}|pwk} die
                  Uraufführung der vier Einakter \emph{Lebendige
                     Stunden}\pwindex{Lebendige Stunden. Vier Einakter@\emph{Lebendige Stunden. Vier Einakter}|pwk} statt. Zu der von Goldmann\pwindex{Goldmann, Paul 31.01.1865 – 25.09.1935@\textsc{Goldmann, Paul} (31.01.1865 – 25.09.1935), \emph{Schriftsteller/Schriftstellerin, Journalist/Journalistin}|pwk}
                  gewünschten Verschiebung kam es nicht.}}}\label{K_L03095-7} in meine kurze Ferienzeit fällt, iſt
               ein Zuſammentreffen, das ſich ausnimmt, als ſei \strikeout{von
                  irge} dieſe Anordnung von einer feindſeligen Hand getroffen worden. Ich werde
               von Dir nicht verlangen, daß Du meinetwegen Deine \textsc{Première\pwindex{Lebendige Stunden. Vier Einakter@\emph{Lebendige Stunden. Vier Einakter}|pwv}} verſchiebſt. Aber mit Rückſicht auf das \label{K_L03095-8v}\edtext{Referat\pwindex{Theater- und Kunstnachrichten [Urauffuehrung von Lebendige Stunden]@\emph{Theater- und Kunstnachrichten [Uraufführung von Lebendige Stunden]}|pwv}}{\lemma{\textnormal{\emph{Referat}}}\Cendnote{\textnormal{[Paul Goldmann\pwindex{Goldmann, Paul 31.01.1865 – 25.09.1935@\textsc{Goldmann, Paul} (31.01.1865 – 25.09.1935), \emph{Schriftsteller/Schriftstellerin, Journalist/Journalistin}|pwk}]: \emph{Theater- und Kunstnachrichten. [Zur Uraufführung von
                        Lebendige Stunden]}\pwindex{Theater- und Kunstnachrichten [Urauffuehrung von Lebendige Stunden]@\emph{Theater- und Kunstnachrichten [Uraufführung von Lebendige Stunden]}|pwk}. In: \emph{Neue Freie
                        Presse}\pwindex{Neue Freie Presse@\emph{Neue Freie Presse}|pwk}, Nr. 13.422, 5. 1. 1902,
                     Morgenblatt, S. 8–9. Später erschien noch ein ausführlicheres Feuilleton\pwindex{Berliner Theater. (»Lebendige Stunden« von Arthur Schnitzler.)@\emph{Berliner Theater. (»Lebendige Stunden« von Arthur Schnitzler.)}|pwkv}: Paul Goldmann\pwindex{Goldmann, Paul 31.01.1865 – 25.09.1935@\textsc{Goldmann, Paul} (31.01.1865 – 25.09.1935), \emph{Schriftsteller/Schriftstellerin, Journalist/Journalistin}|pwk}: \emph{Berliner Theater. (»Lebendige Stunden« von Arthur
                        Schnitzler)}\pwindex{Berliner Theater. (»Lebendige Stunden« von Arthur Schnitzler.)@\emph{Berliner Theater. (»Lebendige Stunden« von Arthur Schnitzler.)}|pwk}. In: \emph{Neue Freie
                        Presse}\pwindex{Neue Freie Presse@\emph{Neue Freie Presse}|pwk}, Nr. 13.438, 22. 1. 1902,
                     Morgenblatt, S. 1–4.}}}\label{K_L03095-8} in der N.
                  Fr. Pr.\pwindex{Neue Freie Presse@\emph{Neue Freie Presse}|pw}\textcolor{gray}{×}, das doch von großer Wichtigkeit
               ſein wird, könnteſt Du ſchon eine Verſchiebung um ein paar Tage vornehmen, unter
               igend einem Vorwande. Ich werde ſehen, ob ich hier einen anſtändigen und verläßlichen
                  {\pb}Vertreter finden kann. Wenn nicht, ſo werde ich
               meinen Urlaub abkürzen und zur \textsc{Première\pwindex{Lebendige Stunden. Vier Einakter@\emph{Lebendige Stunden. Vier Einakter}|pwv}} zurückkommen.\pend
           
\pstart
           Viele herzliche Grüße Dir und den Mädeln\pwindex{Schnitzler, Olga 17.01.1882 – 13.01.1970@\textsc{Schnitzler, Olga} (17.01.1882 – 13.01.1970), \emph{Schauspieler/Schauspielerin, Sänger/Sängerin}|pwv}\pwindex{Steinrueck, Elisabeth 19.11.1885 – 07.04.1920@\textsc{Steinrück, Elisabeth} (19.11.1885 – 07.04.1920)|pwv}! {\\[\baselineskip]}Dein {\\[\baselineskip]}\spacefill\mbox{Paul Goldmn}\pend
           \leftskip=0em{}\selectlanguage{ngerman}\endnumbering\briefempfaengerindex{Schnitzler, Arthur@\textsc{Schnitzler, Arthur}!zzzGoldmann, Paul@\emph{von Paul Goldmann}!1901-12-131@{13. 12. {[}1901{]}}|)be}\mylabel{L03095h}  \normalsize

\doendnotes{C}
\bigskip
\vfill

\clearpage

\footnotesize

\lohead{\textsc{register}}

% Definiere theindex-Environment komplett neu ohne reledmac
\makeatletter
\renewenvironment{theindex}{%
  \section*{\indexname}%
  \setlength{\parindent}{0pt}%
  \setlength{\parskip}{0pt plus 0.3pt}%
  \let\item\@idxitem
}{%
  \clearpage
}
\makeatother

\IfFileExists{\jobname-pw.ind}{\input{\jobname-pw.ind}}{}

\end{document}

      