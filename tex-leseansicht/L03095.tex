%% latex-leseansicht-vorspann.tex
%% Vorspann für die Leseansicht.
%% Lädt die gemeinsame Datei latex-vorspann.tex mit nicht gesetztem Schalter.

\newif\ifkorrekturansicht
\korrekturansichtfalse

\input{../tex-inputs/latex-vorspann}


\section[ Paul Goldmann an Arthur Schnitzler, 13. 12. [1901]]{L03095 Paul Goldmann an Arthur Schnitzler,  13. 12. [1901]}
\nopagebreak\mylabel{L03095v}
\rehead{ }\normalsize\beginnumbering\briefempfaengerindex{Schnitzler, Arthur@\textsc{Schnitzler, Arthur}!zzzGoldmann, Paul@\emph{von Paul Goldmann}!1901-12-131@{13. 12. [1901]}|(be}
\toendnotes[C]{\smallbreak\pagebreak[2]}
\correspDesc{Versand  durch Paul Goldmann am 13. 12. [1901] in Berlin
\newline{}Erhalt  durch Arthur Schnitzler im Zeitraum [14. 12. 1901 – 18. 12. 1901?] in Wien}\toendnotes[C]{\smallbreak}
\Standort{DLA, A:Schnitzler, HS.NZ85.1.3171.}
\physDesc{Brief, 1 Blatt, 4 Seiten, 1746 Zeichen
\newline{}Handschrift: blaue Tinte, deutsche Kurrent
\newline{}Schnitzler: 1) mit Bleistift das Jahr »901« vermerkt  2) mit rotem Buntstift eine Unterstreichung}\toendnotes[C]{\smallbreak}
\pstart
           \raggedleft{}{\pb}\textcolor{gray}{\textbf{DESSAUERSTRASSE 19}}\oindex{Dessauer Straße@\textbf{Dessauer Straße}, \emph{Straße}|pw}\pend
           
\pstart
           Berlin\oindex{Berlin@\textbf{Berlin}, \emph{Hauptstadt}|pw}, 13. Dezember.\pend
           
\pstart\center{}Mein lieber Freund,\pend\vspace{0.5em}
\pstart
           Das \label{K_L03095-1v}\edtext{Verhalten des Volkstheaters\orgindex{Volkstheater@Volkstheater|pw}}{\lemma{\textnormal{\emph{Verhalten des Volkstheaters}}}\Cendnote{\textnormal{hinsichtlich einer möglichen Aufführung der \emph{Lebendigen Stunden}\pwindex{Schnitzler, Arthur 15.\,5.\,1862 Wien – 21.\,10.\,1931 ebd.@\textsc{Schnitzler, Arthur} (15.\,5.\,1862 Wien – 21.\,10.\,1931 ebd.), \emph{Schriftsteller, Mediziner}!Lebendige Stunden. Vier Einakter@\strich\emph{Lebendige Stunden. Vier Einakter}|pwk} am \emph{Volkstheater}\orgindex{Volkstheater@Volkstheater|pwk};
                     siehe A. S.: \emph{Tagebuch}, 6. 12. 1901 und 10. 12. 1901 sowie XXXX Auszeichnungsfehler: Dokument L01184 nicht gefunden, XXXX Auszeichnungsfehler: Dokument L01185 nicht gefunden und XXXX Auszeichnungsfehler: Dokument L01189 nicht gefunden.}}}\label{K_L03095-1} iſt{ }ſkandalös, und Dein \label{K_L03095-2v}\edtext{Brief}{\lemma{\textnormal{\emph{Brief}}}\Cendnote{\textnormal{Siehe Hermann Bahr, Arthur Schnitzler: \emph{Briefwechsel, Aufzeichnungen, Dokumente (1891–1931)}, Arthur Schnitzler an Emerich von Bukovics, 11. 12. 1901.
               }}}\label{K_L03095-2} iſt unter dieſen Umſtänden nur der Ausdruck legitimer Entrüſtung. Ob es aber
                  \uline{klug} war, die Beziehungen ganz abzulehnen, kann
               ich von hier aus nicht beurtheilen. Dazu bedarf ich Deiner mündlichen Aufklärungen.
               Herr \textsc{Bahr\pwindex{Bahr, Hermann 19.\,7.\,1863 Linz – 15.\,1.\,1934 München@\textsc{Bahr, Hermann} (19.\,7.\,1863 Linz – 15.\,1.\,1934 München), \emph{Schriftsteller, Kritiker}|pw}}{ }ſcheint da wieder eine feine Rolle geſpielt zu haben. Wie aber wird die Zukunft{ }ſein? Wenn Du \label{K_L03095-3v}\edtext{in Wien\oindex{Wien@\textbf{Wien}, \emph{Verwaltungsgebiet}|pw} kein Theater mehr}{\lemma{\textnormal{\emph{in … mehr}}}\Cendnote{\textnormal{Anspielung auf die vorjährige verzögerte Ablehnung Paul Schlenthers\pwindex{Schlenther, Paul 20.\,8.\,1854 Chernyakhovsk – 30.\,4.\,1916 Berlin@\textsc{Schlenther, Paul} (20.\,8.\,1854 Chernyakhovsk – 30.\,4.\,1916 Berlin), \emph{Schriftsteller, Kritiker, Theaterleiter}|pwk}, den \emph{Schleier der Beatrice}\pwindex{Schnitzler, Arthur 15.\,5.\,1862 Wien – 21.\,10.\,1931 ebd.@\textsc{Schnitzler, Arthur} (15.\,5.\,1862 Wien – 21.\,10.\,1931 ebd.), \emph{Schriftsteller, Mediziner}!Schleier der Beatrice. Schauspiel in fünf Akten@\strich\emph{Der Schleier der Beatrice. Schauspiel in fünf Akten}|pwk} am \emph{Burgtheater}\orgindex{Burgtheater@Burgtheater|pwk} zu inszenieren.}}}\label{K_L03095-3} haſt, wirſt Du,{ }ſo denke ich mir, nach
                  Berlin\oindex{Berlin@\textbf{Berlin}, \emph{Hauptstadt}|pw} überſiedeln. Hier wirſt Du die Stellung
               finden, die man Dir in Wien\oindex{Wien@\textbf{Wien}, \emph{Verwaltungsgebiet}|pw} verſagt. Und {\pb}auch \strikeout{\textcolor{gray}{h}\textcolor{gray}{×}} Deine Weiterentwickelung könnte nur günſtig \strikeout{beein} beeinflußt werden, wenn Du die engen Wien\oindex{Wien@\textbf{Wien}, \emph{Verwaltungsgebiet}|pw}er Verhältniſſe verließeſt und in die große Welt hinauszögeſt.\pend
           
\pstart
           Die \label{K_L03095-4v}\edtext{Karte, die wir Dir{ }ſandten}{\lemma{\textnormal{\emph{Karte, … sandten}}}\Cendnote{\textnormal{nicht ermittelt}}}\label{K_L03095-4}, war in der That
               bei \label{K_L03095-5v}\edtext{Dr. \textsc{Friedmann\pwindex{Friedmann, Alfred 26.\,10.\,1845 Frankfurt am Main – 13.\,2.\,1923 Berlin@\textsc{Friedmann, Alfred} (26.\,10.\,1845 Frankfurt am Main – 13.\,2.\,1923 Berlin), \emph{Schriftsteller}|pwu}}}{\lemma{\textnormal{\emph{Dr. Friedmann}}}\Cendnote{\textnormal{möglicherweise der Schriftsteller Alfred Friedmann\pwindex{Friedmann, Alfred 26.\,10.\,1845 Frankfurt am Main – 13.\,2.\,1923 Berlin@\textsc{Friedmann, Alfred} (26.\,10.\,1845 Frankfurt am Main – 13.\,2.\,1923 Berlin), \emph{Schriftsteller}|pwk}, der in Berlin\oindex{Berlin@\textbf{Berlin}, \emph{Hauptstadt}|pwk} wohnte}}}\label{K_L03095-5} geſchrieben.\pend
           
\pstart
           Warum führt der Akademiſch-Literariſche Verein\orgindex{Akademischer Verein für Kunst und Literatur@Akademischer Verein für Kunst und Literatur|pw},
               der{ }ſich in Wien\oindex{Wien@\textbf{Wien}, \emph{Verwaltungsgebiet}|pw} begründet hat, nicht den »Schleier der \textsc{Beatrice}\pwindex{Schnitzler, Arthur 15.\,5.\,1862 Wien – 21.\,10.\,1931 ebd.@\textsc{Schnitzler, Arthur} (15.\,5.\,1862 Wien – 21.\,10.\,1931 ebd.), \emph{Schriftsteller, Mediziner}!Schleier der Beatrice. Schauspiel in fünf Akten@\strich\emph{Der Schleier der Beatrice. Schauspiel in fünf Akten}|pw}« auf?\pend
           
\pstart
           Ich hoffe um Weihnachten herum etwa 14 Tage in Frankfurt\oindex{Frankfurt am Main@\textbf{Frankfurt am Main}, \emph{Hauptstadt}|pw} bleiben zu können bis zur Wiedererö\textcolor{gray}{f}fnung des
               Reichstags\orgindex{Reichstag@Reichstag|pw} (\label{K_L03095-6v}\edtext{8. Jänner}{\lemma{\textnormal{\emph{8. Jänner}}}\Cendnote{\textnormal{Goldmann\pwindex{Goldmann, Paul 31.\,1.\,1865 Breslau – 25.\,9.\,1935 Wien@\textsc{Goldmann, Paul} (31.\,1.\,1865 Breslau – 25.\,9.\,1935 Wien), \emph{Schriftsteller, Journalist}|pwk} war ab dem 4. 1. 1902 wieder in Berlin\oindex{Berlin@\textbf{Berlin}, \emph{Hauptstadt}|pwk}, vgl. XXXX Auszeichnungsfehler: Dokument L03098 nicht gefunden.}}}\label{K_L03095-6}). Ich bin {\pb}unbeſchreiblich heruntergearbeitet und bedarf der
               Ruhe und Erholung. Daß Deine \label{K_L03095-7v}\edtext{\textsc{Première\eventindex{Deutsches Theater Berlin@\textbf{Deutsches Theater Berlin}!Uraufführung von Lebendige Stunden, 4.1.1902@Uraufführung von Lebendige Stunden, 4.1.1902|pwv}\pwindex{Schnitzler, Arthur 15.\,5.\,1862 Wien – 21.\,10.\,1931 ebd.@\textsc{Schnitzler, Arthur} (15.\,5.\,1862 Wien – 21.\,10.\,1931 ebd.), \emph{Schriftsteller, Mediziner}!Lebendige Stunden. Vier Einakter@\strich\emph{Lebendige Stunden. Vier Einakter}|pwv}}}{\lemma{\textnormal{\emph{Première}}}\Cendnote{\textnormal{Am 4. 1. 1902 fand am Deutschen Theater\oindex{Deutsches Theater Berlin@\textbf{Deutsches Theater Berlin}, \emph{Theater}|pwk} in Berlin\oindex{Berlin@\textbf{Berlin}, \emph{Hauptstadt}|pwk} die
                  Uraufführung der vier Einakter \emph{Lebendige
                        Stunden}\pwindex{Schnitzler, Arthur 15.\,5.\,1862 Wien – 21.\,10.\,1931 ebd.@\textsc{Schnitzler, Arthur} (15.\,5.\,1862 Wien – 21.\,10.\,1931 ebd.), \emph{Schriftsteller, Mediziner}!Lebendige Stunden. Vier Einakter@\strich\emph{Lebendige Stunden. Vier Einakter}|pwk}\eventindex{Deutsches Theater Berlin@\textbf{Deutsches Theater Berlin}!Uraufführung von Lebendige Stunden, 4.1.1902@Uraufführung von Lebendige Stunden, 4.1.1902|pwk} statt. Zu der von Goldmann\pwindex{Goldmann, Paul 31.\,1.\,1865 Breslau – 25.\,9.\,1935 Wien@\textsc{Goldmann, Paul} (31.\,1.\,1865 Breslau – 25.\,9.\,1935 Wien), \emph{Schriftsteller, Journalist}|pwk}
                  gewünschten Verschiebung kam es nicht.}}}\label{K_L03095-7} in meine kurze Ferienzeit fällt, iſt
               ein Zuſammentreffen, das{ }ſich ausnimmt, als{ }ſei \strikeout{von
                  irge} dieſe Anordnung von einer feindſeligen Hand getroffen worden. Ich werde
               von Dir nicht verlangen, daß Du meinetwegen Deine \textsc{Première\eventindex{Deutsches Theater Berlin@\textbf{Deutsches Theater Berlin}!Uraufführung von Lebendige Stunden, 4.1.1902@Uraufführung von Lebendige Stunden, 4.1.1902|pwv}\pwindex{Schnitzler, Arthur 15.\,5.\,1862 Wien – 21.\,10.\,1931 ebd.@\textsc{Schnitzler, Arthur} (15.\,5.\,1862 Wien – 21.\,10.\,1931 ebd.), \emph{Schriftsteller, Mediziner}!Lebendige Stunden. Vier Einakter@\strich\emph{Lebendige Stunden. Vier Einakter}|pwv}} verſchiebſt. Aber mit Rückſicht auf das \label{K_L03095-8v}\edtext{Referat\pwindex{Theater- und Kunstnachrichten [Uraufführung von Lebendige Stunden]@\emph{Theater- und Kunstnachrichten [Uraufführung von Lebendige Stunden]}|pwv}}{\lemma{\textnormal{\emph{Referat}}}\Cendnote{\textnormal{[Paul Goldmann\pwindex{Goldmann, Paul 31.\,1.\,1865 Breslau – 25.\,9.\,1935 Wien@\textsc{Goldmann, Paul} (31.\,1.\,1865 Breslau – 25.\,9.\,1935 Wien), \emph{Schriftsteller, Journalist}|pwk}]: \emph{Theater- und Kunstnachrichten. [Zur Uraufführung von
                        Lebendige Stunden\eventindex{Deutsches Theater Berlin@\textbf{Deutsches Theater Berlin}!Uraufführung von Lebendige Stunden, 4.1.1902@Uraufführung von Lebendige Stunden, 4.1.1902|pwk}]}\pwindex{Theater- und Kunstnachrichten [Uraufführung von Lebendige Stunden]@\emph{Theater- und Kunstnachrichten [Uraufführung von Lebendige Stunden]}|pwk}. In: \emph{Neue Freie
                        Presse}\pwindex{Neue Freie Presse@\emph{Neue Freie Presse}|pwk}, Nr. 13.422, 5. 1. 1902,
                     Morgenblatt, S. 8–9. Später erschien noch ein ausführlicheres Feuilleton\pwindex{Goldmann, Paul 31.\,1.\,1865 Breslau – 25.\,9.\,1935 Wien@\textsc{Goldmann, Paul} (31.\,1.\,1865 Breslau – 25.\,9.\,1935 Wien), \emph{Schriftsteller, Journalist}!Berliner Theater. (»Lebendige Stunden« von Arthur Schnitzler.)@\strich\emph{Berliner Theater. (»Lebendige Stunden« von Arthur Schnitzler.)}|pwkv}: Paul Goldmann\pwindex{Goldmann, Paul 31.\,1.\,1865 Breslau – 25.\,9.\,1935 Wien@\textsc{Goldmann, Paul} (31.\,1.\,1865 Breslau – 25.\,9.\,1935 Wien), \emph{Schriftsteller, Journalist}|pwk}: \emph{Berliner Theater. (»Lebendige Stunden« von Arthur
                        Schnitzler)}\pwindex{Goldmann, Paul 31.\,1.\,1865 Breslau – 25.\,9.\,1935 Wien@\textsc{Goldmann, Paul} (31.\,1.\,1865 Breslau – 25.\,9.\,1935 Wien), \emph{Schriftsteller, Journalist}!Berliner Theater. (»Lebendige Stunden« von Arthur Schnitzler.)@\strich\emph{Berliner Theater. (»Lebendige Stunden« von Arthur Schnitzler.)}|pwk}. In: \emph{Neue Freie
                        Presse}\pwindex{Neue Freie Presse@\emph{Neue Freie Presse}|pwk}, Nr. 13.438, 22. 1. 1902,
                     Morgenblatt, S. 1–4.}}}\label{K_L03095-8} in der N.
                  Fr. Pr.\pwindex{Neue Freie Presse@\emph{Neue Freie Presse}|pw}\textcolor{gray}{×}, das doch von großer Wichtigkeit{ }ſein wird, könnteſt Du{ }ſchon eine Verſchiebung um ein paar Tage vornehmen, unter
               igend einem Vorwande. Ich werde{ }ſehen, ob ich hier einen anſtändigen und verläßlichen
                  {\pb}Vertreter finden kann. Wenn nicht,{ }ſo werde ich
               meinen Urlaub abkürzen und zur \textsc{Première\eventindex{Deutsches Theater Berlin@\textbf{Deutsches Theater Berlin}!Uraufführung von Lebendige Stunden, 4.1.1902@Uraufführung von Lebendige Stunden, 4.1.1902|pwv}\pwindex{Schnitzler, Arthur 15.\,5.\,1862 Wien – 21.\,10.\,1931 ebd.@\textsc{Schnitzler, Arthur} (15.\,5.\,1862 Wien – 21.\,10.\,1931 ebd.), \emph{Schriftsteller, Mediziner}!Lebendige Stunden. Vier Einakter@\strich\emph{Lebendige Stunden. Vier Einakter}|pwv}} zurückkommen.\pend
           
\pstart
           Viele herzliche Grüße Dir und den Mädeln\pwindex{Schnitzler, Olga 17.\,1.\,1882 Wien – 13.\,1.\,1970 Lugano@\textsc{Schnitzler, Olga} (17.\,1.\,1882 Wien – 13.\,1.\,1970 Lugano), \emph{Schauspielerin, Sängerin}|pwv}\pwindex{Steinrück, Elisabeth 19.\,11.\,1885 – 7.\,4.\,1920 Partenkirchen@\textsc{Steinrück, Elisabeth} (19.\,11.\,1885 – 7.\,4.\,1920 Partenkirchen)|pwv}! {\\[\baselineskip]}Dein {\\[\baselineskip]}\spacefill\mbox{Paul Goldmn}\pend
           \leftskip=0em{}\selectlanguage{ngerman}\endnumbering\briefempfaengerindex{Schnitzler, Arthur@\textsc{Schnitzler, Arthur}!zzzGoldmann, Paul@\emph{von Paul Goldmann}!1901-12-131@{13. 12. [1901]}|)be}\mylabel{L03095h}  \newcommand{\dateiname}{L03095}\newcommand{\titel}{Paul Goldmann an Arthur Schnitzler, 13. 12. [1901]}\newcommand{\editorInnen}{Martin Anton Müller und Laura Untner}%% latex-leseansicht-abspann.tex
%% Abspann für die Leseansicht.
%% Der Schalter \ifkorrekturansicht ist bereits durch den Vorspann gesetzt.

%% latex-abspann.tex
%% Gemeinsamer Abspann für Korrekturansicht und Leseansicht.
%% Setzt den Schalter \ifkorrekturansicht voraus (gesetzt in den
%% einbindenden Dateien latex-korrekturansicht-abspann.tex bzw.
%% latex-leseansicht-abspann.tex).
%% ---------------------------------------------------------------

\normalsize

% Das esempio-Environment wird nur in der Leseansicht benötigt
\ifkorrekturansicht\else
\newenvironment{esempio}[3]%
{
    \vspace{1.5ex}
    \rlap{\underline{#1}}
    \par
    \setlength{\parindent}{0cm}
    \nopagebreak
    \leftskip=#2cm
    \rightskip=#3cm
}
{
    \par
}
\fi

\doendnotes{C}
\bigskip
\vfill

\clearpage

\footnotesize

\ifkorrekturansicht
  \lohead{\textsc{register}}
\fi

% theindex-Environment neu definieren ohne reledmac
\makeatletter
\renewenvironment{theindex}{%
  \ifkorrekturansicht
    \section*{\indexname}%
  \else
    \subsubsection*{Index der erwähnten Entitäten}%
  \fi
  \setlength{\parindent}{0pt}%
  \setlength{\parskip}{0pt plus 0.3pt}%
  \let\item\@idxitem
}{%
  \ifkorrekturansicht\clearpage\fi
}
\makeatother

\IfFileExists{\jobname-pw.ind}{\input{\jobname-pw.ind}}{}

% Quellenangabe nur in der Leseansicht
\ifkorrekturansicht\else
% Fallback-Definitionen, falls die .tex-Datei \titel etc. nicht gesetzt hat
\providecommand{\titel}{}
\providecommand{\editorInnen}{}
\providecommand{\dateiname}{\jobname}

\vspace{3cm}

\vfill

\footnotesize
\textsc{Quelle}: \titel. Herausgegeben von {\editorInnen}. In: \emph{Arthur Schnitzler: Briefwechsel mit Autorinnen und Autoren}.
 Digitale Edition, https://schnitzler-briefe.acdh.oeaw.ac.at/{\dateiname}.html (Stand \today)
\fi

\end{document}


