%% latex-korrekturansicht-vorspann.tex
%% Vorspann für die Korrekturansicht.
%% Lädt die gemeinsame Datei latex-vorspann.tex mit gesetztem Schalter.

\newif\ifkorrekturansicht
\korrekturansichttrue

\input{../tex-inputs/latex-vorspann}


\section[Peter Altenberg an Arthur Schnitzler, {[}10.? 5. 1912{]}]{L02062 Peter Altenberg an Arthur Schnitzler, {[}10.? 5. 1912{]}}
\nopagebreak\mylabel{L02062v}
\rehead{ }\normalsize\beginnumbering\briefempfaengerindex{Schnitzler, Arthur@\textsc{Schnitzler, Arthur}!zzzAltenberg, Peter@\emph{von Peter Altenberg}!1912-05-101@{{[}10.? 5. 1912{]}}|(be}
\toendnotes[C]{\smallbreak\pagebreak[2]}\Standort{CUL, Schnitzler, B 2.}
\physDesc{Brief, 1 Blatt, 2 Seiten, 278 Zeichen
\newline{}Handschrift: blaue Tinte, deutsche Kurrent
\newline{}Schnitzler: mit rotem Buntstift beschrieben: »(an Tisch Mai
                                       1912« 
\newline{}Ordnung: mit Bleistift von unbekannter Hand neben den Wunsch nach einem
                                 Exemplar Vermerk: »erledigt« }\toendnotes[C]{\smallbreak}
\pstart
           \noindent{}\raggedleft{}{\pb}\textcolor{gray}{\textbf{Motto: \emph{Schneeglöcklein, läutest den Frühling
                     ein,}}}\pend
           
\pstart
           \raggedleft{}\textcolor{gray}{\textbf{\emph{Für mich begräbst du den herrlichen Winter.}}}\pend
           
\pstart
           \centering{}\textcolor{gray}{\textbf{HOTEL PANHANS AM SEMMERING\oindex{Hotel Panhans@\textbf{Hotel Panhans}, \emph{Hotel (K.HTL)}|pw}}}\pend
           
\pstart
           \centering{}\textcolor{gray}{\textbf{mit dazugehörigem Hotel Erzherzog
                     Johann\oindex{Hotel Erzherzog Johann@\textbf{Hotel Erzherzog Johann}, \emph{Hotel (K.HTL)}|pw}.}}\pend
           
\pstart
           
\pstart
           \textcolor{gray}{\textbf{1025 m Seehöhe.}}\pend
           
\pstart
           \raggedleft{}\textcolor{gray}{\textbf{1025 m Seehöhe.}}\pend
           \pend
           
\pstart
           \centering{}\textcolor{gray}{\textbf{400 Zimmer und Salons, meist mit Balkons, Gesellschaftsloggien
                  und gemeinsame Terrassen für Freiluft- und Liegekuren in jedem Stockwerke.}}\pend
           
\pstart
           \centering{}\textcolor{gray}{\textbf{Komplette Appartements mit Bad, Dusche und Toilette. Überall
                  elektrisches Licht und Warmwasserheizung, welche in jedem Zimmer genau regulierbar
                  (auch Wohnungen mit Öfen). Hausarzt, Apotheke, Lift. Photographische Dunkelkammer,
                  Automobil-Remise.}}\pend
           
\pstart
           \centering{}\textcolor{gray}{\textbf{Großes Kaffeehaus, luxuriöse Halle, Konversations-, Spiel-,
                  Lese-, Musik- und Damensalons. Feinstes Orchester vom 20. Juni bis 20. September
                  und vom 20. Dezember bis 20. März.}}\pend
           
\pstart
           \centering{}\textcolor{gray}{\textbf{Neben dem Hotel befindet sich das schmucke Semmering-Kirchlein\oindex{Kirche zur heiligen Familie@\textbf{Kirche zur heiligen Familie}, \emph{Kirche (K.KRC)}|pw} (jeden Tag heilige Messe).}}\pend
           
\pstart
           \centering{}\textcolor{gray}{\textbf{Wintersportplatz und Höhenkurort allerersten Ranges.}}\pend
           
\pstart
           \centering{}\textcolor{gray}{\textbf{Mittelpunkt des hiesigen Wintersports.}}\pend
           
\pstart
           \centering{}\textcolor{gray}{\textbf{Sitz des Österreichischen
                     Wintersport-Klubs\orgindex{Oesterreichischer Wintersport-Klub@Österreichischer Wintersport-Klub|pw} im Hotel Erzherzog
                     Johann\oindex{Hotel Erzherzog Johann@\textbf{Hotel Erzherzog Johann}, \emph{Hotel (K.HTL)}|pw}.}}\pend
           
\pstart
           \centering{}\textcolor{gray}{\textbf{Eigene Hochwildjagd, Forellenfischerei, Reitpferde. Fahrräder
                  und Wintersportrequisiten.}}\pend
           
\pstart
           \centering{}\textcolor{gray}{\textbf{Tennis-, Croquet-, Eislauf-, Ski- und Rodelplätze.}}\pend
           
\pstart
           \centering{}\textcolor{gray}{\textbf{Elektrischer Aufzug für Personen und Sportgeräte bei der 4 km
                  langen Rodel- und Bobbahn.}}\pend
           
\pstart
           \centering{}\textcolor{gray}{\textbf{Bade- und Wasserkur unter Leitung bewährter Ärzte. Kohlensäure-,
                  elektrische Dampfbäder, Inhalationen System Dr. Bulling\pwindex{Bulling, Anton 1853/1854 – 1918-05-12@\textsc{Bulling, Anton} (1853/1854 – 1918-05-12), \emph{Mediziner/Medizinerin}|pw}. Hochquellenleitung.}}\pend
           
\pstart
           \centering{}\textcolor{gray}{\textbf{Bester Nachkurort nach Karlsbad\oindex{Karlsbad@\textbf{Karlsbad}, \emph{P.PPLA}|pw}, Marienbad\oindex{Marienbad@\textbf{Marienbad}, \emph{P.PPL}|pw}, Franzensbad\oindex{Franzensbad@\textbf{Franzensbad}, \emph{P.PPL}|pw}, Teplitz\oindex{Teplice@\textbf{Teplice}, \emph{P.PPL}|pw},
                     Abbazia\oindex{Opatija@\textbf{Opatija}, \emph{P.PPLA2}|pw}, Meran\oindex{Opatija@\textbf{Opatija}, \emph{P.PPLA2}|pw}, Grado\oindex{Grado@\textbf{Grado}, \emph{P.PPLA3}|pw}, Gastein\oindex{Bad Gastein@\textbf{Bad Gastein}, \emph{P.PPLA3}|pw}, Pestyan\oindex{Piešťany@\textbf{Piešťany}, \emph{P.PPLA2}|pw}, Davos\oindex{Davos@\textbf{Davos}, \emph{P.PPL}|pw} usw. Winterkuren.}}\pend
           
\pstart
           \centering{}\textcolor{gray}{\textbf{Kammerlieferant der Kaiserl. Hoheiten Erzh. Franz Ferdinand\pwindex{Franz Ferdinand von Oesterreich-Este 18.12.1863 – 28.06.1914@\textsc{Franz Ferdinand von Österreich-Este} (18.12.1863 – 28.06.1914), \emph{Erzherzog/Erzherzogin, Thronfolger/Thronfolgerin}|pw}, Erzh. Karl\pwindex{Karl I. von Oesterreich-Ungarn 17.08.1887 – 01.04.1922@\textsc{Karl I. von Österreich-Ungarn} (17.08.1887 – 01.04.1922), \emph{Kaiser/Kaiserin, Offizier/Offizierin, Kaiserlicher Rat/Kaiserliche Rätin}|pw} und Erzh. Stephan\pwindex{Karl Stephan von Oesterreich 05.09.1860 – 07.04.1933@\textsc{Karl Stephan von Österreich} (05.09.1860 – 07.04.1933), \emph{Erzherzog/Erzherzogin}|pw}.}}\pend
           
\pstart
           \centering{}\textcolor{gray}{\textbf{Sieben zum Hotel gehörige Villen mit Küchen und
                  Herrschaftsstallungen.}}\pend
           
\pstart
           \centering{}\textcolor{gray}{\textbf{Vom Allerhöchsten Hofe und der hohen Aristokratie seit vielen
                  Jahren sehr bevorzugt.}}\pend
           
\pstart
           \centering{}\textcolor{gray}{\textbf{Acht Jahre Sommeraufenthalt des Reichskanzlers Fürsten Bülow\pwindex{Buelow, Bernhard von 03.05.1849 – 28.10.1929@\textsc{Bülow, Bernhard von} (03.05.1849 – 28.10.1929), \emph{Politiker/Politikerin}|pw}.}}\pend
           
\pstart
           \raggedleft{}\textcolor{gray}{\textbf{\textbf{Franz Panhans\pwindex{Panhans, Franz 1869-03-04 – 1913-09-20@\textsc{Panhans, Franz} (1869-03-04 – 1913-09-20), \emph{Hotelier/Hotelière, Hotelbesitzer/Hotelbesitzerin}|pw}}, Besitzer und persönlicher Leiter.}}\pend
           
\pstart
           \raggedleft{}\textcolor{gray}{\textbf{Semmering, am ..........}}\pend
           
\pstart
           Ich bitte ſehr, es dem Herrn \uline{\textsc{D\textsuperscript{r}} Arthur Schnitzler} mitzuteilen, daſs ich noch nie eine ſo feine Novelle
               geleſen habe wie: »\textsc{Der Tod {\pb}des
                     Junggesellen}\pwindex{Tod des Junggesellen. Novelle@\emph{Der Tod des Junggesellen. Novelle}|pw}« in ſeinem \label{K_L02062-1v}\edtext{neuen Buche}{\lemma{\textnormal{\emph{neuen Buche}}}\Cendnote{\textnormal{Schnitzler hatte am 6. 5. 1912
                  sein erstes Exemplar von \emph{Masken und Wunder. Novellen}\pwindex{Masken und Wunder. Novellen@\emph{Masken und Wunder. Novellen}|pwk} in der Hand. Da er im Mai keinen Aufenthalt
                  am Semmering\oindex{Semmering@\textbf{Semmering}, \emph{A.ADM3}|pwk} im \emph{Tagebuch}\pwindex{Tagebuch@\emph{Tagebuch}|pwk} erwähnt, bietet sich nur die Reise vom
                  10. 5. 1912 bis zum 11. 5. 1912 nach Triest\oindex{Triest@\textbf{Triest}, \emph{A.ADM3}|pwk} an, auf
                  der er den Semmering\oindex{Semmering@\textbf{Semmering}, \emph{A.ADM3}|pwk} passierte und
                  möglicherweise Zwischenstation eingelegt hatte und ein Exemplar »an Tisch« überbringen ließ.}}}\label{K_L02062-1}: »\uline{\textsc{Masken und Wunder}}\pwindex{Masken und Wunder. Novellen@\emph{Masken und Wunder. Novellen}|pw}«!\pend
           
\pstart
           Auch bitte ich um ein Exemplar dieſes Buches\pwindex{Masken und Wunder. Novellen@\emph{Masken und Wunder. Novellen}|pwv} gratis.\pend
           
\pstart
           Ihr{\\[\baselineskip]}\spacefill\mbox{Peter Altenberg}\pend
           \leftskip=0em{}
\pstart
           \noindent{}Semmering, Hotel Panhans\oindex{Hotel Panhans@\textbf{Hotel Panhans}, \emph{Hotel (K.HTL)}|pw}\pend
           \selectlanguage{ngerman}\endnumbering\briefempfaengerindex{Schnitzler, Arthur@\textsc{Schnitzler, Arthur}!zzzAltenberg, Peter@\emph{von Peter Altenberg}!1912-05-101@{{[}10.? 5. 1912{]}}|)be}\mylabel{L02062h}  \normalsize

\doendnotes{C}
\bigskip
\vfill

\clearpage

\footnotesize

\lohead{\textsc{register}}

% Definiere theindex-Environment komplett neu ohne reledmac
\makeatletter
\renewenvironment{theindex}{%
  \section*{\indexname}%
  \setlength{\parindent}{0pt}%
  \setlength{\parskip}{0pt plus 0.3pt}%
  \let\item\@idxitem
}{%
  \clearpage
}
\makeatother

\IfFileExists{\jobname-pw.ind}{\input{\jobname-pw.ind}}{}

\end{document}

      