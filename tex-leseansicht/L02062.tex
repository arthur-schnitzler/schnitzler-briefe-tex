%% latex-leseansicht-vorspann.tex
%% Vorspann für die Leseansicht.
%% Lädt die gemeinsame Datei latex-vorspann.tex mit nicht gesetztem Schalter.

\newif\ifkorrekturansicht
\korrekturansichtfalse

\input{../tex-inputs/latex-vorspann}


               \section[Peter Altenberg an Arthur Schnitzler, {[}10.? 5. 1912{]}]{ Peter Altenberg an Arthur Schnitzler, {[}10.? 5. 1912{]}}\nopagebreak\mylabel{v}\rehead{ }\begin{ledgroupsized}[t]{13cm}\normalsize\beginnumbering\briefempfaengerindex{Schnitzler, Arthur@\textsc{Schnitzler, Arthur}!zzzAltenberg, Peter@\emph{von Peter Altenberg}!1912-05-101@{{[}10.? 5. 1912{]}}|(be} \toendnotes[C]{\smallbreak\pagebreak[2]} \Standort{CUL, Schnitzler, B 2.}
\physDesc{Brief, 1 Blatt, 2 Seiten
\newline{}Handschrift: blaue Tinte, deutsche Kurrent
\newline{}Schnitzler: mit rotem Buntstift beschrieben: »(an Tisch Mai
                  1912« \newline{}Ordnung: mit Bleistift von unbekannter Hand neben den Wunsch nach einem
                                            Exemplar Vermerk: »erledigt« }\toendnotes[C]{\smallbreak}\pstart
           \noindent{}\raggedleft{}{\pb}\textcolor{gray}{\textbf{Motto: \emph{Schneeglöcklein, läutest den
                            Frühling ein,}}}\pend
           \pstart
           \noindent{}\raggedleft{}\textcolor{gray}{\textbf{\emph{Für mich begräbst du den herrlichen Winter.}}}\pend
           \pstart
           \noindent{}\centering{}\textcolor{gray}{\textbf{HOTEL PANHANS AM SEMMERING\oindex{Hotel Panhans@\textbf{Hotel Panhans}|pw}}}\pend
           \pstart
           \noindent{}\centering{}\textcolor{gray}{\textbf{mit dazugehörigem Hotel
                            Erzherzog Johann\oindex{Hotel Erzherzog Johann@\textbf{Hotel Erzherzog Johann}|pw}.}}\pend
           \pstart
           \noindent{}\textcolor{gray}{\textbf{1025 m Seehöhe.}}\hfill \textcolor{gray}{\textbf{1025 m Seehöhe.}}\pend
           \pstart
           \centering{}\textcolor{gray}{\textbf{400 Zimmer und Salons, meist mit Balkons,
                        Gesellschaftsloggien und gemeinsame Terrassen für Freiluft- und Liegekuren
                        in jedem Stockwerke.}}\pend
           \pstart
           \noindent{}\centering{}\textcolor{gray}{\textbf{Komplette Appartements mit Bad, Dusche und Toilette.
                        Überall elektrisches Licht und Warmwasserheizung, welche in jedem Zimmer
                        genau regulierbar (auch Wohnungen mit Öfen). Hausarzt, Apotheke, Lift.
                        Photographische Dunkelkammer, Automobil-Remise.}}\pend
           \pstart
           \noindent{}\centering{}\textcolor{gray}{\textbf{Großes Kaffeehaus, luxuriöse Halle, Konversations-, Spiel-,
                        Lese-, Musik- und Damensalons. Feinstes Orchester vom 20. Juni bis
                        20. September und vom 20. Dezember bis 20. März.}}\pend
           \pstart
           \noindent{}\centering{}\textcolor{gray}{\textbf{Neben dem Hotel befindet sich das schmucke Semmering-Kirchlein\oindex{Kirche zur heiligen Familie@\textbf{Kirche zur heiligen Familie}|pw} (jeden Tag heilige Messe).}}\pend
           \pstart
           \noindent{}\centering{}\textcolor{gray}{\textbf{Wintersportplatz und Höhenkurort allerersten Ranges.}}\pend
           \pstart
           \noindent{}\centering{}\textcolor{gray}{\textbf{Mittelpunkt des hiesigen Wintersports.}}\pend
           \pstart
           \noindent{}\centering{}\textcolor{gray}{\textbf{Sitz des Österreichischen
                            Wintersport-Klubs\orgindex{Oesterreichischer Wintersport-Klub@Österreichischer Wintersport-Klub|pw} im Hotel Erzherzog
                            Johann\oindex{Hotel Erzherzog Johann@\textbf{Hotel Erzherzog Johann}|pw}.}}\pend
           \pstart
           \noindent{}\centering{}\textcolor{gray}{\textbf{Eigene Hochwildjagd, Forellenfischerei, Reitpferde.
                        Fahrräder und Wintersportrequisiten.}}\pend
           \pstart
           \noindent{}\centering{}\textcolor{gray}{\textbf{Tennis-, Croquet-, Eislauf-, Ski- und Rodelplätze.}}\pend
           \pstart
           \noindent{}\centering{}\textcolor{gray}{\textbf{Elektrischer Aufzug für Personen und Sportgeräte bei der
                        4 km langen Rodel- und Bobbahn.}}\pend
           \pstart
           \noindent{}\centering{}\textcolor{gray}{\textbf{Bade- und Wasserkur unter Leitung bewährter Ärzte.
                        Kohlensäure-, elektrische Dampfbäder, Inhalationen System Dr. Bulling\pwindex{Bulling, Anton 1853/1854 – 1918-05-12@\textsc{Bulling, Anton} (1853/1854 – 1918-05-12), \emph{Mediziner}|pw}. Hochquellenleitung.}}\pend
           \pstart
           \noindent{}\centering{}\textcolor{gray}{\textbf{Bester Nachkurort nach Karlsbad\oindex{Karlsbad@\textbf{Karlsbad}|pw}, Marienbad\oindex{Marienbad@\textbf{Marienbad}|pw}, Franzensbad\oindex{Franzensbad@\textbf{Franzensbad}|pw}, Teplitz\oindex{Teplice@\textbf{Teplice}|pw}, Abbazia\oindex{Opatija@\textbf{Opatija}|pw}, Meran\oindex{Opatija@\textbf{Opatija}|pw}, Grado\oindex{Grado@\textbf{Grado}|pw}, Gastein\oindex{Bad Gastein@\textbf{Bad Gastein}|pw}, Pestyan\oindex{Piešťany@\textbf{Piešťany}|pw}, Davos\oindex{Davos@\textbf{Davos}|pw} usw.
                        Winterkuren.}}\pend
           \pstart
           \noindent{}\centering{}\textcolor{gray}{\textbf{Kammerlieferant der Kaiserl. Hoheiten Erzh. Franz Ferdinand\pwindex{Franz Ferdinand von Oesterreich-Este 18.12.1863 – 28.06.1914@\textsc{Franz Ferdinand von Österreich-Este} (18.12.1863 – 28.06.1914), \emph{Erzherzog, Thronfolger}|pw}, Erzh. Karl\pwindex{Karl I. von Oesterreich-Ungarn 17.08.1887 – 01.04.1922@\textsc{Karl I. von Österreich-Ungarn} (17.08.1887 – 01.04.1922), \emph{Kaiser}|pw} und Erzh. Stephan\pwindex{Karl Stephan von Oesterreich 05.09.1860 – 07.04.1933@\textsc{Karl Stephan von Österreich} (05.09.1860 – 07.04.1933), \emph{Erzherzog}|pw}.}}\pend
           \pstart
           \noindent{}\centering{}\textcolor{gray}{\textbf{Sieben zum Hotel gehörige Villen mit Küchen und
                        Herrschaftsstallungen.}}\pend
           \pstart
           \noindent{}\centering{}\textcolor{gray}{\textbf{Vom Allerhöchsten Hofe und der hohen Aristokratie seit
                        vielen Jahren sehr bevorzugt.}}\pend
           \pstart
           \noindent{}\centering{}\textcolor{gray}{\textbf{Acht Jahre Sommeraufenthalt des Reichskanzlers Fürsten Bülow\pwindex{Buelow, Bernhard von 03.05.1849 – 28.10.1929@\textsc{Bülow, Bernhard von} (03.05.1849 – 28.10.1929), \emph{Politiker}|pw}.}}\pend
           \pstart
           \noindent{}\raggedleft{}\textcolor{gray}{\textbf{\textbf{Franz Panhans\pwindex{Panhans, Franz 1869-03-04 – 1913-09-20@\textsc{Panhans, Franz} (1869-03-04 – 1913-09-20), \emph{Hotelbesitzer}|pw}}, Besitzer und persönlicher Leiter.}}\pend
           \pstart
           \noindent{}\raggedleft{}\textcolor{gray}{\textbf{Semmering, am ..........}}\pend
           \pstart
           \noindent{}Ich bitte ſehr, es dem Herrn \uline{\textsc{D\textsuperscript{r}} Arthur Schnitzler} mitzuteilen, daſs ich noch nie eine ſo feine
                    Novelle geleſen habe wie: »\textsc{Der Tod {\pb}des
                            Junggesellen}\pwindex{Schnitzler, Arthur 15.05.1862 – 21.10.1931@\textsc{Schnitzler, Arthur} (15.05.1862 – 21.10.1931), \emph{Schriftsteller, Mediziner}!Tod des Junggesellen1. 4. 1908@\strich\emph{Der Tod des Junggesellen} {[}1. 4. 1908{]}|pw}\pwindex{Schnitzler, Arthur 15.05.1862 – 21.10.1931@\textsc{Schnitzler, Arthur} (15.05.1862 – 21.10.1931), \emph{Schriftsteller, Mediziner}!Tod des Junggesellen1. 4. 1908@\strich\emph{Der Tod des Junggesellen} {[}1. 4. 1908{]}|pw}« in ſeinem \label{K_L02062_1v}\edtext{neuen Buche}{\lemma{\textnormal{\emph{neuen Buche}}}\Cendnote{\textnormal{Schnitzler\pwindex{Schnitzler, Arthur 15.05.1862 – 21.10.1931@\textsc{Schnitzler, Arthur} (15.05.1862 – 21.10.1931), \emph{Schriftsteller, Mediziner}|pwk} hatte am
                            6. 5. 1912 sein erstes Exemplar in der Hand. Nachdem er im
                            Mai keinen Aufenthalt am Semmering\oindex{Semmering@\textbf{Semmering}|pwk} im \emph{Tagebuch}\pwindex{Schnitzler, Arthur 15.05.1862 – 21.10.1931@\textsc{Schnitzler, Arthur} (15.05.1862 – 21.10.1931), \emph{Schriftsteller, Mediziner}!Tagebuch1981 – 2000@\strich\emph{Tagebuch} {[}1981 – 2000{]}|pwk} erwähnt,
                        bietet sich nur die Reise 10.–11. 5. 1912 nach Triest\oindex{Triest@\textbf{Triest}|pwk} an, auf der er den Semmering\oindex{Semmering@\textbf{Semmering}|pwk} passiert und möglicherweise Zwischenstation
                        einlegte.}}}\label{K_L02062_1h}: »\uline{\textsc{Masken und Wunder}}\pwindex{Schnitzler, Arthur 15.05.1862 – 21.10.1931@\textsc{Schnitzler, Arthur} (15.05.1862 – 21.10.1931), \emph{Schriftsteller, Mediziner}!Masken und Wunder. Novellen1912-05-06@\strich\emph{Masken und Wunder. Novellen} {[}1912-05-06{]}|pw}«!\pend
           \pstart
           Auch bitte ich um ein Exemplar dieſes Buches\pwindex{Schnitzler, Arthur 15.05.1862 – 21.10.1931@\textsc{Schnitzler, Arthur} (15.05.1862 – 21.10.1931), \emph{Schriftsteller, Mediziner}!Masken und Wunder. Novellen1912-05-06@\strich\emph{Masken und Wunder. Novellen} {[}1912-05-06{]}|pwv} gratis.\pend
           \pstart
           Ihr{\\[\baselineskip]}\spacefill\mbox{Peter Altenberg}\pend
           \leftskip=0em{}\pstart
           \noindent{}Semmering, Hotel Panhans\oindex{Hotel Panhans@\textbf{Hotel Panhans}|pw}\pend
                     \endnumbering\briefempfaengerindex{Schnitzler, Arthur@\textsc{Schnitzler, Arthur}!zzzAltenberg, Peter@\emph{von Peter Altenberg}!1912-05-101@{{[}10.? 5. 1912{]}}|)be}\mylabel{h}\end{ledgroupsized}  \newcommand{\dateiname}{L02062}\newcommand{\titel}{Peter Altenberg an Arthur Schnitzler, [10.? 5. 1912]}\newcommand{\editorInnen}{Martin Anton Müller und Gerd-Hermann Susen}
            \footnotesize
\begin{ledgroupsized}[t]{11.5cm}
\doendnotes{C}
\end{ledgroupsized}
         %% latex-leseansicht-abspann.tex
%% Abspann für die Leseansicht.
%% Der Schalter \ifkorrekturansicht ist bereits durch den Vorspann gesetzt.

%% latex-abspann.tex
%% Gemeinsamer Abspann für Korrekturansicht und Leseansicht.
%% Setzt den Schalter \ifkorrekturansicht voraus (gesetzt in den
%% einbindenden Dateien latex-korrekturansicht-abspann.tex bzw.
%% latex-leseansicht-abspann.tex).
%% ---------------------------------------------------------------

\normalsize

% Das esempio-Environment wird nur in der Leseansicht benötigt
\ifkorrekturansicht\else
\newenvironment{esempio}[3]%
{
    \vspace{1.5ex}
    \rlap{\underline{#1}}
    \par
    \setlength{\parindent}{0cm}
    \nopagebreak
    \leftskip=#2cm
    \rightskip=#3cm
}
{
    \par
}
\fi

\doendnotes{C}
\bigskip
\vfill

\clearpage

\footnotesize

\ifkorrekturansicht
  \lohead{\textsc{register}}
\fi

% theindex-Environment neu definieren ohne reledmac
\makeatletter
\renewenvironment{theindex}{%
  \ifkorrekturansicht
    \section*{\indexname}%
  \else
    \subsubsection*{Index der erwähnten Entitäten}%
  \fi
  \setlength{\parindent}{0pt}%
  \setlength{\parskip}{0pt plus 0.3pt}%
  \let\item\@idxitem
}{%
  \ifkorrekturansicht\clearpage\fi
}
\makeatother

\IfFileExists{\jobname-pw.ind}{\input{\jobname-pw.ind}}{}

% Quellenangabe nur in der Leseansicht
\ifkorrekturansicht\else
% Fallback-Definitionen, falls die .tex-Datei \titel etc. nicht gesetzt hat
\providecommand{\titel}{}
\providecommand{\editorInnen}{}
\providecommand{\dateiname}{\jobname}

\vspace{3cm}

\vfill

\footnotesize
\textsc{Quelle}: \titel. Herausgegeben von {\editorInnen}. In: \emph{Arthur Schnitzler: Briefwechsel mit Autorinnen und Autoren}.
 Digitale Edition, https://schnitzler-briefe.acdh.oeaw.ac.at/{\dateiname}.html (Stand \today)
\fi

\end{document}


      