%% latex-korrekturansicht-vorspann.tex
%% Vorspann für die Korrekturansicht.
%% Lädt die gemeinsame Datei latex-vorspann.tex mit gesetztem Schalter.

\newif\ifkorrekturansicht
\korrekturansichttrue

\input{../tex-inputs/latex-vorspann}


\section[Richard Beer-Hofmann an Arthur Schnitzler, 10. 8. 1913]{L02148 Richard Beer-Hofmann an Arthur Schnitzler, 10. 8. 1913}
\nopagebreak\mylabel{L02148v}
\rehead{ }\normalsize\beginnumbering\briefempfaengerindex{Schnitzler, Arthur@\textsc{Schnitzler, Arthur}!zzzBeer-Hofmann, Richard@\emph{von Richard Beer-Hofmann}!1913-08-101@{10. 8. 1913}|(be}
\toendnotes[C]{\smallbreak\pagebreak[2]}\Standort{CUL, Schnitzler, B 8.}
\physDesc{Bildpostkarte, 401 Zeichen
\newline{}Handschrift: Bleistift, lateinische Kurrent
\newline{}Versand: Stempel: »\nobreak{}\oindex{Santa Maria Elisabetta@\textbf{Santa Maria Elisabetta}, \emph{Bezirk (A.BZK)}|pwk}S. Elisabetta di Lido
                                       (Venezia), 10. 8. 13.\nobreak{}«.  
\newline{}Ordnung: mit Bleistift von unbekannter Hand nummeriert:
                                    »253« }
\buchAbdrucke{\weitereDrucke{Arthur Schnitzler, Richard Beer-Hofmann: \emph{Briefwechsel 1891–1931}. Wien, Zürich: \emph{Europaverlag} 1992, S. 218.} }\toendnotes[C]{\smallbreak}\pstart{}{\pb}Herrn\pend{}\pstart{}Arthur Schnitzler\pend{}\pstart{}Insel Brioni\oindex{Brijuni@\textbf{Brijuni}, \emph{P.PPL}|pw}\pend{}\pstart{}Austria\oindex{Oesterreich@\textbf{Österreich}, \emph{A.PCLI}|pw}.\pend{}{\bigskip}
\pstart
           \noindent{}\centering{}{\pb}\textcolor{gray}{\textbf{Lido\oindex{Lido@\textbf{Lido}, \emph{P.PPL}|pw} – Venezia\oindex{Venedig@\textbf{Venedig}, \emph{P.PPLA}|pw}. Hôtel des Bains\oindex{Grand Hotel des Bains@\textbf{Grand Hotel des Bains}, \emph{Hotel (K.HTL)}|pw}.}}\pend
           \vspace{1em}
\pstart
           \noindent{}{\pb}Lieber Arthur! Dies ist nun unsere Nordsee! Brioni\oindex{Brijuni@\textbf{Brijuni}, \emph{P.PPL}|pw} würde ich gerne sehen – mit Ihnen als Staffage – aber es
               würde die kurze Zeit die ich hierbleibe zersplittern. Bubis\pwindex{Beer-Hofmann, Gabriel 09.01.1901 – 24.03.1971@\textsc{Beer-Hofmann, Gabriel} (09.01.1901 – 24.03.1971), \emph{Schriftsteller/Schriftstellerin, Filmagent/Filmagentin}|pw} wegen – der Aufnahmsprüfung in die III machen und dazu
               vorbereitet werden muss soll ich schon am 1 Sept in Wien\oindex{Wien@\textbf{Wien}, \emph{A.ADM2}|pw}{ }sein. Wann sind Sie zurück?\pend
           
\pstart
           \label{T_L02148-1v}\edtext{Ihnen, Ihrer Frau\pwindex{Schnitzler, Olga 17.01.1882 – 13.01.1970@\textsc{Schnitzler, Olga} (17.01.1882 – 13.01.1970), \emph{Schauspieler/Schauspielerin, Sänger/Sängerin}|pwv}}{\lemma{\textnormal{\emph{Ihnen, Ihrer Frau}}}\Cendnote{\textnormal{ab hier oberhalb und verkehrt zum
                  Text}}}\label{T_L02148-1} u. d. Kindern\pwindex{Schnitzler, Heinrich 09.08.1902 – 12.07.1982@\textsc{Schnitzler, Heinrich} (09.08.1902 – 12.07.1982), \emph{Regisseur/Regisseurin, Schauspieler/Schauspielerin}|pwv}\pwindex{Cappellini, Lili 13.09.1909 – 26.07.1928@\textsc{Cappellini, Lili} (13.09.1909 – 26.07.1928)|pwv} herzliche Grüsse von uns Allen!\spacefill\mbox{R.}\pend
           \selectlanguage{ngerman}\endnumbering\briefempfaengerindex{Schnitzler, Arthur@\textsc{Schnitzler, Arthur}!zzzBeer-Hofmann, Richard@\emph{von Richard Beer-Hofmann}!1913-08-101@{10. 8. 1913}|)be}\mylabel{L02148h}  \normalsize

\doendnotes{C}
\bigskip
\vfill

\clearpage

\footnotesize

\lohead{\textsc{register}}

% Definiere theindex-Environment komplett neu ohne reledmac
\makeatletter
\renewenvironment{theindex}{%
  \section*{\indexname}%
  \setlength{\parindent}{0pt}%
  \setlength{\parskip}{0pt plus 0.3pt}%
  \let\item\@idxitem
}{%
  \clearpage
}
\makeatother

\IfFileExists{\jobname-pw.ind}{\input{\jobname-pw.ind}}{}

\end{document}

      