\input{../tex-inputs/latex-pdf-vorspann}
\begin{center}
            \textcolor{red}{ENTWURF. ENTZIFFERUNG NOCH NICHT KORREKTURGELESEN}
                      \end{center}
            
               \section[Richard Dehmel an Arthur Schnitzler, 14. 8. 1912]{ Richard Dehmel an Arthur Schnitzler, 14. 8. 1912}\nopagebreak\mylabel{v}\rehead{ }\begin{ledgroupsized}[t]{13cm}\normalsize\beginnumbering\briefempfaengerindex{Schnitzler, Arthur@\textsc{Schnitzler, Arthur}!zzzDehmel, Richard@\emph{von Richard Dehmel}!1912-08-141@{14. 8. 1912}|(be} \toendnotes[C]{\smallbreak\pagebreak[2]} \Standort{DLA, A:Schnitzler, HS.71.73.}
\physDesc{Brief, 1 Blatt, 1 Seite
\newline{}Handschrift: schwarze Tinte, lateinische Kurrent\newline{}Beilage: Druck, 1 Blatt, 2 Seiten, Antiqua 
\newline{}Schnitzler: 1) mit Bleistift beschriftet »Deh\textcolor{gray}{m}« 2) mit rotem Buntstift eine Unterstreichung}\buchAbdrucke{\weitereDrucke{Hans-Ulrich Lindken: \emph{Arthur Schnitzler. Aspekte und Akzente. Materialien zu
                                Leben und Werk}. Frankfurt am Main, Bern, Göttingen: \emph{Peter Lang} 1984, S. 221–222 (Europäische Hochschulschriften, Reihe 1, Deutsche
                                Sprache und Literatur, 754).} }\toendnotes[C]{\smallbreak}\pstart
           {\pb}Blankenese \textsuperscript{b}/Hamburg\oindex{Blankenese@\textbf{Blankenese}|pw}, 14. 8. 12. \pend
           \pstart{}Lieber Herr Schnitzler!\pend\pstart
           Professor Köster\pwindex{Koester, Albert 1862-11-07 – 1924-05-29@\textsc{Köster, Albert} (1862-11-07 – 1924-05-29), \emph{Schriftsteller, Germanist, Theaterwissenschaftler}|pw} schreibt mir, daß Sie ihm
                    Ihre Unterschrift zu dem anhängenden Aufruf\pwindex{Aufruf zur Kritik an der Verwendung von Spendengeldern fuer Autoren durch die Deutsche Schillerstiftung1912@\emph{Aufruf zur Kritik an der Verwendung von Spendengeldern für Autoren durch die Deutsche Schillerstiftung} {[}1912{]}|pwv} nicht geschickt haben. Ich kann mir nicht
                    denken, daß Sie es mit Absicht unterlassen haben. Es ist doch tatsächlich ein
                    blödsinniger Unfug, daß soviel schönes Geld an Stümper verplempert wird, während
                    sich tüchtige Leute als Schuldenmacher durchschinden müssen und dabei Zeit und
                    Arbeitslust einbüßen. Ich meine, da muß Jeder, auf den die öffentliche Meinung
                    hört, seinen Namen hergeben, um diese faule Wirtschaft endlich ändern zu helfen.
                    Bitte schreiben Sie mir \uline{gleich} eine zusagende
                    Zeile, eh Sie’s wieder vergessen! \pend
           \pstart
           Mit einem sehr ergebenen Gruß{\\[\baselineskip]}Ihr\spacefill\mbox{Dehmel.}\pend
           \leftskip=0em{}{\bigskip}\pstart
           \raggedleft{}{\pb}Juni
                        1912.\pend
           \pstart{}Euer Hochwohlgeboren\pend\pstart
           wird zur Kenntnis gekommen sein, daß am 17. März ds. Js.
                    eine Versammlung angesehener Schriftsteller in Berlin\oindex{Berlin@\textbf{Berlin}|pw} die Verwaltungsberichte der \so{Deutschen Schillerstiftung}\orgindex{Deutsche Schillerstiftung@Deutsche Schillerstiftung|pw} geprüft und in
                    sehr vielen Fällen die Verwendung der Stiftungsgelder als satzungswidrig
                    befunden hat. Zugleich wurden die Unterzeichneten damit betraut, Schritte zu
                    tun, die eine dauernde Abstellung dieses Mißstandes durchsetzen könnten.\pend
           \pstart
           Nach den Satzungen der Stiftung ist es ihr \so{Hauptzweck},
                    die verfügbaren Gelder als Ehrengaben an Schriftsteller zu verteilen, die einer
                    Unterstützung bedürftig und würdig sind, »vorzugsweise an solche, die sich
                    dichterischer Formen bedient haben«. Die Würdigkeit ist ausdrücklich dahin
                    begrenzt, daß ein »Verdienst um die Nationalliteratur« vorliegen müsse.
                    Tatsächlich aber sind in den letzten Jahrzehnten die Stiftungsgelder großenteils
                    an literarisch wertlose Personen vergeben worden, während bedürftige Dichter und
                    Schriftsteller, deren Wert heute weithin anerkannt ist, entweder gar keine oder
                    ungenügende Unterstützung empfingen.\pend
           \pstart
           Wenn man erwägt, daß die Stiftung jetzt jährlich etwa 80000 Mark
                    auszuspenden hat — im letzten Jahre waren es über 82000 —: dann fragt man
                    mit Verwunderung, wieso sich ein deutscher Dichter von Bedeutung überhaupt noch
                    in Not befinden kann. Was könnte man ausrichten mit \so{so reichlichen Mitteln}, wenn sie nicht immer wieder in kleinen Almosen an
                    die breite Menge der Schwächlinge verzettelt würden, sondern in wirklich
                    nennenswerten Ehrenspenden den stark Begabten zugute kämen! Man hat eingewendet,
                    der Wortlaut der Satzungen erschwere die Austeilung größerer Spenden; aber die
                    Erschwerung ist kein Hinderungsgrund und muß eben irgendwie überwunden werden.
                    Es tut not, junge Kräfte, die sich bereits bewährt haben, vor Verkümmerung zu
                    bewahren und den reifen die Ausdauer in der Durchführung ungewöhnlicher Pläne zu
                    sichern.\pend
           \pstart
           Wir verkennen nicht, wie schwierig es ist, die jeweils Würdigsten auszuwählen,
                    besonders in unsrer geistig vielspältigen Zeit, die immerfort neue Vorstöße nach
                    den \so{verschiedensten Richtungen} macht. Wir möchten
                    deshalb den Verwaltern der Stiftung\orgindex{Deutsche Schillerstiftung@Deutsche Schillerstiftung|pwv}
                    bei dieser schwierigen Aufgabe an die Hand gehn; um aber nicht in den Verdacht
                    zu geraten, daß wir einseitige Ziele verfolgen, ersuchen wir hierdurch eine
                    große Anzahl namhafter Mitarbeiter am deutschen Geist, sich mit uns
                    zusammenzutun und dem Verwaltungsrat Vorschläge zu {\pb}machen, wie das ihm anvertraute Nationalvermögen wohl am ersprießlichsten zu
                    verwenden sei.\pend
           \pstart
           Unsre Absicht ist, den Zentralvorstand der Schillerstiftung\orgindex{Deutsche Schillerstiftung@Deutsche Schillerstiftung|pw} zu ersuchen, daß er alljährlich eine gewisse Summe,
                    und wäre es nur \so{die Hälfte} der auszuspendenden
                    Zinsgelder, an \so{einige wenige} Schriftsteller,
                    insbesondere Dichter, verteilen möge, die ein aus unserm Berufskreise zu
                    ernennender Vertrauensmann (oder eine Gruppe von Vertrauensleuten) ihm \so{jedesmal vorschlagen soll}. Wenn die wenigen
                    Persönlichkeiten, für deren Begabung wir vor der Mit- und Nachwelt die
                    Verantwortung auf uns nehmen, je nach Bedürfnis Ehrengehälter von ausreichender
                    Höhe und Dauer empfangen, so sichert das in der Tat ihre Schaffensfreiheit, oder
                    später nötigenfalls ihren Ruhestand, zu ihrer und unsres Volkes Ehre. Der Rest
                    der verfügbaren Zinssumme möge dann immerhin wie bisher den gewöhnlicheren
                    Anwärtern in kleineren Gaben verabreicht werden.\pend
           \pstart
           Natürlich können und wollen wir nicht verlangen, daß sich die Verwaltung der Schillerstiftung\orgindex{Deutsche Schillerstiftung@Deutsche Schillerstiftung|pw} unserm Urteil in bezug auf die
                    Würdigkeit der vorzuschlagenden Schriftsteller ein für allemal unterwerfe. Wir
                    wollen uns mit der Verwaltung vorerst nur darüber verständigen, ob sie \so{grundsätzlich} bereit sein würde, die Vorschläge
                    unsres Vertrauensmannes (oder unsrer Vertrauensleute) regelmäßig
                    entgegenzunehmen und wohlwollend zu erwägen. Die Verwaltung wird darauf um so
                    eher eingehen, je mehr Namen von anerkanntem Wert unter unserm Antrag vereinigt
                    stehen, und zwar gerade auch solche, die vielleicht Anspruch auf die Hilfsmittel
                    der Stiftung haben.\pend
           \pstart
           Wenn Euer Hochwohlgeboren geneigt si nd, uns für diesen Zweck \so{Ihre Unterschrift} zur Verfügung zu stellen, so bitten
                    wir Sie, Ihr Einverständnis \so{binnen längstens vierzehn Tagen} dem mitunterzeichneten Geheimen Hofrat Professor Dr. Köster\pwindex{Koester, Albert 1862-11-07 – 1924-05-29@\textsc{Köster, Albert} (1862-11-07 – 1924-05-29), \emph{Schriftsteller, Germanist, Theaterwissenschaftler}|pw} kundzugeben, unter der Adresse: Leipzig-Gohlis, Schönhausenstraße 6\oindex{Fritz-Seger-Strasse@\textbf{Fritz-Seger-Straße}|pw}. \pend
           \pstart \spacefill\mbox{Max Bernstein\pwindex{Bernstein, Max 12.05.1854 – 08.03.1925@\textsc{Bernstein, Max} (12.05.1854 – 08.03.1925), \emph{Schriftsteller, Kritiker, Rechtsanwalt}|pw}.}\spacefill\mbox{Richard Dehmel.}\spacefill\mbox{Albert Köster\pwindex{Koester, Albert 1862-11-07 – 1924-05-29@\textsc{Köster, Albert} (1862-11-07 – 1924-05-29), \emph{Schriftsteller, Germanist, Theaterwissenschaftler}|pw}.}\spacefill\mbox{Wilhelm Schäfer\pwindex{Schaefer, Wilhelm 20.01.1868 – 19.01.1952@\textsc{Schäfer, Wilhelm} (20.01.1868 – 19.01.1952), \emph{Schriftsteller, Publizist}|pw}.}\spacefill\mbox{Paul
                        Schlenther\pwindex{Schlenther, Paul 20.08.1854 – 30.04.1916@\textsc{Schlenther, Paul} (20.08.1854 – 30.04.1916), \emph{Schriftsteller, Kritiker, Theaterleiter}|pw}.}\pend{}\endnumbering\briefempfaengerindex{Schnitzler, Arthur@\textsc{Schnitzler, Arthur}!zzzDehmel, Richard@\emph{von Richard Dehmel}!1912-08-141@{14. 8. 1912}|)be}\mylabel{h}\end{ledgroupsized}  \newcommand{\dateiname}{L02084}\newcommand{\titel}{Richard Dehmel an Arthur Schnitzler, 14. 8. 1912}\newcommand{\editorInnen}{ Martin Anton Müller und Gerd-Hermann Susen}\input{../tex-inputs/latex-pdf-abspann}
      