%% latex-leseansicht-vorspann.tex
%% Vorspann für die Leseansicht.
%% Lädt die gemeinsame Datei latex-vorspann.tex mit nicht gesetztem Schalter.

\newif\ifkorrekturansicht
\korrekturansichtfalse

\input{../tex-inputs/latex-vorspann}


\section[Wilhelm Bölsche an Arthur Schnitzler, 17. 9. 1890]{L00004 Wilhelm Bölsche an Arthur Schnitzler, 17. 9. 1890}
\nopagebreak\mylabel{L00004v}
\rehead{ }\normalsize\beginnumbering\briefempfaengerindex{Schnitzler, Arthur@\textsc{Schnitzler, Arthur}!zzzBölsche, Wilhelm@\emph{von Wilhelm Bölsche}!1890-09-171@{17. 9. 1890}|(be}
\toendnotes[C]{\smallbreak\pagebreak[2]}
\correspDesc{Versand  durch Wilhelm Bölsche am 17. 9. 1890 in Berlin
\newline{}Erhalt  durch Arthur Schnitzler im Zeitraum [18. 9. 1890
                  – 22. 9. 1890?] in Wien}\toendnotes[C]{\smallbreak}
\Standort{TMW, HS Schn 1/63/1.}
\physDesc{Brief, 1 Blatt, 2 Seiten, 454 Zeichen
\newline{}Handschrift: schwarze Tinte, deutsche Kurrent
\newline{}Schnitzler: mit rotem Buntstift nummeriert: »1« }\toendnotes[C]{\smallbreak}
\pstart
           \centering{}{\pb}\textcolor{gray}{\textbf{\textsc{Freie Bühne\pwindex{Freie Bühne für modernes Leben@\emph{Freie Bühne für modernes Leben}|pw}}}}\pend
           
\pstart
           \centering{}\textcolor{gray}{\textbf{\textsc{für modernes Leben.}}}\pend
           
\pstart
           \centering{}\textcolor{gray}{\textbf{\textsc{Herausgegeben von \textbf{Otto Brahm}\pwindex{Brahm, Otto 5.\,2.\,1856 Hamburg – 28.\,11.\,1912 Berlin@\textsc{Brahm, Otto} (5.\,2.\,1856 Hamburg – 28.\,11.\,1912 Berlin), \emph{Theaterleiter, Regisseur}|pw}.}}}\pend
           
\pstart
           \textcolor{gray}{\textbf{Verlag und Expedition: S. Fischer\orgindex{S. Fischer Verlag@S. Fischer Verlag|pw}.}}\pend
           
\pstart
           \textcolor{gray}{\textbf{Sprechstunden: Mittwoch und Freitag 12–2 Uhr.}}\pend
           
\pstart
           \textcolor{gray}{\textbf{Alle für die Redaction bestimmten Sendungen (Beiträge,
                     Recensions-Exempl.) bitten wir \textbf{ohne Angabe eines
                        Personennamens} an die Redaction der Wochenschrift »\so{Freie Bühne}\pwindex{Freie Bühne für modernes Leben@\emph{Freie Bühne für modernes Leben}|pw}« Berlin W. Link-Strasse 25\oindex{Linkstraße@\textbf{Linkstraße}, \emph{Straße}|pw} zu
                     addressiren.}}\pend
           
\pstart
           \textcolor{gray}{\textbf{Wir ersuchen unsere geehrten Mitarbeiter, jedes Manuscript
                     auf der ersten Seite mit ihrer genauen Adresse zu versehen.}}\pend
           
\pstart
           \raggedleft{}\textcolor{gray}{\textbf{\textsc{Berlin\oindex{Berlin@\textbf{Berlin}, \emph{Hauptstadt}|pw}}, den}}{ }17. IX. \textcolor{gray}{\textbf{189}}0.\pend
           
\pstart
           \raggedleft{}\textcolor{gray}{\textbf{W. Link-Straße 25\oindex{Linkstraße@\textbf{Linkstraße}, \emph{Straße}|pw}.}}\pend
           
\pstart\center{}Hochgeehrter Herr Doktor!\pend\vspace{0.5em}
\pstart
           Ihre dramatiſche Skizze\pwindex{Schnitzler, Arthur 15.\,5.\,1862 Wien – 21.\,10.\,1931 ebd.@\textsc{Schnitzler, Arthur} (15.\,5.\,1862 Wien – 21.\,10.\,1931 ebd.), \emph{Schriftsteller, Mediziner}!Aus der Kaffeehausecke@\strich\emph{Aus der Kaffeehausecke}|pwv} habe
               ich mit Intereſſe geleſen, kann mich aber doch nicht recht mit ihr befreunden. Der
               Grundgedanke iſt originell, aber der Dialog{ }ſagt mir nicht zu. Bei breiterer
               Ausmalung würde man an den Fall glauben, –{ }ſo grell nicht! Es iſt eben eine
               verzweifelt{ }ſchwere Sache um{ }ſolche Skizzen. Doch bitte ich recht{ }ſehr, gelegentlich
               etwas anderes einzuſenden.\pend
           
\pstart
           Mit vorzüglicher Hochachtung{\\[\baselineskip]}\spacefill\mbox{Wilhelm Bölsche.}\pend
           \leftskip=0em{}\selectlanguage{ngerman}\endnumbering\briefempfaengerindex{Schnitzler, Arthur@\textsc{Schnitzler, Arthur}!zzzBölsche, Wilhelm@\emph{von Wilhelm Bölsche}!1890-09-171@{17. 9. 1890}|)be}\mylabel{L00004h}  \newcommand{\dateiname}{L00004}\newcommand{\titel}{Wilhelm Bölsche an Arthur Schnitzler, 17. 9. 1890}\newcommand{\editorInnen}{Martin Anton Müller und Gerd-Hermann Susen}%% latex-leseansicht-abspann.tex
%% Abspann für die Leseansicht.
%% Der Schalter \ifkorrekturansicht ist bereits durch den Vorspann gesetzt.

%% latex-abspann.tex
%% Gemeinsamer Abspann für Korrekturansicht und Leseansicht.
%% Setzt den Schalter \ifkorrekturansicht voraus (gesetzt in den
%% einbindenden Dateien latex-korrekturansicht-abspann.tex bzw.
%% latex-leseansicht-abspann.tex).
%% ---------------------------------------------------------------

\normalsize

% Das esempio-Environment wird nur in der Leseansicht benötigt
\ifkorrekturansicht\else
\newenvironment{esempio}[3]%
{
    \vspace{1.5ex}
    \rlap{\underline{#1}}
    \par
    \setlength{\parindent}{0cm}
    \nopagebreak
    \leftskip=#2cm
    \rightskip=#3cm
}
{
    \par
}
\fi

\doendnotes{C}
\bigskip
\vfill

\clearpage

\footnotesize

\ifkorrekturansicht
  \lohead{\textsc{register}}
\fi

% theindex-Environment neu definieren ohne reledmac
\makeatletter
\renewenvironment{theindex}{%
  \ifkorrekturansicht
    \section*{\indexname}%
  \else
    \subsubsection*{Index der erwähnten Entitäten}%
  \fi
  \setlength{\parindent}{0pt}%
  \setlength{\parskip}{0pt plus 0.3pt}%
  \let\item\@idxitem
}{%
  \ifkorrekturansicht\clearpage\fi
}
\makeatother

\IfFileExists{\jobname-pw.ind}{\input{\jobname-pw.ind}}{}

% Quellenangabe nur in der Leseansicht
\ifkorrekturansicht\else
% Fallback-Definitionen, falls die .tex-Datei \titel etc. nicht gesetzt hat
\providecommand{\titel}{}
\providecommand{\editorInnen}{}
\providecommand{\dateiname}{\jobname}

\vspace{3cm}

\vfill

\footnotesize
\textsc{Quelle}: \titel. Herausgegeben von {\editorInnen}. In: \emph{Arthur Schnitzler: Briefwechsel mit Autorinnen und Autoren}.
 Digitale Edition, https://schnitzler-briefe.acdh.oeaw.ac.at/{\dateiname}.html (Stand \today)
\fi

\end{document}


