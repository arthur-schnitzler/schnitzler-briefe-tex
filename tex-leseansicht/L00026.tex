%% latex-korrekturansicht-vorspann.tex
%% Vorspann für die Korrekturansicht.
%% Lädt die gemeinsame Datei latex-vorspann.tex mit gesetztem Schalter.

\newif\ifkorrekturansicht
\korrekturansichttrue

\input{../tex-inputs/latex-vorspann}


\section[Hugo von Hofmannsthal an Arthur Schnitzler, {[}Anfang August{]} 1891]{L00026 Hugo von Hofmannsthal an Arthur Schnitzler, {[}Anfang August{]}
               1891}
\nopagebreak\mylabel{L00026v}
\rehead{ }\normalsize\beginnumbering\briefempfaengerindex{Schnitzler, Arthur@\textsc{Schnitzler, Arthur}!zzzHofmannsthal, Hugo von@\emph{von Hugo von Hofmannsthal}!1891-08-011@{{[}Anfang August{]} 1891}|(be}
\toendnotes[C]{\smallbreak\pagebreak[2]}\Standort{CUL, Schnitzler, B 43.}
\physDesc{Brief, 1 Blatt, 4 Seiten, 2271 Zeichen
\newline{}Handschrift: schwarze Tinte, deutsche Kurrent
\newline{}Schnitzler: mit Bleistift datiert: »Anf Jul 91« 
\newline{}Ordnung: mit Bleistift von unbekannter Hand nummeriert:
                                 »3« }
\buchAbdrucke{\weitereDrucke{1) Hugo von Hofmannsthal: \emph{Briefe. 1890–1901}. Berlin: \emph{S. Fischer} 1935, S. 23–24.} \weitereDrucke{2) Hugo von Hofmannsthal, Arthur Schnitzler: \emph{Briefwechsel}. Frankfurt am Main: \emph{S. Fischer} 1964, S. 10–11.} }\toendnotes[C]{\smallbreak}
\pstart
           \noindent{}{\pb}Ich danke Ihnen wirklich für
               Ihren Brief. Sie müſſen ihn ſehr ſchwer geſchrieben haben. Ich habe das damals
               empfunden und empfinde es jetzt wieder.\pend
           
\pstart
           \label{K_L00026-1v}\edtext{Damals}{\lemma{\textnormal{\emph{Damals}}}\Cendnote{\textnormal{zwischen dem 22. und 31. 7. 1891, vgl. Hugo von Hofmannsthal\pwindex{Hofmannsthal, Hugo von 1874-02-01 – 1929-07-15@\textsc{Hofmannsthal, Hugo von} (1874-02-01 – 1929-07-15), \emph{Schriftsteller/Schriftstellerin}|pwk}: \emph{Aufzeichnungen}. Herausgegeben von Rudolf Hirsch † und Ellen Ritter † in
                     Zusammenarbeit mit Konrad Heumann und Peter Michael Braunwarth. Frankfurt am
                     Main: \emph{S. Fischer}\orgindex{S. Fischer Verlag@S. Fischer Verlag|pwk}{ }2013, S. 128 (\emph{Sämtliche Werke},
                     XXXIX).}}}\label{K_L00026-1} – um mich, als ich ihn las, ſtanden \textsc{Robert}\pwindex{Hirschfeld, Robert 17.09.1857 – 02.04.1914@\textsc{Hirschfeld, Robert} (17.09.1857 – 02.04.1914), \emph{Journalist/Journalistin, Musikkritiker/Musikkritikerin}|pw} und \textsc{Olga} Hirſchfeld\pwindex{Hirschfeld, Olga 11.10.1872 – 01.03.1940@\textsc{Hirschfeld, Olga} (11.10.1872 – 01.03.1940)|pw}, Schwarzkopf\pwindex{Schwarzkopf, Gustav 07.11.1853 – 13.11.1939@\textsc{Schwarzkopf, Gustav} (07.11.1853 – 13.11.1939), \emph{Schriftsteller/Schriftstellerin}|pw} und \textsc{Boris Fan-Junk}\pwindex{Van-Jung, Boris 15.10.1872 – 03.10.1899@\textsc{Van-Jung, Boris} (15.10.1872 – 03.10.1899), \emph{Mediziner/Medizinerin}|pw} – berührte er mich wie eine Erinnerung an Längſtvergeſſenes,
               Unerreichbar-fernes. Sie fragten nach meinen Arbeiten. Sie gedachten gemeinſamer
               Pläne. Um mich und in mir waren neue Dinge, Gleiten, Plätſchern, Rieſeln, Auflöſung,
               vages Verſchwimmen. Ich kann nicht arbeiten. Heute ſo wenig als damals. Noch weniger
                  {\pb}vielleicht. Ich gleite, ich
               treibe. Kein Gedanke cryſtalliſiert ſich und es wird kein Vers. Ich kann nicht weiter
               denken als Stunden.\pend
           
\pstart
           Aber mir iſt wohl. Anders wohl, neu wohl, wechſelnd wohl. Ich fühle mich wachſen.
               Wollt ich mich zwingen, müſst ich verzweifeln\strikeout{d},
               abwartend ſehe ich mir fluthen zu und empfinde ein glückliches Michbeſcheiden, das
               gute Schweſtergefühl zur Reſignation. Wäre nur mehr Sonne. So aber bin ich
               verſchnupft und krank möcht ich nicht werden, denn ich kann jetzt das Alleinſein
               nicht brauchen. Wenn Sie vielleicht in der Kunſtchronik\pwindex{Allgemeine Kunst-Chronik@\emph{Allgemeine Kunst-Chronik}|pw} meinem \label{K_L00026-2v}\edtext{Salzburg\oindex{Salzburg@\textbf{Salzburg}, \emph{A.ADM2}|pw}erbericht\pwindex{Mozart-Centenarfeier in Salzburg@\emph{Die Mozart-Centenarfeier in Salzburg}|pwv}}{\lemma{\textnormal{\emph{Salzburgerbericht}}}\Cendnote{\textnormal{Loris\pwindex{Hofmannsthal, Hugo von 1874-02-01 – 1929-07-15@\textsc{Hofmannsthal, Hugo von} (1874-02-01 – 1929-07-15), \emph{Schriftsteller/Schriftstellerin}|pwk}: \emph{Die Mozart-Centenarfeier in Salzburg}\pwindex{Mozart-Centenarfeier in Salzburg@\emph{Die Mozart-Centenarfeier in Salzburg}|pwk}. In: \emph{Allgemeine Kunst-Chronik}\pwindex{Allgemeine Kunst-Chronik@\emph{Allgemeine Kunst-Chronik}|pwk}, Bd. 15, Nr. 16, 1. August-Heft,
                        1. 8. 1891, S. 423–433.}}}\label{K_L00026-2}{ }\label{K_L00026-3v}\edtext{begegnen}{\lemma{\textnormal{\emph{begegnen}}}\Cendnote{\textnormal{Die \emph{Mozart\pwindex{Mozart, Wolfgang Amadeus 27.01.1756 – 05.12.1791@\textsc{Mozart, Wolfgang Amadeus} (27.01.1756 – 05.12.1791), \emph{Komponist/Komponistin}|pwk}-Zentenarfeier}\orgindex{Mozart Zentenarfeier 14.–17. 7. 1891@Mozart Zentenarfeier 14.–17. 7. 1891|pwk} fand vom
                  14. 7. 1891 bis zum 17. 7. 1891 in Salzburg\oindex{Salzburg@\textbf{Salzburg}, \emph{A.ADM2}|pwk}{ }statt. Dadurch ist die Datierung von Schnitzler mit »Anf Jul 91« auszuschließen. Wahrscheinlicher antwortet der Brief auf Schnitzlers Schreiben vom 27. 7. 1891. Das Erscheinen des Artikels\pwindex{Mozart-Centenarfeier in Salzburg@\emph{Die Mozart-Centenarfeier in Salzburg}|pwkv} begrenzt die Datierung nach
                  hinten auf Anfang August.}}}\label{K_L00026-3}, ſo laſſen Sie ſich von mir {\pb}ein paar Vorworte ſagen. Ich habe
               dort in 4 Tagen und 2 Nächten die concentrierteſte Menge von Eindrücken
               zuſammengetrunken, die mein Nervenſyſtem überhaupt vorläufig erträgt. Den Bericht
               habe ich im vollſtändigen Halbſchlaf geſchrieben in dem ſeltſamen Zuſtand, wo das
               Gehirn loſe Bilder, Geſprächstheile der letzten Nacht mit ſchmerzender Deutlichkeit
               bis zum Ekel reproduciert. Wenn der Bericht überhaupt deutſch iſt (ich habe ihn noch
               nicht bekommen) dann ſchläft in mir ein unbewuſster Reporter, \label{K_L00026-4v}\edtext{\textsc{qui parfois se réveille}}{\lemma{\textnormal{\emph{qui parfois se réveille}}}\Cendnote{\textnormal{französisch: der gelegentlich erwacht;
                  Zitat in der Gestalt nicht nachweisbar.}}}\label{K_L00026-4} wie \textsc{Ste. Beuve}\pwindex{Sainte-Beuve, Charles Augustin de 23.12.1804 – 13.10.1869@\textsc{Sainte-Beuve, Charles Augustin de} (23.12.1804 – 13.10.1869), \emph{Schriftsteller/Schriftstellerin}|pw}{ }ſagt. D\textsuperscript{r}\textsc{Hoffmann}\pwindex{Beer-Hofmann, Richard 1866-07-11 – 1945-09-26@\textsc{Beer-Hofmann, Richard} (1866-07-11 – 1945-09-26), \emph{Schriftsteller/Schriftstellerin}|pw} hat mir auf einen 4 Seiten langen Brief nach Wien\oindex{Wien@\textbf{Wien}, \emph{A.ADM2}|pw} nicht geantwortet; ich habe ihm nach {\pb}\textsc{Markt-Aussee}\oindex{Bad Aussee@\textbf{Bad Aussee}, \emph{P.PPLA3}|pw} (??) geſchrieben er ſoll doch zum Teufel hieher kommen. Warum kommt er denn
               nicht?!!! Ich arbeite \uline{garnichts} und hoffe daß die
               Comités der Freien Bühne\orgindex{»Freie Buehne« Verein fuer moderne Literatur@»Freie Bühne« Verein für moderne Literatur|pw} das Gegentheil
               thuen.\pend
           
\pstart
           Während der Eiſenbahnfahrt nach Wien\oindex{Wien@\textbf{Wien}, \emph{A.ADM2}|pw}
                  (15 September) ſchreibe ich\pend
           
\pstart
           1.) die letzte Scene von »Geſtern\pwindex{Gestern. Dramatische Studie in einem Akt in Versen@\emph{Gestern. Dramatische Studie in einem Akt in Versen}|pw}«\pend
           
\pstart
           2.) \label{K_L00026-5v}\edtext{\textsc{Maurice Barrès}\pwindex{Barres, Maurice 1862-08-19 – 1923-12-04@\textsc{Barrès, Maurice} (1862-08-19 – 1923-12-04), \emph{Schriftsteller/Schriftstellerin}|pw}\pwindex{Maurice Barres@\emph{Maurice Barrès}|pwv}}{\lemma{\textnormal{\emph{Maurice Barrès}}}\Cendnote{\textnormal{Loris\pwindex{Hofmannsthal, Hugo von 1874-02-01 – 1929-07-15@\textsc{Hofmannsthal, Hugo von} (1874-02-01 – 1929-07-15), \emph{Schriftsteller/Schriftstellerin}|pwk}: \emph{Maurice Barrès}\pwindex{Maurice Barres@\emph{Maurice Barrès}|pwk}. In: \emph{Moderne
                        Rundschau}\pwindex{Moderne Rundschau@\emph{Moderne Rundschau}|pwk}, Bd. 4, H. 1, 1. 10. 1891,
                  S. 15–18.}}}\label{K_L00026-5}\textsc{, eine Studie}\pend
           
\pstart
           3.) \textsc{eine psychologische Novelle\pwindex{Age of Innocence@\emph{Age of Innocence}|pwv} aus einem 12jährigen Kinderkopf}\pend
           
\pstart
           4.) \textsc{Conway\pwindex{Conway, Hugh 26.12.1847 – 15.05.1885@\textsc{Conway, Hugh} (26.12.1847 – 15.05.1885), \emph{Schriftsteller/Schriftstellerin}|pw}, der Novellist der Telepathie}\pend
           
\pstart
           5.) \textsc{das grosse Buch\pwindex{Englisches Leben@\emph{Englisches Leben}|pwv} von }\label{K_L00026-6v}\edtext{\textsc{1891 in England\oindex{England@\textbf{England}, \emph{A.ADM1}|pw}}}{\lemma{\textnormal{\emph{1891 in England}}}\Cendnote{\textnormal{Loris\pwindex{Hofmannsthal, Hugo von 1874-02-01 – 1929-07-15@\textsc{Hofmannsthal, Hugo von} (1874-02-01 – 1929-07-15), \emph{Schriftsteller/Schriftstellerin}|pwk}: \emph{Englisches Leben}\pwindex{Englisches Leben@\emph{Englisches Leben}|pwk}. In: \emph{Moderne
                        Rundschau}\pwindex{Moderne Rundschau@\emph{Moderne Rundschau}|pwk}, Bd. 4, H. 5, 1. 12. 1891,
                  S. 174–177.}}}\label{K_L00026-6}.\pend
           
\pstart
           \centering{}\textsc{Telle est la vie!}\pend
           \pstart \spacefill\mbox{Loris.}\pend{}\selectlanguage{ngerman}\endnumbering\briefempfaengerindex{Schnitzler, Arthur@\textsc{Schnitzler, Arthur}!zzzHofmannsthal, Hugo von@\emph{von Hugo von Hofmannsthal}!1891-08-011@{{[}Anfang August{]} 1891}|)be}\mylabel{L00026h}  \normalsize

\doendnotes{C}
\bigskip
\vfill

\clearpage

\footnotesize

\lohead{\textsc{register}}

% Definiere theindex-Environment komplett neu ohne reledmac
\makeatletter
\renewenvironment{theindex}{%
  \section*{\indexname}%
  \setlength{\parindent}{0pt}%
  \setlength{\parskip}{0pt plus 0.3pt}%
  \let\item\@idxitem
}{%
  \clearpage
}
\makeatother

\IfFileExists{\jobname-pw.ind}{\input{\jobname-pw.ind}}{}

\end{document}

      