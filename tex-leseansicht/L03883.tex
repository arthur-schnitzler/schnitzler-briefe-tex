%% latex-leseansicht-vorspann.tex
%% Vorspann für die Leseansicht.
%% Lädt die gemeinsame Datei latex-vorspann.tex mit nicht gesetztem Schalter.

\newif\ifkorrekturansicht
\korrekturansichtfalse

\input{../tex-inputs/latex-vorspann}


\section[Romain Rolland an Arthur Schnitzler, 6. 2. 1916]{L03883 Romain Rolland an Arthur Schnitzler, 6. 2. 1916}
\nopagebreak\mylabel{L03883v}
\rehead{ }\normalsize\beginnumbering\briefempfaengerindex{Schnitzler, Arthur@\textsc{Schnitzler, Arthur}!zzzRolland, Romain@\emph{von Romain Rolland}!1916-02-061@{6. 2. 1916}|(be}
\toendnotes[C]{\smallbreak\pagebreak[2]}
\correspDesc{Versand  durch Romain Rolland am 6. 2. 1916 in Genf
\newline{}Erhalt  durch Arthur Schnitzler im Zeitraum [7. 2. 1916
                  – 11. 2. 1916?] in Wien}\toendnotes[C]{\smallbreak}
\Standort{CUL, Schnitzler, B 86.}
\physDesc{Brief, 1 Blatt, 2 Seiten, 627 Zeichen
\newline{}Handschrift: schwarze Tinte, lateinische Kurrent}\toendnotes[C]{\smallbreak}
\pstart
           \raggedleft{}{\pb}\begin{otherlanguage}{french}\label{K_L03883-1v}\edtext{Dimanche 6 fév. 1916}{\lemma{\textnormal{\emph{Dimanche 6 fév. 1916}}}\Cendnote{\textnormal{französisch: »Sonntag, 6. Februar 1916{ / }Sehr geehrter Herr,{ / }ich danke Ihnen herzlich für die Freundlichkeit, mir dieses Telegramm
                              zu senden (das mich in Genf\oindex{Genf@\textbf{Genf}|pw} erst
                              heute, am 6. Februar, erreicht hat). Es rührt mich sehr, dass man
                              trotz so vieler Sorgen an meinen armen Geburtstag denken konnte; und
                              ich empfinde eine besondere Dankbarkeit gegenüber der kleinen Gruppe
                              von Wiener Freunden, die mir ihre Sympathie bewahren.{ / }Man spürt, dass es in der Welt so viele zurückgedrängte Freundschaften
                              gibt, die morgen zutage treten werden! Und es ist unsere Aufgabe, wir,
                              die wir die \strikeout{vorangehen} Vorboten unserer Völker sind, schon heute das Morgen zu
                              leben.{ / }Empfangen Sie, sehr geehrter Herr, den Ausdruck meiner ganzen
                              Ergebenheit{ / }Romain Rolland« }}}\label{K_L03883-1}\end{otherlanguage}\pend
           
\pstart{}\begin{otherlanguage}{french}Cher Monsieur\end{otherlanguage}\pend\vspace{0.5em}
\pstart
           \begin{otherlanguage}{french}Je vous remercie cordialement de la bonté que vous avez eue de
                  m’envoyer cette \label{K_L03883-2v}\edtext{dépêche}{\lemma{\textnormal{\emph{dépêche}}}\Cendnote{\textnormal{XXXX Auszeichnungsfehler: Dokument L04214 nicht gefunden.}}}\label{K_L03883-2} (qui m’est arrivée à Genève\oindex{Genf@\textbf{Genf}|pw} qu’aujourd’hui, 6 février). Je suis bien touché que
                  l’on puisse penser à mon pauvre anniversaire, au milieu de tant de soucis; et j’ai
                  une particulière gratitude au petit groupe d’amis viennois\oindex{Wien@\textbf{Wien}, \emph{Verwaltungsgebiet}|pw}, qui me gardent leur sympathie.\end{otherlanguage}\pend
           
\pstart
           \begin{otherlanguage}{french}On sent qu’il y a dans {\pb}le monde tant d’amitiés comprimées, qui
                  se feront jour, demain ! Et c’est notre rôle à nous, qui \strikeout{devançons} sommes les fourriers de nos peuples, de vivre déjà
                  demain.\end{otherlanguage}\pend
           
\pstart
           \begin{otherlanguage}{french}Veuillez croire, cher Monsieur, à mes sentiments tout
                  dévoués\end{otherlanguage}{\\[\baselineskip]}\spacefill\mbox{Romain Rolland}\pend
           \leftskip=0em{}\selectlanguage{ngerman}\endnumbering\briefempfaengerindex{Schnitzler, Arthur@\textsc{Schnitzler, Arthur}!zzzRolland, Romain@\emph{von Romain Rolland}!1916-02-061@{6. 2. 1916}|)be}\mylabel{L03883h}
\begin{anhang}
\end{anhang}\newcommand{\dateiname}{L03883}\newcommand{\titel}{Romain Rolland an Arthur Schnitzler, 6. 2. 1916}\newcommand{\editorInnen}{Selma Jahnke und Martin Anton Müller}%% latex-leseansicht-abspann.tex
%% Abspann für die Leseansicht.
%% Der Schalter \ifkorrekturansicht ist bereits durch den Vorspann gesetzt.

%% latex-abspann.tex
%% Gemeinsamer Abspann für Korrekturansicht und Leseansicht.
%% Setzt den Schalter \ifkorrekturansicht voraus (gesetzt in den
%% einbindenden Dateien latex-korrekturansicht-abspann.tex bzw.
%% latex-leseansicht-abspann.tex).
%% ---------------------------------------------------------------

\normalsize

% Das esempio-Environment wird nur in der Leseansicht benötigt
\ifkorrekturansicht\else
\newenvironment{esempio}[3]%
{
    \vspace{1.5ex}
    \rlap{\underline{#1}}
    \par
    \setlength{\parindent}{0cm}
    \nopagebreak
    \leftskip=#2cm
    \rightskip=#3cm
}
{
    \par
}
\fi

\doendnotes{C}
\bigskip
\vfill

\clearpage

\footnotesize

\ifkorrekturansicht
  \lohead{\textsc{register}}
\fi

% theindex-Environment neu definieren ohne reledmac
\makeatletter
\renewenvironment{theindex}{%
  \ifkorrekturansicht
    \section*{\indexname}%
  \else
    \subsubsection*{Index der erwähnten Entitäten}%
  \fi
  \setlength{\parindent}{0pt}%
  \setlength{\parskip}{0pt plus 0.3pt}%
  \let\item\@idxitem
}{%
  \ifkorrekturansicht\clearpage\fi
}
\makeatother

\IfFileExists{\jobname-pw.ind}{\input{\jobname-pw.ind}}{}

% Quellenangabe nur in der Leseansicht
\ifkorrekturansicht\else
% Fallback-Definitionen, falls die .tex-Datei \titel etc. nicht gesetzt hat
\providecommand{\titel}{}
\providecommand{\editorInnen}{}
\providecommand{\dateiname}{\jobname}

\vspace{3cm}

\vfill

\footnotesize
\textsc{Quelle}: \titel. Herausgegeben von {\editorInnen}. In: \emph{Arthur Schnitzler: Briefwechsel mit Autorinnen und Autoren}.
 Digitale Edition, https://schnitzler-briefe.acdh.oeaw.ac.at/{\dateiname}.html (Stand \today)
\fi

\end{document}


