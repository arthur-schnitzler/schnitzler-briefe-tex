%% latex-leseansicht-vorspann.tex
%% Vorspann für die Leseansicht.
%% Lädt die gemeinsame Datei latex-vorspann.tex mit nicht gesetztem Schalter.

\newif\ifkorrekturansicht
\korrekturansichtfalse

\input{../tex-inputs/latex-vorspann}


\section[ Paul Goldmann an Arthur Schnitzler, 12. 11. {[}1899{]}]{L02893 Paul Goldmann an Arthur Schnitzler,  12. 11. [1899]}
\nopagebreak\mylabel{L02893v}
\rehead{ }\normalsize\beginnumbering\briefempfaengerindex{Schnitzler, Arthur@\textsc{Schnitzler, Arthur}!zzzGoldmann, Paul@\emph{von Paul Goldmann}!1899-11-121@{12. 11. [1899]}|(be}
\toendnotes[C]{\smallbreak\pagebreak[2]}
\correspDesc{Versand  durch Paul Goldmann am 12. 11. [1899] in Frankfurt am Main
\newline{}Erhalt  durch Arthur Schnitzler im Zeitraum [13. 11. 1899 – 17. 11. 1899?] in Wien}\toendnotes[C]{\smallbreak}
\Standort{DLA, A:Schnitzler, HS.NZ85.1.3169.}
\physDesc{Brief, 2 Blätter, 7 Seiten, 4049 Zeichen
\newline{}Handschrift: blaue Tinte, deutsche Kurrent
\newline{}Schnitzler: 1) mit Bleistift das Jahr »99.« vermerkt  2) mit rotem Buntstift fünf Unterstreichungen}\toendnotes[C]{\smallbreak}
\pstart
           \centering{}{\pb}Frankfurt\oindex{Frankfurt am Main@\textbf{Frankfurt am Main}, \emph{Hauptstadt}|pw}, 12. November.\pend
           
\pstart{}Mein lieber Freund,\pend\vspace{0.5em}
\pstart
           Seit zwei Wochen muß ich meinen Onkel\pwindex{Mamroth, Fedor 21.\,2.\,1851 Breslau – 25.\,6.\,1907 Frankfurt am Main@\textsc{Mamroth, Fedor} (21.\,2.\,1851 Breslau – 25.\,6.\,1907 Frankfurt am Main), \emph{Journalist, Kritiker}|pwv} vertreten u. habe allein das Feuilleton\pwindex{Frankfurter Zeitung@\emph{Frankfurter Zeitung}|pwv} zu redigiren, was \strikeout{b} bei unſerem Blatte\pwindex{Frankfurter Zeitung@\emph{Frankfurter Zeitung}|pwv}
               eine ungeheure Arbeit iſt, welche den ganzen Tag und einen Theil der Nacht ausfüllt.
               Keine freie Viertelſtunde alſo. Seitdem ich Deinen letzten lieben Brief erhielt, will
               ich Dir{ }ſchreiben und leide{ }ſehr darunter, daß ich es nicht kann. Heut gibt endlich der Sonntag die Möglichkeit zur
               Ausführung des lang gehegten Vorſatzes.\pend
           
\pstart
           Auf Deinen letzten Brief hätte ich Mancherlei zu fragen; aber ich fürchte, ich komme{ }ſchon zu{ }ſpät. In der \label{K_L02893-1v}\edtext{Affaire \textsc{Schlenther\pwindex{Schlenther, Paul 20.\,8.\,1854 Chernyakhovsk – 30.\,4.\,1916 Berlin@\textsc{Schlenther, Paul} (20.\,8.\,1854 Chernyakhovsk – 30.\,4.\,1916 Berlin), \emph{Schriftsteller, Kritiker, Theaterleiter}|pw}}}{\lemma{\textnormal{\emph{Affaire Schlenther}}}\Cendnote{\textnormal{\emph{Der grüne Kakadu}\pwindex{Schnitzler, Arthur 15.\,5.\,1862 Wien – 21.\,10.\,1931 ebd.@\textsc{Schnitzler, Arthur} (15.\,5.\,1862 Wien – 21.\,10.\,1931 ebd.), \emph{Schriftsteller, Mediziner}!grüne Kakadu. Groteske in einem Akt@\strich\emph{Der grüne Kakadu. Groteske in einem Akt}|pwk} wurde nach nur sechs
                  Aufführungen vom Spielplan\orgindex{Burgtheater@Burgtheater|pwkv}
                  genommen. Am 26. 10. 1899 war Direktor Paul
                     Schlenther\pwindex{Schlenther, Paul 20.\,8.\,1854 Chernyakhovsk – 30.\,4.\,1916 Berlin@\textsc{Schlenther, Paul} (20.\,8.\,1854 Chernyakhovsk – 30.\,4.\,1916 Berlin), \emph{Schriftsteller, Kritiker, Theaterleiter}|pwk} bei Schnitzler zu Hause
                  und teilte ihm mit, dass die \emph{Zensurbehörde}\orgindex{K. u. k. Zensurstelle@K. u. k. Zensurstelle|pwk} die
                  weitere Aufführung verbiete, ohne das aber mit einem schriftlichen Urteil zu
                  bestätigen, worüber sich Schnitzler
                  zusätzlich ärgerte. Erst Jahre später, am 4. 12. 1905, erfuhr Schnitzler den eigentlichen Grund: »Erzh. Gisela\pwindex{Gisela von Österreich 12.\,7.\,1856 Laxenburg – 27.\,7.\,1932 München@\textsc{Gisela von Österreich} (12.\,7.\,1856 Laxenburg – 27.\,7.\,1932 München), \emph{Erzherzogin}|pw} war drin und indignirt, weil Haeberle\pwindex{Speidel-Haeberle, Else 11.\,7.\,1877 Stuttgart – 21.\,7.\,1937 Augustenfeld@\textsc{Speidel-Haeberle, Else} (11.\,7.\,1877 Stuttgart – 21.\,7.\,1937 Augustenfeld), \emph{Schauspielerin}|pw} (Michette) sich an den Dessous
                     der Marquise (Mitterwurzer\pwindex{Mitterwurzer, Wilhelmine 27.\,3.\,1848 Freiburg im Breisgau – 3.\,8.\,1909 Wien@\textsc{Mitterwurzer, Wilhelmine} (27.\,3.\,1848 Freiburg im Breisgau – 3.\,8.\,1909 Wien), \emph{Schauspielerin}|pw}) zu schaffen
                     machte. –«}}}\label{K_L02893-1} nämlich möchte ich immer wieder zur Mäßigung rathen.
               Es{ }ſteht etwas{ }ſehr Wichtiges auf dem Spiele: Dein \label{K_L02893-2v}\edtext{neues Stück\pwindex{Schnitzler, Arthur 15.\,5.\,1862 Wien – 21.\,10.\,1931 ebd.@\textsc{Schnitzler, Arthur} (15.\,5.\,1862 Wien – 21.\,10.\,1931 ebd.), \emph{Schriftsteller, Mediziner}!Schleier der Beatrice. Schauspiel in fünf Akten@\strich\emph{Der Schleier der Beatrice. Schauspiel in fünf Akten}|pwv}}{\lemma{\textnormal{\emph{neues Stück}}}\Cendnote{\textnormal{Goldmanns\pwindex{Goldmann, Paul 31.\,1.\,1865 Breslau – 25.\,9.\,1935 Wien@\textsc{Goldmann, Paul} (31.\,1.\,1865 Breslau – 25.\,9.\,1935 Wien), \emph{Schriftsteller, Journalist}|pwk} Hinweis darauf, dass Schnitzler, wenn er zu lautstark protestiere,
                  die Aufführung von \emph{Der Schleier der Beatrice}\pwindex{Schnitzler, Arthur 15.\,5.\,1862 Wien – 21.\,10.\,1931 ebd.@\textsc{Schnitzler, Arthur} (15.\,5.\,1862 Wien – 21.\,10.\,1931 ebd.), \emph{Schriftsteller, Mediziner}!Schleier der Beatrice. Schauspiel in fünf Akten@\strich\emph{Der Schleier der Beatrice. Schauspiel in fünf Akten}|pwk}
                  in Gefahr bringe, hatte etwas Prophetisches. Das Stück wurde von Schlenther\pwindex{Schlenther, Paul 20.\,8.\,1854 Chernyakhovsk – 30.\,4.\,1916 Berlin@\textsc{Schlenther, Paul} (20.\,8.\,1854 Chernyakhovsk – 30.\,4.\,1916 Berlin), \emph{Schriftsteller, Kritiker, Theaterleiter}|pwk} zwar anfänglich für das \emph{Burgtheater}\orgindex{Burgtheater@Burgtheater|pwk} akzeptiert, die Zusage aber (neuerlich ohne
                  Transparenz) nach ein paar Monaten zurückgezogen (siehe XXXX Auszeichnungsfehler: Dokument L01073 nicht gefunden), was zu einem Skandal führte (siehe Hermann Bahr, Arthur Schnitzler: \emph{Briefwechsel, Aufzeichnungen, Dokumente (1891–1931)}, Hermann Bahr, Julius Bauer, J. J. David, Robert Hirschfeld, Felix Salten, Ludwig Speidel: Erklärung, 14. 9. 1900). \emph{Der Schleier der Beatrice}\pwindex{Schnitzler, Arthur 15.\,5.\,1862 Wien – 21.\,10.\,1931 ebd.@\textsc{Schnitzler, Arthur} (15.\,5.\,1862 Wien – 21.\,10.\,1931 ebd.), \emph{Schriftsteller, Mediziner}!Schleier der Beatrice. Schauspiel in fünf Akten@\strich\emph{Der Schleier der Beatrice. Schauspiel in fünf Akten}|pwk} wurde schließlich am 1. 12. 1900 im Lobe-Theater\oindex{Lobe-Theater@\textbf{Lobe-Theater}, \emph{Theater}|pwk} in Breslau\oindex{Breslau@\textbf{Breslau}|pwk} uraufgeführt.}}}\label{K_L02893-2}. Was liegt demgegenüber an den drei Einaktern\pwindex{Schnitzler, Arthur 15.\,5.\,1862 Wien – 21.\,10.\,1931 ebd.@\textsc{Schnitzler, Arthur} (15.\,5.\,1862 Wien – 21.\,10.\,1931 ebd.), \emph{Schriftsteller, Mediziner}!grüne Kakadu – Paracelsus – Die Gefährtin. Drei Einakter@\strich\emph{Der grüne Kakadu – Paracelsus – Die Gefährtin. Drei Einakter}|pwv}, die überdies \label{K_L02893-3v}\edtext{überall in Deutſchland\oindex{Deutschland@\textbf{Deutschland}|pw}}{\lemma{\textnormal{\emph{überall in Deutschland}}}\Cendnote{\textnormal{Hervorzuheben ist der Erfolg am Deutschen Theater Berlin\oindex{Deutsches Theater Berlin@\textbf{Deutsches Theater Berlin}, \emph{Theater}|pwk}. Die Einakter\pwindex{Schnitzler, Arthur 15.\,5.\,1862 Wien – 21.\,10.\,1931 ebd.@\textsc{Schnitzler, Arthur} (15.\,5.\,1862 Wien – 21.\,10.\,1931 ebd.), \emph{Schriftsteller, Mediziner}!grüne Kakadu – Paracelsus – Die Gefährtin. Drei Einakter@\strich\emph{Der grüne Kakadu – Paracelsus – Die Gefährtin. Drei Einakter}|pwkv} wurden dort fast dreißigmal
                  aufgeführt und waren damit Schnitzlers bislang größter Erfolg.}}}\label{K_L02893-3}{ }{\pb}mit Erfolg gegeben werden,{ }ſo daß Du{ }ſchließlich
               auf die weitere Aufführung in Wien\oindex{Wien@\textbf{Wien}, \emph{Verwaltungsgebiet}|pw} verzichten
               kannſt. Alle Lebenskunſt kommt oft darauf hinaus\textcolor{gray}{,} kleine
               Conceſſionen zu machen, um große Ziele zu erreichen. Das große Ziel iſt, daß das Burgtheater\orgindex{Burgtheater@Burgtheater|pw} Dein neues Stück\pwindex{Schnitzler, Arthur 15.\,5.\,1862 Wien – 21.\,10.\,1931 ebd.@\textsc{Schnitzler, Arthur} (15.\,5.\,1862 Wien – 21.\,10.\,1931 ebd.), \emph{Schriftsteller, Mediziner}!Schleier der Beatrice. Schauspiel in fünf Akten@\strich\emph{Der Schleier der Beatrice. Schauspiel in fünf Akten}|pwv}{ }ſpielt. Ich finde, daß dir \textsc{Schlenther\pwindex{Schlenther, Paul 20.\,8.\,1854 Chernyakhovsk – 30.\,4.\,1916 Berlin@\textsc{Schlenther, Paul} (20.\,8.\,1854 Chernyakhovsk – 30.\,4.\,1916 Berlin), \emph{Schriftsteller, Kritiker, Theaterleiter}|pw}} durch{ }ſeinen \label{K_L02893-4v}\edtext{Beſuch}{\lemma{\textnormal{\emph{Besuch}}}\Cendnote{\textnormal{Siehe A. S.: \emph{Tagebuch}, 26. 10. 1899.
               }}}\label{K_L02893-4} bei Dir bereits alle mögliche Satisfaktion gegeben hat, und ich meine, Du{ }ſollteſt darauf verzichten, ihn weiter zu demüthigen. Alles Sturmlaufen \strikeout{\textcolor{gray}{nu}} nützt übrigens nichts. Du wirſt dadurch nicht einen feigen und verlogenen
               Menſchen zum Muth und zur Wahrheit \strikeout{\textcolor{gray}{br}i\textcolor{gray}{n}} zwingen, und Öſterreich\oindex{Österreich@\textbf{Österreich}|pw} wirſt Du auch
               nicht ändern. Ich hätte dem \textsc{Schlenther\pwindex{Schlenther, Paul 20.\,8.\,1854 Chernyakhovsk – 30.\,4.\,1916 Berlin@\textsc{Schlenther, Paul} (20.\,8.\,1854 Chernyakhovsk – 30.\,4.\,1916 Berlin), \emph{Schriftsteller, Kritiker, Theaterleiter}|pw}} an Deiner Stelle geradezu geſagt: »Gut, laſſen wir’s gehen, aber{ }ſpielen Sie
               mein neues Stück\pwindex{Schnitzler, Arthur 15.\,5.\,1862 Wien – 21.\,10.\,1931 ebd.@\textsc{Schnitzler, Arthur} (15.\,5.\,1862 Wien – 21.\,10.\,1931 ebd.), \emph{Schriftsteller, Mediziner}!Schleier der Beatrice. Schauspiel in fünf Akten@\strich\emph{Der Schleier der Beatrice. Schauspiel in fünf Akten}|pwv}!« Und {\pb}wenn es nicht{ }ſchon zu{ }ſpät iſt, möchte ich Dir
               rathen, die Verhandlungen noch in dieſem Sinne zu führen. Kommt es aber zum offenen
               Conflict,{ }ſo brauche ich Dir nicht erſt zu{ }ſagen, daß Du unbedingt auf mich rechnen
               kannſt,{ }ſolange ich das Feuilleton\pwindex{Frankfurter Zeitung@\emph{Frankfurter Zeitung}|pwv} redigire. Wenn freilich mein Onkel\pwindex{Mamroth, Fedor 21.\,2.\,1851 Breslau – 25.\,6.\,1907 Frankfurt am Main@\textsc{Mamroth, Fedor} (21.\,2.\,1851 Breslau – 25.\,6.\,1907 Frankfurt am Main), \emph{Journalist, Kritiker}|pwv} wieder zurück iſt,{ }ſo wird wieder der \label{K_L02893-5v}\edtext{Einfluß{ }ſeiner Frau\pwindex{Mamroth, Johanna 19.\,5.\,1872 Frankfurt am Main – 12.\,9.\,1910@\textsc{Mamroth, Johanna} (19.\,5.\,1872 Frankfurt am Main – 12.\,9.\,1910)|pwv}}{\lemma{\textnormal{\emph{Einfluß seiner Frau}}}\Cendnote{\textnormal{Siehe zum Einfluss Johanna Mamroths\pwindex{Mamroth, Johanna 19.\,5.\,1872 Frankfurt am Main – 12.\,9.\,1910@\textsc{Mamroth, Johanna} (19.\,5.\,1872 Frankfurt am Main – 12.\,9.\,1910)|pwk} auf Fedor Mamroths\pwindex{Mamroth, Fedor 21.\,2.\,1851 Breslau – 25.\,6.\,1907 Frankfurt am Main@\textsc{Mamroth, Fedor} (21.\,2.\,1851 Breslau – 25.\,6.\,1907 Frankfurt am Main), \emph{Journalist, Kritiker}|pwk} feuilletonistische Arbeit auch XXXX Auszeichnungsfehler: Dokument L02792 nicht gefunden.}}}\label{K_L02893-5} auf das Feuilleton der Frankfurter Zeitung\pwindex{Frankfurter Zeitung@\emph{Frankfurter Zeitung}|pw} beginnen, und dann bin ich
               machtlos, und Du kannſt auf nichts mehr rechnen.\pend
           
\pstart
           An \textsc{Wassermann\pwindex{Wassermann, Jakob 10.\,3.\,1873 Fürth – 1.\,1.\,1934 Altaussee@\textsc{Wassermann, Jakob} (10.\,3.\,1873 Fürth – 1.\,1.\,1934 Altaussee), \emph{Schriftsteller}|pw}} habe ich – Dir zuliebe – einen \label{K_L02893-6v}\edtext{mahnenden Brief}{\lemma{\textnormal{\emph{mahnenden Brief}}}\Cendnote{\textnormal{Schnitzler dürfte versucht haben, Jakob Wassermann\pwindex{Wassermann, Jakob 10.\,3.\,1873 Fürth – 1.\,1.\,1934 Altaussee@\textsc{Wassermann, Jakob} (10.\,3.\,1873 Fürth – 1.\,1.\,1934 Altaussee), \emph{Schriftsteller}|pwk} hinsichtlich seiner
                  nicht zufriedenstellenden Arbeit für die \emph{Frankfurter
                     Zeitung}\orgindex{Frankfurter Zeitung@Frankfurter Zeitung|pwk} vor Goldmann\pwindex{Goldmann, Paul 31.\,1.\,1865 Breslau – 25.\,9.\,1935 Wien@\textsc{Goldmann, Paul} (31.\,1.\,1865 Breslau – 25.\,9.\,1935 Wien), \emph{Schriftsteller, Journalist}|pwk} zu
                  verteidigen. Siehe XXXX Auszeichnungsfehler: Dokument L02892 nicht gefunden, XXXX Auszeichnungsfehler: Dokument L02897 nicht gefunden, XXXX Auszeichnungsfehler: Dokument L02898 nicht gefunden und XXXX Auszeichnungsfehler: Dokument L02900 nicht gefunden.}}}\label{K_L02893-6}{ }ſchreiben laſſen. Wenn er dennoch eines Tages fällt,{ }ſo werde ich \textsc{Schwarzkopf\pwindex{Schwarzkopf, Gustav 7.\,11.\,1853 Wien – 13.\,11.\,1939 ebd.@\textsc{Schwarzkopf, Gustav} (7.\,11.\,1853 Wien – 13.\,11.\,1939 ebd.), \emph{Schriftsteller}|pw}} und \textsc{Hirschfeld\pwindex{Hirschfeld, Robert 17.\,9.\,1857 Žďár nad Sázavou – 2.\,4.\,1914 Salzburg@\textsc{Hirschfeld, Robert} (17.\,9.\,1857 Žďár nad Sázavou – 2.\,4.\,1914 Salzburg), \emph{Journalist, Musikkritiker}|pw}}{ }\strikeout{\textcolor{gray}{×}} als{ }ſeine Nachfolger\orgindex{Frankfurter Zeitung@Frankfurter Zeitung|pwv}
               empfehlen.\pend
           
\pstart
           {\pb}Ich hätte – trotz meines Nichtſchreibens – gehofft,
               in dieſen Wochen wieder etwas von Dir zu hören. Wenn Du auf meine Antwort gewartet
               haſt,{ }ſo laß’ mich jetzt nicht länger ohne Nachricht und{ }ſchreibe mir, wie Du lebſt
               und was Du \label{K_L02893-7v}\edtext{arbeiteſt}{\lemma{\textnormal{\emph{arbeitest}}}\Cendnote{\textnormal{Am 12. 11. 1899 begann Schnitzler die humoristisch angelegte Erzählung \emph{Der Leuchtkäfer}\pwindex{Schnitzler, Arthur 15.\,5.\,1862 Wien – 21.\,10.\,1931 ebd.@\textsc{Schnitzler, Arthur} (15.\,5.\,1862 Wien – 21.\,10.\,1931 ebd.), \emph{Schriftsteller, Mediziner}!Leuchtkäfer@\strich\emph{Der Leuchtkäfer}|pwk}. Noch zehn Jahre später, am
                     3. 9. 1909,
                  vermerkte er eine Überarbeitung der
                  posthum veröffentlichten Erzählung\pwindex{Schnitzler, Arthur 15.\,5.\,1862 Wien – 21.\,10.\,1931 ebd.@\textsc{Schnitzler, Arthur} (15.\,5.\,1862 Wien – 21.\,10.\,1931 ebd.), \emph{Schriftsteller, Mediziner}!Leuchtkäfer@\strich\emph{Der Leuchtkäfer}|pwkv} in seinem \emph{Tagebuch}\pwindex{Schnitzler, Arthur 15.\,5.\,1862 Wien – 21.\,10.\,1931 ebd.@\textsc{Schnitzler, Arthur} (15.\,5.\,1862 Wien – 21.\,10.\,1931 ebd.), \emph{Schriftsteller, Mediziner}!Tagebuch@\strich\emph{Tagebuch}|pwk}.}}}\label{K_L02893-7}.\pend
           
\pstart
           In meinem Leben bereiten{ }ſich große Stürme und vielleicht{ }ſehr{ }ſchwerwiegende
               Ereigniſſe vor. Mein Verhältniß zu ihr\pwindex{Rottenberg, Theodore 7.\,9.\,1875 – 5.\,4.\,1945 Limburg an der Lahn@\textsc{Rottenberg, Theodore} (7.\,9.\,1875 – 5.\,4.\,1945 Limburg an der Lahn)|pwv} iſt glücklich, dank der Befliſſenheit einiger intimer
                  \label{K_L02893-8v}\edtext{Freundinnen}{\lemma{\textnormal{\emph{Freundinnen}}}\Cendnote{\textnormal{nicht ermittelt}}}\label{K_L02893-8} und auch infolge ihrer eigenen
               Unvorſichtigkeit, zum \label{K_L02893-9v}\edtext{öffentlichen
                  Gerücht}{\lemma{\textnormal{\emph{öffentlichen
                  Gerücht}}}\Cendnote{\textnormal{Siehe XXXX Auszeichnungsfehler: Dokument L02890 nicht gefunden.
               }}}\label{K_L02893-9} geworden. Die ganze Stadt\oindex{Frankfurt am Main@\textbf{Frankfurt am Main}, \emph{Hauptstadt}|pwv}{ }ſpricht zur Zeit davon. Es heißt,{ }ſie werde{ }ſich von ihrem Manne\pwindex{Rottenberg, Ludwig 11.\,10.\,1864 Czernowitz – 6.\,5.\,1932 Frankfurt am Main@\textsc{Rottenberg, Ludwig} (11.\,10.\,1864 Czernowitz – 6.\,5.\,1932 Frankfurt am Main), \emph{Kapellmeister}|pwv}{ }\label{K_L02893-10v}\edtext{ſcheiden laſſen und mich
                  heirathen}{\lemma{\textnormal{\emph{scheiden … heirathen}}}\Cendnote{\textnormal{Dazu kam es nicht.}}}\label{K_L02893-10}. Der
               Klatſch iſt{ }ſo arg geworden, daß mein Chefredakteur\pwindex{Sonnemann, Leopold 29.\,10.\,1831 Höchberg – 30.\,10.\,1909 Frankfurt am Main@\textsc{Sonnemann, Leopold} (29.\,10.\,1831 Höchberg – 30.\,10.\,1909 Frankfurt am Main), \emph{Journalist, Herausgeber}|pwv} bei mir hat anfragen laſſen, ob er begründet{ }ſei. Ein hieſiges {\pb}Klatſchblatt, die »Sonne\pwindex{Sonne@\emph{Die Sonne}|pw}«, hat bereits einen \label{K_L02893-11v}\edtext{\uline{Artikel\pwindex{?? [Zur Affaire von Paul Goldmann und Theodore Rottenberg]@\emph{?? [Zur Affaire von Paul Goldmann und Theodore Rottenberg]}|pwv}}}{\lemma{\textnormal{\emph{Artikel}}}\Cendnote{\textnormal{Es konnte kein Exemplar der \emph{Zeitschrift}\pwindex{Sonne@\emph{Die Sonne}|pwk} aus dem betreffenden Zeitraum
                  nachgewiesen werden.}}}\label{K_L02893-11}{ }\strikeout{darü}{ }\strikeout{da} darüber gebracht. Der \label{K_L02893-12v}\edtext{Gemahl\pwindex{Rottenberg, Ludwig 11.\,10.\,1864 Czernowitz – 6.\,5.\,1932 Frankfurt am Main@\textsc{Rottenberg, Ludwig} (11.\,10.\,1864 Czernowitz – 6.\,5.\,1932 Frankfurt am Main), \emph{Kapellmeister}|pwv} in Wien\oindex{Wien@\textbf{Wien}, \emph{Verwaltungsgebiet}|pw}}{\lemma{\textnormal{\emph{Gemahl in Wien}}}\Cendnote{\textnormal{Siehe XXXX Auszeichnungsfehler: Dokument L02889 nicht gefunden.
               }}}\label{K_L02893-12} weiß noch nichts. Aber er{ }ſoll in einigen Tagen zurückkommen, und dann wird
               die Geſchichte wohl losgehen. Es kommt dazu, daß ſie\pwindex{Rottenberg, Theodore 7.\,9.\,1875 – 5.\,4.\,1945 Limburg an der Lahn@\textsc{Rottenberg, Theodore} (7.\,9.\,1875 – 5.\,4.\,1945 Limburg an der Lahn)|pwv}, von einem plötzlichen Wahrheitsdrang befallen, erklärt,{ }ſie werde ihrem Manne\pwindex{Rottenberg, Ludwig 11.\,10.\,1864 Czernowitz – 6.\,5.\,1932 Frankfurt am Main@\textsc{Rottenberg, Ludwig} (11.\,10.\,1864 Czernowitz – 6.\,5.\,1932 Frankfurt am Main), \emph{Kapellmeister}|pwv}
               gegenüber nicht Alles ableugnen können. Mit banger Sorge{ }ſehe ich der Kataſtrophe
               entgegen, die kaum mehr aufzuhalten iſt. Wenn ihr Mann\pwindex{Rottenberg, Ludwig 11.\,10.\,1864 Czernowitz – 6.\,5.\,1932 Frankfurt am Main@\textsc{Rottenberg, Ludwig} (11.\,10.\,1864 Czernowitz – 6.\,5.\,1932 Frankfurt am Main), \emph{Kapellmeister}|pwv}{ }ſie verſtößt, muß ich natürlich{ }ſie aufnehmen. Und was{ }ſoll ich in meinen Verhältniſſen, wo ich meine Mutter\pwindex{Goldmann, Clementine 15.\,5.\,1842 Breslau – 24.\,2.\,1924 Frankfurt am Main@\textsc{Goldmann, Clementine} (15.\,5.\,1842 Breslau – 24.\,2.\,1924 Frankfurt am Main)|pwv} und mich gerade durchbringe, plötzlich mit einer Frau
               anfangen?\pend
           
\pstart
           {\pb}Unter dieſen Umſtänden iſt mir dieſe kleine Stadt\oindex{Frankfurt am Main@\textbf{Frankfurt am Main}, \emph{Hauptstadt}|pwv} mit ihrer
                  giftigen\textcolor{gray}{,} ganz ohne Noth bösartigen und gemeinen Klatſchſucht
               erſt recht zum Ekel geworden, und ich beklage bitter, daß{ }ſich mein \label{K_L02893-13v}\edtext{Engagement nach Berlin\oindex{Berlin@\textbf{Berlin}, \emph{Hauptstadt}|pw} für die Neue Freie
                  Preſſe\orgindex{Neue Freie Presse@Neue Freie Presse|pw} zerſchlagen}{\lemma{\textnormal{\emph{Engagement … zerschlagen}}}\Cendnote{\textnormal{Siehe XXXX Auszeichnungsfehler: Dokument L02892 nicht gefunden und XXXX Auszeichnungsfehler: Dokument L02896 nicht gefunden.
               }}}\label{K_L02893-13} hat. Hörſt Du irgend etwas, wie es mit \textsc{Frischauer\pwindex{Frischauer, Berthold 9.\,9.\,1851 Brünn – 4.\,2.\,1924 Wien@\textsc{Frischauer, Berthold} (9.\,9.\,1851 Brünn – 4.\,2.\,1924 Wien), \emph{Journalist}|pw}}{ }ſteht? Und weißt Du vielleicht, wer \label{K_L02893-14v}\edtext{jetzt in Paris\oindex{Paris@\textbf{Paris}, \emph{Hauptstadt}|pw} für die
                  N. Fr. Pr.\orgindex{Neue Freie Presse@Neue Freie Presse|pw}}{\lemma{\textnormal{\emph{jetzt … Pr.}}}\Cendnote{\textnormal{Nicht geklärt, vgl. XXXX Auszeichnungsfehler: Dokument L02892 nicht gefunden.
               }}}\label{K_L02893-14} iſt?\pend
           
\pstart
           Grüße mir \textsc{Richard\pwindex{Beer-Hofmann, Richard 11.\,7.\,1866 Wien – 26.\,9.\,1945 New York City@\textsc{Beer-Hofmann, Richard} (11.\,7.\,1866 Wien – 26.\,9.\,1945 New York City), \emph{Schriftsteller}|pw}}, \textsc{Schwarzkopf\pwindex{Schwarzkopf, Gustav 7.\,11.\,1853 Wien – 13.\,11.\,1939 ebd.@\textsc{Schwarzkopf, Gustav} (7.\,11.\,1853 Wien – 13.\,11.\,1939 ebd.), \emph{Schriftsteller}|pw}}, Deinen Bruder\pwindex{Schnitzler, Julius 13.\,7.\,1865 Wien – 29.\,6.\,1939 ebd.@\textsc{Schnitzler, Julius} (13.\,7.\,1865 Wien – 29.\,6.\,1939 ebd.), \emph{Chirurg}|pwv}, Deinen
                  Schwager\pwindex{Hajek, Markus 25.\,11.\,1861 Vršac – 4.\,4.\,1941 London@\textsc{Hajek, Markus} (25.\,11.\,1861 Vršac – 4.\,4.\,1941 London), \emph{Mediziner, Laryngologe}|pwv} und alle die
               anderen lieben Menſchen; empfiehl’ mich Deiner Frau Mutter\pwindex{Schnitzler, Louise 8.\,7.\,1840 Kőszeg – 9.\,9.\,1911 Wien@\textsc{Schnitzler, Louise} (8.\,7.\,1840 Kőszeg – 9.\,9.\,1911 Wien)|pwv}{ }{\pb}und{ }ſei Du{ }ſelbſt von Herzen gegrüßt –\pend
           
\pstart
           von Deinem treuen {\\[\baselineskip]}\spacefill\mbox{Paul Goldmann.}\pend
           \leftskip=0em{}\selectlanguage{ngerman}\endnumbering\briefempfaengerindex{Schnitzler, Arthur@\textsc{Schnitzler, Arthur}!zzzGoldmann, Paul@\emph{von Paul Goldmann}!1899-11-121@{12. 11. [1899]}|)be}\mylabel{L02893h}  \newcommand{\dateiname}{L02893}\newcommand{\titel}{Paul Goldmann an Arthur Schnitzler, 12. 11. [1899]}\newcommand{\editorInnen}{Martin Anton Müller und Laura Untner}%% latex-leseansicht-abspann.tex
%% Abspann für die Leseansicht.
%% Der Schalter \ifkorrekturansicht ist bereits durch den Vorspann gesetzt.

%% latex-abspann.tex
%% Gemeinsamer Abspann für Korrekturansicht und Leseansicht.
%% Setzt den Schalter \ifkorrekturansicht voraus (gesetzt in den
%% einbindenden Dateien latex-korrekturansicht-abspann.tex bzw.
%% latex-leseansicht-abspann.tex).
%% ---------------------------------------------------------------

\normalsize

% Das esempio-Environment wird nur in der Leseansicht benötigt
\ifkorrekturansicht\else
\newenvironment{esempio}[3]%
{
    \vspace{1.5ex}
    \rlap{\underline{#1}}
    \par
    \setlength{\parindent}{0cm}
    \nopagebreak
    \leftskip=#2cm
    \rightskip=#3cm
}
{
    \par
}
\fi

\doendnotes{C}
\bigskip
\vfill

\clearpage

\footnotesize

\ifkorrekturansicht
  \lohead{\textsc{register}}
\fi

% theindex-Environment neu definieren ohne reledmac
\makeatletter
\renewenvironment{theindex}{%
  \ifkorrekturansicht
    \section*{\indexname}%
  \else
    \subsubsection*{Index der erwähnten Entitäten}%
  \fi
  \setlength{\parindent}{0pt}%
  \setlength{\parskip}{0pt plus 0.3pt}%
  \let\item\@idxitem
}{%
  \ifkorrekturansicht\clearpage\fi
}
\makeatother

\IfFileExists{\jobname-pw.ind}{\input{\jobname-pw.ind}}{}

% Quellenangabe nur in der Leseansicht
\ifkorrekturansicht\else
% Fallback-Definitionen, falls die .tex-Datei \titel etc. nicht gesetzt hat
\providecommand{\titel}{}
\providecommand{\editorInnen}{}
\providecommand{\dateiname}{\jobname}

\vspace{3cm}

\vfill

\footnotesize
\textsc{Quelle}: \titel. Herausgegeben von {\editorInnen}. In: \emph{Arthur Schnitzler: Briefwechsel mit Autorinnen und Autoren}.
 Digitale Edition, https://schnitzler-briefe.acdh.oeaw.ac.at/{\dateiname}.html (Stand \today)
\fi

\end{document}


