%% latex-leseansicht-vorspann.tex
%% Vorspann für die Leseansicht.
%% Lädt die gemeinsame Datei latex-vorspann.tex mit nicht gesetztem Schalter.

\newif\ifkorrekturansicht
\korrekturansichtfalse

\input{../tex-inputs/latex-vorspann}


         
         \renewcommand{\erwaehntePersonen}{Personen: Richard Beer-Hofmann, Berthold Frischauer,  Gisela von Österreich, Clementine Goldmann, Markus Hajek, Robert Hirschfeld, Fedor Mamroth, Johanna Mamroth, Wilhelmine Mitterwurzer, Theodore Rottenberg, Ludwig Rottenberg, Paul Schlenther, Julius Schnitzler, Louise Schnitzler, Gustav Schwarzkopf, Leopold Sonnemann, Else Speidel-Haeberle, Jakob Wassermann}
         \renewcommand{\erwaehnteInstitutionen}{Institutionen: Burgtheater, Frankfurter Zeitung, K. u. k. Zensurstelle, Neue Freie Presse}
         \renewcommand{\erwaehnteOrte}{Orte: Berlin, Breslau, Deutsches Theater Berlin, Deutschland, Frankfurt am Main, Lobe-Theater, Paris, Wien, Österreich}
         \renewcommand{\erwaehnteWerke}{Werke: ?? [Zur Affaire von Paul Goldmann und Theodore Rottenberg], Der Leuchtkäfer, Der Schleier der Beatrice. Schauspiel in fünf Akten, Der grüne Kakadu – Paracelsus – Die Gefährtin. Drei Einakter, Der grüne Kakadu. Groteske in einem Akt, Die Sonne, Frankfurter Zeitung, Tagebuch}
               \section[ Paul Goldmann an Arthur Schnitzler, 12. 11. {[}1899{]}]{ Paul Goldmann an Arthur Schnitzler, 12. 11. {[}1899{]}}\nopagebreak\mylabel{v}\rehead{ }\begin{ledgroupsized}[t]{13cm}\normalsize\beginnumbering \toendnotes[C]{\smallbreak\pagebreak[2]} \Standort{DLA, A:Schnitzler, HS.NZ85.1.3169.}
\physDesc{Brief, 2 Blätter, 7 Seiten, 4049 Zeichen
\newline{}Handschrift: blaue Tinte, deutsche Kurrent
\newline{}Schnitzler: 1) mit Bleistift das Jahr »99.« vermerkt  2) mit rotem Buntstift fünf Unterstreichungen}\toendnotes[C]{\smallbreak}\pstart
           \centering{}{\pb}Frankfurt\oindex{Frankfurt am Main@\textbf{Frankfurt am Main}|pw}, 12. November.\pend
           \pstart{}Mein lieber Freund,\pend\pstart
           Seit zwei Wochen muß ich meinen Onkel\pwindex{Mamroth, Fedor 21.02.1851 – 25.06.1907@\textsc{Mamroth, Fedor} (21.02.1851 – 25.06.1907), \emph{Journalist, Kritiker}|pwv} vertreten u. habe allein das Feuilleton\pwindex{?? Werk@Nicht ermittelte Verfasserinnen und Verfasser!Frankfurter Zeitung1856 – 1943@\emph{Frankfurter Zeitung} {[}1856 – 1943{]}|pwv} zu redigiren, was \strikeout{b} bei unſerem Blatte\pwindex{?? Werk@Nicht ermittelte Verfasserinnen und Verfasser!Frankfurter Zeitung1856 – 1943@\emph{Frankfurter Zeitung} {[}1856 – 1943{]}|pwv}
               eine ungeheure Arbeit iſt, welche den ganzen Tag und einen Theil der Nacht ausfüllt.
               Keine freie Viertelſtunde alſo. Seitdem ich Deinen letzten lieben Brief erhielt, will
               ich Dir ſchreiben und leide ſehr darunter, daß ich es nicht kann. Heut gibt endlich der Sonntag die Möglichkeit zur
               Ausführung des lang gehegten Vorſatzes.\pend
           \pstart
           Auf Deinen letzten Brief hätte ich Mancherlei zu fragen; aber ich fürchte, ich komme
               ſchon zu ſpät. In der \label{K_L02893-1v}\edtext{Affaire \textsc{Schlenther\pwindex{Schlenther, Paul 20.08.1854 – 30.04.1916@\textsc{Schlenther, Paul} (20.08.1854 – 30.04.1916), \emph{Schriftsteller, Kritiker, Theaterleiter}|pw}}}{\lemma{\textnormal{\emph{Affaire Schlenther}}}\Cendnote{\textnormal{\emph{Der grüne Kakadu}\pwindex{Schnitzler, Arthur 15.05.1862 – 21.10.1931@\textsc{Schnitzler, Arthur} (15.05.1862 – 21.10.1931), \emph{Schriftsteller, Mediziner}!gruene Kakadu. Groteske in einem Akt1. 3. 1899@\strich\emph{Der grüne Kakadu. Groteske in einem Akt} {[}1. 3. 1899{]}|pwk} wurde nach nur sechs
                  Aufführungen vom Spielplan\orgindex{Burgtheater@Burgtheater|pwkv}
                  genommen. Am 26. 10. 1899 war Direktor Paul
                     Schlenther\pwindex{Schlenther, Paul 20.08.1854 – 30.04.1916@\textsc{Schlenther, Paul} (20.08.1854 – 30.04.1916), \emph{Schriftsteller, Kritiker, Theaterleiter}|pwk} bei Schnitzler\pwindex{Schnitzler, Arthur 15.05.1862 – 21.10.1931@\textsc{Schnitzler, Arthur} (15.05.1862 – 21.10.1931), \emph{Schriftsteller, Mediziner}|pwk} zu Hause
                  und teilte ihm mit, dass die \emph{Zensurbehörde}\orgindex{K. u. k. Zensurstelle@K. u. k. Zensurstelle|pwk} die
                  weitere Aufführung verbiete, ohne das aber mit einem schriftlichen Urteil zu
                  bestätigen, worüber sich Schnitzler\pwindex{Schnitzler, Arthur 15.05.1862 – 21.10.1931@\textsc{Schnitzler, Arthur} (15.05.1862 – 21.10.1931), \emph{Schriftsteller, Mediziner}|pwk}
                  zusätzlich ärgerte. Erst Jahre später, am 4. 12. 1905, erfuhr Schnitzler\pwindex{Schnitzler, Arthur 15.05.1862 – 21.10.1931@\textsc{Schnitzler, Arthur} (15.05.1862 – 21.10.1931), \emph{Schriftsteller, Mediziner}|pwk} den eigentlichen Grund: »Erzh. Gisela\pwindex{Gisela von Oesterreich 1856-07-12 – 1932-07-27@\textsc{Gisela von Österreich} (1856-07-12 – 1932-07-27), \emph{Erzherzogin}|pw} war drin und indignirt, weil Haeberle\pwindex{Speidel-Haeberle, Else 11.07.1877 – 21.07.1937@\textsc{Speidel-Haeberle, Else} (11.07.1877 – 21.07.1937), \emph{Schauspielerin}|pw} (Michette) sich an den Dessous
                     der Marquise (Mitterwurzer\pwindex{Mitterwurzer, Wilhelmine 27.03.1848 – 03.08.1909@\textsc{Mitterwurzer, Wilhelmine} (27.03.1848 – 03.08.1909), \emph{Schauspielerin}|pw}) zu schaffen
                     machte.–«}}}\label{K_L02893-1h} nämlich möchte ich immer wieder zur Mäßigung rathen.
               Es ſteht etwas ſehr Wichtiges auf dem Spiele: Dein \label{K_L02893-66v}\edtext{neues Stück\pwindex{Schnitzler, Arthur 15.05.1862 – 21.10.1931@\textsc{Schnitzler, Arthur} (15.05.1862 – 21.10.1931), \emph{Schriftsteller, Mediziner}!Schleier der Beatrice. Schauspiel in fuenf Akten1900-12-01@\strich\emph{Der Schleier der Beatrice. Schauspiel in fünf Akten} {[}1900-12-01{]}|pwv}}{\lemma{\textnormal{\emph{neues Stück}}}\Cendnote{\textnormal{Goldmann\pwindex{Goldmann, Paul 31.01.1865 – 25.09.1935@\textsc{Goldmann, Paul} (31.01.1865 – 25.09.1935), \emph{Schriftsteller, Journalist}|pwk}s Hinweis darauf, dass Schnitzler\pwindex{Schnitzler, Arthur 15.05.1862 – 21.10.1931@\textsc{Schnitzler, Arthur} (15.05.1862 – 21.10.1931), \emph{Schriftsteller, Mediziner}|pwk}, wenn er zu lautstark protestiere,
                  die Aufführung von \emph{Der Schleier der Beatrice}\pwindex{Schnitzler, Arthur 15.05.1862 – 21.10.1931@\textsc{Schnitzler, Arthur} (15.05.1862 – 21.10.1931), \emph{Schriftsteller, Mediziner}!Schleier der Beatrice. Schauspiel in fuenf Akten1900-12-01@\strich\emph{Der Schleier der Beatrice. Schauspiel in fünf Akten} {[}1900-12-01{]}|pwk}
                  in Gefahr bringe, hatte etwas Prophetisches. Das Stück wurde von Schlenther\pwindex{Schlenther, Paul 20.08.1854 – 30.04.1916@\textsc{Schlenther, Paul} (20.08.1854 – 30.04.1916), \emph{Schriftsteller, Kritiker, Theaterleiter}|pwk} zwar anfänglich für das \emph{Burgtheater}\orgindex{Burgtheater@Burgtheater|pwk} akzeptiert, die Zusage aber (neuerlich ohne
                  Transparenz) nach ein paar Monaten zurückgezogen (siehe Richard Beer-Hofmann an Arthur Schnitzler, 14. 9. 1900), was zu einem Skandal führte (siehe Bahr/Schnitzler, T030017). \emph{Der Schleier der Beatrice}\pwindex{Schnitzler, Arthur 15.05.1862 – 21.10.1931@\textsc{Schnitzler, Arthur} (15.05.1862 – 21.10.1931), \emph{Schriftsteller, Mediziner}!Schleier der Beatrice. Schauspiel in fuenf Akten1900-12-01@\strich\emph{Der Schleier der Beatrice. Schauspiel in fünf Akten} {[}1900-12-01{]}|pwk} wurde schließlich am 1. 12. 1900 im Lobe-Theater\oindex{Lobe-Theater@\textbf{Lobe-Theater}|pwk} in Breslau\oindex{Breslau@\textbf{Breslau}|pwk} uraufgeführt.}}}\label{K_L02893-66h}. Was liegt demgegenüber an den drei Einaktern\pwindex{Schnitzler, Arthur 15.05.1862 – 21.10.1931@\textsc{Schnitzler, Arthur} (15.05.1862 – 21.10.1931), \emph{Schriftsteller, Mediziner}!gruene Kakadu – Paracelsus – Die Gefaehrtin. Drei Einakter1898 – 1899@\strich\emph{Der grüne Kakadu – Paracelsus – Die Gefährtin. Drei Einakter} {[}1898 – 1899{]}|pwv}, die überdies \label{K_L02893-2v}\edtext{überall in Deutſchland\oindex{Deutschland@\textbf{Deutschland}|pw}}{\lemma{\textnormal{\emph{überall in Deutſchland}}}\Cendnote{\textnormal{Hervorzuheben ist der Erfolg am Deutschen Theater Berlin\oindex{Deutsches Theater Berlin@\textbf{Deutsches Theater Berlin}|pwk}. Die Einakter\pwindex{Schnitzler, Arthur 15.05.1862 – 21.10.1931@\textsc{Schnitzler, Arthur} (15.05.1862 – 21.10.1931), \emph{Schriftsteller, Mediziner}!gruene Kakadu – Paracelsus – Die Gefaehrtin. Drei Einakter1898 – 1899@\strich\emph{Der grüne Kakadu – Paracelsus – Die Gefährtin. Drei Einakter} {[}1898 – 1899{]}|pwkv} wurden dort fast 30 Mal
                  aufgeführt und waren damit Schnitzler\pwindex{Schnitzler, Arthur 15.05.1862 – 21.10.1931@\textsc{Schnitzler, Arthur} (15.05.1862 – 21.10.1931), \emph{Schriftsteller, Mediziner}|pwk}s
                  bislang größter Erfolg.}}}\label{K_L02893-2h}{ }{\pb}mit Erfolg gegeben werden, ſo daß Du ſchließlich
               auf die weitere Aufführung in Wien\oindex{Wien@\textbf{Wien}|pw} verzichten
               kannſt. Alle Lebenskunſt kommt oft darauf hinaus\textcolor{gray}{,} kleine
               Conceſſionen zu machen, um große Ziele zu erreichen. Das große Ziel iſt, daß das Burgtheater\orgindex{Burgtheater@Burgtheater|pw} Dein neues Stück\pwindex{Schnitzler, Arthur 15.05.1862 – 21.10.1931@\textsc{Schnitzler, Arthur} (15.05.1862 – 21.10.1931), \emph{Schriftsteller, Mediziner}!Schleier der Beatrice. Schauspiel in fuenf Akten1900-12-01@\strich\emph{Der Schleier der Beatrice. Schauspiel in fünf Akten} {[}1900-12-01{]}|pwv} ſpielt. Ich finde, daß dir \textsc{Schlenther\pwindex{Schlenther, Paul 20.08.1854 – 30.04.1916@\textsc{Schlenther, Paul} (20.08.1854 – 30.04.1916), \emph{Schriftsteller, Kritiker, Theaterleiter}|pw}} durch ſeinen \label{K_L02893-4v}\edtext{Beſuch}{\lemma{\textnormal{\emph{Beſuch}}}\Cendnote{\textnormal{siehe A. S.: \emph{Tagebuch}, 26. 10. 1899}}}\label{K_L02893-4h} bei Dir bereits alle mögliche Satisfaktion gegeben hat, und ich meine, Du
               ſollteſt darauf verzichten, ihn weiter zu demüthigen. Alles Sturmlaufen \strikeout{\textcolor{gray}{nu}} nützt übrigens nichts. Du wirſt dadurch nicht einen feigen und verlogenen
               Menſchen zum Muth und zur Wahrheit \strikeout{\textcolor{gray}{br}i\textcolor{gray}{n}} zwingen, und Öſterreich\oindex{Oesterreich@\textbf{Österreich}|pw} wirſt Du auch
               nicht ändern. Ich hätte dem \textsc{Schlenther\pwindex{Schlenther, Paul 20.08.1854 – 30.04.1916@\textsc{Schlenther, Paul} (20.08.1854 – 30.04.1916), \emph{Schriftsteller, Kritiker, Theaterleiter}|pw}} an Deiner Stelle geradezu geſagt: »Gut, laſſen wir’s gehen, aber ſpielen Sie
               mein neues Stück\pwindex{Schnitzler, Arthur 15.05.1862 – 21.10.1931@\textsc{Schnitzler, Arthur} (15.05.1862 – 21.10.1931), \emph{Schriftsteller, Mediziner}!Schleier der Beatrice. Schauspiel in fuenf Akten1900-12-01@\strich\emph{Der Schleier der Beatrice. Schauspiel in fünf Akten} {[}1900-12-01{]}|pwv}!« Und {\pb}wenn es nicht ſchon zu ſpät iſt, möchte ich Dir
               rathen, die Verhandlungen noch in dieſem Sinne zu führen. Kommt es aber zum offenen
               Conflict, ſo brauche ich Dir nicht erſt zu ſagen, daß Du unbedingt auf mich rechnen
               kannſt, ſolange ich das Feuilleton\pwindex{?? Werk@Nicht ermittelte Verfasserinnen und Verfasser!Frankfurter Zeitung1856 – 1943@\emph{Frankfurter Zeitung} {[}1856 – 1943{]}|pwv} redigire. Wenn freilich mein Onkel\pwindex{Mamroth, Fedor 21.02.1851 – 25.06.1907@\textsc{Mamroth, Fedor} (21.02.1851 – 25.06.1907), \emph{Journalist, Kritiker}|pwv} wieder zurück iſt, ſo wird wieder der \label{K_L02893-7v}\edtext{Einfluß ſeiner Frau\pwindex{Mamroth, Johanna 1872-05-19 – 1910-09-12@\textsc{Mamroth, Johanna} (1872-05-19 – 1910-09-12)|pwv}}{\lemma{\textnormal{\emph{Einfluß ſeiner Frau}}}\Cendnote{\textnormal{Siehe zum Einfluss Johanna Mamroth\pwindex{Mamroth, Johanna 1872-05-19 – 1910-09-12@\textsc{Mamroth, Johanna} (1872-05-19 – 1910-09-12)|pwk}s auf Fedor Mamroth\pwindex{Mamroth, Fedor 21.02.1851 – 25.06.1907@\textsc{Mamroth, Fedor} (21.02.1851 – 25.06.1907), \emph{Journalist, Kritiker}|pwk}s feuilletonistische Arbeit auch Paul Goldmann an Arthur Schnitzler, 2. [1.? 1897].}}}\label{K_L02893-7h} auf das Feuilleton der Frankfurter Zeitung\pwindex{?? Werk@Nicht ermittelte Verfasserinnen und Verfasser!Frankfurter Zeitung1856 – 1943@\emph{Frankfurter Zeitung} {[}1856 – 1943{]}|pw} beginnen, und dann bin ich
               machtlos, und Du kannſt auf nichts mehr rechnen.\pend
           \pstart
           An \textsc{Wassermann\pwindex{Wassermann, Jakob 10.03.1873 – 01.01.1934@\textsc{Wassermann, Jakob} (10.03.1873 – 01.01.1934), \emph{Schriftsteller}|pw}} habe ich – Dir zuliebe – einen \label{K_L02893-8v}\edtext{mahnenden Brief}{\lemma{\textnormal{\emph{mahnenden Brief}}}\Cendnote{\textnormal{Schnitzler\pwindex{Schnitzler, Arthur 15.05.1862 – 21.10.1931@\textsc{Schnitzler, Arthur} (15.05.1862 – 21.10.1931), \emph{Schriftsteller, Mediziner}|pwk} dürfte versucht haben, Jakob Wassermann\pwindex{Wassermann, Jakob 10.03.1873 – 01.01.1934@\textsc{Wassermann, Jakob} (10.03.1873 – 01.01.1934), \emph{Schriftsteller}|pwk} hinsichtlich seiner
                  unzufriedenstellenden Arbeit für die \emph{Frankfurter
                     Zeitung}\orgindex{Frankfurter Zeitung@Frankfurter Zeitung|pwk} vor Goldmann\pwindex{Goldmann, Paul 31.01.1865 – 25.09.1935@\textsc{Goldmann, Paul} (31.01.1865 – 25.09.1935), \emph{Schriftsteller, Journalist}|pwk} zu
                  verteidigen. Siehe Paul Goldmann an Arthur Schnitzler, 26. 10. 1899, 6. 12. [1899], 11. 12. [1899] und 23. 12. [1899].}}}\label{K_L02893-8h}
               ſchreiben laſſen. Wenn er dennoch eines Tages fällt, ſo werde ich \textsc{Schwarzkopf\pwindex{Schwarzkopf, Gustav 07.11.1853 – 13.11.1939@\textsc{Schwarzkopf, Gustav} (07.11.1853 – 13.11.1939), \emph{Schriftsteller}|pw}} und \textsc{Hirschfeld\pwindex{Hirschfeld, Robert 17.09.1857 – 02.04.1914@\textsc{Hirschfeld, Robert} (17.09.1857 – 02.04.1914), \emph{Journalist, Kritiker}|pw}}{ }\strikeout{\textcolor{gray}{×}} als ſeine Nachfolger\orgindex{Frankfurter Zeitung@Frankfurter Zeitung|pwv}
               empfehlen.\pend
           \pstart
           {\pb}Ich hätte – trotz meines Nichtſchreibens – gehofft,
               in dieſen Wochen wieder etwas von Dir zu hören. Wenn Du auf meine Antwort gewartet
               haſt, ſo laß’ mich jetzt nicht länger ohne Nachricht und ſchreibe mir, wie Du lebſt
               und was Du \label{K_L02893-9v}\edtext{arbeiteſt}{\lemma{\textnormal{\emph{arbeiteſt}}}\Cendnote{\textnormal{Am 12. 11. 1899 begann Schnitzler\pwindex{Schnitzler, Arthur 15.05.1862 – 21.10.1931@\textsc{Schnitzler, Arthur} (15.05.1862 – 21.10.1931), \emph{Schriftsteller, Mediziner}|pwk} die humoristisch angelegte Erzählung \emph{Der Leuchtkäfer}\pwindex{Schnitzler, Arthur 15.05.1862 – 21.10.1931@\textsc{Schnitzler, Arthur} (15.05.1862 – 21.10.1931), \emph{Schriftsteller, Mediziner}!Leuchtkaefer1898-11-12 – 1909-09-03@\strich\emph{Der Leuchtkäfer} {[}1898-11-12 – 1909-09-03{]}|pwk}. Noch zehn Jahre später, am
                     3. 9. 1909,
                  notierte Schnitzler\pwindex{Schnitzler, Arthur 15.05.1862 – 21.10.1931@\textsc{Schnitzler, Arthur} (15.05.1862 – 21.10.1931), \emph{Schriftsteller, Mediziner}|pwk} eine Überarbeitung der
                  posthum veröffentlichten Erzählung\pwindex{Schnitzler, Arthur 15.05.1862 – 21.10.1931@\textsc{Schnitzler, Arthur} (15.05.1862 – 21.10.1931), \emph{Schriftsteller, Mediziner}!Leuchtkaefer1898-11-12 – 1909-09-03@\strich\emph{Der Leuchtkäfer} {[}1898-11-12 – 1909-09-03{]}|pwkv} in seinem \emph{Tagebuch}\pwindex{Schnitzler, Arthur 15.05.1862 – 21.10.1931@\textsc{Schnitzler, Arthur} (15.05.1862 – 21.10.1931), \emph{Schriftsteller, Mediziner}!Tagebuch1981 – 2000@\strich\emph{Tagebuch} {[}1981 – 2000{]}|pwk}.}}}\label{K_L02893-9h}.\pend
           \pstart
           In meinem Leben bereiten ſich große Stürme und vielleicht ſehr ſchwerwiegende
               Ereigniſſe vor. Mein Verhältniß zu ihr\pwindex{Rottenberg, Theodore 1875-09-07 – 1945-04-05@\textsc{Rottenberg, Theodore} (1875-09-07 – 1945-04-05)|pwv} iſt glücklich, dank der Befliſſenheit einiger intimer
                  \label{K_L02893-67v}\edtext{Freundinnen}{\lemma{\textnormal{\emph{Freundinnen}}}\Cendnote{\textnormal{nicht ermittelt}}}\label{K_L02893-67h} und auch infolge ihrer eigenen
               Unvorſichtigkeit, zum \label{K_L02893-45v}\edtext{öffentlichen
                  Gerücht}{\lemma{\textnormal{\emph{öffentlichen
                  Gerücht}}}\Cendnote{\textnormal{siehe Paul Goldmann an Arthur Schnitzler, 11. 10. [1899]}}}\label{K_L02893-45h} geworden. Die ganze Stadt\oindex{Frankfurt am Main@\textbf{Frankfurt am Main}|pwv} ſpricht zur Zeit davon. Es heißt, ſie werde ſich von ihrem Manne\pwindex{Rottenberg, Ludwig 11.10.1864 – 6.5.1932@\textsc{Rottenberg, Ludwig} (11.10.1864 – 6.5.1932), \emph{Kapellmeister}|pwv}{ }\label{K_L02893-77v}\edtext{ſcheiden laſſen und mich
                  heirathen}{\lemma{\textnormal{\emph{ſcheiden … heirathen}}}\Cendnote{\textnormal{nicht geschehen}}}\label{K_L02893-77h}. Der
               Klatſch iſt ſo arg geworden, daß mein Chefredakteur\pwindex{Sonnemann, Leopold 1831-10-29 – 1909-10-30@\textsc{Sonnemann, Leopold} (1831-10-29 – 1909-10-30), \emph{Journalist, Herausgeber}|pwv} bei mir hat anfragen laſſen, ob er begründet
               ſei. Ein hieſiges {\pb}Klatſchblatt, die »Sonne\pwindex{?? Werk@Nicht ermittelte Verfasserinnen und Verfasser!Sonne1891 – 1930@\emph{Die Sonne} {[}1891 – 1930{]}|pw}«, hat bereits einen \label{K_L02893-99v}\edtext{\uline{Artikel\pwindex{?? Werk@Nicht ermittelte Verfasserinnen und Verfasser!?? [Zur Affaire von Paul Goldmann und Theodore Rottenberg]November? 1899@\emph{?? [Zur Affaire von Paul Goldmann und Theodore Rottenberg]} {[}November? 1899{]}|pwv}}}{\lemma{\textnormal{\emph{Artikel}}}\Cendnote{\textnormal{Es konnte kein Exemplar der \emph{Zeitschrift}\pwindex{?? Werk@Nicht ermittelte Verfasserinnen und Verfasser!Sonne1891 – 1930@\emph{Die Sonne} {[}1891 – 1930{]}|pwk} aus dem betreffenden Zeitraum
                  nachgewiesen werden.}}}\label{K_L02893-99h}{ }\strikeout{darü}{ }\strikeout{da} darüber gebracht. Der \label{K_L02893-55v}\edtext{Gemahl\pwindex{Rottenberg, Ludwig 11.10.1864 – 6.5.1932@\textsc{Rottenberg, Ludwig} (11.10.1864 – 6.5.1932), \emph{Kapellmeister}|pwv} in Wien\oindex{Wien@\textbf{Wien}|pw}}{\lemma{\textnormal{\emph{Gemahl in Wien}}}\Cendnote{\textnormal{siehe Paul Goldmann an Arthur Schnitzler, 8. 10. [1899]}}}\label{K_L02893-55h} weiß noch nichts. Aber er ſoll in einigen Tagen zurückkommen, und dann wird
               die Geſchichte wohl losgehen. Es kommt dazu, daß ſie\pwindex{Rottenberg, Theodore 1875-09-07 – 1945-04-05@\textsc{Rottenberg, Theodore} (1875-09-07 – 1945-04-05)|pwv}, von einem plötzlichen Wahrheitsdrang befallen, erklärt,
               ſie werde ihrem Manne\pwindex{Rottenberg, Ludwig 11.10.1864 – 6.5.1932@\textsc{Rottenberg, Ludwig} (11.10.1864 – 6.5.1932), \emph{Kapellmeister}|pwv}
               gegenüber nicht Alles ableugnen können. Mit banger Sorge ſehe ich der Kataſtrophe
               entgegen, die kaum mehr aufzuhalten iſt. Wenn ihr Mann\pwindex{Rottenberg, Ludwig 11.10.1864 – 6.5.1932@\textsc{Rottenberg, Ludwig} (11.10.1864 – 6.5.1932), \emph{Kapellmeister}|pwv} ſie verſtößt, muß ich natürlich ſie aufnehmen. Und was
               ſoll ich in meinen Verhältniſſen, wo ich meine Mutter\pwindex{Goldmann, Clementine 1842-05-15 – 1924-02-24@\textsc{Goldmann, Clementine} (1842-05-15 – 1924-02-24)|pwv} und mich gerade durchbringe, plötzlich mit einer Frau
               anfangen?\pend
           \pstart
           {\pb}Unter dieſen Umſtänden iſt mir dieſe kleine Stadt\oindex{Frankfurt am Main@\textbf{Frankfurt am Main}|pwv} mit ihrer
                  giftigen\textcolor{gray}{,} ganz ohne Noth bösartigen und gemeinen Klatſchſucht
               erſt recht zum Ekel geworden, und ich beklage bitter, daß ſich mein \label{K_L02893-13v}\edtext{Engagement nach Berlin\oindex{Berlin@\textbf{Berlin}|pw} für die Neue Freie
                  Preſſe\orgindex{Neue Freie Presse@Neue Freie Presse|pw} zerſchlagen}{\lemma{\textnormal{\emph{Engagement … zerſchlagen}}}\Cendnote{\textnormal{siehe Paul Goldmann an Arthur Schnitzler, 26. 10. 1899 und 4. 12. [1899]}}}\label{K_L02893-13h} hat. Hörſt Du irgend etwas, wie es mit \textsc{Frischauer\pwindex{Frischauer, Berthold 1851-09-09 – 1924-02-04@\textsc{Frischauer, Berthold} (1851-09-09 – 1924-02-04), \emph{Journalist}|pw}} ſteht? Und weißt Du vielleicht, wer \label{K_L02893-14v}\edtext{jetzt in Paris\oindex{Paris@\textbf{Paris}|pw} für die
                  N. Fr. Pr.\orgindex{Neue Freie Presse@Neue Freie Presse|pw}}{\lemma{\textnormal{\emph{jetzt … Pr.}}}\Cendnote{\textnormal{nicht geklärt, vgl. Paul Goldmann an Arthur Schnitzler, 26. 10. 1899}}}\label{K_L02893-14h} iſt?\pend
           \pstart
           Grüße mir \textsc{Richard\pwindex{Beer-Hofmann, Richard 1866-07-11 – 1945-09-26@\textsc{Beer-Hofmann, Richard} (1866-07-11 – 1945-09-26), \emph{Schriftsteller}|pw}}, \textsc{Schwarzkopf\pwindex{Schwarzkopf, Gustav 07.11.1853 – 13.11.1939@\textsc{Schwarzkopf, Gustav} (07.11.1853 – 13.11.1939), \emph{Schriftsteller}|pw}}, Deinen Bruder\pwindex{Schnitzler, Julius 13.07.1865 – 29.06.1939@\textsc{Schnitzler, Julius} (13.07.1865 – 29.06.1939), \emph{Mediziner}|pwv}, Deinen
                  Schwager\pwindex{Hajek, Markus 25.11.1861 – 04.04.1941@\textsc{Hajek, Markus} (25.11.1861 – 04.04.1941), \emph{Mediziner, Mediziner}|pwv} und alle die
               anderen lieben Menſchen; empfiehl’ mich Deiner Frau Mutter\pwindex{Schnitzler, Louise 1840-07-08 – 1911-09-09@\textsc{Schnitzler, Louise} (1840-07-08 – 1911-09-09)|pwv}{ }{\pb}und ſei Du ſelbſt von Herzen gegrüßt –\pend
           \pstart
           von Deinem treuen {\\[\baselineskip]}\spacefill\mbox{Paul Goldmann.}\pend
           \leftskip=0em{}
         
         \endnumbering\mylabel{h}\end{ledgroupsized}  \newcommand{\dateiname}{L02893}\newcommand{\titel}{Paul Goldmann an Arthur Schnitzler, 12. 11. [1899]}\newcommand{\editorInnen}{Martin Anton Müller und Laura Untner}%% latex-leseansicht-abspann.tex
%% Abspann für die Leseansicht.
%% Der Schalter \ifkorrekturansicht ist bereits durch den Vorspann gesetzt.

%% latex-abspann.tex
%% Gemeinsamer Abspann für Korrekturansicht und Leseansicht.
%% Setzt den Schalter \ifkorrekturansicht voraus (gesetzt in den
%% einbindenden Dateien latex-korrekturansicht-abspann.tex bzw.
%% latex-leseansicht-abspann.tex).
%% ---------------------------------------------------------------

\normalsize

% Das esempio-Environment wird nur in der Leseansicht benötigt
\ifkorrekturansicht\else
\newenvironment{esempio}[3]%
{
    \vspace{1.5ex}
    \rlap{\underline{#1}}
    \par
    \setlength{\parindent}{0cm}
    \nopagebreak
    \leftskip=#2cm
    \rightskip=#3cm
}
{
    \par
}
\fi

\doendnotes{C}
\bigskip
\vfill

\clearpage

\footnotesize

\ifkorrekturansicht
  \lohead{\textsc{register}}
\fi

% theindex-Environment neu definieren ohne reledmac
\makeatletter
\renewenvironment{theindex}{%
  \ifkorrekturansicht
    \section*{\indexname}%
  \else
    \subsubsection*{Index der erwähnten Entitäten}%
  \fi
  \setlength{\parindent}{0pt}%
  \setlength{\parskip}{0pt plus 0.3pt}%
  \let\item\@idxitem
}{%
  \ifkorrekturansicht\clearpage\fi
}
\makeatother

\IfFileExists{\jobname-pw.ind}{\input{\jobname-pw.ind}}{}

% Quellenangabe nur in der Leseansicht
\ifkorrekturansicht\else
% Fallback-Definitionen, falls die .tex-Datei \titel etc. nicht gesetzt hat
\providecommand{\titel}{}
\providecommand{\editorInnen}{}
\providecommand{\dateiname}{\jobname}

\vspace{3cm}

\vfill

\footnotesize
\textsc{Quelle}: \titel. Herausgegeben von {\editorInnen}. In: \emph{Arthur Schnitzler: Briefwechsel mit Autorinnen und Autoren}.
 Digitale Edition, https://schnitzler-briefe.acdh.oeaw.ac.at/{\dateiname}.html (Stand \today)
\fi

\end{document}


      