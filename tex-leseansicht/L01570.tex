%% latex-korrekturansicht-vorspann.tex
%% Vorspann für die Korrekturansicht.
%% Lädt die gemeinsame Datei latex-vorspann.tex mit gesetztem Schalter.

\newif\ifkorrekturansicht
\korrekturansichttrue

\input{../tex-inputs/latex-vorspann}


\section[Albert Ehrenstein an Arthur Schnitzler, 21. 12. 1905]{L01570 Albert Ehrenstein an Arthur Schnitzler, 21. 12. 1905}
\nopagebreak\mylabel{L01570v}
\rehead{ }\normalsize\beginnumbering\briefempfaengerindex{Schnitzler, Arthur@\textsc{Schnitzler, Arthur}!zzzEhrenstein, Albert@\emph{von Albert Ehrenstein}!1905-12-211@{21. 12. 1905}|(be}
\toendnotes[C]{\smallbreak\pagebreak[2]}\Standort{CUL, Schnitzler, B 30.}
\physDesc{Brief, 1 Blatt, 1 Seite, 383 Zeichen
\newline{}Handschrift: schwarze Tinte, deutsche Kurrent}
\buchAbdrucke{\weitereDrucke{Albert Ehrenstein: \emph{Briefe}. München: \emph{Boer} 1989, S. 19.} }\toendnotes[C]{\smallbreak}
\pstart
           \raggedleft{}{\pb}21. XII. 1905. \pend
           
\pstart{}\textsc{Sehr geehrter Herr Doktor!}\pend\vspace{0.5em}
\pstart
           Allzugroße Nachſicht ſcheint ſich zu rächen in Geſtalt von noch ſieben \label{K_L01570-1v}\edtext{Trauerſchwänkchen}{\lemma{\textnormal{\emph{Trauerſchwänkchen}}}\Cendnote{\textnormal{\emph{Amok}\pwindex{Amok@\emph{Amok}|pwk} wird von Schnitzler als »Trauerschwank« bezeichnet
                     (A. S.: \emph{Tagebuch}, 6. 12. 1905). Die anderen sind nicht
                  identifiziert.}}}\label{K_L01570-1}, die ein armer Bakkalaureus, ſtark \label{K_L01570-2v}\edtext{gedäftet}{\lemma{\textnormal{\emph{gedäftet}}}\Cendnote{\textnormal{Kleinlaut geworden – Schnitzler hatte Ehrenstein\pwindex{Ehrenstein, Albert 23.12.1886 – 08.04.1950@\textsc{Ehrenstein, Albert} (23.12.1886 – 08.04.1950), \emph{Schriftsteller/Schriftstellerin}|pwk} am
                     12. 12. 1905 und
                  am 20. 12. 1905
                  mündlich sein Urteil mitgeteilt.}}}\label{K_L01570-2} und dankbar auch dafür, Herrn Doktor
               vorzulegen wagt. In der Hoffnung Herrn Doktors Geduld und Liebenswürdigkeit durch
               dieſen Skizzenkranz nicht gar zu arg mißbraucht zu haben, verbleibt\pend
           
\pstart
           Ergebenſt{\\[\baselineskip]}\spacefill\mbox{Albert Ehrenstein.}\pend
           \leftskip=0em{}\selectlanguage{ngerman}\endnumbering\briefempfaengerindex{Schnitzler, Arthur@\textsc{Schnitzler, Arthur}!zzzEhrenstein, Albert@\emph{von Albert Ehrenstein}!1905-12-211@{21. 12. 1905}|)be}\mylabel{L01570h}  \normalsize

\doendnotes{C}
\bigskip
\vfill

\clearpage

\footnotesize

\lohead{\textsc{register}}

% Definiere theindex-Environment komplett neu ohne reledmac
\makeatletter
\renewenvironment{theindex}{%
  \section*{\indexname}%
  \setlength{\parindent}{0pt}%
  \setlength{\parskip}{0pt plus 0.3pt}%
  \let\item\@idxitem
}{%
  \clearpage
}
\makeatother

\IfFileExists{\jobname-pw.ind}{\input{\jobname-pw.ind}}{}

\end{document}

      