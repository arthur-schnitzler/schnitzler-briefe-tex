%% latex-leseansicht-vorspann.tex
%% Vorspann für die Leseansicht.
%% Lädt die gemeinsame Datei latex-vorspann.tex mit nicht gesetztem Schalter.

\newif\ifkorrekturansicht
\korrekturansichtfalse

\input{../tex-inputs/latex-vorspann}


\section[Elsa Plessner an Arthur Schnitzler, 12. 10. 1900]{L03728 Elsa Plessner an Arthur Schnitzler, 12. 10. 1900}
\nopagebreak\mylabel{L03728v}
\rehead{ }\normalsize\beginnumbering\briefempfaengerindex{Schnitzler, Arthur@\textsc{Schnitzler, Arthur}!zzzPlessner, Elsa@\emph{von Elsa Plessner}!1900-10-122@{12. 10. 1900}|(be}
\toendnotes[C]{\smallbreak\pagebreak[2]}
\correspDesc{Versand  durch Elsa Plessner am 12. 10. 1900 in Wien
\newline{}Erhalt  durch Arthur Schnitzler im Zeitraum [12. 10. 1900 – 15. 10. 1900?] in Wien}\toendnotes[C]{\smallbreak}
\Standort{DLA, A:Schnitzler, HS.1985.1.419.}
\physDesc{Brief, 2 Blätter, 8 Seiten, 5293 Zeichen
\newline{}Handschrift: schwarze Tinte, lateinische Kurrent}\toendnotes[C]{\smallbreak}
\pstart
           {\pb}Wien I. Kärnthnerstrasse N\textsuperscript{o} 10\oindex{Wien@\textbf{Wien}!I., Innere Stadt@\textbf{I., Innere Stadt}!Kärntner Straße 10@\textbf{Kärntner Straße 10}, \emph{Wohngebäude}|pw}\pend
           
\pstart
           \raggedleft{}den 12. Oktober 1900\pend
           
\pstart{}Verehrter Herr Doctor!\pend\vspace{0.5em}
\pstart
           \label{K_L03728-1v}\edtext{Sie wünschen}{\lemma{\textnormal{\emph{Sie wünschen}}}\Cendnote{\textnormal{Schnitzlers Brief ist nicht
                  überliefert.}}}\label{K_L03728-1} die chronologische Reihenfolge der Arbeiten zu wissen und
               indem Sie mir eine Erwiderung auf meine Bemerkung über meine Entwickelung
               versprechen, \label{K_L03728-2v}\edtext{kündigen Sie mir – deutlich genug für mein zartes Verständnis – einen
               neuerlichen Putzer}{\lemma{\textnormal{\emph{kündigen … Putzer}}}\Cendnote{\textnormal{Putzer: österreichisch-bairisch: Tadel, Rüge. Ob und wie Schnitzler auf
               den vorliegenden Brief geantwortet hat, ist ungewiss. Es hinterlässt einen unhöflichen Eindruck, wenn er geschwiegen hätte. Aber eine ausführliche
               Antwort dürfte wiederum eine Reaktion von Plessner\pwindex{Plessner, Elsa 22.\,8.\,1875 Wien – 7.\,5.\,1932 Alicante@\textsc{Plessner, Elsa} (22.\,8.\,1875 Wien – 7.\,5.\,1932 Alicante), \emph{Schriftstellerin}|pwk} hervorgerufen haben – und eine solche liegt 
                  nicht vor. Auch anlässlich der Uraufführung von \emph{Die Ehrlosen}\pwindex{Plessner, Elsa 22.\,8.\,1875 Wien – 7.\,5.\,1932 Alicante@\textsc{Plessner, Elsa} (22.\,8.\,1875 Wien – 7.\,5.\,1932 Alicante), \emph{Schriftstellerin}!Ehrlosen. Schauspiel in drei Acten@\strich\emph{Die Ehrlosen. Schauspiel in drei Acten}|pwk} dürfte es zu keiner Kontaktaufnahme
                  gekommen sein. Ein \emph{Tagebuch}\pwindex{Schnitzler, Arthur 15. 5. 1862 Wien – 21. 10. 1931 ebd.@\textsc{Schnitzler, Arthur} (15. 5. 1862 Wien – 21. 10. 1931 ebd.), \emph{Schriftsteller, Mediziner}!Tagebuch@\strich\emph{Tagebuch}|pwk}-Eintrag Schnitzlers zum
                  21. 11. 1909 (und ein persönliches Treffen wenige Tage später) belegt, dass es späterhin eine soziale Wahrnehmung auf Distanz gab.}}}\label{K_L03728-2}
               an. – Zuerst entspreche ich beiliegend Ihrem Wunsche und dann muss ich – zu Ihrer
               Orientirung \uline{vor} dem Putzer – etwas weiter ausholen.
               Sie finden also zuerst die merkwürdige Thatsache dass »Baby\pwindex{Plessner, Elsa 22.\,8.\,1875 Wien – 7.\,5.\,1932 Alicante@\textsc{Plessner, Elsa} (22.\,8.\,1875 Wien – 7.\,5.\,1932 Alicante), \emph{Schriftstellerin}!Baby@\strich\emph{Baby}|pw}« von allen die älteste Arbeit ist. Es ist mir sehr
               verständlich, dass »Baby\pwindex{Plessner, Elsa 22.\,8.\,1875 Wien – 7.\,5.\,1932 Alicante@\textsc{Plessner, Elsa} (22.\,8.\,1875 Wien – 7.\,5.\,1932 Alicante), \emph{Schriftstellerin}!Baby@\strich\emph{Baby}|pw}« Ihnen nicht allzusehr
               missfällt, denn es beweist neuerdings, dass das Wesensverwandte jeden Menschen
               anzieht. Es ist ja auf 1000 Schritte sichtbar, dass diese Geschich{[}te{]}\pwindex{Plessner, Elsa 22.\,8.\,1875 Wien – 7.\,5.\,1932 Alicante@\textsc{Plessner, Elsa} (22.\,8.\,1875 Wien – 7.\,5.\,1932 Alicante), \emph{Schriftstellerin}!Baby@\strich\emph{Baby}|pwv}{ }{\pb}unter Ihrem \strikeout{D}{ }\uline{directen} Einfluss entstanden ist und – bitte um
               Verzeihung für die Arroganz – Leo, der Held könnte ganz gut – Anatol\pwindex{Schnitzler, Arthur 15. 5. 1862 Wien – 21. 10. 1931 ebd.@\textsc{Schnitzler, Arthur} (15. 5. 1862 Wien – 21. 10. 1931 ebd.), \emph{Schriftsteller, Mediziner}!Anatol@\strich\emph{Anatol}|pwv} – heißen, was natürlich nichts \strikeout{zu} daran ändert, dass die Geschichte\pwindex{Plessner, Elsa 22.\,8.\,1875 Wien – 7.\,5.\,1932 Alicante@\textsc{Plessner, Elsa} (22.\,8.\,1875 Wien – 7.\,5.\,1932 Alicante), \emph{Schriftstellerin}!Baby@\strich\emph{Baby}|pwv} selbstverständlich nicht im
               entferntesten an Ihren »Anatol\pwindex{Schnitzler, Arthur 15. 5. 1862 Wien – 21. 10. 1931 ebd.@\textsc{Schnitzler, Arthur} (15. 5. 1862 Wien – 21. 10. 1931 ebd.), \emph{Schriftsteller, Mediziner}!Anatol@\strich\emph{Anatol}|pw}« heranreicht. Das
               heißt mit anderen Worten: »Ich hatte mich so in Ihr Buch\pwindex{Schnitzler, Arthur 15. 5. 1862 Wien – 21. 10. 1931 ebd.@\textsc{Schnitzler, Arthur} (15. 5. 1862 Wien – 21. 10. 1931 ebd.), \emph{Schriftsteller, Mediziner}!Anatol@\strich\emph{Anatol}|pwv} hereingelesen, dass ich auf einmal Ihre Sprache
               sprach«. – Vielleicht interessiert Sie die Thatsache, dass »\label{K_L03728-3v}\edtext{Baby\pwindex{Plessner, Elsa 22.\,8.\,1875 Wien – 7.\,5.\,1932 Alicante@\textsc{Plessner, Elsa} (22.\,8.\,1875 Wien – 7.\,5.\,1932 Alicante), \emph{Schriftstellerin}!Baby@\strich\emph{Baby}|pw}}{\lemma{\textnormal{\emph{Baby}}}\Cendnote{\textnormal{Die Geschichte handelt davon, dass eine
                  Frau ihrem Liebhaber das gemeinsame Kind vorstellt, das bislang am Land aufgezogen
                  wird. Sie möchte, dass der Vater das Kind als Anlass nimmt, sie zu heiraten. Zwar
                  empfindet dieser väterliche Gefühle, doch zu einer Eheschließung kann er sich
                  nicht durchringen. Die Frau erwähnt, mit einem »Baron« in Kontakt
                  zu stehen, der sie heiraten wolle und das Kind als seines aufzuziehen bereit
                  sei.}}}\label{K_L03728-3}« zwei Tage nach dem \label{K_L03728-4v}\edtext{Tod
               meines Vaters\pwindex{Plessner, Louis 3.\,12.\,1847 Bielsko-Biała – 19.\,9.\,1895 Wien@\textsc{Plessner, Louis} (3.\,12.\,1847 Bielsko-Biała – 19.\,9.\,1895 Wien), \emph{Journalist, Kaufmann}|pwv}}{\lemma{\textnormal{\emph{Tod
               meines Vaters}}}\Cendnote{\textnormal{Louis Plessner\pwindex{Plessner, Louis 3.\,12.\,1847 Bielsko-Biała – 19.\,9.\,1895 Wien@\textsc{Plessner, Louis} (3.\,12.\,1847 Bielsko-Biała – 19.\,9.\,1895 Wien), \emph{Journalist, Kaufmann}|pwk} starb am
                     19. 9. 1895 in Wien\oindex{Wien@\textbf{Wien}, \emph{Verwaltungsgebiet}|pwk}.}}}\label{K_L03728-4}
               entstanden ist. – Außerdem theile ich Ihnen im Vertrauen mit, dass der Held dieser
               Geschichte \label{K_L03728-5v}\edtext{eigentlich \uline{Kainz\pwindex{Kainz, Josef 2.\,1.\,1858 Mosonmagyaróvár – 20.\,9.\,1910 Wien@\textsc{Kainz, Josef} (2.\,1.\,1858 Mosonmagyaróvár – 20.\,9.\,1910 Wien), \emph{Schauspieler}|pw}}}{\lemma{\textnormal{\emph{eigentlich Kainz}}}\Cendnote{\textnormal{Die hier gegebenen Hinweise passen mit
                  historischen Fakten zusammen. 
                  Die aus Ungarn\oindex{Ungarn@\textbf{Ungarn}|pwk}
                  stammende Schauspielerin Jolán Ramazetter\pwindex{Ramazetta, Jolantha *~8.\,2.\,1859 Veszprém@\textsc{Ramazetta, Jolantha} (*~8.\,2.\,1859 Veszprém), \emph{Schauspielerin}|pwk},
                  die den eingedeutschten Bühnennamen Jolantha
                     Ramazetta\pwindex{Ramazetta, Jolantha *~8.\,2.\,1859 Veszprém@\textsc{Ramazetta, Jolantha} (*~8.\,2.\,1859 Veszprém), \emph{Schauspielerin}|pwk} verwendete, war um 1884 gemeinsam mit Kainz\pwindex{Kainz, Josef 2.\,1.\,1858 Mosonmagyaróvár – 20.\,9.\,1910 Wien@\textsc{Kainz, Josef} (2.\,1.\,1858 Mosonmagyaróvár – 20.\,9.\,1910 Wien), \emph{Schauspieler}|pwk} am \emph{Deutschen Theater}\orgindex{Deutsches Theater Berlin@Deutsches Theater Berlin|pwk} in Berlin\oindex{Berlin@\textbf{Berlin}, \emph{Hauptstadt}|pwk}
                  engagiert. Am 28. 7. 1884 übernahm das \emph{Neue Wiener Tagblatt}\pwindex{Neues Wiener Tagblatt@\emph{Neues Wiener Tagblatt}|pwk} eine Meldung des \emph{Berliner Börsen-Couriers}\pwindex{Berliner Börsen-Courier@\emph{Berliner Börsen-Courier}|pwk}, dass Kainz\pwindex{Kainz, Josef 2.\,1.\,1858 Mosonmagyaróvár – 20.\,9.\,1910 Wien@\textsc{Kainz, Josef} (2.\,1.\,1858 Mosonmagyaróvár – 20.\,9.\,1910 Wien), \emph{Schauspieler}|pwk} und Ramazetta\pwindex{Ramazetta, Jolantha *~8.\,2.\,1859 Veszprém@\textsc{Ramazetta, Jolantha} (*~8.\,2.\,1859 Veszprém), \emph{Schauspielerin}|pwk}
                  verlobt seien (Nr. 207, S. 3). Die Rekonstruktion der nächsten Ereignisse
                  gelingt nur rückwirkend, über den Tod des Sohnes.  Am 14. 10. 1911 starb in
                     Budapest\oindex{Budapest@\textbf{Budapest}, \emph{Hauptstadt}|pwk} der Journalist und Sprachlehrer
                     Lajos Staél-Dergy\pwindex{Stael-Dergy, Lajos 1885/1886 – 14.\,10.\,1911 Budapest@\textsc{Stael-Dergy, Lajos} (1885/1886 – 14.\,10.\,1911 Budapest), \emph{Journalist, Sprachlehrer}|pwk} an einer
                  Schussverletzung. Er war zu diesem Zeitpunkt 26 Jahre alt und nach Paris\oindex{Paris@\textbf{Paris}, \emph{Hauptstadt}|pwk} zuständig. Als Mutter wird Jolán Ramazetter\pwindex{Ramazetta, Jolantha *~8.\,2.\,1859 Veszprém@\textsc{Ramazetta, Jolantha} (*~8.\,2.\,1859 Veszprém), \emph{Schauspielerin}|pwk} genannt, als Vater Lajos Staél-Dergy\pwindex{Stael-Dergy, Louis @\textsc{Stael-Dergy, Louis}|pwk} (der Vorname dürfte dem
                  französischen ›Louis‹ entsprechen). Zum mutmaßlichen Zeitpunkt der Geburt – um 1885 – war Jolán
                     Ramazetter\pwindex{Ramazetta, Jolantha *~8.\,2.\,1859 Veszprém@\textsc{Ramazetta, Jolantha} (*~8.\,2.\,1859 Veszprém), \emph{Schauspielerin}|pwk} mehrere Monate an
                  einem Theater in Sankt Petersburg\oindex{Sankt Petersburg@\textbf{Sankt Petersburg}|pwk} engagiert.
                  Danach nahm sie Engagements in Paris\oindex{Paris@\textbf{Paris}, \emph{Hauptstadt}|pwk} an. Als
                     Kainz\pwindex{Kainz, Josef 2.\,1.\,1858 Mosonmagyaróvár – 20.\,9.\,1910 Wien@\textsc{Kainz, Josef} (2.\,1.\,1858 Mosonmagyaróvár – 20.\,9.\,1910 Wien), \emph{Schauspieler}|pwk} im Sterben lag, erfuhr Schnitzler auch direkt von der Existenz des
                  gemeinsamen Kindes\pwindex{Stael-Dergy, Lajos 1885/1886 – 14.\,10.\,1911 Budapest@\textsc{Stael-Dergy, Lajos} (1885/1886 – 14.\,10.\,1911 Budapest), \emph{Journalist, Sprachlehrer}|pwkv}, vgl. A. S.: \emph{Tagebuch}, 11. 9. 1910.}}}\label{K_L03728-5} ist, um
               dessen Kind\pwindex{Stael-Dergy, Lajos 1885/1886 – 14.\,10.\,1911 Budapest@\textsc{Stael-Dergy, Lajos} (1885/1886 – 14.\,10.\,1911 Budapest), \emph{Journalist, Sprachlehrer}|pwv} es sich (vor
               fünf Jahren nach dem \label{K_L03728-6v}\edtext{Tod seiner ersten Frau\pwindex{Kainz, Sara 26.\,3.\,1853 St. Louis – 24.\,6.\,1893 Berlin@\textsc{Kainz, Sara} (26.\,3.\,1853 St. Louis – 24.\,6.\,1893 Berlin), \emph{Schriftstellerin}|pwv}}{\lemma{\textnormal{\emph{Tod seiner ersten Frau}}}\Cendnote{\textnormal{Sara Kainz\pwindex{Kainz, Sara 26.\,3.\,1853 St. Louis – 24.\,6.\,1893 Berlin@\textsc{Kainz, Sara} (26.\,3.\,1853 St. Louis – 24.\,6.\,1893 Berlin), \emph{Schriftstellerin}|pwk} war am 24. 6. 1893 in Berlin\oindex{Berlin@\textbf{Berlin}, \emph{Hauptstadt}|pwk} gestorben.}}}\label{K_L03728-6}) handelte. Ich habe die Geschichte\pwindex{Plessner, Elsa 22.\,8.\,1875 Wien – 7.\,5.\,1932 Alicante@\textsc{Plessner, Elsa} (22.\,8.\,1875 Wien – 7.\,5.\,1932 Alicante), \emph{Schriftstellerin}!Baby@\strich\emph{Baby}|pwv} – die natürlich anders sich
               abspielte – {\pb}direct von Rosie Hutzler\pwindex{Hutzler, Rosine @\textsc{Hutzler, Rosine}, \emph{Schauspielerin}|pw}, seiner Stieftochter erfahren. Die Mutter –
               ehemals Mitglied des »Deutschen Theaters\orgindex{Deutsches Theater Berlin@Deutsches Theater Berlin|pw}{[}«{]} – Fräulein Ramacetta\pwindex{Ramazetta, Jolantha *~8.\,2.\,1859 Veszprém@\textsc{Ramazetta, Jolantha} (*~8.\,2.\,1859 Veszprém), \emph{Schauspielerin}|pw}
               ist in Paris\oindex{Paris@\textbf{Paris}, \emph{Hauptstadt}|pw} an einen Baron\pwindex{Stael-Dergy, Louis @\textsc{Stael-Dergy, Louis}|pwv} verheirathet, der das etwa
               zwölfjährige Kind\pwindex{Stael-Dergy, Lajos 1885/1886 – 14.\,10.\,1911 Budapest@\textsc{Stael-Dergy, Lajos} (1885/1886 – 14.\,10.\,1911 Budapest), \emph{Journalist, Sprachlehrer}|pwv} adoptirt
               hat. – – – Nach dieser kleinen Abschweifung in die \label{K_L03728-7v}\edtext{\begin{otherlanguage}{french}chronique scandaleuse\end{otherlanguage}}{\lemma{\textnormal{\emph{chronique scandaleuse}}}\Cendnote{\textnormal{französisch: Skandalchronik,
                  Klatschberichterstattung}}}\label{K_L03728-7}{ }kehre ich zur Materie \introOben{}dieser
                  Epistel\introOben{} zurück. –\pend
           
\pstart
           Das ganze Buch »D. g. Käfig\pwindex{Plessner, Elsa 22.\,8.\,1875 Wien – 7.\,5.\,1932 Alicante@\textsc{Plessner, Elsa} (22.\,8.\,1875 Wien – 7.\,5.\,1932 Alicante), \emph{Schriftstellerin}!gläserne Käfig. Skizzen und Novellen@\strich\emph{Der gläserne Käfig. Skizzen und Novellen}|pw}« hat keinen anderen
               Zweck als den, meiner im Anfang December stattfindenden \label{K_L03728-8v}\edtext{Première\eventindex{Volkstheater@\textbf{Volkstheater}!Uraufführung von Die Ehrlosen, 16.3.1901@Uraufführung von Die Ehrlosen, 16.3.1901|pw}}{\lemma{\textnormal{\emph{Première}}}\Cendnote{\textnormal{Die Premiere\eventindex{Volkstheater@\textbf{Volkstheater}!Uraufführung von Die Ehrlosen, 16.3.1901@Uraufführung von Die Ehrlosen, 16.3.1901|pwkv} von Plessners\pwindex{Plessner, Elsa 22.\,8.\,1875 Wien – 7.\,5.\,1932 Alicante@\textsc{Plessner, Elsa} (22.\,8.\,1875 Wien – 7.\,5.\,1932 Alicante), \emph{Schriftstellerin}|pwk} Schauspiel \emph{Die Ehrlosen}\pwindex{Plessner, Elsa 22.\,8.\,1875 Wien – 7.\,5.\,1932 Alicante@\textsc{Plessner, Elsa} (22.\,8.\,1875 Wien – 7.\,5.\,1932 Alicante), \emph{Schriftstellerin}!Ehrlosen. Schauspiel in drei Acten@\strich\emph{Die Ehrlosen. Schauspiel in drei Acten}|pwk}
                  fand erst am 16. 3. 1901 am Volkstheater statt.}}}\label{K_L03728-8} zu präludieren
               und ein paar Talentproben in die Welt der Premièrenbesucher zu schleudern, damit ich
               nicht ganz wie ein rother Hund behandelt werde, wenn man gar nichts von mir weiß und
               kennt. Glauben Sie ja nicht, dass ich mich irgend welchen Illusionen über den {\pb}Wert des Buches\pwindex{Plessner, Elsa 22.\,8.\,1875 Wien – 7.\,5.\,1932 Alicante@\textsc{Plessner, Elsa} (22.\,8.\,1875 Wien – 7.\,5.\,1932 Alicante), \emph{Schriftstellerin}!gläserne Käfig. Skizzen und Novellen@\strich\emph{Der gläserne Käfig. Skizzen und Novellen}|pwv} hingebe. Aber da ich meine novellistische
               »Thätigkeit« seit 2 Jahren abgeschlossen habe – (\label{K_L03728-9v}\edtext{»Der neue Lehrer\pwindex{Plessner, Elsa 22.\,8.\,1875 Wien – 7.\,5.\,1932 Alicante@\textsc{Plessner, Elsa} (22.\,8.\,1875 Wien – 7.\,5.\,1932 Alicante), \emph{Schriftstellerin}!neue Lehrer. Novelle@\strich\emph{Der neue Lehrer. Novelle}|pw}« war
               das letzte}{\lemma{\textnormal{\emph{»Der … letzte}}}\Cendnote{\textnormal{Am XXXX Auszeichnungsfehler: Dokument L03717 nicht gefunden erwähnte Plessner\pwindex{Plessner, Elsa 22.\,8.\,1875 Wien – 7.\,5.\,1932 Alicante@\textsc{Plessner, Elsa} (22.\,8.\,1875 Wien – 7.\,5.\,1932 Alicante), \emph{Schriftstellerin}|pwk} ihren längsten Prosatext \emph{Der neue Lehrer}\pwindex{Plessner, Elsa 22.\,8.\,1875 Wien – 7.\,5.\,1932 Alicante@\textsc{Plessner, Elsa} (22.\,8.\,1875 Wien – 7.\,5.\,1932 Alicante), \emph{Schriftstellerin}!neue Lehrer. Novelle@\strich\emph{Der neue Lehrer. Novelle}|pwk} erstmals, ohne jedoch den
                  Titel zu nennen.}}}\label{K_L03728-9}) hat es mir Spaß gemacht, die besseren Arbeiten dieser
               Sorte zu einem Debut zusammenzufassen. –\pend
           
\pstart
           Ich muss Sie bitten mir zu glauben, dass ich mein Vertrauen \strikeout{und meine} nicht so offen in der Hand zu jedermanns Belieben herumtrage.
               Aber da Sie sich kennen und Ihre Fähigkeit zu verstehen, werden Sie es begreiflich
               finden, dass ich gerade bei Ihnen Verständnis suchte und noch suche, denn einen
               Menschen muss man doch haben, bei dem man sich ausjammern kann, ohne dass er es
               anders deutet. Das heißt mit kurzen Worten: Ich bin seit mehr als einem Jahr an einem
               toten {\pb}Punkt meiner Entwickelung angelangt, den ich
               nicht überwinden kann. Seit dem »ersten
               Capitel\pwindex{Plessner, Elsa 22.\,8.\,1875 Wien – 7.\,5.\,1932 Alicante@\textsc{Plessner, Elsa} (22.\,8.\,1875 Wien – 7.\,5.\,1932 Alicante), \emph{Schriftstellerin}!erste Kapitel. Schauspiel in drei Akten@\strich\emph{Das erste Kapitel. Schauspiel in drei Akten}|pw}« habe ich außer zu Briefen nicht die Feder in die Hand genommen und
               nicht eine Zeile schreiben können. Ich würde mich wieder für »fertig« halten, wenn
               Sie mir das nicht seinerzeit nach Meran\oindex{Meran@\textbf{Meran}, \emph{Hauptstadt}|pw} so
               nachdrücklich verwiesen hätten. Aber eine so fürchterliche Zeit absoluter Leere und
               Unfähigkeit wie dieses Jahr habe ich noch nie durchgemacht und zu einer
               Zeit, wo mein brennender äußerer Ehrgeiz \strikeout{eigentlich}
               zu seinem Rechte zu kommen beginnt – bin ich eigentlich so sterbensunglücklich wie
               ein Mensch es nur sein kann!\pend
           
\pstart
           Vielleicht ist es das Warten auf die Première\eventindex{Volkstheater@\textbf{Volkstheater}!Uraufführung von Die Ehrlosen, 16.3.1901@Uraufführung von Die Ehrlosen, 16.3.1901|pw},
               das mich so lähmt – aber was mache ich, wenn die »Ehrlosen\pwindex{Plessner, Elsa 22.\,8.\,1875 Wien – 7.\,5.\,1932 Alicante@\textsc{Plessner, Elsa} (22.\,8.\,1875 Wien – 7.\,5.\,1932 Alicante), \emph{Schriftstellerin}!Ehrlosen. Schauspiel in drei Acten@\strich\emph{Die Ehrlosen. Schauspiel in drei Acten}|pw}« {\pb}durchfallen, was doch immerhin
               möglich ist? Bei dem absoluten Versagen aller meiner innerlichen Lebensmöglichkeiten
               sehe ich nichts weiter vor mir, wenn auch mein äußerer Lebenszweck unerreichbar ist.
               Ich habe die schönsten und wertvollsten Jahre meines Lebens vergehen lassen, ohne
               nach rechts und links zu schauen wie andere Mädchen, habe mit Scheuklappen auf mein
               künstlerisches Ziel hingearbeitet und im Gefühle einer gewissen inneren Kraft auf
               Manches verzichtet, um mich nicht zu verzetteln und zu zersplittern – und wenn ich
               mir jetzt vorstelle, dass das Alles umsonst war, könnte ich weinen um jeden Ball, auf
               dem ich mir den Kopf zerbrochen habe um eine Arbeit, statt zu tanzen und mich – zu
               amüsieren. – – Ich habe auf {\pb}der ganzen Welt nichts,
               als meine Arbeit – ob gut oder schlecht ist eigentlich egal. Aber wenn ich nicht
               einmal mehr arbeiten kann – ? – Also wenn Sie jetzt noch vo\substVorne{}\textsuperscript{m}\substDazwischen{}n\substHinten{} Entwickelung in Bezug auf mich sprechen wollen, so können sie nur von der
               Zeit sprechen, die weit hinter mir liegt! Zu dem Standpunkt der alten Arbeiten kann
               ich nicht zurück und vor mir liegt kein Weg mehr. Außer Sie sehen weiter und mehr als
               ich selbst.\pend
           
\pstart
           Das musste ich Ihnen noch vorher sagen und \strikeout{dass ich}
               Sie mit den Voraussetzungen bekannt machen \strikeout{musste},
               aus denen Sie Ihre Schlüsse ziehen {\pb}können. Ich bin
               neugierig wie dieselben ausfallen werden.\pend
           
\pstart
           Herzlich und stets verehrend{\\[\baselineskip]}Ihre{\\[\baselineskip]}\spacefill\mbox{Elsa Plessner.}\pend
           \leftskip=0em{}{\vspace{1\baselineskip}}\settowidth{\longeste}{Baby                                      x}\settowidth{\longestz}{(September 95)}\settowidth{\longestd}{Meran}\settowidth{\longestv}{}\settowidth{\longestf}{}\addtolength\longeste{1em}
        \addtolength\longestz{1em}
        \addtolength\longestd{1em}
      \pstart\noindent\makebox[\the\longeste][l]{Baby\pwindex{Plessner, Elsa 22.\,8.\,1875 Wien – 7.\,5.\,1932 Alicante@\textsc{Plessner, Elsa} (22.\,8.\,1875 Wien – 7.\,5.\,1932 Alicante), \emph{Schriftstellerin}!Baby@\strich\emph{Baby}|pw}                         
                                    
                        }\makebox[\the\longestz][l]{(September 95)}
                  \makebox[\the\longestd][l]{}\pend\pstart\noindent\makebox[\the\longeste][l]{Begräbnistag\pwindex{Plessner, Elsa 22.\,8.\,1875 Wien – 7.\,5.\,1932 Alicante@\textsc{Plessner, Elsa} (22.\,8.\,1875 Wien – 7.\,5.\,1932 Alicante), \emph{Schriftstellerin}!Begräbnißtag@\strich\emph{Der Begräbnißtag}|pw}}\makebox[\the\longestz][l]{(95}
                  \makebox[\the\longestd][l]{}\pend\pstart\noindent\makebox[\the\longeste][l]{Selbstmörder\pwindex{Plessner, Elsa 22.\,8.\,1875 Wien – 7.\,5.\,1932 Alicante@\textsc{Plessner, Elsa} (22.\,8.\,1875 Wien – 7.\,5.\,1932 Alicante), \emph{Schriftstellerin}!Leiter der Seele@\strich\emph{Die Leiter der Seele}|pw}}\makebox[\the\longestz][l]{96}
                  \makebox[\the\longestd][l]{}\pend\pstart\noindent\makebox[\the\longeste][l]{Im Feuer geprüft\pwindex{Plessner, Elsa 22.\,8.\,1875 Wien – 7.\,5.\,1932 Alicante@\textsc{Plessner, Elsa} (22.\,8.\,1875 Wien – 7.\,5.\,1932 Alicante), \emph{Schriftstellerin}!Im Feuer geprüft@\strich\emph{Im Feuer geprüft}|pw}}\makebox[\the\longestz][l]{96}
                  \makebox[\the\longestd][l]{}\pend\pstart\noindent\makebox[\the\longeste][l]{Widerschein\pwindex{Plessner, Elsa 22.\,8.\,1875 Wien – 7.\,5.\,1932 Alicante@\textsc{Plessner, Elsa} (22.\,8.\,1875 Wien – 7.\,5.\,1932 Alicante), \emph{Schriftstellerin}!Im Widerschein@\strich\emph{Im Widerschein}|pw}}\makebox[\the\longestz][l]{96}
                  \makebox[\the\longestd][l]{}\pend\pstart\noindent\makebox[\the\longeste][l]{Am Wege\pwindex{Plessner, Elsa 22.\,8.\,1875 Wien – 7.\,5.\,1932 Alicante@\textsc{Plessner, Elsa} (22.\,8.\,1875 Wien – 7.\,5.\,1932 Alicante), \emph{Schriftstellerin}!Am Wege@\strich\emph{Am Wege}|pw}}\makebox[\the\longestz][l]{96}
                  \makebox[\the\longestd][l]{}\pend\pstart\noindent\makebox[\the\longeste][l]{Cassenchef\pwindex{Plessner, Elsa 22.\,8.\,1875 Wien – 7.\,5.\,1932 Alicante@\textsc{Plessner, Elsa} (22.\,8.\,1875 Wien – 7.\,5.\,1932 Alicante), \emph{Schriftstellerin}!Herr Cassenchef@\strich\emph{Der Herr Cassenchef}|pw}}\makebox[\the\longestz][l]{97}
                  \makebox[\the\longestd][l]{\label{T_L03728-1v}\edtext{Meran\oindex{Meran@\textbf{Meran}, \emph{Hauptstadt}|pw}}{\lemma{\textnormal{\emph{Meran}}}\Cendnote{\textnormal{Die Angabe des Entstehungsortes
                              Meran\oindex{Meran@\textbf{Meran}, \emph{Hauptstadt}|pwk} wird mit geschweifter
                           Klammer auch auf die nächsten beiden Zeilen bezogen.}}}\label{T_L03728-1}}\pend\pstart\noindent\makebox[\the\longeste][l]{Ein Brief\pwindex{Plessner, Elsa 22.\,8.\,1875 Wien – 7.\,5.\,1932 Alicante@\textsc{Plessner, Elsa} (22.\,8.\,1875 Wien – 7.\,5.\,1932 Alicante), \emph{Schriftstellerin}!Brief@\strich\emph{Ein Brief}|pw}}\makebox[\the\longestz][l]{97}
                  \makebox[\the\longestd][l]{{[}Meran\oindex{Meran@\textbf{Meran}, \emph{Hauptstadt}|pw}{]}}\pend\pstart\noindent\makebox[\the\longeste][l]{Der gläserne Käfig\pwindex{Plessner, Elsa 22.\,8.\,1875 Wien – 7.\,5.\,1932 Alicante@\textsc{Plessner, Elsa} (22.\,8.\,1875 Wien – 7.\,5.\,1932 Alicante), \emph{Schriftstellerin}!gläserne Käfig. Eine Parabel@\strich\emph{Der gläserne Käfig. Eine Parabel}|pw}}\makebox[\the\longestz][l]{97}
                  \makebox[\the\longestd][l]{{[}Meran\oindex{Meran@\textbf{Meran}, \emph{Hauptstadt}|pw}{]}}\pend\pstart\noindent\makebox[\the\longeste][l]{Meine Freundin Clotilde\pwindex{Plessner, Elsa 22.\,8.\,1875 Wien – 7.\,5.\,1932 Alicante@\textsc{Plessner, Elsa} (22.\,8.\,1875 Wien – 7.\,5.\,1932 Alicante), \emph{Schriftstellerin}!Meine Freundin Clotilde@\strich\emph{Meine Freundin Clotilde}|pw}}\makebox[\the\longestz][l]{97}
                  \makebox[\the\longestd][l]{}\pend\pstart\noindent\makebox[\the\longeste][l]{Reminiscenz\pwindex{Plessner, Elsa 22.\,8.\,1875 Wien – 7.\,5.\,1932 Alicante@\textsc{Plessner, Elsa} (22.\,8.\,1875 Wien – 7.\,5.\,1932 Alicante), \emph{Schriftstellerin}!Reminiscenz@\strich\emph{Reminiscenz}|pw}}\makebox[\the\longestz][l]{97}
                  \makebox[\the\longestd][l]{}\pend\pstart\noindent\makebox[\the\longeste][l]{Warten\pwindex{Plessner, Elsa 22.\,8.\,1875 Wien – 7.\,5.\,1932 Alicante@\textsc{Plessner, Elsa} (22.\,8.\,1875 Wien – 7.\,5.\,1932 Alicante), \emph{Schriftstellerin}!Warten. Novelle@\strich\emph{Warten. Novelle}|pw}}\makebox[\the\longestz][l]{98}
                  \makebox[\the\longestd][l]{}\pend\pstart\noindent\makebox[\the\longeste][l]{Warum\pwindex{Plessner, Elsa 22.\,8.\,1875 Wien – 7.\,5.\,1932 Alicante@\textsc{Plessner, Elsa} (22.\,8.\,1875 Wien – 7.\,5.\,1932 Alicante), \emph{Schriftstellerin}!Warum?@\strich\emph{Warum?}|pw}}\makebox[\the\longestz][l]{9\substVorne{}\textsuperscript{9}\substDazwischen{}8\substHinten{}}
                  \makebox[\the\longestd][l]{}\pend\pstart\noindent\makebox[\the\longeste][l]{Der neue Lehrer\pwindex{Plessner, Elsa 22.\,8.\,1875 Wien – 7.\,5.\,1932 Alicante@\textsc{Plessner, Elsa} (22.\,8.\,1875 Wien – 7.\,5.\,1932 Alicante), \emph{Schriftstellerin}!neue Lehrer. Novelle@\strich\emph{Der neue Lehrer. Novelle}|pw}}\makebox[\the\longestz][l]{(Juli 9\substVorne{}\textsuperscript{9}\substDazwischen{}8\substHinten{})}
                  \makebox[\the\longestd][l]{}\pend\pstart\noindent\makebox[\the\longeste][l]{(Die Ehrlosen\pwindex{Plessner, Elsa 22.\,8.\,1875 Wien – 7.\,5.\,1932 Alicante@\textsc{Plessner, Elsa} (22.\,8.\,1875 Wien – 7.\,5.\,1932 Alicante), \emph{Schriftstellerin}!Ehrlosen. Schauspiel in drei Acten@\strich\emph{Die Ehrlosen. Schauspiel in drei Acten}|pw}}\makebox[\the\longestz][l]{November 98)}
                  \makebox[\the\longestd][l]{}\pend\pstart\noindent\makebox[\the\longeste][l]{(Das erste Capitel\pwindex{Plessner, Elsa 22.\,8.\,1875 Wien – 7.\,5.\,1932 Alicante@\textsc{Plessner, Elsa} (22.\,8.\,1875 Wien – 7.\,5.\,1932 Alicante), \emph{Schriftstellerin}!erste Kapitel. Schauspiel in drei Akten@\strich\emph{Das erste Kapitel. Schauspiel in drei Akten}|pw}}\makebox[\the\longestz][l]{October 99) }
                  \makebox[\the\longestd][l]{}\pend
\pstart
           und sonst keine Zeile.\pend
           \selectlanguage{ngerman}\endnumbering\briefempfaengerindex{Schnitzler, Arthur@\textsc{Schnitzler, Arthur}!zzzPlessner, Elsa@\emph{von Elsa Plessner}!1900-10-122@{12. 10. 1900}|)be}\mylabel{L03728h}  \newcommand{\dateiname}{L03728}\newcommand{\titel}{Elsa Plessner an Arthur Schnitzler, 12. 10. 1900}\newcommand{\editorInnen}{Selma Jahnke und Martin Anton Müller}%% latex-leseansicht-abspann.tex
%% Abspann für die Leseansicht.
%% Der Schalter \ifkorrekturansicht ist bereits durch den Vorspann gesetzt.

%% latex-abspann.tex
%% Gemeinsamer Abspann für Korrekturansicht und Leseansicht.
%% Setzt den Schalter \ifkorrekturansicht voraus (gesetzt in den
%% einbindenden Dateien latex-korrekturansicht-abspann.tex bzw.
%% latex-leseansicht-abspann.tex).
%% ---------------------------------------------------------------

\normalsize

% Das esempio-Environment wird nur in der Leseansicht benötigt
\ifkorrekturansicht\else
\newenvironment{esempio}[3]%
{
    \vspace{1.5ex}
    \rlap{\underline{#1}}
    \par
    \setlength{\parindent}{0cm}
    \nopagebreak
    \leftskip=#2cm
    \rightskip=#3cm
}
{
    \par
}
\fi

\doendnotes{C}
\bigskip
\vfill

\clearpage

\footnotesize

\ifkorrekturansicht
  \lohead{\textsc{register}}
\fi

% theindex-Environment neu definieren ohne reledmac
\makeatletter
\renewenvironment{theindex}{%
  \ifkorrekturansicht
    \section*{\indexname}%
  \else
    \subsubsection*{Index der erwähnten Entitäten}%
  \fi
  \setlength{\parindent}{0pt}%
  \setlength{\parskip}{0pt plus 0.3pt}%
  \let\item\@idxitem
}{%
  \ifkorrekturansicht\clearpage\fi
}
\makeatother

\IfFileExists{\jobname-pw.ind}{\input{\jobname-pw.ind}}{}

% Quellenangabe nur in der Leseansicht
\ifkorrekturansicht\else
% Fallback-Definitionen, falls die .tex-Datei \titel etc. nicht gesetzt hat
\providecommand{\titel}{}
\providecommand{\editorInnen}{}
\providecommand{\dateiname}{\jobname}

\vspace{3cm}

\vfill

\footnotesize
\textsc{Quelle}: \titel. Herausgegeben von {\editorInnen}. In: \emph{Arthur Schnitzler: Briefwechsel mit Autorinnen und Autoren}.
 Digitale Edition, https://schnitzler-briefe.acdh.oeaw.ac.at/{\dateiname}.html (Stand \today)
\fi

\end{document}


