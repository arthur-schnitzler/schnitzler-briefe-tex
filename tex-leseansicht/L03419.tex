%% latex-leseansicht-vorspann.tex
%% Vorspann für die Leseansicht.
%% Lädt die gemeinsame Datei latex-vorspann.tex mit nicht gesetztem Schalter.

\newif\ifkorrekturansicht
\korrekturansichtfalse

\input{../tex-inputs/latex-vorspann}

\begin{center}
            \textcolor{red}{ENTWURF, NICHT FERTIG KORRIGIERT}
                      \end{center}
            
         
         \renewcommand{\erwaehntePersonen}{Personen: Hedwig Fischer, Elisabeth Maas, Olga Schnitzler}
         \renewcommand{\erwaehnteOrte}{Orte: Berlin, Edmund-Weiß-Gasse, Hotel Savoy, Noordwijk, Wien, XVIII., Währing}
         \renewcommand{\erwaehnteWerke}{Werke: Der einsame Weg. Schauspiel in fünf Akten}
               \section[Felix Salten u. a. an Arthur Schnitzler, 19. 4. 1906]{ Felix Salten u. a. an Arthur Schnitzler, 19. 4. 1906}\nopagebreak\mylabel{v}\rehead{ }\begin{ledgroupsized}[t]{13cm}\normalsize\beginnumbering \toendnotes[C]{\smallbreak\pagebreak[2]} \Standort{CUL, Schnitzler, B 89, B 1.}
\physDesc{Postkarte, 766 Zeichen
\newline{}Handschrift Felix Salten: schwarze Tinte, lateinische Kurrent\newline{}Handschrift Ottilie Salten: schwarze Tinte\newline{}Handschrift Otto Brahm: schwarze Tinte, lateinische Kurrent\newline{}Handschrift Clara Jonas: schwarze Tinte, lateinische Kurrent\newline{}Handschrift Emil Heilbut: schwarze Tinte, lateinische Kurrent\newline{}Handschrift Samuel Fischer: schwarze Tinte, lateinische Kurrent\newline{}Handschrift Hedwig Fischer: schwarze Tinte, deutsche Kurrent\newline{}Handschrift Elisabeth Maas: Bleistift, lateinische Kurrent
\newline{}Versand: Stempel: »\nobreak{}\oindex{Berlin@\textbf{Berlin}|pwk}Berlin \textcolor{gray}{N.W.}, 20. 4. 06, 5–6V\nobreak{}«.  
\newline{}Ordnung: mit Bleistift von unbekannter Hand nummeriert:
                                    »210« }\toendnotes[C]{\smallbreak}\pstart{}\textcolor{gray}{\textbf{SAVOY-HOTEL, BERLIN N. W.\oindex{Hotel Savoy@\textbf{Hotel Savoy}|pw}}}\pend{}{\bigskip}\pstart{}{\pb}Herrn D\textsuperscript{r} Arthur Schnitzler\pend{}\pstart{}Wien XVIII.\oindex{XVIII., Waehring@\textbf{XVIII., Währing}|pw}\pend{}\pstart{}Spöttelgasse 7\oindex{Edmund-Weiss-Gasse@\textbf{Edmund-Weiß-Gasse}|pw}\pend{}{\bigskip}\pstart
           {\pb}Donnerstag{ }Abds. nach dem »Einsamen
                  Weg\pwindex{Schnitzler, Arthur 15.05.1862 – 21.10.1931@\textsc{Schnitzler, Arthur} (15.05.1862 – 21.10.1931), \emph{Schriftsteller, Mediziner}!einsame Weg. Schauspiel in fuenf Akten1904@\strich\emph{Der einsame Weg. Schauspiel in fünf Akten} {[}1904{]}|pw}«\pend
           \pstart
           Wir sind alle ziemlich kaput – aber auf eine edle Weise. (Es gibt kaum eine
               vornehmere Manier, den Leuten die Lebensfreude abzugewöhnen, als dieses schöne Stück\pwindex{Schnitzler, Arthur 15.05.1862 – 21.10.1931@\textsc{Schnitzler, Arthur} (15.05.1862 – 21.10.1931), \emph{Schriftsteller, Mediziner}!einsame Weg. Schauspiel in fuenf Akten1904@\strich\emph{Der einsame Weg. Schauspiel in fünf Akten} {[}1904{]}|pwv})\pend
           \pstart Viele herzliche Grüße Ihnen u. Olga\pwindex{Schnitzler, Olga 17.01.1882 – 13.01.1970@\textsc{Schnitzler, Olga} (17.01.1882 – 13.01.1970), \emph{Schauspielerin, Sängerin}|pw}. Ihr
                  \spacefill\mbox{Salten}\pend{}\pstart {[}hs. Ottilie Salten:{]} \spacefill\mbox{Otti}\pend{}\pstart
           \noindent{}{[}hs. Brahm:{]} Trotz einer miserabeln Aufführung hat mir dieses Werk\pwindex{Schnitzler, Arthur 15.05.1862 – 21.10.1931@\textsc{Schnitzler, Arthur} (15.05.1862 – 21.10.1931), \emph{Schriftsteller, Mediziner}!einsame Weg. Schauspiel in fuenf Akten1904@\strich\emph{Der einsame Weg. Schauspiel in fünf Akten} {[}1904{]}|pwv} wieder sehr gefallen.\pend
           \pstart Herzlich \spacefill\mbox{OBrahm}\pend{}\pstart
           \noindent{}{[}hs. Jonas:{]} Es war doch sehr schön + alles Uebrige werde ich Ihnen
               den Sommer in Nordwijk\oindex{Noordwijk@\textbf{Noordwijk}|pw} sagen.\pend
           \pstart Herzlichste Grüße Ihnen + Ihrer lieben Frau\pwindex{Schnitzler, Olga 17.01.1882 – 13.01.1970@\textsc{Schnitzler, Olga} (17.01.1882 – 13.01.1970), \emph{Schauspielerin, Sängerin}|pwv}. \spacefill\mbox{Clara Jonas}\pend{}\pstart
           \noindent{}{[}hs. Heilbut:{]} Von Ihrem Werk\pwindex{Schnitzler, Arthur 15.05.1862 – 21.10.1931@\textsc{Schnitzler, Arthur} (15.05.1862 – 21.10.1931), \emph{Schriftsteller, Mediziner}!einsame Weg. Schauspiel in fuenf Akten1904@\strich\emph{Der einsame Weg. Schauspiel in fünf Akten} {[}1904{]}|pwv} tiefergriffen grüsst Sie herzlich Ihr
                  \spacefill\mbox{Heilbut}\pend
           \pstart
           \noindent{}{[}hs. Samuel Fischer:{]} Vielen Dank und herzlichen Gruß\pend
           \pstart
           von Ihrem \spacefill\mbox{S. Fischer}.\pend
           \pstart
           \noindent{}{[}hs. Hedwig Fischer:{]} Der »Einſame Weg\pwindex{Schnitzler, Arthur 15.05.1862 – 21.10.1931@\textsc{Schnitzler, Arthur} (15.05.1862 – 21.10.1931), \emph{Schriftsteller, Mediziner}!einsame Weg. Schauspiel in fuenf Akten1904@\strich\emph{Der einsame Weg. Schauspiel in fünf Akten} {[}1904{]}|pw}{[}«{]} hat eine herrliche Auferſtehung gefeiert u wir denken Ihrer in
               Dankbarkeit.\pend
           \pstart Ihre \spacefill\mbox{Hedwig Fischer}\pend{}\pstart
           \noindent{}Herzlichen Gruss\pend
           \pstart {[}hs. Maas:{]} \spacefill\mbox{Lili Jonas}.\pend{}
         
         \endnumbering\mylabel{h}\end{ledgroupsized}\begin{anhang}\end{anhang}\newcommand{\dateiname}{L03419}\newcommand{\titel}{Felix Salten u. a. an Arthur Schnitzler, 19. 4. 1906}\newcommand{\editorInnen}{Martin Anton Müller und Laura Untner}%% latex-leseansicht-abspann.tex
%% Abspann für die Leseansicht.
%% Der Schalter \ifkorrekturansicht ist bereits durch den Vorspann gesetzt.

%% latex-abspann.tex
%% Gemeinsamer Abspann für Korrekturansicht und Leseansicht.
%% Setzt den Schalter \ifkorrekturansicht voraus (gesetzt in den
%% einbindenden Dateien latex-korrekturansicht-abspann.tex bzw.
%% latex-leseansicht-abspann.tex).
%% ---------------------------------------------------------------

\normalsize

% Das esempio-Environment wird nur in der Leseansicht benötigt
\ifkorrekturansicht\else
\newenvironment{esempio}[3]%
{
    \vspace{1.5ex}
    \rlap{\underline{#1}}
    \par
    \setlength{\parindent}{0cm}
    \nopagebreak
    \leftskip=#2cm
    \rightskip=#3cm
}
{
    \par
}
\fi

\doendnotes{C}
\bigskip
\vfill

\clearpage

\footnotesize

\ifkorrekturansicht
  \lohead{\textsc{register}}
\fi

% theindex-Environment neu definieren ohne reledmac
\makeatletter
\renewenvironment{theindex}{%
  \ifkorrekturansicht
    \section*{\indexname}%
  \else
    \subsubsection*{Index der erwähnten Entitäten}%
  \fi
  \setlength{\parindent}{0pt}%
  \setlength{\parskip}{0pt plus 0.3pt}%
  \let\item\@idxitem
}{%
  \ifkorrekturansicht\clearpage\fi
}
\makeatother

\IfFileExists{\jobname-pw.ind}{\input{\jobname-pw.ind}}{}

% Quellenangabe nur in der Leseansicht
\ifkorrekturansicht\else
% Fallback-Definitionen, falls die .tex-Datei \titel etc. nicht gesetzt hat
\providecommand{\titel}{}
\providecommand{\editorInnen}{}
\providecommand{\dateiname}{\jobname}

\vspace{3cm}

\vfill

\footnotesize
\textsc{Quelle}: \titel. Herausgegeben von {\editorInnen}. In: \emph{Arthur Schnitzler: Briefwechsel mit Autorinnen und Autoren}.
 Digitale Edition, https://schnitzler-briefe.acdh.oeaw.ac.at/{\dateiname}.html (Stand \today)
\fi

\end{document}


      