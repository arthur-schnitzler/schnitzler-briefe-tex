%% latex-korrekturansicht-vorspann.tex
%% Vorspann für die Korrekturansicht.
%% Lädt die gemeinsame Datei latex-vorspann.tex mit gesetztem Schalter.

\newif\ifkorrekturansicht
\korrekturansichttrue

\input{../tex-inputs/latex-vorspann}


\section[ Felix Salten u. a. an Arthur Schnitzler, 19. 4. 1906]{L03419 Felix Salten u. a. an Arthur Schnitzler, 19. 4. 1906}
\nopagebreak\mylabel{L03419v}
\rehead{ }\normalsize\beginnumbering\briefempfaengerindex{Schnitzler, Arthur@\textsc{Schnitzler, Arthur}!zzzMaas, Elisabeth@\emph{von Elisabeth Maas}!1906-04-191@{19. 4. 1906}|(be}\briefempfaengerindex{Schnitzler, Arthur@\textsc{Schnitzler, Arthur}!zzzFischer, Hedwig@\emph{von Hedwig Fischer}!1906-04-191@{19. 4. 1906}|(be}\briefempfaengerindex{Schnitzler, Arthur@\textsc{Schnitzler, Arthur}!zzzFischer, Samuel@\emph{von Samuel Fischer}!1906-04-191@{19. 4. 1906}|(be}\briefempfaengerindex{Schnitzler, Arthur@\textsc{Schnitzler, Arthur}!zzzHeilbut, Emil@\emph{von Emil Heilbut}!1906-04-191@{19. 4. 1906}|(be}\briefempfaengerindex{Schnitzler, Arthur@\textsc{Schnitzler, Arthur}!zzzJonas, Clara@\emph{von Clara Jonas}!1906-04-191@{19. 4. 1906}|(be}\briefempfaengerindex{Schnitzler, Arthur@\textsc{Schnitzler, Arthur}!zzzBrahm, Otto@\emph{von Otto Brahm}!1906-04-191@{19. 4. 1906}|(be}\briefempfaengerindex{Schnitzler, Arthur@\textsc{Schnitzler, Arthur}!zzzSalten, Ottilie@\emph{von Ottilie Salten}!1906-04-191@{19. 4. 1906}|(be}\briefempfaengerindex{Schnitzler, Arthur@\textsc{Schnitzler, Arthur}!zzzSalten, Felix@\emph{von Felix Salten}!1906-04-191@{19. 4. 1906}|(be}
\toendnotes[C]{\smallbreak\pagebreak[2]}\Standort{CUL, Schnitzler, B 89, B 1.}
\physDesc{Postkarte, 766 Zeichen
\newline{}Handschrift Felix Salten: schwarze Tinte, lateinische Kurrent
\newline{}Handschrift Ottilie Salten: schwarze Tinte
\newline{}Handschrift Otto Brahm: schwarze Tinte, lateinische Kurrent
\newline{}Handschrift Clara Jonas: schwarze Tinte, lateinische Kurrent
\newline{}Handschrift Emil Heilbut: schwarze Tinte, lateinische Kurrent
\newline{}Handschrift Samuel Fischer: schwarze Tinte, lateinische Kurrent
\newline{}Handschrift Hedwig Fischer: schwarze Tinte, deutsche Kurrent
\newline{}Handschrift Elisabeth Maas: Bleistift, lateinische Kurrent
\newline{}Versand: Stempel: »\nobreak{}\oindex{Berlin@\textbf{Berlin}, \emph{P.PPLC}|pwk}Berlin\textcolor{gray}{,}
                                          N\textcolor{gray}{.} W\textcolor{gray}{.} 7, 20. 4. 06, 5–6 V.\nobreak{}«.  
\newline{}Ordnung: mit Bleistift von unbekannter Hand nummeriert: »210« }\toendnotes[C]{\smallbreak}\pstart{}\textcolor{gray}{\textbf{SAVOY-HOTEL, BERLIN N. W.\oindex{Hotel Savoy [Berlin]@\textbf{Hotel Savoy [Berlin]}, \emph{Hotel (K.HTL)}|pw}}}\pend{}{\bigskip}\pstart{}{\pb}Herrn D\textsuperscript{r} Arthur Schnitzler\pend{}\pstart{}Wien XVIII.\oindex{XVIII., Waehring@\textbf{XVIII., Währing}, \emph{A.ADM3}|pw}\pend{}\pstart{}Spöttelgasse 7\oindex{Edmund-Weiss-Gasse 7@\textbf{Edmund-Weiß-Gasse 7}, \emph{Wohngebäude (K.WHS)}|pw}\pend{}{\bigskip}\vspace{1em}
\pstart
           {\pb}Donnerstag{ }Abds. nach dem »Einsamen
                  Weg\pwindex{einsame Weg. Schauspiel in fuenf Akten@\emph{Der einsame Weg. Schauspiel in fünf Akten}|pw}«.\pend
           \vspace{0.5em}
\pstart
           Wir sind alle ziemlich kaput – aber auf eine edle Weise. (Es gibt kaum eine
               vornehmere Manier, den Leuten die Lebensfreude abzugewöhnen, als dieses schöne Stück\pwindex{einsame Weg. Schauspiel in fuenf Akten@\emph{Der einsame Weg. Schauspiel in fünf Akten}|pwv})\pend
           \pstart Viele herzliche Grüße Ihnen u. Olga\pwindex{Schnitzler, Olga 17.01.1882 – 13.01.1970@\textsc{Schnitzler, Olga} (17.01.1882 – 13.01.1970), \emph{Schauspieler/Schauspielerin, Sänger/Sängerin}|pw}. Ihr
                  \spacefill\mbox{Salten}\pend{}\selectlanguage{ngerman}\vspace{1em}
\pstart
           \noindent{}{[}hs. :{]} \spacefill\mbox{Otti}\pend
           \selectlanguage{ngerman}\vspace{1em}
\pstart
           \noindent{}{[}hs. :{]} Trotz einer miserabeln \label{K_L03419-1v}\edtext{Aufführung}{\lemma{\textnormal{\emph{Aufführung}}}\Cendnote{\textnormal{Am
                     19. 4. 1906 wurde \emph{Der einsame Weg}\pwindex{einsame Weg. Schauspiel in fuenf Akten@\emph{Der einsame Weg. Schauspiel in fünf Akten}|pwk} vom \emph{Lessing-Theater}\orgindex{Lessing-Theater@Lessing-Theater|pwk} in Berlin\oindex{Berlin@\textbf{Berlin}, \emph{P.PPLC}|pwk} als
                  Neuaufnahme gegeben. Hintergrund bildete das bevorstehende Gastspiel in Wien\oindex{Wien@\textbf{Wien}, \emph{A.ADM2}|pwk}, für das das Stück\pwindex{einsame Weg. Schauspiel in fuenf Akten@\emph{Der einsame Weg. Schauspiel in fünf Akten}|pwkv} fix gesetzt war. Die Rolle von Julian Fichtner\pwindex{einsame Weg. Schauspiel in fuenf Akten@\emph{Der einsame Weg. Schauspiel in fünf Akten}|pwkv} wurde aber
                  nicht mehr wie bei der Uraufführung von Rudolf
                     Rittner\pwindex{Rittner, Rudolf 30.06.1869 – 04.02.1943@\textsc{Rittner, Rudolf} (30.06.1869 – 04.02.1943), \emph{Theaterleiter/Theaterleiterin, Schauspieler/Schauspielerin}|pwk}, sondern von Emanuel
                     Reicher\pwindex{Reicher, Emanuel 18.06.1849 – 15.05.1924@\textsc{Reicher, Emanuel} (18.06.1849 – 15.05.1924), \emph{Schauspieler/Schauspielerin}|pwk} gespielt. Das führte in den folgenden Wochen zu verschiedenen
                  (erfolglosen) Versuchen, Rittner\pwindex{Rittner, Rudolf 30.06.1869 – 04.02.1943@\textsc{Rittner, Rudolf} (30.06.1869 – 04.02.1943), \emph{Theaterleiter/Theaterleiterin, Schauspieler/Schauspielerin}|pwk} zur
                  Rückkehr zu bewegen, vgl. 
                     \emph{Der Briefwechsel Arthur Schnitzler – Otto Brahm}.
                     Vollständige Ausgabe. Herausgegeben, eingeleitet und erläutert von Oskar
                     Seidlin. Tübingen: \emph{Niemeyer}{ }1975, S. 225–228, Felix Salten an Arthur Schnitzler, 21. 4. [1906], Felix Salten an Arthur Schnitzler, 22. – 23. 4. 1906.}}}\label{K_L03419-1} hat
               mir dieses Werk\pwindex{einsame Weg. Schauspiel in fuenf Akten@\emph{Der einsame Weg. Schauspiel in fünf Akten}|pwv} wieder sehr
               gefallen. Herzlich \spacefill\mbox{OBrahm}\pend
           \selectlanguage{ngerman}\vspace{1em}
\pstart
           \noindent{}{[}hs. :{]} Es war doch sehr schön + alles Uebrige werde ich Ihnen
                  \label{K_L03419-2v}\edtext{den Sommer in Nordwijk\oindex{Noordwijk@\textbf{Noordwijk}, \emph{A.ADM2}|pw}}{\lemma{\textnormal{\emph{den Sommer in Nordwijk}}}\Cendnote{\textnormal{Schnitzler plante bis in den Juni (vgl. Arthur Schnitzler an Hermann Bahr, 24. – 25. 6. 1906) von Marienlyst\oindex{Marienlyst@\textbf{Marienlyst}, \emph{S.EST}|pwk} an den Strand von Noordwijk\oindex{Noordwijk@\textbf{Noordwijk}, \emph{A.ADM2}|pwk} zu übersiedeln. Dazu kam es nicht. }}}\label{K_L03419-2} sagen. Herzlichste
               Grüße Ihnen + Ihrer lieben Frau\pwindex{Schnitzler, Olga 17.01.1882 – 13.01.1970@\textsc{Schnitzler, Olga} (17.01.1882 – 13.01.1970), \emph{Schauspieler/Schauspielerin, Sänger/Sängerin}|pwv}. \spacefill\mbox{Clara Jonas}\pend
           \selectlanguage{ngerman}\vspace{1em}
\pstart
           \noindent{}{[}hs. :{]} Von Ihrem Werk\pwindex{einsame Weg. Schauspiel in fuenf Akten@\emph{Der einsame Weg. Schauspiel in fünf Akten}|pwv} tiefergriffen grüsst Sie herzlich Ihr
                  \spacefill\mbox{Heilbut}\pend
           \selectlanguage{ngerman}\vspace{1em}
\pstart
           \noindent{}{[}hs. :{]} Vielen Dank und herzlichen Gruß von Ihrem \spacefill\mbox{S.
                  Fischer.}\pend
           \selectlanguage{ngerman}\vspace{1em}
\pstart
           \noindent{}{[}hs. :{]} Der »Einſame Weg\pwindex{einsame Weg. Schauspiel in fuenf Akten@\emph{Der einsame Weg. Schauspiel in fünf Akten}|pw}{[}«{]} hat eine herrliche \label{K_L03419-3v}\edtext{Auferſtehung}{\lemma{\textnormal{\emph{Auferſtehung}}}\Cendnote{\textnormal{Das
                     Stück\pwindex{einsame Weg. Schauspiel in fuenf Akten@\emph{Der einsame Weg. Schauspiel in fünf Akten}|pwkv} war bereits 1904 am Deutschen Theater
                     Berlin\oindex{Deutsches Theater Berlin@\textbf{Deutsches Theater Berlin}, \emph{Theater (K.THE)}|pwk} uraufgeführt worden.}}}\label{K_L03419-3} gefeiert u wir denken Ihrer in
               Dankbarkeit. Ihre \spacefill\mbox{Hedwig Fischer}\pend
           \selectlanguage{ngerman}\vspace{1em}
\pstart
           \noindent{}{[}hs. :{]} Herzlichen Gruss \spacefill\mbox{Lili Jonas.}\pend
           \selectlanguage{ngerman}\endnumbering\briefempfaengerindex{Schnitzler, Arthur@\textsc{Schnitzler, Arthur}!zzzMaas, Elisabeth@\emph{von Elisabeth Maas}!1906-04-191@{19. 4. 1906}|)be}\briefempfaengerindex{Schnitzler, Arthur@\textsc{Schnitzler, Arthur}!zzzFischer, Hedwig@\emph{von Hedwig Fischer}!1906-04-191@{19. 4. 1906}|)be}\briefempfaengerindex{Schnitzler, Arthur@\textsc{Schnitzler, Arthur}!zzzFischer, Samuel@\emph{von Samuel Fischer}!1906-04-191@{19. 4. 1906}|)be}\briefempfaengerindex{Schnitzler, Arthur@\textsc{Schnitzler, Arthur}!zzzHeilbut, Emil@\emph{von Emil Heilbut}!1906-04-191@{19. 4. 1906}|)be}\briefempfaengerindex{Schnitzler, Arthur@\textsc{Schnitzler, Arthur}!zzzJonas, Clara@\emph{von Clara Jonas}!1906-04-191@{19. 4. 1906}|)be}\briefempfaengerindex{Schnitzler, Arthur@\textsc{Schnitzler, Arthur}!zzzBrahm, Otto@\emph{von Otto Brahm}!1906-04-191@{19. 4. 1906}|)be}\briefempfaengerindex{Schnitzler, Arthur@\textsc{Schnitzler, Arthur}!zzzSalten, Ottilie@\emph{von Ottilie Salten}!1906-04-191@{19. 4. 1906}|)be}\briefempfaengerindex{Schnitzler, Arthur@\textsc{Schnitzler, Arthur}!zzzSalten, Felix@\emph{von Felix Salten}!1906-04-191@{19. 4. 1906}|)be}\mylabel{L03419h}  \normalsize

\doendnotes{C}
\bigskip
\vfill

\clearpage

\footnotesize

\lohead{\textsc{register}}

% Definiere theindex-Environment komplett neu ohne reledmac
\makeatletter
\renewenvironment{theindex}{%
  \section*{\indexname}%
  \setlength{\parindent}{0pt}%
  \setlength{\parskip}{0pt plus 0.3pt}%
  \let\item\@idxitem
}{%
  \clearpage
}
\makeatother

\IfFileExists{\jobname-pw.ind}{\input{\jobname-pw.ind}}{}

\end{document}

      