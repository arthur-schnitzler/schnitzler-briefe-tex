%% latex-korrekturansicht-vorspann.tex
%% Vorspann für die Korrekturansicht.
%% Lädt die gemeinsame Datei latex-vorspann.tex mit gesetztem Schalter.

\newif\ifkorrekturansicht
\korrekturansichttrue

\input{../tex-inputs/latex-vorspann}


\section[Gabriel Beer-Hofmann an Arthur Schnitzler, 12. 10. 1926]{L02479 Gabriel Beer-Hofmann an Arthur Schnitzler, 12. 10. 1926}
\nopagebreak\mylabel{L02479v}
\rehead{ }\normalsize\beginnumbering\briefempfaengerindex{Schnitzler, Arthur@\textsc{Schnitzler, Arthur}!zzzBeer-Hofmann, Gabriel@\emph{von Gabriel Beer-Hofmann}!1926-10-121@{12. 10. 1926}|(be}
\toendnotes[C]{\smallbreak\pagebreak[2]}\Standort{CUL, Schnitzler, B 8.}
\physDesc{Brief, 1 Blatt, 2 Seiten, 1015 Zeichen
\newline{}Handschrift: blaue Tinte, lateinische Kurrent
\newline{}Schnitzler: mit Bleistift beschriftet: »Bab BH« 
\newline{}Ordnung: mit Bleistift von unbekannter Hand nummeriert:
                                    »272« }
\buchAbdrucke{\weitereDrucke{Arthur Schnitzler, Richard Beer-Hofmann: \emph{Briefwechsel 1891–1931}. Wien, Zürich: \emph{Europaverlag} 1992, S. 229.} }
\pstart
           \textcolor{gray}{\textbf{{\pb}Am Ausgang des Hauptbahnhof\oindex{Hauptbahnhof Hamburg@\textbf{Hauptbahnhof Hamburg}, \emph{Bahnhofsgebäude (K.BHF)}|pw}es}}\hfill \textcolor{gray}{\textbf{Kirchenallee Nr. 35–36\oindex{Kirchenallee@\textbf{Kirchenallee}, \emph{Straße (K.STR)}|pw}, gegenüber}}\pend
           
\pstart
           {\dotssix}Ankunftsseite{\dotssix}\hfill {\dots}Ausgang Hauptbahnhof{\dots}\pend
           
\pstart
           \centering{}\textcolor{gray}{\textbf{Hotel Reichshof\oindex{Hotel Reichshof@\textbf{Hotel Reichshof}, \emph{Hotel (K.HTL)}|pw} Hamburg}}\pend
           
\pstart
           \centering{}\textcolor{gray}{\textbf{Direktion: Emil Langer\pwindex{Langer, Anton-Emil 1864 – 1928@\textsc{Langer, Anton-Emil} (1864 – 1928), \emph{Hotelbesitzer/Hotelbesitzerin}|pw}}}\pend
           
\pstart
           \centering{}\textcolor{gray}{\textbf{Mehr als 300 Zimmer und Salons}}\pend
           
\pstart
           \centering{}\textcolor{gray}{\textbf{50 Badezimmer}}\pend
           
\pstart
           \textcolor{gray}{\textbf{Telegramm-Adresse:}}\hfill \textcolor{gray}{\textbf{Fernsprecher:}}\pend
           
\pstart
           \textcolor{gray}{\textbf{Reichshof Hamburg}}\hfill \textcolor{gray}{\textbf{Alster 870, 2836, 2837}}\pend
           
\pstart
           \centering{}\textcolor{gray}{\textbf{Im Frühstücks-Saal: Grosses und Abendessen nach der
                     Karte}}\pend
           
\pstart
           \centering{}\textcolor{gray}{\textbf{Kachel-Waschtische mit fliessendem kalten und warmen Wasser
                     in allen Zimmern}}\pend
           
\pstart
           \textcolor{gray}{\textbf{Fernsprecher in allen Zimmern}}\pend
           
\pstart
           \textcolor{gray}{\textbf{Auto-Unterstand für 20 Automobile}}\pend
           
\pstart
           \textcolor{gray}{\textbf{Rasier- und Frisier-Salon im Hause}}\pend
           
\pstart
           \raggedleft{}\textcolor{gray}{\textbf{Hamburg\oindex{Hamburg@\textbf{Hamburg}, \emph{P.PPLA}|pw}, den}}{ }12. Oktober \textcolor{gray}{\textbf{192}}6\pend
           
\pstart
           \raggedleft{}\textcolor{gray}{\textbf{Kirchenallee Nr. 35–36\oindex{Kirchenallee@\textbf{Kirchenallee}, \emph{Straße (K.STR)}|pw}}}\pend
           
\pstart{}Verehrter, lieber Doktor Schnitzler!\pend\vspace{0.5em}
\pstart
           Wie sehr es mir Wunsch und Bedürfniss gewesen wäre, mich von Ihnen zu verabschieden,
               so war es mir doch schliesslich zeitlich unmöglich. Trotz aller Vorbereitungen war
               meine Abreise doch überstürzt. –\pend
           
\pstart
           Ich hätte Sie, lieber Herr Doktor, wie auch ganz besonders gerne Lily\pwindex{Cappellini, Lili 13.09.1909 – 26.07.1928@\textsc{Cappellini, Lili} (13.09.1909 – 26.07.1928)|pw} noch einmal gesehen. –\pend
           
\pstart
           Nach ein paar Tagen Berlin\oindex{Berlin@\textbf{Berlin}, \emph{P.PPLC}|pw} und drei kalten und
               verregneten Tagen in Hamburg\oindex{Hamburg@\textbf{Hamburg}, \emph{P.PPLA}|pw}, fahre ich morgen
               mit der »Thuringia« nach New-York\oindex{New York City@\textbf{New York City}, \emph{P.PPL}|pw}.\pend
           
\pstart
           Zwölf Tage Seefahrt – wie sehr habe ich mir dies – seit Jahren – gewünscht und jetzt
               wird es Erfüllung – wie ein Traum zauberhaft und unglaublich –\pend
           
\pstart
           Ich habe leider nicht die Adresse (Venedig\oindex{Venedig@\textbf{Venedig}, \emph{P.PPLA}|pw}) von
                  Lily\pwindex{Cappellini, Lili 13.09.1909 – 26.07.1928@\textsc{Cappellini, Lili} (13.09.1909 – 26.07.1928)|pw}.\pend
           
\pstart
           Es ist doch nicht unbescheiden, wenn ich Sie, lieber Herr Doktor {\pb}bitte, Lily\pwindex{Cappellini, Lili 13.09.1909 – 26.07.1928@\textsc{Cappellini, Lili} (13.09.1909 – 26.07.1928)|pw} sehr schön und herzlich von mir zu grüssen. Ich will ihr
               gleich von drüben schreiben.\pend
           
\pstart
           Inzwischen, Ihnen, lieber Doktor Schnitzler und der lieben Lily\pwindex{Cappellini, Lili 13.09.1909 – 26.07.1928@\textsc{Cappellini, Lili} (13.09.1909 – 26.07.1928)|pw}, alle guten Wünsche für die nächste Zeit\pend
           
\pstart
           von ganzem Herzen{\\[\baselineskip]}Ihr{\\[\baselineskip]}\spacefill\mbox{Gabriel Beer-Hofmann}\pend
           \leftskip=0em{}\selectlanguage{ngerman}\endnumbering\briefempfaengerindex{Schnitzler, Arthur@\textsc{Schnitzler, Arthur}!zzzBeer-Hofmann, Gabriel@\emph{von Gabriel Beer-Hofmann}!1926-10-121@{12. 10. 1926}|)be}\mylabel{L02479h}  \normalsize

\doendnotes{C}
\bigskip
\vfill

\clearpage

\footnotesize

\lohead{\textsc{register}}

% Definiere theindex-Environment komplett neu ohne reledmac
\makeatletter
\renewenvironment{theindex}{%
  \section*{\indexname}%
  \setlength{\parindent}{0pt}%
  \setlength{\parskip}{0pt plus 0.3pt}%
  \let\item\@idxitem
}{%
  \clearpage
}
\makeatother

\IfFileExists{\jobname-pw.ind}{\input{\jobname-pw.ind}}{}

\end{document}

      