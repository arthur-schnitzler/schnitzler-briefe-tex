%% latex-leseansicht-vorspann.tex
%% Vorspann für die Leseansicht.
%% Lädt die gemeinsame Datei latex-vorspann.tex mit nicht gesetztem Schalter.

\newif\ifkorrekturansicht
\korrekturansichtfalse

\input{../tex-inputs/latex-vorspann}


         
         \renewcommand{\erwaehntePersonen}{Personen: Richard Beer-Hofmann, Anton-Emil Langer, Lili Schnitzler}
         \renewcommand{\erwaehnteOrte}{Orte: Berlin, Hamburg, Hauptbahnhof, Hotel Reichshof, Kirchenallee, New York City, Venedig, Wien}
         \renewcommand{\erwaehnteWerke}{
               \section[Gabriel Beer-Hofmann an Arthur Schnitzler, 12. 10. 1926]{ Gabriel Beer-Hofmann an Arthur Schnitzler,
                    12. 10. 1926}\nopagebreak\mylabel{v}\rehead{ }\begin{ledgroupsized}[t]{13cm}\normalsize\beginnumbering \toendnotes[C]{\smallbreak\pagebreak[2]} \Standort{CUL, Schnitzler, B 8.}
\physDesc{Brief, 1 Blatt, 2 Seiten
\newline{}Handschrift: blaue Tinte, lateinische Kurrent
\newline{}Schnitzler: mit Bleistift beschriftet: »Bab BH« \newline{}Ordnung: mit Bleistift von unbekannter Hand nummeriert:
                                        »272« }\buchAbdrucke{\weitereDrucke{Arthur Schnitzler, Richard Beer-Hofmann: \emph{Briefwechsel 1891–1931}. Hg. Konstanze Fliedl. Wien, Zürich: \emph{Europaverlag} 1992, S. 229.} }\pstart
           \noindent{}\textcolor{gray}{\textbf{{\pb}Am Ausgang des
                                    Hauptbahnhof\oindex{Hauptbahnhof@\textbf{Hauptbahnhof}|pw}es}}\hfill \textcolor{gray}{\textbf{Kirchenallee Nr. 35–36\oindex{Kirchenallee@\textbf{Kirchenallee}|pw},
                                gegenüber}}\pend
           \pstart
           {\dotssix}Ankunftsseite{\dotssix}\hfill {\dots}Ausgang Hauptbahnhof{\dots}\pend
           \pstart
           \centering{}\textcolor{gray}{\textbf{Hotel Reichshof\oindex{Hotel Reichshof@\textbf{Hotel Reichshof}|pw} Hamburg}}\pend
           \pstart
           \noindent{}\centering{}\textcolor{gray}{\textbf{Direktion: Emil
                                Langer\pwindex{Langer, Anton-Emil 1864 – 1928@\textsc{Langer, Anton-Emil} (1864 – 1928), \emph{Hotelbesitzer}|pw}}}\pend
           \pstart
           \noindent{}\centering{}\textcolor{gray}{\textbf{Mehr als 300 Zimmer und Salons}}\pend
           \pstart
           \noindent{}\centering{}\textcolor{gray}{\textbf{50 Badezimmer}}\pend
           \pstart
           \noindent{}\textcolor{gray}{\textbf{Telegramm-Adresse:}}\hfill \textcolor{gray}{\textbf{Fernsprecher:}}\pend
           \pstart
           \textcolor{gray}{\textbf{Reichshof Hamburg}}\hfill \textcolor{gray}{\textbf{Alster 870, 2836, 2837}}\pend
           \pstart
           \centering{}\textcolor{gray}{\textbf{Im Frühstücks-Saal: Grosses und Abendessen nach der
                            Karte}}\pend
           \pstart
           \noindent{}\centering{}\textcolor{gray}{\textbf{Kachel-Waschtische mit fliessendem kalten und warmen
                            Wasser in allen Zimmern}}\pend
           \pstart
           \noindent{}\textcolor{gray}{\textbf{Fernsprecher in allen Zimmern}}\pend
           \pstart
           \textcolor{gray}{\textbf{Auto-Unterstand für 20 Automobile}}\pend
           \pstart
           \textcolor{gray}{\textbf{Rasier- und Frisier-Salon im Hause}}\pend
           \pstart
           \raggedleft{}\textcolor{gray}{\textbf{Hamburg\oindex{Hamburg@\textbf{Hamburg}|pw}, den}}{ }12. Oktober \textcolor{gray}{\textbf{192}}6\pend
           \pstart
           \raggedleft{}\textcolor{gray}{\textbf{Kirchenallee Nr. 35–36\oindex{Kirchenallee@\textbf{Kirchenallee}|pw}}}\pend
           \pstart{}Verehrter, lieber Doktor Schnitzler!\pend\pstart
           Wie sehr es mir Wunsch und Bedürfniss gewesen wäre, mich von Ihnen zu
                    verabschieden, so war es mir doch schliesslich zeitlich unmöglich. Trotz aller
                    Vorbereitungen war meine Abreise doch überstürzt. –\pend
           \pstart
           Ich hätte Sie, lieber Herr Doktor, wie auch ganz besonders gerne Lily\pwindex{Schnitzler, Lili 13.09.1909 – 26.07.1928@\textsc{Schnitzler, Lili} (13.09.1909 – 26.07.1928)|pw} noch einmal gesehen. –\pend
           \pstart
           Nach ein paar Tagen Berlin\oindex{Berlin@\textbf{Berlin}|pw} und drei kalten und
                    verregneten Tagen in Hamburg\oindex{Hamburg@\textbf{Hamburg}|pw}, fahre ich morgen
                    mit der »Thuringia« nach New-York\oindex{New York City@\textbf{New York City}|pw}.\pend
           \pstart
           Zwölf Tage Seefahrt – wie sehr habe ich mir dies – seit Jahren – gewünscht und
                    jetzt wird es Erfüllung – wie ein Traum zauberhaft und unglaublich –\pend
           \pstart
           Ich habe leider nicht die Adresse (Venedig\oindex{Venedig@\textbf{Venedig}|pw}) von
                        Lily\pwindex{Schnitzler, Lili 13.09.1909 – 26.07.1928@\textsc{Schnitzler, Lili} (13.09.1909 – 26.07.1928)|pw}.\pend
           \pstart
           Es ist doch nicht unbescheiden, wenn ich Sie, lieber Herr Doktor {\pb}bitte, Lily\pwindex{Schnitzler, Lili 13.09.1909 – 26.07.1928@\textsc{Schnitzler, Lili} (13.09.1909 – 26.07.1928)|pw} sehr schön und herzlich von mir zu grüssen. Ich will
                    ihr gleich von drüben schreiben.\pend
           \pstart
           Inzwischen, Ihnen, lieber Doktor Schnitzler und der lieben Lily\pwindex{Schnitzler, Lili 13.09.1909 – 26.07.1928@\textsc{Schnitzler, Lili} (13.09.1909 – 26.07.1928)|pw}, alle guten Wünsche für die nächste Zeit\pend
           \pstart
           von ganzem Herzen{\\[\baselineskip]}Ihr{\\[\baselineskip]}\spacefill\mbox{Gabriel Beer-Hofmann}\pend
           \leftskip=0em{}
         
         \endnumbering\mylabel{h}\end{ledgroupsized}  \newcommand{\dateiname}{L02479}\newcommand{\titel}{Gabriel Beer-Hofmann an Arthur Schnitzler, 12. 10. 1926}\newcommand{\editorInnen}{Martin Anton Müller und Gerd-Hermann Susen}%% latex-leseansicht-abspann.tex
%% Abspann für die Leseansicht.
%% Der Schalter \ifkorrekturansicht ist bereits durch den Vorspann gesetzt.

%% latex-abspann.tex
%% Gemeinsamer Abspann für Korrekturansicht und Leseansicht.
%% Setzt den Schalter \ifkorrekturansicht voraus (gesetzt in den
%% einbindenden Dateien latex-korrekturansicht-abspann.tex bzw.
%% latex-leseansicht-abspann.tex).
%% ---------------------------------------------------------------

\normalsize

% Das esempio-Environment wird nur in der Leseansicht benötigt
\ifkorrekturansicht\else
\newenvironment{esempio}[3]%
{
    \vspace{1.5ex}
    \rlap{\underline{#1}}
    \par
    \setlength{\parindent}{0cm}
    \nopagebreak
    \leftskip=#2cm
    \rightskip=#3cm
}
{
    \par
}
\fi

\doendnotes{C}
\bigskip
\vfill

\clearpage

\footnotesize

\ifkorrekturansicht
  \lohead{\textsc{register}}
\fi

% theindex-Environment neu definieren ohne reledmac
\makeatletter
\renewenvironment{theindex}{%
  \ifkorrekturansicht
    \section*{\indexname}%
  \else
    \subsubsection*{Index der erwähnten Entitäten}%
  \fi
  \setlength{\parindent}{0pt}%
  \setlength{\parskip}{0pt plus 0.3pt}%
  \let\item\@idxitem
}{%
  \ifkorrekturansicht\clearpage\fi
}
\makeatother

\IfFileExists{\jobname-pw.ind}{\input{\jobname-pw.ind}}{}

% Quellenangabe nur in der Leseansicht
\ifkorrekturansicht\else
% Fallback-Definitionen, falls die .tex-Datei \titel etc. nicht gesetzt hat
\providecommand{\titel}{}
\providecommand{\editorInnen}{}
\providecommand{\dateiname}{\jobname}

\vspace{3cm}

\vfill

\footnotesize
\textsc{Quelle}: \titel. Herausgegeben von {\editorInnen}. In: \emph{Arthur Schnitzler: Briefwechsel mit Autorinnen und Autoren}.
 Digitale Edition, https://schnitzler-briefe.acdh.oeaw.ac.at/{\dateiname}.html (Stand \today)
\fi

\end{document}


      