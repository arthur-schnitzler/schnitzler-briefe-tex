%% latex-leseansicht-vorspann.tex
%% Vorspann für die Leseansicht.
%% Lädt die gemeinsame Datei latex-vorspann.tex mit nicht gesetztem Schalter.

\newif\ifkorrekturansicht
\korrekturansichtfalse

\input{../tex-inputs/latex-vorspann}


\section[ Paul Goldmann an Arthur Schnitzler, 23. 7. {[}1903{]}]{L03378 Paul Goldmann an Arthur Schnitzler,  23. 7. [1903]}
\nopagebreak\mylabel{L03378v}
\rehead{ }\normalsize\beginnumbering\briefempfaengerindex{Schnitzler, Arthur@\textsc{Schnitzler, Arthur}!zzzGoldmann, Paul@\emph{von Paul Goldmann}!1903-07-231@{23. 7. [1903]}|(be}
\toendnotes[C]{\smallbreak\pagebreak[2]}
\correspDesc{Versand  durch Paul Goldmann am 23. 7. [1903] in Berlin
\newline{}Erhalt  durch Arthur Schnitzler im Zeitraum [24. 7. 1903
                  – 28. 7. 1903?] in Wien}\toendnotes[C]{\smallbreak}
\Standort{DLA, A:Schnitzler, HS.NZ85.1.3173.}
\physDesc{Brief, 1 Blatt, 3 Seiten, 1036 Zeichen
\newline{}Handschrift: blaue Tinte, deutsche Kurrent
\newline{}Schnitzler: mit Bleistift das Jahr »903« vermerkt }\toendnotes[C]{\smallbreak}
\pstart
           \raggedleft{}{\pb}\textcolor{gray}{\textbf{DESSAUERSTRASSE 19\oindex{Dessauer Straße@\textbf{Dessauer Straße}, \emph{Straße}|pw}}}\pend
           
\pstart
           Berlin\oindex{Berlin@\textbf{Berlin}, \emph{Hauptstadt}|pw}, 23. Juli.\pend
           
\pstart\center{}Mein lieber Freund,\pend\vspace{0.5em}
\pstart
           Unſere Briefe haben{ }ſich gekreuzt. Wenn \label{K_L03378-1v}\edtext{»ſie\pwindex{Rottenberg, Theodore 7.\,9.\,1875 – 5.\,4.\,1945 Limburg an der Lahn@\textsc{Rottenberg, Theodore} (7.\,9.\,1875 – 5.\,4.\,1945 Limburg an der Lahn)|pwv}«}{\lemma{\textnormal{\emph{»sie«}}}\Cendnote{\textnormal{Siehe XXXX Auszeichnungsfehler: Dokument L03377 nicht gefunden.
               }}}\label{K_L03378-1} mit mir kommt (was noch{ }ſehr ungewiß iſt), werde ich wohl{ }ſo \label{K_L03378-2v}\edtext{zwiſchen dem 5. und 10. Auguſt in Wien\oindex{Wien@\textbf{Wien}, \emph{Verwaltungsgebiet}|pw} eintreffen, um von da nach Tirol\oindex{Tirol@\textbf{Tirol}, \emph{Land}|pw}\oindex{Südtirol@\textbf{Südtirol}, \emph{Verwaltungsgebiet}|pw} weiterzufahren}{\lemma{\textnormal{\emph{zwischen … weiterzufahren}}}\Cendnote{\textnormal{Siehe XXXX Auszeichnungsfehler: Dokument L03375 nicht gefunden.
               }}}\label{K_L03378-2}. Biſt Du dann noch in Wien\oindex{Wien@\textbf{Wien}, \emph{Verwaltungsgebiet}|pw}? Kommt »ſie\pwindex{Rottenberg, Theodore 7.\,9.\,1875 – 5.\,4.\,1945 Limburg an der Lahn@\textsc{Rottenberg, Theodore} (7.\,9.\,1875 – 5.\,4.\,1945 Limburg an der Lahn)|pwv}« nicht mit,{ }ſo gehe ich
               vielleicht nach Marienbad\oindex{Marienbad@\textbf{Marienbad}|pw} zur Kur.\pend
           
\pstart
           Bitte nochmals: empfiehl’ mir {\pb}eine{ }ſchön gelegene,
               kühle und billige Tirol\oindex{Tirol@\textbf{Tirol}, \emph{Land}|pw}\oindex{Südtirol@\textbf{Südtirol}, \emph{Verwaltungsgebiet}|pw}er Sommerſtation,
               wo man nicht allzuſehr \label{K_L03378-3v}\edtext{unter
                  Beobachtung}{\lemma{\textnormal{\emph{unter
                  Beobachtung}}}\Cendnote{\textnormal{Siehe XXXX Auszeichnungsfehler: Dokument L03377 nicht gefunden.
               }}}\label{K_L03378-3}{ }ſteht. \textsc{Richard\pwindex{Beer-Hofmann, Richard 11.\,7.\,1866 Wien – 26.\,9.\,1945 New York City@\textsc{Beer-Hofmann, Richard} (11.\,7.\,1866 Wien – 26.\,9.\,1945 New York City), \emph{Schriftsteller}|pw}} widerräth \textsc{Eppan\oindex{Eppan an der Weinstraße@\textbf{Eppan an der Weinstraße}, \emph{Verwaltungsgebiet}|pw}} als zu heiß.\pend
           
\pstart
           Warum regſt Du Dich über die \label{K_L03378-4v}\edtext{Indiskretionen der \strikeout{\textcolor{gray}{×}} Zeitungen}{\lemma{\textnormal{\emph{Indiskretionen der Zeitungen}}}\Cendnote{\textnormal{Zeitungsmeldungen hatten
                  die bevorstehende Hochzeit von Schnitzler
                  und Olga Gussmann\pwindex{Schnitzler, Olga 17.\,1.\,1882 Wien – 13.\,1.\,1970 Lugano@\textsc{Schnitzler, Olga} (17.\,1.\,1882 Wien – 13.\,1.\,1970 Lugano), \emph{Schauspielerin, Sängerin}|pwk} gebracht, beispielsweise: »– \so{Dr.}\hspace*{1em}\so{Arthur Schnitzler} vermählt{ }ſich in den allernächſten Tagen in aller Stille mit Fräulein
                           Olga \so{Gußmann}\pwindex{Schnitzler, Olga 17.\,1.\,1882 Wien – 13.\,1.\,1970 Lugano@\textsc{Schnitzler, Olga} (17.\,1.\,1882 Wien – 13.\,1.\,1970 Lugano), \emph{Schauspielerin, Sängerin}|pw}.\pwindex{Dr. Arthur Schnitzler vermählt sich@\emph{Dr. Arthur Schnitzler vermählt sich}|pw}« (\emph{Prager Tagblatt}\pwindex{Prager Tagblatt@\emph{Prager Tagblatt}|pwk}, Jg. 27, Nr. 191, 15. 7. 1903, Morgen-Ausgabe, S. 8.)}}}\label{K_L03378-4}{ }ſo
               auf? Das{ }ſind doch die natürlichen Begleiterſcheinungen der Berühmtheit. Wenn man{ }ſo
               in der Öffentlichkeit{ }ſteht, wie Du, muß man{ }ſich {\pb}auch gefallen laſſen, daß die Öffentlichkeit{ }ſich mit Einem beſchäftigt. Ich finde
               darum die Zeitungen gar nicht{ }ſo »widerlich«. Und{ }ſchließlich: was{ }ſchadet es auch,
               daß{ }ſie melden, was doch bald wahr{ }ſein wird. Sei nicht{ }ſo nervös, mein lieber, alter
               (entſchuldige!) Freund!\pend
           
\pstart
           Grüße \textsc{Olga\pwindex{Schnitzler, Olga 17.\,1.\,1882 Wien – 13.\,1.\,1970 Lugano@\textsc{Schnitzler, Olga} (17.\,1.\,1882 Wien – 13.\,1.\,1970 Lugano), \emph{Schauspielerin, Sängerin}|pw}} und \textsc{Heinrich\pwindex{Schnitzler, Heinrich 9.\,8.\,1902 Hinterbrühl – 12.\,7.\,1982 Wien@\textsc{Schnitzler, Heinrich} (9.\,8.\,1902 Hinterbrühl – 12.\,7.\,1982 Wien), \emph{Regisseur, Schauspieler}|pw}} und{ }ſei{ }ſelbſt vielmals und herzlichſt gegrüßt von Deinem {\\[\baselineskip]}\spacefill\mbox{Paul Goldmn}\pend
           \leftskip=0em{}\selectlanguage{ngerman}\endnumbering\briefempfaengerindex{Schnitzler, Arthur@\textsc{Schnitzler, Arthur}!zzzGoldmann, Paul@\emph{von Paul Goldmann}!1903-07-231@{23. 7. [1903]}|)be}\mylabel{L03378h}  \newcommand{\dateiname}{L03378}\newcommand{\titel}{Paul Goldmann an Arthur Schnitzler, 23. 7. [1903]}\newcommand{\editorInnen}{Martin Anton Müller und Laura Untner}%% latex-leseansicht-abspann.tex
%% Abspann für die Leseansicht.
%% Der Schalter \ifkorrekturansicht ist bereits durch den Vorspann gesetzt.

%% latex-abspann.tex
%% Gemeinsamer Abspann für Korrekturansicht und Leseansicht.
%% Setzt den Schalter \ifkorrekturansicht voraus (gesetzt in den
%% einbindenden Dateien latex-korrekturansicht-abspann.tex bzw.
%% latex-leseansicht-abspann.tex).
%% ---------------------------------------------------------------

\normalsize

% Das esempio-Environment wird nur in der Leseansicht benötigt
\ifkorrekturansicht\else
\newenvironment{esempio}[3]%
{
    \vspace{1.5ex}
    \rlap{\underline{#1}}
    \par
    \setlength{\parindent}{0cm}
    \nopagebreak
    \leftskip=#2cm
    \rightskip=#3cm
}
{
    \par
}
\fi

\doendnotes{C}
\bigskip
\vfill

\clearpage

\footnotesize

\ifkorrekturansicht
  \lohead{\textsc{register}}
\fi

% theindex-Environment neu definieren ohne reledmac
\makeatletter
\renewenvironment{theindex}{%
  \ifkorrekturansicht
    \section*{\indexname}%
  \else
    \subsubsection*{Index der erwähnten Entitäten}%
  \fi
  \setlength{\parindent}{0pt}%
  \setlength{\parskip}{0pt plus 0.3pt}%
  \let\item\@idxitem
}{%
  \ifkorrekturansicht\clearpage\fi
}
\makeatother

\IfFileExists{\jobname-pw.ind}{\input{\jobname-pw.ind}}{}

% Quellenangabe nur in der Leseansicht
\ifkorrekturansicht\else
% Fallback-Definitionen, falls die .tex-Datei \titel etc. nicht gesetzt hat
\providecommand{\titel}{}
\providecommand{\editorInnen}{}
\providecommand{\dateiname}{\jobname}

\vspace{3cm}

\vfill

\footnotesize
\textsc{Quelle}: \titel. Herausgegeben von {\editorInnen}. In: \emph{Arthur Schnitzler: Briefwechsel mit Autorinnen und Autoren}.
 Digitale Edition, https://schnitzler-briefe.acdh.oeaw.ac.at/{\dateiname}.html (Stand \today)
\fi

\end{document}


