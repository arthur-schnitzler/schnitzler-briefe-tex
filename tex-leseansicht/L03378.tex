%% latex-korrekturansicht-vorspann.tex
%% Vorspann für die Korrekturansicht.
%% Lädt die gemeinsame Datei latex-vorspann.tex mit gesetztem Schalter.

\newif\ifkorrekturansicht
\korrekturansichttrue

\input{../tex-inputs/latex-vorspann}


\section[ Paul Goldmann an Arthur Schnitzler, 23. 7. {[}1903{]}]{L03378 Paul Goldmann an Arthur Schnitzler, 23. 7. {[}1903{]}}
\nopagebreak\mylabel{L03378v}
\rehead{ }\normalsize\beginnumbering\briefempfaengerindex{Schnitzler, Arthur@\textsc{Schnitzler, Arthur}!zzzGoldmann, Paul@\emph{von Paul Goldmann}!1903-07-231@{23. 7. {[}1903{]}}|(be}
\toendnotes[C]{\smallbreak\pagebreak[2]}\Standort{DLA, A:Schnitzler, HS.NZ85.1.3173.}
\physDesc{Brief, 1 Blatt, 3 Seiten, 1036 Zeichen
\newline{}Handschrift: blaue Tinte, deutsche Kurrent
\newline{}Schnitzler: mit Bleistift das Jahr »903« vermerkt }\toendnotes[C]{\smallbreak}
\pstart
           \raggedleft{}{\pb}\textcolor{gray}{\textbf{DESSAUERSTRASSE 19\oindex{Dessauer Strasse@\textbf{Dessauer Straße}, \emph{Straße (K.STR)}|pw}}}\pend
           
\pstart
           Berlin\oindex{Berlin@\textbf{Berlin}, \emph{P.PPLC}|pw}, 23. Juli.\pend
           
\pstart\center{}Mein lieber Freund,\pend\vspace{0.5em}
\pstart
           Unſere Briefe haben ſich gekreuzt. Wenn \label{K_L03378-1v}\edtext{»ſie\pwindex{Rottenberg, Theodore 1875-09-07 – 1945-04-05@\textsc{Rottenberg, Theodore} (1875-09-07 – 1945-04-05)|pwv}«}{\lemma{\textnormal{\emph{»ſie«}}}\Cendnote{\textnormal{Siehe Paul Goldmann an Arthur Schnitzler, 19. 7. [1903].
               }}}\label{K_L03378-1} mit mir kommt (was noch ſehr ungewiß iſt), werde ich wohl ſo \label{K_L03378-2v}\edtext{zwiſchen dem 5. und 10. Auguſt in Wien\oindex{Wien@\textbf{Wien}, \emph{A.ADM2}|pw} eintreffen, um von da nach Tirol\oindex{Tirol@\textbf{Tirol}, \emph{A.ADM1}|pw}\oindex{Suedtirol@\textbf{Südtirol}, \emph{A.ADM2}|pw} weiterzufahren}{\lemma{\textnormal{\emph{zwiſchen … weiterzufahren}}}\Cendnote{\textnormal{Siehe Paul Goldmann an Arthur Schnitzler, 27. 6. [1903].
               }}}\label{K_L03378-2}. Biſt Du dann noch in Wien\oindex{Wien@\textbf{Wien}, \emph{A.ADM2}|pw}? Kommt »ſie\pwindex{Rottenberg, Theodore 1875-09-07 – 1945-04-05@\textsc{Rottenberg, Theodore} (1875-09-07 – 1945-04-05)|pwv}« nicht mit, ſo gehe ich
               vielleicht nach Marienbad\oindex{Marienbad@\textbf{Marienbad}, \emph{P.PPL}|pw} zur Kur.\pend
           
\pstart
           Bitte nochmals: empfiehl’ mir {\pb}eine ſchön gelegene,
               kühle und billige Tirol\oindex{Tirol@\textbf{Tirol}, \emph{A.ADM1}|pw}\oindex{Suedtirol@\textbf{Südtirol}, \emph{A.ADM2}|pw}er Sommerſtation,
               wo man nicht allzuſehr \label{K_L03378-3v}\edtext{unter
                  Beobachtung}{\lemma{\textnormal{\emph{unter
                  Beobachtung}}}\Cendnote{\textnormal{Siehe Paul Goldmann an Arthur Schnitzler, 19. 7. [1903].
               }}}\label{K_L03378-3} ſteht. \textsc{Richard\pwindex{Beer-Hofmann, Richard 1866-07-11 – 1945-09-26@\textsc{Beer-Hofmann, Richard} (1866-07-11 – 1945-09-26), \emph{Schriftsteller/Schriftstellerin}|pw}} widerräth \textsc{Eppan\oindex{Eppan an der Weinstrasse@\textbf{Eppan an der Weinstraße}, \emph{A.ADM3}|pw}} als zu heiß.\pend
           
\pstart
           Warum regſt Du Dich über die \label{K_L03378-4v}\edtext{Indiskretionen der \strikeout{\textcolor{gray}{×}} Zeitungen}{\lemma{\textnormal{\emph{Indiskretionen der Zeitungen}}}\Cendnote{\textnormal{Zeitungsmeldungen hatten
                  die bevorstehende Hochzeit von Schnitzler
                  und Olga Gussmann\pwindex{Schnitzler, Olga 17.01.1882 – 13.01.1970@\textsc{Schnitzler, Olga} (17.01.1882 – 13.01.1970), \emph{Schauspieler/Schauspielerin, Sänger/Sängerin}|pwk} gebracht, beispielsweise: »– \so{Dr.{ }}\so{Arthur Schnitzler} vermählt ſich in den allernächſten Tagen in aller Stille mit Fräulein
                           Olga \so{Gußmann}\pwindex{Schnitzler, Olga 17.01.1882 – 13.01.1970@\textsc{Schnitzler, Olga} (17.01.1882 – 13.01.1970), \emph{Schauspieler/Schauspielerin, Sänger/Sängerin}|pw}.\pwindex{Dr. Arthur Schnitzler vermaehlt sich@\emph{Dr. Arthur Schnitzler vermählt sich}|pw}« (\emph{Prager Tagblatt}\pwindex{Prager Tagblatt@\emph{Prager Tagblatt}|pwk}, Jg. 27, Nr. 191, 15. 7. 1903, Morgen-Ausgabe, S. 8.)}}}\label{K_L03378-4} ſo
               auf? Das ſind doch die natürlichen Begleiterſcheinungen der Berühmtheit. Wenn man ſo
               in der Öffentlichkeit ſteht, wie Du, muß man ſich {\pb}auch gefallen laſſen, daß die Öffentlichkeit ſich mit Einem beſchäftigt. Ich finde
               darum die Zeitungen gar nicht ſo »widerlich«. Und ſchließlich: was ſchadet es auch,
               daß ſie melden, was doch bald wahr ſein wird. Sei nicht ſo nervös, mein lieber, alter
               (entſchuldige!) Freund!\pend
           
\pstart
           Grüße \textsc{Olga\pwindex{Schnitzler, Olga 17.01.1882 – 13.01.1970@\textsc{Schnitzler, Olga} (17.01.1882 – 13.01.1970), \emph{Schauspieler/Schauspielerin, Sänger/Sängerin}|pw}} und \textsc{Heinrich\pwindex{Schnitzler, Heinrich 09.08.1902 – 12.07.1982@\textsc{Schnitzler, Heinrich} (09.08.1902 – 12.07.1982), \emph{Regisseur/Regisseurin, Schauspieler/Schauspielerin}|pw}} und ſei ſelbſt vielmals und herzlichſt gegrüßt von Deinem {\\[\baselineskip]}\spacefill\mbox{Paul Goldmn}\pend
           \leftskip=0em{}\selectlanguage{ngerman}\endnumbering\briefempfaengerindex{Schnitzler, Arthur@\textsc{Schnitzler, Arthur}!zzzGoldmann, Paul@\emph{von Paul Goldmann}!1903-07-231@{23. 7. {[}1903{]}}|)be}\mylabel{L03378h}  \normalsize

\doendnotes{C}
\bigskip
\vfill

\clearpage

\footnotesize

\lohead{\textsc{register}}

% Definiere theindex-Environment komplett neu ohne reledmac
\makeatletter
\renewenvironment{theindex}{%
  \section*{\indexname}%
  \setlength{\parindent}{0pt}%
  \setlength{\parskip}{0pt plus 0.3pt}%
  \let\item\@idxitem
}{%
  \clearpage
}
\makeatother

\IfFileExists{\jobname-pw.ind}{\input{\jobname-pw.ind}}{}

\end{document}

      