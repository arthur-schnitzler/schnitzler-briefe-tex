%% latex-leseansicht-vorspann.tex
%% Vorspann für die Leseansicht.
%% Lädt die gemeinsame Datei latex-vorspann.tex mit nicht gesetztem Schalter.

\newif\ifkorrekturansicht
\korrekturansichtfalse

\input{../tex-inputs/latex-vorspann}

\begin{center}
            \textcolor{red}{ENTWURF, NICHT FERTIG KORRIGIERT}
                      \end{center}
            
         
         \renewcommand{\erwaehntePersonen}{Personen: Richard Beer-Hofmann, Theodore Rottenberg, Olga Schnitzler, Heinrich Schnitzler}
         \renewcommand{\erwaehnteOrte}{Orte: Berlin, Dessauer Straße, Eppan an der Weinstraße, Marienbad, Südtirol, Tirol, Wien}
         \renewcommand{\erwaehnteWerke}{Werke: Dr. Arthur Schnitzler vermählt sich, Prager Tagblatt}
               \section[ Paul Goldmann an Arthur Schnitzler, 23. 7. {[}1903{]}]{ Paul Goldmann an Arthur Schnitzler, 23. 7. {[}1903{]}}\nopagebreak\mylabel{v}\rehead{ }\begin{ledgroupsized}[t]{13cm}\normalsize\beginnumbering \toendnotes[C]{\smallbreak\pagebreak[2]} \Standort{DLA, A:Schnitzler, HS.NZ85.1.3173.}
\physDesc{Brief, 1 Blatt, 3 Seiten
\newline{}Handschrift: blaue Tinte, deutsche Kurrent
\newline{}Schnitzler: mit Bleistift das Jahr »{[}1{]}903« vermerkt }\toendnotes[C]{\smallbreak}\pstart
           \noindent{}\raggedleft{}{\pb}\textcolor{gray}{\textbf{DESSAUERSTRASSE 19\oindex{Dessauer Strasse@\textbf{Dessauer Straße}|pw}}}\pend
           \pstart
           Berlin\oindex{Berlin@\textbf{Berlin}|pw}, 23. Juli.\pend
           \pstart\center{}Mein lieber Freund,\pend\pstart
           Unſere Briefe haben ſich gekreuzt. Wenn \label{K_L03378-1v}\edtext{»ſie\pwindex{Rottenberg, Theodore 1875-09-07 – 1945-04-05@\textsc{Rottenberg, Theodore} (1875-09-07 – 1945-04-05)|pwv}«}{\lemma{\textnormal{\emph{»ſie«}}}\Cendnote{\textnormal{siehe Paul Goldmann an Arthur Schnitzler, 19. 7. [1903]}}}\label{K_L03378-1h} mit mir kommt (was noch ſehr ungewiß iſt), werde ich wohl ſo \label{K_L03378-2v}\edtext{zwiſchen dem 5. und 10. Auguſt in Wien\oindex{Wien@\textbf{Wien}|pw} eintreffen, um von da nach Tirol\oindex{Tirol@\textbf{Tirol}|pw}\oindex{Suedtirol@\textbf{Südtirol}|pw} weiterzufahren}{\lemma{\textnormal{\emph{zwiſchen … weiterzufahren}}}\Cendnote{\textnormal{siehe Paul Goldmann an Arthur Schnitzler, 27. 6. [1903]}}}\label{K_L03378-2h}. Biſt Du dann noch in Wien\oindex{Wien@\textbf{Wien}|pw}? Kommt »ſie\pwindex{Rottenberg, Theodore 1875-09-07 – 1945-04-05@\textsc{Rottenberg, Theodore} (1875-09-07 – 1945-04-05)|pwv}« nicht mit, ſo gehe ich
               vielleicht nach Marienbad\oindex{Marienbad@\textbf{Marienbad}|pw} zur Kur.\pend
           \pstart
           Bitte nochmals: empfiehl’ mir {\pb}eine ſchön gelegene,
               kühle und billige Tirol\oindex{Tirol@\textbf{Tirol}|pw}\oindex{Suedtirol@\textbf{Südtirol}|pw}er Sommerſtation,
               wo man nicht allzuſehr \label{K_L03378-4v}\edtext{unter
                  Beobachtung}{\lemma{\textnormal{\emph{unter
                  Beobachtung}}}\Cendnote{\textnormal{siehe Paul Goldmann an Arthur Schnitzler, 19. 7. [1903]}}}\label{K_L03378-4h} ſteht. \textsc{Richard\pwindex{Beer-Hofmann, Richard 1866-07-11 – 1945-09-26@\textsc{Beer-Hofmann, Richard} (1866-07-11 – 1945-09-26), \emph{Schriftsteller}|pw}} widerräth \textsc{Eppan\oindex{Eppan an der Weinstrasse@\textbf{Eppan an der Weinstraße}|pw}} als zu heiß.\pend
           \pstart
           Warum regſt Du Dich über die \label{K_L03378-6v}\edtext{Indiskretionen der \strikeout{\textcolor{gray}{×}} Zeitungen}{\lemma{\textnormal{\emph{Indiskretionen der Zeitungen}}}\Cendnote{\textnormal{Zeitungsmeldungen hatten
                  die bevorstehende Hochzeit von Schnitzler\pwindex{Schnitzler, Arthur 15.05.1862 – 21.10.1931@\textsc{Schnitzler, Arthur} (15.05.1862 – 21.10.1931), \emph{Schriftsteller, Mediziner}|pwk}
                  und Olga Gussmann\pwindex{Schnitzler, Olga 17.01.1882 – 13.01.1970@\textsc{Schnitzler, Olga} (17.01.1882 – 13.01.1970), \emph{Schauspielerin, Sängerin}|pwk} gebracht, beispielsweise: »– \so{Dr.{ }}\so{Arthur Schnitzler}\pwindex{Schnitzler, Arthur 15.05.1862 – 21.10.1931@\textsc{Schnitzler, Arthur} (15.05.1862 – 21.10.1931), \emph{Schriftsteller, Mediziner}|pw} vermählt ſich in den allernächſten Tagen in aller Stille mit Fräulein
                           Olga \so{Gußmann}\pwindex{Schnitzler, Olga 17.01.1882 – 13.01.1970@\textsc{Schnitzler, Olga} (17.01.1882 – 13.01.1970), \emph{Schauspielerin, Sängerin}|pw}.\pwindex{?? Werk@Nicht ermittelte Verfasserinnen und Verfasser!Dr. Arthur Schnitzler vermaehlt sich1903-07-15@\emph{Dr. Arthur Schnitzler vermählt sich} {[}1903-07-15{]}|pw}«, \emph{Prager Tagblatt}\pwindex{?? Werk@Nicht ermittelte Verfasserinnen und Verfasser!Prager Tagblatt1876 – 1939@\emph{Prager Tagblatt} {[}1876 – 1939{]}|pwk}, Jg. 27, Nr. 191,
                        15. 7. 1903, Morgen-Ausgabe, S. 8.}}}\label{K_L03378-6h} ſo auf? Das
               ſind doch die natürlichen Begleiterſcheinungen der Berühmtheit. Wenn man ſo in der
               Öffentlichkeit ſteht, wie Du, muß man ſich {\pb}auch
               gefallen laſſen, daß die Öffentlichkeit ſich mit Einem beſchäftigt. Ich finde darum
               die Zeitungen gar nicht ſo »widerlich«. Und ſchließlich: was ſchadet es auch, daß ſie
               melden, was doch bald wahr ſein wird? Sei nicht ſo nervös, mein lieber, alter
               (entſchuldige!) Freund!\pend
           \pstart
           Grüße \textsc{Olga\pwindex{Schnitzler, Olga 17.01.1882 – 13.01.1970@\textsc{Schnitzler, Olga} (17.01.1882 – 13.01.1970), \emph{Schauspielerin, Sängerin}|pw}} und \textsc{Heinrich\pwindex{Schnitzler, Heinrich 09.08.1902 – 12.07.1982@\textsc{Schnitzler, Heinrich} (09.08.1902 – 12.07.1982), \emph{Regisseur, Schauspieler}|pw}} und ſei ſelbſt vielmals und herzlichſt gegrüßt von Deinem {\\[\baselineskip]}\spacefill\mbox{Paul Goldm}\pend
           \leftskip=0em{}
         
         \endnumbering\mylabel{h}\end{ledgroupsized}  \newcommand{\dateiname}{L03378}\newcommand{\titel}{Paul Goldmann an Arthur Schnitzler, 23. 7. [1903]}\newcommand{\editorInnen}{Martin Anton Müller und Laura Untner}%% latex-leseansicht-abspann.tex
%% Abspann für die Leseansicht.
%% Der Schalter \ifkorrekturansicht ist bereits durch den Vorspann gesetzt.

%% latex-abspann.tex
%% Gemeinsamer Abspann für Korrekturansicht und Leseansicht.
%% Setzt den Schalter \ifkorrekturansicht voraus (gesetzt in den
%% einbindenden Dateien latex-korrekturansicht-abspann.tex bzw.
%% latex-leseansicht-abspann.tex).
%% ---------------------------------------------------------------

\normalsize

% Das esempio-Environment wird nur in der Leseansicht benötigt
\ifkorrekturansicht\else
\newenvironment{esempio}[3]%
{
    \vspace{1.5ex}
    \rlap{\underline{#1}}
    \par
    \setlength{\parindent}{0cm}
    \nopagebreak
    \leftskip=#2cm
    \rightskip=#3cm
}
{
    \par
}
\fi

\doendnotes{C}
\bigskip
\vfill

\clearpage

\footnotesize

\ifkorrekturansicht
  \lohead{\textsc{register}}
\fi

% theindex-Environment neu definieren ohne reledmac
\makeatletter
\renewenvironment{theindex}{%
  \ifkorrekturansicht
    \section*{\indexname}%
  \else
    \subsubsection*{Index der erwähnten Entitäten}%
  \fi
  \setlength{\parindent}{0pt}%
  \setlength{\parskip}{0pt plus 0.3pt}%
  \let\item\@idxitem
}{%
  \ifkorrekturansicht\clearpage\fi
}
\makeatother

\IfFileExists{\jobname-pw.ind}{\input{\jobname-pw.ind}}{}

% Quellenangabe nur in der Leseansicht
\ifkorrekturansicht\else
% Fallback-Definitionen, falls die .tex-Datei \titel etc. nicht gesetzt hat
\providecommand{\titel}{}
\providecommand{\editorInnen}{}
\providecommand{\dateiname}{\jobname}

\vspace{3cm}

\vfill

\footnotesize
\textsc{Quelle}: \titel. Herausgegeben von {\editorInnen}. In: \emph{Arthur Schnitzler: Briefwechsel mit Autorinnen und Autoren}.
 Digitale Edition, https://schnitzler-briefe.acdh.oeaw.ac.at/{\dateiname}.html (Stand \today)
\fi

\end{document}


      