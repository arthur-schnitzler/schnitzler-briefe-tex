%% latex-korrekturansicht-vorspann.tex
%% Vorspann für die Korrekturansicht.
%% Lädt die gemeinsame Datei latex-vorspann.tex mit gesetztem Schalter.

\newif\ifkorrekturansicht
\korrekturansichttrue

\input{../tex-inputs/latex-vorspann}


\section[Arthur Schnitzler an Hermann Bahr, 10. – 12. 9. 1901]{L01171 Arthur Schnitzler an Hermann Bahr, 10. – 12. 9. 1901}
\nopagebreak\mylabel{L01171v}
\rehead{ }\normalsize\beginnumbering\briefempfaengerindex{Bahr, Hermann@\textsc{Bahr, Hermann}!zzzSchnitzler, Arthur@\emph{von Arthur Schnitzler}!1901-09-121@{10. – 12. 9. 1901}|(be}
\toendnotes[C]{\smallbreak\pagebreak[2]}\Standort{TMW, HS AM 23343 Ba.}
\physDesc{Brief, 1 Blatt, 4 Seiten, 1238 Zeichen
\newline{}Handschrift: schwarze Tinte, deutsche Kurrent
\newline{}Ordnung: 1) Lochung  2) mit Bleistift von unbekannter Hand (falsch) datiert):
                                    »16. 5. 01«}
\buchAbdrucke{\weitereDrucke{1) Arthur Schnitzler: \emph{The Letters of Arthur Schnitzler to Hermann Bahr}. Chapel Hill: \emph{The University of North Carolina Press} 1978, S. 70.} \weitereDrucke{2) Hermann Bahr, Arthur Schnitzler: \emph{Briefwechsel, Aufzeichnungen, Dokumente (1891–1931)}. Göttingen: \emph{Wallstein} 2018, S. 213–214.} }\toendnotes[C]{\smallbreak}
\pstart
           \noindent{}{\pb}mein lieber
                  Hermann, ich ſchicke dir \introOben{}heute\introOben{} die 3 Einakter\pwindex{Frau mit dem Dolche@\emph{Die Frau mit dem Dolche}|pwv}\pwindex{Literatur@\emph{Literatur}|pwv}\pwindex{Lebendige Stunden@\emph{Lebendige Stunden}|pwv}.
               Mein Bedenken, die \label{K_L01171-1v}\edtext{Kürze des Abends
                  betreffend}{\lemma{\textnormal{\emph{Kürze … betreffend}}}\Cendnote{\textnormal{Vgl. A. S.: \emph{Tagebuch}, 6. 9. 1901.
               }}}\label{K_L01171-1}, iſt wieder rege geworden; und ich habe die Abſicht, einen vierten Einakter\pwindex{Puppenspieler. Studie in einem Aufzuge@\emph{Der Puppenspieler. Studie in einem Aufzuge}|pwv}, der mir geſtern einfiel
               und in Sinn und Form zu den bis jetzt vorliegenden paſſt, zu ſchreiben. Ob ich gleich
               die rechte Sti{\geminationm}ung dafür finden werde, iſt natürlich
               noch nicht ausgemacht. Jedenfalls bitt’ {\pb}ich dich, vor allem
               einmal dieſe 3 Stücke\pwindex{Frau mit dem Dolche@\emph{Die Frau mit dem Dolche}|pwv}\pwindex{Literatur@\emph{Literatur}|pwv}\pwindex{Lebendige Stunden@\emph{Lebendige Stunden}|pwv} zu leſen, u. zw. in der Reihenfolge »\introOben{}1)\introOben{}{ }Die Frau mit dem Dolch\pwindex{Frau mit dem Dolche@\emph{Die Frau mit dem Dolche}|pw}«. 2) Lebendige Stunden\pwindex{Lebendige Stunden@\emph{Lebendige Stunden}|pw}\damage{.} 3.) Literatur\pwindex{Literatur@\emph{Literatur}|pw}. Es wäre ſchade, wenn
               der Abend an einem ſo äußerlichen Moment, wie dem der Kürze ſcheitern ſollte.
               Allerdings glaube ich, dſs dieſes Bedenken weniger für Wien\oindex{Wien@\textbf{Wien}, \emph{A.ADM2}|pw} als für Berlin\oindex{Berlin@\textbf{Berlin}, \emph{P.PPLC}|pw} in Betracht
               käme.\pend
           
\pstart
           Wenns dir recht iſt, ko{\geminationm} ich wieder {\pb}einmal in den
               Vormittagſtunden zu dir hinaus, ſobald du die Sachen geleſen haſt; es eilt \uline{durchaus nicht}.\pend
           
\pstart
           herzlich grüßt dich{\\[\baselineskip]}dein \spacefill\mbox{Arthur}\pend
           \leftskip=0em{}
\pstart
           Wien\oindex{Wien@\textbf{Wien}, \emph{A.ADM2}|pw}{ }10. 9. 901\pend
           \selectlanguage{ngerman}\vspace{1em}
\pstart
           \noindent{}Der Zufall fügte es, daſs ich, durch ein teleph. Erſuchen Kadelburgs\pwindex{Kadelburg, Heinrich 14.02.1856 – 13.07.1910@\textsc{Kadelburg, Heinrich} (14.02.1856 – 13.07.1910), \emph{Schriftsteller/Schriftstellerin, Regisseur/Regisseurin, Schauspieler/Schauspielerin}|pw} veranlaſſt, die Stücke\pwindex{Frau mit dem Dolche@\emph{Die Frau mit dem Dolche}|pwv}\pwindex{Literatur@\emph{Literatur}|pwv}\pwindex{Lebendige Stunden@\emph{Lebendige Stunden}|pwv} in der Direktion
               überreichte. Ich bat, daſs man ſie {\pb}dir zukommen ließe, was
               wohl bereits geſchehen iſt\pend
           
\pstart
           Indeſs hab ich den vierten
                  Einakter\pwindex{Puppenspieler. Studie in einem Aufzuge@\emph{Der Puppenspieler. Studie in einem Aufzuge}|pwv} zu ſchreiben begonnen und hoffe, daſs er ſich, wie vielleicht noch
               ein \label{K_L01171-2v}\edtext{fünfter\pwindex{letzten Masken@\emph{Die letzten Masken}|pwv}}{\lemma{\textnormal{\emph{fünfter}}}\Cendnote{\textnormal{\emph{Die letzten Masken}\pwindex{letzten Masken@\emph{Die letzten Masken}|pwk}; am 6. 9. 1901{ }schrieb Schnitzler an diesem und am \emph{Puppenspieler}\pwindex{Puppenspieler. Studie in einem Aufzuge@\emph{Der Puppenspieler. Studie in einem Aufzuge}|pwk}. Die Unterscheidung zwischen den zwei Stoffen
                  ergibt sich aus der Formulierung »gestern einfiel« im vorliegenden Brief, womit nur ein neuer Stoff gemeint sein kann. Bereits seit 
                  Frühjahr existierte eine erste dramatische Fassung von \emph{Die letzten Masken}\pwindex{letzten Masken@\emph{Die letzten Masken}|pwk} (Vgl. Arthur Schnitzler an Hermann Bahr, [14. 3.? 1901]). Die Arbeit ging schnell voran, sodass am
                     22. 9. 1901{ }\emph{Die letzten Masken}\pwindex{letzten Masken@\emph{Die letzten Masken}|pwk} vorlag,
                  während \emph{Der Puppenspieler}\pwindex{Puppenspieler. Studie in einem Aufzuge@\emph{Der Puppenspieler. Studie in einem Aufzuge}|pwk} »noch auf
                     ein oder zwei gute Stunden zur Vollendung« wartete (\emph{Der Briefwechsel Arthur Schnitzler – Otto Brahm}.
                        Vollständige Ausgabe. Herausgegeben, eingeleitet und erläutert von Oskar
                        Seidlin. Tübingen: \emph{Niemeyer}{ }1975, S. 95).}}}\label{K_L01171-2} dem
               Cyklus gut einfügen wird\pend
           \pstart herzlichſt \spacefill\mbox{A.}\pend{}
\pstart
           12. 9. 901. \pend
           \selectlanguage{ngerman}\endnumbering\briefempfaengerindex{Bahr, Hermann@\textsc{Bahr, Hermann}!zzzSchnitzler, Arthur@\emph{von Arthur Schnitzler}!1901-09-121@{10. – 12. 9. 1901}|)be}\mylabel{L01171h}  \normalsize

\doendnotes{C}
\bigskip
\vfill

\clearpage

\footnotesize

\lohead{\textsc{register}}

% Definiere theindex-Environment komplett neu ohne reledmac
\makeatletter
\renewenvironment{theindex}{%
  \section*{\indexname}%
  \setlength{\parindent}{0pt}%
  \setlength{\parskip}{0pt plus 0.3pt}%
  \let\item\@idxitem
}{%
  \clearpage
}
\makeatother

\IfFileExists{\jobname-pw.ind}{\input{\jobname-pw.ind}}{}

\end{document}

      