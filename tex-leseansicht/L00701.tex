%% latex-korrekturansicht-vorspann.tex
%% Vorspann für die Korrekturansicht.
%% Lädt die gemeinsame Datei latex-vorspann.tex mit gesetztem Schalter.

\newif\ifkorrekturansicht
\korrekturansichttrue

\input{../tex-inputs/latex-vorspann}


\section[Georg Brandes an Arthur Schnitzler, {[}13. 7. 1897{]}]{L00701 Georg Brandes an Arthur Schnitzler, {[}13. 7. 1897{]}}
\nopagebreak\mylabel{L00701v}
\rehead{ }\normalsize\beginnumbering\briefempfaengerindex{Schnitzler, Arthur@\textsc{Schnitzler, Arthur}!zzzBrandes, Georg@\emph{von Georg Brandes}!1897-07-131@{{[}13. 7. 1897{]}}|(be}
\toendnotes[C]{\smallbreak\pagebreak[2]}\Standort{CUL, Schnitzler, B 17.}
\physDesc{Brief, 1 Blatt, 2 Seiten, 1098 Zeichen
\newline{}Handschrift Schreibkraft: blaue Tinte, lateinische Kurrent
\newline{}Handschrift Georg Brandes: blaue Tinte, lateinische Kurrent (\noindent{}Unterschrift)
\newline{}Schnitzler: mit schwarzer Tinte datiert: »etwa 13. Juli 97« und mit Bleistift nummeriert: »6« }
\buchAbdrucke{\weitereDrucke{Georg Brandes, Arthur Schnitzler: \emph{Ein Briefwechsel}. Bern: \emph{Francke} 1956, S. 63–64.} }\toendnotes[C]{\smallbreak}
\pstart{}{\pb}Lieber und verehrter Herr
                  Schnitzler!\pend\vspace{0.5em}
\pstart
           Ich kann leider nicht mit eigener Hand Ihren liebenswürdigen Brief beantworten. Seit
                  Ende April bin ich krank, habe eine heftige Aderentzündung, die mich
               zwingt ganz still zu liegen, und habe im Juni eine schwere
               Lungenentzündung durchgemacht, die mich dem Tode nah brachte. Jetzt ist die Lunge
               einigermassen heil, doch in der eigentlichen Krankheit ist noch keine Konvalescenz
               eingetreten. Ich werde voraussichtlich noch mehr als einen Monat im Bette bleiben
               müssen. Mein ganzer Sommer ist dahin. Ich habe grosse Schmerzen ausgestanden und bin
               noch sehr leidend.\pend
           
\pstart
           Es freut mich sehr, dass Sie etwas in {\pb}meinem Buch\pwindex{William Shakespeare@\emph{William Shakespeare}|pwv} über Shakespeare\pwindex{Shakespeare, William 23.4.1564? – 03.05.1616@\textsc{Shakespeare, William} (23.4.1564? – 03.05.1616), \emph{Schauspieler/Schauspielerin, Dramatiker/Dramatikerin}|pw} gefunden haben. Ich lese in dieser Zeit die Korrekturbogen der
               zweiten deutschen Ausgabe und bin über die fürchterliche Sprache ganz erschreckt. Es
               wimmelt von den plumpsten Misverständnischen meines dänischen\oindex{Daenemark@\textbf{Dänemark}, \emph{A.PCLI}|pw} Textes; ich schreibe um und verbessere ins unendliche.\pend
           
\pstart
           Ich bitte Sie Ihre Freunde sehr herzlich von mir zu grüssen. Hr Goldmann\pwindex{Goldmann, Paul 31.01.1865 – 25.09.1935@\textsc{Goldmann, Paul} (31.01.1865 – 25.09.1935), \emph{Schriftsteller/Schriftstellerin, Journalist/Journalistin}|pw} verstummte mir gegenüber plötzlich. Sie sind mir aber
               alle drei gleich lieb.\pend
           
\pstart
           Ihr ganz ergebener{\\[\baselineskip]}\spacefill\mbox{{[}hs. :{]} Georg Brandes}\pend
           \leftskip=0em{}\selectlanguage{ngerman}\endnumbering\briefempfaengerindex{Schnitzler, Arthur@\textsc{Schnitzler, Arthur}!zzzBrandes, Georg@\emph{von Georg Brandes}!1897-07-131@{{[}13. 7. 1897{]}}|)be}\mylabel{L00701h}  \normalsize

\doendnotes{C}
\bigskip
\vfill

\clearpage

\footnotesize

\lohead{\textsc{register}}

% Definiere theindex-Environment komplett neu ohne reledmac
\makeatletter
\renewenvironment{theindex}{%
  \section*{\indexname}%
  \setlength{\parindent}{0pt}%
  \setlength{\parskip}{0pt plus 0.3pt}%
  \let\item\@idxitem
}{%
  \clearpage
}
\makeatother

\IfFileExists{\jobname-pw.ind}{\input{\jobname-pw.ind}}{}

\end{document}

      