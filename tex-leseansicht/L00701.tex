%% latex-leseansicht-vorspann.tex
%% Vorspann für die Leseansicht.
%% Lädt die gemeinsame Datei latex-vorspann.tex mit nicht gesetztem Schalter.

\newif\ifkorrekturansicht
\korrekturansichtfalse

\input{../tex-inputs/latex-vorspann}


         
         \renewcommand{\erwaehntePersonen}{Personen: Georg Brandes, Paul Goldmann, William Shakespeare}
         \renewcommand{\erwaehnteOrte}{Orte: Bad Ischl, Dänemark, Kopenhagen}
         \renewcommand{\erwaehnteWerke}{Werke: William Shakespeare}
               \section[Georg Brandes an Arthur Schnitzler, {[}13. 7. 1897{]}]{ Georg Brandes an Arthur Schnitzler, {[}13. 7. 1897{]}}\nopagebreak\mylabel{v}\rehead{ }\begin{ledgroupsized}[t]{13cm}\normalsize\beginnumbering \toendnotes[C]{\smallbreak\pagebreak[2]} \Standort{CUL, Schnitzler, B 17.}
\physDesc{Brief, 1 Blatt, 2 Seiten, 1098 Zeichen
\newline{}Handschrift Schreibkraft: blaue Tinte, lateinische Kurrent\newline{}Handschrift Georg Brandes: blaue Tinte, lateinische Kurrent (\noindent{}Unterschrift)
\newline{}Schnitzler: mit schwarzer Tinte datiert: »etwa 13. Juli 97« und mit Bleistift nummeriert: »6« }\buchAbdrucke{\weitereDrucke{Georg Brandes, Arthur Schnitzler: \emph{Ein Briefwechsel}. Hg. Kurt Bergel. Bern: \emph{Francke} 1956, S. 63–64.} }\toendnotes[C]{\smallbreak}\pstart{}{\pb}Lieber und verehrter Herr
                  Schnitzler!\pend\pstart
           Ich kann leider nicht mit eigener Hand Ihren liebenswürdigen Brief beantworten. Seit
                  Ende April bin ich krank, habe eine heftige Aderentzündung, die mich
               zwingt ganz still zu liegen, und habe im Juni eine schwere
               Lungenentzündung durchgemacht, die mich dem Tode nah brachte. Jetzt ist die Lunge
               einigermassen heil, doch in der eigentlichen Krankheit ist noch keine Konvalescenz
               eingetreten. Ich werde voraussichtlich noch mehr als einen Monat im Bette bleiben
               müssen. Mein ganzer Sommer ist dahin. Ich habe grosse Schmerzen ausgestanden und bin
               noch sehr leidend.\pend
           \pstart
           Es freut mich sehr, dass Sie etwas in {\pb}meinem Buch\pwindex{Brandes, Georg 04.02.1842 – 19.02.1927@\textsc{Brandes, Georg} (04.02.1842 – 19.02.1927)!William Shakespeare1895 – 1896@\strich\emph{William Shakespeare} {[}1895 – 1896{]}|pwv} über Shakespeare\pwindex{Shakespeare, William 23.4.1564? – 03.05.1616@\textsc{Shakespeare, William} (23.4.1564? – 03.05.1616), \emph{Schauspieler, Dramatiker}|pw} gefunden haben. Ich lese in dieser Zeit die Korrekturbogen der
               zweiten deutschen Ausgabe und bin über die fürchterliche Sprache ganz erschreckt. Es
               wimmelt von den plumpsten Misverständnischen meines dänischen\oindex{Daenemark@\textbf{Dänemark}|pw} Textes; ich schreibe um und verbessere ins unendliche.\pend
           \pstart
           Ich bitte Sie Ihre Freunde sehr herzlich von mir zu grüssen. Hr Goldmann\pwindex{Goldmann, Paul 31.01.1865 – 25.09.1935@\textsc{Goldmann, Paul} (31.01.1865 – 25.09.1935), \emph{Schriftsteller, Journalist}|pw} verstummte mir gegenüber plötzlich. Sie sind mir aber
               alle drei gleich lieb.\pend
           \pstart
           Ihr ganz ergebener{\\[\baselineskip]}\spacefill\mbox{{[}hs. Brandes:{]} Georg Brandes}\pend
           \leftskip=0em{}
         
         \endnumbering\mylabel{h}\end{ledgroupsized}  \newcommand{\dateiname}{L00701}\newcommand{\titel}{Georg Brandes an Arthur Schnitzler, [13. 7. 1897]}\newcommand{\editorInnen}{Martin Anton Müller und Gerd-Hermann Susen}%% latex-leseansicht-abspann.tex
%% Abspann für die Leseansicht.
%% Der Schalter \ifkorrekturansicht ist bereits durch den Vorspann gesetzt.

%% latex-abspann.tex
%% Gemeinsamer Abspann für Korrekturansicht und Leseansicht.
%% Setzt den Schalter \ifkorrekturansicht voraus (gesetzt in den
%% einbindenden Dateien latex-korrekturansicht-abspann.tex bzw.
%% latex-leseansicht-abspann.tex).
%% ---------------------------------------------------------------

\normalsize

% Das esempio-Environment wird nur in der Leseansicht benötigt
\ifkorrekturansicht\else
\newenvironment{esempio}[3]%
{
    \vspace{1.5ex}
    \rlap{\underline{#1}}
    \par
    \setlength{\parindent}{0cm}
    \nopagebreak
    \leftskip=#2cm
    \rightskip=#3cm
}
{
    \par
}
\fi

\doendnotes{C}
\bigskip
\vfill

\clearpage

\footnotesize

\ifkorrekturansicht
  \lohead{\textsc{register}}
\fi

% theindex-Environment neu definieren ohne reledmac
\makeatletter
\renewenvironment{theindex}{%
  \ifkorrekturansicht
    \section*{\indexname}%
  \else
    \subsubsection*{Index der erwähnten Entitäten}%
  \fi
  \setlength{\parindent}{0pt}%
  \setlength{\parskip}{0pt plus 0.3pt}%
  \let\item\@idxitem
}{%
  \ifkorrekturansicht\clearpage\fi
}
\makeatother

\IfFileExists{\jobname-pw.ind}{\input{\jobname-pw.ind}}{}

% Quellenangabe nur in der Leseansicht
\ifkorrekturansicht\else
% Fallback-Definitionen, falls die .tex-Datei \titel etc. nicht gesetzt hat
\providecommand{\titel}{}
\providecommand{\editorInnen}{}
\providecommand{\dateiname}{\jobname}

\vspace{3cm}

\vfill

\footnotesize
\textsc{Quelle}: \titel. Herausgegeben von {\editorInnen}. In: \emph{Arthur Schnitzler: Briefwechsel mit Autorinnen und Autoren}.
 Digitale Edition, https://schnitzler-briefe.acdh.oeaw.ac.at/{\dateiname}.html (Stand \today)
\fi

\end{document}


      