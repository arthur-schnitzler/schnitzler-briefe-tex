%% latex-leseansicht-vorspann.tex
%% Vorspann für die Leseansicht.
%% Lädt die gemeinsame Datei latex-vorspann.tex mit nicht gesetztem Schalter.

\newif\ifkorrekturansicht
\korrekturansichtfalse

\input{../tex-inputs/latex-vorspann}

\begin{center}
            \textcolor{red}{ENTWURF, NICHT FERTIG KORRIGIERT}
                      \end{center}
            
         
         \renewcommand{\erwaehntePersonen}{Personen:  Alfons XIII., Oskar Bie, Hedwig Fischer, Samuel Fischer, Julius von Gans-Ludassy, Herbert Ginsberg, Leopold Godowsky, Maximilian Harden, Moritz Heimann, Theodor Herzl, Hugo von Hofmannsthal, Siegfried Jacobsohn, Alfred Kerr, Fritz Kreisler, Richard Metzl, Arthur Nikisch, Felix Poppenberg, Anna Katharina Rehmann, Max Reinhardt, Rudolf Rittner, Vasilij Ilʹič Safonov, Felix Salten, Paul Salten, Ottilie Salten, Olga Schnitzler, Heinrich Schnitzler,  Victoria Eugénie von Spanien, Ida d’Albert}
         \renewcommand{\erwaehnteOrte}{Orte: Bansin, Berlin, Cadiz, Friedrichstraße, Gibraltar, Granada, Heringsdorf, Italien, Lissabon, Madrid, Marienlyst, Ostsee, Sevilla, Skodsborg, Spanien, Tanger, Tiergarten, Toledo, Wien, Świnoujście}
         \renewcommand{\erwaehnteWerke}{Werke: Die neue Rundschau, Herr Wenzel auf Rehberg. Novelle}
               \section[ Felix Salten an Arthur Schnitzler, 1. 5. 1906]{ Felix Salten an Arthur Schnitzler, 1. 5. 1906}\nopagebreak\mylabel{v}\rehead{ }\begin{ledgroupsized}[t]{13cm}\normalsize\beginnumbering \toendnotes[C]{\smallbreak\pagebreak[2]} \Standort{CUL, Schnitzler, B 89, B 1.}
\physDesc{Brief, 1 Blatt, 2 Seiten, 3412 Zeichen
\newline{}Handschrift: schwarze Tinte, lateinische Kurrent
\newline{}Schnitzler: mit rotem Buntstift eine Unterstreichung 
\newline{}Ordnung: mit Bleistift von unbekannter Hand nummeriert: »212« }\toendnotes[C]{\smallbreak}\pstart
           \raggedleft{}{\pb}Berlin\oindex{Berlin@\textbf{Berlin}|pw}, 1. Mai 06.\pend
           \pstart
           Lieber, die \label{K_L03422-1v}\edtext{Radpartie}{\lemma{\textnormal{\emph{Radpartie}}}\Cendnote{\textnormal{siehe Felix Salten an Arthur Schnitzler, 28. 3. 1906}}}\label{K_L03422-1h}, ja, wenn ich heute nur wüßte, wie und was in drei, vier Wochen sein wird.
               Ich fürchte, die Radpartie wird sich nicht machen laßen. Vorläufig nämlich ist es
               beschloßen, dass ich am 20. od. 21. nach Madrid\oindex{Madrid@\textbf{Madrid}|pw}
               fahre, zur \label{K_L03422-2v}\edtext{Königshochzeit\pwindex{Alfons XIII. 1886-05-17 – 1941-02-18@\textsc{Alfons XIII.} (1886-05-17 – 1941-02-18), \emph{König}|pwv}\pwindex{Victoria Eugenie von Spanien 1887-10-24 – 1969-04-15@\textsc{Victoria Eugénie von Spanien} (1887-10-24 – 1969-04-15), \emph{Regentin}|pwv}}{\lemma{\textnormal{\emph{Königshochzeit}}}\Cendnote{\textnormal{Am 17. 5. 1906 heirateten in Madrid\oindex{Madrid@\textbf{Madrid}|pwk}
                  König Alfonso XIII. von Spanien\pwindex{Alfons XIII. 1886-05-17 – 1941-02-18@\textsc{Alfons XIII.} (1886-05-17 – 1941-02-18), \emph{König}|pwk} und Victoria Eugénie von Battenberg\pwindex{Victoria Eugenie von Spanien 1887-10-24 – 1969-04-15@\textsc{Victoria Eugénie von Spanien} (1887-10-24 – 1969-04-15), \emph{Regentin}|pwk}.}}}\label{K_L03422-2h}. Da
               käme ich erst am 10. Juni wieder zurück, weil ich
               natürlich Toledo\oindex{Toledo@\textbf{Toledo}|pw}, Sevilla\oindex{Sevilla@\textbf{Sevilla}|pw}, Cadiz\oindex{Cadiz@\textbf{Cadiz}|pw}, Tanger\oindex{Tanger@\textbf{Tanger}|pw}, Gibraltar\oindex{Gibraltar@\textbf{Gibraltar}|pw}, Granada\oindex{Granada@\textbf{Granada}|pw} mitnehme, und der
               Weg zurück über Lissabon\oindex{Lissabon@\textbf{Lissabon}|pw} führe. Da gäbe es dann
               – ausser dem contractlichen Urlaub – keine Absenz mehr. Und die vier Wochen im Juli will ich still an einem Fleck sitzen, Tennis spielen
               und arbeiten. (Ich bin im Begriff, die \label{K_L03422-3v}\edtext{Herzl\pwindex{Herzl, Theodor 1860-05-02 – 1904-07-03@\textsc{Herzl, Theodor} (1860-05-02 – 1904-07-03), \emph{Schriftsteller, Journalist}|pw}-Biographie}{\lemma{\textnormal{\emph{Herzl-Biographie}}}\Cendnote{\textnormal{Eine Biografie Herzl\pwindex{Herzl, Theodor 1860-05-02 – 1904-07-03@\textsc{Herzl, Theodor} (1860-05-02 – 1904-07-03), \emph{Schriftsteller, Journalist}|pwk}s
                  wurde von Salten\pwindex{Salten, Felix 06.09.1869 – 08.10.1945@\textsc{Salten, Felix} (06.09.1869 – 08.10.1945), \emph{Schriftsteller, Journalist}|pwk} nie geschrieben.}}}\label{K_L03422-3h} zu
               übernehmen, was ich mir als eine Art von Denkmal-Portrait sehr schön denke.) Mit dem
                  \label{K_L03422-4v}\edtext{Seebad\oindex{Marienlyst@\textbf{Marienlyst}|pw}}{\lemma{\textnormal{\emph{Seebad}}}\Cendnote{\textnormal{siehe Felix Salten an Arthur Schnitzler, 28. 3. 1906}}}\label{K_L03422-4h} ist das so: wir müßen doch im Juni schon aufs
               Land, der Kinder\pwindex{Rehmann, Anna Katharina 18.08.1904 – 27.03.1977@\textsc{Rehmann, Anna Katharina} (18.08.1904 – 27.03.1977), \emph{Schauspielerin, Übersetzerin}|pwv}\pwindex{Salten, Paul 11.08.1903 – 08.05.1937@\textsc{Salten, Paul} (11.08.1903 – 08.05.1937), \emph{Filmcutter}|pwv}
               wegen. Otti\pwindex{Salten, Ottilie 07.03.1868 – 22.06.1942@\textsc{Salten, Ottilie} (07.03.1868 – 22.06.1942), \emph{Schauspielerin}|pw} und die Kinder\pwindex{Rehmann, Anna Katharina 18.08.1904 – 27.03.1977@\textsc{Rehmann, Anna Katharina} (18.08.1904 – 27.03.1977), \emph{Schauspielerin, Übersetzerin}|pwv}\pwindex{Salten, Paul 11.08.1903 – 08.05.1937@\textsc{Salten, Paul} (11.08.1903 – 08.05.1937), \emph{Filmcutter}|pwv} gehen Juni, Juli, August, bis Mitte September an die See\oindex{Ostsee@\textbf{Ostsee}|pwv}. Da wird eine Wohnung
               genommen und Wirtschaft geführt. Möglichst nahe, damit ich über Sonntag einmal hin,
                  Otti\pwindex{Salten, Ottilie 07.03.1868 – 22.06.1942@\textsc{Salten, Ottilie} (07.03.1868 – 22.06.1942), \emph{Schauspielerin}|pw} manchmal zu mir in die Stadt\oindex{Berlin@\textbf{Berlin}|pwv} kommen kann. Also Bansin\oindex{Bansin@\textbf{Bansin}|pw}, Swinemünde\oindex{Swinoujście@\textbf{Świnoujście}|pw} oder Heringsdorf\oindex{Heringsdorf@\textbf{Heringsdorf}|pw}. \uline{Deshalb} kann ich dann für den Juli nicht alles nach Skodsborg\oindex{Skodsborg@\textbf{Skodsborg}|pw}
               verlegen. Es ist einfach eine Sache des Geldes. Und bin ich selbst frei, möchte ich
               doch bei den Kindern\pwindex{Rehmann, Anna Katharina 18.08.1904 – 27.03.1977@\textsc{Rehmann, Anna Katharina} (18.08.1904 – 27.03.1977), \emph{Schauspielerin, Übersetzerin}|pwv}\pwindex{Salten, Paul 11.08.1903 – 08.05.1937@\textsc{Salten, Paul} (11.08.1903 – 08.05.1937), \emph{Filmcutter}|pwv} sein.\pend
           \pstart
           Wenn sich die spani\oindex{Spanien@\textbf{Spanien}|pwv}sche Reise
               nun doch nicht macht, schreibe ich Ihnen rechtzeitig wegen der Radtour.\pend
           \pstart
           Mein Brief an Hugo\pwindex{Hofmannsthal, Hugo von 1874-02-01 – 1929-07-15@\textsc{Hofmannsthal, Hugo von} (1874-02-01 – 1929-07-15), \emph{Schriftsteller}|pw} mit der starken Verstimmung
               gegen Berlin\oindex{Berlin@\textbf{Berlin}|pw} datirt weit zurück, war im März noch geschrieben, während er in Italien\oindex{Italien@\textbf{Italien}|pw} war. Seither hat sich die Sache genau um die
               Frühlingssonne verbessert. Ich schreibe selten, weil ich mit organisatorischen
               Arbeiten beschäftigt bin, weil ich productiv einiges componire, und die Stadt\oindex{Berlin@\textbf{Berlin}|pwv} noch zu wenig als
               publizistische Anregung fühle. Es würden Reisebriefe werden, und das wäre falsch. Ich
               bin froh, dass mich meine Selbstcontrolle {\pb}vor solchen Verfehlungen ebenso
               wie vor allzufrühen, taktlosen Vertraulichkeiten mit dieser Stadt\oindex{Berlin@\textbf{Berlin}|pwv} bewahrt.\pend
           \pstart
           Wie \label{K_L03422-5v}\edtext{Herr Wenzel\pwindex{Salten, Felix 06.09.1869 – 08.10.1945@\textsc{Salten, Felix} (06.09.1869 – 08.10.1945), \emph{Schriftsteller, Journalist}!Herr Wenzel auf Rehberg. Novelle1906-05-01@\strich\emph{Herr Wenzel auf Rehberg. Novelle} {[}1906-05-01{]}|pw}}{\lemma{\textnormal{\emph{Herr Wenzel}}}\Cendnote{\textnormal{Felix Salten\pwindex{Salten, Felix 06.09.1869 – 08.10.1945@\textsc{Salten, Felix} (06.09.1869 – 08.10.1945), \emph{Schriftsteller, Journalist}|pwk}: \emph{Herr Wenzel auf Rehberg. Novelle}\pwindex{Salten, Felix 06.09.1869 – 08.10.1945@\textsc{Salten, Felix} (06.09.1869 – 08.10.1945), \emph{Schriftsteller, Journalist}!Herr Wenzel auf Rehberg. Novelle1906-05-01@\strich\emph{Herr Wenzel auf Rehberg. Novelle} {[}1906-05-01{]}|pwk}. In: \emph{Die neue Rundschau}\pwindex{?? Werk@Nicht ermittelte Verfasserinnen und Verfasser!neue Rundschau1904@\emph{Die neue Rundschau} {[}1904{]}|pwk}, Jg. 17, H. 5, Mai 1906, S. 544–576.}}}\label{K_L03422-5h} aufgenommen wird, bin ich
               neugierig. Es ist das erstemal, dass ich eine Novelle von mir in der Correctur ohne
               Desperation und tiefe Niedergeschlagenheit lesen konnte.\pend
           \pstart
           Mein Verkehr hier? Ab und zu Heimann\pwindex{Heimann, Moritz 19.07.1868 – 22.09.1925@\textsc{Heimann, Moritz} (19.07.1868 – 22.09.1925), \emph{Schriftsteller, Verlagslektor}|pw}, Jakobsohn\pwindex{Jacobsohn, Siegfried 28.01.1881 – 03.12.1926@\textsc{Jacobsohn, Siegfried} (28.01.1881 – 03.12.1926), \emph{Journalist, Kritiker, Publizist}|pw}. Dann Rittner\pwindex{Rittner, Rudolf 30.06.1869 – 04.02.1943@\textsc{Rittner, Rudolf} (30.06.1869 – 04.02.1943), \emph{Theaterleiter, Schauspieler}|pw}. Und Fischers\pwindex{Fischer, Hedwig 08.09.1871 – 11.04.1952@\textsc{Fischer, Hedwig} (08.09.1871 – 11.04.1952)|pw}\pwindex{Fischer, Samuel 24.12.1859 – 15.10.1934@\textsc{Fischer, Samuel} (24.12.1859 – 15.10.1934), \emph{Verleger}|pw}, die mir aus der Nähe immer sympathischer werden. Selten Reinhardt\pwindex{Reinhardt, Max 09.09.1873 – 30.10.1943@\textsc{Reinhardt, Max} (09.09.1873 – 30.10.1943), \emph{Theaterleiter, Regisseur, Schauspieler}|pw} und seine Leute, manchmal Bie\pwindex{Bie, Oskar 09.02.1864 – 21.04.1938@\textsc{Bie, Oskar} (09.02.1864 – 21.04.1938), \emph{Schriftsteller, Journalist, Redakteur}|pw} (sehr lieb und fein) und Poppenberg\pwindex{Poppenberg, Felix 13.10.1869 – 18.10.1915@\textsc{Poppenberg, Felix} (13.10.1869 – 18.10.1915), \emph{Schriftsteller, Kritiker}|pw}, zwei, drei lange Gespräche mit Kerr\pwindex{Kerr, Alfred 25.12.1867 – 12.10.1948@\textsc{Kerr, Alfred} (25.12.1867 – 12.10.1948), \emph{Schriftsteller, Kritiker}|pw}; fast garnicht mehr Harden\pwindex{Harden, Maximilian 20.10.1861 – 30.10.1927@\textsc{Harden, Maximilian} (20.10.1861 – 30.10.1927), \emph{Schriftsteller, Publizist}|pw}. Dazwischen die Gesellschaften, denen sich nicht ausweichen läßt. Bei
               meinem Schwager\pwindex{Metzl, Richard 20.04.1870 – 31.10.1941@\textsc{Metzl, Richard} (20.04.1870 – 31.10.1941), \emph{Regisseur, Schauspieler, Theatersekretär}|pwv} Musikleute:
                  Safonoff\pwindex{Safonov, Vasilij Ilʹic 1852-02-06 – 1918-02-27@\textsc{Safonov, Vasilij Ilʹič} (1852-02-06 – 1918-02-27), \emph{Dirigent, Pianist, Musiker}|pw}, Godowski\pwindex{Godowsky, Leopold 13.02.1870 – 21.11.1938@\textsc{Godowsky, Leopold} (13.02.1870 – 21.11.1938), \emph{Komponist, Pianist}|pw}, Nikisch\pwindex{Nikisch, Arthur 12.10.1855 – 23.01.1922@\textsc{Nikisch, Arthur} (12.10.1855 – 23.01.1922), \emph{Dirigent}|pw}, Kreisler\pwindex{Kreisler, Fritz 02.02.1875 – 29.01.1962@\textsc{Kreisler, Fritz} (02.02.1875 – 29.01.1962), \emph{Komponist, Pianist}|pw}. Hie und da eine ärgerliche, manchmal eine nette
               Stunde mit Frau Fulda\pwindex{DAlbert, Ida 05.12.1869 – 1926-10-06@\textsc{d’Albert, Ida} (05.12.1869 – 1926-10-06)|pw}. Das ist alles; ist
               genug, ist – gelegentlich sogar zu viel. Ich will lieber lesen, will jetzt viel, sehr
               viel lesen; lerne ein bischen spanisch und gehe mit Otti\pwindex{Salten, Ottilie 07.03.1868 – 22.06.1942@\textsc{Salten, Ottilie} (07.03.1868 – 22.06.1942), \emph{Schauspielerin}|pw} im Thiergarten\oindex{Tiergarten@\textbf{Tiergarten}|pw} spazieren, wo es –
               unglaublich aber wahr – gerade jetzt einfach märchenhaft schön ist.\pend
           \pstart
           Otti\pwindex{Salten, Ottilie 07.03.1868 – 22.06.1942@\textsc{Salten, Ottilie} (07.03.1868 – 22.06.1942), \emph{Schauspielerin}|pw} läßt Frau Olga\pwindex{Schnitzler, Olga 17.01.1882 – 13.01.1970@\textsc{Schnitzler, Olga} (17.01.1882 – 13.01.1970), \emph{Schauspielerin, Sängerin}|pw} um Entschuldigung bitten, weil sie ihren lieben Brief
               noch nicht beantworten konnte. Sie hat sich erst die linke Hand verbrannt, und kaum
               die halbwegs gut war, wieder die rechte verbrüht. Da wir nicht hoffen, dass sie jetzt
               wieder von vorne anfängt, rechnen wir darauf, dass sie bald wieder den Gebrauch all
               ihrer Gliedmaßen erlangt. Die Kinder\pwindex{Rehmann, Anna Katharina 18.08.1904 – 27.03.1977@\textsc{Rehmann, Anna Katharina} (18.08.1904 – 27.03.1977), \emph{Schauspielerin, Übersetzerin}|pwv}\pwindex{Salten, Paul 11.08.1903 – 08.05.1937@\textsc{Salten, Paul} (11.08.1903 – 08.05.1937), \emph{Filmcutter}|pwv} sind reizend, und wir alle grüßen Sie alle\pwindex{Schnitzler, Olga 17.01.1882 – 13.01.1970@\textsc{Schnitzler, Olga} (17.01.1882 – 13.01.1970), \emph{Schauspielerin, Sängerin}|pwv}\pwindex{Schnitzler, Heinrich 09.08.1902 – 12.07.1982@\textsc{Schnitzler, Heinrich} (09.08.1902 – 12.07.1982), \emph{Regisseur, Schauspieler}|pwv} aufs Herzlichste.\pend
           \pstart Ihr \spacefill\mbox{Salten}\pend{}\pstart
           \noindent{}\label{K_L03422-6v}\edtext{\textcolor{gray}{NB}}{\lemma{\textnormal{\emph{NB}}}\Cendnote{\textnormal{nota bene; lateinisch: merke
                     wohl}}}\label{K_L03422-6h}. Heute sahen wir Ludaßy\pwindex{Gans-Ludassy, Julius von 13.04.1858 – 30.09.1922@\textsc{Gans-Ludassy, Julius von} (13.04.1858 – 30.09.1922), \emph{Schriftsteller, Journalist, Herausgeber}|pw} in der Friedrichstraße\oindex{Friedrichstrasse@\textbf{Friedrichstraße}|pw}. Wir haben sehr gestaunt, weil wir dachten, er sei – wie
                  lange schon! – gestorben.\pend
           \pstart
           D\textsuperscript{r} Ginsberg\pwindex{Ginsberg, Herbert 1881-09-27 – 1962-11-05@\textsc{Ginsberg, Herbert} (1881-09-27 – 1962-11-05), \emph{Industrieller, Kunstsammler, Bankier}|pw}
                  schrieb mir sehr entzückt über die freundl. \label{K_L03422-7v}\edtext{Aufnahme}{\lemma{\textnormal{\emph{Aufnahme}}}\Cendnote{\textnormal{siehe Felix Salten an Arthur Schnitzler, 8. 4. 1906}}}\label{K_L03422-7h} bei Ihnen. Vielen Dank!\pend
           
         
         \endnumbering\mylabel{h}\end{ledgroupsized}  \newcommand{\dateiname}{L03422}\newcommand{\titel}{Felix Salten an Arthur Schnitzler, 1. 5. 1906}\newcommand{\editorInnen}{Martin Anton Müller und Laura Untner}%% latex-leseansicht-abspann.tex
%% Abspann für die Leseansicht.
%% Der Schalter \ifkorrekturansicht ist bereits durch den Vorspann gesetzt.

%% latex-abspann.tex
%% Gemeinsamer Abspann für Korrekturansicht und Leseansicht.
%% Setzt den Schalter \ifkorrekturansicht voraus (gesetzt in den
%% einbindenden Dateien latex-korrekturansicht-abspann.tex bzw.
%% latex-leseansicht-abspann.tex).
%% ---------------------------------------------------------------

\normalsize

% Das esempio-Environment wird nur in der Leseansicht benötigt
\ifkorrekturansicht\else
\newenvironment{esempio}[3]%
{
    \vspace{1.5ex}
    \rlap{\underline{#1}}
    \par
    \setlength{\parindent}{0cm}
    \nopagebreak
    \leftskip=#2cm
    \rightskip=#3cm
}
{
    \par
}
\fi

\doendnotes{C}
\bigskip
\vfill

\clearpage

\footnotesize

\ifkorrekturansicht
  \lohead{\textsc{register}}
\fi

% theindex-Environment neu definieren ohne reledmac
\makeatletter
\renewenvironment{theindex}{%
  \ifkorrekturansicht
    \section*{\indexname}%
  \else
    \subsubsection*{Index der erwähnten Entitäten}%
  \fi
  \setlength{\parindent}{0pt}%
  \setlength{\parskip}{0pt plus 0.3pt}%
  \let\item\@idxitem
}{%
  \ifkorrekturansicht\clearpage\fi
}
\makeatother

\IfFileExists{\jobname-pw.ind}{\input{\jobname-pw.ind}}{}

% Quellenangabe nur in der Leseansicht
\ifkorrekturansicht\else
% Fallback-Definitionen, falls die .tex-Datei \titel etc. nicht gesetzt hat
\providecommand{\titel}{}
\providecommand{\editorInnen}{}
\providecommand{\dateiname}{\jobname}

\vspace{3cm}

\vfill

\footnotesize
\textsc{Quelle}: \titel. Herausgegeben von {\editorInnen}. In: \emph{Arthur Schnitzler: Briefwechsel mit Autorinnen und Autoren}.
 Digitale Edition, https://schnitzler-briefe.acdh.oeaw.ac.at/{\dateiname}.html (Stand \today)
\fi

\end{document}


      