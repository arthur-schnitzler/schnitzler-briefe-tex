%% latex-leseansicht-vorspann.tex
%% Vorspann für die Leseansicht.
%% Lädt die gemeinsame Datei latex-vorspann.tex mit nicht gesetztem Schalter.

\newif\ifkorrekturansicht
\korrekturansichtfalse

\input{../tex-inputs/latex-vorspann}


\section[ Felix Salten an Arthur Schnitzler, 1. 5. 1906]{L03422 Felix Salten an Arthur Schnitzler,  1. 5. 1906}
\nopagebreak\mylabel{L03422v}
\rehead{ }\normalsize\beginnumbering\briefempfaengerindex{Schnitzler, Arthur@\textsc{Schnitzler, Arthur}!zzzSalten, Felix@\emph{von Felix Salten}!1906-05-011@{1. 5. 1906}|(be}
\toendnotes[C]{\smallbreak\pagebreak[2]}
\correspDesc{Versand  durch Felix Salten am 1. 5. 1906 in Berlin
\newline{}Erhalt  durch Arthur Schnitzler im Zeitraum [2. 5. 1906
                  – 6. 5. 1906?] in Wien}\toendnotes[C]{\smallbreak}
\Standort{CUL, Schnitzler, B 89, B 1.}
\physDesc{Brief, 1 Blatt, 2 Seiten, 3410 Zeichen
\newline{}Handschrift: schwarze Tinte, lateinische Kurrent
\newline{}Schnitzler: mit rotem Buntstift eine Unterstreichung 
\newline{}Ordnung: mit Bleistift von unbekannter Hand nummeriert: »212« }\toendnotes[C]{\smallbreak}
\pstart
           \raggedleft{}{\pb}Berlin\oindex{Berlin@\textbf{Berlin}, \emph{Hauptstadt}|pw}, 1. Mai 06.\pend
           \vspace{0.5em}
\pstart
           Lieber, die \label{K_L03422-1v}\edtext{Radpartie}{\lemma{\textnormal{\emph{Radpartie}}}\Cendnote{\textnormal{Siehe XXXX Auszeichnungsfehler: Dokument L03416 nicht gefunden.
               }}}\label{K_L03422-1}, ja, wenn ich heute nur wüßte, wie und was in drei, vier Wochen sein wird.
               Ich fürchte, die Radpartie wird sich nicht machen laßen. Vorläufig nämlich ist es
               beschloßen, dass ich am 20. od. 21. nach Madrid\oindex{Madrid@\textbf{Madrid}, \emph{Hauptstadt}|pw}
               fahre, zur \label{K_L03422-2v}\edtext{Königshochzeit\pwindex{Alfons XIII. 17.\,5.\,1886 Madrid – 18.\,2.\,1941 Rom@\textsc{Alfons XIII.} (17.\,5.\,1886 Madrid – 18.\,2.\,1941 Rom), \emph{König}|pwv}\pwindex{Victoria Eugénie von Spanien 24.\,10.\,1887 Schloss Balmoral – 15.\,4.\,1969 Lausanne@\textsc{Victoria Eugénie von Spanien} (24.\,10.\,1887 Schloss Balmoral – 15.\,4.\,1969 Lausanne), \emph{Regentin}|pwv}}{\lemma{\textnormal{\emph{Königshochzeit}}}\Cendnote{\textnormal{Am 17. 5. 1906 heirateten in Madrid\oindex{Madrid@\textbf{Madrid}, \emph{Hauptstadt}|pwk}
                  König Alfonso XIII. von Spanien\pwindex{Alfons XIII. 17.\,5.\,1886 Madrid – 18.\,2.\,1941 Rom@\textsc{Alfons XIII.} (17.\,5.\,1886 Madrid – 18.\,2.\,1941 Rom), \emph{König}|pwk} und Victoria Eugénie von Battenberg\pwindex{Victoria Eugénie von Spanien 24.\,10.\,1887 Schloss Balmoral – 15.\,4.\,1969 Lausanne@\textsc{Victoria Eugénie von Spanien} (24.\,10.\,1887 Schloss Balmoral – 15.\,4.\,1969 Lausanne), \emph{Regentin}|pwk}.}}}\label{K_L03422-2}. Da
               käme ich erst am 10. Juni wieder zurück, weil ich
               natürlich Toledo\oindex{Toledo@\textbf{Toledo}, \emph{Verwaltungsgebiet}|pw}, Sevilla\oindex{Sevilla@\textbf{Sevilla}|pw}, Cadiz\oindex{Cadiz@\textbf{Cadiz}, \emph{Verwaltungsgebiet}|pw}, Tanger\oindex{Tanger@\textbf{Tanger}, \emph{Verwaltungsgebiet}|pw}, Gibraltar\oindex{Gibraltar@\textbf{Gibraltar}, \emph{Exterritoriales Gebiet}|pw}, Granada\oindex{Granada@\textbf{Granada}, \emph{Verwaltungsgebiet}|pw} mitnehme, und der
               Weg zurück über Lissabon\oindex{Lissabon@\textbf{Lissabon}, \emph{Hauptstadt}|pw} führe. Da gäbe es dann
               – ausser dem contractlichen Urlaub – keine Absenz mehr. Und die vier Wochen im Juli will ich still an einem Fleck sitzen, Tennis spielen
               und arbeiten. (Ich bin im Begriff, die \label{K_L03422-3v}\edtext{Herzl\pwindex{Herzl, Theodor 2.\,5.\,1860 Budapest – 3.\,7.\,1904 Edlach@\textsc{Herzl, Theodor} (2.\,5.\,1860 Budapest – 3.\,7.\,1904 Edlach), \emph{Schriftsteller, Journalist}|pw}-Biographie}{\lemma{\textnormal{\emph{Herzl-Biographie}}}\Cendnote{\textnormal{Eine Biografie Herzls\pwindex{Herzl, Theodor 2.\,5.\,1860 Budapest – 3.\,7.\,1904 Edlach@\textsc{Herzl, Theodor} (2.\,5.\,1860 Budapest – 3.\,7.\,1904 Edlach), \emph{Schriftsteller, Journalist}|pwk}
                  wurde von Salten\pwindex{Salten, Felix 6.\,9.\,1869 Budapest – 8.\,10.\,1945 Zürich@\textsc{Salten, Felix} (6.\,9.\,1869 Budapest – 8.\,10.\,1945 Zürich), \emph{Schriftsteller, Journalist, Chefredakteur}|pwk} nie geschrieben.}}}\label{K_L03422-3} zu
               übernehmen, was ich mir als eine Art von Denkmal-Portrait sehr schön denke.) Mit dem
                  \label{K_L03422-4v}\edtext{Seebad}{\lemma{\textnormal{\emph{Seebad}}}\Cendnote{\textnormal{Siehe XXXX Auszeichnungsfehler: Dokument L03416 nicht gefunden.
               }}}\label{K_L03422-4} ist das so: wir müßen doch im Juni schon aufs
               Land, der Kinder\pwindex{Rehmann, Anna Katharina 18.\,8.\,1904 Wien – 27.\,3.\,1977 Zürich@\textsc{Rehmann, Anna Katharina} (18.\,8.\,1904 Wien – 27.\,3.\,1977 Zürich), \emph{Schauspielerin, Übersetzerin}|pwv}\pwindex{Salten, Paul 11.\,8.\,1903 Wien – 8.\,5.\,1937 ebd.@\textsc{Salten, Paul} (11.\,8.\,1903 Wien – 8.\,5.\,1937 ebd.), \emph{Filmcutter}|pwv}
               wegen. Otti\pwindex{Salten, Ottilie 7.\,3.\,1868 Prag – 22.\,6.\,1942 Zürich@\textsc{Salten, Ottilie} (7.\,3.\,1868 Prag – 22.\,6.\,1942 Zürich), \emph{Schauspielerin}|pw} und die Kinder\pwindex{Rehmann, Anna Katharina 18.\,8.\,1904 Wien – 27.\,3.\,1977 Zürich@\textsc{Rehmann, Anna Katharina} (18.\,8.\,1904 Wien – 27.\,3.\,1977 Zürich), \emph{Schauspielerin, Übersetzerin}|pwv}\pwindex{Salten, Paul 11.\,8.\,1903 Wien – 8.\,5.\,1937 ebd.@\textsc{Salten, Paul} (11.\,8.\,1903 Wien – 8.\,5.\,1937 ebd.), \emph{Filmcutter}|pwv} gehen Juni, Juli, August, bis Mitte September an die See\oindex{Ostsee@\textbf{Ostsee}|pwv}. Da wird eine Wohnung
               genommen und Wirtschaft geführt. Möglichst nahe, damit ich über Sonntag einmal hin,
                  Otti\pwindex{Salten, Ottilie 7.\,3.\,1868 Prag – 22.\,6.\,1942 Zürich@\textsc{Salten, Ottilie} (7.\,3.\,1868 Prag – 22.\,6.\,1942 Zürich), \emph{Schauspielerin}|pw} manchmal zu mir in die Stadt\oindex{Berlin@\textbf{Berlin}, \emph{Hauptstadt}|pwv} kommen kann. Also Bansin\oindex{Bansin@\textbf{Bansin}|pw}, Swinemünde\oindex{Świnoujście@\textbf{Świnoujście}, \emph{Hauptstadt}|pw} oder Heringsdorf\oindex{Heringsdorf@\textbf{Heringsdorf}, \emph{Hauptstadt}|pw}. \uline{Deshalb} kann ich dann für den Juli nicht alles nach Skodsborg\oindex{Skodsborg@\textbf{Skodsborg}|pw}
               verlegen. Es ist einfach eine Sache des Geldes. Und bin ich selbst frei, möchte ich
               doch bei den Kindern\pwindex{Rehmann, Anna Katharina 18.\,8.\,1904 Wien – 27.\,3.\,1977 Zürich@\textsc{Rehmann, Anna Katharina} (18.\,8.\,1904 Wien – 27.\,3.\,1977 Zürich), \emph{Schauspielerin, Übersetzerin}|pwv}\pwindex{Salten, Paul 11.\,8.\,1903 Wien – 8.\,5.\,1937 ebd.@\textsc{Salten, Paul} (11.\,8.\,1903 Wien – 8.\,5.\,1937 ebd.), \emph{Filmcutter}|pwv} sein.\pend
           
\pstart
           Wenn sich die spani\oindex{Spanien@\textbf{Spanien}|pwv}sche Reise
               nun doch nicht macht, schreibe ich Ihnen rechtzeitig wegen der Radtour.\pend
           
\pstart
           Mein \label{K_L03422-5v}\edtext{Brief an Hugo\pwindex{Hofmannsthal, Hugo von 1.\,2.\,1874 Wien – 15.\,7.\,1929 Rodaun@\textsc{Hofmannsthal, Hugo von} (1.\,2.\,1874 Wien – 15.\,7.\,1929 Rodaun), \emph{Schriftsteller}|pw} mit der starken Verstimmung}{\lemma{\textnormal{\emph{Brief … Verstimmung}}}\Cendnote{\textnormal{»Ich habe alle die Fremdheiten dieses Landes jetzt zu
                     verdauen, und alle die Bräuche, Zustände u. s. w. durch die es mich enttäuscht,
                     irgendwie zur Kenntnis zu nehmen. Thatsächlich lebt man hier in russischen
                     Verhältnissen, lebt in einem Polizeistaat, in welchem die Menschen auf eine
                     ekelerregende Weise von Demut zur Frechheit, von Furcht zur Rohheit taumeln.
                     Alle führen die Worte: ›Zuverläßigkeit‹, ›Wahrheit‹, ›Treue‹ u. s. w. beständig
                     im Mund, und alle sind unzuverläßig, verlogen, treulos. Es ist ein Preussen\oindex{Preußen@\textbf{Preußen}|pw}, wie es \uline{vor}{ }Hardenberg\pwindex{Hardenberg, Karl August von 31.\,5.\,1750 Essenrode – 26.\,11.\,1822 Genua@\textsc{Hardenberg, Karl August von} (31.\,5.\,1750 Essenrode – 26.\,11.\,1822 Genua), \emph{Politiker}|pw} und Stein\pwindex{Stein, Heinrich Friedrich Karl vom und zum 25.\,10.\,1757 Nassau – 29.\,6.\,1831 Cappenberg@\textsc{Stein, Heinrich Friedrich Karl vom und zum} (25.\,10.\,1757 Nassau – 29.\,6.\,1831 Cappenberg), \emph{Politiker}|pw}, wie es vor Jena\oindex{Jena@\textbf{Jena}, \emph{Hauptstadt}|pw} und Auerstädt\oindex{Auerstedt@\textbf{Auerstedt}|pw} gewesen:
                     corrupt, niedrig, schandbar.« Felix Salten an Hugo von Hofmannsthal\pwindex{Hofmannsthal, Hugo von 1.\,2.\,1874 Wien – 15.\,7.\,1929 Rodaun@\textsc{Hofmannsthal, Hugo von} (1.\,2.\,1874 Wien – 15.\,7.\,1929 Rodaun), \emph{Schriftsteller}|pwk},
                        9. 3. 1906, \emph{Freies Deutsches
                        Hochstift}, Hs-30865,25. Zit. n. Marcel Atze: \emph{»Unser aller Feldmarschall mit der Feder«. Felix Saltens halbes Jahrhundert
                        als Journalist.} In: Marcel Atze, unter Mitarbeit von Tanja Gausterer
                     (Herausgeber): \emph{Im Schatten von Bambi. Felix Salten entdeckt die Wiener
                        Moderne. Leben und Werk}.
                     Salzburg/Wien:
                        \emph{Residenz}{ }2020, S. 260–289, hier 281.}}}\label{K_L03422-5}
               gegen Berlin\oindex{Berlin@\textbf{Berlin}, \emph{Hauptstadt}|pw} datirt weit zurück, war im März noch geschrieben, während er in Italien\oindex{Italien@\textbf{Italien}|pw} war. Seither hat sich die Sache genau um die
               Frühlingssonne verbessert. Ich schreibe selten, weil ich mit organisatorischen
               Arbeiten beschäftigt bin, weil ich productiv einiges componire, und die Stadt\oindex{Berlin@\textbf{Berlin}, \emph{Hauptstadt}|pwv} noch zu wenig als
               publizistische Anregung fühle. Es würden Reisebriefe werden, und das wäre falsch. Ich
               bin froh, dass mich meine Selbstcontrolle {\pb}vor solchen Verfehlungen ebenso
               wie vor allzufrühen, taktlosen Vertraulichkeiten mit dieser Stadt\oindex{Berlin@\textbf{Berlin}, \emph{Hauptstadt}|pwv} bewahrt.\pend
           
\pstart
           Wie \label{K_L03422-6v}\edtext{Herr Wenzel\pwindex{Salten, Felix 6.\,9.\,1869 Budapest – 8.\,10.\,1945 Zürich@\textsc{Salten, Felix} (6.\,9.\,1869 Budapest – 8.\,10.\,1945 Zürich), \emph{Schriftsteller, Journalist, Chefredakteur}!Herr Wenzel auf Rehberg. Novelle@\strich\emph{Herr Wenzel auf Rehberg. Novelle}|pw}}{\lemma{\textnormal{\emph{Herr Wenzel}}}\Cendnote{\textnormal{Felix Salten\pwindex{Salten, Felix 6.\,9.\,1869 Budapest – 8.\,10.\,1945 Zürich@\textsc{Salten, Felix} (6.\,9.\,1869 Budapest – 8.\,10.\,1945 Zürich), \emph{Schriftsteller, Journalist, Chefredakteur}|pwk}: \emph{Herr Wenzel auf Rehberg. Novelle}\pwindex{Salten, Felix 6.\,9.\,1869 Budapest – 8.\,10.\,1945 Zürich@\textsc{Salten, Felix} (6.\,9.\,1869 Budapest – 8.\,10.\,1945 Zürich), \emph{Schriftsteller, Journalist, Chefredakteur}!Herr Wenzel auf Rehberg. Novelle@\strich\emph{Herr Wenzel auf Rehberg. Novelle}|pwk}. In: \emph{Die neue Rundschau}\pwindex{neue Rundschau@\emph{Die neue Rundschau}|pwk}, Jg. 17, H. 5, Mai 1906, S. 544–576.}}}\label{K_L03422-6} aufgenommen wird, bin ich
               neugierig. Es ist das erstemal, dass ich eine Novelle von mir in der Correctur ohne
               Desperation und tiefe Niedergeschlagenheit lesen konnte.\pend
           
\pstart
           Mein Verkehr hier? Ab und zu Heimann\pwindex{Heimann, Moritz 19.\,7.\,1868 Werder – 22.\,9.\,1925 Berlin@\textsc{Heimann, Moritz} (19.\,7.\,1868 Werder – 22.\,9.\,1925 Berlin), \emph{Schriftsteller, Verlagslektor}|pw}, Jakobsohn\pwindex{Jacobsohn, Siegfried 28.\,1.\,1881 Berlin – 3.\,12.\,1926 ebd.@\textsc{Jacobsohn, Siegfried} (28.\,1.\,1881 Berlin – 3.\,12.\,1926 ebd.), \emph{Journalist, Kritiker, Publizist}|pw}. Dann Rittner\pwindex{Rittner, Rudolf 30.\,6.\,1869 Bílý Potok – 4.\,2.\,1943 ebd.@\textsc{Rittner, Rudolf} (30.\,6.\,1869 Bílý Potok – 4.\,2.\,1943 ebd.), \emph{Theaterleiter, Schauspieler}|pw}. Und Fischers\pwindex{Fischer, Hedwig 8.\,9.\,1871 Szczecin – 11.\,4.\,1952 Königstein im Taunus@\textsc{Fischer, Hedwig} (8.\,9.\,1871 Szczecin – 11.\,4.\,1952 Königstein im Taunus)|pw}\pwindex{Fischer, Samuel 24.\,12.\,1859 Liptovský Mikuláš – 15.\,10.\,1934 Berlin@\textsc{Fischer, Samuel} (24.\,12.\,1859 Liptovský Mikuláš – 15.\,10.\,1934 Berlin), \emph{Verleger}|pw}, die mir aus der Nähe immer sympathischer werden. Selten Reinhardt\pwindex{Reinhardt, Max 9.\,9.\,1873 Baden bei Wien – 30.\,10.\,1943 New York City@\textsc{Reinhardt, Max} (9.\,9.\,1873 Baden bei Wien – 30.\,10.\,1943 New York City), \emph{Theaterleiter, Regisseur, Schauspieler}|pw} und seine Leute, manchmal Bie\pwindex{Bie, Oskar 9.\,2.\,1864 Breslau – 21.\,4.\,1938 Berlin@\textsc{Bie, Oskar} (9.\,2.\,1864 Breslau – 21.\,4.\,1938 Berlin), \emph{Schriftsteller, Journalist, Redakteur}|pw} (sehr lieb und fein) und Poppenberg\pwindex{Poppenberg, Felix 13.\,10.\,1869 Charlottenburg – 18.\,10.\,1915 ebd.@\textsc{Poppenberg, Felix} (13.\,10.\,1869 Charlottenburg – 18.\,10.\,1915 ebd.), \emph{Schriftsteller, Kritiker}|pw}, zwei, drei lange Gespräche mit Kerr\pwindex{Kerr, Alfred 25.\,12.\,1867 Breslau – 12.\,10.\,1948 Hamburg@\textsc{Kerr, Alfred} (25.\,12.\,1867 Breslau – 12.\,10.\,1948 Hamburg), \emph{Schriftsteller, Kritiker}|pw}; fast garnicht mehr Harden\pwindex{Harden, Maximilian 20.\,10.\,1861 Berlin – 30.\,10.\,1927 Montana@\textsc{Harden, Maximilian} (20.\,10.\,1861 Berlin – 30.\,10.\,1927 Montana), \emph{Schriftsteller, Publizist}|pw}. Dazwischen die Gesellschaften, denen sich nicht ausweichen läßt. Bei
               meinem Schwager\pwindex{Metzl, Richard 20.\,4.\,1870 Prag – 31.\,10.\,1941 Paris@\textsc{Metzl, Richard} (20.\,4.\,1870 Prag – 31.\,10.\,1941 Paris), \emph{Regisseur, Schauspieler, Theatersekretär}|pwv} Musikleute:
                  Safonoff\pwindex{Safonov, Vasilij Ilʹič 6.\,2.\,1852 Ishcherskaya – 27.\,2.\,1918 Kislovodsk@\textsc{Safonov, Vasilij Ilʹič} (6.\,2.\,1852 Ishcherskaya – 27.\,2.\,1918 Kislovodsk), \emph{Dirigent, Pianist, Musiker}|pw}, Godowski\pwindex{Godowsky, Leopold 13.\,2.\,1870 Vilnius – 21.\,11.\,1938 New York City@\textsc{Godowsky, Leopold} (13.\,2.\,1870 Vilnius – 21.\,11.\,1938 New York City), \emph{Komponist, Pianist}|pw}, Nikisch\pwindex{Nikisch, Arthur 12.\,10.\,1855 Mosonszentmiklós – 23.\,1.\,1922 Leipzig@\textsc{Nikisch, Arthur} (12.\,10.\,1855 Mosonszentmiklós – 23.\,1.\,1922 Leipzig), \emph{Dirigent}|pw}, Kreisler\pwindex{Kreisler, Fritz 2.\,2.\,1875 Wien – 29.\,1.\,1962 New York City@\textsc{Kreisler, Fritz} (2.\,2.\,1875 Wien – 29.\,1.\,1962 New York City), \emph{Komponist, Pianist}|pw}. Hie und da eine ärgerliche, manchmal eine nette
               Stunde mit Frau Fulda\pwindex{d’Albert, Ida 5.\,12.\,1869 Wien – 6.\,10.\,1926 Berlin@\textsc{d’Albert, Ida} (5.\,12.\,1869 Wien – 6.\,10.\,1926 Berlin), \emph{Schauspielerin}|pw}. Das ist alles; ist
               genug, ist – gelegentlich sogar zu viel. Ich will lieber lesen, will jetzt viel, sehr
               viel lesen; lerne ein bischen spanisch und gehe mit Otti\pwindex{Salten, Ottilie 7.\,3.\,1868 Prag – 22.\,6.\,1942 Zürich@\textsc{Salten, Ottilie} (7.\,3.\,1868 Prag – 22.\,6.\,1942 Zürich), \emph{Schauspielerin}|pw} im Thiergarten\oindex{Tiergarten@\textbf{Tiergarten}, \emph{Ehemaliger Ort}|pw} spazieren, wo es –
               unglaublich aber wahr – gerade jetzt einfach märchenhaft schön ist.\pend
           
\pstart
           Otti\pwindex{Salten, Ottilie 7.\,3.\,1868 Prag – 22.\,6.\,1942 Zürich@\textsc{Salten, Ottilie} (7.\,3.\,1868 Prag – 22.\,6.\,1942 Zürich), \emph{Schauspielerin}|pw} läßt Frau Olga\pwindex{Schnitzler, Olga 17.\,1.\,1882 Wien – 13.\,1.\,1970 Lugano@\textsc{Schnitzler, Olga} (17.\,1.\,1882 Wien – 13.\,1.\,1970 Lugano), \emph{Schauspielerin, Sängerin}|pw} um Entschuldigung bitten, weil sie ihren lieben Brief
               noch nicht beantworten konnte. Sie hat sich erst die linke Hand verbrannt, und kaum
               die halbwegs gut war, wieder die rechte verbrüht. Da wir nicht hoffen, dass sie jetzt
               wieder von vorne anfängt, rechnen wir darauf, dass sie bald wieder den Gebrauch all
               ihrer Gliedmaßen erlangt. Die Kinder\pwindex{Rehmann, Anna Katharina 18.\,8.\,1904 Wien – 27.\,3.\,1977 Zürich@\textsc{Rehmann, Anna Katharina} (18.\,8.\,1904 Wien – 27.\,3.\,1977 Zürich), \emph{Schauspielerin, Übersetzerin}|pwv}\pwindex{Salten, Paul 11.\,8.\,1903 Wien – 8.\,5.\,1937 ebd.@\textsc{Salten, Paul} (11.\,8.\,1903 Wien – 8.\,5.\,1937 ebd.), \emph{Filmcutter}|pwv} sind reizend, und wir alle grüßen Sie alle\pwindex{Schnitzler, Olga 17.\,1.\,1882 Wien – 13.\,1.\,1970 Lugano@\textsc{Schnitzler, Olga} (17.\,1.\,1882 Wien – 13.\,1.\,1970 Lugano), \emph{Schauspielerin, Sängerin}|pwv}\pwindex{Schnitzler, Heinrich 9.\,8.\,1902 Hinterbrühl – 12.\,7.\,1982 Wien@\textsc{Schnitzler, Heinrich} (9.\,8.\,1902 Hinterbrühl – 12.\,7.\,1982 Wien), \emph{Regisseur, Schauspieler}|pwv} aufs Herzlichste.\pend
           \pstart Ihr \spacefill\mbox{Salten}\pend{}
\pstart
           \noindent{}\label{K_L03422-7v}\edtext{\textcolor{gray}{NB}}{\lemma{\textnormal{\emph{NB}}}\Cendnote{\textnormal{nota bene; lateinisch: merke
                     wohl}}}\label{K_L03422-7}. Heute sahen wir Ludaßy\pwindex{Gans-Ludassy, Julius von 13.\,4.\,1858 Wien – 30.\,9.\,1922 ebd.@\textsc{Gans-Ludassy, Julius von} (13.\,4.\,1858 Wien – 30.\,9.\,1922 ebd.), \emph{Schriftsteller, Journalist, Herausgeber}|pw} in der Friedrichstraße\oindex{Friedrichstraße [Berlin]@\textbf{Friedrichstraße [Berlin]}, \emph{Straße}|pw}. Wir haben sehr gestaunt, weil wir dachten, er sei – wie
                  lange schon! – gestorben.\pend
           
\pstart
           D\textsuperscript{r}Ginsberg\pwindex{Ginsberg, Herbert 27.\,9.\,1881 Berlin – 5.\,11.\,1962 New York City@\textsc{Ginsberg, Herbert} (27.\,9.\,1881 Berlin – 5.\,11.\,1962 New York City), \emph{Industrieller, Kunstsammler, Bankier}|pw}
                  schrieb mir sehr entzückt über die freundl. \label{K_L03422-8v}\edtext{Aufnahme}{\lemma{\textnormal{\emph{Aufnahme}}}\Cendnote{\textnormal{Siehe XXXX Auszeichnungsfehler: Dokument L03417 nicht gefunden.
                  }}}\label{K_L03422-8} bei Ihnen. Vielen Dank!\pend
           \selectlanguage{ngerman}\endnumbering\briefempfaengerindex{Schnitzler, Arthur@\textsc{Schnitzler, Arthur}!zzzSalten, Felix@\emph{von Felix Salten}!1906-05-011@{1. 5. 1906}|)be}\mylabel{L03422h}  \newcommand{\dateiname}{L03422}\newcommand{\titel}{Felix Salten an Arthur Schnitzler, 1. 5. 1906}\newcommand{\editorInnen}{Martin Anton Müller und Laura Untner}%% latex-leseansicht-abspann.tex
%% Abspann für die Leseansicht.
%% Der Schalter \ifkorrekturansicht ist bereits durch den Vorspann gesetzt.

%% latex-abspann.tex
%% Gemeinsamer Abspann für Korrekturansicht und Leseansicht.
%% Setzt den Schalter \ifkorrekturansicht voraus (gesetzt in den
%% einbindenden Dateien latex-korrekturansicht-abspann.tex bzw.
%% latex-leseansicht-abspann.tex).
%% ---------------------------------------------------------------

\normalsize

% Das esempio-Environment wird nur in der Leseansicht benötigt
\ifkorrekturansicht\else
\newenvironment{esempio}[3]%
{
    \vspace{1.5ex}
    \rlap{\underline{#1}}
    \par
    \setlength{\parindent}{0cm}
    \nopagebreak
    \leftskip=#2cm
    \rightskip=#3cm
}
{
    \par
}
\fi

\doendnotes{C}
\bigskip
\vfill

\clearpage

\footnotesize

\ifkorrekturansicht
  \lohead{\textsc{register}}
\fi

% theindex-Environment neu definieren ohne reledmac
\makeatletter
\renewenvironment{theindex}{%
  \ifkorrekturansicht
    \section*{\indexname}%
  \else
    \subsubsection*{Index der erwähnten Entitäten}%
  \fi
  \setlength{\parindent}{0pt}%
  \setlength{\parskip}{0pt plus 0.3pt}%
  \let\item\@idxitem
}{%
  \ifkorrekturansicht\clearpage\fi
}
\makeatother

\IfFileExists{\jobname-pw.ind}{\input{\jobname-pw.ind}}{}

% Quellenangabe nur in der Leseansicht
\ifkorrekturansicht\else
% Fallback-Definitionen, falls die .tex-Datei \titel etc. nicht gesetzt hat
\providecommand{\titel}{}
\providecommand{\editorInnen}{}
\providecommand{\dateiname}{\jobname}

\vspace{3cm}

\vfill

\footnotesize
\textsc{Quelle}: \titel. Herausgegeben von {\editorInnen}. In: \emph{Arthur Schnitzler: Briefwechsel mit Autorinnen und Autoren}.
 Digitale Edition, https://schnitzler-briefe.acdh.oeaw.ac.at/{\dateiname}.html (Stand \today)
\fi

\end{document}


