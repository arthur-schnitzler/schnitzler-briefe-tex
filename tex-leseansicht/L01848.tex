%% latex-korrekturansicht-vorspann.tex
%% Vorspann für die Korrekturansicht.
%% Lädt die gemeinsame Datei latex-vorspann.tex mit gesetztem Schalter.

\newif\ifkorrekturansicht
\korrekturansichttrue

\input{../tex-inputs/latex-vorspann}


\section[Arthur Schnitzler an Hermann Bahr, 22. 6. 1909]{L01848 Arthur Schnitzler an Hermann Bahr, 22. 6. 1909}
\nopagebreak\mylabel{L01848v}
\rehead{ }\normalsize\beginnumbering\briefempfaengerindex{Bahr, Hermann@\textsc{Bahr, Hermann}!zzzSchnitzler, Arthur@\emph{von Arthur Schnitzler}!1909-06-221@{22. 6. 1909}|(be}
\toendnotes[C]{\smallbreak\pagebreak[2]}\Standort{TMW, HS AM 60167 Ba.}
\physDesc{Briefkarte, 2 Blätter, 3 Seiten, , 799 Zeichen
\newline{}Handschrift: schwarze Tinte, deutsche Kurrent
\newline{}Ordnung: Lochung }
\buchAbdrucke{\weitereDrucke{1) Arthur Schnitzler: \emph{The Letters of Arthur Schnitzler to Hermann Bahr}. Chapel Hill: \emph{The University of North Carolina Press} 1978, S. 103.} \weitereDrucke{2) Hermann Bahr, Arthur Schnitzler: \emph{Briefwechsel, Aufzeichnungen, Dokumente (1891–1931)}. Göttingen: \emph{Wallstein} 2018, S. 418.} }\toendnotes[C]{\smallbreak}
\pstart
           {\pb}\textcolor{gray}{\textbf{Dr. Arthur Schnitzler}}\hfill 22. 6. 09\pend
           
\pstart
           \textcolor{gray}{\textbf{Wien XVIII. Spoettelgasse 7\oindex{Edmund-Weiss-Gasse 7@\textbf{Edmund-Weiß-Gasse 7}, \emph{Wohngebäude (K.WHS)}|pw}.}}\pend
           \vspace{0.5em}
\pstart
           mein lieber Herma{\geminationn}, geſtern iſt das Tagebuch\pwindex{Tagebuch [Berlin: Paul Cassirer]@\emph{Tagebuch [Berlin: Paul Cassirer]}|pw} geko{\geminationm}en und
               neulich die Drut\pwindex{Drut. Roman@\emph{Drut. Roman}|pw}, die meine Frau\pwindex{Schnitzler, Olga 17.01.1882 – 13.01.1970@\textsc{Schnitzler, Olga} (17.01.1882 – 13.01.1970), \emph{Schauspieler/Schauspielerin, Sänger/Sängerin}|pwv}{ }ſofort für ſich beanſprucht und mit großem
               Entzücken geleſen hat. Auch Burkhard\pwindex{Burckhard, Max Eugen 14.07.1854 – 16.03.1912@\textsc{Burckhard, Max Eugen} (14.07.1854 – 16.03.1912), \emph{Schriftsteller/Schriftstellerin, Rechtswissenschaftler/Rechtswissenschaftlerin, Theaterleiter/Theaterleiterin}|pw} hat mir
               in \textsc{St Gilgen\oindex{St. Gilgen@\textbf{St. Gilgen}, \emph{A.ADM3}|pw}} viel ſchönes darüber geſagt. Ja ſo ſpricht man übereinander und ſieht und
               ſpricht ſich nie. \label{K_L01848-1v}\edtext{Einer w\damage{ird}{ }{\pb}übrig bleiben und
               ſagen: »{\dots}{ }Schade{\dotsfour}«}{\lemma{\textnormal{\emph{Einer … Schade«}}}\Cendnote{\textnormal{Vgl. Hermann Bahr an Arthur Schnitzler, 28. 6. 1909, Arthur Schnitzler an Hermann Bahr, 16. 2. 1930.
               }}}\label{K_L01848-1}\pend
           
\pstart
           \damage{Wi}r ſind von Gilgen\oindex{St. Gilgen@\textbf{St. Gilgen}, \emph{A.ADM3}|pw} zurückgehetzt, weil
               unſer Bub\pwindex{Schnitzler, Heinrich 09.08.1902 – 12.07.1982@\textsc{Schnitzler, Heinrich} (09.08.1902 – 12.07.1982), \emph{Regisseur/Regisseurin, Schauspieler/Schauspielerin}|pwv} eine Art
               Keuchhuſten hat, recht leicht bis jetzt. Nächſte Woche fahren wir nach Edlach\oindex{Edlach@\textbf{Edlach}, \emph{P.PPL}|pw}, ich mit der Drut\pwindex{Drut. Roman@\emph{Drut. Roman}|pw} und dem Tagebuch\pwindex{Tagebuch [Berlin: Paul Cassirer]@\emph{Tagebuch [Berlin: Paul Cassirer]}|pw} und
               freu mich ſchon ſehr. Mit dem Danken ko{\geminationm}t man ja nicht
               nach bei dir. Ich war auch nicht ſehr faul – aber wie ko{\geminationm}t man ſich gegen dich vor! Mit Burckhard\pwindex{Burckhard, Max Eugen 14.07.1854 – 16.03.1912@\textsc{Burckhard, Max Eugen} (14.07.1854 – 16.03.1912), \emph{Schriftsteller/Schriftstellerin, Rechtswissenschaftler/Rechtswissenschaftlerin, Theaterleiter/Theaterleiterin}|pw} war
               ich auf ſeiner {\pb}Alm
               oben; ich finde es geht ihm recht gut, er war lebendig, fidel geradezu und jung.\pend
           
\pstart
           Wir\pwindex{Schnitzler, Olga 17.01.1882 – 13.01.1970@\textsc{Schnitzler, Olga} (17.01.1882 – 13.01.1970), \emph{Schauspieler/Schauspielerin, Sänger/Sängerin}|pwv} grüßen dich
               herzlichſt.{\\[\baselineskip]}Dein getreuer{\\[\baselineskip]}\spacefill\mbox{Arthur}\pend
           \leftskip=0em{}\selectlanguage{ngerman}\endnumbering\briefempfaengerindex{Bahr, Hermann@\textsc{Bahr, Hermann}!zzzSchnitzler, Arthur@\emph{von Arthur Schnitzler}!1909-06-221@{22. 6. 1909}|)be}\mylabel{L01848h}  \normalsize

\doendnotes{C}
\bigskip
\vfill

\clearpage

\footnotesize

\lohead{\textsc{register}}

% Definiere theindex-Environment komplett neu ohne reledmac
\makeatletter
\renewenvironment{theindex}{%
  \section*{\indexname}%
  \setlength{\parindent}{0pt}%
  \setlength{\parskip}{0pt plus 0.3pt}%
  \let\item\@idxitem
}{%
  \clearpage
}
\makeatother

\IfFileExists{\jobname-pw.ind}{\input{\jobname-pw.ind}}{}

\end{document}

      