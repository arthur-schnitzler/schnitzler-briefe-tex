%% latex-leseansicht-vorspann.tex
%% Vorspann für die Leseansicht.
%% Lädt die gemeinsame Datei latex-vorspann.tex mit nicht gesetztem Schalter.

\newif\ifkorrekturansicht
\korrekturansichtfalse

\input{../tex-inputs/latex-vorspann}


\section[Georg Brandes: Widmungsexemplar Armand Carrel für Arthur Schnitzler, {[}nach dem 16. 11. 1912{]}]{L02100 Georg Brandes: Widmungsexemplar Armand Carrel für Arthur Schnitzler, {[}nach dem 16. 11. 1912{]}}
\nopagebreak\mylabel{L02100v}
\rehead{ }\normalsize\beginnumbering\briefempfaengerindex{Schnitzler, Arthur@\textsc{Schnitzler, Arthur}!zzzBrandes, Georg@\emph{von Georg Brandes}!1912-12-312@{{[}nach dem 16. 11. 1912{]}}|(be}
\toendnotes[C]{\smallbreak\pagebreak[2]}
\correspDesc{Versand  durch Georg Brandes im Zeitraum [nach dem 16. 11. 1912] \textbf{Ort fehlend} 
\newline{}Erhalt  durch Arthur Schnitzler im Zeitraum [nach dem 16. 11. 1912] in Wien}\toendnotes[C]{\smallbreak}
\Standort{DLA, G:Schnitzler, Arthur (Sammlung Heinrich Schnitzler).}
\physDesc{Widmung am Vorsatzblatt, 128 Zeichen
\newline{}Handschrift: schwarze Tinte, lateinische Kurrent
\newline{}Ordnung: mit Bleistift von unbekannter Hand das Pseudonym der
                                 Übersetzerin aufgelöst: »Prager Mathilde\pwindex{Holm, Erich 3.\,1.\,1844 Prag – 1.\,2.\,1921 Wien@\textsc{Holm, Erich} (3.\,1.\,1844 Prag – 1.\,2.\,1921 Wien), \emph{Schriftstellerin, Übersetzerin, Literaturwissenschaftlerin}|pw}« }\toendnotes[C]{\smallbreak}
\pstart
           \noindent{}{\pb}An Arthur Schnitzler\pend
           
\pstart
           Diese Bagatelle, \label{K_L02100-1v}\edtext{Diomedes’ Geschenk an
                  Glaukos}{\lemma{\textnormal{\emph{Diomedes’ … Glaukos}}}\Cendnote{\textnormal{Glaukos erneuert den
                  Freundschaftsbund, er gibt Diomedes eine goldene, dieser ihm eine eherne
                  Rüstung.}}}\label{K_L02100-1}, (Ilias\pwindex{\textcolor{red}{\textsuperscript{XXXX indx1}}!Ilias@\strich\emph{Ilias}|pw} IV 235) soll nur ein
               Zeichen treuer Freundschaft\pend
           \pstart \spacefill\mbox{G.B.}\pend{}{\vspace{1\baselineskip}}
\pstart
           \centering{}\textcolor{gray}{\textbf{Armand Carrel\pwindex{Brandes, Georg 4.\,2.\,1842 Kopenhagen – 19.\,2.\,1927 ebd.@\textsc{Brandes, Georg} (4.\,2.\,1842 Kopenhagen – 19.\,2.\,1927 ebd.)!Armand Carrel@\strich\emph{Armand Carrel}|pw}}}\pend
           \selectlanguage{ngerman}\vspace{1em}{\vspace{1\baselineskip}}
\pstart
           \centering{}{\pb}\textcolor{gray}{\textbf{Armand Carrel\pwindex{Brandes, Georg 4.\,2.\,1842 Kopenhagen – 19.\,2.\,1927 ebd.@\textsc{Brandes, Georg} (4.\,2.\,1842 Kopenhagen – 19.\,2.\,1927 ebd.)!Armand Carrel@\strich\emph{Armand Carrel}|pw}}}\pend
           
\pstart
           \centering{}\textcolor{gray}{\textbf{Von}}\pend
           
\pstart
           \centering{}\textcolor{gray}{\textbf{Georg Brandes}}\pend
           
\pstart
           \centering{}\textcolor{gray}{\textbf{Autoriſierte Überſetzung von \so{Erich Holm}\pwindex{Holm, Erich 3.\,1.\,1844 Prag – 1.\,2.\,1921 Wien@\textsc{Holm, Erich} (3.\,1.\,1844 Prag – 1.\,2.\,1921 Wien), \emph{Schriftstellerin, Übersetzerin, Literaturwissenschaftlerin}|pw}}}\pend
           {\vspace{1\baselineskip}}
\pstart
           \centering{}\textcolor{gray}{\textbf{Stuttgart\oindex{Karlsruhe@\textbf{Karlsruhe}, \emph{Hauptstadt}|pw} und Berlin\oindex{Berlin@\textbf{Berlin}, \emph{Hauptstadt}|pw}{ }\label{K_L02100-2v}\edtext{1913}{\lemma{\textnormal{\emph{1913}}}\Cendnote{\textnormal{am 16. 11. 1912 vom \emph{Börsenblatt für den deutschen Buchhandel}\pwindex{Börsenblatt für den Deutschen Buchhandel@\emph{Börsenblatt für den Deutschen Buchhandel}|pwk}
                     als Neuerscheinung gemeldet}}}\label{K_L02100-2}}}\pend
           
\pstart
           \centering{}\textcolor{gray}{\textbf{J. G. Cotta’ſche Buchhandlung Nachfolger\orgindex{J.G. Cotta’sche Buchhandlung Nachfolger@J.G. Cotta’sche Buchhandlung Nachfolger|pw}}}\pend
           \selectlanguage{ngerman}\endnumbering\briefempfaengerindex{Schnitzler, Arthur@\textsc{Schnitzler, Arthur}!zzzBrandes, Georg@\emph{von Georg Brandes}!1912-11-162@{{[}nach dem 16. 11. 1912{]}}|)be}\mylabel{L02100h}  \newcommand{\dateiname}{L02100}\newcommand{\titel}{Georg Brandes: Widmungsexemplar Armand Carrel für Arthur Schnitzler, [nach dem 16. 11. 1912]}\newcommand{\editorInnen}{Martin Anton Müller und Gerd-Hermann Susen}%% latex-leseansicht-abspann.tex
%% Abspann für die Leseansicht.
%% Der Schalter \ifkorrekturansicht ist bereits durch den Vorspann gesetzt.

%% latex-abspann.tex
%% Gemeinsamer Abspann für Korrekturansicht und Leseansicht.
%% Setzt den Schalter \ifkorrekturansicht voraus (gesetzt in den
%% einbindenden Dateien latex-korrekturansicht-abspann.tex bzw.
%% latex-leseansicht-abspann.tex).
%% ---------------------------------------------------------------

\normalsize

% Das esempio-Environment wird nur in der Leseansicht benötigt
\ifkorrekturansicht\else
\newenvironment{esempio}[3]%
{
    \vspace{1.5ex}
    \rlap{\underline{#1}}
    \par
    \setlength{\parindent}{0cm}
    \nopagebreak
    \leftskip=#2cm
    \rightskip=#3cm
}
{
    \par
}
\fi

\doendnotes{C}
\bigskip
\vfill

\clearpage

\footnotesize

\ifkorrekturansicht
  \lohead{\textsc{register}}
\fi

% theindex-Environment neu definieren ohne reledmac
\makeatletter
\renewenvironment{theindex}{%
  \ifkorrekturansicht
    \section*{\indexname}%
  \else
    \subsubsection*{Index der erwähnten Entitäten}%
  \fi
  \setlength{\parindent}{0pt}%
  \setlength{\parskip}{0pt plus 0.3pt}%
  \let\item\@idxitem
}{%
  \ifkorrekturansicht\clearpage\fi
}
\makeatother

\IfFileExists{\jobname-pw.ind}{\input{\jobname-pw.ind}}{}

% Quellenangabe nur in der Leseansicht
\ifkorrekturansicht\else
% Fallback-Definitionen, falls die .tex-Datei \titel etc. nicht gesetzt hat
\providecommand{\titel}{}
\providecommand{\editorInnen}{}
\providecommand{\dateiname}{\jobname}

\vspace{3cm}

\vfill

\footnotesize
\textsc{Quelle}: \titel. Herausgegeben von {\editorInnen}. In: \emph{Arthur Schnitzler: Briefwechsel mit Autorinnen und Autoren}.
 Digitale Edition, https://schnitzler-briefe.acdh.oeaw.ac.at/{\dateiname}.html (Stand \today)
\fi

\end{document}


