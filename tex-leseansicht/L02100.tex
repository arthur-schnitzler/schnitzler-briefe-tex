%% latex-korrekturansicht-vorspann.tex
%% Vorspann für die Korrekturansicht.
%% Lädt die gemeinsame Datei latex-vorspann.tex mit gesetztem Schalter.

\newif\ifkorrekturansicht
\korrekturansichttrue

\input{../tex-inputs/latex-vorspann}


\section[Georg Brandes: Widmungsexemplar Armand Carrel für Arthur Schnitzler, {[}nach dem 16. 11. 1912{]}]{L02100 Georg Brandes: Widmungsexemplar Armand Carrel für Arthur Schnitzler,
               {[}nach dem 16. 11. 1912{]}}
\nopagebreak\mylabel{L02100v}
\rehead{ }\normalsize\beginnumbering\briefempfaengerindex{Schnitzler, Arthur@\textsc{Schnitzler, Arthur}!zzzBrandes, Georg@\emph{von Georg Brandes}!1912-12-312@{{[}nach dem 16. 11. 1912{]}}|(be}
\toendnotes[C]{\smallbreak\pagebreak[2]}\Standort{DLA, G:Schnitzler, Arthur (Sammlung Heinrich Schnitzler).}
\physDesc{Widmung am Vorsatzblatt, 128 Zeichen
\newline{}Handschrift: schwarze Tinte, lateinische Kurrent
\newline{}Ordnung: mit Bleistift von unbekannter Hand das Pseudonym der
                                 Übersetzerin aufgelöst: »Prager Mathilde\pwindex{Holm, Erich 1844-01-03 – 1921-02-01@\textsc{Holm, Erich} (1844-01-03 – 1921-02-01), \emph{Schriftsteller/Schriftstellerin, Übersetzer/Übersetzerin, Literaturwissenschaftler/Literaturwissenschaftlerin}|pw}« }\toendnotes[C]{\smallbreak}
\pstart
           \noindent{}{\pb}An Arthur Schnitzler\pend
           
\pstart
           Diese Bagatelle, \label{K_L02100-1v}\edtext{Diomedes’ Geschenk an
                  Glaukos}{\lemma{\textnormal{\emph{Diomedes’ … Glaukos}}}\Cendnote{\textnormal{Glaukos erneuert den
                  Freundschaftsbund, er gibt Diomedes eine goldene, dieser ihm eine eherne
                  Rüstung.}}}\label{K_L02100-1}, (Ilias\pwindex{Ilias@\emph{Ilias}|pw} IV 235) soll nur ein
               Zeichen treuer Freundschaft\pend
           \pstart \spacefill\mbox{G.B.}\pend{}{\vspace{1\baselineskip}}
\pstart
           \centering{}\textcolor{gray}{\textbf{Armand Carrel\pwindex{Armand Carrel@\emph{Armand Carrel}|pw}}}\pend
           \selectlanguage{ngerman}\vspace{1em}{\vspace{1\baselineskip}}
\pstart
           \centering{}{\pb}\textcolor{gray}{\textbf{Armand Carrel\pwindex{Armand Carrel@\emph{Armand Carrel}|pw}}}\pend
           
\pstart
           \centering{}\textcolor{gray}{\textbf{Von}}\pend
           
\pstart
           \centering{}\textcolor{gray}{\textbf{Georg Brandes}}\pend
           
\pstart
           \centering{}\textcolor{gray}{\textbf{Autoriſierte Überſetzung von \so{Erich Holm}\pwindex{Holm, Erich 1844-01-03 – 1921-02-01@\textsc{Holm, Erich} (1844-01-03 – 1921-02-01), \emph{Schriftsteller/Schriftstellerin, Übersetzer/Übersetzerin, Literaturwissenschaftler/Literaturwissenschaftlerin}|pw}}}\pend
           {\vspace{1\baselineskip}}
\pstart
           \centering{}\textcolor{gray}{\textbf{Stuttgart\oindex{Karlsruhe@\textbf{Karlsruhe}, \emph{P.PPLA2}|pw} und Berlin\oindex{Berlin@\textbf{Berlin}, \emph{P.PPLC}|pw}{ }\label{K_L02100-2v}\edtext{1913}{\lemma{\textnormal{\emph{1913}}}\Cendnote{\textnormal{am 16. 11. 1912 vom \emph{Börsenblatt für den deutschen Buchhandel}\pwindex{Boersenblatt fuer den Deutschen Buchhandel@\emph{Börsenblatt für den Deutschen Buchhandel}|pwk}
                     als Neuerscheinung gemeldet}}}\label{K_L02100-2}}}\pend
           
\pstart
           \centering{}\textcolor{gray}{\textbf{J. G. Cotta’ſche Buchhandlung Nachfolger\orgindex{J.G. Cotta sche Buchhandlung Nachfolger@J.G. Cotta’sche Buchhandlung Nachfolger|pw}}}\pend
           \selectlanguage{ngerman}\endnumbering\briefempfaengerindex{Schnitzler, Arthur@\textsc{Schnitzler, Arthur}!zzzBrandes, Georg@\emph{von Georg Brandes}!1912-11-162@{{[}nach dem 16. 11. 1912{]}}|)be}\mylabel{L02100h}  \normalsize

\doendnotes{C}
\bigskip
\vfill

\clearpage

\footnotesize

\lohead{\textsc{register}}

% Definiere theindex-Environment komplett neu ohne reledmac
\makeatletter
\renewenvironment{theindex}{%
  \section*{\indexname}%
  \setlength{\parindent}{0pt}%
  \setlength{\parskip}{0pt plus 0.3pt}%
  \let\item\@idxitem
}{%
  \clearpage
}
\makeatother

\IfFileExists{\jobname-pw.ind}{\input{\jobname-pw.ind}}{}

\end{document}

      