%% latex-leseansicht-vorspann.tex
%% Vorspann für die Leseansicht.
%% Lädt die gemeinsame Datei latex-vorspann.tex mit nicht gesetztem Schalter.

\newif\ifkorrekturansicht
\korrekturansichtfalse

\input{../tex-inputs/latex-vorspann}


         
         \renewcommand{\erwaehntePersonen}{Personen: Georg Brandes, Erich Holm}
         \renewcommand{\erwaehnteInstitutionen}{Institutionen: J.G. Cotta’sche Buchhandlung Nachfolger}
         \renewcommand{\erwaehnteOrte}{Orte: Berlin, Karlsruhe, Wien}
         \renewcommand{\erwaehnteWerke}{Werke: Armand Carrel, Börsenblatt für den Deutschen Buchhandel, Ilias}
               \section[Georg Brandes: Widmungsexemplar Armand Carrel für Arthur Schnitzler, {[}nach dem 16. 11. 1912{]}]{ Georg Brandes: Widmungsexemplar Armand Carrel für Arthur Schnitzler,
               {[}nach dem 16. 11. 1912{]}}\nopagebreak\mylabel{v}\rehead{ }\begin{ledgroupsized}[t]{13cm}\normalsize\beginnumbering \toendnotes[C]{\smallbreak\pagebreak[2]} \Standort{DLA, G:Schnitzler, Arthur (Sammlung Heinrich Schnitzler).}
\physDesc{Widmung am Vorsatzblatt, 128 Zeichen
\newline{}Handschrift: schwarze Tinte, lateinische Kurrent
\newline{}Ordnung: mit Bleistift von unbekannter Hand das Pseudonym der
                                 Übersetzerin aufgelöst: »Prager Mathilde\pwindex{Holm, Erich 1844-01-03 – 1921-02-01@\textsc{Holm, Erich} (1844-01-03 – 1921-02-01), \emph{Schriftstellerin, Übersetzerin, Wissenschaftlerin >> Literaturwissenschaftler}|pw}« }\toendnotes[C]{\smallbreak}\pstart
           \noindent{}{\pb}An Arthur Schnitzler\pend
           \pstart
           Diese Bagatelle, \label{K_L02100-1v}\edtext{Diomedes’ Geschenk an
                  Glaukos}{\lemma{\textnormal{\emph{Diomedes’ … Glaukos}}}\Cendnote{\textnormal{Glaukos erneuert den
                  Freundschaftsbund, er gibt Diomedes eine goldene, dieser ihm eine eherne
                  Rüstung.}}}\label{K_L02100-1h}, (Ilias\pwindex{\textcolor{red}{\textsuperscript{XXXX1 indx}}!IliasNone@\strich\emph{Ilias} {[}None{]}|pw} IV 235) soll nur ein
               Zeichen treuer Freundschaft\pend
           \pstart \spacefill\mbox{G.B.}\pend{}{\bigskip}\pstart
           \noindent{}\centering{}\textcolor{gray}{\textbf{Armand Carrel\pwindex{Brandes, Georg 04.02.1842 – 19.02.1927@\textsc{Brandes, Georg} (04.02.1842 – 19.02.1927)!Armand Carrel1911@\strich\emph{Armand Carrel} {[}1911{]}|pw}}}\pend
           {\bigskip}\pstart
           \noindent{}\centering{}{\pb}\textcolor{gray}{\textbf{Armand Carrel\pwindex{Brandes, Georg 04.02.1842 – 19.02.1927@\textsc{Brandes, Georg} (04.02.1842 – 19.02.1927)!Armand Carrel1911@\strich\emph{Armand Carrel} {[}1911{]}|pw}}}\pend
           \pstart
           \noindent{}\centering{}\textcolor{gray}{\textbf{Von}}\pend
           \pstart
           \noindent{}\centering{}\textcolor{gray}{\textbf{Georg Brandes}}\pend
           \pstart
           \noindent{}\centering{}\textcolor{gray}{\textbf{Autoriſierte Überſetzung von \so{Erich Holm}\pwindex{Holm, Erich 1844-01-03 – 1921-02-01@\textsc{Holm, Erich} (1844-01-03 – 1921-02-01), \emph{Schriftstellerin, Übersetzerin, Wissenschaftlerin >> Literaturwissenschaftler}|pw}}}\pend
           {\bigskip}\pstart
           \noindent{}\centering{}\textcolor{gray}{\textbf{Stuttgart\oindex{Karlsruhe@\textbf{Karlsruhe}|pw} und Berlin\oindex{Berlin@\textbf{Berlin}|pw}{ }\label{K_L02100-2v}\edtext{1913}{\lemma{\textnormal{\emph{1913}}}\Cendnote{\textnormal{am 16. 11. 1912 vom \emph{Börsenblatt für den deutschen Buchhandel}\pwindex{?? Werk@Nicht ermittelte Verfasserinnen und Verfasser!Boersenblatt fuer den Deutschen Buchhandel1843-01-03@\emph{Börsenblatt für den Deutschen Buchhandel} {[}1843-01-03{]}|pwk}
                     als Neuerscheinung gemeldet}}}\label{K_L02100-2h}}}\pend
           \pstart
           \noindent{}\centering{}\textcolor{gray}{\textbf{J. G. Cotta’ſche Buchhandlung Nachfolger\orgindex{J.G. Cotta sche Buchhandlung Nachfolger@J.G. Cotta’sche Buchhandlung Nachfolger|pw}}}\pend
           
         
         \endnumbering\mylabel{h}\end{ledgroupsized}  \newcommand{\dateiname}{L02100}\newcommand{\titel}{Georg Brandes: Widmungsexemplar Armand Carrel für Arthur Schnitzler, [nach dem 16. 11. 1912]}\newcommand{\editorInnen}{Martin Anton Müller und Gerd-Hermann Susen}%% latex-leseansicht-abspann.tex
%% Abspann für die Leseansicht.
%% Der Schalter \ifkorrekturansicht ist bereits durch den Vorspann gesetzt.

%% latex-abspann.tex
%% Gemeinsamer Abspann für Korrekturansicht und Leseansicht.
%% Setzt den Schalter \ifkorrekturansicht voraus (gesetzt in den
%% einbindenden Dateien latex-korrekturansicht-abspann.tex bzw.
%% latex-leseansicht-abspann.tex).
%% ---------------------------------------------------------------

\normalsize

% Das esempio-Environment wird nur in der Leseansicht benötigt
\ifkorrekturansicht\else
\newenvironment{esempio}[3]%
{
    \vspace{1.5ex}
    \rlap{\underline{#1}}
    \par
    \setlength{\parindent}{0cm}
    \nopagebreak
    \leftskip=#2cm
    \rightskip=#3cm
}
{
    \par
}
\fi

\doendnotes{C}
\bigskip
\vfill

\clearpage

\footnotesize

\ifkorrekturansicht
  \lohead{\textsc{register}}
\fi

% theindex-Environment neu definieren ohne reledmac
\makeatletter
\renewenvironment{theindex}{%
  \ifkorrekturansicht
    \section*{\indexname}%
  \else
    \subsubsection*{Index der erwähnten Entitäten}%
  \fi
  \setlength{\parindent}{0pt}%
  \setlength{\parskip}{0pt plus 0.3pt}%
  \let\item\@idxitem
}{%
  \ifkorrekturansicht\clearpage\fi
}
\makeatother

\IfFileExists{\jobname-pw.ind}{\input{\jobname-pw.ind}}{}

% Quellenangabe nur in der Leseansicht
\ifkorrekturansicht\else
% Fallback-Definitionen, falls die .tex-Datei \titel etc. nicht gesetzt hat
\providecommand{\titel}{}
\providecommand{\editorInnen}{}
\providecommand{\dateiname}{\jobname}

\vspace{3cm}

\vfill

\footnotesize
\textsc{Quelle}: \titel. Herausgegeben von {\editorInnen}. In: \emph{Arthur Schnitzler: Briefwechsel mit Autorinnen und Autoren}.
 Digitale Edition, https://schnitzler-briefe.acdh.oeaw.ac.at/{\dateiname}.html (Stand \today)
\fi

\end{document}


      