%% latex-leseansicht-vorspann.tex
%% Vorspann für die Leseansicht.
%% Lädt die gemeinsame Datei latex-vorspann.tex mit nicht gesetztem Schalter.

\newif\ifkorrekturansicht
\korrekturansichtfalse

\input{../tex-inputs/latex-vorspann}


\section[Hugo von Hofmannsthal an Arthur Schnitzler, 18. 8. 1904]{L01429 Hugo von Hofmannsthal an Arthur Schnitzler, 18. 8. 1904}
\nopagebreak\mylabel{L01429v}
\rehead{ }\normalsize\beginnumbering\briefempfaengerindex{Schnitzler, Arthur@\textsc{Schnitzler, Arthur}!zzzHofmannsthal, Hugo von@\emph{von Hugo von Hofmannsthal}!1904-08-181@{18. 8. 1904}|(be}
\toendnotes[C]{\smallbreak\pagebreak[2]}
\correspDesc{Versand  durch Hugo von Hofmannsthal am 18. 8. 1904 in Bad Aussee
\newline{}Erhalt  durch Arthur Schnitzler am 19. 8. 1904 in Wien}\toendnotes[C]{\smallbreak}
\Standort{CUL, Schnitzler, B 43.}
\physDesc{Postkarte, 458 Zeichen
\newline{}Handschrift: schwarze Tinte, deutsche Kurrent
\newline{}Versand: 1) Stempel: »\nobreak{}\oindex{Bad Aussee@\textbf{Bad Aussee}, \emph{Hauptstadt}|pwk}Aussee in Steiermark, 18 8 04\nobreak{}«.   2) Stempel: »\nobreak{}\oindex{XVIII., Währing@\textbf{XVIII., Währing}, \emph{Verwaltungsgebiet}|pwk}18/1 Wien, 19. 8. 04, 3.N, Bestellt\nobreak{}«. 
\newline{}Ordnung: 1) mit Bleistift von unbekannter Hand nummeriert: »\strikeout{223}«  2) mit Bleistift von unbekannter Hand nummeriert:
                                    »233«}
\buchAbdrucke{\weitereDrucke{Hugo von Hofmannsthal, Arthur Schnitzler: \emph{Briefwechsel}. Herausgegeben von Therese Nickl und Heinrich Schnitzler. Frankfurt am Main: \emph{S. Fischer} 1964, S. 197.} }\toendnotes[C]{\smallbreak}\pstart{}{\pb}\textsc{Herrn D\textsuperscript{r} Arthur Schnitzler}\pend{}\pstart{}\textsc{Wien}\oindex{Wien@\textbf{Wien}, \emph{Verwaltungsgebiet}|pw}\pend{}\pstart{}\textsc{XVIII Spöttelgasse 7}.\oindex{Wien@\textbf{Wien}!XVIII., Währing@\textbf{XVIII., Währing}!Edmund-Weiß-Gasse 7@\textbf{Edmund-Weiß-Gasse 7}, \emph{Wohngebäude}|pw}\pend{}{\bigskip}\vspace{1em}
\pstart
           \raggedleft{}{\pb}18. VIII.\pend
           \vspace{0.5em}
\pstart
           lieber, den Inhalt Ihres großen Briefes werde ich mit \textsc{V. S.}\pwindex{Vansittart, Robert Gilbert 25.\,6.\,1881 Farnham – 14.\,2.\,1957 Denham@\textsc{Vansittart, Robert Gilbert} (25.\,6.\,1881 Farnham – 14.\,2.\,1957 Denham), \emph{Diplomat}|pw} genau durchſprechen und{ }ſeiner Energie wird es gewiſs gelingen, Ordnung in die
               Sache zu bringen. Jetzt etwas anderes; bitte{ }ſchreiben Sie mir gleich, ob Ihr Plan,
                  erſte Tage September{ }Salzka{\geminationm}ergut\oindex{Salzkammergut@\textbf{Salzkammergut}, \emph{Region}|pw}
               feſtſteht. Gerty\pwindex{Hofmannsthal, Gertrude von 16.\,3.\,1880 Wien – 9.\,11.\,1959 Paddington@\textsc{Hofmannsthal, Gertrude von} (16.\,3.\,1880 Wien – 9.\,11.\,1959 Paddington)|pw} bringt 25\textsuperscript{ten} VIII die Kinder\pwindex{Zimmer, Christiane 14.\,5.\,1902 Rodaun – 5.\,1.\,1987 New York City@\textsc{Zimmer, Christiane} (14.\,5.\,1902 Rodaun – 5.\,1.\,1987 New York City)|pwv}\pwindex{Hofmannsthal, Raimund von 26.\,5.\,1906 Rodaun – 20.\,3.\,1974 London@\textsc{Hofmannsthal, Raimund von} (26.\,5.\,1906 Rodaun – 20.\,3.\,1974 London)|pwv}\pwindex{Hofmannsthal, Franz von 20.\,10.\,1903 Wien – 13.\,7.\,1929 ebd.@\textsc{Hofmannsthal, Franz von} (20.\,10.\,1903 Wien – 13.\,7.\,1929 ebd.)|pwv} nach Rodaun\oindex{Wien@\textbf{Wien}!XXIII., Liesing@\textbf{XXIII., Liesing}!Rodaun@\textbf{Rodaun}, \emph{Region}|pw} zurück und wäre \uline{ſehr} erfreut, mit Ihnen und Olga\pwindex{Schnitzler, Olga 17.\,1.\,1882 Wien – 13.\,1.\,1970 Lugano@\textsc{Schnitzler, Olga} (17.\,1.\,1882 Wien – 13.\,1.\,1970 Lugano), \emph{Schauspielerin, Sängerin}|pw} etwa
               den 2\textsuperscript{ten}, 3\textsuperscript{ten} September nach Iſchl\oindex{Bad Ischl@\textbf{Bad Ischl}|pw} zu fahren.\pend
           \pstart Ihr \spacefill\mbox{Hugo.}\pend{}\selectlanguage{ngerman}\endnumbering\briefempfaengerindex{Schnitzler, Arthur@\textsc{Schnitzler, Arthur}!zzzHofmannsthal, Hugo von@\emph{von Hugo von Hofmannsthal}!1904-08-181@{18. 8. 1904}|)be}\mylabel{L01429h}  \newcommand{\dateiname}{L01429}\newcommand{\titel}{Hugo von Hofmannsthal an Arthur Schnitzler, 18. 8. 1904}\newcommand{\editorInnen}{Martin Anton Müller und Gerd-Hermann Susen}%% latex-leseansicht-abspann.tex
%% Abspann für die Leseansicht.
%% Der Schalter \ifkorrekturansicht ist bereits durch den Vorspann gesetzt.

%% latex-abspann.tex
%% Gemeinsamer Abspann für Korrekturansicht und Leseansicht.
%% Setzt den Schalter \ifkorrekturansicht voraus (gesetzt in den
%% einbindenden Dateien latex-korrekturansicht-abspann.tex bzw.
%% latex-leseansicht-abspann.tex).
%% ---------------------------------------------------------------

\normalsize

% Das esempio-Environment wird nur in der Leseansicht benötigt
\ifkorrekturansicht\else
\newenvironment{esempio}[3]%
{
    \vspace{1.5ex}
    \rlap{\underline{#1}}
    \par
    \setlength{\parindent}{0cm}
    \nopagebreak
    \leftskip=#2cm
    \rightskip=#3cm
}
{
    \par
}
\fi

\doendnotes{C}
\bigskip
\vfill

\clearpage

\footnotesize

\ifkorrekturansicht
  \lohead{\textsc{register}}
\fi

% theindex-Environment neu definieren ohne reledmac
\makeatletter
\renewenvironment{theindex}{%
  \ifkorrekturansicht
    \section*{\indexname}%
  \else
    \subsubsection*{Index der erwähnten Entitäten}%
  \fi
  \setlength{\parindent}{0pt}%
  \setlength{\parskip}{0pt plus 0.3pt}%
  \let\item\@idxitem
}{%
  \ifkorrekturansicht\clearpage\fi
}
\makeatother

\IfFileExists{\jobname-pw.ind}{\input{\jobname-pw.ind}}{}

% Quellenangabe nur in der Leseansicht
\ifkorrekturansicht\else
% Fallback-Definitionen, falls die .tex-Datei \titel etc. nicht gesetzt hat
\providecommand{\titel}{}
\providecommand{\editorInnen}{}
\providecommand{\dateiname}{\jobname}

\vspace{3cm}

\vfill

\footnotesize
\textsc{Quelle}: \titel. Herausgegeben von {\editorInnen}. In: \emph{Arthur Schnitzler: Briefwechsel mit Autorinnen und Autoren}.
 Digitale Edition, https://schnitzler-briefe.acdh.oeaw.ac.at/{\dateiname}.html (Stand \today)
\fi

\end{document}


