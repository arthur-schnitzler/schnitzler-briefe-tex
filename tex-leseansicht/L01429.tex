%% latex-leseansicht-vorspann.tex
%% Vorspann für die Leseansicht.
%% Lädt die gemeinsame Datei latex-vorspann.tex mit nicht gesetztem Schalter.

\newif\ifkorrekturansicht
\korrekturansichtfalse

\input{../tex-inputs/latex-vorspann}


         
         \renewcommand{\erwaehntePersonen}{Personen: Gertrude von Hofmannsthal, Christiane von Hofmannsthal, Raimund von Hofmannsthal, Franz von Hofmannsthal, Olga Schnitzler, Robert Gilbert Vansittart}
         \renewcommand{\erwaehnteOrte}{Orte: Bad Aussee, Bad Ischl, Edmund-Weiß-Gasse, Rodaun, Salzkammergut, Wien, XVIII., Währing}
         \renewcommand{\erwaehnteWerke}{}
               \section[Hugo von Hofmannsthal an Arthur Schnitzler, 18. 8. 1904]{ Hugo von Hofmannsthal an Arthur Schnitzler, 18. 8. 1904}\nopagebreak\mylabel{v}\rehead{ }\begin{ledgroupsized}[t]{13cm}\normalsize\beginnumbering \toendnotes[C]{\smallbreak\pagebreak[2]} \Standort{CUL, Schnitzler, B 43.}
\physDesc{Postkarte
\newline{}Handschrift: schwarze Tinte, deutsche Kurrent\newline{}Versand: 1) Stempel: »\nobreak{}\oindex{Bad Aussee@\textbf{Bad Aussee}|pwk}Aussee in Steiermark, 18 8 04\nobreak{}«.   2) Stempel: »\nobreak{}\oindex{XVIII., Waehring@\textbf{XVIII., Währing}|pwk}18/1 Wien, 19. 8. 04, 3.N, Bestellt\nobreak{}«. \newline{}Ordnung: 1) mit Bleistift von unbekannter Hand
                           nummeriert: »\strikeout{223}«  2) mit Bleistift von unbekannter Hand nummeriert: »233«}\buchAbdrucke{\weitereDrucke{Hugo von Hofmannsthal, Arthur Schnitzler: \emph{Briefwechsel}. Hg. Therese Nickl und Heinrich Schnitzler. Frankfurt am Main: \emph{S. Fischer} 1964, S. 197.} }\toendnotes[C]{\smallbreak}\pstart{}{\pb}\textsc{Herrn D\textsuperscript{r} Arthur
                  Schnitzler}\pend{}\pstart{}\textsc{Wien}\oindex{Wien@\textbf{Wien}|pw}\pend{}\pstart{}\textsc{XVIII Spöttelgasse 7}.\oindex{Edmund-Weiss-Gasse@\textbf{Edmund-Weiß-Gasse}|pw}\pend{}{\bigskip}\pstart
           \raggedleft{}{\pb}18. VIII.\pend
           \pstart
           lieber, den Inhalt Ihres großen Briefes werde ich mit \textsc{V. S.}\pwindex{Vansittart, Robert Gilbert 25.06.1881 – 14.02.1957@\textsc{Vansittart, Robert Gilbert} (25.06.1881 – 14.02.1957), \emph{Diplomat}|pw} genau
               durchſprechen und ſeiner Energie wird es gewiſs gelingen, Ordnung in die Sache zu
               bringen. Jetzt etwas anderes; bitte ſchreiben Sie mir gleich, ob Ihr Plan,
                  erſte Tage September{ }Salzka{\geminationm}ergut\oindex{Salzkammergut@\textbf{Salzkammergut}|pw} feſtſteht.
                  Gerty\pwindex{Hofmannsthal, Gertrude von 16.03.1880 – 09.11.1959@\textsc{Hofmannsthal, Gertrude von} (16.03.1880 – 09.11.1959)|pw} bringt 25\textsuperscript{ten} VIII die Kinder\pwindex{Hofmannsthal, Christiane von 14.05.1902 – 05.01.1987@\textsc{Hofmannsthal, Christiane von} (14.05.1902 – 05.01.1987)|pwv}\pwindex{Hofmannsthal, Raimund von 26.5.1906 – 20.03.1974@\textsc{Hofmannsthal, Raimund von} (26.5.1906 – 20.03.1974)|pwv}\pwindex{Hofmannsthal, Franz von 20.10.1903 – 13.07.1929@\textsc{Hofmannsthal, Franz von} (20.10.1903 – 13.07.1929)|pwv} nach Rodaun\oindex{Rodaun@\textbf{Rodaun}|pw} zurück
               und wäre \uline{ſehr} erfreut, mit Ihnen und Olga\pwindex{Schnitzler, Olga 17.01.1882 – 13.01.1970@\textsc{Schnitzler, Olga} (17.01.1882 – 13.01.1970), \emph{Schauspielerin, Sängerin}|pw} etwa den 2\textsuperscript{ten},
                     3\textsuperscript{ten} September nach Iſchl\oindex{Bad Ischl@\textbf{Bad Ischl}|pw} zu fahren.\pend
           \pstart Ihr \spacefill\mbox{Hugo}.\pend{}
         
         \endnumbering\mylabel{h}\end{ledgroupsized}  \newcommand{\dateiname}{L01429}\newcommand{\titel}{Hugo von Hofmannsthal an Arthur Schnitzler, 18. 8. 1904}\newcommand{\editorInnen}{Martin Anton Müller und Gerd-Hermann Susen}%% latex-leseansicht-abspann.tex
%% Abspann für die Leseansicht.
%% Der Schalter \ifkorrekturansicht ist bereits durch den Vorspann gesetzt.

%% latex-abspann.tex
%% Gemeinsamer Abspann für Korrekturansicht und Leseansicht.
%% Setzt den Schalter \ifkorrekturansicht voraus (gesetzt in den
%% einbindenden Dateien latex-korrekturansicht-abspann.tex bzw.
%% latex-leseansicht-abspann.tex).
%% ---------------------------------------------------------------

\normalsize

% Das esempio-Environment wird nur in der Leseansicht benötigt
\ifkorrekturansicht\else
\newenvironment{esempio}[3]%
{
    \vspace{1.5ex}
    \rlap{\underline{#1}}
    \par
    \setlength{\parindent}{0cm}
    \nopagebreak
    \leftskip=#2cm
    \rightskip=#3cm
}
{
    \par
}
\fi

\doendnotes{C}
\bigskip
\vfill

\clearpage

\footnotesize

\ifkorrekturansicht
  \lohead{\textsc{register}}
\fi

% theindex-Environment neu definieren ohne reledmac
\makeatletter
\renewenvironment{theindex}{%
  \ifkorrekturansicht
    \section*{\indexname}%
  \else
    \subsubsection*{Index der erwähnten Entitäten}%
  \fi
  \setlength{\parindent}{0pt}%
  \setlength{\parskip}{0pt plus 0.3pt}%
  \let\item\@idxitem
}{%
  \ifkorrekturansicht\clearpage\fi
}
\makeatother

\IfFileExists{\jobname-pw.ind}{\input{\jobname-pw.ind}}{}

% Quellenangabe nur in der Leseansicht
\ifkorrekturansicht\else
% Fallback-Definitionen, falls die .tex-Datei \titel etc. nicht gesetzt hat
\providecommand{\titel}{}
\providecommand{\editorInnen}{}
\providecommand{\dateiname}{\jobname}

\vspace{3cm}

\vfill

\footnotesize
\textsc{Quelle}: \titel. Herausgegeben von {\editorInnen}. In: \emph{Arthur Schnitzler: Briefwechsel mit Autorinnen und Autoren}.
 Digitale Edition, https://schnitzler-briefe.acdh.oeaw.ac.at/{\dateiname}.html (Stand \today)
\fi

\end{document}


      