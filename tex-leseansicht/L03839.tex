%% latex-leseansicht-vorspann.tex
%% Vorspann für die Leseansicht.
%% Lädt die gemeinsame Datei latex-vorspann.tex mit nicht gesetztem Schalter.

\newif\ifkorrekturansicht
\korrekturansichtfalse

\input{../tex-inputs/latex-vorspann}


\section[Theodor Herzl an Arthur Schnitzler, 17. 12. 1894]{L03839 Theodor Herzl an Arthur Schnitzler, 17. 12. 1894}
\nopagebreak\mylabel{L03839v}
\rehead{ }\normalsize\beginnumbering\briefempfaengerindex{Schnitzler, Arthur@\textsc{Schnitzler, Arthur}!zzzHerzl, Theodor@\emph{von Theodor Herzl}!1894-12-171@{17. 12. 1894}|(be}
\toendnotes[C]{\smallbreak\pagebreak[2]}
\correspDesc{Versand  durch Theodor Herzl am 17. 12. 1894 in Paris
\newline{}Erhalt  durch Arthur Schnitzler im Zeitraum [18. 12. 1894 – 22. 12. 1894?] in Wien}\toendnotes[C]{\smallbreak}
\Standort{CUL, Schnitzler, B 39.}
\physDesc{Brief, 1 Blatt, 3 Seiten, 2556 Zeichen
\newline{}Handschrift: schwarze Tinte, lateinische Kurrent
\newline{}Ordnung: 1) mit Bleistift von unbekannter Hand nummeriert: »18«  2) mit blauem Buntstift von Leon Kellner\pwindex{Kellner, Leon 17.\,4.\,1859 Tarnów – 5.\,12.\,1928 Wien@\textsc{Kellner, Leon} (17.\,4.\,1859 Tarnów – 5.\,12.\,1928 Wien), \emph{Zionist, Literaturhistoriker, Anglist}|pw} Markierung von Stellen für
                                 die Publikation 3) mit rotem Buntstift zwei Anstreichungen}
\buchAbdrucke{\weitereDrucke{1) \pwindex{Kellner, Leon 17.\,4.\,1859 Tarnów – 5.\,12.\,1928 Wien@\textsc{Kellner, Leon} (17.\,4.\,1859 Tarnów – 5.\,12.\,1928 Wien), \emph{Zionist, Literaturhistoriker, Anglist}!Theodor Herzls Lehrjahre (1860–1895). Nach den handschriftlichen Quellen@\strich\emph{Theodor Herzls Lehrjahre (1860–1895). Nach den handschriftlichen Quellen}|pwk}\emph{[Auszug].} In: Leon Kellner: \emph{Theodor Herzls Lehrjahre (1860–1895). Nach den handschriftlichen Quellen}. Wien, Berlin: \emph{R. Löwit-Verlag} 1920, S. 148.} \weitereDrucke{2) \emph{Herzl-Briefe}. Herausgegeben und eingeleitet Manfred Georg. Berlin: \emph{Brandusche Verlagsbuchhandlung} [1935], S. 42–44.} \weitereDrucke{3) Theodor Herzl: \emph{Briefe und
                        autobiographische Notizen 1866–1895}. Bearbeitet von Johannes Wachten in Zusammenarbeit mit Chaya Harel, Daisy Tycho und Manfred Winkler. Berlin, Frankfurt am Main, Wien: \emph{Propyläen} 1983, S. 562–563 (Briefe und Tagebücher. Herausgegeben von Alex Bein, Hermann Greive, Moshe Schaerf, Julius H. Schoeps und Johannes Wachten, 1).} }\toendnotes[C]{\smallbreak}
\pstart
           {\pb}\textcolor{gray}{\textbf{NOUVELLE PRESSE LIBRE}}\orgindex{Neue Freie Presse@Neue Freie Presse|pw}\hfill \textcolor{gray}{\textbf{8, RUE DE MONCEAU
                        }}\oindex{8, rue de Monceau@\textbf{8, rue de Monceau}, \emph{Wohngebäude}|pw}\pend
           
\pstart
           \textcolor{gray}{\textbf{D\textsuperscript{r}{ }TH. HERZL}}\hfill 17. XII. 94\pend
           
\pstart{}Mein lieber Freund!\pend\vspace{0.5em}
\pstart
           Dank für Ihren lieben Brief. Aber was
      sind Sie für ein Vernachlässiger von
       Details! Haben Sie eine Maschine oder
      werden Sie eine kriegen? Der Copist
      kann ja schon anfangen, sich einzuüben.
      Geben Sie ihm ein Buch zum Abspielen.
      Sie werden morgen, spätestens übermorgen den Anfang\pwindex{Herzl, Theodor 2.\,5.\,1860 Budapest – 3.\,7.\,1904 Edlach@\textsc{Herzl, Theodor} (2.\,5.\,1860 Budapest – 3.\,7.\,1904 Edlach), \emph{Schriftsteller, Journalist}!neue Ghetto. Schauspiel in vier Acten@\strich\emph{Das neue Ghetto. Schauspiel in vier Acten}|pwv} zugeschickt bekommen,
               vielleicht schon die ganzen 2 Acte\pwindex{Herzl, Theodor 2.\,5.\,1860 Budapest – 3.\,7.\,1904 Edlach@\textsc{Herzl, Theodor} (2.\,5.\,1860 Budapest – 3.\,7.\,1904 Edlach), \emph{Schriftsteller, Journalist}!neue Ghetto. Schauspiel in vier Acten@\strich\emph{Das neue Ghetto. Schauspiel in vier Acten}|pwv}.
      Das Feilen wird mir sauerer als
      das Schreiben, das so begeistert war.
      Die eine fehlende Scene im 2 Akt
      krieg ich gar nicht heraus. Na, ich
      setz mich an. Es muss gehen!\pend
           
\pstart
           Jetzt wo ich die Sache wieder durchlebe könnt’ ich besser auf Ihre
      ersten Einwendungen antworten. Ich
      finde nicht, dass zu wenig »sympathische«
      Figuren da sind. Und wenn auch,
      soll ich meine Misanthropie fälschen? {\pb}Soll ich gerade dort wo es mir Niemand
      glauben wird lauter wunderedle Menschen
      sehen u. zeigen? Nein, Freund, das geht
      nicht. Ich will mich auch nicht mehr \label{K_L03839-1v}\edtext{emasculiren}{\lemma{\textnormal{\emph{emasculiren}}}\Cendnote{\textnormal{entmannen, abschwächen}}}\label{K_L03839-1} irgend einem Erfolg zu Liebe.
      So rosig meine Augen überhaupt sehen
      können, war’s schon gemacht. Ich will
      durchaus keine Vertheidigung oder »Rettung«
      der \label{K_L03839-2v}\edtext{J.}{\lemma{\textnormal{\emph{J.}}}\Cendnote{\textnormal{Juden}}}\label{K_L03839-2} machen, ich will die Frage
      nur mit aller Macht zur Discussion
      stellen! Die Kritiker und das Volk
      sollen dann vertheidigen oder anklagen.
      Komm ich \introOben{}nur\introOben{} auf die Bühne, so ist der
      Zweck erreicht. Was weiter geschieht
      ist mir Wurscht. Ich pfeif auf das
      Geld obwol ich beinahe keines und
      auf den Ruhm ohwol ich gar keinen
      habe. Ich will gar kein sympathischer
      Dichter sein. Aussprechen will ich mich
      von der Leber u. vom Herzen weg.
      Wenn dieses Stück\pwindex{Herzl, Theodor 2.\,5.\,1860 Budapest – 3.\,7.\,1904 Edlach@\textsc{Herzl, Theodor} (2.\,5.\,1860 Budapest – 3.\,7.\,1904 Edlach), \emph{Schriftsteller, Journalist}!neue Ghetto. Schauspiel in vier Acten@\strich\emph{Das neue Ghetto. Schauspiel in vier Acten}|pwv} in der Welt ist
      wird mir leichter um Herz und
      Leber sein.\pend
           
\pstart
           Was aber Ihr feiner Dichtersinn richtig
         herausgefunden hat, ist: dass im Stück\pwindex{Herzl, Theodor 2.\,5.\,1860 Budapest – 3.\,7.\,1904 Edlach@\textsc{Herzl, Theodor} (2.\,5.\,1860 Budapest – 3.\,7.\,1904 Edlach), \emph{Schriftsteller, Journalist}!neue Ghetto. Schauspiel in vier Acten@\strich\emph{Das neue Ghetto. Schauspiel in vier Acten}|pwv}{ }{\pb}mehrere andere verworren mitschwingen.
      Diese Stücke leben auch schon lange
      und so stark in mir, dass sie bei
      der Loslösung dieses Stückes\pwindex{Herzl, Theodor 2.\,5.\,1860 Budapest – 3.\,7.\,1904 Edlach@\textsc{Herzl, Theodor} (2.\,5.\,1860 Budapest – 3.\,7.\,1904 Edlach), \emph{Schriftsteller, Journalist}!neue Ghetto. Schauspiel in vier Acten@\strich\emph{Das neue Ghetto. Schauspiel in vier Acten}|pwv} es ein
      bischen befleckt haben. Ich werde
      also nur diese Unreinheiten weglöschen
      müssen. \strikeout{\textcolor{gray}{×}\-\textcolor{gray}{×}\-\textcolor{gray}{×}\-\textcolor{gray}{×}\-\textcolor{gray}{×}{ }\textcolor{gray}{×}\-\textcolor{gray}{×}\-\textcolor{gray}{×}\-\textcolor{gray}{×}\-\textcolor{gray}{×}\-\textcolor{gray}{×}{ }\textcolor{gray}{×}\-\textcolor{gray}{×}\-\textcolor{gray}{×}\-\textcolor{gray}{×}\-\textcolor{gray}{×}{ }\textcolor{gray}{×}\-\textcolor{gray}{×}\-\textcolor{gray}{×}}\pend
           
\pstart
           Grössere Gesänge schlafen noch auf
      den ehernen Saiten. Wenn ich mir
      eines Tages die Freiheit vom Taglöhnern
      erwerbe, kommen die höheren Sachen.
      Ich habe noch einen ganzen Frühling
      in mir – vielleicht blüht er noch
      einmal heraus.\pend
           
\pstart
           Die Glosse\pwindex{Herzl, Theodor 2.\,5.\,1860 Budapest – 3.\,7.\,1904 Edlach@\textsc{Herzl, Theodor} (2.\,5.\,1860 Budapest – 3.\,7.\,1904 Edlach), \emph{Schriftsteller, Journalist}!Glosse. Lustspiel in einem Act@\strich\emph{Die Glosse. Lustspiel in einem Act}|pw} hab’ ich ins Massengrab
      meiner alten Stücke geworfen.
      Es ist ja kein Zweifel, dass ich
      mir die Aufführung irgendwo
      »richten« könnte. Das mag ich
      nicht. Begreifen Sie nicht, dass
      mir die Journalisten-Machereien
      beim Theater \label{K_L03839-3v}\edtext{repugniren}{\lemma{\textnormal{\emph{repugniren}}}\Cendnote{\textnormal{widerstreben}}}\label{K_L03839-3}? Lieber
      nicht aufgeführt, als aus Gefälligkeit
      oder Furcht.\pend
           
\pstart
           Sobald Sie den ersten Theil des
         Mscpts\pwindex{Herzl, Theodor 2.\,5.\,1860 Budapest – 3.\,7.\,1904 Edlach@\textsc{Herzl, Theodor} (2.\,5.\,1860 Budapest – 3.\,7.\,1904 Edlach), \emph{Schriftsteller, Journalist}!neue Ghetto. Schauspiel in vier Acten@\strich\emph{Das neue Ghetto. Schauspiel in vier Acten}|pwv} haben, bitte ich um die Anzeige
      ob der Copist begonnen hat, ob er
      schreibt oder abklopft.\pend
           
\pstart
           Herzliche Grüsse von Ihrem Freund {\\[\baselineskip]}\spacefill\mbox{Th. H.}\pend
           \leftskip=0em{}\selectlanguage{ngerman}\endnumbering\briefempfaengerindex{Schnitzler, Arthur@\textsc{Schnitzler, Arthur}!zzzHerzl, Theodor@\emph{von Theodor Herzl}!1894-12-171@{17. 12. 1894}|)be}\mylabel{L03839h}
\begin{anhang}
\end{anhang}\newcommand{\dateiname}{L03839}\newcommand{\titel}{Theodor Herzl an Arthur Schnitzler, 17. 12. 1894}\newcommand{\editorInnen}{Selma Jahnke und Martin Anton Müller}%% latex-leseansicht-abspann.tex
%% Abspann für die Leseansicht.
%% Der Schalter \ifkorrekturansicht ist bereits durch den Vorspann gesetzt.

%% latex-abspann.tex
%% Gemeinsamer Abspann für Korrekturansicht und Leseansicht.
%% Setzt den Schalter \ifkorrekturansicht voraus (gesetzt in den
%% einbindenden Dateien latex-korrekturansicht-abspann.tex bzw.
%% latex-leseansicht-abspann.tex).
%% ---------------------------------------------------------------

\normalsize

% Das esempio-Environment wird nur in der Leseansicht benötigt
\ifkorrekturansicht\else
\newenvironment{esempio}[3]%
{
    \vspace{1.5ex}
    \rlap{\underline{#1}}
    \par
    \setlength{\parindent}{0cm}
    \nopagebreak
    \leftskip=#2cm
    \rightskip=#3cm
}
{
    \par
}
\fi

\doendnotes{C}
\bigskip
\vfill

\clearpage

\footnotesize

\ifkorrekturansicht
  \lohead{\textsc{register}}
\fi

% theindex-Environment neu definieren ohne reledmac
\makeatletter
\renewenvironment{theindex}{%
  \ifkorrekturansicht
    \section*{\indexname}%
  \else
    \subsubsection*{Index der erwähnten Entitäten}%
  \fi
  \setlength{\parindent}{0pt}%
  \setlength{\parskip}{0pt plus 0.3pt}%
  \let\item\@idxitem
}{%
  \ifkorrekturansicht\clearpage\fi
}
\makeatother

\IfFileExists{\jobname-pw.ind}{\input{\jobname-pw.ind}}{}

% Quellenangabe nur in der Leseansicht
\ifkorrekturansicht\else
% Fallback-Definitionen, falls die .tex-Datei \titel etc. nicht gesetzt hat
\providecommand{\titel}{}
\providecommand{\editorInnen}{}
\providecommand{\dateiname}{\jobname}

\vspace{3cm}

\vfill

\footnotesize
\textsc{Quelle}: \titel. Herausgegeben von {\editorInnen}. In: \emph{Arthur Schnitzler: Briefwechsel mit Autorinnen und Autoren}.
 Digitale Edition, https://schnitzler-briefe.acdh.oeaw.ac.at/{\dateiname}.html (Stand \today)
\fi

\end{document}


