%% latex-korrekturansicht-vorspann.tex
%% Vorspann für die Korrekturansicht.
%% Lädt die gemeinsame Datei latex-vorspann.tex mit gesetztem Schalter.

\newif\ifkorrekturansicht
\korrekturansichttrue

\input{../tex-inputs/latex-vorspann}


\section[Hermann Bahr an Arthur Schnitzler, {[}31. 10. 1903{]}]{L01333 Hermann Bahr an Arthur Schnitzler, {[}31. 10. 1903{]}}
\nopagebreak\mylabel{L01333v}
\rehead{ }\normalsize\beginnumbering\briefempfaengerindex{Schnitzler, Arthur@\textsc{Schnitzler, Arthur}!zzzBahr, Hermann@\emph{von Hermann Bahr}!1903-10-312@{{[}31. 10. 1903{]}}|(be}
\toendnotes[C]{\smallbreak\pagebreak[2]}\Standort{CUL, Schnitzler, B 5b.}
\physDesc{Brief, 1 Blatt, 1 Seite, 127 Zeichen
\newline{}Handschrift: schwarze Tinte, deutsche Kurrent
\newline{}Ordnung: von Schnitzler mit Bleistift die datiert: »31/10  903.« Mit Bleistift von unbekannter
                                 Hand nummeriert: »101« }
\buchAbdrucke{\weitereDrucke{Hermann Bahr, Arthur Schnitzler: \emph{Briefwechsel, Aufzeichnungen, Dokumente (1891–1931)}. Göttingen: \emph{Wallstein} 2018, S. 273.} }\toendnotes[C]{\smallbreak}
\pstart
           \centering{}{\pb}\textcolor{gray}{\textbf{Redaktion des Neuen Wiener Tagblatt.\orgindex{Neues Wiener Tagblatt@Neues Wiener Tagblatt|pw}}}\pend
           
\pstart
           \centering{}\textcolor{gray}{\textbf{\textsc{Wien, I., Rothenturmstrasse,
                        Steyrerhof\oindex{Steyrerhof@\textbf{Steyrerhof}, \emph{Gebäude (K.GBD)}|pw}.}}}\pend
           
\pstart
           \centering{}\textcolor{gray}{\textbf{Telegramm-Adresse: Tagblatt\orgindex{Neues Wiener Tagblatt@Neues Wiener Tagblatt|pw}, Steyrerhof, Wien\oindex{Steyrerhof@\textbf{Steyrerhof}, \emph{Gebäude (K.GBD)}|pw}. –
                     Telephon-Nummern 16584 und 16588.}}\pend
           
\pstart
           \centering{}\textcolor{gray}{\textbf{Interurbanes Telephon Nr. 12036.}}\pend
           \vspace{0.5em}
\pstart
           Ich freue mich ſehr, lieber Arthur, Dich morgen bei mir zu
               ſehen.\pend
           
\pstart
           Herzlichſt{\\[\baselineskip]}Dein{\\[\baselineskip]}\spacefill\mbox{H}\pend
           \leftskip=0em{}
\pstart
           \noindent{}Eben hab ich mein neues Büchel\pwindex{Dialog vom Tragischen@\emph{Dialog vom Tragischen}|pwv} an Dich abgeſchickt.\pend
           \selectlanguage{ngerman}\endnumbering\briefempfaengerindex{Schnitzler, Arthur@\textsc{Schnitzler, Arthur}!zzzBahr, Hermann@\emph{von Hermann Bahr}!1903-10-312@{{[}31. 10. 1903{]}}|)be}\mylabel{L01333h}  \normalsize

\doendnotes{C}
\bigskip
\vfill

\clearpage

\footnotesize

\lohead{\textsc{register}}

% Definiere theindex-Environment komplett neu ohne reledmac
\makeatletter
\renewenvironment{theindex}{%
  \section*{\indexname}%
  \setlength{\parindent}{0pt}%
  \setlength{\parskip}{0pt plus 0.3pt}%
  \let\item\@idxitem
}{%
  \clearpage
}
\makeatother

\IfFileExists{\jobname-pw.ind}{\input{\jobname-pw.ind}}{}

\end{document}

      