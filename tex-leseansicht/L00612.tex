%% latex-leseansicht-vorspann.tex
%% Vorspann für die Leseansicht.
%% Lädt die gemeinsame Datei latex-vorspann.tex mit nicht gesetztem Schalter.

\newif\ifkorrekturansicht
\korrekturansichtfalse

\input{../tex-inputs/latex-vorspann}


               \section[Peter Altenberg an Arthur Schnitzler, {[}30.? 10. 1896{]}]{ Peter Altenberg an Arthur Schnitzler, {[}30.? 10. 1896{]}}\nopagebreak\mylabel{v}\rehead{ }\begin{ledgroupsized}[t]{13cm}\normalsize\beginnumbering\briefempfaengerindex{Schnitzler, Arthur@\textsc{Schnitzler, Arthur}!zzzAltenberg, Peter@\emph{von Peter Altenberg}!1896-10-301@{{[}30.? 10. 1896{]}}|(be} \toendnotes[C]{\smallbreak\pagebreak[2]} \Standort{CUL, Schnitzler, B 2.}
\physDesc{Brief, 1 Blatt, 3 Seiten
\newline{}Handschrift: schwarze Tinte, deutsche Kurrent
\newline{}Schnitzler: 1) mit Bleistift auf das falsche Jahr datiert: »Nov 97« 2) mit rotem Buntstift eine Unterstreichung\newline{}Ordnung: mit Bleistift von unbekannter Hand nummeriert:
                                 »6« }\buchAbdrucke{\weitereDrucke{1) Kurt Bergel: \emph{Arthur Schnitzlers unveröffentlichte Tragikomödie Das Wort.} In: \emph{Studies in Arthur Schnitzler. Centennial Commemorative
                        Volume}. Hg. Herbert W. Reichert und Herman Salinger. Chapel Hill: \emph{University of North Carolina Press} 1963, S. 20 (UNC Studies in the Germanic Languages and Literatures, 42).} \weitereDrucke{2) Arthur Schnitzler: \emph{Das Wort. Tragikomödie in fünf Akten. Fragment}. Aus dem Nachlaß hg. und eingel. von Kurt Bergel. Frankfurt am Main: \emph{S. Fischer} 1966, S. 8–9.} \weitereDrucke{3) Peter Altenberg: \emph{Die Selbsterfindung eines Dichters. Briefe und Dokumente
                        1892–1896}. Hg. und mit einem Nachwort von Leo A. Lensing. Göttingen: \emph{Wallstein} 2009, S. 77.} }\toendnotes[C]{\smallbreak}\pstart{}{\pb}Lieber \textsc{D\textsuperscript{r.}} Arthur Schnitzler:\pend\pstart
           Sie können ſich gar nicht vorſtellen, wie tief mich ihre wunderbare Aufmerkſamkeit
               ergriffen hat.\pend
           \pstart
           Sie haben einem Bankrottirer des Lebens zu ſeinen ſparſamen Augenblicken des Glückes
               einen heiligen Augenblick hinzugefügt.\pend
           \pstart
           Mögen Sie, edler Sieger im Leben, nicht ſich wundern, wenn Einer, der durch
               körperliche, ſeeliſche und ökonomiſche Leiden beſiegt und zerdrückt \introOben{}iſt\introOben{}, manchesmal mit Verwunderung auf Jene blickt, {\pb}welchen das Schickſal freundlicher
               lächelt. Mögen Sie mir es verzeihen, der ich die »\uline{ewige
                  Bewegung}«, das »\uline{innere Stürmen}« für das
               Schönſte halte, wenn ich mit Verwunderung auf ihren innigeren Freundeskreis blicke,
               in welchem uralte Greiſe wie Leo Ebermann\pwindex{Ebermann, Leo 16.07.1863 – 09.10.1914@\textsc{Ebermann, Leo} (16.07.1863 – 09.10.1914), \emph{Schriftsteller, Journalist, Rechtswissenschaftler}|pw} und Gustav Schwarzkopf\pwindex{Schwarzkopf, Gustav 07.11.1853 – 13.11.1939@\textsc{Schwarzkopf, Gustav} (07.11.1853 – 13.11.1939), \emph{Schriftsteller}|pw}{ }Stammſitze haben.\pend
           \pstart
           Merkwürdig, Sie waren der Erſte, der mir über meine Manuſkripte erlöſende Worte
               ſagte. Nun bringen Sie mir ein wundervolles Urtheil{ }{\pb}von G. Hauptmann\pwindex{Hauptmann, Gerhart 15.11.1862 – 06.06.1946@\textsc{Hauptmann, Gerhart} (15.11.1862 – 06.06.1946), \emph{Schriftsteller}|pw}.\pend
           \pstart
           Sie haben ſich i{\geminationm}er fein und zart gegen mich
               benommen.\pend
           \pstart
           Möge in kommender Zeit ein freundſchaftliche\strikeout{s}res
               Zuſammenleben mir Gelegenheit geben, meine keimenden Neigungen auswachſsen zu laſſen.
               Das wünſche ich mir!\pend
           \pstart
           Schreiben Sie mir aus Berlin\oindex{Berlin@\textbf{Berlin}|pw}. Sie erleben dort
               gewiſs ſehr viel. Ich ſelbſt lebe in Sehnſucht nach meiner ſchwarzen Freundin \label{K_L00612_1v}\edtext{\uline{\textsc{Nahbadûh}}\pwindex{Nah-Badû @\textsc{Nah-Badû}, \emph{Schausteller/Schaustellerin}|pw}}{\lemma{\textnormal{\emph{Nahbadûh}}}\Cendnote{\textnormal{Dabei handelt es sich um eine der
                  Schaustellerinnen des in Wien\oindex{Wien@\textbf{Wien}|pwk} errichteten Afrika-Dorfes\oindex{Schloss Schoenbrunn@\textbf{Schloß Schönbrunn}|pwkv}, das Altenberg\pwindex{Altenberg, Peter 09.03.1859 – 08.01.1919@\textsc{Altenberg, Peter} (09.03.1859 – 08.01.1919), \emph{Schriftsteller}|pwk} frequentierte. Seine Liebe zu
                  derselben kommt im Buch \emph{Ashantee}\pwindex{Altenberg, Peter 09.03.1859 – 08.01.1919@\textsc{Altenberg, Peter} (09.03.1859 – 08.01.1919), \emph{Schriftsteller}!Ashantee1897@\strich\emph{Ashantee} {[}1897{]}|pwk} (Berlin:
                        \emph{S. Fischer}\orgindex{S. Fischer Verlag@S. Fischer Verlag|pwk}1897) mehrfach zum Ausdruck. Es handelt sich dabei aber nicht um eine
                  literarische Figur, sondern um die Literarisierung einer Leidenschaft, wie Georg Hirschfeld\pwindex{Hirschfeld, Georg 11.02.1873 – 17.01.1942@\textsc{Hirschfeld, Georg} (11.02.1873 – 17.01.1942), \emph{Schriftsteller}|pwk} andeutet (Georg Hirschfeld\pwindex{Hirschfeld, Georg 11.02.1873 – 17.01.1942@\textsc{Hirschfeld, Georg} (11.02.1873 – 17.01.1942), \emph{Schriftsteller}|pwk}: \emph{Wiener Erinnerungen}\pwindex{Hirschfeld, Georg 11.02.1873 – 17.01.1942@\textsc{Hirschfeld, Georg} (11.02.1873 – 17.01.1942), \emph{Schriftsteller}!Wiener Erinnerungen20. 12. 1931@\strich\emph{Wiener Erinnerungen} {[}20. 12. 1931{]}|pwk}. In: \emph{Neue
                        Freie Presse}\pwindex{Neue Freie Presse1864 – 1939@\emph{Neue Freie Presse}|pwk}, Nr. 24163, 20. 12. 1931,
                  S. 31).}}}\label{K_L00612_1h}, dieſem »\label{K_L00612_2v}\edtext{letzten Wahnſinne meiner Seele}{\lemma{\textnormal{\emph{letzten … Seele}}}\Cendnote{\textnormal{Sofern
                  es als Zitat gemeint ist, könnte es auf Lord
                     Byron\pwindex{Byron, George Gordon Noel Lord 22.01.1788 – 19.04.1824@\textsc{Byron, George Gordon Noël Lord} (22.01.1788 – 19.04.1824), \emph{Schriftsteller/Schriftstellerin}|pwk} (\emph{The Giaour}\pwindex{Byron, George Gordon Noel Lord 22.01.1788 – 19.04.1824@\textsc{Byron, George Gordon Noël Lord} (22.01.1788 – 19.04.1824), \emph{Schriftsteller/Schriftstellerin}!Giaur1813@\strich\emph{Der Giaur} {[}1813{]}|pwk}: »The cherish’d madness of my heart\pwindex{Byron, George Gordon Noel Lord 22.01.1788 – 19.04.1824@\textsc{Byron, George Gordon Noël Lord} (22.01.1788 – 19.04.1824), \emph{Schriftsteller/Schriftstellerin}!Giaur1813@\strich\emph{Der Giaur} {[}1813{]}|pwkv}«, deutsch
                     »Geliebter Wahnsinn meiner Seele«, \emph{Lord Byron\pwindex{Byron, George Gordon Noel Lord 22.01.1788 – 19.04.1824@\textsc{Byron, George Gordon Noël Lord} (22.01.1788 – 19.04.1824), \emph{Schriftsteller/Schriftstellerin}|pwk}’s sämmtliche Werke}\pwindex{Byron, George Gordon Noel Lord 22.01.1788 – 19.04.1824@\textsc{Byron, George Gordon Noël Lord} (22.01.1788 – 19.04.1824), \emph{Schriftsteller/Schriftstellerin}!Saemmtliche Werke1839@\strich\emph{Sämmtliche Werke} {[}1839{]}|pwk}. Nach
                     den Anforderungen unserer Zeit neu übersetzt von Mehreren. Siebenter Band.
                     Stuttgart: \emph{Hoffmann’sche Verlags-Buchhandlung}{ }1839, S. 96) oder Friedrich
                     Halm\pwindex{Halm, Friedrich 02.04.1806 – 22.05.1871@\textsc{Halm, Friedrich} (02.04.1806 – 22.05.1871), \emph{Schriftsteller}|pwk} (»O Wahnsinn meiner
                     Seele, / Der Wirklichkeit in leerem Traum vermengt!\pwindex{Byron, George Gordon Noel Lord 22.01.1788 – 19.04.1824@\textsc{Byron, George Gordon Noël Lord} (22.01.1788 – 19.04.1824), \emph{Schriftsteller/Schriftstellerin}!Giaur1813@\strich\emph{Der Giaur} {[}1813{]}|pwkv}«, \emph{Griseldis}\pwindex{Byron, George Gordon Noel Lord 22.01.1788 – 19.04.1824@\textsc{Byron, George Gordon Noël Lord} (22.01.1788 – 19.04.1824), \emph{Schriftsteller/Schriftstellerin}!Giaur1813@\strich\emph{Der Giaur} {[}1813{]}|pwk}. Dramatisches Gedicht von Friedrich Halm\pwindex{Halm, Friedrich 02.04.1806 – 22.05.1871@\textsc{Halm, Friedrich} (02.04.1806 – 22.05.1871), \emph{Schriftsteller}|pwk}. Wien: \emph{Carl
                        Gerold}{ }1837, S. 109) zurückgehen.}}}\label{K_L00612_2h}«!\pend
           \pstart Ihr \spacefill\mbox{Peter Altenberg}\pend{}\endnumbering\briefempfaengerindex{Schnitzler, Arthur@\textsc{Schnitzler, Arthur}!zzzAltenberg, Peter@\emph{von Peter Altenberg}!1896-10-301@{{[}30.? 10. 1896{]}}|)be}\mylabel{h}\end{ledgroupsized}  \newcommand{\dateiname}{L00612}\newcommand{\titel}{Peter Altenberg an Arthur Schnitzler, [30.? 10. 1896]}\newcommand{\editorInnen}{Martin Anton Müller und Gerd-Hermann Susen}
            \footnotesize
\begin{ledgroupsized}[t]{11.5cm}
\doendnotes{C}
\end{ledgroupsized}
         %% latex-leseansicht-abspann.tex
%% Abspann für die Leseansicht.
%% Der Schalter \ifkorrekturansicht ist bereits durch den Vorspann gesetzt.

%% latex-abspann.tex
%% Gemeinsamer Abspann für Korrekturansicht und Leseansicht.
%% Setzt den Schalter \ifkorrekturansicht voraus (gesetzt in den
%% einbindenden Dateien latex-korrekturansicht-abspann.tex bzw.
%% latex-leseansicht-abspann.tex).
%% ---------------------------------------------------------------

\normalsize

% Das esempio-Environment wird nur in der Leseansicht benötigt
\ifkorrekturansicht\else
\newenvironment{esempio}[3]%
{
    \vspace{1.5ex}
    \rlap{\underline{#1}}
    \par
    \setlength{\parindent}{0cm}
    \nopagebreak
    \leftskip=#2cm
    \rightskip=#3cm
}
{
    \par
}
\fi

\doendnotes{C}
\bigskip
\vfill

\clearpage

\footnotesize

\ifkorrekturansicht
  \lohead{\textsc{register}}
\fi

% theindex-Environment neu definieren ohne reledmac
\makeatletter
\renewenvironment{theindex}{%
  \ifkorrekturansicht
    \section*{\indexname}%
  \else
    \subsubsection*{Index der erwähnten Entitäten}%
  \fi
  \setlength{\parindent}{0pt}%
  \setlength{\parskip}{0pt plus 0.3pt}%
  \let\item\@idxitem
}{%
  \ifkorrekturansicht\clearpage\fi
}
\makeatother

\IfFileExists{\jobname-pw.ind}{\input{\jobname-pw.ind}}{}

% Quellenangabe nur in der Leseansicht
\ifkorrekturansicht\else
% Fallback-Definitionen, falls die .tex-Datei \titel etc. nicht gesetzt hat
\providecommand{\titel}{}
\providecommand{\editorInnen}{}
\providecommand{\dateiname}{\jobname}

\vspace{3cm}

\vfill

\footnotesize
\textsc{Quelle}: \titel. Herausgegeben von {\editorInnen}. In: \emph{Arthur Schnitzler: Briefwechsel mit Autorinnen und Autoren}.
 Digitale Edition, https://schnitzler-briefe.acdh.oeaw.ac.at/{\dateiname}.html (Stand \today)
\fi

\end{document}


      