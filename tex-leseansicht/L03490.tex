%% latex-leseansicht-vorspann.tex
%% Vorspann für die Leseansicht.
%% Lädt die gemeinsame Datei latex-vorspann.tex mit nicht gesetztem Schalter.

\newif\ifkorrekturansicht
\korrekturansichtfalse

\input{../tex-inputs/latex-vorspann}

\begin{center}
            \textcolor{red}{ENTWURF, NICHT FERTIG KORRIGIERT}
                      \end{center}
            
         
         \renewcommand{\erwaehntePersonen}{Personen: Felix Salten, Ottilie Salten, Olga Schnitzler, Karl Werkmann}
         \renewcommand{\erwaehnteInstitutionen}{Institutionen: Burgtheater, Die Zeit, Franz-Grillparzer-Preis}
         \renewcommand{\erwaehnteOrte}{Orte: Heiligenstadt, Wien}
         \renewcommand{\erwaehnteWerke}{Werke: Der Schleier der Beatrice. Schauspiel in fünf Akten, Verleihung des Grillparzer-Preises an Artur Schnitzler, Zwischenspiel. Komödie in drei Akten}
               \section[ Felix Salten an Arthur Schnitzler, 15. 1. 1908]{ Felix Salten an Arthur Schnitzler, 15. 1. 1908}\nopagebreak\mylabel{v}\rehead{ }\begin{ledgroupsized}[t]{13cm}\normalsize\beginnumbering \toendnotes[C]{\smallbreak\pagebreak[2]} \Standort{CUL, Schnitzler, B 89, B 1.}
\physDesc{Brief, 1 Blatt, 1 Seite, 672 Zeichen
\newline{}Handschrift: schwarze Tinte, lateinische Kurrent
\newline{}Schnitzler: mit Bleistift Vermerk: »\textsc{Salten}« 
\newline{}Ordnung: mit Bleistift von unbekannter Hand nummeriert: »239« }\toendnotes[C]{\smallbreak}\pstart
           \raggedleft{}{\pb}Heiligenstadt\oindex{Heiligenstadt@\textbf{Heiligenstadt}|pw}, 15. I. 08\pend
           \pstart{}Lieber,\pend\pstart
           eben wird mir aus der Redaktion\orgindex{Zeit@Die Zeit|pwv}
               telefonirt, dass Ihr »Zwischenspiel\pwindex{Schnitzler, Arthur 15.05.1862 – 21.10.1931@\textsc{Schnitzler, Arthur} (15.05.1862 – 21.10.1931), \emph{Schriftsteller, Mediziner}!Zwischenspiel. Komoedie in drei Akten1905-10-12@\strich\emph{Zwischenspiel. Komödie in drei Akten} {[}1905-10-12{]}|pw}« den \label{K_L03490-1v}\edtext{Grillparzer-Preis\orgindex{Franz-Grillparzer-Preis@Franz-Grillparzer-Preis|pw}}{\lemma{\textnormal{\emph{Grillparzer-Preis}}}\Cendnote{\textnormal{Das Auswahlkomitee hatte am 15. 1. 1908
                  entschieden, dass Schnitzler\pwindex{Schnitzler, Arthur 15.05.1862 – 21.10.1931@\textsc{Schnitzler, Arthur} (15.05.1862 – 21.10.1931), \emph{Schriftsteller, Mediziner}|pwk} für seine
                  Komödie \emph{Zwischenspiel}\pwindex{Schnitzler, Arthur 15.05.1862 – 21.10.1931@\textsc{Schnitzler, Arthur} (15.05.1862 – 21.10.1931), \emph{Schriftsteller, Mediziner}!Zwischenspiel. Komoedie in drei Akten1905-10-12@\strich\emph{Zwischenspiel. Komödie in drei Akten} {[}1905-10-12{]}|pwk} der mit 5.000 Kronen
                  dotierte \emph{Grillparzer-Preis}\orgindex{Franz-Grillparzer-Preis@Franz-Grillparzer-Preis|pwk} verliehen werden
                  soll. In den Jahren zuvor war er zwar immer wieder als Favorit gehandelt worden,
                  doch stellte das Zerwürfnis mit dem \emph{Burgtheater}\orgindex{Burgtheater@Burgtheater|pwk} in Folge der Rückgabe von \emph{Der
                     Schleier der Beatrice}\pwindex{Schnitzler, Arthur 15.05.1862 – 21.10.1931@\textsc{Schnitzler, Arthur} (15.05.1862 – 21.10.1931), \emph{Schriftsteller, Mediziner}!Schleier der Beatrice. Schauspiel in fuenf Akten1900-12-01@\strich\emph{Der Schleier der Beatrice. Schauspiel in fünf Akten} {[}1900-12-01{]}|pwk} (1901) ein Hindernis dar.
                  Seit Sommer 1905 war der Konflikt behoben und Schnitzler\pwindex{Schnitzler, Arthur 15.05.1862 – 21.10.1931@\textsc{Schnitzler, Arthur} (15.05.1862 – 21.10.1931), \emph{Schriftsteller, Mediziner}|pwk} konnte wieder bei der Preisvergabe\orgindex{Franz-Grillparzer-Preis@Franz-Grillparzer-Preis|pwkv} berücksichtigt
                  werden.}}}\label{K_L03490-1h} bekam. Ich habe eine große Freude drüber, und sende Ihnen meinen
               herzlichen Glückwunsch. Es war das Beste, was die Herren tun konnten, – wenn es ihnen
               auch, wie’s scheint, \label{K_L03490-2v}\edtext{nicht so bald
               eingefallen ist}{\lemma{\textnormal{\emph{nicht … ist}}}\Cendnote{\textnormal{Salten\pwindex{Salten, Felix 06.09.1869 – 08.10.1945@\textsc{Salten, Felix} (06.09.1869 – 08.10.1945), \emph{Schriftsteller, Journalist}|pwk} kannte also bereits das Interview\pwindex{Verleihung des Grillparzer-Preises an Artur Schnitzler15. 01. 1908@\emph{Verleihung des Grillparzer-Preises an Artur Schnitzler} {[}15. 01. 1908{]}|pwkv} von Karl Werkmann\pwindex{Werkmann, Karl 14.09.1878 – 24.12.1951@\textsc{Werkmann, Karl} (14.09.1878 – 24.12.1951), \emph{Journalist}|pwk}, siehe A. S.: \emph{»Das Zeitlose ist von kürzester Dauer«}, [Karl Werkmann]: Verleihung des Grillparzer-Preises an Artur Schnitzler, 16. 1. 1908}}}\label{K_L03490-2h} – und
               hoffentlich kommt diese Freude auch in einem guten Moment, und es \label{K_L03490-3v}\edtext{geht Ihrer Frau\pwindex{Schnitzler, Olga 17.01.1882 – 13.01.1970@\textsc{Schnitzler, Olga} (17.01.1882 – 13.01.1970), \emph{Schauspielerin, Sängerin}|pwv} immer besser und besser}{\lemma{\textnormal{\emph{geht … besser}}}\Cendnote{\textnormal{siehe Felix Salten an Arthur Schnitzler, [10. 12. 1907]}}}\label{K_L03490-3h}.\pend
           \pstart
           Wir sind alle krank. Influenza. Und wir liegen auch alle seit Samstag im Bett. Otti\pwindex{Salten, Ottilie 07.03.1868 – 22.06.1942@\textsc{Salten, Ottilie} (07.03.1868 – 22.06.1942), \emph{Schauspielerin}|pw} hat sogar
               eine Blinddarmreizung. Aber wir hoffen, dass nächste Woche alles wieder gut ist.\pend
           \pstart
           Nochmals herzliche Glückwünsche, und viele Güße an Sie u. Frau Olga\pwindex{Schnitzler, Olga 17.01.1882 – 13.01.1970@\textsc{Schnitzler, Olga} (17.01.1882 – 13.01.1970), \emph{Schauspielerin, Sängerin}|pw}.\pend
           \pstart
           Ihr {\\[\baselineskip]}\spacefill\mbox{Salten}\pend
           \leftskip=0em{}
         
         \endnumbering\mylabel{h}\end{ledgroupsized}  \newcommand{\dateiname}{L03490}\newcommand{\titel}{Felix Salten an Arthur Schnitzler, 15. 1. 1908}\newcommand{\editorInnen}{Martin Anton Müller und Laura Untner}%% latex-leseansicht-abspann.tex
%% Abspann für die Leseansicht.
%% Der Schalter \ifkorrekturansicht ist bereits durch den Vorspann gesetzt.

%% latex-abspann.tex
%% Gemeinsamer Abspann für Korrekturansicht und Leseansicht.
%% Setzt den Schalter \ifkorrekturansicht voraus (gesetzt in den
%% einbindenden Dateien latex-korrekturansicht-abspann.tex bzw.
%% latex-leseansicht-abspann.tex).
%% ---------------------------------------------------------------

\normalsize

% Das esempio-Environment wird nur in der Leseansicht benötigt
\ifkorrekturansicht\else
\newenvironment{esempio}[3]%
{
    \vspace{1.5ex}
    \rlap{\underline{#1}}
    \par
    \setlength{\parindent}{0cm}
    \nopagebreak
    \leftskip=#2cm
    \rightskip=#3cm
}
{
    \par
}
\fi

\doendnotes{C}
\bigskip
\vfill

\clearpage

\footnotesize

\ifkorrekturansicht
  \lohead{\textsc{register}}
\fi

% theindex-Environment neu definieren ohne reledmac
\makeatletter
\renewenvironment{theindex}{%
  \ifkorrekturansicht
    \section*{\indexname}%
  \else
    \subsubsection*{Index der erwähnten Entitäten}%
  \fi
  \setlength{\parindent}{0pt}%
  \setlength{\parskip}{0pt plus 0.3pt}%
  \let\item\@idxitem
}{%
  \ifkorrekturansicht\clearpage\fi
}
\makeatother

\IfFileExists{\jobname-pw.ind}{\input{\jobname-pw.ind}}{}

% Quellenangabe nur in der Leseansicht
\ifkorrekturansicht\else
% Fallback-Definitionen, falls die .tex-Datei \titel etc. nicht gesetzt hat
\providecommand{\titel}{}
\providecommand{\editorInnen}{}
\providecommand{\dateiname}{\jobname}

\vspace{3cm}

\vfill

\footnotesize
\textsc{Quelle}: \titel. Herausgegeben von {\editorInnen}. In: \emph{Arthur Schnitzler: Briefwechsel mit Autorinnen und Autoren}.
 Digitale Edition, https://schnitzler-briefe.acdh.oeaw.ac.at/{\dateiname}.html (Stand \today)
\fi

\end{document}


      