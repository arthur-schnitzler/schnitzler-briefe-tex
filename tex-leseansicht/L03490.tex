%% latex-korrekturansicht-vorspann.tex
%% Vorspann für die Korrekturansicht.
%% Lädt die gemeinsame Datei latex-vorspann.tex mit gesetztem Schalter.

\newif\ifkorrekturansicht
\korrekturansichttrue

\input{../tex-inputs/latex-vorspann}


\section[ Felix Salten an Arthur Schnitzler, 15. 1. 1908]{L03490 Felix Salten an Arthur Schnitzler, 15. 1. 1908}
\nopagebreak\mylabel{L03490v}
\rehead{ }\normalsize\beginnumbering\briefempfaengerindex{Schnitzler, Arthur@\textsc{Schnitzler, Arthur}!zzzSalten, Felix@\emph{von Felix Salten}!1908-01-151@{15. 1. 1908}|(be}
\toendnotes[C]{\smallbreak\pagebreak[2]}\Standort{CUL, Schnitzler, B 89, B 1.}
\physDesc{Brief, 1 Blatt, 1 Seite, 672 Zeichen
\newline{}Handschrift: schwarze Tinte, lateinische Kurrent
\newline{}Schnitzler: mit Bleistift Vermerk: »\textsc{Salten}« 
\newline{}Ordnung: mit Bleistift von unbekannter Hand nummeriert: »239« }\toendnotes[C]{\smallbreak}
\pstart
           \raggedleft{}{\pb}Heiligenstadt\oindex{Heiligenstadt@\textbf{Heiligenstadt}, \emph{P.PPL}|pw}, 15. I. 08\pend
           
\pstart{}Lieber,\pend\vspace{0.5em}
\pstart
           eben wird mir aus der Redaktion\orgindex{Zeit@Die Zeit|pwv}
               telefonirt, dass Ihr »Zwischenspiel\pwindex{Zwischenspiel. Komoedie in drei Akten@\emph{Zwischenspiel. Komödie in drei Akten}|pw}« den \label{K_L03490-1v}\edtext{Grillparzer-Preis\orgindex{Franz-Grillparzer-Preis@Franz-Grillparzer-Preis|pw}}{\lemma{\textnormal{\emph{Grillparzer-Preis}}}\Cendnote{\textnormal{Das Auswahlkomitee hatte am 15. 1. 1908
                  entschieden, Schnitzler für seine
                  Komödie \emph{Zwischenspiel}\pwindex{Zwischenspiel. Komoedie in drei Akten@\emph{Zwischenspiel. Komödie in drei Akten}|pwk} den mit 5000 Kronen
                  dotierten \emph{Grillparzer-Preis}\orgindex{Franz-Grillparzer-Preis@Franz-Grillparzer-Preis|pwk} zu verleihen. In
                  den Jahren zuvor war er zwar immer wieder als Favorit gehandelt worden, doch
                  stellte das Zerwürfnis mit dem \emph{Burgtheater}\orgindex{Burgtheater@Burgtheater|pwk} in
                  Folge der Rückgabe von \emph{Der Schleier der
                     Beatrice}\pwindex{Schleier der Beatrice. Schauspiel in fuenf Akten@\emph{Der Schleier der Beatrice. Schauspiel in fünf Akten}|pwk} (1901) ein Hindernis dar. Seit Sommer 1905 war der Konflikt behoben und Schnitzler konnte wieder bei der Preisvergabe\orgindex{Franz-Grillparzer-Preis@Franz-Grillparzer-Preis|pwkv} berücksichtigt
                  werden.}}}\label{K_L03490-1} bekam. Ich habe eine große Freude drüber, und sende Ihnen meinen
               herzlichen Glückwunsch. Es war das Beste, was die Herren tun konnten, – wenn es ihnen
               auch, wie’s scheint, \label{K_L03490-2v}\edtext{nicht so bald
               eingefallen ist}{\lemma{\textnormal{\emph{nicht … ist}}}\Cendnote{\textnormal{Salten\pwindex{Salten, Felix 06.09.1869 – 08.10.1945@\textsc{Salten, Felix} (06.09.1869 – 08.10.1945), \emph{Schriftsteller/Schriftstellerin, Journalist/Journalistin, Chefredakteur/Chefredakteurin}|pwk} kannte also bereits das Interview\pwindex{Verleihung des Grillparzer-Preises an Artur Schnitzler@\emph{Verleihung des Grillparzer-Preises an Artur Schnitzler}|pwkv}, das am nächsten Tag in seiner Zeitung
                  erscheinen sollte: A. S.: \emph{»Das Zeitlose ist von kürzester Dauer«}, [Karl Werkmann]: Verleihung des Grillparzer-Preises an Artur Schnitzler, 15. 1. 1908.
               }}}\label{K_L03490-2} – und hoffentlich kommt diese Freude auch in einem guten Moment, und es
                  \label{K_L03490-3v}\edtext{geht Ihrer Frau\pwindex{Schnitzler, Olga 17.01.1882 – 13.01.1970@\textsc{Schnitzler, Olga} (17.01.1882 – 13.01.1970), \emph{Schauspieler/Schauspielerin, Sänger/Sängerin}|pwv} immer besser und besser}{\lemma{\textnormal{\emph{geht … besser}}}\Cendnote{\textnormal{Vgl. Felix Salten an Arthur Schnitzler, [10. 12. 1907].
               }}}\label{K_L03490-3}.\pend
           
\pstart
           Wir sind alle krank. Influenza. Und wir liegen auch alle seit Samstag im Bett. Otti\pwindex{Salten, Ottilie 07.03.1868 – 22.06.1942@\textsc{Salten, Ottilie} (07.03.1868 – 22.06.1942), \emph{Schauspieler/Schauspielerin}|pw} hat sogar
               eine Blinddarmreizung. Aber wir hoffen, dass nächste Woche alles wieder gut ist.\pend
           
\pstart
           Nochmals herzliche Glückwünsche, und viele Güße an Sie u. Frau Olga\pwindex{Schnitzler, Olga 17.01.1882 – 13.01.1970@\textsc{Schnitzler, Olga} (17.01.1882 – 13.01.1970), \emph{Schauspieler/Schauspielerin, Sänger/Sängerin}|pw}.\pend
           
\pstart
           Ihr {\\[\baselineskip]}\spacefill\mbox{Salten}\pend
           \leftskip=0em{}\selectlanguage{ngerman}\endnumbering\briefempfaengerindex{Schnitzler, Arthur@\textsc{Schnitzler, Arthur}!zzzSalten, Felix@\emph{von Felix Salten}!1908-01-151@{15. 1. 1908}|)be}\mylabel{L03490h}  \normalsize

\doendnotes{C}
\bigskip
\vfill

\clearpage

\footnotesize

\lohead{\textsc{register}}

% Definiere theindex-Environment komplett neu ohne reledmac
\makeatletter
\renewenvironment{theindex}{%
  \section*{\indexname}%
  \setlength{\parindent}{0pt}%
  \setlength{\parskip}{0pt plus 0.3pt}%
  \let\item\@idxitem
}{%
  \clearpage
}
\makeatother

\IfFileExists{\jobname-pw.ind}{\input{\jobname-pw.ind}}{}

\end{document}

      