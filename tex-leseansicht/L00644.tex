%% latex-korrekturansicht-vorspann.tex
%% Vorspann für die Korrekturansicht.
%% Lädt die gemeinsame Datei latex-vorspann.tex mit gesetztem Schalter.

\newif\ifkorrekturansicht
\korrekturansichttrue

\input{../tex-inputs/latex-vorspann}


\section[Hugo von Hofmannsthal an Arthur Schnitzler, 9. 2. 1897]{L00644 Hugo von Hofmannsthal an Arthur Schnitzler, 9. 2. 1897}
\nopagebreak\mylabel{L00644v}
\rehead{ }\normalsize\beginnumbering\briefempfaengerindex{Schnitzler, Arthur@\textsc{Schnitzler, Arthur}!zzzHofmannsthal, Hugo von@\emph{von Hugo von Hofmannsthal}!1897-02-091@{9. 2. 1897}|(be}
\toendnotes[C]{\smallbreak\pagebreak[2]}\Standort{CUL, Schnitzler, B 43.}
\physDesc{Kartenbrief, 678 Zeichen
\newline{}Handschrift: schwarze Tinte, deutsche Kurrent
\newline{}Versand: 1) Rohrpost  2) Stempel: »\nobreak{}\oindex{III., Landstrasse@\textbf{III., Landstraße}, \emph{A.ADM3}|pwk}Wien 3/3, 9 II 97, 12–N\nobreak{}«.  3) Stempel: »\nobreak{}\oindex{IX., Alsergrund@\textbf{IX., Alsergrund}, \emph{A.ADM3}|pwk}Wien 9/2, 9 II 97, 12 50N\nobreak{}«. 
\newline{}Schnitzler: mit Bleistift datiert: »9/2 97« 
\newline{}Ordnung: mit Bleistift von unbekannter Hand nummeriert:
                                    »86« }
\buchAbdrucke{\weitereDrucke{Hugo von Hofmannsthal, Arthur Schnitzler: \emph{Briefwechsel}. Frankfurt am Main: \emph{S. Fischer} 1964, S. 77.} }\toendnotes[C]{\smallbreak}\pstart{}{\pb}Herrn D\textsuperscript{r} Arthur Schnitzler\pend{}\pstart{}Wien\oindex{Wien@\textbf{Wien}, \emph{A.ADM2}|pw}\pend{}\pstart{}IX Franckgasse 1\oindex{Frankgasse 1@\textbf{Frankgasse 1}, \emph{Wohngebäude (K.WHS)}|pw}\pend{}{\bigskip}\vspace{1em}
\pstart
           \raggedleft{}{\pb}Dienstag.\pend
           
\pstart{}lieber Arthur\pend\vspace{0.5em}
\pstart
           wollen Sie mir einen großen \label{K_L00644-1v}\edtext{Gefallen}{\lemma{\textnormal{\emph{Gefallen}}}\Cendnote{\textnormal{Hofmannsthal\pwindex{Hofmannsthal, Hugo von 1874-02-01 – 1929-07-15@\textsc{Hofmannsthal, Hugo von} (1874-02-01 – 1929-07-15), \emph{Schriftsteller/Schriftstellerin}|pwk} glaubte zu diesem Zeitpunkt,
                     Hermine Benedict\pwindex{Schaffgotsch, Hermine von 25.11.1871 – 25.11.1928@\textsc{Schaffgotsch, Hermine von} (25.11.1871 – 25.11.1928)|pwk} wäre in ihn verliebt.
                  Die Klärung der Sache, die auch Schnitzler
                  als dritten, nicht amourös Interessierten involvierte, zog sich bis in den
                     März.}}}\label{K_L00644-1} thuen? telephonieren Sie zwiſchen 2 und
                  4 der Minnie\pwindex{Schaffgotsch, Hermine von 25.11.1871 – 25.11.1928@\textsc{Schaffgotsch, Hermine von} (25.11.1871 – 25.11.1928)|pw} 12140 und fragen
               Sie irgend etwas gleichgiltiges z. B. Sie hätten gehört, daſs Sonntag
               die \label{K_L00644-2v}\edtext{2\textsuperscript{te}
                  Vorſtellung}{\lemma{\textnormal{\emph{2\textsuperscript{te} Vorſtellung}}}\Cendnote{\textnormal{Privatinszenierung von Hofmannsthals\pwindex{Hofmannsthal, Hugo von 1874-02-01 – 1929-07-15@\textsc{Hofmannsthal, Hugo von} (1874-02-01 – 1929-07-15), \emph{Schriftsteller/Schriftstellerin}|pwk}{ }\emph{Was die Braut geträumt hat. Ein Gelegenheitsgedicht}\pwindex{Was die Braut getraeumt hat. Ein Gelegenheitsgedicht@\emph{Was die Braut geträumt hat. Ein Gelegenheitsgedicht}|pwk}, die
                  zweite Vorstellung fand am Donnerstag, den 18. 2. 1897 statt.}}}\label{K_L00644-2}{ }ſein ſoll, ob es wahr iſt?\pend
           
\pstart
           und wenn Sie mit ihr ſelbſt ſprechen können und es unauffällig ſich anknüpfen läſst
               (an das Hereinfahren Freitag{ }abend) fragen Sie ſie, wie es ihr geht und ſchreiben mir das \uline{pneumatiſch}, bitte! Wenn Sie aber nur für \uline{möglich} halten, daſs es auffallen oder daſs man den Zuſa{\geminationm}enhang errathen könnte, ſo iſt natürlich beſſer Sie
               laſſen es und ich thue es ſelber. Aber bitte antworten Sie jedenfalls!\hspace*{3.5em}Ihr\spacefill\mbox{Hugo.}\pend
           \selectlanguage{ngerman}\endnumbering\briefempfaengerindex{Schnitzler, Arthur@\textsc{Schnitzler, Arthur}!zzzHofmannsthal, Hugo von@\emph{von Hugo von Hofmannsthal}!1897-02-091@{9. 2. 1897}|)be}\mylabel{L00644h}  \normalsize

\doendnotes{C}
\bigskip
\vfill

\clearpage

\footnotesize

\lohead{\textsc{register}}

% Definiere theindex-Environment komplett neu ohne reledmac
\makeatletter
\renewenvironment{theindex}{%
  \section*{\indexname}%
  \setlength{\parindent}{0pt}%
  \setlength{\parskip}{0pt plus 0.3pt}%
  \let\item\@idxitem
}{%
  \clearpage
}
\makeatother

\IfFileExists{\jobname-pw.ind}{\input{\jobname-pw.ind}}{}

\end{document}

      