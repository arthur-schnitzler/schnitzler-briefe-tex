%% latex-leseansicht-vorspann.tex
%% Vorspann für die Leseansicht.
%% Lädt die gemeinsame Datei latex-vorspann.tex mit nicht gesetztem Schalter.

\newif\ifkorrekturansicht
\korrekturansichtfalse

\input{../tex-inputs/latex-vorspann}


         
         \newcommand{\erwaehntePersonen}{Personen: Hermine von Schaffgotsch}
         \newcommand{\erwaehnteOrte}{Orte: Frankgasse, III., Landstraße, IX., Alsergrund, Wien}
         \newcommand{\erwaehnteWerke}{Werke: Was die Braut geträumt hat. Ein Gelegenheitsgedicht}
               \section[Hugo von Hofmannsthal an Arthur Schnitzler, 9. 2. 1897]{ Hugo von Hofmannsthal an Arthur Schnitzler, 9. 2. 1897}\nopagebreak\mylabel{v}\rehead{ }\begin{ledgroupsized}[t]{13cm}\normalsize\beginnumbering \toendnotes[C]{\smallbreak\pagebreak[2]} \Standort{CUL, Schnitzler, B 43.}
\physDesc{Kartenbrief
\newline{}Handschrift: schwarze Tinte, deutsche Kurrent\newline{}Versand: 1) Rohrpost  2) Stempel: »\nobreak{}\oindex{III., Landstrasse@\textbf{III., Landstraße}|pwk}Wien 3/3, 9 II 97, 12–N\nobreak{}«.  3) Stempel: »\nobreak{}\oindex{IX., Alsergrund@\textbf{IX., Alsergrund}|pwk}Wien 9/2, 9 II 97, 12 50N\nobreak{}«. 
\newline{}Schnitzler: mit Bleistift datiert: »9/2 97« \newline{}Ordnung: mit Bleistift von unbekannter Hand nummeriert:
                                    »86« }\buchAbdrucke{\weitereDrucke{Hugo von Hofmannsthal, Arthur Schnitzler: \emph{Briefwechsel}. Hg. Therese Nickl und Heinrich Schnitzler. Frankfurt am Main: \emph{S. Fischer} 1964, S. 77.} }\toendnotes[C]{\smallbreak}\pstart{}{\pb}\textcolor{gray}{\textbf{An}}\pend{}\pstart{}Herrn D\textsuperscript{r} Arthur Schnitzler\pend{}\pstart{}\textcolor{gray}{\textbf{in}}{ }Wien\oindex{Wien@\textbf{Wien}|pw}\pend{}\pstart{}IX Franckgasse 1\oindex{Frankgasse@\textbf{Frankgasse}|pw}\pend{}{\bigskip}\pstart
           \raggedleft{}{\pb}Dienstag.\pend
           \pstart{}lieber Arthur\pend\pstart
           wollen Sie mir einen großen \label{K_L00644_1v}\edtext{Gefallen}{\lemma{\textnormal{\emph{Gefallen}}}\Cendnote{\textnormal{Hofmannsthal\pwindex{Hofmannsthal, Hugo von 1874-02-01 – 1929-07-15@\textsc{Hofmannsthal, Hugo von} (1874-02-01 – 1929-07-15), \emph{Schriftsteller}|pwk} glaubte zu diesem Zeitpunkt, Hermine Benedict\pwindex{Schaffgotsch, Hermine von 25.11.1871 – 25.11.1928@\textsc{Schaffgotsch, Hermine von} (25.11.1871 – 25.11.1928)|pwk} wäre in ihn verliebt. Die
                  Klärung der Sache, die auch Schnitzler\pwindex{Schnitzler, Arthur 15.05.1862 – 21.10.1931@\textsc{Schnitzler, Arthur} (15.05.1862 – 21.10.1931), \emph{Schriftsteller, Mediziner}|pwk} als
                  dritten, nicht amourös Interessierten involviert, zieht sich bis in den
                     März.}}}\label{K_L00644_1h} thuen? telephonieren Sie zwiſchen 2 und
                  4 der Minnie\pwindex{Schaffgotsch, Hermine von 25.11.1871 – 25.11.1928@\textsc{Schaffgotsch, Hermine von} (25.11.1871 – 25.11.1928)|pw} 12140 und fragen Sie
               irgend etwas gleichgiltiges z. B. Sie hätten gehört, daſs Sonntag die
                  \label{K_L00644_2v}\edtext{2\textsuperscript{te}
                  Vorſtellung}{\lemma{\textnormal{\emph{2te
                  Vorſtellung}}}\Cendnote{\textnormal{Privatinszenierung
                  von Hofmannsthal\pwindex{Hofmannsthal, Hugo von 1874-02-01 – 1929-07-15@\textsc{Hofmannsthal, Hugo von} (1874-02-01 – 1929-07-15), \emph{Schriftsteller}|pwk}s \emph{Was die Braut geträumt hat. Ein Gelegenheitsgedicht}\pwindex{Hofmannsthal, Hugo von 1874-02-01 – 1929-07-15@\textsc{Hofmannsthal, Hugo von} (1874-02-01 – 1929-07-15), \emph{Schriftsteller}!Was die Braut getraeumt hat. Ein Gelegenheitsgedicht1896@\strich\emph{Was die Braut geträumt hat. Ein Gelegenheitsgedicht} {[}1896{]}|pwk}, die zweite
                  Vorstellung fand am Donnerstag, den 18. 2. 1897 statt.}}}\label{K_L00644_2h}{ }ſein ſoll, ob es wahr iſt?\pend
           \pstart
           und wenn Sie mit ihr ſelbſt ſprechen können und es unauffällig ſich anknüpfen läſst
               (an das Hereinfahren Freitag{ }abend) fragen Sie ſie, wie es ihr geht und ſchreiben mir das \uline{pneumatiſch}, bitte! Wenn Sie aber nur für \uline{möglich} halten, daſs es auffallen oder daſs man den Zuſa{\geminationm}enhang errathen könnte, ſo iſt natürlich beſſer Sie
               laſſen es und ich thue es ſelber. Aber bitte antworten Sie jedenfalls!\hspace*{3.5em}Ihr\spacefill\mbox{Hugo.}\pend
           
         
         \endnumbering\mylabel{h}\end{ledgroupsized}  \newcommand{\dateiname}{L00644}\newcommand{\titel}{Hugo von Hofmannsthal an Arthur Schnitzler, 9. 2. 1897}\newcommand{\editorInnen}{Martin Anton Müller und Gerd-Hermann Susen}%% latex-leseansicht-abspann.tex
%% Abspann für die Leseansicht.
%% Der Schalter \ifkorrekturansicht ist bereits durch den Vorspann gesetzt.

%% latex-abspann.tex
%% Gemeinsamer Abspann für Korrekturansicht und Leseansicht.
%% Setzt den Schalter \ifkorrekturansicht voraus (gesetzt in den
%% einbindenden Dateien latex-korrekturansicht-abspann.tex bzw.
%% latex-leseansicht-abspann.tex).
%% ---------------------------------------------------------------

\normalsize

% Das esempio-Environment wird nur in der Leseansicht benötigt
\ifkorrekturansicht\else
\newenvironment{esempio}[3]%
{
    \vspace{1.5ex}
    \rlap{\underline{#1}}
    \par
    \setlength{\parindent}{0cm}
    \nopagebreak
    \leftskip=#2cm
    \rightskip=#3cm
}
{
    \par
}
\fi

\doendnotes{C}
\bigskip
\vfill

\clearpage

\footnotesize

\ifkorrekturansicht
  \lohead{\textsc{register}}
\fi

% theindex-Environment neu definieren ohne reledmac
\makeatletter
\renewenvironment{theindex}{%
  \ifkorrekturansicht
    \section*{\indexname}%
  \else
    \subsubsection*{Index der erwähnten Entitäten}%
  \fi
  \setlength{\parindent}{0pt}%
  \setlength{\parskip}{0pt plus 0.3pt}%
  \let\item\@idxitem
}{%
  \ifkorrekturansicht\clearpage\fi
}
\makeatother

\IfFileExists{\jobname-pw.ind}{\input{\jobname-pw.ind}}{}

% Quellenangabe nur in der Leseansicht
\ifkorrekturansicht\else
% Fallback-Definitionen, falls die .tex-Datei \titel etc. nicht gesetzt hat
\providecommand{\titel}{}
\providecommand{\editorInnen}{}
\providecommand{\dateiname}{\jobname}

\vspace{3cm}

\vfill

\footnotesize
\textsc{Quelle}: \titel. Herausgegeben von {\editorInnen}. In: \emph{Arthur Schnitzler: Briefwechsel mit Autorinnen und Autoren}.
 Digitale Edition, https://schnitzler-briefe.acdh.oeaw.ac.at/{\dateiname}.html (Stand \today)
\fi

\end{document}


      