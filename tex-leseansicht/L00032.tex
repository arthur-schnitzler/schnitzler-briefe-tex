%% latex-korrekturansicht-vorspann.tex
%% Vorspann für die Korrekturansicht.
%% Lädt die gemeinsame Datei latex-vorspann.tex mit gesetztem Schalter.

\newif\ifkorrekturansicht
\korrekturansichttrue

\input{../tex-inputs/latex-vorspann}


\section[Hugo von Hofmannsthal an Arthur Schnitzler, 12. 8. 1891]{L00032 Hugo von Hofmannsthal an Arthur Schnitzler, 12. 8. 1891}
\nopagebreak\mylabel{L00032v}
\rehead{ }\normalsize\beginnumbering\briefempfaengerindex{Schnitzler, Arthur@\textsc{Schnitzler, Arthur}!zzzHofmannsthal, Hugo von@\emph{von Hugo von Hofmannsthal}!1891-08-122@{12. 8. 1891}|(be}
\toendnotes[C]{\smallbreak\pagebreak[2]}\Standort{CUL, Schnitzler, B 43.}
\physDesc{Kartenbrief, 350 Zeichen
\newline{}Handschrift: 1) schwarze Tinte, deutsche Kurrent\hspace{1em}2) schwarze Tinte, lateinische Kurrent (\noindent{}Adresse)\hspace{1em}
\newline{}Versand: 1) Stempel: »\nobreak{}\oindex{Strobl@\textbf{Strobl}, \emph{A.ADM3}|pwk}Strobl, 12. 8. 91\nobreak{}«.   2) Stempel: »\nobreak{}\oindex{VI., Mariahilf@\textbf{VI., Mariahilf}, \emph{A.ADM3}|pwk}Wien VI 1, 13. 8. 91, 8–9½ V.\nobreak{}«. 
\newline{}Schnitzler: mit Bleistift auf der Textseite zusätzlich datiert: »12. 8 91« 
\newline{}Ordnung: mit Bleistift von unbekannter Hand nummeriert:
                                 »5« }
\buchAbdrucke{\weitereDrucke{Hugo von Hofmannsthal, Arthur Schnitzler: \emph{Briefwechsel}. Frankfurt am Main: \emph{S. Fischer} 1964, S. 12.} }\toendnotes[C]{\smallbreak}\pstart{}{\pb}D\textsuperscript{r}
                  Arthur Schnitzler\pend{}\pstart{}Wien\oindex{Wien@\textbf{Wien}, \emph{A.ADM2}|pw}\pend{}\pstart{}I Kärthnerring 12\oindex{Kaerntnerring 12/Boesendorferstrasse 11@\textbf{Kärntnerring 12/Bösendorferstraße 11}, \emph{Wohngebäude (K.WHS)}|pw}\pend{}{\bigskip}\vspace{1em}
\pstart\center{}{\pb}Lieber Freund!\pend\vspace{0.5em}
\pstart
           Infolge Feſtvorbereitungen für \label{K_L00032-1v}\edtext{Kaiſer\pwindex{Franz Joseph I. von Oesterreich-Ungarn 18.08.1830 – 21.11.1916@\textsc{Franz Joseph I. von Österreich-Ungarn} (18.08.1830 – 21.11.1916), \emph{Kaiser/Kaiserin}|pwv}beſuch}{\lemma{\textnormal{\emph{Kaiſerbeſuch}}}\Cendnote{\textnormal{Am 11. 8. 1891 besuchte Kaiser
                     Franz Joseph I.\pwindex{Franz Joseph I. von Oesterreich-Ungarn 18.08.1830 – 21.11.1916@\textsc{Franz Joseph I. von Österreich-Ungarn} (18.08.1830 – 21.11.1916), \emph{Kaiser/Kaiserin}|pwk}{ }Ischl\oindex{Bad Ischl@\textbf{Bad Ischl}, \emph{P.PPL}|pwk}, um sich dort mit König Alexander von Serbien\pwindex{Serbien, Koenig, Alexander I. 02.08.1876 – 29.05.1903@\textsc{Serbien, König, Alexander I.} (02.08.1876 – 29.05.1903), \emph{Regent/Regentin}|pwk} zu treffen.}}}\label{K_L00032-1} ganz Comité,
               kurz blöd, mache ich Ihnen folgende Vorſchläge: Da Strobl\oindex{Strobl@\textbf{Strobl}, \emph{A.ADM3}|pw} Paradies, Iſchl\oindex{Bad Ischl@\textbf{Bad Ischl}, \emph{P.PPL}|pw}{ }Schweineſtall ſo erwarte ich sie und Hoffmann\pwindex{Beer-Hofmann, Richard 1866-07-11 – 1945-09-26@\textsc{Beer-Hofmann, Richard} (1866-07-11 – 1945-09-26), \emph{Schriftsteller/Schriftstellerin}|pw} an einem der beiden Tage \uline{beſtimmteſtens}.\pend
           
\pstart
           Wenn das unmöglich, beſtimmen Sie
               mir ein Iſchl\oindex{Bad Ischl@\textbf{Bad Ischl}, \emph{P.PPL}|pw}er \textsc{rendezvous}. Sehen müſſen wir uns.\pend
           \pstart \spacefill\mbox{Loris.}\pend{}\selectlanguage{ngerman}\endnumbering\briefempfaengerindex{Schnitzler, Arthur@\textsc{Schnitzler, Arthur}!zzzHofmannsthal, Hugo von@\emph{von Hugo von Hofmannsthal}!1891-08-122@{12. 8. 1891}|)be}\mylabel{L00032h}  \normalsize

\doendnotes{C}
\bigskip
\vfill

\clearpage

\footnotesize

\lohead{\textsc{register}}

% Definiere theindex-Environment komplett neu ohne reledmac
\makeatletter
\renewenvironment{theindex}{%
  \section*{\indexname}%
  \setlength{\parindent}{0pt}%
  \setlength{\parskip}{0pt plus 0.3pt}%
  \let\item\@idxitem
}{%
  \clearpage
}
\makeatother

\IfFileExists{\jobname-pw.ind}{\input{\jobname-pw.ind}}{}

\end{document}

      