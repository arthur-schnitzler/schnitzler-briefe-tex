%% latex-leseansicht-vorspann.tex
%% Vorspann für die Leseansicht.
%% Lädt die gemeinsame Datei latex-vorspann.tex mit nicht gesetztem Schalter.

\newif\ifkorrekturansicht
\korrekturansichtfalse

\input{../tex-inputs/latex-vorspann}


\section[Hugo von Hofmannsthal an Arthur Schnitzler, 12. 8. 1891]{L00032 Hugo von Hofmannsthal an Arthur Schnitzler, 12. 8. 1891}
\nopagebreak\mylabel{L00032v}
\rehead{ }\normalsize\beginnumbering\briefempfaengerindex{Schnitzler, Arthur@\textsc{Schnitzler, Arthur}!zzzHofmannsthal, Hugo von@\emph{von Hugo von Hofmannsthal}!1891-08-122@{12. 8. 1891}|(be}
\toendnotes[C]{\smallbreak\pagebreak[2]}
\correspDesc{Versand  durch Hugo von Hofmannsthal am 12. 8. 1891 in Strobl
\newline{}Erhalt  durch Arthur Schnitzler am 13. 8. 1891 in Wien}\toendnotes[C]{\smallbreak}
\Standort{CUL, Schnitzler, B 43.}
\physDesc{Kartenbrief, 350 Zeichen
\newline{}Handschrift: schwarze Tinte, deutsche Kurrent
\newline{}Versand: 1) Stempel: »\nobreak{}\oindex{Strobl@\textbf{Strobl}, \emph{Verwaltungsgebiet}|pwk}Strobl, 12. 8. 91\nobreak{}«.   2) Stempel: »\nobreak{}\oindex{VI., Mariahilf@\textbf{VI., Mariahilf}, \emph{Verwaltungsgebiet}|pwk}Wien VI 1, 13. 8. 91, 8–9½ V.\nobreak{}«. 
\newline{}Schnitzler: mit Bleistift auf der Textseite zusätzlich datiert: »12. 8 91« 
\newline{}Ordnung: mit Bleistift von unbekannter Hand nummeriert:
                                 »5« }
\buchAbdrucke{\weitereDrucke{Hugo von Hofmannsthal, Arthur Schnitzler: \emph{Briefwechsel}. Herausgegeben von Therese Nickl und Heinrich Schnitzler. Frankfurt am Main: \emph{S. Fischer} 1964, S. 12.} }\toendnotes[C]{\smallbreak}\pstart{}{\pb}\textsc{D\textsuperscript{r}
                  Arthur Schnitzler}\pend{}\pstart{}\textsc{Wien\oindex{Wien@\textbf{Wien}, \emph{Verwaltungsgebiet}|pw}}\pend{}\pstart{}\textsc{I Kärthnerring 12\oindex{Wien@\textbf{Wien}!I., Innere Stadt@\textbf{I., Innere Stadt}!Kärntnerring 12/Bösendorferstraße 11@\textbf{Kärntnerring 12/Bösendorferstraße 11}, \emph{Wohngebäude}|pw}}\pend{}{\bigskip}\vspace{1em}
\pstart\center{}{\pb}Lieber Freund!\pend\vspace{0.5em}
\pstart
           Infolge Feſtvorbereitungen für \label{K_L00032-1v}\edtext{Kaiſer\pwindex{Franz Joseph I. von Österreich-Ungarn 18.\,8.\,1830 Wien – 21.\,11.\,1916 ebd.@\textsc{Franz Joseph I. von Österreich-Ungarn} (18.\,8.\,1830 Wien – 21.\,11.\,1916 ebd.), \emph{Kaiser}|pwv}beſuch}{\lemma{\textnormal{\emph{Kaiserbesuch}}}\Cendnote{\textnormal{Am 11. 8. 1891 besuchte Kaiser
                     Franz Joseph I.\pwindex{Franz Joseph I. von Österreich-Ungarn 18.\,8.\,1830 Wien – 21.\,11.\,1916 ebd.@\textsc{Franz Joseph I. von Österreich-Ungarn} (18.\,8.\,1830 Wien – 21.\,11.\,1916 ebd.), \emph{Kaiser}|pwk}{ }Ischl\oindex{Bad Ischl@\textbf{Bad Ischl}|pwk}, um sich dort mit König Alexander von Serbien\pwindex{Serbien, König, Alexander I. 2.\,8.\,1876 Belgrad – 29.\,5.\,1903 ebd.@\textsc{Serbien, König, Alexander I.} (2.\,8.\,1876 Belgrad – 29.\,5.\,1903 ebd.), \emph{Regent}|pwk} zu treffen.}}}\label{K_L00032-1} ganz Comité,
               kurz blöd, mache ich Ihnen folgende Vorſchläge: Da Strobl\oindex{Strobl@\textbf{Strobl}, \emph{Verwaltungsgebiet}|pw} Paradies, Iſchl\oindex{Bad Ischl@\textbf{Bad Ischl}|pw}{ }Schweineſtall{ }ſo erwarte ich sie und Hoffmann\pwindex{Beer-Hofmann, Richard 11.\,7.\,1866 Wien – 26.\,9.\,1945 New York City@\textsc{Beer-Hofmann, Richard} (11.\,7.\,1866 Wien – 26.\,9.\,1945 New York City), \emph{Schriftsteller}|pw} an einem der beiden Tage \uline{beſtimmteſtens}.\pend
           
\pstart
           Wenn das unmöglich, beſtimmen Sie
               mir ein Iſchl\oindex{Bad Ischl@\textbf{Bad Ischl}|pw}er \textsc{rendezvous}. Sehen müſſen wir uns.\pend
           \pstart \spacefill\mbox{Loris.}\pend{}\selectlanguage{ngerman}\endnumbering\briefempfaengerindex{Schnitzler, Arthur@\textsc{Schnitzler, Arthur}!zzzHofmannsthal, Hugo von@\emph{von Hugo von Hofmannsthal}!1891-08-122@{12. 8. 1891}|)be}\mylabel{L00032h}  \newcommand{\dateiname}{L00032}\newcommand{\titel}{Hugo von Hofmannsthal an Arthur Schnitzler, 12. 8. 1891}\newcommand{\editorInnen}{Martin Anton Müller und Gerd-Hermann Susen}%% latex-leseansicht-abspann.tex
%% Abspann für die Leseansicht.
%% Der Schalter \ifkorrekturansicht ist bereits durch den Vorspann gesetzt.

%% latex-abspann.tex
%% Gemeinsamer Abspann für Korrekturansicht und Leseansicht.
%% Setzt den Schalter \ifkorrekturansicht voraus (gesetzt in den
%% einbindenden Dateien latex-korrekturansicht-abspann.tex bzw.
%% latex-leseansicht-abspann.tex).
%% ---------------------------------------------------------------

\normalsize

% Das esempio-Environment wird nur in der Leseansicht benötigt
\ifkorrekturansicht\else
\newenvironment{esempio}[3]%
{
    \vspace{1.5ex}
    \rlap{\underline{#1}}
    \par
    \setlength{\parindent}{0cm}
    \nopagebreak
    \leftskip=#2cm
    \rightskip=#3cm
}
{
    \par
}
\fi

\doendnotes{C}
\bigskip
\vfill

\clearpage

\footnotesize

\ifkorrekturansicht
  \lohead{\textsc{register}}
\fi

% theindex-Environment neu definieren ohne reledmac
\makeatletter
\renewenvironment{theindex}{%
  \ifkorrekturansicht
    \section*{\indexname}%
  \else
    \subsubsection*{Index der erwähnten Entitäten}%
  \fi
  \setlength{\parindent}{0pt}%
  \setlength{\parskip}{0pt plus 0.3pt}%
  \let\item\@idxitem
}{%
  \ifkorrekturansicht\clearpage\fi
}
\makeatother

\IfFileExists{\jobname-pw.ind}{\input{\jobname-pw.ind}}{}

% Quellenangabe nur in der Leseansicht
\ifkorrekturansicht\else
% Fallback-Definitionen, falls die .tex-Datei \titel etc. nicht gesetzt hat
\providecommand{\titel}{}
\providecommand{\editorInnen}{}
\providecommand{\dateiname}{\jobname}

\vspace{3cm}

\vfill

\footnotesize
\textsc{Quelle}: \titel. Herausgegeben von {\editorInnen}. In: \emph{Arthur Schnitzler: Briefwechsel mit Autorinnen und Autoren}.
 Digitale Edition, https://schnitzler-briefe.acdh.oeaw.ac.at/{\dateiname}.html (Stand \today)
\fi

\end{document}


