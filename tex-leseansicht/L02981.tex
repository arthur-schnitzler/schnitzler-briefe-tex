%% latex-korrekturansicht-vorspann.tex
%% Vorspann für die Korrekturansicht.
%% Lädt die gemeinsame Datei latex-vorspann.tex mit gesetztem Schalter.

\newif\ifkorrekturansicht
\korrekturansichttrue

\input{../tex-inputs/latex-vorspann}


\section[ Arthur Schnitzler an Felix Salten, 4. 3. 1903]{L02981 Arthur Schnitzler an Felix Salten, 4. 3. 1903}
\nopagebreak\mylabel{L02981v}
\rehead{ }\normalsize\beginnumbering\briefempfaengerindex{Salten, Felix@\textsc{Salten, Felix}!zzzSchnitzler, Arthur@\emph{von Arthur Schnitzler}!1903-03-042@{4. 3. 1903}|(be}
\toendnotes[C]{\smallbreak\pagebreak[2]}\Standort{Wienbibliothek im Rathaus, ZPH 1681, 2.1.516.}
\physDesc{Brief, 1 Blatt, 3 Seiten, 834 Zeichen
\newline{}Handschrift: Bleistift, deutsche Kurrent
\newline{}Ordnung: mit Bleistift von unbekannter Hand Nummerierung der Doppelseiten des Konvoluts:
                                    »57«–»58« }\toendnotes[C]{\smallbreak}
\pstart
           \raggedleft{}{\pb}4. 3. 903\pend
           
\pstart
           \raggedleft{}Abds{ }\textsc{Berlin\oindex{Berlin@\textbf{Berlin}, \emph{P.PPLC}|pw}}\pend
           \vspace{0.5em}
\pstart
           lieber Freund, meinem \label{K_L02981-1v}\edtext{Brief von heute{ }Nachmittg}{\lemma{\textnormal{\emph{Brief … Nachmittg}}}\Cendnote{\textnormal{Arthur Schnitzler an Felix Salten, 4. 3. 1903.
               }}}\label{K_L02981-1} iſt nachzutragen: als ich das Hotel\oindex{Palasthotel Berlin@\textbf{Palasthotel Berlin}, \emph{Hotel (K.HTL)}|pwv} verlieſs, erwartete mich M. H.\pwindex{Horwitz, Mirjam 1882-06-15 – 1967-09-26@\textsc{Horwitz, Mirjam} (1882-06-15 – 1967-09-26), \emph{Theaterleiter/Theaterleiterin, Schauspieler/Schauspielerin}|pw},
               ſie zeigte mir den Brief, den Sie an den \label{K_L02981-2v}\edtext{Vertrauten\pwindex{Landesmann, Adolf 24.10.1867 – 1925-01-05@\textsc{Landesmann, Adolf} (24.10.1867 – 1925-01-05), \emph{Bankangestellter/Bankangestellte}|pwv} geſchrieben; ich
               hatte ihn (kleine Welt!) geſtern{ }Abend bei Brahm\pwindex{Brahm, Otto 05.02.1856 – 28.11.1912@\textsc{Brahm, Otto} (05.02.1856 – 28.11.1912), \emph{Theaterleiter/Theaterleiterin, Regisseur/Regisseurin}|pw} kennen
                  gelernt}{\lemma{\textnormal{\emph{Vertrauten … gelernt}}}\Cendnote{\textnormal{Die Identifizierung gelingt
                  durch Ausschluss: Von der Abendgesellschaft am 3. 3. 1903 war einzig Adolf Landesmann\pwindex{Landesmann, Adolf 24.10.1867 – 1925-01-05@\textsc{Landesmann, Adolf} (24.10.1867 – 1925-01-05), \emph{Bankangestellter/Bankangestellte}|pwk}{ }Schnitzler zuvor nicht bekannt
                  gewesen.}}}\label{K_L02981-2}{\dotstwo} ich entledigte mich meines Auftrags ganz geſchickt; ſie
                  {\pb}möchte ihre Briefe zurück haben – ich
               rieth ihr, dem keinerlei Werth beizulegen; theile Ihnen aber, \substVorne{}\textsuperscript{ihrer}\substDazwischen{}M\pwindex{Horwitz, Mirjam 1882-06-15 – 1967-09-26@\textsc{Horwitz, Mirjam} (1882-06-15 – 1967-09-26), \emph{Theaterleiter/Theaterleiterin, Schauspieler/Schauspielerin}|pw}.s
               \substHinten{} Bitte entſprechend, d\substVorne{}\textsuperscript{en}\substDazwischen{}ie\substHinten{}ſen Wunſch mit. Thränen, etwas Bläſſe; mehr Zorn als Kränkung wie mir
               ſcheint. Im ganzen kein Anlaſs ſich aufzuregen.\pend
           
\pstart
           – Ich habe hier auch die \label{K_L02981-3v}\edtext{Geſpräche des göttlichen {\pb}\textsc{Aretin}\pwindex{Gespraeche des goettlichen Pietro Aretino@\emph{Die Gespräche des göttlichen Pietro Aretino}|pw}}{\lemma{\textnormal{\emph{Geſpräche … Aretin}}}\Cendnote{\textnormal{Siehe Felix Salten an Arthur Schnitzler, 3. 3. 1903.
               }}}\label{K_L02981-3} geleſen; nicht ganz ohne Enttäuſchg. Ich hoffe Ihre
               rö\textcolor{gray}{mi}ſche Buhlerin wird intereſſantere Dinge zu erzählen\pwindex{Vom goettlichen Aretino@\emph{Vom göttlichen Aretino}|pwv} wiſſen. Amuſirt hat mich am
               meiſten die kleine Pippa\pwindex{Gespraeche des goettlichen Pietro Aretino@\emph{Die Gespräche des göttlichen Pietro Aretino}|pwv} mit
               ihrem dummen Hineinreden.\pend
           
\pstart
           Leben Sie wohl. Herzlichſt Ihr {\\[\baselineskip]}\spacefill\mbox{A.}\pend
           \leftskip=0em{}\selectlanguage{ngerman}\endnumbering\briefempfaengerindex{Salten, Felix@\textsc{Salten, Felix}!zzzSchnitzler, Arthur@\emph{von Arthur Schnitzler}!1903-03-042@{4. 3. 1903}|)be}\mylabel{L02981h}  \normalsize

\doendnotes{C}
\bigskip
\vfill

\clearpage

\footnotesize

\lohead{\textsc{register}}

% Definiere theindex-Environment komplett neu ohne reledmac
\makeatletter
\renewenvironment{theindex}{%
  \section*{\indexname}%
  \setlength{\parindent}{0pt}%
  \setlength{\parskip}{0pt plus 0.3pt}%
  \let\item\@idxitem
}{%
  \clearpage
}
\makeatother

\IfFileExists{\jobname-pw.ind}{\input{\jobname-pw.ind}}{}

\end{document}

      