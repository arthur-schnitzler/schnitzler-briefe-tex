%% latex-leseansicht-vorspann.tex
%% Vorspann für die Leseansicht.
%% Lädt die gemeinsame Datei latex-vorspann.tex mit nicht gesetztem Schalter.

\newif\ifkorrekturansicht
\korrekturansichtfalse

\input{../tex-inputs/latex-vorspann}

\begin{center}
            \textcolor{red}{ENTWURF, NICHT FERTIG KORRIGIERT}
                      \end{center}
            
         
         \renewcommand{\erwaehntePersonen}{Personen: Otto Brahm, Mirjam Horwitz, Adolf Landesmann, Felix Salten}
         \renewcommand{\erwaehnteOrte}{Orte: Berlin, Wien}
         \renewcommand{\erwaehnteWerke}{Werke: Die Gespräche des göttlichen Pietro Aretino}
               \section[Arthur Schnitzler an Felix Salten, 4. 3. 1903]{ Arthur Schnitzler an Felix Salten, 4. 3. 1903}\nopagebreak\mylabel{v}\rehead{ }\begin{ledgroupsized}[t]{13cm}\normalsize\beginnumbering \toendnotes[C]{\smallbreak\pagebreak[2]} \Standort{Wienbibliothek im Rathaus, ZPH 1681, 2.1.516.}
\physDesc{Brief, 1 Blatt, 3 Seiten, 845 Zeichen
\newline{}Handschrift: Bleistift, deutsche Kurrent
\newline{}Ordnung: mit Bleistift von unbekannter Hand Nummerierung der Blätter des
                                 Konvoluts: »57«–»58« }\toendnotes[C]{\smallbreak}\pstart
           \raggedleft{}{\pb}4. 3. 03\pend
           \pstart
           \raggedleft{}Abds{ }\textsc{Berlin\oindex{Berlin@\textbf{Berlin}|pw}}\pend
           \pstart
           lieber Freund, meinem Brief von heute Nachmittg iſt nachzutragen:
               als ich das Hotel verlieſs, erwartete mich M. H.\pwindex{Horwitz, Mirjam 1882-06-15 – 1967-09-26@\textsc{Horwitz, Mirjam} (1882-06-15 – 1967-09-26), \emph{Theaterleiterin, Schauspielerin}|pw}, ſie zeigte mir den Brief, den Sie an den \label{K_L02981-11v}\edtext{Vertrauten\pwindex{Landesmann, Adolf 24.10.1867 – 1925-01-05@\textsc{Landesmann, Adolf} (24.10.1867 – 1925-01-05), \emph{Angestellter}|pwv} geſchrieben; ich
               hatte ihn (kleine Welt!) geſtern Abend bei Brahm\pwindex{Brahm, Otto 05.02.1856 – 28.11.1912@\textsc{Brahm, Otto} (05.02.1856 – 28.11.1912), \emph{Theaterleiter, Regisseur}|pw} kennen gelernt}{\lemma{\textnormal{\emph{Vertrauten … gelernt}}}\Cendnote{\textnormal{Die
                  Identifizierung gelingt mit einem Ausschlusskriterium: Von der Abendgesellschaft
                  am 3. 3. 1903 war
                  einzig Adolf Landesmann\pwindex{Landesmann, Adolf 24.10.1867 – 1925-01-05@\textsc{Landesmann, Adolf} (24.10.1867 – 1925-01-05), \emph{Angestellter}|pwk}{ }Schnitzler\pwindex{Schnitzler, Arthur 15.05.1862 – 21.10.1931@\textsc{Schnitzler, Arthur} (15.05.1862 – 21.10.1931), \emph{Schriftsteller, Mediziner}|pwk} zuvor nicht bekannt.}}}\label{K_L02981-11h}{ }{\dotstwo} ich entledigte mich meines Auftrags ganz geſchickt; ſie
                  {\pb}möchte ihre Briefe zurück haben – ich
               rieth ihr, dem keinerlei Werth beizulegen; theile Ihnen aber, ihrer \introOben{}{[}({]}M\pwindex{Horwitz, Mirjam 1882-06-15 – 1967-09-26@\textsc{Horwitz, Mirjam} (1882-06-15 – 1967-09-26), \emph{Theaterleiterin, Schauspielerin}|pw}.s{[}){]}\introOben{} Bitte entſprechend, dieſen Wunſch mit. Thränen, etwas Kliſche; mehr Zorn als
               Kränkung wie mir ſcheint. Im ganzen kein Anlaſs ſich aufzuregen. \pend
           \pstart
           – Ich habe hier auch die Geſpräche des göttlichen {\pb}\textsc{Aretin}\pwindex{\textcolor{red}{\textsuperscript{XXXX1 indx}}!Gespraeche des goettlichen Pietro Aretino1903@\strich\emph{Die Gespräche des göttlichen Pietro Aretino} {[}1903{]}|pw} geleſen; nicht ganz ohne Enttäuſchg. Ich hoffe Ihre rö\textcolor{gray}{mi}ſche
               Buhlerin wird intereſſantere Dinge zu erzählen wiſſen. Amuſirt hat mich am meiſten
               die kleine Skizze\textcolor{red}{\textsuperscript{\textbf{KEY}}} mit ihren dummen
                  Hineinreden. \pend
           \pstart
           Leben Sie wohl. Herzlichſt Ihr {\\[\baselineskip]}\spacefill\mbox{A.}\pend
           \leftskip=0em{}
         
         \endnumbering\mylabel{h}\end{ledgroupsized}\begin{anhang}\end{anhang}\newcommand{\dateiname}{L02981}\newcommand{\titel}{Arthur Schnitzler an Felix Salten, 4. 3. 1903}\newcommand{\editorInnen}{Martin Anton Müller und Laura Untner}%% latex-leseansicht-abspann.tex
%% Abspann für die Leseansicht.
%% Der Schalter \ifkorrekturansicht ist bereits durch den Vorspann gesetzt.

%% latex-abspann.tex
%% Gemeinsamer Abspann für Korrekturansicht und Leseansicht.
%% Setzt den Schalter \ifkorrekturansicht voraus (gesetzt in den
%% einbindenden Dateien latex-korrekturansicht-abspann.tex bzw.
%% latex-leseansicht-abspann.tex).
%% ---------------------------------------------------------------

\normalsize

% Das esempio-Environment wird nur in der Leseansicht benötigt
\ifkorrekturansicht\else
\newenvironment{esempio}[3]%
{
    \vspace{1.5ex}
    \rlap{\underline{#1}}
    \par
    \setlength{\parindent}{0cm}
    \nopagebreak
    \leftskip=#2cm
    \rightskip=#3cm
}
{
    \par
}
\fi

\doendnotes{C}
\bigskip
\vfill

\clearpage

\footnotesize

\ifkorrekturansicht
  \lohead{\textsc{register}}
\fi

% theindex-Environment neu definieren ohne reledmac
\makeatletter
\renewenvironment{theindex}{%
  \ifkorrekturansicht
    \section*{\indexname}%
  \else
    \subsubsection*{Index der erwähnten Entitäten}%
  \fi
  \setlength{\parindent}{0pt}%
  \setlength{\parskip}{0pt plus 0.3pt}%
  \let\item\@idxitem
}{%
  \ifkorrekturansicht\clearpage\fi
}
\makeatother

\IfFileExists{\jobname-pw.ind}{\input{\jobname-pw.ind}}{}

% Quellenangabe nur in der Leseansicht
\ifkorrekturansicht\else
% Fallback-Definitionen, falls die .tex-Datei \titel etc. nicht gesetzt hat
\providecommand{\titel}{}
\providecommand{\editorInnen}{}
\providecommand{\dateiname}{\jobname}

\vspace{3cm}

\vfill

\footnotesize
\textsc{Quelle}: \titel. Herausgegeben von {\editorInnen}. In: \emph{Arthur Schnitzler: Briefwechsel mit Autorinnen und Autoren}.
 Digitale Edition, https://schnitzler-briefe.acdh.oeaw.ac.at/{\dateiname}.html (Stand \today)
\fi

\end{document}


      