%% latex-leseansicht-vorspann.tex
%% Vorspann für die Leseansicht.
%% Lädt die gemeinsame Datei latex-vorspann.tex mit nicht gesetztem Schalter.

\newif\ifkorrekturansicht
\korrekturansichtfalse

\input{../tex-inputs/latex-vorspann}


\section[ Arthur Schnitzler an Felix Salten, 4. 3. 1903]{L02981 Arthur Schnitzler an Felix Salten,  4. 3. 1903}
\nopagebreak\mylabel{L02981v}
\rehead{ }\normalsize\beginnumbering\briefempfaengerindex{Salten, Felix@\textsc{Salten, Felix}!zzzSchnitzler, Arthur@\emph{von Arthur Schnitzler}!1903-03-042@{4. 3. 1903}|(be}
\toendnotes[C]{\smallbreak\pagebreak[2]}
\correspDesc{Versand  durch Arthur Schnitzler am 4. 3. 1903 in Berlin
\newline{}Erhalt  durch Felix Salten im Zeitraum [5. 3. 1903
                  – 9. 3. 1903?] in Wien}\toendnotes[C]{\smallbreak}
\Standort{Wienbibliothek im Rathaus, ZPH 1681, 2.1.516.}
\physDesc{Brief, 1 Blatt, 3 Seiten, 834 Zeichen
\newline{}Handschrift: Bleistift, deutsche Kurrent
\newline{}Ordnung: mit Bleistift von unbekannter Hand Nummerierung der Doppelseiten des Konvoluts:
                                    »57«–»58« }\toendnotes[C]{\smallbreak}
\pstart
           \raggedleft{}{\pb}4. 3. 903\pend
           
\pstart
           \raggedleft{}Abds{ }\textsc{Berlin\oindex{Berlin@\textbf{Berlin}, \emph{Hauptstadt}|pw}}\pend
           \vspace{0.5em}
\pstart
           lieber Freund, meinem \label{K_L02981-1v}\edtext{Brief von heute{ }Nachmittg}{\lemma{\textnormal{\emph{Brief … Nachmittg}}}\Cendnote{\textnormal{XXXX Auszeichnungsfehler: Dokument L02980 nicht gefunden.
               }}}\label{K_L02981-1} iſt nachzutragen: als ich das Hotel\oindex{Palasthotel Berlin@\textbf{Palasthotel Berlin}, \emph{Hotel}|pwv} verlieſs, erwartete mich M. H.\pwindex{Horwitz, Mirjam 15.\,6.\,1882 Berlin – 26.\,9.\,1967 Lütjensee@\textsc{Horwitz, Mirjam} (15.\,6.\,1882 Berlin – 26.\,9.\,1967 Lütjensee), \emph{Theaterleiterin, Schauspielerin}|pw},{ }ſie zeigte mir den Brief, den Sie an den \label{K_L02981-2v}\edtext{Vertrauten\pwindex{Landesmann, Adolf 24.\,10.\,1867 Baden bei Wien – 5.\,1.\,1925 Berlin@\textsc{Landesmann, Adolf} (24.\,10.\,1867 Baden bei Wien – 5.\,1.\,1925 Berlin), \emph{Bankangestellter}|pwv} geſchrieben; ich
               hatte ihn (kleine Welt!) geſtern{ }Abend bei Brahm\pwindex{Brahm, Otto 5.\,2.\,1856 Hamburg – 28.\,11.\,1912 Berlin@\textsc{Brahm, Otto} (5.\,2.\,1856 Hamburg – 28.\,11.\,1912 Berlin), \emph{Theaterleiter, Regisseur}|pw} kennen
                  gelernt}{\lemma{\textnormal{\emph{Vertrauten … gelernt}}}\Cendnote{\textnormal{Die Identifizierung gelingt
                  durch Ausschluss: Von der Abendgesellschaft am 3. 3. 1903 war einzig Adolf Landesmann\pwindex{Landesmann, Adolf 24.\,10.\,1867 Baden bei Wien – 5.\,1.\,1925 Berlin@\textsc{Landesmann, Adolf} (24.\,10.\,1867 Baden bei Wien – 5.\,1.\,1925 Berlin), \emph{Bankangestellter}|pwk}{ }Schnitzler zuvor nicht bekannt
                  gewesen.}}}\label{K_L02981-2}{\dotstwo} ich entledigte mich meines Auftrags ganz geſchickt;{ }ſie
                  {\pb}möchte ihre Briefe zurück haben – ich
               rieth ihr, dem keinerlei Werth beizulegen; theile Ihnen aber, \substVorne{}\textsuperscript{ihrer}\substDazwischen{}M\pwindex{Horwitz, Mirjam 15.\,6.\,1882 Berlin – 26.\,9.\,1967 Lütjensee@\textsc{Horwitz, Mirjam} (15.\,6.\,1882 Berlin – 26.\,9.\,1967 Lütjensee), \emph{Theaterleiterin, Schauspielerin}|pw}.s\substHinten{} Bitte entſprechend, d\substVorne{}\textsuperscript{en}\substDazwischen{}ie\substHinten{}ſen Wunſch mit. Thränen, etwas Bläſſe; mehr Zorn als Kränkung wie mir{ }ſcheint. Im ganzen kein Anlaſs{ }ſich aufzuregen.\pend
           
\pstart
           – Ich habe hier auch die \label{K_L02981-3v}\edtext{Geſpräche des göttlichen {\pb}\textsc{Aretin}\pwindex{\textcolor{red}{\textsuperscript{XXXX indx1}}!Gespräche des göttlichen Pietro Aretino@\strich\emph{Die Gespräche des göttlichen Pietro Aretino}|pw}}{\lemma{\textnormal{\emph{Gespräche … Aretin}}}\Cendnote{\textnormal{Siehe XXXX Auszeichnungsfehler: Dokument L03339 nicht gefunden.
               }}}\label{K_L02981-3} geleſen; nicht ganz ohne Enttäuſchg. Ich hoffe Ihre
               rö\textcolor{gray}{mi}ſche Buhlerin wird intereſſantere Dinge zu erzählen\pwindex{Salten, Felix 6.\,9.\,1869 Budapest – 8.\,10.\,1945 Zürich@\textsc{Salten, Felix} (6.\,9.\,1869 Budapest – 8.\,10.\,1945 Zürich), \emph{Schriftsteller, Journalist, Chefredakteur}!Vom göttlichen Aretino@\strich\emph{Vom göttlichen Aretino}|pwv} wiſſen. Amuſirt hat mich am
               meiſten die kleine Pippa\pwindex{\textcolor{red}{\textsuperscript{XXXX indx1}}!Gespräche des göttlichen Pietro Aretino@\strich\emph{Die Gespräche des göttlichen Pietro Aretino}|pwv} mit
               ihrem dummen Hineinreden.\pend
           
\pstart
           Leben Sie wohl. Herzlichſt Ihr {\\[\baselineskip]}\spacefill\mbox{A.}\pend
           \leftskip=0em{}\selectlanguage{ngerman}\endnumbering\briefempfaengerindex{Salten, Felix@\textsc{Salten, Felix}!zzzSchnitzler, Arthur@\emph{von Arthur Schnitzler}!1903-03-042@{4. 3. 1903}|)be}\mylabel{L02981h}  \newcommand{\dateiname}{L02981}\newcommand{\titel}{Arthur Schnitzler an Felix Salten, 4. 3. 1903}\newcommand{\editorInnen}{Martin Anton Müller und Laura Untner}%% latex-leseansicht-abspann.tex
%% Abspann für die Leseansicht.
%% Der Schalter \ifkorrekturansicht ist bereits durch den Vorspann gesetzt.

%% latex-abspann.tex
%% Gemeinsamer Abspann für Korrekturansicht und Leseansicht.
%% Setzt den Schalter \ifkorrekturansicht voraus (gesetzt in den
%% einbindenden Dateien latex-korrekturansicht-abspann.tex bzw.
%% latex-leseansicht-abspann.tex).
%% ---------------------------------------------------------------

\normalsize

% Das esempio-Environment wird nur in der Leseansicht benötigt
\ifkorrekturansicht\else
\newenvironment{esempio}[3]%
{
    \vspace{1.5ex}
    \rlap{\underline{#1}}
    \par
    \setlength{\parindent}{0cm}
    \nopagebreak
    \leftskip=#2cm
    \rightskip=#3cm
}
{
    \par
}
\fi

\doendnotes{C}
\bigskip
\vfill

\clearpage

\footnotesize

\ifkorrekturansicht
  \lohead{\textsc{register}}
\fi

% theindex-Environment neu definieren ohne reledmac
\makeatletter
\renewenvironment{theindex}{%
  \ifkorrekturansicht
    \section*{\indexname}%
  \else
    \subsubsection*{Index der erwähnten Entitäten}%
  \fi
  \setlength{\parindent}{0pt}%
  \setlength{\parskip}{0pt plus 0.3pt}%
  \let\item\@idxitem
}{%
  \ifkorrekturansicht\clearpage\fi
}
\makeatother

\IfFileExists{\jobname-pw.ind}{\input{\jobname-pw.ind}}{}

% Quellenangabe nur in der Leseansicht
\ifkorrekturansicht\else
% Fallback-Definitionen, falls die .tex-Datei \titel etc. nicht gesetzt hat
\providecommand{\titel}{}
\providecommand{\editorInnen}{}
\providecommand{\dateiname}{\jobname}

\vspace{3cm}

\vfill

\footnotesize
\textsc{Quelle}: \titel. Herausgegeben von {\editorInnen}. In: \emph{Arthur Schnitzler: Briefwechsel mit Autorinnen und Autoren}.
 Digitale Edition, https://schnitzler-briefe.acdh.oeaw.ac.at/{\dateiname}.html (Stand \today)
\fi

\end{document}


