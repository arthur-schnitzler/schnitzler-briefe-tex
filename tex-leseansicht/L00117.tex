%% latex-korrekturansicht-vorspann.tex
%% Vorspann für die Korrekturansicht.
%% Lädt die gemeinsame Datei latex-vorspann.tex mit gesetztem Schalter.

\newif\ifkorrekturansicht
\korrekturansichttrue

\input{../tex-inputs/latex-vorspann}


\section[Hugo von Hofmannsthal an Arthur Schnitzler, 23. 8. {[}1892{]}]{L00117 Hugo von Hofmannsthal an Arthur Schnitzler, 23. 8. {[}1892{]}}
\nopagebreak\mylabel{L00117v}
\rehead{ }\normalsize\beginnumbering\briefempfaengerindex{Schnitzler, Arthur@\textsc{Schnitzler, Arthur}!zzzHofmannsthal, Hugo von@\emph{von Hugo von Hofmannsthal}!1892-08-231@{23. 8. {[}1892{]}}|(be}
\toendnotes[C]{\smallbreak\pagebreak[2]}\Standort{CUL, Schnitzler, B 43.}
\physDesc{Brief, 1 Blatt, 1 Seite, 196 Zeichen
\newline{}Handschrift: Bleistift, deutsche Kurrent
\newline{}Schnitzler: mit Bleistift die Jahreszahl ergänzt: »92« 
\newline{}Ordnung: 1) mit Bleistift von unbekannter Hand nummeriert: »\strikeout{1}«  2) mit Bleistift von unbekannter Hand nummeriert:
                                 »0«}
\buchAbdrucke{\weitereDrucke{Hugo von Hofmannsthal, Arthur Schnitzler: \emph{Briefwechsel}. Frankfurt am Main: \emph{S. Fischer} 1964, S. 28.} }\toendnotes[C]{\smallbreak}
\pstart
           \raggedleft{}{\pb}23. 8\pend
           
\pstart\center{}Lieber Arthur.\pend\vspace{0.5em}
\pstart
           Ich habe bei der entſetzlichen Hitze an einer heftigen Beinhautentzündung etc.
               gelitten. An meinem \label{K_L00117-1v}\edtext{Reiſeproject}{\lemma{\textnormal{\emph{Reiſeproject}}}\Cendnote{\textnormal{die Maturareise nach Frankreich\oindex{Frankreich@\textbf{Frankreich}, \emph{A.PCLI}|pwk}}}}\label{K_L00117-1} iſt nichts geändert. Ich freue mich, Sie noch in Iſchl\oindex{Bad Ischl@\textbf{Bad Ischl}, \emph{P.PPL}|pw} zu ſehen.\pend
           \pstart \spacefill\mbox{Hugo.}\pend{}\selectlanguage{ngerman}\endnumbering\briefempfaengerindex{Schnitzler, Arthur@\textsc{Schnitzler, Arthur}!zzzHofmannsthal, Hugo von@\emph{von Hugo von Hofmannsthal}!1892-08-231@{23. 8. {[}1892{]}}|)be}\mylabel{L00117h}  \normalsize

\doendnotes{C}
\bigskip
\vfill

\clearpage

\footnotesize

\lohead{\textsc{register}}

% Definiere theindex-Environment komplett neu ohne reledmac
\makeatletter
\renewenvironment{theindex}{%
  \section*{\indexname}%
  \setlength{\parindent}{0pt}%
  \setlength{\parskip}{0pt plus 0.3pt}%
  \let\item\@idxitem
}{%
  \clearpage
}
\makeatother

\IfFileExists{\jobname-pw.ind}{\input{\jobname-pw.ind}}{}

\end{document}

      