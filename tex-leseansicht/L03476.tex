%% latex-korrekturansicht-vorspann.tex
%% Vorspann für die Korrekturansicht.
%% Lädt die gemeinsame Datei latex-vorspann.tex mit gesetztem Schalter.

\newif\ifkorrekturansicht
\korrekturansichttrue

\input{../tex-inputs/latex-vorspann}


\section[ Paul Goldmann an Arthur Schnitzler, 10. 9. 1911]{L03476 Paul Goldmann an Arthur Schnitzler, 10. 9. 1911}
\nopagebreak\mylabel{L03476v}
\rehead{ }\normalsize\beginnumbering\briefempfaengerindex{Schnitzler, Arthur@\textsc{Schnitzler, Arthur}!zzzGoldmann, Paul@\emph{von Paul Goldmann}!1911-09-102@{10. 9. 1911}|(be}
\toendnotes[C]{\smallbreak\pagebreak[2]}\Standort{DLA, A:Schnitzler, HS.NZ85.1.3176.}
\physDesc{Brief, 1 Blatt, 2 Seiten, 423 Zeichen
\newline{}Handschrift: schwarze Tinte, deutsche Kurrent
\newline{}Schnitzler: mit Bleistift »\textsc{Goldma{[}nn{]}\pwindex{Goldmann, Paul 31.01.1865 – 25.09.1935@\textsc{Goldmann, Paul} (31.01.1865 – 25.09.1935), \emph{Schriftsteller/Schriftstellerin, Journalist/Journalistin}|pw}}« vermerkt }\toendnotes[C]{\smallbreak}
\pstart
           \centering{}{\pb}\textcolor{gray}{\textbf{\textbf{\textsc{Hotel Kaiserin Elisabeth\oindex{Hotel Kaiserin Elisabeth [Bad Ischl]@\textbf{Hotel Kaiserin Elisabeth [Bad Ischl]}, \emph{Hotel (K.HTL)}|pw}}}}}\pend
           
\pstart
           \textcolor{gray}{\textbf{INTERURBAN. TELEPHON No. 12}}\hfill \textcolor{gray}{\textbf{Eigentümer: K.\pwindex{Seeauer, Karl @\textsc{Seeauer, Karl}, \emph{Hotelier/Hotelière}|pw}{ }{\kaufmannsund}{ }H. SEEAUER\pwindex{Seeauer, Helene @\textsc{Seeauer, Helene}, \emph{Hotelier/Hotelière}|pw}}}\pend
           
\pstart
           \textcolor{gray}{\textbf{Telegr.-Adr.:}}{ }{\\}\textcolor{gray}{\textbf{ELISABETHHOTEL\oindex{Hotel Kaiserin Elisabeth [Bad Ischl]@\textbf{Hotel Kaiserin Elisabeth [Bad Ischl]}, \emph{Hotel (K.HTL)}|pw}, BAD ISCHL\oindex{Bad Ischl@\textbf{Bad Ischl}, \emph{P.PPL}|pw}.}}\hfill \textcolor{gray}{\textbf{\textsc{Bad Ischl\oindex{Bad Ischl@\textbf{Bad Ischl}, \emph{P.PPL}|pw}}}}\pend
           
\pstart
           \textcolor{gray}{\textbf{LIFT\hspace*{5em}BÄDER.}}\hfill \textcolor{gray}{\textbf{= Salzkammergut\oindex{Salzkammergut@\textbf{Salzkammergut}, \emph{L.RGN}|pw}, Österreich\oindex{Oesterreich@\textbf{Österreich}, \emph{A.PCLI}|pw}. =}}\pend
           
\pstart
           10. 9. 11.\pend
           
\pstart{}Lieber Freund,\pend\vspace{0.5em}
\pstart
           Tief ergriffen von der Nachricht, die ich eben erfahre, habe ich das Bedürfnis, Dir
               teilnehmend die Hand zu drücken u. Dir zu ſagen, daß ich den \label{K_L03476-1v}\edtext{Tod Deiner Mutter\pwindex{Schnitzler, Louise 1840-07-08 – 1911-09-09@\textsc{Schnitzler, Louise} (1840-07-08 – 1911-09-09)|pwv}}{\lemma{\textnormal{\emph{Tod Deiner Mutter}}}\Cendnote{\textnormal{Louise Schnitzler\pwindex{Schnitzler, Louise 1840-07-08 – 1911-09-09@\textsc{Schnitzler, Louise} (1840-07-08 – 1911-09-09)|pwk} war am 9. 9. 1911
                  verstorben.}}}\label{K_L03476-1} mit Dir betraure. Das iſt das Schwerſte, das einen Menſchen
               treffen \strikeout{K} kann, u. ich wünſche Dir die Kraft, dieſen
               Schickſalsſchlag zu {\pb}ertragen. Ich werde Deiner
                  Mutter\pwindex{Schnitzler, Louise 1840-07-08 – 1911-09-09@\textsc{Schnitzler, Louise} (1840-07-08 – 1911-09-09)|pwv} die Güte u.
               Freundſchaft, die ſie mir erwieſen, nie vergeſſen. {\\}\spacefill\mbox{Paul Goldmann.}\pend
           \selectlanguage{ngerman}\endnumbering\briefempfaengerindex{Schnitzler, Arthur@\textsc{Schnitzler, Arthur}!zzzGoldmann, Paul@\emph{von Paul Goldmann}!1911-09-102@{10. 9. 1911}|)be}\mylabel{L03476h}  \normalsize

\doendnotes{C}
\bigskip
\vfill

\clearpage

\footnotesize

\lohead{\textsc{register}}

% Definiere theindex-Environment komplett neu ohne reledmac
\makeatletter
\renewenvironment{theindex}{%
  \section*{\indexname}%
  \setlength{\parindent}{0pt}%
  \setlength{\parskip}{0pt plus 0.3pt}%
  \let\item\@idxitem
}{%
  \clearpage
}
\makeatother

\IfFileExists{\jobname-pw.ind}{\input{\jobname-pw.ind}}{}

\end{document}

      