%% latex-leseansicht-vorspann.tex
%% Vorspann für die Leseansicht.
%% Lädt die gemeinsame Datei latex-vorspann.tex mit nicht gesetztem Schalter.

\newif\ifkorrekturansicht
\korrekturansichtfalse

\input{../tex-inputs/latex-vorspann}

\begin{center}
            \textcolor{red}{ENTWURF, NICHT FERTIG KORRIGIERT}
                      \end{center}
            
         
         \renewcommand{\erwaehntePersonen}{Personen: Paul Goldmann, Louise Schnitzler, Karl Seeauer, Helene Seeauer}
         \renewcommand{\erwaehnteOrte}{Orte: Bad Ischl, Hotel Kaiserin Elisabeth, Salzkammergut, Wien, Österreich}
         \renewcommand{\erwaehnteWerke}{}
               \section[ Paul Goldmann an Arthur Schnitzler, 10. 9. 1911]{ Paul Goldmann an Arthur Schnitzler, 10. 9. 1911}\nopagebreak\mylabel{v}\rehead{ }\begin{ledgroupsized}[t]{13cm}\normalsize\beginnumbering\briefempfaengerindex{Schnitzler, Arthur@\textsc{Schnitzler, Arthur}!zzzGoldmann, Paul@\emph{von Paul Goldmann}!1911-09-102@{10. 9. 1911}|(be} \toendnotes[C]{\smallbreak\pagebreak[2]} \Standort{DLA, A:Schnitzler, HS.NZ85.1.3176.}
\physDesc{Brief, 1 Blatt, 2 Seiten, 423 Zeichen
\newline{}Handschrift: schwarze Tinte, deutsche Kurrent
\newline{}Schnitzler: mit Bleistift Vermerk »\textsc{Goldma\textcolor{gray}{nn}\pwindex{Goldmann, Paul 31.01.1865 – 25.09.1935@\textsc{Goldmann, Paul} (31.01.1865 – 25.09.1935), \emph{Schriftsteller, Journalist}|pw}}« }\toendnotes[C]{\smallbreak}\pstart
           \noindent{}\centering{}{\pb}\textcolor{gray}{\textbf{\textbf{\textsc{Hotel Kaiserin Elisabeth\oindex{Hotel Kaiserin Elisabeth@\textbf{Hotel Kaiserin Elisabeth}|pw}}}}}\pend
           \pstart
           \noindent{}\textcolor{gray}{\textbf{INTERURBAN. TELEPHON No. 12}}\hfill \textcolor{gray}{\textbf{Eigentümer: K.\pwindex{Seeauer, Karl @\textsc{Seeauer, Karl}, \emph{Hotelier}|pw}{ }{\kaufmannsund}{ }H. SEEAUER\pwindex{Seeauer, Helene @\textsc{Seeauer, Helene}, \emph{Hotelière}|pw}}}\pend
           \pstart
           \textcolor{gray}{\textbf{Telegr.-Adr.:}}{ }{\\}\textcolor{gray}{\textbf{ELISABETHHOTEL\oindex{Hotel Kaiserin Elisabeth@\textbf{Hotel Kaiserin Elisabeth}|pw}, BAD ISCHL\oindex{Bad Ischl@\textbf{Bad Ischl}|pw}.}}\hfill \textcolor{gray}{\textbf{\textsc{Bad Ischl\oindex{Bad Ischl@\textbf{Bad Ischl}|pw}}}}\pend
           \pstart
           \textcolor{gray}{\textbf{LIFT\hspace*{5em}BÄDER.}}\hfill \textcolor{gray}{\textbf{= Salzkammergut\oindex{Salzkammergut@\textbf{Salzkammergut}|pw}, Österreich\oindex{Oesterreich@\textbf{Österreich}|pw}. =}}\pend
           \pstart
           10. 9. 11.\pend
           \pstart{}Lieber Freund,\pend\pstart
           Tief ergriffen von der Nachricht, die ich eben erfahre, habe ich das Bedürfnis, Dir
               teilnehmend die Hand zu drücken u. Dir zu ſagen, daß ich den \label{K_L03476-1v}\edtext{Tod Deiner Mutter\pwindex{Schnitzler, Louise 1840-07-08 – 1911-09-09@\textsc{Schnitzler, Louise} (1840-07-08 – 1911-09-09)|pwv}}{\lemma{\textnormal{\emph{Tod Deiner Mutter}}}\Cendnote{\textnormal{Louise Schnitzler\pwindex{Schnitzler, Louise 1840-07-08 – 1911-09-09@\textsc{Schnitzler, Louise} (1840-07-08 – 1911-09-09)|pwk} war am 9. 9. 1911
                  verstorben.}}}\label{K_L03476-1h} mit Dir betraure. Das iſt das Schwerſte, das einen Menſchen
               treffen \strikeout{K} kann, u. ich wünſche Dir die Kraft, dieſen
               Schickſalsſchlag zu {\pb}ertragen. Ich werde Deiner
                  Mutter\pwindex{Schnitzler, Louise 1840-07-08 – 1911-09-09@\textsc{Schnitzler, Louise} (1840-07-08 – 1911-09-09)|pwv} die Güte u.
               Freundſchaft, die ſie mir erwieſen, nie vergeſſen. {\\}\spacefill\mbox{Paul Goldmann.}\pend
           
         
         \endnumbering\mylabel{h}\end{ledgroupsized}  \newcommand{\dateiname}{L03476}\newcommand{\titel}{Paul Goldmann an Arthur Schnitzler, 10. 9. 1911}\newcommand{\editorInnen}{Martin Anton Müller und Laura Untner}%% latex-leseansicht-abspann.tex
%% Abspann für die Leseansicht.
%% Der Schalter \ifkorrekturansicht ist bereits durch den Vorspann gesetzt.

%% latex-abspann.tex
%% Gemeinsamer Abspann für Korrekturansicht und Leseansicht.
%% Setzt den Schalter \ifkorrekturansicht voraus (gesetzt in den
%% einbindenden Dateien latex-korrekturansicht-abspann.tex bzw.
%% latex-leseansicht-abspann.tex).
%% ---------------------------------------------------------------

\normalsize

% Das esempio-Environment wird nur in der Leseansicht benötigt
\ifkorrekturansicht\else
\newenvironment{esempio}[3]%
{
    \vspace{1.5ex}
    \rlap{\underline{#1}}
    \par
    \setlength{\parindent}{0cm}
    \nopagebreak
    \leftskip=#2cm
    \rightskip=#3cm
}
{
    \par
}
\fi

\doendnotes{C}
\bigskip
\vfill

\clearpage

\footnotesize

\ifkorrekturansicht
  \lohead{\textsc{register}}
\fi

% theindex-Environment neu definieren ohne reledmac
\makeatletter
\renewenvironment{theindex}{%
  \ifkorrekturansicht
    \section*{\indexname}%
  \else
    \subsubsection*{Index der erwähnten Entitäten}%
  \fi
  \setlength{\parindent}{0pt}%
  \setlength{\parskip}{0pt plus 0.3pt}%
  \let\item\@idxitem
}{%
  \ifkorrekturansicht\clearpage\fi
}
\makeatother

\IfFileExists{\jobname-pw.ind}{\input{\jobname-pw.ind}}{}

% Quellenangabe nur in der Leseansicht
\ifkorrekturansicht\else
% Fallback-Definitionen, falls die .tex-Datei \titel etc. nicht gesetzt hat
\providecommand{\titel}{}
\providecommand{\editorInnen}{}
\providecommand{\dateiname}{\jobname}

\vspace{3cm}

\vfill

\footnotesize
\textsc{Quelle}: \titel. Herausgegeben von {\editorInnen}. In: \emph{Arthur Schnitzler: Briefwechsel mit Autorinnen und Autoren}.
 Digitale Edition, https://schnitzler-briefe.acdh.oeaw.ac.at/{\dateiname}.html (Stand \today)
\fi

\end{document}


      