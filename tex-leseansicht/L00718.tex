%% latex-korrekturansicht-vorspann.tex
%% Vorspann für die Korrekturansicht.
%% Lädt die gemeinsame Datei latex-vorspann.tex mit gesetztem Schalter.

\newif\ifkorrekturansicht
\korrekturansichttrue

\input{../tex-inputs/latex-vorspann}


\section[Hugo von Hofmannsthal an Arthur Schnitzler, 26. 8. 1897]{L00718 Hugo von Hofmannsthal an Arthur Schnitzler, 26. 8. 1897}
\nopagebreak\mylabel{L00718v}
\rehead{ }\normalsize\beginnumbering\briefempfaengerindex{Schnitzler, Arthur@\textsc{Schnitzler, Arthur}!zzzHofmannsthal, Hugo von@\emph{von Hugo von Hofmannsthal}!1897-08-261@{26. 8. 1897}|(be}
\toendnotes[C]{\smallbreak\pagebreak[2]}\Standort{CUL, Schnitzler, B 43.}
\physDesc{Postkarte, 352 Zeichen
\newline{}Handschrift: 1) Bleistift, deutsche Kurrent\hspace{1em}2) Bleistift, lateinische Kurrent (\noindent{}Adresse)\hspace{1em}
\newline{}Versand: 1) Stempel: »\nobreak{}\oindex{Varese@\textbf{Varese}, \emph{A.ADM3}|pwk}Varese, 27. 8. 97, 8 M\nobreak{}«.   2) Stempel: »\nobreak{}\oindex{IX., Alsergrund@\textbf{IX., Alsergrund}, \emph{A.ADM3}|pwk}Wien 9/3 72, 29. 8. 97, 8.V, Bestellt\nobreak{}«. 
\newline{}Ordnung: 1) mit Bleistift von unbekannter Hand nummeriert:
                                    »95A«  2) mit Bleistift von unbekannter Hand nummeriert:
                                    »102«}
\buchAbdrucke{\weitereDrucke{Hugo von Hofmannsthal, Arthur Schnitzler: \emph{Briefwechsel}. Frankfurt am Main: \emph{S. Fischer} 1964, S. 95.} }\pstart{}{\pb}Herrn D\textsuperscript{r} Arthur Schnitzler\pend{}\pstart{}Wien\oindex{Wien@\textbf{Wien}, \emph{A.ADM2}|pw}\pend{}\pstart{}IX Franckgasse 1\oindex{Frankgasse 1@\textbf{Frankgasse 1}, \emph{Wohngebäude (K.WHS)}|pw}\pend{}\pstart{}Austria\oindex{Oesterreich@\textbf{Österreich}, \emph{A.PCLI}|pw}\pend{}\pstart{}per Ala\pend{}{\bigskip}\vspace{1em}
\pstart
           
\pstart
           {\pb}26 VIII\pend
           
\pstart
           \raggedleft{}\textsc{Varese\oindex{Varese@\textbf{Varese}, \emph{A.ADM3}|pw} per Milano\oindex{Mailand@\textbf{Mailand}, \emph{P.PPLA}|pw}}{\\}\textsc{Hôtel d’Italie\oindex{Grand Hotel Varese@\textbf{Grand Hotel Varese}, \emph{Hotel (K.HTL)}|pw}}\pend
           \pend
           
\pstart{}mein lieber Arthur\pend\vspace{0.5em}
\pstart
           ich bin ſo zufrieden und glücklich wie glaub ich in meinem Leben nicht, ganz
                  überſchwe{\geminationm}t von Plänen und Halbfertigem. Vielleicht
               hör ich etwas von Ihnen, ich bleibe bis 10. September hier.\pend
           
\pstart
           Ihr{\\[\baselineskip]}\spacefill\mbox{Hugo}\pend
           \leftskip=0em{}
\pstart
           \noindent{}ich dank Ihnen herzlich für vieles wegen Poldy\pwindex{Andrian-Werburg, Leopold von 09.05.1875 – 19.11.1951@\textsc{Andrian-Werburg, Leopold von} (09.05.1875 – 19.11.1951), \emph{Schriftsteller/Schriftstellerin, Diplomat/Diplomatin}|pw}.\pend
           \selectlanguage{ngerman}\endnumbering\briefempfaengerindex{Schnitzler, Arthur@\textsc{Schnitzler, Arthur}!zzzHofmannsthal, Hugo von@\emph{von Hugo von Hofmannsthal}!1897-08-261@{26. 8. 1897}|)be}\mylabel{L00718h}  \normalsize

\doendnotes{C}
\bigskip
\vfill

\clearpage

\footnotesize

\lohead{\textsc{register}}

% Definiere theindex-Environment komplett neu ohne reledmac
\makeatletter
\renewenvironment{theindex}{%
  \section*{\indexname}%
  \setlength{\parindent}{0pt}%
  \setlength{\parskip}{0pt plus 0.3pt}%
  \let\item\@idxitem
}{%
  \clearpage
}
\makeatother

\IfFileExists{\jobname-pw.ind}{\input{\jobname-pw.ind}}{}

\end{document}

      