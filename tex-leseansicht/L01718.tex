%% latex-leseansicht-vorspann.tex
%% Vorspann für die Leseansicht.
%% Lädt die gemeinsame Datei latex-vorspann.tex mit nicht gesetztem Schalter.

\newif\ifkorrekturansicht
\korrekturansichtfalse

\input{../tex-inputs/latex-vorspann}


\section[Arthur Schnitzler an Stefan Großmann, 9. 10. 1907]{L01718 Arthur Schnitzler an Stefan Großmann, 9. 10. 1907}
\nopagebreak\mylabel{L01718v}
\rehead{ }\normalsize\beginnumbering\briefempfaengerindex{Großmann, Stefan@\textsc{Großmann, Stefan}!zzzSchnitzler, Arthur@\emph{von Arthur Schnitzler}!1907-10-091@{9. 10. 1907}|(be}
\toendnotes[C]{\smallbreak\pagebreak[2]}
\correspDesc{Versand  durch Arthur Schnitzler am 9. 10. 1907 in Wien
\newline{}Erhalt  durch Stefan Großmann im Zeitraum [9. 10. 1907
                  – 13. 10. 1907?] in Wien}\toendnotes[C]{\smallbreak}
\Standort{DLA, A:Schnitzler, HS.NZ85.1.896.}
\physDesc{Brief, Durchschlag, 1 Blatt, 1 Seite, 847 Zeichen
\newline{}Schreibmaschine
\newline{}Handschrift: 1) Bleistift, deutsche Kurrent (\noindent{}eine Ergänzung)\hspace{1em}2) roter Buntstift, deutsche Kurrent (\noindent{}vier Unterstreichungen)\hspace{1em}}\toendnotes[C]{\smallbreak}
\pstart
           \raggedleft{}{\pb}9. Okt. 07.\pend
           
\pstart{}Sehr geehrter Herr,\pend\vspace{0.5em}
\pstart
           Die beiden Titel, die Sie in meinem vorigen Brief nicht lesen konnten waren »das neue Lied\pwindex{Schnitzler, Arthur 15.\,5.\,1862 Wien – 21.\,10.\,1931 ebd.@\textsc{Schnitzler, Arthur} (15.\,5.\,1862 Wien – 21.\,10.\,1931 ebd.), \emph{Schriftsteller, Mediziner}!neue Lied@\strich\emph{Das neue Lied}|pw}« und die »letzten Masken\pwindex{Schnitzler, Arthur 15.\,5.\,1862 Wien – 21.\,10.\,1931 ebd.@\textsc{Schnitzler, Arthur} (15.\,5.\,1862 Wien – 21.\,10.\,1931 ebd.), \emph{Schriftsteller, Mediziner}!letzten Masken@\strich\emph{Die letzten Masken}|pw}«. Das erste, eine Novelle aus der Sammlung
                  »Dämmerseelen\pwindex{Schnitzler, Arthur 15.\,5.\,1862 Wien – 21.\,10.\,1931 ebd.@\textsc{Schnitzler, Arthur} (15.\,5.\,1862 Wien – 21.\,10.\,1931 ebd.), \emph{Schriftsteller, Mediziner}!Dämmerseele@\strich\emph{Dämmerseele}|pw}«, das zweite ein Einakter aus
               dem Zyklus »lebendige Stunden\pwindex{Schnitzler, Arthur 15.\,5.\,1862 Wien – 21.\,10.\,1931 ebd.@\textsc{Schnitzler, Arthur} (15.\,5.\,1862 Wien – 21.\,10.\,1931 ebd.), \emph{Schriftsteller, Mediziner}!Lebendige Stunden. Vier Einakter@\strich\emph{Lebendige Stunden. Vier Einakter}|pw}«, beide nicht
               besonders heiter und wohl auch zu lang.\pend
           
\pstart
           Kennen Sie vielleicht die kleine Novellette »Exzentrik\pwindex{Schnitzler, Arthur 15.\,5.\,1862 Wien – 21.\,10.\,1931 ebd.@\textsc{Schnitzler, Arthur} (15.\,5.\,1862 Wien – 21.\,10.\,1931 ebd.), \emph{Schriftsteller, Mediziner}!Excentric@\strich\emph{Excentric}|pw}« aus der Sammlung »die griechische
                  Tänzerin\pwindex{Schnitzler, Arthur 15.\,5.\,1862 Wien – 21.\,10.\,1931 ebd.@\textsc{Schnitzler, Arthur} (15.\,5.\,1862 Wien – 21.\,10.\,1931 ebd.), \emph{Schriftsteller, Mediziner}!griechische Tänzerin. Novellette@\strich\emph{Die griechische Tänzerin. Novellette}|pw}«? Sie wird von den Leuten im Allgemeinen für lustig gehalten, hat
               sich schon einigemale als Vorlesestück bewährt. Wollen Sie vielleicht die Güte haben
               sie sich anzusehen und mir zu sagen, ob Sie sie für den Abschluss des Abends für
               geeignet halten.\pend
           
\pstart
           Ich bitte Sie auch mir womöglich die Hausnummer mitzuteilen, wo ich in der Königseggasse\oindex{Wien@\textbf{Wien}!VI., Mariahilf@\textbf{VI., Mariahilf}!Königseggasse@\textbf{Königseggasse}, \emph{Straße}|pw} lesen soll.\pend
           
\pstart
           Ihrer freundlichen Antwort entgegensehend{\\[\baselineskip]}Ihr sehr ergebener\pend
           \leftskip=0em{}{\vspace{1\baselineskip}}
\pstart
           \noindent{}\label{T_L01718-1v}\edtext{Herrn}{\lemma{\textnormal{\emph{Herrn}}}\Cendnote{\textnormal{nachträglich handschriftlich ergänzt}}}\label{T_L01718-1} Stefan Grossmann,
                     Wien\oindex{Wien@\textbf{Wien}, \emph{Verwaltungsgebiet}|pw}\pend
           \selectlanguage{ngerman}\endnumbering\briefempfaengerindex{Großmann, Stefan@\textsc{Großmann, Stefan}!zzzSchnitzler, Arthur@\emph{von Arthur Schnitzler}!1907-10-091@{9. 10. 1907}|)be}\mylabel{L01718h}  \newcommand{\dateiname}{L01718}\newcommand{\titel}{Arthur Schnitzler an Stefan Großmann, 9. 10. 1907}\newcommand{\editorInnen}{Martin Anton Müller und Gerd-Hermann Susen}%% latex-leseansicht-abspann.tex
%% Abspann für die Leseansicht.
%% Der Schalter \ifkorrekturansicht ist bereits durch den Vorspann gesetzt.

%% latex-abspann.tex
%% Gemeinsamer Abspann für Korrekturansicht und Leseansicht.
%% Setzt den Schalter \ifkorrekturansicht voraus (gesetzt in den
%% einbindenden Dateien latex-korrekturansicht-abspann.tex bzw.
%% latex-leseansicht-abspann.tex).
%% ---------------------------------------------------------------

\normalsize

% Das esempio-Environment wird nur in der Leseansicht benötigt
\ifkorrekturansicht\else
\newenvironment{esempio}[3]%
{
    \vspace{1.5ex}
    \rlap{\underline{#1}}
    \par
    \setlength{\parindent}{0cm}
    \nopagebreak
    \leftskip=#2cm
    \rightskip=#3cm
}
{
    \par
}
\fi

\doendnotes{C}
\bigskip
\vfill

\clearpage

\footnotesize

\ifkorrekturansicht
  \lohead{\textsc{register}}
\fi

% theindex-Environment neu definieren ohne reledmac
\makeatletter
\renewenvironment{theindex}{%
  \ifkorrekturansicht
    \section*{\indexname}%
  \else
    \subsubsection*{Index der erwähnten Entitäten}%
  \fi
  \setlength{\parindent}{0pt}%
  \setlength{\parskip}{0pt plus 0.3pt}%
  \let\item\@idxitem
}{%
  \ifkorrekturansicht\clearpage\fi
}
\makeatother

\IfFileExists{\jobname-pw.ind}{\input{\jobname-pw.ind}}{}

% Quellenangabe nur in der Leseansicht
\ifkorrekturansicht\else
% Fallback-Definitionen, falls die .tex-Datei \titel etc. nicht gesetzt hat
\providecommand{\titel}{}
\providecommand{\editorInnen}{}
\providecommand{\dateiname}{\jobname}

\vspace{3cm}

\vfill

\footnotesize
\textsc{Quelle}: \titel. Herausgegeben von {\editorInnen}. In: \emph{Arthur Schnitzler: Briefwechsel mit Autorinnen und Autoren}.
 Digitale Edition, https://schnitzler-briefe.acdh.oeaw.ac.at/{\dateiname}.html (Stand \today)
\fi

\end{document}


