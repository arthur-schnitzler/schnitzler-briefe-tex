%% latex-korrekturansicht-vorspann.tex
%% Vorspann für die Korrekturansicht.
%% Lädt die gemeinsame Datei latex-vorspann.tex mit gesetztem Schalter.

\newif\ifkorrekturansicht
\korrekturansichttrue

\input{../tex-inputs/latex-vorspann}


\section[Arthur Schnitzler an Stefan Großmann, 9. 10. 1907]{L01718 Arthur Schnitzler an Stefan Großmann, 9. 10. 1907}
\nopagebreak\mylabel{L01718v}
\rehead{ }\normalsize\beginnumbering\briefempfaengerindex{Grossmann, Stefan@\textsc{Großmann, Stefan}!zzzSchnitzler, Arthur@\emph{von Arthur Schnitzler}!1907-10-091@{9. 10. 1907}|(be}
\toendnotes[C]{\smallbreak\pagebreak[2]}\Standort{DLA, A:Schnitzler, HS.NZ85.1.896.}
\physDesc{Brief, Durchschlag1 Blatt, 1 Seite, 847 Zeichen
\newline{}Schreibmaschine
\newline{}Handschrift: 1) Bleistift, deutsche Kurrent (\noindent{}eine Ergänzung)\hspace{1em}2) roter Buntstift, deutsche Kurrent (\noindent{}vier Unterstreichungen)\hspace{1em}}\toendnotes[C]{\smallbreak}
\pstart
           \raggedleft{}{\pb}9. Okt. 07. \pend
           
\pstart{}Sehr geehrter Herr,\pend\vspace{0.5em}
\pstart
           Die beiden Titel, die Sie in meinem vorigen Brief nicht lesen konnten waren »das neue Lied\pwindex{neue Lied@\emph{Das neue Lied}|pw}« und die »letzten Masken\pwindex{letzten Masken@\emph{Die letzten Masken}|pw}«. Das erste, eine Novelle aus der Sammlung
                  »Dämmerseelen\pwindex{Daemmerseele@\emph{Dämmerseele}|pw}«, das zweite ein Einakter aus
               dem Zyklus »lebendige Stunden\pwindex{Lebendige Stunden. Vier Einakter@\emph{Lebendige Stunden. Vier Einakter}|pw}«, beide nicht
               besonders heiter und wohl auch zu lang.\pend
           
\pstart
           Kennen Sie vielleicht die kleine Novellette »Exzentrik\pwindex{Excentric@\emph{Excentric}|pw}« aus der Sammlung »die griechische
                  Tänzerin\pwindex{griechische Taenzerin. Novellette@\emph{Die griechische Tänzerin. Novellette}|pw}«? Sie wird von den Leuten im Allgemeinen für lustig gehalten, hat
               sich schon einigemale als Vorlesestück bewährt. Wollen Sie vielleicht die Güte haben
               sie sich anzusehen und mir zu sagen, ob Sie sie für den Abschluss des Abends für
               geeignet halten.\pend
           
\pstart
           Ich bitte Sie auch mir womöglich die Hausnummer mitzuteilen, wo ich in der Königseggasse\oindex{Koenigseggasse@\textbf{Königseggasse}, \emph{Straße (K.STR)}|pw} lesen soll.\pend
           
\pstart
           Ihrer freundlichen Antwort entgegensehend{\\[\baselineskip]}Ihr sehr ergebener\pend
           \leftskip=0em{}{\vspace{1\baselineskip}}
\pstart
           \noindent{}\label{T_L01718-1v}\edtext{Herrn}{\lemma{\textnormal{\emph{Herrn}}}\Cendnote{\textnormal{nachträglich handschriftlich ergänzt}}}\label{T_L01718-1} Stefan Grossmann,
                     Wien\oindex{Wien@\textbf{Wien}, \emph{A.ADM2}|pw}\pend
           \selectlanguage{ngerman}\endnumbering\briefempfaengerindex{Grossmann, Stefan@\textsc{Großmann, Stefan}!zzzSchnitzler, Arthur@\emph{von Arthur Schnitzler}!1907-10-091@{9. 10. 1907}|)be}\mylabel{L01718h}  \normalsize

\doendnotes{C}
\bigskip
\vfill

\clearpage

\footnotesize

\lohead{\textsc{register}}

% Definiere theindex-Environment komplett neu ohne reledmac
\makeatletter
\renewenvironment{theindex}{%
  \section*{\indexname}%
  \setlength{\parindent}{0pt}%
  \setlength{\parskip}{0pt plus 0.3pt}%
  \let\item\@idxitem
}{%
  \clearpage
}
\makeatother

\IfFileExists{\jobname-pw.ind}{\input{\jobname-pw.ind}}{}

\end{document}

      