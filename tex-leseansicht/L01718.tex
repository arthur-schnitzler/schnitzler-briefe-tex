%% latex-leseansicht-vorspann.tex
%% Vorspann für die Leseansicht.
%% Lädt die gemeinsame Datei latex-vorspann.tex mit nicht gesetztem Schalter.

\newif\ifkorrekturansicht
\korrekturansichtfalse

\input{../tex-inputs/latex-vorspann}

\begin{center}
            \textcolor{red}{ENTWURF. ENTZIFFERUNG NOCH NICHT KORREKTURGELESEN}
                      \end{center}
            
               \section[Arthur Schnitzler an Stefan Großmann, 9. 10. 1907]{ Arthur Schnitzler an Stefan Großmann, 9. 10. 1907}\nopagebreak\mylabel{v}\rehead{ }\begin{ledgroupsized}[t]{13cm}\normalsize\beginnumbering\briefempfaengerindex{Grossmann, Stefan@\textsc{Großmann, Stefan}!zzzSchnitzler, Arthur@\emph{von Arthur Schnitzler}!1907-10-091@{9. 10. 1907}|(be} \toendnotes[C]{\smallbreak\pagebreak[2]} \Standort{DLA, A:Schnitzler, HS.NZ85.1.896.}
\physDesc{Brief, 1 Blatt, 1 Seite, maschineller Durchschlag
\newline{}Schreibmaschine
\newline{}Handschrift: 1) Bleistift, deutsche Kurrent (\noindent{}eine Ergänzung)\hspace{1em}2) roter Buntstift, deutsche Kurrent (\noindent{}vier Unterstreichungen)\hspace{1em}}\toendnotes[C]{\smallbreak}\pstart
           \raggedleft{}{\pb}9. Okt. 07. \pend
           \pstart{}Sehr geehrter Herr,\pend\pstart
           Die beiden Titel, die Sie in meinem vorigen Brief nicht lesen konnten waren »das neue Lied\pwindex{Schnitzler, Arthur 15.05.1862 – 21.10.1931@\textsc{Schnitzler, Arthur} (15.05.1862 – 21.10.1931), \emph{Schriftsteller, Mediziner}!neue Lied23. 04. 1905@\strich\emph{Das neue Lied} {[}23. 04. 1905{]}|pw}« und die »letzten Masken\pwindex{Schnitzler, Arthur 15.05.1862 – 21.10.1931@\textsc{Schnitzler, Arthur} (15.05.1862 – 21.10.1931), \emph{Schriftsteller, Mediziner}!letzten Masken1901@\strich\emph{Die letzten Masken} {[}1901{]}|pw}«. Das erste, eine Novelle aus der Sammlung
                        »Dämmerseelen\pwindex{Schnitzler, Arthur 15.05.1862 – 21.10.1931@\textsc{Schnitzler, Arthur} (15.05.1862 – 21.10.1931), \emph{Schriftsteller, Mediziner}!Fremde18.5.1902 – 18.5.1902@\strich\emph{Die Fremde} {[}18.5.1902 – 18.5.1902{]}|pw}«, das zweite ein Einakter
                    aus dem Zyklus »lebendige Stunden\pwindex{Schnitzler, Arthur 15.05.1862 – 21.10.1931@\textsc{Schnitzler, Arthur} (15.05.1862 – 21.10.1931), \emph{Schriftsteller, Mediziner}!Lebendige Stunden. Vier Einakter1901-12-23@\strich\emph{Lebendige Stunden. Vier Einakter} {[}1901-12-23{]}|pw}«, beide
                    nicht besonders heiter und wohl auch zu lang.\pend
           \pstart
           Kennen Sie vielleicht die kleine Novellette »Exzentrik\pwindex{Schnitzler, Arthur 15.05.1862 – 21.10.1931@\textsc{Schnitzler, Arthur} (15.05.1862 – 21.10.1931), \emph{Schriftsteller, Mediziner}!Excentric16. 07. 1902@\strich\emph{Excentric} {[}16. 07. 1902{]}|pw}« aus der Sammlung »die
                        griechische Tänzerin\pwindex{Schnitzler, Arthur 15.05.1862 – 21.10.1931@\textsc{Schnitzler, Arthur} (15.05.1862 – 21.10.1931), \emph{Schriftsteller, Mediziner}!griechische Taenzerin. Novellette28. 09. 1902@\strich\emph{Die griechische Tänzerin. Novellette} {[}28. 09. 1902{]}|pw}«? Sie wird von den Leuten im Allgemeinen für
                    lustig gehalten, hat sich schon einigemale als Vorlesestück bewährt. Wollen Sie
                    vielleicht die Güte haben sie sich anzusehen und mir zu sagen, ob Sie sie für
                    den Abschluss des Abends für geeignet halten.\pend
           \pstart
           Ich bitte Sie auch mir womöglich die Hausnummer mitzuteilen, wo ich in der Königseggasse\oindex{Koenigseggasse@\textbf{Königseggasse}|pw} lesen soll.\pend
           \pstart
           Ihrer freundlichen Antwort entgegensehend{\\[\baselineskip]}Ihr sehr ergebener\pend
           \leftskip=0em{}{\bigskip}\pstart
           \noindent{}\label{T_L01718_1v}\edtext{Herrn}{\lemma{\textnormal{\emph{Herrn}}}\Cendnote{\textnormal{nachträglich handschriftlich ergänzt}}}\label{T_L01718_1h}
                        Stefan Grossmann, Wien\oindex{Wien@\textbf{Wien}|pw}\pend
           \endnumbering\briefempfaengerindex{Grossmann, Stefan@\textsc{Großmann, Stefan}!zzzSchnitzler, Arthur@\emph{von Arthur Schnitzler}!1907-10-091@{9. 10. 1907}|)be}\mylabel{h}\end{ledgroupsized}  \newcommand{\dateiname}{L01718}\newcommand{\titel}{Arthur Schnitzler an Stefan Großmann, 9. 10. 1907}\newcommand{\editorInnen}{Martin Anton Müller und Gerd-Hermann Susen}%% latex-leseansicht-abspann.tex
%% Abspann für die Leseansicht.
%% Der Schalter \ifkorrekturansicht ist bereits durch den Vorspann gesetzt.

%% latex-abspann.tex
%% Gemeinsamer Abspann für Korrekturansicht und Leseansicht.
%% Setzt den Schalter \ifkorrekturansicht voraus (gesetzt in den
%% einbindenden Dateien latex-korrekturansicht-abspann.tex bzw.
%% latex-leseansicht-abspann.tex).
%% ---------------------------------------------------------------

\normalsize

% Das esempio-Environment wird nur in der Leseansicht benötigt
\ifkorrekturansicht\else
\newenvironment{esempio}[3]%
{
    \vspace{1.5ex}
    \rlap{\underline{#1}}
    \par
    \setlength{\parindent}{0cm}
    \nopagebreak
    \leftskip=#2cm
    \rightskip=#3cm
}
{
    \par
}
\fi

\doendnotes{C}
\bigskip
\vfill

\clearpage

\footnotesize

\ifkorrekturansicht
  \lohead{\textsc{register}}
\fi

% theindex-Environment neu definieren ohne reledmac
\makeatletter
\renewenvironment{theindex}{%
  \ifkorrekturansicht
    \section*{\indexname}%
  \else
    \subsubsection*{Index der erwähnten Entitäten}%
  \fi
  \setlength{\parindent}{0pt}%
  \setlength{\parskip}{0pt plus 0.3pt}%
  \let\item\@idxitem
}{%
  \ifkorrekturansicht\clearpage\fi
}
\makeatother

\IfFileExists{\jobname-pw.ind}{\input{\jobname-pw.ind}}{}

% Quellenangabe nur in der Leseansicht
\ifkorrekturansicht\else
% Fallback-Definitionen, falls die .tex-Datei \titel etc. nicht gesetzt hat
\providecommand{\titel}{}
\providecommand{\editorInnen}{}
\providecommand{\dateiname}{\jobname}

\vspace{3cm}

\vfill

\footnotesize
\textsc{Quelle}: \titel. Herausgegeben von {\editorInnen}. In: \emph{Arthur Schnitzler: Briefwechsel mit Autorinnen und Autoren}.
 Digitale Edition, https://schnitzler-briefe.acdh.oeaw.ac.at/{\dateiname}.html (Stand \today)
\fi

\end{document}


      