%% latex-korrekturansicht-vorspann.tex
%% Vorspann für die Korrekturansicht.
%% Lädt die gemeinsame Datei latex-vorspann.tex mit gesetztem Schalter.

\newif\ifkorrekturansicht
\korrekturansichttrue

\input{../tex-inputs/latex-vorspann}


\section[Arthur Schnitzler an Robert Adam, 18. 6. 1915]{L02208 Arthur Schnitzler an Robert Adam, 18. 6. 1915}
\nopagebreak\mylabel{L02208v}
\rehead{ }\normalsize\beginnumbering\briefempfaengerindex{Adam, Robert@\textsc{Adam, Robert}!zzzSchnitzler, Arthur@\emph{von Arthur Schnitzler}!1915-06-181@{18. 6. 1915}|(be}
\toendnotes[C]{\smallbreak\pagebreak[2]}\Standort{DLA, 96.34.1/12.}
\physDesc{Briefkarte, , Umschlag, 1046 Zeichen
\newline{}Handschrift: schwarze Tinte, lateinische Kurrent
\newline{}Versand: Stempel: »\nobreak{}Wien\nobreak{}«.  }\toendnotes[C]{\smallbreak}\pstart{}{\pb}\textcolor{gray}{\textbf{Dr. Arthur Schnitzler}}\pend{}\pstart{}\textcolor{gray}{\textbf{Wien XVIII. Sternwartestrasse 71\oindex{Sternwartestrasse 71@\textbf{Sternwartestraße 71}, \emph{Wohngebäude (K.WHS)}|pw}}}\pend{}{\bigskip}\pstart{}{\pb}Herrn Dr. Rob. Ad. Pollak\pend{}\pstart{}k.k.-Bezirksrichter\pend{}\pstart{}Zistersdorf\oindex{Zistersdorf@\textbf{Zistersdorf}, \emph{A.ADM3}|pw}.\pend{}{\bigskip}\vspace{1em}
\pstart
           {\pb}\textcolor{gray}{\textbf{Dr. Arthur Schnitzler}}\hfill 18. 6. 15.\pend
           
\pstart
           \textcolor{gray}{\textbf{Wien XVIII. Sternwartestrasse 71\oindex{Sternwartestrasse 71@\textbf{Sternwartestraße 71}, \emph{Wohngebäude (K.WHS)}|pw}}}\pend
           
\pstart{}Verehrter Herr Adam,\pend\vspace{0.5em}
\pstart
           mit besonderm Vergnügen habe ich Ihre freundliche Manuscriptsendung\pwindex{Fremde@\emph{Der Fremde}|pwv} empfangen, mit wirklichem, innersten
               Interesse die sechs Scenen gelesen, und wüßte nicht, was Sie davon abhalten sollte,
               diese vornehme we{\geminationn} auch nicht in allen Theilen gleich
               starke, und in manchen rhythmischen Eigenheiten nicht durchaus einleuchtende Dichtung
               dem Publikum oder auch den Theatern vorzulegen. Gewiß werden viele (und nicht die
               urtheilselosesten) {\pb}\introOben{}Leute\introOben{} mit gleichem Antheil und zuweilen mit tieferer
               Bewegung die Scenen auf sich wirken lassen – in denen manchen nun auch eine
               Theaterwirkung zu stecken scheint. Freilich werden nicht viele Bühnen für diese
               eigenartige Sache in Betracht kommen. We{\geminationn} Sie im Laufe
               der nächsten Zeit nach Wien\oindex{Wien@\textbf{Wien}, \emph{A.ADM2}|pw} kämen, lassen Sie
               michs vielleicht wissen; es wäre mir ein Vergnügen, Sie wieder zu sprechen –
               eventuell auch zu dem problematischen Capitel der praktischen Möglichkeiten Ihrer Arbeit\pwindex{Fremde@\emph{Der Fremde}|pwv}.\pend
           
\pstart
           Verbindlich grüßend u dankend{\\[\baselineskip]}Ihr sehr ergebner{\\[\baselineskip]}\spacefill\mbox{Arthur Schnitzler}\pend
           \leftskip=0em{}\selectlanguage{ngerman}\endnumbering\briefempfaengerindex{Adam, Robert@\textsc{Adam, Robert}!zzzSchnitzler, Arthur@\emph{von Arthur Schnitzler}!1915-06-181@{18. 6. 1915}|)be}\mylabel{L02208h}  \normalsize

\doendnotes{C}
\bigskip
\vfill

\clearpage

\footnotesize

\lohead{\textsc{register}}

% Definiere theindex-Environment komplett neu ohne reledmac
\makeatletter
\renewenvironment{theindex}{%
  \section*{\indexname}%
  \setlength{\parindent}{0pt}%
  \setlength{\parskip}{0pt plus 0.3pt}%
  \let\item\@idxitem
}{%
  \clearpage
}
\makeatother

\IfFileExists{\jobname-pw.ind}{\input{\jobname-pw.ind}}{}

\end{document}

      