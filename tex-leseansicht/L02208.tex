%% latex-leseansicht-vorspann.tex
%% Vorspann für die Leseansicht.
%% Lädt die gemeinsame Datei latex-vorspann.tex mit nicht gesetztem Schalter.

\newif\ifkorrekturansicht
\korrekturansichtfalse

\input{../tex-inputs/latex-vorspann}


         
         \renewcommand{\erwaehntePersonen}{Personen: Robert Adam}
         \renewcommand{\erwaehnteOrte}{Orte: Sternwartestraße 71, Wien, Zistersdorf}
         \renewcommand{\erwaehnteWerke}{Werke: Der Fremde}
               \section[Arthur Schnitzler an Robert Adam, 18. 6. 1915]{ Arthur Schnitzler an Robert Adam, 18. 6. 1915}\nopagebreak\mylabel{v}\rehead{ }\begin{ledgroupsized}[t]{13cm}\normalsize\beginnumbering\briefempfaengerindex{Adam, Robert@\textsc{Adam, Robert}!zzzSchnitzler, Arthur@\emph{von Arthur Schnitzler}!1915-06-181@{18. 6. 1915}|(be} \toendnotes[C]{\smallbreak\pagebreak[2]} \Standort{DLA, 96.34.1/12.}
\physDesc{Briefkarte, , Umschlag, 1046 Zeichen
\newline{}Handschrift: schwarze Tinte, lateinische Kurrent
\newline{}Versand: Stempel: »\nobreak{}Wien\nobreak{}«.  }\toendnotes[C]{\smallbreak}\pstart{}{\pb}\textcolor{gray}{\textbf{Dr. Arthur Schnitzler}}\pend{}\pstart{}\textcolor{gray}{\textbf{Wien XVIII. Sternwartestrasse 71\oindex{Sternwartestrasse 71@\textbf{Sternwartestraße 71}|pw}}}\pend{}{\bigskip}\pstart{}{\pb}Herrn Dr. Rob. Ad. Pollak\pend{}\pstart{}k.k.-Bezirksrichter\pend{}\pstart{}Zistersdorf\oindex{Zistersdorf@\textbf{Zistersdorf}|pw}.\pend{}{\bigskip}\pstart
           \noindent{}{\pb}\textcolor{gray}{\textbf{Dr. Arthur Schnitzler}}\hfill 18. 6. 15.\pend
           \pstart
           \textcolor{gray}{\textbf{Wien XVIII. Sternwartestrasse 71\oindex{Sternwartestrasse 71@\textbf{Sternwartestraße 71}|pw}}}\pend
           \pstart{}Verehrter Herr Adam,\pend\pstart
           mit besonderm Vergnügen habe ich Ihre freundliche Manuscriptsendung\pwindex{Adam, Robert 20.04.1877 – 16.10.1961@\textsc{Adam, Robert} (20.04.1877 – 16.10.1961), \emph{Schriftsteller, Richter}!Fremde@\strich\emph{Der Fremde}|pwv} empfangen, mit wirklichem, innersten
               Interesse die sechs Scenen gelesen, und wüßte nicht, was Sie davon abhalten sollte,
               diese vornehme we{\geminationn} auch nicht in allen Theilen gleich
               starke, und in manchen rhythmischen Eigenheiten nicht durchaus einleuchtende Dichtung
               dem Publikum oder auch den Theatern vorzulegen. Gewiß werden viele (und nicht die
               urtheilselosesten) {\pb}\introOben{}Leute\introOben{} mit gleichem Antheil und zuweilen mit tieferer
               Bewegung die Scenen auf sich wirken lassen – in denen manchen nun auch eine
               Theaterwirkung zu stecken scheint. Freilich werden nicht viele Bühnen für diese
               eigenartige Sache in Betracht kommen. We{\geminationn} Sie im Laufe
               der nächsten Zeit nach Wien\oindex{Wien@\textbf{Wien}|pw} kämen, lassen Sie
               michs vielleicht wissen; es wäre mir ein Vergnügen, Sie wieder zu sprechen –
               eventuell auch zu dem problematischen Capitel der praktischen Möglichkeiten Ihrer Arbeit\pwindex{Adam, Robert 20.04.1877 – 16.10.1961@\textsc{Adam, Robert} (20.04.1877 – 16.10.1961), \emph{Schriftsteller, Richter}!Fremde@\strich\emph{Der Fremde}|pwv}.\pend
           \pstart
           Verbindlich grüßend u dankend{\\[\baselineskip]}Ihr sehr ergebner{\\[\baselineskip]}\spacefill\mbox{Arthur Schnitzler}\pend
           \leftskip=0em{}
         
         \endnumbering\mylabel{h}\end{ledgroupsized}  \newcommand{\dateiname}{L02208}\newcommand{\titel}{Arthur Schnitzler an Robert Adam, 18. 6. 1915}\newcommand{\editorInnen}{Martin Anton Müller und Gerd-Hermann Susen}%% latex-leseansicht-abspann.tex
%% Abspann für die Leseansicht.
%% Der Schalter \ifkorrekturansicht ist bereits durch den Vorspann gesetzt.

%% latex-abspann.tex
%% Gemeinsamer Abspann für Korrekturansicht und Leseansicht.
%% Setzt den Schalter \ifkorrekturansicht voraus (gesetzt in den
%% einbindenden Dateien latex-korrekturansicht-abspann.tex bzw.
%% latex-leseansicht-abspann.tex).
%% ---------------------------------------------------------------

\normalsize

% Das esempio-Environment wird nur in der Leseansicht benötigt
\ifkorrekturansicht\else
\newenvironment{esempio}[3]%
{
    \vspace{1.5ex}
    \rlap{\underline{#1}}
    \par
    \setlength{\parindent}{0cm}
    \nopagebreak
    \leftskip=#2cm
    \rightskip=#3cm
}
{
    \par
}
\fi

\doendnotes{C}
\bigskip
\vfill

\clearpage

\footnotesize

\ifkorrekturansicht
  \lohead{\textsc{register}}
\fi

% theindex-Environment neu definieren ohne reledmac
\makeatletter
\renewenvironment{theindex}{%
  \ifkorrekturansicht
    \section*{\indexname}%
  \else
    \subsubsection*{Index der erwähnten Entitäten}%
  \fi
  \setlength{\parindent}{0pt}%
  \setlength{\parskip}{0pt plus 0.3pt}%
  \let\item\@idxitem
}{%
  \ifkorrekturansicht\clearpage\fi
}
\makeatother

\IfFileExists{\jobname-pw.ind}{\input{\jobname-pw.ind}}{}

% Quellenangabe nur in der Leseansicht
\ifkorrekturansicht\else
% Fallback-Definitionen, falls die .tex-Datei \titel etc. nicht gesetzt hat
\providecommand{\titel}{}
\providecommand{\editorInnen}{}
\providecommand{\dateiname}{\jobname}

\vspace{3cm}

\vfill

\footnotesize
\textsc{Quelle}: \titel. Herausgegeben von {\editorInnen}. In: \emph{Arthur Schnitzler: Briefwechsel mit Autorinnen und Autoren}.
 Digitale Edition, https://schnitzler-briefe.acdh.oeaw.ac.at/{\dateiname}.html (Stand \today)
\fi

\end{document}


      