%% latex-leseansicht-vorspann.tex
%% Vorspann für die Leseansicht.
%% Lädt die gemeinsame Datei latex-vorspann.tex mit nicht gesetztem Schalter.

\newif\ifkorrekturansicht
\korrekturansichtfalse

\input{../tex-inputs/latex-vorspann}


         
         \renewcommand{\erwaehntePersonen}{Personen: Paula Beer-Hofmann, Hugo von Hofmannsthal}
         \renewcommand{\erwaehnteInstitutionen}{Institutionen: Die Zeit. Wiener Wochenschrift}
         \renewcommand{\erwaehnteOrte}{Orte: Brixen, Franzensfeste, Hotel und Pension Rudolfshöhe (Leopold Petter), München, Vahrn, Wien}
         \renewcommand{\erwaehnteWerke}{Werke: Der Tod Georgs, Der Tod Georgs. Fragment}
               \section[Richard Beer-Hofmann an Arthur Schnitzler, 12. 9. 1899]{ Richard Beer-Hofmann an Arthur Schnitzler, 12. 9. 1899}\nopagebreak\mylabel{v}\rehead{ }\begin{ledgroupsized}[t]{13cm}\normalsize\beginnumbering \toendnotes[C]{\smallbreak\pagebreak[2]} \Standort{CUL, Schnitzler, B 8.}
\physDesc{Brief, 1 Blatt, 2 Seiten, 1156 Zeichen
\newline{}Handschrift: schwarze Tinte, lateinische Kurrent
\newline{}Ordnung: mit Bleistift von unbekannter Hand nummeriert:
                                    »141« }\buchAbdrucke{\weitereDrucke{Arthur Schnitzler, Richard Beer-Hofmann: \emph{Briefwechsel 1891–1931}. Hg. Konstanze Fliedl. Wien, Zürich: \emph{Europaverlag} 1992, S. 136–137.} }\toendnotes[C]{\smallbreak}\pstart
           \raggedleft{}{\pb}Vahrn\oindex{Vahrn@\textbf{Vahrn}|pw}{ }12/IX 99\pend
           \pstart
           Lieber Arthur! Ihre Karte gestern, heute Ihren Brief vom
                  9. erhalten. Ich habe ihn mehr errathen als gelesen; was heisst
                  \textcolor{gray}{durch allerlei.}{ }Hugo\pwindex{Hofmannsthal, Hugo von 1874-02-01 – 1929-07-15@\textsc{Hofmannsthal, Hugo von} (1874-02-01 – 1929-07-15), \emph{Schriftsteller}|pw}s Brief vom 7. daß er herko{\geminationm}en will habe ich gestern erhalten, und ihm telegrafirt
               er möge nur kommen. Ich arbeite täglich, und komme – wenn auch langsam vorwärts. In
               der »Zeit\orgindex{Zeit. Wiener Wochenschrift@Die Zeit. Wiener Wochenschrift|pw}« werden voraussichtlich nur die ersten
                  \label{K_L00973-1v}\edtext{2. Cap.\pwindex{Beer-Hofmann, Richard 1866-07-11 – 1945-09-26@\textsc{Beer-Hofmann, Richard} (1866-07-11 – 1945-09-26), \emph{Schriftsteller}!Tod Georgs. Fragment4.11.1899 – 25.11.1899@\strich\emph{Der Tod Georgs. Fragment} {[}4.11.1899 – 25.11.1899{]}|pwv}}{\lemma{\textnormal{\emph{2. Cap.}}}\Cendnote{\textnormal{Es erschien nur das gekürzte zweite Kapitel\pwindex{Beer-Hofmann, Richard 1866-07-11 – 1945-09-26@\textsc{Beer-Hofmann, Richard} (1866-07-11 – 1945-09-26), \emph{Schriftsteller}!Tod Georgs1900@\strich\emph{Der Tod Georgs} {[}1900{]}|pwkv} in vier
                  Teilen zwischen 4. und 25. 11. 1899.}}}\label{K_L00973-1h} erscheinen.
               Das Ganze würden sie in \uline{10} Fortsetz. tranchiren
               müssen, und das Buch könnte erst Mitte Dez. erscheinen. Das wäre zu langweilig. Wer
               wird also auf dem Titel figuriren? Schon entschieden? Ich {\pb}mache Sie aufmerksam: In München\oindex{Muenchen@\textbf{München}|pw} geht um 9.10 Nachts ein Zug
               ab, der um 4.36 Früh in Brixen\oindex{Brixen@\textbf{Brixen}|pw} ist. Von da 20
               Minuten Wagen nach \strikeout{V}{ }Vahrn\oindex{Vahrn@\textbf{Vahrn}|pw}. Außerdem ein N. S. Express, der um
                  9.55{ }\substVorne{}\textsuperscript{Früh}\substDazwischen{}Vorm\substHinten{} von München\oindex{Muenchen@\textbf{München}|pw} abgeht, um 3.02
                  Nachm. in Franzensfeste\oindex{Franzensfeste@\textbf{Franzensfeste}|pw} ist; \strikeout{von} (in Brixen\oindex{Brixen@\textbf{Brixen}|pw}
               hält er nicht). Von Franzensfeste\oindex{Franzensfeste@\textbf{Franzensfeste}|pw} mit dem Wagen
               circa 9–10 Kilom. hieher. Es ist hier angenehm, ruhig, bei der table d’hôte nur Paula\pwindex{Beer-Hofmann, Paula 25.02.1879 – 30.10.1939@\textsc{Beer-Hofmann, Paula} (25.02.1879 – 30.10.1939)|pw} und ich inbegriffen \uline{4} Personen. Abends, wie bei Petter\oindex{Hotel und Pension Rudolfshoehe (Leopold Petter)@\textbf{Hotel und Pension Rudolfshöhe (Leopold Petter)}|pw},
               an separaten Tischen. Lärchen und Edelkastanienwald. Gegenüber Weingelände.
               Vielleicht ko{\geminationm}en Sie? Man soll ja doch so spät als
               möglich nach Wien\oindex{Wien@\textbf{Wien}|pw}?\pend
           \pstart
           Herzlichst{\\[\baselineskip]}Ihr{\\[\baselineskip]}\spacefill\mbox{Richard}\pend
           \leftskip=0em{}
         
         \endnumbering\mylabel{h}\end{ledgroupsized}  \newcommand{\dateiname}{L00973}\newcommand{\titel}{Richard Beer-Hofmann an Arthur Schnitzler, 12. 9. 1899}\newcommand{\editorInnen}{Martin Anton Müller und Gerd-Hermann Susen}%% latex-leseansicht-abspann.tex
%% Abspann für die Leseansicht.
%% Der Schalter \ifkorrekturansicht ist bereits durch den Vorspann gesetzt.

%% latex-abspann.tex
%% Gemeinsamer Abspann für Korrekturansicht und Leseansicht.
%% Setzt den Schalter \ifkorrekturansicht voraus (gesetzt in den
%% einbindenden Dateien latex-korrekturansicht-abspann.tex bzw.
%% latex-leseansicht-abspann.tex).
%% ---------------------------------------------------------------

\normalsize

% Das esempio-Environment wird nur in der Leseansicht benötigt
\ifkorrekturansicht\else
\newenvironment{esempio}[3]%
{
    \vspace{1.5ex}
    \rlap{\underline{#1}}
    \par
    \setlength{\parindent}{0cm}
    \nopagebreak
    \leftskip=#2cm
    \rightskip=#3cm
}
{
    \par
}
\fi

\doendnotes{C}
\bigskip
\vfill

\clearpage

\footnotesize

\ifkorrekturansicht
  \lohead{\textsc{register}}
\fi

% theindex-Environment neu definieren ohne reledmac
\makeatletter
\renewenvironment{theindex}{%
  \ifkorrekturansicht
    \section*{\indexname}%
  \else
    \subsubsection*{Index der erwähnten Entitäten}%
  \fi
  \setlength{\parindent}{0pt}%
  \setlength{\parskip}{0pt plus 0.3pt}%
  \let\item\@idxitem
}{%
  \ifkorrekturansicht\clearpage\fi
}
\makeatother

\IfFileExists{\jobname-pw.ind}{\input{\jobname-pw.ind}}{}

% Quellenangabe nur in der Leseansicht
\ifkorrekturansicht\else
% Fallback-Definitionen, falls die .tex-Datei \titel etc. nicht gesetzt hat
\providecommand{\titel}{}
\providecommand{\editorInnen}{}
\providecommand{\dateiname}{\jobname}

\vspace{3cm}

\vfill

\footnotesize
\textsc{Quelle}: \titel. Herausgegeben von {\editorInnen}. In: \emph{Arthur Schnitzler: Briefwechsel mit Autorinnen und Autoren}.
 Digitale Edition, https://schnitzler-briefe.acdh.oeaw.ac.at/{\dateiname}.html (Stand \today)
\fi

\end{document}


      