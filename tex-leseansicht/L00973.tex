%% latex-korrekturansicht-vorspann.tex
%% Vorspann für die Korrekturansicht.
%% Lädt die gemeinsame Datei latex-vorspann.tex mit gesetztem Schalter.

\newif\ifkorrekturansicht
\korrekturansichttrue

\input{../tex-inputs/latex-vorspann}


\section[Richard Beer-Hofmann an Arthur Schnitzler, 12. 9. 1899]{L00973 Richard Beer-Hofmann an Arthur Schnitzler, 12. 9. 1899}
\nopagebreak\mylabel{L00973v}
\rehead{ }\normalsize\beginnumbering\briefempfaengerindex{Schnitzler, Arthur@\textsc{Schnitzler, Arthur}!zzzBeer-Hofmann, Richard@\emph{von Richard Beer-Hofmann}!1899-09-121@{12. 9. 1899}|(be}
\toendnotes[C]{\smallbreak\pagebreak[2]}\Standort{CUL, Schnitzler, B 8.}
\physDesc{Brief, 1 Blatt, 2 Seiten, 1156 Zeichen
\newline{}Handschrift: schwarze Tinte, lateinische Kurrent
\newline{}Ordnung: mit Bleistift von unbekannter Hand nummeriert:
                                    »141« }
\buchAbdrucke{\weitereDrucke{Arthur Schnitzler, Richard Beer-Hofmann: \emph{Briefwechsel 1891–1931}. Wien, Zürich: \emph{Europaverlag} 1992, S. 136–137.} }\toendnotes[C]{\smallbreak}
\pstart
           \raggedleft{}{\pb}Vahrn\oindex{Vahrn@\textbf{Vahrn}, \emph{P.PPLA3}|pw}{ }12/IX 99\pend
           \vspace{0.5em}
\pstart
           Lieber Arthur! Ihre Karte gestern, heute Ihren Brief vom
                  9. erhalten. Ich habe ihn mehr errathen als gelesen; was heisst
                  \textcolor{gray}{durch allerlei.}{ }Hugos\pwindex{Hofmannsthal, Hugo von 1874-02-01 – 1929-07-15@\textsc{Hofmannsthal, Hugo von} (1874-02-01 – 1929-07-15), \emph{Schriftsteller/Schriftstellerin}|pw} Brief vom 7. daß er herko{\geminationm}en will habe ich gestern erhalten, und ihm telegrafirt
               er möge nur kommen. Ich arbeite täglich, und komme – wenn auch langsam vorwärts. In
               der »Zeit\orgindex{Zeit. Wiener Wochenschrift@Die Zeit. Wiener Wochenschrift|pw}« werden voraussichtlich nur die ersten
                  \label{K_L00973-1v}\edtext{2. Cap.\pwindex{Tod Georgs. Fragment@\emph{Der Tod Georgs. Fragment}|pwv}}{\lemma{\textnormal{\emph{2. Cap.}}}\Cendnote{\textnormal{Es erschien nur das gekürzte zweite Kapitel\pwindex{Tod Georgs@\emph{Der Tod Georgs}|pwkv} als Vorabdruck in vier
                  Teilen: Richard Beer-Hofmann\pwindex{Beer-Hofmann, Richard 1866-07-11 – 1945-09-26@\textsc{Beer-Hofmann, Richard} (1866-07-11 – 1945-09-26), \emph{Schriftsteller/Schriftstellerin}|pwk}: \emph{Der Tod Georgs. Fragment}\pwindex{Tod Georgs@\emph{Der Tod Georgs}|pwk}. In:
                     \emph{Die Zeit. Wiener Wochenschrift}\pwindex{Zeit. Wiener Wochenschrift@\emph{Die Zeit. Wiener Wochenschrift}|pwk}, Bd. 21: Nr. 266, 4. 11. 1899, S. 77–80.
                  \emph{(Fortsetzung)}\pwindex{Tod Georgs@\emph{Der Tod Georgs}|pwk}. In: Nr. 267, 11. 11. 1899, S. 95–96.
                  \emph{(Fortsetzung) [II]}\pwindex{Tod Georgs@\emph{Der Tod Georgs}|pwk}. In: Nr. 268, 18. 11. 1899, S. 111–118.
                  \emph{(Schluss)}\pwindex{Tod Georgs@\emph{Der Tod Georgs}|pwk}. In: Nr. 269, 25. 11. 1899, S. 127–128.}}}\label{K_L00973-1} erscheinen.
               Das Ganze würden sie in \uline{10} Fortsetz. tranchiren
               müssen, und das Buch könnte erst Mitte Dez. erscheinen. Das wäre zu langweilig. Wer
               wird also auf dem Titel figuriren? Schon entschieden? Ich {\pb}mache Sie aufmerksam: In München\oindex{Muenchen@\textbf{München}, \emph{P.PPLA}|pw} geht um 9.10 Nachts ein Zug
               ab, der um 4.36 Früh in Brixen\oindex{Brixen@\textbf{Brixen}, \emph{P.PPLA3}|pw} ist. Von da 20
               Minuten Wagen nach \strikeout{V}{ }Vahrn\oindex{Vahrn@\textbf{Vahrn}, \emph{P.PPLA3}|pw}. Außerdem ein N. S. Express, der um
                  9.55{ }\substVorne{}\textsuperscript{Früh}\substDazwischen{}Vorm\substHinten{} von München\oindex{Muenchen@\textbf{München}, \emph{P.PPLA}|pw} abgeht, um 3.02
                  Nachm. in Franzensfeste\oindex{Franzensfeste@\textbf{Franzensfeste}, \emph{A.ADM3}|pw} ist; \strikeout{von} (in Brixen\oindex{Brixen@\textbf{Brixen}, \emph{P.PPLA3}|pw}
               hält er nicht). Von Franzensfeste\oindex{Franzensfeste@\textbf{Franzensfeste}, \emph{A.ADM3}|pw} mit dem Wagen
               circa 9–10 Kilom. hieher. Es ist hier angenehm, ruhig, bei der table d’hôte nur Paula\pwindex{Beer-Hofmann, Paula 25.02.1879 – 30.10.1939@\textsc{Beer-Hofmann, Paula} (25.02.1879 – 30.10.1939)|pw} und ich inbegriffen \uline{4} Personen. Abends, wie bei Petter\oindex{Hotel und Pension Rudolfshoehe (Leopold Petter)@\textbf{Hotel und Pension Rudolfshöhe (Leopold Petter)}, \emph{Hotel (K.HTL)}|pw},
               an separaten Tischen. Lärchen und Edelkastanienwald. Gegenüber Weingelände.
               Vielleicht ko{\geminationm}en Sie? Man soll ja doch so spät als
               möglich nach Wien\oindex{Wien@\textbf{Wien}, \emph{A.ADM2}|pw}?\pend
           
\pstart
           Herzlichst{\\[\baselineskip]}Ihr{\\[\baselineskip]}\spacefill\mbox{Richard}\pend
           \leftskip=0em{}\selectlanguage{ngerman}\endnumbering\briefempfaengerindex{Schnitzler, Arthur@\textsc{Schnitzler, Arthur}!zzzBeer-Hofmann, Richard@\emph{von Richard Beer-Hofmann}!1899-09-121@{12. 9. 1899}|)be}\mylabel{L00973h}  \normalsize

\doendnotes{C}
\bigskip
\vfill

\clearpage

\footnotesize

\lohead{\textsc{register}}

% Definiere theindex-Environment komplett neu ohne reledmac
\makeatletter
\renewenvironment{theindex}{%
  \section*{\indexname}%
  \setlength{\parindent}{0pt}%
  \setlength{\parskip}{0pt plus 0.3pt}%
  \let\item\@idxitem
}{%
  \clearpage
}
\makeatother

\IfFileExists{\jobname-pw.ind}{\input{\jobname-pw.ind}}{}

\end{document}

      