%% latex-leseansicht-vorspann.tex
%% Vorspann für die Leseansicht.
%% Lädt die gemeinsame Datei latex-vorspann.tex mit nicht gesetztem Schalter.

\newif\ifkorrekturansicht
\korrekturansichtfalse

\input{../tex-inputs/latex-vorspann}

\begin{center}
            \textcolor{red}{ENTWURF, NICHT FERTIG KORRIGIERT}
                      \end{center}
            
         
         \renewcommand{\erwaehntePersonen}{Personen: Charlotte Pohl-Glas,  Reisner, Adele Reisner}
         \renewcommand{\erwaehnteInstitutionen}{Institutionen: Berliner Neueste Nachrichten, Münchener General-Anzeiger}
         \renewcommand{\erwaehnteOrte}{Orte: Hörlgasse, Wien}
         \renewcommand{\erwaehnteWerke}{Werke: Tagebuch}
               \section[Felix Salten an Arthur Schnitzler, {[}zwischen 4. und 13. 9.? 1894{]}]{ Felix Salten an Arthur Schnitzler, {[}zwischen 4. und
               13. 9.? 1894{]}}\nopagebreak\mylabel{v}\rehead{ }\begin{ledgroupsized}[t]{13cm}\normalsize\beginnumbering \toendnotes[C]{\smallbreak\pagebreak[2]} \Standort{CUL, Schnitzler, B 89, A 1.}
\physDesc{Visitenkarte
\newline{}Handschrift: Bleistift, lateinische Kurrent
\newline{}Schnitzler: mit Bleistift datiert: »94« }\toendnotes[C]{\smallbreak}\pstart
           \noindent{}{\pb}\textcolor{gray}{\textbf{FELIX SALTEN}}\pend
           \pstart
           \textcolor{gray}{\textbf{WIEN\oindex{Wien@\textbf{Wien}|pw},}}\hfill \textcolor{gray}{\textbf{»Berliner Neueste
                        Nachrichten\orgindex{Berliner Neueste Nachrichten@Berliner Neueste Nachrichten|pw}.«}}\pend
           \pstart
           \textcolor{gray}{\textbf{IX., Hörlgasse 16\oindex{Hoerlgasse@\textbf{Hörlgasse}|pw}.}}\hfill \textcolor{gray}{\textbf{ »Münchener
                        General-Anzeiger\orgindex{Muenchener General-Anzeiger@Münchener General-Anzeiger|pw}.«}}\pend
           \pstart
           {\pb}Lieber Frd, ich habe jetzt \label{K_L03134-1v}\edtext{Rendezvous\pwindex{Pohl-Glas, Charlotte 1873-01-01 – 1944-02-15@\textsc{Pohl-Glas, Charlotte} (1873-01-01 – 1944-02-15), \emph{Schriftstellerin, Politikerin, Sozialistin}|pwuv}}{\lemma{\textnormal{\emph{Rendezvous}}}\Cendnote{\textnormal{Da diese Visitenkarte Salten\pwindex{Salten, Felix 06.09.1869 – 08.10.1945@\textsc{Salten, Felix} (06.09.1869 – 08.10.1945), \emph{Schriftsteller, Journalist}|pwk}s nur für den Zeitraum [6. 9. 1894] bis zum 15. 9. 189[4?] belegt ist, ist es wahrscheinlich, dass
                  auch diese Karte nach Schnitzler\pwindex{Schnitzler, Arthur 15.05.1862 – 21.10.1931@\textsc{Schnitzler, Arthur} (15.05.1862 – 21.10.1931), \emph{Schriftsteller, Mediziner}|pwk}s Heimkehr
                  nach Wien\oindex{Wien@\textbf{Wien}|pwk} im September 1894
                  übermittelt wurde. Nimmt man zudem an, dass ein »Rendezvous« Salten\pwindex{Salten, Felix 06.09.1869 – 08.10.1945@\textsc{Salten, Felix} (06.09.1869 – 08.10.1945), \emph{Schriftsteller, Journalist}|pwk}s mit Lotte
                     Glas\pwindex{Pohl-Glas, Charlotte 1873-01-01 – 1944-02-15@\textsc{Pohl-Glas, Charlotte} (1873-01-01 – 1944-02-15), \emph{Schriftstellerin, Politikerin, Sozialistin}|pwk} gemeint ist, so schränkt sich der Zeitraum weiter ein, da diese am
                     [11. 9. 1894] bereits
                  ihre Haftstrafe angetreten hatte.}}}\label{K_L03134-1h} und kann deshalb nicht ko{\geminationm}en. Es ist möglich, dass wir, dh. ich u. »sie\pwindex{Pohl-Glas, Charlotte 1873-01-01 – 1944-02-15@\textsc{Pohl-Glas, Charlotte} (1873-01-01 – 1944-02-15), \emph{Schriftstellerin, Politikerin, Sozialistin}|pwuv}« mit der
                  \label{K_L03134-22v}\edtext{Reisner\pwindex{Reisner @\textsc{Reisner}|pw}}{\lemma{\textnormal{\emph{Reisner}}}\Cendnote{\textnormal{Obzwar die Person
                  bislang nicht genauer identifiziert werden konnte, ist anzunehmen, dass damit
                  nicht die im Register des \emph{Tagebuch}\pwindex{Schnitzler, Arthur 15.05.1862 – 21.10.1931@\textsc{Schnitzler, Arthur} (15.05.1862 – 21.10.1931), \emph{Schriftsteller, Mediziner}!Tagebuch1981 – 2000@\strich\emph{Tagebuch} {[}1981 – 2000{]}|pwk}s
                  angeführte Adele Reisner\pwindex{Reisner, Adele *~17.12.1882@\textsc{Reisner, Adele} (*~17.12.1882), \emph{Tänzerin}|pwk} gemeint ist, da
                  diese zu diesem Zeitpunkt noch nicht einmal 12 Jahre alt war. Wahrscheinlicher ist, dass
                  sich auch die Einträge zu Adele Reisner\pwindex{Reisner, Adele *~17.12.1882@\textsc{Reisner, Adele} (*~17.12.1882), \emph{Tänzerin}|pwk} im \emph{Tagebuch}\pwindex{Schnitzler, Arthur 15.05.1862 – 21.10.1931@\textsc{Schnitzler, Arthur} (15.05.1862 – 21.10.1931), \emph{Schriftsteller, Mediziner}!Tagebuch1981 – 2000@\strich\emph{Tagebuch} {[}1981 – 2000{]}|pwk} auf
                     die vorliegende Person beziehen.}}}\label{K_L03134-22h} zusammen
               soupiren, für diesen Fall telephonire ich Sie an, oder bitte laßen Sie mir sagen, wo
               ich Sie zwischen ½ 8 u. ½ 9 treffen kann. Ohne dass Sie
               sich binden, natürlich.\pend
           \pstart Herzlichst \spacefill\mbox{Salten}\pend{}
         
         \endnumbering\mylabel{h}\end{ledgroupsized}\begin{anhang}\end{anhang}\newcommand{\dateiname}{L03134}\newcommand{\titel}{Felix Salten an Arthur Schnitzler, [zwischen 4. und 13. 9.? 1894]}\newcommand{\editorInnen}{Martin Anton Müller und Laura Untner}%% latex-leseansicht-abspann.tex
%% Abspann für die Leseansicht.
%% Der Schalter \ifkorrekturansicht ist bereits durch den Vorspann gesetzt.

%% latex-abspann.tex
%% Gemeinsamer Abspann für Korrekturansicht und Leseansicht.
%% Setzt den Schalter \ifkorrekturansicht voraus (gesetzt in den
%% einbindenden Dateien latex-korrekturansicht-abspann.tex bzw.
%% latex-leseansicht-abspann.tex).
%% ---------------------------------------------------------------

\normalsize

% Das esempio-Environment wird nur in der Leseansicht benötigt
\ifkorrekturansicht\else
\newenvironment{esempio}[3]%
{
    \vspace{1.5ex}
    \rlap{\underline{#1}}
    \par
    \setlength{\parindent}{0cm}
    \nopagebreak
    \leftskip=#2cm
    \rightskip=#3cm
}
{
    \par
}
\fi

\doendnotes{C}
\bigskip
\vfill

\clearpage

\footnotesize

\ifkorrekturansicht
  \lohead{\textsc{register}}
\fi

% theindex-Environment neu definieren ohne reledmac
\makeatletter
\renewenvironment{theindex}{%
  \ifkorrekturansicht
    \section*{\indexname}%
  \else
    \subsubsection*{Index der erwähnten Entitäten}%
  \fi
  \setlength{\parindent}{0pt}%
  \setlength{\parskip}{0pt plus 0.3pt}%
  \let\item\@idxitem
}{%
  \ifkorrekturansicht\clearpage\fi
}
\makeatother

\IfFileExists{\jobname-pw.ind}{\input{\jobname-pw.ind}}{}

% Quellenangabe nur in der Leseansicht
\ifkorrekturansicht\else
% Fallback-Definitionen, falls die .tex-Datei \titel etc. nicht gesetzt hat
\providecommand{\titel}{}
\providecommand{\editorInnen}{}
\providecommand{\dateiname}{\jobname}

\vspace{3cm}

\vfill

\footnotesize
\textsc{Quelle}: \titel. Herausgegeben von {\editorInnen}. In: \emph{Arthur Schnitzler: Briefwechsel mit Autorinnen und Autoren}.
 Digitale Edition, https://schnitzler-briefe.acdh.oeaw.ac.at/{\dateiname}.html (Stand \today)
\fi

\end{document}


      