%% latex-leseansicht-vorspann.tex
%% Vorspann für die Leseansicht.
%% Lädt die gemeinsame Datei latex-vorspann.tex mit nicht gesetztem Schalter.

\newif\ifkorrekturansicht
\korrekturansichtfalse

\input{../tex-inputs/latex-vorspann}


\section[Felix Salten an Arthur Schnitzler, {[}zwischen 7. und 12. 9.? 1894{]}]{L03134 Felix Salten an Arthur Schnitzler, {[}zwischen 7. und 12. 9.? 1894{]}}
\nopagebreak\mylabel{L03134v}
\rehead{ }\normalsize\beginnumbering\briefempfaengerindex{Schnitzler, Arthur@\textsc{Schnitzler, Arthur}!zzzSalten, Felix@\emph{von Felix Salten}!1894-09-122@{{[}zwischen 7. und 12. 9.? 1894{]}}|(be}
\toendnotes[C]{\smallbreak\pagebreak[2]}
\correspDesc{Versand  durch Felix Salten im Zeitraum [zwischen 7. und 12. 9.? 1894] in Wien
\newline{}Erhalt  durch Arthur Schnitzler im Zeitraum [zwischen 7. und 12. 9.? 1894] in Wien}\toendnotes[C]{\smallbreak}
\Standort{CUL, Schnitzler, B 89, A 1.}
\physDesc{Visitenkarte, 307 Zeichen
\newline{}Handschrift: Bleistift, lateinische Kurrent
\newline{}Schnitzler: mit Bleistift datiert: »9\textcolor{gray}{4}« 
\newline{}Ordnung: mit Bleistift von unbekannter Hand nummeriert: »36a« }\toendnotes[C]{\smallbreak}
\pstart
           \centering{}{\pb}\textcolor{gray}{\textbf{FELIX SALTEN}}\pend
           
\pstart
           \textcolor{gray}{\textbf{WIEN\oindex{Wien@\textbf{Wien}, \emph{Verwaltungsgebiet}|pw},}}\hfill \textcolor{gray}{\textbf{»Berliner Neueste
                           Nachrichten\orgindex{Berliner Neueste Nachrichten@Berliner Neueste Nachrichten|pw}.«}}\pend
           
\pstart
           \textcolor{gray}{\textbf{IX., Hörlgasse 16\oindex{Wien@\textbf{Wien}!IX., Alsergrund@\textbf{IX., Alsergrund}!Hörlgasse 16@\textbf{Hörlgasse 16}, \emph{Wohngebäude}|pw}.}}\hfill \textcolor{gray}{\textbf{»Münchener
                           General-Anzeiger\orgindex{Münchener General-Anzeiger@Münchener General-Anzeiger|pw}.«}}\pend
           \vspace{0.5em}
\pstart
           {\pb}Lieber Frd, ich habe jetzt \label{K_L03134-1v}\edtext{Rendezvous\pwindex{Pohl-Glas, Charlotte 1.\,1.\,1873 Wien – 15.\,2.\,1944 Zürich@\textsc{Pohl-Glas, Charlotte} (1.\,1.\,1873 Wien – 15.\,2.\,1944 Zürich), \emph{Schriftstellerin, Politikerin, Sozialistin}|pwuv}}{\lemma{\textnormal{\emph{Rendezvous}}}\Cendnote{\textnormal{Da diese Visitenkarte Saltens\pwindex{Salten, Felix 6.\,9.\,1869 Budapest – 8.\,10.\,1945 Zürich@\textsc{Salten, Felix} (6.\,9.\,1869 Budapest – 8.\,10.\,1945 Zürich), \emph{Schriftsteller, Journalist, Chefredakteur}|pwk} nur für den Zeitraum vom XXXX Auszeichnungsfehler: Dokument L03144 nicht gefunden bis zum XXXX Auszeichnungsfehler: Dokument L03147 nicht gefunden belegt ist, ist
                  es wahrscheinlich, dass auch diese Karte nach Schnitzlers Heimkehr nach Wien\oindex{Wien@\textbf{Wien}, \emph{Verwaltungsgebiet}|pwk} im
                     September 1894 übermittelt wurde, wobei Salten\pwindex{Salten, Felix 6.\,9.\,1869 Budapest – 8.\,10.\,1945 Zürich@\textsc{Salten, Felix} (6.\,9.\,1869 Budapest – 8.\,10.\,1945 Zürich), \emph{Schriftsteller, Journalist, Chefredakteur}|pwk} bis zum XXXX Auszeichnungsfehler: Dokument L03144 nicht gefunden nicht von Schnitzlers Rückkehr gewusst haben dürfte. Nimmt man zudem
                  an, dass ein »Rendezvous« Saltens\pwindex{Salten, Felix 6.\,9.\,1869 Budapest – 8.\,10.\,1945 Zürich@\textsc{Salten, Felix} (6.\,9.\,1869 Budapest – 8.\,10.\,1945 Zürich), \emph{Schriftsteller, Journalist, Chefredakteur}|pwk} mit Lotte Glas\pwindex{Pohl-Glas, Charlotte 1.\,1.\,1873 Wien – 15.\,2.\,1944 Zürich@\textsc{Pohl-Glas, Charlotte} (1.\,1.\,1873 Wien – 15.\,2.\,1944 Zürich), \emph{Schriftstellerin, Politikerin, Sozialistin}|pwk} gemeint ist,
                  so schränkt sich der Zeitraum weiter ein, denn diese trat am XXXX Auszeichnungsfehler: Dokument L03145 nicht gefunden ihre Haftstrafe
                  an.}}}\label{K_L03134-1} und kann deshalb nicht ko{\geminationm}en. Es ist
               möglich, dass wir, dh. ich u. »sie\pwindex{Pohl-Glas, Charlotte 1.\,1.\,1873 Wien – 15.\,2.\,1944 Zürich@\textsc{Pohl-Glas, Charlotte} (1.\,1.\,1873 Wien – 15.\,2.\,1944 Zürich), \emph{Schriftstellerin, Politikerin, Sozialistin}|pwuv}« mit der \label{K_L03134-2v}\edtext{Reisner\pwindex{Reisner @\textsc{Reisner}|pw}}{\lemma{\textnormal{\emph{Reisner}}}\Cendnote{\textnormal{Obzwar die Person\pwindex{Reisner @\textsc{Reisner}|pwkv} bislang nicht genauer
                  identifiziert werden konnte, ist anzunehmen, dass damit nicht die im Register des
                     \emph{Tagebuchs}\pwindex{Schnitzler, Arthur 15.\,5.\,1862 Wien – 21.\,10.\,1931 ebd.@\textsc{Schnitzler, Arthur} (15.\,5.\,1862 Wien – 21.\,10.\,1931 ebd.), \emph{Schriftsteller, Mediziner}!Tagebuch@\strich\emph{Tagebuch}|pwk} angeführte Adele Reisner\pwindex{Reisner, Adele *~17.\,12.\,1882 Wien@\textsc{Reisner, Adele} (*~17.\,12.\,1882 Wien), \emph{Tänzerin}|pwk} gemeint ist, da diese zu diesem Zeitpunkt
                  noch nicht einmal 12 Jahre alt war. Wahrscheinlicher ist, dass sich auch die
                  Einträge zu Adele Reisner\pwindex{Reisner, Adele *~17.\,12.\,1882 Wien@\textsc{Reisner, Adele} (*~17.\,12.\,1882 Wien), \emph{Tänzerin}|pwk} im \emph{Tagebuch}\pwindex{Schnitzler, Arthur 15.\,5.\,1862 Wien – 21.\,10.\,1931 ebd.@\textsc{Schnitzler, Arthur} (15.\,5.\,1862 Wien – 21.\,10.\,1931 ebd.), \emph{Schriftsteller, Mediziner}!Tagebuch@\strich\emph{Tagebuch}|pwk} auf die vorliegende Person\pwindex{Reisner @\textsc{Reisner}|pwkv} beziehen.}}}\label{K_L03134-2} zusammen
               soupiren, für diesen Fall telephonire ich Sie an, oder bitte laßen Sie mir sagen, wo
               ich Sie zwischen ½ 8 u. ½ 9 treffen kann. Ohne dass Sie
               sich binden, natürlich.\pend
           
\pstart
           Herzlichst {\\[\baselineskip]}\spacefill\mbox{Salten}\pend
           \leftskip=0em{}\selectlanguage{ngerman}\endnumbering\briefempfaengerindex{Schnitzler, Arthur@\textsc{Schnitzler, Arthur}!zzzSalten, Felix@\emph{von Felix Salten}!1894-09-072@{{[}zwischen 7. und 12. 9.? 1894{]}}|)be}\mylabel{L03134h}  \newcommand{\dateiname}{L03134}\newcommand{\titel}{Felix Salten an Arthur Schnitzler, [zwischen 7. und 12. 9.? 1894]}\newcommand{\editorInnen}{Martin Anton Müller und Laura Untner}%% latex-leseansicht-abspann.tex
%% Abspann für die Leseansicht.
%% Der Schalter \ifkorrekturansicht ist bereits durch den Vorspann gesetzt.

%% latex-abspann.tex
%% Gemeinsamer Abspann für Korrekturansicht und Leseansicht.
%% Setzt den Schalter \ifkorrekturansicht voraus (gesetzt in den
%% einbindenden Dateien latex-korrekturansicht-abspann.tex bzw.
%% latex-leseansicht-abspann.tex).
%% ---------------------------------------------------------------

\normalsize

% Das esempio-Environment wird nur in der Leseansicht benötigt
\ifkorrekturansicht\else
\newenvironment{esempio}[3]%
{
    \vspace{1.5ex}
    \rlap{\underline{#1}}
    \par
    \setlength{\parindent}{0cm}
    \nopagebreak
    \leftskip=#2cm
    \rightskip=#3cm
}
{
    \par
}
\fi

\doendnotes{C}
\bigskip
\vfill

\clearpage

\footnotesize

\ifkorrekturansicht
  \lohead{\textsc{register}}
\fi

% theindex-Environment neu definieren ohne reledmac
\makeatletter
\renewenvironment{theindex}{%
  \ifkorrekturansicht
    \section*{\indexname}%
  \else
    \subsubsection*{Index der erwähnten Entitäten}%
  \fi
  \setlength{\parindent}{0pt}%
  \setlength{\parskip}{0pt plus 0.3pt}%
  \let\item\@idxitem
}{%
  \ifkorrekturansicht\clearpage\fi
}
\makeatother

\IfFileExists{\jobname-pw.ind}{\input{\jobname-pw.ind}}{}

% Quellenangabe nur in der Leseansicht
\ifkorrekturansicht\else
% Fallback-Definitionen, falls die .tex-Datei \titel etc. nicht gesetzt hat
\providecommand{\titel}{}
\providecommand{\editorInnen}{}
\providecommand{\dateiname}{\jobname}

\vspace{3cm}

\vfill

\footnotesize
\textsc{Quelle}: \titel. Herausgegeben von {\editorInnen}. In: \emph{Arthur Schnitzler: Briefwechsel mit Autorinnen und Autoren}.
 Digitale Edition, https://schnitzler-briefe.acdh.oeaw.ac.at/{\dateiname}.html (Stand \today)
\fi

\end{document}


