%% latex-korrekturansicht-vorspann.tex
%% Vorspann für die Korrekturansicht.
%% Lädt die gemeinsame Datei latex-vorspann.tex mit gesetztem Schalter.

\newif\ifkorrekturansicht
\korrekturansichttrue

\input{../tex-inputs/latex-vorspann}


\section[Felix Salten an Arthur Schnitzler, {[}zwischen 7. und 12. 9.? 1894{]}]{L03134 Felix Salten an Arthur Schnitzler,
               {[}zwischen 7. und 12. 9.? 1894{]}}
\nopagebreak\mylabel{L03134v}
\rehead{ }\normalsize\beginnumbering\briefempfaengerindex{Schnitzler, Arthur@\textsc{Schnitzler, Arthur}!zzzSalten, Felix@\emph{von Felix Salten}!1894-09-122@{{[}zwischen 7. und 12. 9.? 1894{]}}|(be}
\toendnotes[C]{\smallbreak\pagebreak[2]}\Standort{CUL, Schnitzler, B 89, A 1.}
\physDesc{Visitenkarte, 307 Zeichen
\newline{}Handschrift: Bleistift, lateinische Kurrent
\newline{}Schnitzler: mit Bleistift datiert: »9\textcolor{gray}{4}« 
\newline{}Ordnung: mit Bleistift von unbekannter Hand nummeriert: »36a« }\toendnotes[C]{\smallbreak}
\pstart
           \centering{}{\pb}\textcolor{gray}{\textbf{FELIX SALTEN}}\pend
           
\pstart
           \textcolor{gray}{\textbf{WIEN\oindex{Wien@\textbf{Wien}, \emph{A.ADM2}|pw},}}\hfill \textcolor{gray}{\textbf{»Berliner Neueste
                           Nachrichten\orgindex{Berliner Neueste Nachrichten@Berliner Neueste Nachrichten|pw}.«}}\pend
           
\pstart
           \textcolor{gray}{\textbf{IX., Hörlgasse 16\oindex{Hoerlgasse 16@\textbf{Hörlgasse 16}, \emph{Wohngebäude (K.WHS)}|pw}.}}\hfill \textcolor{gray}{\textbf{ »Münchener
                           General-Anzeiger\orgindex{Muenchener General-Anzeiger@Münchener General-Anzeiger|pw}.«}}\pend
           \vspace{0.5em}
\pstart
           {\pb}Lieber Frd, ich habe jetzt \label{K_L03134-1v}\edtext{Rendezvous\pwindex{Pohl-Glas, Charlotte 1873-01-01 – 1944-02-15@\textsc{Pohl-Glas, Charlotte} (1873-01-01 – 1944-02-15), \emph{Schriftsteller/Schriftstellerin, Politiker/Politikerin, Sozialist/Sozialistin}|pwuv}}{\lemma{\textnormal{\emph{Rendezvous}}}\Cendnote{\textnormal{Da diese Visitenkarte Saltens\pwindex{Salten, Felix 06.09.1869 – 08.10.1945@\textsc{Salten, Felix} (06.09.1869 – 08.10.1945), \emph{Schriftsteller/Schriftstellerin, Journalist/Journalistin, Chefredakteur/Chefredakteurin}|pwk} nur für den Zeitraum vom [6. 9. 1894] bis zum 15. 9. 189[4?] belegt ist, ist
                  es wahrscheinlich, dass auch diese Karte nach Schnitzlers Heimkehr nach Wien\oindex{Wien@\textbf{Wien}, \emph{A.ADM2}|pwk} im
                     September 1894 übermittelt wurde, wobei Salten\pwindex{Salten, Felix 06.09.1869 – 08.10.1945@\textsc{Salten, Felix} (06.09.1869 – 08.10.1945), \emph{Schriftsteller/Schriftstellerin, Journalist/Journalistin, Chefredakteur/Chefredakteurin}|pwk} bis zum [6. 9. 1894] nicht von Schnitzlers Rückkehr gewusst haben dürfte. Nimmt man zudem
                  an, dass ein »Rendezvous« Saltens\pwindex{Salten, Felix 06.09.1869 – 08.10.1945@\textsc{Salten, Felix} (06.09.1869 – 08.10.1945), \emph{Schriftsteller/Schriftstellerin, Journalist/Journalistin, Chefredakteur/Chefredakteurin}|pwk} mit Lotte Glas\pwindex{Pohl-Glas, Charlotte 1873-01-01 – 1944-02-15@\textsc{Pohl-Glas, Charlotte} (1873-01-01 – 1944-02-15), \emph{Schriftsteller/Schriftstellerin, Politiker/Politikerin, Sozialist/Sozialistin}|pwk} gemeint ist,
                  so schränkt sich der Zeitraum weiter ein, denn diese trat am [11. 9. 1894] ihre Haftstrafe
                  an.}}}\label{K_L03134-1} und kann deshalb nicht ko{\geminationm}en. Es ist
               möglich, dass wir, dh. ich u. »sie\pwindex{Pohl-Glas, Charlotte 1873-01-01 – 1944-02-15@\textsc{Pohl-Glas, Charlotte} (1873-01-01 – 1944-02-15), \emph{Schriftsteller/Schriftstellerin, Politiker/Politikerin, Sozialist/Sozialistin}|pwuv}« mit der \label{K_L03134-2v}\edtext{Reisner\pwindex{Reisner @\textsc{Reisner}|pw}}{\lemma{\textnormal{\emph{Reisner}}}\Cendnote{\textnormal{Obzwar die Person\pwindex{Reisner @\textsc{Reisner}|pwkv} bislang nicht genauer
                  identifiziert werden konnte, ist anzunehmen, dass damit nicht die im Register des
                     \emph{Tagebuchs}\pwindex{Tagebuch@\emph{Tagebuch}|pwk} angeführte Adele Reisner\pwindex{Reisner, Adele *~17.12.1882@\textsc{Reisner, Adele} (*~17.12.1882), \emph{Tänzer/Tänzerin}|pwk} gemeint ist, da diese zu diesem Zeitpunkt
                  noch nicht einmal 12 Jahre alt war. Wahrscheinlicher ist, dass sich auch die
                  Einträge zu Adele Reisner\pwindex{Reisner, Adele *~17.12.1882@\textsc{Reisner, Adele} (*~17.12.1882), \emph{Tänzer/Tänzerin}|pwk} im \emph{Tagebuch}\pwindex{Tagebuch@\emph{Tagebuch}|pwk} auf die vorliegende Person\pwindex{Reisner @\textsc{Reisner}|pwkv} beziehen.}}}\label{K_L03134-2} zusammen
               soupiren, für diesen Fall telephonire ich Sie an, oder bitte laßen Sie mir sagen, wo
               ich Sie zwischen ½ 8 u. ½ 9 treffen kann. Ohne dass Sie
               sich binden, natürlich.\pend
           
\pstart
           Herzlichst {\\[\baselineskip]}\spacefill\mbox{Salten}\pend
           \leftskip=0em{}\selectlanguage{ngerman}\endnumbering\briefempfaengerindex{Schnitzler, Arthur@\textsc{Schnitzler, Arthur}!zzzSalten, Felix@\emph{von Felix Salten}!1894-09-072@{{[}zwischen 7. und 12. 9.? 1894{]}}|)be}\mylabel{L03134h}  \normalsize

\doendnotes{C}
\bigskip
\vfill

\clearpage

\footnotesize

\lohead{\textsc{register}}

% Definiere theindex-Environment komplett neu ohne reledmac
\makeatletter
\renewenvironment{theindex}{%
  \section*{\indexname}%
  \setlength{\parindent}{0pt}%
  \setlength{\parskip}{0pt plus 0.3pt}%
  \let\item\@idxitem
}{%
  \clearpage
}
\makeatother

\IfFileExists{\jobname-pw.ind}{\input{\jobname-pw.ind}}{}

\end{document}

      