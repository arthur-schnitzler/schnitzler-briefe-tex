%% latex-korrekturansicht-vorspann.tex
%% Vorspann für die Korrekturansicht.
%% Lädt die gemeinsame Datei latex-vorspann.tex mit gesetztem Schalter.

\newif\ifkorrekturansicht
\korrekturansichttrue

\input{../tex-inputs/latex-vorspann}


\section[ Paul Goldmann an Arthur Schnitzler und Olga Gussmann, 7. 7. {[}1901{]}]{L03072 Paul Goldmann an Arthur Schnitzler und Olga
               Gussmann, 7. 7. {[}1901{]}}
\nopagebreak\mylabel{L03072v}
\rehead{ }\normalsize\beginnumbering\briefempfaengerindex{Schnitzler, Olga@\textsc{Schnitzler, Olga}!zzzGoldmann, Paul@\emph{von Paul Goldmann}!1901-07-071@{7. 7. {[}1901{]}}|(be}\briefempfaengerindex{Schnitzler, Arthur@\textsc{Schnitzler, Arthur}!zzzGoldmann, Paul@\emph{von Paul Goldmann}!1901-07-071@{7. 7. {[}1901{]}}|(be}
\toendnotes[C]{\smallbreak\pagebreak[2]}\Standort{DLA, A:Schnitzler, HS.NZ85.1.3171.}
\physDesc{Brief, 1 Blatt, 4 Seiten, 1566 Zeichen
\newline{}Handschrift: blaue Tinte, deutsche Kurrent
\newline{}Schnitzler: mit rotem Buntstift zwei Unterstreichungen }\toendnotes[C]{\smallbreak}
\pstart
           \raggedleft{}{\pb}\textcolor{gray}{\textbf{DESSAUERSTRASSE 19}}\oindex{Dessauer Strasse@\textbf{Dessauer Straße}, \emph{Straße (K.STR)}|pw}\pend
           
\pstart
           Berlin\oindex{Berlin@\textbf{Berlin}, \emph{P.PPLC}|pw}, 7. Juli.\pend
           
\pstart\center{}Mein lieber Freund,\pend\vspace{0.5em}
\pstart
           Endlich zieht Vernunft in Eure Reiſepläne ein, und ich freue mich ſehr darüber und
               über die Ausſicht, Euch doch zu \label{K_L03072-1v}\edtext{ſehen}{\lemma{\textnormal{\emph{ſehen}}}\Cendnote{\textnormal{Siehe Paul Goldmann an Arthur Schnitzler, 26. 4. [1901].
               }}}\label{K_L03072-1}. Ich gehe ſo zwiſchen dem 20. u. 25. von hier fort, bleibe einen oder zwei Tage in Dresden\oindex{Dresden@\textbf{Dresden}, \emph{P.PPLA}|pw} und Wien\oindex{Wien@\textbf{Wien}, \emph{A.ADM2}|pw}, gehe dann meinetwegen nach dem \label{K_L03072-2v}\edtext{Wörtherſee\oindex{Woerthersee@\textbf{Wörthersee}, \emph{H.LK}|pw}}{\lemma{\textnormal{\emph{Wörtherſee}}}\Cendnote{\textnormal{Siehe Paul Goldmann an Arthur Schnitzler, 13. 5. [1901].
               }}}\label{K_L03072-2} und komme von da aus ſehr gern zu Euch. \textsc{St. Ulrich}\oindex{Urtijei@\textbf{Urtijëi}, \emph{P.PPLA3}|pw} im {\pb}Grödener Thal\oindex{Val Gardena@\textbf{Val Gardena}, \emph{T.VAL}|pw} würde mir beſonders gefallen.
               Denn ſeit Jahren wünſche ich, das Grödener Thal\oindex{Val Gardena@\textbf{Val Gardena}, \emph{T.VAL}|pw}
               kennen zu lernen. Bitte, halt’ alſo dieſes Projekt feſt. Vielleicht können wir dann
               auch von dort aus ein paar Tage in die Berge ſteigen.\pend
           
\pstart
           Ich höre, daß die »Zeit\orgindex{Zeit@Die Zeit|pw}\orgindex{Zeit. Wiener Wochenschrift@Die Zeit. Wiener Wochenschrift|pw}« von 1. Oktober ab \label{K_L03072-3v}\edtext{Tagesblatt}{\lemma{\textnormal{\emph{Tagesblatt}}}\Cendnote{\textnormal{\emph{Die Zeit}\pwindex{Zeit@\emph{Die Zeit}|pwk} wurde erst ab dem 27. 9. 1902 (bis 31. 8. 1919) als Tageszeitung von Heinrich Kanner\pwindex{Kanner, Heinrich 09.11.1864 – 15.02.1930@\textsc{Kanner, Heinrich} (09.11.1864 – 15.02.1930), \emph{Herausgeber/Herausgeberin, Publizist/Publizistin}|pwk} und Isidor Singer\pwindex{Singer, Isidor 16.01.1857 – 08.12.1927@\textsc{Singer, Isidor} (16.01.1857 – 08.12.1927), \emph{Journalist/Journalistin, Herausgeber/Herausgeberin, Soziologe/Soziologin}|pwk}
                  herausgegeben. Bis zum 29. 10. 1904 erschien \emph{Die Zeit}\pwindex{Zeit. Wiener Wochenschrift@\emph{Die Zeit. Wiener Wochenschrift}|pwk} parallel als Wochenschrift. Die \emph{Neue Freie Presse}\pwindex{Neue Freie Presse@\emph{Neue Freie Presse}|pwk} ersetzte sie nicht.}}}\label{K_L03072-3}
               wird mit 1 Million \textsc{Kronen} Capital. Weißt Du etwas davon?
               Kommt es dazu, ſo bedeutet {\pb}das, nach meiner
               Überzeugung, den Anfang vom Ende der N. Fr. Pr.\orgindex{Neue Freie Presse@Neue Freie Presse|pw}
               So ſetzt auch \textsc{Dr. Kanner\pwindex{Kanner, Heinrich 09.11.1864 – 15.02.1930@\textsc{Kanner, Heinrich} (09.11.1864 – 15.02.1930), \emph{Herausgeber/Herausgeberin, Publizist/Publizistin}|pw}} ſeinen Lebensplan durch. Nur ich, – ich allein bleibe auf der Strecke. Es iſt
               martervoll!\pend
           
\pstart
           Viele treue Grüße! {\\[\baselineskip]}Dein {\\[\baselineskip]}\spacefill\mbox{Paul Goldmann.}\pend
           \leftskip=0em{}\selectlanguage{ngerman}\vspace{1em}{\vspace{1\baselineskip}}
\pstart
           Liebes Fräulein \textsc{Olga}, Ich
               danke Ihnen für Ihren lieben und guten Brief. Jetzt, bitte, ſetzen Sie noch durch,
               daß wir ins Grödener Thal\oindex{Val Gardena@\textbf{Val Gardena}, \emph{T.VAL}|pw}{ }{\pb}gehen. Ich möchte ſehr gern dorthin, was für \textsc{Arthur} immerhin einen ausreichenden Grund bilden könnte,
                  \strikeout{\textcolor{gray}{in}d\textcolor{gray}{e}s} ſich für einen anderen Ort zu
               entſchließen. Auch ich möchte, gleich Ihnen, ſtillſitzen und Ruhe, Ruhe haben. Über
                  \textsc{Kerr\pwindex{Kerr, Alfred 25.12.1867 – 12.10.1948@\textsc{Kerr, Alfred} (25.12.1867 – 12.10.1948), \emph{Schriftsteller/Schriftstellerin, Kritiker/Kritikerin}|pw}} ſprechen wir mündlich. Er wird übrigens nur nachkommen und nicht mitkommen
               können. Ihrem lieben Schweſterchen\pwindex{Steinrueck, Elisabeth 19.11.1885 – 07.04.1920@\textsc{Steinrück, Elisabeth} (19.11.1885 – 07.04.1920)|pwv} wünſche ich gute Beſſerung. Haben Sie keine Sorgen! Wenn ſie
                  \textsc{Arthurs} Behandlung bisher ausgehalten hat, wird ſie auch
               davonkommen. Sie iſt eine widerſtandsfähige Natur.\pend
           
\pstart
           Herzlichſt Ihr {\\[\baselineskip]}\spacefill\mbox{Dr. Paul Goldmann.}\pend
           \leftskip=0em{}\selectlanguage{ngerman}\endnumbering\briefempfaengerindex{Schnitzler, Olga@\textsc{Schnitzler, Olga}!zzzGoldmann, Paul@\emph{von Paul Goldmann}!1901-07-071@{7. 7. {[}1901{]}}|)be}\briefempfaengerindex{Schnitzler, Arthur@\textsc{Schnitzler, Arthur}!zzzGoldmann, Paul@\emph{von Paul Goldmann}!1901-07-071@{7. 7. {[}1901{]}}|)be}\mylabel{L03072h}  \normalsize

\doendnotes{C}
\bigskip
\vfill

\clearpage

\footnotesize

\lohead{\textsc{register}}

% Definiere theindex-Environment komplett neu ohne reledmac
\makeatletter
\renewenvironment{theindex}{%
  \section*{\indexname}%
  \setlength{\parindent}{0pt}%
  \setlength{\parskip}{0pt plus 0.3pt}%
  \let\item\@idxitem
}{%
  \clearpage
}
\makeatother

\IfFileExists{\jobname-pw.ind}{\input{\jobname-pw.ind}}{}

\end{document}

      