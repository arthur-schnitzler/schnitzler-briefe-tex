%% latex-leseansicht-vorspann.tex
%% Vorspann für die Leseansicht.
%% Lädt die gemeinsame Datei latex-vorspann.tex mit nicht gesetztem Schalter.

\newif\ifkorrekturansicht
\korrekturansichtfalse

\input{../tex-inputs/latex-vorspann}


         
         \renewcommand{\erwaehntePersonen}{Personen: Paul Goldmann, Heinrich Kanner, Alfred Kerr, Olga Schnitzler, Isidor Singer, Elisabeth Steinrück}
         \renewcommand{\erwaehnteInstitutionen}{Institutionen: Die Zeit, Die Zeit. Wiener Wochenschrift, Neue Freie Presse}
         \renewcommand{\erwaehnteOrte}{Orte: Berlin, Dessauer Straße, Dresden, St. Anton am Arlberg, Urtijëi, Val Gardena, Wien, Wörthersee}
         \renewcommand{\erwaehnteWerke}{Werke: Die Zeit, Die Zeit. Wiener Wochenschrift, Neue Freie Presse}
               \section[ Paul Goldmann an Arthur Schnitzler und Olga Gussmann, 7. 7. {[}1901{]}]{ Paul Goldmann an Arthur Schnitzler und Olga
               Gussmann, 7. 7. {[}1901{]}}\nopagebreak\mylabel{v}\rehead{ }\begin{ledgroupsized}[t]{13cm}\normalsize\beginnumbering \toendnotes[C]{\smallbreak\pagebreak[2]} \Standort{DLA, A:Schnitzler, HS.NZ85.1.3171.}
\physDesc{Brief, 1 Blatt, 4 Seiten, 1566 Zeichen
\newline{}Handschrift: blaue Tinte, deutsche Kurrent
\newline{}Schnitzler: mit rotem Buntstift zwei Unterstreichungen }\toendnotes[C]{\smallbreak}\pstart
           \noindent{}\raggedleft{}{\pb}\textcolor{gray}{\textbf{DESSAUERSTRASSE 19}}\oindex{Dessauer Strasse@\textbf{Dessauer Straße}|pw}\pend
           \pstart
           Berlin\oindex{Berlin@\textbf{Berlin}|pw}, 7. Juli.\pend
           \pstart\center{}Mein lieber Freund,\pend\pstart
           Endlich zieht Vernunft in Eure Reiſepläne ein, und ich freue mich ſehr darüber und
               über die Ausſicht, Euch doch zu \label{K_L03072-1v}\edtext{ſehen}{\lemma{\textnormal{\emph{ſehen}}}\Cendnote{\textnormal{siehe Paul Goldmann an Arthur Schnitzler, 26. 4. [1901]}}}\label{K_L03072-1h}. Ich gehe ſo zwiſchen dem 20. u. 25. von hier fort, bleibe einen oder zwei Tage in Dresden\oindex{Dresden@\textbf{Dresden}|pw} und Wien\oindex{Wien@\textbf{Wien}|pw}, gehe dann meinetwegen nach dem \label{K_L03072-2v}\edtext{Wörtherſee\oindex{Woerthersee@\textbf{Wörthersee}|pw}}{\lemma{\textnormal{\emph{Wörtherſee}}}\Cendnote{\textnormal{siehe Paul Goldmann an Arthur Schnitzler, 13. 5. [1901]}}}\label{K_L03072-2h} und komme von da aus ſehr gern zu Euch. \textsc{St. Ulrich}\oindex{Urtijei@\textbf{Urtijëi}|pw} im {\pb}Grödener Thal\oindex{Val Gardena@\textbf{Val Gardena}|pw} würde mir beſonders gefallen.
               Denn ſeit Jahren wünſche ich, das Grödener Thal\oindex{Val Gardena@\textbf{Val Gardena}|pw}
               kennen zu lernen. Bitte, halt’ alſo dieſes Projekt feſt. Vielleicht können wir dann
               auch von dort aus ein paar Tage in die Berge ſteigen.\pend
           \pstart
           Ich höre, daß die »Zeit\orgindex{Zeit@Die Zeit|pw}\orgindex{Zeit. Wiener Wochenschrift@Die Zeit. Wiener Wochenschrift|pw}« von 1. Oktober ab \label{K_L03072-3v}\edtext{Tagesblatt}{\lemma{\textnormal{\emph{Tagesblatt}}}\Cendnote{\textnormal{\emph{Die Zeit}\pwindex{Zeit1902-09-27 – 1919@\emph{Die Zeit} {[}1902-09-27 – 1919{]}|pwk} wurde erst ab dem 27. 9. 1902 (bis 31. 8. 1919) als Tageszeitung von Heinrich Kanner\pwindex{Kanner, Heinrich 09.11.1864 – 15.02.1930@\textsc{Kanner, Heinrich} (09.11.1864 – 15.02.1930), \emph{Herausgeber, Publizist}|pwk} und Isidor Singer\pwindex{Singer, Isidor 16.01.1857 – 08.12.1927@\textsc{Singer, Isidor} (16.01.1857 – 08.12.1927), \emph{Journalist, Herausgeber, Soziologe}|pwk}
                  herausgegeben. Bis zum 29. 10. 1904 erschien \emph{Die Zeit}\pwindex{Zeit. Wiener Wochenschrift1894 – 1904@\emph{Die Zeit. Wiener Wochenschrift} {[}1894 – 1904{]}|pwk} parallel als Wochenschrift. Die \emph{Neue Freie Presse}\pwindex{Neue Freie Presse1864 – 1939@\emph{Neue Freie Presse} {[}1864 – 1939{]}|pwk} ersetzte sie nicht.}}}\label{K_L03072-3h}
               wird mit 1 Million \textsc{Kronen} Capital. Weißt Du etwas davon?
               Kommt es dazu, ſo bedeutet {\pb}das, nach meiner
               Überzeugung, den Anfang vom Ende der N. Fr. Pr.\orgindex{Neue Freie Presse@Neue Freie Presse|pw}
               So ſetzt auch \textsc{Dr. Kanner\pwindex{Kanner, Heinrich 09.11.1864 – 15.02.1930@\textsc{Kanner, Heinrich} (09.11.1864 – 15.02.1930), \emph{Herausgeber, Publizist}|pw}} ſeinen Lebensplan durch. Nur ich, – ich allein bleibe auf der Strecke. Es iſt
               martervoll!\pend
           \pstart
           Viele treue Grüße! {\\[\baselineskip]}Dein {\\[\baselineskip]}\spacefill\mbox{Paul Goldmann.}\pend
           \leftskip=0em{}{\bigskip}\pstart
           \noindent{}Liebes Fräulein \textsc{Olga}, Ich
               danke Ihnen für Ihren lieben und guten Brief. Jetzt, bitte, ſetzen Sie noch durch,
               daß wir ins Grödener Thal\oindex{Val Gardena@\textbf{Val Gardena}|pw}{ }{\pb}gehen. Ich möchte ſehr gern dorthin, was für \textsc{Arthur} immerhin einen ausreichenden Grund bilden könnte,
                  \strikeout{\textcolor{gray}{in}d\textcolor{gray}{e}s} ſich für einen anderen Ort zu
               entſchließen. Auch ich möchte, gleich Ihnen, ſtillſitzen und Ruhe, Ruhe haben. Über
                  \textsc{Kerr\pwindex{Kerr, Alfred 25.12.1867 – 12.10.1948@\textsc{Kerr, Alfred} (25.12.1867 – 12.10.1948), \emph{Schriftsteller, Kritiker}|pw}} ſprechen wir mündlich. Er wird übrigens nur nachkommen und nicht mitkommen
               können. Ihrem lieben Schweſterchen\pwindex{Steinrueck, Elisabeth 19.11.1885 – 07.04.1920@\textsc{Steinrück, Elisabeth} (19.11.1885 – 07.04.1920)|pwv} wünſche ich gute Beſſerung. Haben Sie keine Sorgen! Wenn ſie
                  \textsc{Arthurs} Behandlung bisher ausgehalten hat, wird ſie auch
               davonkommen. Sie iſt eine widerſtandsfähige Natur.\pend
           \pstart
           Herzlichſt Ihr {\\[\baselineskip]}\spacefill\mbox{Dr. Paul Goldmann.}\pend
           \leftskip=0em{}
         
         \endnumbering\mylabel{h}\end{ledgroupsized}  \newcommand{\dateiname}{L03072}\newcommand{\titel}{Paul Goldmann an Arthur Schnitzler und Olga Gussmann, 7. 7. [1901]}\newcommand{\editorInnen}{Martin Anton Müller und Laura Untner}%% latex-leseansicht-abspann.tex
%% Abspann für die Leseansicht.
%% Der Schalter \ifkorrekturansicht ist bereits durch den Vorspann gesetzt.

%% latex-abspann.tex
%% Gemeinsamer Abspann für Korrekturansicht und Leseansicht.
%% Setzt den Schalter \ifkorrekturansicht voraus (gesetzt in den
%% einbindenden Dateien latex-korrekturansicht-abspann.tex bzw.
%% latex-leseansicht-abspann.tex).
%% ---------------------------------------------------------------

\normalsize

% Das esempio-Environment wird nur in der Leseansicht benötigt
\ifkorrekturansicht\else
\newenvironment{esempio}[3]%
{
    \vspace{1.5ex}
    \rlap{\underline{#1}}
    \par
    \setlength{\parindent}{0cm}
    \nopagebreak
    \leftskip=#2cm
    \rightskip=#3cm
}
{
    \par
}
\fi

\doendnotes{C}
\bigskip
\vfill

\clearpage

\footnotesize

\ifkorrekturansicht
  \lohead{\textsc{register}}
\fi

% theindex-Environment neu definieren ohne reledmac
\makeatletter
\renewenvironment{theindex}{%
  \ifkorrekturansicht
    \section*{\indexname}%
  \else
    \subsubsection*{Index der erwähnten Entitäten}%
  \fi
  \setlength{\parindent}{0pt}%
  \setlength{\parskip}{0pt plus 0.3pt}%
  \let\item\@idxitem
}{%
  \ifkorrekturansicht\clearpage\fi
}
\makeatother

\IfFileExists{\jobname-pw.ind}{\input{\jobname-pw.ind}}{}

% Quellenangabe nur in der Leseansicht
\ifkorrekturansicht\else
% Fallback-Definitionen, falls die .tex-Datei \titel etc. nicht gesetzt hat
\providecommand{\titel}{}
\providecommand{\editorInnen}{}
\providecommand{\dateiname}{\jobname}

\vspace{3cm}

\vfill

\footnotesize
\textsc{Quelle}: \titel. Herausgegeben von {\editorInnen}. In: \emph{Arthur Schnitzler: Briefwechsel mit Autorinnen und Autoren}.
 Digitale Edition, https://schnitzler-briefe.acdh.oeaw.ac.at/{\dateiname}.html (Stand \today)
\fi

\end{document}


      