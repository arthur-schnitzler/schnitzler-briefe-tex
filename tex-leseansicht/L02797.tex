%% latex-korrekturansicht-vorspann.tex
%% Vorspann für die Korrekturansicht.
%% Lädt die gemeinsame Datei latex-vorspann.tex mit gesetztem Schalter.

\newif\ifkorrekturansicht
\korrekturansichttrue

\input{../tex-inputs/latex-vorspann}


\section[Vally Rosengart an Arthur Schnitzler, {[}16. 1. 1898{]}]{L02797 Vally Rosengart an Arthur Schnitzler, {[}16. 1. 1898{]}}
\nopagebreak\mylabel{L02797v}
\rehead{ }\normalsize\beginnumbering\briefempfaengerindex{Schnitzler, Arthur@\textsc{Schnitzler, Arthur}!zzzRosengart, Vally@\emph{von Vally Rosengart}!1898-01-162@{{[}16. 1. 1898{]}}|(be}
\toendnotes[C]{\smallbreak\pagebreak[2]}\Standort{DLA, A:Schnitzler, HS.NZ85.1.4334.}
\physDesc{Telegramm, 424 Zeichen
\newline{}maschinell
\newline{}Schnitzler: mit rotem Buntstift drei Unterstreichungen 
\newline{}Ordnung: beschnitten }\toendnotes[C]{\smallbreak}
\pstart
           \centering{}{\pb}de frankfurtm\oindex{Frankfurt am Main@\textbf{Frankfurt am Main}, \emph{P.PPLA3}|pw}
                  854 62 16/1{ }4 58 s=\pend
           \vspace{0.5em}
\pstart
           paul\pwindex{Goldmann, Paul 31.01.1865 – 25.09.1935@\textsc{Goldmann, Paul} (31.01.1865 – 25.09.1935), \emph{Schriftsteller/Schriftstellerin, Journalist/Journalistin}|pw} wuenscht, ohne persoenlich
               hervorzutreten, fuer \label{K_L02797-1v}\edtext{schlenthers\pwindex{Schlenther, Paul 20.08.1854 – 30.04.1916@\textsc{Schlenther, Paul} (20.08.1854 – 30.04.1916), \emph{Schriftsteller/Schriftstellerin, Kritiker/Kritikerin, Theaterleiter/Theaterleiterin}|pw} nachfolge}{\lemma{\textnormal{\emph{schlenthers nachfolge}}}\Cendnote{\textnormal{Paul Schlenther\pwindex{Schlenther, Paul 20.08.1854 – 30.04.1916@\textsc{Schlenther, Paul} (20.08.1854 – 30.04.1916), \emph{Schriftsteller/Schriftstellerin, Kritiker/Kritikerin, Theaterleiter/Theaterleiterin}|pwk} war 1886 als Nachfolger von Theodor
                     Fontane\pwindex{Fontane, Theodor 30.12.1819 – 20.09.1898@\textsc{Fontane, Theodor} (30.12.1819 – 20.09.1898), \emph{Schriftsteller/Schriftstellerin, Kritiker/Kritikerin, Apotheker/Apothekerin}|pwk} zur \emph{Vossischen Zeitung}\orgindex{Vossische Zeitung@Vossische Zeitung|pwk}
                  gekommen. Als er Mitte Januar 1898 zum neuen \emph{Burgtheater}\orgindex{Burgtheater@Burgtheater|pwk}-Direktor ernannt wurde, wurde seine
                  Position vakant. Damit erläutern sich zwei bislang kryptische Stellen in der
                  Korrespondenz zwischen Schnitzler und Otto Brahm\pwindex{Brahm, Otto 05.02.1856 – 28.11.1912@\textsc{Brahm, Otto} (05.02.1856 – 28.11.1912), \emph{Theaterleiter/Theaterleiterin, Regisseur/Regisseurin}|pwk}. Schnitzler kontaktierte Brahm\pwindex{Brahm, Otto 05.02.1856 – 28.11.1912@\textsc{Brahm, Otto} (05.02.1856 – 28.11.1912), \emph{Theaterleiter/Theaterleiterin, Regisseur/Regisseurin}|pwk} im
                  Sinne Goldmanns\pwindex{Goldmann, Paul 31.01.1865 – 25.09.1935@\textsc{Goldmann, Paul} (31.01.1865 – 25.09.1935), \emph{Schriftsteller/Schriftstellerin, Journalist/Journalistin}|pwk}, woraufhin Brahm\pwindex{Brahm, Otto 05.02.1856 – 28.11.1912@\textsc{Brahm, Otto} (05.02.1856 – 28.11.1912), \emph{Theaterleiter/Theaterleiterin, Regisseur/Regisseurin}|pwk} am 18. 1. 1898 mit einem Telegramm antwortete: »Ihren Kandidaten\pwindex{Goldmann, Paul 31.01.1865 – 25.09.1935@\textsc{Goldmann, Paul} (31.01.1865 – 25.09.1935), \emph{Schriftsteller/Schriftstellerin, Journalist/Journalistin}|pwv}{ }Schlenther\pwindex{Schlenther, Paul 20.08.1854 – 30.04.1916@\textsc{Schlenther, Paul} (20.08.1854 – 30.04.1916), \emph{Schriftsteller/Schriftstellerin, Kritiker/Kritikerin, Theaterleiter/Theaterleiterin}|pw} empfohlen.« Am
                     21. 1. 1898 verfasste Schlenther\pwindex{Schlenther, Paul 20.08.1854 – 30.04.1916@\textsc{Schlenther, Paul} (20.08.1854 – 30.04.1916), \emph{Schriftsteller/Schriftstellerin, Kritiker/Kritikerin, Theaterleiter/Theaterleiterin}|pwk} einen Brief an Schnitzler, in dem er angibt, er habe die Anfrage an den Chefredakteur
                     Friedrich Stephany\pwindex{Stephany, Friedrich 1830-03-14 – 1912@\textsc{Stephany, Friedrich} (1830-03-14 – 1912), \emph{Redakteur/Redakteurin, Publizist/Publizistin}|pwk} weitergereicht, doch
                  dürfte die Stelle erst im Herbst nachbesetzt werden. In einem Antwortbrief Schnitzlers an Brahm\pwindex{Brahm, Otto 05.02.1856 – 28.11.1912@\textsc{Brahm, Otto} (05.02.1856 – 28.11.1912), \emph{Theaterleiter/Theaterleiterin, Regisseur/Regisseurin}|pwk} vom 22. 1. 1898 wird
                  die Sache zum letzten Mal angesprochen: »Für Ihre Verwendung betreffs Goldmann\pwindex{Goldmann, Paul 31.01.1865 – 25.09.1935@\textsc{Goldmann, Paul} (31.01.1865 – 25.09.1935), \emph{Schriftsteller/Schriftstellerin, Journalist/Journalistin}|pw} noch einmal herzlichsten Dank.
                     Er selbst wußte nichts davon; nur seine Verwandten; heute weiß er es natürlich.
                     Halten Sie einen Erfolg für möglich?« (\emph{Der Briefwechsel Arthur Schnitzler – Otto Brahm}.
                        Vollständige Ausgabe. Herausgegeben, eingeleitet und erläutert von Oskar
                        Seidlin. Tübingen: \emph{Niemeyer}{ }1975, S. 42–43). Zugleich erlaubt diese Stelle die Datierung zusammen mit der Monats- und Tagesangabe in der
                  Übermittlungszeile des Telegramms.}}}\label{K_L02797-1} bei vossischer\orgindex{Vossische Zeitung@Vossische Zeitung|pw} zu candidiren und bittet sie, schnellstens und nachdruecklichst
               in diesem sinne zu wirken. vielleicht machen sie brahm\pwindex{Brahm, Otto 05.02.1856 – 28.11.1912@\textsc{Brahm, Otto} (05.02.1856 – 28.11.1912), \emph{Theaterleiter/Theaterleiterin, Regisseur/Regisseurin}|pw} telegraphisch aufmerksam, dasz goldmann\pwindex{Goldmann, Paul 31.01.1865 – 25.09.1935@\textsc{Goldmann, Paul} (31.01.1865 – 25.09.1935), \emph{Schriftsteller/Schriftstellerin, Journalist/Journalistin}|pw} zu haben waere, betonen seine glaenzende eignung und ersuchen brahm\pwindex{Brahm, Otto 05.02.1856 – 28.11.1912@\textsc{Brahm, Otto} (05.02.1856 – 28.11.1912), \emph{Theaterleiter/Theaterleiterin, Regisseur/Regisseurin}|pw} zu interveniren. herzlichen dank fuer
               alles, was sie dem freunde\pwindex{Goldmann, Paul 31.01.1865 – 25.09.1935@\textsc{Goldmann, Paul} (31.01.1865 – 25.09.1935), \emph{Schriftsteller/Schriftstellerin, Journalist/Journalistin}|pwv}
               thun = \spacefill\mbox{rosengart-goldmann.}\pend
           \selectlanguage{ngerman}\endnumbering\briefempfaengerindex{Schnitzler, Arthur@\textsc{Schnitzler, Arthur}!zzzRosengart, Vally@\emph{von Vally Rosengart}!1898-01-162@{{[}16. 1. 1898{]}}|)be}\mylabel{L02797h}  \normalsize

\doendnotes{C}
\bigskip
\vfill

\clearpage

\footnotesize

\lohead{\textsc{register}}

% Definiere theindex-Environment komplett neu ohne reledmac
\makeatletter
\renewenvironment{theindex}{%
  \section*{\indexname}%
  \setlength{\parindent}{0pt}%
  \setlength{\parskip}{0pt plus 0.3pt}%
  \let\item\@idxitem
}{%
  \clearpage
}
\makeatother

\IfFileExists{\jobname-pw.ind}{\input{\jobname-pw.ind}}{}

\end{document}

      