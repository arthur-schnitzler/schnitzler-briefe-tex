%% latex-leseansicht-vorspann.tex
%% Vorspann für die Leseansicht.
%% Lädt die gemeinsame Datei latex-vorspann.tex mit nicht gesetztem Schalter.

\newif\ifkorrekturansicht
\korrekturansichtfalse

\input{../tex-inputs/latex-vorspann}


\section[Vally Rosengart an Arthur Schnitzler, {[}16. 1. 1898{]}]{L02797 Vally Rosengart an Arthur Schnitzler, {[}16. 1. 1898{]}}
\nopagebreak\mylabel{L02797v}
\rehead{ }\normalsize\beginnumbering\briefempfaengerindex{Schnitzler, Arthur@\textsc{Schnitzler, Arthur}!zzzRosengart, Vally@\emph{von Vally Rosengart}!1898-01-162@{{[}16. 1. 1898{]}}|(be}
\toendnotes[C]{\smallbreak\pagebreak[2]}
\correspDesc{Versand  durch Vally Rosengart am [16. 1. 1898] in Frankfurt am Main
\newline{}Erhalt  durch Arthur Schnitzler am [16. 1. 1898] in Wien}\toendnotes[C]{\smallbreak}
\Standort{DLA, A:Schnitzler, HS.NZ85.1.4334.}
\physDesc{Telegramm, 424 Zeichen
\newline{}maschinell
\newline{}Schnitzler: mit rotem Buntstift drei Unterstreichungen 
\newline{}Ordnung: beschnitten }\toendnotes[C]{\smallbreak}
\pstart
           \centering{}{\pb}de frankfurtm\oindex{Frankfurt am Main@\textbf{Frankfurt am Main}, \emph{Hauptstadt}|pw}
                  854 62 16/1{ }4 58 s=\pend
           \vspace{0.5em}
\pstart
           paul\pwindex{Goldmann, Paul 31.\,1.\,1865 Breslau – 25.\,9.\,1935 Wien@\textsc{Goldmann, Paul} (31.\,1.\,1865 Breslau – 25.\,9.\,1935 Wien), \emph{Schriftsteller, Journalist}|pw} wuenscht, ohne persoenlich
               hervorzutreten, fuer \label{K_L02797-1v}\edtext{schlenthers\pwindex{Schlenther, Paul 20.\,8.\,1854 Chernyakhovsk – 30.\,4.\,1916 Berlin@\textsc{Schlenther, Paul} (20.\,8.\,1854 Chernyakhovsk – 30.\,4.\,1916 Berlin), \emph{Schriftsteller, Kritiker, Theaterleiter}|pw} nachfolge}{\lemma{\textnormal{\emph{schlenthers nachfolge}}}\Cendnote{\textnormal{Paul Schlenther\pwindex{Schlenther, Paul 20.\,8.\,1854 Chernyakhovsk – 30.\,4.\,1916 Berlin@\textsc{Schlenther, Paul} (20.\,8.\,1854 Chernyakhovsk – 30.\,4.\,1916 Berlin), \emph{Schriftsteller, Kritiker, Theaterleiter}|pwk} war 1886 als Nachfolger von Theodor
                     Fontane\pwindex{Fontane, Theodor 30.\,12.\,1819 Neuruppin – 20.\,9.\,1898 Berlin@\textsc{Fontane, Theodor} (30.\,12.\,1819 Neuruppin – 20.\,9.\,1898 Berlin), \emph{Schriftsteller, Kritiker, Apotheker}|pwk} zur \emph{Vossischen Zeitung}\orgindex{Vossische Zeitung@Vossische Zeitung|pwk}
                  gekommen. Als er Mitte Januar 1898 zum neuen \emph{Burgtheater}\orgindex{Burgtheater@Burgtheater|pwk}-Direktor ernannt wurde, wurde seine
                  Position vakant. Damit erläutern sich zwei bislang kryptische Stellen in der
                  Korrespondenz zwischen Schnitzler und Otto Brahm\pwindex{Brahm, Otto 5.\,2.\,1856 Hamburg – 28.\,11.\,1912 Berlin@\textsc{Brahm, Otto} (5.\,2.\,1856 Hamburg – 28.\,11.\,1912 Berlin), \emph{Theaterleiter, Regisseur}|pwk}. Schnitzler kontaktierte Brahm\pwindex{Brahm, Otto 5.\,2.\,1856 Hamburg – 28.\,11.\,1912 Berlin@\textsc{Brahm, Otto} (5.\,2.\,1856 Hamburg – 28.\,11.\,1912 Berlin), \emph{Theaterleiter, Regisseur}|pwk} im
                  Sinne Goldmanns\pwindex{Goldmann, Paul 31.\,1.\,1865 Breslau – 25.\,9.\,1935 Wien@\textsc{Goldmann, Paul} (31.\,1.\,1865 Breslau – 25.\,9.\,1935 Wien), \emph{Schriftsteller, Journalist}|pwk}, woraufhin Brahm\pwindex{Brahm, Otto 5.\,2.\,1856 Hamburg – 28.\,11.\,1912 Berlin@\textsc{Brahm, Otto} (5.\,2.\,1856 Hamburg – 28.\,11.\,1912 Berlin), \emph{Theaterleiter, Regisseur}|pwk} am 18. 1. 1898 mit einem Telegramm antwortete: »Ihren Kandidaten\pwindex{Goldmann, Paul 31.\,1.\,1865 Breslau – 25.\,9.\,1935 Wien@\textsc{Goldmann, Paul} (31.\,1.\,1865 Breslau – 25.\,9.\,1935 Wien), \emph{Schriftsteller, Journalist}|pwv}{ }Schlenther\pwindex{Schlenther, Paul 20.\,8.\,1854 Chernyakhovsk – 30.\,4.\,1916 Berlin@\textsc{Schlenther, Paul} (20.\,8.\,1854 Chernyakhovsk – 30.\,4.\,1916 Berlin), \emph{Schriftsteller, Kritiker, Theaterleiter}|pw} empfohlen.« Am
                     21. 1. 1898 verfasste Schlenther\pwindex{Schlenther, Paul 20.\,8.\,1854 Chernyakhovsk – 30.\,4.\,1916 Berlin@\textsc{Schlenther, Paul} (20.\,8.\,1854 Chernyakhovsk – 30.\,4.\,1916 Berlin), \emph{Schriftsteller, Kritiker, Theaterleiter}|pwk} einen Brief an Schnitzler, in dem er angibt, er habe die Anfrage an den Chefredakteur
                     Friedrich Stephany\pwindex{Stephany, Friedrich 14.\,3.\,1830 – 1912@\textsc{Stephany, Friedrich} (14.\,3.\,1830 – 1912), \emph{Redakteur, Publizist}|pwk} weitergereicht, doch
                  dürfte die Stelle erst im Herbst nachbesetzt werden. In einem Antwortbrief Schnitzlers an Brahm\pwindex{Brahm, Otto 5.\,2.\,1856 Hamburg – 28.\,11.\,1912 Berlin@\textsc{Brahm, Otto} (5.\,2.\,1856 Hamburg – 28.\,11.\,1912 Berlin), \emph{Theaterleiter, Regisseur}|pwk} vom 22. 1. 1898 wird
                  die Sache zum letzten Mal angesprochen: »Für Ihre Verwendung betreffs Goldmann\pwindex{Goldmann, Paul 31.\,1.\,1865 Breslau – 25.\,9.\,1935 Wien@\textsc{Goldmann, Paul} (31.\,1.\,1865 Breslau – 25.\,9.\,1935 Wien), \emph{Schriftsteller, Journalist}|pw} noch einmal herzlichsten Dank.
                     Er selbst wußte nichts davon; nur seine Verwandten; heute weiß er es natürlich.
                     Halten Sie einen Erfolg für möglich?« (\emph{Der Briefwechsel Arthur Schnitzler – Otto Brahm}.
                        Vollständige Ausgabe. Herausgegeben, eingeleitet und erläutert von Oskar
                        Seidlin. Tübingen: \emph{Niemeyer}{ }1975, S. 42–43). Zugleich erlaubt diese Stelle die Datierung zusammen mit der Monats- und Tagesangabe in der
                  Übermittlungszeile des Telegramms.}}}\label{K_L02797-1} bei vossischer\orgindex{Vossische Zeitung@Vossische Zeitung|pw} zu candidiren und bittet sie, schnellstens und nachdruecklichst
               in diesem sinne zu wirken. vielleicht machen sie brahm\pwindex{Brahm, Otto 5.\,2.\,1856 Hamburg – 28.\,11.\,1912 Berlin@\textsc{Brahm, Otto} (5.\,2.\,1856 Hamburg – 28.\,11.\,1912 Berlin), \emph{Theaterleiter, Regisseur}|pw} telegraphisch aufmerksam, dasz goldmann\pwindex{Goldmann, Paul 31.\,1.\,1865 Breslau – 25.\,9.\,1935 Wien@\textsc{Goldmann, Paul} (31.\,1.\,1865 Breslau – 25.\,9.\,1935 Wien), \emph{Schriftsteller, Journalist}|pw} zu haben waere, betonen seine glaenzende eignung und ersuchen brahm\pwindex{Brahm, Otto 5.\,2.\,1856 Hamburg – 28.\,11.\,1912 Berlin@\textsc{Brahm, Otto} (5.\,2.\,1856 Hamburg – 28.\,11.\,1912 Berlin), \emph{Theaterleiter, Regisseur}|pw} zu interveniren. herzlichen dank fuer
               alles, was sie dem freunde\pwindex{Goldmann, Paul 31.\,1.\,1865 Breslau – 25.\,9.\,1935 Wien@\textsc{Goldmann, Paul} (31.\,1.\,1865 Breslau – 25.\,9.\,1935 Wien), \emph{Schriftsteller, Journalist}|pwv}
               thun = \spacefill\mbox{rosengart-goldmann.}\pend
           \selectlanguage{ngerman}\endnumbering\briefempfaengerindex{Schnitzler, Arthur@\textsc{Schnitzler, Arthur}!zzzRosengart, Vally@\emph{von Vally Rosengart}!1898-01-162@{{[}16. 1. 1898{]}}|)be}\mylabel{L02797h}  \newcommand{\dateiname}{L02797}\newcommand{\titel}{Vally Rosengart an Arthur Schnitzler, [16. 1. 1898]}\newcommand{\editorInnen}{Martin Anton Müller und Laura Untner}%% latex-leseansicht-abspann.tex
%% Abspann für die Leseansicht.
%% Der Schalter \ifkorrekturansicht ist bereits durch den Vorspann gesetzt.

%% latex-abspann.tex
%% Gemeinsamer Abspann für Korrekturansicht und Leseansicht.
%% Setzt den Schalter \ifkorrekturansicht voraus (gesetzt in den
%% einbindenden Dateien latex-korrekturansicht-abspann.tex bzw.
%% latex-leseansicht-abspann.tex).
%% ---------------------------------------------------------------

\normalsize

% Das esempio-Environment wird nur in der Leseansicht benötigt
\ifkorrekturansicht\else
\newenvironment{esempio}[3]%
{
    \vspace{1.5ex}
    \rlap{\underline{#1}}
    \par
    \setlength{\parindent}{0cm}
    \nopagebreak
    \leftskip=#2cm
    \rightskip=#3cm
}
{
    \par
}
\fi

\doendnotes{C}
\bigskip
\vfill

\clearpage

\footnotesize

\ifkorrekturansicht
  \lohead{\textsc{register}}
\fi

% theindex-Environment neu definieren ohne reledmac
\makeatletter
\renewenvironment{theindex}{%
  \ifkorrekturansicht
    \section*{\indexname}%
  \else
    \subsubsection*{Index der erwähnten Entitäten}%
  \fi
  \setlength{\parindent}{0pt}%
  \setlength{\parskip}{0pt plus 0.3pt}%
  \let\item\@idxitem
}{%
  \ifkorrekturansicht\clearpage\fi
}
\makeatother

\IfFileExists{\jobname-pw.ind}{\input{\jobname-pw.ind}}{}

% Quellenangabe nur in der Leseansicht
\ifkorrekturansicht\else
% Fallback-Definitionen, falls die .tex-Datei \titel etc. nicht gesetzt hat
\providecommand{\titel}{}
\providecommand{\editorInnen}{}
\providecommand{\dateiname}{\jobname}

\vspace{3cm}

\vfill

\footnotesize
\textsc{Quelle}: \titel. Herausgegeben von {\editorInnen}. In: \emph{Arthur Schnitzler: Briefwechsel mit Autorinnen und Autoren}.
 Digitale Edition, https://schnitzler-briefe.acdh.oeaw.ac.at/{\dateiname}.html (Stand \today)
\fi

\end{document}


