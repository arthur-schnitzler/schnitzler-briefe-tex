%% latex-leseansicht-vorspann.tex
%% Vorspann für die Leseansicht.
%% Lädt die gemeinsame Datei latex-vorspann.tex mit nicht gesetztem Schalter.

\newif\ifkorrekturansicht
\korrekturansichtfalse

\input{../tex-inputs/latex-vorspann}


\section[Arthur Schnitzler an Gustav Schwarzkopf, 10. 8. 1901]{L04078 Arthur Schnitzler an Gustav Schwarzkopf, 10. 8. 1901}
\nopagebreak\mylabel{L04078v}
\rehead{ }\normalsize\beginnumbering\briefempfaengerindex{Schwarzkopf, Gustav@\textsc{Schwarzkopf, Gustav}!zzzSchnitzler, Arthur@\emph{von Arthur Schnitzler}!1901-08-104@{10. 8. 1901}|(be}
\toendnotes[C]{\smallbreak\pagebreak[2]}
\correspDesc{Versand  durch Arthur Schnitzler am 10. 8. 1901 in Vahrn
\newline{}Erhalt  durch Gustav Schwarzkopf im Zeitraum [11. 8. 1901 – 15. 8. 1901?] in Wien}\toendnotes[C]{\smallbreak}
\Standort{CUL, Schnitzler, B 96.}
\physDesc{Brief, 1 Blatt, 2 Seiten, 732 Zeichen
\newline{}Handschrift: schwarze Tinte, deutsche Kurrent}\toendnotes[C]{\smallbreak}
\pstart
           \raggedleft{}{\pb}\textsc{Vahrn\oindex{Vahrn@\textbf{Vahrn}, \emph{Hauptstadt}|pw}},
                     10. 8. 901.\pend
           \vspace{0.5em}
\pstart
           lieber Guſtav, vor allen muſs ich ein Unrecht gut machen; der Verfaſſer\pwindex{Silberer, Geza 1.\,12.\,1876 Vršac – 5.\,4.\,1938 Wien@\textsc{Silberer, Geza} (1.\,12.\,1876 Vršac – 5.\,4.\,1938 Wien), \emph{Schriftsteller, Journalist}|pwv} der \textsc{Irene von Bien\pwindex{Silberer, Geza 1.\,12.\,1876 Vršac – 5.\,4.\,1938 Wien@\textsc{Silberer, Geza} (1.\,12.\,1876 Vršac – 5.\,4.\,1938 Wien), \emph{Schriftsteller, Journalist}!Irene van Bien@\strich\emph{Irene van Bien}|pw}} ist nicht \textsc{Adele Schreiber\pwindex{Schreiber, Adele 29.\,4.\,1872 Wien – 20.\,2.\,1957 Herrliberg@\textsc{Schreiber, Adele} (29.\,4.\,1872 Wien – 20.\,2.\,1957 Herrliberg), \emph{Schriftstellerin, Politikerin, Pädagogin}|pw}}, ſondern ein männliches Weſen, namens Silberer\pwindex{Silberer, Geza 1.\,12.\,1876 Vršac – 5.\,4.\,1938 Wien@\textsc{Silberer, Geza} (1.\,12.\,1876 Vršac – 5.\,4.\,1938 Wien), \emph{Schriftsteller, Journalist}|pw}.
               Dies erzählt mir Goldma{\geminationn}\pwindex{Goldmann, Paul 31.\,1.\,1865 Breslau – 25.\,9.\,1935 Wien@\textsc{Goldmann, Paul} (31.\,1.\,1865 Breslau – 25.\,9.\,1935 Wien), \emph{Schriftsteller, Journalist}|pw}, den ich nemlich \label{K_L04078-1v}\edtext{in \textsc{Welsberg\oindex{Welsberg-Taisten@\textbf{Welsberg-Taisten}, \emph{Verwaltungsgebiet}|pw}} getroffen}{\lemma{\textnormal{\emph{in Welsberg getroffen}}}\Cendnote{\textnormal{Vgl. A. S.: \emph{Tagebuch}, 7. 8. 1901.}}}\label{K_L04078-1}. \label{K_L04078-2v}\edtext{Montag treffen wir in \textsc{Bozen\oindex{Bozen@\textbf{Bozen}, \emph{Hauptstadt}|pw}}}{\lemma{\textnormal{\emph{Montag … Bozen}}}\Cendnote{\textnormal{Das Treffen verschob sich um einen Tag auf Dienstag, vgl. A. S.: \emph{Tagebuch}, 13. 8. 1901.}}}\label{K_L04078-2} zuſa{\geminationm}en, fahren da{\geminationn}
               nach Trient\oindex{Trient@\textbf{Trient}|pw} u gedenken Freitag den \label{K_L04078-3v}\edtext{16. in
               \textsc{Welsberg\oindex{Welsberg-Taisten@\textbf{Welsberg-Taisten}, \emph{Verwaltungsgebiet}|pw}}}{\lemma{\textnormal{\emph{16. in
               Welsberg}}}\Cendnote{\textnormal{Dieser Termin hielt, vgl. A. S.: \emph{Wiener Schnitzler}, 16. 8. 1901.}}}\label{K_L04078-3} zu ſein, wo wir etwa 14 Tage bleiben. W.\oindex{Welsberg-Taisten@\textbf{Welsberg-Taisten}, \emph{Verwaltungsgebiet}|pw}{ }\textsc{resp.} das Bad Waldbrunn\oindex{Wildbad Waldbrunn@\textbf{Wildbad Waldbrunn}, \emph{Spa}|pw}{ }{\pb}wo wir wohnen werden, liegt entzückend;
               Penſion 3 fl 50. Von Wien\oindex{Wien@\textbf{Wien}, \emph{Verwaltungsgebiet}|pw} in etwa 12 Stunden
               erreichbar. Puſterthal\oindex{Pustertal@\textbf{Pustertal}, \emph{Tal}|pw}, nahe Toblach\oindex{Toblach@\textbf{Toblach}, \emph{Verwaltungsgebiet}|pw}. Wollen Sie ſich nicht doch in letzter Minute entſchließen? Wir
               würden uns alle ſo ſehr freuen. Eine raſche Antwort trifft mich in Trient\oindex{Trient@\textbf{Trient}|pw}{ }\introOben{}(post
                  rest.)\introOben{}, andernfalls \textsc{Bad Waldbrunn\oindex{Wildbad Waldbrunn@\textbf{Wildbad Waldbrunn}, \emph{Spa}|pw}} bei Welsberg\oindex{Welsberg-Taisten@\textbf{Welsberg-Taisten}, \emph{Verwaltungsgebiet}|pw} (Puſtertal\oindex{Pustertal@\textbf{Pustertal}, \emph{Tal}|pw}.)\pend
           
\pstart
           Viele herzliche Grüße.{\\[\baselineskip]} Ihr{\\[\baselineskip]}\spacefill\mbox{Arth Sch}\pend
           \leftskip=0em{}\selectlanguage{ngerman}\endnumbering\briefempfaengerindex{Schwarzkopf, Gustav@\textsc{Schwarzkopf, Gustav}!zzzSchnitzler, Arthur@\emph{von Arthur Schnitzler}!1901-08-104@{10. 8. 1901}|)be}\mylabel{L04078h}
\begin{anhang}
\end{anhang}\newcommand{\dateiname}{L04078}\newcommand{\titel}{Arthur Schnitzler an Gustav Schwarzkopf, 10. 8. 1901}\newcommand{\editorInnen}{Herausgegeben von Jahnke, SelmaMüller, Martin Anton}%% latex-leseansicht-abspann.tex
%% Abspann für die Leseansicht.
%% Der Schalter \ifkorrekturansicht ist bereits durch den Vorspann gesetzt.

%% latex-abspann.tex
%% Gemeinsamer Abspann für Korrekturansicht und Leseansicht.
%% Setzt den Schalter \ifkorrekturansicht voraus (gesetzt in den
%% einbindenden Dateien latex-korrekturansicht-abspann.tex bzw.
%% latex-leseansicht-abspann.tex).
%% ---------------------------------------------------------------

\normalsize

% Das esempio-Environment wird nur in der Leseansicht benötigt
\ifkorrekturansicht\else
\newenvironment{esempio}[3]%
{
    \vspace{1.5ex}
    \rlap{\underline{#1}}
    \par
    \setlength{\parindent}{0cm}
    \nopagebreak
    \leftskip=#2cm
    \rightskip=#3cm
}
{
    \par
}
\fi

\doendnotes{C}
\bigskip
\vfill

\clearpage

\footnotesize

\ifkorrekturansicht
  \lohead{\textsc{register}}
\fi

% theindex-Environment neu definieren ohne reledmac
\makeatletter
\renewenvironment{theindex}{%
  \ifkorrekturansicht
    \section*{\indexname}%
  \else
    \subsubsection*{Index der erwähnten Entitäten}%
  \fi
  \setlength{\parindent}{0pt}%
  \setlength{\parskip}{0pt plus 0.3pt}%
  \let\item\@idxitem
}{%
  \ifkorrekturansicht\clearpage\fi
}
\makeatother

\IfFileExists{\jobname-pw.ind}{\input{\jobname-pw.ind}}{}

% Quellenangabe nur in der Leseansicht
\ifkorrekturansicht\else
% Fallback-Definitionen, falls die .tex-Datei \titel etc. nicht gesetzt hat
\providecommand{\titel}{}
\providecommand{\editorInnen}{}
\providecommand{\dateiname}{\jobname}

\vspace{3cm}

\vfill

\footnotesize
\textsc{Quelle}: \titel. Herausgegeben von {\editorInnen}. In: \emph{Arthur Schnitzler: Briefwechsel mit Autorinnen und Autoren}.
 Digitale Edition, https://schnitzler-briefe.acdh.oeaw.ac.at/{\dateiname}.html (Stand \today)
\fi

\end{document}


