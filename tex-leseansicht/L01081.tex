\input{../tex-inputs/latex-pdf-vorspann}
\begin{center}
            \textcolor{red}{ENTWURF. ENTZIFFERUNG NOCH NICHT KORREKTURGELESEN}
                      \end{center}
            
               \section[Hugo von Hofmannsthal an Arthur Schnitzler, 27. 10. 1900]{ Hugo von Hofmannsthal an Arthur Schnitzler, 27. 10. 1900}\nopagebreak\mylabel{v}\rehead{ }\begin{ledgroupsized}[t]{13cm}\normalsize\beginnumbering\briefempfaengerindex{Schnitzler, Arthur@\textsc{Schnitzler, Arthur}!zzzHofmannsthal, Hugo von@\emph{von Hugo von Hofmannsthal}!1900-10-272@{27. 10. 1900}|(be} \toendnotes[C]{\smallbreak\pagebreak[2]} \Standort{CUL, Schnitzler, B 43.}
\physDesc{Postkarte
\newline{}Handschrift: schwarze Tinte, deutsche Kurrent\newline{}Versand: 1) Rohrpost 2) Stempel: »\nobreak{}\oindex{III., Landstrasse@\textbf{III., Landstraße}|pwk}Wien 3/3, 27 X 00, 11 10V\nobreak{}«. 3) Stempel: »\nobreak{}\oindex{IX., Alsergrund@\textbf{IX., Alsergrund}|pwk}Wien 9/2, 22 X 00, 12–\textcolor{gray}{1}\nobreak{}«. 
\newline{}Schnitzler: mit Bleistift datiert: »11/10 900.« \newline{}Ordnung: 1) mit Bleistift von unbekannter Hand nummeriert: »168« 2) mit Bleistift von unbekannter Hand nummeriert: »175«}\buchAbdrucke{\weitereDrucke{Hugo von Hofmannsthal, Arthur Schnitzler: \emph{Briefwechsel}. Hg. Therese Nickl und Heinrich Schnitzler. Frankfurt am Main: \emph{S. Fischer} 1964, S. 145.} }\pstart{}{\pb}\textsc{Herrn D\textsuperscript{r} Arthur Schnitzler}\pend{}\pstart{}\textsc{IX. Franckgasse 1.\oindex{Frankgasse@\textbf{Frankgasse}|pw}}\pend{}\pstart{}\textsc{Wien\oindex{Wien@\textbf{Wien}|pw}}\pend{}{\bigskip}\pstart
           \noindent{}{\pb}lieber Arthur, ich freue mich ſo ſehr Sie wieder zu ſehen. Ich werde
               Ihnen erzählen wie es kommt, daſs ich wenig Zeit habe. Heute abend kann ich nicht.
               Morgen möchte ich aber den Abend bei Richard\pwindex{Beer-Hofmann, Richard 11.07.1866 – 26.09.1945@\textsc{Beer-Hofmann, Richard} (11.07.1866 – 26.09.1945), \emph{Schriftsteller}|pw}{ }ſein und ſchon früh hinko{\geminationm}en. Bitte machen Sie es auch möglich.\pend
           \pstart
           Von Herzen{\\[\baselineskip]}Ihr\spacefill\mbox{Hugo}\pend
           \leftskip=0em{}\endnumbering\briefempfaengerindex{Schnitzler, Arthur@\textsc{Schnitzler, Arthur}!zzzHofmannsthal, Hugo von@\emph{von Hugo von Hofmannsthal}!1900-10-272@{27. 10. 1900}|)be}\mylabel{h}\end{ledgroupsized}  \newcommand{\dateiname}{L01081}\newcommand{\titel}{Hugo von Hofmannsthal an Arthur Schnitzler, 27. 10. 1900}\newcommand{\editorInnen}{Martin Anton Müller und Gerd-Hermann Susen}\input{../tex-inputs/latex-pdf-abspann}
      