%% latex-korrekturansicht-vorspann.tex
%% Vorspann für die Korrekturansicht.
%% Lädt die gemeinsame Datei latex-vorspann.tex mit gesetztem Schalter.

\newif\ifkorrekturansicht
\korrekturansichttrue

\input{../tex-inputs/latex-vorspann}


\section[Arthur Schnitzler an Franz Blei, 14. 12. 1909]{L01902 Arthur Schnitzler an Franz Blei, 14. 12. 1909}
\nopagebreak\mylabel{L01902v}
\rehead{ }\normalsize\beginnumbering\briefempfaengerindex{Blei, Franz@\textsc{Blei, Franz}!zzzSchnitzler, Arthur@\emph{von Arthur Schnitzler}!1909-12-142@{14. 12. 1909}|(be}
\toendnotes[C]{\smallbreak\pagebreak[2]}\Standort{DLA, A:Schnitzler, HS.NZ85.1.403.}
\physDesc{Brief, Durchschlag1 Blatt, 1 Seite, 483 Zeichen
\newline{}Schreibmaschine}\toendnotes[C]{\smallbreak}
\pstart
           \raggedleft{}{\pb}14. 12. 1909.\pend
           
\pstart{}Verehrtester Dr. Blei!\pend\vspace{0.5em}
\pstart
           Ich komme heute auf Ihre freundliche Aufforderung zur Mitarbeiterschaft am Hyperion\orgindex{Hyperion@Hyperion|pw} zurück. Teilen Sie mir gütigst recht bald
               mit, ob Sie prinzipiell geneigt wären das Vorspiel zu meinem neuen Stück\pwindex{junge Medardus. Dramatische Historie in einem Vorspiel und fuenf Aufzuegen@\emph{Der junge Medardus. Dramatische Historie in einem Vorspiel und fünf Aufzügen}|pwv}, eine so ziemlich in sich
               abgeschlossene Sache im Ganzen nahezu von der Ausdehnung des »Grünen Kakadu\pwindex{gruene Kakadu. Groteske in einem Akt@\emph{Der grüne Kakadu. Groteske in einem Akt}|pw}«, abzudrucken, und dafür im Annahmefall 1000 Mark
               zu zahlen.\pend
           
\pstart
           Mit verbindlichem Gruss{\\[\baselineskip]}Ihr ergebener\pend
           \leftskip=0em{}
\pstart
           \noindent{}Herrn Dr. Franz Blei, Herausgeber des Hyperion\orgindex{Hyperion@Hyperion|pw}, München\oindex{Muenchen@\textbf{München}, \emph{P.PPLA}|pw}.\pend
           \selectlanguage{ngerman}\endnumbering\briefempfaengerindex{Blei, Franz@\textsc{Blei, Franz}!zzzSchnitzler, Arthur@\emph{von Arthur Schnitzler}!1909-12-142@{14. 12. 1909}|)be}\mylabel{L01902h}  \normalsize

\doendnotes{C}
\bigskip
\vfill

\clearpage

\footnotesize

\lohead{\textsc{register}}

% Definiere theindex-Environment komplett neu ohne reledmac
\makeatletter
\renewenvironment{theindex}{%
  \section*{\indexname}%
  \setlength{\parindent}{0pt}%
  \setlength{\parskip}{0pt plus 0.3pt}%
  \let\item\@idxitem
}{%
  \clearpage
}
\makeatother

\IfFileExists{\jobname-pw.ind}{\input{\jobname-pw.ind}}{}

\end{document}

      