%% latex-leseansicht-vorspann.tex
%% Vorspann für die Leseansicht.
%% Lädt die gemeinsame Datei latex-vorspann.tex mit nicht gesetztem Schalter.

\newif\ifkorrekturansicht
\korrekturansichtfalse

\input{../tex-inputs/latex-vorspann}


\section[Stefan Zweig an Arthur Schnitzler, 6. {[}11.?{]} 1911]{L03630 Stefan Zweig an Arthur Schnitzler, 6. [11.?] 1911}
\nopagebreak\mylabel{L03630v}
\rehead{ }\normalsize\beginnumbering\briefempfaengerindex{Schnitzler, Arthur@\textsc{Schnitzler, Arthur}!zzzZweig, Stefan@\emph{von Stefan Zweig}!1911-11-061@{6. [11.?] 1911}|(be}
\toendnotes[C]{\smallbreak\pagebreak[2]}
\correspDesc{Versand  durch Stefan Zweig am 6. [11.?] 1911 in Wien
\newline{}Weiterleitung  im Zeitraum [6. 11. 1911
                  – 7. 11. 1911?] in Wien
\newline{}Erhalt  durch Arthur Schnitzler im Zeitraum [8. 11. 1911
                  – 9. 11. 1911?] in Hamburg}\toendnotes[C]{\smallbreak}
\Standort{CUL, Schnitzler, B 118.}
\physDesc{Bildpostkarte, 622 Zeichen
\newline{}Handschrift: schwarze Tinte, lateinische Kurrent
\newline{}Versand: Stempel: »\nobreak{}\oindex{VIII., Josefstadt@\textbf{VIII., Josefstadt}, \emph{Verwaltungsgebiet}|pwk}8/\textcolor{gray}{×} Wien, 6. \textcolor{gray}{X}I. 11, 5\nobreak{}«.  }
\buchAbdrucke{\weitereDrucke{Stefan Zweig: \emph{Briefwechsel mit Hermann Bahr, Sigmund Freud, Rainer Maria
                        Rilke und Arthur Schnitzler}. Herausgegeben von Jeffrey B. Berlin, Hans-Ulrich Lindken und Donald A. Prater. Frankfurt am Main: \emph{S. Fischer} 1987, S. 367–368.} }\toendnotes[C]{\smallbreak}\pstart{}{\pb}D\textsuperscript{r} Artur
                  Schnitzler\pend{}\pstart{}Wien – Cottage\oindex{Wien@\textbf{Wien}!XVIII., Währing@\textbf{XVIII., Währing}!Währinger Cottage@\textbf{Währinger Cottage}, \emph{Teil eines besiedelten Ortes}|pw}\pend{}\pstart{}\strikeout{Cott}{ }\label{K_L03630-1v}\edtext{Sternwartestrasse 72}{\lemma{\textnormal{\emph{Sternwartestrasse 72}}}\Cendnote{\textnormal{Zweig\pwindex{Zweig, Stefan 28.\,11.\,1881 Wien – 23.\,2.\,1942 Petrópolis@\textsc{Zweig, Stefan} (28.\,11.\,1881 Wien – 23.\,2.\,1942 Petrópolis), \emph{Schriftsteller}|pwk} wechselt bei der Adressierung
                        seiner Schreiben an Schnitzler immer
                        wieder zwischen der falschen Hausnummer »72« und der
                        richtigen »71«.}}}\label{K_L03630-1}\oindex{Wien@\textbf{Wien}!XVIII., Währing@\textbf{XVIII., Währing}!Sternwartestraße 71@\textbf{Sternwartestraße 71}, \emph{Wohngebäude}|pw}\pend{}{\bigskip}
\pstart
           {\pb}\textcolor{gray}{\textbf{WIEN}}\oindex{Wien@\textbf{Wien}, \emph{Verwaltungsgebiet}|pw}\hfill \textcolor{gray}{\textbf{Justiz-Palast}}\oindex{Wien@\textbf{Wien}!I., Innere Stadt@\textbf{I., Innere Stadt}!Justizpalast Wien@\textbf{Justizpalast Wien}, \emph{Gebäude}|pw}\pend
           \vspace{1em}
\pstart
           \noindent{}{\pb}Verehrter Herr Doktor,{ }\label{K_L03630-2v}\edtext{Paul Morisse\pwindex{Morisse, Paul 11.\,3.\,1866 Rouen – 28.\,9.\,1946 Paris@\textsc{Morisse, Paul} (11.\,3.\,1866 Rouen – 28.\,9.\,1946 Paris), \emph{Übersetzer}|pw}}{\lemma{\textnormal{\emph{Paul Morisse}}}\Cendnote{\textnormal{Nach der ersten Kontaktaufnahme im
                     Februar 1911 (siehe XXXX Auszeichnungsfehler: Dokument L03636 nicht gefunden) betrieb Morisse\pwindex{Morisse, Paul 11.\,3.\,1866 Rouen – 28.\,9.\,1946 Paris@\textsc{Morisse, Paul} (11.\,3.\,1866 Rouen – 28.\,9.\,1946 Paris), \emph{Übersetzer}|pwk} den
                  Plan der Übersetzung von \emph{Das weite Land}\pwindex{Schnitzler, Arthur 15.\,5.\,1862 Wien – 21.\,10.\,1931 ebd.@\textsc{Schnitzler, Arthur} (15.\,5.\,1862 Wien – 21.\,10.\,1931 ebd.), \emph{Schriftsteller, Mediziner}!weite Land. Tragikomödie in fünf Akten@\strich\emph{Das weite Land. Tragikomödie in fünf Akten}|pwk} in
                  den folgenden Monaten ernsthafter. Er nahm Kontakt mit S. Fischer\pwindex{Fischer, Samuel 24.\,12.\,1859 Liptovský Mikuláš – 15.\,10.\,1934 Berlin@\textsc{Fischer, Samuel} (24.\,12.\,1859 Liptovský Mikuláš – 15.\,10.\,1934 Berlin), \emph{Verleger}|pwk} auf und bekam am 17. 1.1912 die formelle Erlaubnis für die Übersetzung
                  von Schnitzler. (Schnitzler traf seine Entscheidung nach Rücksprache mit André Antoine\pwindex{Antoine, André 31.\,1.\,1858 Limoges – 23.\,10.\,1943 Le Pouliguen@\textsc{Antoine, André} (31.\,1.\,1858 Limoges – 23.\,10.\,1943 Le Pouliguen), \emph{Theaterleiter, Schauspieler}|pwk}, weil auch Maurice Rémon\pwindex{Rémon, Maurice 27.\,11.\,1861 Paris – 20.\,6.\,1945 Mérignac@\textsc{Rémon, Maurice} (27.\,11.\,1861 Paris – 20.\,6.\,1945 Mérignac), \emph{Übersetzer}|pwk} die Übersetzungsrechte erbeten hatte.) 
                  Zugleich versuchte Morisse\pwindex{Morisse, Paul 11.\,3.\,1866 Rouen – 28.\,9.\,1946 Paris@\textsc{Morisse, Paul} (11.\,3.\,1866 Rouen – 28.\,9.\,1946 Paris), \emph{Übersetzer}|pwk}, ein Theater
                  für die Inszenierung zu finden. Für die Übersetzungsarbeit sicherte er sich eine
                  Mitarbeiterin, Henriette Charasson\pwindex{Charasson, Henriette 6.\,1.\,1884 Le Havre – 24.\,12.\,1972 Châteauroux@\textsc{Charasson, Henriette} (6.\,1.\,1884 Le Havre – 24.\,12.\,1972 Châteauroux), \emph{Schriftstellerin}|pwk}. Außer
                  einer Zeitungsmeldung, in der die Übersetzung unter dem Titel »\emph{le Pays mystérieux}\pwindex{Schnitzler, Arthur 15.\,5.\,1862 Wien – 21.\,10.\,1931 ebd.@\textsc{Schnitzler, Arthur} (15.\,5.\,1862 Wien – 21.\,10.\,1931 ebd.), \emph{Schriftsteller, Mediziner}!Le Pays Inconnu@\strich\emph{Le Pays Inconnu}|pwuk}« angekündigt wurde, scheint
                  sich die Sache schnell zerschlagen zu haben. Im Nachlass Schnitzlers in der \emph{Cambridge University Library} finden sich in der Mappe 244 mehrere Durchschläge einer französischen
                  Übersetzung, bei der kein finaler Titel, sondern nur handschriftliche Titelangaben
                  angebracht wurden: »Le Pays Inconnu\pwindex{Schnitzler, Arthur 15.\,5.\,1862 Wien – 21.\,10.\,1931 ebd.@\textsc{Schnitzler, Arthur} (15.\,5.\,1862 Wien – 21.\,10.\,1931 ebd.), \emph{Schriftsteller, Mediziner}!Le Pays Inconnu@\strich\emph{Le Pays Inconnu}|pw}«, »Le Pays de l’Ame\pwindex{Schnitzler, Arthur 15.\,5.\,1862 Wien – 21.\,10.\,1931 ebd.@\textsc{Schnitzler, Arthur} (15.\,5.\,1862 Wien – 21.\,10.\,1931 ebd.), \emph{Schriftsteller, Mediziner}!Le Pays Inconnu@\strich\emph{Le Pays Inconnu}|pw}« und »Le Pays Lontain\pwindex{Schnitzler, Arthur 15.\,5.\,1862 Wien – 21.\,10.\,1931 ebd.@\textsc{Schnitzler, Arthur} (15.\,5.\,1862 Wien – 21.\,10.\,1931 ebd.), \emph{Schriftsteller, Mediziner}!Le Pays Inconnu@\strich\emph{Le Pays Inconnu}|pw}«. Ob es sich dabei um die Übersetzung von Morisse\pwindex{Morisse, Paul 11.\,3.\,1866 Rouen – 28.\,9.\,1946 Paris@\textsc{Morisse, Paul} (11.\,3.\,1866 Rouen – 28.\,9.\,1946 Paris), \emph{Übersetzer}|pwk}/Charasson\pwindex{Charasson, Henriette 6.\,1.\,1884 Le Havre – 24.\,12.\,1972 Châteauroux@\textsc{Charasson, Henriette} (6.\,1.\,1884 Le Havre – 24.\,12.\,1972 Châteauroux), \emph{Schriftstellerin}|pwk}
                  handelt, ist unklar.}}}\label{K_L03630-2}, dem ich seinerzeit das »Weite Land\pwindex{Schnitzler, Arthur 15.\,5.\,1862 Wien – 21.\,10.\,1931 ebd.@\textsc{Schnitzler, Arthur} (15.\,5.\,1862 Wien – 21.\,10.\,1931 ebd.), \emph{Schriftsteller, Mediziner}!weite Land. Tragikomödie in fünf Akten@\strich\emph{Das weite Land. Tragikomödie in fünf Akten}|pw}« zur Übertragung empfahl, möchte gern an das Werk\pwindex{Schnitzler, Arthur 15.\,5.\,1862 Wien – 21.\,10.\,1931 ebd.@\textsc{Schnitzler, Arthur} (15.\,5.\,1862 Wien – 21.\,10.\,1931 ebd.), \emph{Schriftsteller, Mediziner}!Le Pays Inconnu@\strich\emph{Le Pays Inconnu}|pwv} gehen. Ich will Ihnen
               heute nur wiederholen, dass M.\pwindex{Morisse, Paul 11.\,3.\,1866 Rouen – 28.\,9.\,1946 Paris@\textsc{Morisse, Paul} (11.\,3.\,1866 Rouen – 28.\,9.\,1946 Paris), \emph{Übersetzer}|pw} sowohl deutsch
               wie französisch glänzend beherrscht und ein ernster tüchtiger Übersetzer mit vielen
               literarischen Beziehungen ist, den ich Ihnen auf das wärmste empfehlen kann. Ich
               reise heute nach Meran\oindex{Meran@\textbf{Meran}, \emph{Hauptstadt}|pw}, obwohl es mir gar nicht
               schlecht geht.\pend
           
\pstart
           Das Haus am Meer\pwindex{Zweig, Stefan 28.\,11.\,1881 Wien – 23.\,2.\,1942 Petrópolis@\textsc{Zweig, Stefan} (28.\,11.\,1881 Wien – 23.\,2.\,1942 Petrópolis), \emph{Schriftsteller}!Haus am Meer. Ein Schauspiel in zwei Teilen (drei Aufzügen)@\strich\emph{Das Haus am Meer. Ein Schauspiel in zwei Teilen (drei Aufzügen)}|pw} ist von einem halben Dutzend
               erster Bühnen bereits erworben.\pend
           \pstart {\pb}Mit vielen Grüssen an Ihre Frau Gemahlin\pwindex{Schnitzler, Olga 17.\,1.\,1882 Wien – 13.\,1.\,1970 Lugano@\textsc{Schnitzler, Olga} (17.\,1.\,1882 Wien – 13.\,1.\,1970 Lugano), \emph{Schauspielerin, Sängerin}|pwv} und Sie Ihr stets
               getreuer \spacefill\mbox{Stefan Zweig}\pend{}\selectlanguage{ngerman}\endnumbering\briefempfaengerindex{Schnitzler, Arthur@\textsc{Schnitzler, Arthur}!zzzZweig, Stefan@\emph{von Stefan Zweig}!1911-11-061@{6. [11.?] 1911}|)be}\mylabel{L03630h}  \newcommand{\dateiname}{L03630}\newcommand{\titel}{Stefan Zweig an Arthur Schnitzler, 6. [11.?] 1911}\newcommand{\editorInnen}{Selma Jahnke und Martin Anton Müller}%% latex-leseansicht-abspann.tex
%% Abspann für die Leseansicht.
%% Der Schalter \ifkorrekturansicht ist bereits durch den Vorspann gesetzt.

%% latex-abspann.tex
%% Gemeinsamer Abspann für Korrekturansicht und Leseansicht.
%% Setzt den Schalter \ifkorrekturansicht voraus (gesetzt in den
%% einbindenden Dateien latex-korrekturansicht-abspann.tex bzw.
%% latex-leseansicht-abspann.tex).
%% ---------------------------------------------------------------

\normalsize

% Das esempio-Environment wird nur in der Leseansicht benötigt
\ifkorrekturansicht\else
\newenvironment{esempio}[3]%
{
    \vspace{1.5ex}
    \rlap{\underline{#1}}
    \par
    \setlength{\parindent}{0cm}
    \nopagebreak
    \leftskip=#2cm
    \rightskip=#3cm
}
{
    \par
}
\fi

\doendnotes{C}
\bigskip
\vfill

\clearpage

\footnotesize

\ifkorrekturansicht
  \lohead{\textsc{register}}
\fi

% theindex-Environment neu definieren ohne reledmac
\makeatletter
\renewenvironment{theindex}{%
  \ifkorrekturansicht
    \section*{\indexname}%
  \else
    \subsubsection*{Index der erwähnten Entitäten}%
  \fi
  \setlength{\parindent}{0pt}%
  \setlength{\parskip}{0pt plus 0.3pt}%
  \let\item\@idxitem
}{%
  \ifkorrekturansicht\clearpage\fi
}
\makeatother

\IfFileExists{\jobname-pw.ind}{\input{\jobname-pw.ind}}{}

% Quellenangabe nur in der Leseansicht
\ifkorrekturansicht\else
% Fallback-Definitionen, falls die .tex-Datei \titel etc. nicht gesetzt hat
\providecommand{\titel}{}
\providecommand{\editorInnen}{}
\providecommand{\dateiname}{\jobname}

\vspace{3cm}

\vfill

\footnotesize
\textsc{Quelle}: \titel. Herausgegeben von {\editorInnen}. In: \emph{Arthur Schnitzler: Briefwechsel mit Autorinnen und Autoren}.
 Digitale Edition, https://schnitzler-briefe.acdh.oeaw.ac.at/{\dateiname}.html (Stand \today)
\fi

\end{document}


