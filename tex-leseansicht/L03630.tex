%% latex-korrekturansicht-vorspann.tex
%% Vorspann für die Korrekturansicht.
%% Lädt die gemeinsame Datei latex-vorspann.tex mit gesetztem Schalter.

\newif\ifkorrekturansicht
\korrekturansichttrue

\input{../tex-inputs/latex-vorspann}


\section[Stefan Zweig an Arthur Schnitzler, 6. {[}11.?{]} 1911]{L03630 Stefan Zweig an Arthur Schnitzler, 6. {[}11.?{]} 1911}
\nopagebreak\mylabel{L03630v}
\rehead{ }\normalsize\beginnumbering\briefempfaengerindex{Schnitzler, Arthur@\textsc{Schnitzler, Arthur}!zzzZweig, Stefan@\emph{von Stefan Zweig}!1911-11-061@{6. {[}11.?{]} 1911}|(be}
\toendnotes[C]{\smallbreak\pagebreak[2]}\Standort{CUL, Schnitzler, B 118.}
\physDesc{Bildpostkarte, 622 Zeichen
\newline{}Handschrift: schwarze Tinte, lateinische Kurrent
\newline{}Versand: Stempel: »\nobreak{}\oindex{VIII., Josefstadt@\textbf{VIII., Josefstadt}, \emph{A.ADM3}|pwk}8/\textcolor{gray}{×} Wien, 6. \textcolor{gray}{X}I. 11, 5\nobreak{}«.  }
\buchAbdrucke{\weitereDrucke{Stefan Zweig: \emph{Briefwechsel mit Hermann Bahr, Sigmund Freud, Rainer Maria
                        Rilke und Arthur Schnitzler}. Frankfurt am Main: \emph{S. Fischer} 1987, S. 367–368.} }\toendnotes[C]{\smallbreak}\pstart{}{\pb}D\textsuperscript{r} Artur
                  Schnitzler\pend{}\pstart{}Wien – Cottage\oindex{Waehringer Cottage@\textbf{Währinger Cottage}, \emph{Teil eines besiedelten Ortes (A.BSOX)}|pw}\pend{}\pstart{}\strikeout{Cott}{ }\label{K_L03630-1v}\edtext{Sternwartestrasse 72}{\lemma{\textnormal{\emph{Sternwartestrasse 72}}}\Cendnote{\textnormal{Zweig\pwindex{Zweig, Stefan 28.11.1881 – 23.02.1942@\textsc{Zweig, Stefan} (28.11.1881 – 23.02.1942), \emph{Schriftsteller/Schriftstellerin}|pwk} wechselt bei der Adressierung
                        seiner Schreiben an Schnitzler immer
                        wieder zwischen der falschen Hausnummer »72« und der
                        richtigen »71«.}}}\label{K_L03630-1}\oindex{Sternwartestrasse 71@\textbf{Sternwartestraße 71}, \emph{Wohngebäude (K.WHS)}|pw}\pend{}{\bigskip}
\pstart
           {\pb}\textcolor{gray}{\textbf{WIEN}}\oindex{Wien@\textbf{Wien}, \emph{A.ADM2}|pw}\hfill \textcolor{gray}{\textbf{Justiz-Palast}}\oindex{Justizpalast Wien@\textbf{Justizpalast Wien}, \emph{Gebäude (K.GBD)}|pw}\pend
           \vspace{1em}
\pstart
           \noindent{}{\pb}Verehrter Herr Doktor,{ }\label{K_L03630-2v}\edtext{Paul Morisse\pwindex{Morisse, Paul 1866-03-11 – 1946-09-28@\textsc{Morisse, Paul} (1866-03-11 – 1946-09-28), \emph{Übersetzer/Übersetzerin}|pw}}{\lemma{\textnormal{\emph{Paul Morisse}}}\Cendnote{\textnormal{Nach der ersten Kontaktaufnahme im
                     Februar 1911 (siehe Stefan Zweig an Arthur Schnitzler, 21. 2. 1911) betrieb Morisse\pwindex{Morisse, Paul 1866-03-11 – 1946-09-28@\textsc{Morisse, Paul} (1866-03-11 – 1946-09-28), \emph{Übersetzer/Übersetzerin}|pwk} den
                  Plan der Übersetzung von \emph{Das weite Land}\pwindex{weite Land. Tragikomoedie in fuenf Akten@\emph{Das weite Land. Tragikomödie in fünf Akten}|pwk} in
                  den folgenden Monaten ernsthafter. Er nahm Kontakt mit S. Fischer\pwindex{Fischer, Samuel 24.12.1859 – 15.10.1934@\textsc{Fischer, Samuel} (24.12.1859 – 15.10.1934), \emph{Verleger/Verlegerin}|pwk} auf und bekam die Erlaubnis für die Übersetzung
                  von Schnitzler. (Schnitzler traf seine Entscheidung nach 
                  Rücksprache mit André Antoine\pwindex{Antoine, Andre 1858-01-31 – 1943-10-23@\textsc{Antoine, André} (1858-01-31 – 1943-10-23), \emph{Theaterleiter/Theaterleiterin, Schauspieler/Schauspielerin}|pwk}, weil auch Maurice Rémon\pwindex{Remon, Maurice 27.11.1861 – 20.06.1945@\textsc{Rémon, Maurice} (27.11.1861 – 20.06.1945), \emph{Übersetzer/Übersetzerin}|pwk} die Übersetzungsrechte erbeten
                  hatte.) Zugleich versuchte Morisse\pwindex{Morisse, Paul 1866-03-11 – 1946-09-28@\textsc{Morisse, Paul} (1866-03-11 – 1946-09-28), \emph{Übersetzer/Übersetzerin}|pwk}, ein
                  Theater für die Inszenierung zu finden. Für die Übersetzungsarbeit sicherte er
                  sich eine Mitarbeiterin, Henriette
                     Charasson\pwindex{Charasson, Henriette 1884-01-06 – 1972-12-24@\textsc{Charasson, Henriette} (1884-01-06 – 1972-12-24), \emph{Schriftsteller/Schriftstellerin}|pwk}. Außer einer Zeitungsmeldung, in der die Übersetzung unter dem
                  Titel »\emph{le Pays mystérieux}\pwindex{Le Pays Inconnu@\emph{Le Pays Inconnu}|pwuk}«
                  angekündigt wurde, scheint sich die Sache schnell zerschlagen zu haben. Im
                  Nachlass Schnitzlers in der \emph{Cambridge University Library} finden sich in der Mappe 244 mehrere Durchschläge einer französischen
                  Übersetzung, bei der kein finaler Titel, sondern nur handschriftliche Titelangaben
                  angebracht wurden: »Le Pays Inconnu\pwindex{Le Pays Inconnu@\emph{Le Pays Inconnu}|pw}«, »Le Pays de l’Ame\pwindex{Le Pays Inconnu@\emph{Le Pays Inconnu}|pw}« und »Le Pays Lontain\pwindex{Le Pays Inconnu@\emph{Le Pays Inconnu}|pw}«. Ob es sich dabei um die Übersetzung von Morisse\pwindex{Morisse, Paul 1866-03-11 – 1946-09-28@\textsc{Morisse, Paul} (1866-03-11 – 1946-09-28), \emph{Übersetzer/Übersetzerin}|pwk}/Charasson\pwindex{Charasson, Henriette 1884-01-06 – 1972-12-24@\textsc{Charasson, Henriette} (1884-01-06 – 1972-12-24), \emph{Schriftsteller/Schriftstellerin}|pwk}
                  handelt, ist unklar.}}}\label{K_L03630-2}, dem ich seinerzeit das »Weite Land\pwindex{weite Land. Tragikomoedie in fuenf Akten@\emph{Das weite Land. Tragikomödie in fünf Akten}|pw}« zur Übertragung empfahl, möchte gern an das Werk\pwindex{Le Pays Inconnu@\emph{Le Pays Inconnu}|pwv} gehen. Ich will Ihnen
               heute nur wiederholen, dass M.\pwindex{Morisse, Paul 1866-03-11 – 1946-09-28@\textsc{Morisse, Paul} (1866-03-11 – 1946-09-28), \emph{Übersetzer/Übersetzerin}|pw} sowohl deutsch
               wie französisch glänzend beherrscht und ein ernster tüchtiger Übersetzer mit vielen
               literarischen Beziehungen ist, den ich Ihnen auf das wärmste empfehlen kann. Ich
               reise heute nach Meran\oindex{Meran@\textbf{Meran}, \emph{P.PPLA3}|pw}, obwohl es mir gar nicht
               schlecht geht.\pend
           
\pstart
           Das Haus am Meer\pwindex{Haus am Meer. Ein Schauspiel in zwei Teilen (drei Aufzuegen)@\emph{Das Haus am Meer. Ein Schauspiel in zwei Teilen (drei Aufzügen)}|pw} ist von einem halben Dutzend
               erster Bühnen bereits erworben.\pend
           \pstart {\pb}Mit vielen Grüssen an Ihre Frau Gemahlin\pwindex{Schnitzler, Olga 17.01.1882 – 13.01.1970@\textsc{Schnitzler, Olga} (17.01.1882 – 13.01.1970), \emph{Schauspieler/Schauspielerin, Sänger/Sängerin}|pwv} und Sie Ihr stets
               getreuer \spacefill\mbox{Stefan Zweig}\pend{}\selectlanguage{ngerman}\endnumbering\briefempfaengerindex{Schnitzler, Arthur@\textsc{Schnitzler, Arthur}!zzzZweig, Stefan@\emph{von Stefan Zweig}!1911-11-061@{6. {[}11.?{]} 1911}|)be}\mylabel{L03630h}  \normalsize

\doendnotes{C}
\bigskip
\vfill

\clearpage

\footnotesize

\lohead{\textsc{register}}

% Definiere theindex-Environment komplett neu ohne reledmac
\makeatletter
\renewenvironment{theindex}{%
  \section*{\indexname}%
  \setlength{\parindent}{0pt}%
  \setlength{\parskip}{0pt plus 0.3pt}%
  \let\item\@idxitem
}{%
  \clearpage
}
\makeatother

\IfFileExists{\jobname-pw.ind}{\input{\jobname-pw.ind}}{}

\end{document}

      