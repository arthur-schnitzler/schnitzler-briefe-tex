%% latex-leseansicht-vorspann.tex
%% Vorspann für die Leseansicht.
%% Lädt die gemeinsame Datei latex-vorspann.tex mit nicht gesetztem Schalter.

\newif\ifkorrekturansicht
\korrekturansichtfalse

\input{../tex-inputs/latex-vorspann}


\section[Karl Kraus an Arthur Schnitzler, 5. 5. 1893]{L00212 Karl Kraus an Arthur Schnitzler, 5. 5. 1893}
\nopagebreak\mylabel{L00212v}
\rehead{ }\normalsize\beginnumbering\briefempfaengerindex{Schnitzler, Arthur@\textsc{Schnitzler, Arthur}!zzzKraus, Karl@\emph{von Karl Kraus}!1893-05-051@{5. 5. 1893}|(be}
\toendnotes[C]{\smallbreak\pagebreak[2]}
\correspDesc{Versand  durch Karl Kraus am 5. 5. 1893 in Wien
\newline{}Erhalt  durch Arthur Schnitzler im Zeitraum [5. 5. 1893
                  – 9. 5. 1893?] in Wien}\toendnotes[C]{\smallbreak}
\Standort{DLA, A:Schnitzler, HS.NZ85.1.3790, S. 11–12.}
\physDesc{Brief, maschinenschriftliche Abschrift, 1 Blatt, 1 Seite, 338 Zeichen
\newline{}Schreibmaschine}
\buchAbdrucke{\weitereDrucke{\emph{Karl Kraus und Arthur Schnitzler. Eine Dokumentation.}Herausgegeben von Reinhard Urbach In: \emph{Literatur und Kritik}, Bd. 49, Oktober 1970, S. 518.} }\toendnotes[C]{\smallbreak}
\pstart
           \raggedleft{}{\pb}Wien\oindex{Wien@\textbf{Wien}, \emph{Verwaltungsgebiet}|pw}, 5. Mai 1893.\pend
           \vspace{0.5em}
\pstart
           Liebster Herr Doctor! Beiliegend sende ich Ihnen den Kritikausschnitt\pwindex{Kraus, Karl 28.\,4.\,1874 Jičín – 12.\,6.\,1936 Wien@\textsc{Kraus, Karl} (28.\,4.\,1874 Jičín – 12.\,6.\,1936 Wien), \emph{Schriftsteller, Publizist, Schriftsteller}!Wiener Dichter@\strich\emph{Wiener Dichter}|pwv} aus Nr. 18 des Magazin\pwindex{Magazin für die Literatur des Auslandes@\emph{Magazin für die Literatur des Auslandes}|pw} (6. Mai), das mir eben
               zuging. – Leider konnte ich gestern{ }½ 10 nicht im Trauerhause\oindex{Wien@\textbf{Wien}!I., Innere Stadt@\textbf{I., Innere Stadt}!Burgring@\textbf{Burgring}, \emph{Straße}|pwv} erscheinen, da ich die Parte erst vormit{\pb}tags gestern erhielt. Nochmals auf diesem Wege mein
               herzlichstes Beileid\pwindex{Schnitzler, Johann 10.\,4.\,1835 Nagykanizsa – 2.\,5.\,1893 Wien@\textsc{Schnitzler, Johann} (10.\,4.\,1835 Nagykanizsa – 2.\,5.\,1893 Wien), \emph{Laryngologe}|pwv} und
               viel Grüsse von Ihrem\pend
           \pstart treuen \spacefill\mbox{Karl Kraus}\pend{}\selectlanguage{ngerman}\endnumbering\briefempfaengerindex{Schnitzler, Arthur@\textsc{Schnitzler, Arthur}!zzzKraus, Karl@\emph{von Karl Kraus}!1893-05-051@{5. 5. 1893}|)be}\mylabel{L00212h}  \newcommand{\dateiname}{L00212}\newcommand{\titel}{Karl Kraus an Arthur Schnitzler, 5. 5. 1893}\newcommand{\editorInnen}{Martin Anton Müller und Gerd-Hermann Susen}%% latex-leseansicht-abspann.tex
%% Abspann für die Leseansicht.
%% Der Schalter \ifkorrekturansicht ist bereits durch den Vorspann gesetzt.

%% latex-abspann.tex
%% Gemeinsamer Abspann für Korrekturansicht und Leseansicht.
%% Setzt den Schalter \ifkorrekturansicht voraus (gesetzt in den
%% einbindenden Dateien latex-korrekturansicht-abspann.tex bzw.
%% latex-leseansicht-abspann.tex).
%% ---------------------------------------------------------------

\normalsize

% Das esempio-Environment wird nur in der Leseansicht benötigt
\ifkorrekturansicht\else
\newenvironment{esempio}[3]%
{
    \vspace{1.5ex}
    \rlap{\underline{#1}}
    \par
    \setlength{\parindent}{0cm}
    \nopagebreak
    \leftskip=#2cm
    \rightskip=#3cm
}
{
    \par
}
\fi

\doendnotes{C}
\bigskip
\vfill

\clearpage

\footnotesize

\ifkorrekturansicht
  \lohead{\textsc{register}}
\fi

% theindex-Environment neu definieren ohne reledmac
\makeatletter
\renewenvironment{theindex}{%
  \ifkorrekturansicht
    \section*{\indexname}%
  \else
    \subsubsection*{Index der erwähnten Entitäten}%
  \fi
  \setlength{\parindent}{0pt}%
  \setlength{\parskip}{0pt plus 0.3pt}%
  \let\item\@idxitem
}{%
  \ifkorrekturansicht\clearpage\fi
}
\makeatother

\IfFileExists{\jobname-pw.ind}{\input{\jobname-pw.ind}}{}

% Quellenangabe nur in der Leseansicht
\ifkorrekturansicht\else
% Fallback-Definitionen, falls die .tex-Datei \titel etc. nicht gesetzt hat
\providecommand{\titel}{}
\providecommand{\editorInnen}{}
\providecommand{\dateiname}{\jobname}

\vspace{3cm}

\vfill

\footnotesize
\textsc{Quelle}: \titel. Herausgegeben von {\editorInnen}. In: \emph{Arthur Schnitzler: Briefwechsel mit Autorinnen und Autoren}.
 Digitale Edition, https://schnitzler-briefe.acdh.oeaw.ac.at/{\dateiname}.html (Stand \today)
\fi

\end{document}


