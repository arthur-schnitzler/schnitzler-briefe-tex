%% latex-korrekturansicht-vorspann.tex
%% Vorspann für die Korrekturansicht.
%% Lädt die gemeinsame Datei latex-vorspann.tex mit gesetztem Schalter.

\newif\ifkorrekturansicht
\korrekturansichttrue

\input{../tex-inputs/latex-vorspann}


\section[Karl Kraus an Arthur Schnitzler, 5. 5. 1893]{L00212 Karl Kraus an Arthur Schnitzler, 5. 5. 1893}
\nopagebreak\mylabel{L00212v}
\rehead{ }\normalsize\beginnumbering\briefempfaengerindex{Schnitzler, Arthur@\textsc{Schnitzler, Arthur}!zzzKraus, Karl@\emph{von Karl Kraus}!1893-05-051@{5. 5. 1893}|(be}
\toendnotes[C]{\smallbreak\pagebreak[2]}\Standort{DLA, A:Schnitzler, HS.NZ85.1.3790, S. 11–12.}
\physDesc{Brief, maschinenschriftliche Abschrift1 Blatt, 1 Seite, 338 Zeichen
\newline{}Schreibmaschine}
\buchAbdrucke{\weitereDrucke{\emph{Literatur und Kritik}, Bd. 49, Oktober 1970, S. 518.} }\toendnotes[C]{\smallbreak}
\pstart
           \raggedleft{}{\pb}Wien\oindex{Wien@\textbf{Wien}, \emph{A.ADM2}|pw}, 5. Mai 1893. \pend
           \vspace{0.5em}
\pstart
           Liebster Herr Doctor! Beiliegend sende ich Ihnen den Kritikausschnitt\pwindex{Wiener Dichter@\emph{Wiener Dichter}|pwv} aus Nr. 18 des Magazin\pwindex{Magazin fuer die Literatur des Auslandes@\emph{Magazin für die Literatur des Auslandes}|pw} (6. Mai), das mir eben
               zuging. – Leider konnte ich gestern{ }½ 10 nicht im Trauerhause\oindex{Burgring@\textbf{Burgring}, \emph{Straße (K.STR)}|pwv} erscheinen, da ich die Parte erst vormit{\pb}tags gestern erhielt. Nochmals auf diesem Wege mein
               herzlichstes Beileid\pwindex{Schnitzler, Johann 10.04.1835 – 02.05.1893@\textsc{Schnitzler, Johann} (10.04.1835 – 02.05.1893), \emph{Laryngologe/Laryngologin}|pwv} und
               viel Grüsse von Ihrem\pend
           \pstart treuen \spacefill\mbox{Karl Kraus}\pend{}\selectlanguage{ngerman}\endnumbering\briefempfaengerindex{Schnitzler, Arthur@\textsc{Schnitzler, Arthur}!zzzKraus, Karl@\emph{von Karl Kraus}!1893-05-051@{5. 5. 1893}|)be}\mylabel{L00212h}  \normalsize

\doendnotes{C}
\bigskip
\vfill

\clearpage

\footnotesize

\lohead{\textsc{register}}

% Definiere theindex-Environment komplett neu ohne reledmac
\makeatletter
\renewenvironment{theindex}{%
  \section*{\indexname}%
  \setlength{\parindent}{0pt}%
  \setlength{\parskip}{0pt plus 0.3pt}%
  \let\item\@idxitem
}{%
  \clearpage
}
\makeatother

\IfFileExists{\jobname-pw.ind}{\input{\jobname-pw.ind}}{}

\end{document}

      