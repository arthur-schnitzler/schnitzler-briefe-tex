%% latex-leseansicht-vorspann.tex
%% Vorspann für die Leseansicht.
%% Lädt die gemeinsame Datei latex-vorspann.tex mit nicht gesetztem Schalter.

\newif\ifkorrekturansicht
\korrekturansichtfalse

\input{../tex-inputs/latex-vorspann}

\begin{center}
            \textcolor{red}{ENTWURF, NICHT FERTIG KORRIGIERT}
                      \end{center}
            
         
         \renewcommand{\erwaehntePersonen}{Personen:  Luise von Mecklenburg-Strelitz, Olga Schnitzler}
         \renewcommand{\erwaehnteOrte}{Orte: Edmund-Weiß-Gasse 7, Königs Wusterhausen, Pfuhl’s Hotel, Wien, XVIII., Währing}
         \renewcommand{\erwaehnteWerke}{}
               \section[ Felix Salten u. a. an Arthur Schnitzler, 4. 6. 1906]{ Felix Salten u. a. an Arthur Schnitzler, 4. 6. 1906}\nopagebreak\mylabel{v}\rehead{ }\begin{ledgroupsized}[t]{13cm}\normalsize\beginnumbering \toendnotes[C]{\smallbreak\pagebreak[2]} \Standort{CUL, Schnitzler, B 89, B 1.}
\physDesc{Bildpostkarte, 258 Zeichen
\newline{}Handschrift Felix Salten: schwarze Tinte, lateinische Kurrent\newline{}Handschrift Ottilie Salten: schwarze Tinte\newline{}Handschrift Samuel Fischer: schwarze Tinte, lateinische Kurrent\newline{}Handschrift Hedwig Fischer: schwarze Tinte, deutsche Kurrent\newline{}Handschrift  ?? [Frau in Königs Wusterhausen]: blauer Buntstift, lateinische Kurrent
\newline{}Versand: Stempel: »\nobreak{}\oindex{Koenigs Wusterhausen@\textbf{Königs Wusterhausen}|pwk}Königs-Wusterhaus\textcolor{gray}{en} b, 4. 6. 06, 8–9 N.\nobreak{}«.  
\newline{}Schnitzler: mit Bleistift datiert: »4/6« 
\newline{}Ordnung: mit Bleistift von unbekannter Hand nummeriert: »217« }\toendnotes[C]{\smallbreak}\pstart{}{\pb}Herrn D\textsuperscript{r} Arthur Schnitzler\pend{}\pstart{}Wien XVIII.\oindex{XVIII., Waehring@\textbf{XVIII., Währing}|pw}\pend{}\pstart{}Spöttelgaſse 7\oindex{Edmund-Weiss-Gasse 7@\textbf{Edmund-Weiß-Gasse 7}|pw}\pend{}{\bigskip}\pstart
           \noindent{}\centering{}{\pb}\textcolor{gray}{\textbf{Gruss aus Königs-Wusterhausen\oindex{Koenigs Wusterhausen@\textbf{Königs Wusterhausen}|pw}}}\pend
           \pstart
           \noindent{}\centering{}\textcolor{gray}{\textbf{Historisches Buffet aus der Zeit der Königin Louise\pwindex{Luise von Mecklenburg-Strelitz 1776-03-10 – 1810-07-19@\textsc{Luise von Mecklenburg-Strelitz} (1776-03-10 – 1810-07-19), \emph{Königin, Regentin}|pw}}}\pend
           \pstart
           \noindent{}\centering{}\textcolor{gray}{\textbf{Pfuhl’s Hôtel\oindex{Pfuhl s Hotel@\textbf{Pfuhl’s Hotel}|pw}}}\pend
           \pstart
           \noindent{}\centering{}\textcolor{gray}{\textbf{Parthie aus dem Park}}\pend
           \pstart
           Lieber, ja, \label{K_L03426-1v}\edtext{krank}{\lemma{\textnormal{\emph{krank}}}\Cendnote{\textnormal{siehe Felix Salten an Arthur Schnitzler, 6. 7. 1906}}}\label{K_L03426-1h} war ich; aber es geht wieder besser. Brief folgt.\pend
           \pstart
           herzlichst für Sie {\kaufmannsund}{ }Olga\pwindex{Schnitzler, Olga 17.01.1882 – 13.01.1970@\textsc{Schnitzler, Olga} (17.01.1882 – 13.01.1970), \emph{Schauspielerin, Sängerin}|pw}{\\[\baselineskip]}Ihr \spacefill\mbox{Salten}{\\[\baselineskip]}{[}hs. Ottilie Salten:{]} \spacefill\mbox{Ottilie}\pend
           \leftskip=0em{}\pstart {[}hs. Samuel Fischer:{]} Viele Grüße von Ihrem \spacefill\mbox{SFischer}\pend{}\pstart
           \noindent{}{[}hs. Hedwig Fischer:{]} \textsc{Hedwig Fischer} grüßt Sie u. Ihre Frau\pwindex{Schnitzler, Olga 17.01.1882 – 13.01.1970@\textsc{Schnitzler, Olga} (17.01.1882 – 13.01.1970), \emph{Schauspielerin, Sängerin}|pwv}.\pend
           \pstart
           \noindent{}{[}hs. ?? [Frau in Königs Wusterhausen]:{]} Eine \label{K_L03426-2v}\edtext{Verehrerin}{\lemma{\textnormal{\emph{Verehrerin}}}\Cendnote{\textnormal{nicht
                  identifiziert}}}\label{K_L03426-2h} grüßt auch noch herzlich.\pend
           
         
         \endnumbering\mylabel{h}\end{ledgroupsized}  \newcommand{\dateiname}{L03426}\newcommand{\titel}{Felix Salten u. a. an Arthur Schnitzler, 4. 6. 1906}\newcommand{\editorInnen}{Martin Anton Müller und Laura Untner}%% latex-leseansicht-abspann.tex
%% Abspann für die Leseansicht.
%% Der Schalter \ifkorrekturansicht ist bereits durch den Vorspann gesetzt.

%% latex-abspann.tex
%% Gemeinsamer Abspann für Korrekturansicht und Leseansicht.
%% Setzt den Schalter \ifkorrekturansicht voraus (gesetzt in den
%% einbindenden Dateien latex-korrekturansicht-abspann.tex bzw.
%% latex-leseansicht-abspann.tex).
%% ---------------------------------------------------------------

\normalsize

% Das esempio-Environment wird nur in der Leseansicht benötigt
\ifkorrekturansicht\else
\newenvironment{esempio}[3]%
{
    \vspace{1.5ex}
    \rlap{\underline{#1}}
    \par
    \setlength{\parindent}{0cm}
    \nopagebreak
    \leftskip=#2cm
    \rightskip=#3cm
}
{
    \par
}
\fi

\doendnotes{C}
\bigskip
\vfill

\clearpage

\footnotesize

\ifkorrekturansicht
  \lohead{\textsc{register}}
\fi

% theindex-Environment neu definieren ohne reledmac
\makeatletter
\renewenvironment{theindex}{%
  \ifkorrekturansicht
    \section*{\indexname}%
  \else
    \subsubsection*{Index der erwähnten Entitäten}%
  \fi
  \setlength{\parindent}{0pt}%
  \setlength{\parskip}{0pt plus 0.3pt}%
  \let\item\@idxitem
}{%
  \ifkorrekturansicht\clearpage\fi
}
\makeatother

\IfFileExists{\jobname-pw.ind}{\input{\jobname-pw.ind}}{}

% Quellenangabe nur in der Leseansicht
\ifkorrekturansicht\else
% Fallback-Definitionen, falls die .tex-Datei \titel etc. nicht gesetzt hat
\providecommand{\titel}{}
\providecommand{\editorInnen}{}
\providecommand{\dateiname}{\jobname}

\vspace{3cm}

\vfill

\footnotesize
\textsc{Quelle}: \titel. Herausgegeben von {\editorInnen}. In: \emph{Arthur Schnitzler: Briefwechsel mit Autorinnen und Autoren}.
 Digitale Edition, https://schnitzler-briefe.acdh.oeaw.ac.at/{\dateiname}.html (Stand \today)
\fi

\end{document}


      