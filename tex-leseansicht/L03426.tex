%% latex-korrekturansicht-vorspann.tex
%% Vorspann für die Korrekturansicht.
%% Lädt die gemeinsame Datei latex-vorspann.tex mit gesetztem Schalter.

\newif\ifkorrekturansicht
\korrekturansichttrue

\input{../tex-inputs/latex-vorspann}


\section[ Felix Salten u. a. an Arthur Schnitzler, 4. 6. 1906]{L03426 Felix Salten u. a. an Arthur Schnitzler, 4. 6. 1906}
\nopagebreak\mylabel{L03426v}
\rehead{ }\normalsize\beginnumbering\briefempfaengerindex{Schnitzler, Arthur@\textsc{Schnitzler, Arthur}!zzz?? [Frau in Koenigs Wusterhausen], @\emph{von  ?? [Frau in Königs Wusterhausen]}!1906-06-041@{4. 6. 1906}|(be}\briefempfaengerindex{Schnitzler, Arthur@\textsc{Schnitzler, Arthur}!zzzFischer, Hedwig@\emph{von Hedwig Fischer}!1906-06-041@{4. 6. 1906}|(be}\briefempfaengerindex{Schnitzler, Arthur@\textsc{Schnitzler, Arthur}!zzzFischer, Samuel@\emph{von Samuel Fischer}!1906-06-041@{4. 6. 1906}|(be}\briefempfaengerindex{Schnitzler, Arthur@\textsc{Schnitzler, Arthur}!zzzSalten, Ottilie@\emph{von Ottilie Salten}!1906-06-041@{4. 6. 1906}|(be}\briefempfaengerindex{Schnitzler, Arthur@\textsc{Schnitzler, Arthur}!zzzSalten, Felix@\emph{von Felix Salten}!1906-06-041@{4. 6. 1906}|(be}
\toendnotes[C]{\smallbreak\pagebreak[2]}\Standort{CUL, Schnitzler, B 89, B 1.}
\physDesc{Bildpostkarte, 258 Zeichen
\newline{}Handschrift Felix Salten: schwarze Tinte, lateinische Kurrent
\newline{}Handschrift Ottilie Salten: schwarze Tinte
\newline{}Handschrift Samuel Fischer: schwarze Tinte, lateinische Kurrent
\newline{}Handschrift Hedwig Fischer: schwarze Tinte, deutsche Kurrent
\newline{}Handschrift  ?? [Frau in Königs Wusterhausen]: Bleistift, lateinische Kurrent
\newline{}Versand: Stempel: »\nobreak{}\oindex{Koenigs Wusterhausen@\textbf{Königs Wusterhausen}, \emph{P.PPL}|pwk}Königs-Wusterhaus\textcolor{gray}{en}, 4. 6. 06, 8–9 N.\nobreak{}«.  
\newline{}Schnitzler: mit Bleistift datiert: »4/6« 
\newline{}Ordnung: mit Bleistift von unbekannter Hand nummeriert: »217« }\toendnotes[C]{\smallbreak}\pstart{}{\pb}Herrn D\textsuperscript{r} Arthur Schnitzler\pend{}\pstart{}Wien XVIII.\oindex{XVIII., Waehring@\textbf{XVIII., Währing}, \emph{A.ADM3}|pw}\pend{}\pstart{}Spöttelgaſse 7\oindex{Edmund-Weiss-Gasse 7@\textbf{Edmund-Weiß-Gasse 7}, \emph{Wohngebäude (K.WHS)}|pw}\pend{}{\bigskip}
\pstart
           \noindent{}\centering{}{\pb}\textcolor{gray}{\textbf{Gruss aus Königs-Wusterhausen\oindex{Koenigs Wusterhausen@\textbf{Königs Wusterhausen}, \emph{P.PPL}|pw}}}\pend
           
\pstart
           \centering{}\textcolor{gray}{\textbf{Historisches Buffet aus der Zeit der Königin Louise\pwindex{Luise von Mecklenburg-Strelitz 1776-03-10 – 1810-07-19@\textsc{Luise von Mecklenburg-Strelitz} (1776-03-10 – 1810-07-19), \emph{König/Königin, Regent/Regentin}|pw}}}\pend
           
\pstart
           \centering{}\textcolor{gray}{\textbf{Pfuhl’s Hôtel\oindex{Pfuhl s Hotel@\textbf{Pfuhl’s Hotel}, \emph{Hotel (K.HTL)}|pw}}}\pend
           
\pstart
           \centering{}\textcolor{gray}{\textbf{Parthie aus dem Park}}\pend
           \vspace{1em}
\pstart
           \noindent{}{\pb}Lieber, ja, \label{K_L03426-1v}\edtext{krank}{\lemma{\textnormal{\emph{krank}}}\Cendnote{\textnormal{Siehe Felix Salten an Arthur Schnitzler, 6. 7. 1906.
               }}}\label{K_L03426-1} war ich: aber es geht wieder besser. Brief folgt.\pend
           
\pstart
           herzlichst für Sie {\kaufmannsund}{ }Olga\pwindex{Schnitzler, Olga 17.01.1882 – 13.01.1970@\textsc{Schnitzler, Olga} (17.01.1882 – 13.01.1970), \emph{Schauspieler/Schauspielerin, Sänger/Sängerin}|pw}{\\[\baselineskip]}Ihr \spacefill\mbox{Salten}{\\[\baselineskip]}{[}hs. :{]} \spacefill\mbox{Ottilie}\pend
           \leftskip=0em{}\selectlanguage{ngerman}\vspace{1em}\pstart {[}hs. :{]} Viele Grüße von Ihrem \spacefill\mbox{SFischer}\pend{}\selectlanguage{ngerman}\vspace{1em}
\pstart
           \noindent{}{[}hs. :{]} \textsc{Hedwig Fischer} grüßt Sie u. Ihre Frau\pwindex{Schnitzler, Olga 17.01.1882 – 13.01.1970@\textsc{Schnitzler, Olga} (17.01.1882 – 13.01.1970), \emph{Schauspieler/Schauspielerin, Sänger/Sängerin}|pwv}.\pend
           \selectlanguage{ngerman}\vspace{1em}
\pstart
           \noindent{}{[}hs. :{]} Eine \label{K_L03426-2v}\edtext{Verehrerin}{\lemma{\textnormal{\emph{Verehrerin}}}\Cendnote{\textnormal{nicht
                  identifiziert}}}\label{K_L03426-2} grüßt auch noch herzlich.\pend
           \selectlanguage{ngerman}\endnumbering\briefempfaengerindex{Schnitzler, Arthur@\textsc{Schnitzler, Arthur}!zzz?? [Frau in Koenigs Wusterhausen], @\emph{von  ?? [Frau in Königs Wusterhausen]}!1906-06-041@{4. 6. 1906}|)be}\briefempfaengerindex{Schnitzler, Arthur@\textsc{Schnitzler, Arthur}!zzzFischer, Hedwig@\emph{von Hedwig Fischer}!1906-06-041@{4. 6. 1906}|)be}\briefempfaengerindex{Schnitzler, Arthur@\textsc{Schnitzler, Arthur}!zzzFischer, Samuel@\emph{von Samuel Fischer}!1906-06-041@{4. 6. 1906}|)be}\briefempfaengerindex{Schnitzler, Arthur@\textsc{Schnitzler, Arthur}!zzzSalten, Ottilie@\emph{von Ottilie Salten}!1906-06-041@{4. 6. 1906}|)be}\briefempfaengerindex{Schnitzler, Arthur@\textsc{Schnitzler, Arthur}!zzzSalten, Felix@\emph{von Felix Salten}!1906-06-041@{4. 6. 1906}|)be}\mylabel{L03426h}  \normalsize

\doendnotes{C}
\bigskip
\vfill

\clearpage

\footnotesize

\lohead{\textsc{register}}

% Definiere theindex-Environment komplett neu ohne reledmac
\makeatletter
\renewenvironment{theindex}{%
  \section*{\indexname}%
  \setlength{\parindent}{0pt}%
  \setlength{\parskip}{0pt plus 0.3pt}%
  \let\item\@idxitem
}{%
  \clearpage
}
\makeatother

\IfFileExists{\jobname-pw.ind}{\input{\jobname-pw.ind}}{}

\end{document}

      