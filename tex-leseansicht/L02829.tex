%% latex-leseansicht-vorspann.tex
%% Vorspann für die Leseansicht.
%% Lädt die gemeinsame Datei latex-vorspann.tex mit nicht gesetztem Schalter.

\newif\ifkorrekturansicht
\korrekturansichtfalse

\input{../tex-inputs/latex-vorspann}


         
         \renewcommand{\erwaehntePersonen}{Personen: Richard Beer-Hofmann, Paula Beer-Hofmann, Mirjam Beer-Hofmann, Paul Goldmann, Theodor Herzl, Marie Reinhard, Felix Salten, Leopold Sonnemann, Leo Van-Jung}
         \renewcommand{\erwaehnteInstitutionen}{Institutionen: Frankfurter Zeitung}
         \renewcommand{\erwaehnteOrte}{Orte: Bad Ischl, Deutsches Haus, Neues Deutsches Theater, Paris, Prag, Salzburg, Wien, rue de la Bourse}
         \renewcommand{\erwaehnteWerke}{Werke: Das Vermächtnis. Schauspiel in drei Akten, Die Toten schweigen, Freiwild. Schauspiel in 3 Akten, Weihnachts-Einkäufe}
               \section[ Paul Goldmann an Arthur Schnitzler, 15. 10. {[}1897{]}]{ Paul Goldmann an Arthur Schnitzler, 15. 10. {[}1897{]}}\nopagebreak\mylabel{v}\rehead{ }\begin{ledgroupsized}[t]{13cm}\normalsize\beginnumbering \toendnotes[C]{\smallbreak\pagebreak[2]} \Standort{DLA, A:Schnitzler, HS.NZ85.1.3167.}
\physDesc{Brief, 1 Blatt, 4 Seiten, 1982 Zeichen
\newline{}Handschrift: blaue Tinte, deutsche Kurrent
\newline{}Schnitzler: 1) mit Bleistift das Jahr »97« vermerkt  2) mit rotem Buntstift vier Unterstreichungen}\toendnotes[C]{\smallbreak}\pstart
           \noindent{}{\pb}\textcolor{gray}{\textbf{\textbf{Frankfurter Zeitung\orgindex{Frankfurter Zeitung@Frankfurter Zeitung|pw}}}}\pend
           \pstart
           \textcolor{gray}{\textbf{(\begin{otherlanguage}{french}Gazette de Francfort\end{otherlanguage}\orgindex{Frankfurter Zeitung@Frankfurter Zeitung|pw}).}}\pend
           \pstart
           \textcolor{gray}{\textbf{\textbf{\begin{otherlanguage}{french}Fondateur M.\end{otherlanguage}{ }L. Sonnemann\pwindex{Sonnemann, Leopold 1831-10-29 – 1909-10-30@\textsc{Sonnemann, Leopold} (1831-10-29 – 1909-10-30), \emph{Journalist, Herausgeber}|pw}.}}}\pend
           \pstart
           \begin{otherlanguage}{french}\textcolor{gray}{\textbf{Journal politique, financier,}}\end{otherlanguage}\pend
           \pstart
           \begin{otherlanguage}{french}\textcolor{gray}{\textbf{commercial et littéraire.}}\end{otherlanguage}\pend
           \pstart
           \begin{otherlanguage}{french}\textcolor{gray}{\textbf{\textbf{Paraissant trois fois par jour.}}}\end{otherlanguage}\hfill \textsc{Paris\oindex{Paris@\textbf{Paris}|pw}}, 15. Oktober.\pend
           \pstart
           \begin{otherlanguage}{french}\textcolor{gray}{\textbf{\textbf{Bureau à Paris\oindex{Paris@\textbf{Paris}|pw}}}}\end{otherlanguage}\pend
           \pstart
           \begin{otherlanguage}{french}\textcolor{gray}{\textbf{\textbf{10 \so{Rue de la Bourse}\oindex{rue de la Bourse@\textbf{rue de la Bourse}|pw}.}}}\end{otherlanguage}\pend
           \pstart\center{}Mein lieber Freund,\pend\pstart
           Ich wollte Dir täglich ſchreiben, habe aber jetzt ausnahmsweiſe viel zu thun. Heut erſt kann ich Dir für Deinen lieben Brief danken,
               der mich wahrhaft beruhigt hat. Ich war wirklich ſchon in Sorge, weil ich ſo lange
               nichts \introOben{}von Dir\introOben{} hörte.\pend
           \pstart
           Wenn von dem Allen nur das Eine zurückbleibt, daß Du »Sie\pwindex{Reinhard, Marie 1871-03-13 – 1899-03-18@\textsc{Reinhard, Marie} (1871-03-13 – 1899-03-18), \emph{Gesangspädagogin}|pwv}« lieber haſt als je, ſo weiß ich, wozu
               es gut war. Ich glaube immer mehr, daß »Sie\pwindex{Reinhard, Marie 1871-03-13 – 1899-03-18@\textsc{Reinhard, Marie} (1871-03-13 – 1899-03-18), \emph{Gesangspädagogin}|pwv}« in Deinem Leben die Treue, die Ruhe, die Ordnung
               darſtellt. Je feſter Du mit ihr verbunden biſt, umſo beſſer iſts für Dich. Wie
               herrlich doch {\pb}das Leben waltet! Auch Noth und Tod
               ſind ihm nur ein Mittel, um neue Liebe hervorzurufen{\dotsfive}\pend
           \pstart
           Auch die ſonſtigen Mittheilungen Deines Briefes haben mich ſehr befriedigt. Wenn das
                  Stück\pwindex{Schnitzler, Arthur 15.05.1862 – 21.10.1931@\textsc{Schnitzler, Arthur} (15.05.1862 – 21.10.1931), \emph{Schriftsteller, Mediziner}!Vermaechtnis. Schauspiel in drei Akten1898@\strich\emph{Das Vermächtnis. Schauspiel in drei Akten} {[}1898{]}|pwv} ſo weit
               iſt, bekomme ichs wohl einmal auf einen Tag im Manuſkript zu ſehen? Zu düſter
               ſollteſt Du es freilich nicht machen. Kannſt Du nicht eine heitere oder wenigſtens
               verſöhnende Epiſoden-Figur einflicken? {\dotsfive}\pend
           \pstart
           Ich habe Dir noch nicht geſagt, wie ſehr ich mich in \textsc{\label{K_L02829-1v}\edtext{Salzburg\oindex{Salzburg@\textbf{Salzburg}|pw}}{\lemma{\textnormal{\emph{Salzburg}}}\Cendnote{\textnormal{wohl ein Aufenthalt nach der Abreise
                     aus Ischl\oindex{Bad Ischl@\textbf{Bad Ischl}|pwk}, also Anfang
                        September 1897}}}\label{K_L02829-1h}} mit dem \textsc{Leo\pwindex{Van-Jung, Leo 15.10.1866 – 02.07.1939@\textsc{Van-Jung, Leo} (15.10.1866 – 02.07.1939), \emph{Gesangspädagoge, Mathematiker}|pw}} gefreut habe. Was für ein lieber Menſch! Er kommt mir vor wie ein treuer Löwe.
                  {\pb}\textsc{Richard\pwindex{Beer-Hofmann, Richard 1866-07-11 – 1945-09-26@\textsc{Beer-Hofmann, Richard} (1866-07-11 – 1945-09-26), \emph{Schriftsteller}|pw}} hatte ſein Möglichſtes gethan, um ihn davon abzureden, nach \textsc{Salzburg\oindex{Salzburg@\textbf{Salzburg}|pw}} zu kommen!\pend
           \pstart
           Von \textsc{Richard\pwindex{Beer-Hofmann, Richard 1866-07-11 – 1945-09-26@\textsc{Beer-Hofmann, Richard} (1866-07-11 – 1945-09-26), \emph{Schriftsteller}|pw}} höre ich natürlich kein Wort. Vielleicht ſchreibſt Du mir einmal eine Zeile,
               wie es ihm, \textsc{Paula\pwindex{Beer-Hofmann, Paula 25.02.1879 – 30.10.1939@\textsc{Beer-Hofmann, Paula} (25.02.1879 – 30.10.1939)|pw}} und »\textsc{Mirjam\pwindex{Beer-Hofmann, Mirjam 04.09.1897 – 24.12.1984@\textsc{Beer-Hofmann, Mirjam} (04.09.1897 – 24.12.1984)|pw}}« geht? Auch \label{K_L02829-2v}\edtext{\textsc{Salten\pwindex{Salten, Felix 06.09.1869 – 08.10.1945@\textsc{Salten, Felix} (06.09.1869 – 08.10.1945), \emph{Schriftsteller, Journalist}|pw}}, den ich in \textsc{Salzburg\oindex{Salzburg@\textbf{Salzburg}|pw}} ſah}{\lemma{\textnormal{\emph{Salten, … ſah}}}\Cendnote{\textnormal{vgl. Felix Salten an Arthur Schnitzler, 3. 9. [1897]}}}\label{K_L02829-2h}, hat mir ſehr gut gefallen. Iſt ein charmanter Menſch geworden. Daß Dir \textsc{Herzl\pwindex{Herzl, Theodor 1860-05-02 – 1904-07-03@\textsc{Herzl, Theodor} (1860-05-02 – 1904-07-03), \emph{Schriftsteller, Journalist}|pw}} zuwider iſt, glaub’ ich gern. So viel Prätention und nichts dahinter! So
               geiſtreich und ſo urtheilslos! Und ſo gar keinen Zuſammenhang mit dem wirklichen \strikeout{Leb} Leben. Aber ſchwarzer Bart und impoſantes
               Auftreten. Das ſind die Leute, die im Journalismus die großen Erfolge haben.\pend
           \pstart
           {\pb}Bitte, ſchreib’ mir, ob Du \label{K_L02829-3v}\edtext{nach \textsc{Prag\oindex{Prag@\textbf{Prag}|pw}} vorleſen}{\lemma{\textnormal{\emph{nach Prag vorleſen}}}\Cendnote{\textnormal{Schnitzler\pwindex{Schnitzler, Arthur 15.05.1862 – 21.10.1931@\textsc{Schnitzler, Arthur} (15.05.1862 – 21.10.1931), \emph{Schriftsteller, Mediziner}|pwk} hielt sich von 24. 11. 1897 bis 28. 11. 1897 in Prag\oindex{Prag@\textbf{Prag}|pwk} auf. Am 25. 11. 1897 las er im Deutschen Haus\oindex{Deutsches Haus@\textbf{Deutsches Haus}|pwkv} aus \emph{Die Toten schweigen}\pwindex{Schnitzler, Arthur 15.05.1862 – 21.10.1931@\textsc{Schnitzler, Arthur} (15.05.1862 – 21.10.1931), \emph{Schriftsteller, Mediziner}!Toten schweigen01. 10. 1897@\strich\emph{Die Toten schweigen} {[}01. 10. 1897{]}|pwk} und \emph{Weihnachts-Einkäufe}\pwindex{Schnitzler, Arthur 15.05.1862 – 21.10.1931@\textsc{Schnitzler, Arthur} (15.05.1862 – 21.10.1931), \emph{Schriftsteller, Mediziner}!Weihnachts-Einkaeufe24. 12. 1891@\strich\emph{Weihnachts-Einkäufe} {[}24. 12. 1891{]}|pwk}. Am 27. 11. 1897 fand in Schnitzler\pwindex{Schnitzler, Arthur 15.05.1862 – 21.10.1931@\textsc{Schnitzler, Arthur} (15.05.1862 – 21.10.1931), \emph{Schriftsteller, Mediziner}|pwk}s Anwesenheit die Premiere von \emph{Freiwild}\pwindex{Schnitzler, Arthur 15.05.1862 – 21.10.1931@\textsc{Schnitzler, Arthur} (15.05.1862 – 21.10.1931), \emph{Schriftsteller, Mediziner}!Freiwild. Schauspiel in 3 Akten1896@\strich\emph{Freiwild. Schauspiel in 3 Akten} {[}1896{]}|pwk} im Neuen Deutschen
                     Theater\oindex{Neues Deutsches Theater@\textbf{Neues Deutsches Theater}|pwk} statt.}}}\label{K_L02829-3h} gehſt? Und wann?\pend
           \pstart
           Von mir ſchreibe ich Dir lieber nichts. Es iſt die alte Geſchichte, ohne einen Zug
               von Änderung, \substVorne{}\textsuperscript{höchſtens}{\allowbreak}\substDazwischen{}eher\substHinten{} ſchlimmer als beſſer. Das iſt wirklich nicht intereſſant.\pend
           \pstart
           Grüße Deine Freundin\pwindex{Reinhard, Marie 1871-03-13 – 1899-03-18@\textsc{Reinhard, Marie} (1871-03-13 – 1899-03-18), \emph{Gesangspädagogin}|pwv} und ſei
               Du ſelbſt von Herzen gegrüßt!\pend
           \pstart
           Dein {\\[\baselineskip]}\spacefill\mbox{Paul Goldmnn}\pend
           \leftskip=0em{}
         
         \endnumbering\mylabel{h}\end{ledgroupsized}  \newcommand{\dateiname}{L02829}\newcommand{\titel}{Paul Goldmann an Arthur Schnitzler, 15. 10. [1897]}\newcommand{\editorInnen}{Martin Anton Müller und Laura Untner}%% latex-leseansicht-abspann.tex
%% Abspann für die Leseansicht.
%% Der Schalter \ifkorrekturansicht ist bereits durch den Vorspann gesetzt.

%% latex-abspann.tex
%% Gemeinsamer Abspann für Korrekturansicht und Leseansicht.
%% Setzt den Schalter \ifkorrekturansicht voraus (gesetzt in den
%% einbindenden Dateien latex-korrekturansicht-abspann.tex bzw.
%% latex-leseansicht-abspann.tex).
%% ---------------------------------------------------------------

\normalsize

% Das esempio-Environment wird nur in der Leseansicht benötigt
\ifkorrekturansicht\else
\newenvironment{esempio}[3]%
{
    \vspace{1.5ex}
    \rlap{\underline{#1}}
    \par
    \setlength{\parindent}{0cm}
    \nopagebreak
    \leftskip=#2cm
    \rightskip=#3cm
}
{
    \par
}
\fi

\doendnotes{C}
\bigskip
\vfill

\clearpage

\footnotesize

\ifkorrekturansicht
  \lohead{\textsc{register}}
\fi

% theindex-Environment neu definieren ohne reledmac
\makeatletter
\renewenvironment{theindex}{%
  \ifkorrekturansicht
    \section*{\indexname}%
  \else
    \subsubsection*{Index der erwähnten Entitäten}%
  \fi
  \setlength{\parindent}{0pt}%
  \setlength{\parskip}{0pt plus 0.3pt}%
  \let\item\@idxitem
}{%
  \ifkorrekturansicht\clearpage\fi
}
\makeatother

\IfFileExists{\jobname-pw.ind}{\input{\jobname-pw.ind}}{}

% Quellenangabe nur in der Leseansicht
\ifkorrekturansicht\else
% Fallback-Definitionen, falls die .tex-Datei \titel etc. nicht gesetzt hat
\providecommand{\titel}{}
\providecommand{\editorInnen}{}
\providecommand{\dateiname}{\jobname}

\vspace{3cm}

\vfill

\footnotesize
\textsc{Quelle}: \titel. Herausgegeben von {\editorInnen}. In: \emph{Arthur Schnitzler: Briefwechsel mit Autorinnen und Autoren}.
 Digitale Edition, https://schnitzler-briefe.acdh.oeaw.ac.at/{\dateiname}.html (Stand \today)
\fi

\end{document}


      