%% latex-leseansicht-vorspann.tex
%% Vorspann für die Leseansicht.
%% Lädt die gemeinsame Datei latex-vorspann.tex mit nicht gesetztem Schalter.

\newif\ifkorrekturansicht
\korrekturansichtfalse

\input{../tex-inputs/latex-vorspann}


\section[ Paul Goldmann an Arthur Schnitzler, 15. 10. [1897]]{L02829 Paul Goldmann an Arthur Schnitzler,  15. 10. [1897]}
\nopagebreak\mylabel{L02829v}
\rehead{ }\normalsize\beginnumbering\briefempfaengerindex{Schnitzler, Arthur@\textsc{Schnitzler, Arthur}!zzzGoldmann, Paul@\emph{von Paul Goldmann}!1897-10-151@{15. 10. [1897]}|(be}
\toendnotes[C]{\smallbreak\pagebreak[2]}
\correspDesc{Versand  durch Paul Goldmann am 15. 10. [1897] in Paris
\newline{}Erhalt  durch Arthur Schnitzler im Zeitraum [16. 10. 1897 – 20. 10. 1897?] in Wien}\toendnotes[C]{\smallbreak}
\Standort{DLA, A:Schnitzler, HS.NZ85.1.3167.}
\physDesc{Brief, 1 Blatt, 4 Seiten, 1982 Zeichen
\newline{}Handschrift: blaue Tinte, deutsche Kurrent
\newline{}Schnitzler: 1) mit Bleistift das Jahr »97« vermerkt  2) mit rotem Buntstift vier Unterstreichungen}\toendnotes[C]{\smallbreak}
\pstart
           {\pb}\textcolor{gray}{\textbf{\textbf{Frankfurter Zeitung\orgindex{Frankfurter Zeitung@Frankfurter Zeitung|pw}}}}\pend
           
\pstart
           \textcolor{gray}{\textbf{(\begin{otherlanguage}{french}Gazette de Francfort\end{otherlanguage}\orgindex{Frankfurter Zeitung@Frankfurter Zeitung|pw}).}}\pend
           
\pstart
           \textcolor{gray}{\textbf{\textbf{\begin{otherlanguage}{french}Fondateur M.\end{otherlanguage}{ }L. Sonnemann\pwindex{Sonnemann, Leopold 29.\,10.\,1831 Höchberg – 30.\,10.\,1909 Frankfurt am Main@\textsc{Sonnemann, Leopold} (29.\,10.\,1831 Höchberg – 30.\,10.\,1909 Frankfurt am Main), \emph{Journalist, Herausgeber}|pw}.}}}\pend
           
\pstart
           \begin{otherlanguage}{french}\textcolor{gray}{\textbf{Journal politique, financier,}}\end{otherlanguage}\pend
           
\pstart
           \begin{otherlanguage}{french}\textcolor{gray}{\textbf{commercial et littéraire.}}\end{otherlanguage}\pend
           
\pstart
           \begin{otherlanguage}{french}\textcolor{gray}{\textbf{\textbf{Paraissant trois fois par jour.}}}\end{otherlanguage}\hfill \textsc{Paris\oindex{Paris@\textbf{Paris}, \emph{Hauptstadt}|pw}}, 15. Oktober.\pend
           
\pstart
           \begin{otherlanguage}{french}\textcolor{gray}{\textbf{\textbf{Bureau à Paris\oindex{Paris@\textbf{Paris}, \emph{Hauptstadt}|pw}}}}\end{otherlanguage}\pend
           
\pstart
           \begin{otherlanguage}{french}\textcolor{gray}{\textbf{\textbf{10 \so{Rue de la Bourse}\oindex{rue de la Bourse@\textbf{rue de la Bourse}, \emph{Straße}|pw}.}}}\end{otherlanguage}\pend
           
\pstart\center{}Mein lieber Freund,\pend\vspace{0.5em}
\pstart
           Ich wollte Dir täglich{ }ſchreiben, habe aber jetzt ausnahmsweiſe viel zu thun. Heut erſt kann ich Dir für Deinen lieben Brief danken,
               der mich wahrhaft beruhigt hat. Ich war wirklich{ }ſchon in Sorge, weil ich{ }ſo lange
               nichts \introOben{}von Dir\introOben{} hörte.\pend
           
\pstart
           Wenn von dem Allen nur das Eine zurückbleibt, daß Du »Sie\pwindex{Reinhard, Marie 13.\,3.\,1871 Wien – 18.\,3.\,1899 ebd.@\textsc{Reinhard, Marie} (13.\,3.\,1871 Wien – 18.\,3.\,1899 ebd.), \emph{Gesangspädagogin}|pwv}« lieber haſt als je,{ }ſo weiß ich, wozu
               es gut war. Ich glaube immer mehr, daß »Sie\pwindex{Reinhard, Marie 13.\,3.\,1871 Wien – 18.\,3.\,1899 ebd.@\textsc{Reinhard, Marie} (13.\,3.\,1871 Wien – 18.\,3.\,1899 ebd.), \emph{Gesangspädagogin}|pwv}« in Deinem Leben die Treue, die Ruhe, die Ordnung
               darſtellt. Je feſter Du mit ihr verbunden biſt, umſo beſſer iſts für Dich. Wie
               herrlich doch {\pb}das Leben waltet! Auch Noth und Tod{ }ſind ihm nur ein Mittel, um neue Liebe hervorzurufen{\dotsfive}\pend
           
\pstart
           Auch die{ }ſonſtigen Mittheilungen Deines Briefes haben mich{ }ſehr befriedigt. Wenn das
                  Stück\pwindex{Schnitzler, Arthur 15.\,5.\,1862 Wien – 21.\,10.\,1931 ebd.@\textsc{Schnitzler, Arthur} (15.\,5.\,1862 Wien – 21.\,10.\,1931 ebd.), \emph{Schriftsteller, Mediziner}!Vermächtnis. Schauspiel in drei Akten@\strich\emph{Das Vermächtnis. Schauspiel in drei Akten}|pwv}{ }ſo weit
               iſt, bekomme ichs wohl einmal auf einen Tag im Manuſkript zu{ }ſehen? Zu düſter{ }ſollteſt Du es freilich nicht machen. Kannſt Du nicht eine heitere oder wenigſtens
               verſöhnende Epiſoden-Figur einflicken? {\dotsfive}\pend
           
\pstart
           Ich habe Dir noch nicht geſagt, wie{ }ſehr ich mich in \textsc{\label{K_L02829-1v}\edtext{Salzburg\oindex{Salzburg@\textbf{Salzburg}, \emph{Verwaltungsgebiet}|pw}}{\lemma{\textnormal{\emph{Salzburg}}}\Cendnote{\textnormal{wohl ein Aufenthalt nach der Abreise
                     aus Ischl\oindex{Bad Ischl@\textbf{Bad Ischl}|pwk}, also Anfang September 1897}}}\label{K_L02829-1}} mit dem \textsc{Leo\pwindex{Van-Jung, Leo 15.\,10.\,1866 Odessa – 2.\,7.\,1939 Riga@\textsc{Van-Jung, Leo} (15.\,10.\,1866 Odessa – 2.\,7.\,1939 Riga), \emph{Gesangspädagoge, Mathematiker}|pw}} gefreut habe. Was für ein lieber Menſch! Er kommt mir vor wie ein treuer Löwe.
                  {\pb}\textsc{Richard\pwindex{Beer-Hofmann, Richard 11.\,7.\,1866 Wien – 26.\,9.\,1945 New York City@\textsc{Beer-Hofmann, Richard} (11.\,7.\,1866 Wien – 26.\,9.\,1945 New York City), \emph{Schriftsteller}|pw}} hatte{ }ſein Möglichſtes gethan, um ihn davon abzureden, nach \textsc{Salzburg\oindex{Salzburg@\textbf{Salzburg}, \emph{Verwaltungsgebiet}|pw}} zu kommen!\pend
           
\pstart
           Von \textsc{Richard\pwindex{Beer-Hofmann, Richard 11.\,7.\,1866 Wien – 26.\,9.\,1945 New York City@\textsc{Beer-Hofmann, Richard} (11.\,7.\,1866 Wien – 26.\,9.\,1945 New York City), \emph{Schriftsteller}|pw}} höre ich natürlich kein Wort. Vielleicht{ }ſchreibſt Du mir einmal eine Zeile,
               wie es ihm, \textsc{Paula\pwindex{Beer-Hofmann, Paula 25.\,2.\,1879 Wien – 30.\,10.\,1939 Zürich@\textsc{Beer-Hofmann, Paula} (25.\,2.\,1879 Wien – 30.\,10.\,1939 Zürich)|pw}} und »\textsc{Mirjam\pwindex{Beer-Hofmann, Mirjam 4.\,9.\,1897 Wien – 24.\,12.\,1984 New York City@\textsc{Beer-Hofmann, Mirjam} (4.\,9.\,1897 Wien – 24.\,12.\,1984 New York City)|pw}}« geht? Auch \label{K_L02829-2v}\edtext{\textsc{Salten\pwindex{Salten, Felix 6.\,9.\,1869 Budapest – 8.\,10.\,1945 Zürich@\textsc{Salten, Felix} (6.\,9.\,1869 Budapest – 8.\,10.\,1945 Zürich), \emph{Schriftsteller, Journalist, Chefredakteur}|pw}}, den ich in \textsc{Salzburg\oindex{Salzburg@\textbf{Salzburg}, \emph{Verwaltungsgebiet}|pw}}{ }ſah}{\lemma{\textnormal{\emph{Salten, … sah}}}\Cendnote{\textnormal{Vgl. XXXX Auszeichnungsfehler: Dokument L03274 nicht gefunden.
               }}}\label{K_L02829-2}, hat mir{ }ſehr gut gefallen. Iſt ein charmanter Menſch geworden. Daß Dir \textsc{Herzl\pwindex{Herzl, Theodor 2.\,5.\,1860 Budapest – 3.\,7.\,1904 Edlach@\textsc{Herzl, Theodor} (2.\,5.\,1860 Budapest – 3.\,7.\,1904 Edlach), \emph{Schriftsteller, Journalist}|pw}} zuwider iſt, glaub’ ich gern. So viel Prätention und nichts dahinter! So
               geiſtreich und{ }ſo urtheilslos! Und{ }ſo gar keinen Zuſammenhang mit dem wirklichen \strikeout{Leb} Leben. Aber{ }ſchwarzer Bart und impoſantes
               Auftreten. Das{ }ſind die Leute, die im Journalismus die großen Erfolge haben.\pend
           
\pstart
           {\pb}Bitte,{ }ſchreib’ mir, ob Du \label{K_L02829-3v}\edtext{nach \textsc{Prag\oindex{Prag@\textbf{Prag}, \emph{Land}|pw}} vorleſen}{\lemma{\textnormal{\emph{nach Prag vorlesen}}}\Cendnote{\textnormal{Schnitzler hielt sich vom 24. 11. 1897 bis zum 28. 11. 1897 in Prag\oindex{Prag@\textbf{Prag}, \emph{Land}|pwk} auf. Am 25. 11. 1897 las er im Deutschen Haus\oindex{Deutsches Haus [Prag]@\textbf{Deutsches Haus [Prag]}, \emph{Gebäude}|pwkv} aus \emph{Die Toten schweigen}\pwindex{Schnitzler, Arthur 15.\,5.\,1862 Wien – 21.\,10.\,1931 ebd.@\textsc{Schnitzler, Arthur} (15.\,5.\,1862 Wien – 21.\,10.\,1931 ebd.), \emph{Schriftsteller, Mediziner}!Toten schweigen@\strich\emph{Die Toten schweigen}|pwk} und \emph{Weihnachts-Einkäufe}\pwindex{Schnitzler, Arthur 15.\,5.\,1862 Wien – 21.\,10.\,1931 ebd.@\textsc{Schnitzler, Arthur} (15.\,5.\,1862 Wien – 21.\,10.\,1931 ebd.), \emph{Schriftsteller, Mediziner}!Weihnachts-Einkäufe@\strich\emph{Weihnachts-Einkäufe}|pwk}. Am 27. 11. 1897 fand in Schnitzlers Anwesenheit die Premiere von \emph{Freiwild}\pwindex{Schnitzler, Arthur 15.\,5.\,1862 Wien – 21.\,10.\,1931 ebd.@\textsc{Schnitzler, Arthur} (15.\,5.\,1862 Wien – 21.\,10.\,1931 ebd.), \emph{Schriftsteller, Mediziner}!Freiwild. Schauspiel in 3 Akten@\strich\emph{Freiwild. Schauspiel in 3 Akten}|pwk} im Neuen Deutschen
                     Theater\oindex{Neues Deutsches Theater@\textbf{Neues Deutsches Theater}, \emph{Theater}|pwk} statt.}}}\label{K_L02829-3} gehſt? Und wann?\pend
           
\pstart
           Von mir{ }ſchreibe ich Dir lieber nichts. Es iſt die alte Geſchichte, ohne einen Zug
               von Änderung, \substVorne{}\textsuperscript{höchſtens}\substDazwischen{}eher\substHinten{}{ }ſchlimmer als beſſer. Das iſt wirklich nicht intereſſant.\pend
           
\pstart
           Grüße Deine Freundin\pwindex{Reinhard, Marie 13.\,3.\,1871 Wien – 18.\,3.\,1899 ebd.@\textsc{Reinhard, Marie} (13.\,3.\,1871 Wien – 18.\,3.\,1899 ebd.), \emph{Gesangspädagogin}|pwv} und{ }ſei
               Du{ }ſelbſt von Herzen gegrüßt!\pend
           
\pstart
           Dein {\\[\baselineskip]}\spacefill\mbox{Paul Goldmnn}\pend
           \leftskip=0em{}\selectlanguage{ngerman}\endnumbering\briefempfaengerindex{Schnitzler, Arthur@\textsc{Schnitzler, Arthur}!zzzGoldmann, Paul@\emph{von Paul Goldmann}!1897-10-151@{15. 10. [1897]}|)be}\mylabel{L02829h}  \newcommand{\dateiname}{L02829}\newcommand{\titel}{Paul Goldmann an Arthur Schnitzler, 15. 10. [1897]}\newcommand{\editorInnen}{Martin Anton Müller und Laura Untner}%% latex-leseansicht-abspann.tex
%% Abspann für die Leseansicht.
%% Der Schalter \ifkorrekturansicht ist bereits durch den Vorspann gesetzt.

%% latex-abspann.tex
%% Gemeinsamer Abspann für Korrekturansicht und Leseansicht.
%% Setzt den Schalter \ifkorrekturansicht voraus (gesetzt in den
%% einbindenden Dateien latex-korrekturansicht-abspann.tex bzw.
%% latex-leseansicht-abspann.tex).
%% ---------------------------------------------------------------

\normalsize

% Das esempio-Environment wird nur in der Leseansicht benötigt
\ifkorrekturansicht\else
\newenvironment{esempio}[3]%
{
    \vspace{1.5ex}
    \rlap{\underline{#1}}
    \par
    \setlength{\parindent}{0cm}
    \nopagebreak
    \leftskip=#2cm
    \rightskip=#3cm
}
{
    \par
}
\fi

\doendnotes{C}
\bigskip
\vfill

\clearpage

\footnotesize

\ifkorrekturansicht
  \lohead{\textsc{register}}
\fi

% theindex-Environment neu definieren ohne reledmac
\makeatletter
\renewenvironment{theindex}{%
  \ifkorrekturansicht
    \section*{\indexname}%
  \else
    \subsubsection*{Index der erwähnten Entitäten}%
  \fi
  \setlength{\parindent}{0pt}%
  \setlength{\parskip}{0pt plus 0.3pt}%
  \let\item\@idxitem
}{%
  \ifkorrekturansicht\clearpage\fi
}
\makeatother

\IfFileExists{\jobname-pw.ind}{\input{\jobname-pw.ind}}{}

% Quellenangabe nur in der Leseansicht
\ifkorrekturansicht\else
% Fallback-Definitionen, falls die .tex-Datei \titel etc. nicht gesetzt hat
\providecommand{\titel}{}
\providecommand{\editorInnen}{}
\providecommand{\dateiname}{\jobname}

\vspace{3cm}

\vfill

\footnotesize
\textsc{Quelle}: \titel. Herausgegeben von {\editorInnen}. In: \emph{Arthur Schnitzler: Briefwechsel mit Autorinnen und Autoren}.
 Digitale Edition, https://schnitzler-briefe.acdh.oeaw.ac.at/{\dateiname}.html (Stand \today)
\fi

\end{document}


