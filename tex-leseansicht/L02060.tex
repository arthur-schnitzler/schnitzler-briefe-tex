%% latex-leseansicht-vorspann.tex
%% Vorspann für die Leseansicht.
%% Lädt die gemeinsame Datei latex-vorspann.tex mit nicht gesetztem Schalter.

\newif\ifkorrekturansicht
\korrekturansichtfalse

\input{../tex-inputs/latex-vorspann}


         
         \renewcommand{\erwaehntePersonen}{Personen: Hermann Bahr, Anna Bahr-Mildenburg, Olga Schnitzler}
         \renewcommand{\erwaehnteInstitutionen}{Institutionen: Staatsoper}
         \renewcommand{\erwaehnteOrte}{Orte: Wien}
         \renewcommand{\erwaehnteWerke}{}
               \section[Hermann Bahr an Olga Schnitzler, 27. 4. 1912]{ Hermann Bahr an Olga Schnitzler, 27. 4. 1912}\nopagebreak\mylabel{v}\rehead{ }\begin{ledgroupsized}[t]{13cm}\normalsize\beginnumbering\briefempfaengerindex{Schnitzler, Olga@\textsc{Schnitzler, Olga}!zzzBahr, Hermann@\emph{von Hermann Bahr}!1912-04-271@{27. 4. 1912}|(be} \toendnotes[C]{\smallbreak\pagebreak[2]} \Standort{CUL, Schnitzler, B 5b.}
\physDesc{Brief, 1 Blatt, 2 Seiten, 1062 Zeichen
\newline{}Handschrift: schwarze Tinte, deutsche Kurrent
\newline{}Schnitzler: mit Bleistift ergänzt »\textsc{Bahr}« 
\newline{}Ordnung: mit Bleistift von unbekannter Hand nummeriert:
                                    »172« }\buchAbdrucke{\weitereDrucke{Hermann Bahr, Arthur Schnitzler: \emph{Briefwechsel, Aufzeichnungen, Dokumente (1891–1931)}. Hg. Kurt Ifkovits und Martin Anton Müller. Göttingen: \emph{Wallstein} 2018, S. 470.} }\toendnotes[C]{\smallbreak}\pstart
           \raggedleft{}{\pb}27. 4. 12\pend
           \pstart\center{}Sehr verehrte liebe gnädige Frau!\pend\pstart
           Meine Frau\pwindex{Bahr-Mildenburg, Anna 29.11.1872 – 27.01.1947@\textsc{Bahr-Mildenburg, Anna} (29.11.1872 – 27.01.1947), \emph{Sängerin}|pwv} dankt Ihnen
               herzlichſt für Ihre liebe Einladung, der ſie ſo gern folgen würde, wenns nur irgend
               ging! Es geht aber leider nicht, weil ſie gerade jetzt von den ſämmtlichen
               Freundinnen oder Bekannten, die ſie ſich in den \strikeout{zwölf}{ }\label{K_L02060-1v}\edtext{vierzehn Wiener\oindex{Wien@\textbf{Wien}|pw} Jahren}{\lemma{\textnormal{\emph{vierzehn Wiener Jahren}}}\Cendnote{\textnormal{Am
                     1. 6. 1898 wurde sie Ensemblemitglied der \emph{Wiener Hofoper}\orgindex{Staatsoper@Staatsoper|pwk}.}}}\label{K_L02060-1h} angesammelt hat, dringend
               aufgefordert wird, ſie müßte nun bevor wir Wien\oindex{Wien@\textbf{Wien}|pw}
               verlaſſen, noch einmal zu ihnen kommen; ſie hätte alſo vierzehn Tage rein mit
               Beſuchen zuzubringen, da ſagt ſie lieber allen Nein. Nun können Sie ſich aber
               vorſtellen, wie eiferſüchtig {\pb}dieſe ſämmtlichen
               Freundinnen darüber wachen, daß ſie wenigſtens auch bei den anderen nicht erſcheint,
               und Sie können ſich den Lärm vorſtellen, w\substVorne{}\textsuperscript{ie}\substDazwischen{}enn\substHinten{}{ }ſie\pwindex{Bahr-Mildenburg, Anna 29.11.1872 – 27.01.1947@\textsc{Bahr-Mildenburg, Anna} (29.11.1872 – 27.01.1947), \emph{Sängerin}|pwv} auch nur eine einzige
               Ausnahme machte. Da Sie ja ſelbſt ſo glücklich ſind, weiblichen Geſchlechts zu ſein,
               werden Sie ja dieſe femininen Feinheiten beſſer zu würdigen verſtehen als ich ſelbſt
               und ſich Donnerstag mit mir begnügen, der ſich unendlich freut, mit
               Ihnen beiden\pwindex{Schnitzler, Arthur 15.05.1862 – 21.10.1931@\textsc{Schnitzler, Arthur} (15.05.1862 – 21.10.1931), \emph{Schriftsteller, Mediziner}|pwv} zuſammen zu
               ſein.\pend
           \pstart
           Mit den ſchönsten Grüßen von Haus zu Haus{\\[\baselineskip]}immer Ihr alternder{\\[\baselineskip]}\spacefill\mbox{HermannBahr}\pend
           \leftskip=0em{}
         
         \endnumbering\mylabel{h}\end{ledgroupsized}  \newcommand{\dateiname}{L02060}\newcommand{\titel}{Hermann Bahr an Olga Schnitzler, 27. 4. 1912}\newcommand{\editorInnen}{Martin Anton Müller und Gerd-Hermann Susen}%% latex-leseansicht-abspann.tex
%% Abspann für die Leseansicht.
%% Der Schalter \ifkorrekturansicht ist bereits durch den Vorspann gesetzt.

%% latex-abspann.tex
%% Gemeinsamer Abspann für Korrekturansicht und Leseansicht.
%% Setzt den Schalter \ifkorrekturansicht voraus (gesetzt in den
%% einbindenden Dateien latex-korrekturansicht-abspann.tex bzw.
%% latex-leseansicht-abspann.tex).
%% ---------------------------------------------------------------

\normalsize

% Das esempio-Environment wird nur in der Leseansicht benötigt
\ifkorrekturansicht\else
\newenvironment{esempio}[3]%
{
    \vspace{1.5ex}
    \rlap{\underline{#1}}
    \par
    \setlength{\parindent}{0cm}
    \nopagebreak
    \leftskip=#2cm
    \rightskip=#3cm
}
{
    \par
}
\fi

\doendnotes{C}
\bigskip
\vfill

\clearpage

\footnotesize

\ifkorrekturansicht
  \lohead{\textsc{register}}
\fi

% theindex-Environment neu definieren ohne reledmac
\makeatletter
\renewenvironment{theindex}{%
  \ifkorrekturansicht
    \section*{\indexname}%
  \else
    \subsubsection*{Index der erwähnten Entitäten}%
  \fi
  \setlength{\parindent}{0pt}%
  \setlength{\parskip}{0pt plus 0.3pt}%
  \let\item\@idxitem
}{%
  \ifkorrekturansicht\clearpage\fi
}
\makeatother

\IfFileExists{\jobname-pw.ind}{\input{\jobname-pw.ind}}{}

% Quellenangabe nur in der Leseansicht
\ifkorrekturansicht\else
% Fallback-Definitionen, falls die .tex-Datei \titel etc. nicht gesetzt hat
\providecommand{\titel}{}
\providecommand{\editorInnen}{}
\providecommand{\dateiname}{\jobname}

\vspace{3cm}

\vfill

\footnotesize
\textsc{Quelle}: \titel. Herausgegeben von {\editorInnen}. In: \emph{Arthur Schnitzler: Briefwechsel mit Autorinnen und Autoren}.
 Digitale Edition, https://schnitzler-briefe.acdh.oeaw.ac.at/{\dateiname}.html (Stand \today)
\fi

\end{document}


      