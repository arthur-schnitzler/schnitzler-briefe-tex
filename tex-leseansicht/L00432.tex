%% latex-korrekturansicht-vorspann.tex
%% Vorspann für die Korrekturansicht.
%% Lädt die gemeinsame Datei latex-vorspann.tex mit gesetztem Schalter.

\newif\ifkorrekturansicht
\korrekturansichttrue

\input{../tex-inputs/latex-vorspann}


\section[Karl Kraus an Arthur Schnitzler, 25. 4. 1895]{L00432 Karl Kraus an Arthur Schnitzler, 25. 4. 1895}
\nopagebreak\mylabel{L00432v}
\rehead{ }\normalsize\beginnumbering\briefempfaengerindex{Schnitzler, Arthur@\textsc{Schnitzler, Arthur}!zzzKraus, Karl@\emph{von Karl Kraus}!1895-04-251@{25. 4. 1895}|(be}
\toendnotes[C]{\smallbreak\pagebreak[2]}\Standort{CUL, Schnitzler, B 55.}
\physDesc{Brief, 1 Blatt, 1 Seite, 388 Zeichen
\newline{}Handschrift: schwarze Tinte, deutsche Kurrent}
\buchAbdrucke{\weitereDrucke{\emph{Literatur und Kritik}, Bd. 49, Oktober 1970, S. 522.} }\toendnotes[C]{\smallbreak}
\pstart
           {\pb}\textcolor{gray}{\textbf{KARL KRAUS}}\hfill \textcolor{gray}{\textbf{WIEN\oindex{Wien@\textbf{Wien}, \emph{A.ADM2}|pw}}}, 25. 4. \textcolor{gray}{\textbf{189}}5.\pend
           
\pstart
           \raggedleft{}\textcolor{gray}{\textbf{\textsc{I. Maximilianstrasse 13\oindex{Mahlerstrasse@\textbf{Mahlerstraße}, \emph{Straße (K.STR)}|pw}}}}.\pend
           
\pstart{}Lieber Doktor,\pend\vspace{0.5em}
\pstart
           zu unſerer Wette:\pend
           
\pstart
           Ich erkundigte mich im Regiezimmer des Burgtheaters\oindex{Burgtheater@\textbf{Burgtheater}, \emph{S.THTR}|pw} und Herr \textsc{Lorai}\pwindex{Lorey, Christian 10.08.1840 – 01.08.1906@\textsc{Lorey, Christian} (10.08.1840 – 01.08.1906), \emph{Theateransager/Theateransagerin}|pw} hat mir folgende Auskunft ertheilt:\pend
           
\pstart
           »Herr Schreiner\pwindex{Schreiner, Jakob 14.06.1854 – 26.01.1942@\textsc{Schreiner, Jakob} (14.06.1854 – 26.01.1942), \emph{Schauspieler/Schauspielerin}|pw} hat den Lerſe\pwindex{Goetz von Berlichingen mit der eisernen Hand@\emph{Götz von Berlichingen mit der eisernen Hand}|pwv} in ›Götz
                  v. Berlichingen\pwindex{Goetz von Berlichingen mit der eisernen Hand@\emph{Götz von Berlichingen mit der eisernen Hand}|pw}‹ \uuline{ſehr häufig} geſpielt.«\pend
           
\pstart
           – »Das ſind die kurzen Sätze. Ich kann nichts dafür. – – – – –«\pend
           
\pstart
           Beſtens grüßend{\\[\baselineskip]}Ihr ganz ergebener{\\[\baselineskip]}\spacefill\mbox{KarlKraus}\pend
           \leftskip=0em{}
\pstart
           \noindent{}\textsc{NB}. Herr \textsc{Lorai}\pwindex{Lorey, Christian 10.08.1840 – 01.08.1906@\textsc{Lorey, Christian} (10.08.1840 – 01.08.1906), \emph{Theateransager/Theateransagerin}|pw} wird Ihnen die mir gegebenen Auskünfte gerne wiederholen.\pend
           \selectlanguage{ngerman}\endnumbering\briefempfaengerindex{Schnitzler, Arthur@\textsc{Schnitzler, Arthur}!zzzKraus, Karl@\emph{von Karl Kraus}!1895-04-251@{25. 4. 1895}|)be}\mylabel{L00432h}  \normalsize

\doendnotes{C}
\bigskip
\vfill

\clearpage

\footnotesize

\lohead{\textsc{register}}

% Definiere theindex-Environment komplett neu ohne reledmac
\makeatletter
\renewenvironment{theindex}{%
  \section*{\indexname}%
  \setlength{\parindent}{0pt}%
  \setlength{\parskip}{0pt plus 0.3pt}%
  \let\item\@idxitem
}{%
  \clearpage
}
\makeatother

\IfFileExists{\jobname-pw.ind}{\input{\jobname-pw.ind}}{}

\end{document}

      