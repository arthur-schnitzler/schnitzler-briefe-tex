%% latex-leseansicht-vorspann.tex
%% Vorspann für die Leseansicht.
%% Lädt die gemeinsame Datei latex-vorspann.tex mit nicht gesetztem Schalter.

\newif\ifkorrekturansicht
\korrekturansichtfalse

\input{../tex-inputs/latex-vorspann}


\section[Karl Kraus an Arthur Schnitzler, 25. 4. 1895]{L00432 Karl Kraus an Arthur Schnitzler, 25. 4. 1895}
\nopagebreak\mylabel{L00432v}
\rehead{ }\normalsize\beginnumbering\briefempfaengerindex{Schnitzler, Arthur@\textsc{Schnitzler, Arthur}!zzzKraus, Karl@\emph{von Karl Kraus}!1895-04-251@{25. 4. 1895}|(be}
\toendnotes[C]{\smallbreak\pagebreak[2]}
\correspDesc{Versand  durch Karl Kraus am 25. 4. 1895 in Wien
\newline{}Erhalt  durch Arthur Schnitzler im Zeitraum [25. 4. 1895
                  – 29. 4. 1895?] in Wien}\toendnotes[C]{\smallbreak}
\Standort{CUL, Schnitzler, B 55.}
\physDesc{Brief, 1 Blatt, 1 Seite, 388 Zeichen
\newline{}Handschrift: schwarze Tinte, deutsche Kurrent}
\buchAbdrucke{\weitereDrucke{\emph{Karl Kraus und Arthur Schnitzler. Eine Dokumentation.}Herausgegeben von Reinhard Urbach In: \emph{Literatur und Kritik}, Bd. 49, Oktober 1970, S. 522.} }\toendnotes[C]{\smallbreak}
\pstart
           {\pb}\textcolor{gray}{\textbf{KARL KRAUS}}\hfill \textcolor{gray}{\textbf{WIEN\oindex{Wien@\textbf{Wien}, \emph{Verwaltungsgebiet}|pw}}}, 25. 4. \textcolor{gray}{\textbf{189}}5.\pend
           
\pstart
           \raggedleft{}\textcolor{gray}{\textbf{\textsc{I. Maximilianstrasse 13\oindex{Wien@\textbf{Wien}!I., Innere Stadt@\textbf{I., Innere Stadt}!Mahlerstraße@\textbf{Mahlerstraße}, \emph{Straße}|pw}}}}.\pend
           
\pstart{}Lieber Doktor,\pend\vspace{0.5em}
\pstart
           zu unſerer Wette:\pend
           
\pstart
           Ich erkundigte mich im Regiezimmer des Burgtheaters\oindex{Wien@\textbf{Wien}!I., Innere Stadt@\textbf{I., Innere Stadt}!Burgtheater@\textbf{Burgtheater}, \emph{Theater}|pw} und Herr \textsc{Lorai}\pwindex{Lorey, Christian 10.\,8.\,1840 Bad Salzungen – 1.\,8.\,1906 Emmersdorf an der Donau@\textsc{Lorey, Christian} (10.\,8.\,1840 Bad Salzungen – 1.\,8.\,1906 Emmersdorf an der Donau), \emph{Theateransager}|pw} hat mir folgende Auskunft ertheilt:\pend
           
\pstart
           »Herr Schreiner\pwindex{Schreiner, Jakob 14.\,6.\,1854 Gaweinstal – 26.\,1.\,1942 Wien@\textsc{Schreiner, Jakob} (14.\,6.\,1854 Gaweinstal – 26.\,1.\,1942 Wien), \emph{Schauspieler}|pw} hat den Lerſe\pwindex{Götz von Berlichingen mit der eisernen Hand@\emph{Götz von Berlichingen mit der eisernen Hand}|pwv} in ›Götz
                  v. Berlichingen\pwindex{Götz von Berlichingen mit der eisernen Hand@\emph{Götz von Berlichingen mit der eisernen Hand}|pw}‹ \uuline{ſehr häufig} geſpielt.«\pend
           
\pstart
           – »Das{ }ſind die kurzen Sätze. Ich kann nichts dafür. – – – – –«\pend
           
\pstart
           Beſtens grüßend{\\[\baselineskip]}Ihr ganz ergebener{\\[\baselineskip]}\spacefill\mbox{KarlKraus}\pend
           \leftskip=0em{}
\pstart
           \noindent{}\textsc{NB}. Herr \textsc{Lorai}\pwindex{Lorey, Christian 10.\,8.\,1840 Bad Salzungen – 1.\,8.\,1906 Emmersdorf an der Donau@\textsc{Lorey, Christian} (10.\,8.\,1840 Bad Salzungen – 1.\,8.\,1906 Emmersdorf an der Donau), \emph{Theateransager}|pw} wird Ihnen die mir gegebenen Auskünfte gerne wiederholen.\pend
           \selectlanguage{ngerman}\endnumbering\briefempfaengerindex{Schnitzler, Arthur@\textsc{Schnitzler, Arthur}!zzzKraus, Karl@\emph{von Karl Kraus}!1895-04-251@{25. 4. 1895}|)be}\mylabel{L00432h}  \newcommand{\dateiname}{L00432}\newcommand{\titel}{Karl Kraus an Arthur Schnitzler, 25. 4. 1895}\newcommand{\editorInnen}{Martin Anton Müller und Gerd-Hermann Susen}%% latex-leseansicht-abspann.tex
%% Abspann für die Leseansicht.
%% Der Schalter \ifkorrekturansicht ist bereits durch den Vorspann gesetzt.

%% latex-abspann.tex
%% Gemeinsamer Abspann für Korrekturansicht und Leseansicht.
%% Setzt den Schalter \ifkorrekturansicht voraus (gesetzt in den
%% einbindenden Dateien latex-korrekturansicht-abspann.tex bzw.
%% latex-leseansicht-abspann.tex).
%% ---------------------------------------------------------------

\normalsize

% Das esempio-Environment wird nur in der Leseansicht benötigt
\ifkorrekturansicht\else
\newenvironment{esempio}[3]%
{
    \vspace{1.5ex}
    \rlap{\underline{#1}}
    \par
    \setlength{\parindent}{0cm}
    \nopagebreak
    \leftskip=#2cm
    \rightskip=#3cm
}
{
    \par
}
\fi

\doendnotes{C}
\bigskip
\vfill

\clearpage

\footnotesize

\ifkorrekturansicht
  \lohead{\textsc{register}}
\fi

% theindex-Environment neu definieren ohne reledmac
\makeatletter
\renewenvironment{theindex}{%
  \ifkorrekturansicht
    \section*{\indexname}%
  \else
    \subsubsection*{Index der erwähnten Entitäten}%
  \fi
  \setlength{\parindent}{0pt}%
  \setlength{\parskip}{0pt plus 0.3pt}%
  \let\item\@idxitem
}{%
  \ifkorrekturansicht\clearpage\fi
}
\makeatother

\IfFileExists{\jobname-pw.ind}{\input{\jobname-pw.ind}}{}

% Quellenangabe nur in der Leseansicht
\ifkorrekturansicht\else
% Fallback-Definitionen, falls die .tex-Datei \titel etc. nicht gesetzt hat
\providecommand{\titel}{}
\providecommand{\editorInnen}{}
\providecommand{\dateiname}{\jobname}

\vspace{3cm}

\vfill

\footnotesize
\textsc{Quelle}: \titel. Herausgegeben von {\editorInnen}. In: \emph{Arthur Schnitzler: Briefwechsel mit Autorinnen und Autoren}.
 Digitale Edition, https://schnitzler-briefe.acdh.oeaw.ac.at/{\dateiname}.html (Stand \today)
\fi

\end{document}


