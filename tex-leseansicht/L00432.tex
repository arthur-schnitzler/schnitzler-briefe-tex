%% latex-leseansicht-vorspann.tex
%% Vorspann für die Leseansicht.
%% Lädt die gemeinsame Datei latex-vorspann.tex mit nicht gesetztem Schalter.

\newif\ifkorrekturansicht
\korrekturansichtfalse

\input{../tex-inputs/latex-vorspann}


         
         \renewcommand{\erwaehntePersonen}{Personen: Karl Kraus, Christian Lorey, Jakob Schreiner}
         \renewcommand{\erwaehnteOrte}{Orte: Burgtheater, Mahlerstraße, Wien}
         \renewcommand{\erwaehnteWerke}{Werke: Götz von Berlichingen}
               \section[Karl Kraus an Arthur Schnitzler, 25. 4. 1895]{ Karl Kraus an Arthur Schnitzler, 25. 4. 1895}\nopagebreak\mylabel{v}\rehead{ }\begin{ledgroupsized}[t]{13cm}\normalsize\beginnumbering\briefempfaengerindex{Schnitzler, Arthur@\textsc{Schnitzler, Arthur}!zzzKraus, Karl@\emph{von Karl Kraus}!1895-04-251@{25. 4. 1895}|(be} \toendnotes[C]{\smallbreak\pagebreak[2]} \Standort{CUL, Schnitzler, B 55.}
\physDesc{Brief, 1 Blatt, 1 Seite, 388 Zeichen
\newline{}Handschrift: schwarze Tinte, deutsche Kurrent}\buchAbdrucke{\weitereDrucke{\emph{Karl Kraus und Arthur Schnitzler. Eine Dokumentation.} Hg. Reinhard Urbach. In: \emph{Literatur und Kritik}, Bd. 49, Oktober 1970, S. 522.} }\toendnotes[C]{\smallbreak}\pstart
           \noindent{}{\pb}\textcolor{gray}{\textbf{KARL KRAUS}}\hfill \textcolor{gray}{\textbf{WIEN\oindex{Wien@\textbf{Wien}|pw}}}, 25. 4. \textcolor{gray}{\textbf{189}}5.\pend
           \pstart
           \raggedleft{}\textcolor{gray}{\textbf{\textsc{I. Maximilianstrasse 13\oindex{Mahlerstrasse@\textbf{Mahlerstraße}|pw}}}}.\pend
           \pstart{}Lieber Doktor,\pend\pstart
           zu unſerer Wette:\pend
           \pstart
           Ich erkundigte mich im Regiezimmer des Burgtheater\oindex{Burgtheater@\textbf{Burgtheater}|pw}s und Herr \textsc{Lorai}\pwindex{Lorey, Christian 10.08.1840 – 01.08.1906@\textsc{Lorey, Christian} (10.08.1840 – 01.08.1906), \emph{Theateransager}|pw} hat mir folgende Auskunft ertheilt:\pend
           \pstart
           »Herr Schreiner\pwindex{Schreiner, Jakob 14.06.1854 – 26.01.1942@\textsc{Schreiner, Jakob} (14.06.1854 – 26.01.1942), \emph{Schauspieler}|pw} hat den Lerſe\pwindex{\textcolor{red}{\textsuperscript{XXXX1 indx}}!Goetz von Berlichingen1773@\strich\emph{Götz von Berlichingen} {[}1773{]}|pwv} in ›Götz
                  v. Berlichingen\pwindex{\textcolor{red}{\textsuperscript{XXXX1 indx}}!Goetz von Berlichingen1773@\strich\emph{Götz von Berlichingen} {[}1773{]}|pw}‹ \uuline{ſehr häufig} geſpielt.«\pend
           \pstart
           – »Das ſind die kurzen Sätze. Ich kann nichts dafür. – – – – –«\pend
           \pstart
           Beſtens grüßend{\\[\baselineskip]}Ihr ganz ergebener{\\[\baselineskip]}\spacefill\mbox{KarlKraus}\pend
           \leftskip=0em{}\pstart
           \noindent{}\textsc{NB}. Herr \textsc{Lorai}\pwindex{Lorey, Christian 10.08.1840 – 01.08.1906@\textsc{Lorey, Christian} (10.08.1840 – 01.08.1906), \emph{Theateransager}|pw} wird Ihnen die mir gegebenen Auskünfte gerne wiederholen.\pend
           
         
         \endnumbering\mylabel{h}\end{ledgroupsized}  \newcommand{\dateiname}{L00432}\newcommand{\titel}{Karl Kraus an Arthur Schnitzler, 25. 4. 1895}\newcommand{\editorInnen}{Martin Anton Müller und Gerd-Hermann Susen}%% latex-leseansicht-abspann.tex
%% Abspann für die Leseansicht.
%% Der Schalter \ifkorrekturansicht ist bereits durch den Vorspann gesetzt.

%% latex-abspann.tex
%% Gemeinsamer Abspann für Korrekturansicht und Leseansicht.
%% Setzt den Schalter \ifkorrekturansicht voraus (gesetzt in den
%% einbindenden Dateien latex-korrekturansicht-abspann.tex bzw.
%% latex-leseansicht-abspann.tex).
%% ---------------------------------------------------------------

\normalsize

% Das esempio-Environment wird nur in der Leseansicht benötigt
\ifkorrekturansicht\else
\newenvironment{esempio}[3]%
{
    \vspace{1.5ex}
    \rlap{\underline{#1}}
    \par
    \setlength{\parindent}{0cm}
    \nopagebreak
    \leftskip=#2cm
    \rightskip=#3cm
}
{
    \par
}
\fi

\doendnotes{C}
\bigskip
\vfill

\clearpage

\footnotesize

\ifkorrekturansicht
  \lohead{\textsc{register}}
\fi

% theindex-Environment neu definieren ohne reledmac
\makeatletter
\renewenvironment{theindex}{%
  \ifkorrekturansicht
    \section*{\indexname}%
  \else
    \subsubsection*{Index der erwähnten Entitäten}%
  \fi
  \setlength{\parindent}{0pt}%
  \setlength{\parskip}{0pt plus 0.3pt}%
  \let\item\@idxitem
}{%
  \ifkorrekturansicht\clearpage\fi
}
\makeatother

\IfFileExists{\jobname-pw.ind}{\input{\jobname-pw.ind}}{}

% Quellenangabe nur in der Leseansicht
\ifkorrekturansicht\else
% Fallback-Definitionen, falls die .tex-Datei \titel etc. nicht gesetzt hat
\providecommand{\titel}{}
\providecommand{\editorInnen}{}
\providecommand{\dateiname}{\jobname}

\vspace{3cm}

\vfill

\footnotesize
\textsc{Quelle}: \titel. Herausgegeben von {\editorInnen}. In: \emph{Arthur Schnitzler: Briefwechsel mit Autorinnen und Autoren}.
 Digitale Edition, https://schnitzler-briefe.acdh.oeaw.ac.at/{\dateiname}.html (Stand \today)
\fi

\end{document}


      