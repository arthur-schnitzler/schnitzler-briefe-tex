%% latex-leseansicht-vorspann.tex
%% Vorspann für die Leseansicht.
%% Lädt die gemeinsame Datei latex-vorspann.tex mit nicht gesetztem Schalter.

\newif\ifkorrekturansicht
\korrekturansichtfalse

\input{../tex-inputs/latex-vorspann}


\section[Arthur Schnitzler an Hugo von Hofmannsthal, 12. 7. 1893]{L00236 Arthur Schnitzler an Hugo von Hofmannsthal, 12. 7. 1893}
\nopagebreak\mylabel{L00236v}
\rehead{ }\normalsize\beginnumbering\briefempfaengerindex{Hofmannsthal, Hugo von@\textsc{Hofmannsthal, Hugo von}!zzzSchnitzler, Arthur@\emph{von Arthur Schnitzler}!1893-07-121@{12. 7. 1893}|(be}
\toendnotes[C]{\smallbreak\pagebreak[2]}
\correspDesc{Versand  durch Arthur Schnitzler am 12. 7. 1893 in Bad Ischl
\newline{}Erhalt  durch Hugo von Hofmannsthal im Zeitraum [13. 7. 1893
                  – 17. 7. 1893?] in Wien}\toendnotes[C]{\smallbreak}
\Standort{FDH, Hs-30885,36.}
\physDesc{Brief, 1 Blatt, 4 Seiten, 901 Zeichen (Briefpapier mit Trauerrand)
\newline{}Handschrift: schwarze Tinte, deutsche Kurrent
\newline{}Ordnung: mit Bleistift von Schnitzler mutmaßlich bei der Durchsicht der Korrespondenz
                                    1929  datiert: »12. 7. 93« }
\buchAbdrucke{\weitereDrucke{Hugo von Hofmannsthal, Arthur Schnitzler: \emph{Briefwechsel}. Herausgegeben von Therese Nickl und Heinrich Schnitzler. Frankfurt am Main: \emph{S. Fischer} 1964, S. 40.} }\toendnotes[C]{\smallbreak}
\pstart{}{\pb}Lieber Loris,\pend\vspace{0.5em}
\pstart
           meine \label{K_L00236-1v}\edtext{Einakter\pwindex{Schnitzler, Arthur 15.\,5.\,1862 Wien – 21.\,10.\,1931 ebd.@\textsc{Schnitzler, Arthur} (15.\,5.\,1862 Wien – 21.\,10.\,1931 ebd.), \emph{Schriftsteller, Mediziner}!Abschiedssouper@\strich\emph{Abschiedssouper}|pwv}\pwindex{Schnitzler, Arthur 15.\,5.\,1862 Wien – 21.\,10.\,1931 ebd.@\textsc{Schnitzler, Arthur} (15.\,5.\,1862 Wien – 21.\,10.\,1931 ebd.), \emph{Schriftsteller, Mediziner}!Frage an das Schicksal@\strich\emph{Die Frage an das Schicksal}|pwv}}{\lemma{\textnormal{\emph{Einakter}}}\Cendnote{\textnormal{Nur \emph{Abschiedssouper}\pwindex{Schnitzler, Arthur 15.\,5.\,1862 Wien – 21.\,10.\,1931 ebd.@\textsc{Schnitzler, Arthur} (15.\,5.\,1862 Wien – 21.\,10.\,1931 ebd.), \emph{Schriftsteller, Mediziner}!Abschiedssouper@\strich\emph{Abschiedssouper}|pwk} wurde gegeben.}}}\label{K_L00236-1}{ }ſind Freitag. Erſte Probe geſtern –
                  Anatol\pwindex{Schnitzler, Arthur 15.\,5.\,1862 Wien – 21.\,10.\,1931 ebd.@\textsc{Schnitzler, Arthur} (15.\,5.\,1862 Wien – 21.\,10.\,1931 ebd.), \emph{Schriftsteller, Mediziner}!Abschiedssouper@\strich\emph{Abschiedssouper}|pwv} (Herr \textsc{Hoefer}\pwindex{Höfer, Emil 14.\,5.\,1864 Wien – 1.\,5.\,1940 München@\textsc{Höfer, Emil} (14.\,5.\,1864 Wien – 1.\,5.\,1940 München), \emph{Schauspieler}|pw}) erſchien einfach nicht. – Ich nahm mit \textsc{Jarno}\pwindex{Jarno, Josef 24.\,8.\,1865 Budapest – 11.\,1.\,1932 Wien@\textsc{Jarno, Josef} (24.\,8.\,1865 Budapest – 11.\,1.\,1932 Wien), \emph{Theaterleiter, Schauspieler}|pw} die Stücke\pwindex{Schnitzler, Arthur 15.\,5.\,1862 Wien – 21.\,10.\,1931 ebd.@\textsc{Schnitzler, Arthur} (15.\,5.\,1862 Wien – 21.\,10.\,1931 ebd.), \emph{Schriftsteller, Mediziner}!Abschiedssouper@\strich\emph{Abschiedssouper}|pwv}\pwindex{Schnitzler, Arthur 15.\,5.\,1862 Wien – 21.\,10.\,1931 ebd.@\textsc{Schnitzler, Arthur} (15.\,5.\,1862 Wien – 21.\,10.\,1931 ebd.), \emph{Schriftsteller, Mediziner}!Frage an das Schicksal@\strich\emph{Die Frage an das Schicksal}|pwv}
               durch; Inſcenierung, Stellung etc. – Die \textsc{Griebl}\pwindex{Griebl, Marie 27.\,2.\,1872 Baden bei Wien – 8.\,6.\,1952 Wien@\textsc{Griebl, Marie} (27.\,2.\,1872 Baden bei Wien – 8.\,6.\,1952 Wien), \emph{Schauspielerin, Sängerin}|pw} gibt die \textsc{Annie}\pwindex{Schnitzler, Arthur 15.\,5.\,1862 Wien – 21.\,10.\,1931 ebd.@\textsc{Schnitzler, Arthur} (15.\,5.\,1862 Wien – 21.\,10.\,1931 ebd.), \emph{Schriftsteller, Mediziner}!Abschiedssouper@\strich\emph{Abschiedssouper}|pwv}. –\pend
           
\pstart
           Urtheil \textsc{Friese}\pwindex{Friese, Carl Adolph 24.\,10.\,1831 Bamberg – 24.\,1.\,1900 Wien@\textsc{Friese, Carl Adolph} (24.\,10.\,1831 Bamberg – 24.\,1.\,1900 Wien), \emph{Regisseur, Schauspieler}|pw}’s: Es iſt {\pb}ein Skandal,{ }ſo was aufzuführen. – Frau
                  \textsc{Friese}\pwindex{Skura, Josefine 1841 – 1913@\textsc{Skura, Josefine} (1841 – 1913), \emph{Schauspielerin}|pw} (dieſe alte Stabscanaille, wie \textsc{Jarno}\pwindex{Jarno, Josef 24.\,8.\,1865 Budapest – 11.\,1.\,1932 Wien@\textsc{Jarno, Josef} (24.\,8.\,1865 Budapest – 11.\,1.\,1932 Wien), \emph{Theaterleiter, Schauspieler}|pw}{ }ſagt) hat{ }ſich \uline{geſchämt}, wie{ }ſie das Abſch.-\textsc{souper}\pwindex{Schnitzler, Arthur 15.\,5.\,1862 Wien – 21.\,10.\,1931 ebd.@\textsc{Schnitzler, Arthur} (15.\,5.\,1862 Wien – 21.\,10.\,1931 ebd.), \emph{Schriftsteller, Mediziner}!Abschiedssouper@\strich\emph{Abschiedssouper}|pw} geleſen. –\pend
           
\pstart
           Die Cenſur{ }ſtrich: \uline{am Buſen geruht} u{ }ſetzte dafür \uline{gekoſt}. –\pend
           
\pstart
           – Ob mir die Geſchichte für Berlin\oindex{Berlin@\textbf{Berlin}, \emph{Hauptstadt}|pw} nützen wird,
               iſt nicht abzuſehen – da \textsc{Jarno}\pwindex{Jarno, Josef 24.\,8.\,1865 Budapest – 11.\,1.\,1932 Wien@\textsc{Jarno, Josef} (24.\,8.\,1865 Budapest – 11.\,1.\,1932 Wien), \emph{Theaterleiter, Schauspieler}|pw} höchſt un{\pb}verläßlich zu{ }ſein{ }ſcheint. Ihm hat die
                  Frage a. d. Sch.\pwindex{Schnitzler, Arthur 15.\,5.\,1862 Wien – 21.\,10.\,1931 ebd.@\textsc{Schnitzler, Arthur} (15.\,5.\,1862 Wien – 21.\,10.\,1931 ebd.), \emph{Schriftsteller, Mediziner}!Frage an das Schicksal@\strich\emph{Die Frage an das Schicksal}|pw}{ }ſchon 150 Mark getragen –{ }ſo viel bekam jeder der
               Mitwirkenden bei \label{K_L00236-2v}\edtext{\textsc{Grelling}\pwindex{Grelling, Richard 11.\,6.\,1853 Berlin – 15.\,1.\,1929 ebd.@\textsc{Grelling, Richard} (11.\,6.\,1853 Berlin – 15.\,1.\,1929 ebd.), \emph{Schriftsteller, Rechtsanwalt, Publizist}|pw}}{\lemma{\textnormal{\emph{Grelling}}}\Cendnote{\textnormal{Privataufführung bei Richard Grelling\pwindex{Grelling, Richard 11.\,6.\,1853 Berlin – 15.\,1.\,1929 ebd.@\textsc{Grelling, Richard} (11.\,6.\,1853 Berlin – 15.\,1.\,1929 ebd.), \emph{Schriftsteller, Rechtsanwalt, Publizist}|pwk} kurz vor dem
                  14. 1. 1891.}}}\label{K_L00236-2}. –\pend
           
\pstart
           Gearbeitet hab ich beinah nichts; alles ungewiſſe,{ }ſo nichtig es{ }ſein mag,
               beſchäftigt nach außen hin u macht daher nervös, – Hoffentlich haben {\pb}Sie Ihre glückliche Verſeſti{\geminationm}ung wiedergefunden. – Schade, daſs Sie Freitag nicht da{ }ſind.\pend
           
\pstart
           Herzlichen Gruß{\\[\baselineskip]}Ihr\spacefill\mbox{Arth.}\pend
           \leftskip=0em{}
\pstart
           \textsc{Ischl}\oindex{Bad Ischl@\textbf{Bad Ischl}|pw}, 12. 7. 93.\pend
           \selectlanguage{ngerman}\endnumbering\briefempfaengerindex{Hofmannsthal, Hugo von@\textsc{Hofmannsthal, Hugo von}!zzzSchnitzler, Arthur@\emph{von Arthur Schnitzler}!1893-07-121@{12. 7. 1893}|)be}\mylabel{L00236h}  \newcommand{\dateiname}{L00236}\newcommand{\titel}{Arthur Schnitzler an Hugo von Hofmannsthal, 12. 7. 1893}\newcommand{\editorInnen}{Martin Anton Müller und Gerd-Hermann Susen}%% latex-leseansicht-abspann.tex
%% Abspann für die Leseansicht.
%% Der Schalter \ifkorrekturansicht ist bereits durch den Vorspann gesetzt.

%% latex-abspann.tex
%% Gemeinsamer Abspann für Korrekturansicht und Leseansicht.
%% Setzt den Schalter \ifkorrekturansicht voraus (gesetzt in den
%% einbindenden Dateien latex-korrekturansicht-abspann.tex bzw.
%% latex-leseansicht-abspann.tex).
%% ---------------------------------------------------------------

\normalsize

% Das esempio-Environment wird nur in der Leseansicht benötigt
\ifkorrekturansicht\else
\newenvironment{esempio}[3]%
{
    \vspace{1.5ex}
    \rlap{\underline{#1}}
    \par
    \setlength{\parindent}{0cm}
    \nopagebreak
    \leftskip=#2cm
    \rightskip=#3cm
}
{
    \par
}
\fi

\doendnotes{C}
\bigskip
\vfill

\clearpage

\footnotesize

\ifkorrekturansicht
  \lohead{\textsc{register}}
\fi

% theindex-Environment neu definieren ohne reledmac
\makeatletter
\renewenvironment{theindex}{%
  \ifkorrekturansicht
    \section*{\indexname}%
  \else
    \subsubsection*{Index der erwähnten Entitäten}%
  \fi
  \setlength{\parindent}{0pt}%
  \setlength{\parskip}{0pt plus 0.3pt}%
  \let\item\@idxitem
}{%
  \ifkorrekturansicht\clearpage\fi
}
\makeatother

\IfFileExists{\jobname-pw.ind}{\input{\jobname-pw.ind}}{}

% Quellenangabe nur in der Leseansicht
\ifkorrekturansicht\else
% Fallback-Definitionen, falls die .tex-Datei \titel etc. nicht gesetzt hat
\providecommand{\titel}{}
\providecommand{\editorInnen}{}
\providecommand{\dateiname}{\jobname}

\vspace{3cm}

\vfill

\footnotesize
\textsc{Quelle}: \titel. Herausgegeben von {\editorInnen}. In: \emph{Arthur Schnitzler: Briefwechsel mit Autorinnen und Autoren}.
 Digitale Edition, https://schnitzler-briefe.acdh.oeaw.ac.at/{\dateiname}.html (Stand \today)
\fi

\end{document}


