%% latex-leseansicht-vorspann.tex
%% Vorspann für die Leseansicht.
%% Lädt die gemeinsame Datei latex-vorspann.tex mit nicht gesetztem Schalter.

\newif\ifkorrekturansicht
\korrekturansichtfalse

\input{../tex-inputs/latex-vorspann}


               \section[Arthur Schnitzler an Hugo von Hofmannsthal, 12. 7. 1893]{ Arthur Schnitzler an Hugo von Hofmannsthal, 12. 7. 1893}\nopagebreak\mylabel{v}\rehead{ }\begin{ledgroupsized}[t]{13cm}\normalsize\beginnumbering\briefempfaengerindex{Hofmannsthal, Hugo von@\textsc{Hofmannsthal, Hugo von}!zzzSchnitzler, Arthur@\emph{von Arthur Schnitzler}!1893-07-121@{12. 7. 1893}|(be} \toendnotes[C]{\smallbreak\pagebreak[2]} \Standort{FDH, Hs-30885,36.}
\physDesc{Brief, 1 Blatt (Briefpapier mit Trauerrand), 4 Seiten
\newline{}Handschrift: schwarze Tinte, deutsche Kurrent\newline{}Ordnung: von Schnitzler mutmaßlich bei der Durchsicht der Korrespondenz 1929 mit
                                    Bleistift datiert: »12. 7. 93« }\buchAbdrucke{\weitereDrucke{Hugo von Hofmannsthal, Arthur Schnitzler: \emph{Briefwechsel}. Hg. Therese Nickl und Heinrich Schnitzler. Frankfurt am Main: \emph{S. Fischer} 1964, S. 40.} }\toendnotes[C]{\smallbreak}\pstart{}{\pb}Lieber Loris,\pend\pstart
           meine \label{K_L00236_1v}\edtext{Einakter\pwindex{Schnitzler, Arthur 15.05.1862 – 21.10.1931@\textsc{Schnitzler, Arthur} (15.05.1862 – 21.10.1931), \emph{Schriftsteller, Mediziner}!Abschiedssouper1892@\strich\emph{Abschiedssouper} {[}1892{]}|pwv}\pwindex{Schnitzler, Arthur 15.05.1862 – 21.10.1931@\textsc{Schnitzler, Arthur} (15.05.1862 – 21.10.1931), \emph{Schriftsteller, Mediziner}!Frage an das Schicksal01. 05. 1890@\strich\emph{Die Frage an das Schicksal} {[}01. 05. 1890{]}|pwv}}{\lemma{\textnormal{\emph{Einakter}}}\Cendnote{\textnormal{Nur \emph{Abschiedssouper}\pwindex{Schnitzler, Arthur 15.05.1862 – 21.10.1931@\textsc{Schnitzler, Arthur} (15.05.1862 – 21.10.1931), \emph{Schriftsteller, Mediziner}!Abschiedssouper1892@\strich\emph{Abschiedssouper} {[}1892{]}|pwk} wurde gegeben.}}}\label{K_L00236_1h}{ }ſind
                        Freitag. Erſte Probe geſtern – Anatol\pwindex{Schnitzler, Arthur 15.05.1862 – 21.10.1931@\textsc{Schnitzler, Arthur} (15.05.1862 – 21.10.1931), \emph{Schriftsteller, Mediziner}!Abschiedssouper1892@\strich\emph{Abschiedssouper} {[}1892{]}|pwv} (Herr \textsc{Hoefer}\pwindex{Hoefer, Emil 14.05.1864 – 01.05.1940@\textsc{Höfer, Emil} (14.05.1864 – 01.05.1940), \emph{Schauspieler}|pw}) erſchien einfach nicht. – Ich nahm mit \textsc{Jarno}\pwindex{Jarno, Josef 24.08.1865 – 11.01.1932@\textsc{Jarno, Josef} (24.08.1865 – 11.01.1932), \emph{Theaterleiter, Schauspieler}|pw} die Stücke\pwindex{Schnitzler, Arthur 15.05.1862 – 21.10.1931@\textsc{Schnitzler, Arthur} (15.05.1862 – 21.10.1931), \emph{Schriftsteller, Mediziner}!Abschiedssouper1892@\strich\emph{Abschiedssouper} {[}1892{]}|pwv}\pwindex{Schnitzler, Arthur 15.05.1862 – 21.10.1931@\textsc{Schnitzler, Arthur} (15.05.1862 – 21.10.1931), \emph{Schriftsteller, Mediziner}!Frage an das Schicksal01. 05. 1890@\strich\emph{Die Frage an das Schicksal} {[}01. 05. 1890{]}|pwv} durch;
                    Inſcenierung, Stellung etc. – Die \textsc{Griebl}\pwindex{Griebl, Marie 1872-02-27 – 1952-06-08@\textsc{Griebl, Marie} (1872-02-27 – 1952-06-08), \emph{Schauspielerin, Sängerin}|pw} gibt die \textsc{Annie}\pwindex{Schnitzler, Arthur 15.05.1862 – 21.10.1931@\textsc{Schnitzler, Arthur} (15.05.1862 – 21.10.1931), \emph{Schriftsteller, Mediziner}!Abschiedssouper1892@\strich\emph{Abschiedssouper} {[}1892{]}|pwv}. – \pend
           \pstart
           Urtheil \textsc{Friese}\pwindex{Friese, Carl Adolph 24.10.1831 – 24.01.1900@\textsc{Friese, Carl Adolph} (24.10.1831 – 24.01.1900), \emph{Regisseur, Schauspieler}|pw}’s: Es iſt {\pb}ein Skandal, ſo was
                    aufzuführen. – Frau \textsc{Friese}\pwindex{Skura, Josefine 1841 – 1913@\textsc{Skura, Josefine} (1841 – 1913), \emph{Schauspielerin}|pw} (dieſe alte Stabscanaille, wie \textsc{Jarno}\pwindex{Jarno, Josef 24.08.1865 – 11.01.1932@\textsc{Jarno, Josef} (24.08.1865 – 11.01.1932), \emph{Theaterleiter, Schauspieler}|pw}{ }ſagt) hat ſich \uline{geſchämt}, wie ſie das
                        Abſch.-\textsc{souper}\pwindex{Schnitzler, Arthur 15.05.1862 – 21.10.1931@\textsc{Schnitzler, Arthur} (15.05.1862 – 21.10.1931), \emph{Schriftsteller, Mediziner}!Abschiedssouper1892@\strich\emph{Abschiedssouper} {[}1892{]}|pw} geleſen. –\pend
           \pstart
           Die Cenſur ſtrich: \uline{am Buſen geruht} u ſetzte dafür
                        \uline{gekoſt}. –\pend
           \pstart
           – Ob mir die Geſchichte für Berlin\oindex{Berlin@\textbf{Berlin}|pw} nützen
                    wird, iſt nicht abzuſehen – da \textsc{Jarno}\pwindex{Jarno, Josef 24.08.1865 – 11.01.1932@\textsc{Jarno, Josef} (24.08.1865 – 11.01.1932), \emph{Theaterleiter, Schauspieler}|pw} höchſt un{\pb}verläßlich zu ſein ſcheint.
               Ihm hat die Frage a. d. Sch.\pwindex{Schnitzler, Arthur 15.05.1862 – 21.10.1931@\textsc{Schnitzler, Arthur} (15.05.1862 – 21.10.1931), \emph{Schriftsteller, Mediziner}!Frage an das Schicksal01. 05. 1890@\strich\emph{Die Frage an das Schicksal} {[}01. 05. 1890{]}|pw}{ }ſchon 150 Mark
                    getragen – ſo viel bekam jeder der Mitwirkenden bei \label{K_L00236_2v}\edtext{\textsc{Grelling}\pwindex{Grelling, Richard 11.06.1853 – 15.01.1929@\textsc{Grelling, Richard} (11.06.1853 – 15.01.1929), \emph{Schriftsteller, Rechtsanwalt, Publizist}|pw}}{\lemma{\textnormal{\emph{Grelling}}}\Cendnote{\textnormal{Privataufführung bei Richard Grelling\pwindex{Grelling, Richard 11.06.1853 – 15.01.1929@\textsc{Grelling, Richard} (11.06.1853 – 15.01.1929), \emph{Schriftsteller, Rechtsanwalt, Publizist}|pwk} kurz vor dem
                            14. 1. 1891.}}}\label{K_L00236_2h}. –\pend
           \pstart
           Gearbeitet hab ich beinah nichts; alles ungewiſſe, ſo nichtig es ſein mag,
                    beſchäftigt nach außen hin u macht daher nervös, – Hoffentlich haben {\pb}Sie Ihre glückliche Verſeſti{\geminationm}ung wiedergefunden. – Schade, daſs Sie Freitag
                    nicht da ſind.\pend
           \pstart
           Herzlichen Gruß{\\[\baselineskip]}Ihr\spacefill\mbox{Arth.}\pend
           \leftskip=0em{}\pstart
           \textsc{Ischl}\oindex{Bad Ischl@\textbf{Bad Ischl}|pw}, 12. 7. 93.\pend
           \endnumbering\briefempfaengerindex{Hofmannsthal, Hugo von@\textsc{Hofmannsthal, Hugo von}!zzzSchnitzler, Arthur@\emph{von Arthur Schnitzler}!1893-07-121@{12. 7. 1893}|)be}\mylabel{h}\end{ledgroupsized}  \newcommand{\dateiname}{L00236}\newcommand{\titel}{Arthur Schnitzler an Hugo von Hofmannsthal, 12. 7. 1893}\newcommand{\editorInnen}{Martin Anton Müller und Gerd-Hermann Susen}
            \footnotesize
\begin{ledgroupsized}[t]{11.5cm}
\doendnotes{C}
\end{ledgroupsized}
         %% latex-leseansicht-abspann.tex
%% Abspann für die Leseansicht.
%% Der Schalter \ifkorrekturansicht ist bereits durch den Vorspann gesetzt.

%% latex-abspann.tex
%% Gemeinsamer Abspann für Korrekturansicht und Leseansicht.
%% Setzt den Schalter \ifkorrekturansicht voraus (gesetzt in den
%% einbindenden Dateien latex-korrekturansicht-abspann.tex bzw.
%% latex-leseansicht-abspann.tex).
%% ---------------------------------------------------------------

\normalsize

% Das esempio-Environment wird nur in der Leseansicht benötigt
\ifkorrekturansicht\else
\newenvironment{esempio}[3]%
{
    \vspace{1.5ex}
    \rlap{\underline{#1}}
    \par
    \setlength{\parindent}{0cm}
    \nopagebreak
    \leftskip=#2cm
    \rightskip=#3cm
}
{
    \par
}
\fi

\doendnotes{C}
\bigskip
\vfill

\clearpage

\footnotesize

\ifkorrekturansicht
  \lohead{\textsc{register}}
\fi

% theindex-Environment neu definieren ohne reledmac
\makeatletter
\renewenvironment{theindex}{%
  \ifkorrekturansicht
    \section*{\indexname}%
  \else
    \subsubsection*{Index der erwähnten Entitäten}%
  \fi
  \setlength{\parindent}{0pt}%
  \setlength{\parskip}{0pt plus 0.3pt}%
  \let\item\@idxitem
}{%
  \ifkorrekturansicht\clearpage\fi
}
\makeatother

\IfFileExists{\jobname-pw.ind}{\input{\jobname-pw.ind}}{}

% Quellenangabe nur in der Leseansicht
\ifkorrekturansicht\else
% Fallback-Definitionen, falls die .tex-Datei \titel etc. nicht gesetzt hat
\providecommand{\titel}{}
\providecommand{\editorInnen}{}
\providecommand{\dateiname}{\jobname}

\vspace{3cm}

\vfill

\footnotesize
\textsc{Quelle}: \titel. Herausgegeben von {\editorInnen}. In: \emph{Arthur Schnitzler: Briefwechsel mit Autorinnen und Autoren}.
 Digitale Edition, https://schnitzler-briefe.acdh.oeaw.ac.at/{\dateiname}.html (Stand \today)
\fi

\end{document}


      