%% latex-korrekturansicht-vorspann.tex
%% Vorspann für die Korrekturansicht.
%% Lädt die gemeinsame Datei latex-vorspann.tex mit gesetztem Schalter.

\newif\ifkorrekturansicht
\korrekturansichttrue

\input{../tex-inputs/latex-vorspann}


\section[Arthur Schnitzler an Hugo von Hofmannsthal, 12. 7. 1893]{L00236 Arthur Schnitzler an Hugo von Hofmannsthal, 12. 7. 1893}
\nopagebreak\mylabel{L00236v}
\rehead{ }\normalsize\beginnumbering\briefempfaengerindex{Hofmannsthal, Hugo von@\textsc{Hofmannsthal, Hugo von}!zzzSchnitzler, Arthur@\emph{von Arthur Schnitzler}!1893-07-121@{12. 7. 1893}|(be}
\toendnotes[C]{\smallbreak\pagebreak[2]}\Standort{FDH, Hs-30885,36.}
\physDesc{Brief, 1 Blatt, 4 Seiten, 901 Zeichen (Briefpapier mit Trauerrand)
\newline{}Handschrift: schwarze Tinte, deutsche Kurrent
\newline{}Ordnung: mit Bleistift von Schnitzler mutmaßlich bei der Durchsicht der Korrespondenz
                                    1929  datiert: »12. 7. 93« }
\buchAbdrucke{\weitereDrucke{Hugo von Hofmannsthal, Arthur Schnitzler: \emph{Briefwechsel}. Frankfurt am Main: \emph{S. Fischer} 1964, S. 40.} }\toendnotes[C]{\smallbreak}
\pstart{}{\pb}Lieber Loris,\pend\vspace{0.5em}
\pstart
           meine \label{K_L00236-1v}\edtext{Einakter\pwindex{Abschiedssouper@\emph{Abschiedssouper}|pwv}\pwindex{Frage an das Schicksal@\emph{Die Frage an das Schicksal}|pwv}}{\lemma{\textnormal{\emph{Einakter}}}\Cendnote{\textnormal{Nur \emph{Abschiedssouper}\pwindex{Abschiedssouper@\emph{Abschiedssouper}|pwk} wurde gegeben.}}}\label{K_L00236-1}{ }ſind Freitag. Erſte Probe geſtern –
                  Anatol\pwindex{Abschiedssouper@\emph{Abschiedssouper}|pwv} (Herr \textsc{Hoefer}\pwindex{Hoefer, Emil 14.05.1864 – 01.05.1940@\textsc{Höfer, Emil} (14.05.1864 – 01.05.1940), \emph{Schauspieler/Schauspielerin}|pw}) erſchien einfach nicht. – Ich nahm mit \textsc{Jarno}\pwindex{Jarno, Josef 24.08.1865 – 11.01.1932@\textsc{Jarno, Josef} (24.08.1865 – 11.01.1932), \emph{Theaterleiter/Theaterleiterin, Schauspieler/Schauspielerin}|pw} die Stücke\pwindex{Abschiedssouper@\emph{Abschiedssouper}|pwv}\pwindex{Frage an das Schicksal@\emph{Die Frage an das Schicksal}|pwv}
               durch; Inſcenierung, Stellung etc. – Die \textsc{Griebl}\pwindex{Griebl, Marie 1872-02-27 – 1952-06-08@\textsc{Griebl, Marie} (1872-02-27 – 1952-06-08), \emph{Schauspieler/Schauspielerin, Sänger/Sängerin}|pw} gibt die \textsc{Annie}\pwindex{Abschiedssouper@\emph{Abschiedssouper}|pwv}. – \pend
           
\pstart
           Urtheil \textsc{Friese}\pwindex{Friese, Carl Adolph 24.10.1831 – 24.01.1900@\textsc{Friese, Carl Adolph} (24.10.1831 – 24.01.1900), \emph{Regisseur/Regisseurin, Schauspieler/Schauspielerin}|pw}’s: Es iſt {\pb}ein Skandal, ſo was aufzuführen. – Frau
                  \textsc{Friese}\pwindex{Skura, Josefine 1841 – 1913@\textsc{Skura, Josefine} (1841 – 1913), \emph{Schauspieler/Schauspielerin}|pw} (dieſe alte Stabscanaille, wie \textsc{Jarno}\pwindex{Jarno, Josef 24.08.1865 – 11.01.1932@\textsc{Jarno, Josef} (24.08.1865 – 11.01.1932), \emph{Theaterleiter/Theaterleiterin, Schauspieler/Schauspielerin}|pw}{ }ſagt) hat ſich \uline{geſchämt}, wie ſie das Abſch.-\textsc{souper}\pwindex{Abschiedssouper@\emph{Abschiedssouper}|pw} geleſen. –\pend
           
\pstart
           Die Cenſur ſtrich: \uline{am Buſen geruht} u ſetzte dafür \uline{gekoſt}. –\pend
           
\pstart
           – Ob mir die Geſchichte für Berlin\oindex{Berlin@\textbf{Berlin}, \emph{P.PPLC}|pw} nützen wird,
               iſt nicht abzuſehen – da \textsc{Jarno}\pwindex{Jarno, Josef 24.08.1865 – 11.01.1932@\textsc{Jarno, Josef} (24.08.1865 – 11.01.1932), \emph{Theaterleiter/Theaterleiterin, Schauspieler/Schauspielerin}|pw} höchſt un{\pb}verläßlich zu ſein ſcheint. Ihm hat die
                  Frage a. d. Sch.\pwindex{Frage an das Schicksal@\emph{Die Frage an das Schicksal}|pw}{ }ſchon 150 Mark getragen – ſo viel bekam jeder der
               Mitwirkenden bei \label{K_L00236-2v}\edtext{\textsc{Grelling}\pwindex{Grelling, Richard 11.06.1853 – 15.01.1929@\textsc{Grelling, Richard} (11.06.1853 – 15.01.1929), \emph{Schriftsteller/Schriftstellerin, Rechtsanwalt/Rechtsanwältin, Publizist/Publizistin}|pw}}{\lemma{\textnormal{\emph{Grelling}}}\Cendnote{\textnormal{Privataufführung bei Richard Grelling\pwindex{Grelling, Richard 11.06.1853 – 15.01.1929@\textsc{Grelling, Richard} (11.06.1853 – 15.01.1929), \emph{Schriftsteller/Schriftstellerin, Rechtsanwalt/Rechtsanwältin, Publizist/Publizistin}|pwk} kurz vor dem
                  14. 1. 1891.}}}\label{K_L00236-2}. –\pend
           
\pstart
           Gearbeitet hab ich beinah nichts; alles ungewiſſe, ſo nichtig es ſein mag,
               beſchäftigt nach außen hin u macht daher nervös, – Hoffentlich haben {\pb}Sie Ihre glückliche Verſeſti{\geminationm}ung wiedergefunden. – Schade, daſs Sie Freitag nicht da ſind.\pend
           
\pstart
           Herzlichen Gruß{\\[\baselineskip]}Ihr\spacefill\mbox{Arth.}\pend
           \leftskip=0em{}
\pstart
           \textsc{Ischl}\oindex{Bad Ischl@\textbf{Bad Ischl}, \emph{P.PPL}|pw}, 12. 7. 93.\pend
           \selectlanguage{ngerman}\endnumbering\briefempfaengerindex{Hofmannsthal, Hugo von@\textsc{Hofmannsthal, Hugo von}!zzzSchnitzler, Arthur@\emph{von Arthur Schnitzler}!1893-07-121@{12. 7. 1893}|)be}\mylabel{L00236h}  \normalsize

\doendnotes{C}
\bigskip
\vfill

\clearpage

\footnotesize

\lohead{\textsc{register}}

% Definiere theindex-Environment komplett neu ohne reledmac
\makeatletter
\renewenvironment{theindex}{%
  \section*{\indexname}%
  \setlength{\parindent}{0pt}%
  \setlength{\parskip}{0pt plus 0.3pt}%
  \let\item\@idxitem
}{%
  \clearpage
}
\makeatother

\IfFileExists{\jobname-pw.ind}{\input{\jobname-pw.ind}}{}

\end{document}

      