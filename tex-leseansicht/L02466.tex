%% latex-leseansicht-vorspann.tex
%% Vorspann für die Leseansicht.
%% Lädt die gemeinsame Datei latex-vorspann.tex mit nicht gesetztem Schalter.

\newif\ifkorrekturansicht
\korrekturansichtfalse

\input{../tex-inputs/latex-vorspann}


\section[Hugo Hofmannsthal an Arthur Schnitzler, 9. 3. {[}1926{]}]{L02466 Hugo Hofmannsthal an Arthur Schnitzler, 9. 3. [1926]}
\nopagebreak\mylabel{L02466v}
\rehead{ }\normalsize\beginnumbering\briefempfaengerindex{Schnitzler, Arthur@\textsc{Schnitzler, Arthur}!zzzHofmannsthal, Hugo von@\emph{von Hugo von Hofmannsthal}!1926-03-091@{9. 3. [1926]}|(be}
\toendnotes[C]{\smallbreak\pagebreak[2]}
\correspDesc{Versand  durch Hugo von Hofmannsthal am 9. 3. [1926] in Rodaun
\newline{}Erhalt  durch Arthur Schnitzler im Zeitraum [10. 3. 1926
                  – 14. 3. 1926?] in Wien}\toendnotes[C]{\smallbreak}
\Standort{CUL, Schnitzler, B 43.}
\physDesc{Brief, 2 Blätter, 4 Seiten, 1912 Zeichen (das zweite Blatt nummeriert mit:
                                 »II.«)
\newline{}Handschrift: schwarze Tinte, lateinische Kurrent
\newline{}Schnitzler: mit Bleistift die Jahreszahl ergänzt: »26« und beschriftet: »\textsc{Hugo}«. Datiert: »9/3 26« 
\newline{}Ordnung: 1) mit Bleistift von unbekannter Hand auf der zweiten Seite der
                                 Vermerk »Abgeschrieben« und auf der vierten,
                                 ansonsten unbeschriebenen Seite der Name: »{\pb}Hofmannsthal«  2) mit Bleistift von unbekannter Hand nummeriert:
                                    »370« 3) mit Bleistift von unbekannter Hand nummeriert:
                                    »379«}
\buchAbdrucke{\weitereDrucke{Hugo von Hofmannsthal, Arthur Schnitzler: \emph{Briefwechsel}. Herausgegeben von Therese Nickl und Heinrich Schnitzler. Frankfurt am Main: \emph{S. Fischer} 1964, S. 305.} }
\pstart
           {\pb}Rodaun\oindex{Wien@\textbf{Wien}!XXIII., Liesing@\textbf{XXIII., Liesing}!Rodaun@\textbf{Rodaun}, \emph{Region}|pw}{ }9 III\pend
           
\pstart{}mein lieber Arthur\pend\vspace{0.5em}
\pstart
           Lili\pwindex{Cappellini, Lili 13.\,9.\,1909 Wien – 26.\,7.\,1928 Venedig@\textsc{Cappellini, Lili} (13.\,9.\,1909 Wien – 26.\,7.\,1928 Venedig)|pw}, das hübsche, Hüte wechselnde, schwer
               wiederzuerkennende Wesen sagt mir, dass Sie schon eine ganze Weile zurück sind, in
               dessen ich Sie noch in Deutschland\oindex{Deutschland@\textbf{Deutschland}|pw} glaubte.\pend
           
\pstart
           Sie soll mir nur freundlich verzeihen und mich immer etwas vertraulich anlächeln.
               Denn ich habe nicht etwa ein schlechtes Physiognomieengedächtnis, sondern etwas viel
               Sonderbareres. Meine Phantasie verändert mir das Erinnerungsbild, sie gestaltet um,
               verschärft einen besti{\geminationm}ten Zug, und tritt dann das
               Original vor mich, so weigert sich die Phantasie, die Identität anzuerke{\geminationn}en. Ich grüße infolgedessen in einem Theater oder auf
               der Gasse fast nur fremde Menschen, deren Gesichter ich mit vermeintlichen Gesichtern
               in einen plausiblen {\pb}Zusammenhang
               bringe. Außerdem aber habe ich schlechte Augen.\hspace*{1.5em}Soviel nun von Lili\pwindex{Cappellini, Lili 13.\,9.\,1909 Wien – 26.\,7.\,1928 Venedig@\textsc{Cappellini, Lili} (13.\,9.\,1909 Wien – 26.\,7.\,1928 Venedig)|pw} u. meinen schwierigen,
               durch wechselnde Hüte und wechselnden Ausdruck noch erschwerten Begegnungen mit ihr.
               Jetzt aber eine Bitte, eine Quälerei, eine mehr zu den vielen die jede Post bringt.
               Aber ich wage es, denn es handelt sich darum, einem ordentlichen, in die schwierigste
               Lage geratenen Menschen zu helfen. Der Verleger Erich Reiss\pwindex{Reiss, Erich 24.\,1.\,1887 Berlin – 8.\,5.\,1951 New York City@\textsc{Reiss, Erich} (24.\,1.\,1887 Berlin – 8.\,5.\,1951 New York City), \emph{Verleger}|pw} (Verleger von Brandes\pwindex{Brandes, Georg 4.\,2.\,1842 Kopenhagen – 19.\,2.\,1927 ebd.@\textsc{Brandes, Georg} (4.\,2.\,1842 Kopenhagen – 19.\,2.\,1927 ebd.)|pw} und
               anderen, fast lauter guter Sachen) ist zusammengebrochen. Es wäre ihm vom größten
               Nutzen, vor allem moralisch, wenn Sie (ebenso wie ich) die Güte haben wollten, ein
               paar Zeilen in Maschinschrift zu dictieren, worin Sie bekunden dass der Verlag Erich Reiss\orgindex{Erich-Reiss-Verlag@Erich-Reiss-Verlag|pw}{ }{\pb}ein Unternehmen von culturellem
               Wert war.\pend
           
\pstart
           Bitte tun Sie es auch mir zu lieb, ich kenne den Menschen seit vielen Jahren, und
               durchaus im Guten.\pend
           
\pstart
           Ein paar überaus liebe Zeilen, die Sie mir vor vielen Wochen schrieben, klingen immer
               in mir nach.\hspace*{1.5em}Soll ich, wenn es freundlicher wird, zu
               einem Vormittagsspaziergang hinüber ko{\geminationm}en?\hspace*{1.5em}Oder gibts eine andere Form der Begegnung, die Ihnen
               nicht beschwerend ist?\pend
           \pstart In Freundschaft Ihr\spacefill\mbox{Hugo.}\pend{}
\pstart
           \noindent{}PS Die Sache mit E. Reiss\pwindex{Reiss, Erich 24.\,1.\,1887 Berlin – 8.\,5.\,1951 New York City@\textsc{Reiss, Erich} (24.\,1.\,1887 Berlin – 8.\,5.\,1951 New York City), \emph{Verleger}|pw} ist, soviel ich
                  verstehe, dringend eilig!\pend
           \selectlanguage{ngerman}\endnumbering\briefempfaengerindex{Schnitzler, Arthur@\textsc{Schnitzler, Arthur}!zzzHofmannsthal, Hugo von@\emph{von Hugo von Hofmannsthal}!1926-03-091@{9. 3. [1926]}|)be}\mylabel{L02466h}  \newcommand{\dateiname}{L02466}\newcommand{\titel}{Hugo Hofmannsthal an Arthur Schnitzler, 9. 3. [1926]}\newcommand{\editorInnen}{Martin Anton Müller und Gerd-Hermann Susen}%% latex-leseansicht-abspann.tex
%% Abspann für die Leseansicht.
%% Der Schalter \ifkorrekturansicht ist bereits durch den Vorspann gesetzt.

%% latex-abspann.tex
%% Gemeinsamer Abspann für Korrekturansicht und Leseansicht.
%% Setzt den Schalter \ifkorrekturansicht voraus (gesetzt in den
%% einbindenden Dateien latex-korrekturansicht-abspann.tex bzw.
%% latex-leseansicht-abspann.tex).
%% ---------------------------------------------------------------

\normalsize

% Das esempio-Environment wird nur in der Leseansicht benötigt
\ifkorrekturansicht\else
\newenvironment{esempio}[3]%
{
    \vspace{1.5ex}
    \rlap{\underline{#1}}
    \par
    \setlength{\parindent}{0cm}
    \nopagebreak
    \leftskip=#2cm
    \rightskip=#3cm
}
{
    \par
}
\fi

\doendnotes{C}
\bigskip
\vfill

\clearpage

\footnotesize

\ifkorrekturansicht
  \lohead{\textsc{register}}
\fi

% theindex-Environment neu definieren ohne reledmac
\makeatletter
\renewenvironment{theindex}{%
  \ifkorrekturansicht
    \section*{\indexname}%
  \else
    \subsubsection*{Index der erwähnten Entitäten}%
  \fi
  \setlength{\parindent}{0pt}%
  \setlength{\parskip}{0pt plus 0.3pt}%
  \let\item\@idxitem
}{%
  \ifkorrekturansicht\clearpage\fi
}
\makeatother

\IfFileExists{\jobname-pw.ind}{\input{\jobname-pw.ind}}{}

% Quellenangabe nur in der Leseansicht
\ifkorrekturansicht\else
% Fallback-Definitionen, falls die .tex-Datei \titel etc. nicht gesetzt hat
\providecommand{\titel}{}
\providecommand{\editorInnen}{}
\providecommand{\dateiname}{\jobname}

\vspace{3cm}

\vfill

\footnotesize
\textsc{Quelle}: \titel. Herausgegeben von {\editorInnen}. In: \emph{Arthur Schnitzler: Briefwechsel mit Autorinnen und Autoren}.
 Digitale Edition, https://schnitzler-briefe.acdh.oeaw.ac.at/{\dateiname}.html (Stand \today)
\fi

\end{document}


