%% latex-korrekturansicht-vorspann.tex
%% Vorspann für die Korrekturansicht.
%% Lädt die gemeinsame Datei latex-vorspann.tex mit gesetztem Schalter.

\newif\ifkorrekturansicht
\korrekturansichttrue

\input{../tex-inputs/latex-vorspann}


\section[Arthur Schnitzler an Richard Beer-Hofmann, 23. 10. 1896]{L00608 Arthur Schnitzler an Richard Beer-Hofmann, 23. 10. 1896}
\nopagebreak\mylabel{L00608v}
\rehead{ }\normalsize\beginnumbering\briefempfaengerindex{Beer-Hofmann, Richard@\textsc{Beer-Hofmann, Richard}!zzzSchnitzler, Arthur@\emph{von Arthur Schnitzler}!1896-10-231@{23. 10. 1896}|(be}
\toendnotes[C]{\smallbreak\pagebreak[2]}\Standort{YCGL, MSS 31.}
\physDesc{Brief, 1 Blatt, 1 Seite, Umschlag, 249 Zeichen
\newline{}Handschrift: Bleistift, deutsche Kurrent
\newline{}Versand: Stempel: »\nobreak{}\oindex{I., Innere Stadt@\textbf{I., Innere Stadt}, \emph{A.ADM3}|pwk}Wien 1/1, 23. 10. 96, 6–7N\nobreak{}«.  }
\buchAbdrucke{\weitereDrucke{Arthur Schnitzler, Richard Beer-Hofmann: \emph{Briefwechsel 1891–1931}. Wien, Zürich: \emph{Europaverlag} 1992, S. 99.} }\toendnotes[C]{\smallbreak}\pstart{}{\pb}Herrn \textsc{Dr. Rich.
                     Beer-Hofmann}\pend{}\pstart{}Wien\oindex{Wien@\textbf{Wien}, \emph{A.ADM2}|pw}\pend{}\pstart{}I. \textsc{Wollzeile 15}\oindex{Wollzeile@\textbf{Wollzeile}, \emph{Straße (K.STR)}|pw},\pend{}\pstart{}4. Stock.\pend{}{\bigskip}\vspace{1em}
\pstart{}{\pb}Lieber Richard.\pend\vspace{0.5em}
\pstart
           Morgen geh ich ins \textsc{Raimund}theat (Ebne Erde u erſter Stock\pwindex{Zu ebener Erde und erster Stock oder Die Launen des Gluecks@\emph{Zu ebener Erde und erster Stock oder Die Launen des Glücks}|pw}.)\pend
           
\pstart
           Montag reiſe ich ſpäteſtens ab. \textsc{Première}\pwindex{Freiwild. Schauspiel in 3 Akten@\emph{Freiwild. Schauspiel in 3 Akten}|pwv}{ }ſchon am 31. –\pend
           
\pstart
           Schreib\textcolor{gray}{en} Sie mir wo Sie morgen Abend ſind \textsc{etc}.\pend
           
\pstart
           Herzlich der Ihre{\\[\baselineskip]}\spacefill\mbox{Arthur}\pend
           \leftskip=0em{}\selectlanguage{ngerman}\endnumbering\briefempfaengerindex{Beer-Hofmann, Richard@\textsc{Beer-Hofmann, Richard}!zzzSchnitzler, Arthur@\emph{von Arthur Schnitzler}!1896-10-231@{23. 10. 1896}|)be}\mylabel{L00608h}  \normalsize

\doendnotes{C}
\bigskip
\vfill

\clearpage

\footnotesize

\lohead{\textsc{register}}

% Definiere theindex-Environment komplett neu ohne reledmac
\makeatletter
\renewenvironment{theindex}{%
  \section*{\indexname}%
  \setlength{\parindent}{0pt}%
  \setlength{\parskip}{0pt plus 0.3pt}%
  \let\item\@idxitem
}{%
  \clearpage
}
\makeatother

\IfFileExists{\jobname-pw.ind}{\input{\jobname-pw.ind}}{}

\end{document}

      