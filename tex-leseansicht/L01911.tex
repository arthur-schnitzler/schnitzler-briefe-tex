%% latex-korrekturansicht-vorspann.tex
%% Vorspann für die Korrekturansicht.
%% Lädt die gemeinsame Datei latex-vorspann.tex mit gesetztem Schalter.

\newif\ifkorrekturansicht
\korrekturansichttrue

\input{../tex-inputs/latex-vorspann}


\section[Richard Beer-Hofmann an Arthur Schnitzler, 18. 1. 1910]{L01911 Richard Beer-Hofmann an Arthur Schnitzler, 18. 1. 1910}
\nopagebreak\mylabel{L01911v}
\rehead{ }\normalsize\beginnumbering\briefempfaengerindex{Schnitzler, Arthur@\textsc{Schnitzler, Arthur}!zzzBeer-Hofmann, Richard@\emph{von Richard Beer-Hofmann}!1910-01-181@{18. 1. 1910}|(be}
\toendnotes[C]{\smallbreak\pagebreak[2]}\Standort{CUL, Schnitzler, B 8.}
\physDesc{Kartenbrief, 340 Zeichen
\newline{}Handschrift: Bleistift, lateinische Kurrent
\newline{}Versand: ohne postalischen Übermittlungsvermerk 
\newline{}Schnitzler: mit Bleistift beschriftet: »\textsc{BH}« 
\newline{}Ordnung: mit Bleistift von unbekannter Hand nummeriert:
                                    »227« }\toendnotes[C]{\smallbreak}\pstart{}{\pb}Herrn\pend{}\pstart{}Arthur Schnitzler\pend{}\pstart{}Spöttelgasse\oindex{Edmund-Weiss-Gasse 7@\textbf{Edmund-Weiß-Gasse 7}, \emph{Wohngebäude (K.WHS)}|pw} 7\pend{}{\bigskip}\vspace{1em}
\pstart
           \raggedleft{}{\pb}18/I 10\pend
           
\pstart{}Lieber Arthur!\pend\vspace{0.5em}
\pstart
           Bitte, veranlassen Sie, dass das bewusste \label{K_L01911-1v}\edtext{Fräulein\pwindex{Reiter, Anna @\textsc{Reiter, Anna}, \emph{Hausschneider/Hausschneiderin}|pwv}}{\lemma{\textnormal{\emph{Fräulein}}}\Cendnote{\textnormal{Vgl. A. S.: \emph{Tagebuch}, 19. 1. 1910.
               }}}\label{K_L01911-1}{ }\uline{nicht} zwischen halbdrei – halbvier,
               sondern erst wenn Sie von Ihnen weggeht – also zwischen 6 und
                  7 zu uns kommt{[}.{]}{ }{\pb}Sie collidirt sonst mit den
               Fräuleins die wir von Schallingers\pwindex{Schallinger @\textsc{Schallinger}|pw}\pwindex{Schallinger @\textsc{Schallinger}|pw}
               erwarten.\pend
           
\pstart
           Herzlichst mit allen guten Wünschen für \label{K_L01911-2v}\edtext{Dresden\oindex{Dresden@\textbf{Dresden}, \emph{P.PPLA}|pw}}{\lemma{\textnormal{\emph{Dresden}}}\Cendnote{\textnormal{Schnitzler reiste am 20. 1. 1910 zur
                  Premiere von \emph{Der Schleier der
                  Pierrette}\pwindex{Schleier der Pierrette. Pantomime in drei Bildern@\emph{Der Schleier der Pierrette. Pantomime in drei Bildern}|pwk}.}}}\label{K_L01911-2}{\\[\baselineskip]}Ihr{\\[\baselineskip]}\spacefill\mbox{Richard}\pend
           \leftskip=0em{}\selectlanguage{ngerman}\endnumbering\briefempfaengerindex{Schnitzler, Arthur@\textsc{Schnitzler, Arthur}!zzzBeer-Hofmann, Richard@\emph{von Richard Beer-Hofmann}!1910-01-181@{18. 1. 1910}|)be}\mylabel{L01911h}  \normalsize

\doendnotes{C}
\bigskip
\vfill

\clearpage

\footnotesize

\lohead{\textsc{register}}

% Definiere theindex-Environment komplett neu ohne reledmac
\makeatletter
\renewenvironment{theindex}{%
  \section*{\indexname}%
  \setlength{\parindent}{0pt}%
  \setlength{\parskip}{0pt plus 0.3pt}%
  \let\item\@idxitem
}{%
  \clearpage
}
\makeatother

\IfFileExists{\jobname-pw.ind}{\input{\jobname-pw.ind}}{}

\end{document}

      