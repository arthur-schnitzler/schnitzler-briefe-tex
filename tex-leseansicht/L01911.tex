%% latex-leseansicht-vorspann.tex
%% Vorspann für die Leseansicht.
%% Lädt die gemeinsame Datei latex-vorspann.tex mit nicht gesetztem Schalter.

\newif\ifkorrekturansicht
\korrekturansichtfalse

\input{../tex-inputs/latex-vorspann}


\section[Richard Beer-Hofmann an Arthur Schnitzler, 18. 1. 1910]{L01911 Richard Beer-Hofmann an Arthur Schnitzler, 18. 1. 1910}
\nopagebreak\mylabel{L01911v}
\rehead{ }\normalsize\beginnumbering\briefempfaengerindex{Schnitzler, Arthur@\textsc{Schnitzler, Arthur}!zzzBeer-Hofmann, Richard@\emph{von Richard Beer-Hofmann}!1910-01-182@{18. 1. 1910}|(be}
\toendnotes[C]{\smallbreak\pagebreak[2]}
\correspDesc{Versand  durch Richard Beer-Hofmann am 18. 1. 1910 in Wien
\newline{}Erhalt  durch Arthur Schnitzler am 18. 1. 1910 in Wien}\toendnotes[C]{\smallbreak}
\Standort{CUL, Schnitzler, B 8.}
\physDesc{Kartenbrief, 340 Zeichen
\newline{}Handschrift: Bleistift, lateinische Kurrent
\newline{}Versand: ohne postalischen Übermittlungsvermerk 
\newline{}Schnitzler: mit Bleistift beschriftet: »\textsc{BH}« 
\newline{}Ordnung: mit Bleistift von unbekannter Hand nummeriert:
                                    »227« }\toendnotes[C]{\smallbreak}\pstart{}{\pb}Herrn\pend{}\pstart{}Arthur Schnitzler\pend{}\pstart{}Spöttelgasse\oindex{Wien@\textbf{Wien}!XVIII., Währing@\textbf{XVIII., Währing}!Edmund-Weiß-Gasse 7@\textbf{Edmund-Weiß-Gasse 7}, \emph{Wohngebäude}|pw} 7\pend{}{\bigskip}\vspace{1em}
\pstart
           \raggedleft{}{\pb}18/I 10\pend
           
\pstart{}Lieber Arthur!\pend\vspace{0.5em}
\pstart
           Bitte, veranlassen Sie, dass das bewusste \label{K_L01911-1v}\edtext{Fräulein\pwindex{Reiter, Anna @\textsc{Reiter, Anna}, \emph{Hausschneiderin}|pwv}}{\lemma{\textnormal{\emph{Fräulein}}}\Cendnote{\textnormal{Vgl. A. S.: \emph{Tagebuch}, 19. 1. 1910.
               }}}\label{K_L01911-1}{ }\uline{nicht} zwischen halbdrei – halbvier,
               sondern erst wenn Sie von Ihnen weggeht – also zwischen 6 und
                  7 zu uns kommt{[}.{]}{ }{\pb}Sie collidirt sonst mit den
               Fräuleins die wir von Schallingers\pwindex{Schallinger @\textsc{Schallinger}|pw}\pwindex{Schallinger @\textsc{Schallinger}|pw}
               erwarten.\pend
           
\pstart
           Herzlichst mit allen guten Wünschen für \label{K_L01911-2v}\edtext{Dresden\oindex{Dresden@\textbf{Dresden}|pw}}{\lemma{\textnormal{\emph{Dresden}}}\Cendnote{\textnormal{Schnitzler reiste am 20. 1. 1910 zur
                  Uraufführung\eventindex{Semperoper@\textbf{Semperoper}!Uraufführung von Der Schleier der Pierrette, Premiere von Versiegelt, 22.1.1910@Uraufführung von Der Schleier der Pierrette, Premiere von Versiegelt, 22.1.1910|pwkv} von \emph{Der Schleier der
                  Pierrette}\pwindex{Schnitzler, Arthur 15.\,5.\,1862 Wien – 21.\,10.\,1931 ebd.@\textsc{Schnitzler, Arthur} (15.\,5.\,1862 Wien – 21.\,10.\,1931 ebd.), \emph{Schriftsteller, Mediziner}!Schleier der Pierrette. Pantomime in drei Bildern@\strich\emph{Der Schleier der Pierrette. Pantomime in drei Bildern}|pwk}.}}}\label{K_L01911-2}{\\[\baselineskip]}Ihr{\\[\baselineskip]}\spacefill\mbox{Richard}\pend
           \leftskip=0em{}\selectlanguage{ngerman}\endnumbering\briefempfaengerindex{Schnitzler, Arthur@\textsc{Schnitzler, Arthur}!zzzBeer-Hofmann, Richard@\emph{von Richard Beer-Hofmann}!1910-01-182@{18. 1. 1910}|)be}\mylabel{L01911h}  \newcommand{\dateiname}{L01911}\newcommand{\titel}{Richard Beer-Hofmann an Arthur Schnitzler, 18. 1. 1910}\newcommand{\editorInnen}{Martin Anton Müller und Gerd-Hermann Susen}%% latex-leseansicht-abspann.tex
%% Abspann für die Leseansicht.
%% Der Schalter \ifkorrekturansicht ist bereits durch den Vorspann gesetzt.

%% latex-abspann.tex
%% Gemeinsamer Abspann für Korrekturansicht und Leseansicht.
%% Setzt den Schalter \ifkorrekturansicht voraus (gesetzt in den
%% einbindenden Dateien latex-korrekturansicht-abspann.tex bzw.
%% latex-leseansicht-abspann.tex).
%% ---------------------------------------------------------------

\normalsize

% Das esempio-Environment wird nur in der Leseansicht benötigt
\ifkorrekturansicht\else
\newenvironment{esempio}[3]%
{
    \vspace{1.5ex}
    \rlap{\underline{#1}}
    \par
    \setlength{\parindent}{0cm}
    \nopagebreak
    \leftskip=#2cm
    \rightskip=#3cm
}
{
    \par
}
\fi

\doendnotes{C}
\bigskip
\vfill

\clearpage

\footnotesize

\ifkorrekturansicht
  \lohead{\textsc{register}}
\fi

% theindex-Environment neu definieren ohne reledmac
\makeatletter
\renewenvironment{theindex}{%
  \ifkorrekturansicht
    \section*{\indexname}%
  \else
    \subsubsection*{Index der erwähnten Entitäten}%
  \fi
  \setlength{\parindent}{0pt}%
  \setlength{\parskip}{0pt plus 0.3pt}%
  \let\item\@idxitem
}{%
  \ifkorrekturansicht\clearpage\fi
}
\makeatother

\IfFileExists{\jobname-pw.ind}{\input{\jobname-pw.ind}}{}

% Quellenangabe nur in der Leseansicht
\ifkorrekturansicht\else
% Fallback-Definitionen, falls die .tex-Datei \titel etc. nicht gesetzt hat
\providecommand{\titel}{}
\providecommand{\editorInnen}{}
\providecommand{\dateiname}{\jobname}

\vspace{3cm}

\vfill

\footnotesize
\textsc{Quelle}: \titel. Herausgegeben von {\editorInnen}. In: \emph{Arthur Schnitzler: Briefwechsel mit Autorinnen und Autoren}.
 Digitale Edition, https://schnitzler-briefe.acdh.oeaw.ac.at/{\dateiname}.html (Stand \today)
\fi

\end{document}


