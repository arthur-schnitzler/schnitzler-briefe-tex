%% latex-leseansicht-vorspann.tex
%% Vorspann für die Leseansicht.
%% Lädt die gemeinsame Datei latex-vorspann.tex mit nicht gesetztem Schalter.

\newif\ifkorrekturansicht
\korrekturansichtfalse

\input{../tex-inputs/latex-vorspann}


               \section[Richard Beer-Hofmann an Arthur Schnitzler, 18. 1. 1910]{ Richard Beer-Hofmann an Arthur Schnitzler, 18. 1. 1910}\nopagebreak\mylabel{v}\rehead{ }\begin{ledgroupsized}[t]{13cm}\normalsize\beginnumbering\briefempfaengerindex{Schnitzler, Arthur@\textsc{Schnitzler, Arthur}!zzzBeer-Hofmann, Richard@\emph{von Richard Beer-Hofmann}!1910-01-181@{18. 1. 1910}|(be} \toendnotes[C]{\smallbreak\pagebreak[2]} \Standort{CUL, Schnitzler, B 8.}
\physDesc{Kartenbrief, 1 Blatt, 4 Seiten
\newline{}Handschrift: Bleistift, lateinische Kurrent\newline{}Versand: ohne postalischen Übermittlungsvermerk 
\newline{}Schnitzler: mit Bleistift beschriftet: »\textsc{BH}« \newline{}Ordnung: mit Bleistift von unbekannter Hand nummeriert:
                              »227« }\toendnotes[C]{\smallbreak}\pstart{}{\pb}Herrn\pend{}\pstart{}Arthur Schnitzler\pend{}\pstart{}Spöttelgasse\oindex{Edmund-Weiss-Gasse@\textbf{Edmund-Weiß-Gasse}|pw} 7\pend{}{\bigskip}\pstart
           \raggedleft{}{\pb}18/I 10\pend
           \pstart{}Lieber Arthur!\pend\pstart
           Bitte, veranlassen Sie, dass das bewusste \label{KLL01911_AS-v}\edtext{Fräulein\pwindex{Reiter, Anna @\textsc{Reiter, Anna}, \emph{Hausschneiderin}|pwv}}{\lemma{\textnormal{\emph{Fräulein}}}\Cendnote{\textnormal{vgl. A. S.: \emph{Tagebuch}, 19. 1. 1910}}}\label{KLL01911_AS-h}{ }\uline{nicht} zwischen halbdrei – halbvier,
               sondern erst wenn Sie von Ihnen weggeht – also zwischen 6 und
                  7 zu uns kommt{[}.{]}{ }{\pb}Sie collidirt sonst mit den
               Fräuleins die wir von Schallingers\pwindex{Schallinger 1910-01-18 – 1910-01-18@\textsc{Schallinger} (1910-01-18 – 1910-01-18)|pw}\pwindex{Schallinger 1910-01-18 – 1910-01-18@\textsc{Schallinger} (1910-01-18 – 1910-01-18)|pw}
               erwarten.\pend
           \pstart
           Herzlichst mit allen guten Wünschen für \label{KLL01911_AS-1v}\edtext{Dresden\oindex{Dresden@\textbf{Dresden}|pw}}{\lemma{\textnormal{\emph{Dresden}}}\Cendnote{\textnormal{Schnitzler\pwindex{Schnitzler, Arthur 15.05.1862 – 21.10.1931@\textsc{Schnitzler, Arthur} (15.05.1862 – 21.10.1931), \emph{Schriftsteller, Mediziner}|pwk} reiste am 20. 1. 1910 zur Premiere
                  von \emph{Der Schleier der Pierrette}\pwindex{Schnitzler, Arthur 15.05.1862 – 21.10.1931@\textsc{Schnitzler, Arthur} (15.05.1862 – 21.10.1931), \emph{Schriftsteller, Mediziner}!Schleier der Pierrette1910-01-22@\strich\emph{Der Schleier der Pierrette} {[}1910-01-22{]}|pwk}.}}}\label{KLL01911_AS-1h}{\\[\baselineskip]}Ihr{\\[\baselineskip]}\spacefill\mbox{Richard}\pend
           \leftskip=0em{}          \endnumbering\briefempfaengerindex{Schnitzler, Arthur@\textsc{Schnitzler, Arthur}!zzzBeer-Hofmann, Richard@\emph{von Richard Beer-Hofmann}!1910-01-181@{18. 1. 1910}|)be}\mylabel{h}\end{ledgroupsized}  \newcommand{\dateiname}{L01911}\newcommand{\titel}{Richard Beer-Hofmann an Arthur Schnitzler, 18. 1. 1910}\newcommand{\editorInnen}{Martin Anton Müller und Gerd-Hermann Susen}
            \footnotesize
\begin{ledgroupsized}[t]{11.5cm}
\doendnotes{C}
\end{ledgroupsized}
         %% latex-leseansicht-abspann.tex
%% Abspann für die Leseansicht.
%% Der Schalter \ifkorrekturansicht ist bereits durch den Vorspann gesetzt.

%% latex-abspann.tex
%% Gemeinsamer Abspann für Korrekturansicht und Leseansicht.
%% Setzt den Schalter \ifkorrekturansicht voraus (gesetzt in den
%% einbindenden Dateien latex-korrekturansicht-abspann.tex bzw.
%% latex-leseansicht-abspann.tex).
%% ---------------------------------------------------------------

\normalsize

% Das esempio-Environment wird nur in der Leseansicht benötigt
\ifkorrekturansicht\else
\newenvironment{esempio}[3]%
{
    \vspace{1.5ex}
    \rlap{\underline{#1}}
    \par
    \setlength{\parindent}{0cm}
    \nopagebreak
    \leftskip=#2cm
    \rightskip=#3cm
}
{
    \par
}
\fi

\doendnotes{C}
\bigskip
\vfill

\clearpage

\footnotesize

\ifkorrekturansicht
  \lohead{\textsc{register}}
\fi

% theindex-Environment neu definieren ohne reledmac
\makeatletter
\renewenvironment{theindex}{%
  \ifkorrekturansicht
    \section*{\indexname}%
  \else
    \subsubsection*{Index der erwähnten Entitäten}%
  \fi
  \setlength{\parindent}{0pt}%
  \setlength{\parskip}{0pt plus 0.3pt}%
  \let\item\@idxitem
}{%
  \ifkorrekturansicht\clearpage\fi
}
\makeatother

\IfFileExists{\jobname-pw.ind}{\input{\jobname-pw.ind}}{}

% Quellenangabe nur in der Leseansicht
\ifkorrekturansicht\else
% Fallback-Definitionen, falls die .tex-Datei \titel etc. nicht gesetzt hat
\providecommand{\titel}{}
\providecommand{\editorInnen}{}
\providecommand{\dateiname}{\jobname}

\vspace{3cm}

\vfill

\footnotesize
\textsc{Quelle}: \titel. Herausgegeben von {\editorInnen}. In: \emph{Arthur Schnitzler: Briefwechsel mit Autorinnen und Autoren}.
 Digitale Edition, https://schnitzler-briefe.acdh.oeaw.ac.at/{\dateiname}.html (Stand \today)
\fi

\end{document}


      