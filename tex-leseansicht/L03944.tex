%% latex-leseansicht-vorspann.tex
%% Vorspann für die Leseansicht.
%% Lädt die gemeinsame Datei latex-vorspann.tex mit nicht gesetztem Schalter.

\newif\ifkorrekturansicht
\korrekturansichtfalse

\input{../tex-inputs/latex-vorspann}


\section[Arthur Schnitzler an Theodor Herzl, 13. 6. 1893]{L03944 Arthur Schnitzler an Theodor Herzl, 13. 6. 1893}
\nopagebreak\mylabel{L03944v}
\rehead{ }\normalsize\beginnumbering\briefempfaengerindex{Herzl, Theodor@\textsc{Herzl, Theodor}!zzzSchnitzler, Arthur@\emph{von Arthur Schnitzler}!1893-06-131@{13. 6. 1893}|(be}
\toendnotes[C]{\smallbreak\pagebreak[2]}
\correspDesc{Versand  durch Arthur Schnitzler am 13. 6. 1893 in Wien
\newline{}Erhalt  durch Theodor Herzl im Zeitraum [14. 6. 1893 –
                  15. 6. 1893] in Paris}\toendnotes[C]{\smallbreak}
\Standort{Jerusalem, Central Zionist Archives, H1:1924-6.}
\physDesc{Brief, 2 Blätter, 7 Seiten, 2309 Zeichen
\newline{}Handschrift: schwarze Tinte, deutsche Kurrent
\newline{}Ordnung: mit Bleistift von unbekannter Hand innerhalb das Konvoluts paginiert:
                                    »21«–»23« }
\buchAbdrucke{\weitereDrucke{Arthur Schnitzler: \emph{Briefe 1875–1912}. Herausgegeben von Therese Nickl und Heinrich Schnitzler. Frankfurt am Main: \emph{S. Fischer} 1981, S. 206–207.} }\toendnotes[C]{\smallbreak}
\pstart{}{\pb}Lieber Freund,\pend\vspace{0.5em}
\pstart
           Sie dieſen So{\geminationm}er in Wien\oindex{Wien@\textbf{Wien}, \emph{Verwaltungsgebiet}|pw} zu{ }ſehen, wäre mir eine große Freude. Ich{ }ſelbſt dürfte mich kaum von
               hier entfernen; – abgeſehen von der Zeit \label{K_L03944-1v}\edtext{Mitte Auguſt bis September, wo ich mich in den Dienſt des
                  Vaterlandes}{\lemma{\textnormal{\emph{Mitte … Vaterlandes}}}\Cendnote{\textnormal{Dazu kam es nicht.}}}\label{K_L03944-1}{ }ſtellen, dh. einrücken muſs. Wahrſcheinlich \textsc{Bruck}\oindex{Bruck an der Leitha@\textbf{Bruck an der Leitha}, \emph{Verwaltungsgebiet}|pw}. – Sonſt ka{\geminationn}
               ich kaum von hier weg; ohne gerade viel zu thun zu haben, bin ich gebunden.
               Jedenfalls {\pb}haben Sie die Güte, mir näheres über Ihre
               Ankunft mitzutheilen, wie über Ihre Pläne überhaupt. – Ich weiſs auch nicht, ob
               das \label{K_L03944-2v}\edtext{Ereignis\pwindex{Neumann, Margarethe 20.\,5.\,1893 Paris – 15.\,3.\,1943 Konzentrationslager Theresienstadt@\textsc{Neumann, Margarethe} (20.\,5.\,1893 Paris – 15.\,3.\,1943 Konzentrationslager Theresienstadt)|pwv}}{\lemma{\textnormal{\emph{Ereignis}}}\Cendnote{\textnormal{Die Geburt von Margarethe Herzl\pwindex{Neumann, Margarethe 20.\,5.\,1893 Paris – 15.\,3.\,1943 Konzentrationslager Theresienstadt@\textsc{Neumann, Margarethe} (20.\,5.\,1893 Paris – 15.\,3.\,1943 Konzentrationslager Theresienstadt)|pwk}, die am 20. 5. 1893
                  stattfand.}}}\label{K_L03944-2}, welches Sie in Ihrem \label{K_L03944-3v}\edtext{letzten Briefe}{\lemma{\textnormal{\emph{letzten Briefe}}}\Cendnote{\textnormal{XXXX Auszeichnungsfehler: Dokument L03829 nicht gefunden.}}}\label{K_L03944-3} aviſiren, bereits
               eingetroffen iſt. Jedenfalls – viel Glück dazu! –\pend
           
\pstart
           Was mich anbelangt,{ }ſo hab ich mich »\label{K_L03944-4v}\edtext{derfangen}{\lemma{\textnormal{\emph{derfangen}}}\Cendnote{\textnormal{dialektal:
                  zusammengefangen; im Sinne von: einen (Schicksals-)Schlag überlebt und wieder zum
                  Alltag zurückgekehrt. Schnitzlers{ }Vater\pwindex{Schnitzler, Johann 10.\,4.\,1835 Nagykanizsa – 2.\,5.\,1893 Wien@\textsc{Schnitzler, Johann} (10.\,4.\,1835 Nagykanizsa – 2.\,5.\,1893 Wien), \emph{Laryngologe}|pwkv} war am 2. 5. 1894 verstorben.}}}\label{K_L03944-4}«,{ }ſo gut es ging, u. verſuche da u
               dort wieder ins Arbeiten zu ko{\geminationm}en. Klinik, Praxis u d\substVorne{}\textsuperscript{ie}\substDazwischen{}as\substHinten{}{ }\label{K_L03944-5v}\edtext{mediz. Journal\orgindex{Internationale klinische Rundschau@Internationale klinische Rundschau|pwv} das ich {\pb}leite}{\lemma{\textnormal{\emph{mediz. … leite}}}\Cendnote{\textnormal{Schnitzler betreute zwischen Ende 1887 und bis zum September 1894 die von seinem Vater
                  begründete \emph{Internationale klinische
                  Rundschau}\orgindex{Internationale klinische Rundschau@Internationale klinische Rundschau|pwk}.}}}\label{K_L03944-5}, nehmen mir viel Zeit weg, laſſen mir aber innerlich eine
               gewiſſe Freiheit. Die Praxis ni{\geminationm}t nemlich auch Zeit weg,
               ohne daſs man Patienten hat, das ist das arge, und we{\geminationn}{ }ſtatt 1 oder 2 Leuten 16–20 in die Ordination\oindex{Wien@\textbf{Wien}!I., Innere Stadt@\textbf{I., Innere Stadt}!Wohnung und Ordination Arthur Schnitzler Grillparzerstraße 7/3. Stock@\textbf{Wohnung und Ordination Arthur Schnitzler Grillparzerstraße 7/3. Stock}, \emph{Ordination}|pwv} kämen,{ }ſo gäbe das kaum mehr zu thun, und hätte
               doch{ }ſeine Vortheile. – Ich \label{K_L03944-6v}\edtext{ſchreibe\pwindex{Schnitzler, Arthur 15.\,5.\,1862 Wien – 21.\,10.\,1931 ebd.@\textsc{Schnitzler, Arthur} (15.\,5.\,1862 Wien – 21.\,10.\,1931 ebd.), \emph{Schriftsteller, Mediziner}!kleine Komödie@\strich\emph{Die kleine Komödie}|pwv} jetzt meiſtens{ }ſpät
               Abends,{ }ſo um die Mitternacht herum, im Kaffehaus\oindex{Wien@\textbf{Wien}!I., Innere Stadt@\textbf{I., Innere Stadt}!Café Reichsrath (Inh. Karl Auböck)@\textbf{Café Reichsrath (Inh. Karl Auböck)}, \emph{Kaffeehaus}|pwv}}{\lemma{\textnormal{\emph{schreibe … Kaffehaus}}}\Cendnote{\textnormal{Vgl. A. S.: \emph{Tagebuch}, 29. 5. 1893.}}}\label{K_L03944-6}. Dort,
               beim Rathaus\oindex{Wien@\textbf{Wien}!I., Innere Stadt@\textbf{I., Innere Stadt}!Rathaus [Wien]@\textbf{Rathaus [Wien]}, \emph{Verwaltungsgebäude}|pw}, {\pb}dem Park\oindex{Wien@\textbf{Wien}!I., Innere Stadt@\textbf{I., Innere Stadt}!Rathauspark@\textbf{Rathauspark}, \emph{Park}|pw} vis à vis. – Es{ }ſoll etwas zärtliches
               und \substVorne{}\textsuperscript{komiſches}\substDazwischen{}luſtiges\substHinten{} werden, – der geheime Trieb iſt aber offenbar der: – ich will wieder{ }ſchreiben lernen. – Zum »Flüchtling\pwindex{Herzl, Theodor 2.\,5.\,1860 Budapest – 3.\,7.\,1904 Edlach@\textsc{Herzl, Theodor} (2.\,5.\,1860 Budapest – 3.\,7.\,1904 Edlach), \emph{Schriftsteller, Journalist}!Flüchtling. Lustspiel in einem Aufzug@\strich\emph{Der Flüchtling. Lustspiel in einem Aufzug}|pwv}« in Berlin\oindex{Berlin@\textbf{Berlin}, \emph{Hauptstadt}|pw} muſs ich Ihnen
               noch gratuliren. Ueberhaupt wächſt meine Hochachtung für Menſchen, die aufgeführt
               werden, i{\geminationm}er mehr,{ }ſeit ich{ }ſehe, wie weit der Weg vom
                  Angeno{\geminationm}enwerden zum Aufgeführtwerden iſt. – {\pb}In Prag\oindex{Prag@\textbf{Prag}, \emph{Land}|pw} bin ich über die
               Moral des Intendanten Dr. Schleſinger\pwindex{Schlesinger, Ludwig 13.\,10.\,1838 Litvínov – 24.\,12.\,1899 Prager Neustadt@\textsc{Schlesinger, Ludwig} (13.\,10.\,1838 Litvínov – 24.\,12.\,1899 Prager Neustadt), \emph{Politiker, Theaterleiter, Historiker}|pw}
               geſtolpert, der \label{T_L03944-1v}\edtext{über das Märchen\pwindex{Schnitzler, Arthur 15.\,5.\,1862 Wien – 21.\,10.\,1931 ebd.@\textsc{Schnitzler, Arthur} (15.\,5.\,1862 Wien – 21.\,10.\,1931 ebd.), \emph{Schriftsteller, Mediziner}!Märchen. Schauspiel in drei Aufzügen@\strich\emph{Das Märchen. Schauspiel in drei Aufzügen}|pw}}{\lemma{\textnormal{\emph{über das Märchen}}}\Cendnote{\textnormal{Durch Umstellungszeichen vor »›empört‹ war«
                  geschoben.}}}\label{T_L03944-1} »empört« war, – und von Berlin\oindex{Berlin@\textbf{Berlin}, \emph{Hauptstadt}|pw} aus werde ich nachdrücklich verachtet; man beantworte\strikeout{ſ}t weder meine
               höflichen noch meine – andern Briefe. »\label{K_L03944-7v}\edtext{Man{ }ſpuckt aus u. geht weiter}{\lemma{\textnormal{\emph{Man … weiter}}}\Cendnote{\textnormal{XXXX Auszeichnungsfehler: Dokument L03823 nicht gefunden.}}}\label{K_L03944-7}«{ }ſchrieben Sie mir einmal.
               Sie haben ja \uline{ſo} Recht! – Aber merkwürdigerweiſe hilft
               auch das Spucken und Weitergehn nichts. Die Direktoren u. ähnliches denkt {\pb}ſich eben:
               – »Man wiſcht{ }ſich ab und{ }ſchurkt weiter.« – Und{ }ſie{ }ſind die Klügeren –{ }ſie koſtet’s
               nur das Schnupftuch, aber uns die Lungen. –\pend
           
\pstart
           Na genug für heute, mein lieber Herr Doktor; ich hoffe recht bald von Ihnen zu hören. –
               Und haben Sie noch keine genauen So{\geminationm}erpläne,{ }ſo berichten Sie mir wenigſtens in
               2 Zeilen, wie’s Ihnen geht.\pend
           \pstart {\pb}Ihr herzlich ergebner \spacefill\mbox{DrArthSchnitzl}\pend{}
\pstart
           13. 6. 93\pend
           \selectlanguage{ngerman}\endnumbering\briefempfaengerindex{Herzl, Theodor@\textsc{Herzl, Theodor}!zzzSchnitzler, Arthur@\emph{von Arthur Schnitzler}!1893-06-131@{13. 6. 1893}|)be}\mylabel{L03944h}
\begin{anhang}
\end{anhang}\newcommand{\dateiname}{L03944}\newcommand{\titel}{Arthur Schnitzler an Theodor Herzl, 13. 6. 1893}\newcommand{\editorInnen}{Herausgegeben von Jahnke, SelmaMüller, Martin Anton}%% latex-leseansicht-abspann.tex
%% Abspann für die Leseansicht.
%% Der Schalter \ifkorrekturansicht ist bereits durch den Vorspann gesetzt.

%% latex-abspann.tex
%% Gemeinsamer Abspann für Korrekturansicht und Leseansicht.
%% Setzt den Schalter \ifkorrekturansicht voraus (gesetzt in den
%% einbindenden Dateien latex-korrekturansicht-abspann.tex bzw.
%% latex-leseansicht-abspann.tex).
%% ---------------------------------------------------------------

\normalsize

% Das esempio-Environment wird nur in der Leseansicht benötigt
\ifkorrekturansicht\else
\newenvironment{esempio}[3]%
{
    \vspace{1.5ex}
    \rlap{\underline{#1}}
    \par
    \setlength{\parindent}{0cm}
    \nopagebreak
    \leftskip=#2cm
    \rightskip=#3cm
}
{
    \par
}
\fi

\doendnotes{C}
\bigskip
\vfill

\clearpage

\footnotesize

\ifkorrekturansicht
  \lohead{\textsc{register}}
\fi

% theindex-Environment neu definieren ohne reledmac
\makeatletter
\renewenvironment{theindex}{%
  \ifkorrekturansicht
    \section*{\indexname}%
  \else
    \subsubsection*{Index der erwähnten Entitäten}%
  \fi
  \setlength{\parindent}{0pt}%
  \setlength{\parskip}{0pt plus 0.3pt}%
  \let\item\@idxitem
}{%
  \ifkorrekturansicht\clearpage\fi
}
\makeatother

\IfFileExists{\jobname-pw.ind}{\input{\jobname-pw.ind}}{}

% Quellenangabe nur in der Leseansicht
\ifkorrekturansicht\else
% Fallback-Definitionen, falls die .tex-Datei \titel etc. nicht gesetzt hat
\providecommand{\titel}{}
\providecommand{\editorInnen}{}
\providecommand{\dateiname}{\jobname}

\vspace{3cm}

\vfill

\footnotesize
\textsc{Quelle}: \titel. Herausgegeben von {\editorInnen}. In: \emph{Arthur Schnitzler: Briefwechsel mit Autorinnen und Autoren}.
 Digitale Edition, https://schnitzler-briefe.acdh.oeaw.ac.at/{\dateiname}.html (Stand \today)
\fi

\end{document}


