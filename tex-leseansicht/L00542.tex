%% latex-korrekturansicht-vorspann.tex
%% Vorspann für die Korrekturansicht.
%% Lädt die gemeinsame Datei latex-vorspann.tex mit gesetztem Schalter.

\newif\ifkorrekturansicht
\korrekturansichttrue

\input{../tex-inputs/latex-vorspann}


\section[Arthur Schnitzler an Georg Brandes, 25. 4. 1896]{L00542 Arthur Schnitzler an Georg Brandes, 25. 4. 1896}
\nopagebreak\mylabel{L00542v}
\rehead{ }\normalsize\beginnumbering\briefempfaengerindex{Brandes, Georg@\textsc{Brandes, Georg}!zzzSchnitzler, Arthur@\emph{von Arthur Schnitzler}!1896-04-251@{25. 4. 1896}|(be}
\toendnotes[C]{\smallbreak\pagebreak[2]}\Standort{Kopenhagen, Det Kongelige Bibliotek, Georg Brandes Arkiv, box 125.}
\physDesc{Brief, 1 Blatt, 3 Seiten, 742 Zeichen
\newline{}Handschrift: schwarze Tinte, deutsche Kurrent
\newline{}Ordnung: mit Bleistift von unbekannter Hand auf der ersten Seite
                                    »Schnitzler« vermerkt, datiert: »22/4 96« und nummeriert: »2« }
\buchAbdrucke{\weitereDrucke{Georg Brandes, Arthur Schnitzler: \emph{Ein Briefwechsel}. Bern: \emph{Francke} 1956, S. 57.} }\toendnotes[C]{\smallbreak}
\pstart\center{}{\pb}Sehr verehrter Herr,\pend\vspace{0.5em}
\pstart
           Ihr freundlicher Brief hat mich aufs höchſte erfreut. Ich habe das Buch\pwindex{Liebelei. Schauspiel in drei Akten@\emph{Liebelei. Schauspiel in drei Akten}|pwv} nur einigen perſönlichen Beka{\geminationn}ten gegeben – und ich darf mir wohl geſtatten, Ihrer
               Bemerkung, daß ich »in meinem Erfolg« Ihrer vergeſſen habe, als Scherz aufzufaſſen.
                  {\pb}Oder halten Sie mich für ſo ſtupid, daſs der
               Zufall eines Erfolges mich in meiner Stellung zu Menſchen, die ich bewundere,
               verändern könnte? So nehme ich alſo jene Bemerkung lieber als eine liebenswürdige
               Aufforderung, auf die ich ſtolz bin, und bitte Sie um die Ehre, auch dieſes
               verſpätete Exemplar\pwindex{Liebelei. Schauspiel in drei Akten@\emph{Liebelei. Schauspiel in drei Akten}|pwv} gütigſt
               entgegen zu nehmen.\pend
           \pstart {\pb}In der Hoffnung, Ihnen doch auch einmal
               perſönlich begegnen zu dürfen, bleibe ich mit verbindlichſten Grüßen Ihr dankbar
               ergebner \spacefill\mbox{ArthurSchnitzler}\pend{}
\pstart
           Wien\oindex{Wien@\textbf{Wien}, \emph{A.ADM2}|pw}{ }25. 4. 96.\pend
           \selectlanguage{ngerman}\endnumbering\briefempfaengerindex{Brandes, Georg@\textsc{Brandes, Georg}!zzzSchnitzler, Arthur@\emph{von Arthur Schnitzler}!1896-04-251@{25. 4. 1896}|)be}\mylabel{L00542h}  \normalsize

\doendnotes{C}
\bigskip
\vfill

\clearpage

\footnotesize

\lohead{\textsc{register}}

% Definiere theindex-Environment komplett neu ohne reledmac
\makeatletter
\renewenvironment{theindex}{%
  \section*{\indexname}%
  \setlength{\parindent}{0pt}%
  \setlength{\parskip}{0pt plus 0.3pt}%
  \let\item\@idxitem
}{%
  \clearpage
}
\makeatother

\IfFileExists{\jobname-pw.ind}{\input{\jobname-pw.ind}}{}

\end{document}

      