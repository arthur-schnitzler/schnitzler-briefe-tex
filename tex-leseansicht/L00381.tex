%% latex-korrekturansicht-vorspann.tex
%% Vorspann für die Korrekturansicht.
%% Lädt die gemeinsame Datei latex-vorspann.tex mit gesetztem Schalter.

\newif\ifkorrekturansicht
\korrekturansichttrue

\input{../tex-inputs/latex-vorspann}


\section[Richard Beer-Hofmann an Arthur Schnitzler, 12. 10. 1894]{L00381 Richard Beer-Hofmann an Arthur Schnitzler, 12. 10. 1894}
\nopagebreak\mylabel{L00381v}
\rehead{ }\normalsize\beginnumbering\briefempfaengerindex{Schnitzler, Arthur@\textsc{Schnitzler, Arthur}!zzzBeer-Hofmann, Richard@\emph{von Richard Beer-Hofmann}!1894-10-121@{12. 10. 1894}|(be}
\toendnotes[C]{\smallbreak\pagebreak[2]}\Standort{CUL, Schnitzler, B 8.}
\physDesc{Postkarte, 260 Zeichen
\newline{}Handschrift: Bleistift, lateinische Kurrent
\newline{}Versand: 1) Stempel: »\nobreak{}\oindex{Hotel Quirinale@\textbf{Hotel Quirinale}, \emph{Hotel (K.HTL)}|pwk}Grand Hôtel du Quirinal
                                       ROME, 12 OTT. 94, Tenu par Alessandro Marroni\nobreak{}«.   2) Stempel: »\nobreak{}\oindex{Roma Termini@\textbf{Roma Termini}, \emph{S.RSTN}|pwk}Roma Ferrovia, 12 10.–94, 11M\nobreak{}«.  3) Stempel: »\nobreak{}\oindex{IX., Alsergrund@\textbf{IX., Alsergrund}, \emph{A.ADM3}|pwk}Wien 9/3, 14. 10. 94, 8.V, Bestellt\nobreak{}«. 
\newline{}Schnitzler: 1) mit Bleistift von unbekannter Hand nummeriert:
                                    »45«  2) mit Bleistift von unbekannter Hand Nummerierung
                                 wiederholt}\pstart{}{\pb}\textcolor{gray}{\textbf{A}}n D\textsuperscript{r} Arthur
                  Schnitzler\pend{}\pstart{}Austria\oindex{Oesterreich@\textbf{Österreich}, \emph{A.PCLI}|pw}\pend{}\pstart{}Wien\oindex{Wien@\textbf{Wien}, \emph{A.ADM2}|pw}\pend{}\pstart{}IX Frankgasse 1\oindex{Frankgasse 1@\textbf{Frankgasse 1}, \emph{Wohngebäude (K.WHS)}|pw}\pend{}{\bigskip}\vspace{1em}
\pstart
           \noindent{}{\pb}Lieber Arthur! Soeben
               erhalte Karte. \strikeout{Alle Zeit\pwindex{Zeit. Wiener Wochenschrift@\emph{Die Zeit. Wiener Wochenschrift}|pw}} Brief »Lieber Bek.« erhalten »\uline{Zeit\pwindex{Zeit. Wiener Wochenschrift@\emph{Die Zeit. Wiener Wochenschrift}|pw}}« nicht. Bitte senden Sie Brief u Zeit\pwindex{Zeit. Wiener Wochenschrift@\emph{Die Zeit. Wiener Wochenschrift}|pw} nach
                  Neapel\oindex{Neapel@\textbf{Neapel}, \emph{P.PPLA}|pw} (Napoli\oindex{Neapel@\textbf{Neapel}, \emph{P.PPLA}|pw}) a posta ferma. Schreibe morgen ausführlicher. Bin zu abgehetzt.\pend
           \pstart Herzlichst\spacefill\mbox{Richard}\pend{}\selectlanguage{ngerman}\endnumbering\briefempfaengerindex{Schnitzler, Arthur@\textsc{Schnitzler, Arthur}!zzzBeer-Hofmann, Richard@\emph{von Richard Beer-Hofmann}!1894-10-121@{12. 10. 1894}|)be}\mylabel{L00381h}  \normalsize

\doendnotes{C}
\bigskip
\vfill

\clearpage

\footnotesize

\lohead{\textsc{register}}

% Definiere theindex-Environment komplett neu ohne reledmac
\makeatletter
\renewenvironment{theindex}{%
  \section*{\indexname}%
  \setlength{\parindent}{0pt}%
  \setlength{\parskip}{0pt plus 0.3pt}%
  \let\item\@idxitem
}{%
  \clearpage
}
\makeatother

\IfFileExists{\jobname-pw.ind}{\input{\jobname-pw.ind}}{}

\end{document}

      