%% latex-leseansicht-vorspann.tex
%% Vorspann für die Leseansicht.
%% Lädt die gemeinsame Datei latex-vorspann.tex mit nicht gesetztem Schalter.

\newif\ifkorrekturansicht
\korrekturansichtfalse

\input{../tex-inputs/latex-vorspann}


\section[Arthur Schnitzler an Richard Beer-Hofmann, 15. 9. 1899]{L00974 Arthur Schnitzler an Richard Beer-Hofmann, 15. 9. 1899}
\nopagebreak\mylabel{L00974v}
\rehead{ }\normalsize\beginnumbering\briefempfaengerindex{Beer-Hofmann, Richard@\textsc{Beer-Hofmann, Richard}!zzzSchnitzler, Arthur@\emph{von Arthur Schnitzler}!1899-09-151@{15. 9. 1899}|(be}
\toendnotes[C]{\smallbreak\pagebreak[2]}
\correspDesc{Versand  durch Arthur Schnitzler am 15. 9. 1899 in München
\newline{}Erhalt  durch Richard Beer-Hofmann am 17. 9. 1899 in Vahrn}\toendnotes[C]{\smallbreak}
\Standort{YCGL, MSS 31.}
\physDesc{Bildpostkarte, 324 Zeichen
\newline{}Handschrift: Bleistift, deutsche Kurrent
\newline{}Versand: 1) Stempel: »\nobreak{}\oindex{München@\textbf{München}|pwk}München, 15 Sep 99, 6–7Nm\nobreak{}«.   2) Stempel: »\nobreak{}17. 9. 99\nobreak{}«. 
\newline{}Zusatz: Postkartenmotiv von Otto
                                    Strützel\pwindex{Strützel, Otto 2.\,9.\,1855 Dessau – 25.\,12.\,1930 München@\textsc{Strützel, Otto} (2.\,9.\,1855 Dessau – 25.\,12.\,1930 München), \emph{Maler, Illustrator}|pw} }\toendnotes[C]{\smallbreak}\pstart{}{\pb}Hrn \textsc{Dr. Richard
                     Beer-Hofmann}\pend{}\pstart{}\textsc{Vahrn}\oindex{Vahrn@\textbf{Vahrn}, \emph{Hauptstadt}|pw} bei \textsc{Brixen\oindex{Brixen@\textbf{Brixen}, \emph{Hauptstadt}|pw}}\pend{}\pstart{}\textsc{Tirol}\oindex{Südtirol@\textbf{Südtirol}, \emph{Verwaltungsgebiet}|pw}\pend{}{\bigskip}
\pstart
           \noindent{}\centering{}{\pb}\textcolor{gray}{\textbf{München\oindex{München@\textbf{München}|pw}, Frauenkirche\oindex{Frauenkirche@\textbf{Frauenkirche}, \emph{Kirche}|pw}}}\pend
           \vspace{1em}
\pstart
           \noindent{}{\pb}lieber Richard, ich fahre von hier (nicht ganz direct)
               wahrſcheinlich Frankfurt\oindex{Frankfurt am Main@\textbf{Frankfurt am Main}, \emph{Hauptstadt}|pw} zu Goldmann\pwindex{Goldmann, Paul 31.\,1.\,1865 Breslau – 25.\,9.\,1935 Wien@\textsc{Goldmann, Paul} (31.\,1.\,1865 Breslau – 25.\,9.\,1935 Wien), \emph{Schriftsteller, Journalist}|pw}; nehme an, eventuell Mittwoch dort zu{ }ſein. Iſt Hugo\pwindex{Hofmannsthal, Hugo von 1.\,2.\,1874 Wien – 15.\,7.\,1929 Rodaun@\textsc{Hofmannsthal, Hugo von} (1.\,2.\,1874 Wien – 15.\,7.\,1929 Rodaun), \emph{Schriftsteller}|pw} bei Ihnen? Von Fr.\oindex{Frankfurt am Main@\textbf{Frankfurt am Main}, \emph{Hauptstadt}|pw} fahr ich nach – pardon – will ich nach Berlin\oindex{Berlin@\textbf{Berlin}, \emph{Hauptstadt}|pw} fahren. Bitte Nachricht Frkf a M\oindex{Frankfurt am Main@\textbf{Frankfurt am Main}, \emph{Hauptstadt}|pw}{ }\textsc{post rest}.\pend
           
\pstart
           \label{T_L00974-1v}\edtext{Herzlich}{\lemma{\textnormal{\emph{Herzlich}}}\Cendnote{\textnormal{Grußformel über dem Text am rechten Rand.}}}\label{T_L00974-1}{\\[\baselineskip]}Ihr\spacefill\mbox{A.}\pend
           \leftskip=0em{}
\pstart
           \noindent{}\label{T_L00974-2v}\edtext{Danke für Ihren l. Brief.}{\lemma{\textnormal{\emph{Danke … Brief.}}}\Cendnote{\textnormal{am linken Rand neben dem Bild}}}\label{T_L00974-2}\pend
           \selectlanguage{ngerman}\endnumbering\briefempfaengerindex{Beer-Hofmann, Richard@\textsc{Beer-Hofmann, Richard}!zzzSchnitzler, Arthur@\emph{von Arthur Schnitzler}!1899-09-151@{15. 9. 1899}|)be}\mylabel{L00974h}  \newcommand{\dateiname}{L00974}\newcommand{\titel}{Arthur Schnitzler an Richard Beer-Hofmann, 15. 9. 1899}\newcommand{\editorInnen}{Martin Anton Müller und Gerd-Hermann Susen}%% latex-leseansicht-abspann.tex
%% Abspann für die Leseansicht.
%% Der Schalter \ifkorrekturansicht ist bereits durch den Vorspann gesetzt.

%% latex-abspann.tex
%% Gemeinsamer Abspann für Korrekturansicht und Leseansicht.
%% Setzt den Schalter \ifkorrekturansicht voraus (gesetzt in den
%% einbindenden Dateien latex-korrekturansicht-abspann.tex bzw.
%% latex-leseansicht-abspann.tex).
%% ---------------------------------------------------------------

\normalsize

% Das esempio-Environment wird nur in der Leseansicht benötigt
\ifkorrekturansicht\else
\newenvironment{esempio}[3]%
{
    \vspace{1.5ex}
    \rlap{\underline{#1}}
    \par
    \setlength{\parindent}{0cm}
    \nopagebreak
    \leftskip=#2cm
    \rightskip=#3cm
}
{
    \par
}
\fi

\doendnotes{C}
\bigskip
\vfill

\clearpage

\footnotesize

\ifkorrekturansicht
  \lohead{\textsc{register}}
\fi

% theindex-Environment neu definieren ohne reledmac
\makeatletter
\renewenvironment{theindex}{%
  \ifkorrekturansicht
    \section*{\indexname}%
  \else
    \subsubsection*{Index der erwähnten Entitäten}%
  \fi
  \setlength{\parindent}{0pt}%
  \setlength{\parskip}{0pt plus 0.3pt}%
  \let\item\@idxitem
}{%
  \ifkorrekturansicht\clearpage\fi
}
\makeatother

\IfFileExists{\jobname-pw.ind}{\input{\jobname-pw.ind}}{}

% Quellenangabe nur in der Leseansicht
\ifkorrekturansicht\else
% Fallback-Definitionen, falls die .tex-Datei \titel etc. nicht gesetzt hat
\providecommand{\titel}{}
\providecommand{\editorInnen}{}
\providecommand{\dateiname}{\jobname}

\vspace{3cm}

\vfill

\footnotesize
\textsc{Quelle}: \titel. Herausgegeben von {\editorInnen}. In: \emph{Arthur Schnitzler: Briefwechsel mit Autorinnen und Autoren}.
 Digitale Edition, https://schnitzler-briefe.acdh.oeaw.ac.at/{\dateiname}.html (Stand \today)
\fi

\end{document}


