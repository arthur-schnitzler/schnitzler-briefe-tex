\input{../tex-inputs/latex-pdf-vorspann}
\begin{center}
            \textcolor{red}{ENTWURF. ENTZIFFERUNG NOCH NICHT KORREKTURGELESEN}
                      \end{center}
            
               \section[Arthur Schnitzler an Richard Beer-Hofmann, 15. 9. 1899]{ Arthur Schnitzler an Richard Beer-Hofmann, 15. 9. 1899}\nopagebreak\mylabel{v}\rehead{ }\begin{ledgroupsized}[t]{13cm}\normalsize\beginnumbering\briefempfaengerindex{Beer-Hofmann, Richard@\textsc{Beer-Hofmann, Richard}!zzzSchnitzler, Arthur@\emph{von Arthur Schnitzler}!1899-09-151@{15. 9. 1899}|(be} \toendnotes[C]{\smallbreak\pagebreak[2]} \Standort{YCGL, MSS 31.}
\physDesc{Bildpostkarte
\newline{}Handschrift: Bleistift, deutsche Kurrent\newline{}Versand: 1) Stempel: »\nobreak{}\oindex{Muenchen@\textbf{München}|pwk}München, 15 Sep 99, 6–7Nm\nobreak{}«.  2) Stempel: »\nobreak{}17. 9. 99\nobreak{}«. \newline{}Zusatz: Postkartenmotiv von Otto
                                    Strützel\pwindex{Struetzel, Otto 1855-09-02 – 1930-12-25@\textsc{Strützel, Otto} (1855-09-02 – 1930-12-25), \emph{Maler, Illustrator}|pw} }\toendnotes[C]{\smallbreak}\pstart{}{\pb}Hrn \textsc{Dr. Richard
                     Beer-Hofmann}\pend{}\pstart{}\textsc{Vahrn}\oindex{Vahrn@\textbf{Vahrn}|pw} bei \textsc{Brixen\oindex{Brixen@\textbf{Brixen}|pw}}\pend{}\pstart{}\textsc{Tirol}\oindex{Suedtirol@\textbf{Südtirol}|pw}\pend{}{\bigskip}\pstart
           \noindent{}\centering{}\textcolor{gray}{\textbf{{\pb}München\oindex{Muenchen@\textbf{München}|pw}, Frauenkirche\oindex{Frauenkirche@\textbf{Frauenkirche}|pw}}}\pend
           \pstart
           lieber Richard, ich fahre von hier (nicht ganz direct)
               wahrſcheinlich Frankfurt\oindex{Frankfurt am Main@\textbf{Frankfurt am Main}|pw} zu Goldmann\pwindex{Goldmann, Paul 31.01.1865 – 25.09.1935@\textsc{Goldmann, Paul} (31.01.1865 – 25.09.1935), \emph{Schriftsteller, Journalist}|pw}; nehme an, eventuell Mittwoch dort zu
               ſein. Iſt Hugo\pwindex{Hofmannsthal, Hugo von 01.02.1874 – 15.07.1929@\textsc{Hofmannsthal, Hugo von} (01.02.1874 – 15.07.1929), \emph{Schriftsteller}|pw} bei Ihnen? Von Fr.\oindex{Frankfurt am Main@\textbf{Frankfurt am Main}|pw} fahr ich nach – pardon – will ich nach Berlin\oindex{Berlin@\textbf{Berlin}|pw} fahren. Bitte Nachricht Frkf
                  a M\oindex{Frankfurt am Main@\textbf{Frankfurt am Main}|pw}{ }\textsc{post rest}.\pend
           \pstart
           \label{T_L00974_1v}\edtext{Herzlich}{\lemma{\textnormal{\emph{Herzlich}}}\Cendnote{\textnormal{Grußformel über dem Text am rechten Rand.}}}\label{T_L00974_1h}{\\[\baselineskip]}Ihr\spacefill\mbox{A.}\pend
           \leftskip=0em{}\pstart
           \noindent{}\label{T_L00974_2v}\edtext{Danke für Ihren l. Brief.}{\lemma{\textnormal{\emph{Danke … Brief.}}}\Cendnote{\textnormal{am linken Rand neben dem Bild}}}\label{T_L00974_2h}\pend
           \endnumbering\briefempfaengerindex{Beer-Hofmann, Richard@\textsc{Beer-Hofmann, Richard}!zzzSchnitzler, Arthur@\emph{von Arthur Schnitzler}!1899-09-151@{15. 9. 1899}|)be}\mylabel{h}\end{ledgroupsized}  \newcommand{\dateiname}{L00974}\newcommand{\titel}{Arthur Schnitzler an Richard Beer-Hofmann, 15. 9. 1899}\newcommand{\editorInnen}{Martin Anton Müller und Gerd-Hermann Susen}\input{../tex-inputs/latex-pdf-abspann}
      