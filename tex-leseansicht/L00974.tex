%% latex-korrekturansicht-vorspann.tex
%% Vorspann für die Korrekturansicht.
%% Lädt die gemeinsame Datei latex-vorspann.tex mit gesetztem Schalter.

\newif\ifkorrekturansicht
\korrekturansichttrue

\input{../tex-inputs/latex-vorspann}


\section[Arthur Schnitzler an Richard Beer-Hofmann, 15. 9. 1899]{L00974 Arthur Schnitzler an Richard Beer-Hofmann, 15. 9. 1899}
\nopagebreak\mylabel{L00974v}
\rehead{ }\normalsize\beginnumbering\briefempfaengerindex{Beer-Hofmann, Richard@\textsc{Beer-Hofmann, Richard}!zzzSchnitzler, Arthur@\emph{von Arthur Schnitzler}!1899-09-151@{15. 9. 1899}|(be}
\toendnotes[C]{\smallbreak\pagebreak[2]}\Standort{YCGL, MSS 31.}
\physDesc{Bildpostkarte, 324 Zeichen
\newline{}Handschrift: Bleistift, deutsche Kurrent
\newline{}Versand: 1) Stempel: »\nobreak{}\oindex{Muenchen@\textbf{München}, \emph{P.PPLA}|pwk}München, 15 Sep 99, 6–7Nm\nobreak{}«.   2) Stempel: »\nobreak{}17. 9. 99\nobreak{}«. 
\newline{}Zusatz: Postkartenmotiv von Otto
                                    Strützel\pwindex{Struetzel, Otto 1855-09-02 – 1930-12-25@\textsc{Strützel, Otto} (1855-09-02 – 1930-12-25), \emph{Maler/Malerin, Illustrator/Illustratorin}|pw} }\toendnotes[C]{\smallbreak}\pstart{}{\pb}Hrn \textsc{Dr. Richard
                     Beer-Hofmann}\pend{}\pstart{}\textsc{Vahrn}\oindex{Vahrn@\textbf{Vahrn}, \emph{P.PPLA3}|pw} bei \textsc{Brixen\oindex{Brixen@\textbf{Brixen}, \emph{P.PPLA3}|pw}}\pend{}\pstart{}\textsc{Tirol}\oindex{Suedtirol@\textbf{Südtirol}, \emph{A.ADM2}|pw}\pend{}{\bigskip}
\pstart
           \noindent{}\centering{}{\pb}\textcolor{gray}{\textbf{München\oindex{Muenchen@\textbf{München}, \emph{P.PPLA}|pw}, Frauenkirche\oindex{Frauenkirche@\textbf{Frauenkirche}, \emph{Kirche (K.KRC)}|pw}}}\pend
           \vspace{1em}
\pstart
           \noindent{}{\pb}lieber Richard, ich fahre von hier (nicht ganz direct)
               wahrſcheinlich Frankfurt\oindex{Frankfurt am Main@\textbf{Frankfurt am Main}, \emph{P.PPLA3}|pw} zu Goldmann\pwindex{Goldmann, Paul 31.01.1865 – 25.09.1935@\textsc{Goldmann, Paul} (31.01.1865 – 25.09.1935), \emph{Schriftsteller/Schriftstellerin, Journalist/Journalistin}|pw}; nehme an, eventuell Mittwoch dort zu
               ſein. Iſt Hugo\pwindex{Hofmannsthal, Hugo von 1874-02-01 – 1929-07-15@\textsc{Hofmannsthal, Hugo von} (1874-02-01 – 1929-07-15), \emph{Schriftsteller/Schriftstellerin}|pw} bei Ihnen? Von Fr.\oindex{Frankfurt am Main@\textbf{Frankfurt am Main}, \emph{P.PPLA3}|pw} fahr ich nach – pardon – will ich nach Berlin\oindex{Berlin@\textbf{Berlin}, \emph{P.PPLC}|pw} fahren. Bitte Nachricht Frkf a M\oindex{Frankfurt am Main@\textbf{Frankfurt am Main}, \emph{P.PPLA3}|pw}{ }\textsc{post rest}.\pend
           
\pstart
           \label{T_L00974-1v}\edtext{Herzlich}{\lemma{\textnormal{\emph{Herzlich}}}\Cendnote{\textnormal{Grußformel über dem Text am rechten Rand.}}}\label{T_L00974-1}{\\[\baselineskip]}Ihr\spacefill\mbox{A.}\pend
           \leftskip=0em{}
\pstart
           \noindent{}\label{T_L00974-2v}\edtext{Danke für Ihren l. Brief.}{\lemma{\textnormal{\emph{Danke … Brief.}}}\Cendnote{\textnormal{am linken Rand neben dem Bild}}}\label{T_L00974-2}\pend
           \selectlanguage{ngerman}\endnumbering\briefempfaengerindex{Beer-Hofmann, Richard@\textsc{Beer-Hofmann, Richard}!zzzSchnitzler, Arthur@\emph{von Arthur Schnitzler}!1899-09-151@{15. 9. 1899}|)be}\mylabel{L00974h}  \normalsize

\doendnotes{C}
\bigskip
\vfill

\clearpage

\footnotesize

\lohead{\textsc{register}}

% Definiere theindex-Environment komplett neu ohne reledmac
\makeatletter
\renewenvironment{theindex}{%
  \section*{\indexname}%
  \setlength{\parindent}{0pt}%
  \setlength{\parskip}{0pt plus 0.3pt}%
  \let\item\@idxitem
}{%
  \clearpage
}
\makeatother

\IfFileExists{\jobname-pw.ind}{\input{\jobname-pw.ind}}{}

\end{document}

      