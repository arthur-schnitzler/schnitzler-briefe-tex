%% latex-korrekturansicht-vorspann.tex
%% Vorspann für die Korrekturansicht.
%% Lädt die gemeinsame Datei latex-vorspann.tex mit gesetztem Schalter.

\newif\ifkorrekturansicht
\korrekturansichttrue

\input{../tex-inputs/latex-vorspann}


\section[ Felix Salten an Arthur Schnitzler, 28. 6. 1910]{L03549 Felix Salten an Arthur Schnitzler, 28. 6. 1910}
\nopagebreak\mylabel{L03549v}
\rehead{ }\normalsize\beginnumbering\briefempfaengerindex{Schnitzler, Arthur@\textsc{Schnitzler, Arthur}!zzzSalten, Felix@\emph{von Felix Salten}!1910-06-282@{28. 6. 1910}|(be}
\toendnotes[C]{\smallbreak\pagebreak[2]}\Standort{CUL, Schnitzler, B 89, B 2.}
\physDesc{Postkarte, 468 Zeichen
\newline{}Handschrift: schwarze Tinte, lateinische Kurrent
\newline{}Versand: Stempel: »\nobreak{}\oindex{Unterach am Attersee@\textbf{Unterach am Attersee}, \emph{P.PPL}|pwk}Unterach am Attersee, 28/6 10, 5\nobreak{}«.  
\newline{}Ordnung: mit Bleistift von unbekannter Hand nummeriert: »264« }\toendnotes[C]{\smallbreak}\pstart{}{\pb}Salten.\pend{}\pstart{}Unterach a. Attersee\oindex{Unterach am Attersee@\textbf{Unterach am Attersee}, \emph{P.PPL}|pw}. Berghof\oindex{Berghof@\textbf{Berghof}, \emph{Wohngebäude (K.WHS)}|pw}.\pend{}{\bigskip}\pstart{}Herrn\pend{}\pstart{}D\textsuperscript{r} Arthur Schnitzler\pend{}\pstart{}Wien\oindex{Wien@\textbf{Wien}, \emph{A.ADM2}|pw}\pend{}\pstart{}XVIII. Spöttelgaße 7\oindex{Edmund-Weiss-Gasse 7@\textbf{Edmund-Weiß-Gasse 7}, \emph{Wohngebäude (K.WHS)}|pw}\pend{}{\bigskip}\vspace{1em}
\pstart
           \raggedleft{}{\pb}28. VI. 10\pend
           
\pstart{}Lieber,\pend\vspace{0.5em}
\pstart
           vielen Dank! Ich freu mich, dass es\pwindex{Kuenstler sollen reden@\emph{Künstler sollen reden}|pwv} Ihnen gefallen hat, und bin froh, dass diese Sache\pwindex{Kuenstler sollen reden@\emph{Künstler sollen reden}|pwv} auch sonst – wie es scheint – \substVorne{}\textsuperscript{I}\substDazwischen{}i\substHinten{}hre Wirkung tut. Wir leben hier\oindex{Unterach am Attersee@\textbf{Unterach am Attersee}, \emph{P.PPL}|pwv} sehr angenehm, sehr still, und ich arbeite viel. Es regnet oft, aber
               das verdirbt uns, wenigstens bisher, den Aufenthalt nicht. Alles Schöne zur Arbeit am
                  Haus\oindex{Sternwartestrasse 71@\textbf{Sternwartestraße 71}, \emph{Wohngebäude (K.WHS)}|pwv} und zum übrigen
               Arbeiten. Herzliche Grüße von uns\pwindex{Salten, Ottilie 07.03.1868 – 22.06.1942@\textsc{Salten, Ottilie} (07.03.1868 – 22.06.1942), \emph{Schauspieler/Schauspielerin}|pwv} zu Ihnen.\pend
           
\pstart
           Ihr {\\[\baselineskip]}\spacefill\mbox{F. S.}\pend
           \leftskip=0em{}\selectlanguage{ngerman}\endnumbering\briefempfaengerindex{Schnitzler, Arthur@\textsc{Schnitzler, Arthur}!zzzSalten, Felix@\emph{von Felix Salten}!1910-06-282@{28. 6. 1910}|)be}\mylabel{L03549h}  \normalsize

\doendnotes{C}
\bigskip
\vfill

\clearpage

\footnotesize

\lohead{\textsc{register}}

% Definiere theindex-Environment komplett neu ohne reledmac
\makeatletter
\renewenvironment{theindex}{%
  \section*{\indexname}%
  \setlength{\parindent}{0pt}%
  \setlength{\parskip}{0pt plus 0.3pt}%
  \let\item\@idxitem
}{%
  \clearpage
}
\makeatother

\IfFileExists{\jobname-pw.ind}{\input{\jobname-pw.ind}}{}

\end{document}

      