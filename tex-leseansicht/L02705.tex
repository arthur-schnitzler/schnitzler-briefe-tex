%% latex-leseansicht-vorspann.tex
%% Vorspann für die Leseansicht.
%% Lädt die gemeinsame Datei latex-vorspann.tex mit nicht gesetztem Schalter.

\newif\ifkorrekturansicht
\korrekturansichtfalse

\input{../tex-inputs/latex-vorspann}


         
         \renewcommand{\erwaehntePersonen}{Personen: Paul Goldmann, Leopold Sonnemann}
         \renewcommand{\erwaehnteInstitutionen}{Institutionen: Frankfurter Zeitung}
         \renewcommand{\erwaehnteOrte}{Orte: Paris, Wien, rue Richelieu}
         \renewcommand{\erwaehnteWerke}{}
               \section[Paul Goldmann an Arthur Schnitzler, 25. 2. {[}1893{]}]{ Paul Goldmann an Arthur Schnitzler, 25. 2. {[}1893{]}}\nopagebreak\mylabel{v}\rehead{ }\begin{ledgroupsized}[t]{13cm}\normalsize\beginnumbering\briefempfaengerindex{Schnitzler, Arthur@\textsc{Schnitzler, Arthur}!zzzGoldmann, Paul@\emph{von Paul Goldmann}!1893-02-251@{25. 2. {[}1893{]}}|(be} \toendnotes[C]{\smallbreak\pagebreak[2]} \Standort{DLA, A:Schnitzler, HS.NZ85.1.3163.}
\physDesc{Brief, 1 Blatt, 3 Seiten, 1594 Zeichen
\newline{}Handschrift: schwarze Tinte, deutsche Kurrent
\newline{}Schnitzler: mit Bleistift das Jahr »93« vermerkt }\toendnotes[C]{\smallbreak}\pstart
           \noindent{}{\pb}\textcolor{gray}{\textbf{\textbf{Frankfurter Zeitung\orgindex{Frankfurter Zeitung@Frankfurter Zeitung|pw}.}}}\pend
           \pstart
           \textcolor{gray}{\textbf{\textbf{(\begin{otherlanguage}{french}Gazette de Francfort\end{otherlanguage}\orgindex{Frankfurter Zeitung@Frankfurter Zeitung|pw}.)}}}\pend
           \pstart
           \textcolor{gray}{\textbf{\begin{otherlanguage}{french}Directeur\end{otherlanguage}{ }\textbf{M. L. Sonnemann\pwindex{Sonnemann, Leopold 1831-10-29 – 1909-10-30@\textsc{Sonnemann, Leopold} (1831-10-29 – 1909-10-30), \emph{Journalist, Herausgeber}|pw}.}}}\hfill \textsc{Paris\oindex{Paris@\textbf{Paris}|pw}}, 25. Februar.\pend
           \pstart
           \begin{otherlanguage}{french}\textcolor{gray}{\textbf{Journal politique, financier,}}\end{otherlanguage}\pend
           \pstart
           \begin{otherlanguage}{french}\textcolor{gray}{\textbf{commercial et litteraire.}}\end{otherlanguage}\pend
           \pstart
           \begin{otherlanguage}{french}\textcolor{gray}{\textbf{\textbf{Paraissant trois fois par jour}}}\end{otherlanguage}\pend
           \pstart
           \begin{otherlanguage}{french}\textcolor{gray}{\textbf{\textbf{Bureaux à Paris\oindex{Paris@\textbf{Paris}|pw}:}}}\end{otherlanguage}\pend
           \pstart
           \begin{otherlanguage}{french}\textcolor{gray}{\textbf{\textbf{rue Richelieu 75\oindex{rue Richelieu@\textbf{rue Richelieu}|pw}.}}}\end{otherlanguage}\pend
           \pstart\center{}Mein lieber Freund!\pend\pstart
           Ich hätte Dir ſchon längſt für Deinen ſo lieben Brief danken ſollen. Aber in
               Zuſtänden wie der meinige hat man nicht immer die moraliſche Energie, ſich zum
               Schreiben aufzuraffen. Sich in die Berufsarbeit zu vergraben, ſich daran zu betrinken
               und zu betäuben – das bringt man zuſammen. Aber wenn man mit denen ſich beſchäftigen
               ſoll, die Einem lieb und theuer ſind, ſo kommt Einem die ganze Entſetzlichkeit zum
               Bewußtſein, in der man ſich befindet – durch die Erin{\pb}nerung, den Contraſt mit früher \textsc{etc}. Du wirſt das
               verſtehen und mir nicht zürnen.\pend
           \pstart
           Aber ich muß Dir doch ſagen, daß mir Dein lieber Brief unendlich wohlgethan hat.
               Nicht wegen des Inhalts, der viel zu ſehr nach Troſt ausſieht, als daß ich ein Wort
               davon glauben könnte, – aber wegen der treuen freundſchaftlichen Geſinnung, der
               Herzensgüte, an die ich armer Verlaſſener und Verlorener nicht mehr gewöhnt bin. Laß’
               Dir alſo von ganzem Herzen dafür danken{\dotsfour}\pend
           \pstart
           Der \label{K_L02705-1v}\edtext{Verlauf}{\lemma{\textnormal{\emph{Verlauf}}}\Cendnote{\textnormal{Bezug auf Goldmann\pwindex{Goldmann, Paul 31.01.1865 – 25.09.1935@\textsc{Goldmann, Paul} (31.01.1865 – 25.09.1935), \emph{Schriftsteller, Journalist}|pwk}s
                  Geschlechtskrankheit, vgl. Paul Goldmann an Arthur Schnitzler, 6. 2. [1893]}}}\label{K_L02705-1h} iſt der gewöhnliche. Ich bin im erſten Anfangsſtadium. Die Symptome ſtellen
               ſich ſicher, aber ſehr langſam eines nach dem {\pb}andern ein. Die eigentlich ernſte Behandlung wird wohl erſt nächſte Woche beginnen.
               Ich bin auf das Schlimmſte vorbereitet und wohl Mann genug, um mein Loos bis zum Ende
               zu tragen.\pend
           \pstart
           Du biſt der Einzige, der darum weiß. Das war wohl auch vielleicht Unrecht. Aber die
               Schwachheit iſt entſchuldbar. Man erſtickt unter der Laſt, und es iſt eine
               Erleichterung, es wenigſtens Einem ſagen zu können.\pend
           \pstart
           Grüß’ Dich Gott, mein lieber Arthur! Schreib’ mir, bitte, wie es Dir geht, und recht
               ausführlich.\pend
           \pstart
           Dein {\\[\baselineskip]}treuer {\\[\baselineskip]}\spacefill\mbox{Paul Goldmn.}\pend
           \leftskip=0em{}
         
         \endnumbering\mylabel{h}\end{ledgroupsized}  \newcommand{\dateiname}{L02705}\newcommand{\titel}{Paul Goldmann an Arthur Schnitzler, 25. 2. [1893]}\newcommand{\editorInnen}{Martin Anton Müller und Laura Untner}%% latex-leseansicht-abspann.tex
%% Abspann für die Leseansicht.
%% Der Schalter \ifkorrekturansicht ist bereits durch den Vorspann gesetzt.

%% latex-abspann.tex
%% Gemeinsamer Abspann für Korrekturansicht und Leseansicht.
%% Setzt den Schalter \ifkorrekturansicht voraus (gesetzt in den
%% einbindenden Dateien latex-korrekturansicht-abspann.tex bzw.
%% latex-leseansicht-abspann.tex).
%% ---------------------------------------------------------------

\normalsize

% Das esempio-Environment wird nur in der Leseansicht benötigt
\ifkorrekturansicht\else
\newenvironment{esempio}[3]%
{
    \vspace{1.5ex}
    \rlap{\underline{#1}}
    \par
    \setlength{\parindent}{0cm}
    \nopagebreak
    \leftskip=#2cm
    \rightskip=#3cm
}
{
    \par
}
\fi

\doendnotes{C}
\bigskip
\vfill

\clearpage

\footnotesize

\ifkorrekturansicht
  \lohead{\textsc{register}}
\fi

% theindex-Environment neu definieren ohne reledmac
\makeatletter
\renewenvironment{theindex}{%
  \ifkorrekturansicht
    \section*{\indexname}%
  \else
    \subsubsection*{Index der erwähnten Entitäten}%
  \fi
  \setlength{\parindent}{0pt}%
  \setlength{\parskip}{0pt plus 0.3pt}%
  \let\item\@idxitem
}{%
  \ifkorrekturansicht\clearpage\fi
}
\makeatother

\IfFileExists{\jobname-pw.ind}{\input{\jobname-pw.ind}}{}

% Quellenangabe nur in der Leseansicht
\ifkorrekturansicht\else
% Fallback-Definitionen, falls die .tex-Datei \titel etc. nicht gesetzt hat
\providecommand{\titel}{}
\providecommand{\editorInnen}{}
\providecommand{\dateiname}{\jobname}

\vspace{3cm}

\vfill

\footnotesize
\textsc{Quelle}: \titel. Herausgegeben von {\editorInnen}. In: \emph{Arthur Schnitzler: Briefwechsel mit Autorinnen und Autoren}.
 Digitale Edition, https://schnitzler-briefe.acdh.oeaw.ac.at/{\dateiname}.html (Stand \today)
\fi

\end{document}


      