\input{../tex-inputs/latex-pdf-vorspann}
\begin{center}
            \textcolor{red}{ENTWURF. ENTZIFFERUNG NOCH NICHT KORREKTURGELESEN}
                      \end{center}
            
               \section[Paul Goldmann an Arthur Schnitzler, 25. 2. {[}1893{]}]{ Paul Goldmann an Arthur Schnitzler, 25. 2. {[}1893{]}}\nopagebreak\mylabel{v}\rehead{ }\begin{ledgroupsized}[t]{13cm}\normalsize\beginnumbering\briefempfaengerindex{Schnitzler, Arthur@\textsc{Schnitzler, Arthur}!zzzGoldmann, Paul@\emph{von Paul Goldmann}!1893-02-251@{25. 2. {[}1893{]}}|(be} \toendnotes[C]{\smallbreak\pagebreak[2]} \Standort{DLA, A:Schnitzler, HS.NZ85.1.3163.}
\physDesc{Brief, 1 Blatt, 3 Seiten
\newline{}Handschrift: schwarze Tinte, deutsche Kurrent
\newline{}Schnitzler: mit Bleistift das Jahr »93« vermerkt }\toendnotes[C]{\smallbreak}\pstart
           \noindent{}{\pb}\textcolor{gray}{\textbf{\textbf{Frankfurter Zeitung\orgindex{Frankfurter Zeitung@Frankfurter Zeitung|pw}.}}}\pend
           \pstart
           \textcolor{gray}{\textbf{\textbf{(\begin{otherlanguage}{french}Gazette de Francfort\end{otherlanguage}\orgindex{Frankfurter Zeitung@Frankfurter Zeitung|pw}.)}}}\pend
           \pstart
           \textcolor{gray}{\textbf{\begin{otherlanguage}{french}Directeur\pwindex{Sonnemann, Leopold 1831-10-29 – 1909-10-30@\textsc{Sonnemann, Leopold} (1831-10-29 – 1909-10-30), \emph{Journalist, Herausgeber}|pwv}\end{otherlanguage}{ }\textbf{M. L. Sonnemann\pwindex{Sonnemann, Leopold 1831-10-29 – 1909-10-30@\textsc{Sonnemann, Leopold} (1831-10-29 – 1909-10-30), \emph{Journalist, Herausgeber}|pw}.}}}\hfill \textsc{Paris\oindex{Paris@\textbf{Paris}|pw}}, 25. Februar.\pend
           \pstart
           \begin{otherlanguage}{french}\textcolor{gray}{\textbf{Journal\pwindex{Frankfurter Zeitung1856 – 1943@\emph{Frankfurter Zeitung}|pw} politique, financier,}}\end{otherlanguage}\pend
           \pstart
           \begin{otherlanguage}{french}\textcolor{gray}{\textbf{commercial et litteraire.}}\end{otherlanguage}\pend
           \pstart
           \begin{otherlanguage}{french}\textcolor{gray}{\textbf{\textbf{Paraissant trois fois par jour}}}\end{otherlanguage}\pend
           \pstart
           \begin{otherlanguage}{french}\textcolor{gray}{\textbf{\textbf{Bureaux à Paris\oindex{Paris@\textbf{Paris}|pw}:}}}\end{otherlanguage}\pend
           \pstart
           \begin{otherlanguage}{french}\textcolor{gray}{\textbf{\textbf{rue Richelieu 75\oindex{rue Richelieu@\textbf{rue Richelieu}|pw}.}}}\end{otherlanguage}\pend
           \pstart
           Mein lieber Freund!\pend
           \pstart
           Ich hätte Dir ſchon längſt für Deinen ſo lieben Brief danken ſollen. Aber in
               Zuſtänden wie der meinige hat man nicht immer die moraliſche Energie, ſich zum
               Schreiben aufzuraffen. Sich in die Berufsarbeit zu vergraben, ſich daran zu betrinken
               und zu betäuben – das bringt man zuſammen. Aber wenn man mit denen ſich beſchäftigen
               ſoll, die Einem lieb und theuer ſind, ſo kommt Einem die ganze Entſetzlichkeit zum
               Bewußtſein, in der man ſich befindet – durch die Erin{\pb}nerung, den Contraſt mit früher \textsc{etc}. Du wirſt das
               verſtehen und mir nicht zürnen.\pend
           \pstart
           Aber ich muß Dir doch ſagen, daß mir dein lieber Brief unendlich wohlgethan hat.
               Nicht wegen des Inhalts, der viel zu ſehr nach Troſt ausſieht, als daß ich ein Wort
               davon glauben könnte, – aber wegen der treuen freundschaftlichen Geſinnung, der
               Herzensgüte, an die ich armer Verlaſſener und Verlorener nicht mehr gewöhnt bin. Laß’
               Dir alſo von ganzem Herzen dafür danken{\dotsfour}\pend
           \pstart
           Der \label{K_L02705-1v}\edtext{Verlauf}{\lemma{\textnormal{\emph{Verlauf}}}\Cendnote{\textnormal{Siehe Paul Goldmann an Arthur Schnitzler, 6. 2. [1893]}}}\label{K_L02705-1h} iſt der gewöhnliche. Ich bin im erſten Anfangsſtadium. Die Symptome ſtellen
               ſich ſicher, aber ſehr langſam eines nach dem {\pb}andern ein. Die eigentlich ernſte Behandlung wird wohl erſt nächſte Woche beginnen.
               Ich bin auf das Schlimmſte vorbereitet und wohl Mann genug, um mein Loos bis zum Ende
               zu tragen. Du biſt der Einzige, der darum weiß. Das war wohl auch vielleicht Unrecht.
               Aber die Schwachheit iſt entſchuldbar. Man erſtickt unter der Laſt, und es iſt eine
               Erleichterung, es wenigſtens Einem ſagen zu können.\pend
           \pstart
           Grüß’ Dich Gott, mein lieber Arthur! Schreib’ mir bitte, wie es Dir geht, und recht
               ausführlich.\pend
           \pstart
           Dein {\\[\baselineskip]}treuer {\\[\baselineskip]}\spacefill\mbox{Paul Goldm.}\pend
           \leftskip=0em{}\endnumbering\briefempfaengerindex{Schnitzler, Arthur@\textsc{Schnitzler, Arthur}!zzzGoldmann, Paul@\emph{von Paul Goldmann}!1893-02-251@{25. 2. {[}1893{]}}|)be}\mylabel{h}\end{ledgroupsized}\begin{anhang}\end{anhang}\newcommand{\dateiname}{L02705}\newcommand{\titel}{Paul Goldmann an Arthur Schnitzler, 25. 2. [1893]}\newcommand{\editorInnen}{Martin Anton Müller und Laura Untner}\input{../tex-inputs/latex-pdf-abspann}
      