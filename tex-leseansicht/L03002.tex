%% latex-leseansicht-vorspann.tex
%% Vorspann für die Leseansicht.
%% Lädt die gemeinsame Datei latex-vorspann.tex mit nicht gesetztem Schalter.

\newif\ifkorrekturansicht
\korrekturansichtfalse

\input{../tex-inputs/latex-vorspann}

\begin{center}
            \textcolor{red}{ENTWURF, NICHT FERTIG KORRIGIERT}
                      \end{center}
            
         
         \newcommand{\erwaehntePersonen}{Personen: Hermann Bahr, Otto Brahm, Felix Salten, Albert von Speidel}
         \newcommand{\erwaehnteOrte}{Orte: Berlin, Edmund-Weiß-Gasse, Kantstraße, Wien}
         \newcommand{\erwaehnteWerke}{Werke: Der Ruf des Lebens. Schauspiel in drei Akten}
               \section[Arthur Schnitzler an Felix Salten, 30. 1. 1906]{ Arthur Schnitzler an Felix Salten, 30. 1. 1906}\nopagebreak\mylabel{v}\rehead{ }\begin{ledgroupsized}[t]{13cm}\normalsize\beginnumbering \toendnotes[C]{\smallbreak\pagebreak[2]} \Standort{Wienbibliothek im Rathaus, ZPH 1681, 2.1.516.}
\physDesc{
\newline{}Handschrift: , deutsche Kurrent}\toendnotes[C]{\smallbreak}\pstart
           \noindent{}{\pb}\textcolor{gray}{\textbf{Dr. Arthur Schnitzler}}\hfill 30. 1. 906\pend
           \pstart
           \textcolor{gray}{\textbf{Wien, XVIII.
                        Spoettelgasse 7\oindex{Edmund-Weiss-Gasse@\textbf{Edmund-Weiß-Gasse}|pw}.}}\pend
           \pstart
           lieber, zum Einzug in Berlin\oindex{Berlin@\textbf{Berlin}|pw} und
               in die neue Wohnung wünſchen wir Ihnen Alles erdenkliche gute u ſchöne. Am 17. etwa
               denken wir nach Berlin\oindex{Berlin@\textbf{Berlin}|pw} zu fahren, wo die Pr. des
                  »Ruf\pwindex{Schnitzler, Arthur 15.05.1862 – 21.10.1931@\textsc{Schnitzler, Arthur} (15.05.1862 – 21.10.1931), \emph{Schriftsteller, Mediziner}!Ruf des Lebens. Schauspiel in drei Akten1906-02-20@\strich\emph{Der Ruf des Lebens. Schauspiel in drei Akten} {[}1906-02-20{]}|pw}« am 24. ſtattfinden ſoll;
               ſehr möglich aber wär es, daſs ich {\pb}um den 5. Feber herum auf einige Tage hinfahre, theils zu den
               Arrangirproben, theils zu Brahm\pwindex{Brahm, Otto 05.02.1856 – 28.11.1912@\textsc{Brahm, Otto} (05.02.1856 – 28.11.1912), \emph{Theaterleiter, Regisseur}|pw}s fünfzigſtem. \pend
           \pstart
           – Von Bahr\pwindex{Bahr, Hermann 19.07.1863 – 15.01.1934@\textsc{Bahr, Hermann} (19.07.1863 – 15.01.1934), \emph{Schriftsteller, Kritiker}|pw} erhielt ich geſtern Nachricht, daſs
               ihm der Intendant\pwindex{Speidel, Albert von 26.01.1858 – 01.09.1912@\textsc{Speidel, Albert von} (26.01.1858 – 01.09.1912), \emph{Theaterleiter}|pwv} die
               Genehmigung zur Annahme des »Ruf\pwindex{Schnitzler, Arthur 15.05.1862 – 21.10.1931@\textsc{Schnitzler, Arthur} (15.05.1862 – 21.10.1931), \emph{Schriftsteller, Mediziner}!Ruf des Lebens. Schauspiel in drei Akten1906-02-20@\strich\emph{Der Ruf des Lebens. Schauspiel in drei Akten} {[}1906-02-20{]}|pw}« (die er
               dringend verlangt hatte) verweigert hat. Er fügt hinzu: »Es iſt das nur ein Glied in
               der Kette, {\pb}von kleinen
               Gemeinheiten, durch welche man mich jetzt aus meinem Contract hinausekeln will, was
               vermutlich gelingen wird.« (Bitte das vorläufig als vertraulich zu behandeln, ich
               meine natürlich gegenüber Berlin\oindex{Berlin@\textbf{Berlin}|pw}er Bekannten). \pend
           \pstart
           Wenn ich komme, melde ich mich natürlich gleich. {\\[\baselineskip]}Von Herzen, mit Grüßen
               {\\[\baselineskip]}vom Spöttel\oindex{Edmund-Weiss-Gasse@\textbf{Edmund-Weiß-Gasse}|pw} nach Kant\oindex{Kantstrasse@\textbf{Kantstraße}|pw} Ihr \spacefill\mbox{A.}\pend
           \leftskip=0em{}
         
         \endnumbering\mylabel{h}\end{ledgroupsized}\begin{anhang}\end{anhang}\newcommand{\dateiname}{L03002}\newcommand{\titel}{Arthur Schnitzler an Felix Salten, 30. 1. 1906}\newcommand{\editorInnen}{Martin Anton Müller und Laura Untner}%% latex-leseansicht-abspann.tex
%% Abspann für die Leseansicht.
%% Der Schalter \ifkorrekturansicht ist bereits durch den Vorspann gesetzt.

%% latex-abspann.tex
%% Gemeinsamer Abspann für Korrekturansicht und Leseansicht.
%% Setzt den Schalter \ifkorrekturansicht voraus (gesetzt in den
%% einbindenden Dateien latex-korrekturansicht-abspann.tex bzw.
%% latex-leseansicht-abspann.tex).
%% ---------------------------------------------------------------

\normalsize

% Das esempio-Environment wird nur in der Leseansicht benötigt
\ifkorrekturansicht\else
\newenvironment{esempio}[3]%
{
    \vspace{1.5ex}
    \rlap{\underline{#1}}
    \par
    \setlength{\parindent}{0cm}
    \nopagebreak
    \leftskip=#2cm
    \rightskip=#3cm
}
{
    \par
}
\fi

\doendnotes{C}
\bigskip
\vfill

\clearpage

\footnotesize

\ifkorrekturansicht
  \lohead{\textsc{register}}
\fi

% theindex-Environment neu definieren ohne reledmac
\makeatletter
\renewenvironment{theindex}{%
  \ifkorrekturansicht
    \section*{\indexname}%
  \else
    \subsubsection*{Index der erwähnten Entitäten}%
  \fi
  \setlength{\parindent}{0pt}%
  \setlength{\parskip}{0pt plus 0.3pt}%
  \let\item\@idxitem
}{%
  \ifkorrekturansicht\clearpage\fi
}
\makeatother

\IfFileExists{\jobname-pw.ind}{\input{\jobname-pw.ind}}{}

% Quellenangabe nur in der Leseansicht
\ifkorrekturansicht\else
% Fallback-Definitionen, falls die .tex-Datei \titel etc. nicht gesetzt hat
\providecommand{\titel}{}
\providecommand{\editorInnen}{}
\providecommand{\dateiname}{\jobname}

\vspace{3cm}

\vfill

\footnotesize
\textsc{Quelle}: \titel. Herausgegeben von {\editorInnen}. In: \emph{Arthur Schnitzler: Briefwechsel mit Autorinnen und Autoren}.
 Digitale Edition, https://schnitzler-briefe.acdh.oeaw.ac.at/{\dateiname}.html (Stand \today)
\fi

\end{document}


      