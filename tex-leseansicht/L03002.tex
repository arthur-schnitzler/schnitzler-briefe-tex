%% latex-korrekturansicht-vorspann.tex
%% Vorspann für die Korrekturansicht.
%% Lädt die gemeinsame Datei latex-vorspann.tex mit gesetztem Schalter.

\newif\ifkorrekturansicht
\korrekturansichttrue

\input{../tex-inputs/latex-vorspann}


\section[ Arthur Schnitzler an Felix Salten, 30. 1. 1906]{L03002 Arthur Schnitzler an Felix Salten, 30. 1. 1906}
\nopagebreak\mylabel{L03002v}
\rehead{ }\normalsize\beginnumbering\briefempfaengerindex{Salten, Felix@\textsc{Salten, Felix}!zzzSchnitzler, Arthur@\emph{von Arthur Schnitzler}!1906-01-302@{30. 1. 1906}|(be}
\toendnotes[C]{\smallbreak\pagebreak[2]}\Standort{Wienbibliothek im Rathaus, ZPH 1681, 2.1.516.}
\physDesc{Brief, 1 Blatt, 3 Seiten, 846 Zeichen
\newline{}Handschrift: schwarze Tinte, deutsche Kurrent
\newline{}Ordnung: mit Bleistift von unbekannter Hand Nummerierung der Doppelseiten des Konvoluts:
                                    »24«–»25« }\toendnotes[C]{\smallbreak}
\pstart
           {\pb}\textcolor{gray}{\textbf{Dr. Arthur Schnitzler}}\hfill 30. 1. 906\pend
           
\pstart
           \textcolor{gray}{\textbf{Wien, XVIII. Spoettelgasse 7\oindex{Edmund-Weiss-Gasse 7@\textbf{Edmund-Weiß-Gasse 7}, \emph{Wohngebäude (K.WHS)}|pw}.}}\pend
           \vspace{0.5em}
\pstart
           lieber, zum Einzug in Berlin\oindex{Berlin@\textbf{Berlin}, \emph{P.PPLC}|pw} und
               in die neue Wohnung\oindex{Kantstrasse@\textbf{Kantstraße}, \emph{Straße (K.STR)}|pwv} wünſchen
               wir Ihnen Alles erdenkliche gute u ſchöne. \label{K_L03002-1v}\edtext{Am 17. etwa}{\lemma{\textnormal{\emph{Am 17. etwa}}}\Cendnote{\textnormal{Die Abreise fand am Abend des
                     16. 2. 1906
                  statt.}}}\label{K_L03002-1} denken wir nach Berlin\oindex{Berlin@\textbf{Berlin}, \emph{P.PPLC}|pw} zu
               fahren, wo die \label{K_L03002-2v}\edtext{Pr. des »Ruf\pwindex{Ruf des Lebens. Schauspiel in drei Akten@\emph{Der Ruf des Lebens. Schauspiel in drei Akten}|pw}« am 24.}{\lemma{\textnormal{\emph{Pr. des »Ruf« am 24.}}}\Cendnote{\textnormal{Am 24. 2. 1906 fand die deutschsprachige
                  Uraufführung von \emph{Der Ruf des Lebens}\pwindex{Ruf des Lebens. Schauspiel in drei Akten@\emph{Der Ruf des Lebens. Schauspiel in drei Akten}|pwk} am \emph{Lessing-Theater}\orgindex{Lessing-Theater@Lessing-Theater|pwk} statt.}}}\label{K_L03002-2} ſtattfinden ſoll;
               ſehr möglich aber wär es, daſs ich {\pb}\label{K_L03002-3v}\edtext{um den 5. Feber}{\lemma{\textnormal{\emph{um den 5. Feber}}}\Cendnote{\textnormal{Am 3. 2. 1906 fuhr Schnitzler nach Berlin\oindex{Berlin@\textbf{Berlin}, \emph{P.PPLC}|pwk}, am 5. 2. 1906 und am
                  Folgetag fanden Arrangierproben statt. Der 7. 2. 1906 war der Tag der Rückreise.}}}\label{K_L03002-3} herum auf einige Tage hinfahre, theils zu den
               Arrangirproben, theils zu \label{K_L03002-4v}\edtext{Brahms\pwindex{Brahm, Otto 05.02.1856 – 28.11.1912@\textsc{Brahm, Otto} (05.02.1856 – 28.11.1912), \emph{Theaterleiter/Theaterleiterin, Regisseur/Regisseurin}|pw}
               fünfzigſtem}{\lemma{\textnormal{\emph{Brahms
               fünfzigſtem}}}\Cendnote{\textnormal{Vgl. A. S.: \emph{Tagebuch}, 5. 2. 1906.
               }}}\label{K_L03002-4}.\pend
           
\pstart
           – Von \label{K_L03002-5v}\edtext{Bahr\pwindex{Bahr, Hermann 19.07.1863 – 15.01.1934@\textsc{Bahr, Hermann} (19.07.1863 – 15.01.1934), \emph{Schriftsteller/Schriftstellerin, Kritiker/Kritikerin}|pw} erhielt ich geſtern Nachricht}{\lemma{\textnormal{\emph{Bahr … Nachricht}}}\Cendnote{\textnormal{Siehe Hermann Bahr an Arthur Schnitzler, 29. 1. 1906.
               }}}\label{K_L03002-5}, daſs ihm der \label{K_L03002-6v}\edtext{Intendant\pwindex{Speidel, Albert von 26.01.1858 – 01.09.1912@\textsc{Speidel, Albert von} (26.01.1858 – 01.09.1912), \emph{Theaterleiter/Theaterleiterin}|pwv} die Genehmigung
               zur Annahme}{\lemma{\textnormal{\emph{Intendant … Annahme}}}\Cendnote{\textnormal{Bahr\pwindex{Bahr, Hermann 19.07.1863 – 15.01.1934@\textsc{Bahr, Hermann} (19.07.1863 – 15.01.1934), \emph{Schriftsteller/Schriftstellerin, Kritiker/Kritikerin}|pwk} war zum Oberregisseur des \emph{Münchener Hoftheaters}\orgindex{Nationaltheater Muenchen@Nationaltheater München|pwk} ernannt worden. Aufgrund
                  von öffentlichem konservativem Gegenwind kam es zur Vertragsauflösung.}}}\label{K_L03002-6} des
                  »Ruf\pwindex{Ruf des Lebens. Schauspiel in drei Akten@\emph{Der Ruf des Lebens. Schauspiel in drei Akten}|pw}« (die er dringend verlangt hatte)
               verweigert hat. Er fügt hinzu: »Es iſt das nur ein Glied in der Kette {\pb}von kleinen Gemeinheiten, durch welche man
               mich jetzt aus meinem Contract hinausekeln will, was vermuthlich gelingen wird.«
               (bitte das vorläufig als vertraulich zu behandeln, ich meine natürlich gegenüber Berlin\oindex{Berlin@\textbf{Berlin}, \emph{P.PPLC}|pw}er Bekannten).\pend
           
\pstart
           Wenn ich komme, melde ich mich natürlich gleich. {\\[\baselineskip]}Von Herzen, mit Grüßen von
                  Spöttel\oindex{Edmund-Weiss-Gasse 7@\textbf{Edmund-Weiß-Gasse 7}, \emph{Wohngebäude (K.WHS)}|pw} nach \label{K_L03002-7v}\edtext{Kant\oindex{Kantstrasse@\textbf{Kantstraße}, \emph{Straße (K.STR)}|pw}}{\lemma{\textnormal{\emph{Kant}}}\Cendnote{\textnormal{Salten\pwindex{Salten, Felix 06.09.1869 – 08.10.1945@\textsc{Salten, Felix} (06.09.1869 – 08.10.1945), \emph{Schriftsteller/Schriftstellerin, Journalist/Journalistin, Chefredakteur/Chefredakteurin}|pwk} hatte in Berlin\oindex{Berlin@\textbf{Berlin}, \emph{P.PPLC}|pwk} eine Unterkunft in der Kantstraße 34\oindex{Kantstrasse@\textbf{Kantstraße}, \emph{Straße (K.STR)}|pwk} bezogen, vgl. Felix Salten an Arthur Schnitzler, 29. 1. 1906.}}}\label{K_L03002-7} Ihr \spacefill\mbox{A.}\pend
           \leftskip=0em{}\selectlanguage{ngerman}\endnumbering\briefempfaengerindex{Salten, Felix@\textsc{Salten, Felix}!zzzSchnitzler, Arthur@\emph{von Arthur Schnitzler}!1906-01-302@{30. 1. 1906}|)be}\mylabel{L03002h}  \normalsize

\doendnotes{C}
\bigskip
\vfill

\clearpage

\footnotesize

\lohead{\textsc{register}}

% Definiere theindex-Environment komplett neu ohne reledmac
\makeatletter
\renewenvironment{theindex}{%
  \section*{\indexname}%
  \setlength{\parindent}{0pt}%
  \setlength{\parskip}{0pt plus 0.3pt}%
  \let\item\@idxitem
}{%
  \clearpage
}
\makeatother

\IfFileExists{\jobname-pw.ind}{\input{\jobname-pw.ind}}{}

\end{document}

      