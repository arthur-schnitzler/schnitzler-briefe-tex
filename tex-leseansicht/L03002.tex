%% latex-leseansicht-vorspann.tex
%% Vorspann für die Leseansicht.
%% Lädt die gemeinsame Datei latex-vorspann.tex mit nicht gesetztem Schalter.

\newif\ifkorrekturansicht
\korrekturansichtfalse

\input{../tex-inputs/latex-vorspann}


\section[ Arthur Schnitzler an Felix Salten, 30. 1. 1906]{L03002 Arthur Schnitzler an Felix Salten,  30. 1. 1906}
\nopagebreak\mylabel{L03002v}
\rehead{ }\normalsize\beginnumbering\briefempfaengerindex{Salten, Felix@\textsc{Salten, Felix}!zzzSchnitzler, Arthur@\emph{von Arthur Schnitzler}!1906-01-302@{30. 1. 1906}|(be}
\toendnotes[C]{\smallbreak\pagebreak[2]}
\correspDesc{Versand  durch Arthur Schnitzler am 30. 1. 1906 in Wien
\newline{}Erhalt  durch Felix Salten im Zeitraum [31. 1. 1906
                  – 4. 2. 1906?] in Berlin}\toendnotes[C]{\smallbreak}
\Standort{Wienbibliothek im Rathaus, ZPH 1681, 2.1.516.}
\physDesc{Brief, 1 Blatt, 3 Seiten, 846 Zeichen
\newline{}Handschrift: schwarze Tinte, deutsche Kurrent
\newline{}Ordnung: mit Bleistift von unbekannter Hand Nummerierung der Doppelseiten des Konvoluts:
                                    »24«–»25« }\toendnotes[C]{\smallbreak}
\pstart
           {\pb}\textcolor{gray}{\textbf{Dr. Arthur Schnitzler}}\hfill 30. 1. 906\pend
           
\pstart
           \textcolor{gray}{\textbf{Wien, XVIII. Spoettelgasse 7\oindex{Wien@\textbf{Wien}!XVIII., Währing@\textbf{XVIII., Währing}!Edmund-Weiß-Gasse 7@\textbf{Edmund-Weiß-Gasse 7}, \emph{Wohngebäude}|pw}.}}\pend
           \vspace{0.5em}
\pstart
           lieber, zum Einzug in Berlin\oindex{Berlin@\textbf{Berlin}, \emph{Hauptstadt}|pw} und
               in die neue Wohnung\oindex{Kantstraße@\textbf{Kantstraße}, \emph{Straße}|pwv} wünſchen
               wir Ihnen Alles erdenkliche gute u{ }ſchöne. \label{K_L03002-1v}\edtext{Am 17. etwa}{\lemma{\textnormal{\emph{Am 17. etwa}}}\Cendnote{\textnormal{Die Abreise fand am Abend des
                     16. 2. 1906
                  statt.}}}\label{K_L03002-1} denken wir nach Berlin\oindex{Berlin@\textbf{Berlin}, \emph{Hauptstadt}|pw} zu
               fahren, wo die \label{K_L03002-2v}\edtext{Pr. des »Ruf\pwindex{Schnitzler, Arthur 15.\,5.\,1862 Wien – 21.\,10.\,1931 ebd.@\textsc{Schnitzler, Arthur} (15.\,5.\,1862 Wien – 21.\,10.\,1931 ebd.), \emph{Schriftsteller, Mediziner}!Ruf des Lebens. Schauspiel in drei Akten@\strich\emph{Der Ruf des Lebens. Schauspiel in drei Akten}|pw}«\eventindex{Lessing-Theater@\textbf{Lessing-Theater}!Premiere von Der Ruf des Lebens, 24.2.1906@Premiere von Der Ruf des Lebens, 24.2.1906|pwv} am 24.}{\lemma{\textnormal{\emph{Pr. des »Ruf« am 24.}}}\Cendnote{\textnormal{Am 24. 2. 1906 fand die deutschsprachige
                  Uraufführung von \emph{Der Ruf des Lebens}\pwindex{Schnitzler, Arthur 15.\,5.\,1862 Wien – 21.\,10.\,1931 ebd.@\textsc{Schnitzler, Arthur} (15.\,5.\,1862 Wien – 21.\,10.\,1931 ebd.), \emph{Schriftsteller, Mediziner}!Ruf des Lebens. Schauspiel in drei Akten@\strich\emph{Der Ruf des Lebens. Schauspiel in drei Akten}|pwk}\eventindex{Lessing-Theater@\textbf{Lessing-Theater}!Premiere von Der Ruf des Lebens, 24.2.1906@Premiere von Der Ruf des Lebens, 24.2.1906|pwk} am \emph{Lessing-Theater}\orgindex{Lessing-Theater@Lessing-Theater|pwk} statt.}}}\label{K_L03002-2}{ }ſtattfinden{ }ſoll;{ }ſehr möglich aber wär es, daſs ich {\pb}\label{K_L03002-3v}\edtext{um den 5. Feber}{\lemma{\textnormal{\emph{um den 5. Feber}}}\Cendnote{\textnormal{Am 3. 2. 1906 fuhr Schnitzler nach Berlin\oindex{Berlin@\textbf{Berlin}, \emph{Hauptstadt}|pwk}, am 5. 2. 1906 und am
                  Folgetag fanden Arrangierproben\eventindex{Lessing-Theater@\textbf{Lessing-Theater}!Arrangierprobe von Der Ruf des Lebens, 5.2.1906@Arrangierprobe von Der Ruf des Lebens, 5.2.1906|pwkv}\eventindex{Lessing-Theater@\textbf{Lessing-Theater}!Arrangierprobe von Der Ruf des Lebens, 6.2.1906@Arrangierprobe von Der Ruf des Lebens, 6.2.1906|pwkv} statt. Der 7. 2. 1906 war der Tag der Rückreise.}}}\label{K_L03002-3} herum auf einige Tage hinfahre, theils zu den
               Arrangirproben\eventindex{Lessing-Theater@\textbf{Lessing-Theater}!Arrangierprobe von Der Ruf des Lebens, 5.2.1906@Arrangierprobe von Der Ruf des Lebens, 5.2.1906|pwv}\eventindex{Lessing-Theater@\textbf{Lessing-Theater}!Arrangierprobe von Der Ruf des Lebens, 6.2.1906@Arrangierprobe von Der Ruf des Lebens, 6.2.1906|pwv}, theils zu \label{K_L03002-4v}\edtext{Brahms\pwindex{Brahm, Otto 5.\,2.\,1856 Hamburg – 28.\,11.\,1912 Berlin@\textsc{Brahm, Otto} (5.\,2.\,1856 Hamburg – 28.\,11.\,1912 Berlin), \emph{Theaterleiter, Regisseur}|pw}
               fünfzigſtem\eventindex{Kronprinzenufer 7/In den Zelten 14@\textbf{Kronprinzenufer 7/In den Zelten 14}!Feier des 50. Geburtstages von Otto Brahm@Feier des 50. Geburtstages von Otto Brahm|pwv}}{\lemma{\textnormal{\emph{Brahms
               fünfzigstem}}}\Cendnote{\textnormal{Vgl. A. S.: \emph{Tagebuch}, 5. 2. 1906.
               }}}\label{K_L03002-4}.\pend
           
\pstart
           – Von \label{K_L03002-5v}\edtext{Bahr\pwindex{Bahr, Hermann 19.\,7.\,1863 Linz – 15.\,1.\,1934 München@\textsc{Bahr, Hermann} (19.\,7.\,1863 Linz – 15.\,1.\,1934 München), \emph{Schriftsteller, Kritiker}|pw} erhielt ich geſtern Nachricht}{\lemma{\textnormal{\emph{Bahr … Nachricht}}}\Cendnote{\textnormal{Siehe XXXX Auszeichnungsfehler: Dokument L01577 nicht gefunden.
               }}}\label{K_L03002-5}, daſs ihm der \label{K_L03002-6v}\edtext{Intendant\pwindex{Speidel, Albert von 26.\,1.\,1858 München – 1.\,9.\,1912 ebd.@\textsc{Speidel, Albert von} (26.\,1.\,1858 München – 1.\,9.\,1912 ebd.), \emph{Theaterleiter}|pwv} die Genehmigung
               zur Annahme}{\lemma{\textnormal{\emph{Intendant … Annahme}}}\Cendnote{\textnormal{Bahr\pwindex{Bahr, Hermann 19.\,7.\,1863 Linz – 15.\,1.\,1934 München@\textsc{Bahr, Hermann} (19.\,7.\,1863 Linz – 15.\,1.\,1934 München), \emph{Schriftsteller, Kritiker}|pwk} war zum Oberregisseur des \emph{Münchener Hoftheaters}\orgindex{Nationaltheater München@Nationaltheater München|pwk} ernannt worden. Aufgrund
                  von öffentlichem konservativem Gegenwind kam es zur Vertragsauflösung.}}}\label{K_L03002-6} des
                  »Ruf\pwindex{Schnitzler, Arthur 15.\,5.\,1862 Wien – 21.\,10.\,1931 ebd.@\textsc{Schnitzler, Arthur} (15.\,5.\,1862 Wien – 21.\,10.\,1931 ebd.), \emph{Schriftsteller, Mediziner}!Ruf des Lebens. Schauspiel in drei Akten@\strich\emph{Der Ruf des Lebens. Schauspiel in drei Akten}|pw}« (die er dringend verlangt hatte)
               verweigert hat. Er fügt hinzu: »Es iſt das nur ein Glied in der Kette {\pb}von kleinen Gemeinheiten, durch welche man
               mich jetzt aus meinem Contract hinausekeln will, was vermuthlich gelingen wird.«
               (bitte das vorläufig als vertraulich zu behandeln, ich meine natürlich gegenüber Berlin\oindex{Berlin@\textbf{Berlin}, \emph{Hauptstadt}|pw}er Bekannten).\pend
           
\pstart
           Wenn ich komme, melde ich mich natürlich gleich. {\\[\baselineskip]}Von Herzen, mit Grüßen von
                  Spöttel\oindex{Wien@\textbf{Wien}!XVIII., Währing@\textbf{XVIII., Währing}!Edmund-Weiß-Gasse 7@\textbf{Edmund-Weiß-Gasse 7}, \emph{Wohngebäude}|pw} nach \label{K_L03002-7v}\edtext{Kant\oindex{Kantstraße@\textbf{Kantstraße}, \emph{Straße}|pw}}{\lemma{\textnormal{\emph{Kant}}}\Cendnote{\textnormal{Salten\pwindex{Salten, Felix 6.\,9.\,1869 Budapest – 8.\,10.\,1945 Zürich@\textsc{Salten, Felix} (6.\,9.\,1869 Budapest – 8.\,10.\,1945 Zürich), \emph{Schriftsteller, Journalist, Chefredakteur}|pwk} hatte in Berlin\oindex{Berlin@\textbf{Berlin}, \emph{Hauptstadt}|pwk} eine Unterkunft in der Kantstraße 34\oindex{Kantstraße@\textbf{Kantstraße}, \emph{Straße}|pwk} bezogen, vgl. XXXX Auszeichnungsfehler: Dokument L03413 nicht gefunden.}}}\label{K_L03002-7} Ihr \spacefill\mbox{A.}\pend
           \leftskip=0em{}\selectlanguage{ngerman}\endnumbering\briefempfaengerindex{Salten, Felix@\textsc{Salten, Felix}!zzzSchnitzler, Arthur@\emph{von Arthur Schnitzler}!1906-01-302@{30. 1. 1906}|)be}\mylabel{L03002h}  \newcommand{\dateiname}{L03002}\newcommand{\titel}{Arthur Schnitzler an Felix Salten, 30. 1. 1906}\newcommand{\editorInnen}{Martin Anton Müller und Laura Untner}%% latex-leseansicht-abspann.tex
%% Abspann für die Leseansicht.
%% Der Schalter \ifkorrekturansicht ist bereits durch den Vorspann gesetzt.

%% latex-abspann.tex
%% Gemeinsamer Abspann für Korrekturansicht und Leseansicht.
%% Setzt den Schalter \ifkorrekturansicht voraus (gesetzt in den
%% einbindenden Dateien latex-korrekturansicht-abspann.tex bzw.
%% latex-leseansicht-abspann.tex).
%% ---------------------------------------------------------------

\normalsize

% Das esempio-Environment wird nur in der Leseansicht benötigt
\ifkorrekturansicht\else
\newenvironment{esempio}[3]%
{
    \vspace{1.5ex}
    \rlap{\underline{#1}}
    \par
    \setlength{\parindent}{0cm}
    \nopagebreak
    \leftskip=#2cm
    \rightskip=#3cm
}
{
    \par
}
\fi

\doendnotes{C}
\bigskip
\vfill

\clearpage

\footnotesize

\ifkorrekturansicht
  \lohead{\textsc{register}}
\fi

% theindex-Environment neu definieren ohne reledmac
\makeatletter
\renewenvironment{theindex}{%
  \ifkorrekturansicht
    \section*{\indexname}%
  \else
    \subsubsection*{Index der erwähnten Entitäten}%
  \fi
  \setlength{\parindent}{0pt}%
  \setlength{\parskip}{0pt plus 0.3pt}%
  \let\item\@idxitem
}{%
  \ifkorrekturansicht\clearpage\fi
}
\makeatother

\IfFileExists{\jobname-pw.ind}{\input{\jobname-pw.ind}}{}

% Quellenangabe nur in der Leseansicht
\ifkorrekturansicht\else
% Fallback-Definitionen, falls die .tex-Datei \titel etc. nicht gesetzt hat
\providecommand{\titel}{}
\providecommand{\editorInnen}{}
\providecommand{\dateiname}{\jobname}

\vspace{3cm}

\vfill

\footnotesize
\textsc{Quelle}: \titel. Herausgegeben von {\editorInnen}. In: \emph{Arthur Schnitzler: Briefwechsel mit Autorinnen und Autoren}.
 Digitale Edition, https://schnitzler-briefe.acdh.oeaw.ac.at/{\dateiname}.html (Stand \today)
\fi

\end{document}


