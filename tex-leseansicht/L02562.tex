\input{../tex-inputs/latex-pdf-vorspann}
\begin{center}
            \textcolor{red}{ENTWURF. ENTZIFFERUNG NOCH NICHT KORREKTURGELESEN}
                      \end{center}
            
               \section[Olga und Arthur Schnitzler an Richard und Paula Beer-Hofmann, {[}19. 5. 1914{]}]{ Olga und Arthur Schnitzler an Richard und Paula Beer-Hofmann,
               {[}19. 5. 1914{]}}\nopagebreak\mylabel{v}\rehead{ }\begin{ledgroupsized}[t]{13cm}\normalsize\beginnumbering\briefempfaengerindex{Beer-Hofmann, Paula@\textsc{Beer-Hofmann, Paula}!zzzSchnitzler, Arthur@\emph{von Arthur Schnitzler}!1914-05-191@{{[}19. 5. 1914{]}}|(be}\briefempfaengerindex{Beer-Hofmann, Paula@\textsc{Beer-Hofmann, Paula}!zzzSchnitzler, Olga@\emph{von Olga Schnitzler}!1914-05-191@{{[}19. 5. 1914{]}}|(be}\briefempfaengerindex{Beer-Hofmann, Richard@\textsc{Beer-Hofmann, Richard}!zzzSchnitzler, Arthur@\emph{von Arthur Schnitzler}!1914-05-191@{{[}19. 5. 1914{]}}|(be}\briefempfaengerindex{Beer-Hofmann, Richard@\textsc{Beer-Hofmann, Richard}!zzzSchnitzler, Olga@\emph{von Olga Schnitzler}!1914-05-191@{{[}19. 5. 1914{]}}|(be} \toendnotes[C]{\smallbreak\pagebreak[2]} \Standort{YCGL, MSS 31.}
\physDesc{Bildpostkarte
\newline{}Handschrift Arthur Schnitzler: Bleistift, deutsche Kurrent\newline{}Handschrift Olga Schnitzler: Bleistift, lateinische Kurrent\newline{}Versand: 1) Stempel: »\nobreak{}\textcolor{gray}{20. 5.}\nobreak{}«.  2) Der Versandweg ist unklar, da eine Briefmarke des deutschen Reiches zum Einsatz kommt, die 
                                 Kreuzfahrt aber am 20. 5. 1910 von Süden 
                                 kommend erst Southampton\oindex{Southampton@\textbf{Southampton}|pw} erreicht\newline{}Ordnung: mit Bleistift von unbekannter Hand datiert:
                              »19. 5. 14.« }\pstart{}{\pb}Austria\oindex{Oesterreich@\textbf{Österreich}|pw}\pend{}\pstart{}Herrn u. Frau\pend{}\pstart{}D\textsuperscript{r} Richard Beer-Hofmann\pend{}\pstart{}Vienna\oindex{Wien@\textbf{Wien}|pw}\pend{}\pstart{}XVIII Hasenauerstr. 59\oindex{Hasenauerstrasse@\textbf{Hasenauerstraße}|pw}.\pend{}{\bigskip}\pstart
           \noindent{}\centering{}{\pb}\textcolor{gray}{\textbf{NORDDEUTSCHER LLOYD BREMEN\orgindex{Norddeutscher Lloyd@Norddeutscher Lloyd|pw}}}{\\}\textcolor{gray}{\textbf{Reichspostdampfer »York«}}\pend
           \pstart
           \noindent{}{\pb}Herzliche Grüsse!\pend
           \pstart
           morgen hoffen wir, in Southampten\oindex{Southampton@\textbf{Southampton}|pw} zu sein,
               übermorgen Antwerpen\oindex{Antwerpen@\textbf{Antwerpen}|pw}, Freitag{ }Amsterdam\oindex{Amsterdam@\textbf{Amsterdam}|pw}.\pend
           \pstart Ihre\spacefill\mbox{OlgaS.}\pend{}\pstart
           {\pb}{[}hs. Schnitzler:{]} Herzlichſt Ihr\spacefill\mbox{Arthur}\pend
           \pstart
           Eben passiren wir ziemlich ſeitlich den Golf von Biscaya\oindex{Biskaya@\textbf{Biskaya}|pw} bei
               ruhiger See.\pend
           \pstart
           Die Fahrt heut den 7. Tag, herrlich\pend
           \endnumbering\briefempfaengerindex{Beer-Hofmann, Paula@\textsc{Beer-Hofmann, Paula}!zzzSchnitzler, Arthur@\emph{von Arthur Schnitzler}!1914-05-191@{{[}19. 5. 1914{]}}|)be}\briefempfaengerindex{Beer-Hofmann, Paula@\textsc{Beer-Hofmann, Paula}!zzzSchnitzler, Olga@\emph{von Olga Schnitzler}!1914-05-191@{{[}19. 5. 1914{]}}|)be}\briefempfaengerindex{Beer-Hofmann, Richard@\textsc{Beer-Hofmann, Richard}!zzzSchnitzler, Arthur@\emph{von Arthur Schnitzler}!1914-05-191@{{[}19. 5. 1914{]}}|)be}\briefempfaengerindex{Beer-Hofmann, Richard@\textsc{Beer-Hofmann, Richard}!zzzSchnitzler, Olga@\emph{von Olga Schnitzler}!1914-05-191@{{[}19. 5. 1914{]}}|)be}\mylabel{h}\end{ledgroupsized}  \newcommand{\dateiname}{L02562}\newcommand{\titel}{Olga und Arthur Schnitzler an Richard und Paula Beer-Hofmann, [19. 5. 1914]}\newcommand{\editorInnen}{Martin Anton Müller und Gerd-Hermann Susen}\input{../tex-inputs/latex-pdf-abspann}
      