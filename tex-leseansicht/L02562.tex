%% latex-leseansicht-vorspann.tex
%% Vorspann für die Leseansicht.
%% Lädt die gemeinsame Datei latex-vorspann.tex mit nicht gesetztem Schalter.

\newif\ifkorrekturansicht
\korrekturansichtfalse

\input{../tex-inputs/latex-vorspann}


         
         \newcommand{\erwaehntePersonen}{Personen: Richard Beer-Hofmann, Paula Beer-Hofmann}
         \newcommand{\erwaehnteInstitutionen}{Institutionen: Norddeutscher Lloyd}
         \newcommand{\erwaehnteOrte}{Orte: Amsterdam, Antwerpen, Biskaya, Hasenauerstraße, Southampton, Wien, Österreich}
         \newcommand{\erwaehnteWerke}{
               \section[Olga und Arthur Schnitzler an Richard und Paula Beer-Hofmann, {[}19. 5. 1914{]}]{ Olga und Arthur Schnitzler an Richard und Paula Beer-Hofmann,
               {[}19. 5. 1914{]}}\nopagebreak\mylabel{v}\rehead{ }\begin{ledgroupsized}[t]{13cm}\normalsize\beginnumbering \toendnotes[C]{\smallbreak\pagebreak[2]} \Standort{YCGL, MSS 31.}
\physDesc{Bildpostkarte
\newline{}Handschrift Arthur Schnitzler: Bleistift, deutsche Kurrent\newline{}Handschrift Olga Schnitzler: Bleistift, lateinische Kurrent\newline{}Versand: 1) Stempel: »\nobreak{}\textcolor{gray}{20. 5.}\nobreak{}«.   2) Der Versandweg ist unklar, da eine Briefmarke des deutschen Reiches zum Einsatz kommt, die 
                                 Kreuzfahrt aber am 20. 5. 1910 von Süden 
                                 kommend erst Southampton\oindex{Southampton@\textbf{Southampton}|pw} erreicht\newline{}Ordnung: mit Bleistift von unbekannter Hand datiert:
                              »19. 5. 14.« }\pstart{}{\pb}Austria\oindex{Oesterreich@\textbf{Österreich}|pw}\pend{}\pstart{}Herrn u. Frau\pend{}\pstart{}D\textsuperscript{r} Richard Beer-Hofmann\pend{}\pstart{}Vienna\oindex{Wien@\textbf{Wien}|pw}\pend{}\pstart{}XVIII Hasenauerstr. 59\oindex{Hasenauerstrasse@\textbf{Hasenauerstraße}|pw}.\pend{}{\bigskip}\pstart
           \noindent{}\centering{}{\pb}\textcolor{gray}{\textbf{NORDDEUTSCHER LLOYD BREMEN\orgindex{Norddeutscher Lloyd@Norddeutscher Lloyd|pw}}}{\\}\textcolor{gray}{\textbf{Reichspostdampfer »York«}}\pend
           \pstart
           \noindent{}{\pb}Herzliche Grüsse!\pend
           \pstart
           morgen hoffen wir, in Southampten\oindex{Southampton@\textbf{Southampton}|pw} zu sein,
               übermorgen Antwerpen\oindex{Antwerpen@\textbf{Antwerpen}|pw}, Freitag{ }Amsterdam\oindex{Amsterdam@\textbf{Amsterdam}|pw}.\pend
           \pstart Ihre\spacefill\mbox{OlgaS.}\pend{}\pstart
           {\pb}{[}hs. Arthur Schnitzler:{]} Herzlichſt Ihr\spacefill\mbox{Arthur}\pend
           \pstart
           Eben passiren wir ziemlich ſeitlich den Golf von Biscaya\oindex{Biskaya@\textbf{Biskaya}|pw} bei
               ruhiger See.\pend
           \pstart
           Die Fahrt heut den 7. Tag, herrlich\pend
           
         
         \endnumbering\mylabel{h}\end{ledgroupsized}  \newcommand{\dateiname}{L02562}\newcommand{\titel}{Olga und Arthur Schnitzler an Richard und Paula Beer-Hofmann, [19. 5. 1914]}\newcommand{\editorInnen}{Martin Anton Müller und Gerd-Hermann Susen}%% latex-leseansicht-abspann.tex
%% Abspann für die Leseansicht.
%% Der Schalter \ifkorrekturansicht ist bereits durch den Vorspann gesetzt.

%% latex-abspann.tex
%% Gemeinsamer Abspann für Korrekturansicht und Leseansicht.
%% Setzt den Schalter \ifkorrekturansicht voraus (gesetzt in den
%% einbindenden Dateien latex-korrekturansicht-abspann.tex bzw.
%% latex-leseansicht-abspann.tex).
%% ---------------------------------------------------------------

\normalsize

% Das esempio-Environment wird nur in der Leseansicht benötigt
\ifkorrekturansicht\else
\newenvironment{esempio}[3]%
{
    \vspace{1.5ex}
    \rlap{\underline{#1}}
    \par
    \setlength{\parindent}{0cm}
    \nopagebreak
    \leftskip=#2cm
    \rightskip=#3cm
}
{
    \par
}
\fi

\doendnotes{C}
\bigskip
\vfill

\clearpage

\footnotesize

\ifkorrekturansicht
  \lohead{\textsc{register}}
\fi

% theindex-Environment neu definieren ohne reledmac
\makeatletter
\renewenvironment{theindex}{%
  \ifkorrekturansicht
    \section*{\indexname}%
  \else
    \subsubsection*{Index der erwähnten Entitäten}%
  \fi
  \setlength{\parindent}{0pt}%
  \setlength{\parskip}{0pt plus 0.3pt}%
  \let\item\@idxitem
}{%
  \ifkorrekturansicht\clearpage\fi
}
\makeatother

\IfFileExists{\jobname-pw.ind}{\input{\jobname-pw.ind}}{}

% Quellenangabe nur in der Leseansicht
\ifkorrekturansicht\else
% Fallback-Definitionen, falls die .tex-Datei \titel etc. nicht gesetzt hat
\providecommand{\titel}{}
\providecommand{\editorInnen}{}
\providecommand{\dateiname}{\jobname}

\vspace{3cm}

\vfill

\footnotesize
\textsc{Quelle}: \titel. Herausgegeben von {\editorInnen}. In: \emph{Arthur Schnitzler: Briefwechsel mit Autorinnen und Autoren}.
 Digitale Edition, https://schnitzler-briefe.acdh.oeaw.ac.at/{\dateiname}.html (Stand \today)
\fi

\end{document}


      