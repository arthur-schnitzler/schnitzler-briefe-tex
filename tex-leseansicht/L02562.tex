%% latex-korrekturansicht-vorspann.tex
%% Vorspann für die Korrekturansicht.
%% Lädt die gemeinsame Datei latex-vorspann.tex mit gesetztem Schalter.

\newif\ifkorrekturansicht
\korrekturansichttrue

\input{../tex-inputs/latex-vorspann}


\section[Olga und Arthur Schnitzler an Richard und Paula Beer-Hofmann, {[}19. 5. 1914{]}]{L02562 Olga und Arthur Schnitzler an Richard und Paula Beer-Hofmann,
               {[}19. 5. 1914{]}}
\nopagebreak\mylabel{L02562v}
\rehead{ }\normalsize\beginnumbering\briefempfaengerindex{Beer-Hofmann, Paula@\textsc{Beer-Hofmann, Paula}!zzzSchnitzler, Arthur@\emph{von Arthur Schnitzler}!1914-05-191@{{[}19. 5. 1914{]}}|(be}\briefempfaengerindex{Beer-Hofmann, Paula@\textsc{Beer-Hofmann, Paula}!zzzSchnitzler, Olga@\emph{von Olga Schnitzler}!1914-05-191@{{[}19. 5. 1914{]}}|(be}\briefempfaengerindex{Beer-Hofmann, Richard@\textsc{Beer-Hofmann, Richard}!zzzSchnitzler, Arthur@\emph{von Arthur Schnitzler}!1914-05-191@{{[}19. 5. 1914{]}}|(be}\briefempfaengerindex{Beer-Hofmann, Richard@\textsc{Beer-Hofmann, Richard}!zzzSchnitzler, Olga@\emph{von Olga Schnitzler}!1914-05-191@{{[}19. 5. 1914{]}}|(be}
\toendnotes[C]{\smallbreak\pagebreak[2]}\Standort{YCGL, MSS 31.}
\physDesc{Bildpostkarte, 303 Zeichen
\newline{}Handschrift Arthur Schnitzler: Bleistift, deutsche Kurrent
\newline{}Handschrift Olga Schnitzler: Bleistift, lateinische Kurrent
\newline{}Versand: 1) Stempel: »\nobreak{}\textcolor{gray}{20. 5.}\nobreak{}«.   2) Der Versandweg ist unklar, da eine Briefmarke des deutschen
                                 Reiches zum Einsatz kommt, die Kreuzfahrt aber am 20. 5. 1910 von Süden kommend erst Southampton\oindex{Southampton@\textbf{Southampton}, \emph{P.PPLA2}|pw} erreicht
\newline{}Ordnung: mit Bleistift von unbekannter Hand datiert: »19. 5. 14.« }\pstart{}{\pb}Austria\oindex{Oesterreich@\textbf{Österreich}, \emph{A.PCLI}|pw}\pend{}\pstart{}Herrn u. Frau\pend{}\pstart{}D\textsuperscript{r} Richard Beer-Hofmann\pend{}\pstart{}Vienna\oindex{Wien@\textbf{Wien}, \emph{A.ADM2}|pw}\pend{}\pstart{}XVIII Hasenauerstr. 59\oindex{Hasenauerstrasse 59@\textbf{Hasenauerstraße 59}, \emph{Wohngebäude (K.WHS)}|pw}.\pend{}{\bigskip}
\pstart
           \noindent{}\centering{}{\pb}\textcolor{gray}{\textbf{NORDDEUTSCHER LLOYD BREMEN\orgindex{Norddeutscher Lloyd@Norddeutscher Lloyd|pw}}}{\\}\textcolor{gray}{\textbf{Reichspostdampfer »York«}}\pend
           \vspace{1em}
\pstart
           \noindent{}{\pb}Herzliche Grüsse!\pend
           
\pstart
           morgen hoffen wir, in Southampten\oindex{Southampton@\textbf{Southampton}, \emph{P.PPLA2}|pw} zu sein,
               übermorgen Antwerpen\oindex{Antwerpen@\textbf{Antwerpen}, \emph{A.ADM4}|pw}, Freitag{ }Amsterdam\oindex{Amsterdam@\textbf{Amsterdam}, \emph{P.PPLC}|pw}.\pend
           \pstart Ihre\spacefill\mbox{OlgaS.}\pend{}
\pstart
           {\pb}{[}hs. :{]} Herzlichſt Ihr\spacefill\mbox{Arthur}\pend
           
\pstart
           Eben passiren wir ziemlich ſeitlich den Golf von
                  Biscaya\oindex{Biskaya@\textbf{Biskaya}, \emph{Bucht (N.BCT)}|pw} bei ruhiger See.\pend
           
\pstart
           Die Fahrt heut den 7. Tag, herrlich\pend
           \selectlanguage{ngerman}\endnumbering\briefempfaengerindex{Beer-Hofmann, Paula@\textsc{Beer-Hofmann, Paula}!zzzSchnitzler, Arthur@\emph{von Arthur Schnitzler}!1914-05-191@{{[}19. 5. 1914{]}}|)be}\briefempfaengerindex{Beer-Hofmann, Paula@\textsc{Beer-Hofmann, Paula}!zzzSchnitzler, Olga@\emph{von Olga Schnitzler}!1914-05-191@{{[}19. 5. 1914{]}}|)be}\briefempfaengerindex{Beer-Hofmann, Richard@\textsc{Beer-Hofmann, Richard}!zzzSchnitzler, Arthur@\emph{von Arthur Schnitzler}!1914-05-191@{{[}19. 5. 1914{]}}|)be}\briefempfaengerindex{Beer-Hofmann, Richard@\textsc{Beer-Hofmann, Richard}!zzzSchnitzler, Olga@\emph{von Olga Schnitzler}!1914-05-191@{{[}19. 5. 1914{]}}|)be}\mylabel{L02562h}  \normalsize

\doendnotes{C}
\bigskip
\vfill

\clearpage

\footnotesize

\lohead{\textsc{register}}

% Definiere theindex-Environment komplett neu ohne reledmac
\makeatletter
\renewenvironment{theindex}{%
  \section*{\indexname}%
  \setlength{\parindent}{0pt}%
  \setlength{\parskip}{0pt plus 0.3pt}%
  \let\item\@idxitem
}{%
  \clearpage
}
\makeatother

\IfFileExists{\jobname-pw.ind}{\input{\jobname-pw.ind}}{}

\end{document}

      