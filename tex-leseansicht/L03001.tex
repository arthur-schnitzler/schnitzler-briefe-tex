%% latex-korrekturansicht-vorspann.tex
%% Vorspann für die Korrekturansicht.
%% Lädt die gemeinsame Datei latex-vorspann.tex mit gesetztem Schalter.

\newif\ifkorrekturansicht
\korrekturansichttrue

\input{../tex-inputs/latex-vorspann}


\section[ Arthur Schnitzler an Felix Salten, 20. 12. 1905]{L03001 Arthur Schnitzler an Felix Salten, 20. 12. 1905}
\nopagebreak\mylabel{L03001v}
\rehead{ }\normalsize\beginnumbering\briefempfaengerindex{Salten, Felix@\textsc{Salten, Felix}!zzzSchnitzler, Arthur@\emph{von Arthur Schnitzler}!1905-12-201@{20. 12. 1905}|(be}
\toendnotes[C]{\smallbreak\pagebreak[2]}\Standort{Wienbibliothek im Rathaus, ZPH 1681, 2.1.516.}
\physDesc{Brief, 1 Blatt, 3 Seiten, 1070 Zeichen
\newline{}Handschrift: schwarze Tinte, deutsche Kurrent
\newline{}Ordnung: mit Bleistift von unbekannter Hand Nummerierung der Doppelseiten des
                                 Konvoluts: »14«–»15« }
\buchAbdrucke{\weitereDrucke{Arthur Schnitzler: \emph{Briefe 1875–1912}. Frankfurt am Main: \emph{S. Fischer} 1981, S. 522–523.} }\toendnotes[C]{\smallbreak}
\pstart
           {\pb}\textcolor{gray}{\textbf{Dr. Arthur Schnitzler}}\hfill 20. 12. 905\pend
           
\pstart
           \textcolor{gray}{\textbf{Wien, XVIII. Spoettelgasse 7\oindex{Edmund-Weiss-Gasse 7@\textbf{Edmund-Weiß-Gasse 7}, \emph{Wohngebäude (K.WHS)}|pw}.}}\pend
           \vspace{0.5em}
\pstart
           lieber, herzlichen Dank für das \label{K_L03001-1v}\edtext{Königsbüchel\pwindex{Buch der Koenige@\emph{Das Buch der Könige}|pw}}{\lemma{\textnormal{\emph{Königsbüchel}}}\Cendnote{\textnormal{Siehe Felix Salten: Widmungsexemplar Das Buch der Könige für Arthur
               Schnitzler, [zwischen 1. und 20. 12.] 1905.
               }}}\label{K_L03001-1}, deſſen Köſtlich- u Koſtbarkeiten wiederzugenießen ich mich ſchon ſehr
               freue.\pend
           
\pstart
           Ferner: eine Anzahl ſogenannter \label{K_L03001-2v}\edtext{Aphorismen\pwindex{Bemerkungen@\emph{Bemerkungen}|pwv}}{\lemma{\textnormal{\emph{Aphorismen}}}\Cendnote{\textnormal{Arthur Schnitzler: \emph{Bemerkungen}\pwindex{Bemerkungen@\emph{Bemerkungen}|pwk}. In: \emph{Österreichische Rundschau}\pwindex{Oesterreichische Rundschau@\emph{Österreichische Rundschau}|pwk}. Bd. 5, Nr. 60/61, 21. 12. 1905, S. 395–396.}}}\label{K_L03001-2} lag ſchon für die
               Weihnachtszeit bereit – da kam ein wahrer Brandbrief von \textsc{Glossy\pwindex{Glossy, Karl 07.03.1848 – 09.09.1937@\textsc{Glossy, Karl} (07.03.1848 – 09.09.1937), \emph{Schriftsteller/Schriftstellerin, Museumsleiter/Museumsleiterin, Zensurbeirat/Zensurbeirätin}|pw}} (der mich ſchon ſeit Gründg der Oe. Rdſch.\orgindex{Oesterreichische Rundschau@Österreichische Rundschau|pw}
               heftig um Beiträge angeht und der (wörtlich) »vor Aufregung phyſiſch {\pb}erkrankt ſei, durch meine neuerliche
               Absage–«) – nun und ich ſandte ihm die paar Nichtigkeiten\pwindex{Bemerkungen@\emph{Bemerkungen}|pwv}, in der angenehmen Gewißheit, daſs \textsc{Singer\pwindex{Singer, Isidor 16.01.1857 – 08.12.1927@\textsc{Singer, Isidor} (16.01.1857 – 08.12.1927), \emph{Journalist/Journalistin, Herausgeber/Herausgeberin, Soziologe/Soziologin}|pw}} und \textsc{Kanners\pwindex{Kanner, Heinrich 09.11.1864 – 15.02.1930@\textsc{Kanner, Heinrich} (09.11.1864 – 15.02.1930), \emph{Herausgeber/Herausgeberin, Publizist/Publizistin}|pw}} Geſundheit durch mein
               Fernbleiben unerſchüttert bleiben. (Und nun hab ich wieder einmal die feſte Abſicht,
               mit nichts mehr in die Oeffentlichkeit zu ko{\geminationm}en, eh ich
               wieder was ganz ordentliches herausgebracht habe.)\pend
           
\pstart
           Drittens. \label{K_L03001-3v}\edtext{Morgen Donnerſtag}{\lemma{\textnormal{\emph{Morgen Donnerſtag}}}\Cendnote{\textnormal{Arthur und Olga Schnitzler\pwindex{Schnitzler, Olga 17.01.1882 – 13.01.1970@\textsc{Schnitzler, Olga} (17.01.1882 – 13.01.1970), \emph{Schauspieler/Schauspielerin, Sänger/Sängerin}|pwk} sahen sich \emph{Immer
                     modern}\pwindex{Immer modern. Sechs Szenen aus dem modernen Pariser Leben@\emph{Immer modern. Sechs Szenen aus dem modernen Pariser Leben}|pwk} von Henri Léon Lavedan an, vgl. A. S.: \emph{Tagebuch}, 21. 12. 1905. Ein anschließender Besuch im Riedhof\oindex{Riedhof@\textbf{Riedhof}, \emph{Lokal (K.LKL)}|pwk} ist nicht belegt. Auch ein
                  Zusammentreffen mit Salten\pwindex{Salten, Felix 06.09.1869 – 08.10.1945@\textsc{Salten, Felix} (06.09.1869 – 08.10.1945), \emph{Schriftsteller/Schriftstellerin, Journalist/Journalistin, Chefredakteur/Chefredakteurin}|pwk} ist nicht
                  nachweisbar.}}}\label{K_L03001-3} gehn {\pb}wir\pwindex{Schnitzler, Olga 17.01.1882 – 13.01.1970@\textsc{Schnitzler, Olga} (17.01.1882 – 13.01.1970), \emph{Schauspieler/Schauspielerin, Sänger/Sängerin}|pwv} ins Joſefſtädter Theater\oindex{Theater in der Josefstadt@\textbf{Theater in der Josefstadt}, \emph{Theater (K.THE)}|pw}, und wären ſehr erfreut, nachher (im Riedhof\oindex{Riedhof@\textbf{Riedhof}, \emph{Lokal (K.LKL)}|pw} wie u wo neulich) mit Ihnen beiden\pwindex{Salten, Ottilie 07.03.1868 – 22.06.1942@\textsc{Salten, Ottilie} (07.03.1868 – 22.06.1942), \emph{Schauspieler/Schauspielerin}|pwv} zuſa{\geminationm}entreffen zu können. Und we{\geminationn} Sie verhindert ſind, geben Sie ein andres Rendevous oder ko{\geminationm}en zu uns. \label{K_L03001-4v}\edtext{Mittwoch}{\lemma{\textnormal{\emph{Mittwoch}}}\Cendnote{\textnormal{Siehe A. S.: \emph{Tagebuch}, 27. 12. 1905. Salten\pwindex{Salten, Felix 06.09.1869 – 08.10.1945@\textsc{Salten, Felix} (06.09.1869 – 08.10.1945), \emph{Schriftsteller/Schriftstellerin, Journalist/Journalistin, Chefredakteur/Chefredakteurin}|pwk} war nicht bei der privaten Lesung, bei
                  der Jakob Wassermann\pwindex{Wassermann, Jakob 10.03.1873 – 01.01.1934@\textsc{Wassermann, Jakob} (10.03.1873 – 01.01.1934), \emph{Schriftsteller/Schriftstellerin}|pwk} seine Novelle \emph{Clarissa Mirabel}\pwindex{Clarissa Mirabel@\emph{Clarissa Mirabel}|pwk} vortrug.}}}\label{K_L03001-4} ſind Sie wohl
               auch zur \textsc{Wasserm\pwindex{Wassermann, Jakob 10.03.1873 – 01.01.1934@\textsc{Wassermann, Jakob} (10.03.1873 – 01.01.1934), \emph{Schriftsteller/Schriftstellerin}|pw}}. Vorleſung\pwindex{Clarissa Mirabel@\emph{Clarissa Mirabel}|pwv} geladen? Und
               am \label{K_L03001-5v}\edtext{\textsc{Se{\geminationm}ering\oindex{Semmering@\textbf{Semmering}, \emph{A.ADM3}|pw}}, Jänner}{\lemma{\textnormal{\emph{Semmering, Jänner}}}\Cendnote{\textnormal{Schnitzler fuhr selbst erst wieder im Herbst 1906 auf den
                     Semmering\oindex{Semmering@\textbf{Semmering}, \emph{A.ADM3}|pwk}.}}}\label{K_L03001-5}, halten wir doch
               feſt?\pend
           
\pstart
           Herzlichſt Ihr {\\[\baselineskip]}\spacefill\mbox{A.}\pend
           \leftskip=0em{}\selectlanguage{ngerman}\endnumbering\briefempfaengerindex{Salten, Felix@\textsc{Salten, Felix}!zzzSchnitzler, Arthur@\emph{von Arthur Schnitzler}!1905-12-201@{20. 12. 1905}|)be}\mylabel{L03001h}  \normalsize

\doendnotes{C}
\bigskip
\vfill

\clearpage

\footnotesize

\lohead{\textsc{register}}

% Definiere theindex-Environment komplett neu ohne reledmac
\makeatletter
\renewenvironment{theindex}{%
  \section*{\indexname}%
  \setlength{\parindent}{0pt}%
  \setlength{\parskip}{0pt plus 0.3pt}%
  \let\item\@idxitem
}{%
  \clearpage
}
\makeatother

\IfFileExists{\jobname-pw.ind}{\input{\jobname-pw.ind}}{}

\end{document}

      