%% latex-leseansicht-vorspann.tex
%% Vorspann für die Leseansicht.
%% Lädt die gemeinsame Datei latex-vorspann.tex mit nicht gesetztem Schalter.

\newif\ifkorrekturansicht
\korrekturansichtfalse

\input{../tex-inputs/latex-vorspann}


\section[ Arthur Schnitzler an Felix Salten, 20. 12. 1905]{L03001 Arthur Schnitzler an Felix Salten,  20. 12. 1905}
\nopagebreak\mylabel{L03001v}
\rehead{ }\normalsize\beginnumbering\briefempfaengerindex{Salten, Felix@\textsc{Salten, Felix}!zzzSchnitzler, Arthur@\emph{von Arthur Schnitzler}!1905-12-201@{20. 12. 1905}|(be}
\toendnotes[C]{\smallbreak\pagebreak[2]}
\correspDesc{Versand  durch Arthur Schnitzler am 20. 12. 1905 in Wien
\newline{}Erhalt  durch Felix Salten im Zeitraum [20. 12. 1905 – 23. 12. 1905?] in Wien}\toendnotes[C]{\smallbreak}
\Standort{Wienbibliothek im Rathaus, ZPH 1681, 2.1.516.}
\physDesc{Brief, 1 Blatt, 3 Seiten, 1070 Zeichen
\newline{}Handschrift: schwarze Tinte, deutsche Kurrent
\newline{}Ordnung: mit Bleistift von unbekannter Hand Nummerierung der Doppelseiten des
                                 Konvoluts: »14«–»15« }
\buchAbdrucke{\weitereDrucke{1) Arthur Schnitzler: \emph{Vom jungen Herzl. (Ein Brief aus dem Jahre 1892).} In: \emph{Jüdischer Almanach 5670}. Herausgegeben aus Anlass des 25-semestrigen Jubiläums von der (1910), S. 102–103.} \weitereDrucke{2) \pwindex{Kellner, Leon 17.\,4.\,1859 Tarnów – 5.\,12.\,1928 Wien@\textsc{Kellner, Leon} (17.\,4.\,1859 Tarnów – 5.\,12.\,1928 Wien), \emph{Zionist, Literaturhistoriker, Anglist}!Theodor Herzls Lehrjahre (1860–1895). Nach den handschriftlichen Quellen@\strich\emph{Theodor Herzls Lehrjahre (1860–1895). Nach den handschriftlichen Quellen}|pwk}Leon Kellner: \emph{Theodor Herzls Lehrjahre. 1860–1895. Nach den
                        handschriftlichen Quellen}. Wien, Berlin: \emph{R. Löwit} 1920, S. 108–111.} \weitereDrucke{3) \pwindex{Neues Wiener Journal@\emph{Neues Wiener Journal}|pwk}h. m. [= Hermann Menkes]: \emph{Briefwechsel zwischen Theodor Herzl und Artur Schnitzler. Lehrjahre des berühmten Zionistenführers.} In: \emph{Neues Wiener Journal}, Jg. 28, Nr. 9540, 29.\,5.\,1920, S. 3–4.} \weitereDrucke{4) Arthur Schnitzler: \emph{Briefe 1875–1912}. Herausgegeben von Therese Nickl und Heinrich Schnitzler. Frankfurt am Main: \emph{S. Fischer} 1981, S. 522–523.} \weitereDrucke{5) Arthur Schnitzler: \emph{»Das Zeitlose ist von kürzester Dauer«. Interviews, Meinungen, Proteste}. Göttingen: \emph{Wallstein} 2023 \url{https://schnitzler-interviews.acdh.oeaw.ac.at/M170.html}, S. 470–472.} }\toendnotes[C]{\smallbreak}
\pstart
           {\pb}\textcolor{gray}{\textbf{Dr. Arthur Schnitzler}}\hfill 20. 12. 905\pend
           
\pstart
           \textcolor{gray}{\textbf{Wien, XVIII. Spoettelgasse 7\oindex{Wien@\textbf{Wien}!XVIII., Währing@\textbf{XVIII., Währing}!Edmund-Weiß-Gasse 7@\textbf{Edmund-Weiß-Gasse 7}, \emph{Wohngebäude}|pw}.}}\pend
           \vspace{0.5em}
\pstart
           lieber, herzlichen Dank für das \label{K_L03001-1v}\edtext{Königsbüchel\pwindex{Salten, Felix 6.\,9.\,1869 Budapest – 8.\,10.\,1945 Zürich@\textsc{Salten, Felix} (6.\,9.\,1869 Budapest – 8.\,10.\,1945 Zürich), \emph{Schriftsteller, Journalist, Chefredakteur}!Buch der Könige@\strich\emph{Das Buch der Könige}|pw}}{\lemma{\textnormal{\emph{Königsbüchel}}}\Cendnote{\textnormal{Siehe XXXX Auszeichnungsfehler: Dokument L03050 nicht gefunden. }}}\label{K_L03001-1}, deſſen
               Köſtlich- u Koſtbarkeiten wiederzugenießen ich mich{ }ſchon{ }ſehr freue.\pend
           
\pstart
           Ferner: eine Anzahl{ }ſogenannter \label{K_L03001-2v}\edtext{Aphorismen\pwindex{Schnitzler, Arthur 15.\,5.\,1862 Wien – 21.\,10.\,1931 ebd.@\textsc{Schnitzler, Arthur} (15.\,5.\,1862 Wien – 21.\,10.\,1931 ebd.), \emph{Schriftsteller, Mediziner}!Bemerkungen@\strich\emph{Bemerkungen}|pwv}}{\lemma{\textnormal{\emph{Aphorismen}}}\Cendnote{\textnormal{Arthur Schnitzler: \emph{Bemerkungen}\pwindex{Schnitzler, Arthur 15.\,5.\,1862 Wien – 21.\,10.\,1931 ebd.@\textsc{Schnitzler, Arthur} (15.\,5.\,1862 Wien – 21.\,10.\,1931 ebd.), \emph{Schriftsteller, Mediziner}!Bemerkungen@\strich\emph{Bemerkungen}|pwk}. In: \emph{Österreichische Rundschau}\pwindex{Österreichische Rundschau@\emph{Österreichische Rundschau}|pwk}. Bd. 5, Nr. 60/61, 21. 12. 1905, S. 395–396.}}}\label{K_L03001-2} lag{ }ſchon für die
               Weihnachtszeit bereit – da kam ein wahrer Brandbrief von \textsc{Glossy\pwindex{Glossy, Karl 7.\,3.\,1848 Wien – 9.\,9.\,1937 ebd.@\textsc{Glossy, Karl} (7.\,3.\,1848 Wien – 9.\,9.\,1937 ebd.), \emph{Schriftsteller, Museumsleiter, Zensurbeirat}|pw}} (der mich{ }ſchon{ }ſeit Gründg der Oe. Rdſch.\orgindex{Österreichische Rundschau@Österreichische Rundschau|pw}
               heftig um Beiträge angeht und der (wörtlich) »vor Aufregung phyſiſch {\pb}erkrankt{ }ſei, durch meine neuerliche
               Absage–«) – nun und ich{ }ſandte ihm die paar Nichtigkeiten\pwindex{Schnitzler, Arthur 15.\,5.\,1862 Wien – 21.\,10.\,1931 ebd.@\textsc{Schnitzler, Arthur} (15.\,5.\,1862 Wien – 21.\,10.\,1931 ebd.), \emph{Schriftsteller, Mediziner}!Bemerkungen@\strich\emph{Bemerkungen}|pwv}, in der angenehmen Gewißheit, daſs \textsc{Singer\pwindex{Singer, Isidor 16.\,1.\,1857 Budapest – 8.\,12.\,1927 Wien@\textsc{Singer, Isidor} (16.\,1.\,1857 Budapest – 8.\,12.\,1927 Wien), \emph{Journalist, Herausgeber, Soziologe}|pw}} und \textsc{Kanners\pwindex{Kanner, Heinrich 9.\,11.\,1864 Galați – 15.\,2.\,1930 Wien@\textsc{Kanner, Heinrich} (9.\,11.\,1864 Galați – 15.\,2.\,1930 Wien), \emph{Herausgeber, Publizist}|pw}} Geſundheit durch mein Fernbleiben unerſchüttert bleiben. (Und nun hab ich
               wieder einmal die feſte Abſicht, mit nichts mehr in die Oeffentlichkeit zu ko{\geminationm}en, eh ich wieder was ganz ordentliches herausgebracht
               habe.)\pend
           
\pstart
           Drittens. \label{K_L03001-3v}\edtext{Morgen Donnerſtag}{\lemma{\textnormal{\emph{Morgen Donnerstag}}}\Cendnote{\textnormal{Arthur und Olga Schnitzler\pwindex{Schnitzler, Olga 17.\,1.\,1882 Wien – 13.\,1.\,1970 Lugano@\textsc{Schnitzler, Olga} (17.\,1.\,1882 Wien – 13.\,1.\,1970 Lugano), \emph{Schauspielerin, Sängerin}|pwk} sahen sich \emph{Immer
                     modern}\pwindex{\textcolor{red}{\textsuperscript{XXXX indx1}}!Immer modern. Sechs Szenen aus dem modernen Pariser Leben@\strich\emph{Immer modern. Sechs Szenen aus dem modernen Pariser Leben}|pwk} von Henri Léon Lavedan an, vgl. A. S.: \emph{Tagebuch}, 21. 12. 1905. Ein anschließender Besuch im Riedhof\oindex{Wien@\textbf{Wien}!VIII., Josefstadt@\textbf{VIII., Josefstadt}!Riedhof@\textbf{Riedhof}, \emph{Lokal}|pwk} ist nicht belegt. Auch ein
                  Zusammentreffen mit Salten\pwindex{Salten, Felix 6.\,9.\,1869 Budapest – 8.\,10.\,1945 Zürich@\textsc{Salten, Felix} (6.\,9.\,1869 Budapest – 8.\,10.\,1945 Zürich), \emph{Schriftsteller, Journalist, Chefredakteur}|pwk} ist nicht
                  nachweisbar.}}}\label{K_L03001-3} gehn {\pb}wir\pwindex{Schnitzler, Olga 17.\,1.\,1882 Wien – 13.\,1.\,1970 Lugano@\textsc{Schnitzler, Olga} (17.\,1.\,1882 Wien – 13.\,1.\,1970 Lugano), \emph{Schauspielerin, Sängerin}|pwv} ins Joſefſtädter Theater\oindex{Wien@\textbf{Wien}!VIII., Josefstadt@\textbf{VIII., Josefstadt}!Theater in der Josefstadt@\textbf{Theater in der Josefstadt}, \emph{Theater}|pw}, und wären{ }ſehr erfreut, nachher (im Riedhof\oindex{Wien@\textbf{Wien}!VIII., Josefstadt@\textbf{VIII., Josefstadt}!Riedhof@\textbf{Riedhof}, \emph{Lokal}|pw} wie u wo neulich) mit Ihnen beiden\pwindex{Salten, Ottilie 7.\,3.\,1868 Prag – 22.\,6.\,1942 Zürich@\textsc{Salten, Ottilie} (7.\,3.\,1868 Prag – 22.\,6.\,1942 Zürich), \emph{Schauspielerin}|pwv} zuſa{\geminationm}entreffen zu können. Und we{\geminationn} Sie verhindert{ }ſind, geben Sie ein andres Rendevous oder ko{\geminationm}en zu uns. \label{K_L03001-4v}\edtext{Mittwoch}{\lemma{\textnormal{\emph{Mittwoch}}}\Cendnote{\textnormal{Siehe A. S.: \emph{Tagebuch}, 27. 12. 1905. Salten\pwindex{Salten, Felix 6.\,9.\,1869 Budapest – 8.\,10.\,1945 Zürich@\textsc{Salten, Felix} (6.\,9.\,1869 Budapest – 8.\,10.\,1945 Zürich), \emph{Schriftsteller, Journalist, Chefredakteur}|pwk} war nicht bei der privaten Lesung, bei
                  der Jakob Wassermann\pwindex{Wassermann, Jakob 10.\,3.\,1873 Fürth – 1.\,1.\,1934 Altaussee@\textsc{Wassermann, Jakob} (10.\,3.\,1873 Fürth – 1.\,1.\,1934 Altaussee), \emph{Schriftsteller}|pwk} seine Novelle \emph{Clarissa Mirabel}\pwindex{Wassermann, Jakob 10.\,3.\,1873 Fürth – 1.\,1.\,1934 Altaussee@\textsc{Wassermann, Jakob} (10.\,3.\,1873 Fürth – 1.\,1.\,1934 Altaussee), \emph{Schriftsteller}!Clarissa Mirabel@\strich\emph{Clarissa Mirabel}|pwk} vortrug.}}}\label{K_L03001-4}{ }ſind Sie wohl
               auch zur \textsc{Wasserm\pwindex{Wassermann, Jakob 10.\,3.\,1873 Fürth – 1.\,1.\,1934 Altaussee@\textsc{Wassermann, Jakob} (10.\,3.\,1873 Fürth – 1.\,1.\,1934 Altaussee), \emph{Schriftsteller}|pw}}. Vorleſung\pwindex{Wassermann, Jakob 10.\,3.\,1873 Fürth – 1.\,1.\,1934 Altaussee@\textsc{Wassermann, Jakob} (10.\,3.\,1873 Fürth – 1.\,1.\,1934 Altaussee), \emph{Schriftsteller}!Clarissa Mirabel@\strich\emph{Clarissa Mirabel}|pwv} geladen? Und
               am \label{K_L03001-5v}\edtext{\textsc{Se{\geminationm}ering\oindex{Semmering@\textbf{Semmering}, \emph{Verwaltungsgebiet}|pw}}, Jänner}{\lemma{\textnormal{\emph{Semmering, Jänner}}}\Cendnote{\textnormal{Schnitzler fuhr selbst erst wieder im
                     Herbst 1906 auf den Semmering\oindex{Semmering@\textbf{Semmering}, \emph{Verwaltungsgebiet}|pwk}.}}}\label{K_L03001-5}, halten wir doch feſt?\pend
           
\pstart
           Herzlichſt Ihr {\\[\baselineskip]}\spacefill\mbox{A.}\pend
           \leftskip=0em{}\selectlanguage{ngerman}\endnumbering\briefempfaengerindex{Salten, Felix@\textsc{Salten, Felix}!zzzSchnitzler, Arthur@\emph{von Arthur Schnitzler}!1905-12-201@{20. 12. 1905}|)be}\mylabel{L03001h}  \newcommand{\dateiname}{L03001}\newcommand{\titel}{Arthur Schnitzler an Felix Salten, 20. 12. 1905}\newcommand{\editorInnen}{Martin Anton Müller und Laura Untner}%% latex-leseansicht-abspann.tex
%% Abspann für die Leseansicht.
%% Der Schalter \ifkorrekturansicht ist bereits durch den Vorspann gesetzt.

%% latex-abspann.tex
%% Gemeinsamer Abspann für Korrekturansicht und Leseansicht.
%% Setzt den Schalter \ifkorrekturansicht voraus (gesetzt in den
%% einbindenden Dateien latex-korrekturansicht-abspann.tex bzw.
%% latex-leseansicht-abspann.tex).
%% ---------------------------------------------------------------

\normalsize

% Das esempio-Environment wird nur in der Leseansicht benötigt
\ifkorrekturansicht\else
\newenvironment{esempio}[3]%
{
    \vspace{1.5ex}
    \rlap{\underline{#1}}
    \par
    \setlength{\parindent}{0cm}
    \nopagebreak
    \leftskip=#2cm
    \rightskip=#3cm
}
{
    \par
}
\fi

\doendnotes{C}
\bigskip
\vfill

\clearpage

\footnotesize

\ifkorrekturansicht
  \lohead{\textsc{register}}
\fi

% theindex-Environment neu definieren ohne reledmac
\makeatletter
\renewenvironment{theindex}{%
  \ifkorrekturansicht
    \section*{\indexname}%
  \else
    \subsubsection*{Index der erwähnten Entitäten}%
  \fi
  \setlength{\parindent}{0pt}%
  \setlength{\parskip}{0pt plus 0.3pt}%
  \let\item\@idxitem
}{%
  \ifkorrekturansicht\clearpage\fi
}
\makeatother

\IfFileExists{\jobname-pw.ind}{\input{\jobname-pw.ind}}{}

% Quellenangabe nur in der Leseansicht
\ifkorrekturansicht\else
% Fallback-Definitionen, falls die .tex-Datei \titel etc. nicht gesetzt hat
\providecommand{\titel}{}
\providecommand{\editorInnen}{}
\providecommand{\dateiname}{\jobname}

\vspace{3cm}

\vfill

\footnotesize
\textsc{Quelle}: \titel. Herausgegeben von {\editorInnen}. In: \emph{Arthur Schnitzler: Briefwechsel mit Autorinnen und Autoren}.
 Digitale Edition, https://schnitzler-briefe.acdh.oeaw.ac.at/{\dateiname}.html (Stand \today)
\fi

\end{document}


