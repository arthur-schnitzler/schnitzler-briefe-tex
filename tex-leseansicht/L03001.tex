%% latex-leseansicht-vorspann.tex
%% Vorspann für die Leseansicht.
%% Lädt die gemeinsame Datei latex-vorspann.tex mit nicht gesetztem Schalter.

\newif\ifkorrekturansicht
\korrekturansichtfalse

\input{../tex-inputs/latex-vorspann}

\begin{center}
            \textcolor{red}{ENTWURF, NICHT FERTIG KORRIGIERT}
                      \end{center}
            
         
         \newcommand{\erwaehntePersonen}{Personen: Karl Glossy, Heinrich Kanner, Felix Salten, Isidor Singer, Jakob Wassermann}
         \newcommand{\erwaehnteInstitutionen}{Institutionen: Österreichische Rundschau}
         \newcommand{\erwaehnteOrte}{Orte: Riedhof, Semmering, Theater in der Josefstadt, Wien}
         \newcommand{\erwaehnteWerke}{
               \section[Arthur Schnitzler an Felix Salten, 20. 12. 1905]{ Arthur Schnitzler an Felix Salten, 20. 12. 1905}\nopagebreak\mylabel{v}\rehead{ }\begin{ledgroupsized}[t]{13cm}\normalsize\beginnumbering \toendnotes[C]{\smallbreak\pagebreak[2]} \Standort{Wienbibliothek im Rathaus, ZPH 1681, 2.1.516.}
\physDesc{
\newline{}Handschrift: , deutsche Kurrent}\toendnotes[C]{\smallbreak}\pstart
           \noindent{}{\pb}\textcolor{gray}{\textbf{Dr. Arthur Schnitzler}}\hfill 20. 12. 905\pend
           \pstart
           \textcolor{gray}{\textbf{Wien, XVIII.
                        Spoettelgasse 7\oindex{XXXX Ortsangabe fehlt|pw}.}}\pend
           \pstart
           lieber, herzlichen Dank für das Königsbüchel\textcolor{red}{\textsuperscript{\textbf{KEY}}}, deſſen Köſtlich- u Koſtbarkeiten wiederzugenießen ich mich
               ſchon ſehr freue. \pend
           \pstart
           Ferner: eine Anzahl ſogenannter Aphorismen\textcolor{red}{\textsuperscript{\textbf{KEY}}} lag ſchon für
               die Weihnachtszeit bereit – da kam ein wahrer Brandbrief von \textsc{Glossy\pwindex{Glossy, Karl 07.03.1848 – 09.09.1937@\textsc{Glossy, Karl} (07.03.1848 – 09.09.1937), \emph{Schriftsteller, Museumsleiter, Zensurbeirat}|pw}} (der mich ſchon ſeit Gründg der Oe. Rdſch.\orgindex{Oesterreichische Rundschau@Österreichische Rundschau|pw}
               heftig um Beiträge angeht aus der (wörtlich) »vor Aufregung phyſiſch {\pb}erkrankt ſei, durch meine neuerliche
               Absage–«) – nun und ich ſandte ihm die par Nichtigkeiten, in der angenehmen
               Gewißheit, daſs \textsc{Singer\pwindex{Singer, Isidor 16.01.1857 – 08.12.1927@\textsc{Singer, Isidor} (16.01.1857 – 08.12.1927), \emph{Journalist, Herausgeber, Soziologe}|pw}} und \textsc{Kanner\pwindex{Kanner, Heinrich 09.11.1864 – 15.02.1930@\textsc{Kanner, Heinrich} (09.11.1864 – 15.02.1930), \emph{Herausgeber, Publizist}|pw}s } Geſundheit durch mein
               Fernbleiben unerſchüttert bleiben. (Und nun hab ich wieder einmal die feſte Abſicht,
               mit nichts mehr in die Oeffentlichkeit zu ko{\geminationm}en, eh ich
               wieder was ganz ordentliches herausgebracht habe.) \pend
           \pstart
           Drittens. Morgen Donnerſtag gehn {\pb}wir ins Joſefſtädter Theater\oindex{Theater in der Josefstadt@\textbf{Theater in der Josefstadt}|pw}, und wären ſehr
               erfreut, nachher (im Riedhof\oindex{Riedhof@\textbf{Riedhof}|pw} u wo neulich) mit
               Ihnen zuſa{\geminationm}entreffen zu können. Und we{\geminationn} Sie verhindert ſind, geben Sie ein andres Rendevous
               oder ko{\geminationm}en zu uns. Mittwoch ſind Sie wohl auch zur \textsc{Wasserm\pwindex{Wassermann, Jakob 10.03.1873 – 01.01.1934@\textsc{Wassermann, Jakob} (10.03.1873 – 01.01.1934), \emph{Schriftsteller}|pw}} Vorleſung geladen? Und am \textsc{Se{\geminationm}ering\oindex{Semmering@\textbf{Semmering}|pw}}, Jänner, halten wir doch feſt? \pend
           \pstart
           Herzlichſt Ihr {\\[\baselineskip]}\spacefill\mbox{A.}\pend
           \leftskip=0em{}
         
         \endnumbering\mylabel{h}\end{ledgroupsized}\begin{anhang}\end{anhang}\newcommand{\dateiname}{L03001}\newcommand{\titel}{Arthur Schnitzler an Felix Salten, 20. 12. 1905}\newcommand{\editorInnen}{Martin Anton Müller und Laura Untner}%% latex-leseansicht-abspann.tex
%% Abspann für die Leseansicht.
%% Der Schalter \ifkorrekturansicht ist bereits durch den Vorspann gesetzt.

%% latex-abspann.tex
%% Gemeinsamer Abspann für Korrekturansicht und Leseansicht.
%% Setzt den Schalter \ifkorrekturansicht voraus (gesetzt in den
%% einbindenden Dateien latex-korrekturansicht-abspann.tex bzw.
%% latex-leseansicht-abspann.tex).
%% ---------------------------------------------------------------

\normalsize

% Das esempio-Environment wird nur in der Leseansicht benötigt
\ifkorrekturansicht\else
\newenvironment{esempio}[3]%
{
    \vspace{1.5ex}
    \rlap{\underline{#1}}
    \par
    \setlength{\parindent}{0cm}
    \nopagebreak
    \leftskip=#2cm
    \rightskip=#3cm
}
{
    \par
}
\fi

\doendnotes{C}
\bigskip
\vfill

\clearpage

\footnotesize

\ifkorrekturansicht
  \lohead{\textsc{register}}
\fi

% theindex-Environment neu definieren ohne reledmac
\makeatletter
\renewenvironment{theindex}{%
  \ifkorrekturansicht
    \section*{\indexname}%
  \else
    \subsubsection*{Index der erwähnten Entitäten}%
  \fi
  \setlength{\parindent}{0pt}%
  \setlength{\parskip}{0pt plus 0.3pt}%
  \let\item\@idxitem
}{%
  \ifkorrekturansicht\clearpage\fi
}
\makeatother

\IfFileExists{\jobname-pw.ind}{\input{\jobname-pw.ind}}{}

% Quellenangabe nur in der Leseansicht
\ifkorrekturansicht\else
% Fallback-Definitionen, falls die .tex-Datei \titel etc. nicht gesetzt hat
\providecommand{\titel}{}
\providecommand{\editorInnen}{}
\providecommand{\dateiname}{\jobname}

\vspace{3cm}

\vfill

\footnotesize
\textsc{Quelle}: \titel. Herausgegeben von {\editorInnen}. In: \emph{Arthur Schnitzler: Briefwechsel mit Autorinnen und Autoren}.
 Digitale Edition, https://schnitzler-briefe.acdh.oeaw.ac.at/{\dateiname}.html (Stand \today)
\fi

\end{document}


      