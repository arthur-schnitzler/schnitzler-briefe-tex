%% latex-leseansicht-vorspann.tex
%% Vorspann für die Leseansicht.
%% Lädt die gemeinsame Datei latex-vorspann.tex mit nicht gesetztem Schalter.

\newif\ifkorrekturansicht
\korrekturansichtfalse

\input{../tex-inputs/latex-vorspann}


         
         \renewcommand{\erwaehntePersonen}{Personen: Josef Hoffmann, Rudolf Huber-Wiesenthal, Koloman Moser, Olga Schnitzler, Grethe Wiesenthal, Elsa Wiesenthal, Berta Wiesenthal, Friedrich Wärndorfer}
         \renewcommand{\erwaehnteOrte}{Orte: Taubstummengasse, Wien}
         \renewcommand{\erwaehnteWerke}{}
               \section[Max Mell an Arthur Schnitzler, 4. 11. 1907]{ Max Mell an Arthur Schnitzler, 4. 11. 1907}\nopagebreak\mylabel{v}\rehead{ }\begin{ledgroupsized}[t]{13cm}\normalsize\beginnumbering \toendnotes[C]{\smallbreak\pagebreak[2]} \Standort{DLA, A:Schnitzler, HS.NZ85.1.4055, S. [6].}
\physDesc{Brief, Maschinenschriftliche Abschrift, 1 Blatt, 1 Seite
\newline{}Schreibmaschine}\toendnotes[C]{\smallbreak}\pstart
           \raggedleft{}{\pb}4. November 1907.\pend
           \pstart\center{}Verehrter Herr Doktor,\pend\pstart
           die Tänzerinnen Schwestern Wiesenthal\pwindex{Wiesenthal, Grethe 09.12.1885 – 24.06.1970@\textsc{Wiesenthal, Grethe} (09.12.1885 – 24.06.1970), \emph{Tänzerin}|pw}\pwindex{Wiesenthal, Elsa 1887-06-22 – 1967-02-20@\textsc{Wiesenthal, Elsa} (1887-06-22 – 1967-02-20), \emph{Tänzerin}|pw}\pwindex{Wiesenthal, Berta 1892-08-10 – 1953-10-02@\textsc{Wiesenthal, Berta} (1892-08-10 – 1953-10-02), \emph{Tänzerin}|pw} veranstalten am \label{K_L01728-1v}\edtext{Mittwoch}{\lemma{\textnormal{\emph{Mittwoch}}}\Cendnote{\textnormal{Schnitzler\pwindex{Schnitzler, Arthur 15.05.1862 – 21.10.1931@\textsc{Schnitzler, Arthur} (15.05.1862 – 21.10.1931), \emph{Schriftsteller, Mediziner}|pwk} nahm die
                        Einladung nicht an, er war am 6. 11. 1907 auf einer Verbandssitzung.}}}\label{K_L01728-1h} einen
                    Tanzabend – sollte es Sie und Ihre Frau Gemahlin\pwindex{Schnitzler, Olga 17.01.1882 – 13.01.1970@\textsc{Schnitzler, Olga} (17.01.1882 – 13.01.1970), \emph{Schauspielerin, Sängerin}|pwv} interessieren, so kommen Sie doch bitte dazu! Es
                    findet im Atelier des Malers Huber\pwindex{Huber-Wiesenthal, Rudolf 1884-10-17 – 1983-10-16@\textsc{Huber-Wiesenthal, Rudolf} (1884-10-17 – 1983-10-16), \emph{Maler}|pw}, IV. Taubstummengasse 2\oindex{Taubstummengasse@\textbf{Taubstummengasse}|pw}, statt, um ½ 8
                        abends, und es werden ausser mir nur noch Kolo Moser\pwindex{Moser, Koloman 1868-03-30 – 1918-10-18@\textsc{Moser, Koloman} (1868-03-30 – 1918-10-18), \emph{Maler, Grafiker}|pw} und Josef
                        Hoffmann\pwindex{Hoffmann, Josef 22.07.1831 – 31.01.1904@\textsc{Hoffmann, Josef} (22.07.1831 – 31.01.1904), \emph{Maler}|pw} dort sein, allenfalls Waerndorfer\pwindex{Waerndorfer, Friedrich 05.05.1868 – 09.08.1939@\textsc{Wärndorfer, Friedrich} (05.05.1868 – 09.08.1939), \emph{Industrieller}|pw}. Die Wiesenthals\pwindex{Wiesenthal, Grethe 09.12.1885 – 24.06.1970@\textsc{Wiesenthal, Grethe} (09.12.1885 – 24.06.1970), \emph{Tänzerin}|pw}\pwindex{Wiesenthal, Elsa 1887-06-22 – 1967-02-20@\textsc{Wiesenthal, Elsa} (1887-06-22 – 1967-02-20), \emph{Tänzerin}|pw}\pwindex{Wiesenthal, Berta 1892-08-10 – 1953-10-02@\textsc{Wiesenthal, Berta} (1892-08-10 – 1953-10-02), \emph{Tänzerin}|pw} wären über Ihr Kommen sehr erfreut, ich wurde gebeten, Sie
                    zu benachrichtigen.\pend
           \pstart
           Mit vielen Empfehlungen{\\[\baselineskip]}Ihr stets ergebener{\\[\baselineskip]}\spacefill\mbox{Max
                        Mell}\pend
           \leftskip=0em{}
         
         \endnumbering\mylabel{h}\end{ledgroupsized}  \newcommand{\dateiname}{L01728}\newcommand{\titel}{Max Mell an Arthur Schnitzler, 4. 11. 1907}\newcommand{\editorInnen}{Martin Anton Müller und Gerd-Hermann Susen}%% latex-leseansicht-abspann.tex
%% Abspann für die Leseansicht.
%% Der Schalter \ifkorrekturansicht ist bereits durch den Vorspann gesetzt.

%% latex-abspann.tex
%% Gemeinsamer Abspann für Korrekturansicht und Leseansicht.
%% Setzt den Schalter \ifkorrekturansicht voraus (gesetzt in den
%% einbindenden Dateien latex-korrekturansicht-abspann.tex bzw.
%% latex-leseansicht-abspann.tex).
%% ---------------------------------------------------------------

\normalsize

% Das esempio-Environment wird nur in der Leseansicht benötigt
\ifkorrekturansicht\else
\newenvironment{esempio}[3]%
{
    \vspace{1.5ex}
    \rlap{\underline{#1}}
    \par
    \setlength{\parindent}{0cm}
    \nopagebreak
    \leftskip=#2cm
    \rightskip=#3cm
}
{
    \par
}
\fi

\doendnotes{C}
\bigskip
\vfill

\clearpage

\footnotesize

\ifkorrekturansicht
  \lohead{\textsc{register}}
\fi

% theindex-Environment neu definieren ohne reledmac
\makeatletter
\renewenvironment{theindex}{%
  \ifkorrekturansicht
    \section*{\indexname}%
  \else
    \subsubsection*{Index der erwähnten Entitäten}%
  \fi
  \setlength{\parindent}{0pt}%
  \setlength{\parskip}{0pt plus 0.3pt}%
  \let\item\@idxitem
}{%
  \ifkorrekturansicht\clearpage\fi
}
\makeatother

\IfFileExists{\jobname-pw.ind}{\input{\jobname-pw.ind}}{}

% Quellenangabe nur in der Leseansicht
\ifkorrekturansicht\else
% Fallback-Definitionen, falls die .tex-Datei \titel etc. nicht gesetzt hat
\providecommand{\titel}{}
\providecommand{\editorInnen}{}
\providecommand{\dateiname}{\jobname}

\vspace{3cm}

\vfill

\footnotesize
\textsc{Quelle}: \titel. Herausgegeben von {\editorInnen}. In: \emph{Arthur Schnitzler: Briefwechsel mit Autorinnen und Autoren}.
 Digitale Edition, https://schnitzler-briefe.acdh.oeaw.ac.at/{\dateiname}.html (Stand \today)
\fi

\end{document}


      