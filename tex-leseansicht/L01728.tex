%% latex-korrekturansicht-vorspann.tex
%% Vorspann für die Korrekturansicht.
%% Lädt die gemeinsame Datei latex-vorspann.tex mit gesetztem Schalter.

\newif\ifkorrekturansicht
\korrekturansichttrue

\input{../tex-inputs/latex-vorspann}


\section[Max Mell an Arthur Schnitzler, 4. 11. 1907]{L01728 Max Mell an Arthur Schnitzler, 4. 11. 1907}
\nopagebreak\mylabel{L01728v}
\rehead{ }\normalsize\beginnumbering\briefempfaengerindex{Schnitzler, Arthur@\textsc{Schnitzler, Arthur}!zzzMell, Max@\emph{von Max Mell}!1907-11-041@{4. 11. 1907}|(be}
\toendnotes[C]{\smallbreak\pagebreak[2]}\Standort{DLA, A:Schnitzler, HS.NZ85.1.4055, S. [6].}
\physDesc{Brief, maschinenschriftliche Abschrift1 Blatt, 1 Seite, 515 Zeichen
\newline{}Schreibmaschine}\toendnotes[C]{\smallbreak}
\pstart
           \raggedleft{}{\pb}4. November 1907.\pend
           
\pstart\center{}Verehrter Herr Doktor,\pend\vspace{0.5em}
\pstart
           die Tänzerinnen Schwestern Wiesenthal\pwindex{Wiesenthal, Grethe 09.12.1885 – 24.06.1970@\textsc{Wiesenthal, Grethe} (09.12.1885 – 24.06.1970), \emph{Tänzer/Tänzerin}|pw}\pwindex{Wiesenthal, Elsa 1887-06-22 – 1967-02-20@\textsc{Wiesenthal, Elsa} (1887-06-22 – 1967-02-20), \emph{Tänzer/Tänzerin}|pw}\pwindex{Wiesenthal, Berta 1892-08-10 – 1953-10-02@\textsc{Wiesenthal, Berta} (1892-08-10 – 1953-10-02), \emph{Tänzer/Tänzerin}|pw} veranstalten am \label{K_L01728-1v}\edtext{Mittwoch}{\lemma{\textnormal{\emph{Mittwoch}}}\Cendnote{\textnormal{Schnitzler nahm die Einladung nicht an, er
                  war am 6. 11. 1907
                  auf einer Verbandssitzung.}}}\label{K_L01728-1} einen Tanzabend – sollte es Sie und Ihre Frau
                  Gemahlin\pwindex{Schnitzler, Olga 17.01.1882 – 13.01.1970@\textsc{Schnitzler, Olga} (17.01.1882 – 13.01.1970), \emph{Schauspieler/Schauspielerin, Sänger/Sängerin}|pwv} interessieren, so
               kommen Sie doch bitte dazu! Es findet im Atelier des Malers Huber\pwindex{Huber-Wiesenthal, Rudolf 1884-10-17 – 1983-10-16@\textsc{Huber-Wiesenthal, Rudolf} (1884-10-17 – 1983-10-16), \emph{Maler/Malerin}|pw}, IV.
                  Taubstummengasse 2\oindex{Taubstummengasse@\textbf{Taubstummengasse}, \emph{Straße (K.STR)}|pw}, statt, um ½ 8 abends, und es werden ausser
               mir nur noch Kolo Moser\pwindex{Moser, Koloman 1868-03-30 – 1918-10-18@\textsc{Moser, Koloman} (1868-03-30 – 1918-10-18), \emph{Maler/Malerin, Grafiker/Grafikerin}|pw} und Josef Hoffmann\pwindex{Hoffmann, Josef 22.07.1831 – 31.01.1904@\textsc{Hoffmann, Josef} (22.07.1831 – 31.01.1904), \emph{Maler/Malerin}|pw} dort sein, allenfalls Waerndorfer\pwindex{Waerndorfer, Friedrich 05.05.1868 – 09.08.1939@\textsc{Wärndorfer, Friedrich} (05.05.1868 – 09.08.1939), \emph{Industrieller/Industrielle, Mäzen/Mäzenin, Unternehmer/Unternehmerin}|pw}. Die Wiesenthals\pwindex{Wiesenthal, Grethe 09.12.1885 – 24.06.1970@\textsc{Wiesenthal, Grethe} (09.12.1885 – 24.06.1970), \emph{Tänzer/Tänzerin}|pw}\pwindex{Wiesenthal, Elsa 1887-06-22 – 1967-02-20@\textsc{Wiesenthal, Elsa} (1887-06-22 – 1967-02-20), \emph{Tänzer/Tänzerin}|pw}\pwindex{Wiesenthal, Berta 1892-08-10 – 1953-10-02@\textsc{Wiesenthal, Berta} (1892-08-10 – 1953-10-02), \emph{Tänzer/Tänzerin}|pw} wären über Ihr Kommen sehr
               erfreut, ich wurde gebeten, Sie zu benachrichtigen.\pend
           
\pstart
           Mit vielen Empfehlungen{\\[\baselineskip]}Ihr stets ergebener{\\[\baselineskip]}\spacefill\mbox{Max Mell}\pend
           \leftskip=0em{}\selectlanguage{ngerman}\endnumbering\briefempfaengerindex{Schnitzler, Arthur@\textsc{Schnitzler, Arthur}!zzzMell, Max@\emph{von Max Mell}!1907-11-041@{4. 11. 1907}|)be}\mylabel{L01728h}  \normalsize

\doendnotes{C}
\bigskip
\vfill

\clearpage

\footnotesize

\lohead{\textsc{register}}

% Definiere theindex-Environment komplett neu ohne reledmac
\makeatletter
\renewenvironment{theindex}{%
  \section*{\indexname}%
  \setlength{\parindent}{0pt}%
  \setlength{\parskip}{0pt plus 0.3pt}%
  \let\item\@idxitem
}{%
  \clearpage
}
\makeatother

\IfFileExists{\jobname-pw.ind}{\input{\jobname-pw.ind}}{}

\end{document}

      