%% latex-leseansicht-vorspann.tex
%% Vorspann für die Leseansicht.
%% Lädt die gemeinsame Datei latex-vorspann.tex mit nicht gesetztem Schalter.

\newif\ifkorrekturansicht
\korrekturansichtfalse

\input{../tex-inputs/latex-vorspann}


\section[Max Mell an Arthur Schnitzler, 4. 11. 1907]{L01728 Max Mell an Arthur Schnitzler, 4. 11. 1907}
\nopagebreak\mylabel{L01728v}
\rehead{ }\normalsize\beginnumbering\briefempfaengerindex{Schnitzler, Arthur@\textsc{Schnitzler, Arthur}!zzzMell, Max@\emph{von Max Mell}!1907-11-041@{4. 11. 1907}|(be}
\toendnotes[C]{\smallbreak\pagebreak[2]}
\correspDesc{Versand  durch Max Mell am 4. 11. 1907 in Wien
\newline{}Erhalt  durch Arthur Schnitzler im Zeitraum [4. 11. 1907
                  – 8. 11. 1907?] in Wien}\toendnotes[C]{\smallbreak}
\Standort{DLA, A:Schnitzler, HS.NZ85.1.4055, S. [6].}
\physDesc{Brief, maschinenschriftliche Abschrift, 1 Blatt, 1 Seite, 515 Zeichen
\newline{}Schreibmaschine}\toendnotes[C]{\smallbreak}
\pstart
           \raggedleft{}{\pb}4. November 1907.\pend
           
\pstart\center{}Verehrter Herr Doktor,\pend\vspace{0.5em}
\pstart
           die Tänzerinnen Schwestern Wiesenthal\pwindex{Wiesenthal, Grethe 9.\,12.\,1885 Wien – 24.\,6.\,1970 ebd.@\textsc{Wiesenthal, Grethe} (9.\,12.\,1885 Wien – 24.\,6.\,1970 ebd.), \emph{Tänzerin}|pw}\pwindex{Wiesenthal, Elsa 22.\,6.\,1887 Wien – 20.\,2.\,1967 Zürich@\textsc{Wiesenthal, Elsa} (22.\,6.\,1887 Wien – 20.\,2.\,1967 Zürich), \emph{Tänzerin}|pw}\pwindex{Wiesenthal, Berta 10.\,8.\,1892 Wien – 2.\,10.\,1953 Stockerau@\textsc{Wiesenthal, Berta} (10.\,8.\,1892 Wien – 2.\,10.\,1953 Stockerau), \emph{Tänzerin}|pw} veranstalten am \label{K_L01728-1v}\edtext{Mittwoch}{\lemma{\textnormal{\emph{Mittwoch}}}\Cendnote{\textnormal{Schnitzler nahm die Einladung nicht an, er
                  war am 6. 11. 1907
                  auf einer Verbandssitzung.}}}\label{K_L01728-1} einen Tanzabend – sollte es Sie und Ihre Frau
                  Gemahlin\pwindex{Schnitzler, Olga 17.\,1.\,1882 Wien – 13.\,1.\,1970 Lugano@\textsc{Schnitzler, Olga} (17.\,1.\,1882 Wien – 13.\,1.\,1970 Lugano), \emph{Schauspielerin, Sängerin}|pwv} interessieren, so
               kommen Sie doch bitte dazu! Es findet im Atelier des Malers Huber\pwindex{Huber-Wiesenthal, Rudolf 17.\,10.\,1884 Wien – 16.\,10.\,1983 Zürich@\textsc{Huber-Wiesenthal, Rudolf} (17.\,10.\,1884 Wien – 16.\,10.\,1983 Zürich), \emph{Maler}|pw}, IV.
                  Taubstummengasse 2\oindex{Wien@\textbf{Wien}!IV., Wieden@\textbf{IV., Wieden}!Taubstummengasse@\textbf{Taubstummengasse}, \emph{Straße}|pw}, statt, um ½ 8 abends, und es werden ausser
               mir nur noch Kolo Moser\pwindex{Moser, Koloman 30.\,3.\,1868 Wien – 18.\,10.\,1918 ebd.@\textsc{Moser, Koloman} (30.\,3.\,1868 Wien – 18.\,10.\,1918 ebd.), \emph{Maler, Grafiker}|pw} und Josef Hoffmann\pwindex{Hoffmann, Josef 22.\,7.\,1831 Wien – 31.\,1.\,1904 ebd.@\textsc{Hoffmann, Josef} (22.\,7.\,1831 Wien – 31.\,1.\,1904 ebd.), \emph{Maler}|pw} dort sein, allenfalls Waerndorfer\pwindex{Wärndorfer, Friedrich 5.\,5.\,1868 Wien – 9.\,8.\,1939 Bryn Mawr@\textsc{Wärndorfer, Friedrich} (5.\,5.\,1868 Wien – 9.\,8.\,1939 Bryn Mawr), \emph{Industrieller, Mäzen, Unternehmer}|pw}. Die Wiesenthals\pwindex{Wiesenthal, Grethe 9.\,12.\,1885 Wien – 24.\,6.\,1970 ebd.@\textsc{Wiesenthal, Grethe} (9.\,12.\,1885 Wien – 24.\,6.\,1970 ebd.), \emph{Tänzerin}|pw}\pwindex{Wiesenthal, Elsa 22.\,6.\,1887 Wien – 20.\,2.\,1967 Zürich@\textsc{Wiesenthal, Elsa} (22.\,6.\,1887 Wien – 20.\,2.\,1967 Zürich), \emph{Tänzerin}|pw}\pwindex{Wiesenthal, Berta 10.\,8.\,1892 Wien – 2.\,10.\,1953 Stockerau@\textsc{Wiesenthal, Berta} (10.\,8.\,1892 Wien – 2.\,10.\,1953 Stockerau), \emph{Tänzerin}|pw} wären über Ihr Kommen sehr
               erfreut, ich wurde gebeten, Sie zu benachrichtigen.\pend
           
\pstart
           Mit vielen Empfehlungen{\\[\baselineskip]}Ihr stets ergebener{\\[\baselineskip]}\spacefill\mbox{Max Mell}\pend
           \leftskip=0em{}\selectlanguage{ngerman}\endnumbering\briefempfaengerindex{Schnitzler, Arthur@\textsc{Schnitzler, Arthur}!zzzMell, Max@\emph{von Max Mell}!1907-11-041@{4. 11. 1907}|)be}\mylabel{L01728h}  \newcommand{\dateiname}{L01728}\newcommand{\titel}{Max Mell an Arthur Schnitzler, 4. 11. 1907}\newcommand{\editorInnen}{Martin Anton Müller und Gerd-Hermann Susen}%% latex-leseansicht-abspann.tex
%% Abspann für die Leseansicht.
%% Der Schalter \ifkorrekturansicht ist bereits durch den Vorspann gesetzt.

%% latex-abspann.tex
%% Gemeinsamer Abspann für Korrekturansicht und Leseansicht.
%% Setzt den Schalter \ifkorrekturansicht voraus (gesetzt in den
%% einbindenden Dateien latex-korrekturansicht-abspann.tex bzw.
%% latex-leseansicht-abspann.tex).
%% ---------------------------------------------------------------

\normalsize

% Das esempio-Environment wird nur in der Leseansicht benötigt
\ifkorrekturansicht\else
\newenvironment{esempio}[3]%
{
    \vspace{1.5ex}
    \rlap{\underline{#1}}
    \par
    \setlength{\parindent}{0cm}
    \nopagebreak
    \leftskip=#2cm
    \rightskip=#3cm
}
{
    \par
}
\fi

\doendnotes{C}
\bigskip
\vfill

\clearpage

\footnotesize

\ifkorrekturansicht
  \lohead{\textsc{register}}
\fi

% theindex-Environment neu definieren ohne reledmac
\makeatletter
\renewenvironment{theindex}{%
  \ifkorrekturansicht
    \section*{\indexname}%
  \else
    \subsubsection*{Index der erwähnten Entitäten}%
  \fi
  \setlength{\parindent}{0pt}%
  \setlength{\parskip}{0pt plus 0.3pt}%
  \let\item\@idxitem
}{%
  \ifkorrekturansicht\clearpage\fi
}
\makeatother

\IfFileExists{\jobname-pw.ind}{\input{\jobname-pw.ind}}{}

% Quellenangabe nur in der Leseansicht
\ifkorrekturansicht\else
% Fallback-Definitionen, falls die .tex-Datei \titel etc. nicht gesetzt hat
\providecommand{\titel}{}
\providecommand{\editorInnen}{}
\providecommand{\dateiname}{\jobname}

\vspace{3cm}

\vfill

\footnotesize
\textsc{Quelle}: \titel. Herausgegeben von {\editorInnen}. In: \emph{Arthur Schnitzler: Briefwechsel mit Autorinnen und Autoren}.
 Digitale Edition, https://schnitzler-briefe.acdh.oeaw.ac.at/{\dateiname}.html (Stand \today)
\fi

\end{document}


