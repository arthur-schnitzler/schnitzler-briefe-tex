%% latex-leseansicht-vorspann.tex
%% Vorspann für die Leseansicht.
%% Lädt die gemeinsame Datei latex-vorspann.tex mit nicht gesetztem Schalter.

\newif\ifkorrekturansicht
\korrekturansichtfalse

\input{../tex-inputs/latex-vorspann}


\section[Arthur Schnitzler an Hugo von Hofmannsthal, 29. 9. 1899]{L00983 Arthur Schnitzler an Hugo von Hofmannsthal, 29. 9. 1899}
\nopagebreak\mylabel{L00983v}
\rehead{ }\normalsize\beginnumbering\briefempfaengerindex{Hofmannsthal, Hugo von@\textsc{Hofmannsthal, Hugo von}!zzzSchnitzler, Arthur@\emph{von Arthur Schnitzler}!1899-09-292@{29. 9. 1899}|(be}
\toendnotes[C]{\smallbreak\pagebreak[2]}
\correspDesc{Versand  durch Arthur Schnitzler am 29. 9. 1899 in Wiesbaden
\newline{}Erhalt  durch Hugo von Hofmannsthal im Zeitraum [30. 9. 1899
                  – 4. 10. 1899?] in Venedig}\toendnotes[C]{\smallbreak}
\Standort{FDH, Hs-30885,87.}
\physDesc{Brief, 1 Blatt, 4 Seiten, 1024 Zeichen
\newline{}Handschrift: Bleistift, deutsche Kurrent
\newline{}Ordnung: mit Bleistift von Schnitzler mutmaßlich bei der Durchsicht der
                                 Korrespondenz 1929 Ergänzung der Jahreszahl »99« sowie des Ortes »\textsc{Wiesbaden}\oindex{Wiesbaden@\textbf{Wiesbaden}|pw}« }
\buchAbdrucke{\weitereDrucke{Hugo von Hofmannsthal, Arthur Schnitzler: \emph{Briefwechsel}. Herausgegeben von Therese Nickl und Heinrich Schnitzler. Frankfurt am Main: \emph{S. Fischer} 1964, S. 131.} }\toendnotes[C]{\smallbreak}
\pstart
           \raggedleft{}{\pb}Freitag 29. 9.\pend
           \vspace{0.5em}
\pstart
           mein lieber Hugo, das geht{ }ſchon{ }ſo mit den Stücken. Am leichteſten{ }ſind{ }ſie we{\geminationn}{ }ſie einem grad einfallen, – da{ }ſind{ }ſie beinah
               fertig. Über meines will ich nichts{ }ſagen – mein Vertrauen wechſelt; das höchſte und
               wohl auch das höhere iſt mir nun einmal {\pb}verſagt; ich will
               für die Momente dankbar{ }ſein, in denen ich eine gewiſſe innere Fülle empfinde.  –\pend
           
\pstart
           Ich bleibe hier noch bis zum Dinſtag, fahre da{\geminationn} nach Berlin\oindex{Berlin@\textbf{Berlin}, \emph{Hauptstadt}|pw} (\textsc{Hotel Savoy}\oindex{Hotel Savoy [Berlin]@\textbf{Hotel Savoy [Berlin]}, \emph{Hotel}|pw}, bitte{ }ſchreiben Sie mir hin)\pend
           
\pstart
           – Die paar Tage mit \textsc{Beatrice}\pwindex{Schnitzler, Arthur 15.\,5.\,1862 Wien – 21.\,10.\,1931 ebd.@\textsc{Schnitzler, Arthur} (15.\,5.\,1862 Wien – 21.\,10.\,1931 ebd.), \emph{Schriftsteller, Mediziner}!Schleier der Beatrice. Schauspiel in fünf Akten@\strich\emph{Der Schleier der Beatrice. Schauspiel in fünf Akten}|pw}{ }{\pb}(München\oindex{München@\textbf{München}|pw}, Nürnberg\oindex{Nürnberg@\textbf{Nürnberg}|pw}) waren ziemlich, ja ganz ungeſtört;
               eigentlich wirklich hübſch. Seit zehn Tagen hab ich erſt einmal, ganz flüchtig von
                  ihr\pwindex{Reinhard, Marie 13.\,3.\,1871 Wien – 18.\,3.\,1899 ebd.@\textsc{Reinhard, Marie} (13.\,3.\,1871 Wien – 18.\,3.\,1899 ebd.), \emph{Gesangspädagogin}|pwv} gehört. – In Frankfurt\oindex{Frankfurt am Main@\textbf{Frankfurt am Main}, \emph{Hauptstadt}|pw} freute ich mich Paul Goldm\pwindex{Goldmann, Paul 31.\,1.\,1865 Breslau – 25.\,9.\,1935 Wien@\textsc{Goldmann, Paul} (31.\,1.\,1865 Breslau – 25.\,9.\,1935 Wien), \emph{Schriftsteller, Journalist}|pw} in{ }ſozuſagen glücklichrer Sti{\geminationm}ung zu{ }ſehn als je. – Hier leb ich ganz allein, in
               einem{ }ſchönen, angenehmen Hotel\oindex{Hôtel du Parc {\kaufmannsund} Bristol@\textbf{Hôtel du Parc {\kaufmannsund} Bristol}, \emph{Hotel}|pwv}, bin heut (i{\geminationm}er{ }ſchlechtes Wetter) zum
               erſten Mal geradelt; arbeite nicht wenig; habe natürlich zuweilen Stunden von einer
               unbeſchreiblichen Traurigkeit. Ich glaube, ich werde immer mehr arbeiten,{ }ſolang’s
               eben geht.\pend
           \pstart Von Herzen Ihr \spacefill\mbox{Arthur.}\pend{}\selectlanguage{ngerman}\endnumbering\briefempfaengerindex{Hofmannsthal, Hugo von@\textsc{Hofmannsthal, Hugo von}!zzzSchnitzler, Arthur@\emph{von Arthur Schnitzler}!1899-09-292@{29. 9. 1899}|)be}\mylabel{L00983h}  \newcommand{\dateiname}{L00983}\newcommand{\titel}{Arthur Schnitzler an Hugo von Hofmannsthal, 29. 9. 1899}\newcommand{\editorInnen}{Martin Anton Müller und Gerd-Hermann Susen}%% latex-leseansicht-abspann.tex
%% Abspann für die Leseansicht.
%% Der Schalter \ifkorrekturansicht ist bereits durch den Vorspann gesetzt.

%% latex-abspann.tex
%% Gemeinsamer Abspann für Korrekturansicht und Leseansicht.
%% Setzt den Schalter \ifkorrekturansicht voraus (gesetzt in den
%% einbindenden Dateien latex-korrekturansicht-abspann.tex bzw.
%% latex-leseansicht-abspann.tex).
%% ---------------------------------------------------------------

\normalsize

% Das esempio-Environment wird nur in der Leseansicht benötigt
\ifkorrekturansicht\else
\newenvironment{esempio}[3]%
{
    \vspace{1.5ex}
    \rlap{\underline{#1}}
    \par
    \setlength{\parindent}{0cm}
    \nopagebreak
    \leftskip=#2cm
    \rightskip=#3cm
}
{
    \par
}
\fi

\doendnotes{C}
\bigskip
\vfill

\clearpage

\footnotesize

\ifkorrekturansicht
  \lohead{\textsc{register}}
\fi

% theindex-Environment neu definieren ohne reledmac
\makeatletter
\renewenvironment{theindex}{%
  \ifkorrekturansicht
    \section*{\indexname}%
  \else
    \subsubsection*{Index der erwähnten Entitäten}%
  \fi
  \setlength{\parindent}{0pt}%
  \setlength{\parskip}{0pt plus 0.3pt}%
  \let\item\@idxitem
}{%
  \ifkorrekturansicht\clearpage\fi
}
\makeatother

\IfFileExists{\jobname-pw.ind}{\input{\jobname-pw.ind}}{}

% Quellenangabe nur in der Leseansicht
\ifkorrekturansicht\else
% Fallback-Definitionen, falls die .tex-Datei \titel etc. nicht gesetzt hat
\providecommand{\titel}{}
\providecommand{\editorInnen}{}
\providecommand{\dateiname}{\jobname}

\vspace{3cm}

\vfill

\footnotesize
\textsc{Quelle}: \titel. Herausgegeben von {\editorInnen}. In: \emph{Arthur Schnitzler: Briefwechsel mit Autorinnen und Autoren}.
 Digitale Edition, https://schnitzler-briefe.acdh.oeaw.ac.at/{\dateiname}.html (Stand \today)
\fi

\end{document}


