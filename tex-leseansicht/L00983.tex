%% latex-korrekturansicht-vorspann.tex
%% Vorspann für die Korrekturansicht.
%% Lädt die gemeinsame Datei latex-vorspann.tex mit gesetztem Schalter.

\newif\ifkorrekturansicht
\korrekturansichttrue

\input{../tex-inputs/latex-vorspann}


\section[Arthur Schnitzler an Hugo von Hofmannsthal, 29. 9. 1899]{L00983 Arthur Schnitzler an Hugo von Hofmannsthal, 29. 9. 1899}
\nopagebreak\mylabel{L00983v}
\rehead{ }\normalsize\beginnumbering\briefempfaengerindex{Hofmannsthal, Hugo von@\textsc{Hofmannsthal, Hugo von}!zzzSchnitzler, Arthur@\emph{von Arthur Schnitzler}!1899-09-292@{29. 9. 1899}|(be}
\toendnotes[C]{\smallbreak\pagebreak[2]}\Standort{FDH, Hs-30885,87.}
\physDesc{Brief, 1 Blatt, 4 Seiten, 1024 Zeichen
\newline{}Handschrift: Bleistift, deutsche Kurrent
\newline{}Ordnung: mit Bleistift von Schnitzler mutmaßlich bei der Durchsicht der
                                 Korrespondenz 1929 Ergänzung der Jahreszahl »99« sowie des Ortes »\textsc{Wiesbaden}\oindex{Wiesbaden@\textbf{Wiesbaden}, \emph{P.PPLA}|pw}« }
\buchAbdrucke{\weitereDrucke{Hugo von Hofmannsthal, Arthur Schnitzler: \emph{Briefwechsel}. Frankfurt am Main: \emph{S. Fischer} 1964, S. 131.} }\toendnotes[C]{\smallbreak}
\pstart
           \raggedleft{}{\pb}Freitag 29. 9.\pend
           \vspace{0.5em}
\pstart
           mein lieber Hugo, das geht ſchon ſo mit den Stücken. Am leichteſten
               ſind ſie we{\geminationn}{ }ſie einem grad einfallen, – da ſind ſie beinah
               fertig. Über meines will ich nichts ſagen – mein Vertrauen wechſelt; das höchſte und
               wohl auch das höhere iſt mir nun einmal {\pb}verſagt; ich will
               für die Momente dankbar ſein, in denen ich eine gewiſſe innere Fülle empfinde.  –\pend
           
\pstart
           Ich bleibe hier noch bis zum Dinſtag, fahre da{\geminationn} nach Berlin\oindex{Berlin@\textbf{Berlin}, \emph{P.PPLC}|pw} (\textsc{Hotel Savoy}\oindex{Hotel Savoy [Berlin]@\textbf{Hotel Savoy [Berlin]}, \emph{Hotel (K.HTL)}|pw}, bitte ſchreiben Sie mir hin)\pend
           
\pstart
           – Die paar Tage mit \textsc{Beatrice}\pwindex{Schleier der Beatrice. Schauspiel in fuenf Akten@\emph{Der Schleier der Beatrice. Schauspiel in fünf Akten}|pw}{ }{\pb}(München\oindex{Muenchen@\textbf{München}, \emph{P.PPLA}|pw}, Nürnberg\oindex{Nuernberg@\textbf{Nürnberg}, \emph{P.PPL}|pw}) waren ziemlich, ja ganz ungeſtört;
               eigentlich wirklich hübſch. Seit zehn Tagen hab ich erſt einmal, ganz flüchtig von
                  ihr\pwindex{Reinhard, Marie 1871-03-13 – 1899-03-18@\textsc{Reinhard, Marie} (1871-03-13 – 1899-03-18), \emph{Gesangspädagoge/Gesangspädagogin}|pwv} gehört. – In Frankfurt\oindex{Frankfurt am Main@\textbf{Frankfurt am Main}, \emph{P.PPLA3}|pw} freute ich mich Paul Goldm\pwindex{Goldmann, Paul 31.01.1865 – 25.09.1935@\textsc{Goldmann, Paul} (31.01.1865 – 25.09.1935), \emph{Schriftsteller/Schriftstellerin, Journalist/Journalistin}|pw} in ſozuſagen glücklichrer Sti{\geminationm}ung zu ſehn als je. – Hier leb ich ganz allein, in
               einem ſchönen, angenehmen Hotel\oindex{Hôtel du Parc {\kaufmannsund} Bristol@\textbf{Hôtel du Parc {\kaufmannsund} Bristol}, \emph{Hotel (K.HTL)}|pwv}, bin heut (i{\geminationm}er ſchlechtes Wetter) zum
               erſten Mal geradelt; arbeite nicht wenig; habe natürlich zuweilen Stunden von einer
               unbeſchreiblichen Traurigkeit. Ich glaube, ich werde immer mehr arbeiten, ſolang’s
               eben geht.\pend
           \pstart Von Herzen Ihr \spacefill\mbox{Arthur.}\pend{}\selectlanguage{ngerman}\endnumbering\briefempfaengerindex{Hofmannsthal, Hugo von@\textsc{Hofmannsthal, Hugo von}!zzzSchnitzler, Arthur@\emph{von Arthur Schnitzler}!1899-09-292@{29. 9. 1899}|)be}\mylabel{L00983h}  \normalsize

\doendnotes{C}
\bigskip
\vfill

\clearpage

\footnotesize

\lohead{\textsc{register}}

% Definiere theindex-Environment komplett neu ohne reledmac
\makeatletter
\renewenvironment{theindex}{%
  \section*{\indexname}%
  \setlength{\parindent}{0pt}%
  \setlength{\parskip}{0pt plus 0.3pt}%
  \let\item\@idxitem
}{%
  \clearpage
}
\makeatother

\IfFileExists{\jobname-pw.ind}{\input{\jobname-pw.ind}}{}

\end{document}

      