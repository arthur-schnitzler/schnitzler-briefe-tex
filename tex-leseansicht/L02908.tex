%% latex-korrekturansicht-vorspann.tex
%% Vorspann für die Korrekturansicht.
%% Lädt die gemeinsame Datei latex-vorspann.tex mit gesetztem Schalter.

\newif\ifkorrekturansicht
\korrekturansichttrue

\input{../tex-inputs/latex-vorspann}


\section[ Paul Goldmann an Arthur Schnitzler, 24. 3. {[}1900{]}]{L02908 Paul Goldmann an Arthur Schnitzler, 24. 3. {[}1900{]}}
\nopagebreak\mylabel{L02908v}
\rehead{ }\normalsize\beginnumbering\briefempfaengerindex{Schnitzler, Arthur@\textsc{Schnitzler, Arthur}!zzzGoldmann, Paul@\emph{von Paul Goldmann}!1900-03-241@{24. 3. {[}1900{]}}|(be}
\toendnotes[C]{\smallbreak\pagebreak[2]}\Standort{DLA, A:Schnitzler, HS.NZ85.1.3170.}
\physDesc{Brief, 1 Blatt, 1 Seite, 361 Zeichen
\newline{}Handschrift: blaue Tinte, deutsche Kurrent
\newline{}Schnitzler: 1) mit Bleistift das Jahr »900« vermerkt  2) mit rotem Buntstift eine Unterstreichung}\toendnotes[C]{\smallbreak}
\pstart
           {\pb}\textcolor{gray}{\textbf{DESSAUERSTRASSE 19}}\oindex{Dessauer Strasse@\textbf{Dessauer Straße}, \emph{Straße (K.STR)}|pw}\pend
           
\pstart
           \raggedleft{}Berlin\oindex{Berlin@\textbf{Berlin}, \emph{P.PPLC}|pw}, 24. März.\pend
           
\pstart{}Mein lieber Freund,\pend\vspace{0.5em}
\pstart
           Ich danke Dir für die Überſendung des \label{K_L02908-1v}\edtext{\textsc{Hoffmannsthal\pwindex{Hofmannsthal, Hugo von 1874-02-01 – 1929-07-15@\textsc{Hofmannsthal, Hugo von} (1874-02-01 – 1929-07-15), \emph{Schriftsteller/Schriftstellerin}|pw}}’ſchen Vorſpiels\pwindex{Vorspiel zur Antigone des Sophokles@\emph{Vorspiel zur Antigone des Sophokles}|pwv}}{\lemma{\textnormal{\emph{Hoffmannsthal’ſchen Vorſpiels}}}\Cendnote{\textnormal{Hugo von Hofmannsthal\pwindex{Hofmannsthal, Hugo von 1874-02-01 – 1929-07-15@\textsc{Hofmannsthal, Hugo von} (1874-02-01 – 1929-07-15), \emph{Schriftsteller/Schriftstellerin}|pwk} hatte Schnitzler gebeten, sein \emph{Vorspiel zur Antigone des Sophokles}\pwindex{Vorspiel zur Antigone des Sophokles@\emph{Vorspiel zur Antigone des Sophokles}|pwk} an Goldmann\pwindex{Goldmann, Paul 31.01.1865 – 25.09.1935@\textsc{Goldmann, Paul} (31.01.1865 – 25.09.1935), \emph{Schriftsteller/Schriftstellerin, Journalist/Journalistin}|pwk} zu übersenden. Vgl. Hugo von Hofmannsthal an Arthur Schnitzler, 15. 3. [1900], Hugo August von Hofmannsthal an Arthur Schnitzler, 22. 3. 1900
                  und Arthur Schnitzler an Hugo von Hofmannsthal, 23. 3. 1900.}}}\label{K_L02908-1}. Ich finde es
               abſcheulich.\pend
           
\pstart
           Haſt Du meinen \label{K_L02908-2v}\edtext{Brief von \strikeout{\textcolor{gray}{×}}{ }vorgeſtern}{\lemma{\textnormal{\emph{Brief von  vorgeſtern}}}\Cendnote{\textnormal{Paul Goldmann an Arthur Schnitzler, 22. 3. [1900].
               }}}\label{K_L02908-2} nicht erhalten?\pend
           
\pstart
           Ich danke Dir für die \label{K_L02908-3v}\edtext{Mittheilung der
               Äußerung der Frau \textsc{Bürger}\pwindex{Burger, Caroline 11.07.1869 – 15.03.1959@\textsc{Burger, Caroline} (11.07.1869 – 15.03.1959)|pwu}}{\lemma{\textnormal{\emph{Mittheilung … Bürger}}}\Cendnote{\textnormal{Die Stelle ist nicht mit Sicherheit aufzuschlüsseln. Es könnte sich um eine
                  Aussage von Caroline Burger\pwindex{Burger, Caroline 11.07.1869 – 15.03.1959@\textsc{Burger, Caroline} (11.07.1869 – 15.03.1959)|pwk} handeln, die
                  ältere Schwester von Marie Reinhard\pwindex{Reinhard, Marie 1871-03-13 – 1899-03-18@\textsc{Reinhard, Marie} (1871-03-13 – 1899-03-18), \emph{Gesangspädagoge/Gesangspädagogin}|pwk}, mit der
                     Schnitzler in Kontakt stand.}}}\label{K_L02908-3}, die
               mich ſehr gefreut hat.\pend
           
\pstart
           Haſt Du die prachtvolle \label{K_L02908-4v}\edtext{\textsc{Dante\pwindex{Dante Alighieri um 1265 – 22.09.1321@\textsc{Dante Alighieri} (um 1265 – 22.09.1321), \emph{Schriftsteller/Schriftstellerin}|pw}}-Biographie\pwindex{Dante@\emph{Dante}|pw} von \textsc{Federn\pwindex{Federn, Karl 02.02.1868 – 22.03.1943@\textsc{Federn, Karl} (02.02.1868 – 22.03.1943), \emph{Schriftsteller/Schriftstellerin, Übersetzer/Übersetzerin}|pw}}}{\lemma{\textnormal{\emph{Dante-Biographie von Federn}}}\Cendnote{\textnormal{Schnitzler las Karl Federns\pwindex{Federn, Karl 02.02.1868 – 22.03.1943@\textsc{Federn, Karl} (02.02.1868 – 22.03.1943), \emph{Schriftsteller/Schriftstellerin, Übersetzer/Übersetzerin}|pwk}{ }Dante\pwindex{Dante Alighieri um 1265 – 22.09.1321@\textsc{Dante Alighieri} (um 1265 – 22.09.1321), \emph{Schriftsteller/Schriftstellerin}|pwk}-Biographie\pwindex{Dante@\emph{Dante}|pwkv} (zuerst unter dem
                  Titel \emph{Dante}\pwindex{Dante@\emph{Dante}|pwk} erschienen, später auch unter \emph{Dante und seine Zeit}\pwindex{Dante@\emph{Dante}|pwk}) im Mai 1900 (vgl. Arthur Schnitzler an Georg Brandes, 3. 5. 1900).}}}\label{K_L02908-4} ſchon geleſen?\pend
           
\pstart
           Viele treue Grüße! {\\[\baselineskip]}Dein {\\[\baselineskip]}\spacefill\mbox{Paul Goldmann}\pend
           \leftskip=0em{}\selectlanguage{ngerman}\endnumbering\briefempfaengerindex{Schnitzler, Arthur@\textsc{Schnitzler, Arthur}!zzzGoldmann, Paul@\emph{von Paul Goldmann}!1900-03-241@{24. 3. {[}1900{]}}|)be}\mylabel{L02908h}  \normalsize

\doendnotes{C}
\bigskip
\vfill

\clearpage

\footnotesize

\lohead{\textsc{register}}

% Definiere theindex-Environment komplett neu ohne reledmac
\makeatletter
\renewenvironment{theindex}{%
  \section*{\indexname}%
  \setlength{\parindent}{0pt}%
  \setlength{\parskip}{0pt plus 0.3pt}%
  \let\item\@idxitem
}{%
  \clearpage
}
\makeatother

\IfFileExists{\jobname-pw.ind}{\input{\jobname-pw.ind}}{}

\end{document}

      