%% latex-leseansicht-vorspann.tex
%% Vorspann für die Leseansicht.
%% Lädt die gemeinsame Datei latex-vorspann.tex mit nicht gesetztem Schalter.

\newif\ifkorrekturansicht
\korrekturansichtfalse

\input{../tex-inputs/latex-vorspann}

\begin{center}
            \textcolor{red}{ENTWURF, NICHT FERTIG KORRIGIERT}
                      \end{center}
            
         
         \newcommand{\erwaehntePersonen}{Personen: Caroline Burger,  Dante Alighieri, Karl Federn, Hugo von Hofmannsthal}
         \newcommand{\erwaehnteInstitutionen}{}
         \newcommand{\erwaehnteOrte}{Orte: Berlin, Dessauer Straße, Wien}
         \newcommand{\erwaehnteWerke}{Werke: Dante, Vorspiel zur Antigone des Sophokles}
               \section[ Paul Goldmann an Arthur Schnitzler, 24. 3. {[}1900{]}]{ Paul Goldmann an Arthur Schnitzler, 24. 3. {[}1900{]}}\nopagebreak\mylabel{v}\rehead{ }\begin{ledgroupsized}[t]{13cm}\normalsize\beginnumbering \toendnotes[C]{\smallbreak\pagebreak[2]} \Standort{DLA, A:Schnitzler, HS.NZ85.1.3170.}
\physDesc{Brief, 1 Blatt, 1 Seite
\newline{}Handschrift: blaue Tinte, deutsche Kurrent
\newline{}Schnitzler: 1) mit Bleistift das Jahr »{[}1{]}900« vermerkt  2) mit rotem Buntstift eine Unterstreichung}\toendnotes[C]{\smallbreak}\pstart{}{\pb}\textcolor{gray}{\textbf{DESSAUERSTRASSE 19}}\oindex{Dessauer Strasse@\textbf{Dessauer Straße}|pw}\pend{}{\bigskip}\pstart
           \raggedleft{}Berlin\oindex{Berlin@\textbf{Berlin}|pw}, 24. März.\pend
           \pstart\center{}Mein lieber Freund,\pend\pstart
           Ich danke Dir für die Überſendung des \label{K_L02908-3v}\edtext{\textsc{Hoffmannsthal\pwindex{Hofmannsthal, Hugo von 1874-02-01 – 1929-07-15@\textsc{Hofmannsthal, Hugo von} (1874-02-01 – 1929-07-15), \emph{Schriftsteller}|pw}}’ſchen Vorſpiel\pwindex{Hofmannsthal, Hugo von 1874-02-01 – 1929-07-15@\textsc{Hofmannsthal, Hugo von} (1874-02-01 – 1929-07-15), \emph{Schriftsteller}!Vorspiel zur Antigone des Sophokles26. 3. 1900@\strich\emph{Vorspiel zur Antigone des Sophokles} {[}26. 3. 1900{]}|pwv}s}{\lemma{\textnormal{\emph{Hoffmannsthal’ſchen Vorſpiels}}}\Cendnote{\textnormal{Hugo von Hofmannsthal\pwindex{Hofmannsthal, Hugo von 1874-02-01 – 1929-07-15@\textsc{Hofmannsthal, Hugo von} (1874-02-01 – 1929-07-15), \emph{Schriftsteller}|pwk} hatte Schnitzler\pwindex{Schnitzler, Arthur 15.05.1862 – 21.10.1931@\textsc{Schnitzler, Arthur} (15.05.1862 – 21.10.1931), \emph{Schriftsteller, Mediziner}|pwk} gebeten, sein \emph{Vorspiel zur Antigone des Sophokles}\pwindex{Hofmannsthal, Hugo von 1874-02-01 – 1929-07-15@\textsc{Hofmannsthal, Hugo von} (1874-02-01 – 1929-07-15), \emph{Schriftsteller}!Vorspiel zur Antigone des Sophokles26. 3. 1900@\strich\emph{Vorspiel zur Antigone des Sophokles} {[}26. 3. 1900{]}|pwk} an Goldmann\pwindex{Goldmann, Paul 31.01.1865 – 25.09.1935@\textsc{Goldmann, Paul} (31.01.1865 – 25.09.1935), \emph{Schriftsteller, Journalist}|pwk} zu übersenden. Vgl. Hugo von Hofmannsthal an Arthur Schnitzler,
               15. 3. [1900], Hugo August von Hofmannsthal an Arthur Schnitzler,
                    22. 3. 1900
                  und Arthur Schnitzler an Hugo von Hofmannsthal, 23. 3. 1900.}}}\label{K_L02908-3h}. Ich finde es
               abſcheulich.\pend
           \pstart
           Haſt Du meinen \label{K_L02908-1v}\edtext{Brief von \strikeout{\textcolor{gray}{×}}{ }vorgeſtern}{\lemma{\textnormal{\emph{Brief von  vorgeſtern}}}\Cendnote{\textnormal{Paul Goldmann an Arthur Schnitzler, 22. 3. [1900]}}}\label{K_L02908-1h} nicht erhalten?\pend
           \pstart
           Ich danke Dir für die \label{K_L02908-4v}\edtext{Mittheilung}{\lemma{\textnormal{\emph{Mittheilung}}}\Cendnote{\textnormal{Bezug unklar}}}\label{K_L02908-4h} der Äußerung der Frau \textsc{Bürger}\pwindex{Burger, Caroline 11.07.1869 – 15.03.1959@\textsc{Burger, Caroline} (11.07.1869 – 15.03.1959)|pwv}, die mich ſehr gefreut hat.\pend
           \pstart
           Haſt Du die prachtvolle \label{K_L02908-2v}\edtext{\textsc{Dante\pwindex{Dante Alighieri um 1265 – 22.09.1321@\textsc{Dante Alighieri} (um 1265 – 22.09.1321), \emph{Schriftsteller}|pw}}-Biographie\pwindex{Federn, Karl 02.02.1868 – 22.03.1943@\textsc{Federn, Karl} (02.02.1868 – 22.03.1943), \emph{Schriftsteller, Übersetzer}!Dante1899@\strich\emph{Dante} {[}1899{]}|pw} von \textsc{Federn\pwindex{Federn, Karl 02.02.1868 – 22.03.1943@\textsc{Federn, Karl} (02.02.1868 – 22.03.1943), \emph{Schriftsteller, Übersetzer}|pw}}}{\lemma{\textnormal{\emph{Dante-Biographie von Federn}}}\Cendnote{\textnormal{Schnitzler\pwindex{Schnitzler, Arthur 15.05.1862 – 21.10.1931@\textsc{Schnitzler, Arthur} (15.05.1862 – 21.10.1931), \emph{Schriftsteller, Mediziner}|pwk} las Karl Federn\pwindex{Federn, Karl 02.02.1868 – 22.03.1943@\textsc{Federn, Karl} (02.02.1868 – 22.03.1943), \emph{Schriftsteller, Übersetzer}|pwk}s Dante\pwindex{Dante Alighieri um 1265 – 22.09.1321@\textsc{Dante Alighieri} (um 1265 – 22.09.1321), \emph{Schriftsteller}|pwk}-Biographie\pwindex{Federn, Karl 02.02.1868 – 22.03.1943@\textsc{Federn, Karl} (02.02.1868 – 22.03.1943), \emph{Schriftsteller, Übersetzer}!Dante1899@\strich\emph{Dante} {[}1899{]}|pwkv} (zuerst unter dem
                  Titel \emph{Dante}\pwindex{Federn, Karl 02.02.1868 – 22.03.1943@\textsc{Federn, Karl} (02.02.1868 – 22.03.1943), \emph{Schriftsteller, Übersetzer}!Dante1899@\strich\emph{Dante} {[}1899{]}|pwk} erschienen, später auch unter \emph{Dante und seine Zeit}\pwindex{Federn, Karl 02.02.1868 – 22.03.1943@\textsc{Federn, Karl} (02.02.1868 – 22.03.1943), \emph{Schriftsteller, Übersetzer}!Dante1899@\strich\emph{Dante} {[}1899{]}|pwk}) im Mai 1900 (vgl. Arthur Schnitzler an Georg Brandes, 3. 5. 1900).}}}\label{K_L02908-2h} ſchon geleſen?\pend
           \pstart
           Viele treue Grüße! {\\[\baselineskip]}Dein {\\[\baselineskip]}\spacefill\mbox{Paul Goldmann}\pend
           \leftskip=0em{}
         
         \endnumbering\mylabel{h}\end{ledgroupsized}\begin{anhang}\end{anhang}\newcommand{\dateiname}{L02908}\newcommand{\titel}{Paul Goldmann an Arthur Schnitzler, 24. 3. [1900]}\newcommand{\editorInnen}{Martin Anton Müller und Laura Untner}%% latex-leseansicht-abspann.tex
%% Abspann für die Leseansicht.
%% Der Schalter \ifkorrekturansicht ist bereits durch den Vorspann gesetzt.

%% latex-abspann.tex
%% Gemeinsamer Abspann für Korrekturansicht und Leseansicht.
%% Setzt den Schalter \ifkorrekturansicht voraus (gesetzt in den
%% einbindenden Dateien latex-korrekturansicht-abspann.tex bzw.
%% latex-leseansicht-abspann.tex).
%% ---------------------------------------------------------------

\normalsize

% Das esempio-Environment wird nur in der Leseansicht benötigt
\ifkorrekturansicht\else
\newenvironment{esempio}[3]%
{
    \vspace{1.5ex}
    \rlap{\underline{#1}}
    \par
    \setlength{\parindent}{0cm}
    \nopagebreak
    \leftskip=#2cm
    \rightskip=#3cm
}
{
    \par
}
\fi

\doendnotes{C}
\bigskip
\vfill

\clearpage

\footnotesize

\ifkorrekturansicht
  \lohead{\textsc{register}}
\fi

% theindex-Environment neu definieren ohne reledmac
\makeatletter
\renewenvironment{theindex}{%
  \ifkorrekturansicht
    \section*{\indexname}%
  \else
    \subsubsection*{Index der erwähnten Entitäten}%
  \fi
  \setlength{\parindent}{0pt}%
  \setlength{\parskip}{0pt plus 0.3pt}%
  \let\item\@idxitem
}{%
  \ifkorrekturansicht\clearpage\fi
}
\makeatother

\IfFileExists{\jobname-pw.ind}{\input{\jobname-pw.ind}}{}

% Quellenangabe nur in der Leseansicht
\ifkorrekturansicht\else
% Fallback-Definitionen, falls die .tex-Datei \titel etc. nicht gesetzt hat
\providecommand{\titel}{}
\providecommand{\editorInnen}{}
\providecommand{\dateiname}{\jobname}

\vspace{3cm}

\vfill

\footnotesize
\textsc{Quelle}: \titel. Herausgegeben von {\editorInnen}. In: \emph{Arthur Schnitzler: Briefwechsel mit Autorinnen und Autoren}.
 Digitale Edition, https://schnitzler-briefe.acdh.oeaw.ac.at/{\dateiname}.html (Stand \today)
\fi

\end{document}


      