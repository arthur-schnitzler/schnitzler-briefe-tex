%% latex-leseansicht-vorspann.tex
%% Vorspann für die Leseansicht.
%% Lädt die gemeinsame Datei latex-vorspann.tex mit nicht gesetztem Schalter.

\newif\ifkorrekturansicht
\korrekturansichtfalse

\input{../tex-inputs/latex-vorspann}


\section[ Paul Goldmann an Arthur Schnitzler, 24. 3. [1900]]{L02908 Paul Goldmann an Arthur Schnitzler,  24. 3. [1900]}
\nopagebreak\mylabel{L02908v}
\rehead{ }\normalsize\beginnumbering\briefempfaengerindex{Schnitzler, Arthur@\textsc{Schnitzler, Arthur}!zzzGoldmann, Paul@\emph{von Paul Goldmann}!1900-03-241@{24. 3. [1900]}|(be}
\toendnotes[C]{\smallbreak\pagebreak[2]}
\correspDesc{Versand  durch Paul Goldmann am 24. 3. [1900] in Berlin
\newline{}Erhalt  durch Arthur Schnitzler im Zeitraum [25. 3. 1900
                  – 27. 3. 1900?] in Wien}\toendnotes[C]{\smallbreak}
\Standort{DLA, A:Schnitzler, HS.NZ85.1.3170.}
\physDesc{Brief, 1 Blatt, 1 Seite, 361 Zeichen
\newline{}Handschrift: blaue Tinte, deutsche Kurrent
\newline{}Schnitzler: 1) mit Bleistift das Jahr »900« vermerkt  2) mit rotem Buntstift eine Unterstreichung}\toendnotes[C]{\smallbreak}
\pstart
           {\pb}\textcolor{gray}{\textbf{DESSAUERSTRASSE 19}}\oindex{Dessauer Straße@\textbf{Dessauer Straße}, \emph{Straße}|pw}\pend
           
\pstart
           \raggedleft{}Berlin\oindex{Berlin@\textbf{Berlin}, \emph{Hauptstadt}|pw}, 24. März.\pend
           
\pstart{}Mein lieber Freund,\pend\vspace{0.5em}
\pstart
           Ich danke Dir für die Überſendung des \label{K_L02908-1v}\edtext{\textsc{Hoffmannsthal\pwindex{Hofmannsthal, Hugo von 1.\,2.\,1874 Wien – 15.\,7.\,1929 Rodaun@\textsc{Hofmannsthal, Hugo von} (1.\,2.\,1874 Wien – 15.\,7.\,1929 Rodaun), \emph{Schriftsteller}|pw}}’ſchen Vorſpiels\pwindex{Hofmannsthal, Hugo von 1.\,2.\,1874 Wien – 15.\,7.\,1929 Rodaun@\textsc{Hofmannsthal, Hugo von} (1.\,2.\,1874 Wien – 15.\,7.\,1929 Rodaun), \emph{Schriftsteller}!Vorspiel zur Antigone des Sophokles@\strich\emph{Vorspiel zur Antigone des Sophokles}|pwv}}{\lemma{\textnormal{\emph{Hoffmannsthal’schen Vorspiels}}}\Cendnote{\textnormal{Hugo von Hofmannsthal\pwindex{Hofmannsthal, Hugo von 1.\,2.\,1874 Wien – 15.\,7.\,1929 Rodaun@\textsc{Hofmannsthal, Hugo von} (1.\,2.\,1874 Wien – 15.\,7.\,1929 Rodaun), \emph{Schriftsteller}|pwk} hatte Schnitzler gebeten, sein \emph{Vorspiel zur Antigone des Sophokles}\pwindex{Hofmannsthal, Hugo von 1.\,2.\,1874 Wien – 15.\,7.\,1929 Rodaun@\textsc{Hofmannsthal, Hugo von} (1.\,2.\,1874 Wien – 15.\,7.\,1929 Rodaun), \emph{Schriftsteller}!Vorspiel zur Antigone des Sophokles@\strich\emph{Vorspiel zur Antigone des Sophokles}|pwk} an Goldmann\pwindex{Goldmann, Paul 31.\,1.\,1865 Breslau – 25.\,9.\,1935 Wien@\textsc{Goldmann, Paul} (31.\,1.\,1865 Breslau – 25.\,9.\,1935 Wien), \emph{Schriftsteller, Journalist}|pwk} zu übersenden. Vgl. XXXX Auszeichnungsfehler: Dokument L01021 nicht gefunden, XXXX Auszeichnungsfehler: Dokument L01023 nicht gefunden
                  und XXXX Auszeichnungsfehler: Dokument L01024 nicht gefunden.}}}\label{K_L02908-1}. Ich finde es
               abſcheulich.\pend
           
\pstart
           Haſt Du meinen \label{K_L02908-2v}\edtext{Brief von \strikeout{\textcolor{gray}{×}}{ }vorgeſtern}{\lemma{\textnormal{\emph{Brief von  vorgestern}}}\Cendnote{\textnormal{XXXX Auszeichnungsfehler: Dokument L02907 nicht gefunden.
               }}}\label{K_L02908-2} nicht erhalten?\pend
           
\pstart
           Ich danke Dir für die \label{K_L02908-3v}\edtext{Mittheilung der
               Äußerung der Frau \textsc{Bürger}\pwindex{Burger, Caroline 11.\,7.\,1869 Wien – 15.\,3.\,1959 ebd.@\textsc{Burger, Caroline} (11.\,7.\,1869 Wien – 15.\,3.\,1959 ebd.)|pwu}}{\lemma{\textnormal{\emph{Mittheilung … Bürger}}}\Cendnote{\textnormal{Die Stelle ist nicht mit Sicherheit aufzuschlüsseln. Es könnte sich um eine
                  Aussage von Caroline Burger\pwindex{Burger, Caroline 11.\,7.\,1869 Wien – 15.\,3.\,1959 ebd.@\textsc{Burger, Caroline} (11.\,7.\,1869 Wien – 15.\,3.\,1959 ebd.)|pwk} handeln, die
                  ältere Schwester von Marie Reinhard\pwindex{Reinhard, Marie 13.\,3.\,1871 Wien – 18.\,3.\,1899 ebd.@\textsc{Reinhard, Marie} (13.\,3.\,1871 Wien – 18.\,3.\,1899 ebd.), \emph{Gesangspädagogin}|pwk}, mit der
                     Schnitzler in Kontakt stand.}}}\label{K_L02908-3}, die
               mich{ }ſehr gefreut hat.\pend
           
\pstart
           Haſt Du die prachtvolle \label{K_L02908-4v}\edtext{\textsc{Dante\pwindex{Dante Alighieri um 1265 Florenz – 22.\,9.\,1321 Ravenna@\textsc{Dante Alighieri} (um 1265 Florenz – 22.\,9.\,1321 Ravenna), \emph{Schriftsteller}|pw}}-Biographie\pwindex{Federn, Karl 2.\,2.\,1868 Wien – 22.\,3.\,1943 London@\textsc{Federn, Karl} (2.\,2.\,1868 Wien – 22.\,3.\,1943 London), \emph{Schriftsteller, Übersetzer}!Dante@\strich\emph{Dante}|pw} von \textsc{Federn\pwindex{Federn, Karl 2.\,2.\,1868 Wien – 22.\,3.\,1943 London@\textsc{Federn, Karl} (2.\,2.\,1868 Wien – 22.\,3.\,1943 London), \emph{Schriftsteller, Übersetzer}|pw}}}{\lemma{\textnormal{\emph{Dante-Biographie von Federn}}}\Cendnote{\textnormal{Schnitzler las Karl Federns\pwindex{Federn, Karl 2.\,2.\,1868 Wien – 22.\,3.\,1943 London@\textsc{Federn, Karl} (2.\,2.\,1868 Wien – 22.\,3.\,1943 London), \emph{Schriftsteller, Übersetzer}|pwk}{ }Dante\pwindex{Dante Alighieri um 1265 Florenz – 22.\,9.\,1321 Ravenna@\textsc{Dante Alighieri} (um 1265 Florenz – 22.\,9.\,1321 Ravenna), \emph{Schriftsteller}|pwk}-Biographie\pwindex{Federn, Karl 2.\,2.\,1868 Wien – 22.\,3.\,1943 London@\textsc{Federn, Karl} (2.\,2.\,1868 Wien – 22.\,3.\,1943 London), \emph{Schriftsteller, Übersetzer}!Dante@\strich\emph{Dante}|pwkv} (zuerst unter dem
                  Titel \emph{Dante}\pwindex{Federn, Karl 2.\,2.\,1868 Wien – 22.\,3.\,1943 London@\textsc{Federn, Karl} (2.\,2.\,1868 Wien – 22.\,3.\,1943 London), \emph{Schriftsteller, Übersetzer}!Dante@\strich\emph{Dante}|pwk} erschienen, später auch unter \emph{Dante und seine Zeit}\pwindex{Federn, Karl 2.\,2.\,1868 Wien – 22.\,3.\,1943 London@\textsc{Federn, Karl} (2.\,2.\,1868 Wien – 22.\,3.\,1943 London), \emph{Schriftsteller, Übersetzer}!Dante@\strich\emph{Dante}|pwk}) im Mai 1900 (vgl. XXXX Auszeichnungsfehler: Dokument L01034 nicht gefunden).}}}\label{K_L02908-4}{ }ſchon geleſen?\pend
           
\pstart
           Viele treue Grüße! {\\[\baselineskip]}Dein {\\[\baselineskip]}\spacefill\mbox{Paul Goldmann}\pend
           \leftskip=0em{}\selectlanguage{ngerman}\endnumbering\briefempfaengerindex{Schnitzler, Arthur@\textsc{Schnitzler, Arthur}!zzzGoldmann, Paul@\emph{von Paul Goldmann}!1900-03-241@{24. 3. [1900]}|)be}\mylabel{L02908h}  \newcommand{\dateiname}{L02908}\newcommand{\titel}{Paul Goldmann an Arthur Schnitzler, 24. 3. [1900]}\newcommand{\editorInnen}{Martin Anton Müller und Laura Untner}%% latex-leseansicht-abspann.tex
%% Abspann für die Leseansicht.
%% Der Schalter \ifkorrekturansicht ist bereits durch den Vorspann gesetzt.

%% latex-abspann.tex
%% Gemeinsamer Abspann für Korrekturansicht und Leseansicht.
%% Setzt den Schalter \ifkorrekturansicht voraus (gesetzt in den
%% einbindenden Dateien latex-korrekturansicht-abspann.tex bzw.
%% latex-leseansicht-abspann.tex).
%% ---------------------------------------------------------------

\normalsize

% Das esempio-Environment wird nur in der Leseansicht benötigt
\ifkorrekturansicht\else
\newenvironment{esempio}[3]%
{
    \vspace{1.5ex}
    \rlap{\underline{#1}}
    \par
    \setlength{\parindent}{0cm}
    \nopagebreak
    \leftskip=#2cm
    \rightskip=#3cm
}
{
    \par
}
\fi

\doendnotes{C}
\bigskip
\vfill

\clearpage

\footnotesize

\ifkorrekturansicht
  \lohead{\textsc{register}}
\fi

% theindex-Environment neu definieren ohne reledmac
\makeatletter
\renewenvironment{theindex}{%
  \ifkorrekturansicht
    \section*{\indexname}%
  \else
    \subsubsection*{Index der erwähnten Entitäten}%
  \fi
  \setlength{\parindent}{0pt}%
  \setlength{\parskip}{0pt plus 0.3pt}%
  \let\item\@idxitem
}{%
  \ifkorrekturansicht\clearpage\fi
}
\makeatother

\IfFileExists{\jobname-pw.ind}{\input{\jobname-pw.ind}}{}

% Quellenangabe nur in der Leseansicht
\ifkorrekturansicht\else
% Fallback-Definitionen, falls die .tex-Datei \titel etc. nicht gesetzt hat
\providecommand{\titel}{}
\providecommand{\editorInnen}{}
\providecommand{\dateiname}{\jobname}

\vspace{3cm}

\vfill

\footnotesize
\textsc{Quelle}: \titel. Herausgegeben von {\editorInnen}. In: \emph{Arthur Schnitzler: Briefwechsel mit Autorinnen und Autoren}.
 Digitale Edition, https://schnitzler-briefe.acdh.oeaw.ac.at/{\dateiname}.html (Stand \today)
\fi

\end{document}


