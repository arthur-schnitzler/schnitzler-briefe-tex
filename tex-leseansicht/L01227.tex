%% latex-leseansicht-vorspann.tex
%% Vorspann für die Leseansicht.
%% Lädt die gemeinsame Datei latex-vorspann.tex mit nicht gesetztem Schalter.

\newif\ifkorrekturansicht
\korrekturansichtfalse

\input{../tex-inputs/latex-vorspann}


         
         \renewcommand{\erwaehntePersonen}{Personen: Hermann Bahr}
         \renewcommand{\erwaehnteInstitutionen}{Institutionen: Neues Wiener Tagblatt}
         \renewcommand{\erwaehnteOrte}{Orte: Matrei am Brenner, Ober Sankt Veit, Veitlissengasse, Wien, XIII., Hietzing}
         \renewcommand{\erwaehnteWerke}{Werke: Entgegen}
               \section[Hugo von Hofmannsthal und Arthur Schnitzler an Hermann Bahr, 3. 7. 1902]{ Hugo von Hofmannsthal und Arthur Schnitzler an Hermann Bahr,
               3. 7. 1902}\nopagebreak\mylabel{v}\rehead{ }\begin{ledgroupsized}[t]{13cm}\normalsize\beginnumbering \toendnotes[C]{\smallbreak\pagebreak[2]} \Standort{TMW, HS AM 49090 Ba.}
\physDesc{Bildpostkarte
\newline{}Handschrift Hugo von Hofmannsthal: 1) Bleistift, deutsche Kurrent\hspace{1em}2) Bleistift, lateinische Kurrent (\noindent{}Adresse)\hspace{1em}\newline{}Handschrift Arthur Schnitzler: Bleistift, deutsche Kurrent\newline{}Versand: 1) Stempel: »\nobreak{}\oindex{Matrei am Brenner@\textbf{Matrei am Brenner}|pwk}Deutsch-Matrei, 3/7 {[}1902{]}\nobreak{}«.   2) Stempel: »\nobreak{}\oindex{XIII., Hietzing@\textbf{XIII., Hietzing}|pwk}Wien 13, 4. 7. 02, 11. V, Bestellt\nobreak{}«. \newline{}Ordnung: Lochung }\buchAbdrucke{\weitereDrucke{Hermann Bahr, Arthur Schnitzler: \emph{Briefwechsel, Aufzeichnungen, Dokumente (1891–1931)}. Hg. Kurt Ifkovits und Martin Anton Müller. Göttingen: \emph{Wallstein} 2018, S. 240.} }\toendnotes[C]{\smallbreak}\pstart{}{\pb}Herrn Hermann
                  Bahr\pend{}\pstart{}Redacteur\pend{}\pstart{}Wien\oindex{Wien@\textbf{Wien}|pw}\pend{}\pstart{}XIII\textsubscript{7}
                     Veitlissengasse\oindex{Veitlissengasse@\textbf{Veitlissengasse}|pw}\pend{}\pstart{}in Ober St Veit\oindex{Ober Sankt Veit@\textbf{Ober Sankt Veit}|pw}.\pend{}{\bigskip}\pstart
           \noindent{}\centering{}\textcolor{gray}{\textbf{{\pb}MATREI\oindex{Matrei am Brenner@\textbf{Matrei am Brenner}|pw}.}}\pend
           \pstart
           \label{T_L01227_1v}\edtext{Das was Sie}{\lemma{\textnormal{\emph{Das was Sie}}}\Cendnote{\textnormal{quer am rechten Rand}}}\label{T_L01227_1h} über’n \label{K_L01227_1v}\edtext{Automobil\pwindex{Bahr, Hermann 19.07.1863 – 15.01.1934@\textsc{Bahr, Hermann} (19.07.1863 – 15.01.1934), \emph{Schriftsteller, Kritiker}!Entgegen01. 07. 1902@\strich\emph{Entgegen} {[}01. 07. 1902{]}|pwv}}{\lemma{\textnormal{\emph{Automobil}}}\Cendnote{\textnormal{In \emph{Entgegen}\pwindex{Bahr, Hermann 19.07.1863 – 15.01.1934@\textsc{Bahr, Hermann} (19.07.1863 – 15.01.1934), \emph{Schriftsteller, Kritiker}!Entgegen01. 07. 1902@\strich\emph{Entgegen} {[}01. 07. 1902{]}|pwk} (\emph{Neues Wiener Tagblatt}\orgindex{Neues Wiener Tagblatt@Neues Wiener Tagblatt|pwk}, Jg. 36, Nr. 179,
                        1. 7. 1902, S. 1–2) schildert Bahr\pwindex{Bahr, Hermann 19.07.1863 – 15.01.1934@\textsc{Bahr, Hermann} (19.07.1863 – 15.01.1934), \emph{Schriftsteller, Kritiker}|pwk} ein Automobilrennen.}}}\label{K_L01227_1h} geſchrieben haben, war \uline{ſehr} gut.\pend
           \pstart
           3 VII 02.\pend
           \pstart Viele Grüße\hspace*{1.5em}\spacefill\mbox{Hugo}\pend{}\pstart
           \noindent{}{[}hs. Schnitzler:{]} \label{T_L01227_2v}\edtext{Ich auch}{\lemma{\textnormal{\emph{Ich auch}}}\Cendnote{\textnormal{am rechten Rand auf dem Kopf}}}\label{T_L01227_2h}\pend
           \pstart \spacefill\mbox{Arthur.}\pend{}
         
         \endnumbering\mylabel{h}\end{ledgroupsized}  \newcommand{\dateiname}{L01227}\newcommand{\titel}{Hugo von Hofmannsthal und Arthur Schnitzler an Hermann Bahr, 3. 7. 1902}\newcommand{\editorInnen}{ Kurt Ifkovits,  Martin Anton Müller}%% latex-leseansicht-abspann.tex
%% Abspann für die Leseansicht.
%% Der Schalter \ifkorrekturansicht ist bereits durch den Vorspann gesetzt.

%% latex-abspann.tex
%% Gemeinsamer Abspann für Korrekturansicht und Leseansicht.
%% Setzt den Schalter \ifkorrekturansicht voraus (gesetzt in den
%% einbindenden Dateien latex-korrekturansicht-abspann.tex bzw.
%% latex-leseansicht-abspann.tex).
%% ---------------------------------------------------------------

\normalsize

% Das esempio-Environment wird nur in der Leseansicht benötigt
\ifkorrekturansicht\else
\newenvironment{esempio}[3]%
{
    \vspace{1.5ex}
    \rlap{\underline{#1}}
    \par
    \setlength{\parindent}{0cm}
    \nopagebreak
    \leftskip=#2cm
    \rightskip=#3cm
}
{
    \par
}
\fi

\doendnotes{C}
\bigskip
\vfill

\clearpage

\footnotesize

\ifkorrekturansicht
  \lohead{\textsc{register}}
\fi

% theindex-Environment neu definieren ohne reledmac
\makeatletter
\renewenvironment{theindex}{%
  \ifkorrekturansicht
    \section*{\indexname}%
  \else
    \subsubsection*{Index der erwähnten Entitäten}%
  \fi
  \setlength{\parindent}{0pt}%
  \setlength{\parskip}{0pt plus 0.3pt}%
  \let\item\@idxitem
}{%
  \ifkorrekturansicht\clearpage\fi
}
\makeatother

\IfFileExists{\jobname-pw.ind}{\input{\jobname-pw.ind}}{}

% Quellenangabe nur in der Leseansicht
\ifkorrekturansicht\else
% Fallback-Definitionen, falls die .tex-Datei \titel etc. nicht gesetzt hat
\providecommand{\titel}{}
\providecommand{\editorInnen}{}
\providecommand{\dateiname}{\jobname}

\vspace{3cm}

\vfill

\footnotesize
\textsc{Quelle}: \titel. Herausgegeben von {\editorInnen}. In: \emph{Arthur Schnitzler: Briefwechsel mit Autorinnen und Autoren}.
 Digitale Edition, https://schnitzler-briefe.acdh.oeaw.ac.at/{\dateiname}.html (Stand \today)
\fi

\end{document}


      