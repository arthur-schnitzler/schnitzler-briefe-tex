%% latex-korrekturansicht-vorspann.tex
%% Vorspann für die Korrekturansicht.
%% Lädt die gemeinsame Datei latex-vorspann.tex mit gesetztem Schalter.

\newif\ifkorrekturansicht
\korrekturansichttrue

\input{../tex-inputs/latex-vorspann}


\section[Hugo von Hofmannsthal und Arthur Schnitzler an Hermann Bahr, 3. 7. 1902]{L01227 Hugo von Hofmannsthal und Arthur Schnitzler an Hermann Bahr,
               3. 7. 1902}
\nopagebreak\mylabel{L01227v}
\rehead{ }\normalsize\beginnumbering\briefempfaengerindex{Bahr, Hermann@\textsc{Bahr, Hermann}!zzzSchnitzler, Arthur@\emph{von Arthur Schnitzler}!1902-07-031@{3. 7. 1902}|(be}\briefempfaengerindex{Bahr, Hermann@\textsc{Bahr, Hermann}!zzzHofmannsthal, Hugo von@\emph{von Hugo von Hofmannsthal}!1902-07-031@{3. 7. 1902}|(be}
\toendnotes[C]{\smallbreak\pagebreak[2]}\Standort{TMW, HS AM 49090 Ba.}
\physDesc{Bildpostkarte, 162 Zeichen
\newline{}Handschrift Hugo von Hofmannsthal: 1) Bleistift, deutsche Kurrent\hspace{1em}2) Bleistift, lateinische Kurrent (\noindent{}Adresse)\hspace{1em}
\newline{}Handschrift Arthur Schnitzler: Bleistift, deutsche Kurrent
\newline{}Versand: 1) Stempel: »\nobreak{}\oindex{Matrei am Brenner@\textbf{Matrei am Brenner}, \emph{A.ADM3}|pwk}Deutsch-Matrei, 3/7 {[}1902{]}\nobreak{}«.   2) Stempel: »\nobreak{}\oindex{XIII., Hietzing@\textbf{XIII., Hietzing}, \emph{A.ADM3}|pwk}Wien 13, 4. 7. 02, 11. V, Bestellt\nobreak{}«. 
\newline{}Ordnung: Lochung }
\buchAbdrucke{\weitereDrucke{Hermann Bahr, Arthur Schnitzler: \emph{Briefwechsel, Aufzeichnungen, Dokumente (1891–1931)}. Göttingen: \emph{Wallstein} 2018, S. 240.} }\toendnotes[C]{\smallbreak}\pstart{}{\pb}Herrn Hermann
                  Bahr\pend{}\pstart{}Redacteur\pend{}\pstart{}Wien\oindex{Wien@\textbf{Wien}, \emph{A.ADM2}|pw}\pend{}\pstart{}XIII\textsubscript{7}
                     Veitlissengasse\oindex{Veitlissengasse@\textbf{Veitlissengasse}, \emph{Straße (K.STR)}|pw}\pend{}\pstart{}in Ober St Veit\oindex{Ober Sankt Veit@\textbf{Ober Sankt Veit}, \emph{P.PPLX}|pw}.\pend{}{\bigskip}
\pstart
           \noindent{}\centering{}{\pb}\textcolor{gray}{\textbf{MATREI\oindex{Matrei am Brenner@\textbf{Matrei am Brenner}, \emph{A.ADM3}|pw}.}}\pend
           \vspace{1em}
\pstart
           \noindent{}{\pb}\label{T_L01227-1v}\edtext{Das was Sie}{\lemma{\textnormal{\emph{Das was Sie}}}\Cendnote{\textnormal{quer am rechten Rand}}}\label{T_L01227-1} über’n \label{K_L01227-1v}\edtext{Automobil\pwindex{Entgegen@\emph{Entgegen}|pwv}}{\lemma{\textnormal{\emph{Automobil}}}\Cendnote{\textnormal{In \emph{Entgegen}\pwindex{Entgegen@\emph{Entgegen}|pwk} (\emph{Neues Wiener Tagblatt}\orgindex{Neues Wiener Tagblatt@Neues Wiener Tagblatt|pwk}, Jg. 36, Nr. 179,
                        1. 7. 1902, S. 1–2) schildert Bahr\pwindex{Bahr, Hermann 19.07.1863 – 15.01.1934@\textsc{Bahr, Hermann} (19.07.1863 – 15.01.1934), \emph{Schriftsteller/Schriftstellerin, Kritiker/Kritikerin}|pwk} ein Automobilrennen.}}}\label{K_L01227-1} geſchrieben haben, war \uline{ſehr} gut.\pend
           
\pstart
           3 VII 02.\pend
           \pstart Viele Grüße\hspace*{1.5em}\spacefill\mbox{Hugo}\pend{}\selectlanguage{ngerman}\vspace{1em}
\pstart
           \noindent{}{[}hs. :{]} \label{T_L01227-2v}\edtext{Ich auch}{\lemma{\textnormal{\emph{Ich auch}}}\Cendnote{\textnormal{am rechten Rand auf dem Kopf}}}\label{T_L01227-2}\pend
           \pstart \spacefill\mbox{Arthur.}\pend{}\selectlanguage{ngerman}\endnumbering\briefempfaengerindex{Bahr, Hermann@\textsc{Bahr, Hermann}!zzzSchnitzler, Arthur@\emph{von Arthur Schnitzler}!1902-07-031@{3. 7. 1902}|)be}\briefempfaengerindex{Bahr, Hermann@\textsc{Bahr, Hermann}!zzzHofmannsthal, Hugo von@\emph{von Hugo von Hofmannsthal}!1902-07-031@{3. 7. 1902}|)be}\mylabel{L01227h}  \normalsize

\doendnotes{C}
\bigskip
\vfill

\clearpage

\footnotesize

\lohead{\textsc{register}}

% Definiere theindex-Environment komplett neu ohne reledmac
\makeatletter
\renewenvironment{theindex}{%
  \section*{\indexname}%
  \setlength{\parindent}{0pt}%
  \setlength{\parskip}{0pt plus 0.3pt}%
  \let\item\@idxitem
}{%
  \clearpage
}
\makeatother

\IfFileExists{\jobname-pw.ind}{\input{\jobname-pw.ind}}{}

\end{document}

      