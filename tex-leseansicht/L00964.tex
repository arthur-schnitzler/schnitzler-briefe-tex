%% latex-korrekturansicht-vorspann.tex
%% Vorspann für die Korrekturansicht.
%% Lädt die gemeinsame Datei latex-vorspann.tex mit gesetztem Schalter.

\newif\ifkorrekturansicht
\korrekturansichttrue

\input{../tex-inputs/latex-vorspann}


\section[Arthur Schnitzler an Gerhart Hauptmann, 25. 8. 1899]{L00964 Arthur Schnitzler an Gerhart Hauptmann, 25. 8. 1899}
\nopagebreak\mylabel{L00964v}
\rehead{ }\normalsize\beginnumbering\briefempfaengerindex{Hauptmann, Gerhart@\textsc{Hauptmann, Gerhart}!zzzSchnitzler, Arthur@\emph{von Arthur Schnitzler}!1899-08-251@{25. 8. 1899}|(be}
\toendnotes[C]{\smallbreak\pagebreak[2]}\Standort{Staatsbibliothek Berlin – Preußischer Kulturbesitz, GHBrBl A:Schnitzler (2,3).}
\physDesc{Brief, 1 Blatt, 4 Seiten, 903 Zeichen
\newline{}Handschrift: schwarze Tinte, deutsche Kurrent
\newline{}Ordnung: mit Bleistift von unbekannter Hand
                                 nummeriert: »2« }
\buchAbdrucke{\weitereDrucke{Arthur Schnitzler: \emph{Briefe 1875–1912}. Frankfurt am Main: \emph{S. Fischer} 1981, S. 373.} }\toendnotes[C]{\smallbreak}
\pstart
           \raggedleft{}{\pb}Ischl, Rudolfshöhe\oindex{Hotel und Pension Rudolfshoehe (Leopold Petter)@\textbf{Hotel und Pension Rudolfshöhe (Leopold Petter)}, \emph{Hotel (K.HTL)}|pw}{\\}25. 8. 9\textcolor{gray}{9}. \pend
           
\pstart{}Lieber Herr Hauptmann,\pend\vspace{0.5em}
\pstart
           etwas verſpätet danke ich Ihnen für Ihre freundliche Antwort. Ich darf Ihnen wohl
               ſagen, dſs ich ſie ungefähr ſo erwartet und an Ihrer Stelle dieſelbe gegeben hätte.
               Nun iſt der Heraus{\pb}geber\pwindex{Singer, Isidor 16.01.1857 – 08.12.1927@\textsc{Singer, Isidor} (16.01.1857 – 08.12.1927), \emph{Journalist/Journalistin, Herausgeber/Herausgeberin, Soziologe/Soziologin}|pwv} von der ganzen Idee mit den
               vielen Namen und den großen Namen abgeko{\geminationm}en, was ich
               ſehr vernünftig finde.\pend
           
\pstart
           Ich bin jetzt in Iſchl\oindex{Bad Ischl@\textbf{Bad Ischl}, \emph{P.PPL}|pw}, Hofmannsthal\pwindex{Hofmannsthal, Hugo von 1874-02-01 – 1929-07-15@\textsc{Hofmannsthal, Hugo von} (1874-02-01 – 1929-07-15), \emph{Schriftsteller/Schriftstellerin}|pw} desgleichen, in derſelben Pension\oindex{Hotel und Pension Rudolfshoehe (Leopold Petter)@\textbf{Hotel und Pension Rudolfshöhe (Leopold Petter)}, \emph{Hotel (K.HTL)}|pwv}, und jeder von uns hat einen eigenen
                  {\pb}Balkon zum Dichten.\pend
           
\pstart
           Es freut mich dſs Sie ſich ſo freundlich meiner erinnern und mich bald einmal wieder
               zu sehen wünschen – aber ob \uline{inner}halb oder \uline{außer}halb der Stadtmauern kann ich Ihrem Brief nicht
               entnehmen: in Ihrer Schrift ſieht {\pb}»innen«
               genau ſo aus wie »außen« – ſo arg iſts bei mir hoffentlich nicht.\pend
           
\pstart
           Wie immer und wo i{\geminationm}er; Sie können mir glauben daſs es
               wenige Menſchen gibt, die ich ſo gerne bald wiederſehen möchte als Sie.\pend
           
\pstart
           Ganz der Ihre{\\[\baselineskip]}\spacefill\mbox{Arthur Schnitzler}\pend
           \leftskip=0em{}\selectlanguage{ngerman}\endnumbering\briefempfaengerindex{Hauptmann, Gerhart@\textsc{Hauptmann, Gerhart}!zzzSchnitzler, Arthur@\emph{von Arthur Schnitzler}!1899-08-251@{25. 8. 1899}|)be}\mylabel{L00964h}  \normalsize

\doendnotes{C}
\bigskip
\vfill

\clearpage

\footnotesize

\lohead{\textsc{register}}

% Definiere theindex-Environment komplett neu ohne reledmac
\makeatletter
\renewenvironment{theindex}{%
  \section*{\indexname}%
  \setlength{\parindent}{0pt}%
  \setlength{\parskip}{0pt plus 0.3pt}%
  \let\item\@idxitem
}{%
  \clearpage
}
\makeatother

\IfFileExists{\jobname-pw.ind}{\input{\jobname-pw.ind}}{}

\end{document}

      