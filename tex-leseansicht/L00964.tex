%% latex-leseansicht-vorspann.tex
%% Vorspann für die Leseansicht.
%% Lädt die gemeinsame Datei latex-vorspann.tex mit nicht gesetztem Schalter.

\newif\ifkorrekturansicht
\korrekturansichtfalse

\input{../tex-inputs/latex-vorspann}


         
         \renewcommand{\erwaehntePersonen}{Personen: Gerhart Hauptmann, Hugo von Hofmannsthal, Isidor Singer}
         \renewcommand{\erwaehnteOrte}{Orte: Bad Ischl, Hotel und Pension Rudolfshöhe (Leopold Petter), Insel Hiddensee}
         \renewcommand{\erwaehnteWerke}{}
               \section[Arthur Schnitzler an Gerhart Hauptmann, 25. 8. 1899]{ Arthur Schnitzler an Gerhart Hauptmann, 25. 8. 1899}\nopagebreak\mylabel{v}\rehead{ }\begin{ledgroupsized}[t]{13cm}\normalsize\beginnumbering \toendnotes[C]{\smallbreak\pagebreak[2]} \Standort{Staatsbibliothek Berlin – Preußischer Kulturbesitz, GHBrBl A:Schnitzler (2,3).}
\physDesc{Brief, 1 Blatt, 4 Seiten, 903 Zeichen
\newline{}Handschrift: schwarze Tinte, deutsche Kurrent
\newline{}Ordnung: mit Bleistift von unbekannter Hand
                                 nummeriert: »2« }\buchAbdrucke{\weitereDrucke{Arthur Schnitzler: \emph{Briefe 1875–1912}. Hg. Therese Nickl und Heinrich Schnitzler. Frankfurt am Main: \emph{S. Fischer} 1981, S. 373.} }\toendnotes[C]{\smallbreak}\pstart
           \raggedleft{}{\pb}Ischl, Rudolfshöhe\oindex{Hotel und Pension Rudolfshoehe (Leopold Petter)@\textbf{Hotel und Pension Rudolfshöhe (Leopold Petter)}|pw}{\\}25. 8. 9\textcolor{gray}{9}. \pend
           \pstart{}Lieber Herr Hauptmann,\pend\pstart
           etwas verſpätet danke ich Ihnen für Ihre freundliche Antwort. Ich darf Ihnen wohl
               ſagen, dſs ich ſie ungefähr ſo erwartet und an Ihrer Stelle dieſelbe gegeben hätte.
               Nun iſt der Heraus{\pb}geber\pwindex{Singer, Isidor 16.01.1857 – 08.12.1927@\textsc{Singer, Isidor} (16.01.1857 – 08.12.1927), \emph{Journalist, Herausgeber, Soziologe}|pwv} von der ganzen Idee mit den
               vielen Namen und den großen Namen abgeko{\geminationm}en, was ich
               ſehr vernünftig finde.\pend
           \pstart
           Ich bin jetzt in Iſchl\oindex{Bad Ischl@\textbf{Bad Ischl}|pw}, Hofmannsthal\pwindex{Hofmannsthal, Hugo von 1874-02-01 – 1929-07-15@\textsc{Hofmannsthal, Hugo von} (1874-02-01 – 1929-07-15), \emph{Schriftsteller}|pw} desgleichen, in derſelben Pension\oindex{Hotel und Pension Rudolfshoehe (Leopold Petter)@\textbf{Hotel und Pension Rudolfshöhe (Leopold Petter)}|pwv}, und jeder von uns hat einen eigenen
                  {\pb}Balkon zum Dichten.\pend
           \pstart
           Es freut mich dſs Sie ſich ſo freundlich meiner erinnern und mich bald einmal wieder
               zu sehen wünschen – aber ob \uline{inner}halb oder \uline{außer}halb der Stadtmauern kann ich Ihrem Brief nicht
               entnehmen: in Ihrer Schrift ſieht {\pb}»innen«
               genau ſo aus wie »außen« – ſo arg iſts bei mir hoffentlich nicht.\pend
           \pstart
           Wie immer und wo i{\geminationm}er; Sie können mir glauben daſs es
               wenige Menſchen gibt, die ich ſo gerne bald wiederſehen möchte als Sie.\pend
           \pstart
           Ganz der Ihre{\\[\baselineskip]}Arthur Schnitzler\pend
           \leftskip=0em{}
         
         \endnumbering\mylabel{h}\end{ledgroupsized}  \newcommand{\dateiname}{L00964}\newcommand{\titel}{Arthur Schnitzler an Gerhart Hauptmann, 25. 8. 1899}\newcommand{\editorInnen}{ Martin Anton Müller und Gerd-Hermann Susen}%% latex-leseansicht-abspann.tex
%% Abspann für die Leseansicht.
%% Der Schalter \ifkorrekturansicht ist bereits durch den Vorspann gesetzt.

%% latex-abspann.tex
%% Gemeinsamer Abspann für Korrekturansicht und Leseansicht.
%% Setzt den Schalter \ifkorrekturansicht voraus (gesetzt in den
%% einbindenden Dateien latex-korrekturansicht-abspann.tex bzw.
%% latex-leseansicht-abspann.tex).
%% ---------------------------------------------------------------

\normalsize

% Das esempio-Environment wird nur in der Leseansicht benötigt
\ifkorrekturansicht\else
\newenvironment{esempio}[3]%
{
    \vspace{1.5ex}
    \rlap{\underline{#1}}
    \par
    \setlength{\parindent}{0cm}
    \nopagebreak
    \leftskip=#2cm
    \rightskip=#3cm
}
{
    \par
}
\fi

\doendnotes{C}
\bigskip
\vfill

\clearpage

\footnotesize

\ifkorrekturansicht
  \lohead{\textsc{register}}
\fi

% theindex-Environment neu definieren ohne reledmac
\makeatletter
\renewenvironment{theindex}{%
  \ifkorrekturansicht
    \section*{\indexname}%
  \else
    \subsubsection*{Index der erwähnten Entitäten}%
  \fi
  \setlength{\parindent}{0pt}%
  \setlength{\parskip}{0pt plus 0.3pt}%
  \let\item\@idxitem
}{%
  \ifkorrekturansicht\clearpage\fi
}
\makeatother

\IfFileExists{\jobname-pw.ind}{\input{\jobname-pw.ind}}{}

% Quellenangabe nur in der Leseansicht
\ifkorrekturansicht\else
% Fallback-Definitionen, falls die .tex-Datei \titel etc. nicht gesetzt hat
\providecommand{\titel}{}
\providecommand{\editorInnen}{}
\providecommand{\dateiname}{\jobname}

\vspace{3cm}

\vfill

\footnotesize
\textsc{Quelle}: \titel. Herausgegeben von {\editorInnen}. In: \emph{Arthur Schnitzler: Briefwechsel mit Autorinnen und Autoren}.
 Digitale Edition, https://schnitzler-briefe.acdh.oeaw.ac.at/{\dateiname}.html (Stand \today)
\fi

\end{document}


      