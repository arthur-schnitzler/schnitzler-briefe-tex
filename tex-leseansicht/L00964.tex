%% latex-leseansicht-vorspann.tex
%% Vorspann für die Leseansicht.
%% Lädt die gemeinsame Datei latex-vorspann.tex mit nicht gesetztem Schalter.

\newif\ifkorrekturansicht
\korrekturansichtfalse

\input{../tex-inputs/latex-vorspann}


\section[Arthur Schnitzler an Gerhart Hauptmann, 25. 8. 1899]{L00964 Arthur Schnitzler an Gerhart Hauptmann, 25. 8. 1899}
\nopagebreak\mylabel{L00964v}
\rehead{ }\normalsize\beginnumbering\briefempfaengerindex{Hauptmann, Gerhart@\textsc{Hauptmann, Gerhart}!zzzSchnitzler, Arthur@\emph{von Arthur Schnitzler}!1899-08-251@{25. 8. 1899}|(be}
\toendnotes[C]{\smallbreak\pagebreak[2]}
\correspDesc{Versand  durch Arthur Schnitzler am 25. 8. 1899 in Bad Ischl
\newline{}Erhalt  durch Gerhart Hauptmann im Zeitraum [26. 8. 1899
                  – 30. 8. 1899?] in Insel Hiddensee}\toendnotes[C]{\smallbreak}
\Standort{Staatsbibliothek Berlin – Preußischer Kulturbesitz, GHBrBl A:Schnitzler (2,3).}
\physDesc{Brief, 1 Blatt, 4 Seiten, 903 Zeichen
\newline{}Handschrift: schwarze Tinte, deutsche Kurrent
\newline{}Ordnung: mit Bleistift von unbekannter Hand
                                 nummeriert: »2« }
\buchAbdrucke{\weitereDrucke{Arthur Schnitzler: \emph{Briefe 1875–1912}. Herausgegeben von Therese Nickl und Heinrich Schnitzler. Frankfurt am Main: \emph{S. Fischer} 1981, S. 373.} }\toendnotes[C]{\smallbreak}
\pstart
           \raggedleft{}{\pb}Ischl, Rudolfshöhe\oindex{Hotel und Pension Rudolfshöhe (Leopold Petter)@\textbf{Hotel und Pension Rudolfshöhe (Leopold Petter)}, \emph{Hotel}|pw}{\\}25. 8. 9\textcolor{gray}{9}.\pend
           
\pstart{}Lieber Herr Hauptmann,\pend\vspace{0.5em}
\pstart
           etwas verſpätet danke ich Ihnen für Ihre freundliche Antwort. Ich darf Ihnen wohl{ }ſagen, dſs ich{ }ſie ungefähr{ }ſo erwartet und an Ihrer Stelle dieſelbe gegeben hätte.
               Nun iſt der Heraus{\pb}geber\pwindex{Singer, Isidor 16.\,1.\,1857 Budapest – 8.\,12.\,1927 Wien@\textsc{Singer, Isidor} (16.\,1.\,1857 Budapest – 8.\,12.\,1927 Wien), \emph{Journalist, Herausgeber, Soziologe}|pwv} von der ganzen Idee mit den
               vielen Namen und den großen Namen abgeko{\geminationm}en, was ich{ }ſehr vernünftig finde.\pend
           
\pstart
           Ich bin jetzt in Iſchl\oindex{Bad Ischl@\textbf{Bad Ischl}|pw}, Hofmannsthal\pwindex{Hofmannsthal, Hugo von 1.\,2.\,1874 Wien – 15.\,7.\,1929 Rodaun@\textsc{Hofmannsthal, Hugo von} (1.\,2.\,1874 Wien – 15.\,7.\,1929 Rodaun), \emph{Schriftsteller}|pw} desgleichen, in derſelben Pension\oindex{Hotel und Pension Rudolfshöhe (Leopold Petter)@\textbf{Hotel und Pension Rudolfshöhe (Leopold Petter)}, \emph{Hotel}|pwv}, und jeder von uns hat einen eigenen
                  {\pb}Balkon zum Dichten.\pend
           
\pstart
           Es freut mich dſs Sie{ }ſich{ }ſo freundlich meiner erinnern und mich bald einmal wieder
               zu sehen wünschen – aber ob \uline{inner}halb oder \uline{außer}halb der Stadtmauern kann ich Ihrem Brief nicht
               entnehmen: in Ihrer Schrift{ }ſieht {\pb}»innen«
               genau{ }ſo aus wie »außen« –{ }ſo arg iſts bei mir hoffentlich nicht.\pend
           
\pstart
           Wie immer und wo i{\geminationm}er; Sie können mir glauben daſs es
               wenige Menſchen gibt, die ich{ }ſo gerne bald wiederſehen möchte als Sie.\pend
           
\pstart
           Ganz der Ihre{\\[\baselineskip]}\spacefill\mbox{Arthur Schnitzler}\pend
           \leftskip=0em{}\selectlanguage{ngerman}\endnumbering\briefempfaengerindex{Hauptmann, Gerhart@\textsc{Hauptmann, Gerhart}!zzzSchnitzler, Arthur@\emph{von Arthur Schnitzler}!1899-08-251@{25. 8. 1899}|)be}\mylabel{L00964h}  \newcommand{\dateiname}{L00964}\newcommand{\titel}{Arthur Schnitzler an Gerhart Hauptmann, 25. 8. 1899}\newcommand{\editorInnen}{Herausgegeben von Martin Anton Müller}%% latex-leseansicht-abspann.tex
%% Abspann für die Leseansicht.
%% Der Schalter \ifkorrekturansicht ist bereits durch den Vorspann gesetzt.

%% latex-abspann.tex
%% Gemeinsamer Abspann für Korrekturansicht und Leseansicht.
%% Setzt den Schalter \ifkorrekturansicht voraus (gesetzt in den
%% einbindenden Dateien latex-korrekturansicht-abspann.tex bzw.
%% latex-leseansicht-abspann.tex).
%% ---------------------------------------------------------------

\normalsize

% Das esempio-Environment wird nur in der Leseansicht benötigt
\ifkorrekturansicht\else
\newenvironment{esempio}[3]%
{
    \vspace{1.5ex}
    \rlap{\underline{#1}}
    \par
    \setlength{\parindent}{0cm}
    \nopagebreak
    \leftskip=#2cm
    \rightskip=#3cm
}
{
    \par
}
\fi

\doendnotes{C}
\bigskip
\vfill

\clearpage

\footnotesize

\ifkorrekturansicht
  \lohead{\textsc{register}}
\fi

% theindex-Environment neu definieren ohne reledmac
\makeatletter
\renewenvironment{theindex}{%
  \ifkorrekturansicht
    \section*{\indexname}%
  \else
    \subsubsection*{Index der erwähnten Entitäten}%
  \fi
  \setlength{\parindent}{0pt}%
  \setlength{\parskip}{0pt plus 0.3pt}%
  \let\item\@idxitem
}{%
  \ifkorrekturansicht\clearpage\fi
}
\makeatother

\IfFileExists{\jobname-pw.ind}{\input{\jobname-pw.ind}}{}

% Quellenangabe nur in der Leseansicht
\ifkorrekturansicht\else
% Fallback-Definitionen, falls die .tex-Datei \titel etc. nicht gesetzt hat
\providecommand{\titel}{}
\providecommand{\editorInnen}{}
\providecommand{\dateiname}{\jobname}

\vspace{3cm}

\vfill

\footnotesize
\textsc{Quelle}: \titel. Herausgegeben von {\editorInnen}. In: \emph{Arthur Schnitzler: Briefwechsel mit Autorinnen und Autoren}.
 Digitale Edition, https://schnitzler-briefe.acdh.oeaw.ac.at/{\dateiname}.html (Stand \today)
\fi

\end{document}


