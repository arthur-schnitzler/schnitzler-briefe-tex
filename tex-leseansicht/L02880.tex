%% latex-korrekturansicht-vorspann.tex
%% Vorspann für die Korrekturansicht.
%% Lädt die gemeinsame Datei latex-vorspann.tex mit gesetztem Schalter.

\newif\ifkorrekturansicht
\korrekturansichttrue

\input{../tex-inputs/latex-vorspann}


\section[ Paul Goldmann an Arthur Schnitzler, 16. 7. 1899]{L02880 Paul Goldmann an Arthur Schnitzler, 16. 7. 1899}
\nopagebreak\mylabel{L02880v}
\rehead{ }\normalsize\beginnumbering\briefempfaengerindex{Schnitzler, Arthur@\textsc{Schnitzler, Arthur}!zzzGoldmann, Paul@\emph{von Paul Goldmann}!1899-07-162@{16. 7. 1899}|(be}
\toendnotes[C]{\smallbreak\pagebreak[2]}\Standort{DLA, A:Schnitzler, HS.NZ85.1.3169.}
\physDesc{Brief, 1 Blatt, 2 Seiten, 816 Zeichen
\newline{}Handschrift: schwarze Tinte, deutsche Kurrent
\newline{}Schnitzler: mit rotem Buntstift eine Unterstreichung }\toendnotes[C]{\smallbreak}
\pstart
           {\pb}\textcolor{gray}{\textbf{\textbf{Frankfurter Zeitung}}}\orgindex{Frankfurter Zeitung@Frankfurter Zeitung|pw}\hfill \textcolor{gray}{\textbf{\textbf{Frankfurt a. M.\oindex{Frankfurt am Main@\textbf{Frankfurt am Main}, \emph{P.PPLA3}|pw},}}}{ }16. Juli \textcolor{gray}{\textbf{189}}9\pend
           
\pstart
           \textcolor{gray}{\textbf{und}}\pend
           
\pstart
           \textcolor{gray}{\textbf{Handelsblatt.}}\pend
           
\pstart
           \textcolor{gray}{\textbf{\textbf{Redaktion\orgindex{Frankfurter Zeitung@Frankfurter Zeitung|pwv}.}\noindent{}\textcolor{gray}{\textbf{Für die Redaktion\orgindex{Frankfurter Zeitung@Frankfurter Zeitung|pwv} beſtimmte Briefe und Sendungen wolle man
                                 \so{nicht} an die Perſon eines Redakteurs,
                              ſondern ſtets \textbf{an die Redaktion der Frankfurter Zeitung\orgindex{Frankfurter Zeitung@Frankfurter Zeitung|pw}} adreſſiren. }}}}\pend
           
\pstart
           \textcolor{gray}{\textbf{Telegramm-Adreſſe:}}\pend
           
\pstart
           \textcolor{gray}{\textbf{\textbf{Zeitung\orgindex{Frankfurter Zeitung@Frankfurter Zeitung|pwv}{ }Frankfurt Main\oindex{Frankfurt am Main@\textbf{Frankfurt am Main}, \emph{P.PPLA3}|pw}.}}}\pend
           
\pstart\center{}Mein lieber Freund,\pend\vspace{0.5em}
\pstart
           Alle meine Sommerpläne haben ſich wieder geändert. Nächſte Woche muß ich nach Bayreuth\oindex{Bayreuth@\textbf{Bayreuth}, \emph{P.PPLA2}|pw}, gegen Mitte Auguſt nach \textsc{Rennes}\oindex{Rennes@\textbf{Rennes}, \emph{P.PPLA}|pw}. Im September bin ich in Frankfurt\oindex{Frankfurt am Main@\textbf{Frankfurt am Main}, \emph{P.PPLA3}|pw}, um meinen Onkel\pwindex{Mamroth, Fedor 21.02.1851 – 25.06.1907@\textsc{Mamroth, Fedor} (21.02.1851 – 25.06.1907), \emph{Journalist/Journalistin, Kritiker/Kritikerin}|pwv} zu vertreten. Im Oktober
               will ich meinen Urlaub nehmen und nach Italien\oindex{Italien@\textbf{Italien}, \emph{A.PCLI}|pw}
                  (Florenz\oindex{Florenz@\textbf{Florenz}, \emph{P.PPLA}|pw} u. Rom\oindex{Rom@\textbf{Rom}, \emph{P.PPLC}|pw}) gehen. Könnteſt Du nicht da \label{K_L02880-1v}\edtext{mitkommen}{\lemma{\textnormal{\emph{mitkommen}}}\Cendnote{\textnormal{Dazu kam es nicht.}}}\label{K_L02880-1}? Jedenfalls, bitte, richte Dich ſo ein, daß Du \label{K_L02880-2v}\edtext{im September
               nach Frankfurt\oindex{Frankfurt am Main@\textbf{Frankfurt am Main}, \emph{P.PPLA3}|pw}}{\lemma{\textnormal{\emph{im … Frankfurt}}}\Cendnote{\textnormal{Schnitzler war vom 19. 9. 1899 bis zum 24. 9. 1899 in Frankfurt am Main\oindex{Frankfurt am Main@\textbf{Frankfurt am Main}, \emph{P.PPLA3}|pwk}.}}}\label{K_L02880-2} kommſt. Gib’ mir eine
                  {\pb}kurze Nachricht (Adreſſe immer Frankfurter Zeitung\orgindex{Frankfurter Zeitung@Frankfurter Zeitung|pwv}), wie es Dir geht, wie
               Du Dich in \label{K_L02880-3v}\edtext{Slavonien\oindex{Slawonien@\textbf{Slawonien}, \emph{L.RGN}|pw}}{\lemma{\textnormal{\emph{Slavonien}}}\Cendnote{\textnormal{Siehe Paul Goldmann an Arthur Schnitzler, 2. 7. 1899.
               }}}\label{K_L02880-3} behagt haſt, wo Du \label{K_L02880-4v}\edtext{jetzt
                  ſteckſt}{\lemma{\textnormal{\emph{jetzt
                  ſteckſt}}}\Cendnote{\textnormal{Schnitzler hielt sich noch in Wien\oindex{Wien@\textbf{Wien}, \emph{A.ADM2}|pwk} auf, reiste jedoch am 17. 7. 1899 nach Velden am Wörthersee\oindex{Velden am Woerthersee@\textbf{Velden am Wörthersee}, \emph{P.PPL}|pwk} ab.}}}\label{K_L02880-4}? Vielleicht bei
                  \textsc{Richard\pwindex{Beer-Hofmann, Richard 1866-07-11 – 1945-09-26@\textsc{Beer-Hofmann, Richard} (1866-07-11 – 1945-09-26), \emph{Schriftsteller/Schriftstellerin}|pw}}? Dann grüß’ ihn vielmals von mir und frag’ ihn, ob er \strikeout{m\textcolor{gray}{e}i} mein Buch\pwindex{Sommer in China. Reisebilder@\emph{Ein Sommer in China. Reisebilder}|pwv} bekommen hat? Der Schuft hat, wie gewöhnlich, nicht
               geantwortet.\pend
           
\pstart
           Viele treue Grüße! {\\[\baselineskip]}Dein {\\[\baselineskip]}\spacefill\mbox{Paul Goldmann}\pend
           \leftskip=0em{}
\pstart
           \noindent{}Kennſt Du \label{K_L02880-5v}\edtext{\textsc{Hettners\pwindex{Hettner, Hermann 1821-03-12 – 1882-05-29@\textsc{Hettner, Hermann} (1821-03-12 – 1882-05-29), \emph{Museumsdirektor/Museumsdirektorin, Literaturhistoriker/Literaturhistorikerin, Kunsthistoriker/Kunsthistorikerin}|pw}}{ }Franzöſiſche
                     Literaturgeſchichte\pwindex{Geschichte der franzoesischen Literatur im achtzehnten Jahrhundert@\emph{Geschichte der französischen Literatur im achtzehnten Jahrhundert}|pwv}}{\lemma{\textnormal{\emph{Hettners … Literaturgeſchichte}}}\Cendnote{\textnormal{Hermann Hettner\pwindex{Hettner, Hermann 1821-03-12 – 1882-05-29@\textsc{Hettner, Hermann} (1821-03-12 – 1882-05-29), \emph{Museumsdirektor/Museumsdirektorin, Literaturhistoriker/Literaturhistorikerin, Kunsthistoriker/Kunsthistorikerin}|pwk}: \emph{Geschichte der französischen Literatur im achtzehnten
                           Jahrhundert}\pwindex{Geschichte der franzoesischen Literatur im achtzehnten Jahrhundert@\emph{Geschichte der französischen Literatur im achtzehnten Jahrhundert}|pwk}. Braunschweig\oindex{Braunschweig@\textbf{Braunschweig}, \emph{P.PPLA3}|pwk}: \emph{Friedrich Vieweg und Sohn}\orgindex{Friedrich Vieweg und Sohn@Friedrich Vieweg und Sohn|pwk}{ }1860. Eine Lektüre des Werks\pwindex{Geschichte der franzoesischen Literatur im achtzehnten Jahrhundert@\emph{Geschichte der französischen Literatur im achtzehnten Jahrhundert}|pwkv} durch Schnitzler ist nicht
                     bekannt.}}}\label{K_L02880-5}? Feines, geſcheites, gediegenes Werk. Bitte zu leſen.\pend
           \selectlanguage{ngerman}\endnumbering\briefempfaengerindex{Schnitzler, Arthur@\textsc{Schnitzler, Arthur}!zzzGoldmann, Paul@\emph{von Paul Goldmann}!1899-07-162@{16. 7. 1899}|)be}\mylabel{L02880h}  \normalsize

\doendnotes{C}
\bigskip
\vfill

\clearpage

\footnotesize

\lohead{\textsc{register}}

% Definiere theindex-Environment komplett neu ohne reledmac
\makeatletter
\renewenvironment{theindex}{%
  \section*{\indexname}%
  \setlength{\parindent}{0pt}%
  \setlength{\parskip}{0pt plus 0.3pt}%
  \let\item\@idxitem
}{%
  \clearpage
}
\makeatother

\IfFileExists{\jobname-pw.ind}{\input{\jobname-pw.ind}}{}

\end{document}

      