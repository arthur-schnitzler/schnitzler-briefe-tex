%% latex-leseansicht-vorspann.tex
%% Vorspann für die Leseansicht.
%% Lädt die gemeinsame Datei latex-vorspann.tex mit nicht gesetztem Schalter.

\newif\ifkorrekturansicht
\korrekturansichtfalse

\input{../tex-inputs/latex-vorspann}


         
         \renewcommand{\erwaehntePersonen}{Personen: Richard Beer-Hofmann, Paul Goldmann, Hermann Hettner, Fedor Mamroth}
         \renewcommand{\erwaehnteInstitutionen}{Institutionen: Frankfurter Zeitung, Friedrich Vieweg und Sohn}
         \renewcommand{\erwaehnteOrte}{Orte: Bayreuth, Braunschweig, Florenz, Frankfurt am Main, Italien, Rennes, Rom, Slawonien, Velden am Wörthersee, Wien}
         \renewcommand{\erwaehnteWerke}{Werke: Ein Sommer in China. Reisebilder, Geschichte der französischen Literatur im achtzehnten Jahrhundert}
               \section[ Paul Goldmann an Arthur Schnitzler, 16. 7. 1899]{ Paul Goldmann an Arthur Schnitzler, 16. 7. 1899}\nopagebreak\mylabel{v}\rehead{ }\begin{ledgroupsized}[t]{13cm}\normalsize\beginnumbering \toendnotes[C]{\smallbreak\pagebreak[2]} \Standort{DLA, A:Schnitzler, HS.NZ85.1.3169.}
\physDesc{Brief, 1 Blatt, 2 Seiten, 816 Zeichen
\newline{}Handschrift: schwarze Tinte, deutsche Kurrent
\newline{}Schnitzler: mit rotem Buntstift eine Unterstreichung }\toendnotes[C]{\smallbreak}\pstart
           \noindent{}{\pb}\textcolor{gray}{\textbf{\textbf{Frankfurter Zeitung}}}\orgindex{Frankfurter Zeitung@Frankfurter Zeitung|pw}\hfill \textcolor{gray}{\textbf{\textbf{Frankfurt a. M.\oindex{Frankfurt am Main@\textbf{Frankfurt am Main}|pw},}}}{ }16. Juli \textcolor{gray}{\textbf{189}}9\pend
           \pstart
           \textcolor{gray}{\textbf{und}}\pend
           \pstart
           \textcolor{gray}{\textbf{Handelsblatt.}}\pend
           \pstart
           \textcolor{gray}{\textbf{\textbf{Redaktion\orgindex{Frankfurter Zeitung@Frankfurter Zeitung|pwv}.}\footnote{\noindent{}\textcolor{gray}{\textbf{Für die Redaktion\orgindex{Frankfurter Zeitung@Frankfurter Zeitung|pwv} beſtimmte Briefe und Sendungen wolle man
                                 \so{nicht} an die Perſon eines Redakteurs,
                              ſondern ſtets \textbf{an die Redaktion der Frankfurter Zeitung\orgindex{Frankfurter Zeitung@Frankfurter Zeitung|pw}} adreſſiren. }}}}}\pend
           \pstart
           \textcolor{gray}{\textbf{Telegramm-Adreſſe:}}\pend
           \pstart
           \textcolor{gray}{\textbf{\textbf{Zeitung\orgindex{Frankfurter Zeitung@Frankfurter Zeitung|pwv}{ }Frankfurt Main\oindex{Frankfurt am Main@\textbf{Frankfurt am Main}|pw}.}}}\pend
           \pstart\center{}Mein lieber Freund,\pend\pstart
           Alle meine Sommerpläne haben ſich wieder geändert. Nächſte Woche muß ich nach Bayreuth\oindex{Bayreuth@\textbf{Bayreuth}|pw}, gegen Mitte Auguſt nach \textsc{Rennes}\oindex{Rennes@\textbf{Rennes}|pw}. Im September bin ich in Frankfurt\oindex{Frankfurt am Main@\textbf{Frankfurt am Main}|pw}, um meinen Onkel\pwindex{Mamroth, Fedor 21.02.1851 – 25.06.1907@\textsc{Mamroth, Fedor} (21.02.1851 – 25.06.1907), \emph{Journalist, Kritiker}|pwv} zu vertreten. Im Oktober
               will ich meinen Urlaub nehmen und nach Italien\oindex{Italien@\textbf{Italien}|pw}
                  (Florenz\oindex{Florenz@\textbf{Florenz}|pw} u. Rom\oindex{Rom@\textbf{Rom}|pw}) gehen. Könnteſt Du nicht da \label{K_L02880-1v}\edtext{mitkommen}{\lemma{\textnormal{\emph{mitkommen}}}\Cendnote{\textnormal{nicht
                  geschehen}}}\label{K_L02880-1h}? Jedenfalls, bitte, richte Dich ſo ein, daß Du \label{K_L02880-2v}\edtext{im September
               nach Frankfurt\oindex{Frankfurt am Main@\textbf{Frankfurt am Main}|pw}}{\lemma{\textnormal{\emph{im … Frankfurt}}}\Cendnote{\textnormal{Schnitzler\pwindex{Schnitzler, Arthur 15.05.1862 – 21.10.1931@\textsc{Schnitzler, Arthur} (15.05.1862 – 21.10.1931), \emph{Schriftsteller, Mediziner}|pwk} war von 19. 9. 1899 bis 24. 9. 1899 in Frankfurt am Main\oindex{Frankfurt am Main@\textbf{Frankfurt am Main}|pwk}.}}}\label{K_L02880-2h} kommſt. Gib’ mir eine
                  {\pb}kurze Nachricht (Adreſſe immer Frankfurter Zeitung\orgindex{Frankfurter Zeitung@Frankfurter Zeitung|pwv}), wie es Dir geht, wie
               Du Dich in \label{K_L02880-3v}\edtext{Slavonien\oindex{Slawonien@\textbf{Slawonien}|pw}}{\lemma{\textnormal{\emph{Slavonien}}}\Cendnote{\textnormal{siehe Paul Goldmann an Arthur Schnitzler, 2. 7. 1899}}}\label{K_L02880-3h} behagt haſt, wo Du \label{K_L02880-4v}\edtext{jetzt
                  ſteckſt}{\lemma{\textnormal{\emph{jetzt
                  ſteckſt}}}\Cendnote{\textnormal{Schnitzler\pwindex{Schnitzler, Arthur 15.05.1862 – 21.10.1931@\textsc{Schnitzler, Arthur} (15.05.1862 – 21.10.1931), \emph{Schriftsteller, Mediziner}|pwk} hielt sich noch in Wien\oindex{Wien@\textbf{Wien}|pwk} auf, reiste jedoch am 17. 7. 1899 nach Velden am Wörthersee\oindex{Velden am Woerthersee@\textbf{Velden am Wörthersee}|pwk} ab.}}}\label{K_L02880-4h}? Vielleicht bei
                  \textsc{Richard\pwindex{Beer-Hofmann, Richard 1866-07-11 – 1945-09-26@\textsc{Beer-Hofmann, Richard} (1866-07-11 – 1945-09-26), \emph{Schriftsteller}|pw}}? Dann grüß’ ihn vielmals von mir und frag’ ihn, ob er \strikeout{m\textcolor{gray}{e}i} mein Buch\pwindex{Goldmann, Paul 31.01.1865 – 25.09.1935@\textsc{Goldmann, Paul} (31.01.1865 – 25.09.1935), \emph{Schriftsteller, Journalist}!Sommer in China. Reisebilder1899-05-02@\strich\emph{Ein Sommer in China. Reisebilder} {[}1899-05-02{]}|pwv} bekommen hat? Der Schuft hat, wie gewöhnlich, nicht
               geantwortet.\pend
           \pstart
           Viele treue Grüße! {\\[\baselineskip]}Dein {\\[\baselineskip]}\spacefill\mbox{Paul Goldmann}\pend
           \leftskip=0em{}\pstart
           \noindent{}Kennſt Du \label{K_L02880-5v}\edtext{\textsc{Hettner\pwindex{Hettner, Hermann 1821-03-12 – 1882-05-29@\textsc{Hettner, Hermann} (1821-03-12 – 1882-05-29), \emph{Museumsdirektor, Literaturhistoriker, Kunsthistoriker}|pw}s}{ }Franzöſiſche
                     Literaturgeſchichte\pwindex{Hettner, Hermann 1821-03-12 – 1882-05-29@\textsc{Hettner, Hermann} (1821-03-12 – 1882-05-29), \emph{Museumsdirektor, Literaturhistoriker, Kunsthistoriker}!Geschichte der franzoesischen Literatur im achtzehnten Jahrhundert1860@\strich\emph{Geschichte der französischen Literatur im achtzehnten Jahrhundert} {[}1860{]}|pwv}}{\lemma{\textnormal{\emph{Hettners … Literaturgeſchichte}}}\Cendnote{\textnormal{Hermann Hettner\pwindex{Hettner, Hermann 1821-03-12 – 1882-05-29@\textsc{Hettner, Hermann} (1821-03-12 – 1882-05-29), \emph{Museumsdirektor, Literaturhistoriker, Kunsthistoriker}|pwk}: \emph{Geschichte der französischen Literatur im achtzehnten
                           Jahrhundert}\pwindex{Hettner, Hermann 1821-03-12 – 1882-05-29@\textsc{Hettner, Hermann} (1821-03-12 – 1882-05-29), \emph{Museumsdirektor, Literaturhistoriker, Kunsthistoriker}!Geschichte der franzoesischen Literatur im achtzehnten Jahrhundert1860@\strich\emph{Geschichte der französischen Literatur im achtzehnten Jahrhundert} {[}1860{]}|pwk}. Braunschweig\oindex{Braunschweig@\textbf{Braunschweig}|pwk}: \emph{Friedrich Vieweg und Sohn}\orgindex{Friedrich Vieweg und Sohn@Friedrich Vieweg und Sohn|pwk}{ }1860. Eine Lektüre des Werk\pwindex{Hettner, Hermann 1821-03-12 – 1882-05-29@\textsc{Hettner, Hermann} (1821-03-12 – 1882-05-29), \emph{Museumsdirektor, Literaturhistoriker, Kunsthistoriker}!Geschichte der franzoesischen Literatur im achtzehnten Jahrhundert1860@\strich\emph{Geschichte der französischen Literatur im achtzehnten Jahrhundert} {[}1860{]}|pwkv}s durch Schnitzler\pwindex{Schnitzler, Arthur 15.05.1862 – 21.10.1931@\textsc{Schnitzler, Arthur} (15.05.1862 – 21.10.1931), \emph{Schriftsteller, Mediziner}|pwk} ist nicht
                     bekannt.}}}\label{K_L02880-5h}? Feines, geſcheites, gediegenes Werk. Bitte zu leſen.\pend
           
         
         \endnumbering\mylabel{h}\end{ledgroupsized}  \newcommand{\dateiname}{L02880}\newcommand{\titel}{Paul Goldmann an Arthur Schnitzler, 16. 7. 1899}\newcommand{\editorInnen}{Martin Anton Müller und Laura Untner}%% latex-leseansicht-abspann.tex
%% Abspann für die Leseansicht.
%% Der Schalter \ifkorrekturansicht ist bereits durch den Vorspann gesetzt.

%% latex-abspann.tex
%% Gemeinsamer Abspann für Korrekturansicht und Leseansicht.
%% Setzt den Schalter \ifkorrekturansicht voraus (gesetzt in den
%% einbindenden Dateien latex-korrekturansicht-abspann.tex bzw.
%% latex-leseansicht-abspann.tex).
%% ---------------------------------------------------------------

\normalsize

% Das esempio-Environment wird nur in der Leseansicht benötigt
\ifkorrekturansicht\else
\newenvironment{esempio}[3]%
{
    \vspace{1.5ex}
    \rlap{\underline{#1}}
    \par
    \setlength{\parindent}{0cm}
    \nopagebreak
    \leftskip=#2cm
    \rightskip=#3cm
}
{
    \par
}
\fi

\doendnotes{C}
\bigskip
\vfill

\clearpage

\footnotesize

\ifkorrekturansicht
  \lohead{\textsc{register}}
\fi

% theindex-Environment neu definieren ohne reledmac
\makeatletter
\renewenvironment{theindex}{%
  \ifkorrekturansicht
    \section*{\indexname}%
  \else
    \subsubsection*{Index der erwähnten Entitäten}%
  \fi
  \setlength{\parindent}{0pt}%
  \setlength{\parskip}{0pt plus 0.3pt}%
  \let\item\@idxitem
}{%
  \ifkorrekturansicht\clearpage\fi
}
\makeatother

\IfFileExists{\jobname-pw.ind}{\input{\jobname-pw.ind}}{}

% Quellenangabe nur in der Leseansicht
\ifkorrekturansicht\else
% Fallback-Definitionen, falls die .tex-Datei \titel etc. nicht gesetzt hat
\providecommand{\titel}{}
\providecommand{\editorInnen}{}
\providecommand{\dateiname}{\jobname}

\vspace{3cm}

\vfill

\footnotesize
\textsc{Quelle}: \titel. Herausgegeben von {\editorInnen}. In: \emph{Arthur Schnitzler: Briefwechsel mit Autorinnen und Autoren}.
 Digitale Edition, https://schnitzler-briefe.acdh.oeaw.ac.at/{\dateiname}.html (Stand \today)
\fi

\end{document}


      