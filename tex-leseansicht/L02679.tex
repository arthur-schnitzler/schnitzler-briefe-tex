%% latex-leseansicht-vorspann.tex
%% Vorspann für die Leseansicht.
%% Lädt die gemeinsame Datei latex-vorspann.tex mit nicht gesetztem Schalter.

\newif\ifkorrekturansicht
\korrekturansichtfalse

\input{../tex-inputs/latex-vorspann}

\begin{center}
            \textcolor{red}{ENTWURF. ENTZIFFERUNG NOCH NICHT KORREKTURGELESEN}
                      \end{center}
            
               \section[Paul Goldmann an Arthur Schnitzler, {[}21. 3. 1899?{]}]{ Paul Goldmann an Arthur Schnitzler, {[}21. 3. 1899?{]}}\nopagebreak\mylabel{v}\rehead{ }\begin{ledgroupsized}[t]{13cm}\normalsize\beginnumbering\briefempfaengerindex{Schnitzler, Arthur@\textsc{Schnitzler, Arthur}!zzzGoldmann, Paul@\emph{von Paul Goldmann}!1899-03-211@{{[}21. 3. 1899?{]}}|(be} \toendnotes[C]{\smallbreak\pagebreak[2]} \Standort{DLA, A:Schnitzler, HS.NZ85.1.3169.}
\physDesc{Telegramm
\newline{}maschinell
\newline{}Schnitzler: mit Bleistift datiert: »März 99« \newline{}Ordnung: beschnitten }\toendnotes[C]{\smallbreak}\pstart
           \noindent{}\centering{}{\pb}fr frankfurtmain\oindex{Frankfurt am Main@\textbf{Frankfurt am Main}|pw}
               9+ 73219 \label{K_L02679-1v}\edtext{21 3}{\lemma{\textnormal{\emph{21 3}}}\Cendnote{\textnormal{Obwohl inhaltlich nicht auf den Tod von
                     Marie Reinhard\pwindex{Reinhard, Marie 13.03.1871 – 18.03.1899@\textsc{Reinhard, Marie} (13.03.1871 – 18.03.1899), \emph{Gesangspädagogin}|pwk} am 18. 3. 1899 bezugnehmend
                  und mit falscher Spationierung wiedergegeben, dürfte das Telegramm am
                     21. 3. 1899 übermittelt worden sein.}}}\label{K_L02679-1h}1{ }1 20=\pend
           \pstart
           \noindent{}situation wieder vollstaendig ins schwanken gerathen +\pend
           \pstart
           sobald etwas definitives entschieden schrejbe ich dir =\pend
           \pstart grusz \spacefill\mbox{goldmann +}\pend{}\endnumbering\briefempfaengerindex{Schnitzler, Arthur@\textsc{Schnitzler, Arthur}!zzzGoldmann, Paul@\emph{von Paul Goldmann}!1899-03-211@{{[}21. 3. 1899?{]}}|)be}\mylabel{h}\end{ledgroupsized}\begin{anhang}\end{anhang}\newcommand{\dateiname}{L02679}\newcommand{\titel}{Paul Goldmann an Arthur Schnitzler, [21. 3. 1899?]}\newcommand{\editorInnen}{Martin Anton Müller und Laura Untner}%% latex-leseansicht-abspann.tex
%% Abspann für die Leseansicht.
%% Der Schalter \ifkorrekturansicht ist bereits durch den Vorspann gesetzt.

%% latex-abspann.tex
%% Gemeinsamer Abspann für Korrekturansicht und Leseansicht.
%% Setzt den Schalter \ifkorrekturansicht voraus (gesetzt in den
%% einbindenden Dateien latex-korrekturansicht-abspann.tex bzw.
%% latex-leseansicht-abspann.tex).
%% ---------------------------------------------------------------

\normalsize

% Das esempio-Environment wird nur in der Leseansicht benötigt
\ifkorrekturansicht\else
\newenvironment{esempio}[3]%
{
    \vspace{1.5ex}
    \rlap{\underline{#1}}
    \par
    \setlength{\parindent}{0cm}
    \nopagebreak
    \leftskip=#2cm
    \rightskip=#3cm
}
{
    \par
}
\fi

\doendnotes{C}
\bigskip
\vfill

\clearpage

\footnotesize

\ifkorrekturansicht
  \lohead{\textsc{register}}
\fi

% theindex-Environment neu definieren ohne reledmac
\makeatletter
\renewenvironment{theindex}{%
  \ifkorrekturansicht
    \section*{\indexname}%
  \else
    \subsubsection*{Index der erwähnten Entitäten}%
  \fi
  \setlength{\parindent}{0pt}%
  \setlength{\parskip}{0pt plus 0.3pt}%
  \let\item\@idxitem
}{%
  \ifkorrekturansicht\clearpage\fi
}
\makeatother

\IfFileExists{\jobname-pw.ind}{\input{\jobname-pw.ind}}{}

% Quellenangabe nur in der Leseansicht
\ifkorrekturansicht\else
% Fallback-Definitionen, falls die .tex-Datei \titel etc. nicht gesetzt hat
\providecommand{\titel}{}
\providecommand{\editorInnen}{}
\providecommand{\dateiname}{\jobname}

\vspace{3cm}

\vfill

\footnotesize
\textsc{Quelle}: \titel. Herausgegeben von {\editorInnen}. In: \emph{Arthur Schnitzler: Briefwechsel mit Autorinnen und Autoren}.
 Digitale Edition, https://schnitzler-briefe.acdh.oeaw.ac.at/{\dateiname}.html (Stand \today)
\fi

\end{document}


      