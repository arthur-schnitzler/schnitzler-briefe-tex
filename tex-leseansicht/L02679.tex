%% latex-korrekturansicht-vorspann.tex
%% Vorspann für die Korrekturansicht.
%% Lädt die gemeinsame Datei latex-vorspann.tex mit gesetztem Schalter.

\newif\ifkorrekturansicht
\korrekturansichttrue

\input{../tex-inputs/latex-vorspann}


\section[Paul Goldmann an Arthur Schnitzler, {[}31. 1. 1899?{]}]{L02679 Paul Goldmann an Arthur Schnitzler, {[}31. 1. 1899?{]}}
\nopagebreak\mylabel{L02679v}
\rehead{ }\normalsize\beginnumbering\briefempfaengerindex{Schnitzler, Arthur@\textsc{Schnitzler, Arthur}!zzzGoldmann, Paul@\emph{von Paul Goldmann}!1899-01-311@{{[}31. 1. 1899?{]}}|(be}
\toendnotes[C]{\smallbreak\pagebreak[2]}\Standort{DLA, A:Schnitzler, HS.NZ85.1.3169.}
\physDesc{Telegramm, 158 Zeichen
\newline{}maschinell
\newline{}Schnitzler: mit Bleistift datiert auf den Monat »März« und das Jahr »99«  
\newline{}Ordnung: beschnitten }\toendnotes[C]{\smallbreak}
\pstart
           \noindent{}\centering{}{\pb}fr frankfurtmain\oindex{Frankfurt am Main@\textbf{Frankfurt am Main}, \emph{P.PPLA3}|pw}
               9+ 73219 21 \label{K_L02679-1v}\edtext{31 1}{\lemma{\textnormal{\emph{31 1}}}\Cendnote{\textnormal{Inhaltlich dürfte sich das Telegramm auf
                  die geplante Mitarbeit Goldmanns\pwindex{Goldmann, Paul 31.01.1865 – 25.09.1935@\textsc{Goldmann, Paul} (31.01.1865 – 25.09.1935), \emph{Schriftsteller/Schriftstellerin, Journalist/Journalistin}|pwk} bei der
                     \emph{Neuen Freien Presse}\orgindex{Neue Freie Presse@Neue Freie Presse|pwk} beziehen. Die Datierung
                     Schnitzlers auf »März« lässt sich nicht ohne argumentative Verrenkungen mit den
                  Korrespondenzstücken aus diesem Zeitraum in Einklang bringen, da zu diesem Punkt die Anstellung bei der \emph{Neuen Freien
                     Presse}\orgindex{Neue Freie Presse@Neue Freie Presse|pwk} bereits (fürs Erste) abgetan war. Im Gegensatz dazu verteten wir die Ansicht,
                  dass die Empfangszeile des Telegramms nur eine zweistellige Uhrzeit
                     »20« angibt und die Ziffern davor das Datum darstellen. Das
                  würde den langen Abstand zwischen Goldmanns\pwindex{Goldmann, Paul 31.01.1865 – 25.09.1935@\textsc{Goldmann, Paul} (31.01.1865 – 25.09.1935), \emph{Schriftsteller/Schriftstellerin, Journalist/Journalistin}|pwk} Abreise aus Wien\oindex{Wien@\textbf{Wien}, \emph{A.ADM2}|pwk}{ }Mitte Januar 1899 und seinem nächsten Schreiben (Paul Goldmann an Arthur Schnitzler, 5. 3. [1899]) erklären.}}}\label{K_L02679-1}{ }20=\pend
           
\pstart
           situation wieder vollstaendig ins schwanken gerathen +\pend
           
\pstart
           sobald etwas definitives entschieden schrejbe ich dir =\pend
           \pstart grusz \spacefill\mbox{goldmann +}\pend{}\selectlanguage{ngerman}\endnumbering\briefempfaengerindex{Schnitzler, Arthur@\textsc{Schnitzler, Arthur}!zzzGoldmann, Paul@\emph{von Paul Goldmann}!1899-01-311@{{[}31. 1. 1899?{]}}|)be}\mylabel{L02679h}  \normalsize

\doendnotes{C}
\bigskip
\vfill

\clearpage

\footnotesize

\lohead{\textsc{register}}

% Definiere theindex-Environment komplett neu ohne reledmac
\makeatletter
\renewenvironment{theindex}{%
  \section*{\indexname}%
  \setlength{\parindent}{0pt}%
  \setlength{\parskip}{0pt plus 0.3pt}%
  \let\item\@idxitem
}{%
  \clearpage
}
\makeatother

\IfFileExists{\jobname-pw.ind}{\input{\jobname-pw.ind}}{}

\end{document}

      