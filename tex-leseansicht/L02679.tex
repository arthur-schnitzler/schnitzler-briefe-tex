%% latex-leseansicht-vorspann.tex
%% Vorspann für die Leseansicht.
%% Lädt die gemeinsame Datei latex-vorspann.tex mit nicht gesetztem Schalter.

\newif\ifkorrekturansicht
\korrekturansichtfalse

\input{../tex-inputs/latex-vorspann}


         
         \renewcommand{\erwaehntePersonen}{Personen: Paul Goldmann}
         \renewcommand{\erwaehnteInstitutionen}{Institutionen: Neue Freie Presse}
         \renewcommand{\erwaehnteOrte}{Orte: Frankfurt am Main, Wien}
         \renewcommand{\erwaehnteWerke}{}
               \section[Paul Goldmann an Arthur Schnitzler, {[}31. 1. 1899?{]}]{ Paul Goldmann an Arthur Schnitzler, {[}31. 1. 1899?{]}}\nopagebreak\mylabel{v}\rehead{ }\begin{ledgroupsized}[t]{13cm}\normalsize\beginnumbering\briefempfaengerindex{Schnitzler, Arthur@\textsc{Schnitzler, Arthur}!zzzGoldmann, Paul@\emph{von Paul Goldmann}!1899-01-311@{{[}31. 1. 1899?{]}}|(be} \toendnotes[C]{\smallbreak\pagebreak[2]} \Standort{DLA, A:Schnitzler, HS.NZ85.1.3169.}
\physDesc{Telegramm, 158 Zeichen
\newline{}maschinell
\newline{}Schnitzler: mit Bleistift datiert auf den Monat »März« und das Jahr »99«  
\newline{}Ordnung: beschnitten }\toendnotes[C]{\smallbreak}\pstart
           \noindent{}\centering{}{\pb}fr frankfurtmain\oindex{Frankfurt am Main@\textbf{Frankfurt am Main}|pw}
               9+ 73219 21 \label{K_L02679-1v}\edtext{31 1}{\lemma{\textnormal{\emph{31 1}}}\Cendnote{\textnormal{Inhaltlich dürfte sich das Telegramm auf
                  die geplante Mitarbeit Goldmanns\pwindex{Goldmann, Paul 31.01.1865 – 25.09.1935@\textsc{Goldmann, Paul} (31.01.1865 – 25.09.1935), \emph{Schriftsteller, Journalist}|pwk} bei der
                     \emph{Neuen Freien Presse}\orgindex{Neue Freie Presse@Neue Freie Presse|pwk} beziehen. Die Datierung
                     Schnitzlers\pwindex{Schnitzler, Arthur 15.05.1862 – 21.10.1931@\textsc{Schnitzler, Arthur} (15.05.1862 – 21.10.1931), \emph{Schriftsteller, Mediziner}|pwk} auf »März« lässt sich nicht ohne argumentative Verrenkungen mit den
                  Korrespondenzstücken aus diesem Zeitraum in Einklang bringen, da zu diesem Punkt die Anstellung bei der \emph{Neuen Freien
                     Presse}\orgindex{Neue Freie Presse@Neue Freie Presse|pwk} bereits (fürs Erste) abgetan war. Im Gegensatz dazu verteten wir die Ansicht,
                  dass die Empfangszeile des Telegramms nur eine zweistellige Uhrzeit
                     »20« angibt und die Ziffern davor das Datum darstellen. Das
                  würde den langen Abstand zwischen Goldmanns\pwindex{Goldmann, Paul 31.01.1865 – 25.09.1935@\textsc{Goldmann, Paul} (31.01.1865 – 25.09.1935), \emph{Schriftsteller, Journalist}|pwk} Abreise aus Wien\oindex{Wien@\textbf{Wien}|pwk}{ }Mitte Januar 1899 und seinem nächsten Schreiben (Paul Goldmann an Arthur Schnitzler, 5. 3. [1899]) erklären.}}}\label{K_L02679-1h}{ }20=\pend
           \pstart
           \noindent{}situation wieder vollstaendig ins schwanken gerathen +\pend
           \pstart
           sobald etwas definitives entschieden schrejbe ich dir =\pend
           \pstart grusz \spacefill\mbox{goldmann +}\pend{}
         
         \endnumbering\mylabel{h}\end{ledgroupsized}  \newcommand{\dateiname}{L02679}\newcommand{\titel}{Paul Goldmann an Arthur Schnitzler, [31. 1. 1899?]}\newcommand{\editorInnen}{Martin Anton Müller und Laura Untner}%% latex-leseansicht-abspann.tex
%% Abspann für die Leseansicht.
%% Der Schalter \ifkorrekturansicht ist bereits durch den Vorspann gesetzt.

%% latex-abspann.tex
%% Gemeinsamer Abspann für Korrekturansicht und Leseansicht.
%% Setzt den Schalter \ifkorrekturansicht voraus (gesetzt in den
%% einbindenden Dateien latex-korrekturansicht-abspann.tex bzw.
%% latex-leseansicht-abspann.tex).
%% ---------------------------------------------------------------

\normalsize

% Das esempio-Environment wird nur in der Leseansicht benötigt
\ifkorrekturansicht\else
\newenvironment{esempio}[3]%
{
    \vspace{1.5ex}
    \rlap{\underline{#1}}
    \par
    \setlength{\parindent}{0cm}
    \nopagebreak
    \leftskip=#2cm
    \rightskip=#3cm
}
{
    \par
}
\fi

\doendnotes{C}
\bigskip
\vfill

\clearpage

\footnotesize

\ifkorrekturansicht
  \lohead{\textsc{register}}
\fi

% theindex-Environment neu definieren ohne reledmac
\makeatletter
\renewenvironment{theindex}{%
  \ifkorrekturansicht
    \section*{\indexname}%
  \else
    \subsubsection*{Index der erwähnten Entitäten}%
  \fi
  \setlength{\parindent}{0pt}%
  \setlength{\parskip}{0pt plus 0.3pt}%
  \let\item\@idxitem
}{%
  \ifkorrekturansicht\clearpage\fi
}
\makeatother

\IfFileExists{\jobname-pw.ind}{\input{\jobname-pw.ind}}{}

% Quellenangabe nur in der Leseansicht
\ifkorrekturansicht\else
% Fallback-Definitionen, falls die .tex-Datei \titel etc. nicht gesetzt hat
\providecommand{\titel}{}
\providecommand{\editorInnen}{}
\providecommand{\dateiname}{\jobname}

\vspace{3cm}

\vfill

\footnotesize
\textsc{Quelle}: \titel. Herausgegeben von {\editorInnen}. In: \emph{Arthur Schnitzler: Briefwechsel mit Autorinnen und Autoren}.
 Digitale Edition, https://schnitzler-briefe.acdh.oeaw.ac.at/{\dateiname}.html (Stand \today)
\fi

\end{document}


      