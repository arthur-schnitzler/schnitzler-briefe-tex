%% latex-leseansicht-vorspann.tex
%% Vorspann für die Leseansicht.
%% Lädt die gemeinsame Datei latex-vorspann.tex mit nicht gesetztem Schalter.

\newif\ifkorrekturansicht
\korrekturansichtfalse

\input{../tex-inputs/latex-vorspann}

\begin{center}
            \textcolor{red}{ENTWURF, NICHT FERTIG KORRIGIERT}
                      \end{center}
            
         
         \newcommand{\erwaehntePersonen}{Personen: }
         \newcommand{\erwaehnteInstitutionen}{Institutionen: Neue Freie Presse}
         \newcommand{\erwaehnteOrte}{Orte: Frankfurt am Main, Wien}
         \newcommand{\erwaehnteWerke}{
               \section[Paul Goldmann an Arthur Schnitzler, {[}31. 1. 1899?{]}]{ Paul Goldmann an Arthur Schnitzler, {[}31. 1. 1899?{]}}\nopagebreak\mylabel{v}\rehead{ }\begin{ledgroupsized}[t]{13cm}\normalsize\beginnumbering \toendnotes[C]{\smallbreak\pagebreak[2]} \Standort{DLA, A:Schnitzler, HS.NZ85.1.3169.}
\physDesc{Telegramm
\newline{}maschinell
\newline{}Schnitzler: mit Bleistift datiert: »März 99« \newline{}Ordnung: beschnitten }\toendnotes[C]{\smallbreak}\pstart
           \noindent{}\centering{}{\pb}fr frankfurtmain\oindex{Frankfurt am Main@\textbf{Frankfurt am Main}|pw}
               9+ 73219 21\label{K_L02679-1v}\edtext{31 1}{\lemma{\textnormal{\emph{31 1}}}\Cendnote{\textnormal{Inhaltlich dürfte sich das Telegramm auf
                  die geplante Mitarbeit Goldmann\pwindex{Goldmann, Paul 31.01.1865 – 25.09.1935@\textsc{Goldmann, Paul} (31.01.1865 – 25.09.1935), \emph{Schriftsteller, Journalist}|pwk}s bei der
                     \emph{Neuen Freien Presse}\orgindex{Neue Freie Presse@Neue Freie Presse|pwk} beziehen. Die Datierung
                     Schnitzler\pwindex{Schnitzler, Arthur 15.05.1862 – 21.10.1931@\textsc{Schnitzler, Arthur} (15.05.1862 – 21.10.1931), \emph{Schriftsteller, Mediziner}|pwk}s auf »März 99« lässt sich nicht ohne Mühen mit den erhaltenen Korrespondenzstücken in
                  Einklang bringen, da zu diesem Zeitpunkt die Anstellung bei der \emph{Neuen Freien Presse}\orgindex{Neue Freie Presse@Neue Freie Presse|pwk} bereits (fürs Erste) abgetan ist. Hier
                  wird die Ansicht vertreten, dass die Empfangszeile des Telegramms nur eine
                  zweistellige Uhrzeit »20« angibt und die Ziffern davor das Datum
                  darstellen. Das würde den langen Abstand zwischen Goldmann\pwindex{Goldmann, Paul 31.01.1865 – 25.09.1935@\textsc{Goldmann, Paul} (31.01.1865 – 25.09.1935), \emph{Schriftsteller, Journalist}|pwk}s Abreise aus Wien\oindex{Wien@\textbf{Wien}|pwk}Mitte Januar 1899 und seinem nächsten Schreiben (Paul Goldmann an Arthur Schnitzler, 5. 3. [1899]) erklären.}}}\label{K_L02679-1h}1{ }1 20=\pend
           \pstart
           \noindent{}situation wieder vollstaendig ins schwanken gerathen +\pend
           \pstart
           sobald etwas definitives entschieden schrejbe ich dir =\pend
           \pstart grusz \spacefill\mbox{goldmann +}\pend{}
         
         \endnumbering\mylabel{h}\end{ledgroupsized}  \newcommand{\dateiname}{L02679}\newcommand{\titel}{Paul Goldmann an Arthur Schnitzler, [31. 1. 1899?]}\newcommand{\editorInnen}{Martin Anton Müller und Laura Untner}%% latex-leseansicht-abspann.tex
%% Abspann für die Leseansicht.
%% Der Schalter \ifkorrekturansicht ist bereits durch den Vorspann gesetzt.

%% latex-abspann.tex
%% Gemeinsamer Abspann für Korrekturansicht und Leseansicht.
%% Setzt den Schalter \ifkorrekturansicht voraus (gesetzt in den
%% einbindenden Dateien latex-korrekturansicht-abspann.tex bzw.
%% latex-leseansicht-abspann.tex).
%% ---------------------------------------------------------------

\normalsize

% Das esempio-Environment wird nur in der Leseansicht benötigt
\ifkorrekturansicht\else
\newenvironment{esempio}[3]%
{
    \vspace{1.5ex}
    \rlap{\underline{#1}}
    \par
    \setlength{\parindent}{0cm}
    \nopagebreak
    \leftskip=#2cm
    \rightskip=#3cm
}
{
    \par
}
\fi

\doendnotes{C}
\bigskip
\vfill

\clearpage

\footnotesize

\ifkorrekturansicht
  \lohead{\textsc{register}}
\fi

% theindex-Environment neu definieren ohne reledmac
\makeatletter
\renewenvironment{theindex}{%
  \ifkorrekturansicht
    \section*{\indexname}%
  \else
    \subsubsection*{Index der erwähnten Entitäten}%
  \fi
  \setlength{\parindent}{0pt}%
  \setlength{\parskip}{0pt plus 0.3pt}%
  \let\item\@idxitem
}{%
  \ifkorrekturansicht\clearpage\fi
}
\makeatother

\IfFileExists{\jobname-pw.ind}{\input{\jobname-pw.ind}}{}

% Quellenangabe nur in der Leseansicht
\ifkorrekturansicht\else
% Fallback-Definitionen, falls die .tex-Datei \titel etc. nicht gesetzt hat
\providecommand{\titel}{}
\providecommand{\editorInnen}{}
\providecommand{\dateiname}{\jobname}

\vspace{3cm}

\vfill

\footnotesize
\textsc{Quelle}: \titel. Herausgegeben von {\editorInnen}. In: \emph{Arthur Schnitzler: Briefwechsel mit Autorinnen und Autoren}.
 Digitale Edition, https://schnitzler-briefe.acdh.oeaw.ac.at/{\dateiname}.html (Stand \today)
\fi

\end{document}


      