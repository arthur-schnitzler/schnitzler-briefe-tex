%% latex-leseansicht-vorspann.tex
%% Vorspann für die Leseansicht.
%% Lädt die gemeinsame Datei latex-vorspann.tex mit nicht gesetztem Schalter.

\newif\ifkorrekturansicht
\korrekturansichtfalse

\input{../tex-inputs/latex-vorspann}


\section[Stefan Zweig an Arthur Schnitzler, {[}1924–1928{]}]{L03676 Stefan Zweig an Arthur Schnitzler, {[}1924–1928{]}}
\nopagebreak\mylabel{L03676v}
\rehead{ }\normalsize\beginnumbering\briefempfaengerindex{Schnitzler, Arthur@\textsc{Schnitzler, Arthur}!zzzZweig, Stefan@\emph{von Stefan Zweig}!1928-12-311@{{[}1924–1928{]}}|(be}
\toendnotes[C]{\smallbreak\pagebreak[2]}
\correspDesc{Versand  durch Stefan Zweig im Zeitraum [1924–1928] \textbf{Ort fehlend} 
\newline{}Erhalt  durch Arthur Schnitzler im Zeitraum [1924–1928] in Wien}\toendnotes[C]{\smallbreak}
\Standort{CUL, Schnitzler, B 118.}
\physDesc{Brief, 1 Blatt, 1 Seite, 326 Zeichen
\newline{}Handschrift: schwarze Tinte, lateinische Kurrent
\newline{}Schnitzler: mit rotem Buntstift eine Unterstreichung }
\buchAbdrucke{\weitereDrucke{Stefan Zweig: \emph{Briefwechsel mit Hermann Bahr, Sigmund Freud, Rainer Maria
                        Rilke und Arthur Schnitzler}. Herausgegeben von Jeffrey B. Berlin, Hans-Ulrich Lindken und Donald A. Prater. Frankfurt am Main: \emph{S. Fischer} 1987, S. 449.} }\toendnotes[C]{\smallbreak}
\pstart
           \noindent{}{\pb}Verehrter Herr Doktor, der Dozent der Universität London\orgindex{University of London@University of London|pw}{ }M\textsuperscript{r}{ }J. Isaacs\pwindex{Isaacs, Jacob 6.\,12.\,1896 London – 12.\,5.\,1973 ebd.@\textsc{Isaacs, Jacob} (6.\,12.\,1896 London – 12.\,5.\,1973 ebd.), \emph{Universitätslehrer}|pw} würde auf seiner Reise nach Wien\oindex{Wien@\textbf{Wien}, \emph{Verwaltungsgebiet}|pw} Sie ungemein gerne sehen: als \label{K_L03676-1v}\edtext{Dozent der englischen\oindex{Vereinigtes Königreich@\textbf{Vereinigtes Königreich}|pw} Literatur}{\lemma{\textnormal{\emph{Dozent … Literatur}}}\Cendnote{\textnormal{Diese
                  Position hatte Jacob Isaacs\pwindex{Isaacs, Jacob 6.\,12.\,1896 London – 12.\,5.\,1973 ebd.@\textsc{Isaacs, Jacob} (6.\,12.\,1896 London – 12.\,5.\,1973 ebd.), \emph{Universitätslehrer}|pwk} von
                     1924 bis 1928 am \emph{King’s College}\orgindex{King’s College London@King’s College London|pwk} inne, so dass das vorliegende Schreiben
                  in diesem Zeitraum gelaufen sein dürfte. Schnitzler erwähnt kein Treffen im \emph{Tagebuch}\pwindex{Schnitzler, Arthur 15.\,5.\,1862 Wien – 21.\,10.\,1931 ebd.@\textsc{Schnitzler, Arthur} (15.\,5.\,1862 Wien – 21.\,10.\,1931 ebd.), \emph{Schriftsteller, Mediziner}!Tagebuch@\strich\emph{Tagebuch}|pwk}. }}}\label{K_L03676-1} kann er Ihnen wohl in mancher Auskunft nützlich sein
               und Sie würden, glaube ich, es nicht bedauern, ihm eine Stunde zu schenken.\pend
           
\pstart
           In alter inniger Verehrung Ihr{\\[\baselineskip]}\spacefill\mbox{Stefan Zweig}\pend
           \leftskip=0em{}\selectlanguage{ngerman}\endnumbering\briefempfaengerindex{Schnitzler, Arthur@\textsc{Schnitzler, Arthur}!zzzZweig, Stefan@\emph{von Stefan Zweig}!1924-01-011@{{[}1924–1928{]}}|)be}\mylabel{L03676h}
\begin{anhang}
\end{anhang}\newcommand{\dateiname}{L03676}\newcommand{\titel}{Stefan Zweig an Arthur Schnitzler, [1924–1928]}\newcommand{\editorInnen}{Selma Jahnke und Martin Anton Müller}%% latex-leseansicht-abspann.tex
%% Abspann für die Leseansicht.
%% Der Schalter \ifkorrekturansicht ist bereits durch den Vorspann gesetzt.

%% latex-abspann.tex
%% Gemeinsamer Abspann für Korrekturansicht und Leseansicht.
%% Setzt den Schalter \ifkorrekturansicht voraus (gesetzt in den
%% einbindenden Dateien latex-korrekturansicht-abspann.tex bzw.
%% latex-leseansicht-abspann.tex).
%% ---------------------------------------------------------------

\normalsize

% Das esempio-Environment wird nur in der Leseansicht benötigt
\ifkorrekturansicht\else
\newenvironment{esempio}[3]%
{
    \vspace{1.5ex}
    \rlap{\underline{#1}}
    \par
    \setlength{\parindent}{0cm}
    \nopagebreak
    \leftskip=#2cm
    \rightskip=#3cm
}
{
    \par
}
\fi

\doendnotes{C}
\bigskip
\vfill

\clearpage

\footnotesize

\ifkorrekturansicht
  \lohead{\textsc{register}}
\fi

% theindex-Environment neu definieren ohne reledmac
\makeatletter
\renewenvironment{theindex}{%
  \ifkorrekturansicht
    \section*{\indexname}%
  \else
    \subsubsection*{Index der erwähnten Entitäten}%
  \fi
  \setlength{\parindent}{0pt}%
  \setlength{\parskip}{0pt plus 0.3pt}%
  \let\item\@idxitem
}{%
  \ifkorrekturansicht\clearpage\fi
}
\makeatother

\IfFileExists{\jobname-pw.ind}{\input{\jobname-pw.ind}}{}

% Quellenangabe nur in der Leseansicht
\ifkorrekturansicht\else
% Fallback-Definitionen, falls die .tex-Datei \titel etc. nicht gesetzt hat
\providecommand{\titel}{}
\providecommand{\editorInnen}{}
\providecommand{\dateiname}{\jobname}

\vspace{3cm}

\vfill

\footnotesize
\textsc{Quelle}: \titel. Herausgegeben von {\editorInnen}. In: \emph{Arthur Schnitzler: Briefwechsel mit Autorinnen und Autoren}.
 Digitale Edition, https://schnitzler-briefe.acdh.oeaw.ac.at/{\dateiname}.html (Stand \today)
\fi

\end{document}


