%% latex-korrekturansicht-vorspann.tex
%% Vorspann für die Korrekturansicht.
%% Lädt die gemeinsame Datei latex-vorspann.tex mit gesetztem Schalter.

\newif\ifkorrekturansicht
\korrekturansichttrue

\input{../tex-inputs/latex-vorspann}


\section[Hugo Hofmannsthal an Arthur Schnitzler, 10. 12. 1925]{L02457 Hugo Hofmannsthal an Arthur Schnitzler, 10. 12. 1925}
\nopagebreak\mylabel{L02457v}
\rehead{ }\normalsize\beginnumbering\briefempfaengerindex{Schnitzler, Arthur@\textsc{Schnitzler, Arthur}!zzzHofmannsthal, Hugo von@\emph{von Hugo von Hofmannsthal}!1925-12-101@{10. 12. 1925}|(be}
\toendnotes[C]{\smallbreak\pagebreak[2]}\Standort{CUL, Schnitzler, B 43.}
\physDesc{Postkarte, 431 Zeichen
\newline{}Handschrift: schwarze Tinte, lateinische Kurrent
\newline{}Versand: Stempel: »\nobreak{}\oindex{Rodaun@\textbf{Rodaun}, \emph{A.ADM4}|pwk}Rodaun, 10 \textcolor{gray}{12} 25, 12V\nobreak{}«.  
\newline{}Ordnung: 1) mit Bleistift von unbekannter Hand nummeriert: »\strikeout{288}\strikeout{289}\strikeout{354}\strikeout{367}\strikeout{193}«  2) mit Bleistift von unbekannter Hand nummeriert:
                                    »391«}
\buchAbdrucke{\weitereDrucke{Hugo von Hofmannsthal, Arthur Schnitzler: \emph{Briefwechsel}. Frankfurt am Main: \emph{S. Fischer} 1964, S. 304.} }\toendnotes[C]{\smallbreak}\pstart{}{\pb}Herrn D\textsuperscript{r} Arthur Schnitzler\pend{}\pstart{}Wien\oindex{Wien@\textbf{Wien}, \emph{A.ADM2}|pw}\pend{}\pstart{}XVIII Sternwartestrasse 71\oindex{Sternwartestrasse 71@\textbf{Sternwartestraße 71}, \emph{Wohngebäude (K.WHS)}|pw}\pend{}{\bigskip}\vspace{1em}
\pstart
           \raggedleft{}{\pb}Rodaun\oindex{Rodaun@\textbf{Rodaun}, \emph{A.ADM4}|pw}, Do{\geminationn}erstag\pend
           \vspace{0.5em}
\pstart
           Mit der allergrößten Freude, lieber Arthur, an jedem beliebigen Nachmittg oder Abend
               der nächsten Woche \label{K_L02457-1v}\edtext{ab
                  Dienstag}{\lemma{\textnormal{\emph{ab
                  Dienstag}}}\Cendnote{\textnormal{Tatsächlich entschied sich Schnitzler für Dienstag, den 16. 12. 1925, um \emph{Der Gang zum Weiher}\pwindex{Gang zum Weiher. Dramatische Dichtung@\emph{Der Gang zum Weiher. Dramatische Dichtung}|pwk} in privatem Kreis
                  vorzulesen. Anwesend war auch Hofmannsthal\pwindex{Hofmannsthal, Hugo von 1874-02-01 – 1929-07-15@\textsc{Hofmannsthal, Hugo von} (1874-02-01 – 1929-07-15), \emph{Schriftsteller/Schriftstellerin}|pwk}.}}}\label{K_L02457-1}. Vielleicht {\pb}fangen Sie ziemlich früh an
                     (7\textsuperscript{h}?) ich bin so gar kein Nachtmensch.\pend
           
\pstart
           Ein Auto, um in die Stadt zu fahren, wird man ja beko{\geminationm}en
                  kö{\geminationn}en? (Ich meine natürlich ein Taxi.)\pend
           
\pstart
           Also bitte telegraphiren Sie mir den Tag, den Sie wählen.\pend
           
\pstart
           Herzlich Ihr{\\[\baselineskip]}\spacefill\mbox{Hugo.}\pend
           \leftskip=0em{}\selectlanguage{ngerman}\endnumbering\briefempfaengerindex{Schnitzler, Arthur@\textsc{Schnitzler, Arthur}!zzzHofmannsthal, Hugo von@\emph{von Hugo von Hofmannsthal}!1925-12-101@{10. 12. 1925}|)be}\mylabel{L02457h}  \normalsize

\doendnotes{C}
\bigskip
\vfill

\clearpage

\footnotesize

\lohead{\textsc{register}}

% Definiere theindex-Environment komplett neu ohne reledmac
\makeatletter
\renewenvironment{theindex}{%
  \section*{\indexname}%
  \setlength{\parindent}{0pt}%
  \setlength{\parskip}{0pt plus 0.3pt}%
  \let\item\@idxitem
}{%
  \clearpage
}
\makeatother

\IfFileExists{\jobname-pw.ind}{\input{\jobname-pw.ind}}{}

\end{document}

      