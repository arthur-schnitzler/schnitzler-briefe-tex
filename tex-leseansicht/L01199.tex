%% latex-korrekturansicht-vorspann.tex
%% Vorspann für die Korrekturansicht.
%% Lädt die gemeinsame Datei latex-vorspann.tex mit gesetztem Schalter.

\newif\ifkorrekturansicht
\korrekturansichttrue

\input{../tex-inputs/latex-vorspann}


\section[Arthur Schnitzler an Hermann Bahr, 25. 1. 1902]{L01199 Arthur Schnitzler an Hermann Bahr, 25. 1. 1902}
\nopagebreak\mylabel{L01199v}
\rehead{ }\normalsize\beginnumbering\briefempfaengerindex{Bahr, Hermann@\textsc{Bahr, Hermann}!zzzSchnitzler, Arthur@\emph{von Arthur Schnitzler}!1902-01-251@{25. 1. 1902}|(be}
\toendnotes[C]{\smallbreak\pagebreak[2]}\Standort{TMW, HS AM 23349 Ba.}
\physDesc{Brief, 1 Blatt, 2 Seiten, 419 Zeichen
\newline{}Handschrift: schwarze Tinte, deutsche Kurrent
\newline{}Ordnung: 1) Lochung  2) mit Bleistift von unbekannter Hand datiert:
                                    »25. I. 02«}
\buchAbdrucke{\weitereDrucke{1) Arthur Schnitzler: \emph{The Letters of Arthur Schnitzler to Hermann Bahr}. Chapel Hill: \emph{The University of North Carolina Press} 1978, S. 74.} \weitereDrucke{2) Hermann Bahr, Arthur Schnitzler: \emph{Briefwechsel, Aufzeichnungen, Dokumente (1891–1931)}. Göttingen: \emph{Wallstein} 2018, S. 226.} }\toendnotes[C]{\smallbreak}
\pstart{}{\pb}mein lieber
                  Hermann;\pend\vspace{0.5em}
\pstart
           ich danke dir ſehr. Du haſt Dinge über mich geſagt, die mich ganz beſonders gefreut
               haben; – ich wollte ſie endlich hören, wollte ſie vor allem von dir hören. Nicht das
               beiläufige über den Grillparzer Preis\orgindex{Franz-Grillparzer-Preis@Franz-Grillparzer-Preis|pw} meine ich,
               ſondern das {\pb}allgemeine. Jemand, der heute deinen Artikel\pwindex{Grillparzerpreis@\emph{Der Grillparzerpreis}|pwv} las, ſagte: »Es iſt ganz einfach, Ihr seid \introOben{}a\textcolor{gray}{l}\damage{le}\introOben{} beide mit der Zeit anſtändige Leute geworden.«\pend
           
\pstart
           herzlichen Gruſs{\\[\baselineskip]}dein \spacefill\mbox{Arthur}\pend
           \leftskip=0em{}
\pstart
           2\substVorne{}\textsuperscript{4}\substDazwischen{}5\substHinten{}. 1. 902\pend
           \selectlanguage{ngerman}\endnumbering\briefempfaengerindex{Bahr, Hermann@\textsc{Bahr, Hermann}!zzzSchnitzler, Arthur@\emph{von Arthur Schnitzler}!1902-01-251@{25. 1. 1902}|)be}\mylabel{L01199h}  \normalsize

\doendnotes{C}
\bigskip
\vfill

\clearpage

\footnotesize

\lohead{\textsc{register}}

% Definiere theindex-Environment komplett neu ohne reledmac
\makeatletter
\renewenvironment{theindex}{%
  \section*{\indexname}%
  \setlength{\parindent}{0pt}%
  \setlength{\parskip}{0pt plus 0.3pt}%
  \let\item\@idxitem
}{%
  \clearpage
}
\makeatother

\IfFileExists{\jobname-pw.ind}{\input{\jobname-pw.ind}}{}

\end{document}

      