%% latex-leseansicht-vorspann.tex
%% Vorspann für die Leseansicht.
%% Lädt die gemeinsame Datei latex-vorspann.tex mit nicht gesetztem Schalter.

\newif\ifkorrekturansicht
\korrekturansichtfalse

\input{../tex-inputs/latex-vorspann}


         
         \newcommand{\erwaehntePersonen}{Personen: Hermann Bahr}
         \newcommand{\erwaehnteInstitutionen}{Institutionen: Franz-Grillparzer-Preis}
         \newcommand{\erwaehnteOrte}{Orte: Wien}
         \newcommand{\erwaehnteWerke}{Werke: Der Grillparzerpreis}
               \section[Arthur Schnitzler an Hermann Bahr, 25. 1. 1902]{ Arthur Schnitzler an Hermann Bahr, 25. 1. 1902}\nopagebreak\mylabel{v}\rehead{ }\begin{ledgroupsized}[t]{13cm}\normalsize\beginnumbering \toendnotes[C]{\smallbreak\pagebreak[2]} \Standort{TMW, HS AM 23349 Ba.}
\physDesc{Brief, 1 Blatt, 2 Seiten
\newline{}Handschrift: schwarze Tinte, deutsche Kurrent\newline{}Ordnung: 1) Lochung  2) mit Bleistift von unbekannter Hand datiert:
                              »25. I. 02«}\buchAbdrucke{\weitereDrucke{1) \emph{25. 1. 1902.} In: Arthur Schnitzler: \emph{The Letters of Arthur Schnitzler to Hermann Bahr}. Edited, annotated, and with an introduction, by Donald G.
                        Daviau. Chapel Hill: \emph{The University of North Carolina Press} 1978, S. 74 (University of North Carolina studies in the Germanic languages
                        and literatures, 89).} \weitereDrucke{2) Hermann Bahr, Arthur Schnitzler: \emph{Briefwechsel, Aufzeichnungen, Dokumente (1891–1931)}. Hg. Kurt Ifkovits und Martin Anton Müller. Göttingen: \emph{Wallstein} 2018, S. 226.} }\toendnotes[C]{\smallbreak}\pstart{}{\pb}mein lieber Hermann;\pend\pstart
           ich danke dir ſehr. Du haſt Dinge über mich geſagt, die mich ganz beſonders gefreut
               haben; – ich wollte ſie endlich hören, wollte ſie vor allem von dir hören. Nicht das
               beiläufige über den Grillparzer Preis\orgindex{Franz-Grillparzer-Preis@Franz-Grillparzer-Preis|pw}
               meine ich, ſondern das {\pb}allgemeine. Jemand, der heute
               deinen Artikel\pwindex{Bahr, Hermann 19.07.1863 – 15.01.1934@\textsc{Bahr, Hermann} (19.07.1863 – 15.01.1934), \emph{Schriftsteller, Kritiker}!Grillparzerpreis23. 01. 1902@\strich\emph{Der Grillparzerpreis} {[}23. 01. 1902{]}|pwv} las, ſagte: »Es
               iſt ganz einfach, Ihr seid \introOben{}a\textcolor{gray}{l}\damage{le}\introOben{} beide mit der Zeit anſtändige Leute geworden.«\pend
           \pstart
           herzlichen Gruſs{\\[\baselineskip]}dein \spacefill\mbox{Arthur}\pend
           \leftskip=0em{}\pstart
           2\substVorne{}\textsuperscript{4}\substDazwischen{}5\substHinten{}. 1. 902\pend
           
         
         \endnumbering\mylabel{h}\end{ledgroupsized}  \newcommand{\dateiname}{L01199}\newcommand{\titel}{Arthur Schnitzler an Hermann Bahr, 25. 1. 1902}\newcommand{\editorInnen}{ Kurt Ifkovits,  Martin Anton Müller}%% latex-leseansicht-abspann.tex
%% Abspann für die Leseansicht.
%% Der Schalter \ifkorrekturansicht ist bereits durch den Vorspann gesetzt.

%% latex-abspann.tex
%% Gemeinsamer Abspann für Korrekturansicht und Leseansicht.
%% Setzt den Schalter \ifkorrekturansicht voraus (gesetzt in den
%% einbindenden Dateien latex-korrekturansicht-abspann.tex bzw.
%% latex-leseansicht-abspann.tex).
%% ---------------------------------------------------------------

\normalsize

% Das esempio-Environment wird nur in der Leseansicht benötigt
\ifkorrekturansicht\else
\newenvironment{esempio}[3]%
{
    \vspace{1.5ex}
    \rlap{\underline{#1}}
    \par
    \setlength{\parindent}{0cm}
    \nopagebreak
    \leftskip=#2cm
    \rightskip=#3cm
}
{
    \par
}
\fi

\doendnotes{C}
\bigskip
\vfill

\clearpage

\footnotesize

\ifkorrekturansicht
  \lohead{\textsc{register}}
\fi

% theindex-Environment neu definieren ohne reledmac
\makeatletter
\renewenvironment{theindex}{%
  \ifkorrekturansicht
    \section*{\indexname}%
  \else
    \subsubsection*{Index der erwähnten Entitäten}%
  \fi
  \setlength{\parindent}{0pt}%
  \setlength{\parskip}{0pt plus 0.3pt}%
  \let\item\@idxitem
}{%
  \ifkorrekturansicht\clearpage\fi
}
\makeatother

\IfFileExists{\jobname-pw.ind}{\input{\jobname-pw.ind}}{}

% Quellenangabe nur in der Leseansicht
\ifkorrekturansicht\else
% Fallback-Definitionen, falls die .tex-Datei \titel etc. nicht gesetzt hat
\providecommand{\titel}{}
\providecommand{\editorInnen}{}
\providecommand{\dateiname}{\jobname}

\vspace{3cm}

\vfill

\footnotesize
\textsc{Quelle}: \titel. Herausgegeben von {\editorInnen}. In: \emph{Arthur Schnitzler: Briefwechsel mit Autorinnen und Autoren}.
 Digitale Edition, https://schnitzler-briefe.acdh.oeaw.ac.at/{\dateiname}.html (Stand \today)
\fi

\end{document}


      