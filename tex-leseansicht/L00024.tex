%% latex-korrekturansicht-vorspann.tex
%% Vorspann für die Korrekturansicht.
%% Lädt die gemeinsame Datei latex-vorspann.tex mit gesetztem Schalter.

\newif\ifkorrekturansicht
\korrekturansichttrue

\input{../tex-inputs/latex-vorspann}


\section[Max Burckhard an Arthur Schnitzler, 14. 7. 1891]{L00024 Max Burckhard an Arthur Schnitzler, 14. 7. 1891}
\nopagebreak\mylabel{L00024v}
\rehead{ }\normalsize\beginnumbering\briefempfaengerindex{Schnitzler, Arthur@\textsc{Schnitzler, Arthur}!zzzBurckhard, Max Eugen@\emph{von Max Eugen Burckhard}!1891-07-141@{14. 7. 1891}|(be}
\toendnotes[C]{\smallbreak\pagebreak[2]}\Standort{DLA, A:Schnitzler, HS.NZ85.1.2665, S. [2].}
\physDesc{Brief, maschinenschriftliche Abschrift1 Blatt, 1 Seite, 398 Zeichen
\newline{}Schreibmaschine}
\buchAbdrucke{\weitereDrucke{1) \pwindex{Schnitzlers Einzug ins Burgtheater@\emph{Schnitzlers Einzug ins Burgtheater}|pwk}\pwindex{Neue Freie Presse@\emph{Neue Freie Presse}|pwk}\emph{Neue Freie Presse}, Nr. 24162, 19. 12. 1931, S. 14.} \weitereDrucke{2) \pwindex{Schnitzlers Einzug ins Burgtheater@\emph{Schnitzlers Einzug ins Burgtheater}|pwk}\emph{Wiener Studien und Dokumente}. Wien: \emph{Steyrermühl} 1933, S. 166–168.} \weitereDrucke{3) Hans-Ulrich Lindken: \emph{Arthur Schnitzler. Aspekte und Akzente. Materialien zu Leben
                        und Werk}. Frankfurt am Main, Bern, Göttingen: \emph{Peter Lang} 1984, S. 243–246.} }\toendnotes[C]{\smallbreak}
\pstart
           \raggedleft{}{\pb}Wien\oindex{Wien@\textbf{Wien}, \emph{A.ADM2}|pw}, 14. Juli 1891.\pend
           
\pstart\center{}Sehr geehrter Herr Doctor!\pend\vspace{0.5em}
\pstart
           Mit grossem Interesse habe ich Ihr liebenswürdig phantastisches dramatisches Gedicht
                  Alkandis Lied\pwindex{Alkandi s Lied@\emph{Alkandi’s Lied}|pw} gelesen. Leider gestatten mir
               die Repertoir{[}e{]}verhältnisse nicht, auf die Aufführung von
               Einaktern so viel Mühe \label{T_L00024-1v}\edtext{zu}{\lemma{\textnormal{\emph{zu}}}\Cendnote{\textnormal{Die Abschrift hat »uu«.}}}\label{T_L00024-1}
               verwenden, als dies bei Kostümstücken, und speziell bei vorliegendem der Fall sein
               müsste.\pend
           
\pstart
           Mit verbindlichstem Danke{\\[\baselineskip]}hochachtungsvoll{\\[\baselineskip]}\spacefill\mbox{Dr. Burckhard.}\pend
           \leftskip=0em{}\selectlanguage{ngerman}\endnumbering\briefempfaengerindex{Schnitzler, Arthur@\textsc{Schnitzler, Arthur}!zzzBurckhard, Max Eugen@\emph{von Max Eugen Burckhard}!1891-07-141@{14. 7. 1891}|)be}\mylabel{L00024h}  \normalsize

\doendnotes{C}
\bigskip
\vfill

\clearpage

\footnotesize

\lohead{\textsc{register}}

% Definiere theindex-Environment komplett neu ohne reledmac
\makeatletter
\renewenvironment{theindex}{%
  \section*{\indexname}%
  \setlength{\parindent}{0pt}%
  \setlength{\parskip}{0pt plus 0.3pt}%
  \let\item\@idxitem
}{%
  \clearpage
}
\makeatother

\IfFileExists{\jobname-pw.ind}{\input{\jobname-pw.ind}}{}

\end{document}

      