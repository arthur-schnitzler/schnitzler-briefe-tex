\input{../tex-inputs/latex-pdf-vorspann}
\begin{center}
            \textcolor{red}{ENTWURF. ENTZIFFERUNG NOCH NICHT KORREKTURGELESEN}
                      \end{center}
            
               \section[Max Burckhard an Arthur Schnitzler, 14. 7. 1891]{ Max Burckhard an Arthur Schnitzler, 14. 7. 1891}\nopagebreak\mylabel{v}\rehead{ }\begin{ledgroupsized}[t]{13cm}\normalsize\beginnumbering\briefempfaengerindex{Schnitzler, Arthur@\textsc{Schnitzler, Arthur}!zzzBurckhard, Max Eugen@\emph{von Max Eugen Burckhard}!1891-07-141@{14. 7. 1891}|(be} \toendnotes[C]{\smallbreak\pagebreak[2]} \Standort{DLA, A:Schnitzler, HS.NZ85.1.2665, S. [2].}
\physDesc{maschinelle Abschrift}\buchAbdrucke{\weitereDrucke{1) \pwindex{Glossy, Karl 07.03.1848 – 09.09.1937@\textsc{Glossy, Karl} (07.03.1848 – 09.09.1937), \emph{Schriftsteller, Museumsleiter, Zensurbeirat}!Schnitzlers Einzug ins Burgtheater19. 12. 1931@\strich\emph{Schnitzlers Einzug ins Burgtheater} {[}19. 12. 1931{]}|pwk}\pwindex{Neue Freie Presse1864 – 1939@\emph{Neue Freie Presse}|pwk}Karl Glossy: \emph{Schnitzlers Einzug ins Burgtheater. Unbekannte Briefe des Dichters.} In: \emph{Neue Freie Presse}, Nr. 24162, 19. 12. 1931, S. 14.} \weitereDrucke{2) \pwindex{Glossy, Karl 07.03.1848 – 09.09.1937@\textsc{Glossy, Karl} (07.03.1848 – 09.09.1937), \emph{Schriftsteller, Museumsleiter, Zensurbeirat}!Schnitzlers Einzug ins Burgtheater19. 12. 1931@\strich\emph{Schnitzlers Einzug ins Burgtheater} {[}19. 12. 1931{]}|pwk}Karl Glossy: \emph{Schnitzlers Einzug ins Burgtheater. Unbekannte Briefe des Dichters.} In: \emph{Wiener Studien und Dokumente}. Zum 85. Geburtstag des Verfassers hg. von seinen
                                Freunden. Wien: \emph{Steyrermühl} 1933, S. 166–168.} \weitereDrucke{3) Hans-Ulrich Lindken: \emph{Arthur Schnitzler. Aspekte und Akzente. Materialien zu
                                Leben und Werk}. Frankfurt am Main, Bern, Göttingen: \emph{Peter Lang} 1984, S. 243–246 (Europäische Hochschulschriften, Reihe 1, Deutsche
                                Sprache und Literatur, 754).} }\toendnotes[C]{\smallbreak}\pstart
           \raggedleft{}{\pb}Wien\oindex{Wien@\textbf{Wien}|pw}, 14. Juli 1891.\pend
           \pstart\center{}Sehr geehrter Herr Doctor!\pend\pstart
           Mit grossem Interesse habe ich Ihr liebenswürdig phantastisches dramatisches
                    Gedicht Alkandis Lied\pwindex{Schnitzler, Arthur 15.05.1862 – 21.10.1931@\textsc{Schnitzler, Arthur} (15.05.1862 – 21.10.1931), \emph{Schriftsteller, Mediziner}!Alkandi s Lied15.8.1890 – 1.9.1890@\strich\emph{Alkandi’s Lied} {[}15.8.1890 – 1.9.1890{]}|pw} gelesen. Leider
                    gestatten mir die Repertoir{[}e{]}verhältnisse nicht, auf die
                    Aufführung von Einaktern so viel Mühe \label{T_L00024_1v}\edtext{zu}{\lemma{\textnormal{\emph{zu}}}\Cendnote{\textnormal{die Abschrift hat
                            »uu«}}}\label{T_L00024_1h} verwenden, als dies bei Kostümstücken, und
                    speziell bei vorliegendem der Fall sein müsste.\pend
           \pstart
           Mit verbindlichstem Danke{\\[\baselineskip]}hochachtungsvoll{\\[\baselineskip]}\spacefill\mbox{Dr. Burckhard.}\pend
           \leftskip=0em{}\endnumbering\briefempfaengerindex{Schnitzler, Arthur@\textsc{Schnitzler, Arthur}!zzzBurckhard, Max Eugen@\emph{von Max Eugen Burckhard}!1891-07-141@{14. 7. 1891}|)be}\mylabel{h}\end{ledgroupsized}  \newcommand{\dateiname}{L00024}\newcommand{\titel}{Max Burckhard an Arthur Schnitzler, 14. 7. 1891}\newcommand{\editorInnen}{Martin Anton Müller und Gerd-Hermann Susen}\input{../tex-inputs/latex-pdf-abspann}
      