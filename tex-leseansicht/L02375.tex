%% latex-leseansicht-vorspann.tex
%% Vorspann für die Leseansicht.
%% Lädt die gemeinsame Datei latex-vorspann.tex mit nicht gesetztem Schalter.

\newif\ifkorrekturansicht
\korrekturansichtfalse

\input{../tex-inputs/latex-vorspann}


         
         \renewcommand{\erwaehntePersonen}{Personen: Frieda Pollak, Berta Zuckerkandl}
         \renewcommand{\erwaehnteInstitutionen}{Institutionen: Rikola Verlag}
         \renewcommand{\erwaehnteOrte}{Orte: Rodaun, Salzburg, Stallburggasse, Wien}
         \renewcommand{\erwaehnteWerke}{Werke: Tod und Verklärung op. 24}
               \section[Hugo Hofmannsthal an Arthur Schnitzler, 28. 1. 192{[}2{]}]{ Hugo Hofmannsthal an Arthur Schnitzler, 28. 1. 192{[}2{]}}\nopagebreak\mylabel{v}\rehead{ }\begin{ledgroupsized}[t]{13cm}\normalsize\beginnumbering \toendnotes[C]{\smallbreak\pagebreak[2]} \Standort{CUL, Schnitzler, B 43.}
\physDesc{Brief, 1 Blatt, 4 Seiten, 1382 Zeichen
\newline{}Handschrift: schwarze Tinte, lateinische Kurrent
\newline{}Schnitzler: mit Bleistift beschriftet: »\textsc{Hugo}« und die hintere Ziffer der Jahreszahl des Datums durch
                                 Überschreiben korrigiert: »2« 
\newline{}Ordnung: 1) mit Bleistift von Frieda
                                    Pollak\pwindex{Pollak, Frieda 08.12.1881 – 13.07.1937@\textsc{Pollak, Frieda} (08.12.1881 – 13.07.1937), \emph{Sekretärin}|pw} (?) mit dem Buchstaben »A«
                                 (Abgeschrieben/Abschrift) gekennzeichnet  2) mit Bleistift von unbekannter Hand nummeriert: »\strikeout{367}« 3) mit Bleistift von unbekannter Hand nummeriert:
                                    »371«}\buchAbdrucke{\weitereDrucke{Hugo von Hofmannsthal, Arthur Schnitzler: \emph{Briefwechsel}. Hg. Therese Nickl und Heinrich Schnitzler. Frankfurt am Main: \emph{S. Fischer} 1964, S. 295.} }\toendnotes[C]{\smallbreak}\pstart
           \raggedleft{}{\pb}Rodaun\oindex{Rodaun@\textbf{Rodaun}|pw}{ }28 I 21. \pend
           \pstart{}mein lieber Arthur\pend\pstart
           es freut mich riesig von B. Z.\pwindex{Zuckerkandl, Berta 13.04.1864 – 16.10.1945@\textsc{Zuckerkandl, Berta} (13.04.1864 – 16.10.1945), \emph{Journalistin, Übersetzerin}|pw} zu hören dass
               Sie zu dem \label{K_L02375-1v}\edtext{Vorlesen}{\lemma{\textnormal{\emph{Vorlesen}}}\Cendnote{\textnormal{Die Vorlesung fand am 3. 2. 1922 im Salon
                  von Berta Zuckerkandl\pwindex{Zuckerkandl, Berta 13.04.1864 – 16.10.1945@\textsc{Zuckerkandl, Berta} (13.04.1864 – 16.10.1945), \emph{Journalistin, Übersetzerin}|pwk} statt.}}}\label{K_L02375-1h} des Welttheaters\pwindex{\textcolor{red}{\textsuperscript{XXXX1 indx}}!Tod und Verklaerung op. 241890@\strich\emph{Tod und Verklärung op. 24} {[}1890{]}|pw} ko{\geminationm}en
               wollen – es ist ja keine Vorles\uline{ung}, sondern wirklich
               ein bescheidenes \uline{Vorlesen} an ein paar alte und ein
               paar neuere Bekannte u. Fremde an diesem zwanglosen neutralen Ort (in der Stallburggasse\oindex{Stallburggasse@\textbf{Stallburggasse}|pw} sind nur 8 Sessel und das {\pb}grosse Zi{\geminationm}er heizt sich elend) und es ist mir natürlich ein
               liebes Geschenk, dass Sie da sein wollen.\pend
           \pstart
           Es ist mir immer ein bissl trüb in Erinnerung dass ich Sie, einen so nahen Menschen,
               mit dem ich mir nie im Leben \uline{halb} begegnet bin, in
               diesem So{\geminationm}er nur in diesen Salzburg\oindex{Salzburg@\textbf{Salzburg}|pw}er Tagen gesehen habe, in einem noch {\pb}währenden Übelbefinden u. einer
               Beschäftigtheit wie sie dort entsteht (sie bezog sich ja auf das noch unentstandene
               Welttheater) – nur wie durch einen Schleier. –\pend
           \pstart
           Ich bitte Sie um einen Rat, Arthur, den Sie mir am Freitag{ }\uline{mündlich} geben können. Während ich um Broterwerbes
               willen {\pb}fast über meine Kräfte
               Arbeit auf mich nehme (Schriftstellerische, nicht Dichterische, die muss ich fast
               zurückdrängen) bin ich andererseits unvertraut mit dem was man in Anpassung an den
               veränderten Zustand verlangen u. beko{\geminationm}en müsste: so
               dies: welche Forderung hätte man (Sie oder ich, wir kommen beide in Frage) für
               Überlassung eines Werkes für eine Luxusausgabe an den Rikolaverlag\orgindex{Rikola Verlag@Rikola Verlag|pw} vernünftigerweise einmalig zu verlangen?\pend
           \pstart
           \label{T_L02375-1v}\edtext{Also hoffentlich auf Wiedersehen
                  Freitag!}{\lemma{\textnormal{\emph{Also … Freitag!}}}\Cendnote{\textnormal{ab hier quer am
                  linken Rand}}}\label{T_L02375-1h}\pend
           \pstart Ihr \spacefill\mbox{Hugo.}\pend{}
         
         \endnumbering\mylabel{h}\end{ledgroupsized}  \newcommand{\dateiname}{L02375}\newcommand{\titel}{Hugo Hofmannsthal an Arthur Schnitzler, 28. 1. 192[2]}\newcommand{\editorInnen}{Martin Anton Müller und Gerd-Hermann Susen}%% latex-leseansicht-abspann.tex
%% Abspann für die Leseansicht.
%% Der Schalter \ifkorrekturansicht ist bereits durch den Vorspann gesetzt.

%% latex-abspann.tex
%% Gemeinsamer Abspann für Korrekturansicht und Leseansicht.
%% Setzt den Schalter \ifkorrekturansicht voraus (gesetzt in den
%% einbindenden Dateien latex-korrekturansicht-abspann.tex bzw.
%% latex-leseansicht-abspann.tex).
%% ---------------------------------------------------------------

\normalsize

% Das esempio-Environment wird nur in der Leseansicht benötigt
\ifkorrekturansicht\else
\newenvironment{esempio}[3]%
{
    \vspace{1.5ex}
    \rlap{\underline{#1}}
    \par
    \setlength{\parindent}{0cm}
    \nopagebreak
    \leftskip=#2cm
    \rightskip=#3cm
}
{
    \par
}
\fi

\doendnotes{C}
\bigskip
\vfill

\clearpage

\footnotesize

\ifkorrekturansicht
  \lohead{\textsc{register}}
\fi

% theindex-Environment neu definieren ohne reledmac
\makeatletter
\renewenvironment{theindex}{%
  \ifkorrekturansicht
    \section*{\indexname}%
  \else
    \subsubsection*{Index der erwähnten Entitäten}%
  \fi
  \setlength{\parindent}{0pt}%
  \setlength{\parskip}{0pt plus 0.3pt}%
  \let\item\@idxitem
}{%
  \ifkorrekturansicht\clearpage\fi
}
\makeatother

\IfFileExists{\jobname-pw.ind}{\input{\jobname-pw.ind}}{}

% Quellenangabe nur in der Leseansicht
\ifkorrekturansicht\else
% Fallback-Definitionen, falls die .tex-Datei \titel etc. nicht gesetzt hat
\providecommand{\titel}{}
\providecommand{\editorInnen}{}
\providecommand{\dateiname}{\jobname}

\vspace{3cm}

\vfill

\footnotesize
\textsc{Quelle}: \titel. Herausgegeben von {\editorInnen}. In: \emph{Arthur Schnitzler: Briefwechsel mit Autorinnen und Autoren}.
 Digitale Edition, https://schnitzler-briefe.acdh.oeaw.ac.at/{\dateiname}.html (Stand \today)
\fi

\end{document}


      