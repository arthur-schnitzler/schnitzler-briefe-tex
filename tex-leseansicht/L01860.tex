%% latex-korrekturansicht-vorspann.tex
%% Vorspann für die Korrekturansicht.
%% Lädt die gemeinsame Datei latex-vorspann.tex mit gesetztem Schalter.

\newif\ifkorrekturansicht
\korrekturansichttrue

\input{../tex-inputs/latex-vorspann}


\section[Arthur Schnitzler an Anna von Liliencron, 24. 7. 1909]{L01860 Arthur Schnitzler an Anna von Liliencron, 24. 7. 1909}
\nopagebreak\mylabel{L01860v}
\rehead{ }\normalsize\beginnumbering\briefempfaengerindex{Liliencron, Anna von@\textsc{Liliencron, Anna von}!zzzSchnitzler, Arthur@\emph{von Arthur Schnitzler}!1909-07-242@{24. 7. 1909}|(be}
\toendnotes[C]{\smallbreak\pagebreak[2]}\Standort{Wien, Österreichische Gesellschaft für Literatur, Kopienarchiv Schnitzler, Liliencron.}
\physDesc{Telegramm, Fotokopie180 Zeichen
\newline{}Handschrift einer Schreibkraft: Bleistift, deutsche Kurrent
\newline{}Versand: »\noindent{}\begin{center}\textcolor{gray}{\textbf{Telegraphie des Deutschen Reichs\oindex{Deutschland@\textbf{Deutschland}, \emph{A.PCLI}|pw}.}}\end{center}{ / }\begin{center}\textcolor{gray}{\textbf{Amt}}{ }\textcolor{gray}{\textbf{\textit{Altrahlstedt}}}\oindex{Rahlstedt@\textbf{Rahlstedt}, \emph{P.PPLX}|pw}\end{center}{ / }\textcolor{gray}{\textbf{Leitung Nr.}} 471{ / }\textcolor{gray}{\textbf{Telegramm Nr.}} 20{ / }\textcolor{gray}{\textbf{\textbf{Aufgenommen} von}}{ }Hbg\oindex{Hamburg@\textbf{Hamburg}, \emph{P.PPLA}|pw}{ / }\textcolor{gray}{\textbf{am}}{ }24/7{ / }\textcolor{gray}{\textbf{um}}{ }5 \textcolor{gray}{\textbf{Uhr}} 20 \textcolor{gray}{\textbf{Min.}}n{ / }\textcolor{gray}{\textbf{durch}}{ }\textcolor{gray}{Ln}« 
\newline{}Ordnung: von unbekannter Hand beschriftet: »Kondolenztelegramm an
                                    die Witwe Detlev von
                                       Liliencrons\pwindex{Liliencron, Detlev von 03.06.1844 – 22.07.1909@\textsc{Liliencron, Detlev von} (03.06.1844 – 22.07.1909), \emph{Schriftsteller/Schriftstellerin, Dichter/Dichterin, Dramatiker/Dramatikerin}|pw}. Juli 1909 aus Edlach\oindex{Edlach@\textbf{Edlach}, \emph{P.PPL}|pw}« 
\newline{}Zusatz: Original nicht nachweisbar }\toendnotes[C]{\smallbreak}\pstart{}{\pb}Baronin Liliencron\pend{}\pstart{}\textcolor{gray}{\textbf{\textit{Altrahlstedt}}} b. Hmb\oindex{Rahlstedt@\textbf{Rahlstedt}, \emph{P.PPLX}|pw}\pend{}{\bigskip}\vspace{1em}
\pstart
           {\pb}\textcolor{gray}{\textbf{Telegramm aus}}{ }Edlach\oindex{Edlach@\textbf{Edlach}, \emph{P.PPL}|pw}\hfill 21 \textcolor{gray}{\textbf{W. den}}{ }24/7{ }\textcolor{gray}{\textbf{um}}{ }3 \textcolor{gray}{\textbf{Uhr}} 20 \textcolor{gray}{\textbf{Min.}}{ }n\pend
           \vspace{0.5em}
\pstart
           \strikeout{Lei} Innigſter Teilnahme an dem Hinſcheiden Ihres
               Gatten des verehrten und geliebten Dichters\pwindex{Liliencron, Detlev von 03.06.1844 – 22.07.1909@\textsc{Liliencron, Detlev von} (03.06.1844 – 22.07.1909), \emph{Schriftsteller/Schriftstellerin, Dichter/Dichterin, Dramatiker/Dramatikerin}|pwv} verſichert Sie\pend
           \pstart \spacefill\mbox{Arthur Schnitzler}\pend{}\selectlanguage{ngerman}\endnumbering\briefempfaengerindex{Liliencron, Anna von@\textsc{Liliencron, Anna von}!zzzSchnitzler, Arthur@\emph{von Arthur Schnitzler}!1909-07-242@{24. 7. 1909}|)be}\mylabel{L01860h}  \normalsize

\doendnotes{C}
\bigskip
\vfill

\clearpage

\footnotesize

\lohead{\textsc{register}}

% Definiere theindex-Environment komplett neu ohne reledmac
\makeatletter
\renewenvironment{theindex}{%
  \section*{\indexname}%
  \setlength{\parindent}{0pt}%
  \setlength{\parskip}{0pt plus 0.3pt}%
  \let\item\@idxitem
}{%
  \clearpage
}
\makeatother

\IfFileExists{\jobname-pw.ind}{\input{\jobname-pw.ind}}{}

\end{document}

      