%% latex-leseansicht-vorspann.tex
%% Vorspann für die Leseansicht.
%% Lädt die gemeinsame Datei latex-vorspann.tex mit nicht gesetztem Schalter.

\newif\ifkorrekturansicht
\korrekturansichtfalse

\input{../tex-inputs/latex-vorspann}


\section[Arthur Schnitzler an Anna von Liliencron, 24. 7. 1909]{L01860 Arthur Schnitzler an Anna von Liliencron, 24. 7. 1909}
\nopagebreak\mylabel{L01860v}
\rehead{ }\normalsize\beginnumbering\briefempfaengerindex{Liliencron, Anna von@\textsc{Liliencron, Anna von}!zzzSchnitzler, Arthur@\emph{von Arthur Schnitzler}!1909-07-242@{24. 7. 1909}|(be}
\toendnotes[C]{\smallbreak\pagebreak[2]}
\correspDesc{Versand  durch Arthur Schnitzler am 24. 7. 1909 in Edlach
\newline{}Erhalt  durch Anna von Liliencron am 24. 7. 1909 in Hamburg}\toendnotes[C]{\smallbreak}
\Standort{Wien, Österreichische Gesellschaft für Literatur, Kopienarchiv Schnitzler, Liliencron.}
\physDesc{Telegramm, Fotokopie, 180 Zeichen
\newline{}HandschriftX2 einer Schreibkraft: Bleistift, deutsche Kurrent
\newline{}Versand: »\noindent{}\begin{center}\textcolor{gray}{\textbf{Telegraphie des Deutschen Reichs\oindex{Deutschland@\textbf{Deutschland}|pw}.}}\end{center}{ / }\begin{center}\textcolor{gray}{\textbf{Amt}}{ }\textcolor{gray}{\textbf{\textit{Altrahlstedt}}}\oindex{Rahlstedt@\textbf{Rahlstedt}, \emph{Ehemaliger Ort}|pw}\end{center}{ / }\textcolor{gray}{\textbf{Leitung Nr.}} 471{ / }\textcolor{gray}{\textbf{Telegramm Nr.}} 20{ / }\textcolor{gray}{\textbf{\textbf{Aufgenommen} von}}{ }Hbg\oindex{Hamburg@\textbf{Hamburg}|pw}{ / }\textcolor{gray}{\textbf{am}}{ }24/7{ / }\textcolor{gray}{\textbf{um}}{ }5 \textcolor{gray}{\textbf{Uhr}} 20 \textcolor{gray}{\textbf{Min.}}n{ / }\textcolor{gray}{\textbf{durch}}{ }\textcolor{gray}{Ln}« 
\newline{}Ordnung: von unbekannter Hand beschriftet: »Kondolenztelegramm an
                                    die Witwe Detlev von
                                       Liliencrons\pwindex{Liliencron, Detlev von 3.\,6.\,1844 Kiel – 22.\,7.\,1909 Rahlstedt@\textsc{Liliencron, Detlev von} (3.\,6.\,1844 Kiel – 22.\,7.\,1909 Rahlstedt), \emph{Schriftsteller, Dichter, Dramatiker}|pw}. Juli 1909 aus Edlach\oindex{Edlach@\textbf{Edlach}|pw}« 
\newline{}Zusatz: Original nicht nachweisbar }\toendnotes[C]{\smallbreak}\pstart{}{\pb}Baronin Liliencron\pend{}\pstart{}\textcolor{gray}{\textbf{\textit{Altrahlstedt}}} b. Hmb\oindex{Rahlstedt@\textbf{Rahlstedt}, \emph{Ehemaliger Ort}|pw}\pend{}{\bigskip}\vspace{1em}
\pstart
           {\pb}\textcolor{gray}{\textbf{Telegramm aus}}{ }Edlach\oindex{Edlach@\textbf{Edlach}|pw}\hfill 21 \textcolor{gray}{\textbf{W. den}}{ }24/7{ }\textcolor{gray}{\textbf{um}}{ }3 \textcolor{gray}{\textbf{Uhr}} 20 \textcolor{gray}{\textbf{Min.}}{ }n\pend
           \vspace{0.5em}
\pstart
           \strikeout{Lei} Innigſter Teilnahme an dem Hinſcheiden Ihres
               Gatten des verehrten und geliebten Dichters\pwindex{Liliencron, Detlev von 3.\,6.\,1844 Kiel – 22.\,7.\,1909 Rahlstedt@\textsc{Liliencron, Detlev von} (3.\,6.\,1844 Kiel – 22.\,7.\,1909 Rahlstedt), \emph{Schriftsteller, Dichter, Dramatiker}|pwv} verſichert Sie\pend
           \pstart \spacefill\mbox{Arthur Schnitzler}\pend{}\selectlanguage{ngerman}\endnumbering\briefempfaengerindex{Liliencron, Anna von@\textsc{Liliencron, Anna von}!zzzSchnitzler, Arthur@\emph{von Arthur Schnitzler}!1909-07-242@{24. 7. 1909}|)be}\mylabel{L01860h}  \newcommand{\dateiname}{L01860}\newcommand{\titel}{Arthur Schnitzler an Anna von Liliencron, 24. 7. 1909}\newcommand{\editorInnen}{Martin Anton Müller und Gerd-Hermann Susen}%% latex-leseansicht-abspann.tex
%% Abspann für die Leseansicht.
%% Der Schalter \ifkorrekturansicht ist bereits durch den Vorspann gesetzt.

%% latex-abspann.tex
%% Gemeinsamer Abspann für Korrekturansicht und Leseansicht.
%% Setzt den Schalter \ifkorrekturansicht voraus (gesetzt in den
%% einbindenden Dateien latex-korrekturansicht-abspann.tex bzw.
%% latex-leseansicht-abspann.tex).
%% ---------------------------------------------------------------

\normalsize

% Das esempio-Environment wird nur in der Leseansicht benötigt
\ifkorrekturansicht\else
\newenvironment{esempio}[3]%
{
    \vspace{1.5ex}
    \rlap{\underline{#1}}
    \par
    \setlength{\parindent}{0cm}
    \nopagebreak
    \leftskip=#2cm
    \rightskip=#3cm
}
{
    \par
}
\fi

\doendnotes{C}
\bigskip
\vfill

\clearpage

\footnotesize

\ifkorrekturansicht
  \lohead{\textsc{register}}
\fi

% theindex-Environment neu definieren ohne reledmac
\makeatletter
\renewenvironment{theindex}{%
  \ifkorrekturansicht
    \section*{\indexname}%
  \else
    \subsubsection*{Index der erwähnten Entitäten}%
  \fi
  \setlength{\parindent}{0pt}%
  \setlength{\parskip}{0pt plus 0.3pt}%
  \let\item\@idxitem
}{%
  \ifkorrekturansicht\clearpage\fi
}
\makeatother

\IfFileExists{\jobname-pw.ind}{\input{\jobname-pw.ind}}{}

% Quellenangabe nur in der Leseansicht
\ifkorrekturansicht\else
% Fallback-Definitionen, falls die .tex-Datei \titel etc. nicht gesetzt hat
\providecommand{\titel}{}
\providecommand{\editorInnen}{}
\providecommand{\dateiname}{\jobname}

\vspace{3cm}

\vfill

\footnotesize
\textsc{Quelle}: \titel. Herausgegeben von {\editorInnen}. In: \emph{Arthur Schnitzler: Briefwechsel mit Autorinnen und Autoren}.
 Digitale Edition, https://schnitzler-briefe.acdh.oeaw.ac.at/{\dateiname}.html (Stand \today)
\fi

\end{document}


