%% latex-leseansicht-vorspann.tex
%% Vorspann für die Leseansicht.
%% Lädt die gemeinsame Datei latex-vorspann.tex mit nicht gesetztem Schalter.

\newif\ifkorrekturansicht
\korrekturansichtfalse

\input{../tex-inputs/latex-vorspann}


\section[Elsa Plessner an Arthur Schnitzler, 21.\,1.\,1899]{L03721 Elsa Plessner an Arthur Schnitzler, 21.\,1.\,1899}
\nopagebreak\mylabel{L03721v}
\rehead{ }\normalsize\beginnumbering\briefempfaengerindex{Schnitzler, Arthur@\textsc{Schnitzler, Arthur}!zzzPlessner, Elsa@\emph{von Elsa Plessner}!1899-01-211@{21.\,1.\,1899}|(be}
\toendnotes[C]{\smallbreak\pagebreak[2]}
\correspDesc{Versand  durch Elsa Plessner am 21. 1. 1899 in Wien
\newline{}Erhalt  durch Arthur Schnitzler im Zeitraum [22. 1. 1899
                  – 26. 1. 1899?] in Wien}\toendnotes[C]{\smallbreak}
\Standort{DLA, A:Schnitzler, HS.1985.1.419.}
\physDesc{Brief, 1 Blatt, 1 Seite, 340 Zeichen (Briefpapier mit Blumenmotiv (Klee))
\newline{}Handschrift: schwarze Tinte, lateinische Kurrent}\toendnotes[C]{\smallbreak}
\pstart
           \raggedleft{}{\pb}den 21./1. 99.\pend
           \vspace{0.5em}
\pstart
           Meine Familie\pwindex{Askonas, Johanna Leonie 20.\,11.\,1877 Wien – 30.\,7.\,1930 ebd.@\textsc{Askonas, Johanna Leonie} (20.\,11.\,1877 Wien – 30.\,7.\,1930 ebd.), \emph{Pensionsinhaberin}|pwv}\pwindex{Plessner, Clementine 7.\,12.\,1855 Wien – 27.\,2.\,1943 Konzentrationslager Theresienstadt@\textsc{Plessner, Clementine} (7.\,12.\,1855 Wien – 27.\,2.\,1943 Konzentrationslager Theresienstadt), \emph{Schauspielerin, Filmschauspielerin}|pwv}
               hat von mir ein Stück ertrotzt!! Rache – \uline{dieses}{ }Stück\pwindex{Plessner, Elsa 22.\,8.\,1875 Wien – 7.\,5.\,1932 Alicante@\textsc{Plessner, Elsa} (22.\,8.\,1875 Wien – 7.\,5.\,1932 Alicante), \emph{Schriftstellerin}!Ehrlosen. Schauspiel in drei Acten@\strich\emph{Die Ehrlosen. Schauspiel in drei Acten}|pwv}!! \label{K_L03721-1v}\edtext{»Singende
                  Flammen\pwindex{Plessner, Elsa 22.\,8.\,1875 Wien – 7.\,5.\,1932 Alicante@\textsc{Plessner, Elsa} (22.\,8.\,1875 Wien – 7.\,5.\,1932 Alicante), \emph{Schriftstellerin}!Ehrlosen. Schauspiel in drei Acten@\strich\emph{Die Ehrlosen. Schauspiel in drei Acten}|pw}«}{\lemma{\textnormal{\emph{»Singende
                  Flammen«}}}\Cendnote{\textnormal{Es handelt sich um e\emph{Die Ehrlosen}\pwindex{Plessner, Elsa 22.\,8.\,1875 Wien – 7.\,5.\,1932 Alicante@\textsc{Plessner, Elsa} (22.\,8.\,1875 Wien – 7.\,5.\,1932 Alicante), \emph{Schriftstellerin}!Ehrlosen. Schauspiel in drei Acten@\strich\emph{Die Ehrlosen. Schauspiel in drei Acten}|pwk}, vgl. XXXX Auszeichnungsfehler: Dokument L03725 nicht gefunden.}}}\label{K_L03721-1} für Vettern und Tanten!\pend
           
\pstart
           Sie sind ein Engel –! Ich habe Angst! – ! Schließlich sende ich Ihnen
               herzlich Grüße! – Und Angst habe ich! – – Furchtbare!! – Seien Sie voll gnädiger \uline{Strenge}! – Sie haben bis jetzt bei mir immer Recht
               gehabt!\pend
           \pstart \spacefill\mbox{Elsa Plessner.}\pend{}\selectlanguage{ngerman}\endnumbering\briefempfaengerindex{Schnitzler, Arthur@\textsc{Schnitzler, Arthur}!zzzPlessner, Elsa@\emph{von Elsa Plessner}!1899-01-211@{21.\,1.\,1899}|)be}\mylabel{L03721h}  \newcommand{\dateiname}{L03721}\newcommand{\titel}{Elsa Plessner an Arthur Schnitzler, 21. 1. 1899}\newcommand{\editorInnen}{Selma Jahnke und Martin Anton Müller}%% latex-leseansicht-abspann.tex
%% Abspann für die Leseansicht.
%% Der Schalter \ifkorrekturansicht ist bereits durch den Vorspann gesetzt.

%% latex-abspann.tex
%% Gemeinsamer Abspann für Korrekturansicht und Leseansicht.
%% Setzt den Schalter \ifkorrekturansicht voraus (gesetzt in den
%% einbindenden Dateien latex-korrekturansicht-abspann.tex bzw.
%% latex-leseansicht-abspann.tex).
%% ---------------------------------------------------------------

\normalsize

% Das esempio-Environment wird nur in der Leseansicht benötigt
\ifkorrekturansicht\else
\newenvironment{esempio}[3]%
{
    \vspace{1.5ex}
    \rlap{\underline{#1}}
    \par
    \setlength{\parindent}{0cm}
    \nopagebreak
    \leftskip=#2cm
    \rightskip=#3cm
}
{
    \par
}
\fi

\doendnotes{C}
\bigskip
\vfill

\clearpage

\footnotesize

\ifkorrekturansicht
  \lohead{\textsc{register}}
\fi

% theindex-Environment neu definieren ohne reledmac
\makeatletter
\renewenvironment{theindex}{%
  \ifkorrekturansicht
    \section*{\indexname}%
  \else
    \subsubsection*{Index der erwähnten Entitäten}%
  \fi
  \setlength{\parindent}{0pt}%
  \setlength{\parskip}{0pt plus 0.3pt}%
  \let\item\@idxitem
}{%
  \ifkorrekturansicht\clearpage\fi
}
\makeatother

\IfFileExists{\jobname-pw.ind}{\input{\jobname-pw.ind}}{}

% Quellenangabe nur in der Leseansicht
\ifkorrekturansicht\else
% Fallback-Definitionen, falls die .tex-Datei \titel etc. nicht gesetzt hat
\providecommand{\titel}{}
\providecommand{\editorInnen}{}
\providecommand{\dateiname}{\jobname}

\vspace{3cm}

\vfill

\footnotesize
\textsc{Quelle}: \titel. Herausgegeben von {\editorInnen}. In: \emph{Arthur Schnitzler: Briefwechsel mit Autorinnen und Autoren}.
 Digitale Edition, https://schnitzler-briefe.acdh.oeaw.ac.at/{\dateiname}.html (Stand \today)
\fi

\end{document}


