%% latex-korrekturansicht-vorspann.tex
%% Vorspann für die Korrekturansicht.
%% Lädt die gemeinsame Datei latex-vorspann.tex mit gesetztem Schalter.

\newif\ifkorrekturansicht
\korrekturansichttrue

\input{../tex-inputs/latex-vorspann}


\section[Elsa Plessner an Arthur Schnitzler, 21. 1. 1899]{L03721 Elsa Plessner an Arthur Schnitzler, 21. 1. 1899}
\nopagebreak\mylabel{L03721v}
\rehead{ }\normalsize\beginnumbering\briefempfaengerindex{Schnitzler, Arthur@\textsc{Schnitzler, Arthur}!zzzPlessner, Elsa@\emph{von Elsa Plessner}!1899-01-211@{21. 1. 1899}|(be}
\toendnotes[C]{\smallbreak\pagebreak[2]}\Standort{DLA, A:Schnitzler, HS.1985.1.419.}
\physDesc{Brief, 1 Blatt, 1 Seite, 341 Zeichen (Briefpapier mit Blumenmotiv (Klee))
\newline{}Handschrift: , lateinische Kurrent}\toendnotes[C]{\smallbreak}
\pstart
           {\pb}
                  den
                  21. /1. 99.\pend
           \vspace{0.5em}
\pstart
           Meine
               Familie\pwindex{Askonas, Johanna Leonie 1877-11-20 – 1930-07-30@\textsc{Askonas, Johanna Leonie} (1877-11-20 – 1930-07-30), \emph{Pensionsinhaber/Pensionsinhaberin}|pwv}\pwindex{Plessner, Clementine 1855-12-07 – 1943-02-27@\textsc{Plessner, Clementine} (1855-12-07 – 1943-02-27), \emph{Schauspieler/Schauspielerin, Filmschauspieler/Filmschauspielerin}|pwv} hat von
      mir ein Stück ertrotzt!!
      Rache – \uline{dieses}{ }Stück\pwindex{Ehrlosen. Schauspiel in drei Acten@\emph{Die Ehrlosen. Schauspiel in drei Acten}|pwv}!!
               \label{K_L03721-1v}\edtext{»Singende Flammen\pwindex{Ehrlosen. Schauspiel in drei Acten@\emph{Die Ehrlosen. Schauspiel in drei Acten}|pw}«}{\lemma{\textnormal{\emph{»Singende Flammen«}}}\Cendnote{\textnormal{Bei dem dem Brief beiliegenden Schauspiel handelte es sich vermutlich um eine frühe Version von \emph{Die Ehrlosen}\pwindex{Ehrlosen. Schauspiel in drei Acten@\emph{Die Ehrlosen. Schauspiel in drei Acten}|pwk}.}}}\label{K_L03721-1}
      für Vettern und Tanten!
      \pend
           
\pstart
           Sie sind ein Engel –
      Ich habe Angst! – !
      Schließlich sende ich
      Ihnen herzlich Grüße! –
      Und Angst habe ich! – –
      Furchtbare!! – Seien Sie
      voll gnädiger \uline{Strenge}! – Sie
      haben bis jetzt bei mir immer
      Recht gehabt!
      \pend
           \pstart \spacefill\mbox{Elsa Plessner}\pend{}\selectlanguage{ngerman}\endnumbering\briefempfaengerindex{Schnitzler, Arthur@\textsc{Schnitzler, Arthur}!zzzPlessner, Elsa@\emph{von Elsa Plessner}!1899-01-211@{21. 1. 1899}|)be}\mylabel{L03721h}
\begin{anhang}
\end{anhang}\normalsize

\doendnotes{C}
\bigskip
\vfill

\clearpage

\footnotesize

\lohead{\textsc{register}}

% Definiere theindex-Environment komplett neu ohne reledmac
\makeatletter
\renewenvironment{theindex}{%
  \section*{\indexname}%
  \setlength{\parindent}{0pt}%
  \setlength{\parskip}{0pt plus 0.3pt}%
  \let\item\@idxitem
}{%
  \clearpage
}
\makeatother

\IfFileExists{\jobname-pw.ind}{\input{\jobname-pw.ind}}{}

\end{document}

      