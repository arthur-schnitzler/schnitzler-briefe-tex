\input{../tex-inputs/latex-pdf-vorspann}
\begin{center}
            \textcolor{red}{ENTWURF. ENTZIFFERUNG NOCH NICHT KORREKTURGELESEN}
                      \end{center}
            
               \section[Georg Brandes an Arthur Schnitzler, {[}11. 5. 1923{]}]{ Georg Brandes an Arthur Schnitzler, {[}11. 5. 1923{]}}\nopagebreak\mylabel{v}\rehead{ }\begin{ledgroupsized}[t]{13cm}\normalsize\beginnumbering\briefempfaengerindex{Schnitzler, Arthur@\textsc{Schnitzler, Arthur}!zzzBrandes, Georg@\emph{von Georg Brandes}!1923-05-111@{{[}11. 5. 1923{]}}|(be} \toendnotes[C]{\smallbreak\pagebreak[2]} \Standort{Kopenhagen, Det Kongelige Bibliotek, Georg Brandes Arkiv, box 125.}
\physDesc{Brief, 1 Blatt, 1 Seite
\newline{}Handschrift: Bleistift, lateinische Kurrent
\newline{}Schnitzler: datiert: »Mai 923« \newline{}Ordnung: 1) mit Bleistift von unbekannter Hand in der rechten oberen Ecke notiert: »\uline{erg.}« 2) mit Bleistift von unbekannter Hand nummeriert: »53«}\buchAbdrucke{\weitereDrucke{Georg Brandes, Arthur Schnitzler: \emph{Ein Briefwechsel}. Hg. Kurt Bergel. Bern: \emph{Francke} 1956, S. 138.} }\toendnotes[C]{\smallbreak}\pstart
           \raggedleft{}{\pb}Allégade 31\oindex{Allegade@\textbf{Allégade}|pw}{\\}Dr. Meisens\pwindex{Meisen, Valdemar 20.07.1878 – 18.04.1934@\textsc{Meisen, Valdemar} (20.07.1878 – 18.04.1934), \emph{Mediziner, Chirurg, Sanatoriumsleiter}|pw} Klinik\oindex{Meisen s Klinik@\textbf{Meisen’s Klinik}|pw}{\\}Freitag\pend
           \pstart{}Lieber Schnitzler\pend\pstart
           Wegen eines Unwohlseins bin ich seit ein Paar Wochen auf einer Klinik\oindex{Meisen s Klinik@\textbf{Meisen’s Klinik}|pwv}. Es ist mir ein wahrer Trauer, Sie
                    nicht in diesen Tagen bei mir empfangen zu können; \uline{muss}
               Sie aber sehen.\pend
           \pstart
           Bitte suchen Sie mich morgen Sonnabend etwa um 2 und
                    bleiben Sie ruhig bis gegen 5. Ihre \label{K_L02398_1v}\edtext{Vorlesung}{\lemma{\textnormal{\emph{Vorlesung}}}\Cendnote{\textnormal{am
                            12. 5. 1923}}}\label{K_L02398_1h} findet ja erst
                        Abends ſtatt.\pend
           \pstart
           Mit tausend Grüssen{\\[\baselineskip]}Ihr Freund{\\[\baselineskip]}\spacefill\mbox{Georg Brandes}\pend
           \leftskip=0em{}\endnumbering\briefempfaengerindex{Schnitzler, Arthur@\textsc{Schnitzler, Arthur}!zzzBrandes, Georg@\emph{von Georg Brandes}!1923-05-111@{{[}11. 5. 1923{]}}|)be}\mylabel{h}\end{ledgroupsized}  \newcommand{\dateiname}{L02398}\newcommand{\titel}{Georg Brandes an Arthur Schnitzler, [11. 5. 1923]}\newcommand{\editorInnen}{Martin Anton Müller und Gerd-Hermann Susen}\input{../tex-inputs/latex-pdf-abspann}
      