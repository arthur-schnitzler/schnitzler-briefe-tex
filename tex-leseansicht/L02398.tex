%% latex-leseansicht-vorspann.tex
%% Vorspann für die Leseansicht.
%% Lädt die gemeinsame Datei latex-vorspann.tex mit nicht gesetztem Schalter.

\newif\ifkorrekturansicht
\korrekturansichtfalse

\input{../tex-inputs/latex-vorspann}


\section[Georg Brandes an Arthur Schnitzler, {{[}}11. 5. 1923{{]}}]{L02398 Georg Brandes an Arthur Schnitzler, {[}11. 5. 1923{]}}
\nopagebreak\mylabel{L02398v}
\rehead{ }\normalsize\beginnumbering\briefempfaengerindex{Schnitzler, Arthur@\textsc{Schnitzler, Arthur}!zzzBrandes, Georg@\emph{von Georg Brandes}!1923-05-111@{{[}11. 5. 1923{]}}|(be}
\toendnotes[C]{\smallbreak\pagebreak[2]}
\correspDesc{Versand  durch Georg Brandes am [11. 5. 1923] in Kopenhagen
\newline{}Erhalt  durch Arthur Schnitzler im Zeitraum [11. 5. 1923
                  – 12. 5. 1923?] in Kopenhagen}\toendnotes[C]{\smallbreak}
\Standort{CUL, Schnitzler, B 17.}
\physDesc{Brief, 1 Blatt, 1 Seite, 385 Zeichen
\newline{}Handschrift: Bleistift, lateinische Kurrent
\newline{}Schnitzler: datiert: »Mai 923« 
\newline{}Ordnung: 1) mit Bleistift von unbekannter Hand in der rechten oberen Ecke
                                 notiert: »\uline{erg.}«  2) mit Bleistift von unbekannter Hand nummeriert:
                                    »53«}
\buchAbdrucke{\weitereDrucke{Georg Brandes, Arthur Schnitzler: \emph{Ein Briefwechsel}. Herausgegeben von Kurt Bergel. Bern: \emph{Francke} 1956, S. 138.} }\toendnotes[C]{\smallbreak}
\pstart
           \raggedleft{}{\pb}Allégade 31\oindex{Allégade@\textbf{Allégade}, \emph{Straße}|pw}{\\}Dr. Meisens\pwindex{Meisen, Valdemar 20.\,7.\,1878 Randers – 18.\,4.\,1934 Kopenhagen@\textsc{Meisen, Valdemar} (20.\,7.\,1878 Randers – 18.\,4.\,1934 Kopenhagen), \emph{Mediziner, Chirurg, Sanatoriumsleiter}|pw} Klinik\oindex{Meisen’s Klinik@\textbf{Meisen’s Klinik}, \emph{Krankenhaus}|pw}{\\}Freitag\pend
           
\pstart{}Lieber Schnitzler\pend\vspace{0.5em}
\pstart
           Wegen eines Unwohlseins bin ich seit ein Paar Wochen auf einer Klinik\oindex{Meisen’s Klinik@\textbf{Meisen’s Klinik}, \emph{Krankenhaus}|pwv}. Es ist mir ein wahrer Trauer, Sie
               nicht in diesen Tagen bei mir empfangen zu können; \uline{muss} Sie aber sehen.\pend
           
\pstart
           Bitte suchen Sie mich morgen Sonnabend etwa um 2 und
               bleiben Sie ruhig bis gegen 5. Ihre \label{K_L02398-1v}\edtext{Vorlesung}{\lemma{\textnormal{\emph{Vorlesung}}}\Cendnote{\textnormal{Vgl. A. S.: \emph{Tagebuch}, 12. 5. 1923.
               }}}\label{K_L02398-1} findet ja erst Abends{ }ſtatt.\pend
           
\pstart
           Mit tausend Grüssen{\\[\baselineskip]}Ihr Freund{\\[\baselineskip]}\spacefill\mbox{Georg Brandes}\pend
           \leftskip=0em{}\selectlanguage{ngerman}\endnumbering\briefempfaengerindex{Schnitzler, Arthur@\textsc{Schnitzler, Arthur}!zzzBrandes, Georg@\emph{von Georg Brandes}!1923-05-111@{{[}11. 5. 1923{]}}|)be}\mylabel{L02398h}  \newcommand{\dateiname}{L02398}\newcommand{\titel}{Georg Brandes an Arthur Schnitzler, [11. 5. 1923]}\newcommand{\editorInnen}{Martin Anton Müller und Gerd-Hermann Susen}%% latex-leseansicht-abspann.tex
%% Abspann für die Leseansicht.
%% Der Schalter \ifkorrekturansicht ist bereits durch den Vorspann gesetzt.

%% latex-abspann.tex
%% Gemeinsamer Abspann für Korrekturansicht und Leseansicht.
%% Setzt den Schalter \ifkorrekturansicht voraus (gesetzt in den
%% einbindenden Dateien latex-korrekturansicht-abspann.tex bzw.
%% latex-leseansicht-abspann.tex).
%% ---------------------------------------------------------------

\normalsize

% Das esempio-Environment wird nur in der Leseansicht benötigt
\ifkorrekturansicht\else
\newenvironment{esempio}[3]%
{
    \vspace{1.5ex}
    \rlap{\underline{#1}}
    \par
    \setlength{\parindent}{0cm}
    \nopagebreak
    \leftskip=#2cm
    \rightskip=#3cm
}
{
    \par
}
\fi

\doendnotes{C}
\bigskip
\vfill

\clearpage

\footnotesize

\ifkorrekturansicht
  \lohead{\textsc{register}}
\fi

% theindex-Environment neu definieren ohne reledmac
\makeatletter
\renewenvironment{theindex}{%
  \ifkorrekturansicht
    \section*{\indexname}%
  \else
    \subsubsection*{Index der erwähnten Entitäten}%
  \fi
  \setlength{\parindent}{0pt}%
  \setlength{\parskip}{0pt plus 0.3pt}%
  \let\item\@idxitem
}{%
  \ifkorrekturansicht\clearpage\fi
}
\makeatother

\IfFileExists{\jobname-pw.ind}{\input{\jobname-pw.ind}}{}

% Quellenangabe nur in der Leseansicht
\ifkorrekturansicht\else
% Fallback-Definitionen, falls die .tex-Datei \titel etc. nicht gesetzt hat
\providecommand{\titel}{}
\providecommand{\editorInnen}{}
\providecommand{\dateiname}{\jobname}

\vspace{3cm}

\vfill

\footnotesize
\textsc{Quelle}: \titel. Herausgegeben von {\editorInnen}. In: \emph{Arthur Schnitzler: Briefwechsel mit Autorinnen und Autoren}.
 Digitale Edition, https://schnitzler-briefe.acdh.oeaw.ac.at/{\dateiname}.html (Stand \today)
\fi

\end{document}


