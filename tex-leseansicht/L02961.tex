%% latex-leseansicht-vorspann.tex
%% Vorspann für die Leseansicht.
%% Lädt die gemeinsame Datei latex-vorspann.tex mit nicht gesetztem Schalter.

\newif\ifkorrekturansicht
\korrekturansichtfalse

\input{../tex-inputs/latex-vorspann}


         
         \renewcommand{\erwaehntePersonen}{Personen: Felix Salten, Josefine Lydia von Weisswasser}
         \renewcommand{\erwaehnteOrte}{Orte: Dölsach, Pressbaum, Wien}
         \renewcommand{\erwaehnteWerke}{}
               \section[ Arthur Schnitzler an Felix Salten, 17. 8. 1893]{ Arthur Schnitzler an Felix Salten, 17. 8. 1893}\nopagebreak\mylabel{v}\rehead{ }\begin{ledgroupsized}[t]{13cm}\normalsize\beginnumbering\briefempfaengerindex{Salten, Felix@\textsc{Salten, Felix}!zzzSchnitzler, Arthur@\emph{von Arthur Schnitzler}!1893-08-171@{17. 8. 1893}|(be} \toendnotes[C]{\smallbreak\pagebreak[2]} \Standort{Wienbibliothek im Rathaus, ZPH 1681, 2.1.516.}
\physDesc{Brief, 1 Blatt, 4 Seiten, 688 Zeichen (Briefpapier mit Trauerrand)
\newline{}Handschrift: Bleistift, deutsche Kurrent
\newline{}Ordnung: mit Bleistift von unbekannter Hand Nummerierung der Doppelseiten des Konvoluts:
                                    »78«–»79« }\buchAbdrucke{\weitereDrucke{Arthur Schnitzler: \emph{Briefe 1875–1912}. Hg. Therese Nickl und Heinrich Schnitzler. Frankfurt am Main: \emph{S. Fischer} 1981, S. 213.} }\toendnotes[C]{\smallbreak}\pstart
           \raggedleft{}{\pb}\uline{17. 8. 93}\pend
           \pstart{}Lieber Freund,\pend\pstart
           ich ka{\geminationn}{ }\label{K_L02961-1v}\edtext{Montag oder Dinſtg bei
               Ihnen ſein}{\lemma{\textnormal{\emph{Montag … ſein}}}\Cendnote{\textnormal{Siehe Felix Salten an Arthur Schnitzler, 14. 8. 1893.
               }}}\label{K_L02961-1h}. Aber ſchreiben Sie mir gefälligſt, \uline{wohin} ich
               fahren ſoll, wo Sie mich erwarten wollen, {\pb}und, ſoweit dies möglich, wie unſre Partie ſich eigentlich geſtalten wird. –\pend
           \pstart
           Sie müſſen mir \uline{gleich} ſchreiben. –\pend
           \pstart
           Plötzlich iſt eine unterträgliche Hitze über Wien\oindex{Wien@\textbf{Wien}|pw}
               hereingebrochen. {\pb}Heute{ }früh kam ich \textsc{per}{ }\textsc{Bic.} aus Preßbaum\oindex{Pressbaum@\textbf{Pressbaum}|pw}
               herein, wo ich eine \label{K_L02961-2v}\edtext{Nacht der »Liebe«\pwindex{Weisswasser, Josefine Lydia von *~01.03.1864@\textsc{Weisswasser, Josefine Lydia von} (*~01.03.1864)|pwv}}{\lemma{\textnormal{\emph{Nacht der »Liebe«}}}\Cendnote{\textnormal{Siehe A. S.: \emph{Tagebuch}, 16. 8. 1893.
               }}}\label{K_L02961-2h} verbracht hatte. Dumpfiges Gaſthofzi{\geminationm}er mit ſchlechten Betten – der Abend
               vorher war ganz ſchön; – denn was lügt einem die Si{\geminationn}lichkeit nach dem {\pb}Nachtmahl \introOben{}nicht\introOben{} alles vor! – Wodurch ſie\pwindex{Weisswasser, Josefine Lydia von *~01.03.1864@\textsc{Weisswasser, Josefine Lydia von} (*~01.03.1864)|pwv} ſich von den
               Weibern unterſcheidet, die auch vor dem Nachtmahl lügen. –\pend
           \pstart
           – Leben Sie wohl, ſeien Sie herzlich gegrüßt, {\\[\baselineskip]}\spacefill\mbox{Arthur}\pend
           \leftskip=0em{}
         
         \endnumbering\mylabel{h}\end{ledgroupsized}  \newcommand{\dateiname}{L02961}\newcommand{\titel}{Arthur Schnitzler an Felix Salten, 17. 8. 1893}\newcommand{\editorInnen}{Martin Anton Müller und Laura Untner}%% latex-leseansicht-abspann.tex
%% Abspann für die Leseansicht.
%% Der Schalter \ifkorrekturansicht ist bereits durch den Vorspann gesetzt.

%% latex-abspann.tex
%% Gemeinsamer Abspann für Korrekturansicht und Leseansicht.
%% Setzt den Schalter \ifkorrekturansicht voraus (gesetzt in den
%% einbindenden Dateien latex-korrekturansicht-abspann.tex bzw.
%% latex-leseansicht-abspann.tex).
%% ---------------------------------------------------------------

\normalsize

% Das esempio-Environment wird nur in der Leseansicht benötigt
\ifkorrekturansicht\else
\newenvironment{esempio}[3]%
{
    \vspace{1.5ex}
    \rlap{\underline{#1}}
    \par
    \setlength{\parindent}{0cm}
    \nopagebreak
    \leftskip=#2cm
    \rightskip=#3cm
}
{
    \par
}
\fi

\doendnotes{C}
\bigskip
\vfill

\clearpage

\footnotesize

\ifkorrekturansicht
  \lohead{\textsc{register}}
\fi

% theindex-Environment neu definieren ohne reledmac
\makeatletter
\renewenvironment{theindex}{%
  \ifkorrekturansicht
    \section*{\indexname}%
  \else
    \subsubsection*{Index der erwähnten Entitäten}%
  \fi
  \setlength{\parindent}{0pt}%
  \setlength{\parskip}{0pt plus 0.3pt}%
  \let\item\@idxitem
}{%
  \ifkorrekturansicht\clearpage\fi
}
\makeatother

\IfFileExists{\jobname-pw.ind}{\input{\jobname-pw.ind}}{}

% Quellenangabe nur in der Leseansicht
\ifkorrekturansicht\else
% Fallback-Definitionen, falls die .tex-Datei \titel etc. nicht gesetzt hat
\providecommand{\titel}{}
\providecommand{\editorInnen}{}
\providecommand{\dateiname}{\jobname}

\vspace{3cm}

\vfill

\footnotesize
\textsc{Quelle}: \titel. Herausgegeben von {\editorInnen}. In: \emph{Arthur Schnitzler: Briefwechsel mit Autorinnen und Autoren}.
 Digitale Edition, https://schnitzler-briefe.acdh.oeaw.ac.at/{\dateiname}.html (Stand \today)
\fi

\end{document}


      