%% latex-leseansicht-vorspann.tex
%% Vorspann für die Leseansicht.
%% Lädt die gemeinsame Datei latex-vorspann.tex mit nicht gesetztem Schalter.

\newif\ifkorrekturansicht
\korrekturansichtfalse

\input{../tex-inputs/latex-vorspann}


\section[Hugo und Gerty Hofmannsthal an Arthur Schnitzler, Mitte Juni 1928]{L02502 Hugo und Gerty Hofmannsthal an Arthur Schnitzler, Mitte Juni 1928}
\nopagebreak\mylabel{L02502v}
\rehead{ }\normalsize\beginnumbering\briefempfaengerindex{Schnitzler, Arthur@\textsc{Schnitzler, Arthur}!zzzHofmannsthal, Gertrude von@\emph{von Gertrude von Hofmannsthal}!1928-06-151@{Mitte Juni 1928}|(be}\briefempfaengerindex{Schnitzler, Arthur@\textsc{Schnitzler, Arthur}!zzzHofmannsthal, Hugo von@\emph{von Hugo von Hofmannsthal}!1928-06-151@{Mitte Juni 1928}|(be}
\toendnotes[C]{\smallbreak\pagebreak[2]}
\correspDesc{Versand  durch Hugo Hofmannsthal, Gerty Hofmannsthal am Mitte Juni 1928 in Wien
\newline{}Erhalt  durch Arthur Schnitzler im Zeitraum [15. 6. 1928
                  – 19. 6. 1928?] in Wien}\toendnotes[C]{\smallbreak}
\Standort{Wien, Österreichische Gesellschaft für Literatur, Schnitzler, Varia.}
\physDesc{Briefkarte, Fotokopie, 195 Zeichen
\newline{}Druck
\newline{}Schnitzler: vermutlich mit zwei roten Unterstreichungen am Original }\toendnotes[C]{\smallbreak}
\pstart
           \noindent{}{\pb}Herr und Frau von Hofmannsthal geben Nachricht von der
               Vermälung ihrer Tochter Christiane\pwindex{Zimmer, Christiane 14.\,5.\,1902 Rodaun – 5.\,1.\,1987 New York City@\textsc{Zimmer, Christiane} (14.\,5.\,1902 Rodaun – 5.\,1.\,1987 New York City)|pw}
               mit D\textsuperscript{r}{ }Heinrich
                  Zimmer\pwindex{Zimmer, Heinrich 6.\,12.\,1890 Greifswald – 20.\,3.\,1943 New York City@\textsc{Zimmer, Heinrich} (6.\,12.\,1890 Greifswald – 20.\,3.\,1943 New York City), \emph{Indologe}|pw}, Professor der indischen Philologie an der Universität Heidelberg\orgindex{Ruprecht-Karls-Universität Heidelberg@Ruprecht-Karls-Universität Heidelberg|pw}.\pend
           
\pstart
           Rodaun\oindex{Wien@\textbf{Wien}!XXIII., Liesing@\textbf{XXIII., Liesing}!Rodaun@\textbf{Rodaun}, \emph{Region}|pw}, im \label{K_L02502-1v}\edtext{Juni 1928}{\lemma{\textnormal{\emph{Juni 1928}}}\Cendnote{\textnormal{Die Hochzeit fand am
                        14. 6. 1928 statt.}}}\label{K_L02502-1}.\pend
           \selectlanguage{ngerman}\endnumbering\briefempfaengerindex{Schnitzler, Arthur@\textsc{Schnitzler, Arthur}!zzzHofmannsthal, Gertrude von@\emph{von Gertrude von Hofmannsthal}!1928-06-151@{Mitte Juni 1928}|)be}\briefempfaengerindex{Schnitzler, Arthur@\textsc{Schnitzler, Arthur}!zzzHofmannsthal, Hugo von@\emph{von Hugo von Hofmannsthal}!1928-06-151@{Mitte Juni 1928}|)be}\mylabel{L02502h}  \newcommand{\dateiname}{L02502}\newcommand{\titel}{Hugo und Gerty Hofmannsthal an Arthur Schnitzler, Mitte Juni 1928}\newcommand{\editorInnen}{Martin Anton Müller und Gerd-Hermann Susen}%% latex-leseansicht-abspann.tex
%% Abspann für die Leseansicht.
%% Der Schalter \ifkorrekturansicht ist bereits durch den Vorspann gesetzt.

%% latex-abspann.tex
%% Gemeinsamer Abspann für Korrekturansicht und Leseansicht.
%% Setzt den Schalter \ifkorrekturansicht voraus (gesetzt in den
%% einbindenden Dateien latex-korrekturansicht-abspann.tex bzw.
%% latex-leseansicht-abspann.tex).
%% ---------------------------------------------------------------

\normalsize

% Das esempio-Environment wird nur in der Leseansicht benötigt
\ifkorrekturansicht\else
\newenvironment{esempio}[3]%
{
    \vspace{1.5ex}
    \rlap{\underline{#1}}
    \par
    \setlength{\parindent}{0cm}
    \nopagebreak
    \leftskip=#2cm
    \rightskip=#3cm
}
{
    \par
}
\fi

\doendnotes{C}
\bigskip
\vfill

\clearpage

\footnotesize

\ifkorrekturansicht
  \lohead{\textsc{register}}
\fi

% theindex-Environment neu definieren ohne reledmac
\makeatletter
\renewenvironment{theindex}{%
  \ifkorrekturansicht
    \section*{\indexname}%
  \else
    \subsubsection*{Index der erwähnten Entitäten}%
  \fi
  \setlength{\parindent}{0pt}%
  \setlength{\parskip}{0pt plus 0.3pt}%
  \let\item\@idxitem
}{%
  \ifkorrekturansicht\clearpage\fi
}
\makeatother

\IfFileExists{\jobname-pw.ind}{\input{\jobname-pw.ind}}{}

% Quellenangabe nur in der Leseansicht
\ifkorrekturansicht\else
% Fallback-Definitionen, falls die .tex-Datei \titel etc. nicht gesetzt hat
\providecommand{\titel}{}
\providecommand{\editorInnen}{}
\providecommand{\dateiname}{\jobname}

\vspace{3cm}

\vfill

\footnotesize
\textsc{Quelle}: \titel. Herausgegeben von {\editorInnen}. In: \emph{Arthur Schnitzler: Briefwechsel mit Autorinnen und Autoren}.
 Digitale Edition, https://schnitzler-briefe.acdh.oeaw.ac.at/{\dateiname}.html (Stand \today)
\fi

\end{document}


