%% latex-leseansicht-vorspann.tex
%% Vorspann für die Leseansicht.
%% Lädt die gemeinsame Datei latex-vorspann.tex mit nicht gesetztem Schalter.

\newif\ifkorrekturansicht
\korrekturansichtfalse

\input{../tex-inputs/latex-vorspann}


\section[Hugo von Hofmannsthal an Arthur Schnitzler, {{[}}16. 8. 1891{{]}}]{L00034 Hugo von Hofmannsthal an Arthur Schnitzler, {[}16. 8. 1891{]}}
\nopagebreak\mylabel{L00034v}
\rehead{ }\normalsize\beginnumbering\briefempfaengerindex{Schnitzler, Arthur@\textsc{Schnitzler, Arthur}!zzzHofmannsthal, Hugo von@\emph{von Hugo von Hofmannsthal}!1891-08-161@{{[}16. 8. 1891{]}}|(be}
\toendnotes[C]{\smallbreak\pagebreak[2]}
\correspDesc{Versand  durch Hugo von Hofmannsthal am [16. 8. 1891] in Strobl
\newline{}Weiterleitung  im Zeitraum [16. 8. 1891
                  – 17. 8. 1891?] in Bad Ischl
\newline{}Erhalt  durch Arthur Schnitzler im Zeitraum [17. 8. 1891
                  – 21. 8. 1891?] in Wien}\toendnotes[C]{\smallbreak}
\Standort{CUL, Schnitzler, B 43.}
\physDesc{Visitenkarte, 207 Zeichen
\newline{}Handschrift: Bleistift, deutsche Kurrent
\newline{}Schnitzler: mit Bleistift auf der Namensseite datiert »16/8 91« 
\newline{}Ordnung: mit Bleistift von unbekannter Hand auf der Rückseite nummeriert:
                                    »6« }
\buchAbdrucke{\weitereDrucke{Hugo von Hofmannsthal, Arthur Schnitzler: \emph{Briefwechsel}. Herausgegeben von Therese Nickl und Heinrich Schnitzler. Frankfurt am Main: \emph{S. Fischer} 1964, S. 12.} }
\pstart{}{\pb}Liebſter Freund!\pend\vspace{0.5em}
\pstart
           Heute nacht vielleicht infolge{ }ſchlechter Champignons{ }ſehr unwohl kann heute kaum{ }ſtehen. Seien Sie und Richard\pwindex{Beer-Hofmann, Richard 11.\,7.\,1866 Wien – 26.\,9.\,1945 New York City@\textsc{Beer-Hofmann, Richard} (11.\,7.\,1866 Wien – 26.\,9.\,1945 New York City), \emph{Schriftsteller}|pw} nicht bös und
               behandeln Sie meine Unarten als Object der Analyſe.\pend
           
\pstart
           {\pb}Herzlichst{\\[\baselineskip]}\spacefill\mbox{Loris.}\pend
           \leftskip=0em{}
\pstart
           \noindent{}\centering{}\textcolor{gray}{\textbf{D\textsuperscript{r.}{ }Hugo von Hofmannsthal\pwindex{Hofmannsthal, Hugo August von 21.\,12.\,1841 Wien – 8.\,12.\,1915 ebd.@\textsc{Hofmannsthal, Hugo August von} (21.\,12.\,1841 Wien – 8.\,12.\,1915 ebd.), \emph{Bankdirektor}|pw}}}\pend
           \selectlanguage{ngerman}\endnumbering\briefempfaengerindex{Schnitzler, Arthur@\textsc{Schnitzler, Arthur}!zzzHofmannsthal, Hugo von@\emph{von Hugo von Hofmannsthal}!1891-08-161@{{[}16. 8. 1891{]}}|)be}\mylabel{L00034h}  \newcommand{\dateiname}{L00034}\newcommand{\titel}{Hugo von Hofmannsthal an Arthur Schnitzler, [16. 8. 1891]}\newcommand{\editorInnen}{Martin Anton Müller und Gerd-Hermann Susen}%% latex-leseansicht-abspann.tex
%% Abspann für die Leseansicht.
%% Der Schalter \ifkorrekturansicht ist bereits durch den Vorspann gesetzt.

%% latex-abspann.tex
%% Gemeinsamer Abspann für Korrekturansicht und Leseansicht.
%% Setzt den Schalter \ifkorrekturansicht voraus (gesetzt in den
%% einbindenden Dateien latex-korrekturansicht-abspann.tex bzw.
%% latex-leseansicht-abspann.tex).
%% ---------------------------------------------------------------

\normalsize

% Das esempio-Environment wird nur in der Leseansicht benötigt
\ifkorrekturansicht\else
\newenvironment{esempio}[3]%
{
    \vspace{1.5ex}
    \rlap{\underline{#1}}
    \par
    \setlength{\parindent}{0cm}
    \nopagebreak
    \leftskip=#2cm
    \rightskip=#3cm
}
{
    \par
}
\fi

\doendnotes{C}
\bigskip
\vfill

\clearpage

\footnotesize

\ifkorrekturansicht
  \lohead{\textsc{register}}
\fi

% theindex-Environment neu definieren ohne reledmac
\makeatletter
\renewenvironment{theindex}{%
  \ifkorrekturansicht
    \section*{\indexname}%
  \else
    \subsubsection*{Index der erwähnten Entitäten}%
  \fi
  \setlength{\parindent}{0pt}%
  \setlength{\parskip}{0pt plus 0.3pt}%
  \let\item\@idxitem
}{%
  \ifkorrekturansicht\clearpage\fi
}
\makeatother

\IfFileExists{\jobname-pw.ind}{\input{\jobname-pw.ind}}{}

% Quellenangabe nur in der Leseansicht
\ifkorrekturansicht\else
% Fallback-Definitionen, falls die .tex-Datei \titel etc. nicht gesetzt hat
\providecommand{\titel}{}
\providecommand{\editorInnen}{}
\providecommand{\dateiname}{\jobname}

\vspace{3cm}

\vfill

\footnotesize
\textsc{Quelle}: \titel. Herausgegeben von {\editorInnen}. In: \emph{Arthur Schnitzler: Briefwechsel mit Autorinnen und Autoren}.
 Digitale Edition, https://schnitzler-briefe.acdh.oeaw.ac.at/{\dateiname}.html (Stand \today)
\fi

\end{document}


