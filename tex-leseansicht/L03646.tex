%% latex-korrekturansicht-vorspann.tex
%% Vorspann für die Korrekturansicht.
%% Lädt die gemeinsame Datei latex-vorspann.tex mit gesetztem Schalter.

\newif\ifkorrekturansicht
\korrekturansichttrue

\input{../tex-inputs/latex-vorspann}


\section[Stefan Zweig an Arthur Schnitzler, 20. 2. 1914]{L03646 Stefan Zweig an Arthur Schnitzler, 20. 2. 1914}
\nopagebreak\mylabel{L03646v}
\rehead{ }\normalsize\beginnumbering\briefempfaengerindex{Schnitzler, Arthur@\textsc{Schnitzler, Arthur}!zzzZweig, Stefan@\emph{von Stefan Zweig}!1914-02-201@{20. 2. 1914}|(be}
\toendnotes[C]{\smallbreak\pagebreak[2]}\Standort{CUL, Schnitzler, B 118.}
\physDesc{Brief, 1 Blatt, 4 Seiten, 2159 Zeichen
\newline{}Handschrift: lila Tinte, lateinische Kurrent
\newline{}Schnitzler: 1) mit Bleistift »\textsc{Zweig}«  2) mit rotem Buntstift eine Markierung}
\buchAbdrucke{\weitereDrucke{Stefan Zweig: \emph{Briefwechsel mit Hermann Bahr, Sigmund Freud, Rainer Maria
                        Rilke und Arthur Schnitzler}. Frankfurt am Main: \emph{S. Fischer} 1987, S. 379–380.} }\toendnotes[C]{\smallbreak}
\pstart
           {\pb}\textcolor{gray}{\textbf{SZ}}\hfill \textcolor{gray}{\textbf{VIII. KOCHGASSE 8\oindex{Kochgasse 8@\textbf{Kochgasse 8}, \emph{Wohngebäude (K.WHS)}|pw}}}\pend
           
\pstart
           \raggedleft{}\textcolor{gray}{\textbf{WIEN\oindex{Wien@\textbf{Wien}, \emph{A.ADM2}|pw},}}{ }20 II. 14\pend
           \vspace{0.5em}
\pstart
           Sehr verehrter Herr Doktor, empfangen Sie meinen herzlichen
               Glückwunsch zur erfolgreichen \label{K_L03646-1v}\edtext{Auferstehung}{\lemma{\textnormal{\emph{Auferstehung}}}\Cendnote{\textnormal{Der einsame Weg wurde
                  am 13. 2. 1904 am Deutschen Theater\oindex{Deutsches Theater Berlin@\textbf{Deutsches Theater Berlin}, \emph{Theater (K.THE)}|pwk}
                  in Berlin\oindex{Berlin@\textbf{Berlin}, \emph{P.PPLC}|pwk} mit mäßigem Erfolg uraufgeführt.
                  Erst zehn Jahre später, am 19. 2.1914, fand die Wiener\oindex{Wien@\textbf{Wien}, \emph{A.ADM2}|pwk} Erstaufführung am Burgtheater\oindex{Burgtheater@\textbf{Burgtheater}, \emph{S.THTR}|pwk} statt.}}}\label{K_L03646-1} des »Einsamen
                  Weges\pwindex{einsame Weg. Schauspiel in fuenf Akten@\emph{Der einsame Weg. Schauspiel in fünf Akten}|pw}«. Durch alle Unzulänglichkeit mancher Darsteller habe ich
                  gestern abends wieder diese Menschen gespürt, die seit Jahren wie
               wirklich begegnete in meinem Leben sind und bin froh wieder Ihrer Meisterschaft
               bewusst geworden. Ich habe ja keine Berufung darüber zu sprechen, aber immer bei
               Ihren Werken, wenn ich \strikeout{ihnen} auf der Bühne oder im
               Buch {\pb}ihnen neuerlich nahe trete, ist es
               mir Bedürfnis ein Wort an Sie zu richten, Ihnen irgendwie zu danken für Alles was Sie
               uns gegeben haben. Ich meine nie das Einzelne damit und gerade gestern,
               im »Einsamen Weg\pwindex{einsame Weg. Schauspiel in fuenf Akten@\emph{Der einsame Weg. Schauspiel in fünf Akten}|pw}« der mir vor Jahren, als er das
               letzte war, auch das liebste Ihrer Stücke schien, ist mir bewusst geworden, wie stark
               in Ihrer Kunst seitdem der Zusammenschluss aller innerlichen Kräfte geworden ist, wie
               Manches, was hier noch Andeutung ist, im »Weiten
                  Land\pwindex{weite Land. Tragikomoedie in fuenf Akten@\emph{Das weite Land. Tragikomödie in fünf Akten}|pw}« und der »Frau Beate\pwindex{Frau Beate und ihr Sohn. Novelle@\emph{Frau Beate und ihr Sohn. Novelle}|pw}« schöpferisch
                  \strikeout{au} sich ausgebaut hat. Es ist für mich ein grosser
               Genuss, spüren zu dürfen wie organisch sich über das einzelne Werk hinaus Ihre Motive
               entwickeln, wie Ihr ganzes Schaffen gleichsam {\pb}symphonisch in Anklang und Widerklang
                  \strikeout{die} gewisse persönlichste menschliche Themen
               verarbeitet und wie eigentlich alle diese einzelnen Dramen von einer gewissen Ferne
               der Jahre,\substVorne{}\textsuperscript{auf}\substDazwischen{}von\substHinten{} der notwendigen Erhebung des Alters gesehen, eine complexe Gesammtheit
               bilden. Es gibt darum für mich eigentlich nicht \substVorne{}\textsuperscript{f }\substDazwischen{}w\substHinten{}ie für die meisten (die Ihre Werke in erster Linie als Theaterstücke werten)
               ein Mehr oder Minder des Gefallens, ich wehre mich gegen den Vergleich und freue
               mich, Ihnen immer und immer wieder für das Ganze danken zu können. Es sei
                  heute und nicht zum letzten Mal aus aufrichtiger Empfindung
               getan.\pend
           
\pstart
           Lebhaft leid ist es mir, dass Sie mir so lange nicht erlaubten, Sie sehen zu dürfen.
               Ich reise nun {\pb}wieder von Wien\oindex{Wien@\textbf{Wien}, \emph{A.ADM2}|pw} fort, mit kurzer Unterbrechung eigentlich für
               lange, aber ich nehme alle Verehrung und Liebe für Sie und Ihr Werk getreulich mit.
               Nur meines Dankes einen Teil wollte ich Ihnen heute zurücklassen, da mir
                  gestern Ihr Stück\pwindex{einsame Weg. Schauspiel in fuenf Akten@\emph{Der einsame Weg. Schauspiel in fünf Akten}|pwv} diese Verpflichtung neuerlich voll bewusst gemacht.\pend
           \pstart In inniger Verehrung Ihr \spacefill\mbox{Stefan Zweig}\pend{}\selectlanguage{ngerman}\endnumbering\briefempfaengerindex{Schnitzler, Arthur@\textsc{Schnitzler, Arthur}!zzzZweig, Stefan@\emph{von Stefan Zweig}!1914-02-201@{20. 2. 1914}|)be}\mylabel{L03646h}
\begin{anhang}
\end{anhang}\normalsize

\doendnotes{C}
\bigskip
\vfill

\clearpage

\footnotesize

\lohead{\textsc{register}}

% Definiere theindex-Environment komplett neu ohne reledmac
\makeatletter
\renewenvironment{theindex}{%
  \section*{\indexname}%
  \setlength{\parindent}{0pt}%
  \setlength{\parskip}{0pt plus 0.3pt}%
  \let\item\@idxitem
}{%
  \clearpage
}
\makeatother

\IfFileExists{\jobname-pw.ind}{\input{\jobname-pw.ind}}{}

\end{document}

      