%% latex-korrekturansicht-vorspann.tex
%% Vorspann für die Korrekturansicht.
%% Lädt die gemeinsame Datei latex-vorspann.tex mit gesetztem Schalter.

\newif\ifkorrekturansicht
\korrekturansichttrue

\input{../tex-inputs/latex-vorspann}


\section[Arthur Schnitzler an Richard Beer-Hofmann, 13. 9. 1893]{L00262 Arthur Schnitzler an Richard Beer-Hofmann, 13. 9. 1893}
\nopagebreak\mylabel{L00262v}
\rehead{ }\normalsize\beginnumbering\briefempfaengerindex{Beer-Hofmann, Richard@\textsc{Beer-Hofmann, Richard}!zzzSchnitzler, Arthur@\emph{von Arthur Schnitzler}!1893-09-131@{13. 9. 1893}|(be}
\toendnotes[C]{\smallbreak\pagebreak[2]}\Standort{YCGL, MSS 31.}
\physDesc{Brief, 1 Blatt, 3 Seiten, Umschlag, 590 Zeichen (Briefpapier und Umschlag mit Trauerrand )
\newline{}Handschrift: Bleistift, deutsche Kurrent
\newline{}Versand: 1) Stempel: »\nobreak{}Wien 1/1, 1\textcolor{gray}{3}. 9. 9\textcolor{gray}{3}, 11–12 N\nobreak{}«.   2) Stempel: »\nobreak{}\oindex{Znaim@\textbf{Znaim}, \emph{P.PPLA2}|pwk}\textcolor{gray}{Zn}ojmo, 14 9 93, 10{[}–12{]}\textcolor{gray}{N}\nobreak{}«. }
\buchAbdrucke{\weitereDrucke{Arthur Schnitzler, Richard Beer-Hofmann: \emph{Briefwechsel 1891–1931}. Wien, Zürich: \emph{Europaverlag} 1992, S. 53.} }\pstart{}{\pb}Herrn \textsc{Dr. Richard
                     Beer-Hofmann}\pend{}\pstart{}k. u. k. Lieutenant in der Ref. beim k. k. Inf. Regimente Nr. 99\pend{}\pstart{}in Znaim\oindex{Znaim@\textbf{Znaim}, \emph{P.PPLA2}|pw}. \pend{}{\bigskip}\vspace{1em}
\pstart{}{\pb}Lieber Richard,\pend\vspace{0.5em}
\pstart
           Ihre Karte fand ich Montag, als ich von Reichenau\oindex{Reichenau an der Rax@\textbf{Reichenau an der Rax}, \emph{A.ADM3}|pw} zurück kam; habe ſehr bedauert, dß ich Sie verſäumen mußte. –\pend
           
\pstart
           Samſtag fahre ich auf 2–3 Tage nach Salzburg\oindex{Salzburg@\textbf{Salzburg}, \emph{A.ADM2}|pw}, wo
               ſich Goldma{\geminationn}\pwindex{Goldmann, Paul 31.01.1865 – 25.09.1935@\textsc{Goldmann, Paul} (31.01.1865 – 25.09.1935), \emph{Schriftsteller/Schriftstellerin, Journalist/Journalistin}|pw} be{\pb}findet. –\pend
           
\pstart
           Geſtern hab ich den Vertrag mit dem \textsc{Dtsch. Volksth}.\oindex{Volkstheater@\textbf{Volkstheater}, \emph{Theater (K.THE)}|pw} unterſchrieben, nach welchem das
                  M.\pwindex{Maerchen. Schauspiel in drei Aufzuegen@\emph{Das Märchen. Schauspiel in drei Aufzügen}|pw} vor 1. Dezember 93 in Scene
               gehen müſſte, – »in würdiger Aufführung« wie es im Vertrag heißt. –\pend
           
\pstart
           {\pb}Laſſen Sie was von sich hören, ko{\geminationm}en Sie in guter Sti{\geminationm}ung
               zurück und ſeien Sie herzlich gegrüßt!\pend
           \pstart Ihr\spacefill\mbox{Arthur}\pend{}
\pstart
           Wien\oindex{Wien@\textbf{Wien}, \emph{A.ADM2}|pw}{ }13. 9 93.\pend
           \selectlanguage{ngerman}\endnumbering\briefempfaengerindex{Beer-Hofmann, Richard@\textsc{Beer-Hofmann, Richard}!zzzSchnitzler, Arthur@\emph{von Arthur Schnitzler}!1893-09-131@{13. 9. 1893}|)be}\mylabel{L00262h}  \normalsize

\doendnotes{C}
\bigskip
\vfill

\clearpage

\footnotesize

\lohead{\textsc{register}}

% Definiere theindex-Environment komplett neu ohne reledmac
\makeatletter
\renewenvironment{theindex}{%
  \section*{\indexname}%
  \setlength{\parindent}{0pt}%
  \setlength{\parskip}{0pt plus 0.3pt}%
  \let\item\@idxitem
}{%
  \clearpage
}
\makeatother

\IfFileExists{\jobname-pw.ind}{\input{\jobname-pw.ind}}{}

\end{document}

      