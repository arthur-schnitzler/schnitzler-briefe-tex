%% latex-leseansicht-vorspann.tex
%% Vorspann für die Leseansicht.
%% Lädt die gemeinsame Datei latex-vorspann.tex mit nicht gesetztem Schalter.

\newif\ifkorrekturansicht
\korrekturansichtfalse

\input{../tex-inputs/latex-vorspann}


         
         \renewcommand{\erwaehntePersonen}{Personen: Richard Beer-Hofmann, Paul Goldmann}
         \renewcommand{\erwaehnteOrte}{Orte: Reichenau an der Rax, Salzburg, Volkstheater, Wien, Znaim}
         \renewcommand{\erwaehnteWerke}{Werke: Das Märchen. Schauspiel in drei Aufzügen}
               \section[Arthur Schnitzler an Richard Beer-Hofmann, 13. 9. 1893]{ Arthur Schnitzler an Richard Beer-Hofmann, 13. 9. 1893}\nopagebreak\mylabel{v}\rehead{ }\begin{ledgroupsized}[t]{13cm}\normalsize\beginnumbering \toendnotes[C]{\smallbreak\pagebreak[2]} \Standort{YCGL, MSS 31.}
\physDesc{Brief, 1 Blatt, 3 Seiten, , , , Umschlag (Briefpapier und Umschlag mit Trauerrand
                              )
\newline{}Handschrift: Bleistift, deutsche Kurrent\newline{}Versand: 1) Stempel: »\nobreak{}Wien 1/1, 1\textcolor{gray}{3}. 9. 9\textcolor{gray}{3}, 11–12 N\nobreak{}«.   2) Stempel: »\nobreak{}\oindex{Znaim@\textbf{Znaim}|pwk}\textcolor{gray}{Zn}ojmo, 14 9 93, 10{[}–12{]} \textcolor{gray}{N}\nobreak{}«. }\buchAbdrucke{\weitereDrucke{Arthur Schnitzler, Richard Beer-Hofmann: \emph{Briefwechsel 1891–1931}. Hg. Konstanze Fliedl. Wien, Zürich: \emph{Europaverlag} 1992, S. 53.} }\pstart{}{\pb}Herrn \textsc{Dr. Richard
                     Beer-Hofmann}\pend{}\pstart{}k. u. k. Lieutenant in der Ref. beim k. k. Inf. Regimente Nr. 99\pend{}\pstart{}in Znaim\oindex{Znaim@\textbf{Znaim}|pw}. \pend{}{\bigskip}\pstart{}{\pb}Lieber Richard,\pend\pstart
           Ihre Karte fand ich Montag, als ich von Reichenau\oindex{Reichenau an der Rax@\textbf{Reichenau an der Rax}|pw} zurück kam; habe ſehr bedauert, dß ich Sie verſäumen mußte. –\pend
           \pstart
           Samſtag fahre ich auf 2–3 Tage nach Salzburg\oindex{Salzburg@\textbf{Salzburg}|pw}, wo ſich
                  Goldma{\geminationn}\pwindex{Goldmann, Paul 31.01.1865 – 25.09.1935@\textsc{Goldmann, Paul} (31.01.1865 – 25.09.1935), \emph{Schriftsteller, Journalist}|pw} be{\pb}findet. –\pend
           \pstart
           Geſtern hab ich den Vertrag mit dem \textsc{Dtsch. Volksth}.\oindex{Volkstheater@\textbf{Volkstheater}|pw} unterſchrieben, nach welchem das
                  M.\pwindex{Schnitzler, Arthur 15.05.1862 – 21.10.1931@\textsc{Schnitzler, Arthur} (15.05.1862 – 21.10.1931), \emph{Schriftsteller, Mediziner}!Maerchen. Schauspiel in drei Aufzuegen1893-12-01@\strich\emph{Das Märchen. Schauspiel in drei Aufzügen} {[}1893-12-01{]}|pw} vor 1. Dezember 93 in Scene
               gehen müſſte, – »in würdiger Aufführung« wie es im Vertrag heißt. –\pend
           \pstart
           {\pb}Laſſen Sie was von sich hören, ko{\geminationm}en Sie in guter Sti{\geminationm}ung
               zurück und ſeien Sie herzlich gegrüßt!\pend
           \pstart Ihr\spacefill\mbox{Arthur}\pend{}\pstart
           Wien\oindex{Wien@\textbf{Wien}|pw}{ }13. 9 93.\pend
           
         
         \endnumbering\mylabel{h}\end{ledgroupsized}  \newcommand{\dateiname}{L00262}\newcommand{\titel}{Arthur Schnitzler an Richard Beer-Hofmann, 13. 9. 1893}\newcommand{\editorInnen}{Martin Anton Müller und Gerd-Hermann Susen}%% latex-leseansicht-abspann.tex
%% Abspann für die Leseansicht.
%% Der Schalter \ifkorrekturansicht ist bereits durch den Vorspann gesetzt.

%% latex-abspann.tex
%% Gemeinsamer Abspann für Korrekturansicht und Leseansicht.
%% Setzt den Schalter \ifkorrekturansicht voraus (gesetzt in den
%% einbindenden Dateien latex-korrekturansicht-abspann.tex bzw.
%% latex-leseansicht-abspann.tex).
%% ---------------------------------------------------------------

\normalsize

% Das esempio-Environment wird nur in der Leseansicht benötigt
\ifkorrekturansicht\else
\newenvironment{esempio}[3]%
{
    \vspace{1.5ex}
    \rlap{\underline{#1}}
    \par
    \setlength{\parindent}{0cm}
    \nopagebreak
    \leftskip=#2cm
    \rightskip=#3cm
}
{
    \par
}
\fi

\doendnotes{C}
\bigskip
\vfill

\clearpage

\footnotesize

\ifkorrekturansicht
  \lohead{\textsc{register}}
\fi

% theindex-Environment neu definieren ohne reledmac
\makeatletter
\renewenvironment{theindex}{%
  \ifkorrekturansicht
    \section*{\indexname}%
  \else
    \subsubsection*{Index der erwähnten Entitäten}%
  \fi
  \setlength{\parindent}{0pt}%
  \setlength{\parskip}{0pt plus 0.3pt}%
  \let\item\@idxitem
}{%
  \ifkorrekturansicht\clearpage\fi
}
\makeatother

\IfFileExists{\jobname-pw.ind}{\input{\jobname-pw.ind}}{}

% Quellenangabe nur in der Leseansicht
\ifkorrekturansicht\else
% Fallback-Definitionen, falls die .tex-Datei \titel etc. nicht gesetzt hat
\providecommand{\titel}{}
\providecommand{\editorInnen}{}
\providecommand{\dateiname}{\jobname}

\vspace{3cm}

\vfill

\footnotesize
\textsc{Quelle}: \titel. Herausgegeben von {\editorInnen}. In: \emph{Arthur Schnitzler: Briefwechsel mit Autorinnen und Autoren}.
 Digitale Edition, https://schnitzler-briefe.acdh.oeaw.ac.at/{\dateiname}.html (Stand \today)
\fi

\end{document}


      