%% latex-leseansicht-vorspann.tex
%% Vorspann für die Leseansicht.
%% Lädt die gemeinsame Datei latex-vorspann.tex mit nicht gesetztem Schalter.

\newif\ifkorrekturansicht
\korrekturansichtfalse

\input{../tex-inputs/latex-vorspann}

\begin{center}
            \textcolor{red}{ENTWURF, NICHT FERTIG KORRIGIERT}
                      \end{center}
            
         
         \renewcommand{\erwaehntePersonen}{Personen: Felix Salten, Ottilie Salten}
         \renewcommand{\erwaehnteOrte}{Orte: Wien}
         \renewcommand{\erwaehnteWerke}{Werke: Die Zeit, Zum großen Wurstel. Burleske in einem Akt}
               \section[Arthur Schnitzler an Felix Salten, 8. 2. 1905]{ Arthur Schnitzler an Felix Salten, 8. 2. 1905}\nopagebreak\mylabel{v}\rehead{ }\begin{ledgroupsized}[t]{13cm}\normalsize\beginnumbering \toendnotes[C]{\smallbreak\pagebreak[2]} \Standort{Wienbibliothek im Rathaus, ZPH 1681, 2.1.516.}
\physDesc{Brief, 1 Blatt, 2 Seiten
\newline{}Handschrift: schwarze Tinte, deutsche Kurrent\newline{}Ordnung: mit Bleistift von unbekannter Hand Nummerierung der ungeraden Seiten: »28« }\toendnotes[C]{\smallbreak}\pstart
           \raggedleft{}\textsc{{\pb}Wien\oindex{Wien@\textbf{Wien}|pw}}, 8. 2. 905\pend
           \pstart{}lieber,\pend\pstart
           erſtens frage ich Sie, ob Sie am Sonntag Abend mit Ihrer Frau\pwindex{Salten, Ottilie 07.03.1868 – 22.06.1942@\textsc{Salten, Ottilie} (07.03.1868 – 22.06.1942), \emph{Schauspielerin}|pwv} bei uns nachtmahlen
               wollen, was uns ſehr freuen würde. \pend
           \pstart
           Zweitens ſchicke ich Ihnen hier ein \textsc{Manuscript\pwindex{Schnitzler, Arthur 15.05.1862 – 21.10.1931@\textsc{Schnitzler, Arthur} (15.05.1862 – 21.10.1931), \emph{Schriftsteller, Mediziner}!Zum grossen Wurstel. Burleske in einem Akt23. 04. 1905@\strich\emph{Zum großen Wurstel. Burleske in einem Akt} {[}23. 04. 1905{]}|pwv}\pwindex{Schnitzler, Arthur 15.05.1862 – 21.10.1931@\textsc{Schnitzler, Arthur} (15.05.1862 – 21.10.1931), \emph{Schriftsteller, Mediziner}!Zum grossen Wurstel. Burleske in einem Akt23. 04. 1905@\strich\emph{Zum großen Wurstel. Burleske in einem Akt} {[}23. 04. 1905{]}|pwv}}. Es ſind die einſtigen Marionetten\pwindex{Schnitzler, Arthur 15.05.1862 – 21.10.1931@\textsc{Schnitzler, Arthur} (15.05.1862 – 21.10.1931), \emph{Schriftsteller, Mediziner}!Zum grossen Wurstel. Burleske in einem Akt23. 04. 1905@\strich\emph{Zum großen Wurstel. Burleske in einem Akt} {[}23. 04. 1905{]}|pw}\pwindex{Schnitzler, Arthur 15.05.1862 – 21.10.1931@\textsc{Schnitzler, Arthur} (15.05.1862 – 21.10.1931), \emph{Schriftsteller, Mediziner}!Zum grossen Wurstel. Burleske in einem Akt23. 04. 1905@\strich\emph{Zum großen Wurstel. Burleske in einem Akt} {[}23. 04. 1905{]}|pw} (die
               natürlich auch noch niemals gedruckt waren) höchſt umgearbeitet, und ich frage Sie,
               ob Sie das Stückerl für die \label{K_L02997-1v}\edtext{Oſternu{\geminationm}er\pwindex{Zeit1902-09-27 – 1919@\emph{Die Zeit} {[}1902-09-27 – 1919{]}|pwv}}{\lemma{\textnormal{\emph{Oſternummer}}}\Cendnote{\textnormal{Arthur Schnitzler\pwindex{Schnitzler, Arthur 15.05.1862 – 21.10.1931@\textsc{Schnitzler, Arthur} (15.05.1862 – 21.10.1931), \emph{Schriftsteller, Mediziner}|pwk}: \emph{Zum großen Wurstel. Burleske in einem Akt}\pwindex{Schnitzler, Arthur 15.05.1862 – 21.10.1931@\textsc{Schnitzler, Arthur} (15.05.1862 – 21.10.1931), \emph{Schriftsteller, Mediziner}!Zum grossen Wurstel. Burleske in einem Akt23. 04. 1905@\strich\emph{Zum großen Wurstel. Burleske in einem Akt} {[}23. 04. 1905{]}|pwk}\pwindex{Schnitzler, Arthur 15.05.1862 – 21.10.1931@\textsc{Schnitzler, Arthur} (15.05.1862 – 21.10.1931), \emph{Schriftsteller, Mediziner}!Zum grossen Wurstel. Burleske in einem Akt23. 04. 1905@\strich\emph{Zum großen Wurstel. Burleske in einem Akt} {[}23. 04. 1905{]}|pwk}. In: \emph{Die Zeit}\pwindex{Zeit1902-09-27 – 1919@\emph{Die Zeit} {[}1902-09-27 – 1919{]}|pwk}, Jg. 4, Nr. 926,
                        23. 4. 1905, Beilage: Oster-Zeit, S. 3–7.
               }}}\label{K_L02997-1h} haben wollen. Ich ſchicke es Ihnen deshalb ſo früh, weil ich Ihnen, für {\pb}den Fall der Annahme, vorſchlagen möchte, es
               illuſtriren zu laſſen, \introOben{}mir\introOben{} wofür es ſich ſehr zu eignen
               ſcheint – natürlich bin ich da{\geminationn} ſehr gern bereit, \strikeout{den} mich mit dem Illuſtrator, den Sie wählen würden,
               über die Details zu beſprechen. (Eventuell wäre mit dieſem Scherz die ganze Oſterbeilage\pwindex{Zeit1902-09-27 – 1919@\emph{Die Zeit} {[}1902-09-27 – 1919{]}|pwv} ausgefüllt.) Als
               Honorar wrüde ich 600 Kronen beanſpruchen. \pend
           \pstart
           Seien Sie herzlich gegrüßt {\\[\baselineskip]}Ihr {\\[\baselineskip]}\spacefill\mbox{ArthSch}\pend
           \leftskip=0em{}
         
         \endnumbering\mylabel{h}\end{ledgroupsized}\begin{anhang}\end{anhang}\newcommand{\dateiname}{L02997}\newcommand{\titel}{Arthur Schnitzler an Felix Salten, 8. 2. 1905}\newcommand{\editorInnen}{Martin Anton Müller und Laura Untner}%% latex-leseansicht-abspann.tex
%% Abspann für die Leseansicht.
%% Der Schalter \ifkorrekturansicht ist bereits durch den Vorspann gesetzt.

%% latex-abspann.tex
%% Gemeinsamer Abspann für Korrekturansicht und Leseansicht.
%% Setzt den Schalter \ifkorrekturansicht voraus (gesetzt in den
%% einbindenden Dateien latex-korrekturansicht-abspann.tex bzw.
%% latex-leseansicht-abspann.tex).
%% ---------------------------------------------------------------

\normalsize

% Das esempio-Environment wird nur in der Leseansicht benötigt
\ifkorrekturansicht\else
\newenvironment{esempio}[3]%
{
    \vspace{1.5ex}
    \rlap{\underline{#1}}
    \par
    \setlength{\parindent}{0cm}
    \nopagebreak
    \leftskip=#2cm
    \rightskip=#3cm
}
{
    \par
}
\fi

\doendnotes{C}
\bigskip
\vfill

\clearpage

\footnotesize

\ifkorrekturansicht
  \lohead{\textsc{register}}
\fi

% theindex-Environment neu definieren ohne reledmac
\makeatletter
\renewenvironment{theindex}{%
  \ifkorrekturansicht
    \section*{\indexname}%
  \else
    \subsubsection*{Index der erwähnten Entitäten}%
  \fi
  \setlength{\parindent}{0pt}%
  \setlength{\parskip}{0pt plus 0.3pt}%
  \let\item\@idxitem
}{%
  \ifkorrekturansicht\clearpage\fi
}
\makeatother

\IfFileExists{\jobname-pw.ind}{\input{\jobname-pw.ind}}{}

% Quellenangabe nur in der Leseansicht
\ifkorrekturansicht\else
% Fallback-Definitionen, falls die .tex-Datei \titel etc. nicht gesetzt hat
\providecommand{\titel}{}
\providecommand{\editorInnen}{}
\providecommand{\dateiname}{\jobname}

\vspace{3cm}

\vfill

\footnotesize
\textsc{Quelle}: \titel. Herausgegeben von {\editorInnen}. In: \emph{Arthur Schnitzler: Briefwechsel mit Autorinnen und Autoren}.
 Digitale Edition, https://schnitzler-briefe.acdh.oeaw.ac.at/{\dateiname}.html (Stand \today)
\fi

\end{document}


      