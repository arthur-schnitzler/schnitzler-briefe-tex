%% latex-korrekturansicht-vorspann.tex
%% Vorspann für die Korrekturansicht.
%% Lädt die gemeinsame Datei latex-vorspann.tex mit gesetztem Schalter.

\newif\ifkorrekturansicht
\korrekturansichttrue

\input{../tex-inputs/latex-vorspann}


\section[ Arthur Schnitzler an Felix Salten, 8. 2. 1905]{L02997 Arthur Schnitzler an Felix Salten, 8. 2. 1905}
\nopagebreak\mylabel{L02997v}
\rehead{ }\normalsize\beginnumbering\briefempfaengerindex{Salten, Felix@\textsc{Salten, Felix}!zzzSchnitzler, Arthur@\emph{von Arthur Schnitzler}!1905-02-081@{8. 2. 1905}|(be}
\toendnotes[C]{\smallbreak\pagebreak[2]}\Standort{Wienbibliothek im Rathaus, ZPH 1681, 2.1.516.}
\physDesc{Brief, 1 Blatt, 2 Seiten, 786 Zeichen
\newline{}Handschrift: schwarze Tinte, deutsche Kurrent
\newline{}Ordnung: mit Bleistift von unbekannter Hand nummeriert: »28« }\toendnotes[C]{\smallbreak}
\pstart
           \raggedleft{}\textsc{{\pb}Wien\oindex{Wien@\textbf{Wien}, \emph{A.ADM2}|pw}}, 8. 2. 905\pend
           
\pstart{}lieber,\pend\vspace{0.5em}
\pstart
           erſtens frage ich Sie, ob Sie am \label{K_L02997-1v}\edtext{Sonntag{ }Abend mit Ihrer Frau\pwindex{Salten, Ottilie 07.03.1868 – 22.06.1942@\textsc{Salten, Ottilie} (07.03.1868 – 22.06.1942), \emph{Schauspieler/Schauspielerin}|pwv} bei uns nachtmahlen}{\lemma{\textnormal{\emph{Sonntag … nachtmahlen}}}\Cendnote{\textnormal{Siehe A. S.: \emph{Tagebuch}, 12. 2. 1905.
               }}}\label{K_L02997-1} wollen, was uns ſehr freuen würde.\pend
           
\pstart
           Zweitens ſchicke ich Ihnen hier ein \textsc{Manuscript\pwindex{Zum grossen Wurstel. Burleske in einem Akt@\emph{Zum großen Wurstel. Burleske in einem Akt}|pwv}}. Es ſind die \label{K_L02997-2v}\edtext{einſtigen Marionetten\pwindex{Zum grossen Wurstel. Burleske in einem Akt@\emph{Zum großen Wurstel. Burleske in einem Akt}|pw}}{\lemma{\textnormal{\emph{einſtigen Marionetten}}}\Cendnote{\textnormal{Am 8. 3. 1901 führte das \emph{Überbrettl}\orgindex{Ueberbrettl@Überbrettl|pwk} unter dem Titel \emph{Marionetten}\pwindex{Zum grossen Wurstel. Burleske in einem Akt@\emph{Zum großen Wurstel. Burleske in einem Akt}|pwk} die Burleske auf, die durchfiel. Schnitzler hatte seither die Szene unter dem
                  neuen Titel \emph{Zum großen Wurstel. Burleske in einem
                     Akt}\pwindex{Zum grossen Wurstel. Burleske in einem Akt@\emph{Zum großen Wurstel. Burleske in einem Akt}|pwk} überarbeitet. Den Titel \emph{Marionetten}\pwindex{Marionetten. Drei Einakter@\emph{Marionetten. Drei Einakter}|pwk} verwendete er 1906 für die Buchausgabe, die
                  diese Szene und zwei andere vereinte.}}}\label{K_L02997-2} (die natürlich auch noch niemals
               gedruckt waren) höchſt umgearbeitet; und ich frage Sie, ob Sie das Stückerl\pwindex{Zum grossen Wurstel. Burleske in einem Akt@\emph{Zum großen Wurstel. Burleske in einem Akt}|pwv} für die \label{K_L02997-3v}\edtext{Oſternu{\geminationm}er\pwindex{Zeit@\emph{Die Zeit}|pwv}}{\lemma{\textnormal{\emph{Oſternummer}}}\Cendnote{\textnormal{Arthur Schnitzler: \emph{Zum großen Wurstel. Burleske in einem Akt}\pwindex{Zum grossen Wurstel. Burleske in einem Akt@\emph{Zum großen Wurstel. Burleske in einem Akt}|pwk}. In: \emph{Die Zeit}\pwindex{Zeit@\emph{Die Zeit}|pwk}, Jg. 4, Nr. 926, 23. 4. 1905, Beilage: Oster-Zeit, S. 3–7.
               }}}\label{K_L02997-3} haben wollen. Ich ſchicke es Ihnen deshalb ſo früh, weil ich Ihnen, für {\pb}den Fall der Annahme, vorſchlagen möchte, es
               illuſtriren zu laſſen, wofür es ſich \introOben{}mir\introOben{} ſehr zu eignen
               ſcheint – natürlich bin ich da{\geminationn} ſehr gern bereit, \strikeout{den} mich mit dem \label{K_L02997-66v}\edtext{Illuſtrator\pwindex{Czegka, Berta 30.07.1880 – 04.11.1954@\textsc{Czegka, Berta} (30.07.1880 – 04.11.1954), \emph{Maler/Malerin}|pwv}}{\lemma{\textnormal{\emph{Illuſtrator}}}\Cendnote{\textnormal{Vor und nach dem
                  Text des Erstdrucks findet sich jeweils eine Illustration von Berta Czegka\pwindex{Czegka, Berta 30.07.1880 – 04.11.1954@\textsc{Czegka, Berta} (30.07.1880 – 04.11.1954), \emph{Maler/Malerin}|pwk}.}}}\label{K_L02997-66}, den Sie wählen würden, über die
               Details zu beſprechen. (Eventuell wäre mit dieſem Scherz die ganze Oſterbeilage\pwindex{Zeit@\emph{Die Zeit}|pwv} ausgefüllt.) Als Honorar würd
               ich 600 Kronen beanſpruchen.\pend
           
\pstart
           Seien Sie herzlich gegrüßt. {\\[\baselineskip]}Ihr {\\[\baselineskip]}\spacefill\mbox{Arth Sch}\pend
           \leftskip=0em{}\selectlanguage{ngerman}\endnumbering\briefempfaengerindex{Salten, Felix@\textsc{Salten, Felix}!zzzSchnitzler, Arthur@\emph{von Arthur Schnitzler}!1905-02-081@{8. 2. 1905}|)be}\mylabel{L02997h}  \normalsize

\doendnotes{C}
\bigskip
\vfill

\clearpage

\footnotesize

\lohead{\textsc{register}}

% Definiere theindex-Environment komplett neu ohne reledmac
\makeatletter
\renewenvironment{theindex}{%
  \section*{\indexname}%
  \setlength{\parindent}{0pt}%
  \setlength{\parskip}{0pt plus 0.3pt}%
  \let\item\@idxitem
}{%
  \clearpage
}
\makeatother

\IfFileExists{\jobname-pw.ind}{\input{\jobname-pw.ind}}{}

\end{document}

      