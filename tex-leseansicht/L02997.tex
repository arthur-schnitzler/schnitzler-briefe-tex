%% latex-leseansicht-vorspann.tex
%% Vorspann für die Leseansicht.
%% Lädt die gemeinsame Datei latex-vorspann.tex mit nicht gesetztem Schalter.

\newif\ifkorrekturansicht
\korrekturansichtfalse

\input{../tex-inputs/latex-vorspann}


\section[ Arthur Schnitzler an Felix Salten, 8. 2. 1905]{L02997 Arthur Schnitzler an Felix Salten,  8. 2. 1905}
\nopagebreak\mylabel{L02997v}
\rehead{ }\normalsize\beginnumbering\briefempfaengerindex{Salten, Felix@\textsc{Salten, Felix}!zzzSchnitzler, Arthur@\emph{von Arthur Schnitzler}!1905-02-081@{8. 2. 1905}|(be}
\toendnotes[C]{\smallbreak\pagebreak[2]}
\correspDesc{Versand  durch Arthur Schnitzler am 8. 2. 1905 in Wien
\newline{}Erhalt  durch Felix Salten im Zeitraum [8. 2. 1905
                  – 11. 2. 1905?] in Wien}\toendnotes[C]{\smallbreak}
\Standort{Wienbibliothek im Rathaus, ZPH 1681, 2.1.516.}
\physDesc{Brief, 1 Blatt, 2 Seiten, 786 Zeichen
\newline{}Handschrift: schwarze Tinte, deutsche Kurrent
\newline{}Ordnung: mit Bleistift von unbekannter Hand nummeriert: »28« }\toendnotes[C]{\smallbreak}
\pstart
           \raggedleft{}\textsc{{\pb}Wien\oindex{Wien@\textbf{Wien}, \emph{Verwaltungsgebiet}|pw}}, 8. 2. 905\pend
           
\pstart{}lieber,\pend\vspace{0.5em}
\pstart
           erſtens frage ich Sie, ob Sie am \label{K_L02997-1v}\edtext{Sonntag{ }Abend mit Ihrer Frau\pwindex{Salten, Ottilie 7.\,3.\,1868 Prag – 22.\,6.\,1942 Zürich@\textsc{Salten, Ottilie} (7.\,3.\,1868 Prag – 22.\,6.\,1942 Zürich), \emph{Schauspielerin}|pwv} bei uns nachtmahlen}{\lemma{\textnormal{\emph{Sonntag … nachtmahlen}}}\Cendnote{\textnormal{Siehe A. S.: \emph{Tagebuch}, 12. 2. 1905.
               }}}\label{K_L02997-1} wollen, was uns{ }ſehr freuen würde.\pend
           
\pstart
           Zweitens{ }ſchicke ich Ihnen hier ein \textsc{Manuscript\pwindex{Schnitzler, Arthur 15.\,5.\,1862 Wien – 21.\,10.\,1931 ebd.@\textsc{Schnitzler, Arthur} (15.\,5.\,1862 Wien – 21.\,10.\,1931 ebd.), \emph{Schriftsteller, Mediziner}!Zum großen Wurstel. Burleske in einem Akt@\strich\emph{Zum großen Wurstel. Burleske in einem Akt}|pwv}}. Es{ }ſind die \label{K_L02997-2v}\edtext{einſtigen Marionetten\pwindex{Schnitzler, Arthur 15.\,5.\,1862 Wien – 21.\,10.\,1931 ebd.@\textsc{Schnitzler, Arthur} (15.\,5.\,1862 Wien – 21.\,10.\,1931 ebd.), \emph{Schriftsteller, Mediziner}!Zum großen Wurstel. Burleske in einem Akt@\strich\emph{Zum großen Wurstel. Burleske in einem Akt}|pw}}{\lemma{\textnormal{\emph{einstigen Marionetten}}}\Cendnote{\textnormal{Am 8. 3. 1901 führte das \emph{Überbrettl}\orgindex{Überbrettl@Überbrettl|pwk} unter dem Titel \emph{Marionetten}\pwindex{Schnitzler, Arthur 15.\,5.\,1862 Wien – 21.\,10.\,1931 ebd.@\textsc{Schnitzler, Arthur} (15.\,5.\,1862 Wien – 21.\,10.\,1931 ebd.), \emph{Schriftsteller, Mediziner}!Zum großen Wurstel. Burleske in einem Akt@\strich\emph{Zum großen Wurstel. Burleske in einem Akt}|pwk} die Burleske auf, die durchfiel. Schnitzler hatte seither die Szene unter dem
                  neuen Titel \emph{Zum großen Wurstel. Burleske in einem
                     Akt}\pwindex{Schnitzler, Arthur 15.\,5.\,1862 Wien – 21.\,10.\,1931 ebd.@\textsc{Schnitzler, Arthur} (15.\,5.\,1862 Wien – 21.\,10.\,1931 ebd.), \emph{Schriftsteller, Mediziner}!Zum großen Wurstel. Burleske in einem Akt@\strich\emph{Zum großen Wurstel. Burleske in einem Akt}|pwk} überarbeitet. Den Titel \emph{Marionetten}\pwindex{Schnitzler, Arthur 15.\,5.\,1862 Wien – 21.\,10.\,1931 ebd.@\textsc{Schnitzler, Arthur} (15.\,5.\,1862 Wien – 21.\,10.\,1931 ebd.), \emph{Schriftsteller, Mediziner}!Marionetten. Drei Einakter@\strich\emph{Marionetten. Drei Einakter}|pwk} verwendete er 1906 für die Buchausgabe, die
                  diese Szene und zwei andere vereinte.}}}\label{K_L02997-2} (die natürlich auch noch niemals
               gedruckt waren) höchſt umgearbeitet; und ich frage Sie, ob Sie das Stückerl\pwindex{Schnitzler, Arthur 15.\,5.\,1862 Wien – 21.\,10.\,1931 ebd.@\textsc{Schnitzler, Arthur} (15.\,5.\,1862 Wien – 21.\,10.\,1931 ebd.), \emph{Schriftsteller, Mediziner}!Zum großen Wurstel. Burleske in einem Akt@\strich\emph{Zum großen Wurstel. Burleske in einem Akt}|pwv} für die \label{K_L02997-3v}\edtext{Oſternu{\geminationm}er\pwindex{Zeit@\emph{Die Zeit}|pwv}}{\lemma{\textnormal{\emph{Osternummer}}}\Cendnote{\textnormal{Arthur Schnitzler: \emph{Zum großen Wurstel. Burleske in einem Akt}\pwindex{Schnitzler, Arthur 15.\,5.\,1862 Wien – 21.\,10.\,1931 ebd.@\textsc{Schnitzler, Arthur} (15.\,5.\,1862 Wien – 21.\,10.\,1931 ebd.), \emph{Schriftsteller, Mediziner}!Zum großen Wurstel. Burleske in einem Akt@\strich\emph{Zum großen Wurstel. Burleske in einem Akt}|pwk}. In: \emph{Die Zeit}\pwindex{Zeit@\emph{Die Zeit}|pwk}, Jg. 4, Nr. 926, 23. 4. 1905, Beilage: Oster-Zeit, S. 3–7.
               }}}\label{K_L02997-3} haben wollen. Ich{ }ſchicke es Ihnen deshalb{ }ſo früh, weil ich Ihnen, für {\pb}den Fall der Annahme, vorſchlagen möchte, es
               illuſtriren zu laſſen, wofür es{ }ſich \introOben{}mir\introOben{}{ }ſehr zu eignen{ }ſcheint – natürlich bin ich da{\geminationn}{ }ſehr gern bereit, \strikeout{den} mich mit dem \label{K_L02997-66v}\edtext{Illuſtrator\pwindex{Czegka, Berta 30.\,7.\,1880 Feldkirch – 4.\,11.\,1954 Hall in Tirol@\textsc{Czegka, Berta} (30.\,7.\,1880 Feldkirch – 4.\,11.\,1954 Hall in Tirol), \emph{Malerin}|pwv}}{\lemma{\textnormal{\emph{Illustrator}}}\Cendnote{\textnormal{Vor und nach dem
                  Text des Erstdrucks findet sich jeweils eine Illustration von Berta Czegka\pwindex{Czegka, Berta 30.\,7.\,1880 Feldkirch – 4.\,11.\,1954 Hall in Tirol@\textsc{Czegka, Berta} (30.\,7.\,1880 Feldkirch – 4.\,11.\,1954 Hall in Tirol), \emph{Malerin}|pwk}.}}}\label{K_L02997-66}, den Sie wählen würden, über die
               Details zu beſprechen. (Eventuell wäre mit dieſem Scherz die ganze Oſterbeilage\pwindex{Zeit@\emph{Die Zeit}|pwv} ausgefüllt.) Als Honorar würd
               ich 600 Kronen beanſpruchen.\pend
           
\pstart
           Seien Sie herzlich gegrüßt. {\\[\baselineskip]}Ihr {\\[\baselineskip]}\spacefill\mbox{Arth Sch}\pend
           \leftskip=0em{}\selectlanguage{ngerman}\endnumbering\briefempfaengerindex{Salten, Felix@\textsc{Salten, Felix}!zzzSchnitzler, Arthur@\emph{von Arthur Schnitzler}!1905-02-081@{8. 2. 1905}|)be}\mylabel{L02997h}  \newcommand{\dateiname}{L02997}\newcommand{\titel}{Arthur Schnitzler an Felix Salten, 8. 2. 1905}\newcommand{\editorInnen}{Martin Anton Müller und Laura Untner}%% latex-leseansicht-abspann.tex
%% Abspann für die Leseansicht.
%% Der Schalter \ifkorrekturansicht ist bereits durch den Vorspann gesetzt.

%% latex-abspann.tex
%% Gemeinsamer Abspann für Korrekturansicht und Leseansicht.
%% Setzt den Schalter \ifkorrekturansicht voraus (gesetzt in den
%% einbindenden Dateien latex-korrekturansicht-abspann.tex bzw.
%% latex-leseansicht-abspann.tex).
%% ---------------------------------------------------------------

\normalsize

% Das esempio-Environment wird nur in der Leseansicht benötigt
\ifkorrekturansicht\else
\newenvironment{esempio}[3]%
{
    \vspace{1.5ex}
    \rlap{\underline{#1}}
    \par
    \setlength{\parindent}{0cm}
    \nopagebreak
    \leftskip=#2cm
    \rightskip=#3cm
}
{
    \par
}
\fi

\doendnotes{C}
\bigskip
\vfill

\clearpage

\footnotesize

\ifkorrekturansicht
  \lohead{\textsc{register}}
\fi

% theindex-Environment neu definieren ohne reledmac
\makeatletter
\renewenvironment{theindex}{%
  \ifkorrekturansicht
    \section*{\indexname}%
  \else
    \subsubsection*{Index der erwähnten Entitäten}%
  \fi
  \setlength{\parindent}{0pt}%
  \setlength{\parskip}{0pt plus 0.3pt}%
  \let\item\@idxitem
}{%
  \ifkorrekturansicht\clearpage\fi
}
\makeatother

\IfFileExists{\jobname-pw.ind}{\input{\jobname-pw.ind}}{}

% Quellenangabe nur in der Leseansicht
\ifkorrekturansicht\else
% Fallback-Definitionen, falls die .tex-Datei \titel etc. nicht gesetzt hat
\providecommand{\titel}{}
\providecommand{\editorInnen}{}
\providecommand{\dateiname}{\jobname}

\vspace{3cm}

\vfill

\footnotesize
\textsc{Quelle}: \titel. Herausgegeben von {\editorInnen}. In: \emph{Arthur Schnitzler: Briefwechsel mit Autorinnen und Autoren}.
 Digitale Edition, https://schnitzler-briefe.acdh.oeaw.ac.at/{\dateiname}.html (Stand \today)
\fi

\end{document}


