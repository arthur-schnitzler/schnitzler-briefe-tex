%% latex-leseansicht-vorspann.tex
%% Vorspann für die Leseansicht.
%% Lädt die gemeinsame Datei latex-vorspann.tex mit nicht gesetztem Schalter.

\newif\ifkorrekturansicht
\korrekturansichtfalse

\input{../tex-inputs/latex-vorspann}


         
         \renewcommand{\erwaehntePersonen}{Personen: Olga Schnitzler, Elisabeth Steinrück}
         \renewcommand{\erwaehnteOrte}{Orte: Berlin, Brühl, Dessauer Straße, Edmund-Weiß-Gasse, Hinterbrühl, Wien}
         \renewcommand{\erwaehnteWerke}{}
               \section[ Paul Goldmann an Arthur Schnitzler, 16. 5. {[}1902{]}]{ Paul Goldmann an Arthur Schnitzler, 16. 5. {[}1902{]}}\nopagebreak\mylabel{v}\rehead{ }\begin{ledgroupsized}[t]{13cm}\normalsize\beginnumbering \toendnotes[C]{\smallbreak\pagebreak[2]} \Standort{DLA, A:Schnitzler, HS.NZ85.1.3172.}
\physDesc{Brief, 1 Blatt, 2 Seiten, 633 Zeichen
\newline{}Handschrift: blaue Tinte, deutsche Kurrent
\newline{}Schnitzler: mit Bleistift das Jahr »1902« vermerkt }\toendnotes[C]{\smallbreak}\pstart
           \noindent{}\raggedleft{}{\pb}\textcolor{gray}{\textbf{DESSAUERSTRASSE 19}}\oindex{Dessauer Strasse@\textbf{Dessauer Straße}|pw}\pend
           \pstart
           Berlin\oindex{Berlin@\textbf{Berlin}|pw}, 16. Mai.\pend
           \pstart\center{}Mein lieber Freund,\pend\pstart
           Ich habe mich alſo entſchloſſen, zu fahren, nur weiß ich noch nicht, ob ich Samſtag{ }früh oder Samſtag{ }Abend fahre. Da ich mir auch denke, daß Du jedenfalls ſchon \label{K_L03209-1v}\edtext{Samſtag{ }Abend}{\lemma{\textnormal{\emph{Samſtag Abend}}}\Cendnote{\textnormal{Schnitzler\pwindex{Schnitzler, Arthur 15.05.1862 – 21.10.1931@\textsc{Schnitzler, Arthur} (15.05.1862 – 21.10.1931), \emph{Schriftsteller, Mediziner}|pwk} fuhr Freitagabend, am 16. 5. 1902, nach Hinterbrühl\oindex{Hinterbruehl@\textbf{Hinterbrühl}|pwk} und dürfte bis zum 17. 5. 1902 geblieben sein.}}}\label{K_L03209-1h} nach der Brühl\oindex{Bruehl@\textbf{Brühl}|pw} fahren möchteſt, ſo will ich Dich in
               Deinen Diſpoſitionen auf keinen Fall ſtören und werde Dir über {\pb}meine Ankunft nichts Näheres mittheilen. Sonntag{ }früh komme ich in Deine Wohnung\oindex{Edmund-Weiss-Gasse@\textbf{Edmund-Weiß-Gasse}|pwv}. Wenn Du in der \label{K_L03209-4v}\edtext{Brühl\oindex{Bruehl@\textbf{Brühl}|pw}}{\lemma{\textnormal{\emph{Brühl}}}\Cendnote{\textnormal{Schnitzler\pwindex{Schnitzler, Arthur 15.05.1862 – 21.10.1931@\textsc{Schnitzler, Arthur} (15.05.1862 – 21.10.1931), \emph{Schriftsteller, Mediziner}|pwk} und Goldmann\pwindex{Goldmann, Paul 31.01.1865 – 25.09.1935@\textsc{Goldmann, Paul} (31.01.1865 – 25.09.1935), \emph{Schriftsteller, Journalist}|pwk} trafen am 18. 5. 1902 in Wien\oindex{Wien@\textbf{Wien}|pwk} aufeinander. Am 19. 5. 1902 machten sie gemeinsam einen Ausflug nach Hinterbrühl\oindex{Hinterbruehl@\textbf{Hinterbrühl}|pwk}.}}}\label{K_L03209-4h} biſt, ſo hinterlaſſe mir einen Brief mit
               Angabe der Adreſſe. Grüße die Mädeln\pwindex{Schnitzler, Olga 17.01.1882 – 13.01.1970@\textsc{Schnitzler, Olga} (17.01.1882 – 13.01.1970), \emph{Schauspielerin, Sängerin}|pwv}\pwindex{Steinrueck, Elisabeth 19.11.1885 – 07.04.1920@\textsc{Steinrück, Elisabeth} (19.11.1885 – 07.04.1920)|pwv} nicht wieder auf das Herzlichſte. Ich komme
               beim beſten Willen nicht mehr dazu, auf ihre lieben Briefe zu antworten.\pend
           \pstart
           Von Herzen {\\[\baselineskip]}Dein {\\[\baselineskip]}\spacefill\mbox{Paul Goldmann}\pend
           \leftskip=0em{}
         
         \endnumbering\mylabel{h}\end{ledgroupsized}  \newcommand{\dateiname}{L03209}\newcommand{\titel}{Paul Goldmann an Arthur Schnitzler, 16. 5. [1902]}\newcommand{\editorInnen}{Martin Anton Müller und Laura Untner}%% latex-leseansicht-abspann.tex
%% Abspann für die Leseansicht.
%% Der Schalter \ifkorrekturansicht ist bereits durch den Vorspann gesetzt.

%% latex-abspann.tex
%% Gemeinsamer Abspann für Korrekturansicht und Leseansicht.
%% Setzt den Schalter \ifkorrekturansicht voraus (gesetzt in den
%% einbindenden Dateien latex-korrekturansicht-abspann.tex bzw.
%% latex-leseansicht-abspann.tex).
%% ---------------------------------------------------------------

\normalsize

% Das esempio-Environment wird nur in der Leseansicht benötigt
\ifkorrekturansicht\else
\newenvironment{esempio}[3]%
{
    \vspace{1.5ex}
    \rlap{\underline{#1}}
    \par
    \setlength{\parindent}{0cm}
    \nopagebreak
    \leftskip=#2cm
    \rightskip=#3cm
}
{
    \par
}
\fi

\doendnotes{C}
\bigskip
\vfill

\clearpage

\footnotesize

\ifkorrekturansicht
  \lohead{\textsc{register}}
\fi

% theindex-Environment neu definieren ohne reledmac
\makeatletter
\renewenvironment{theindex}{%
  \ifkorrekturansicht
    \section*{\indexname}%
  \else
    \subsubsection*{Index der erwähnten Entitäten}%
  \fi
  \setlength{\parindent}{0pt}%
  \setlength{\parskip}{0pt plus 0.3pt}%
  \let\item\@idxitem
}{%
  \ifkorrekturansicht\clearpage\fi
}
\makeatother

\IfFileExists{\jobname-pw.ind}{\input{\jobname-pw.ind}}{}

% Quellenangabe nur in der Leseansicht
\ifkorrekturansicht\else
% Fallback-Definitionen, falls die .tex-Datei \titel etc. nicht gesetzt hat
\providecommand{\titel}{}
\providecommand{\editorInnen}{}
\providecommand{\dateiname}{\jobname}

\vspace{3cm}

\vfill

\footnotesize
\textsc{Quelle}: \titel. Herausgegeben von {\editorInnen}. In: \emph{Arthur Schnitzler: Briefwechsel mit Autorinnen und Autoren}.
 Digitale Edition, https://schnitzler-briefe.acdh.oeaw.ac.at/{\dateiname}.html (Stand \today)
\fi

\end{document}


      