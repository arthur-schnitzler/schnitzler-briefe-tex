%% latex-leseansicht-vorspann.tex
%% Vorspann für die Leseansicht.
%% Lädt die gemeinsame Datei latex-vorspann.tex mit nicht gesetztem Schalter.

\newif\ifkorrekturansicht
\korrekturansichtfalse

\input{../tex-inputs/latex-vorspann}


\section[Ferdinand von Saar an Arthur Schnitzler, 11. 10. 1902]{L01240 Ferdinand von Saar an Arthur Schnitzler, 11. 10. 1902}
\nopagebreak\mylabel{L01240v}
\rehead{ }\normalsize\beginnumbering\briefempfaengerindex{Schnitzler, Arthur@\textsc{Schnitzler, Arthur}!zzzSaar, Ferdinand von@\emph{von Ferdinand von Saar}!1902-10-111@{11. 10. 1902}|(be}
\toendnotes[C]{\smallbreak\pagebreak[2]}
\correspDesc{Versand  durch Ferdindand von Saar am 11. 10. 1902 in Wien
\newline{}Erhalt  durch Arthur Schnitzler im Zeitraum [11. 10. 1902 – 15. 10. 1902?] in Wien}\toendnotes[C]{\smallbreak}
\Standort{CUL, Schnitzler, B 88.}
\physDesc{Brief, 1 Blatt, 1 Seite, 183 Zeichen
\newline{}Handschrift: schwarze Tinte, deutsche Kurrent
\newline{}Schnitzler: mit Bleistift nummeriert: »7« }\toendnotes[C]{\smallbreak}
\pstart
           \raggedleft{}{\pb}\textsc{Wien-Döbling}\oindex{XIX., Döbling@\textbf{XIX., Döbling}, \emph{Verwaltungsgebiet}|pw}.{ }11/10. 1902.\pend
           \vspace{0.5em}
\pstart
           Herzlichen Dank, verehrteſter Poet, für Ihre{ }ſo freundliche \label{K_L01240-1v}\edtext{Kundgebung}{\lemma{\textnormal{\emph{Kundgebung}}}\Cendnote{\textnormal{Zu
                  seinem Geburtstag am 30. 9. 1902 wurde Saar\pwindex{Saar, Ferdinand von 30.\,9.\,1833 Wien – 24.\,7.\,1906 ebd.@\textsc{Saar, Ferdinand von} (30.\,9.\,1833 Wien – 24.\,7.\,1906 ebd.), \emph{Schriftsteller}|pwk} eine Adresse verschiedener Schriftsteller und ein
                  Widmungsband überreicht, der von Schnitzler{ }\emph{Liebelei. Erstes Bild}\pwindex{Schnitzler, Arthur 15.\,5.\,1862 Wien – 21.\,10.\,1931 ebd.@\textsc{Schnitzler, Arthur} (15.\,5.\,1862 Wien – 21.\,10.\,1931 ebd.), \emph{Schriftsteller, Mediziner}!Liebelei. Erstes Bild@\strich\emph{Liebelei. Erstes Bild}|pwk} enthielt. (\emph{Widmungen zur Feier des siebzigsten
                        Geburtstages Ferdinand von Saar’s}\pwindex{Widmungen zur Feier des siebzigsten Geburtstages Ferdinand von Saar’s.@\emph{Widmungen zur Feier des siebzigsten Geburtstages Ferdinand von Saar’s.}|pwk}. Herausgegeben von Richard Specht\pwindex{Specht, Richard 7.\,12.\,1870 Wien – 18.\,3.\,1932 ebd.@\textsc{Specht, Richard} (7.\,12.\,1870 Wien – 18.\,3.\,1932 ebd.), \emph{Schriftsteller, Journalist, Kritiker}|pwk}. Buchschmuck von A. F. Seligmann\pwindex{Seligmann, Adalbert Franz 2.\,4.\,1862 Wien – 13.\,12.\,1945 ebd.@\textsc{Seligmann, Adalbert Franz} (2.\,4.\,1862 Wien – 13.\,12.\,1945 ebd.), \emph{Maler, Publizist}|pwk}. Wien: \emph{Wiener Verlag}\orgindex{Wiener Verlag@Wiener Verlag|pwk}{ }1903, S. 175–196. Der Widmungsband erschien, auf
                     1903 vordatiert, am 14. 11. 1902. Folglich, weil er
                  noch nicht verfügbar war, dürfte hier die Adresse gemeint sein.}}}\label{K_L01240-1} an meinem
                  »70\textsuperscript{ten}«!\pend
           
\pstart
           Mit allen guten Wünſchen und{\\[\baselineskip]}in treuer Erinnerung{\\[\baselineskip]}Ihr{\\[\baselineskip]}\spacefill\mbox{Ferdinand von Saar.}\pend
           \leftskip=0em{}\selectlanguage{ngerman}\endnumbering\briefempfaengerindex{Schnitzler, Arthur@\textsc{Schnitzler, Arthur}!zzzSaar, Ferdinand von@\emph{von Ferdinand von Saar}!1902-10-111@{11. 10. 1902}|)be}\mylabel{L01240h}  \newcommand{\dateiname}{L01240}\newcommand{\titel}{Ferdinand von Saar an Arthur Schnitzler, 11. 10. 1902}\newcommand{\editorInnen}{Martin Anton Müller und Gerd-Hermann Susen}%% latex-leseansicht-abspann.tex
%% Abspann für die Leseansicht.
%% Der Schalter \ifkorrekturansicht ist bereits durch den Vorspann gesetzt.

%% latex-abspann.tex
%% Gemeinsamer Abspann für Korrekturansicht und Leseansicht.
%% Setzt den Schalter \ifkorrekturansicht voraus (gesetzt in den
%% einbindenden Dateien latex-korrekturansicht-abspann.tex bzw.
%% latex-leseansicht-abspann.tex).
%% ---------------------------------------------------------------

\normalsize

% Das esempio-Environment wird nur in der Leseansicht benötigt
\ifkorrekturansicht\else
\newenvironment{esempio}[3]%
{
    \vspace{1.5ex}
    \rlap{\underline{#1}}
    \par
    \setlength{\parindent}{0cm}
    \nopagebreak
    \leftskip=#2cm
    \rightskip=#3cm
}
{
    \par
}
\fi

\doendnotes{C}
\bigskip
\vfill

\clearpage

\footnotesize

\ifkorrekturansicht
  \lohead{\textsc{register}}
\fi

% theindex-Environment neu definieren ohne reledmac
\makeatletter
\renewenvironment{theindex}{%
  \ifkorrekturansicht
    \section*{\indexname}%
  \else
    \subsubsection*{Index der erwähnten Entitäten}%
  \fi
  \setlength{\parindent}{0pt}%
  \setlength{\parskip}{0pt plus 0.3pt}%
  \let\item\@idxitem
}{%
  \ifkorrekturansicht\clearpage\fi
}
\makeatother

\IfFileExists{\jobname-pw.ind}{\input{\jobname-pw.ind}}{}

% Quellenangabe nur in der Leseansicht
\ifkorrekturansicht\else
% Fallback-Definitionen, falls die .tex-Datei \titel etc. nicht gesetzt hat
\providecommand{\titel}{}
\providecommand{\editorInnen}{}
\providecommand{\dateiname}{\jobname}

\vspace{3cm}

\vfill

\footnotesize
\textsc{Quelle}: \titel. Herausgegeben von {\editorInnen}. In: \emph{Arthur Schnitzler: Briefwechsel mit Autorinnen und Autoren}.
 Digitale Edition, https://schnitzler-briefe.acdh.oeaw.ac.at/{\dateiname}.html (Stand \today)
\fi

\end{document}


