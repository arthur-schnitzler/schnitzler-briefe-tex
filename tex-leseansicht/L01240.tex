%% latex-korrekturansicht-vorspann.tex
%% Vorspann für die Korrekturansicht.
%% Lädt die gemeinsame Datei latex-vorspann.tex mit gesetztem Schalter.

\newif\ifkorrekturansicht
\korrekturansichttrue

\input{../tex-inputs/latex-vorspann}


\section[Ferdinand von Saar an Arthur Schnitzler, 11. 10. 1902]{L01240 Ferdinand von Saar an Arthur Schnitzler, 11. 10. 1902}
\nopagebreak\mylabel{L01240v}
\rehead{ }\normalsize\beginnumbering\briefempfaengerindex{Schnitzler, Arthur@\textsc{Schnitzler, Arthur}!zzzSaar, Ferdinand von@\emph{von Ferdinand von Saar}!1902-10-111@{11. 10. 1902}|(be}
\toendnotes[C]{\smallbreak\pagebreak[2]}\Standort{CUL, Schnitzler, B 88.}
\physDesc{Brief, 1 Blatt, 1 Seite, 183 Zeichen
\newline{}Handschrift: schwarze Tinte, deutsche Kurrent
\newline{}Schnitzler: mit Bleistift nummeriert: »7« }\toendnotes[C]{\smallbreak}
\pstart
           \raggedleft{}{\pb}\textsc{Wien-Döbling}\oindex{XIX., Doebling@\textbf{XIX., Döbling}, \emph{A.ADM3}|pw}.{ }11/10. 1902.\pend
           \vspace{0.5em}
\pstart
           Herzlichen Dank, verehrteſter Poet, für Ihre ſo freundliche \label{K_L01240-1v}\edtext{Kundgebung}{\lemma{\textnormal{\emph{Kundgebung}}}\Cendnote{\textnormal{Zu
                  seinem Geburtstag am 30. 9. 1902 wurde Saar\pwindex{Saar, Ferdinand von 30.09.1833 – 24.07.1906@\textsc{Saar, Ferdinand von} (30.09.1833 – 24.07.1906), \emph{Schriftsteller/Schriftstellerin}|pwk} eine Adresse verschiedener Schriftsteller und ein
                  Widmungsband überreicht, der von Schnitzler{ }\emph{Liebelei. Erstes Bild}\pwindex{Liebelei. Erstes Bild@\emph{Liebelei. Erstes Bild}|pwk} enthielt. (\emph{Widmungen zur Feier des siebzigsten
                        Geburtstages Ferdinand von Saar’s}\pwindex{Widmungen zur Feier des siebzigsten Geburtstages Ferdinand von Saar s.@\emph{Widmungen zur Feier des siebzigsten Geburtstages Ferdinand von Saar’s.}|pwk}. Herausgegeben von Richard Specht\pwindex{Specht, Richard 07.12.1870 – 18.03.1932@\textsc{Specht, Richard} (07.12.1870 – 18.03.1932), \emph{Schriftsteller/Schriftstellerin, Journalist/Journalistin, Kritiker/Kritikerin}|pwk}. Buchschmuck von A. F. Seligmann\pwindex{Seligmann, Adalbert Franz 02.04.1862 – 13.12.1945@\textsc{Seligmann, Adalbert Franz} (02.04.1862 – 13.12.1945), \emph{Maler/Malerin, Publizist/Publizistin}|pwk}. Wien: \emph{Wiener Verlag}\orgindex{Wiener Verlag@Wiener Verlag|pwk}{ }1903, S. 175–196. Der Widmungsband erschien, auf
                     1903 vordatiert, am 14. 11. 1902. Folglich, weil er
                  noch nicht verfügbar war, dürfte hier die Adresse gemeint sein.}}}\label{K_L01240-1} an meinem
                  »70\textsuperscript{ten}«!\pend
           
\pstart
           Mit allen guten Wünſchen und{\\[\baselineskip]}in treuer Erinnerung{\\[\baselineskip]}Ihr{\\[\baselineskip]}\spacefill\mbox{Ferdinand von Saar.}\pend
           \leftskip=0em{}\selectlanguage{ngerman}\endnumbering\briefempfaengerindex{Schnitzler, Arthur@\textsc{Schnitzler, Arthur}!zzzSaar, Ferdinand von@\emph{von Ferdinand von Saar}!1902-10-111@{11. 10. 1902}|)be}\mylabel{L01240h}  \normalsize

\doendnotes{C}
\bigskip
\vfill

\clearpage

\footnotesize

\lohead{\textsc{register}}

% Definiere theindex-Environment komplett neu ohne reledmac
\makeatletter
\renewenvironment{theindex}{%
  \section*{\indexname}%
  \setlength{\parindent}{0pt}%
  \setlength{\parskip}{0pt plus 0.3pt}%
  \let\item\@idxitem
}{%
  \clearpage
}
\makeatother

\IfFileExists{\jobname-pw.ind}{\input{\jobname-pw.ind}}{}

\end{document}

      