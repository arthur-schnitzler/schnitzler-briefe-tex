%% latex-leseansicht-vorspann.tex
%% Vorspann für die Leseansicht.
%% Lädt die gemeinsame Datei latex-vorspann.tex mit nicht gesetztem Schalter.

\newif\ifkorrekturansicht
\korrekturansichtfalse

\input{../tex-inputs/latex-vorspann}


\section[ Arthur Schnitzler an Felix Salten, 30. 5. 1908]{L03012 Arthur Schnitzler an Felix Salten,  30. 5. 1908}
\nopagebreak\mylabel{L03012v}
\rehead{ }\normalsize\beginnumbering\briefempfaengerindex{Salten, Felix@\textsc{Salten, Felix}!zzzSchnitzler, Arthur@\emph{von Arthur Schnitzler}!1908-05-301@{30. 5. 1908}|(be}
\toendnotes[C]{\smallbreak\pagebreak[2]}
\correspDesc{Versand  durch Arthur Schnitzler am 30. 5. 1908 in Wien
\newline{}Erhalt  durch Felix Salten im Zeitraum [30. 5. 1908
                  – 2. 6. 1908?] in Wien}\toendnotes[C]{\smallbreak}
\Standort{Wienbibliothek im Rathaus, ZPH 1681, 2.1.516.}
\physDesc{Brief, 1 Blatt, 2 Seiten, 321 Zeichen
\newline{}Handschrift: schwarze Tinte, deutsche Kurrent
\newline{}Ordnung: mit Bleistift von unbekannter Hand nummeriert: »\textcolor{gray}{1}8« }
\buchAbdrucke{\weitereDrucke{Arthur Schnitzler: \emph{Briefe 1875–1912}. Herausgegeben von Therese Nickl und Heinrich Schnitzler. Frankfurt am Main: \emph{S. Fischer} 1981, S. 578.} }\toendnotes[C]{\smallbreak}
\pstart
           {\pb}\textcolor{gray}{\textbf{Dr. Arthur Schnitzler}}\hfill 30. 5. 908.\pend
           
\pstart
           \textcolor{gray}{\textbf{Wien XVIII. Spoettelgasse 7\oindex{Wien@\textbf{Wien}!XVIII., Währing@\textbf{XVIII., Währing}!Edmund-Weiß-Gasse 7@\textbf{Edmund-Weiß-Gasse 7}, \emph{Wohngebäude}|pw}.}}\pend
           \vspace{0.5em}
\pstart
           mein lieber, ich ka{\geminationn} Ihnen gar nicht{ }ſagen, \uline{wie} ich mich \label{K_L03012-1v}\edtext{gefreut}{\lemma{\textnormal{\emph{gefreut}}}\Cendnote{\textnormal{Salten\pwindex{Salten, Felix 6.\,9.\,1869 Budapest – 8.\,10.\,1945 Zürich@\textsc{Salten, Felix} (6.\,9.\,1869 Budapest – 8.\,10.\,1945 Zürich), \emph{Schriftsteller, Journalist, Chefredakteur}|pwk} hatte die allererste Rezension von
                     \emph{Der Weg ins Freie}\pwindex{Schnitzler, Arthur 15.\,5.\,1862 Wien – 21.\,10.\,1931 ebd.@\textsc{Schnitzler, Arthur} (15.\,5.\,1862 Wien – 21.\,10.\,1931 ebd.), \emph{Schriftsteller, Mediziner}!Weg ins Freie. Roman@\strich\emph{Der Weg ins Freie. Roman}|pwk} verfasst: Felix Salten\pwindex{Salten, Felix 6.\,9.\,1869 Budapest – 8.\,10.\,1945 Zürich@\textsc{Salten, Felix} (6.\,9.\,1869 Budapest – 8.\,10.\,1945 Zürich), \emph{Schriftsteller, Journalist, Chefredakteur}|pwk}: \emph{Schnitzlers Wiener Roman}\pwindex{Salten, Felix 6.\,9.\,1869 Budapest – 8.\,10.\,1945 Zürich@\textsc{Salten, Felix} (6.\,9.\,1869 Budapest – 8.\,10.\,1945 Zürich), \emph{Schriftsteller, Journalist, Chefredakteur}!Schnitzlers Wiener Roman@\strich\emph{Schnitzlers Wiener Roman}|pwk}. In: \emph{Die Zeit}\pwindex{Zeit@\emph{Die Zeit}|pwk}, Jg. 7, Nr. 2042, 30. 5. 1908, Morgenblatt, S. 1–2. Die Rezension verweist auf
                  die Buchausgabe, die ihm aber zu diesem Zeitpunkt nur als Vorabexemplar
                  vorgelegen haben könnte. Wahrscheinlicher ist, dass ihm Schnitzler den Text des 6. und (letzten) Teils des
                  Vorabdrucks in der \emph{Der neuen Rundschau}\pwindex{neue Rundschau@\emph{Die neue Rundschau}|pwk} oder
                  sonst eine Druckfahne zur Verfügung gestellt hatte (vgl. XXXX Auszeichnungsfehler: Dokument L03509 nicht gefunden). Schnitzler zeigte sich im \emph{Tagebuch}\pwindex{Schnitzler, Arthur 15.\,5.\,1862 Wien – 21.\,10.\,1931 ebd.@\textsc{Schnitzler, Arthur} (15.\,5.\,1862 Wien – 21.\,10.\,1931 ebd.), \emph{Schriftsteller, Mediziner}!Tagebuch@\strich\emph{Tagebuch}|pwk} gerührt: »In der Zeit\pwindex{Zeit@\emph{Die Zeit}|pw}{ }Feuilleton\pwindex{Salten, Felix 6.\,9.\,1869 Budapest – 8.\,10.\,1945 Zürich@\textsc{Salten, Felix} (6.\,9.\,1869 Budapest – 8.\,10.\,1945 Zürich), \emph{Schriftsteller, Journalist, Chefredakteur}!Schnitzlers Wiener Roman@\strich\emph{Schnitzlers Wiener Roman}|pwv}{ }Salten\pwindex{Salten, Felix 6.\,9.\,1869 Budapest – 8.\,10.\,1945 Zürich@\textsc{Salten, Felix} (6.\,9.\,1869 Budapest – 8.\,10.\,1945 Zürich), \emph{Schriftsteller, Journalist, Chefredakteur}|pw}’s über den Roman\pwindex{Schnitzler, Arthur 15.\,5.\,1862 Wien – 21.\,10.\,1931 ebd.@\textsc{Schnitzler, Arthur} (15.\,5.\,1862 Wien – 21.\,10.\,1931 ebd.), \emph{Schriftsteller, Mediziner}!Weg ins Freie. Roman@\strich\emph{Der Weg ins Freie. Roman}|pwv}. Sehr schön; fast ergreifend –
                     ohne Einschränkung. – Schrieb ihm.«}}}\label{K_L03012-1} habe. Aber Sie können{ }ſichs
               ja denken. Daſs Sie der Erſte{ }ſind, der{ }ſich vernehmen lieſs\pwindex{Salten, Felix 6.\,9.\,1869 Budapest – 8.\,10.\,1945 Zürich@\textsc{Salten, Felix} (6.\,9.\,1869 Budapest – 8.\,10.\,1945 Zürich), \emph{Schriftsteller, Journalist, Chefredakteur}!Schnitzlers Wiener Roman@\strich\emph{Schnitzlers Wiener Roman}|pwv}, und{ }ſo, gerade{ }ſo, bedeutet mir {\pb}viel – vielleicht \label{K_L03012-2v}\edtext{mehr als Sie vermuthen}{\lemma{\textnormal{\emph{mehr als Sie vermuthen}}}\Cendnote{\textnormal{Vgl. XXXX Auszeichnungsfehler: Dokument L03491 nicht gefunden.
               }}}\label{K_L03012-2}. An gewiſſen Stellen{ }ſind mir Thränen geko{\geminationm}en.
               »\label{K_L03012-3v}\edtext{Naja {\dotstwo} weil’s wahr is}{\lemma{\textnormal{\emph{Naja  weil’s wahr is}}}\Cendnote{\textnormal{Vgl. XXXX Auszeichnungsfehler: Dokument L00871 nicht gefunden.
               }}}\label{K_L03012-3}{ }{\dotstwo}«\pend
           
\pstart
           Von Herzen {\\[\baselineskip]}Ihr \spacefill\mbox{Arthur}\pend
           \leftskip=0em{}\selectlanguage{ngerman}\endnumbering\briefempfaengerindex{Salten, Felix@\textsc{Salten, Felix}!zzzSchnitzler, Arthur@\emph{von Arthur Schnitzler}!1908-05-301@{30. 5. 1908}|)be}\mylabel{L03012h}  \newcommand{\dateiname}{L03012}\newcommand{\titel}{Arthur Schnitzler an Felix Salten, 30. 5. 1908}\newcommand{\editorInnen}{Martin Anton Müller und Laura Untner}%% latex-leseansicht-abspann.tex
%% Abspann für die Leseansicht.
%% Der Schalter \ifkorrekturansicht ist bereits durch den Vorspann gesetzt.

%% latex-abspann.tex
%% Gemeinsamer Abspann für Korrekturansicht und Leseansicht.
%% Setzt den Schalter \ifkorrekturansicht voraus (gesetzt in den
%% einbindenden Dateien latex-korrekturansicht-abspann.tex bzw.
%% latex-leseansicht-abspann.tex).
%% ---------------------------------------------------------------

\normalsize

% Das esempio-Environment wird nur in der Leseansicht benötigt
\ifkorrekturansicht\else
\newenvironment{esempio}[3]%
{
    \vspace{1.5ex}
    \rlap{\underline{#1}}
    \par
    \setlength{\parindent}{0cm}
    \nopagebreak
    \leftskip=#2cm
    \rightskip=#3cm
}
{
    \par
}
\fi

\doendnotes{C}
\bigskip
\vfill

\clearpage

\footnotesize

\ifkorrekturansicht
  \lohead{\textsc{register}}
\fi

% theindex-Environment neu definieren ohne reledmac
\makeatletter
\renewenvironment{theindex}{%
  \ifkorrekturansicht
    \section*{\indexname}%
  \else
    \subsubsection*{Index der erwähnten Entitäten}%
  \fi
  \setlength{\parindent}{0pt}%
  \setlength{\parskip}{0pt plus 0.3pt}%
  \let\item\@idxitem
}{%
  \ifkorrekturansicht\clearpage\fi
}
\makeatother

\IfFileExists{\jobname-pw.ind}{\input{\jobname-pw.ind}}{}

% Quellenangabe nur in der Leseansicht
\ifkorrekturansicht\else
% Fallback-Definitionen, falls die .tex-Datei \titel etc. nicht gesetzt hat
\providecommand{\titel}{}
\providecommand{\editorInnen}{}
\providecommand{\dateiname}{\jobname}

\vspace{3cm}

\vfill

\footnotesize
\textsc{Quelle}: \titel. Herausgegeben von {\editorInnen}. In: \emph{Arthur Schnitzler: Briefwechsel mit Autorinnen und Autoren}.
 Digitale Edition, https://schnitzler-briefe.acdh.oeaw.ac.at/{\dateiname}.html (Stand \today)
\fi

\end{document}


