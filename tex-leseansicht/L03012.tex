%% latex-korrekturansicht-vorspann.tex
%% Vorspann für die Korrekturansicht.
%% Lädt die gemeinsame Datei latex-vorspann.tex mit gesetztem Schalter.

\newif\ifkorrekturansicht
\korrekturansichttrue

\input{../tex-inputs/latex-vorspann}


\section[ Arthur Schnitzler an Felix Salten, 30. 5. 1908]{L03012 Arthur Schnitzler an Felix Salten, 30. 5. 1908}
\nopagebreak\mylabel{L03012v}
\rehead{ }\normalsize\beginnumbering\briefempfaengerindex{Salten, Felix@\textsc{Salten, Felix}!zzzSchnitzler, Arthur@\emph{von Arthur Schnitzler}!1908-05-301@{30. 5. 1908}|(be}
\toendnotes[C]{\smallbreak\pagebreak[2]}\Standort{Wienbibliothek im Rathaus, ZPH 1681, 2.1.516.}
\physDesc{Brief, 1 Blatt, 2 Seiten, 321 Zeichen
\newline{}Handschrift: schwarze Tinte, deutsche Kurrent
\newline{}Ordnung: mit Bleistift von unbekannter Hand nummeriert: »\textcolor{gray}{1}8« }
\buchAbdrucke{\weitereDrucke{Arthur Schnitzler: \emph{Briefe 1875–1912}. Frankfurt am Main: \emph{S. Fischer} 1981, S. 578.} }\toendnotes[C]{\smallbreak}
\pstart
           {\pb}\textcolor{gray}{\textbf{Dr. Arthur Schnitzler}}\hfill 30. 5. 908. \pend
           
\pstart
           \textcolor{gray}{\textbf{Wien XVIII. Spoettelgasse 7\oindex{Edmund-Weiss-Gasse 7@\textbf{Edmund-Weiß-Gasse 7}, \emph{Wohngebäude (K.WHS)}|pw}.}}\pend
           \vspace{0.5em}
\pstart
           mein lieber, ich ka{\geminationn} Ihnen gar nicht
               ſagen, \uline{wie} ich mich \label{K_L03012-1v}\edtext{gefreut}{\lemma{\textnormal{\emph{gefreut}}}\Cendnote{\textnormal{Salten\pwindex{Salten, Felix 06.09.1869 – 08.10.1945@\textsc{Salten, Felix} (06.09.1869 – 08.10.1945), \emph{Schriftsteller/Schriftstellerin, Journalist/Journalistin, Chefredakteur/Chefredakteurin}|pwk} hatte die allererste Rezension von
                     \emph{Der Weg ins Freie}\pwindex{Weg ins Freie. Roman@\emph{Der Weg ins Freie. Roman}|pwk} verfasst: Felix Salten\pwindex{Salten, Felix 06.09.1869 – 08.10.1945@\textsc{Salten, Felix} (06.09.1869 – 08.10.1945), \emph{Schriftsteller/Schriftstellerin, Journalist/Journalistin, Chefredakteur/Chefredakteurin}|pwk}: \emph{Schnitzlers Wiener Roman}\pwindex{Schnitzlers Wiener Roman@\emph{Schnitzlers Wiener Roman}|pwk}. In: \emph{Die Zeit}\pwindex{Zeit@\emph{Die Zeit}|pwk}, Jg. 7, Nr. 2042, 30. 5. 1908, Morgenblatt, S. 1–2. Die Rezension verweist auf
                  die Buchausgabe, die ihm aber zu diesem Zeitpunkt nur als Vorabexemplar
                  vorgelegen haben könnte. Wahrscheinlicher ist, dass ihm Schnitzler den Text des 6. und (letzten) Teils des
                  Vorabdrucks in der \emph{Der neuen Rundschau}\pwindex{neue Rundschau@\emph{Die neue Rundschau}|pwk} oder
                  sonst eine Druckfahne zur Verfügung gestellt hatte (vgl. Felix Salten an Arthur Schnitzler, 16. 1. 1908). Schnitzler zeigte sich im \emph{Tagebuch}\pwindex{Tagebuch@\emph{Tagebuch}|pwk} gerührt: »In der Zeit\pwindex{Zeit@\emph{Die Zeit}|pw}{ }Feuilleton\pwindex{Schnitzlers Wiener Roman@\emph{Schnitzlers Wiener Roman}|pwv}{ }Salten\pwindex{Salten, Felix 06.09.1869 – 08.10.1945@\textsc{Salten, Felix} (06.09.1869 – 08.10.1945), \emph{Schriftsteller/Schriftstellerin, Journalist/Journalistin, Chefredakteur/Chefredakteurin}|pw}’s über den Roman\pwindex{Weg ins Freie. Roman@\emph{Der Weg ins Freie. Roman}|pwv}. Sehr schön; fast ergreifend –
                     ohne Einschränkung. – Schrieb ihm.«}}}\label{K_L03012-1} habe. Aber Sie können ſichs
               ja denken. Daſs Sie der Erſte ſind, der ſich vernehmen lieſs\pwindex{Schnitzlers Wiener Roman@\emph{Schnitzlers Wiener Roman}|pwv}, und ſo, gerade ſo, bedeutet mir {\pb}viel – vielleicht \label{K_L03012-2v}\edtext{mehr als Sie vermuthen}{\lemma{\textnormal{\emph{mehr als Sie vermuthen}}}\Cendnote{\textnormal{Vgl. Felix Salten an Arthur Schnitzler, 26. 1. 1908.
               }}}\label{K_L03012-2}. An gewiſſen Stellen ſind mir Thränen geko{\geminationm}en.
               »\label{K_L03012-3v}\edtext{Naja {\dotstwo} weil’s wahr is}{\lemma{\textnormal{\emph{Naja  weil’s wahr is}}}\Cendnote{\textnormal{Vgl. Richard Beer-Hofmann an Arthur Schnitzler, 24. 12. 1898.
               }}}\label{K_L03012-3}{ }{\dotstwo}«\pend
           
\pstart
           Von Herzen {\\[\baselineskip]}Ihr \spacefill\mbox{Arthur}\pend
           \leftskip=0em{}\selectlanguage{ngerman}\endnumbering\briefempfaengerindex{Salten, Felix@\textsc{Salten, Felix}!zzzSchnitzler, Arthur@\emph{von Arthur Schnitzler}!1908-05-301@{30. 5. 1908}|)be}\mylabel{L03012h}  \normalsize

\doendnotes{C}
\bigskip
\vfill

\clearpage

\footnotesize

\lohead{\textsc{register}}

% Definiere theindex-Environment komplett neu ohne reledmac
\makeatletter
\renewenvironment{theindex}{%
  \section*{\indexname}%
  \setlength{\parindent}{0pt}%
  \setlength{\parskip}{0pt plus 0.3pt}%
  \let\item\@idxitem
}{%
  \clearpage
}
\makeatother

\IfFileExists{\jobname-pw.ind}{\input{\jobname-pw.ind}}{}

\end{document}

      