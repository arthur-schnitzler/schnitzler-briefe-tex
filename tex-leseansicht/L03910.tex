%% latex-leseansicht-vorspann.tex
%% Vorspann für die Leseansicht.
%% Lädt die gemeinsame Datei latex-vorspann.tex mit nicht gesetztem Schalter.

\newif\ifkorrekturansicht
\korrekturansichtfalse

\input{../tex-inputs/latex-vorspann}


\section[Arthur Schnitzler an Theodor Herzl, 17. 11. 1894]{L03910 Arthur Schnitzler an Theodor Herzl, 17. 11. 1894}
\nopagebreak\mylabel{L03910v}
\rehead{ }\normalsize\beginnumbering\briefempfaengerindex{Herzl, Theodor@\textsc{Herzl, Theodor}!zzzSchnitzler, Arthur@\emph{von Arthur Schnitzler}!1894-11-172@{17. 11. 1894}|(be}
\toendnotes[C]{\smallbreak\pagebreak[2]}
\correspDesc{Versand  durch Arthur Schnitzler am 17. 11. 1894 in Wien
\newline{}Erhalt  durch Theodor Herzl in Wien}\toendnotes[C]{\smallbreak}
\Standort{Jerusalem, Central Zionist Archives, H1:1924-15.}
\physDesc{,  Blätter,  Seiten
\newline{}Handschrift: , deutsche Kurrent}
\buchAbdrucke{\weitereDrucke{Arthur Schnitzler: \emph{Briefe 1875–1912}. Herausgegeben von Therese Nickl und Heinrich Schnitzler. Frankfurt am Main: \emph{S. Fischer} 1981, S. 237–239.} }\toendnotes[C]{\smallbreak}
\pstart
           {\pb}Wien\oindex{Wien@\textbf{Wien}, \emph{Verwaltungsgebiet}|pw}, 17. Nov. 94.\pend
           
\pstart{}Lieber Freund!\pend\vspace{0.5em}
\pstart
           Mit Ihrem Stück\pwindex{Herzl, Theodor 2.\,5.\,1860 Budapest – 3.\,7.\,1904 Edlach@\textsc{Herzl, Theodor} (2.\,5.\,1860 Budapest – 3.\,7.\,1904 Edlach), \emph{Schriftsteller, Journalist}!neue Ghetto. Schauspiel in vier Acten@\strich\emph{Das neue Ghetto. Schauspiel in vier Acten}|pwv} haben Sie dem
               Theater ein neues Milieu entdeckt und haben eine Reihe von Geſtalten geſchaffen, die
               Athem des Lebens haben. Und \uline{neue} Geſtalten – manche,
               an die man{ }ſich bis jetzt nicht herangetraut hat. Die beſten Figuren{ }ſind die, die
               aus{ }ſich herausreden, ganz naiv; – da haben Sie mit ein paar Strichen glänzend
               gezeichnet; Charlotte\pwindex{Herzl, Theodor 2.\,5.\,1860 Budapest – 3.\,7.\,1904 Edlach@\textsc{Herzl, Theodor} (2.\,5.\,1860 Budapest – 3.\,7.\,1904 Edlach), \emph{Schriftsteller, Journalist}!neue Ghetto. Schauspiel in vier Acten@\strich\emph{Das neue Ghetto. Schauspiel in vier Acten}|pwv}
               z. B. – Wassermann\pwindex{Herzl, Theodor 2.\,5.\,1860 Budapest – 3.\,7.\,1904 Edlach@\textsc{Herzl, Theodor} (2.\,5.\,1860 Budapest – 3.\,7.\,1904 Edlach), \emph{Schriftsteller, Journalist}!neue Ghetto. Schauspiel in vier Acten@\strich\emph{Das neue Ghetto. Schauspiel in vier Acten}|pw} iſt ausgezeichnet; der
               iſt wohl berufen, Ihnen häufig nachge{\pb}dichtet zu werden.
               Dieses aber verfällt gegen Schluſs in den Fehler Ihrer Hauptperſon; – er erläutert{ }ſich. Sie unterſchätzen Ihre Charakteriſirungskunſt – man kennt Herrn \textsc{Wassermann\pwindex{Herzl, Theodor 2.\,5.\,1860 Budapest – 3.\,7.\,1904 Edlach@\textsc{Herzl, Theodor} (2.\,5.\,1860 Budapest – 3.\,7.\,1904 Edlach), \emph{Schriftsteller, Journalist}!neue Ghetto. Schauspiel in vier Acten@\strich\emph{Das neue Ghetto. Schauspiel in vier Acten}|pwv}} längst, bevor er anfängt von{ }ſich zu erzählen. Ueber das Stück\pwindex{Herzl, Theodor 2.\,5.\,1860 Budapest – 3.\,7.\,1904 Edlach@\textsc{Herzl, Theodor} (2.\,5.\,1860 Budapest – 3.\,7.\,1904 Edlach), \emph{Schriftsteller, Journalist}!neue Ghetto. Schauspiel in vier Acten@\strich\emph{Das neue Ghetto. Schauspiel in vier Acten}|pwv} als ganzes iſt etwas ähnliches zu{ }ſagen. Es hat{ }ſoviel echtes Leben und iſt in seiner Entwicklung{ }ſo natürlich, daſs
               Sie auf \uline{kleine Abſichtlichkeiten der Ausführung} wohl
               verzichten dürften, welche die große Abſicht des Stoffes \substVorne{}\textsuperscript{ſchädigen }\substDazwischen{}verwirren\substHinten{}.  {\pb}Am meiſten hab ich in dieſem Sinne gegen den
               Schlußſatz des Stücks\pwindex{Herzl, Theodor 2.\,5.\,1860 Budapest – 3.\,7.\,1904 Edlach@\textsc{Herzl, Theodor} (2.\,5.\,1860 Budapest – 3.\,7.\,1904 Edlach), \emph{Schriftsteller, Journalist}!neue Ghetto. Schauspiel in vier Acten@\strich\emph{Das neue Ghetto. Schauspiel in vier Acten}|pwv}
               einzuwenden, den eigentlichen Schluſsſatz, den der{ }ſterbende \textsc{Jacob Samuel\pwindex{Herzl, Theodor 2.\,5.\,1860 Budapest – 3.\,7.\,1904 Edlach@\textsc{Herzl, Theodor} (2.\,5.\,1860 Budapest – 3.\,7.\,1904 Edlach), \emph{Schriftsteller, Journalist}!neue Ghetto. Schauspiel in vier Acten@\strich\emph{Das neue Ghetto. Schauspiel in vier Acten}|pwv}} zu{ }ſprechen hat. Laſſen Sie ihn lieber wortlos{ }ſterben – dieſer Tod{ }ſagt mehr
               beſſeres, ich glaube{ }ſelbſt, was ganz andres als der Sterbende ſelbſt. Der
               Sterbende ſagt: »Juden, Brüder, man
                  wird euch erſt wieder leben laſſen, wenn ihr zu ſterben wiſſt.\pwindex{Herzl, Theodor 2.\,5.\,1860 Budapest – 3.\,7.\,1904 Edlach@\textsc{Herzl, Theodor} (2.\,5.\,1860 Budapest – 3.\,7.\,1904 Edlach), \emph{Schriftsteller, Journalist}!neue Ghetto. Schauspiel in vier Acten@\strich\emph{Das neue Ghetto. Schauspiel in vier Acten}|pwv}« – Sein Tod
               aber{ }ſpricht: Dieſer arme Teufel und edle Menſch muſs ſich von einem erbärmlichen {\pb}Haderlumpen einfach deshalb niederschießen lassen – weil er
               als Jud geboren iſt! – – Es gab eine Zeit, wo die Juden zu tauſenden auf den
               Scheiterhaufen verbrannt wurden. Sie haben zu ſterben gewußt. Und man hat{ }ſie nicht
               leben laſſen – deswegen. – So fährt Ihr Drama\pwindex{Herzl, Theodor 2.\,5.\,1860 Budapest – 3.\,7.\,1904 Edlach@\textsc{Herzl, Theodor} (2.\,5.\,1860 Budapest – 3.\,7.\,1904 Edlach), \emph{Schriftsteller, Journalist}!neue Ghetto. Schauspiel in vier Acten@\strich\emph{Das neue Ghetto. Schauspiel in vier Acten}|pwv}, nachdem es ſicher u.{ }ſchön ſeinen Weg hingebrauſt
               ist, – auf einem falſchen Geleiſe ein. –\pend
           
\pstart
           – Eine Figur wäre event. noch in das Stück\pwindex{Herzl, Theodor 2.\,5.\,1860 Budapest – 3.\,7.\,1904 Edlach@\textsc{Herzl, Theodor} (2.\,5.\,1860 Budapest – 3.\,7.\,1904 Edlach), \emph{Schriftsteller, Journalist}!neue Ghetto. Schauspiel in vier Acten@\strich\emph{Das neue Ghetto. Schauspiel in vier Acten}|pwv} hineinzuſtellen, die als Gegenſpieler wirkſam wäre:
               ein jüdiſcher Couleur{\pb}ſtudent, der nach 30. Menſuren
               chaſſirt wird, weil er ein Jude iſt. – Eventuell noch ein anderer Student, der
               dem kathol. Geſellenverein\orgindex{Kolpingwerk@Kolpingwerk|pw} angehört und sich au\substVorne{}\textsuperscript{f}\substDazwischen{}s\substHinten{} »Katholizismus« nicht ſchlägt – und daher ſehr verehrt wird! – Und noch eine
               Figur ſcheint mir in dem reichen Bild zu fehlen, das Sie von einer gewiſſen jüd.
               Geſellschaft \strikeout{\textcolor{gray}{×}\-\textcolor{gray}{×}\-\textcolor{gray}{×}\-\textcolor{gray}{×}} entwerfen. – D. i. eine ſympathiſche Frau (oder
               Mädel) Gibt es nemlich auch. Oder es wäre wenigſtens {\pb}zu
               zeigen, wie ein urſprünglich gut veranlagtes Mädel durch Hellmanniſche\pwindex{Herzl, Theodor 2.\,5.\,1860 Budapest – 3.\,7.\,1904 Edlach@\textsc{Herzl, Theodor} (2.\,5.\,1860 Budapest – 3.\,7.\,1904 Edlach), \emph{Schriftsteller, Journalist}!neue Ghetto. Schauspiel in vier Acten@\strich\emph{Das neue Ghetto. Schauspiel in vier Acten}|pwv} Erziehung verkommt.
               Ließe ſich vielleicht gar nicht ſo ſchwer an Hermine\pwindex{Herzl, Theodor 2.\,5.\,1860 Budapest – 3.\,7.\,1904 Edlach@\textsc{Herzl, Theodor} (2.\,5.\,1860 Budapest – 3.\,7.\,1904 Edlach), \emph{Schriftsteller, Journalist}!neue Ghetto. Schauspiel in vier Acten@\strich\emph{Das neue Ghetto. Schauspiel in vier Acten}|pwv} zeigen, die ſcharf aber doch ein bischen outrirt
               gezeichnet iſt. Man begreift gar nicht, daſs ein ſo hochſtehender Menſch wie Jacob\pwindex{Herzl, Theodor 2.\,5.\,1860 Budapest – 3.\,7.\,1904 Edlach@\textsc{Herzl, Theodor} (2.\,5.\,1860 Budapest – 3.\,7.\,1904 Edlach), \emph{Schriftsteller, Journalist}!neue Ghetto. Schauspiel in vier Acten@\strich\emph{Das neue Ghetto. Schauspiel in vier Acten}|pwv} ihn heiratet. Das wäre
               dann gleich motivirt, wenn die guten Züge noch an ihr zu entdecken wären.\pend
           
\pstart
           – Ganz meiſterhaft ſind die {\pb}alten \textsc{Samuel\pwindex{Herzl, Theodor 2.\,5.\,1860 Budapest – 3.\,7.\,1904 Edlach@\textsc{Herzl, Theodor} (2.\,5.\,1860 Budapest – 3.\,7.\,1904 Edlach), \emph{Schriftsteller, Journalist}!neue Ghetto. Schauspiel in vier Acten@\strich\emph{Das neue Ghetto. Schauspiel in vier Acten}|pwv}}. Nun redet die Frau ein bischen zu gewollt, im 1. Akt\pwindex{Herzl, Theodor 2.\,5.\,1860 Budapest – 3.\,7.\,1904 Edlach@\textsc{Herzl, Theodor} (2.\,5.\,1860 Budapest – 3.\,7.\,1904 Edlach), \emph{Schriftsteller, Journalist}!neue Ghetto. Schauspiel in vier Acten@\strich\emph{Das neue Ghetto. Schauspiel in vier Acten}|pwv} beſonders. \textsc{Wurzlechner\pwindex{Herzl, Theodor 2.\,5.\,1860 Budapest – 3.\,7.\,1904 Edlach@\textsc{Herzl, Theodor} (2.\,5.\,1860 Budapest – 3.\,7.\,1904 Edlach), \emph{Schriftsteller, Journalist}!neue Ghetto. Schauspiel in vier Acten@\strich\emph{Das neue Ghetto. Schauspiel in vier Acten}|pwv}} verſtehe ich nicht ganz. Ich glaub, in Ihrem Streben nach Objectivität haben
               Sie ihn geradezu ſympathiſch zu machen verſucht. Aber, glauben Sie mir, er iſt ein
               ganz ordinairer Kerl. Geben Sie ihm wenigſtens ſtärkere Motive, wenn er von Jacob\pwindex{Herzl, Theodor 2.\,5.\,1860 Budapest – 3.\,7.\,1904 Edlach@\textsc{Herzl, Theodor} (2.\,5.\,1860 Budapest – 3.\,7.\,1904 Edlach), \emph{Schriftsteller, Journalist}!neue Ghetto. Schauspiel in vier Acten@\strich\emph{Das neue Ghetto. Schauspiel in vier Acten}|pwv} Abſchied ni{\geminationm}t. Oder
               laſſen Sie dieſe Infamie ſchon im erſten Akt vermuthen. Oder: Jacob\pwindex{Herzl, Theodor 2.\,5.\,1860 Budapest – 3.\,7.\,1904 Edlach@\textsc{Herzl, Theodor} (2.\,5.\,1860 Budapest – 3.\,7.\,1904 Edlach), \emph{Schriftsteller, Journalist}!neue Ghetto. Schauspiel in vier Acten@\strich\emph{Das neue Ghetto. Schauspiel in vier Acten}|pwv}{\pb}ſelbſt merkt, daſs dem \textsc{Wurzlechner\pwindex{Herzl, Theodor 2.\,5.\,1860 Budapest – 3.\,7.\,1904 Edlach@\textsc{Herzl, Theodor} (2.\,5.\,1860 Budapest – 3.\,7.\,1904 Edlach), \emph{Schriftsteller, Journalist}!neue Ghetto. Schauspiel in vier Acten@\strich\emph{Das neue Ghetto. Schauspiel in vier Acten}|pwv}}{ }ſein Verkehr mit den Juden in der Carrière von ihm,{ }ſchadet u. er
               legt es ihm nahe, zu{ }ſcheiden. Oder – was mir am liebſten wäre. \textsc{Jacob\pwindex{Herzl, Theodor 2.\,5.\,1860 Budapest – 3.\,7.\,1904 Edlach@\textsc{Herzl, Theodor} (2.\,5.\,1860 Budapest – 3.\,7.\,1904 Edlach), \emph{Schriftsteller, Journalist}!neue Ghetto. Schauspiel in vier Acten@\strich\emph{Das neue Ghetto. Schauspiel in vier Acten}|pwv}} schmeißt den Kerl wie er sich windet und dreht, einfach hinaus. Als Secundant
               empfehle ich da{\geminationn} für das Duell mit \textsc{Schramm\pwindex{Herzl, Theodor 2.\,5.\,1860 Budapest – 3.\,7.\,1904 Edlach@\textsc{Herzl, Theodor} (2.\,5.\,1860 Budapest – 3.\,7.\,1904 Edlach), \emph{Schriftsteller, Journalist}!neue Ghetto. Schauspiel in vier Acten@\strich\emph{Das neue Ghetto. Schauspiel in vier Acten}|pwv}} den neu zu ſchaffenden Studenten mit den 30 Menſuren. (Er könnte der Neffe
               dieſes köſtlichen \textsc{Wassermann\pwindex{Herzl, Theodor 2.\,5.\,1860 Budapest – 3.\,7.\,1904 Edlach@\textsc{Herzl, Theodor} (2.\,5.\,1860 Budapest – 3.\,7.\,1904 Edlach), \emph{Schriftsteller, Journalist}!neue Ghetto. Schauspiel in vier Acten@\strich\emph{Das neue Ghetto. Schauspiel in vier Acten}|pw}} sein.) –\pend
           
\pstart
           {\pb} – Als zufällige Beiſpiele für die früher erwähnten
               kleinen Abſichtlichkeiten der Ausführung: –\pend
           
\pstart
           Seite 1. \uline{Köchin}: \uline{Halt Juden. Die Juden haben
                     alles Geld\pwindex{Herzl, Theodor 2.\,5.\,1860 Budapest – 3.\,7.\,1904 Edlach@\textsc{Herzl, Theodor} (2.\,5.\,1860 Budapest – 3.\,7.\,1904 Edlach), \emph{Schriftsteller, Journalist}!neue Ghetto. Schauspiel in vier Acten@\strich\emph{Das neue Ghetto. Schauspiel in vier Acten}|pwv}}. »Sind halt Juden« –{ }ſagt dasſelbe; wirkt ſtärker. (das Entrée \textsc{Bichlers\pwindex{Herzl, Theodor 2.\,5.\,1860 Budapest – 3.\,7.\,1904 Edlach@\textsc{Herzl, Theodor} (2.\,5.\,1860 Budapest – 3.\,7.\,1904 Edlach), \emph{Schriftsteller, Journalist}!neue Ghetto. Schauspiel in vier Acten@\strich\emph{Das neue Ghetto. Schauspiel in vier Acten}|pwv}} behagt mir nicht ſehr)\pend
           
\pstart
           Seite 60. \textsc{\uline{Jacob\pwindex{Herzl, Theodor 2.\,5.\,1860 Budapest – 3.\,7.\,1904 Edlach@\textsc{Herzl, Theodor} (2.\,5.\,1860 Budapest – 3.\,7.\,1904 Edlach), \emph{Schriftsteller, Journalist}!neue Ghetto. Schauspiel in vier Acten@\strich\emph{Das neue Ghetto. Schauspiel in vier Acten}|pwv}}}: \uline{Jetzt kannſt du das auch
                     auffaſſen, dſs die Juden Hunde{ }ſind\pwindex{Herzl, Theodor 2.\,5.\,1860 Budapest – 3.\,7.\,1904 Edlach@\textsc{Herzl, Theodor} (2.\,5.\,1860 Budapest – 3.\,7.\,1904 Edlach), \emph{Schriftsteller, Journalist}!neue Ghetto. Schauspiel in vier Acten@\strich\emph{Das neue Ghetto. Schauspiel in vier Acten}|pwv}} – Hier iſt die Abſicht deutlich – bis zur Verſti{\geminationm}ung. – »Auch der Jude mit dem wunden Ehrgefühl\pwindex{Herzl, Theodor 2.\,5.\,1860 Budapest – 3.\,7.\,1904 Edlach@\textsc{Herzl, Theodor} (2.\,5.\,1860 Budapest – 3.\,7.\,1904 Edlach), \emph{Schriftsteller, Journalist}!neue Ghetto. Schauspiel in vier Acten@\strich\emph{Das neue Ghetto. Schauspiel in vier Acten}|pwv}« {\pb}will mir nicht gefallen – geben Sie Ihrem Jacob\pwindex{Herzl, Theodor 2.\,5.\,1860 Budapest – 3.\,7.\,1904 Edlach@\textsc{Herzl, Theodor} (2.\,5.\,1860 Budapest – 3.\,7.\,1904 Edlach), \emph{Schriftsteller, Journalist}!neue Ghetto. Schauspiel in vier Acten@\strich\emph{Das neue Ghetto. Schauspiel in vier Acten}|pwv} etwas mehr innere Freiheit. Der
               Grundgedanke leidet nicht darunter, und die Perſon wird uns{ }ſympathiſcher. Glauben
               Sie nicht? Und hier sah ich es wieder: Die Figur des \uline{Kraftjuden} fehlt mir geradezu in Ihrem Stück\pwindex{Herzl, Theodor 2.\,5.\,1860 Budapest – 3.\,7.\,1904 Edlach@\textsc{Herzl, Theodor} (2.\,5.\,1860 Budapest – 3.\,7.\,1904 Edlach), \emph{Schriftsteller, Journalist}!neue Ghetto. Schauspiel in vier Acten@\strich\emph{Das neue Ghetto. Schauspiel in vier Acten}|pwv}. Es iſt ja nicht wahr, daſs in dem \textsc{Ghetto}, das Sie meinen, alle Juden gedrückt oder i{\geminationn}erlich
               schäbig herumlaufen. Es \uline{gibt}{\pb}andre – und gerade \uline{die}
               werden von den Antiſemiten am tiefſten gehaſſt. Etwas in der Art müßte auch in dem
                  Stück\pwindex{Herzl, Theodor 2.\,5.\,1860 Budapest – 3.\,7.\,1904 Edlach@\textsc{Herzl, Theodor} (2.\,5.\,1860 Budapest – 3.\,7.\,1904 Edlach), \emph{Schriftsteller, Journalist}!neue Ghetto. Schauspiel in vier Acten@\strich\emph{Das neue Ghetto. Schauspiel in vier Acten}|pwv} geſagt werden. IhrStück\pwindex{Herzl, Theodor 2.\,5.\,1860 Budapest – 3.\,7.\,1904 Edlach@\textsc{Herzl, Theodor} (2.\,5.\,1860 Budapest – 3.\,7.\,1904 Edlach), \emph{Schriftsteller, Journalist}!neue Ghetto. Schauspiel in vier Acten@\strich\emph{Das neue Ghetto. Schauspiel in vier Acten}|pwv} ist kühn, – ich
               möchte es auch trotzig haben. Und vor allem laſſen die Ihren Helden nicht ſo
               ergeben ſterben. Ich hab es ſchon anfangs geſagt – jetzt fällt es mir wieder ein
               – Sie ſehn, wie ernſt es mir damit ist! –\pend
           
\pstart
           {\pb}Bühnenwirkſamkeit – ſoweit das vorher zu ſagen iſt – muſs
               Ihr Stück\pwindex{Herzl, Theodor 2.\,5.\,1860 Budapest – 3.\,7.\,1904 Edlach@\textsc{Herzl, Theodor} (2.\,5.\,1860 Budapest – 3.\,7.\,1904 Edlach), \emph{Schriftsteller, Journalist}!neue Ghetto. Schauspiel in vier Acten@\strich\emph{Das neue Ghetto. Schauspiel in vier Acten}|pwv} haben – ob ein
               Theater d\substVorne{}\textsuperscript{i }\substDazwischen{}e\substHinten{}n Muth haben wird, es aufzuführen –? – Doch davon kann{ }ſpäter
               geſprochen werden. Ich freue mich das Stück\pwindex{Herzl, Theodor 2.\,5.\,1860 Budapest – 3.\,7.\,1904 Edlach@\textsc{Herzl, Theodor} (2.\,5.\,1860 Budapest – 3.\,7.\,1904 Edlach), \emph{Schriftsteller, Journalist}!neue Ghetto. Schauspiel in vier Acten@\strich\emph{Das neue Ghetto. Schauspiel in vier Acten}|pwv} (welches Sie doch aufrichtig ein »Trauerſpiel«
               ne{\geminationn}en ſollten), ſehr bald wieder zu leſen, und wenn Sie finden ſollten, daſs von den
               paar Bemerkungen, die ich mir erlaubt habe, einige der Ueberlegung werth ſind, ſo
               werde ich das viel{\pb}leicht in der nächſten Abschrift zu
               erkennen im Stande ſein. Mein herzliches Vergnügen, nach langen Jahren wieder einmal
               ein »\textsc{Originalmanuscript}« von Ihnen durchleſen zu dürfen,
               kann ich Ihnen nicht verſchweigen.\pend
           
\pstart
           Seien Sie vielmals gegrüßt und bedankt.{\\[\baselineskip]}Stets der Ihre{\\[\baselineskip]}\spacefill\mbox{ArthurSchnitzler}\pend
           \leftskip=0em{}\selectlanguage{ngerman}\endnumbering\briefempfaengerindex{Herzl, Theodor@\textsc{Herzl, Theodor}!zzzSchnitzler, Arthur@\emph{von Arthur Schnitzler}!1894-11-172@{17. 11. 1894}|)be}\mylabel{L03910h}
\begin{anhang}
\end{anhang}\newcommand{\dateiname}{L03910}\newcommand{\titel}{Arthur Schnitzler an Theodor Herzl, 17. 11. 1894}\newcommand{\editorInnen}{Herausgegeben von Jahnke, SelmaMüller, Martin Anton}%% latex-leseansicht-abspann.tex
%% Abspann für die Leseansicht.
%% Der Schalter \ifkorrekturansicht ist bereits durch den Vorspann gesetzt.

%% latex-abspann.tex
%% Gemeinsamer Abspann für Korrekturansicht und Leseansicht.
%% Setzt den Schalter \ifkorrekturansicht voraus (gesetzt in den
%% einbindenden Dateien latex-korrekturansicht-abspann.tex bzw.
%% latex-leseansicht-abspann.tex).
%% ---------------------------------------------------------------

\normalsize

% Das esempio-Environment wird nur in der Leseansicht benötigt
\ifkorrekturansicht\else
\newenvironment{esempio}[3]%
{
    \vspace{1.5ex}
    \rlap{\underline{#1}}
    \par
    \setlength{\parindent}{0cm}
    \nopagebreak
    \leftskip=#2cm
    \rightskip=#3cm
}
{
    \par
}
\fi

\doendnotes{C}
\bigskip
\vfill

\clearpage

\footnotesize

\ifkorrekturansicht
  \lohead{\textsc{register}}
\fi

% theindex-Environment neu definieren ohne reledmac
\makeatletter
\renewenvironment{theindex}{%
  \ifkorrekturansicht
    \section*{\indexname}%
  \else
    \subsubsection*{Index der erwähnten Entitäten}%
  \fi
  \setlength{\parindent}{0pt}%
  \setlength{\parskip}{0pt plus 0.3pt}%
  \let\item\@idxitem
}{%
  \ifkorrekturansicht\clearpage\fi
}
\makeatother

\IfFileExists{\jobname-pw.ind}{\input{\jobname-pw.ind}}{}

% Quellenangabe nur in der Leseansicht
\ifkorrekturansicht\else
% Fallback-Definitionen, falls die .tex-Datei \titel etc. nicht gesetzt hat
\providecommand{\titel}{}
\providecommand{\editorInnen}{}
\providecommand{\dateiname}{\jobname}

\vspace{3cm}

\vfill

\footnotesize
\textsc{Quelle}: \titel. Herausgegeben von {\editorInnen}. In: \emph{Arthur Schnitzler: Briefwechsel mit Autorinnen und Autoren}.
 Digitale Edition, https://schnitzler-briefe.acdh.oeaw.ac.at/{\dateiname}.html (Stand \today)
\fi

\end{document}


