\input{../tex-inputs/latex-pdf-vorspann}
\begin{center}
            \textcolor{red}{ENTWURF. ENTZIFFERUNG NOCH NICHT KORREKTURGELESEN}
                      \end{center}
            
               \section[Hermann Bahr an Arthur Schnitzler, 28. 9. 1910]{ Hermann Bahr an Arthur Schnitzler, 28. 9. 1910}\nopagebreak\mylabel{v}\rehead{ }\begin{ledgroupsized}[t]{13cm}\normalsize\beginnumbering\briefempfaengerindex{Schnitzler, Arthur@\textsc{Schnitzler, Arthur}!zzzBahr, Hermann@\emph{von Hermann Bahr}!1910-09-281@{28. 9. 1910}|(be} \toendnotes[C]{\smallbreak\pagebreak[2]} \Standort{CUL, Schnitzler, B 5b.}
\physDesc{Visitenkarte
\newline{}Handschrift Lisa Clarus: schwarze Tinte, lateinische Kurrent\newline{}Ordnung: mit Bleistift von unbekannter Hand
                           nummeriert: »167« }\buchAbdrucke{\weitereDrucke{Hermann Bahr, Arthur Schnitzler: \emph{Briefwechsel, Aufzeichnungen, Dokumente (1891–1931)}. Hg. Kurt Ifkovits und Martin Anton Müller. Göttingen: \emph{Wallstein} 2018, S. 438.} }\toendnotes[C]{\smallbreak}\pstart
           \noindent{}\centering{}{\pb}\textcolor{gray}{\textbf{Hermann
                  Bahr}}\pend
           \pstart
           \noindent{}dankt Dir herzlichst für Deinen lieben Brief – und je eher das Stück\pwindex{Schnitzler, Arthur 15.05.1862 – 21.10.1931@\textsc{Schnitzler, Arthur} (15.05.1862 – 21.10.1931), \emph{Schriftsteller, Mediziner}!junge Medardus. Dramatische Historie in einem Vorspiel und fuenf Aufzuegen1910-10-26@\strich\emph{Der junge Medardus. Dramatische Historie in einem Vorspiel und fünf Aufzügen} {[}1910-10-26{]}|pwv} kommt, desto lieber! Und die schönsten
               Grüsse von Haus zu Haus!\pend
           \pstart
           28. 9. 10.\pend
           \endnumbering\briefempfaengerindex{Schnitzler, Arthur@\textsc{Schnitzler, Arthur}!zzzBahr, Hermann@\emph{von Hermann Bahr}!1910-09-281@{28. 9. 1910}|)be}\mylabel{h}\end{ledgroupsized}  \newcommand{\dateiname}{L01960}\newcommand{\titel}{Hermann Bahr an Arthur Schnitzler, 28. 9. 1910}\newcommand{\editorInnen}{ Kurt Ifkovits,  Martin Anton Müller}\input{../tex-inputs/latex-pdf-abspann}
      