%% latex-leseansicht-vorspann.tex
%% Vorspann für die Leseansicht.
%% Lädt die gemeinsame Datei latex-vorspann.tex mit nicht gesetztem Schalter.

\newif\ifkorrekturansicht
\korrekturansichtfalse

\input{../tex-inputs/latex-vorspann}


\section[Hermann Bahr an Arthur Schnitzler, 28. 9. 1910]{L01960 Hermann Bahr an Arthur Schnitzler, 28. 9. 1910}
\nopagebreak\mylabel{L01960v}
\rehead{ }\normalsize\beginnumbering\briefempfaengerindex{Schnitzler, Arthur@\textsc{Schnitzler, Arthur}!zzzBahr, Hermann@\emph{von Hermann Bahr}!1910-09-281@{28. 9. 1910}|(be}
\toendnotes[C]{\smallbreak\pagebreak[2]}
\correspDesc{Versand  durch Hermann Bahr am 28. 9. 1910 in Wien
\newline{}Erhalt  durch Arthur Schnitzler im Zeitraum [28. 9. 1910
                  – 2. 10. 1910?] in Wien}\toendnotes[C]{\smallbreak}
\Standort{CUL, Schnitzler, B 5b.}
\physDesc{Visitenkarte, 140 Zeichen
\newline{}HandschriftX2  : schwarze Tinte, lateinische Kurrent
\newline{}Ordnung: mit Bleistift von unbekannter Hand nummeriert:
                                    »167« }
\buchAbdrucke{\weitereDrucke{Hermann Bahr, Arthur Schnitzler: \emph{Briefwechsel, Aufzeichnungen, Dokumente (1891–1931)}. Herausgegeben von Kurt Ifkovits und Martin Anton Müller. Göttingen: \emph{Wallstein} 2018, S. 438.} }\toendnotes[C]{\smallbreak}
\pstart
           \noindent{}\centering{}{\pb}\textcolor{gray}{\textbf{Hermann Bahr}}\pend
           
\pstart
           dankt Dir herzlichst für Deinen lieben Brief – und je eher das Stück\pwindex{Schnitzler, Arthur 15.\,5.\,1862 Wien – 21.\,10.\,1931 ebd.@\textsc{Schnitzler, Arthur} (15.\,5.\,1862 Wien – 21.\,10.\,1931 ebd.), \emph{Schriftsteller, Mediziner}!junge Medardus. Dramatische Historie in einem Vorspiel und fünf Aufzügen@\strich\emph{Der junge Medardus. Dramatische Historie in einem Vorspiel und fünf Aufzügen}|pwv} kommt, desto lieber! Und die
               schönsten Grüsse von Haus zu Haus!\pend
           
\pstart
           28. 9. 10.\pend
           \selectlanguage{ngerman}\endnumbering\briefempfaengerindex{Schnitzler, Arthur@\textsc{Schnitzler, Arthur}!zzzBahr, Hermann@\emph{von Hermann Bahr}!1910-09-281@{28. 9. 1910}|)be}\mylabel{L01960h}  \newcommand{\dateiname}{L01960}\newcommand{\titel}{Hermann Bahr an Arthur Schnitzler, 28. 9. 1910}\newcommand{\editorInnen}{Herausgegeben von Martin Anton Müller}%% latex-leseansicht-abspann.tex
%% Abspann für die Leseansicht.
%% Der Schalter \ifkorrekturansicht ist bereits durch den Vorspann gesetzt.

%% latex-abspann.tex
%% Gemeinsamer Abspann für Korrekturansicht und Leseansicht.
%% Setzt den Schalter \ifkorrekturansicht voraus (gesetzt in den
%% einbindenden Dateien latex-korrekturansicht-abspann.tex bzw.
%% latex-leseansicht-abspann.tex).
%% ---------------------------------------------------------------

\normalsize

% Das esempio-Environment wird nur in der Leseansicht benötigt
\ifkorrekturansicht\else
\newenvironment{esempio}[3]%
{
    \vspace{1.5ex}
    \rlap{\underline{#1}}
    \par
    \setlength{\parindent}{0cm}
    \nopagebreak
    \leftskip=#2cm
    \rightskip=#3cm
}
{
    \par
}
\fi

\doendnotes{C}
\bigskip
\vfill

\clearpage

\footnotesize

\ifkorrekturansicht
  \lohead{\textsc{register}}
\fi

% theindex-Environment neu definieren ohne reledmac
\makeatletter
\renewenvironment{theindex}{%
  \ifkorrekturansicht
    \section*{\indexname}%
  \else
    \subsubsection*{Index der erwähnten Entitäten}%
  \fi
  \setlength{\parindent}{0pt}%
  \setlength{\parskip}{0pt plus 0.3pt}%
  \let\item\@idxitem
}{%
  \ifkorrekturansicht\clearpage\fi
}
\makeatother

\IfFileExists{\jobname-pw.ind}{\input{\jobname-pw.ind}}{}

% Quellenangabe nur in der Leseansicht
\ifkorrekturansicht\else
% Fallback-Definitionen, falls die .tex-Datei \titel etc. nicht gesetzt hat
\providecommand{\titel}{}
\providecommand{\editorInnen}{}
\providecommand{\dateiname}{\jobname}

\vspace{3cm}

\vfill

\footnotesize
\textsc{Quelle}: \titel. Herausgegeben von {\editorInnen}. In: \emph{Arthur Schnitzler: Briefwechsel mit Autorinnen und Autoren}.
 Digitale Edition, https://schnitzler-briefe.acdh.oeaw.ac.at/{\dateiname}.html (Stand \today)
\fi

\end{document}


