%% latex-leseansicht-vorspann.tex
%% Vorspann für die Leseansicht.
%% Lädt die gemeinsame Datei latex-vorspann.tex mit nicht gesetztem Schalter.

\newif\ifkorrekturansicht
\korrekturansichtfalse

\input{../tex-inputs/latex-vorspann}


\section[Arthur Schnitzler an Felix Braun, 19. 10. 1924]{L02416 Arthur Schnitzler an Felix Braun, 19. 10. 1924}
\nopagebreak\mylabel{L02416v}
\rehead{ }\normalsize\beginnumbering\briefempfaengerindex{Braun, Felix@\textsc{Braun, Felix}!zzzSchnitzler, Arthur@\emph{von Arthur Schnitzler}!1924-10-191@{19. 10. 1924}|(be}
\toendnotes[C]{\smallbreak\pagebreak[2]}
\correspDesc{Versand  durch Arthur Schnitzler am 19. 10. 1924 in Wien
\newline{}Erhalt  durch Felix Braun im Zeitraum [19. 10. 1924 – 23. 10. 1924?] in Wien}\toendnotes[C]{\smallbreak}
\Standort{Wienbibliothek im Rathaus, H.I.N.-198.046.}
\physDesc{Postkarte, 570 Zeichen
\newline{}Handschrift: schwarze Tinte, lateinische Kurrent
\newline{}Versand: Stempel: »\nobreak{}\oindex{XVIII., Währing@\textbf{XVIII., Währing}, \emph{Verwaltungsgebiet}|pwk}18/1 Wien 110, 20. X. 24, 8\nobreak{}«.  }\toendnotes[C]{\smallbreak}\pstart{}{\pb}\label{T_L02416-1v}\edtext{\textcolor{gray}{\textbf{A. S.}}}{\lemma{\textnormal{\emph{A. S.}}}\Cendnote{\textnormal{ovaler Absenderkleber}}}\label{T_L02416-1}\pend{}\pstart{}\textcolor{gray}{\textbf{WIEN, XVIII.}}\oindex{XVIII., Währing@\textbf{XVIII., Währing}, \emph{Verwaltungsgebiet}|pw}\pend{}\pstart{}\textcolor{gray}{\textbf{STERNWARTESTR. 71}}\oindex{Wien@\textbf{Wien}!XVIII., Währing@\textbf{XVIII., Währing}!Sternwartestraße 71@\textbf{Sternwartestraße 71}, \emph{Wohngebäude}|pw}\pend{}{\bigskip}\pstart{}Hrn Felix Braun\pend{}\pstart{}Wien XIX\oindex{XIX., Döbling@\textbf{XIX., Döbling}, \emph{Verwaltungsgebiet}|pw}\pend{}\pstart{}Sieveringerstr 191\oindex{Wien@\textbf{Wien}!XIX., Döbling@\textbf{XIX., Döbling}!Sieveringer Straße@\textbf{Sieveringer Straße}, \emph{Straße}|pw}\pend{}{\bigskip}\vspace{1em}
\pstart
           \raggedleft{}{\pb}Wien\oindex{Wien@\textbf{Wien}, \emph{Verwaltungsgebiet}|pw}, 19. 10. 924\pend
           \vspace{0.5em}
\pstart
           Verehrter und lieber Herr Felix Braun,  für Ihren schönen Brief
               seien Sie sehr herzlich bedankt, ebenso wie für die beiden Bücher\pwindex{Braun, Felix 4.\,11.\,1885 Wien – 29.\,11.\,1973 Klosterneuburg@\textsc{Braun, Felix} (4.\,11.\,1885 Wien – 29.\,11.\,1973 Klosterneuburg), \emph{Schriftsteller}!Wunderstunden. Drei Erzählungen@\strich\emph{Wunderstunden. Drei Erzählungen}|pwv}\pwindex{Braun, Felix 4.\,11.\,1885 Wien – 29.\,11.\,1973 Klosterneuburg@\textsc{Braun, Felix} (4.\,11.\,1885 Wien – 29.\,11.\,1973 Klosterneuburg), \emph{Schriftsteller}!unsichtbare Gast@\strich\emph{Der unsichtbare Gast}|pwv}, \substVorne{}\textsuperscript{die}\substDazwischen{}von denen\substHinten{} ich eben das eine, die »Wunderstunden\pwindex{Braun, Felix 4.\,11.\,1885 Wien – 29.\,11.\,1973 Klosterneuburg@\textsc{Braun, Felix} (4.\,11.\,1885 Wien – 29.\,11.\,1973 Klosterneuburg), \emph{Schriftsteller}!Wunderstunden. Drei Erzählungen@\strich\emph{Wunderstunden. Drei Erzählungen}|pw}«
               mit innigstem Vergnügen gelesen habe. Wir begegnen einander hoffentlich beide einmal
               wieder – ich wünschte sehr Sie fühlten meine aufrichtige Sympathie auch aus diesen
               paar geschrie{\pb}benen Worten, wie ich mich
               der Ihrigen in wohlthuender Weise gewiſs zu fühlen glaube. Ich drücke Ihnen die Hand
               als Ihr herzlich ergebner\pend
           \pstart \spacefill\mbox{Arthur Schnitzler}\pend{}\selectlanguage{ngerman}\endnumbering\briefempfaengerindex{Braun, Felix@\textsc{Braun, Felix}!zzzSchnitzler, Arthur@\emph{von Arthur Schnitzler}!1924-10-191@{19. 10. 1924}|)be}\mylabel{L02416h}  \newcommand{\dateiname}{L02416}\newcommand{\titel}{Arthur Schnitzler an Felix Braun, 19. 10. 1924}\newcommand{\editorInnen}{Martin Anton Müller und Gerd-Hermann Susen}%% latex-leseansicht-abspann.tex
%% Abspann für die Leseansicht.
%% Der Schalter \ifkorrekturansicht ist bereits durch den Vorspann gesetzt.

%% latex-abspann.tex
%% Gemeinsamer Abspann für Korrekturansicht und Leseansicht.
%% Setzt den Schalter \ifkorrekturansicht voraus (gesetzt in den
%% einbindenden Dateien latex-korrekturansicht-abspann.tex bzw.
%% latex-leseansicht-abspann.tex).
%% ---------------------------------------------------------------

\normalsize

% Das esempio-Environment wird nur in der Leseansicht benötigt
\ifkorrekturansicht\else
\newenvironment{esempio}[3]%
{
    \vspace{1.5ex}
    \rlap{\underline{#1}}
    \par
    \setlength{\parindent}{0cm}
    \nopagebreak
    \leftskip=#2cm
    \rightskip=#3cm
}
{
    \par
}
\fi

\doendnotes{C}
\bigskip
\vfill

\clearpage

\footnotesize

\ifkorrekturansicht
  \lohead{\textsc{register}}
\fi

% theindex-Environment neu definieren ohne reledmac
\makeatletter
\renewenvironment{theindex}{%
  \ifkorrekturansicht
    \section*{\indexname}%
  \else
    \subsubsection*{Index der erwähnten Entitäten}%
  \fi
  \setlength{\parindent}{0pt}%
  \setlength{\parskip}{0pt plus 0.3pt}%
  \let\item\@idxitem
}{%
  \ifkorrekturansicht\clearpage\fi
}
\makeatother

\IfFileExists{\jobname-pw.ind}{\input{\jobname-pw.ind}}{}

% Quellenangabe nur in der Leseansicht
\ifkorrekturansicht\else
% Fallback-Definitionen, falls die .tex-Datei \titel etc. nicht gesetzt hat
\providecommand{\titel}{}
\providecommand{\editorInnen}{}
\providecommand{\dateiname}{\jobname}

\vspace{3cm}

\vfill

\footnotesize
\textsc{Quelle}: \titel. Herausgegeben von {\editorInnen}. In: \emph{Arthur Schnitzler: Briefwechsel mit Autorinnen und Autoren}.
 Digitale Edition, https://schnitzler-briefe.acdh.oeaw.ac.at/{\dateiname}.html (Stand \today)
\fi

\end{document}


