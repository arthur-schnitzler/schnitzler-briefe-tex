%% latex-korrekturansicht-vorspann.tex
%% Vorspann für die Korrekturansicht.
%% Lädt die gemeinsame Datei latex-vorspann.tex mit gesetztem Schalter.

\newif\ifkorrekturansicht
\korrekturansichttrue

\input{../tex-inputs/latex-vorspann}


\section[Arthur Schnitzler an Felix Braun, 19. 10. 1924]{L02416 Arthur Schnitzler an Felix Braun, 19. 10. 1924}
\nopagebreak\mylabel{L02416v}
\rehead{ }\normalsize\beginnumbering\briefempfaengerindex{Braun, Felix@\textsc{Braun, Felix}!zzzSchnitzler, Arthur@\emph{von Arthur Schnitzler}!1924-10-191@{19. 10. 1924}|(be}
\toendnotes[C]{\smallbreak\pagebreak[2]}\Standort{Wienbibliothek im Rathaus, H.I.N.-198.046.}
\physDesc{Postkarte, 570 Zeichen
\newline{}Handschrift: schwarze Tinte, lateinische Kurrent
\newline{}Versand: Stempel: »\nobreak{}\oindex{XVIII., Waehring@\textbf{XVIII., Währing}, \emph{A.ADM3}|pwk}18/1 Wien 110, 20. X. 24, 8\nobreak{}«.  }\toendnotes[C]{\smallbreak}\pstart{}{\pb}\label{T_L02416-1v}\edtext{\textcolor{gray}{\textbf{A. S.}}}{\lemma{\textnormal{\emph{A. S.}}}\Cendnote{\textnormal{ovaler Absenderkleber}}}\label{T_L02416-1}\pend{}\pstart{}\textcolor{gray}{\textbf{WIEN, XVIII.}}\oindex{XVIII., Waehring@\textbf{XVIII., Währing}, \emph{A.ADM3}|pw}\pend{}\pstart{}\textcolor{gray}{\textbf{STERNWARTESTR. 71}}\oindex{Sternwartestrasse 71@\textbf{Sternwartestraße 71}, \emph{Wohngebäude (K.WHS)}|pw}\pend{}{\bigskip}\pstart{}Hrn Felix Braun\pend{}\pstart{}Wien XIX\oindex{XIX., Doebling@\textbf{XIX., Döbling}, \emph{A.ADM3}|pw}\pend{}\pstart{}Sieveringerstr 191\oindex{Sieveringer Strasse@\textbf{Sieveringer Straße}, \emph{Straße (K.STR)}|pw}\pend{}{\bigskip}\vspace{1em}
\pstart
           \raggedleft{}{\pb}Wien\oindex{Wien@\textbf{Wien}, \emph{A.ADM2}|pw}, 19. 10. 924\pend
           \vspace{0.5em}
\pstart
           Verehrter und lieber Herr Felix Braun,  für Ihren schönen Brief
               seien Sie sehr herzlich bedankt, ebenso wie für die beiden Bücher\pwindex{Wunderstunden. Drei Erzaehlungen@\emph{Wunderstunden. Drei Erzählungen}|pwv}\pwindex{unsichtbare Gast@\emph{Der unsichtbare Gast}|pwv}, \substVorne{}\textsuperscript{die}\substDazwischen{}von denen\substHinten{} ich eben das eine, die »Wunderstunden\pwindex{Wunderstunden. Drei Erzaehlungen@\emph{Wunderstunden. Drei Erzählungen}|pw}«
               mit innigstem Vergnügen gelesen habe. Wir begegnen einander hoffentlich beide einmal
               wieder – ich wünschte sehr Sie fühlten meine aufrichtige Sympathie auch aus diesen
               paar geschrie{\pb}benen Worten, wie ich mich
               der Ihrigen in wohlthuender Weise gewiſs zu fühlen glaube. Ich drücke Ihnen die Hand
               als Ihr herzlich ergebner\pend
           \pstart \spacefill\mbox{Arthur Schnitzler}\pend{}\selectlanguage{ngerman}\endnumbering\briefempfaengerindex{Braun, Felix@\textsc{Braun, Felix}!zzzSchnitzler, Arthur@\emph{von Arthur Schnitzler}!1924-10-191@{19. 10. 1924}|)be}\mylabel{L02416h}  \normalsize

\doendnotes{C}
\bigskip
\vfill

\clearpage

\footnotesize

\lohead{\textsc{register}}

% Definiere theindex-Environment komplett neu ohne reledmac
\makeatletter
\renewenvironment{theindex}{%
  \section*{\indexname}%
  \setlength{\parindent}{0pt}%
  \setlength{\parskip}{0pt plus 0.3pt}%
  \let\item\@idxitem
}{%
  \clearpage
}
\makeatother

\IfFileExists{\jobname-pw.ind}{\input{\jobname-pw.ind}}{}

\end{document}

      