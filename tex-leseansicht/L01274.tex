%% latex-korrekturansicht-vorspann.tex
%% Vorspann für die Korrekturansicht.
%% Lädt die gemeinsame Datei latex-vorspann.tex mit gesetztem Schalter.

\newif\ifkorrekturansicht
\korrekturansichttrue

\input{../tex-inputs/latex-vorspann}


\section[Hugo von Hofmannsthal an Arthur Schnitzler, 2. 3. 1903]{L01274 Hugo von Hofmannsthal an Arthur Schnitzler, 2. 3. 1903}
\nopagebreak\mylabel{L01274v}
\rehead{ }\normalsize\beginnumbering\briefempfaengerindex{Schnitzler, Arthur@\textsc{Schnitzler, Arthur}!zzzHofmannsthal, Hugo von@\emph{von Hugo von Hofmannsthal}!1903-03-021@{2. 3. 1903}|(be}
\toendnotes[C]{\smallbreak\pagebreak[2]}\Standort{CUL, Schnitzler, B 43.}
\physDesc{Bildpostkarte, 122 Zeichen
\newline{}Handschrift: schwarze Tinte, lateinische Kurrent
\newline{}Versand: 1) Stempel: »\nobreak{}\oindex{Garmisch-Partenkirchen@\textbf{Garmisch-Partenkirchen}, \emph{P.PPLA3}|pwk}Breslau, 2. 3. 03, 1–2V\nobreak{}«.   2) Stempel: »\nobreak{}\oindex{Berlin@\textbf{Berlin}, \emph{P.PPLC}|pwk}Berlin, 3. 3. 03, Bestellt vom Postamte 6\nobreak{}«. 
\newline{}Schnitzler: mit Bleistift die Jahreszahl ergänzt: »902« 
\newline{}Ordnung: 1) mit Bleistift von unbekannter Hand nummeriert: »\strikeout{225}«  2) mit Bleistift von unbekannter Hand nummeriert: »195«}
\buchAbdrucke{\weitereDrucke{Hugo von Hofmannsthal, Arthur Schnitzler: \emph{Briefwechsel}. Frankfurt am Main: \emph{S. Fischer} 1964, S. 168.} }\toendnotes[C]{\smallbreak}\pstart{}{\pb}Herrn D\textsuperscript{r} Arthur Schnitzler\pend{}\pstart{}Berlin\oindex{Berlin@\textbf{Berlin}, \emph{P.PPLC}|pw}\pend{}\pstart{}Schuhmannstrasse\oindex{Schumannstrasse@\textbf{Schumannstraße}, \emph{Straße (K.STR)}|pw}, Deutsches Theater\oindex{Deutsches Theater Berlin@\textbf{Deutsches Theater Berlin}, \emph{Theater (K.THE)}|pw}\pend{}{\bigskip}
\pstart
           \noindent{}\centering{}{\pb}\textcolor{gray}{\textbf{Breslau\oindex{Breslau@\textbf{Breslau}, \emph{P.PPLA}|pw}{ }Südpark-Restaurant\oindex{Suedpark-Restaurant@\textbf{Südpark-Restaurant}, \emph{Lokal (K.LKL)}|pw}.}}\pend
           \vspace{1em}
\pstart
           \noindent{}{\pb}Gruss an Sie, Brahm\pwindex{Brahm, Otto 05.02.1856 – 28.11.1912@\textsc{Brahm, Otto} (05.02.1856 – 28.11.1912), \emph{Theaterleiter/Theaterleiterin, Regisseur/Regisseurin}|pw} und den wackern alten \label{K_L01274-1v}\edtext{Chiaveluzzi\pwindex{Schleier der Beatrice. Schauspiel in fuenf Akten@\emph{Der Schleier der Beatrice. Schauspiel in fünf Akten}|pwv}}{\lemma{\textnormal{\emph{Chiaveluzzi}}}\Cendnote{\textnormal{Figur aus \emph{Der Schleier der Beatrice}\pwindex{Schleier der Beatrice. Schauspiel in fuenf Akten@\emph{Der Schleier der Beatrice. Schauspiel in fünf Akten}|pwk}; bei der Berlin\oindex{Berlin@\textbf{Berlin}, \emph{P.PPLC}|pwk}er Inszenierung wurde die Rolle von Adolf Kurth\pwindex{Kurth, Adolf @\textsc{Kurth, Adolf}, \emph{Regisseur/Regisseurin, Schauspieler/Schauspielerin}|pwk} gespielt.}}}\label{K_L01274-1}.\pend
           
\pstart
           2 März\pend
           \selectlanguage{ngerman}\endnumbering\briefempfaengerindex{Schnitzler, Arthur@\textsc{Schnitzler, Arthur}!zzzHofmannsthal, Hugo von@\emph{von Hugo von Hofmannsthal}!1903-03-021@{2. 3. 1903}|)be}\mylabel{L01274h}  \normalsize

\doendnotes{C}
\bigskip
\vfill

\clearpage

\footnotesize

\lohead{\textsc{register}}

% Definiere theindex-Environment komplett neu ohne reledmac
\makeatletter
\renewenvironment{theindex}{%
  \section*{\indexname}%
  \setlength{\parindent}{0pt}%
  \setlength{\parskip}{0pt plus 0.3pt}%
  \let\item\@idxitem
}{%
  \clearpage
}
\makeatother

\IfFileExists{\jobname-pw.ind}{\input{\jobname-pw.ind}}{}

\end{document}

      