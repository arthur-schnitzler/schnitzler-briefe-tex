%% latex-korrekturansicht-vorspann.tex
%% Vorspann für die Korrekturansicht.
%% Lädt die gemeinsame Datei latex-vorspann.tex mit gesetztem Schalter.

\newif\ifkorrekturansicht
\korrekturansichttrue

\input{../tex-inputs/latex-vorspann}


\section[Paul Goldmann an Arthur Schnitzler, 14. 6. 1889]{L02639 Paul Goldmann an Arthur Schnitzler, 14. 6. 1889}
\nopagebreak\mylabel{L02639v}
\rehead{ }\normalsize\beginnumbering\briefempfaengerindex{Schnitzler, Arthur@\textsc{Schnitzler, Arthur}!zzzGoldmann, Paul@\emph{von Paul Goldmann}!1889-06-141@{14. 6. 1889}|(be}
\toendnotes[C]{\smallbreak\pagebreak[2]}\Standort{DLA, A:Schnitzler, HS.NZ85.1.3162.}
\physDesc{Brief, 1 Blatt, 3 Seiten, 1212 Zeichen
\newline{}Handschrift: blaue Tinte, deutsche Kurrent
\newline{}Schnitzler: mit rotem Buntstift zwei Unterstreichungen }\toendnotes[C]{\smallbreak}
\pstart
           \centering{}{\pb}\textcolor{gray}{\textbf{\textbf{Adminiſtration: VII.
                           Seidengaſſe 7\oindex{Seidengasse@\textbf{Seidengasse}, \emph{Straße (K.STR)}|pw}} (Jos. Eberle {\kaufmannsund} Co.\orgindex{Josef Eberle Stein-, Buch und Musikaliendruckerei@Josef Eberle Stein-, Buch und Musikaliendruckerei|pw})}}\pend
           
\pstart
           \centering{}\textcolor{gray}{\textbf{An der Schönen Blauen Donau\orgindex{der schoenen blauen Donau@An der schönen blauen Donau|pw}}}\pend
           
\pstart
           \centering{}\textcolor{gray}{\textbf{Chef-Redacteur: Dr. F.
                        Mamroth\pwindex{Mamroth, Fedor 21.02.1851 – 25.06.1907@\textsc{Mamroth, Fedor} (21.02.1851 – 25.06.1907), \emph{Journalist/Journalistin, Kritiker/Kritikerin}|pw}. – Redaction: IX.,
                        Berggaſſe 31\oindex{Berggasse@\textbf{Berggasse}, \emph{Straße (K.STR)}|pw}.}}\pend
           
\pstart
           \raggedleft{}\textcolor{gray}{\textbf{Wien\oindex{Wien@\textbf{Wien}, \emph{A.ADM2}|pw}, den}}{ }14. Juni \textcolor{gray}{\textbf{18}}89.\pend
           
\pstart\center{}Sehr geehrter Herr Doctor!\pend\vspace{0.5em}
\pstart
           Soeben erhalte ich von Herrn \textsc{Boxer\pwindex{Boxer, Oswald 1860-05-29 – 1892-01-26@\textsc{Boxer, Oswald} (1860-05-29 – 1892-01-26), \emph{Journalist/Journalistin}|pw}} die gewünſchte \label{K_L02639-1v}\edtext{Empfehlung}{\lemma{\textnormal{\emph{Empfehlung}}}\Cendnote{\textnormal{Es handelt sich um ein Empfehlungsschreiben
                  für die im Folgenden angesprochene Kontaktaufnahme mit Paul Lindau\pwindex{Lindau, Paul 03.06.1839 – 31.01.1919@\textsc{Lindau, Paul} (03.06.1839 – 31.01.1919), \emph{Schriftsteller/Schriftstellerin, Kritiker/Kritikerin, Theaterleiter/Theaterleiterin}|pwk}. Die erhaltene Korrespondenz Schnitzlers mit Lindau\pwindex{Lindau, Paul 03.06.1839 – 31.01.1919@\textsc{Lindau, Paul} (03.06.1839 – 31.01.1919), \emph{Schriftsteller/Schriftstellerin, Kritiker/Kritikerin, Theaterleiter/Theaterleiterin}|pwk} beginnt 1895.}}}\label{K_L02639-1}. Ich halte es für ſehr günftig,
               daß er ſelbſt es übernommen hat\substVorne{}\textsuperscript{.}\substDazwischen{},\substHinten{} Ihnen dieſe Empfehlung zu geben, da College \textsc{Boxer\pwindex{Boxer, Oswald 1860-05-29 – 1892-01-26@\textsc{Boxer, Oswald} (1860-05-29 – 1892-01-26), \emph{Journalist/Journalistin}|pw}}, wie ich weiß, zu all den Herren der Berlin\oindex{Berlin@\textbf{Berlin}, \emph{P.PPLC}|pw}er Schriftſteller-Welt infolge ſeiner einflußreichen Stellung als Correſpondent\pwindex{Boxer, Oswald 1860-05-29 – 1892-01-26@\textsc{Boxer, Oswald} (1860-05-29 – 1892-01-26), \emph{Journalist/Journalistin}|pwv} dreier großer
                  Wien\oindex{Wien@\textbf{Wien}, \emph{A.ADM2}|pw}er \label{K_L02639-2v}\edtext{Blätter\orgindex{Presse@Die Presse|pwv}}{\lemma{\textnormal{\emph{Blätter}}}\Cendnote{\textnormal{Oswald Boxer\pwindex{Boxer, Oswald 1860-05-29 – 1892-01-26@\textsc{Boxer, Oswald} (1860-05-29 – 1892-01-26), \emph{Journalist/Journalistin}|pwk} arbeitete jedenfalls als Berlin\oindex{Berlin@\textbf{Berlin}, \emph{P.PPLC}|pwk}er Korrespondent der \emph{Presse}\orgindex{Presse@Die Presse|pwk}.}}}\label{K_L02639-2} ſehr gute Beziehungen hat.\pend
           
\pstart
           Wenn ich mir nun erlauben {\pb}darf, Ihnen noch
               weiterhin einen Rath zu geben, ſo geht derſelbe dahin: Überſenden Sie das \label{K_L02639-3v}\edtext{Manuſcript}{\lemma{\textnormal{\emph{Manuſcript}}}\Cendnote{\textnormal{nicht identifiziert}}}\label{K_L02639-3} dem \textsc{Paul Lindau\pwindex{Lindau, Paul 03.06.1839 – 31.01.1919@\textsc{Lindau, Paul} (03.06.1839 – 31.01.1919), \emph{Schriftsteller/Schriftstellerin, Kritiker/Kritikerin, Theaterleiter/Theaterleiterin}|pw}}{ }\uline{bald}, damit er die Sendung erhält, bevor er in’s Bad
               fährt; adreſſiren Sie ferner an ihn direct, \uline{nicht} an
               die Redaction\orgindex{Nord und Sued@Nord und Süd|pwuv};
               nun legen Sie in Ihrem Begleitſchreiben ganz offen den Grund des Empfehlungs-Briefes
               dar: daß \strikeout{es} Ihnen nichts ferner gelegen, als dadurch
               ſein Urtheil beeinfluſſen zu wollen, daß Sie im Gegentheil – was Ihnen, als
                  unbekannte\textcolor{gray}{n} jüngern Litteraten ſonſt vielleicht unmöglich
               geweſen wäre – dadurch nur erreichen wollten, daß Ihr Manuscript von ihm \uline{geleſen} werde.\pend
           
\pstart
           Die \label{K_L02639-4v}\edtext{Wärterin\pwindex{Waerterin]@\emph{[Die Wärterin]}|pw}}{\lemma{\textnormal{\emph{Wärterin}}}\Cendnote{\textnormal{Bezug unklar. Eventuell handelt es sich
                  um eine Ausarbeitung der folgenden Notiz: »Die junge Frau bei dem Assistenzarzt des Spitals. Er hat Dienst. Eine
                        Wärterin ruft ihn ab. Ein Selbstmörder ist gebracht worden, sterbend. Sie
                        ist fortgegangen, findet ihren Mann nicht zuhause. Bringt die Photographie
                        ihres Manns ins Spital, frägt den Geliebten: ›Ist’s der?‹ – Ja, es ist der
                        Selbstmörder.{ / }Einakter: Gespräch der Bedienerin mit der Frau. Zurückkehren des
                        Sekundararztes. Er schickt die Frau nach Hause. Der Freund kommt. Oder eine
                        Wärterin kommt: Die Identität ist festgestellt.« (vgl. Arthur Schnitzler: \emph{Entworfenes und
                     Verworfenes. Aus dem Nachlaß}. Herausgegeben von Reinhard Urbach.
                     Frankfurt/Main:
                        \emph{S. Fischer}{ }1977, S. 27.)}}}\label{K_L02639-4} haben Sie hoffentlich ſchon
               herausgeputzt; einen hübſchen, markanten Titel werden Sie wohl noch finden; und dann
                  {\pb}– Glückauf zur \label{K_L02639-5v}\edtext{Fahrt}{\lemma{\textnormal{\emph{Fahrt}}}\Cendnote{\textnormal{nicht
                  ermittelt}}}\label{K_L02639-5}! {\dots}\pend
           
\pstart
           Ich empfehle mich Ihnen Hochachtungsvoll {\\[\baselineskip]}Ihr ergebener {\\[\baselineskip]}\spacefill\mbox{Dr. Paul Goldmann}\pend
           \leftskip=0em{}\selectlanguage{ngerman}\endnumbering\briefempfaengerindex{Schnitzler, Arthur@\textsc{Schnitzler, Arthur}!zzzGoldmann, Paul@\emph{von Paul Goldmann}!1889-06-141@{14. 6. 1889}|)be}\mylabel{L02639h}  \normalsize

\doendnotes{C}
\bigskip
\vfill

\clearpage

\footnotesize

\lohead{\textsc{register}}

% Definiere theindex-Environment komplett neu ohne reledmac
\makeatletter
\renewenvironment{theindex}{%
  \section*{\indexname}%
  \setlength{\parindent}{0pt}%
  \setlength{\parskip}{0pt plus 0.3pt}%
  \let\item\@idxitem
}{%
  \clearpage
}
\makeatother

\IfFileExists{\jobname-pw.ind}{\input{\jobname-pw.ind}}{}

\end{document}

      