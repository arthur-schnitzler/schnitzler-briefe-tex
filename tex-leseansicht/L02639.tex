%% latex-leseansicht-vorspann.tex
%% Vorspann für die Leseansicht.
%% Lädt die gemeinsame Datei latex-vorspann.tex mit nicht gesetztem Schalter.

\newif\ifkorrekturansicht
\korrekturansichtfalse

\input{../tex-inputs/latex-vorspann}


\section[Paul Goldmann an Arthur Schnitzler, 14. 6. 1889]{L02639 Paul Goldmann an Arthur Schnitzler, 14. 6. 1889}
\nopagebreak\mylabel{L02639v}
\rehead{ }\normalsize\beginnumbering\briefempfaengerindex{Schnitzler, Arthur@\textsc{Schnitzler, Arthur}!zzzGoldmann, Paul@\emph{von Paul Goldmann}!1889-06-141@{14. 6. 1889}|(be}
\toendnotes[C]{\smallbreak\pagebreak[2]}
\correspDesc{Versand  durch Paul Goldmann am 14. 6. 1889 in Wien
\newline{}Erhalt  durch Arthur Schnitzler im Zeitraum [14. 6. 1889
                  – 18. 6. 1889?] in Wien}\toendnotes[C]{\smallbreak}
\Standort{DLA, A:Schnitzler, HS.NZ85.1.3162.}
\physDesc{Brief, 1 Blatt, 3 Seiten, 1212 Zeichen
\newline{}Handschrift: blaue Tinte, deutsche Kurrent
\newline{}Schnitzler: mit rotem Buntstift zwei Unterstreichungen }\toendnotes[C]{\smallbreak}
\pstart
           \centering{}{\pb}\textcolor{gray}{\textbf{\textbf{Adminiſtration: VII.
                           Seidengaſſe 7\oindex{Wien@\textbf{Wien}!VII., Neubau@\textbf{VII., Neubau}!Seidengasse@\textbf{Seidengasse}, \emph{Straße}|pw}} (Jos. Eberle {\kaufmannsund} Co.\orgindex{Josef Eberle Stein-, Buch und Musikaliendruckerei@Josef Eberle Stein-, Buch und Musikaliendruckerei|pw})}}\pend
           
\pstart
           \centering{}\textcolor{gray}{\textbf{An der Schönen Blauen Donau\orgindex{der schönen blauen Donau@An der schönen blauen Donau|pw}}}\pend
           
\pstart
           \centering{}\textcolor{gray}{\textbf{Chef-Redacteur: Dr. F.
                        Mamroth\pwindex{Mamroth, Fedor 21.\,2.\,1851 Breslau – 25.\,6.\,1907 Frankfurt am Main@\textsc{Mamroth, Fedor} (21.\,2.\,1851 Breslau – 25.\,6.\,1907 Frankfurt am Main), \emph{Journalist, Kritiker}|pw}. – Redaction: IX.,
                        Berggaſſe 31\oindex{Wien@\textbf{Wien}!IX., Alsergrund@\textbf{IX., Alsergrund}!Berggasse@\textbf{Berggasse}, \emph{Straße}|pw}.}}\pend
           
\pstart
           \raggedleft{}\textcolor{gray}{\textbf{Wien\oindex{Wien@\textbf{Wien}, \emph{Verwaltungsgebiet}|pw}, den}}{ }14. Juni \textcolor{gray}{\textbf{18}}89.\pend
           
\pstart\center{}Sehr geehrter Herr Doctor!\pend\vspace{0.5em}
\pstart
           Soeben erhalte ich von Herrn \textsc{Boxer\pwindex{Boxer, Oswald 29.\,5.\,1860 Wien – 26.\,1.\,1892 São Paulo@\textsc{Boxer, Oswald} (29.\,5.\,1860 Wien – 26.\,1.\,1892 São Paulo), \emph{Journalist}|pw}} die gewünſchte \label{K_L02639-1v}\edtext{Empfehlung}{\lemma{\textnormal{\emph{Empfehlung}}}\Cendnote{\textnormal{Es handelt sich um ein Empfehlungsschreiben
                  für die im Folgenden angesprochene Kontaktaufnahme mit Paul Lindau\pwindex{Lindau, Paul 3.\,6.\,1839 Magdeburg – 31.\,1.\,1919 Berlin@\textsc{Lindau, Paul} (3.\,6.\,1839 Magdeburg – 31.\,1.\,1919 Berlin), \emph{Schriftsteller, Kritiker, Theaterleiter}|pwk}. Die erhaltene Korrespondenz Schnitzlers mit Lindau\pwindex{Lindau, Paul 3.\,6.\,1839 Magdeburg – 31.\,1.\,1919 Berlin@\textsc{Lindau, Paul} (3.\,6.\,1839 Magdeburg – 31.\,1.\,1919 Berlin), \emph{Schriftsteller, Kritiker, Theaterleiter}|pwk} beginnt 1895.}}}\label{K_L02639-1}. Ich halte es für{ }ſehr günftig,
               daß er{ }ſelbſt es übernommen hat\substVorne{}\textsuperscript{.}\substDazwischen{},\substHinten{} Ihnen dieſe Empfehlung zu geben, da College \textsc{Boxer\pwindex{Boxer, Oswald 29.\,5.\,1860 Wien – 26.\,1.\,1892 São Paulo@\textsc{Boxer, Oswald} (29.\,5.\,1860 Wien – 26.\,1.\,1892 São Paulo), \emph{Journalist}|pw}}, wie ich weiß, zu all den Herren der Berlin\oindex{Berlin@\textbf{Berlin}, \emph{Hauptstadt}|pw}er Schriftſteller-Welt infolge{ }ſeiner einflußreichen Stellung als Correſpondent\pwindex{Boxer, Oswald 29.\,5.\,1860 Wien – 26.\,1.\,1892 São Paulo@\textsc{Boxer, Oswald} (29.\,5.\,1860 Wien – 26.\,1.\,1892 São Paulo), \emph{Journalist}|pwv} dreier großer
                  Wien\oindex{Wien@\textbf{Wien}, \emph{Verwaltungsgebiet}|pw}er \label{K_L02639-2v}\edtext{Blätter\orgindex{Presse@Die Presse|pwv}}{\lemma{\textnormal{\emph{Blätter}}}\Cendnote{\textnormal{Oswald Boxer\pwindex{Boxer, Oswald 29.\,5.\,1860 Wien – 26.\,1.\,1892 São Paulo@\textsc{Boxer, Oswald} (29.\,5.\,1860 Wien – 26.\,1.\,1892 São Paulo), \emph{Journalist}|pwk} arbeitete jedenfalls als Berlin\oindex{Berlin@\textbf{Berlin}, \emph{Hauptstadt}|pwk}er Korrespondent der \emph{Presse}\orgindex{Presse@Die Presse|pwk}.}}}\label{K_L02639-2}{ }ſehr gute Beziehungen hat.\pend
           
\pstart
           Wenn ich mir nun erlauben {\pb}darf, Ihnen noch
               weiterhin einen Rath zu geben,{ }ſo geht derſelbe dahin: Überſenden Sie das \label{K_L02639-3v}\edtext{Manuſcript}{\lemma{\textnormal{\emph{Manuscript}}}\Cendnote{\textnormal{nicht identifiziert}}}\label{K_L02639-3} dem \textsc{Paul Lindau\pwindex{Lindau, Paul 3.\,6.\,1839 Magdeburg – 31.\,1.\,1919 Berlin@\textsc{Lindau, Paul} (3.\,6.\,1839 Magdeburg – 31.\,1.\,1919 Berlin), \emph{Schriftsteller, Kritiker, Theaterleiter}|pw}}{ }\uline{bald}, damit er die Sendung erhält, bevor er in’s Bad
               fährt; adreſſiren Sie ferner an ihn direct, \uline{nicht} an
               die Redaction\orgindex{Nord und Süd@Nord und Süd|pwuv};
               nun legen Sie in Ihrem Begleitſchreiben ganz offen den Grund des Empfehlungs-Briefes
               dar: daß \strikeout{es} Ihnen nichts ferner gelegen, als dadurch{ }ſein Urtheil beeinfluſſen zu wollen, daß Sie im Gegentheil – was Ihnen, als
                  unbekannte\textcolor{gray}{n} jüngern Litteraten{ }ſonſt vielleicht unmöglich
               geweſen wäre – dadurch nur erreichen wollten, daß Ihr Manuscript von ihm \uline{geleſen} werde.\pend
           
\pstart
           Die \label{K_L02639-4v}\edtext{Wärterin\pwindex{Schnitzler, Arthur 15.\,5.\,1862 Wien – 21.\,10.\,1931 ebd.@\textsc{Schnitzler, Arthur} (15.\,5.\,1862 Wien – 21.\,10.\,1931 ebd.), \emph{Schriftsteller, Mediziner}!Wärterin]@\strich\emph{[Die Wärterin]}|pw}}{\lemma{\textnormal{\emph{Wärterin}}}\Cendnote{\textnormal{Bezug unklar. Eventuell handelt es sich
                  um eine Ausarbeitung der folgenden Notiz: »Die junge Frau bei dem Assistenzarzt des Spitals. Er hat Dienst. Eine
                        Wärterin ruft ihn ab. Ein Selbstmörder ist gebracht worden, sterbend. Sie
                        ist fortgegangen, findet ihren Mann nicht zuhause. Bringt die Photographie
                        ihres Manns ins Spital, frägt den Geliebten: ›Ist’s der?‹ – Ja, es ist der
                        Selbstmörder.{ / }Einakter: Gespräch der Bedienerin mit der Frau. Zurückkehren des
                        Sekundararztes. Er schickt die Frau nach Hause. Der Freund kommt. Oder eine
                        Wärterin kommt: Die Identität ist festgestellt.« (vgl. Arthur Schnitzler: \emph{Entworfenes und
                     Verworfenes. Aus dem Nachlaß}. Herausgegeben von Reinhard Urbach.
                     Frankfurt/Main:
                        \emph{S. Fischer}{ }1977, S. 27.)}}}\label{K_L02639-4} haben Sie hoffentlich{ }ſchon
               herausgeputzt; einen hübſchen, markanten Titel werden Sie wohl noch finden; und dann
                  {\pb}– Glückauf zur \label{K_L02639-5v}\edtext{Fahrt}{\lemma{\textnormal{\emph{Fahrt}}}\Cendnote{\textnormal{nicht
                  ermittelt}}}\label{K_L02639-5}! {\dots}\pend
           
\pstart
           Ich empfehle mich Ihnen Hochachtungsvoll {\\[\baselineskip]}Ihr ergebener {\\[\baselineskip]}\spacefill\mbox{Dr. Paul Goldmann}\pend
           \leftskip=0em{}\selectlanguage{ngerman}\endnumbering\briefempfaengerindex{Schnitzler, Arthur@\textsc{Schnitzler, Arthur}!zzzGoldmann, Paul@\emph{von Paul Goldmann}!1889-06-141@{14. 6. 1889}|)be}\mylabel{L02639h}  \newcommand{\dateiname}{L02639}\newcommand{\titel}{Paul Goldmann an Arthur Schnitzler, 14. 6. 1889}\newcommand{\editorInnen}{Martin Anton Müller und Laura Untner}%% latex-leseansicht-abspann.tex
%% Abspann für die Leseansicht.
%% Der Schalter \ifkorrekturansicht ist bereits durch den Vorspann gesetzt.

%% latex-abspann.tex
%% Gemeinsamer Abspann für Korrekturansicht und Leseansicht.
%% Setzt den Schalter \ifkorrekturansicht voraus (gesetzt in den
%% einbindenden Dateien latex-korrekturansicht-abspann.tex bzw.
%% latex-leseansicht-abspann.tex).
%% ---------------------------------------------------------------

\normalsize

% Das esempio-Environment wird nur in der Leseansicht benötigt
\ifkorrekturansicht\else
\newenvironment{esempio}[3]%
{
    \vspace{1.5ex}
    \rlap{\underline{#1}}
    \par
    \setlength{\parindent}{0cm}
    \nopagebreak
    \leftskip=#2cm
    \rightskip=#3cm
}
{
    \par
}
\fi

\doendnotes{C}
\bigskip
\vfill

\clearpage

\footnotesize

\ifkorrekturansicht
  \lohead{\textsc{register}}
\fi

% theindex-Environment neu definieren ohne reledmac
\makeatletter
\renewenvironment{theindex}{%
  \ifkorrekturansicht
    \section*{\indexname}%
  \else
    \subsubsection*{Index der erwähnten Entitäten}%
  \fi
  \setlength{\parindent}{0pt}%
  \setlength{\parskip}{0pt plus 0.3pt}%
  \let\item\@idxitem
}{%
  \ifkorrekturansicht\clearpage\fi
}
\makeatother

\IfFileExists{\jobname-pw.ind}{\input{\jobname-pw.ind}}{}

% Quellenangabe nur in der Leseansicht
\ifkorrekturansicht\else
% Fallback-Definitionen, falls die .tex-Datei \titel etc. nicht gesetzt hat
\providecommand{\titel}{}
\providecommand{\editorInnen}{}
\providecommand{\dateiname}{\jobname}

\vspace{3cm}

\vfill

\footnotesize
\textsc{Quelle}: \titel. Herausgegeben von {\editorInnen}. In: \emph{Arthur Schnitzler: Briefwechsel mit Autorinnen und Autoren}.
 Digitale Edition, https://schnitzler-briefe.acdh.oeaw.ac.at/{\dateiname}.html (Stand \today)
\fi

\end{document}


