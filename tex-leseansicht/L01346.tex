%% latex-leseansicht-vorspann.tex
%% Vorspann für die Leseansicht.
%% Lädt die gemeinsame Datei latex-vorspann.tex mit nicht gesetztem Schalter.

\newif\ifkorrekturansicht
\korrekturansichtfalse

\input{../tex-inputs/latex-vorspann}


         
         \newcommand{\erwaehntePersonen}{Personen: Marc Henry}
         \newcommand{\erwaehnteInstitutionen}{Institutionen: Die elf Scharfrichter, Münchener Post}
         \newcommand{\erwaehnteOrte}{Orte: Arcisstraße, München, Wien}
         \newcommand{\erwaehnteWerke}{Werke: Ein Reigen, Reigen. Zehn Dialoge}
               \section[Franz Blei an Arthur Schnitzler, 25. 11. 1903]{ Franz Blei an Arthur Schnitzler, 25. 11. 1903}\nopagebreak\mylabel{v}\rehead{ }\begin{ledgroupsized}[t]{13cm}\normalsize\beginnumbering \toendnotes[C]{\smallbreak\pagebreak[2]} \Standort{CUL, Schnitzler, B 14.}
\physDesc{Brief, 1 Blatt, 1 Seite
\newline{}Handschrift: schwarze Tinte, lateinische Kurrent
\newline{}Schnitzler: mit rotem Buntstift eine Unterstreichung \newline{}Ordnung: von Schnitzler mit Bleistift beschriftet: »\textsc{Blei}«, von unbekannter Hand mit Bleistift nummeriert:
                                        »1« }\toendnotes[C]{\smallbreak}\pstart
           \raggedleft{}{\pb}München, Arcisstrasse 19\oindex{Arcisstrasse@\textbf{Arcisstraße}|pw}\pend
           \pstart\raggedleft{}Sehr geehrter Herr Doktor,\pend\pstart
           auf meine Anfrage theilt mit die Direktion der 11 Scharfrichter\orgindex{elf Scharfrichter@Die elf Scharfrichter|pw} mit, dass die Tantièmen für den vierzehnmal
                    gespielten Dialog\pwindex{Schnitzler, Arthur 15.05.1862 – 21.10.1931@\textsc{Schnitzler, Arthur} (15.05.1862 – 21.10.1931), \emph{Schriftsteller, Mediziner}!Reigen. Zehn Dialoge1900@\strich\emph{Reigen. Zehn Dialoge} {[}1900{]}|pwv} 82 Mark
                    betragen. Direktor Henry\pwindex{Henry, Marc 02.04.1873 – 24.09.1943@\textsc{Henry, Marc} (02.04.1873 – 24.09.1943), \emph{Schriftsteller, Theaterleiter, Kabarettist}|pw} wird Ihnen den
                    Betrag \label{K_L01346_1v}\edtext{am 10. December}{\lemma{\textnormal{\emph{am 10. December}}}\Cendnote{\textnormal{An dem Tag sollten die \emph{11 Scharfrichter}\orgindex{elf Scharfrichter@Die elf Scharfrichter|pwk} in Wien\oindex{Wien@\textbf{Wien}|pwk} auftreten.}}}\label{K_L01346_1h} in Wien\oindex{Wien@\textbf{Wien}|pw} zustellen.\pend
           \pstart
           Den beiliegenden Ausschnitt\pwindex{?? Werk@Nicht ermittelte Verfasserinnen und Verfasser!Reigen26. 11. 1903@\emph{Ein Reigen} {[}26. 11. 1903{]}|pwv}
                    finde ich in der heutigen »Münchner Poſt\orgindex{Muenchener Post@Münchener Post|pw}«, er
                    wird Sie interessieren.\pend
           \pstart
           Mit besten Grüssen{\\[\baselineskip]}Ihr ergebener{\\[\baselineskip]}\spacefill\mbox{Franz Blei}\pend
           \leftskip=0em{}\pstart
           25. 11. 1903\pend
           
         
         \endnumbering\mylabel{h}\end{ledgroupsized}  \newcommand{\dateiname}{L01346}\newcommand{\titel}{Franz Blei an Arthur Schnitzler, 25. 11. 1903}\newcommand{\editorInnen}{Martin Anton Müller und Gerd-Hermann Susen}%% latex-leseansicht-abspann.tex
%% Abspann für die Leseansicht.
%% Der Schalter \ifkorrekturansicht ist bereits durch den Vorspann gesetzt.

%% latex-abspann.tex
%% Gemeinsamer Abspann für Korrekturansicht und Leseansicht.
%% Setzt den Schalter \ifkorrekturansicht voraus (gesetzt in den
%% einbindenden Dateien latex-korrekturansicht-abspann.tex bzw.
%% latex-leseansicht-abspann.tex).
%% ---------------------------------------------------------------

\normalsize

% Das esempio-Environment wird nur in der Leseansicht benötigt
\ifkorrekturansicht\else
\newenvironment{esempio}[3]%
{
    \vspace{1.5ex}
    \rlap{\underline{#1}}
    \par
    \setlength{\parindent}{0cm}
    \nopagebreak
    \leftskip=#2cm
    \rightskip=#3cm
}
{
    \par
}
\fi

\doendnotes{C}
\bigskip
\vfill

\clearpage

\footnotesize

\ifkorrekturansicht
  \lohead{\textsc{register}}
\fi

% theindex-Environment neu definieren ohne reledmac
\makeatletter
\renewenvironment{theindex}{%
  \ifkorrekturansicht
    \section*{\indexname}%
  \else
    \subsubsection*{Index der erwähnten Entitäten}%
  \fi
  \setlength{\parindent}{0pt}%
  \setlength{\parskip}{0pt plus 0.3pt}%
  \let\item\@idxitem
}{%
  \ifkorrekturansicht\clearpage\fi
}
\makeatother

\IfFileExists{\jobname-pw.ind}{\input{\jobname-pw.ind}}{}

% Quellenangabe nur in der Leseansicht
\ifkorrekturansicht\else
% Fallback-Definitionen, falls die .tex-Datei \titel etc. nicht gesetzt hat
\providecommand{\titel}{}
\providecommand{\editorInnen}{}
\providecommand{\dateiname}{\jobname}

\vspace{3cm}

\vfill

\footnotesize
\textsc{Quelle}: \titel. Herausgegeben von {\editorInnen}. In: \emph{Arthur Schnitzler: Briefwechsel mit Autorinnen und Autoren}.
 Digitale Edition, https://schnitzler-briefe.acdh.oeaw.ac.at/{\dateiname}.html (Stand \today)
\fi

\end{document}


      