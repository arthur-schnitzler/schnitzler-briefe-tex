%% latex-korrekturansicht-vorspann.tex
%% Vorspann für die Korrekturansicht.
%% Lädt die gemeinsame Datei latex-vorspann.tex mit gesetztem Schalter.

\newif\ifkorrekturansicht
\korrekturansichttrue

\input{../tex-inputs/latex-vorspann}


\section[Franz Blei an Arthur Schnitzler, 25. 11. 1903]{L01346 Franz Blei an Arthur Schnitzler, 25. 11. 1903}
\nopagebreak\mylabel{L01346v}
\rehead{ }\normalsize\beginnumbering\briefempfaengerindex{Schnitzler, Arthur@\textsc{Schnitzler, Arthur}!zzzBlei, Franz@\emph{von Franz Blei}!1903-11-251@{25. 11. 1903}|(be}
\toendnotes[C]{\smallbreak\pagebreak[2]}\Standort{CUL, Schnitzler, B 14.}
\physDesc{Brief, 1 Blatt, 1 Seite, 402 Zeichen
\newline{}Handschrift: schwarze Tinte, lateinische Kurrent
\newline{}Schnitzler: mit rotem Buntstift eine Unterstreichung 
\newline{}Ordnung: von Schnitzler mit Bleistift beschriftet: »\textsc{Blei}«, von unbekannter Hand mit Bleistift nummeriert:
                                    »1« }\toendnotes[C]{\smallbreak}
\pstart
           \raggedleft{}{\pb}München, Arcisstrasse 19\oindex{Arcisstrasse@\textbf{Arcisstraße}, \emph{Straße (K.STR)}|pw}\pend
           
\pstart\raggedleft{}Sehr geehrter Herr Doktor,\pend\vspace{0.5em}
\pstart
           auf meine Anfrage theilt mit die Direktion der 11 Scharfrichter\orgindex{elf Scharfrichter@Die elf Scharfrichter|pw} mit, dass die Tantièmen für den vierzehnmal gespielten Dialog\pwindex{Reigen. Zehn Dialoge@\emph{Reigen. Zehn Dialoge}|pwv} 82 Mark betragen.
               Direktor Henry\pwindex{Henry, Marc 02.04.1873 – 24.09.1943@\textsc{Henry, Marc} (02.04.1873 – 24.09.1943), \emph{Schriftsteller/Schriftstellerin, Theaterleiter/Theaterleiterin, Kabarettist/Kabarettistin}|pw} wird Ihnen den Betrag \label{K_L01346-1v}\edtext{am 10. December}{\lemma{\textnormal{\emph{am 10. December}}}\Cendnote{\textnormal{An dem Tag sollten die \emph{11 Scharfrichter}\orgindex{elf Scharfrichter@Die elf Scharfrichter|pwk} in Wien\oindex{Wien@\textbf{Wien}, \emph{A.ADM2}|pwk} auftreten.}}}\label{K_L01346-1} in Wien\oindex{Wien@\textbf{Wien}, \emph{A.ADM2}|pw} zustellen.\pend
           
\pstart
           Den beiliegenden Ausschnitt\pwindex{Reigen@\emph{Ein Reigen}|pwv}
               finde ich in der heutigen »Münchner Poſt\orgindex{Muenchener Post@Münchener Post|pw}«, er
               wird Sie interessieren.\pend
           
\pstart
           Mit besten Grüssen{\\[\baselineskip]}Ihr ergebener{\\[\baselineskip]}\spacefill\mbox{Franz Blei}\pend
           \leftskip=0em{}
\pstart
           25. 11. 1903\pend
           \selectlanguage{ngerman}\endnumbering\briefempfaengerindex{Schnitzler, Arthur@\textsc{Schnitzler, Arthur}!zzzBlei, Franz@\emph{von Franz Blei}!1903-11-251@{25. 11. 1903}|)be}\mylabel{L01346h}  \normalsize

\doendnotes{C}
\bigskip
\vfill

\clearpage

\footnotesize

\lohead{\textsc{register}}

% Definiere theindex-Environment komplett neu ohne reledmac
\makeatletter
\renewenvironment{theindex}{%
  \section*{\indexname}%
  \setlength{\parindent}{0pt}%
  \setlength{\parskip}{0pt plus 0.3pt}%
  \let\item\@idxitem
}{%
  \clearpage
}
\makeatother

\IfFileExists{\jobname-pw.ind}{\input{\jobname-pw.ind}}{}

\end{document}

      