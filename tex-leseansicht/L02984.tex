%% latex-korrekturansicht-vorspann.tex
%% Vorspann für die Korrekturansicht.
%% Lädt die gemeinsame Datei latex-vorspann.tex mit gesetztem Schalter.

\newif\ifkorrekturansicht
\korrekturansichttrue

\input{../tex-inputs/latex-vorspann}


\section[ Arthur Schnitzler an Felix Salten, 12. 10. {[}1903{]}]{L02984 Arthur Schnitzler an Felix Salten, 12. 10. {[}1903{]}}
\nopagebreak\mylabel{L02984v}
\rehead{ }\normalsize\beginnumbering\briefempfaengerindex{Salten, Felix@\textsc{Salten, Felix}!zzzSchnitzler, Arthur@\emph{von Arthur Schnitzler}!1903-10-123@{12. 10. {[}1903{]}}|(be}
\toendnotes[C]{\smallbreak\pagebreak[2]}\Standort{Wienbibliothek im Rathaus, ZPH 1681, 2.1.516.}
\physDesc{Brief, 1 Blatt, 3 Seiten, 654 Zeichen
\newline{}Handschrift: Bleistift, deutsche Kurrent
\newline{}Ordnung: mit Bleistift von unbekannter Hand Nummerierung der Doppelseiten des Konvoluts:
                                    »1«–»2« }\toendnotes[C]{\smallbreak}
\pstart
           \raggedleft{}{\pb}Montag Abd 12/10.\pend
           \vspace{0.5em}
\pstart
           lieber, ich werde \label{K_L02984-1v}\edtext{Freitag um 5}{\lemma{\textnormal{\emph{Freitag um 5}}}\Cendnote{\textnormal{Siehe Felix Salten an Arthur Schnitzler, [12. 10. 1903].
               }}}\label{K_L02984-1} gern bei Ihnen ſein. Ihrem Wunſch von einer Discuſſion abzuſehen reſpektire
               ich; mir ſei nur die monologiſche Äußerung geſtattet, daſs ſich in meinen innern
               Beziehungen zu Ihnen nichts geändert hat, daſs es mir wahrhaft leid thut, ſo ſelten
               mit Ihnen zu reden, daſs es {\pb}einen »Kreis«
               überhaupt nicht mehr gibt, und daſs ich nicht nur wünſche, ſondern auch hoffe, \strikeout{daſs} von Herzen hoffe, es werde ſich in unſrem Verkehr
               die Unbefangenheit und Herzlichkeit wieder einſtellen, die – gewiſs nicht durch meine
               Schuld allein – zu ſchwinden begann und die ich – es iſt {\pb}und bleibt ein Monolog, – aufrichtig
               vermiſſe.\pend
           
\pstart
           Ihr {\\[\baselineskip]}\spacefill\mbox{Arthur}\pend
           \leftskip=0em{}\selectlanguage{ngerman}\endnumbering\briefempfaengerindex{Salten, Felix@\textsc{Salten, Felix}!zzzSchnitzler, Arthur@\emph{von Arthur Schnitzler}!1903-10-123@{12. 10. {[}1903{]}}|)be}\mylabel{L02984h}  \normalsize

\doendnotes{C}
\bigskip
\vfill

\clearpage

\footnotesize

\lohead{\textsc{register}}

% Definiere theindex-Environment komplett neu ohne reledmac
\makeatletter
\renewenvironment{theindex}{%
  \section*{\indexname}%
  \setlength{\parindent}{0pt}%
  \setlength{\parskip}{0pt plus 0.3pt}%
  \let\item\@idxitem
}{%
  \clearpage
}
\makeatother

\IfFileExists{\jobname-pw.ind}{\input{\jobname-pw.ind}}{}

\end{document}

      