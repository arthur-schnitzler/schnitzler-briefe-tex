%% latex-leseansicht-vorspann.tex
%% Vorspann für die Leseansicht.
%% Lädt die gemeinsame Datei latex-vorspann.tex mit nicht gesetztem Schalter.

\newif\ifkorrekturansicht
\korrekturansichtfalse

\input{../tex-inputs/latex-vorspann}


\section[Arthur Schnitzler an Hermann Bahr, 26. 10. 1901]{L01183 Arthur Schnitzler an Hermann Bahr, 26. 10. 1901}
\nopagebreak\mylabel{L01183v}
\rehead{ }\normalsize\beginnumbering\briefempfaengerindex{Bahr, Hermann@\textsc{Bahr, Hermann}!zzzSchnitzler, Arthur@\emph{von Arthur Schnitzler}!1901-10-261@{26. 10. 1901}|(be}
\toendnotes[C]{\smallbreak\pagebreak[2]}
\correspDesc{Versand  durch Arthur Schnitzler am 26. 10. 1901 in Wien
\newline{}Erhalt  durch Hermann Bahr im Zeitraum [26. 10. 1901 – 30. 10. 1901?] in Wien}\toendnotes[C]{\smallbreak}
\Standort{TMW, HS AM 37430 Ba.}
\physDesc{Brief, 1 Blatt, 3 Seiten, 636 Zeichen
\newline{}Handschrift: schwarze Tinte, deutsche Kurrent
\newline{}Ordnung: Lochung }
\buchAbdrucke{\weitereDrucke{1) \emph{26. 10. 1901.} In: Arthur Schnitzler: \emph{The Letters of Arthur Schnitzler to Hermann Bahr}. Edited, annotated, and with an introduction, by Donald G. Daviau. Chapel Hill: \emph{The University of North Carolina Press} 1978, S. 72 (University of North Carolina studies in the Germanic languages
                        and literatures, 89).} \weitereDrucke{2) Hermann Bahr, Arthur Schnitzler: \emph{Briefwechsel, Aufzeichnungen, Dokumente (1891–1931)}. Herausgegeben von Kurt Ifkovits und Martin Anton Müller. Göttingen: \emph{Wallstein} 2018, S. 216.} }\toendnotes[C]{\smallbreak}
\pstart{}{\pb}lieber
                  Hermann,\pend\vspace{0.5em}
\pstart
           ich danke dir{ }ſehr für dein neues \label{K_L01183-1v}\edtext{Buch\pwindex{Bahr, Hermann 19.\,7.\,1863 Linz – 15.\,1.\,1934 München@\textsc{Bahr, Hermann} (19.\,7.\,1863 Linz – 15.\,1.\,1934 München), \emph{Schriftsteller, Kritiker}!Wirkung in die Ferne und Anderes@\strich\emph{Wirkung in die Ferne und Anderes}|pwv}}{\lemma{\textnormal{\emph{Buch}}}\Cendnote{\textnormal{Hermann Bahr\pwindex{Bahr, Hermann 19.\,7.\,1863 Linz – 15.\,1.\,1934 München@\textsc{Bahr, Hermann} (19.\,7.\,1863 Linz – 15.\,1.\,1934 München), \emph{Schriftsteller, Kritiker}|pwk}: \emph{Wirkung in die Ferne und Anderes}\pwindex{Bahr, Hermann 19.\,7.\,1863 Linz – 15.\,1.\,1934 München@\textsc{Bahr, Hermann} (19.\,7.\,1863 Linz – 15.\,1.\,1934 München), \emph{Schriftsteller, Kritiker}!Wirkung in die Ferne und Anderes@\strich\emph{Wirkung in die Ferne und Anderes}|pwk}. Wien: \emph{Wiener Verlag}\orgindex{Wiener Verlag@Wiener Verlag|pwk}{ }1902.}}}\label{K_L01183-1}. Die \label{K_L01183-2v}\edtext{Titelnovelle\pwindex{Bahr, Hermann 19.\,7.\,1863 Linz – 15.\,1.\,1934 München@\textsc{Bahr, Hermann} (19.\,7.\,1863 Linz – 15.\,1.\,1934 München), \emph{Schriftsteller, Kritiker}!Wirkung in die Ferne@\strich\emph{Wirkung in die Ferne}|pwv}}{\lemma{\textnormal{\emph{Titelnovelle}}}\Cendnote{\textnormal{\emph{Wirkung in die Ferne}\pwindex{Bahr, Hermann 19.\,7.\,1863 Linz – 15.\,1.\,1934 München@\textsc{Bahr, Hermann} (19.\,7.\,1863 Linz – 15.\,1.\,1934 München), \emph{Schriftsteller, Kritiker}!Wirkung in die Ferne@\strich\emph{Wirkung in die Ferne}|pwk}, zuerst erschienen
                     in: \emph{Neues Wiener Tagblatt}\orgindex{Neues Wiener Tagblatt@Neues Wiener Tagblatt|pwk}, Jg. 34, Nr. 103,
                        15. 4. 1900, S. 79–85.}}}\label{K_L01183-2} hat mich beſonders
               intereſſirt; du haſt vielleicht bemerkt, daſs in der Erzählg des Puppenſpielers\pwindex{Schnitzler, Arthur 15.\,5.\,1862 Wien – 21.\,10.\,1931 ebd.@\textsc{Schnitzler, Arthur} (15.\,5.\,1862 Wien – 21.\,10.\,1931 ebd.), \emph{Schriftsteller, Mediziner}!Puppenspieler. Studie in einem Aufzuge@\strich\emph{Der Puppenspieler. Studie in einem Aufzuge}|pwv} von dem \label{K_L01183-3v}\edtext{Mann in der Eiſenbahn}{\lemma{\textnormal{\emph{Mann in der Eisenbahn}}}\Cendnote{\textnormal{Arthur Schnitzler: \emph{Marionetten. Drei Einakter}\pwindex{Schnitzler, Arthur 15.\,5.\,1862 Wien – 21.\,10.\,1931 ebd.@\textsc{Schnitzler, Arthur} (15.\,5.\,1862 Wien – 21.\,10.\,1931 ebd.), \emph{Schriftsteller, Mediziner}!Marionetten. Drei Einakter@\strich\emph{Marionetten. Drei Einakter}|pwk}. Berlin: \emph{S. Fischer}\orgindex{S. Fischer Verlag@S. Fischer Verlag|pwk}{ }1906, S. 18–19.}}}\label{K_L01183-3} ein ähnliches Thema leicht angerührt
               iſt. In {\pb}dem Geſpräch
                  »\label{K_L01183-4v}\edtext{Räuber u \damage{M}örder\pwindex{Bahr, Hermann 19.\,7.\,1863 Linz – 15.\,1.\,1934 München@\textsc{Bahr, Hermann} (19.\,7.\,1863 Linz – 15.\,1.\,1934 München), \emph{Schriftsteller, Kritiker}!Räuber und Mörder@\strich\emph{Räuber und Mörder}|pw}}{\lemma{\textnormal{\emph{Räuber u Mörder}}}\Cendnote{\textnormal{\emph{Räuber und Mörder}\pwindex{Bahr, Hermann 19.\,7.\,1863 Linz – 15.\,1.\,1934 München@\textsc{Bahr, Hermann} (19.\,7.\,1863 Linz – 15.\,1.\,1934 München), \emph{Schriftsteller, Kritiker}!Räuber und Mörder@\strich\emph{Räuber und Mörder}|pwk}, zuerst erschienen in:
                        \emph{Neues Wiener Tagblatt}\orgindex{Neues Wiener Tagblatt@Neues Wiener Tagblatt|pwk}, Jg. 34, Nr. 151,
                        3. 6. 1900, S. 2–3.}}}\label{K_L01183-4}« erzählſt du ganz flüchtig
               eine Geſchichte, die mir ein geborner Schwank{ }ſcheint: von dem Hofrath, der dem Dieb
               bietet, ihn nicht anzuzeigen. Wäre ich der \label{K_L01183-5v}\edtext{liebe Auguſtin\orgindex{Jung-Wiener Theater zum Lieben Augustin@Jung-Wiener Theater zum Lieben Augustin|pw}}{\lemma{\textnormal{\emph{liebe Augustin}}}\Cendnote{\textnormal{von Salten\pwindex{Salten, Felix 6.\,9.\,1869 Budapest – 8.\,10.\,1945 Zürich@\textsc{Salten, Felix} (6.\,9.\,1869 Budapest – 8.\,10.\,1945 Zürich), \emph{Schriftsteller, Journalist, Chefredakteur}|pwk} geleitetes Kabarett}}}\label{K_L01183-5},{ }ſo redete ich dir zu, die Scene zu{ }ſchreiben. –\pend
           
\pstart
           Manches hab ich{ }ſchon gekannt, und mit Vergnügen wieder {\pb}geleſen. Lieb iſt die
                  \label{K_L01183-6v}\edtext{Pantomime\pwindex{Bahr, Hermann 19.\,7.\,1863 Linz – 15.\,1.\,1934 München@\textsc{Bahr, Hermann} (19.\,7.\,1863 Linz – 15.\,1.\,1934 München), \emph{Schriftsteller, Kritiker}!Pantomime vom braven Manne@\strich\emph{Die Pantomime vom braven Manne}|pw}}{\lemma{\textnormal{\emph{Pantomime}}}\Cendnote{\textnormal{\emph{Die Pantomime vom braven Manne}\pwindex{Bahr, Hermann 19.\,7.\,1863 Linz – 15.\,1.\,1934 München@\textsc{Bahr, Hermann} (19.\,7.\,1863 Linz – 15.\,1.\,1934 München), \emph{Schriftsteller, Kritiker}!Pantomime vom braven Manne@\strich\emph{Die Pantomime vom braven Manne}|pwk}, zuerst
                     erschienen in: \emph{Das Magazin für Litteratur}\pwindex{Magazin für die Literatur des Auslandes@\emph{Magazin für die Literatur des Auslandes}|pwk},
                     Jg. 62, Nr. 6, 11. 2. 1893, Sp. 93–95.}}}\label{K_L01183-6}. Wird{ }ſie wer
                  \label{K_L01183-7v}\edtext{componiren}{\lemma{\textnormal{\emph{componiren}}}\Cendnote{\textnormal{Vgl. XXXX Auszeichnungsfehler: Dokument L02299 nicht gefunden.
               }}}\label{K_L01183-7}?\pend
           
\pstart
           Ich grüß dich herzlich{\\[\baselineskip]}dein{\\[\baselineskip]}\spacefill\mbox{Arthur}\pend
           \leftskip=0em{}
\pstart
           26. X. 901\pend
           \selectlanguage{ngerman}\endnumbering\briefempfaengerindex{Bahr, Hermann@\textsc{Bahr, Hermann}!zzzSchnitzler, Arthur@\emph{von Arthur Schnitzler}!1901-10-261@{26. 10. 1901}|)be}\mylabel{L01183h}  \newcommand{\dateiname}{L01183}\newcommand{\titel}{Arthur Schnitzler an Hermann Bahr, 26. 10. 1901}\newcommand{\editorInnen}{Herausgegeben von Martin Anton Müller}%% latex-leseansicht-abspann.tex
%% Abspann für die Leseansicht.
%% Der Schalter \ifkorrekturansicht ist bereits durch den Vorspann gesetzt.

%% latex-abspann.tex
%% Gemeinsamer Abspann für Korrekturansicht und Leseansicht.
%% Setzt den Schalter \ifkorrekturansicht voraus (gesetzt in den
%% einbindenden Dateien latex-korrekturansicht-abspann.tex bzw.
%% latex-leseansicht-abspann.tex).
%% ---------------------------------------------------------------

\normalsize

% Das esempio-Environment wird nur in der Leseansicht benötigt
\ifkorrekturansicht\else
\newenvironment{esempio}[3]%
{
    \vspace{1.5ex}
    \rlap{\underline{#1}}
    \par
    \setlength{\parindent}{0cm}
    \nopagebreak
    \leftskip=#2cm
    \rightskip=#3cm
}
{
    \par
}
\fi

\doendnotes{C}
\bigskip
\vfill

\clearpage

\footnotesize

\ifkorrekturansicht
  \lohead{\textsc{register}}
\fi

% theindex-Environment neu definieren ohne reledmac
\makeatletter
\renewenvironment{theindex}{%
  \ifkorrekturansicht
    \section*{\indexname}%
  \else
    \subsubsection*{Index der erwähnten Entitäten}%
  \fi
  \setlength{\parindent}{0pt}%
  \setlength{\parskip}{0pt plus 0.3pt}%
  \let\item\@idxitem
}{%
  \ifkorrekturansicht\clearpage\fi
}
\makeatother

\IfFileExists{\jobname-pw.ind}{\input{\jobname-pw.ind}}{}

% Quellenangabe nur in der Leseansicht
\ifkorrekturansicht\else
% Fallback-Definitionen, falls die .tex-Datei \titel etc. nicht gesetzt hat
\providecommand{\titel}{}
\providecommand{\editorInnen}{}
\providecommand{\dateiname}{\jobname}

\vspace{3cm}

\vfill

\footnotesize
\textsc{Quelle}: \titel. Herausgegeben von {\editorInnen}. In: \emph{Arthur Schnitzler: Briefwechsel mit Autorinnen und Autoren}.
 Digitale Edition, https://schnitzler-briefe.acdh.oeaw.ac.at/{\dateiname}.html (Stand \today)
\fi

\end{document}


