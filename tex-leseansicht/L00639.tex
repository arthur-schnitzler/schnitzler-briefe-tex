%% latex-leseansicht-vorspann.tex
%% Vorspann für die Leseansicht.
%% Lädt die gemeinsame Datei latex-vorspann.tex mit nicht gesetztem Schalter.

\newif\ifkorrekturansicht
\korrekturansichtfalse

\input{../tex-inputs/latex-vorspann}


         
         \newcommand{\erwaehntePersonen}{Personen: Richard Beer-Hofmann, Astrid Brandes, Paul Goldmann, Henrik Ibsen, Edith Philipp, William Shakespeare}
         \newcommand{\erwaehnteInstitutionen}{}
         \newcommand{\erwaehnteOrte}{Orte: Havnegade, Kopenhagen, Wien}
         \newcommand{\erwaehnteWerke}{Werke: Anatol, Freiwild. Schauspiel in 3 Akten, William Shakespeare}
               \section[Georg Brandes an Arthur Schnitzler, 16. 1. 1897]{ Georg Brandes an Arthur Schnitzler, 16. 1. 1897}\nopagebreak\mylabel{v}\rehead{ }\begin{ledgroupsized}[t]{13cm}\normalsize\beginnumbering \toendnotes[C]{\smallbreak\pagebreak[2]} \Standort{CUL, Schnitzler, B 17.}
\physDesc{Brief, 2 Blätter, 2 Seiten, maschinelle Abschrift\newline{}Ordnung: Zusammen mit acht anderen gehört der Brief zu den als
                                    »Several originals missing« bezeichneten, die
                                 durch einen Zettel in der Mappe ausgewiesen sind (nur zwei sind
                                 tatsächlich verschollen). }\buchAbdrucke{\weitereDrucke{Georg Brandes, Arthur Schnitzler: \emph{Ein Briefwechsel}. Hg. Kurt Bergel. Bern: \emph{Francke} 1956, S. 60–61.} }\toendnotes[C]{\smallbreak}\pstart
           \raggedleft{}{\pb}Kopenhagen\oindex{Kopenhagen@\textbf{Kopenhagen}|pw}, 16.{ }Jän.{ }1897\pend
           \pstart\center{}Liebster Herr Schnitzler.\pend\pstart
           Sie wissen: Alles was Sie schreiben interessiert mich, deshalb war ich auf Ihr Stück\pwindex{Schnitzler, Arthur 15.05.1862 – 21.10.1931@\textsc{Schnitzler, Arthur} (15.05.1862 – 21.10.1931), \emph{Schriftsteller, Mediziner}!Freiwild. Schauspiel in 3 Akten1896@\strich\emph{Freiwild. Schauspiel in 3 Akten} {[}1896{]}|pwv} gespannt und natürlich es
               hat meine \label{T_L00639_1v}\edtext{Erwartungen}{\lemma{\textnormal{\emph{Erwartungen}}}\Cendnote{\textnormal{Tippfehler:
                  »Erwqrtungen«}}}\label{T_L00639_1h} nicht getäuscht. Es interessiert lebhaft, es
               spannt und hält in Atem bis zum letzten Wort.\pend
           \pstart
           Es mag sein wie Sie sagen, dass es etwas trocken wirkt, d. h. etwas knapp,
               thesenartig, wenn es auch nicht so gefühlt ist. Ich verstehe Sie recht wohl wenn Sie
               sagen, dass die weibliche Hauptfigur einen »Sprung« bekam. Der Ausdruck war mir neu,
               aber die Sache ist mir bekannt. Das ist sogar auch mir einmal geschehen und es macht
               immerhin einen unangenehmen Eindruck, kann auch der Produktion schädlich sein. Ibsen\pwindex{Ibsen, Henrik 20.03.1828 – 23.05.1906@\textsc{Ibsen, Henrik} (20.03.1828 – 23.05.1906), \emph{Schriftsteller}|pw} sagte mir einmal: Ich kenne zuletzt die
               Personen, die ich darstellen werde\strikeout{n}, so genau, dass
               ich bei meinem Mann sogar die zwei Knöpfe sehe hinten an seinem Rock, die er selbst
               nicht sieht{\dots} so lange haben Sie sich mit diesen Personen
               nicht beschäftigt, dass Sie diese zwei Knöpfe gesehen haben. Deshalb sind die
               Gestalten vielleicht nicht rund, nicht stereoskopisch genug. Die Liebe zwischen Paul
               und Anna ist zu knapp behandelt, nicht individuell genug, nur indiciert. Auch scheint
               es mir gewissermassen ein Fehler, dass die vielen so schön und lebendig gezeichneten
               Nebenpersonen – meisterhaft sind sie, und mit so viel Kenntnis und Erfahrung
               hervorgebracht – dass diese also ganz und gar nicht in die Handlung eingreifen. Das
               ist mangelhafte Technik, nicht wahr?\pend
           \pstart
           Alle diese Einwendungen mache ich um mein Renomée als Kritiker nicht ganz
               preiszugeben, denn mein Vergnügen ist nur Sie zu loben. Wir werden alle dümmer, wenn
               man uns lobt, aber wir werden es ohnehin, und es gibt keine angenehmere Weise, dümmer
               zu werden. Deshalb liebe ich selbst so sehr gelobt zu werden. Als ich noch meine
               beiden kleinen Mädchen\pwindex{Philipp, Edith 17.01.1879 – 1968-02-16@\textsc{Philipp, Edith} (17.01.1879 – 1968-02-16)|pwv}\pwindex{Brandes, Astrid 28.03.1880 – 19.11.1890@\textsc{Brandes, Astrid} (28.03.1880 – 19.11.1890)|pwv}
               hatte – {\pb}ich habe eins\pwindex{Brandes, Astrid 28.03.1880 – 19.11.1890@\textsc{Brandes, Astrid} (28.03.1880 – 19.11.1890)|pwv} durch den Tod verloren –
               lernte ich sie auf die Frage: »Wo wird man jeden Tag dümmer?« zu antworten den Markt
               und die Nummer, wo ich damals wohnte, und sie thaten das mit Bravour. Jetzt werde ich
               jeden Tag dümmer in Havnegade\oindex{Havnegade@\textbf{Havnegade}|pw}, obwohl ich mehr
               geschimpft werde als gelobt. Sonderbar, ich hatte Sie mir nach »Anatol\pwindex{Schnitzler, Arthur 15.05.1862 – 21.10.1931@\textsc{Schnitzler, Arthur} (15.05.1862 – 21.10.1931), \emph{Schriftsteller, Mediziner}!Anatol1892-10-29@\strich\emph{Anatol} {[}1892-10-29{]}|pw}« ganz anders vorgestellt, leichtsinnig, frivol,
               leichtlebig. Sie sind es kaum je gewesen, glaub ich jetzt. Sie sind ja sehr, sehr
               ernst, für einen Wien\oindex{Wien@\textbf{Wien}|pw}er sogar unglaublich ernst.\pend
           \pstart
           Ich habe eine demütige Bitte an Sie. Lesen Sie einmal mein fürchterlich dickes Buch\pwindex{Brandes, Georg 04.02.1842 – 19.02.1927@\textsc{Brandes, Georg} (04.02.1842 – 19.02.1927)!William Shakespeare1895 – 1896@\strich\emph{William Shakespeare} {[}1895 – 1896{]}|pwv} über Shakespeare\pwindex{Shakespeare, William 23.4.1564? – 03.05.1616@\textsc{Shakespeare, William} (23.4.1564? – 03.05.1616), \emph{Schauspieler, Dramatiker}|pw} – in dieser grässlichen deutschen Uebersetzung, wo
               alle Musik der Sprache fort ist und der Sinn nur annähernd wiedergegeben und sagen
               Sie zum Vergelt mir \uline{Ihre} Meinung darüber. Ich habe
               dort ein Stück Psychologie kühner Art versucht und die ganze deutsche Kritik hat sich
               mir \uline{überlegen} gefühlt; ich verachte aber diese Kritik
               mehr als sie mich verachtet, und das heisst etwas.\pend
           \pstart
           Ich war sehr glücklich, heute von Herrn Beer
                  Hofmann\pwindex{Beer-Hofmann, Richard 1866-07-11 – 1945-09-26@\textsc{Beer-Hofmann, Richard} (1866-07-11 – 1945-09-26), \emph{Schriftsteller}|pw} Brief zu bekommen, werde ihm sehr schnell schreiben, liebe ihn sehr.
               Sie und er und Goldmann\pwindex{Goldmann, Paul 31.01.1865 – 25.09.1935@\textsc{Goldmann, Paul} (31.01.1865 – 25.09.1935), \emph{Schriftsteller, Journalist}|pw} sind staunenswerth,
               unerhört, \uline{Freunde}. Dass es noch so etwas gibt! Was
               ich derartiges hatte ist längst todt, und ich glaube nicht mehr daran.\pend
           \pstart
           Ihr{\\[\baselineskip]}\spacefill\mbox{Georg Brandes}\pend
           \leftskip=0em{}
         
         \endnumbering\mylabel{h}\end{ledgroupsized}  \newcommand{\dateiname}{L00639}\newcommand{\titel}{Georg Brandes an Arthur Schnitzler, 16. 1. 1897}\newcommand{\editorInnen}{Martin Anton Müller und Gerd-Hermann Susen}%% latex-leseansicht-abspann.tex
%% Abspann für die Leseansicht.
%% Der Schalter \ifkorrekturansicht ist bereits durch den Vorspann gesetzt.

%% latex-abspann.tex
%% Gemeinsamer Abspann für Korrekturansicht und Leseansicht.
%% Setzt den Schalter \ifkorrekturansicht voraus (gesetzt in den
%% einbindenden Dateien latex-korrekturansicht-abspann.tex bzw.
%% latex-leseansicht-abspann.tex).
%% ---------------------------------------------------------------

\normalsize

% Das esempio-Environment wird nur in der Leseansicht benötigt
\ifkorrekturansicht\else
\newenvironment{esempio}[3]%
{
    \vspace{1.5ex}
    \rlap{\underline{#1}}
    \par
    \setlength{\parindent}{0cm}
    \nopagebreak
    \leftskip=#2cm
    \rightskip=#3cm
}
{
    \par
}
\fi

\doendnotes{C}
\bigskip
\vfill

\clearpage

\footnotesize

\ifkorrekturansicht
  \lohead{\textsc{register}}
\fi

% theindex-Environment neu definieren ohne reledmac
\makeatletter
\renewenvironment{theindex}{%
  \ifkorrekturansicht
    \section*{\indexname}%
  \else
    \subsubsection*{Index der erwähnten Entitäten}%
  \fi
  \setlength{\parindent}{0pt}%
  \setlength{\parskip}{0pt plus 0.3pt}%
  \let\item\@idxitem
}{%
  \ifkorrekturansicht\clearpage\fi
}
\makeatother

\IfFileExists{\jobname-pw.ind}{\input{\jobname-pw.ind}}{}

% Quellenangabe nur in der Leseansicht
\ifkorrekturansicht\else
% Fallback-Definitionen, falls die .tex-Datei \titel etc. nicht gesetzt hat
\providecommand{\titel}{}
\providecommand{\editorInnen}{}
\providecommand{\dateiname}{\jobname}

\vspace{3cm}

\vfill

\footnotesize
\textsc{Quelle}: \titel. Herausgegeben von {\editorInnen}. In: \emph{Arthur Schnitzler: Briefwechsel mit Autorinnen und Autoren}.
 Digitale Edition, https://schnitzler-briefe.acdh.oeaw.ac.at/{\dateiname}.html (Stand \today)
\fi

\end{document}


      