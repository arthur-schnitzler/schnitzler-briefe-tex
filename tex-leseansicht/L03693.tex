%% latex-korrekturansicht-vorspann.tex
%% Vorspann für die Korrekturansicht.
%% Lädt die gemeinsame Datei latex-vorspann.tex mit gesetztem Schalter.

\newif\ifkorrekturansicht
\korrekturansichttrue

\input{../tex-inputs/latex-vorspann}


\section[Stefan Zweig an Arthur Schnitzler, 2. 4. 1930]{L03693 Stefan Zweig an Arthur Schnitzler, 2. 4. 1930}
\nopagebreak\mylabel{L03693v}
\rehead{ }\normalsize\beginnumbering\briefempfaengerindex{Schnitzler, Arthur@\textsc{Schnitzler, Arthur}!zzzZweig, Stefan@\emph{von Stefan Zweig}!1930-04-021@{2. 4. 1930}|(be}
\toendnotes[C]{\smallbreak\pagebreak[2]}\Standort{CUL, Schnitzler, B 118.}
\physDesc{Brief, 1 Blatt, 1 Seite, 414 Zeichen
\newline{}Schreibmaschine
\newline{}Handschrift: Bleistift, lateinische Kurrent (\noindent{}Unterschrift, Korrektur)
\newline{}Schnitzler: 1) mit rotem Buntstift beschriftet: »\textsc{Zweig}«  2) mit rotem Buntstift eine Unterstreichung}
\buchAbdrucke{\weitereDrucke{Stefan Zweig: \emph{Briefwechsel mit Hermann Bahr, Sigmund Freud, Rainer Maria
                        Rilke und Arthur Schnitzler}. Frankfurt am Main: \emph{S. Fischer} 1987, S. 449.} }\toendnotes[C]{\smallbreak}
\pstart
           {\pb}\textcolor{gray}{\textbf{SZ}}\hfill \textcolor{gray}{\textbf{KAPUZINERBERG 5\oindex{Paschinger Schloessl@\textbf{Paschinger Schlössl}, \emph{Wohngebäude (K.WHS)}|pw},}}\pend
           
\pstart
           \raggedleft{}\textcolor{gray}{\textbf{SALZBURG\oindex{Salzburg@\textbf{Salzburg}, \emph{A.ADM2}|pw}}}{ }2. April 1930. \pend
           
\pstart{}Sehr verehrter Herr Doktor!\pend\vspace{0.5em}
\pstart
           Ich bin nicht Besitzer Ihrer geheimen Telefonnummer, vielleicht sind Sie so lieb, sie
               meiner Wiener\oindex{Wien@\textbf{Wien}, \emph{A.ADM2}|pw} Adresse (IX., Garnisongasse 10\oindex{Garnisongasse 10@\textbf{Garnisongasse 10}, \emph{Wohngebäude (K.WHS)}|pw}, Tel.Nr. A 26-0-57) anzuvertrauen und
               mir zu sagen, wann ich \substVorne{}\textsuperscript{s}\substDazwischen{}S\substHinten{}ie \label{K_L03693-1v}\edtext{wieder
               einmal sehen}{\lemma{\textnormal{\emph{wieder
               einmal sehen}}}\Cendnote{\textnormal{Laut Schnitzlers{ }\emph{Tagebuch}\pwindex{Tagebuch@\emph{Tagebuch}|pwk} kam es am 6. 4. 1930 zu einem
                  Telefonat und am 7. 4. 1930 zu einer Abendeinladung.}}}\label{K_L03693-1} dürfte; ich bin endlich
               wieder einmal eine Woche in Wien\oindex{Wien@\textbf{Wien}, \emph{A.ADM2}|pw}, um ein wenig die
                  \label{K_L03693-2v}\edtext{Proben\pwindex{Lamm des Armen. Tragikomoedie in drei Akten@\emph{Das Lamm des Armen. Tragikomödie in drei Akten}|pwv}}{\lemma{\textnormal{\emph{Proben}}}\Cendnote{\textnormal{Stefan Zweigs\pwindex{Zweig, Stefan 28.11.1881 – 23.02.1942@\textsc{Zweig, Stefan} (28.11.1881 – 23.02.1942), \emph{Schriftsteller/Schriftstellerin}|pwk} Theaterstück \emph{Das Lamm des Armen}\pwindex{Lamm des Armen. Tragikomoedie in drei Akten@\emph{Das Lamm des Armen. Tragikomödie in drei Akten}|pwk} wurde am 12. 4. 1930 im
                     Wiener\oindex{Wien@\textbf{Wien}, \emph{A.ADM2}|pwk}{ }Burgtheater\oindex{Burgtheater@\textbf{Burgtheater}, \emph{S.THTR}|pwk} erstaufgeführt.}}}\label{K_L03693-2}
               mitanzusehen.\pend
           
\pstart
           Ihr immer getreuer{\\[\baselineskip]}\spacefill\mbox{{[}hs.:{]} Stefan Zweig}\pend
           \leftskip=0em{}
\pstart
           \noindent{}Herrn Dr. Artur Schnitzler{\\}\uline{Wien, XVIII\oindex{XVIII., Waehring@\textbf{XVIII., Währing}, \emph{A.ADM3}|pw}.}\pend
           \selectlanguage{ngerman}\endnumbering\briefempfaengerindex{Schnitzler, Arthur@\textsc{Schnitzler, Arthur}!zzzZweig, Stefan@\emph{von Stefan Zweig}!1930-04-021@{2. 4. 1930}|)be}\mylabel{L03693h}
\begin{anhang}
\end{anhang}\normalsize

\doendnotes{C}
\bigskip
\vfill

\clearpage

\footnotesize

\lohead{\textsc{register}}

% Definiere theindex-Environment komplett neu ohne reledmac
\makeatletter
\renewenvironment{theindex}{%
  \section*{\indexname}%
  \setlength{\parindent}{0pt}%
  \setlength{\parskip}{0pt plus 0.3pt}%
  \let\item\@idxitem
}{%
  \clearpage
}
\makeatother

\IfFileExists{\jobname-pw.ind}{\input{\jobname-pw.ind}}{}

\end{document}

      