%% latex-leseansicht-vorspann.tex
%% Vorspann für die Leseansicht.
%% Lädt die gemeinsame Datei latex-vorspann.tex mit nicht gesetztem Schalter.

\newif\ifkorrekturansicht
\korrekturansichtfalse

\input{../tex-inputs/latex-vorspann}


\section[Friedrich M. Fels an Arthur Schnitzler, {[}17. 5. 1894{]}]{L00326 Friedrich M. Fels an Arthur Schnitzler, {[}17. 5. 1894{]}}
\nopagebreak\mylabel{L00326v}
\rehead{ }\normalsize\beginnumbering\briefempfaengerindex{Schnitzler, Arthur@\textsc{Schnitzler, Arthur}!zzzFels, Friedrich Michael@\emph{von Friedrich Michael Fels}!1894-05-171@{{[}17. 5. 1894{]}}|(be}
\toendnotes[C]{\smallbreak\pagebreak[2]}
\correspDesc{Versand  durch Friedrich M. Fels am [17. 5. 1894] in Wien
\newline{}Erhalt  durch Arthur Schnitzler im Zeitraum [17. 5. 1894
                  – 21. 5. 1894?] in Wien}\toendnotes[C]{\smallbreak}
\Standort{DLA, A:Schnitzler, HS.NZ85.1.2956.}
\physDesc{Brief, 1 Blatt, 2 Seiten, 1652 Zeichen
\newline{}Handschrift: schwarze Tinte, lateinische Kurrent
\newline{}Schnitzler: mit Bleistift datiert »17/5 94« und nummeriert: »12« }\toendnotes[C]{\smallbreak}
\pstart{}{\pb}Lieber Dr. Schnitzler!\pend\vspace{0.5em}
\pstart
           I. Verzeihen Sie mir den unfrankierten Brief; aber we{\geminationn}
               ich mich auf den Kopf stelle, ko{\geminationm}en keine 3 Kr zum
               Vorschein. Ich müsste also höchstens Ihr »Mährchen\pwindex{Schnitzler, Arthur 15.\,5.\,1862 Wien – 21.\,10.\,1931 ebd.@\textsc{Schnitzler, Arthur} (15.\,5.\,1862 Wien – 21.\,10.\,1931 ebd.), \emph{Schriftsteller, Mediziner}!Märchen. Schauspiel in drei Aufzügen@\strich\emph{Das Märchen. Schauspiel in drei Aufzügen}|pw}« zum Antiquar tragen – und da zahlen Sie jedenfalls lieber
               Strafporto. Verzeihen Sie ferner das kaum recht dicke Papier; aber {\dots} Grund wie vorhin.\pend
           
\pstart
           II. Da Sie die Liebenswürdigkeit hatten, \label{K_L00326-1v}\edtext{Beer-Hofma{\geminationn}\pwindex{Beer-Hofmann, Richard 11.\,7.\,1866 Wien – 26.\,9.\,1945 New York City@\textsc{Beer-Hofmann, Richard} (11.\,7.\,1866 Wien – 26.\,9.\,1945 New York City), \emph{Schriftsteller}|pw} zu schreiben}{\lemma{\textnormal{\emph{Beer-Hofmann zu schreiben}}}\Cendnote{\textnormal{Siehe XXXX Auszeichnungsfehler: Dokument L00323 nicht gefunden.}}}\label{K_L00326-1}, haben Sie vielleicht die grössere Liebenswürdigkeit, ihm noch
               einmal zu schreiben. Ganz abgesehen davon, dass ich, im Vertrauen auf ihn, so
               leichtgläubig war, vorgestern ordentlich zu essen und den ganzen von Ihnen erhaltenen
               Gulden aufzubrauchen, dass ich also seit vorgestern gar nichts zum Leben habe, wäre
               es mir wirklich unangenehm und ein Verlust, we{\geminationn} ich
               nicht baldmöglichst in die Kunstausstellung und am Samstag zum Augartenfest\orgindex{Augartenfest@Augartenfest|pw} gehen kö{\geminationn}te. Also bitte, schreiben Sie Beer-Hofma{\geminationn}\pwindex{Beer-Hofmann, Richard 11.\,7.\,1866 Wien – 26.\,9.\,1945 New York City@\textsc{Beer-Hofmann, Richard} (11.\,7.\,1866 Wien – 26.\,9.\,1945 New York City), \emph{Schriftsteller}|pw} nochmals und entschuldigen Sie mir die Mühe, die ich Ihnen verursache. Ich
               wollte Sie heute früh aufsuchen; doch Ihre Betten hingen bereits {\pb}unter dem Fenster, dass Sie kaum zu Hause waren; auch
               wollte die elektrische Klingel durchaus nicht »thun«.\pend
           
\pstart
           III. Um die Annehmlichkeiten meines Lebens voll zu machen, scheint meine Hauswirthin\pwindex{?? [Vermieterin von F. M. Fels] @\textsc{?? [Vermieterin von F. M. Fels]}|pwv} im Sterben zu
               liegen. Offen gestanden, ich fühle kein Mitleid mit dem armen, jungen Weib, viel eher
               ein bischen Neid auf \substVorne{}\textsuperscript{S}\substDazwischen{}s\substHinten{}ie.\pend
           
\pstart
           Bestens grüsst{\\[\baselineskip]}Ihr{\\[\baselineskip]}dankbarergebener{\\[\baselineskip]}\spacefill\mbox{Fels}\pend
           \leftskip=0em{}
\pstart
           \noindent{}Wien XVIII, Exnergasse 3\textsuperscript{III. St. Th. 22}\oindex{Wien@\textbf{Wien}!XVIII., Währing@\textbf{XVIII., Währing}!Krütznergasse@\textbf{Krütznergasse}, \emph{Straße}|pw}\pend
           
\pstart
           N. B. Ich merke jetzt, dass der \uline{letzte} Satz sehr
                  nach Pose ausschaut; aber, nach gründlicher Gewissenserforschung, muss ich sagen,
                  dass ich, als ich ihn niederschrieb; durchaus nicht an Pose gedacht \label{T_L00326-1v}\edtext{habe}{\lemma{\textnormal{\emph{habe}}}\Cendnote{\textnormal{Er schreibt: »haben«.}}}\label{T_L00326-1}. Bitte, von dieser
                  Rechtfertigung Notiz zu nehmen. \spacefill\mbox{F}\pend
           \selectlanguage{ngerman}\endnumbering\briefempfaengerindex{Schnitzler, Arthur@\textsc{Schnitzler, Arthur}!zzzFels, Friedrich Michael@\emph{von Friedrich Michael Fels}!1894-05-171@{{[}17. 5. 1894{]}}|)be}\mylabel{L00326h}  \newcommand{\dateiname}{L00326}\newcommand{\titel}{Friedrich M. Fels an Arthur Schnitzler, [17. 5. 1894]}\newcommand{\editorInnen}{Martin Anton Müller und Gerd-Hermann Susen}%% latex-leseansicht-abspann.tex
%% Abspann für die Leseansicht.
%% Der Schalter \ifkorrekturansicht ist bereits durch den Vorspann gesetzt.

%% latex-abspann.tex
%% Gemeinsamer Abspann für Korrekturansicht und Leseansicht.
%% Setzt den Schalter \ifkorrekturansicht voraus (gesetzt in den
%% einbindenden Dateien latex-korrekturansicht-abspann.tex bzw.
%% latex-leseansicht-abspann.tex).
%% ---------------------------------------------------------------

\normalsize

% Das esempio-Environment wird nur in der Leseansicht benötigt
\ifkorrekturansicht\else
\newenvironment{esempio}[3]%
{
    \vspace{1.5ex}
    \rlap{\underline{#1}}
    \par
    \setlength{\parindent}{0cm}
    \nopagebreak
    \leftskip=#2cm
    \rightskip=#3cm
}
{
    \par
}
\fi

\doendnotes{C}
\bigskip
\vfill

\clearpage

\footnotesize

\ifkorrekturansicht
  \lohead{\textsc{register}}
\fi

% theindex-Environment neu definieren ohne reledmac
\makeatletter
\renewenvironment{theindex}{%
  \ifkorrekturansicht
    \section*{\indexname}%
  \else
    \subsubsection*{Index der erwähnten Entitäten}%
  \fi
  \setlength{\parindent}{0pt}%
  \setlength{\parskip}{0pt plus 0.3pt}%
  \let\item\@idxitem
}{%
  \ifkorrekturansicht\clearpage\fi
}
\makeatother

\IfFileExists{\jobname-pw.ind}{\input{\jobname-pw.ind}}{}

% Quellenangabe nur in der Leseansicht
\ifkorrekturansicht\else
% Fallback-Definitionen, falls die .tex-Datei \titel etc. nicht gesetzt hat
\providecommand{\titel}{}
\providecommand{\editorInnen}{}
\providecommand{\dateiname}{\jobname}

\vspace{3cm}

\vfill

\footnotesize
\textsc{Quelle}: \titel. Herausgegeben von {\editorInnen}. In: \emph{Arthur Schnitzler: Briefwechsel mit Autorinnen und Autoren}.
 Digitale Edition, https://schnitzler-briefe.acdh.oeaw.ac.at/{\dateiname}.html (Stand \today)
\fi

\end{document}


