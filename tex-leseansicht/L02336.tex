%% latex-leseansicht-vorspann.tex
%% Vorspann für die Leseansicht.
%% Lädt die gemeinsame Datei latex-vorspann.tex mit nicht gesetztem Schalter.

\newif\ifkorrekturansicht
\korrekturansichtfalse

\input{../tex-inputs/latex-vorspann}


         
         \renewcommand{\erwaehntePersonen}{Personen: Gertrude von Hofmannsthal, Frieda Pollak}
         \renewcommand{\erwaehnteOrte}{Orte: Rodaun, Stallburggasse, Sternwartestraße, Wien}
         \renewcommand{\erwaehnteWerke}{Werke: Die Schwestern oder Casanova in Spa. Lustspiel in Versen}
               \section[Hugo Hofmannsthal an Arthur Schnitzler, 15. 2. 1920]{ Hugo Hofmannsthal an Arthur Schnitzler, 15. 2. 1920}\nopagebreak\mylabel{v}\rehead{ }\begin{ledgroupsized}[t]{13cm}\normalsize\beginnumbering \toendnotes[C]{\smallbreak\pagebreak[2]} \Standort{CUL, Schnitzler, B 43.}
\physDesc{Postkarte, 346 Zeichen
\newline{}Handschrift: 1) schwarze Tinte, deutsche Kurrent\hspace{1em}2) schwarze Tinte, lateinische Kurrent (\noindent{}Adresse)\hspace{1em}
\newline{}Versand: Stempel: »\nobreak{}\oindex{Rodaun@\textbf{Rodaun}|pwk}Rodaun, 16 2 20, 7–8V\nobreak{}«.  
\newline{}Ordnung: 1) mit Bleistift von Frieda
                                    Pollak\pwindex{Pollak, Frieda 08.12.1881 – 13.07.1937@\textsc{Pollak, Frieda} (08.12.1881 – 13.07.1937), \emph{Sekretärin}|pw} (?) mit dem Buchstaben »A«
                                 (Abgeschrieben/Abschrift) gekennzeichnet  2) mit Bleistift von unbekannter Hand nummeriert: »\strikeout{261}« 3) mit Bleistift von unbekannter Hand nummeriert:
                                    »368«}\buchAbdrucke{\weitereDrucke{Hugo von Hofmannsthal, Arthur Schnitzler: \emph{Briefwechsel}. Hg. Therese Nickl und Heinrich Schnitzler. Frankfurt am Main: \emph{S. Fischer} 1964, S. 291.} }\pstart{}{\pb}Herrn D\textsuperscript{r} Arthur Schnitzler\pend{}\pstart{}Wien\oindex{Wien@\textbf{Wien}|pw}\pend{}\pstart{}XVIII. Sternwartestrasse 71\oindex{Sternwartestrasse@\textbf{Sternwartestraße}|pw}.\pend{}{\bigskip}\pstart
           {\pb}Rodaun\oindex{Rodaun@\textbf{Rodaun}|pw}{\\}15 II. 20\pend
           \pstart{}mein lieber Arthur\pend\pstart
           ich liege ſeit 5 Tagen hier mit rheumatiſcher Grippe. Gerty\pwindex{Hofmannsthal, Gertrude von 16.03.1880 – 09.11.1959@\textsc{Hofmannsthal, Gertrude von} (16.03.1880 – 09.11.1959)|pw}{ }{\pb}liegt anhaltend mit erhöhter
               Temperatur u. geringen Schmerzen in der Stallburggaſſe\oindex{Stallburggasse@\textbf{Stallburggasse}|pw}.\pend
           \pstart
           Freue mich, Sie wiederzuſehen, ſobald alles beſſer.\pend
           \pstart
           Die »Schweſtern\pwindex{Schnitzler, Arthur 15.05.1862 – 21.10.1931@\textsc{Schnitzler, Arthur} (15.05.1862 – 21.10.1931), \emph{Schriftsteller, Mediziner}!Schwestern oder Casanova in Spa. Lustspiel in Versen01. 10. 1919@\strich\emph{Die Schwestern oder Casanova in Spa. Lustspiel in Versen} {[}01. 10. 1919{]}|pw}« machten mir eine unterhaltende
               Stunde.\pend
           \pstart Ihr \spacefill\mbox{Hugo.}\pend{}
         
         \endnumbering\mylabel{h}\end{ledgroupsized}  \newcommand{\dateiname}{L02336}\newcommand{\titel}{Hugo Hofmannsthal an Arthur Schnitzler, 15. 2. 1920}\newcommand{\editorInnen}{Martin Anton Müller und Gerd-Hermann Susen}%% latex-leseansicht-abspann.tex
%% Abspann für die Leseansicht.
%% Der Schalter \ifkorrekturansicht ist bereits durch den Vorspann gesetzt.

%% latex-abspann.tex
%% Gemeinsamer Abspann für Korrekturansicht und Leseansicht.
%% Setzt den Schalter \ifkorrekturansicht voraus (gesetzt in den
%% einbindenden Dateien latex-korrekturansicht-abspann.tex bzw.
%% latex-leseansicht-abspann.tex).
%% ---------------------------------------------------------------

\normalsize

% Das esempio-Environment wird nur in der Leseansicht benötigt
\ifkorrekturansicht\else
\newenvironment{esempio}[3]%
{
    \vspace{1.5ex}
    \rlap{\underline{#1}}
    \par
    \setlength{\parindent}{0cm}
    \nopagebreak
    \leftskip=#2cm
    \rightskip=#3cm
}
{
    \par
}
\fi

\doendnotes{C}
\bigskip
\vfill

\clearpage

\footnotesize

\ifkorrekturansicht
  \lohead{\textsc{register}}
\fi

% theindex-Environment neu definieren ohne reledmac
\makeatletter
\renewenvironment{theindex}{%
  \ifkorrekturansicht
    \section*{\indexname}%
  \else
    \subsubsection*{Index der erwähnten Entitäten}%
  \fi
  \setlength{\parindent}{0pt}%
  \setlength{\parskip}{0pt plus 0.3pt}%
  \let\item\@idxitem
}{%
  \ifkorrekturansicht\clearpage\fi
}
\makeatother

\IfFileExists{\jobname-pw.ind}{\input{\jobname-pw.ind}}{}

% Quellenangabe nur in der Leseansicht
\ifkorrekturansicht\else
% Fallback-Definitionen, falls die .tex-Datei \titel etc. nicht gesetzt hat
\providecommand{\titel}{}
\providecommand{\editorInnen}{}
\providecommand{\dateiname}{\jobname}

\vspace{3cm}

\vfill

\footnotesize
\textsc{Quelle}: \titel. Herausgegeben von {\editorInnen}. In: \emph{Arthur Schnitzler: Briefwechsel mit Autorinnen und Autoren}.
 Digitale Edition, https://schnitzler-briefe.acdh.oeaw.ac.at/{\dateiname}.html (Stand \today)
\fi

\end{document}


      