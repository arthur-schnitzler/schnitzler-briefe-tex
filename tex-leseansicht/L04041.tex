%% latex-leseansicht-vorspann.tex
%% Vorspann für die Leseansicht.
%% Lädt die gemeinsame Datei latex-vorspann.tex mit nicht gesetztem Schalter.

\newif\ifkorrekturansicht
\korrekturansichtfalse

\input{../tex-inputs/latex-vorspann}


\section[Arthur Schnitzler an Gustav Schwarzkopf, {{[}}23. oder 24. 11. 1896?{{]}}]{L04041 Arthur Schnitzler an Gustav Schwarzkopf, {[}23. oder 24. 11. 1896?{]}}
\nopagebreak\mylabel{L04041v}
\rehead{ }\normalsize\beginnumbering\briefempfaengerindex{Schwarzkopf, Gustav@\textsc{Schwarzkopf, Gustav}!zzzSchnitzler, Arthur@\emph{von Arthur Schnitzler}!1896-11-241@{{[}23. oder 24. 11. 1896?{]}}|(be}
\toendnotes[C]{\smallbreak\pagebreak[2]}
\correspDesc{Versand  durch Arthur Schnitzler im Zeitraum [23. oder 24. 11. 1896?] in Wien
\newline{}Erhalt  durch Gustav Schwarzkopf im Zeitraum [23. 11. 1896 – 26. 11. 1896?] in Wien}\toendnotes[C]{\smallbreak}
\Standort{CUL, Schnitzler, B 96.}
\physDesc{Brief, 1 Blatt, 2 Seiten, 245 Zeichen
\newline{}Handschrift: Bleistift, deutsche Kurrent}\toendnotes[C]{\smallbreak}
\pstart{}{\pb}lieber Guſtav,\pend\vspace{0.5em}
\pstart
           beifolgend ein \label{K_L04214-1v}\edtext{Telegramm von Goldmann\pwindex{Goldmann, Paul 31.\,1.\,1865 Breslau – 25.\,9.\,1935 Wien@\textsc{Goldmann, Paul} (31.\,1.\,1865 Breslau – 25.\,9.\,1935 Wien), \emph{Schriftsteller, Journalist}|pw}}{\lemma{\textnormal{\emph{Telegramm von Goldmann}}}\Cendnote{\textnormal{Beilagen nicht erhalten. Geht man davon aus, das beide hier
                  angesprochenen Korrespondenzstücke von Schwarzkopf\pwindex{Schwarzkopf, Gustav 7.\,11.\,1853 Wien – 13.\,11.\,1939 ebd.@\textsc{Schwarzkopf, Gustav} (7.\,11.\,1853 Wien – 13.\,11.\,1939 ebd.), \emph{Schriftsteller}|pwk} an Schnitzler
                  retourniert wurden und in dessen Nachlass überliefert sind, dürfte es sich bei dem
                  Telegramm um diese Sendung handeln: XXXX Auszeichnungsfehler: Dokument L02689 nicht gefunden (vgl. A. S.: \emph{Tagebuch}, 23. 11. 1896). Das Korrespondenzstück wäre demnach }}}\label{K_L04214-1}, ferner ein \label{K_L04214-2v}\edtext{Brief von Blumenthal\pwindex{Blumenthal, Oskar 13.\,3.\,1852 Berlin – 24.\,4.\,1917 ebd.@\textsc{Blumenthal, Oskar} (13.\,3.\,1852 Berlin – 24.\,4.\,1917 ebd.), \emph{Schriftsteller, Journalist, Theaterleiter}|pw}}{\lemma{\textnormal{\emph{Brief von Blumenthal}}}\Cendnote{\textnormal{Vermutlich XXXX Auszeichnungsfehler: Dokument L00623 nicht gefunden, vgl. A. S.: \emph{Tagebuch}, 16. 11. 1896. Entsprechend wäre der 23. oder 24. 11. 1893
                     ein wahrscheinlicher Termin für das vorliegende, undatierte Korrespondenzstück.
                     
                     Im \emph{Tagebuch}\pwindex{Schnitzler, Arthur 15. 5. 1862 Wien – 21. 10. 1931 ebd.@\textsc{Schnitzler, Arthur} (15. 5. 1862 Wien – 21. 10. 1931 ebd.), \emph{Schriftsteller, Mediziner}!Tagebuch@\strich\emph{Tagebuch}|pwk} erwähnt
                  Schnitzler in diesen Tagen keine Begegnungen mit Schwarzkopf\pwindex{Schwarzkopf, Gustav 7.\,11.\,1853 Wien – 13.\,11.\,1939 ebd.@\textsc{Schwarzkopf, Gustav} (7.\,11.\,1853 Wien – 13.\,11.\,1939 ebd.), \emph{Schriftsteller}|pwk}.}}}\label{K_L04214-2}.\pend
           
\pstart
           – Ich ſchau nach 7 etwa bei Ihnen hinauf (bin ſoweit auf dem Weg) bitte
                  \uline{erwarten Sie mich nicht}; treffe ich Sie nicht,{ }ſo ko{\geminationm} ich mor{\pb}gen Vormittag.\pend
           
\pstart
           Herzlichen Gruſs,{\\[\baselineskip]} Ihr{\\[\baselineskip]}\spacefill\mbox{A.}\pend
           \leftskip=0em{}\selectlanguage{ngerman}\endnumbering\briefempfaengerindex{Schwarzkopf, Gustav@\textsc{Schwarzkopf, Gustav}!zzzSchnitzler, Arthur@\emph{von Arthur Schnitzler}!1896-11-231@{{[}23. oder 24. 11. 1896?{]}}|)be}\mylabel{L04041h}
\begin{anhang}
\end{anhang}\newcommand{\dateiname}{L04041}\newcommand{\titel}{Arthur Schnitzler an Gustav Schwarzkopf, [23. oder 24. 11. 1896?]}\newcommand{\editorInnen}{Herausgegeben von Jahnke, SelmaMüller, Martin Anton}%% latex-leseansicht-abspann.tex
%% Abspann für die Leseansicht.
%% Der Schalter \ifkorrekturansicht ist bereits durch den Vorspann gesetzt.

%% latex-abspann.tex
%% Gemeinsamer Abspann für Korrekturansicht und Leseansicht.
%% Setzt den Schalter \ifkorrekturansicht voraus (gesetzt in den
%% einbindenden Dateien latex-korrekturansicht-abspann.tex bzw.
%% latex-leseansicht-abspann.tex).
%% ---------------------------------------------------------------

\normalsize

% Das esempio-Environment wird nur in der Leseansicht benötigt
\ifkorrekturansicht\else
\newenvironment{esempio}[3]%
{
    \vspace{1.5ex}
    \rlap{\underline{#1}}
    \par
    \setlength{\parindent}{0cm}
    \nopagebreak
    \leftskip=#2cm
    \rightskip=#3cm
}
{
    \par
}
\fi

\doendnotes{C}
\bigskip
\vfill

\clearpage

\footnotesize

\ifkorrekturansicht
  \lohead{\textsc{register}}
\fi

% theindex-Environment neu definieren ohne reledmac
\makeatletter
\renewenvironment{theindex}{%
  \ifkorrekturansicht
    \section*{\indexname}%
  \else
    \subsubsection*{Index der erwähnten Entitäten}%
  \fi
  \setlength{\parindent}{0pt}%
  \setlength{\parskip}{0pt plus 0.3pt}%
  \let\item\@idxitem
}{%
  \ifkorrekturansicht\clearpage\fi
}
\makeatother

\IfFileExists{\jobname-pw.ind}{\input{\jobname-pw.ind}}{}

% Quellenangabe nur in der Leseansicht
\ifkorrekturansicht\else
% Fallback-Definitionen, falls die .tex-Datei \titel etc. nicht gesetzt hat
\providecommand{\titel}{}
\providecommand{\editorInnen}{}
\providecommand{\dateiname}{\jobname}

\vspace{3cm}

\vfill

\footnotesize
\textsc{Quelle}: \titel. Herausgegeben von {\editorInnen}. In: \emph{Arthur Schnitzler: Briefwechsel mit Autorinnen und Autoren}.
 Digitale Edition, https://schnitzler-briefe.acdh.oeaw.ac.at/{\dateiname}.html (Stand \today)
\fi

\end{document}


