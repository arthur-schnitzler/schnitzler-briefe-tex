%% latex-leseansicht-vorspann.tex
%% Vorspann für die Leseansicht.
%% Lädt die gemeinsame Datei latex-vorspann.tex mit nicht gesetztem Schalter.

\newif\ifkorrekturansicht
\korrekturansichtfalse

\input{../tex-inputs/latex-vorspann}


         
         \renewcommand{\erwaehntePersonen}{Personen: Eduard Bacher, Richard Beer-Hofmann, Moriz Benedikt, Theodor Herzl, Fedor Mamroth,  N., Marie Reinhard, Emil Schiff}
         \renewcommand{\erwaehnteInstitutionen}{Institutionen: Frankfurter Zeitung, Neue Freie Presse}
         \renewcommand{\erwaehnteOrte}{Orte: Bastille, Berlin, Burgtheater, Central-Hotel, Deutschland, Frankfurt am Main, Paris, Redaktion der Frankfurter Zeitung, Russland, Wien, Österreich}
         \renewcommand{\erwaehnteWerke}{Werke: Der grüne Kakadu – Paracelsus – Die Gefährtin. Drei Einakter, Der grüne Kakadu. Groteske in einem Akt, Die Gefährtin. Schauspiel in einem Akt, Frankfurter Zeitung, Neue Deutsche Rundschau, Paracelsus. Versspiel in einem Akt, Tagebuch}
               \section[ Paul Goldmann an Arthur Schnitzler, 5. 3. {[}1899{]}]{ Paul Goldmann an Arthur Schnitzler, 5. 3. {[}1899{]}}\nopagebreak\mylabel{v}\rehead{ }\begin{ledgroupsized}[t]{13cm}\normalsize\beginnumbering \toendnotes[C]{\smallbreak\pagebreak[2]} \Standort{DLA, A:Schnitzler, HS.NZ85.1.3169.}
\physDesc{Brief, 2 Blätter, 8 Seiten, 7159 Zeichen
\newline{}Handschrift: schwarze Tinte, deutsche Kurrent
\newline{}Schnitzler: 1) mit Bleistift das Jahr »99« vermerkt  2) mit rotem Buntstift eine Unterstreichung}\toendnotes[C]{\smallbreak}\pstart
           \raggedleft{}{\pb}Frankfurt\oindex{Frankfurt am Main@\textbf{Frankfurt am Main}|pw}, 5. März.\pend
           \pstart\center{}Mein lieber Freund,\pend\pstart
           Ich komme aus \textsc{Paris\oindex{Paris@\textbf{Paris}|pw}} zurück und höre hier, daß Du mit Deinen drei Einaktern\pwindex{Schnitzler, Arthur 15.05.1862 – 21.10.1931@\textsc{Schnitzler, Arthur} (15.05.1862 – 21.10.1931), \emph{Schriftsteller, Mediziner}!gruene Kakadu – Paracelsus – Die Gefaehrtin. Drei Einakter1898 – 1899@\strich\emph{Der grüne Kakadu – Paracelsus – Die Gefährtin. Drei Einakter} {[}1898 – 1899{]}|pwv} wieder einen großen und ſchönen \label{K_L02868-1v}\edtext{Erfolg}{\lemma{\textnormal{\emph{Erfolg}}}\Cendnote{\textnormal{Der Einakterzyklus\pwindex{Schnitzler, Arthur 15.05.1862 – 21.10.1931@\textsc{Schnitzler, Arthur} (15.05.1862 – 21.10.1931), \emph{Schriftsteller, Mediziner}!gruene Kakadu – Paracelsus – Die Gefaehrtin. Drei Einakter1898 – 1899@\strich\emph{Der grüne Kakadu – Paracelsus – Die Gefährtin. Drei Einakter} {[}1898 – 1899{]}|pwkv} bestehend aus den Stücken \emph{Der grüne Kakadu}\pwindex{Schnitzler, Arthur 15.05.1862 – 21.10.1931@\textsc{Schnitzler, Arthur} (15.05.1862 – 21.10.1931), \emph{Schriftsteller, Mediziner}!gruene Kakadu. Groteske in einem Akt1. 3. 1899@\strich\emph{Der grüne Kakadu. Groteske in einem Akt} {[}1. 3. 1899{]}|pwk}, \emph{Paracelsus}\pwindex{Schnitzler, Arthur 15.05.1862 – 21.10.1931@\textsc{Schnitzler, Arthur} (15.05.1862 – 21.10.1931), \emph{Schriftsteller, Mediziner}!Paracelsus. Versspiel in einem Akt01. 11. 1898@\strich\emph{Paracelsus. Versspiel in einem Akt} {[}01. 11. 1898{]}|pwk} und \emph{Die Gefährtin}\pwindex{Schnitzler, Arthur 15.05.1862 – 21.10.1931@\textsc{Schnitzler, Arthur} (15.05.1862 – 21.10.1931), \emph{Schriftsteller, Mediziner}!Gefaehrtin. Schauspiel in einem Akt1899-03-01@\strich\emph{Die Gefährtin. Schauspiel in einem Akt} {[}1899-03-01{]}|pwk} wurde am 1. 3. 1899 im Wien\oindex{Wien@\textbf{Wien}|pwk}er Burgtheater\oindex{Burgtheater@\textbf{Burgtheater}|pwk} uraufgeführt.}}}\label{K_L02868-1h} gehabt. Ich freue mich darüber von
               Herzen und beglückwünſche Dich aufs Wärmſte. Geleſen habe ich noch keine Kritik, aber
               ich denke, ich finde die Wien\oindex{Wien@\textbf{Wien}|pw}er Blätter morgen hier im Büreau\oindex{Redaktion der Frankfurter Zeitung@\textbf{Redaktion der Frankfurter Zeitung}|pwv}. Den \label{K_L02868-88v}\edtext{»Grünen Kakadu\pwindex{Schnitzler, Arthur 15.05.1862 – 21.10.1931@\textsc{Schnitzler, Arthur} (15.05.1862 – 21.10.1931), \emph{Schriftsteller, Mediziner}!gruene Kakadu. Groteske in einem Akt1. 3. 1899@\strich\emph{Der grüne Kakadu. Groteske in einem Akt} {[}1. 3. 1899{]}|pw}« las ich}{\lemma{\textnormal{\emph{»Grünen Kakadu« las ich}}}\Cendnote{\textnormal{\emph{Der grüne Kakadu}\pwindex{Schnitzler, Arthur 15.05.1862 – 21.10.1931@\textsc{Schnitzler, Arthur} (15.05.1862 – 21.10.1931), \emph{Schriftsteller, Mediziner}!gruene Kakadu. Groteske in einem Akt1. 3. 1899@\strich\emph{Der grüne Kakadu. Groteske in einem Akt} {[}1. 3. 1899{]}|pwk} wurde zuerst in der \emph{Neuen Deutschen Rundschau}\pwindex{Neue Deutsche Rundschau1894-01-01 – 1903-12-31@\emph{Neue Deutsche Rundschau} {[}1894-01-01 – 1903-12-31{]}|pwk} (Jg. 10, H. 3,
                        März 1899, S. 282–308) gedruckt, Goldmann\pwindex{Goldmann, Paul 31.01.1865 – 25.09.1935@\textsc{Goldmann, Paul} (31.01.1865 – 25.09.1935), \emph{Schriftsteller, Journalist}|pwk} hätte also bereits den Erstdruck
                  lesen können. Er besaß aber ein Manuskript (vgl. Paul Goldmann an Arthur Schnitzler, 12. 3. [1899]). Dieses dürfte Goldmann\pwindex{Goldmann, Paul 31.01.1865 – 25.09.1935@\textsc{Goldmann, Paul} (31.01.1865 – 25.09.1935), \emph{Schriftsteller, Journalist}|pwk} in etwa Mitte Januar erhalten haben,
                  da er im \emph{Tagebuch}\pwindex{Schnitzler, Arthur 15.05.1862 – 21.10.1931@\textsc{Schnitzler, Arthur} (15.05.1862 – 21.10.1931), \emph{Schriftsteller, Mediziner}!Tagebuch1981 – 2000@\strich\emph{Tagebuch} {[}1981 – 2000{]}|pwk}{ }Schnitzler\pwindex{Schnitzler, Arthur 15.05.1862 – 21.10.1931@\textsc{Schnitzler, Arthur} (15.05.1862 – 21.10.1931), \emph{Schriftsteller, Mediziner}|pwk}s am 17. 1. 1899 zum
                  letzten Mal als sich in Wien\oindex{Wien@\textbf{Wien}|pwk} aufhaltend erwähnt
                  wird.}}}\label{K_L02868-88h} noch auf der Reiſe von Wien\oindex{Wien@\textbf{Wien}|pw} nach
                  Frankfurt\oindex{Frankfurt am Main@\textbf{Frankfurt am Main}|pw}. Ein vortreffliches Stück\pwindex{Schnitzler, Arthur 15.05.1862 – 21.10.1931@\textsc{Schnitzler, Arthur} (15.05.1862 – 21.10.1931), \emph{Schriftsteller, Mediziner}!gruene Kakadu. Groteske in einem Akt1. 3. 1899@\strich\emph{Der grüne Kakadu. Groteske in einem Akt} {[}1. 3. 1899{]}|pwv}. Da ich aber etwas ganz Vollendetes
               erwartete, hat es mich doch auch ein wenig enttäuſcht. Ich erhoffte Revolution und
                  Baſtille\oindex{Bastille@\textbf{Bastille}|pw}nſturm, fand aber zuletzt doch nur
               wieder eine Liebesgeſchichte mit einem Theatermädel. Anderſeits iſt es, glaube ich,
               in der Ausführung eines Deiner beſten Stücke und bedeutet doch \strikeout{\textcolor{gray}{einen}} auch einen gewaltigen Schritt nach vorwärts \strikeout{von
                     \textcolor{gray}{dem} alten T} von Deinem alten Ton und Deinen alten
               Stoffen zu irgend etwas Neuem, das ſehr ſchön werden wird.\pend
           \pstart
           {\pb}Mein lieber Freund, ich komme alſo nicht nach Wien\oindex{Wien@\textbf{Wien}|pw}. Es war ein quälendes wochenlanges Ringen und
               ein ſchwerer Entſchluß. Wie alle Entſchlüſſe im Augenblick nachdem man ſie gefaßt
               hat, erſcheint mir auch dieſer jetzt recht tadelnswerth. Aber das war zu
               erwarten.\pend
           \pstart
           Als ich von Wien\oindex{Wien@\textbf{Wien}|pw} nach Frankfurt\oindex{Frankfurt am Main@\textbf{Frankfurt am Main}|pw} kam und ſich in Frankfurt\oindex{Frankfurt am Main@\textbf{Frankfurt am Main}|pw} die Wien\oindex{Wien@\textbf{Wien}|pw}er Eindrücke zu klären
               begannen, ſchien es mir zunächſt unmöglich, mich wieder in den Wien\oindex{Wien@\textbf{Wien}|pw}er Journalismus zu fügen, nachdem ich Jahre lang unter
               größeren und freieren Verhältniſſen gelebt. Und nachdem ich Jahre lang in der »Frankfurter Zeitung\orgindex{Frankfurter Zeitung@Frankfurter Zeitung|pw}« gearbeitet, wo ich
               ungehindert meine Anſichten entfalten konnte und eigentlich nur mein Gewiſſen um Rath
               zu fragen brauchte, erſchien es mir unmöglich, mich in die \label{K_L02868-2v}\edtext{»Neue Freie Preſſe\orgindex{Neue Freie Presse@Neue Freie Presse|pw}«
                  { }\strikeout{\textcolor{gray}{einfügen}} hineinzufinden}{\lemma{\textnormal{\emph{»Neue … hineinzufinden}}}\Cendnote{\textnormal{als Redakteur für
                  ausländische Politik in Wien\oindex{Wien@\textbf{Wien}|pwk}}}}\label{K_L02868-2h} mit ihrer Rückſichtennehmerei und Cliquen-Wirthſchaft, welche verlangt, daß
               man Dieſes beſchönigt und Jenes verſchweigt und daß \label{K_L02868-3v}\edtext{man \textsc{Herzl\pwindex{Herzl, Theodor 1860-05-02 – 1904-07-03@\textsc{Herzl, Theodor} (1860-05-02 – 1904-07-03), \emph{Schriftsteller, Journalist}|pw}s} durchgefallene Stücke}{\lemma{\textnormal{\emph{man … Stücke}}}\Cendnote{\textnormal{Theodor Herzl\pwindex{Herzl, Theodor 1860-05-02 – 1904-07-03@\textsc{Herzl, Theodor} (1860-05-02 – 1904-07-03), \emph{Schriftsteller, Journalist}|pwk} verantwortete das Feuilleton
                  der \emph{Neuen Freien Presse}\orgindex{Neue Freie Presse@Neue Freie Presse|pwk}. Goldmann\pwindex{Goldmann, Paul 31.01.1865 – 25.09.1935@\textsc{Goldmann, Paul} (31.01.1865 – 25.09.1935), \emph{Schriftsteller, Journalist}|pwk} behauptete, dass die Berichterstattung über
                  dessen Stücke ungerechtfertigt positiv ausgefallen wäre.}}}\label{K_L02868-3h} als die {\pb}Meiſterwerke eines genialen Schriftſteller\pwindex{Herzl, Theodor 1860-05-02 – 1904-07-03@\textsc{Herzl, Theodor} (1860-05-02 – 1904-07-03), \emph{Schriftsteller, Journalist}|pwv}s dem Publicum anpreiſt. \strikeout{M} Mir grauſte ferner vor dem Arbeitsgebiet, das mir
               zugewieſen werden ſollte, der ausländiſchen Politik, während doch mein ganzes
               Beſtreben dahin geht, möglichſt aus der Politik heraus in die Literatur oder
               wenigſtens in den mit Literatur ſich beſchäftigenden Journalismus zu kommen. Und mir
               grauſte vor der Rieſen-Arbeit, die man mir in Wien\oindex{Wien@\textbf{Wien}|pw}
               aufbürden wollte, vor der Stellung des Redaktions-\label{K_L02868-4v}\edtext{Culis}{\lemma{\textnormal{\emph{Culis}}}\Cendnote{\textnormal{Kuli,
                  englisch/hindi: Tagelöhner, Verrichter minderer Dienste}}}\label{K_L02868-4h}, der alle Laſten
               trägt, vor der rückſichtsloſen Ausbeutung der Sklavenhalter in Wien\oindex{Wien@\textbf{Wien}|pw} (während die Sklavenhalter in Frankfurt\oindex{Frankfurt am Main@\textbf{Frankfurt am Main}|pw} doch ein wenig \strikeout{rü\textcolor{gray}{c}} rückſichtsvoller ausbeuten). Es iſt wahr, als Compenſation für das Alles hatte
               ich Euch in Wien\oindex{Wien@\textbf{Wien}|pw}. \strikeout{E} Gewiß, die ſchönſte aller Compenſationen. Aber \strikeout{an} die Hauptſache im Leben iſt die Arbeit, die man thut. Davon geht alle
               Sonne, alles Behagen aus. Und wenn man in ſeinen Wirkungskreis nicht hineinpaßt, ſo
               iſt das Daſein in ſeinem Wichtigſten verfehlt und man wird tiefunglücklich, trotz
               allen Verkehrs {\pb}mit ſehr lieben Menſchen. Beſſer
               eine Arbeit, die Einem wenigſtens einigermaßen zuſagt, und keine lieben Menſchen,
               als, wenn man ſchon einmal wählen muß, liebe Menſchen und eine widerwärtige Arbeit.
                  \introOben{}Hier muß man Stoiker ſein und darf ſeinem weichen Herzen nicht
                  nachgeben.\introOben{} Auch kommt dazu, daß Jeder von Euch jetzt ſein eigenes Leben lebt
               und daß ich von \strikeout{\textcolor{gray}{K}} Keinem, ſelbſt vom nächſten Freunde nicht, beanſpruchen darf, er ſolle mir
               mein Leben leben helfen. Während dieſer Zeit wurde ich in Frankfurt\oindex{Frankfurt am Main@\textbf{Frankfurt am Main}|pw} ſehr zum Bleiben gedrängt. Ich ſah, daß \strikeout{es} man in der Redaktion\orgindex{Frankfurter Zeitung@Frankfurter Zeitung|pwv} mich achtete und ſchätzte, merkte auch, daß das
               Publicum auf mich hielt. Und ich dachte mir, daß es eigentlich Wahnſinn wäre, zehn
               Jahre Arbeit, die ich in das Blatt\pwindex{?? Werk@Nicht ermittelte Verfasserinnen und Verfasser!Frankfurter Zeitung1856 – 1943@\emph{Frankfurter Zeitung} {[}1856 – 1943{]}|pwv} hier geſteckt, wegzuwerfen\strikeout{,} und nach
                  Wien\oindex{Wien@\textbf{Wien}|pw} zu gehen, wo kein Menſch mich kennt, wo
               nicht einmal Ihr mehr etwas von meinen Leiſtungen wißt, wo ich von Anfang anfangen
                  \strikeout{müßte} und mir Schritt für Schritt, unter Gott weiß
               welchen Kämpfen, {\pb}eine Stellung erſt ſchaffen müßte,
               die ich hier bereits beſitze. Zukunft endlich (wenn ich überhaupt Zukunft habe) gibt
               es doch nur in Deutſchland\oindex{Deutschland@\textbf{Deutschland}|pw}, nicht in Öſterreich\oindex{Oesterreich@\textbf{Österreich}|pw}. Dazu kam noch Allerlei, was die
               Familie angeht.\pend
           \pstart
           Immerhin wollte ich mit der »Neuen Freien Preſſe\orgindex{Neue Freie Presse@Neue Freie Presse|pw}«
               nicht gleich \strikeout{\textcolor{gray}{ab}} abbrechen und \strikeout{ſp\textcolor{gray}{a}} ſpann die Sache weiter. Wir waren verblieben (die \label{K_L02868-56v}\edtext{Chefredacteur\pwindex{Bacher, Eduard 07.03.1846 – 16.01.1908@\textsc{Bacher, Eduard} (07.03.1846 – 16.01.1908), \emph{Journalist, Herausgeber}|pwv}s}{\lemma{\textnormal{\emph{Chefredacteurs}}}\Cendnote{\textnormal{Seit dem Frühjahr 1879 war Eduard Bacher\pwindex{Bacher, Eduard 07.03.1846 – 16.01.1908@\textsc{Bacher, Eduard} (07.03.1846 – 16.01.1908), \emph{Journalist, Herausgeber}|pwk}
                  Chefredakteur der \emph{Neuen Freien Presse}\orgindex{Neue Freie Presse@Neue Freie Presse|pwk}. Es ist
                  nicht gänzlich geklärt, mit wem Goldmann\pwindex{Goldmann, Paul 31.01.1865 – 25.09.1935@\textsc{Goldmann, Paul} (31.01.1865 – 25.09.1935), \emph{Schriftsteller, Journalist}|pwk} in
                  dieser Zeit zusätzlich Kontakt hatte. Vermutlich war es Moriz Benedikt\pwindex{Benedikt, Moriz 27.05.1849 – 18.03.1920@\textsc{Benedikt, Moriz} (27.05.1849 – 18.03.1920), \emph{Journalist, Herausgeber}|pwk} (siehe Paul Goldmann an Arthur Schnitzler, 26. 10. 1899).}}}\label{K_L02868-56h} und ich), daß zur Beſiegelung meines
               Eintritts in die Redaktion\orgindex{Neue Freie Presse@Neue Freie Presse|pwv}
               Vertragsbriefe ausgetauſcht werden ſollten. Ich ſandte einen früheren Brief von \textsc{Bacher\pwindex{Bacher, Eduard 07.03.1846 – 16.01.1908@\textsc{Bacher, Eduard} (07.03.1846 – 16.01.1908), \emph{Journalist, Herausgeber}|pw}}, den dieſer behufs Aufſetzung des Vertrages gewünſcht hatte, an ihn zurück und
               bat um Überſendung des Vertragsbriefes. Wenige Tage darauf ſtarb \label{K_L02868-5v}\edtext{\textsc{Schiff\pwindex{Schiff, Emil 30.5.1849 – 23.1.1899@\textsc{Schiff, Emil} (30.5.1849 – 23.1.1899), \emph{Journalist}|pw}}}{\lemma{\textnormal{\emph{Schiff}}}\Cendnote{\textnormal{Emil Schiff\pwindex{Schiff, Emil 30.5.1849 – 23.1.1899@\textsc{Schiff, Emil} (30.5.1849 – 23.1.1899), \emph{Journalist}|pwk} verstarb am 23. 1. 1899.}}}\label{K_L02868-5h}, der Berlin\oindex{Berlin@\textbf{Berlin}|pw}er Correſpondent der N.
                  Fr. Pr.\orgindex{Neue Freie Presse@Neue Freie Presse|pw}; ich bekam von der Redaktion\orgindex{Neue Freie Presse@Neue Freie Presse|pwv} ein Telegramm mit der Aufforderung, den \label{K_L02868-123v}\edtext{Berlin\oindex{Berlin@\textbf{Berlin}|pw}er Correſpondenten\pwindex{N. @\textsc{N.}, \emph{Journalist}|pwv}}{\lemma{\textnormal{\emph{Berliner Correſpondenten}}}\Cendnote{\textnormal{Es handelte sich wohl um jenen
                  Korrespondenten, der unter dem Kürzel »N.\pwindex{N. @\textsc{N.}, \emph{Journalist}|pwkv}« schrieb. Der ganze Name konnte nicht ermittelt
                  werden.}}}\label{K_L02868-123h} der Frankfurter Zeitung\orgindex{Frankfurter Zeitung@Frankfurter Zeitung|pw}{ }\strikeout{als} als Nachfolger für \textsc{Schiff\pwindex{Schiff, Emil 30.5.1849 – 23.1.1899@\textsc{Schiff, Emil} (30.5.1849 – 23.1.1899), \emph{Journalist}|pw}} zu engagiren. {\pb}Ich telegraphirte \introOben{}und ſchrieb\introOben{} zurück, das ginge aus dieſem und jenem Grunde
               nicht, bot mich aber zugleich als Nachfolger \textsc{Schiff\pwindex{Schiff, Emil 30.5.1849 – 23.1.1899@\textsc{Schiff, Emil} (30.5.1849 – 23.1.1899), \emph{Journalist}|pw}s} in Berlin\oindex{Berlin@\textbf{Berlin}|pw} an. In der That wäre mir die Stellung in Berlin\oindex{Berlin@\textbf{Berlin}|pw} lieber geweſen, \strikeout{als die} als die in Wien\oindex{Wien@\textbf{Wien}|pw}. Ich hätte von
                  Berlin\oindex{Berlin@\textbf{Berlin}|pw} aus über Theater und Kunſt geſchrieben
               und wäre auch der Wien\oindex{Wien@\textbf{Wien}|pw}er Redaktions-Wirthſchaft in
                  Berlin\oindex{Berlin@\textbf{Berlin}|pw} ſehr \strikeout{entrü\textcolor{gray}{c}kt} entrückt geweſen. Meiner Anſicht nach hätte
               die N. Fr. Pr.\orgindex{Neue Freie Presse@Neue Freie Presse|pw} in mir einen recht geeigneten
               Correſpondenten für Berlin\oindex{Berlin@\textbf{Berlin}|pw} gehabt. Seit jenem
               Augenblick nun (Ende Januar) habe ich \strikeout{vo} von der N. Fr.
                  Pr.\orgindex{Neue Freie Presse@Neue Freie Presse|pw} kein Wort mehr gehört. Mehr als vier Wochen vergingen, \substVorne{}\textsuperscript{ohne{ }\textcolor{gray}{dieſe ich}}{\allowbreak}\substDazwischen{}und ich bekam\substHinten{} nicht nur keinen Beſcheid über mein Anerbieten bezüglich des \strikeout{Wien\oindex{Wien@\textbf{Wien}|pw}er Poſt\textcolor{gray}{e}}{ }Berlin\oindex{Berlin@\textbf{Berlin}|pw}er Poſtens, ſondern auch nicht einmal den
               Vertragsbrief, den die Leute mir ſofort hätten ſchicken müſſen. Ich wartete und
               wartete (dies der Grund, weshalb ich Dir ſo lange nicht geſchrieben), hielt es
               natürlich für unter {\pb}meiner Würde zu drängen, und
               nachdem bis zum Ende Februar immer noch weder Beſcheid
               noch Vertrag aus Wien\oindex{Wien@\textbf{Wien}|pw} eingetroffen waren,
               unterzeichnete ich einen neuen Vertrag mit der Frankfurter Zeitung\orgindex{Frankfurter Zeitung@Frankfurter Zeitung|pw}. Geſtern aber habe ich
               ein Telegramm von \textsc{Bacher\pwindex{Bacher, Eduard 07.03.1846 – 16.01.1908@\textsc{Bacher, Eduard} (07.03.1846 – 16.01.1908), \emph{Journalist, Herausgeber}|pw}} erhalten, der ſehr erzürnt darüber iſt, daß ich nicht am 1. März, wie mündlich\strikeout{,}
               beſprochen, in der Redaktion\orgindex{Neue Freie Presse@Neue Freie Presse|pwv} in
                  Wien\oindex{Wien@\textbf{Wien}|pw} angetreten bin! Ich habe ihm den
               Sachverhalt auseinandergeſetzt, und nach dieſem Telegramm wird mir das Verhalten der
               Leute noch räthſelhafter als zuvor.\pend
           \pstart
           In Frankfurt\oindex{Frankfurt am Main@\textbf{Frankfurt am Main}|pw} trete ich in die Feuilleton-Redaktion\orgindex{Frankfurter Zeitung@Frankfurter Zeitung|pwv} ein, als \label{K_L02868-11v}\edtext{\textsc{Adlatus}}{\lemma{\textnormal{\emph{Adlatus}}}\Cendnote{\textnormal{Gehilfe}}}\label{K_L02868-11h} von \textsc{Dr. Mamroth\pwindex{Mamroth, Fedor 21.02.1851 – 25.06.1907@\textsc{Mamroth, Fedor} (21.02.1851 – 25.06.1907), \emph{Journalist, Kritiker}|pw}}, und ſoll zu Reiſe-Miſſionen verwendet werden (im Herbſt nach Rußland\oindex{Russland@\textbf{Russland}|pw}, im nächſten Frühjahr zur Pariſ\oindex{Paris@\textbf{Paris}|pw}er Weltausſtellung, zu großen \textsc{\begin{otherlanguage}{french}Premièren\end{otherlanguage}} in Deutſchland\oindex{Deutschland@\textbf{Deutschland}|pw} und zu ähnlichen Anläſſen).
                  \strikeout{S\textcolor{gray}{o}{ }\textcolor{gray}{×}\-\textcolor{gray}{×}} So finde ich mich denn, nach ſo viel Wirrſal und Schwanken, \strikeout{\textcolor{gray}{×}\-\textcolor{gray}{×}h\textcolor{gray}{×}\-\textcolor{gray}{×}} auf einmal in der kleinen Stadt\oindex{Frankfurt am Main@\textbf{Frankfurt am Main}|pwv}, einſam, ohne Freunde, unter läſtigen Familien-{\pb}Verhältniſſen. \strikeout{\textcolor{gray}{Fe}} Fern von der großen Welt\substVorne{}\textsuperscript{!}\substDazwischen{}.\substHinten{} Und mir iſt, als ſei eine Thür hinter mir ins Schloß gefallen.\pend
           \pstart
           Habe ich recht gehandelt oder falſch? Wird \strikeout{\textcolor{gray}{×}\-\textcolor{gray}{×}s} dieſe neue Exiſtenz zu
               ertragen ſein? Ich weiß es nicht.\pend
           \pstart
           Bitte, zeig’ dem \textsc{Richard\pwindex{Beer-Hofmann, Richard 1866-07-11 – 1945-09-26@\textsc{Beer-Hofmann, Richard} (1866-07-11 – 1945-09-26), \emph{Schriftsteller}|pw}} dieſen Brief (wenn es ihn intereſſirt). Sonſt aber betrachte das Mitgetheilte
               als vertraulich; und wenn man \strikeout{d} Dich fragt, warum ich
               nicht zur N. Fr. Pr.\orgindex{Neue Freie Presse@Neue Freie Presse|pw} gekommen bin, ſo \strikeout{ſ\textcolor{gray}{pric}h} ſage, daß die Verhandlungen
               ſich in die Länge gezogen haben und daß die Sache noch unentſchieden iſt. Ich möchte
               mir nämlich, wenn es ginge, ein{[}e{]} Hinterthür für die Zukunft
               offen laſſen.\pend
           \pstart
           Bitte, ſchreib’ mir bald, liebſter Freund, und vor Allem: \label{K_L02868-45v}\edtext{komm’ demnächſt nach Frankfurt\oindex{Frankfurt am Main@\textbf{Frankfurt am Main}|pw}}{\lemma{\textnormal{\emph{komm’ … Frankfurt}}}\Cendnote{\textnormal{Schnitzler\pwindex{Schnitzler, Arthur 15.05.1862 – 21.10.1931@\textsc{Schnitzler, Arthur} (15.05.1862 – 21.10.1931), \emph{Schriftsteller, Mediziner}|pwk} war das nächste Mal von 19. 9. 1899 bis 23. 9. 1899 in Frankfurt am Main\oindex{Frankfurt am Main@\textbf{Frankfurt am Main}|pwk}.}}}\label{K_L02868-45h}!\pend
           \pstart
           Viele treue Grüße! {\\[\baselineskip]}Dein {\\[\baselineskip]}\spacefill\mbox{Paul Goldmann}\pend
           \leftskip=0em{}\pstart
           \noindent{}Adreſſe: \textsc{Hotel Central\oindex{Central-Hotel@\textbf{Central-Hotel}|pw}}, Frankfurt \textsuperscript{a}/\textsubscript{M.}\oindex{Frankfurt am Main@\textbf{Frankfurt am Main}|pw}\pend
           \pstart
           Grüße an Deine Freundin\pwindex{Reinhard, Marie 1871-03-13 – 1899-03-18@\textsc{Reinhard, Marie} (1871-03-13 – 1899-03-18), \emph{Gesangspädagogin}|pwv}!\pend
           
         
         \endnumbering\mylabel{h}\end{ledgroupsized}  \newcommand{\dateiname}{L02868}\newcommand{\titel}{Paul Goldmann an Arthur Schnitzler, 5. 3. [1899]}\newcommand{\editorInnen}{Martin Anton Müller und Laura Untner}%% latex-leseansicht-abspann.tex
%% Abspann für die Leseansicht.
%% Der Schalter \ifkorrekturansicht ist bereits durch den Vorspann gesetzt.

%% latex-abspann.tex
%% Gemeinsamer Abspann für Korrekturansicht und Leseansicht.
%% Setzt den Schalter \ifkorrekturansicht voraus (gesetzt in den
%% einbindenden Dateien latex-korrekturansicht-abspann.tex bzw.
%% latex-leseansicht-abspann.tex).
%% ---------------------------------------------------------------

\normalsize

% Das esempio-Environment wird nur in der Leseansicht benötigt
\ifkorrekturansicht\else
\newenvironment{esempio}[3]%
{
    \vspace{1.5ex}
    \rlap{\underline{#1}}
    \par
    \setlength{\parindent}{0cm}
    \nopagebreak
    \leftskip=#2cm
    \rightskip=#3cm
}
{
    \par
}
\fi

\doendnotes{C}
\bigskip
\vfill

\clearpage

\footnotesize

\ifkorrekturansicht
  \lohead{\textsc{register}}
\fi

% theindex-Environment neu definieren ohne reledmac
\makeatletter
\renewenvironment{theindex}{%
  \ifkorrekturansicht
    \section*{\indexname}%
  \else
    \subsubsection*{Index der erwähnten Entitäten}%
  \fi
  \setlength{\parindent}{0pt}%
  \setlength{\parskip}{0pt plus 0.3pt}%
  \let\item\@idxitem
}{%
  \ifkorrekturansicht\clearpage\fi
}
\makeatother

\IfFileExists{\jobname-pw.ind}{\input{\jobname-pw.ind}}{}

% Quellenangabe nur in der Leseansicht
\ifkorrekturansicht\else
% Fallback-Definitionen, falls die .tex-Datei \titel etc. nicht gesetzt hat
\providecommand{\titel}{}
\providecommand{\editorInnen}{}
\providecommand{\dateiname}{\jobname}

\vspace{3cm}

\vfill

\footnotesize
\textsc{Quelle}: \titel. Herausgegeben von {\editorInnen}. In: \emph{Arthur Schnitzler: Briefwechsel mit Autorinnen und Autoren}.
 Digitale Edition, https://schnitzler-briefe.acdh.oeaw.ac.at/{\dateiname}.html (Stand \today)
\fi

\end{document}


      