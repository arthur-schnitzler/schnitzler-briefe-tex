%% latex-leseansicht-vorspann.tex
%% Vorspann für die Leseansicht.
%% Lädt die gemeinsame Datei latex-vorspann.tex mit nicht gesetztem Schalter.

\newif\ifkorrekturansicht
\korrekturansichtfalse

\input{../tex-inputs/latex-vorspann}


\section[ Paul Goldmann an Arthur Schnitzler, 5. 3. [1899]]{L02868 Paul Goldmann an Arthur Schnitzler,  5. 3. [1899]}
\nopagebreak\mylabel{L02868v}
\rehead{ }\normalsize\beginnumbering\briefempfaengerindex{Schnitzler, Arthur@\textsc{Schnitzler, Arthur}!zzzGoldmann, Paul@\emph{von Paul Goldmann}!1899-03-051@{5. 3. [1899]}|(be}
\toendnotes[C]{\smallbreak\pagebreak[2]}
\correspDesc{Versand  durch Paul Goldmann am 5. 3. [1899] in Frankfurt am Main
\newline{}Erhalt  durch Arthur Schnitzler im Zeitraum [6. 3. 1899
                  – 10. 3. 1899?] in Wien}\toendnotes[C]{\smallbreak}
\Standort{DLA, A:Schnitzler, HS.NZ85.1.3169.}
\physDesc{Brief, 2 Blätter, 8 Seiten, 7159 Zeichen
\newline{}Handschrift: schwarze Tinte, deutsche Kurrent
\newline{}Schnitzler: 1) mit Bleistift das Jahr »99« vermerkt  2) mit rotem Buntstift eine Unterstreichung}\toendnotes[C]{\smallbreak}
\pstart
           \raggedleft{}{\pb}Frankfurt\oindex{Frankfurt am Main@\textbf{Frankfurt am Main}, \emph{Hauptstadt}|pw}, 5. März.\pend
           
\pstart\center{}Mein lieber Freund,\pend\vspace{0.5em}
\pstart
           Ich komme aus \textsc{Paris\oindex{Paris@\textbf{Paris}, \emph{Hauptstadt}|pw}} zurück und höre hier, daß Du mit Deinen drei Einaktern\pwindex{Schnitzler, Arthur 15.\,5.\,1862 Wien – 21.\,10.\,1931 ebd.@\textsc{Schnitzler, Arthur} (15.\,5.\,1862 Wien – 21.\,10.\,1931 ebd.), \emph{Schriftsteller, Mediziner}!grüne Kakadu – Paracelsus – Die Gefährtin. Drei Einakter@\strich\emph{Der grüne Kakadu – Paracelsus – Die Gefährtin. Drei Einakter}|pwv} wieder einen großen und{ }ſchönen \label{K_L02868-1v}\edtext{Erfolg}{\lemma{\textnormal{\emph{Erfolg}}}\Cendnote{\textnormal{Der Einakterzyklus\pwindex{Schnitzler, Arthur 15.\,5.\,1862 Wien – 21.\,10.\,1931 ebd.@\textsc{Schnitzler, Arthur} (15.\,5.\,1862 Wien – 21.\,10.\,1931 ebd.), \emph{Schriftsteller, Mediziner}!grüne Kakadu – Paracelsus – Die Gefährtin. Drei Einakter@\strich\emph{Der grüne Kakadu – Paracelsus – Die Gefährtin. Drei Einakter}|pwkv}, bestehend aus den Stücken \emph{Der grüne Kakadu}\pwindex{Schnitzler, Arthur 15.\,5.\,1862 Wien – 21.\,10.\,1931 ebd.@\textsc{Schnitzler, Arthur} (15.\,5.\,1862 Wien – 21.\,10.\,1931 ebd.), \emph{Schriftsteller, Mediziner}!grüne Kakadu. Groteske in einem Akt@\strich\emph{Der grüne Kakadu. Groteske in einem Akt}|pwk}, \emph{Paracelsus}\pwindex{Schnitzler, Arthur 15.\,5.\,1862 Wien – 21.\,10.\,1931 ebd.@\textsc{Schnitzler, Arthur} (15.\,5.\,1862 Wien – 21.\,10.\,1931 ebd.), \emph{Schriftsteller, Mediziner}!Paracelsus. Versspiel in einem Akt@\strich\emph{Paracelsus. Versspiel in einem Akt}|pwk} und \emph{Die Gefährtin}\pwindex{Schnitzler, Arthur 15.\,5.\,1862 Wien – 21.\,10.\,1931 ebd.@\textsc{Schnitzler, Arthur} (15.\,5.\,1862 Wien – 21.\,10.\,1931 ebd.), \emph{Schriftsteller, Mediziner}!Gefährtin. Schauspiel in einem Akt@\strich\emph{Die Gefährtin. Schauspiel in einem Akt}|pwk}, wurde am 1. 3. 1899 am \emph{Burgtheater}\orgindex{Burgtheater@Burgtheater|pwk} uraufgeführt.}}}\label{K_L02868-1} gehabt. Ich freue mich darüber von
               Herzen und beglückwünſche Dich aufs Wärmſte. Geleſen habe ich noch keine Kritik, aber
               ich denke, ich finde die Wien\oindex{Wien@\textbf{Wien}, \emph{Verwaltungsgebiet}|pw}er Blätter morgen hier im Büreau\oindex{Redaktion der Frankfurter Zeitung@\textbf{Redaktion der Frankfurter Zeitung}, \emph{Redaktionsgebäude}|pwv}. Den \label{K_L02868-2v}\edtext{»Grünen Kakadu\pwindex{Schnitzler, Arthur 15.\,5.\,1862 Wien – 21.\,10.\,1931 ebd.@\textsc{Schnitzler, Arthur} (15.\,5.\,1862 Wien – 21.\,10.\,1931 ebd.), \emph{Schriftsteller, Mediziner}!grüne Kakadu. Groteske in einem Akt@\strich\emph{Der grüne Kakadu. Groteske in einem Akt}|pw}« las ich}{\lemma{\textnormal{\emph{»Grünen Kakadu« las ich}}}\Cendnote{\textnormal{\emph{Der grüne Kakadu}\pwindex{Schnitzler, Arthur 15.\,5.\,1862 Wien – 21.\,10.\,1931 ebd.@\textsc{Schnitzler, Arthur} (15.\,5.\,1862 Wien – 21.\,10.\,1931 ebd.), \emph{Schriftsteller, Mediziner}!grüne Kakadu. Groteske in einem Akt@\strich\emph{Der grüne Kakadu. Groteske in einem Akt}|pwk} wurde zuerst in der \emph{Neuen Deutschen Rundschau}\pwindex{Neue Deutsche Rundschau@\emph{Neue Deutsche Rundschau}|pwk} (Jg. 10, H. 3,
                        März 1899, S. 282–308) gedruckt, Goldmann\pwindex{Goldmann, Paul 31.\,1.\,1865 Breslau – 25.\,9.\,1935 Wien@\textsc{Goldmann, Paul} (31.\,1.\,1865 Breslau – 25.\,9.\,1935 Wien), \emph{Schriftsteller, Journalist}|pwk} hätte also bereits den Erstdruck
                  lesen können. Er besaß aber ein Manuskript (vgl. XXXX Auszeichnungsfehler: Dokument L02869 nicht gefunden). Dieses dürfte Goldmann\pwindex{Goldmann, Paul 31.\,1.\,1865 Breslau – 25.\,9.\,1935 Wien@\textsc{Goldmann, Paul} (31.\,1.\,1865 Breslau – 25.\,9.\,1935 Wien), \emph{Schriftsteller, Journalist}|pwk} etwa Mitte Januar erhalten haben,
                  da er im \emph{Tagebuch}\pwindex{Schnitzler, Arthur 15.\,5.\,1862 Wien – 21.\,10.\,1931 ebd.@\textsc{Schnitzler, Arthur} (15.\,5.\,1862 Wien – 21.\,10.\,1931 ebd.), \emph{Schriftsteller, Mediziner}!Tagebuch@\strich\emph{Tagebuch}|pwk}{ }Schnitzlers am 17. 1. 1899 zum
                  letzten Mal als sich in Wien\oindex{Wien@\textbf{Wien}, \emph{Verwaltungsgebiet}|pwk} aufhaltend erwähnt
                  wird.}}}\label{K_L02868-2} noch auf der Reiſe von Wien\oindex{Wien@\textbf{Wien}, \emph{Verwaltungsgebiet}|pw} nach
                  Frankfurt\oindex{Frankfurt am Main@\textbf{Frankfurt am Main}, \emph{Hauptstadt}|pw}. Ein vortreffliches Stück\pwindex{Schnitzler, Arthur 15.\,5.\,1862 Wien – 21.\,10.\,1931 ebd.@\textsc{Schnitzler, Arthur} (15.\,5.\,1862 Wien – 21.\,10.\,1931 ebd.), \emph{Schriftsteller, Mediziner}!grüne Kakadu. Groteske in einem Akt@\strich\emph{Der grüne Kakadu. Groteske in einem Akt}|pwv}. Da ich aber etwas ganz Vollendetes
               erwartete, hat es mich doch auch ein wenig enttäuſcht. Ich erhoffte Revolution und
                  Baſtille\oindex{Bastille@\textbf{Bastille}, \emph{Gebäude}|pw}nſturm, fand aber zuletzt doch nur
               wieder eine Liebesgeſchichte mit einem Theatermädel. Anderſeits iſt es, glaube ich,
               in der Ausführung eines Deiner beſten Stücke und bedeutet doch \strikeout{\textcolor{gray}{einen}} auch einen gewaltigen Schritt nach vorwärts \strikeout{von
                     \textcolor{gray}{dem} alten T} von Deinem alten Ton und Deinen alten
               Stoffen zu irgend etwas Neuem, das{ }ſehr{ }ſchön werden wird.\pend
           
\pstart
           {\pb}Mein lieber Freund, ich komme alſo nicht nach Wien\oindex{Wien@\textbf{Wien}, \emph{Verwaltungsgebiet}|pw}. Es war ein quälendes wochenlanges Ringen und
               ein{ }ſchwerer Entſchluß. Wie alle Entſchlüſſe im Augenblick nachdem man{ }ſie gefaßt
               hat, erſcheint mir auch dieſer jetzt recht tadelnswerth. Aber das war zu
               erwarten.\pend
           
\pstart
           Als ich von Wien\oindex{Wien@\textbf{Wien}, \emph{Verwaltungsgebiet}|pw} nach Frankfurt\oindex{Frankfurt am Main@\textbf{Frankfurt am Main}, \emph{Hauptstadt}|pw} kam und{ }ſich in Frankfurt\oindex{Frankfurt am Main@\textbf{Frankfurt am Main}, \emph{Hauptstadt}|pw} die Wien\oindex{Wien@\textbf{Wien}, \emph{Verwaltungsgebiet}|pw}er Eindrücke zu klären
               begannen,{ }ſchien es mir zunächſt unmöglich, mich wieder in den Wien\oindex{Wien@\textbf{Wien}, \emph{Verwaltungsgebiet}|pw}er Journalismus zu fügen, nachdem ich Jahre lang unter
               größeren und freieren Verhältniſſen gelebt. Und nachdem ich Jahre lang in der »Frankfurter Zeitung\orgindex{Frankfurter Zeitung@Frankfurter Zeitung|pw}« gearbeitet, wo ich
               ungehindert meine Anſichten entfalten konnte und eigentlich nur mein Gewiſſen um Rath
               zu fragen brauchte, erſchien es mir unmöglich, mich in die \label{K_L02868-3v}\edtext{»Neue Freie Preſſe\orgindex{Neue Freie Presse@Neue Freie Presse|pw}«
                  { }\strikeout{\textcolor{gray}{einfügen}} hineinzufinden}{\lemma{\textnormal{\emph{»Neue … hineinzufinden}}}\Cendnote{\textnormal{als Redakteur für
                  ausländische Politik in Wien\oindex{Wien@\textbf{Wien}, \emph{Verwaltungsgebiet}|pwk}}}}\label{K_L02868-3} mit ihrer Rückſichtennehmerei und Cliquen-Wirthſchaft, welche verlangt, daß
               man Dieſes beſchönigt und Jenes verſchweigt und daß \label{K_L02868-4v}\edtext{man \textsc{Herzls\pwindex{Herzl, Theodor 2.\,5.\,1860 Budapest – 3.\,7.\,1904 Edlach@\textsc{Herzl, Theodor} (2.\,5.\,1860 Budapest – 3.\,7.\,1904 Edlach), \emph{Schriftsteller, Journalist}|pw}} durchgefallene Stücke}{\lemma{\textnormal{\emph{man … Stücke}}}\Cendnote{\textnormal{Theodor Herzl\pwindex{Herzl, Theodor 2.\,5.\,1860 Budapest – 3.\,7.\,1904 Edlach@\textsc{Herzl, Theodor} (2.\,5.\,1860 Budapest – 3.\,7.\,1904 Edlach), \emph{Schriftsteller, Journalist}|pwk} verantwortete das Feuilleton
                  der \emph{Neuen Freien Presse}\orgindex{Neue Freie Presse@Neue Freie Presse|pwk}. Goldmann\pwindex{Goldmann, Paul 31.\,1.\,1865 Breslau – 25.\,9.\,1935 Wien@\textsc{Goldmann, Paul} (31.\,1.\,1865 Breslau – 25.\,9.\,1935 Wien), \emph{Schriftsteller, Journalist}|pwk} behauptete, dass die Berichterstattung über
                  dessen Stücke ungerechtfertigt positiv ausgefallen wäre.}}}\label{K_L02868-4} als die {\pb}Meiſterwerke eines genialen Schriftſtellers\pwindex{Herzl, Theodor 2.\,5.\,1860 Budapest – 3.\,7.\,1904 Edlach@\textsc{Herzl, Theodor} (2.\,5.\,1860 Budapest – 3.\,7.\,1904 Edlach), \emph{Schriftsteller, Journalist}|pwv} dem Publicum anpreiſt. \strikeout{M} Mir grauſte ferner vor dem Arbeitsgebiet, das mir
               zugewieſen werden{ }ſollte, der ausländiſchen Politik, während doch mein ganzes
               Beſtreben dahin geht, möglichſt aus der Politik heraus in die Literatur oder
               wenigſtens in den mit Literatur{ }ſich beſchäftigenden Journalismus zu kommen. Und mir
               grauſte vor der Rieſen-Arbeit, die man mir in Wien\oindex{Wien@\textbf{Wien}, \emph{Verwaltungsgebiet}|pw}
               aufbürden wollte, vor der Stellung des Redaktions-\label{K_L02868-5v}\edtext{Culis}{\lemma{\textnormal{\emph{Culis}}}\Cendnote{\textnormal{Kuli,
                  englisch/hindi: Tagelöhner, Verrichter minderer Dienste}}}\label{K_L02868-5}, der alle Laſten
               trägt, vor der rückſichtsloſen Ausbeutung der Sklavenhalter in Wien\oindex{Wien@\textbf{Wien}, \emph{Verwaltungsgebiet}|pw} (während die Sklavenhalter in Frankfurt\oindex{Frankfurt am Main@\textbf{Frankfurt am Main}, \emph{Hauptstadt}|pw} doch ein wenig \strikeout{rü\textcolor{gray}{c}} rückſichtsvoller ausbeuten). Es iſt wahr, als Compenſation für das Alles hatte
               ich Euch in Wien\oindex{Wien@\textbf{Wien}, \emph{Verwaltungsgebiet}|pw}. \strikeout{E} Gewiß, die{ }ſchönſte aller Compenſationen. Aber \strikeout{an} die Hauptſache im Leben iſt die Arbeit, die man thut. Davon geht alle
               Sonne, alles Behagen aus. Und wenn man in{ }ſeinen Wirkungskreis nicht hineinpaßt,{ }ſo
               iſt das Daſein in{ }ſeinem Wichtigſten verfehlt und man wird tiefunglücklich, trotz
               allen Verkehrs {\pb}mit{ }ſehr lieben Menſchen. Beſſer
               eine Arbeit, die Einem wenigſtens einigermaßen zuſagt, und keine lieben Menſchen,
               als, wenn man{ }ſchon einmal wählen muß, liebe Menſchen und eine widerwärtige Arbeit.
                  \introOben{}Hier muß man Stoiker{ }ſein und darf{ }ſeinem weichen Herzen nicht
                  nachgeben.\introOben{} Auch kommt dazu, daß Jeder von Euch jetzt{ }ſein eigenes Leben lebt
               und daß ich von \strikeout{\textcolor{gray}{K}} Keinem,{ }ſelbſt vom nächſten Freunde nicht, beanſpruchen darf, er{ }ſolle mir
               mein Leben leben helfen. Während dieſer Zeit wurde ich in Frankfurt\oindex{Frankfurt am Main@\textbf{Frankfurt am Main}, \emph{Hauptstadt}|pw}{ }ſehr zum Bleiben gedrängt. Ich{ }ſah, daß \strikeout{es} man in der Redaktion\orgindex{Frankfurter Zeitung@Frankfurter Zeitung|pwv} mich achtete und{ }ſchätzte, merkte auch, daß das
               Publicum auf mich hielt. Und ich dachte mir, daß es eigentlich Wahnſinn wäre, zehn
               Jahre Arbeit, die ich in das Blatt\pwindex{Frankfurter Zeitung@\emph{Frankfurter Zeitung}|pwv} hier geſteckt, wegzuwerfen\strikeout{,} und nach
                  Wien\oindex{Wien@\textbf{Wien}, \emph{Verwaltungsgebiet}|pw} zu gehen, wo kein Menſch mich kennt, wo
               nicht einmal Ihr mehr etwas von meinen Leiſtungen wißt, wo ich von Anfang anfangen
                  \strikeout{müßte} und mir Schritt für Schritt, unter Gott weiß
               welchen Kämpfen, {\pb}eine Stellung erſt{ }ſchaffen müßte,
               die ich hier bereits beſitze. Zukunft endlich (wenn ich überhaupt Zukunft habe) gibt
               es doch nur in Deutſchland\oindex{Deutschland@\textbf{Deutschland}|pw}, nicht in Öſterreich\oindex{Österreich@\textbf{Österreich}|pw}. Dazu kam noch Allerlei, was die
               Familie angeht.\pend
           
\pstart
           Immerhin wollte ich mit der »Neuen Freien Preſſe\orgindex{Neue Freie Presse@Neue Freie Presse|pw}«
               nicht gleich \strikeout{\textcolor{gray}{ab}} abbrechen und \strikeout{ſp\textcolor{gray}{a}}{ }ſpann die Sache weiter. Wir waren verblieben (die \label{K_L02868-6v}\edtext{Chefredacteurs\pwindex{Bacher, Eduard 7.\,3.\,1846 Postoloprty – 16.\,1.\,1908 Wien@\textsc{Bacher, Eduard} (7.\,3.\,1846 Postoloprty – 16.\,1.\,1908 Wien), \emph{Journalist, Herausgeber}|pwv}}{\lemma{\textnormal{\emph{Chefredacteurs}}}\Cendnote{\textnormal{Seit dem Frühjahr 1879 war Eduard Bacher\pwindex{Bacher, Eduard 7.\,3.\,1846 Postoloprty – 16.\,1.\,1908 Wien@\textsc{Bacher, Eduard} (7.\,3.\,1846 Postoloprty – 16.\,1.\,1908 Wien), \emph{Journalist, Herausgeber}|pwk}
                  Chefredakteur der \emph{Neuen Freien Presse}\orgindex{Neue Freie Presse@Neue Freie Presse|pwk}. Es ist
                  nicht gänzlich geklärt, mit wem Goldmann\pwindex{Goldmann, Paul 31.\,1.\,1865 Breslau – 25.\,9.\,1935 Wien@\textsc{Goldmann, Paul} (31.\,1.\,1865 Breslau – 25.\,9.\,1935 Wien), \emph{Schriftsteller, Journalist}|pwk} in
                  dieser Zeit zusätzlich Kontakt hatte. Vermutlich war es Moriz Benedikt\pwindex{Benedikt, Moriz 27.\,5.\,1849 Kvačice – 18.\,3.\,1920 Wien@\textsc{Benedikt, Moriz} (27.\,5.\,1849 Kvačice – 18.\,3.\,1920 Wien), \emph{Journalist, Herausgeber}|pwk} (siehe XXXX Auszeichnungsfehler: Dokument L02892 nicht gefunden).}}}\label{K_L02868-6} und ich), daß zur Beſiegelung meines
               Eintritts in die Redaktion\orgindex{Neue Freie Presse@Neue Freie Presse|pwv}
               Vertragsbriefe ausgetauſcht werden{ }ſollten. Ich{ }ſandte einen früheren Brief von \textsc{Bacher\pwindex{Bacher, Eduard 7.\,3.\,1846 Postoloprty – 16.\,1.\,1908 Wien@\textsc{Bacher, Eduard} (7.\,3.\,1846 Postoloprty – 16.\,1.\,1908 Wien), \emph{Journalist, Herausgeber}|pw}}, den dieſer behufs Aufſetzung des Vertrages gewünſcht hatte, an ihn zurück und
               bat um Überſendung des Vertragsbriefes. Wenige Tage darauf{ }ſtarb \label{K_L02868-7v}\edtext{\textsc{Schiff\pwindex{Schiff, Emil 30.\,5.\,1849 Roudnice nad Labem – 23.\,1.\,1899 Berlin@\textsc{Schiff, Emil} (30.\,5.\,1849 Roudnice nad Labem – 23.\,1.\,1899 Berlin), \emph{Journalist}|pw}}}{\lemma{\textnormal{\emph{Schiff}}}\Cendnote{\textnormal{Emil Schiff\pwindex{Schiff, Emil 30.\,5.\,1849 Roudnice nad Labem – 23.\,1.\,1899 Berlin@\textsc{Schiff, Emil} (30.\,5.\,1849 Roudnice nad Labem – 23.\,1.\,1899 Berlin), \emph{Journalist}|pwk} verstarb am 23. 1. 1899.}}}\label{K_L02868-7}, der Berlin\oindex{Berlin@\textbf{Berlin}, \emph{Hauptstadt}|pw}er Correſpondent der N.
                  Fr. Pr.\orgindex{Neue Freie Presse@Neue Freie Presse|pw}; ich bekam von der Redaktion\orgindex{Neue Freie Presse@Neue Freie Presse|pwv} ein Telegramm mit der Aufforderung, den \label{K_L02868-8v}\edtext{Berlin\oindex{Berlin@\textbf{Berlin}, \emph{Hauptstadt}|pw}er Correſpondenten\pwindex{N. @\textsc{N.}, \emph{Journalist}|pwv}}{\lemma{\textnormal{\emph{Berliner Correspondenten}}}\Cendnote{\textnormal{Es handelte sich wohl um jenen
                  Korrespondenten, der unter dem Kürzel »N.\pwindex{N. @\textsc{N.}, \emph{Journalist}|pwkv}« schrieb. Der ganze Name konnte nicht ermittelt
                  werden.}}}\label{K_L02868-8} der Frankfurter Zeitung\orgindex{Frankfurter Zeitung@Frankfurter Zeitung|pw}{ }\strikeout{als} als Nachfolger für \textsc{Schiff\pwindex{Schiff, Emil 30.\,5.\,1849 Roudnice nad Labem – 23.\,1.\,1899 Berlin@\textsc{Schiff, Emil} (30.\,5.\,1849 Roudnice nad Labem – 23.\,1.\,1899 Berlin), \emph{Journalist}|pw}} zu engagiren. {\pb}Ich telegraphirte \introOben{}und{ }ſchrieb\introOben{} zurück, das ginge aus dieſem und jenem Grunde
               nicht, bot mich aber zugleich als Nachfolger \textsc{Schiffs\pwindex{Schiff, Emil 30.\,5.\,1849 Roudnice nad Labem – 23.\,1.\,1899 Berlin@\textsc{Schiff, Emil} (30.\,5.\,1849 Roudnice nad Labem – 23.\,1.\,1899 Berlin), \emph{Journalist}|pw}} in Berlin\oindex{Berlin@\textbf{Berlin}, \emph{Hauptstadt}|pw} an. In der That wäre mir die Stellung in Berlin\oindex{Berlin@\textbf{Berlin}, \emph{Hauptstadt}|pw} lieber geweſen, \strikeout{als die} als die in Wien\oindex{Wien@\textbf{Wien}, \emph{Verwaltungsgebiet}|pw}. Ich hätte von
                  Berlin\oindex{Berlin@\textbf{Berlin}, \emph{Hauptstadt}|pw} aus über Theater und Kunſt geſchrieben
               und wäre auch der Wien\oindex{Wien@\textbf{Wien}, \emph{Verwaltungsgebiet}|pw}er Redaktions-Wirthſchaft in
                  Berlin\oindex{Berlin@\textbf{Berlin}, \emph{Hauptstadt}|pw}{ }ſehr \strikeout{entrü\textcolor{gray}{c}kt} entrückt geweſen. Meiner Anſicht nach hätte
               die N. Fr. Pr.\orgindex{Neue Freie Presse@Neue Freie Presse|pw} in mir einen recht geeigneten
               Correſpondenten für Berlin\oindex{Berlin@\textbf{Berlin}, \emph{Hauptstadt}|pw} gehabt. Seit jenem
               Augenblick nun (Ende Januar) habe ich \strikeout{vo} von der N. Fr.
                  Pr.\orgindex{Neue Freie Presse@Neue Freie Presse|pw} kein Wort mehr gehört. Mehr als vier Wochen vergingen, \substVorne{}\textsuperscript{ohne{ }\textcolor{gray}{dieſe ich}}\substDazwischen{}und ich bekam\substHinten{} nicht nur keinen Beſcheid über mein Anerbieten bezüglich des \strikeout{Wien\oindex{Wien@\textbf{Wien}, \emph{Verwaltungsgebiet}|pw}er Poſt\textcolor{gray}{e}}{ }Berlin\oindex{Berlin@\textbf{Berlin}, \emph{Hauptstadt}|pw}er Poſtens,{ }ſondern auch nicht einmal den
               Vertragsbrief, den die Leute mir{ }ſofort hätten{ }ſchicken müſſen. Ich wartete und
               wartete (dies der Grund, weshalb ich Dir{ }ſo lange nicht geſchrieben), hielt es
               natürlich für unter {\pb}meiner Würde zu drängen, und
               nachdem bis zum Ende Februar immer noch weder Beſcheid
               noch Vertrag aus Wien\oindex{Wien@\textbf{Wien}, \emph{Verwaltungsgebiet}|pw} eingetroffen waren,
               unterzeichnete ich einen neuen Vertrag mit der Frankfurter Zeitung\orgindex{Frankfurter Zeitung@Frankfurter Zeitung|pw}. Geſtern aber habe ich
               ein Telegramm von \textsc{Bacher\pwindex{Bacher, Eduard 7.\,3.\,1846 Postoloprty – 16.\,1.\,1908 Wien@\textsc{Bacher, Eduard} (7.\,3.\,1846 Postoloprty – 16.\,1.\,1908 Wien), \emph{Journalist, Herausgeber}|pw}} erhalten, der{ }ſehr erzürnt darüber iſt, daß ich nicht am 1. März, wie mündlich\strikeout{,}
               beſprochen, in der Redaktion\orgindex{Neue Freie Presse@Neue Freie Presse|pwv} in
                  Wien\oindex{Wien@\textbf{Wien}, \emph{Verwaltungsgebiet}|pw} angetreten bin! Ich habe ihm den
               Sachverhalt auseinandergeſetzt, und nach dieſem Telegramm wird mir das Verhalten der
               Leute noch räthſelhafter als zuvor.\pend
           
\pstart
           In Frankfurt\oindex{Frankfurt am Main@\textbf{Frankfurt am Main}, \emph{Hauptstadt}|pw} trete ich in die Feuilleton-Redaktion\orgindex{Frankfurter Zeitung@Frankfurter Zeitung|pwv} ein, als \label{K_L02868-9v}\edtext{\textsc{Adlatus}}{\lemma{\textnormal{\emph{Adlatus}}}\Cendnote{\textnormal{Gehilfe}}}\label{K_L02868-9} von \textsc{Dr. Mamroth\pwindex{Mamroth, Fedor 21.\,2.\,1851 Breslau – 25.\,6.\,1907 Frankfurt am Main@\textsc{Mamroth, Fedor} (21.\,2.\,1851 Breslau – 25.\,6.\,1907 Frankfurt am Main), \emph{Journalist, Kritiker}|pw}}, und{ }ſoll zu Reiſe-Miſſionen verwendet werden (im Herbſt nach Rußland\oindex{Russland@\textbf{Russland}|pw}, im nächſten Frühjahr zur Pariſ\oindex{Paris@\textbf{Paris}, \emph{Hauptstadt}|pw}er Weltausſtellung, zu großen \textsc{\begin{otherlanguage}{french}Premièren\end{otherlanguage}} in Deutſchland\oindex{Deutschland@\textbf{Deutschland}|pw} und zu ähnlichen Anläſſen).
                  \strikeout{S\textcolor{gray}{o}{ }\textcolor{gray}{×}\-\textcolor{gray}{×}} So finde ich mich denn, nach{ }ſo viel Wirrſal und Schwanken, \strikeout{\textcolor{gray}{×}\-\textcolor{gray}{×}h\textcolor{gray}{×}\-\textcolor{gray}{×}} auf einmal in der kleinen Stadt\oindex{Frankfurt am Main@\textbf{Frankfurt am Main}, \emph{Hauptstadt}|pwv}, einſam, ohne Freunde, unter läſtigen Familien-{\pb}Verhältniſſen. \strikeout{\textcolor{gray}{Fe}} Fern von der großen Welt\substVorne{}\textsuperscript{!}\substDazwischen{}.\substHinten{} Und mir iſt, als{ }ſei eine Thür hinter mir ins Schloß gefallen.\pend
           
\pstart
           Habe ich recht gehandelt oder falſch? Wird \strikeout{\textcolor{gray}{×}\-\textcolor{gray}{×}s} dieſe neue Exiſtenz zu
               ertragen{ }ſein? Ich weiß es nicht.\pend
           
\pstart
           Bitte, zeig’ dem \textsc{Richard\pwindex{Beer-Hofmann, Richard 11.\,7.\,1866 Wien – 26.\,9.\,1945 New York City@\textsc{Beer-Hofmann, Richard} (11.\,7.\,1866 Wien – 26.\,9.\,1945 New York City), \emph{Schriftsteller}|pw}} dieſen Brief (wenn es ihn intereſſirt). Sonſt aber betrachte das Mitgetheilte
               als vertraulich; und wenn man \strikeout{d} Dich fragt, warum ich
               nicht zur N. Fr. Pr.\orgindex{Neue Freie Presse@Neue Freie Presse|pw} gekommen bin,{ }ſo \strikeout{ſ\textcolor{gray}{pric}h}{ }ſage, daß die Verhandlungen{ }ſich in die Länge gezogen haben und daß die Sache noch unentſchieden iſt. Ich möchte
               mir nämlich, wenn es ginge, ein{[}e{]} Hinterthür für die Zukunft
               offen laſſen.\pend
           
\pstart
           Bitte,{ }ſchreib’ mir bald, liebſter Freund, und vor Allem: \label{K_L02868-10v}\edtext{komm’ demnächſt nach Frankfurt\oindex{Frankfurt am Main@\textbf{Frankfurt am Main}, \emph{Hauptstadt}|pw}}{\lemma{\textnormal{\emph{komm’ … Frankfurt}}}\Cendnote{\textnormal{Schnitzler war das nächste Mal vom 19. 9. 1899 bis zum 23. 9. 1899 in Frankfurt am Main\oindex{Frankfurt am Main@\textbf{Frankfurt am Main}, \emph{Hauptstadt}|pwk}.}}}\label{K_L02868-10}!\pend
           
\pstart
           Viele treue Grüße! {\\[\baselineskip]}Dein {\\[\baselineskip]}\spacefill\mbox{Paul Goldmann}\pend
           \leftskip=0em{}
\pstart
           \noindent{}Adreſſe: \textsc{Hotel Central\oindex{Central-Hotel@\textbf{Central-Hotel}, \emph{Hotel}|pw}}, Frankfurt \textsuperscript{a}/\textsubscript{M.}\oindex{Frankfurt am Main@\textbf{Frankfurt am Main}, \emph{Hauptstadt}|pw}\pend
           
\pstart
           Grüße an Deine Freundin\pwindex{Reinhard, Marie 13.\,3.\,1871 Wien – 18.\,3.\,1899 ebd.@\textsc{Reinhard, Marie} (13.\,3.\,1871 Wien – 18.\,3.\,1899 ebd.), \emph{Gesangspädagogin}|pwv}!\pend
           \selectlanguage{ngerman}\endnumbering\briefempfaengerindex{Schnitzler, Arthur@\textsc{Schnitzler, Arthur}!zzzGoldmann, Paul@\emph{von Paul Goldmann}!1899-03-051@{5. 3. [1899]}|)be}\mylabel{L02868h}  \newcommand{\dateiname}{L02868}\newcommand{\titel}{Paul Goldmann an Arthur Schnitzler, 5. 3. [1899]}\newcommand{\editorInnen}{Martin Anton Müller und Laura Untner}%% latex-leseansicht-abspann.tex
%% Abspann für die Leseansicht.
%% Der Schalter \ifkorrekturansicht ist bereits durch den Vorspann gesetzt.

%% latex-abspann.tex
%% Gemeinsamer Abspann für Korrekturansicht und Leseansicht.
%% Setzt den Schalter \ifkorrekturansicht voraus (gesetzt in den
%% einbindenden Dateien latex-korrekturansicht-abspann.tex bzw.
%% latex-leseansicht-abspann.tex).
%% ---------------------------------------------------------------

\normalsize

% Das esempio-Environment wird nur in der Leseansicht benötigt
\ifkorrekturansicht\else
\newenvironment{esempio}[3]%
{
    \vspace{1.5ex}
    \rlap{\underline{#1}}
    \par
    \setlength{\parindent}{0cm}
    \nopagebreak
    \leftskip=#2cm
    \rightskip=#3cm
}
{
    \par
}
\fi

\doendnotes{C}
\bigskip
\vfill

\clearpage

\footnotesize

\ifkorrekturansicht
  \lohead{\textsc{register}}
\fi

% theindex-Environment neu definieren ohne reledmac
\makeatletter
\renewenvironment{theindex}{%
  \ifkorrekturansicht
    \section*{\indexname}%
  \else
    \subsubsection*{Index der erwähnten Entitäten}%
  \fi
  \setlength{\parindent}{0pt}%
  \setlength{\parskip}{0pt plus 0.3pt}%
  \let\item\@idxitem
}{%
  \ifkorrekturansicht\clearpage\fi
}
\makeatother

\IfFileExists{\jobname-pw.ind}{\input{\jobname-pw.ind}}{}

% Quellenangabe nur in der Leseansicht
\ifkorrekturansicht\else
% Fallback-Definitionen, falls die .tex-Datei \titel etc. nicht gesetzt hat
\providecommand{\titel}{}
\providecommand{\editorInnen}{}
\providecommand{\dateiname}{\jobname}

\vspace{3cm}

\vfill

\footnotesize
\textsc{Quelle}: \titel. Herausgegeben von {\editorInnen}. In: \emph{Arthur Schnitzler: Briefwechsel mit Autorinnen und Autoren}.
 Digitale Edition, https://schnitzler-briefe.acdh.oeaw.ac.at/{\dateiname}.html (Stand \today)
\fi

\end{document}


