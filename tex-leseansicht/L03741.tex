%% latex-korrekturansicht-vorspann.tex
%% Vorspann für die Korrekturansicht.
%% Lädt die gemeinsame Datei latex-vorspann.tex mit gesetztem Schalter.

\newif\ifkorrekturansicht
\korrekturansichttrue

\input{../tex-inputs/latex-vorspann}


\section[Arthur Schnitzler an Stefan Zweig, 16. 1. 1928]{L03741 Arthur Schnitzler an Stefan Zweig, 16. 1. 1928}
\nopagebreak\mylabel{L03741v}
\rehead{ }\normalsize\beginnumbering\briefempfaengerindex{Zweig, Stefan@\textsc{Zweig, Stefan}!zzzSchnitzler, Arthur@\emph{von Arthur Schnitzler}!1928-01-161@{16. 1. 1928}|(be}
\toendnotes[C]{\smallbreak\pagebreak[2]}\Standort{Jerusalem, National Library of Israel, ARC. Ms. Var. 305 1 58 Stefan Zweig Collection.}
\physDesc{Brief, 1 Blatt, 2 Seiten, 3222 Zeichen
\newline{}Schreibmaschine
\newline{}Handschrift: Bleistift, lateinische Kurrent (\noindent{}marginale Korrekturen, eine Ergänzung, Schlussformel und Unterschrift)}
\buchAbdrucke{\weitereDrucke{Arthur Schnitzler: \emph{Briefe 1913–1931}. Frankfurt am Main: \emph{S. Fischer} 1984, S. 525–526.} }\toendnotes[C]{\smallbreak}
\pstart
           {\pb}\textcolor{gray}{\textbf{D\textsuperscript{R} ARTHUR SCHNITZLER}}\hfill 16. 1. 1928.\pend
           
\pstart
           \textcolor{gray}{\textbf{WIEN, XVIII.
                        STERNWARTESTRASSE 71\oindex{Sternwartestrasse 71@\textbf{Sternwartestraße 71}, \emph{Wohngebäude (K.WHS)}|pw}.}}\pend
           
\pstart{}Lieber und verehrter Doktor Zweig.\pend\vspace{0.5em}
\pstart
           Ich danke Ihnen herzlichst für all das Herzliche und Gute, das Sie zu meinem »Buch der Sprüche und Bedenken\pwindex{Buch der Sprueche und Bedenken@\emph{Buch der Sprüche und Bedenken}|pw}« sagen und kann
               auch begreifen, dass manche Bemerkungen, die Ihnen nach Unmut und Empfindlichkeit zu
               schmecken scheinen, ein wenig verdrossen haben. Aber ich glaube, auch diese Stellen
               gehören in das Buch\pwindex{Buch der Sprueche und Bedenken@\emph{Buch der Sprüche und Bedenken}|pwv}, es wäre
               eine Unaufrichtigkeit, vielleicht wirklich eine Art von Ueberheblichkeit gewesen, zu
               verschweigen, dass man gelegentlich auch solchen Stimmungen wenn auch nicht gerade
               »unterworfen« ist, sie gelegentlich doch durchzumachen hat. Ich glaube sogar, dass
               mir das seltener passiert als vielen Andern, die in der Oeffentlichkeit stehen und
               dass ich es viel rascher überwinde. Und ich glaube sogar, dass diese Verstimmungen
               oder Ekelgefühle oder Empörungen zuweilen oder zum Teil eher allgemeinen als
               persönlichen Motiven entstammen. Leichtfertigkeit, Unbedenklichkeit und
               Unverschämtheit einer gewissen Sorte von Kritik hat schon früh meine Aufmerksamkeit
               und meinen Unwillen erregt, zu einer Zeit schon als ich für meine Person mit der
               Kritik nicht das Geringste zu tun hatte. Denken Sie, dass einer meiner ersten
               essayistischen Versuche – ich war damals 18 Jahre alt – den Titel führte »Ueber die Grenzen der erlaubten Kritik\pwindex{Grenzen der Kritik@\emph{Grenzen der Kritik}|pw}«. Also
               hier war schon eine Art Problem für mich, lange ehe ich private Erfahrungen zu
               sammeln begann und es wäre sehr möglich, dass ich über das Problem als solches recht
               bald mich etwas ausführlicher äussern werde. Es wird dann gewiss nicht schaden, wenn
               meine Bemerkungen von Salz und Pfeffer eigener Erlebnisse ein wenig gewürzt sein
               sollten.\pend
           
\pstart
           Ich höre, dass in Russland\oindex{Russland@\textbf{Russland}, \emph{A.PCLI}|pw} demnächst eine
                  \label{K_L03741-1v}\edtext{Uebersetzung\pwindex{Sobranie socinenij@\emph{Sobranie sočinenij}|pwv} Ihrer
                  Gesam{[}m{]}elten Werke}{\lemma{\textnormal{\emph{Uebersetzung … Werke}}}\Cendnote{\textnormal{Eine deutschsprachige Werkausgabe gab es zu diesem Zeitpunkt nicht, es handelte
                  sich also nicht um eine Übersetzung. Zur russischen Werkausgabe\pwindex{Sobranie socinenij@\emph{Sobranie sočinenij}|pwkv}{ }Stefan Zweigs\pwindex{Zweig, Stefan 28.11.1881 – 23.02.1942@\textsc{Zweig, Stefan} (28.11.1881 – 23.02.1942), \emph{Schriftsteller/Schriftstellerin}|pwk} mit Vorwort Maxim Gorkis\pwindex{Gorkij, Maxim 1868-03-28 – 1936-06-18@\textsc{Gorkij, Maxim} (1868-03-28 – 1936-06-18), \emph{Schriftsteller/Schriftstellerin}|pwk} und Einleitung\pwindex{Stefan Zweig. Versuch eines Bildnisses@\emph{Stefan Zweig. Versuch eines Bildnisses}|pwkv} von Richard
                     Specht\pwindex{Specht, Richard 07.12.1870 – 18.03.1932@\textsc{Specht, Richard} (07.12.1870 – 18.03.1932), \emph{Schriftsteller/Schriftstellerin, Journalist/Journalistin, Kritiker/Kritikerin}|pwk}, die zwischen 1927 und 1932 in zwölf
                  Bänden beim Verlag \emph{Wremja}\orgindex{Wremja@Wremja|pwk} entstand, vgl.
                     Konstantin Asadowski: \emph{Stefan Zweig in der UdSSR}.
                     In: Ders.: \emph{Russisch-deutsche Verflechtungen. Ausgewählte
                        Beiträge zur Literatur- und Kulturgeschichte des 19. und
                        20. Jahrhunderts}. Herausgegeben von Fedor Poljakov und Natalia
                     Bakshi. In: \emph{Schriftenreihe des Instituts für russisch-deutsche
                        Literatur- {\kaufmannsund} Kulturbeziehungen an der RGGU
                        Moskau, Band 24}, Paderborn: \emph{Fink
                        Brill}{ }2022, S. 291–313, hier S. 298–299.}}}\label{K_L03741-1} erscheinen soll. (Richard Specht\pwindex{Specht, Richard 07.12.1870 – 18.03.1932@\textsc{Specht, Richard} (07.12.1870 – 18.03.1932), \emph{Schriftsteller/Schriftstellerin, Journalist/Journalistin, Kritiker/Kritikerin}|pw} hat mich dieser Tage seine
               wohlgelungene Vorrede\pwindex{Stefan Zweig. Versuch eines Bildnisses@\emph{Stefan Zweig. Versuch eines Bildnisses}|pwv} dazu
               lesen lassen). Nun sind {\pb}doch gewiss vorher schon viele
               Ihrer Werke in russischer\oindex{Russland@\textbf{Russland}, \emph{A.PCLI}|pw} Sprache
               herausgekommen, ohne dass man Sie dafür bezahlt oder auch nur \introOben{}davon\introOben{} verständigt hätte. Mir ist das so ziemlich mit allen meinen Werken
               geschehen. Werden nun für Ihre russische\oindex{Russland@\textbf{Russland}, \emph{A.PCLI}|pw}{ }Gesamtausgabe\pwindex{Sobranie socinenij@\emph{Sobranie sočinenij}|pwv} diese
               nichtautorisierten Uebersetzungen benützt? Sind Sie gegen den Weitervertrieb dieser
               älteren, gestohlenen Ausgaben auf irgend eine Weise geschützt? Können Sie mir, wenn
               auch nicht ziffermässig genau, doch ein Wort über die Bedingungen sagen, zu welchen
               jener (welcher\orgindex{Wremja@Wremja|pwv}?) russische\oindex{Russland@\textbf{Russland}, \emph{A.PCLI}|pw} Verlag Ihre Gesamtausgabe\pwindex{Sobranie socinenij@\emph{Sobranie sočinenij}|pwv} erworben hat? Ich
               meinerseits werde vorläufig von den bolschewistischen Verlegern (auf legalem Wege)
               genau so bestohlen, wie früher von den zaristischen und es scheint auch nicht zu
               gelingen meinen neuen Roman\pwindex{Therese. Chronik eines Frauenlebens@\emph{Therese. Chronik eines Frauenlebens}|pwv}
               vor Erscheinen in deutscher Ausgabe, die nahe bevorsteht, in Russland\oindex{Russland@\textbf{Russland}, \emph{A.PCLI}|pw} unterzubringen. Natürlich wird er, sobald er nur in
               deutscher Sprache da ist, wie alle meine früheren Sachen »honorarfrei« ins Russische\oindex{Russland@\textbf{Russland}, \emph{A.PCLI}|pw} übersetzt werden. Vielleicht können
               Sie mir zu diesem Thema gelegentlich etwas Nützliches sagen?\pend
           
\pstart
           An Ihren meisterlichen »Sternstunden« der
                  Menschheit\pwindex{Sternstunden der Menschheit@\emph{Sternstunden der Menschheit}|pw} habe ich eine rechte Freude gehabt und wünsche das versprochene
               neue »Dreimeisterbuch\pwindex{Drei Dichter ihres Lebens. Casanova – Stendhal – Tolstoi@\emph{Drei Dichter ihres Lebens. Casanova – Stendhal – Tolstoi}|pwv}« und
               die kleine Komödie\pwindex{Quiproquo. Komoedie in drei Akten@\emph{Quiproquo. Komödie in drei Akten}|pwv} bald
               herbei. Auch ich bin mit allerlei, wenn auch zum Teil nur sehr spielerisch,
               beschäftigt.\pend
           
\pstart
           Auf Wiedersehen und herzlichste Grüsse{\\[\baselineskip]}{[}hs.:{]} Ihr{\\[\baselineskip]}\spacefill\mbox{Arthur Schnitzler}\pend
           \leftskip=0em{}
\pstart
           \noindent{}{[}ms.:{]} Herrn Dr. Stephan Zweig,\pend
           
\pstart
           Salzburg\oindex{Salzburg@\textbf{Salzburg}, \emph{A.ADM2}|pw}.\pend
           \selectlanguage{ngerman}\endnumbering\briefempfaengerindex{Zweig, Stefan@\textsc{Zweig, Stefan}!zzzSchnitzler, Arthur@\emph{von Arthur Schnitzler}!1928-01-161@{16. 1. 1928}|)be}\mylabel{L03741h}
\begin{anhang}
\end{anhang}\normalsize

\doendnotes{C}
\bigskip
\vfill

\clearpage

\footnotesize

\lohead{\textsc{register}}

% Definiere theindex-Environment komplett neu ohne reledmac
\makeatletter
\renewenvironment{theindex}{%
  \section*{\indexname}%
  \setlength{\parindent}{0pt}%
  \setlength{\parskip}{0pt plus 0.3pt}%
  \let\item\@idxitem
}{%
  \clearpage
}
\makeatother

\IfFileExists{\jobname-pw.ind}{\input{\jobname-pw.ind}}{}

\end{document}

      