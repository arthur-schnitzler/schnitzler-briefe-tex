%% latex-leseansicht-vorspann.tex
%% Vorspann für die Leseansicht.
%% Lädt die gemeinsame Datei latex-vorspann.tex mit nicht gesetztem Schalter.

\newif\ifkorrekturansicht
\korrekturansichtfalse

\input{../tex-inputs/latex-vorspann}


\section[ Paul Goldmann an Arthur Schnitzler, {[}28. – 31. 3. 1898?{]}]{L02849 Paul Goldmann an Arthur Schnitzler,  [28. – 31. 3. 1898?]}
\nopagebreak\mylabel{L02849v}
\rehead{ }\normalsize\beginnumbering\briefempfaengerindex{Schnitzler, Arthur@\textsc{Schnitzler, Arthur}!zzzGoldmann, Paul@\emph{von Paul Goldmann}!1898-03-311@{{[}28. – 31. 3. 1898?{]}}|(be}
\toendnotes[C]{\smallbreak\pagebreak[2]}
\correspDesc{Versand  durch Paul Goldmann im Zeitraum [28. – 31. 3. 1898?] in [Frankfurt am Main]
\newline{}Erhalt  durch Arthur Schnitzler im Zeitraum [29. 3. 1898 – 5. 4. 1898?] in Wien}\toendnotes[C]{\smallbreak}
\Standort{DLA, A:Schnitzler, HS.NZ85.1.3168.}
\physDesc{Brief, 1 Blatt, 4 Seiten, 1251 Zeichen
\newline{}Handschrift: schwarze Tinte, deutsche Kurrent
\newline{}Schnitzler: 1) mit Bleistift »Ende \textsc{März 98}« vermerkt  2) mit rotem Buntstift zwei Unterstreichungen}\toendnotes[C]{\smallbreak}
\pstart
           \raggedleft{}{\pb}\textcolor{gray}{\textbf{\strikeout{\label{K_L02849-1v}\edtext{Große Eſchenheimerſtraße 1\oindex{Große Eschenheimer Straße@\textbf{Große Eschenheimer Straße}, \emph{Straße}|pw}}{\lemma{\textnormal{\emph{Große Eschenheimerstraße 1}}}\Cendnote{\textnormal{Adresse von Goldmanns\pwindex{Goldmann, Paul 31.\,1.\,1865 Breslau – 25.\,9.\,1935 Wien@\textsc{Goldmann, Paul} (31.\,1.\,1865 Breslau – 25.\,9.\,1935 Wien), \emph{Schriftsteller, Journalist}|pwk} Schwester Vally\pwindex{Rosengart, Vally 29.\,12.\,1866 Breslau – nach 1926@\textsc{Rosengart, Vally} (29.\,12.\,1866 Breslau – nach 1926)|pwk} und seinem Schwager Josef Rosengart\pwindex{Rosengart, Josef 8.\,2.\,1860 Laupheim – 4.\,8.\,1927 Frankfurt am Main@\textsc{Rosengart, Josef} (8.\,2.\,1860 Laupheim – 4.\,8.\,1927 Frankfurt am Main), \emph{Arzt}|pwk}}}}\label{K_L02849-1}.}}}\pend
           
\pstart\center{}Mein lieber Freund,\pend\vspace{0.5em}
\pstart
           Ich danke Dir für Deinen lieben Brief, den ich hier\oindex{Frankfurt am Main@\textbf{Frankfurt am Main}, \emph{Hauptstadt}|pwv} fand.\pend
           
\pstart
           Es geht nicht, nach \textsc{Wien\oindex{Wien@\textbf{Wien}, \emph{Verwaltungsgebiet}|pw}} zu kommen. Die Zeit reicht nicht aus. Es thut mir unendlich leid, daß ich{ }ſo
               hinausfahren{ }ſoll, ohne einen guten Händedruck von Dir mitzunehmen.\pend
           
\pstart
           \label{K_L02849-2v}\edtext{Samſtag}{\lemma{\textnormal{\emph{Samstag}}}\Cendnote{\textnormal{Damit dürfte der 2. 4. 1898 gemeint sein. Goldmann\pwindex{Goldmann, Paul 31.\,1.\,1865 Breslau – 25.\,9.\,1935 Wien@\textsc{Goldmann, Paul} (31.\,1.\,1865 Breslau – 25.\,9.\,1935 Wien), \emph{Schriftsteller, Journalist}|pwk} kam spätestens am 4. 4. 1898 in
                     Genua\oindex{Genua@\textbf{Genua}|pwk} an (vgl. XXXX Auszeichnungsfehler: Dokument L02847 nicht gefunden). Schnitzler datierte den vorliegenden Brief auf »Ende \textsc{März 98}«, was den Schluss zulässt, dass er zwischen Montag, 28., und Donnerstag, 31. 3. 1898, verfasst wurde.}}}\label{K_L02849-2}{ }früh fahre ich von hier\oindex{Frankfurt am Main@\textbf{Frankfurt am Main}, \emph{Hauptstadt}|pwv} nach \textsc{Genua\oindex{Genua@\textbf{Genua}|pw}}. Am 5.{ }ſteige ich dort aufs Schiff. {\pb}Ich habe viel Angſt vor der Seekrankheit und noch
               mehr davor, daß ich den \strikeout{\textcolor{gray}{G}} Aufgaben meiner Reiſe \label{K_L02849-3v}\edtext{journaliſtiſch-ſchriftſtelleriſch nicht gewachſen}{\lemma{\textnormal{\emph{journalistisch-schriftstellerisch nicht gewachsen}}}\Cendnote{\textnormal{Dessen ungeachtet entstand in dieser Zeit die erste
                  Feuilletonsammlung Goldmanns\pwindex{Goldmann, Paul 31.\,1.\,1865 Breslau – 25.\,9.\,1935 Wien@\textsc{Goldmann, Paul} (31.\,1.\,1865 Breslau – 25.\,9.\,1935 Wien), \emph{Schriftsteller, Journalist}|pwk}: \emph{Ein Sommer in China}\pwindex{Goldmann, Paul 31.\,1.\,1865 Breslau – 25.\,9.\,1935 Wien@\textsc{Goldmann, Paul} (31.\,1.\,1865 Breslau – 25.\,9.\,1935 Wien), \emph{Schriftsteller, Journalist}!Sommer in China. Reisebilder@\strich\emph{Ein Sommer in China. Reisebilder}|pwk} (Frankfurt am Main\oindex{Frankfurt am Main@\textbf{Frankfurt am Main}, \emph{Hauptstadt}|pwk}: \emph{Rütten {\kaufmannsund} Loening}\orgindex{Rütten und Loening@Rütten {\kaufmannsund}  Loening|pwk}{ }1899, 2 Bände).}}}\label{K_L02849-3}{ }ſein werde.\pend
           
\pstart
           Es freut mich unendlich, daß Du \label{K_L02849-4v}\edtext{arbeiteſt}{\lemma{\textnormal{\emph{arbeitest}}}\Cendnote{\textnormal{womöglich Bezug auf die
                  Fertigstellung von \emph{Die Gefährtin}\pwindex{Schnitzler, Arthur 15.\,5.\,1862 Wien – 21.\,10.\,1931 ebd.@\textsc{Schnitzler, Arthur} (15.\,5.\,1862 Wien – 21.\,10.\,1931 ebd.), \emph{Schriftsteller, Mediziner}!Gefährtin. Schauspiel in einem Akt@\strich\emph{Die Gefährtin. Schauspiel in einem Akt}|pwk}, vgl. A. S.: \emph{Tagebuch}, 28. 3. 1898.}}}\label{K_L02849-4}. Laß’ Deine Stimmung{ }ſein, wie{ }ſie will, und arbeite weiter. Dadurch wird am
               Ende auch die Stimmung beſſer werden. Alle Mißſtimmung kommt ja doch nur daher, daß
                  {\pb}man \strikeout{\textcolor{gray}{×}} über{ }ſich nachdenkt. Das muß man unter allen Umſtänden vermeiden, und Arbeit
               iſt das beſte Mittel hierzu.\pend
           
\pstart
           Schreib’ mir, bitte, noch ein Wort über Dein Ergehen nach \textsc{Genova\oindex{Genua@\textbf{Genua}|pw}}, \label{K_L02849-5v}\edtext{\begin{otherlanguage}{italian}\textsc{ferma in posta}\end{otherlanguage}}{\lemma{\textnormal{\emph{ferma in posta}}}\Cendnote{\textnormal{italienisch: postlagernd}}}\label{K_L02849-5}. Auch
               während ich unterwegs bin, mußt Du mir regelmäßig über Dich berichten. Ich theile Dir
               noch das Nähere über Adreſſe u. Sonſtiges mit.\pend
           
\pstart
           {\pb}Vor meiner Abreiſe aus \textsc{Paris\oindex{Paris@\textbf{Paris}, \emph{Hauptstadt}|pw}} war ich noch ein oder zwei Mal mit \label{K_L02849-6v}\edtext{\textsc{Frau}{ }\substVorne{}\textsuperscript{\textsc{Bahr}}\substDazwischen{}\textsc{Bahr\pwindex{Bahr, Rosa 26.\,10.\,1871 Prag – 17.\,2.\,1940 Berlin@\textsc{Bahr, Rosa} (26.\,10.\,1871 Prag – 17.\,2.\,1940 Berlin), \emph{Schauspielerin}|pw}}\substHinten{} zuſammen (Saumenſch}{\lemma{\textnormal{\emph{Frau … (Saumensch}}}\Cendnote{\textnormal{Der Umgang
                  mit Rosa Bahr\pwindex{Bahr, Rosa 26.\,10.\,1871 Prag – 17.\,2.\,1940 Berlin@\textsc{Bahr, Rosa} (26.\,10.\,1871 Prag – 17.\,2.\,1940 Berlin), \emph{Schauspielerin}|pwk} wurde von mehreren Seiten als
                  schwierig geschildert.}}}\label{K_L02849-6}!)\pend
           
\pstart
           Die Meinigen haben Alle viel nach Dir gefragt und grüßen Dich herzlich.\pend
           
\pstart
           Grüße mir den \textsc{Richard\pwindex{Beer-Hofmann, Richard 11.\,7.\,1866 Wien – 26.\,9.\,1945 New York City@\textsc{Beer-Hofmann, Richard} (11.\,7.\,1866 Wien – 26.\,9.\,1945 New York City), \emph{Schriftsteller}|pw}} und den \textsc{Leo\pwindex{Van-Jung, Leo 15.\,10.\,1866 Odessa – 2.\,7.\,1939 Riga@\textsc{Van-Jung, Leo} (15.\,10.\,1866 Odessa – 2.\,7.\,1939 Riga), \emph{Gesangspädagoge, Mathematiker}|pw}} und{ }ſei Du{ }ſelbſt von Herzen gegrüßt!\pend
           
\pstart
           Dein treuer {\\[\baselineskip]}\spacefill\mbox{Paul Goldmann}\pend
           \leftskip=0em{}\selectlanguage{ngerman}\endnumbering\briefempfaengerindex{Schnitzler, Arthur@\textsc{Schnitzler, Arthur}!zzzGoldmann, Paul@\emph{von Paul Goldmann}!1898-03-281@{{[}28. – 31. 3. 1898?{]}}|)be}\mylabel{L02849h}  \newcommand{\dateiname}{L02849}\newcommand{\titel}{Paul Goldmann an Arthur Schnitzler, [28. – 31. 3. 1898?]}\newcommand{\editorInnen}{Martin Anton Müller und Laura Untner}%% latex-leseansicht-abspann.tex
%% Abspann für die Leseansicht.
%% Der Schalter \ifkorrekturansicht ist bereits durch den Vorspann gesetzt.

%% latex-abspann.tex
%% Gemeinsamer Abspann für Korrekturansicht und Leseansicht.
%% Setzt den Schalter \ifkorrekturansicht voraus (gesetzt in den
%% einbindenden Dateien latex-korrekturansicht-abspann.tex bzw.
%% latex-leseansicht-abspann.tex).
%% ---------------------------------------------------------------

\normalsize

% Das esempio-Environment wird nur in der Leseansicht benötigt
\ifkorrekturansicht\else
\newenvironment{esempio}[3]%
{
    \vspace{1.5ex}
    \rlap{\underline{#1}}
    \par
    \setlength{\parindent}{0cm}
    \nopagebreak
    \leftskip=#2cm
    \rightskip=#3cm
}
{
    \par
}
\fi

\doendnotes{C}
\bigskip
\vfill

\clearpage

\footnotesize

\ifkorrekturansicht
  \lohead{\textsc{register}}
\fi

% theindex-Environment neu definieren ohne reledmac
\makeatletter
\renewenvironment{theindex}{%
  \ifkorrekturansicht
    \section*{\indexname}%
  \else
    \subsubsection*{Index der erwähnten Entitäten}%
  \fi
  \setlength{\parindent}{0pt}%
  \setlength{\parskip}{0pt plus 0.3pt}%
  \let\item\@idxitem
}{%
  \ifkorrekturansicht\clearpage\fi
}
\makeatother

\IfFileExists{\jobname-pw.ind}{\input{\jobname-pw.ind}}{}

% Quellenangabe nur in der Leseansicht
\ifkorrekturansicht\else
% Fallback-Definitionen, falls die .tex-Datei \titel etc. nicht gesetzt hat
\providecommand{\titel}{}
\providecommand{\editorInnen}{}
\providecommand{\dateiname}{\jobname}

\vspace{3cm}

\vfill

\footnotesize
\textsc{Quelle}: \titel. Herausgegeben von {\editorInnen}. In: \emph{Arthur Schnitzler: Briefwechsel mit Autorinnen und Autoren}.
 Digitale Edition, https://schnitzler-briefe.acdh.oeaw.ac.at/{\dateiname}.html (Stand \today)
\fi

\end{document}


