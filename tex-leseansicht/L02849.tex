%% latex-korrekturansicht-vorspann.tex
%% Vorspann für die Korrekturansicht.
%% Lädt die gemeinsame Datei latex-vorspann.tex mit gesetztem Schalter.

\newif\ifkorrekturansicht
\korrekturansichttrue

\input{../tex-inputs/latex-vorspann}


\section[ Paul Goldmann an Arthur Schnitzler, {[}28. – 31. 3. 1898?{]}]{L02849 Paul Goldmann an Arthur Schnitzler, {[}28. – 31. 3. 1898?{]}}
\nopagebreak\mylabel{L02849v}
\rehead{ }\normalsize\beginnumbering\briefempfaengerindex{Schnitzler, Arthur@\textsc{Schnitzler, Arthur}!zzzGoldmann, Paul@\emph{von Paul Goldmann}!1898-03-311@{{[}28. – 31. 3. 1898?{]}}|(be}
\toendnotes[C]{\smallbreak\pagebreak[2]}\Standort{DLA, A:Schnitzler, HS.NZ85.1.3168.}
\physDesc{Brief, 1 Blatt, 4 Seiten, 1251 Zeichen
\newline{}Handschrift: schwarze Tinte, deutsche Kurrent
\newline{}Schnitzler: 1) mit Bleistift »Ende \textsc{März 98}« vermerkt  2) mit rotem Buntstift zwei Unterstreichungen}\toendnotes[C]{\smallbreak}
\pstart
           \raggedleft{}{\pb}\textcolor{gray}{\textbf{\strikeout{\label{K_L02849-1v}\edtext{Große Eſchenheimerſtraße 1\oindex{Grosse Eschenheimer Strasse@\textbf{Große Eschenheimer Straße}, \emph{Straße (K.STR)}|pw}}{\lemma{\textnormal{\emph{Große Eſchenheimerſtraße 1}}}\Cendnote{\textnormal{Adresse von Goldmanns\pwindex{Goldmann, Paul 31.01.1865 – 25.09.1935@\textsc{Goldmann, Paul} (31.01.1865 – 25.09.1935), \emph{Schriftsteller/Schriftstellerin, Journalist/Journalistin}|pwk} Schwester Vally\pwindex{Rosengart, Vally 1866-12-29 – nach 1926@\textsc{Rosengart, Vally} (1866-12-29 – nach 1926)|pwk} und seinem Schwager Josef Rosengart\pwindex{Rosengart, Josef 1860-02-08 – 1927-08-04@\textsc{Rosengart, Josef} (1860-02-08 – 1927-08-04), \emph{Arzt/Ärztin}|pwk}}}}\label{K_L02849-1}.}}}\pend
           
\pstart\center{}Mein lieber Freund,\pend\vspace{0.5em}
\pstart
           Ich danke Dir für Deinen lieben Brief, den ich hier\oindex{Frankfurt am Main@\textbf{Frankfurt am Main}, \emph{P.PPLA3}|pwv} fand.\pend
           
\pstart
           Es geht nicht, nach \textsc{Wien\oindex{Wien@\textbf{Wien}, \emph{A.ADM2}|pw}} zu kommen. Die Zeit reicht nicht aus. Es thut mir unendlich leid, daß ich ſo
               hinausfahren ſoll, ohne einen guten Händedruck von Dir mitzunehmen.\pend
           
\pstart
           \label{K_L02849-2v}\edtext{Samſtag}{\lemma{\textnormal{\emph{Samſtag}}}\Cendnote{\textnormal{Damit dürfte der 2. 4. 1898 gemeint sein. Goldmann\pwindex{Goldmann, Paul 31.01.1865 – 25.09.1935@\textsc{Goldmann, Paul} (31.01.1865 – 25.09.1935), \emph{Schriftsteller/Schriftstellerin, Journalist/Journalistin}|pwk} kam spätestens am 4. 4. 1898 in
                     Genua\oindex{Genua@\textbf{Genua}, \emph{P.PPLA}|pwk} an (vgl. Paul Goldmann an Arthur Schnitzler, 4. 4. 1898). Schnitzler datierte den vorliegenden Brief auf »Ende \textsc{März 98}«, was den Schluss zulässt, dass er zwischen Montag, 28., und Donnerstag,
                  31. 3. 1898, verfasst wurde.}}}\label{K_L02849-2}{ }früh fahre ich von hier\oindex{Frankfurt am Main@\textbf{Frankfurt am Main}, \emph{P.PPLA3}|pwv} nach \textsc{Genua\oindex{Genua@\textbf{Genua}, \emph{P.PPLA}|pw}}. Am 5. ſteige ich dort aufs Schiff. {\pb}Ich habe viel Angſt vor der Seekrankheit und noch
               mehr davor, daß ich den \strikeout{\textcolor{gray}{G}} Aufgaben meiner Reiſe \label{K_L02849-3v}\edtext{journaliſtiſch-ſchriftſtelleriſch nicht gewachſen}{\lemma{\textnormal{\emph{journaliſtiſch-ſchriftſtelleriſch nicht gewachſen}}}\Cendnote{\textnormal{Dessen ungeachtet entstand in dieser Zeit die erste
                  Feuilletonsammlung Goldmanns\pwindex{Goldmann, Paul 31.01.1865 – 25.09.1935@\textsc{Goldmann, Paul} (31.01.1865 – 25.09.1935), \emph{Schriftsteller/Schriftstellerin, Journalist/Journalistin}|pwk}: \emph{Ein Sommer in China}\pwindex{Sommer in China. Reisebilder@\emph{Ein Sommer in China. Reisebilder}|pwk} (Frankfurt am Main\oindex{Frankfurt am Main@\textbf{Frankfurt am Main}, \emph{P.PPLA3}|pwk}: \emph{Rütten {\kaufmannsund} Loening}\orgindex{Ruetten und Loening@Rütten {\kaufmannsund}  Loening|pwk}{ }1899, 2 Bände).}}}\label{K_L02849-3} ſein werde.\pend
           
\pstart
           Es freut mich unendlich, daß Du \label{K_L02849-4v}\edtext{arbeiteſt}{\lemma{\textnormal{\emph{arbeiteſt}}}\Cendnote{\textnormal{womöglich Bezug auf die
                  Fertigstellung von \emph{Die Gefährtin}\pwindex{Gefaehrtin. Schauspiel in einem Akt@\emph{Die Gefährtin. Schauspiel in einem Akt}|pwk}, vgl. A. S.: \emph{Tagebuch}, 28. 3. 1898.}}}\label{K_L02849-4}. Laß’ Deine Stimmung ſein, wie ſie will, und arbeite weiter. Dadurch wird am
               Ende auch die Stimmung beſſer werden. Alle Mißſtimmung kommt ja doch nur daher, daß
                  {\pb}man \strikeout{\textcolor{gray}{×}} über ſich nachdenkt. Das muß man unter allen Umſtänden vermeiden, und Arbeit
               iſt das beſte Mittel hierzu.\pend
           
\pstart
           Schreib’ mir, bitte, noch ein Wort über Dein Ergehen nach \textsc{Genova\oindex{Genua@\textbf{Genua}, \emph{P.PPLA}|pw}}, \label{K_L02849-5v}\edtext{\begin{otherlanguage}{italian}\textsc{ferma in posta}\end{otherlanguage}}{\lemma{\textnormal{\emph{ferma in posta}}}\Cendnote{\textnormal{italienisch: postlagernd}}}\label{K_L02849-5}. Auch
               während ich unterwegs bin, mußt Du mir regelmäßig über Dich berichten. Ich theile Dir
               noch das Nähere über Adreſſe u. Sonſtiges mit. \pend
           
\pstart
           {\pb}Vor meiner Abreiſe aus \textsc{Paris\oindex{Paris@\textbf{Paris}, \emph{P.PPLC}|pw}} war ich noch ein oder zwei Mal mit \label{K_L02849-6v}\edtext{\textsc{Frau}{ }\substVorne{}\textsuperscript{\textsc{Bahr}}\substDazwischen{}\textsc{Bahr\pwindex{Bahr, Rosa 26.10.1871 – 17.02.1940@\textsc{Bahr, Rosa} (26.10.1871 – 17.02.1940), \emph{Schauspieler/Schauspielerin}|pw}}\substHinten{} zuſammen (Saumenſch}{\lemma{\textnormal{\emph{Frau … (Saumenſch}}}\Cendnote{\textnormal{Der Umgang
                  mit Rosa Bahr\pwindex{Bahr, Rosa 26.10.1871 – 17.02.1940@\textsc{Bahr, Rosa} (26.10.1871 – 17.02.1940), \emph{Schauspieler/Schauspielerin}|pwk} wurde von mehreren Seiten als
                  schwierig geschildert.}}}\label{K_L02849-6}!) \pend
           
\pstart
           Die Meinigen haben Alle viel nach Dir gefragt und grüßen Dich herzlich.\pend
           
\pstart
           Grüße mir den \textsc{Richard\pwindex{Beer-Hofmann, Richard 1866-07-11 – 1945-09-26@\textsc{Beer-Hofmann, Richard} (1866-07-11 – 1945-09-26), \emph{Schriftsteller/Schriftstellerin}|pw}} und den \textsc{Leo\pwindex{Van-Jung, Leo 15.10.1866 – 02.07.1939@\textsc{Van-Jung, Leo} (15.10.1866 – 02.07.1939), \emph{Gesangspädagoge/Gesangspädagogin, Mathematiker/Mathematikerin}|pw}} und ſei Du ſelbſt von Herzen gegrüßt!\pend
           
\pstart
           Dein treuer {\\[\baselineskip]}\spacefill\mbox{Paul Goldmann}\pend
           \leftskip=0em{}\selectlanguage{ngerman}\endnumbering\briefempfaengerindex{Schnitzler, Arthur@\textsc{Schnitzler, Arthur}!zzzGoldmann, Paul@\emph{von Paul Goldmann}!1898-03-281@{{[}28. – 31. 3. 1898?{]}}|)be}\mylabel{L02849h}  \normalsize

\doendnotes{C}
\bigskip
\vfill

\clearpage

\footnotesize

\lohead{\textsc{register}}

% Definiere theindex-Environment komplett neu ohne reledmac
\makeatletter
\renewenvironment{theindex}{%
  \section*{\indexname}%
  \setlength{\parindent}{0pt}%
  \setlength{\parskip}{0pt plus 0.3pt}%
  \let\item\@idxitem
}{%
  \clearpage
}
\makeatother

\IfFileExists{\jobname-pw.ind}{\input{\jobname-pw.ind}}{}

\end{document}

      