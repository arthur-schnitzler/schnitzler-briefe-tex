%% latex-leseansicht-vorspann.tex
%% Vorspann für die Leseansicht.
%% Lädt die gemeinsame Datei latex-vorspann.tex mit nicht gesetztem Schalter.

\newif\ifkorrekturansicht
\korrekturansichtfalse

\input{../tex-inputs/latex-vorspann}


\section[Stefan Zweig an Arthur Schnitzler, 15. 3. 1913]{L03642 Stefan Zweig an Arthur Schnitzler, 15. 3. 1913}
\nopagebreak\mylabel{L03642v}
\rehead{ }\normalsize\beginnumbering\briefempfaengerindex{Schnitzler, Arthur@\textsc{Schnitzler, Arthur}!zzzZweig, Stefan@\emph{von Stefan Zweig}!1913-03-151@{15. 3. 1913}|(be}
\toendnotes[C]{\smallbreak\pagebreak[2]}
\correspDesc{Versand  durch Stefan Zweig am 15. 3. 1913 in Paris
\newline{}Erhalt  durch Arthur Schnitzler im Zeitraum [16. 3. 1913
                  – 20. 3. 1913?] in Wien}\toendnotes[C]{\smallbreak}
\Standort{CUL, Schnitzler, B 118.}
\physDesc{Brief, 1 Blatt, 3 Seiten, 2001 Zeichen
\newline{}Handschrift: schwarze Tinte, lateinische Kurrent
\newline{}Schnitzler: 1) mit Bleistift »\textsc{Zweig}«  2) mit rotem Buntstift zwei Unterstreichungen}
\buchAbdrucke{\weitereDrucke{Stefan Zweig: \emph{Briefwechsel mit Hermann Bahr, Sigmund Freud, Rainer Maria
                        Rilke und Arthur Schnitzler}. Herausgegeben von Jeffrey B. Berlin, Hans-Ulrich Lindken und Donald A. Prater. Frankfurt am Main: \emph{S. Fischer} 1987, S. 372–374.} }\toendnotes[C]{\smallbreak}
\pstart
           {\pb}Hotel Beaujolais\oindex{Hôtel de Beaujolais@\textbf{Hôtel de Beaujolais}, \emph{Hotel}|pw}\hfill 15. März 1913\pend
           
\pstart
           15, rue de Beaujolais\oindex{Hôtel de Beaujolais@\textbf{Hôtel de Beaujolais}, \emph{Hotel}|pw}\pend
           
\pstart
           Paris\oindex{Paris@\textbf{Paris}, \emph{Hauptstadt}|pw} –\pend
           
\pstart{}Verehrter Herr Doktor,\pend\vspace{0.5em}
\pstart
           seit einiger Zeit \label{K_L03642-1v}\edtext{in Paris\oindex{Paris@\textbf{Paris}, \emph{Hauptstadt}|pw}}{\lemma{\textnormal{\emph{in Paris}}}\Cendnote{\textnormal{Zweig\pwindex{Zweig, Stefan 28.\,11.\,1881 Wien – 23.\,2.\,1942 Petrópolis@\textsc{Zweig, Stefan} (28.\,11.\,1881 Wien – 23.\,2.\,1942 Petrópolis), \emph{Schriftsteller}|pwk} verbrachte die Zeit vom
                  4. 3. 1913 bis zum 23. 4. 1913 in Paris\oindex{Paris@\textbf{Paris}, \emph{Hauptstadt}|pwk}.}}}\label{K_L03642-1} habe ich heute Paul Morisse\pwindex{Morisse, Paul 11.\,3.\,1866 Rouen – 28.\,9.\,1946 Paris@\textsc{Morisse, Paul} (11.\,3.\,1866 Rouen – 28.\,9.\,1946 Paris), \emph{Übersetzer}|pw} zum erstenmal gesprochen und eile
               mich, Ihnen sein Stillschweigen zu erklären. Morisse\pwindex{Morisse, Paul 11.\,3.\,1866 Rouen – 28.\,9.\,1946 Paris@\textsc{Morisse, Paul} (11.\,3.\,1866 Rouen – 28.\,9.\,1946 Paris), \emph{Übersetzer}|pw} hat Ihr Stück\pwindex{Schnitzler, Arthur 15.\,5.\,1862 Wien – 21.\,10.\,1931 ebd.@\textsc{Schnitzler, Arthur} (15.\,5.\,1862 Wien – 21.\,10.\,1931 ebd.), \emph{Schriftsteller, Mediziner}!weite Land. Tragikomödie in fünf Akten@\strich\emph{Das weite Land. Tragikomödie in fünf Akten}|pwv}
               längst übersetzt, sogar eigens in München\oindex{München@\textbf{München}|pw} einer
               Aufführung beigewohnt und gibt sich alle Mühe. Wenn er Ihnen \label{K_L03642-2v}\edtext{nicht schrieb}{\lemma{\textnormal{\emph{nicht schrieb}}}\Cendnote{\textnormal{Im Nachlass 
                  Schnitzlers ist kein Korrespondenzstück von Morisse\pwindex{Morisse, Paul 11.\,3.\,1866 Rouen – 28.\,9.\,1946 Paris@\textsc{Morisse, Paul} (11.\,3.\,1866 Rouen – 28.\,9.\,1946 Paris), \emph{Übersetzer}|pwk} überliefert, das nach dem 14. 4. 1912 
               abgefasst wurde.}}}\label{K_L03642-2}, so war es
               einzig die Scheu, nichts Negatives melden zu wollen. Es bedeutet ja für Sie nichts
               Peinliches, wenn ich es nun übernehme Ihnen zu sagen, dass bei zwei Theatern\orgindex{Odéon@Odéon|pwv} seine Schritte vergeblich gewesen
               sind, so sehr man das Werk\pwindex{Schnitzler, Arthur 15.\,5.\,1862 Wien – 21.\,10.\,1931 ebd.@\textsc{Schnitzler, Arthur} (15.\,5.\,1862 Wien – 21.\,10.\,1931 ebd.), \emph{Schriftsteller, Mediziner}!weite Land. Tragikomödie in fünf Akten@\strich\emph{Das weite Land. Tragikomödie in fünf Akten}|pwv}
               rühmte, auch Antoine\pwindex{Antoine, André 31.\,1.\,1858 Limoges – 23.\,10.\,1943 Le Pouliguen@\textsc{Antoine, André} (31.\,1.\,1858 Limoges – 23.\,10.\,1943 Le Pouliguen), \emph{Theaterleiter, Schauspieler}|pw} konnte sich nicht
               entscheiden. Augenb{[}lick{]}lich liegt es beim Theater des Variétés\orgindex{Théâtre des Variétés@Théâtre des Variétés|pw}, wo die Hoffnungen auf schwachen Füssen
               stehn, besonders bei der jetzi{\pb}gen
               politischen Lage, wo die Aufführung deutscher Werke geringer Sympathie begegnet.\pend
           
\pstart
           Sicher wäre das \uline{Theater des Arts\orgindex{Théâtre Hébertot@Théâtre Hébertot|pw}} das jetzt modernste von Paris\oindex{Paris@\textbf{Paris}, \emph{Hauptstadt}|pw}, das Shaw\pwindex{Shaw, George Bernard 26.\,7.\,1856 Dublin – 2.\,11.\,1950 Ayot Saint Lawrence@\textsc{Shaw, George Bernard} (26.\,7.\,1856 Dublin – 2.\,11.\,1950 Ayot Saint Lawrence), \emph{Schriftsteller}|pw}, Hebbel\pwindex{Hebbel, Friedrich 18.\,3.\,1813 Wesselburen – 13.\,12.\,1863 Wien@\textsc{Hebbel, Friedrich} (18.\,3.\,1813 Wesselburen – 13.\,12.\,1863 Wien), \emph{Schriftsteller}|pw}, die jungen Franzosen\oindex{Frankreich@\textbf{Frankreich}|pw} spielt. Es
               ist natürlich ein \begin{otherlanguage}{french}a-coté\end{otherlanguage}-Theater und trägt gar
               nichts oder beinahe so viel: Morisse\pwindex{Morisse, Paul 11.\,3.\,1866 Rouen – 28.\,9.\,1946 Paris@\textsc{Morisse, Paul} (11.\,3.\,1866 Rouen – 28.\,9.\,1946 Paris), \emph{Übersetzer}|pw} wagte
               Ihnen dies nicht anzubieten, etwas Deklassierendes ist \strikeout{natürlich} dabei nicht zu finden und die Presse vollzählig vertreten. Hier
               müssten Sie entscheiden.\pend
           
\pstart
           Auch ist er bereit, das Werk\pwindex{Schnitzler, Arthur 15.\,5.\,1862 Wien – 21.\,10.\,1931 ebd.@\textsc{Schnitzler, Arthur} (15.\,5.\,1862 Wien – 21.\,10.\,1931 ebd.), \emph{Schriftsteller, Mediziner}!weite Land. Tragikomödie in fünf Akten@\strich\emph{Das weite Land. Tragikomödie in fünf Akten}|pwv}
               sofort als Buch erscheinen zu lassen, nur soll dies in Frankreich\oindex{Frankreich@\textbf{Frankreich}|pw} gewissermassen einen schweigenden Verzicht auf die Aufführung
               bedeuten.\pend
           
\pstart
           Ich hoffe, verehrter Herr Doktor, klar berichtet zu haben. Morisse\pwindex{Morisse, Paul 11.\,3.\,1866 Rouen – 28.\,9.\,1946 Paris@\textsc{Morisse, Paul} (11.\,3.\,1866 Rouen – 28.\,9.\,1946 Paris), \emph{Übersetzer}|pw} hat sich alle Mühe gegeben, Sie wissen ja selbst,
               wie schwer Paris\oindex{Paris@\textbf{Paris}, \emph{Hauptstadt}|pw} zu erobern ist. Jedesfalls stehe
               ich hier ganz zu Ihrer Verfügung, falls Sie irgend eine bestimmte Aus{\pb}kunft wünschen, ich bleibe noch drei Wochen
                  zumindest. Mein Leben ist hier vielfältig
               durch die Stadt\oindex{Paris@\textbf{Paris}, \emph{Hauptstadt}|pwv} und doch
               geschlossener durch das Fremdsein, das nur die Freundschaft einiger guter Menschen
               zum doppelten Glück macht. Bewahren Sie mir gutes Gedenken, überbringen Sie Ihrer
               Frau Gemahlin\pwindex{Schnitzler, Olga 17.\,1.\,1882 Wien – 13.\,1.\,1970 Lugano@\textsc{Schnitzler, Olga} (17.\,1.\,1882 Wien – 13.\,1.\,1970 Lugano), \emph{Schauspielerin, Sängerin}|pwv} beste
               Empfehlungen und seien Sie aufrichtigst gegrüsst von Ihrem treu ergebenen\pend
           \pstart \spacefill\mbox{Stefan Zweig}\pend{}
\pstart
           \noindent{}Paul Morissen’s\pwindex{Morisse, Paul 11.\,3.\,1866 Rouen – 28.\,9.\,1946 Paris@\textsc{Morisse, Paul} (11.\,3.\,1866 Rouen – 28.\,9.\,1946 Paris), \emph{Übersetzer}|pw} Adresse ist{\\}Mercure de France\orgindex{Mercure de France@Mercure de France|pw}{\\}26, rue de Condé\oindex{Hôtel Charles-Testu@\textbf{Hôtel Charles-Testu}, \emph{Gebäude}|pw}\pend
           \selectlanguage{ngerman}\endnumbering\briefempfaengerindex{Schnitzler, Arthur@\textsc{Schnitzler, Arthur}!zzzZweig, Stefan@\emph{von Stefan Zweig}!1913-03-151@{15. 3. 1913}|)be}\mylabel{L03642h}  \newcommand{\dateiname}{L03642}\newcommand{\titel}{Stefan Zweig an Arthur Schnitzler, 15. 3. 1913}\newcommand{\editorInnen}{Selma Jahnke und Martin Anton Müller}%% latex-leseansicht-abspann.tex
%% Abspann für die Leseansicht.
%% Der Schalter \ifkorrekturansicht ist bereits durch den Vorspann gesetzt.

%% latex-abspann.tex
%% Gemeinsamer Abspann für Korrekturansicht und Leseansicht.
%% Setzt den Schalter \ifkorrekturansicht voraus (gesetzt in den
%% einbindenden Dateien latex-korrekturansicht-abspann.tex bzw.
%% latex-leseansicht-abspann.tex).
%% ---------------------------------------------------------------

\normalsize

% Das esempio-Environment wird nur in der Leseansicht benötigt
\ifkorrekturansicht\else
\newenvironment{esempio}[3]%
{
    \vspace{1.5ex}
    \rlap{\underline{#1}}
    \par
    \setlength{\parindent}{0cm}
    \nopagebreak
    \leftskip=#2cm
    \rightskip=#3cm
}
{
    \par
}
\fi

\doendnotes{C}
\bigskip
\vfill

\clearpage

\footnotesize

\ifkorrekturansicht
  \lohead{\textsc{register}}
\fi

% theindex-Environment neu definieren ohne reledmac
\makeatletter
\renewenvironment{theindex}{%
  \ifkorrekturansicht
    \section*{\indexname}%
  \else
    \subsubsection*{Index der erwähnten Entitäten}%
  \fi
  \setlength{\parindent}{0pt}%
  \setlength{\parskip}{0pt plus 0.3pt}%
  \let\item\@idxitem
}{%
  \ifkorrekturansicht\clearpage\fi
}
\makeatother

\IfFileExists{\jobname-pw.ind}{\input{\jobname-pw.ind}}{}

% Quellenangabe nur in der Leseansicht
\ifkorrekturansicht\else
% Fallback-Definitionen, falls die .tex-Datei \titel etc. nicht gesetzt hat
\providecommand{\titel}{}
\providecommand{\editorInnen}{}
\providecommand{\dateiname}{\jobname}

\vspace{3cm}

\vfill

\footnotesize
\textsc{Quelle}: \titel. Herausgegeben von {\editorInnen}. In: \emph{Arthur Schnitzler: Briefwechsel mit Autorinnen und Autoren}.
 Digitale Edition, https://schnitzler-briefe.acdh.oeaw.ac.at/{\dateiname}.html (Stand \today)
\fi

\end{document}


