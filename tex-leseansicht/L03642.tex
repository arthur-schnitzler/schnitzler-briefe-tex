%% latex-korrekturansicht-vorspann.tex
%% Vorspann für die Korrekturansicht.
%% Lädt die gemeinsame Datei latex-vorspann.tex mit gesetztem Schalter.

\newif\ifkorrekturansicht
\korrekturansichttrue

\input{../tex-inputs/latex-vorspann}


\section[Stefan Zweig an Arthur Schnitzler, 15. 3. 1913]{L03642 Stefan Zweig an Arthur Schnitzler, 15. 3. 1913}
\nopagebreak\mylabel{L03642v}
\rehead{ }\normalsize\beginnumbering\briefempfaengerindex{Schnitzler, Arthur@\textsc{Schnitzler, Arthur}!zzzZweig, Stefan@\emph{von Stefan Zweig}!1913-03-151@{15. 3. 1913}|(be}
\toendnotes[C]{\smallbreak\pagebreak[2]}\Standort{CUL, Schnitzler, B 118.}
\physDesc{Brief, 1 Blatt, 3 Seiten, 2004 Zeichen
\newline{}Handschrift: schwarze Tinte, lateinische Kurrent
\newline{}Schnitzler: 1) mit Bleistift »\textsc{Zweig}«  2) mit rotem Buntstift eine Unterstreichung}
\buchAbdrucke{\weitereDrucke{Stefan Zweig: \emph{Briefwechsel mit Hermann Bahr, Sigmund Freud, Rainer Maria
                        Rilke und Arthur Schnitzler}. Frankfurt am Main: \emph{S. Fischer} 1987, S. 372–374.} }\toendnotes[C]{\smallbreak}
\pstart
           {\pb}Hotel
                     Beaujolais\oindex{Hôtel de Beaujolais@\textbf{Hôtel de Beaujolais}, \emph{Hotel (K.HTL)}|pw}\pend
           
\pstart
           15, rue de Beaujolais\oindex{Hôtel de Beaujolais@\textbf{Hôtel de Beaujolais}, \emph{Hotel (K.HTL)}|pw}\pend
           
\pstart
           Paris\oindex{Paris@\textbf{Paris}, \emph{P.PPLC}|pw} –\pend
           
\pstart
           \raggedleft{}15. März 1913\pend
           
\pstart{}Verehrter Herr Doktor,\pend\vspace{0.5em}
\pstart
           seit einiger Zeit in Paris\oindex{Paris@\textbf{Paris}, \emph{P.PPLC}|pw} habe ich heute Paul Morisse\pwindex{Morisse, Paul 1866-03-11 – 1946-09-28@\textsc{Morisse, Paul} (1866-03-11 – 1946-09-28), \emph{Übersetzer/Übersetzerin}|pw} zum erstenmal gesprochen und eile
               mich, Ihnen sein Stillschweigen zu erklären. Morisse\pwindex{Morisse, Paul 1866-03-11 – 1946-09-28@\textsc{Morisse, Paul} (1866-03-11 – 1946-09-28), \emph{Übersetzer/Übersetzerin}|pw} hat Ihr Stück\pwindex{weite Land. Tragikomoedie in fuenf Akten@\emph{Das weite Land. Tragikomödie in fünf Akten}|pwv} längst übersetzt, sogar
               eigens in München\oindex{Muenchen@\textbf{München}, \emph{P.PPLA}|pw} einer Aufführung beigewohnt und gibt
               sich alle Mühe. Wenn er Ihnen nicht schrieb, so war es einzig die Scheu, nichts
               Negatives melden zu wollen. Es bedeutet ja für Sie nichts Peinliches, wenn ich es nun
               übernehme Ihnen zu sagen, dass bei zwei Theatern seine Schritte vergeblich gewesen
               sind, so sehr man das Werk\pwindex{weite Land. Tragikomoedie in fuenf Akten@\emph{Das weite Land. Tragikomödie in fünf Akten}|pwv} rühmte, auch Antoine\pwindex{Antoine, Andre 1858-01-31 – 1943-10-23@\textsc{Antoine, André} (1858-01-31 – 1943-10-23), \emph{Theaterleiter/Theaterleiterin, Schauspieler/Schauspielerin}|pw} konnte sich nicht entscheiden. Augenb{[}lick{]}lich liegt es beim Theater des Variétés\oindex{Theâtre des Varietes@\textbf{Théâtre des Variétés}, \emph{Theater (K.THE)}|pw}, wo die Hoffnungen auf schwachen Füssen
               stehn, besonders bei der jetzi{\pb}gen politischen Lage, wo die
               Aufführung deutscher Werke geringer Sympathie begegnet.\pend
           
\pstart
           Sicher wäre das Theater des ArtsXXXX ORGangabe fehlt das jetzt modernste von Paris\oindex{Paris@\textbf{Paris}, \emph{P.PPLC}|pw}, das Shaw\pwindex{Shaw, George Bernard 26.07.1856 – 02.11.1950@\textsc{Shaw, George Bernard} (26.07.1856 – 02.11.1950), \emph{Schriftsteller/Schriftstellerin}|pw}, Hebbel\pwindex{Hebbel, Friedrich 18.03.1813 – 13.12.1863@\textsc{Hebbel, Friedrich} (18.03.1813 – 13.12.1863), \emph{Schriftsteller/Schriftstellerin}|pw}, die jungen Franzosen\oindex{Frankreich@\textbf{Frankreich}, \emph{A.PCLI}|pw} spielt. Es ist natürlich
                  ein \begin{otherlanguage}{french}a-coté\end{otherlanguage}-Theater und trägt gar nichts oder beinahe so viel: Morisse\pwindex{Morisse, Paul 1866-03-11 – 1946-09-28@\textsc{Morisse, Paul} (1866-03-11 – 1946-09-28), \emph{Übersetzer/Übersetzerin}|pw} wagte Ihnen dies nicht anzubieten, etwas
                  Deklassierendes ist \strikeout{natürlic}h dabei nicht zu
                  finden und die Presse vollzählig vertreten. Hier müssten Sie entscheiden.\pend
           
\pstart
           Auch ist
                     er bereit, das Werk\pwindex{weite Land. Tragikomoedie in fuenf Akten@\emph{Das weite Land. Tragikomödie in fünf Akten}|pwv} sofort als Buch erscheinen zu
                     lassen, nur soll dies in Frankreich\oindex{Frankreich@\textbf{Frankreich}, \emph{A.PCLI}|pw} gewissermassen
                     einen schweigenden Verzicht auf die Aufführung bedeuten.\pend
           
\pstart
           Ich hoffe, verehrter Herr
                  Doktor, klar berichtet zu haben. Morisse\pwindex{Morisse, Paul 1866-03-11 – 1946-09-28@\textsc{Morisse, Paul} (1866-03-11 – 1946-09-28), \emph{Übersetzer/Übersetzerin}|pw}
                  hat sich alle Mühe gegeben, Sie wissen ja selbst, wie schwer Paris\oindex{Paris@\textbf{Paris}, \emph{P.PPLC}|pw} zu erobern ist. Jedesfalls stehe ich hier ganz zu Ihrer
                  Verfügung, falls Sie irgend eine bestimmte Aus{\pb}kunft wünschen, ich bleibe noch
                  \label{K_L03642-1v}\edtext{drei Wochen}{\lemma{\textnormal{\emph{drei Wochen}}}\Cendnote{\textnormal{Stefan Zweig\pwindex{Zweig, Stefan 28.11.1881 – 23.02.1942@\textsc{Zweig, Stefan} (28.11.1881 – 23.02.1942), \emph{Schriftsteller/Schriftstellerin}|pwk} verbrachte die Zeit vom 4. 3. bis zum 23. 4. 1913 in Paris\oindex{Paris@\textbf{Paris}, \emph{P.PPLC}|pwk}.}}}\label{K_L03642-1} zumindest. Mein Leben ist hier vielfältig durch die Stadt\oindex{Paris@\textbf{Paris}, \emph{P.PPLC}|pwv} und doch
                  geschlossener durch das Fremdsein, das nur die Freundschaft einiger guter Menschen
                  zum doppelten Glück macht. Bewahren Sie mir gutes Gedenken, überbringen Sie Ihrer
                  Frau Gemahlin\pwindex{Schnitzler, Olga 17.01.1882 – 13.01.1970@\textsc{Schnitzler, Olga} (17.01.1882 – 13.01.1970), \emph{Schauspieler/Schauspielerin, Sänger/Sängerin}|pwv} beste
                        Empfehlungen und seien Sie aufrichtigst gegrüsst von Ihrem treu ergebenen\pend
           \pstart \spacefill\mbox{Stefan Zweig}\pend{}
\pstart
           \noindent{}Paul Morissen's\pwindex{Morisse, Paul 1866-03-11 – 1946-09-28@\textsc{Morisse, Paul} (1866-03-11 – 1946-09-28), \emph{Übersetzer/Übersetzerin}|pw} Adresse
                     ist{\\}Mercure de France\orgindex{Mercure de France@Mercure de France|pw}{\\}
                     26, rue de Condé\oindex{Hôtel Charles-Testu@\textbf{Hôtel Charles-Testu}, \emph{Gebäude (K.GBD)}|pw}\pend
           \selectlanguage{ngerman}\endnumbering\briefempfaengerindex{Schnitzler, Arthur@\textsc{Schnitzler, Arthur}!zzzZweig, Stefan@\emph{von Stefan Zweig}!1913-03-151@{15. 3. 1913}|)be}\mylabel{L03642h}
\begin{anhang}
\end{anhang}\normalsize

\doendnotes{C}
\bigskip
\vfill

\clearpage

\footnotesize

\lohead{\textsc{register}}

% Definiere theindex-Environment komplett neu ohne reledmac
\makeatletter
\renewenvironment{theindex}{%
  \section*{\indexname}%
  \setlength{\parindent}{0pt}%
  \setlength{\parskip}{0pt plus 0.3pt}%
  \let\item\@idxitem
}{%
  \clearpage
}
\makeatother

\IfFileExists{\jobname-pw.ind}{\input{\jobname-pw.ind}}{}

\end{document}

      