%% latex-korrekturansicht-vorspann.tex
%% Vorspann für die Korrekturansicht.
%% Lädt die gemeinsame Datei latex-vorspann.tex mit gesetztem Schalter.

\newif\ifkorrekturansicht
\korrekturansichttrue

\input{../tex-inputs/latex-vorspann}


\section[ Felix Salten an Arthur Schnitzler, 30. 3. 1921]{L03570 Felix Salten an Arthur Schnitzler, 30. 3. 1921}
\nopagebreak\mylabel{L03570v}
\rehead{ }\normalsize\beginnumbering\briefempfaengerindex{Schnitzler, Arthur@\textsc{Schnitzler, Arthur}!zzzSalten, Felix@\emph{von Felix Salten}!1921-03-302@{30. 3. 1921}|(be}
\toendnotes[C]{\smallbreak\pagebreak[2]}\Standort{CUL, Schnitzler, B 89, B 2.}
\physDesc{Bildpostkarte, 221 Zeichen
\newline{}Handschrift: schwarze Tinte, lateinische Kurrent
\newline{}Versand: Stempel: »\nobreak{}\oindex{Hildesheim@\textbf{Hildesheim}, \emph{P.PPLA3}|pwk}Hildesheim 2, 30. 3. 21, 6–7 N\nobreak{}«.  
\newline{}Ordnung: 1) mit Bleistift von Frieda Pollak\pwindex{Pollak, Frieda 08.12.1881 – 13.07.1937@\textsc{Pollak, Frieda} (08.12.1881 – 13.07.1937), \emph{Sekretär/Sekretärin}|pw} (?) mit
                                 dem Buchstaben »A« (Abgeschrieben/Abschrift)
                                 gekennzeichnet  2) mit Bleistift von unbekannter Hand nummeriert: »283«}\toendnotes[C]{\smallbreak}\pstart{}{\pb}Herrn\pend{}\pstart{}D\textsuperscript{r} Arthur Schnitzler\pend{}\pstart{}Wien\oindex{Wien@\textbf{Wien}, \emph{A.ADM2}|pw}\pend{}\pstart{}XVIII. Sternwartestraße 71\oindex{Sternwartestrasse 71@\textbf{Sternwartestraße 71}, \emph{Wohngebäude (K.WHS)}|pw}\pend{}{\bigskip}
\pstart
           {\pb}\textcolor{gray}{\textbf{Hildesheim\oindex{Hildesheim@\textbf{Hildesheim}, \emph{P.PPLA3}|pw}.}}\hfill \textcolor{gray}{\textbf{Tempelherrenhaus\oindex{Tempelhaus [Hildesheim]@\textbf{Tempelhaus [Hildesheim]}, \emph{Sakralbau (K.SAK)}|pw}.}}\pend
           \vspace{1em}
\pstart
           \noindent{}{\pb}Lieber,{ }hier\oindex{Hildesheim@\textbf{Hildesheim}, \emph{P.PPLA3}|pwv} verbringe ich, ganz
               unverhofft, einen stillen Tag. Die Stadt\oindex{Hildesheim@\textbf{Hildesheim}, \emph{P.PPLA3}|pwv} ist verblüffend schön. Morgen bin ich
               in Berlin\oindex{Berlin@\textbf{Berlin}, \emph{P.PPLC}|pw}.\pend
           
\pstart
           Alles Herzliche Ihr {\\[\baselineskip]}\spacefill\mbox{Felix Salten}\pend
           \leftskip=0em{}
\pstart
           Hildesheim\oindex{Hildesheim@\textbf{Hildesheim}, \emph{P.PPLA3}|pw}, 30. 3. 21\pend
           \selectlanguage{ngerman}\endnumbering\briefempfaengerindex{Schnitzler, Arthur@\textsc{Schnitzler, Arthur}!zzzSalten, Felix@\emph{von Felix Salten}!1921-03-302@{30. 3. 1921}|)be}\mylabel{L03570h}  \normalsize

\doendnotes{C}
\bigskip
\vfill

\clearpage

\footnotesize

\lohead{\textsc{register}}

% Definiere theindex-Environment komplett neu ohne reledmac
\makeatletter
\renewenvironment{theindex}{%
  \section*{\indexname}%
  \setlength{\parindent}{0pt}%
  \setlength{\parskip}{0pt plus 0.3pt}%
  \let\item\@idxitem
}{%
  \clearpage
}
\makeatother

\IfFileExists{\jobname-pw.ind}{\input{\jobname-pw.ind}}{}

\end{document}

      