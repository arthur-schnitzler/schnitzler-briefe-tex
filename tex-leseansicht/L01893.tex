%% latex-leseansicht-vorspann.tex
%% Vorspann für die Leseansicht.
%% Lädt die gemeinsame Datei latex-vorspann.tex mit nicht gesetztem Schalter.

\newif\ifkorrekturansicht
\korrekturansichtfalse

\input{../tex-inputs/latex-vorspann}


\section[Arthur Schnitzler an Richard Beer-Hofmann, 7. 12. 1909]{L01893 Arthur Schnitzler an Richard Beer-Hofmann, 7. 12. 1909}
\nopagebreak\mylabel{L01893v}
\rehead{ }\normalsize\beginnumbering\briefempfaengerindex{Beer-Hofmann, Richard@\textsc{Beer-Hofmann, Richard}!zzzSchnitzler, Arthur@\emph{von Arthur Schnitzler}!1909-12-071@{7. 12. 1909}|(be}
\toendnotes[C]{\smallbreak\pagebreak[2]}
\correspDesc{Versand  durch Arthur Schnitzler am 7. 12. 1909 in Wien
\newline{}Erhalt  durch Richard Beer-Hofmann am 7. 12. 1909 in Wien}\toendnotes[C]{\smallbreak}
\Standort{YCGL, MSS 31.}
\physDesc{Visitenkarte, 258 Zeichen
\newline{}Handschrift: Bleistift, deutsche Kurrent}
\buchAbdrucke{\weitereDrucke{Arthur Schnitzler, Richard Beer-Hofmann: \emph{Briefwechsel 1891–1931}. Herausgegeben von Konstanze Fliedl. Wien, Zürich: \emph{Europaverlag} 1992, S. 196.} }\toendnotes[C]{\smallbreak}
\pstart
           \raggedleft{}{\pb}\substVorne{}\textsuperscript{6}\substDazwischen{}7\substHinten{}/12 09.\pend
           
\pstart
           \centering{}\textcolor{gray}{\textbf{D\textsuperscript{r} Arthur Schnitzler}}\pend
           \vspace{0.5em}
\pstart
           {\pb}lieber Richard, ich höre von
               verſchiedenen Seiten dſs im Apollo-Theater\oindex{Wien@\textbf{Wien}!VI., Mariahilf@\textbf{VI., Mariahilf}!Apollo-Theater [Wien]@\textbf{Apollo-Theater [Wien]}, \emph{Theater}|pw} ein
                  \label{K_L01893-1v}\edtext{Plagiat\pwindex{\textcolor{red}{\textsuperscript{XXXX indx1}}!schwarze Mali. Sketch@\strich\emph{Die schwarze Mali. Sketch}|pwv}\pwindex{\textcolor{red}{\textsuperscript{XXXX indx1}}!schwarze Mali. Sketch@\strich\emph{Die schwarze Mali. Sketch}|pwv}}{\lemma{\textnormal{\emph{Plagiat}}}\Cendnote{\textnormal{In \emph{Ma
                     gosse}\pwindex{\textcolor{red}{\textsuperscript{XXXX indx1}}!schwarze Mali. Sketch@\strich\emph{Die schwarze Mali. Sketch}|pwk}\pwindex{\textcolor{red}{\textsuperscript{XXXX indx1}}!schwarze Mali. Sketch@\strich\emph{Die schwarze Mali. Sketch}|pwk} (\emph{Die schwarze Mali}\pwindex{\textcolor{red}{\textsuperscript{XXXX indx1}}!schwarze Mali. Sketch@\strich\emph{Die schwarze Mali. Sketch}|pwk}\pwindex{\textcolor{red}{\textsuperscript{XXXX indx1}}!schwarze Mali. Sketch@\strich\emph{Die schwarze Mali. Sketch}|pwk}) heuert ein
                  Wirt Schauspieler an, die den Gästen ein Kriminalstück vorspielen.}}}\label{K_L01893-1} des Kakadu\pwindex{Schnitzler, Arthur 15.\,5.\,1862 Wien – 21.\,10.\,1931 ebd.@\textsc{Schnitzler, Arthur} (15.\,5.\,1862 Wien – 21.\,10.\,1931 ebd.), \emph{Schriftsteller, Mediziner}!grüne Kakadu. Groteske in einem Akt@\strich\emph{Der grüne Kakadu. Groteske in einem Akt}|pw} »nach dem franzöſiſchen« geſpielt wird; es
               liegt mir daran die Sache{ }ſo bald als möglich zu{ }ſehen – wollen Sie \introOben{}Beide\pwindex{Beer-Hofmann, Paula 25.\,2.\,1879 Wien – 30.\,10.\,1939 Zürich@\textsc{Beer-Hofmann, Paula} (25.\,2.\,1879 Wien – 30.\,10.\,1939 Zürich)|pwv}\introOben{} heute mit uns eine Loge\pwindex{\textcolor{red}{\textsuperscript{XXXX indx1}}!schwarze Mali. Sketch@\strich\emph{Die schwarze Mali. Sketch}|pwv}\pwindex{\textcolor{red}{\textsuperscript{XXXX indx1}}!schwarze Mali. Sketch@\strich\emph{Die schwarze Mali. Sketch}|pwv}
               nehmen? Herzlichst Ihr\pend
           \pstart \spacefill\mbox{A.}\pend{}\selectlanguage{ngerman}\endnumbering\briefempfaengerindex{Beer-Hofmann, Richard@\textsc{Beer-Hofmann, Richard}!zzzSchnitzler, Arthur@\emph{von Arthur Schnitzler}!1909-12-071@{7. 12. 1909}|)be}\mylabel{L01893h}  \newcommand{\dateiname}{L01893}\newcommand{\titel}{Arthur Schnitzler an Richard Beer-Hofmann, 7. 12. 1909}\newcommand{\editorInnen}{Martin Anton Müller und Gerd-Hermann Susen}%% latex-leseansicht-abspann.tex
%% Abspann für die Leseansicht.
%% Der Schalter \ifkorrekturansicht ist bereits durch den Vorspann gesetzt.

%% latex-abspann.tex
%% Gemeinsamer Abspann für Korrekturansicht und Leseansicht.
%% Setzt den Schalter \ifkorrekturansicht voraus (gesetzt in den
%% einbindenden Dateien latex-korrekturansicht-abspann.tex bzw.
%% latex-leseansicht-abspann.tex).
%% ---------------------------------------------------------------

\normalsize

% Das esempio-Environment wird nur in der Leseansicht benötigt
\ifkorrekturansicht\else
\newenvironment{esempio}[3]%
{
    \vspace{1.5ex}
    \rlap{\underline{#1}}
    \par
    \setlength{\parindent}{0cm}
    \nopagebreak
    \leftskip=#2cm
    \rightskip=#3cm
}
{
    \par
}
\fi

\doendnotes{C}
\bigskip
\vfill

\clearpage

\footnotesize

\ifkorrekturansicht
  \lohead{\textsc{register}}
\fi

% theindex-Environment neu definieren ohne reledmac
\makeatletter
\renewenvironment{theindex}{%
  \ifkorrekturansicht
    \section*{\indexname}%
  \else
    \subsubsection*{Index der erwähnten Entitäten}%
  \fi
  \setlength{\parindent}{0pt}%
  \setlength{\parskip}{0pt plus 0.3pt}%
  \let\item\@idxitem
}{%
  \ifkorrekturansicht\clearpage\fi
}
\makeatother

\IfFileExists{\jobname-pw.ind}{\input{\jobname-pw.ind}}{}

% Quellenangabe nur in der Leseansicht
\ifkorrekturansicht\else
% Fallback-Definitionen, falls die .tex-Datei \titel etc. nicht gesetzt hat
\providecommand{\titel}{}
\providecommand{\editorInnen}{}
\providecommand{\dateiname}{\jobname}

\vspace{3cm}

\vfill

\footnotesize
\textsc{Quelle}: \titel. Herausgegeben von {\editorInnen}. In: \emph{Arthur Schnitzler: Briefwechsel mit Autorinnen und Autoren}.
 Digitale Edition, https://schnitzler-briefe.acdh.oeaw.ac.at/{\dateiname}.html (Stand \today)
\fi

\end{document}


