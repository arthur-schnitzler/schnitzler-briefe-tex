%% latex-korrekturansicht-vorspann.tex
%% Vorspann für die Korrekturansicht.
%% Lädt die gemeinsame Datei latex-vorspann.tex mit gesetztem Schalter.

\newif\ifkorrekturansicht
\korrekturansichttrue

\input{../tex-inputs/latex-vorspann}


\section[Arthur Schnitzler an Richard Beer-Hofmann, 7. 12. 1909]{L01893 Arthur Schnitzler an Richard Beer-Hofmann, 7. 12. 1909}
\nopagebreak\mylabel{L01893v}
\rehead{ }\normalsize\beginnumbering\briefempfaengerindex{Beer-Hofmann, Richard@\textsc{Beer-Hofmann, Richard}!zzzSchnitzler, Arthur@\emph{von Arthur Schnitzler}!1909-12-071@{7. 12. 1909}|(be}
\toendnotes[C]{\smallbreak\pagebreak[2]}\Standort{YCGL, MSS 31.}
\physDesc{Visitenkarte, 258 Zeichen
\newline{}Handschrift: Bleistift, deutsche Kurrent}
\buchAbdrucke{\weitereDrucke{Arthur Schnitzler, Richard Beer-Hofmann: \emph{Briefwechsel 1891–1931}. Wien, Zürich: \emph{Europaverlag} 1992, S. 196.} }\toendnotes[C]{\smallbreak}
\pstart
           \raggedleft{}{\pb}\substVorne{}\textsuperscript{6}\substDazwischen{}7\substHinten{}/12 09.\pend
           
\pstart
           \centering{}\textcolor{gray}{\textbf{D\textsuperscript{r} Arthur Schnitzler}}\pend
           \vspace{0.5em}
\pstart
           {\pb}lieber Richard, ich höre von
               verſchiedenen Seiten dſs im Apollo-Theater\oindex{Apollo-Theater [Wien]@\textbf{Apollo-Theater [Wien]}, \emph{Theater (K.THE)}|pw} ein
                  \label{K_L01893-1v}\edtext{Plagiat\pwindex{schwarze Mali. Sketch@\emph{Die schwarze Mali. Sketch}|pwv}}{\lemma{\textnormal{\emph{Plagiat}}}\Cendnote{\textnormal{In \emph{Ma
                     gosse}\pwindex{schwarze Mali. Sketch@\emph{Die schwarze Mali. Sketch}|pwk} (\emph{Die schwarze Mali}\pwindex{schwarze Mali. Sketch@\emph{Die schwarze Mali. Sketch}|pwk}) heuert ein
                  Wirt Schauspieler an, die den Gästen ein Kriminalstück vorspielen.}}}\label{K_L01893-1} des Kakadu\pwindex{gruene Kakadu. Groteske in einem Akt@\emph{Der grüne Kakadu. Groteske in einem Akt}|pw} »nach dem franzöſiſchen« geſpielt wird; es
               liegt mir daran die Sache ſo bald als möglich zu ſehen – wollen Sie \introOben{}Beide\pwindex{Beer-Hofmann, Paula 25.02.1879 – 30.10.1939@\textsc{Beer-Hofmann, Paula} (25.02.1879 – 30.10.1939)|pwv}\introOben{} heute mit uns eine Loge\pwindex{schwarze Mali. Sketch@\emph{Die schwarze Mali. Sketch}|pwv}
               nehmen? Herzlichst Ihr\pend
           \pstart \spacefill\mbox{A.}\pend{}\selectlanguage{ngerman}\endnumbering\briefempfaengerindex{Beer-Hofmann, Richard@\textsc{Beer-Hofmann, Richard}!zzzSchnitzler, Arthur@\emph{von Arthur Schnitzler}!1909-12-071@{7. 12. 1909}|)be}\mylabel{L01893h}  \normalsize

\doendnotes{C}
\bigskip
\vfill

\clearpage

\footnotesize

\lohead{\textsc{register}}

% Definiere theindex-Environment komplett neu ohne reledmac
\makeatletter
\renewenvironment{theindex}{%
  \section*{\indexname}%
  \setlength{\parindent}{0pt}%
  \setlength{\parskip}{0pt plus 0.3pt}%
  \let\item\@idxitem
}{%
  \clearpage
}
\makeatother

\IfFileExists{\jobname-pw.ind}{\input{\jobname-pw.ind}}{}

\end{document}

      