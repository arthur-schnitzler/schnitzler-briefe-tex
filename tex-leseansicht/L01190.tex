%% latex-korrekturansicht-vorspann.tex
%% Vorspann für die Korrekturansicht.
%% Lädt die gemeinsame Datei latex-vorspann.tex mit gesetztem Schalter.

\newif\ifkorrekturansicht
\korrekturansichttrue

\input{../tex-inputs/latex-vorspann}


\section[Arthur Schnitzler: Widmungsexemplar Das Märchen für Gerty von Hofmannsthal, 20. 12. 1901]{L01190 Arthur Schnitzler: Widmungsexemplar Das Märchen für Gerty von
               Hofmannsthal, 20. 12. 1901}
\nopagebreak\mylabel{L01190v}
\rehead{ }\normalsize\beginnumbering\briefempfaengerindex{Hofmannsthal, Gertrude von@\textsc{Hofmannsthal, Gertrude von}!zzzSchnitzler, Arthur@\emph{von Arthur Schnitzler}!1901-12-201@{20. 12. 1901}|(be}
\toendnotes[C]{\smallbreak\pagebreak[2]}\Standort{FDH, FDH 3230.}
\physDesc{Widmung am Vorsatzblatt, 98 Zeichen
\newline{}Handschrift: schwarze Tinte, deutsche Kurrent}
\buchAbdrucke{\weitereDrucke{Hugo von Hofmannsthal: \emph{Bibliothek}. Frankfurt am Main: \emph{S. Fischer} 2011, S. 605.} }
\pstart
           \noindent{}{\pb}Frau \textsc{Gerty von
                  Hofmannsthal}{\\}mit herzlichem Gruſs{\\}ein Märchen aus uralten Zeiten\pend
           \pstart \spacefill\mbox{ArthSchn}\pend{}
\pstart
           20. 12. 901.\pend
           \selectlanguage{ngerman}\vspace{1em}{\vspace{1\baselineskip}}
\pstart
           \centering{}{\pb}\textcolor{gray}{\textbf{\textbf{Arthur Schnitzler}}}\pend
           
\pstart
           \centering{}\textcolor{gray}{\textbf{\textbf{\so{Das Märchen}}\pwindex{Maerchen. Schauspiel in drei Aufzuegen@\emph{Das Märchen. Schauspiel in drei Aufzügen}|pw}}}\pend
           
\pstart
           \centering{}\textcolor{gray}{\textbf{Schauſpiel in drei Aufzügen}}\pend
           
\pstart
           \centering{}\textcolor{gray}{\textbf{Zweite Auflage}}\pend
           {\vspace{1\baselineskip}}
\pstart
           \centering{}\textcolor{gray}{\textbf{\so{Berlin}\oindex{Berlin@\textbf{Berlin}, \emph{P.PPLC}|pw}{ }1902}}\pend
           
\pstart
           \centering{}\textcolor{gray}{\textbf{\so{S. Fiſcher, Verlag}\orgindex{S. Fischer Verlag@S. Fischer Verlag|pw}}}\pend
           \selectlanguage{ngerman}\endnumbering\briefempfaengerindex{Hofmannsthal, Gertrude von@\textsc{Hofmannsthal, Gertrude von}!zzzSchnitzler, Arthur@\emph{von Arthur Schnitzler}!1901-12-201@{20. 12. 1901}|)be}\mylabel{L01190h}  \normalsize

\doendnotes{C}
\bigskip
\vfill

\clearpage

\footnotesize

\lohead{\textsc{register}}

% Definiere theindex-Environment komplett neu ohne reledmac
\makeatletter
\renewenvironment{theindex}{%
  \section*{\indexname}%
  \setlength{\parindent}{0pt}%
  \setlength{\parskip}{0pt plus 0.3pt}%
  \let\item\@idxitem
}{%
  \clearpage
}
\makeatother

\IfFileExists{\jobname-pw.ind}{\input{\jobname-pw.ind}}{}

\end{document}

      