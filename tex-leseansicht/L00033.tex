%% latex-leseansicht-vorspann.tex
%% Vorspann für die Leseansicht.
%% Lädt die gemeinsame Datei latex-vorspann.tex mit nicht gesetztem Schalter.

\newif\ifkorrekturansicht
\korrekturansichtfalse

\input{../tex-inputs/latex-vorspann}

\begin{center}
            \textcolor{red}{ENTWURF. ENTZIFFERUNG NOCH NICHT KORREKTURGELESEN}
                      \end{center}
            
               \section[Arthur Schnitzler an Richard Beer-Hofmann, 14. 8. 1891]{ Arthur Schnitzler an Richard Beer-Hofmann,
               14. 8. 1891}\nopagebreak\mylabel{v}\rehead{ }\begin{ledgroupsized}[t]{13cm}\normalsize\beginnumbering\briefempfaengerindex{Beer-Hofmann, Richard@\textsc{Beer-Hofmann, Richard}!zzzSchnitzler, Arthur@\emph{von Arthur Schnitzler}!1891-08-141@{14. 8. 1891}|(be} \toendnotes[C]{\smallbreak\pagebreak[2]} \Standort{YCGL, MSS 31.}
\physDesc{Briefkarte, Umschlag
\newline{}Handschrift: schwarze Tinte, deutsche Kurrent\newline{}Versand: 1) Stempel: »\nobreak{}Wien, 14 8 9\textcolor{gray}{1}, 3.N\nobreak{}«.  2) Stempel: »\nobreak{}\oindex{Bad Aussee@\textbf{Bad Aussee}|pwk}Aussee, {[}15 8{]} 91\nobreak{}«. }\buchAbdrucke{\weitereDrucke{Arthur Schnitzler, Richard Beer-Hofmann: \emph{Briefwechsel 1891–1931}. Hg. Konstanze Fliedl. Wien, Zürich: \emph{Europaverlag} 1992, S. 31–32.} }\pstart{}{\pb}\textsc{Herrn Dr. Rich. Beer-Hofmann}\pend{}\pstart{}\textsc{Aussee}\oindex{Bad Aussee@\textbf{Bad Aussee}|pw}\pend{}\pstart{}\textsc{Steiermark}\oindex{Steiermark@\textbf{Steiermark}|pw}\pend{}{\bigskip}\pstart
           \noindent{}{\pb}Lieber Richard, ko{\geminationm}en Sie, we{\geminationn} es geht, So{\geminationn}tag 16.
                  Auguſt{ }Vormittag nach Iſchl\oindex{Bad Ischl@\textbf{Bad Ischl}|pw}. Meine Adreſſe
               dort \textsc{\uline{Pension Leopold\oindex{Hotel und Pension Rudolfshoehe (Leopold Petter)@\textbf{Hotel und Pension Rudolfshöhe (Leopold Petter)}|pw}}}. Telegrafiren Sie mir eventuell dahin die Stunde Ihrer Ankunft. Ich {\pb}denke, wir fahren dann zu \textsc{Loris}\pwindex{Hofmannsthal, Hugo von 01.02.1874 – 15.07.1929@\textsc{Hofmannsthal, Hugo von} (01.02.1874 – 15.07.1929), \emph{Schriftsteller}|pw} nach \textsc{Strobl\oindex{Strobl@\textbf{Strobl}|pw}} hinüber. Oder, beſſer, ich werde ihn bitten, auch nach Iſchl\oindex{Bad Ischl@\textbf{Bad Ischl}|pw} zu ko{\geminationm}en. Ich freue mich
               ſehr, mit Ihnen beiſa{\geminationm}en zu ſein.\pend
           \pstart Mit herzlichem Gruß Ihr \spacefill\mbox{Arthur.}\pend{}\endnumbering\briefempfaengerindex{Beer-Hofmann, Richard@\textsc{Beer-Hofmann, Richard}!zzzSchnitzler, Arthur@\emph{von Arthur Schnitzler}!1891-08-141@{14. 8. 1891}|)be}\mylabel{h}\end{ledgroupsized}  \newcommand{\dateiname}{L00033}\newcommand{\titel}{Arthur Schnitzler an Richard Beer-Hofmann, 14. 8. 1891}\newcommand{\editorInnen}{Martin Anton Müller und Gerd-Hermann Susen}%% latex-leseansicht-abspann.tex
%% Abspann für die Leseansicht.
%% Der Schalter \ifkorrekturansicht ist bereits durch den Vorspann gesetzt.

%% latex-abspann.tex
%% Gemeinsamer Abspann für Korrekturansicht und Leseansicht.
%% Setzt den Schalter \ifkorrekturansicht voraus (gesetzt in den
%% einbindenden Dateien latex-korrekturansicht-abspann.tex bzw.
%% latex-leseansicht-abspann.tex).
%% ---------------------------------------------------------------

\normalsize

% Das esempio-Environment wird nur in der Leseansicht benötigt
\ifkorrekturansicht\else
\newenvironment{esempio}[3]%
{
    \vspace{1.5ex}
    \rlap{\underline{#1}}
    \par
    \setlength{\parindent}{0cm}
    \nopagebreak
    \leftskip=#2cm
    \rightskip=#3cm
}
{
    \par
}
\fi

\doendnotes{C}
\bigskip
\vfill

\clearpage

\footnotesize

\ifkorrekturansicht
  \lohead{\textsc{register}}
\fi

% theindex-Environment neu definieren ohne reledmac
\makeatletter
\renewenvironment{theindex}{%
  \ifkorrekturansicht
    \section*{\indexname}%
  \else
    \subsubsection*{Index der erwähnten Entitäten}%
  \fi
  \setlength{\parindent}{0pt}%
  \setlength{\parskip}{0pt plus 0.3pt}%
  \let\item\@idxitem
}{%
  \ifkorrekturansicht\clearpage\fi
}
\makeatother

\IfFileExists{\jobname-pw.ind}{\input{\jobname-pw.ind}}{}

% Quellenangabe nur in der Leseansicht
\ifkorrekturansicht\else
% Fallback-Definitionen, falls die .tex-Datei \titel etc. nicht gesetzt hat
\providecommand{\titel}{}
\providecommand{\editorInnen}{}
\providecommand{\dateiname}{\jobname}

\vspace{3cm}

\vfill

\footnotesize
\textsc{Quelle}: \titel. Herausgegeben von {\editorInnen}. In: \emph{Arthur Schnitzler: Briefwechsel mit Autorinnen und Autoren}.
 Digitale Edition, https://schnitzler-briefe.acdh.oeaw.ac.at/{\dateiname}.html (Stand \today)
\fi

\end{document}


      