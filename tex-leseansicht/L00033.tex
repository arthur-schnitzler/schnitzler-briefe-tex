%% latex-korrekturansicht-vorspann.tex
%% Vorspann für die Korrekturansicht.
%% Lädt die gemeinsame Datei latex-vorspann.tex mit gesetztem Schalter.

\newif\ifkorrekturansicht
\korrekturansichttrue

\input{../tex-inputs/latex-vorspann}


\section[Arthur Schnitzler an Richard Beer-Hofmann, 14. 8. 1891]{L00033 Arthur Schnitzler an Richard Beer-Hofmann,14. 8. 1891}
\nopagebreak\mylabel{L00033v}
\rehead{ }\normalsize\beginnumbering\briefempfaengerindex{Beer-Hofmann, Richard@\textsc{Beer-Hofmann, Richard}!zzzSchnitzler, Arthur@\emph{von Arthur Schnitzler}!1891-08-141@{14. 8. 1891}|(be}
\toendnotes[C]{\smallbreak\pagebreak[2]}\Standort{YCGL, MSS 31.}
\physDesc{Briefkarte, , 407 Zeichen
\newline{}Handschrift: 1) schwarze Tinte, deutsche Kurrent\hspace{1em}2) schwarze Tinte, lateinische Kurrent (\noindent{}Adresse)\hspace{1em}
\newline{}Versand: 1) Stempel: »\nobreak{}Wien, 14 8 9\textcolor{gray}{1}, 3.N\nobreak{}«.   2) Stempel: »\nobreak{}\oindex{Bad Aussee@\textbf{Bad Aussee}|pwk}Aussee, {[}15 8{]} 91\nobreak{}«. }
\buchAbdrucke{\weitereDrucke{Arthur Schnitzler, Richard Beer-Hofmann: \emph{Briefwechsel 1891–1931}. Wien, Zürich: \emph{Europaverlag} 1992, S. 31–32.} }\pstart{}{\pb}Herrn Dr. Rich. Beer-Hofmann\pend{}\pstart{}Aussee\oindex{Bad Aussee@\textbf{Bad Aussee}|pw}\pend{}\pstart{}Steiermark\oindex{Steiermark@\textbf{Steiermark}|pw}\pend{}{\bigskip}\vspace{1em}
\pstart
           \noindent{}{\pb}Lieber Richard, ko{\geminationm}en Sie, we{\geminationn} es geht, So{\geminationn}tag 16.
                  Auguſt{ }Vormittag nach Iſchl\oindex{Bad Ischl@\textbf{Bad Ischl}|pw}. Meine Adreſſe
               dort \textsc{\uline{Pension Leopold\oindex{Hotel und Pension Rudolfshoehe (Leopold Petter)@\textbf{Hotel und Pension Rudolfshöhe (Leopold Petter)}|pw}}}. Telegrafiren Sie mir eventuell dahin die Stunde Ihrer Ankunft. Ich {\pb}denke, wir fahren dann zu \textsc{Loris}\pwindex{Hofmannsthal, Hugo von 1.\,2.\,1874 Wien – 15.\,7.\,1929 Rodaun@\textsc{Hofmannsthal, Hugo von} (1.\,2.\,1874 Wien – 15.\,7.\,1929 Rodaun), \emph{Schriftsteller}|pw} nach \textsc{Strobl\oindex{Strobl@\textbf{Strobl}|pw}} hinüber. Oder, beſſer, ich werde ihn bitten, auch nach Iſchl\oindex{Bad Ischl@\textbf{Bad Ischl}|pw} zu ko{\geminationm}en. Ich freue mich{ }ſehr, mit Ihnen beiſa{\geminationm}en zu{ }ſein.\pend
           \pstart Mit herzlichem Gruß Ihr \spacefill\mbox{Arthur.}\pend{}\selectlanguage{ngerman}\endnumbering\briefempfaengerindex{Beer-Hofmann, Richard@\textsc{Beer-Hofmann, Richard}!zzzSchnitzler, Arthur@\emph{von Arthur Schnitzler}!1891-08-141@{14. 8. 1891}|)be}\mylabel{L00033h}  \normalsize

\doendnotes{C}
\bigskip
\vfill

\clearpage

\footnotesize

\lohead{\textsc{register}}

% Definiere theindex-Environment komplett neu ohne reledmac
\makeatletter
\renewenvironment{theindex}{%
  \section*{\indexname}%
  \setlength{\parindent}{0pt}%
  \setlength{\parskip}{0pt plus 0.3pt}%
  \let\item\@idxitem
}{%
  \clearpage
}
\makeatother

\IfFileExists{\jobname-pw.ind}{\input{\jobname-pw.ind}}{}

\end{document}

      