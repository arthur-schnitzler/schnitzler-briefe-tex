\input{../tex-inputs/latex-pdf-vorspann}
\begin{center}
            \textcolor{red}{ENTWURF. ENTZIFFERUNG NOCH NICHT KORREKTURGELESEN}
                      \end{center}
            
               \section[Georg Brandes an Arthur Schnitzler, 18. 10. 1914]{ Georg Brandes an Arthur Schnitzler, 18. 10. 1914}\nopagebreak\mylabel{v}\rehead{ }\begin{ledgroupsized}[t]{13cm}\normalsize\beginnumbering\briefempfaengerindex{Schnitzler, Arthur@\textsc{Schnitzler, Arthur}!zzzBrandes, Georg@\emph{von Georg Brandes}!1914-10-181@{18. 10. 1914}|(be} \toendnotes[C]{\smallbreak\pagebreak[2]} \Standort{CUL, Schnitzler, B 17.}
\physDesc{Postkarte
\newline{}Handschrift: schwarze Tinte, lateinische Kurrent\newline{}Versand: Stempel: »\nobreak{}\oindex{Kopenhagen@\textbf{Kopenhagen}|pwk}Kjøbenhavn, 18. 10. 14, 4–5E\nobreak{}«.  \newline{}Ordnung: mit Bleistift von unbekannter Hand nummeriert: »43« }\buchAbdrucke{\weitereDrucke{Georg Brandes, Arthur Schnitzler: \emph{Ein Briefwechsel}. Hg. Kurt Bergel. Bern: \emph{Francke} 1956, S. 110–111.} }\toendnotes[C]{\smallbreak}\pstart{}{\pb}Herrn Dr. Arthur
                        Schnitzler\pend{}\pstart{}Sternwartestrasse 71\oindex{Sternwartestrasse@\textbf{Sternwartestraße}|pw}\pend{}\pstart{}Wien XVIII\oindex{XVIII., Waehring@\textbf{XVIII., Währing}|pw}\pend{}{\bigskip}\pstart
           \raggedleft{}{\pb}Kopenhagen\oindex{Kopenhagen@\textbf{Kopenhagen}|pw}{ }18. October 14\pend
           \pstart
           Verehrter Freund! Da ich erfuhr, dass die schwedische Akademie\orgindex{Kungliga Vetenskapsakademien@Kungliga Vetenskapsakademien|pw}
               sich trotz ihres früheren Planes
                    entschlossen hatte \label{K_L02198_1v}\edtext{auch in diesem
                        Jahr}{\lemma{\textnormal{\emph{auch in diesem
                        Jahr}}}\Cendnote{\textnormal{1914 wurde der Nobelpreis für
                        Literatur – im Gegensatz zu den Preisen für Physik, Medizin und Chemie –
                        nicht vergeben. 1915 erhielt Romain
                            Rolland\pwindex{Rolland, Romain 29.01.1866 – 30.12.1944@\textsc{Rolland, Romain} (29.01.1866 – 30.12.1944), \emph{Schriftsteller}|pwk} den Literaturnobelpreis.}}}\label{K_L02198_1h} die Preise\orgindex{Nobelpreis@Nobelpreis|pwv} auszuteilen, wendete ich mich an
                    das Comité. Man theilt mir mit:\pend
           \pstart
           Es ist ein purer Irrthum, dass Sie in diesem Jahre vorgeschlagen gewesen, Sie
                    sind überhaupt nie in Vorschlag gekommen. Die Pflicht des Schweigens verbietet
                    ihnen, mir mitzutheilen, \uline{wer} vorgeschlagen ist.
                    Aber man lässt mich verstehen, dass der Preis schon weggegeben ist.\pend
           \pstart
           Wollen Sie im \uline{folgenden} Jahr in Betracht kommen,
                    müssen Sie \uuline{vor Ausgang des kommenden \strikeout{Jahres} Januars} von so vielen und so
                    wuchtigen Namen wie möglich vorgeschlagen werden. Unter diesen ist der meinige
                    Ihnen sicher. Man wird Sie wol aber kaum als »Idealist« auffassen, was die
                    Bedingung ist. – Die schönsten Grüsse in dieser traurigen Zeit.\pend
           \pstart
           Ihr{\\[\baselineskip]}\spacefill\mbox{Georg Brandes}\pend
           \leftskip=0em{}\endnumbering\briefempfaengerindex{Schnitzler, Arthur@\textsc{Schnitzler, Arthur}!zzzBrandes, Georg@\emph{von Georg Brandes}!1914-10-181@{18. 10. 1914}|)be}\mylabel{h}\end{ledgroupsized}  \newcommand{\dateiname}{L02198}\newcommand{\titel}{Georg Brandes an Arthur Schnitzler, 18. 10. 1914}\newcommand{\editorInnen}{Martin Anton Müller und Gerd-Hermann Susen}\input{../tex-inputs/latex-pdf-abspann}
      