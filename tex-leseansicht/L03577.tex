%% latex-leseansicht-vorspann.tex
%% Vorspann für die Leseansicht.
%% Lädt die gemeinsame Datei latex-vorspann.tex mit nicht gesetztem Schalter.

\newif\ifkorrekturansicht
\korrekturansichtfalse

\input{../tex-inputs/latex-vorspann}


\section[ Felix Salten an Arthur Schnitzler, 2. 4. 1922]{L03577 Felix Salten an Arthur Schnitzler,  2. 4. 1922}
\nopagebreak\mylabel{L03577v}
\rehead{ }\normalsize\beginnumbering\briefempfaengerindex{Schnitzler, Arthur@\textsc{Schnitzler, Arthur}!zzzSalten, Felix@\emph{von Felix Salten}!1922-04-021@{2. 4. 1922}|(be}
\toendnotes[C]{\smallbreak\pagebreak[2]}
\correspDesc{Versand  durch Felix Salten am 2. 4. 1922 in Rom
\newline{}Übermittlung  am 3. 4. 1922 in Rom
\newline{}Erhalt  durch Arthur Schnitzler im Zeitraum [3. 4. 1922
                  – 7. 4. 1922?] in Wien}\toendnotes[C]{\smallbreak}
\Standort{CUL, Schnitzler, B 89, B 2.}
\physDesc{Bildpostkarte, 115 Zeichen
\newline{}Handschrift: schwarze Tinte, lateinische Kurrent
\newline{}Versand: Stempel: »\nobreak{}\oindex{Roma Termini@\textbf{Roma Termini}, \emph{Bahnhof}|pwk}Roma Ferrovia, 3. IV 1922, 13–14\nobreak{}«.  
\newline{}Ordnung: 1) mit Bleistift von Frieda Pollak\pwindex{Pollak, Frieda 8.\,12.\,1881 Wien – 13.\,7.\,1937 ebd.@\textsc{Pollak, Frieda} (8.\,12.\,1881 Wien – 13.\,7.\,1937 ebd.), \emph{Sekretärin}|pw} (?) mit
                                 dem Buchstaben »A« (Abgeschrieben/Abschrift)
                                 gekennzeichnet  2) mit Bleistift von unbekannter Hand nummeriert: »2\substVorne{}\textsuperscript{\textcolor{gray}{89}}\substDazwischen{}90\substHinten{}«}\pstart{}{\pb}Austria\oindex{Österreich@\textbf{Österreich}|pw}\pend{}\pstart{}Herrn D\textsuperscript{r} Arthur Schnitzler\pend{}\pstart{}XVIII. Sternwartestraße 71\oindex{Wien@\textbf{Wien}!XVIII., Währing@\textbf{XVIII., Währing}!Sternwartestraße 71@\textbf{Sternwartestraße 71}, \emph{Wohngebäude}|pw}\pend{}\pstart{}Wien (\begin{otherlanguage}{english}Vienna\end{otherlanguage})\oindex{Wien@\textbf{Wien}, \emph{Verwaltungsgebiet}|pw}\pend{}{\bigskip}
\pstart
           \noindent{}\centering{}{\pb}\textcolor{gray}{\textbf{Roma\oindex{Rom@\textbf{Rom}, \emph{Hauptstadt}|pw} – Scalinata di Piazza di Spagna\oindex{Spanische Treppe@\textbf{Spanische Treppe}, \emph{Monument}|pw}}}\pend
           \vspace{1em}
\pstart
           \noindent{}{\pb}Viele herzliche Grüße {\\}Ihr {\\}\spacefill\mbox{Felix Salten}\pend
           
\pstart
           2. 4. 22\pend
           \selectlanguage{ngerman}\endnumbering\briefempfaengerindex{Schnitzler, Arthur@\textsc{Schnitzler, Arthur}!zzzSalten, Felix@\emph{von Felix Salten}!1922-04-021@{2. 4. 1922}|)be}\mylabel{L03577h}  \newcommand{\dateiname}{L03577}\newcommand{\titel}{Felix Salten an Arthur Schnitzler, 2. 4. 1922}\newcommand{\editorInnen}{Martin Anton Müller und Laura Untner}%% latex-leseansicht-abspann.tex
%% Abspann für die Leseansicht.
%% Der Schalter \ifkorrekturansicht ist bereits durch den Vorspann gesetzt.

%% latex-abspann.tex
%% Gemeinsamer Abspann für Korrekturansicht und Leseansicht.
%% Setzt den Schalter \ifkorrekturansicht voraus (gesetzt in den
%% einbindenden Dateien latex-korrekturansicht-abspann.tex bzw.
%% latex-leseansicht-abspann.tex).
%% ---------------------------------------------------------------

\normalsize

% Das esempio-Environment wird nur in der Leseansicht benötigt
\ifkorrekturansicht\else
\newenvironment{esempio}[3]%
{
    \vspace{1.5ex}
    \rlap{\underline{#1}}
    \par
    \setlength{\parindent}{0cm}
    \nopagebreak
    \leftskip=#2cm
    \rightskip=#3cm
}
{
    \par
}
\fi

\doendnotes{C}
\bigskip
\vfill

\clearpage

\footnotesize

\ifkorrekturansicht
  \lohead{\textsc{register}}
\fi

% theindex-Environment neu definieren ohne reledmac
\makeatletter
\renewenvironment{theindex}{%
  \ifkorrekturansicht
    \section*{\indexname}%
  \else
    \subsubsection*{Index der erwähnten Entitäten}%
  \fi
  \setlength{\parindent}{0pt}%
  \setlength{\parskip}{0pt plus 0.3pt}%
  \let\item\@idxitem
}{%
  \ifkorrekturansicht\clearpage\fi
}
\makeatother

\IfFileExists{\jobname-pw.ind}{\input{\jobname-pw.ind}}{}

% Quellenangabe nur in der Leseansicht
\ifkorrekturansicht\else
% Fallback-Definitionen, falls die .tex-Datei \titel etc. nicht gesetzt hat
\providecommand{\titel}{}
\providecommand{\editorInnen}{}
\providecommand{\dateiname}{\jobname}

\vspace{3cm}

\vfill

\footnotesize
\textsc{Quelle}: \titel. Herausgegeben von {\editorInnen}. In: \emph{Arthur Schnitzler: Briefwechsel mit Autorinnen und Autoren}.
 Digitale Edition, https://schnitzler-briefe.acdh.oeaw.ac.at/{\dateiname}.html (Stand \today)
\fi

\end{document}


