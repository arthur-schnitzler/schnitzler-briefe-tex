%% latex-leseansicht-vorspann.tex
%% Vorspann für die Leseansicht.
%% Lädt die gemeinsame Datei latex-vorspann.tex mit nicht gesetztem Schalter.

\newif\ifkorrekturansicht
\korrekturansichtfalse

\input{../tex-inputs/latex-vorspann}


\section[Theodor Herzl an Arthur Schnitzler, 5. 5. 1895]{L03860 Theodor Herzl an Arthur Schnitzler, 5. 5. 1895}
\nopagebreak\mylabel{L03860v}
\rehead{ }\normalsize\beginnumbering\briefempfaengerindex{Schnitzler, Arthur@\textsc{Schnitzler, Arthur}!zzzHerzl, Theodor@\emph{von Theodor Herzl}!1895-05-051@{5. 5. 1895}|(be}
\toendnotes[C]{\smallbreak\pagebreak[2]}
\correspDesc{Versand  durch Theodor Herzl am 5. 5. 1895 in Paris
\newline{}Erhalt  durch Arthur Schnitzler im Zeitraum [5. 5. 1895
                  – 8. 5. 1895?] in Wien}\toendnotes[C]{\smallbreak}
\Standort{CUL, Schnitzler, B 39.}
\physDesc{Brief, 1 Blatt, 2 Seiten, 1531 Zeichen
\newline{}Handschrift: schwarze Tinte, lateinische Kurrent
\newline{}Schnitzler: mit Bleistift Vermerk: »{\pb}\textsc{Herzl}« 
\newline{}Ordnung: mit Bleistift von unbekannter Hand nummeriert: »39« }
\buchAbdrucke{\weitereDrucke{Theodor Herzl: \emph{Briefe Anfang Mai 1895 – Anfang Dezember 1898}. Bearbeitet von Barbara Schäfer in Zusammenarbeit mit Sofia Gelmann, Chaya Harel, Ines Rubin und Daisy Ticho. Berlin, Frankfurt am Main, Wien: \emph{Propyläen} 1990, S. 36 (Briefe und Tagebücher. Herausgegeben von Alex Bein, Hermann Greive, Moshe Schaerf, Julius H. Schoeps und Johannes Wachten, 4).} }\toendnotes[C]{\smallbreak}
\pstart
           {\pb}\textcolor{gray}{\textbf{NOUVELLE PRESSE LIBRE\orgindex{Neue Freie Presse@Neue Freie Presse|pw}}}\hfill \textcolor{gray}{\textbf{Paris\oindex{Paris@\textbf{Paris}, \emph{Hauptstadt}|pw}, le}}{ }5 Mai \textcolor{gray}{\textbf{1895}}\pend
           
\pstart
           \textcolor{gray}{\textbf{D\textsuperscript{r}{ }THÉODORE HERZL}}\pend
           
\pstart
           37 rue Cambon\oindex{37, Rue Cambon@\textbf{37, Rue Cambon}, \emph{Wohngebäude}|pw}\pend
           
\pstart{}Lieber Freund!\pend\vspace{0.5em}
\pstart
           Zwei Worte in Eile.\pend
           
\pstart
           Fischer\pwindex{Fischer, Samuel 24.\,12.\,1859 Liptovský Mikuláš – 15.\,10.\,1934 Berlin@\textsc{Fischer, Samuel} (24.\,12.\,1859 Liptovský Mikuláš – 15.\,10.\,1934 Berlin), \emph{Verleger}|pw} hat sichs zu lang überlegt. Jetzt will
               ich nicht mehr. Unter anderen Gründen hiefür erwähne ich zwei: 1° ich weiss nicht,
               wie nach dem neuen \label{K_L03860-1v}\edtext{deutschen\oindex{Deutschland@\textbf{Deutschland}|pw} Umsturzgesetz}{\lemma{\textnormal{\emph{deutschen Umsturzgesetz}}}\Cendnote{\textnormal{Gemeint ist die seit Dezember 1894 in Deutschland\oindex{Deutschland@\textbf{Deutschland}|pwk} diskutierte sogenannte
                  Umsturzvorlage, die eine Einschränkung von Presse- und Wissenschaftsfreiheit
                  anvisierte. Sie wurde jedoch am 11. 5. 1895 in zweiter Lesung vom \emph{Parlament}\orgindex{Reichstag@Reichstag|pwk} zurückgewiesen.}}}\label{K_L03860-1} die Buchausgabe\pwindex{Herzl, Theodor 2.\,5.\,1860 Budapest – 3.\,7.\,1904 Edlach@\textsc{Herzl, Theodor} (2.\,5.\,1860 Budapest – 3.\,7.\,1904 Edlach), \emph{Schriftsteller, Journalist}!neue Ghetto. Schauspiel in vier Acten@\strich\emph{Das neue Ghetto. Schauspiel in vier Acten}|pwv} (wegen des \label{K_L03860-2v}\edtext{Rittmeisters}{\lemma{\textnormal{\emph{Rittmeisters}}}\Cendnote{\textnormal{Die Figur des Rittmeister von Schramms ist der Antagonist in
                     Herzls\pwindex{Herzl, Theodor 2.\,5.\,1860 Budapest – 3.\,7.\,1904 Edlach@\textsc{Herzl, Theodor} (2.\,5.\,1860 Budapest – 3.\,7.\,1904 Edlach), \emph{Schriftsteller, Journalist}|pwk} Schauspiel \emph{Das neue Ghetto}\pwindex{Herzl, Theodor 2.\,5.\,1860 Budapest – 3.\,7.\,1904 Edlach@\textsc{Herzl, Theodor} (2.\,5.\,1860 Budapest – 3.\,7.\,1904 Edlach), \emph{Schriftsteller, Journalist}!neue Ghetto. Schauspiel in vier Acten@\strich\emph{Das neue Ghetto. Schauspiel in vier Acten}|pwk}. Schramm beutet die Arbeiter seines Bergwerks
                  skrupellos aus, ist nur auf Gewinn bedacht und überzeugter Antisemit.}}}\label{K_L03860-2})
               behandelt würde. Abortiren will ich das Werk\pwindex{Herzl, Theodor 2.\,5.\,1860 Budapest – 3.\,7.\,1904 Edlach@\textsc{Herzl, Theodor} (2.\,5.\,1860 Budapest – 3.\,7.\,1904 Edlach), \emph{Schriftsteller, Journalist}!neue Ghetto. Schauspiel in vier Acten@\strich\emph{Das neue Ghetto. Schauspiel in vier Acten}|pwv} im Buchhandel nicht lassen.\pend
           
\pstart
           2° Jetzt könnte das Buch\pwindex{Herzl, Theodor 2.\,5.\,1860 Budapest – 3.\,7.\,1904 Edlach@\textsc{Herzl, Theodor} (2.\,5.\,1860 Budapest – 3.\,7.\,1904 Edlach), \emph{Schriftsteller, Journalist}!neue Ghetto. Schauspiel in vier Acten@\strich\emph{Das neue Ghetto. Schauspiel in vier Acten}|pwv} doch
               erst wieder im Herbst erscheinen. So will ich den verlorenen
                  Sommer zu anderen Anbringungsversuchen (notamment Prag\oindex{Prag@\textbf{Prag}, \emph{Land}|pw}) verwenden.\pend
           
\pstart
           Ich bitte Sie also das Manuscript\pwindex{Herzl, Theodor 2.\,5.\,1860 Budapest – 3.\,7.\,1904 Edlach@\textsc{Herzl, Theodor} (2.\,5.\,1860 Budapest – 3.\,7.\,1904 Edlach), \emph{Schriftsteller, Journalist}!neue Ghetto. Schauspiel in vier Acten@\strich\emph{Das neue Ghetto. Schauspiel in vier Acten}|pwv} sofort von Fischer\pwindex{Fischer, Samuel 24.\,12.\,1859 Liptovský Mikuláš – 15.\,10.\,1934 Berlin@\textsc{Fischer, Samuel} (24.\,12.\,1859 Liptovský Mikuláš – 15.\,10.\,1934 Berlin), \emph{Verleger}|pw}
                  zurückver{\pb}langen zu lassen u. bis zur
               weiteren Verfügung, die mein nächster langer Plauderbrief enthalten wird, bei sich zu
               behalten. Ich koste Sie in Briefmarken allein schon ein Vermögen – – – – – –
               (Verlegenheitsgedankenstriche.)\pend
           
\pstart
           Welcher Baumgarten\pwindex{Baumgarten, Julius 26.\,5.\,1860 Wien – 28.\,8.\,1934 ebd.@\textsc{Baumgarten, Julius} (26.\,5.\,1860 Wien – 28.\,8.\,1934 ebd.), \emph{Anwalt}|pwv} ersetzt Schick\pwindex{Schik, Friedrich *~6.\,9.\,1857 Wien@\textsc{Schik, Friedrich} (*~6.\,9.\,1857 Wien), \emph{Notar, Journalist, Dramaturg}|pw}? Der einst im Deutsch-oestreichischen Leseverein\orgindex{Deutsch-Österreichischer Leseverein der Wiener Hochschulen@Deutsch-Österreichischer Leseverein der Wiener Hochschulen|pw} war? Um Gotteswillen, das ist eine
               gräuliche Plaudertasche. Nur sorgen Sie dafür, dass er das Manuscript\pwindex{Herzl, Theodor 2.\,5.\,1860 Budapest – 3.\,7.\,1904 Edlach@\textsc{Herzl, Theodor} (2.\,5.\,1860 Budapest – 3.\,7.\,1904 Edlach), \emph{Schriftsteller, Journalist}!neue Ghetto. Schauspiel in vier Acten@\strich\emph{Das neue Ghetto. Schauspiel in vier Acten}|pwv} nicht liest! Ueberhaupt wird’s
               mir leider schon zu viel abgelüftet!\pend
           
\pstart
           Müller G.\pwindex{Müller-Guttenbrunn, Adam 22.\,10.\,1852 Zăbrani – 5.\,1.\,1923 Wien@\textsc{Müller-Guttenbrunn, Adam} (22.\,10.\,1852 Zăbrani – 5.\,1.\,1923 Wien), \emph{Schriftsteller, Theaterleiter, Beamter}|pw} hat den richtigen Einfall gehabt,
               als er an meinen Tact appellirte. Bitten um Schonung sind nie ungehört an mich
               gekommen. Ich werde ihn nicht erwähnen. Ueber die Tabarin\pwindex{Herzl, Theodor 2.\,5.\,1860 Budapest – 3.\,7.\,1904 Edlach@\textsc{Herzl, Theodor} (2.\,5.\,1860 Budapest – 3.\,7.\,1904 Edlach), \emph{Schriftsteller, Journalist}!Tabarin. Schauspiel in einem Act. Frei nach Catulle Mendès@\strich\emph{Tabarin. Schauspiel in einem Act. Frei nach Catulle Mendès}|pw}-Kritiken\pwindex{Kalbeck, Max 4.\,1.\,1850 Breslau – 4.\,5.\,1921 Wien@\textsc{Kalbeck, Max} (4.\,1.\,1850 Breslau – 4.\,5.\,1921 Wien), \emph{Journalist}!Burgtheater. [Tabarin und Verbotene Früchte]@\strich\emph{Burgtheater. [Tabarin und Verbotene Früchte]}|pwv}
               habe ich mich krumm u. bucklig gelacht. Nein, was diese Herren für Kunstverständige
               sind.\pend
           
\pstart
           Ich grüsse Sie herzlich Ihr getreuer{\\[\baselineskip]}\spacefill\mbox{Th H.}\pend
           \leftskip=0em{}
\pstart
           \noindent{}\label{T_L03860-1v}\edtext{Die Palme gebührt \label{K_L03860-3v}\edtext{Kalbeck\pwindex{Kalbeck, Max 4.\,1.\,1850 Breslau – 4.\,5.\,1921 Wien@\textsc{Kalbeck, Max} (4.\,1.\,1850 Breslau – 4.\,5.\,1921 Wien), \emph{Journalist}|pw}}{\lemma{\textnormal{\emph{Kalbeck}}}\Cendnote{\textnormal{M. K.\pwindex{Kalbeck, Max 4.\,1.\,1850 Breslau – 4.\,5.\,1921 Wien@\textsc{Kalbeck, Max} (4.\,1.\,1850 Breslau – 4.\,5.\,1921 Wien), \emph{Journalist}|pwk} [ = Max Kalbeck\pwindex{Kalbeck, Max 4.\,1.\,1850 Breslau – 4.\,5.\,1921 Wien@\textsc{Kalbeck, Max} (4.\,1.\,1850 Breslau – 4.\,5.\,1921 Wien), \emph{Journalist}|pwk}]: \emph{Burgtheater}\pwindex{Kalbeck, Max 4.\,1.\,1850 Breslau – 4.\,5.\,1921 Wien@\textsc{Kalbeck, Max} (4.\,1.\,1850 Breslau – 4.\,5.\,1921 Wien), \emph{Journalist}!Burgtheater. [Tabarin und Verbotene Früchte]@\strich\emph{Burgtheater. [Tabarin und Verbotene Früchte]}|pwk}. In: \emph{Neues Wiener
                           Tagblatt}\pwindex{Neues Wiener Tagblatt@\emph{Neues Wiener Tagblatt}|pwk}, Jg. 29, Nr. 120, 3. 5. 1895,
                     S. 5.}}}\label{K_L03860-3} der mich als »\label{K_L03860-4v}\edtext{flotten Verkäufer alter
                     Stoffe\pwindex{Kalbeck, Max 4.\,1.\,1850 Breslau – 4.\,5.\,1921 Wien@\textsc{Kalbeck, Max} (4.\,1.\,1850 Breslau – 4.\,5.\,1921 Wien), \emph{Journalist}!Burgtheater. [Tabarin und Verbotene Früchte]@\strich\emph{Burgtheater. [Tabarin und Verbotene Früchte]}|pwv}}{\lemma{\textnormal{\emph{flotten … Stoffe}}}\Cendnote{\textnormal{Die Stelle lautet: »Die Herren
                           Emil \so{Gött}\pwindex{Gött, Emil 13.\,5.\,1864 Jechtingen – 13.\,4.\,1908 Freiburg im Breisgau@\textsc{Gött, Emil} (13.\,5.\,1864 Jechtingen – 13.\,4.\,1908 Freiburg im Breisgau), \emph{Schriftsteller, Landwirt, Dramatiker}|pw} und Theodor \so{Herzl}\pwindex{Herzl, Theodor 2.\,5.\,1860 Budapest – 3.\,7.\,1904 Edlach@\textsc{Herzl, Theodor} (2.\,5.\,1860 Budapest – 3.\,7.\,1904 Edlach), \emph{Schriftsteller, Journalist}|pw} haben mit dem großen Miguel de \so{Cervantes}\pwindex{Cervantes Saavedra, Miguel de 29.\,9.\,1547 Alcalá de Henares – 22.\,4.\,1616 Madrid@\textsc{Cervantes Saavedra, Miguel de} (29.\,9.\,1547 Alcalá de Henares – 22.\,4.\,1616 Madrid), \emph{Schriftsteller}|pw} und dem kleinen Catulle \so{Mendès}\pwindex{Mendès, Catulle 20.\,5.\,1841 Bordeaux – 8.\,2.\,1909 Saint-Germain-en-Laye@\textsc{Mendès, Catulle} (20.\,5.\,1841 Bordeaux – 8.\,2.\,1909 Saint-Germain-en-Laye), \emph{Schriftsteller}|pw}
                        ein Compagniegeschäft etablirt, um als flotte Verkäufer ein paar alte Stoffe
                        an den Mann zu bringen.«}}}\label{K_L03860-4}« hinstellt. Ich treibe also
                  Hosenhandel. Und Gött\pwindex{Gött, Emil 13.\,5.\,1864 Jechtingen – 13.\,4.\,1908 Freiburg im Breisgau@\textsc{Gött, Emil} (13.\,5.\,1864 Jechtingen – 13.\,4.\,1908 Freiburg im Breisgau), \emph{Schriftsteller, Landwirt, Dramatiker}|pw} ist formgewandt\pwindex{Kalbeck, Max 4.\,1.\,1850 Breslau – 4.\,5.\,1921 Wien@\textsc{Kalbeck, Max} (4.\,1.\,1850 Breslau – 4.\,5.\,1921 Wien), \emph{Journalist}!Burgtheater. [Tabarin und Verbotene Früchte]@\strich\emph{Burgtheater. [Tabarin und Verbotene Früchte]}|pwv}  ich nicht. Ich habe
                  nämlich gegen seinen Principal\pwindex{Singer, Wilhelm 26.\,11.\,1847 Bzenec – 10.\,10.\,1917 Wien@\textsc{Singer, Wilhelm} (26.\,11.\,1847 Bzenec – 10.\,10.\,1917 Wien), \emph{Journalist, Chefredakteur}|pwv} gestimmt in der Concordia\orgindex{Concordia. Journalisten- und Schriftstellerverein@Concordia. Journalisten- und Schriftstellerverein|pw}.}{\lemma{\textnormal{\emph{Die … Concordia.}}}\Cendnote{\textnormal{Text entlang des
                     Mittelfalzes geschrieben}}}\label{T_L03860-1}\pend
           \selectlanguage{ngerman}\endnumbering\briefempfaengerindex{Schnitzler, Arthur@\textsc{Schnitzler, Arthur}!zzzHerzl, Theodor@\emph{von Theodor Herzl}!1895-05-051@{5. 5. 1895}|)be}\mylabel{L03860h}
\begin{anhang}
\end{anhang}\newcommand{\dateiname}{L03860}\newcommand{\titel}{Theodor Herzl an Arthur Schnitzler, 5. 5. 1895}\newcommand{\editorInnen}{Selma Jahnke und Martin Anton Müller}%% latex-leseansicht-abspann.tex
%% Abspann für die Leseansicht.
%% Der Schalter \ifkorrekturansicht ist bereits durch den Vorspann gesetzt.

%% latex-abspann.tex
%% Gemeinsamer Abspann für Korrekturansicht und Leseansicht.
%% Setzt den Schalter \ifkorrekturansicht voraus (gesetzt in den
%% einbindenden Dateien latex-korrekturansicht-abspann.tex bzw.
%% latex-leseansicht-abspann.tex).
%% ---------------------------------------------------------------

\normalsize

% Das esempio-Environment wird nur in der Leseansicht benötigt
\ifkorrekturansicht\else
\newenvironment{esempio}[3]%
{
    \vspace{1.5ex}
    \rlap{\underline{#1}}
    \par
    \setlength{\parindent}{0cm}
    \nopagebreak
    \leftskip=#2cm
    \rightskip=#3cm
}
{
    \par
}
\fi

\doendnotes{C}
\bigskip
\vfill

\clearpage

\footnotesize

\ifkorrekturansicht
  \lohead{\textsc{register}}
\fi

% theindex-Environment neu definieren ohne reledmac
\makeatletter
\renewenvironment{theindex}{%
  \ifkorrekturansicht
    \section*{\indexname}%
  \else
    \subsubsection*{Index der erwähnten Entitäten}%
  \fi
  \setlength{\parindent}{0pt}%
  \setlength{\parskip}{0pt plus 0.3pt}%
  \let\item\@idxitem
}{%
  \ifkorrekturansicht\clearpage\fi
}
\makeatother

\IfFileExists{\jobname-pw.ind}{\input{\jobname-pw.ind}}{}

% Quellenangabe nur in der Leseansicht
\ifkorrekturansicht\else
% Fallback-Definitionen, falls die .tex-Datei \titel etc. nicht gesetzt hat
\providecommand{\titel}{}
\providecommand{\editorInnen}{}
\providecommand{\dateiname}{\jobname}

\vspace{3cm}

\vfill

\footnotesize
\textsc{Quelle}: \titel. Herausgegeben von {\editorInnen}. In: \emph{Arthur Schnitzler: Briefwechsel mit Autorinnen und Autoren}.
 Digitale Edition, https://schnitzler-briefe.acdh.oeaw.ac.at/{\dateiname}.html (Stand \today)
\fi

\end{document}


