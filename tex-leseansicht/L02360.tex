%% latex-korrekturansicht-vorspann.tex
%% Vorspann für die Korrekturansicht.
%% Lädt die gemeinsame Datei latex-vorspann.tex mit gesetztem Schalter.

\newif\ifkorrekturansicht
\korrekturansichttrue

\input{../tex-inputs/latex-vorspann}


\section[Arthur Schnitzler an Hermann Bahr, 7. 2. 1921]{L02360 Arthur Schnitzler an Hermann Bahr, 7. 2. 1921}
\nopagebreak\mylabel{L02360v}
\rehead{ }\normalsize\beginnumbering\briefempfaengerindex{Bahr, Hermann@\textsc{Bahr, Hermann}!zzzSchnitzler, Arthur@\emph{von Arthur Schnitzler}!1921-02-071@{7. 2. 1921}|(be}
\toendnotes[C]{\smallbreak\pagebreak[2]}\Standort{TMW, HS AM 23396 Ba.}
\physDesc{Brief, 1 Blatt, 1 Seite, 1062 Zeichen
\newline{}Schreibmaschine
\newline{}Handschrift: 1) schwarze Tinte, lateinische Kurrent (\noindent{}Unterschrift und Grußformel)\hspace{1em}2) Bleistift, lateinische Kurrent (\noindent{}Korrekturen)\hspace{1em}}\Standort{DLA, A:Schnitzler, 85.1.294/7.}
\physDesc{Durchschlag1 Blatt, 1 Seite, 1062 Zeichen
\newline{}Schreibmaschine}
\buchAbdrucke{\weitereDrucke{1) Arthur Schnitzler: \emph{The Letters of Arthur Schnitzler to Hermann Bahr}. Chapel Hill: \emph{The University of North Carolina Press} 1978, S. 115.} \weitereDrucke{2) Hermann Bahr, Arthur Schnitzler: \emph{Briefwechsel, Aufzeichnungen, Dokumente (1891–1931)}. Göttingen: \emph{Wallstein} 2018, S. 540.} }\toendnotes[C]{\smallbreak}
\pstart
           {\pb}\textcolor{gray}{\textbf{D\textsuperscript{r} Arthur Schnitzler}}\hfill 7. 2. 1921.\pend
           
\pstart
           \textcolor{gray}{\textbf{Wien. XVIII. Sternwartestrasse 71\oindex{Sternwartestrasse 71@\textbf{Sternwartestraße 71}, \emph{Wohngebäude (K.WHS)}|pw}.}}\pend
           
\pstart{}Lieber Hermann.\pend\vspace{0.5em}
\pstart
           Am \label{K_L02360-1v}\edtext{20. Feber}{\lemma{\textnormal{\emph{20. Feber}}}\Cendnote{\textnormal{Eigentlich am 21., wobei die Unsicherheit über den genauen Geburtstag in der Presse
                  verbreitet war.}}}\label{K_L02360-1} feiert Popper-Lynkeus\pwindex{Popper-Lynkeus, Josef 21.02.1838 – 22.12.1921@\textsc{Popper-Lynkeus, Josef} (21.02.1838 – 22.12.1921), \emph{Schriftsteller/Schriftstellerin}|pw}
               seinen 83. Geburtstag. \substVorne{}\textsuperscript{Es}\substDazwischen{}Das\substHinten{} fängt wie ein Aufruf an, aber es ist nur eine Bitte. Es wäre von einiger
               Bedeutung, insbesondere mit Rücksicht auf die bevorstehende \label{K_L02360-2v}\edtext{Ausgabe\pwindex{Krieg, Wehrpflicht und Staatsverfassung@\emph{Krieg, Wehrpflicht und Staatsverfassung}|pwv}}{\lemma{\textnormal{\emph{Ausgabe}}}\Cendnote{\textnormal{Eine Werkausgabe erschien nicht, nur ein
                  neuer Titel: Josef Popper-Lynkeus\pwindex{Popper-Lynkeus, Josef 21.02.1838 – 22.12.1921@\textsc{Popper-Lynkeus, Josef} (21.02.1838 – 22.12.1921), \emph{Schriftsteller/Schriftstellerin}|pwk}: \emph{Krieg, Wehrpflicht und Staatsverfassung}\pwindex{Krieg, Wehrpflicht und Staatsverfassung@\emph{Krieg, Wehrpflicht und Staatsverfassung}|pwk}. Wien, Berlin,
                     Leipzig, München: \emph{Rikola}\orgindex{Rikola Verlag@Rikola Verlag|pwk}{ }1921.}}}\label{K_L02360-2} der Popper-Lynkeu’schen\pwindex{Popper-Lynkeus, Josef 21.02.1838 – 22.12.1921@\textsc{Popper-Lynkeus, Josef} (21.02.1838 – 22.12.1921), \emph{Schriftsteller/Schriftstellerin}|pw}
               Werke im \label{K_L02360-3v}\edtext{Verlag Kola\orgindex{Rikola Verlag@Rikola Verlag|pw}}{\lemma{\textnormal{\emph{Verlag Kola}}}\Cendnote{\textnormal{Gemeint ist der Wien\oindex{Wien@\textbf{Wien}, \emph{A.ADM2}|pwk}er Verlag \emph{Rikola}\orgindex{Rikola Verlag@Rikola Verlag|pwk}, der
                  von Richard Kola\pwindex{Kola, Richard 12.08.1872 – 11.03.1939@\textsc{Kola, Richard} (12.08.1872 – 11.03.1939), \emph{Schriftsteller/Schriftstellerin, Verleger/Verlegerin, Bankier/Bankierin}|pwk}{ }Ende 1920 mit Unterstützung Schnitzlers gegründet worden war und für den sich in der Folge auch Bahr\pwindex{Bahr, Hermann 19.07.1863 – 15.01.1934@\textsc{Bahr, Hermann} (19.07.1863 – 15.01.1934), \emph{Schriftsteller/Schriftstellerin, Kritiker/Kritikerin}|pwk} engagierte.}}}\label{K_L02360-3}, wenn an diesem Tag
               von einigen führenden Geistern die rechten Worte über ihn gesagt würden. Man hat mich
               gebeten Dich zu fragen, ob Du vielleicht in Deinem Tagebuch (der 20.
                  Feber ist gerade ein Sonntag) über Popper-Lynkeus\pwindex{Popper-Lynkeus, Josef 21.02.1838 – 22.12.1921@\textsc{Popper-Lynkeus, Josef} (21.02.1838 – 22.12.1921), \emph{Schriftsteller/Schriftstellerin}|pw}, den Du ja, wie ich weiss, liebst und verehrst, schreiben
               wolltest. Wäre Dir diesmal irgend eine andere Form, ein anderer Rahmen genehm, so
               steht es natürlich ganz bei Dir. Es wäre von hohem Wert (wie ich glaube auch für den
               Elan des Verlages), wenn Du am 20. Februar unter denen nicht fehltest,
               die ein paar Worte über das Werk und das Wesen von Popper-Lynkeus\pwindex{Popper-Lynkeus, Josef 21.02.1838 – 22.12.1921@\textsc{Popper-Lynkeus, Josef} (21.02.1838 – 22.12.1921), \emph{Schriftsteller/Schriftstellerin}|pw} sagen wollten.\pend
           
\pstart
           Ich höre\introOben{}, –\introOben{} und lese es auch aus Deinem Tagebuch\pwindex{Tagebuch [Kolumne im Neuen Wiener Journal]@\emph{Tagebuch [Kolumne im Neuen Wiener Journal]}|pw} heraus, dass Du Dich wohlbefindest. Hoffentlich habe ich
               doch bald wieder Gelegenheit mich auch persönlich davon zu überzeugen.\pend
           
\pstart
           {[}hs.:{]} Mit herzlichen Grüßen{\\[\baselineskip]}Dein{\\[\baselineskip]}\spacefill\mbox{Arthur}\pend
           \leftskip=0em{}\selectlanguage{ngerman}\endnumbering\briefempfaengerindex{Bahr, Hermann@\textsc{Bahr, Hermann}!zzzSchnitzler, Arthur@\emph{von Arthur Schnitzler}!1921-02-071@{7. 2. 1921}|)be}\mylabel{L02360h}  \normalsize

\doendnotes{C}
\bigskip
\vfill

\clearpage

\footnotesize

\lohead{\textsc{register}}

% Definiere theindex-Environment komplett neu ohne reledmac
\makeatletter
\renewenvironment{theindex}{%
  \section*{\indexname}%
  \setlength{\parindent}{0pt}%
  \setlength{\parskip}{0pt plus 0.3pt}%
  \let\item\@idxitem
}{%
  \clearpage
}
\makeatother

\IfFileExists{\jobname-pw.ind}{\input{\jobname-pw.ind}}{}

\end{document}

      