%% latex-leseansicht-vorspann.tex
%% Vorspann für die Leseansicht.
%% Lädt die gemeinsame Datei latex-vorspann.tex mit nicht gesetztem Schalter.

\newif\ifkorrekturansicht
\korrekturansichtfalse

\input{../tex-inputs/latex-vorspann}


               \section[Arthur Schnitzler an Hermann Bahr, 7. 2. 1921]{ Arthur Schnitzler an Hermann Bahr, 7. 2. 1921}\nopagebreak\mylabel{v}\rehead{ }\begin{ledgroupsized}[t]{13cm}\normalsize\beginnumbering\briefempfaengerindex{Bahr, Hermann@\textsc{Bahr, Hermann}!zzzSchnitzler, Arthur@\emph{von Arthur Schnitzler}!1921-02-071@{7. 2. 1921}|(be} \toendnotes[C]{\smallbreak\pagebreak[2]} \Standort{TMW, HS AM 23396 Ba.}
\physDesc{Brief, 1 Blatt, 1 Seite
\newline{}Schreibmaschine
\newline{}Handschrift: 1) schwarze Tinte, lateinische Kurrent (\noindent{}Unterschrift und Grußformel)\hspace{1em}2) Bleistift, lateinische Kurrent (\noindent{}Korrekturen)\hspace{1em}}\Standort{DLA, A:Schnitzler, 85.1.294/7.}
\physDesc{Brief, maschineller Durchschlag
\newline{}Schreibmaschine}\buchAbdrucke{\weitereDrucke{1) \emph{7. 2. 1921.} In: Arthur Schnitzler: \emph{The Letters of Arthur Schnitzler to Hermann Bahr}. Edited, annotated, and with an introduction, by Donald G.
                        Daviau. Chapel Hill: \emph{The University of North Carolina Press} 1978, S. 115 (University of North Carolina studies in the Germanic languages
                        and literatures, 89).} \weitereDrucke{2) Hermann Bahr, Arthur Schnitzler: \emph{Briefwechsel, Aufzeichnungen, Dokumente (1891–1931)}. Hg. Kurt Ifkovits und Martin Anton Müller. Göttingen: \emph{Wallstein} 2018, S. 540.} }\toendnotes[C]{\smallbreak}\pstart
           \noindent{}{\pb}\textcolor{gray}{\textbf{D\textsuperscript{r } Arthur Schnitzler}}\hfill 7. 2. 1921.\pend
           \pstart
           \textcolor{gray}{\textbf{Wien. XVIII. Sternwartestrasse 71\oindex{Sternwartestrasse@\textbf{Sternwartestraße}|pw}.}}\pend
           \pstart{}Lieber Hermann.\pend\pstart
           Am \label{K_L02360_1v}\edtext{20. Feber}{\lemma{\textnormal{\emph{20. Feber}}}\Cendnote{\textnormal{Eigentlich am 21., wobei die Unsicherheit über den genauen Geburtstag in der
                  Presse verbreitet war.}}}\label{K_L02360_1h} feiert Popper-Lynkeus\pwindex{Popper-Lynkeus, Josef 21.02.1838 – 22.12.1921@\textsc{Popper-Lynkeus, Josef} (21.02.1838 – 22.12.1921), \emph{Schriftsteller}|pw} seinen 83. Geburtstag. \substVorne{}\textsuperscript{Es}\substDazwischen{}Das\substHinten{} fängt wie ein Aufruf an, aber es ist nur eine Bitte. Es wäre von einiger
               Bedeutung, insbesondere mit Rücksicht auf die bevorstehende \label{K_L02360_2v}\edtext{Ausgabe\pwindex{Popper-Lynkeus, Josef 21.02.1838 – 22.12.1921@\textsc{Popper-Lynkeus, Josef} (21.02.1838 – 22.12.1921), \emph{Schriftsteller}!Krieg, Wehrpflicht und Staatsverfassung1921@\strich\emph{Krieg, Wehrpflicht und Staatsverfassung} {[}1921{]}|pwv}}{\lemma{\textnormal{\emph{Ausgabe}}}\Cendnote{\textnormal{Eine Werkausgabe erschien nicht, nur ein
                  neuer Titel: Josef Popper-Lynkeus\pwindex{Popper-Lynkeus, Josef 21.02.1838 – 22.12.1921@\textsc{Popper-Lynkeus, Josef} (21.02.1838 – 22.12.1921), \emph{Schriftsteller}|pwk}: \emph{Krieg, Wehrpflicht und Staatsverfassung}\pwindex{Popper-Lynkeus, Josef 21.02.1838 – 22.12.1921@\textsc{Popper-Lynkeus, Josef} (21.02.1838 – 22.12.1921), \emph{Schriftsteller}!Krieg, Wehrpflicht und Staatsverfassung1921@\strich\emph{Krieg, Wehrpflicht und Staatsverfassung} {[}1921{]}|pwk}. Wien, Berlin,
                     Leipzig, München: \emph{Rikola}\orgindex{Rikola@Rikola|pwk}{ }1921.}}}\label{K_L02360_2h} der Popper-Lynkeu’schen\pwindex{Popper-Lynkeus, Josef 21.02.1838 – 22.12.1921@\textsc{Popper-Lynkeus, Josef} (21.02.1838 – 22.12.1921), \emph{Schriftsteller}|pw} Werke
               im \label{K_L02360_3v}\edtext{Verlag Kola\orgindex{Rikola@Rikola|pw}}{\lemma{\textnormal{\emph{Verlag Kola}}}\Cendnote{\textnormal{Gemeint ist der Wien\oindex{Wien@\textbf{Wien}|pwk}er Verlag \emph{Rikola}\orgindex{Rikola@Rikola|pwk}, der
                  von Richard Kola\pwindex{Kola, Richard 12.08.1872 – 11.03.1939@\textsc{Kola, Richard} (12.08.1872 – 11.03.1939), \emph{Schriftsteller/Schriftstellerin, Verleger/Verlegerin, Bankier/Bankierin}|pwk}{ }Ende 1920 mit Unterstützung Schnitzlers\pwindex{Schnitzler, Arthur 15.05.1862 – 21.10.1931@\textsc{Schnitzler, Arthur} (15.05.1862 – 21.10.1931), \emph{Schriftsteller, Mediziner}|pwk} gegründet wurde und für den in der Folge auch Bahr\pwindex{Bahr, Hermann 19.07.1863 – 15.01.1934@\textsc{Bahr, Hermann} (19.07.1863 – 15.01.1934), \emph{Schriftsteller, Kritiker}|pwk} tätig wurde.}}}\label{K_L02360_3h}, wenn an diesem Tag
               von einigen führenden Geistern die rechten Worte über ihn gesagt würden. Man hat mich
               gebeten Dich zu fragen, ob Du vielleicht in Deinem Tagebuch (der 20.
                  Feber ist gerade ein Sonntag) über Popper-Lynkeus\pwindex{Popper-Lynkeus, Josef 21.02.1838 – 22.12.1921@\textsc{Popper-Lynkeus, Josef} (21.02.1838 – 22.12.1921), \emph{Schriftsteller}|pw}, den Du ja, wie ich weiss, liebst und verehrst, schreiben
               wolltest. Wäre Dir diesmal irgend eine andere Form, ein anderer Rahmen genehm, so
               steht es natürlich ganz bei Dir. Es wäre von hohem Wert (wie ich glaube auch für den
               Elan des Verlages), wenn Du am 20. Februar unter denen nicht fehltest,
               die ein paar Worte über das Werk und das Wesen von Popper-Lynkeus\pwindex{Popper-Lynkeus, Josef 21.02.1838 – 22.12.1921@\textsc{Popper-Lynkeus, Josef} (21.02.1838 – 22.12.1921), \emph{Schriftsteller}|pw} sagen wollten.\pend
           \pstart
           Ich höre\introOben{}, –\introOben{} und lese es auch aus Deinem Tagebuch\pwindex{Bahr, Hermann 19.07.1863 – 15.01.1934@\textsc{Bahr, Hermann} (19.07.1863 – 15.01.1934), \emph{Schriftsteller, Kritiker}!Tagebuch [Kolumne im Neuen Wiener Journal]24.12.1916 – 1931@\strich\emph{Tagebuch [Kolumne im Neuen Wiener Journal]} {[}24.12.1916 – 1931{]}|pw} heraus, dass Du Dich wohlbefindest. Hoffentlich habe ich
               doch bald wieder Gelegenheit mich auch persönlich davon zu überzeugen.\pend
           \pstart
           {[}hs.:{]} Mit herzlichen Grüßen{\\[\baselineskip]}Dein{\\[\baselineskip]}\spacefill\mbox{Arthur}\pend
           \leftskip=0em{}\endnumbering\briefempfaengerindex{Bahr, Hermann@\textsc{Bahr, Hermann}!zzzSchnitzler, Arthur@\emph{von Arthur Schnitzler}!1921-02-071@{7. 2. 1921}|)be}\mylabel{h}\end{ledgroupsized}  \newcommand{\dateiname}{L02360}\newcommand{\titel}{Arthur Schnitzler an Hermann Bahr, 7. 2. 1921}\newcommand{\editorInnen}{ Kurt Ifkovits,  Martin Anton Müller}
            \footnotesize
\begin{ledgroupsized}[t]{11.5cm}
\doendnotes{C}
\end{ledgroupsized}
         %% latex-leseansicht-abspann.tex
%% Abspann für die Leseansicht.
%% Der Schalter \ifkorrekturansicht ist bereits durch den Vorspann gesetzt.

%% latex-abspann.tex
%% Gemeinsamer Abspann für Korrekturansicht und Leseansicht.
%% Setzt den Schalter \ifkorrekturansicht voraus (gesetzt in den
%% einbindenden Dateien latex-korrekturansicht-abspann.tex bzw.
%% latex-leseansicht-abspann.tex).
%% ---------------------------------------------------------------

\normalsize

% Das esempio-Environment wird nur in der Leseansicht benötigt
\ifkorrekturansicht\else
\newenvironment{esempio}[3]%
{
    \vspace{1.5ex}
    \rlap{\underline{#1}}
    \par
    \setlength{\parindent}{0cm}
    \nopagebreak
    \leftskip=#2cm
    \rightskip=#3cm
}
{
    \par
}
\fi

\doendnotes{C}
\bigskip
\vfill

\clearpage

\footnotesize

\ifkorrekturansicht
  \lohead{\textsc{register}}
\fi

% theindex-Environment neu definieren ohne reledmac
\makeatletter
\renewenvironment{theindex}{%
  \ifkorrekturansicht
    \section*{\indexname}%
  \else
    \subsubsection*{Index der erwähnten Entitäten}%
  \fi
  \setlength{\parindent}{0pt}%
  \setlength{\parskip}{0pt plus 0.3pt}%
  \let\item\@idxitem
}{%
  \ifkorrekturansicht\clearpage\fi
}
\makeatother

\IfFileExists{\jobname-pw.ind}{\input{\jobname-pw.ind}}{}

% Quellenangabe nur in der Leseansicht
\ifkorrekturansicht\else
% Fallback-Definitionen, falls die .tex-Datei \titel etc. nicht gesetzt hat
\providecommand{\titel}{}
\providecommand{\editorInnen}{}
\providecommand{\dateiname}{\jobname}

\vspace{3cm}

\vfill

\footnotesize
\textsc{Quelle}: \titel. Herausgegeben von {\editorInnen}. In: \emph{Arthur Schnitzler: Briefwechsel mit Autorinnen und Autoren}.
 Digitale Edition, https://schnitzler-briefe.acdh.oeaw.ac.at/{\dateiname}.html (Stand \today)
\fi

\end{document}


      