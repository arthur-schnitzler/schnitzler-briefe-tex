%% latex-korrekturansicht-vorspann.tex
%% Vorspann für die Korrekturansicht.
%% Lädt die gemeinsame Datei latex-vorspann.tex mit gesetztem Schalter.

\newif\ifkorrekturansicht
\korrekturansichttrue

\input{../tex-inputs/latex-vorspann}


\section[ Paul Goldmann an Arthur Schnitzler, 10. 5. {[}1900{]}]{L02915 Paul Goldmann an Arthur Schnitzler, 10. 5. {[}1900{]}}
\nopagebreak\mylabel{L02915v}
\rehead{ }\normalsize\beginnumbering\briefempfaengerindex{Schnitzler, Arthur@\textsc{Schnitzler, Arthur}!zzzGoldmann, Paul@\emph{von Paul Goldmann}!1900-05-101@{10. 5. {[}1900{]}}|(be}
\toendnotes[C]{\smallbreak\pagebreak[2]}\Standort{DLA, A:Schnitzler, HS.NZ85.1.3170.}
\physDesc{Brief, 1 Blatt, 3 Seiten, 1326 Zeichen
\newline{}Handschrift: blaue Tinte, deutsche Kurrent
\newline{}Schnitzler: 1) mit Bleistift das Jahr »900« vermerkt  2) mit rotem Buntstift eine Unterstreichung}\toendnotes[C]{\smallbreak}
\pstart
           {\pb}\textcolor{gray}{\textbf{DESSAUERSTRASSE 19}}\oindex{Dessauer Strasse@\textbf{Dessauer Straße}, \emph{Straße (K.STR)}|pw}\pend
           
\pstart
           \raggedleft{}Berlin\oindex{Berlin@\textbf{Berlin}, \emph{P.PPLC}|pw}, 10. Mai.\pend
           
\pstart\center{}Mein lieber Freund,\pend\vspace{0.5em}
\pstart
           Als ich das letzte Mal in Wien\oindex{Wien@\textbf{Wien}, \emph{A.ADM2}|pw} war, ſprachen wir
               über \textsc{Rudolf Lothar\pwindex{Lothar, Rudolf 23.2.1865 – 2.10.1943@\textsc{Lothar, Rudolf} (23.2.1865 – 2.10.1943), \emph{Schriftsteller/Schriftstellerin, Journalist/Journalistin, Theaterdirektor/Theaterdirektorin}|pw}}, und Du ſagteſt, er ſei ein anſtändiger Menſch. Laß’ Dir folgenden Beitrag zu
               ſeiner Anſtändigkeit liefern:\pend
           
\pstart
           Heut bekomme ich einen Brief von der Redaktion der N. Fr. Pr.\orgindex{Neue Freie Presse@Neue Freie Presse|pw}, welcher mich informirt, daß \textsc{Lothar\pwindex{Lothar, Rudolf 23.2.1865 – 2.10.1943@\textsc{Lothar, Rudolf} (23.2.1865 – 2.10.1943), \emph{Schriftsteller/Schriftstellerin, Journalist/Journalistin, Theaterdirektor/Theaterdirektorin}|pw}} bei \textsc{Benedikt\pwindex{Benedikt, Moriz 27.05.1849 – 18.03.1920@\textsc{Benedikt, Moriz} (27.05.1849 – 18.03.1920), \emph{Journalist/Journalistin, Herausgeber/Herausgeberin}|pw}} war und erwirkt hat, daß ich über ſein Stück\pwindex{Koenig Harlekin. Maskenspiel in vier Aufzuegen@\emph{König Harlekin. Maskenspiel in vier Aufzügen}|pwv}, welches das \label{K_L02915-1v}\edtext{Volkstheater\orgindex{Volkstheater@Volkstheater|pw}}{\lemma{\textnormal{\emph{Volkstheater}}}\Cendnote{\textnormal{Nachdem die für den 31. 3. 1900 geplante Wien\oindex{Wien@\textbf{Wien}, \emph{A.ADM2}|pwk}er Premiere von Rudolf Lothars\pwindex{Lothar, Rudolf 23.2.1865 – 2.10.1943@\textsc{Lothar, Rudolf} (23.2.1865 – 2.10.1943), \emph{Schriftsteller/Schriftstellerin, Journalist/Journalistin, Theaterdirektor/Theaterdirektorin}|pwk}
                  Satire \emph{König Harlekin}\pwindex{Koenig Harlekin. Maskenspiel in vier Aufzuegen@\emph{König Harlekin. Maskenspiel in vier Aufzügen}|pwk} aus Zensurgründen
                  abgesagt worden war, kam das Stück am 19. 5. 1900 als
                  Gastspiel des Wien\oindex{Wien@\textbf{Wien}, \emph{A.ADM2}|pwk}er \emph{Volkstheaters}\orgindex{Volkstheater@Volkstheater|pwk} am Deutschen
                     Theater Berlin\oindex{Deutsches Theater Berlin@\textbf{Deutsches Theater Berlin}, \emph{Theater (K.THE)}|pwk} zur Uraufführung. In Wien\oindex{Wien@\textbf{Wien}, \emph{A.ADM2}|pwk} fand die Premiere am 14. 9. 1901
                  statt. Das Stück\pwindex{Koenig Harlekin. Maskenspiel in vier Aufzuegen@\emph{König Harlekin. Maskenspiel in vier Aufzügen}|pwkv} wurde
                  aufgrund seiner politischen Tendenzen europa\oindex{Europa@\textbf{Europa}, \emph{Kontinent (A.KNT)}|pwk}weit zensiert.}}}\label{K_L02915-1}{ }hier\oindex{Berlin@\textbf{Berlin}, \emph{P.PPLC}|pwv} zur Aufführung bringt, \uline{nicht} referire. Demgemäß erhalte ich die Weiſung, \strikeout{d\textcolor{gray}{em}} den »\textsc{König Harlekin\pwindex{Koenig Harlekin. Maskenspiel in vier Aufzuegen@\emph{König Harlekin. Maskenspiel in vier Aufzügen}|pw}}« aus meinem Referat auszuſchalten.\pend
           
\pstart
           {\pb}Das heißt alſo: Dieſer Burſche\pwindex{Lothar, Rudolf 23.2.1865 – 2.10.1943@\textsc{Lothar, Rudolf} (23.2.1865 – 2.10.1943), \emph{Schriftsteller/Schriftstellerin, Journalist/Journalistin, Theaterdirektor/Theaterdirektorin}|pwv} weiß ſehr wohl, daß ich nicht lüge
               und daß ich, wenn ſein Stück\pwindex{Koenig Harlekin. Maskenspiel in vier Aufzuegen@\emph{König Harlekin. Maskenspiel in vier Aufzügen}|pwv},
               wie vorauszuſehen, einen Mißerfolg haben wird, einen Mißerfolg conſtatiren werde.
               Darum benutzt er ſeinen Einfluß, um mich aus meinem Kritiker-Amt zu verdrängen und um
                  \introOben{}dann\introOben{} ſelbſt an die N. Fr.
                  Pr.\orgindex{Neue Freie Presse@Neue Freie Presse|pw}{ }\strikeout{d} gefälſchte \label{K_L02915-2v}\edtext{Berichte}{\lemma{\textnormal{\emph{Berichte}}}\Cendnote{\textnormal{Siehe Paul Goldmann an Arthur Schnitzler, 27. 5. [1900].
               }}}\label{K_L02915-2} abzuſenden \textsc{resp.} ſie durch eine Kreatur abſenden zu
               laſſen.\pend
           
\pstart
           Was \strikeout{\textcolor{gray}{×}} ich Dir da ſage, iſt Dienſtgeheimniß, und ich muß Dich daher um ſtrengſte
               Diskretion bitten.\pend
           
\pstart
           {\pb}Hingegen würdeſt Du mir einen großen Gefallen
               erweiſen, wenn Du allen Freunden und Bekannten mittheilen wollteſt, ich hätte Dir
               geſchrieben, daß ich über \textsc{Lothars\pwindex{Lothar, Rudolf 23.2.1865 – 2.10.1943@\textsc{Lothar, Rudolf} (23.2.1865 – 2.10.1943), \emph{Schriftsteller/Schriftstellerin, Journalist/Journalistin, Theaterdirektor/Theaterdirektorin}|pw}}{ }Stück\pwindex{Koenig Harlekin. Maskenspiel in vier Aufzuegen@\emph{König Harlekin. Maskenspiel in vier Aufzügen}|pwv} weder im Feuilleton\pwindex{Neue Freie Presse@\emph{Neue Freie Presse}|pwv} noch in der Theaterrubrik\pwindex{Neue Freie Presse@\emph{Neue Freie Presse}|pwv} berichten würde.\pend
           
\pstart
           Was treibſt Du ſonſt, mein lieber Freund? Mache mir bald wieder einmal die Freude
               eines Briefes.\pend
           
\pstart
           Viele treue Grüße! {\\[\baselineskip]}Dein {\\[\baselineskip]}\spacefill\mbox{Paul Goldmann.}\pend
           \leftskip=0em{}\selectlanguage{ngerman}\endnumbering\briefempfaengerindex{Schnitzler, Arthur@\textsc{Schnitzler, Arthur}!zzzGoldmann, Paul@\emph{von Paul Goldmann}!1900-05-101@{10. 5. {[}1900{]}}|)be}\mylabel{L02915h}  \normalsize

\doendnotes{C}
\bigskip
\vfill

\clearpage

\footnotesize

\lohead{\textsc{register}}

% Definiere theindex-Environment komplett neu ohne reledmac
\makeatletter
\renewenvironment{theindex}{%
  \section*{\indexname}%
  \setlength{\parindent}{0pt}%
  \setlength{\parskip}{0pt plus 0.3pt}%
  \let\item\@idxitem
}{%
  \clearpage
}
\makeatother

\IfFileExists{\jobname-pw.ind}{\input{\jobname-pw.ind}}{}

\end{document}

      