%% latex-leseansicht-vorspann.tex
%% Vorspann für die Leseansicht.
%% Lädt die gemeinsame Datei latex-vorspann.tex mit nicht gesetztem Schalter.

\newif\ifkorrekturansicht
\korrekturansichtfalse

\input{../tex-inputs/latex-vorspann}


         
         \renewcommand{\erwaehntePersonen}{Personen: Moriz Benedikt, Rudolf Lothar}
         \renewcommand{\erwaehnteInstitutionen}{Institutionen: Neue Freie Presse, Volkstheater}
         \renewcommand{\erwaehnteOrte}{Orte: Berlin, Dessauer Straße, Deutsches Theater Berlin, Europa, Wien}
         \renewcommand{\erwaehnteWerke}{Werke: König Harlekin. Maskenspiel in vier Aufzügen, Neue Freie Presse}
               \section[ Paul Goldmann an Arthur Schnitzler, 10. 5. {[}1900{]}]{ Paul Goldmann an Arthur Schnitzler, 10. 5. {[}1900{]}}\nopagebreak\mylabel{v}\rehead{ }\begin{ledgroupsized}[t]{13cm}\normalsize\beginnumbering \toendnotes[C]{\smallbreak\pagebreak[2]} \Standort{DLA, A:Schnitzler, HS.NZ85.1.3170.}
\physDesc{Brief, 1 Blatt, 3 Seiten, 1326 Zeichen
\newline{}Handschrift: blaue Tinte, deutsche Kurrent
\newline{}Schnitzler: 1) mit Bleistift das Jahr »900« vermerkt  2) mit rotem Buntstift eine Unterstreichung}\toendnotes[C]{\smallbreak}\pstart
           \noindent{}{\pb}\textcolor{gray}{\textbf{DESSAUERSTRASSE 19}}\oindex{Dessauer Strasse@\textbf{Dessauer Straße}|pw}\pend
           \pstart
           \raggedleft{}Berlin\oindex{Berlin@\textbf{Berlin}|pw}, 10. Mai.\pend
           \pstart\center{}Mein lieber Freund,\pend\pstart
           Als ich das letzte Mal in Wien\oindex{Wien@\textbf{Wien}|pw} war, ſprachen wir
               über \textsc{Rudolf Lothar\pwindex{Lothar, Rudolf 23.2.1865 – 2.10.1943@\textsc{Lothar, Rudolf} (23.2.1865 – 2.10.1943), \emph{Schriftsteller, Journalist, Theaterdirektor}|pw}}, und Du ſagteſt, er ſei ein anſtändiger Menſch. Laß’ Dir folgenden Beitrag zu
               ſeiner Anſtändigkeit liefern:\pend
           \pstart
           Heut bekomme ich einen Brief von der Redaktion der N. Fr. Pr.\orgindex{Neue Freie Presse@Neue Freie Presse|pw}, welcher mich informirt, daß \textsc{Lothar\pwindex{Lothar, Rudolf 23.2.1865 – 2.10.1943@\textsc{Lothar, Rudolf} (23.2.1865 – 2.10.1943), \emph{Schriftsteller, Journalist, Theaterdirektor}|pw}} bei \textsc{Benedikt\pwindex{Benedikt, Moriz 27.05.1849 – 18.03.1920@\textsc{Benedikt, Moriz} (27.05.1849 – 18.03.1920), \emph{Journalist, Herausgeber}|pw}} war und erwirkt hat, daß ich über ſein Stück\pwindex{Lothar, Rudolf 23.2.1865 – 2.10.1943@\textsc{Lothar, Rudolf} (23.2.1865 – 2.10.1943), \emph{Schriftsteller, Journalist, Theaterdirektor}!Koenig Harlekin. Maskenspiel in vier Aufzuegen1900@\strich\emph{König Harlekin. Maskenspiel in vier Aufzügen} {[}1900{]}|pwv}, welches das \label{K_L02915-1v}\edtext{Volkstheater\orgindex{Volkstheater@Volkstheater|pw}}{\lemma{\textnormal{\emph{Volkstheater}}}\Cendnote{\textnormal{Nachdem die für den 31. 3. 1900 geplante Wien\oindex{Wien@\textbf{Wien}|pwk}er Premiere von Rudolf Lothar\pwindex{Lothar, Rudolf 23.2.1865 – 2.10.1943@\textsc{Lothar, Rudolf} (23.2.1865 – 2.10.1943), \emph{Schriftsteller, Journalist, Theaterdirektor}|pwk}s
                  Satire \emph{König Harlekin}\pwindex{Lothar, Rudolf 23.2.1865 – 2.10.1943@\textsc{Lothar, Rudolf} (23.2.1865 – 2.10.1943), \emph{Schriftsteller, Journalist, Theaterdirektor}!Koenig Harlekin. Maskenspiel in vier Aufzuegen1900@\strich\emph{König Harlekin. Maskenspiel in vier Aufzügen} {[}1900{]}|pwk} aus Zensurgründen
                  abgesagt worden war, kam es am 19. 5. 1900 als
                  Gastspiel des Wien\oindex{Wien@\textbf{Wien}|pwk}er \emph{Volkstheater}\orgindex{Volkstheater@Volkstheater|pwk}s am Deutschen
                     Theater Berlin\oindex{Deutsches Theater Berlin@\textbf{Deutsches Theater Berlin}|pwk} zur Uraufführung. In Wien\oindex{Wien@\textbf{Wien}|pwk} fand die Premiere am 14. 9. 1901
                  statt. Das Stück\pwindex{Lothar, Rudolf 23.2.1865 – 2.10.1943@\textsc{Lothar, Rudolf} (23.2.1865 – 2.10.1943), \emph{Schriftsteller, Journalist, Theaterdirektor}!Koenig Harlekin. Maskenspiel in vier Aufzuegen1900@\strich\emph{König Harlekin. Maskenspiel in vier Aufzügen} {[}1900{]}|pwkv} wurde
                  aufgrund seiner politischen Tendenzen europa\oindex{Europa@\textbf{Europa}|pwk}weit zensiert.}}}\label{K_L02915-1h}{ }hier\oindex{Berlin@\textbf{Berlin}|pwv} zur Aufführung bringt, \uline{nicht} referire. Demgemäß erhalte ich die Weiſung, \strikeout{d\textcolor{gray}{em}} den »\textsc{König Harlekin\pwindex{Lothar, Rudolf 23.2.1865 – 2.10.1943@\textsc{Lothar, Rudolf} (23.2.1865 – 2.10.1943), \emph{Schriftsteller, Journalist, Theaterdirektor}!Koenig Harlekin. Maskenspiel in vier Aufzuegen1900@\strich\emph{König Harlekin. Maskenspiel in vier Aufzügen} {[}1900{]}|pw}}« aus meinem Referat auszuſchalten.\pend
           \pstart
           {\pb}Das heißt alſo: Dieſer Burſche\pwindex{Lothar, Rudolf 23.2.1865 – 2.10.1943@\textsc{Lothar, Rudolf} (23.2.1865 – 2.10.1943), \emph{Schriftsteller, Journalist, Theaterdirektor}|pwv} weiß ſehr wohl, daß ich nicht lüge
               und daß ich, wenn ſein Stück\pwindex{Lothar, Rudolf 23.2.1865 – 2.10.1943@\textsc{Lothar, Rudolf} (23.2.1865 – 2.10.1943), \emph{Schriftsteller, Journalist, Theaterdirektor}!Koenig Harlekin. Maskenspiel in vier Aufzuegen1900@\strich\emph{König Harlekin. Maskenspiel in vier Aufzügen} {[}1900{]}|pwv},
               wie vorauszuſehen, einen Mißerfolg haben wird, einen Mißerfolg conſtatiren werde.
               Darum benutzt er ſeinen Einfluß, um mich aus meinem Kritiker-Amt zu verdrängen und um
                  \introOben{}dann\introOben{} ſelbſt an die N. Fr.
                  Pr.\orgindex{Neue Freie Presse@Neue Freie Presse|pw}{ }\strikeout{d} gefälſchte \label{K_L02915-2v}\edtext{Berichte}{\lemma{\textnormal{\emph{Berichte}}}\Cendnote{\textnormal{siehe Paul Goldmann an Arthur Schnitzler, 27. 5. [1900]}}}\label{K_L02915-2h} abzuſenden \textsc{resp.} ſie durch eine Kreatur abſenden zu
               laſſen.\pend
           \pstart
           Was \strikeout{\textcolor{gray}{×}} ich Dir da ſage, iſt Dienſtgeheimniß, und ich muß Dich daher um ſtrengſte
               Diskretion bitten.\pend
           \pstart
           {\pb}Hingegen würdeſt Du mir einen großen Gefallen
               erweiſen, wenn Du allen Freunden und Bekannten mittheilen wollteſt, ich hätte Dir
               geſchrieben, daß ich über \textsc{Lothar\pwindex{Lothar, Rudolf 23.2.1865 – 2.10.1943@\textsc{Lothar, Rudolf} (23.2.1865 – 2.10.1943), \emph{Schriftsteller, Journalist, Theaterdirektor}|pw}s}{ }Stück\pwindex{Lothar, Rudolf 23.2.1865 – 2.10.1943@\textsc{Lothar, Rudolf} (23.2.1865 – 2.10.1943), \emph{Schriftsteller, Journalist, Theaterdirektor}!Koenig Harlekin. Maskenspiel in vier Aufzuegen1900@\strich\emph{König Harlekin. Maskenspiel in vier Aufzügen} {[}1900{]}|pwv} weder im Feuilleton\pwindex{Neue Freie Presse1864 – 1939@\emph{Neue Freie Presse} {[}1864 – 1939{]}|pwv} noch in der Theaterrubrik\pwindex{Neue Freie Presse1864 – 1939@\emph{Neue Freie Presse} {[}1864 – 1939{]}|pwv} berichten würde.\pend
           \pstart
           Was treibſt Du ſonſt, mein lieber Freund? Mache mir bald wieder einmal die Freude
               eines Briefes.\pend
           \pstart
           Viele treue Grüße! {\\[\baselineskip]}Dein {\\[\baselineskip]}\spacefill\mbox{Paul Goldmann.}\pend
           \leftskip=0em{}
         
         \endnumbering\mylabel{h}\end{ledgroupsized}  \newcommand{\dateiname}{L02915}\newcommand{\titel}{Paul Goldmann an Arthur Schnitzler, 10. 5. [1900]}\newcommand{\editorInnen}{Martin Anton Müller und Laura Untner}%% latex-leseansicht-abspann.tex
%% Abspann für die Leseansicht.
%% Der Schalter \ifkorrekturansicht ist bereits durch den Vorspann gesetzt.

%% latex-abspann.tex
%% Gemeinsamer Abspann für Korrekturansicht und Leseansicht.
%% Setzt den Schalter \ifkorrekturansicht voraus (gesetzt in den
%% einbindenden Dateien latex-korrekturansicht-abspann.tex bzw.
%% latex-leseansicht-abspann.tex).
%% ---------------------------------------------------------------

\normalsize

% Das esempio-Environment wird nur in der Leseansicht benötigt
\ifkorrekturansicht\else
\newenvironment{esempio}[3]%
{
    \vspace{1.5ex}
    \rlap{\underline{#1}}
    \par
    \setlength{\parindent}{0cm}
    \nopagebreak
    \leftskip=#2cm
    \rightskip=#3cm
}
{
    \par
}
\fi

\doendnotes{C}
\bigskip
\vfill

\clearpage

\footnotesize

\ifkorrekturansicht
  \lohead{\textsc{register}}
\fi

% theindex-Environment neu definieren ohne reledmac
\makeatletter
\renewenvironment{theindex}{%
  \ifkorrekturansicht
    \section*{\indexname}%
  \else
    \subsubsection*{Index der erwähnten Entitäten}%
  \fi
  \setlength{\parindent}{0pt}%
  \setlength{\parskip}{0pt plus 0.3pt}%
  \let\item\@idxitem
}{%
  \ifkorrekturansicht\clearpage\fi
}
\makeatother

\IfFileExists{\jobname-pw.ind}{\input{\jobname-pw.ind}}{}

% Quellenangabe nur in der Leseansicht
\ifkorrekturansicht\else
% Fallback-Definitionen, falls die .tex-Datei \titel etc. nicht gesetzt hat
\providecommand{\titel}{}
\providecommand{\editorInnen}{}
\providecommand{\dateiname}{\jobname}

\vspace{3cm}

\vfill

\footnotesize
\textsc{Quelle}: \titel. Herausgegeben von {\editorInnen}. In: \emph{Arthur Schnitzler: Briefwechsel mit Autorinnen und Autoren}.
 Digitale Edition, https://schnitzler-briefe.acdh.oeaw.ac.at/{\dateiname}.html (Stand \today)
\fi

\end{document}


      