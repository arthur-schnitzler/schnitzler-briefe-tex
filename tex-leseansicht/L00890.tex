%% latex-korrekturansicht-vorspann.tex
%% Vorspann für die Korrekturansicht.
%% Lädt die gemeinsame Datei latex-vorspann.tex mit gesetztem Schalter.

\newif\ifkorrekturansicht
\korrekturansichttrue

\input{../tex-inputs/latex-vorspann}


\section[Arthur Schnitzler an Hermann Bahr, Antwort auf eine Umfrage, 15. 2. 1899]{L00890 Arthur Schnitzler an Hermann Bahr, Antwort auf eine Umfrage,
               15. 2. 1899}
\nopagebreak\mylabel{L00890v}
\rehead{ }\normalsize\beginnumbering\briefempfaengerindex{Bahr, Hermann@\textsc{Bahr, Hermann}!zzzSchnitzler, Arthur@\emph{von Arthur Schnitzler}!1899-02-151@{15. 2. 1899}|(be}
\toendnotes[C]{\smallbreak\pagebreak[2]}\buchAlsQuelle{\pwindex{Zeit. Wiener Wochenschrift@\emph{Die Zeit. Wiener Wochenschrift}|pwk}\emph{Die Zeit}, Bd. 18, Nr. 229, 18. 2. 1899, S. 104–106, hier: S. 105.}
\buchAbdrucke{\weitereDrucke{Hermann Bahr, Arthur Schnitzler: \emph{Briefwechsel, Aufzeichnungen, Dokumente (1891–1931)}. Göttingen: \emph{Wallstein} 2018, S. 167–168.} }\toendnotes[C]{\smallbreak}
\pstart
           \raggedleft{}{\pb}Wien\oindex{Wien@\textbf{Wien}, \emph{A.ADM2}|pw}, 15. Februar
                     1899. \pend
           
\pstart
           \centering{}\so{Lieber Bahr!}\pend
           \vspace{0.5em}
\pstart
           \label{K_L00890-1v}\edtext{Ob ein gerufener Autor erſcheinen
                  ſoll}{\lemma{\textnormal{\emph{Ob … ſoll}}}\Cendnote{\textnormal{Der Brief erschien zusammen mit
                  weiteren Antworten nach folgender wohl von Bahr\pwindex{Bahr, Hermann 19.07.1863 – 15.01.1934@\textsc{Bahr, Hermann} (19.07.1863 – 15.01.1934), \emph{Schriftsteller/Schriftstellerin, Kritiker/Kritikerin}|pwk} verfasster Einleitung: »Zu dem Aufsatze ›Premièren\pwindex{Premieren. (Zur Premiere des Lustspiels »Unser Kaethchen« von Theodor Herzl im Deutschen Volkstheater am 4. Februar 1898)@\emph{Premièren. (Zur Première des Lustspiels »Unser Käthchen« von Theodor Herzl im Deutschen Volkstheater am 4. Februar 1898)}|pw}‹ in Nr. 228 der ›Zeit\pwindex{Zeit. Wiener Wochenschrift@\emph{Die Zeit. Wiener Wochenschrift}|pw}‹, welcher anregte, dass sich die Autoren bei ihren
                     Premièren nicht mehr dem Publicum zeigen sollen, sind uns folgende Zuschriften
                     zugekommen:« Die anderen Antworten, durchweg in Form eines an Bahr
                  gerichteten Briefes: Emerich von Bukovics\pwindex{Bukovics, Emerich von 28.02.1844 – 04.07.1905@\textsc{Bukovics, Emerich von} (28.02.1844 – 04.07.1905), \emph{Journalist/Journalistin, Theaterleiter/Theaterleiterin}|pwk},
                     Ernst Gettke\pwindex{Gettke, Ernst 08.10.1841 – 04.12.1912@\textsc{Gettke, Ernst} (08.10.1841 – 04.12.1912), \emph{Schriftsteller/Schriftstellerin, Theaterleiter/Theaterleiterin, Regisseur/Regisseurin}|pwk}, Leo Ebermann\pwindex{Ebermann, Leo 16.07.1863 – 09.10.1914@\textsc{Ebermann, Leo} (16.07.1863 – 09.10.1914), \emph{Schriftsteller/Schriftstellerin, Journalist/Journalistin, Rechtswissenschaftler/Rechtswissenschaftlerin}|pwk}, Carl
                     Karlweis\pwindex{Karlweis, Carl 23.11.1850 – 27.10.1901@\textsc{Karlweis, Carl} (23.11.1850 – 27.10.1901), \emph{Schriftsteller/Schriftstellerin}|pwk}, Philipp Langmann\pwindex{Langmann, Philipp 05.02.1862 – 22.05.1931@\textsc{Langmann, Philipp} (05.02.1862 – 22.05.1931), \emph{Schriftsteller/Schriftstellerin, Journalist/Journalistin}|pwk}, Victor Léon\pwindex{Leon, Victor 4.1.1858 – 23.2.1940@\textsc{Léon, Victor} (4.1.1858 – 23.2.1940), \emph{Schriftsteller/Schriftstellerin, Dramaturg/Dramaturgin}|pwk}, Oskar Blumenthal\pwindex{Blumenthal, Oskar 13.03.1852 – 24.04.1917@\textsc{Blumenthal, Oskar} (13.03.1852 – 24.04.1917), \emph{Schriftsteller/Schriftstellerin, Journalist/Journalistin, Theaterleiter/Theaterleiterin}|pwk}, Ernst
                     von Wildenbruch\pwindex{Wildenbruch, Ernst von 03.02.1845 – 15.01.1909@\textsc{Wildenbruch, Ernst von} (03.02.1845 – 15.01.1909), \emph{Schriftsteller/Schriftstellerin}|pwk} und Otto Erich
                     Hartleben\pwindex{Hartleben, Otto Erich 03.06.1864 – 11.02.1905@\textsc{Hartleben, Otto Erich} (03.06.1864 – 11.02.1905), \emph{Schriftsteller/Schriftstellerin}|pwk}; die Antwort von Max Grube\pwindex{Grube, Max 1854-03-25 – 1934-12-25@\textsc{Grube, Max} (1854-03-25 – 1934-12-25), \emph{Theaterleiter/Theaterleiterin, Schauspieler/Schauspielerin}|pwk}
                  in Gestalt eines Gedichts. Auf eine Reaktion\pwindex{Unser Kaethchen« [Brief an Hermann Bahr]@\emph{»Unser Käthchen« [Brief an Hermann Bahr]}|pwkv}{ }Theodor Herzls\pwindex{Herzl, Theodor 1860-05-02 – 1904-07-03@\textsc{Herzl, Theodor} (1860-05-02 – 1904-07-03), \emph{Schriftsteller/Schriftstellerin, Journalist/Journalistin}|pwk} in der \emph{Neuen Freien Presse}\pwindex{Neue Freie Presse@\emph{Neue Freie Presse}|pwk} vom 12. 2. 1899
                     (Nr. 12.384, S. 8) wurde hingewiesen.}}}\label{K_L00890-1} oder nicht? Nichts iſt
               gleichgiltiger für das innere Schickſal der Première; nichts gleichgiltiger für das
               fernere Schickſal des betreffenden Stückes. Jeder Autor möge es daher in jedem Falle
               halten, wie es ihm beliebt. In Geſchmacks- und Stimmungsfragen gibt es keine
               Solidarität.\pend
           
\pstart
           Herzlichen Gruß. \label{K_L00890-2v}\edtext{Dein}{\lemma{\textnormal{\emph{Dein}}}\Cendnote{\textnormal{Drei weitere Antworten geben
                  Duzbrüderschaft mit Bahr zu erkennen: Bukovics\pwindex{Bukovics, Emerich von 28.02.1844 – 04.07.1905@\textsc{Bukovics, Emerich von} (28.02.1844 – 04.07.1905), \emph{Journalist/Journalistin, Theaterleiter/Theaterleiterin}|pwk}, Ebermann\pwindex{Ebermann, Leo 16.07.1863 – 09.10.1914@\textsc{Ebermann, Leo} (16.07.1863 – 09.10.1914), \emph{Schriftsteller/Schriftstellerin, Journalist/Journalistin, Rechtswissenschaftler/Rechtswissenschaftlerin}|pwk} und Karlweis\pwindex{Karlweis, Carl 23.11.1850 – 27.10.1901@\textsc{Karlweis, Carl} (23.11.1850 – 27.10.1901), \emph{Schriftsteller/Schriftstellerin}|pwk}.}}}\label{K_L00890-2} ergebener{\\[\baselineskip]}\spacefill\mbox{Arthur Schnitzler.}\pend
           \leftskip=0em{}\selectlanguage{ngerman}\endnumbering\briefempfaengerindex{Bahr, Hermann@\textsc{Bahr, Hermann}!zzzSchnitzler, Arthur@\emph{von Arthur Schnitzler}!1899-02-151@{15. 2. 1899}|)be}\mylabel{L00890h}  \normalsize

\doendnotes{C}
\bigskip
\vfill

\clearpage

\footnotesize

\lohead{\textsc{register}}

% Definiere theindex-Environment komplett neu ohne reledmac
\makeatletter
\renewenvironment{theindex}{%
  \section*{\indexname}%
  \setlength{\parindent}{0pt}%
  \setlength{\parskip}{0pt plus 0.3pt}%
  \let\item\@idxitem
}{%
  \clearpage
}
\makeatother

\IfFileExists{\jobname-pw.ind}{\input{\jobname-pw.ind}}{}

\end{document}

      