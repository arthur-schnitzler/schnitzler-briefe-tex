%% latex-leseansicht-vorspann.tex
%% Vorspann für die Leseansicht.
%% Lädt die gemeinsame Datei latex-vorspann.tex mit nicht gesetztem Schalter.

\newif\ifkorrekturansicht
\korrekturansichtfalse

\input{../tex-inputs/latex-vorspann}


         
         \renewcommand{\erwaehntePersonen}{Personen: Hermann Bahr, Oskar Blumenthal, Emerich von Bukovics, Leo Ebermann, Ernst Gettke, Max Grube, Otto Erich Hartleben, Theodor Herzl, Carl Karlweis, Philipp Langmann, Victor Léon, Ernst von Wildenbruch}
         \renewcommand{\erwaehnteOrte}{Orte: Wien}
         \renewcommand{\erwaehnteWerke}{Werke: Die Zeit. Wiener Wochenschrift, Neue Freie Presse, Premièren. (Zur Première des Lustspiels »Unser Käthchen« von Theodor Herzl im Deutschen Volkstheater am 4. Februar 1898), »Unser Käthchen« [Brief an Hermann Bahr]}
               \section[Arthur Schnitzler an Hermann Bahr, Antwort auf eine Umfrage, 15. 2. 1899]{ Arthur Schnitzler an Hermann Bahr, Antwort auf eine Umfrage,
               15. 2. 1899}\nopagebreak\mylabel{v}\rehead{ }\begin{ledgroupsized}[t]{13cm}\normalsize\beginnumbering \toendnotes[C]{\smallbreak\pagebreak[2]} \buchAlsQuelle{\pwindex{Zeit. Wiener Wochenschrift1894 – 1904@\emph{Die Zeit. Wiener Wochenschrift} {[}1894 – 1904{]}|pwk}Arthur Schnitzler: \emph{[Das Erscheinen der Autoren].} In: \emph{Die Zeit}, Bd. 18, Nr. 229, 18. 2. 1899, S. 104–106, hier: S. 105.}\buchAbdrucke{\weitereDrucke{Hermann Bahr, Arthur Schnitzler: \emph{Briefwechsel, Aufzeichnungen, Dokumente (1891–1931)}. Hg. Kurt Ifkovits und Martin Anton Müller. Göttingen: \emph{Wallstein} 2018, S. 167–168.} }\toendnotes[C]{\smallbreak}\pstart
           \raggedleft{}{\pb}Wien\oindex{Wien@\textbf{Wien}|pw}, 15. Februar
                     1899. \pend
           \pstart
           \centering{}\so{Lieber Bahr!}\pend
           \pstart
           \label{K_L00890_1v}\edtext{Ob ein gerufener Autor erſcheinen
                  ſoll}{\lemma{\textnormal{\emph{Ob … ſoll}}}\Cendnote{\textnormal{Der Brief erschien zusammen mit
                  weiteren Antworten nach folgender wohl von Bahr\pwindex{Bahr, Hermann 19.07.1863 – 15.01.1934@\textsc{Bahr, Hermann} (19.07.1863 – 15.01.1934), \emph{Schriftsteller, Kritiker}|pwk} verfasster Einleitung: »Zu dem Aufsatze ›Premièren\pwindex{Bahr, Hermann 19.07.1863 – 15.01.1934@\textsc{Bahr, Hermann} (19.07.1863 – 15.01.1934), \emph{Schriftsteller, Kritiker}!Premieren. (Zur Premiere des Lustspiels »Unser Kaethchen« von Theodor Herzl im Deutschen Volkstheater am 4. Februar 1898)11. 02. 1899@\strich\emph{Premièren. (Zur Première des Lustspiels »Unser Käthchen« von Theodor Herzl im Deutschen Volkstheater am 4. Februar 1898)} {[}11. 02. 1899{]}|pw}‹ in Nr. 228 der ›Zeit\pwindex{Zeit. Wiener Wochenschrift1894 – 1904@\emph{Die Zeit. Wiener Wochenschrift} {[}1894 – 1904{]}|pw}‹, welcher anregte, dass sich die Autoren bei ihren Premièren
                     nicht mehr dem Publicum zeigen sollen, sind uns folgende Zuschriften
                     zugekommen:« Die anderen Antworten, durchwegs in Form eines an Bahr
                  gerichteten Briefes: Emerich von Bukovics\pwindex{Bukovics, Emerich von 28.02.1844 – 04.07.1905@\textsc{Bukovics, Emerich von} (28.02.1844 – 04.07.1905), \emph{Journalist, Theaterleiter}|pwk}, Ernst Gettke\pwindex{Gettke, Ernst 08.10.1841 – 04.12.1912@\textsc{Gettke, Ernst} (08.10.1841 – 04.12.1912), \emph{Schriftsteller, Theaterleiter, Regisseur}|pwk}, Leo
                     Ebermann\pwindex{Ebermann, Leo 16.07.1863 – 09.10.1914@\textsc{Ebermann, Leo} (16.07.1863 – 09.10.1914), \emph{Schriftsteller, Journalist, Rechtswissenschaftler}|pwk}, Carl Karlweis\pwindex{Karlweis, Carl 23.11.1850 – 27.10.1901@\textsc{Karlweis, Carl} (23.11.1850 – 27.10.1901), \emph{Schriftsteller}|pwk}, Philipp Langmann\pwindex{Langmann, Philipp 05.02.1862 – 22.05.1931@\textsc{Langmann, Philipp} (05.02.1862 – 22.05.1931), \emph{Schriftsteller, Journalist, Schriftsteller}|pwk}, Victor Léon\pwindex{Leon, Victor 4.1.1858 – 23.2.1940@\textsc{Léon, Victor} (4.1.1858 – 23.2.1940), \emph{Schriftsteller, Dramaturg}|pwk}, Oskar Blumenthal\pwindex{Blumenthal, Oskar 13.03.1852 – 24.04.1917@\textsc{Blumenthal, Oskar} (13.03.1852 – 24.04.1917), \emph{Schriftsteller, Journalist, Theaterleiter}|pwk}, Ernst von Wildenbruch\pwindex{Wildenbruch, Ernst von 03.02.1845 – 15.01.1909@\textsc{Wildenbruch, Ernst von} (03.02.1845 – 15.01.1909), \emph{Schriftsteller}|pwk} und Otto Erich Hartleben\pwindex{Hartleben, Otto Erich 03.06.1864 – 11.02.1905@\textsc{Hartleben, Otto Erich} (03.06.1864 – 11.02.1905), \emph{Schriftsteller}|pwk}; die Antwort von Max Grube\pwindex{Grube, Max 1854-03-25 – 1934-12-25@\textsc{Grube, Max} (1854-03-25 – 1934-12-25), \emph{Theaterleiter, Schauspieler}|pwk} in Gestalt eines Gedichts. Auf eine Reaktion\pwindex{Herzl, Theodor 1860-05-02 – 1904-07-03@\textsc{Herzl, Theodor} (1860-05-02 – 1904-07-03), \emph{Schriftsteller, Journalist}!Unser Kaethchen« [Brief an Hermann Bahr]12. 02. 1899@\strich\emph{»Unser Käthchen« [Brief an Hermann Bahr]} {[}12. 02. 1899{]}|pwkv}{ }Theodor Herzls\pwindex{Herzl, Theodor 1860-05-02 – 1904-07-03@\textsc{Herzl, Theodor} (1860-05-02 – 1904-07-03), \emph{Schriftsteller, Journalist}|pwk} in der \emph{Neuen Freien Presse}\pwindex{Neue Freie Presse1864 – 1939@\emph{Neue Freie Presse} {[}1864 – 1939{]}|pwk} vom 12. 2. 1899
                     (Nr. 12384, S. 8) wird hingewiesen.}}}\label{K_L00890_1h} oder nicht? Nichts iſt
               gleichgiltiger für das innere Schickſal der Première; nichts gleichgiltiger für das
               fernere Schickſal des betreffenden Stückes. Jeder Autor möge es daher in jedem Falle
               halten, wie es ihm beliebt. In Geſchmacks- und Stimmungsfragen gibt es keine
               Solidarität.\pend
           \pstart
           Herzlichen Gruß. \label{K_L00890_2v}\edtext{Dein}{\lemma{\textnormal{\emph{Dein}}}\Cendnote{\textnormal{Drei weitere Antworten geben
                  Duzbrüderschaft mit Bahr zu erkennen: Bukovics\pwindex{Bukovics, Emerich von 28.02.1844 – 04.07.1905@\textsc{Bukovics, Emerich von} (28.02.1844 – 04.07.1905), \emph{Journalist, Theaterleiter}|pwk},
                     Ebermann\pwindex{Ebermann, Leo 16.07.1863 – 09.10.1914@\textsc{Ebermann, Leo} (16.07.1863 – 09.10.1914), \emph{Schriftsteller, Journalist, Rechtswissenschaftler}|pwk} und Karlweis\pwindex{Karlweis, Carl 23.11.1850 – 27.10.1901@\textsc{Karlweis, Carl} (23.11.1850 – 27.10.1901), \emph{Schriftsteller}|pwk}.}}}\label{K_L00890_2h} ergebener{\\[\baselineskip]}\spacefill\mbox{Arthur Schnitzler.}\pend
           \leftskip=0em{}
         
         \endnumbering\mylabel{h}\end{ledgroupsized}  \newcommand{\dateiname}{L00890}\newcommand{\titel}{Arthur Schnitzler an Hermann Bahr, Antwort auf eine Umfrage, 15. 2. 1899}\newcommand{\editorInnen}{Martin Anton Müller und Gerd-Hermann Susen}%% latex-leseansicht-abspann.tex
%% Abspann für die Leseansicht.
%% Der Schalter \ifkorrekturansicht ist bereits durch den Vorspann gesetzt.

%% latex-abspann.tex
%% Gemeinsamer Abspann für Korrekturansicht und Leseansicht.
%% Setzt den Schalter \ifkorrekturansicht voraus (gesetzt in den
%% einbindenden Dateien latex-korrekturansicht-abspann.tex bzw.
%% latex-leseansicht-abspann.tex).
%% ---------------------------------------------------------------

\normalsize

% Das esempio-Environment wird nur in der Leseansicht benötigt
\ifkorrekturansicht\else
\newenvironment{esempio}[3]%
{
    \vspace{1.5ex}
    \rlap{\underline{#1}}
    \par
    \setlength{\parindent}{0cm}
    \nopagebreak
    \leftskip=#2cm
    \rightskip=#3cm
}
{
    \par
}
\fi

\doendnotes{C}
\bigskip
\vfill

\clearpage

\footnotesize

\ifkorrekturansicht
  \lohead{\textsc{register}}
\fi

% theindex-Environment neu definieren ohne reledmac
\makeatletter
\renewenvironment{theindex}{%
  \ifkorrekturansicht
    \section*{\indexname}%
  \else
    \subsubsection*{Index der erwähnten Entitäten}%
  \fi
  \setlength{\parindent}{0pt}%
  \setlength{\parskip}{0pt plus 0.3pt}%
  \let\item\@idxitem
}{%
  \ifkorrekturansicht\clearpage\fi
}
\makeatother

\IfFileExists{\jobname-pw.ind}{\input{\jobname-pw.ind}}{}

% Quellenangabe nur in der Leseansicht
\ifkorrekturansicht\else
% Fallback-Definitionen, falls die .tex-Datei \titel etc. nicht gesetzt hat
\providecommand{\titel}{}
\providecommand{\editorInnen}{}
\providecommand{\dateiname}{\jobname}

\vspace{3cm}

\vfill

\footnotesize
\textsc{Quelle}: \titel. Herausgegeben von {\editorInnen}. In: \emph{Arthur Schnitzler: Briefwechsel mit Autorinnen und Autoren}.
 Digitale Edition, https://schnitzler-briefe.acdh.oeaw.ac.at/{\dateiname}.html (Stand \today)
\fi

\end{document}


      