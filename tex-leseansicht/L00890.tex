%% latex-leseansicht-vorspann.tex
%% Vorspann für die Leseansicht.
%% Lädt die gemeinsame Datei latex-vorspann.tex mit nicht gesetztem Schalter.

\newif\ifkorrekturansicht
\korrekturansichtfalse

\input{../tex-inputs/latex-vorspann}


\section[Arthur Schnitzler an Hermann Bahr, Antwort auf eine Umfrage, 15. 2. 1899]{L00890 Arthur Schnitzler an Hermann Bahr, Antwort auf eine Umfrage, 15. 2. 1899}
\nopagebreak\mylabel{L00890v}
\rehead{ }\normalsize\beginnumbering\briefempfaengerindex{Bahr, Hermann@\textsc{Bahr, Hermann}!zzzSchnitzler, Arthur@\emph{von Arthur Schnitzler}!1899-02-151@{15. 2. 1899}|(be}
\toendnotes[C]{\smallbreak\pagebreak[2]}
\correspDesc{Versand  durch Arthur Schnitzler am 15. 2. 1899 in Wien
\newline{}Erhalt  durch Hermann Bahr im Zeitraum [15. 2. 1899
                  – 19. 2. 1899?] in Wien}\toendnotes[C]{\smallbreak}
\buchAlsQuelle{\pwindex{Zeit. Wiener Wochenschrift@\emph{Die Zeit. Wiener Wochenschrift}|pwk}Arthur Schnitzler: \emph{[Das Erscheinen der Autoren].} In: \emph{Die Zeit}, Bd. 18, Nr. 229, 18. 2. 1899, S. 104–106, hier: S. 105.}
\buchAbdrucke{\weitereDrucke{Hermann Bahr, Arthur Schnitzler: \emph{Briefwechsel, Aufzeichnungen, Dokumente (1891–1931)}. Herausgegeben von Kurt Ifkovits und Martin Anton Müller. Göttingen: \emph{Wallstein} 2018, S. 167–168.} }\toendnotes[C]{\smallbreak}
\pstart
           \raggedleft{}{\pb}Wien\oindex{Wien@\textbf{Wien}, \emph{Verwaltungsgebiet}|pw}, 15. Februar 1899.\pend
           
\pstart
           \centering{}\so{Lieber Bahr!}\pend
           \vspace{0.5em}
\pstart
           \label{K_L00890-1v}\edtext{Ob ein gerufener Autor erſcheinen{ }ſoll}{\lemma{\textnormal{\emph{Ob … soll}}}\Cendnote{\textnormal{Der Brief erschien zusammen mit
                  weiteren Antworten nach folgender wohl von Bahr\pwindex{Bahr, Hermann 19.\,7.\,1863 Linz – 15.\,1.\,1934 München@\textsc{Bahr, Hermann} (19.\,7.\,1863 Linz – 15.\,1.\,1934 München), \emph{Schriftsteller, Kritiker}|pwk} verfasster Einleitung: »Zu dem Aufsatze ›Premièren\pwindex{Bahr, Hermann 19.\,7.\,1863 Linz – 15.\,1.\,1934 München@\textsc{Bahr, Hermann} (19.\,7.\,1863 Linz – 15.\,1.\,1934 München), \emph{Schriftsteller, Kritiker}!Premièren. (Zur Première des Lustspiels »Unser Käthchen« von Theodor Herzl im Deutschen Volkstheater am 4. Februar 1898)@\strich\emph{Premièren. (Zur Première des Lustspiels »Unser Käthchen« von Theodor Herzl im Deutschen Volkstheater am 4. Februar 1898)}|pw}‹ in Nr. 228 der ›Zeit\pwindex{Zeit. Wiener Wochenschrift@\emph{Die Zeit. Wiener Wochenschrift}|pw}‹, welcher anregte, dass sich die Autoren bei ihren
                     Premièren nicht mehr dem Publicum zeigen sollen, sind uns folgende Zuschriften
                     zugekommen:« Die anderen Antworten, durchweg in Form eines an Bahr
                  gerichteten Briefes: Emerich von Bukovics\pwindex{Bukovics, Emerich von 28.\,2.\,1844 Wien – 4.\,7.\,1905 ebd.@\textsc{Bukovics, Emerich von} (28.\,2.\,1844 Wien – 4.\,7.\,1905 ebd.), \emph{Journalist, Theaterleiter}|pwk},
                     Ernst Gettke\pwindex{Gettke, Ernst 8.\,10.\,1841 Berlin – 4.\,12.\,1912 ebd.@\textsc{Gettke, Ernst} (8.\,10.\,1841 Berlin – 4.\,12.\,1912 ebd.), \emph{Schriftsteller, Theaterleiter, Regisseur}|pwk}, Leo Ebermann\pwindex{Ebermann, Leo 16.\,7.\,1863 Draganovka – 9.\,10.\,1914 Wien@\textsc{Ebermann, Leo} (16.\,7.\,1863 Draganovka – 9.\,10.\,1914 Wien), \emph{Schriftsteller, Journalist, Rechtswissenschaftler}|pwk}, Carl
                     Karlweis\pwindex{Karlweis, Carl 23.\,11.\,1850 Wien – 27.\,10.\,1901 ebd.@\textsc{Karlweis, Carl} (23.\,11.\,1850 Wien – 27.\,10.\,1901 ebd.), \emph{Schriftsteller}|pwk}, Philipp Langmann\pwindex{Langmann, Philipp 5.\,2.\,1862 Brünn – 22.\,5.\,1931 Wien@\textsc{Langmann, Philipp} (5.\,2.\,1862 Brünn – 22.\,5.\,1931 Wien), \emph{Schriftsteller, Journalist}|pwk}, Victor Léon\pwindex{Léon, Victor 4.\,1.\,1858 Senica – 23.\,2.\,1940 Wien@\textsc{Léon, Victor} (4.\,1.\,1858 Senica – 23.\,2.\,1940 Wien), \emph{Schriftsteller, Dramaturg}|pwk}, Oskar Blumenthal\pwindex{Blumenthal, Oskar 13.\,3.\,1852 Berlin – 24.\,4.\,1917 ebd.@\textsc{Blumenthal, Oskar} (13.\,3.\,1852 Berlin – 24.\,4.\,1917 ebd.), \emph{Schriftsteller, Journalist, Theaterleiter}|pwk}, Ernst
                     von Wildenbruch\pwindex{Wildenbruch, Ernst von 3.\,2.\,1845 Beirut – 15.\,1.\,1909 Berlin@\textsc{Wildenbruch, Ernst von} (3.\,2.\,1845 Beirut – 15.\,1.\,1909 Berlin), \emph{Schriftsteller}|pwk} und Otto Erich
                     Hartleben\pwindex{Hartleben, Otto Erich 3.\,6.\,1864 Clausthal-Zellerfeld – 11.\,2.\,1905 Salò@\textsc{Hartleben, Otto Erich} (3.\,6.\,1864 Clausthal-Zellerfeld – 11.\,2.\,1905 Salò), \emph{Schriftsteller}|pwk}; die Antwort von Max Grube\pwindex{Grube, Max 25.\,3.\,1854 Tartu – 25.\,12.\,1934 Meiningen@\textsc{Grube, Max} (25.\,3.\,1854 Tartu – 25.\,12.\,1934 Meiningen), \emph{Theaterleiter, Schauspieler}|pwk}
                  in Gestalt eines Gedichts. Auf eine Reaktion\pwindex{Herzl, Theodor 2.\,5.\,1860 Budapest – 3.\,7.\,1904 Edlach@\textsc{Herzl, Theodor} (2.\,5.\,1860 Budapest – 3.\,7.\,1904 Edlach), \emph{Schriftsteller, Journalist}!Unser Käthchen« [Brief an Hermann Bahr]@\strich\emph{»Unser Käthchen« [Brief an Hermann Bahr]}|pwkv}{ }Theodor Herzls\pwindex{Herzl, Theodor 2.\,5.\,1860 Budapest – 3.\,7.\,1904 Edlach@\textsc{Herzl, Theodor} (2.\,5.\,1860 Budapest – 3.\,7.\,1904 Edlach), \emph{Schriftsteller, Journalist}|pwk} in der \emph{Neuen Freien Presse}\pwindex{Neue Freie Presse@\emph{Neue Freie Presse}|pwk} vom 12. 2. 1899
                     (Nr. 12.384, S. 8) wurde hingewiesen.}}}\label{K_L00890-1} oder nicht? Nichts iſt
               gleichgiltiger für das innere Schickſal der Première; nichts gleichgiltiger für das
               fernere Schickſal des betreffenden Stückes. Jeder Autor möge es daher in jedem Falle
               halten, wie es ihm beliebt. In Geſchmacks- und Stimmungsfragen gibt es keine
               Solidarität.\pend
           
\pstart
           Herzlichen Gruß. \label{K_L00890-2v}\edtext{Dein}{\lemma{\textnormal{\emph{Dein}}}\Cendnote{\textnormal{Drei weitere Antworten geben
                  Duzbrüderschaft mit Bahr zu erkennen: Bukovics\pwindex{Bukovics, Emerich von 28.\,2.\,1844 Wien – 4.\,7.\,1905 ebd.@\textsc{Bukovics, Emerich von} (28.\,2.\,1844 Wien – 4.\,7.\,1905 ebd.), \emph{Journalist, Theaterleiter}|pwk}, Ebermann\pwindex{Ebermann, Leo 16.\,7.\,1863 Draganovka – 9.\,10.\,1914 Wien@\textsc{Ebermann, Leo} (16.\,7.\,1863 Draganovka – 9.\,10.\,1914 Wien), \emph{Schriftsteller, Journalist, Rechtswissenschaftler}|pwk} und Karlweis\pwindex{Karlweis, Carl 23.\,11.\,1850 Wien – 27.\,10.\,1901 ebd.@\textsc{Karlweis, Carl} (23.\,11.\,1850 Wien – 27.\,10.\,1901 ebd.), \emph{Schriftsteller}|pwk}.}}}\label{K_L00890-2} ergebener{\\[\baselineskip]}\spacefill\mbox{Arthur Schnitzler.}\pend
           \leftskip=0em{}\selectlanguage{ngerman}\endnumbering\briefempfaengerindex{Bahr, Hermann@\textsc{Bahr, Hermann}!zzzSchnitzler, Arthur@\emph{von Arthur Schnitzler}!1899-02-151@{15. 2. 1899}|)be}\mylabel{L00890h}  \newcommand{\dateiname}{L00890}\newcommand{\titel}{Arthur Schnitzler an Hermann Bahr, Antwort auf eine Umfrage, 15. 2. 1899}\newcommand{\editorInnen}{Kurt Ifkovits und Martin Anton Müller}%% latex-leseansicht-abspann.tex
%% Abspann für die Leseansicht.
%% Der Schalter \ifkorrekturansicht ist bereits durch den Vorspann gesetzt.

%% latex-abspann.tex
%% Gemeinsamer Abspann für Korrekturansicht und Leseansicht.
%% Setzt den Schalter \ifkorrekturansicht voraus (gesetzt in den
%% einbindenden Dateien latex-korrekturansicht-abspann.tex bzw.
%% latex-leseansicht-abspann.tex).
%% ---------------------------------------------------------------

\normalsize

% Das esempio-Environment wird nur in der Leseansicht benötigt
\ifkorrekturansicht\else
\newenvironment{esempio}[3]%
{
    \vspace{1.5ex}
    \rlap{\underline{#1}}
    \par
    \setlength{\parindent}{0cm}
    \nopagebreak
    \leftskip=#2cm
    \rightskip=#3cm
}
{
    \par
}
\fi

\doendnotes{C}
\bigskip
\vfill

\clearpage

\footnotesize

\ifkorrekturansicht
  \lohead{\textsc{register}}
\fi

% theindex-Environment neu definieren ohne reledmac
\makeatletter
\renewenvironment{theindex}{%
  \ifkorrekturansicht
    \section*{\indexname}%
  \else
    \subsubsection*{Index der erwähnten Entitäten}%
  \fi
  \setlength{\parindent}{0pt}%
  \setlength{\parskip}{0pt plus 0.3pt}%
  \let\item\@idxitem
}{%
  \ifkorrekturansicht\clearpage\fi
}
\makeatother

\IfFileExists{\jobname-pw.ind}{\input{\jobname-pw.ind}}{}

% Quellenangabe nur in der Leseansicht
\ifkorrekturansicht\else
% Fallback-Definitionen, falls die .tex-Datei \titel etc. nicht gesetzt hat
\providecommand{\titel}{}
\providecommand{\editorInnen}{}
\providecommand{\dateiname}{\jobname}

\vspace{3cm}

\vfill

\footnotesize
\textsc{Quelle}: \titel. Herausgegeben von {\editorInnen}. In: \emph{Arthur Schnitzler: Briefwechsel mit Autorinnen und Autoren}.
 Digitale Edition, https://schnitzler-briefe.acdh.oeaw.ac.at/{\dateiname}.html (Stand \today)
\fi

\end{document}


