%% latex-korrekturansicht-vorspann.tex
%% Vorspann für die Korrekturansicht.
%% Lädt die gemeinsame Datei latex-vorspann.tex mit gesetztem Schalter.

\newif\ifkorrekturansicht
\korrekturansichttrue

\input{../tex-inputs/latex-vorspann}


\section[ Paul Goldmann an Arthur Schnitzler, 14. 7. {[}1902{]}]{L03213 Paul Goldmann an Arthur Schnitzler, 14. 7. {[}1902{]}}
\nopagebreak\mylabel{L03213v}
\rehead{ }\normalsize\beginnumbering\briefempfaengerindex{Schnitzler, Arthur@\textsc{Schnitzler, Arthur}!zzzGoldmann, Paul@\emph{von Paul Goldmann}!1902-07-141@{14. 7. {[}1902{]}}|(be}
\toendnotes[C]{\smallbreak\pagebreak[2]}\Standort{DLA, A:Schnitzler, HS.NZ85.1.3172.}
\physDesc{Brief, 1 Blatt, 2 Seiten, 507 Zeichen
\newline{}Handschrift: blaue Tinte, deutsche Kurrent
\newline{}Schnitzler: mit Bleistift das Jahr »902« vermerkt }\toendnotes[C]{\smallbreak}
\pstart
           \raggedleft{}{\pb}\textcolor{gray}{\textbf{DESSAUERSTRASSE 19}}\oindex{Dessauer Strasse@\textbf{Dessauer Straße}, \emph{Straße (K.STR)}|pw}\pend
           
\pstart
           Berlin\oindex{Berlin@\textbf{Berlin}, \emph{P.PPLC}|pw}, 14. Juli.\pend
           
\pstart\center{}Mein lieber Freund,\pend\vspace{0.5em}
\pstart
           Höre ich bald von Dir? Wie war die \label{K_L03213-1v}\edtext{Reiſe}{\lemma{\textnormal{\emph{Reiſe}}}\Cendnote{\textnormal{Schnitzler reiste zwischen 27. 6. 1902 und 7. 7. 1902 nach Salzburg\oindex{Salzburg@\textbf{Salzburg}, \emph{A.ADM2}|pwk}, Nordtirol\oindex{Tirol@\textbf{Tirol}, \emph{A.ADM1}|pwk} und Südtirol\oindex{Suedtirol@\textbf{Südtirol}, \emph{A.ADM2}|pwk}.}}}\label{K_L03213-1}? Biſt
               Du glücklich zurück? Was macht \textsc{Olga\pwindex{Schnitzler, Olga 17.01.1882 – 13.01.1970@\textsc{Schnitzler, Olga} (17.01.1882 – 13.01.1970), \emph{Schauspieler/Schauspielerin, Sänger/Sängerin}|pw}}?\pend
           
\pstart
           Wirſt Du die »\textsc{Beatrice\pwindex{Schleier der Beatrice. Schauspiel in fuenf Akten@\emph{Der Schleier der Beatrice. Schauspiel in fünf Akten}|pw}}« dem \label{K_L03213-2v}\edtext{\textsc{Dr\textcolor{gray}{-}{ }Löwenfeld\pwindex{Loewenfeld, Raphael 11.02.1854 – 28.12.1910@\textsc{Löwenfeld, Raphael} (11.02.1854 – 28.12.1910), \emph{Theaterleiter/Theaterleiterin}|pw}}}{\lemma{\textnormal{\emph{Dr- Löwenfeld}}}\Cendnote{\textnormal{Schnitzler verhandelte sowohl mit Raphael Löwenfeld\pwindex{Loewenfeld, Raphael 11.02.1854 – 28.12.1910@\textsc{Löwenfeld, Raphael} (11.02.1854 – 28.12.1910), \emph{Theaterleiter/Theaterleiterin}|pwk}, dem Leiter des \emph{Schiller-Theaters}\orgindex{Schiller-Theater@Schiller-Theater|pwk}, als auch mit Otto Brahm\pwindex{Brahm, Otto 05.02.1856 – 28.11.1912@\textsc{Brahm, Otto} (05.02.1856 – 28.11.1912), \emph{Theaterleiter/Theaterleiterin, Regisseur/Regisseurin}|pwk}, dem Leiter des \emph{Deutschen Theaters}\orgindex{Deutsches Theater Berlin@Deutsches Theater Berlin|pwk}, wegen einer Aufführung von \emph{Der Schleier der Beatrice}\pwindex{Schleier der Beatrice. Schauspiel in fuenf Akten@\emph{Der Schleier der Beatrice. Schauspiel in fünf Akten}|pwk} (vgl. A. S.: \emph{Tagebuch}, 17. 7. 1902). Die Berlin\oindex{Berlin@\textbf{Berlin}, \emph{P.PPLC}|pwk}er Premiere fand am 7. 3. 1903 am Deutschen Theater\oindex{Deutsches Theater Berlin@\textbf{Deutsches Theater Berlin}, \emph{Theater (K.THE)}|pwk} statt. Siehe auch Arthur Schnitzler an Hugo von Hofmannsthal, 7. 10. 1902.}}}\label{K_L03213-2} geben?\pend
           
\pstart
           Dieſer Tage las ich \label{K_L03213-3v}\edtext{»\textsc{\begin{otherlanguage}{french}Fort comme la mort\pwindex{Fort comme la mort@\emph{Fort comme la mort}|pw}\end{otherlanguage}}«}{\lemma{\textnormal{\emph{»Fort comme la mort«}}}\Cendnote{\textnormal{Guy de Maupassant\pwindex{Maupassant, Guy de 05.08.1850 – 07.07.1893@\textsc{Maupassant, Guy de} (05.08.1850 – 07.07.1893), \emph{Schriftsteller/Schriftstellerin}|pwk}: \emph{Fort comme la mort}\pwindex{Fort comme la mort@\emph{Fort comme la mort}|pwk}. Paris\oindex{Paris@\textbf{Paris}, \emph{P.PPLC}|pwk}: \emph{Paul Ollendorf}\orgindex{Paul Ollendorff@Paul Ollendorff|pwk}{ }1889. Siehe A. S.: \emph{Lektüren}, Frankreich.}}}\label{K_L03213-3}, das mich tief ergriffen hat. Nie iſt das Altwerden ſo geſchildert
               worden. {\pb}Es iſt übrigens Dein Stoff: der alternde
               Junggeſelle, der das junge Mädchen \strikeout{liebt} liebt. Wenn
               Du das Buch\pwindex{Fort comme la mort@\emph{Fort comme la mort}|pwv} nicht kennſt, mußt
               Du es ſchleunigſt leſen.\pend
           
\pstart
           Ich danke Dir für Deine lieben Karten \strikeout{\textcolor{gray}{aus}} von unterwegs.\pend
           
\pstart
           Viele treue Grüße!{\\[\baselineskip]}Dein{\\[\baselineskip]}\spacefill\mbox{Paul Goldm}\pend
           \leftskip=0em{}\selectlanguage{ngerman}\endnumbering\briefempfaengerindex{Schnitzler, Arthur@\textsc{Schnitzler, Arthur}!zzzGoldmann, Paul@\emph{von Paul Goldmann}!1902-07-141@{14. 7. {[}1902{]}}|)be}\mylabel{L03213h}  \normalsize

\doendnotes{C}
\bigskip
\vfill

\clearpage

\footnotesize

\lohead{\textsc{register}}

% Definiere theindex-Environment komplett neu ohne reledmac
\makeatletter
\renewenvironment{theindex}{%
  \section*{\indexname}%
  \setlength{\parindent}{0pt}%
  \setlength{\parskip}{0pt plus 0.3pt}%
  \let\item\@idxitem
}{%
  \clearpage
}
\makeatother

\IfFileExists{\jobname-pw.ind}{\input{\jobname-pw.ind}}{}

\end{document}

      