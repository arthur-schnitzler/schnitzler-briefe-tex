%% latex-leseansicht-vorspann.tex
%% Vorspann für die Leseansicht.
%% Lädt die gemeinsame Datei latex-vorspann.tex mit nicht gesetztem Schalter.

\newif\ifkorrekturansicht
\korrekturansichtfalse

\input{../tex-inputs/latex-vorspann}


\section[ Paul Goldmann an Arthur Schnitzler, 14. 7. {[}1902{]}]{L03213 Paul Goldmann an Arthur Schnitzler,  14. 7. [1902]}
\nopagebreak\mylabel{L03213v}
\rehead{ }\normalsize\beginnumbering\briefempfaengerindex{Schnitzler, Arthur@\textsc{Schnitzler, Arthur}!zzzGoldmann, Paul@\emph{von Paul Goldmann}!1902-07-141@{14. 7. [1902]}|(be}
\toendnotes[C]{\smallbreak\pagebreak[2]}
\correspDesc{Versand  durch Paul Goldmann am 14. 7. [1902] in Berlin
\newline{}Erhalt  durch Arthur Schnitzler im Zeitraum [15. 7. 1902
                  – 19. 7. 1902?] in Wien}\toendnotes[C]{\smallbreak}
\Standort{DLA, A:Schnitzler, HS.NZ85.1.3172.}
\physDesc{Brief, 1 Blatt, 2 Seiten, 507 Zeichen
\newline{}Handschrift: blaue Tinte, deutsche Kurrent
\newline{}Schnitzler: mit Bleistift das Jahr »902« vermerkt }\toendnotes[C]{\smallbreak}
\pstart
           \raggedleft{}{\pb}\textcolor{gray}{\textbf{DESSAUERSTRASSE 19}}\oindex{Dessauer Straße@\textbf{Dessauer Straße}, \emph{Straße}|pw}\pend
           
\pstart
           Berlin\oindex{Berlin@\textbf{Berlin}, \emph{Hauptstadt}|pw}, 14. Juli.\pend
           
\pstart\center{}Mein lieber Freund,\pend\vspace{0.5em}
\pstart
           Höre ich bald von Dir? Wie war die \label{K_L03213-1v}\edtext{Reiſe}{\lemma{\textnormal{\emph{Reise}}}\Cendnote{\textnormal{Schnitzler reiste zwischen 27. 6. 1902 und 7. 7. 1902 nach Salzburg\oindex{Salzburg@\textbf{Salzburg}, \emph{Verwaltungsgebiet}|pwk}, Nordtirol\oindex{Tirol@\textbf{Tirol}, \emph{Land}|pwk} und Südtirol\oindex{Südtirol@\textbf{Südtirol}, \emph{Verwaltungsgebiet}|pwk}.}}}\label{K_L03213-1}? Biſt
               Du glücklich zurück? Was macht \textsc{Olga\pwindex{Schnitzler, Olga 17.\,1.\,1882 Wien – 13.\,1.\,1970 Lugano@\textsc{Schnitzler, Olga} (17.\,1.\,1882 Wien – 13.\,1.\,1970 Lugano), \emph{Schauspielerin, Sängerin}|pw}}?\pend
           
\pstart
           Wirſt Du die »\textsc{Beatrice\pwindex{Schnitzler, Arthur 15.\,5.\,1862 Wien – 21.\,10.\,1931 ebd.@\textsc{Schnitzler, Arthur} (15.\,5.\,1862 Wien – 21.\,10.\,1931 ebd.), \emph{Schriftsteller, Mediziner}!Schleier der Beatrice. Schauspiel in fünf Akten@\strich\emph{Der Schleier der Beatrice. Schauspiel in fünf Akten}|pw}}« dem \label{K_L03213-2v}\edtext{\textsc{Dr\textcolor{gray}{-}{ }Löwenfeld\pwindex{Löwenfeld, Raphael 11.\,2.\,1854 Poznan – 28.\,12.\,1910 Berlin@\textsc{Löwenfeld, Raphael} (11.\,2.\,1854 Poznan – 28.\,12.\,1910 Berlin), \emph{Theaterleiter}|pw}}}{\lemma{\textnormal{\emph{Dr- Löwenfeld}}}\Cendnote{\textnormal{Schnitzler verhandelte sowohl mit Raphael Löwenfeld\pwindex{Löwenfeld, Raphael 11.\,2.\,1854 Poznan – 28.\,12.\,1910 Berlin@\textsc{Löwenfeld, Raphael} (11.\,2.\,1854 Poznan – 28.\,12.\,1910 Berlin), \emph{Theaterleiter}|pwk}, dem Leiter des \emph{Schiller-Theaters}\orgindex{Schiller-Theater@Schiller-Theater|pwk}, als auch mit Otto Brahm\pwindex{Brahm, Otto 5.\,2.\,1856 Hamburg – 28.\,11.\,1912 Berlin@\textsc{Brahm, Otto} (5.\,2.\,1856 Hamburg – 28.\,11.\,1912 Berlin), \emph{Theaterleiter, Regisseur}|pwk}, dem Leiter des \emph{Deutschen Theaters}\orgindex{Deutsches Theater Berlin@Deutsches Theater Berlin|pwk}, wegen einer Aufführung von \emph{Der Schleier der Beatrice}\pwindex{Schnitzler, Arthur 15.\,5.\,1862 Wien – 21.\,10.\,1931 ebd.@\textsc{Schnitzler, Arthur} (15.\,5.\,1862 Wien – 21.\,10.\,1931 ebd.), \emph{Schriftsteller, Mediziner}!Schleier der Beatrice. Schauspiel in fünf Akten@\strich\emph{Der Schleier der Beatrice. Schauspiel in fünf Akten}|pwk} (vgl. A. S.: \emph{Tagebuch}, 17. 7. 1902). Die Berlin\oindex{Berlin@\textbf{Berlin}, \emph{Hauptstadt}|pwk}er Premiere fand am 7. 3. 1903 am Deutschen Theater\oindex{Deutsches Theater Berlin@\textbf{Deutsches Theater Berlin}, \emph{Theater}|pwk} statt. Siehe auch XXXX Auszeichnungsfehler: Dokument L01238 nicht gefunden.}}}\label{K_L03213-2} geben?\pend
           
\pstart
           Dieſer Tage las ich \label{K_L03213-3v}\edtext{»\textsc{\begin{otherlanguage}{french}Fort comme la mort\pwindex{Maupassant, Guy de 5.\,8.\,1850 Tourville-sur-Arques – 7.\,7.\,1893 Paris@\textsc{Maupassant, Guy de} (5.\,8.\,1850 Tourville-sur-Arques – 7.\,7.\,1893 Paris), \emph{Schriftsteller}!Fort comme la mort@\strich\emph{Fort comme la mort}|pw}\end{otherlanguage}}«}{\lemma{\textnormal{\emph{»Fort comme la mort«}}}\Cendnote{\textnormal{Guy de Maupassant\pwindex{Maupassant, Guy de 5.\,8.\,1850 Tourville-sur-Arques – 7.\,7.\,1893 Paris@\textsc{Maupassant, Guy de} (5.\,8.\,1850 Tourville-sur-Arques – 7.\,7.\,1893 Paris), \emph{Schriftsteller}|pwk}: \emph{Fort comme la mort}\pwindex{Maupassant, Guy de 5.\,8.\,1850 Tourville-sur-Arques – 7.\,7.\,1893 Paris@\textsc{Maupassant, Guy de} (5.\,8.\,1850 Tourville-sur-Arques – 7.\,7.\,1893 Paris), \emph{Schriftsteller}!Fort comme la mort@\strich\emph{Fort comme la mort}|pwk}. Paris\oindex{Paris@\textbf{Paris}, \emph{Hauptstadt}|pwk}: \emph{Paul Ollendorf}\orgindex{Paul Ollendorff@Paul Ollendorff|pwk}{ }1889. Siehe A. S.: \emph{Lektüren}, Frankreich.}}}\label{K_L03213-3}, das mich tief ergriffen hat. Nie iſt das Altwerden{ }ſo geſchildert
               worden. {\pb}Es iſt übrigens Dein Stoff: der alternde
               Junggeſelle, der das junge Mädchen \strikeout{liebt} liebt. Wenn
               Du das Buch\pwindex{Maupassant, Guy de 5.\,8.\,1850 Tourville-sur-Arques – 7.\,7.\,1893 Paris@\textsc{Maupassant, Guy de} (5.\,8.\,1850 Tourville-sur-Arques – 7.\,7.\,1893 Paris), \emph{Schriftsteller}!Fort comme la mort@\strich\emph{Fort comme la mort}|pwv} nicht kennſt, mußt
               Du es{ }ſchleunigſt leſen.\pend
           
\pstart
           Ich danke Dir für Deine lieben Karten \strikeout{\textcolor{gray}{aus}} von unterwegs.\pend
           
\pstart
           Viele treue Grüße!{\\[\baselineskip]}Dein{\\[\baselineskip]}\spacefill\mbox{Paul Goldm}\pend
           \leftskip=0em{}\selectlanguage{ngerman}\endnumbering\briefempfaengerindex{Schnitzler, Arthur@\textsc{Schnitzler, Arthur}!zzzGoldmann, Paul@\emph{von Paul Goldmann}!1902-07-141@{14. 7. [1902]}|)be}\mylabel{L03213h}  \newcommand{\dateiname}{L03213}\newcommand{\titel}{Paul Goldmann an Arthur Schnitzler, 14. 7. [1902]}\newcommand{\editorInnen}{Martin Anton Müller und Laura Untner}%% latex-leseansicht-abspann.tex
%% Abspann für die Leseansicht.
%% Der Schalter \ifkorrekturansicht ist bereits durch den Vorspann gesetzt.

%% latex-abspann.tex
%% Gemeinsamer Abspann für Korrekturansicht und Leseansicht.
%% Setzt den Schalter \ifkorrekturansicht voraus (gesetzt in den
%% einbindenden Dateien latex-korrekturansicht-abspann.tex bzw.
%% latex-leseansicht-abspann.tex).
%% ---------------------------------------------------------------

\normalsize

% Das esempio-Environment wird nur in der Leseansicht benötigt
\ifkorrekturansicht\else
\newenvironment{esempio}[3]%
{
    \vspace{1.5ex}
    \rlap{\underline{#1}}
    \par
    \setlength{\parindent}{0cm}
    \nopagebreak
    \leftskip=#2cm
    \rightskip=#3cm
}
{
    \par
}
\fi

\doendnotes{C}
\bigskip
\vfill

\clearpage

\footnotesize

\ifkorrekturansicht
  \lohead{\textsc{register}}
\fi

% theindex-Environment neu definieren ohne reledmac
\makeatletter
\renewenvironment{theindex}{%
  \ifkorrekturansicht
    \section*{\indexname}%
  \else
    \subsubsection*{Index der erwähnten Entitäten}%
  \fi
  \setlength{\parindent}{0pt}%
  \setlength{\parskip}{0pt plus 0.3pt}%
  \let\item\@idxitem
}{%
  \ifkorrekturansicht\clearpage\fi
}
\makeatother

\IfFileExists{\jobname-pw.ind}{\input{\jobname-pw.ind}}{}

% Quellenangabe nur in der Leseansicht
\ifkorrekturansicht\else
% Fallback-Definitionen, falls die .tex-Datei \titel etc. nicht gesetzt hat
\providecommand{\titel}{}
\providecommand{\editorInnen}{}
\providecommand{\dateiname}{\jobname}

\vspace{3cm}

\vfill

\footnotesize
\textsc{Quelle}: \titel. Herausgegeben von {\editorInnen}. In: \emph{Arthur Schnitzler: Briefwechsel mit Autorinnen und Autoren}.
 Digitale Edition, https://schnitzler-briefe.acdh.oeaw.ac.at/{\dateiname}.html (Stand \today)
\fi

\end{document}


