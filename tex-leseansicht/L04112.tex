%% latex-leseansicht-vorspann.tex
%% Vorspann für die Leseansicht.
%% Lädt die gemeinsame Datei latex-vorspann.tex mit nicht gesetztem Schalter.

\newif\ifkorrekturansicht
\korrekturansichtfalse

\input{../tex-inputs/latex-vorspann}


\section[Arthur Schnitzler an Gustav Schwarzkopf, 26. 4. 1897]{L04112 Arthur Schnitzler an Gustav Schwarzkopf, 26. 4. 1897}
\nopagebreak\mylabel{L04112v}
\rehead{ }\normalsize\beginnumbering\briefempfaengerindex{Schwarzkopf, Gustav@\textsc{Schwarzkopf, Gustav}!zzzSchnitzler, Arthur@\emph{von Arthur Schnitzler}!1897-04-264@{26. 4. 1897}|(be}
\toendnotes[C]{\smallbreak\pagebreak[2]}
\correspDesc{Versand  durch Arthur Schnitzler am 26. 4. 1897 in Paris
\newline{}Erhalt  durch Gustav Schwarzkopf am 28. 4. 1897 in Wien}\toendnotes[C]{\smallbreak}
\Standort{CUL, Schnitzler, B 96.}
\physDesc{Postkarte, 672 Zeichen
\newline{}Handschrift: schwarze Tinte, deutsche Kurrent
\newline{}Versand: 1) Stempel: »\nobreak{}\oindex{Lueg@\textbf{Lueg}, \emph{Teil eines besiedelten Ortes}|pwk}Paris R. Lafayette, 26 Avril 97, 8\textsuperscript{E}\nobreak{}«.   2) Stempel: »\nobreak{}\oindex{I., Innere Stadt@\textbf{I., Innere Stadt}, \emph{Verwaltungsgebiet}|pwk}Wien 1/1, 28. 4. 97, 9–10½V., Bestellt\nobreak{}«. }\toendnotes[C]{\smallbreak}\pstart{}{\pb}\textsc{Gustav
                     Schwarzkopf}\pend{}\pstart{}\textsc{Wien}\oindex{Wien@\textbf{Wien}, \emph{Verwaltungsgebiet}|pw}\pend{}\pstart{}\textsc{I. Tiefer
                     Graben 23}\oindex{Wien@\textbf{Wien}!I., Innere Stadt@\textbf{I., Innere Stadt}!Tiefer Graben 23@\textbf{Tiefer Graben 23}, \emph{Wohngebäude}|pw}.\pend{}{\bigskip}\vspace{1em}
\pstart
           \noindent{}{\pb}Lieber Guſtav, da ich mir die N. R.\pwindex{Neue Revue. Wiener Literatur-Zeitung@\emph{Neue Revue. Wiener Literatur-Zeitung}|pw}
               nachſchicken laſſe, weiſs ich was Sie über \textsc{Zacconi\pwindex{Zacconi, Ermete 14.\,9.\,1857 Montecchio Emilia – 14.\,10.\,1948 Viareggio@\textsc{Zacconi, Ermete} (14.\,9.\,1857 Montecchio Emilia – 14.\,10.\,1948 Viareggio), \emph{Regisseur, Schauspieler}|pw}}{ }denken\pwindex{Schwarzkopf, Gustav 7.\,11.\,1853 Wien – 13.\,11.\,1939 ebd.@\textsc{Schwarzkopf, Gustav} (7.\,11.\,1853 Wien – 13.\,11.\,1939 ebd.), \emph{Schriftsteller}!italienische Gastspielgesellschaft@\strich\emph{Die italienische Gastspielgesellschaft}|pwv} – aber ſonſt nichts von Ihnen.
               Dem helfen Sie gelegentlich durch ein Anzahl an Zeilen ab; nicht wahr?
               »Reiſeberichte« von mir erwarten Sie nicht, wie ich mir denke; we{\geminationn} man bedenkt: wozu ich zehn Briefbogen bräuchte, das
               läßt ſich in Wien\oindex{Wien@\textbf{Wien}, \emph{Verwaltungsgebiet}|pw} erzählen – zwiſchen dem
               Entſchluß aus dem Kaffeehaus nach Haus zu gehen und deſſen Ausführung. – Im
               ganzen bin ich über zwei \textcolor{gray}{Dinge} recht froh: daſs ich \uline{nicht} in Wien\oindex{Wien@\textbf{Wien}, \emph{Verwaltungsgebiet}|pw} bin –
               und daſs ich in Paris\oindex{Paris@\textbf{Paris}, \emph{Hauptstadt}|pw} bin. – Ach {\dots} Ahnen
               Sie, wie geldgierig dieſer Seufzer iſt? –\pend
           
\pstart
           Herzliche Grüße!{\\[\baselineskip]} Ihr \spacefill\mbox{Arthur Sch.}\pend
           \leftskip=0em{}
\pstart
           \noindent{}5 rue \introOben{}de\introOben{} Maubeuge\oindex{5, rue de Maubeuge@\textbf{5, rue de Maubeuge}, \emph{Wohngebäude}|pw}\pend
           
\pstart
           \textsc{Paris\oindex{Paris@\textbf{Paris}, \emph{Hauptstadt}|pw}}{ }26/4 97\pend
           \selectlanguage{ngerman}\endnumbering\briefempfaengerindex{Schwarzkopf, Gustav@\textsc{Schwarzkopf, Gustav}!zzzSchnitzler, Arthur@\emph{von Arthur Schnitzler}!1897-04-264@{26. 4. 1897}|)be}\mylabel{L04112h}
\begin{anhang}
\end{anhang}\newcommand{\dateiname}{L04112}\newcommand{\titel}{Arthur Schnitzler an Gustav Schwarzkopf, 26. 4. 1897}\newcommand{\editorInnen}{Herausgegeben von Jahnke, SelmaMüller, Martin Anton}%% latex-leseansicht-abspann.tex
%% Abspann für die Leseansicht.
%% Der Schalter \ifkorrekturansicht ist bereits durch den Vorspann gesetzt.

%% latex-abspann.tex
%% Gemeinsamer Abspann für Korrekturansicht und Leseansicht.
%% Setzt den Schalter \ifkorrekturansicht voraus (gesetzt in den
%% einbindenden Dateien latex-korrekturansicht-abspann.tex bzw.
%% latex-leseansicht-abspann.tex).
%% ---------------------------------------------------------------

\normalsize

% Das esempio-Environment wird nur in der Leseansicht benötigt
\ifkorrekturansicht\else
\newenvironment{esempio}[3]%
{
    \vspace{1.5ex}
    \rlap{\underline{#1}}
    \par
    \setlength{\parindent}{0cm}
    \nopagebreak
    \leftskip=#2cm
    \rightskip=#3cm
}
{
    \par
}
\fi

\doendnotes{C}
\bigskip
\vfill

\clearpage

\footnotesize

\ifkorrekturansicht
  \lohead{\textsc{register}}
\fi

% theindex-Environment neu definieren ohne reledmac
\makeatletter
\renewenvironment{theindex}{%
  \ifkorrekturansicht
    \section*{\indexname}%
  \else
    \subsubsection*{Index der erwähnten Entitäten}%
  \fi
  \setlength{\parindent}{0pt}%
  \setlength{\parskip}{0pt plus 0.3pt}%
  \let\item\@idxitem
}{%
  \ifkorrekturansicht\clearpage\fi
}
\makeatother

\IfFileExists{\jobname-pw.ind}{\input{\jobname-pw.ind}}{}

% Quellenangabe nur in der Leseansicht
\ifkorrekturansicht\else
% Fallback-Definitionen, falls die .tex-Datei \titel etc. nicht gesetzt hat
\providecommand{\titel}{}
\providecommand{\editorInnen}{}
\providecommand{\dateiname}{\jobname}

\vspace{3cm}

\vfill

\footnotesize
\textsc{Quelle}: \titel. Herausgegeben von {\editorInnen}. In: \emph{Arthur Schnitzler: Briefwechsel mit Autorinnen und Autoren}.
 Digitale Edition, https://schnitzler-briefe.acdh.oeaw.ac.at/{\dateiname}.html (Stand \today)
\fi

\end{document}


