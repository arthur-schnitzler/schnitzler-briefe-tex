%% latex-leseansicht-vorspann.tex
%% Vorspann für die Leseansicht.
%% Lädt die gemeinsame Datei latex-vorspann.tex mit nicht gesetztem Schalter.

\newif\ifkorrekturansicht
\korrekturansichtfalse

\input{../tex-inputs/latex-vorspann}


         
         \newcommand{\erwaehntePersonen}{Personen: }
         \newcommand{\erwaehnteInstitutionen}{}
         \newcommand{\erwaehnteOrte}{Orte: Wien, Zdislavice}
         \newcommand{\erwaehnteWerke}{
               \section[Marie von Ebner-Eschenbach an Arthur Schnitzler, 13. 9. 1910]{ Marie von Ebner-Eschenbach an Arthur Schnitzler, 13. 9. 1910}\nopagebreak\mylabel{v}\rehead{ }\begin{ledgroupsized}[t]{13cm}\normalsize\beginnumbering \toendnotes[C]{\smallbreak\pagebreak[2]} \Standort{DLA, A:Schnitzler, HS.NZ85.1.2822.}
\physDesc{Brief, 1 Blatt, 1 Seite
\newline{}Druck\newline{}Zusatz: Druck von »Theyer {\kaufmannsund}
                                    Hardtmudth« }\toendnotes[C]{\smallbreak}\stanza{}{\pb}\label{K_L02580-1v}\edtext{Sehr alt bin ich}{\lemma{\textnormal{\emph{Sehr alt bin ich}}}\Cendnote{\textnormal{Sie feierte am 13. 9. 1910 ihren 80. Geburtstag.}}}\label{K_L02580-1h}, Ihr Freunde und
                  Verwandten,\newverse{}und nicht imſtand, geliebte Gratulanten,\newverse{}zu danken ſo für Eure Huld und Güte,\newverse{}wie mich verlangt gar innig im Gemüte.\newverse{}Doch habt Geduld; vielleicht erſcheint der Tag,\newverse{}an dem zu Kraft ich wieder kommen mag,\newverse{}und was ich jetzt muß ſtill im Herzen tragen,\newverse{}aufjubelnd darf mit heller Stimme ſagen.\newverse{}Laßt nur die Zeit, die liebe Zeit verfließen,\newverse{}ein neu Beginnen dankbar mich genießen;\newverse{}geraten erſt in Zug die Zehn mal acht,\newverse{}dann fühl’ ich wieder mich ganz jung gemacht.\newverse{}Dann führt vielleicht zum Siege noch mein Ringen\newverse{}und ſpendet, was ich heut’ entbehren muß,\newverse{}die Fähigkeit, Euch würdig darzubringen\newverse{}aus voller Seele meinen Dankesgruß.\stanzaend{}\pstart \spacefill\mbox{\so{Marie von Ebner-Eſchenbach.}}\pend{}\pstart
           Zdißlawitz\oindex{Zdislavice@\textbf{Zdislavice}|pw}, 13. September
                  1910.\pend
           
         
         \endnumbering\mylabel{h}\end{ledgroupsized}  \newcommand{\dateiname}{L02580}\newcommand{\titel}{Marie von Ebner-Eschenbach an Arthur Schnitzler, 13. 9. 1910}\newcommand{\editorInnen}{Martin Anton Müller und Laura Untner}%% latex-leseansicht-abspann.tex
%% Abspann für die Leseansicht.
%% Der Schalter \ifkorrekturansicht ist bereits durch den Vorspann gesetzt.

%% latex-abspann.tex
%% Gemeinsamer Abspann für Korrekturansicht und Leseansicht.
%% Setzt den Schalter \ifkorrekturansicht voraus (gesetzt in den
%% einbindenden Dateien latex-korrekturansicht-abspann.tex bzw.
%% latex-leseansicht-abspann.tex).
%% ---------------------------------------------------------------

\normalsize

% Das esempio-Environment wird nur in der Leseansicht benötigt
\ifkorrekturansicht\else
\newenvironment{esempio}[3]%
{
    \vspace{1.5ex}
    \rlap{\underline{#1}}
    \par
    \setlength{\parindent}{0cm}
    \nopagebreak
    \leftskip=#2cm
    \rightskip=#3cm
}
{
    \par
}
\fi

\doendnotes{C}
\bigskip
\vfill

\clearpage

\footnotesize

\ifkorrekturansicht
  \lohead{\textsc{register}}
\fi

% theindex-Environment neu definieren ohne reledmac
\makeatletter
\renewenvironment{theindex}{%
  \ifkorrekturansicht
    \section*{\indexname}%
  \else
    \subsubsection*{Index der erwähnten Entitäten}%
  \fi
  \setlength{\parindent}{0pt}%
  \setlength{\parskip}{0pt plus 0.3pt}%
  \let\item\@idxitem
}{%
  \ifkorrekturansicht\clearpage\fi
}
\makeatother

\IfFileExists{\jobname-pw.ind}{\input{\jobname-pw.ind}}{}

% Quellenangabe nur in der Leseansicht
\ifkorrekturansicht\else
% Fallback-Definitionen, falls die .tex-Datei \titel etc. nicht gesetzt hat
\providecommand{\titel}{}
\providecommand{\editorInnen}{}
\providecommand{\dateiname}{\jobname}

\vspace{3cm}

\vfill

\footnotesize
\textsc{Quelle}: \titel. Herausgegeben von {\editorInnen}. In: \emph{Arthur Schnitzler: Briefwechsel mit Autorinnen und Autoren}.
 Digitale Edition, https://schnitzler-briefe.acdh.oeaw.ac.at/{\dateiname}.html (Stand \today)
\fi

\end{document}


      