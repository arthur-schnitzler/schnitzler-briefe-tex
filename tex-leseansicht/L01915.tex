%% latex-leseansicht-vorspann.tex
%% Vorspann für die Leseansicht.
%% Lädt die gemeinsame Datei latex-vorspann.tex mit nicht gesetztem Schalter.

\newif\ifkorrekturansicht
\korrekturansichtfalse

\input{../tex-inputs/latex-vorspann}


\section[Hugo von Hofmannsthal an Arthur Schnitzler, 1. 3. 1910]{L01915 Hugo von Hofmannsthal an Arthur Schnitzler, 1. 3. 1910}
\nopagebreak\mylabel{L01915v}
\rehead{ }\normalsize\beginnumbering\briefempfaengerindex{Schnitzler, Arthur@\textsc{Schnitzler, Arthur}!zzzHofmannsthal, Hugo von@\emph{von Hugo von Hofmannsthal}!1910-03-011@{1. 3. 1910}|(be}
\toendnotes[C]{\smallbreak\pagebreak[2]}
\correspDesc{Versand  durch Hugo von Hofmannsthal am 1. 3. 1910 in Weimar
\newline{}Erhalt  durch Arthur Schnitzler im Zeitraum [2. 3. 1910
                  – 6. 3. 1910?] in Wien}\toendnotes[C]{\smallbreak}
\Standort{CUL, Schnitzler, B 43.}
\physDesc{Bildpostkarte, 209 Zeichen
\newline{}Handschrift: schwarze Tinte, deutsche Kurrent
\newline{}Versand: Stempel: »\nobreak{}\oindex{Weimar@\textbf{Weimar}, \emph{Verwaltungsgebiet}|pwk}Weimar\nobreak{}«.  
\newline{}Schnitzler: mit Bleistift die Jahreszahl ergänzt: »910« 
\newline{}Ordnung: 1) mit Bleistift von unbekannter Hand nummeriert: »\strikeout{317}«  2) mit Bleistift von unbekannter Hand nummeriert: »314«}
\buchAbdrucke{\weitereDrucke{Hugo von Hofmannsthal, Arthur Schnitzler: \emph{Briefwechsel}. Herausgegeben von Therese Nickl und Heinrich Schnitzler. Frankfurt am Main: \emph{S. Fischer} 1964, S. 248.} }\toendnotes[C]{\smallbreak}\pstart{}\textsc{{\pb}Herrn D\textsuperscript{r} Arthur Schnitzler}\pend{}\pstart{}\textsc{Wien\oindex{Wien@\textbf{Wien}, \emph{Verwaltungsgebiet}|pw}}\pend{}\pstart{}\textsc{XVIII Spöttelgasse 7\oindex{Wien@\textbf{Wien}!XVIII., Währing@\textbf{XVIII., Währing}!Edmund-Weiß-Gasse 7@\textbf{Edmund-Weiß-Gasse 7}, \emph{Wohngebäude}|pw}}\pend{}{\bigskip}
\pstart
           \noindent{}\centering{}{\pb}\textcolor{gray}{\textbf{Anna Amalia\pwindex{Sachsen-Weimar und Eisenach, Anna Amalia von 24.\,10.\,1739 Wolfenbüttel – 10.\,4.\,1807 Weimar@\textsc{Sachsen-Weimar und Eisenach, Anna Amalia von} (24.\,10.\,1739 Wolfenbüttel – 10.\,4.\,1807 Weimar), \emph{Herzogin}|pw} und ihre Begleiter in der Villa d’Este\oindex{Villa d’Este@\textbf{Villa d’Este}, \emph{Gebäude}|pw}. Herder\pwindex{Herder, Johann Gottfried von 25.\,8.\,1744 Morąg – 18.\,12.\,1803 Weimar@\textsc{Herder, Johann Gottfried von} (25.\,8.\,1744 Morąg – 18.\,12.\,1803 Weimar), \emph{Schriftsteller, Philosoph, Theologe}|pw}, Göchhausen\pwindex{von Göchhausen, Luise Ernestine Christiane Juliane 13.\,2.\,1752 – 7.\,9.\,1807@\textsc{von Göchhausen, Luise Ernestine Christiane Juliane} (13.\,2.\,1752 – 7.\,9.\,1807), \emph{Hofdame}|pw},
                     Angelika Kauffmann\pwindex{Kauffmann, Angelika 30.\,10.\,1741 – 5.\,11.\,1807@\textsc{Kauffmann, Angelika} (30.\,10.\,1741 – 5.\,11.\,1807), \emph{Malerin}|pw}. Aquarell\pwindex{Schütz, Johann Georg 16.\,5.\,1755 – 11.\,5.\,1813@\textsc{Schütz, Johann Georg} (16.\,5.\,1755 – 11.\,5.\,1813), \emph{Maler}!Anna Amalia und ihre Begleiter im Garten der Villa d’Este zu Tivoli@\strich\emph{Anna Amalia und ihre Begleiter im Garten der Villa d’Este zu Tivoli}|pwv} v. Schütz\pwindex{Schütz, Johann Georg 16.\,5.\,1755 – 11.\,5.\,1813@\textsc{Schütz, Johann Georg} (16.\,5.\,1755 – 11.\,5.\,1813), \emph{Maler}|pw}. (Schloss Tiefurt\oindex{Schloss Tiefurt@\textbf{Schloss Tiefurt}|pw}
                  bei Weimar\oindex{Weimar@\textbf{Weimar}, \emph{Verwaltungsgebiet}|pw}.)}}\pend
           \vspace{1em}
\pstart
           \raggedleft{}{\pb}Weimar\oindex{Weimar@\textbf{Weimar}, \emph{Verwaltungsgebiet}|pw}{ }1. III.\pend
           \vspace{0.5em}
\pstart
           Danke Ihnen, lieber Arthur,{ }ſehr für die lieben Zeilen (und wünſche
               mir{ }ſehr mündlich weiteres). Wir{ }ſind Montag zurück und freuen uns auf
               Euch.\pend
           \pstart \spacefill\mbox{Hugo.}\pend{}\selectlanguage{ngerman}\endnumbering\briefempfaengerindex{Schnitzler, Arthur@\textsc{Schnitzler, Arthur}!zzzHofmannsthal, Hugo von@\emph{von Hugo von Hofmannsthal}!1910-03-011@{1. 3. 1910}|)be}\mylabel{L01915h}  \newcommand{\dateiname}{L01915}\newcommand{\titel}{Hugo von Hofmannsthal an Arthur Schnitzler, 1. 3. 1910}\newcommand{\editorInnen}{Martin Anton Müller und Gerd-Hermann Susen}%% latex-leseansicht-abspann.tex
%% Abspann für die Leseansicht.
%% Der Schalter \ifkorrekturansicht ist bereits durch den Vorspann gesetzt.

%% latex-abspann.tex
%% Gemeinsamer Abspann für Korrekturansicht und Leseansicht.
%% Setzt den Schalter \ifkorrekturansicht voraus (gesetzt in den
%% einbindenden Dateien latex-korrekturansicht-abspann.tex bzw.
%% latex-leseansicht-abspann.tex).
%% ---------------------------------------------------------------

\normalsize

% Das esempio-Environment wird nur in der Leseansicht benötigt
\ifkorrekturansicht\else
\newenvironment{esempio}[3]%
{
    \vspace{1.5ex}
    \rlap{\underline{#1}}
    \par
    \setlength{\parindent}{0cm}
    \nopagebreak
    \leftskip=#2cm
    \rightskip=#3cm
}
{
    \par
}
\fi

\doendnotes{C}
\bigskip
\vfill

\clearpage

\footnotesize

\ifkorrekturansicht
  \lohead{\textsc{register}}
\fi

% theindex-Environment neu definieren ohne reledmac
\makeatletter
\renewenvironment{theindex}{%
  \ifkorrekturansicht
    \section*{\indexname}%
  \else
    \subsubsection*{Index der erwähnten Entitäten}%
  \fi
  \setlength{\parindent}{0pt}%
  \setlength{\parskip}{0pt plus 0.3pt}%
  \let\item\@idxitem
}{%
  \ifkorrekturansicht\clearpage\fi
}
\makeatother

\IfFileExists{\jobname-pw.ind}{\input{\jobname-pw.ind}}{}

% Quellenangabe nur in der Leseansicht
\ifkorrekturansicht\else
% Fallback-Definitionen, falls die .tex-Datei \titel etc. nicht gesetzt hat
\providecommand{\titel}{}
\providecommand{\editorInnen}{}
\providecommand{\dateiname}{\jobname}

\vspace{3cm}

\vfill

\footnotesize
\textsc{Quelle}: \titel. Herausgegeben von {\editorInnen}. In: \emph{Arthur Schnitzler: Briefwechsel mit Autorinnen und Autoren}.
 Digitale Edition, https://schnitzler-briefe.acdh.oeaw.ac.at/{\dateiname}.html (Stand \today)
\fi

\end{document}


