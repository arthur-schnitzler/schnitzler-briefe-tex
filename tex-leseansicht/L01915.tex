%% latex-korrekturansicht-vorspann.tex
%% Vorspann für die Korrekturansicht.
%% Lädt die gemeinsame Datei latex-vorspann.tex mit gesetztem Schalter.

\newif\ifkorrekturansicht
\korrekturansichttrue

\input{../tex-inputs/latex-vorspann}


\section[Hugo von Hofmannsthal an Arthur Schnitzler, 1. 3. 1910]{L01915 Hugo von Hofmannsthal an Arthur Schnitzler, 1. 3. 1910}
\nopagebreak\mylabel{L01915v}
\rehead{ }\normalsize\beginnumbering\briefempfaengerindex{Schnitzler, Arthur@\textsc{Schnitzler, Arthur}!zzzHofmannsthal, Hugo von@\emph{von Hugo von Hofmannsthal}!1910-03-011@{1. 3. 1910}|(be}
\toendnotes[C]{\smallbreak\pagebreak[2]}\Standort{CUL, Schnitzler, B 43.}
\physDesc{Bildpostkarte, 209 Zeichen
\newline{}Handschrift: 1) schwarze Tinte, deutsche Kurrent\hspace{1em}2) schwarze Tinte, lateinische Kurrent (\noindent{}Adresse)\hspace{1em}
\newline{}Versand: Stempel: »\nobreak{}\oindex{Weimar@\textbf{Weimar}, \emph{A.ADM3}|pwk}Weimar\nobreak{}«.  
\newline{}Schnitzler: mit Bleistift die Jahreszahl ergänzt: »910« 
\newline{}Ordnung: 1) mit Bleistift von unbekannter Hand nummeriert: »\strikeout{317}«  2) mit Bleistift von unbekannter Hand nummeriert: »314«}
\buchAbdrucke{\weitereDrucke{Hugo von Hofmannsthal, Arthur Schnitzler: \emph{Briefwechsel}. Frankfurt am Main: \emph{S. Fischer} 1964, S. 248.} }\toendnotes[C]{\smallbreak}\pstart{}{\pb}Herrn D\textsuperscript{r} Arthur Schnitzler\pend{}\pstart{}Wien\oindex{Wien@\textbf{Wien}, \emph{A.ADM2}|pw}\pend{}\pstart{}XVIII Spöttelgasse 7\oindex{Edmund-Weiss-Gasse 7@\textbf{Edmund-Weiß-Gasse 7}, \emph{Wohngebäude (K.WHS)}|pw}\pend{}{\bigskip}
\pstart
           \noindent{}\centering{}{\pb}\textcolor{gray}{\textbf{Anna Amalia\pwindex{Sachsen-Weimar und Eisenach, Anna Amalia von 1739-10-24 – 1807-04-10@\textsc{Sachsen-Weimar und Eisenach, Anna Amalia von} (1739-10-24 – 1807-04-10), \emph{Herzog/Herzogin}|pw} und ihre Begleiter in der Villa d’Este\oindex{Villa DEste@\textbf{Villa d’Este}, \emph{Gebäude (K.GBD)}|pw}. Herder\pwindex{Herder, Johann Gottfried von 25.8.1744 – 18.12.1803@\textsc{Herder, Johann Gottfried von} (25.8.1744 – 18.12.1803), \emph{Schriftsteller/Schriftstellerin, Philosoph/Philosophin, Theologe/Theologin}|pw}, Göchhausen\pwindex{von Goechhausen, Luise Ernestine Christiane Juliane 1752-02-13 – 1807-09-07@\textsc{von Göchhausen, Luise Ernestine Christiane Juliane} (1752-02-13 – 1807-09-07), \emph{männliche Hofdame/Hofdame}|pw},
                     Angelika Kauffmann\pwindex{Kauffmann, Angelika 1741-10-30 – 1807-11-05@\textsc{Kauffmann, Angelika} (1741-10-30 – 1807-11-05), \emph{Maler/Malerin}|pw}. Aquarell\pwindex{Anna Amalia und ihre Begleiter im Garten der Villa DEste zu Tivoli@\emph{Anna Amalia und ihre Begleiter im Garten der Villa d’Este zu Tivoli}|pwv} v. Schütz\pwindex{Schuetz, Johann Georg 1755-05-16 – 1813-05-11@\textsc{Schütz, Johann Georg} (1755-05-16 – 1813-05-11), \emph{Maler/Malerin}|pw}. (Schloss Tiefurt\oindex{Schloss Tiefurt@\textbf{Schloss Tiefurt}, \emph{P.PPL}|pw}
                  bei Weimar\oindex{Weimar@\textbf{Weimar}, \emph{A.ADM3}|pw}.)}}\pend
           \vspace{1em}
\pstart
           \raggedleft{}{\pb}Weimar\oindex{Weimar@\textbf{Weimar}, \emph{A.ADM3}|pw}{ }1. III.\pend
           \vspace{0.5em}
\pstart
           Danke Ihnen, lieber Arthur, ſehr für die lieben Zeilen (und wünſche
               mir ſehr mündlich weiteres). Wir ſind Montag zurück und freuen uns auf
               Euch.\pend
           \pstart \spacefill\mbox{Hugo.}\pend{}\selectlanguage{ngerman}\endnumbering\briefempfaengerindex{Schnitzler, Arthur@\textsc{Schnitzler, Arthur}!zzzHofmannsthal, Hugo von@\emph{von Hugo von Hofmannsthal}!1910-03-011@{1. 3. 1910}|)be}\mylabel{L01915h}  \normalsize

\doendnotes{C}
\bigskip
\vfill

\clearpage

\footnotesize

\lohead{\textsc{register}}

% Definiere theindex-Environment komplett neu ohne reledmac
\makeatletter
\renewenvironment{theindex}{%
  \section*{\indexname}%
  \setlength{\parindent}{0pt}%
  \setlength{\parskip}{0pt plus 0.3pt}%
  \let\item\@idxitem
}{%
  \clearpage
}
\makeatother

\IfFileExists{\jobname-pw.ind}{\input{\jobname-pw.ind}}{}

\end{document}

      