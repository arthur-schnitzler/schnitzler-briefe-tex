%% latex-leseansicht-vorspann.tex
%% Vorspann für die Leseansicht.
%% Lädt die gemeinsame Datei latex-vorspann.tex mit nicht gesetztem Schalter.

\newif\ifkorrekturansicht
\korrekturansichtfalse

\input{../tex-inputs/latex-vorspann}


         
         \renewcommand{\erwaehntePersonen}{Personen: Johann Gottfried von Herder, Angelika Kauffmann, Anna Amalia von Sachsen-Weimar und Eisenach, Johann Georg Schütz, Luise Ernestine Christiane Juliane von Göchhausen}
         \renewcommand{\erwaehnteOrte}{Orte: Edmund-Weiß-Gasse, Schloss Tiefurt, Villa d’Este, Weimar, Wien}
         \renewcommand{\erwaehnteWerke}{Werke: Anna Amalia und ihre Begleiter im Garten der Villa d’Este zu Tivoli}
               \section[Hugo von Hofmannsthal an Arthur Schnitzler, 1. 3. 1910]{ Hugo von Hofmannsthal an Arthur Schnitzler, 1. 3. 1910}\nopagebreak\mylabel{v}\rehead{ }\begin{ledgroupsized}[t]{13cm}\normalsize\beginnumbering \toendnotes[C]{\smallbreak\pagebreak[2]} \Standort{CUL, Schnitzler, B 43.}
\physDesc{Bildpostkarte, 209 Zeichen
\newline{}Handschrift: 1) schwarze Tinte, deutsche Kurrent\hspace{1em}2) schwarze Tinte, lateinische Kurrent (\noindent{}Adresse)\hspace{1em}
\newline{}Versand: Stempel: »\nobreak{}\oindex{Weimar@\textbf{Weimar}|pwk}Weimar\nobreak{}«.  
\newline{}Schnitzler: mit Bleistift die Jahreszahl ergänzt: »910« 
\newline{}Ordnung: 1) mit Bleistift von unbekannter Hand nummeriert: »\strikeout{317}«  2) mit Bleistift von unbekannter Hand nummeriert:
                                    »314«}\buchAbdrucke{\weitereDrucke{Hugo von Hofmannsthal, Arthur Schnitzler: \emph{Briefwechsel}. Hg. Therese Nickl und Heinrich Schnitzler. Frankfurt am Main: \emph{S. Fischer} 1964, S. 248.} }\toendnotes[C]{\smallbreak}\pstart{}{\pb}Herrn D\textsuperscript{r} Arthur Schnitzler\pend{}\pstart{}Wien\oindex{Wien@\textbf{Wien}|pw}\pend{}\pstart{}XVIII Spöttelgasse 7\oindex{XXXX Ortsangabe fehlt|pw}\pend{}{\bigskip}\pstart
           \noindent{}\centering{}\textcolor{gray}{\textbf{{\pb}Anna Amalia\pwindex{Sachsen-Weimar und Eisenach, Anna Amalia von 1739-10-24 – 1807-04-10@\textsc{Sachsen-Weimar und Eisenach, Anna Amalia von} (1739-10-24 – 1807-04-10), \emph{Herzogin}|pw} und ihre Begleiter in der
                        Villa d’Este\oindex{Villa DEste@\textbf{Villa d’Este}|pw}. Herder\pwindex{Herder, Johann Gottfried von 25.8.1744 – 18.12.1803@\textsc{Herder, Johann Gottfried von} (25.8.1744 – 18.12.1803), \emph{Schriftsteller}|pw}, Göchhausen\pwindex{von Goechhausen, Luise Ernestine Christiane Juliane 1752-02-13 – 1807-09-07@\textsc{von Göchhausen, Luise Ernestine Christiane Juliane} (1752-02-13 – 1807-09-07), \emph{Hofdame}|pw}, Angelika Kauffmann\pwindex{Kauffmann, Angelika 1741-10-30 – 1807-11-05@\textsc{Kauffmann, Angelika} (1741-10-30 – 1807-11-05), \emph{Bildende Künstlerin >> Maler}|pw}.
                        Aquarell\pwindex{Schuetz, Johann Georg 1755-05-16 – 1813-05-11@\textsc{Schütz, Johann Georg} (1755-05-16 – 1813-05-11), \emph{Bildender Künstler}!Anna Amalia und ihre Begleiter im Garten der Villa DEste zu Tivoli1789@\strich\emph{Anna Amalia und ihre Begleiter im Garten der Villa d’Este zu Tivoli} {[}1789{]}|pwv} v. Schütz\pwindex{Schuetz, Johann Georg 1755-05-16 – 1813-05-11@\textsc{Schütz, Johann Georg} (1755-05-16 – 1813-05-11), \emph{Bildender Künstler}|pw}. (Schloss Tiefurt\oindex{Schloss Tiefurt@\textbf{Schloss Tiefurt}|pw} bei Weimar\oindex{Weimar@\textbf{Weimar}|pw}.)}}\pend
           \pstart
           \raggedleft{}Weimar\oindex{Weimar@\textbf{Weimar}|pw}{ }1. III.\pend
           \pstart
           Danke Ihnen, lieber Arthur, ſehr für die lieben Zeilen (und wünſche
               mir ſehr mündlich weiteres). Wir ſind Montag zurück und freuen uns auf
               Euch.\pend
           \pstart \spacefill\mbox{Hugo.}\pend{}
         
         \endnumbering\mylabel{h}\end{ledgroupsized}  \newcommand{\dateiname}{L01915}\newcommand{\titel}{Hugo von Hofmannsthal an Arthur Schnitzler, 1. 3. 1910}\newcommand{\editorInnen}{Martin Anton Müller und Gerd-Hermann Susen}%% latex-leseansicht-abspann.tex
%% Abspann für die Leseansicht.
%% Der Schalter \ifkorrekturansicht ist bereits durch den Vorspann gesetzt.

%% latex-abspann.tex
%% Gemeinsamer Abspann für Korrekturansicht und Leseansicht.
%% Setzt den Schalter \ifkorrekturansicht voraus (gesetzt in den
%% einbindenden Dateien latex-korrekturansicht-abspann.tex bzw.
%% latex-leseansicht-abspann.tex).
%% ---------------------------------------------------------------

\normalsize

% Das esempio-Environment wird nur in der Leseansicht benötigt
\ifkorrekturansicht\else
\newenvironment{esempio}[3]%
{
    \vspace{1.5ex}
    \rlap{\underline{#1}}
    \par
    \setlength{\parindent}{0cm}
    \nopagebreak
    \leftskip=#2cm
    \rightskip=#3cm
}
{
    \par
}
\fi

\doendnotes{C}
\bigskip
\vfill

\clearpage

\footnotesize

\ifkorrekturansicht
  \lohead{\textsc{register}}
\fi

% theindex-Environment neu definieren ohne reledmac
\makeatletter
\renewenvironment{theindex}{%
  \ifkorrekturansicht
    \section*{\indexname}%
  \else
    \subsubsection*{Index der erwähnten Entitäten}%
  \fi
  \setlength{\parindent}{0pt}%
  \setlength{\parskip}{0pt plus 0.3pt}%
  \let\item\@idxitem
}{%
  \ifkorrekturansicht\clearpage\fi
}
\makeatother

\IfFileExists{\jobname-pw.ind}{\input{\jobname-pw.ind}}{}

% Quellenangabe nur in der Leseansicht
\ifkorrekturansicht\else
% Fallback-Definitionen, falls die .tex-Datei \titel etc. nicht gesetzt hat
\providecommand{\titel}{}
\providecommand{\editorInnen}{}
\providecommand{\dateiname}{\jobname}

\vspace{3cm}

\vfill

\footnotesize
\textsc{Quelle}: \titel. Herausgegeben von {\editorInnen}. In: \emph{Arthur Schnitzler: Briefwechsel mit Autorinnen und Autoren}.
 Digitale Edition, https://schnitzler-briefe.acdh.oeaw.ac.at/{\dateiname}.html (Stand \today)
\fi

\end{document}


      