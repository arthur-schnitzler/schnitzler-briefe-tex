%% latex-leseansicht-vorspann.tex
%% Vorspann für die Leseansicht.
%% Lädt die gemeinsame Datei latex-vorspann.tex mit nicht gesetztem Schalter.

\newif\ifkorrekturansicht
\korrekturansichtfalse

\input{../tex-inputs/latex-vorspann}


\section[Hugo von Hofmannsthal an Arthur Schnitzler, 4. 8. {[}1892{]}]{L00111 Hugo von Hofmannsthal an Arthur Schnitzler, 4. 8. [1892]}
\nopagebreak\mylabel{L00111v}
\rehead{ }\normalsize\beginnumbering\briefempfaengerindex{Schnitzler, Arthur@\textsc{Schnitzler, Arthur}!zzzHofmannsthal, Hugo von@\emph{von Hugo von Hofmannsthal}!1892-08-041@{4. 8. [1892]}|(be}
\toendnotes[C]{\smallbreak\pagebreak[2]}
\correspDesc{Versand  durch Hugo von Hofmannsthal am 4. 8. [1892] in Bad Fusch
\newline{}Erhalt  durch Arthur Schnitzler im Zeitraum [5. 8. 1892
                  – 9. 8. 1892?] in Wien}\toendnotes[C]{\smallbreak}
\Standort{CUL, Schnitzler, B 43.}
\physDesc{Brief, 1 Blatt, 4 Seiten, 1695 Zeichen (aufgeprägtes Wappen)
\newline{}Handschrift: schwarze Tinte, deutsche Kurrent
\newline{}Schnitzler: mit Bleistift die Jahreszahl ergänzt: »92« 
\newline{}Ordnung: mit Bleistift von unbekannter Hand nummeriert:
                                    »29« }
\buchAbdrucke{\weitereDrucke{1) Hugo von Hofmannsthal: \emph{Briefe an Freunde.} In: \emph{Die neue Rundschau}, Jg. 41, Nr. 4, April 1930, S. 513–514.} \weitereDrucke{2) Hugo von Hofmannsthal: \emph{Briefe. 1890–1901}. Berlin: \emph{S. Fischer} 1935, S. 60–61.} \weitereDrucke{3) Hugo von Hofmannsthal, Arthur Schnitzler: \emph{Briefwechsel}. Herausgegeben von Therese Nickl und Heinrich Schnitzler. Frankfurt am Main: \emph{S. Fischer} 1964, S. 26.} }\toendnotes[C]{\smallbreak}
\pstart
           \raggedleft{}{\pb}Fuſch\oindex{Bad Fusch@\textbf{Bad Fusch}|pw}{\\}4 VIII.\pend
           
\pstart\center{}Lieber Arthur.\pend\vspace{0.5em}
\pstart
           Da haben Sie die \label{K_L00111-1v}\edtext{Märchen\pwindex{Schnitzler, Arthur 15.\,5.\,1862 Wien – 21.\,10.\,1931 ebd.@\textsc{Schnitzler, Arthur} (15.\,5.\,1862 Wien – 21.\,10.\,1931 ebd.), \emph{Schriftsteller, Mediziner}!Märchen. Schauspiel in drei Aufzügen@\strich\emph{Das Märchen. Schauspiel in drei Aufzügen}|pw}kritik}{\lemma{\textnormal{\emph{Märchenkritik}}}\Cendnote{\textnormal{nicht publizierte und nicht erhaltene Kritik}}}\label{K_L00111-1} der \textsc{Herzfeld}\pwindex{Herzfeld, Marie 20.\,3.\,1855 Kőszeg – 22.\,9.\,1940 Mining@\textsc{Herzfeld, Marie} (20.\,3.\,1855 Kőszeg – 22.\,9.\,1940 Mining), \emph{Schriftstellerin, Übersetzerin}|pw}. Ich habe ihr für die{ }ſympathiſche Ausführlichkeit gedankt und ihr von dem
               Erſcheinen des Anatol\pwindex{Schnitzler, Arthur 15.\,5.\,1862 Wien – 21.\,10.\,1931 ebd.@\textsc{Schnitzler, Arthur} (15.\,5.\,1862 Wien – 21.\,10.\,1931 ebd.), \emph{Schriftsteller, Mediziner}!Anatol@\strich\emph{Anatol}|pw}-Buches geſprochen; wie
               heißt denn der Verlag\orgindex{Bibliographisches Bureau@Bibliographisches Bureau|pw}? –\pend
           
\pstart
           Ich habe den erſten Act\pwindex{Hofmannsthal, Hugo von 1.\,2.\,1874 Wien – 15.\,7.\,1929 Rodaun@\textsc{Hofmannsthal, Hugo von} (1.\,2.\,1874 Wien – 15.\,7.\,1929 Rodaun), \emph{Schriftsteller}!Ascanio und Gioconda@\strich\emph{Ascanio und Gioconda}|pwv}
               (654 Verſe) vollendet, den zweiten\pwindex{Hofmannsthal, Hugo von 1.\,2.\,1874 Wien – 15.\,7.\,1929 Rodaun@\textsc{Hofmannsthal, Hugo von} (1.\,2.\,1874 Wien – 15.\,7.\,1929 Rodaun), \emph{Schriftsteller}!Ascanio und Gioconda@\strich\emph{Ascanio und Gioconda}|pwv} beinahe.\pend
           
\pstart
           Unſere Art zu arbeiten (im Drama) iſt nicht gar{ }ſo verſchieden, wie Sie anzunehmen{ }ſcheinen; was ich {\pb}aus{ }ſpäteren
               Acten vorausarbeiten kann,{ }ſind nicht geſchloſſene Scenen,{ }ſondern reine
               Farbenſkizzen: Worte und Dialogſtellen, die oft dann gar nicht wirklich aufgenommen
               werden, mir aber als Parfümflaſchen, als Stimmungs-Accumulatoren und -Condenſatoren
               dienen, damit die Suggeſtion im Laufe der Detailarbeit nicht verloren geht; das ganze
               hängt wahrſcheinlich mit meiner Ihnen gegenüber mehr lyriſchen, mehr auf Farbe
               hinarbeitenden Technik zuſammen. Wie lange {\pb}bleiben Sie in Wien\oindex{Wien@\textbf{Wien}, \emph{Verwaltungsgebiet}|pw}? kann man Ihnen während der Waffenübung{ }ſchreiben?\pend
           
\pstart
           Ich freue mich{ }ſehr auf die Novelle\pwindex{Schnitzler, Arthur 15.\,5.\,1862 Wien – 21.\,10.\,1931 ebd.@\textsc{Schnitzler, Arthur} (15.\,5.\,1862 Wien – 21.\,10.\,1931 ebd.), \emph{Schriftsteller, Mediziner}!Sterben. Novelle@\strich\emph{Sterben. Novelle}|pw}; ich hoffe
               Sie werden nichts vor meiner Rückkehr vorleſen.\pend
           
\pstart
           Ich bin vom 7\textsuperscript{ten} – 31\textsuperscript{ten} Auguſt in Strobl bei Iſchl\oindex{Strobl@\textbf{Strobl}, \emph{Verwaltungsgebiet}|pw}.\pend
           
\pstart
           Herzlichſt grüßend{\\[\baselineskip]}\spacefill\mbox{Loris.}\pend
           \leftskip=0em{}
\pstart
           \noindent{}\textsc{P. S.} Was die \textsc{Herzfeld}\pwindex{Herzfeld, Marie 20.\,3.\,1855 Kőszeg – 22.\,9.\,1940 Mining@\textsc{Herzfeld, Marie} (20.\,3.\,1855 Kőszeg – 22.\,9.\,1940 Mining), \emph{Schriftstellerin, Übersetzerin}|pw} von nothwendiger Technik \strikeout{und} für
                  Bühnenfernwirkung und von »concentrierter« Natürlichkeit des Dialog’s{ }ſagt,{ }ſcheint mir{ }ſehr vernünftig; {\pb}es iſt dies thatſächlich die Erfahrung des allerletzten Theaterjahres für jeden
                  Objectiven und für künftige Arbeiten nicht unwichtig: ganz die gleichen
                  Rathſchläge, mit zahlloſen anderen höchſt wertvollen, finde ich in den kritiſchen
                  Studien von Otto Ludwig\pwindex{Ludwig, Otto 12.\,2.\,1813 Eisfeld – 25.\,2.\,1865 Dresden@\textsc{Ludwig, Otto} (12.\,2.\,1813 Eisfeld – 25.\,2.\,1865 Dresden), \emph{Schriftsteller, Schriftsteller, Krimiautor}|pw}, aus denen ich hier
                  mit Genuſs und innerer Freude eine Menge lerne. Über Technik des dramatiſchen
                  Dramas zum Unterſchied vom herrſchenden Novellendrama muſs überhaupt nächſten
                  Winter bei Ihnen{ }ſehr viel geredet werden.\pend
           \selectlanguage{ngerman}\endnumbering\briefempfaengerindex{Schnitzler, Arthur@\textsc{Schnitzler, Arthur}!zzzHofmannsthal, Hugo von@\emph{von Hugo von Hofmannsthal}!1892-08-041@{4. 8. [1892]}|)be}\mylabel{L00111h}  \newcommand{\dateiname}{L00111}\newcommand{\titel}{Hugo von Hofmannsthal an Arthur Schnitzler, 4. 8. [1892]}\newcommand{\editorInnen}{Martin Anton Müller und Gerd-Hermann Susen}%% latex-leseansicht-abspann.tex
%% Abspann für die Leseansicht.
%% Der Schalter \ifkorrekturansicht ist bereits durch den Vorspann gesetzt.

%% latex-abspann.tex
%% Gemeinsamer Abspann für Korrekturansicht und Leseansicht.
%% Setzt den Schalter \ifkorrekturansicht voraus (gesetzt in den
%% einbindenden Dateien latex-korrekturansicht-abspann.tex bzw.
%% latex-leseansicht-abspann.tex).
%% ---------------------------------------------------------------

\normalsize

% Das esempio-Environment wird nur in der Leseansicht benötigt
\ifkorrekturansicht\else
\newenvironment{esempio}[3]%
{
    \vspace{1.5ex}
    \rlap{\underline{#1}}
    \par
    \setlength{\parindent}{0cm}
    \nopagebreak
    \leftskip=#2cm
    \rightskip=#3cm
}
{
    \par
}
\fi

\doendnotes{C}
\bigskip
\vfill

\clearpage

\footnotesize

\ifkorrekturansicht
  \lohead{\textsc{register}}
\fi

% theindex-Environment neu definieren ohne reledmac
\makeatletter
\renewenvironment{theindex}{%
  \ifkorrekturansicht
    \section*{\indexname}%
  \else
    \subsubsection*{Index der erwähnten Entitäten}%
  \fi
  \setlength{\parindent}{0pt}%
  \setlength{\parskip}{0pt plus 0.3pt}%
  \let\item\@idxitem
}{%
  \ifkorrekturansicht\clearpage\fi
}
\makeatother

\IfFileExists{\jobname-pw.ind}{\input{\jobname-pw.ind}}{}

% Quellenangabe nur in der Leseansicht
\ifkorrekturansicht\else
% Fallback-Definitionen, falls die .tex-Datei \titel etc. nicht gesetzt hat
\providecommand{\titel}{}
\providecommand{\editorInnen}{}
\providecommand{\dateiname}{\jobname}

\vspace{3cm}

\vfill

\footnotesize
\textsc{Quelle}: \titel. Herausgegeben von {\editorInnen}. In: \emph{Arthur Schnitzler: Briefwechsel mit Autorinnen und Autoren}.
 Digitale Edition, https://schnitzler-briefe.acdh.oeaw.ac.at/{\dateiname}.html (Stand \today)
\fi

\end{document}


