%% latex-leseansicht-vorspann.tex
%% Vorspann für die Leseansicht.
%% Lädt die gemeinsame Datei latex-vorspann.tex mit nicht gesetztem Schalter.

\newif\ifkorrekturansicht
\korrekturansichtfalse

\input{../tex-inputs/latex-vorspann}


\section[Richard Beer-Hofmann an Arthur Schnitzler, 29. 8. 1907]{L01703 Richard Beer-Hofmann an Arthur Schnitzler, 29. 8. 1907}
\nopagebreak\mylabel{L01703v}
\rehead{ }\normalsize\beginnumbering\briefempfaengerindex{Schnitzler, Arthur@\textsc{Schnitzler, Arthur}!zzzBeer-Hofmann, Richard@\emph{von Richard Beer-Hofmann}!1907-08-291@{29. 8. 1907}|(be}
\toendnotes[C]{\smallbreak\pagebreak[2]}
\correspDesc{Versand  durch Richard Beer-Hofmann am 29. 8. 1907 in Velden am Wörthersee
\newline{}Erhalt  durch Arthur Schnitzler im Zeitraum [30. 8. 1907
                  – 3. 9. 1907?] in Karersee}\toendnotes[C]{\smallbreak}
\Standort{CUL, Schnitzler, B 8.}
\physDesc{Brief, 1 Blatt, 2 Seiten, 1004 Zeichen
\newline{}Handschrift: Bleistift, lateinische Kurrent
\newline{}Ordnung: mit Bleistift von unbekannter Hand nummeriert:
                                    »212« }
\buchAbdrucke{\weitereDrucke{Arthur Schnitzler, Richard Beer-Hofmann: \emph{Briefwechsel 1891–1931}. Herausgegeben von Konstanze Fliedl. Wien, Zürich: \emph{Europaverlag} 1992, S. 184.} }\toendnotes[C]{\smallbreak}
\pstart
           \raggedleft{}{\pb}Velden\oindex{Velden am Wörthersee@\textbf{Velden am Wörthersee}|pw}{ }29/VIII 07\pend
           \vspace{0.5em}
\pstart
           Lieber Arthur! Wir haben überlegt: Es wäre mit drei Kindern\pwindex{Beer-Hofmann, Naëmah 20.\,12.\,1898 Wien – 10.\,11.\,1971 New York City@\textsc{Beer-Hofmann, Naëmah} (20.\,12.\,1898 Wien – 10.\,11.\,1971 New York City)|pwv}\pwindex{Beer-Hofmann, Mirjam 4.\,9.\,1897 Wien – 24.\,12.\,1984 New York City@\textsc{Beer-Hofmann, Mirjam} (4.\,9.\,1897 Wien – 24.\,12.\,1984 New York City)|pwv}\pwindex{Beer-Hofmann, Gabriel 9.\,1.\,1901 Wien – 24.\,3.\,1971 St Albans@\textsc{Beer-Hofmann, Gabriel} (9.\,1.\,1901 Wien – 24.\,3.\,1971 St Albans), \emph{Schriftsteller, Filmagent}|pwv} u.
               der Christine\pwindex{Christine @\textsc{Christine}, \emph{Kinderbetreuerin}|pw} – (6 in einem Wagen) nicht schön
               4 Tage im Wagen bis Bozen\oindex{Bozen@\textbf{Bozen}, \emph{Hauptstadt}|pw} zu fahren. Auch für
               das täglich Aus und Einpacken – täglich wo anders übernachten – sind bessere Nerven
               nötig, als Paula\pwindex{Beer-Hofmann, Paula 25.\,2.\,1879 Wien – 30.\,10.\,1939 Zürich@\textsc{Beer-Hofmann, Paula} (25.\,2.\,1879 Wien – 30.\,10.\,1939 Zürich)|pw} augenblicklich hat. Sie hat
               nur den Wunsch viel zu schlafen, ruhig zu sitzen, und in sehr heisser Sonne zu
               braten. So drängt Alles nach dem Lido\oindex{Lido@\textbf{Lido}|pw}, den wir in
               nicht ganz sieben Stunden von hier, erreichen können.\pend
           
\pstart
           {\pb}Ich reise also Samstag hier ab –
               bin es – wenn Sie dies lesen hoffentlich schon – übernachte in Villach\oindex{Villach@\textbf{Villach}, \emph{Verwaltungsgebiet}|pw} und fahre Sonntag Früh nach Venedig\oindex{Venedig@\textbf{Venedig}|pw}, – vorläufig Bauer-Grünwald\oindex{Grand Hotel Bauer-Grünwald@\textbf{Grand Hotel Bauer-Grünwald}, \emph{Hotel}|pw}, bis wir Zi{\geminationm}er auf dem Lido\oindex{Lido@\textbf{Lido}|pw} beko{\geminationm}en. So
               werde ich Sie erst wieder in Wien\oindex{Wien@\textbf{Wien}, \emph{Verwaltungsgebiet}|pw} sehen, ausser
               Sie wählen den Rückweg über Venedig\oindex{Venedig@\textbf{Venedig}|pw} – was ja auch
               einiges für sich hätte. Im Herbst erhoffe ich mir \strikeout{so} ein paar schöne Tage mit Spaziergängen mit Ihnen –
               hier folgt eine Schilderung Wiens\oindex{Wien@\textbf{Wien}, \emph{Verwaltungsgebiet}|pw} im Herbst – von
               Ihnen besser besorgt als von mir. Von Herzen\pend
           \pstart Ihr \spacefill\mbox{Richard}\pend{}
\pstart
           Grüsse an Frau Olga\pwindex{Schnitzler, Olga 17.\,1.\,1882 Wien – 13.\,1.\,1970 Lugano@\textsc{Schnitzler, Olga} (17.\,1.\,1882 Wien – 13.\,1.\,1970 Lugano), \emph{Schauspielerin, Sängerin}|pw} von Paula\pwindex{Beer-Hofmann, Paula 25.\,2.\,1879 Wien – 30.\,10.\,1939 Zürich@\textsc{Beer-Hofmann, Paula} (25.\,2.\,1879 Wien – 30.\,10.\,1939 Zürich)|pw} u mir\pend
           \selectlanguage{ngerman}\endnumbering\briefempfaengerindex{Schnitzler, Arthur@\textsc{Schnitzler, Arthur}!zzzBeer-Hofmann, Richard@\emph{von Richard Beer-Hofmann}!1907-08-291@{29. 8. 1907}|)be}\mylabel{L01703h}  \newcommand{\dateiname}{L01703}\newcommand{\titel}{Richard Beer-Hofmann an Arthur Schnitzler, 29. 8. 1907}\newcommand{\editorInnen}{Martin Anton Müller und Gerd-Hermann Susen}%% latex-leseansicht-abspann.tex
%% Abspann für die Leseansicht.
%% Der Schalter \ifkorrekturansicht ist bereits durch den Vorspann gesetzt.

%% latex-abspann.tex
%% Gemeinsamer Abspann für Korrekturansicht und Leseansicht.
%% Setzt den Schalter \ifkorrekturansicht voraus (gesetzt in den
%% einbindenden Dateien latex-korrekturansicht-abspann.tex bzw.
%% latex-leseansicht-abspann.tex).
%% ---------------------------------------------------------------

\normalsize

% Das esempio-Environment wird nur in der Leseansicht benötigt
\ifkorrekturansicht\else
\newenvironment{esempio}[3]%
{
    \vspace{1.5ex}
    \rlap{\underline{#1}}
    \par
    \setlength{\parindent}{0cm}
    \nopagebreak
    \leftskip=#2cm
    \rightskip=#3cm
}
{
    \par
}
\fi

\doendnotes{C}
\bigskip
\vfill

\clearpage

\footnotesize

\ifkorrekturansicht
  \lohead{\textsc{register}}
\fi

% theindex-Environment neu definieren ohne reledmac
\makeatletter
\renewenvironment{theindex}{%
  \ifkorrekturansicht
    \section*{\indexname}%
  \else
    \subsubsection*{Index der erwähnten Entitäten}%
  \fi
  \setlength{\parindent}{0pt}%
  \setlength{\parskip}{0pt plus 0.3pt}%
  \let\item\@idxitem
}{%
  \ifkorrekturansicht\clearpage\fi
}
\makeatother

\IfFileExists{\jobname-pw.ind}{\input{\jobname-pw.ind}}{}

% Quellenangabe nur in der Leseansicht
\ifkorrekturansicht\else
% Fallback-Definitionen, falls die .tex-Datei \titel etc. nicht gesetzt hat
\providecommand{\titel}{}
\providecommand{\editorInnen}{}
\providecommand{\dateiname}{\jobname}

\vspace{3cm}

\vfill

\footnotesize
\textsc{Quelle}: \titel. Herausgegeben von {\editorInnen}. In: \emph{Arthur Schnitzler: Briefwechsel mit Autorinnen und Autoren}.
 Digitale Edition, https://schnitzler-briefe.acdh.oeaw.ac.at/{\dateiname}.html (Stand \today)
\fi

\end{document}


