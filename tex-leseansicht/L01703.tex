%% latex-korrekturansicht-vorspann.tex
%% Vorspann für die Korrekturansicht.
%% Lädt die gemeinsame Datei latex-vorspann.tex mit gesetztem Schalter.

\newif\ifkorrekturansicht
\korrekturansichttrue

\input{../tex-inputs/latex-vorspann}


\section[Richard Beer-Hofmann an Arthur Schnitzler, 29. 8. 1907]{L01703 Richard Beer-Hofmann an Arthur Schnitzler, 29. 8. 1907}
\nopagebreak\mylabel{L01703v}
\rehead{ }\normalsize\beginnumbering\briefempfaengerindex{Schnitzler, Arthur@\textsc{Schnitzler, Arthur}!zzzBeer-Hofmann, Richard@\emph{von Richard Beer-Hofmann}!1907-08-291@{29. 8. 1907}|(be}
\toendnotes[C]{\smallbreak\pagebreak[2]}\Standort{CUL, Schnitzler, B 8.}
\physDesc{Brief, 1 Blatt, 2 Seiten, 1004 Zeichen
\newline{}Handschrift: Bleistift, lateinische Kurrent
\newline{}Ordnung: mit Bleistift von unbekannter Hand nummeriert:
                                    »212« }
\buchAbdrucke{\weitereDrucke{Arthur Schnitzler, Richard Beer-Hofmann: \emph{Briefwechsel 1891–1931}. Wien, Zürich: \emph{Europaverlag} 1992, S. 184.} }\toendnotes[C]{\smallbreak}
\pstart
           \raggedleft{}{\pb}Velden\oindex{Velden am Woerthersee@\textbf{Velden am Wörthersee}, \emph{P.PPL}|pw}{ }29/VIII 07\pend
           \vspace{0.5em}
\pstart
           Lieber Arthur! Wir haben überlegt: Es wäre mit drei Kindern\pwindex{Beer-Hofmann, Naemah 20.12.1898 – 10.11.1971@\textsc{Beer-Hofmann, Naëmah} (20.12.1898 – 10.11.1971)|pwv}\pwindex{Beer-Hofmann, Mirjam 04.09.1897 – 24.12.1984@\textsc{Beer-Hofmann, Mirjam} (04.09.1897 – 24.12.1984)|pwv}\pwindex{Beer-Hofmann, Gabriel 09.01.1901 – 24.03.1971@\textsc{Beer-Hofmann, Gabriel} (09.01.1901 – 24.03.1971), \emph{Schriftsteller/Schriftstellerin, Filmagent/Filmagentin}|pwv} u.
               der Christine\pwindex{Christine @\textsc{Christine}, \emph{Kinderbetreuer/Kinderbetreuerin}|pw} – (6 in einem Wagen) nicht schön
               4 Tage im Wagen bis Bozen\oindex{Bozen@\textbf{Bozen}, \emph{P.PPLA2}|pw} zu fahren. Auch für
               das täglich Aus und Einpacken – täglich wo anders übernachten – sind bessere Nerven
               nötig, als Paula\pwindex{Beer-Hofmann, Paula 25.02.1879 – 30.10.1939@\textsc{Beer-Hofmann, Paula} (25.02.1879 – 30.10.1939)|pw} augenblicklich hat. Sie hat
               nur den Wunsch viel zu schlafen, ruhig zu sitzen, und in sehr heisser Sonne zu
               braten. So drängt Alles nach dem Lido\oindex{Lido@\textbf{Lido}, \emph{P.PPL}|pw}, den wir in
               nicht ganz sieben Stunden von hier, erreichen können.\pend
           
\pstart
           {\pb}Ich reise also Samstag hier ab –
               bin es – wenn Sie dies lesen hoffentlich schon – übernachte in Villach\oindex{Villach@\textbf{Villach}, \emph{A.ADM3}|pw} und fahre Sonntag Früh nach Venedig\oindex{Venedig@\textbf{Venedig}, \emph{P.PPLA}|pw}, – vorläufig Bauer-Grünwald\oindex{Grand Hotel Bauer-Gruenwald@\textbf{Grand Hotel Bauer-Grünwald}, \emph{Hotel (K.HTL)}|pw}, bis wir Zi{\geminationm}er auf dem Lido\oindex{Lido@\textbf{Lido}, \emph{P.PPL}|pw} beko{\geminationm}en. So
               werde ich Sie erst wieder in Wien\oindex{Wien@\textbf{Wien}, \emph{A.ADM2}|pw} sehen, ausser
               Sie wählen den Rückweg über Venedig\oindex{Venedig@\textbf{Venedig}, \emph{P.PPLA}|pw} – was ja auch
               einiges für sich hätte. Im Herbst erhoffe ich mir \strikeout{so} ein paar schöne Tage mit Spaziergängen mit Ihnen –
               hier folgt eine Schilderung Wiens\oindex{Wien@\textbf{Wien}, \emph{A.ADM2}|pw} im Herbst – von
               Ihnen besser besorgt als von mir. Von Herzen\pend
           \pstart Ihr \spacefill\mbox{Richard}\pend{}
\pstart
           Grüsse an Frau Olga\pwindex{Schnitzler, Olga 17.01.1882 – 13.01.1970@\textsc{Schnitzler, Olga} (17.01.1882 – 13.01.1970), \emph{Schauspieler/Schauspielerin, Sänger/Sängerin}|pw} von Paula\pwindex{Beer-Hofmann, Paula 25.02.1879 – 30.10.1939@\textsc{Beer-Hofmann, Paula} (25.02.1879 – 30.10.1939)|pw} u mir\pend
           \selectlanguage{ngerman}\endnumbering\briefempfaengerindex{Schnitzler, Arthur@\textsc{Schnitzler, Arthur}!zzzBeer-Hofmann, Richard@\emph{von Richard Beer-Hofmann}!1907-08-291@{29. 8. 1907}|)be}\mylabel{L01703h}  \normalsize

\doendnotes{C}
\bigskip
\vfill

\clearpage

\footnotesize

\lohead{\textsc{register}}

% Definiere theindex-Environment komplett neu ohne reledmac
\makeatletter
\renewenvironment{theindex}{%
  \section*{\indexname}%
  \setlength{\parindent}{0pt}%
  \setlength{\parskip}{0pt plus 0.3pt}%
  \let\item\@idxitem
}{%
  \clearpage
}
\makeatother

\IfFileExists{\jobname-pw.ind}{\input{\jobname-pw.ind}}{}

\end{document}

      