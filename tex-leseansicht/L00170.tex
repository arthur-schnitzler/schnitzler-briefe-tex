%% latex-leseansicht-vorspann.tex
%% Vorspann für die Leseansicht.
%% Lädt die gemeinsame Datei latex-vorspann.tex mit nicht gesetztem Schalter.

\newif\ifkorrekturansicht
\korrekturansichtfalse

\input{../tex-inputs/latex-vorspann}


\section[Arthur Schnitzler an Hugo von Hofmannsthal, {[}1. 2. 1893{]}]{L00170 Arthur Schnitzler an Hugo von Hofmannsthal, {[}1. 2. 1893{]}}
\nopagebreak\mylabel{L00170v}
\rehead{ }\normalsize\beginnumbering\briefempfaengerindex{Hofmannsthal, Hugo von@\textsc{Hofmannsthal, Hugo von}!zzzSchnitzler, Arthur@\emph{von Arthur Schnitzler}!1893-02-012@{{[}1. 2. 1893{]}}|(be}
\toendnotes[C]{\smallbreak\pagebreak[2]}
\correspDesc{Versand  durch Arthur Schnitzler am [1. 2. 1893] in Wien
\newline{}Erhalt  durch Hugo von Hofmannsthal im Zeitraum [1. 2. 1893
                  – 5. 2. 1893?] in Wien}\toendnotes[C]{\smallbreak}
\Standort{FDH, Hs-30885,33.}
\physDesc{Briefkarte, 1188 Zeichen
\newline{}Handschrift: schwarze Tinte, deutsche Kurrent
\newline{}Ordnung: mit Bleistift von Schnitzler mutmaßlich bei der Durchsicht der Korrespondenz
                                    1929  datiert: »\substVorne{}\textsuperscript{91}\substDazwischen{}Anfang 93\substHinten{}« }
\buchAbdrucke{\weitereDrucke{Hugo von Hofmannsthal, Arthur Schnitzler: \emph{Briefwechsel}. Herausgegeben von Therese Nickl und Heinrich Schnitzler. Frankfurt am Main: \emph{S. Fischer} 1964, S. 34.} }\toendnotes[C]{\smallbreak}
\pstart{}{\pb}Mein lieber Hugo,\pend\vspace{0.5em}
\pstart
           Fels\pwindex{Fels, Friedrich Michael *~1864 Bad Dürkheim@\textsc{Fels, Friedrich Michael} (*~1864 Bad Dürkheim), \emph{Journalist}|pw} befindet{ }ſich bereits beſſer; ernſtere
               Beſorgniſſe{ }ſind nun wohl auszuſchließen. Hingegen wäre nunmehr Ihre{ }ſ. Z.
               beſprochene Liebenswürdigkeit{ }ſehr erwünſcht, u die Idee mit den Freunden ohne
                  Namensne{\geminationn}ung iſt{ }ſehr gut, und raſcher Durchführung
               zu empfehlen. –\pend
           
\pstart
           Die Arbeit Engländers\pwindex{Altenberg, Peter 9.\,3.\,1859 Wien – 8.\,1.\,1919 ebd.@\textsc{Altenberg, Peter} (9.\,3.\,1859 Wien – 8.\,1.\,1919 ebd.), \emph{Schriftsteller}|pw} iſt über Sölneß\pwindex{\textcolor{red}{\textsuperscript{XXXX indx1}}!Baumeister Solness. Schauspiel in drei Aufzügen  |@\strich\emph{Baumeister Solness. Schauspiel in drei Aufzügen |}|pw}; Schick\pwindex{Schik, Friedrich *~6.\,9.\,1857 Wien@\textsc{Schik, Friedrich} (*~6.\,9.\,1857 Wien), \emph{Notar, Journalist, Dramaturg}|pw} richtete das Ihnen übermittelte Erſuchen an mich. –\pend
           
\pstart
           Was{ }ſoll ich der akad. Vereinigung\orgindex{Wiener Akademische Vereinigung@Wiener Akademische Vereinigung|pw} ins Exemplar\pwindex{Schnitzler, Arthur 15.\,5.\,1862 Wien – 21.\,10.\,1931 ebd.@\textsc{Schnitzler, Arthur} (15.\,5.\,1862 Wien – 21.\,10.\,1931 ebd.), \emph{Schriftsteller, Mediziner}!Anatol@\strich\emph{Anatol}|pwv}{ }ſchreiben, ich ke{\geminationn}
               mich da gar nicht aus? – Teltſch\pwindex{Telcs, Ede 12.\,5.\,1872 Baja – 1948 Budapest@\textsc{Telcs, Ede} (12.\,5.\,1872 Baja – 1948 Budapest), \emph{Bildhauer}|pw} erhält eins,
                  {\pb}ſobald ich wieder welche von Berlin\orgindex{Bibliographisches Bureau@Bibliographisches Bureau|pw} beko{\geminationm}e, in ein paar Tagen; ich
               grüſs ihn herzlich. – Sah heute im Gewerbemuſeum\oindex{Wien@\textbf{Wien}!I., Innere Stadt@\textbf{I., Innere Stadt}!Museum für Angewandte Kunst@\textbf{Museum für Angewandte Kunst}, \emph{Museum}|pw}
               Ihr \label{K_L00170-1v}\edtext{Relief\pwindex{Telcs, Ede 12.\,5.\,1872 Baja – 1948 Budapest@\textsc{Telcs, Ede} (12.\,5.\,1872 Baja – 1948 Budapest), \emph{Bildhauer}!Hugo von Hofmannsthal@\strich\emph{Hugo von Hofmannsthal}|pwv}}{\lemma{\textnormal{\emph{Relief}}}\Cendnote{\textnormal{Das Relief\pwindex{Telcs, Ede 12.\,5.\,1872 Baja – 1948 Budapest@\textsc{Telcs, Ede} (12.\,5.\,1872 Baja – 1948 Budapest), \emph{Bildhauer}!Hugo von Hofmannsthal@\strich\emph{Hugo von Hofmannsthal}|pwkv} befindet sich heute in der Sammlung Richard
                     und Hilda Mises, \emph{Houghton Library}\orgindex{Houghton Library@Houghton Library|pwk},
                     Harvard.}}}\label{K_L00170-1}. Plötzlich lag es da, zwiſchen einem pompej\oindex{Pompeji@\textbf{Pompeji}, \emph{Ausgrabung}|pw}aniſchen Tiſchfuſs und einem Nürnberg\oindex{Nürnberg@\textbf{Nürnberg}|pw}er Hanswurſt. – Ich glaube, es iſt{ }ſehr gut, hab’ aber
               kein gutes Licht gehabt. –\pend
           
\pstart
           \textsc{Salten}\pwindex{Salten, Felix 6.\,9.\,1869 Budapest – 8.\,10.\,1945 Zürich@\textsc{Salten, Felix} (6.\,9.\,1869 Budapest – 8.\,10.\,1945 Zürich), \emph{Schriftsteller, Journalist, Chefredakteur}|pw}{ }ſoll Mitte März fort. – \label{K_L00170-2v}\edtext{Familie\pwindex{Schnitzler, Arthur 15.\,5.\,1862 Wien – 21.\,10.\,1931 ebd.@\textsc{Schnitzler, Arthur} (15.\,5.\,1862 Wien – 21.\,10.\,1931 ebd.), \emph{Schriftsteller, Mediziner}!Familie@\strich\emph{Familie}|pw} beendet}{\lemma{\textnormal{\emph{Familie beendet}}}\Cendnote{\textnormal{Das erlaubt die Datierung des Briefes nach dem 24. 1. 1893, da dieser
                  Tag sowohl im \emph{Tagebuch}\pwindex{Schnitzler, Arthur 15.\,5.\,1862 Wien – 21.\,10.\,1931 ebd.@\textsc{Schnitzler, Arthur} (15.\,5.\,1862 Wien – 21.\,10.\,1931 ebd.), \emph{Schriftsteller, Mediziner}!Tagebuch@\strich\emph{Tagebuch}|pwk} als auch am Manuskript
                  (vgl. Arthur Schnitzler: \emph{Entworfenes und Verworfenes. Aus dem Nachlaß}. Herausgegeben von Reinhard Urbach. Frankfurt/Main: \emph{S. Fischer}{ }1977, S. 508)
                  als Datum des Abschlusses genannt wird.}}}\label{K_L00170-2}, traue mich nicht \strikeout{zu}{ }ſie durchzuleſen; fürchte mich vor der grauſamen
               Gewißheit. Abſicht: Ende Feber auf 10–14 Tage in die Wärme, von der
               Klinik und dem grauen Leben weg, das Stück\pwindex{Schnitzler, Arthur 15.\,5.\,1862 Wien – 21.\,10.\,1931 ebd.@\textsc{Schnitzler, Arthur} (15.\,5.\,1862 Wien – 21.\,10.\,1931 ebd.), \emph{Schriftsteller, Mediziner}!Familie@\strich\emph{Familie}|pwv} im Koffer. \label{K_L00170-3v}\edtext{Schreibe jetzt »Verwandlungen\pwindex{Schnitzler, Arthur 15.\,5.\,1862 Wien – 21.\,10.\,1931 ebd.@\textsc{Schnitzler, Arthur} (15.\,5.\,1862 Wien – 21.\,10.\,1931 ebd.), \emph{Schriftsteller, Mediziner}!kleine Komödie@\strich\emph{Die kleine Komödie}|pw}«}{\lemma{\textnormal{\emph{Schreibe jetzt »Verwandlungen«}}}\Cendnote{\textnormal{Am 1. 2. 1893 nahm Schnitzler die Arbeit an \emph{Verwandlungen}\pwindex{Schnitzler, Arthur 15.\,5.\,1862 Wien – 21.\,10.\,1931 ebd.@\textsc{Schnitzler, Arthur} (15.\,5.\,1862 Wien – 21.\,10.\,1931 ebd.), \emph{Schriftsteller, Mediziner}!kleine Komödie@\strich\emph{Die kleine Komödie}|pwk} wieder auf, was, gemeinsam mit den Datierungen
                  der vorangehenden zwei Korrespondenzstücke, auf die hier geantwortet wird, nach
                  vorne hin beschränkt.}}}\label{K_L00170-3}, Novellette in Briefen, u gehe heut Abend auf die
                  \label{K_L00170-4v}\edtext{Redoute}{\lemma{\textnormal{\emph{Redoute}}}\Cendnote{\textnormal{Finaler Hinweis zur Datierung: Am 1. 2. 1893
                  besuchte Schnitzler die Redoute der Hofoper\oindex{Wien@\textbf{Wien}!I., Innere Stadt@\textbf{I., Innere Stadt}!Oper@\textbf{Oper}, \emph{Oper}|pwk}.}}}\label{K_L00170-4}, weil ich ein Lebemann bin. –
               Ihr herzlich ergebener Arthur, welcher Sie bald zu{ }ſehen und zu hören verlangt. –\pend
           \selectlanguage{ngerman}\endnumbering\briefempfaengerindex{Hofmannsthal, Hugo von@\textsc{Hofmannsthal, Hugo von}!zzzSchnitzler, Arthur@\emph{von Arthur Schnitzler}!1893-02-012@{{[}1. 2. 1893{]}}|)be}\mylabel{L00170h}  \newcommand{\dateiname}{L00170}\newcommand{\titel}{Arthur Schnitzler an Hugo von Hofmannsthal, [1. 2. 1893]}\newcommand{\editorInnen}{Martin Anton Müller und Gerd-Hermann Susen}%% latex-leseansicht-abspann.tex
%% Abspann für die Leseansicht.
%% Der Schalter \ifkorrekturansicht ist bereits durch den Vorspann gesetzt.

%% latex-abspann.tex
%% Gemeinsamer Abspann für Korrekturansicht und Leseansicht.
%% Setzt den Schalter \ifkorrekturansicht voraus (gesetzt in den
%% einbindenden Dateien latex-korrekturansicht-abspann.tex bzw.
%% latex-leseansicht-abspann.tex).
%% ---------------------------------------------------------------

\normalsize

% Das esempio-Environment wird nur in der Leseansicht benötigt
\ifkorrekturansicht\else
\newenvironment{esempio}[3]%
{
    \vspace{1.5ex}
    \rlap{\underline{#1}}
    \par
    \setlength{\parindent}{0cm}
    \nopagebreak
    \leftskip=#2cm
    \rightskip=#3cm
}
{
    \par
}
\fi

\doendnotes{C}
\bigskip
\vfill

\clearpage

\footnotesize

\ifkorrekturansicht
  \lohead{\textsc{register}}
\fi

% theindex-Environment neu definieren ohne reledmac
\makeatletter
\renewenvironment{theindex}{%
  \ifkorrekturansicht
    \section*{\indexname}%
  \else
    \subsubsection*{Index der erwähnten Entitäten}%
  \fi
  \setlength{\parindent}{0pt}%
  \setlength{\parskip}{0pt plus 0.3pt}%
  \let\item\@idxitem
}{%
  \ifkorrekturansicht\clearpage\fi
}
\makeatother

\IfFileExists{\jobname-pw.ind}{\input{\jobname-pw.ind}}{}

% Quellenangabe nur in der Leseansicht
\ifkorrekturansicht\else
% Fallback-Definitionen, falls die .tex-Datei \titel etc. nicht gesetzt hat
\providecommand{\titel}{}
\providecommand{\editorInnen}{}
\providecommand{\dateiname}{\jobname}

\vspace{3cm}

\vfill

\footnotesize
\textsc{Quelle}: \titel. Herausgegeben von {\editorInnen}. In: \emph{Arthur Schnitzler: Briefwechsel mit Autorinnen und Autoren}.
 Digitale Edition, https://schnitzler-briefe.acdh.oeaw.ac.at/{\dateiname}.html (Stand \today)
\fi

\end{document}


