%% latex-korrekturansicht-vorspann.tex
%% Vorspann für die Korrekturansicht.
%% Lädt die gemeinsame Datei latex-vorspann.tex mit gesetztem Schalter.

\newif\ifkorrekturansicht
\korrekturansichttrue

\input{../tex-inputs/latex-vorspann}


\section[Arthur Schnitzler an Hugo von Hofmannsthal, {[}1. 2. 1893{]}]{L00170 Arthur Schnitzler an Hugo von Hofmannsthal, {[}1. 2. 1893{]}}
\nopagebreak\mylabel{L00170v}
\rehead{ }\normalsize\beginnumbering\briefempfaengerindex{Hofmannsthal, Hugo von@\textsc{Hofmannsthal, Hugo von}!zzzSchnitzler, Arthur@\emph{von Arthur Schnitzler}!1893-02-012@{{[}1. 2. 1893{]}}|(be}
\toendnotes[C]{\smallbreak\pagebreak[2]}\Standort{FDH, Hs-30885,33.}
\physDesc{Briefkarte, 1188 Zeichen
\newline{}Handschrift: schwarze Tinte, deutsche Kurrent
\newline{}Ordnung: mit Bleistift von Schnitzler mutmaßlich bei der Durchsicht der Korrespondenz
                                    1929  datiert: »\substVorne{}\textsuperscript{91}\substDazwischen{}Anfang 93\substHinten{}« }
\buchAbdrucke{\weitereDrucke{Hugo von Hofmannsthal, Arthur Schnitzler: \emph{Briefwechsel}. Frankfurt am Main: \emph{S. Fischer} 1964, S. 34.} }\toendnotes[C]{\smallbreak}
\pstart{}{\pb}Mein lieber Hugo,\pend\vspace{0.5em}
\pstart
           Fels\pwindex{Fels, Friedrich Michael *~1864@\textsc{Fels, Friedrich Michael} (*~1864), \emph{Journalist/Journalistin}|pw} befindet ſich bereits beſſer; ernſtere
               Beſorgniſſe ſind nun wohl auszuſchließen. Hingegen wäre nunmehr Ihre ſ. Z.
               beſprochene Liebenswürdigkeit ſehr erwünſcht, u die Idee mit den Freunden ohne
                  Namensne{\geminationn}ung iſt ſehr gut, und raſcher Durchführung
               zu empfehlen. –\pend
           
\pstart
           Die Arbeit Engländers\pwindex{Altenberg, Peter 09.03.1859 – 08.01.1919@\textsc{Altenberg, Peter} (09.03.1859 – 08.01.1919), \emph{Schriftsteller/Schriftstellerin}|pw} iſt über Sölneß\pwindex{Baumeister Solness@\emph{Baumeister Solness}|pw}; Schick\pwindex{Schik, Friedrich *~06.09.1857@\textsc{Schik, Friedrich} (*~06.09.1857), \emph{Journalist/Journalistin, Dramaturg/Dramaturgin}|pw} richtete das Ihnen übermittelte Erſuchen an mich. –\pend
           
\pstart
           Was ſoll ich der akad. Vereinigung\orgindex{Wiener Akademische Vereinigung@Wiener Akademische Vereinigung|pw} ins Exemplar\pwindex{Anatol@\emph{Anatol}|pwv}{ }ſchreiben, ich ke{\geminationn}
               mich da gar nicht aus? – Teltſch\pwindex{Telcs, Ede 1872-05-12 – 1948@\textsc{Telcs, Ede} (1872-05-12 – 1948), \emph{Bildhauer/Bildhauerin}|pw} erhält eins,
                  {\pb}ſobald ich wieder welche von Berlin\orgindex{Bibliographisches Bureau@Bibliographisches Bureau|pw} beko{\geminationm}e, in ein paar Tagen; ich
               grüſs ihn herzlich. – Sah heute im Gewerbemuſeum\oindex{Museum fuer Angewandte Kunst@\textbf{Museum für Angewandte Kunst}, \emph{Museum (K.MUS)}|pw}
               Ihr \label{K_L00170-1v}\edtext{Relief\pwindex{Hugo von Hofmannsthal@\emph{Hugo von Hofmannsthal}|pwv}}{\lemma{\textnormal{\emph{Relief}}}\Cendnote{\textnormal{Das Relief\pwindex{Hugo von Hofmannsthal@\emph{Hugo von Hofmannsthal}|pwkv} befindet sich heute in der Sammlung Richard
                     und Hilda Mises, \emph{Houghton Library}\orgindex{Houghton Library@Houghton Library|pwk},
                     Harvard.}}}\label{K_L00170-1}. Plötzlich lag es da, zwiſchen einem pompej\oindex{Pompeji@\textbf{Pompeji}, \emph{S.ANS}|pw}aniſchen Tiſchfuſs und einem Nürnberg\oindex{Nuernberg@\textbf{Nürnberg}, \emph{P.PPL}|pw}er Hanswurſt. – Ich glaube, es iſt ſehr gut, hab’ aber
               kein gutes Licht gehabt. –\pend
           
\pstart
           \textsc{Salten}\pwindex{Salten, Felix 06.09.1869 – 08.10.1945@\textsc{Salten, Felix} (06.09.1869 – 08.10.1945), \emph{Schriftsteller/Schriftstellerin, Journalist/Journalistin, Chefredakteur/Chefredakteurin}|pw}{ }ſoll Mitte März fort. – \label{K_L00170-2v}\edtext{Familie\pwindex{Familie@\emph{Familie}|pw} beendet}{\lemma{\textnormal{\emph{Familie beendet}}}\Cendnote{\textnormal{Das erlaubt die Datierung des Briefes nach dem 24. 1. 1893, da dieser
                  Tag sowohl im \emph{Tagebuch}\pwindex{Tagebuch@\emph{Tagebuch}|pwk} als auch am Manuskript
                  (vgl. Arthur Schnitzler: \emph{Entworfenes und Verworfenes. Aus dem Nachlaß}. Herausgegeben von Reinhard Urbach. Frankfurt/Main: \emph{S. Fischer}{ }1977, S. 508)
                  als Datum des Abschlusses genannt wird.}}}\label{K_L00170-2}, traue mich nicht \strikeout{zu}{ }ſie durchzuleſen; fürchte mich vor der grauſamen
               Gewißheit. Abſicht: Ende Feber auf 10–14 Tage in die Wärme, von der
               Klinik und dem grauen Leben weg, das Stück\pwindex{Familie@\emph{Familie}|pwv} im Koffer. \label{K_L00170-3v}\edtext{Schreibe jetzt »Verwandlungen\pwindex{kleine Komoedie@\emph{Die kleine Komödie}|pw}«}{\lemma{\textnormal{\emph{Schreibe jetzt »Verwandlungen«}}}\Cendnote{\textnormal{Am 1. 2. 1893 nahm Schnitzler die Arbeit an \emph{Verwandlungen}\pwindex{kleine Komoedie@\emph{Die kleine Komödie}|pwk} wieder auf, was, gemeinsam mit den Datierungen
                  der vorangehenden zwei Korrespondenzstücke, auf die hier geantwortet wird, nach
                  vorne hin beschränkt.}}}\label{K_L00170-3}, Novellette in Briefen, u gehe heut Abend auf die
                  \label{K_L00170-4v}\edtext{Redoute}{\lemma{\textnormal{\emph{Redoute}}}\Cendnote{\textnormal{Finaler Hinweis zur Datierung: Am 1. 2. 1893
                  besuchte Schnitzler die Redoute der Hofoper\oindex{Oper@\textbf{Oper}, \emph{Oper (K.OPR)}|pwk}.}}}\label{K_L00170-4}, weil ich ein Lebemann bin. –
               Ihr herzlich ergebener Arthur, welcher Sie bald zu ſehen und zu hören verlangt. –\pend
           \selectlanguage{ngerman}\endnumbering\briefempfaengerindex{Hofmannsthal, Hugo von@\textsc{Hofmannsthal, Hugo von}!zzzSchnitzler, Arthur@\emph{von Arthur Schnitzler}!1893-02-012@{{[}1. 2. 1893{]}}|)be}\mylabel{L00170h}  \normalsize

\doendnotes{C}
\bigskip
\vfill

\clearpage

\footnotesize

\lohead{\textsc{register}}

% Definiere theindex-Environment komplett neu ohne reledmac
\makeatletter
\renewenvironment{theindex}{%
  \section*{\indexname}%
  \setlength{\parindent}{0pt}%
  \setlength{\parskip}{0pt plus 0.3pt}%
  \let\item\@idxitem
}{%
  \clearpage
}
\makeatother

\IfFileExists{\jobname-pw.ind}{\input{\jobname-pw.ind}}{}

\end{document}

      