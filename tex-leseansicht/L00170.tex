%% latex-leseansicht-vorspann.tex
%% Vorspann für die Leseansicht.
%% Lädt die gemeinsame Datei latex-vorspann.tex mit nicht gesetztem Schalter.

\newif\ifkorrekturansicht
\korrekturansichtfalse

\input{../tex-inputs/latex-vorspann}


         
         \renewcommand{\erwaehntePersonen}{Personen: Peter Altenberg, Friedrich Michael Fels, Hugo von Hofmannsthal, Felix Salten, Friedrich Schik, Ede Telcs}
         \renewcommand{\erwaehnteInstitutionen}{Institutionen: Bibliographisches Bureau, Houghton Library, Wiener Akademische Vereinigung}
         \renewcommand{\erwaehnteOrte}{Orte: Nürnberg, Oper, Pompei, Wien, Österreichisches Museum für Kunst und Industrie}
         \renewcommand{\erwaehnteWerke}{Werke: Anatol, Baumeister Solness, Die kleine Komödie, Familie, Hugo von Hofmannsthal, Tagebuch}
               \section[Arthur Schnitzler an Hugo von Hofmannsthal, {[}1. 2. 1893{]}]{ Arthur Schnitzler an Hugo von Hofmannsthal, {[}1. 2. 1893{]}}\nopagebreak\mylabel{v}\rehead{ }\begin{ledgroupsized}[t]{13cm}\normalsize\beginnumbering \toendnotes[C]{\smallbreak\pagebreak[2]} \Standort{FDH, Hs-30885,33.}
\physDesc{Briefkarte, 1189 Zeichen
\newline{}Handschrift: schwarze Tinte, deutsche Kurrent
\newline{}Ordnung: mit Bleistift von Schnitzler mutmaßlich bei der Durchsicht der Korrespondenz
                                    1929  datiert: »\substVorne{}\textsuperscript{91}\substDazwischen{}Anfang 93\substHinten{}« }\buchAbdrucke{\weitereDrucke{Hugo von Hofmannsthal, Arthur Schnitzler: \emph{Briefwechsel}. Hg. Therese Nickl und Heinrich Schnitzler. Frankfurt am Main: \emph{S. Fischer} 1964, S. 34.} }\toendnotes[C]{\smallbreak}\pstart{}{\pb}Mein lieber Hugo,\pend\pstart
           Fels\pwindex{Fels, Friedrich Michael *~1864@\textsc{Fels, Friedrich Michael} (*~1864), \emph{Journalist}|pw} befindet ſich bereits beſſer; ernſtere
               Beſorgniſſe ſind nun wohl auszuſchließen. Hingegen wäre nunmehr Ihre ſ. Z.
               beſprochene Liebenswürdigkeit ſehr erwünſcht, u die Idee mit den Freunden ohne
                  Namensne{\geminationn}ung iſt ſehr gut, und raſcher Durchführung
               zu empfehlen. –\pend
           \pstart
           Die Arbeit Engländer\pwindex{Altenberg, Peter 09.03.1859 – 08.01.1919@\textsc{Altenberg, Peter} (09.03.1859 – 08.01.1919), \emph{Schriftsteller}|pw}s iſt über Sölneß\pwindex{\textcolor{red}{\textsuperscript{XXXX1 indx}}!Baumeister Solness1892@\strich\emph{Baumeister Solness} {[}1892{]}|pw}; Schick\pwindex{Schik, Friedrich *~06.09.1857@\textsc{Schik, Friedrich} (*~06.09.1857), \emph{Journalist, Dramaturg}|pw} richtete das Ihnen übermittelte Erſuchen an mich. –\pend
           \pstart
           Was ſoll ich der akad. Vereinigung\orgindex{Wiener Akademische Vereinigung@Wiener Akademische Vereinigung|pw} ins Exemplar\pwindex{Schnitzler, Arthur 15.05.1862 – 21.10.1931@\textsc{Schnitzler, Arthur} (15.05.1862 – 21.10.1931), \emph{Schriftsteller, Mediziner}!Anatol1892-10-29@\strich\emph{Anatol} {[}1892-10-29{]}|pwv}{ }ſchreiben, ich ke{\geminationn}
               mich da gar nicht aus? – Teltſch\pwindex{Telcs, Ede 1872-05-12 – 1948@\textsc{Telcs, Ede} (1872-05-12 – 1948), \emph{Bildender Künstler}|pw} erhält eins,
                  {\pb}ſobald ich wieder welche von Berlin\orgindex{Bibliographisches Bureau@Bibliographisches Bureau|pw} beko{\geminationm}e, in ein paar Tagen; ich
               grüſs ihn herzlich. – Sah heute im Gewerbemuſeum\oindex{Oesterreichisches Museum fuer Kunst und Industrie@\textbf{Österreichisches Museum für Kunst und Industrie}|pw}
               Ihr \label{K_L00170_1v}\edtext{Relief\pwindex{Telcs, Ede 1872-05-12 – 1948@\textsc{Telcs, Ede} (1872-05-12 – 1948), \emph{Bildender Künstler}!Hugo von Hofmannsthal1892@\strich\emph{Hugo von Hofmannsthal} {[}1892{]}|pwv}}{\lemma{\textnormal{\emph{Relief}}}\Cendnote{\textnormal{Das Relief\pwindex{Telcs, Ede 1872-05-12 – 1948@\textsc{Telcs, Ede} (1872-05-12 – 1948), \emph{Bildender Künstler}!Hugo von Hofmannsthal1892@\strich\emph{Hugo von Hofmannsthal} {[}1892{]}|pwkv} befindet sich heute in der Sammlung Richard
                     und Hilda Mises, \emph{Houghton Library}\orgindex{Houghton Library@Houghton Library|pwk},
                     Harvard.}}}\label{K_L00170_1h}. Plötzlich lag es da, zwiſchen einem pompej\oindex{Pompei@\textbf{Pompei}|pw}aniſchen Tiſchfuſs und einem Nürnberg\oindex{Nuernberg@\textbf{Nürnberg}|pw}er Hanswurſt. – Ich glaube, es iſt ſehr gut, hab’ aber
               kein gutes Licht gehabt. –\pend
           \pstart
           \textsc{Salten}\pwindex{Salten, Felix 06.09.1869 – 08.10.1945@\textsc{Salten, Felix} (06.09.1869 – 08.10.1945), \emph{Schriftsteller, Journalist}|pw}{ }ſoll Mitte März fort. – \label{K_L00170_2v}\edtext{Familie\pwindex{Schnitzler, Arthur 15.05.1862 – 21.10.1931@\textsc{Schnitzler, Arthur} (15.05.1862 – 21.10.1931), \emph{Schriftsteller, Mediziner}!Familie1977@\strich\emph{Familie} {[}1977{]}|pw} beendet}{\lemma{\textnormal{\emph{Familie beendet}}}\Cendnote{\textnormal{Das erlaubt die Datierung des Briefes nach dem 24. 1. 1893, da dieser
                  Tag sowohl im \emph{Tagebuch}\pwindex{Schnitzler, Arthur 15.05.1862 – 21.10.1931@\textsc{Schnitzler, Arthur} (15.05.1862 – 21.10.1931), \emph{Schriftsteller, Mediziner}!Tagebuch1981 – 2000@\strich\emph{Tagebuch} {[}1981 – 2000{]}|pwk} als auch am Manuskript
                     (vgl. \emph{Entworfenes und Verworfenes} 508)
                  als Datum des Abschlusses genannt wird.}}}\label{K_L00170_2h}, traue mich nicht \strikeout{zu}{ }ſie durchzuleſen; fürchte mich vor der grauſamen
               Gewißheit. Abſicht: Ende Feber auf 10–14 Tage in die Wärme, von der
               Klinik und dem grauen Leben weg, das Stück\pwindex{Schnitzler, Arthur 15.05.1862 – 21.10.1931@\textsc{Schnitzler, Arthur} (15.05.1862 – 21.10.1931), \emph{Schriftsteller, Mediziner}!Familie1977@\strich\emph{Familie} {[}1977{]}|pwv} im Koffer. \label{K_L00170_3v}\edtext{Schreibe jetzt »Verwandlungen\pwindex{Schnitzler, Arthur 15.05.1862 – 21.10.1931@\textsc{Schnitzler, Arthur} (15.05.1862 – 21.10.1931), \emph{Schriftsteller, Mediziner}!kleine Komoedie1895-08-01@\strich\emph{Die kleine Komödie} {[}1895-08-01{]}|pw}«}{\lemma{\textnormal{\emph{Schreibe jetzt »Verwandlungen«}}}\Cendnote{\textnormal{Am 1. 2. 1893 nahm Schnitzler\pwindex{Schnitzler, Arthur 15.05.1862 – 21.10.1931@\textsc{Schnitzler, Arthur} (15.05.1862 – 21.10.1931), \emph{Schriftsteller, Mediziner}|pwk} die Arbeit an \emph{Verwandlungen}\pwindex{Schnitzler, Arthur 15.05.1862 – 21.10.1931@\textsc{Schnitzler, Arthur} (15.05.1862 – 21.10.1931), \emph{Schriftsteller, Mediziner}!kleine Komoedie1895-08-01@\strich\emph{Die kleine Komödie} {[}1895-08-01{]}|pwk} wieder auf, was, gemeinsam mit den Datierungen
                  der vorangehenden zwei Korrespondenzstücke, auf die hier geantwortet wird, nach
                  vorne hin beschränkt.}}}\label{K_L00170_3h}, Novellette in Briefen, u gehe heut Abend auf die
                  \label{K_L00170_4v}\edtext{Redoute}{\lemma{\textnormal{\emph{Redoute}}}\Cendnote{\textnormal{Finaler Hinweis zur Datierung: Am 1. 2. 1893
                  besuchte Schnitzler\pwindex{Schnitzler, Arthur 15.05.1862 – 21.10.1931@\textsc{Schnitzler, Arthur} (15.05.1862 – 21.10.1931), \emph{Schriftsteller, Mediziner}|pwk} die Redoute der Hofoper\oindex{Oper@\textbf{Oper}|pwk}.}}}\label{K_L00170_4h}, weil ich ein Lebemann bin. –
               Ihr herzlich ergebener Arthur, welcher Sie bald zu ſehen und zu hören verlangt. –\pend
           
         
         \endnumbering\mylabel{h}\end{ledgroupsized}  \newcommand{\dateiname}{L00170}\newcommand{\titel}{Arthur Schnitzler an Hugo von Hofmannsthal, [1. 2. 1893]}\newcommand{\editorInnen}{Martin Anton Müller und Gerd-Hermann Susen}%% latex-leseansicht-abspann.tex
%% Abspann für die Leseansicht.
%% Der Schalter \ifkorrekturansicht ist bereits durch den Vorspann gesetzt.

%% latex-abspann.tex
%% Gemeinsamer Abspann für Korrekturansicht und Leseansicht.
%% Setzt den Schalter \ifkorrekturansicht voraus (gesetzt in den
%% einbindenden Dateien latex-korrekturansicht-abspann.tex bzw.
%% latex-leseansicht-abspann.tex).
%% ---------------------------------------------------------------

\normalsize

% Das esempio-Environment wird nur in der Leseansicht benötigt
\ifkorrekturansicht\else
\newenvironment{esempio}[3]%
{
    \vspace{1.5ex}
    \rlap{\underline{#1}}
    \par
    \setlength{\parindent}{0cm}
    \nopagebreak
    \leftskip=#2cm
    \rightskip=#3cm
}
{
    \par
}
\fi

\doendnotes{C}
\bigskip
\vfill

\clearpage

\footnotesize

\ifkorrekturansicht
  \lohead{\textsc{register}}
\fi

% theindex-Environment neu definieren ohne reledmac
\makeatletter
\renewenvironment{theindex}{%
  \ifkorrekturansicht
    \section*{\indexname}%
  \else
    \subsubsection*{Index der erwähnten Entitäten}%
  \fi
  \setlength{\parindent}{0pt}%
  \setlength{\parskip}{0pt plus 0.3pt}%
  \let\item\@idxitem
}{%
  \ifkorrekturansicht\clearpage\fi
}
\makeatother

\IfFileExists{\jobname-pw.ind}{\input{\jobname-pw.ind}}{}

% Quellenangabe nur in der Leseansicht
\ifkorrekturansicht\else
% Fallback-Definitionen, falls die .tex-Datei \titel etc. nicht gesetzt hat
\providecommand{\titel}{}
\providecommand{\editorInnen}{}
\providecommand{\dateiname}{\jobname}

\vspace{3cm}

\vfill

\footnotesize
\textsc{Quelle}: \titel. Herausgegeben von {\editorInnen}. In: \emph{Arthur Schnitzler: Briefwechsel mit Autorinnen und Autoren}.
 Digitale Edition, https://schnitzler-briefe.acdh.oeaw.ac.at/{\dateiname}.html (Stand \today)
\fi

\end{document}


      