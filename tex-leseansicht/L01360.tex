%% latex-leseansicht-vorspann.tex
%% Vorspann für die Leseansicht.
%% Lädt die gemeinsame Datei latex-vorspann.tex mit nicht gesetztem Schalter.

\newif\ifkorrekturansicht
\korrekturansichtfalse

\input{../tex-inputs/latex-vorspann}


\section[Arthur Schnitzler an Franz Blei, 8. 1. 1904]{L01360 Arthur Schnitzler an Franz Blei, 8. 1. 1904}
\nopagebreak\mylabel{L01360v}
\rehead{ }\normalsize\beginnumbering\briefempfaengerindex{Blei, Franz@\textsc{Blei, Franz}!zzzSchnitzler, Arthur@\emph{von Arthur Schnitzler}!1904-01-082@{8. 1. 1904}|(be}
\toendnotes[C]{\smallbreak\pagebreak[2]}
\correspDesc{Versand  durch Arthur Schnitzler am 8. 1. 1904 in Wien
\newline{}Erhalt  durch Franz Blei im Zeitraum [8. 1. 1904
                  – 12. 1. 1904?] \textbf{Ort fehlend} }\toendnotes[C]{\smallbreak}
\Standort{DLA, A:Schnitzler, HS.NZ85.1.403.}
\physDesc{Brief, Durchschlag, 1 Blatt, 1 Seite, 666 Zeichen
\newline{}Schreibmaschine
\newline{}Handschrift: roter Buntstift, lateinische Kurrent (\noindent{}»Fr Blei« und vier Unterstreichungen)
\newline{}Editorischer Hinweis: Die Zeichen des Textverlusts werden stillschweigend ergänzt,
                                 sofern sie inhaltlich verlässlich zu erschließen sind. }\toendnotes[C]{\smallbreak}
\pstart
           \raggedleft{}{\pb}Wien\oindex{Wien@\textbf{Wien}, \emph{Verwaltungsgebiet}|pw}, 8. Januar 1904.{\\}XVIII. Spöttelg. 7\oindex{Wien@\textbf{Wien}!XVIII., Währing@\textbf{XVIII., Währing}!Edmund-Weiß-Gasse 7@\textbf{Edmund-Weiß-Gasse 7}, \emph{Wohngebäude}|pw}.\pend
           
\pstart{}Sehr geehrter Herr Blei!\pend\vspace{0.5em}
\pstart
           Für Ihre freundlichen Nachrichten danke ich sehr. Könnte ich nicht wissen, warum mein
                  englischer\oindex{England@\textbf{England}, \emph{Land}|pw}{ }Verleger\pwindex{Bates, Alfred @\textsc{Bates, Alfred}, \emph{Herausgeber}|pwv} »\begin{otherlanguage}{english}distinctly shady\end{otherlanguage}« sein soll? Jedenfalls habe ich bis
                  1. Juli 1906 in Hinsicht auf den »Kakadu\pwindex{Schnitzler, Arthur 15.\,5.\,1862 Wien – 21.\,10.\,1931 ebd.@\textsc{Schnitzler, Arthur} (15.\,5.\,1862 Wien – 21.\,10.\,1931 ebd.), \emph{Schriftsteller, Mediziner}!grüne Kakadu. Groteske in einem Akt@\strich\emph{Der grüne Kakadu. Groteske in einem Akt}|pw}« Vertrag, der mich bindet.\pend
           
\pstart
           In Betreff eventuellen Verlags meiner Novellen bei Heinemann\orgindex{William Heinemann Ltd@William Heinemann Ltd|pw} erwarte ich gern präzisere Anträge.\pend
           
\pstart
           Dass ich das Honorar von den Scharfrichtern\orgindex{elf Scharfrichter@Die elf Scharfrichter|pw} noch
               immer nicht bekommen habe, kann ich Ihnen bei dem besten Willen nicht verhehlen.\pend
           
\pstart
           Mir hat es recht leid getan, Sie in Wien\oindex{Wien@\textbf{Wien}, \emph{Verwaltungsgebiet}|pw} nicht
               gesehen \damage{zu} haben; bei den \label{K_L01360-1v}\edtext{Scharfrichtern\orgindex{elf Scharfrichter@Die elf Scharfrichter|pw} im Savoy\oindex{Wien@\textbf{Wien}!VII., Neubau@\textbf{VII., Neubau}!Hotel Savoy [Wien]@\textbf{Hotel Savoy [Wien]}, \emph{Hotel}|pw}}{\lemma{\textnormal{\emph{Scharfrichtern im Savoy}}}\Cendnote{\textnormal{Schnitzler hatte mit seiner Gattin
                  Olga\pwindex{Schnitzler, Olga 17.\,1.\,1882 Wien – 13.\,1.\,1970 Lugano@\textsc{Schnitzler, Olga} (17.\,1.\,1882 Wien – 13.\,1.\,1970 Lugano), \emph{Schauspielerin, Sängerin}|pwk} 
                  am 10. 12. 1903 den Auftritt der \emph{Elf Scharfrichter}\orgindex{elf Scharfrichter@Die elf Scharfrichter|pwk}
                  im Hotel Savoy\oindex{Wien@\textbf{Wien}!VII., Neubau@\textbf{VII., Neubau}!Hotel Savoy [Wien]@\textbf{Hotel Savoy [Wien]}, \emph{Hotel}|pwk} besucht.
                  }}}\label{K_L01360-1} hat
               es mir sehr behagt.\pend
           
\pstart
           Mit verbindlichem Gruss{\\[\baselineskip]} Ihr aufrichtig ergebener\pend
           \leftskip=0em{}\selectlanguage{ngerman}\endnumbering\briefempfaengerindex{Blei, Franz@\textsc{Blei, Franz}!zzzSchnitzler, Arthur@\emph{von Arthur Schnitzler}!1904-01-082@{8. 1. 1904}|)be}\mylabel{L01360h}  \newcommand{\dateiname}{L01360}\newcommand{\titel}{Arthur Schnitzler an Franz Blei, 8. 1. 1904}\newcommand{\editorInnen}{Martin Anton Müller und Gerd-Hermann Susen}%% latex-leseansicht-abspann.tex
%% Abspann für die Leseansicht.
%% Der Schalter \ifkorrekturansicht ist bereits durch den Vorspann gesetzt.

%% latex-abspann.tex
%% Gemeinsamer Abspann für Korrekturansicht und Leseansicht.
%% Setzt den Schalter \ifkorrekturansicht voraus (gesetzt in den
%% einbindenden Dateien latex-korrekturansicht-abspann.tex bzw.
%% latex-leseansicht-abspann.tex).
%% ---------------------------------------------------------------

\normalsize

% Das esempio-Environment wird nur in der Leseansicht benötigt
\ifkorrekturansicht\else
\newenvironment{esempio}[3]%
{
    \vspace{1.5ex}
    \rlap{\underline{#1}}
    \par
    \setlength{\parindent}{0cm}
    \nopagebreak
    \leftskip=#2cm
    \rightskip=#3cm
}
{
    \par
}
\fi

\doendnotes{C}
\bigskip
\vfill

\clearpage

\footnotesize

\ifkorrekturansicht
  \lohead{\textsc{register}}
\fi

% theindex-Environment neu definieren ohne reledmac
\makeatletter
\renewenvironment{theindex}{%
  \ifkorrekturansicht
    \section*{\indexname}%
  \else
    \subsubsection*{Index der erwähnten Entitäten}%
  \fi
  \setlength{\parindent}{0pt}%
  \setlength{\parskip}{0pt plus 0.3pt}%
  \let\item\@idxitem
}{%
  \ifkorrekturansicht\clearpage\fi
}
\makeatother

\IfFileExists{\jobname-pw.ind}{\input{\jobname-pw.ind}}{}

% Quellenangabe nur in der Leseansicht
\ifkorrekturansicht\else
% Fallback-Definitionen, falls die .tex-Datei \titel etc. nicht gesetzt hat
\providecommand{\titel}{}
\providecommand{\editorInnen}{}
\providecommand{\dateiname}{\jobname}

\vspace{3cm}

\vfill

\footnotesize
\textsc{Quelle}: \titel. Herausgegeben von {\editorInnen}. In: \emph{Arthur Schnitzler: Briefwechsel mit Autorinnen und Autoren}.
 Digitale Edition, https://schnitzler-briefe.acdh.oeaw.ac.at/{\dateiname}.html (Stand \today)
\fi

\end{document}


