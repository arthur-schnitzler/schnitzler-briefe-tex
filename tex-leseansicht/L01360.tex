%% latex-korrekturansicht-vorspann.tex
%% Vorspann für die Korrekturansicht.
%% Lädt die gemeinsame Datei latex-vorspann.tex mit gesetztem Schalter.

\newif\ifkorrekturansicht
\korrekturansichttrue

\input{../tex-inputs/latex-vorspann}


\section[Arthur Schnitzler an Franz Blei, 8. 1. 1904]{L01360 Arthur Schnitzler an Franz Blei, 8. 1. 1904}
\nopagebreak\mylabel{L01360v}
\rehead{ }\normalsize\beginnumbering\briefempfaengerindex{Blei, Franz@\textsc{Blei, Franz}!zzzSchnitzler, Arthur@\emph{von Arthur Schnitzler}!1904-01-082@{8. 1. 1904}|(be}
\toendnotes[C]{\smallbreak\pagebreak[2]}\Standort{DLA, A:Schnitzler, HS.NZ85.1.403.}
\physDesc{Brief, Durchschlag1 Blatt, 1 Seite, 666 Zeichen
\newline{}Schreibmaschine
\newline{}Handschrift: roter Buntstift, lateinische Kurrent (\noindent{}»Fr Blei« und vier Unterstreichungen)
\newline{}Editorischer Hinweis: Die Zeichen des Textverlusts werden stillschweigend ergänzt,
                                 sofern sie inhaltlich verlässlich zu erschließen sind. }\toendnotes[C]{\smallbreak}
\pstart
           \raggedleft{}{\pb}Wien\oindex{Wien@\textbf{Wien}, \emph{A.ADM2}|pw}, 8. Januar 1904.{\\}XVIII. Spöttelg. 7\oindex{Edmund-Weiss-Gasse 7@\textbf{Edmund-Weiß-Gasse 7}, \emph{Wohngebäude (K.WHS)}|pw}.\pend
           
\pstart{}Sehr geehrter Herr Blei!\pend\vspace{0.5em}
\pstart
           Für Ihre freundlichen Nachrichten danke ich sehr. Könnte ich nicht wissen, warum mein
                  englischer\oindex{England@\textbf{England}, \emph{A.ADM1}|pw}{ }Verleger\pwindex{Bates, Alfred @\textsc{Bates, Alfred}, \emph{Herausgeber/Herausgeberin}|pwv} »\begin{otherlanguage}{english}distinctly shady\end{otherlanguage}« sein soll? Jedenfalls habe ich bis
                  1. Juli 1906 in Hinsicht auf den »Kakadu\pwindex{gruene Kakadu. Groteske in einem Akt@\emph{Der grüne Kakadu. Groteske in einem Akt}|pw}« Vertrag, der mich bindet.\pend
           
\pstart
           In Betreff eventuellen Verlags meiner Novellen bei Heinemann\orgindex{William Heinemann Ltd@William Heinemann Ltd|pw} erwarte ich gern präzisere Anträge.\pend
           
\pstart
           Dass ich das Honorar von den Scharfrichtern\orgindex{elf Scharfrichter@Die elf Scharfrichter|pw} noch
               immer nicht bekommen habe, kann ich Ihnen bei dem besten Willen nicht verhehlen.\pend
           
\pstart
           Mir hat es recht leid getan, Sie in Wien\oindex{Wien@\textbf{Wien}, \emph{A.ADM2}|pw} nicht
               gesehen \damage{zu} haben; bei den \label{K_L01360-1v}\edtext{Scharfrichtern\orgindex{elf Scharfrichter@Die elf Scharfrichter|pw} im Savoy\oindex{Hotel Savoy [Wien]@\textbf{Hotel Savoy [Wien]}, \emph{Hotel (K.HTL)}|pw}}{\lemma{\textnormal{\emph{Scharfrichtern im Savoy}}}\Cendnote{\textnormal{Schnitzler hatte mit seiner Gattin
                  Olga\pwindex{Schnitzler, Olga 17.01.1882 – 13.01.1970@\textsc{Schnitzler, Olga} (17.01.1882 – 13.01.1970), \emph{Schauspieler/Schauspielerin, Sänger/Sängerin}|pwk} 
                  am 10. 12. 1903 den Auftritt der \emph{Elf Scharfrichter}\orgindex{elf Scharfrichter@Die elf Scharfrichter|pwk}
                  im Hotel Savoy\oindex{Hotel Savoy [Wien]@\textbf{Hotel Savoy [Wien]}, \emph{Hotel (K.HTL)}|pwk} besucht.
                  }}}\label{K_L01360-1} hat
               es mir sehr behagt.\pend
           
\pstart
           Mit verbindlichem Gruss{\\[\baselineskip]} Ihr aufrichtig ergebener\pend
           \leftskip=0em{}\selectlanguage{ngerman}\endnumbering\briefempfaengerindex{Blei, Franz@\textsc{Blei, Franz}!zzzSchnitzler, Arthur@\emph{von Arthur Schnitzler}!1904-01-082@{8. 1. 1904}|)be}\mylabel{L01360h}  \normalsize

\doendnotes{C}
\bigskip
\vfill

\clearpage

\footnotesize

\lohead{\textsc{register}}

% Definiere theindex-Environment komplett neu ohne reledmac
\makeatletter
\renewenvironment{theindex}{%
  \section*{\indexname}%
  \setlength{\parindent}{0pt}%
  \setlength{\parskip}{0pt plus 0.3pt}%
  \let\item\@idxitem
}{%
  \clearpage
}
\makeatother

\IfFileExists{\jobname-pw.ind}{\input{\jobname-pw.ind}}{}

\end{document}

      