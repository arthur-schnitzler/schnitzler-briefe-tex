%% latex-leseansicht-vorspann.tex
%% Vorspann für die Leseansicht.
%% Lädt die gemeinsame Datei latex-vorspann.tex mit nicht gesetztem Schalter.

\newif\ifkorrekturansicht
\korrekturansichtfalse

\input{../tex-inputs/latex-vorspann}


               \section[Hermann Bahr an Arthur Schnitzler, {[}18.?{]} 8. 1905]{ Hermann Bahr an Arthur Schnitzler, {[}18.?{]} 8. 1905}\nopagebreak\mylabel{v}\rehead{ }\begin{ledgroupsized}[t]{13cm}\normalsize\beginnumbering\briefempfaengerindex{Schnitzler, Arthur@\textsc{Schnitzler, Arthur}!zzzBahr, Hermann@\emph{von Hermann Bahr}!1905-08-181@{{[}18.?{]} 8. 1905}|(be} \toendnotes[C]{\smallbreak\pagebreak[2]} \Standort{CUL, Schnitzler, B 5b.}
\physDesc{Bildpostkarte
\newline{}Handschrift: Bleistift, deutsche Kurrent\newline{}Versand: Stempel: »\nobreak{}\oindex{Salzburg@\textbf{Salzburg}|pwk}Salzburg, \textcolor{gray}{18}. VIII. 05\nobreak{}«.  \newline{}Ordnung: mit Bleistift von unbekannter Hand nummeriert:
                                    »130« }\buchAbdrucke{\weitereDrucke{Hermann Bahr, Arthur Schnitzler: \emph{Briefwechsel, Aufzeichnungen, Dokumente (1891–1931)}. Hg. Kurt Ifkovits und Martin Anton Müller. Göttingen: \emph{Wallstein} 2018, S. 350.} }\toendnotes[C]{\smallbreak}\pstart{}{\pb}\textsc{Artur Schnitzler}\pend{}\pstart{}\textsc{Wien XVIII}\oindex{XVIII., Waehring@\textbf{XVIII., Währing}|pw}\pend{}\pstart{}\textsc{Spöttelgasse 7}\oindex{Edmund-Weiss-Gasse@\textbf{Edmund-Weiß-Gasse}|pw}\pend{}{\bigskip}\pstart
           \noindent{}\centering{}\textcolor{gray}{\textbf{{\pb}Das Weiser\pwindex{Weiser, Ignatz Anton von 1701-03-01 – 1785-12-26@\textsc{Weiser, Ignatz Anton von} (1701-03-01 – 1785-12-26), \emph{Politiker}|pwv}haus am alten Markt in Salzburg\oindex{Salzburg@\textbf{Salzburg}|pw} um 1800 (jetzt Salzburger Sparkasse,
                        Ludwig-Viktorplatz\oindex{Alter Markt@\textbf{Alter Markt}|pw})}}\pend
           \pstart
           \label{K_L01543_1v}\edtext{Vanjung\pwindex{Van-Jung, Leo 15.10.1866 – 02.07.1939@\textsc{Van-Jung, Leo} (15.10.1866 – 02.07.1939), \emph{Gesangspädagoge, Mathematiker}|pw} erzält mir eben}{\lemma{\textnormal{\emph{Vanjung erzält mir eben}}}\Cendnote{\textnormal{Am 12. 8. 1905 hatte Van
                     Jung\pwindex{Van-Jung, Leo 15.10.1866 – 02.07.1939@\textsc{Van-Jung, Leo} (15.10.1866 – 02.07.1939), \emph{Gesangspädagoge, Mathematiker}|pwk} bei Schnitzler\pwindex{Schnitzler, Arthur 15.05.1862 – 21.10.1931@\textsc{Schnitzler, Arthur} (15.05.1862 – 21.10.1931), \emph{Schriftsteller, Mediziner}|pwk}{ }\emph{Zwischenspiel}\pwindex{Schnitzler, Arthur 15.05.1862 – 21.10.1931@\textsc{Schnitzler, Arthur} (15.05.1862 – 21.10.1931), \emph{Schriftsteller, Mediziner}!Zwischenspiel. Komoedie in drei Akten1905-10-12 – 1905-10-12@\strich\emph{Zwischenspiel. Komödie in drei Akten} {[}1905-10-12 – 1905-10-12{]}|pwk} und \emph{Ruf des Lebens}\pwindex{Schnitzler, Arthur 15.05.1862 – 21.10.1931@\textsc{Schnitzler, Arthur} (15.05.1862 – 21.10.1931), \emph{Schriftsteller, Mediziner}!Ruf des Lebens. Schauspiel in drei Akten1906-02-20@\strich\emph{Der Ruf des Lebens. Schauspiel in drei Akten} {[}1906-02-20{]}|pwk} vorgelesen bekommen. Vom 18. bis zum
                     20. 8. 1905 war Bahr\pwindex{Bahr, Hermann 19.07.1863 – 15.01.1934@\textsc{Bahr, Hermann} (19.07.1863 – 15.01.1934), \emph{Schriftsteller, Kritiker}|pwk} in Salzburg\oindex{Salzburg@\textbf{Salzburg}|pwk} (Bahr: \emph{Tagebücher, Skizzenhefte, Notizbücher}
                     IV,424).}}}\label{K_L01543_1h} von Deinen beiden Stücken\pwindex{Schnitzler, Arthur 15.05.1862 – 21.10.1931@\textsc{Schnitzler, Arthur} (15.05.1862 – 21.10.1931), \emph{Schriftsteller, Mediziner}!Zwischenspiel. Komoedie in drei Akten1905-10-12 – 1905-10-12@\strich\emph{Zwischenspiel. Komödie in drei Akten} {[}1905-10-12 – 1905-10-12{]}|pwv}\pwindex{Schnitzler, Arthur 15.05.1862 – 21.10.1931@\textsc{Schnitzler, Arthur} (15.05.1862 – 21.10.1931), \emph{Schriftsteller, Mediziner}!Ruf des Lebens. Schauspiel in drei Akten1906-02-20@\strich\emph{Der Ruf des Lebens. Schauspiel in drei Akten} {[}1906-02-20{]}|pwv}, ich freu mich ſehr und bin ungeheuer
                  neugierig.\hspace*{1.5em}Herzlichſt \spacefill\mbox{Hermann}\pend
           \endnumbering\briefempfaengerindex{Schnitzler, Arthur@\textsc{Schnitzler, Arthur}!zzzBahr, Hermann@\emph{von Hermann Bahr}!1905-08-181@{{[}18.?{]} 8. 1905}|)be}\mylabel{h}\end{ledgroupsized}  \newcommand{\dateiname}{L01543}\newcommand{\titel}{Hermann Bahr an Arthur Schnitzler, [18.?] 8. 1905}\newcommand{\editorInnen}{ Kurt Ifkovits,  Martin Anton Müller}
            \footnotesize
\begin{ledgroupsized}[t]{11.5cm}
\doendnotes{C}
\end{ledgroupsized}
         %% latex-leseansicht-abspann.tex
%% Abspann für die Leseansicht.
%% Der Schalter \ifkorrekturansicht ist bereits durch den Vorspann gesetzt.

%% latex-abspann.tex
%% Gemeinsamer Abspann für Korrekturansicht und Leseansicht.
%% Setzt den Schalter \ifkorrekturansicht voraus (gesetzt in den
%% einbindenden Dateien latex-korrekturansicht-abspann.tex bzw.
%% latex-leseansicht-abspann.tex).
%% ---------------------------------------------------------------

\normalsize

% Das esempio-Environment wird nur in der Leseansicht benötigt
\ifkorrekturansicht\else
\newenvironment{esempio}[3]%
{
    \vspace{1.5ex}
    \rlap{\underline{#1}}
    \par
    \setlength{\parindent}{0cm}
    \nopagebreak
    \leftskip=#2cm
    \rightskip=#3cm
}
{
    \par
}
\fi

\doendnotes{C}
\bigskip
\vfill

\clearpage

\footnotesize

\ifkorrekturansicht
  \lohead{\textsc{register}}
\fi

% theindex-Environment neu definieren ohne reledmac
\makeatletter
\renewenvironment{theindex}{%
  \ifkorrekturansicht
    \section*{\indexname}%
  \else
    \subsubsection*{Index der erwähnten Entitäten}%
  \fi
  \setlength{\parindent}{0pt}%
  \setlength{\parskip}{0pt plus 0.3pt}%
  \let\item\@idxitem
}{%
  \ifkorrekturansicht\clearpage\fi
}
\makeatother

\IfFileExists{\jobname-pw.ind}{\input{\jobname-pw.ind}}{}

% Quellenangabe nur in der Leseansicht
\ifkorrekturansicht\else
% Fallback-Definitionen, falls die .tex-Datei \titel etc. nicht gesetzt hat
\providecommand{\titel}{}
\providecommand{\editorInnen}{}
\providecommand{\dateiname}{\jobname}

\vspace{3cm}

\vfill

\footnotesize
\textsc{Quelle}: \titel. Herausgegeben von {\editorInnen}. In: \emph{Arthur Schnitzler: Briefwechsel mit Autorinnen und Autoren}.
 Digitale Edition, https://schnitzler-briefe.acdh.oeaw.ac.at/{\dateiname}.html (Stand \today)
\fi

\end{document}


      