%% latex-leseansicht-vorspann.tex
%% Vorspann für die Leseansicht.
%% Lädt die gemeinsame Datei latex-vorspann.tex mit nicht gesetztem Schalter.

\newif\ifkorrekturansicht
\korrekturansichtfalse

\input{../tex-inputs/latex-vorspann}


         
         \renewcommand{\erwaehntePersonen}{Personen: Rolf von Brockdorff, Eugen Gura,  Paulus, Felix Salten, Rudolf Strauss}
         \renewcommand{\erwaehnteInstitutionen}{Institutionen: Wiener Allgemeine Zeitung}
         \renewcommand{\erwaehnteOrte}{Orte: Ronacher, Wien}
         \renewcommand{\erwaehnteWerke}{Werke: Liebelei. Eine Wiener Zeitschrift, Wiener Allgemeine Zeitung}
               \section[ Felix Salten an Arthur Schnitzler, {[}12?. 12. 1895{]}]{ Felix Salten an Arthur Schnitzler, {[}12?. 12. 1895{]}}\nopagebreak\mylabel{v}\rehead{ }\begin{ledgroupsized}[t]{13cm}\normalsize\beginnumbering\briefempfaengerindex{Schnitzler, Arthur@\textsc{Schnitzler, Arthur}!zzzSalten, Felix@\emph{von Felix Salten}!1895-12-121@{{[}12?. 12. 1895{]}}|(be} \toendnotes[C]{\smallbreak\pagebreak[2]} \Standort{CUL, Schnitzler, B 89, A 1.}
\physDesc{Brief, 1 Blatt, 1 Seite, 405 Zeichen
\newline{}Handschrift: Bleistift, lateinische Kurrent
\newline{}Schnitzler: mit Bleistift datiert: »11/12 95« 
\newline{}Ordnung: mit Bleistift von unbekannter Hand nummeriert: »67« }\toendnotes[C]{\smallbreak}\pstart{}{\pb}Lieber \label{K_L03167-1v}\edtext{F.}{\lemma{\textnormal{\emph{F.}}}\Cendnote{\textnormal{Freund}}}\label{K_L03167-1h}\pend\pstart
           Es soll bei uns\pwindex{?? Werk@Nicht ermittelte Verfasserinnen und Verfasser!Wiener Allgemeine Zeitung1.3.1880 – 11.2.1934@\emph{Wiener Allgemeine Zeitung} {[}1.3.1880 – 11.2.1934{]}|pwv} eine scharfe
               Notiz gegen die \label{K_L03167-2v}\edtext{Zeitung »Liebelei\pwindex{Liebelei. Eine Wiener Zeitschrift1896-01-01@\emph{Liebelei. Eine Wiener Zeitschrift} {[}1896-01-01{]}|pw}«}{\lemma{\textnormal{\emph{Zeitung »Liebelei«}}}\Cendnote{\textnormal{Ab 1. 1. 1896 erschien die von
                     Rolf von Brockdorff\pwindex{Brockdorff, Rolf von @\textsc{Brockdorff, Rolf von}, \emph{Schriftsteller, Herausgeber}|pwk} und Rudolf Strauss\pwindex{Strauss, Rudolf 25.07.1874 – 06.11.1943@\textsc{Strauss, Rudolf} (25.07.1874 – 06.11.1943), \emph{Schriftsteller, Journalist, Redakteur}|pwk} herausgegebene Zeitschrift
                     \emph{Liebelei}\pwindex{Liebelei. Eine Wiener Zeitschrift1896-01-01@\emph{Liebelei. Eine Wiener Zeitschrift} {[}1896-01-01{]}|pwk}. Im Dezember 1895 findet sich keine Kritik daran in der \emph{Wiener Allgemeinen Zeitung}\pwindex{?? Werk@Nicht ermittelte Verfasserinnen und Verfasser!Wiener Allgemeine Zeitung1.3.1880 – 11.2.1934@\emph{Wiener Allgemeine Zeitung} {[}1.3.1880 – 11.2.1934{]}|pwk}.}}}\label{K_L03167-2h} geschrieben werden.
               Soll ich das verhindern, oder begünstigen? Ich habe die Empfindung, als ob Sie jetzt
               ganz gut ein Wort gegen diese Sache sagen könnten. Aber es geht auch, wenn die »W\textsuperscript{r} Allgemeine\orgindex{Wiener Allgemeine Zeitung@Wiener Allgemeine Zeitung|pw}«, quasi
               als Ihr Officiosus in dieser Notiz Ihre Stellung zu dem Unternehmen\pwindex{Liebelei. Eine Wiener Zeitschrift1896-01-01@\emph{Liebelei. Eine Wiener Zeitschrift} {[}1896-01-01{]}|pwv} erklärt.\pend
           \pstart
           Wollen Sie \label{K_L03167-3v}\edtext{heute}{\lemma{\textnormal{\emph{heute}}}\Cendnote{\textnormal{Schnitzler\pwindex{Schnitzler, Arthur 15.05.1862 – 21.10.1931@\textsc{Schnitzler, Arthur} (15.05.1862 – 21.10.1931), \emph{Schriftsteller, Mediziner}|pwk} datierte den Brief mit »11/12 95«, das angesprochene Konzert von Eugen
                     Gura\pwindex{Gura, Eugen 08.11.1842 – 26.08.1906@\textsc{Gura, Eugen} (08.11.1842 – 26.08.1906), \emph{Sänger}|pwk} fand jedoch am 12. 12. 1895 statt, weswegen sich Schnitzler\pwindex{Schnitzler, Arthur 15.05.1862 – 21.10.1931@\textsc{Schnitzler, Arthur} (15.05.1862 – 21.10.1931), \emph{Schriftsteller, Mediziner}|pwk} mit der Datumsangabe um einen Tag vertan haben dürfte.
                  Alternativ wäre es möglich, dass Salten\pwindex{Salten, Felix 06.09.1869 – 08.10.1945@\textsc{Salten, Felix} (06.09.1869 – 08.10.1945), \emph{Schriftsteller, Journalist}|pwk} den
                  Brief am 11.{ }abends verfasste und also das »heute« vordatierte –
                  wissend, dass es erst am Folgetag in den Händen Schnitzler\pwindex{Schnitzler, Arthur 15.05.1862 – 21.10.1931@\textsc{Schnitzler, Arthur} (15.05.1862 – 21.10.1931), \emph{Schriftsteller, Mediziner}|pwk}s sein dürfte. Auffällig ist, dass sich auch für das vorhergehende
                  Korrespondenzstück eine ähnliche Argumentation rechtfertigen ließe, siehe Felix Salten an Arthur Schnitzler, [16. 11. 1895].}}}\label{K_L03167-3h} nach \label{K_L03167-4v}\edtext{Gura\pwindex{Gura, Eugen 08.11.1842 – 26.08.1906@\textsc{Gura, Eugen} (08.11.1842 – 26.08.1906), \emph{Sänger}|pw} zum 
               \uline{Paulus\pwindex{Paulus 1845-02-06 – 1908-06-01@\textsc{Paulus} (1845-02-06 – 1908-06-01), \emph{Sänger}|pw}} (Ronacher\oindex{Ronacher@\textbf{Ronacher}|pw})}{\lemma{\textnormal{\emph{Gura … (Ronacher)}}}\Cendnote{\textnormal{Schnitzler\pwindex{Schnitzler, Arthur 15.05.1862 – 21.10.1931@\textsc{Schnitzler, Arthur} (15.05.1862 – 21.10.1931), \emph{Schriftsteller, Mediziner}|pwk} besuchte zuerst das Konzert von Eugen Gura\pwindex{Gura, Eugen 08.11.1842 – 26.08.1906@\textsc{Gura, Eugen} (08.11.1842 – 26.08.1906), \emph{Sänger}|pwk}, dann ging er tatsächlich ins Ronacher\oindex{Ronacher@\textbf{Ronacher}|pwk}, siehe A. S.: \emph{Tagebuch}, 12. 12. 1895.}}}\label{K_L03167-4h}
               gehen?\pend
           \pstart
           Ihr {\\[\baselineskip]}\spacefill\mbox{Salten}\pend
           \leftskip=0em{}
         
         \endnumbering\mylabel{h}\end{ledgroupsized}  \newcommand{\dateiname}{L03167}\newcommand{\titel}{Felix Salten an Arthur Schnitzler, [12?. 12. 1895]}\newcommand{\editorInnen}{Martin Anton Müller und Laura Untner}%% latex-leseansicht-abspann.tex
%% Abspann für die Leseansicht.
%% Der Schalter \ifkorrekturansicht ist bereits durch den Vorspann gesetzt.

%% latex-abspann.tex
%% Gemeinsamer Abspann für Korrekturansicht und Leseansicht.
%% Setzt den Schalter \ifkorrekturansicht voraus (gesetzt in den
%% einbindenden Dateien latex-korrekturansicht-abspann.tex bzw.
%% latex-leseansicht-abspann.tex).
%% ---------------------------------------------------------------

\normalsize

% Das esempio-Environment wird nur in der Leseansicht benötigt
\ifkorrekturansicht\else
\newenvironment{esempio}[3]%
{
    \vspace{1.5ex}
    \rlap{\underline{#1}}
    \par
    \setlength{\parindent}{0cm}
    \nopagebreak
    \leftskip=#2cm
    \rightskip=#3cm
}
{
    \par
}
\fi

\doendnotes{C}
\bigskip
\vfill

\clearpage

\footnotesize

\ifkorrekturansicht
  \lohead{\textsc{register}}
\fi

% theindex-Environment neu definieren ohne reledmac
\makeatletter
\renewenvironment{theindex}{%
  \ifkorrekturansicht
    \section*{\indexname}%
  \else
    \subsubsection*{Index der erwähnten Entitäten}%
  \fi
  \setlength{\parindent}{0pt}%
  \setlength{\parskip}{0pt plus 0.3pt}%
  \let\item\@idxitem
}{%
  \ifkorrekturansicht\clearpage\fi
}
\makeatother

\IfFileExists{\jobname-pw.ind}{\input{\jobname-pw.ind}}{}

% Quellenangabe nur in der Leseansicht
\ifkorrekturansicht\else
% Fallback-Definitionen, falls die .tex-Datei \titel etc. nicht gesetzt hat
\providecommand{\titel}{}
\providecommand{\editorInnen}{}
\providecommand{\dateiname}{\jobname}

\vspace{3cm}

\vfill

\footnotesize
\textsc{Quelle}: \titel. Herausgegeben von {\editorInnen}. In: \emph{Arthur Schnitzler: Briefwechsel mit Autorinnen und Autoren}.
 Digitale Edition, https://schnitzler-briefe.acdh.oeaw.ac.at/{\dateiname}.html (Stand \today)
\fi

\end{document}


      