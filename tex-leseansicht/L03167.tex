%% latex-leseansicht-vorspann.tex
%% Vorspann für die Leseansicht.
%% Lädt die gemeinsame Datei latex-vorspann.tex mit nicht gesetztem Schalter.

\newif\ifkorrekturansicht
\korrekturansichtfalse

\input{../tex-inputs/latex-vorspann}

\begin{center}
            \textcolor{red}{ENTWURF, NICHT FERTIG KORRIGIERT}
                      \end{center}
            
         
         \renewcommand{\erwaehntePersonen}{Personen: Rolf von Brockdorff, Eugen Gura, Rudolf Strauss}
         \renewcommand{\erwaehnteInstitutionen}{Institutionen: Wiener Allgemeine Zeitung}
         \renewcommand{\erwaehnteOrte}{Orte: Ronacher, Wien}
         \renewcommand{\erwaehnteWerke}{Werke: Liebelei. Eine Wiener Zeitschrift, Wiener Allgemeine Zeitung}
               \section[Felix Salten an Arthur Schnitzler, {[}12. 12. 1895{]}]{ Felix Salten an Arthur Schnitzler, {[}12. 12. 1895{]}}\nopagebreak\mylabel{v}\rehead{ }\begin{ledgroupsized}[t]{13cm}\normalsize\beginnumbering \toendnotes[C]{\smallbreak\pagebreak[2]} \Standort{CUL, Schnitzler, B 89, A 1.}
\physDesc{Brief, 1 Blatt, 1 Seite
\newline{}Handschrift: Bleistift, lateinische Kurrent
\newline{}Schnitzler: mit Bleistift datiert: »11/12 95« \newline{}Ordnung: mit Bleistift von unbekannter Hand nummeriert: »67« }\toendnotes[C]{\smallbreak}\pstart
           \noindent{}{\pb}Lieber F. Es soll bei uns eine scharfe Notiz gegen die \label{K_L03167-11v}\edtext{Zeitung »Liebelei\pwindex{Liebelei. Eine Wiener Zeitschrift1896-01-01@\emph{Liebelei. Eine Wiener Zeitschrift} {[}1896-01-01{]}|pw}«}{\lemma{\textnormal{\emph{Zeitung »Liebelei«}}}\Cendnote{\textnormal{Ab
                     1. 1. 1896 erschien die von Rolf v. Brockdorff\pwindex{Brockdorff, Rolf von @\textsc{Brockdorff, Rolf von}, \emph{Herausgeber, Schriftsteller}|pwk} und Rudolf Strauss\pwindex{Strauss, Rudolf 25.07.1874 – 06.11.1943@\textsc{Strauss, Rudolf} (25.07.1874 – 06.11.1943), \emph{Schriftsteller, Journalist, Redakteur}|pwk} herausgegebene Zeitschrift \emph{Liebelei}\pwindex{Liebelei. Eine Wiener Zeitschrift1896-01-01@\emph{Liebelei. Eine Wiener Zeitschrift} {[}1896-01-01{]}|pwk}. Im Dezember 1895 findet sich keine
                  Kritik daran in der \emph{Wiener Allgemeinen
                     Zeitung}\pwindex{?? Werk@Nicht ermittelte Verfasserinnen und Verfasser!Wiener Allgemeine Zeitung1.3.1880 – 11.2.1934@\emph{Wiener Allgemeine Zeitung} {[}1.3.1880 – 11.2.1934{]}|pwk}.}}}\label{K_L03167-11h} geschrieben werden. Soll ich das verhindern, oder
               begünstigen? Ich habe die Empfindung, als ob Sie jetzt ganz gut ein Wort gegen diese
               Sache sagen könnten. \pend
           \pstart
           Aber es geht auch, wenn die »W\textsuperscript{r} Allgemeine\orgindex{Wiener Allgemeine Zeitung@Wiener Allgemeine Zeitung|pw}« quasi als Ihr Officiosus, in dieser Notiz Ihre
               Stellung zu dem Unternehmen erklärt. \pend
           \pstart
           Wollen Sie \label{K_L03167-1v}\edtext{heute}{\lemma{\textnormal{\emph{heute}}}\Cendnote{\textnormal{Schnitzler\pwindex{Schnitzler, Arthur 15.05.1862 – 21.10.1931@\textsc{Schnitzler, Arthur} (15.05.1862 – 21.10.1931), \emph{Schriftsteller, Mediziner}|pwk} datiert auf »11/12 95«, doch
                  fand das Konzert am 12. 12. 1895 statt, so dass er sich mit der Datumsangabe um einen Tag
                  vertun dürfte. Alternativ wäre es möglich, dass Salten\pwindex{Salten, Felix 06.09.1869 – 08.10.1945@\textsc{Salten, Felix} (06.09.1869 – 08.10.1945), \emph{Schriftsteller, Journalist}|pwk} den Brief am 11. abends verfasste und also das
                     »heute« vordatierte, wissend dass es erst am Folgetag in den
                  Händen Schnitzler\pwindex{Schnitzler, Arthur 15.05.1862 – 21.10.1931@\textsc{Schnitzler, Arthur} (15.05.1862 – 21.10.1931), \emph{Schriftsteller, Mediziner}|pwk}s sein dürfte.}}}\label{K_L03167-1h} nach
                  Gura\pwindex{Gura, Eugen 08.11.1842 – 26.08.1906@\textsc{Gura, Eugen} (08.11.1842 – 26.08.1906), \emph{Sänger}|pw} zum \uline{Paulus\textcolor{red}{\textsuperscript{XXXX indx}}} (Ronacher\oindex{Ronacher@\textbf{Ronacher}|pw}) gehen. \pend
           \pstart Ihr \spacefill\mbox{Salten}\pend{}
         
         \endnumbering\mylabel{h}\end{ledgroupsized}\begin{anhang}\end{anhang}\newcommand{\dateiname}{L03167}\newcommand{\titel}{Felix Salten an Arthur Schnitzler, [12. 12. 1895]}\newcommand{\editorInnen}{Martin Anton Müller und Laura Untner}%% latex-leseansicht-abspann.tex
%% Abspann für die Leseansicht.
%% Der Schalter \ifkorrekturansicht ist bereits durch den Vorspann gesetzt.

%% latex-abspann.tex
%% Gemeinsamer Abspann für Korrekturansicht und Leseansicht.
%% Setzt den Schalter \ifkorrekturansicht voraus (gesetzt in den
%% einbindenden Dateien latex-korrekturansicht-abspann.tex bzw.
%% latex-leseansicht-abspann.tex).
%% ---------------------------------------------------------------

\normalsize

% Das esempio-Environment wird nur in der Leseansicht benötigt
\ifkorrekturansicht\else
\newenvironment{esempio}[3]%
{
    \vspace{1.5ex}
    \rlap{\underline{#1}}
    \par
    \setlength{\parindent}{0cm}
    \nopagebreak
    \leftskip=#2cm
    \rightskip=#3cm
}
{
    \par
}
\fi

\doendnotes{C}
\bigskip
\vfill

\clearpage

\footnotesize

\ifkorrekturansicht
  \lohead{\textsc{register}}
\fi

% theindex-Environment neu definieren ohne reledmac
\makeatletter
\renewenvironment{theindex}{%
  \ifkorrekturansicht
    \section*{\indexname}%
  \else
    \subsubsection*{Index der erwähnten Entitäten}%
  \fi
  \setlength{\parindent}{0pt}%
  \setlength{\parskip}{0pt plus 0.3pt}%
  \let\item\@idxitem
}{%
  \ifkorrekturansicht\clearpage\fi
}
\makeatother

\IfFileExists{\jobname-pw.ind}{\input{\jobname-pw.ind}}{}

% Quellenangabe nur in der Leseansicht
\ifkorrekturansicht\else
% Fallback-Definitionen, falls die .tex-Datei \titel etc. nicht gesetzt hat
\providecommand{\titel}{}
\providecommand{\editorInnen}{}
\providecommand{\dateiname}{\jobname}

\vspace{3cm}

\vfill

\footnotesize
\textsc{Quelle}: \titel. Herausgegeben von {\editorInnen}. In: \emph{Arthur Schnitzler: Briefwechsel mit Autorinnen und Autoren}.
 Digitale Edition, https://schnitzler-briefe.acdh.oeaw.ac.at/{\dateiname}.html (Stand \today)
\fi

\end{document}


      