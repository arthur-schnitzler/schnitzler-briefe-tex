%% latex-korrekturansicht-vorspann.tex
%% Vorspann für die Korrekturansicht.
%% Lädt die gemeinsame Datei latex-vorspann.tex mit gesetztem Schalter.

\newif\ifkorrekturansicht
\korrekturansichttrue

\input{../tex-inputs/latex-vorspann}


\section[ Felix Salten an Arthur Schnitzler, {[}12?. 12. 1895{]}]{L03167 Felix Salten an Arthur Schnitzler, {[}12?. 12. 1895{]}}
\nopagebreak\mylabel{L03167v}
\rehead{ }\normalsize\beginnumbering\briefempfaengerindex{Schnitzler, Arthur@\textsc{Schnitzler, Arthur}!zzzSalten, Felix@\emph{von Felix Salten}!1895-12-121@{{[}12?. 12. 1895{]}}|(be}
\toendnotes[C]{\smallbreak\pagebreak[2]}\Standort{CUL, Schnitzler, B 89, A 1.}
\physDesc{Brief, 1 Blatt, 1 Seite, 405 Zeichen
\newline{}Handschrift: Bleistift, lateinische Kurrent
\newline{}Schnitzler: mit Bleistift datiert: »11/12 95« 
\newline{}Ordnung: mit Bleistift von unbekannter Hand nummeriert: »67« }\toendnotes[C]{\smallbreak}
\pstart{}{\pb}Lieber \label{K_L03167-1v}\edtext{F.}{\lemma{\textnormal{\emph{F.}}}\Cendnote{\textnormal{Freund}}}\label{K_L03167-1}\pend\vspace{0.5em}
\pstart
           Es soll bei uns\pwindex{Wiener Allgemeine Zeitung@\emph{Wiener Allgemeine Zeitung}|pwv} eine scharfe
               Notiz gegen die \label{K_L03167-2v}\edtext{Zeitung »Liebelei\pwindex{Liebelei. Eine Wiener Zeitschrift@\emph{Liebelei. Eine Wiener Zeitschrift}|pw}«}{\lemma{\textnormal{\emph{Zeitung »Liebelei«}}}\Cendnote{\textnormal{Ab 1. 1. 1896 erschien die von
                     Rolf von Brockdorff\pwindex{Brockdorff, Rolf von @\textsc{Brockdorff, Rolf von}, \emph{Schriftsteller/Schriftstellerin, Herausgeber/Herausgeberin}|pwk} und Rudolf Strauss\pwindex{Strauss, Rudolf 25.07.1874 – 06.11.1943@\textsc{Strauss, Rudolf} (25.07.1874 – 06.11.1943), \emph{Schriftsteller/Schriftstellerin, Journalist/Journalistin, Redakteur/Redakteurin}|pwk} herausgegebene Zeitschrift
                     \emph{Liebelei}\pwindex{Liebelei. Eine Wiener Zeitschrift@\emph{Liebelei. Eine Wiener Zeitschrift}|pwk}. Im Dezember 1895 findet sich keine Kritik daran in der \emph{Wiener Allgemeinen Zeitung}\pwindex{Wiener Allgemeine Zeitung@\emph{Wiener Allgemeine Zeitung}|pwk}.}}}\label{K_L03167-2} geschrieben werden.
               Soll ich das verhindern, oder begünstigen? Ich habe die Empfindung, als ob Sie jetzt
               ganz gut ein Wort gegen diese Sache sagen könnten. Aber es geht auch, wenn die »W\textsuperscript{r} Allgemeine\orgindex{Wiener Allgemeine Zeitung@Wiener Allgemeine Zeitung|pw}«, quasi
               als Ihr Officiosus in dieser Notiz Ihre Stellung zu dem Unternehmen\pwindex{Liebelei. Eine Wiener Zeitschrift@\emph{Liebelei. Eine Wiener Zeitschrift}|pwv} erklärt.\pend
           
\pstart
           Wollen Sie \label{K_L03167-3v}\edtext{heute}{\lemma{\textnormal{\emph{heute}}}\Cendnote{\textnormal{Schnitzler datierte den Brief mit »11/12 95«, das angesprochene Konzert von Eugen
                     Gura\pwindex{Gura, Eugen 08.11.1842 – 26.08.1906@\textsc{Gura, Eugen} (08.11.1842 – 26.08.1906), \emph{Sänger/Sängerin}|pwk} fand jedoch am 12. 12. 1895 statt, weswegen sich Schnitzler mit der Datumsangabe um einen Tag vertan haben dürfte.
                  Alternativ wäre es möglich, dass Salten\pwindex{Salten, Felix 06.09.1869 – 08.10.1945@\textsc{Salten, Felix} (06.09.1869 – 08.10.1945), \emph{Schriftsteller/Schriftstellerin, Journalist/Journalistin, Chefredakteur/Chefredakteurin}|pwk} den
                  Brief am 11.{ }abends verfasste und also das »heute« vordatierte –
                  wissend, dass das Korrespondenzstück erst am Folgetag in den Händen Schnitzlers sein dürfte. Auffällig ist, dass sich auch für das vorhergehende
                  Schreiben eine ähnliche Argumentation rechtfertigen lässt, siehe Felix Salten an Arthur Schnitzler, [16. 11. 1895].}}}\label{K_L03167-3} nach \label{K_L03167-4v}\edtext{Gura\pwindex{Gura, Eugen 08.11.1842 – 26.08.1906@\textsc{Gura, Eugen} (08.11.1842 – 26.08.1906), \emph{Sänger/Sängerin}|pw} zum 
               \uline{Paulus\pwindex{Paulus 1845-02-06 – 1908-06-01@\textsc{Paulus} (1845-02-06 – 1908-06-01), \emph{Sänger/Sängerin}|pw}} (Ronacher\oindex{Ronacher@\textbf{Ronacher}, \emph{Theater (K.THE)}|pw})}{\lemma{\textnormal{\emph{Gura … (Ronacher)}}}\Cendnote{\textnormal{Schnitzler besuchte zuerst das Konzert von Eugen Gura\pwindex{Gura, Eugen 08.11.1842 – 26.08.1906@\textsc{Gura, Eugen} (08.11.1842 – 26.08.1906), \emph{Sänger/Sängerin}|pwk}, dann ging er tatsächlich ins Ronacher\oindex{Ronacher@\textbf{Ronacher}, \emph{Theater (K.THE)}|pwk}, siehe A. S.: \emph{Tagebuch}, 12. 12. 1895.}}}\label{K_L03167-4}
               gehen?\pend
           
\pstart
           Ihr {\\[\baselineskip]}\spacefill\mbox{Salten}\pend
           \leftskip=0em{}\selectlanguage{ngerman}\endnumbering\briefempfaengerindex{Schnitzler, Arthur@\textsc{Schnitzler, Arthur}!zzzSalten, Felix@\emph{von Felix Salten}!1895-12-121@{{[}12?. 12. 1895{]}}|)be}\mylabel{L03167h}  \normalsize

\doendnotes{C}
\bigskip
\vfill

\clearpage

\footnotesize

\lohead{\textsc{register}}

% Definiere theindex-Environment komplett neu ohne reledmac
\makeatletter
\renewenvironment{theindex}{%
  \section*{\indexname}%
  \setlength{\parindent}{0pt}%
  \setlength{\parskip}{0pt plus 0.3pt}%
  \let\item\@idxitem
}{%
  \clearpage
}
\makeatother

\IfFileExists{\jobname-pw.ind}{\input{\jobname-pw.ind}}{}

\end{document}

      