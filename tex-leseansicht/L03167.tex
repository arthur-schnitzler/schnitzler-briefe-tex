%% latex-leseansicht-vorspann.tex
%% Vorspann für die Leseansicht.
%% Lädt die gemeinsame Datei latex-vorspann.tex mit nicht gesetztem Schalter.

\newif\ifkorrekturansicht
\korrekturansichtfalse

\input{../tex-inputs/latex-vorspann}


\section[ Felix Salten an Arthur Schnitzler, [12?. 12. 1895]]{L03167 Felix Salten an Arthur Schnitzler,  [12?. 12. 1895]}
\nopagebreak\mylabel{L03167v}
\rehead{ }\normalsize\beginnumbering\briefempfaengerindex{Schnitzler, Arthur@\textsc{Schnitzler, Arthur}!zzzSalten, Felix@\emph{von Felix Salten}!1895-12-121@{{[}12?. 12. 1895{]}}|(be}
\toendnotes[C]{\smallbreak\pagebreak[2]}
\correspDesc{Versand  durch Felix Salten am [12?. 12. 1895] in Wien
\newline{}Erhalt  durch Arthur Schnitzler am [12?. 12. 1895?] in Wien}\toendnotes[C]{\smallbreak}
\Standort{CUL, Schnitzler, B 89, A 1.}
\physDesc{Brief, 1 Blatt, 1 Seite, 405 Zeichen
\newline{}Handschrift: Bleistift, lateinische Kurrent
\newline{}Schnitzler: mit Bleistift datiert: »11/12 95« 
\newline{}Ordnung: mit Bleistift von unbekannter Hand nummeriert: »67« }\toendnotes[C]{\smallbreak}
\pstart{}{\pb}Lieber \label{K_L03167-1v}\edtext{F.}{\lemma{\textnormal{\emph{F.}}}\Cendnote{\textnormal{Freund}}}\label{K_L03167-1}\pend\vspace{0.5em}
\pstart
           Es soll bei uns\pwindex{Wiener Allgemeine Zeitung@\emph{Wiener Allgemeine Zeitung}|pwv} eine scharfe
               Notiz gegen die \label{K_L03167-2v}\edtext{Zeitung »Liebelei\pwindex{Liebelei. Eine Wiener Zeitschrift@\emph{Liebelei. Eine Wiener Zeitschrift}|pw}«}{\lemma{\textnormal{\emph{Zeitung »Liebelei«}}}\Cendnote{\textnormal{Ab 1. 1. 1896 erschien die von
                     Rolf von Brockdorff\pwindex{Brockdorff, Rolf von @\textsc{Brockdorff, Rolf von}, \emph{Schriftsteller, Herausgeber}|pwk} und Rudolf Strauss\pwindex{Strauss, Rudolf 25.\,7.\,1874 Bielsko-Biała – 6.\,11.\,1943 Wien@\textsc{Strauss, Rudolf} (25.\,7.\,1874 Bielsko-Biała – 6.\,11.\,1943 Wien), \emph{Schriftsteller, Journalist, Redakteur}|pwk} herausgegebene Zeitschrift
                     \emph{Liebelei}\pwindex{Liebelei. Eine Wiener Zeitschrift@\emph{Liebelei. Eine Wiener Zeitschrift}|pwk}. Im Dezember 1895 findet sich keine Kritik daran in der \emph{Wiener Allgemeinen Zeitung}\pwindex{Wiener Allgemeine Zeitung@\emph{Wiener Allgemeine Zeitung}|pwk}.}}}\label{K_L03167-2} geschrieben werden.
               Soll ich das verhindern, oder begünstigen? Ich habe die Empfindung, als ob Sie jetzt
               ganz gut ein Wort gegen diese Sache sagen könnten. Aber es geht auch, wenn die »W\textsuperscript{r} Allgemeine\orgindex{Wiener Allgemeine Zeitung@Wiener Allgemeine Zeitung|pw}«, quasi
               als Ihr Officiosus in dieser Notiz Ihre Stellung zu dem Unternehmen\pwindex{Liebelei. Eine Wiener Zeitschrift@\emph{Liebelei. Eine Wiener Zeitschrift}|pwv} erklärt.\pend
           
\pstart
           Wollen Sie \label{K_L03167-3v}\edtext{heute}{\lemma{\textnormal{\emph{heute}}}\Cendnote{\textnormal{Schnitzler datierte den Brief mit »11/12 95«, das angesprochene Konzert von Eugen
                     Gura\pwindex{Gura, Eugen 8.\,11.\,1842 Brežany – 26.\,8.\,1906 Aufkirchen@\textsc{Gura, Eugen} (8.\,11.\,1842 Brežany – 26.\,8.\,1906 Aufkirchen), \emph{Sänger}|pwk} fand jedoch am 12. 12. 1895 statt, weswegen sich Schnitzler mit der Datumsangabe um einen Tag vertan haben dürfte.
                  Alternativ wäre es möglich, dass Salten\pwindex{Salten, Felix 6.\,9.\,1869 Budapest – 8.\,10.\,1945 Zürich@\textsc{Salten, Felix} (6.\,9.\,1869 Budapest – 8.\,10.\,1945 Zürich), \emph{Schriftsteller, Journalist, Chefredakteur}|pwk} den
                  Brief am 11.{ }abends verfasste und also das »heute« vordatierte –
                  wissend, dass das Korrespondenzstück erst am Folgetag in den Händen Schnitzlers sein dürfte. Auffällig ist, dass sich auch für das vorhergehende
                  Schreiben eine ähnliche Argumentation rechtfertigen lässt, siehe XXXX Auszeichnungsfehler: Dokument L03166 nicht gefunden.}}}\label{K_L03167-3} nach \label{K_L03167-4v}\edtext{Gura\pwindex{Gura, Eugen 8.\,11.\,1842 Brežany – 26.\,8.\,1906 Aufkirchen@\textsc{Gura, Eugen} (8.\,11.\,1842 Brežany – 26.\,8.\,1906 Aufkirchen), \emph{Sänger}|pw} zum 
               \uline{Paulus\pwindex{Paulus 6.\,2.\,1845 Saint-Esprit – 1.\,6.\,1908 Saint-Mandé@\textsc{Paulus} (6.\,2.\,1845 Saint-Esprit – 1.\,6.\,1908 Saint-Mandé), \emph{Sänger}|pw}} (Ronacher\oindex{Wien@\textbf{Wien}!I., Innere Stadt@\textbf{I., Innere Stadt}!Ronacher@\textbf{Ronacher}, \emph{Theater}|pw})}{\lemma{\textnormal{\emph{Gura … (Ronacher)}}}\Cendnote{\textnormal{Schnitzler besuchte zuerst das Konzert von Eugen Gura\pwindex{Gura, Eugen 8.\,11.\,1842 Brežany – 26.\,8.\,1906 Aufkirchen@\textsc{Gura, Eugen} (8.\,11.\,1842 Brežany – 26.\,8.\,1906 Aufkirchen), \emph{Sänger}|pwk}, dann ging er tatsächlich ins Ronacher\oindex{Wien@\textbf{Wien}!I., Innere Stadt@\textbf{I., Innere Stadt}!Ronacher@\textbf{Ronacher}, \emph{Theater}|pwk}, siehe A. S.: \emph{Tagebuch}, 12. 12. 1895.}}}\label{K_L03167-4}
               gehen?\pend
           
\pstart
           Ihr {\\[\baselineskip]}\spacefill\mbox{Salten}\pend
           \leftskip=0em{}\selectlanguage{ngerman}\endnumbering\briefempfaengerindex{Schnitzler, Arthur@\textsc{Schnitzler, Arthur}!zzzSalten, Felix@\emph{von Felix Salten}!1895-12-121@{{[}12?. 12. 1895{]}}|)be}\mylabel{L03167h}  \newcommand{\dateiname}{L03167}\newcommand{\titel}{Felix Salten an Arthur Schnitzler, [12?. 12. 1895]}\newcommand{\editorInnen}{Martin Anton Müller und Laura Untner}%% latex-leseansicht-abspann.tex
%% Abspann für die Leseansicht.
%% Der Schalter \ifkorrekturansicht ist bereits durch den Vorspann gesetzt.

%% latex-abspann.tex
%% Gemeinsamer Abspann für Korrekturansicht und Leseansicht.
%% Setzt den Schalter \ifkorrekturansicht voraus (gesetzt in den
%% einbindenden Dateien latex-korrekturansicht-abspann.tex bzw.
%% latex-leseansicht-abspann.tex).
%% ---------------------------------------------------------------

\normalsize

% Das esempio-Environment wird nur in der Leseansicht benötigt
\ifkorrekturansicht\else
\newenvironment{esempio}[3]%
{
    \vspace{1.5ex}
    \rlap{\underline{#1}}
    \par
    \setlength{\parindent}{0cm}
    \nopagebreak
    \leftskip=#2cm
    \rightskip=#3cm
}
{
    \par
}
\fi

\doendnotes{C}
\bigskip
\vfill

\clearpage

\footnotesize

\ifkorrekturansicht
  \lohead{\textsc{register}}
\fi

% theindex-Environment neu definieren ohne reledmac
\makeatletter
\renewenvironment{theindex}{%
  \ifkorrekturansicht
    \section*{\indexname}%
  \else
    \subsubsection*{Index der erwähnten Entitäten}%
  \fi
  \setlength{\parindent}{0pt}%
  \setlength{\parskip}{0pt plus 0.3pt}%
  \let\item\@idxitem
}{%
  \ifkorrekturansicht\clearpage\fi
}
\makeatother

\IfFileExists{\jobname-pw.ind}{\input{\jobname-pw.ind}}{}

% Quellenangabe nur in der Leseansicht
\ifkorrekturansicht\else
% Fallback-Definitionen, falls die .tex-Datei \titel etc. nicht gesetzt hat
\providecommand{\titel}{}
\providecommand{\editorInnen}{}
\providecommand{\dateiname}{\jobname}

\vspace{3cm}

\vfill

\footnotesize
\textsc{Quelle}: \titel. Herausgegeben von {\editorInnen}. In: \emph{Arthur Schnitzler: Briefwechsel mit Autorinnen und Autoren}.
 Digitale Edition, https://schnitzler-briefe.acdh.oeaw.ac.at/{\dateiname}.html (Stand \today)
\fi

\end{document}


