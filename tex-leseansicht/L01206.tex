%% latex-leseansicht-vorspann.tex
%% Vorspann für die Leseansicht.
%% Lädt die gemeinsame Datei latex-vorspann.tex mit nicht gesetztem Schalter.

\newif\ifkorrekturansicht
\korrekturansichtfalse

\input{../tex-inputs/latex-vorspann}


\section[Arthur Schnitzler an Richard Beer-Hofmann, 26. 2. 1902]{L01206 Arthur Schnitzler an Richard Beer-Hofmann, 26. 2. 1902}
\nopagebreak\mylabel{L01206v}
\rehead{ }\normalsize\beginnumbering\briefempfaengerindex{Beer-Hofmann, Richard@\textsc{Beer-Hofmann, Richard}!zzzSchnitzler, Arthur@\emph{von Arthur Schnitzler}!1902-02-261@{26. 2. 1902}|(be}
\toendnotes[C]{\smallbreak\pagebreak[2]}
\correspDesc{Versand  durch Arthur Schnitzler am 26. 2. 1902 in Wien
\newline{}Erhalt  durch Richard Beer-Hofmann am 27. 2. 1902 in Rodaun}\toendnotes[C]{\smallbreak}
\Standort{YCGL, MSS 31.}
\physDesc{Brief, 1 Blatt, 3 Seiten, Kuvert, 355 Zeichen
\newline{}Handschrift: Bleistift, deutsche Kurrent
\newline{}Versand: 1) Stempel: »\nobreak{}\oindex{Wien@\textbf{Wien}, \emph{Verwaltungsgebiet}|pwk}Wien, 27. \textcolor{gray}{2}. 0\textcolor{gray}{2}, \textcolor{gray}{11}V\nobreak{}«.   2) Stempel: »\nobreak{}\oindex{Wien@\textbf{Wien}!XXIII., Liesing@\textbf{XXIII., Liesing}!Rodaun@\textbf{Rodaun}, \emph{Region}|pwk}{\pb}Rodaun, 27 2 02\nobreak{}«. }
\buchAbdrucke{\weitereDrucke{Arthur Schnitzler, Richard Beer-Hofmann: \emph{Briefwechsel 1891–1931}. Herausgegeben von Konstanze Fliedl. Wien, Zürich: \emph{Europaverlag} 1992, S. 157.} }\toendnotes[C]{\smallbreak}\pstart{}{\pb}Herrn \textsc{Dr Richard
                     Beer-Hofmann}\pend{}\pstart{}\textsc{Rodaun} bei Lieſing\oindex{Wien@\textbf{Wien}!XXIII., Liesing@\textbf{XXIII., Liesing}!Rodaun@\textbf{Rodaun}, \emph{Region}|pw}\pend{}\pstart{}\textsc{Liesinger}ſtraße 1\oindex{Liesingerstraße@\textbf{Liesingerstraße}, \emph{Straße}|pw}. \pend{}{\bigskip}\vspace{1em}
\pstart
           \raggedleft{}{\pb}\uline{Mittwoch}.\pend
           \vspace{0.5em}
\pstart
           lieber Richard, ich bin vorläufig wieder nach Wien\oindex{Wien@\textbf{Wien}, \emph{Verwaltungsgebiet}|pw} gezogen, habe noch \label{K_L01206-1v}\edtext{keine »Villa«}{\lemma{\textnormal{\emph{keine »Villa«}}}\Cendnote{\textnormal{Er
                  war auf Suche nach einer gemeinsamen Unterkunft für seine Familie.}}}\label{K_L01206-1}, bin{ }ſehr enervirt. Die {\pb}Damen\pwindex{Schnitzler, Olga 17.\,1.\,1882 Wien – 13.\,1.\,1970 Lugano@\textsc{Schnitzler, Olga} (17.\,1.\,1882 Wien – 13.\,1.\,1970 Lugano), \emph{Schauspielerin, Sängerin}|pwv}\pwindex{Steinrück, Elisabeth 19.\,11.\,1885 – 7.\,4.\,1920 Partenkirchen@\textsc{Steinrück, Elisabeth} (19.\,11.\,1885 – 7.\,4.\,1920 Partenkirchen)|pwv}{ }ſind zu \textsc{Schönberger}\oindex{Zum goldenen Stern@\textbf{Zum goldenen Stern}, \emph{Lokal}|pw} gezogen, da es im Kurhaus\oindex{Cur- und Wasserheilanstalt Hinterbrühl@\textbf{Cur- und Wasserheilanstalt Hinterbrühl}, \emph{Sanatorium}|pw} auf die Dauer
               abſolut nicht auszuhalten war.\pend
           
\pstart
           Ich weiſs noch nicht, wa{\geminationn} ich nach Rodaun\oindex{Wien@\textbf{Wien}!XXIII., Liesing@\textbf{XXIII., Liesing}!Rodaun@\textbf{Rodaun}, \emph{Region}|pw}{ }{\pb}komme, grüßen Sie bitte auch Hugo\pwindex{Hofmannsthal, Hugo von 1.\,2.\,1874 Wien – 15.\,7.\,1929 Rodaun@\textsc{Hofmannsthal, Hugo von} (1.\,2.\,1874 Wien – 15.\,7.\,1929 Rodaun), \emph{Schriftsteller}|pw}.\pend
           
\pstart
           Herzlichſt{\\[\baselineskip]}Ihr{\\[\baselineskip]}\spacefill\mbox{A.}\pend
           \leftskip=0em{}\selectlanguage{ngerman}\endnumbering\briefempfaengerindex{Beer-Hofmann, Richard@\textsc{Beer-Hofmann, Richard}!zzzSchnitzler, Arthur@\emph{von Arthur Schnitzler}!1902-02-261@{26. 2. 1902}|)be}\mylabel{L01206h}  \newcommand{\dateiname}{L01206}\newcommand{\titel}{Arthur Schnitzler an Richard Beer-Hofmann, 26. 2. 1902}\newcommand{\editorInnen}{Martin Anton Müller und Gerd-Hermann Susen}%% latex-leseansicht-abspann.tex
%% Abspann für die Leseansicht.
%% Der Schalter \ifkorrekturansicht ist bereits durch den Vorspann gesetzt.

%% latex-abspann.tex
%% Gemeinsamer Abspann für Korrekturansicht und Leseansicht.
%% Setzt den Schalter \ifkorrekturansicht voraus (gesetzt in den
%% einbindenden Dateien latex-korrekturansicht-abspann.tex bzw.
%% latex-leseansicht-abspann.tex).
%% ---------------------------------------------------------------

\normalsize

% Das esempio-Environment wird nur in der Leseansicht benötigt
\ifkorrekturansicht\else
\newenvironment{esempio}[3]%
{
    \vspace{1.5ex}
    \rlap{\underline{#1}}
    \par
    \setlength{\parindent}{0cm}
    \nopagebreak
    \leftskip=#2cm
    \rightskip=#3cm
}
{
    \par
}
\fi

\doendnotes{C}
\bigskip
\vfill

\clearpage

\footnotesize

\ifkorrekturansicht
  \lohead{\textsc{register}}
\fi

% theindex-Environment neu definieren ohne reledmac
\makeatletter
\renewenvironment{theindex}{%
  \ifkorrekturansicht
    \section*{\indexname}%
  \else
    \subsubsection*{Index der erwähnten Entitäten}%
  \fi
  \setlength{\parindent}{0pt}%
  \setlength{\parskip}{0pt plus 0.3pt}%
  \let\item\@idxitem
}{%
  \ifkorrekturansicht\clearpage\fi
}
\makeatother

\IfFileExists{\jobname-pw.ind}{\input{\jobname-pw.ind}}{}

% Quellenangabe nur in der Leseansicht
\ifkorrekturansicht\else
% Fallback-Definitionen, falls die .tex-Datei \titel etc. nicht gesetzt hat
\providecommand{\titel}{}
\providecommand{\editorInnen}{}
\providecommand{\dateiname}{\jobname}

\vspace{3cm}

\vfill

\footnotesize
\textsc{Quelle}: \titel. Herausgegeben von {\editorInnen}. In: \emph{Arthur Schnitzler: Briefwechsel mit Autorinnen und Autoren}.
 Digitale Edition, https://schnitzler-briefe.acdh.oeaw.ac.at/{\dateiname}.html (Stand \today)
\fi

\end{document}


