%% latex-korrekturansicht-vorspann.tex
%% Vorspann für die Korrekturansicht.
%% Lädt die gemeinsame Datei latex-vorspann.tex mit gesetztem Schalter.

\newif\ifkorrekturansicht
\korrekturansichttrue

\input{../tex-inputs/latex-vorspann}


\section[ Paul Goldmann an Arthur Schnitzler, 24. 5. {[}1896{]}]{L02775 Paul Goldmann an Arthur Schnitzler, 24. 5. {[}1896{]}}
\nopagebreak\mylabel{L02775v}
\rehead{ }\normalsize\beginnumbering\briefempfaengerindex{Schnitzler, Arthur@\textsc{Schnitzler, Arthur}!zzzGoldmann, Paul@\emph{von Paul Goldmann}!1896-05-241@{24. 5. {[}1896{]}}|(be}
\toendnotes[C]{\smallbreak\pagebreak[2]}\Standort{DLA, A:Schnitzler, HS.NZ85.1.3166.}
\physDesc{Brief, 2 Blätter, 8 Seiten, 2442 Zeichen
\newline{}Handschrift: blaue Tinte, deutsche Kurrent
\newline{}Schnitzler: 1) mit Bleistift das Jahr »96« vermerkt  2) mit rotem Buntstift auf der ersten Seite »\textsc{Kerr\pwindex{Kerr, Alfred 25.12.1867 – 12.10.1948@\textsc{Kerr, Alfred} (25.12.1867 – 12.10.1948), \emph{Schriftsteller/Schriftstellerin, Kritiker/Kritikerin}|pw}}« vermerkt und insgesamt drei Unterstreichungen}\toendnotes[C]{\smallbreak}
\pstart
           {\pb}\textcolor{gray}{\textbf{\textbf{Frankfurter Zeitung\orgindex{Frankfurter Zeitung@Frankfurter Zeitung|pw}}}}\pend
           
\pstart
           \textcolor{gray}{\textbf{(\begin{otherlanguage}{french}Gazette de Francfort\end{otherlanguage}\orgindex{Frankfurter Zeitung@Frankfurter Zeitung|pw}).}}\pend
           
\pstart
           \textcolor{gray}{\textbf{\textbf{\begin{otherlanguage}{french}Fondateur M.\end{otherlanguage}{ }L. Sonnemann\pwindex{Sonnemann, Leopold 1831-10-29 – 1909-10-30@\textsc{Sonnemann, Leopold} (1831-10-29 – 1909-10-30), \emph{Journalist/Journalistin, Herausgeber/Herausgeberin}|pw}.}}}\pend
           
\pstart
           \begin{otherlanguage}{french}\textcolor{gray}{\textbf{Journal\pwindex{Frankfurter Zeitung@\emph{Frankfurter Zeitung}|pwv} politique,
                        financier,}}\end{otherlanguage}\pend
           
\pstart
           \begin{otherlanguage}{french}\textcolor{gray}{\textbf{commercial et littéraire.}}\end{otherlanguage}\pend
           
\pstart
           \begin{otherlanguage}{french}\textcolor{gray}{\textbf{\textbf{Paraissant trois fois par jour.}}}\end{otherlanguage}\pend
           
\pstart
           \begin{otherlanguage}{french}\textcolor{gray}{\textbf{\textbf{Bureau à Paris\oindex{Paris@\textbf{Paris}, \emph{P.PPLC}|pw}}}}\end{otherlanguage}\hfill \textsc{Paris\oindex{Paris@\textbf{Paris}, \emph{P.PPLC}|pw}}, 24. Mai.\pend
           
\pstart
           \begin{otherlanguage}{french}\textcolor{gray}{\textbf{\textbf{24. Rue Feydeau\oindex{rue Feydeau@\textbf{rue Feydeau}, \emph{Straße (K.STR)}|pw}.}}}\end{otherlanguage}\pend
           
\pstart\center{}Mein lieber Freund,\pend\vspace{0.5em}
\pstart
           Vielen Dank für die »Freie
                  Bühne\pwindex{Neue Deutsche Rundschau@\emph{Neue Deutsche Rundschau}|pwv}«, die ich anbei zurückſende. (Das heißt nicht »anbei«. Ich behalte ſie
               noch bis Dienſtag, um ſie \textsc{M. Schefer\pwindex{Schefer, Christian 1866-07-14 – Februar 1944@\textsc{Schefer, Christian} (1866-07-14 – Februar 1944), \emph{Journalist/Journalistin, Lehrer/Lehrerin}|pw}} zu zeigen, der mich an dieſem Tage beſuchen
               kommt). Der \label{K_L02775-1v}\edtext{Artikel\pwindex{Arthur Schnitzler@\emph{Arthur Schnitzler}|pwv}}{\lemma{\textnormal{\emph{Artikel}}}\Cendnote{\textnormal{Alfred Kerr\pwindex{Kerr, Alfred 25.12.1867 – 12.10.1948@\textsc{Kerr, Alfred} (25.12.1867 – 12.10.1948), \emph{Schriftsteller/Schriftstellerin, Kritiker/Kritikerin}|pwk}: \emph{Arthur Schnitzler}\pwindex{Arthur Schnitzler@\emph{Arthur Schnitzler}|pwk}. In: \emph{Neue Deutsche Rundschau (Freie Bühne)}\pwindex{Neue Deutsche Rundschau@\emph{Neue Deutsche Rundschau}|pwk}, Jg. 7, H. 3, März 1896, S. 287–292.}}}\label{K_L02775-1} iſt höchſt
               intereſſant. Ich freue mich über den ſchönen Enthuſiasmus, den mein lieber \textsc{Arthur} erregt. Auch ſagt der {\pb}Verfaſſer\pwindex{Kerr, Alfred 25.12.1867 – 12.10.1948@\textsc{Kerr, Alfred} (25.12.1867 – 12.10.1948), \emph{Schriftsteller/Schriftstellerin, Kritiker/Kritikerin}|pwv} manches Richtige.
               Im Allgemeinen aber ſind \strikeout{\textcolor{gray}{E}} mir ſeine kraftgenialiſche Art und Styl nicht ſehr ſympathiſch.\pend
           
\pstart
           \label{K_L02775-2v}\edtext{Beifolgenden Brief}{\lemma{\textnormal{\emph{Beifolgenden Brief}}}\Cendnote{\textnormal{Beilage nicht erhalten, Verfasser nicht
                  identifiziert}}}\label{K_L02775-2} empfehle ich \strikeout{D\textcolor{gray}{ic}h} Dir aufs Wärmſte zur bejahenden Beantwortung.
               Verfaſſer iſt ein Vetter\pwindex{?? [Vetter von Heinrich Kanner] @\textsc{?? [Vetter von Heinrich Kanner]}|pwv} von
                  \textsc{Kanner\pwindex{Kanner, Heinrich 09.11.1864 – 15.02.1930@\textsc{Kanner, Heinrich} (09.11.1864 – 15.02.1930), \emph{Herausgeber/Herausgeberin, Publizist/Publizistin}|pw}} – kreuzbraver Menſch\pwindex{?? [Vetter von Heinrich Kanner] @\textsc{?? [Vetter von Heinrich Kanner]}|pwv} –
               ſelbſt ſchwer lungenleidend, der wohl im »Sterben\pwindex{Sterben. Novelle@\emph{Sterben. Novelle}|pw}« ein Stück {\pb}ſeines Schickſals
               gefunden hat.\pend
           
\pstart
           Über den \label{K_L02775-3v}\edtext{»Vortrag\pwindex{Poesie und Leben. Aus einem Vortrage@\emph{Poesie und Leben. Aus einem Vortrage}|pwv}«}{\lemma{\textnormal{\emph{»Vortrag«}}}\Cendnote{\textnormal{Hugo von Hofmannsthal\pwindex{Hofmannsthal, Hugo von 1874-02-01 – 1929-07-15@\textsc{Hofmannsthal, Hugo von} (1874-02-01 – 1929-07-15), \emph{Schriftsteller/Schriftstellerin}|pwk}: \emph{Poesie und Leben}\pwindex{Poesie und Leben. Aus einem Vortrage@\emph{Poesie und Leben. Aus einem Vortrage}|pwk}. In: \emph{Die Zeit}\pwindex{Zeit. Wiener Wochenschrift@\emph{Die Zeit. Wiener Wochenschrift}|pwk}, Bd. 7, Nr. 85, 16. 5. 1896,
                     S. 104–106.}}}\label{K_L02775-3} von \textsc{Loris\pwindex{Hofmannsthal, Hugo von 1874-02-01 – 1929-07-15@\textsc{Hofmannsthal, Hugo von} (1874-02-01 – 1929-07-15), \emph{Schriftsteller/Schriftstellerin}|pw}}, den die letzte »Zeit\pwindex{Zeit. Wiener Wochenschrift@\emph{Die Zeit. Wiener Wochenschrift}|pw}« gebracht, war ich
               wüthend. Ich verſtehe nicht ein Wort von dem, was er will. Und dann Stellen, wie:
                  »Eine neue und kühne Verbindung
                  von Worten iſt das wundervollſte Geſchenk für die Seelen und nichts geringeres als
                  ein Standbild des Knaben \textsc{Antinous\pwindex{Antinoos @\textsc{Antinoos}|pw}} oder eine große gewölbte Pforte\pwindex{Poesie und Leben. Aus einem Vortrage@\emph{Poesie und Leben. Aus einem Vortrage}|pwv}«. Das iſt doch unerhört! Was iſt eine
               große gewölbte {\pb}Pforte für die Seelen? Und was hat
               das, zum Teufel, mit dem Standbild des Knaben \textsc{Antinous\pwindex{Antinoos @\textsc{Antinoos}|pw}} zu thun? Ich will nicht ausſchließen, daß das wirklich empfunden iſt. Aber wenn
               auch – ſo thut das eine ganz unerhörte Empfindungen-Verwirrung dar. Auch iſt es eine
               verfluchte Schlamperei, ſich ſo gehen zu laſſen und jede \label{K_L02775-4v}\edtext{\textsc{\begin{otherlanguage}{french}incohérence\end{otherlanguage}}}{\lemma{\textnormal{\emph{incohérence}}}\Cendnote{\textnormal{französisch: mangelnder
                  Zusammenhang}}}\label{K_L02775-4} auszuſprechen, die Einem durchs Hirn fährt, \strikeout{die \textcolor{gray}{×}\-\textcolor{gray}{×}\-\textcolor{gray}{×}\-\textcolor{gray}{×}\-\textcolor{gray}{×}\-\textcolor{gray}{×}\-\textcolor{gray}{×}{ }\textcolor{gray}{×}\-\textcolor{gray}{×}\-\textcolor{gray}{×}\-\textcolor{gray}{×}{ }\textcolor{gray}{wird}} in der Überzeugung, das {\pb}ſei genial. Auch
               wird die Literatur auf dieſe Weiſe zu einer Geheim-Sprache, die nur mehr ein paar
               Eingeweihte verſtehen. Dieſer junge Mann\pwindex{Hofmannsthal, Hugo von 1874-02-01 – 1929-07-15@\textsc{Hofmannsthal, Hugo von} (1874-02-01 – 1929-07-15), \emph{Schriftsteller/Schriftstellerin}|pwv} ſchreibt doch fürs Publicum. Und wenn er ſich nicht mehr
               ſo ausdrücken kann, daß ihn das Publicum versteht – wenn ſeine Gedanken einen Flug
               nehmen, {\pb}wo die Maſſe ihm nicht nach kann und wo er
               ſelbſt kaum noch mit kann – dann ſoll er eben \strikeout{kei\textcolor{gray}{n}} nichts mehr drucken laſſen und keine Vorträge halten. Hübſch iſt auch, daß es
               einmal heißt, »bei den neueren
                  deutſchen ſogenannten Dichtern\pwindex{Poesie und Leben. Aus einem Vortrage@\emph{Poesie und Leben. Aus einem Vortrage}|pwv}«. Und weiter unten: »Sie wundern ſich, daß Ihnen \uline{ein Dichter} die Regeln lobt \textsc{etc.}\pwindex{Poesie und Leben. Aus einem Vortrage@\emph{Poesie und Leben. Aus einem Vortrage}|pwv}« Alſo größenwahnſinnig {\pb}iſt dieſer junge Mann\pwindex{Hofmannsthal, Hugo von 1874-02-01 – 1929-07-15@\textsc{Hofmannsthal, Hugo von} (1874-02-01 – 1929-07-15), \emph{Schriftsteller/Schriftstellerin}|pwv} auch ſchon. Worauf hin?
               Mit dem »jungen \textsc{Goethe\pwindex{Goethe, Johann Wolfgang von 1749-08-28 – 1832-03-22@\textsc{Goethe, Johann Wolfgang von} (1749-08-28 – 1832-03-22), \emph{Schriftsteller/Schriftstellerin}|pw}}« iſt es bisher nichts geworden. Bisher hat es eigentlich nur in einem Punkte
               geſtimmt: in der Jugend.\pend
           
\pstart
           Nein, iſt dieſer arme kleine Burſch\pwindex{Hofmannsthal, Hugo von 1874-02-01 – 1929-07-15@\textsc{Hofmannsthal, Hugo von} (1874-02-01 – 1929-07-15), \emph{Schriftsteller/Schriftstellerin}|pwv} verdorben worden\strikeout{!} von \textsc{Bahr\pwindex{Bahr, Hermann 19.07.1863 – 15.01.1934@\textsc{Bahr, Hermann} (19.07.1863 – 15.01.1934), \emph{Schriftsteller/Schriftstellerin, Kritiker/Kritikerin}|pw}}, dieſem verfluchten Pfuſcher und Schurken!\pend
           
\pstart
           {\pb}Grüß’ Dich Gott, liebſter Freund.\pend
           
\pstart
           Auch ſchreibſt Du mir wohl nächſtens einmal.\pend
           
\pstart
           Dein {\\[\baselineskip]}treuer {\\[\baselineskip]}\spacefill\mbox{Paul Goldmann}\pend
           \leftskip=0em{}\selectlanguage{ngerman}\endnumbering\briefempfaengerindex{Schnitzler, Arthur@\textsc{Schnitzler, Arthur}!zzzGoldmann, Paul@\emph{von Paul Goldmann}!1896-05-241@{24. 5. {[}1896{]}}|)be}\mylabel{L02775h}  \normalsize

\doendnotes{C}
\bigskip
\vfill

\clearpage

\footnotesize

\lohead{\textsc{register}}

% Definiere theindex-Environment komplett neu ohne reledmac
\makeatletter
\renewenvironment{theindex}{%
  \section*{\indexname}%
  \setlength{\parindent}{0pt}%
  \setlength{\parskip}{0pt plus 0.3pt}%
  \let\item\@idxitem
}{%
  \clearpage
}
\makeatother

\IfFileExists{\jobname-pw.ind}{\input{\jobname-pw.ind}}{}

\end{document}

      