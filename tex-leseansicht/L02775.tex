%% latex-leseansicht-vorspann.tex
%% Vorspann für die Leseansicht.
%% Lädt die gemeinsame Datei latex-vorspann.tex mit nicht gesetztem Schalter.

\newif\ifkorrekturansicht
\korrekturansichtfalse

\input{../tex-inputs/latex-vorspann}


         
         \newcommand{\erwaehntePersonen}{Personen:  ?? [Vetter von Heinrich Kanner],  Antinoos, Hermann Bahr, Johann Wolfgang von Goethe, Hugo von Hofmannsthal, Heinrich Kanner, Alfred Kerr, Christian Schefer, Leopold Sonnemann}
         \newcommand{\erwaehnteInstitutionen}{Institutionen: Frankfurter Zeitung}
         \newcommand{\erwaehnteOrte}{Orte: Paris, Wien, rue Feydeau}
         \newcommand{\erwaehnteWerke}{Werke: Arthur Schnitzler, Die Zeit. Wiener Wochenschrift, Frankfurter Zeitung, Neue Deutsche Rundschau, Poesie und Leben. Aus einem Vortrage, Sterben. Novelle}
               \section[ Paul Goldmann an Arthur Schnitzler, 24. 5. {[}1896{]}]{ Paul Goldmann an Arthur Schnitzler, 24. 5. {[}1896{]}}\nopagebreak\mylabel{v}\rehead{ }\begin{ledgroupsized}[t]{13cm}\normalsize\beginnumbering \toendnotes[C]{\smallbreak\pagebreak[2]} \Standort{DLA, A:Schnitzler, HS.NZ85.1.3166.}
\physDesc{Brief, 2 Blätter, 8 Seiten
\newline{}Handschrift: blaue Tinte, deutsche Kurrent
\newline{}Schnitzler: 1) mit Bleistift das Jahr »96« vermerkt  2) mit rotem Buntstift auf der ersten Seite »\textsc{Kerr\pwindex{Kerr, Alfred 25.12.1867 – 12.10.1948@\textsc{Kerr, Alfred} (25.12.1867 – 12.10.1948), \emph{Schriftsteller, Kritiker}|pw}}« vermerkt und insgesamt drei Unterstreichungen}\toendnotes[C]{\smallbreak}\pstart
           \noindent{}{\pb}\textcolor{gray}{\textbf{\textbf{Frankfurter Zeitung\orgindex{Frankfurter Zeitung@Frankfurter Zeitung|pw}}}}\pend
           \pstart
           \textcolor{gray}{\textbf{(\begin{otherlanguage}{french}Gazette de Francfort\end{otherlanguage}\orgindex{Frankfurter Zeitung@Frankfurter Zeitung|pw}).}}\pend
           \pstart
           \textcolor{gray}{\textbf{\textbf{\begin{otherlanguage}{french}Fondateur M.\end{otherlanguage}{ }L. Sonnemann\pwindex{Sonnemann, Leopold 1831-10-29 – 1909-10-30@\textsc{Sonnemann, Leopold} (1831-10-29 – 1909-10-30), \emph{Journalist, Herausgeber}|pw}.}}}\pend
           \pstart
           \begin{otherlanguage}{french}\textcolor{gray}{\textbf{Journal\pwindex{?? Werk@Nicht ermittelte Verfasserinnen und Verfasser!Frankfurter Zeitung1856 – 1943@\emph{Frankfurter Zeitung} {[}1856 – 1943{]}|pwv} politique,
                        financier,}}\end{otherlanguage}\pend
           \pstart
           \begin{otherlanguage}{french}\textcolor{gray}{\textbf{commercial et littéraire.}}\end{otherlanguage}\pend
           \pstart
           \begin{otherlanguage}{french}\textcolor{gray}{\textbf{\textbf{Paraissant trois fois par jour.}}}\end{otherlanguage}\pend
           \pstart
           \begin{otherlanguage}{french}\textcolor{gray}{\textbf{\textbf{Bureau à Paris\oindex{Paris@\textbf{Paris}|pw}}}}\end{otherlanguage}\hfill \textsc{Paris\oindex{Paris@\textbf{Paris}|pw}}, 24. Mai.\pend
           \pstart
           \begin{otherlanguage}{french}\textcolor{gray}{\textbf{\textbf{24. Rue Feydeau\oindex{rue Feydeau@\textbf{rue Feydeau}|pw}.}}}\end{otherlanguage}\pend
           \pstart\center{}Mein lieber Freund,\pend\pstart
           Vielen Dank für die »Freie
                  Bühne\pwindex{Neue Deutsche Rundschau1894-01-01 – 1903-12-31@\emph{Neue Deutsche Rundschau} {[}1894-01-01 – 1903-12-31{]}|pwv}«, die ich anbei zurückſende. (Das heißt nicht »anbei«. Ich behalte ſie
               noch bis Dienſtag, um ſie \textsc{M. Schefer\pwindex{Schefer, Christian 1866-07-14 – Februar 1944@\textsc{Schefer, Christian} (1866-07-14 – Februar 1944), \emph{Journalist, Lehrer}|pw}} zu zeigen, der mich an dieſem Tage beſuchen
               kommt). Der \label{K_L02775-88v}\edtext{Artikel\pwindex{Kerr, Alfred 25.12.1867 – 12.10.1948@\textsc{Kerr, Alfred} (25.12.1867 – 12.10.1948), \emph{Schriftsteller, Kritiker}!Arthur Schnitzler1896-03@\strich\emph{Arthur Schnitzler} {[}1896-03{]}|pwv}}{\lemma{\textnormal{\emph{Artikel}}}\Cendnote{\textnormal{Alfred Kerr\pwindex{Kerr, Alfred 25.12.1867 – 12.10.1948@\textsc{Kerr, Alfred} (25.12.1867 – 12.10.1948), \emph{Schriftsteller, Kritiker}|pwk}: \emph{Arthur Schnitzler}\pwindex{Kerr, Alfred 25.12.1867 – 12.10.1948@\textsc{Kerr, Alfred} (25.12.1867 – 12.10.1948), \emph{Schriftsteller, Kritiker}!Arthur Schnitzler1896-03@\strich\emph{Arthur Schnitzler} {[}1896-03{]}|pwk}. In: \emph{Neue Deutsche Rundschau (Freie Bühne)}\pwindex{Neue Deutsche Rundschau1894-01-01 – 1903-12-31@\emph{Neue Deutsche Rundschau} {[}1894-01-01 – 1903-12-31{]}|pwk}, Jg. 7, H. 3, März 1896, S. 287–292.}}}\label{K_L02775-88h} iſt höchſt
               intereſſant. Ich freue mich über den ſchönen Enthuſiasmus, den mein lieber \textsc{Arthur} erregt. Auch ſagt der {\pb}Verfaſſer\pwindex{Kerr, Alfred 25.12.1867 – 12.10.1948@\textsc{Kerr, Alfred} (25.12.1867 – 12.10.1948), \emph{Schriftsteller, Kritiker}|pwv} manches Richtige.
               Im Allgemeinen aber ſind \strikeout{\textcolor{gray}{E}} mir ſeine kraftgenialiſche Art und Styl nicht ſehr ſympathiſch.\pend
           \pstart
           \label{K_L02775-24v}\edtext{Beifolgenden Brief}{\lemma{\textnormal{\emph{Beifolgenden Brief}}}\Cendnote{\textnormal{Beilage nicht erhalten, Verfasser nicht
                  identifiziert}}}\label{K_L02775-24h} empfehle ich \strikeout{D\textcolor{gray}{ic}h} Dir aufs Wärmſte zur bejahenden Beantwortung.
               Verfaſſer iſt ein Vetter\pwindex{?? [Vetter von Heinrich Kanner] @\textsc{?? [Vetter von Heinrich Kanner]}|pwv} von
                  \textsc{Kanner\pwindex{Kanner, Heinrich 09.11.1864 – 15.02.1930@\textsc{Kanner, Heinrich} (09.11.1864 – 15.02.1930), \emph{Herausgeber, Publizist}|pw}} – kreuzbraver Menſch\pwindex{?? [Vetter von Heinrich Kanner] @\textsc{?? [Vetter von Heinrich Kanner]}|pwv} –
               ſelbſt ſchwer lungenleidend, der wohl im »Sterben\pwindex{Schnitzler, Arthur 15.05.1862 – 21.10.1931@\textsc{Schnitzler, Arthur} (15.05.1862 – 21.10.1931), \emph{Schriftsteller, Mediziner}!Sterben. Novelle1894-10-01 – 1894-12-01@\strich\emph{Sterben. Novelle} {[}1894-10-01 – 1894-12-01{]}|pw}« ein Stück {\pb}ſeines Schickſals
               gefunden hat.\pend
           \pstart
           Über den \label{K_L02775-13v}\edtext{»Vortrag\pwindex{Hofmannsthal, Hugo von 1874-02-01 – 1929-07-15@\textsc{Hofmannsthal, Hugo von} (1874-02-01 – 1929-07-15), \emph{Schriftsteller}!Poesie und Leben. Aus einem Vortrage1896-05-16@\strich\emph{Poesie und Leben. Aus einem Vortrage} {[}1896-05-16{]}|pwv}«}{\lemma{\textnormal{\emph{»Vortrag«}}}\Cendnote{\textnormal{Hugo von Hofmannsthal\pwindex{Hofmannsthal, Hugo von 1874-02-01 – 1929-07-15@\textsc{Hofmannsthal, Hugo von} (1874-02-01 – 1929-07-15), \emph{Schriftsteller}|pwk}: \emph{Poesie und Leben}\pwindex{Hofmannsthal, Hugo von 1874-02-01 – 1929-07-15@\textsc{Hofmannsthal, Hugo von} (1874-02-01 – 1929-07-15), \emph{Schriftsteller}!Poesie und Leben. Aus einem Vortrage1896-05-16@\strich\emph{Poesie und Leben. Aus einem Vortrage} {[}1896-05-16{]}|pwk}. In: \emph{Die Zeit}\pwindex{Zeit. Wiener Wochenschrift1894 – 1904@\emph{Die Zeit. Wiener Wochenschrift} {[}1894 – 1904{]}|pwk}, Bd. 7, Nr. 85, 16. 5. 1896,
                     S. 104–106.}}}\label{K_L02775-13h} von \textsc{Loris\pwindex{Hofmannsthal, Hugo von 1874-02-01 – 1929-07-15@\textsc{Hofmannsthal, Hugo von} (1874-02-01 – 1929-07-15), \emph{Schriftsteller}|pw}}, den die letzte »Zeit\pwindex{Zeit. Wiener Wochenschrift1894 – 1904@\emph{Die Zeit. Wiener Wochenschrift} {[}1894 – 1904{]}|pw}« gebracht, war ich
               wüthend. Ich verſtehe nicht ein Wort von dem, was er will. Und dann Stellen, wie:
                  »Eine neue und kühne Verbindung
                  von Worten iſt das wundervollſte Geſchenk für die Seelen und nichts geringeres als
                  ein Standbild des Knaben \textsc{Antinous\pwindex{Antinoos @\textsc{Antinoos}|pw}} oder eine große gewölbte Pforte\pwindex{Hofmannsthal, Hugo von 1874-02-01 – 1929-07-15@\textsc{Hofmannsthal, Hugo von} (1874-02-01 – 1929-07-15), \emph{Schriftsteller}!Poesie und Leben. Aus einem Vortrage1896-05-16@\strich\emph{Poesie und Leben. Aus einem Vortrage} {[}1896-05-16{]}|pwv}«. Das iſt doch unerhört! Was iſt eine
               große gewölbte {\pb}Pforte für die Seelen? Und was hat
               das, zum Teufel, mit dem Standbild des Knaben \textsc{Antinous\pwindex{Antinoos @\textsc{Antinoos}|pw}} zu thun? Ich will nicht ausſchließen, daß das wirklich empfunden iſt. Aber wenn
               auch – ſo thut das eine ganz unerhörte Empfindungen-Verwirrung dar. Auch iſt es eine
               verfluchte Schlamperei, ſich ſo gehen zu laſſen und jede \label{K_L02775-22v}\edtext{\textsc{\begin{otherlanguage}{french}incohérence\end{otherlanguage}}}{\lemma{\textnormal{\emph{incohérence}}}\Cendnote{\textnormal{französisch: mangelnder
                  Zusammenhang}}}\label{K_L02775-22h} auszuſprechen, die Einem durchs Hirn fährt, \strikeout{die \textcolor{gray}{×}\-\textcolor{gray}{×}\-\textcolor{gray}{×}\-\textcolor{gray}{×}\-\textcolor{gray}{×}\-\textcolor{gray}{×}\-\textcolor{gray}{×}{ }\textcolor{gray}{×}\-\textcolor{gray}{×}\-\textcolor{gray}{×}\-\textcolor{gray}{×}{ }\textcolor{gray}{wird}} in der Überzeugung, das {\pb}ſei genial. Auch
               wird die Literatur auf dieſe Weiſe zu einer Geheim-Sprache, die nur mehr ein paar
               Eingeweihte verſtehen. Dieſer junge Mann\pwindex{Hofmannsthal, Hugo von 1874-02-01 – 1929-07-15@\textsc{Hofmannsthal, Hugo von} (1874-02-01 – 1929-07-15), \emph{Schriftsteller}|pwv} ſchreibt doch fürs Publicum. Und wenn er ſich nicht mehr
               ſo ausdrücken kann, daß ihn das Publicum versteht – wenn ſeine Gedanken einen Flug
               nehmen, {\pb}wo die Maſſe ihm nicht nach kann und wo er
               ſelbſt kaum noch mit kann – dann ſoll er eben \strikeout{kei\textcolor{gray}{n}} nichts mehr drucken laſſen und keine Vorträge halten. Hübſch iſt auch, daß es
               einmal heißt, »bei den neueren
                  deutſchen ſogenannten Dichtern\pwindex{Hofmannsthal, Hugo von 1874-02-01 – 1929-07-15@\textsc{Hofmannsthal, Hugo von} (1874-02-01 – 1929-07-15), \emph{Schriftsteller}!Poesie und Leben. Aus einem Vortrage1896-05-16@\strich\emph{Poesie und Leben. Aus einem Vortrage} {[}1896-05-16{]}|pwv}«. Und weiter unten: »Sie wundern ſich, daß Ihnen \uline{ein Dichter} die Regeln lobt \textsc{etc.}\pwindex{Hofmannsthal, Hugo von 1874-02-01 – 1929-07-15@\textsc{Hofmannsthal, Hugo von} (1874-02-01 – 1929-07-15), \emph{Schriftsteller}!Poesie und Leben. Aus einem Vortrage1896-05-16@\strich\emph{Poesie und Leben. Aus einem Vortrage} {[}1896-05-16{]}|pwv}« Alſo größenwahnſinnig {\pb}iſt dieſer junge Mann\pwindex{Hofmannsthal, Hugo von 1874-02-01 – 1929-07-15@\textsc{Hofmannsthal, Hugo von} (1874-02-01 – 1929-07-15), \emph{Schriftsteller}|pwv} auch ſchon. Worauf hin?
               Mit dem »jungen \textsc{Goethe\pwindex{Goethe, Johann Wolfgang von 1749-08-28 – 1832-03-22@\textsc{Goethe, Johann Wolfgang von} (1749-08-28 – 1832-03-22), \emph{Schriftsteller}|pw}}« iſt es bisher nichts geworden. Bisher hat es eigentlich nur in einem Punkte
               geſtimmt: in der Jugend.\pend
           \pstart
           Nein, iſt dieſer arme kleine Burſch\pwindex{Hofmannsthal, Hugo von 1874-02-01 – 1929-07-15@\textsc{Hofmannsthal, Hugo von} (1874-02-01 – 1929-07-15), \emph{Schriftsteller}|pwv} verdorben worden\strikeout{!} von \textsc{Bahr\pwindex{Bahr, Hermann 19.07.1863 – 15.01.1934@\textsc{Bahr, Hermann} (19.07.1863 – 15.01.1934), \emph{Schriftsteller, Kritiker}|pw}}, dieſem verfluchten Pfuſcher und Schurken!\pend
           \pstart
           {\pb}Grüß’ Dich Gott, liebſter Freund.\pend
           \pstart
           Auch ſchreibſt Du mir wohl nächſtens einmal.\pend
           \pstart
           Dein {\\[\baselineskip]}treuer {\\[\baselineskip]}\spacefill\mbox{Paul Goldmann}\pend
           \leftskip=0em{}
         
         \endnumbering\mylabel{h}\end{ledgroupsized}  \newcommand{\dateiname}{L02775}\newcommand{\titel}{Paul Goldmann an Arthur Schnitzler, 24. 5. [1896]}\newcommand{\editorInnen}{Martin Anton Müller und Laura Untner}%% latex-leseansicht-abspann.tex
%% Abspann für die Leseansicht.
%% Der Schalter \ifkorrekturansicht ist bereits durch den Vorspann gesetzt.

%% latex-abspann.tex
%% Gemeinsamer Abspann für Korrekturansicht und Leseansicht.
%% Setzt den Schalter \ifkorrekturansicht voraus (gesetzt in den
%% einbindenden Dateien latex-korrekturansicht-abspann.tex bzw.
%% latex-leseansicht-abspann.tex).
%% ---------------------------------------------------------------

\normalsize

% Das esempio-Environment wird nur in der Leseansicht benötigt
\ifkorrekturansicht\else
\newenvironment{esempio}[3]%
{
    \vspace{1.5ex}
    \rlap{\underline{#1}}
    \par
    \setlength{\parindent}{0cm}
    \nopagebreak
    \leftskip=#2cm
    \rightskip=#3cm
}
{
    \par
}
\fi

\doendnotes{C}
\bigskip
\vfill

\clearpage

\footnotesize

\ifkorrekturansicht
  \lohead{\textsc{register}}
\fi

% theindex-Environment neu definieren ohne reledmac
\makeatletter
\renewenvironment{theindex}{%
  \ifkorrekturansicht
    \section*{\indexname}%
  \else
    \subsubsection*{Index der erwähnten Entitäten}%
  \fi
  \setlength{\parindent}{0pt}%
  \setlength{\parskip}{0pt plus 0.3pt}%
  \let\item\@idxitem
}{%
  \ifkorrekturansicht\clearpage\fi
}
\makeatother

\IfFileExists{\jobname-pw.ind}{\input{\jobname-pw.ind}}{}

% Quellenangabe nur in der Leseansicht
\ifkorrekturansicht\else
% Fallback-Definitionen, falls die .tex-Datei \titel etc. nicht gesetzt hat
\providecommand{\titel}{}
\providecommand{\editorInnen}{}
\providecommand{\dateiname}{\jobname}

\vspace{3cm}

\vfill

\footnotesize
\textsc{Quelle}: \titel. Herausgegeben von {\editorInnen}. In: \emph{Arthur Schnitzler: Briefwechsel mit Autorinnen und Autoren}.
 Digitale Edition, https://schnitzler-briefe.acdh.oeaw.ac.at/{\dateiname}.html (Stand \today)
\fi

\end{document}


      