%% latex-korrekturansicht-vorspann.tex
%% Vorspann für die Korrekturansicht.
%% Lädt die gemeinsame Datei latex-vorspann.tex mit gesetztem Schalter.

\newif\ifkorrekturansicht
\korrekturansichttrue

\input{../tex-inputs/latex-vorspann}


\section[Paul Goldmann an Arthur Schnitzler, 27. 4. {[}1902?{]}]{L02633 Paul Goldmann an Arthur Schnitzler, 27. 4. {[}1902?{]}}
\nopagebreak\mylabel{L02633v}
\rehead{ }\normalsize\beginnumbering\briefempfaengerindex{Schnitzler, Arthur@\textsc{Schnitzler, Arthur}!zzzGoldmann, Paul@\emph{von Paul Goldmann}!1902-04-271@{27. 4. {[}1902?{]}}|(be}
\toendnotes[C]{\smallbreak\pagebreak[2]}\Standort{DLA, A:Schnitzler, HS.NZ85.1.3170.}
\physDesc{Telegramm, 191 Zeichen
\newline{}maschinell
\newline{}Versand: 1) Stempel: »\nobreak{}\oindex{IX., Alsergrund@\textbf{IX., Alsergrund}, \emph{A.ADM3}|pwk}{[}Wien{]} 9/2, 27 IV 00\nobreak{}«.   2) »\noindent{}\textcolor{gray}{\textbf{\textit{27 Apr}}}{ / }\textcolor{gray}{\textbf{\textit{Zaunegger\pwindex{Zaunegger, Anton @\textsc{Zaunegger, Anton}, \emph{Telegrafenbeamter/Telegrafenbeamtin}|pw}}}}{ / }\textcolor{gray}{\textbf{\textit{Ausgefertigt 27 Apr{ }8 10}}}« 3) mit Bleistift von unbekannter Hand Vermerk des Postrayons:
                                    »71«
\newline{}Ordnung: beschnitten }\toendnotes[C]{\smallbreak}\pstart{}{\pb}= arthur schnitzler wien\oindex{Wien@\textbf{Wien}, \emph{A.ADM2}|pw}\pend{}\pstart{}neuntbezirk frankgasse\oindex{Frankgasse 1@\textbf{Frankgasse 1}, \emph{Wohngebäude (K.WHS)}|pw} =\pend{}{\bigskip}\vspace{1em}
\pstart
           {\pb}v{ }berlin\oindex{Berlin@\textbf{Berlin}, \emph{P.PPLC}|pw} 68646 24 27{ }6 33 S\pend
           \vspace{0.5em}
\pstart
           ich glaube nicht dasz die \label{K_L02633-1v}\edtext{notizen\pwindex{litterarisch-dramatisches Hochstapler-Stuecklein@\emph{Ein litterarisch-dramatisches Hochstapler-Stücklein}|pwv}}{\lemma{\textnormal{\emph{notizen}}}\Cendnote{\textnormal{Dieses Telegramm ist im Nachlass den
                  Korrespondenzstücken des Jahres 1900 zugeordnet. Die Datierung dürfte
                  auf den abgeschnitten überlieferten Stempel zurückgehen, der sichtbar die
                  Zeichenfolge »27 IV 00« enthält. Ob es sich dabei um einen falsch
                  eingestellten Stempel handelt oder ob es hier um Reste der Uhrzeit geht, bleibt
                  unklar. Das Telegramm dürfte jedenfalls zu jenem des Vortags (Paul Goldmann an Arthur Schnitzler, 26. 4. 1902) gehören.}}}\label{K_L02633-1} irgendwelche folgen haben werden;
               sie sind nur taktlos und albern. herzlichst = \spacefill\mbox{goldmann}\pend
           \selectlanguage{ngerman}\endnumbering\briefempfaengerindex{Schnitzler, Arthur@\textsc{Schnitzler, Arthur}!zzzGoldmann, Paul@\emph{von Paul Goldmann}!1902-04-271@{27. 4. {[}1902?{]}}|)be}\mylabel{L02633h}  \normalsize

\doendnotes{C}
\bigskip
\vfill

\clearpage

\footnotesize

\lohead{\textsc{register}}

% Definiere theindex-Environment komplett neu ohne reledmac
\makeatletter
\renewenvironment{theindex}{%
  \section*{\indexname}%
  \setlength{\parindent}{0pt}%
  \setlength{\parskip}{0pt plus 0.3pt}%
  \let\item\@idxitem
}{%
  \clearpage
}
\makeatother

\IfFileExists{\jobname-pw.ind}{\input{\jobname-pw.ind}}{}

\end{document}

      