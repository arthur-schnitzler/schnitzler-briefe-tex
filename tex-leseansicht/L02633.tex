%% latex-leseansicht-vorspann.tex
%% Vorspann für die Leseansicht.
%% Lädt die gemeinsame Datei latex-vorspann.tex mit nicht gesetztem Schalter.

\newif\ifkorrekturansicht
\korrekturansichtfalse

\input{../tex-inputs/latex-vorspann}


\section[Paul Goldmann an Arthur Schnitzler, 27. 4. {[}1902?{]}]{L02633 Paul Goldmann an Arthur Schnitzler, 27. 4. [1902?]}
\nopagebreak\mylabel{L02633v}
\rehead{ }\normalsize\beginnumbering\briefempfaengerindex{Schnitzler, Arthur@\textsc{Schnitzler, Arthur}!zzzGoldmann, Paul@\emph{von Paul Goldmann}!1902-04-271@{27. 4. [1902?]}|(be}
\toendnotes[C]{\smallbreak\pagebreak[2]}
\correspDesc{Versand  durch Paul Goldmann am 27. 4. [1902?] in Berlin
\newline{}Erhalt  durch Arthur Schnitzler am 27. 4. [1902?] in Wien}\toendnotes[C]{\smallbreak}
\Standort{DLA, A:Schnitzler, HS.NZ85.1.3170.}
\physDesc{Telegramm, 191 Zeichen
\newline{}maschinell
\newline{}Versand: 1) Stempel: »\nobreak{}\oindex{IX., Alsergrund@\textbf{IX., Alsergrund}, \emph{Verwaltungsgebiet}|pwk}{[}Wien{]} 9/2, 27 IV 00\nobreak{}«.   2) »\noindent{}\textcolor{gray}{\textbf{\textit{27 Apr}}}{ / }\textcolor{gray}{\textbf{\textit{Zaunegger\pwindex{Zaunegger, Anton @\textsc{Zaunegger, Anton}, \emph{Telegrafenbeamter}|pw}}}}{ / }\textcolor{gray}{\textbf{\textit{Ausgefertigt 27 Apr{ }8 10}}}« 3) mit Bleistift von unbekannter Hand Vermerk des Postrayons:
                                    »71«
\newline{}Ordnung: beschnitten }\toendnotes[C]{\smallbreak}\pstart{}{\pb}= arthur schnitzler wien\oindex{Wien@\textbf{Wien}, \emph{Verwaltungsgebiet}|pw}\pend{}\pstart{}neuntbezirk frankgasse\oindex{Wien@\textbf{Wien}!IX., Alsergrund@\textbf{IX., Alsergrund}!Frankgasse 1@\textbf{Frankgasse 1}, \emph{Wohngebäude}|pw} =\pend{}{\bigskip}\vspace{1em}
\pstart
           {\pb}v{ }berlin\oindex{Berlin@\textbf{Berlin}, \emph{Hauptstadt}|pw} 68646 24 27{ }6 33 S\pend
           \vspace{0.5em}
\pstart
           ich glaube nicht dasz die \label{K_L02633-1v}\edtext{notizen\pwindex{\textcolor{red}{\textsuperscript{XXXX indx1}}!litterarisch-dramatisches Hochstapler-Stücklein@\strich\emph{Ein litterarisch-dramatisches Hochstapler-Stücklein}|pwv}}{\lemma{\textnormal{\emph{notizen}}}\Cendnote{\textnormal{Dieses Telegramm ist im Nachlass den
                  Korrespondenzstücken des Jahres 1900 zugeordnet. Die Datierung dürfte
                  auf den abgeschnitten überlieferten Stempel zurückgehen, der sichtbar die
                  Zeichenfolge »27 IV 00« enthält. Ob es sich dabei um einen falsch
                  eingestellten Stempel handelt oder ob es hier um Reste der Uhrzeit geht, bleibt
                  unklar. Das Telegramm dürfte jedenfalls zu jenem des Vortags (XXXX Auszeichnungsfehler: Dokument L02634 nicht gefunden) gehören.}}}\label{K_L02633-1} irgendwelche folgen haben werden;
               sie sind nur taktlos und albern. herzlichst = \spacefill\mbox{goldmann}\pend
           \selectlanguage{ngerman}\endnumbering\briefempfaengerindex{Schnitzler, Arthur@\textsc{Schnitzler, Arthur}!zzzGoldmann, Paul@\emph{von Paul Goldmann}!1902-04-271@{27. 4. [1902?]}|)be}\mylabel{L02633h}  \newcommand{\dateiname}{L02633}\newcommand{\titel}{Paul Goldmann an Arthur Schnitzler, 27. 4. [1902?]}\newcommand{\editorInnen}{Martin Anton Müller und Laura Untner}%% latex-leseansicht-abspann.tex
%% Abspann für die Leseansicht.
%% Der Schalter \ifkorrekturansicht ist bereits durch den Vorspann gesetzt.

%% latex-abspann.tex
%% Gemeinsamer Abspann für Korrekturansicht und Leseansicht.
%% Setzt den Schalter \ifkorrekturansicht voraus (gesetzt in den
%% einbindenden Dateien latex-korrekturansicht-abspann.tex bzw.
%% latex-leseansicht-abspann.tex).
%% ---------------------------------------------------------------

\normalsize

% Das esempio-Environment wird nur in der Leseansicht benötigt
\ifkorrekturansicht\else
\newenvironment{esempio}[3]%
{
    \vspace{1.5ex}
    \rlap{\underline{#1}}
    \par
    \setlength{\parindent}{0cm}
    \nopagebreak
    \leftskip=#2cm
    \rightskip=#3cm
}
{
    \par
}
\fi

\doendnotes{C}
\bigskip
\vfill

\clearpage

\footnotesize

\ifkorrekturansicht
  \lohead{\textsc{register}}
\fi

% theindex-Environment neu definieren ohne reledmac
\makeatletter
\renewenvironment{theindex}{%
  \ifkorrekturansicht
    \section*{\indexname}%
  \else
    \subsubsection*{Index der erwähnten Entitäten}%
  \fi
  \setlength{\parindent}{0pt}%
  \setlength{\parskip}{0pt plus 0.3pt}%
  \let\item\@idxitem
}{%
  \ifkorrekturansicht\clearpage\fi
}
\makeatother

\IfFileExists{\jobname-pw.ind}{\input{\jobname-pw.ind}}{}

% Quellenangabe nur in der Leseansicht
\ifkorrekturansicht\else
% Fallback-Definitionen, falls die .tex-Datei \titel etc. nicht gesetzt hat
\providecommand{\titel}{}
\providecommand{\editorInnen}{}
\providecommand{\dateiname}{\jobname}

\vspace{3cm}

\vfill

\footnotesize
\textsc{Quelle}: \titel. Herausgegeben von {\editorInnen}. In: \emph{Arthur Schnitzler: Briefwechsel mit Autorinnen und Autoren}.
 Digitale Edition, https://schnitzler-briefe.acdh.oeaw.ac.at/{\dateiname}.html (Stand \today)
\fi

\end{document}


