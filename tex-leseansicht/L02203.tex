%% latex-korrekturansicht-vorspann.tex
%% Vorspann für die Korrekturansicht.
%% Lädt die gemeinsame Datei latex-vorspann.tex mit gesetztem Schalter.

\newif\ifkorrekturansicht
\korrekturansichttrue

\input{../tex-inputs/latex-vorspann}


\section[Arthur Schnitzler an Hermann Bahr, 9. 2. 1915]{L02203 Arthur Schnitzler an Hermann Bahr, 9. 2. 1915}
\nopagebreak\mylabel{L02203v}
\rehead{ }\normalsize\beginnumbering\briefempfaengerindex{Bahr, Hermann@\textsc{Bahr, Hermann}!zzzSchnitzler, Arthur@\emph{von Arthur Schnitzler}!1915-02-091@{9. 2. 1915}|(be}
\toendnotes[C]{\smallbreak\pagebreak[2]}\Standort{TMW, HS AM 60138 Ba.}
\physDesc{Briefkarte, 753 Zeichen
\newline{}Handschrift: schwarze Tinte, deutsche Kurrent}
\buchAbdrucke{\weitereDrucke{1) Arthur Schnitzler: \emph{The Letters of Arthur Schnitzler to Hermann Bahr}. Chapel Hill: \emph{The University of North Carolina Press} 1978, S. 114.} \weitereDrucke{2) Hermann Bahr, Arthur Schnitzler: \emph{Briefwechsel, Aufzeichnungen, Dokumente (1891–1931)}. Göttingen: \emph{Wallstein} 2018, S. 497.} }\toendnotes[C]{\smallbreak}
\pstart
           {\pb}\textcolor{gray}{\textbf{Dr. Arthur Schnitzler}}\hfill 9. 2. 915\pend
           
\pstart
           \textcolor{gray}{\textbf{Wien XVIII. Sternwartestrasse 71\oindex{Sternwartestrasse 71@\textbf{Sternwartestraße 71}, \emph{Wohngebäude (K.WHS)}|pw}}}\pend
           \vspace{0.5em}
\pstart
           lieber Hermann, der Buchhändler Heller\pwindex{Heller, Hugo 08.05.1870 – 29.11.1923@\textsc{Heller, Hugo} (08.05.1870 – 29.11.1923), \emph{Verleger/Verlegerin, Buchhändler/Buchhändlerin}|pw} theilt mir mit daſs er deiner verehrten Gattin\pwindex{Bahr-Mildenburg, Anna 29.11.1872 – 27.01.1947@\textsc{Bahr-Mildenburg, Anna} (29.11.1872 – 27.01.1947), \emph{Sänger/Sängerin}|pwv}{ }\label{K_L02203-1v}\edtext{geſchrieben}{\lemma{\textnormal{\emph{geſchrieben}}}\Cendnote{\textnormal{am 6. 2. 1915 (\emph{Theatermuseum Wien}, AM 27.957 BaM.)}}}\label{K_L02203-1}, ob ſie
               hier nicht zu einem \label{K_L02203-2v}\edtext{wohlthätigen
                  Zwecke}{\lemma{\textnormal{\emph{wohlthätigen
                  Zwecke}}}\Cendnote{\textnormal{Vgl. A. S.: \emph{Tagebuch}, 13. 12. 1915.
               }}}\label{K_L02203-2}{ }Schubert\pwindex{Schubert, Franz Peter 31.01.1797 – 19.11.1828@\textsc{Schubert, Franz Peter} (31.01.1797 – 19.11.1828), \emph{Komponist/Komponistin}|pw} Lieder ſingen möchte – und da ich
               daraufhin mich begreiflicherweiſe äußerte: das möcht ich gern hören, – bittet er
               mich, \strikeout{als} dieſen Wunſch, dieſe Sehnſucht {\pb}(ich theile ſie
               wahrſcheinlich mit vielen) dir direct zu übermitteln. Das thu ich – in der Empfindung
               etwas unbeſcheiden – aber doch deiner Nachſicht gewiſs zu sein. Im übrigen wär es,
               auch abgeſehn von den Schubert\pwindex{Schubert, Franz Peter 31.01.1797 – 19.11.1828@\textsc{Schubert, Franz Peter} (31.01.1797 – 19.11.1828), \emph{Komponist/Komponistin}|pw} Liedern, die
               deine Frau\pwindex{Bahr-Mildenburg, Anna 29.11.1872 – 27.01.1947@\textsc{Bahr-Mildenburg, Anna} (29.11.1872 – 27.01.1947), \emph{Sänger/Sängerin}|pwv}{ }ſo herrlich ſingen ſoll, ſchön, we{\geminationn} man ſich wieder einmal ſehen und ſprechen kö{\geminationn}te – in dieſer – Zeit, für die das Adjectiv doch erſt
               gefunden werden müſſte!\pend
           
\pstart
           Von Herzen mit Grüßen von Haus zu Haus{\\[\baselineskip]}dein \spacefill\mbox{Arthur}\pend
           \leftskip=0em{}\selectlanguage{ngerman}\endnumbering\briefempfaengerindex{Bahr, Hermann@\textsc{Bahr, Hermann}!zzzSchnitzler, Arthur@\emph{von Arthur Schnitzler}!1915-02-091@{9. 2. 1915}|)be}\mylabel{L02203h}  \normalsize

\doendnotes{C}
\bigskip
\vfill

\clearpage

\footnotesize

\lohead{\textsc{register}}

% Definiere theindex-Environment komplett neu ohne reledmac
\makeatletter
\renewenvironment{theindex}{%
  \section*{\indexname}%
  \setlength{\parindent}{0pt}%
  \setlength{\parskip}{0pt plus 0.3pt}%
  \let\item\@idxitem
}{%
  \clearpage
}
\makeatother

\IfFileExists{\jobname-pw.ind}{\input{\jobname-pw.ind}}{}

\end{document}

      