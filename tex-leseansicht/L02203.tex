%% latex-leseansicht-vorspann.tex
%% Vorspann für die Leseansicht.
%% Lädt die gemeinsame Datei latex-vorspann.tex mit nicht gesetztem Schalter.

\newif\ifkorrekturansicht
\korrekturansichtfalse

\input{../tex-inputs/latex-vorspann}


\section[Arthur Schnitzler an Hermann Bahr, 9. 2. 1915]{L02203 Arthur Schnitzler an Hermann Bahr, 9. 2. 1915}
\nopagebreak\mylabel{L02203v}
\rehead{ }\normalsize\beginnumbering\briefempfaengerindex{Bahr, Hermann@\textsc{Bahr, Hermann}!zzzSchnitzler, Arthur@\emph{von Arthur Schnitzler}!1915-02-091@{9. 2. 1915}|(be}
\toendnotes[C]{\smallbreak\pagebreak[2]}
\correspDesc{Versand  durch Arthur Schnitzler am 9. 2. 1915 in Wien
\newline{}Erhalt  durch Hermann Bahr im Zeitraum [10. 2. 1915
                  – 14. 2. 1915?] in Salzburg}\toendnotes[C]{\smallbreak}
\Standort{TMW, HS AM 60138 Ba.}
\physDesc{Briefkarte, 753 Zeichen
\newline{}Handschrift: schwarze Tinte, deutsche Kurrent}
\buchAbdrucke{\weitereDrucke{1) \emph{9. 2. 1915.} In: Arthur Schnitzler: \emph{The Letters of Arthur Schnitzler to Hermann Bahr}. Edited, annotated, and with an introduction, by Donald G. Daviau. Chapel Hill: \emph{The University of North Carolina Press} 1978, S. 114 (University of North Carolina studies in the Germanic languages
                        and literatures, 89).} \weitereDrucke{2) Hermann Bahr, Arthur Schnitzler: \emph{Briefwechsel, Aufzeichnungen, Dokumente (1891–1931)}. Herausgegeben von Kurt Ifkovits und Martin Anton Müller. Göttingen: \emph{Wallstein} 2018, S. 497.} }\toendnotes[C]{\smallbreak}
\pstart
           {\pb}\textcolor{gray}{\textbf{Dr. Arthur Schnitzler}}\hfill 9. 2. 915\pend
           
\pstart
           \textcolor{gray}{\textbf{Wien XVIII. Sternwartestrasse 71\oindex{Wien@\textbf{Wien}!XVIII., Währing@\textbf{XVIII., Währing}!Sternwartestraße 71@\textbf{Sternwartestraße 71}, \emph{Wohngebäude}|pw}}}\pend
           \vspace{0.5em}
\pstart
           lieber Hermann, der Buchhändler Heller\pwindex{Heller, Hugo 8.\,5.\,1870 Székesfehérvár – 29.\,11.\,1923 Wien@\textsc{Heller, Hugo} (8.\,5.\,1870 Székesfehérvár – 29.\,11.\,1923 Wien), \emph{Verleger, Buchhändler}|pw} theilt mir mit daſs er deiner verehrten Gattin\pwindex{Bahr-Mildenburg, Anna 29.\,11.\,1872 Wien – 27.\,1.\,1947 ebd.@\textsc{Bahr-Mildenburg, Anna} (29.\,11.\,1872 Wien – 27.\,1.\,1947 ebd.), \emph{Sängerin}|pwv}{ }\label{K_L02203-1v}\edtext{geſchrieben}{\lemma{\textnormal{\emph{geschrieben}}}\Cendnote{\textnormal{am 6. 2. 1915 (\emph{Theatermuseum Wien}, AM 27.957 BaM.)}}}\label{K_L02203-1}, ob{ }ſie
               hier nicht zu einem \label{K_L02203-2v}\edtext{wohlthätigen
                  Zwecke}{\lemma{\textnormal{\emph{wohlthätigen
                  Zwecke}}}\Cendnote{\textnormal{Vgl. A. S.: \emph{Kulturveranstaltungen}, 13. 12. 1915.
               }}}\label{K_L02203-2}{ }Schubert\pwindex{Schubert, Franz Peter 31.\,1.\,1797 Lichtental [Wien] – 19.\,11.\,1828 Wien@\textsc{Schubert, Franz Peter} (31.\,1.\,1797 Lichtental [Wien] – 19.\,11.\,1828 Wien), \emph{Komponist}|pw} Lieder{ }ſingen möchte – und da ich
               daraufhin mich begreiflicherweiſe äußerte: das möcht ich gern hören, – bittet er
               mich, \strikeout{als} dieſen Wunſch, dieſe Sehnſucht {\pb}(ich theile{ }ſie
               wahrſcheinlich mit vielen) dir direct zu übermitteln. Das thu ich – in der Empfindung
               etwas unbeſcheiden – aber doch deiner Nachſicht gewiſs zu sein. Im übrigen wär es,
               auch abgeſehn von den Schubert\pwindex{Schubert, Franz Peter 31.\,1.\,1797 Lichtental [Wien] – 19.\,11.\,1828 Wien@\textsc{Schubert, Franz Peter} (31.\,1.\,1797 Lichtental [Wien] – 19.\,11.\,1828 Wien), \emph{Komponist}|pw} Liedern, die
               deine Frau\pwindex{Bahr-Mildenburg, Anna 29.\,11.\,1872 Wien – 27.\,1.\,1947 ebd.@\textsc{Bahr-Mildenburg, Anna} (29.\,11.\,1872 Wien – 27.\,1.\,1947 ebd.), \emph{Sängerin}|pwv}{ }ſo herrlich{ }ſingen{ }ſoll,{ }ſchön, we{\geminationn} man{ }ſich wieder einmal{ }ſehen und{ }ſprechen kö{\geminationn}te – in dieſer – Zeit, für die das Adjectiv doch erſt
               gefunden werden müſſte!\pend
           
\pstart
           Von Herzen mit Grüßen von Haus zu Haus{\\[\baselineskip]}dein \spacefill\mbox{Arthur}\pend
           \leftskip=0em{}\selectlanguage{ngerman}\endnumbering\briefempfaengerindex{Bahr, Hermann@\textsc{Bahr, Hermann}!zzzSchnitzler, Arthur@\emph{von Arthur Schnitzler}!1915-02-091@{9. 2. 1915}|)be}\mylabel{L02203h}  \newcommand{\dateiname}{L02203}\newcommand{\titel}{Arthur Schnitzler an Hermann Bahr, 9. 2. 1915}\newcommand{\editorInnen}{Herausgegeben von Martin Anton Müller}%% latex-leseansicht-abspann.tex
%% Abspann für die Leseansicht.
%% Der Schalter \ifkorrekturansicht ist bereits durch den Vorspann gesetzt.

%% latex-abspann.tex
%% Gemeinsamer Abspann für Korrekturansicht und Leseansicht.
%% Setzt den Schalter \ifkorrekturansicht voraus (gesetzt in den
%% einbindenden Dateien latex-korrekturansicht-abspann.tex bzw.
%% latex-leseansicht-abspann.tex).
%% ---------------------------------------------------------------

\normalsize

% Das esempio-Environment wird nur in der Leseansicht benötigt
\ifkorrekturansicht\else
\newenvironment{esempio}[3]%
{
    \vspace{1.5ex}
    \rlap{\underline{#1}}
    \par
    \setlength{\parindent}{0cm}
    \nopagebreak
    \leftskip=#2cm
    \rightskip=#3cm
}
{
    \par
}
\fi

\doendnotes{C}
\bigskip
\vfill

\clearpage

\footnotesize

\ifkorrekturansicht
  \lohead{\textsc{register}}
\fi

% theindex-Environment neu definieren ohne reledmac
\makeatletter
\renewenvironment{theindex}{%
  \ifkorrekturansicht
    \section*{\indexname}%
  \else
    \subsubsection*{Index der erwähnten Entitäten}%
  \fi
  \setlength{\parindent}{0pt}%
  \setlength{\parskip}{0pt plus 0.3pt}%
  \let\item\@idxitem
}{%
  \ifkorrekturansicht\clearpage\fi
}
\makeatother

\IfFileExists{\jobname-pw.ind}{\input{\jobname-pw.ind}}{}

% Quellenangabe nur in der Leseansicht
\ifkorrekturansicht\else
% Fallback-Definitionen, falls die .tex-Datei \titel etc. nicht gesetzt hat
\providecommand{\titel}{}
\providecommand{\editorInnen}{}
\providecommand{\dateiname}{\jobname}

\vspace{3cm}

\vfill

\footnotesize
\textsc{Quelle}: \titel. Herausgegeben von {\editorInnen}. In: \emph{Arthur Schnitzler: Briefwechsel mit Autorinnen und Autoren}.
 Digitale Edition, https://schnitzler-briefe.acdh.oeaw.ac.at/{\dateiname}.html (Stand \today)
\fi

\end{document}


