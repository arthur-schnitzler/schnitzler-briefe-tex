%% latex-leseansicht-vorspann.tex
%% Vorspann für die Leseansicht.
%% Lädt die gemeinsame Datei latex-vorspann.tex mit nicht gesetztem Schalter.

\newif\ifkorrekturansicht
\korrekturansichtfalse

\input{../tex-inputs/latex-vorspann}


\section[Arthur Schnitzler an Theodor Herzl, 16. 5. 1895]{L03933 Arthur Schnitzler an Theodor Herzl, 16. 5. 1895}
\nopagebreak\mylabel{L03933v}
\rehead{ }\normalsize\beginnumbering\briefempfaengerindex{Herzl, Theodor@\textsc{Herzl, Theodor}!zzzSchnitzler, Arthur@\emph{von Arthur Schnitzler}!1895-05-161@{16. 5. 1895}|(be}
\toendnotes[C]{\smallbreak\pagebreak[2]}
\correspDesc{Versand  durch Arthur Schnitzler am 16. 5. 1895 in Wien
\newline{}Erhalt  durch Theodor Herzl in Wien}\toendnotes[C]{\smallbreak}
\Standort{Jerusalem, Central Zionist Archives, H1:1925-18.}
\physDesc{,  Blätter,  Seiten
\newline{}Handschrift: , deutsche Kurrent}
\buchAbdrucke{\weitereDrucke{Arthur Schnitzler: \emph{Briefe 1875–1912}. Herausgegeben von Therese Nickl und Heinrich Schnitzler. Frankfurt am Main: \emph{S. Fischer} 1981, S. 257–258.} }\toendnotes[C]{\smallbreak}
\pstart{}{\pb}Mein lieber Freund!\pend\vspace{0.5em}
\pstart
           unſre Briefe haben{ }ſich diesmal wohl gekreuzt, und Sie wiſſen{ }ſchon, dſs das \textsc{Mscrpt\pwindex{Herzl, Theodor 2.\,5.\,1860 Budapest – 3.\,7.\,1904 Edlach@\textsc{Herzl, Theodor} (2.\,5.\,1860 Budapest – 3.\,7.\,1904 Edlach), \emph{Schriftsteller, Journalist}!neue Ghetto. Schauspiel in vier Acten@\strich\emph{Das neue Ghetto. Schauspiel in vier Acten}|pwv}} wieder in meinen Händen ist. Es kann jeden Moment abgehen. Sie{ }ſagen: das von \textsc{Blumenthal\pwindex{Blumenthal, Oskar 13.\,3.\,1852 Berlin – 24.\,4.\,1917 ebd.@\textsc{Blumenthal, Oskar} (13.\,3.\,1852 Berlin – 24.\,4.\,1917 ebd.), \emph{Schriftsteller, Journalist, Theaterleiter}|pw}}{ }ſchon zurückgelangte \textsc{Mscrpt\pwindex{Herzl, Theodor 2.\,5.\,1860 Budapest – 3.\,7.\,1904 Edlach@\textsc{Herzl, Theodor} (2.\,5.\,1860 Budapest – 3.\,7.\,1904 Edlach), \emph{Schriftsteller, Journalist}!neue Ghetto. Schauspiel in vier Acten@\strich\emph{Das neue Ghetto. Schauspiel in vier Acten}|pwv}}. Ich eri{\geminationn}ere Sie, daſs es jetzt nicht dort war,{ }ſondern dſs ich, Ihrem Auftrag
            entſprechend als \textsc{Schnabel}{ }\textsc{Bl.\pwindex{Blumenthal, Oskar 13.\,3.\,1852 Berlin – 24.\,4.\,1917 ebd.@\textsc{Blumenthal, Oskar} (13.\,3.\,1852 Berlin – 24.\,4.\,1917 ebd.), \emph{Schriftsteller, Journalist, Theaterleiter}|pw}} fragte, ob er, nach Berückſichtigg der M.
               G.\pwindex{Müller-Guttenbrunn, Adam 22.\,10.\,1852 Zăbrani – 5.\,1.\,1923 Wien@\textsc{Müller-Guttenbrunn, Adam} (22.\,10.\,1852 Zăbrani – 5.\,1.\,1923 Wien), \emph{Schriftsteller, Theaterleiter, Beamter}|pw}s Ablehnungswürde noch einmal etc – (ich hielt mich ganz nach dem von {\pb}Ihnen abgegebenen Wortlaute. Da kam dann zehn oder zwölf Tage
            keine Antwort. Nun ging es an \textsc{Fischer\pwindex{Fischer, Samuel 24.\,12.\,1859 Liptovský Mikuláš – 15.\,10.\,1934 Berlin@\textsc{Fischer, Samuel} (24.\,12.\,1859 Liptovský Mikuláš – 15.\,10.\,1934 Berlin), \emph{Verleger}|pw}} – der Begleitbrief entſprach natürlich auch vollkommen dem von Ihnen angegebenen
            Wortlaut und der Mann\pwindex{Fischer, Samuel 24.\,12.\,1859 Liptovský Mikuláš – 15.\,10.\,1934 Berlin@\textsc{Fischer, Samuel} (24.\,12.\,1859 Liptovský Mikuláš – 15.\,10.\,1934 Berlin), \emph{Verleger}|pwv} lieſs
            einfach vierzehn Tage oder gar 3 Wochen nichts von{ }ſich hören. Nun{ }ſandte ers – nicht
            auf brüſke Rückforderung,{ }ſondern auf höfliches Erſuchen{ }ſich zu entſcheiden oder
            zurückzuſchicken, – ohne eine Silbe oder Ent{\pb}ſchuldigg, an \textsc{Baumgarten\pwindex{Baumgarten, Julius 26.\,5.\,1860 Wien – 28.\,8.\,1934 ebd.@\textsc{Baumgarten, Julius} (26.\,5.\,1860 Wien – 28.\,8.\,1934 ebd.), \emph{Anwalt}|pw}}{ }\textsc{retour}, der es mir unter Kreuzband wie es geko{\geminationm}en war,
               zuſtellen lieſs. B.\pwindex{Baumgarten, Julius 26.\,5.\,1860 Wien – 28.\,8.\,1934 ebd.@\textsc{Baumgarten, Julius} (26.\,5.\,1860 Wien – 28.\,8.\,1934 ebd.), \emph{Anwalt}|pw} hat das \textsc{Mscrpt}.\pwindex{Herzl, Theodor 2.\,5.\,1860 Budapest – 3.\,7.\,1904 Edlach@\textsc{Herzl, Theodor} (2.\,5.\,1860 Budapest – 3.\,7.\,1904 Edlach), \emph{Schriftsteller, Journalist}!neue Ghetto. Schauspiel in vier Acten@\strich\emph{Das neue Ghetto. Schauspiel in vier Acten}|pwv} daher mit keinem Aug geſehen, da ich es
            perſönlich an \textsc{Fischer\pwindex{Fischer, Samuel 24.\,12.\,1859 Liptovský Mikuláš – 15.\,10.\,1934 Berlin@\textsc{Fischer, Samuel} (24.\,12.\,1859 Liptovský Mikuláš – 15.\,10.\,1934 Berlin), \emph{Verleger}|pw}} auf die Poſt gab. – Nun, wie gefällt Ihnen die{ }ſelbſtgewählte Rolle des
            »unbekannten Dichters«?– Glauben Sie mir, daſs ich Ihren Widerwillen{ }ſozuſagen
            begeiſtert mitfühle. Und man ist wehrlos. –\pend
           
\pstart
           {\pb}Tabarin\pwindex{Herzl, Theodor 2.\,5.\,1860 Budapest – 3.\,7.\,1904 Edlach@\textsc{Herzl, Theodor} (2.\,5.\,1860 Budapest – 3.\,7.\,1904 Edlach), \emph{Schriftsteller, Journalist}!Tabarin. Schauspiel in einem Act. Frei nach Catulle Mendès@\strich\emph{Tabarin. Schauspiel in einem Act. Frei nach Catulle Mendès}|pw} hab ich \label{K_L03933-1v}\edtext{neulich\eventindex{Burgtheater@\textbf{Burgtheater}!Aufführung von Tabarin, Die Wespe, 7.5.1895@Aufführung von Tabarin, Die Wespe, 7.5.1895|pw}}{\lemma{\textnormal{\emph{neulich}}}\Cendnote{\textnormal{A. S.: \emph{Kulturveranstaltungen}, 7. 5. 1895.}}}\label{K_L03933-1} geſehen; es wirkte{ }ſehr gut und wird{ }ſich auf dem \textsc{Repèrtoire} halten. We{\geminationn} es Sie nicht langweilt, möchte ich eine
            Einwendung gegen eine Scene erheben. Es iſt der kurze Monolog, den T. \pwindex{Herzl, Theodor 2.\,5.\,1860 Budapest – 3.\,7.\,1904 Edlach@\textsc{Herzl, Theodor} (2.\,5.\,1860 Budapest – 3.\,7.\,1904 Edlach), \emph{Schriftsteller, Journalist}!Tabarin. Schauspiel in einem Act. Frei nach Catulle Mendès@\strich\emph{Tabarin. Schauspiel in einem Act. Frei nach Catulle Mendès}|pw}der Bühnen-bühne hält, gleich nachdem er{ }ſeine Frau mit dem
            Soldaten entdeckt hat; –{ }ſie packt währenddem ihre Sachen zuſa{\geminationm}en. Ich verſtehe die
            theatraliſchen Gründe für dieſen Aufſchub in der Handlung, aber ich und manche andere
            vernünftige Beurtheiler
               {\pb}fanden, daſs die Scene als unwahr wirkt. Man
            begreift nicht, daſs sich Tabarin\pwindex{Herzl, Theodor 2.\,5.\,1860 Budapest – 3.\,7.\,1904 Edlach@\textsc{Herzl, Theodor} (2.\,5.\,1860 Budapest – 3.\,7.\,1904 Edlach), \emph{Schriftsteller, Journalist}!Tabarin. Schauspiel in einem Act. Frei nach Catulle Mendès@\strich\emph{Tabarin. Schauspiel in einem Act. Frei nach Catulle Mendès}|pwv}
            nicht{ }ſofort auf{ }ſeine Frau{ }ſtürzt – man begreift aber noch weniger, daſs die Frau nicht
            wenigstens die Zeit benutzt, die T.\pwindex{Herzl, Theodor 2.\,5.\,1860 Budapest – 3.\,7.\,1904 Edlach@\textsc{Herzl, Theodor} (2.\,5.\,1860 Budapest – 3.\,7.\,1904 Edlach), \emph{Schriftsteller, Journalist}!Tabarin. Schauspiel in einem Act. Frei nach Catulle Mendès@\strich\emph{Tabarin. Schauspiel in einem Act. Frei nach Catulle Mendès}|pwv} monologiſirt, um davonzulaufen. – Es iſt{ }ſchade, daſs Sie die Aufführung
            nicht geſehen haben, von der Sie viel Freude gehabt hätten. Es war wunderbar, was die
               Sandrock\pwindex{Sandrock, Adele 19.\,8.\,1863 Rotterdam – 30.\,8.\,1937 Berlin@\textsc{Sandrock, Adele} (19.\,8.\,1863 Rotterdam – 30.\,8.\,1937 Berlin), \emph{Schauspielerin}|pw} mit ihren wenigen Worten für eine
            lebendige {\pb}Leiſtung bot. Daſs \textsc{Mitterwurzer\pwindex{Mitterwurzer, Friedrich 16.\,10.\,1844 Dresden – 13.\,2.\,1897 Wien@\textsc{Mitterwurzer, Friedrich} (16.\,10.\,1844 Dresden – 13.\,2.\,1897 Wien), \emph{Schauspieler}|pw}} vorzüglich war, iſt nicht merkwürdig – es gäbe viele Schauſpieler, die in dieſer
            Rolle gut wären, die ja \strikeout{an{ }ſich} ſo unfehlbare Wirkungen
            in{ }ſich trägt. Ich möchte u. a. Sonnenthal\pwindex{Sonnenthal, Adolf von 21.\,12.\,1834 Budapest – 4.\,4.\,1909 Prag@\textsc{Sonnenthal, Adolf von} (21.\,12.\,1834 Budapest – 4.\,4.\,1909 Prag), \emph{Schauspieler}|pw} oder Robert\pwindex{Robert, Emerich 21.\,5.\,1847 Budapest – 29.\,5.\,1899 Würzburg@\textsc{Robert, Emerich} (21.\,5.\,1847 Budapest – 29.\,5.\,1899 Würzburg), \emph{Regisseur, Schauspieler}|pw} als Tabarin\pwindex{Herzl, Theodor 2.\,5.\,1860 Budapest – 3.\,7.\,1904 Edlach@\textsc{Herzl, Theodor} (2.\,5.\,1860 Budapest – 3.\,7.\,1904 Edlach), \emph{Schriftsteller, Journalist}!Tabarin. Schauspiel in einem Act. Frei nach Catulle Mendès@\strich\emph{Tabarin. Schauspiel in einem Act. Frei nach Catulle Mendès}|pwv}{ }ſehn. –\pend
           
\pstart
           – Ich\pwindex{Schnitzler, Arthur 15.\,5.\,1862 Wien – 21.\,10.\,1931 ebd.@\textsc{Schnitzler, Arthur} (15.\,5.\,1862 Wien – 21.\,10.\,1931 ebd.), \emph{Schriftsteller, Mediziner}!Liebelei. Schauspiel in drei Akten@\strich\emph{Liebelei. Schauspiel in drei Akten}|pwv} ko{\geminationm}e natürlich heuer
            nicht mehr dran u. bin froh darüber. Wirkliche in der \textcolor{gray}{jetzigen}
               Saiſon Gründe, mich \strikeout{heuer} nicht aufzuführen –
            lagen allerdings keine vor, – außer daſs \strikeout{es} mir die
            Aufführung für März verſprochen {\pb}worden war. Aber
            das{ }ſcheint ja beim Theater{ }ſchon zu genügen. –\pend
           
\pstart
           Seien Sie vielmals herzlich gegrüßt{\\[\baselineskip]}von Ihrem treu ergebn{\\[\baselineskip]}\spacefill\mbox{ArthSch}\pend
           \leftskip=0em{}
\pstart
           16/10 95.\pend
           \selectlanguage{ngerman}\endnumbering\briefempfaengerindex{Herzl, Theodor@\textsc{Herzl, Theodor}!zzzSchnitzler, Arthur@\emph{von Arthur Schnitzler}!1895-05-161@{16. 5. 1895}|)be}\mylabel{L03933h}
\begin{anhang}
\end{anhang}\newcommand{\dateiname}{L03933}\newcommand{\titel}{Arthur Schnitzler an Theodor Herzl, 16. 5. 1895}\newcommand{\editorInnen}{Herausgegeben von Jahnke, SelmaMüller, Martin Anton}%% latex-leseansicht-abspann.tex
%% Abspann für die Leseansicht.
%% Der Schalter \ifkorrekturansicht ist bereits durch den Vorspann gesetzt.

%% latex-abspann.tex
%% Gemeinsamer Abspann für Korrekturansicht und Leseansicht.
%% Setzt den Schalter \ifkorrekturansicht voraus (gesetzt in den
%% einbindenden Dateien latex-korrekturansicht-abspann.tex bzw.
%% latex-leseansicht-abspann.tex).
%% ---------------------------------------------------------------

\normalsize

% Das esempio-Environment wird nur in der Leseansicht benötigt
\ifkorrekturansicht\else
\newenvironment{esempio}[3]%
{
    \vspace{1.5ex}
    \rlap{\underline{#1}}
    \par
    \setlength{\parindent}{0cm}
    \nopagebreak
    \leftskip=#2cm
    \rightskip=#3cm
}
{
    \par
}
\fi

\doendnotes{C}
\bigskip
\vfill

\clearpage

\footnotesize

\ifkorrekturansicht
  \lohead{\textsc{register}}
\fi

% theindex-Environment neu definieren ohne reledmac
\makeatletter
\renewenvironment{theindex}{%
  \ifkorrekturansicht
    \section*{\indexname}%
  \else
    \subsubsection*{Index der erwähnten Entitäten}%
  \fi
  \setlength{\parindent}{0pt}%
  \setlength{\parskip}{0pt plus 0.3pt}%
  \let\item\@idxitem
}{%
  \ifkorrekturansicht\clearpage\fi
}
\makeatother

\IfFileExists{\jobname-pw.ind}{\input{\jobname-pw.ind}}{}

% Quellenangabe nur in der Leseansicht
\ifkorrekturansicht\else
% Fallback-Definitionen, falls die .tex-Datei \titel etc. nicht gesetzt hat
\providecommand{\titel}{}
\providecommand{\editorInnen}{}
\providecommand{\dateiname}{\jobname}

\vspace{3cm}

\vfill

\footnotesize
\textsc{Quelle}: \titel. Herausgegeben von {\editorInnen}. In: \emph{Arthur Schnitzler: Briefwechsel mit Autorinnen und Autoren}.
 Digitale Edition, https://schnitzler-briefe.acdh.oeaw.ac.at/{\dateiname}.html (Stand \today)
\fi

\end{document}


