%% latex-leseansicht-vorspann.tex
%% Vorspann für die Leseansicht.
%% Lädt die gemeinsame Datei latex-vorspann.tex mit nicht gesetztem Schalter.

\newif\ifkorrekturansicht
\korrekturansichtfalse

\input{../tex-inputs/latex-vorspann}


\section[Arthur Schnitzler an Richard Beer-Hofmann, 16. 12. 1895]{L00522 Arthur Schnitzler an Richard Beer-Hofmann, 16. 12. 1895}
\nopagebreak\mylabel{L00522v}
\rehead{ }\normalsize\beginnumbering\briefempfaengerindex{Beer-Hofmann, Richard@\textsc{Beer-Hofmann, Richard}!zzzSchnitzler, Arthur@\emph{von Arthur Schnitzler}!1895-12-161@{16. 12. 1895}|(be}
\toendnotes[C]{\smallbreak\pagebreak[2]}
\correspDesc{Versand  durch Arthur Schnitzler am 16. 12. 1895 in Wien
\newline{}Erhalt  durch Richard Beer-Hofmann am 17. 12. 1895 in Wien}\toendnotes[C]{\smallbreak}
\Standort{YCGL, MSS 31.}
\physDesc{Brief, 1 Blatt, 2 Seiten, Kuvert, 409 Zeichen
\newline{}Handschrift: schwarze Tinte, deutsche Kurrent
\newline{}Versand: 1) Stempel: »\nobreak{}\oindex{III., Landstraße@\textbf{III., Landstraße}, \emph{Verwaltungsgebiet}|pwk}Wien 3/1, 16. \textcolor{gray}{1}2. 95, 6–7\textcolor{gray}{S}\nobreak{}«.   2) Stempel: »\nobreak{}\oindex{I., Innere Stadt@\textbf{I., Innere Stadt}, \emph{Verwaltungsgebiet}|pwk}{\pb}Wien 1/1, 17. 12. 95, Bestellt\nobreak{}«. }
\buchAbdrucke{\weitereDrucke{Arthur Schnitzler, Richard Beer-Hofmann: \emph{Briefwechsel 1891–1931}. Herausgegeben von Konstanze Fliedl. Wien, Zürich: \emph{Europaverlag} 1992, S. 89.} }\pstart{}{\pb}Herrn \textsc{Dr. Richard
                     Beer-Hofmann}\pend{}\pstart{}Wien\oindex{Wien@\textbf{Wien}, \emph{Verwaltungsgebiet}|pw}\pend{}\pstart{}\textsc{I Wollzeile 15\oindex{Wien@\textbf{Wien}!I., Innere Stadt@\textbf{I., Innere Stadt}!Wollzeile 15 (»Berthahof«)@\textbf{Wollzeile 15 (»Berthahof«)}, \emph{Wohngebäude}|pw}}.\pend{}{\bigskip}\vspace{1em}
\pstart{}{\pb}Lieber Richard,\pend\vspace{0.5em}
\pstart
           eben war Frau Lou\pwindex{Andreas-Salomé, Lou 12.\,2.\,1861 Sankt Petersburg – 5.\,2.\,1937 Göttingen@\textsc{Andreas-Salomé, Lou} (12.\,2.\,1861 Sankt Petersburg – 5.\,2.\,1937 Göttingen), \emph{Schriftstellerin}|pw} bei mir. Haben Sie morgen
                  Dinſtag{ }Abend Zeit? Ich erinnere mich, Sie äußerten irgend was dergleichen. Ich
               bin bei \textsc{Rosé}\pwindex{Rosé, Arnold 24.\,10.\,1863 Iași – 25.\,8.\,1946 London@\textsc{Rosé, Arnold} (24.\,10.\,1863 Iași – 25.\,8.\,1946 London), \emph{Violinist}|pw}; iſts Ihnen recht,{ }ſo hole ich {\pb}von dort aus
                  (½ 10) Sie, u wir zuſa{\geminationm}en Fr. Lou\pwindex{Andreas-Salomé, Lou 12.\,2.\,1861 Sankt Petersburg – 5.\,2.\,1937 Göttingen@\textsc{Andreas-Salomé, Lou} (12.\,2.\,1861 Sankt Petersburg – 5.\,2.\,1937 Göttingen), \emph{Schriftstellerin}|pw} ab. Oder Sie holen Sie  früher
               ab und{ }ſagen mir, wo ich Sie nach \textsc{Rosé}\pwindex{Rosé, Arnold 24.\,10.\,1863 Iași – 25.\,8.\,1946 London@\textsc{Rosé, Arnold} (24.\,10.\,1863 Iași – 25.\,8.\,1946 London), \emph{Violinist}|pw} finde. \textsc{Grstdl}\oindex{Wien@\textbf{Wien}!I., Innere Stadt@\textbf{I., Innere Stadt}!Café Griensteidl@\textbf{Café Griensteidl}, \emph{Kaffeehaus}|pw} iſt wohl in letzterem Falle das einfachſte.\pend
           \pstart Herzlich Ihr \spacefill\mbox{Arthur}\pend{}\selectlanguage{ngerman}\endnumbering\briefempfaengerindex{Beer-Hofmann, Richard@\textsc{Beer-Hofmann, Richard}!zzzSchnitzler, Arthur@\emph{von Arthur Schnitzler}!1895-12-161@{16. 12. 1895}|)be}\mylabel{L00522h}  \newcommand{\dateiname}{L00522}\newcommand{\titel}{Arthur Schnitzler an Richard Beer-Hofmann, 16. 12. 1895}\newcommand{\editorInnen}{Martin Anton Müller und Gerd-Hermann Susen}%% latex-leseansicht-abspann.tex
%% Abspann für die Leseansicht.
%% Der Schalter \ifkorrekturansicht ist bereits durch den Vorspann gesetzt.

%% latex-abspann.tex
%% Gemeinsamer Abspann für Korrekturansicht und Leseansicht.
%% Setzt den Schalter \ifkorrekturansicht voraus (gesetzt in den
%% einbindenden Dateien latex-korrekturansicht-abspann.tex bzw.
%% latex-leseansicht-abspann.tex).
%% ---------------------------------------------------------------

\normalsize

% Das esempio-Environment wird nur in der Leseansicht benötigt
\ifkorrekturansicht\else
\newenvironment{esempio}[3]%
{
    \vspace{1.5ex}
    \rlap{\underline{#1}}
    \par
    \setlength{\parindent}{0cm}
    \nopagebreak
    \leftskip=#2cm
    \rightskip=#3cm
}
{
    \par
}
\fi

\doendnotes{C}
\bigskip
\vfill

\clearpage

\footnotesize

\ifkorrekturansicht
  \lohead{\textsc{register}}
\fi

% theindex-Environment neu definieren ohne reledmac
\makeatletter
\renewenvironment{theindex}{%
  \ifkorrekturansicht
    \section*{\indexname}%
  \else
    \subsubsection*{Index der erwähnten Entitäten}%
  \fi
  \setlength{\parindent}{0pt}%
  \setlength{\parskip}{0pt plus 0.3pt}%
  \let\item\@idxitem
}{%
  \ifkorrekturansicht\clearpage\fi
}
\makeatother

\IfFileExists{\jobname-pw.ind}{\input{\jobname-pw.ind}}{}

% Quellenangabe nur in der Leseansicht
\ifkorrekturansicht\else
% Fallback-Definitionen, falls die .tex-Datei \titel etc. nicht gesetzt hat
\providecommand{\titel}{}
\providecommand{\editorInnen}{}
\providecommand{\dateiname}{\jobname}

\vspace{3cm}

\vfill

\footnotesize
\textsc{Quelle}: \titel. Herausgegeben von {\editorInnen}. In: \emph{Arthur Schnitzler: Briefwechsel mit Autorinnen und Autoren}.
 Digitale Edition, https://schnitzler-briefe.acdh.oeaw.ac.at/{\dateiname}.html (Stand \today)
\fi

\end{document}


