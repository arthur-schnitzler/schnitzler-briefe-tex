%% latex-leseansicht-vorspann.tex
%% Vorspann für die Leseansicht.
%% Lädt die gemeinsame Datei latex-vorspann.tex mit nicht gesetztem Schalter.

\newif\ifkorrekturansicht
\korrekturansichtfalse

\input{../tex-inputs/latex-vorspann}


\section[Felix Braun an Arthur Schnitzler, 2. 10. 1924]{L02415 Felix Braun an Arthur Schnitzler, 2. 10. 1924}
\nopagebreak\mylabel{L02415v}
\rehead{ }\normalsize\beginnumbering\briefempfaengerindex{Schnitzler, Arthur@\textsc{Schnitzler, Arthur}!zzzBraun, Felix@\emph{von Felix Braun}!1924-10-021@{2. 10. 1924}|(be}
\toendnotes[C]{\smallbreak\pagebreak[2]}
\correspDesc{Versand  durch Felix Braun am 2. 10. 1924 in Wien
\newline{}Erhalt  durch Arthur Schnitzler im Zeitraum [2. 10. 1924
                  – 6. 10. 1924?] in Wien}\toendnotes[C]{\smallbreak}
\Standort{DLA, A:Schnitzler, HS.NZ85.1.2604,3.}
\physDesc{Brief, 1 Blatt, 2 Seiten, 1502 Zeichen
\newline{}Handschrift: schwarze Tinte, deutsche Kurrent
\newline{}Schnitzler: 1) mit Bleistift beschriftet: »\textsc{Braun}« und »\textsc{Sievringerstr. 191}\oindex{Wien@\textbf{Wien}!XIX., Döbling@\textbf{XIX., Döbling}!Sieveringer Straße@\textbf{Sieveringer Straße}, \emph{Straße}|pw}«  2) mit rotem Buntstift mehrere Unterstreichungen}\toendnotes[C]{\smallbreak}
\pstart
           \centering{}{\pb}Wien\oindex{Wien@\textbf{Wien}, \emph{Verwaltungsgebiet}|pw}, den 2. X. 1924\pend
           
\pstart{}Sehr verehrter Herr Doktor!\pend\vspace{0.5em}
\pstart
           Als ich heute vom Verlag Fiſcher\orgindex{S. Fischer Verlag@S. Fischer Verlag|pw} Ihre neue Komödie\pwindex{Schnitzler, Arthur 15.\,5.\,1862 Wien – 21.\,10.\,1931 ebd.@\textsc{Schnitzler, Arthur} (15.\,5.\,1862 Wien – 21.\,10.\,1931 ebd.), \emph{Schriftsteller, Mediziner}!Komödie der Verführung. In drei Akten@\strich\emph{Komödie der Verführung. In drei Akten}|pwv} erhielt und in dem Buch\pwindex{Schnitzler, Arthur 15.\,5.\,1862 Wien – 21.\,10.\,1931 ebd.@\textsc{Schnitzler, Arthur} (15.\,5.\,1862 Wien – 21.\,10.\,1931 ebd.), \emph{Schriftsteller, Mediziner}!Komödie der Verführung. In drei Akten@\strich\emph{Komödie der Verführung. In drei Akten}|pwv} den Vermerk: »Im Auftrag
               des Verfaſſers« fand, war ich{ }ſehr{ }ſtolz und erfreut:{ }ſeien Sie herzlich bedankt für
               dieſe Auszeichnung!\pend
           
\pstart
           Ich habe auch das Werk\pwindex{Schnitzler, Arthur 15.\,5.\,1862 Wien – 21.\,10.\,1931 ebd.@\textsc{Schnitzler, Arthur} (15.\,5.\,1862 Wien – 21.\,10.\,1931 ebd.), \emph{Schriftsteller, Mediziner}!Komödie der Verführung. In drei Akten@\strich\emph{Komödie der Verführung. In drei Akten}|pwv}{ }ſofort
               zu leſen begonnen und jetzt – es ist{ }ſpät nachts – den ergreifenden, tiefen dritten
               Akt beendigt.\pend
           
\pstart
           Es iſt ein großes, reines Dichtwerk, eine Art dramatiſcher Roman, wenn ich mich{ }ſo
               ausdrücken darf. Die Geſtalt Falkenirs\pwindex{Schnitzler, Arthur 15.\,5.\,1862 Wien – 21.\,10.\,1931 ebd.@\textsc{Schnitzler, Arthur} (15.\,5.\,1862 Wien – 21.\,10.\,1931 ebd.), \emph{Schriftsteller, Mediziner}!Komödie der Verführung. In drei Akten@\strich\emph{Komödie der Verführung. In drei Akten}|pwv} ging mir am nächſten. In Aurelie\pwindex{Schnitzler, Arthur 15.\,5.\,1862 Wien – 21.\,10.\,1931 ebd.@\textsc{Schnitzler, Arthur} (15.\,5.\,1862 Wien – 21.\,10.\,1931 ebd.), \emph{Schriftsteller, Mediziner}!Komödie der Verführung. In drei Akten@\strich\emph{Komödie der Verführung. In drei Akten}|pwv} iſt das Weibliche als das Allmögliche des Erlebens
               endgültig geſtaltet; Falkenirs\pwindex{Schnitzler, Arthur 15.\,5.\,1862 Wien – 21.\,10.\,1931 ebd.@\textsc{Schnitzler, Arthur} (15.\,5.\,1862 Wien – 21.\,10.\,1931 ebd.), \emph{Schriftsteller, Mediziner}!Komödie der Verführung. In drei Akten@\strich\emph{Komödie der Verführung. In drei Akten}|pwv}
               Schuld geht daran hervor. Man lebt{ }ſich{ }ſehr in dieſe Welt ein und möchte{ }ſich eine
               Fortſetzung wün{\pb}ſchen. Den erſten und den dritten Akt
               halte ich für die{ }ſchönſten des Stückes; der zweite{ }ſteht für mein Gefühl etwas
               zurück. Der dritte iſt myſtiſch, wächſt gegen den Schluß immer höher ins
               Bedeutungsvolle und gewinnt immer noch an Poeſie. Über Einzelheiten voll tiefen
               Einblicks möchte ich in dieſem kurzen Brief gar nicht erſt{ }ſprechen. Die{ }ſchöne
               Stelle über die Liebe als Kampf nur darf ich hervorheben. Das Buch\pwindex{Schnitzler, Arthur 15.\,5.\,1862 Wien – 21.\,10.\,1931 ebd.@\textsc{Schnitzler, Arthur} (15.\,5.\,1862 Wien – 21.\,10.\,1931 ebd.), \emph{Schriftsteller, Mediziner}!Komödie der Verführung. In drei Akten@\strich\emph{Komödie der Verführung. In drei Akten}|pwv} hat mir viel gegeben: ich danke Ihnen
               von Herzen dafür!\pend
           
\pstart
           Darf ich Sie nun bitten, verehrter Herr Doktor, als eine – freilich im Abſtand zu
               betrachtende – Gegengabe meine beiden letzterſchienenen Bücher\pwindex{Braun, Felix 4.\,11.\,1885 Wien – 29.\,11.\,1973 Klosterneuburg@\textsc{Braun, Felix} (4.\,11.\,1885 Wien – 29.\,11.\,1973 Klosterneuburg), \emph{Schriftsteller}!Wunderstunden. Drei Erzählungen@\strich\emph{Wunderstunden. Drei Erzählungen}|pwv}\pwindex{Braun, Felix 4.\,11.\,1885 Wien – 29.\,11.\,1973 Klosterneuburg@\textsc{Braun, Felix} (4.\,11.\,1885 Wien – 29.\,11.\,1973 Klosterneuburg), \emph{Schriftsteller}!unsichtbare Gast@\strich\emph{Der unsichtbare Gast}|pwv} von mir anzunehmen? Es
               würde mich{ }ſehr freuen, wenn Ihnen das eine oder das andere ein weniges zu{ }ſagen
               hätte.\pend
           
\pstart
           In dieſer Zuverſicht bin ich, verehrter Herr Doktor, Ihr ergebener{\\[\baselineskip]}\spacefill\mbox{Felix Braun.}\pend
           \leftskip=0em{}\selectlanguage{ngerman}\endnumbering\briefempfaengerindex{Schnitzler, Arthur@\textsc{Schnitzler, Arthur}!zzzBraun, Felix@\emph{von Felix Braun}!1924-10-021@{2. 10. 1924}|)be}\mylabel{L02415h}  \newcommand{\dateiname}{L02415}\newcommand{\titel}{Felix Braun an Arthur Schnitzler, 2. 10. 1924}\newcommand{\editorInnen}{Martin Anton Müller und Gerd-Hermann Susen}%% latex-leseansicht-abspann.tex
%% Abspann für die Leseansicht.
%% Der Schalter \ifkorrekturansicht ist bereits durch den Vorspann gesetzt.

%% latex-abspann.tex
%% Gemeinsamer Abspann für Korrekturansicht und Leseansicht.
%% Setzt den Schalter \ifkorrekturansicht voraus (gesetzt in den
%% einbindenden Dateien latex-korrekturansicht-abspann.tex bzw.
%% latex-leseansicht-abspann.tex).
%% ---------------------------------------------------------------

\normalsize

% Das esempio-Environment wird nur in der Leseansicht benötigt
\ifkorrekturansicht\else
\newenvironment{esempio}[3]%
{
    \vspace{1.5ex}
    \rlap{\underline{#1}}
    \par
    \setlength{\parindent}{0cm}
    \nopagebreak
    \leftskip=#2cm
    \rightskip=#3cm
}
{
    \par
}
\fi

\doendnotes{C}
\bigskip
\vfill

\clearpage

\footnotesize

\ifkorrekturansicht
  \lohead{\textsc{register}}
\fi

% theindex-Environment neu definieren ohne reledmac
\makeatletter
\renewenvironment{theindex}{%
  \ifkorrekturansicht
    \section*{\indexname}%
  \else
    \subsubsection*{Index der erwähnten Entitäten}%
  \fi
  \setlength{\parindent}{0pt}%
  \setlength{\parskip}{0pt plus 0.3pt}%
  \let\item\@idxitem
}{%
  \ifkorrekturansicht\clearpage\fi
}
\makeatother

\IfFileExists{\jobname-pw.ind}{\input{\jobname-pw.ind}}{}

% Quellenangabe nur in der Leseansicht
\ifkorrekturansicht\else
% Fallback-Definitionen, falls die .tex-Datei \titel etc. nicht gesetzt hat
\providecommand{\titel}{}
\providecommand{\editorInnen}{}
\providecommand{\dateiname}{\jobname}

\vspace{3cm}

\vfill

\footnotesize
\textsc{Quelle}: \titel. Herausgegeben von {\editorInnen}. In: \emph{Arthur Schnitzler: Briefwechsel mit Autorinnen und Autoren}.
 Digitale Edition, https://schnitzler-briefe.acdh.oeaw.ac.at/{\dateiname}.html (Stand \today)
\fi

\end{document}


