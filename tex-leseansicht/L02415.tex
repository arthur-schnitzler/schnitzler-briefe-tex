\input{../tex-inputs/latex-pdf-vorspann}
\begin{center}
            \textcolor{red}{ENTWURF. ENTZIFFERUNG NOCH NICHT KORREKTURGELESEN}
                      \end{center}
            
               \section[Felix Braun an Arthur Schnitzler, 2. 10. 1924]{ Felix Braun an Arthur Schnitzler, 2. 10. 1924}\nopagebreak\mylabel{v}\rehead{ }\begin{ledgroupsized}[t]{13cm}\normalsize\beginnumbering\briefempfaengerindex{Schnitzler, Arthur@\textsc{Schnitzler, Arthur}!zzzBraun, Felix@\emph{von Felix Braun}!1924-10-021@{2. 10. 1924}|(be} \toendnotes[C]{\smallbreak\pagebreak[2]} \Standort{DLA, A:Schnitzler, HS.NZ85.1.2604,3.}
\physDesc{Brief, 1 Blatt, 2 Seiten
\newline{}Handschrift: schwarze Tinte, deutsche Kurrent
\newline{}Schnitzler: 1) mit Bleistift beschriftet: »\textsc{Braun}« und »\textsc{Sievringerstr. 191}\oindex{Sieveringer Strasse@\textbf{Sieveringer Straße}|pw}« 2) mit rotem Buntstift mehrere Unterstreichungen}\toendnotes[C]{\smallbreak}\pstart
           \centering{}{\pb}Wien\oindex{Wien@\textbf{Wien}|pw}, den 2. X. 1924\pend
           \pstart{}Sehr verehrter Herr Doktor!\pend\pstart
           Als ich heute vom Verlag Fiſcher\orgindex{S. Fischer Verlag@S. Fischer Verlag|pw} Ihre neue Komödie\pwindex{Schnitzler, Arthur 15.05.1862 – 21.10.1931@\textsc{Schnitzler, Arthur} (15.05.1862 – 21.10.1931), \emph{Schriftsteller, Mediziner}!Komoedie der Verfuehrung. In drei Akten1924@\strich\emph{Komödie der Verführung. In drei Akten} {[}1924{]}|pwv} erhielt und in dem
                        Buch\pwindex{Schnitzler, Arthur 15.05.1862 – 21.10.1931@\textsc{Schnitzler, Arthur} (15.05.1862 – 21.10.1931), \emph{Schriftsteller, Mediziner}!Komoedie der Verfuehrung. In drei Akten1924@\strich\emph{Komödie der Verführung. In drei Akten} {[}1924{]}|pwv} den Vermerk: »Im
                    Auftrag des Verfaſſers« fand, war ich ſehr ſtolz und erfreut: ſeien Sie herzlich
                    bedankt für dieſe Auszeichnung!\pend
           \pstart
           Ich habe auch das Werk\pwindex{Schnitzler, Arthur 15.05.1862 – 21.10.1931@\textsc{Schnitzler, Arthur} (15.05.1862 – 21.10.1931), \emph{Schriftsteller, Mediziner}!Komoedie der Verfuehrung. In drei Akten1924@\strich\emph{Komödie der Verführung. In drei Akten} {[}1924{]}|pwv} ſofort
                    zu leſen begonnen und jetzt – es ist ſpät nachts – den ergreifenden, tiefen
                    dritten Akt beendigt.\pend
           \pstart
           Es iſt ein großes, reines Dichtwerk, eine Art dramatiſcher Roman, wenn ich mich
                    ſo ausdrücken darf. Die Geſtalt Falkenirs\pwindex{Schnitzler, Arthur 15.05.1862 – 21.10.1931@\textsc{Schnitzler, Arthur} (15.05.1862 – 21.10.1931), \emph{Schriftsteller, Mediziner}!Komoedie der Verfuehrung. In drei Akten1924@\strich\emph{Komödie der Verführung. In drei Akten} {[}1924{]}|pwv} ging mir am nächſten. In Aurelie\pwindex{Schnitzler, Arthur 15.05.1862 – 21.10.1931@\textsc{Schnitzler, Arthur} (15.05.1862 – 21.10.1931), \emph{Schriftsteller, Mediziner}!Komoedie der Verfuehrung. In drei Akten1924@\strich\emph{Komödie der Verführung. In drei Akten} {[}1924{]}|pwv} iſt das Weibliche als das Allmögliche des
                    Erlebens endgültig geſtaltet; Falkenir\pwindex{Schnitzler, Arthur 15.05.1862 – 21.10.1931@\textsc{Schnitzler, Arthur} (15.05.1862 – 21.10.1931), \emph{Schriftsteller, Mediziner}!Komoedie der Verfuehrung. In drei Akten1924@\strich\emph{Komödie der Verführung. In drei Akten} {[}1924{]}|pwv}s Schuld geht daran hervor. Man lebt ſich ſehr in dieſe Welt
                    ein und möchte ſich eine Fortſetzung wün{\pb}ſchen. Den erſten und den
                    dritten Akt halte ich für die ſchönſten des Stückes; der zweite ſteht für mein
                    Gefühl etwas zurück. Der dritte iſt myſtiſch, wächſt gegen den Schluß immer
                    höher ins Bedeutungsvolle und gewinnt immer noch an Poeſie. Über Einzelheiten
                    voll tiefen Einblicks möchte ich in dieſem kurzen Brief gar nicht erſt ſprechen.
                    Die ſchöne Stelle über die Liebe als Kampf nur darf ich hervorheben. Das Buch\pwindex{Schnitzler, Arthur 15.05.1862 – 21.10.1931@\textsc{Schnitzler, Arthur} (15.05.1862 – 21.10.1931), \emph{Schriftsteller, Mediziner}!Komoedie der Verfuehrung. In drei Akten1924@\strich\emph{Komödie der Verführung. In drei Akten} {[}1924{]}|pwv} hat mir viel gegeben:
                    ich danke Ihnen von Herzen dafür!\pend
           \pstart
           Darf ich Sie nun bitten, verehrter Herr Doktor, als eine – freilich im Abſtand zu
                    betrachtende – Gegengabe meine beiden letzterſchienenen Bücher\pwindex{Braun, Felix 04.11.1885 – 29.11.1973@\textsc{Braun, Felix} (04.11.1885 – 29.11.1973), \emph{Schriftsteller}!Wunderstunden. Drei Erzaehlungen1923@\strich\emph{Wunderstunden. Drei Erzählungen} {[}1923{]}|pwv}\pwindex{Braun, Felix 04.11.1885 – 29.11.1973@\textsc{Braun, Felix} (04.11.1885 – 29.11.1973), \emph{Schriftsteller}!unsichtbare Gast1924@\strich\emph{Der unsichtbare Gast} {[}1924{]}|pwv} von mir anzunehmen? Es
                    würde mich ſehr freuen, wenn Ihnen das eine oder das andere ein weniges zu ſagen
                    hätte.\pend
           \pstart
           In dieſer Zuverſicht bin ich, verehrter Herr Doktor, Ihr ergebener{\\[\baselineskip]}\spacefill\mbox{Felix Braun.}\pend
           \leftskip=0em{}\endnumbering\briefempfaengerindex{Schnitzler, Arthur@\textsc{Schnitzler, Arthur}!zzzBraun, Felix@\emph{von Felix Braun}!1924-10-021@{2. 10. 1924}|)be}\mylabel{h}\end{ledgroupsized}  \newcommand{\dateiname}{L02415}\newcommand{\titel}{Felix Braun an Arthur Schnitzler, 2. 10. 1924}\newcommand{\editorInnen}{Martin Anton Müller und Gerd-Hermann Susen}\input{../tex-inputs/latex-pdf-abspann}
      