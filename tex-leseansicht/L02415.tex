%% latex-korrekturansicht-vorspann.tex
%% Vorspann für die Korrekturansicht.
%% Lädt die gemeinsame Datei latex-vorspann.tex mit gesetztem Schalter.

\newif\ifkorrekturansicht
\korrekturansichttrue

\input{../tex-inputs/latex-vorspann}


\section[Felix Braun an Arthur Schnitzler, 2. 10. 1924]{L02415 Felix Braun an Arthur Schnitzler, 2. 10. 1924}
\nopagebreak\mylabel{L02415v}
\rehead{ }\normalsize\beginnumbering\briefempfaengerindex{Schnitzler, Arthur@\textsc{Schnitzler, Arthur}!zzzBraun, Felix@\emph{von Felix Braun}!1924-10-021@{2. 10. 1924}|(be}
\toendnotes[C]{\smallbreak\pagebreak[2]}\Standort{DLA, A:Schnitzler, HS.NZ85.1.2604,3.}
\physDesc{Brief, 1 Blatt, 2 Seiten, 1502 Zeichen
\newline{}Handschrift: schwarze Tinte, deutsche Kurrent
\newline{}Schnitzler: 1) mit Bleistift beschriftet: »\textsc{Braun}« und »\textsc{Sievringerstr. 191}\oindex{Sieveringer Strasse@\textbf{Sieveringer Straße}, \emph{Straße (K.STR)}|pw}«  2) mit rotem Buntstift mehrere Unterstreichungen}\toendnotes[C]{\smallbreak}
\pstart
           \centering{}{\pb}Wien\oindex{Wien@\textbf{Wien}, \emph{A.ADM2}|pw}, den 2. X. 1924\pend
           
\pstart{}Sehr verehrter Herr Doktor!\pend\vspace{0.5em}
\pstart
           Als ich heute vom Verlag Fiſcher\orgindex{S. Fischer Verlag@S. Fischer Verlag|pw} Ihre neue Komödie\pwindex{Komoedie der Verfuehrung. In drei Akten@\emph{Komödie der Verführung. In drei Akten}|pwv} erhielt und in dem Buch\pwindex{Komoedie der Verfuehrung. In drei Akten@\emph{Komödie der Verführung. In drei Akten}|pwv} den Vermerk: »Im Auftrag
               des Verfaſſers« fand, war ich ſehr ſtolz und erfreut: ſeien Sie herzlich bedankt für
               dieſe Auszeichnung!\pend
           
\pstart
           Ich habe auch das Werk\pwindex{Komoedie der Verfuehrung. In drei Akten@\emph{Komödie der Verführung. In drei Akten}|pwv} ſofort
               zu leſen begonnen und jetzt – es ist ſpät nachts – den ergreifenden, tiefen dritten
               Akt beendigt.\pend
           
\pstart
           Es iſt ein großes, reines Dichtwerk, eine Art dramatiſcher Roman, wenn ich mich ſo
               ausdrücken darf. Die Geſtalt Falkenirs\pwindex{Komoedie der Verfuehrung. In drei Akten@\emph{Komödie der Verführung. In drei Akten}|pwv} ging mir am nächſten. In Aurelie\pwindex{Komoedie der Verfuehrung. In drei Akten@\emph{Komödie der Verführung. In drei Akten}|pwv} iſt das Weibliche als das Allmögliche des Erlebens
               endgültig geſtaltet; Falkenirs\pwindex{Komoedie der Verfuehrung. In drei Akten@\emph{Komödie der Verführung. In drei Akten}|pwv}
               Schuld geht daran hervor. Man lebt ſich ſehr in dieſe Welt ein und möchte ſich eine
               Fortſetzung wün{\pb}ſchen. Den erſten und den dritten Akt
               halte ich für die ſchönſten des Stückes; der zweite ſteht für mein Gefühl etwas
               zurück. Der dritte iſt myſtiſch, wächſt gegen den Schluß immer höher ins
               Bedeutungsvolle und gewinnt immer noch an Poeſie. Über Einzelheiten voll tiefen
               Einblicks möchte ich in dieſem kurzen Brief gar nicht erſt ſprechen. Die ſchöne
               Stelle über die Liebe als Kampf nur darf ich hervorheben. Das Buch\pwindex{Komoedie der Verfuehrung. In drei Akten@\emph{Komödie der Verführung. In drei Akten}|pwv} hat mir viel gegeben: ich danke Ihnen
               von Herzen dafür!\pend
           
\pstart
           Darf ich Sie nun bitten, verehrter Herr Doktor, als eine – freilich im Abſtand zu
               betrachtende – Gegengabe meine beiden letzterſchienenen Bücher\pwindex{Wunderstunden. Drei Erzaehlungen@\emph{Wunderstunden. Drei Erzählungen}|pwv}\pwindex{unsichtbare Gast@\emph{Der unsichtbare Gast}|pwv} von mir anzunehmen? Es
               würde mich ſehr freuen, wenn Ihnen das eine oder das andere ein weniges zu ſagen
               hätte.\pend
           
\pstart
           In dieſer Zuverſicht bin ich, verehrter Herr Doktor, Ihr ergebener{\\[\baselineskip]}\spacefill\mbox{Felix Braun.}\pend
           \leftskip=0em{}\selectlanguage{ngerman}\endnumbering\briefempfaengerindex{Schnitzler, Arthur@\textsc{Schnitzler, Arthur}!zzzBraun, Felix@\emph{von Felix Braun}!1924-10-021@{2. 10. 1924}|)be}\mylabel{L02415h}  \normalsize

\doendnotes{C}
\bigskip
\vfill

\clearpage

\footnotesize

\lohead{\textsc{register}}

% Definiere theindex-Environment komplett neu ohne reledmac
\makeatletter
\renewenvironment{theindex}{%
  \section*{\indexname}%
  \setlength{\parindent}{0pt}%
  \setlength{\parskip}{0pt plus 0.3pt}%
  \let\item\@idxitem
}{%
  \clearpage
}
\makeatother

\IfFileExists{\jobname-pw.ind}{\input{\jobname-pw.ind}}{}

\end{document}

      