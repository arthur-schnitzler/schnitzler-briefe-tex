%% latex-korrekturansicht-vorspann.tex
%% Vorspann für die Korrekturansicht.
%% Lädt die gemeinsame Datei latex-vorspann.tex mit gesetztem Schalter.

\newif\ifkorrekturansicht
\korrekturansichttrue

\input{../tex-inputs/latex-vorspann}


\section[Paul Goldmann an Arthur Schnitzler, 29. 11. {[}1895{]}]{L02757 Paul Goldmann an Arthur Schnitzler, 29. 11. {[}1895{]}}
\nopagebreak\mylabel{L02757v}
\rehead{ }\normalsize\beginnumbering\briefempfaengerindex{Schnitzler, Arthur@\textsc{Schnitzler, Arthur}!zzzGoldmann, Paul@\emph{von Paul Goldmann}!1895-11-291@{29. 11. {[}1895{]}}|(be}
\toendnotes[C]{\smallbreak\pagebreak[2]}\Standort{DLA, A:Schnitzler, HS.NZ85.1.3165.}
\physDesc{Brief, 1 Blatt, 4 Seiten, 1335 Zeichen
\newline{}Handschrift: blaue Tinte, deutsche Kurrent
\newline{}Schnitzler: 1) mit Bleistift das Jahr »95« vermerkt  2) mit rotem Buntstift eine Unterstreichung sowie den Schreibirrtum »Liebelei\pwindex{Liebelei. Schauspiel in drei Akten@\emph{Liebelei. Schauspiel in drei Akten}|pw}« auf der zweiten Seite umrahmt und dazu »\textsc{Kl. K.\pwindex{kleine Komoedie@\emph{Die kleine Komödie}|pwv}}« (Kleine Komödie\pwindex{kleine Komoedie@\emph{Die kleine Komödie}|pw})
                                 vermerkt}\toendnotes[C]{\smallbreak}
\pstart
           {\pb}\textcolor{gray}{\textbf{\textbf{Frankfurter Zeitung\orgindex{Frankfurter Zeitung@Frankfurter Zeitung|pw}}}}\pend
           
\pstart
           \textcolor{gray}{\textbf{(\begin{otherlanguage}{french}Gazette de Francfort\end{otherlanguage}\orgindex{Frankfurter Zeitung@Frankfurter Zeitung|pw}). }}\pend
           
\pstart
           \textcolor{gray}{\textbf{\textbf{\begin{otherlanguage}{french}Fondateur M. L.
                              Sonnemann\pwindex{Sonnemann, Leopold 1831-10-29 – 1909-10-30@\textsc{Sonnemann, Leopold} (1831-10-29 – 1909-10-30), \emph{Journalist/Journalistin, Herausgeber/Herausgeberin}|pw}\end{otherlanguage}.}}}\pend
           
\pstart
           \begin{otherlanguage}{french}\textcolor{gray}{\textbf{Journal politique, financier,}}\end{otherlanguage}\pend
           
\pstart
           \begin{otherlanguage}{french}\textcolor{gray}{\textbf{commercial et littéraire.}}\end{otherlanguage}\pend
           
\pstart
           \begin{otherlanguage}{french}\textcolor{gray}{\textbf{\textbf{Paraissant trois fois par jour.}}}\end{otherlanguage}\pend
           
\pstart
           \begin{otherlanguage}{french}\textcolor{gray}{\textbf{\textbf{Bureau à Paris\oindex{Paris@\textbf{Paris}, \emph{P.PPLC}|pw}:}}}\end{otherlanguage}\pend
           
\pstart
           \begin{otherlanguage}{french}\textcolor{gray}{\textbf{\textbf{24. Rue Feydeau\oindex{rue Feydeau@\textbf{rue Feydeau}, \emph{Straße (K.STR)}|pw}.}}}\end{otherlanguage}\hfill \textsc{Paris\oindex{Paris@\textbf{Paris}, \emph{P.PPLC}|pw}}, 29. November.\pend
           
\pstart\center{}Mein lieber Freund,\pend\vspace{0.5em}
\pstart
           Dieſen Deinen Brief habe ich mit Sorge aufgemacht. Was wirſt Du ſagen? Ich bin ſo
               ſchuldbewußt! Aber ich finde keinen Vorwurf. Gott ſei Dank\textcolor{gray}{!}\pend
           
\pstart
           Tolle Arbeit, liebſter Freund, tolle Arbeit und wüſtes Leben! Ich komme zu nichts
               mehr. Aber in einigen Tagen ſchreibe ich Dir doch.\pend
           
\pstart
           Hier die Druckſachen. Die Bemerkungen dazu muß ich mir für ſpäter aufſparen. Denn
               gleich geht die Kammer\orgindex{Franzoesische Abgeordnetenkammer@Französische Abgeordnetenkammer|pw} an.\pend
           
\pstart
           {\pb}Die Überſetzung\pwindex{petite comedie. Mœurs viennois@\emph{La petite comédie. Mœurs viennois}|pwv} der »\label{K_L02757-1v}\edtext{Liebelei\pwindex{Liebelei. Schauspiel in drei Akten@\emph{Liebelei. Schauspiel in drei Akten}|pw}}{\lemma{\textnormal{\emph{Liebelei}}}\Cendnote{\textnormal{Schreibirrtum: Er meint \emph{Die kleine Komödie}\pwindex{kleine Komoedie@\emph{Die kleine Komödie}|pwk}.
               }}}\label{K_L02757-1}« ſinde ich vorzüglich. Schreib’, bittae, an Frau \textsc{Aubry\pwindex{Aubry, [MMe. Georges] @\textsc{Aubry, [MMe. Georges]}, \emph{Übersetzer/Übersetzerin}|pw}} – deutſch – ein artiges Wort darüber; danke auch dem Manne\pwindex{Aubry, Georges †~1923@\textsc{Aubry, Georges} (†~1923), \emph{Redakteur/Redakteurin}|pwv}, daß er es in die »\textsc{Liberté\pwindex{Liberte@\emph{La Liberté}|pw}}« gebracht hat; denn das war nicht leicht \strikeout{durz}
               durchzuſetzen bei dem prüden u. etwas chauviniſtiſchen \textsc{Bourgéois}-Blatte\pwindex{Liberte@\emph{La Liberté}|pwv}. \introOben{}(Adreſſe \textsc{10. Rue Caron\oindex{rue Caron@\textbf{rue Caron}, \emph{Straße (K.STR)}|pw}}).\introOben{} Die Exemplare\pwindex{Liberte@\emph{La Liberté}|pwv}
               will ich Dir zu verſchaffen ſuchen; aber ich fürchte, man wird ſie zahlen müſſen.\pend
           
\pstart
           {\pb}Vielen Dank für die \textsc{Strauss\pwindex{Strauss, Johann 25.10.1825 – 03.06.1899@\textsc{Strauss, Johann} (25.10.1825 – 03.06.1899), \emph{Komponist/Komponistin, Dirigent/Dirigentin}|pw}}-Empfehlung. Auch hat mir \textsc{Richard\pwindex{Beer-Hofmann, Richard 1866-07-11 – 1945-09-26@\textsc{Beer-Hofmann, Richard} (1866-07-11 – 1945-09-26), \emph{Schriftsteller/Schriftstellerin}|pw}} den \textsc{Hogarth\pwindex{Hogarth, William 1697-11-10 – 1764-10-25@\textsc{Hogarth, William} (1697-11-10 – 1764-10-25), \emph{Maler/Malerin, Kupferstecher/Kupferstecherin}|pw}} geſchickt, wofür ich ihm von Herzen danke. Auch ihm ſchreibe ich einen dieſer
               Tage.\pend
           
\pstart
           \textsc{Herzl\pwindex{Herzl, Theodor 1860-05-02 – 1904-07-03@\textsc{Herzl, Theodor} (1860-05-02 – 1904-07-03), \emph{Schriftsteller/Schriftstellerin, Journalist/Journalistin}|pw}} war hier. Er iſt mir unſagbar widerwärtig.\pend
           
\pstart
           Wüſtes Leben, mein lieber Freund! Ich will in \textsc{Paris\oindex{Paris@\textbf{Paris}, \emph{P.PPLC}|pw}} verſchwinden, will mich gegen draußen abſperren, von wo mir jeder Luftzug die
               Kunde meiner {\pb}verfehlten Exiſtenz bringt. Bin müde,
               zu kämpfen, und möchte leben, oh nur ein einziges Mal!\pend
           
\pstart
           Grüß’ Dich Gott! {\\[\baselineskip]}Dein treuer {\\[\baselineskip]}\spacefill\mbox{Paul Goldmann}\pend
           \leftskip=0em{}
\pstart
           \noindent{}Viele Grüße an die liebe Frau\pwindex{Andreas-Salome, Lou 12.02.1861 – 05.02.1937@\textsc{Andreas-Salomé, Lou} (12.02.1861 – 05.02.1937), \emph{Schriftsteller/Schriftstellerin}|pwv}, die wieder in \textsc{Wien\oindex{Wien@\textbf{Wien}, \emph{A.ADM2}|pw}} iſt.\pend
           \selectlanguage{ngerman}\endnumbering\briefempfaengerindex{Schnitzler, Arthur@\textsc{Schnitzler, Arthur}!zzzGoldmann, Paul@\emph{von Paul Goldmann}!1895-11-291@{29. 11. {[}1895{]}}|)be}\mylabel{L02757h}  \normalsize

\doendnotes{C}
\bigskip
\vfill

\clearpage

\footnotesize

\lohead{\textsc{register}}

% Definiere theindex-Environment komplett neu ohne reledmac
\makeatletter
\renewenvironment{theindex}{%
  \section*{\indexname}%
  \setlength{\parindent}{0pt}%
  \setlength{\parskip}{0pt plus 0.3pt}%
  \let\item\@idxitem
}{%
  \clearpage
}
\makeatother

\IfFileExists{\jobname-pw.ind}{\input{\jobname-pw.ind}}{}

\end{document}

      