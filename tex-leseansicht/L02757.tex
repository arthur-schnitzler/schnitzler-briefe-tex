%% latex-leseansicht-vorspann.tex
%% Vorspann für die Leseansicht.
%% Lädt die gemeinsame Datei latex-vorspann.tex mit nicht gesetztem Schalter.

\newif\ifkorrekturansicht
\korrekturansichtfalse

\input{../tex-inputs/latex-vorspann}


         
         \renewcommand{\erwaehntePersonen}{Personen: Lou Andreas-Salomé, [MMe. Georges] Aubry, Georges Aubry, Richard Beer-Hofmann, Paul Goldmann, Theodor Herzl, William Hogarth, Leopold Sonnemann, Johann Strauss}
         \renewcommand{\erwaehnteInstitutionen}{Institutionen: Frankfurter Zeitung, Französische Abgeordnetenkammer}
         \renewcommand{\erwaehnteOrte}{Orte: Paris, Wien, rue Caron, rue Feydeau}
         \renewcommand{\erwaehnteWerke}{Werke: Die kleine Komödie, La Liberté, La petite comédie. Mœurs viennois, Liebelei. Schauspiel in drei Akten}
               \section[Paul Goldmann an Arthur Schnitzler, 29. 11. {[}1895{]}]{ Paul Goldmann an Arthur Schnitzler, 29. 11. {[}1895{]}}\nopagebreak\mylabel{v}\rehead{ }\begin{ledgroupsized}[t]{13cm}\normalsize\beginnumbering \toendnotes[C]{\smallbreak\pagebreak[2]} \Standort{DLA, A:Schnitzler, HS.NZ85.1.3165.}
\physDesc{Brief, 1 Blatt, 4 Seiten, 1335 Zeichen
\newline{}Handschrift: blaue Tinte, deutsche Kurrent
\newline{}Schnitzler: 1) mit Bleistift das Jahr »95« vermerkt  2) mit rotem Buntstift eine Unterstreichung sowie den Schreibirrtum »Liebelei\pwindex{Schnitzler, Arthur 15.05.1862 – 21.10.1931@\textsc{Schnitzler, Arthur} (15.05.1862 – 21.10.1931), \emph{Schriftsteller, Mediziner}!Liebelei. Schauspiel in drei Akten1895-10-09@\strich\emph{Liebelei. Schauspiel in drei Akten} {[}1895-10-09{]}|pw}« auf der zweiten Seite umrahmt und dazu »\textsc{Kl. K.\pwindex{Schnitzler, Arthur 15.05.1862 – 21.10.1931@\textsc{Schnitzler, Arthur} (15.05.1862 – 21.10.1931), \emph{Schriftsteller, Mediziner}!kleine Komoedie1895-08-01@\strich\emph{Die kleine Komödie} {[}1895-08-01{]}|pwv}}« (Kleine Komödie\pwindex{Schnitzler, Arthur 15.05.1862 – 21.10.1931@\textsc{Schnitzler, Arthur} (15.05.1862 – 21.10.1931), \emph{Schriftsteller, Mediziner}!kleine Komoedie1895-08-01@\strich\emph{Die kleine Komödie} {[}1895-08-01{]}|pw})
                                 vermerkt}\toendnotes[C]{\smallbreak}\pstart
           \noindent{}{\pb}\textcolor{gray}{\textbf{\textbf{Frankfurter Zeitung\orgindex{Frankfurter Zeitung@Frankfurter Zeitung|pw}}}}\pend
           \pstart
           \textcolor{gray}{\textbf{(\begin{otherlanguage}{french}Gazette de Francfort\end{otherlanguage}\orgindex{Frankfurter Zeitung@Frankfurter Zeitung|pw}). }}\pend
           \pstart
           \textcolor{gray}{\textbf{\textbf{\begin{otherlanguage}{french}Fondateur M. L.
                              Sonnemann\pwindex{Sonnemann, Leopold 1831-10-29 – 1909-10-30@\textsc{Sonnemann, Leopold} (1831-10-29 – 1909-10-30), \emph{Journalist, Herausgeber}|pw}\end{otherlanguage}.}}}\pend
           \pstart
           \begin{otherlanguage}{french}\textcolor{gray}{\textbf{Journal politique, financier,}}\end{otherlanguage}\pend
           \pstart
           \begin{otherlanguage}{french}\textcolor{gray}{\textbf{commercial et littéraire.}}\end{otherlanguage}\pend
           \pstart
           \begin{otherlanguage}{french}\textcolor{gray}{\textbf{\textbf{Paraissant trois fois par jour.}}}\end{otherlanguage}\pend
           \pstart
           \begin{otherlanguage}{french}\textcolor{gray}{\textbf{\textbf{Bureau à Paris\oindex{Paris@\textbf{Paris}|pw}:}}}\end{otherlanguage}\pend
           \pstart
           \begin{otherlanguage}{french}\textcolor{gray}{\textbf{\textbf{24. Rue Feydeau\oindex{rue Feydeau@\textbf{rue Feydeau}|pw}.}}}\end{otherlanguage}\hfill \textsc{Paris\oindex{Paris@\textbf{Paris}|pw}}, 29. November.\pend
           \pstart\center{}Mein lieber Freund,\pend\pstart
           Dieſen Deinen Brief habe ich mit Sorge aufgemacht. Was wirſt Du ſagen? Ich bin ſo
               ſchuldbewußt! Aber ich finde keinen Vorwurf. Gott ſei Dank\textcolor{gray}{!}\pend
           \pstart
           Tolle Arbeit, liebſter Freund, tolle Arbeit und wüſtes Leben! Ich komme zu nichts
               mehr. Aber in einigen Tagen ſchreibe ich Dir doch.\pend
           \pstart
           Hier die Druckſachen. Die Bemerkungen dazu muß ich mir für ſpäter aufſparen. Denn
               gleich geht die Kammer\orgindex{Franzoesische Abgeordnetenkammer@Französische Abgeordnetenkammer|pw} an.\pend
           \pstart
           {\pb}Die Überſetzung\pwindex{Schnitzler, Arthur 15.05.1862 – 21.10.1931@\textsc{Schnitzler, Arthur} (15.05.1862 – 21.10.1931), \emph{Schriftsteller, Mediziner}!petite comedie. Mœurs viennois1895-11-19 – 1895-11-28@\strich\emph{La petite comédie. Mœurs viennois} {[}1895-11-19 – 1895-11-28{]}|pwv} der »\label{K_L02757-1v}\edtext{Liebelei\pwindex{Schnitzler, Arthur 15.05.1862 – 21.10.1931@\textsc{Schnitzler, Arthur} (15.05.1862 – 21.10.1931), \emph{Schriftsteller, Mediziner}!Liebelei. Schauspiel in drei Akten1895-10-09@\strich\emph{Liebelei. Schauspiel in drei Akten} {[}1895-10-09{]}|pw}}{\lemma{\textnormal{\emph{Liebelei}}}\Cendnote{\textnormal{Schreibirrtum: er meint \emph{Die kleine Komödie}\pwindex{Schnitzler, Arthur 15.05.1862 – 21.10.1931@\textsc{Schnitzler, Arthur} (15.05.1862 – 21.10.1931), \emph{Schriftsteller, Mediziner}!kleine Komoedie1895-08-01@\strich\emph{Die kleine Komödie} {[}1895-08-01{]}|pwk}}}}\label{K_L02757-1h}« ſinde ich vorzüglich. Schreib’, bittae, an Frau \textsc{Aubry\pwindex{Aubry, [MMe. Georges] @\textsc{Aubry, [MMe. Georges]}, \emph{Übersetzerin}|pw}} – deutſch – ein artiges Wort darüber; danke auch dem Manne\pwindex{Aubry, Georges †~1923@\textsc{Aubry, Georges} (†~1923), \emph{Redakteur}|pwv}, daß er es in die »\textsc{Liberté\pwindex{?? Werk@Nicht ermittelte Verfasserinnen und Verfasser!Liberte1865-07-16 – 1940-06-11@\emph{La Liberté} {[}1865-07-16 – 1940-06-11{]}|pw}}« gebracht hat; denn das war nicht leicht \strikeout{durz}
               durchzuſetzen bei dem prüden u. etwas chauviniſtiſchen \textsc{Bourgéois}-Blatte\pwindex{?? Werk@Nicht ermittelte Verfasserinnen und Verfasser!Liberte1865-07-16 – 1940-06-11@\emph{La Liberté} {[}1865-07-16 – 1940-06-11{]}|pwv}. \introOben{}(Adreſſe \textsc{10. Rue Caron\oindex{rue Caron@\textbf{rue Caron}|pw}}).\introOben{} Die Exemplare\pwindex{?? Werk@Nicht ermittelte Verfasserinnen und Verfasser!Liberte1865-07-16 – 1940-06-11@\emph{La Liberté} {[}1865-07-16 – 1940-06-11{]}|pwv}
               will ich Dir zu verſchaffen ſuchen; aber ich fürchte, man wird ſie zahlen müſſen.\pend
           \pstart
           {\pb}Vielen Dank für die \textsc{Strauss\pwindex{Strauss, Johann 25.10.1825 – 03.06.1899@\textsc{Strauss, Johann} (25.10.1825 – 03.06.1899), \emph{Komponist, Dirigent}|pw}}-Empfehlung. Auch hat mir \textsc{Richard\pwindex{Beer-Hofmann, Richard 1866-07-11 – 1945-09-26@\textsc{Beer-Hofmann, Richard} (1866-07-11 – 1945-09-26), \emph{Schriftsteller}|pw}} den \textsc{Hogarth\pwindex{Hogarth, William 1697-11-10 – 1764-10-25@\textsc{Hogarth, William} (1697-11-10 – 1764-10-25), \emph{Bildender Künstler, Kupferstecher}|pw}} geſchickt, wofür ich ihm von Herzen danke. Auch ihm ſchreibe ich einen dieſer
               Tage.\pend
           \pstart
           \textsc{Herzl\pwindex{Herzl, Theodor 1860-05-02 – 1904-07-03@\textsc{Herzl, Theodor} (1860-05-02 – 1904-07-03), \emph{Schriftsteller, Journalist}|pw}} war hier. Er iſt mir unſagbar widerwärtig.\pend
           \pstart
           Wüſtes Leben, mein lieber Freund! Ich will in \textsc{Paris\oindex{Paris@\textbf{Paris}|pw}} verſchwinden, will mich gegen draußen abſperren, von wo mir jeder Luftzug die
               Kunde meiner {\pb}verfehlten Exiſtenz bringt. Bin müde,
               zu kämpfen, und möchte leben, oh nur ein einziges Mal!\pend
           \pstart
           Grüß’ Dich Gott! {\\[\baselineskip]}Dein treuer {\\[\baselineskip]}\spacefill\mbox{Paul Goldmann}\pend
           \leftskip=0em{}\pstart
           \noindent{}Viele Grüße an die liebe Frau\pwindex{Andreas-Salome, Lou 12.02.1861 – 05.02.1937@\textsc{Andreas-Salomé, Lou} (12.02.1861 – 05.02.1937), \emph{Schriftstellerin}|pwv}, die wieder in \textsc{Wien\oindex{Wien@\textbf{Wien}|pw}} iſt.\pend
           
         
         \endnumbering\mylabel{h}\end{ledgroupsized}  \newcommand{\dateiname}{L02757}\newcommand{\titel}{Paul Goldmann an Arthur Schnitzler, 29. 11. [1895]}\newcommand{\editorInnen}{Martin Anton Müller und Laura Untner}%% latex-leseansicht-abspann.tex
%% Abspann für die Leseansicht.
%% Der Schalter \ifkorrekturansicht ist bereits durch den Vorspann gesetzt.

%% latex-abspann.tex
%% Gemeinsamer Abspann für Korrekturansicht und Leseansicht.
%% Setzt den Schalter \ifkorrekturansicht voraus (gesetzt in den
%% einbindenden Dateien latex-korrekturansicht-abspann.tex bzw.
%% latex-leseansicht-abspann.tex).
%% ---------------------------------------------------------------

\normalsize

% Das esempio-Environment wird nur in der Leseansicht benötigt
\ifkorrekturansicht\else
\newenvironment{esempio}[3]%
{
    \vspace{1.5ex}
    \rlap{\underline{#1}}
    \par
    \setlength{\parindent}{0cm}
    \nopagebreak
    \leftskip=#2cm
    \rightskip=#3cm
}
{
    \par
}
\fi

\doendnotes{C}
\bigskip
\vfill

\clearpage

\footnotesize

\ifkorrekturansicht
  \lohead{\textsc{register}}
\fi

% theindex-Environment neu definieren ohne reledmac
\makeatletter
\renewenvironment{theindex}{%
  \ifkorrekturansicht
    \section*{\indexname}%
  \else
    \subsubsection*{Index der erwähnten Entitäten}%
  \fi
  \setlength{\parindent}{0pt}%
  \setlength{\parskip}{0pt plus 0.3pt}%
  \let\item\@idxitem
}{%
  \ifkorrekturansicht\clearpage\fi
}
\makeatother

\IfFileExists{\jobname-pw.ind}{\input{\jobname-pw.ind}}{}

% Quellenangabe nur in der Leseansicht
\ifkorrekturansicht\else
% Fallback-Definitionen, falls die .tex-Datei \titel etc. nicht gesetzt hat
\providecommand{\titel}{}
\providecommand{\editorInnen}{}
\providecommand{\dateiname}{\jobname}

\vspace{3cm}

\vfill

\footnotesize
\textsc{Quelle}: \titel. Herausgegeben von {\editorInnen}. In: \emph{Arthur Schnitzler: Briefwechsel mit Autorinnen und Autoren}.
 Digitale Edition, https://schnitzler-briefe.acdh.oeaw.ac.at/{\dateiname}.html (Stand \today)
\fi

\end{document}


      