%% latex-leseansicht-vorspann.tex
%% Vorspann für die Leseansicht.
%% Lädt die gemeinsame Datei latex-vorspann.tex mit nicht gesetztem Schalter.

\newif\ifkorrekturansicht
\korrekturansichtfalse

\input{../tex-inputs/latex-vorspann}

\begin{center}
            \textcolor{red}{ENTWURF. ENTZIFFERUNG NOCH NICHT KORREKTURGELESEN}
                      \end{center}
            
               \section[Arthur Schnitzler an Thomas Mann, 26. 6. 1922]{ Arthur Schnitzler an Thomas Mann, 26. 6. 1922}\nopagebreak\mylabel{v}\rehead{ }\begin{ledgroupsized}[t]{13cm}\normalsize\beginnumbering\briefempfaengerindex{Mann, Thomas@\textsc{Mann, Thomas}!zzzSchnitzler, Arthur@\emph{von Arthur Schnitzler}!1922-06-262@{26. 6. 1922}|(be} \toendnotes[C]{\smallbreak\pagebreak[2]} \Standort{Zürich, Thomas-Mann-Archiv, B-II-SCHNM-1.}
\physDesc{Briefkarte
\newline{}Handschrift: schwarze Tinte, lateinische Kurrent}\buchAbdrucke{\weitereDrucke{Hertha Krotkoff: \emph{Arthur Schnitzler – Thomas Mann: Briefe.} In: \emph{Modern Austrian Literature}, Jg. 7 (1974) Nr. 1/2, S. 17–18.} }\toendnotes[C]{\smallbreak}\pstart
           \raggedleft{}{\pb}Wien\oindex{Wien@\textbf{Wien}|pw}{ }26. 6. 22\pend
           \pstart
           Verehrter und lieber Herr Thomas Mann, erlauben Sie, dſs ich
                    Ihnen Mr. Scofield Thayer\pwindex{Thayer, Scofield 12.12.1889 – 09.07.1982@\textsc{Thayer, Scofield} (12.12.1889 – 09.07.1982), \emph{Journalist, Herausgeber}|pw} vorstelle, den
                    Herausgeber der »Dial\orgindex{Dial@The Dial|pw},« der Ihre Werke liebt
                    und bewundert. Mr. Thayer\pwindex{Thayer, Scofield 12.12.1889 – 09.07.1982@\textsc{Thayer, Scofield} (12.12.1889 – 09.07.1982), \emph{Journalist, Herausgeber}|pw} hat sich fast ein
                    Jahr lang in Wien\oindex{Wien@\textbf{Wien}|pw} aufgehalten, ich habe höchst
                    anregende Stunden mit ihm verbracht; und so kostbar Ihre Zeit ist – ich glaube,
                    daß auch Ihnen die Beka{\geminationn}tschaft mit diesem auf
                    vielen Gebieten interessanten, um die {\pb}Verbreitung deutscher Literatur in Amerika\oindex{Amerika@\textbf{Amerika}|pw}
                    höchst verdienten und wahrhaft liebenswürdigen jungen Mannes\pwindex{Thayer, Scofield 12.12.1889 – 09.07.1982@\textsc{Thayer, Scofield} (12.12.1889 – 09.07.1982), \emph{Journalist, Herausgeber}|pwv} nicht unangenehm sein wird.\pend
           \pstart
           Darf ich hier meinen herzlichen Dank für die schönen Worte\pwindex{\textcolor{red}{\textsuperscript{XXXX1 indx}}!Arthur Schnitzler zu seinem sechzigsten Geburtstag01. 05. 1922@\strich\emph{Arthur Schnitzler zu seinem sechzigsten Geburtstag} {[}01. 05. 1922{]}|pwv}\pwindex{\textcolor{red}{\textsuperscript{XXXX1 indx}}!Arthur Schnitzler zu seinem sechzigsten Geburtstag01. 05. 1922@\strich\emph{Arthur Schnitzler zu seinem sechzigsten Geburtstag} {[}01. 05. 1922{]}|pwv}\pwindex{\textcolor{red}{\textsuperscript{XXXX1 indx}}!Arthur Schnitzler zu seinem sechzigsten Geburtstag01. 05. 1922@\strich\emph{Arthur Schnitzler zu seinem sechzigsten Geburtstag} {[}01. 05. 1922{]}|pwv}\pwindex{\textcolor{red}{\textsuperscript{XXXX1 indx}}!Arthur Schnitzler zu seinem sechzigsten Geburtstag01. 05. 1922@\strich\emph{Arthur Schnitzler zu seinem sechzigsten Geburtstag} {[}01. 05. 1922{]}|pwv}\pwindex{\textcolor{red}{\textsuperscript{XXXX1 indx}}!Arthur Schnitzler zu seinem sechzigsten Geburtstag01. 05. 1922@\strich\emph{Arthur Schnitzler zu seinem sechzigsten Geburtstag} {[}01. 05. 1922{]}|pwv}\pwindex{\textcolor{red}{\textsuperscript{XXXX1 indx}}!Arthur Schnitzler zu seinem sechzigsten Geburtstag01. 05. 1922@\strich\emph{Arthur Schnitzler zu seinem sechzigsten Geburtstag} {[}01. 05. 1922{]}|pwv}\pwindex{\textcolor{red}{\textsuperscript{XXXX1 indx}}!Arthur Schnitzler zu seinem sechzigsten Geburtstag01. 05. 1922@\strich\emph{Arthur Schnitzler zu seinem sechzigsten Geburtstag} {[}01. 05. 1922{]}|pwv}\pwindex{\textcolor{red}{\textsuperscript{XXXX1 indx}}!Arthur Schnitzler zu seinem sechzigsten Geburtstag01. 05. 1922@\strich\emph{Arthur Schnitzler zu seinem sechzigsten Geburtstag} {[}01. 05. 1922{]}|pwv}\pwindex{\textcolor{red}{\textsuperscript{XXXX1 indx}}!Arthur Schnitzler zu seinem sechzigsten Geburtstag01. 05. 1922@\strich\emph{Arthur Schnitzler zu seinem sechzigsten Geburtstag} {[}01. 05. 1922{]}|pwv}\pwindex{\textcolor{red}{\textsuperscript{XXXX1 indx}}!Arthur Schnitzler zu seinem sechzigsten Geburtstag01. 05. 1922@\strich\emph{Arthur Schnitzler zu seinem sechzigsten Geburtstag} {[}01. 05. 1922{]}|pwv}\pwindex{\textcolor{red}{\textsuperscript{XXXX1 indx}}!Arthur Schnitzler zu seinem sechzigsten Geburtstag01. 05. 1922@\strich\emph{Arthur Schnitzler zu seinem sechzigsten Geburtstag} {[}01. 05. 1922{]}|pwv}\pwindex{Mann, Thomas 06.06.1875 – 12.08.1955@\textsc{Mann, Thomas} (06.06.1875 – 12.08.1955), \emph{Schriftsteller}!Arthur Schnitzler zu seinem sechzigsten Geburtstag01. 05. 1922@\strich\emph{Arthur Schnitzler zu seinem sechzigsten Geburtstag} {[}01. 05. 1922{]}|pwv}\pwindex{\textcolor{red}{\textsuperscript{XXXX1 indx}}!Arthur Schnitzler zu seinem sechzigsten Geburtstag01. 05. 1922@\strich\emph{Arthur Schnitzler zu seinem sechzigsten Geburtstag} {[}01. 05. 1922{]}|pwv}\pwindex{\textcolor{red}{\textsuperscript{XXXX1 indx}}!Arthur Schnitzler zu seinem sechzigsten Geburtstag01. 05. 1922@\strich\emph{Arthur Schnitzler zu seinem sechzigsten Geburtstag} {[}01. 05. 1922{]}|pwv}\pwindex{\textcolor{red}{\textsuperscript{XXXX1 indx}}!Arthur Schnitzler zu seinem sechzigsten Geburtstag01. 05. 1922@\strich\emph{Arthur Schnitzler zu seinem sechzigsten Geburtstag} {[}01. 05. 1922{]}|pwv} anschließen, die Sie mir zu meinem immerhin
                    sechzigsten Geburtstag in der N. R.\pwindex{neue Rundschau1904@\emph{Die neue Rundschau}|pw} gewidmet
                    haben?\pend
           \pstart
           Ich sehe Sie hoffentlich bald wieder; und grüße Sie in freundschaftlicher
                    Bewunderung als Ihr ergebener \spacefill\mbox{Arthur Schnitzler}\pend
           \endnumbering\briefempfaengerindex{Mann, Thomas@\textsc{Mann, Thomas}!zzzSchnitzler, Arthur@\emph{von Arthur Schnitzler}!1922-06-262@{26. 6. 1922}|)be}\mylabel{h}\end{ledgroupsized}  \newcommand{\dateiname}{L02390}\newcommand{\titel}{Arthur Schnitzler an Thomas Mann, 26. 6. 1922}\newcommand{\editorInnen}{Martin Anton Müller und Gerd-Hermann Susen}%% latex-leseansicht-abspann.tex
%% Abspann für die Leseansicht.
%% Der Schalter \ifkorrekturansicht ist bereits durch den Vorspann gesetzt.

%% latex-abspann.tex
%% Gemeinsamer Abspann für Korrekturansicht und Leseansicht.
%% Setzt den Schalter \ifkorrekturansicht voraus (gesetzt in den
%% einbindenden Dateien latex-korrekturansicht-abspann.tex bzw.
%% latex-leseansicht-abspann.tex).
%% ---------------------------------------------------------------

\normalsize

% Das esempio-Environment wird nur in der Leseansicht benötigt
\ifkorrekturansicht\else
\newenvironment{esempio}[3]%
{
    \vspace{1.5ex}
    \rlap{\underline{#1}}
    \par
    \setlength{\parindent}{0cm}
    \nopagebreak
    \leftskip=#2cm
    \rightskip=#3cm
}
{
    \par
}
\fi

\doendnotes{C}
\bigskip
\vfill

\clearpage

\footnotesize

\ifkorrekturansicht
  \lohead{\textsc{register}}
\fi

% theindex-Environment neu definieren ohne reledmac
\makeatletter
\renewenvironment{theindex}{%
  \ifkorrekturansicht
    \section*{\indexname}%
  \else
    \subsubsection*{Index der erwähnten Entitäten}%
  \fi
  \setlength{\parindent}{0pt}%
  \setlength{\parskip}{0pt plus 0.3pt}%
  \let\item\@idxitem
}{%
  \ifkorrekturansicht\clearpage\fi
}
\makeatother

\IfFileExists{\jobname-pw.ind}{\input{\jobname-pw.ind}}{}

% Quellenangabe nur in der Leseansicht
\ifkorrekturansicht\else
% Fallback-Definitionen, falls die .tex-Datei \titel etc. nicht gesetzt hat
\providecommand{\titel}{}
\providecommand{\editorInnen}{}
\providecommand{\dateiname}{\jobname}

\vspace{3cm}

\vfill

\footnotesize
\textsc{Quelle}: \titel. Herausgegeben von {\editorInnen}. In: \emph{Arthur Schnitzler: Briefwechsel mit Autorinnen und Autoren}.
 Digitale Edition, https://schnitzler-briefe.acdh.oeaw.ac.at/{\dateiname}.html (Stand \today)
\fi

\end{document}


      