%% latex-korrekturansicht-vorspann.tex
%% Vorspann für die Korrekturansicht.
%% Lädt die gemeinsame Datei latex-vorspann.tex mit gesetztem Schalter.

\newif\ifkorrekturansicht
\korrekturansichttrue

\input{../tex-inputs/latex-vorspann}


\section[Arthur Schnitzler an Thomas Mann, 26. 6. 1922]{L02390 Arthur Schnitzler an Thomas Mann, 26. 6. 1922}
\nopagebreak\mylabel{L02390v}
\rehead{ }\normalsize\beginnumbering\briefempfaengerindex{Mann, Thomas@\textsc{Mann, Thomas}!zzzSchnitzler, Arthur@\emph{von Arthur Schnitzler}!1922-06-262@{26. 6. 1922}|(be}
\toendnotes[C]{\smallbreak\pagebreak[2]}\Standort{Zürich, Thomas-Mann-Archiv, B-II-SCHNM-1.}
\physDesc{Briefkarte, 806 Zeichen
\newline{}Handschrift: schwarze Tinte, lateinische Kurrent}
\buchAbdrucke{\weitereDrucke{\emph{Modern Austrian Literature}, Jg. 7 (1974) Nr. 1/2, S. 17–18.} }\toendnotes[C]{\smallbreak}
\pstart
           \raggedleft{}{\pb}Wien\oindex{Wien@\textbf{Wien}, \emph{A.ADM2}|pw}{ }26. 6. 22\pend
           \vspace{0.5em}
\pstart
           Verehrter und lieber Herr Thomas Mann, erlauben Sie, dſs ich Ihnen
               Mr. Scofield Thayer\pwindex{Thayer, Scofield 12.12.1889 – 09.07.1982@\textsc{Thayer, Scofield} (12.12.1889 – 09.07.1982), \emph{Journalist/Journalistin, Herausgeber/Herausgeberin}|pw} vorstelle, den Herausgeber
               der »Dial\orgindex{Dial@The Dial|pw},« der Ihre Werke liebt und bewundert.
               Mr. Thayer\pwindex{Thayer, Scofield 12.12.1889 – 09.07.1982@\textsc{Thayer, Scofield} (12.12.1889 – 09.07.1982), \emph{Journalist/Journalistin, Herausgeber/Herausgeberin}|pw} hat sich fast ein Jahr lang in Wien\oindex{Wien@\textbf{Wien}, \emph{A.ADM2}|pw} aufgehalten, ich habe höchst anregende Stunden
               mit ihm verbracht; und so kostbar Ihre Zeit ist – ich glaube, daß auch Ihnen die
                  Beka{\geminationn}tschaft mit diesem auf vielen Gebieten
               interessanten, um die {\pb}Verbreitung deutscher Literatur in Amerika\oindex{Amerika@\textbf{Amerika}, \emph{kein passender Code gefunden}|pw}
               höchst verdienten und wahrhaft liebenswürdigen jungen Mannes\pwindex{Thayer, Scofield 12.12.1889 – 09.07.1982@\textsc{Thayer, Scofield} (12.12.1889 – 09.07.1982), \emph{Journalist/Journalistin, Herausgeber/Herausgeberin}|pwv} nicht unangenehm sein wird.\pend
           
\pstart
           Darf ich hier meinen herzlichen Dank für die schönen Worte\pwindex{Arthur Schnitzler. Zu seinem sechzigsten Geburtstag (15. Mai 1922)@\emph{Arthur Schnitzler. Zu seinem sechzigsten Geburtstag (15. Mai 1922)}|pwv} anschließen, die Sie mir zu meinem immerhin
               sechzigsten Geburtstag in der N. R.\pwindex{neue Rundschau@\emph{Die neue Rundschau}|pw} gewidmet
               haben?\pend
           
\pstart
           Ich sehe Sie hoffentlich bald wieder; und grüße Sie in freundschaftlicher Bewunderung
               als Ihr ergebener \spacefill\mbox{Arthur Schnitzler}\pend
           \selectlanguage{ngerman}\endnumbering\briefempfaengerindex{Mann, Thomas@\textsc{Mann, Thomas}!zzzSchnitzler, Arthur@\emph{von Arthur Schnitzler}!1922-06-262@{26. 6. 1922}|)be}\mylabel{L02390h}  \normalsize

\doendnotes{C}
\bigskip
\vfill

\clearpage

\footnotesize

\lohead{\textsc{register}}

% Definiere theindex-Environment komplett neu ohne reledmac
\makeatletter
\renewenvironment{theindex}{%
  \section*{\indexname}%
  \setlength{\parindent}{0pt}%
  \setlength{\parskip}{0pt plus 0.3pt}%
  \let\item\@idxitem
}{%
  \clearpage
}
\makeatother

\IfFileExists{\jobname-pw.ind}{\input{\jobname-pw.ind}}{}

\end{document}

      