%% latex-leseansicht-vorspann.tex
%% Vorspann für die Leseansicht.
%% Lädt die gemeinsame Datei latex-vorspann.tex mit nicht gesetztem Schalter.

\newif\ifkorrekturansicht
\korrekturansichtfalse

\input{../tex-inputs/latex-vorspann}


\section[Karl Kraus u. a. an Richard Dehmel, 10. 2. 1894]{L00297 Karl Kraus u. a. an Richard Dehmel, 10. 2. 1894}
\nopagebreak\mylabel{L00297v}
\rehead{ }\normalsize\beginnumbering\briefempfaengerindex{Dehmel, Richard@\textsc{Dehmel, Richard}!zzzHofmannsthal, Hugo von@\emph{von Hugo von Hofmannsthal}!1894-02-101@{10. 2. 1894}|(be}\briefempfaengerindex{Dehmel, Richard@\textsc{Dehmel, Richard}!zzzBeer-Hofmann, Richard@\emph{von Richard Beer-Hofmann}!1894-02-101@{10. 2. 1894}|(be}\briefempfaengerindex{Dehmel, Richard@\textsc{Dehmel, Richard}!zzzSchnitzler, Arthur@\emph{von Arthur Schnitzler}!1894-02-101@{10. 2. 1894}|(be}\briefempfaengerindex{Dehmel, Richard@\textsc{Dehmel, Richard}!zzzKraus, Karl@\emph{von Karl Kraus}!1894-02-101@{10. 2. 1894}|(be}
\toendnotes[C]{\smallbreak\pagebreak[2]}
\correspDesc{Versand  durch Karl Kraus, Arthur Schnitzler, Richard Beer-Hofmann, Hugo von Hofmannsthal am 10. 2. 1894 in Wien
\newline{}Erhalt  durch Richard Dehmel am 12. 2. 1894 in Berlin-Pankow}\toendnotes[C]{\smallbreak}
\Standort{Hamburg, Staats- und Universitätsbibliothek, DA:Br:K:282.}
\physDesc{Kartenbrief, 477 Zeichen
\newline{}Handschrift Karl Kraus: schwarze Tinte, deutsche Kurrent
\newline{}Handschrift Arthur Schnitzler: schwarze Tinte, deutsche Kurrent
\newline{}Handschrift Richard Beer-Hofmann: schwarze Tinte, deutsche Kurrent
\newline{}Handschrift Hugo von Hofmannsthal: schwarze Tinte, deutsche Kurrent
\newline{}Versand: 1) Stempel: »\nobreak{}\oindex{I., Innere Stadt@\textbf{I., Innere Stadt}, \emph{Verwaltungsgebiet}|pwk}Wien 1/1, 11. 2. 94, 8–9V\nobreak{}«.   2) Stempel: »\nobreak{}\oindex{Berlin-Pankow@\textbf{Berlin-Pankow}, \emph{Ehemaliger Ort}|pwk}Pankow bei Berlin, 12. 2. 94, 10–11V.\nobreak{}«. }
\buchAbdrucke{\weitereDrucke{Joachim Kersten, Friedrich Pfäfflin: \emph{Detlev von Liliencron entdeckt, gefeiert und gelesen von Karl
                        Kraus}. Göttingen: \emph{Wallstein} 2016, S. 116–117.} }\toendnotes[C]{\smallbreak}\pstart{}{\pb}Abſender: Karl Kraus, I.
                     Maximilianſtr 13\oindex{Wien@\textbf{Wien}!I., Innere Stadt@\textbf{I., Innere Stadt}!Mahlerstraße@\textbf{Mahlerstraße}, \emph{Straße}|pw}. \pend{}\pstart{}Wien\oindex{Wien@\textbf{Wien}, \emph{Verwaltungsgebiet}|pw}\pend{}\pstart{}Loris\pend{}\pstart{}Schnitzler\pend{}\pstart{}Beer-Hofmann\pend{}{\bigskip}\pstart{}Herrn\pend{}\pstart{}Richard Dehmel\pend{}\pstart{}Pankow bei Berlin\oindex{Berlin-Pankow@\textbf{Berlin-Pankow}, \emph{Ehemaliger Ort}|pw}, Parkstr. 25.\oindex{Parkstraße@\textbf{Parkstraße}, \emph{Straße}|pw}\pend{}{\bigskip}\vspace{1em}
\pstart
           {\pb}Wien\oindex{Wien@\textbf{Wien}, \emph{Verwaltungsgebiet}|pw}, \label{K_L00297-1v}\edtext{10. II. 93}{\lemma{\textnormal{\emph{10. II. 93}}}\Cendnote{\textnormal{Die Datierung ist, wie aus den
                     Poststempeln ersichtlich wird, um ein Jahr falsch.}}}\label{K_L00297-1}.\pend
           \vspace{0.5em}
\pstart
           Café Central\oindex{Wien@\textbf{Wien}!I., Innere Stadt@\textbf{I., Innere Stadt}!Café Central@\textbf{Café Central}, \emph{Kaffeehaus}|pw} – die Secession\introOben{}isten\introOben{} der Secession (\uline{nicht mehr} das altberühmte Café Grienſteidl\oindex{Wien@\textbf{Wien}!I., Innere Stadt@\textbf{I., Innere Stadt}!Café Griensteidl@\textbf{Café Griensteidl}, \emph{Kaffeehaus}|pw} oder »Steinkrügl«, wie Liliencron\pwindex{Liliencron, Detlev von 3.\,6.\,1844 Kiel – 22.\,7.\,1909 Rahlstedt@\textsc{Liliencron, Detlev von} (3.\,6.\,1844 Kiel – 22.\,7.\,1909 Rahlstedt), \emph{Schriftsteller, Dichter, Dramatiker}|pw}{ }ſagt)\pend
           
\pstart
           Liebſter Dehmel, viele{ }ſchöne Grüße, Sie welttiefer Völkerpsycholog. Meinen Brief
               haben Sie wohl{ }ſchon!\pend
           
\pstart
           Gruß an Bierbaum\pwindex{Bierbaum, Otto Julius 28.\,6.\,1865 Zielona Góra – 1.\,2.\,1910 Dresden@\textsc{Bierbaum, Otto Julius} (28.\,6.\,1865 Zielona Góra – 1.\,2.\,1910 Dresden)|pw}, Schlaf\pwindex{Schlaf, Johannes 21.\,6.\,1862 Querfurt – 2.\,2.\,1941 ebd.@\textsc{Schlaf, Johannes} (21.\,6.\,1862 Querfurt – 2.\,2.\,1941 ebd.), \emph{Schriftsteller}|pw}, Scheerbart\pwindex{Scheerbart, Paul 8.\,1.\,1863 Danzig – 15.\,10.\,1915 Berlin@\textsc{Scheerbart, Paul} (8.\,1.\,1863 Danzig – 15.\,10.\,1915 Berlin), \emph{Schriftsteller}|pw}, Halbe\pwindex{Halbe, Max 4.\,10.\,1865 Gmina Suchy Dąb – 30.\,11.\,1944 Neuötting@\textsc{Halbe, Max} (4.\,10.\,1865 Gmina Suchy Dąb – 30.\,11.\,1944 Neuötting), \emph{Schriftsteller}|pw}! Ihr \spacefill\mbox{Karl Kraus.}\pend
           
\pstart
           \spacefill\mbox{{[}hs. Hofmannsthal:{]} Richard Beer-Hofmann\footnote{\noindent{}\emph{Novellen}\pwindex{Beer-Hofmann, Richard 11.\,7.\,1866 Wien – 26.\,9.\,1945 New York City@\textsc{Beer-Hofmann, Richard} (11.\,7.\,1866 Wien – 26.\,9.\,1945 New York City), \emph{Schriftsteller}!Novellen@\strich\emph{Novellen}|pwk}. Berlin\oindex{Berlin@\textbf{Berlin}, \emph{Hauptstadt}|pwk}{ }\emph{Freund {\kaufmannsund}
                              Jäckel}\orgindex{Freund und Jeckel@Freund {\kaufmannsund}  Jeckel|pwk}{ }1893}\footnote{\noindent{}dieser Dichter hat nicht selbst unterschrieben, weil er nicht schreiben kann
                        aber er sitzt auch da. Loris.}}\pend
           
\pstart
           \spacefill\mbox{Loris}\pend
           
\pstart
           {[}hs. Schnitzler:{]} Herzliche Grüße \spacefill\mbox{Arthur Schnitzler}\pend
           \selectlanguage{ngerman}\endnumbering\briefempfaengerindex{Dehmel, Richard@\textsc{Dehmel, Richard}!zzzHofmannsthal, Hugo von@\emph{von Hugo von Hofmannsthal}!1894-02-101@{10. 2. 1894}|)be}\briefempfaengerindex{Dehmel, Richard@\textsc{Dehmel, Richard}!zzzBeer-Hofmann, Richard@\emph{von Richard Beer-Hofmann}!1894-02-101@{10. 2. 1894}|)be}\briefempfaengerindex{Dehmel, Richard@\textsc{Dehmel, Richard}!zzzSchnitzler, Arthur@\emph{von Arthur Schnitzler}!1894-02-101@{10. 2. 1894}|)be}\briefempfaengerindex{Dehmel, Richard@\textsc{Dehmel, Richard}!zzzKraus, Karl@\emph{von Karl Kraus}!1894-02-101@{10. 2. 1894}|)be}\mylabel{L00297h}  \newcommand{\dateiname}{L00297}\newcommand{\titel}{Karl Kraus u. a. an Richard Dehmel, 10. 2. 1894}\newcommand{\editorInnen}{Herausgegeben von Martin Anton Müller}%% latex-leseansicht-abspann.tex
%% Abspann für die Leseansicht.
%% Der Schalter \ifkorrekturansicht ist bereits durch den Vorspann gesetzt.

%% latex-abspann.tex
%% Gemeinsamer Abspann für Korrekturansicht und Leseansicht.
%% Setzt den Schalter \ifkorrekturansicht voraus (gesetzt in den
%% einbindenden Dateien latex-korrekturansicht-abspann.tex bzw.
%% latex-leseansicht-abspann.tex).
%% ---------------------------------------------------------------

\normalsize

% Das esempio-Environment wird nur in der Leseansicht benötigt
\ifkorrekturansicht\else
\newenvironment{esempio}[3]%
{
    \vspace{1.5ex}
    \rlap{\underline{#1}}
    \par
    \setlength{\parindent}{0cm}
    \nopagebreak
    \leftskip=#2cm
    \rightskip=#3cm
}
{
    \par
}
\fi

\doendnotes{C}
\bigskip
\vfill

\clearpage

\footnotesize

\ifkorrekturansicht
  \lohead{\textsc{register}}
\fi

% theindex-Environment neu definieren ohne reledmac
\makeatletter
\renewenvironment{theindex}{%
  \ifkorrekturansicht
    \section*{\indexname}%
  \else
    \subsubsection*{Index der erwähnten Entitäten}%
  \fi
  \setlength{\parindent}{0pt}%
  \setlength{\parskip}{0pt plus 0.3pt}%
  \let\item\@idxitem
}{%
  \ifkorrekturansicht\clearpage\fi
}
\makeatother

\IfFileExists{\jobname-pw.ind}{\input{\jobname-pw.ind}}{}

% Quellenangabe nur in der Leseansicht
\ifkorrekturansicht\else
% Fallback-Definitionen, falls die .tex-Datei \titel etc. nicht gesetzt hat
\providecommand{\titel}{}
\providecommand{\editorInnen}{}
\providecommand{\dateiname}{\jobname}

\vspace{3cm}

\vfill

\footnotesize
\textsc{Quelle}: \titel. Herausgegeben von {\editorInnen}. In: \emph{Arthur Schnitzler: Briefwechsel mit Autorinnen und Autoren}.
 Digitale Edition, https://schnitzler-briefe.acdh.oeaw.ac.at/{\dateiname}.html (Stand \today)
\fi

\end{document}


