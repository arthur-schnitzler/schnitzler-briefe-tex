\input{../tex-inputs/latex-pdf-vorspann}
\begin{center}
            \textcolor{red}{ENTWURF. ENTZIFFERUNG NOCH NICHT KORREKTURGELESEN}
                      \end{center}
            
               \section[Karl Kraus u.a. an Richard Dehmel, 10. 2. 1894]{ Karl Kraus u.a. an Richard Dehmel, 10. 2. 1894}\nopagebreak\mylabel{v}\rehead{ }\begin{ledgroupsized}[t]{13cm}\normalsize\beginnumbering\briefempfaengerindex{Dehmel, Richard@\textsc{Dehmel, Richard}!zzzHofmannsthal, Hugo von@\emph{von Hugo von Hofmannsthal}!1894-02-101@{10. 2. 1894}|(be}\briefempfaengerindex{Dehmel, Richard@\textsc{Dehmel, Richard}!zzzBeer-Hofmann, Richard@\emph{von Richard Beer-Hofmann}!1894-02-101@{10. 2. 1894}|(be}\briefempfaengerindex{Dehmel, Richard@\textsc{Dehmel, Richard}!zzzSchnitzler, Arthur@\emph{von Arthur Schnitzler}!1894-02-101@{10. 2. 1894}|(be}\briefempfaengerindex{Dehmel, Richard@\textsc{Dehmel, Richard}!zzzKraus, Karl@\emph{von Karl Kraus}!1894-02-101@{10. 2. 1894}|(be} \toendnotes[C]{\smallbreak\pagebreak[2]} \Standort{Hamburg, Staats- und Universitätsbibliothek, DA:Br:K:282.}
\physDesc{Kartenbrief
\newline{}Handschrift Karl Kraus: schwarze Tinte, deutsche Kurrent\newline{}Handschrift Arthur Schnitzler: schwarze Tinte, deutsche Kurrent\newline{}Handschrift Richard Beer-Hofmann: schwarze Tinte, deutsche Kurrent\newline{}Handschrift Hugo von Hofmannsthal: schwarze Tinte, deutsche Kurrent\newline{}Versand: 1) Stempel: »\nobreak{}\oindex{I., Innere Stadt@\textbf{I., Innere Stadt}|pwk}Wien 1/1, 11. 2. 94, 8–9V\nobreak{}«.  2) Stempel: »\nobreak{}\oindex{Berlin-Pankow@\textbf{Berlin-Pankow}|pwk}Pankow bei
                                                  Berlin, 12. 2. 94, 10–11V.\nobreak{}«. }\buchAbdrucke{\weitereDrucke{Joachim Kersten, Friedrich Pfäfflin: \emph{Detlev von Liliencron entdeckt, gefeiert und gelesen
                                von Karl Kraus}. Göttingen: \emph{Wallstein} 2016, S. 116–117.} }\toendnotes[C]{\smallbreak}\pstart{}{\pb}Abſender: Karl Kraus, I. Maximilianſtr 13\oindex{Mahlerstrasse@\textbf{Mahlerstraße}|pw}. \pend{}\pstart{}Wien\oindex{Wien@\textbf{Wien}|pw}\pend{}\pstart{}Loris\pend{}\pstart{}Schnitzler\pend{}\pstart{}Beer-Hofmann\pend{}{\bigskip}\pstart{}Herrn\pend{}\pstart{}Richard Dehmel\pend{}\pstart{}Pankow bei Berlin\oindex{Berlin-Pankow@\textbf{Berlin-Pankow}|pw}, Parkstr. 25.\oindex{Parkstrasse@\textbf{Parkstraße}|pw}\pend{}{\bigskip}\pstart
           {\pb}Wien\oindex{Wien@\textbf{Wien}|pw}, \label{K_L00297_1v}\edtext{10. II. 93}{\lemma{\textnormal{\emph{10. II. 93}}}\Cendnote{\textnormal{Die Datierung ist, wie aus den
                            Poststempeln ersichtlich wird, um ein Jahr falsch.}}}\label{K_L00297_1h}.\pend
           \pstart
           Café Central\oindex{Cafe Central@\textbf{Café Central}|pw} – die Secession\introOben{}isten\introOben{} der Secession
                        (\uline{nicht mehr} das altberühmte Café Grienſteidl\oindex{Cafe Griensteidl@\textbf{Café Griensteidl}|pw} oder »Steinkrügl«, wie Liliencron\pwindex{Liliencron, Detlev von 03.06.1844 – 22.07.1909@\textsc{Liliencron, Detlev von} (03.06.1844 – 22.07.1909)|pw}{ }ſagt)\pend
           \pstart
           Liebſter Dehmel, viele ſchöne Grüße, Sie welttiefer Völkerpsycholog. Meinen Brief
                    haben Sie wohl ſchon!\pend
           \pstart
           Gruß an Bierbaum\pwindex{Bierbaum, Otto Julius 28.06.1865 – 01.02.1910@\textsc{Bierbaum, Otto Julius} (28.06.1865 – 01.02.1910)|pw}, Schlaf\pwindex{Schlaf, Johannes 21.06.1862 – 02.02.1941@\textsc{Schlaf, Johannes} (21.06.1862 – 02.02.1941), \emph{Schriftsteller}|pw}, Scheerbart\pwindex{Scheerbart, Paul 08.01.1863 – 15.10.1915@\textsc{Scheerbart, Paul} (08.01.1863 – 15.10.1915), \emph{Schriftsteller}|pw}, Halbe\pwindex{Halbe, Max 04.10.1865 – 30.11.1944@\textsc{Halbe, Max} (04.10.1865 – 30.11.1944), \emph{Schriftsteller}|pw}! Ihr \spacefill\mbox{Karl Kraus.}\pend
           \pstart
           \spacefill\mbox{{[}hs. Hofmannsthal:{]} Richard Beer-Hofmann}\footnote{\noindent{}\emph{Novellen}\pwindex{Beer-Hofmann, Richard 11.07.1866 – 26.09.1945@\textsc{Beer-Hofmann, Richard} (11.07.1866 – 26.09.1945), \emph{Schriftsteller}!Novellen1. 12. 1893@\strich\emph{Novellen} {[}1. 12. 1893{]}|pwk}. Berlin\oindex{Berlin@\textbf{Berlin}|pwk}{ }\emph{Freund {\kaufmannsund}
                                    Jäckel}\orgindex{Freund und Jeckel@Freund {\kaufmannsund}  Jeckel|pwk}{ }1893}\footnote{\noindent{}dieser Dichter hat nicht selbst unterschrieben, weil er nicht schreiben
                            kann aber er sitzt auch da. Loris.}\pend
           \pstart
           \spacefill\mbox{Loris}\pend
           \pstart
           {[}hs. Schnitzler:{]} Herzliche Grüße \spacefill\mbox{Arthur Schnitzler}\pend
           \endnumbering\briefempfaengerindex{Dehmel, Richard@\textsc{Dehmel, Richard}!zzzHofmannsthal, Hugo von@\emph{von Hugo von Hofmannsthal}!1894-02-101@{10. 2. 1894}|)be}\briefempfaengerindex{Dehmel, Richard@\textsc{Dehmel, Richard}!zzzBeer-Hofmann, Richard@\emph{von Richard Beer-Hofmann}!1894-02-101@{10. 2. 1894}|)be}\briefempfaengerindex{Dehmel, Richard@\textsc{Dehmel, Richard}!zzzSchnitzler, Arthur@\emph{von Arthur Schnitzler}!1894-02-101@{10. 2. 1894}|)be}\briefempfaengerindex{Dehmel, Richard@\textsc{Dehmel, Richard}!zzzKraus, Karl@\emph{von Karl Kraus}!1894-02-101@{10. 2. 1894}|)be}\mylabel{h}\end{ledgroupsized}  \newcommand{\dateiname}{L00297}\newcommand{\titel}{Karl Kraus u.a. an Richard Dehmel, 10. 2. 1894}\newcommand{\editorInnen}{ Martin Anton Müller und Gerd-Hermann Susen}\input{../tex-inputs/latex-pdf-abspann}
      