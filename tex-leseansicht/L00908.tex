%% latex-leseansicht-vorspann.tex
%% Vorspann für die Leseansicht.
%% Lädt die gemeinsame Datei latex-vorspann.tex mit nicht gesetztem Schalter.

\newif\ifkorrekturansicht
\korrekturansichtfalse

\input{../tex-inputs/latex-vorspann}

\begin{center}
            \textcolor{red}{ENTWURF. ENTZIFFERUNG NOCH NICHT KORREKTURGELESEN}
                      \end{center}
            
               \section[Arthur Schnitzler an Hugo von Hofmannsthal, 22. 3. 1899]{ Arthur Schnitzler an Hugo von Hofmannsthal, 22. 3. 1899}\nopagebreak\mylabel{v}\rehead{ }\begin{ledgroupsized}[t]{13cm}\normalsize\beginnumbering\briefempfaengerindex{Hofmannsthal, Hugo von@\textsc{Hofmannsthal, Hugo von}!zzzSchnitzler, Arthur@\emph{von Arthur Schnitzler}!1899-03-221@{22. 3. 1899}|(be} \toendnotes[C]{\smallbreak\pagebreak[2]} \Standort{FDH, Hs-30885,80.}
\physDesc{Brief, 1 Blatt, 4 Seiten
\newline{}Handschrift: Bleistift, deutsche Kurrent}\buchAbdrucke{\weitereDrucke{1) Hugo von Hofmannsthal, Arthur Schnitzler: \emph{Briefwechsel}. Hg. Therese Nickl und Heinrich Schnitzler. Frankfurt am Main: \emph{S. Fischer} 1964, S. 119–120.} \weitereDrucke{2) Arthur Schnitzler: \emph{Briefe 1875–1912}. Hg. Therese Nickl und Heinrich Schnitzler. Frankfurt am Main: \emph{S. Fischer} 1981, S. 369.} }\toendnotes[C]{\smallbreak}\pstart
           \raggedleft{}{\pb}\uline{22. 3. 99}\pend
           \pstart
           Mein lieber Hugo! ich danke Ihnen ſehr dſs Sie noch einmal bei
                    mir waren. Was ſoll ich Ihnen heute weiter ſagen. Ein Tag ist ſchrecklicher als
                    der andre; es iſt viel grauenvoller und hoffnungsloſer als irgend ein Wort
                    darüber. Ich habe das Gefühl, fertig zu ſein; Zeichen genug werden mir geſandt!
                    Vom Morgen aus der Ausblick ins leere, {\pb}leere – die
                    Erinnerungen an ihr\pwindex{Reinhard, Marie 13.03.1871 – 18.03.1899@\textsc{Reinhard, Marie} (13.03.1871 – 18.03.1899), \emph{Gesangspädagogin}|pwv} Leben
                    voll Pein, an ihren Tod von einer grenzenloſen Entſetzlichkeit{\dotstwo} die letzten Blicke, die letzten Worte unvergeßlich
                    – die letzte Angſt auf i{\geminationm}er alles zerſtörend, was
                    noch ko{\geminationm}en könnte. Eine ungeheure Gleichgiltigkeit
                    gegen alles, was mir auch Inhalt des Lebens ſchien – ſchauen ins leere, {\pb}greifen ins leere, ja{\geminationm}ern ins leere.\pend
           \pstart
           Vielleicht fahre ich auf einen Tag nach Graz\oindex{Graz@\textbf{Graz}|pw}, wo
                    ihre Schweſter\pwindex{Burger, Caroline 11.07.1869 – 15.03.1959@\textsc{Burger, Caroline} (11.07.1869 – 15.03.1959)|pwv} und jetzt
                    auch ihr Vater\pwindex{Reinhard, Carl 01.03.1868 – 1904-09-29@\textsc{Reinhard, Carl} (01.03.1868 – 1904-09-29), \emph{Kapellmeister}|pwv} u von
                    morgen an ihre Mutter\pwindex{Reinhard, Therese 13.12.1844 – 25.03.1926@\textsc{Reinhard, Therese} (13.12.1844 – 25.03.1926)|pwv} iſt.
                    Alle Menſchen ſind ſehr gut zu mir; – ich möchte danken können. Eine Einſamkeit
                    ohne gleichen – ich muß dran denken, wie ich doch i{\geminationm}er die Menſchen zu ſchildern verſucht habe, die ihr geliebteſtes verlieren –
                        {\pb}es gibt eben etwas, das nicht auszudrücken iſt –
                    ſo gut wie die Ewigkeit, die Unendlichkeit: – die Einſamkeit, das \uline{Ver}einſamtſein; \uline{ver}einſamt \uline{werden}.\pend
           \pstart
           Leben Sie wohl, liebſter Hugo. Ko{\geminationm}en Sie bald
                    zurück!? Bitte ſchreiben Sie mir nur äußere Vorkommniſſe, \uline{nichts}{ }\uuline{da}\uline{rüber}.\pend
           \pstart
           – Sagen Sie es Brahm\pwindex{Brahm, Otto 05.02.1856 – 28.11.1912@\textsc{Brahm, Otto} (05.02.1856 – 28.11.1912), \emph{Theaterleiter, Regisseur}|pw} u Hirſchfeld\pwindex{Hirschfeld, Georg 11.02.1873 – 17.01.1942@\textsc{Hirschfeld, Georg} (11.02.1873 – 17.01.1942), \emph{Schriftsteller}|pw}, damit ſie’s wiſſen, we{\geminationn} ich komme.\pend
           \pstart Von Herzen Ihr \spacefill\mbox{Arthur}\pend{}\endnumbering\briefempfaengerindex{Hofmannsthal, Hugo von@\textsc{Hofmannsthal, Hugo von}!zzzSchnitzler, Arthur@\emph{von Arthur Schnitzler}!1899-03-221@{22. 3. 1899}|)be}\mylabel{h}\end{ledgroupsized}  \newcommand{\dateiname}{L00908}\newcommand{\titel}{Arthur Schnitzler an Hugo von Hofmannsthal, 22. 3. 1899}\newcommand{\editorInnen}{Martin Anton Müller und Gerd-Hermann Susen}%% latex-leseansicht-abspann.tex
%% Abspann für die Leseansicht.
%% Der Schalter \ifkorrekturansicht ist bereits durch den Vorspann gesetzt.

%% latex-abspann.tex
%% Gemeinsamer Abspann für Korrekturansicht und Leseansicht.
%% Setzt den Schalter \ifkorrekturansicht voraus (gesetzt in den
%% einbindenden Dateien latex-korrekturansicht-abspann.tex bzw.
%% latex-leseansicht-abspann.tex).
%% ---------------------------------------------------------------

\normalsize

% Das esempio-Environment wird nur in der Leseansicht benötigt
\ifkorrekturansicht\else
\newenvironment{esempio}[3]%
{
    \vspace{1.5ex}
    \rlap{\underline{#1}}
    \par
    \setlength{\parindent}{0cm}
    \nopagebreak
    \leftskip=#2cm
    \rightskip=#3cm
}
{
    \par
}
\fi

\doendnotes{C}
\bigskip
\vfill

\clearpage

\footnotesize

\ifkorrekturansicht
  \lohead{\textsc{register}}
\fi

% theindex-Environment neu definieren ohne reledmac
\makeatletter
\renewenvironment{theindex}{%
  \ifkorrekturansicht
    \section*{\indexname}%
  \else
    \subsubsection*{Index der erwähnten Entitäten}%
  \fi
  \setlength{\parindent}{0pt}%
  \setlength{\parskip}{0pt plus 0.3pt}%
  \let\item\@idxitem
}{%
  \ifkorrekturansicht\clearpage\fi
}
\makeatother

\IfFileExists{\jobname-pw.ind}{\input{\jobname-pw.ind}}{}

% Quellenangabe nur in der Leseansicht
\ifkorrekturansicht\else
% Fallback-Definitionen, falls die .tex-Datei \titel etc. nicht gesetzt hat
\providecommand{\titel}{}
\providecommand{\editorInnen}{}
\providecommand{\dateiname}{\jobname}

\vspace{3cm}

\vfill

\footnotesize
\textsc{Quelle}: \titel. Herausgegeben von {\editorInnen}. In: \emph{Arthur Schnitzler: Briefwechsel mit Autorinnen und Autoren}.
 Digitale Edition, https://schnitzler-briefe.acdh.oeaw.ac.at/{\dateiname}.html (Stand \today)
\fi

\end{document}


      