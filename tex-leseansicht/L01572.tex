%% latex-korrekturansicht-vorspann.tex
%% Vorspann für die Korrekturansicht.
%% Lädt die gemeinsame Datei latex-vorspann.tex mit gesetztem Schalter.

\newif\ifkorrekturansicht
\korrekturansichttrue

\input{../tex-inputs/latex-vorspann}


\section[Adolf Treibl an Arthur Schnitzler, 15. 1. 1906]{L01572 Adolf Treibl an Arthur Schnitzler, 15. 1. 1906}
\nopagebreak\mylabel{L01572v}
\rehead{ }\normalsize\beginnumbering\briefempfaengerindex{Schnitzler, Arthur@\textsc{Schnitzler, Arthur}!zzzTreibl, Adolf@\emph{von Adolf Treibl}!1906-01-151@{15 1. 1906}|(be}
\toendnotes[C]{\smallbreak\pagebreak[2]}\Standort{DLA, A:Schnitzler, HS.NZ85.1.4815,3.}
\physDesc{Brief, 1 Blatt, 4 Seiten, 1115 Zeichen
\newline{}Handschrift: schwarze Tinte, deutsche Kurrent
\newline{}Schnitzler: mit Bleistift beschriftet: »\textsc{Treibl (Ehrenstein}« }\toendnotes[C]{\smallbreak}
\pstart
           {\pb}\textsc{Euer Hochwohlgeboren}\pend
           
\pstart{}\textsc{Hochverehrter Herr Doctor}\pend\vspace{0.5em}
\pstart
           Namens meines \label{K_L01572-1v}\edtext{Schwagers}{\lemma{\textnormal{\emph{Schwagers}}}\Cendnote{\textnormal{Treibl war mit einer Tante
                  mütterlicherseits von Albert Ehrenstein\pwindex{Ehrenstein, Albert 23.12.1886 – 08.04.1950@\textsc{Ehrenstein, Albert} (23.12.1886 – 08.04.1950), \emph{Schriftsteller/Schriftstellerin}|pwk}
                  verheiratet.}}}\label{K_L01572-1} Herrn \textsc{Alex Ehrenstein}\pwindex{Ehrenstein, Alexander 29.03.1857 – 29.05.1925@\textsc{Ehrenstein, Alexander} (29.03.1857 – 29.05.1925), \emph{Kassier/Kassierin}|pw} und ſeiner Frau\pwindex{Ehrenstein, Charlotte 21.04.1867 – 02.02.1941@\textsc{Ehrenstein, Charlotte} (21.04.1867 – 02.02.1941)|pwv} beehre
               ich mich den verbindlichſten Dank für die warme Teilnahme auszudrücken, die Euer
               Hochwohlgeboren dem lieben \textsc{Albert}\pwindex{Ehrenstein, Albert 23.12.1886 – 08.04.1950@\textsc{Ehrenstein, Albert} (23.12.1886 – 08.04.1950), \emph{Schriftsteller/Schriftstellerin}|pw} zuteil werden laſſen. {\pb}Dem Opfer, das Sie mit
               Ihrem \label{K_L01572-2v}\edtext{geſtrigen Beſuch}{\lemma{\textnormal{\emph{geſtrigen Beſuch}}}\Cendnote{\textnormal{Vgl. A. S.: \emph{Tagebuch}, 14. 1. 1906.
               }}}\label{K_L01572-2} nicht nur dem Patienten ſondern auch ſeinen mitleidenden Eltern\pwindex{Ehrenstein, Alexander 29.03.1857 – 29.05.1925@\textsc{Ehrenstein, Alexander} (29.03.1857 – 29.05.1925), \emph{Kassier/Kassierin}|pwv}\pwindex{Ehrenstein, Charlotte 21.04.1867 – 02.02.1941@\textsc{Ehrenstein, Charlotte} (21.04.1867 – 02.02.1941)|pwv} gebracht haben, wird,
               deſſen können hochverehrter Herr Doktor ſich verſichert halten, ein treueſt und
               dankbarest Gedenken immer bewahrt werden.\pend
           
\pstart
           Der Zuſtand des lieben \textsc{Albert}\pwindex{Ehrenstein, Albert 23.12.1886 – 08.04.1950@\textsc{Ehrenstein, Albert} (23.12.1886 – 08.04.1950), \emph{Schriftsteller/Schriftstellerin}|pw} iſt über Nacht wohl ruhiger geworden, doch lautet {\pb}die Auskunft des zu Rate gezogenen Arztes \textsc{D\textsuperscript{r}{ }Alfred Adler\pwindex{Adler, Alfred 07.02.1870 – 28.05.1937@\textsc{Adler, Alfred} (07.02.1870 – 28.05.1937), \emph{Psychiater/Psychiaterin, Neurologe/Neurologin}|pw}}, den ich als \textsc{Psychologen} und \textsc{Diagnostiker} hochschätze nichts weniger als befriedigend. Er ſchließt auf
                  \textsc{acute Paranoia} und empfiehlt die Abgabe in ein
               Sanatorium.\pend
           
\pstart
           Während ich dies ſchreibe iſt die Schwägerin\pwindex{Ehrenstein, Charlotte 21.04.1867 – 02.02.1941@\textsc{Ehrenstein, Charlotte} (21.04.1867 – 02.02.1941)|pwv} in \textsc{Ob. Döbling}\oindex{Oberdoebling@\textbf{Oberdöbling}, \emph{Bezirk (A.BZK)}|pw} um die Aufnahme in das Sanatorium \textsc{Obersteiner}\oindex{Sanatorium Obersteiner@\textbf{Sanatorium Obersteiner}, \emph{Sanatorium (K.SAN)}|pw} vorzubereiten.\pend
           
\pstart
           Indem ich unſeren herzlichſten Dank wiederhole {\pb}bitte
               ich dem lieben \textsc{Albert}\pwindex{Ehrenstein, Albert 23.12.1886 – 08.04.1950@\textsc{Ehrenstein, Albert} (23.12.1886 – 08.04.1950), \emph{Schriftsteller/Schriftstellerin}|pw} die \textsc{Sympathien} gütigſt zu bewahren, die, wie ich
               begreife, ihn mit gerechtem Stolz erfüllen.\pend
           
\pstart
           In vollkommener Hochachtung{\\[\baselineskip]}ergebenſt{\\[\baselineskip]}\spacefill\mbox{Adolf Treibl}\pend
           \leftskip=0em{}
\pstart
           Wien\oindex{Wien@\textbf{Wien}, \emph{A.ADM2}|pw}{ }15/I 06\pend
           \selectlanguage{ngerman}\endnumbering\briefempfaengerindex{Schnitzler, Arthur@\textsc{Schnitzler, Arthur}!zzzTreibl, Adolf@\emph{von Adolf Treibl}!1906-01-151@{15 1. 1906}|)be}\mylabel{L01572h}  \normalsize

\doendnotes{C}
\bigskip
\vfill

\clearpage

\footnotesize

\lohead{\textsc{register}}

% Definiere theindex-Environment komplett neu ohne reledmac
\makeatletter
\renewenvironment{theindex}{%
  \section*{\indexname}%
  \setlength{\parindent}{0pt}%
  \setlength{\parskip}{0pt plus 0.3pt}%
  \let\item\@idxitem
}{%
  \clearpage
}
\makeatother

\IfFileExists{\jobname-pw.ind}{\input{\jobname-pw.ind}}{}

\end{document}

      