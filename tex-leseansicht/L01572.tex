%% latex-leseansicht-vorspann.tex
%% Vorspann für die Leseansicht.
%% Lädt die gemeinsame Datei latex-vorspann.tex mit nicht gesetztem Schalter.

\newif\ifkorrekturansicht
\korrekturansichtfalse

\input{../tex-inputs/latex-vorspann}


         
         \renewcommand{\erwaehntePersonen}{Personen: Alfred Adler, Albert Ehrenstein, Alexander Ehrenstein, Charlotte Ehrenstein}
         \renewcommand{\erwaehnteOrte}{Orte: Oberdöbling, Sanatorium Obersteiner, Wien}
         \renewcommand{\erwaehnteWerke}{}
               \section[Adolf Treibl an Arthur Schnitzler, 15. 1. 1906]{ Adolf Treibl an Arthur Schnitzler, 15. 1. 1906}\nopagebreak\mylabel{v}\rehead{ }\begin{ledgroupsized}[t]{13cm}\normalsize\beginnumbering \toendnotes[C]{\smallbreak\pagebreak[2]} \Standort{DLA, A:Schnitzler, HS.NZ85.1.4815,3.}
\physDesc{Brief, 1 Blatt, 4 Seiten, 1115 Zeichen
\newline{}Handschrift: schwarze Tinte, deutsche Kurrent
\newline{}Schnitzler: mit Bleistift beschriftet: »\textsc{Treibl (Ehrenstein}« }\toendnotes[C]{\smallbreak}\pstart
           \noindent{}{\pb}\textsc{Euer Hochwohlgeboren}\pend
           \pstart{}\textsc{Hochverehrter Herr Doctor}\pend\pstart
           Namens meines \label{K_L01572_1v}\edtext{Schwagers}{\lemma{\textnormal{\emph{Schwagers}}}\Cendnote{\textnormal{Treibl war mit einer Tante
                  mütterlicherseits von Albert Ehrenstein\pwindex{Ehrenstein, Albert 23.12.1886 – 08.04.1950@\textsc{Ehrenstein, Albert} (23.12.1886 – 08.04.1950), \emph{Schriftsteller}|pwk}
                  verheiratet.}}}\label{K_L01572_1h} Herrn \textsc{Alex Ehrenstein}\pwindex{Ehrenstein, Alexander 29.03.1857 – 29.05.1925@\textsc{Ehrenstein, Alexander} (29.03.1857 – 29.05.1925), \emph{Kassier}|pw} und ſeiner Frau\pwindex{Ehrenstein, Charlotte 21.04.1867 – 02.02.1941@\textsc{Ehrenstein, Charlotte} (21.04.1867 – 02.02.1941)|pwv} beehre
               ich mich den verbindlichſten Dank für die warme Teilnahme auszudrücken, die Euer
               Hochwohlgeboren dem lieben \textsc{Albert}\pwindex{Ehrenstein, Albert 23.12.1886 – 08.04.1950@\textsc{Ehrenstein, Albert} (23.12.1886 – 08.04.1950), \emph{Schriftsteller}|pw} zuteil werden laſſen. {\pb}Dem Opfer, das Sie mit
               Ihrem \label{K_L01572_2v}\edtext{geſtrigen Beſuch}{\lemma{\textnormal{\emph{geſtrigen Beſuch}}}\Cendnote{\textnormal{vgl. A. S.: \emph{Tagebuch}, 14. 1. 1906}}}\label{K_L01572_2h} nicht nur dem Patienten ſondern auch ſeinen mitleidenden Eltern\pwindex{Ehrenstein, Alexander 29.03.1857 – 29.05.1925@\textsc{Ehrenstein, Alexander} (29.03.1857 – 29.05.1925), \emph{Kassier}|pwv}\pwindex{Ehrenstein, Charlotte 21.04.1867 – 02.02.1941@\textsc{Ehrenstein, Charlotte} (21.04.1867 – 02.02.1941)|pwv} gebracht haben, wird,
               deſſen können hochverehrter Herr Doktor ſich verſichert halten, ein treueſt und
               dankbarest Gedenken immer bewahrt werden.\pend
           \pstart
           Der Zuſtand des lieben \textsc{Albert}\pwindex{Ehrenstein, Albert 23.12.1886 – 08.04.1950@\textsc{Ehrenstein, Albert} (23.12.1886 – 08.04.1950), \emph{Schriftsteller}|pw} iſt über Nacht wohl ruhiger geworden, doch lautet {\pb}die Auskunft des zu Rate gezogenen Arztes \textsc{D\textsuperscript{r}{ }Alfred Adler\pwindex{Adler, Alfred 07.02.1870 – 28.05.1937@\textsc{Adler, Alfred} (07.02.1870 – 28.05.1937), \emph{Psychiater, Mediziner}|pw}}, den ich als \textsc{Psychologen} und \textsc{Diagnostiker} hochschätze nichts weniger als befriedigend. Er ſchließt auf
                  \textsc{acute Paranoia} und empfiehlt die Abgabe in ein
               Sanatorium.\pend
           \pstart
           Während ich dies ſchreibe iſt die Schwägerin\pwindex{Ehrenstein, Charlotte 21.04.1867 – 02.02.1941@\textsc{Ehrenstein, Charlotte} (21.04.1867 – 02.02.1941)|pwv} in \textsc{Ob. Döbling}\oindex{Oberdoebling@\textbf{Oberdöbling}|pw} um die Aufnahme in das Sanatorium \textsc{Obersteiner}\oindex{Sanatorium Obersteiner@\textbf{Sanatorium Obersteiner}|pw} vorzubereiten.\pend
           \pstart
           Indem ich unſeren herzlichſten Dank wiederhole {\pb}bitte
               ich dem lieben \textsc{Albert}\pwindex{Ehrenstein, Albert 23.12.1886 – 08.04.1950@\textsc{Ehrenstein, Albert} (23.12.1886 – 08.04.1950), \emph{Schriftsteller}|pw} die \textsc{Sympathien} gütigſt zu bewahren, die, wie ich
               begreife, ihn mit gerechtem Stolz erfüllen.\pend
           \pstart
           In vollkommener Hochachtung{\\[\baselineskip]}ergebenſt{\\[\baselineskip]}\spacefill\mbox{Adolf Treibl}\pend
           \leftskip=0em{}\pstart
           Wien\oindex{Wien@\textbf{Wien}|pw}{ }15/I 06\pend
           
         
         \endnumbering\mylabel{h}\end{ledgroupsized}  \newcommand{\dateiname}{L01572}\newcommand{\titel}{Adolf Treibl an Arthur Schnitzler, 15. 1. 1906}\newcommand{\editorInnen}{Martin Anton Müller und Gerd-Hermann Susen}%% latex-leseansicht-abspann.tex
%% Abspann für die Leseansicht.
%% Der Schalter \ifkorrekturansicht ist bereits durch den Vorspann gesetzt.

%% latex-abspann.tex
%% Gemeinsamer Abspann für Korrekturansicht und Leseansicht.
%% Setzt den Schalter \ifkorrekturansicht voraus (gesetzt in den
%% einbindenden Dateien latex-korrekturansicht-abspann.tex bzw.
%% latex-leseansicht-abspann.tex).
%% ---------------------------------------------------------------

\normalsize

% Das esempio-Environment wird nur in der Leseansicht benötigt
\ifkorrekturansicht\else
\newenvironment{esempio}[3]%
{
    \vspace{1.5ex}
    \rlap{\underline{#1}}
    \par
    \setlength{\parindent}{0cm}
    \nopagebreak
    \leftskip=#2cm
    \rightskip=#3cm
}
{
    \par
}
\fi

\doendnotes{C}
\bigskip
\vfill

\clearpage

\footnotesize

\ifkorrekturansicht
  \lohead{\textsc{register}}
\fi

% theindex-Environment neu definieren ohne reledmac
\makeatletter
\renewenvironment{theindex}{%
  \ifkorrekturansicht
    \section*{\indexname}%
  \else
    \subsubsection*{Index der erwähnten Entitäten}%
  \fi
  \setlength{\parindent}{0pt}%
  \setlength{\parskip}{0pt plus 0.3pt}%
  \let\item\@idxitem
}{%
  \ifkorrekturansicht\clearpage\fi
}
\makeatother

\IfFileExists{\jobname-pw.ind}{\input{\jobname-pw.ind}}{}

% Quellenangabe nur in der Leseansicht
\ifkorrekturansicht\else
% Fallback-Definitionen, falls die .tex-Datei \titel etc. nicht gesetzt hat
\providecommand{\titel}{}
\providecommand{\editorInnen}{}
\providecommand{\dateiname}{\jobname}

\vspace{3cm}

\vfill

\footnotesize
\textsc{Quelle}: \titel. Herausgegeben von {\editorInnen}. In: \emph{Arthur Schnitzler: Briefwechsel mit Autorinnen und Autoren}.
 Digitale Edition, https://schnitzler-briefe.acdh.oeaw.ac.at/{\dateiname}.html (Stand \today)
\fi

\end{document}


      