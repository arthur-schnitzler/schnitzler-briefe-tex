%% latex-leseansicht-vorspann.tex
%% Vorspann für die Leseansicht.
%% Lädt die gemeinsame Datei latex-vorspann.tex mit nicht gesetztem Schalter.

\newif\ifkorrekturansicht
\korrekturansichtfalse

\input{../tex-inputs/latex-vorspann}


\section[Adolf Treibl an Arthur Schnitzler, 15. 1. 1906]{L01572 Adolf Treibl an Arthur Schnitzler, 15. 1. 1906}
\nopagebreak\mylabel{L01572v}
\rehead{ }\normalsize\beginnumbering\briefempfaengerindex{Schnitzler, Arthur@\textsc{Schnitzler, Arthur}!zzzTreibl, Adolf@\emph{von Adolf Treibl}!1906-01-151@{15 1. 1906}|(be}
\toendnotes[C]{\smallbreak\pagebreak[2]}
\correspDesc{Versand  durch Adolf Treibl am 15 1. 1906 in Wien
\newline{}Erhalt  durch Arthur Schnitzler im Zeitraum [15. 1. 1906
                  – 19. 1. 1906?] in Wien}\toendnotes[C]{\smallbreak}
\Standort{DLA, A:Schnitzler, HS.NZ85.1.4815,3.}
\physDesc{Brief, 1 Blatt, 4 Seiten, 1115 Zeichen
\newline{}Handschrift: schwarze Tinte, deutsche Kurrent
\newline{}Schnitzler: mit Bleistift beschriftet: »\textsc{Treibl (Ehrenstein}« }\toendnotes[C]{\smallbreak}
\pstart
           {\pb}\textsc{Euer Hochwohlgeboren}\pend
           
\pstart{}\textsc{Hochverehrter Herr Doctor}\pend\vspace{0.5em}
\pstart
           Namens meines \label{K_L01572-1v}\edtext{Schwagers}{\lemma{\textnormal{\emph{Schwagers}}}\Cendnote{\textnormal{Treibl war mit einer Tante
                  mütterlicherseits von Albert Ehrenstein\pwindex{Ehrenstein, Albert 23.\,12.\,1886 Wien – 8.\,4.\,1950 New York City@\textsc{Ehrenstein, Albert} (23.\,12.\,1886 Wien – 8.\,4.\,1950 New York City), \emph{Schriftsteller}|pwk}
                  verheiratet.}}}\label{K_L01572-1} Herrn \textsc{Alex Ehrenstein}\pwindex{Ehrenstein, Alexander 29.\,3.\,1857 Skalice – 29.\,5.\,1925 Wien@\textsc{Ehrenstein, Alexander} (29.\,3.\,1857 Skalice – 29.\,5.\,1925 Wien), \emph{Kassier}|pw} und{ }ſeiner Frau\pwindex{Ehrenstein, Charlotte 21.\,4.\,1867 Vrádište – 2.\,2.\,1941 New York City@\textsc{Ehrenstein, Charlotte} (21.\,4.\,1867 Vrádište – 2.\,2.\,1941 New York City)|pwv} beehre
               ich mich den verbindlichſten Dank für die warme Teilnahme auszudrücken, die Euer
               Hochwohlgeboren dem lieben \textsc{Albert}\pwindex{Ehrenstein, Albert 23.\,12.\,1886 Wien – 8.\,4.\,1950 New York City@\textsc{Ehrenstein, Albert} (23.\,12.\,1886 Wien – 8.\,4.\,1950 New York City), \emph{Schriftsteller}|pw} zuteil werden laſſen. {\pb}Dem Opfer, das Sie mit
               Ihrem \label{K_L01572-2v}\edtext{geſtrigen Beſuch}{\lemma{\textnormal{\emph{gestrigen Besuch}}}\Cendnote{\textnormal{Vgl. A. S.: \emph{Tagebuch}, 14. 1. 1906.
               }}}\label{K_L01572-2} nicht nur dem Patienten{ }ſondern auch{ }ſeinen mitleidenden Eltern\pwindex{Ehrenstein, Alexander 29.\,3.\,1857 Skalice – 29.\,5.\,1925 Wien@\textsc{Ehrenstein, Alexander} (29.\,3.\,1857 Skalice – 29.\,5.\,1925 Wien), \emph{Kassier}|pwv}\pwindex{Ehrenstein, Charlotte 21.\,4.\,1867 Vrádište – 2.\,2.\,1941 New York City@\textsc{Ehrenstein, Charlotte} (21.\,4.\,1867 Vrádište – 2.\,2.\,1941 New York City)|pwv} gebracht haben, wird,
               deſſen können hochverehrter Herr Doktor{ }ſich verſichert halten, ein treueſt und
               dankbarest Gedenken immer bewahrt werden.\pend
           
\pstart
           Der Zuſtand des lieben \textsc{Albert}\pwindex{Ehrenstein, Albert 23.\,12.\,1886 Wien – 8.\,4.\,1950 New York City@\textsc{Ehrenstein, Albert} (23.\,12.\,1886 Wien – 8.\,4.\,1950 New York City), \emph{Schriftsteller}|pw} iſt über Nacht wohl ruhiger geworden, doch lautet {\pb}die Auskunft des zu Rate gezogenen Arztes \textsc{D\textsuperscript{r}{ }Alfred Adler\pwindex{Adler, Alfred 7.\,2.\,1870 Wien – 28.\,5.\,1937 Aberdeen@\textsc{Adler, Alfred} (7.\,2.\,1870 Wien – 28.\,5.\,1937 Aberdeen), \emph{Psychiater, Neurologe}|pw}}, den ich als \textsc{Psychologen} und \textsc{Diagnostiker} hochschätze nichts weniger als befriedigend. Er{ }ſchließt auf
                  \textsc{acute Paranoia} und empfiehlt die Abgabe in ein
               Sanatorium.\pend
           
\pstart
           Während ich dies{ }ſchreibe iſt die Schwägerin\pwindex{Ehrenstein, Charlotte 21.\,4.\,1867 Vrádište – 2.\,2.\,1941 New York City@\textsc{Ehrenstein, Charlotte} (21.\,4.\,1867 Vrádište – 2.\,2.\,1941 New York City)|pwv} in \textsc{Ob. Döbling}\oindex{Wien@\textbf{Wien}!XIX., Döbling@\textbf{XIX., Döbling}!Oberdöbling@\textbf{Oberdöbling}, \emph{Bezirk}|pw} um die Aufnahme in das Sanatorium \textsc{Obersteiner}\oindex{Wien@\textbf{Wien}!XIX., Döbling@\textbf{XIX., Döbling}!Sanatorium Obersteiner@\textbf{Sanatorium Obersteiner}, \emph{Sanatorium}|pw} vorzubereiten.\pend
           
\pstart
           Indem ich unſeren herzlichſten Dank wiederhole {\pb}bitte
               ich dem lieben \textsc{Albert}\pwindex{Ehrenstein, Albert 23.\,12.\,1886 Wien – 8.\,4.\,1950 New York City@\textsc{Ehrenstein, Albert} (23.\,12.\,1886 Wien – 8.\,4.\,1950 New York City), \emph{Schriftsteller}|pw} die \textsc{Sympathien} gütigſt zu bewahren, die, wie ich
               begreife, ihn mit gerechtem Stolz erfüllen.\pend
           
\pstart
           In vollkommener Hochachtung{\\[\baselineskip]}ergebenſt{\\[\baselineskip]}\spacefill\mbox{Adolf Treibl}\pend
           \leftskip=0em{}
\pstart
           Wien\oindex{Wien@\textbf{Wien}, \emph{Verwaltungsgebiet}|pw}{ }15/I 06\pend
           \selectlanguage{ngerman}\endnumbering\briefempfaengerindex{Schnitzler, Arthur@\textsc{Schnitzler, Arthur}!zzzTreibl, Adolf@\emph{von Adolf Treibl}!1906-01-151@{15 1. 1906}|)be}\mylabel{L01572h}  \newcommand{\dateiname}{L01572}\newcommand{\titel}{Adolf Treibl an Arthur Schnitzler, 15. 1. 1906}\newcommand{\editorInnen}{Martin Anton Müller und Gerd-Hermann Susen}%% latex-leseansicht-abspann.tex
%% Abspann für die Leseansicht.
%% Der Schalter \ifkorrekturansicht ist bereits durch den Vorspann gesetzt.

%% latex-abspann.tex
%% Gemeinsamer Abspann für Korrekturansicht und Leseansicht.
%% Setzt den Schalter \ifkorrekturansicht voraus (gesetzt in den
%% einbindenden Dateien latex-korrekturansicht-abspann.tex bzw.
%% latex-leseansicht-abspann.tex).
%% ---------------------------------------------------------------

\normalsize

% Das esempio-Environment wird nur in der Leseansicht benötigt
\ifkorrekturansicht\else
\newenvironment{esempio}[3]%
{
    \vspace{1.5ex}
    \rlap{\underline{#1}}
    \par
    \setlength{\parindent}{0cm}
    \nopagebreak
    \leftskip=#2cm
    \rightskip=#3cm
}
{
    \par
}
\fi

\doendnotes{C}
\bigskip
\vfill

\clearpage

\footnotesize

\ifkorrekturansicht
  \lohead{\textsc{register}}
\fi

% theindex-Environment neu definieren ohne reledmac
\makeatletter
\renewenvironment{theindex}{%
  \ifkorrekturansicht
    \section*{\indexname}%
  \else
    \subsubsection*{Index der erwähnten Entitäten}%
  \fi
  \setlength{\parindent}{0pt}%
  \setlength{\parskip}{0pt plus 0.3pt}%
  \let\item\@idxitem
}{%
  \ifkorrekturansicht\clearpage\fi
}
\makeatother

\IfFileExists{\jobname-pw.ind}{\input{\jobname-pw.ind}}{}

% Quellenangabe nur in der Leseansicht
\ifkorrekturansicht\else
% Fallback-Definitionen, falls die .tex-Datei \titel etc. nicht gesetzt hat
\providecommand{\titel}{}
\providecommand{\editorInnen}{}
\providecommand{\dateiname}{\jobname}

\vspace{3cm}

\vfill

\footnotesize
\textsc{Quelle}: \titel. Herausgegeben von {\editorInnen}. In: \emph{Arthur Schnitzler: Briefwechsel mit Autorinnen und Autoren}.
 Digitale Edition, https://schnitzler-briefe.acdh.oeaw.ac.at/{\dateiname}.html (Stand \today)
\fi

\end{document}


