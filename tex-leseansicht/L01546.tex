%% latex-leseansicht-vorspann.tex
%% Vorspann für die Leseansicht.
%% Lädt die gemeinsame Datei latex-vorspann.tex mit nicht gesetztem Schalter.

\newif\ifkorrekturansicht
\korrekturansichtfalse

\input{../tex-inputs/latex-vorspann}


\section[Arthur Schnitzler an Richard Beer-Hofmann, 14. 9. 1905]{L01546 Arthur Schnitzler an Richard Beer-Hofmann, 14. 9. 1905}
\nopagebreak\mylabel{L01546v}
\rehead{ }\normalsize\beginnumbering\briefempfaengerindex{Beer-Hofmann, Richard@\textsc{Beer-Hofmann, Richard}!zzzSchnitzler, Arthur@\emph{von Arthur Schnitzler}!1905-09-141@{14. 9. 1905}|(be}
\toendnotes[C]{\smallbreak\pagebreak[2]}
\correspDesc{Versand  durch Arthur Schnitzler am 14. 9. 1905 in Wien
\newline{}Weiterleitung  in Rodaun
\newline{}Erhalt  durch Richard Beer-Hofmann im Zeitraum [15. 9. 1905
                  – 19. 9. 1905?] in Venedig}\toendnotes[C]{\smallbreak}
\Standort{YCGL, MSS 31.}
\physDesc{Brief, 1 Blatt, 4 Seiten, Kuvert, 1488 Zeichen
\newline{}Handschrift: schwarze Tinte, deutsche Kurrent
\newline{}Versand: 1) Stempel: »\nobreak{}\oindex{XVIII., Währing@\textbf{XVIII., Währing}, \emph{Verwaltungsgebiet}|pwk}18/1 Wien, 14. IX. 0\textcolor{gray}{5}, 6\nobreak{}«.   2) Stempel: »\nobreak{}\oindex{Wien@\textbf{Wien}!XXIII., Liesing@\textbf{XXIII., Liesing}!Rodaun@\textbf{Rodaun}, \emph{Region}|pwk}Rodaun\nobreak{}«.  3) Stempel: »\nobreak{}\oindex{Venedig@\textbf{Venedig}|pwk}\textcolor{gray}{Venezia}\nobreak{}«.  4) mit blauer Tinte von unbekannter Hand in Lateinschrift zur
                                 Adresse hinzugefügt: »derzeit \uline{Venedig\oindex{Venedig@\textbf{Venedig}|pw}}{ }Lido Grand Hotel\oindex{Grand Hotel des Bains@\textbf{Grand Hotel des Bains}, \emph{Hotel}|pw}{ }\uline{Italien\oindex{Italien@\textbf{Italien}|pw}}«}
\buchAbdrucke{\weitereDrucke{Arthur Schnitzler, Richard Beer-Hofmann: \emph{Briefwechsel 1891–1931}. Herausgegeben von Konstanze Fliedl. Wien, Zürich: \emph{Europaverlag} 1992, S. 175.} }\toendnotes[C]{\smallbreak}\pstart{}{\pb}\textcolor{gray}{\textbf{Dr. Arthur Schnitzler}}\pend{}\pstart{}\textcolor{gray}{\textbf{Wien XVIII. Spoettelgasse 7\oindex{Wien@\textbf{Wien}!XVIII., Währing@\textbf{XVIII., Währing}!Edmund-Weiß-Gasse 7@\textbf{Edmund-Weiß-Gasse 7}, \emph{Wohngebäude}|pw}.}}\pend{}{\bigskip}\pstart{}{\pb}\textsc{Herrn Dr. Richard Beer-Hofmann}\pend{}\pstart{}\textsc{Rodaun}\oindex{Wien@\textbf{Wien}!XXIII., Liesing@\textbf{XXIII., Liesing}!Rodaun@\textbf{Rodaun}, \emph{Region}|pw}\pend{}\pstart{}\textsc{bei Liesing\oindex{XXIII., Liesing@\textbf{XXIII., Liesing}, \emph{Verwaltungsgebiet}|pw}}\pend{}\pstart{}\textsc{Liesingerstr.} 2\oindex{Liesingerstraße@\textbf{Liesingerstraße}, \emph{Straße}|pw}.\pend{}\pstart{}ev. nachzuſenden.\pend{}{\bigskip}\vspace{1em}
\pstart
           \raggedleft{}{\pb}Wien\oindex{Wien@\textbf{Wien}, \emph{Verwaltungsgebiet}|pw}{ }14. 9. 905\pend
           \vspace{0.5em}
\pstart
           lieber Richard, ich habe erwartet, eine Nachricht von Ihnen zu
                  beko{\geminationm}en, we{\geminationn} Sie
               irgendwo gelandet{ }ſind, und da ich nicht weiſs, ob Sie{ }ſchon, noch, überhaupt am Lido\oindex{Lido@\textbf{Lido}|pw}{ }ſind und in welchem Hotel, richte ich dieſe
               Zeilen an Ihre Rodaun\oindex{Wien@\textbf{Wien}!XXIII., Liesing@\textbf{XXIII., Liesing}!Rodaun@\textbf{Rodaun}, \emph{Region}|pw}er Adreſſe. Der Brief an \textsc{Mir. Horwitz}\pwindex{Horwitz, Mirjam 15.\,6.\,1882 Berlin – 26.\,9.\,1967 Lütjensee@\textsc{Horwitz, Mirjam} (15.\,6.\,1882 Berlin – 26.\,9.\,1967 Lütjensee), \emph{Theaterleiterin, Schauspielerin}|pw} iſt längſt beſorgt, übrigens ko{\geminationm}t Adreſſatin
               morgen hier an (mit dem \textsc{Roland}\orgindex{Roland von Berlin@Roland von Berlin|pw} von Berlin\oindex{Berlin@\textbf{Berlin}, \emph{Hauptstadt}|pw}, was kein Liebhaber,{ }ſondern ein
                  \textsc{Caba{\pb}ret} iſt). Ob und
               wann ich in dieſem Herbſt noch wegkomme, iſt ungewiſs, da ich wahrſcheinlich{ }ſehr
               bald Burgtheater\oindex{Wien@\textbf{Wien}!I., Innere Stadt@\textbf{I., Innere Stadt}!Burgtheater@\textbf{Burgtheater}, \emph{Theater}|pw}proben haben dürfte. (Sie haben
               wohl geleſen; näheres mündlich, die Sache iſt mir höchſt angenehm; Schl.\pwindex{Schlenther, Paul 20.\,8.\,1854 Chernyakhovsk – 30.\,4.\,1916 Berlin@\textsc{Schlenther, Paul} (20.\,8.\,1854 Chernyakhovsk – 30.\,4.\,1916 Berlin), \emph{Schriftsteller, Kritiker, Theaterleiter}|pw} hatte{ }ſich über Brahm\pwindex{Brahm, Otto 5.\,2.\,1856 Hamburg – 28.\,11.\,1912 Berlin@\textsc{Brahm, Otto} (5.\,2.\,1856 Hamburg – 28.\,11.\,1912 Berlin), \emph{Theaterleiter, Regisseur}|pw} an mich gewandt.) Auch mit dem zweiten Stück\pwindex{Schnitzler, Arthur 15.\,5.\,1862 Wien – 21.\,10.\,1931 ebd.@\textsc{Schnitzler, Arthur} (15.\,5.\,1862 Wien – 21.\,10.\,1931 ebd.), \emph{Schriftsteller, Mediziner}!Ruf des Lebens. Schauspiel in drei Akten@\strich\emph{Der Ruf des Lebens. Schauspiel in drei Akten}|pwv}, das zur Zeit der Vorleſung im 3. Akt
               noch höchſt unſicher war, bin ich jetzt glaub ich leidlich fertig – oder ka{\geminationn} nur ni{\geminationm}er weiter, was
               aufs gleiche {\pb}herausko{\geminationm}t.
               – Wahrſcheinlich kriegt auch das zweite der Brahm\pwindex{Brahm, Otto 5.\,2.\,1856 Hamburg – 28.\,11.\,1912 Berlin@\textsc{Brahm, Otto} (5.\,2.\,1856 Hamburg – 28.\,11.\,1912 Berlin), \emph{Theaterleiter, Regisseur}|pw}; mit \textsc{Reinhardt}\pwindex{Reinhardt, Max 9.\,9.\,1873 Baden bei Wien – 30.\,10.\,1943 New York City@\textsc{Reinhardt, Max} (9.\,9.\,1873 Baden bei Wien – 30.\,10.\,1943 New York City), \emph{Theaterleiter, Regisseur, Schauspieler}|pw} und den Seinen iſt einfach nicht zu verhandeln. Sie depeſchiren einem von
               Briefen, die auf dem Wege{ }ſind – und die nie geſchrieben wurden – und das iſt noch
               nicht das ärgſte. Auch darüber mündlich. –\pend
           
\pstart
           Sagen Sie mir doch ein Wort, wo Sie{ }ſind, wie lang Sie bleiben, wann Sie kommen, wie
               es Paula\pwindex{Beer-Hofmann, Paula 25.\,2.\,1879 Wien – 30.\,10.\,1939 Zürich@\textsc{Beer-Hofmann, Paula} (25.\,2.\,1879 Wien – 30.\,10.\,1939 Zürich)|pw} geht und den Kindern\pwindex{Beer-Hofmann, Naëmah 20.\,12.\,1898 Wien – 10.\,11.\,1971 New York City@\textsc{Beer-Hofmann, Naëmah} (20.\,12.\,1898 Wien – 10.\,11.\,1971 New York City)|pwv}\pwindex{Beer-Hofmann, Mirjam 4.\,9.\,1897 Wien – 24.\,12.\,1984 New York City@\textsc{Beer-Hofmann, Mirjam} (4.\,9.\,1897 Wien – 24.\,12.\,1984 New York City)|pwv}\pwindex{Beer-Hofmann, Gabriel 9.\,1.\,1901 Wien – 24.\,3.\,1971 St Albans@\textsc{Beer-Hofmann, Gabriel} (9.\,1.\,1901 Wien – 24.\,3.\,1971 St Albans), \emph{Schriftsteller, Filmagent}|pwv} –\pend
           
\pstart
           {\pb}Wir{ }ſpielen täglich Tennis, und bald hoff ich wieder
               in ein geordnetes Arbeiten zu gerathen. Olga\pwindex{Schnitzler, Olga 17.\,1.\,1882 Wien – 13.\,1.\,1970 Lugano@\textsc{Schnitzler, Olga} (17.\,1.\,1882 Wien – 13.\,1.\,1970 Lugano), \emph{Schauspielerin, Sängerin}|pw},
               die Sie alle herzlich grüßt, iſt{ }ſehr wohl, Heinrich\pwindex{Schnitzler, Heinrich 9.\,8.\,1902 Hinterbrühl – 12.\,7.\,1982 Wien@\textsc{Schnitzler, Heinrich} (9.\,8.\,1902 Hinterbrühl – 12.\,7.\,1982 Wien), \emph{Regisseur, Schauspieler}|pw} desgleichen –{ }ſchreiben Sie bitte!\pend
           
\pstart
           Von Herzen Ihr{\\[\baselineskip]}\spacefill\mbox{A.}\pend
           \leftskip=0em{}\selectlanguage{ngerman}\endnumbering\briefempfaengerindex{Beer-Hofmann, Richard@\textsc{Beer-Hofmann, Richard}!zzzSchnitzler, Arthur@\emph{von Arthur Schnitzler}!1905-09-141@{14. 9. 1905}|)be}\mylabel{L01546h}  \newcommand{\dateiname}{L01546}\newcommand{\titel}{Arthur Schnitzler an Richard Beer-Hofmann, 14. 9. 1905}\newcommand{\editorInnen}{Martin Anton Müller und Gerd-Hermann Susen}%% latex-leseansicht-abspann.tex
%% Abspann für die Leseansicht.
%% Der Schalter \ifkorrekturansicht ist bereits durch den Vorspann gesetzt.

%% latex-abspann.tex
%% Gemeinsamer Abspann für Korrekturansicht und Leseansicht.
%% Setzt den Schalter \ifkorrekturansicht voraus (gesetzt in den
%% einbindenden Dateien latex-korrekturansicht-abspann.tex bzw.
%% latex-leseansicht-abspann.tex).
%% ---------------------------------------------------------------

\normalsize

% Das esempio-Environment wird nur in der Leseansicht benötigt
\ifkorrekturansicht\else
\newenvironment{esempio}[3]%
{
    \vspace{1.5ex}
    \rlap{\underline{#1}}
    \par
    \setlength{\parindent}{0cm}
    \nopagebreak
    \leftskip=#2cm
    \rightskip=#3cm
}
{
    \par
}
\fi

\doendnotes{C}
\bigskip
\vfill

\clearpage

\footnotesize

\ifkorrekturansicht
  \lohead{\textsc{register}}
\fi

% theindex-Environment neu definieren ohne reledmac
\makeatletter
\renewenvironment{theindex}{%
  \ifkorrekturansicht
    \section*{\indexname}%
  \else
    \subsubsection*{Index der erwähnten Entitäten}%
  \fi
  \setlength{\parindent}{0pt}%
  \setlength{\parskip}{0pt plus 0.3pt}%
  \let\item\@idxitem
}{%
  \ifkorrekturansicht\clearpage\fi
}
\makeatother

\IfFileExists{\jobname-pw.ind}{\input{\jobname-pw.ind}}{}

% Quellenangabe nur in der Leseansicht
\ifkorrekturansicht\else
% Fallback-Definitionen, falls die .tex-Datei \titel etc. nicht gesetzt hat
\providecommand{\titel}{}
\providecommand{\editorInnen}{}
\providecommand{\dateiname}{\jobname}

\vspace{3cm}

\vfill

\footnotesize
\textsc{Quelle}: \titel. Herausgegeben von {\editorInnen}. In: \emph{Arthur Schnitzler: Briefwechsel mit Autorinnen und Autoren}.
 Digitale Edition, https://schnitzler-briefe.acdh.oeaw.ac.at/{\dateiname}.html (Stand \today)
\fi

\end{document}


