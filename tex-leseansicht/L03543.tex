%% latex-leseansicht-vorspann.tex
%% Vorspann für die Leseansicht.
%% Lädt die gemeinsame Datei latex-vorspann.tex mit nicht gesetztem Schalter.

\newif\ifkorrekturansicht
\korrekturansichtfalse

\input{../tex-inputs/latex-vorspann}


\section[Ludwig Ganghofer an Arthur Schnitzler, 30. 4. [1899]]{L03543 Ludwig Ganghofer an Arthur Schnitzler, 30. 4. [1899]}
\nopagebreak\mylabel{L03543v}
\rehead{ }\normalsize\beginnumbering\briefempfaengerindex{Schnitzler, Arthur@\textsc{Schnitzler, Arthur}!zzzGanghofer, Ludwig@\emph{von Ludwig Ganghofer}!1899-04-304@{30. 4. [1899]}|(be}
\toendnotes[C]{\smallbreak\pagebreak[2]}
\correspDesc{Versand  durch Ludwig Ganghofer am 30. 4. [1899] in München
\newline{}Erhalt  durch Arthur Schnitzler am 30. 4. [1899] in Berlin}\toendnotes[C]{\smallbreak}
\Standort{CUL, Schnitzler, B 775.}
\physDesc{Telegramm, 373 Zeichen (Vordruck: »\textcolor{gray}{\textbf{\textbf{\textsc{Berlin\oindex{Berlin@\textbf{Berlin}, \emph{Hauptstadt}|pw},}}{ }Haupt-Telegraphenamt\oindex{Haupttelegrafenamt@\textbf{Haupttelegrafenamt}, \emph{Bürogebäude}|pw}}}.«)
\newline{}maschinell
\newline{}Versand: 1) mit Bleistift rückseitiger Vermerk: »\noindent{}{\pb}Adrſ. wohnt Savoy-Hôtel\oindex{Hotel Savoy [Berlin]@\textbf{Hotel Savoy [Berlin]}, \emph{Hotel}|pw}{ }Friedrichſtr\oindex{Friedrichstraße [Berlin]@\textbf{Friedrichstraße [Berlin]}, \emph{Straße}|pw}{ / }Bote Frimmel\pwindex{Frimmel @\textsc{Frimmel}, \emph{Briefträger}|pw}«  2) mit rotem Buntstift vier Unterstreichungen und »\textcolor{gray}{K}« für Kakadu\pwindex{Schnitzler, Arthur 15.\,5.\,1862 Wien – 21.\,10.\,1931 ebd.@\textsc{Schnitzler, Arthur} (15.\,5.\,1862 Wien – 21.\,10.\,1931 ebd.), \emph{Schriftsteller, Mediziner}!grüne Kakadu. Groteske in einem Akt@\strich\emph{Der grüne Kakadu. Groteske in einem Akt}|pw}?}\toendnotes[C]{\smallbreak}
\pstart
           \centering{}{\pb}fr muenchen\oindex{München@\textbf{München}|pw} tel 55 30/4{ }9 m =\pend
           \vspace{0.5em}
\pstart
           kann jhnen zu meiner freude mitteilen dass gruener
                  kakadu\pwindex{Schnitzler, Arthur 15.\,5.\,1862 Wien – 21.\,10.\,1931 ebd.@\textsc{Schnitzler, Arthur} (15.\,5.\,1862 Wien – 21.\,10.\,1931 ebd.), \emph{Schriftsteller, Mediziner}!grüne Kakadu. Groteske in einem Akt@\strich\emph{Der grüne Kakadu. Groteske in einem Akt}|pw}{ }gestern{ }abend bei wirklich musterhafter \label{K_L03543-1v}\edtext{auffuehrung}{\lemma{\textnormal{\emph{auffuehrung}}}\Cendnote{\textnormal{Am
                     29. 4. 1899 hatten am Residenztheater\oindex{Residenztheater München@\textbf{Residenztheater München}, \emph{Theater}|pwk} in München\oindex{München@\textbf{München}|pwk} die Premieren von \emph{Traum eines
                     Frühlingsmorgens}\pwindex{D’Annunzio, Gabriele 12.\,3.\,1863 Pescara – 1.\,3.\,1938 Cargnacco@\textsc{D’Annunzio, Gabriele} (12.\,3.\,1863 Pescara – 1.\,3.\,1938 Cargnacco), \emph{Schriftsteller}!Traum eines Frühlingsmorgens@\strich\emph{Traum eines Frühlingsmorgens}|pwk} von Gabriele
                        d’Annunzio\pwindex{D’Annunzio, Gabriele 12.\,3.\,1863 Pescara – 1.\,3.\,1938 Cargnacco@\textsc{D’Annunzio, Gabriele} (12.\,3.\,1863 Pescara – 1.\,3.\,1938 Cargnacco), \emph{Schriftsteller}|pwk}, \emph{Mein Fürst}\pwindex{Scholz, Wilhelm von 15.\,7.\,1874 Berlin – 29.\,5.\,1969 Schloss Seeheim@\textsc{Scholz, Wilhelm von} (15.\,7.\,1874 Berlin – 29.\,5.\,1969 Schloss Seeheim), \emph{Schriftsteller, Kulturfunktionär}!Mein Fürst@\strich\emph{Mein Fürst}|pwk} von Wilhelm von Scholz\pwindex{Scholz, Wilhelm von 15.\,7.\,1874 Berlin – 29.\,5.\,1969 Schloss Seeheim@\textsc{Scholz, Wilhelm von} (15.\,7.\,1874 Berlin – 29.\,5.\,1969 Schloss Seeheim), \emph{Schriftsteller, Kulturfunktionär}|pwk} und Schnitzlers{ }\emph{Der grüne
                     Kakadu}\pwindex{Schnitzler, Arthur 15.\,5.\,1862 Wien – 21.\,10.\,1931 ebd.@\textsc{Schnitzler, Arthur} (15.\,5.\,1862 Wien – 21.\,10.\,1931 ebd.), \emph{Schriftsteller, Mediziner}!grüne Kakadu. Groteske in einem Akt@\strich\emph{Der grüne Kakadu. Groteske in einem Akt}|pwk} stattgefunden.}}}\label{K_L03543-1} durch die ersten kraefte der hofbuehne\orgindex{Residenztheater München@Residenztheater München|pw} einen so stuermischen erfolg errang wie ihn das residenztheater\oindex{Residenztheater München@\textbf{Residenztheater München}, \emph{Theater}|pw} seit jahren nicht erlebte. nach
               schluss des stueck\pwindex{Schnitzler, Arthur 15.\,5.\,1862 Wien – 21.\,10.\,1931 ebd.@\textsc{Schnitzler, Arthur} (15.\,5.\,1862 Wien – 21.\,10.\,1931 ebd.), \emph{Schriftsteller, Mediziner}!grüne Kakadu. Groteske in einem Akt@\strich\emph{Der grüne Kakadu. Groteske in einem Akt}|pwv}es wurden
               die darsteller ein dutzend mal hervorgejubelt mit bestem gruss =\pend
           \pstart \spacefill\mbox{ludwig ganghofer .+}\pend{}\selectlanguage{ngerman}\endnumbering\briefempfaengerindex{Schnitzler, Arthur@\textsc{Schnitzler, Arthur}!zzzGanghofer, Ludwig@\emph{von Ludwig Ganghofer}!1899-04-304@{30. 4. [1899]}|)be}\mylabel{L03543h}  \newcommand{\dateiname}{L03543}\newcommand{\titel}{Ludwig Ganghofer an Arthur Schnitzler, 30. 4. [1899]}\newcommand{\editorInnen}{Martin Anton Müller und Gerd-Hermann Susen}%% latex-leseansicht-abspann.tex
%% Abspann für die Leseansicht.
%% Der Schalter \ifkorrekturansicht ist bereits durch den Vorspann gesetzt.

%% latex-abspann.tex
%% Gemeinsamer Abspann für Korrekturansicht und Leseansicht.
%% Setzt den Schalter \ifkorrekturansicht voraus (gesetzt in den
%% einbindenden Dateien latex-korrekturansicht-abspann.tex bzw.
%% latex-leseansicht-abspann.tex).
%% ---------------------------------------------------------------

\normalsize

% Das esempio-Environment wird nur in der Leseansicht benötigt
\ifkorrekturansicht\else
\newenvironment{esempio}[3]%
{
    \vspace{1.5ex}
    \rlap{\underline{#1}}
    \par
    \setlength{\parindent}{0cm}
    \nopagebreak
    \leftskip=#2cm
    \rightskip=#3cm
}
{
    \par
}
\fi

\doendnotes{C}
\bigskip
\vfill

\clearpage

\footnotesize

\ifkorrekturansicht
  \lohead{\textsc{register}}
\fi

% theindex-Environment neu definieren ohne reledmac
\makeatletter
\renewenvironment{theindex}{%
  \ifkorrekturansicht
    \section*{\indexname}%
  \else
    \subsubsection*{Index der erwähnten Entitäten}%
  \fi
  \setlength{\parindent}{0pt}%
  \setlength{\parskip}{0pt plus 0.3pt}%
  \let\item\@idxitem
}{%
  \ifkorrekturansicht\clearpage\fi
}
\makeatother

\IfFileExists{\jobname-pw.ind}{\input{\jobname-pw.ind}}{}

% Quellenangabe nur in der Leseansicht
\ifkorrekturansicht\else
% Fallback-Definitionen, falls die .tex-Datei \titel etc. nicht gesetzt hat
\providecommand{\titel}{}
\providecommand{\editorInnen}{}
\providecommand{\dateiname}{\jobname}

\vspace{3cm}

\vfill

\footnotesize
\textsc{Quelle}: \titel. Herausgegeben von {\editorInnen}. In: \emph{Arthur Schnitzler: Briefwechsel mit Autorinnen und Autoren}.
 Digitale Edition, https://schnitzler-briefe.acdh.oeaw.ac.at/{\dateiname}.html (Stand \today)
\fi

\end{document}


