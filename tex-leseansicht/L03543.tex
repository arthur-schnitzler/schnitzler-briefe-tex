%% latex-korrekturansicht-vorspann.tex
%% Vorspann für die Korrekturansicht.
%% Lädt die gemeinsame Datei latex-vorspann.tex mit gesetztem Schalter.

\newif\ifkorrekturansicht
\korrekturansichttrue

\input{../tex-inputs/latex-vorspann}


\section[Ludwig Ganghofer an Arthur Schnitzler, 30. 4. {[}1899{]}]{L03543 Ludwig Ganghofer an Arthur Schnitzler, 30. 4. {[}1899{]}}
\nopagebreak\mylabel{L03543v}
\rehead{ }\normalsize\beginnumbering\briefempfaengerindex{Schnitzler, Arthur@\textsc{Schnitzler, Arthur}!zzzGanghofer, Ludwig@\emph{von Ludwig Ganghofer}!1899-04-301@{30. 4. {[}1899{]}}|(be}
\toendnotes[C]{\smallbreak\pagebreak[2]}\Standort{CUL, Schnitzler, B 775.}
\physDesc{Telegramm, 373 Zeichen (Vordruck: »\textcolor{gray}{\textbf{\textbf{\textsc{Berlin\oindex{Berlin@\textbf{Berlin}, \emph{P.PPLC}|pw},}}{ }Haupt-Telegraphenamt\oindex{Haupttelegrafenamt@\textbf{Haupttelegrafenamt}, \emph{Bürogebäude (K.BUR)}|pw}}}.«)
\newline{}maschinell
\newline{}Versand: 1) mit Bleistift rückseitiger Vermerk: »\noindent{}{\pb}Adrſ. wohnt Savoy-Hôtel\oindex{Hotel Savoy [Berlin]@\textbf{Hotel Savoy [Berlin]}, \emph{Hotel (K.HTL)}|pw}{ }Friedrichſtr\oindex{Friedrichstrasse [Berlin]@\textbf{Friedrichstraße [Berlin]}, \emph{Straße (K.STR)}|pw}{ / }Bote Frimmel\pwindex{Frimmel @\textsc{Frimmel}, \emph{Briefträger/Briefträgerin}|pw}«  2) mit rotem Buntstift vier Unterstreichungen und »\textcolor{gray}{K}« für Kakadu\pwindex{gruene Kakadu. Groteske in einem Akt@\emph{Der grüne Kakadu. Groteske in einem Akt}|pw}?}\toendnotes[C]{\smallbreak}
\pstart
           \centering{}{\pb}fr muenchen\oindex{Muenchen@\textbf{München}, \emph{P.PPLA}|pw} tel 55 30/4{ }9 m =\pend
           \vspace{0.5em}
\pstart
           kann jhnen zu meiner freude mitteilen dass gruener
                  kakadu\pwindex{gruene Kakadu. Groteske in einem Akt@\emph{Der grüne Kakadu. Groteske in einem Akt}|pw}{ }gestern{ }abend bei wirklich musterhafter \label{K_L03543-1v}\edtext{auffuehrung}{\lemma{\textnormal{\emph{auffuehrung}}}\Cendnote{\textnormal{Am
                     29. 4. 1899 hatten am Residenztheater\oindex{Residenztheater Muenchen@\textbf{Residenztheater München}, \emph{Theater (K.THE)}|pwk} in München\oindex{Muenchen@\textbf{München}, \emph{P.PPLA}|pwk} die Premieren von \emph{Traum eines
                     Frühlingsmorgens}\pwindex{Traum eines Fruehlingsmorgens@\emph{Traum eines Frühlingsmorgens}|pwk} von Gabriele
                        d’Annunzio\pwindex{DAnnunzio, Gabriele 12.03.1863 – 01.03.1938@\textsc{D’Annunzio, Gabriele} (12.03.1863 – 01.03.1938), \emph{Schriftsteller/Schriftstellerin}|pwk}, \emph{Mein Fürst}\pwindex{Mein Fuerst@\emph{Mein Fürst}|pwk} von Wilhelm von Scholz\pwindex{Scholz, Wilhelm von 15.07.1874 – 29.05.1969@\textsc{Scholz, Wilhelm von} (15.07.1874 – 29.05.1969), \emph{Schriftsteller/Schriftstellerin, Kulturfunktionär/Kulturfunktionärin}|pwk} und Schnitzlers{ }\emph{Der grüne
                     Kakadu}\pwindex{gruene Kakadu. Groteske in einem Akt@\emph{Der grüne Kakadu. Groteske in einem Akt}|pwk} stattgefunden.}}}\label{K_L03543-1} durch die ersten kraefte der hofbuehne\orgindex{Residenztheater Muenchen@Residenztheater München|pw} einen so stuermischen erfolg errang wie ihn das residenztheater\oindex{Residenztheater Muenchen@\textbf{Residenztheater München}, \emph{Theater (K.THE)}|pw} seit jahren nicht erlebte. nach
               schluss des stueck\pwindex{gruene Kakadu. Groteske in einem Akt@\emph{Der grüne Kakadu. Groteske in einem Akt}|pwv}es wurden
               die darsteller ein dutzend mal hervorgejubelt mit bestem gruss =\pend
           \pstart \spacefill\mbox{ludwig ganghofer .+}\pend{}\selectlanguage{ngerman}\endnumbering\briefempfaengerindex{Schnitzler, Arthur@\textsc{Schnitzler, Arthur}!zzzGanghofer, Ludwig@\emph{von Ludwig Ganghofer}!1899-04-301@{30. 4. {[}1899{]}}|)be}\mylabel{L03543h}  \normalsize

\doendnotes{C}
\bigskip
\vfill

\clearpage

\footnotesize

\lohead{\textsc{register}}

% Definiere theindex-Environment komplett neu ohne reledmac
\makeatletter
\renewenvironment{theindex}{%
  \section*{\indexname}%
  \setlength{\parindent}{0pt}%
  \setlength{\parskip}{0pt plus 0.3pt}%
  \let\item\@idxitem
}{%
  \clearpage
}
\makeatother

\IfFileExists{\jobname-pw.ind}{\input{\jobname-pw.ind}}{}

\end{document}

      