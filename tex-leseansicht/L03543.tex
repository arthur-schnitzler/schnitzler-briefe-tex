%% latex-leseansicht-vorspann.tex
%% Vorspann für die Leseansicht.
%% Lädt die gemeinsame Datei latex-vorspann.tex mit nicht gesetztem Schalter.

\newif\ifkorrekturansicht
\korrekturansichtfalse

\input{../tex-inputs/latex-vorspann}

\begin{center}
            \textcolor{red}{ENTWURF, NICHT FERTIG KORRIGIERT}
                      \end{center}
            
         
         \renewcommand{\erwaehntePersonen}{Personen: Gabriele D’Annunzio, Ludwig Ganghofer, Wilhelm von Scholz,  [T?]immel}
         \renewcommand{\erwaehnteInstitutionen}{Institutionen: Residenztheater München}
         \renewcommand{\erwaehnteOrte}{Orte: Berlin, Friedrichstraße, Haupttelegrafenamt, Hotel Savoy, München, Residenztheater München}
         \renewcommand{\erwaehnteWerke}{Werke: Der grüne Kakadu. Groteske in einem Akt, Mein Fürst, Traum eines Frühlingsmorgens}
               \section[Ludwig Ganghofer an Arthur Schnitzler, 30. 4. {[}1899{]}]{ Ludwig Ganghofer an Arthur Schnitzler, 30. 4. {[}1899{]}}\nopagebreak\mylabel{v}\rehead{ }\begin{ledgroupsized}[t]{13cm}\normalsize\beginnumbering \toendnotes[C]{\smallbreak\pagebreak[2]} \Standort{CUL, Schnitzler, B 775.}
\physDesc{Telegramm, 373 Zeichen (Vordruck
                                 \textcolor{gray}{\textbf{\textbf{Berlin\oindex{Berlin@\textbf{Berlin}|pw}, }Haupt-Telegraphenamt\oindex{Haupttelegrafenamt@\textbf{Haupttelegrafenamt}|pw}}})
\newline{}maschinell
\newline{}Versand: 1) mit Bleistift rückseitiger Vermerk: »\noindent{}{\pb}\textcolor{gray}{Adrſ.} wohnt Savoy-Hôtel\oindex{Hotel Savoy@\textbf{Hotel Savoy}|pw}{ }Friedrichſtr\oindex{Friedrichstrasse@\textbf{Friedrichstraße}|pw}{ / }Bote \textcolor{gray}{T}immel\pwindex{[T?]immel @\textsc{[T?]immel}, \emph{Dienstbote/Dienstbotin}|pw}«  2) mit rotem Buntstift vier Unterstreichungen und eine nicht
                                 entzifferte Paraphe}\toendnotes[C]{\smallbreak}\pstart
           \noindent{}\centering{}{\pb}fr muenchen\oindex{Muenchen@\textbf{München}|pw} tel 55 30/4{ }9m = \pend
           \pstart
           kann jhnen zu meiner freude mitteilen dass gruener
                  kakadu\pwindex{Schnitzler, Arthur 15.05.1862 – 21.10.1931@\textsc{Schnitzler, Arthur} (15.05.1862 – 21.10.1931), \emph{Schriftsteller, Mediziner}!gruene Kakadu. Groteske in einem Akt1. 3. 1899@\strich\emph{Der grüne Kakadu. Groteske in einem Akt} {[}1. 3. 1899{]}|pw} gestern abend bei wirklich musterhafter \label{K_L03543-1v}\edtext{auffuehrung}{\lemma{\textnormal{\emph{auffuehrung}}}\Cendnote{\textnormal{Am 29. 4. 1899 fanden
                     am Residenztheater\oindex{Residenztheater Muenchen@\textbf{Residenztheater München}|pwk} in München\oindex{Muenchen@\textbf{München}|pwk} die
                     Premieren von \emph{Traum eines Frühlingsmorgens}\pwindex{DAnnunzio, Gabriele 12.03.1863 – 01.03.1938@\textsc{D’Annunzio, Gabriele} (12.03.1863 – 01.03.1938), \emph{Schriftsteller}!Traum eines Fruehlingsmorgens1897-06-15@\strich\emph{Traum eines Frühlingsmorgens} {[}1897-06-15{]}|pwk} von Gabriele d’Annunzio\pwindex{DAnnunzio, Gabriele 12.03.1863 – 01.03.1938@\textsc{D’Annunzio, Gabriele} (12.03.1863 – 01.03.1938), \emph{Schriftsteller}|pwk}, \emph{Mein Fürst}\pwindex{Scholz, Wilhelm von 15.07.1874 – 29.05.1969@\textsc{Scholz, Wilhelm von} (15.07.1874 – 29.05.1969), \emph{Schriftsteller, Kulturfunktionär}!Mein Fuerst1898@\strich\emph{Mein Fürst} {[}1898{]}|pwk} von Wilhelm
                        von Scholz\pwindex{Scholz, Wilhelm von 15.07.1874 – 29.05.1969@\textsc{Scholz, Wilhelm von} (15.07.1874 – 29.05.1969), \emph{Schriftsteller, Kulturfunktionär}|pwk} und Schnitzler\pwindex{Schnitzler, Arthur 15.05.1862 – 21.10.1931@\textsc{Schnitzler, Arthur} (15.05.1862 – 21.10.1931), \emph{Schriftsteller, Mediziner}|pwk}s \emph{Der grüne Kakadu}\pwindex{Schnitzler, Arthur 15.05.1862 – 21.10.1931@\textsc{Schnitzler, Arthur} (15.05.1862 – 21.10.1931), \emph{Schriftsteller, Mediziner}!gruene Kakadu. Groteske in einem Akt1. 3. 1899@\strich\emph{Der grüne Kakadu. Groteske in einem Akt} {[}1. 3. 1899{]}|pwk}
                     statt.}}}\label{K_L03543-1h} durch die ersten
               kraefte der hofbuehne\orgindex{Residenztheater Muenchen@Residenztheater München|pw} einen so stuermischen
               erfolg errang wie ihn das residenztheater\oindex{Residenztheater Muenchen@\textbf{Residenztheater München}|pw} seit
               jahren nicht erlebte. nach schluss des stueckes wurden die darsteller ein dutzend mal
               hervorgejubelt mit bestem gruss =\pend
           \pstart \spacefill\mbox{ludwig ganghofer .–}\pend{}
         
         \endnumbering\mylabel{h}\end{ledgroupsized}\begin{anhang}\end{anhang}\newcommand{\dateiname}{L03543}\newcommand{\titel}{Ludwig Ganghofer an Arthur Schnitzler, 30. 4. [1899]}\newcommand{\editorInnen}{Martin Anton Müller und Laura Untner}%% latex-leseansicht-abspann.tex
%% Abspann für die Leseansicht.
%% Der Schalter \ifkorrekturansicht ist bereits durch den Vorspann gesetzt.

%% latex-abspann.tex
%% Gemeinsamer Abspann für Korrekturansicht und Leseansicht.
%% Setzt den Schalter \ifkorrekturansicht voraus (gesetzt in den
%% einbindenden Dateien latex-korrekturansicht-abspann.tex bzw.
%% latex-leseansicht-abspann.tex).
%% ---------------------------------------------------------------

\normalsize

% Das esempio-Environment wird nur in der Leseansicht benötigt
\ifkorrekturansicht\else
\newenvironment{esempio}[3]%
{
    \vspace{1.5ex}
    \rlap{\underline{#1}}
    \par
    \setlength{\parindent}{0cm}
    \nopagebreak
    \leftskip=#2cm
    \rightskip=#3cm
}
{
    \par
}
\fi

\doendnotes{C}
\bigskip
\vfill

\clearpage

\footnotesize

\ifkorrekturansicht
  \lohead{\textsc{register}}
\fi

% theindex-Environment neu definieren ohne reledmac
\makeatletter
\renewenvironment{theindex}{%
  \ifkorrekturansicht
    \section*{\indexname}%
  \else
    \subsubsection*{Index der erwähnten Entitäten}%
  \fi
  \setlength{\parindent}{0pt}%
  \setlength{\parskip}{0pt plus 0.3pt}%
  \let\item\@idxitem
}{%
  \ifkorrekturansicht\clearpage\fi
}
\makeatother

\IfFileExists{\jobname-pw.ind}{\input{\jobname-pw.ind}}{}

% Quellenangabe nur in der Leseansicht
\ifkorrekturansicht\else
% Fallback-Definitionen, falls die .tex-Datei \titel etc. nicht gesetzt hat
\providecommand{\titel}{}
\providecommand{\editorInnen}{}
\providecommand{\dateiname}{\jobname}

\vspace{3cm}

\vfill

\footnotesize
\textsc{Quelle}: \titel. Herausgegeben von {\editorInnen}. In: \emph{Arthur Schnitzler: Briefwechsel mit Autorinnen und Autoren}.
 Digitale Edition, https://schnitzler-briefe.acdh.oeaw.ac.at/{\dateiname}.html (Stand \today)
\fi

\end{document}


      