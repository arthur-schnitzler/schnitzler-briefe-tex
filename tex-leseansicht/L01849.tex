%% latex-korrekturansicht-vorspann.tex
%% Vorspann für die Korrekturansicht.
%% Lädt die gemeinsame Datei latex-vorspann.tex mit gesetztem Schalter.

\newif\ifkorrekturansicht
\korrekturansichttrue

\input{../tex-inputs/latex-vorspann}


\section[Richard Beer-Hofmann an Arthur Schnitzler, 26. 6. 1909]{L01849 Richard Beer-Hofmann an Arthur Schnitzler, 26. 6. 1909}
\nopagebreak\mylabel{L01849v}
\rehead{ }\normalsize\beginnumbering\briefempfaengerindex{Schnitzler, Arthur@\textsc{Schnitzler, Arthur}!zzzBeer-Hofmann, Richard@\emph{von Richard Beer-Hofmann}!1909-06-261@{26. 6. 1909}|(be}
\toendnotes[C]{\smallbreak\pagebreak[2]}\Standort{CUL, Schnitzler, B 8.}
\physDesc{Brief, 1 Blatt, 2 Seiten, 464 Zeichen
\newline{}Handschrift: Bleistift, lateinische Kurrent
\newline{}Schnitzler: mit Bleistift beschriftet: »\textsc{Beerhofm.}« 
\newline{}Ordnung: mit Bleistift von unbekannter Hand nummeriert:
                                    »219« }
\buchAbdrucke{\weitereDrucke{Arthur Schnitzler, Richard Beer-Hofmann: \emph{Briefwechsel 1891–1931}. Wien, Zürich: \emph{Europaverlag} 1992, S. 193.} }\toendnotes[C]{\smallbreak}
\pstart
           \raggedleft{}{\pb}\substVorne{}\textsuperscript{22}\substDazwischen{}26\substHinten{}/VI 09\pend
           \vspace{0.5em}
\pstart
           Lieber Arthur! Sie waren \label{K_L01849-1v}\edtext{vorgestern}{\lemma{\textnormal{\emph{vorgestern}}}\Cendnote{\textnormal{Vgl. A. S.: \emph{Tagebuch}, 24. 6. 1909.
               }}}\label{K_L01849-1}{ }Abends bei uns als wir schon im Türkenschanzpark\oindex{Tuerkenschanzpark@\textbf{Türkenschanzpark}, \emph{Park (K.PRK)}|pw} waren. Wir waren in bewusster, Ihnen odioser, Gesellschaft.
               Wir gehen heute wieder hinüber, haben dort Rendezvous mit Leo\pwindex{Van-Jung, Leo 15.10.1866 – 02.07.1939@\textsc{Van-Jung, Leo} (15.10.1866 – 02.07.1939), \emph{Gesangspädagoge/Gesangspädagogin, Mathematiker/Mathematikerin}|pw}, Bella Wengeroff\pwindex{Vengerova, Isabella 01.03.1877 – 07.02.1956@\textsc{Vengerova, Isabella} (01.03.1877 – 07.02.1956), \emph{Musikpädagoge/Musikpädagogin, Pianist/Pianistin}|pw}, Kaufmann\pwindex{Kaufmann, Arthur 04.04.1872 – 25.07.1938@\textsc{Kaufmann, Arthur} (04.04.1872 – 25.07.1938), \emph{Rechtswissenschaftler/Rechtswissenschaftlerin, Privatgelehrte/Privatgelehrte, Privatier/Privatière}|pw}; vielleicht ko{\geminationm}en Sie doch? (Ich bemerke soeben dass {\pb}»doch« keinen Sinn hat.) Also
               »auch«! Wir \label{K_L01849-2v}\edtext{reisen}{\lemma{\textnormal{\emph{reisen}}}\Cendnote{\textnormal{nach Pichl
                     am See\oindex{Pichl am See@\textbf{Pichl am See}, \emph{P.PPL}|pwk}}}}\label{K_L01849-2} (– nein, \uline{wollen} reisen – nein reisen \uline{sicher}, nein – Schicksal mach \uuline{\edtext{D}{\Cendnote{dreifach unterstrichen}}}ir selber den »Dreh« der \uuline{\edtext{D}{\Cendnote{dreifach unterstrichen}}}ir passt)
                  Dienstag{ }Früh.\pend
           
\pstart
           Herzlichst{\\[\baselineskip]}\spacefill\mbox{Richard}\pend
           \leftskip=0em{}\selectlanguage{ngerman}\endnumbering\briefempfaengerindex{Schnitzler, Arthur@\textsc{Schnitzler, Arthur}!zzzBeer-Hofmann, Richard@\emph{von Richard Beer-Hofmann}!1909-06-261@{26. 6. 1909}|)be}\mylabel{L01849h}  \normalsize

\doendnotes{C}
\bigskip
\vfill

\clearpage

\footnotesize

\lohead{\textsc{register}}

% Definiere theindex-Environment komplett neu ohne reledmac
\makeatletter
\renewenvironment{theindex}{%
  \section*{\indexname}%
  \setlength{\parindent}{0pt}%
  \setlength{\parskip}{0pt plus 0.3pt}%
  \let\item\@idxitem
}{%
  \clearpage
}
\makeatother

\IfFileExists{\jobname-pw.ind}{\input{\jobname-pw.ind}}{}

\end{document}

      