%% latex-leseansicht-vorspann.tex
%% Vorspann für die Leseansicht.
%% Lädt die gemeinsame Datei latex-vorspann.tex mit nicht gesetztem Schalter.

\newif\ifkorrekturansicht
\korrekturansichtfalse

\input{../tex-inputs/latex-vorspann}


\section[Richard Beer-Hofmann an Arthur Schnitzler, 26. 6. 1909]{L01849 Richard Beer-Hofmann an Arthur Schnitzler, 26. 6. 1909}
\nopagebreak\mylabel{L01849v}
\rehead{ }\normalsize\beginnumbering\briefempfaengerindex{Schnitzler, Arthur@\textsc{Schnitzler, Arthur}!zzzBeer-Hofmann, Richard@\emph{von Richard Beer-Hofmann}!1909-06-261@{26. 6. 1909}|(be}
\toendnotes[C]{\smallbreak\pagebreak[2]}
\correspDesc{Versand  durch Richard Beer-Hofmann am 26. 6. 1909 in Wien
\newline{}Erhalt  durch Arthur Schnitzler im Zeitraum [26. 6. 1909
                  – 30. 6. 1909?] in Wien}\toendnotes[C]{\smallbreak}
\Standort{CUL, Schnitzler, B 8.}
\physDesc{Brief, 1 Blatt, 2 Seiten, 464 Zeichen
\newline{}Handschrift: Bleistift, lateinische Kurrent
\newline{}Schnitzler: mit Bleistift beschriftet: »\textsc{Beerhofm.}« 
\newline{}Ordnung: mit Bleistift von unbekannter Hand nummeriert:
                                    »219« }
\buchAbdrucke{\weitereDrucke{Arthur Schnitzler, Richard Beer-Hofmann: \emph{Briefwechsel 1891–1931}. Herausgegeben von Konstanze Fliedl. Wien, Zürich: \emph{Europaverlag} 1992, S. 193.} }\toendnotes[C]{\smallbreak}
\pstart
           \raggedleft{}{\pb}\substVorne{}\textsuperscript{22}\substDazwischen{}26\substHinten{}/VI 09\pend
           \vspace{0.5em}
\pstart
           Lieber Arthur! Sie waren \label{K_L01849-1v}\edtext{vorgestern}{\lemma{\textnormal{\emph{vorgestern}}}\Cendnote{\textnormal{Vgl. A. S.: \emph{Tagebuch}, 24. 6. 1909.
               }}}\label{K_L01849-1}{ }Abends bei uns als wir schon im Türkenschanzpark\oindex{Wien@\textbf{Wien}!XVIII., Währing@\textbf{XVIII., Währing}!Türkenschanzpark@\textbf{Türkenschanzpark}, \emph{Park}|pw} waren. Wir waren in bewusster, Ihnen odioser, Gesellschaft.
               Wir gehen heute wieder hinüber, haben dort Rendezvous mit Leo\pwindex{Van-Jung, Leo 15.\,10.\,1866 Odessa – 2.\,7.\,1939 Riga@\textsc{Van-Jung, Leo} (15.\,10.\,1866 Odessa – 2.\,7.\,1939 Riga), \emph{Gesangspädagoge, Mathematiker}|pw}, Bella Wengeroff\pwindex{Vengerova, Isabella 1.\,3.\,1877 Minsk – 7.\,2.\,1956 New York City@\textsc{Vengerova, Isabella} (1.\,3.\,1877 Minsk – 7.\,2.\,1956 New York City), \emph{Musikpädagogin, Pianistin}|pw}, Kaufmann\pwindex{Kaufmann, Arthur 4.\,4.\,1872 Iași – 25.\,7.\,1938 Wien@\textsc{Kaufmann, Arthur} (4.\,4.\,1872 Iași – 25.\,7.\,1938 Wien), \emph{Rechtswissenschaftler, Privatgelehrte, Privatier}|pw}; vielleicht ko{\geminationm}en Sie doch? (Ich bemerke soeben dass {\pb}»doch« keinen Sinn hat.) Also
               »auch«! Wir \label{K_L01849-2v}\edtext{reisen}{\lemma{\textnormal{\emph{reisen}}}\Cendnote{\textnormal{nach Pichl
                     am See\oindex{Pichl am See@\textbf{Pichl am See}|pwk}}}}\label{K_L01849-2} (– nein, \uline{wollen} reisen – nein reisen \uline{sicher}, nein – Schicksal mach \uuline{\edtext{D}{\Cendnote{dreifach unterstrichen}}}ir selber den »Dreh« der \uuline{\edtext{D}{\Cendnote{dreifach unterstrichen}}}ir passt)
                  Dienstag{ }Früh.\pend
           
\pstart
           Herzlichst{\\[\baselineskip]}\spacefill\mbox{Richard}\pend
           \leftskip=0em{}\selectlanguage{ngerman}\endnumbering\briefempfaengerindex{Schnitzler, Arthur@\textsc{Schnitzler, Arthur}!zzzBeer-Hofmann, Richard@\emph{von Richard Beer-Hofmann}!1909-06-261@{26. 6. 1909}|)be}\mylabel{L01849h}  \newcommand{\dateiname}{L01849}\newcommand{\titel}{Richard Beer-Hofmann an Arthur Schnitzler, 26. 6. 1909}\newcommand{\editorInnen}{Martin Anton Müller und Gerd-Hermann Susen}%% latex-leseansicht-abspann.tex
%% Abspann für die Leseansicht.
%% Der Schalter \ifkorrekturansicht ist bereits durch den Vorspann gesetzt.

%% latex-abspann.tex
%% Gemeinsamer Abspann für Korrekturansicht und Leseansicht.
%% Setzt den Schalter \ifkorrekturansicht voraus (gesetzt in den
%% einbindenden Dateien latex-korrekturansicht-abspann.tex bzw.
%% latex-leseansicht-abspann.tex).
%% ---------------------------------------------------------------

\normalsize

% Das esempio-Environment wird nur in der Leseansicht benötigt
\ifkorrekturansicht\else
\newenvironment{esempio}[3]%
{
    \vspace{1.5ex}
    \rlap{\underline{#1}}
    \par
    \setlength{\parindent}{0cm}
    \nopagebreak
    \leftskip=#2cm
    \rightskip=#3cm
}
{
    \par
}
\fi

\doendnotes{C}
\bigskip
\vfill

\clearpage

\footnotesize

\ifkorrekturansicht
  \lohead{\textsc{register}}
\fi

% theindex-Environment neu definieren ohne reledmac
\makeatletter
\renewenvironment{theindex}{%
  \ifkorrekturansicht
    \section*{\indexname}%
  \else
    \subsubsection*{Index der erwähnten Entitäten}%
  \fi
  \setlength{\parindent}{0pt}%
  \setlength{\parskip}{0pt plus 0.3pt}%
  \let\item\@idxitem
}{%
  \ifkorrekturansicht\clearpage\fi
}
\makeatother

\IfFileExists{\jobname-pw.ind}{\input{\jobname-pw.ind}}{}

% Quellenangabe nur in der Leseansicht
\ifkorrekturansicht\else
% Fallback-Definitionen, falls die .tex-Datei \titel etc. nicht gesetzt hat
\providecommand{\titel}{}
\providecommand{\editorInnen}{}
\providecommand{\dateiname}{\jobname}

\vspace{3cm}

\vfill

\footnotesize
\textsc{Quelle}: \titel. Herausgegeben von {\editorInnen}. In: \emph{Arthur Schnitzler: Briefwechsel mit Autorinnen und Autoren}.
 Digitale Edition, https://schnitzler-briefe.acdh.oeaw.ac.at/{\dateiname}.html (Stand \today)
\fi

\end{document}


