%% latex-leseansicht-vorspann.tex
%% Vorspann für die Leseansicht.
%% Lädt die gemeinsame Datei latex-vorspann.tex mit nicht gesetztem Schalter.

\newif\ifkorrekturansicht
\korrekturansichtfalse

\input{../tex-inputs/latex-vorspann}


         
         \renewcommand{\erwaehntePersonen}{Personen: Michael Georg Conrad, Theodore Rottenberg}
         \renewcommand{\erwaehnteOrte}{Orte: Berlin, Dessauer Straße, Wien}
         \renewcommand{\erwaehnteWerke}{Werke: Arthur Schnitzler [Reigen-Privatdruck], Aus der Theaterwelt. (Der gefährlichste Feind der Theatersaison. – Eine interessante Novität Arthur Schnitzler’s. – Dessous der »Familie Wawroch«. – Der Naturalismus in der Desinfektionsanstalt. – Der Claquechef des Deutschen Volkstheaters in …, Die Gesellschaft. Monatsschrift für Litteratur, Kunst und Sozialpolitik, Fremden-Blatt, Reigen. Zehn Dialoge}
               \section[ Paul Goldmann an Arthur Schnitzler, 27. 4. {[}1900{]}]{ Paul Goldmann an Arthur Schnitzler, 27. 4. {[}1900{]}}\nopagebreak\mylabel{v}\rehead{ }\begin{ledgroupsized}[t]{13cm}\normalsize\beginnumbering \toendnotes[C]{\smallbreak\pagebreak[2]} \Standort{DLA, A:Schnitzler, HS.NZ85.1.3170.}
\physDesc{Brief, 1 Blatt, 2 Seiten, 902 Zeichen
\newline{}Handschrift: blaue Tinte, deutsche Kurrent
\newline{}Schnitzler: mit Bleistift das Jahr »900« vermerkt }\toendnotes[C]{\smallbreak}\pstart
           \noindent{}{\pb}\textcolor{gray}{\textbf{DESSAUERSTRASSE 19}}\oindex{Dessauer Strasse@\textbf{Dessauer Straße}|pw}\hfill Berlin\oindex{Berlin@\textbf{Berlin}|pw}, 27. April.\pend
           \pstart\center{}Mein lieber Freund,\pend\pstart
           Ich war ſehr erſtaunt, als ich ſah, daß die Sache mit dem »Reigen\pwindex{Schnitzler, Arthur 15.05.1862 – 21.10.1931@\textsc{Schnitzler, Arthur} (15.05.1862 – 21.10.1931), \emph{Schriftsteller, Mediziner}!Reigen. Zehn Dialoge1900@\strich\emph{Reigen. Zehn Dialoge} {[}1900{]}|pw}« in die \label{K_L02913-1v}\edtext{Zeitungen}{\lemma{\textnormal{\emph{Zeitungen}}}\Cendnote{\textnormal{Am
                     22. 4. 1900 brachte das \emph{Fremdenblatt}\pwindex{?? Werk@Nicht ermittelte Verfasserinnen und Verfasser!Fremden-Blatt1.7.1847 – 22.3.1919@\emph{Fremden-Blatt} {[}1.7.1847 – 22.3.1919{]}|pwk} folgende Meldung\pwindex{?? Werk@Nicht ermittelte Verfasserinnen und Verfasser!Aus der Theaterwelt. (Der gefaehrlichste Feind der Theatersaison. – Eine
                  interessante Novitaet Arthur Schnitzler s. – Dessous der »Familie Wawroch«. – Der
                  Naturalismus in der Desinfektionsanstalt. – Der Claquechef des Deutschen
                  Volkstheaters in …1900-04-22@\emph{Aus der Theaterwelt. (Der gefährlichste Feind der Theatersaison. – Eine interessante Novität Arthur Schnitzler’s. – Dessous der »Familie Wawroch«. – Der Naturalismus in der Desinfektionsanstalt. – Der Claquechef des Deutschen Volkstheaters in …} {[}1900-04-22{]}|pwkv} in ihrer Kolumne über Ereignisse in Theaterkreisen: Schnitzler\pwindex{Schnitzler, Arthur 15.05.1862 – 21.10.1931@\textsc{Schnitzler, Arthur} (15.05.1862 – 21.10.1931), \emph{Schriftsteller, Mediziner}|pwk} »hat ein neues Buch
                     geschrieben, aber kein dramatisches. Es nennt sich ›\so{Reigen}\pwindex{Schnitzler, Arthur 15.05.1862 – 21.10.1931@\textsc{Schnitzler, Arthur} (15.05.1862 – 21.10.1931), \emph{Schriftsteller, Mediziner}!Reigen. Zehn Dialoge1900@\strich\emph{Reigen. Zehn Dialoge} {[}1900{]}|pw}‹ und schildert – wie sagt man nur, was? – die verschiedenartigen
                     Gestalten, welche Liebe annimmt, wenn sie in der ärmsten Volksschichte oder bei
                     armen Leuten, beim Kleinbürger oder beim wohlhabenden Bourgeois bis hinauf in
                     den vornehmen Gesellschaftskreisen erscheint. Damen, welche das Buch\pwindex{Schnitzler, Arthur 15.05.1862 – 21.10.1931@\textsc{Schnitzler, Arthur} (15.05.1862 – 21.10.1931), \emph{Schriftsteller, Mediziner}!Reigen. Zehn Dialoge1900@\strich\emph{Reigen. Zehn Dialoge} {[}1900{]}|pwv} kaufen wollen, würden aber
                     vergeblich vor dem Buchhandlungsgehilfen erröthen. Denn der Verfasser hat das
                        Buch\pwindex{Schnitzler, Arthur 15.05.1862 – 21.10.1931@\textsc{Schnitzler, Arthur} (15.05.1862 – 21.10.1931), \emph{Schriftsteller, Mediziner}!Reigen. Zehn Dialoge1900@\strich\emph{Reigen. Zehn Dialoge} {[}1900{]}|pwv} nur in
                     zweihundert Exemplaren als Manuskript drucken lassen, um diese an einen
                     ausgewählten Kreis von Herren zu versenden. Die geringe Auflage des Buches
                     gestattete dem Verfasser, die Vorrede in jedem Exemplare mit seiner
                     eigenhändigen Unterschrift zu versehen – eine Aufmerksamkeit, die das Buch
                     jedem Besitzer umso interessanter erscheinen läßt.« Ähnlich lautende
                  Meldungen wurden in Folge auch außerhalb Wien\oindex{Wien@\textbf{Wien}|pwk}s
                  abgedruckt, beispielsweise: M. G. C.\pwindex{Conrad, Michael Georg 05.04.1846 – 20.12.1927@\textsc{Conrad, Michael Georg} (05.04.1846 – 20.12.1927), \emph{Schriftsteller, Kritiker}|pwkv} [ = Michael Georg Conrad\pwindex{Conrad, Michael Georg 05.04.1846 – 20.12.1927@\textsc{Conrad, Michael Georg} (05.04.1846 – 20.12.1927), \emph{Schriftsteller, Kritiker}|pwk}]: \emph{Arthur Schnitzler}\pwindex{Arthur Schnitzler [Reigen-Privatdruck]1900@\emph{Arthur Schnitzler [Reigen-Privatdruck]} {[}1900{]}|pwk}. In: \emph{Die Gesellschaft. Halbmonatschrift für Litteratur, Kunst und
                        Sozialpolitik}\pwindex{Gesellschaft. Monatsschrift fuer Litteratur, Kunst und
                  Sozialpolitik1885 – 1902@\emph{Die Gesellschaft. Monatsschrift für Litteratur, Kunst und Sozialpolitik} {[}1885 – 1902{]}|pwk}, Jg. 16, Bd. 3, H. 4, 1900,
                     S. 251.}}}\label{K_L02913-1h} gekommen iſt, und die betreffenden Notizen in den Wien\oindex{Wien@\textbf{Wien}|pw}er Blättern ſind eine Albernheit oder eine
               Perfidie. Gefahr könnte erſt entſtehen, wenn Du von irgendwelchem Lumpenhunde beim
                  \strikeout{der} Staatsanwalt denuncirt würdeſt. Und da man
               immer mit ſolchen Lumpenhunden rechnen muß, und da Vorſicht niemals ſchaden kann, {\pb}möchte ich Dir rathen, einen verläßlichen Advokaten
               zu conſultiren, ob man Dir irgend Etwas anhaben kann. Ich glaube zwar nicht, aber es
               iſt immer gut, für alle Fälle \strikeout{be} vorbereitet zu ſein.
               Du aber mußt dafür ſorgen (und haſt jedenfalls ſchon dafür geſorgt), daß das Buch\pwindex{Schnitzler, Arthur 15.05.1862 – 21.10.1931@\textsc{Schnitzler, Arthur} (15.05.1862 – 21.10.1931), \emph{Schriftsteller, Mediziner}!Reigen. Zehn Dialoge1900@\strich\emph{Reigen. Zehn Dialoge} {[}1900{]}|pwv} nur in die Hände ſicherer
               Leute kommt. Vor allen \strikeout{Di} Dingen nicht in weibliche
               Hände! Was man einer Frau gibt, trägt man auf den offenen Markt. \label{K_L02913-2v}\edtext{Ich weiß ein Lied davon zu ſingen.}{\lemma{\textnormal{\emph{Ich … ſingen.}}}\Cendnote{\textnormal{vermutlich Bezug auf Goldmann\pwindex{Goldmann, Paul 31.01.1865 – 25.09.1935@\textsc{Goldmann, Paul} (31.01.1865 – 25.09.1935), \emph{Schriftsteller, Journalist}|pwk}s Beziehung mit Theodore Rottenberg\pwindex{Rottenberg, Theodore 1875-09-07 – 1945-04-05@\textsc{Rottenberg, Theodore} (1875-09-07 – 1945-04-05)|pwk}, siehe Paul Goldmann an Arthur Schnitzler, 12. 11. [1899]}}}\label{K_L02913-2h}\pend
           \pstart
           Viele treue Grüße! {\\[\baselineskip]}Dein {\\[\baselineskip]}\spacefill\mbox{Paul Goldmann.}\pend
           \leftskip=0em{}
         
         \endnumbering\mylabel{h}\end{ledgroupsized}  \newcommand{\dateiname}{L02913}\newcommand{\titel}{Paul Goldmann an Arthur Schnitzler, 27. 4. [1900]}\newcommand{\editorInnen}{Martin Anton Müller und Laura Untner}%% latex-leseansicht-abspann.tex
%% Abspann für die Leseansicht.
%% Der Schalter \ifkorrekturansicht ist bereits durch den Vorspann gesetzt.

%% latex-abspann.tex
%% Gemeinsamer Abspann für Korrekturansicht und Leseansicht.
%% Setzt den Schalter \ifkorrekturansicht voraus (gesetzt in den
%% einbindenden Dateien latex-korrekturansicht-abspann.tex bzw.
%% latex-leseansicht-abspann.tex).
%% ---------------------------------------------------------------

\normalsize

% Das esempio-Environment wird nur in der Leseansicht benötigt
\ifkorrekturansicht\else
\newenvironment{esempio}[3]%
{
    \vspace{1.5ex}
    \rlap{\underline{#1}}
    \par
    \setlength{\parindent}{0cm}
    \nopagebreak
    \leftskip=#2cm
    \rightskip=#3cm
}
{
    \par
}
\fi

\doendnotes{C}
\bigskip
\vfill

\clearpage

\footnotesize

\ifkorrekturansicht
  \lohead{\textsc{register}}
\fi

% theindex-Environment neu definieren ohne reledmac
\makeatletter
\renewenvironment{theindex}{%
  \ifkorrekturansicht
    \section*{\indexname}%
  \else
    \subsubsection*{Index der erwähnten Entitäten}%
  \fi
  \setlength{\parindent}{0pt}%
  \setlength{\parskip}{0pt plus 0.3pt}%
  \let\item\@idxitem
}{%
  \ifkorrekturansicht\clearpage\fi
}
\makeatother

\IfFileExists{\jobname-pw.ind}{\input{\jobname-pw.ind}}{}

% Quellenangabe nur in der Leseansicht
\ifkorrekturansicht\else
% Fallback-Definitionen, falls die .tex-Datei \titel etc. nicht gesetzt hat
\providecommand{\titel}{}
\providecommand{\editorInnen}{}
\providecommand{\dateiname}{\jobname}

\vspace{3cm}

\vfill

\footnotesize
\textsc{Quelle}: \titel. Herausgegeben von {\editorInnen}. In: \emph{Arthur Schnitzler: Briefwechsel mit Autorinnen und Autoren}.
 Digitale Edition, https://schnitzler-briefe.acdh.oeaw.ac.at/{\dateiname}.html (Stand \today)
\fi

\end{document}


      