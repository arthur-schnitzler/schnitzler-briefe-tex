%% latex-leseansicht-vorspann.tex
%% Vorspann für die Leseansicht.
%% Lädt die gemeinsame Datei latex-vorspann.tex mit nicht gesetztem Schalter.

\newif\ifkorrekturansicht
\korrekturansichtfalse

\input{../tex-inputs/latex-vorspann}


\section[ Paul Goldmann an Arthur Schnitzler, 27. 4. [1900]]{L02913 Paul Goldmann an Arthur Schnitzler,  27. 4. [1900]}
\nopagebreak\mylabel{L02913v}
\rehead{ }\normalsize\beginnumbering\briefempfaengerindex{Schnitzler, Arthur@\textsc{Schnitzler, Arthur}!zzzGoldmann, Paul@\emph{von Paul Goldmann}!1900-04-271@{27. 4. [1900]}|(be}
\toendnotes[C]{\smallbreak\pagebreak[2]}
\correspDesc{Versand  durch Paul Goldmann am 27. 4. [1900] in Berlin
\newline{}Erhalt  durch Arthur Schnitzler im Zeitraum [28. 4. 1900
                  – 2. 5. 1900?] in Wien}\toendnotes[C]{\smallbreak}
\Standort{DLA, A:Schnitzler, HS.NZ85.1.3170.}
\physDesc{Brief, 1 Blatt, 2 Seiten, 902 Zeichen
\newline{}Handschrift: blaue Tinte, deutsche Kurrent
\newline{}Schnitzler: mit Bleistift das Jahr »900« vermerkt }\toendnotes[C]{\smallbreak}
\pstart
           {\pb}\textcolor{gray}{\textbf{DESSAUERSTRASSE 19}}\oindex{Dessauer Straße@\textbf{Dessauer Straße}, \emph{Straße}|pw}\hfill Berlin\oindex{Berlin@\textbf{Berlin}, \emph{Hauptstadt}|pw}, 27. April.\pend
           
\pstart\center{}Mein lieber Freund,\pend\vspace{0.5em}
\pstart
           Ich war{ }ſehr erſtaunt, als ich{ }ſah, daß die Sache mit dem »Reigen\pwindex{Schnitzler, Arthur 15.\,5.\,1862 Wien – 21.\,10.\,1931 ebd.@\textsc{Schnitzler, Arthur} (15.\,5.\,1862 Wien – 21.\,10.\,1931 ebd.), \emph{Schriftsteller, Mediziner}!Reigen. Zehn Dialoge@\strich\emph{Reigen. Zehn Dialoge}|pw}« in die \label{K_L02913-1v}\edtext{Zeitungen}{\lemma{\textnormal{\emph{Zeitungen}}}\Cendnote{\textnormal{Am
                     22. 4. 1900 brachte das \emph{Fremdenblatt}\pwindex{Fremden-Blatt@\emph{Fremden-Blatt}|pwk} folgende Meldung\pwindex{Aus der Theaterwelt. (Der gefährlichste Feind der Theatersaison. – Eine interessante Novität Arthur Schnitzler’s. – Dessous der »Familie Wawroch«. – Der Naturalismus in der Desinfektionsanstalt. – Der Claquechef des Deutschen Volkstheaters in …@\emph{Aus der Theaterwelt. (Der gefährlichste Feind der Theatersaison. – Eine interessante Novität Arthur Schnitzler’s. – Dessous der »Familie Wawroch«. – Der Naturalismus in der Desinfektionsanstalt. – Der Claquechef des Deutschen Volkstheaters in …}|pwkv} in seiner Kolumne über Ereignisse in Theaterkreisen: Schnitzler »hat ein neues Buch
                     geschrieben, aber kein dramatisches. Es nennt sich ›\so{Reigen}\pwindex{Schnitzler, Arthur 15.\,5.\,1862 Wien – 21.\,10.\,1931 ebd.@\textsc{Schnitzler, Arthur} (15.\,5.\,1862 Wien – 21.\,10.\,1931 ebd.), \emph{Schriftsteller, Mediziner}!Reigen. Zehn Dialoge@\strich\emph{Reigen. Zehn Dialoge}|pw}‹ und schildert – wie sagt man nur, was? – die verschiedenartigen
                     Gestalten, welche Liebe annimmt, wenn sie in der ärmsten Volksschichte oder bei
                     armen Leuten, beim Kleinbürger oder beim wohlhabenden Bourgeois bis hinauf in
                     den vornehmen Gesellschaftskreisen erscheint. Damen, welche das Buch\pwindex{Schnitzler, Arthur 15.\,5.\,1862 Wien – 21.\,10.\,1931 ebd.@\textsc{Schnitzler, Arthur} (15.\,5.\,1862 Wien – 21.\,10.\,1931 ebd.), \emph{Schriftsteller, Mediziner}!Reigen. Zehn Dialoge@\strich\emph{Reigen. Zehn Dialoge}|pwv} kaufen wollen, würden aber
                     vergeblich vor dem Buchhandlungsgehilfen erröthen. Denn der Verfasser hat das
                        Buch\pwindex{Schnitzler, Arthur 15.\,5.\,1862 Wien – 21.\,10.\,1931 ebd.@\textsc{Schnitzler, Arthur} (15.\,5.\,1862 Wien – 21.\,10.\,1931 ebd.), \emph{Schriftsteller, Mediziner}!Reigen. Zehn Dialoge@\strich\emph{Reigen. Zehn Dialoge}|pwv} nur in
                     zweihundert Exemplaren als Manuskript drucken lassen, um diese an einen
                     ausgewählten Kreis von Herren zu versenden. Die geringe Auflage des Buches
                     gestattete dem Verfasser, die Vorrede in jedem Exemplare mit seiner
                     eigenhändigen Unterschrift zu versehen – eine Aufmerksamkeit, die das Buch
                     jedem Besitzer umso interessanter erscheinen läßt.« Ähnlich lautende
                  Meldungen wurden in Folge auch außerhalb Wiens\oindex{Wien@\textbf{Wien}, \emph{Verwaltungsgebiet}|pwk}
                  abgedruckt, beispielsweise: M. G. C.\pwindex{Conrad, Michael Georg 5.\,4.\,1846 Gnodstadt – 20.\,12.\,1927 München@\textsc{Conrad, Michael Georg} (5.\,4.\,1846 Gnodstadt – 20.\,12.\,1927 München), \emph{Schriftsteller, Kritiker}|pwkv} [ = Michael Georg Conrad\pwindex{Conrad, Michael Georg 5.\,4.\,1846 Gnodstadt – 20.\,12.\,1927 München@\textsc{Conrad, Michael Georg} (5.\,4.\,1846 Gnodstadt – 20.\,12.\,1927 München), \emph{Schriftsteller, Kritiker}|pwk}]: \emph{Arthur Schnitzler}\pwindex{Conrad, Michael Georg 5.\,4.\,1846 Gnodstadt – 20.\,12.\,1927 München@\textsc{Conrad, Michael Georg} (5.\,4.\,1846 Gnodstadt – 20.\,12.\,1927 München), \emph{Schriftsteller, Kritiker}!Arthur Schnitzler [Reigen-Privatdruck]@\strich\emph{Arthur Schnitzler [Reigen-Privatdruck]}|pwk}. In: \emph{Die Gesellschaft. Halbmonatschrift für Litteratur, Kunst und
                        Sozialpolitik}\pwindex{Gesellschaft. Monatsschrift für Litteratur, Kunst und Sozialpolitik@\emph{Die Gesellschaft. Monatsschrift für Litteratur, Kunst und Sozialpolitik}|pwk}, Jg. 16, Bd. 3, H. 4, 1900,
                     S. 251.}}}\label{K_L02913-1} gekommen iſt, und die betreffenden Notizen in den Wien\oindex{Wien@\textbf{Wien}, \emph{Verwaltungsgebiet}|pw}er Blättern{ }ſind eine Albernheit oder eine
               Perfidie. Gefahr könnte erſt entſtehen, wenn Du von irgendwelchem Lumpenhunde beim
                  \strikeout{der} Staatsanwalt denuncirt würdeſt. Und da man
               immer mit{ }ſolchen Lumpenhunden rechnen muß, und da Vorſicht niemals{ }ſchaden kann, {\pb}möchte ich Dir rathen, einen verläßlichen Advokaten
               zu conſultiren, ob man Dir irgend Etwas anhaben kann. Ich glaube zwar nicht, aber es
               iſt immer gut, für alle Fälle \strikeout{be} vorbereitet zu{ }ſein.
               Du aber mußt dafür{ }ſorgen (und haſt jedenfalls{ }ſchon dafür geſorgt), daß das Buch\pwindex{Schnitzler, Arthur 15.\,5.\,1862 Wien – 21.\,10.\,1931 ebd.@\textsc{Schnitzler, Arthur} (15.\,5.\,1862 Wien – 21.\,10.\,1931 ebd.), \emph{Schriftsteller, Mediziner}!Reigen. Zehn Dialoge@\strich\emph{Reigen. Zehn Dialoge}|pwv} nur in die Hände{ }ſicherer
               Leute kommt. Vor allen \strikeout{Di} Dingen nicht in weibliche
               Hände! Was man einer Frau gibt, trägt man auf den offenen Markt. \label{K_L02913-2v}\edtext{Ich weiß ein Lied davon zu{ }ſingen.}{\lemma{\textnormal{\emph{Ich … singen.}}}\Cendnote{\textnormal{vermutlich Bezug auf Goldmanns\pwindex{Goldmann, Paul 31.\,1.\,1865 Breslau – 25.\,9.\,1935 Wien@\textsc{Goldmann, Paul} (31.\,1.\,1865 Breslau – 25.\,9.\,1935 Wien), \emph{Schriftsteller, Journalist}|pwk} Beziehung mit Theodore Rottenberg\pwindex{Rottenberg, Theodore 7.\,9.\,1875 – 5.\,4.\,1945 Limburg an der Lahn@\textsc{Rottenberg, Theodore} (7.\,9.\,1875 – 5.\,4.\,1945 Limburg an der Lahn)|pwk}, siehe XXXX Auszeichnungsfehler: Dokument L02893 nicht gefunden.}}}\label{K_L02913-2}\pend
           
\pstart
           Viele treue Grüße! {\\[\baselineskip]}Dein {\\[\baselineskip]}\spacefill\mbox{Paul Goldmann.}\pend
           \leftskip=0em{}\selectlanguage{ngerman}\endnumbering\briefempfaengerindex{Schnitzler, Arthur@\textsc{Schnitzler, Arthur}!zzzGoldmann, Paul@\emph{von Paul Goldmann}!1900-04-271@{27. 4. [1900]}|)be}\mylabel{L02913h}  \newcommand{\dateiname}{L02913}\newcommand{\titel}{Paul Goldmann an Arthur Schnitzler, 27. 4. [1900]}\newcommand{\editorInnen}{Martin Anton Müller und Laura Untner}%% latex-leseansicht-abspann.tex
%% Abspann für die Leseansicht.
%% Der Schalter \ifkorrekturansicht ist bereits durch den Vorspann gesetzt.

%% latex-abspann.tex
%% Gemeinsamer Abspann für Korrekturansicht und Leseansicht.
%% Setzt den Schalter \ifkorrekturansicht voraus (gesetzt in den
%% einbindenden Dateien latex-korrekturansicht-abspann.tex bzw.
%% latex-leseansicht-abspann.tex).
%% ---------------------------------------------------------------

\normalsize

% Das esempio-Environment wird nur in der Leseansicht benötigt
\ifkorrekturansicht\else
\newenvironment{esempio}[3]%
{
    \vspace{1.5ex}
    \rlap{\underline{#1}}
    \par
    \setlength{\parindent}{0cm}
    \nopagebreak
    \leftskip=#2cm
    \rightskip=#3cm
}
{
    \par
}
\fi

\doendnotes{C}
\bigskip
\vfill

\clearpage

\footnotesize

\ifkorrekturansicht
  \lohead{\textsc{register}}
\fi

% theindex-Environment neu definieren ohne reledmac
\makeatletter
\renewenvironment{theindex}{%
  \ifkorrekturansicht
    \section*{\indexname}%
  \else
    \subsubsection*{Index der erwähnten Entitäten}%
  \fi
  \setlength{\parindent}{0pt}%
  \setlength{\parskip}{0pt plus 0.3pt}%
  \let\item\@idxitem
}{%
  \ifkorrekturansicht\clearpage\fi
}
\makeatother

\IfFileExists{\jobname-pw.ind}{\input{\jobname-pw.ind}}{}

% Quellenangabe nur in der Leseansicht
\ifkorrekturansicht\else
% Fallback-Definitionen, falls die .tex-Datei \titel etc. nicht gesetzt hat
\providecommand{\titel}{}
\providecommand{\editorInnen}{}
\providecommand{\dateiname}{\jobname}

\vspace{3cm}

\vfill

\footnotesize
\textsc{Quelle}: \titel. Herausgegeben von {\editorInnen}. In: \emph{Arthur Schnitzler: Briefwechsel mit Autorinnen und Autoren}.
 Digitale Edition, https://schnitzler-briefe.acdh.oeaw.ac.at/{\dateiname}.html (Stand \today)
\fi

\end{document}


