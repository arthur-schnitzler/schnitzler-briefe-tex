%% latex-leseansicht-vorspann.tex
%% Vorspann für die Leseansicht.
%% Lädt die gemeinsame Datei latex-vorspann.tex mit nicht gesetztem Schalter.

\newif\ifkorrekturansicht
\korrekturansichtfalse

\input{../tex-inputs/latex-vorspann}

\begin{center}
            \textcolor{red}{ENTWURF, NICHT FERTIG KORRIGIERT}
                      \end{center}
            
         
         \newcommand{\erwaehntePersonen}{Personen: Michael Georg Conrad, Theodore Rottenberg}
         \newcommand{\erwaehnteInstitutionen}{}
         \newcommand{\erwaehnteOrte}{Orte: Berlin, Dessauer Straße, Wien}
         \newcommand{\erwaehnteWerke}{Werke: Arthur Schnitzler [Reigen-Privatdruck], Die Gesellschaft. Monatsschrift für Litteratur, Kunst und Sozialpolitik, Reigen. Zehn Dialoge}
               \section[ Paul Goldmann an Arthur Schnitzler, 27. 4. {[}1900{]}]{ Paul Goldmann an Arthur Schnitzler, 27. 4. {[}1900{]}}\nopagebreak\mylabel{v}\rehead{ }\begin{ledgroupsized}[t]{13cm}\normalsize\beginnumbering \toendnotes[C]{\smallbreak\pagebreak[2]} \Standort{DLA, A:Schnitzler, HS.NZ85.1.3170.}
\physDesc{Brief, 1 Blatt, 2 Seiten
\newline{}Handschrift: blaue Tinte, deutsche Kurrent
\newline{}Schnitzler: mit Bleistift das Jahr »{[}1{]}900« vermerkt }\toendnotes[C]{\smallbreak}\pstart
           \raggedleft{}{\pb}Berlin\oindex{Berlin@\textbf{Berlin}|pw}, 27. April.\pend
           \pstart
           \textcolor{gray}{\textbf{DESSAUERSTRASSE 19}}\oindex{Dessauer Strasse@\textbf{Dessauer Straße}|pw}\pend
           \pstart\center{}Mein lieber Freund,\pend\pstart
           Ich war ſehr erſtaunt, als ich ſah, daß die Sache mit dem »Reigen\pwindex{Schnitzler, Arthur 15.05.1862 – 21.10.1931@\textsc{Schnitzler, Arthur} (15.05.1862 – 21.10.1931), \emph{Schriftsteller, Mediziner}!Reigen. Zehn Dialoge1900@\strich\emph{Reigen. Zehn Dialoge} {[}1900{]}|pw}« in die \label{K_L02913-1v}\edtext{Zeitungen}{\lemma{\textnormal{\emph{Zeitungen}}}\Cendnote{\textnormal{Siehe zum Beispiel M. G. C.\pwindex{Conrad, Michael Georg 05.04.1846 – 20.12.1927@\textsc{Conrad, Michael Georg} (05.04.1846 – 20.12.1927), \emph{Schriftsteller, Kritiker}|pwkv} [=Michael Georg Conrad\pwindex{Conrad, Michael Georg 05.04.1846 – 20.12.1927@\textsc{Conrad, Michael Georg} (05.04.1846 – 20.12.1927), \emph{Schriftsteller, Kritiker}|pwk}]: \emph{Arthur Schnitzler}\pwindex{Arthur Schnitzler [Reigen-Privatdruck]1900@\emph{Arthur Schnitzler [Reigen-Privatdruck]} {[}1900{]}|pwk}. In: \emph{Die Gesellschaft. Halbmonatschrift für Litteratur, Kunst und
                        Sozialpolitik}\pwindex{Gesellschaft. Monatsschrift fuer Litteratur, Kunst und Sozialpolitik1885 – 1902@\emph{Die Gesellschaft. Monatsschrift für Litteratur, Kunst und Sozialpolitik} {[}1885 – 1902{]}|pwk}, Jg. 16, Bd. 3, H. 4, 1900,
                     S. 251.}}}\label{K_L02913-1h} gekommen iſt, und die betreffenden Notizen in den Wien\oindex{Wien@\textbf{Wien}|pw}er Blätern ſind eine Albernheit oder eine
               Perfidie. Gefahr könnte erſt entſtehen, wenn Du von irgendwelchem Lumpenhunde beim
                  \strikeout{der} Staatsanwalt denuncirt würdeſt. Und da man
               immer mit ſolchen Lumpenhunden rechnen muß, und da Vorſicht niemals ſchaden kann, {\pb}möchte ich Dir rathen, einen verläßlichen Advokaten
               zu conſultiren, ob man Dir irgend Etwas anhaben kann. Ich glaube zwar nicht, aber es
               iſt immer gut, für alle Fälle \strikeout{be} vorbereitet zu ſein.
               Du aber mußt dafür ſorgen (und haſt jedenfalls ſchon dafür geſorgt), daß das Buch\pwindex{Schnitzler, Arthur 15.05.1862 – 21.10.1931@\textsc{Schnitzler, Arthur} (15.05.1862 – 21.10.1931), \emph{Schriftsteller, Mediziner}!Reigen. Zehn Dialoge1900@\strich\emph{Reigen. Zehn Dialoge} {[}1900{]}|pwv} nur in die Hände ſicherer
               Leute kommt. Vor allen \strikeout{Di} Dingen nicht in weibliche
               Hände! Was man einer Frau gibt, trägt man auf den offenen Markt. \label{K_L02913-3v}\edtext{Ich weiß ein Lied davon zu ſingen.}{\lemma{\textnormal{\emph{Ich … ſingen.}}}\Cendnote{\textnormal{Bezug auf seine Affäre mit Theodore Rottenberg\pwindex{Rottenberg, Theodore 1875-09-07 – 1945-04-05@\textsc{Rottenberg, Theodore} (1875-09-07 – 1945-04-05)|pwk}, vgl. Paul Goldmann an Arthur Schnitzler, 8. 10. [1899] und Paul Goldmann an Arthur Schnitzler, 11. 10. [1899]}}}\label{K_L02913-3h}\pend
           \pstart
           Viele treue Grüße! {\\[\baselineskip]}Dein {\\[\baselineskip]}\spacefill\mbox{Paul Goldmann.}\pend
           \leftskip=0em{}
         
         \endnumbering\mylabel{h}\end{ledgroupsized}\begin{anhang}\end{anhang}\newcommand{\dateiname}{L02913}\newcommand{\titel}{Paul Goldmann an Arthur Schnitzler, 27. 4. [1900]}\newcommand{\editorInnen}{Martin Anton Müller und Laura Untner}%% latex-leseansicht-abspann.tex
%% Abspann für die Leseansicht.
%% Der Schalter \ifkorrekturansicht ist bereits durch den Vorspann gesetzt.

%% latex-abspann.tex
%% Gemeinsamer Abspann für Korrekturansicht und Leseansicht.
%% Setzt den Schalter \ifkorrekturansicht voraus (gesetzt in den
%% einbindenden Dateien latex-korrekturansicht-abspann.tex bzw.
%% latex-leseansicht-abspann.tex).
%% ---------------------------------------------------------------

\normalsize

% Das esempio-Environment wird nur in der Leseansicht benötigt
\ifkorrekturansicht\else
\newenvironment{esempio}[3]%
{
    \vspace{1.5ex}
    \rlap{\underline{#1}}
    \par
    \setlength{\parindent}{0cm}
    \nopagebreak
    \leftskip=#2cm
    \rightskip=#3cm
}
{
    \par
}
\fi

\doendnotes{C}
\bigskip
\vfill

\clearpage

\footnotesize

\ifkorrekturansicht
  \lohead{\textsc{register}}
\fi

% theindex-Environment neu definieren ohne reledmac
\makeatletter
\renewenvironment{theindex}{%
  \ifkorrekturansicht
    \section*{\indexname}%
  \else
    \subsubsection*{Index der erwähnten Entitäten}%
  \fi
  \setlength{\parindent}{0pt}%
  \setlength{\parskip}{0pt plus 0.3pt}%
  \let\item\@idxitem
}{%
  \ifkorrekturansicht\clearpage\fi
}
\makeatother

\IfFileExists{\jobname-pw.ind}{\input{\jobname-pw.ind}}{}

% Quellenangabe nur in der Leseansicht
\ifkorrekturansicht\else
% Fallback-Definitionen, falls die .tex-Datei \titel etc. nicht gesetzt hat
\providecommand{\titel}{}
\providecommand{\editorInnen}{}
\providecommand{\dateiname}{\jobname}

\vspace{3cm}

\vfill

\footnotesize
\textsc{Quelle}: \titel. Herausgegeben von {\editorInnen}. In: \emph{Arthur Schnitzler: Briefwechsel mit Autorinnen und Autoren}.
 Digitale Edition, https://schnitzler-briefe.acdh.oeaw.ac.at/{\dateiname}.html (Stand \today)
\fi

\end{document}


      