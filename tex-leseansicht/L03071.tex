%% latex-leseansicht-vorspann.tex
%% Vorspann für die Leseansicht.
%% Lädt die gemeinsame Datei latex-vorspann.tex mit nicht gesetztem Schalter.

\newif\ifkorrekturansicht
\korrekturansichtfalse

\input{../tex-inputs/latex-vorspann}


\section[Paul Goldmann an Arthur Schnitzler, Olga und Elisabeth Gussmann, 3. 7. [1901]]{L03071 Paul Goldmann an Arthur Schnitzler, Olga und Elisabeth
               Gussmann,  3. 7. [1901]}
\nopagebreak\mylabel{L03071v}
\rehead{ }\normalsize\beginnumbering\briefempfaengerindex{Steinrück, Elisabeth@\textsc{Steinrück, Elisabeth}!zzzGoldmann, Paul@\emph{von Paul Goldmann}!1901-07-033@{3. 7. [1901]}|(be}\briefempfaengerindex{Schnitzler, Olga@\textsc{Schnitzler, Olga}!zzzGoldmann, Paul@\emph{von Paul Goldmann}!1901-07-033@{3. 7. [1901]}|(be}\briefempfaengerindex{Schnitzler, Arthur@\textsc{Schnitzler, Arthur}!zzzGoldmann, Paul@\emph{von Paul Goldmann}!1901-07-033@{3. 7. [1901]}|(be}
\toendnotes[C]{\smallbreak\pagebreak[2]}
\correspDesc{Versand  durch Paul Goldmann am 3. 7. [1901] in Berlin
\newline{}Erhalt  durch Arthur Schnitzler, Olga Gussmann, Elisabeth Gussmann im Zeitraum [4. 7. 1901
                  – 8. 7. 1901?] in St. Anton am Arlberg}\toendnotes[C]{\smallbreak}
\Standort{DLA, A:Schnitzler, HS.NZ85.1.3171.}
\physDesc{Brief, 2 Blätter, 7 Seiten, 2438 Zeichen
\newline{}Handschrift: blaue Tinte, deutsche Kurrent
\newline{}Schnitzler: mit rotem Buntstift eine Unterstreichung }\toendnotes[C]{\smallbreak}
\pstart
           \raggedleft{}{\pb}\textcolor{gray}{\textbf{DESSAUERSTRASSE 19}}\oindex{Dessauer Straße@\textbf{Dessauer Straße}, \emph{Straße}|pw}\pend
           
\pstart
           Berlin\oindex{Berlin@\textbf{Berlin}, \emph{Hauptstadt}|pw}, 3. Juli.\pend
           
\pstart\center{}Mein lieber Freund,\pend\vspace{0.5em}
\pstart
           Ich habe mich{ }ſehr mit Deinem und der kleinen \textsc{Liesl} Briefe
               gefreut.\pend
           
\pstart
           Du kannſt Dir denken, wie gern ich mit Euch Allen zuſammenſein würde. Aber Du machſt
               es mir gar zu \strikeout{ſch\textcolor{gray}{we}}{ }ſchwer; und wenn Du nach der \label{K_L03071-1v}\edtext{Schweiz\oindex{Schweiz@\textbf{Schweiz}|pw}}{\lemma{\textnormal{\emph{Schweiz}}}\Cendnote{\textnormal{Schnitzler war im Sommer 1901 nicht in der Schweiz\oindex{Schweiz@\textbf{Schweiz}|pwk}. Er und
                     Goldmann\pwindex{Goldmann, Paul 31.\,1.\,1865 Breslau – 25.\,9.\,1935 Wien@\textsc{Goldmann, Paul} (31.\,1.\,1865 Breslau – 25.\,9.\,1935 Wien), \emph{Schriftsteller, Journalist}|pwk} trafen sich trotzdem, vgl. XXXX Auszeichnungsfehler: Dokument L03064 nicht gefunden.}}}\label{K_L03071-1} gehſt, wird es
               ganz unmöglich{ }ſein. Ich bekomme eine Freikarte auf der Südbahn\orgindex{Südbahn-Gesellschaft@Südbahn-Gesellschaft|pw}. Danach muß ich mich richten, bei meinen beſchränkten
               Geldmitteln. Wenn Du alſo mit mir {\pb}zuſammen{ }ſein
               willſt,{ }ſo mußt Du mir entgegenkommen. Das heißt alſo: Gehſt Du nach \introOben{}Kärnthen\oindex{Kärnten@\textbf{Kärnten}, \emph{Land}|pw} oder\introOben{}{ }Tirol\oindex{Tirol@\textbf{Tirol}, \emph{Land}|pw}, nach Südtirol\oindex{Südtirol@\textbf{Südtirol}, \emph{Verwaltungsgebiet}|pw} womöglich,{ }ſo werden wir uns{ }ſehen. Wenn \strikeout{\textcolor{gray}{nic}h\textcolor{gray}{t}} nicht,{ }ſo werde ich diesmal meinen Urlaub in Öſterreich\oindex{Österreich@\textbf{Österreich}|pw} verbringen, ohne Dir die Hand drücken zu können, und das wird{ }ſehr traurig{ }ſein. Im Übrigen denke ich mir: Ihr Zwei{ }ſeid glücklich miteinander.
               Gewiß, ich würde Euch nicht{ }ſtören. Aber{ }ſoll ich mir das anthun, ich Einſamer, {\pb}dem Alles verſagt iſt, in der Nähe eines{ }ſo großen
               Glücks zu leben?\pend
           
\pstart
           Theile mir alſo \introOben{}(und zwar möglichſt raſch)\introOben{}{ }\strikeout{noch N\textcolor{gray}{ä}} Jedenfalls noch Näheres über Deine Reiſepläne mit! \textsc{Kerr\pwindex{Kerr, Alfred 25.\,12.\,1867 Breslau – 12.\,10.\,1948 Hamburg@\textsc{Kerr, Alfred} (25.\,12.\,1867 Breslau – 12.\,10.\,1948 Hamburg), \emph{Schriftsteller, Kritiker}|pw}} möchte auch mit Dir und mir zuſammen{ }ſein. Soll ich ihm{ }ſagen, wo Du biſt? Und
               mit wem? Einſtweilen haben \textsc{Kerr\pwindex{Kerr, Alfred 25.\,12.\,1867 Breslau – 12.\,10.\,1948 Hamburg@\textsc{Kerr, Alfred} (25.\,12.\,1867 Breslau – 12.\,10.\,1948 Hamburg), \emph{Schriftsteller, Kritiker}|pw}} und ich feſtgeſetzt, daß wir uns am Wörtherſee\oindex{Wörthersee@\textbf{Wörthersee}, \emph{See}|pw} treffen und vielleicht \label{K_L03071-2v}\edtext{zuſammen hingehn}{\lemma{\textnormal{\emph{zusammen hingehn}}}\Cendnote{\textnormal{Dazu kam es nicht.}}}\label{K_L03071-2}\substVorne{}\textsuperscript{?}\substDazwischen{}.\substHinten{}\pend
           
\pstart
           {\pb}Sei vielmals und von Herzen gegrüßt von {\\[\baselineskip]}Deinem {\\[\baselineskip]}\spacefill\mbox{Paul Goldmnn}\pend
           \leftskip=0em{}\selectlanguage{ngerman}\vspace{1em}{\vspace{1\baselineskip}}
\pstart
           Liebes Fräulein \textsc{Olga}, Ich
               danke für Ihre lieben Zeilen und freue mich auf Ihren Brief. Könnten Sie nicht den
                  \textsc{Arthur} beſtimmen, daß er nach Tirol\oindex{Tirol@\textbf{Tirol}, \emph{Land}|pw} oder Kärnthen\oindex{Kärnten@\textbf{Kärnten}, \emph{Land}|pw} geht{ }ſtatt nach der Schweiz\oindex{Schweiz@\textbf{Schweiz}|pw}? Nach mir richtet er{ }ſich nicht; das weiß ich aus Erfahrung. Aber wenn Sie es verlangen, richtet er{ }ſich
               vielleicht nach Ihnen. Das Ganze kann ja ein Geheimniß bleiben zwiſchen uns \strikeout{Beiden.} Beiden.\pend
           
\pstart
           Herzlichſt Ihr {\\[\baselineskip]}\spacefill\mbox{Dr. Paul Goldmann.}\pend
           \leftskip=0em{}\selectlanguage{ngerman}\vspace{1em}{\vspace{1\baselineskip}}
\pstart
           {\pb}Liebes Fräulein Liesl,\pend
           
\pstart
           Mir fällt ein, daß ich Ihnen auch gleich antworten möchte. Ich danke Ihnen für Ihr
               liebes Briefchen, und es thut mir unendlich leid, daß Sie{ }ſoviel \label{K_L03071-3v}\edtext{Kummer}{\lemma{\textnormal{\emph{Kummer}}}\Cendnote{\textnormal{Elisabeth Gussmann\pwindex{Steinrück, Elisabeth 19.\,11.\,1885 – 7.\,4.\,1920 Partenkirchen@\textsc{Steinrück, Elisabeth} (19.\,11.\,1885 – 7.\,4.\,1920 Partenkirchen)|pwk} dürfte erkrankt sein,
                     vgl. XXXX Auszeichnungsfehler: Dokument L03072 nicht gefunden.}}}\label{K_L03071-3} gehabt
               haben. Aber warten Sie nur, es wird{ }ſchon beſſer kommen. Ich möchte Sie gern
               wiederſehen und ein Bischen mit Ihnen plaudern und Sie quietſchen hören (quietſchen
               Sie noch{ }ſo gut?). Aber dieſer {\pb}Schurke, der \textsc{Arthur} (bitte, \strikeout{S\textcolor{gray}{a}}{ }ſagen Sie es ihm \strikeout{\textcolor{gray}{×}\-\textcolor{gray}{×}\-\textcolor{gray}{×}} nicht, daß ich ihn Schurke genannt habe) will nach der Schweiz\oindex{Schweiz@\textbf{Schweiz}|pw} gehen. So macht er es mir unmöglich, mit Ihnen
               zuſammenzukommen. Ich glaube, er thut es abſichtlich. Er will beide Schweſtern ganz
               für{ }ſich haben und gönnt{ }ſie Keinem. Er war immer{ }ſo ein Intriguant.\pend
           
\pstart
           {\pb}Bitte,{ }ſchreiben Sie mir bald wieder, und{ }ſeien Sie
               herzlichſt gegrüßt von {\\[\baselineskip]}Ihrem {\\[\baselineskip]}\spacefill\mbox{Dr. Paul Goldmann.}\pend
           \leftskip=0em{}\selectlanguage{ngerman}\endnumbering\briefempfaengerindex{Steinrück, Elisabeth@\textsc{Steinrück, Elisabeth}!zzzGoldmann, Paul@\emph{von Paul Goldmann}!1901-07-033@{3. 7. [1901]}|)be}\briefempfaengerindex{Schnitzler, Olga@\textsc{Schnitzler, Olga}!zzzGoldmann, Paul@\emph{von Paul Goldmann}!1901-07-033@{3. 7. [1901]}|)be}\briefempfaengerindex{Schnitzler, Arthur@\textsc{Schnitzler, Arthur}!zzzGoldmann, Paul@\emph{von Paul Goldmann}!1901-07-033@{3. 7. [1901]}|)be}\mylabel{L03071h}  \newcommand{\dateiname}{L03071}\newcommand{\titel}{Paul Goldmann an Arthur Schnitzler, Olga und Elisabeth Gussmann, 3. 7. [1901]}\newcommand{\editorInnen}{Martin Anton Müller und Laura Untner}%% latex-leseansicht-abspann.tex
%% Abspann für die Leseansicht.
%% Der Schalter \ifkorrekturansicht ist bereits durch den Vorspann gesetzt.

%% latex-abspann.tex
%% Gemeinsamer Abspann für Korrekturansicht und Leseansicht.
%% Setzt den Schalter \ifkorrekturansicht voraus (gesetzt in den
%% einbindenden Dateien latex-korrekturansicht-abspann.tex bzw.
%% latex-leseansicht-abspann.tex).
%% ---------------------------------------------------------------

\normalsize

% Das esempio-Environment wird nur in der Leseansicht benötigt
\ifkorrekturansicht\else
\newenvironment{esempio}[3]%
{
    \vspace{1.5ex}
    \rlap{\underline{#1}}
    \par
    \setlength{\parindent}{0cm}
    \nopagebreak
    \leftskip=#2cm
    \rightskip=#3cm
}
{
    \par
}
\fi

\doendnotes{C}
\bigskip
\vfill

\clearpage

\footnotesize

\ifkorrekturansicht
  \lohead{\textsc{register}}
\fi

% theindex-Environment neu definieren ohne reledmac
\makeatletter
\renewenvironment{theindex}{%
  \ifkorrekturansicht
    \section*{\indexname}%
  \else
    \subsubsection*{Index der erwähnten Entitäten}%
  \fi
  \setlength{\parindent}{0pt}%
  \setlength{\parskip}{0pt plus 0.3pt}%
  \let\item\@idxitem
}{%
  \ifkorrekturansicht\clearpage\fi
}
\makeatother

\IfFileExists{\jobname-pw.ind}{\input{\jobname-pw.ind}}{}

% Quellenangabe nur in der Leseansicht
\ifkorrekturansicht\else
% Fallback-Definitionen, falls die .tex-Datei \titel etc. nicht gesetzt hat
\providecommand{\titel}{}
\providecommand{\editorInnen}{}
\providecommand{\dateiname}{\jobname}

\vspace{3cm}

\vfill

\footnotesize
\textsc{Quelle}: \titel. Herausgegeben von {\editorInnen}. In: \emph{Arthur Schnitzler: Briefwechsel mit Autorinnen und Autoren}.
 Digitale Edition, https://schnitzler-briefe.acdh.oeaw.ac.at/{\dateiname}.html (Stand \today)
\fi

\end{document}


