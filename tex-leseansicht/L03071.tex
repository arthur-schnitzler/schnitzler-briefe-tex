%% latex-leseansicht-vorspann.tex
%% Vorspann für die Leseansicht.
%% Lädt die gemeinsame Datei latex-vorspann.tex mit nicht gesetztem Schalter.

\newif\ifkorrekturansicht
\korrekturansichtfalse

\input{../tex-inputs/latex-vorspann}


         
         \renewcommand{\erwaehntePersonen}{Personen: Alfred Kerr, Olga Schnitzler, Elisabeth Steinrück}
         \renewcommand{\erwaehnteInstitutionen}{Institutionen: Südbahn-Gesellschaft}
         \renewcommand{\erwaehnteOrte}{Orte: Berlin, Dessauer Straße, Kärnten, Schweiz, St. Anton am Arlberg, Südtirol, Tirol, Wörthersee, Österreich}
         \renewcommand{\erwaehnteWerke}{}
               \section[Paul Goldmann an Arthur Schnitzler, Olga und Elisabeth Gussmann, 3. 7. {[}1901{]}]{ Paul Goldmann an Arthur Schnitzler, Olga und Elisabeth
               Gussmann, 3. 7. {[}1901{]}}\nopagebreak\mylabel{v}\rehead{ }\begin{ledgroupsized}[t]{13cm}\normalsize\beginnumbering \toendnotes[C]{\smallbreak\pagebreak[2]} \Standort{DLA, A:Schnitzler, HS.NZ85.1.3171.}
\physDesc{Brief, 2 Blätter, 7 Seiten, 2438 Zeichen
\newline{}Handschrift: blaue Tinte, deutsche Kurrent
\newline{}Schnitzler: mit rotem Buntstift eine Unterstreichung }\toendnotes[C]{\smallbreak}\pstart
           \noindent{}\raggedleft{}{\pb}\textcolor{gray}{\textbf{DESSAUERSTRASSE 19}}\oindex{Dessauer Strasse@\textbf{Dessauer Straße}|pw}\pend
           \pstart
           Berlin\oindex{Berlin@\textbf{Berlin}|pw}, 3. Juli.\pend
           \pstart\center{}Mein lieber Freund,\pend\pstart
           Ich habe mich ſehr mit Deinem und der kleinen \textsc{Liesl} Briefe
               gefreut.\pend
           \pstart
           Du kannſt Dir denken, wie gern ich mit Euch Allen zuſammenſein würde. Aber Du machſt
               es mir gar zu \strikeout{ſch\textcolor{gray}{we}} ſchwer; und wenn Du nach der \label{K_L03071-1v}\edtext{Schweiz\oindex{Schweiz@\textbf{Schweiz}|pw}}{\lemma{\textnormal{\emph{Schweiz}}}\Cendnote{\textnormal{Schnitzler\pwindex{Schnitzler, Arthur 15.05.1862 – 21.10.1931@\textsc{Schnitzler, Arthur} (15.05.1862 – 21.10.1931), \emph{Schriftsteller, Mediziner}|pwk} war im Sommer 1901 nicht in der Schweiz\oindex{Schweiz@\textbf{Schweiz}|pwk}. Er und
                     Goldmann\pwindex{Goldmann, Paul 31.01.1865 – 25.09.1935@\textsc{Goldmann, Paul} (31.01.1865 – 25.09.1935), \emph{Schriftsteller, Journalist}|pwk} trafen sich trotzdem, vgl. Paul Goldmann an Arthur Schnitzler, 26. 4. [1901].}}}\label{K_L03071-1h} gehſt, wird es
               ganz unmöglich ſein. Ich bekomme eine Freikarte auf der Südbahn\orgindex{Suedbahn-Gesellschaft@Südbahn-Gesellschaft|pw}. Danach muß ich mich richten, bei meinen beſchränkten
               Geldmitteln. Wenn Du alſo mit mir {\pb}zuſammen ſein
               willſt, ſo mußt Du mir entgegenkommen. Das heißt alſo: Gehſt Du nach \introOben{}Kärnthen\oindex{Kaernten@\textbf{Kärnten}|pw} oder\introOben{}{ }Tirol\oindex{Tirol@\textbf{Tirol}|pw}, nach Südtirol\oindex{Suedtirol@\textbf{Südtirol}|pw} womöglich, ſo werden wir uns ſehen. Wenn \strikeout{\textcolor{gray}{nic}h\textcolor{gray}{t}} nicht, ſo werde ich diesmal meinen Urlaub in Öſterreich\oindex{Oesterreich@\textbf{Österreich}|pw} verbringen, ohne Dir die Hand drücken zu können, und das wird
               ſehr traurig ſein. Im Übrigen denke ich mir: Ihr Zwei ſeid glücklich miteinander.
               Gewiß, ich würde Euch nicht ſtören. Aber ſoll ich mir das anthun, ich Einſamer, {\pb}dem Alles verſagt iſt, in der Nähe eines ſo großen
               Glücks zu leben?\pend
           \pstart
           Theile mir alſo \introOben{}(und zwar möglichſt raſch)\introOben{}{ }\strikeout{noch N\textcolor{gray}{ä}} Jedenfalls noch Näheres über Deine Reiſepläne mit! \textsc{Kerr\pwindex{Kerr, Alfred 25.12.1867 – 12.10.1948@\textsc{Kerr, Alfred} (25.12.1867 – 12.10.1948), \emph{Schriftsteller, Kritiker}|pw}} möchte auch mit Dir und mir zuſammen ſein. Soll ich ihm ſagen, wo Du biſt? Und
               mit wem? Einſtweilen haben \textsc{Kerr\pwindex{Kerr, Alfred 25.12.1867 – 12.10.1948@\textsc{Kerr, Alfred} (25.12.1867 – 12.10.1948), \emph{Schriftsteller, Kritiker}|pw}} und ich feſtgeſetzt, daß wir uns am Wörtherſee\oindex{Woerthersee@\textbf{Wörthersee}|pw} treffen und vielleicht \label{K_L03071-2v}\edtext{zuſammen hingehn}{\lemma{\textnormal{\emph{zuſammen hingehn}}}\Cendnote{\textnormal{nicht geschehen}}}\label{K_L03071-2h}\substVorne{}\textsuperscript{?}\substDazwischen{}.\substHinten{}\pend
           \pstart
           {\pb}Sei vielmals und von Herzen gegrüßt von {\\[\baselineskip]}Deinem {\\[\baselineskip]}\spacefill\mbox{Paul Goldmnn}\pend
           \leftskip=0em{}{\bigskip}\pstart
           \noindent{}Liebes Fräulein \textsc{Olga}, Ich
               danke für Ihre lieben Zeilen und freue mich auf Ihren Brief. Könnten Sie nicht den
                  \textsc{Arthur} beſtimmen, daß er nach Tirol\oindex{Tirol@\textbf{Tirol}|pw} oder Kärnthen\oindex{Kaernten@\textbf{Kärnten}|pw} geht
               ſtatt nach der Schweiz\oindex{Schweiz@\textbf{Schweiz}|pw}? Nach mir richtet er
               ſich nicht; das weiß ich aus Erfahrung. Aber wenn Sie es verlangen, richtet er ſich
               vielleicht nach Ihnen. Das Ganze kann ja ein Geheimniß bleiben zwiſchen uns \strikeout{Beiden.} Beiden.\pend
           \pstart
           Herzlichſt Ihr {\\[\baselineskip]}\spacefill\mbox{Dr. Paul Goldmann.}\pend
           \leftskip=0em{}{\bigskip}\pstart
           \noindent{}{\pb}Liebes Fräulein Liesl,\pend
           \pstart
           Mir fällt ein, daß ich Ihnen auch gleich antworten möchte. Ich danke Ihnen für Ihr
               liebes Briefchen, und es thut mir unendlich leid, daß Sie ſoviel \label{K_L03071-3v}\edtext{Kummer}{\lemma{\textnormal{\emph{Kummer}}}\Cendnote{\textnormal{Elisabeth Gussmann\pwindex{Steinrueck, Elisabeth 19.11.1885 – 07.04.1920@\textsc{Steinrück, Elisabeth} (19.11.1885 – 07.04.1920)|pwk} dürfte erkrankt sein,
                     vgl. Paul Goldmann an Arthur Schnitzler und Olga
               Gussmann, 7. 7. [1901].}}}\label{K_L03071-3h} gehabt
               haben. Aber warten Sie nur, es wird ſchon beſſer kommen. Ich möchte Sie gern
               wiederſehen und ein Bischen mit Ihnen plaudern und Sie quietſchen hören (quietſchen
               Sie noch ſo gut?). Aber dieſer {\pb}Schurke, der \textsc{Arthur} (bitte, \strikeout{S\textcolor{gray}{a}} ſagen Sie es ihm \strikeout{\textcolor{gray}{×}\-\textcolor{gray}{×}\-\textcolor{gray}{×}} nicht, daß ich ihn Schurke genannt habe) will nach der Schweiz\oindex{Schweiz@\textbf{Schweiz}|pw} gehen. So macht er es mir unmöglich, mit Ihnen
               zuſammenzukommen. Ich glaube, er thut es abſichtlich. Er will beide Schweſtern ganz
               für ſich haben und gönnt ſie Keinem. Er war immer ſo ein Intriguant.\pend
           \pstart
           {\pb}Bitte, ſchreiben Sie mir bald wieder, und ſeien Sie
               herzlichſt gegrüßt von {\\[\baselineskip]}Ihrem {\\[\baselineskip]}\spacefill\mbox{Dr. Paul Goldmann.}\pend
           \leftskip=0em{}
         
         \endnumbering\mylabel{h}\end{ledgroupsized}  \newcommand{\dateiname}{L03071}\newcommand{\titel}{Paul Goldmann an Arthur Schnitzler, Olga und Elisabeth Gussmann, 3. 7. [1901]}\newcommand{\editorInnen}{Martin Anton Müller und Laura Untner}%% latex-leseansicht-abspann.tex
%% Abspann für die Leseansicht.
%% Der Schalter \ifkorrekturansicht ist bereits durch den Vorspann gesetzt.

%% latex-abspann.tex
%% Gemeinsamer Abspann für Korrekturansicht und Leseansicht.
%% Setzt den Schalter \ifkorrekturansicht voraus (gesetzt in den
%% einbindenden Dateien latex-korrekturansicht-abspann.tex bzw.
%% latex-leseansicht-abspann.tex).
%% ---------------------------------------------------------------

\normalsize

% Das esempio-Environment wird nur in der Leseansicht benötigt
\ifkorrekturansicht\else
\newenvironment{esempio}[3]%
{
    \vspace{1.5ex}
    \rlap{\underline{#1}}
    \par
    \setlength{\parindent}{0cm}
    \nopagebreak
    \leftskip=#2cm
    \rightskip=#3cm
}
{
    \par
}
\fi

\doendnotes{C}
\bigskip
\vfill

\clearpage

\footnotesize

\ifkorrekturansicht
  \lohead{\textsc{register}}
\fi

% theindex-Environment neu definieren ohne reledmac
\makeatletter
\renewenvironment{theindex}{%
  \ifkorrekturansicht
    \section*{\indexname}%
  \else
    \subsubsection*{Index der erwähnten Entitäten}%
  \fi
  \setlength{\parindent}{0pt}%
  \setlength{\parskip}{0pt plus 0.3pt}%
  \let\item\@idxitem
}{%
  \ifkorrekturansicht\clearpage\fi
}
\makeatother

\IfFileExists{\jobname-pw.ind}{\input{\jobname-pw.ind}}{}

% Quellenangabe nur in der Leseansicht
\ifkorrekturansicht\else
% Fallback-Definitionen, falls die .tex-Datei \titel etc. nicht gesetzt hat
\providecommand{\titel}{}
\providecommand{\editorInnen}{}
\providecommand{\dateiname}{\jobname}

\vspace{3cm}

\vfill

\footnotesize
\textsc{Quelle}: \titel. Herausgegeben von {\editorInnen}. In: \emph{Arthur Schnitzler: Briefwechsel mit Autorinnen und Autoren}.
 Digitale Edition, https://schnitzler-briefe.acdh.oeaw.ac.at/{\dateiname}.html (Stand \today)
\fi

\end{document}


      