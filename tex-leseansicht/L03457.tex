%% latex-leseansicht-vorspann.tex
%% Vorspann für die Leseansicht.
%% Lädt die gemeinsame Datei latex-vorspann.tex mit nicht gesetztem Schalter.

\newif\ifkorrekturansicht
\korrekturansichtfalse

\input{../tex-inputs/latex-vorspann}


\section[ Paul Goldmann und Josef Rosengart an Arthur Schnitzler, 10. 9. 1904]{L03457 Paul Goldmann und Josef Rosengart an Arthur
               Schnitzler,  10. 9. 1904}
\nopagebreak\mylabel{L03457v}
\rehead{ }\normalsize\beginnumbering\briefempfaengerindex{Schnitzler, Arthur@\textsc{Schnitzler, Arthur}!zzzRosengart, Josef@\emph{von Josef Rosengart}!1904-09-102@{10. 9. 1904}|(be}\briefempfaengerindex{Schnitzler, Arthur@\textsc{Schnitzler, Arthur}!zzzGoldmann, Paul@\emph{von Paul Goldmann}!1904-09-102@{10. 9. 1904}|(be}
\toendnotes[C]{\smallbreak\pagebreak[2]}
\correspDesc{Versand  durch Paul Goldmann, Josef Rosengart am 10. 9. 1904 in Como
\newline{}Übermittlung  am 11. 9. 1904 in Cernobbio
\newline{}Erhalt  durch Arthur Schnitzler im Zeitraum [13. 9. 1904
                  – 16. 9. 1904?] in Pass Lueg}\toendnotes[C]{\smallbreak}
\Standort{DLA, A:Schnitzler, HS.NZ85.1.3174.}
\physDesc{Bildpostkarte, 113 Zeichen
\newline{}Handschrift Paul Goldmann: Bleistift, deutsche Kurrent
\newline{}Handschrift Josef Rosengart: Bleistift
\newline{}Versand: Stempel: »\nobreak{}\oindex{Cernobbio@\textbf{Cernobbio}, \emph{Verwaltungsgebiet}|pwk}Cernobbio (Como), 11 9 04\nobreak{}«.  
\newline{}Schnitzler: mit Bleistift das Jahr »904« vermerkt }\pstart{}\textsc{{\pb}Herrn}\pend{}\pstart{}\textsc{Dr. Arthur Schnitzler}\pend{}\pstart{}\textsc{XVIII. Spöttelgaſse 7\oindex{Wien@\textbf{Wien}!XVIII., Währing@\textbf{XVIII., Währing}!Edmund-Weiß-Gasse 7@\textbf{Edmund-Weiß-Gasse 7}, \emph{Wohngebäude}|pw}}\pend{}\pstart{}\textsc{Wien\oindex{Wien@\textbf{Wien}, \emph{Verwaltungsgebiet}|pw}}\pend{}\pstart{}\textcolor{gray}{\textbf{(}}Austria\oindex{Österreich@\textbf{Österreich}|pw}\textcolor{gray}{\textbf{)}}\pend{}{\bigskip}
\pstart
           {\pb}\textcolor{gray}{\textbf{Como\oindex{Como@\textbf{Como}, \emph{Hauptstadt}|pw}.}}\hfill \textcolor{gray}{\textbf{\begin{otherlanguage}{italian}Il Duomo\oindex{Dom zu Como@\textbf{Dom zu Como}, \emph{Kirche}|pw}\end{otherlanguage}.}}\pend
           \vspace{1em}
\pstart
           {\pb}Como\oindex{Como@\textbf{Como}, \emph{Hauptstadt}|pw}{ }10. Sept.\pend
           \vspace{0.5em}
\pstart
           Herzlichen Gruß!\pend
           
\pstart
           \spacefill\mbox{Paul Goldmann.}{\\[\baselineskip]}{[}hs. Rosengart:{]} \spacefill\mbox{\textcolor{gray}{D\textsuperscript{r}R}osengart}\pend
           \leftskip=0em{}\selectlanguage{ngerman}\endnumbering\briefempfaengerindex{Schnitzler, Arthur@\textsc{Schnitzler, Arthur}!zzzRosengart, Josef@\emph{von Josef Rosengart}!1904-09-102@{10. 9. 1904}|)be}\briefempfaengerindex{Schnitzler, Arthur@\textsc{Schnitzler, Arthur}!zzzGoldmann, Paul@\emph{von Paul Goldmann}!1904-09-102@{10. 9. 1904}|)be}\mylabel{L03457h}  \newcommand{\dateiname}{L03457}\newcommand{\titel}{Paul Goldmann und Josef Rosengart an Arthur Schnitzler, 10. 9. 1904}\newcommand{\editorInnen}{Martin Anton Müller und Laura Untner}%% latex-leseansicht-abspann.tex
%% Abspann für die Leseansicht.
%% Der Schalter \ifkorrekturansicht ist bereits durch den Vorspann gesetzt.

%% latex-abspann.tex
%% Gemeinsamer Abspann für Korrekturansicht und Leseansicht.
%% Setzt den Schalter \ifkorrekturansicht voraus (gesetzt in den
%% einbindenden Dateien latex-korrekturansicht-abspann.tex bzw.
%% latex-leseansicht-abspann.tex).
%% ---------------------------------------------------------------

\normalsize

% Das esempio-Environment wird nur in der Leseansicht benötigt
\ifkorrekturansicht\else
\newenvironment{esempio}[3]%
{
    \vspace{1.5ex}
    \rlap{\underline{#1}}
    \par
    \setlength{\parindent}{0cm}
    \nopagebreak
    \leftskip=#2cm
    \rightskip=#3cm
}
{
    \par
}
\fi

\doendnotes{C}
\bigskip
\vfill

\clearpage

\footnotesize

\ifkorrekturansicht
  \lohead{\textsc{register}}
\fi

% theindex-Environment neu definieren ohne reledmac
\makeatletter
\renewenvironment{theindex}{%
  \ifkorrekturansicht
    \section*{\indexname}%
  \else
    \subsubsection*{Index der erwähnten Entitäten}%
  \fi
  \setlength{\parindent}{0pt}%
  \setlength{\parskip}{0pt plus 0.3pt}%
  \let\item\@idxitem
}{%
  \ifkorrekturansicht\clearpage\fi
}
\makeatother

\IfFileExists{\jobname-pw.ind}{\input{\jobname-pw.ind}}{}

% Quellenangabe nur in der Leseansicht
\ifkorrekturansicht\else
% Fallback-Definitionen, falls die .tex-Datei \titel etc. nicht gesetzt hat
\providecommand{\titel}{}
\providecommand{\editorInnen}{}
\providecommand{\dateiname}{\jobname}

\vspace{3cm}

\vfill

\footnotesize
\textsc{Quelle}: \titel. Herausgegeben von {\editorInnen}. In: \emph{Arthur Schnitzler: Briefwechsel mit Autorinnen und Autoren}.
 Digitale Edition, https://schnitzler-briefe.acdh.oeaw.ac.at/{\dateiname}.html (Stand \today)
\fi

\end{document}


