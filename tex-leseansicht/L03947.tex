%% latex-leseansicht-vorspann.tex
%% Vorspann für die Leseansicht.
%% Lädt die gemeinsame Datei latex-vorspann.tex mit nicht gesetztem Schalter.

\newif\ifkorrekturansicht
\korrekturansichtfalse

\input{../tex-inputs/latex-vorspann}


\section[Arthur Schnitzler an Berta Zuckerkandl, 25. 10. 1923]{L03947 Arthur Schnitzler an Berta Zuckerkandl, 25. 10. 1923}
\nopagebreak\mylabel{L03947v}
\rehead{ }\normalsize\beginnumbering\briefempfaengerindex{Zuckerkandl, Berta@\textsc{Zuckerkandl, Berta}!zzzSchnitzler, Arthur@\emph{von Arthur Schnitzler}!1923-10-251@{25. 10. 1923}|(be}
\toendnotes[C]{\smallbreak\pagebreak[2]}
\correspDesc{Versand  durch Arthur Schnitzler am 25. 10. 1923 in Wien
\newline{}Erhalt  durch Berta Zuckerkandl im Zeitraum [25. 10. 1923 – 28. 10. 1923?] in Wien}\toendnotes[C]{\smallbreak}
\Standort{DLA, HS.1985.1.2282.}
\physDesc{Brief, Durchschlag, 1 Blatt, 2 Seiten, 3174 Zeichen
\newline{}Schreibmaschine
\newline{}Handschrift: roter Buntstift, lateinische Kurrent (\noindent{}beschriftet: »\uline{Zuckerkandl}«, sechs Unterstreichungen)}\toendnotes[C]{\smallbreak}
\pstart
           \raggedleft{}{\pb}25. 10. 1923.\pend
           
\pstart{}Liebe und verehrte Frau Hofrätin.\pend\vspace{0.5em}
\pstart
           Die Absicht des Verlags Stock\orgindex{Éditions Stock@Éditions Stock|pw} zwei Bände meiner
               Werke herauszugeben, ist mir natürlich sehr erfreulich. Sprechen wir zuerst von dem
               Dramenband. Am meisten würde sich empfehlen eine Zusammenstellung der beiden Zyklen
                  »Komödie der Worte\pwindex{Schnitzler, Arthur 15. 5. 1862 Wien – 21. 10. 1931 ebd.@\textsc{Schnitzler, Arthur} (15. 5. 1862 Wien – 21. 10. 1931 ebd.), \emph{Schriftsteller, Mediziner}!Komödie der Worte. Drei Einakter@\strich\emph{Komödie der Worte. Drei Einakter}|pw}« (also »Stunde des Erkennens\pwindex{Schnitzler, Arthur 15. 5. 1862 Wien – 21. 10. 1931 ebd.@\textsc{Schnitzler, Arthur} (15. 5. 1862 Wien – 21. 10. 1931 ebd.), \emph{Schriftsteller, Mediziner}!Stunde des Erkennens@\strich\emph{Stunde des Erkennens}|pw}«, »Grosse Szene\pwindex{Schnitzler, Arthur 15. 5. 1862 Wien – 21. 10. 1931 ebd.@\textsc{Schnitzler, Arthur} (15. 5. 1862 Wien – 21. 10. 1931 ebd.), \emph{Schriftsteller, Mediziner}!Große Szene@\strich\emph{Große Szene}|pw}« und »Bachusfest\pwindex{Schnitzler, Arthur 15. 5. 1862 Wien – 21. 10. 1931 ebd.@\textsc{Schnitzler, Arthur} (15. 5. 1862 Wien – 21. 10. 1931 ebd.), \emph{Schriftsteller, Mediziner}!Bacchusfest@\strich\emph{Das Bacchusfest}|pw}«) und
                  „Lebendige Stunden\pwindex{Schnitzler, Arthur 15. 5. 1862 Wien – 21. 10. 1931 ebd.@\textsc{Schnitzler, Arthur} (15. 5. 1862 Wien – 21. 10. 1931 ebd.), \emph{Schriftsteller, Mediziner}!Lebendige Stunden. Vier Einakter@\strich\emph{Lebendige Stunden. Vier Einakter}|pw}" (also »Lebendige Stunden\pwindex{Schnitzler, Arthur 15. 5. 1862 Wien – 21. 10. 1931 ebd.@\textsc{Schnitzler, Arthur} (15. 5. 1862 Wien – 21. 10. 1931 ebd.), \emph{Schriftsteller, Mediziner}!Lebendige Stunden@\strich\emph{Lebendige Stunden}|pw}«, »Frau mit
                  dem Dolch\pwindex{Schnitzler, Arthur 15. 5. 1862 Wien – 21. 10. 1931 ebd.@\textsc{Schnitzler, Arthur} (15. 5. 1862 Wien – 21. 10. 1931 ebd.), \emph{Schriftsteller, Mediziner}!Frau mit dem Dolche@\strich\emph{Die Frau mit dem Dolche}|pw}«, »Letzte Masken\pwindex{Schnitzler, Arthur 15. 5. 1862 Wien – 21. 10. 1931 ebd.@\textsc{Schnitzler, Arthur} (15. 5. 1862 Wien – 21. 10. 1931 ebd.), \emph{Schriftsteller, Mediziner}!letzten Masken@\strich\emph{Die letzten Masken}|pw}«, »Literatur\pwindex{Schnitzler, Arthur 15. 5. 1862 Wien – 21. 10. 1931 ebd.@\textsc{Schnitzler, Arthur} (15. 5. 1862 Wien – 21. 10. 1931 ebd.), \emph{Schriftsteller, Mediziner}!Literatur@\strich\emph{Literatur}|pw}«). Dazu eventuell noch einige andere
               Einakter: »Puppenspieler\pwindex{Schnitzler, Arthur 15. 5. 1862 Wien – 21. 10. 1931 ebd.@\textsc{Schnitzler, Arthur} (15. 5. 1862 Wien – 21. 10. 1931 ebd.), \emph{Schriftsteller, Mediziner}!Puppenspieler. Studie in einem Aufzuge@\strich\emph{Der Puppenspieler. Studie in einem Aufzuge}|pw}«, »Der grüne Kakadu\pwindex{Schnitzler, Arthur 15. 5. 1862 Wien – 21. 10. 1931 ebd.@\textsc{Schnitzler, Arthur} (15. 5. 1862 Wien – 21. 10. 1931 ebd.), \emph{Schriftsteller, Mediziner}!grüne Kakadu. Groteske in einem Akt@\strich\emph{Der grüne Kakadu. Groteske in einem Akt}|pw}«, »Die
                  Gefährtin\pwindex{Schnitzler, Arthur 15. 5. 1862 Wien – 21. 10. 1931 ebd.@\textsc{Schnitzler, Arthur} (15. 5. 1862 Wien – 21. 10. 1931 ebd.), \emph{Schriftsteller, Mediziner}!Gefährtin. Schauspiel in einem Akt@\strich\emph{Die Gefährtin. Schauspiel in einem Akt}|pw}«. »Der grosse Wurstl\pwindex{Schnitzler, Arthur 15. 5. 1862 Wien – 21. 10. 1931 ebd.@\textsc{Schnitzler, Arthur} (15. 5. 1862 Wien – 21. 10. 1931 ebd.), \emph{Schriftsteller, Mediziner}!Zum großen Wurstel. Burleske in einem Akt@\strich\emph{Zum großen Wurstel. Burleske in einem Akt}|pw}« käme
               meiner Ansicht nach weniger in Betracht, wenigstens für den Anfang.\pend
           
\pstart
           Zwei Fragen sind vor allem zu erwägen: Autorisation und Uebersetzung. »Komödie der Worte\pwindex{Schnitzler, Arthur 15. 5. 1862 Wien – 21. 10. 1931 ebd.@\textsc{Schnitzler, Arthur} (15. 5. 1862 Wien – 21. 10. 1931 ebd.), \emph{Schriftsteller, Mediziner}!Komödie der Worte. Drei Einakter@\strich\emph{Komödie der Worte. Drei Einakter}|pw}«, »Puppenspieler\pwindex{Schnitzler, Arthur 15. 5. 1862 Wien – 21. 10. 1931 ebd.@\textsc{Schnitzler, Arthur} (15. 5. 1862 Wien – 21. 10. 1931 ebd.), \emph{Schriftsteller, Mediziner}!Puppenspieler. Studie in einem Aufzuge@\strich\emph{Der Puppenspieler. Studie in einem Aufzuge}|pw}« sind ganz frei. »Die lebendigen Stunden\pwindex{Schnitzler, Arthur 15. 5. 1862 Wien – 21. 10. 1931 ebd.@\textsc{Schnitzler, Arthur} (15. 5. 1862 Wien – 21. 10. 1931 ebd.), \emph{Schriftsteller, Mediziner}!Lebendige Stunden. Vier Einakter@\strich\emph{Lebendige Stunden. Vier Einakter}|pw}« sind seinerzeit von Rémon\pwindex{Rémon, Maurice 27.\,11.\,1861 Paris – 20.\,6.\,1945 Mérignac@\textsc{Rémon, Maurice} (27.\,11.\,1861 Paris – 20.\,6.\,1945 Mérignac), \emph{Übersetzer}|pw} und Mme. Vallentin\pwindex{Valentin, Noémi @\textsc{Valentin, Noémi}, \emph{Übersetzerin}|pw}{ }übersetzt\pwindex{Schnitzler, Arthur 15. 5. 1862 Wien – 21. 10. 1931 ebd.@\textsc{Schnitzler, Arthur} (15. 5. 1862 Wien – 21. 10. 1931 ebd.), \emph{Schriftsteller, Mediziner}!Heures vives@\strich\emph{Heures vives}|pwv} worden, einzelne
                  \label{K_L03947-1v}\edtext{in Zeitschriften erschienen}{\lemma{\textnormal{\emph{in … erschienen}}}\Cendnote{\textnormal{\emph{La femme au poignard}\pwindex{Schnitzler, Arthur 15. 5. 1862 Wien – 21. 10. 1931 ebd.@\textsc{Schnitzler, Arthur} (15. 5. 1862 Wien – 21. 10. 1931 ebd.), \emph{Schriftsteller, Mediziner}!femme au poignard@\strich\emph{La femme au poignard}|pwk}. In: \emph{Revue de Paris}\pwindex{Revue de Paris@\emph{La Revue de Paris}|pwk}, Jg. 19, Reihe 3
                     (Mai–Juni), 15. 5. 1912, S. 225–238, \emph{Les Derniers masques. Comédie en un act}\pwindex{Schnitzler, Arthur 15. 5. 1862 Wien – 21. 10. 1931 ebd.@\textsc{Schnitzler, Arthur} (15. 5. 1862 Wien – 21. 10. 1931 ebd.), \emph{Schriftsteller, Mediziner}!Derniers masques. Comédie en un act@\strich\emph{Les Derniers masques. Comédie en un act}|pwk}.
                     In: \emph{Revue bleue. La Revue politique et
                        littéraire}\pwindex{Revue politique et littéraire@\emph{La Revue politique et littéraire}|pwk}, Jg. 50, 2. Semester, Nr. 20, 11. 11. 1912,
                     S. 618–622 sowie Nr. 21, 23. 11. 1912, S. 652–657 und \emph{Littérature. Comédie en en act}\pwindex{Schnitzler, Arthur 15. 5. 1862 Wien – 21. 10. 1931 ebd.@\textsc{Schnitzler, Arthur} (15. 5. 1862 Wien – 21. 10. 1931 ebd.), \emph{Schriftsteller, Mediziner}!Littérature. Comédie en en act@\strich\emph{Littérature. Comédie en en act}|pwk}. In: \emph{La Revue bleue. La Revue politique et
                        littéraire}\pwindex{Revue politique et littéraire@\emph{La Revue politique et littéraire}|pwk}, Jahrgang 52, 1. Semester, Nr. 1, 3. 1. 1914,
                     S. 11–16 sowie Nr. 2, S. 44–50.}}}\label{K_L03947-1}; \label{K_L03947-7v}\edtext{aufgeführt wurde nichts}{\lemma{\textnormal{\emph{aufgeführt wurde nichts}}}\Cendnote{\textnormal{Tatsächlich wurde der Einakter \emph{Die letzten Masken}\pwindex{Schnitzler, Arthur 15. 5. 1862 Wien – 21. 10. 1931 ebd.@\textsc{Schnitzler, Arthur} (15. 5. 1862 Wien – 21. 10. 1931 ebd.), \emph{Schriftsteller, Mediziner}!letzten Masken@\strich\emph{Die letzten Masken}|pwk} in der Übersetzung\pwindex{Schnitzler, Arthur 15. 5. 1862 Wien – 21. 10. 1931 ebd.@\textsc{Schnitzler, Arthur} (15. 5. 1862 Wien – 21. 10. 1931 ebd.), \emph{Schriftsteller, Mediziner}!Derniers masques. Comédie en un act@\strich\emph{Les Derniers masques. Comédie en un act}|pwkv} von Rémon\pwindex{Rémon, Maurice 27.\,11.\,1861 Paris – 20.\,6.\,1945 Mérignac@\textsc{Rémon, Maurice} (27.\,11.\,1861 Paris – 20.\,6.\,1945 Mérignac), \emph{Übersetzer}|pwk} unter Regie von Aurélien
                     Lugné-Poe\pwindex{\textcolor{red}{\textsuperscript{XXXX indx1}}|pwk} ab dem 1. 4. 1912XXXX EVENT-Angabe fehlt durch das \emph{Théâtre de
                     l'Oeuvre}XXXX ORGangabe fehlt im Théâtre Antoine\oindex{XXXX Ortsangabe fehlt|pwk}
                  aufgeführt, vgl. Karl Zieger: \emph{Arthur Schnitzler et la
                        France 1894–1938. Enquête sur une réception},
                     Villeneuve d’Ascq: \emph{Presses Universitaires du
                        Septentrion} 2012, S. 104–108.}}}\label{K_L03947-7}. Meine
               Honorare bewegten sich um den Nullpunkt. Ich nehme an, dass ich nach 15–20 Jahren
               doch wohl wieder frei über meine Einakter\pwindex{Schnitzler, Arthur 15. 5. 1862 Wien – 21. 10. 1931 ebd.@\textsc{Schnitzler, Arthur} (15. 5. 1862 Wien – 21. 10. 1931 ebd.), \emph{Schriftsteller, Mediziner}!Lebendige Stunden. Vier Einakter@\strich\emph{Lebendige Stunden. Vier Einakter}|pwv} verfügen kann. Wo sich Mme. Vallentin\pwindex{Valentin, Noémi @\textsc{Valentin, Noémi}, \emph{Übersetzerin}|pw} und M. Rémon\pwindex{Rémon, Maurice 27.\,11.\,1861 Paris – 20.\,6.\,1945 Mérignac@\textsc{Rémon, Maurice} (27.\,11.\,1861 Paris – 20.\,6.\,1945 Mérignac), \emph{Übersetzer}|pw} aufhalten ist
               mir unbekannt, sie haben gewiss kein Recht mehr Ansprüche zu stellen, aber die
               Editeurs Stock\orgindex{Éditions Stock@Éditions Stock|pw} werden ja leicht eruieren können,
               wo sich Herr Maurice Rémon\pwindex{Rémon, Maurice 27.\,11.\,1861 Paris – 20.\,6.\,1945 Mérignac@\textsc{Rémon, Maurice} (27.\,11.\,1861 Paris – 20.\,6.\,1945 Mérignac), \emph{Übersetzer}|pw}{ }heute aufhält. Er ist der Mitübersetzer des »Anatol\pwindex{Schnitzler, Arthur 15. 5. 1862 Wien – 21. 10. 1931 ebd.@\textsc{Schnitzler, Arthur} (15. 5. 1862 Wien – 21. 10. 1931 ebd.), \emph{Schriftsteller, Mediziner}!Anatol@\strich\emph{Anatol}|pw}\textcolor{red}{\textsuperscript{XXXX indx2}}«. Ich glaube aber keineswegs eine alleinige
               Autorisation erteilt zu haben. »Der Grüne
                  Kakadu\pwindex{Schnitzler, Arthur 15. 5. 1862 Wien – 21. 10. 1931 ebd.@\textsc{Schnitzler, Arthur} (15. 5. 1862 Wien – 21. 10. 1931 ebd.), \emph{Schriftsteller, Mediziner}!grüne Kakadu. Groteske in einem Akt@\strich\emph{Der grüne Kakadu. Groteske in einem Akt}|pw}\pwindex{Schnitzler, Arthur 15. 5. 1862 Wien – 21. 10. 1931 ebd.@\textsc{Schnitzler, Arthur} (15. 5. 1862 Wien – 21. 10. 1931 ebd.), \emph{Schriftsteller, Mediziner}!Au Perroquet Vert@\strich\emph{Au Perroquet Vert}|pw}«, seinerzeit \label{K_L03947-2v}\edtext{bei Antoine\pwindex{Antoine, André 31.\,1.\,1858 Limoges – 23.\,10.\,1943 Le Pouliguen@\textsc{Antoine, André} (31.\,1.\,1858 Limoges – 23.\,10.\,1943 Le Pouliguen), \emph{Theaterleiter, Schauspieler}|pw}{ }aufgeführt\eventindex{Théâtre Antoine-Simone Berriau@\textbf{Théâtre Antoine-Simone Berriau}!Premiere von Au Perroquet Vert, 7.11.1903@Premiere von Au Perroquet Vert, 7.11.1903|pwv}}{\lemma{\textnormal{\emph{bei Antoine aufgeführt}}}\Cendnote{\textnormal{Die Theaterpremiere\eventindex{Théâtre Antoine-Simone Berriau@\textbf{Théâtre Antoine-Simone Berriau}!Premiere von Au Perroquet Vert, 7.11.1903@Premiere von Au Perroquet Vert, 7.11.1903|pwk} von \emph{Au Perroquet
                     Vert}\pwindex{Schnitzler, Arthur 15. 5. 1862 Wien – 21. 10. 1931 ebd.@\textsc{Schnitzler, Arthur} (15. 5. 1862 Wien – 21. 10. 1931 ebd.), \emph{Schriftsteller, Mediziner}!Au Perroquet Vert@\strich\emph{Au Perroquet Vert}|pwk} fand am 7. 11. 1903 am \emph{Théâtre Antoine}\orgindex{Théâtre Antoine@Théâtre Antoine|pwk} statt.}}}\label{K_L03947-2}, meines Wissens nie
               gedruckt, ist für die Buchausgabe jedesfalls frei. »Puppenspieler\pwindex{Schnitzler, Arthur 15. 5. 1862 Wien – 21. 10. 1931 ebd.@\textsc{Schnitzler, Arthur} (15. 5. 1862 Wien – 21. 10. 1931 ebd.), \emph{Schriftsteller, Mediziner}!Puppenspieler. Studie in einem Aufzuge@\strich\emph{Der Puppenspieler. Studie in einem Aufzuge}|pw}« frei.\pend
           
\pstart
           {\pb}Die Frage ist nur, wer soll all die Sachen übersetzen?
               Lieber mit der Herausgabe warten als eine mässige Uebersetzung erscheinen lassen. Wer
               wird insbesondere die Verse übertragen? Wie schön, wenn man sich der Mitarbeit Geraldys\pwindex{Géraldy, Paul 6.\,3.\,1885 Paris – 9.\,3.\,1983 Neuilly-sur-Seine@\textsc{Géraldy, Paul} (6.\,3.\,1885 Paris – 9.\,3.\,1983 Neuilly-sur-Seine), \emph{Schriftsteller}|pw} versichern könnte.\pend
           
\pstart
           Nun die materielle Frage. Ich glaube wohl, dass ich 10 {\%} vom
               Ladenpreis verlangen dürfte, und eine entsprechende Garantiesumme bei Abschluss des
               Vertrages. Nehmen wir also an, der Band kostet 5 frncs, 1000 Exemplare 5000, so kämen
               auf mich 500 frcs. Werden gleich 3000 gedruckt, 1500 frcs. u. s. w. Als Garantie wäre
               die Hälfte, eventuell auch mehr zu zahlen. Im weiteren Verlaufe, viertel- resp.
               halbjährige Abrechnung. Aufführungsrechte müssten mir vollkommen reserviert bleiben.
               Das bedürfe weiterer Abmachungen von Fall zu Fall. Auch für die Aufführungsfrage wäre
               ja wohl Geraldys\pwindex{Géraldy, Paul 6.\,3.\,1885 Paris – 9.\,3.\,1983 Neuilly-sur-Seine@\textsc{Géraldy, Paul} (6.\,3.\,1885 Paris – 9.\,3.\,1983 Neuilly-sur-Seine), \emph{Schriftsteller}|pw} Mitwirkung unschätzbar.\pend
           
\pstart
           Die Auswahl der Novellen auf nächstens. Vielleicht verschieben wir das bis zu Ihrer
               Wiederkehr, liebe Frau Hofrätin. Ich denke in diesen Band sollte man nur kürzere
               Novellen aufnehmen, »Casanovas Heimfahrt\pwindex{Schnitzler, Arthur 15. 5. 1862 Wien – 21. 10. 1931 ebd.@\textsc{Schnitzler, Arthur} (15. 5. 1862 Wien – 21. 10. 1931 ebd.), \emph{Schriftsteller, Mediziner}!Casanovas Heimfahrt@\strich\emph{Casanovas Heimfahrt}|pw}« sollte
               wohl als Buch für sich erscheinen.\pend
           
\pstart
           Seien sie vorläufig aufs Allerherzlichste bedankt für Ihre freundlichen Bemühungen.
               Wenn Sie also wirklich die Güte haben wollen die Verhandlungen weiter für mich zu
               führen, eventuell auch schon Theatermöglichkeiten und dergleichen mit Geraldy\pwindex{Géraldy, Paul 6.\,3.\,1885 Paris – 9.\,3.\,1983 Neuilly-sur-Seine@\textsc{Géraldy, Paul} (6.\,3.\,1885 Paris – 9.\,3.\,1983 Neuilly-sur-Seine), \emph{Schriftsteller}|pw} zu besprechen, so bleibt es natürlich
               bei den finanziellen Abmachungen, über die wir noch im Einzelnen mündlich sprechen
               werden. Ihre Teilaberschaft begänne selbstverständlich schon bei den \label{K_L03947-3v}\edtext{\begin{otherlanguage}{french}à-valois\end{otherlanguage}}{\lemma{\textnormal{\emph{à-valois}}}\Cendnote{\textnormal{französisch les à-valoirs:
                  Vorschüsse}}}\label{K_L03947-3}. Jedenfalls bin ich Ihnen vom Herzen dankbar für alles, was Sie
               bisher in Frankreich\oindex{Frankreich@\textbf{Frankreich}|pw} für mich getan haben und tun
               wollen.\pend
           \selectlanguage{ngerman}\endnumbering\briefempfaengerindex{Zuckerkandl, Berta@\textsc{Zuckerkandl, Berta}!zzzSchnitzler, Arthur@\emph{von Arthur Schnitzler}!1923-10-251@{25. 10. 1923}|)be}\mylabel{L03947h}
\begin{anhang}
\end{anhang}\newcommand{\dateiname}{L03947}\newcommand{\titel}{Arthur Schnitzler an Berta Zuckerkandl, 25. 10. 1923}\newcommand{\editorInnen}{Herausgegeben von Jahnke, SelmaMüller, Martin Anton}%% latex-leseansicht-abspann.tex
%% Abspann für die Leseansicht.
%% Der Schalter \ifkorrekturansicht ist bereits durch den Vorspann gesetzt.

%% latex-abspann.tex
%% Gemeinsamer Abspann für Korrekturansicht und Leseansicht.
%% Setzt den Schalter \ifkorrekturansicht voraus (gesetzt in den
%% einbindenden Dateien latex-korrekturansicht-abspann.tex bzw.
%% latex-leseansicht-abspann.tex).
%% ---------------------------------------------------------------

\normalsize

% Das esempio-Environment wird nur in der Leseansicht benötigt
\ifkorrekturansicht\else
\newenvironment{esempio}[3]%
{
    \vspace{1.5ex}
    \rlap{\underline{#1}}
    \par
    \setlength{\parindent}{0cm}
    \nopagebreak
    \leftskip=#2cm
    \rightskip=#3cm
}
{
    \par
}
\fi

\doendnotes{C}
\bigskip
\vfill

\clearpage

\footnotesize

\ifkorrekturansicht
  \lohead{\textsc{register}}
\fi

% theindex-Environment neu definieren ohne reledmac
\makeatletter
\renewenvironment{theindex}{%
  \ifkorrekturansicht
    \section*{\indexname}%
  \else
    \subsubsection*{Index der erwähnten Entitäten}%
  \fi
  \setlength{\parindent}{0pt}%
  \setlength{\parskip}{0pt plus 0.3pt}%
  \let\item\@idxitem
}{%
  \ifkorrekturansicht\clearpage\fi
}
\makeatother

\IfFileExists{\jobname-pw.ind}{\input{\jobname-pw.ind}}{}

% Quellenangabe nur in der Leseansicht
\ifkorrekturansicht\else
% Fallback-Definitionen, falls die .tex-Datei \titel etc. nicht gesetzt hat
\providecommand{\titel}{}
\providecommand{\editorInnen}{}
\providecommand{\dateiname}{\jobname}

\vspace{3cm}

\vfill

\footnotesize
\textsc{Quelle}: \titel. Herausgegeben von {\editorInnen}. In: \emph{Arthur Schnitzler: Briefwechsel mit Autorinnen und Autoren}.
 Digitale Edition, https://schnitzler-briefe.acdh.oeaw.ac.at/{\dateiname}.html (Stand \today)
\fi

\end{document}


