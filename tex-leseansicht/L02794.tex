%% latex-leseansicht-vorspann.tex
%% Vorspann für die Leseansicht.
%% Lädt die gemeinsame Datei latex-vorspann.tex mit nicht gesetztem Schalter.

\newif\ifkorrekturansicht
\korrekturansichtfalse

\input{../tex-inputs/latex-vorspann}


\section[ Paul Goldmann an Arthur Schnitzler, 2. 12. {[}1896{]}]{L02794 Paul Goldmann an Arthur Schnitzler,  2. 12. [1896]}
\nopagebreak\mylabel{L02794v}
\rehead{ }\normalsize\beginnumbering\briefempfaengerindex{Schnitzler, Arthur@\textsc{Schnitzler, Arthur}!zzzGoldmann, Paul@\emph{von Paul Goldmann}!1896-12-021@{2. 12. [1896]}|(be}
\toendnotes[C]{\smallbreak\pagebreak[2]}
\correspDesc{Versand  durch Paul Goldmann am 2. 12. [1896] in Paris
\newline{}Erhalt  durch Arthur Schnitzler im Zeitraum [3. 12. 1896
                  – 7. 12. 1896?] in Wien}\toendnotes[C]{\smallbreak}
\Standort{DLA, A:Schnitzler, HS.NZ85.1.3166.}
\physDesc{Brief, 3 Blätter, 9 Seiten, 4634 Zeichen
\newline{}Handschrift: blaue Tinte, deutsche Kurrent
\newline{}Beilage: aufgeklebter Ausschnitt aus einem Brief von Recha Wolff\pwindex{Wolff, Recha *~1838@\textsc{Wolff, Recha} (*~1838)|pw} an Theodor Wolff\pwindex{Wolff, Theodor 2.\,8.\,1868 Berlin – 23.\,9.\,1943 ebd.@\textsc{Wolff, Theodor} (2.\,8.\,1868 Berlin – 23.\,9.\,1943 ebd.), \emph{Schriftsteller, Journalist}|pw}, schwarze Tinte, deutsche
                                 Kurrent 
\newline{}Schnitzler: 1) mit Bleistift das Jahr »96« vermerkt  2) mir rotem Buntstift vier Unterstreichungen}\toendnotes[C]{\smallbreak}
\pstart
           {\pb}\textcolor{gray}{\textbf{\textbf{Frankfurter Zeitung\orgindex{Frankfurter Zeitung@Frankfurter Zeitung|pw}}}}\pend
           
\pstart
           \textcolor{gray}{\textbf{(\begin{otherlanguage}{french}Gazette de Francfort\end{otherlanguage}\orgindex{Frankfurter Zeitung@Frankfurter Zeitung|pw}).}}\pend
           
\pstart
           \textcolor{gray}{\textbf{\textbf{\begin{otherlanguage}{french}Fondateur M.\end{otherlanguage}{ }L. Sonnemann\pwindex{Sonnemann, Leopold 29.\,10.\,1831 Höchberg – 30.\,10.\,1909 Frankfurt am Main@\textsc{Sonnemann, Leopold} (29.\,10.\,1831 Höchberg – 30.\,10.\,1909 Frankfurt am Main), \emph{Journalist, Herausgeber}|pw}.}}}\pend
           
\pstart
           \begin{otherlanguage}{french}\textcolor{gray}{\textbf{Journal\pwindex{Frankfurter Zeitung@\emph{Frankfurter Zeitung}|pwv} politique,
                        financier,}}\end{otherlanguage}\pend
           
\pstart
           \begin{otherlanguage}{french}\textcolor{gray}{\textbf{commercial et littéraire.}}\end{otherlanguage}\pend
           
\pstart
           \begin{otherlanguage}{french}\textcolor{gray}{\textbf{\textbf{Paraissant trois fois par jour.}}}\end{otherlanguage}\hfill \textsc{Paris\oindex{Paris@\textbf{Paris}, \emph{Hauptstadt}|pw}}, 2. December.\pend
           
\pstart
           \begin{otherlanguage}{french}\textcolor{gray}{\textbf{\textbf{Bureau à Paris\oindex{Paris@\textbf{Paris}, \emph{Hauptstadt}|pw}}}}\end{otherlanguage}\pend
           
\pstart
           \begin{otherlanguage}{french}\textcolor{gray}{\textbf{\textbf{24. Rue Feydeau\oindex{rue Feydeau@\textbf{rue Feydeau}, \emph{Straße}|pw}.}}}\end{otherlanguage}\pend
           
\pstart\center{}Mein lieber Freund,\pend\vspace{0.5em}
\pstart
           Mir{ }ſcheint, in meinen letzten Brief hat{ }ſich{ }ſehr gegen meinen Willen ein falſcher
               Ton eingeſchlichen. Du haſt etwas vom »Berühmtwerden« herausgehört? Ich{ }ſchwöre Dir,
               ich bin durchdrungen von der Nichtigkeit und \label{K_L02794-1v}\edtext{Unbedeutenheit}{\lemma{\textnormal{\emph{Unbedeutenheit}}}\Cendnote{\textnormal{zu der Zeit längst veraltete Form von
                  »Unbedeutendheit«}}}\label{K_L02794-1} aller jener Vorgänge. Ich habe mich{ }ſogar im Verdacht,
               daß ich ein \strikeout{we\textcolor{gray}{n}} wenig Komödie geſpielt habe. Ich \strikeout{\textcolor{gray}{×}} glaube, ich hätte mich vielleicht doch nicht geſchlagen, wenn ich nicht gar{ }ſo{ }ſicher darauf gerechnet hätte, der Andere werde mich nicht erſchießen. Du wirſt ja{ }ſelbſt auch{ }ſehen, wie raſch das Alles vergeſſen werden {\pb}wird, wie bald ich in mein Dunkel zurückkehren
               werde, nachdem ein flüchtiger Lichtſtrahl von draußen auf mich gefallen. Ich glaube{ }ſogar, ich habe es von Anfang an ein wenig auf dieſen Lichtſtrahl angelegt. Ich habe
               für Gerechtigkeit eintreten und zugleich mir etwas \strikeout{Rekla} Reklame machen wollen. Ich habe mit{ }ſchlauer Berechnung von Anfang an
               geſehen, daß die ganze Angelegenheit ein gutes Mittel{ }ſei, auf anſtändige Weiſe von
               mir reden zu machen. Gewiß war auch die Empörung über das Unrecht dabei. Ich will
               mich nicht{ }ſchlechter machen, als ich bin, aber Du machſt {\pb}mich viel zu gut. Etwas Derartiges, wie Deinen
               entzückenden Glückwunſchbrief von neulich habe ich nicht verdient. So wie ich Dirs
               eben geſagt{ }ſtehen die Dinge und nicht anders, und ich möchte nicht, daß es einen
               Schatten von Unehrlichkeit gebe zwiſchen Dir und mir.\pend
           
\pstart
           Jetzt will ich Dir noch{ }ſagen, daß ich geſtern einen
               Brief von \textsc{Georg Brandes\pwindex{Brandes, Georg 4.\,2.\,1842 Kopenhagen – 19.\,2.\,1927 ebd.@\textsc{Brandes, Georg} (4.\,2.\,1842 Kopenhagen – 19.\,2.\,1927 ebd.)|pw}} erhielt, worin er mir, zu meiner freudigen Überraſchung,{ }ſchreibt, er habe mich
                  \label{K_L02794-2v}\edtext{in \textsc{Kopenhagen\oindex{Kopenhagen@\textbf{Kopenhagen}, \emph{Hauptstadt}|pw}} liebgewonnen}{\lemma{\textnormal{\emph{in … liebgewonnen}}}\Cendnote{\textnormal{Im Rahmen der Skandinavien\oindex{Dänemark@\textbf{Dänemark}|pwkv}\oindex{Schweden@\textbf{Schweden}|pwkv}\oindex{Norwegen@\textbf{Norwegen}|pwkv}-Reise im Sommer 1896 waren sich auch  Goldmann\pwindex{Goldmann, Paul 31.\,1.\,1865 Breslau – 25.\,9.\,1935 Wien@\textsc{Goldmann, Paul} (31.\,1.\,1865 Breslau – 25.\,9.\,1935 Wien), \emph{Schriftsteller, Journalist}|pwk} und Georg Brandes\pwindex{Brandes, Georg 4.\,2.\,1842 Kopenhagen – 19.\,2.\,1927 ebd.@\textsc{Brandes, Georg} (4.\,2.\,1842 Kopenhagen – 19.\,2.\,1927 ebd.)|pwk} begegnet, jedenfalls am 21. 8. 1896.}}}\label{K_L02794-2};
               will Dir außerdem{ }ſagen, daß ich \label{K_L02794-3v}\edtext{\textsc{Herzls\pwindex{Herzl, Theodor 2.\,5.\,1860 Budapest – 3.\,7.\,1904 Edlach@\textsc{Herzl, Theodor} (2.\,5.\,1860 Budapest – 3.\,7.\,1904 Edlach), \emph{Schriftsteller, Journalist}|pw}} Art, mich jetzt zu {\pb}überſchätzen}{\lemma{\textnormal{\emph{Herzls … überschätzen}}}\Cendnote{\textnormal{Gemeint ist wohl: nach dem Pistolenduell; siehe XXXX Auszeichnungsfehler: Dokument L02684 nicht gefunden.}}}\label{K_L02794-3}, ebenſo
               lächerlich finde, wie{ }ſeine bisherige Art, mich zu unterſchätzen (der Mann iſt immer
               urtheilslos,{ }ſo oder{ }ſo); und will Dich erſuchen, dem \label{K_L02794-4v}\edtext{Artikel\pwindex{Verschiedenes [Goldmann und Millevoye]@\emph{Verschiedenes [Goldmann und Millevoye]}|pwv} des »\textsc{Figaro\pwindex{Le Figaro@\emph{Le Figaro}|pw}}«, den Du im \strikeout{Bo}{ }Börſen-Courier\pwindex{Berliner Börsen-Zeitung@\emph{Berliner Börsen-Zeitung}|pwv} gefunden}{\lemma{\textnormal{\emph{Artikel … gefunden}}}\Cendnote{\textnormal{Maurice Leudet\pwindex{Leudet, Maurice *~1858@\textsc{Leudet, Maurice} (*~1858), \emph{Journalist}|pwk}: \emph{L’Affaire Millevoye–Goldmann}\pwindex{Leudet, Maurice *~1858@\textsc{Leudet, Maurice} (*~1858), \emph{Journalist}!Affaire Millevoye-Goldmann@\strich\emph{L’Affaire Millevoye-Goldmann}|pwk}. In: \emph{Le Figaro}\pwindex{Le Figaro@\emph{Le Figaro}|pwk}, Jg. 42, Nr. 326, 21. 11. 1896, S. 1–2. [O. V.]: \emph{Verschiedenes}\pwindex{Verschiedenes [Goldmann und Millevoye]@\emph{Verschiedenes [Goldmann und Millevoye]}|pwk}. In: \emph{Berliner Börsen-Zeitung}\pwindex{Berliner Börsen-Zeitung@\emph{Berliner Börsen-Zeitung}|pwk}, Jg. 42, Nr. 531, 24. 11. 1896, Morgen-Ausgabe, S. 12.}}}\label{K_L02794-4}, nicht das
               mindeſte Gewicht beizulegen. Im »\textsc{Figaro\pwindex{Le Figaro@\emph{Le Figaro}|pw}}« werden{ }ſolche Dinge nur gedruckt, wenn man{ }ſie bezahlt. Der Mann\pwindex{Leudet, Maurice *~1858@\textsc{Leudet, Maurice} (*~1858), \emph{Journalist}|pwv}, der dieſen Artikel\pwindex{Verschiedenes [Goldmann und Millevoye]@\emph{Verschiedenes [Goldmann und Millevoye]}|pwv} geſchrieben, iſt ein erbärmliches
               Subject, unfähig, irgend Jemandem aus freien Stücken Gerechtigkeit zu erweiſen. Ich
               vermuthe, daß der Artikel\pwindex{Verschiedenes [Goldmann und Millevoye]@\emph{Verschiedenes [Goldmann und Millevoye]}|pwv} von
               der Familie \textsc{Dreyfus\pwindex{Dreyfus, Alfred 9.\,10.\,1859 Mulhouse – 12.\,7.\,1935 Paris@\textsc{Dreyfus, Alfred} (9.\,10.\,1859 Mulhouse – 12.\,7.\,1935 Paris), \emph{Militär}|pwv}} herrührt, {\pb}und wenn man ihn aufmerkſam lieſt,{ }ſo iſt er \strikeout{\textcolor{gray}{ein}}, unter dem Vorwand \strikeout{\textcolor{gray}{v}} von mir zu{ }ſprechen, ein geſchicktes \textsc{Plaidoyer\pwindex{Verschiedenes [Goldmann und Millevoye]@\emph{Verschiedenes [Goldmann und Millevoye]}|pwv}} für den \strikeout{Verurt}{ }Verurtheilten\pwindex{Dreyfus, Alfred 9.\,10.\,1859 Mulhouse – 12.\,7.\,1935 Paris@\textsc{Dreyfus, Alfred} (9.\,10.\,1859 Mulhouse – 12.\,7.\,1935 Paris), \emph{Militär}|pwv}. Und nun wollen
               wir kein Wort mehr von der ganzen Geſchichte reden, nicht wahr?\pend
           
\pstart
           Nach \strikeout{alle} Allem, was in den letzten Wochen \label{K_L02794-5v}\edtext{zwiſchen mir und mir}{\lemma{\textnormal{\emph{zwischen mir und mir}}}\Cendnote{\textnormal{vermutlich eine wörtliche Übersetzung von
                     »entre moi et moi-même«}}}\label{K_L02794-5} geſtanden, bin ich jetzt wieder
               allein \label{K_L02794-6v}\edtext{\begin{otherlanguage}{french}\textsc{en tête-à-tête avec moi-même}\end{otherlanguage}}{\lemma{\textnormal{\emph{en … moi-même}}}\Cendnote{\textnormal{französisch: mit mir selbst von
                  Angesicht zu Angesicht}}}\label{K_L02794-6}. Und da{ }ſehe ich erſt ganz deutlich, daß alles
               Äußere Schwindel war, und daß ich unfähig bin {\pb}zur
               wahren Leiſtung: ein gutes Buch, ein gutes Stück. Und nicht einmal die Liebe will
               kommen. Nie, nie ein geliebtes Weſen in die Arme geſchloſſen! Und \label{K_L02794-7v}\edtext{morgen iſt die Jugend zu Ende}{\lemma{\textnormal{\emph{morgen … Ende}}}\Cendnote{\textnormal{metaphorisch gemeint, Goldmann\pwindex{Goldmann, Paul 31.\,1.\,1865 Breslau – 25.\,9.\,1935 Wien@\textsc{Goldmann, Paul} (31.\,1.\,1865 Breslau – 25.\,9.\,1935 Wien), \emph{Schriftsteller, Journalist}|pwk} hatte nicht
                  Geburtstag}}}\label{K_L02794-7}! Und es will nicht kommen! Das iſt troſtlos; und dann gehts
               recht{ }ſchlimm mit meinen Augen, und ich fürchte, blind zu werden{\dots}\pend
           
\pstart
           Entſchuldige, daß ich Dir gar{ }ſo viel von mir{ }ſpreche. Ich freue mich, zu hören, daß
               Du wieder arbeiteſt und daß Dir die Arbeit{ }ſeeliſch gut thut. Die \label{K_L02794-8v}\edtext{Sachen\pwindex{Schnitzler, Arthur 15.\,5.\,1862 Wien – 21.\,10.\,1931 ebd.@\textsc{Schnitzler, Arthur} (15.\,5.\,1862 Wien – 21.\,10.\,1931 ebd.), \emph{Schriftsteller, Mediziner}!Reigen. Zehn Dialoge@\strich\emph{Reigen. Zehn Dialoge}|pwv}, mit denen Du
               beſchäftigt biſt}{\lemma{\textnormal{\emph{Sachen, … bist}}}\Cendnote{\textnormal{Am 23. 11. 1896 hatte Schnitzler am \emph{Reigen}\pwindex{Schnitzler, Arthur 15.\,5.\,1862 Wien – 21.\,10.\,1931 ebd.@\textsc{Schnitzler, Arthur} (15.\,5.\,1862 Wien – 21.\,10.\,1931 ebd.), \emph{Schriftsteller, Mediziner}!Reigen. Zehn Dialoge@\strich\emph{Reigen. Zehn Dialoge}|pwk} zu schreiben begonnen. Enthusiasmus für dieses neue Stück\pwindex{Schnitzler, Arthur 15.\,5.\,1862 Wien – 21.\,10.\,1931 ebd.@\textsc{Schnitzler, Arthur} (15.\,5.\,1862 Wien – 21.\,10.\,1931 ebd.), \emph{Schriftsteller, Mediziner}!Reigen. Zehn Dialoge@\strich\emph{Reigen. Zehn Dialoge}|pwkv} klingt etwa im \emph{Tagebuch}\pwindex{Schnitzler, Arthur 15.\,5.\,1862 Wien – 21.\,10.\,1931 ebd.@\textsc{Schnitzler, Arthur} (15.\,5.\,1862 Wien – 21.\,10.\,1931 ebd.), \emph{Schriftsteller, Mediziner}!Tagebuch@\strich\emph{Tagebuch}|pwk}-Eintrag vom 27. 11. 1896 durch: »Schrieb mit Laune
                        die 4. Scene\pwindex{Schnitzler, Arthur 15.\,5.\,1862 Wien – 21.\,10.\,1931 ebd.@\textsc{Schnitzler, Arthur} (15.\,5.\,1862 Wien – 21.\,10.\,1931 ebd.), \emph{Schriftsteller, Mediziner}!Reigen. Zehn Dialoge@\strich\emph{Reigen. Zehn Dialoge}|pwv} des Hemic\pwindex{Schnitzler, Arthur 15.\,5.\,1862 Wien – 21.\,10.\,1931 ebd.@\textsc{Schnitzler, Arthur} (15.\,5.\,1862 Wien – 21.\,10.\,1931 ebd.), \emph{Schriftsteller, Mediziner}!Reigen. Zehn Dialoge@\strich\emph{Reigen. Zehn Dialoge}|pwv}.«}}}\label{K_L02794-8},
               dürften {\pb}Dir{ }ſehr »liegen«. Wie denkſt Du aber doch
               über das hiſtoriſche \strikeout{Wie}{ }\label{K_L02794-9v}\edtext{Wien\oindex{Wien@\textbf{Wien}, \emph{Verwaltungsgebiet}|pw}er Stück}{\lemma{\textnormal{\emph{Wiener Stück}}}\Cendnote{\textnormal{Siehe A. S.: \emph{Tagebuch}, 22. 11. 1896.
               }}}\label{K_L02794-9}? Vielleicht mit einem jungen Componiſten, der ein Bischen alte und neue Wien\oindex{Wien@\textbf{Wien}, \emph{Verwaltungsgebiet}|pw}er Muſik dazu machen würde? Würde Dich dieſe
               Abwechſelung nicht einmal reizen? Oder willſt Du fürs Erſte überhaupt kein größeres
               Stück{ }ſchreiben? Auch das würde ich{ }ſehr billigen. Und wann kommt Dein \label{K_L02794-10v}\edtext{Buch\pwindex{Schnitzler, Arthur 15.\,5.\,1862 Wien – 21.\,10.\,1931 ebd.@\textsc{Schnitzler, Arthur} (15.\,5.\,1862 Wien – 21.\,10.\,1931 ebd.), \emph{Schriftsteller, Mediziner}!Frau des Weisen. Novelletten@\strich\emph{Die Frau des Weisen. Novelletten}|pwv} bei \textsc{Fischer\orgindex{S. Fischer Verlag@S. Fischer Verlag|pw}}}{\lemma{\textnormal{\emph{Buch bei Fischer}}}\Cendnote{\textnormal{Im August 1896 vereinbarten S.
                     Fischer\pwindex{Fischer, Samuel 24.\,12.\,1859 Liptovský Mikuláš – 15.\,10.\,1934 Berlin@\textsc{Fischer, Samuel} (24.\,12.\,1859 Liptovský Mikuláš – 15.\,10.\,1934 Berlin), \emph{Verleger}|pwk} und Schnitzler, eine Sammlung
                  seiner Novelletten als Buch zu veröffentlichen. \emph{Die Frau des Weisen}\pwindex{Schnitzler, Arthur 15.\,5.\,1862 Wien – 21.\,10.\,1931 ebd.@\textsc{Schnitzler, Arthur} (15.\,5.\,1862 Wien – 21.\,10.\,1931 ebd.), \emph{Schriftsteller, Mediziner}!Frau des Weisen. Novelletten@\strich\emph{Die Frau des Weisen. Novelletten}|pwk} erschien aber erst am 3. 5. 1898.}}}\label{K_L02794-10}?\pend
           
\pstart
           Wer iſt dieſer \textsc{Stephan Grossmann\pwindex{Großmann, Stefan 19.\,5.\,1875 Wien – 3.\,1.\,1935 ebd.@\textsc{Großmann, Stefan} (19.\,5.\,1875 Wien – 3.\,1.\,1935 ebd.), \emph{Schriftsteller, Journalist}|pw}}, den Du mir geſchickt haſt? Ich habe mich für ihn verwendet, und heut wird mir ein \label{K_L02794-11v}\edtext{Zeitungs-Ausſchnitt}{\lemma{\textnormal{\emph{Zeitungs-Ausschnitt}}}\Cendnote{\textnormal{Mehrere Tageszeitungen berichteten über die Verhaftung des
                  Anarchisten und Journalisten Stephan
                     Großmann\pwindex{Großmann, Stefan 19.\,5.\,1875 Wien – 3.\,1.\,1935 ebd.@\textsc{Großmann, Stefan} (19.\,5.\,1875 Wien – 3.\,1.\,1935 ebd.), \emph{Schriftsteller, Journalist}|pwk} in Berlin\oindex{Berlin@\textbf{Berlin}, \emph{Hauptstadt}|pwk}. Siehe etwa
                     [O. V.]: \emph{Verhaftung eines Wiener
                        Anarchisten in Berlin}\pwindex{Verhaftung eines Wiener Anarchisten in Berlin@\emph{Verhaftung eines Wiener Anarchisten in Berlin}|pwk}. In: \emph{Arbeiter-Zeitung}\pwindex{Arbeiter-Zeitung@\emph{Arbeiter-Zeitung}|pwk}, Jg. 8, Nr. 297, 28. 10. 1896, Morgenblatt, S. 5–6.}}}\label{K_L02794-11} geſchickt, worin{ }ſteht, daß {\pb}er{ }ſich der Berlin\oindex{Berlin@\textbf{Berlin}, \emph{Hauptstadt}|pw}er Polizei\orgindex{Polizeidirektion Berlin@Polizeidirektion Berlin|pwv} als Spitzel angeboten habe. \strikeout{H\textcolor{gray}{×}\-\textcolor{gray}{×}} Ich habe ihm\pwindex{Großmann, Stefan 19.\,5.\,1875 Wien – 3.\,1.\,1935 ebd.@\textsc{Großmann, Stefan} (19.\,5.\,1875 Wien – 3.\,1.\,1935 ebd.), \emph{Schriftsteller, Journalist}|pwv} geſagt,
               daß er, da er mit einer Empfehlung von Dir bei mir erſchienen iſt, \strikeout{v\textcolor{gray}{o}n v\textcolor{gray}{o}} in meinen Augen von vornherein gegen alle Zeitungen Recht hat. Aber er hat{ }ſich \strikeout{\textcolor{gray}{m}i\textcolor{gray}{s}} ungeſchickt gerechtfertigt; das kann freilich auch Befangenheit{ }ſein; \strikeout{imm\textcolor{gray}{e}} darum möchte ich gern in zwei Worten hören, wie Du über den Fall denkſt?\pend
           
\pstart
           Iſt es wahr, daß die \label{K_L02794-12v}\edtext{»Allgemeine Zeitung\orgindex{Wiener Allgemeine Zeitung@Wiener Allgemeine Zeitung|pwv}« in andere Hände}{\lemma{\textnormal{\emph{»Allgemeine … Hände}}}\Cendnote{\textnormal{Mit Jahresende 1896 übergab
                  der mit einer Cousine\pwindex{Gans-Ludassy, Olga von 5.\,6.\,1867 Wien – 18.\,8.\,1948 Islip@\textsc{Gans-Ludassy, Olga von} (5.\,6.\,1867 Wien – 18.\,8.\,1948 Islip)|pwkv}{ }Schnitzlers verheiratete Julius Gans-Ludassy\pwindex{Gans-Ludassy, Julius von 13.\,4.\,1858 Wien – 30.\,9.\,1922 ebd.@\textsc{Gans-Ludassy, Julius von} (13.\,4.\,1858 Wien – 30.\,9.\,1922 ebd.), \emph{Schriftsteller, Journalist, Herausgeber}|pwk} die Herausgabe der \emph{Wiener Allgemeinen Zeitung}\orgindex{Wiener Allgemeine Zeitung@Wiener Allgemeine Zeitung|pwk} an August Krawani\pwindex{Krawani, August 6.\,10.\,1829 Ptuj – 4.\,11.\,1900 Wien@\textsc{Krawani, August} (6.\,10.\,1829 Ptuj – 4.\,11.\,1900 Wien), \emph{Schriftsteller, Journalist}|pwk}, der zu diesem Zeitpunkt beinahe siebzig
                  Jahre alt war. Julian Sternberg\pwindex{Sternberg, Julian 8.\,11.\,1868 Wien – 28.\,6.\,1945 Havanna@\textsc{Sternberg, Julian} (8.\,11.\,1868 Wien – 28.\,6.\,1945 Havanna), \emph{Journalist}|pwk} war seit
                  einem Jahr als Chefredakteur im Amt und wurde am 30. 6. 1897 von Josef Münz\pwindex{Münz, Josef @\textsc{Münz, Josef}, \emph{Journalist}|pwk}
                  abgelöst. Die Personalwechsel bedeuteten für Salten\pwindex{Salten, Felix 6.\,9.\,1869 Budapest – 8.\,10.\,1945 Zürich@\textsc{Salten, Felix} (6.\,9.\,1869 Budapest – 8.\,10.\,1945 Zürich), \emph{Schriftsteller, Journalist, Chefredakteur}|pwk}, der seit 1894 am Blatt\orgindex{Wiener Allgemeine Zeitung@Wiener Allgemeine Zeitung|pwkv} mitarbeitete, zu verschiedenen
                  Zeiten unterschiedliche Aufgaben, er verlor aber seine Stelle nicht.}}}\label{K_L02794-12}
               übergeht? Was wird aus \textsc{Salten\pwindex{Salten, Felix 6.\,9.\,1869 Budapest – 8.\,10.\,1945 Zürich@\textsc{Salten, Felix} (6.\,9.\,1869 Budapest – 8.\,10.\,1945 Zürich), \emph{Schriftsteller, Journalist, Chefredakteur}|pw}}? {\dots}\pend
           
\pstart
           Sei nochmals von ganzem Herzen bedankt für Deine treue Antheilnahme an den letzten
               Vorgängen. Tauſend herzliche Grüße! Dein \spacefill\mbox{Paul Goldmn}\pend
           
\pstart
           \label{T_L02794-1v}\edtext{Grüße \textsc{Richard\pwindex{Beer-Hofmann, Richard 11.\,7.\,1866 Wien – 26.\,9.\,1945 New York City@\textsc{Beer-Hofmann, Richard} (11.\,7.\,1866 Wien – 26.\,9.\,1945 New York City), \emph{Schriftsteller}|pw}} und \textsc{Leo\pwindex{Van-Jung, Leo 15.\,10.\,1866 Odessa – 2.\,7.\,1939 Riga@\textsc{Van-Jung, Leo} (15.\,10.\,1866 Odessa – 2.\,7.\,1939 Riga), \emph{Gesangspädagoge, Mathematiker}|pw}}! Und schreib’ mir recht bald!}{\lemma{\textnormal{\emph{Grüße … bald!}}}\Cendnote{\textnormal{seitlich am linken Rand}}}\label{T_L02794-1}\pend
           
\pstart
           \label{T_L02794-2v}\edtext{Die \label{K_L02794-13v}\edtext{Kritiken}{\lemma{\textnormal{\emph{Kritiken}}}\Cendnote{\textnormal{Rezensionen der Uraufführung\eventindex{Deutsches Theater Berlin@\textbf{Deutsches Theater Berlin}!Uraufführung von Freiwild, 3.11.1896@Uraufführung von Freiwild, 3.11.1896|pwkv} von \emph{Freiwild}\pwindex{Schnitzler, Arthur 15.\,5.\,1862 Wien – 21.\,10.\,1931 ebd.@\textsc{Schnitzler, Arthur} (15.\,5.\,1862 Wien – 21.\,10.\,1931 ebd.), \emph{Schriftsteller, Mediziner}!Freiwild. Schauspiel in 3 Akten@\strich\emph{Freiwild. Schauspiel in 3 Akten}|pwk}}}}\label{K_L02794-13}{ }ſende ich Dir demnächſt zurück}{\lemma{\textnormal{\emph{Die … zurück}}}\Cendnote{\textnormal{kopfüber am oberen Rand}}}\label{T_L02794-2}\pend
           \selectlanguage{ngerman}\vspace{1em}{\vspace{1\baselineskip}}
\pstart
           \noindent{}{\pb}\label{K_L02794-14v}\edtext{Dies iſt ein Ausſchnitt}{\lemma{\textnormal{\emph{Dies ist ein Ausschnitt}}}\Cendnote{\textnormal{Die Ergänzung dieses undatierten Blattes
                  zu diesem Brief muss gerechtfertigt werden. Als eigenes Korrespondenzstück wirkt
                  es zu zusammenhanglos. Es entspricht auch nicht den sonstigen Usancen der
                  Korrespondenz, derartige Petitessen separat zu senden. Die inhärente Datierung des
                  Briefs von Recha Wolff\pwindex{Wolff, Recha *~1838@\textsc{Wolff, Recha} (*~1838)|pwk} auf den Tag nach
                  einer Berlin\oindex{Berlin@\textbf{Berlin}, \emph{Hauptstadt}|pwk}er Aufführung von \emph{Freiwild}\pwindex{Schnitzler, Arthur 15.\,5.\,1862 Wien – 21.\,10.\,1931 ebd.@\textsc{Schnitzler, Arthur} (15.\,5.\,1862 Wien – 21.\,10.\,1931 ebd.), \emph{Schriftsteller, Mediziner}!Freiwild. Schauspiel in 3 Akten@\strich\emph{Freiwild. Schauspiel in 3 Akten}|pwk} erlaubt, ihren Brief zeitlich einzugrenzen: Das
                     Stück\pwindex{Schnitzler, Arthur 15.\,5.\,1862 Wien – 21.\,10.\,1931 ebd.@\textsc{Schnitzler, Arthur} (15.\,5.\,1862 Wien – 21.\,10.\,1931 ebd.), \emph{Schriftsteller, Mediziner}!Freiwild. Schauspiel in 3 Akten@\strich\emph{Freiwild. Schauspiel in 3 Akten}|pwkv} stand zwischen
                  3. 11. 1896 und 16. 11. 1896 am \emph{Deutschen Theater}\orgindex{Deutsches Theater Berlin@Deutsches Theater Berlin|pwk} auf dem
                  Programm. Entsprechend könnte der Ausschnitt jedem der Briefstücke des Novembers 1896 zugeordnet werden. Markant ist jedoch der
                  divergierende Farbton der von Goldmann\pwindex{Goldmann, Paul 31.\,1.\,1865 Breslau – 25.\,9.\,1935 Wien@\textsc{Goldmann, Paul} (31.\,1.\,1865 Breslau – 25.\,9.\,1935 Wien), \emph{Schriftsteller, Journalist}|pwk}
                  verwendeten Tinte. Am meisten stimmt er mit dem vorliegenden Brief überein, womit
                  die Zuordnung vorgenommen werden konnte.}}}\label{K_L02794-14} aus einem Briefe, den mein College
                  \textsc{Th. Wolff\pwindex{Wolff, Theodor 2.\,8.\,1868 Berlin – 23.\,9.\,1943 ebd.@\textsc{Wolff, Theodor} (2.\,8.\,1868 Berlin – 23.\,9.\,1943 ebd.), \emph{Schriftsteller, Journalist}|pw}} dieſer Tage von{ }ſeiner Mutter\pwindex{Wolff, Recha *~1838@\textsc{Wolff, Recha} (*~1838)|pwv} erhalten hat:\pend
           \selectlanguage{ngerman}\vspace{1em}
\pstart
           \noindent{}{[}hs. Wolff:{]} recht zu{ }ſagen. Gestern war ich mit \textsc{Martha\pwindex{Wolff, Marta 6.\,9.\,1871 Berlin – 22.\,9.\,1942 Konzentrationslager Theresienstadt@\textsc{Wolff, Marta} (6.\,9.\,1871 Berlin – 22.\,9.\,1942 Konzentrationslager Theresienstadt), \emph{Fotografin}|pw}} am Deutſchen Theater\oindex{Deutsches Theater Berlin@\textbf{Deutsches Theater Berlin}, \emph{Theater}|pw}, \textcolor{gray}{wo}
               wir einen wirklichen Genuß hatten. »Freiwild\pwindex{Schnitzler, Arthur 15.\,5.\,1862 Wien – 21.\,10.\,1931 ebd.@\textsc{Schnitzler, Arthur} (15.\,5.\,1862 Wien – 21.\,10.\,1931 ebd.), \emph{Schriftsteller, Mediziner}!Freiwild. Schauspiel in 3 Akten@\strich\emph{Freiwild. Schauspiel in 3 Akten}|pw}«
               von Schnitzler iſt das Schönſte, was ich{ }ſeit lange geſehen, und geſpielt wurde
               geradezu vollendet\textcolor{gray}{.}\pend
           \selectlanguage{ngerman}\endnumbering\briefempfaengerindex{Schnitzler, Arthur@\textsc{Schnitzler, Arthur}!zzzGoldmann, Paul@\emph{von Paul Goldmann}!1896-12-021@{2. 12. [1896]}|)be}\mylabel{L02794h}  \newcommand{\dateiname}{L02794}\newcommand{\titel}{Paul Goldmann an Arthur Schnitzler, 2. 12. [1896]}\newcommand{\editorInnen}{Martin Anton Müller und Laura Untner}%% latex-leseansicht-abspann.tex
%% Abspann für die Leseansicht.
%% Der Schalter \ifkorrekturansicht ist bereits durch den Vorspann gesetzt.

%% latex-abspann.tex
%% Gemeinsamer Abspann für Korrekturansicht und Leseansicht.
%% Setzt den Schalter \ifkorrekturansicht voraus (gesetzt in den
%% einbindenden Dateien latex-korrekturansicht-abspann.tex bzw.
%% latex-leseansicht-abspann.tex).
%% ---------------------------------------------------------------

\normalsize

% Das esempio-Environment wird nur in der Leseansicht benötigt
\ifkorrekturansicht\else
\newenvironment{esempio}[3]%
{
    \vspace{1.5ex}
    \rlap{\underline{#1}}
    \par
    \setlength{\parindent}{0cm}
    \nopagebreak
    \leftskip=#2cm
    \rightskip=#3cm
}
{
    \par
}
\fi

\doendnotes{C}
\bigskip
\vfill

\clearpage

\footnotesize

\ifkorrekturansicht
  \lohead{\textsc{register}}
\fi

% theindex-Environment neu definieren ohne reledmac
\makeatletter
\renewenvironment{theindex}{%
  \ifkorrekturansicht
    \section*{\indexname}%
  \else
    \subsubsection*{Index der erwähnten Entitäten}%
  \fi
  \setlength{\parindent}{0pt}%
  \setlength{\parskip}{0pt plus 0.3pt}%
  \let\item\@idxitem
}{%
  \ifkorrekturansicht\clearpage\fi
}
\makeatother

\IfFileExists{\jobname-pw.ind}{\input{\jobname-pw.ind}}{}

% Quellenangabe nur in der Leseansicht
\ifkorrekturansicht\else
% Fallback-Definitionen, falls die .tex-Datei \titel etc. nicht gesetzt hat
\providecommand{\titel}{}
\providecommand{\editorInnen}{}
\providecommand{\dateiname}{\jobname}

\vspace{3cm}

\vfill

\footnotesize
\textsc{Quelle}: \titel. Herausgegeben von {\editorInnen}. In: \emph{Arthur Schnitzler: Briefwechsel mit Autorinnen und Autoren}.
 Digitale Edition, https://schnitzler-briefe.acdh.oeaw.ac.at/{\dateiname}.html (Stand \today)
\fi

\end{document}


