%% latex-leseansicht-vorspann.tex
%% Vorspann für die Leseansicht.
%% Lädt die gemeinsame Datei latex-vorspann.tex mit nicht gesetztem Schalter.

\newif\ifkorrekturansicht
\korrekturansichtfalse

\input{../tex-inputs/latex-vorspann}


         
         \renewcommand{\erwaehntePersonen}{Personen: Richard Beer-Hofmann, Georg Brandes, Alfred Dreyfus, Samuel Fischer, Olga von Gans-Ludassy, Julius von Gans-Ludassy, Stefan Großmann, Theodor Herzl, August Krawani, Maurice Leudet, Josef Münz, Felix Salten, Leopold Sonnemann, Julian Sternberg, Leo Van-Jung, Recha Wolff, Theodor Wolff, Marta Wolff}
         \renewcommand{\erwaehnteInstitutionen}{Institutionen: Frankfurter Zeitung, Polizeidirektion Berlin, S. Fischer Verlag, Wiener Allgemeine Zeitung}
         \renewcommand{\erwaehnteOrte}{Orte: Berlin, Deutsches Theater Berlin, Dänemark, Kopenhagen, Norwegen, Paris, Schweden, Wien, rue Feydeau}
         \renewcommand{\erwaehnteWerke}{Werke: Arbeiter-Zeitung, Berliner Börsen-Zeitung, Die Frau des Weisen. Novelletten, Frankfurter Zeitung, Freiwild. Schauspiel in 3 Akten, Le Figaro, L’Affaire Millevoye-Goldmann, Reigen. Zehn Dialoge, Tagebuch, Verhaftung eines Wiener Anarchisten in Berlin, Verschiedenes [Goldmann und Millevoye]}
               \section[ Paul Goldmann an Arthur Schnitzler, 2. 12. {[}1896{]}]{ Paul Goldmann an Arthur Schnitzler, 2. 12. {[}1896{]}}\nopagebreak\mylabel{v}\rehead{ }\begin{ledgroupsized}[t]{13cm}\normalsize\beginnumbering \toendnotes[C]{\smallbreak\pagebreak[2]} \Standort{DLA, A:Schnitzler, HS.NZ85.1.3166.}
\physDesc{Brief, 3 Blätter, 9 Seiten, 4634 Zeichen
\newline{}Handschrift: blaue Tinte, deutsche Kurrent
\newline{}Beilage: aufgeklebter Ausschnitt aus einem Brief von Recha Wolff\pwindex{Wolff, Recha *~1838@\textsc{Wolff, Recha} (*~1838)|pw} an Theodor Wolff\pwindex{Wolff, Theodor 1868-08-02 – 1943-09-23@\textsc{Wolff, Theodor} (1868-08-02 – 1943-09-23), \emph{Schriftsteller, Journalist}|pw}, schwarze Tinte, deutsche
                                 Kurrent 
\newline{}Schnitzler: 1) mit Bleistift das Jahr »96« vermerkt  2) mir rotem Buntstift vier Unterstreichungen}\toendnotes[C]{\smallbreak}\pstart
           \noindent{}{\pb}\textcolor{gray}{\textbf{\textbf{Frankfurter Zeitung\orgindex{Frankfurter Zeitung@Frankfurter Zeitung|pw}}}}\pend
           \pstart
           \textcolor{gray}{\textbf{(\begin{otherlanguage}{french}Gazette de Francfort\end{otherlanguage}\orgindex{Frankfurter Zeitung@Frankfurter Zeitung|pw}).}}\pend
           \pstart
           \textcolor{gray}{\textbf{\textbf{\begin{otherlanguage}{french}Fondateur M.\end{otherlanguage}{ }L. Sonnemann\pwindex{Sonnemann, Leopold 1831-10-29 – 1909-10-30@\textsc{Sonnemann, Leopold} (1831-10-29 – 1909-10-30), \emph{Journalist, Herausgeber}|pw}.}}}\pend
           \pstart
           \begin{otherlanguage}{french}\textcolor{gray}{\textbf{Journal\pwindex{?? Werk@Nicht ermittelte Verfasserinnen und Verfasser!Frankfurter Zeitung1856 – 1943@\emph{Frankfurter Zeitung} {[}1856 – 1943{]}|pwv} politique,
                        financier,}}\end{otherlanguage}\pend
           \pstart
           \begin{otherlanguage}{french}\textcolor{gray}{\textbf{commercial et littéraire.}}\end{otherlanguage}\pend
           \pstart
           \begin{otherlanguage}{french}\textcolor{gray}{\textbf{\textbf{Paraissant trois fois par jour.}}}\end{otherlanguage}\hfill \textsc{Paris\oindex{Paris@\textbf{Paris}|pw}}, 2. December.\pend
           \pstart
           \begin{otherlanguage}{french}\textcolor{gray}{\textbf{\textbf{Bureau à Paris\oindex{Paris@\textbf{Paris}|pw}}}}\end{otherlanguage}\pend
           \pstart
           \begin{otherlanguage}{french}\textcolor{gray}{\textbf{\textbf{24. Rue Feydeau\oindex{rue Feydeau@\textbf{rue Feydeau}|pw}.}}}\end{otherlanguage}\pend
           \pstart\center{}Mein lieber Freund,\pend\pstart
           Mir ſcheint, in meinen letzten Brief hat ſich ſehr gegen meinen Willen ein falſcher
               Ton eingeſchlichen. Du haſt etwas vom »Berühmtwerden« herausgehört? Ich ſchwöre Dir,
               ich bin durchdrungen von der Nichtigkeit und \label{K_L02794-111v}\edtext{Unbedeutenheit}{\lemma{\textnormal{\emph{Unbedeutenheit}}}\Cendnote{\textnormal{zu der Zeit längst veraltete Form von
                  »Unbedeutendheit«}}}\label{K_L02794-111h} aller jener Vorgänge. Ich habe mich ſogar im Verdacht,
               daß ich ein \strikeout{we\textcolor{gray}{n}} wenig Komödie geſpielt habe. Ich \strikeout{\textcolor{gray}{×}} glaube, ich hätte mich vielleicht doch nicht geſchlagen, wenn ich nicht gar ſo
               ſicher darauf gerechnet hätte, der Andere werde mich nicht erſchießen. Du wirſt ja
               ſelbſt auch ſehen, wie raſch das Alles vergeſſen werden {\pb}wird, wie bald ich in mein Dunkel zurückkehren
               werde, nachdem ein flüchtiger Lichtſtrahl von draußen auf mich gefallen. Ich glaube
               ſogar, ich habe es von Anfang an ein wenig auf dieſen Lichtſtrahl angelegt. Ich habe
               für Gerechtigkeit eintreten und zugleich mir etwas \strikeout{Rekla} Reklame machen wollen. Ich habe mit ſchlauer Berechnung von Anfang an
               geſehen, daß die ganze Angelegenheit ein gutes Mittel ſei, auf anſtändige Weiſe von
               mir reden zu machen. Gewiß war auch die Empörung über das Unrecht dabei. Ich will
               mich nicht ſchlechter machen, als ich bin, aber Du machſt {\pb}mich viel zu gut. Etwas Derartiges, wie Deinen
               entzückenden Glückwunſchbrief von neulich habe ich nicht verdient. So wie ich Dirs
               eben geſagt ſtehen die Dinge und nicht anders, und ich möchte nicht, daß es einen
               Schatten von Unehrlichkeit gebe zwiſchen Dir und mir.\pend
           \pstart
           Jetzt will ich Dir noch ſagen, daß ich geſtern einen
               Brief von \textsc{Georg Brandes\pwindex{Brandes, Georg 04.02.1842 – 19.02.1927@\textsc{Brandes, Georg} (04.02.1842 – 19.02.1927)|pw}} erhielt, worin er mir, zu meiner freudigen Überraſchung, ſchreibt, er habe mich
                  \label{K_L02794-1v}\edtext{in \textsc{Kopenhagen\oindex{Kopenhagen@\textbf{Kopenhagen}|pw}} liebgewonnen}{\lemma{\textnormal{\emph{in … liebgewonnen}}}\Cendnote{\textnormal{Im Rahmen der Skandinavien\oindex{Daenemark@\textbf{Dänemark}|pwkv}\oindex{Schweden@\textbf{Schweden}|pwkv}\oindex{Norwegen@\textbf{Norwegen}|pwkv}-Reise im Sommer 1896 traf Goldmann\pwindex{Goldmann, Paul 31.01.1865 – 25.09.1935@\textsc{Goldmann, Paul} (31.01.1865 – 25.09.1935), \emph{Schriftsteller, Journalist}|pwk} auch auf Georg Brandes\pwindex{Brandes, Georg 04.02.1842 – 19.02.1927@\textsc{Brandes, Georg} (04.02.1842 – 19.02.1927)|pwk}, jedenfalls am 21. 8. 1896.}}}\label{K_L02794-1h};
               will Dir außerdem ſagen, daß ich \label{K_L02794-2v}\edtext{\textsc{Herzl\pwindex{Herzl, Theodor 1860-05-02 – 1904-07-03@\textsc{Herzl, Theodor} (1860-05-02 – 1904-07-03), \emph{Schriftsteller, Journalist}|pw}s} Art, mich jetzt zu {\pb}überſchätzen}{\lemma{\textnormal{\emph{Herzls … überſchätzen}}}\Cendnote{\textnormal{gemeint ist wohl: nach dem Pistolenduell (siehe Arthur Schnitzler an Paul Goldmann, 21. 11. 1896)}}}\label{K_L02794-2h}, ebenſo
               lächerlich finde, wie ſeine bisherige Art, mich zu unterſchätzen (der Mann iſt immer
               urtheilslos, ſo oder ſo); und will Dich erſuchen, dem \label{K_L02794-3v}\edtext{Artikel\pwindex{?? Werk@Nicht ermittelte Verfasserinnen und Verfasser!Verschiedenes [Goldmann und Millevoye]1896-11-24@\emph{Verschiedenes [Goldmann und Millevoye]} {[}1896-11-24{]}|pwv} des »\textsc{Figaro\pwindex{Le Figaro1826-01-15@\emph{Le Figaro} {[}1826-01-15{]}|pw}}«, den Du im \strikeout{Bo}{ }Börſen-Courier\pwindex{?? Werk@Nicht ermittelte Verfasserinnen und Verfasser!Berliner Boersen-ZeitungNone@\emph{Berliner Börsen-Zeitung} {[}None{]}|pwv} gefunden}{\lemma{\textnormal{\emph{Artikel … gefunden}}}\Cendnote{\textnormal{Maurice Leudet\pwindex{Leudet, Maurice *~1858@\textsc{Leudet, Maurice} (*~1858), \emph{Journalist}|pwk}: \emph{L’Affaire Millevoye-Goldmann}\pwindex{Leudet, Maurice *~1858@\textsc{Leudet, Maurice} (*~1858), \emph{Journalist}!Affaire Millevoye-Goldmann1896-11-21@\strich\emph{L’Affaire Millevoye-Goldmann} {[}1896-11-21{]}|pwk}. In: \emph{Le Figaro}\pwindex{Le Figaro1826-01-15@\emph{Le Figaro} {[}1826-01-15{]}|pwk}, Jg. 42, Nr. 326, 21. 11. 1896, S. 1–2. [O. V.]: \emph{Verschiedenes}\pwindex{?? Werk@Nicht ermittelte Verfasserinnen und Verfasser!Verschiedenes [Goldmann und Millevoye]1896-11-24@\emph{Verschiedenes [Goldmann und Millevoye]} {[}1896-11-24{]}|pwk}. In: \emph{Berliner Börsen-Zeitung}\pwindex{?? Werk@Nicht ermittelte Verfasserinnen und Verfasser!Berliner Boersen-ZeitungNone@\emph{Berliner Börsen-Zeitung} {[}None{]}|pwk}, Jg. 42, Nr. 531, 24. 11. 1896, Morgen-Ausgabe, S. 12.}}}\label{K_L02794-3h}, nicht das
               mindeſte Gewicht beizulegen. Im »\textsc{Figaro\pwindex{Le Figaro1826-01-15@\emph{Le Figaro} {[}1826-01-15{]}|pw}}« werden ſolche Dinge nur gedruckt, wenn man ſie bezahlt. Der Mann\pwindex{Leudet, Maurice *~1858@\textsc{Leudet, Maurice} (*~1858), \emph{Journalist}|pwv}, der dieſen Artikel\pwindex{?? Werk@Nicht ermittelte Verfasserinnen und Verfasser!Verschiedenes [Goldmann und Millevoye]1896-11-24@\emph{Verschiedenes [Goldmann und Millevoye]} {[}1896-11-24{]}|pwv} geſchrieben, iſt ein erbärmliches
               Subject, unfähig, irgend Jemandem aus freien Stücken Gerechtigkeit zu erweiſen. Ich
               vermuthe, daß der Artikel\pwindex{?? Werk@Nicht ermittelte Verfasserinnen und Verfasser!Verschiedenes [Goldmann und Millevoye]1896-11-24@\emph{Verschiedenes [Goldmann und Millevoye]} {[}1896-11-24{]}|pwv} von
               der Familie \textsc{Dreyfus\pwindex{Dreyfus, Alfred 1859-10-09 – 1935-07-12@\textsc{Dreyfus, Alfred} (1859-10-09 – 1935-07-12), \emph{Militär}|pwv}} herrührt, {\pb}und wenn man ihn aufmerkſam lieſt,
               ſo iſt er \strikeout{\textcolor{gray}{ein}}, unter dem Vorwand \strikeout{\textcolor{gray}{v}} von mir zu ſprechen, ein geſchicktes \textsc{Plaidoyer\pwindex{?? Werk@Nicht ermittelte Verfasserinnen und Verfasser!Verschiedenes [Goldmann und Millevoye]1896-11-24@\emph{Verschiedenes [Goldmann und Millevoye]} {[}1896-11-24{]}|pwv}} für den \strikeout{Verurt}{ }Verurtheilten\pwindex{Dreyfus, Alfred 1859-10-09 – 1935-07-12@\textsc{Dreyfus, Alfred} (1859-10-09 – 1935-07-12), \emph{Militär}|pwv}. Und nun wollen
               wir kein Wort mehr von der ganzen Geſchichte reden, nicht wahr?\pend
           \pstart
           Nach \strikeout{alle} Allem, was in den letzten Wochen \label{K_L02794-44v}\edtext{zwiſchen mir und mir}{\lemma{\textnormal{\emph{zwiſchen mir und mir}}}\Cendnote{\textnormal{vermutlich eine wörtliche Übersetzung von
                     »entre moi et moi-même«}}}\label{K_L02794-44h} geſtanden, bin ich jetzt wieder
               allein \label{K_L02794-6v}\edtext{\begin{otherlanguage}{french}\textsc{en tête-à-tête avec moi-même}\end{otherlanguage}}{\lemma{\textnormal{\emph{en … moi-même}}}\Cendnote{\textnormal{französisch: mit mir selbst von
                  Angesicht zu Angesicht}}}\label{K_L02794-6h}. Und da ſehe ich erſt ganz deutlich, daß alles
               Äußere Schwindel war, und daß ich unfähig bin {\pb}zur
               wahren Leiſtung: ein gutes Buch, ein gutes Stück. Und nicht einmal die Liebe will
               kommen. Nie, nie ein geliebtes Weſen in die Arme geſchloſſen! Und \label{K_L02794-89v}\edtext{morgen iſt die Jugend zu Ende}{\lemma{\textnormal{\emph{morgen … Ende}}}\Cendnote{\textnormal{metaphorisch gemeint, er hatte nicht
                  Geburtstag}}}\label{K_L02794-89h}! Und es will nicht kommen! Das iſt troſtlos; und dann gehts
               recht ſchlimm mit meinen Augen, und ich fürchte, blind zu werden{\dots}\pend
           \pstart
           Entſchuldige, daß ich Dir gar ſo viel von mir ſpreche. Ich freue mich, zu hören, daß
               Du wieder arbeiteſt und daß Dir die Arbeit ſeeliſch gut thut. Die \label{K_L02794-7v}\edtext{Sachen\pwindex{Schnitzler, Arthur 15.05.1862 – 21.10.1931@\textsc{Schnitzler, Arthur} (15.05.1862 – 21.10.1931), \emph{Schriftsteller, Mediziner}!Reigen. Zehn Dialoge1900@\strich\emph{Reigen. Zehn Dialoge} {[}1900{]}|pwv}, mit denen Du
               beſchäftigt biſt}{\lemma{\textnormal{\emph{Sachen, … biſt}}}\Cendnote{\textnormal{Am 23. 11. 1896 begann Schnitzler\pwindex{Schnitzler, Arthur 15.05.1862 – 21.10.1931@\textsc{Schnitzler, Arthur} (15.05.1862 – 21.10.1931), \emph{Schriftsteller, Mediziner}|pwk} am \emph{Reigen}\pwindex{Schnitzler, Arthur 15.05.1862 – 21.10.1931@\textsc{Schnitzler, Arthur} (15.05.1862 – 21.10.1931), \emph{Schriftsteller, Mediziner}!Reigen. Zehn Dialoge1900@\strich\emph{Reigen. Zehn Dialoge} {[}1900{]}|pwk} zu schreiben. Enthusiasmus für dieses neue Stück\pwindex{Schnitzler, Arthur 15.05.1862 – 21.10.1931@\textsc{Schnitzler, Arthur} (15.05.1862 – 21.10.1931), \emph{Schriftsteller, Mediziner}!Reigen. Zehn Dialoge1900@\strich\emph{Reigen. Zehn Dialoge} {[}1900{]}|pwkv} klingt etwa im \emph{Tagebuch}\pwindex{Schnitzler, Arthur 15.05.1862 – 21.10.1931@\textsc{Schnitzler, Arthur} (15.05.1862 – 21.10.1931), \emph{Schriftsteller, Mediziner}!Tagebuch1981 – 2000@\strich\emph{Tagebuch} {[}1981 – 2000{]}|pwk}-Eintrag vom 27. 11. 1896 durch: »Schrieb mit Laune
                        die 4. Scene\pwindex{Schnitzler, Arthur 15.05.1862 – 21.10.1931@\textsc{Schnitzler, Arthur} (15.05.1862 – 21.10.1931), \emph{Schriftsteller, Mediziner}!Reigen. Zehn Dialoge1900@\strich\emph{Reigen. Zehn Dialoge} {[}1900{]}|pwv} des Hemic\pwindex{Schnitzler, Arthur 15.05.1862 – 21.10.1931@\textsc{Schnitzler, Arthur} (15.05.1862 – 21.10.1931), \emph{Schriftsteller, Mediziner}!Reigen. Zehn Dialoge1900@\strich\emph{Reigen. Zehn Dialoge} {[}1900{]}|pwv}.«.}}}\label{K_L02794-7h},
               dürften {\pb}Dir ſehr »liegen«. Wie denkſt Du aber doch
               über das hiſtoriſche \strikeout{Wie}{ }\label{K_L02794-9v}\edtext{Wien\oindex{Wien@\textbf{Wien}|pw}er Stück}{\lemma{\textnormal{\emph{Wiener Stück}}}\Cendnote{\textnormal{siehe A. S.: \emph{Tagebuch}, 22. 11. 1896}}}\label{K_L02794-9h}? Vielleicht mit einem jungen Componiſten, der ein Bischen alte und neue Wien\oindex{Wien@\textbf{Wien}|pw}er Muſik dazu machen würde? Würde Dich dieſe
               Abwechſelung nicht einmal reizen? Oder willſt Du fürs Erſte überhaupt kein größeres
               Stück ſchreiben? Auch das würde ich ſehr billigen. Und wann kommt Dein \label{K_L02794-11v}\edtext{Buch\pwindex{Schnitzler, Arthur 15.05.1862 – 21.10.1931@\textsc{Schnitzler, Arthur} (15.05.1862 – 21.10.1931), \emph{Schriftsteller, Mediziner}!Frau des Weisen. Novelletten1898-05-03@\strich\emph{Die Frau des Weisen. Novelletten} {[}1898-05-03{]}|pwv} bei \textsc{Fischer\orgindex{S. Fischer Verlag@S. Fischer Verlag|pw}}}{\lemma{\textnormal{\emph{Buch bei Fischer}}}\Cendnote{\textnormal{Im August 1896 vereinbarten S.
                     Fischer\pwindex{Fischer, Samuel 24.12.1859 – 15.10.1934@\textsc{Fischer, Samuel} (24.12.1859 – 15.10.1934), \emph{Verleger}|pwk} und Schnitzler\pwindex{Schnitzler, Arthur 15.05.1862 – 21.10.1931@\textsc{Schnitzler, Arthur} (15.05.1862 – 21.10.1931), \emph{Schriftsteller, Mediziner}|pwk} eine Sammlung
                  seiner Novelletten als Buch zu veröffentlichen. \emph{Die Frau des Weisen}\pwindex{Schnitzler, Arthur 15.05.1862 – 21.10.1931@\textsc{Schnitzler, Arthur} (15.05.1862 – 21.10.1931), \emph{Schriftsteller, Mediziner}!Frau des Weisen. Novelletten1898-05-03@\strich\emph{Die Frau des Weisen. Novelletten} {[}1898-05-03{]}|pwk} erschien aber erst am 3. 5. 1898.}}}\label{K_L02794-11h}?\pend
           \pstart
           Wer iſt dieſer \textsc{Stephan Grossmann\pwindex{Grossmann, Stefan 19.05.1875 – 03.01.1935@\textsc{Großmann, Stefan} (19.05.1875 – 03.01.1935), \emph{Schriftsteller, Journalist}|pw}}, den Du mir geſchickt haſt? Ich habe mich für ihn verwendet, und heut wird mir ein \label{K_L02794-12v}\edtext{Zeitungs-Ausſchnitt}{\lemma{\textnormal{\emph{Zeitungs-Ausſchnitt}}}\Cendnote{\textnormal{Mehrere Tageszeitungen berichteten über die Verhaftung des
                  Anarchisten und Journalisten Stephan
                     Großmann\pwindex{Grossmann, Stefan 19.05.1875 – 03.01.1935@\textsc{Großmann, Stefan} (19.05.1875 – 03.01.1935), \emph{Schriftsteller, Journalist}|pwk} in Berlin\oindex{Berlin@\textbf{Berlin}|pwk}. Siehe etwa
                     [O. V.]: \emph{Verhaftung eines Wiener
                        Anarchisten in Berlin}\pwindex{?? Werk@Nicht ermittelte Verfasserinnen und Verfasser!Verhaftung eines Wiener Anarchisten in Berlin1896-10-28@\emph{Verhaftung eines Wiener Anarchisten in Berlin} {[}1896-10-28{]}|pwk}. In: \emph{Arbeiter-Zeitung}\pwindex{Arbeiter-Zeitung12.7.1881 – 31.10.1991@\emph{Arbeiter-Zeitung} {[}12.7.1881 – 31.10.1991{]}|pwk}, Jg. 8, Nr. 297, 28. 10. 1896, Morgenblatt, S. 5–6.}}}\label{K_L02794-12h} geſchickt, worin
               ſteht, daß {\pb}er ſich der Berlin\oindex{Berlin@\textbf{Berlin}|pw}er Polizei\orgindex{Polizeidirektion Berlin@Polizeidirektion Berlin|pwv} als Spitzel angeboten habe. \strikeout{H\textcolor{gray}{×}\-\textcolor{gray}{×}} Ich habe ihm\pwindex{Grossmann, Stefan 19.05.1875 – 03.01.1935@\textsc{Großmann, Stefan} (19.05.1875 – 03.01.1935), \emph{Schriftsteller, Journalist}|pwv} geſagt,
               daß er, da er mit einer Empfehlung von Dir bei mir erſchienen iſt, \strikeout{v\textcolor{gray}{o}n v\textcolor{gray}{o}} in meinen Augen von vornherein gegen alle Zeitungen Recht hat. Aber er hat
               ſich \strikeout{\textcolor{gray}{m}i\textcolor{gray}{s}} ungeſchickt gerechtfertigt; das kann freilich auch Befangenheit ſein; \strikeout{imm\textcolor{gray}{e}} darum möchte ich gern in zwei Worten hören, wie Du über den Fall denkſt?\pend
           \pstart
           Iſt es wahr, daß die \label{K_L02794-14v}\edtext{»Allgemeine Zeitung\orgindex{Wiener Allgemeine Zeitung@Wiener Allgemeine Zeitung|pwv}« in andere Hände}{\lemma{\textnormal{\emph{»Allgemeine … Hände}}}\Cendnote{\textnormal{Mit Jahresende 1896 übergab
                  der mit einer Cousine\pwindex{Gans-Ludassy, Olga von 05.06.1867 – 1948-08-18@\textsc{Gans-Ludassy, Olga von} (05.06.1867 – 1948-08-18)|pwkv}{ }Schnitzler\pwindex{Schnitzler, Arthur 15.05.1862 – 21.10.1931@\textsc{Schnitzler, Arthur} (15.05.1862 – 21.10.1931), \emph{Schriftsteller, Mediziner}|pwk}s verheiratete Julius Gans-Ludassy\pwindex{Gans-Ludassy, Julius von 13.04.1858 – 30.09.1922@\textsc{Gans-Ludassy, Julius von} (13.04.1858 – 30.09.1922), \emph{Schriftsteller, Journalist, Herausgeber}|pwk} die Herausgabe der \emph{Wiener Allgemeinen Zeitung}\orgindex{Wiener Allgemeine Zeitung@Wiener Allgemeine Zeitung|pwk} an August Krawani\pwindex{Krawani, August 1829-10-06 – 1900-11-04@\textsc{Krawani, August} (1829-10-06 – 1900-11-04), \emph{Journalist, Schriftsteller}|pwk}, der zu diesem Zeitpunkt beinahe siebzig
                  Jahre alt war. Julian Sternberg\pwindex{Sternberg, Julian 08.11.1868 – 28.06. 1945@\textsc{Sternberg, Julian} (08.11.1868 – 28.06. 1945), \emph{Journalist}|pwk} war seit
                  einem Jahr als Chefredakteur im Amt und wurde am 30. 6. 1897 von Josef Münz\pwindex{Muenz, Josef @\textsc{Münz, Josef}, \emph{Journalist}|pwk}
                  abgelöst. Die Personalwechsel bedeuteten für Salten\pwindex{Salten, Felix 06.09.1869 – 08.10.1945@\textsc{Salten, Felix} (06.09.1869 – 08.10.1945), \emph{Schriftsteller, Journalist}|pwk}, der seit 1894 am Blatt\orgindex{Wiener Allgemeine Zeitung@Wiener Allgemeine Zeitung|pwkv} mitarbeitete, zu verschiedenen
                  Zeiten unterschiedliche Aufgaben, er verlor aber seine Stelle nicht.}}}\label{K_L02794-14h}
               übergeht? Was wird aus \textsc{Salten\pwindex{Salten, Felix 06.09.1869 – 08.10.1945@\textsc{Salten, Felix} (06.09.1869 – 08.10.1945), \emph{Schriftsteller, Journalist}|pw}}? {\dots}\pend
           \pstart
           Sei nochmals von ganzem Herzen bedankt für Deine treue Antheilnahme an den letzten
               Vorgängen. Tauſend herzliche Grüße! Dein \spacefill\mbox{Paul Goldmn}\pend
           \pstart
           \label{T_L02794-1v}\edtext{Grüße \textsc{Richard\pwindex{Beer-Hofmann, Richard 1866-07-11 – 1945-09-26@\textsc{Beer-Hofmann, Richard} (1866-07-11 – 1945-09-26), \emph{Schriftsteller}|pw}} und \textsc{Leo\pwindex{Van-Jung, Leo 15.10.1866 – 02.07.1939@\textsc{Van-Jung, Leo} (15.10.1866 – 02.07.1939), \emph{Gesangspädagoge, Mathematiker}|pw}}! Und schreib’ mir recht bald!}{\lemma{\textnormal{\emph{Grüße … bald!}}}\Cendnote{\textnormal{seitlich am linken Rand}}}\label{T_L02794-1h}\pend
           \pstart
           \label{T_L02794-2v}\edtext{Die \label{K_L02794-23v}\edtext{Kritiken}{\lemma{\textnormal{\emph{Kritiken}}}\Cendnote{\textnormal{Rezensionen der Uraufführung von \emph{Freiwild}\pwindex{Schnitzler, Arthur 15.05.1862 – 21.10.1931@\textsc{Schnitzler, Arthur} (15.05.1862 – 21.10.1931), \emph{Schriftsteller, Mediziner}!Freiwild. Schauspiel in 3 Akten1896@\strich\emph{Freiwild. Schauspiel in 3 Akten} {[}1896{]}|pwk}}}}\label{K_L02794-23h} ſende ich Dir demnächſt zurück}{\lemma{\textnormal{\emph{Die … zurück}}}\Cendnote{\textnormal{kopfüber am oberen Rand}}}\label{T_L02794-2h}\pend
           {\bigskip}\pstart
           \noindent{}{\pb}\label{K_L02790-8765v}\edtext{Dies iſt ein Ausſchnitt}{\lemma{\textnormal{\emph{Dies iſt ein Ausſchnitt}}}\Cendnote{\textnormal{Die Ergänzung dieses undatierten Blattes
                  zu diesem Brief muss gerechtfertigt werden. Als eigenes Korrespondenzstück wirkt
                  es zu zusammenhanglos. Es entspricht auch nicht den sonstigen Usancen der
                  Korrespondenz, derartige Petitessen separat zu senden. Die inhärente Datierung des
                  Briefs von Recha Wolff\pwindex{Wolff, Recha *~1838@\textsc{Wolff, Recha} (*~1838)|pwk} auf den Tag nach
                  einer Berlin\oindex{Berlin@\textbf{Berlin}|pwk}er Aufführung von \emph{Freiwild}\pwindex{Schnitzler, Arthur 15.05.1862 – 21.10.1931@\textsc{Schnitzler, Arthur} (15.05.1862 – 21.10.1931), \emph{Schriftsteller, Mediziner}!Freiwild. Schauspiel in 3 Akten1896@\strich\emph{Freiwild. Schauspiel in 3 Akten} {[}1896{]}|pwk} erlaubt, ihren Brief zeitlich einzugrenzen: Das
                     Stück\pwindex{Schnitzler, Arthur 15.05.1862 – 21.10.1931@\textsc{Schnitzler, Arthur} (15.05.1862 – 21.10.1931), \emph{Schriftsteller, Mediziner}!Freiwild. Schauspiel in 3 Akten1896@\strich\emph{Freiwild. Schauspiel in 3 Akten} {[}1896{]}|pwkv} stand zwischen
                     3. 11. 1896 und 16. 11. 1896 am Deutschen Theater\oindex{Deutsches Theater Berlin@\textbf{Deutsches Theater Berlin}|pwk} am
                  Programm. Entsprechend könnte der Ausschnitt jedem der Briefstücke des Novembers 1896 zugeordnet werden. Markant ist jedoch der
                  divergierende Farbton der von Goldmann\pwindex{Goldmann, Paul 31.01.1865 – 25.09.1935@\textsc{Goldmann, Paul} (31.01.1865 – 25.09.1935), \emph{Schriftsteller, Journalist}|pwk}
                  verwendeten Tinte. Am meisten stimmt er mit dem vorliegenden Brief überein, womit
                  die Zuordnung vorgenommen werden konnte.}}}\label{K_L02790-8765h} aus einem Briefe, den mein College
                  \textsc{Th. Wolff\pwindex{Wolff, Theodor 1868-08-02 – 1943-09-23@\textsc{Wolff, Theodor} (1868-08-02 – 1943-09-23), \emph{Schriftsteller, Journalist}|pw}} dieſer Tage von ſeiner Mutter\pwindex{Wolff, Recha *~1838@\textsc{Wolff, Recha} (*~1838)|pwv} erhalten hat:\pend
           \pstart
           \noindent{}{[}hs. Wolff:{]} recht zu ſagen. Gestern war ich mit \textsc{Martha\pwindex{Wolff, Marta 1871-09-06 – 1942-09-22@\textsc{Wolff, Marta} (1871-09-06 – 1942-09-22), \emph{Bildende Künstlerin >> Fotograf}|pw}} am Deutſchen Theater\oindex{Deutsches Theater Berlin@\textbf{Deutsches Theater Berlin}|pw}, \textcolor{gray}{wo}
               wir einen wirklichen Genuß hatten. »Freiwild\pwindex{Schnitzler, Arthur 15.05.1862 – 21.10.1931@\textsc{Schnitzler, Arthur} (15.05.1862 – 21.10.1931), \emph{Schriftsteller, Mediziner}!Freiwild. Schauspiel in 3 Akten1896@\strich\emph{Freiwild. Schauspiel in 3 Akten} {[}1896{]}|pw}«
               von Schnitzler iſt das Schönſte, was ich ſeit lange geſehen, und geſpielt wurde
               geradezu vollendet\textcolor{gray}{.}\pend
           
         
         \endnumbering\mylabel{h}\end{ledgroupsized}  \newcommand{\dateiname}{L02794}\newcommand{\titel}{Paul Goldmann an Arthur Schnitzler, 2. 12. [1896]}\newcommand{\editorInnen}{Martin Anton Müller und Laura Untner}%% latex-leseansicht-abspann.tex
%% Abspann für die Leseansicht.
%% Der Schalter \ifkorrekturansicht ist bereits durch den Vorspann gesetzt.

%% latex-abspann.tex
%% Gemeinsamer Abspann für Korrekturansicht und Leseansicht.
%% Setzt den Schalter \ifkorrekturansicht voraus (gesetzt in den
%% einbindenden Dateien latex-korrekturansicht-abspann.tex bzw.
%% latex-leseansicht-abspann.tex).
%% ---------------------------------------------------------------

\normalsize

% Das esempio-Environment wird nur in der Leseansicht benötigt
\ifkorrekturansicht\else
\newenvironment{esempio}[3]%
{
    \vspace{1.5ex}
    \rlap{\underline{#1}}
    \par
    \setlength{\parindent}{0cm}
    \nopagebreak
    \leftskip=#2cm
    \rightskip=#3cm
}
{
    \par
}
\fi

\doendnotes{C}
\bigskip
\vfill

\clearpage

\footnotesize

\ifkorrekturansicht
  \lohead{\textsc{register}}
\fi

% theindex-Environment neu definieren ohne reledmac
\makeatletter
\renewenvironment{theindex}{%
  \ifkorrekturansicht
    \section*{\indexname}%
  \else
    \subsubsection*{Index der erwähnten Entitäten}%
  \fi
  \setlength{\parindent}{0pt}%
  \setlength{\parskip}{0pt plus 0.3pt}%
  \let\item\@idxitem
}{%
  \ifkorrekturansicht\clearpage\fi
}
\makeatother

\IfFileExists{\jobname-pw.ind}{\input{\jobname-pw.ind}}{}

% Quellenangabe nur in der Leseansicht
\ifkorrekturansicht\else
% Fallback-Definitionen, falls die .tex-Datei \titel etc. nicht gesetzt hat
\providecommand{\titel}{}
\providecommand{\editorInnen}{}
\providecommand{\dateiname}{\jobname}

\vspace{3cm}

\vfill

\footnotesize
\textsc{Quelle}: \titel. Herausgegeben von {\editorInnen}. In: \emph{Arthur Schnitzler: Briefwechsel mit Autorinnen und Autoren}.
 Digitale Edition, https://schnitzler-briefe.acdh.oeaw.ac.at/{\dateiname}.html (Stand \today)
\fi

\end{document}


      