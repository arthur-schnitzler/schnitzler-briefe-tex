%% latex-leseansicht-vorspann.tex
%% Vorspann für die Leseansicht.
%% Lädt die gemeinsame Datei latex-vorspann.tex mit nicht gesetztem Schalter.

\newif\ifkorrekturansicht
\korrekturansichtfalse

\input{../tex-inputs/latex-vorspann}


         
         \renewcommand{\erwaehntePersonen}{Personen: Felix Salten}
         \renewcommand{\erwaehnteInstitutionen}{Institutionen: Stadttheater Teplitz}
         \renewcommand{\erwaehnteOrte}{Orte: Frankgasse 1, IX., Alsergrund, Teplice, Wien}
         \renewcommand{\erwaehnteWerke}{}
               \section[ Felix Salten an Arthur Schnitzler, 1{[}3{]}. 5. 1899]{ Felix Salten an Arthur Schnitzler, 1{[}3{]}. 5. 1899}\nopagebreak\mylabel{v}\rehead{ }\begin{ledgroupsized}[t]{13cm}\normalsize\beginnumbering\briefempfaengerindex{Schnitzler, Arthur@\textsc{Schnitzler, Arthur}!zzzSalten, Felix@\emph{von Felix Salten}!1899-05-131@{13. 5. 1899}|(be} \toendnotes[C]{\smallbreak\pagebreak[2]} \Standort{CUL, Schnitzler, B 89, A 2.}
\physDesc{Kartenbrief, 331 Zeichen
\newline{}Handschrift: Bleistift, lateinische Kurrent
\newline{}Versand: Stempel: »\nobreak{}1/1 Wien, 1{[}3{]}. 5. 99, 11–12 N\nobreak{}«. Stempel: »\nobreak{}\oindex{IX., Alsergrund@\textbf{IX., Alsergrund}|pwk}Wien 9/3 72, 14. 5. 99, 9. V, Bestellt\nobreak{}«.  
\newline{}Schnitzler: mit Bleistift datiert: »13/5 99« 
\newline{}Ordnung: mit Bleistift von unbekannter Hand nummeriert: »116« }\toendnotes[C]{\smallbreak}\pstart{}{\pb}Herrn D\textsuperscript{r} Arthur Schnitzler\pend{}\pstart{}Wien IX.\oindex{IX., Alsergrund@\textbf{IX., Alsergrund}|pw}\pend{}\pstart{}Frankgaße N\textsuperscript{o} 1\oindex{Frankgasse 1@\textbf{Frankgasse 1}|pw}\pend{}{\bigskip}\pstart{}{\pb}Lieber,\pend\pstart
           ich fahre jetzt nach Teplitz\oindex{Teplice@\textbf{Teplice}|pw}, – \label{K_L03292-1v}\edtext{vielleicht glückt es mir diesmal doch}{\lemma{\textnormal{\emph{vielleicht … doch}}}\Cendnote{\textnormal{siehe Felix Salten an Arthur Schnitzler, 6. 5. 1899}}}\label{K_L03292-1h}. Das Geld hab ich mir theilweise aufgetrieben. Ich weiß nicht, soll ich mir
                  \strikeout{diesmal} das Theater\orgindex{Stadttheater Teplitz@Stadttheater Teplitz|pwv} wünschen oder nicht.\pend
           \pstart
           Montag bin ich wieder in Wien\oindex{Wien@\textbf{Wien}|pw}, u. Montag ist auch schon alles \label{K_L03292-55v}\edtext{entschieden}{\lemma{\textnormal{\emph{entschieden}}}\Cendnote{\textnormal{Direktor wurde Emanuel
                     Raul\pwindex{\textcolor{red}{\textsuperscript{XXXX1 indx}}|pwk}.}}}\label{K_L03292-55h}.\pend
           \pstart
           Herzlichstes von Ihrem {\\[\baselineskip]}\spacefill\mbox{Salten}\pend
           \leftskip=0em{}
         
         \endnumbering\mylabel{h}\end{ledgroupsized}  \newcommand{\dateiname}{L03292}\newcommand{\titel}{Felix Salten an Arthur Schnitzler, 1[3]. 5. 1899}\newcommand{\editorInnen}{Martin Anton Müller und Laura Untner}%% latex-leseansicht-abspann.tex
%% Abspann für die Leseansicht.
%% Der Schalter \ifkorrekturansicht ist bereits durch den Vorspann gesetzt.

%% latex-abspann.tex
%% Gemeinsamer Abspann für Korrekturansicht und Leseansicht.
%% Setzt den Schalter \ifkorrekturansicht voraus (gesetzt in den
%% einbindenden Dateien latex-korrekturansicht-abspann.tex bzw.
%% latex-leseansicht-abspann.tex).
%% ---------------------------------------------------------------

\normalsize

% Das esempio-Environment wird nur in der Leseansicht benötigt
\ifkorrekturansicht\else
\newenvironment{esempio}[3]%
{
    \vspace{1.5ex}
    \rlap{\underline{#1}}
    \par
    \setlength{\parindent}{0cm}
    \nopagebreak
    \leftskip=#2cm
    \rightskip=#3cm
}
{
    \par
}
\fi

\doendnotes{C}
\bigskip
\vfill

\clearpage

\footnotesize

\ifkorrekturansicht
  \lohead{\textsc{register}}
\fi

% theindex-Environment neu definieren ohne reledmac
\makeatletter
\renewenvironment{theindex}{%
  \ifkorrekturansicht
    \section*{\indexname}%
  \else
    \subsubsection*{Index der erwähnten Entitäten}%
  \fi
  \setlength{\parindent}{0pt}%
  \setlength{\parskip}{0pt plus 0.3pt}%
  \let\item\@idxitem
}{%
  \ifkorrekturansicht\clearpage\fi
}
\makeatother

\IfFileExists{\jobname-pw.ind}{\input{\jobname-pw.ind}}{}

% Quellenangabe nur in der Leseansicht
\ifkorrekturansicht\else
% Fallback-Definitionen, falls die .tex-Datei \titel etc. nicht gesetzt hat
\providecommand{\titel}{}
\providecommand{\editorInnen}{}
\providecommand{\dateiname}{\jobname}

\vspace{3cm}

\vfill

\footnotesize
\textsc{Quelle}: \titel. Herausgegeben von {\editorInnen}. In: \emph{Arthur Schnitzler: Briefwechsel mit Autorinnen und Autoren}.
 Digitale Edition, https://schnitzler-briefe.acdh.oeaw.ac.at/{\dateiname}.html (Stand \today)
\fi

\end{document}


      