%% latex-korrekturansicht-vorspann.tex
%% Vorspann für die Korrekturansicht.
%% Lädt die gemeinsame Datei latex-vorspann.tex mit gesetztem Schalter.

\newif\ifkorrekturansicht
\korrekturansichttrue

\input{../tex-inputs/latex-vorspann}


\section[ Felix Salten an Arthur Schnitzler, 1{[}3{]}. 5. 1899]{L03292 Felix Salten an Arthur Schnitzler, 1{[}3{]}. 5. 1899}
\nopagebreak\mylabel{L03292v}
\rehead{ }\normalsize\beginnumbering\briefempfaengerindex{Schnitzler, Arthur@\textsc{Schnitzler, Arthur}!zzzSalten, Felix@\emph{von Felix Salten}!1899-05-131@{13. 5. 1899}|(be}
\toendnotes[C]{\smallbreak\pagebreak[2]}\Standort{CUL, Schnitzler, B 89, A 2.}
\physDesc{Kartenbrief, 331 Zeichen
\newline{}Handschrift: Bleistift, lateinische Kurrent
\newline{}Versand: Stempel: »\nobreak{}1/1 Wien, 1{[}3{]}. 5. 99, 11–12 N\nobreak{}«. Stempel: »\nobreak{}\oindex{IX., Alsergrund@\textbf{IX., Alsergrund}, \emph{A.ADM3}|pwk}Wien 9/3 72, 14. 5. 99, 9. V, Bestellt\nobreak{}«.  
\newline{}Schnitzler: mit Bleistift datiert: »13/5 99« 
\newline{}Ordnung: mit Bleistift von unbekannter Hand nummeriert: »116« }\toendnotes[C]{\smallbreak}\pstart{}{\pb}Herrn D\textsuperscript{r} Arthur Schnitzler\pend{}\pstart{}Wien IX.\oindex{IX., Alsergrund@\textbf{IX., Alsergrund}, \emph{A.ADM3}|pw}\pend{}\pstart{}Frankgaße N\textsuperscript{o} 1\oindex{Frankgasse 1@\textbf{Frankgasse 1}, \emph{Wohngebäude (K.WHS)}|pw}\pend{}{\bigskip}\vspace{1em}
\pstart{}{\pb}Lieber,\pend\vspace{0.5em}
\pstart
           ich fahre jetzt nach Teplitz\oindex{Teplice@\textbf{Teplice}, \emph{P.PPL}|pw}, – \label{K_L03292-1v}\edtext{vielleicht glückt es mir diesmal doch}{\lemma{\textnormal{\emph{vielleicht … doch}}}\Cendnote{\textnormal{Siehe Felix Salten an Arthur Schnitzler, 6. 5. 1899.
               }}}\label{K_L03292-1}. Das Geld hab ich mir theilweise aufgetrieben. Ich weiß nicht, soll ich mir
                  \strikeout{diesmal} das Theater\orgindex{Stadttheater Teplitz@Stadttheater Teplitz|pwv} wünschen oder nicht.\pend
           
\pstart
           Montag bin ich wieder in Wien\oindex{Wien@\textbf{Wien}, \emph{A.ADM2}|pw}, u. Montag ist auch schon alles \label{K_L03292-2v}\edtext{entschieden}{\lemma{\textnormal{\emph{entschieden}}}\Cendnote{\textnormal{Direktor wurde Emanuel
                     Raul\pwindex{Raul, Emanuel 1843-05-13 – 1916-04-19@\textsc{Raul, Emanuel} (1843-05-13 – 1916-04-19), \emph{Theaterdirektor/Theaterdirektorin}|pwk}.}}}\label{K_L03292-2}.\pend
           
\pstart
           Herzlichstes von Ihrem {\\[\baselineskip]}\spacefill\mbox{Salten}\pend
           \leftskip=0em{}\selectlanguage{ngerman}\endnumbering\briefempfaengerindex{Schnitzler, Arthur@\textsc{Schnitzler, Arthur}!zzzSalten, Felix@\emph{von Felix Salten}!1899-05-131@{13. 5. 1899}|)be}\mylabel{L03292h}  \normalsize

\doendnotes{C}
\bigskip
\vfill

\clearpage

\footnotesize

\lohead{\textsc{register}}

% Definiere theindex-Environment komplett neu ohne reledmac
\makeatletter
\renewenvironment{theindex}{%
  \section*{\indexname}%
  \setlength{\parindent}{0pt}%
  \setlength{\parskip}{0pt plus 0.3pt}%
  \let\item\@idxitem
}{%
  \clearpage
}
\makeatother

\IfFileExists{\jobname-pw.ind}{\input{\jobname-pw.ind}}{}

\end{document}

      