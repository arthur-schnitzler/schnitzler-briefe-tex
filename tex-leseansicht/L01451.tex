%% latex-leseansicht-vorspann.tex
%% Vorspann für die Leseansicht.
%% Lädt die gemeinsame Datei latex-vorspann.tex mit nicht gesetztem Schalter.

\newif\ifkorrekturansicht
\korrekturansichtfalse

\input{../tex-inputs/latex-vorspann}


\section[Arthur Schnitzler an Hugo von Hofmannsthal, 2. 10. 1904]{L01451 Arthur Schnitzler an Hugo von Hofmannsthal, 2. 10. 1904}
\nopagebreak\mylabel{L01451v}
\rehead{ }\normalsize\beginnumbering\briefempfaengerindex{Hofmannsthal, Hugo von@\textsc{Hofmannsthal, Hugo von}!zzzSchnitzler, Arthur@\emph{von Arthur Schnitzler}!1904-10-021@{2. 10. 1904}|(be}
\toendnotes[C]{\smallbreak\pagebreak[2]}
\correspDesc{Versand  durch Arthur Schnitzler am 2. 10. 1904 in Wien
\newline{}Erhalt  durch Hugo von Hofmannsthal am 3. 10. 1904 in Rodaun}\toendnotes[C]{\smallbreak}
\Standort{FDH, Hs-30885,116.}
\physDesc{Kartenbrief, 542 Zeichen
\newline{}Handschrift: schwarze Tinte, deutsche Kurrent
\newline{}Versand: 1) Stempel: »\nobreak{}\oindex{XVIII., Währing@\textbf{XVIII., Währing}, \emph{Verwaltungsgebiet}|pwk}{[}Wi{]}en 110, 3. X. 04, IX\nobreak{}«.   2) Stempel: »\nobreak{}\oindex{Wien@\textbf{Wien}!XXIII., Liesing@\textbf{XXIII., Liesing}!Rodaun@\textbf{Rodaun}, \emph{Region}|pwk}Rodaun, 3. {[}10.{]} 04\nobreak{}«. }
\buchAbdrucke{\weitereDrucke{Hugo von Hofmannsthal, Arthur Schnitzler: \emph{Briefwechsel}. Herausgegeben von Therese Nickl und Heinrich Schnitzler. Frankfurt am Main: \emph{S. Fischer} 1964, S. 203.} }\toendnotes[C]{\smallbreak}\pstart{}{\pb}\damage{Herr}n \textsc{Dr Hugo v Hofmannsthal}\pend{}\pstart{}\textsc{Rodaun \textsuperscript{b}/Liesing}\oindex{Wien@\textbf{Wien}!XXIII., Liesing@\textbf{XXIII., Liesing}!Rodaun@\textbf{Rodaun}, \emph{Region}|pw}\pend{}\pstart{}\textsc{Badgasse 5}\oindex{Wien@\textbf{Wien}!XXIII., Liesing@\textbf{XXIII., Liesing}!Badgasse@\textbf{Badgasse}, \emph{Straße}|pw}. \pend{}{\bigskip}\vspace{1em}
\pstart
           \raggedleft{}{\pb}Wien\oindex{Wien@\textbf{Wien}, \emph{Verwaltungsgebiet}|pw}, 2. 10. 904\pend
           \vspace{0.5em}
\pstart
           lieber, in d\substVorne{}\textsuperscript{er}\substDazwischen{}ieſer\substHinten{} Woche werden wir uns kaum{ }ſehen können; – es fügt{ }ſich gerade, daſs allerlei
                  zuſa{\geminationm}enko{\geminationm}t: \label{K_L01451-1v}\edtext{\textsc{Duse}\pwindex{Duse, Eleonora 3.\,10.\,1858 Vigevano – 21.\,4.\,1924 Pittsburgh@\textsc{Duse, Eleonora} (3.\,10.\,1858 Vigevano – 21.\,4.\,1924 Pittsburgh), \emph{Schauspielerin}|pw}}{\lemma{\textnormal{\emph{Duse}}}\Cendnote{\textnormal{Schnitzler besuchte am 6. 10. 1904 das Gastspiel von Eleonora Duse\pwindex{Duse, Eleonora 3.\,10.\,1858 Vigevano – 21.\,4.\,1924 Pittsburgh@\textsc{Duse, Eleonora} (3.\,10.\,1858 Vigevano – 21.\,4.\,1924 Pittsburgh), \emph{Schauspielerin}|pwk} am \emph{Theater an der Wien}\orgindex{Theater an der Wien@Theater an der Wien|pwk}. Sie spielte die Titelrolle von \emph{Die Kameliendame}\pwindex{\textcolor{red}{\textsuperscript{XXXX indx1}}!Dame aux camélias (théâtre)@\strich\emph{La Dame aux camélias (théâtre)}|pwk}.}}}\label{K_L01451-1}, Burgtheater\oindex{Wien@\textbf{Wien}!I., Innere Stadt@\textbf{I., Innere Stadt}!Burgtheater@\textbf{Burgtheater}, \emph{Theater}|pw} (\label{K_L01451-2v}\edtext{Heinrich\pwindex{\textcolor{red}{\textsuperscript{XXXX indx1}}!Henry V@\strich\emph{Henry V}|pw}}{\lemma{\textnormal{\emph{Heinrich}}}\Cendnote{\textnormal{am 8. 10. 1904}}}\label{K_L01451-2}), \label{K_L01451-3v}\edtext{Josefſtadt\oindex{Wien@\textbf{Wien}!VIII., Josefstadt@\textbf{VIII., Josefstadt}!Theater in der Josefstadt@\textbf{Theater in der Josefstadt}, \emph{Theater}|pw}}{\lemma{\textnormal{\emph{Josefstadt}}}\Cendnote{\textnormal{Am 5. 10. 1904 besuchte er \emph{Herzogin Crevette. Schauspiel in fünf Acten}\pwindex{Feydeau, Georges 8.\,12.\,1862 Paris – 5.\,6.\,1921 Rueil-Malmaison@\textsc{Feydeau, Georges} (8.\,12.\,1862 Paris – 5.\,6.\,1921 Rueil-Malmaison), \emph{Schriftsteller}!Herzogin Crevette. Schauspiel in fünf Acten@\strich\emph{Herzogin Crevette. Schauspiel in fünf Acten}|pwk} von Georges Feydeau\pwindex{Feydeau, Georges 8.\,12.\,1862 Paris – 5.\,6.\,1921 Rueil-Malmaison@\textsc{Feydeau, Georges} (8.\,12.\,1862 Paris – 5.\,6.\,1921 Rueil-Malmaison), \emph{Schriftsteller}|pwk}.}}}\label{K_L01451-3}, Familie, und{ }ſo
               müſſen wir das abendliche Hietzing\oindex{XIII., Hietzing@\textbf{XIII., Hietzing}, \emph{Verwaltungsgebiet}|pw} auf Beginn
               nächſter Woche verſchieben. Nachmittags arbeite ich{ }ſo viel als möglich. Wie iſt Ihre
               Eintheilung? Wenn man einmal in den Vormittagsſtunden nach Rodaun\oindex{Wien@\textbf{Wien}!XXIII., Liesing@\textbf{XXIII., Liesing}!Rodaun@\textbf{Rodaun}, \emph{Region}|pw} käme, (wofür ich freilich nicht garantiren kann) würde
               man Sie{ }ſtören?\pend
           
\pstart
           Die Bücher\pwindex{\textcolor{red}{\textsuperscript{XXXX indx1}}!Entweder – Oder@\strich\emph{Entweder – Oder}|pwv}\pwindex{Kunst und Künstler@\emph{Kunst und Künstler}|pwv} haben Sie
               bekommen?\pend
           
\pstart
           Von Herzen Ihr{\\[\baselineskip]}\spacefill\mbox{Arthur}\pend
           \leftskip=0em{}\selectlanguage{ngerman}\endnumbering\briefempfaengerindex{Hofmannsthal, Hugo von@\textsc{Hofmannsthal, Hugo von}!zzzSchnitzler, Arthur@\emph{von Arthur Schnitzler}!1904-10-021@{2. 10. 1904}|)be}\mylabel{L01451h}  \newcommand{\dateiname}{L01451}\newcommand{\titel}{Arthur Schnitzler an Hugo von Hofmannsthal, 2. 10. 1904}\newcommand{\editorInnen}{Martin Anton Müller und Gerd-Hermann Susen}%% latex-leseansicht-abspann.tex
%% Abspann für die Leseansicht.
%% Der Schalter \ifkorrekturansicht ist bereits durch den Vorspann gesetzt.

%% latex-abspann.tex
%% Gemeinsamer Abspann für Korrekturansicht und Leseansicht.
%% Setzt den Schalter \ifkorrekturansicht voraus (gesetzt in den
%% einbindenden Dateien latex-korrekturansicht-abspann.tex bzw.
%% latex-leseansicht-abspann.tex).
%% ---------------------------------------------------------------

\normalsize

% Das esempio-Environment wird nur in der Leseansicht benötigt
\ifkorrekturansicht\else
\newenvironment{esempio}[3]%
{
    \vspace{1.5ex}
    \rlap{\underline{#1}}
    \par
    \setlength{\parindent}{0cm}
    \nopagebreak
    \leftskip=#2cm
    \rightskip=#3cm
}
{
    \par
}
\fi

\doendnotes{C}
\bigskip
\vfill

\clearpage

\footnotesize

\ifkorrekturansicht
  \lohead{\textsc{register}}
\fi

% theindex-Environment neu definieren ohne reledmac
\makeatletter
\renewenvironment{theindex}{%
  \ifkorrekturansicht
    \section*{\indexname}%
  \else
    \subsubsection*{Index der erwähnten Entitäten}%
  \fi
  \setlength{\parindent}{0pt}%
  \setlength{\parskip}{0pt plus 0.3pt}%
  \let\item\@idxitem
}{%
  \ifkorrekturansicht\clearpage\fi
}
\makeatother

\IfFileExists{\jobname-pw.ind}{\input{\jobname-pw.ind}}{}

% Quellenangabe nur in der Leseansicht
\ifkorrekturansicht\else
% Fallback-Definitionen, falls die .tex-Datei \titel etc. nicht gesetzt hat
\providecommand{\titel}{}
\providecommand{\editorInnen}{}
\providecommand{\dateiname}{\jobname}

\vspace{3cm}

\vfill

\footnotesize
\textsc{Quelle}: \titel. Herausgegeben von {\editorInnen}. In: \emph{Arthur Schnitzler: Briefwechsel mit Autorinnen und Autoren}.
 Digitale Edition, https://schnitzler-briefe.acdh.oeaw.ac.at/{\dateiname}.html (Stand \today)
\fi

\end{document}


