%% latex-korrekturansicht-vorspann.tex
%% Vorspann für die Korrekturansicht.
%% Lädt die gemeinsame Datei latex-vorspann.tex mit gesetztem Schalter.

\newif\ifkorrekturansicht
\korrekturansichttrue

\input{../tex-inputs/latex-vorspann}


\section[Arthur Schnitzler an Hugo von Hofmannsthal, 2. 10. 1904]{L01451 Arthur Schnitzler an Hugo von Hofmannsthal, 2. 10. 1904}
\nopagebreak\mylabel{L01451v}
\rehead{ }\normalsize\beginnumbering\briefempfaengerindex{Hofmannsthal, Hugo von@\textsc{Hofmannsthal, Hugo von}!zzzSchnitzler, Arthur@\emph{von Arthur Schnitzler}!1904-10-021@{2. 10. 1904}|(be}
\toendnotes[C]{\smallbreak\pagebreak[2]}\Standort{FDH, Hs-30885,116.}
\physDesc{Kartenbrief, 542 Zeichen
\newline{}Handschrift: schwarze Tinte, deutsche Kurrent
\newline{}Versand: 1) Stempel: »\nobreak{}\oindex{XVIII., Waehring@\textbf{XVIII., Währing}, \emph{A.ADM3}|pwk}{[}Wi{]}en 110, 3. X. 04, IX\nobreak{}«.   2) Stempel: »\nobreak{}\oindex{Rodaun@\textbf{Rodaun}, \emph{A.ADM4}|pwk}Rodaun, 3. {[}10.{]} 04\nobreak{}«. }
\buchAbdrucke{\weitereDrucke{Hugo von Hofmannsthal, Arthur Schnitzler: \emph{Briefwechsel}. Frankfurt am Main: \emph{S. Fischer} 1964, S. 203.} }\toendnotes[C]{\smallbreak}\pstart{}{\pb}\damage{Herr}n \textsc{Dr Hugo v Hofmannsthal}\pend{}\pstart{}\textsc{Rodaun \textsuperscript{b}/Liesing}\oindex{Rodaun@\textbf{Rodaun}, \emph{A.ADM4}|pw}\pend{}\pstart{}\textsc{Badgasse 5}\oindex{Badgasse@\textbf{Badgasse}, \emph{Straße (K.STR)}|pw}. \pend{}{\bigskip}\vspace{1em}
\pstart
           \raggedleft{}{\pb}Wien\oindex{Wien@\textbf{Wien}, \emph{A.ADM2}|pw}, 2. 10. 904\pend
           \vspace{0.5em}
\pstart
           lieber, in d\substVorne{}\textsuperscript{er}\substDazwischen{}ieſer\substHinten{} Woche werden wir uns kaum ſehen können; – es fügt ſich gerade, daſs allerlei
                  zuſa{\geminationm}enko{\geminationm}t: \label{K_L01451-1v}\edtext{\textsc{Duse}\pwindex{Duse, Eleonora 03.10.1858 – 21.04.1924@\textsc{Duse, Eleonora} (03.10.1858 – 21.04.1924), \emph{Schauspieler/Schauspielerin}|pw}}{\lemma{\textnormal{\emph{Duse}}}\Cendnote{\textnormal{Schnitzler besuchte am 6. 10. 1904 das Gastspiel von Eleonora Duse\pwindex{Duse, Eleonora 03.10.1858 – 21.04.1924@\textsc{Duse, Eleonora} (03.10.1858 – 21.04.1924), \emph{Schauspieler/Schauspielerin}|pwk} am \emph{Theater an der Wien}\orgindex{Theater an der Wien@Theater an der Wien|pwk}. Sie spielte die Titelrolle von \emph{Die Kameliendame}\pwindex{Dame aux camelias (theâtre)@\emph{La Dame aux camélias (théâtre)}|pwk}.}}}\label{K_L01451-1}, Burgtheater\oindex{Burgtheater@\textbf{Burgtheater}, \emph{S.THTR}|pw} (\label{K_L01451-2v}\edtext{Heinrich\pwindex{Henry V@\emph{Henry V}|pw}}{\lemma{\textnormal{\emph{Heinrich}}}\Cendnote{\textnormal{am 8. 10. 1904}}}\label{K_L01451-2}), \label{K_L01451-3v}\edtext{Josefſtadt\oindex{Theater in der Josefstadt@\textbf{Theater in der Josefstadt}, \emph{Theater (K.THE)}|pw}}{\lemma{\textnormal{\emph{Josefſtadt}}}\Cendnote{\textnormal{Am 5. 10. 1904 besuchte er \emph{Herzogin Crevette. Schauspiel in fünf Acten}\pwindex{Herzogin Crevette. Schauspiel in fuenf Acten@\emph{Herzogin Crevette. Schauspiel in fünf Acten}|pwk} von Georges Feydeau\pwindex{Feydeau, Georges 08.12.1862 – 05.06.1921@\textsc{Feydeau, Georges} (08.12.1862 – 05.06.1921), \emph{Schriftsteller/Schriftstellerin}|pwk}.}}}\label{K_L01451-3}, Familie, und ſo
               müſſen wir das abendliche Hietzing\oindex{XIII., Hietzing@\textbf{XIII., Hietzing}, \emph{A.ADM3}|pw} auf Beginn
               nächſter Woche verſchieben. Nachmittags arbeite ich ſo viel als möglich. Wie iſt Ihre
               Eintheilung? Wenn man einmal in den Vormittagsſtunden nach Rodaun\oindex{Rodaun@\textbf{Rodaun}, \emph{A.ADM4}|pw} käme, (wofür ich freilich nicht garantiren kann) würde
               man Sie ſtören?\pend
           
\pstart
           Die Bücher\pwindex{Entweder – Oder@\emph{Entweder – Oder}|pwv}\pwindex{Kunst und Kuenstler@\emph{Kunst und Künstler}|pwv} haben Sie
               bekommen? \pend
           
\pstart
           Von Herzen Ihr{\\[\baselineskip]}\spacefill\mbox{Arthur}\pend
           \leftskip=0em{}\selectlanguage{ngerman}\endnumbering\briefempfaengerindex{Hofmannsthal, Hugo von@\textsc{Hofmannsthal, Hugo von}!zzzSchnitzler, Arthur@\emph{von Arthur Schnitzler}!1904-10-021@{2. 10. 1904}|)be}\mylabel{L01451h}  \normalsize

\doendnotes{C}
\bigskip
\vfill

\clearpage

\footnotesize

\lohead{\textsc{register}}

% Definiere theindex-Environment komplett neu ohne reledmac
\makeatletter
\renewenvironment{theindex}{%
  \section*{\indexname}%
  \setlength{\parindent}{0pt}%
  \setlength{\parskip}{0pt plus 0.3pt}%
  \let\item\@idxitem
}{%
  \clearpage
}
\makeatother

\IfFileExists{\jobname-pw.ind}{\input{\jobname-pw.ind}}{}

\end{document}

      