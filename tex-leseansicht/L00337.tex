%% latex-leseansicht-vorspann.tex
%% Vorspann für die Leseansicht.
%% Lädt die gemeinsame Datei latex-vorspann.tex mit nicht gesetztem Schalter.

\newif\ifkorrekturansicht
\korrekturansichtfalse

\input{../tex-inputs/latex-vorspann}


         
         \renewcommand{\erwaehntePersonen}{Personen: Lou Andreas-Salomé, Georg Brandes, Paul Goldmann}
         \renewcommand{\erwaehnteOrte}{Orte: Frankgasse, Paris, Wien}
         \renewcommand{\erwaehnteWerke}{Werke: Anatol, Das Märchen. Schauspiel in drei Aufzügen}
               \section[Arthur Schnitzler an Lou Andreas-Salomé, 13. 6. 1894]{ Arthur Schnitzler an Lou Andreas-Salomé, 13. 6. 1894}\nopagebreak\mylabel{v}\rehead{ }\begin{ledgroupsized}[t]{13cm}\normalsize\beginnumbering \toendnotes[C]{\smallbreak\pagebreak[2]} \Standort{Göttingen, Lou Andreas-Salomé Archiv, Schnitzler.}
\physDesc{Brief, 2 Blätter, 8 Seiten, 2744 Zeichen, Fragment
\newline{}Handschrift: schwarze Tinte, deutsche Kurrent}\toendnotes[C]{\smallbreak}\pstart
           \raggedleft{}{\pb}Wien, IX. \textsc{Frankgasse 1}\oindex{Frankgasse@\textbf{Frankgasse}|pw}.{\\}13. 6. 94.\pend
           \pstart{}Hochverehrte, gnädige Frau,\pend\pstart
           Sie haben Recht: ich bin über Ihren Brief verwundert geweſen. Daſs eine Frau wie Sie,
               gewohnt zwiſchen den tiefſten Problemen wie in ihrem Hausgarten ſpazieren zu wandeln,
               Zeit und Sti{\geminationm}ung fand, ſich mit den beſcheidnen Arbeiten
               eines Unbekannten zu beſchäftigen, mußte mich Wunder nehmen. Aber dieſe Verwunderung
                  {\pb}war ein Gemiſch von Stolz und Freude; – ſie iſt vorläufig der einzige
               Dank, den ich für Sie habe. – Auch überflüßig, gnädige Frau, war ihr Brief, gewiſs, –
               wie ſo vieles ſchöne und gute, ohne das man ja ſchließlich auch weiter exiſtiren
               kann, insbeſondre we{\geminationn}{ }\substVorne{}\textsuperscript{d}\substDazwischen{}m\substHinten{}an es gar nicht erhofft hat. Iſt es aber einmal da, ſo beglückt es ja doch
               tauſendmal mehr als manches noth{\pb}wendige, ohne das man zu Grunde gehen
               müſſte. Sie ſprechen von ſich als von einer Stimme aus dem Publikum und mögen ja
               Recht haben, daſs ſolche Sti{\geminationm}en im allgemeinen wenig
               Freude machen; aber Sie müſſen doch einige Ausnahmen gelten laſſen. Sie machen Freude
               – erſtens we{\geminationn}{ }ſie loben, zweitens we{\geminationn}{ }{\pb}man noch nicht ſonderlich verwöhnt iſt und drittens, we{\geminationn}{ }ſie zufällig jemandem angehören, den man ſeit
               langem kennt und verehrt. Ermeſſen Sie daraus, geſchätzte Sti{\geminationm}e aus dem Publikum, wie herrlich Sie mir erklungen
               ſind! Ein Zufall hat es gefügt, daſs ich gleichzeitig mit dem Ihren einen Brief von
                  \textsc{Georg Brandes}\pwindex{Brandes, Georg 04.02.1842 – 19.02.1927@\textsc{Brandes, Georg} (04.02.1842 – 19.02.1927)|pw} erhielt, der mir im {\pb}Vergleich zu dem Ihren insbeſondre dadurch
               intereſſant iſt, daſs er im Gegenſatz zu Ihnen das »Märchen\pwindex{Schnitzler, Arthur 15.05.1862 – 21.10.1931@\textsc{Schnitzler, Arthur} (15.05.1862 – 21.10.1931), \emph{Schriftsteller, Mediziner}!Maerchen. Schauspiel in drei Aufzuegen1893-12-01@\strich\emph{Das Märchen. Schauspiel in drei Aufzügen} {[}1893-12-01{]}|pw}« ganz beträchtlich über den »Anatol\pwindex{Schnitzler, Arthur 15.05.1862 – 21.10.1931@\textsc{Schnitzler, Arthur} (15.05.1862 – 21.10.1931), \emph{Schriftsteller, Mediziner}!Anatol1892-10-29@\strich\emph{Anatol} {[}1892-10-29{]}|pw}« ſtellt. Ich ſelbſt glaube, daſs im Märchen\pwindex{Schnitzler, Arthur 15.05.1862 – 21.10.1931@\textsc{Schnitzler, Arthur} (15.05.1862 – 21.10.1931), \emph{Schriftsteller, Mediziner}!Maerchen. Schauspiel in drei Aufzuegen1893-12-01@\strich\emph{Das Märchen. Schauspiel in drei Aufzügen} {[}1893-12-01{]}|pw} mehr gutes ſteckt als im Anatol\pwindex{Schnitzler, Arthur 15.05.1862 – 21.10.1931@\textsc{Schnitzler, Arthur} (15.05.1862 – 21.10.1931), \emph{Schriftsteller, Mediziner}!Anatol1892-10-29@\strich\emph{Anatol} {[}1892-10-29{]}|pw}, – daſs aber einzelne \substVorne{}\textsuperscript{aus}\substDazwischen{}von\substHinten{} den Anatol\pwindex{Schnitzler, Arthur 15.05.1862 – 21.10.1931@\textsc{Schnitzler, Arthur} (15.05.1862 – 21.10.1931), \emph{Schriftsteller, Mediziner}!Anatol1892-10-29@\strich\emph{Anatol} {[}1892-10-29{]}|pw}ſcenen als ganzes gelungener
               ſind. Auch weiſs ich nicht, ob man den Fedor Denner\pwindex{Schnitzler, Arthur 15.05.1862 – 21.10.1931@\textsc{Schnitzler, Arthur} (15.05.1862 – 21.10.1931), \emph{Schriftsteller, Mediziner}!Maerchen. Schauspiel in drei Aufzuegen1893-12-01@\strich\emph{Das Märchen. Schauspiel in drei Aufzügen} {[}1893-12-01{]}|pwv} wirklich für überſpannt {\pb}und ſeine
               Empfindung für ſo verzwickt und widerſpruchsvoll halten muſs? Mich dünkt, aber ganze
               Wirrniſs liegt darin, daſs er theoretiſch eine Frage längſt abgethan hat, der er in
               einem concreten Fall noch nicht gewachſen iſt; – er widerſpricht ſich eigentlich
               nicht, er hat ſich nur ſelber misverſtanden. – Auf Ihre vielen freundlichen und
                  auszeich{\pb}nende Worte habe ich natürlich keine Einwendung übrig; aber
               ich ka{\geminationn} es nicht läugnen, daſs ich bei einigen Ihrer
               allzuliebenswürdigen Bemerkungen die gewiſſe Empfindung des Beſchämtſeins hatte wie
               gegenüber Lobſprüchen, die man ja wohl einmal zu verdienen hofft, die aber
               überraſchend und unerwartet Früh geko{\geminationm}en ſind.\pend
           \pstart
           {\pb}Daſs an Ihrem Schreiben, gnädige Frau mein Freund \textsc{Paul Goldmann}\pwindex{Goldmann, Paul 31.01.1865 – 25.09.1935@\textsc{Goldmann, Paul} (31.01.1865 – 25.09.1935), \emph{Schriftsteller, Journalist}|pw} nicht ohne Schuld iſt, brauchen Sie kaum zu ſagen: er trägt die Schuld beinahe
               an allem erfreulichem, das mir in den letzten Jahren begegnet iſt. Ihr Brief gehört
               nun zu den allererfreulichſsten Dingen, die mir paſſiren konnten – und da Sie ſich
               ſelbſt aus den Reihen derjenigen weg {[}Ende des Fragments{]}\pend
           
         
         \endnumbering\mylabel{h}\end{ledgroupsized}  \newcommand{\dateiname}{L00337}\newcommand{\titel}{Arthur Schnitzler an Lou Andreas-Salomé, 13. 6. 1894}\newcommand{\editorInnen}{Martin Anton Müller und Gerd-Hermann Susen}%% latex-leseansicht-abspann.tex
%% Abspann für die Leseansicht.
%% Der Schalter \ifkorrekturansicht ist bereits durch den Vorspann gesetzt.

%% latex-abspann.tex
%% Gemeinsamer Abspann für Korrekturansicht und Leseansicht.
%% Setzt den Schalter \ifkorrekturansicht voraus (gesetzt in den
%% einbindenden Dateien latex-korrekturansicht-abspann.tex bzw.
%% latex-leseansicht-abspann.tex).
%% ---------------------------------------------------------------

\normalsize

% Das esempio-Environment wird nur in der Leseansicht benötigt
\ifkorrekturansicht\else
\newenvironment{esempio}[3]%
{
    \vspace{1.5ex}
    \rlap{\underline{#1}}
    \par
    \setlength{\parindent}{0cm}
    \nopagebreak
    \leftskip=#2cm
    \rightskip=#3cm
}
{
    \par
}
\fi

\doendnotes{C}
\bigskip
\vfill

\clearpage

\footnotesize

\ifkorrekturansicht
  \lohead{\textsc{register}}
\fi

% theindex-Environment neu definieren ohne reledmac
\makeatletter
\renewenvironment{theindex}{%
  \ifkorrekturansicht
    \section*{\indexname}%
  \else
    \subsubsection*{Index der erwähnten Entitäten}%
  \fi
  \setlength{\parindent}{0pt}%
  \setlength{\parskip}{0pt plus 0.3pt}%
  \let\item\@idxitem
}{%
  \ifkorrekturansicht\clearpage\fi
}
\makeatother

\IfFileExists{\jobname-pw.ind}{\input{\jobname-pw.ind}}{}

% Quellenangabe nur in der Leseansicht
\ifkorrekturansicht\else
% Fallback-Definitionen, falls die .tex-Datei \titel etc. nicht gesetzt hat
\providecommand{\titel}{}
\providecommand{\editorInnen}{}
\providecommand{\dateiname}{\jobname}

\vspace{3cm}

\vfill

\footnotesize
\textsc{Quelle}: \titel. Herausgegeben von {\editorInnen}. In: \emph{Arthur Schnitzler: Briefwechsel mit Autorinnen und Autoren}.
 Digitale Edition, https://schnitzler-briefe.acdh.oeaw.ac.at/{\dateiname}.html (Stand \today)
\fi

\end{document}


      