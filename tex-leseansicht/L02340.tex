%% latex-leseansicht-vorspann.tex
%% Vorspann für die Leseansicht.
%% Lädt die gemeinsame Datei latex-vorspann.tex mit nicht gesetztem Schalter.

\newif\ifkorrekturansicht
\korrekturansichtfalse

\input{../tex-inputs/latex-vorspann}


\section[Hugo Hofmannsthal an Olga Schnitzler, 17. 4. 1920]{L02340 Hugo Hofmannsthal an Olga Schnitzler, 17. 4. 1920}
\nopagebreak\mylabel{L02340v}
\rehead{ }\normalsize\beginnumbering\briefempfaengerindex{Schnitzler, Olga@\textsc{Schnitzler, Olga}!zzzHofmannsthal, Hugo von@\emph{von Hugo von Hofmannsthal}!1920-04-172@{17. 4. 1920}|(be}
\toendnotes[C]{\smallbreak\pagebreak[2]}
\correspDesc{Versand  durch Hugo von Hofmannsthal am 17. 4. 1920 in Rodaun
\newline{}Erhalt  durch Olga Schnitzler im Zeitraum [18. 4. 1920
                  – 22. 4. 1920?] in Wien}\toendnotes[C]{\smallbreak}
\Standort{DLA, A:Schnitzler, HS.NZ85.1.5584.}
\physDesc{Brief, 1 Blatt, 4 Seiten, Kuvert, 1135 Zeichen (Briefpapier mit Trauerrand)
\newline{}Handschrift: schwarze Tinte, deutsche Kurrent
\newline{}Versand: Stempel: »\nobreak{}\oindex{Wien@\textbf{Wien}!XXIII., Liesing@\textbf{XXIII., Liesing}!Rodaun@\textbf{Rodaun}, \emph{Region}|pwk}Rodaun, 17{[}. 4. 1920{]}\nobreak{}«.  }\toendnotes[C]{\smallbreak}\pstart{}\textsc{{\pb}Hofma{\geminationn}sthal}\pend{}\pstart{}\textsc{Rodaun\oindex{Wien@\textbf{Wien}!XXIII., Liesing@\textbf{XXIII., Liesing}!Rodaun@\textbf{Rodaun}, \emph{Region}|pw}.}\pend{}{\bigskip}\pstart{}\textsc{Frau Olga Schnitzler}\pend{}\pstart{}\textsc{Wien\oindex{Wien@\textbf{Wien}, \emph{Verwaltungsgebiet}|pw}}\pend{}\pstart{}\textsc{XVIII. Sternwartestrasse \substVorne{}\textsuperscript{5}\substDazwischen{}7\substHinten{}1\oindex{Wien@\textbf{Wien}!XVIII., Währing@\textbf{XVIII., Währing}!Sternwartestraße 71@\textbf{Sternwartestraße 71}, \emph{Wohngebäude}|pw}.}\pend{}{\bigskip}\vspace{1em}
\pstart
           \raggedleft{}{\pb}Rodaun\oindex{Wien@\textbf{Wien}!XXIII., Liesing@\textbf{XXIII., Liesing}!Rodaun@\textbf{Rodaun}, \emph{Region}|pw}{\\}17. IV.\pend
           
\pstart{}liebe Olga\pend\vspace{0.5em}
\pstart
           mit Schmerz hab ich erfahren, daſs Ihre gute liebe Schweſter\pwindex{Steinrück, Elisabeth 19.\,11.\,1885 – 7.\,4.\,1920 Partenkirchen@\textsc{Steinrück, Elisabeth} (19.\,11.\,1885 – 7.\,4.\,1920 Partenkirchen)|pwv} von dieſer finſteren Welt und uns allen auf immer
               fortgegangen iſt. Wie freundlich wäre es,{ }ſie noch immer unter den Lebenden zu
               wiſſen. Es{ }ſchien mir eine Güte von ihr, daſs{ }ſie immer noch dableiben wollte. Dieſes
               unvergleichliche, rührende Weſen\pwindex{Steinrück, Elisabeth 19.\,11.\,1885 – 7.\,4.\,1920 Partenkirchen@\textsc{Steinrück, Elisabeth} (19.\,11.\,1885 – 7.\,4.\,1920 Partenkirchen)|pwv} – ich habe{ }ſie ja, {\pb}würde man{ }ſagen,
               nur wenig gekannt: und doch, wie{ }ſehr iſt{ }ſie auch mir geſtorben! – und davon gibt
               mein innerſtes Gefühl mit nachhaltigem Schmerz mir{ }ſelber Zeugnis. Man brauchte ihr
               nur manchmal begegnet zu{ }ſein – mit welcher zarten feinen unauslöſchbaren Schrift{ }ſchrieb{ }ſich dieſes Weſen\pwindex{Steinrück, Elisabeth 19.\,11.\,1885 – 7.\,4.\,1920 Partenkirchen@\textsc{Steinrück, Elisabeth} (19.\,11.\,1885 – 7.\,4.\,1920 Partenkirchen)|pwv}
               einem ins Herz. Sie haben{ }ſo viel {\pb}verloren – mehr als
               irgend jemand{ }ſicherlich, denn Sie waren die frühen Jahre mit ihr verbunden:{ }ſo fällt
               für Sie{ }ſo nichts zugleich dahin.\pend
           
\pstart
           Wie viel aber auch Arthur verloren hat, was für
               eine gute zärtliche Freundin, das kann ich ahnen – ermeſſen kann ja ein Dritter{ }ſolche Dinge nie. Sagen Sie es ihm, daſs ich oft u. oft an ihn denke.\pend
           
\pstart
           {\pb}Ich bin, liebe Olga, in alter Freundſchaft\pend
           
\pstart
           Herzlich Ihr{\\[\baselineskip]}\spacefill\mbox{Hugo H.}\pend
           \leftskip=0em{}\selectlanguage{ngerman}\endnumbering\briefempfaengerindex{Schnitzler, Olga@\textsc{Schnitzler, Olga}!zzzHofmannsthal, Hugo von@\emph{von Hugo von Hofmannsthal}!1920-04-172@{17. 4. 1920}|)be}\mylabel{L02340h}  \newcommand{\dateiname}{L02340}\newcommand{\titel}{Hugo Hofmannsthal an Olga Schnitzler, 17. 4. 1920}\newcommand{\editorInnen}{Herausgegeben von Martin Anton Müller}%% latex-leseansicht-abspann.tex
%% Abspann für die Leseansicht.
%% Der Schalter \ifkorrekturansicht ist bereits durch den Vorspann gesetzt.

%% latex-abspann.tex
%% Gemeinsamer Abspann für Korrekturansicht und Leseansicht.
%% Setzt den Schalter \ifkorrekturansicht voraus (gesetzt in den
%% einbindenden Dateien latex-korrekturansicht-abspann.tex bzw.
%% latex-leseansicht-abspann.tex).
%% ---------------------------------------------------------------

\normalsize

% Das esempio-Environment wird nur in der Leseansicht benötigt
\ifkorrekturansicht\else
\newenvironment{esempio}[3]%
{
    \vspace{1.5ex}
    \rlap{\underline{#1}}
    \par
    \setlength{\parindent}{0cm}
    \nopagebreak
    \leftskip=#2cm
    \rightskip=#3cm
}
{
    \par
}
\fi

\doendnotes{C}
\bigskip
\vfill

\clearpage

\footnotesize

\ifkorrekturansicht
  \lohead{\textsc{register}}
\fi

% theindex-Environment neu definieren ohne reledmac
\makeatletter
\renewenvironment{theindex}{%
  \ifkorrekturansicht
    \section*{\indexname}%
  \else
    \subsubsection*{Index der erwähnten Entitäten}%
  \fi
  \setlength{\parindent}{0pt}%
  \setlength{\parskip}{0pt plus 0.3pt}%
  \let\item\@idxitem
}{%
  \ifkorrekturansicht\clearpage\fi
}
\makeatother

\IfFileExists{\jobname-pw.ind}{\input{\jobname-pw.ind}}{}

% Quellenangabe nur in der Leseansicht
\ifkorrekturansicht\else
% Fallback-Definitionen, falls die .tex-Datei \titel etc. nicht gesetzt hat
\providecommand{\titel}{}
\providecommand{\editorInnen}{}
\providecommand{\dateiname}{\jobname}

\vspace{3cm}

\vfill

\footnotesize
\textsc{Quelle}: \titel. Herausgegeben von {\editorInnen}. In: \emph{Arthur Schnitzler: Briefwechsel mit Autorinnen und Autoren}.
 Digitale Edition, https://schnitzler-briefe.acdh.oeaw.ac.at/{\dateiname}.html (Stand \today)
\fi

\end{document}


