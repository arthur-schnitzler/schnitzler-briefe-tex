%% latex-leseansicht-vorspann.tex
%% Vorspann für die Leseansicht.
%% Lädt die gemeinsame Datei latex-vorspann.tex mit nicht gesetztem Schalter.

\newif\ifkorrekturansicht
\korrekturansichtfalse

\input{../tex-inputs/latex-vorspann}


         
         \renewcommand{\erwaehntePersonen}{Personen: Olga Schnitzler, Elisabeth Steinrück}
         \renewcommand{\erwaehnteOrte}{Orte: Rodaun, Sternwartestraße, Wien}
         \renewcommand{\erwaehnteWerke}{}
               \section[Hugo von Hofmannsthal an Olga Schnitzler, 17. 4. 1920]{ Hugo von Hofmannsthal an Olga Schnitzler, 17. 4. 1920}\nopagebreak\mylabel{v}\rehead{ }\begin{ledgroupsized}[t]{13cm}\normalsize\beginnumbering \toendnotes[C]{\smallbreak\pagebreak[2]} \Standort{DLA, A:Schnitzler, HS.NZ85.1.5584.}
\physDesc{Brief, 1 Blatt, 4 Seiten, Umschlag, 1135 Zeichen (Briefpapier mit Trauerrand)
\newline{}Handschrift: 1) schwarze Tinte, deutsche Kurrent\hspace{1em}2) schwarze Tinte, lateinische Kurrent (\noindent{}Adresse)\hspace{1em}
\newline{}Versand: Stempel: »\nobreak{}\oindex{Rodaun@\textbf{Rodaun}|pwk}Rodaun, 17{[}. 4. 1920{]}\nobreak{}«.  }\toendnotes[C]{\smallbreak}\pstart{}{\pb}Hofma{\geminationn}sthal\pend{}\pstart{}Rodaun\oindex{Rodaun@\textbf{Rodaun}|pw}.\pend{}{\bigskip}\pstart{}Frau Olga Schnitzler\pend{}\pstart{}Wien\oindex{Wien@\textbf{Wien}|pw}\pend{}\pstart{}XVIII. Sternwartestrasse \substVorne{}\textsuperscript{5}\substDazwischen{}7\substHinten{}1\oindex{Sternwartestrasse@\textbf{Sternwartestraße}|pw}.\pend{}{\bigskip}\pstart
           \raggedleft{}{\pb}Rodaun\oindex{Rodaun@\textbf{Rodaun}|pw}{\\}17. IV.\pend
           \pstart{}liebe Olga\pend\pstart
           mit Schmerz hab ich erfahren, daſs Ihre gute liebe Schweſter\pwindex{Steinrueck, Elisabeth 19.11.1885 – 07.04.1920@\textsc{Steinrück, Elisabeth} (19.11.1885 – 07.04.1920)|pwv} von dieſer finſteren Welt und uns allen auf immer
               fortgegangen iſt. Wie freundlich wäre es, ſie noch immer unter den Lebenden zu
               wiſſen. Es ſchien mir eine Güte von ihr, daſs ſie immer noch dableiben wollte. Dieſes
               unvergleichliche, rührende Weſen\pwindex{Steinrueck, Elisabeth 19.11.1885 – 07.04.1920@\textsc{Steinrück, Elisabeth} (19.11.1885 – 07.04.1920)|pwv} – ich habe ſie ja, {\pb}würde man ſagen,
               nur wenig gekannt: und doch, wie ſehr iſt ſie auch mir geſtorben! – und davon gibt
               mein innerſtes Gefühl mit nachhaltigem Schmerz mir ſelber Zeugnis. Man brauchte ihr
               nur manchmal begegnet zu ſein – mit welcher zarten feinen unauslöſchbaren Schrift
               ſchrieb ſich dieſes Weſen\pwindex{Steinrueck, Elisabeth 19.11.1885 – 07.04.1920@\textsc{Steinrück, Elisabeth} (19.11.1885 – 07.04.1920)|pwv}
               einem ins Herz. Sie haben ſo viel {\pb}verloren – mehr als
               irgend jemand ſicherlich, denn Sie waren die frühen Jahre mit ihr verbunden: ſo fällt
               für Sie ſo nichts zugleich dahin.\pend
           \pstart
           Wie viel aber auch Arthur\pwindex{Schnitzler, Arthur 15.05.1862 – 21.10.1931@\textsc{Schnitzler, Arthur} (15.05.1862 – 21.10.1931), \emph{Schriftsteller, Mediziner}|pw} verloren hat, was für
               eine gute zärtliche Freundin, das kann ich ahnen – ermeſſen kann ja ein Dritter
               ſolche Dinge nie. Sagen Sie es ihm, daſs ich oft u. oft an ihn denke.\pend
           \pstart
           {\pb}Ich bin, liebe Olga, in alter Freundſchaft\pend
           \pstart
           Herzlich Ihr{\\[\baselineskip]}\spacefill\mbox{Hugo H.}\pend
           \leftskip=0em{}
         
         \endnumbering\mylabel{h}\end{ledgroupsized}  \newcommand{\dateiname}{L02340}\newcommand{\titel}{Hugo von Hofmannsthal an Olga Schnitzler, 17. 4. 1920}\newcommand{\editorInnen}{ Martin Anton Müller und Gerd-Hermann Susen}%% latex-leseansicht-abspann.tex
%% Abspann für die Leseansicht.
%% Der Schalter \ifkorrekturansicht ist bereits durch den Vorspann gesetzt.

%% latex-abspann.tex
%% Gemeinsamer Abspann für Korrekturansicht und Leseansicht.
%% Setzt den Schalter \ifkorrekturansicht voraus (gesetzt in den
%% einbindenden Dateien latex-korrekturansicht-abspann.tex bzw.
%% latex-leseansicht-abspann.tex).
%% ---------------------------------------------------------------

\normalsize

% Das esempio-Environment wird nur in der Leseansicht benötigt
\ifkorrekturansicht\else
\newenvironment{esempio}[3]%
{
    \vspace{1.5ex}
    \rlap{\underline{#1}}
    \par
    \setlength{\parindent}{0cm}
    \nopagebreak
    \leftskip=#2cm
    \rightskip=#3cm
}
{
    \par
}
\fi

\doendnotes{C}
\bigskip
\vfill

\clearpage

\footnotesize

\ifkorrekturansicht
  \lohead{\textsc{register}}
\fi

% theindex-Environment neu definieren ohne reledmac
\makeatletter
\renewenvironment{theindex}{%
  \ifkorrekturansicht
    \section*{\indexname}%
  \else
    \subsubsection*{Index der erwähnten Entitäten}%
  \fi
  \setlength{\parindent}{0pt}%
  \setlength{\parskip}{0pt plus 0.3pt}%
  \let\item\@idxitem
}{%
  \ifkorrekturansicht\clearpage\fi
}
\makeatother

\IfFileExists{\jobname-pw.ind}{\input{\jobname-pw.ind}}{}

% Quellenangabe nur in der Leseansicht
\ifkorrekturansicht\else
% Fallback-Definitionen, falls die .tex-Datei \titel etc. nicht gesetzt hat
\providecommand{\titel}{}
\providecommand{\editorInnen}{}
\providecommand{\dateiname}{\jobname}

\vspace{3cm}

\vfill

\footnotesize
\textsc{Quelle}: \titel. Herausgegeben von {\editorInnen}. In: \emph{Arthur Schnitzler: Briefwechsel mit Autorinnen und Autoren}.
 Digitale Edition, https://schnitzler-briefe.acdh.oeaw.ac.at/{\dateiname}.html (Stand \today)
\fi

\end{document}


      