%% latex-korrekturansicht-vorspann.tex
%% Vorspann für die Korrekturansicht.
%% Lädt die gemeinsame Datei latex-vorspann.tex mit gesetztem Schalter.

\newif\ifkorrekturansicht
\korrekturansichttrue

\input{../tex-inputs/latex-vorspann}


\section[ Felix Salten an Arthur Schnitzler, 7. 9. 1919]{L03568 Felix Salten an Arthur Schnitzler, 7. 9. 1919}
\nopagebreak\mylabel{L03568v}
\rehead{ }\normalsize\beginnumbering\briefempfaengerindex{Schnitzler, Arthur@\textsc{Schnitzler, Arthur}!zzzSalten, Felix@\emph{von Felix Salten}!1919-09-071@{7. 9. 1919}|(be}
\toendnotes[C]{\smallbreak\pagebreak[2]}\Standort{CUL, Schnitzler, B 89, B 2.}
\physDesc{Briefkarte, 1037 Zeichen
\newline{}Handschrift: schwarze Tinte, lateinische Kurrent
\newline{}Schnitzler: mit rotem Buntstift eine Unterstreichung 
\newline{}Ordnung: 1) mit Bleistift von Frieda Pollak\pwindex{Pollak, Frieda 08.12.1881 – 13.07.1937@\textsc{Pollak, Frieda} (08.12.1881 – 13.07.1937), \emph{Sekretär/Sekretärin}|pw} (?) mit
                                 dem Buchstaben »A« (Abgeschrieben/Abschrift)
                                 gekennzeichnet  2) mit Bleistift von unbekannter Hand nummeriert: »281«}\toendnotes[C]{\smallbreak}
\pstart
           \raggedleft{}{\pb}7. 9. 19\pend
           
\pstart
           \raggedleft{}Berghof\oindex{Berghof@\textbf{Berghof}, \emph{Wohngebäude (K.WHS)}|pw}.\pend
           
\pstart{}Lieber,\pend\vspace{0.5em}
\pstart
           herzlichen Dank für Ihr \label{K_L03568-1v}\edtext{Telegramm}{\lemma{\textnormal{\emph{Telegramm}}}\Cendnote{\textnormal{Salten\pwindex{Salten, Felix 06.09.1869 – 08.10.1945@\textsc{Salten, Felix} (06.09.1869 – 08.10.1945), \emph{Schriftsteller/Schriftstellerin, Journalist/Journalistin, Chefredakteur/Chefredakteurin}|pwk} hatte am Vortag seinen 50. Geburtstag gefeiert.}}}\label{K_L03568-1}. Wie sehr es mir wohltut
               und mich freut, brauche ich eigentlich kaum zu sagen, möchte es aber doch sagen, um
               Ihnen aufrichtig zu danken. Besonders auch dafür, dass meine Zuneigung, meine
               Verehrung und meine Freundschaft für Sie im Laufe des Lebens nur immer fester,
               überzeugter und inniger werden konnten, und dass auch Sie mir Ihre gute Gesinnung so
               bewahrt haben. Das bleibt nun so, denke ich, ohne der Worte zu bedürfen. Sie haben
               Recht: laßen Sie uns die Stücke Weges noch öfter und näher beisammen bleiben. An mir
               soll’s {\pb}nicht fehlen.\pend
           
\pstart
           Von Olga\pwindex{Schnitzler, Olga 17.01.1882 – 13.01.1970@\textsc{Schnitzler, Olga} (17.01.1882 – 13.01.1970), \emph{Schauspieler/Schauspielerin, Sänger/Sängerin}|pw} bekam ich gestern ein liebes Telegramm. Ich hoffe, sie am Dienstag noch in Salzburg\oindex{Salzburg@\textbf{Salzburg}, \emph{A.ADM2}|pw} sehen und ihr
               danken zu können. Aus etlichen anderen Telegrammen, die heute kamen, wird mir die Befürchtung, es habe in irgend einer \label{K_L03568-2v}\edtext{Wien\oindex{Wien@\textbf{Wien}, \emph{A.ADM2}|pw}er Zeitung von meinem 50. Geburtstag}{\lemma{\textnormal{\emph{Wiener … meinem 50. Geburtstag}}}\Cendnote{\textnormal{Tatsächlich stand es in der Zeitung: [O. V.]: \emph{Felix
                        Saltens fünfzigster Geburtstag}\pwindex{Felix Saltens fuenfzigster Geburtstag@\emph{Felix Saltens fünfzigster Geburtstag}|pwk}. In: \emph{Neue Freie Presse}\pwindex{Neue Freie Presse@\emph{Neue Freie Presse}|pwk}, Nr. 19.768, 6. 9. 1919, Morgenblatt, S. 7. Darin wurde jedoch behauptet,
                  der Geburtstag wäre »morgen«.}}}\label{K_L03568-2} gestanden. Das wäre mir sehr unangenehm!! Donnerstag{ }Abend will ich in Wien\oindex{Wien@\textbf{Wien}, \emph{A.ADM2}|pw} sein. Also,
               auf recht baldiges \label{K_L03568-3v}\edtext{Wiedersehen}{\lemma{\textnormal{\emph{Wiedersehen}}}\Cendnote{\textnormal{Nachweislich sahen sie sich am 18. 9. 1919
                  wieder.}}}\label{K_L03568-3}, nochmals: Danke, und viele herzliche Grüße von Otti\pwindex{Salten, Ottilie 07.03.1868 – 22.06.1942@\textsc{Salten, Ottilie} (07.03.1868 – 22.06.1942), \emph{Schauspieler/Schauspielerin}|pw} wie von mir.\pend
           
\pstart
           Ihr {\\[\baselineskip]}\spacefill\mbox{Felix S.}\pend
           \leftskip=0em{}\selectlanguage{ngerman}\endnumbering\briefempfaengerindex{Schnitzler, Arthur@\textsc{Schnitzler, Arthur}!zzzSalten, Felix@\emph{von Felix Salten}!1919-09-071@{7. 9. 1919}|)be}\mylabel{L03568h}  \normalsize

\doendnotes{C}
\bigskip
\vfill

\clearpage

\footnotesize

\lohead{\textsc{register}}

% Definiere theindex-Environment komplett neu ohne reledmac
\makeatletter
\renewenvironment{theindex}{%
  \section*{\indexname}%
  \setlength{\parindent}{0pt}%
  \setlength{\parskip}{0pt plus 0.3pt}%
  \let\item\@idxitem
}{%
  \clearpage
}
\makeatother

\IfFileExists{\jobname-pw.ind}{\input{\jobname-pw.ind}}{}

\end{document}

      