%% latex-leseansicht-vorspann.tex
%% Vorspann für die Leseansicht.
%% Lädt die gemeinsame Datei latex-vorspann.tex mit nicht gesetztem Schalter.

\newif\ifkorrekturansicht
\korrekturansichtfalse

\input{../tex-inputs/latex-vorspann}


\section[ Felix Salten an Arthur Schnitzler, 7. 9. 1919]{L03568 Felix Salten an Arthur Schnitzler,  7. 9. 1919}
\nopagebreak\mylabel{L03568v}
\rehead{ }\normalsize\beginnumbering\briefempfaengerindex{Schnitzler, Arthur@\textsc{Schnitzler, Arthur}!zzzSalten, Felix@\emph{von Felix Salten}!1919-09-071@{7. 9. 1919}|(be}
\toendnotes[C]{\smallbreak\pagebreak[2]}
\correspDesc{Versand  durch Felix Salten am 7. 9. 1919 in Unterach am Attersee
\newline{}Erhalt  durch Arthur Schnitzler im Zeitraum [8. 9. 1919
                  – 12. 9. 1919?] in Wien}\toendnotes[C]{\smallbreak}
\Standort{CUL, Schnitzler, B 89, B 2.}
\physDesc{Briefkarte, 1037 Zeichen
\newline{}Handschrift: schwarze Tinte, lateinische Kurrent
\newline{}Schnitzler: mit rotem Buntstift eine Unterstreichung 
\newline{}Ordnung: 1) mit Bleistift von Frieda Pollak\pwindex{Pollak, Frieda 8.\,12.\,1881 Wien – 13.\,7.\,1937 ebd.@\textsc{Pollak, Frieda} (8.\,12.\,1881 Wien – 13.\,7.\,1937 ebd.), \emph{Sekretärin}|pw} (?) mit
                                 dem Buchstaben »A« (Abgeschrieben/Abschrift)
                                 gekennzeichnet  2) mit Bleistift von unbekannter Hand nummeriert: »281«}\toendnotes[C]{\smallbreak}
\pstart
           \raggedleft{}{\pb}7. 9. 19\pend
           
\pstart
           \raggedleft{}Berghof\oindex{Berghof@\textbf{Berghof}, \emph{Wohngebäude}|pw}.\pend
           
\pstart{}Lieber,\pend\vspace{0.5em}
\pstart
           herzlichen Dank für Ihr \label{K_L03568-1v}\edtext{Telegramm}{\lemma{\textnormal{\emph{Telegramm}}}\Cendnote{\textnormal{Salten\pwindex{Salten, Felix 6.\,9.\,1869 Budapest – 8.\,10.\,1945 Zürich@\textsc{Salten, Felix} (6.\,9.\,1869 Budapest – 8.\,10.\,1945 Zürich), \emph{Schriftsteller, Journalist, Chefredakteur}|pwk} hatte am Vortag seinen 50. Geburtstag gefeiert.}}}\label{K_L03568-1}. Wie sehr es mir wohltut
               und mich freut, brauche ich eigentlich kaum zu sagen, möchte es aber doch sagen, um
               Ihnen aufrichtig zu danken. Besonders auch dafür, dass meine Zuneigung, meine
               Verehrung und meine Freundschaft für Sie im Laufe des Lebens nur immer fester,
               überzeugter und inniger werden konnten, und dass auch Sie mir Ihre gute Gesinnung so
               bewahrt haben. Das bleibt nun so, denke ich, ohne der Worte zu bedürfen. Sie haben
               Recht: laßen Sie uns die Stücke Weges noch öfter und näher beisammen bleiben. An mir
               soll’s {\pb}nicht fehlen.\pend
           
\pstart
           Von Olga\pwindex{Schnitzler, Olga 17.\,1.\,1882 Wien – 13.\,1.\,1970 Lugano@\textsc{Schnitzler, Olga} (17.\,1.\,1882 Wien – 13.\,1.\,1970 Lugano), \emph{Schauspielerin, Sängerin}|pw} bekam ich gestern ein liebes Telegramm. Ich hoffe, sie am Dienstag noch in Salzburg\oindex{Salzburg@\textbf{Salzburg}, \emph{Verwaltungsgebiet}|pw} sehen und ihr
               danken zu können. Aus etlichen anderen Telegrammen, die heute kamen, wird mir die Befürchtung, es habe in irgend einer \label{K_L03568-2v}\edtext{Wien\oindex{Wien@\textbf{Wien}, \emph{Verwaltungsgebiet}|pw}er Zeitung von meinem 50. Geburtstag}{\lemma{\textnormal{\emph{Wiener … meinem 50. Geburtstag}}}\Cendnote{\textnormal{Tatsächlich stand es in der Zeitung: [O. V.]: \emph{Felix
                        Saltens fünfzigster Geburtstag}\pwindex{Felix Saltens fünfzigster Geburtstag@\emph{Felix Saltens fünfzigster Geburtstag}|pwk}. In: \emph{Neue Freie Presse}\pwindex{Neue Freie Presse@\emph{Neue Freie Presse}|pwk}, Nr. 19.768, 6. 9. 1919, Morgenblatt, S. 7. Darin wurde jedoch behauptet,
                  der Geburtstag wäre »morgen«.}}}\label{K_L03568-2} gestanden. Das wäre mir sehr unangenehm!! Donnerstag{ }Abend will ich in Wien\oindex{Wien@\textbf{Wien}, \emph{Verwaltungsgebiet}|pw} sein. Also,
               auf recht baldiges \label{K_L03568-3v}\edtext{Wiedersehen}{\lemma{\textnormal{\emph{Wiedersehen}}}\Cendnote{\textnormal{Nachweislich sahen sie sich am 18. 9. 1919
                  wieder.}}}\label{K_L03568-3}, nochmals: Danke, und viele herzliche Grüße von Otti\pwindex{Salten, Ottilie 7.\,3.\,1868 Prag – 22.\,6.\,1942 Zürich@\textsc{Salten, Ottilie} (7.\,3.\,1868 Prag – 22.\,6.\,1942 Zürich), \emph{Schauspielerin}|pw} wie von mir.\pend
           
\pstart
           Ihr {\\[\baselineskip]}\spacefill\mbox{Felix S.}\pend
           \leftskip=0em{}\selectlanguage{ngerman}\endnumbering\briefempfaengerindex{Schnitzler, Arthur@\textsc{Schnitzler, Arthur}!zzzSalten, Felix@\emph{von Felix Salten}!1919-09-071@{7. 9. 1919}|)be}\mylabel{L03568h}  \newcommand{\dateiname}{L03568}\newcommand{\titel}{Felix Salten an Arthur Schnitzler, 7. 9. 1919}\newcommand{\editorInnen}{Martin Anton Müller und Laura Untner}%% latex-leseansicht-abspann.tex
%% Abspann für die Leseansicht.
%% Der Schalter \ifkorrekturansicht ist bereits durch den Vorspann gesetzt.

%% latex-abspann.tex
%% Gemeinsamer Abspann für Korrekturansicht und Leseansicht.
%% Setzt den Schalter \ifkorrekturansicht voraus (gesetzt in den
%% einbindenden Dateien latex-korrekturansicht-abspann.tex bzw.
%% latex-leseansicht-abspann.tex).
%% ---------------------------------------------------------------

\normalsize

% Das esempio-Environment wird nur in der Leseansicht benötigt
\ifkorrekturansicht\else
\newenvironment{esempio}[3]%
{
    \vspace{1.5ex}
    \rlap{\underline{#1}}
    \par
    \setlength{\parindent}{0cm}
    \nopagebreak
    \leftskip=#2cm
    \rightskip=#3cm
}
{
    \par
}
\fi

\doendnotes{C}
\bigskip
\vfill

\clearpage

\footnotesize

\ifkorrekturansicht
  \lohead{\textsc{register}}
\fi

% theindex-Environment neu definieren ohne reledmac
\makeatletter
\renewenvironment{theindex}{%
  \ifkorrekturansicht
    \section*{\indexname}%
  \else
    \subsubsection*{Index der erwähnten Entitäten}%
  \fi
  \setlength{\parindent}{0pt}%
  \setlength{\parskip}{0pt plus 0.3pt}%
  \let\item\@idxitem
}{%
  \ifkorrekturansicht\clearpage\fi
}
\makeatother

\IfFileExists{\jobname-pw.ind}{\input{\jobname-pw.ind}}{}

% Quellenangabe nur in der Leseansicht
\ifkorrekturansicht\else
% Fallback-Definitionen, falls die .tex-Datei \titel etc. nicht gesetzt hat
\providecommand{\titel}{}
\providecommand{\editorInnen}{}
\providecommand{\dateiname}{\jobname}

\vspace{3cm}

\vfill

\footnotesize
\textsc{Quelle}: \titel. Herausgegeben von {\editorInnen}. In: \emph{Arthur Schnitzler: Briefwechsel mit Autorinnen und Autoren}.
 Digitale Edition, https://schnitzler-briefe.acdh.oeaw.ac.at/{\dateiname}.html (Stand \today)
\fi

\end{document}


