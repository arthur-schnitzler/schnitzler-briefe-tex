%% latex-korrekturansicht-vorspann.tex
%% Vorspann für die Korrekturansicht.
%% Lädt die gemeinsame Datei latex-vorspann.tex mit gesetztem Schalter.

\newif\ifkorrekturansicht
\korrekturansichttrue

\input{../tex-inputs/latex-vorspann}


\section[ Paul Goldmann an Arthur Schnitzler, 6. 12. {[}1901{]}]{L03094 Paul Goldmann an Arthur Schnitzler, 6. 12. {[}1901{]}}
\nopagebreak\mylabel{L03094v}
\rehead{ }\normalsize\beginnumbering\briefempfaengerindex{Schnitzler, Arthur@\textsc{Schnitzler, Arthur}!zzzGoldmann, Paul@\emph{von Paul Goldmann}!1901-12-061@{6. 12. {[}1901{]}}|(be}
\toendnotes[C]{\smallbreak\pagebreak[2]}\Standort{DLA, A:Schnitzler, HS.NZ85.1.3171.}
\physDesc{Brief, 1 Blatt, 2 Seiten, 1053 Zeichen
\newline{}Handschrift: blaue Tinte, deutsche Kurrent
\newline{}Schnitzler: 1) mit Bleistift das Jahr »901.« vermerkt  2) mit rotem Buntstift zwei Unterstreichungen}\toendnotes[C]{\smallbreak}
\pstart
           \raggedleft{}{\pb}\textcolor{gray}{\textbf{DESSAUERSTRASSE 19}}\oindex{Dessauer Strasse@\textbf{Dessauer Straße}, \emph{Straße (K.STR)}|pw}\pend
           
\pstart
           Berlin\oindex{Berlin@\textbf{Berlin}, \emph{P.PPLC}|pw}, 6. Dezember.\pend
           
\pstart\center{}Mein lieber Freund,\pend\vspace{0.5em}
\pstart
           Ich freue mich ſehr, daß Dir mein \label{K_L03094-1v}\edtext{Feuilleton\pwindex{Berliner Theater. »Der Rothe Hahn.«@\emph{Berliner Theater. »Der Rothe Hahn.«}|pwv}}{\lemma{\textnormal{\emph{Feuilleton}}}\Cendnote{\textnormal{Paul Goldmann\pwindex{Goldmann, Paul 31.01.1865 – 25.09.1935@\textsc{Goldmann, Paul} (31.01.1865 – 25.09.1935), \emph{Schriftsteller/Schriftstellerin, Journalist/Journalistin}|pwk}: \emph{Berliner Theater. »Der Rothe Hahn«}\pwindex{Berliner Theater. »Der Rothe Hahn.«@\emph{Berliner Theater. »Der Rothe Hahn.«}|pwk}. In: \emph{Neue Freie Presse}\pwindex{Neue Freie Presse@\emph{Neue Freie Presse}|pwk}, Nr. 13.391, 4. 12. 1901, Morgenblatt, S. 1–3.}}}\label{K_L03094-1}
               gefallen hat, und danke Dir für Deine lieben Worte. Nur ſehe ich nicht ein, warum Du
               in meinem \label{K_L03094-2v}\edtext{Feuilleton\pwindex{Berliner Theater. »Einsame Menschen« im Deutschen Theater@\emph{Berliner Theater. »Einsame Menschen« im Deutschen Theater}|pwv} über »Einſame Menſchen\pwindex{Einsame Menschen. Drama@\emph{Einsame Menschen. Drama}|pw}«}{\lemma{\textnormal{\emph{Feuilleton … Menſchen«}}}\Cendnote{\textnormal{Paul Goldmann\pwindex{Goldmann, Paul 31.01.1865 – 25.09.1935@\textsc{Goldmann, Paul} (31.01.1865 – 25.09.1935), \emph{Schriftsteller/Schriftstellerin, Journalist/Journalistin}|pwk}: \emph{Berliner Theater. »Einsame Menschen« im Deutschen Theater}\pwindex{Berliner Theater. »Einsame Menschen« im Deutschen Theater@\emph{Berliner Theater. »Einsame Menschen« im Deutschen Theater}|pwk}.
                     In: \emph{Neue Freie Presse}\pwindex{Neue Freie Presse@\emph{Neue Freie Presse}|pwk}, Nr. 13.345, 19. 10. 1901, Morgenblatt, S. 1–3. Siehe auch Paul Goldmann an Arthur Schnitzler, 9. 11. [1901].}}}\label{K_L03094-2} meinen Ton
               mißbilligt haſt, da \strikeout{\textcolor{gray}{er}} in meinem letzten Feuilleton\pwindex{Berliner Theater. »Der Rothe Hahn.«@\emph{Berliner Theater. »Der Rothe Hahn.«}|pwv} der Ton genau derſelbe iſt. Und daß ich im »Biberpelz\pwindex{Biberpelz. Eine Diebskomoedie@\emph{Der Biberpelz. Eine Diebskomödie}|pw}« Einiges anerkannt\pwindex{Berliner Theater. »Der Rothe Hahn.«@\emph{Berliner Theater. »Der Rothe Hahn.«}|pwv} habe, liegt daran, daß der »Biberpelz\pwindex{Biberpelz. Eine Diebskomoedie@\emph{Der Biberpelz. Eine Diebskomödie}|pw}« Gutes enthält, das anzuerkennen iſt, »Einſame Menſchen\pwindex{Einsame Menschen. Drama@\emph{Einsame Menschen. Drama}|pw}« aber nicht das Mindeſte.\pend
           
\pstart
           Wann werde ich Dir wieder ausführlich ſchreiben können? Ich weiß an Arbeit nicht ein
               noch aus.\pend
           
\pstart
           Das \label{K_L03094-3v}\edtext{Buch\pwindex{Verloebnis. Geschichte eines Knaben@\emph{Das Verlöbnis. Geschichte eines Knaben}|pwuv} von \textsc{Frisch\pwindex{Frisch, Efraim 01.03.1873 – 26.11.1942@\textsc{Frisch, Efraim} (01.03.1873 – 26.11.1942), \emph{Schriftsteller/Schriftstellerin, Publizist/Publizistin}|pwu}}}{\lemma{\textnormal{\emph{Buch von Frisch}}}\Cendnote{\textnormal{Im \emph{Börsenblatt für den deutschen Buchhandel}\pwindex{Boersenblatt fuer den Deutschen Buchhandel@\emph{Börsenblatt für den Deutschen Buchhandel}|pwk} ist das Erscheinen von Efraim Frischs\pwindex{Frisch, Efraim 01.03.1873 – 26.11.1942@\textsc{Frisch, Efraim} (01.03.1873 – 26.11.1942), \emph{Schriftsteller/Schriftstellerin, Publizist/Publizistin}|pwk}{ }\emph{Das Verlöbnis. Geschichte eines Knaben}\pwindex{Verloebnis. Geschichte eines Knaben@\emph{Das Verlöbnis. Geschichte eines Knaben}|pwk} am 1. 11. 1901 bei \emph{S.
                     Fischer}\orgindex{S. Fischer Verlag@S. Fischer Verlag|pwk} angezeigt.}}}\label{K_L03094-3} bringſt Du mir wohl nach Berlin\oindex{Berlin@\textbf{Berlin}, \emph{P.PPLC}|pw} mit?\pend
           
\pstart
           Der gewiſſe \label{K_L03094-4v}\edtext{Herr \textsc{Krügler\pwindex{Kruegler @\textsc{Krügler}, \emph{Schriftsteller/Schriftstellerin}|pw}}}{\lemma{\textnormal{\emph{Herr Krügler}}}\Cendnote{\textnormal{Über den hier im Raum stehenden
                  Plagiatsvorwurf gegen Schnitzler ist bislang
                  nichts bekannt.}}}\label{K_L03094-4} iſt ſehr {\pb}gleichgiltig. Er
               wird den Stoff anders behandelt haben, als Du, – deſſen kannſt Du ſicher ſein. Kommt
               es zu einer öffentlichen Diskuſſion, ſo bin ich Zeuge, daß Du mir den Stoff bereits
               vor zwei Jahren erzählt haſt.\pend
           
\pstart
           Mittwoch war ich bei Frau \textsc{Fulda\pwindex{DAlbert, Ida 05.12.1869 – 1926-10-06@\textsc{d’Albert, Ida} (05.12.1869 – 1926-10-06), \emph{Schauspieler/Schauspielerin}|pw}}. Sie war außergewöhnlich entzückt von Dir und ſagte, daß ſie Dich ſehr lieb
               hat.\pend
           
\pstart
           \label{K_L03094-5v}\edtext{Wann kommſt du?}{\lemma{\textnormal{\emph{Wann kommſt du?}}}\Cendnote{\textnormal{Siehe Paul Goldmann an Arthur Schnitzler, 4. 12. [1901].
               }}}\label{K_L03094-5}\pend
           
\pstart
           Grüße die Mädels\pwindex{Schnitzler, Olga 17.01.1882 – 13.01.1970@\textsc{Schnitzler, Olga} (17.01.1882 – 13.01.1970), \emph{Schauspieler/Schauspielerin, Sänger/Sängerin}|pwv}\pwindex{Steinrueck, Elisabeth 19.11.1885 – 07.04.1920@\textsc{Steinrück, Elisabeth} (19.11.1885 – 07.04.1920)|pwv} und ſei ſelbſt vielmals und herzlichſt gegrüßt von {\\[\baselineskip]}Deinem {\\[\baselineskip]}\spacefill\mbox{Paul Goldmn}\pend
           \leftskip=0em{}\selectlanguage{ngerman}\endnumbering\briefempfaengerindex{Schnitzler, Arthur@\textsc{Schnitzler, Arthur}!zzzGoldmann, Paul@\emph{von Paul Goldmann}!1901-12-061@{6. 12. {[}1901{]}}|)be}\mylabel{L03094h}  \normalsize

\doendnotes{C}
\bigskip
\vfill

\clearpage

\footnotesize

\lohead{\textsc{register}}

% Definiere theindex-Environment komplett neu ohne reledmac
\makeatletter
\renewenvironment{theindex}{%
  \section*{\indexname}%
  \setlength{\parindent}{0pt}%
  \setlength{\parskip}{0pt plus 0.3pt}%
  \let\item\@idxitem
}{%
  \clearpage
}
\makeatother

\IfFileExists{\jobname-pw.ind}{\input{\jobname-pw.ind}}{}

\end{document}

      