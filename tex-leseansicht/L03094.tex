%% latex-leseansicht-vorspann.tex
%% Vorspann für die Leseansicht.
%% Lädt die gemeinsame Datei latex-vorspann.tex mit nicht gesetztem Schalter.

\newif\ifkorrekturansicht
\korrekturansichtfalse

\input{../tex-inputs/latex-vorspann}


\section[ Paul Goldmann an Arthur Schnitzler, 6. 12. {[}1901{]}]{L03094 Paul Goldmann an Arthur Schnitzler,  6. 12. [1901]}
\nopagebreak\mylabel{L03094v}
\rehead{ }\normalsize\beginnumbering\briefempfaengerindex{Schnitzler, Arthur@\textsc{Schnitzler, Arthur}!zzzGoldmann, Paul@\emph{von Paul Goldmann}!1901-12-061@{6. 12. [1901]}|(be}
\toendnotes[C]{\smallbreak\pagebreak[2]}
\correspDesc{Versand  durch Paul Goldmann am 6. 12. [1901] in Berlin
\newline{}Erhalt  durch Arthur Schnitzler im Zeitraum [7. 12. 1901
                  – 11. 12. 1901?] in Wien}\toendnotes[C]{\smallbreak}
\Standort{DLA, A:Schnitzler, HS.NZ85.1.3171.}
\physDesc{Brief, 1 Blatt, 2 Seiten, 1053 Zeichen
\newline{}Handschrift: blaue Tinte, deutsche Kurrent
\newline{}Schnitzler: 1) mit Bleistift das Jahr »901.« vermerkt  2) mit rotem Buntstift zwei Unterstreichungen}\toendnotes[C]{\smallbreak}
\pstart
           \raggedleft{}{\pb}\textcolor{gray}{\textbf{DESSAUERSTRASSE 19}}\oindex{Dessauer Straße@\textbf{Dessauer Straße}, \emph{Straße}|pw}\pend
           
\pstart
           Berlin\oindex{Berlin@\textbf{Berlin}, \emph{Hauptstadt}|pw}, 6. Dezember.\pend
           
\pstart\center{}Mein lieber Freund,\pend\vspace{0.5em}
\pstart
           Ich freue mich{ }ſehr, daß Dir mein \label{K_L03094-1v}\edtext{Feuilleton\pwindex{Goldmann, Paul 31.\,1.\,1865 Breslau – 25.\,9.\,1935 Wien@\textsc{Goldmann, Paul} (31.\,1.\,1865 Breslau – 25.\,9.\,1935 Wien), \emph{Schriftsteller, Journalist}!Berliner Theater. »Der Rothe Hahn.«@\strich\emph{Berliner Theater. »Der Rothe Hahn.«}|pwv}}{\lemma{\textnormal{\emph{Feuilleton}}}\Cendnote{\textnormal{Paul Goldmann\pwindex{Goldmann, Paul 31.\,1.\,1865 Breslau – 25.\,9.\,1935 Wien@\textsc{Goldmann, Paul} (31.\,1.\,1865 Breslau – 25.\,9.\,1935 Wien), \emph{Schriftsteller, Journalist}|pwk}: \emph{Berliner Theater. »Der Rothe Hahn«}\pwindex{Goldmann, Paul 31.\,1.\,1865 Breslau – 25.\,9.\,1935 Wien@\textsc{Goldmann, Paul} (31.\,1.\,1865 Breslau – 25.\,9.\,1935 Wien), \emph{Schriftsteller, Journalist}!Berliner Theater. »Der Rothe Hahn.«@\strich\emph{Berliner Theater. »Der Rothe Hahn.«}|pwk}. In: \emph{Neue Freie Presse}\pwindex{Neue Freie Presse@\emph{Neue Freie Presse}|pwk}, Nr. 13.391, 4. 12. 1901, Morgenblatt, S. 1–3.}}}\label{K_L03094-1}
               gefallen hat, und danke Dir für Deine lieben Worte. Nur{ }ſehe ich nicht ein, warum Du
               in meinem \label{K_L03094-2v}\edtext{Feuilleton\pwindex{Goldmann, Paul 31.\,1.\,1865 Breslau – 25.\,9.\,1935 Wien@\textsc{Goldmann, Paul} (31.\,1.\,1865 Breslau – 25.\,9.\,1935 Wien), \emph{Schriftsteller, Journalist}!Berliner Theater. »Einsame Menschen« im Deutschen Theater@\strich\emph{Berliner Theater. »Einsame Menschen« im Deutschen Theater}|pwv} über »Einſame Menſchen\pwindex{\textcolor{red}{\textsuperscript{XXXX indx1}}!Einsame Menschen. Drama@\strich\emph{Einsame Menschen. Drama}|pw}«}{\lemma{\textnormal{\emph{Feuilleton … Menschen«}}}\Cendnote{\textnormal{Paul Goldmann\pwindex{Goldmann, Paul 31.\,1.\,1865 Breslau – 25.\,9.\,1935 Wien@\textsc{Goldmann, Paul} (31.\,1.\,1865 Breslau – 25.\,9.\,1935 Wien), \emph{Schriftsteller, Journalist}|pwk}: \emph{Berliner Theater. »Einsame Menschen« im Deutschen Theater}\pwindex{Goldmann, Paul 31.\,1.\,1865 Breslau – 25.\,9.\,1935 Wien@\textsc{Goldmann, Paul} (31.\,1.\,1865 Breslau – 25.\,9.\,1935 Wien), \emph{Schriftsteller, Journalist}!Berliner Theater. »Einsame Menschen« im Deutschen Theater@\strich\emph{Berliner Theater. »Einsame Menschen« im Deutschen Theater}|pwk}.
                     In: \emph{Neue Freie Presse}\pwindex{Neue Freie Presse@\emph{Neue Freie Presse}|pwk}, Nr. 13.345, 19. 10. 1901, Morgenblatt, S. 1–3. Siehe auch XXXX Auszeichnungsfehler: Dokument L03090 nicht gefunden.}}}\label{K_L03094-2} meinen Ton
               mißbilligt haſt, da \strikeout{\textcolor{gray}{er}} in meinem letzten Feuilleton\pwindex{Goldmann, Paul 31.\,1.\,1865 Breslau – 25.\,9.\,1935 Wien@\textsc{Goldmann, Paul} (31.\,1.\,1865 Breslau – 25.\,9.\,1935 Wien), \emph{Schriftsteller, Journalist}!Berliner Theater. »Der Rothe Hahn.«@\strich\emph{Berliner Theater. »Der Rothe Hahn.«}|pwv} der Ton genau derſelbe iſt. Und daß ich im »Biberpelz\pwindex{\textcolor{red}{\textsuperscript{XXXX indx1}}!Biberpelz. Eine Diebskomödie@\strich\emph{Der Biberpelz. Eine Diebskomödie}|pw}« Einiges anerkannt\pwindex{Goldmann, Paul 31.\,1.\,1865 Breslau – 25.\,9.\,1935 Wien@\textsc{Goldmann, Paul} (31.\,1.\,1865 Breslau – 25.\,9.\,1935 Wien), \emph{Schriftsteller, Journalist}!Berliner Theater. »Der Rothe Hahn.«@\strich\emph{Berliner Theater. »Der Rothe Hahn.«}|pwv} habe, liegt daran, daß der »Biberpelz\pwindex{\textcolor{red}{\textsuperscript{XXXX indx1}}!Biberpelz. Eine Diebskomödie@\strich\emph{Der Biberpelz. Eine Diebskomödie}|pw}« Gutes enthält, das anzuerkennen iſt, »Einſame Menſchen\pwindex{\textcolor{red}{\textsuperscript{XXXX indx1}}!Einsame Menschen. Drama@\strich\emph{Einsame Menschen. Drama}|pw}« aber nicht das Mindeſte.\pend
           
\pstart
           Wann werde ich Dir wieder ausführlich{ }ſchreiben können? Ich weiß an Arbeit nicht ein
               noch aus.\pend
           
\pstart
           Das \label{K_L03094-3v}\edtext{Buch\pwindex{Frisch, Efraim 1.\,3.\,1873 Stryj – 26.\,11.\,1942 Ascona@\textsc{Frisch, Efraim} (1.\,3.\,1873 Stryj – 26.\,11.\,1942 Ascona), \emph{Schriftsteller, Publizist}!Verlöbnis. Geschichte eines Knaben@\strich\emph{Das Verlöbnis. Geschichte eines Knaben}|pwuv} von \textsc{Frisch\pwindex{Frisch, Efraim 1.\,3.\,1873 Stryj – 26.\,11.\,1942 Ascona@\textsc{Frisch, Efraim} (1.\,3.\,1873 Stryj – 26.\,11.\,1942 Ascona), \emph{Schriftsteller, Publizist}|pwu}}}{\lemma{\textnormal{\emph{Buch von Frisch}}}\Cendnote{\textnormal{Im \emph{Börsenblatt für den deutschen Buchhandel}\pwindex{Börsenblatt für den Deutschen Buchhandel@\emph{Börsenblatt für den Deutschen Buchhandel}|pwk} ist das Erscheinen von Efraim Frischs\pwindex{Frisch, Efraim 1.\,3.\,1873 Stryj – 26.\,11.\,1942 Ascona@\textsc{Frisch, Efraim} (1.\,3.\,1873 Stryj – 26.\,11.\,1942 Ascona), \emph{Schriftsteller, Publizist}|pwk}{ }\emph{Das Verlöbnis. Geschichte eines Knaben}\pwindex{Frisch, Efraim 1.\,3.\,1873 Stryj – 26.\,11.\,1942 Ascona@\textsc{Frisch, Efraim} (1.\,3.\,1873 Stryj – 26.\,11.\,1942 Ascona), \emph{Schriftsteller, Publizist}!Verlöbnis. Geschichte eines Knaben@\strich\emph{Das Verlöbnis. Geschichte eines Knaben}|pwk} am 1. 11. 1901 bei \emph{S.
                     Fischer}\orgindex{S. Fischer Verlag@S. Fischer Verlag|pwk} angezeigt.}}}\label{K_L03094-3} bringſt Du mir wohl nach Berlin\oindex{Berlin@\textbf{Berlin}, \emph{Hauptstadt}|pw} mit?\pend
           
\pstart
           Der gewiſſe \label{K_L03094-4v}\edtext{Herr \textsc{Krügler\pwindex{Krügler @\textsc{Krügler}, \emph{Schriftsteller}|pw}}}{\lemma{\textnormal{\emph{Herr Krügler}}}\Cendnote{\textnormal{Über den hier im Raum stehenden
                  Plagiatsvorwurf gegen Schnitzler ist bislang
                  nichts bekannt.}}}\label{K_L03094-4} iſt{ }ſehr {\pb}gleichgiltig. Er
               wird den Stoff anders behandelt haben, als Du, – deſſen kannſt Du{ }ſicher{ }ſein. Kommt
               es zu einer öffentlichen Diskuſſion,{ }ſo bin ich Zeuge, daß Du mir den Stoff bereits
               vor zwei Jahren erzählt haſt.\pend
           
\pstart
           Mittwoch war ich bei Frau \textsc{Fulda\pwindex{d’Albert, Ida 5.\,12.\,1869 Wien – 6.\,10.\,1926 Berlin@\textsc{d’Albert, Ida} (5.\,12.\,1869 Wien – 6.\,10.\,1926 Berlin), \emph{Schauspielerin}|pw}}. Sie war außergewöhnlich entzückt von Dir und{ }ſagte, daß{ }ſie Dich{ }ſehr lieb
               hat.\pend
           
\pstart
           \label{K_L03094-5v}\edtext{Wann kommſt du?}{\lemma{\textnormal{\emph{Wann kommst du?}}}\Cendnote{\textnormal{Siehe XXXX Auszeichnungsfehler: Dokument L03093 nicht gefunden.
               }}}\label{K_L03094-5}\pend
           
\pstart
           Grüße die Mädels\pwindex{Schnitzler, Olga 17.\,1.\,1882 Wien – 13.\,1.\,1970 Lugano@\textsc{Schnitzler, Olga} (17.\,1.\,1882 Wien – 13.\,1.\,1970 Lugano), \emph{Schauspielerin, Sängerin}|pwv}\pwindex{Steinrück, Elisabeth 19.\,11.\,1885 – 7.\,4.\,1920 Partenkirchen@\textsc{Steinrück, Elisabeth} (19.\,11.\,1885 – 7.\,4.\,1920 Partenkirchen)|pwv} und{ }ſei{ }ſelbſt vielmals und herzlichſt gegrüßt von {\\[\baselineskip]}Deinem {\\[\baselineskip]}\spacefill\mbox{Paul Goldmn}\pend
           \leftskip=0em{}\selectlanguage{ngerman}\endnumbering\briefempfaengerindex{Schnitzler, Arthur@\textsc{Schnitzler, Arthur}!zzzGoldmann, Paul@\emph{von Paul Goldmann}!1901-12-061@{6. 12. [1901]}|)be}\mylabel{L03094h}  \newcommand{\dateiname}{L03094}\newcommand{\titel}{Paul Goldmann an Arthur Schnitzler, 6. 12. [1901]}\newcommand{\editorInnen}{Martin Anton Müller und Laura Untner}%% latex-leseansicht-abspann.tex
%% Abspann für die Leseansicht.
%% Der Schalter \ifkorrekturansicht ist bereits durch den Vorspann gesetzt.

%% latex-abspann.tex
%% Gemeinsamer Abspann für Korrekturansicht und Leseansicht.
%% Setzt den Schalter \ifkorrekturansicht voraus (gesetzt in den
%% einbindenden Dateien latex-korrekturansicht-abspann.tex bzw.
%% latex-leseansicht-abspann.tex).
%% ---------------------------------------------------------------

\normalsize

% Das esempio-Environment wird nur in der Leseansicht benötigt
\ifkorrekturansicht\else
\newenvironment{esempio}[3]%
{
    \vspace{1.5ex}
    \rlap{\underline{#1}}
    \par
    \setlength{\parindent}{0cm}
    \nopagebreak
    \leftskip=#2cm
    \rightskip=#3cm
}
{
    \par
}
\fi

\doendnotes{C}
\bigskip
\vfill

\clearpage

\footnotesize

\ifkorrekturansicht
  \lohead{\textsc{register}}
\fi

% theindex-Environment neu definieren ohne reledmac
\makeatletter
\renewenvironment{theindex}{%
  \ifkorrekturansicht
    \section*{\indexname}%
  \else
    \subsubsection*{Index der erwähnten Entitäten}%
  \fi
  \setlength{\parindent}{0pt}%
  \setlength{\parskip}{0pt plus 0.3pt}%
  \let\item\@idxitem
}{%
  \ifkorrekturansicht\clearpage\fi
}
\makeatother

\IfFileExists{\jobname-pw.ind}{\input{\jobname-pw.ind}}{}

% Quellenangabe nur in der Leseansicht
\ifkorrekturansicht\else
% Fallback-Definitionen, falls die .tex-Datei \titel etc. nicht gesetzt hat
\providecommand{\titel}{}
\providecommand{\editorInnen}{}
\providecommand{\dateiname}{\jobname}

\vspace{3cm}

\vfill

\footnotesize
\textsc{Quelle}: \titel. Herausgegeben von {\editorInnen}. In: \emph{Arthur Schnitzler: Briefwechsel mit Autorinnen und Autoren}.
 Digitale Edition, https://schnitzler-briefe.acdh.oeaw.ac.at/{\dateiname}.html (Stand \today)
\fi

\end{document}


