%% latex-leseansicht-vorspann.tex
%% Vorspann für die Leseansicht.
%% Lädt die gemeinsame Datei latex-vorspann.tex mit nicht gesetztem Schalter.

\newif\ifkorrekturansicht
\korrekturansichtfalse

\input{../tex-inputs/latex-vorspann}


         
         \renewcommand{\erwaehntePersonen}{Personen: Efraim Frisch,  Krügler, Olga Schnitzler, Elisabeth Steinrück, Ida d’Albert}
         \renewcommand{\erwaehnteInstitutionen}{Institutionen: S. Fischer Verlag}
         \renewcommand{\erwaehnteOrte}{Orte: Berlin, Dessauer Straße, Wien}
         \renewcommand{\erwaehnteWerke}{Werke: Berliner Theater. »Der Rothe Hahn.«, Berliner Theater. »Einsame Menschen« im Deutschen Theater, Börsenblatt für den Deutschen Buchhandel, Das Verlöbnis. Geschichte eines Knaben, Der Biberpelz. Eine Diebskomödie, Einsame Menschen. Drama, Neue Freie Presse}
               \section[ Paul Goldmann an Arthur Schnitzler, 6. 12. {[}1901{]}]{ Paul Goldmann an Arthur Schnitzler, 6. 12. {[}1901{]}}\nopagebreak\mylabel{v}\rehead{ }\begin{ledgroupsized}[t]{13cm}\normalsize\beginnumbering \toendnotes[C]{\smallbreak\pagebreak[2]} \Standort{DLA, A:Schnitzler, HS.NZ85.1.3171.}
\physDesc{Brief, 1 Blatt, 2 Seiten, 1053 Zeichen
\newline{}Handschrift: blaue Tinte, deutsche Kurrent
\newline{}Schnitzler: 1) mit Bleistift das Jahr »{[}1{]}901.« vermerkt  2) mit rotem Buntstift zwei Unterstreichungen}\toendnotes[C]{\smallbreak}\pstart
           \noindent{}\raggedleft{}{\pb}\textcolor{gray}{\textbf{DESSAUERSTRASSE 19}}\oindex{Dessauer Strasse@\textbf{Dessauer Straße}|pw}\pend
           \pstart
           Berlin\oindex{Berlin@\textbf{Berlin}|pw}, 6. Dezember.\pend
           \pstart\center{}Mein lieber Freund,\pend\pstart
           Ich freue mich ſehr, daß Dir mein \label{K_L03094-2v}\edtext{Feuilleton\pwindex{Goldmann, Paul 31.01.1865 – 25.09.1935@\textsc{Goldmann, Paul} (31.01.1865 – 25.09.1935), \emph{Schriftsteller, Journalist}!Berliner Theater. »Der Rothe Hahn.«1901-12-04@\strich\emph{Berliner Theater. »Der Rothe Hahn.«} {[}1901-12-04{]}|pwv}}{\lemma{\textnormal{\emph{Feuilleton}}}\Cendnote{\textnormal{Paul Goldmann\pwindex{Goldmann, Paul 31.01.1865 – 25.09.1935@\textsc{Goldmann, Paul} (31.01.1865 – 25.09.1935), \emph{Schriftsteller, Journalist}|pwk}: \emph{Berliner Theater. »Der Rothe Hahn.«}\pwindex{Goldmann, Paul 31.01.1865 – 25.09.1935@\textsc{Goldmann, Paul} (31.01.1865 – 25.09.1935), \emph{Schriftsteller, Journalist}!Berliner Theater. »Der Rothe Hahn.«1901-12-04@\strich\emph{Berliner Theater. »Der Rothe Hahn.«} {[}1901-12-04{]}|pwk}. In: \emph{Neue Freie Presse}\pwindex{Neue Freie Presse1864 – 1939@\emph{Neue Freie Presse} {[}1864 – 1939{]}|pwk}, Nr. 13.391, 4. 12. 1901, Morgenblatt, S. 1–3.}}}\label{K_L03094-2h}
               gefallen hat, und danke Dir für Deine lieben Worte. Nur ſehe ich nicht ein, warum Du
               in meinem \label{K_L03094-1v}\edtext{Feuilleton\pwindex{Goldmann, Paul 31.01.1865 – 25.09.1935@\textsc{Goldmann, Paul} (31.01.1865 – 25.09.1935), \emph{Schriftsteller, Journalist}!Berliner Theater. »Einsame Menschen« im Deutschen Theater19. 10. 1901@\strich\emph{Berliner Theater. »Einsame Menschen« im Deutschen Theater} {[}19. 10. 1901{]}|pwv} über »Einſame Menſchen\pwindex{\textcolor{red}{\textsuperscript{XXXX1 indx}}!Einsame Menschen. Drama1891-01-11@\strich\emph{Einsame Menschen. Drama} {[}1891-01-11{]}|pw}«}{\lemma{\textnormal{\emph{Feuilleton … Menſchen«}}}\Cendnote{\textnormal{Paul Goldmann\pwindex{Goldmann, Paul 31.01.1865 – 25.09.1935@\textsc{Goldmann, Paul} (31.01.1865 – 25.09.1935), \emph{Schriftsteller, Journalist}|pwk}: \emph{Berliner Theater. »Einsame Menschen« im Deutschen Theater}\pwindex{Goldmann, Paul 31.01.1865 – 25.09.1935@\textsc{Goldmann, Paul} (31.01.1865 – 25.09.1935), \emph{Schriftsteller, Journalist}!Berliner Theater. »Einsame Menschen« im Deutschen Theater19. 10. 1901@\strich\emph{Berliner Theater. »Einsame Menschen« im Deutschen Theater} {[}19. 10. 1901{]}|pwk}.
                     In: \emph{Neue Freie Presse}\pwindex{Neue Freie Presse1864 – 1939@\emph{Neue Freie Presse} {[}1864 – 1939{]}|pwk}, Nr. 13.345, 19. 10. 1901, Morgenblatt, S. 1–3. Siehe auch Paul Goldmann an Arthur Schnitzler, 9. 11. [1901].}}}\label{K_L03094-1h} meinen Ton
               mißbilligt haſt, da \strikeout{\textcolor{gray}{er}} in meinem letzten Feuilleton\pwindex{Goldmann, Paul 31.01.1865 – 25.09.1935@\textsc{Goldmann, Paul} (31.01.1865 – 25.09.1935), \emph{Schriftsteller, Journalist}!Berliner Theater. »Der Rothe Hahn.«1901-12-04@\strich\emph{Berliner Theater. »Der Rothe Hahn.«} {[}1901-12-04{]}|pwv} der Ton genau derſelbe iſt. Und daß ich im »Biberpelz\pwindex{\textcolor{red}{\textsuperscript{XXXX1 indx}}!Biberpelz. Eine Diebskomoedie1893@\strich\emph{Der Biberpelz. Eine Diebskomödie} {[}1893{]}|pw}« Einiges anerkannt\pwindex{Goldmann, Paul 31.01.1865 – 25.09.1935@\textsc{Goldmann, Paul} (31.01.1865 – 25.09.1935), \emph{Schriftsteller, Journalist}!Berliner Theater. »Der Rothe Hahn.«1901-12-04@\strich\emph{Berliner Theater. »Der Rothe Hahn.«} {[}1901-12-04{]}|pwv} habe, liegt daran, daß der »Biberpelz\pwindex{\textcolor{red}{\textsuperscript{XXXX1 indx}}!Biberpelz. Eine Diebskomoedie1893@\strich\emph{Der Biberpelz. Eine Diebskomödie} {[}1893{]}|pw}« Gutes enthält, das anzuerkennen iſt, »Einſame Menſchen\pwindex{\textcolor{red}{\textsuperscript{XXXX1 indx}}!Einsame Menschen. Drama1891-01-11@\strich\emph{Einsame Menschen. Drama} {[}1891-01-11{]}|pw}« aber nicht das Mindeſte.\pend
           \pstart
           Wann werde ich Dir wieder ausführlich ſchreiben können? Ich weiß an Arbeit nicht ein
               noch aus.\pend
           \pstart
           Das \label{K_L03094-55v}\edtext{Buch\pwindex{Frisch, Efraim 01.03.1873 – 26.11.1942@\textsc{Frisch, Efraim} (01.03.1873 – 26.11.1942), \emph{Schriftsteller, Publizist}!Verloebnis. Geschichte eines Knaben1901-11-01@\strich\emph{Das Verlöbnis. Geschichte eines Knaben} {[}1901-11-01{]}|pwuv} von \textsc{Frisch\pwindex{Frisch, Efraim 01.03.1873 – 26.11.1942@\textsc{Frisch, Efraim} (01.03.1873 – 26.11.1942), \emph{Schriftsteller, Publizist}|pwu}}}{\lemma{\textnormal{\emph{Buch von Frisch}}}\Cendnote{\textnormal{Im \emph{Börsenblatt für den deutschen Buchhandel}\pwindex{?? Werk@Nicht ermittelte Verfasserinnen und Verfasser!Boersenblatt fuer den Deutschen Buchhandel1843-01-03@\emph{Börsenblatt für den Deutschen Buchhandel} {[}1843-01-03{]}|pwk} ist das Erscheinen von Efraim Frisch\pwindex{Frisch, Efraim 01.03.1873 – 26.11.1942@\textsc{Frisch, Efraim} (01.03.1873 – 26.11.1942), \emph{Schriftsteller, Publizist}|pwk}s \emph{Das Verlöbnis. Geschichte eines Knaben}\pwindex{Frisch, Efraim 01.03.1873 – 26.11.1942@\textsc{Frisch, Efraim} (01.03.1873 – 26.11.1942), \emph{Schriftsteller, Publizist}!Verloebnis. Geschichte eines Knaben1901-11-01@\strich\emph{Das Verlöbnis. Geschichte eines Knaben} {[}1901-11-01{]}|pwk} am 1. 11. 1901 bei \emph{S.
                     Fischer}\orgindex{S. Fischer Verlag@S. Fischer Verlag|pwk} angezeigt.}}}\label{K_L03094-55h} bringſt Du mir wohl nach Berlin\oindex{Berlin@\textbf{Berlin}|pw} mit?\pend
           \pstart
           Der gewiſſe \label{K_L03094-112v}\edtext{Herr \textsc{Krügler\pwindex{Kruegler @\textsc{Krügler}, \emph{Schriftsteller}|pw}}}{\lemma{\textnormal{\emph{Herr Krügler}}}\Cendnote{\textnormal{Über den hier im Raum stehenden
                  Plagiatsvorwurf gegen Schnitzler\pwindex{Schnitzler, Arthur 15.05.1862 – 21.10.1931@\textsc{Schnitzler, Arthur} (15.05.1862 – 21.10.1931), \emph{Schriftsteller, Mediziner}|pwk} ist bislang
                  nichts bekannt.}}}\label{K_L03094-112h} iſt ſehr {\pb}gleichgiltig. Er
               wird den Stoff anders behandelt haben, als Du, – deſſen kannſt Du ſicher ſein. Kommt
               es zu einer öffentlichen Diskuſſion, ſo bin ich Zeuge, daß Du mir den Stoff bereits
               vor zwei Jahren erzählt haſt.\pend
           \pstart
           Mittwoch war ich bei Frau \textsc{Fulda\pwindex{DAlbert, Ida 05.12.1869 – 1926-10-06@\textsc{d’Albert, Ida} (05.12.1869 – 1926-10-06)|pw}}. Sie war außergewöhnlich entzückt von Dir und ſagte, daß ſie Dich ſehr lieb
               hat.\pend
           \pstart
           \label{K_L03094-32v}\edtext{Wann kommſt du?}{\lemma{\textnormal{\emph{Wann kommſt du?}}}\Cendnote{\textnormal{siehe Paul Goldmann an Arthur Schnitzler, 4. 12. [1901]}}}\label{K_L03094-32h}\pend
           \pstart
           Grüße die Mädels\pwindex{Schnitzler, Olga 17.01.1882 – 13.01.1970@\textsc{Schnitzler, Olga} (17.01.1882 – 13.01.1970), \emph{Schauspielerin, Sängerin}|pwv}\pwindex{Steinrueck, Elisabeth 19.11.1885 – 07.04.1920@\textsc{Steinrück, Elisabeth} (19.11.1885 – 07.04.1920)|pwv} und ſei ſelbſt vielmals und herzlichſt gegrüßt von {\\[\baselineskip]}Deinem {\\[\baselineskip]}\spacefill\mbox{Paul Goldmn}\pend
           \leftskip=0em{}
         
         \endnumbering\mylabel{h}\end{ledgroupsized}  \newcommand{\dateiname}{L03094}\newcommand{\titel}{Paul Goldmann an Arthur Schnitzler, 6. 12. [1901]}\newcommand{\editorInnen}{Martin Anton Müller und Laura Untner}%% latex-leseansicht-abspann.tex
%% Abspann für die Leseansicht.
%% Der Schalter \ifkorrekturansicht ist bereits durch den Vorspann gesetzt.

%% latex-abspann.tex
%% Gemeinsamer Abspann für Korrekturansicht und Leseansicht.
%% Setzt den Schalter \ifkorrekturansicht voraus (gesetzt in den
%% einbindenden Dateien latex-korrekturansicht-abspann.tex bzw.
%% latex-leseansicht-abspann.tex).
%% ---------------------------------------------------------------

\normalsize

% Das esempio-Environment wird nur in der Leseansicht benötigt
\ifkorrekturansicht\else
\newenvironment{esempio}[3]%
{
    \vspace{1.5ex}
    \rlap{\underline{#1}}
    \par
    \setlength{\parindent}{0cm}
    \nopagebreak
    \leftskip=#2cm
    \rightskip=#3cm
}
{
    \par
}
\fi

\doendnotes{C}
\bigskip
\vfill

\clearpage

\footnotesize

\ifkorrekturansicht
  \lohead{\textsc{register}}
\fi

% theindex-Environment neu definieren ohne reledmac
\makeatletter
\renewenvironment{theindex}{%
  \ifkorrekturansicht
    \section*{\indexname}%
  \else
    \subsubsection*{Index der erwähnten Entitäten}%
  \fi
  \setlength{\parindent}{0pt}%
  \setlength{\parskip}{0pt plus 0.3pt}%
  \let\item\@idxitem
}{%
  \ifkorrekturansicht\clearpage\fi
}
\makeatother

\IfFileExists{\jobname-pw.ind}{\input{\jobname-pw.ind}}{}

% Quellenangabe nur in der Leseansicht
\ifkorrekturansicht\else
% Fallback-Definitionen, falls die .tex-Datei \titel etc. nicht gesetzt hat
\providecommand{\titel}{}
\providecommand{\editorInnen}{}
\providecommand{\dateiname}{\jobname}

\vspace{3cm}

\vfill

\footnotesize
\textsc{Quelle}: \titel. Herausgegeben von {\editorInnen}. In: \emph{Arthur Schnitzler: Briefwechsel mit Autorinnen und Autoren}.
 Digitale Edition, https://schnitzler-briefe.acdh.oeaw.ac.at/{\dateiname}.html (Stand \today)
\fi

\end{document}


      