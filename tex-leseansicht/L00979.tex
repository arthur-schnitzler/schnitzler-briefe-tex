%% latex-leseansicht-vorspann.tex
%% Vorspann für die Leseansicht.
%% Lädt die gemeinsame Datei latex-vorspann.tex mit nicht gesetztem Schalter.

\newif\ifkorrekturansicht
\korrekturansichtfalse

\input{../tex-inputs/latex-vorspann}


         
         \renewcommand{\erwaehntePersonen}{Personen: Richard Beer-Hofmann}
         \renewcommand{\erwaehnteOrte}{Orte: Brixen, Frankfurt am Main, Roßmarkt, Tirol, Vahrn}
         \renewcommand{\erwaehnteWerke}{
               \section[Arthur Schnitzler und Paul Goldmann an Richard Beer-Hofmann, 2{[}0?{]}. 9. 1899]{ Arthur Schnitzler und Paul Goldmann an Richard Beer-Hofmann,
               2{[}0?{]}. 9. 1899}\nopagebreak\mylabel{v}\rehead{ }\begin{ledgroupsized}[t]{13cm}\normalsize\beginnumbering \toendnotes[C]{\smallbreak\pagebreak[2]} \Standort{CUL, Schnitzler, B 8.}
\physDesc{Bildpostkarte
\newline{}Handschrift Arthur Schnitzler: Bleistift, deutsche Kurrent\newline{}Handschrift Paul Goldmann: Bleistift, deutsche Kurrent\newline{}Versand: 1) Stempel: »\nobreak{}\oindex{Frankfurt am Main@\textbf{Frankfurt am Main}|pwk}Frankfurt, {[}2{]}\textcolor{gray}{×}{[}. {]}9. 99, 6–7V\nobreak{}«.   2) Stempel: »\nobreak{}22. 9. 99\nobreak{}«. }\pstart{}{\pb}Herrn \textsc{Dr. Richard
                     Beer-Hofmann}\pend{}\pstart{}\textsc{Vahrn}\oindex{Vahrn@\textbf{Vahrn}|pw}\pend{}\pstart{}bei \textsc{Brixen\oindex{Brixen@\textbf{Brixen}|pw}}\pend{}\pstart{}\textsc{Tirol}\oindex{Tirol@\textbf{Tirol}|pw}\pend{}{\bigskip}\pstart
           \noindent{}\centering{}\textcolor{gray}{\textbf{{\pb}Gruss aus Frankfurt a. M.\oindex{Frankfurt am Main@\textbf{Frankfurt am Main}|pw}{ }Rossmarkt\oindex{Rossmarkt@\textbf{Roßmarkt}|pw}}}\pend
           \pstart
           {\pb}Herzliche Grüße\pend
           \pstart \spacefill\mbox{Arthur}\pend{}\pstart
           \noindent{}{[}hs. Goldmann:{]} Herzlichen Gruß! Vielen Dank für den ſchönen Brief!\pend
           \pstart \spacefill\mbox{P. G.}\pend{}
         
         \endnumbering\mylabel{h}\end{ledgroupsized}  \newcommand{\dateiname}{L00979}\newcommand{\titel}{Arthur Schnitzler und Paul Goldmann an Richard Beer-Hofmann, 2[0?]. 9. 1899}\newcommand{\editorInnen}{Martin Anton Müller und Gerd-Hermann Susen}%% latex-leseansicht-abspann.tex
%% Abspann für die Leseansicht.
%% Der Schalter \ifkorrekturansicht ist bereits durch den Vorspann gesetzt.

%% latex-abspann.tex
%% Gemeinsamer Abspann für Korrekturansicht und Leseansicht.
%% Setzt den Schalter \ifkorrekturansicht voraus (gesetzt in den
%% einbindenden Dateien latex-korrekturansicht-abspann.tex bzw.
%% latex-leseansicht-abspann.tex).
%% ---------------------------------------------------------------

\normalsize

% Das esempio-Environment wird nur in der Leseansicht benötigt
\ifkorrekturansicht\else
\newenvironment{esempio}[3]%
{
    \vspace{1.5ex}
    \rlap{\underline{#1}}
    \par
    \setlength{\parindent}{0cm}
    \nopagebreak
    \leftskip=#2cm
    \rightskip=#3cm
}
{
    \par
}
\fi

\doendnotes{C}
\bigskip
\vfill

\clearpage

\footnotesize

\ifkorrekturansicht
  \lohead{\textsc{register}}
\fi

% theindex-Environment neu definieren ohne reledmac
\makeatletter
\renewenvironment{theindex}{%
  \ifkorrekturansicht
    \section*{\indexname}%
  \else
    \subsubsection*{Index der erwähnten Entitäten}%
  \fi
  \setlength{\parindent}{0pt}%
  \setlength{\parskip}{0pt plus 0.3pt}%
  \let\item\@idxitem
}{%
  \ifkorrekturansicht\clearpage\fi
}
\makeatother

\IfFileExists{\jobname-pw.ind}{\input{\jobname-pw.ind}}{}

% Quellenangabe nur in der Leseansicht
\ifkorrekturansicht\else
% Fallback-Definitionen, falls die .tex-Datei \titel etc. nicht gesetzt hat
\providecommand{\titel}{}
\providecommand{\editorInnen}{}
\providecommand{\dateiname}{\jobname}

\vspace{3cm}

\vfill

\footnotesize
\textsc{Quelle}: \titel. Herausgegeben von {\editorInnen}. In: \emph{Arthur Schnitzler: Briefwechsel mit Autorinnen und Autoren}.
 Digitale Edition, https://schnitzler-briefe.acdh.oeaw.ac.at/{\dateiname}.html (Stand \today)
\fi

\end{document}


      