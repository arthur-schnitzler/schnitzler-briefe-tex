%% latex-korrekturansicht-vorspann.tex
%% Vorspann für die Korrekturansicht.
%% Lädt die gemeinsame Datei latex-vorspann.tex mit gesetztem Schalter.

\newif\ifkorrekturansicht
\korrekturansichttrue

\input{../tex-inputs/latex-vorspann}


\section[ Paul Goldmann an Arthur Schnitzler, {[}10. 8. 1904{]}]{L03453 Paul Goldmann an Arthur Schnitzler, {[}10. 8. 1904{]}}
\nopagebreak\mylabel{L03453v}
\rehead{ }\normalsize\beginnumbering\briefempfaengerindex{Schnitzler, Arthur@\textsc{Schnitzler, Arthur}!zzzGoldmann, Paul@\emph{von Paul Goldmann}!1904-08-101@{{[}10. 8. 1904{]}}|(be}
\toendnotes[C]{\smallbreak\pagebreak[2]}\Standort{DLA, A:Schnitzler, HS.NZ85.1.3174.}
\physDesc{Visitenkarte, 320 Zeichen
\newline{}Handschrift: Bleistift, deutsche Kurrent
\newline{}Schnitzler: mit Bleistift das Datum »10/8 904« vermerkt }\toendnotes[C]{\smallbreak}
\pstart
           \centering{}{\pb}\textcolor{gray}{\textbf{D\textsuperscript{r} Paul Goldmann}}\pend
           
\pstart
           \raggedleft{}\textcolor{gray}{\textbf{»Neue Freie Presse\orgindex{Neue Freie Presse@Neue Freie Presse|pw}«}}\pend
           
\pstart
           \textcolor{gray}{\textbf{DESSAUERSTRASSE 19\oindex{Dessauer Strasse@\textbf{Dessauer Straße}, \emph{Straße (K.STR)}|pw}.}}\pend
           \vspace{0.5em}
\pstart
           {\pb}Mein lieber Freund, Es thut mir unendlich
               leid, Dich verfehlt zu haben. Ich fahre \label{K_L03453-1v}\edtext{heut}{\lemma{\textnormal{\emph{heut}}}\Cendnote{\textnormal{Schnitzler war nicht verreist, hatte aber
                  die letzten Tage mit Ausflügen gefüllt. Nachdem Goldmann\pwindex{Goldmann, Paul 31.01.1865 – 25.09.1935@\textsc{Goldmann, Paul} (31.01.1865 – 25.09.1935), \emph{Schriftsteller/Schriftstellerin, Journalist/Journalistin}|pwk} diese Karte hinterlegt hatte, beschloss er, seine Abreise um
                  einen Tag zu verschieben, um Schnitzler doch
                  noch zu sehen (vgl. Paul Goldmann an Arthur Schnitzler, 10. 8. 1904).}}}\label{K_L03453-1}{ }9 Uhr 40 Abends weiter und muß alſo Wien\oindex{Wien@\textbf{Wien}, \emph{A.ADM2}|pw} verlaſſen, ohne Dich geſehen zu haben. Ich wäre Dir gern noch
               nachgekommen, aber Niemand {\pb}weiß, wohin Ihr\pwindex{Schnitzler, Olga 17.01.1882 – 13.01.1970@\textsc{Schnitzler, Olga} (17.01.1882 – 13.01.1970), \emph{Schauspieler/Schauspielerin, Sänger/Sängerin}|pwv} gegangen ſeid. Hoffentlich
               ſehen wir uns auf der Rückreiſe. Herzliche Grüße Dir und Deiner Frau\pwindex{Schnitzler, Olga 17.01.1882 – 13.01.1970@\textsc{Schnitzler, Olga} (17.01.1882 – 13.01.1970), \emph{Schauspieler/Schauspielerin, Sänger/Sängerin}|pwv}!\pend
           \selectlanguage{ngerman}\endnumbering\briefempfaengerindex{Schnitzler, Arthur@\textsc{Schnitzler, Arthur}!zzzGoldmann, Paul@\emph{von Paul Goldmann}!1904-08-101@{{[}10. 8. 1904{]}}|)be}\mylabel{L03453h}  \normalsize

\doendnotes{C}
\bigskip
\vfill

\clearpage

\footnotesize

\lohead{\textsc{register}}

% Definiere theindex-Environment komplett neu ohne reledmac
\makeatletter
\renewenvironment{theindex}{%
  \section*{\indexname}%
  \setlength{\parindent}{0pt}%
  \setlength{\parskip}{0pt plus 0.3pt}%
  \let\item\@idxitem
}{%
  \clearpage
}
\makeatother

\IfFileExists{\jobname-pw.ind}{\input{\jobname-pw.ind}}{}

\end{document}

      