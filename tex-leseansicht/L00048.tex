%% latex-korrekturansicht-vorspann.tex
%% Vorspann für die Korrekturansicht.
%% Lädt die gemeinsame Datei latex-vorspann.tex mit gesetztem Schalter.

\newif\ifkorrekturansicht
\korrekturansichttrue

\input{../tex-inputs/latex-vorspann}


\section[Arthur Schnitzler: Widmungsexemplar Das Märchen für Hugo August von Hofmannsthal, {[}5.{]} 12. 1891]{L00048 Arthur Schnitzler: Widmungsexemplar Das Märchen für Hugo August von
               Hofmannsthal, {[}5.{]} 12. 1891}
\nopagebreak\mylabel{L00048v}
\rehead{ }\normalsize\beginnumbering\briefempfaengerindex{Hofmannsthal, Hugo August von@\textsc{Hofmannsthal, Hugo August von}!zzzSchnitzler, Arthur@\emph{von Arthur Schnitzler}!1891-12-052@{{[}5.{]} 12. 1891}|(be}
\toendnotes[C]{\smallbreak\pagebreak[2]}\Standort{FDH, FDH 3227.}
\physDesc{Widmung am Umschlag, 47 Zeichen
\newline{}Handschrift: schwarze Tinte, deutsche Kurrent}
\buchAbdrucke{\weitereDrucke{Hugo von Hofmannsthal: \emph{Bibliothek}. Frankfurt am Main: \emph{S. Fischer} 2011, S. 604.} }\toendnotes[C]{\smallbreak}
\pstart
           \noindent{}\raggedleft{}{\pb}Herrn Dr. v. \textsc{Hofma{\geminationn}sthal}\pend
           
\pstart
           \raggedleft{}verehrungsvoll\pend
           \pstart \spacefill\mbox{ArthSch.}\pend{}{\vspace{1\baselineskip}}
\pstart
           \textcolor{gray}{\textbf{\uline{Manuſkript.}}}\pend
           
\pstart
           \centering{}\textcolor{gray}{\textbf{\textbf{\so{Das Märchen}}\pwindex{Maerchen. Schauspiel in drei Aufzuegen@\emph{Das Märchen. Schauspiel in drei Aufzügen}|pw}.}}\pend
           
\pstart
           \centering{}\textcolor{gray}{\textbf{Schauſpiel in drei Aufzügen}}{\\}\textcolor{gray}{\textbf{von}}{\\}\textcolor{gray}{\textbf{\textbf{Arthur Schnitzler.}}}\pend
           {\vspace{1\baselineskip}}
\pstart
           \centering{}\textcolor{gray}{\textbf{\textbf{Wien}\oindex{Wien@\textbf{Wien}, \emph{A.ADM2}|pw}{ }\label{K_L00048-1v}\edtext{\textbf{1891}}{\lemma{\textnormal{\emph{1891}}}\Cendnote{\textnormal{Schnitzler bekam die Drucke des Stücks
                        am 5. 12. 1891. Zwei Tage später hatte es Hugo August
                           Hofmannsthal\pwindex{Hofmannsthal, Hugo August von 21.12.1841 – 08.12.1915@\textsc{Hofmannsthal, Hugo August von} (21.12.1841 – 08.12.1915), \emph{Bankdirektor/Bankdirektorin}|pwk} gelesen.}}}\label{K_L00048-1}.}}\pend
           
\pstart
           \centering{}\textcolor{gray}{\textbf{Druck von Carl Steinhardt {\kaufmannsund} Cie.\orgindex{Carl Steinhardt und Co.@Carl Steinhardt {\kaufmannsund}  Co.|pw} (verantw. Leiter Guſtav Röttig\pwindex{Roettig, Gustav 1855-12-08 – nach 1918@\textsc{Röttig, Gustav} (1855-12-08 – nach 1918), \emph{Redakteur/Redakteurin, Drucker/Druckerin}|pw}).}}\pend
           \selectlanguage{ngerman}\endnumbering\briefempfaengerindex{Hofmannsthal, Hugo August von@\textsc{Hofmannsthal, Hugo August von}!zzzSchnitzler, Arthur@\emph{von Arthur Schnitzler}!1891-12-052@{{[}5.{]} 12. 1891}|)be}\mylabel{L00048h}  \normalsize

\doendnotes{C}
\bigskip
\vfill

\clearpage

\footnotesize

\lohead{\textsc{register}}

% Definiere theindex-Environment komplett neu ohne reledmac
\makeatletter
\renewenvironment{theindex}{%
  \section*{\indexname}%
  \setlength{\parindent}{0pt}%
  \setlength{\parskip}{0pt plus 0.3pt}%
  \let\item\@idxitem
}{%
  \clearpage
}
\makeatother

\IfFileExists{\jobname-pw.ind}{\input{\jobname-pw.ind}}{}

\end{document}

      