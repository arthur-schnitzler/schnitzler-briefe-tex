%% latex-leseansicht-vorspann.tex
%% Vorspann für die Leseansicht.
%% Lädt die gemeinsame Datei latex-vorspann.tex mit nicht gesetztem Schalter.

\newif\ifkorrekturansicht
\korrekturansichtfalse

\input{../tex-inputs/latex-vorspann}


               \section[Arthur Schnitzler: Widmungsexemplar Das Märchen für Hugo August von Hofmannsthal, {[}5.{]} 12. 1891]{ Arthur Schnitzler: Widmungsexemplar Das Märchen für Hugo August von
               Hofmannsthal, {[}5.{]} 12. 1891}\nopagebreak\mylabel{v}\rehead{ }\begin{ledgroupsized}[t]{13cm}\normalsize\beginnumbering\briefempfaengerindex{Hofmannsthal, Hugo August von@\textsc{Hofmannsthal, Hugo August von}!zzzSchnitzler, Arthur@\emph{von Arthur Schnitzler}!1891-12-052@{{[}5.{]} 12. 1891}|(be} \toendnotes[C]{\smallbreak\pagebreak[2]} \Standort{FDH, FDH 3227.}
\physDesc{Widmung am Umschlag
\newline{}Handschrift: schwarze Tinte, deutsche Kurrent}\buchAbdrucke{\weitereDrucke{Hugo von Hofmannsthal: \emph{Bibliothek}. Hg. Ellen Ritter † in Zusammenarbeit mit Dalia Bukauskaité und
                        Konrad Heumann. Frankfurt am Main: \emph{S. Fischer} 2011, S. 604 (Sämtliche Werke. Kritische Ausgabe, XL).} }\toendnotes[C]{\smallbreak}\pstart
           \noindent{}\raggedleft{}{\pb}Herrn Dr. v. \textsc{Hofma{\geminationn}sthal}\pend
           \pstart
           \noindent{}\raggedleft{}verehrungsvoll\pend
           \pstart \spacefill\mbox{ArthSch.}\pend{}{\bigskip}\pstart
           \noindent{}\textcolor{gray}{\textbf{\uline{Manuſkript.}}}\pend
           \pstart
           \centering{}\textcolor{gray}{\textbf{\so{Das Märchen}\pwindex{Schnitzler, Arthur 15.05.1862 – 21.10.1931@\textsc{Schnitzler, Arthur} (15.05.1862 – 21.10.1931), \emph{Schriftsteller, Mediziner}!Maerchen. Schauspiel in drei Aufzuegen1891 – 1891@\strich\emph{Das Märchen. Schauspiel in drei Aufzügen} {[}1891 – 1891{]}|pw}.}}\pend
           \pstart
           \noindent{}\centering{}\textcolor{gray}{\textbf{Schauſpiel in drei Aufzügen}}{\\}\textcolor{gray}{\textbf{von}}{\\}\textcolor{gray}{\textbf{Arthur Schnitzler.}}\pend
           {\bigskip}\pstart
           \noindent{}\centering{}\textcolor{gray}{\textbf{Wien\oindex{Wien@\textbf{Wien}|pw}{ }\label{K_L00048_1v}\edtext{1891}{\lemma{\textnormal{\emph{1891}}}\Cendnote{\textnormal{Schnitzler\pwindex{Schnitzler, Arthur 15.05.1862 – 21.10.1931@\textsc{Schnitzler, Arthur} (15.05.1862 – 21.10.1931), \emph{Schriftsteller, Mediziner}|pwk} bekam die Drucke des Stücks am
                           5. 12. 1891.
                        Zwei Tage später hatte es Hugo August
                           Hofmannsthal\pwindex{Hofmannsthal, Hugo August von 21.12.1841 – 08.12.1915@\textsc{Hofmannsthal, Hugo August von} (21.12.1841 – 08.12.1915), \emph{Bankdirektor}|pwk} gelesen.}}}\label{K_L00048_1h}.}}\pend
           \pstart
           \noindent{}\centering{}\textcolor{gray}{\textbf{Druck von Carl Steinhardt {\kaufmannsund} Cie.\orgindex{Carl Steinhardt und Co.@Carl Steinhardt {\kaufmannsund}  Co.|pw} (verantw. Leiter Guſtav Röttig\pwindex{Roettig, Gustav 1855-12-08 – nach 1918@\textsc{Röttig, Gustav} (1855-12-08 – nach 1918), \emph{Redakteur, Drucker}|pw}).}}\pend
           \endnumbering\briefempfaengerindex{Hofmannsthal, Hugo August von@\textsc{Hofmannsthal, Hugo August von}!zzzSchnitzler, Arthur@\emph{von Arthur Schnitzler}!1891-12-052@{{[}5.{]} 12. 1891}|)be}\mylabel{h}\end{ledgroupsized}  \newcommand{\dateiname}{L00048}\newcommand{\titel}{Arthur Schnitzler: Widmungsexemplar Das Märchen für Hugo August von Hofmannsthal, [5.] 12. 1891}\newcommand{\editorInnen}{Martin Anton Müller und Gerd-Hermann Susen}%% latex-leseansicht-abspann.tex
%% Abspann für die Leseansicht.
%% Der Schalter \ifkorrekturansicht ist bereits durch den Vorspann gesetzt.

%% latex-abspann.tex
%% Gemeinsamer Abspann für Korrekturansicht und Leseansicht.
%% Setzt den Schalter \ifkorrekturansicht voraus (gesetzt in den
%% einbindenden Dateien latex-korrekturansicht-abspann.tex bzw.
%% latex-leseansicht-abspann.tex).
%% ---------------------------------------------------------------

\normalsize

% Das esempio-Environment wird nur in der Leseansicht benötigt
\ifkorrekturansicht\else
\newenvironment{esempio}[3]%
{
    \vspace{1.5ex}
    \rlap{\underline{#1}}
    \par
    \setlength{\parindent}{0cm}
    \nopagebreak
    \leftskip=#2cm
    \rightskip=#3cm
}
{
    \par
}
\fi

\doendnotes{C}
\bigskip
\vfill

\clearpage

\footnotesize

\ifkorrekturansicht
  \lohead{\textsc{register}}
\fi

% theindex-Environment neu definieren ohne reledmac
\makeatletter
\renewenvironment{theindex}{%
  \ifkorrekturansicht
    \section*{\indexname}%
  \else
    \subsubsection*{Index der erwähnten Entitäten}%
  \fi
  \setlength{\parindent}{0pt}%
  \setlength{\parskip}{0pt plus 0.3pt}%
  \let\item\@idxitem
}{%
  \ifkorrekturansicht\clearpage\fi
}
\makeatother

\IfFileExists{\jobname-pw.ind}{\input{\jobname-pw.ind}}{}

% Quellenangabe nur in der Leseansicht
\ifkorrekturansicht\else
% Fallback-Definitionen, falls die .tex-Datei \titel etc. nicht gesetzt hat
\providecommand{\titel}{}
\providecommand{\editorInnen}{}
\providecommand{\dateiname}{\jobname}

\vspace{3cm}

\vfill

\footnotesize
\textsc{Quelle}: \titel. Herausgegeben von {\editorInnen}. In: \emph{Arthur Schnitzler: Briefwechsel mit Autorinnen und Autoren}.
 Digitale Edition, https://schnitzler-briefe.acdh.oeaw.ac.at/{\dateiname}.html (Stand \today)
\fi

\end{document}


      