%% latex-leseansicht-vorspann.tex
%% Vorspann für die Leseansicht.
%% Lädt die gemeinsame Datei latex-vorspann.tex mit nicht gesetztem Schalter.

\newif\ifkorrekturansicht
\korrekturansichtfalse

\input{../tex-inputs/latex-vorspann}

\begin{center}
            \textcolor{red}{ENTWURF, NICHT FERTIG KORRIGIERT}
                      \end{center}
            
         
         \renewcommand{\erwaehntePersonen}{Personen: Paul Goldmann, Theodore Rottenberg, Olga Schnitzler}
         \renewcommand{\erwaehnteOrte}{Orte: Berlin, Dessauer Straße, Dresden, Frankgasse 1, Grand Hotel Wien, Podmokly, Prag, Wien}
         \renewcommand{\erwaehnteWerke}{}
               \section[ Paul Goldmann an Arthur Schnitzler, 7. 8. {[}1903{]}]{ Paul Goldmann an Arthur Schnitzler, 7. 8. {[}1903{]}}\nopagebreak\mylabel{v}\rehead{ }\begin{ledgroupsized}[t]{13cm}\normalsize\beginnumbering \toendnotes[C]{\smallbreak\pagebreak[2]} \Standort{DLA, A:Schnitzler, HS.NZ85.1.3173.}
\physDesc{Brief, 1 Blatt, 2 Seiten, 590 Zeichen
\newline{}Handschrift: blaue Tinte, deutsche Kurrent
\newline{}Schnitzler: mit Bleistift das Jahr »903« vermerkt }\toendnotes[C]{\smallbreak}\pstart
           \noindent{}\raggedleft{}{\pb}\textcolor{gray}{\textbf{DESSAUERSTRASSE 19\oindex{Dessauer Strasse@\textbf{Dessauer Straße}|pw}}}\pend
           \pstart
           Berlin\oindex{Berlin@\textbf{Berlin}|pw}, 7. Auguſt.\pend
           \pstart
           Tauſend Dank für Deinen lieben Brief, mein lieber und \label{K_L03382-1v}\edtext{»egoiſtiſcher«}{\lemma{\textnormal{\emph{»egoiſtiſcher«}}}\Cendnote{\textnormal{Auch wenn es sich aller Wahrscheinlichkeit nach nur um
                     eine Aussage Schnitzler\pwindex{Schnitzler, Arthur 15.05.1862 – 21.10.1931@\textsc{Schnitzler, Arthur} (15.05.1862 – 21.10.1931), \emph{Schriftsteller, Mediziner}|pwk}s vom Typ »aus
                     Eigeninteresse freue ich mich über Dein Kommen« im nicht erhaltenen Brief
                     handeln dürfte, geschieht dies doch in zeitlicher Nähe zu einer ausführlicheren
                     Erklärung Schnitzler\pwindex{Schnitzler, Arthur 15.05.1862 – 21.10.1931@\textsc{Schnitzler, Arthur} (15.05.1862 – 21.10.1931), \emph{Schriftsteller, Mediziner}|pwk}s über seinen lange
                     Zeit egoistischen Zugang bei Werkkonzeptionen (vgl. A. S.: \emph{Tagebuch}, 8. 8. 1903). Es ist zumindest vorstellbar, dass er
                     diese Selbstkritik Goldmann\pwindex{Goldmann, Paul 31.01.1865 – 25.09.1935@\textsc{Goldmann, Paul} (31.01.1865 – 25.09.1935), \emph{Schriftsteller, Journalist}|pwk} mitgeteilt
                     hatte. }}}\label{K_L03382-1h} Freund!{ }Geſtern hatte ich Nachricht von \label{K_L03382-2v}\edtext{»ihr\pwindex{Rottenberg, Theodore 1875-09-07 – 1945-04-05@\textsc{Rottenberg, Theodore} (1875-09-07 – 1945-04-05)|pwv}«}{\lemma{\textnormal{\emph{»ihr«}}}\Cendnote{\textnormal{siehe Paul Goldmann an Arthur Schnitzler, 27. 6. [1903]}}}\label{K_L03382-2h}, daß ſie mit mir kommt. Heut wieder das
               Gegentheil. So geht es ſeit zehn Tagen! Ich kann nicht mehr, und ich habe
               beſchloſſen, morgen, Samſtag, früh nach
                  Wien\oindex{Wien@\textbf{Wien}|pw} zu fahren. Ich komme \label{K_L03382-3v}\edtext{über \textsc{Bodenbach\oindex{Podmokly@\textbf{Podmokly}|pw}}}{\lemma{\textnormal{\emph{über Bodenbach}}}\Cendnote{\textnormal{über die Zugstrecke Dresden\oindex{Dresden@\textbf{Dresden}|pwk}–Prag\oindex{Prag@\textbf{Prag}|pwk}}}}\label{K_L03382-3h} um 10 Uhr 15 (glaube ich) an. Wenn Du
                  Abends ſo lange aufbleibſt, ſo hinterlaß’ mir im \textsc{Grand Hotel\oindex{Grand Hotel Wien@\textbf{Grand Hotel Wien}|pw}} einen Brief, in welchem \textsc{Café} ich Dich \label{K_L03382-4v}\edtext{finden}{\lemma{\textnormal{\emph{finden}}}\Cendnote{\textnormal{Schnitzler\pwindex{Schnitzler, Arthur 15.05.1862 – 21.10.1931@\textsc{Schnitzler, Arthur} (15.05.1862 – 21.10.1931), \emph{Schriftsteller, Mediziner}|pwk} und Olga Gussmann\pwindex{Schnitzler, Olga 17.01.1882 – 13.01.1970@\textsc{Schnitzler, Olga} (17.01.1882 – 13.01.1970), \emph{Schauspielerin, Sängerin}|pwk} verbrachten den Abend des 8. 8. 1903{ }zu Hause\oindex{Frankgasse 1@\textbf{Frankgasse 1}|pwkv}. Goldmann\pwindex{Goldmann, Paul 31.01.1865 – 25.09.1935@\textsc{Goldmann, Paul} (31.01.1865 – 25.09.1935), \emph{Schriftsteller, Journalist}|pwk} traf Schnitzler\pwindex{Schnitzler, Arthur 15.05.1862 – 21.10.1931@\textsc{Schnitzler, Arthur} (15.05.1862 – 21.10.1931), \emph{Schriftsteller, Mediziner}|pwk} am 9. 8. 1903.}}}\label{K_L03382-4h} kann. Bitte, laß’ Dich aber \uline{nicht im Geringſten} ſtören! Höre ich Abends{ }{\pb}nicht von Dir, ſo bin ich Sonntag{ }Vormittag bei Dir.\pend
           \pstart
           Herzlichſt Dein {\\[\baselineskip]}\spacefill\mbox{Paul Goldm}\pend
           \leftskip=0em{}
         
         \endnumbering\mylabel{h}\end{ledgroupsized}\begin{anhang}\end{anhang}\newcommand{\dateiname}{L03382}\newcommand{\titel}{Paul Goldmann an Arthur Schnitzler, 7. 8. [1903]}\newcommand{\editorInnen}{Martin Anton Müller und Laura Untner}%% latex-leseansicht-abspann.tex
%% Abspann für die Leseansicht.
%% Der Schalter \ifkorrekturansicht ist bereits durch den Vorspann gesetzt.

%% latex-abspann.tex
%% Gemeinsamer Abspann für Korrekturansicht und Leseansicht.
%% Setzt den Schalter \ifkorrekturansicht voraus (gesetzt in den
%% einbindenden Dateien latex-korrekturansicht-abspann.tex bzw.
%% latex-leseansicht-abspann.tex).
%% ---------------------------------------------------------------

\normalsize

% Das esempio-Environment wird nur in der Leseansicht benötigt
\ifkorrekturansicht\else
\newenvironment{esempio}[3]%
{
    \vspace{1.5ex}
    \rlap{\underline{#1}}
    \par
    \setlength{\parindent}{0cm}
    \nopagebreak
    \leftskip=#2cm
    \rightskip=#3cm
}
{
    \par
}
\fi

\doendnotes{C}
\bigskip
\vfill

\clearpage

\footnotesize

\ifkorrekturansicht
  \lohead{\textsc{register}}
\fi

% theindex-Environment neu definieren ohne reledmac
\makeatletter
\renewenvironment{theindex}{%
  \ifkorrekturansicht
    \section*{\indexname}%
  \else
    \subsubsection*{Index der erwähnten Entitäten}%
  \fi
  \setlength{\parindent}{0pt}%
  \setlength{\parskip}{0pt plus 0.3pt}%
  \let\item\@idxitem
}{%
  \ifkorrekturansicht\clearpage\fi
}
\makeatother

\IfFileExists{\jobname-pw.ind}{\input{\jobname-pw.ind}}{}

% Quellenangabe nur in der Leseansicht
\ifkorrekturansicht\else
% Fallback-Definitionen, falls die .tex-Datei \titel etc. nicht gesetzt hat
\providecommand{\titel}{}
\providecommand{\editorInnen}{}
\providecommand{\dateiname}{\jobname}

\vspace{3cm}

\vfill

\footnotesize
\textsc{Quelle}: \titel. Herausgegeben von {\editorInnen}. In: \emph{Arthur Schnitzler: Briefwechsel mit Autorinnen und Autoren}.
 Digitale Edition, https://schnitzler-briefe.acdh.oeaw.ac.at/{\dateiname}.html (Stand \today)
\fi

\end{document}


      