%% latex-korrekturansicht-vorspann.tex
%% Vorspann für die Korrekturansicht.
%% Lädt die gemeinsame Datei latex-vorspann.tex mit gesetztem Schalter.

\newif\ifkorrekturansicht
\korrekturansichttrue

\input{../tex-inputs/latex-vorspann}


\section[ Paul Goldmann an Arthur Schnitzler, 7. 8. {[}1903{]}]{L03382 Paul Goldmann an Arthur Schnitzler, 7. 8. {[}1903{]}}
\nopagebreak\mylabel{L03382v}
\rehead{ }\normalsize\beginnumbering\briefempfaengerindex{Schnitzler, Arthur@\textsc{Schnitzler, Arthur}!zzzGoldmann, Paul@\emph{von Paul Goldmann}!1903-08-072@{7. 8. {[}1903{]}}|(be}
\toendnotes[C]{\smallbreak\pagebreak[2]}\Standort{DLA, A:Schnitzler, HS.NZ85.1.3173.}
\physDesc{Brief, 1 Blatt, 2 Seiten, 592 Zeichen
\newline{}Handschrift: blaue Tinte, deutsche Kurrent
\newline{}Schnitzler: mit Bleistift das Jahr »903« vermerkt }\toendnotes[C]{\smallbreak}
\pstart
           \raggedleft{}{\pb}\textcolor{gray}{\textbf{DESSAUERSTRASSE 19\oindex{Dessauer Strasse@\textbf{Dessauer Straße}, \emph{Straße (K.STR)}|pw}}}\pend
           
\pstart
           Berlin\oindex{Berlin@\textbf{Berlin}, \emph{P.PPLC}|pw}, 7. Auguſt.\pend
           \vspace{0.5em}
\pstart
           Tauſend Dank für Deinen lieben Brief, mein lieber und »\label{K_L03382-1v}\edtext{egoiſtiſcher}{\lemma{\textnormal{\emph{egoiſtiſcher}}}\Cendnote{\textnormal{Auch wenn es sich aller Wahrscheinlichkeit nach nur um
                     eine Aussage Schnitzlers vom Typ ›aus
                     Eigeninteresse freue ich mich über Dein Kommen‹ im nicht erhaltenen Brief
                     gehandelt haben dürfte, wurde diese Anmerkung doch in zeitlicher Nähe zu einer
                     ausführlicheren Erklärung Schnitzlers über seinen lange Zeit egoistischen Zugang bei Werkkonzeptionen verfasst (vgl. A. S.: \emph{Tagebuch}, 8. 8. 1903). Es ist
                     zumindest vorstellbar, dass er diese Selbstkritik Goldmann\pwindex{Goldmann, Paul 31.01.1865 – 25.09.1935@\textsc{Goldmann, Paul} (31.01.1865 – 25.09.1935), \emph{Schriftsteller/Schriftstellerin, Journalist/Journalistin}|pwk} mitgeteilt hatte.}}}\label{K_L03382-1}« Freund!{ }Geſtern hatte ich Nachricht von \label{K_L03382-2v}\edtext{»ihr\pwindex{Rottenberg, Theodore 1875-09-07 – 1945-04-05@\textsc{Rottenberg, Theodore} (1875-09-07 – 1945-04-05)|pwv}«}{\lemma{\textnormal{\emph{»ihr«}}}\Cendnote{\textnormal{Siehe Paul Goldmann an Arthur Schnitzler, 27. 6. [1903].
               }}}\label{K_L03382-2}, daß ſie mit mir kommt. Heut wieder das
               Gegentheil. So geht es ſeit zehn Tagen! Ich kann nicht mehr, und ich habe
               beſchloſſen, morgen, Samſtag, früh nach
                  Wien\oindex{Wien@\textbf{Wien}, \emph{A.ADM2}|pw} zu fahren. Ich komme \label{K_L03382-3v}\edtext{über \textsc{Bodenbach\oindex{Podmokly@\textbf{Podmokly}, \emph{P.PPL}|pw}}}{\lemma{\textnormal{\emph{über Bodenbach}}}\Cendnote{\textnormal{Das heißt, er kam über die Zugstrecke Dresden\oindex{Dresden@\textbf{Dresden}, \emph{P.PPLA}|pwk}–Prag\oindex{Prag@\textbf{Prag}, \emph{A.ADM1}|pwk}.}}}\label{K_L03382-3} um 10 Uhr 15 (glaube ich) an. Wenn Du Abends ſo
               lange aufbleibſt, ſo hinterlaß’ mir im \textsc{Grand Hotel\oindex{Grand Hotel Wien@\textbf{Grand Hotel Wien}, \emph{Hotel (K.HTL)}|pw}} einen Brief, in welchem \textsc{Café} ich Dich \label{K_L03382-4v}\edtext{finden}{\lemma{\textnormal{\emph{finden}}}\Cendnote{\textnormal{Schnitzler und Olga Gussmann\pwindex{Schnitzler, Olga 17.01.1882 – 13.01.1970@\textsc{Schnitzler, Olga} (17.01.1882 – 13.01.1970), \emph{Schauspieler/Schauspielerin, Sänger/Sängerin}|pwk} verbrachten den Abend des 8. 8. 1903
                  bei sich 
                  zu Hause\oindex{Frankgasse 1@\textbf{Frankgasse 1}, \emph{Wohngebäude (K.WHS)}|pwkv}. Goldmann\pwindex{Goldmann, Paul 31.01.1865 – 25.09.1935@\textsc{Goldmann, Paul} (31.01.1865 – 25.09.1935), \emph{Schriftsteller/Schriftstellerin, Journalist/Journalistin}|pwk} traf Schnitzler am 9. 8. 1903.}}}\label{K_L03382-4} kann. Bitte, laß’ Dich aber
                  \uline{nicht im Geringſten} ſtören! Höre ich
                  Abends{ }{\pb}nicht von Dir, ſo bin ich Sonntag{ }Vormittag bei Dir.\pend
           
\pstart
           Herzlichſt Dein {\\[\baselineskip]}\spacefill\mbox{Paul Goldm}\pend
           \leftskip=0em{}\selectlanguage{ngerman}\endnumbering\briefempfaengerindex{Schnitzler, Arthur@\textsc{Schnitzler, Arthur}!zzzGoldmann, Paul@\emph{von Paul Goldmann}!1903-08-072@{7. 8. {[}1903{]}}|)be}\mylabel{L03382h}  \normalsize

\doendnotes{C}
\bigskip
\vfill

\clearpage

\footnotesize

\lohead{\textsc{register}}

% Definiere theindex-Environment komplett neu ohne reledmac
\makeatletter
\renewenvironment{theindex}{%
  \section*{\indexname}%
  \setlength{\parindent}{0pt}%
  \setlength{\parskip}{0pt plus 0.3pt}%
  \let\item\@idxitem
}{%
  \clearpage
}
\makeatother

\IfFileExists{\jobname-pw.ind}{\input{\jobname-pw.ind}}{}

\end{document}

      