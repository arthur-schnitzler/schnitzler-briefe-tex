%% latex-leseansicht-vorspann.tex
%% Vorspann für die Leseansicht.
%% Lädt die gemeinsame Datei latex-vorspann.tex mit nicht gesetztem Schalter.

\newif\ifkorrekturansicht
\korrekturansichtfalse

\input{../tex-inputs/latex-vorspann}


\section[ Paul Goldmann an Arthur Schnitzler, 7. 8. {[}1903{]}]{L03382 Paul Goldmann an Arthur Schnitzler,  7. 8. [1903]}
\nopagebreak\mylabel{L03382v}
\rehead{ }\normalsize\beginnumbering\briefempfaengerindex{Schnitzler, Arthur@\textsc{Schnitzler, Arthur}!zzzGoldmann, Paul@\emph{von Paul Goldmann}!1903-08-073@{7. 8. [1903]}|(be}
\toendnotes[C]{\smallbreak\pagebreak[2]}
\correspDesc{Versand  durch Paul Goldmann am 7. 8. [1903] in Berlin
\newline{}Erhalt  durch Arthur Schnitzler im Zeitraum [8. 8. 1903
                  – 12. 8. 1903?] in Wien}\toendnotes[C]{\smallbreak}
\Standort{DLA, A:Schnitzler, HS.NZ85.1.3173.}
\physDesc{Brief, 1 Blatt, 2 Seiten, 592 Zeichen
\newline{}Handschrift: blaue Tinte, deutsche Kurrent
\newline{}Schnitzler: mit Bleistift das Jahr »903« vermerkt }\toendnotes[C]{\smallbreak}
\pstart
           \raggedleft{}{\pb}\textcolor{gray}{\textbf{DESSAUERSTRASSE 19\oindex{Dessauer Straße@\textbf{Dessauer Straße}, \emph{Straße}|pw}}}\pend
           
\pstart
           Berlin\oindex{Berlin@\textbf{Berlin}, \emph{Hauptstadt}|pw}, 7. Auguſt.\pend
           \vspace{0.5em}
\pstart
           Tauſend Dank für Deinen lieben Brief, mein lieber und »\label{K_L03382-1v}\edtext{egoiſtiſcher}{\lemma{\textnormal{\emph{egoistischer}}}\Cendnote{\textnormal{Auch wenn es sich aller Wahrscheinlichkeit nach nur um
                     eine Aussage Schnitzlers vom Typ ›aus
                     Eigeninteresse freue ich mich über Dein Kommen‹ im nicht erhaltenen Brief
                     gehandelt haben dürfte, wurde diese Anmerkung doch in zeitlicher Nähe zu einer
                     ausführlicheren Erklärung Schnitzlers über seinen lange Zeit egoistischen Zugang bei Werkkonzeptionen verfasst (vgl. A. S.: \emph{Tagebuch}, 8. 8. 1903). Es ist
                     zumindest vorstellbar, dass er diese Selbstkritik Goldmann\pwindex{Goldmann, Paul 31.\,1.\,1865 Breslau – 25.\,9.\,1935 Wien@\textsc{Goldmann, Paul} (31.\,1.\,1865 Breslau – 25.\,9.\,1935 Wien), \emph{Schriftsteller, Journalist}|pwk} mitgeteilt hatte.}}}\label{K_L03382-1}« Freund!{ }Geſtern hatte ich Nachricht von \label{K_L03382-2v}\edtext{»ihr\pwindex{Rottenberg, Theodore 7.\,9.\,1875 – 5.\,4.\,1945 Limburg an der Lahn@\textsc{Rottenberg, Theodore} (7.\,9.\,1875 – 5.\,4.\,1945 Limburg an der Lahn)|pwv}«}{\lemma{\textnormal{\emph{»ihr«}}}\Cendnote{\textnormal{Siehe XXXX Auszeichnungsfehler: Dokument L03375 nicht gefunden.
               }}}\label{K_L03382-2}, daß{ }ſie mit mir kommt. Heut wieder das
               Gegentheil. So geht es{ }ſeit zehn Tagen! Ich kann nicht mehr, und ich habe
               beſchloſſen, morgen, Samſtag, früh nach
                  Wien\oindex{Wien@\textbf{Wien}, \emph{Verwaltungsgebiet}|pw} zu fahren. Ich komme \label{K_L03382-3v}\edtext{über \textsc{Bodenbach\oindex{Podmokly@\textbf{Podmokly}|pw}}}{\lemma{\textnormal{\emph{über Bodenbach}}}\Cendnote{\textnormal{Das heißt, er kam über die Zugstrecke Dresden\oindex{Dresden@\textbf{Dresden}|pwk}–Prag\oindex{Prag@\textbf{Prag}, \emph{Land}|pwk}.}}}\label{K_L03382-3} um 10 Uhr 15 (glaube ich) an. Wenn Du Abends{ }ſo
               lange aufbleibſt,{ }ſo hinterlaß’ mir im \textsc{Grand Hotel\oindex{Wien@\textbf{Wien}!I., Innere Stadt@\textbf{I., Innere Stadt}!Grand Hotel Wien@\textbf{Grand Hotel Wien}, \emph{Hotel}|pw}} einen Brief, in welchem \textsc{Café} ich Dich \label{K_L03382-4v}\edtext{finden}{\lemma{\textnormal{\emph{finden}}}\Cendnote{\textnormal{Schnitzler und Olga Gussmann\pwindex{Schnitzler, Olga 17.\,1.\,1882 Wien – 13.\,1.\,1970 Lugano@\textsc{Schnitzler, Olga} (17.\,1.\,1882 Wien – 13.\,1.\,1970 Lugano), \emph{Schauspielerin, Sängerin}|pwk} verbrachten den Abend des 8. 8. 1903
                  bei sich 
                  zu Hause\oindex{Wien@\textbf{Wien}!IX., Alsergrund@\textbf{IX., Alsergrund}!Frankgasse 1@\textbf{Frankgasse 1}, \emph{Wohngebäude}|pwkv}. Goldmann\pwindex{Goldmann, Paul 31.\,1.\,1865 Breslau – 25.\,9.\,1935 Wien@\textsc{Goldmann, Paul} (31.\,1.\,1865 Breslau – 25.\,9.\,1935 Wien), \emph{Schriftsteller, Journalist}|pwk} traf Schnitzler am 9. 8. 1903.}}}\label{K_L03382-4} kann. Bitte, laß’ Dich aber
                  \uline{nicht im Geringſten}{ }ſtören! Höre ich
                  Abends{ }{\pb}nicht von Dir,{ }ſo bin ich Sonntag{ }Vormittag bei Dir.\pend
           
\pstart
           Herzlichſt Dein {\\[\baselineskip]}\spacefill\mbox{Paul Goldm}\pend
           \leftskip=0em{}\selectlanguage{ngerman}\endnumbering\briefempfaengerindex{Schnitzler, Arthur@\textsc{Schnitzler, Arthur}!zzzGoldmann, Paul@\emph{von Paul Goldmann}!1903-08-073@{7. 8. [1903]}|)be}\mylabel{L03382h}  \newcommand{\dateiname}{L03382}\newcommand{\titel}{Paul Goldmann an Arthur Schnitzler, 7. 8. [1903]}\newcommand{\editorInnen}{Martin Anton Müller und Laura Untner}%% latex-leseansicht-abspann.tex
%% Abspann für die Leseansicht.
%% Der Schalter \ifkorrekturansicht ist bereits durch den Vorspann gesetzt.

%% latex-abspann.tex
%% Gemeinsamer Abspann für Korrekturansicht und Leseansicht.
%% Setzt den Schalter \ifkorrekturansicht voraus (gesetzt in den
%% einbindenden Dateien latex-korrekturansicht-abspann.tex bzw.
%% latex-leseansicht-abspann.tex).
%% ---------------------------------------------------------------

\normalsize

% Das esempio-Environment wird nur in der Leseansicht benötigt
\ifkorrekturansicht\else
\newenvironment{esempio}[3]%
{
    \vspace{1.5ex}
    \rlap{\underline{#1}}
    \par
    \setlength{\parindent}{0cm}
    \nopagebreak
    \leftskip=#2cm
    \rightskip=#3cm
}
{
    \par
}
\fi

\doendnotes{C}
\bigskip
\vfill

\clearpage

\footnotesize

\ifkorrekturansicht
  \lohead{\textsc{register}}
\fi

% theindex-Environment neu definieren ohne reledmac
\makeatletter
\renewenvironment{theindex}{%
  \ifkorrekturansicht
    \section*{\indexname}%
  \else
    \subsubsection*{Index der erwähnten Entitäten}%
  \fi
  \setlength{\parindent}{0pt}%
  \setlength{\parskip}{0pt plus 0.3pt}%
  \let\item\@idxitem
}{%
  \ifkorrekturansicht\clearpage\fi
}
\makeatother

\IfFileExists{\jobname-pw.ind}{\input{\jobname-pw.ind}}{}

% Quellenangabe nur in der Leseansicht
\ifkorrekturansicht\else
% Fallback-Definitionen, falls die .tex-Datei \titel etc. nicht gesetzt hat
\providecommand{\titel}{}
\providecommand{\editorInnen}{}
\providecommand{\dateiname}{\jobname}

\vspace{3cm}

\vfill

\footnotesize
\textsc{Quelle}: \titel. Herausgegeben von {\editorInnen}. In: \emph{Arthur Schnitzler: Briefwechsel mit Autorinnen und Autoren}.
 Digitale Edition, https://schnitzler-briefe.acdh.oeaw.ac.at/{\dateiname}.html (Stand \today)
\fi

\end{document}


