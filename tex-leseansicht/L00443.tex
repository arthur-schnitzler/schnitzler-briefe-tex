\input{../tex-inputs/latex-pdf-vorspann}
\begin{center}
            \textcolor{red}{ENTWURF. ENTZIFFERUNG NOCH NICHT KORREKTURGELESEN}
                      \end{center}
            
               \section[Richard Beer-Hofmann an Arthur Schnitzler, {[}18. 5. 1895{]}]{ Richard Beer-Hofmann an Arthur Schnitzler,
               {[}18. 5. 1895{]}}\nopagebreak\mylabel{v}\rehead{ }\begin{ledgroupsized}[t]{13cm}\normalsize\beginnumbering\briefempfaengerindex{Schnitzler, Arthur@\textsc{Schnitzler, Arthur}!zzzBeer-Hofmann, Richard@\emph{von Richard Beer-Hofmann}!1895-05-181@{{[}18. 5. 1895{]}}|(be} \toendnotes[C]{\smallbreak\pagebreak[2]} \Standort{CUL, Schnitzler, B 8.}
\physDesc{Brief, 1 Blatt, 1 Seite
\newline{}Handschrift: Bleistift, deutsche Kurrent
\newline{}Schnitzler: mit Bleistift datiert: »18/5 95« und nummeriert: »56« \newline{}Ordnung: von unbekannter Hand die Nummerierung
                           wiederholt }\toendnotes[C]{\smallbreak}\pstart{}{\pb}Lieber Arthur!\pend\pstart
           Ecke, Orchester, IV Reihe. Sind Sie \label{K_L00443_1v}\edtext{zufrieden}{\lemma{\textnormal{\emph{zufrieden}}}\Cendnote{\textnormal{Es dürfte sich um Karten für die Vorstellung von
                     \emph{Macbeth}\pwindex{\textcolor{red}{\textsuperscript{XXXX1 indx}}!Macbeth1606@\strich\emph{Macbeth} {[}1606{]}|pwk} im Burgtheater\oindex{Burgtheater@\textbf{Burgtheater}|pwk} am 21. 5. 1895
                  handeln.}}}\label{K_L00443_1h}?\pend
           \pstart
           Herzlichst{\\[\baselineskip]}\spacefill\mbox{Richard}\pend
           \leftskip=0em{}\endnumbering\briefempfaengerindex{Schnitzler, Arthur@\textsc{Schnitzler, Arthur}!zzzBeer-Hofmann, Richard@\emph{von Richard Beer-Hofmann}!1895-05-181@{{[}18. 5. 1895{]}}|)be}\mylabel{h}\end{ledgroupsized}  \newcommand{\dateiname}{L00443}\newcommand{\titel}{Richard Beer-Hofmann an Arthur Schnitzler, [18. 5. 1895]}\newcommand{\editorInnen}{Martin Anton Müller und Gerd-Hermann Susen}\input{../tex-inputs/latex-pdf-abspann}
      