%% latex-leseansicht-vorspann.tex
%% Vorspann für die Leseansicht.
%% Lädt die gemeinsame Datei latex-vorspann.tex mit nicht gesetztem Schalter.

\newif\ifkorrekturansicht
\korrekturansichtfalse

\input{../tex-inputs/latex-vorspann}


\section[Georg Brandes an Arthur Schnitzler, 11. 5. 1899]{L00916 Georg Brandes an Arthur Schnitzler, 11. 5. 1899}
\nopagebreak\mylabel{L00916v}
\rehead{ }\normalsize\beginnumbering\briefempfaengerindex{Schnitzler, Arthur@\textsc{Schnitzler, Arthur}!zzzBrandes, Georg@\emph{von Georg Brandes}!1899-05-111@{11. 5. 1899}|(be}
\toendnotes[C]{\smallbreak\pagebreak[2]}
\correspDesc{Versand  durch Georg Brandes am 11. 5. 1899 \textbf{Ort fehlend} 
\newline{}Erhalt  durch Arthur Schnitzler im Zeitraum [11. 5. 1899
                  – 15. 5. 1899?] in Wien}\toendnotes[C]{\smallbreak}
\Standort{CUL, Schnitzler, B 17.}
\physDesc{Brief, 1 Blatt, 4 Seiten, 2692 Zeichen
\newline{}Handschrift: blaue Tinte, lateinische Kurrent
\newline{}Ordnung: mit Bleistift von unbekannter Hand nummeriert:
                                    »15« }
\buchAbdrucke{\weitereDrucke{Georg Brandes, Arthur Schnitzler: \emph{Ein Briefwechsel}. Herausgegeben von Kurt Bergel. Bern: \emph{Francke} 1956, S. 75–76.} }\toendnotes[C]{\smallbreak}
\pstart
           \raggedleft{}{\pb}Kopenhagen\oindex{Kopenhagen@\textbf{Kopenhagen}, \emph{Hauptstadt}|pw}{ }11 Mai 99\pend
           \vspace{0.5em}
\pstart
           Liebster\hspace*{3.5em}Sie haben mich sehr geehrt, indem Sie mir Ihren
               Schmerz gesagt haben. Sie wünschen, dass ich darüber nichts sage, ich antworte \strikeout{denn} nur: ich habe selbst viel erfahren, Verluste
               gelitten, bisweilen recht Hartes ausgestanden; Sie sind jung, ich \introOben{}bin\introOben{} alt, \strikeout{de}ich wage deshalb sonst keinen
               Vergleich, ich glaube aber, wir haben \strikeout{e}Eins gemeinsam,
               den inneren Born, den unversiegbaren Lebenstrieb, dem das Leben immer wieder werth
               wird.\pend
           
\pstart
           Ich kann dies sagen, denn meine Lage scheint meine Worte zu verspotten. Seit
               5 Monaten liege ich zu Bett. Ich heile nicht. Eine Entzündung der Venen folgt bei mir
               immer der anderen, bisweilen bricht die Entzündung auf ein Mal an drei Stellen aus.
               5 Monate im Gefängnis machen eine lange öde Zeit. Ich erhalte mir das Leben {\pb}durch Lesen und Schreiben, erhalte
               auch bisweilen Besuche. Man hat hier eine Volksausgabe\pwindex{Brandes, Georg 4.\,2.\,1842 Kopenhagen – 19.\,2.\,1927 ebd.@\textsc{Brandes, Georg} (4.\,2.\,1842 Kopenhagen – 19.\,2.\,1927 ebd.)!Samlede Skrifter [Gesammelte Werke]@\strich\emph{Samlede Skrifter [Gesammelte Werke]}|pwv} meiner Schriften angefangen (Peter Nansen\pwindex{Nansen, Peter 20.\,1.\,1861 Kopenhagen – 31.\,7.\,1918 Mariager@\textsc{Nansen, Peter} (20.\,1.\,1861 Kopenhagen – 31.\,7.\,1918 Mariager), \emph{Schriftsteller, Journalist, Verleger}|pw} Ihr guter Bekannter ist der
               Urheber) und sie scheint Erfolg zu haben. Man hat circa 5000 Subscribenten und druckt
               6000 Exemplare. Es erscheinen alle 14 Tage 10 Bogen, und es wird etwas über 3 Jahre
               dauern. Dennoch gehören einige meiner grösseren Schriften nicht diesem Verlag. So
               viel Papier habe ich armer geschwärzt.\pend
           
\pstart
           Madame Marni\pwindex{Marni, Jeanne 31.\,1.\,1854 Toulouse – 6.\,1.\,1910 Cannes@\textsc{Marni, Jeanne} (31.\,1.\,1854 Toulouse – 6.\,1.\,1910 Cannes), \emph{Schriftstellerin}|pw}, die ich übrigens nie gesehen
               habe, schrieb mir, dass Goldmann\pwindex{Goldmann, Paul 31.\,1.\,1865 Breslau – 25.\,9.\,1935 Wien@\textsc{Goldmann, Paul} (31.\,1.\,1865 Breslau – 25.\,9.\,1935 Wien), \emph{Schriftsteller, Journalist}|pw} bei ihr
               gewesen war und sich mit Freundschaft meiner erinnert hatte, was mich erfreute. Richard Beer Hofmann\pwindex{Beer-Hofmann, Richard 11.\,7.\,1866 Wien – 26.\,9.\,1945 New York City@\textsc{Beer-Hofmann, Richard} (11.\,7.\,1866 Wien – 26.\,9.\,1945 New York City), \emph{Schriftsteller}|pw} gibt mir nie {\pb}ein Lebenszeichen.\pend
           
\pstart
           Wie gut dass Sie nicht von jenem Schriftsteller heimgesucht wurden! Lasen Sie den kl.
                  Aufsatz\pwindex{Brandes, Georg 4.\,2.\,1842 Kopenhagen – 19.\,2.\,1927 ebd.@\textsc{Brandes, Georg} (4.\,2.\,1842 Kopenhagen – 19.\,2.\,1927 ebd.)!Dänentum in Südjütland@\strich\emph{Das Dänentum in Südjütland}|pwv} pro patria den ich
               in der \uline{Zukunft}\orgindex{Zukunft@Die Zukunft|pw} vom 7 April hatte? \uline{Neue fr. Presse}\orgindex{Neue Freie Presse@Neue Freie Presse|pw} und \uline{Die Zeit}\orgindex{Zeit. Wiener Wochenschrift@Die Zeit. Wiener Wochenschrift|pw} verweigerten, ihn zu drucken. Die Oesterreicher\oindex{Österreich@\textbf{Österreich}|pw} sind preussischer\oindex{Preußen@\textbf{Preußen}|pw} als die
                  Preussen\oindex{Preußen@\textbf{Preußen}|pw}. Das arme Skandinavien\oindex{Skandinavien@\textbf{Skandinavien}|pw}, man peinigt im Süden die Schleswiger\oindex{Südschleswig@\textbf{Südschleswig}, \emph{Hauptstadt}|pw}, im Norden die Finnländer\oindex{Finnland@\textbf{Finnland}|pw}.\pend
           
\pstart
           Ich erhalte Gottlob täglich von den meisten Gegenden Europas\oindex{Europa@\textbf{Europa}|pw} Briefe und Bücher, sonst wäre ich in meinem Elend zu Grunde
               gegangen. Ich lese stetig \uline{L’Aurore}\orgindex{Aurore@L’Aurore|pw} und \uline{Le Siècle}\orgindex{Le Siècle@Le Siècle|pw}, folge so von Tag zu Tag dem Verlauf der Begebenheiten in Frankreich\oindex{Frankreich@\textbf{Frankreich}|pw}. Welches Stück Seelenlehre! Ich habe in meinem {\pb}Leben wenig so Lehrreiches
               gelesen.\pend
           
\pstart
           Ihr Buch\pwindex{Schnitzler, Arthur 15.\,5.\,1862 Wien – 21.\,10.\,1931 ebd.@\textsc{Schnitzler, Arthur} (15.\,5.\,1862 Wien – 21.\,10.\,1931 ebd.), \emph{Schriftsteller, Mediziner}!grüne Kakadu – Paracelsus – Die Gefährtin. Drei Einakter@\strich\emph{Der grüne Kakadu – Paracelsus – Die Gefährtin. Drei Einakter}|pwv} habe ich noch nicht
               erhalten; ich werde es mit derselben ernsten Aufmerksamkeit lesen, womit ich Ihnen
               immer folge. Ich las kürzlich das \uline{Vermächtnis}\pwindex{Schnitzler, Arthur 15.\,5.\,1862 Wien – 21.\,10.\,1931 ebd.@\textsc{Schnitzler, Arthur} (15.\,5.\,1862 Wien – 21.\,10.\,1931 ebd.), \emph{Schriftsteller, Mediziner}!Vermächtnis. Schauspiel in drei Akten@\strich\emph{Das Vermächtnis. Schauspiel in drei Akten}|pw} wieder; es verdient, dass man dazu zurückkehrt. Ein kleiner dummer schwedischer\oindex{Schweden@\textbf{Schweden}|pw}{ }Journalist\pwindex{Paul, Adolf 6.\,1.\,1863 Brommö – 30.\,9.\,1943 Berlin@\textsc{Paul, Adolf} (6.\,1.\,1863 Brommö – 30.\,9.\,1943 Berlin), \emph{Schriftsteller}|pwv} hatte Sie vor
               einigen Tagen in einem Stockholm\oindex{Stockholm@\textbf{Stockholm}, \emph{Hauptstadt}|pw}erblatt\orgindex{Svensk Dagbladet@Svensk Dagbladet|pwv}, das mir zugeschickt
               wird, \label{K_L00916-1v}\edtext{angegriffen\pwindex{Paul, Adolf 6.\,1.\,1863 Brommö – 30.\,9.\,1943 Berlin@\textsc{Paul, Adolf} (6.\,1.\,1863 Brommö – 30.\,9.\,1943 Berlin), \emph{Schriftsteller}!Från Berlins teatrar@\strich\emph{Från Berlins teatrar}|pwv}}{\lemma{\textnormal{\emph{angegriffen}}}\Cendnote{\textnormal{Adolf Paul\pwindex{Paul, Adolf 6.\,1.\,1863 Brommö – 30.\,9.\,1943 Berlin@\textsc{Paul, Adolf} (6.\,1.\,1863 Brommö – 30.\,9.\,1943 Berlin), \emph{Schriftsteller}|pwk}: \emph{Från Berlins teatrar}\pwindex{Paul, Adolf 6.\,1.\,1863 Brommö – 30.\,9.\,1943 Berlin@\textsc{Paul, Adolf} (6.\,1.\,1863 Brommö – 30.\,9.\,1943 Berlin), \emph{Schriftsteller}!Från Berlins teatrar@\strich\emph{Från Berlins teatrar}|pwk}. In: \emph{Svensk Dagbladet}\pwindex{Svenska Dagbladet@\emph{Svenska Dagbladet}|pwk}, Nr. 142, 8. 5. 1899,
                  S. 2.}}}\label{K_L00916-1}; es brannte mir die Finger, dagegen zu schreiben, habe es
               nicht gethan, weil ich ein wenig müde bin und soviele Correcturen täglich zu besorgen
               habe, thue es vielleicht noch. Doch ich kann Ihnen vielleicht einmal auf bessere
               Weise nützlich sein.\pend
           
\pstart
           Ich drücke Ihnen die Hand.\pend
           \pstart Ihr \spacefill\mbox{Georg Brandes}\pend{}\selectlanguage{ngerman}\endnumbering\briefempfaengerindex{Schnitzler, Arthur@\textsc{Schnitzler, Arthur}!zzzBrandes, Georg@\emph{von Georg Brandes}!1899-05-111@{11. 5. 1899}|)be}\mylabel{L00916h}  \newcommand{\dateiname}{L00916}\newcommand{\titel}{Georg Brandes an Arthur Schnitzler, 11. 5. 1899}\newcommand{\editorInnen}{Martin Anton Müller und Gerd-Hermann Susen}%% latex-leseansicht-abspann.tex
%% Abspann für die Leseansicht.
%% Der Schalter \ifkorrekturansicht ist bereits durch den Vorspann gesetzt.

%% latex-abspann.tex
%% Gemeinsamer Abspann für Korrekturansicht und Leseansicht.
%% Setzt den Schalter \ifkorrekturansicht voraus (gesetzt in den
%% einbindenden Dateien latex-korrekturansicht-abspann.tex bzw.
%% latex-leseansicht-abspann.tex).
%% ---------------------------------------------------------------

\normalsize

% Das esempio-Environment wird nur in der Leseansicht benötigt
\ifkorrekturansicht\else
\newenvironment{esempio}[3]%
{
    \vspace{1.5ex}
    \rlap{\underline{#1}}
    \par
    \setlength{\parindent}{0cm}
    \nopagebreak
    \leftskip=#2cm
    \rightskip=#3cm
}
{
    \par
}
\fi

\doendnotes{C}
\bigskip
\vfill

\clearpage

\footnotesize

\ifkorrekturansicht
  \lohead{\textsc{register}}
\fi

% theindex-Environment neu definieren ohne reledmac
\makeatletter
\renewenvironment{theindex}{%
  \ifkorrekturansicht
    \section*{\indexname}%
  \else
    \subsubsection*{Index der erwähnten Entitäten}%
  \fi
  \setlength{\parindent}{0pt}%
  \setlength{\parskip}{0pt plus 0.3pt}%
  \let\item\@idxitem
}{%
  \ifkorrekturansicht\clearpage\fi
}
\makeatother

\IfFileExists{\jobname-pw.ind}{\input{\jobname-pw.ind}}{}

% Quellenangabe nur in der Leseansicht
\ifkorrekturansicht\else
% Fallback-Definitionen, falls die .tex-Datei \titel etc. nicht gesetzt hat
\providecommand{\titel}{}
\providecommand{\editorInnen}{}
\providecommand{\dateiname}{\jobname}

\vspace{3cm}

\vfill

\footnotesize
\textsc{Quelle}: \titel. Herausgegeben von {\editorInnen}. In: \emph{Arthur Schnitzler: Briefwechsel mit Autorinnen und Autoren}.
 Digitale Edition, https://schnitzler-briefe.acdh.oeaw.ac.at/{\dateiname}.html (Stand \today)
\fi

\end{document}


