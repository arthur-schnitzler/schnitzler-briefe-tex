%% latex-leseansicht-vorspann.tex
%% Vorspann für die Leseansicht.
%% Lädt die gemeinsame Datei latex-vorspann.tex mit nicht gesetztem Schalter.

\newif\ifkorrekturansicht
\korrekturansichtfalse

\input{../tex-inputs/latex-vorspann}


         
         \newcommand{\erwaehntePersonen}{Personen: Leopold von Andrian-Werburg, Moritz Mayer}
         \newcommand{\erwaehnteOrte}{Orte: Bahnhof, Frankgasse, Grand Hotel Bauer-Grünwald, IX., Alsergrund, Venedig, Wien, Österreich}
         \newcommand{\erwaehnteWerke}{Werke: Das Märchen. Schauspiel in drei Aufzügen, Die Schmetterlingsschlacht. Komödie in 4 Akten}
               \section[Richard Beer-Hofmann an Arthur Schnitzler, 28. 10. 1894]{ Richard Beer-Hofmann an Arthur Schnitzler, 28. 10. 1894}\nopagebreak\mylabel{v}\rehead{ }\begin{ledgroupsized}[t]{13cm}\normalsize\beginnumbering \toendnotes[C]{\smallbreak\pagebreak[2]} \Standort{CUL, Schnitzler, B 8.}
\physDesc{Postkarte
\newline{}Handschrift: Bleistift, deutsche Kurrent\newline{}Versand: 1) Stempel: »\nobreak{}\oindex{Grand Hotel Bauer-Gruenwald@\textbf{Grand Hotel Bauer-Grünwald}|pwk}Hotel d’Italie {\kaufmannsund}
                              Bauer Bauer Grünwald Venise, 28 Oct. 94\nobreak{}«.   2) Stempel: »\nobreak{}\oindex{Bahnhof@\textbf{Bahnhof}|pwk}Venezia Ferrovia, 28 10–94, 9 S\nobreak{}«.  3) Stempel: »\nobreak{}\oindex{IX., Alsergrund@\textbf{IX., Alsergrund}|pwk}Wien 9/3, 30. 10. 94, 8.V, Bestellt\nobreak{}«. 
\newline{}Schnitzler: mit Bleistift datiert: »28/10 94« und nummeriert: »31« }\buchAbdrucke{\weitereDrucke{Arthur Schnitzler, Richard Beer-Hofmann: \emph{Briefwechsel 1891–1931}. Hg. Konstanze Fliedl. Wien, Zürich: \emph{Europaverlag} 1992, S. 69–70.} }\toendnotes[C]{\smallbreak}\pstart{}{\pb}\textcolor{gray}{\textbf{A}}n\pend{}\pstart{}Herrn D\textsuperscript{r} Arthur Schnitzler\pend{}\pstart{}Wien\oindex{Wien@\textbf{Wien}|pw}\pend{}\pstart{}IX Frankgasse 1\oindex{Frankgasse@\textbf{Frankgasse}|pw}\pend{}\pstart{}Austria\oindex{Oesterreich@\textbf{Österreich}|pw}\pend{}{\bigskip}\pstart
           \raggedleft{}{\pb}Venedig\oindex{Venedig@\textbf{Venedig}|pw}. Sonntag
                     Abends\pend
           \pstart
           Lieber Arthur! Ihren Brief hab ich erhalten. Es ist wahrscheinlich
               daß ich schon Donnerstag in Wien\oindex{Wien@\textbf{Wien}|pw} bin
                  (\uline{Das ist aber njcht officiell}). Jedenfalls
               verständigen Sie mich in meine Wohnung was Donnerstag ist. Den kleinen
                  Andrian\pwindex{Andrian-Werburg, Leopold von 09.05.1875 – 19.11.1951@\textsc{Andrian-Werburg, Leopold von} (09.05.1875 – 19.11.1951), \emph{Schriftsteller, Diplomat}|pw} hab ich hier getroffen. Herr Moritz Mayer\pwindex{Mayer, Moritz @\textsc{Mayer, Moritz}|pw} der Ihr »Märchen\pwindex{Schnitzler, Arthur 15.05.1862 – 21.10.1931@\textsc{Schnitzler, Arthur} (15.05.1862 – 21.10.1931), \emph{Schriftsteller, Mediziner}!Maerchen. Schauspiel in drei Aufzuegen1893-12-01@\strich\emph{Das Märchen. Schauspiel in drei Aufzügen} {[}1893-12-01{]}|pw}« so hasst daß er hier wieder davon zu reden anfieng hebt die
                  »Schmetterlingsschlacht\pwindex{\textcolor{red}{\textsuperscript{XXXX1 indx}}!Schmetterlingsschlacht. Komoedie in 4 Akten1894-10-06@\strich\emph{Die Schmetterlingsschlacht. Komödie in 4 Akten} {[}1894-10-06{]}|pw}« in den Hi{\geminationm}el. Das hat \label{T_L00394_1v}\edtext{ihr
                  noch}{\lemma{\textnormal{\emph{ihr
                  noch}}}\Cendnote{\textnormal{weiter am rechten Rand}}}\label{T_L00394_1h} gefehlt!
                  \spacefill\mbox{Richard}\pend
           
         
         \endnumbering\mylabel{h}\end{ledgroupsized}  \newcommand{\dateiname}{L00394}\newcommand{\titel}{Richard Beer-Hofmann an Arthur Schnitzler, 28. 10. 1894}\newcommand{\editorInnen}{Martin Anton Müller und Gerd-Hermann Susen}%% latex-leseansicht-abspann.tex
%% Abspann für die Leseansicht.
%% Der Schalter \ifkorrekturansicht ist bereits durch den Vorspann gesetzt.

%% latex-abspann.tex
%% Gemeinsamer Abspann für Korrekturansicht und Leseansicht.
%% Setzt den Schalter \ifkorrekturansicht voraus (gesetzt in den
%% einbindenden Dateien latex-korrekturansicht-abspann.tex bzw.
%% latex-leseansicht-abspann.tex).
%% ---------------------------------------------------------------

\normalsize

% Das esempio-Environment wird nur in der Leseansicht benötigt
\ifkorrekturansicht\else
\newenvironment{esempio}[3]%
{
    \vspace{1.5ex}
    \rlap{\underline{#1}}
    \par
    \setlength{\parindent}{0cm}
    \nopagebreak
    \leftskip=#2cm
    \rightskip=#3cm
}
{
    \par
}
\fi

\doendnotes{C}
\bigskip
\vfill

\clearpage

\footnotesize

\ifkorrekturansicht
  \lohead{\textsc{register}}
\fi

% theindex-Environment neu definieren ohne reledmac
\makeatletter
\renewenvironment{theindex}{%
  \ifkorrekturansicht
    \section*{\indexname}%
  \else
    \subsubsection*{Index der erwähnten Entitäten}%
  \fi
  \setlength{\parindent}{0pt}%
  \setlength{\parskip}{0pt plus 0.3pt}%
  \let\item\@idxitem
}{%
  \ifkorrekturansicht\clearpage\fi
}
\makeatother

\IfFileExists{\jobname-pw.ind}{\input{\jobname-pw.ind}}{}

% Quellenangabe nur in der Leseansicht
\ifkorrekturansicht\else
% Fallback-Definitionen, falls die .tex-Datei \titel etc. nicht gesetzt hat
\providecommand{\titel}{}
\providecommand{\editorInnen}{}
\providecommand{\dateiname}{\jobname}

\vspace{3cm}

\vfill

\footnotesize
\textsc{Quelle}: \titel. Herausgegeben von {\editorInnen}. In: \emph{Arthur Schnitzler: Briefwechsel mit Autorinnen und Autoren}.
 Digitale Edition, https://schnitzler-briefe.acdh.oeaw.ac.at/{\dateiname}.html (Stand \today)
\fi

\end{document}


      