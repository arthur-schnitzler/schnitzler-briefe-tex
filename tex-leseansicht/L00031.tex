\input{../tex-inputs/latex-pdf-vorspann}
\begin{center}
            \textcolor{red}{ENTWURF. ENTZIFFERUNG NOCH NICHT KORREKTURGELESEN}
                      \end{center}
            
               \section[Richard Beer-Hofmann an Arthur Schnitzler, 12. 8. 1891]{ Richard Beer-Hofmann an Arthur Schnitzler,
               12. 8. 1891}\nopagebreak\mylabel{v}\rehead{ }\begin{ledgroupsized}[t]{13cm}\normalsize\beginnumbering\briefempfaengerindex{Schnitzler, Arthur@\textsc{Schnitzler, Arthur}!zzzBeer-Hofmann, Richard@\emph{von Richard Beer-Hofmann}!1891-08-121@{12. 8. 1891}|(be} \toendnotes[C]{\smallbreak\pagebreak[2]} \Standort{CUL, Schnitzler, B 8.}
\physDesc{Briefkarte
\newline{}Handschrift: schwarze Tinte, lateinische Kurrent
\newline{}Schnitzler: mit Bleistift nummeriert: »3.« }\buchAbdrucke{\weitereDrucke{Arthur Schnitzler, Richard Beer-Hofmann: \emph{Briefwechsel 1891–1931}. Hg. Konstanze Fliedl. Wien, Zürich: \emph{Europaverlag} 1992, S. 31.} }\pstart
           \noindent{}{\pb}Lieber Arthur! \hspace*{2em}Ich danke Ihnen daß Sie mir trotz meines Schweigens
               schrieben. Mir geht es lange nicht so gut als Sie wünschen. Sti{\geminationm}ung tief unter Null. Bitte schreiben, oder telegrafiren
               Sie; wo und wann wir uns in Ischl\oindex{Bad Ischl@\textbf{Bad Ischl}|pw} treffen sollen.
                  Leo Fan-Jung\pwindex{Van-Jung, Leo 15.10.1866 – 02.07.1939@\textsc{Van-Jung, Leo} (15.10.1866 – 02.07.1939), \emph{Gesangspädagoge, Mathematiker}|pw} ist seit einigen Tagen hier\oindex{Bad Aussee@\textbf{Bad Aussee}|pw} (mit Familie). Ich stehe also zu Verfügung.\pend
           \pstart
           Auf Wiedersehen{\\[\baselineskip]}Ihr\spacefill\mbox{Richard}\pend
           \leftskip=0em{}\pstart
           12. Aug. 91.\pend
           \endnumbering\briefempfaengerindex{Schnitzler, Arthur@\textsc{Schnitzler, Arthur}!zzzBeer-Hofmann, Richard@\emph{von Richard Beer-Hofmann}!1891-08-121@{12. 8. 1891}|)be}\mylabel{h}\end{ledgroupsized}  \newcommand{\dateiname}{L00031}\newcommand{\titel}{Richard Beer-Hofmann an Arthur Schnitzler, 12. 8. 1891}\newcommand{\editorInnen}{Martin Anton Müller und Gerd-Hermann Susen}\input{../tex-inputs/latex-pdf-abspann}
      