%% latex-korrekturansicht-vorspann.tex
%% Vorspann für die Korrekturansicht.
%% Lädt die gemeinsame Datei latex-vorspann.tex mit gesetztem Schalter.

\newif\ifkorrekturansicht
\korrekturansichttrue

\input{../tex-inputs/latex-vorspann}


\section[Richard Beer-Hofmann an Arthur Schnitzler, 12. 8. 1891]{L00031 Richard Beer-Hofmann an Arthur Schnitzler,12. 8. 1891}
\nopagebreak\mylabel{L00031v}
\rehead{ }\normalsize\beginnumbering\briefempfaengerindex{Schnitzler, Arthur@\textsc{Schnitzler, Arthur}!zzzBeer-Hofmann, Richard@\emph{von Richard Beer-Hofmann}!1891-08-121@{12. 8. 1891}|(be}
\toendnotes[C]{\smallbreak\pagebreak[2]}\Standort{CUL, Schnitzler, B 8.}
\physDesc{Briefkarte, 352 Zeichen
\newline{}Handschrift: schwarze Tinte, lateinische Kurrent
\newline{}Schnitzler: mit Bleistift nummeriert: »3.« }
\buchAbdrucke{\weitereDrucke{Arthur Schnitzler, Richard Beer-Hofmann: \emph{Briefwechsel 1891–1931}. Wien, Zürich: \emph{Europaverlag} 1992, S. 31.} }
\pstart
           \noindent{}{\pb}Lieber Arthur!\hspace*{2em}Ich danke Ihnen daß Sie mir trotz meines Schweigens
               schrieben. Mir geht es lange nicht so gut als Sie wünschen. Sti{\geminationm}ung tief unter Null. Bitte schreiben, oder telegrafiren
               Sie; wo und wann wir uns in Ischl\oindex{Bad Ischl@\textbf{Bad Ischl}|pw} treffen sollen.
                  Leo Fan-Jung\pwindex{Van-Jung, Leo 15.\,10.\,1866 Odessa – 2.\,7.\,1939 Riga@\textsc{Van-Jung, Leo} (15.\,10.\,1866 Odessa – 2.\,7.\,1939 Riga), \emph{Gesangspädagoge, Mathematiker}|pw} ist seit einigen Tagen hier\oindex{Bad Aussee@\textbf{Bad Aussee}|pw} (mit Familie). Ich stehe also zu
               Verfügung.\pend
           
\pstart
           Auf Wiedersehen{\\[\baselineskip]}Ihr\spacefill\mbox{Richard}\pend
           \leftskip=0em{}
\pstart
           12. Aug. 91.\pend
           \selectlanguage{ngerman}\endnumbering\briefempfaengerindex{Schnitzler, Arthur@\textsc{Schnitzler, Arthur}!zzzBeer-Hofmann, Richard@\emph{von Richard Beer-Hofmann}!1891-08-121@{12. 8. 1891}|)be}\mylabel{L00031h}  \normalsize

\doendnotes{C}
\bigskip
\vfill

\clearpage

\footnotesize

\lohead{\textsc{register}}

% Definiere theindex-Environment komplett neu ohne reledmac
\makeatletter
\renewenvironment{theindex}{%
  \section*{\indexname}%
  \setlength{\parindent}{0pt}%
  \setlength{\parskip}{0pt plus 0.3pt}%
  \let\item\@idxitem
}{%
  \clearpage
}
\makeatother

\IfFileExists{\jobname-pw.ind}{\input{\jobname-pw.ind}}{}

\end{document}

      