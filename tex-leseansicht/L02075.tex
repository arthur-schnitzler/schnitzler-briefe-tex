%% latex-leseansicht-vorspann.tex
%% Vorspann für die Leseansicht.
%% Lädt die gemeinsame Datei latex-vorspann.tex mit nicht gesetztem Schalter.

\newif\ifkorrekturansicht
\korrekturansichtfalse

\input{../tex-inputs/latex-vorspann}


         
         \renewcommand{\erwaehntePersonen}{Personen: Hugo von Hofmannsthal}
         \renewcommand{\erwaehnteOrte}{Orte: Neues Deutsches Theater, Prag, Wien}
         \renewcommand{\erwaehnteWerke}{Werke: Der einsame Weg. Schauspiel in fünf Akten, Tagebuch}
               \section[Arthur Schnitzler an Hugo von Hofmannsthal, 1{[}3?{]}. 6. 1912]{ Arthur Schnitzler an Hugo von Hofmannsthal, 1{[}3?{]}. 6. 1912}\nopagebreak\mylabel{v}\rehead{ }\begin{ledgroupsized}[t]{13cm}\normalsize\beginnumbering\briefempfaengerindex{Hofmannsthal, Hugo von@\textsc{Hofmannsthal, Hugo von}!zzzSchnitzler, Arthur@\emph{von Arthur Schnitzler}!1912-06-131@{1{[}3?{]}. 6. 1912}|(be} \toendnotes[C]{\smallbreak\pagebreak[2]} \Standort{FDH, Hs-30885,145.}
\physDesc{Brief, 1 Blatt, 3 Seiten, 796 Zeichen (Briefpapier mit Trauerrand)
\newline{}Handschrift: schwarze Tinte, deutsche Kurrent}\buchAbdrucke{\weitereDrucke{Hugo von Hofmannsthal, Arthur Schnitzler: \emph{Briefwechsel}. Hg. Therese Nickl und Heinrich Schnitzler. Frankfurt am Main: \emph{S. Fischer} 1964, S. 268.} }\toendnotes[C]{\smallbreak}\pstart
           {\pb}\textcolor{gray}{\textbf{A. S.}}\hfill Wien\oindex{Wien@\textbf{Wien}|pw}, 12. 6. 912\pend
           \pstart
           Mein lieber Hugo, für Ihren ſchönen Brief, der mir ans Herz
               gegriffen hat, muß ich Ihnen gleich danken. Zu erwidern hab ich nur mit dem Wunſch,
               daſs es zwiſchen uns bleibe, wie es war und iſt, was die unzerſtörbare innere
               Verknüpfg anbelangt – daſs aber die äußern Verknüpfungen ſich etwas {\pb}häufiger ergeben ſollten, als bisher. Denn das
               »Umeinanderwiſſen« iſt zwar ein edles und ſchmackhaftes aber doch ein magers Brod für
               die Seele. Und um gleich den Anfang zu machen, wir möchten gerne \label{K_L02075-1v}\edtext{nächſte Woche}{\lemma{\textnormal{\emph{nächſte Woche}}}\Cendnote{\textnormal{Siehe A. S.: \emph{Tagebuch}, 20. 6. 1912.
               }}}\label{K_L02075-1h} bei Euch angefahren kommen, in den frühen Abendſtunden; gegen Ende, ich
               ſchreibe oder telegrafire den Tag \introOben{}am\introOben{} Montag oder Dinſtag,
                  {\pb}jetzt mach ich mich eben fertig, um \label{K_L02075-2v}\edtext{nach Prag\oindex{Prag@\textbf{Prag}|pw}}{\lemma{\textnormal{\emph{nach Prag}}}\Cendnote{\textnormal{Da er erst für den 13. 6. 1912 im \emph{Tagebuch}\pwindex{\textcolor{red}{\textsuperscript{XXXX1 indx}}!Tagebuch1981 – 2000@\strich\emph{Tagebuch} {[}Hrsg., 1981 – 2000{]}|pwk} festhielt, zu packen und abzureisen,
                  ohnedies nur einen Tag in Prag\oindex{Prag@\textbf{Prag}|pwk} blieb und am
                     15. 6. 1912
                  bereits retour fuhr, dürfte die Datierung Schnitzlers\pwindex{Schnitzler, Arthur 15.05.1862 – 21.10.1931@\textsc{Schnitzler, Arthur} (15.05.1862 – 21.10.1931), \emph{Schriftsteller, Mediziner}|pwk} nicht stimmen. Am 14. 6. 1912 wurde \emph{Der einsame Weg}\pwindex{Schnitzler, Arthur 15.05.1862 – 21.10.1931@\textsc{Schnitzler, Arthur} (15.05.1862 – 21.10.1931), \emph{Schriftsteller, Mediziner}!einsame Weg. Schauspiel in fuenf Akten1904@\strich\emph{Der einsame Weg. Schauspiel in fünf Akten} {[}1904{]}|pwk} am Neuen
                     Deutschen Theater\oindex{Neues Deutsches Theater@\textbf{Neues Deutsches Theater}|pwk} aufgeführt. Laut Ankündigung war es der 12. Teil des
                  »Arthur Schnitzler-Zyklus«.}}}\label{K_L02075-2h} zu fahren, wo ich gezy\substVorne{}\textsuperscript{c}\substDazwischen{}k\substHinten{}elt werde. Ich ſoll mir den Eins. Weg\pwindex{Schnitzler, Arthur 15.05.1862 – 21.10.1931@\textsc{Schnitzler, Arthur} (15.05.1862 – 21.10.1931), \emph{Schriftsteller, Mediziner}!einsame Weg. Schauspiel in fuenf Akten1904@\strich\emph{Der einsame Weg. Schauspiel in fünf Akten} {[}1904{]}|pw}
               vorſpielen laſſen.\pend
           \pstart
           Wir grüßen Euch herzlichſt{\\[\baselineskip]} Ihr{\\[\baselineskip]}\spacefill\mbox{Arthur}\pend
           \leftskip=0em{}
         
         \endnumbering\mylabel{h}\end{ledgroupsized}  \newcommand{\dateiname}{L02075}\newcommand{\titel}{Arthur Schnitzler an Hugo von Hofmannsthal, 1[3?]. 6. 1912}\newcommand{\editorInnen}{Martin Anton Müller und Gerd-Hermann Susen}%% latex-leseansicht-abspann.tex
%% Abspann für die Leseansicht.
%% Der Schalter \ifkorrekturansicht ist bereits durch den Vorspann gesetzt.

%% latex-abspann.tex
%% Gemeinsamer Abspann für Korrekturansicht und Leseansicht.
%% Setzt den Schalter \ifkorrekturansicht voraus (gesetzt in den
%% einbindenden Dateien latex-korrekturansicht-abspann.tex bzw.
%% latex-leseansicht-abspann.tex).
%% ---------------------------------------------------------------

\normalsize

% Das esempio-Environment wird nur in der Leseansicht benötigt
\ifkorrekturansicht\else
\newenvironment{esempio}[3]%
{
    \vspace{1.5ex}
    \rlap{\underline{#1}}
    \par
    \setlength{\parindent}{0cm}
    \nopagebreak
    \leftskip=#2cm
    \rightskip=#3cm
}
{
    \par
}
\fi

\doendnotes{C}
\bigskip
\vfill

\clearpage

\footnotesize

\ifkorrekturansicht
  \lohead{\textsc{register}}
\fi

% theindex-Environment neu definieren ohne reledmac
\makeatletter
\renewenvironment{theindex}{%
  \ifkorrekturansicht
    \section*{\indexname}%
  \else
    \subsubsection*{Index der erwähnten Entitäten}%
  \fi
  \setlength{\parindent}{0pt}%
  \setlength{\parskip}{0pt plus 0.3pt}%
  \let\item\@idxitem
}{%
  \ifkorrekturansicht\clearpage\fi
}
\makeatother

\IfFileExists{\jobname-pw.ind}{\input{\jobname-pw.ind}}{}

% Quellenangabe nur in der Leseansicht
\ifkorrekturansicht\else
% Fallback-Definitionen, falls die .tex-Datei \titel etc. nicht gesetzt hat
\providecommand{\titel}{}
\providecommand{\editorInnen}{}
\providecommand{\dateiname}{\jobname}

\vspace{3cm}

\vfill

\footnotesize
\textsc{Quelle}: \titel. Herausgegeben von {\editorInnen}. In: \emph{Arthur Schnitzler: Briefwechsel mit Autorinnen und Autoren}.
 Digitale Edition, https://schnitzler-briefe.acdh.oeaw.ac.at/{\dateiname}.html (Stand \today)
\fi

\end{document}


      