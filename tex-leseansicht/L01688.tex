%% latex-korrekturansicht-vorspann.tex
%% Vorspann für die Korrekturansicht.
%% Lädt die gemeinsame Datei latex-vorspann.tex mit gesetztem Schalter.

\newif\ifkorrekturansicht
\korrekturansichttrue

\input{../tex-inputs/latex-vorspann}


\section[Hugo von Hofmannsthal an Arthur Schnitzler, 9. 7. 1907]{L01688 Hugo von Hofmannsthal an Arthur Schnitzler, 9. 7. 1907}
\nopagebreak\mylabel{L01688v}
\rehead{ }\normalsize\beginnumbering\briefempfaengerindex{Schnitzler, Arthur@\textsc{Schnitzler, Arthur}!zzzHofmannsthal, Hugo von@\emph{von Hugo von Hofmannsthal}!1907-07-091@{9. 7. 1907}|(be}
\toendnotes[C]{\smallbreak\pagebreak[2]}\Standort{CUL, Schnitzler, B 43.}
\physDesc{Postkarte, 346 Zeichen
\newline{}Handschrift: 1) schwarze Tinte, deutsche Kurrent\hspace{1em}2) schwarze Tinte, lateinische Kurrent (\noindent{}Adresse)\hspace{1em}
\newline{}Versand: 1) Stempel: »\nobreak{}\oindex{Cortina DAmpezzo@\textbf{Cortina d’Ampezzo}, \emph{P.PPLA3}|pwk}Cortina, 9. VII. 07\nobreak{}«.   2) Stempel: »\nobreak{}\oindex{Welsberg-Taisten@\textbf{Welsberg-Taisten}, \emph{A.ADM3}|pwk}Welsbe\textcolor{gray}{r}g, \textcolor{gray}{9. 7. 07}\nobreak{}«. 
\newline{}Schnitzler: mit Bleistift datiert: »9/7 907« 
\newline{}Ordnung: 1) mit Bleistift von unbekannter Hand nummeriert: »\strikeout{280}«  2) mit Bleistift von unbekannter Hand nummeriert:
                                    »282«}
\buchAbdrucke{\weitereDrucke{Hugo von Hofmannsthal, Arthur Schnitzler: \emph{Briefwechsel}. Frankfurt am Main: \emph{S. Fischer} 1964, S. 230.} }\toendnotes[C]{\smallbreak}\pstart{}{\pb}Herrn D\textsuperscript{r} Arthur Schnitzler\pend{}\pstart{}Pension Waldbrunn\oindex{Wildbad Waldbrunn@\textbf{Wildbad Waldbrunn}, \emph{S.SPA}|pw}\pend{}\pstart{}Welsberg\oindex{Welsberg-Taisten@\textbf{Welsberg-Taisten}, \emph{A.ADM3}|pw}\pend{}\pstart{}bei Toblach.\oindex{Toblach@\textbf{Toblach}, \emph{A.ADM3}|pw}\pend{}{\bigskip}\vspace{1em}
\pstart
           \noindent{}{\pb}Sind Sie wirklich dort? Das wäre
               nett.\pend
           
\pstart
           Würde es Ihnen dann paſſen, daſs wir Anfang oder Mitte nächſter Woche auf 2 Tage hin
               kämen? (natürlich ohne Störung Ihrer Arbeitsſtunden, ich arbeite auch\pwindex{Silvia im »Stern«@\emph{Silvia im »Stern«}|pwv}.) Würden wir dort für 2 Nächte
               Unterkunft finden?\hspace*{1.5em}Bitte um Antwort.\pend
           \pstart \spacefill\mbox{Hugo}\pend{}
\pstart
           \noindent{}\textsc{Cortina Hôtel Bellevue}\oindex{Hotel Bellevue [Cortina DAmpezzo]@\textbf{Hotel Bellevue [Cortina d’Ampezzo]}, \emph{Hotel (K.HTL)}|pw}\pend
           \selectlanguage{ngerman}\endnumbering\briefempfaengerindex{Schnitzler, Arthur@\textsc{Schnitzler, Arthur}!zzzHofmannsthal, Hugo von@\emph{von Hugo von Hofmannsthal}!1907-07-091@{9. 7. 1907}|)be}\mylabel{L01688h}  \normalsize

\doendnotes{C}
\bigskip
\vfill

\clearpage

\footnotesize

\lohead{\textsc{register}}

% Definiere theindex-Environment komplett neu ohne reledmac
\makeatletter
\renewenvironment{theindex}{%
  \section*{\indexname}%
  \setlength{\parindent}{0pt}%
  \setlength{\parskip}{0pt plus 0.3pt}%
  \let\item\@idxitem
}{%
  \clearpage
}
\makeatother

\IfFileExists{\jobname-pw.ind}{\input{\jobname-pw.ind}}{}

\end{document}

      