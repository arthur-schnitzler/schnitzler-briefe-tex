%% latex-korrekturansicht-vorspann.tex
%% Vorspann für die Korrekturansicht.
%% Lädt die gemeinsame Datei latex-vorspann.tex mit gesetztem Schalter.

\newif\ifkorrekturansicht
\korrekturansichttrue

\input{../tex-inputs/latex-vorspann}


\section[Michael Georg Conrad an Arthur Schnitzler, 28. 3. 1893]{L00194 Michael Georg Conrad an Arthur Schnitzler, 28. 3. 1893}
\nopagebreak\mylabel{L00194v}
\rehead{ }\normalsize\beginnumbering\briefempfaengerindex{Schnitzler, Arthur@\textsc{Schnitzler, Arthur}!zzzConrad, Michael Georg@\emph{von Michael Georg Conrad}!1893-03-281@{28. 3. 1893}|(be}
\toendnotes[C]{\smallbreak\pagebreak[2]}\Standort{CUL, Schnitzler, B 22.}
\physDesc{Postkarte, 412 Zeichen
\newline{}Handschrift: schwarze Tinte, deutsche Kurrent
\newline{}Versand: 1) Stempel: »\nobreak{}\oindex{Muenchen@\textbf{München}, \emph{P.PPLA}|pwk}München I, 
                                       28 Mär 
                                       {[}93{]}, 7–8 N\nobreak{}«.   2) Stempel: »\nobreak{}Wien 1/\nobreak{}«. 
\newline{}Ordnung: mit rotem Buntstift 
                                 von unbekannter Hand nummeriert:
                                    »
                                 2
                                 «
                               }\toendnotes[C]{\smallbreak}\pstart{}{\pb}
                  Herrn 
                  \textsc{
                     D
                     \textsuperscript{r}}
                   Arthur Schnitzler
               \pend{}\pstart{}Wien I\oindex{I., Innere Stadt@\textbf{I., Innere Stadt}, \emph{A.ADM3}|pw}
                  .
               \pend{}\pstart{}Grillparzerſtr. 7\oindex{Grillparzerstrasse@\textbf{Grillparzerstraße}, \emph{R.ST}|pw}
                  .
               \pend{}{\bigskip}\vspace{1em}
\pstart
           {\pb}München\oindex{Muenchen@\textbf{München}, \emph{P.PPLA}|pw}
                  , 
                  Steinsdorfſtr. 7\oindex{Steinsdorfstrasse@\textbf{Steinsdorfstraße}, \emph{Straße (K.STR)}|pw}
                  .
               \pend
           
\pstart
           \raggedleft{}28. 3. 93
                  .
               \pend
           \vspace{0.5em}
\pstart
           
               Beſten Dank für Ihre Zuſchrift, ſehr geehrter Herr Doktor! Haben Sie ſeit 
               A. L.\pwindex{Alkandi s Lied@\emph{Alkandi’s Lied}|pw}
                nichts mehr veröffentlicht? Stünde ich mit
               der Leitung d. 
               Hoftheaters\orgindex{Nationaltheater Muenchen@Nationaltheater München|pw}
                beſſer, würde ich gern
               perſönlich für Ihr 
               Werk\pwindex{Alkandi s Lied@\emph{Alkandi’s Lied}|pwv}
               
               eintreten. Aber ich habe von dieſer Seite auch noch nichts als Kränkungen erfahren.
               Mit hochachtungsvollem Gruße
               \hspace*{1.5em}
               Ihr ergebener
            \pend
           \pstart \spacefill\mbox{Dr. Conrad.}\pend{}\selectlanguage{ngerman}\endnumbering\briefempfaengerindex{Schnitzler, Arthur@\textsc{Schnitzler, Arthur}!zzzConrad, Michael Georg@\emph{von Michael Georg Conrad}!1893-03-281@{28. 3. 1893}|)be}\mylabel{L00194h}  \normalsize

\doendnotes{C}
\bigskip
\vfill

\clearpage

\footnotesize

\lohead{\textsc{register}}

% Definiere theindex-Environment komplett neu ohne reledmac
\makeatletter
\renewenvironment{theindex}{%
  \section*{\indexname}%
  \setlength{\parindent}{0pt}%
  \setlength{\parskip}{0pt plus 0.3pt}%
  \let\item\@idxitem
}{%
  \clearpage
}
\makeatother

\IfFileExists{\jobname-pw.ind}{\input{\jobname-pw.ind}}{}

\end{document}

      