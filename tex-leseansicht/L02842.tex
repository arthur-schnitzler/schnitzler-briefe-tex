%% latex-korrekturansicht-vorspann.tex
%% Vorspann für die Korrekturansicht.
%% Lädt die gemeinsame Datei latex-vorspann.tex mit gesetztem Schalter.

\newif\ifkorrekturansicht
\korrekturansichttrue

\input{../tex-inputs/latex-vorspann}


\section[ Paul Goldmann an Arthur Schnitzler, 10. 3. {[}1898{]}]{L02842 Paul Goldmann an Arthur Schnitzler, 10. 3. {[}1898{]}}
\nopagebreak\mylabel{L02842v}
\rehead{ }\normalsize\beginnumbering\briefempfaengerindex{Schnitzler, Arthur@\textsc{Schnitzler, Arthur}!zzzGoldmann, Paul@\emph{von Paul Goldmann}!1898-03-102@{10. 3. {[}1898{]}}|(be}
\toendnotes[C]{\smallbreak\pagebreak[2]}\Standort{DLA, A:Schnitzler, HS.NZ85.1.3168.}
\physDesc{Brief, 1 Blatt, 3 Seiten, 1260 Zeichen
\newline{}Handschrift: blaue Tinte, deutsche Kurrent
\newline{}Schnitzler: mit Bleistift das Jahr »98« vermerkt }\toendnotes[C]{\smallbreak}
\pstart
           {\pb}\textcolor{gray}{\textbf{\textbf{Frankfurter Zeitung\orgindex{Frankfurter Zeitung@Frankfurter Zeitung|pw}}}}\pend
           
\pstart
           \textcolor{gray}{\textbf{(\begin{otherlanguage}{french}Gazette de Francfort\end{otherlanguage}\orgindex{Frankfurter Zeitung@Frankfurter Zeitung|pw}).}}\pend
           
\pstart
           \textcolor{gray}{\textbf{\textbf{\begin{otherlanguage}{french}Fondateur M.\end{otherlanguage}{ }L. Sonnemann\pwindex{Sonnemann, Leopold 1831-10-29 – 1909-10-30@\textsc{Sonnemann, Leopold} (1831-10-29 – 1909-10-30), \emph{Journalist/Journalistin, Herausgeber/Herausgeberin}|pw}.}}}\pend
           
\pstart
           \begin{otherlanguage}{french}\textcolor{gray}{\textbf{Journal politique, financier,}}\end{otherlanguage}\pend
           
\pstart
           \begin{otherlanguage}{french}\textcolor{gray}{\textbf{commercial et littéraire.}}\end{otherlanguage}\pend
           
\pstart
           \begin{otherlanguage}{french}\textcolor{gray}{\textbf{\textbf{Paraissant trois fois par jour.}}}\end{otherlanguage}\pend
           
\pstart
           \begin{otherlanguage}{french}\textcolor{gray}{\textbf{\textbf{Bureau à Paris\oindex{Paris@\textbf{Paris}, \emph{P.PPLC}|pw}}}}\end{otherlanguage}\hfill \textsc{Paris\oindex{Paris@\textbf{Paris}, \emph{P.PPLC}|pw}}, 10. März.\pend
           
\pstart
           \begin{otherlanguage}{french}\textcolor{gray}{\textbf{\textbf{10 \so{Rue de la Bourse}\oindex{rue de la Bourse@\textbf{rue de la Bourse}, \emph{Straße (K.STR)}|pw}.}}}\end{otherlanguage}\pend
           \vspace{0.5em}
\pstart
           Die Geographie, mein theurer Freund, iſ niemals Deine
               ſtarke Seite geweſen. Du weißt wieder einmal nicht, wo \textsc{Wien\oindex{Wien@\textbf{Wien}, \emph{A.ADM2}|pw}} liegt. Es gehört eine erſtaunliche Unſchuld des Gemüthes dazu, um zu behaupten,
               daß der nächſte Weg von \textsc{Paris\oindex{Paris@\textbf{Paris}, \emph{P.PPLC}|pw}} nach \textsc{China\oindex{China@\textbf{China}, \emph{A.PCLI}|pw}} über Wien\oindex{Wien@\textbf{Wien}, \emph{A.ADM2}|pw} führt. Aber wenn Du nach \textsc{Genua\oindex{Genua@\textbf{Genua}, \emph{P.PPLA}|pw}} kämſt, ſo würdeſt Du damit zeigen, daß Du ein braver Burſch biſt. \introOben{}(\textsc{N. B.: Genua\oindex{Genua@\textbf{Genua}, \emph{P.PPLA}|pw}} iſt eine italien\oindex{Italien@\textbf{Italien}, \emph{A.PCLI}|pw}iſche
                  Hafenſtadt).\introOben{}\pend
           
\pstart
            Und noch eine Bitte. Haſt Du in Deiner Umgebung Jemanden, {\pb}der mir eine wirkſame Empfehlung an Irgendwen in \textsc{China\oindex{China@\textbf{China}, \emph{A.PCLI}|pw}} oder \textsc{Japan\oindex{Japan@\textbf{Japan}, \emph{A.PCLI}|pw}} geben könnte? Ich bekomme zwar ſchon genug Empfehlungen mit, aber eine mehr
               kann nicht ſchaden, und vielleicht iſt gerade dieſe die eigentlich\strikeout{\textcolor{gray}{e}} nützliche.\pend
           
\pstart
           Du glaubſt, daß Du mich beneideſt? Ich glaube, daß Du mich nicht beneiden ſollſt.
               Ruhelos und friedlos in der Welt herumirren? \strikeout{I\textcolor{gray}{n}s} Ins Weite gehen ſtatt in die Höhe, um ſich
               vorzulügen, daß man {\pb}vorwärts kommt? Ich finde darin
               nichts Beneidenswerthes. Überdies werde ich mich gräßlich \strikeout{bl} blamiren. Endlich werde ich \strikeout{\textcolor{gray}{a}} am Fieber \introOben{}oder\introOben{} an der Peſt \strikeout{\textcolor{gray}{d}r} ſterben oder irgendwo an der großen Mauer\oindex{Chinesische Mauer@\textbf{Chinesische Mauer}, \emph{S.WALL}|pw}{ }\strikeout{ermo} ermordet werden.\pend
           
\pstart
           Bitte, liebſter Freund, ſchreib’ mir nach Frankfurt\oindex{Frankfurt am Main@\textbf{Frankfurt am Main}, \emph{P.PPLA3}|pw} an die Adreſſe meiner Mutter\pwindex{Goldmann, Clementine 1842-05-15 – 1924-02-24@\textsc{Goldmann, Clementine} (1842-05-15 – 1924-02-24)|pwv} (Frau \textsc{Clementine Goldmann\pwindex{Goldmann, Clementine 1842-05-15 – 1924-02-24@\textsc{Goldmann, Clementine} (1842-05-15 – 1924-02-24)|pw}}, \textsc{Rossertstraſse 15}\oindex{Rossertstrasse@\textbf{Rossertstraße}, \emph{Straße (K.STR)}|pw}). Ich gehe wahrſcheinlich ſchon nächſter Tage dahin ab.\pend
           
\pstart
           Herzlichſt {\\[\baselineskip]}Dein {\\[\baselineskip]}\spacefill\mbox{Paul Goldmnn}\pend
           \leftskip=0em{}\selectlanguage{ngerman}\endnumbering\briefempfaengerindex{Schnitzler, Arthur@\textsc{Schnitzler, Arthur}!zzzGoldmann, Paul@\emph{von Paul Goldmann}!1898-03-102@{10. 3. {[}1898{]}}|)be}\mylabel{L02842h}  \normalsize

\doendnotes{C}
\bigskip
\vfill

\clearpage

\footnotesize

\lohead{\textsc{register}}

% Definiere theindex-Environment komplett neu ohne reledmac
\makeatletter
\renewenvironment{theindex}{%
  \section*{\indexname}%
  \setlength{\parindent}{0pt}%
  \setlength{\parskip}{0pt plus 0.3pt}%
  \let\item\@idxitem
}{%
  \clearpage
}
\makeatother

\IfFileExists{\jobname-pw.ind}{\input{\jobname-pw.ind}}{}

\end{document}

      