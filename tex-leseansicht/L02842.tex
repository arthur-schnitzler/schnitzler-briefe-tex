%% latex-leseansicht-vorspann.tex
%% Vorspann für die Leseansicht.
%% Lädt die gemeinsame Datei latex-vorspann.tex mit nicht gesetztem Schalter.

\newif\ifkorrekturansicht
\korrekturansichtfalse

\input{../tex-inputs/latex-vorspann}


         
         \newcommand{\erwaehntePersonen}{Personen: Clementine Goldmann, Leopold Sonnemann}
         \newcommand{\erwaehnteInstitutionen}{Institutionen: Frankfurter Zeitung}
         \newcommand{\erwaehnteOrte}{Orte: China, Chinesische Mauer, Frankfurt am Main, Genua, Italien, Japan, Paris, Rossertstraße, Wien, rue de la Bourse}
         \newcommand{\erwaehnteWerke}{
               \section[ Paul Goldmann an Arthur Schnitzler, 10. 3. {[}1898{]}]{ Paul Goldmann an Arthur Schnitzler, 10. 3. {[}1898{]}}\nopagebreak\mylabel{v}\rehead{ }\begin{ledgroupsized}[t]{13cm}\normalsize\beginnumbering \toendnotes[C]{\smallbreak\pagebreak[2]} \Standort{DLA, A:Schnitzler, HS.NZ85.1.3168.}
\physDesc{Brief, 1 Blatt, 3 Seiten
\newline{}Handschrift: blaue Tinte, deutsche Kurrent
\newline{}Schnitzler: mit Bleistift das Jahr »98« vermerkt }\toendnotes[C]{\smallbreak}\pstart
           \noindent{}{\pb}\textcolor{gray}{\textbf{\textbf{Frankfurter Zeitung\orgindex{Frankfurter Zeitung@Frankfurter Zeitung|pw}}}}\pend
           \pstart
           \textcolor{gray}{\textbf{(\begin{otherlanguage}{french}Gazette de Francfort\end{otherlanguage}\orgindex{Frankfurter Zeitung@Frankfurter Zeitung|pw}).}}\pend
           \pstart
           \textcolor{gray}{\textbf{\textbf{\begin{otherlanguage}{french}Fondateur M.\end{otherlanguage}{ }L. Sonnemann\pwindex{Sonnemann, Leopold 1831-10-29 – 1909-10-30@\textsc{Sonnemann, Leopold} (1831-10-29 – 1909-10-30), \emph{Journalist, Herausgeber}|pw}.}}}\pend
           \pstart
           \begin{otherlanguage}{french}\textcolor{gray}{\textbf{Journal politique, financier,}}\end{otherlanguage}\pend
           \pstart
           \begin{otherlanguage}{french}\textcolor{gray}{\textbf{commercial et littéraire.}}\end{otherlanguage}\pend
           \pstart
           \begin{otherlanguage}{french}\textcolor{gray}{\textbf{\textbf{Paraissant trois fois par jour.}}}\end{otherlanguage}\pend
           \pstart
           \begin{otherlanguage}{french}\textcolor{gray}{\textbf{\textbf{Bureau à Paris\oindex{Paris@\textbf{Paris}|pw}}}}\end{otherlanguage}\hfill \textsc{Paris\oindex{Paris@\textbf{Paris}|pw}}, 10. März.\pend
           \pstart
           \begin{otherlanguage}{french}\textcolor{gray}{\textbf{\textbf{10 \so{Rue de la Bourse}\oindex{rue de la Bourse@\textbf{rue de la Bourse}|pw}.}}}\end{otherlanguage}\pend
           \pstart
           Die Geographie, mein theurer Freund, iſ niemals Deine
               ſtarke Seite geweſen. Du weißt wieder einmal nicht, wo \textsc{Wien\oindex{Wien@\textbf{Wien}|pw}} liegt. Es gehört eine erſtaunliche Unſchuld des Gemüthes dazu, um zu behaupten,
               daß der nächſte Weg von \textsc{Paris\oindex{Paris@\textbf{Paris}|pw}} nach \textsc{China\oindex{China@\textbf{China}|pw}} über Wien\oindex{Wien@\textbf{Wien}|pw} führt. Aber wenn Du nach \textsc{Genua\oindex{Genua@\textbf{Genua}|pw}} kämſt, ſo würdeſt Du damit zeigen, daß Du ein braver Burſch biſt. \introOben{}(\textsc{N. B.: Genua\oindex{Genua@\textbf{Genua}|pw}} iſt eine italien\oindex{Italien@\textbf{Italien}|pw}iſche
                  Hafenſtadt).\introOben{}\pend
           \pstart
            Und noch eine Bitte. Haſt Du in Deiner Umgebung Jemanden, {\pb}der mir eine wirkſame Empfehlung an Irgendwen in \textsc{China\oindex{China@\textbf{China}|pw}} oder \textsc{Japan\oindex{Japan@\textbf{Japan}|pw}} geben könnte? Ich bekomme zwar ſchon genug Empfehlungen mit, aber eine mehr
               kann nicht ſchaden, und vielleicht iſt gerade dieſe die eigentlich\strikeout{\textcolor{gray}{e}} nützliche.\pend
           \pstart
           Du glaubſt, daß Du mich beneideſt? Ich glaube, daß Du mich nicht beneiden ſollſt.
               Ruhelos und friedlos in der Welt herumirren? \strikeout{I\textcolor{gray}{n}s} Ins Weite gehen ſtatt in die Höhe, um ſich
               vorzulügen, daß man {\pb}vorwärts kommt? Ich finde darin
               nichts Beneidenswerthes. Überdies werde ich mich gräßlich \strikeout{bl} blamiren. Endlich werde ich \strikeout{\textcolor{gray}{a}} am Fieber \introOben{}oder\introOben{} an der Peſt \strikeout{\textcolor{gray}{d}r} ſterben oder irgendwo an der großen Mauer\oindex{Chinesische Mauer@\textbf{Chinesische Mauer}|pw}{ }\strikeout{ermo} ermordet werden.\pend
           \pstart
           Bitte, liebſter Freund, ſchreib’ mir nach Frankfurt\oindex{Frankfurt am Main@\textbf{Frankfurt am Main}|pw} an die Adreſſe meiner Mutter\pwindex{Goldmann, Clementine 1842-05-15 – 1924-02-24@\textsc{Goldmann, Clementine} (1842-05-15 – 1924-02-24)|pwv} (Frau \textsc{Clementine Goldmann\pwindex{Goldmann, Clementine 1842-05-15 – 1924-02-24@\textsc{Goldmann, Clementine} (1842-05-15 – 1924-02-24)|pw}}, \textsc{Rossertstraße} 15\oindex{Rossertstrasse@\textbf{Rossertstraße}|pw}). Ich gehe wahrſcheinlich ſchon
               nächſter Tage dahin ab.\pend
           \pstart
           Herzlichſt {\\[\baselineskip]}Dein {\\[\baselineskip]}\spacefill\mbox{Paul Goldmnn}\pend
           \leftskip=0em{}
         
         \endnumbering\mylabel{h}\end{ledgroupsized}  \newcommand{\dateiname}{L02842}\newcommand{\titel}{Paul Goldmann an Arthur Schnitzler, 10. 3. [1898]}\newcommand{\editorInnen}{Martin Anton Müller und Laura Untner}%% latex-leseansicht-abspann.tex
%% Abspann für die Leseansicht.
%% Der Schalter \ifkorrekturansicht ist bereits durch den Vorspann gesetzt.

%% latex-abspann.tex
%% Gemeinsamer Abspann für Korrekturansicht und Leseansicht.
%% Setzt den Schalter \ifkorrekturansicht voraus (gesetzt in den
%% einbindenden Dateien latex-korrekturansicht-abspann.tex bzw.
%% latex-leseansicht-abspann.tex).
%% ---------------------------------------------------------------

\normalsize

% Das esempio-Environment wird nur in der Leseansicht benötigt
\ifkorrekturansicht\else
\newenvironment{esempio}[3]%
{
    \vspace{1.5ex}
    \rlap{\underline{#1}}
    \par
    \setlength{\parindent}{0cm}
    \nopagebreak
    \leftskip=#2cm
    \rightskip=#3cm
}
{
    \par
}
\fi

\doendnotes{C}
\bigskip
\vfill

\clearpage

\footnotesize

\ifkorrekturansicht
  \lohead{\textsc{register}}
\fi

% theindex-Environment neu definieren ohne reledmac
\makeatletter
\renewenvironment{theindex}{%
  \ifkorrekturansicht
    \section*{\indexname}%
  \else
    \subsubsection*{Index der erwähnten Entitäten}%
  \fi
  \setlength{\parindent}{0pt}%
  \setlength{\parskip}{0pt plus 0.3pt}%
  \let\item\@idxitem
}{%
  \ifkorrekturansicht\clearpage\fi
}
\makeatother

\IfFileExists{\jobname-pw.ind}{\input{\jobname-pw.ind}}{}

% Quellenangabe nur in der Leseansicht
\ifkorrekturansicht\else
% Fallback-Definitionen, falls die .tex-Datei \titel etc. nicht gesetzt hat
\providecommand{\titel}{}
\providecommand{\editorInnen}{}
\providecommand{\dateiname}{\jobname}

\vspace{3cm}

\vfill

\footnotesize
\textsc{Quelle}: \titel. Herausgegeben von {\editorInnen}. In: \emph{Arthur Schnitzler: Briefwechsel mit Autorinnen und Autoren}.
 Digitale Edition, https://schnitzler-briefe.acdh.oeaw.ac.at/{\dateiname}.html (Stand \today)
\fi

\end{document}


      