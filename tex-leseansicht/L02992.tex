%% latex-leseansicht-vorspann.tex
%% Vorspann für die Leseansicht.
%% Lädt die gemeinsame Datei latex-vorspann.tex mit nicht gesetztem Schalter.

\newif\ifkorrekturansicht
\korrekturansichtfalse

\input{../tex-inputs/latex-vorspann}

\begin{center}
            \textcolor{red}{ENTWURF, NICHT FERTIG KORRIGIERT}
                      \end{center}
            
         
         \renewcommand{\erwaehntePersonen}{Personen: Gisela Hajek, Felix Salten, Olga Schnitzler, Margot Vallo}
         \renewcommand{\erwaehnteOrte}{Orte: Pötzleinsdorf, Starkfriedgassse, VIII., Josefstadt, Wien}
         \renewcommand{\erwaehnteWerke}{}
               \section[ Arthur Schnitzler an Felix Salten, 27. 7. 1904]{ Arthur Schnitzler an Felix Salten, 27. 7. 1904}\nopagebreak\mylabel{v}\rehead{ }\begin{ledgroupsized}[t]{13cm}\normalsize\beginnumbering \toendnotes[C]{\smallbreak\pagebreak[2]} \Standort{Wienbibliothek im Rathaus, ZPH 1681, 2.1.516.}
\physDesc{Kartenbrief, 314 Zeichen
\newline{}Handschrift: Bleistift, deutsche Kurrent
\newline{}Versand: 1) Stempel: »\nobreak{}\oindex{VIII., Josefstadt@\textbf{VIII., Josefstadt}|pwk}18/1 Wi{[}en{]}, 27. VII. 04, 6\nobreak{}«.   2) Stempel: »\nobreak{}\oindex{VIII., Josefstadt@\textbf{VIII., Josefstadt}|pwk}{\pb}Wien 18/3 1\textcolor{gray}{44}, 27. 7. \textcolor{gray}{0}4, \textcolor{gray}{5} N, Bestellt\nobreak{}«. 
\newline{}Ordnung: mit Bleistift von unbekannter Hand nummeriert: »20« }\toendnotes[C]{\smallbreak}\pstart{}{\pb}Herrn \textsc{Felix
                     Salten}\pend{}\pstart{}Wien Pötzleinsdorf\oindex{Poetzleinsdorf@\textbf{Pötzleinsdorf}|pw}\pend{}\pstart{}Starkfriedgaſſe 12\oindex{Starkfriedgassse@\textbf{Starkfriedgassse}|pw}.\pend{}{\bigskip}\pstart
           \raggedleft{}{\pb}27. 7 904\pend
           \pstart
           lieber, für morgen müſſen wir leider
               abſagen. Sind mit meiner Schweſter\pwindex{Hajek, Gisela 20.12.1867 – 03.02.1953@\textsc{Hajek, Gisela} (20.12.1867 – 03.02.1953)|pwv} das erſte Mal ſeit vielen Wochen (\textsc{Margot\pwindex{Vallo, Margot 1892-09-24 – 1969-12-17@\textsc{Vallo, Margot} (1892-09-24 – 1969-12-17)|pw}\strikeout{t}} hatte Scharlach) u das letzte Mal vor ihrer Abreiſe zuſammen.\pend
           \pstart
           Auf \label{K_L02992-1v}\edtext{nächſte Woche}{\lemma{\textnormal{\emph{nächſte Woche}}}\Cendnote{\textnormal{siehe A. S.: \emph{Tagebuch}, 4. 8. 1904}}}\label{K_L02992-1h}\pend
           \pstart
           Herzlichen Gruß {\\[\baselineskip]}Ihr {\\[\baselineskip]}\spacefill\mbox{A.}\pend
           \leftskip=0em{}\pstart
           \noindent{}\label{T_L02992-1v}\edtext{Die \label{K_L02992-2v}\edtext{Bilder}{\lemma{\textnormal{\emph{Bilder}}}\Cendnote{\textnormal{siehe A. S.: \emph{Tagebuch}, 25. 7. 1904}}}\label{K_L02992-2h} ſind da{[}.{]}{ }Olga\pwindex{Schnitzler, Olga 17.01.1882 – 13.01.1970@\textsc{Schnitzler, Olga} (17.01.1882 – 13.01.1970), \emph{Schauspielerin, Sängerin}|pw} und andre ſind entzückt.}{\lemma{\textnormal{\emph{Die … entzückt.}}}\Cendnote{\textnormal{seitlich am rechten Rand, quer zum
                     Text}}}\label{T_L02992-1h}\pend
           
         
         \endnumbering\mylabel{h}\end{ledgroupsized}  \newcommand{\dateiname}{L02992}\newcommand{\titel}{Arthur Schnitzler an Felix Salten, 27. 7. 1904}\newcommand{\editorInnen}{Martin Anton Müller und Laura Untner}%% latex-leseansicht-abspann.tex
%% Abspann für die Leseansicht.
%% Der Schalter \ifkorrekturansicht ist bereits durch den Vorspann gesetzt.

%% latex-abspann.tex
%% Gemeinsamer Abspann für Korrekturansicht und Leseansicht.
%% Setzt den Schalter \ifkorrekturansicht voraus (gesetzt in den
%% einbindenden Dateien latex-korrekturansicht-abspann.tex bzw.
%% latex-leseansicht-abspann.tex).
%% ---------------------------------------------------------------

\normalsize

% Das esempio-Environment wird nur in der Leseansicht benötigt
\ifkorrekturansicht\else
\newenvironment{esempio}[3]%
{
    \vspace{1.5ex}
    \rlap{\underline{#1}}
    \par
    \setlength{\parindent}{0cm}
    \nopagebreak
    \leftskip=#2cm
    \rightskip=#3cm
}
{
    \par
}
\fi

\doendnotes{C}
\bigskip
\vfill

\clearpage

\footnotesize

\ifkorrekturansicht
  \lohead{\textsc{register}}
\fi

% theindex-Environment neu definieren ohne reledmac
\makeatletter
\renewenvironment{theindex}{%
  \ifkorrekturansicht
    \section*{\indexname}%
  \else
    \subsubsection*{Index der erwähnten Entitäten}%
  \fi
  \setlength{\parindent}{0pt}%
  \setlength{\parskip}{0pt plus 0.3pt}%
  \let\item\@idxitem
}{%
  \ifkorrekturansicht\clearpage\fi
}
\makeatother

\IfFileExists{\jobname-pw.ind}{\input{\jobname-pw.ind}}{}

% Quellenangabe nur in der Leseansicht
\ifkorrekturansicht\else
% Fallback-Definitionen, falls die .tex-Datei \titel etc. nicht gesetzt hat
\providecommand{\titel}{}
\providecommand{\editorInnen}{}
\providecommand{\dateiname}{\jobname}

\vspace{3cm}

\vfill

\footnotesize
\textsc{Quelle}: \titel. Herausgegeben von {\editorInnen}. In: \emph{Arthur Schnitzler: Briefwechsel mit Autorinnen und Autoren}.
 Digitale Edition, https://schnitzler-briefe.acdh.oeaw.ac.at/{\dateiname}.html (Stand \today)
\fi

\end{document}


      