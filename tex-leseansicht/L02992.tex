%% latex-korrekturansicht-vorspann.tex
%% Vorspann für die Korrekturansicht.
%% Lädt die gemeinsame Datei latex-vorspann.tex mit gesetztem Schalter.

\newif\ifkorrekturansicht
\korrekturansichttrue

\input{../tex-inputs/latex-vorspann}


\section[ Arthur Schnitzler an Felix Salten, 27. 7. 1904]{L02992 Arthur Schnitzler an Felix Salten, 27. 7. 1904}
\nopagebreak\mylabel{L02992v}
\rehead{ }\normalsize\beginnumbering\briefempfaengerindex{Salten, Felix@\textsc{Salten, Felix}!zzzSchnitzler, Arthur@\emph{von Arthur Schnitzler}!1904-07-271@{27. 7. 1904}|(be}
\toendnotes[C]{\smallbreak\pagebreak[2]}\Standort{Wienbibliothek im Rathaus, ZPH 1681, 2.1.516.}
\physDesc{Kartenbrief, 314 Zeichen
\newline{}Handschrift: Bleistift, deutsche Kurrent
\newline{}Versand: 1) Stempel: »\nobreak{}\oindex{VIII., Josefstadt@\textbf{VIII., Josefstadt}, \emph{A.ADM3}|pwk}18/1 Wi{[}en{]}, 27. VII. 04, 6\nobreak{}«.   2) Stempel: »\nobreak{}\oindex{VIII., Josefstadt@\textbf{VIII., Josefstadt}, \emph{A.ADM3}|pwk}{\pb}Wien 18/3 1\textcolor{gray}{44}, 27. 7. \textcolor{gray}{0}4, \textcolor{gray}{5} N, Bestellt\nobreak{}«. 
\newline{}Ordnung: mit Bleistift von unbekannter Hand nummeriert: »20« }\toendnotes[C]{\smallbreak}\pstart{}{\pb}Herrn \textsc{Felix
                     Salten}\pend{}\pstart{}Wien Pötzleinsdorf\oindex{Poetzleinsdorf@\textbf{Pötzleinsdorf}, \emph{P.PPLX}|pw}\pend{}\pstart{}Starkfriedgaſſe 12\oindex{Starkfriedgassse@\textbf{Starkfriedgassse}, \emph{Straße (K.STR)}|pw}.\pend{}{\bigskip}\vspace{1em}
\pstart
           \raggedleft{}{\pb}27. 7 904\pend
           \vspace{0.5em}
\pstart
           lieber, für morgen müſſen wir leider
               abſagen. Sind mit meiner Schweſter\pwindex{Hajek, Gisela 20.12.1867 – 03.02.1953@\textsc{Hajek, Gisela} (20.12.1867 – 03.02.1953)|pwv} das erſte Mal ſeit vielen Wochen (\textsc{Margot\pwindex{Vallo, Margot 1892-09-24 – 1969-12-17@\textsc{Vallo, Margot} (1892-09-24 – 1969-12-17)|pw}\strikeout{t}} hatte Scharlach) u das letzte Mal vor ihrer Abreiſe zuſammen.\pend
           
\pstart
           Auf \label{K_L02992-1v}\edtext{nächſte Woche}{\lemma{\textnormal{\emph{nächſte Woche}}}\Cendnote{\textnormal{Siehe A. S.: \emph{Tagebuch}, 4. 8. 1904.
               }}}\label{K_L02992-1}\pend
           
\pstart
           Herzlichen Gruß {\\[\baselineskip]}Ihr {\\[\baselineskip]}\spacefill\mbox{A.}\pend
           \leftskip=0em{}
\pstart
           \noindent{}\label{T_L02992-1v}\edtext{Die \label{K_L02992-2v}\edtext{Bilder}{\lemma{\textnormal{\emph{Bilder}}}\Cendnote{\textnormal{Siehe A. S.: \emph{Tagebuch}, 25. 7. 1904.
                  }}}\label{K_L02992-2} ſind da{[}.{]}{ }Olga\pwindex{Schnitzler, Olga 17.01.1882 – 13.01.1970@\textsc{Schnitzler, Olga} (17.01.1882 – 13.01.1970), \emph{Schauspieler/Schauspielerin, Sänger/Sängerin}|pw} und andre ſind entzückt.}{\lemma{\textnormal{\emph{Die … entzückt.}}}\Cendnote{\textnormal{seitlich am rechten Rand, quer zum
                     Text}}}\label{T_L02992-1}\pend
           \selectlanguage{ngerman}\endnumbering\briefempfaengerindex{Salten, Felix@\textsc{Salten, Felix}!zzzSchnitzler, Arthur@\emph{von Arthur Schnitzler}!1904-07-271@{27. 7. 1904}|)be}\mylabel{L02992h}  \normalsize

\doendnotes{C}
\bigskip
\vfill

\clearpage

\footnotesize

\lohead{\textsc{register}}

% Definiere theindex-Environment komplett neu ohne reledmac
\makeatletter
\renewenvironment{theindex}{%
  \section*{\indexname}%
  \setlength{\parindent}{0pt}%
  \setlength{\parskip}{0pt plus 0.3pt}%
  \let\item\@idxitem
}{%
  \clearpage
}
\makeatother

\IfFileExists{\jobname-pw.ind}{\input{\jobname-pw.ind}}{}

\end{document}

      