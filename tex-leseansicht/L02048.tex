%% latex-korrekturansicht-vorspann.tex
%% Vorspann für die Korrekturansicht.
%% Lädt die gemeinsame Datei latex-vorspann.tex mit gesetztem Schalter.

\newif\ifkorrekturansicht
\korrekturansichttrue

\input{../tex-inputs/latex-vorspann}


\section[Hugo von Hofmannsthal an Arthur Schnitzler, 30. 11. {[}1911{]}]{L02048 Hugo von Hofmannsthal an Arthur Schnitzler, 30. 11. {[}1911{]}}
\nopagebreak\mylabel{L02048v}
\rehead{ }\normalsize\beginnumbering\briefempfaengerindex{Schnitzler, Arthur@\textsc{Schnitzler, Arthur}!zzzHofmannsthal, Hugo von@\emph{von Hugo von Hofmannsthal}!1911-11-302@{30. 11. {[}1911{]}}|(be}
\toendnotes[C]{\smallbreak\pagebreak[2]}\Standort{CUL, Schnitzler, B 43.}
\physDesc{Telegramm, 105 Zeichen
\newline{}maschinell
\newline{}Versand: Stempel des Telegrafenbeamten: »\textcolor{gray}{\textbf{\textit{J. S. Steimetzer\pwindex{Steimetzer, Josef S. @\textsc{Steimetzer, Josef S.}, \emph{Telegrafenbeamter/Telegrafenbeamtin}|pw}}}}« 
\newline{}Schnitzler: mit Bleistift datiert: »30/12 911« 
\newline{}Ordnung: mit Bleistift von unbekannter Hand nummeriert:
                                    »334« }
\buchAbdrucke{\weitereDrucke{Hugo von Hofmannsthal, Arthur Schnitzler: \emph{Briefwechsel}. Frankfurt am Main: \emph{S. Fischer} 1964, S. 264.} }\toendnotes[C]{\smallbreak}
\pstart
           {\pb}berlin\oindex{Berlin@\textbf{Berlin}, \emph{P.PPLC}|pw} fd 954 17 30/11{ }7,38 s =\pend
           \vspace{0.5em}
\pstart
           danke herzlichst fuer so gute liebe worte in etwas baenglichem moment\pwindex{Jedermann. Das Spiel vom Sterben des reichen Mannes@\emph{Jedermann. Das Spiel vom Sterben des reichen Mannes}|pwv}\pend
           \pstart \spacefill\mbox{= hugo .+}\pend{}\selectlanguage{ngerman}\endnumbering\briefempfaengerindex{Schnitzler, Arthur@\textsc{Schnitzler, Arthur}!zzzHofmannsthal, Hugo von@\emph{von Hugo von Hofmannsthal}!1911-11-302@{30. 11. {[}1911{]}}|)be}\mylabel{L02048h}  \normalsize

\doendnotes{C}
\bigskip
\vfill

\clearpage

\footnotesize

\lohead{\textsc{register}}

% Definiere theindex-Environment komplett neu ohne reledmac
\makeatletter
\renewenvironment{theindex}{%
  \section*{\indexname}%
  \setlength{\parindent}{0pt}%
  \setlength{\parskip}{0pt plus 0.3pt}%
  \let\item\@idxitem
}{%
  \clearpage
}
\makeatother

\IfFileExists{\jobname-pw.ind}{\input{\jobname-pw.ind}}{}

\end{document}

      