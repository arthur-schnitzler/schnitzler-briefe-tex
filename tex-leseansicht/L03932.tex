%% latex-leseansicht-vorspann.tex
%% Vorspann für die Leseansicht.
%% Lädt die gemeinsame Datei latex-vorspann.tex mit nicht gesetztem Schalter.

\newif\ifkorrekturansicht
\korrekturansichtfalse

\input{../tex-inputs/latex-vorspann}


\section[Arthur Schnitzler an Theodor Herzl, 30. 6. 1895]{L03932 Arthur Schnitzler an Theodor Herzl, 30. 6. 1895}
\nopagebreak\mylabel{L03932v}
\rehead{ }\normalsize\beginnumbering\briefempfaengerindex{Herzl, Theodor@\textsc{Herzl, Theodor}!zzzSchnitzler, Arthur@\emph{von Arthur Schnitzler}!1895-06-301@{30. 6. 1895}|(be}
\toendnotes[C]{\smallbreak\pagebreak[2]}
\correspDesc{Versand  durch Arthur Schnitzler am 30. 6. 1895 in Wien
\newline{}Erhalt  durch Theodor Herzl in Wien}\toendnotes[C]{\smallbreak}
\Standort{Jerusalem, Central Zionist Archives, H1:1925-17.}
\physDesc{,  Blätter,  Seiten
\newline{}Handschrift: , deutsche Kurrent}
\buchAbdrucke{\weitereDrucke{Arthur Schnitzler: \emph{Briefe 1875–1912}. Herausgegeben von Therese Nickl und Heinrich Schnitzler. Frankfurt am Main: \emph{S. Fischer} 1981, S. 263–264.} }\toendnotes[C]{\smallbreak}
\pstart
           {\pb}Wien\oindex{Wien@\textbf{Wien}, \emph{Verwaltungsgebiet}|pw}{ }30. 6. 95.\pend
           
\pstart{}Lieber Freund!\pend\vspace{0.5em}
\pstart
           die wunderſchöne Sti{\geminationm}ung, von der Ihr Gedankenleben
               jetzt erfüllt \strikeout{ist} und welche in Ihrem \label{K_L03932-1v}\edtext{Brief}{\lemma{\textnormal{\emph{Brief}}}\Cendnote{\textnormal{XXXX Auszeichnungsfehler: Dokument L03896 nicht gefunden}}}\label{K_L03932-1} an mich überſtrömt iſt, freut mich um Ihres und um Ihres Werkes\pwindex{Herzl, Theodor 2.\,5.\,1860 Budapest – 3.\,7.\,1904 Edlach@\textsc{Herzl, Theodor} (2.\,5.\,1860 Budapest – 3.\,7.\,1904 Edlach), \emph{Schriftsteller, Journalist}!Judenstaat. Versuch einer modernen Lösung der Judenfrage@\strich\emph{Der Judenstaat. Versuch einer modernen Lösung der Judenfrage}|pwv} willen. Was iſt es? Wieder ein
               Stück? Wollen Sie mir auch das erſt{ }ſagen, wenn wir uns, wie ich ja mit Sicherheit
               erwarten darf, im Sommer treffen? {\pb}Ich will Ihnen gleich
               mittheilen, daſs ich etwa Mitte Juli nach Iſchl\oindex{Bad Ischl@\textbf{Bad Ischl}|pw} ko{\geminationm}e, nacher will ich mir die böhmiſchen\oindex{Böhmen@\textbf{Böhmen}, \emph{Region}|pw} Bäder anſehen, die ich noch nicht
               kenne. Sie erfahren noch ausführlicheres über meine Adreſſe; nach Wien\oindex{Wien@\textbf{Wien}, \emph{Verwaltungsgebiet}|pw} kö{\geminationn}en Sie mir i{\geminationm}er{ }ſchreiben, da mir die Briefe nachgeſchickt
               werden.\pend
           
\pstart
           Ich{ }ſelbſt hoffe über den Sommer mit einem Stück\pwindex{Schnitzler, Arthur 15.\,5.\,1862 Wien – 21.\,10.\,1931 ebd.@\textsc{Schnitzler, Arthur} (15.\,5.\,1862 Wien – 21.\,10.\,1931 ebd.), \emph{Schriftsteller, Mediziner}!Freiwild. Schauspiel in 3 Akten@\strich\emph{Freiwild. Schauspiel in 3 Akten}|pwv} zu Ende zu ko{\geminationm}en, {\pb}von dem ein halber
                  Akt\pwindex{Schnitzler, Arthur 15.\,5.\,1862 Wien – 21.\,10.\,1931 ebd.@\textsc{Schnitzler, Arthur} (15.\,5.\,1862 Wien – 21.\,10.\,1931 ebd.), \emph{Schriftsteller, Mediziner}!Freiwild. Schauspiel in 3 Akten@\strich\emph{Freiwild. Schauspiel in 3 Akten}|pwv} fertig iſt, deſſen Plan aber bis ins Detail daliegt. Auch kleineres\pwindex{Schnitzler, Arthur 15.\,5.\,1862 Wien – 21.\,10.\,1931 ebd.@\textsc{Schnitzler, Arthur} (15.\,5.\,1862 Wien – 21.\,10.\,1931 ebd.), \emph{Schriftsteller, Mediziner}!Frau des Weisen. Erzählung@\strich\emph{Die Frau des Weisen. Erzählung}|pwv}\pwindex{Schnitzler, Arthur 15.\,5.\,1862 Wien – 21.\,10.\,1931 ebd.@\textsc{Schnitzler, Arthur} (15.\,5.\,1862 Wien – 21.\,10.\,1931 ebd.), \emph{Schriftsteller, Mediziner}!Abschied@\strich\emph{Ein Abschied}|pwv} hoffe ich zuwege zu bringen.\pend
           
\pstart
           Wie ko{\geminationm}t es, daſs die Prag\oindex{Prag@\textbf{Prag}, \emph{Land}|pw}er Entſcheidung{ }ſo lang warten läßt? Haben Sie Hoffnung? –\pend
           
\pstart
           Bleiben Sie, mein lieber Freund, in Ihrer{ }ſchaffensfreudigen Laune und laſſen Sie
               mich bald wieder{ }ſo gutes wie diesmal von Ihnen vernehmen. Wie {\pb}ſchön iſt es doch um unſre Kunſt,{ }ſolang wir mit ihr allein
               bleiben und nicht das{ }ſtechende Verlangen{ }ſpüren, die ganze Welt zu Zeugen unſrer
               Umarmungen zu machen. – Zuerſt Fla{\geminationm}en, Einſamkeit und
               Begeiſterrung – dann – Agenten, Verleger, Wanzen, Publicum. –\pend
           
\pstart
           Leben Sie wohl und{ }ſeien Sie vielmals herzlichſt gegrüßt{\\[\baselineskip]}Ihr treu ergebner
                  \spacefill\mbox{ArthSch}\pend
           \leftskip=0em{}\selectlanguage{ngerman}\endnumbering\briefempfaengerindex{Herzl, Theodor@\textsc{Herzl, Theodor}!zzzSchnitzler, Arthur@\emph{von Arthur Schnitzler}!1895-06-301@{30. 6. 1895}|)be}\mylabel{L03932h}
\begin{anhang}
\end{anhang}\newcommand{\dateiname}{L03932}\newcommand{\titel}{Arthur Schnitzler an Theodor Herzl, 30. 6. 1895}\newcommand{\editorInnen}{Herausgegeben von Jahnke, SelmaMüller, Martin Anton}%% latex-leseansicht-abspann.tex
%% Abspann für die Leseansicht.
%% Der Schalter \ifkorrekturansicht ist bereits durch den Vorspann gesetzt.

%% latex-abspann.tex
%% Gemeinsamer Abspann für Korrekturansicht und Leseansicht.
%% Setzt den Schalter \ifkorrekturansicht voraus (gesetzt in den
%% einbindenden Dateien latex-korrekturansicht-abspann.tex bzw.
%% latex-leseansicht-abspann.tex).
%% ---------------------------------------------------------------

\normalsize

% Das esempio-Environment wird nur in der Leseansicht benötigt
\ifkorrekturansicht\else
\newenvironment{esempio}[3]%
{
    \vspace{1.5ex}
    \rlap{\underline{#1}}
    \par
    \setlength{\parindent}{0cm}
    \nopagebreak
    \leftskip=#2cm
    \rightskip=#3cm
}
{
    \par
}
\fi

\doendnotes{C}
\bigskip
\vfill

\clearpage

\footnotesize

\ifkorrekturansicht
  \lohead{\textsc{register}}
\fi

% theindex-Environment neu definieren ohne reledmac
\makeatletter
\renewenvironment{theindex}{%
  \ifkorrekturansicht
    \section*{\indexname}%
  \else
    \subsubsection*{Index der erwähnten Entitäten}%
  \fi
  \setlength{\parindent}{0pt}%
  \setlength{\parskip}{0pt plus 0.3pt}%
  \let\item\@idxitem
}{%
  \ifkorrekturansicht\clearpage\fi
}
\makeatother

\IfFileExists{\jobname-pw.ind}{\input{\jobname-pw.ind}}{}

% Quellenangabe nur in der Leseansicht
\ifkorrekturansicht\else
% Fallback-Definitionen, falls die .tex-Datei \titel etc. nicht gesetzt hat
\providecommand{\titel}{}
\providecommand{\editorInnen}{}
\providecommand{\dateiname}{\jobname}

\vspace{3cm}

\vfill

\footnotesize
\textsc{Quelle}: \titel. Herausgegeben von {\editorInnen}. In: \emph{Arthur Schnitzler: Briefwechsel mit Autorinnen und Autoren}.
 Digitale Edition, https://schnitzler-briefe.acdh.oeaw.ac.at/{\dateiname}.html (Stand \today)
\fi

\end{document}


