%% latex-korrekturansicht-vorspann.tex
%% Vorspann für die Korrekturansicht.
%% Lädt die gemeinsame Datei latex-vorspann.tex mit gesetztem Schalter.

\newif\ifkorrekturansicht
\korrekturansichttrue

\input{../tex-inputs/latex-vorspann}


\section[Arthur und Olga Schnitzler an Richard Beer-Hofmann, 9. 4. 1914]{L02174 Arthur und Olga Schnitzler an Richard Beer-Hofmann, 9. 4. 1914}
\nopagebreak\mylabel{L02174v}
\rehead{ }\normalsize\beginnumbering\briefempfaengerindex{Beer-Hofmann, Richard@\textsc{Beer-Hofmann, Richard}!zzzSchnitzler, Olga@\emph{von Olga Schnitzler}!1914-04-091@{9. 4. 1914}|(be}\briefempfaengerindex{Beer-Hofmann, Richard@\textsc{Beer-Hofmann, Richard}!zzzSchnitzler, Arthur@\emph{von Arthur Schnitzler}!1914-04-091@{9. 4. 1914}|(be}
\toendnotes[C]{\smallbreak\pagebreak[2]}\Standort{YCGL, MSS 31.}
\physDesc{Bildpostkarte, 242 Zeichen
\newline{}Handschrift Arthur Schnitzler: schwarze Tinte, deutsche Kurrent
\newline{}Handschrift Olga Schnitzler: schwarze Tinte, deutsche Kurrent
\newline{}Versand: 1) Stempel: »\nobreak{}Wien 111, 10. IV. 14, \textcolor{gray}{4}\nobreak{}«.   2) Stempel: »\nobreak{}\oindex{Menton@\textbf{Menton}, \emph{P.PPL}|pwk}Menton Alpes Maritimes, 14–4 14\nobreak{}«.  3) mit blauem Buntstift von unbekannter Hand Adresse gestrichen und
                                 ersetzt durch: »\noindent{}\textsc{Hôtel Regina Palace}\oindex{Hotel Regina Palace@\textbf{Hotel Regina Palace}, \emph{Hotel (K.HTL)}|pw}{ / }\textsc{Menton}\oindex{Menton@\textbf{Menton}, \emph{P.PPL}|pw}«
\newline{}Zusatz: Postkartenmotiv mit Olga und Heinrich\pwindex{Schnitzler, Heinrich 09.08.1902 – 12.07.1982@\textsc{Schnitzler, Heinrich} (09.08.1902 – 12.07.1982), \emph{Regisseur/Regisseurin, Schauspieler/Schauspielerin}|pw} links vor dem Haus und Schnitzler und Lili\pwindex{Cappellini, Lili 13.09.1909 – 26.07.1928@\textsc{Cappellini, Lili} (13.09.1909 – 26.07.1928)|pw} auf dem Söller }
\buchAbdrucke{\weitereDrucke{Arthur Schnitzler, Richard Beer-Hofmann: \emph{Briefwechsel 1891–1931}. Wien, Zürich: \emph{Europaverlag} 1992, S. 219.} }\pstart{}{\pb}Hrn \textsc{Dr. Richard}\pend{}\pstart{}\textsc{Beerhofmann}\pend{}\pstart{}aus Wien\oindex{Wien@\textbf{Wien}, \emph{A.ADM2}|pw}\pend{}\pstart{}\textsc{Kap d’Ail}\oindex{Cap-DAil@\textbf{Cap-d’Ail}, \emph{P.PPL}|pw}\pend{}\pstart{}\textsc{Riviera}\oindex{Riviera@\textbf{Riviera}, \emph{Strand (N.STR)}|pw}\pend{}\pstart{}\textsc{Frankreich\oindex{Frankreich@\textbf{Frankreich}, \emph{A.PCLI}|pw}}\pend{}{\bigskip}
\pstart
           \noindent{}{\pb}\textcolor{gray}{\textbf{Wien, XVIII, Sternwartestr. 71\oindex{Sternwartestrasse 71@\textbf{Sternwartestraße 71}, \emph{Wohngebäude (K.WHS)}|pw}.}}\pend
           \vspace{1em}
\pstart
           {\pb}9. 4. 914.\pend
           \vspace{0.5em}
\pstart
           Herzliche Grüße. Wir freuen uns daſs es Ihnen wohl gefällt u gut geht; und wagen es
               ohne Fragezeichen, die auch nichts nützen, ein gutes Wiederſehen zu erhoffen.\pend
           
\pstart
           Ihr \spacefill\mbox{Arthur}{\\[\baselineskip]}\spacefill\mbox{{[}hs. :{]} Olga}\pend
           \leftskip=0em{}\selectlanguage{ngerman}\endnumbering\briefempfaengerindex{Beer-Hofmann, Richard@\textsc{Beer-Hofmann, Richard}!zzzSchnitzler, Olga@\emph{von Olga Schnitzler}!1914-04-091@{9. 4. 1914}|)be}\briefempfaengerindex{Beer-Hofmann, Richard@\textsc{Beer-Hofmann, Richard}!zzzSchnitzler, Arthur@\emph{von Arthur Schnitzler}!1914-04-091@{9. 4. 1914}|)be}\mylabel{L02174h}  \normalsize

\doendnotes{C}
\bigskip
\vfill

\clearpage

\footnotesize

\lohead{\textsc{register}}

% Definiere theindex-Environment komplett neu ohne reledmac
\makeatletter
\renewenvironment{theindex}{%
  \section*{\indexname}%
  \setlength{\parindent}{0pt}%
  \setlength{\parskip}{0pt plus 0.3pt}%
  \let\item\@idxitem
}{%
  \clearpage
}
\makeatother

\IfFileExists{\jobname-pw.ind}{\input{\jobname-pw.ind}}{}

\end{document}

      