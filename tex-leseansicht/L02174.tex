%% latex-leseansicht-vorspann.tex
%% Vorspann für die Leseansicht.
%% Lädt die gemeinsame Datei latex-vorspann.tex mit nicht gesetztem Schalter.

\newif\ifkorrekturansicht
\korrekturansichtfalse

\input{../tex-inputs/latex-vorspann}


\section[Arthur und Olga Schnitzler an Richard Beer-Hofmann, 9. 4. 1914]{L02174 Arthur und Olga Schnitzler an Richard Beer-Hofmann, 9. 4. 1914}
\nopagebreak\mylabel{L02174v}
\rehead{ }\normalsize\beginnumbering\briefempfaengerindex{Beer-Hofmann, Richard@\textsc{Beer-Hofmann, Richard}!zzzSchnitzler, Olga@\emph{von Olga Schnitzler}!1914-04-091@{9. 4. 1914}|(be}\briefempfaengerindex{Beer-Hofmann, Richard@\textsc{Beer-Hofmann, Richard}!zzzSchnitzler, Arthur@\emph{von Arthur Schnitzler}!1914-04-091@{9. 4. 1914}|(be}
\toendnotes[C]{\smallbreak\pagebreak[2]}
\correspDesc{Versand  durch Arthur Schnitzler, Olga Schnitzler am 9. 4. 1914 in Wien
\newline{}Übermittlung  am 10. 4. 1914 in Wien
\newline{}Weiterleitung  in Cap-d’Ail
\newline{}Erhalt  durch Richard Beer-Hofmann am 14. 4. 1914 in Menton}\toendnotes[C]{\smallbreak}
\Standort{YCGL, MSS 31.}
\physDesc{Bildpostkarte, 242 Zeichen
\newline{}Handschrift Arthur Schnitzler: schwarze Tinte, deutsche Kurrent
\newline{}Handschrift Olga Schnitzler: schwarze Tinte, deutsche Kurrent
\newline{}Versand: 1) Stempel: »\nobreak{}\oindex{Wien@\textbf{Wien}, \emph{Verwaltungsgebiet}|pwk}Wien 111, 10. IV. 14, \textcolor{gray}{4}\nobreak{}«.   2) Stempel: »\nobreak{}\oindex{Menton@\textbf{Menton}|pwk}Menton Alpes Maritimes, 14–4 14\nobreak{}«.  3) mit blauem Buntstift von unbekannter Hand Adresse gestrichen und
                                 ersetzt durch: »\noindent{}\textsc{Hôtel Regina Palace}\oindex{Hotel Regina Palace@\textbf{Hotel Regina Palace}, \emph{Hotel}|pw}{ / }\textsc{Menton}\oindex{Menton@\textbf{Menton}|pw}«
\newline{}Zusatz: Postkartenmotiv mit Olga und Heinrich\pwindex{Schnitzler, Heinrich 9.\,8.\,1902 Hinterbrühl – 12.\,7.\,1982 Wien@\textsc{Schnitzler, Heinrich} (9.\,8.\,1902 Hinterbrühl – 12.\,7.\,1982 Wien), \emph{Regisseur, Schauspieler}|pw} links vor dem Haus und Schnitzler und Lili\pwindex{Cappellini, Lili 13.\,9.\,1909 Wien – 26.\,7.\,1928 Venedig@\textsc{Cappellini, Lili} (13.\,9.\,1909 Wien – 26.\,7.\,1928 Venedig)|pw} auf dem Söller }
\buchAbdrucke{\weitereDrucke{Arthur Schnitzler, Richard Beer-Hofmann: \emph{Briefwechsel 1891–1931}. Herausgegeben von Konstanze Fliedl. Wien, Zürich: \emph{Europaverlag} 1992, S. 219.} }\pstart{}{\pb}Hrn \textsc{Dr. Richard}\pend{}\pstart{}\textsc{Beerhofmann}\pend{}\pstart{}aus Wien\oindex{Wien@\textbf{Wien}, \emph{Verwaltungsgebiet}|pw}\pend{}\pstart{}\textsc{Kap d’Ail}\oindex{Cap-d’Ail@\textbf{Cap-d’Ail}|pw}\pend{}\pstart{}\textsc{Riviera}\oindex{Riviera@\textbf{Riviera}|pw}\pend{}\pstart{}\textsc{Frankreich\oindex{Frankreich@\textbf{Frankreich}|pw}}\pend{}{\bigskip}
\pstart
           \noindent{}{\pb}\textcolor{gray}{\textbf{Wien, XVIII, Sternwartestr. 71\oindex{Wien@\textbf{Wien}!XVIII., Währing@\textbf{XVIII., Währing}!Sternwartestraße 71@\textbf{Sternwartestraße 71}, \emph{Wohngebäude}|pw}.}}\pwindex{\textcolor{red}{\textsuperscript{XXXX indx1}}!Wien, XVIII, Sternwartestr. 71.@\strich\emph{Wien, XVIII, Sternwartestr. 71.}|pw}\pend
           \vspace{1em}
\pstart
           {\pb}9. 4. 914.\pend
           \vspace{0.5em}
\pstart
           Herzliche Grüße. Wir freuen uns daſs es Ihnen wohl gefällt u gut geht; und wagen es
               ohne Fragezeichen, die auch nichts nützen, ein gutes Wiederſehen zu erhoffen.\pend
           
\pstart
           Ihr \spacefill\mbox{Arthur}{\\[\baselineskip]}\spacefill\mbox{{[}hs. Schnitzler:{]} Olga}\pend
           \leftskip=0em{}\selectlanguage{ngerman}\endnumbering\briefempfaengerindex{Beer-Hofmann, Richard@\textsc{Beer-Hofmann, Richard}!zzzSchnitzler, Olga@\emph{von Olga Schnitzler}!1914-04-091@{9. 4. 1914}|)be}\briefempfaengerindex{Beer-Hofmann, Richard@\textsc{Beer-Hofmann, Richard}!zzzSchnitzler, Arthur@\emph{von Arthur Schnitzler}!1914-04-091@{9. 4. 1914}|)be}\mylabel{L02174h}  \newcommand{\dateiname}{L02174}\newcommand{\titel}{Arthur und Olga Schnitzler an Richard Beer-Hofmann, 9. 4. 1914}\newcommand{\editorInnen}{Martin Anton Müller und Gerd-Hermann Susen}%% latex-leseansicht-abspann.tex
%% Abspann für die Leseansicht.
%% Der Schalter \ifkorrekturansicht ist bereits durch den Vorspann gesetzt.

%% latex-abspann.tex
%% Gemeinsamer Abspann für Korrekturansicht und Leseansicht.
%% Setzt den Schalter \ifkorrekturansicht voraus (gesetzt in den
%% einbindenden Dateien latex-korrekturansicht-abspann.tex bzw.
%% latex-leseansicht-abspann.tex).
%% ---------------------------------------------------------------

\normalsize

% Das esempio-Environment wird nur in der Leseansicht benötigt
\ifkorrekturansicht\else
\newenvironment{esempio}[3]%
{
    \vspace{1.5ex}
    \rlap{\underline{#1}}
    \par
    \setlength{\parindent}{0cm}
    \nopagebreak
    \leftskip=#2cm
    \rightskip=#3cm
}
{
    \par
}
\fi

\doendnotes{C}
\bigskip
\vfill

\clearpage

\footnotesize

\ifkorrekturansicht
  \lohead{\textsc{register}}
\fi

% theindex-Environment neu definieren ohne reledmac
\makeatletter
\renewenvironment{theindex}{%
  \ifkorrekturansicht
    \section*{\indexname}%
  \else
    \subsubsection*{Index der erwähnten Entitäten}%
  \fi
  \setlength{\parindent}{0pt}%
  \setlength{\parskip}{0pt plus 0.3pt}%
  \let\item\@idxitem
}{%
  \ifkorrekturansicht\clearpage\fi
}
\makeatother

\IfFileExists{\jobname-pw.ind}{\input{\jobname-pw.ind}}{}

% Quellenangabe nur in der Leseansicht
\ifkorrekturansicht\else
% Fallback-Definitionen, falls die .tex-Datei \titel etc. nicht gesetzt hat
\providecommand{\titel}{}
\providecommand{\editorInnen}{}
\providecommand{\dateiname}{\jobname}

\vspace{3cm}

\vfill

\footnotesize
\textsc{Quelle}: \titel. Herausgegeben von {\editorInnen}. In: \emph{Arthur Schnitzler: Briefwechsel mit Autorinnen und Autoren}.
 Digitale Edition, https://schnitzler-briefe.acdh.oeaw.ac.at/{\dateiname}.html (Stand \today)
\fi

\end{document}


