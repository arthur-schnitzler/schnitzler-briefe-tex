%% latex-leseansicht-vorspann.tex
%% Vorspann für die Leseansicht.
%% Lädt die gemeinsame Datei latex-vorspann.tex mit nicht gesetztem Schalter.

\newif\ifkorrekturansicht
\korrekturansichtfalse

\input{../tex-inputs/latex-vorspann}


         
         \renewcommand{\erwaehntePersonen}{Personen: André Antoine, Hermann Bahr, Stephan Epstein, Friedrich Kovacs, Émile Lutz, Julius Schnitzler}
         \renewcommand{\erwaehnteOrte}{Orte: Wien, rue de l’Assomption}
         \renewcommand{\erwaehnteWerke}{Werke: Der Apostel. Schauspiel in drei Aufzügen, Der Meister. Komödie in drei Akten, Der grüne Kakadu. Groteske in einem Akt}
               \section[Arthur Schnitzler an Hermann Bahr, 24. 6. 1903]{ Arthur Schnitzler an Hermann Bahr, 24. 6. 1903}\nopagebreak\mylabel{v}\rehead{ }\begin{ledgroupsized}[t]{13cm}\normalsize\beginnumbering \toendnotes[C]{\smallbreak\pagebreak[2]} \Standort{TMW, HS AM 23356 Ba.}
\physDesc{Brief, 1 Blatt, 3 Seiten
\newline{}Handschrift: Bleistift, deutsche Kurrent\newline{}Ordnung: Lochung }\buchAbdrucke{\weitereDrucke{1) \emph{24. 6. 1903.} In: Arthur Schnitzler: \emph{The Letters of Arthur Schnitzler to Hermann Bahr}. Edited, annotated, and with an introduction, by Donald G.
                        Daviau. Chapel Hill: \emph{The University of North Carolina Press} 1978, S. 79 (University of North Carolina studies in the Germanic languages
                        and literatures, 89).} \weitereDrucke{2) Hermann Bahr, Arthur Schnitzler: \emph{Briefwechsel, Aufzeichnungen, Dokumente (1891–1931)}. Hg. Kurt Ifkovits und Martin Anton Müller. Göttingen: \emph{Wallstein} 2018, S. 267.} }\toendnotes[C]{\smallbreak}\pstart
           \raggedleft{}{\pb}24. \damage{6}. 903.\pend
           \pstart{}lieber Hermann,\pend\pstart
           Herr Dr \textsc{Stephan Epstein}\pwindex{Epstein, Stephan 12.11.1866 – 1941@\textsc{Epstein, Stephan} (12.11.1866 – 1941), \emph{Schriftsteller, Dramaturg, Übersetzer}|pw} (der mit Hrn Lutz\pwindex{Lutz, Emile 1868-04-08 – 1940-01-18@\textsc{Lutz, Émile} (1868-04-08 – 1940-01-18), \emph{Übersetzer, Dichter}|pw} zuſammen Kakadu\pwindex{Schnitzler, Arthur 15.05.1862 – 21.10.1931@\textsc{Schnitzler, Arthur} (15.05.1862 – 21.10.1931), \emph{Schriftsteller, Mediziner}!gruene Kakadu. Groteske in einem Akt1. 3. 1899@\strich\emph{Der grüne Kakadu. Groteske in einem Akt} {[}1. 3. 1899{]}|pw} ins franzöſiſche überſetzt hat (für \textsc{Antoine\pwindex{Antoine, Andre 1858-01-31 – 1943-10-23@\textsc{Antoine, André} (1858-01-31 – 1943-10-23), \emph{Theaterleiter, Schauspieler}|pw}})) \textsc{Paris, 78 rue de l’Assomption\oindex{rue de l Assomption@\textbf{rue de l’Assomption}|pw}}, bittet mich dich zu fragen, ob du ſein Erſuchen betreffs Überſetzungsrechten
               des \textsc{Apostel\pwindex{Bahr, Hermann 19.07.1863 – 15.01.1934@\textsc{Bahr, Hermann} (19.07.1863 – 15.01.1934), \emph{Schriftsteller, Kritiker}!Apostel. Schauspiel in drei Aufzuegen1901@\strich\emph{Der Apostel. Schauspiel in drei Aufzügen} {[}1901{]}|pw}} ins franz. {\pb}\label{K_L01299_1v}\edtext{erhalten haſt}{\lemma{\textnormal{\emph{erhalten haſt}}}\Cendnote{\textnormal{nicht
                  überliefert}}}\label{K_L01299_1h}. Vielleicht biſt du ſo freundlich ihm
               direct zu antworten? –\pend
           \pstart
           – Mein Bruder\pwindex{Schnitzler, Julius 13.07.1865 – 29.06.1939@\textsc{Schnitzler, Julius} (13.07.1865 – 29.06.1939), \emph{Chirurg}|pwv} nennt mir als
               einen \introOben{}Arzt, der\introOben{} in \introOben{}der\introOben{} neulich von
               uns \label{K_L01299_2v}\edtext{beſprochenen Art ſeine Patienten zu
                  unterſuchen}{\lemma{\textnormal{\emph{beſprochenen … unterſuchen}}}\Cendnote{\textnormal{Vermutlich in Zusammenhang
                  mit der Abfassung von \emph{Der Meister}\pwindex{Bahr, Hermann 19.07.1863 – 15.01.1934@\textsc{Bahr, Hermann} (19.07.1863 – 15.01.1934), \emph{Schriftsteller, Kritiker}!Meister. Komoedie in drei Akten1903@\strich\emph{Der Meister. Komödie in drei Akten} {[}1903{]}|pwk} zu sehen,
                  dessen Hauptfigur ein Alternativmediziner ist.}}}\label{K_L01299_2h} pflegt: Dr \textsc{Kovacs}\pwindex{Kovacs, Friedrich 1861-01-16 – 1931-02-11@\textsc{Kovacs, Friedrich} (1861-01-16 – 1931-02-11), \emph{Mediziner}|pw}. (Ich glaube er kennt ihn nicht persönlich.) –\pend
           \pstart
           {\pb}Herzlichen Gruſs.{\\[\baselineskip]}Dein{\\[\baselineskip]}\spacefill\mbox{A.}\pend
           \leftskip=0em{}
         
         \endnumbering\mylabel{h}\end{ledgroupsized}  \newcommand{\dateiname}{L01299}\newcommand{\titel}{Arthur Schnitzler an Hermann Bahr, 24. 6. 1903}\newcommand{\editorInnen}{ Kurt Ifkovits,  Martin Anton Müller}%% latex-leseansicht-abspann.tex
%% Abspann für die Leseansicht.
%% Der Schalter \ifkorrekturansicht ist bereits durch den Vorspann gesetzt.

%% latex-abspann.tex
%% Gemeinsamer Abspann für Korrekturansicht und Leseansicht.
%% Setzt den Schalter \ifkorrekturansicht voraus (gesetzt in den
%% einbindenden Dateien latex-korrekturansicht-abspann.tex bzw.
%% latex-leseansicht-abspann.tex).
%% ---------------------------------------------------------------

\normalsize

% Das esempio-Environment wird nur in der Leseansicht benötigt
\ifkorrekturansicht\else
\newenvironment{esempio}[3]%
{
    \vspace{1.5ex}
    \rlap{\underline{#1}}
    \par
    \setlength{\parindent}{0cm}
    \nopagebreak
    \leftskip=#2cm
    \rightskip=#3cm
}
{
    \par
}
\fi

\doendnotes{C}
\bigskip
\vfill

\clearpage

\footnotesize

\ifkorrekturansicht
  \lohead{\textsc{register}}
\fi

% theindex-Environment neu definieren ohne reledmac
\makeatletter
\renewenvironment{theindex}{%
  \ifkorrekturansicht
    \section*{\indexname}%
  \else
    \subsubsection*{Index der erwähnten Entitäten}%
  \fi
  \setlength{\parindent}{0pt}%
  \setlength{\parskip}{0pt plus 0.3pt}%
  \let\item\@idxitem
}{%
  \ifkorrekturansicht\clearpage\fi
}
\makeatother

\IfFileExists{\jobname-pw.ind}{\input{\jobname-pw.ind}}{}

% Quellenangabe nur in der Leseansicht
\ifkorrekturansicht\else
% Fallback-Definitionen, falls die .tex-Datei \titel etc. nicht gesetzt hat
\providecommand{\titel}{}
\providecommand{\editorInnen}{}
\providecommand{\dateiname}{\jobname}

\vspace{3cm}

\vfill

\footnotesize
\textsc{Quelle}: \titel. Herausgegeben von {\editorInnen}. In: \emph{Arthur Schnitzler: Briefwechsel mit Autorinnen und Autoren}.
 Digitale Edition, https://schnitzler-briefe.acdh.oeaw.ac.at/{\dateiname}.html (Stand \today)
\fi

\end{document}


      