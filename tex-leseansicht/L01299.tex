%% latex-leseansicht-vorspann.tex
%% Vorspann für die Leseansicht.
%% Lädt die gemeinsame Datei latex-vorspann.tex mit nicht gesetztem Schalter.

\newif\ifkorrekturansicht
\korrekturansichtfalse

\input{../tex-inputs/latex-vorspann}


\section[Arthur Schnitzler an Hermann Bahr, 24. 6. 1903]{L01299 Arthur Schnitzler an Hermann Bahr, 24. 6. 1903}
\nopagebreak\mylabel{L01299v}
\rehead{ }\normalsize\beginnumbering\briefempfaengerindex{Bahr, Hermann@\textsc{Bahr, Hermann}!zzzSchnitzler, Arthur@\emph{von Arthur Schnitzler}!1903-06-241@{24. 6. 1903}|(be}
\toendnotes[C]{\smallbreak\pagebreak[2]}
\correspDesc{Versand  durch Arthur Schnitzler am 24. 6. 1903 in Wien
\newline{}Erhalt  durch Hermann Bahr im Zeitraum [24. 6. 1903
                  – 28. 6. 1903?] in Wien}\toendnotes[C]{\smallbreak}
\Standort{TMW, HS AM 23356 Ba.}
\physDesc{Brief, 1 Blatt, 3 Seiten, 514 Zeichen
\newline{}Handschrift: Bleistift, deutsche Kurrent
\newline{}Ordnung: Lochung }
\buchAbdrucke{\weitereDrucke{1) \emph{24. 6. 1903.} In: Arthur Schnitzler: \emph{The Letters of Arthur Schnitzler to Hermann Bahr}. Edited, annotated, and with an introduction, by Donald G. Daviau. Chapel Hill: \emph{The University of North Carolina Press} 1978, S. 79 (University of North Carolina studies in the Germanic languages
                        and literatures, 89).} \weitereDrucke{2) Hermann Bahr, Arthur Schnitzler: \emph{Briefwechsel, Aufzeichnungen, Dokumente (1891–1931)}. Herausgegeben von Kurt Ifkovits und Martin Anton Müller. Göttingen: \emph{Wallstein} 2018, S. 267.} }\toendnotes[C]{\smallbreak}
\pstart
           \raggedleft{}{\pb}24. \damage{6}. 903.\pend
           
\pstart{}lieber Hermann,\pend\vspace{0.5em}
\pstart
           Herr Dr \textsc{Stephan Epstein}\pwindex{Epstein, Stephan 12.\,11.\,1866 Warschau – 1941 Paris@\textsc{Epstein, Stephan} (12.\,11.\,1866 Warschau – 1941 Paris), \emph{Schriftsteller, Dramaturg, Übersetzer}|pw} (der mit Hrn Lutz\pwindex{Lutz, Émile 8.\,4.\,1868 Saint-Étienne-du-Rouvray – 18.\,1.\,1940 Paris@\textsc{Lutz, Émile} (8.\,4.\,1868 Saint-Étienne-du-Rouvray – 18.\,1.\,1940 Paris), \emph{Übersetzer, Dichter}|pw} zuſammen Kakadu\pwindex{Schnitzler, Arthur 15.\,5.\,1862 Wien – 21.\,10.\,1931 ebd.@\textsc{Schnitzler, Arthur} (15.\,5.\,1862 Wien – 21.\,10.\,1931 ebd.), \emph{Schriftsteller, Mediziner}!grüne Kakadu. Groteske in einem Akt@\strich\emph{Der grüne Kakadu. Groteske in einem Akt}|pw} ins franzöſiſche überſetzt\pwindex{Schnitzler, Arthur 15.\,5.\,1862 Wien – 21.\,10.\,1931 ebd.@\textsc{Schnitzler, Arthur} (15.\,5.\,1862 Wien – 21.\,10.\,1931 ebd.), \emph{Schriftsteller, Mediziner}!Au Perroquet Vert@\strich\emph{Au Perroquet Vert}|pwv} hat (für \textsc{Antoine\pwindex{Antoine, André 31.\,1.\,1858 Limoges – 23.\,10.\,1943 Le Pouliguen@\textsc{Antoine, André} (31.\,1.\,1858 Limoges – 23.\,10.\,1943 Le Pouliguen), \emph{Theaterleiter, Schauspieler}|pw}})) \textsc{Paris, 78 rue de l’Assomption\oindex{rue de l’Assomption@\textbf{rue de l’Assomption}, \emph{Straße}|pw}}, bittet mich dich zu fragen, ob du{ }ſein Erſuchen betreffs Überſetzungsrechten
               des \textsc{Apostel\pwindex{Bahr, Hermann 19.\,7.\,1863 Linz – 15.\,1.\,1934 München@\textsc{Bahr, Hermann} (19.\,7.\,1863 Linz – 15.\,1.\,1934 München), \emph{Schriftsteller, Kritiker}!Apostel. Schauspiel in drei Aufzügen@\strich\emph{Der Apostel. Schauspiel in drei Aufzügen}|pw}} ins franz. {\pb}\label{K_L01299-1v}\edtext{erhalten haſt}{\lemma{\textnormal{\emph{erhalten hast}}}\Cendnote{\textnormal{nicht überliefert}}}\label{K_L01299-1}. Vielleicht biſt
               du{ }ſo freundlich ihm direct zu antworten? –\pend
           
\pstart
           – Mein Bruder\pwindex{Schnitzler, Julius 13.\,7.\,1865 Wien – 29.\,6.\,1939 ebd.@\textsc{Schnitzler, Julius} (13.\,7.\,1865 Wien – 29.\,6.\,1939 ebd.), \emph{Chirurg}|pwv} nennt mir als
               einen \introOben{}Arzt, der\introOben{} in \introOben{}der\introOben{} neulich von
               uns \label{K_L01299-2v}\edtext{beſprochenen Art{ }ſeine Patienten
               zu unterſuchen}{\lemma{\textnormal{\emph{besprochenen … untersuchen}}}\Cendnote{\textnormal{Vermutlich in Zusammenhang
                  mit der Abfassung von \emph{Der Meister}\pwindex{Bahr, Hermann 19.\,7.\,1863 Linz – 15.\,1.\,1934 München@\textsc{Bahr, Hermann} (19.\,7.\,1863 Linz – 15.\,1.\,1934 München), \emph{Schriftsteller, Kritiker}!Meister. Komödie in drei Akten@\strich\emph{Der Meister. Komödie in drei Akten}|pwk}\pwindex{Bahr, Hermann 19.\,7.\,1863 Linz – 15.\,1.\,1934 München@\textsc{Bahr, Hermann} (19.\,7.\,1863 Linz – 15.\,1.\,1934 München), \emph{Schriftsteller, Kritiker}!Meister. Komödie in drei Akten@\strich\emph{Der Meister. Komödie in drei Akten}|pwk} zu sehen,
                  dessen Hauptfigur ein Alternativmediziner ist.}}}\label{K_L01299-2} pflegt: Dr \textsc{Kovacs}\pwindex{Kovacs, Friedrich 16.\,1.\,1861 Wien – 11.\,2.\,1931 ebd.@\textsc{Kovacs, Friedrich} (16.\,1.\,1861 Wien – 11.\,2.\,1931 ebd.), \emph{Mediziner, Internist, Hochschullehrer}|pw}. (Ich glaube er kennt ihn nicht persönlich.) –\pend
           
\pstart
           {\pb}Herzlichen Gruſs.{\\[\baselineskip]}Dein{\\[\baselineskip]}\spacefill\mbox{A.}\pend
           \leftskip=0em{}\selectlanguage{ngerman}\endnumbering\briefempfaengerindex{Bahr, Hermann@\textsc{Bahr, Hermann}!zzzSchnitzler, Arthur@\emph{von Arthur Schnitzler}!1903-06-241@{24. 6. 1903}|)be}\mylabel{L01299h}  \newcommand{\dateiname}{L01299}\newcommand{\titel}{Arthur Schnitzler an Hermann Bahr, 24. 6. 1903}\newcommand{\editorInnen}{Herausgegeben von Martin Anton Müller}%% latex-leseansicht-abspann.tex
%% Abspann für die Leseansicht.
%% Der Schalter \ifkorrekturansicht ist bereits durch den Vorspann gesetzt.

%% latex-abspann.tex
%% Gemeinsamer Abspann für Korrekturansicht und Leseansicht.
%% Setzt den Schalter \ifkorrekturansicht voraus (gesetzt in den
%% einbindenden Dateien latex-korrekturansicht-abspann.tex bzw.
%% latex-leseansicht-abspann.tex).
%% ---------------------------------------------------------------

\normalsize

% Das esempio-Environment wird nur in der Leseansicht benötigt
\ifkorrekturansicht\else
\newenvironment{esempio}[3]%
{
    \vspace{1.5ex}
    \rlap{\underline{#1}}
    \par
    \setlength{\parindent}{0cm}
    \nopagebreak
    \leftskip=#2cm
    \rightskip=#3cm
}
{
    \par
}
\fi

\doendnotes{C}
\bigskip
\vfill

\clearpage

\footnotesize

\ifkorrekturansicht
  \lohead{\textsc{register}}
\fi

% theindex-Environment neu definieren ohne reledmac
\makeatletter
\renewenvironment{theindex}{%
  \ifkorrekturansicht
    \section*{\indexname}%
  \else
    \subsubsection*{Index der erwähnten Entitäten}%
  \fi
  \setlength{\parindent}{0pt}%
  \setlength{\parskip}{0pt plus 0.3pt}%
  \let\item\@idxitem
}{%
  \ifkorrekturansicht\clearpage\fi
}
\makeatother

\IfFileExists{\jobname-pw.ind}{\input{\jobname-pw.ind}}{}

% Quellenangabe nur in der Leseansicht
\ifkorrekturansicht\else
% Fallback-Definitionen, falls die .tex-Datei \titel etc. nicht gesetzt hat
\providecommand{\titel}{}
\providecommand{\editorInnen}{}
\providecommand{\dateiname}{\jobname}

\vspace{3cm}

\vfill

\footnotesize
\textsc{Quelle}: \titel. Herausgegeben von {\editorInnen}. In: \emph{Arthur Schnitzler: Briefwechsel mit Autorinnen und Autoren}.
 Digitale Edition, https://schnitzler-briefe.acdh.oeaw.ac.at/{\dateiname}.html (Stand \today)
\fi

\end{document}


