%% latex-korrekturansicht-vorspann.tex
%% Vorspann für die Korrekturansicht.
%% Lädt die gemeinsame Datei latex-vorspann.tex mit gesetztem Schalter.

\newif\ifkorrekturansicht
\korrekturansichttrue

\input{../tex-inputs/latex-vorspann}


\section[Arthur Schnitzler an Hermann Bahr, 24. 6. 1903]{L01299 Arthur Schnitzler an Hermann Bahr, 24. 6. 1903}
\nopagebreak\mylabel{L01299v}
\rehead{ }\normalsize\beginnumbering\briefempfaengerindex{Bahr, Hermann@\textsc{Bahr, Hermann}!zzzSchnitzler, Arthur@\emph{von Arthur Schnitzler}!1903-06-241@{24. 6. 1903}|(be}
\toendnotes[C]{\smallbreak\pagebreak[2]}\Standort{TMW, HS AM 23356 Ba.}
\physDesc{Brief, 1 Blatt, 3 Seiten, 514 Zeichen
\newline{}Handschrift: Bleistift, deutsche Kurrent
\newline{}Ordnung: Lochung }
\buchAbdrucke{\weitereDrucke{1) Arthur Schnitzler: \emph{The Letters of Arthur Schnitzler to Hermann Bahr}. Chapel Hill: \emph{The University of North Carolina Press} 1978, S. 79.} \weitereDrucke{2) Hermann Bahr, Arthur Schnitzler: \emph{Briefwechsel, Aufzeichnungen, Dokumente (1891–1931)}. Göttingen: \emph{Wallstein} 2018, S. 267.} }\toendnotes[C]{\smallbreak}
\pstart
           \raggedleft{}{\pb}24. \damage{6}. 903.\pend
           
\pstart{}lieber Hermann,\pend\vspace{0.5em}
\pstart
           Herr Dr \textsc{Stephan Epstein}\pwindex{Epstein, Stephan 12.11.1866 – 1941@\textsc{Epstein, Stephan} (12.11.1866 – 1941), \emph{Schriftsteller/Schriftstellerin, Dramaturg/Dramaturgin, Übersetzer/Übersetzerin}|pw} (der mit Hrn Lutz\pwindex{Lutz, Emile 1868-04-08 – 1940-01-18@\textsc{Lutz, Émile} (1868-04-08 – 1940-01-18), \emph{Übersetzer/Übersetzerin, Dichter/Dichterin}|pw} zuſammen Kakadu\pwindex{gruene Kakadu. Groteske in einem Akt@\emph{Der grüne Kakadu. Groteske in einem Akt}|pw} ins franzöſiſche überſetzt\pwindex{Au Perroquet Vert@\emph{Au Perroquet Vert}|pwv} hat (für \textsc{Antoine\pwindex{Antoine, Andre 1858-01-31 – 1943-10-23@\textsc{Antoine, André} (1858-01-31 – 1943-10-23), \emph{Theaterleiter/Theaterleiterin, Schauspieler/Schauspielerin}|pw}})) \textsc{Paris, 78 rue de l’Assomption\oindex{rue de l Assomption@\textbf{rue de l’Assomption}, \emph{Straße (K.STR)}|pw}}, bittet mich dich zu fragen, ob du ſein Erſuchen betreffs Überſetzungsrechten
               des \textsc{Apostel\pwindex{Apostel. Schauspiel in drei Aufzuegen@\emph{Der Apostel. Schauspiel in drei Aufzügen}|pw}} ins franz. {\pb}\label{K_L01299-1v}\edtext{erhalten haſt}{\lemma{\textnormal{\emph{erhalten haſt}}}\Cendnote{\textnormal{nicht überliefert}}}\label{K_L01299-1}. Vielleicht biſt
               du ſo freundlich ihm direct zu antworten? –\pend
           
\pstart
           – Mein Bruder\pwindex{Schnitzler, Julius 13.07.1865 – 29.06.1939@\textsc{Schnitzler, Julius} (13.07.1865 – 29.06.1939), \emph{Chirurg/Chirurgin}|pwv} nennt mir als
               einen \introOben{}Arzt, der\introOben{} in \introOben{}der\introOben{} neulich von
               uns \label{K_L01299-2v}\edtext{beſprochenen Art ſeine Patienten
               zu unterſuchen}{\lemma{\textnormal{\emph{beſprochenen … unterſuchen}}}\Cendnote{\textnormal{Vermutlich in Zusammenhang
                  mit der Abfassung von \emph{Der Meister}\pwindex{Meister. Komoedie in drei Akten@\emph{Der Meister. Komödie in drei Akten}|pwk} zu sehen,
                  dessen Hauptfigur ein Alternativmediziner ist.}}}\label{K_L01299-2} pflegt: Dr \textsc{Kovacs}\pwindex{Kovacs, Friedrich 1861-01-16 – 1931-02-11@\textsc{Kovacs, Friedrich} (1861-01-16 – 1931-02-11), \emph{Mediziner/Medizinerin, Internist/Internistin, Hochschullehrer/Hochschullehrerin}|pw}. (Ich glaube er kennt ihn nicht persönlich.) –\pend
           
\pstart
           {\pb}Herzlichen Gruſs.{\\[\baselineskip]}Dein{\\[\baselineskip]}\spacefill\mbox{A.}\pend
           \leftskip=0em{}\selectlanguage{ngerman}\endnumbering\briefempfaengerindex{Bahr, Hermann@\textsc{Bahr, Hermann}!zzzSchnitzler, Arthur@\emph{von Arthur Schnitzler}!1903-06-241@{24. 6. 1903}|)be}\mylabel{L01299h}  \normalsize

\doendnotes{C}
\bigskip
\vfill

\clearpage

\footnotesize

\lohead{\textsc{register}}

% Definiere theindex-Environment komplett neu ohne reledmac
\makeatletter
\renewenvironment{theindex}{%
  \section*{\indexname}%
  \setlength{\parindent}{0pt}%
  \setlength{\parskip}{0pt plus 0.3pt}%
  \let\item\@idxitem
}{%
  \clearpage
}
\makeatother

\IfFileExists{\jobname-pw.ind}{\input{\jobname-pw.ind}}{}

\end{document}

      