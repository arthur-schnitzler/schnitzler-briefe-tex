%% latex-leseansicht-vorspann.tex
%% Vorspann für die Leseansicht.
%% Lädt die gemeinsame Datei latex-vorspann.tex mit nicht gesetztem Schalter.

\newif\ifkorrekturansicht
\korrekturansichtfalse

\input{../tex-inputs/latex-vorspann}


\section[Hugo von Hofmannsthal an Arthur Schnitzler, {[}1. 12. 1892{]}]{L00141 Hugo von Hofmannsthal an Arthur Schnitzler, {[}1. 12. 1892{]}}
\nopagebreak\mylabel{L00141v}
\rehead{ }\normalsize\beginnumbering\briefempfaengerindex{Schnitzler, Arthur@\textsc{Schnitzler, Arthur}!zzzHofmannsthal, Hugo von@\emph{von Hugo von Hofmannsthal}!1892-12-012@{{[}1. 12. 1892{]}}|(be}
\toendnotes[C]{\smallbreak\pagebreak[2]}
\correspDesc{Versand  durch Hugo von Hofmannsthal am [1. 12. 1892] in Wien
\newline{}Erhalt  durch Arthur Schnitzler im Zeitraum [1. 12. 1892
                  – 5. 12. 1892?] in Wien}\toendnotes[C]{\smallbreak}
\Standort{CUL, Schnitzler, B 43.}
\physDesc{Briefkarte, 223 Zeichen
\newline{}Handschrift: schwarze Tinte, deutsche Kurrent
\newline{}Schnitzler: mit Bleistift nummeriert: »34« und mit Jahreszahl
                                 versehen: »92« }
\buchAbdrucke{\weitereDrucke{Hugo von Hofmannsthal, Arthur Schnitzler: \emph{Briefwechsel}. Herausgegeben von Therese Nickl und Heinrich Schnitzler. Frankfurt am Main: \emph{S. Fischer} 1964, S. 32.} }\toendnotes[C]{\smallbreak}
\pstart
           \textcolor{gray}{\textbf{\label{T_L00141-1v}\edtext{AvH}{\lemma{\textnormal{\emph{AvH}}}\Cendnote{\textnormal{Monogramm der Mutter Anna von Hofmannsthal\pwindex{Hofmannsthal, Anna von 27.\,1.\,1849 Wien – 22.\,3.\,1904 Sanatorium Fürth@\textsc{Hofmannsthal, Anna von} (27.\,1.\,1849 Wien – 22.\,3.\,1904 Sanatorium Fürth)|pwk} mit Krone in Golddruck}}}\label{T_L00141-1}}}\pend
           
\pstart
           \raggedleft{}{\pb}Donnerstag.\pend
           
\pstart{}Lieber Arthur.\pend\vspace{0.5em}
\pstart
           Bitte alſo{ }ſchicken Sie die Photographie dem Devrient\pwindex{Devrient, Max 12.\,12.\,1857 Hannover – 13.\,6.\,1929 Chur@\textsc{Devrient, Max} (12.\,12.\,1857 Hannover – 13.\,6.\,1929 Chur), \emph{Regisseur, Schauspieler}|pw} mit der Bitte um nicht zu langſame Rückſendung in unterſchriebenem
               Zustand für irgend eine Verehrerin. Auf Wiederſehen Sonntag! Besten
               dankend\pend
           \pstart \spacefill\mbox{Loris}\pend{}\selectlanguage{ngerman}\endnumbering\briefempfaengerindex{Schnitzler, Arthur@\textsc{Schnitzler, Arthur}!zzzHofmannsthal, Hugo von@\emph{von Hugo von Hofmannsthal}!1892-12-012@{{[}1. 12. 1892{]}}|)be}\mylabel{L00141h}  \newcommand{\dateiname}{L00141}\newcommand{\titel}{Hugo von Hofmannsthal an Arthur Schnitzler, [1. 12. 1892]}\newcommand{\editorInnen}{Martin Anton Müller und Gerd-Hermann Susen}%% latex-leseansicht-abspann.tex
%% Abspann für die Leseansicht.
%% Der Schalter \ifkorrekturansicht ist bereits durch den Vorspann gesetzt.

%% latex-abspann.tex
%% Gemeinsamer Abspann für Korrekturansicht und Leseansicht.
%% Setzt den Schalter \ifkorrekturansicht voraus (gesetzt in den
%% einbindenden Dateien latex-korrekturansicht-abspann.tex bzw.
%% latex-leseansicht-abspann.tex).
%% ---------------------------------------------------------------

\normalsize

% Das esempio-Environment wird nur in der Leseansicht benötigt
\ifkorrekturansicht\else
\newenvironment{esempio}[3]%
{
    \vspace{1.5ex}
    \rlap{\underline{#1}}
    \par
    \setlength{\parindent}{0cm}
    \nopagebreak
    \leftskip=#2cm
    \rightskip=#3cm
}
{
    \par
}
\fi

\doendnotes{C}
\bigskip
\vfill

\clearpage

\footnotesize

\ifkorrekturansicht
  \lohead{\textsc{register}}
\fi

% theindex-Environment neu definieren ohne reledmac
\makeatletter
\renewenvironment{theindex}{%
  \ifkorrekturansicht
    \section*{\indexname}%
  \else
    \subsubsection*{Index der erwähnten Entitäten}%
  \fi
  \setlength{\parindent}{0pt}%
  \setlength{\parskip}{0pt plus 0.3pt}%
  \let\item\@idxitem
}{%
  \ifkorrekturansicht\clearpage\fi
}
\makeatother

\IfFileExists{\jobname-pw.ind}{\input{\jobname-pw.ind}}{}

% Quellenangabe nur in der Leseansicht
\ifkorrekturansicht\else
% Fallback-Definitionen, falls die .tex-Datei \titel etc. nicht gesetzt hat
\providecommand{\titel}{}
\providecommand{\editorInnen}{}
\providecommand{\dateiname}{\jobname}

\vspace{3cm}

\vfill

\footnotesize
\textsc{Quelle}: \titel. Herausgegeben von {\editorInnen}. In: \emph{Arthur Schnitzler: Briefwechsel mit Autorinnen und Autoren}.
 Digitale Edition, https://schnitzler-briefe.acdh.oeaw.ac.at/{\dateiname}.html (Stand \today)
\fi

\end{document}


