%% latex-korrekturansicht-vorspann.tex
%% Vorspann für die Korrekturansicht.
%% Lädt die gemeinsame Datei latex-vorspann.tex mit gesetztem Schalter.

\newif\ifkorrekturansicht
\korrekturansichttrue

\input{../tex-inputs/latex-vorspann}


\section[Richard Beer-Hofmann an Arthur Schnitzler, {[}vor dem 22. 6. 1893?{]}]{L00223 Richard Beer-Hofmann an Arthur Schnitzler, {[}vor dem
               22. 6. 1893?{]}}
\nopagebreak\mylabel{L00223v}
\rehead{ }\normalsize\beginnumbering\briefempfaengerindex{Schnitzler, Arthur@\textsc{Schnitzler, Arthur}!zzzBeer-Hofmann, Richard@\emph{von Richard Beer-Hofmann}!1893-06-201@{{[}vor dem 22. 6. 1893?{]}}|(be}
\toendnotes[C]{\smallbreak\pagebreak[2]}\Standort{CUL, Schnitzler, B 8.}
\physDesc{Briefkarte, 566 Zeichen
\newline{}Handschrift: Bleistift, lateinische Kurrent
\newline{}Schnitzler: mit Bleistift nummeriert: »16« }
\buchAbdrucke{\weitereDrucke{Arthur Schnitzler, Richard Beer-Hofmann: \emph{Briefwechsel 1891–1931}. Wien, Zürich: \emph{Europaverlag} 1992, S. 44.} }\toendnotes[C]{\smallbreak}
\pstart
           \textcolor{gray}{\textbf{\label{T_L00223-1v}\edtext{RB}{\lemma{\textnormal{\emph{RB}}}\Cendnote{\textnormal{Monogramm in Golddruck}}}\label{T_L00223-1}}}\pend
           
\pstart{}{\pb}Lieber Arthur!\pend\vspace{0.5em}
\pstart
           Wie ich aus den Theaterzetteln entnehme ist Jarno\pwindex{Jarno, Josef 24.08.1865 – 11.01.1932@\textsc{Jarno, Josef} (24.08.1865 – 11.01.1932), \emph{Theaterleiter/Theaterleiterin, Schauspieler/Schauspielerin}|pw} hier a. G. und aber auch als Regisseur (also offenbar für die Saison).
               Schreiben Sie ihm also \uline{er} möge \uline{mich} aufsuchen (motiviren Sie das irgendwie, da es mir nicht passt zu
               ihm zu gehen) sagen {\pb}Sie was von
               Bewunderung für ihn; in Wien\oindex{Wien@\textbf{Wien}, \emph{A.ADM2}|pw} gesehen etc, – ich
               Ihre Intentionen kennen u. s. w. Vielleicht geht es für \uline{Juli} einen Abend mit Ihren Sachen zu geben z. B.\pend
           \leftskip=3em{}
\pstart
           \noindent{}Episode\pwindex{Episode@\emph{Episode}|pw}\pend
           \leftskip=0em{}\leftskip=3em{}
\pstart
           Abschiedssouper\pwindex{Abschiedssouper@\emph{Abschiedssouper}|pw}\pend
           \leftskip=0em{}\leftskip=3em{}
\pstart
           Hochzeitsmorgen\pwindex{Anatols Hochzeitsmorgen@\emph{Anatols Hochzeitsmorgen}|pw}\pend
           \leftskip=0em{}
\pstart
           Ko{\geminationm}en Sie bald, Grüße an alle.\pend
           
\pstart
           Herzlichst{\\[\baselineskip]}\spacefill\mbox{Richard}\pend
           \leftskip=0em{}
\pstart
           \noindent{}\label{T_L00223-2v}\edtext{Ich bin i{\geminationm}er gegen 2 Uhr zu Hause (wegen Jarno\pwindex{Jarno, Josef 24.08.1865 – 11.01.1932@\textsc{Jarno, Josef} (24.08.1865 – 11.01.1932), \emph{Theaterleiter/Theaterleiterin, Schauspieler/Schauspielerin}|pw})}{\lemma{\textnormal{\emph{Ich … Jarno)}}}\Cendnote{\textnormal{zwischen den Zeilen}}}\label{T_L00223-2}\pend
           
\pstart
           \label{K_L00223-1v}\edtext{Tartaglia\pwindex{Felix, Benedikt 28.09.1851 – 02.03.1912@\textsc{Felix, Benedikt} (28.09.1851 – 02.03.1912), \emph{Schauspieler/Schauspielerin, Sänger/Sängerin, Bassist/Bassistin}|pwuv}}{\lemma{\textnormal{\emph{Tartaglia}}}\Cendnote{\textnormal{Womöglich ist Benedikt Felix\pwindex{Felix, Benedikt 28.09.1851 – 02.03.1912@\textsc{Felix, Benedikt} (28.09.1851 – 02.03.1912), \emph{Schauspieler/Schauspielerin, Sänger/Sängerin, Bassist/Bassistin}|pwk} gemeint, der in der abgelaufenen Theatersaison in
                        \emph{Signor Formica}\pwindex{Signor Formica. Komische Oper in drei Akten@\emph{Signor Formica. Komische Oper in drei Akten}|pwk} in der Rolle des
                     Tartaglia aufgetreten war.}}}\label{K_L00223-1} schrieb ich gestern.\pend
           \selectlanguage{ngerman}\endnumbering\briefempfaengerindex{Schnitzler, Arthur@\textsc{Schnitzler, Arthur}!zzzBeer-Hofmann, Richard@\emph{von Richard Beer-Hofmann}!1893-06-201@{{[}vor dem 22. 6. 1893?{]}}|)be}\mylabel{L00223h}  \normalsize

\doendnotes{C}
\bigskip
\vfill

\clearpage

\footnotesize

\lohead{\textsc{register}}

% Definiere theindex-Environment komplett neu ohne reledmac
\makeatletter
\renewenvironment{theindex}{%
  \section*{\indexname}%
  \setlength{\parindent}{0pt}%
  \setlength{\parskip}{0pt plus 0.3pt}%
  \let\item\@idxitem
}{%
  \clearpage
}
\makeatother

\IfFileExists{\jobname-pw.ind}{\input{\jobname-pw.ind}}{}

\end{document}

      