%% latex-leseansicht-vorspann.tex
%% Vorspann für die Leseansicht.
%% Lädt die gemeinsame Datei latex-vorspann.tex mit nicht gesetztem Schalter.

\newif\ifkorrekturansicht
\korrekturansichtfalse

\input{../tex-inputs/latex-vorspann}


\section[Richard Beer-Hofmann an Arthur Schnitzler, {{[}}vor dem 22. 6. 1893?{{]}}]{L00223 Richard Beer-Hofmann an Arthur Schnitzler, {[}vor dem 22. 6. 1893?{]}}
\nopagebreak\mylabel{L00223v}
\rehead{ }\normalsize\beginnumbering\briefempfaengerindex{Schnitzler, Arthur@\textsc{Schnitzler, Arthur}!zzzBeer-Hofmann, Richard@\emph{von Richard Beer-Hofmann}!1893-06-201@{{[}vor dem 22. 6. 1893?{]}}|(be}
\toendnotes[C]{\smallbreak\pagebreak[2]}
\correspDesc{Versand  durch Richard Beer-Hofmann am [vor dem 22. 6. 1893?] in Bad Ischl
\newline{}Erhalt  durch Arthur Schnitzler im Zeitraum [21. 6. 1893
                  – 25. 6. 1893?] in Wien}\toendnotes[C]{\smallbreak}
\Standort{CUL, Schnitzler, B 8.}
\physDesc{Briefkarte, 566 Zeichen
\newline{}Handschrift: Bleistift, lateinische Kurrent
\newline{}Schnitzler: mit Bleistift nummeriert: »16« }
\buchAbdrucke{\weitereDrucke{Arthur Schnitzler, Richard Beer-Hofmann: \emph{Briefwechsel 1891–1931}. Herausgegeben von Konstanze Fliedl. Wien, Zürich: \emph{Europaverlag} 1992, S. 44.} }\toendnotes[C]{\smallbreak}
\pstart
           \textcolor{gray}{\textbf{\label{T_L00223-1v}\edtext{RB}{\lemma{\textnormal{\emph{RB}}}\Cendnote{\textnormal{Monogramm in Golddruck}}}\label{T_L00223-1}}}\pend
           
\pstart{}{\pb}Lieber Arthur!\pend\vspace{0.5em}
\pstart
           Wie ich aus den Theaterzetteln entnehme ist Jarno\pwindex{Jarno, Josef 24.\,8.\,1865 Budapest – 11.\,1.\,1932 Wien@\textsc{Jarno, Josef} (24.\,8.\,1865 Budapest – 11.\,1.\,1932 Wien), \emph{Theaterleiter, Schauspieler}|pw} hier a. G. und aber auch als Regisseur (also offenbar für die Saison).
               Schreiben Sie ihm also \uline{er} möge \uline{mich} aufsuchen (motiviren Sie das irgendwie, da es mir nicht passt zu
               ihm zu gehen) sagen {\pb}Sie was von
               Bewunderung für ihn; in Wien\oindex{Wien@\textbf{Wien}, \emph{Verwaltungsgebiet}|pw} gesehen etc, – ich
               Ihre Intentionen kennen u. s. w. Vielleicht geht es für \uline{Juli} einen Abend mit Ihren Sachen zu geben z. B.\pend
           \leftskip=3em{}
\pstart
           \noindent{}Episode\pwindex{Schnitzler, Arthur 15.\,5.\,1862 Wien – 21.\,10.\,1931 ebd.@\textsc{Schnitzler, Arthur} (15.\,5.\,1862 Wien – 21.\,10.\,1931 ebd.), \emph{Schriftsteller, Mediziner}!Episode@\strich\emph{Episode}|pw}\pend
           \leftskip=0em{}\leftskip=3em{}
\pstart
           Abschiedssouper\pwindex{Schnitzler, Arthur 15.\,5.\,1862 Wien – 21.\,10.\,1931 ebd.@\textsc{Schnitzler, Arthur} (15.\,5.\,1862 Wien – 21.\,10.\,1931 ebd.), \emph{Schriftsteller, Mediziner}!Abschiedssouper@\strich\emph{Abschiedssouper}|pw}\pend
           \leftskip=0em{}\leftskip=3em{}
\pstart
           Hochzeitsmorgen\pwindex{Schnitzler, Arthur 15.\,5.\,1862 Wien – 21.\,10.\,1931 ebd.@\textsc{Schnitzler, Arthur} (15.\,5.\,1862 Wien – 21.\,10.\,1931 ebd.), \emph{Schriftsteller, Mediziner}!Anatols Hochzeitsmorgen@\strich\emph{Anatols Hochzeitsmorgen}|pw}\pend
           \leftskip=0em{}
\pstart
           Ko{\geminationm}en Sie bald, Grüße an alle.\pend
           
\pstart
           Herzlichst{\\[\baselineskip]}\spacefill\mbox{Richard}\pend
           \leftskip=0em{}
\pstart
           \noindent{}\label{T_L00223-2v}\edtext{Ich bin i{\geminationm}er gegen 2 Uhr zu Hause (wegen Jarno\pwindex{Jarno, Josef 24.\,8.\,1865 Budapest – 11.\,1.\,1932 Wien@\textsc{Jarno, Josef} (24.\,8.\,1865 Budapest – 11.\,1.\,1932 Wien), \emph{Theaterleiter, Schauspieler}|pw})}{\lemma{\textnormal{\emph{Ich … Jarno)}}}\Cendnote{\textnormal{zwischen den Zeilen}}}\label{T_L00223-2}\pend
           
\pstart
           \label{K_L00223-1v}\edtext{Tartaglia\pwindex{Felix, Benedikt 28.\,9.\,1851 Budapest – 2.\,3.\,1912 Wien@\textsc{Felix, Benedikt} (28.\,9.\,1851 Budapest – 2.\,3.\,1912 Wien), \emph{Schauspieler, Sänger, Bassist}|pwuv}}{\lemma{\textnormal{\emph{Tartaglia}}}\Cendnote{\textnormal{Womöglich ist Benedikt Felix\pwindex{Felix, Benedikt 28.\,9.\,1851 Budapest – 2.\,3.\,1912 Wien@\textsc{Felix, Benedikt} (28.\,9.\,1851 Budapest – 2.\,3.\,1912 Wien), \emph{Schauspieler, Sänger, Bassist}|pwk} gemeint, der in der abgelaufenen Theatersaison in
                        \emph{Signor Formica}\pwindex{\textcolor{red}{\textsuperscript{XXXX indx1}}!Signor Formica. Komische Oper in drei Akten@\strich\emph{Signor Formica. Komische Oper in drei Akten}|pwk} in der Rolle des
                     Tartaglia aufgetreten war.}}}\label{K_L00223-1} schrieb ich gestern.\pend
           \selectlanguage{ngerman}\endnumbering\briefempfaengerindex{Schnitzler, Arthur@\textsc{Schnitzler, Arthur}!zzzBeer-Hofmann, Richard@\emph{von Richard Beer-Hofmann}!1893-06-201@{{[}vor dem 22. 6. 1893?{]}}|)be}\mylabel{L00223h}  \newcommand{\dateiname}{L00223}\newcommand{\titel}{Richard Beer-Hofmann an Arthur Schnitzler, [vor dem 22. 6. 1893?]}\newcommand{\editorInnen}{Martin Anton Müller und Gerd-Hermann Susen}%% latex-leseansicht-abspann.tex
%% Abspann für die Leseansicht.
%% Der Schalter \ifkorrekturansicht ist bereits durch den Vorspann gesetzt.

%% latex-abspann.tex
%% Gemeinsamer Abspann für Korrekturansicht und Leseansicht.
%% Setzt den Schalter \ifkorrekturansicht voraus (gesetzt in den
%% einbindenden Dateien latex-korrekturansicht-abspann.tex bzw.
%% latex-leseansicht-abspann.tex).
%% ---------------------------------------------------------------

\normalsize

% Das esempio-Environment wird nur in der Leseansicht benötigt
\ifkorrekturansicht\else
\newenvironment{esempio}[3]%
{
    \vspace{1.5ex}
    \rlap{\underline{#1}}
    \par
    \setlength{\parindent}{0cm}
    \nopagebreak
    \leftskip=#2cm
    \rightskip=#3cm
}
{
    \par
}
\fi

\doendnotes{C}
\bigskip
\vfill

\clearpage

\footnotesize

\ifkorrekturansicht
  \lohead{\textsc{register}}
\fi

% theindex-Environment neu definieren ohne reledmac
\makeatletter
\renewenvironment{theindex}{%
  \ifkorrekturansicht
    \section*{\indexname}%
  \else
    \subsubsection*{Index der erwähnten Entitäten}%
  \fi
  \setlength{\parindent}{0pt}%
  \setlength{\parskip}{0pt plus 0.3pt}%
  \let\item\@idxitem
}{%
  \ifkorrekturansicht\clearpage\fi
}
\makeatother

\IfFileExists{\jobname-pw.ind}{\input{\jobname-pw.ind}}{}

% Quellenangabe nur in der Leseansicht
\ifkorrekturansicht\else
% Fallback-Definitionen, falls die .tex-Datei \titel etc. nicht gesetzt hat
\providecommand{\titel}{}
\providecommand{\editorInnen}{}
\providecommand{\dateiname}{\jobname}

\vspace{3cm}

\vfill

\footnotesize
\textsc{Quelle}: \titel. Herausgegeben von {\editorInnen}. In: \emph{Arthur Schnitzler: Briefwechsel mit Autorinnen und Autoren}.
 Digitale Edition, https://schnitzler-briefe.acdh.oeaw.ac.at/{\dateiname}.html (Stand \today)
\fi

\end{document}


