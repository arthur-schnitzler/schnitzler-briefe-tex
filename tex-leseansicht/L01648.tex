%% latex-leseansicht-vorspann.tex
%% Vorspann für die Leseansicht.
%% Lädt die gemeinsame Datei latex-vorspann.tex mit nicht gesetztem Schalter.

\newif\ifkorrekturansicht
\korrekturansichtfalse

\input{../tex-inputs/latex-vorspann}


\section[Max Burckhard an Arthur Schnitzler, 3. 1. 1907]{L01648 Max Burckhard an Arthur Schnitzler, 3. 1. 1907}
\nopagebreak\mylabel{L01648v}
\rehead{ }\normalsize\beginnumbering\briefempfaengerindex{Schnitzler, Arthur@\textsc{Schnitzler, Arthur}!zzzBurckhard, Max Eugen@\emph{von Max Eugen Burckhard}!1907-01-031@{3. 1. 1907}|(be}
\toendnotes[C]{\smallbreak\pagebreak[2]}
\correspDesc{Versand  durch Max Burckhard am 3. 1. 1907 in St. Gilgen
\newline{}Erhalt  durch Arthur Schnitzler im Zeitraum [4. 1. 1907
                  – 8. 1. 1907?] in Wien}\toendnotes[C]{\smallbreak}
\Standort{CUL, Schnitzler, B 20.}
\physDesc{Brief, 1 Blatt, 2 Seiten, 475 Zeichen
\newline{}Handschrift: schwarze Tinte, deutsche Kurrent
\newline{}Ordnung: mit Bleistift von unbekannter Hand nummeriert:
                                    »16« }\toendnotes[C]{\smallbreak}
\pstart
           {\pb}\textcolor{gray}{\textbf{D\textsuperscript{r.} Max Burckhard}}\hfill \textcolor{gray}{\textbf{Wien, IX. Porzellangasse 48\oindex{Wien@\textbf{Wien}!IX., Alsergrund@\textbf{IX., Alsergrund}!Porzellangasse@\textbf{Porzellangasse}, \emph{Straße}|pw}{ }..........}}\pend
           
\pstart
           \raggedleft{}\textcolor{gray}{\textbf{St. Gilgen\oindex{St. Gilgen@\textbf{St. Gilgen}, \emph{Verwaltungsgebiet}|pw}}}{ }3. 1. 07\pend
           
\pstart{}Sehr verehrter lieber Doctor!\pend\vspace{0.5em}
\pstart
           Ich danke Ihnen herzlichſt für Ihre lieben Zeilen. Haben Sie heuer gar keine Luſt zum
               »Rodeln« herzukommen? Geſtern hat es zwar geregnet, heute aber{ }ſchneit es{ }ſchon
               wieder luſtig. »Schlitten« im Hauſe. Auch Sky’s. Herr und Frau Eichinger\pwindex{Eichinger @\textsc{Eichinger}, \emph{Haushälter}|pw}\pwindex{Eichinger @\textsc{Eichinger}, \emph{Haushälterin}|pw} lassen{ }ſich \label{T_L01648-1v}\edtext{{[}nicht{]}}{\lemma{\textnormal{\emph{nicht}}}\Cendnote{\textnormal{handschriftliche Ergänzung von Schnitzler
                  in der maschinellen Abschrift}}}\label{T_L01648-1} nehmen, daſs es Ihnen gar {\pb}nicht gefallen haben muß, da Sie und die
               gnädige Frau nicht mehr kommen.\pend
           
\pstart
           Mit herzlichſten Grüßen und Handkuſs an die verehrte gnädige Frau\pwindex{Schnitzler, Olga 17.\,1.\,1882 Wien – 13.\,1.\,1970 Lugano@\textsc{Schnitzler, Olga} (17.\,1.\,1882 Wien – 13.\,1.\,1970 Lugano), \emph{Schauspielerin, Sängerin}|pwv}\hspace*{1.5em}Ihr\pend
           \pstart getreuer \spacefill\mbox{D\textsuperscript{r}Burckhard}\pend{}\selectlanguage{ngerman}\endnumbering\briefempfaengerindex{Schnitzler, Arthur@\textsc{Schnitzler, Arthur}!zzzBurckhard, Max Eugen@\emph{von Max Eugen Burckhard}!1907-01-031@{3. 1. 1907}|)be}\mylabel{L01648h}  \newcommand{\dateiname}{L01648}\newcommand{\titel}{Max Burckhard an Arthur Schnitzler, 3. 1. 1907}\newcommand{\editorInnen}{Martin Anton Müller und Gerd-Hermann Susen}%% latex-leseansicht-abspann.tex
%% Abspann für die Leseansicht.
%% Der Schalter \ifkorrekturansicht ist bereits durch den Vorspann gesetzt.

%% latex-abspann.tex
%% Gemeinsamer Abspann für Korrekturansicht und Leseansicht.
%% Setzt den Schalter \ifkorrekturansicht voraus (gesetzt in den
%% einbindenden Dateien latex-korrekturansicht-abspann.tex bzw.
%% latex-leseansicht-abspann.tex).
%% ---------------------------------------------------------------

\normalsize

% Das esempio-Environment wird nur in der Leseansicht benötigt
\ifkorrekturansicht\else
\newenvironment{esempio}[3]%
{
    \vspace{1.5ex}
    \rlap{\underline{#1}}
    \par
    \setlength{\parindent}{0cm}
    \nopagebreak
    \leftskip=#2cm
    \rightskip=#3cm
}
{
    \par
}
\fi

\doendnotes{C}
\bigskip
\vfill

\clearpage

\footnotesize

\ifkorrekturansicht
  \lohead{\textsc{register}}
\fi

% theindex-Environment neu definieren ohne reledmac
\makeatletter
\renewenvironment{theindex}{%
  \ifkorrekturansicht
    \section*{\indexname}%
  \else
    \subsubsection*{Index der erwähnten Entitäten}%
  \fi
  \setlength{\parindent}{0pt}%
  \setlength{\parskip}{0pt plus 0.3pt}%
  \let\item\@idxitem
}{%
  \ifkorrekturansicht\clearpage\fi
}
\makeatother

\IfFileExists{\jobname-pw.ind}{\input{\jobname-pw.ind}}{}

% Quellenangabe nur in der Leseansicht
\ifkorrekturansicht\else
% Fallback-Definitionen, falls die .tex-Datei \titel etc. nicht gesetzt hat
\providecommand{\titel}{}
\providecommand{\editorInnen}{}
\providecommand{\dateiname}{\jobname}

\vspace{3cm}

\vfill

\footnotesize
\textsc{Quelle}: \titel. Herausgegeben von {\editorInnen}. In: \emph{Arthur Schnitzler: Briefwechsel mit Autorinnen und Autoren}.
 Digitale Edition, https://schnitzler-briefe.acdh.oeaw.ac.at/{\dateiname}.html (Stand \today)
\fi

\end{document}


