%% latex-korrekturansicht-vorspann.tex
%% Vorspann für die Korrekturansicht.
%% Lädt die gemeinsame Datei latex-vorspann.tex mit gesetztem Schalter.

\newif\ifkorrekturansicht
\korrekturansichttrue

\input{../tex-inputs/latex-vorspann}


\section[Max Burckhard an Arthur Schnitzler, 3. 1. 1907]{L01648 Max Burckhard an Arthur Schnitzler, 3. 1. 1907}
\nopagebreak\mylabel{L01648v}
\rehead{ }\normalsize\beginnumbering\briefempfaengerindex{Schnitzler, Arthur@\textsc{Schnitzler, Arthur}!zzzBurckhard, Max Eugen@\emph{von Max Eugen Burckhard}!1907-01-031@{3. 1. 1907}|(be}
\toendnotes[C]{\smallbreak\pagebreak[2]}\Standort{CUL, Schnitzler, B 20.}
\physDesc{Brief, 1 Blatt, 2 Seiten, 475 Zeichen
\newline{}Handschrift: schwarze Tinte, deutsche Kurrent
\newline{}Ordnung: mit Bleistift von unbekannter Hand nummeriert:
                                    »16« }\toendnotes[C]{\smallbreak}
\pstart
           {\pb}\textcolor{gray}{\textbf{D\textsuperscript{r.} Max Burckhard}}\hfill \textcolor{gray}{\textbf{Wien, IX. Porzellangasse 48\oindex{Porzellangasse@\textbf{Porzellangasse}, \emph{Straße (K.STR)}|pw}{ }..........}}\pend
           
\pstart
           \raggedleft{}\textcolor{gray}{\textbf{St. Gilgen\oindex{St. Gilgen@\textbf{St. Gilgen}, \emph{A.ADM3}|pw}}}{ }3. 1. 07\pend
           
\pstart{}Sehr verehrter lieber Doctor!\pend\vspace{0.5em}
\pstart
           Ich danke Ihnen herzlichſt für Ihre lieben Zeilen. Haben Sie heuer gar keine Luſt zum
               »Rodeln« herzukommen? Geſtern hat es zwar geregnet, heute aber ſchneit es ſchon
               wieder luſtig. »Schlitten« im Hauſe. Auch Sky’s. Herr und Frau Eichinger\pwindex{Eichinger @\textsc{Eichinger}, \emph{Haushälter/Haushälterin}|pw}\pwindex{Eichinger @\textsc{Eichinger}, \emph{Haushälter/Haushälterin}|pw} lassen ſich \label{T_L01648-1v}\edtext{{[}nicht{]}}{\lemma{\textnormal{\emph{nicht}}}\Cendnote{\textnormal{handschriftliche Ergänzung von Schnitzler
                  in der maschinellen Abschrift}}}\label{T_L01648-1} nehmen, daſs es Ihnen gar {\pb}nicht gefallen haben muß, da Sie und die
               gnädige Frau nicht mehr kommen.\pend
           
\pstart
           Mit herzlichſten Grüßen und Handkuſs an die verehrte gnädige Frau\pwindex{Schnitzler, Olga 17.01.1882 – 13.01.1970@\textsc{Schnitzler, Olga} (17.01.1882 – 13.01.1970), \emph{Schauspieler/Schauspielerin, Sänger/Sängerin}|pwv}\hspace*{1.5em}Ihr\pend
           \pstart getreuer \spacefill\mbox{D\textsuperscript{r}Burckhard}\pend{}\selectlanguage{ngerman}\endnumbering\briefempfaengerindex{Schnitzler, Arthur@\textsc{Schnitzler, Arthur}!zzzBurckhard, Max Eugen@\emph{von Max Eugen Burckhard}!1907-01-031@{3. 1. 1907}|)be}\mylabel{L01648h}  \normalsize

\doendnotes{C}
\bigskip
\vfill

\clearpage

\footnotesize

\lohead{\textsc{register}}

% Definiere theindex-Environment komplett neu ohne reledmac
\makeatletter
\renewenvironment{theindex}{%
  \section*{\indexname}%
  \setlength{\parindent}{0pt}%
  \setlength{\parskip}{0pt plus 0.3pt}%
  \let\item\@idxitem
}{%
  \clearpage
}
\makeatother

\IfFileExists{\jobname-pw.ind}{\input{\jobname-pw.ind}}{}

\end{document}

      