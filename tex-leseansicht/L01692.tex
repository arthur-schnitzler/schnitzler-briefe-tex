%% latex-leseansicht-vorspann.tex
%% Vorspann für die Leseansicht.
%% Lädt die gemeinsame Datei latex-vorspann.tex mit nicht gesetztem Schalter.

\newif\ifkorrekturansicht
\korrekturansichtfalse

\input{../tex-inputs/latex-vorspann}


               \section[Max Mell an Arthur Schnitzler, 15. 7. 1907]{ Max Mell an Arthur Schnitzler, 15. 7. 1907}\nopagebreak\mylabel{v}\rehead{ }\begin{ledgroupsized}[t]{13cm}\normalsize\beginnumbering\briefempfaengerindex{Schnitzler, Arthur@\textsc{Schnitzler, Arthur}!zzzMell, Max@\emph{von Max Mell}!1907-07-151@{15. 7. 1907}|(be} \toendnotes[C]{\smallbreak\pagebreak[2]} \Standort{CUL, Schnitzler, B 70.}
\physDesc{Brief, 1 Blatt, 1 Seite
\newline{}Handschrift: schwarze Tinte, deutsche Kurrent
\newline{}Schnitzler: mit Bleistift beschriftet: »\textsc{Mell}« }\toendnotes[C]{\smallbreak}\pstart
           \noindent{}{\pb}15/VII.\hfill 1907\pend
           \pstart
           \centering{}\textcolor{gray}{\textbf{WW WIENER}}\pend
           \pstart
           \noindent{}\centering{}\textcolor{gray}{\textbf{WERKSTÆTTE\orgindex{Wiener Werkstaette@Wiener Werkstätte|pw}}}\pend
           \pstart
           \noindent{}\centering{}\textcolor{gray}{\textbf{7\oindex{VII., Neubau@\textbf{VII., Neubau}|pw}}}\pend
           \pstart
           \noindent{}\centering{}\textcolor{gray}{\textbf{NEUSTIFTGASSE\oindex{Neustiftgasse@\textbf{Neustiftgasse}|pw}}}\pend
           \pstart
           \noindent{}\centering{}\textcolor{gray}{\textbf{32}}\pend
           \pstart{}Sehr verehrter Herr Doktor,\pend\pstart
           im Herbſt will die »Wiener Werkſtätte\orgindex{Wiener Werkstaette@Wiener Werkstätte|pw}« einen
                        \label{K_L01692_1v}\edtext{Almanach}{\lemma{\textnormal{\emph{Almanach}}}\Cendnote{\textnormal{In der hier präsentierten Form kam
                        der Almanach nicht zustande. Erst 1911 erschien ein solcher Almanach.}}}\label{K_L01692_1h} »Die Frau\pwindex{Almanach der Wiener Werkstaette1911 – 1911@\emph{Almanach der Wiener Werkstätte}|pwv}« herausgeben, ich
                    bin mit der Redaktion betraut und bitte Sie nun, mich mit einem Beitrag zu
                    unterſtützen. Hoffentlich können Sie mir dieſe Freude machen! Ich ſoll die
                    Einſendungen bis Anfang September beiſammen haben, was ſchon etwas knapp iſt,
                    aber Waerndorfer\pwindex{Waerndorfer, Friedrich 05.05.1868 – 09.08.1939@\textsc{Wärndorfer, Friedrich} (05.05.1868 – 09.08.1939), \emph{Industrieller}|pw} und Hoffmann\pwindex{Hoffmann, Josef 22.07.1831 – 31.01.1904@\textsc{Hoffmann, Josef} (22.07.1831 – 31.01.1904), \emph{Maler}|pw} konnten ſich ſolange nicht entſchließen. Es iſt
                    ſelbſtverſtändlig, daß Sie nur in die beſte Geſellschaft kommen.\pend
           \pstart
           Es war mir ſehr leid, Sie nicht mehr geſehen zu haben. So wünſch ich Ihnen und
                    Ihrer verehrten Frau\pwindex{Schnitzler, Olga 17.01.1882 – 13.01.1970@\textsc{Schnitzler, Olga} (17.01.1882 – 13.01.1970), \emph{Schauspielerin, Sängerin}|pwv}
                    ſchriftlich, aber nicht minder herzlich recht angenehmen Sommer. – Ich bleib
                    noch da, Mary\pwindex{Mell, Maria 12.07.1885 – 29.10.1954@\textsc{Mell, Maria} (12.07.1885 – 29.10.1954), \emph{Schauspielerin}|pw} iſt in Ungarn\oindex{Ungarn@\textbf{Ungarn}|pw}.\pend
           \pstart
           Mit den beſten Empfehlungen{\\[\baselineskip]}Ihr ſehr ergebener{\\[\baselineskip]}\spacefill\mbox{Max Mell.}\pend
           \leftskip=0em{}\pstart
           II. Wittelsbachſtr. 5\oindex{Wittelsbachstrasse@\textbf{Wittelsbachstraße}|pw}.\pend
                     \endnumbering\briefempfaengerindex{Schnitzler, Arthur@\textsc{Schnitzler, Arthur}!zzzMell, Max@\emph{von Max Mell}!1907-07-151@{15. 7. 1907}|)be}\mylabel{h}\end{ledgroupsized}  \newcommand{\dateiname}{L01692}\newcommand{\titel}{Max Mell an Arthur Schnitzler, 15. 7. 1907}\newcommand{\editorInnen}{Martin Anton Müller und Gerd-Hermann Susen}
            \footnotesize
\begin{ledgroupsized}[t]{11.5cm}
\doendnotes{C}
\end{ledgroupsized}
         %% latex-leseansicht-abspann.tex
%% Abspann für die Leseansicht.
%% Der Schalter \ifkorrekturansicht ist bereits durch den Vorspann gesetzt.

%% latex-abspann.tex
%% Gemeinsamer Abspann für Korrekturansicht und Leseansicht.
%% Setzt den Schalter \ifkorrekturansicht voraus (gesetzt in den
%% einbindenden Dateien latex-korrekturansicht-abspann.tex bzw.
%% latex-leseansicht-abspann.tex).
%% ---------------------------------------------------------------

\normalsize

% Das esempio-Environment wird nur in der Leseansicht benötigt
\ifkorrekturansicht\else
\newenvironment{esempio}[3]%
{
    \vspace{1.5ex}
    \rlap{\underline{#1}}
    \par
    \setlength{\parindent}{0cm}
    \nopagebreak
    \leftskip=#2cm
    \rightskip=#3cm
}
{
    \par
}
\fi

\doendnotes{C}
\bigskip
\vfill

\clearpage

\footnotesize

\ifkorrekturansicht
  \lohead{\textsc{register}}
\fi

% theindex-Environment neu definieren ohne reledmac
\makeatletter
\renewenvironment{theindex}{%
  \ifkorrekturansicht
    \section*{\indexname}%
  \else
    \subsubsection*{Index der erwähnten Entitäten}%
  \fi
  \setlength{\parindent}{0pt}%
  \setlength{\parskip}{0pt plus 0.3pt}%
  \let\item\@idxitem
}{%
  \ifkorrekturansicht\clearpage\fi
}
\makeatother

\IfFileExists{\jobname-pw.ind}{\input{\jobname-pw.ind}}{}

% Quellenangabe nur in der Leseansicht
\ifkorrekturansicht\else
% Fallback-Definitionen, falls die .tex-Datei \titel etc. nicht gesetzt hat
\providecommand{\titel}{}
\providecommand{\editorInnen}{}
\providecommand{\dateiname}{\jobname}

\vspace{3cm}

\vfill

\footnotesize
\textsc{Quelle}: \titel. Herausgegeben von {\editorInnen}. In: \emph{Arthur Schnitzler: Briefwechsel mit Autorinnen und Autoren}.
 Digitale Edition, https://schnitzler-briefe.acdh.oeaw.ac.at/{\dateiname}.html (Stand \today)
\fi

\end{document}


      