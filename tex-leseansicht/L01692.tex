%% latex-korrekturansicht-vorspann.tex
%% Vorspann für die Korrekturansicht.
%% Lädt die gemeinsame Datei latex-vorspann.tex mit gesetztem Schalter.

\newif\ifkorrekturansicht
\korrekturansichttrue

\input{../tex-inputs/latex-vorspann}


\section[Max Mell an Arthur Schnitzler, 15. 7. 1907]{L01692 Max Mell an Arthur Schnitzler, 15. 7. 1907}
\nopagebreak\mylabel{L01692v}
\rehead{ }\normalsize\beginnumbering\briefempfaengerindex{Schnitzler, Arthur@\textsc{Schnitzler, Arthur}!zzzMell, Max@\emph{von Max Mell}!1907-07-151@{15. 7. 1907}|(be}
\toendnotes[C]{\smallbreak\pagebreak[2]}\Standort{CUL, Schnitzler, B 70.}
\physDesc{Brief, 1 Blatt, 1 Seite, 760 Zeichen
\newline{}Handschrift: schwarze Tinte, deutsche Kurrent
\newline{}Schnitzler: mit Bleistift beschriftet: »\textsc{Mell}« }\toendnotes[C]{\smallbreak}
\pstart
           {\pb}15/VII.\hfill 1907\pend
           
\pstart
           \centering{}\textcolor{gray}{\textbf{WW WIENER}}\pend
           
\pstart
           \centering{}\textcolor{gray}{\textbf{WERKSTÆTTE\orgindex{Wiener Werkstaette@Wiener Werkstätte|pw}}}\pend
           
\pstart
           \centering{}\textcolor{gray}{\textbf{7\oindex{VII., Neubau@\textbf{VII., Neubau}, \emph{A.ADM3}|pw}}}\pend
           
\pstart
           \centering{}\textcolor{gray}{\textbf{NEUSTIFTGASSE\oindex{Neustiftgasse@\textbf{Neustiftgasse}, \emph{Straße (K.STR)}|pw}}}\pend
           
\pstart
           \centering{}\textcolor{gray}{\textbf{32}}\pend
           
\pstart{}Sehr verehrter Herr Doktor,\pend\vspace{0.5em}
\pstart
           im Herbſt will die »Wiener Werkſtätte\orgindex{Wiener Werkstaette@Wiener Werkstätte|pw}« einen
                  \label{K_L01692-1v}\edtext{Almanach}{\lemma{\textnormal{\emph{Almanach}}}\Cendnote{\textnormal{In der hier präsentierten Form kam der Almanach nicht zustande.
                  Erst 1911 erschien ein solcher Almanach.}}}\label{K_L01692-1} »Die Frau\pwindex{Almanach der Wiener Werkstaette@\emph{Almanach der Wiener Werkstätte}|pwv}« herausgeben, ich bin mit der
               Redaktion betraut und bitte Sie nun, mich mit einem Beitrag zu unterſtützen.
               Hoffentlich können Sie mir dieſe Freude machen! Ich ſoll die Einſendungen bis Anfang
               September beiſammen haben, was ſchon etwas knapp iſt, aber Waerndorfer\pwindex{Waerndorfer, Friedrich 05.05.1868 – 09.08.1939@\textsc{Wärndorfer, Friedrich} (05.05.1868 – 09.08.1939), \emph{Industrieller/Industrielle, Mäzen/Mäzenin, Unternehmer/Unternehmerin}|pw} und Hoffmann\pwindex{Hoffmann, Josef 22.07.1831 – 31.01.1904@\textsc{Hoffmann, Josef} (22.07.1831 – 31.01.1904), \emph{Maler/Malerin}|pw} konnten ſich ſolange nicht entſchließen. Es iſt ſelbſtverſtändlig,
               daß Sie nur in die beſte Geſellschaft kommen.\pend
           
\pstart
           Es war mir ſehr leid, Sie nicht mehr geſehen zu haben. So wünſch ich Ihnen und Ihrer
               verehrten Frau\pwindex{Schnitzler, Olga 17.01.1882 – 13.01.1970@\textsc{Schnitzler, Olga} (17.01.1882 – 13.01.1970), \emph{Schauspieler/Schauspielerin, Sänger/Sängerin}|pwv} ſchriftlich,
               aber nicht minder herzlich recht angenehmen Sommer. – Ich bleib noch da, Mary\pwindex{Mell, Maria 12.07.1885 – 29.10.1954@\textsc{Mell, Maria} (12.07.1885 – 29.10.1954), \emph{Schauspieler/Schauspielerin}|pw} iſt in Ungarn\oindex{Ungarn@\textbf{Ungarn}, \emph{A.PCLI}|pw}.\pend
           
\pstart
           Mit den beſten Empfehlungen{\\[\baselineskip]}Ihr ſehr ergebener{\\[\baselineskip]}\spacefill\mbox{Max Mell.}\pend
           \leftskip=0em{}
\pstart
           II. Wittelsbachſtr. 5\oindex{Wittelsbachstrasse@\textbf{Wittelsbachstraße}, \emph{Straße (K.STR)}|pw}.\pend
           \selectlanguage{ngerman}\endnumbering\briefempfaengerindex{Schnitzler, Arthur@\textsc{Schnitzler, Arthur}!zzzMell, Max@\emph{von Max Mell}!1907-07-151@{15. 7. 1907}|)be}\mylabel{L01692h}  \normalsize

\doendnotes{C}
\bigskip
\vfill

\clearpage

\footnotesize

\lohead{\textsc{register}}

% Definiere theindex-Environment komplett neu ohne reledmac
\makeatletter
\renewenvironment{theindex}{%
  \section*{\indexname}%
  \setlength{\parindent}{0pt}%
  \setlength{\parskip}{0pt plus 0.3pt}%
  \let\item\@idxitem
}{%
  \clearpage
}
\makeatother

\IfFileExists{\jobname-pw.ind}{\input{\jobname-pw.ind}}{}

\end{document}

      