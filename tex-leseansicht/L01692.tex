%% latex-leseansicht-vorspann.tex
%% Vorspann für die Leseansicht.
%% Lädt die gemeinsame Datei latex-vorspann.tex mit nicht gesetztem Schalter.

\newif\ifkorrekturansicht
\korrekturansichtfalse

\input{../tex-inputs/latex-vorspann}


\section[Max Mell an Arthur Schnitzler, 15. 7. 1907]{L01692 Max Mell an Arthur Schnitzler, 15. 7. 1907}
\nopagebreak\mylabel{L01692v}
\rehead{ }\normalsize\beginnumbering\briefempfaengerindex{Schnitzler, Arthur@\textsc{Schnitzler, Arthur}!zzzMell, Max@\emph{von Max Mell}!1907-07-151@{15. 7. 1907}|(be}
\toendnotes[C]{\smallbreak\pagebreak[2]}
\correspDesc{Versand  durch Max Mell am 15. 7. 1907 in Wien
\newline{}Erhalt  durch Arthur Schnitzler im Zeitraum [15. 7. 1907
                  – 19. 7. 1907?] in Wien}\toendnotes[C]{\smallbreak}
\Standort{CUL, Schnitzler, B 70.}
\physDesc{Brief, 1 Blatt, 1 Seite, 760 Zeichen
\newline{}Handschrift: schwarze Tinte, deutsche Kurrent
\newline{}Schnitzler: mit Bleistift beschriftet: »\textsc{Mell}« }\toendnotes[C]{\smallbreak}
\pstart
           {\pb}15/VII.\hfill 1907\pend
           
\pstart
           \centering{}\textcolor{gray}{\textbf{WW WIENER}}\pend
           
\pstart
           \centering{}\textcolor{gray}{\textbf{WERKSTÆTTE\orgindex{Wiener Werkstätte@Wiener Werkstätte|pw}}}\pend
           
\pstart
           \centering{}\textcolor{gray}{\textbf{7\oindex{VII., Neubau@\textbf{VII., Neubau}, \emph{Verwaltungsgebiet}|pw}}}\pend
           
\pstart
           \centering{}\textcolor{gray}{\textbf{NEUSTIFTGASSE\oindex{Wien@\textbf{Wien}!VII., Neubau@\textbf{VII., Neubau}!Neustiftgasse@\textbf{Neustiftgasse}, \emph{Straße}|pw}}}\pend
           
\pstart
           \centering{}\textcolor{gray}{\textbf{32}}\pend
           
\pstart{}Sehr verehrter Herr Doktor,\pend\vspace{0.5em}
\pstart
           im Herbſt will die »Wiener Werkſtätte\orgindex{Wiener Werkstätte@Wiener Werkstätte|pw}« einen
                  \label{K_L01692-1v}\edtext{Almanach}{\lemma{\textnormal{\emph{Almanach}}}\Cendnote{\textnormal{In der hier präsentierten Form kam der Almanach nicht zustande.
                  Erst 1911 erschien ein solcher Almanach.}}}\label{K_L01692-1} »Die Frau\pwindex{Almanach der Wiener Werkstätte@\emph{Almanach der Wiener Werkstätte}|pwv}« herausgeben, ich bin mit der
               Redaktion betraut und bitte Sie nun, mich mit einem Beitrag zu unterſtützen.
               Hoffentlich können Sie mir dieſe Freude machen! Ich{ }ſoll die Einſendungen bis Anfang
               September beiſammen haben, was{ }ſchon etwas knapp iſt, aber Waerndorfer\pwindex{Wärndorfer, Friedrich 5.\,5.\,1868 Wien – 9.\,8.\,1939 Bryn Mawr@\textsc{Wärndorfer, Friedrich} (5.\,5.\,1868 Wien – 9.\,8.\,1939 Bryn Mawr), \emph{Industrieller, Mäzen, Unternehmer}|pw} und Hoffmann\pwindex{Hoffmann, Josef 22.\,7.\,1831 Wien – 31.\,1.\,1904 ebd.@\textsc{Hoffmann, Josef} (22.\,7.\,1831 Wien – 31.\,1.\,1904 ebd.), \emph{Maler}|pw} konnten{ }ſich{ }ſolange nicht entſchließen. Es iſt{ }ſelbſtverſtändlig,
               daß Sie nur in die beſte Geſellschaft kommen.\pend
           
\pstart
           Es war mir{ }ſehr leid, Sie nicht mehr geſehen zu haben. So wünſch ich Ihnen und Ihrer
               verehrten Frau\pwindex{Schnitzler, Olga 17.\,1.\,1882 Wien – 13.\,1.\,1970 Lugano@\textsc{Schnitzler, Olga} (17.\,1.\,1882 Wien – 13.\,1.\,1970 Lugano), \emph{Schauspielerin, Sängerin}|pwv}{ }ſchriftlich,
               aber nicht minder herzlich recht angenehmen Sommer. – Ich bleib noch da, Mary\pwindex{Mell, Maria 12.\,7.\,1885 Maribor – 29.\,10.\,1954 Wien@\textsc{Mell, Maria} (12.\,7.\,1885 Maribor – 29.\,10.\,1954 Wien), \emph{Schauspielerin}|pw} iſt in Ungarn\oindex{Ungarn@\textbf{Ungarn}|pw}.\pend
           
\pstart
           Mit den beſten Empfehlungen{\\[\baselineskip]}Ihr{ }ſehr ergebener{\\[\baselineskip]}\spacefill\mbox{Max Mell.}\pend
           \leftskip=0em{}
\pstart
           II. Wittelsbachſtr. 5\oindex{Wien@\textbf{Wien}!II., Leopoldstadt@\textbf{II., Leopoldstadt}!Wittelsbachstraße@\textbf{Wittelsbachstraße}, \emph{Straße}|pw}.\pend
           \selectlanguage{ngerman}\endnumbering\briefempfaengerindex{Schnitzler, Arthur@\textsc{Schnitzler, Arthur}!zzzMell, Max@\emph{von Max Mell}!1907-07-151@{15. 7. 1907}|)be}\mylabel{L01692h}  \newcommand{\dateiname}{L01692}\newcommand{\titel}{Max Mell an Arthur Schnitzler, 15. 7. 1907}\newcommand{\editorInnen}{Martin Anton Müller und Gerd-Hermann Susen}%% latex-leseansicht-abspann.tex
%% Abspann für die Leseansicht.
%% Der Schalter \ifkorrekturansicht ist bereits durch den Vorspann gesetzt.

%% latex-abspann.tex
%% Gemeinsamer Abspann für Korrekturansicht und Leseansicht.
%% Setzt den Schalter \ifkorrekturansicht voraus (gesetzt in den
%% einbindenden Dateien latex-korrekturansicht-abspann.tex bzw.
%% latex-leseansicht-abspann.tex).
%% ---------------------------------------------------------------

\normalsize

% Das esempio-Environment wird nur in der Leseansicht benötigt
\ifkorrekturansicht\else
\newenvironment{esempio}[3]%
{
    \vspace{1.5ex}
    \rlap{\underline{#1}}
    \par
    \setlength{\parindent}{0cm}
    \nopagebreak
    \leftskip=#2cm
    \rightskip=#3cm
}
{
    \par
}
\fi

\doendnotes{C}
\bigskip
\vfill

\clearpage

\footnotesize

\ifkorrekturansicht
  \lohead{\textsc{register}}
\fi

% theindex-Environment neu definieren ohne reledmac
\makeatletter
\renewenvironment{theindex}{%
  \ifkorrekturansicht
    \section*{\indexname}%
  \else
    \subsubsection*{Index der erwähnten Entitäten}%
  \fi
  \setlength{\parindent}{0pt}%
  \setlength{\parskip}{0pt plus 0.3pt}%
  \let\item\@idxitem
}{%
  \ifkorrekturansicht\clearpage\fi
}
\makeatother

\IfFileExists{\jobname-pw.ind}{\input{\jobname-pw.ind}}{}

% Quellenangabe nur in der Leseansicht
\ifkorrekturansicht\else
% Fallback-Definitionen, falls die .tex-Datei \titel etc. nicht gesetzt hat
\providecommand{\titel}{}
\providecommand{\editorInnen}{}
\providecommand{\dateiname}{\jobname}

\vspace{3cm}

\vfill

\footnotesize
\textsc{Quelle}: \titel. Herausgegeben von {\editorInnen}. In: \emph{Arthur Schnitzler: Briefwechsel mit Autorinnen und Autoren}.
 Digitale Edition, https://schnitzler-briefe.acdh.oeaw.ac.at/{\dateiname}.html (Stand \today)
\fi

\end{document}


