%% latex-korrekturansicht-vorspann.tex
%% Vorspann für die Korrekturansicht.
%% Lädt die gemeinsame Datei latex-vorspann.tex mit gesetztem Schalter.

\newif\ifkorrekturansicht
\korrekturansichttrue

\input{../tex-inputs/latex-vorspann}


\section[ Felix Salten an Arthur Schnitzler, 23. 7. 1897]{L03271 Felix Salten an Arthur Schnitzler, 23. 7. 1897}
\nopagebreak\mylabel{L03271v}
\rehead{ }\normalsize\beginnumbering\briefempfaengerindex{Schnitzler, Arthur@\textsc{Schnitzler, Arthur}!zzzSalten, Felix@\emph{von Felix Salten}!1897-07-232@{23. 7. 1897}|(be}
\toendnotes[C]{\smallbreak\pagebreak[2]}\Standort{CUL, Schnitzler, B 89, A 2.}
\physDesc{Postkarte, 169 Zeichen
\newline{}Handschrift: Bleistift, lateinische Kurrent
\newline{}Versand: Stempel: »\nobreak{}\oindex{IX., Alsergrund@\textbf{IX., Alsergrund}, \emph{A.ADM3}|pwk}Wien 9/3 72, 23 7. 97, 4–5N\nobreak{}«. Stempel: »\nobreak{}\oindex{Bad Ischl@\textbf{Bad Ischl}, \emph{P.PPL}|pwk}Isch\textcolor{gray}{l}, 24/7 97, 7–\textcolor{gray}{8}\nobreak{}«.  
\newline{}Schnitzler: mit Bleistift datiert: »23/7« 
\newline{}Ordnung: mit Bleistift von unbekannter Hand nummeriert: »94« }
\buchAbdrucke{\weitereDrucke{Hermann Bahr, Arthur Schnitzler: \emph{Briefwechsel, Aufzeichnungen, Dokumente (1891–1931)}. Göttingen: \emph{Wallstein} 2018, S. 150.} }\toendnotes[C]{\smallbreak}\pstart{}{\pb}Herrn D\textsuperscript{r} Arthur Schnitzler\pend{}\pstart{}\strikeout{Wien}\oindex{Wien@\textbf{Wien}, \emph{A.ADM2}|pw}{ }Ischl\oindex{Bad Ischl@\textbf{Bad Ischl}, \emph{P.PPL}|pw}\pend{}\pstart{}Kaltenbach, Pension Petter\oindex{Hotel und Pension Rudolfshoehe (Leopold Petter)@\textbf{Hotel und Pension Rudolfshöhe (Leopold Petter)}, \emph{Hotel (K.HTL)}|pw}.\pend{}{\bigskip}\vspace{1em}
\pstart
           \noindent{}{\pb}Heute hab ich die Quelle jener \label{K_L03271-1v}\edtext{Nachricht\pwindex{Theater, Kunst und Literatur [Agnes Jordan nicht am Burgtheater]@\emph{Theater, Kunst und Literatur [Agnes Jordan nicht am Burgtheater]}|pwv}}{\lemma{\textnormal{\emph{Nachricht}}}\Cendnote{\textnormal{Siehe Felix Salten an Arthur Schnitzler, 22. 7. 1897.
               }}}\label{K_L03271-1} erfahren. – \label{K_L03271-2v}\edtext{B.\pwindex{Bahr, Hermann 19.07.1863 – 15.01.1934@\textsc{Bahr, Hermann} (19.07.1863 – 15.01.1934), \emph{Schriftsteller/Schriftstellerin, Kritiker/Kritikerin}|pw}}{\lemma{\textnormal{\emph{B.}}}\Cendnote{\textnormal{Obwohl sich das Initial auch 
                  auf Max Burckhard\pwindex{Burckhard, Max Eugen 14.07.1854 – 16.03.1912@\textsc{Burckhard, Max Eugen} (14.07.1854 – 16.03.1912), \emph{Schriftsteller/Schriftstellerin, Rechtswissenschaftler/Rechtswissenschaftlerin, Theaterleiter/Theaterleiterin}|pwk} beziehen könnte, wird
                  durch die Vorgeschichte (siehe Felix Salten an Arthur Schnitzler, 17. 7. 1897) deutlich,
                  dass Hermann Bahr\pwindex{Bahr, Hermann 19.07.1863 – 15.01.1934@\textsc{Bahr, Hermann} (19.07.1863 – 15.01.1934), \emph{Schriftsteller/Schriftstellerin, Kritiker/Kritikerin}|pwk} als der Strippenzieher
                  im Hintergrund betrachtet wird, von dem man sich eine solche Information an 
                  die Presse erwartete.}}}\label{K_L03271-2}\pend
           
\pstart
           Das hätte \substVorne{}\textsuperscript{ich}\substDazwischen{}m\substHinten{}an sich eigentlich denken können.\pend
           \pstart Herzlich \spacefill\mbox{S.}\pend{}\selectlanguage{ngerman}\endnumbering\briefempfaengerindex{Schnitzler, Arthur@\textsc{Schnitzler, Arthur}!zzzSalten, Felix@\emph{von Felix Salten}!1897-07-232@{23. 7. 1897}|)be}\mylabel{L03271h}  \normalsize

\doendnotes{C}
\bigskip
\vfill

\clearpage

\footnotesize

\lohead{\textsc{register}}

% Definiere theindex-Environment komplett neu ohne reledmac
\makeatletter
\renewenvironment{theindex}{%
  \section*{\indexname}%
  \setlength{\parindent}{0pt}%
  \setlength{\parskip}{0pt plus 0.3pt}%
  \let\item\@idxitem
}{%
  \clearpage
}
\makeatother

\IfFileExists{\jobname-pw.ind}{\input{\jobname-pw.ind}}{}

\end{document}

      