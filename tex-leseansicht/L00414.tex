%% latex-korrekturansicht-vorspann.tex
%% Vorspann für die Korrekturansicht.
%% Lädt die gemeinsame Datei latex-vorspann.tex mit gesetztem Schalter.

\newif\ifkorrekturansicht
\korrekturansichttrue

\input{../tex-inputs/latex-vorspann}


\section[Arthur Schnitzler an Richard Beer-Hofmann, {[}28. 3. 1895?{]}]{L00414 Arthur Schnitzler an Richard Beer-Hofmann, {[}28. 3. 1895?{]}}
\nopagebreak\mylabel{L00414v}
\rehead{ }\normalsize\beginnumbering\briefempfaengerindex{Beer-Hofmann, Richard@\textsc{Beer-Hofmann, Richard}!zzzSchnitzler, Arthur@\emph{von Arthur Schnitzler}!1895-03-282@{{[}28. 3. 1895?{]}}|(be}
\toendnotes[C]{\smallbreak\pagebreak[2]}\Standort{YCGL, MSS 31.}
\physDesc{Brief, 1 Blatt, 3 Seiten, Umschlag, 432 Zeichen
\newline{}Handschrift: Bleistift, deutsche Kurrent
\newline{}Versand: ohne postalischen Übermittlungsvermerk }\toendnotes[C]{\smallbreak}\pstart{}{\pb}Herrn \textsc{Dr. Richard
                     Beer-Hofmann}\pend{}\pstart{}Wien\oindex{Wien@\textbf{Wien}, \emph{A.ADM2}|pw}\pend{}\pstart{}\textsc{I. Wollzeile 15\oindex{Wollzeile@\textbf{Wollzeile}, \emph{Straße (K.STR)}|pw}, }\pend{}\pstart{}4. Stock\pend{}{\bigskip}\vspace{1em}
\pstart
           {\pb}\textcolor{gray}{\textbf{\label{T_L00414-1v}\edtext{A S}{\lemma{\textnormal{\emph{A S}}}\Cendnote{\textnormal{Prägedruck}}}\label{T_L00414-1}}}\pend
           
\pstart{}Mein lieber Richard\pend\vspace{0.5em}
\pstart
           beiliegendes\pwindex{Jeunes Viennois@\emph{Les Jeunes Viennois}|pwv} erhalte ich
                  \label{K_L00414-1v}\edtext{heute von Paul\pwindex{Goldmann, Paul 31.01.1865 – 25.09.1935@\textsc{Goldmann, Paul} (31.01.1865 – 25.09.1935), \emph{Schriftsteller/Schriftstellerin, Journalist/Journalistin}|pw}}{\lemma{\textnormal{\emph{heute von Paul}}}\Cendnote{\textnormal{Es dürfte sich um die Zusendung des
                  Bürstenabzugs von \emph{Les Jeunes Viennois}\pwindex{Jeunes Viennois@\emph{Les Jeunes Viennois}|pwk}
                  handeln, vgl. Paul Goldmann an Arthur Schnitzler, 28. 3. [1895].
               }}}\label{K_L00414-1} geſandt. Wenn Sie u Hugo\pwindex{Hofmannsthal, Hugo von 1874-02-01 – 1929-07-15@\textsc{Hofmannsthal, Hugo von} (1874-02-01 – 1929-07-15), \emph{Schriftsteller/Schriftstellerin}|pw} es geleſen,
               geben Sie mir’s zurück. Ich hab die betreffd Nu{\geminationm}er
               beſtellt, auch eine für Sie u eine für Hugo\pwindex{Hofmannsthal, Hugo von 1874-02-01 – 1929-07-15@\textsc{Hofmannsthal, Hugo von} (1874-02-01 – 1929-07-15), \emph{Schriftsteller/Schriftstellerin}|pw}.\pend
           
\pstart
           {\pb}– Vielleicht ſeh ich Sie heut Abend doch noch im
               Cafè, ich denk, daſs ich nach zwölf dort bin. Laſſen Sie mich für alle
               Fälle wiſſen, wo Sie u Hugo\pwindex{Hofmannsthal, Hugo von 1874-02-01 – 1929-07-15@\textsc{Hofmannsthal, Hugo von} (1874-02-01 – 1929-07-15), \emph{Schriftsteller/Schriftstellerin}|pw} morgen {\pb}aufzugreifen ſind.\pend
           
\pstart
           Herzlich grüßend{\\[\baselineskip]}Ihr \spacefill\mbox{Arth}\pend
           \leftskip=0em{}\selectlanguage{ngerman}\endnumbering\briefempfaengerindex{Beer-Hofmann, Richard@\textsc{Beer-Hofmann, Richard}!zzzSchnitzler, Arthur@\emph{von Arthur Schnitzler}!1895-03-282@{{[}28. 3. 1895?{]}}|)be}\mylabel{L00414h}  \normalsize

\doendnotes{C}
\bigskip
\vfill

\clearpage

\footnotesize

\lohead{\textsc{register}}

% Definiere theindex-Environment komplett neu ohne reledmac
\makeatletter
\renewenvironment{theindex}{%
  \section*{\indexname}%
  \setlength{\parindent}{0pt}%
  \setlength{\parskip}{0pt plus 0.3pt}%
  \let\item\@idxitem
}{%
  \clearpage
}
\makeatother

\IfFileExists{\jobname-pw.ind}{\input{\jobname-pw.ind}}{}

\end{document}

      