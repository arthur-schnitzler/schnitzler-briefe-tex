%% latex-leseansicht-vorspann.tex
%% Vorspann für die Leseansicht.
%% Lädt die gemeinsame Datei latex-vorspann.tex mit nicht gesetztem Schalter.

\newif\ifkorrekturansicht
\korrekturansichtfalse

\input{../tex-inputs/latex-vorspann}


         
         \renewcommand{\erwaehntePersonen}{Personen: Richard Beer-Hofmann, Paul Goldmann, Hugo von Hofmannsthal}
         \renewcommand{\erwaehnteOrte}{Orte: Wien, Wollzeile}
         \renewcommand{\erwaehnteWerke}{Werke: Les Jeunes Viennois}
               \section[Arthur Schnitzler an Richard Beer-Hofmann, {[}28.  3. 1895?{]}]{ Arthur Schnitzler an Richard Beer-Hofmann,
               {[}28.  3. 1895?{]}}\nopagebreak\mylabel{v}\rehead{ }\begin{ledgroupsized}[t]{13cm}\normalsize\beginnumbering \toendnotes[C]{\smallbreak\pagebreak[2]} \Standort{YCGL, MSS 31.}
\physDesc{Brief, 1 Blatt, 3 Seiten, Umschlag
\newline{}Handschrift: Bleistift, deutsche Kurrent\newline{}Versand: ohne postalischen Übermittlungsvermerk }\toendnotes[C]{\smallbreak}\pstart{}{\pb}Herrn \textsc{Dr. Richard
                     Beer-Hofmann}\pend{}\pstart{}Wien\oindex{Wien@\textbf{Wien}|pw}\pend{}\pstart{}\textsc{I. Wollzeile 15\oindex{Wollzeile@\textbf{Wollzeile}|pw}, }\pend{}\pstart{}4. Stock\pend{}{\bigskip}\pstart
           \noindent{}{\pb}\textcolor{gray}{\textbf{\label{T_L00414-1v}\edtext{A S}{\lemma{\textnormal{\emph{A S}}}\Cendnote{\textnormal{Prägedruck}}}\label{T_L00414-1h}}}\pend
           \pstart{}Mein lieber Richard\pend\pstart
           beiliegendes\pwindex{\textcolor{red}{\textsuperscript{XXXX1 indx}}!Jeunes Viennois01. 04. 1895@\strich\emph{Les Jeunes Viennois} {[}01. 04. 1895{]}|pwv} erhalte ich \label{K_L00414-44v}\edtext{heute von
                  Paul\pwindex{Goldmann, Paul 31.01.1865 – 25.09.1935@\textsc{Goldmann, Paul} (31.01.1865 – 25.09.1935), \emph{Schriftsteller, Journalist}|pw}}{\lemma{\textnormal{\emph{heute von
                  Paul}}}\Cendnote{\textnormal{Es dürfte sich um die Zusendung des Bürstenabzugs von \emph{Les Jeunes Viennois}\pwindex{\textcolor{red}{\textsuperscript{XXXX1 indx}}!Jeunes Viennois01. 04. 1895@\strich\emph{Les Jeunes Viennois} {[}01. 04. 1895{]}|pwk} handeln, vgl. Paul Goldmann an Arthur Schnitzler, 28. 3. [1895]}}}\label{K_L00414-44h} geſandt. Wenn Sie u Hugo\pwindex{Hofmannsthal, Hugo von 1874-02-01 – 1929-07-15@\textsc{Hofmannsthal, Hugo von} (1874-02-01 – 1929-07-15), \emph{Schriftsteller}|pw} es geleſen, geben Sie mir’s zurück. Ich
               hab die betreffd Nu{\geminationm}er beſtellt, auch eine für Sie u
               eine für Hugo\pwindex{Hofmannsthal, Hugo von 1874-02-01 – 1929-07-15@\textsc{Hofmannsthal, Hugo von} (1874-02-01 – 1929-07-15), \emph{Schriftsteller}|pw}.\pend
           \pstart
           {\pb}– Vielleicht ſeh ich Sie heut Abend doch noch im
               Cafè, ich denk, daſs ich nach zwölf dort bin. Laſſen Sie mich für alle
               Fälle wiſſen, wo Sie u Hugo\pwindex{Hofmannsthal, Hugo von 1874-02-01 – 1929-07-15@\textsc{Hofmannsthal, Hugo von} (1874-02-01 – 1929-07-15), \emph{Schriftsteller}|pw} morgen {\pb}aufzugreifen ſind.\pend
           \pstart
           Herzlich grüßend{\\[\baselineskip]}Ihr \spacefill\mbox{Arth}\pend
           \leftskip=0em{}
         
         \endnumbering\mylabel{h}\end{ledgroupsized}  \newcommand{\dateiname}{L00414}\newcommand{\titel}{Arthur Schnitzler an Richard Beer-Hofmann, [28.  3. 1895?]}\newcommand{\editorInnen}{Martin Anton Müller und Gerd-Hermann Susen}%% latex-leseansicht-abspann.tex
%% Abspann für die Leseansicht.
%% Der Schalter \ifkorrekturansicht ist bereits durch den Vorspann gesetzt.

%% latex-abspann.tex
%% Gemeinsamer Abspann für Korrekturansicht und Leseansicht.
%% Setzt den Schalter \ifkorrekturansicht voraus (gesetzt in den
%% einbindenden Dateien latex-korrekturansicht-abspann.tex bzw.
%% latex-leseansicht-abspann.tex).
%% ---------------------------------------------------------------

\normalsize

% Das esempio-Environment wird nur in der Leseansicht benötigt
\ifkorrekturansicht\else
\newenvironment{esempio}[3]%
{
    \vspace{1.5ex}
    \rlap{\underline{#1}}
    \par
    \setlength{\parindent}{0cm}
    \nopagebreak
    \leftskip=#2cm
    \rightskip=#3cm
}
{
    \par
}
\fi

\doendnotes{C}
\bigskip
\vfill

\clearpage

\footnotesize

\ifkorrekturansicht
  \lohead{\textsc{register}}
\fi

% theindex-Environment neu definieren ohne reledmac
\makeatletter
\renewenvironment{theindex}{%
  \ifkorrekturansicht
    \section*{\indexname}%
  \else
    \subsubsection*{Index der erwähnten Entitäten}%
  \fi
  \setlength{\parindent}{0pt}%
  \setlength{\parskip}{0pt plus 0.3pt}%
  \let\item\@idxitem
}{%
  \ifkorrekturansicht\clearpage\fi
}
\makeatother

\IfFileExists{\jobname-pw.ind}{\input{\jobname-pw.ind}}{}

% Quellenangabe nur in der Leseansicht
\ifkorrekturansicht\else
% Fallback-Definitionen, falls die .tex-Datei \titel etc. nicht gesetzt hat
\providecommand{\titel}{}
\providecommand{\editorInnen}{}
\providecommand{\dateiname}{\jobname}

\vspace{3cm}

\vfill

\footnotesize
\textsc{Quelle}: \titel. Herausgegeben von {\editorInnen}. In: \emph{Arthur Schnitzler: Briefwechsel mit Autorinnen und Autoren}.
 Digitale Edition, https://schnitzler-briefe.acdh.oeaw.ac.at/{\dateiname}.html (Stand \today)
\fi

\end{document}


      