%% latex-leseansicht-vorspann.tex
%% Vorspann für die Leseansicht.
%% Lädt die gemeinsame Datei latex-vorspann.tex mit nicht gesetztem Schalter.

\newif\ifkorrekturansicht
\korrekturansichtfalse

\input{../tex-inputs/latex-vorspann}

\begin{center}
            \textcolor{red}{ENTWURF, NICHT FERTIG KORRIGIERT}
                      \end{center}
            
         
         \renewcommand{\erwaehntePersonen}{Personen: Olga Schnitzler}
         \renewcommand{\erwaehnteOrte}{Orte: Berlin, Dessauer Straße, Wien}
         \renewcommand{\erwaehnteWerke}{Werke: Der Schleier der Beatrice. Schauspiel in fünf Akten}
               \section[ Paul Goldmann an Arthur Schnitzler, 2. 9. {[}1902{]}]{ Paul Goldmann an Arthur Schnitzler, 2. 9. {[}1902{]}}\nopagebreak\mylabel{v}\rehead{ }\begin{ledgroupsized}[t]{13cm}\normalsize\beginnumbering \toendnotes[C]{\smallbreak\pagebreak[2]} \Standort{DLA, A:Schnitzler, HS.NZ85.1.3172.}
\physDesc{Brief, 2 Blätter, 7 Seiten
\newline{}Handschrift: blaue Tinte, deutsche Kurrent
\newline{}Schnitzler: 1) mit Bleistift das Jahr »{[}1{]}902«
                                            vermerkt  2) mit rotem Buntstift eine Unterstreichung}\pstart
           \noindent{}\raggedleft{}{\pb}\textcolor{gray}{\textbf{DESSAUERSTRASSE 19}}\oindex{Dessauer Strasse@\textbf{Dessauer Straße}|pw}\pend
           \pstart
           Berlin\oindex{Berlin@\textbf{Berlin}|pw}, 2. Sept.\pend
           \pstart\center{}Mein lieber Freund,\pend\pstart
           Die Paß-Angelegenheit hat mich nicht gar ſo viel Zeit gekoſtet, und ich brauche
                    Dir nicht erſt zu ſagen, daß es mir eine große Freude macht, meine Zeit auf eine
                    Angelegenheit zu verwenden, die Dich (wenn auch nur indirekt) betrifft. Die
                    einwöchentliche Frist müßt Ihr benutzen, um wenigſtens die Ausſtellung eines
                    Interims-Paſſes zu ermöglichen. Sonſt ſtehe ich {\pb} für nichts. Es muß doch noch Rechtsmittel geben, um den Kerl\textcolor{red}{\textsuperscript{\textbf{KEY}}} zu zwingen. Vielleicht iſt, da der Vater\textcolor{red}{\textsuperscript{\textbf{KEY}}} ſo vollſtändig ſeine Pflichten vernachläſſigt, eine frühere
                    Großjährigkeits-Erklärung oder die Beſtellung eines Vormunds möglich. \pend
           \pstart
           Die Ausſicht, Dich bald hier\textcolor{red}{\textsuperscript{\textbf{KEY}}} zu ſehen, bereitet mir
                    große Freude. Freilich werde ich von Deinem Aufenthalt wenig haben, {\pb} da gerade Mitte Oktober meine
                    Arbeit ins Ungeheure wachſen dürfte. \pend
           \pstart
           \textsc{Dr. Hugo Felix\textcolor{red}{\textsuperscript{\textbf{KEY}}}} iſt hier\textcolor{red}{\textsuperscript{\textbf{KEY}}} – ein ſehr lieber Menſch, der mir
                    ausgezeichnet gefällt. Er hat mich \strikeout{gebete}
                    erſucht, Dich zu bitten, Du möchteſt ihm doch die Erlaubniß geben, aus der »\textsc{Beatrice\pwindex{Schnitzler, Arthur 15.05.1862 – 21.10.1931@\textsc{Schnitzler, Arthur} (15.05.1862 – 21.10.1931), \emph{Schriftsteller, Mediziner}!Schleier der Beatrice. Schauspiel in fuenf Akten1900-12-01@\strich\emph{Der Schleier der Beatrice. Schauspiel in fünf Akten} {[}1900-12-01{]}|pw}}«, die er entzückend findet und von der er ſagt, daß ſie ihm herrlich
                    »liegt«, für Italien\textcolor{red}{\textsuperscript{\textbf{KEY}}} eine Oper zu machen. Er will
                    ſich nicht direkt an {\pb} Dich wenden, weil er\textcolor{red}{\textsuperscript{\textbf{KEY}}} fürchtet, Du würdeſt ihm gegenüber, auch wenn
                    Dir der Vorſchlag nicht paßte, mit der Sprache nicht heraus wollen, um ihn nicht
                        \strikeout{\textcolor{gray}{kra}} zu kränken, und würdeſt Dich ſo gebunden fühlen, ſeine Bitte bejahend zu
                    beantworten. Darum hat er mich um meine Vermittelung gebeten, die ich gern
                    übernehme, weil ich überzeugt bin, daß Gutes für beide Theile herauskommen
                    würde, wenn die Angelegenheit {\pb} ſich arrangiren
                    ließe. Ich bitte um eine möglichſt umgehende Antwort, da ich MontagAbend mit \textsc{Felix\textcolor{red}{\textsuperscript{\textbf{KEY}}}} zuſammenſein ſoll und ihm einen Beſcheid bringen möchte. \pend
           \pstart
           Ich danke Dir für die Empfehlung der Werke\textcolor{red}{\textsuperscript{\textbf{KEY}}} von \textsc{Tschechow\textcolor{red}{\textsuperscript{\textbf{KEY}}}}. Ich entdecke dieſer Tage ein entzückendes franzöſiſches Aphorismen-\textcolor{red}{\textsuperscript{\textbf{KEY}}}{\pb}Buch\textcolor{red}{\textsuperscript{\textbf{KEY}}} »\textsc{Maximes de la vie\textcolor{red}{\textsuperscript{\textbf{KEY}}}}« von \textsc{Comtesse Diane\textcolor{red}{\textsuperscript{\textbf{KEY}}}}. Laß’ es Dich die 8 \textsc{MK} nicht reuen, die es
                    koſtet; Du wirſt Freude daran haben. \pend
           \pstart
           Ich hoffe, daß \textsc{Olga\pwindex{Schnitzler, Olga 17.01.1882 – 13.01.1970@\textsc{Schnitzler, Olga} (17.01.1882 – 13.01.1970), \emph{Schauspielerin, Sängerin}|pw}} bald wiederhergeſtellt ſein wird\textcolor{gray}{,} bitte, ſie vielmals
                    von mir zu grüßen, {\pb} und begrüße auch Dich auf
                    das Herzlichſte. {\\[\baselineskip]}Dein\pend
           \leftskip=0em{}\pstart
           {\\[\baselineskip]}\spacefill\mbox{Paul Goldmnn}\pend
           \leftskip=0em{}\pstart
           Ich würde Dir dankbarſein, wenn Du mirmittheilen wollteſt, welchenEindruck die »Zeit\textcolor{red}{\textsuperscript{\textbf{KEY}}}« auf Dich
                        undüberhaupt in Wien\oindex{Wien@\textbf{Wien}|pw} macht?
                    \pend
           
         
         \endnumbering\mylabel{h}\end{ledgroupsized}\begin{anhang}\end{anhang}\newcommand{\dateiname}{L03223}\newcommand{\titel}{Paul Goldmann an Arthur Schnitzler, 2. 9. [1902]}\newcommand{\editorInnen}{Martin Anton Müller und Laura Untner}%% latex-leseansicht-abspann.tex
%% Abspann für die Leseansicht.
%% Der Schalter \ifkorrekturansicht ist bereits durch den Vorspann gesetzt.

%% latex-abspann.tex
%% Gemeinsamer Abspann für Korrekturansicht und Leseansicht.
%% Setzt den Schalter \ifkorrekturansicht voraus (gesetzt in den
%% einbindenden Dateien latex-korrekturansicht-abspann.tex bzw.
%% latex-leseansicht-abspann.tex).
%% ---------------------------------------------------------------

\normalsize

% Das esempio-Environment wird nur in der Leseansicht benötigt
\ifkorrekturansicht\else
\newenvironment{esempio}[3]%
{
    \vspace{1.5ex}
    \rlap{\underline{#1}}
    \par
    \setlength{\parindent}{0cm}
    \nopagebreak
    \leftskip=#2cm
    \rightskip=#3cm
}
{
    \par
}
\fi

\doendnotes{C}
\bigskip
\vfill

\clearpage

\footnotesize

\ifkorrekturansicht
  \lohead{\textsc{register}}
\fi

% theindex-Environment neu definieren ohne reledmac
\makeatletter
\renewenvironment{theindex}{%
  \ifkorrekturansicht
    \section*{\indexname}%
  \else
    \subsubsection*{Index der erwähnten Entitäten}%
  \fi
  \setlength{\parindent}{0pt}%
  \setlength{\parskip}{0pt plus 0.3pt}%
  \let\item\@idxitem
}{%
  \ifkorrekturansicht\clearpage\fi
}
\makeatother

\IfFileExists{\jobname-pw.ind}{\input{\jobname-pw.ind}}{}

% Quellenangabe nur in der Leseansicht
\ifkorrekturansicht\else
% Fallback-Definitionen, falls die .tex-Datei \titel etc. nicht gesetzt hat
\providecommand{\titel}{}
\providecommand{\editorInnen}{}
\providecommand{\dateiname}{\jobname}

\vspace{3cm}

\vfill

\footnotesize
\textsc{Quelle}: \titel. Herausgegeben von {\editorInnen}. In: \emph{Arthur Schnitzler: Briefwechsel mit Autorinnen und Autoren}.
 Digitale Edition, https://schnitzler-briefe.acdh.oeaw.ac.at/{\dateiname}.html (Stand \today)
\fi

\end{document}


      