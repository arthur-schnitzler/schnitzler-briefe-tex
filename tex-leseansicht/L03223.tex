%% latex-leseansicht-vorspann.tex
%% Vorspann für die Leseansicht.
%% Lädt die gemeinsame Datei latex-vorspann.tex mit nicht gesetztem Schalter.

\newif\ifkorrekturansicht
\korrekturansichtfalse

\input{../tex-inputs/latex-vorspann}


         
         \renewcommand{\erwaehntePersonen}{Personen: Marie Suin Beausacq, Hugo Felix, Paul Goldmann, Rudolf Gussmann, Sully Prudhomme, Olga Schnitzler, Elisabeth Steinrück, Anton Pavlovič Čechov}
         \renewcommand{\erwaehnteInstitutionen}{Institutionen: Die Zeit, Paul Ollendorff}
         \renewcommand{\erwaehnteOrte}{Orte: Berlin, Dessauer Straße, Italien, Montreux, Paris, Wien}
         \renewcommand{\erwaehnteWerke}{Werke: Der Schleier der Beatrice. Schauspiel in fünf Akten, Die Zeit, Maximes de la vie. Préface par Sully Prud’homme, Schatten des Todes}
               \section[ Paul Goldmann an Arthur Schnitzler, 2. {[}10. 1902{]}]{ Paul Goldmann an Arthur Schnitzler, 2. {[}10. 1902{]}}\nopagebreak\mylabel{v}\rehead{ }\begin{ledgroupsized}[t]{13cm}\normalsize\beginnumbering\briefempfaengerindex{Schnitzler, Arthur@\textsc{Schnitzler, Arthur}!zzzGoldmann, Paul@\emph{von Paul Goldmann}!1902-10-021@{2. {[}10. 1902{]}}|(be} \toendnotes[C]{\smallbreak\pagebreak[2]} \Standort{DLA, A:Schnitzler, HS.NZ85.1.3172.}
\physDesc{Brief, 2 Blätter, 7 Seiten, 2172 Zeichen
\newline{}Handschrift: blaue Tinte, deutsche Kurrent
\newline{}Schnitzler: 1) mit Bleistift das Jahr »902« vermerkt  2) mit rotem Buntstift eine Unterstreichung}\toendnotes[C]{\smallbreak}\pstart
           \noindent{}\raggedleft{}{\pb}\textcolor{gray}{\textbf{DESSAUERSTRASSE 19}}\oindex{Dessauer Strasse@\textbf{Dessauer Straße}|pw}\pend
           \pstart
           Berlin\oindex{Berlin@\textbf{Berlin}|pw}, \label{K_L03223-1v}\edtext{2. Sept.}{\lemma{\textnormal{\emph{2. Sept.}}}\Cendnote{\textnormal{Die Datierung ist offensichtlich falsch, da Goldmann\pwindex{Goldmann, Paul 31.01.1865 – 25.09.1935@\textsc{Goldmann, Paul} (31.01.1865 – 25.09.1935), \emph{Schriftsteller, Journalist}|pwk} am 1. 9. [1902] noch in
                           Montreux\oindex{Montreux@\textbf{Montreux}|pwk} weilte und eine längere
                        Heimreise plante. Goldmann\pwindex{Goldmann, Paul 31.01.1865 – 25.09.1935@\textsc{Goldmann, Paul} (31.01.1865 – 25.09.1935), \emph{Schriftsteller, Journalist}|pwk}s Brief vom
                           6. 10. [1902]
                        reagiert auf Antworten, zu denen die Fragen im vorliegenden Brief gestellt
                        werden. Deshalb ist ein Irrtum um einen Monat anzunehmen.}}}\label{K_L03223-1h}\pend
           \pstart\center{}Mein lieber Freund,\pend\pstart
           Die \label{K_L03223-2v}\edtext{Paß-Angelegenheit}{\lemma{\textnormal{\emph{Paß-Angelegenheit}}}\Cendnote{\textnormal{siehe Paul Goldmann an Olga Gussmann, 29. 9. [1902]}}}\label{K_L03223-2h} hat mich nicht gar ſo viel Zeit gekoſtet, und ich brauche Dir nicht erſt zu
               ſagen, daß es mir eine große Freude macht, meine Zeit auf eine Angelegenheit zu
               verwenden, die Dich (wenn auch nur indirekt) betrifft. Die vierwöchentliche Friſt
               müßt Ihr\pwindex{Steinrueck, Elisabeth 19.11.1885 – 07.04.1920@\textsc{Steinrück, Elisabeth} (19.11.1885 – 07.04.1920)|pwv}\pwindex{Schnitzler, Olga 17.01.1882 – 13.01.1970@\textsc{Schnitzler, Olga} (17.01.1882 – 13.01.1970), \emph{Schauspielerin, Sängerin}|pwv} benutzen,
               um wenigſtens die Ausſtellung eines Interims-Paſſes zu ermöglichen. Sonſt ſtehe ich
                  {\pb}für nichts. Es muß doch noch Rechtsmittel geben,
               um den Kerl\pwindex{Gussmann, Rudolf 05.03.1842 – 24.01.1921@\textsc{Gussmann, Rudolf} (05.03.1842 – 24.01.1921), \emph{Handelsagent}|pwv} zu zwingen.
               Vielleicht iſt, da der Vater\pwindex{Gussmann, Rudolf 05.03.1842 – 24.01.1921@\textsc{Gussmann, Rudolf} (05.03.1842 – 24.01.1921), \emph{Handelsagent}|pwv}
               ſo vollſtändig ſeine Pflichten vernachläſſigt, eine frühere \label{K_L03223-3v}\edtext{Großjährigkeits-Erklärung}{\lemma{\textnormal{\emph{Großjährigkeits-Erklärung}}}\Cendnote{\textnormal{Elisabeth Gussmann\pwindex{Steinrueck, Elisabeth 19.11.1885 – 07.04.1920@\textsc{Steinrück, Elisabeth} (19.11.1885 – 07.04.1920)|pwk} wurde am
                     19. 11. 1885 geboren, stand also kurz vor ihrem 17. Geburtstag.
                  Das Alter für die Volljährigkeit war üblicherweise 21.}}}\label{K_L03223-3h} oder die Beſtellung
               eines Vormunds möglich.\pend
           \pstart
           Die Ausſicht, Dich \label{K_L03223-4v}\edtext{bald hier\oindex{Berlin@\textbf{Berlin}|pwv}}{\lemma{\textnormal{\emph{bald hier}}}\Cendnote{\textnormal{Schnitzler\pwindex{Schnitzler, Arthur 15.05.1862 – 21.10.1931@\textsc{Schnitzler, Arthur} (15.05.1862 – 21.10.1931), \emph{Schriftsteller, Mediziner}|pwk} war von 13. 10. 1902 bis 18. 10. 1902 in Berlin\oindex{Berlin@\textbf{Berlin}|pwk}. Die beiden trafen sich in dieser Zeit
                  täglich.}}}\label{K_L03223-4h} zu ſehen, bereitet mir große Freude. Freilich werde ich von Deinem
               Aufenthalt wenig haben, {\pb}da gerade Mitte Oktober meine Arbeit ins Ungeheure wachſen
               dürfte.\pend
           \pstart
           \textsc{Dr. Hugo Felix\pwindex{Felix, Hugo 19.11.1866 – 25.08.1934@\textsc{Felix, Hugo} (19.11.1866 – 25.08.1934), \emph{Komponist, Chemiker}|pw}} iſt hier\oindex{Berlin@\textbf{Berlin}|pwv} – ein ſehr
               lieber Menſch, der mir ausgezeichnet gefällt. Er hat mich \strikeout{gebete} erſucht, Dich zu bitten, Du möchteſt ihm doch die Erlaubniß geben,
               aus der »\textsc{Beatrice\pwindex{Schnitzler, Arthur 15.05.1862 – 21.10.1931@\textsc{Schnitzler, Arthur} (15.05.1862 – 21.10.1931), \emph{Schriftsteller, Mediziner}!Schleier der Beatrice. Schauspiel in fuenf Akten1900-12-01@\strich\emph{Der Schleier der Beatrice. Schauspiel in fünf Akten} {[}1900-12-01{]}|pw}}«, die er entzückend findet und von der er ſagt, daß ſie ihm herrlich »liegt«,
               für Italien\oindex{Italien@\textbf{Italien}|pw} eine \label{K_L03223-5v}\edtext{Oper}{\lemma{\textnormal{\emph{Oper}}}\Cendnote{\textnormal{Obwohl Schnitzler\pwindex{Schnitzler, Arthur 15.05.1862 – 21.10.1931@\textsc{Schnitzler, Arthur} (15.05.1862 – 21.10.1931), \emph{Schriftsteller, Mediziner}|pwk} wohl zugestimmt hat (vgl. Paul Goldmann an Arthur Schnitzler, 6. 10. [1902]), ist keine
                  entsprechende Oper des Komponisten Felix
                     Hugo\pwindex{Felix, Hugo 19.11.1866 – 25.08.1934@\textsc{Felix, Hugo} (19.11.1866 – 25.08.1934), \emph{Komponist, Chemiker}|pwk} bekannt.}}}\label{K_L03223-5h} zu machen. Er will ſich nicht direkt an {\pb}Dich wenden, weil er\pwindex{Felix, Hugo 19.11.1866 – 25.08.1934@\textsc{Felix, Hugo} (19.11.1866 – 25.08.1934), \emph{Komponist, Chemiker}|pwv} fürchtet, Du würdeſt ihm gegenüber, auch
               wenn Dir der Vorſchlag nicht paßte, mit der Sprache nicht heraus wollen, um ihn nicht
                  \strikeout{k\textcolor{gray}{ra}} zu kränken, und würdeſt Dich ſo gebunden fühlen, ſeine Bitte bejahend zu
               beantworten. Darum hat er mich um meine Vermittelung gebeten, die ich gern übernehme,
               weil ich überzeugt bin, daß Gutes für beide Theile herauskommen würde, wenn die
               Angelegenheit {\pb}ſich arrangiren ließe. Ich bitte um
               eine möglichſt umgehende Antwort, da ich Montag{ }Abend mit \textsc{Felix\pwindex{Felix, Hugo 19.11.1866 – 25.08.1934@\textsc{Felix, Hugo} (19.11.1866 – 25.08.1934), \emph{Komponist, Chemiker}|pw}} zuſammenſein ſoll und ihm einen Beſcheid bringen möchte.\pend
           \pstart
           Ich danke Dir für die Empfehlung der \label{K_L03223-6v}\edtext{Werke\pwindex{Cechov, Anton Pavlovic 1860-01-17 – 1904-07-15@\textsc{Čechov, Anton Pavlovič} (1860-01-17 – 1904-07-15), \emph{Schriftsteller}!Schatten des Todes1902@\strich\emph{Schatten des Todes} {[}1902{]}|pwv} von \textsc{Tschechow\pwindex{Cechov, Anton Pavlovic 1860-01-17 – 1904-07-15@\textsc{Čechov, Anton Pavlovič} (1860-01-17 – 1904-07-15), \emph{Schriftsteller}|pw}}}{\lemma{\textnormal{\emph{Werke von Tschechow}}}\Cendnote{\textnormal{Schnitzler\pwindex{Schnitzler, Arthur 15.05.1862 – 21.10.1931@\textsc{Schnitzler, Arthur} (15.05.1862 – 21.10.1931), \emph{Schriftsteller, Mediziner}|pwk} hatte nachweislich am 26. 8. 1902 die
                  Novelle \emph{Schatten des Todes}\pwindex{Cechov, Anton Pavlovic 1860-01-17 – 1904-07-15@\textsc{Čechov, Anton Pavlovič} (1860-01-17 – 1904-07-15), \emph{Schriftsteller}!Schatten des Todes1902@\strich\emph{Schatten des Todes} {[}1902{]}|pwk} gelesen.}}}\label{K_L03223-6h}.
               Ich entdeckte dieſer Tage ein entzückendes franzöſiſches Aphorismen-{\pb}Buch\pwindex{Beausacq, Marie Suin 1829-10-03 – 1899-12-19@\textsc{Beausacq, Marie Suin} (1829-10-03 – 1899-12-19), \emph{Schriftstellerin}!Maximes de la vie. Preface par Sully PruDhomme1883@\strich\emph{Maximes de la vie. Préface par Sully Prud’homme} {[}1883{]}|pw}{ }\label{K_L03223-7v}\edtext{»\textsc{Maximes de la vie\pwindex{Beausacq, Marie Suin 1829-10-03 – 1899-12-19@\textsc{Beausacq, Marie Suin} (1829-10-03 – 1899-12-19), \emph{Schriftstellerin}!Maximes de la vie. Preface par Sully PruDhomme1883@\strich\emph{Maximes de la vie. Préface par Sully Prud’homme} {[}1883{]}|pw}}« von \textsc{Comtesse Diane\pwindex{Beausacq, Marie Suin 1829-10-03 – 1899-12-19@\textsc{Beausacq, Marie Suin} (1829-10-03 – 1899-12-19), \emph{Schriftstellerin}|pw}}}{\lemma{\textnormal{\emph{»Maximes … Diane}}}\Cendnote{\textnormal{Comtesse Diane\pwindex{Beausacq, Marie Suin 1829-10-03 – 1899-12-19@\textsc{Beausacq, Marie Suin} (1829-10-03 – 1899-12-19), \emph{Schriftstellerin}|pwk} [ = Marie Suin Beausacq\pwindex{Beausacq, Marie Suin 1829-10-03 – 1899-12-19@\textsc{Beausacq, Marie Suin} (1829-10-03 – 1899-12-19), \emph{Schriftstellerin}|pwk}]: \emph{Maximes de la vie. Préface par Sully
                           Prud’homme\pwindex{Prudhomme, Sully 1839-03-16 – 1907-09-07@\textsc{Prudhomme, Sully} (1839-03-16 – 1907-09-07), \emph{Schriftsteller, Librettist, Nobelpreisträger}|pwk}}\pwindex{Beausacq, Marie Suin 1829-10-03 – 1899-12-19@\textsc{Beausacq, Marie Suin} (1829-10-03 – 1899-12-19), \emph{Schriftstellerin}!Maximes de la vie. Preface par Sully PruDhomme1883@\strich\emph{Maximes de la vie. Préface par Sully Prud’homme} {[}1883{]}|pwk}. Paris\oindex{Paris@\textbf{Paris}|pwk}: \emph{P. Ollendorf}\orgindex{Paul Ollendorff@Paul Ollendorff|pwk}{ }1883. Eine Lektüre durch Schnitzler\pwindex{Schnitzler, Arthur 15.05.1862 – 21.10.1931@\textsc{Schnitzler, Arthur} (15.05.1862 – 21.10.1931), \emph{Schriftsteller, Mediziner}|pwk} ist
                  nicht bekannt.}}}\label{K_L03223-7h}. Laß’ es Dich die 8 \textsc{MK} nicht
               reuen, die es koſtet; Du wirſt Freude daran haben.\pend
           \pstart
           Ich hoffe, daß \textsc{Olga\pwindex{Schnitzler, Olga 17.01.1882 – 13.01.1970@\textsc{Schnitzler, Olga} (17.01.1882 – 13.01.1970), \emph{Schauspielerin, Sängerin}|pw}} bald \label{K_L03223-8v}\edtext{wiederhergeſtellt}{\lemma{\textnormal{\emph{wiederhergeſtellt}}}\Cendnote{\textnormal{siehe A. S.: \emph{Tagebuch}, 30. 9. 1902}}}\label{K_L03223-8h} ſein wird\textcolor{gray}{,} bitte, ſie vielmals von mir zu grüßen, {\pb}und begrüße auch Dich auf das Herzlichſte. {\\[\baselineskip]}Dein {\\[\baselineskip]}\spacefill\mbox{Paul Goldmn}\pend
           \leftskip=0em{}\pstart
           \noindent{}Ich würde Dir dankbar ſein, wenn Du mir mittheilen wollteſt, welchen Eindruck die
                     \label{K_L03223-9v}\edtext{»Zeit\orgindex{Zeit@Die Zeit|pw}\pwindex{Zeit1902-09-27 – 1919@\emph{Die Zeit} {[}1902-09-27 – 1919{]}|pw}«}{\lemma{\textnormal{\emph{»Zeit«}}}\Cendnote{\textnormal{siehe Paul Goldmann an Arthur Schnitzler, 16. 9. [1902]}}}\label{K_L03223-9h} auf Dich
                  und überhaupt in Wien\oindex{Wien@\textbf{Wien}|pw} macht?\pend
           
         
         \endnumbering\mylabel{h}\end{ledgroupsized}  \newcommand{\dateiname}{L03223}\newcommand{\titel}{Paul Goldmann an Arthur Schnitzler, 2. [10. 1902]}\newcommand{\editorInnen}{Martin Anton Müller und Laura Untner}%% latex-leseansicht-abspann.tex
%% Abspann für die Leseansicht.
%% Der Schalter \ifkorrekturansicht ist bereits durch den Vorspann gesetzt.

%% latex-abspann.tex
%% Gemeinsamer Abspann für Korrekturansicht und Leseansicht.
%% Setzt den Schalter \ifkorrekturansicht voraus (gesetzt in den
%% einbindenden Dateien latex-korrekturansicht-abspann.tex bzw.
%% latex-leseansicht-abspann.tex).
%% ---------------------------------------------------------------

\normalsize

% Das esempio-Environment wird nur in der Leseansicht benötigt
\ifkorrekturansicht\else
\newenvironment{esempio}[3]%
{
    \vspace{1.5ex}
    \rlap{\underline{#1}}
    \par
    \setlength{\parindent}{0cm}
    \nopagebreak
    \leftskip=#2cm
    \rightskip=#3cm
}
{
    \par
}
\fi

\doendnotes{C}
\bigskip
\vfill

\clearpage

\footnotesize

\ifkorrekturansicht
  \lohead{\textsc{register}}
\fi

% theindex-Environment neu definieren ohne reledmac
\makeatletter
\renewenvironment{theindex}{%
  \ifkorrekturansicht
    \section*{\indexname}%
  \else
    \subsubsection*{Index der erwähnten Entitäten}%
  \fi
  \setlength{\parindent}{0pt}%
  \setlength{\parskip}{0pt plus 0.3pt}%
  \let\item\@idxitem
}{%
  \ifkorrekturansicht\clearpage\fi
}
\makeatother

\IfFileExists{\jobname-pw.ind}{\input{\jobname-pw.ind}}{}

% Quellenangabe nur in der Leseansicht
\ifkorrekturansicht\else
% Fallback-Definitionen, falls die .tex-Datei \titel etc. nicht gesetzt hat
\providecommand{\titel}{}
\providecommand{\editorInnen}{}
\providecommand{\dateiname}{\jobname}

\vspace{3cm}

\vfill

\footnotesize
\textsc{Quelle}: \titel. Herausgegeben von {\editorInnen}. In: \emph{Arthur Schnitzler: Briefwechsel mit Autorinnen und Autoren}.
 Digitale Edition, https://schnitzler-briefe.acdh.oeaw.ac.at/{\dateiname}.html (Stand \today)
\fi

\end{document}


      