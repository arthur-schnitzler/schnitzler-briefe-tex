%% latex-leseansicht-vorspann.tex
%% Vorspann für die Leseansicht.
%% Lädt die gemeinsame Datei latex-vorspann.tex mit nicht gesetztem Schalter.

\newif\ifkorrekturansicht
\korrekturansichtfalse

\input{../tex-inputs/latex-vorspann}


         
         \renewcommand{\erwaehntePersonen}{Personen: Hermann Bahr, Otto Brahm, Samuel Fischer, Olga Schnitzler}
         \renewcommand{\erwaehnteInstitutionen}{Institutionen: Berliner Tageblatt, Niederösterreichische Statthalterei}
         \renewcommand{\erwaehnteOrte}{Orte: Berlin, Wien}
         \renewcommand{\erwaehnteWerke}{Werke: Reigen. Zehn Dialoge}
               \section[Hermann Bahr an Arthur Schnitzler, 16. 12. {[}1903{]}]{ Hermann Bahr an Arthur Schnitzler, 16. 12. {[}1903{]}}\nopagebreak\mylabel{v}\rehead{ }\begin{ledgroupsized}[t]{13cm}\normalsize\beginnumbering\briefempfaengerindex{Schnitzler, Arthur@\textsc{Schnitzler, Arthur}!zzzBahr, Hermann@\emph{von Hermann Bahr}!1903-12-161@{16. 12. {[}1903{]}}|(be} \toendnotes[C]{\smallbreak\pagebreak[2]} \Standort{CUL, Schnitzler, B 5b.}
\physDesc{Brief, 1 Blatt, 2 Seiten, 505 Zeichen
\newline{}Handschrift: schwarze Tinte, deutsche Kurrent
\newline{}Schnitzler: mit Bleistift Jahreszahl ergänzt: »903.« 
\newline{}Ordnung: mit Bleistift von unbekannter Hand nummeriert:
                                    »104« }\buchAbdrucke{\weitereDrucke{Hermann Bahr, Arthur Schnitzler: \emph{Briefwechsel, Aufzeichnungen, Dokumente (1891–1931)}. Hg. Kurt Ifkovits und Martin Anton Müller. Göttingen: \emph{Wallstein} 2018, S. 286.} }\toendnotes[C]{\smallbreak}\pstart
           \raggedleft{}{\pb}16. 12.\pend
           \pstart\center{}Lieber Arthur!\pend\pstart
           Herzlichſten Dank für Dein liebes Telegramm. Und die beſten Grüße von Brahm\pwindex{Brahm, Otto 05.02.1856 – 28.11.1912@\textsc{Brahm, Otto} (05.02.1856 – 28.11.1912), \emph{Theaterleiter, Regisseur}|pw}, Fiſcher\pwindex{Fischer, Samuel 24.12.1859 – 15.10.1934@\textsc{Fischer, Samuel} (24.12.1859 – 15.10.1934), \emph{Verleger}|pw} und allen möglichen Leuten.\pend
           \pstart
           Im Tageblatt\orgindex{Berliner Tageblatt@Berliner Tageblatt|pw} hatte man mir ſchon beinahe
               verſprochen, den Rek\substVorne{}\textsuperscript{ou}\substDazwischen{}urs\substHinten{}\pwindex{Schnitzler, Arthur 15.05.1862 – 21.10.1931@\textsc{Schnitzler, Arthur} (15.05.1862 – 21.10.1931), \emph{Schriftsteller, Mediziner}!Reigen. Zehn Dialoge1900@\strich\emph{Reigen. Zehn Dialoge} {[}1900{]}|pwv} an die Statthalterei\orgindex{Niederoesterreichische Statthalterei@Niederösterreichische Statthalterei|pw} abzudrucken, dann
               haben ſie aber vorgeſtern blos eine einzige Stelle abgedruckt und dies auch noch mit
               ſehr dummen Bemerkungen. Viel geſcheiter ſind ſie ja in Berlin\oindex{Berlin@\textbf{Berlin}|pw} auch nicht als {\pb}bei
               uns, ſondern nur etwas anſtändiger.\pend
           \pstart
           Ich hoffe Dich bald zu ſehen. Mit den beſten Grüßen an Deine Frau\pwindex{Schnitzler, Olga 17.01.1882 – 13.01.1970@\textsc{Schnitzler, Olga} (17.01.1882 – 13.01.1970), \emph{Schauspielerin, Sängerin}|pwv}\pend
           \pstart
           herzlichſt{\\[\baselineskip]}\spacefill\mbox{H.}\pend
           \leftskip=0em{}
         
         \endnumbering\mylabel{h}\end{ledgroupsized}  \newcommand{\dateiname}{L01352}\newcommand{\titel}{Hermann Bahr an Arthur Schnitzler, 16. 12. [1903]}\newcommand{\editorInnen}{ Kurt Ifkovits,  Martin Anton Müller}%% latex-leseansicht-abspann.tex
%% Abspann für die Leseansicht.
%% Der Schalter \ifkorrekturansicht ist bereits durch den Vorspann gesetzt.

%% latex-abspann.tex
%% Gemeinsamer Abspann für Korrekturansicht und Leseansicht.
%% Setzt den Schalter \ifkorrekturansicht voraus (gesetzt in den
%% einbindenden Dateien latex-korrekturansicht-abspann.tex bzw.
%% latex-leseansicht-abspann.tex).
%% ---------------------------------------------------------------

\normalsize

% Das esempio-Environment wird nur in der Leseansicht benötigt
\ifkorrekturansicht\else
\newenvironment{esempio}[3]%
{
    \vspace{1.5ex}
    \rlap{\underline{#1}}
    \par
    \setlength{\parindent}{0cm}
    \nopagebreak
    \leftskip=#2cm
    \rightskip=#3cm
}
{
    \par
}
\fi

\doendnotes{C}
\bigskip
\vfill

\clearpage

\footnotesize

\ifkorrekturansicht
  \lohead{\textsc{register}}
\fi

% theindex-Environment neu definieren ohne reledmac
\makeatletter
\renewenvironment{theindex}{%
  \ifkorrekturansicht
    \section*{\indexname}%
  \else
    \subsubsection*{Index der erwähnten Entitäten}%
  \fi
  \setlength{\parindent}{0pt}%
  \setlength{\parskip}{0pt plus 0.3pt}%
  \let\item\@idxitem
}{%
  \ifkorrekturansicht\clearpage\fi
}
\makeatother

\IfFileExists{\jobname-pw.ind}{\input{\jobname-pw.ind}}{}

% Quellenangabe nur in der Leseansicht
\ifkorrekturansicht\else
% Fallback-Definitionen, falls die .tex-Datei \titel etc. nicht gesetzt hat
\providecommand{\titel}{}
\providecommand{\editorInnen}{}
\providecommand{\dateiname}{\jobname}

\vspace{3cm}

\vfill

\footnotesize
\textsc{Quelle}: \titel. Herausgegeben von {\editorInnen}. In: \emph{Arthur Schnitzler: Briefwechsel mit Autorinnen und Autoren}.
 Digitale Edition, https://schnitzler-briefe.acdh.oeaw.ac.at/{\dateiname}.html (Stand \today)
\fi

\end{document}


      