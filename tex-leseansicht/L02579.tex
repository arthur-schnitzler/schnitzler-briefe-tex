%% latex-korrekturansicht-vorspann.tex
%% Vorspann für die Korrekturansicht.
%% Lädt die gemeinsame Datei latex-vorspann.tex mit gesetztem Schalter.

\newif\ifkorrekturansicht
\korrekturansichttrue

\input{../tex-inputs/latex-vorspann}


\section[Leo Van-Jung, Fanny Mütter, Richard Beer-Hofmann an Arthur und Olga Schnitzler, {[}24. 9.?{]} 1905]{L02579 Leo Van-Jung, Fanny Mütter, Richard Beer-Hofmann an Arthur und Olga
               Schnitzler, {[}24. 9.?{]} 1905}
\nopagebreak\mylabel{L02579v}
\rehead{ }\normalsize\beginnumbering\briefempfaengerindex{Schnitzler, Olga@\textsc{Schnitzler, Olga}!zzzBeer-Hofmann, Richard@\emph{von Richard Beer-Hofmann}!1905-09-241@{{[}24. 9.?{]} 1905}|(be}\briefempfaengerindex{Schnitzler, Olga@\textsc{Schnitzler, Olga}!zzzMuetter, Franziska@\emph{von Franziska Mütter}!1905-09-241@{{[}24. 9.?{]} 1905}|(be}\briefempfaengerindex{Schnitzler, Olga@\textsc{Schnitzler, Olga}!zzzVan-Jung, Leo@\emph{von Leo Van-Jung}!1905-09-241@{{[}24. 9.?{]} 1905}|(be}\briefempfaengerindex{Schnitzler, Arthur@\textsc{Schnitzler, Arthur}!zzzBeer-Hofmann, Richard@\emph{von Richard Beer-Hofmann}!1905-09-241@{{[}24. 9.?{]} 1905}|(be}\briefempfaengerindex{Schnitzler, Arthur@\textsc{Schnitzler, Arthur}!zzzMuetter, Franziska@\emph{von Franziska Mütter}!1905-09-241@{{[}24. 9.?{]} 1905}|(be}\briefempfaengerindex{Schnitzler, Arthur@\textsc{Schnitzler, Arthur}!zzzVan-Jung, Leo@\emph{von Leo Van-Jung}!1905-09-241@{{[}24. 9.?{]} 1905}|(be}
\toendnotes[C]{\smallbreak\pagebreak[2]}\Standort{DLA, A:Schnitzler, 85.1.4821.}
\physDesc{Bildpostkarte, 650 Zeichen
\newline{}Handschrift Leo Van-Jung: Bleistift, lateinische Kurrent
\newline{}Handschrift Franziska Mütter: Bleistift, deutsche Kurrent
\newline{}Handschrift Richard Beer-Hofmann: Bleistift, lateinische Kurrent
\newline{}Versand: Stempel: »\nobreak{}\oindex{Lido@\textbf{Lido}, \emph{P.PPL}|pwk}S. Elisabett{[}a di Lido
                                          (Venezia){]}, 2\textcolor{gray}{4}{[}9{]} 05\nobreak{}«.  }\toendnotes[C]{\smallbreak}\pstart{}{\pb}Herrn D\textsuperscript{r} Arthur
                  Schnitzler\pend{}\pstart{}Wien XVIII. Spöttelgasse 7\oindex{Edmund-Weiss-Gasse 7@\textbf{Edmund-Weiß-Gasse 7}, \emph{Wohngebäude (K.WHS)}|pw}.\pend{}\pstart{}Austria\oindex{Oesterreich@\textbf{Österreich}, \emph{A.PCLI}|pw}\pend{}\pstart{}Vienna\oindex{Wien@\textbf{Wien}, \emph{A.ADM2}|pw}\pend{}{\bigskip}
\pstart
           \noindent{}\centering{}{\pb}\textcolor{gray}{\textbf{VENEZIA\oindex{Venedig@\textbf{Venedig}, \emph{P.PPLA}|pw} – Accademia di Belle Arti\orgindex{Accademia di belle arti di Venezia@Accademia di belle arti di Venezia|pw} – La Presentazione della Vergine\pwindex{presentazione della Vergine al Tempio@\emph{La presentazione della Vergine al Tempio}|pw} – Tiziano\pwindex{Tizian zwischen 1488 und 1490 – 27.08.1576@\textsc{Tizian} (zwischen 1488 und 1490 – 27.08.1576), \emph{Maler/Malerin}|pw}}}\pend
           \vspace{1em}
\pstart
           \noindent{}{\pb}Lieber Arthur,{ }\uline{erst} heute schreib ich Ihnen, aber nicht weil ich an
               Sie vergessen habe, sondern weil ich mich freue Sie bald wieder zu sehen und von den
                  »\label{K_L02579-1v}\edtext{Sünderinnen\pwindex{Zwischenspiel. Komoedie in drei Akten@\emph{Zwischenspiel. Komödie in drei Akten}|pwv}\pwindex{Ruf des Lebens. Schauspiel in drei Akten@\emph{Der Ruf des Lebens. Schauspiel in drei Akten}|pwv}}{\lemma{\textnormal{\emph{Sünderinnen}}}\Cendnote{\textnormal{Es dürfte sich um eine gemeinsame
                  Bezeichnung für die zwei Stücke \emph{Zwischenspiel}\pwindex{Zwischenspiel. Komoedie in drei Akten@\emph{Zwischenspiel. Komödie in drei Akten}|pwk}
                  und \emph{Der Ruf des Lebens}\pwindex{Ruf des Lebens. Schauspiel in drei Akten@\emph{Der Ruf des Lebens. Schauspiel in drei Akten}|pwk} handeln, die, noch
                  ohne finalen Titel, weitgehend fertig gestellt waren, was in mehreren Zeitungen
                  gemeldet worden war. Van-Jung\pwindex{Van-Jung, Leo 15.10.1866 – 02.07.1939@\textsc{Van-Jung, Leo} (15.10.1866 – 02.07.1939), \emph{Gesangspädagoge/Gesangspädagogin, Mathematiker/Mathematikerin}|pwk} kannte sie
                  beide, da Schnitzler sie ihm am 12. 8. 1905 vorgelesen
                  hatte.}}}\label{K_L02579-1}« zu hören. Einige Zeitungsnotizen haben mich sehr neugierig gemacht.
               Handkuss der lieben Frau Olga und die allerherzlichsten Grüsse Ihnen von Ihrem\pend
           \pstart \spacefill\mbox{Leo.}\pend{}\selectlanguage{ngerman}\vspace{1em}
\pstart
           \noindent{}{[}hs. :{]} Lieber Dr. und liebſte Olga! Ich bleibe noch einige Tage hier und
               werde den lieben Brief Olga’s morgen beantworten. Für heute tauſend Grüße von Ihrer
               alten \spacefill\mbox{Fanny Mütter}\pend
           \selectlanguage{ngerman}\vspace{1em}
\pstart
           \noindent{}{[}hs. :{]} Lieber Arthur! Wir sind – hoffe ich \label{K_L02579-2v}\edtext{Mittwoch oder Donnerstag}{\lemma{\textnormal{\emph{Mittwoch oder Donnerstag}}}\Cendnote{\textnormal{Der Poststempel dieser Karte ist nur bei
                  der Jahresangabe verlässlich zu entziffern. Eine grobe Einordnung lässt sich mit
                     Beer-Hofmanns\pwindex{Beer-Hofmann, Richard 1866-07-11 – 1945-09-26@\textsc{Beer-Hofmann, Richard} (1866-07-11 – 1945-09-26), \emph{Schriftsteller/Schriftstellerin}|pwk} Zusammenstellung seiner
                  Lebensdaten treffen: »Ende August, über Bozen\oindex{Bozen@\textbf{Bozen}, \emph{P.PPLA2}|pw} an den Lido\oindex{Lido@\textbf{Lido}, \emph{P.PPL}|pw} (Hôtel des Bains\oindex{Grand Hotel des Bains@\textbf{Grand Hotel des Bains}, \emph{Hotel (K.HTL)}|pw}), Bella Vengerova\pwindex{Vengerova, Isabella 01.03.1877 – 07.02.1956@\textsc{Vengerova, Isabella} (01.03.1877 – 07.02.1956), \emph{Musikpädagoge/Musikpädagogin, Pianist/Pianistin}|pw}, Arthur
                        Kaufmann\pwindex{Kaufmann, Arthur 04.04.1872 – 25.07.1938@\textsc{Kaufmann, Arthur} (04.04.1872 – 25.07.1938), \emph{Rechtswissenschaftler/Rechtswissenschaftlerin, Privatgelehrte/Privatgelehrte, Privatier/Privatière}|pw}, Leo Van Jung\pwindex{Van-Jung, Leo 15.10.1866 – 02.07.1939@\textsc{Van-Jung, Leo} (15.10.1866 – 02.07.1939), \emph{Gesangspädagoge/Gesangspädagogin, Mathematiker/Mathematikerin}|pw} kommen
                     nach.« Die Tagesangabe des Poststempels ist zweistellig und beginnt mit
                  einer »2«, sodass die Karte Ende August oder Ende
                     September anzusiedeln ist. Letzteres wiederum ist wahrscheinlicher, da
                  es bis zum [7. 10. 1905] zu
                  keinem gemeinsamen Treffen kam.}}}\label{K_L02579-2} in Rodaun\oindex{Rodaun@\textbf{Rodaun}, \emph{A.ADM4}|pw}\pend
           
\pstart
           Freuen uns Sie bald zu sehen.\pend
           \pstart Ihr \spacefill\mbox{Richard}\pend{}\selectlanguage{ngerman}\endnumbering\briefempfaengerindex{Schnitzler, Olga@\textsc{Schnitzler, Olga}!zzzBeer-Hofmann, Richard@\emph{von Richard Beer-Hofmann}!1905-09-241@{{[}24. 9.?{]} 1905}|)be}\briefempfaengerindex{Schnitzler, Olga@\textsc{Schnitzler, Olga}!zzzMuetter, Franziska@\emph{von Franziska Mütter}!1905-09-241@{{[}24. 9.?{]} 1905}|)be}\briefempfaengerindex{Schnitzler, Olga@\textsc{Schnitzler, Olga}!zzzVan-Jung, Leo@\emph{von Leo Van-Jung}!1905-09-241@{{[}24. 9.?{]} 1905}|)be}\briefempfaengerindex{Schnitzler, Arthur@\textsc{Schnitzler, Arthur}!zzzBeer-Hofmann, Richard@\emph{von Richard Beer-Hofmann}!1905-09-241@{{[}24. 9.?{]} 1905}|)be}\briefempfaengerindex{Schnitzler, Arthur@\textsc{Schnitzler, Arthur}!zzzMuetter, Franziska@\emph{von Franziska Mütter}!1905-09-241@{{[}24. 9.?{]} 1905}|)be}\briefempfaengerindex{Schnitzler, Arthur@\textsc{Schnitzler, Arthur}!zzzVan-Jung, Leo@\emph{von Leo Van-Jung}!1905-09-241@{{[}24. 9.?{]} 1905}|)be}\mylabel{L02579h}  \normalsize

\doendnotes{C}
\bigskip
\vfill

\clearpage

\footnotesize

\lohead{\textsc{register}}

% Definiere theindex-Environment komplett neu ohne reledmac
\makeatletter
\renewenvironment{theindex}{%
  \section*{\indexname}%
  \setlength{\parindent}{0pt}%
  \setlength{\parskip}{0pt plus 0.3pt}%
  \let\item\@idxitem
}{%
  \clearpage
}
\makeatother

\IfFileExists{\jobname-pw.ind}{\input{\jobname-pw.ind}}{}

\end{document}

      