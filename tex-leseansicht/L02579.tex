%% latex-leseansicht-vorspann.tex
%% Vorspann für die Leseansicht.
%% Lädt die gemeinsame Datei latex-vorspann.tex mit nicht gesetztem Schalter.

\newif\ifkorrekturansicht
\korrekturansichtfalse

\input{../tex-inputs/latex-vorspann}


               \section[Leo Van-Jung, Fanny Mütter, Richard Beer-Hofmann an Arthur und Olga Schnitzler, {[}24. 9.?{]} 1905]{ Leo Van-Jung, Fanny Mütter, Richard Beer-Hofmann an Arthur und Olga
               Schnitzler, {[}24. 9.?{]} 1905}\nopagebreak\mylabel{v}\rehead{ }\begin{ledgroupsized}[t]{13cm}\normalsize\beginnumbering\briefempfaengerindex{Schnitzler, Olga@\textsc{Schnitzler, Olga}!zzzBeer-Hofmann, Richard@\emph{von Richard Beer-Hofmann}!1905-09-241@{{[}24. 9.?{]} 1905}|(be}\briefempfaengerindex{Schnitzler, Olga@\textsc{Schnitzler, Olga}!zzzMuetter, Franziska@\emph{von Franziska Mütter}!1905-09-241@{{[}24. 9.?{]} 1905}|(be}\briefempfaengerindex{Schnitzler, Olga@\textsc{Schnitzler, Olga}!zzzVan-Jung, Leo@\emph{von Leo Van-Jung}!1905-09-241@{{[}24. 9.?{]} 1905}|(be}\briefempfaengerindex{Schnitzler, Arthur@\textsc{Schnitzler, Arthur}!zzzBeer-Hofmann, Richard@\emph{von Richard Beer-Hofmann}!1905-09-241@{{[}24. 9.?{]} 1905}|(be}\briefempfaengerindex{Schnitzler, Arthur@\textsc{Schnitzler, Arthur}!zzzMuetter, Franziska@\emph{von Franziska Mütter}!1905-09-241@{{[}24. 9.?{]} 1905}|(be}\briefempfaengerindex{Schnitzler, Arthur@\textsc{Schnitzler, Arthur}!zzzVan-Jung, Leo@\emph{von Leo Van-Jung}!1905-09-241@{{[}24. 9.?{]} 1905}|(be} \toendnotes[C]{\smallbreak\pagebreak[2]} \Standort{DLA, A:Schnitzler, 85.1.4821.}
\physDesc{Bildpostkarte
\newline{}Handschrift Leo Van-Jung: Bleistift, lateinische Kurrent\newline{}Handschrift Franziska Mütter: Bleistift, deutsche Kurrent\newline{}Handschrift Richard Beer-Hofmann: Bleistift, lateinische Kurrent\newline{}Versand: Stempel: »\nobreak{}\oindex{Lido@\textbf{Lido}|pwk}S. Elisabett{[}a di Lido
                                          (Venezia){]}, 2\textcolor{gray}{4} {[}9{]} 05\nobreak{}«.  }\toendnotes[C]{\smallbreak}\pstart{}{\pb}Herrn D\textsuperscript{r} Arthur
                  Schnitzler\pend{}\pstart{}Wien XVIII. Spöttelgasse 7\oindex{Edmund-Weiss-Gasse@\textbf{Edmund-Weiß-Gasse}|pw}.\pend{}\pstart{}Austria\oindex{Oesterreich@\textbf{Österreich}|pw}\pend{}\pstart{}Vienna\oindex{Wien@\textbf{Wien}|pw}\pend{}{\bigskip}\pstart
           \noindent{}\centering{}{\pb}\textcolor{gray}{\textbf{VENEZIA\oindex{Venedig@\textbf{Venedig}|pw} – Accademia di Belle Arti\orgindex{Accademia di belle arti di Venezia@Accademia di belle arti di Venezia|pw} – La Presentazione della Vergine\pwindex{Tizian zwischen 1488 und 1490 – 27.08.1576@\textsc{Tizian} (zwischen 1488 und 1490 – 27.08.1576), \emph{Maler}!presentazione della Vergine al Tempio1534 – 1538@\strich\emph{La presentazione della Vergine al Tempio} {[}1534 – 1538{]}|pw} – Tiziano\pwindex{Tizian zwischen 1488 und 1490 – 27.08.1576@\textsc{Tizian} (zwischen 1488 und 1490 – 27.08.1576), \emph{Maler}|pw}}}\pend
           \pstart
           Lieber Arthur, \uline{erst} heute schreib ich Ihnen, aber nicht weil ich an
               Sie vergessen habe, sondern weil ich mich freue Sie bald wieder zu sehen und von den
                  »\label{K_L02579-1v}\edtext{Sünderinnen\pwindex{Schnitzler, Arthur 15.05.1862 – 21.10.1931@\textsc{Schnitzler, Arthur} (15.05.1862 – 21.10.1931), \emph{Schriftsteller, Mediziner}!Zwischenspiel. Komoedie in drei Akten1905-10-12 – 1905-10-12@\strich\emph{Zwischenspiel. Komödie in drei Akten} {[}1905-10-12 – 1905-10-12{]}|pwv}\pwindex{Schnitzler, Arthur 15.05.1862 – 21.10.1931@\textsc{Schnitzler, Arthur} (15.05.1862 – 21.10.1931), \emph{Schriftsteller, Mediziner}!Ruf des Lebens. Schauspiel in drei Akten1906-02-20@\strich\emph{Der Ruf des Lebens. Schauspiel in drei Akten} {[}1906-02-20{]}|pwv}}{\lemma{\textnormal{\emph{Sünderinnen}}}\Cendnote{\textnormal{Es dürfte sich um eine gemeinsame
                  Bezeichnung für die zwei Stücke \emph{Zwischenspiel}\pwindex{Schnitzler, Arthur 15.05.1862 – 21.10.1931@\textsc{Schnitzler, Arthur} (15.05.1862 – 21.10.1931), \emph{Schriftsteller, Mediziner}!Zwischenspiel. Komoedie in drei Akten1905-10-12 – 1905-10-12@\strich\emph{Zwischenspiel. Komödie in drei Akten} {[}1905-10-12 – 1905-10-12{]}|pwk}
                  und \emph{Der Ruf des Lebens}\pwindex{Schnitzler, Arthur 15.05.1862 – 21.10.1931@\textsc{Schnitzler, Arthur} (15.05.1862 – 21.10.1931), \emph{Schriftsteller, Mediziner}!Ruf des Lebens. Schauspiel in drei Akten1906-02-20@\strich\emph{Der Ruf des Lebens. Schauspiel in drei Akten} {[}1906-02-20{]}|pwk} handeln, die, noch ohne
                  finalen Titel weitgehend fertig gestellt waren, was in mehreren Zeitungen gemeldet
                  worden war. Van-Jung\pwindex{Van-Jung, Leo 15.10.1866 – 02.07.1939@\textsc{Van-Jung, Leo} (15.10.1866 – 02.07.1939), \emph{Gesangspädagoge, Mathematiker}|pwk} kannte sie beide, da Schnitzler\pwindex{Schnitzler, Arthur 15.05.1862 – 21.10.1931@\textsc{Schnitzler, Arthur} (15.05.1862 – 21.10.1931), \emph{Schriftsteller, Mediziner}|pwk} sie ihm am 12. 8. 1905 vorgelesen
                  hatte.}}}\label{K_L02579-1h}« zu hören. Einige Zeitungsnotizen haben mich sehr neugierig gemacht.
               Handkuss der lieben Frau Olga und die allerherzlichsten Grüsse Ihnen von Ihrem\pend
           \pstart \spacefill\mbox{Leo.}\pend{}\pstart
           \noindent{}{[}hs. Mütter:{]} Lieber Dr. und liebſte Olga! Ich bleibe noch einige Tage hier und
               werde den lieben Brief Olga’s morgen beantworten. Für heute tauſend Grüße von Ihrer
               alten \spacefill\mbox{Fanny Mütter}\pend
           \pstart
           \noindent{}{[}hs. Beer-Hofmann:{]} Lieber Arthur! Wir sind – hoffe ich \label{K_L02579-2v}\edtext{Mittwoch oder Donnerstag}{\lemma{\textnormal{\emph{Mittwoch oder Donnerstag}}}\Cendnote{\textnormal{Der Poststempel dieser Karte ist nur bei
                  der Jahresangabe verlässlich zu entziffern. Eine grobe Einordnung lässt sich mit
                     Beer-Hofmann\pwindex{Beer-Hofmann, Richard 11.07.1866 – 26.09.1945@\textsc{Beer-Hofmann, Richard} (11.07.1866 – 26.09.1945), \emph{Schriftsteller}|pwk}s Zusammentstellung seiner
                  Lebensdaten treffen: »Ende August, über Bozen\oindex{Bozen@\textbf{Bozen}|pw} an den Lido\oindex{Lido@\textbf{Lido}|pw} (Hôtel des Bains\oindex{Grand Hotel des Bains@\textbf{Grand Hotel des Bains}|pw}), Bella
                        Vengerova\pwindex{Vengerova, Isabella 01.03.1877 – 07.02.1956@\textsc{Vengerova, Isabella} (01.03.1877 – 07.02.1956), \emph{Musikpädagogin, Pianistin}|pw}, Arthur Kaufmann\pwindex{Kaufmann, Arthur 04.04.1872 – 25.07.1938@\textsc{Kaufmann, Arthur} (04.04.1872 – 25.07.1938), \emph{Rechtswissenschaftler, Privatgelehrte, Privatier}|pw}, Leo Van Jung\pwindex{Van-Jung, Leo 15.10.1866 – 02.07.1939@\textsc{Van-Jung, Leo} (15.10.1866 – 02.07.1939), \emph{Gesangspädagoge, Mathematiker}|pw} kommen nach.« Die
                  Tagesangabe des Poststempels ist zweistellig und beginnt mit einer »2«, so dass
                  die Karte Ende August oder Ende September anzusiedeln
                  ist. Letzteres wiederum ist wahrscheinlicher, da es bis zum [7. 10. 1905] zu keinem gemeinsamen Treffen kam.}}}\label{K_L02579-2h}
               in Rodaun\oindex{Rodaun@\textbf{Rodaun}|pw}\pend
           \pstart
           Freuen uns Sie bald zu sehen.\pend
           \pstart Ihr \spacefill\mbox{Richard}\pend{}\endnumbering\briefempfaengerindex{Schnitzler, Olga@\textsc{Schnitzler, Olga}!zzzBeer-Hofmann, Richard@\emph{von Richard Beer-Hofmann}!1905-09-241@{{[}24. 9.?{]} 1905}|)be}\briefempfaengerindex{Schnitzler, Olga@\textsc{Schnitzler, Olga}!zzzMuetter, Franziska@\emph{von Franziska Mütter}!1905-09-241@{{[}24. 9.?{]} 1905}|)be}\briefempfaengerindex{Schnitzler, Olga@\textsc{Schnitzler, Olga}!zzzVan-Jung, Leo@\emph{von Leo Van-Jung}!1905-09-241@{{[}24. 9.?{]} 1905}|)be}\briefempfaengerindex{Schnitzler, Arthur@\textsc{Schnitzler, Arthur}!zzzBeer-Hofmann, Richard@\emph{von Richard Beer-Hofmann}!1905-09-241@{{[}24. 9.?{]} 1905}|)be}\briefempfaengerindex{Schnitzler, Arthur@\textsc{Schnitzler, Arthur}!zzzMuetter, Franziska@\emph{von Franziska Mütter}!1905-09-241@{{[}24. 9.?{]} 1905}|)be}\briefempfaengerindex{Schnitzler, Arthur@\textsc{Schnitzler, Arthur}!zzzVan-Jung, Leo@\emph{von Leo Van-Jung}!1905-09-241@{{[}24. 9.?{]} 1905}|)be}\mylabel{h}\end{ledgroupsized}  \newcommand{\dateiname}{L02579}\newcommand{\titel}{Leo Van-Jung, Fanny Mütter, Richard Beer-Hofmann an Arthur und Olga Schnitzler, [24. 9.?] 1905}\newcommand{\editorInnen}{Martin Anton Müller und Gerd-Hermann Susen}%% latex-leseansicht-abspann.tex
%% Abspann für die Leseansicht.
%% Der Schalter \ifkorrekturansicht ist bereits durch den Vorspann gesetzt.

%% latex-abspann.tex
%% Gemeinsamer Abspann für Korrekturansicht und Leseansicht.
%% Setzt den Schalter \ifkorrekturansicht voraus (gesetzt in den
%% einbindenden Dateien latex-korrekturansicht-abspann.tex bzw.
%% latex-leseansicht-abspann.tex).
%% ---------------------------------------------------------------

\normalsize

% Das esempio-Environment wird nur in der Leseansicht benötigt
\ifkorrekturansicht\else
\newenvironment{esempio}[3]%
{
    \vspace{1.5ex}
    \rlap{\underline{#1}}
    \par
    \setlength{\parindent}{0cm}
    \nopagebreak
    \leftskip=#2cm
    \rightskip=#3cm
}
{
    \par
}
\fi

\doendnotes{C}
\bigskip
\vfill

\clearpage

\footnotesize

\ifkorrekturansicht
  \lohead{\textsc{register}}
\fi

% theindex-Environment neu definieren ohne reledmac
\makeatletter
\renewenvironment{theindex}{%
  \ifkorrekturansicht
    \section*{\indexname}%
  \else
    \subsubsection*{Index der erwähnten Entitäten}%
  \fi
  \setlength{\parindent}{0pt}%
  \setlength{\parskip}{0pt plus 0.3pt}%
  \let\item\@idxitem
}{%
  \ifkorrekturansicht\clearpage\fi
}
\makeatother

\IfFileExists{\jobname-pw.ind}{\input{\jobname-pw.ind}}{}

% Quellenangabe nur in der Leseansicht
\ifkorrekturansicht\else
% Fallback-Definitionen, falls die .tex-Datei \titel etc. nicht gesetzt hat
\providecommand{\titel}{}
\providecommand{\editorInnen}{}
\providecommand{\dateiname}{\jobname}

\vspace{3cm}

\vfill

\footnotesize
\textsc{Quelle}: \titel. Herausgegeben von {\editorInnen}. In: \emph{Arthur Schnitzler: Briefwechsel mit Autorinnen und Autoren}.
 Digitale Edition, https://schnitzler-briefe.acdh.oeaw.ac.at/{\dateiname}.html (Stand \today)
\fi

\end{document}


      