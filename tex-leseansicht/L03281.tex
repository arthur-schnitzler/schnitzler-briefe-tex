%% latex-leseansicht-vorspann.tex
%% Vorspann für die Leseansicht.
%% Lädt die gemeinsame Datei latex-vorspann.tex mit nicht gesetztem Schalter.

\newif\ifkorrekturansicht
\korrekturansichtfalse

\input{../tex-inputs/latex-vorspann}


\section[ Felix Salten an Arthur Schnitzler, 5. 9. {[}1898{]}]{L03281 Felix Salten an Arthur Schnitzler,  5. 9. [1898]}
\nopagebreak\mylabel{L03281v}
\rehead{ }\normalsize\beginnumbering\briefempfaengerindex{Schnitzler, Arthur@\textsc{Schnitzler, Arthur}!zzzSalten, Felix@\emph{von Felix Salten}!1898-09-052@{5. 9. [1898]}|(be}
\toendnotes[C]{\smallbreak\pagebreak[2]}
\correspDesc{Versand  durch Felix Salten am 5. 9. [1898] in Wien
\newline{}Erhalt  durch Arthur Schnitzler im Zeitraum [5. 9. 1898
                  – 7. 9. 1898?] in Wien}\toendnotes[C]{\smallbreak}
\Standort{CUL, Schnitzler, B 89, A 2.}
\physDesc{Brief, 1 Blatt, 1 Seite, 422 Zeichen
\newline{}Handschrift: Bleistift, lateinische Kurrent
\newline{}Ordnung: mit Bleistift von unbekannter Hand nummeriert: »105« }\toendnotes[C]{\smallbreak}
\pstart
           \raggedleft{}{\pb}Hietzing, Wattmanngaße 11\oindex{Wien@\textbf{Wien}!XIII., Hietzing@\textbf{XIII., Hietzing}!Wattmanngasse@\textbf{Wattmanngasse}, \emph{Straße}|pw}{\\}5. Septemb.\pend
           \vspace{0.5em}
\pstart
           Lieber Arthur, ich war die ganze Zeit, vom 4. August bis zum 28., \label{K_L03281-1v}\edtext{fort}{\lemma{\textnormal{\emph{fort}}}\Cendnote{\textnormal{Siehe XXXX Auszeichnungsfehler: Dokument L03280 nicht gefunden.
               }}}\label{K_L03281-1}. Theils in Ungarn\oindex{Ungarn@\textbf{Ungarn}|pw}, theils Reichenhall\oindex{Bad Reichenhall@\textbf{Bad Reichenhall}, \emph{Region}|pw}, und bekam nichts nachgesendet. Am
                  28\textsuperscript{ten} aber war es auch für Ihre \label{K_L03281-2v}\edtext{Genf\oindex{Genf@\textbf{Genf}|pw}er Adreße}{\lemma{\textnormal{\emph{Genfer Adreße}}}\Cendnote{\textnormal{Schnitzlers Aufenthalt in Genf\oindex{Genf@\textbf{Genf}|pwk} dauerte vom 16. 8. 1898 bis zum 18. 8. 1898.}}}\label{K_L03281-2}
               schon zu spät. Also entschuldigen Sie, dass ich nichts hören ließ, und erst heute für Ihre \label{K_L03281-3v}\edtext{lieben Karten}{\lemma{\textnormal{\emph{lieben Karten}}}\Cendnote{\textnormal{nicht überliefert}}}\label{K_L03281-3} danke. Wenn Sie \label{K_L03281-4v}\edtext{schon in Wien\oindex{Wien@\textbf{Wien}, \emph{Verwaltungsgebiet}|pw}}{\lemma{\textnormal{\emph{schon in Wien}}}\Cendnote{\textnormal{Schnitzler war am 3. 9. 1898 nach Wien\oindex{Wien@\textbf{Wien}, \emph{Verwaltungsgebiet}|pwk} zurückgekehrt.}}}\label{K_L03281-4} sind, senden Sie mir
               eine Zeile, wann wir uns sehen können.\pend
           
\pstart
           herzlichst Ihr {\\[\baselineskip]}\spacefill\mbox{Salten}\pend
           \leftskip=0em{}\selectlanguage{ngerman}\endnumbering\briefempfaengerindex{Schnitzler, Arthur@\textsc{Schnitzler, Arthur}!zzzSalten, Felix@\emph{von Felix Salten}!1898-09-052@{5. 9. [1898]}|)be}\mylabel{L03281h}  \newcommand{\dateiname}{L03281}\newcommand{\titel}{Felix Salten an Arthur Schnitzler, 5. 9. [1898]}\newcommand{\editorInnen}{Martin Anton Müller und Laura Untner}%% latex-leseansicht-abspann.tex
%% Abspann für die Leseansicht.
%% Der Schalter \ifkorrekturansicht ist bereits durch den Vorspann gesetzt.

%% latex-abspann.tex
%% Gemeinsamer Abspann für Korrekturansicht und Leseansicht.
%% Setzt den Schalter \ifkorrekturansicht voraus (gesetzt in den
%% einbindenden Dateien latex-korrekturansicht-abspann.tex bzw.
%% latex-leseansicht-abspann.tex).
%% ---------------------------------------------------------------

\normalsize

% Das esempio-Environment wird nur in der Leseansicht benötigt
\ifkorrekturansicht\else
\newenvironment{esempio}[3]%
{
    \vspace{1.5ex}
    \rlap{\underline{#1}}
    \par
    \setlength{\parindent}{0cm}
    \nopagebreak
    \leftskip=#2cm
    \rightskip=#3cm
}
{
    \par
}
\fi

\doendnotes{C}
\bigskip
\vfill

\clearpage

\footnotesize

\ifkorrekturansicht
  \lohead{\textsc{register}}
\fi

% theindex-Environment neu definieren ohne reledmac
\makeatletter
\renewenvironment{theindex}{%
  \ifkorrekturansicht
    \section*{\indexname}%
  \else
    \subsubsection*{Index der erwähnten Entitäten}%
  \fi
  \setlength{\parindent}{0pt}%
  \setlength{\parskip}{0pt plus 0.3pt}%
  \let\item\@idxitem
}{%
  \ifkorrekturansicht\clearpage\fi
}
\makeatother

\IfFileExists{\jobname-pw.ind}{\input{\jobname-pw.ind}}{}

% Quellenangabe nur in der Leseansicht
\ifkorrekturansicht\else
% Fallback-Definitionen, falls die .tex-Datei \titel etc. nicht gesetzt hat
\providecommand{\titel}{}
\providecommand{\editorInnen}{}
\providecommand{\dateiname}{\jobname}

\vspace{3cm}

\vfill

\footnotesize
\textsc{Quelle}: \titel. Herausgegeben von {\editorInnen}. In: \emph{Arthur Schnitzler: Briefwechsel mit Autorinnen und Autoren}.
 Digitale Edition, https://schnitzler-briefe.acdh.oeaw.ac.at/{\dateiname}.html (Stand \today)
\fi

\end{document}


