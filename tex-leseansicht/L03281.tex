%% latex-korrekturansicht-vorspann.tex
%% Vorspann für die Korrekturansicht.
%% Lädt die gemeinsame Datei latex-vorspann.tex mit gesetztem Schalter.

\newif\ifkorrekturansicht
\korrekturansichttrue

\input{../tex-inputs/latex-vorspann}


\section[ Felix Salten an Arthur Schnitzler, 5. 9. {[}1898{]}]{L03281 Felix Salten an Arthur Schnitzler, 5. 9. {[}1898{]}}
\nopagebreak\mylabel{L03281v}
\rehead{ }\normalsize\beginnumbering\briefempfaengerindex{Schnitzler, Arthur@\textsc{Schnitzler, Arthur}!zzzSalten, Felix@\emph{von Felix Salten}!1898-09-052@{5. 9. {[}1898{]}}|(be}
\toendnotes[C]{\smallbreak\pagebreak[2]}\Standort{CUL, Schnitzler, B 89, A 2.}
\physDesc{Brief, 1 Blatt, 1 Seite, 422 Zeichen
\newline{}Handschrift: Bleistift, lateinische Kurrent
\newline{}Ordnung: mit Bleistift von unbekannter Hand nummeriert: »105« }\toendnotes[C]{\smallbreak}
\pstart
           \raggedleft{}{\pb}Hietzing, Wattmanngaße 11\oindex{Wattmanngasse@\textbf{Wattmanngasse}, \emph{Straße (K.STR)}|pw}{\\}5. Septemb.\pend
           \vspace{0.5em}
\pstart
           Lieber Arthur, ich war die ganze Zeit, vom 4. August bis zum 28., \label{K_L03281-1v}\edtext{fort}{\lemma{\textnormal{\emph{fort}}}\Cendnote{\textnormal{Siehe Felix Salten an Arthur Schnitzler, 30. 7. 1898.
               }}}\label{K_L03281-1}. Theils in Ungarn\oindex{Ungarn@\textbf{Ungarn}, \emph{A.PCLI}|pw}, theils Reichenhall\oindex{Bad Reichenhall@\textbf{Bad Reichenhall}, \emph{A.ADM4}|pw}, und bekam nichts nachgesendet. Am
                  28\textsuperscript{ten} aber war es auch für Ihre \label{K_L03281-2v}\edtext{Genf\oindex{Genf@\textbf{Genf}, \emph{P.PPLA}|pw}er Adreße}{\lemma{\textnormal{\emph{Genfer Adreße}}}\Cendnote{\textnormal{Schnitzlers Aufenthalt in Genf\oindex{Genf@\textbf{Genf}, \emph{P.PPLA}|pwk} dauerte vom 16. 8. 1898 bis zum 18. 8. 1898.}}}\label{K_L03281-2}
               schon zu spät. Also entschuldigen Sie, dass ich nichts hören ließ, und erst heute für Ihre \label{K_L03281-3v}\edtext{lieben Karten}{\lemma{\textnormal{\emph{lieben Karten}}}\Cendnote{\textnormal{nicht überliefert}}}\label{K_L03281-3} danke. Wenn Sie \label{K_L03281-4v}\edtext{schon in Wien\oindex{Wien@\textbf{Wien}, \emph{A.ADM2}|pw}}{\lemma{\textnormal{\emph{schon in Wien}}}\Cendnote{\textnormal{Schnitzler war am 3. 9. 1898 nach Wien\oindex{Wien@\textbf{Wien}, \emph{A.ADM2}|pwk} zurückgekehrt.}}}\label{K_L03281-4} sind, senden Sie mir
               eine Zeile, wann wir uns sehen können. \pend
           
\pstart
           herzlichst Ihr {\\[\baselineskip]}\spacefill\mbox{Salten}\pend
           \leftskip=0em{}\selectlanguage{ngerman}\endnumbering\briefempfaengerindex{Schnitzler, Arthur@\textsc{Schnitzler, Arthur}!zzzSalten, Felix@\emph{von Felix Salten}!1898-09-052@{5. 9. {[}1898{]}}|)be}\mylabel{L03281h}  \normalsize

\doendnotes{C}
\bigskip
\vfill

\clearpage

\footnotesize

\lohead{\textsc{register}}

% Definiere theindex-Environment komplett neu ohne reledmac
\makeatletter
\renewenvironment{theindex}{%
  \section*{\indexname}%
  \setlength{\parindent}{0pt}%
  \setlength{\parskip}{0pt plus 0.3pt}%
  \let\item\@idxitem
}{%
  \clearpage
}
\makeatother

\IfFileExists{\jobname-pw.ind}{\input{\jobname-pw.ind}}{}

\end{document}

      