%% latex-leseansicht-vorspann.tex
%% Vorspann für die Leseansicht.
%% Lädt die gemeinsame Datei latex-vorspann.tex mit nicht gesetztem Schalter.

\newif\ifkorrekturansicht
\korrekturansichtfalse

\input{../tex-inputs/latex-vorspann}


\section[Elsa Plessner an Arthur Schnitzler, {{[}}Mitte April 1897{{]}}]{L03694 Elsa Plessner an Arthur Schnitzler, {[}Mitte April 1897{]}}
\nopagebreak\mylabel{L03694v}
\rehead{ }\normalsize\beginnumbering\briefempfaengerindex{Schnitzler, Arthur@\textsc{Schnitzler, Arthur}!zzzPlessner, Elsa@\emph{von Elsa Plessner}!1897-04-212@{{[}Mitte April 1897{]}}|(be}
\toendnotes[C]{\smallbreak\pagebreak[2]}
\correspDesc{Versand  durch Elsa Plessner im Zeitraum [Mitte April 1897] in Wien
\newline{}Erhalt  durch Arthur Schnitzler am 22. 4. 1897 in Paris}\toendnotes[C]{\smallbreak}
\Standort{DLA, A:Schnitzler, 85.1.4198.}
\physDesc{Brief, 1 Blatt, 2 Seiten, 772 Zeichen
\newline{}Handschrift: schwarze Tinte, lateinische Kurrent
\newline{}Schnitzler: 1) mit Bleistift den Empfang datiert: »\substVorne{}\textsuperscript{1}\substDazwischen{}2\substHinten{}2/4 97«  2) mit rotem Buntstift eine Unterstreichung}
\buchAbdrucke{\weitereDrucke{Hermann Bahr, Arthur Schnitzler: \emph{Briefwechsel, Aufzeichnungen, Dokumente (1891–1931)}. Herausgegeben von Kurt Ifkovits und Martin Anton Müller. Göttingen: \emph{Wallstein} 2018, S. 141.} }\toendnotes[C]{\smallbreak}
\pstart
           {\pb}Wien VIII. Florianigasse N\textsuperscript{o} 44\oindex{Wien@\textbf{Wien}!VIII., Josefstadt@\textbf{VIII., Josefstadt}!Florianigasse 44@\textbf{Florianigasse 44}, \emph{Wohngebäude}|pw}.\pend
           
\pstart{}Hochverehrter Herr Doctor!\pend\vspace{0.5em}
\pstart
           Schon wieder einmal komme ich Sie um etwas zu bitten!! – Aber Sie sind ja immer so
               gut. Also die Sache ist die, dass ich bei Herrn H.
                  Bahr\pwindex{Bahr, Hermann 19.\,7.\,1863 Linz – 15.\,1.\,1934 München@\textsc{Bahr, Hermann} (19.\,7.\,1863 Linz – 15.\,1.\,1934 München), \emph{Schriftsteller, Kritiker}|pw} die Novelle, die Sie »Warten\pwindex{Plessner, Elsa 22.\,8.\,1875 Wien – 7.\,5.\,1932 Alicante@\textsc{Plessner, Elsa} (22.\,8.\,1875 Wien – 7.\,5.\,1932 Alicante), \emph{Schriftstellerin}!Warten. Novelle@\strich\emph{Warten. Novelle}|pw}«
               getauft haben, (bei mir hieß sie zuerst »Blätter«) – an die Sie sich hoffentlich
                  \label{K_L03694-1v}\edtext{noch erinnern}{\lemma{\textnormal{\emph{noch erinnern}}}\Cendnote{\textnormal{Plessner\pwindex{Plessner, Elsa 22.\,8.\,1875 Wien – 7.\,5.\,1932 Alicante@\textsc{Plessner, Elsa} (22.\,8.\,1875 Wien – 7.\,5.\,1932 Alicante), \emph{Schriftstellerin}|pwk} hatte die Erzählung\pwindex{Plessner, Elsa 22.\,8.\,1875 Wien – 7.\,5.\,1932 Alicante@\textsc{Plessner, Elsa} (22.\,8.\,1875 Wien – 7.\,5.\,1932 Alicante), \emph{Schriftstellerin}!Warten. Novelle@\strich\emph{Warten. Novelle}|pwkv} am XXXX Auszeichnungsfehler: Dokument L03700 nicht gefunden{ }Schnitzler in einer ersten Fassung zugesandt
                  und am XXXX Auszeichnungsfehler: Dokument L03702 nicht gefunden in
                  einem Paket mit weiteren Texten erneut. Er äußerte seine Zustimmung zu dem Text\pwindex{Plessner, Elsa 22.\,8.\,1875 Wien – 7.\,5.\,1932 Alicante@\textsc{Plessner, Elsa} (22.\,8.\,1875 Wien – 7.\,5.\,1932 Alicante), \emph{Schriftstellerin}!Warten. Novelle@\strich\emph{Warten. Novelle}|pwkv}, vgl. XXXX Auszeichnungsfehler: Dokument L03703 nicht gefunden.}}}\label{K_L03694-1} – für »die Zeit\orgindex{Zeit. Wiener Wochenschrift@Die Zeit. Wiener Wochenschrift|pw}« eingereicht habe, und dass ich Sie nun
               herzlichst bitte, ein – (oder zwei?) \label{K_L03694-2v}\edtext{gute Worte}{\lemma{\textnormal{\emph{gute Worte}}}\Cendnote{\textnormal{Siehe XXXX Auszeichnungsfehler: Dokument L00668 nicht gefunden. }}}\label{K_L03694-2} für mich und
               sie bei genanntem Herrn einzulegen.\pend
           
\pstart
           {\pb}Ich traue mich diesbezüglich nur deshalb an Sie heran,
               weil Ihnen die Arbeit\pwindex{Plessner, Elsa 22.\,8.\,1875 Wien – 7.\,5.\,1932 Alicante@\textsc{Plessner, Elsa} (22.\,8.\,1875 Wien – 7.\,5.\,1932 Alicante), \emph{Schriftstellerin}!Warten. Novelle@\strich\emph{Warten. Novelle}|pwv}
               seinerzeit gefiel. Aber – Sie wissen ja, wie das ist, – ein empfehlendes Wort
               Ihrerseits ist doch zehnmal gewichtiger als die beste Arbeit einer \label{K_L03694-3v}\edtext{\begin{otherlanguage}{french}obscurité\end{otherlanguage}}{\lemma{\textnormal{\emph{obscurité}}}\Cendnote{\textnormal{französisch, sinngemäß:
                  Unerkannten}}}\label{K_L03694-3}. – Also – besten herzlichsten Dank im voraus!\pend
           
\pstart
           In steter Verehrung{\\[\baselineskip]}\spacefill\mbox{Elsa Plessner}\pend
           \leftskip=0em{}\selectlanguage{ngerman}\endnumbering\briefempfaengerindex{Schnitzler, Arthur@\textsc{Schnitzler, Arthur}!zzzPlessner, Elsa@\emph{von Elsa Plessner}!1897-04-152@{{[}Mitte April 1897{]}}|)be}\mylabel{L03694h}  \newcommand{\dateiname}{L03694}\newcommand{\titel}{Elsa Plessner an Arthur Schnitzler, [Mitte April 1897]}\newcommand{\editorInnen}{Kurt Ifkovits, Selma Jahnke und Martin Anton Müller}%% latex-leseansicht-abspann.tex
%% Abspann für die Leseansicht.
%% Der Schalter \ifkorrekturansicht ist bereits durch den Vorspann gesetzt.

%% latex-abspann.tex
%% Gemeinsamer Abspann für Korrekturansicht und Leseansicht.
%% Setzt den Schalter \ifkorrekturansicht voraus (gesetzt in den
%% einbindenden Dateien latex-korrekturansicht-abspann.tex bzw.
%% latex-leseansicht-abspann.tex).
%% ---------------------------------------------------------------

\normalsize

% Das esempio-Environment wird nur in der Leseansicht benötigt
\ifkorrekturansicht\else
\newenvironment{esempio}[3]%
{
    \vspace{1.5ex}
    \rlap{\underline{#1}}
    \par
    \setlength{\parindent}{0cm}
    \nopagebreak
    \leftskip=#2cm
    \rightskip=#3cm
}
{
    \par
}
\fi

\doendnotes{C}
\bigskip
\vfill

\clearpage

\footnotesize

\ifkorrekturansicht
  \lohead{\textsc{register}}
\fi

% theindex-Environment neu definieren ohne reledmac
\makeatletter
\renewenvironment{theindex}{%
  \ifkorrekturansicht
    \section*{\indexname}%
  \else
    \subsubsection*{Index der erwähnten Entitäten}%
  \fi
  \setlength{\parindent}{0pt}%
  \setlength{\parskip}{0pt plus 0.3pt}%
  \let\item\@idxitem
}{%
  \ifkorrekturansicht\clearpage\fi
}
\makeatother

\IfFileExists{\jobname-pw.ind}{\input{\jobname-pw.ind}}{}

% Quellenangabe nur in der Leseansicht
\ifkorrekturansicht\else
% Fallback-Definitionen, falls die .tex-Datei \titel etc. nicht gesetzt hat
\providecommand{\titel}{}
\providecommand{\editorInnen}{}
\providecommand{\dateiname}{\jobname}

\vspace{3cm}

\vfill

\footnotesize
\textsc{Quelle}: \titel. Herausgegeben von {\editorInnen}. In: \emph{Arthur Schnitzler: Briefwechsel mit Autorinnen und Autoren}.
 Digitale Edition, https://schnitzler-briefe.acdh.oeaw.ac.at/{\dateiname}.html (Stand \today)
\fi

\end{document}


