%% latex-korrekturansicht-vorspann.tex
%% Vorspann für die Korrekturansicht.
%% Lädt die gemeinsame Datei latex-vorspann.tex mit gesetztem Schalter.

\newif\ifkorrekturansicht
\korrekturansichttrue

\input{../tex-inputs/latex-vorspann}


\section[Elsa Plessner an Arthur Schnitzler, {[}Mitte April 1897{]}]{L03694 Elsa Plessner an Arthur Schnitzler, {[}Mitte April 1897{]}}
\nopagebreak\mylabel{L03694v}
\rehead{ }\normalsize\beginnumbering\briefempfaengerindex{Schnitzler, Arthur@\textsc{Schnitzler, Arthur}!zzzPlessner, Elsa@\emph{von Elsa Plessner}!1897-04-212@{{[}Mitte April 1897{]}}|(be}
\toendnotes[C]{\smallbreak\pagebreak[2]}\Standort{DLA, A:Schnitzler, 85.1.4198.}
\physDesc{Brief, 1 Blatt, 2 Seiten, 771 Zeichen
\newline{}Handschrift: , lateinische Kurrent
\newline{}Schnitzler: mit Bleistift datiert: »22/4 97« }
\buchAbdrucke{\weitereDrucke{Hermann Bahr, Arthur Schnitzler: \emph{Briefwechsel, Aufzeichnungen, Dokumente (1891–1931)}. Göttingen: \emph{Wallstein} 2018, S. 141.} }\toendnotes[C]{\smallbreak}
\pstart
           \raggedleft{}{\pb}Wien VIII. Florianigasse N\textsuperscript{o} 44\oindex{Florianigasse 44@\textbf{Florianigasse 44}, \emph{Wohngebäude (K.WHS)}|pw}.\pend
           
\pstart{}Hochverehrter Herr Doctor!\pend\vspace{0.5em}
\pstart
           Schon wieder einmal komme ich Sie um etwas zu bitten!!. Aber Sie sind ja immer so
               gut. Also die Sache ist die, dass ich bei Herrn H.
                  Bahr\pwindex{Bahr, Hermann 19.07.1863 – 15.01.1934@\textsc{Bahr, Hermann} (19.07.1863 – 15.01.1934), \emph{Schriftsteller/Schriftstellerin, Kritiker/Kritikerin}|pw} die Novelle, die Sie »Warten\pwindex{Warten@\emph{Warten}|pw}«
               getauft haben, (bei mir hieß sie zuerst »Blätter«) – an die Sie sich hoffentlich
                  \label{K_L03694-1v}\edtext{noch erinnern}{\lemma{\textnormal{\emph{noch erinnern}}}\Cendnote{\textnormal{Plessner\pwindex{Plessner, Elsa 22.08.1875 – 01.05.1932@\textsc{Plessner, Elsa} (22.08.1875 – 01.05.1932), \emph{Schriftsteller/Schriftstellerin}|pwk} hatte die Erzählung\pwindex{Warten@\emph{Warten}|pwkv} am 14. 4. 1896{ }Schnitzler in einer ersten Fassung zugesandt
                  und am 15. 9. 1896 in
                     einem Paket mit weiteren Texten erneut. Er äußerte seine Zustimmung zu dem Text\pwindex{Warten@\emph{Warten}|pwkv},
                     vgl. Elsa Plessner an Arthur Schnitzler, 21. 9. 1896.}}}\label{K_L03694-1} – für »die Zeit\orgindex{Zeit. Wiener Wochenschrift@Die Zeit. Wiener Wochenschrift|pw}« eingereicht habe, und dass ich Sie nun
               herzlichst bitte, ein – (oder zwei?) \label{K_L03694-2v}\edtext{gute Worte}{\lemma{\textnormal{\emph{gute Worte}}}\Cendnote{\textnormal{siehe Arthur Schnitzler an Hermann Bahr, 22. 4. 1897. }}}\label{K_L03694-2} für mich
               und sie bei genanntem Herrn einzulegen.\pend
           
\pstart
           Ich traue mich diesbezüglich nur deshalb an Sie heran, weil Ihnen die Arbeit\pwindex{Warten@\emph{Warten}|pwv} seinerzeit gefiel. Aber – Sie wissen
               ja, wie das ist, – ein empfehlendes Wort Ihrerseits ist doch zehnmal gewichtiger als
               die beste Arbeit einer \label{K_L03694-3v}\edtext{\begin{otherlanguage}{french}obscurité\end{otherlanguage}}{\lemma{\textnormal{\emph{obscurité}}}\Cendnote{\textnormal{französisch, sinngemäß:
                  Unerkannten}}}\label{K_L03694-3}. – Also – besten herzlichsten Dank im voraus!\pend
           
\pstart
           In steter Verehrung{\\[\baselineskip]}\spacefill\mbox{Elsa Plessner}\pend
           \leftskip=0em{}\selectlanguage{ngerman}\endnumbering\briefempfaengerindex{Schnitzler, Arthur@\textsc{Schnitzler, Arthur}!zzzPlessner, Elsa@\emph{von Elsa Plessner}!1897-04-102@{{[}Mitte April 1897{]}}|)be}\mylabel{L03694h}
\begin{anhang}
\end{anhang}\normalsize

\doendnotes{C}
\bigskip
\vfill

\clearpage

\footnotesize

\lohead{\textsc{register}}

% Definiere theindex-Environment komplett neu ohne reledmac
\makeatletter
\renewenvironment{theindex}{%
  \section*{\indexname}%
  \setlength{\parindent}{0pt}%
  \setlength{\parskip}{0pt plus 0.3pt}%
  \let\item\@idxitem
}{%
  \clearpage
}
\makeatother

\IfFileExists{\jobname-pw.ind}{\input{\jobname-pw.ind}}{}

\end{document}

      