%% latex-leseansicht-vorspann.tex
%% Vorspann für die Leseansicht.
%% Lädt die gemeinsame Datei latex-vorspann.tex mit nicht gesetztem Schalter.

\newif\ifkorrekturansicht
\korrekturansichtfalse

\input{../tex-inputs/latex-vorspann}


\section[Arthur Schnitzler an Gustav Schwarzkopf, 17. 7. 1896]{L04116 Arthur Schnitzler an Gustav Schwarzkopf, 17. 7. 1896}
\nopagebreak\mylabel{L04116v}
\rehead{ }\normalsize\beginnumbering\briefempfaengerindex{Schwarzkopf, Gustav@\textsc{Schwarzkopf, Gustav}!zzzSchnitzler, Arthur@\emph{von Arthur Schnitzler}!1896-07-172@{17. 7. 1896}|(be}
\toendnotes[C]{\smallbreak\pagebreak[2]}
\correspDesc{Versand  durch Arthur Schnitzler am 17. 7. 1896 \textbf{Ort fehlend} 
\newline{}Übermittlung  am 18. 7. 1896 in Tromsø
\newline{}Erhalt  durch Gustav Schwarzkopf am 25. 7. 1896 in Wien}\toendnotes[C]{\smallbreak}
\Standort{CUL, Schnitzler, B 96.}
\physDesc{Postkarte, 414 Zeichen
\newline{}Handschrift: Bleistift, deutsche Kurrent
\newline{}Versand: 1) Stempel: »\nobreak{}\oindex{Tromsø@\textbf{Tromsø}|pwk}Tromsø, 18 VII 96\nobreak{}«.   2) Stempel: »\nobreak{}\oindex{I., Innere Stadt@\textbf{I., Innere Stadt}, \emph{Verwaltungsgebiet}|pwk}\textcolor{gray}{Wien 1/1}, 25. 7. \textcolor{gray}{96}, 8–9½V, \textcolor{gray}{Bestellt}\nobreak{}«. }\toendnotes[C]{\smallbreak}\pstart{}{\pb}Austria\oindex{Österreich@\textbf{Österreich}|pw}\pend{}\pstart{}\textsc{Wien}\oindex{Wien@\textbf{Wien}, \emph{Verwaltungsgebiet}|pw}\pend{}\pstart{}Herrn \textsc{Gustav Schwarzkopf}\pend{}\pstart{}Wien\oindex{Wien@\textbf{Wien}, \emph{Verwaltungsgebiet}|pw}\pend{}\pstart{}\textsc{I. Tiefer Graben 23}\oindex{Wien@\textbf{Wien}!I., Innere Stadt@\textbf{I., Innere Stadt}!Tiefer Graben 23@\textbf{Tiefer Graben 23}, \emph{Wohngebäude}|pw}\pend{}{\bigskip}\vspace{1em}
\pstart
           \raggedleft{}{\pb}an Bord der \textsc{Sig Jarl}\orgindex{Sig Jarl@Sig Jarl|pw}{ }17/7 9\textcolor{gray}{6}\pend
           \vspace{0.5em}
\pstart
           Lieber Freund, leſen Sie \label{K_L04116-1v}\edtext{\textsc{Lichtenbergs\pwindex{Lichtenberg, Georg Christoph 1.\,7.\,1742 Ober-Ramstadt – 24.\,2.\,1799 Göttingen@\textsc{Lichtenberg, Georg Christoph} (1.\,7.\,1742 Ober-Ramstadt – 24.\,2.\,1799 Göttingen), \emph{Schriftsteller, Philosoph, Physiker}|pw}}{ }ausgew. Schriften\pwindex{Lichtenberg, Georg Christoph 1.\,7.\,1742 Ober-Ramstadt – 24.\,2.\,1799 Göttingen@\textsc{Lichtenberg, Georg Christoph} (1.\,7.\,1742 Ober-Ramstadt – 24.\,2.\,1799 Göttingen), \emph{Schriftsteller, Philosoph, Physiker}!Georg Christ. Lichtenbergs ausgewählte Schriften@\strich\emph{Georg Christ. Lichtenbergs ausgewählte Schriften}|pw}{ }\introOben{}(U. Bibl.\orgindex{Philipp Reclam jun.@Philipp Reclam jun.|pw})\introOben{} S. 229, Zeile 6 von
               
               unten u. f.}{\lemma{\textnormal{\emph{Lichtenbergs … f.}}}\Cendnote{\textnormal{Ab Zeile 5, Schnitzler dürfte den mit »Warum« beginnenden Satz bezeichnen: »Wer da{ }ſagt, daß der Jude ein Schelm sei, weil
               er geſtohlen habe, der iſt ein \so{Lügner}. Warum haben die Leute ihre Effecten
               nicht beſſer in Acht genommen? Hätte der Jude gefehlt, das ich aber nicht zugebe,{ }ſo hat
               er weiter nichts als{ }ſeine Pflicht gegen{ }ſeinen Nächſten verabſäumt, das iſt Alles;«}}}\label{K_L04116-1} –
      Da{\geminationn} leſen Sie \textsc{Schwarzkopf},
               »Recepte\pwindex{Schwarzkopf, Gustav 7.\,11.\,1853 Wien – 13.\,11.\,1939 ebd.@\textsc{Schwarzkopf, Gustav} (7.\,11.\,1853 Wien – 13.\,11.\,1939 ebd.), \emph{Schriftsteller}!Recepte. Ein Kochbuch für Schriftsteller und Künstler@\strich\emph{Recepte. Ein Kochbuch für Schriftsteller und Künstler}|pw}«, \label{K_L04116-2v}\edtext{Ein neues Geſetz\pwindex{Schwarzkopf, Gustav 7.\,11.\,1853 Wien – 13.\,11.\,1939 ebd.@\textsc{Schwarzkopf, Gustav} (7.\,11.\,1853 Wien – 13.\,11.\,1939 ebd.), \emph{Schriftsteller}!neues Gesetz. Novelle@\strich\emph{Ein neues Gesetz. Novelle}|pw}}{\lemma{\textnormal{\emph{Ein neues Gesetz}}}\Cendnote{\textnormal{In seiner Satire überzeugt
               ein angeklagter Betrüger seinen Verteidiger zu argumentieren, dass die Opfer so nachlässig waren, dass er eigentlich
               eine Entschuldigung von ihnen verdient hätte.}}}\label{K_L04116-2}. –
      da{\geminationn} ko{\geminationm}en Sie nach \textsc{Skottsborg\oindex{Skodsborg@\textbf{Skodsborg}|pw}}.\pend
           
\pstart
           Und außerdem laſſen Sie{ }ſich herzlich grüßen!
      Nach dem 30. treffen mich
      Nachrichten in \textsc{Kopenhagen\oindex{Kopenhagen@\textbf{Kopenhagen}, \emph{Hauptstadt}|pw}}{ }\textsc{post rest}. Aber beſſer iſt
      Sie ko{\geminationm}en ſelbſt.\pend
           
\pstart
           Ihr{\\[\baselineskip]}\spacefill\mbox{ArthSch}\pend
           \leftskip=0em{}\selectlanguage{ngerman}\endnumbering\briefempfaengerindex{Schwarzkopf, Gustav@\textsc{Schwarzkopf, Gustav}!zzzSchnitzler, Arthur@\emph{von Arthur Schnitzler}!1896-07-172@{17. 7. 1896}|)be}\mylabel{L04116h}
\begin{anhang}
\end{anhang}\newcommand{\dateiname}{L04116}\newcommand{\titel}{Arthur Schnitzler an Gustav Schwarzkopf, 17. 7. 1896}\newcommand{\editorInnen}{Herausgegeben von Jahnke, SelmaMüller, Martin Anton}%% latex-leseansicht-abspann.tex
%% Abspann für die Leseansicht.
%% Der Schalter \ifkorrekturansicht ist bereits durch den Vorspann gesetzt.

%% latex-abspann.tex
%% Gemeinsamer Abspann für Korrekturansicht und Leseansicht.
%% Setzt den Schalter \ifkorrekturansicht voraus (gesetzt in den
%% einbindenden Dateien latex-korrekturansicht-abspann.tex bzw.
%% latex-leseansicht-abspann.tex).
%% ---------------------------------------------------------------

\normalsize

% Das esempio-Environment wird nur in der Leseansicht benötigt
\ifkorrekturansicht\else
\newenvironment{esempio}[3]%
{
    \vspace{1.5ex}
    \rlap{\underline{#1}}
    \par
    \setlength{\parindent}{0cm}
    \nopagebreak
    \leftskip=#2cm
    \rightskip=#3cm
}
{
    \par
}
\fi

\doendnotes{C}
\bigskip
\vfill

\clearpage

\footnotesize

\ifkorrekturansicht
  \lohead{\textsc{register}}
\fi

% theindex-Environment neu definieren ohne reledmac
\makeatletter
\renewenvironment{theindex}{%
  \ifkorrekturansicht
    \section*{\indexname}%
  \else
    \subsubsection*{Index der erwähnten Entitäten}%
  \fi
  \setlength{\parindent}{0pt}%
  \setlength{\parskip}{0pt plus 0.3pt}%
  \let\item\@idxitem
}{%
  \ifkorrekturansicht\clearpage\fi
}
\makeatother

\IfFileExists{\jobname-pw.ind}{\input{\jobname-pw.ind}}{}

% Quellenangabe nur in der Leseansicht
\ifkorrekturansicht\else
% Fallback-Definitionen, falls die .tex-Datei \titel etc. nicht gesetzt hat
\providecommand{\titel}{}
\providecommand{\editorInnen}{}
\providecommand{\dateiname}{\jobname}

\vspace{3cm}

\vfill

\footnotesize
\textsc{Quelle}: \titel. Herausgegeben von {\editorInnen}. In: \emph{Arthur Schnitzler: Briefwechsel mit Autorinnen und Autoren}.
 Digitale Edition, https://schnitzler-briefe.acdh.oeaw.ac.at/{\dateiname}.html (Stand \today)
\fi

\end{document}


