%% latex-korrekturansicht-vorspann.tex
%% Vorspann für die Korrekturansicht.
%% Lädt die gemeinsame Datei latex-vorspann.tex mit gesetztem Schalter.

\newif\ifkorrekturansicht
\korrekturansichttrue

\input{../tex-inputs/latex-vorspann}


\section[Arthur Schnitzler an Stefan Zweig, 1. 8. 1923]{L03752 Arthur Schnitzler an Stefan Zweig, 1. 8. 1923}
\nopagebreak\mylabel{L03752v}
\rehead{ }\normalsize\beginnumbering\briefempfaengerindex{Zweig, Stefan@\textsc{Zweig, Stefan}!zzzSchnitzler, Arthur@\emph{von Arthur Schnitzler}!1923-08-011@{1. 8. 1923}|(be}
\toendnotes[C]{\smallbreak\pagebreak[2]}\Standort{Jerusalem, National Library of Israel, ARC. Ms. Var. 305 1 58 Stefan Zweig Collection.}
\physDesc{Postkarte, 1 Blatt, 2 Seiten, 387 Zeichen
\newline{}Handschrift: Bleistift, lateinische Kurrent
\newline{}Versand: 1) Stempel: »\nobreak{}\oindex{IX., Alsergrund@\textbf{IX., Alsergrund}, \emph{A.ADM3}|pwk}
                                          9/\textsubscript{4}
                                             Wien 68, 
                                          1. VIII. 23, XII\nobreak{}«.   2) Stempel: »\nobreak{}600 K{[}ronen{]} einzuheben\nobreak{}«. }\toendnotes[C]{\smallbreak}\pstart{}{\pb}\label{T_L03752-1v}\edtext{\textcolor{gray}{\textbf{A. S.}}}{\lemma{\textnormal{\emph{A. S.}}}\Cendnote{\textnormal{ovaler Absenderkleber}}}\label{T_L03752-1}\pend{}\pstart{}\textcolor{gray}{\textbf{WIEN, XVIII.}}\oindex{XVIII., Waehring@\textbf{XVIII., Währing}, \emph{A.ADM3}|pw}\pend{}\pstart{}\textcolor{gray}{\textbf{STERNWARTESTR. 71}}\oindex{Sternwartestrasse 71@\textbf{Sternwartestraße 71}, \emph{Wohngebäude (K.WHS)}|pw}\pend{}{\bigskip}\pstart{}{\pb}Hn \pend{}\pstart{}Dr. Stephan Zweig\pend{}\pstart{}Salzburg\oindex{Salzburg@\textbf{Salzburg}, \emph{A.ADM2}|pw}\pend{}\pstart{}Kapuzinerberg 5\oindex{Paschinger Schloessl@\textbf{Paschinger Schlössl}, \emph{Wohngebäude (K.WHS)}|pw}\pend{}{\bigskip}\vspace{1em}
\pstart
           \raggedleft{}{\pb}1. 8. 923\pend
           \vspace{0.5em}
\pstart
           lieber Herr Doktor, zu größerer Sicherheit theil ich nochmals mit,
               dſs ich \label{K_L03752-1v}\edtext{Freitag}{\lemma{\textnormal{\emph{Freitag}}}\Cendnote{\textnormal{A. S.: \emph{Wiener Schnitzler}, 3. 8. 1923.
               }}}\label{K_L03752-1}{ }Nm in Salzb.\oindex{Salzburg@\textbf{Salzburg}, \emph{A.ADM2}|pw} anzuko{\geminationm}en u im Oest Hof\oindex{Oesterreichischer Hof@\textbf{Österreichischer Hof}, \emph{Hotel (K.HTL)}|pw} durch Ihre Güte ein Zimmer zu finden hoffe. Hör ich
               nichts weiteres, so denk ich im Oest Hof\oindex{Oesterreichischer Hof@\textbf{Österreichischer Hof}, \emph{Hotel (K.HTL)}|pw} zwischen
                  8 u 9 zu nachtmahlen.\pend
           \pstart Empfehlen Sie mich bitte Ihrer verehrten Gattin\pwindex{Zweig, Friderike Maria 1882-12-04 – 1971-01-18@\textsc{Zweig, Friderike Maria} (1882-12-04 – 1971-01-18), \emph{Schriftsteller/Schriftstellerin}|pwv}, u seien Sie herzlichst gegrüßt von Ihrem
                  \spacefill\mbox{Arthur Schnitzler}\pend{}\selectlanguage{ngerman}\endnumbering\briefempfaengerindex{Zweig, Stefan@\textsc{Zweig, Stefan}!zzzSchnitzler, Arthur@\emph{von Arthur Schnitzler}!1923-08-011@{1. 8. 1923}|)be}\mylabel{L03752h}
\begin{anhang}
\end{anhang}\normalsize

\doendnotes{C}
\bigskip
\vfill

\clearpage

\footnotesize

\lohead{\textsc{register}}

% Definiere theindex-Environment komplett neu ohne reledmac
\makeatletter
\renewenvironment{theindex}{%
  \section*{\indexname}%
  \setlength{\parindent}{0pt}%
  \setlength{\parskip}{0pt plus 0.3pt}%
  \let\item\@idxitem
}{%
  \clearpage
}
\makeatother

\IfFileExists{\jobname-pw.ind}{\input{\jobname-pw.ind}}{}

\end{document}

      