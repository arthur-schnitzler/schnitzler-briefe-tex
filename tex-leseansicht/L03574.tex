%% latex-korrekturansicht-vorspann.tex
%% Vorspann für die Korrekturansicht.
%% Lädt die gemeinsame Datei latex-vorspann.tex mit gesetztem Schalter.

\newif\ifkorrekturansicht
\korrekturansichttrue

\input{../tex-inputs/latex-vorspann}


\section[ Felix und Ottilie Salten an Arthur Schnitzler, 17. {[}8.?{]} 1921]{L03574 Felix und Ottilie Salten an Arthur Schnitzler, 17. {[}8.?{]} 1921}
\nopagebreak\mylabel{L03574v}
\rehead{ }\normalsize\beginnumbering\briefempfaengerindex{Schnitzler, Arthur@\textsc{Schnitzler, Arthur}!zzzSalten, Ottilie@\emph{von Ottilie Salten}!1921-08-171@{17. {[}8.?{]} 1921}|(be}\briefempfaengerindex{Schnitzler, Arthur@\textsc{Schnitzler, Arthur}!zzzSalten, Felix@\emph{von Felix Salten}!1921-08-171@{17. {[}8.?{]} 1921}|(be}
\toendnotes[C]{\smallbreak\pagebreak[2]}\Standort{CUL, Schnitzler, B 89, B 2.}
\physDesc{Bildpostkarte, 377 Zeichen
\newline{}Handschrift Felix Salten: schwarze Tinte, lateinische Kurrent
\newline{}Handschrift Ottilie Salten: schwarze Tinte, deutsche Kurrent
\newline{}Versand: Stempel: »\nobreak{}\oindex{Unterach am Attersee@\textbf{Unterach am Attersee}, \emph{P.PPL}|pwk}\textcolor{gray}{Un}terach am Atter\textcolor{gray}{see}\nobreak{}«.  
\newline{}Ordnung: 1) mit Bleistift von Frieda Pollak\pwindex{Pollak, Frieda 08.12.1881 – 13.07.1937@\textsc{Pollak, Frieda} (08.12.1881 – 13.07.1937), \emph{Sekretär/Sekretärin}|pw} (?) mit
                                 dem Buchstaben »A« (Abgeschrieben/Abschrift)
                                 gekennzeichnet  2) mit Bleistift von unbekannter Hand nummeriert: »287«}\toendnotes[C]{\smallbreak}\pstart{}{\pb}Herrn\pend{}\pstart{}D\textsuperscript{r} Arthur Schnitzler\pend{}\pstart{}Alt-Aussee\oindex{Altaussee@\textbf{Altaussee}, \emph{A.ADM3}|pw}\pend{}\pstart{}Seewirt\oindex{Hotel am See@\textbf{Hotel am See}, \emph{Hotel (K.HTL)}|pw}\pend{}{\bigskip}
\pstart
           \noindent{}\centering{}{\pb}\textcolor{gray}{\textbf{Salzkammergut\oindex{Salzkammergut@\textbf{Salzkammergut}, \emph{L.RGN}|pw}. \textbf{Unterach am Attersee\oindex{Unterach am Attersee@\textbf{Unterach am Attersee}, \emph{P.PPL}|pw}.}}}\pend
           \vspace{1em}
\pstart
           \raggedleft{}{\pb}Berghof\oindex{Berghof@\textbf{Berghof}, \emph{Wohngebäude (K.WHS)}|pw}, \label{K_L03574-1v}\edtext{17. \textcolor{gray}{8}. 21}{\lemma{\textnormal{\emph{17. 8. 21}}}\Cendnote{\textnormal{Die Monatsziffer ist nicht eindeutig
                     lesbar, auch ›9‹ wäre möglich. Durch die Adressierung nach Altaussee\oindex{Altaussee@\textbf{Altaussee}, \emph{A.ADM3}|pwk} und den Inhalt kann der September 1921 jedoch ausgeschlossen werden.}}}\label{K_L03574-1}\pend
           
\pstart{}Lieber,\pend\vspace{0.5em}
\pstart
           werden Sie also \label{K_L03574-2v}\edtext{auf Ihrem Weg nach München\oindex{Muenchen@\textbf{München}, \emph{P.PPLA}|pw} an uns vorüber-kommen oder vorbei
                  gehen}{\lemma{\textnormal{\emph{auf … gehen}}}\Cendnote{\textnormal{Schnitzler reiste über Salzburg\oindex{Salzburg@\textbf{Salzburg}, \emph{A.ADM2}|pwk} und Berchtesgaden\oindex{Berchtesgaden@\textbf{Berchtesgaden}, \emph{P.PPL}|pwk} nach München\oindex{Muenchen@\textbf{München}, \emph{P.PPLA}|pwk}, wo er
                  am 28. 8. 1921
                  ankam. Am 25. 8. 1921 sahen sie sich in Salzburg\oindex{Salzburg@\textbf{Salzburg}, \emph{A.ADM2}|pwk}.}}}\label{K_L03574-2}? Wir würden uns so \uline{sehr}{ }\label{K_L03574-3v}\edtext{freuen, wenn Sie kämen}{\lemma{\textnormal{\emph{freuen, wenn Sie kämen}}}\Cendnote{\textnormal{Zu Schnitzlers Verhältnis zum Berghof\oindex{Berghof@\textbf{Berghof}, \emph{Wohngebäude (K.WHS)}|pwk}{ }siehe Felix Salten an Arthur Schnitzler, [25.? 8. 1892].}}}\label{K_L03574-3} und zwei, drei, vier Tage
               blieben. Je länger, je besser! Es ist sehr still und einsam hier!\pend
           
\pstart
           Alles Herzliche von uns allen {\\[\baselineskip]}Ihr {\\[\baselineskip]}\spacefill\mbox{F. S.}\pend
           \leftskip=0em{}\selectlanguage{ngerman}\vspace{1em}
\pstart
           \noindent{}{[}hs. :{]} Wie ſchön wäre es, wenn Sie kämen! Herzlichſt
                  \spacefill\mbox{Ottilie Salten}\pend
           \selectlanguage{ngerman}\endnumbering\briefempfaengerindex{Schnitzler, Arthur@\textsc{Schnitzler, Arthur}!zzzSalten, Ottilie@\emph{von Ottilie Salten}!1921-08-171@{17. {[}8.?{]} 1921}|)be}\briefempfaengerindex{Schnitzler, Arthur@\textsc{Schnitzler, Arthur}!zzzSalten, Felix@\emph{von Felix Salten}!1921-08-171@{17. {[}8.?{]} 1921}|)be}\mylabel{L03574h}  \normalsize

\doendnotes{C}
\bigskip
\vfill

\clearpage

\footnotesize

\lohead{\textsc{register}}

% Definiere theindex-Environment komplett neu ohne reledmac
\makeatletter
\renewenvironment{theindex}{%
  \section*{\indexname}%
  \setlength{\parindent}{0pt}%
  \setlength{\parskip}{0pt plus 0.3pt}%
  \let\item\@idxitem
}{%
  \clearpage
}
\makeatother

\IfFileExists{\jobname-pw.ind}{\input{\jobname-pw.ind}}{}

\end{document}

      