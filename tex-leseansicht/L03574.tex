%% latex-leseansicht-vorspann.tex
%% Vorspann für die Leseansicht.
%% Lädt die gemeinsame Datei latex-vorspann.tex mit nicht gesetztem Schalter.

\newif\ifkorrekturansicht
\korrekturansichtfalse

\input{../tex-inputs/latex-vorspann}


\section[ Felix und Ottilie Salten an Arthur Schnitzler, 17. [8.?] 1921]{L03574 Felix und Ottilie Salten an Arthur Schnitzler,  17. [8.?] 1921}
\nopagebreak\mylabel{L03574v}
\rehead{ }\normalsize\beginnumbering\briefempfaengerindex{Schnitzler, Arthur@\textsc{Schnitzler, Arthur}!zzzSalten, Ottilie@\emph{von Ottilie Salten}!1921-08-171@{17. [8.?] 1921}|(be}\briefempfaengerindex{Schnitzler, Arthur@\textsc{Schnitzler, Arthur}!zzzSalten, Felix@\emph{von Felix Salten}!1921-08-171@{17. [8.?] 1921}|(be}
\toendnotes[C]{\smallbreak\pagebreak[2]}
\correspDesc{Versand  durch Felix Salten, Ottilie Salten am 17. [8.?] 1921 in Unterach am Attersee
\newline{}Erhalt  durch Arthur Schnitzler im Zeitraum [18. 8. 1921
                  – 22. 8. 1921?] in Altaussee}\toendnotes[C]{\smallbreak}
\Standort{CUL, Schnitzler, B 89, B 2.}
\physDesc{Bildpostkarte, 377 Zeichen
\newline{}Handschrift Felix Salten: schwarze Tinte, lateinische Kurrent
\newline{}Handschrift Ottilie Salten: schwarze Tinte, deutsche Kurrent
\newline{}Versand: Stempel: »\nobreak{}\oindex{Unterach am Attersee@\textbf{Unterach am Attersee}|pwk}\textcolor{gray}{Un}terach am Atter\textcolor{gray}{see}\nobreak{}«.  
\newline{}Ordnung: 1) mit Bleistift von Frieda Pollak\pwindex{Pollak, Frieda 8.\,12.\,1881 Wien – 13.\,7.\,1937 ebd.@\textsc{Pollak, Frieda} (8.\,12.\,1881 Wien – 13.\,7.\,1937 ebd.), \emph{Sekretärin}|pw} (?) mit
                                 dem Buchstaben »A« (Abgeschrieben/Abschrift)
                                 gekennzeichnet  2) mit Bleistift von unbekannter Hand nummeriert: »287«}\toendnotes[C]{\smallbreak}\pstart{}{\pb}Herrn\pend{}\pstart{}D\textsuperscript{r} Arthur Schnitzler\pend{}\pstart{}Alt-Aussee\oindex{Altaussee@\textbf{Altaussee}, \emph{Verwaltungsgebiet}|pw}\pend{}\pstart{}Seewirt\oindex{Hotel am See@\textbf{Hotel am See}, \emph{Hotel}|pw}\pend{}{\bigskip}
\pstart
           \noindent{}\centering{}{\pb}\textcolor{gray}{\textbf{Salzkammergut\oindex{Salzkammergut@\textbf{Salzkammergut}, \emph{Region}|pw}. \textbf{Unterach am Attersee\oindex{Unterach am Attersee@\textbf{Unterach am Attersee}|pw}.}}}\pend
           \vspace{1em}
\pstart
           \raggedleft{}{\pb}Berghof\oindex{Berghof@\textbf{Berghof}, \emph{Wohngebäude}|pw}, \label{K_L03574-1v}\edtext{17. \textcolor{gray}{8}. 21}{\lemma{\textnormal{\emph{17. 8. 21}}}\Cendnote{\textnormal{Die Monatsziffer ist nicht eindeutig
                     lesbar, auch ›9‹ wäre möglich. Durch die Adressierung nach Altaussee\oindex{Altaussee@\textbf{Altaussee}, \emph{Verwaltungsgebiet}|pwk} und den Inhalt kann der September 1921 jedoch ausgeschlossen werden.}}}\label{K_L03574-1}\pend
           
\pstart{}Lieber,\pend\vspace{0.5em}
\pstart
           werden Sie also \label{K_L03574-2v}\edtext{auf Ihrem Weg nach München\oindex{München@\textbf{München}|pw} an uns vorüber-kommen oder vorbei
                  gehen}{\lemma{\textnormal{\emph{auf … gehen}}}\Cendnote{\textnormal{Schnitzler reiste über Salzburg\oindex{Salzburg@\textbf{Salzburg}, \emph{Verwaltungsgebiet}|pwk} und Berchtesgaden\oindex{Berchtesgaden@\textbf{Berchtesgaden}|pwk} nach München\oindex{München@\textbf{München}|pwk}, wo er
                  am 28. 8. 1921
                  ankam. Am 25. 8. 1921 sahen sie sich in Salzburg\oindex{Salzburg@\textbf{Salzburg}, \emph{Verwaltungsgebiet}|pwk}.}}}\label{K_L03574-2}? Wir würden uns so \uline{sehr}{ }\label{K_L03574-3v}\edtext{freuen, wenn Sie kämen}{\lemma{\textnormal{\emph{freuen, wenn Sie kämen}}}\Cendnote{\textnormal{Zu Schnitzlers Verhältnis zum Berghof\oindex{Berghof@\textbf{Berghof}, \emph{Wohngebäude}|pwk}{ }siehe XXXX Auszeichnungsfehler: Dokument L03114 nicht gefunden.}}}\label{K_L03574-3} und zwei, drei, vier Tage
               blieben. Je länger, je besser! Es ist sehr still und einsam hier!\pend
           
\pstart
           Alles Herzliche von uns allen {\\[\baselineskip]}Ihr {\\[\baselineskip]}\spacefill\mbox{F. S.}\pend
           \leftskip=0em{}\selectlanguage{ngerman}\vspace{1em}
\pstart
           \noindent{}{[}hs. Salten:{]} Wie{ }ſchön wäre es, wenn Sie kämen! Herzlichſt
                  \spacefill\mbox{Ottilie Salten}\pend
           \selectlanguage{ngerman}\endnumbering\briefempfaengerindex{Schnitzler, Arthur@\textsc{Schnitzler, Arthur}!zzzSalten, Ottilie@\emph{von Ottilie Salten}!1921-08-171@{17. [8.?] 1921}|)be}\briefempfaengerindex{Schnitzler, Arthur@\textsc{Schnitzler, Arthur}!zzzSalten, Felix@\emph{von Felix Salten}!1921-08-171@{17. [8.?] 1921}|)be}\mylabel{L03574h}  \newcommand{\dateiname}{L03574}\newcommand{\titel}{Felix und Ottilie Salten an Arthur Schnitzler, 17. [8.?] 1921}\newcommand{\editorInnen}{Martin Anton Müller und Laura Untner}%% latex-leseansicht-abspann.tex
%% Abspann für die Leseansicht.
%% Der Schalter \ifkorrekturansicht ist bereits durch den Vorspann gesetzt.

%% latex-abspann.tex
%% Gemeinsamer Abspann für Korrekturansicht und Leseansicht.
%% Setzt den Schalter \ifkorrekturansicht voraus (gesetzt in den
%% einbindenden Dateien latex-korrekturansicht-abspann.tex bzw.
%% latex-leseansicht-abspann.tex).
%% ---------------------------------------------------------------

\normalsize

% Das esempio-Environment wird nur in der Leseansicht benötigt
\ifkorrekturansicht\else
\newenvironment{esempio}[3]%
{
    \vspace{1.5ex}
    \rlap{\underline{#1}}
    \par
    \setlength{\parindent}{0cm}
    \nopagebreak
    \leftskip=#2cm
    \rightskip=#3cm
}
{
    \par
}
\fi

\doendnotes{C}
\bigskip
\vfill

\clearpage

\footnotesize

\ifkorrekturansicht
  \lohead{\textsc{register}}
\fi

% theindex-Environment neu definieren ohne reledmac
\makeatletter
\renewenvironment{theindex}{%
  \ifkorrekturansicht
    \section*{\indexname}%
  \else
    \subsubsection*{Index der erwähnten Entitäten}%
  \fi
  \setlength{\parindent}{0pt}%
  \setlength{\parskip}{0pt plus 0.3pt}%
  \let\item\@idxitem
}{%
  \ifkorrekturansicht\clearpage\fi
}
\makeatother

\IfFileExists{\jobname-pw.ind}{\input{\jobname-pw.ind}}{}

% Quellenangabe nur in der Leseansicht
\ifkorrekturansicht\else
% Fallback-Definitionen, falls die .tex-Datei \titel etc. nicht gesetzt hat
\providecommand{\titel}{}
\providecommand{\editorInnen}{}
\providecommand{\dateiname}{\jobname}

\vspace{3cm}

\vfill

\footnotesize
\textsc{Quelle}: \titel. Herausgegeben von {\editorInnen}. In: \emph{Arthur Schnitzler: Briefwechsel mit Autorinnen und Autoren}.
 Digitale Edition, https://schnitzler-briefe.acdh.oeaw.ac.at/{\dateiname}.html (Stand \today)
\fi

\end{document}


