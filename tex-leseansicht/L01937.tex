%% latex-leseansicht-vorspann.tex
%% Vorspann für die Leseansicht.
%% Lädt die gemeinsame Datei latex-vorspann.tex mit nicht gesetztem Schalter.

\newif\ifkorrekturansicht
\korrekturansichtfalse

\input{../tex-inputs/latex-vorspann}


         
         \renewcommand{\erwaehntePersonen}{Personen: Tilly Wedekind}
         \renewcommand{\erwaehnteOrte}{Orte: München, Münchner Schauspielhaus, Wien}
         \renewcommand{\erwaehnteWerke}{Werke: Die letzten Masken, Komtesse Mizzi oder Der Familientag, Lieutenant Gustl. Novelle, Literatur}
               \section[Frank Wedekind an Arthur Schnitzler, 19. 6. 1910]{ Frank Wedekind an Arthur Schnitzler, 19. 6. 1910}\nopagebreak\mylabel{v}\rehead{ }\begin{ledgroupsized}[t]{13cm}\normalsize\beginnumbering \toendnotes[C]{\smallbreak\pagebreak[2]} \Standort{CUL, Schnitzler, B 111.}
\physDesc{Brief, 1 Blatt, 3 Seiten, 735 Zeichen
\newline{}Handschrift: schwarze Tinte, deutsche Kurrent
\newline{}Schnitzler: 1) mit Bleistift beschriftet: »\textsc{Wedekind}«  2) mit rotem Buntstift eine Unterstreichung}\toendnotes[C]{\smallbreak}\pstart{}{\pb}Sehr verehrter Herr Doctor!\pend\pstart
           \label{K_L01937-1v}\edtext{Neulich}{\lemma{\textnormal{\emph{Neulich}}}\Cendnote{\textnormal{Am 11. 5. 1910 wurde im Schauspielhaus\oindex{Muenchner Schauspielhaus@\textbf{Münchner Schauspielhaus}|pwk} zum ersten Mal \emph{Komtesse Mizzi}\pwindex{Schnitzler, Arthur 15.05.1862 – 21.10.1931@\textsc{Schnitzler, Arthur} (15.05.1862 – 21.10.1931), \emph{Schriftsteller, Mediziner}!Komtesse Mizzi oder Der Familientag1908-04-19@\strich\emph{Komtesse Mizzi oder Der Familientag} {[}1908-04-19{]}|pwk} (gemeinsam mit \emph{Die letzten
                     Masken}\pwindex{Schnitzler, Arthur 15.05.1862 – 21.10.1931@\textsc{Schnitzler, Arthur} (15.05.1862 – 21.10.1931), \emph{Schriftsteller, Mediziner}!letzten Masken1901@\strich\emph{Die letzten Masken} {[}1901{]}|pwk} und \emph{Literatur}\pwindex{Schnitzler, Arthur 15.05.1862 – 21.10.1931@\textsc{Schnitzler, Arthur} (15.05.1862 – 21.10.1931), \emph{Schriftsteller, Mediziner}!Literatur1901@\strich\emph{Literatur} {[}1901{]}|pwk}) gegeben.}}}\label{K_L01937-1h}
               hatte ich die große Freude Conteſſe Mizzi\pwindex{Schnitzler, Arthur 15.05.1862 – 21.10.1931@\textsc{Schnitzler, Arthur} (15.05.1862 – 21.10.1931), \emph{Schriftsteller, Mediziner}!Komtesse Mizzi oder Der Familientag1908-04-19@\strich\emph{Komtesse Mizzi oder Der Familientag} {[}1908-04-19{]}|pw} auf
               der Bühne zu ſehen und bin noch voll vom Genuß der Schönheit dieſes vornehmen
               ſcharfgeſchliffenen Kuntwerks. Conteſſe Mizzi\pwindex{Schnitzler, Arthur 15.05.1862 – 21.10.1931@\textsc{Schnitzler, Arthur} (15.05.1862 – 21.10.1931), \emph{Schriftsteller, Mediziner}!Komtesse Mizzi oder Der Familientag1908-04-19@\strich\emph{Komtesse Mizzi oder Der Familientag} {[}1908-04-19{]}|pw}
               erſcheint mir als eine Meiſterſchöpfung, als der Urtypus der Komödie im beſten Sinne
               des Wortes. {\pb}Als Kunſtwerk ſcheint mir das
               Stück ebenſo ein Unicum zu ſein wie es mir vor \label{K_L01937-2v}\edtext{7 Jahren}{\lemma{\textnormal{\emph{7 Jahren}}}\Cendnote{\textnormal{\emph{Lieutenant Gustl}\pwindex{Schnitzler, Arthur 15.05.1862 – 21.10.1931@\textsc{Schnitzler, Arthur} (15.05.1862 – 21.10.1931), \emph{Schriftsteller, Mediziner}!Lieutenant Gustl. Novelle1900-12-25@\strich\emph{Lieutenant Gustl. Novelle} {[}1900-12-25{]}|pwk} lag bereits
                     1902 in Buchform vor.}}}\label{K_L01937-2h}{ }Leutnant Guſtl\pwindex{Schnitzler, Arthur 15.05.1862 – 21.10.1931@\textsc{Schnitzler, Arthur} (15.05.1862 – 21.10.1931), \emph{Schriftsteller, Mediziner}!Lieutenant Gustl. Novelle1900-12-25@\strich\emph{Lieutenant Gustl. Novelle} {[}1900-12-25{]}|pw} erſchien. Ich kann es mir nicht
               verſagen, Ihnen, dem ich ſchon ſo viele verſchiedenartige Genüſſe verdanke, meiner
               hellen Freude Ausdruck zu geben.\pend
           \pstart
           {\pb}Seien Sie herzlichſt gegrüßt. An unſern
               zufälligen Abenden ist ſehr viel von Ihnen die Rede.\pend
           \pstart
           Mit verbindlichſten Empfehlungen auch von meiner Frau\pwindex{Wedekind, Tilly 11.04.1886 – 20.04.1970@\textsc{Wedekind, Tilly} (11.04.1886 – 20.04.1970), \emph{Schauspielerin}|pwv}\pend
           \pstart
           Ihr ergebener{\\[\baselineskip]}\spacefill\mbox{FrankWedekind.}\pend
           \leftskip=0em{}\pstart
           München\oindex{Muenchen@\textbf{München}|pw}, 19. Juni
                  1910.\pend
           
         
         \endnumbering\mylabel{h}\end{ledgroupsized}  \newcommand{\dateiname}{L01937}\newcommand{\titel}{Frank Wedekind an Arthur Schnitzler, 19. 6. 1910}\newcommand{\editorInnen}{Martin Anton Müller und Gerd-Hermann Susen}%% latex-leseansicht-abspann.tex
%% Abspann für die Leseansicht.
%% Der Schalter \ifkorrekturansicht ist bereits durch den Vorspann gesetzt.

%% latex-abspann.tex
%% Gemeinsamer Abspann für Korrekturansicht und Leseansicht.
%% Setzt den Schalter \ifkorrekturansicht voraus (gesetzt in den
%% einbindenden Dateien latex-korrekturansicht-abspann.tex bzw.
%% latex-leseansicht-abspann.tex).
%% ---------------------------------------------------------------

\normalsize

% Das esempio-Environment wird nur in der Leseansicht benötigt
\ifkorrekturansicht\else
\newenvironment{esempio}[3]%
{
    \vspace{1.5ex}
    \rlap{\underline{#1}}
    \par
    \setlength{\parindent}{0cm}
    \nopagebreak
    \leftskip=#2cm
    \rightskip=#3cm
}
{
    \par
}
\fi

\doendnotes{C}
\bigskip
\vfill

\clearpage

\footnotesize

\ifkorrekturansicht
  \lohead{\textsc{register}}
\fi

% theindex-Environment neu definieren ohne reledmac
\makeatletter
\renewenvironment{theindex}{%
  \ifkorrekturansicht
    \section*{\indexname}%
  \else
    \subsubsection*{Index der erwähnten Entitäten}%
  \fi
  \setlength{\parindent}{0pt}%
  \setlength{\parskip}{0pt plus 0.3pt}%
  \let\item\@idxitem
}{%
  \ifkorrekturansicht\clearpage\fi
}
\makeatother

\IfFileExists{\jobname-pw.ind}{\input{\jobname-pw.ind}}{}

% Quellenangabe nur in der Leseansicht
\ifkorrekturansicht\else
% Fallback-Definitionen, falls die .tex-Datei \titel etc. nicht gesetzt hat
\providecommand{\titel}{}
\providecommand{\editorInnen}{}
\providecommand{\dateiname}{\jobname}

\vspace{3cm}

\vfill

\footnotesize
\textsc{Quelle}: \titel. Herausgegeben von {\editorInnen}. In: \emph{Arthur Schnitzler: Briefwechsel mit Autorinnen und Autoren}.
 Digitale Edition, https://schnitzler-briefe.acdh.oeaw.ac.at/{\dateiname}.html (Stand \today)
\fi

\end{document}


      