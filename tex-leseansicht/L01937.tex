%% latex-korrekturansicht-vorspann.tex
%% Vorspann für die Korrekturansicht.
%% Lädt die gemeinsame Datei latex-vorspann.tex mit gesetztem Schalter.

\newif\ifkorrekturansicht
\korrekturansichttrue

\input{../tex-inputs/latex-vorspann}


\section[Frank Wedekind an Arthur Schnitzler, 19. 6. 1910]{L01937 Frank Wedekind an Arthur Schnitzler, 19. 6. 1910}
\nopagebreak\mylabel{L01937v}
\rehead{ }\normalsize\beginnumbering\briefempfaengerindex{Schnitzler, Arthur@\textsc{Schnitzler, Arthur}!zzzWedekind, Frank@\emph{von Frank Wedekind}!1910-06-192@{19. 6. 1910}|(be}
\toendnotes[C]{\smallbreak\pagebreak[2]}\Standort{CUL, Schnitzler, B 111.}
\physDesc{Brief, 1 Blatt, 3 Seiten, 735 Zeichen
\newline{}Handschrift: schwarze Tinte, deutsche Kurrent
\newline{}Schnitzler: 1) mit Bleistift beschriftet: »\textsc{Wedekind}«  2) mit rotem Buntstift eine Unterstreichung}\toendnotes[C]{\smallbreak}
\pstart{}{\pb}Sehr verehrter Herr Doctor!\pend\vspace{0.5em}
\pstart
           \label{K_L01937-1v}\edtext{Neulich}{\lemma{\textnormal{\emph{Neulich}}}\Cendnote{\textnormal{Am 11. 5. 1910 wurde im Schauspielhaus\oindex{Muenchner Schauspielhaus@\textbf{Münchner Schauspielhaus}, \emph{Theater (K.THE)}|pwk} zum ersten Mal \emph{Komtesse Mizzi}\pwindex{Komtesse Mizzi oder: Der Familientag@\emph{Komtesse Mizzi oder: Der Familientag}|pwk} (gemeinsam mit \emph{Die letzten
                     Masken}\pwindex{letzten Masken@\emph{Die letzten Masken}|pwk} und \emph{Literatur}\pwindex{Literatur@\emph{Literatur}|pwk}) gegeben.}}}\label{K_L01937-1}
               hatte ich die große Freude Conteſſe Mizzi\pwindex{Komtesse Mizzi oder: Der Familientag@\emph{Komtesse Mizzi oder: Der Familientag}|pw} auf
               der Bühne zu ſehen und bin noch voll vom Genuß der Schönheit dieſes vornehmen
               ſcharfgeſchliffenen Kuntwerks. Conteſſe Mizzi\pwindex{Komtesse Mizzi oder: Der Familientag@\emph{Komtesse Mizzi oder: Der Familientag}|pw}
               erſcheint mir als eine Meiſterſchöpfung, als der Urtypus der Komödie im beſten Sinne
               des Wortes. {\pb}Als Kunſtwerk ſcheint mir das
               Stück ebenſo ein Unicum zu ſein wie es mir vor \label{K_L01937-2v}\edtext{7 Jahren}{\lemma{\textnormal{\emph{7 Jahren}}}\Cendnote{\textnormal{\emph{Lieutenant Gustl}\pwindex{Lieutenant Gustl. Novelle@\emph{Lieutenant Gustl. Novelle}|pwk} lag bereits
                     1902 in Buchform vor.}}}\label{K_L01937-2}{ }Leutnant Guſtl\pwindex{Lieutenant Gustl. Novelle@\emph{Lieutenant Gustl. Novelle}|pw} erſchien. Ich kann es mir nicht
               verſagen, Ihnen, dem ich ſchon ſo viele verſchiedenartige Genüſſe verdanke, meiner
               hellen Freude Ausdruck zu geben.\pend
           
\pstart
           {\pb}Seien Sie herzlichſt gegrüßt. An unſern
               zufälligen Abenden ist ſehr viel von Ihnen die Rede.\pend
           
\pstart
           Mit verbindlichſten Empfehlungen auch von meiner Frau\pwindex{Wedekind, Tilly 11.04.1886 – 20.04.1970@\textsc{Wedekind, Tilly} (11.04.1886 – 20.04.1970), \emph{Schauspieler/Schauspielerin}|pwv}\pend
           
\pstart
           Ihr ergebener{\\[\baselineskip]}\spacefill\mbox{FrankWedekind.}\pend
           \leftskip=0em{}
\pstart
           München\oindex{Muenchen@\textbf{München}, \emph{P.PPLA}|pw}, 19. Juni
                  1910.\pend
           \selectlanguage{ngerman}\endnumbering\briefempfaengerindex{Schnitzler, Arthur@\textsc{Schnitzler, Arthur}!zzzWedekind, Frank@\emph{von Frank Wedekind}!1910-06-192@{19. 6. 1910}|)be}\mylabel{L01937h}  \normalsize

\doendnotes{C}
\bigskip
\vfill

\clearpage

\footnotesize

\lohead{\textsc{register}}

% Definiere theindex-Environment komplett neu ohne reledmac
\makeatletter
\renewenvironment{theindex}{%
  \section*{\indexname}%
  \setlength{\parindent}{0pt}%
  \setlength{\parskip}{0pt plus 0.3pt}%
  \let\item\@idxitem
}{%
  \clearpage
}
\makeatother

\IfFileExists{\jobname-pw.ind}{\input{\jobname-pw.ind}}{}

\end{document}

      