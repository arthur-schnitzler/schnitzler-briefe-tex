%% latex-leseansicht-vorspann.tex
%% Vorspann für die Leseansicht.
%% Lädt die gemeinsame Datei latex-vorspann.tex mit nicht gesetztem Schalter.

\newif\ifkorrekturansicht
\korrekturansichtfalse

\input{../tex-inputs/latex-vorspann}


\section[Frank Wedekind an Arthur Schnitzler, 19. 6. 1910]{L01937 Frank Wedekind an Arthur Schnitzler, 19. 6. 1910}
\nopagebreak\mylabel{L01937v}
\rehead{ }\normalsize\beginnumbering\briefempfaengerindex{Schnitzler, Arthur@\textsc{Schnitzler, Arthur}!zzzWedekind, Frank@\emph{von Frank Wedekind}!1910-06-192@{19. 6. 1910}|(be}
\toendnotes[C]{\smallbreak\pagebreak[2]}
\correspDesc{Versand  durch Frank Wedekind am 19. 6. 1910 in München
\newline{}Erhalt  durch Arthur Schnitzler am [20. 6. 1910] in Wien}\toendnotes[C]{\smallbreak}
\Standort{CUL, Schnitzler, B 111.}
\physDesc{Brief, 1 Blatt, 3 Seiten, 735 Zeichen
\newline{}Handschrift: schwarze Tinte, deutsche Kurrent
\newline{}Schnitzler: 1) mit Bleistift beschriftet: »\textsc{Wedekind}«  2) mit rotem Buntstift eine Unterstreichung}
\buchAbdrucke{\weitereDrucke{\emph{Frank Wedekinds Korrespondenz digital}. (7. 10. 2024) \url{https://briefedition.wedekind.h-da.de/view/document/single.xhtml?contentType=1&documentId=1552}.} }\toendnotes[C]{\smallbreak}
\pstart{}{\pb}Sehr verehrter Herr Doctor!\pend\vspace{0.5em}
\pstart
           \label{K_L01937-1v}\edtext{Neulich}{\lemma{\textnormal{\emph{Neulich}}}\Cendnote{\textnormal{Am 11. 5. 1910 wurde im Schauspielhaus\oindex{Münchner Schauspielhaus@\textbf{Münchner Schauspielhaus}, \emph{Theater}|pwk} zum ersten Mal \emph{Komtesse Mizzi}\pwindex{Schnitzler, Arthur 15.\,5.\,1862 Wien – 21.\,10.\,1931 ebd.@\textsc{Schnitzler, Arthur} (15.\,5.\,1862 Wien – 21.\,10.\,1931 ebd.), \emph{Schriftsteller, Mediziner}!Komtesse Mizzi oder: Der Familientag@\strich\emph{Komtesse Mizzi oder: Der Familientag}|pwk} (gemeinsam mit \emph{Die letzten
                     Masken}\pwindex{Schnitzler, Arthur 15.\,5.\,1862 Wien – 21.\,10.\,1931 ebd.@\textsc{Schnitzler, Arthur} (15.\,5.\,1862 Wien – 21.\,10.\,1931 ebd.), \emph{Schriftsteller, Mediziner}!letzten Masken@\strich\emph{Die letzten Masken}|pwk} und \emph{Literatur}\pwindex{Schnitzler, Arthur 15.\,5.\,1862 Wien – 21.\,10.\,1931 ebd.@\textsc{Schnitzler, Arthur} (15.\,5.\,1862 Wien – 21.\,10.\,1931 ebd.), \emph{Schriftsteller, Mediziner}!Literatur@\strich\emph{Literatur}|pwk}) gegeben.}}}\label{K_L01937-1}
               hatte ich die große Freude Conteſſe Mizzi\pwindex{Schnitzler, Arthur 15.\,5.\,1862 Wien – 21.\,10.\,1931 ebd.@\textsc{Schnitzler, Arthur} (15.\,5.\,1862 Wien – 21.\,10.\,1931 ebd.), \emph{Schriftsteller, Mediziner}!Komtesse Mizzi oder: Der Familientag@\strich\emph{Komtesse Mizzi oder: Der Familientag}|pw} auf
               der Bühne zu{ }ſehen und bin noch voll vom Genuß der Schönheit dieſes vornehmen{ }ſcharfgeſchliffenen Kuntwerks. Conteſſe Mizzi\pwindex{Schnitzler, Arthur 15.\,5.\,1862 Wien – 21.\,10.\,1931 ebd.@\textsc{Schnitzler, Arthur} (15.\,5.\,1862 Wien – 21.\,10.\,1931 ebd.), \emph{Schriftsteller, Mediziner}!Komtesse Mizzi oder: Der Familientag@\strich\emph{Komtesse Mizzi oder: Der Familientag}|pw}
               erſcheint mir als eine Meiſterſchöpfung, als der Urtypus der Komödie im beſten Sinne
               des Wortes. {\pb}Als Kunſtwerk{ }ſcheint mir das
               Stück ebenſo ein Unicum zu{ }ſein wie es mir vor \label{K_L01937-2v}\edtext{7 Jahren}{\lemma{\textnormal{\emph{7 Jahren}}}\Cendnote{\textnormal{\emph{Lieutenant Gustl}\pwindex{Schnitzler, Arthur 15.\,5.\,1862 Wien – 21.\,10.\,1931 ebd.@\textsc{Schnitzler, Arthur} (15.\,5.\,1862 Wien – 21.\,10.\,1931 ebd.), \emph{Schriftsteller, Mediziner}!Lieutenant Gustl. Novelle@\strich\emph{Lieutenant Gustl. Novelle}|pwk} lag bereits
                     1902 in Buchform vor.}}}\label{K_L01937-2}{ }Leutnant Guſtl\pwindex{Schnitzler, Arthur 15.\,5.\,1862 Wien – 21.\,10.\,1931 ebd.@\textsc{Schnitzler, Arthur} (15.\,5.\,1862 Wien – 21.\,10.\,1931 ebd.), \emph{Schriftsteller, Mediziner}!Lieutenant Gustl. Novelle@\strich\emph{Lieutenant Gustl. Novelle}|pw} erſchien. Ich kann es mir nicht
               verſagen, Ihnen, dem ich{ }ſchon{ }ſo viele verſchiedenartige Genüſſe verdanke, meiner
               hellen Freude Ausdruck zu geben.\pend
           
\pstart
           {\pb}Seien Sie herzlichſt gegrüßt. An unſern
               zufälligen Abenden ist{ }ſehr viel von Ihnen die Rede.\pend
           
\pstart
           Mit verbindlichſten Empfehlungen auch von meiner Frau\pwindex{Wedekind, Tilly 11.\,4.\,1886 Graz – 20.\,4.\,1970 München@\textsc{Wedekind, Tilly} (11.\,4.\,1886 Graz – 20.\,4.\,1970 München), \emph{Schauspielerin}|pwv}\pend
           
\pstart
           Ihr ergebener{\\[\baselineskip]}\spacefill\mbox{FrankWedekind.}\pend
           \leftskip=0em{}
\pstart
           München\oindex{München@\textbf{München}|pw}, 19. Juni 1910.\pend
           \selectlanguage{ngerman}\vspace{1em}
\pstart
           \noindent{}{\pb}An Arthur \so{Schnitzler} mit herzlichem Dank für »\label{K_L01937-11v}\edtext{Comtesse Mizzi\pwindex{Schnitzler, Arthur 15.\,5.\,1862 Wien – 21.\,10.\,1931 ebd.@\textsc{Schnitzler, Arthur} (15.\,5.\,1862 Wien – 21.\,10.\,1931 ebd.), \emph{Schriftsteller, Mediziner}!Komtesse Mizzi oder: Der Familientag@\strich\emph{Komtesse Mizzi oder: Der Familientag}|pw}}{\lemma{\textnormal{\emph{Comtesse Mizzi}}}\Cendnote{\textnormal{Dem Brief beigelegt
                  dürfte ein Widmungsexemplar von \emph{In allen Wassern gewaschen}\pwindex{Wedekind, Frank 24.\,7.\,1864 Hannover – 9.\,3.\,1918 München@\textsc{Wedekind, Frank} (24.\,7.\,1864 Hannover – 9.\,3.\,1918 München), \emph{Schriftsteller, Schauspieler, Schriftsteller}!In allen Wassern gewaschen. Tragödie in einem Aufzug@\strich\emph{In allen Wassern gewaschen. Tragödie in einem Aufzug}|pwk} sein – insofern es nicht separat versandt wurde. 
                  Jedenfalls vermerkt Schnitzler am 20. 6. 1910
                  die Lektüre im \emph{Tagebuch}\pwindex{Schnitzler, Arthur 15.\,5.\,1862 Wien – 21.\,10.\,1931 ebd.@\textsc{Schnitzler, Arthur} (15.\,5.\,1862 Wien – 21.\,10.\,1931 ebd.), \emph{Schriftsteller, Mediziner}!Tagebuch@\strich\emph{Tagebuch}|pwk} und erwähnt, dass es ihm von Wedekind\pwindex{Wedekind, Frank 24.\,7.\,1864 Hannover – 9.\,3.\,1918 München@\textsc{Wedekind, Frank} (24.\,7.\,1864 Hannover – 9.\,3.\,1918 München), \emph{Schriftsteller, Schauspieler, Schriftsteller}|pwk}
                  gesandt worden war. Das Exemplar ist verschollen, ein Nachweis findet sich im Antiquariatskatalog 574 (Lot 658) von \emph{J. A. Stargardt} (1965).
                  Die digitale Edition der Korrespondenz Wedekinds\pwindex{Wedekind, Frank 24.\,7.\,1864 Hannover – 9.\,3.\,1918 München@\textsc{Wedekind, Frank} (24.\,7.\,1864 Hannover – 9.\,3.\,1918 München), \emph{Schriftsteller, Schauspieler, Schriftsteller}|pwk} führt es als eigenes Korrespondenzstück, .
               }}}\label{K_L01937-11}«\pend
           
\pstart
           München\oindex{München@\textbf{München}|pw}, im Juni 1910.\pend
           \pstart \spacefill\mbox{Frank Wedekind}\pend{}\selectlanguage{ngerman}\endnumbering\briefempfaengerindex{Schnitzler, Arthur@\textsc{Schnitzler, Arthur}!zzzWedekind, Frank@\emph{von Frank Wedekind}!1910-06-192@{19. 6. 1910}|)be}\mylabel{L01937h}  \newcommand{\dateiname}{L01937}\newcommand{\titel}{Frank Wedekind an Arthur Schnitzler, 19. 6. 1910}\newcommand{\editorInnen}{Martin Anton Müller und Gerd-Hermann Susen}%% latex-leseansicht-abspann.tex
%% Abspann für die Leseansicht.
%% Der Schalter \ifkorrekturansicht ist bereits durch den Vorspann gesetzt.

%% latex-abspann.tex
%% Gemeinsamer Abspann für Korrekturansicht und Leseansicht.
%% Setzt den Schalter \ifkorrekturansicht voraus (gesetzt in den
%% einbindenden Dateien latex-korrekturansicht-abspann.tex bzw.
%% latex-leseansicht-abspann.tex).
%% ---------------------------------------------------------------

\normalsize

% Das esempio-Environment wird nur in der Leseansicht benötigt
\ifkorrekturansicht\else
\newenvironment{esempio}[3]%
{
    \vspace{1.5ex}
    \rlap{\underline{#1}}
    \par
    \setlength{\parindent}{0cm}
    \nopagebreak
    \leftskip=#2cm
    \rightskip=#3cm
}
{
    \par
}
\fi

\doendnotes{C}
\bigskip
\vfill

\clearpage

\footnotesize

\ifkorrekturansicht
  \lohead{\textsc{register}}
\fi

% theindex-Environment neu definieren ohne reledmac
\makeatletter
\renewenvironment{theindex}{%
  \ifkorrekturansicht
    \section*{\indexname}%
  \else
    \subsubsection*{Index der erwähnten Entitäten}%
  \fi
  \setlength{\parindent}{0pt}%
  \setlength{\parskip}{0pt plus 0.3pt}%
  \let\item\@idxitem
}{%
  \ifkorrekturansicht\clearpage\fi
}
\makeatother

\IfFileExists{\jobname-pw.ind}{\input{\jobname-pw.ind}}{}

% Quellenangabe nur in der Leseansicht
\ifkorrekturansicht\else
% Fallback-Definitionen, falls die .tex-Datei \titel etc. nicht gesetzt hat
\providecommand{\titel}{}
\providecommand{\editorInnen}{}
\providecommand{\dateiname}{\jobname}

\vspace{3cm}

\vfill

\footnotesize
\textsc{Quelle}: \titel. Herausgegeben von {\editorInnen}. In: \emph{Arthur Schnitzler: Briefwechsel mit Autorinnen und Autoren}.
 Digitale Edition, https://schnitzler-briefe.acdh.oeaw.ac.at/{\dateiname}.html (Stand \today)
\fi

\end{document}


