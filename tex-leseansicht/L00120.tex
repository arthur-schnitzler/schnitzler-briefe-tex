%% latex-korrekturansicht-vorspann.tex
%% Vorspann für die Korrekturansicht.
%% Lädt die gemeinsame Datei latex-vorspann.tex mit gesetztem Schalter.

\newif\ifkorrekturansicht
\korrekturansichttrue

\input{../tex-inputs/latex-vorspann}


\section[Arthur Schnitzler an Richard Beer-Hofmann, {[}zwischen 24. 9. 1892 und 1. 5. 1901{]}]{L00120 Arthur Schnitzler an Richard Beer-Hofmann, {[}zwischen 24. 9. 1892 und
               1. 5. 1901{]}}
\nopagebreak\mylabel{L00120v}
\rehead{ }\normalsize\beginnumbering\briefempfaengerindex{Beer-Hofmann, Richard@\textsc{Beer-Hofmann, Richard}!zzzSchnitzler, Arthur@\emph{von Arthur Schnitzler}!1901-04-301@{{[}zwischen 24. 9. 1892 und
                  1. 5. 1901{]}}|(be}
\toendnotes[C]{\smallbreak\pagebreak[2]}\Standort{YCGL, MSS 31.}
\physDesc{Briefkarte, , Umschlag, 195 Zeichen
\newline{}Handschrift: Bleistift, deutsche Kurrent
\newline{}Versand: ohne postalischen Übermittlungsvermerk }\toendnotes[C]{\smallbreak}\pstart{}{\pb}Hrn \textsc{Dr. Rich.
                     Beer-Hofmann}\pend{}\pstart{}Wien\oindex{Wien@\textbf{Wien}, \emph{A.ADM2}|pw}\pend{}\pstart{}\textsc{I. \label{K_L00120-1v}\edtext{Wollzeile 15}{\lemma{\textnormal{\emph{Wollzeile 15}}}\Cendnote{\textnormal{Das
                           Korrespondenzstück ist undatiert. Durch die Übersiedlung Beer-Hofmanns\pwindex{Beer-Hofmann, Richard 1866-07-11 – 1945-09-26@\textsc{Beer-Hofmann, Richard} (1866-07-11 – 1945-09-26), \emph{Schriftsteller/Schriftstellerin}|pwk}s im September
                              1892 in die Wollzeile\oindex{Wollzeile@\textbf{Wollzeile}, \emph{Straße (K.STR)}|pwk}, wo
                           er bis Ende April 1901 wohnte, ist eine grobe Zuordnung
                           möglich.}}}\label{K_L00120-1}\oindex{Wollzeile@\textbf{Wollzeile}, \emph{Straße (K.STR)}|pw}.}\pend{}{\bigskip}\vspace{1em}
\pstart
           \noindent{}{\pb}lieber Richard,  hier iſt der Herr\pwindex{?? [Mann ohne Winterrock] @\textsc{?? [Mann ohne Winterrock]}|pwv} mit dem Winterrock, \textsc{resp.}
               ohne den Winterrock.\pend
           \pstart Ihr \spacefill\mbox{Arthur.}\pend{}
\pstart
           \noindent{}{\pb}Vielleicht geben Sie ihm auch ein paar Kreuzer. Er
                  fährt nach Linz\oindex{Linz@\textbf{Linz}, \emph{P.PPLA}|pw}.\pend
           \selectlanguage{ngerman}\endnumbering\briefempfaengerindex{Beer-Hofmann, Richard@\textsc{Beer-Hofmann, Richard}!zzzSchnitzler, Arthur@\emph{von Arthur Schnitzler}!1892-09-241@{{[}zwischen 24. 9. 1892 und
                  1. 5. 1901{]}}|)be}\mylabel{L00120h}  \normalsize

\doendnotes{C}
\bigskip
\vfill

\clearpage

\footnotesize

\lohead{\textsc{register}}

% Definiere theindex-Environment komplett neu ohne reledmac
\makeatletter
\renewenvironment{theindex}{%
  \section*{\indexname}%
  \setlength{\parindent}{0pt}%
  \setlength{\parskip}{0pt plus 0.3pt}%
  \let\item\@idxitem
}{%
  \clearpage
}
\makeatother

\IfFileExists{\jobname-pw.ind}{\input{\jobname-pw.ind}}{}

\end{document}

      