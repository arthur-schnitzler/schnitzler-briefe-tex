%% latex-leseansicht-vorspann.tex
%% Vorspann für die Leseansicht.
%% Lädt die gemeinsame Datei latex-vorspann.tex mit nicht gesetztem Schalter.

\newif\ifkorrekturansicht
\korrekturansichtfalse

\input{../tex-inputs/latex-vorspann}


\section[Theodor Herzl an Arthur Schnitzler, 26. 4. 1895]{L03859 Theodor Herzl an Arthur Schnitzler, 26. 4. 1895}
\nopagebreak\mylabel{L03859v}
\rehead{ }\normalsize\beginnumbering\briefempfaengerindex{Schnitzler, Arthur@\textsc{Schnitzler, Arthur}!zzzHerzl, Theodor@\emph{von Theodor Herzl}!1895-04-262@{26. 4. 1895}|(be}
\toendnotes[C]{\smallbreak\pagebreak[2]}
\correspDesc{Versand  durch Theodor Herzl am 26. 4. 1895 in Paris
\newline{}Erhalt  durch Arthur Schnitzler im Zeitraum [27. 4. 1895 – 1. 5. 1895?] in Wien}\toendnotes[C]{\smallbreak}
\Standort{CUL, Schnitzler, B 39.}
\physDesc{Brief, 1 Blatt, 1 Seite, 602 Zeichen
\newline{}Handschrift: schwarze Tinte, lateinische Kurrent
\newline{}Ordnung: mit Bleistift von unbekannter Hand nummeriert: »38« }
\buchAbdrucke{\weitereDrucke{Theodor Herzl: \emph{Briefe und
                        autobiographische Notizen 1866–1895}. Bearbeitet von Johannes Wachten in Zusammenarbeit mit Chaya Harel, Daisy Tycho und Manfred Winkler. Berlin, Frankfurt am Main, Wien: \emph{Propyläen} 1983, S. 584 (Briefe und Tagebücher. Herausgegeben von Alex Bein, Hermann Greive, Moshe Schaerf, Julius H. Schoeps und Johannes Wachten, 1).} }\toendnotes[C]{\smallbreak}
\pstart
           \raggedleft{}{\pb}\uline{37 rue Cambon\oindex{37, Rue Cambon@\textbf{37, Rue Cambon}, \emph{Wohngebäude}|pw}}\pend
           
\pstart
           \raggedleft{}26. IV. 95\pend
           
\pstart{}Lieber Freund!\pend\vspace{0.5em}
\pstart
           Ihre Nachrichten fehlen mir.\pend
           
\pstart
           Haben Sie an Blumenthal\pwindex{Blumenthal, Oskar 13.\,3.\,1852 Berlin – 24.\,4.\,1917 ebd.@\textsc{Blumenthal, Oskar} (13.\,3.\,1852 Berlin – 24.\,4.\,1917 ebd.), \emph{Schriftsteller, Journalist, Theaterleiter}|pw} schreiben
      lassen? Was antwortet der
      vortreffliche Mann\pwindex{Blumenthal, Oskar 13.\,3.\,1852 Berlin – 24.\,4.\,1917 ebd.@\textsc{Blumenthal, Oskar} (13.\,3.\,1852 Berlin – 24.\,4.\,1917 ebd.), \emph{Schriftsteller, Journalist, Theaterleiter}|pwv}?\pend
           
\pstart
           Schnabels Adresse \label{K_L03859-1v}\edtext{\begin{otherlanguage}{french}poste restante\end{otherlanguage}}{\lemma{\textnormal{\emph{poste restante}}}\Cendnote{\textnormal{französisch: postlagernd}}}\label{K_L03859-1}{ }\uline{Bureau Nº 3}\oindex{place de la Madeleine@\textbf{place de la Madeleine}, \emph{Platz}|pwv}.\pend
           
\pstart
           Duncker {\kaufmannsund} Humblot\orgindex{Duncker und Humblot@Duncker {\kaufmannsund}  Humblot|pw}
         will mein
               Palais Bourbon\pwindex{Herzl, Theodor 2.\,5.\,1860 Budapest – 3.\,7.\,1904 Edlach@\textsc{Herzl, Theodor} (2.\,5.\,1860 Budapest – 3.\,7.\,1904 Edlach), \emph{Schriftsteller, Journalist}!Palais Bourbon. Bilder aus dem französischen Parlamentsleben@\strich\emph{Das Palais Bourbon. Bilder aus dem französischen Parlamentsleben}|pw} verlegen. Was
      meinen Sie? Ist das der
      richtige Verlag?\pend
           
\pstart
           Heute fasse ich mich kurz.
               Die \label{K_L03859-2v}\edtext{Tabarin\pwindex{Herzl, Theodor 2.\,5.\,1860 Budapest – 3.\,7.\,1904 Edlach@\textsc{Herzl, Theodor} (2.\,5.\,1860 Budapest – 3.\,7.\,1904 Edlach), \emph{Schriftsteller, Journalist}!Tabarin. Schauspiel in einem Act. Frei nach Catulle Mendès@\strich\emph{Tabarin. Schauspiel in einem Act. Frei nach Catulle Mendès}|pw}-Première\eventindex{Burgtheater@\textbf{Burgtheater}!Premiere von Tabarin und Verbotene Früchte, 2.5.1895@Premiere von Tabarin und Verbotene Früchte, 2.5.1895|pw}}{\lemma{\textnormal{\emph{Tabarin-Première}}}\Cendnote{\textnormal{Herzls\pwindex{Herzl, Theodor 2.\,5.\,1860 Budapest – 3.\,7.\,1904 Edlach@\textsc{Herzl, Theodor} (2.\,5.\,1860 Budapest – 3.\,7.\,1904 Edlach), \emph{Schriftsteller, Journalist}|pwk} Einakter \emph{Tabarin}\pwindex{Herzl, Theodor 2.\,5.\,1860 Budapest – 3.\,7.\,1904 Edlach@\textsc{Herzl, Theodor} (2.\,5.\,1860 Budapest – 3.\,7.\,1904 Edlach), \emph{Schriftsteller, Journalist}!Tabarin. Schauspiel in einem Act. Frei nach Catulle Mendès@\strich\emph{Tabarin. Schauspiel in einem Act. Frei nach Catulle Mendès}|pwk} hatte zehn Jahre zuvor durch den Schauspieler Friedrich Mitterwurzer\pwindex{Mitterwurzer, Friedrich 16.\,10.\,1844 Dresden – 13.\,2.\,1897 Wien@\textsc{Mitterwurzer, Friedrich} (16.\,10.\,1844 Dresden – 13.\,2.\,1897 Wien), \emph{Schauspieler}|pwk} in New York\oindex{New York City@\textbf{New York City}|pwk} seine Uraufführung\eventindex{New York City@\textbf{New York City}!Uraufführung von Tabarin, 23.11.1885@Uraufführung von Tabarin, 23.11.1885|pwkv} erfahren. Die Inszenierung am Burgtheater\oindex{Wien@\textbf{Wien}!I., Innere Stadt@\textbf{I., Innere Stadt}!Burgtheater@\textbf{Burgtheater}, \emph{Theater}|pwk} war die Europapremiere\eventindex{Burgtheater@\textbf{Burgtheater}!Premiere von Tabarin und Verbotene Früchte, 2.5.1895@Premiere von Tabarin und Verbotene Früchte, 2.5.1895|pwkv}.}}}\label{K_L03859-2}{ }Montag
      langweilt mich sehr. Vor zehn
               Jahren wär sie mir recht
      gewesen. Damals wollt ich so
      gern auf die Bühne kommen,
      wärs auch nur als Bearbeiter
      gewesen. Jetzt das! Und dabei
      siehts aus, als würde mir ein
      Gefallen erwiesen. \label{K_L03859-3v}\edtext{\begin{otherlanguage}{french}Misère de moi\end{otherlanguage}}{\lemma{\textnormal{\emph{Misère de moi}}}\Cendnote{\textnormal{französisch: mein Elend, weh mir}}}\label{K_L03859-3}!\pend
           
\pstart
           Ich grüsse Sie herzlich{\\[\baselineskip]}Ihr getreuer \spacefill\mbox{Th H.}\pend
           \leftskip=0em{}\selectlanguage{ngerman}\endnumbering\briefempfaengerindex{Schnitzler, Arthur@\textsc{Schnitzler, Arthur}!zzzHerzl, Theodor@\emph{von Theodor Herzl}!1895-04-262@{26. 4. 1895}|)be}\mylabel{L03859h}
\begin{anhang}
\end{anhang}\newcommand{\dateiname}{L03859}\newcommand{\titel}{Theodor Herzl an Arthur Schnitzler, 26. 4. 1895}\newcommand{\editorInnen}{Selma Jahnke und Martin Anton Müller}%% latex-leseansicht-abspann.tex
%% Abspann für die Leseansicht.
%% Der Schalter \ifkorrekturansicht ist bereits durch den Vorspann gesetzt.

%% latex-abspann.tex
%% Gemeinsamer Abspann für Korrekturansicht und Leseansicht.
%% Setzt den Schalter \ifkorrekturansicht voraus (gesetzt in den
%% einbindenden Dateien latex-korrekturansicht-abspann.tex bzw.
%% latex-leseansicht-abspann.tex).
%% ---------------------------------------------------------------

\normalsize

% Das esempio-Environment wird nur in der Leseansicht benötigt
\ifkorrekturansicht\else
\newenvironment{esempio}[3]%
{
    \vspace{1.5ex}
    \rlap{\underline{#1}}
    \par
    \setlength{\parindent}{0cm}
    \nopagebreak
    \leftskip=#2cm
    \rightskip=#3cm
}
{
    \par
}
\fi

\doendnotes{C}
\bigskip
\vfill

\clearpage

\footnotesize

\ifkorrekturansicht
  \lohead{\textsc{register}}
\fi

% theindex-Environment neu definieren ohne reledmac
\makeatletter
\renewenvironment{theindex}{%
  \ifkorrekturansicht
    \section*{\indexname}%
  \else
    \subsubsection*{Index der erwähnten Entitäten}%
  \fi
  \setlength{\parindent}{0pt}%
  \setlength{\parskip}{0pt plus 0.3pt}%
  \let\item\@idxitem
}{%
  \ifkorrekturansicht\clearpage\fi
}
\makeatother

\IfFileExists{\jobname-pw.ind}{\input{\jobname-pw.ind}}{}

% Quellenangabe nur in der Leseansicht
\ifkorrekturansicht\else
% Fallback-Definitionen, falls die .tex-Datei \titel etc. nicht gesetzt hat
\providecommand{\titel}{}
\providecommand{\editorInnen}{}
\providecommand{\dateiname}{\jobname}

\vspace{3cm}

\vfill

\footnotesize
\textsc{Quelle}: \titel. Herausgegeben von {\editorInnen}. In: \emph{Arthur Schnitzler: Briefwechsel mit Autorinnen und Autoren}.
 Digitale Edition, https://schnitzler-briefe.acdh.oeaw.ac.at/{\dateiname}.html (Stand \today)
\fi

\end{document}


