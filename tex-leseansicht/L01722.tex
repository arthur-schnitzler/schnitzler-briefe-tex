%% latex-korrekturansicht-vorspann.tex
%% Vorspann für die Korrekturansicht.
%% Lädt die gemeinsame Datei latex-vorspann.tex mit gesetztem Schalter.

\newif\ifkorrekturansicht
\korrekturansichttrue

\input{../tex-inputs/latex-vorspann}


\section[Stefan Großmann an Arthur Schnitzler, 12. 10. 1907]{L01722 Stefan Großmann an Arthur Schnitzler, 12. 10. 1907}
\nopagebreak\mylabel{L01722v}
\rehead{ }\normalsize\beginnumbering\briefempfaengerindex{Schnitzler, Arthur@\textsc{Schnitzler, Arthur}!zzzGrossmann, Stefan@\emph{von Stefan Großmann}!1907-10-121@{12. 10. 1907}|(be}
\toendnotes[C]{\smallbreak\pagebreak[2]}\Standort{TMW, HS Schn 2/68/1.}
\physDesc{Brief, 1 Blatt, 2 Seiten, 644 Zeichen (Briefpapier mit Trauerrand)
\newline{}Handschrift: schwarze Tinte, deutsche Kurrent
\newline{}Schnitzler: mit rotem Buntstift zwei Unterstreichungen }\toendnotes[C]{\smallbreak}
\pstart
           {\pb}\textcolor{gray}{\textbf{Freie Volksbühne\orgindex{Wiener Freie Volksbuehne@Wiener Freie Volksbühne|pw}}}\pend
           
\pstart
           \textcolor{gray}{\textbf{Wien VI/\textsubscript{1}\oindex{VI., Mariahilf@\textbf{VI., Mariahilf}, \emph{A.ADM3}|pw}.}}\pend
           
\pstart
           \textcolor{gray}{\textbf{Mariahilferſtraße Nr. 89\oindex{Mariahilfer Strasse@\textbf{Mariahilfer Straße}, \emph{Straße (K.STR)}|pw}.}}\hfill \textcolor{gray}{\textbf{Wien\oindex{Wien@\textbf{Wien}, \emph{A.ADM2}|pw}, am}}{ }12. X. \textcolor{gray}{\textbf{190}}7\pend
           
\pstart
           \textcolor{gray}{\textbf{Poſtſparkaſſen-Konto Nr. 87.544.}}\pend
           
\pstart{}Sehr verehrter Herr.\pend\vspace{0.5em}
\pstart
           Der Saal iſt: VI. Königseggaſſe \uline{10}\oindex{Koenigseggasse@\textbf{Königseggasse}, \emph{Straße (K.STR)}|pw}.\pend
           
\pstart
           Ich habe »\textsc{Excentrik\pwindex{Excentric@\emph{Excentric}|pw}}« u »\textsc{Das Lied\pwindex{neue Lied@\emph{Das neue Lied}|pw}}« geleſen, es wird mir ſchwer zu entſcheiden, die \textsc{Variété}geſchichte\pwindex{neue Lied@\emph{Das neue Lied}|pwv} iſt übermüthiger, die andre \strikeout{Geſchichte}{ }Novelle\pwindex{Excentric@\emph{Excentric}|pwv} ist mir lieber.\pend
           
\pstart
           Wozu Sie ſelbſt mehr Luſt haben, das leſen Sie!\pend
           
\pstart
           Wenn es Ihnen recht wäre, ſo würde ich Sie, geehrter Herr, abends vorher treffen oder
               abholen.\pend
           
\pstart
           Vieles, das ich als als hundsjunger Menſch gedacht und das vielleicht noch in Ihrem
                  {\pb}Gedächtnis haftet,
               könnte ich bei dieſer Gelegenheit revidiren. Aber vielleicht iſt es Ihnen lieber
               allein zu kommen. Dann will ich Sie gewiſs nicht stören.\pend
           
\pstart
           Mit aller Ergebenheit{\\[\baselineskip]}dankbar{\\[\baselineskip]}\spacefill\mbox{Stefan Großmann}\pend
           \leftskip=0em{}\selectlanguage{ngerman}\endnumbering\briefempfaengerindex{Schnitzler, Arthur@\textsc{Schnitzler, Arthur}!zzzGrossmann, Stefan@\emph{von Stefan Großmann}!1907-10-121@{12. 10. 1907}|)be}\mylabel{L01722h}  \normalsize

\doendnotes{C}
\bigskip
\vfill

\clearpage

\footnotesize

\lohead{\textsc{register}}

% Definiere theindex-Environment komplett neu ohne reledmac
\makeatletter
\renewenvironment{theindex}{%
  \section*{\indexname}%
  \setlength{\parindent}{0pt}%
  \setlength{\parskip}{0pt plus 0.3pt}%
  \let\item\@idxitem
}{%
  \clearpage
}
\makeatother

\IfFileExists{\jobname-pw.ind}{\input{\jobname-pw.ind}}{}

\end{document}

      