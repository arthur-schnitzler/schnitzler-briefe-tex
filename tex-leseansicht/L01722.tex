%% latex-leseansicht-vorspann.tex
%% Vorspann für die Leseansicht.
%% Lädt die gemeinsame Datei latex-vorspann.tex mit nicht gesetztem Schalter.

\newif\ifkorrekturansicht
\korrekturansichtfalse

\input{../tex-inputs/latex-vorspann}


         \renewcommand{\erwaehnteInstitutionen}{Institutionen: Wiener Freie Volksbühne}
         \renewcommand{\erwaehnteOrte}{Orte: Königseggasse, Mariahilferstraße, VI., Mariahilf, Wien}
         \renewcommand{\erwaehnteWerke}{Werke: Das neue Lied, Excentric}
               \section[Stefan Großmann an Arthur Schnitzler, 12. 10. 1907]{ Stefan Großmann an Arthur Schnitzler, 12. 10. 1907}\nopagebreak\mylabel{v}\rehead{ }\begin{ledgroupsized}[t]{13cm}\normalsize\beginnumbering \toendnotes[C]{\smallbreak\pagebreak[2]} \Standort{TMW, HS Schn 2/68/1.}
\physDesc{Brief, 1 Blatt, 2 Seiten, 644 Zeichen (Briefpapier mit Trauerrand)
\newline{}Handschrift: schwarze Tinte, deutsche Kurrent
\newline{}Schnitzler: mit rotem Buntstift zwei Unterstreichungen }\toendnotes[C]{\smallbreak}\pstart
           \noindent{}{\pb}\textcolor{gray}{\textbf{Freie Volksbühne\orgindex{Wiener Freie Volksbuehne@Wiener Freie Volksbühne|pw}}}\pend
           \pstart
           \textcolor{gray}{\textbf{Wien VI/\textsubscript{1}\oindex{VI., Mariahilf@\textbf{VI., Mariahilf}|pw}.}}\pend
           \pstart
           \textcolor{gray}{\textbf{Mariahilferſtraße Nr. 89\oindex{Mariahilferstrasse@\textbf{Mariahilferstraße}|pw}.}}\hfill \textcolor{gray}{\textbf{Wien\oindex{Wien@\textbf{Wien}|pw}, am}}{ }12. X. \textcolor{gray}{\textbf{190}}7\pend
           \pstart
           \textcolor{gray}{\textbf{Poſtſparkaſſen-Konto Nr. 87.544.}}\pend
           \pstart{}Sehr verehrter Herr.\pend\pstart
           Der Saal iſt: VI. Königseggaſſe \uline{10}\oindex{Koenigseggasse@\textbf{Königseggasse}|pw}.\pend
           \pstart
           Ich habe »\textsc{Excentrik\pwindex{Schnitzler, Arthur 15.05.1862 – 21.10.1931@\textsc{Schnitzler, Arthur} (15.05.1862 – 21.10.1931), \emph{Schriftsteller, Mediziner}!Excentric16. 07. 1902@\strich\emph{Excentric} {[}16. 07. 1902{]}|pw}}« u »\textsc{Das Lied\pwindex{Schnitzler, Arthur 15.05.1862 – 21.10.1931@\textsc{Schnitzler, Arthur} (15.05.1862 – 21.10.1931), \emph{Schriftsteller, Mediziner}!neue Lied23. 04. 1905@\strich\emph{Das neue Lied} {[}23. 04. 1905{]}|pw}}« geleſen, es wird mir ſchwer zu entſcheiden, die \textsc{Variété}geſchichte\pwindex{Schnitzler, Arthur 15.05.1862 – 21.10.1931@\textsc{Schnitzler, Arthur} (15.05.1862 – 21.10.1931), \emph{Schriftsteller, Mediziner}!neue Lied23. 04. 1905@\strich\emph{Das neue Lied} {[}23. 04. 1905{]}|pwv} iſt übermüthiger, die andre \strikeout{Geſchichte}{ }Novelle\pwindex{Schnitzler, Arthur 15.05.1862 – 21.10.1931@\textsc{Schnitzler, Arthur} (15.05.1862 – 21.10.1931), \emph{Schriftsteller, Mediziner}!Excentric16. 07. 1902@\strich\emph{Excentric} {[}16. 07. 1902{]}|pwv} ist mir lieber.\pend
           \pstart
           Wozu Sie ſelbſt mehr Luſt haben, das leſen Sie!\pend
           \pstart
           Wenn es Ihnen recht wäre, ſo würde ich Sie, geehrter Herr, abends vorher treffen oder
               abholen.\pend
           \pstart
           Vieles, das ich als als hundsjunger Menſch gedacht und das vielleicht noch in Ihrem
                  {\pb}Gedächtnis haftet,
               könnte ich bei dieſer Gelegenheit revidiren. Aber vielleicht iſt es Ihnen lieber
               allein zu kommen. Dann will ich Sie gewiſs nicht stören.\pend
           \pstart
           Mit aller Ergebenheit{\\[\baselineskip]}dankbar{\\[\baselineskip]}\spacefill\mbox{Stefan Großmann}\pend
           \leftskip=0em{}
         
         \endnumbering\mylabel{h}\end{ledgroupsized}  \newcommand{\dateiname}{L01722}\newcommand{\titel}{Stefan Großmann an Arthur Schnitzler, 12. 10. 1907}\newcommand{\editorInnen}{ Martin Anton Müller und Gerd-Hermann Susen}%% latex-leseansicht-abspann.tex
%% Abspann für die Leseansicht.
%% Der Schalter \ifkorrekturansicht ist bereits durch den Vorspann gesetzt.

%% latex-abspann.tex
%% Gemeinsamer Abspann für Korrekturansicht und Leseansicht.
%% Setzt den Schalter \ifkorrekturansicht voraus (gesetzt in den
%% einbindenden Dateien latex-korrekturansicht-abspann.tex bzw.
%% latex-leseansicht-abspann.tex).
%% ---------------------------------------------------------------

\normalsize

% Das esempio-Environment wird nur in der Leseansicht benötigt
\ifkorrekturansicht\else
\newenvironment{esempio}[3]%
{
    \vspace{1.5ex}
    \rlap{\underline{#1}}
    \par
    \setlength{\parindent}{0cm}
    \nopagebreak
    \leftskip=#2cm
    \rightskip=#3cm
}
{
    \par
}
\fi

\doendnotes{C}
\bigskip
\vfill

\clearpage

\footnotesize

\ifkorrekturansicht
  \lohead{\textsc{register}}
\fi

% theindex-Environment neu definieren ohne reledmac
\makeatletter
\renewenvironment{theindex}{%
  \ifkorrekturansicht
    \section*{\indexname}%
  \else
    \subsubsection*{Index der erwähnten Entitäten}%
  \fi
  \setlength{\parindent}{0pt}%
  \setlength{\parskip}{0pt plus 0.3pt}%
  \let\item\@idxitem
}{%
  \ifkorrekturansicht\clearpage\fi
}
\makeatother

\IfFileExists{\jobname-pw.ind}{\input{\jobname-pw.ind}}{}

% Quellenangabe nur in der Leseansicht
\ifkorrekturansicht\else
% Fallback-Definitionen, falls die .tex-Datei \titel etc. nicht gesetzt hat
\providecommand{\titel}{}
\providecommand{\editorInnen}{}
\providecommand{\dateiname}{\jobname}

\vspace{3cm}

\vfill

\footnotesize
\textsc{Quelle}: \titel. Herausgegeben von {\editorInnen}. In: \emph{Arthur Schnitzler: Briefwechsel mit Autorinnen und Autoren}.
 Digitale Edition, https://schnitzler-briefe.acdh.oeaw.ac.at/{\dateiname}.html (Stand \today)
\fi

\end{document}


      