%% latex-leseansicht-vorspann.tex
%% Vorspann für die Leseansicht.
%% Lädt die gemeinsame Datei latex-vorspann.tex mit nicht gesetztem Schalter.

\newif\ifkorrekturansicht
\korrekturansichtfalse

\input{../tex-inputs/latex-vorspann}


\section[Arthur Schnitzler an Hugo von Hofmannsthal, 1. 9. 1895]{L00479 Arthur Schnitzler an Hugo von Hofmannsthal, 1. 9. 1895}
\nopagebreak\mylabel{L00479v}
\rehead{ }\normalsize\beginnumbering\briefempfaengerindex{Hofmannsthal, Hugo von@\textsc{Hofmannsthal, Hugo von}!zzzSchnitzler, Arthur@\emph{von Arthur Schnitzler}!1895-09-011@{1. 9. 1895}|(be}
\toendnotes[C]{\smallbreak\pagebreak[2]}
\correspDesc{Versand  durch Arthur Schnitzler am 1. 9. 1895 in München
\newline{}Erhalt  durch Hugo von Hofmannsthal im Zeitraum [2. 9. 1895
                  – 6. 9. 1895?] in Hodonín}\toendnotes[C]{\smallbreak}
\Standort{FDH, Hs-30885,46.}
\physDesc{Brief, 2 Blätter, 6 Seiten, 2894 Zeichen
\newline{}Handschrift: schwarze Tinte, deutsche Kurrent
\newline{}Ordnung: 1) mit Bleistift von Schnitzler mutmaßlich bei der Durchsicht der
                                 Korrespondenz 1929 auf dem ersten Blatt datiert: »1/9 95«  2) mit Bleistift von unbekannter Hand beschriftet: »München\oindex{München@\textbf{München}|pw}«}
\buchAbdrucke{\weitereDrucke{1) Hugo von Hofmannsthal, Arthur Schnitzler: \emph{Briefwechsel}. Herausgegeben von Therese Nickl und Heinrich Schnitzler. Frankfurt am Main: \emph{S. Fischer} 1964, S. 61–62.} \weitereDrucke{2) Arthur Schnitzler: \emph{Briefe 1875–1912}. Herausgegeben von Therese Nickl und Heinrich Schnitzler. Frankfurt am Main: \emph{S. Fischer} 1981, S. 275–276.} }
\pstart
           \noindent{}{\pb}Lieber Hugo. Von Salzburg\oindex{Salzburg@\textbf{Salzburg}, \emph{Verwaltungsgebiet}|pw} aus, wo
                  Richard\pwindex{Beer-Hofmann, Richard 11.\,7.\,1866 Wien – 26.\,9.\,1945 New York City@\textsc{Beer-Hofmann, Richard} (11.\,7.\,1866 Wien – 26.\,9.\,1945 New York City), \emph{Schriftsteller}|pw}, \textsc{Salten}\pwindex{Salten, Felix 6.\,9.\,1869 Budapest – 8.\,10.\,1945 Zürich@\textsc{Salten, Felix} (6.\,9.\,1869 Budapest – 8.\,10.\,1945 Zürich), \emph{Schriftsteller, Journalist, Chefredakteur}|pw} u. die \textsc{Salomé}\pwindex{Andreas-Salomé, Lou 12.\,2.\,1861 Sankt Petersburg – 5.\,2.\,1937 Göttingen@\textsc{Andreas-Salomé, Lou} (12.\,2.\,1861 Sankt Petersburg – 5.\,2.\,1937 Göttingen), \emph{Schriftstellerin}|pw} zuſa{\geminationm}en waren, fuhren ich u. S.\pwindex{Salten, Felix 6.\,9.\,1869 Budapest – 8.\,10.\,1945 Zürich@\textsc{Salten, Felix} (6.\,9.\,1869 Budapest – 8.\,10.\,1945 Zürich), \emph{Schriftsteller, Journalist, Chefredakteur}|pw} per Rad davon. Das war{ }ſehr{ }ſchön. Man hat{ }ſchon ganz
               aufgehört,{ }ſo mitten durch Dörfer und Flecken zu fahren, mitten dur\damage{ch} das Leben und die Naivität \damage{\textcolor{gray}{eines Ortes}}. Von Stationen aus, wo{ }ſich naturgemäß künſtliches{ }ſa{\geminationm}elt,{ }ſieht man das alles{ }ſchief. Auch die Landſtraßen
               werden wieder lebendig, wachen auf, und man gehört mit zu den Erweckenden. Auch
               Zufälle gibt es wieder, und, das beſte, man hält den Zug an, wo es beliebt. {\pb}Dagegen fällt das mancherlei unangenehme, dſs es regnen
               kann und daſs man naſs u kotig wird u{ }ſtürzt, wenig ins Gewicht. Wir hatten darunter
               genug zu leiden, mußten{ }ſogar in einem Zollhaus{ }ſtundenlang ein beſſres Wetter
               abwarten. Amüſant war es, wie gerade an der bair\oindex{Bayern@\textbf{Bayern}, \emph{Land}|pw}-oeſterr\oindex{Österreich@\textbf{Österreich}|pw} Grenze, zwiſchen Reichenhall\oindex{Bad Reichenhall@\textbf{Bad Reichenhall}, \emph{Region}|pw} u Lofer\oindex{Lofer@\textbf{Lofer}, \emph{Hauptstadt}|pw}, Burckhard\pwindex{Burckhard, Max Eugen 14.\,7.\,1854 Korneuburg – 16.\,3.\,1912 Wien@\textsc{Burckhard, Max Eugen} (14.\,7.\,1854 Korneuburg – 16.\,3.\,1912 Wien), \emph{Schriftsteller, Rechtswissenschaftler, Theaterleiter}|pw} auf einem Rad
               entgegenkam, der von Innsbruck\oindex{Innsbruck@\textbf{Innsbruck}, \emph{Verwaltungsgebiet}|pw} nach Iſchl\oindex{Bad Ischl@\textbf{Bad Ischl}|pw} fuhr. Bei dieſem Menſchen iſt eine Miſchung
               von »reinem Thoren« und gefinkeltem Diplomaten{ }ſehr intereſſant, welche mir i{\geminationm}er zweifelloſer {\pb}wird. Sein
               perſönlicher \textsc{Charme} iſt vielleicht dieſes
               Durchleuchtetwerden eines verworrenen bunten{ }ſelbſt trüben Äußern von innen her.\pend
           
\pstart
           Worüber noch einiges zu{ }ſagen wäre. Hier, in M.\oindex{München@\textbf{München}|pw}
               bin ich{ }ſeit Donnerſtag mit Paul
                  Gldm.\pwindex{Goldmann, Paul 31.\,1.\,1865 Breslau – 25.\,9.\,1935 Wien@\textsc{Goldmann, Paul} (31.\,1.\,1865 Breslau – 25.\,9.\,1935 Wien), \emph{Schriftsteller, Journalist}|pw} zuſa{\geminationm}en, der{ }ſehr gut ausſieht, aber mit
               Schickſal und Ausſichten wenig zufrieden iſt und insbeſondere daran leidet, daſs er{ }ſeine eigene Thätigkeit nicht genügend{ }ſchätzt, weil{ }ſie nicht in der
               wünſchenswerthen Weiſe anerkannt wird. Iſt übrigens wie i{\geminationm}er voll Verſtand, Verſtändnis, Herzlichkeit, Freude am Schönen; wohlthuend in dem,
               was er bringt, und in {\pb}der Art wie er aufni{\geminationm}t. Seit geſtern Abend iſt auch Richard\pwindex{Beer-Hofmann, Richard 11.\,7.\,1866 Wien – 26.\,9.\,1945 New York City@\textsc{Beer-Hofmann, Richard} (11.\,7.\,1866 Wien – 26.\,9.\,1945 New York City), \emph{Schriftsteller}|pw} da, und die Salomé\pwindex{Andreas-Salomé, Lou 12.\,2.\,1861 Sankt Petersburg – 5.\,2.\,1937 Göttingen@\textsc{Andreas-Salomé, Lou} (12.\,2.\,1861 Sankt Petersburg – 5.\,2.\,1937 Göttingen), \emph{Schriftstellerin}|pw}{ }ſoll am 3. od. 4. ko{\geminationm}en. – Im Glaspalaſt\oindex{Glaspalast@\textbf{Glaspalast}, \emph{Gebäude}|pw}
               iſt{ }ſehr wenig gutes, viel mittelmäßiges und zu viel{ }ſchlechtes. Viel mehr iſt in der
                  \textsc{Secession}\oindex{Münchener Secession@\textbf{Münchener Secession}, \emph{Galerie}|pw} zu{ }ſehn; manches, das weit über den Schweinen und weit über den Schnapsflaſchen
               des techniſch ausgezeichneten \textsc{Heyden}\pwindex{Heyden, Hubert 13.\,9.\,1860 – 20.\,1.\,1911@\textsc{Heyden, Hubert} (13.\,9.\,1860 – 20.\,1.\,1911), \emph{Künstler}|pw}{ }ſteht. Die Meiſterſinger\pwindex{\textcolor{red}{\textsuperscript{XXXX indx1}}!Meistersinger von Nürnberg@\strich\emph{Die Meistersinger von Nürnberg}|pw} hab ich{ }ſchon einmal gehört, heute wieder. Neulich Triſtan\pwindex{\textcolor{red}{\textsuperscript{XXXX indx1}}!Tristan und Isolde@\strich\emph{Tristan und Isolde}|pw}, dem arger Schade zugefügt wird, indem
               man{ }ſich einbildet, ihn ungekürzt geben zu können oder gar zu müſſen. An den Geſchwiſter\pwindex{\textcolor{red}{\textsuperscript{XXXX indx1}}!Geschwister. Schauspiel in einem Akt@\strich\emph{Die Geschwister. Schauspiel in einem Akt}|pw}n u am \textsc{Clavigo}\pwindex{\textcolor{red}{\textsuperscript{XXXX indx1}}!Clavigo@\strich\emph{Clavigo}|pw} hab ich mich trotz vieler Mängel der Darſtellung {\pb}neulich tief erfreut. Zum erſten Mal (in den Geſchwiſter\pwindex{\textcolor{red}{\textsuperscript{XXXX indx1}}!Geschwister. Schauspiel in einem Akt@\strich\emph{Die Geschwister. Schauspiel in einem Akt}|pw}n) die Conrad-Ramlo\pwindex{Conrad-Ramlo, Marie 8.\,9.\,1850 München – 1.\,10.\,1921 ebd.@\textsc{Conrad-Ramlo, Marie} (8.\,9.\,1850 München – 1.\,10.\,1921 ebd.), \emph{Schauspielerin}|pw} geſehn,
               die viel zu bedeuten{ }ſcheint. – Heute wird Sedan\oindex{Sedan@\textbf{Sedan}, \emph{Hauptstadt}|pw} gefeiert; Fahnen, Wimpeln, Feſtzeitungen, Feſtvorſtellungen, Menſchen
               auf der Straße hin u her, geſchmückte Stadt – wohl auch einige von Stolz und
               Begeiſterung geſchwellte Herzen, die man zum Glück nicht{ }ſieht. Das andre aber iſt
               ein helles und freundliches Bild.\pend
           
\pstart
           – Freitag den 6. werde ich wohl wieder in Wien\oindex{Wien@\textbf{Wien}, \emph{Verwaltungsgebiet}|pw}{ }ſein;{ }ſchreiben Sie mir von den Manövern aus, wenn
               Sie Zeit haben, noch eine Zeile dahin. Sagen Sie, wie iſt de{\geminationn} eigentlich {\pb}Ihr Rennen
               ausgefallen? –\pend
           
\pstart
           Von Paul\pwindex{Goldmann, Paul 31.\,1.\,1865 Breslau – 25.\,9.\,1935 Wien@\textsc{Goldmann, Paul} (31.\,1.\,1865 Breslau – 25.\,9.\,1935 Wien), \emph{Schriftsteller, Journalist}|pw} u Richard\pwindex{Beer-Hofmann, Richard 11.\,7.\,1866 Wien – 26.\,9.\,1945 New York City@\textsc{Beer-Hofmann, Richard} (11.\,7.\,1866 Wien – 26.\,9.\,1945 New York City), \emph{Schriftsteller}|pw}, wie von mir die herzlichſten Grüße. Jetzt wollen wir, vor der Oper,
               nach \textsc{Nymphenburg}\oindex{Neuhausen-Nymphenburg@\textbf{Neuhausen-Nymphenburg}, \emph{Bezirk}|pw} fahren.\pend
           \pstart Ihr \spacefill\mbox{Arthur}\pend{}
\pstart
           München\oindex{München@\textbf{München}|pw},
                  1. Sept. 95.\pend
           \selectlanguage{ngerman}\endnumbering\briefempfaengerindex{Hofmannsthal, Hugo von@\textsc{Hofmannsthal, Hugo von}!zzzSchnitzler, Arthur@\emph{von Arthur Schnitzler}!1895-09-011@{1. 9. 1895}|)be}\mylabel{L00479h}  \newcommand{\dateiname}{L00479}\newcommand{\titel}{Arthur Schnitzler an Hugo von Hofmannsthal, 1. 9. 1895}\newcommand{\editorInnen}{Martin Anton Müller und Gerd-Hermann Susen}%% latex-leseansicht-abspann.tex
%% Abspann für die Leseansicht.
%% Der Schalter \ifkorrekturansicht ist bereits durch den Vorspann gesetzt.

%% latex-abspann.tex
%% Gemeinsamer Abspann für Korrekturansicht und Leseansicht.
%% Setzt den Schalter \ifkorrekturansicht voraus (gesetzt in den
%% einbindenden Dateien latex-korrekturansicht-abspann.tex bzw.
%% latex-leseansicht-abspann.tex).
%% ---------------------------------------------------------------

\normalsize

% Das esempio-Environment wird nur in der Leseansicht benötigt
\ifkorrekturansicht\else
\newenvironment{esempio}[3]%
{
    \vspace{1.5ex}
    \rlap{\underline{#1}}
    \par
    \setlength{\parindent}{0cm}
    \nopagebreak
    \leftskip=#2cm
    \rightskip=#3cm
}
{
    \par
}
\fi

\doendnotes{C}
\bigskip
\vfill

\clearpage

\footnotesize

\ifkorrekturansicht
  \lohead{\textsc{register}}
\fi

% theindex-Environment neu definieren ohne reledmac
\makeatletter
\renewenvironment{theindex}{%
  \ifkorrekturansicht
    \section*{\indexname}%
  \else
    \subsubsection*{Index der erwähnten Entitäten}%
  \fi
  \setlength{\parindent}{0pt}%
  \setlength{\parskip}{0pt plus 0.3pt}%
  \let\item\@idxitem
}{%
  \ifkorrekturansicht\clearpage\fi
}
\makeatother

\IfFileExists{\jobname-pw.ind}{\input{\jobname-pw.ind}}{}

% Quellenangabe nur in der Leseansicht
\ifkorrekturansicht\else
% Fallback-Definitionen, falls die .tex-Datei \titel etc. nicht gesetzt hat
\providecommand{\titel}{}
\providecommand{\editorInnen}{}
\providecommand{\dateiname}{\jobname}

\vspace{3cm}

\vfill

\footnotesize
\textsc{Quelle}: \titel. Herausgegeben von {\editorInnen}. In: \emph{Arthur Schnitzler: Briefwechsel mit Autorinnen und Autoren}.
 Digitale Edition, https://schnitzler-briefe.acdh.oeaw.ac.at/{\dateiname}.html (Stand \today)
\fi

\end{document}


