%% latex-korrekturansicht-vorspann.tex
%% Vorspann für die Korrekturansicht.
%% Lädt die gemeinsame Datei latex-vorspann.tex mit gesetztem Schalter.

\newif\ifkorrekturansicht
\korrekturansichttrue

\input{../tex-inputs/latex-vorspann}


\section[Arthur Schnitzler an Hugo von Hofmannsthal, 1. 9. 1895]{L00479 Arthur Schnitzler an Hugo von Hofmannsthal, 1. 9. 1895}
\nopagebreak\mylabel{L00479v}
\rehead{ }\normalsize\beginnumbering\briefempfaengerindex{Hofmannsthal, Hugo von@\textsc{Hofmannsthal, Hugo von}!zzzSchnitzler, Arthur@\emph{von Arthur Schnitzler}!1895-09-011@{1. 9. 1895}|(be}
\toendnotes[C]{\smallbreak\pagebreak[2]}\Standort{FDH, Hs-30885,46.}
\physDesc{Brief, 2 Blätter, 6 Seiten, 2894 Zeichen
\newline{}Handschrift: schwarze Tinte, deutsche Kurrent
\newline{}Ordnung: 1) mit Bleistift von Schnitzler mutmaßlich bei der Durchsicht der
                                 Korrespondenz 1929 auf dem ersten Blatt datiert: »1/9 95«  2) mit Bleistift von unbekannter Hand beschriftet: »München\oindex{Muenchen@\textbf{München}, \emph{P.PPLA}|pw}«}
\buchAbdrucke{\weitereDrucke{1) Hugo von Hofmannsthal, Arthur Schnitzler: \emph{Briefwechsel}. Frankfurt am Main: \emph{S. Fischer} 1964, S. 61–62.} \weitereDrucke{2) Arthur Schnitzler: \emph{Briefe 1875–1912}. Frankfurt am Main: \emph{S. Fischer} 1981, S. 275–276.} }
\pstart
           \noindent{}{\pb}Lieber Hugo. Von Salzburg\oindex{Salzburg@\textbf{Salzburg}, \emph{A.ADM2}|pw} aus, wo
                  Richard\pwindex{Beer-Hofmann, Richard 1866-07-11 – 1945-09-26@\textsc{Beer-Hofmann, Richard} (1866-07-11 – 1945-09-26), \emph{Schriftsteller/Schriftstellerin}|pw}, \textsc{Salten}\pwindex{Salten, Felix 06.09.1869 – 08.10.1945@\textsc{Salten, Felix} (06.09.1869 – 08.10.1945), \emph{Schriftsteller/Schriftstellerin, Journalist/Journalistin, Chefredakteur/Chefredakteurin}|pw} u. die \textsc{Salomé}\pwindex{Andreas-Salome, Lou 12.02.1861 – 05.02.1937@\textsc{Andreas-Salomé, Lou} (12.02.1861 – 05.02.1937), \emph{Schriftsteller/Schriftstellerin}|pw} zuſa{\geminationm}en waren, fuhren ich u. S.\pwindex{Salten, Felix 06.09.1869 – 08.10.1945@\textsc{Salten, Felix} (06.09.1869 – 08.10.1945), \emph{Schriftsteller/Schriftstellerin, Journalist/Journalistin, Chefredakteur/Chefredakteurin}|pw} per Rad davon. Das war ſehr ſchön. Man hat ſchon ganz
               aufgehört, ſo mitten durch Dörfer und Flecken zu fahren, mitten dur\damage{ch} das Leben und die Naivität \damage{\textcolor{gray}{eines Ortes}}. Von Stationen aus, wo ſich naturgemäß künſtliches ſa{\geminationm}elt, ſieht man das alles ſchief. Auch die Landſtraßen
               werden wieder lebendig, wachen auf, und man gehört mit zu den Erweckenden. Auch
               Zufälle gibt es wieder, und, das beſte, man hält den Zug an, wo es beliebt. {\pb}Dagegen fällt das mancherlei unangenehme, dſs es regnen
               kann und daſs man naſs u kotig wird u ſtürzt, wenig ins Gewicht. Wir hatten darunter
               genug zu leiden, mußten ſogar in einem Zollhaus ſtundenlang ein beſſres Wetter
               abwarten. Amüſant war es, wie gerade an der bair\oindex{Bayern@\textbf{Bayern}, \emph{A.ADM1}|pw}-oeſterr\oindex{Oesterreich@\textbf{Österreich}, \emph{A.PCLI}|pw} Grenze, zwiſchen Reichenhall\oindex{Bad Reichenhall@\textbf{Bad Reichenhall}, \emph{A.ADM4}|pw} u Lofer\oindex{Lofer@\textbf{Lofer}, \emph{P.PPLA3}|pw}, Burckhard\pwindex{Burckhard, Max Eugen 14.07.1854 – 16.03.1912@\textsc{Burckhard, Max Eugen} (14.07.1854 – 16.03.1912), \emph{Schriftsteller/Schriftstellerin, Rechtswissenschaftler/Rechtswissenschaftlerin, Theaterleiter/Theaterleiterin}|pw} auf einem Rad
               entgegenkam, der von Innsbruck\oindex{Innsbruck@\textbf{Innsbruck}, \emph{A.ADM2}|pw} nach Iſchl\oindex{Bad Ischl@\textbf{Bad Ischl}, \emph{P.PPL}|pw} fuhr. Bei dieſem Menſchen iſt eine Miſchung
               von »reinem Thoren« und gefinkeltem Diplomaten ſehr intereſſant, welche mir i{\geminationm}er zweifelloſer {\pb}wird. Sein
               perſönlicher \textsc{Charme} iſt vielleicht dieſes
               Durchleuchtetwerden eines verworrenen bunten ſelbſt trüben Äußern von innen her.\pend
           
\pstart
           Worüber noch einiges zu ſagen wäre. Hier, in M.\oindex{Muenchen@\textbf{München}, \emph{P.PPLA}|pw}
               bin ich ſeit Donnerſtag mit Paul
                  Gldm.\pwindex{Goldmann, Paul 31.01.1865 – 25.09.1935@\textsc{Goldmann, Paul} (31.01.1865 – 25.09.1935), \emph{Schriftsteller/Schriftstellerin, Journalist/Journalistin}|pw} zuſa{\geminationm}en, der ſehr gut ausſieht, aber mit
               Schickſal und Ausſichten wenig zufrieden iſt und insbeſondere daran leidet, daſs er
               ſeine eigene Thätigkeit nicht genügend ſchätzt, weil ſie nicht in der
               wünſchenswerthen Weiſe anerkannt wird. Iſt übrigens wie i{\geminationm}er voll Verſtand, Verſtändnis, Herzlichkeit, Freude am Schönen; wohlthuend in dem,
               was er bringt, und in {\pb}der Art wie er aufni{\geminationm}t. Seit geſtern Abend iſt auch Richard\pwindex{Beer-Hofmann, Richard 1866-07-11 – 1945-09-26@\textsc{Beer-Hofmann, Richard} (1866-07-11 – 1945-09-26), \emph{Schriftsteller/Schriftstellerin}|pw} da, und die Salomé\pwindex{Andreas-Salome, Lou 12.02.1861 – 05.02.1937@\textsc{Andreas-Salomé, Lou} (12.02.1861 – 05.02.1937), \emph{Schriftsteller/Schriftstellerin}|pw}{ }ſoll am 3. od. 4. ko{\geminationm}en. – Im Glaspalaſt\oindex{Glaspalast@\textbf{Glaspalast}, \emph{Gebäude (K.GBD)}|pw}
               iſt ſehr wenig gutes, viel mittelmäßiges und zu viel ſchlechtes. Viel mehr iſt in der
                  \textsc{Secession}\oindex{Muenchener Secession@\textbf{Münchener Secession}, \emph{Galerie (K.GLR)}|pw} zu ſehn; manches, das weit über den Schweinen und weit über den Schnapsflaſchen
               des techniſch ausgezeichneten \textsc{Heyden}\pwindex{Heyden, Hubert 13.09.1860 – 20.01.1911@\textsc{Heyden, Hubert} (13.09.1860 – 20.01.1911), \emph{Künstler/Künstlerin}|pw}{ }ſteht. Die Meiſterſinger\pwindex{Meistersinger von Nuernberg@\emph{Die Meistersinger von Nürnberg}|pw} hab ich ſchon einmal gehört, heute wieder. Neulich Triſtan\pwindex{Tristan und Isolde@\emph{Tristan und Isolde}|pw}, dem arger Schade zugefügt wird, indem
               man ſich einbildet, ihn ungekürzt geben zu können oder gar zu müſſen. An den Geſchwiſter\pwindex{Geschwister. Schauspiel in einem Akt@\emph{Die Geschwister. Schauspiel in einem Akt}|pw}n u am \textsc{Clavigo}\pwindex{Clavigo@\emph{Clavigo}|pw} hab ich mich trotz vieler Mängel der Darſtellung {\pb}neulich tief erfreut. Zum erſten Mal (in den Geſchwiſter\pwindex{Geschwister. Schauspiel in einem Akt@\emph{Die Geschwister. Schauspiel in einem Akt}|pw}n) die Conrad-Ramlo\pwindex{Conrad-Ramlo, Marie 08.09.1850 – 01.10.1921@\textsc{Conrad-Ramlo, Marie} (08.09.1850 – 01.10.1921), \emph{Schauspieler/Schauspielerin}|pw} geſehn,
               die viel zu bedeuten ſcheint. – Heute wird Sedan\oindex{Sedan@\textbf{Sedan}, \emph{P.PPLA3}|pw} gefeiert; Fahnen, Wimpeln, Feſtzeitungen, Feſtvorſtellungen, Menſchen
               auf der Straße hin u her, geſchmückte Stadt – wohl auch einige von Stolz und
               Begeiſterung geſchwellte Herzen, die man zum Glück nicht ſieht. Das andre aber iſt
               ein helles und freundliches Bild.\pend
           
\pstart
           – Freitag den 6. werde ich wohl wieder in Wien\oindex{Wien@\textbf{Wien}, \emph{A.ADM2}|pw}{ }ſein; ſchreiben Sie mir von den Manövern aus, wenn
               Sie Zeit haben, noch eine Zeile dahin. Sagen Sie, wie iſt de{\geminationn} eigentlich {\pb}Ihr Rennen
               ausgefallen? –\pend
           
\pstart
           Von Paul\pwindex{Goldmann, Paul 31.01.1865 – 25.09.1935@\textsc{Goldmann, Paul} (31.01.1865 – 25.09.1935), \emph{Schriftsteller/Schriftstellerin, Journalist/Journalistin}|pw} u Richard\pwindex{Beer-Hofmann, Richard 1866-07-11 – 1945-09-26@\textsc{Beer-Hofmann, Richard} (1866-07-11 – 1945-09-26), \emph{Schriftsteller/Schriftstellerin}|pw}, wie von mir die herzlichſten Grüße. Jetzt wollen wir, vor der Oper,
               nach \textsc{Nymphenburg}\oindex{Neuhausen-Nymphenburg@\textbf{Neuhausen-Nymphenburg}, \emph{Bezirk (A.BZK)}|pw} fahren.\pend
           \pstart Ihr \spacefill\mbox{Arthur}\pend{}
\pstart
           München\oindex{Muenchen@\textbf{München}, \emph{P.PPLA}|pw},
                  1. Sept. 95.\pend
           \selectlanguage{ngerman}\endnumbering\briefempfaengerindex{Hofmannsthal, Hugo von@\textsc{Hofmannsthal, Hugo von}!zzzSchnitzler, Arthur@\emph{von Arthur Schnitzler}!1895-09-011@{1. 9. 1895}|)be}\mylabel{L00479h}  \normalsize

\doendnotes{C}
\bigskip
\vfill

\clearpage

\footnotesize

\lohead{\textsc{register}}

% Definiere theindex-Environment komplett neu ohne reledmac
\makeatletter
\renewenvironment{theindex}{%
  \section*{\indexname}%
  \setlength{\parindent}{0pt}%
  \setlength{\parskip}{0pt plus 0.3pt}%
  \let\item\@idxitem
}{%
  \clearpage
}
\makeatother

\IfFileExists{\jobname-pw.ind}{\input{\jobname-pw.ind}}{}

\end{document}

      