%% latex-leseansicht-vorspann.tex
%% Vorspann für die Leseansicht.
%% Lädt die gemeinsame Datei latex-vorspann.tex mit nicht gesetztem Schalter.

\newif\ifkorrekturansicht
\korrekturansichtfalse

\input{../tex-inputs/latex-vorspann}


\section[Arthur Schnitzler an Hermann Bahr, 28. 11. 1896]{L00625 Arthur Schnitzler an Hermann Bahr, 28. 11. 1896}
\nopagebreak\mylabel{L00625v}
\rehead{ }\normalsize\beginnumbering\briefempfaengerindex{Bahr, Hermann@\textsc{Bahr, Hermann}!zzzSchnitzler, Arthur@\emph{von Arthur Schnitzler}!1896-11-281@{28. 11. 1896}|(be}
\toendnotes[C]{\smallbreak\pagebreak[2]}
\correspDesc{Versand  durch Arthur Schnitzler am 28. 11. 1896 in Wien
\newline{}Erhalt  durch Hermann Bahr im Zeitraum [28. 11. 1896 – 2. 12. 1896?] in Wien}\toendnotes[C]{\smallbreak}
\Standort{TMW, HS AM 23327 Ba.}
\physDesc{Brief, 1 Blatt, 4 Seiten, 686 Zeichen
\newline{}Handschrift: Bleistift, deutsche Kurrent}
\buchAbdrucke{\weitereDrucke{1) \emph{28. 11. 1896.} In: Arthur Schnitzler: \emph{The Letters of Arthur Schnitzler to Hermann Bahr}. Edited, annotated, and with an introduction, by Donald G. Daviau. Chapel Hill: \emph{The University of North Carolina Press} 1978, S. 59 (University of North Carolina studies in the Germanic languages
                        and literatures, 89).} \weitereDrucke{2) Hermann Bahr, Arthur Schnitzler: \emph{Briefwechsel, Aufzeichnungen, Dokumente (1891–1931)}. Herausgegeben von Kurt Ifkovits und Martin Anton Müller. Göttingen: \emph{Wallstein} 2018, S. 131.} }\toendnotes[C]{\smallbreak}
\pstart
           \raggedleft{}{\pb}Samſtag \damage{2}8. 11. 96.\pend
           
\pstart{}Lieber Hermann,\pend\vspace{0.5em}
\pstart
           als ich neulich bei dir war, hab ich vergeſſen, Dir von Reicher\pwindex{Reicher, Emanuel 18.\,6.\,1849 Bochnia – 15.\,5.\,1924 Berlin@\textsc{Reicher, Emanuel} (18.\,6.\,1849 Bochnia – 15.\,5.\,1924 Berlin), \emph{Schauspieler}|pw} etwas auszurichten, um was er mich in Berlin\oindex{Berlin@\textbf{Berlin}, \emph{Hauptstadt}|pw} gebeten hat. Er hat nemlich die {\pb}Abſicht, im Frühjahr
               mit einem Schauspielenſemble herzuko{\geminationm}en und einige hier
               noch nicht geſpielte Stücke aufzuführen, von denen er noch nicht weiſs, ob, \textsc{resp}. unter welchen Bedingungen die {\pb}Cenſur{ }ſie freigeben
               wird. Er{ }ſcheint auf deinen Rath, vielleicht auch auf deinen Beiſtand zu rechnen. Es
               handelt{ }ſich vor allem um die \label{K_L00625-1v}\edtext{Jugend\pwindex{Halbe, Max 4.\,10.\,1865 Gmina Suchy Dąb – 30.\,11.\,1944 Neuötting@\textsc{Halbe, Max} (4.\,10.\,1865 Gmina Suchy Dąb – 30.\,11.\,1944 Neuötting), \emph{Schriftsteller}!Jugend. Ein Liebesdrama@\strich\emph{Jugend. Ein Liebesdrama}|pw}, ich glaube auch um die Weber\pwindex{Hauptmann, Gerhart 15.\,11.\,1862 Szczawno-Zdrój – 6.\,6.\,1946 Jagniątków@\textsc{Hauptmann, Gerhart} (15.\,11.\,1862 Szczawno-Zdrój – 6.\,6.\,1946 Jagniątków), \emph{Schriftsteller}!Weber. Schauspiel aus den vierziger Jahren@\strich\emph{Die Weber. Schauspiel aus den vierziger Jahren}|pw}}{\lemma{\textnormal{\emph{Jugend, … Weber}}}\Cendnote{\textnormal{\emph{Jugend}\pwindex{Halbe, Max 4.\,10.\,1865 Gmina Suchy Dąb – 30.\,11.\,1944 Neuötting@\textsc{Halbe, Max} (4.\,10.\,1865 Gmina Suchy Dąb – 30.\,11.\,1944 Neuötting), \emph{Schriftsteller}!Jugend. Ein Liebesdrama@\strich\emph{Jugend. Ein Liebesdrama}|pwk} von Max Halbe\pwindex{Halbe, Max 4.\,10.\,1865 Gmina Suchy Dąb – 30.\,11.\,1944 Neuötting@\textsc{Halbe, Max} (4.\,10.\,1865 Gmina Suchy Dąb – 30.\,11.\,1944 Neuötting), \emph{Schriftsteller}|pwk} konnte erst 1901, \emph{Die Weber}\pwindex{Hauptmann, Gerhart 15.\,11.\,1862 Szczawno-Zdrój – 6.\,6.\,1946 Jagniątków@\textsc{Hauptmann, Gerhart} (15.\,11.\,1862 Szczawno-Zdrój – 6.\,6.\,1946 Jagniątków), \emph{Schriftsteller}!Weber. Schauspiel aus den vierziger Jahren@\strich\emph{Die Weber. Schauspiel aus den vierziger Jahren}|pwk} von Gerhart Hauptmann\pwindex{Hauptmann, Gerhart 15.\,11.\,1862 Szczawno-Zdrój – 6.\,6.\,1946 Jagniątków@\textsc{Hauptmann, Gerhart} (15.\,11.\,1862 Szczawno-Zdrój – 6.\,6.\,1946 Jagniątków), \emph{Schriftsteller}|pwk}
                  erst 1904 in Österreich\oindex{Österreich@\textbf{Österreich}|pwk}
                  aufgeführt werden.}}}\label{K_L00625-1}. Näheres hat {\pb}er mir{ }ſelbſt noch
               nicht geſagt; ich nehme an er \label{K_L00625-2v}\edtext{wird dir{ }ſchreiben}{\lemma{\textnormal{\emph{wird dir schreiben}}}\Cendnote{\textnormal{Kein infrage kommender
                  Brief liegt im Nachlass Bahrs\pwindex{Bahr, Hermann 19.\,7.\,1863 Linz – 15.\,1.\,1934 München@\textsc{Bahr, Hermann} (19.\,7.\,1863 Linz – 15.\,1.\,1934 München), \emph{Schriftsteller, Kritiker}|pwk}.}}}\label{K_L00625-2}, und
               dieſe Zeilen bereiten dich nur darauf vor.\pend
           
\pstart
           Herzlich grüßt dich{\\[\baselineskip]}dein \spacefill\mbox{Arthur Sch}\pend
           \leftskip=0em{}\selectlanguage{ngerman}\endnumbering\briefempfaengerindex{Bahr, Hermann@\textsc{Bahr, Hermann}!zzzSchnitzler, Arthur@\emph{von Arthur Schnitzler}!1896-11-281@{28. 11. 1896}|)be}\mylabel{L00625h}  \newcommand{\dateiname}{L00625}\newcommand{\titel}{Arthur Schnitzler an Hermann Bahr, 28. 11. 1896}\newcommand{\editorInnen}{Herausgegeben von Martin Anton Müller}%% latex-leseansicht-abspann.tex
%% Abspann für die Leseansicht.
%% Der Schalter \ifkorrekturansicht ist bereits durch den Vorspann gesetzt.

%% latex-abspann.tex
%% Gemeinsamer Abspann für Korrekturansicht und Leseansicht.
%% Setzt den Schalter \ifkorrekturansicht voraus (gesetzt in den
%% einbindenden Dateien latex-korrekturansicht-abspann.tex bzw.
%% latex-leseansicht-abspann.tex).
%% ---------------------------------------------------------------

\normalsize

% Das esempio-Environment wird nur in der Leseansicht benötigt
\ifkorrekturansicht\else
\newenvironment{esempio}[3]%
{
    \vspace{1.5ex}
    \rlap{\underline{#1}}
    \par
    \setlength{\parindent}{0cm}
    \nopagebreak
    \leftskip=#2cm
    \rightskip=#3cm
}
{
    \par
}
\fi

\doendnotes{C}
\bigskip
\vfill

\clearpage

\footnotesize

\ifkorrekturansicht
  \lohead{\textsc{register}}
\fi

% theindex-Environment neu definieren ohne reledmac
\makeatletter
\renewenvironment{theindex}{%
  \ifkorrekturansicht
    \section*{\indexname}%
  \else
    \subsubsection*{Index der erwähnten Entitäten}%
  \fi
  \setlength{\parindent}{0pt}%
  \setlength{\parskip}{0pt plus 0.3pt}%
  \let\item\@idxitem
}{%
  \ifkorrekturansicht\clearpage\fi
}
\makeatother

\IfFileExists{\jobname-pw.ind}{\input{\jobname-pw.ind}}{}

% Quellenangabe nur in der Leseansicht
\ifkorrekturansicht\else
% Fallback-Definitionen, falls die .tex-Datei \titel etc. nicht gesetzt hat
\providecommand{\titel}{}
\providecommand{\editorInnen}{}
\providecommand{\dateiname}{\jobname}

\vspace{3cm}

\vfill

\footnotesize
\textsc{Quelle}: \titel. Herausgegeben von {\editorInnen}. In: \emph{Arthur Schnitzler: Briefwechsel mit Autorinnen und Autoren}.
 Digitale Edition, https://schnitzler-briefe.acdh.oeaw.ac.at/{\dateiname}.html (Stand \today)
\fi

\end{document}


