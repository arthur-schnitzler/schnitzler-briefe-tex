%% latex-korrekturansicht-vorspann.tex
%% Vorspann für die Korrekturansicht.
%% Lädt die gemeinsame Datei latex-vorspann.tex mit gesetztem Schalter.

\newif\ifkorrekturansicht
\korrekturansichttrue

\input{../tex-inputs/latex-vorspann}


\section[Arthur Schnitzler an Hermann Bahr, 28. 11. 1896]{L00625 Arthur Schnitzler an Hermann Bahr, 28. 11. 1896}
\nopagebreak\mylabel{L00625v}
\rehead{ }\normalsize\beginnumbering\briefempfaengerindex{Bahr, Hermann@\textsc{Bahr, Hermann}!zzzSchnitzler, Arthur@\emph{von Arthur Schnitzler}!1896-11-281@{28. 11. 1896}|(be}
\toendnotes[C]{\smallbreak\pagebreak[2]}\Standort{TMW, HS AM 23327 Ba.}
\physDesc{Brief, 1 Blatt, 4 Seiten, 686 Zeichen
\newline{}Handschrift: Bleistift, deutsche Kurrent}
\buchAbdrucke{\weitereDrucke{1) Arthur Schnitzler: \emph{The Letters of Arthur Schnitzler to Hermann Bahr}. Chapel Hill: \emph{The University of North Carolina Press} 1978, S. 59.} \weitereDrucke{2) Hermann Bahr, Arthur Schnitzler: \emph{Briefwechsel, Aufzeichnungen, Dokumente (1891–1931)}. Göttingen: \emph{Wallstein} 2018, S. 131.} }\toendnotes[C]{\smallbreak}
\pstart
           \raggedleft{}{\pb}Samſtag \damage{2}8. 11. 96.\pend
           
\pstart{}Lieber Hermann,\pend\vspace{0.5em}
\pstart
           als ich neulich bei dir war, hab ich vergeſſen, Dir von Reicher\pwindex{Reicher, Emanuel 18.06.1849 – 15.05.1924@\textsc{Reicher, Emanuel} (18.06.1849 – 15.05.1924), \emph{Schauspieler/Schauspielerin}|pw} etwas auszurichten, um was er mich in Berlin\oindex{Berlin@\textbf{Berlin}, \emph{P.PPLC}|pw} gebeten hat. Er hat nemlich die {\pb}Abſicht, im Frühjahr
               mit einem Schauspielenſemble herzuko{\geminationm}en und einige hier
               noch nicht geſpielte Stücke aufzuführen, von denen er noch nicht weiſs, ob, \textsc{resp}. unter welchen Bedingungen die {\pb}Cenſur ſie freigeben
               wird. Er ſcheint auf deinen Rath, vielleicht auch auf deinen Beiſtand zu rechnen. Es
               handelt ſich vor allem um die \label{K_L00625-1v}\edtext{Jugend\pwindex{Jugend. Ein Liebesdrama@\emph{Jugend. Ein Liebesdrama}|pw}, ich glaube auch um die Weber\pwindex{Weber. Schauspiel aus den vierziger Jahren@\emph{Die Weber. Schauspiel aus den vierziger Jahren}|pw}}{\lemma{\textnormal{\emph{Jugend, … Weber}}}\Cendnote{\textnormal{\emph{Jugend}\pwindex{Jugend. Ein Liebesdrama@\emph{Jugend. Ein Liebesdrama}|pwk} von Max Halbe\pwindex{Halbe, Max 04.10.1865 – 30.11.1944@\textsc{Halbe, Max} (04.10.1865 – 30.11.1944), \emph{Schriftsteller/Schriftstellerin}|pwk} konnte erst 1901, \emph{Die Weber}\pwindex{Weber. Schauspiel aus den vierziger Jahren@\emph{Die Weber. Schauspiel aus den vierziger Jahren}|pwk} von Gerhart Hauptmann\pwindex{Hauptmann, Gerhart 15.11.1862 – 06.06.1946@\textsc{Hauptmann, Gerhart} (15.11.1862 – 06.06.1946), \emph{Schriftsteller/Schriftstellerin}|pwk}
                  erst 1904 in Österreich\oindex{Oesterreich@\textbf{Österreich}, \emph{A.PCLI}|pwk}
                  aufgeführt werden.}}}\label{K_L00625-1}. Näheres hat {\pb}er mir ſelbſt noch
               nicht geſagt; ich nehme an er \label{K_L00625-2v}\edtext{wird dir
                  ſchreiben}{\lemma{\textnormal{\emph{wird dir
                  ſchreiben}}}\Cendnote{\textnormal{Kein infrage kommender
                  Brief liegt im Nachlass Bahrs\pwindex{Bahr, Hermann 19.07.1863 – 15.01.1934@\textsc{Bahr, Hermann} (19.07.1863 – 15.01.1934), \emph{Schriftsteller/Schriftstellerin, Kritiker/Kritikerin}|pwk}.}}}\label{K_L00625-2}, und
               dieſe Zeilen bereiten dich nur darauf vor.\pend
           
\pstart
           Herzlich grüßt dich{\\[\baselineskip]}dein \spacefill\mbox{Arthur Sch}\pend
           \leftskip=0em{}\selectlanguage{ngerman}\endnumbering\briefempfaengerindex{Bahr, Hermann@\textsc{Bahr, Hermann}!zzzSchnitzler, Arthur@\emph{von Arthur Schnitzler}!1896-11-281@{28. 11. 1896}|)be}\mylabel{L00625h}  \normalsize

\doendnotes{C}
\bigskip
\vfill

\clearpage

\footnotesize

\lohead{\textsc{register}}

% Definiere theindex-Environment komplett neu ohne reledmac
\makeatletter
\renewenvironment{theindex}{%
  \section*{\indexname}%
  \setlength{\parindent}{0pt}%
  \setlength{\parskip}{0pt plus 0.3pt}%
  \let\item\@idxitem
}{%
  \clearpage
}
\makeatother

\IfFileExists{\jobname-pw.ind}{\input{\jobname-pw.ind}}{}

\end{document}

      