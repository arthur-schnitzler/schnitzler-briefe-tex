%% latex-leseansicht-vorspann.tex
%% Vorspann für die Leseansicht.
%% Lädt die gemeinsame Datei latex-vorspann.tex mit nicht gesetztem Schalter.

\newif\ifkorrekturansicht
\korrekturansichtfalse

\input{../tex-inputs/latex-vorspann}


         
         \renewcommand{\erwaehntePersonen}{Personen:  ?? [Junge Frau, in die Goldmann Dezember 1895 verliebt ist], Lou Andreas-Salomé, André Antoine, Georges Aubry, [MMe. Georges] Aubry, Hermann Bahr, Moriz Baumfeld, Richard Beer-Hofmann, Alfred von Berger, Jakob Julius David, Julius von Gans-Ludassy, Emil Granichstaedten, Gerhart Hauptmann, Theodor Herzl, Henri de Riaz, Adele Sandrock, Leopold Sonnemann, Hermann Sudermann, Jean Thorel, Charlotte Wolter}
         \renewcommand{\erwaehnteInstitutionen}{Institutionen: Frankfurter Zeitung}
         \renewcommand{\erwaehnteOrte}{Orte: Lyon, Paris, Wien, rue Feydeau}
         \renewcommand{\erwaehnteWerke}{Werke: Aus dem Tagebuch einer Weltdame, Bei Charlotte Wolter, Burgtheater (Liebelei, Schauspiel in drei Acten von Arthur Schnitzler. Rechte der Seele, Schauspiel in einem Act von Guiseppe Giacosa. Zum ersten Mal aufgeführt am 9. October), Burgtheater [Rechte der Seele, Liebelei], Burgtheater. »Rechte der Seele«, Schauspiel in einem Acte von Giuseppe Giacosa: deutsch von Otto Eisenschitz. »Liebelei«, Schauspiel in drei Acten von Arthur Schnitzler. Beide zum erstenmale aufgeführt am 9. October 1895, Deutsches Volkstheater. (»Ein Regentag«, Charakterbild von J. J. David.), Die Frau des Weisen. Erzählung, Die Presse, Die Weber. Schauspiel aus den vierziger Jahren, Die kleine Komödie, Ein Regentag. Charakterbild, Extrapost. Unparteiische Montags-Zeitung, Freiwild. Schauspiel in 3 Akten, Hanneles Himmelfahrt. Traumdichtung in zwei Teilen, La Liberté, La petite comédie. Mœurs viennois, Liebelei. Schauspiel in drei Akten, Montags-Revue. Wochenschrift für Politik, Finanzen, Kunst und Literatur, Wiener Allgemeine Zeitung, Wiener Caricaturen, [Parodie auf Liebelei / Schnitzler]}
               \section[Paul Goldmann an Arthur Schnitzler, 5. 12. {[}1895{]}]{ Paul Goldmann an Arthur Schnitzler, 5. 12. {[}1895{]}}\nopagebreak\mylabel{v}\rehead{ }\begin{ledgroupsized}[t]{13cm}\normalsize\beginnumbering \toendnotes[C]{\smallbreak\pagebreak[2]} \Standort{DLA, A:Schnitzler, HS.NZ85.1.3165.}
\physDesc{Brief, 4 Blätter, 16 Seiten, 6803 Zeichen
\newline{}Handschrift: blaue Tinte, deutsche Kurrent
\newline{}Schnitzler: 1) mit Bleistift das Jahr » 95« vermerkt  2) mit rotem Buntstift acht Unterstreichungen und eine seitliche
                                 Markierung}\toendnotes[C]{\smallbreak}\pstart
           \noindent{}{\pb}\textcolor{gray}{\textbf{\textbf{Frankfurter Zeitung\orgindex{Frankfurter Zeitung@Frankfurter Zeitung|pw}}}}\pend
           \pstart
           \textcolor{gray}{\textbf{(\begin{otherlanguage}{french}Gazette de Francfort\end{otherlanguage}\orgindex{Frankfurter Zeitung@Frankfurter Zeitung|pw}). }}\pend
           \pstart
           \textcolor{gray}{\textbf{\textbf{\begin{otherlanguage}{french}Fondateur M. L.
                              Sonnemann\pwindex{Sonnemann, Leopold 1831-10-29 – 1909-10-30@\textsc{Sonnemann, Leopold} (1831-10-29 – 1909-10-30), \emph{Journalist, Herausgeber}|pw}\end{otherlanguage}.}}}\pend
           \pstart
           \begin{otherlanguage}{french}\textcolor{gray}{\textbf{Journal politique, financier,}}\end{otherlanguage}\pend
           \pstart
           \begin{otherlanguage}{french}\textcolor{gray}{\textbf{commercial et littéraire.}}\end{otherlanguage}\pend
           \pstart
           \begin{otherlanguage}{french}\textcolor{gray}{\textbf{\textbf{Paraissant trois fois par jour.}}}\end{otherlanguage}\pend
           \pstart
           \begin{otherlanguage}{french}\textcolor{gray}{\textbf{\textbf{Bureau à Paris\oindex{Paris@\textbf{Paris}|pw}:}}}\end{otherlanguage}\pend
           \pstart
           \begin{otherlanguage}{french}\textcolor{gray}{\textbf{\textbf{24. Rue Feydeau\oindex{rue Feydeau@\textbf{rue Feydeau}|pw}.}}}\end{otherlanguage}\hfill \textsc{Paris\oindex{Paris@\textbf{Paris}|pw}}, 5. December.\pend
           \pstart\center{}Mein lieber Freund,\pend\pstart
           In Angelegenheit der Aufführung von »Liebelei\pwindex{Schnitzler, Arthur 15.05.1862 – 21.10.1931@\textsc{Schnitzler, Arthur} (15.05.1862 – 21.10.1931), \emph{Schriftsteller, Mediziner}!Liebelei. Schauspiel in drei Akten1895-10-09@\strich\emph{Liebelei. Schauspiel in drei Akten} {[}1895-10-09{]}|pw}«
               in \textsc{Paris\oindex{Paris@\textbf{Paris}|pw}} habe ich geſtern einen Schritt gethan, den ich
               längſt thun wollte. Ich war bei \textsc{Jean Thorel\pwindex{Thorel, Jean 1859-09-11 – 1916-08-20@\textsc{Thorel, Jean} (1859-09-11 – 1916-08-20), \emph{Übersetzer, Schriftsteller}|pw}}, deſſen Namen Du gewiß kennſt. Sehr braver u. gewiſſenhafter Menſch\pwindex{Thorel, Jean 1859-09-11 – 1916-08-20@\textsc{Thorel, Jean} (1859-09-11 – 1916-08-20), \emph{Übersetzer, Schriftsteller}|pwv}, wenig Künſtler, großer Freund \textsc{Hauptmann\pwindex{Hauptmann, Gerhart 15.11.1862 – 06.06.1946@\textsc{Hauptmann, Gerhart} (15.11.1862 – 06.06.1946), \emph{Schriftsteller}|pw}}s\pwindex{Thorel, Jean 1859-09-11 – 1916-08-20@\textsc{Thorel, Jean} (1859-09-11 – 1916-08-20), \emph{Übersetzer, Schriftsteller}|pwv}, von dem er die »Weber\pwindex{Hauptmann, Gerhart 15.11.1862 – 06.06.1946@\textsc{Hauptmann, Gerhart} (15.11.1862 – 06.06.1946), \emph{Schriftsteller}!Weber. Schauspiel aus den vierziger Jahren1892@\strich\emph{Die Weber. Schauspiel aus den vierziger Jahren} {[}1892{]}|pw}« u. »\textsc{Hannele\pwindex{Hauptmann, Gerhart 15.11.1862 – 06.06.1946@\textsc{Hauptmann, Gerhart} (15.11.1862 – 06.06.1946), \emph{Schriftsteller}!Hanneles Himmelfahrt. Traumdichtung in zwei Teilen1893-11-14@\strich\emph{Hanneles Himmelfahrt. Traumdichtung in zwei Teilen} {[}1893-11-14{]}|pw}}« für die Pariſ\oindex{Paris@\textbf{Paris}|pw}er Aufführung überſetzt hat,
                  \textsc{Intimus\pwindex{Thorel, Jean 1859-09-11 – 1916-08-20@\textsc{Thorel, Jean} (1859-09-11 – 1916-08-20), \emph{Übersetzer, Schriftsteller}|pwv}}{ }\strikeout{\textcolor{gray}{v}} von \textsc{Antoine\pwindex{Antoine, Andre 1858-01-31 – 1943-10-23@\textsc{Antoine, André} (1858-01-31 – 1943-10-23), \emph{Theaterleiter, Schauspieler}|pw}}{ }\textsc{etc.} Ich habe ihm von Deinem Stück\pwindex{Schnitzler, Arthur 15.05.1862 – 21.10.1931@\textsc{Schnitzler, Arthur} (15.05.1862 – 21.10.1931), \emph{Schriftsteller, Mediziner}!Liebelei. Schauspiel in drei Akten1895-10-09@\strich\emph{Liebelei. Schauspiel in drei Akten} {[}1895-10-09{]}|pwv} geſprochen, \label{K_L02758-54v}\edtext{\begin{otherlanguage}{french}\textsc{il est très – emballé là-dessus}\end{otherlanguage}}{\lemma{\textnormal{\emph{il … là-dessus}}}\Cendnote{\textnormal{französisch: er ist sehr dafür
                  eingenommen}}}\label{K_L02758-54h}, will es gern \label{K_L02758-1v}\edtext{überſetzen}{\lemma{\textnormal{\emph{überſetzen}}}\Cendnote{\textnormal{Die Übersetzung wurde,
                  obzwar mit einer Summe von 500 Francs bezahlt, nie fertiggestellt. Am
                     16. 6. 1910 setzte Schnitzler\pwindex{Schnitzler, Arthur 15.05.1862 – 21.10.1931@\textsc{Schnitzler, Arthur} (15.05.1862 – 21.10.1931), \emph{Schriftsteller, Mediziner}|pwk}{ }Jean Thorel\pwindex{Thorel, Jean 1859-09-11 – 1916-08-20@\textsc{Thorel, Jean} (1859-09-11 – 1916-08-20), \emph{Übersetzer, Schriftsteller}|pwk} davon in Kenntnis, dass er sich
                  nach vierzehn Jahren nicht mehr an frühere Abmachungen gebunden fühle und er
                  nunmehr über das Recht, \emph{Liebelei}\pwindex{Schnitzler, Arthur 15.05.1862 – 21.10.1931@\textsc{Schnitzler, Arthur} (15.05.1862 – 21.10.1931), \emph{Schriftsteller, Mediziner}!Liebelei. Schauspiel in drei Akten1895-10-09@\strich\emph{Liebelei. Schauspiel in drei Akten} {[}1895-10-09{]}|pwk} übersetzen
                  und auf die Bühne zu bringen, wieder frei verfüge. (\emph{Deutsches Literaturarchiv Marbach},
                  HS.1985.1.2069)}}}\label{K_L02758-1h}, unter der Bedingung freilich, daß es zur Aufführung
                  {\pb}kommt, will Schritte zur Aufführung bei ernſten
               Theatern thun, verlangt aber baldige Einſendung des Buch\pwindex{Schnitzler, Arthur 15.05.1862 – 21.10.1931@\textsc{Schnitzler, Arthur} (15.05.1862 – 21.10.1931), \emph{Schriftsteller, Mediziner}!Liebelei. Schauspiel in drei Akten1895-10-09@\strich\emph{Liebelei. Schauspiel in drei Akten} {[}1895-10-09{]}|pwv}es, im \strikeout{Druk} Druck oder
               auch im Manuſcript. Wenn es irgend geht, ſende ihm die Sache\pwindex{Schnitzler, Arthur 15.05.1862 – 21.10.1931@\textsc{Schnitzler, Arthur} (15.05.1862 – 21.10.1931), \emph{Schriftsteller, Mediziner}!Liebelei. Schauspiel in drei Akten1895-10-09@\strich\emph{Liebelei. Schauspiel in drei Akten} {[}1895-10-09{]}|pwv}, mit einem artigen Briefe, deutſch
               geſchrieben, worin Du Dich entſchuldigſt, daß Du wegen mangelnder franzöſiſcher
               Stylgewandtheit ihm nicht franzöſiſch ſchreibſt. Er wird keine glänzende Überſetzung
               machen; eine gute franzöſiſche Überſetzung bekommſt Du überhaupt nicht, da alle
               überſetzenden Franzoſen mehr oder minder plumpe Handwerker ſind; aber von Allen, die
               ich kenne, {\pb}wird er die Sache noch am Wenigſten
               verhunzen. Damit erledigt ſich wohl von ſelbſt der Brief des jungen \label{K_L02758-2v}\edtext{Mann\pwindex{Riaz, Henri de 1871 – 1951@\textsc{Riaz, Henri de} (1871 – 1951), \emph{Dichter}|pwv}es }{\lemma{\textnormal{\emph{Mannes }}}\Cendnote{\textnormal{Henry de Riaz\pwindex{Riaz, Henri de 1871 – 1951@\textsc{Riaz, Henri de} (1871 – 1951), \emph{Dichter}|pwk}; von ihm finden sich drei
                  Briefe aus dem Zeitraum 1895–1896 im Nachlass Schnitzler\pwindex{Schnitzler, Arthur 15.05.1862 – 21.10.1931@\textsc{Schnitzler, Arthur} (15.05.1862 – 21.10.1931), \emph{Schriftsteller, Mediziner}|pwk}s.}}}\label{K_L02758-2h} aus \textsc{Lyon\oindex{Lyon@\textbf{Lyon}|pw}}, der mir ſonſt ſehr gefällt und ſehr ehrlich zu ſein ſcheint. Aber ich habe
               mich nach ihm erkundigt, kein Menſch kennt den Namen, ſelbſt die \textsc{Lyon\oindex{Lyon@\textbf{Lyon}|pw}er} Journaliſten nicht. \strikeout{D\textcolor{gray}{rum}} Drum iſts wohl beſſer, ſich nicht aufs Unſichere einzulaſſen und lieber den
               geraden Weg, d. h. einen bekannten Überſetzer\pwindex{Thorel, Jean 1859-09-11 – 1916-08-20@\textsc{Thorel, Jean} (1859-09-11 – 1916-08-20), \emph{Übersetzer, Schriftsteller}|pwv} zu wählen. Entſchuldige, daß ich den Brief ſolange behalten. Aber
               wüßteſt Du, was Alles in meinen Kopfe rumort hat, ſeitdem!\pend
           \pstart
           {\pb}Haſt Du an \textsc{Aubry\pwindex{Aubry, Georges †~1923@\textsc{Aubry, Georges} (†~1923), \emph{Redakteur}|pw}} oder Frau\pwindex{Aubry, [MMe. Georges] @\textsc{Aubry, [MMe. Georges]}, \emph{Übersetzerin}|pwv}
               geſchrieben?\pend
           \pstart
           Die kürzlich zurückgeſandten Druckſachen haben mich intereſſirt, wie alles Übrige.
                  \label{K_L02758-22v}\edtext{\textsc{Wolter\pwindex{Baumfeld, Moriz 06.10.1868 – 03.03.1913@\textsc{Baumfeld, Moriz} (06.10.1868 – 03.03.1913), \emph{Journalist, Wissenschaftler, Theaterleiter}!Bei Charlotte Wolter1895-10-21@\strich\emph{Bei Charlotte Wolter} {[}1895-10-21{]}|pwuv}\pwindex{Wolter, Charlotte 01.03.1834 – 14.06.1897@\textsc{Wolter, Charlotte} (01.03.1834 – 14.06.1897), \emph{Schauspielerin}|pw}}}{\lemma{\textnormal{\emph{Wolter}}}\Cendnote{\textnormal{Wahrscheinlich folgende \emph{home story}, die in Schnitzler\pwindex{Schnitzler, Arthur 15.05.1862 – 21.10.1931@\textsc{Schnitzler, Arthur} (15.05.1862 – 21.10.1931), \emph{Schriftsteller, Mediziner}|pwk}s Zeitungsausschnittsammlung an der \emph{University of Exeter} aufbewahrt wird (5. Liebelei, box 10/1): Moriz Baumfeld\pwindex{Baumfeld, Moriz 06.10.1868 – 03.03.1913@\textsc{Baumfeld, Moriz} (06.10.1868 – 03.03.1913), \emph{Journalist, Wissenschaftler, Theaterleiter}|pwk}: \emph{Bei Charlotte Wolter}\pwindex{Baumfeld, Moriz 06.10.1868 – 03.03.1913@\textsc{Baumfeld, Moriz} (06.10.1868 – 03.03.1913), \emph{Journalist, Wissenschaftler, Theaterleiter}!Bei Charlotte Wolter1895-10-21@\strich\emph{Bei Charlotte Wolter} {[}1895-10-21{]}|pwk}. In: \emph{Extrapost}\pwindex{?? Werk@Nicht ermittelte Verfasserinnen und Verfasser!Extrapost. Unparteiische Montags-Zeitung1882 – 1905@\emph{Extrapost. Unparteiische Montags-Zeitung} {[}1882 – 1905{]}|pwk}, Jg. 14, Nr. 718, 21. 10. 1895, S. 1–2. Darin erzählt Charlotte Wolter\pwindex{Wolter, Charlotte 01.03.1834 – 14.06.1897@\textsc{Wolter, Charlotte} (01.03.1834 – 14.06.1897), \emph{Schauspielerin}|pwk}, dass sie nach einem Jahr erstmals wieder
                  im Theater war und das Pech hatte, \emph{Liebelei}\pwindex{Schnitzler, Arthur 15.05.1862 – 21.10.1931@\textsc{Schnitzler, Arthur} (15.05.1862 – 21.10.1931), \emph{Schriftsteller, Mediziner}!Liebelei. Schauspiel in drei Akten1895-10-09@\strich\emph{Liebelei. Schauspiel in drei Akten} {[}1895-10-09{]}|pwk}
                  zu sehen – eine, wie sie fand, völlig kunstlose Arbeit\pwindex{Schnitzler, Arthur 15.05.1862 – 21.10.1931@\textsc{Schnitzler, Arthur} (15.05.1862 – 21.10.1931), \emph{Schriftsteller, Mediziner}!Liebelei. Schauspiel in drei Akten1895-10-09@\strich\emph{Liebelei. Schauspiel in drei Akten} {[}1895-10-09{]}|pwkv}.}}}\label{K_L02758-22h}, die dumme Gans\pwindex{Wolter, Charlotte 01.03.1834 – 14.06.1897@\textsc{Wolter, Charlotte} (01.03.1834 – 14.06.1897), \emph{Schauspielerin}|pwv}, hat mich beluſtigt, \label{K_L02758-23v}\edtext{\textsc{Ludassy\pwindex{Burgtheater. »Rechte der Seele«, Schauspiel in einem Acte von Giuseppe
                  Giacosa: deutsch von Otto Eisenschitz. »Liebelei«, Schauspiel in drei Acten von
                  Arthur Schnitzler. Beide zum erstenmale aufgefuehrt am 9. October 18951895-10-11@\emph{Burgtheater. »Rechte der Seele«, Schauspiel in einem Acte von Giuseppe Giacosa: deutsch von Otto Eisenschitz. »Liebelei«, Schauspiel in drei Acten von Arthur Schnitzler. Beide zum erstenmale aufgeführt am 9. October 1895} {[}1895-10-11{]}|pwuv}\pwindex{Gans-Ludassy, Julius von 13.04.1858 – 30.09.1922@\textsc{Gans-Ludassy, Julius von} (13.04.1858 – 30.09.1922), \emph{Schriftsteller, Journalist, Herausgeber}|pw}}}{\lemma{\textnormal{\emph{Ludassy}}}\Cendnote{\textnormal{Es könnte sich um den Nachtrag der
                  früheren Kritik\pwindex{Burgtheater. »Rechte der Seele«, Schauspiel in einem Acte von Giuseppe
                  Giacosa: deutsch von Otto Eisenschitz. »Liebelei«, Schauspiel in drei Acten von
                  Arthur Schnitzler. Beide zum erstenmale aufgefuehrt am 9. October 18951895-10-11@\emph{Burgtheater. »Rechte der Seele«, Schauspiel in einem Acte von Giuseppe Giacosa: deutsch von Otto Eisenschitz. »Liebelei«, Schauspiel in drei Acten von Arthur Schnitzler. Beide zum erstenmale aufgeführt am 9. October 1895} {[}1895-10-11{]}|pwkv} handeln: L\pwindex{Gans-Ludassy, Julius von 13.04.1858 – 30.09.1922@\textsc{Gans-Ludassy, Julius von} (13.04.1858 – 30.09.1922), \emph{Schriftsteller, Journalist, Herausgeber}|pwkv} [ = Julius von Gans-Ludassy\pwindex{Gans-Ludassy, Julius von 13.04.1858 – 30.09.1922@\textsc{Gans-Ludassy, Julius von} (13.04.1858 – 30.09.1922), \emph{Schriftsteller, Journalist, Herausgeber}|pwk}]: \emph{Burgtheater. »Rechte der Seele«, Schauspiel in einem Acte
                        von Giuseppe Giacosa: deutsch von Otto Eisenschitz. »Liebelei«, Schauspiel
                        in drei Acten von Arthur Schnitzler. Beide zum erstenmale aufgeführt am
                        9. October 1895}\pwindex{Burgtheater. »Rechte der Seele«, Schauspiel in einem Acte von Giuseppe
                  Giacosa: deutsch von Otto Eisenschitz. »Liebelei«, Schauspiel in drei Acten von
                  Arthur Schnitzler. Beide zum erstenmale aufgefuehrt am 9. October 18951895-10-11@\emph{Burgtheater. »Rechte der Seele«, Schauspiel in einem Acte von Giuseppe Giacosa: deutsch von Otto Eisenschitz. »Liebelei«, Schauspiel in drei Acten von Arthur Schnitzler. Beide zum erstenmale aufgeführt am 9. October 1895} {[}1895-10-11{]}|pwk}. In: \emph{Wiener Allgemeine
                        Zeitung}\pwindex{?? Werk@Nicht ermittelte Verfasserinnen und Verfasser!Wiener Allgemeine Zeitung1.3.1880 – 11.2.1934@\emph{Wiener Allgemeine Zeitung} {[}1.3.1880 – 11.2.1934{]}|pwk}, Nr. 5282, 11. 10. 1895,
                     S. 2–3.}}}\label{K_L02758-23h} mag \strikeout{\textcolor{gray}{d}} ich gar nicht – auch Einer, der mit dem Erfolge geht und Dich bei der erſten
               Schwierigkeit im Stich laſſen wird. Die kleine \label{K_L02758-24v}\edtext{Parodie\pwindex{?? Werk@Nicht ermittelte Verfasserinnen und Verfasser!Parodie auf Liebelei / Schnitzler]1895@\emph{[Parodie auf Liebelei / Schnitzler]} {[}1895{]}|pwv}}{\lemma{\textnormal{\emph{Parodie}}}\Cendnote{\textnormal{Eventuell der ungezeichnete Text\pwindex{?? Werk@Nicht ermittelte Verfasserinnen und Verfasser!Aus dem Tagebuch einer Weltdame1895-10-20@\emph{Aus dem Tagebuch einer Weltdame} {[}1895-10-20{]}|pwkv}: \emph{Aus dem Tagebuch einer Weltdame}\pwindex{?? Werk@Nicht ermittelte Verfasserinnen und Verfasser!Aus dem Tagebuch einer Weltdame1895-10-20@\emph{Aus dem Tagebuch einer Weltdame} {[}1895-10-20{]}|pwk}. In: \emph{Wiener Caricaturen}\pwindex{?? Werk@Nicht ermittelte Verfasserinnen und Verfasser!Wiener Caricaturen1881 – 1925@\emph{Wiener Caricaturen} {[}1881 – 1925{]}|pwk}, Jg. 15, Nr. 42,
                        20. 10. 1895, S. 2–3. Nicht so sehr eine Parodie, als
                  eine Satire: Geschildert wird aus der Perspektive einer eher simplen »Dame von
                  Welt«, wie junge Mädchen nicht durch den Besuch der \emph{Liebelei}\pwindex{Schnitzler, Arthur 15.05.1862 – 21.10.1931@\textsc{Schnitzler, Arthur} (15.05.1862 – 21.10.1931), \emph{Schriftsteller, Mediziner}!Liebelei. Schauspiel in drei Akten1895-10-09@\strich\emph{Liebelei. Schauspiel in drei Akten} {[}1895-10-09{]}|pwk}, sondern durch Gespräche in der »stillen
                     Häuslichkeit« in sittliche Gefahr geraten.}}}\label{K_L02758-24h} iſt nicht übel
               gemacht. Daß \label{K_L02758-25v}\edtext{\textsc{Granichstaedten\pwindex{Granichstaedten, Emil 1847-07-08 – 1904-07-02@\textsc{Granichstaedten, Emil} (1847-07-08 – 1904-07-02), \emph{Journalist, Wissenschaftler}!Deutsches Volkstheater. (»Ein Regentag«, Charakterbild von J. J.
                  David.)1895-10-15@\strich\emph{Deutsches Volkstheater. (»Ein Regentag«, Charakterbild von J. J. David.)} {[}1895-10-15{]}|pwv}\pwindex{Granichstaedten, Emil 1847-07-08 – 1904-07-02@\textsc{Granichstaedten, Emil} (1847-07-08 – 1904-07-02), \emph{Journalist, Wissenschaftler}|pw}}}{\lemma{\textnormal{\emph{Granichstaedten}}}\Cendnote{\textnormal{Bezug womöglich auf diese Stelle:
                     »Werden alle die Redlichen, welche das Glück hatten, an Schnitzler\pwindex{Schnitzler, Arthur 15.05.1862 – 21.10.1931@\textsc{Schnitzler, Arthur} (15.05.1862 – 21.10.1931), \emph{Schriftsteller, Mediziner}|pw}’s ›Liebelei\pwindex{Schnitzler, Arthur 15.05.1862 – 21.10.1931@\textsc{Schnitzler, Arthur} (15.05.1862 – 21.10.1931), \emph{Schriftsteller, Mediziner}!Liebelei. Schauspiel in drei Akten1895-10-09@\strich\emph{Liebelei. Schauspiel in drei Akten} {[}1895-10-09{]}|pw}‹ Gefallen zu finden, nun auch für David\pwindex{David, Jakob Julius 1859-02-06 – 1906-11-20@\textsc{David, Jakob Julius} (1859-02-06 – 1906-11-20), \emph{Schriftsteller, Journalist}|pw}’s ›Ein Regentag\pwindex{David, Jakob Julius 1859-02-06 – 1906-11-20@\textsc{David, Jakob Julius} (1859-02-06 – 1906-11-20), \emph{Schriftsteller, Journalist}!Regentag. Charakterbild12. 10. 1895@\strich\emph{Ein Regentag. Charakterbild} {[}12. 10. 1895{]}|pw}‹ das Wort ergreifen und das Lob eines Dichters singen,
                     der sein Werk aus seiner Seele geholt und mit der Beredtsamkeit seines Herzens
                     geschmückt hat? — Mag es gelten, daß man jedes Streben mit Wohlwollen fördern
                     soll. Aber warum offenbart sich dieses Wohlwollen nicht gleich beglückend und
                     gleich allgemein und kräftig bei dem armen Poeten, der nicht die Zeit hat, so
                     viele gewiß redliche Freunde gewiß redlich zu gewinnen, der nicht in der Lage
                     ist, auch in der Gesellschaft als interessanter junger Mann eine Stellung zu
                     haben? Nicht darin liegt die Gefährlichkeit der Camaraderie, daß sie kleine
                     Talente aufbläht, sondern darin, daß sie damit echten Talenten den Weg
                     erschwert, wol auch versperrt. Es ist so leicht, ein ›lieber Kerl‹ zu sein, und
                     die ›lieben Kerle‹ wissen gar nicht, wie viel himmelschreiendes Unrecht sie
                     täglich verschulden.« Emil Granichstaedten\pwindex{Granichstaedten, Emil 1847-07-08 – 1904-07-02@\textsc{Granichstaedten, Emil} (1847-07-08 – 1904-07-02), \emph{Journalist, Wissenschaftler}|pwk}: \emph{Deutsches Volkstheater. (»Ein Regentag«, Charakterbild von
                        J. J. David.)}\pwindex{Granichstaedten, Emil 1847-07-08 – 1904-07-02@\textsc{Granichstaedten, Emil} (1847-07-08 – 1904-07-02), \emph{Journalist, Wissenschaftler}!Deutsches Volkstheater. (»Ein Regentag«, Charakterbild von J. J.
                  David.)1895-10-15@\strich\emph{Deutsches Volkstheater. (»Ein Regentag«, Charakterbild von J. J. David.)} {[}1895-10-15{]}|pwk}. In: \emph{Die Presse}\pwindex{?? Werk@Nicht ermittelte Verfasserinnen und Verfasser!Presse1848-07-03@\emph{Die Presse} {[}1848-07-03{]}|pwk},
                     Jg. 48, Nr. 283, 15. 10. 1895, S. 1–2, hier:
                     S. 2. }}}\label{K_L02758-25h}{ }\substVorne{}\textsuperscript{jed\textcolor{gray}{e}}\substDazwischen{}jede\substHinten{} nur irgend mögliche Gemeinheit begeht, iſt ſelbſtverſtändlich. Du haſt
               Recht, Dich nicht dabei aufzuhalten. Weiterſchreiben iſt die beſte {\pb}Antwort. Zum Haſſen und zum Bekämpfen ſolcher
               perſönlicher Widerſacher haben nur die unproductiven Leute Zeit\substVorne{}\textsuperscript{,}\substDazwischen{}.\substHinten{}{ }\strikeout{wie z. B.} Nur den \textsc{Bahr\pwindex{Bahr, Hermann 19.07.1863 – 15.01.1934@\textsc{Bahr, Hermann} (19.07.1863 – 15.01.1934), \emph{Schriftsteller, Kritiker}|pw}} würde ich an Deiner Stelle doch einſalzen. Das iſt nämlich eine Maßnahme von
               Hygiene des alltäglichen Lebens. Der Burſch\pwindex{Bahr, Hermann 19.07.1863 – 15.01.1934@\textsc{Bahr, Hermann} (19.07.1863 – 15.01.1934), \emph{Schriftsteller, Kritiker}|pwv} darf Dir nicht mehr ins Haus, es muß ein deutlicher
               und klarer Bruch zwiſchen Dir und ihm ſein. Was haſt Du ihm auf das infame \label{K_L02758-4v}\edtext{Billet}{\lemma{\textnormal{\emph{Billet}}}\Cendnote{\textnormal{Gemeint ist die herzliche Gratulation, trotz der mehr als
                  distanzierten Kritik\pwindex{Bahr, Hermann 19.07.1863 – 15.01.1934@\textsc{Bahr, Hermann} (19.07.1863 – 15.01.1934), \emph{Schriftsteller, Kritiker}!Burgtheater (Liebelei, Schauspiel in drei Acten von Arthur Schnitzler. Rechte
                  der Seele, Schauspiel in einem Act von Guiseppe Giacosa. Zum ersten Mal aufgefuehrt
                  am 9. October)1895-10-12@\strich\emph{Burgtheater (Liebelei, Schauspiel in drei Acten von Arthur Schnitzler. Rechte der Seele, Schauspiel in einem Act von Guiseppe Giacosa. Zum ersten Mal aufgeführt am 9. October)} {[}1895-10-12{]}|pwkv} der
                     \emph{Liebelei}\pwindex{Schnitzler, Arthur 15.05.1862 – 21.10.1931@\textsc{Schnitzler, Arthur} (15.05.1862 – 21.10.1931), \emph{Schriftsteller, Mediziner}!Liebelei. Schauspiel in drei Akten1895-10-09@\strich\emph{Liebelei. Schauspiel in drei Akten} {[}1895-10-09{]}|pwk} (Hermann Bahr an Arthur Schnitzler, [12. 10. 1895]).}}}\label{K_L02758-4h} geantwortet, das er Dir nach ſeiner Kritik\pwindex{Bahr, Hermann 19.07.1863 – 15.01.1934@\textsc{Bahr, Hermann} (19.07.1863 – 15.01.1934), \emph{Schriftsteller, Kritiker}!Burgtheater (Liebelei, Schauspiel in drei Acten von Arthur Schnitzler. Rechte
                  der Seele, Schauspiel in einem Act von Guiseppe Giacosa. Zum ersten Mal aufgefuehrt
                  am 9. October)1895-10-12@\strich\emph{Burgtheater (Liebelei, Schauspiel in drei Acten von Arthur Schnitzler. Rechte der Seele, Schauspiel in einem Act von Guiseppe Giacosa. Zum ersten Mal aufgeführt am 9. October)} {[}1895-10-12{]}|pwv}{ }{\pb}zu ſchreiben die Frechheit h\substVorne{}\textsuperscript{\textcolor{gray}{e}}\substDazwischen{}a\substHinten{}tte?\pend
           \pstart
           \label{K_L02758-31v}\edtext{\textsc{Berger\pwindex{Berger, Alfred von 30.04.1853 – 24.08.1912@\textsc{Berger, Alfred von} (30.04.1853 – 24.08.1912), \emph{Schriftsteller, Journalist, Theaterleiter}|pw}s}{ }Feuilleton\pwindex{Berger, Alfred von 30.04.1853 – 24.08.1912@\textsc{Berger, Alfred von} (30.04.1853 – 24.08.1912), \emph{Schriftsteller, Journalist, Theaterleiter}!Burgtheater [Rechte der Seele, Liebelei]1895-10-14@\strich\emph{Burgtheater [Rechte der Seele, Liebelei]} {[}1895-10-14{]}|pwv}}{\lemma{\textnormal{\emph{Bergers Feuilleton}}}\Cendnote{\textnormal{Alfred Freiherr von Berger\pwindex{Berger, Alfred von 30.04.1853 – 24.08.1912@\textsc{Berger, Alfred von} (30.04.1853 – 24.08.1912), \emph{Schriftsteller, Journalist, Theaterleiter}|pwk}: \emph{Burgtheater}\pwindex{Berger, Alfred von 30.04.1853 – 24.08.1912@\textsc{Berger, Alfred von} (30.04.1853 – 24.08.1912), \emph{Schriftsteller, Journalist, Theaterleiter}!Burgtheater [Rechte der Seele, Liebelei]1895-10-14@\strich\emph{Burgtheater [Rechte der Seele, Liebelei]} {[}1895-10-14{]}|pwk}. In: \emph{Montags-Revue}\pwindex{?? Werk@Nicht ermittelte Verfasserinnen und Verfasser!Montags-Revue. Wochenschrift fuer Politik, Finanzen, Kunst und
                  Literatur1874 – 1915@\emph{Montags-Revue. Wochenschrift für Politik, Finanzen, Kunst und Literatur} {[}1874 – 1915{]}|pwk}, Jg. 26, Nr. 41, 14. 10. 1895, S. 1–4.}}}\label{K_L02758-31h} haſt Du mir leider nicht
               geſchickt.\pend
           \pstart
           Daran, daß die Leute Deinen Erfolg Deinen Freunden und Beziehungen zuſchreiben, wirſt
               Du Dich gewöhnen müſſen. Das Geſindel \strikeout{d} kann doch
               nicht rückhaltslos loben; irgend etwas Geringſchätzendes müſſen ſie einfließen
               laſſen. So haben ſie das gefunden. Beim nächſten Erfolg werden ſie ſchon auf etwas
               Neues kommen. Das Alles hat aber nicht die geringſte Bedeutung, {\pb}und mit all’ ihrer Gemeinheit, vorn herum oder
               hinten herum, können ſie Dir nichts Weſentliches \strikeout{rauben.} rauben.\pend
           \pstart
           \textsc{Herzl\pwindex{Herzl, Theodor 1860-05-02 – 1904-07-03@\textsc{Herzl, Theodor} (1860-05-02 – 1904-07-03), \emph{Schriftsteller, Journalist}|pw}} war bei mir und ſagte über Dich \strikeout{wohl\textcolor{gray}{×}} wohlwollend: »Der iſt jetzt der größte Dichter von Wien\oindex{Wien@\textbf{Wien}|pw}«. Auch dieſen wirſt Du bald auf der Gegenſeite finden. Oh
               was für ein widerliches Subject\pwindex{Herzl, Theodor 1860-05-02 – 1904-07-03@\textsc{Herzl, Theodor} (1860-05-02 – 1904-07-03), \emph{Schriftsteller, Journalist}|pwv}! Ich habe nicht die Kraft \strikeout{verhehlt,
                  ihn} gehabt, ihm diesmal den abſtoßenden Eindruck zu verbergen, den er mir
               machte.\pend
           \pstart
           {\pb}Auch \textsc{Sudermann\pwindex{Sudermann, Hermann 30.09.1857 – 21.11.1928@\textsc{Sudermann, Hermann} (30.09.1857 – 21.11.1928), \emph{Schriftsteller}|pw}} iſt mir nicht ſympathiſch. Freilich iſt er zu Dir anders\textcolor{gray}{,}
               wie zu mir. Aber dieſe ſeine Einfachheit \strikeout{iſt eine} iſt
               eine gemachte; und er iſt ſogar eitel darauf, der ſchöne Mann zu ſein. Auch bin ich
               überzeugt, bei \strikeout{Fra} Frauen ſpielt er den Räthſelhaften
               und Dämoniſchen.\pend
           \pstart
           Haſt Du nun wirklich die »Liebelei\pwindex{Schnitzler, Arthur 15.05.1862 – 21.10.1931@\textsc{Schnitzler, Arthur} (15.05.1862 – 21.10.1931), \emph{Schriftsteller, Mediziner}!Liebelei. Schauspiel in drei Akten1895-10-09@\strich\emph{Liebelei. Schauspiel in drei Akten} {[}1895-10-09{]}|pw}« für Dich
               umgearbeitet? Und was macht das neue Stück\pwindex{Schnitzler, Arthur 15.05.1862 – 21.10.1931@\textsc{Schnitzler, Arthur} (15.05.1862 – 21.10.1931), \emph{Schriftsteller, Mediziner}!Freiwild. Schauspiel in 3 Akten1896@\strich\emph{Freiwild. Schauspiel in 3 Akten} {[}1896{]}|pwv}? Werde ich es im Manuſkript zu ſehen
               bekommen, auf {\pb}einen Tag, wie immer? Und was \label{K_L02758-6v}\edtext{ſchreibſt Du ſonſt}{\lemma{\textnormal{\emph{ſchreibſt Du ſonſt}}}\Cendnote{\textnormal{Schnitzler\pwindex{Schnitzler, Arthur 15.05.1862 – 21.10.1931@\textsc{Schnitzler, Arthur} (15.05.1862 – 21.10.1931), \emph{Schriftsteller, Mediziner}|pwk} arbeitete am \emph{Freiwild}\pwindex{Schnitzler, Arthur 15.05.1862 – 21.10.1931@\textsc{Schnitzler, Arthur} (15.05.1862 – 21.10.1931), \emph{Schriftsteller, Mediziner}!Freiwild. Schauspiel in 3 Akten1896@\strich\emph{Freiwild. Schauspiel in 3 Akten} {[}1896{]}|pwk}, ein Schauspiel, mit dem er zu diesem Zeitpunkt
                  sehr unzufrieden war (vgl. A. S.: \emph{Tagebuch}, 2. 12. 1895). Am 5. 12. 1895 begann er zudem die Erzählung \emph{Die Frau des Weisen}\pwindex{Schnitzler, Arthur 15.05.1862 – 21.10.1931@\textsc{Schnitzler, Arthur} (15.05.1862 – 21.10.1931), \emph{Schriftsteller, Mediziner}!Frau des Weisen. Erzaehlung1897-01-02 – 1897-01-16@\strich\emph{Die Frau des Weisen. Erzählung} {[}1897-01-02 – 1897-01-16{]}|pwk} neu.}}}\label{K_L02758-6h}? Und wie und mit wem lebſt
               Du? Was macht die große Tragödin\pwindex{Sandrock, Adele 1863-08-19 – 1937-08-30@\textsc{Sandrock, Adele} (1863-08-19 – 1937-08-30), \emph{Schauspielerin}|pwv}? Wie lange wird die »Liebelei\pwindex{Schnitzler, Arthur 15.05.1862 – 21.10.1931@\textsc{Schnitzler, Arthur} (15.05.1862 – 21.10.1931), \emph{Schriftsteller, Mediziner}!Liebelei. Schauspiel in drei Akten1895-10-09@\strich\emph{Liebelei. Schauspiel in drei Akten} {[}1895-10-09{]}|pw}«
               noch geſpielt werden? Der Erfolg iſt phänomenal. Haſt Du viel Geld verdient? Und das
               ſparſt Du doch hoffentlich? Haſt Du die ſechs \strikeout{E}{ }\label{K_L02758-88v}\edtext{Ausſchnitte}{\lemma{\textnormal{\emph{Ausſchnitte}}}\Cendnote{\textnormal{Beilage nicht erhalten. Eventuell Teile der bis
                     28. 11. 1895 in acht Folgen abgedruckten Übersetzung\pwindex{Schnitzler, Arthur 15.05.1862 – 21.10.1931@\textsc{Schnitzler, Arthur} (15.05.1862 – 21.10.1931), \emph{Schriftsteller, Mediziner}!petite comedie. Mœurs viennois1895-11-19 – 1895-11-28@\strich\emph{La petite comédie. Mœurs viennois} {[}1895-11-19 – 1895-11-28{]}|pwkv} von \emph{Die kleine Komödie}\pwindex{Schnitzler, Arthur 15.05.1862 – 21.10.1931@\textsc{Schnitzler, Arthur} (15.05.1862 – 21.10.1931), \emph{Schriftsteller, Mediziner}!kleine Komoedie1895-08-01@\strich\emph{Die kleine Komödie} {[}1895-08-01{]}|pwk}, \emph{La
                     petite comédie}\pwindex{Schnitzler, Arthur 15.05.1862 – 21.10.1931@\textsc{Schnitzler, Arthur} (15.05.1862 – 21.10.1931), \emph{Schriftsteller, Mediziner}!petite comedie. Mœurs viennois1895-11-19 – 1895-11-28@\strich\emph{La petite comédie. Mœurs viennois} {[}1895-11-19 – 1895-11-28{]}|pwk}. }}}\label{K_L02758-88h} aus der »\textsc{Liberté\pwindex{?? Werk@Nicht ermittelte Verfasserinnen und Verfasser!Liberte1865-07-16 – 1940-06-11@\emph{La Liberté} {[}1865-07-16 – 1940-06-11{]}|pw}}« erhalten, die ich Dir ſenden ließ? Was macht die Frau \textsc{Lou Andreas\pwindex{Andreas-Salome, Lou 12.02.1861 – 05.02.1937@\textsc{Andreas-Salomé, Lou} (12.02.1861 – 05.02.1937), \emph{Schriftstellerin}|pw}}? Was {\pb}macht \textsc{Richard\pwindex{Beer-Hofmann, Richard 1866-07-11 – 1945-09-26@\textsc{Beer-Hofmann, Richard} (1866-07-11 – 1945-09-26), \emph{Schriftsteller}|pw}}? Arbeitet er? Wird was von ihm erſcheinen? {\dotsfive}\pend
           \pstart
           Wir Zwei! In einem Deiner Briefe befindet ſich eine lange und rührende Stelle
               darüber, die mich jetzt beim Wiederleſen nicht weniger bewegt, als beim \strikeout{A\textcolor{gray}{×}f}
               erſten Mal. Es iſt lieb, daß Du Dir ſolche Mühe gibſt, mir die ſchlimmen Dinge
               auszureden. Sprechen muß ich Dir davon, denn ich bin Dir Ehrlichkeit ſchuldig. Von
               Dir aus iſt gewiß nichts zu befürchten. Du wirſt {\pb}Dich nicht ändern, was auch kommen mag, und wirſt einfach und treu bleiben. Aber in
               mir ſitzt das Übel. Ich habe die Empfindung – und ſie kehrt immer wieder, trotz allen
               Ankämpfens dagegen – daß Du mir auf einmal ferner gerückt biſt, als je, daß Du und
               ich jetzt auf zwei ganz verſchiedenen Lebensgefilden ſtehen, die weiter auseinander
               liegen, als \strikeout{\textcolor{gray}{fe}}{ }Wien\oindex{Wien@\textbf{Wien}|pw} und \textsc{Paris\oindex{Paris@\textbf{Paris}|pw}}, und \strikeout{w} durch etwas Weiteres getrennt ſind, als
               durch einen Raum von fünf Jahren. Du und ich, \strikeout{w} wir
               werden jetzt zwei {\pb}verſchiedene Leben führen. Das
                  \strikeout{\textcolor{gray}{×}} kommt nicht plötzlich, aber ganz \strikeout{all} allmälig,
               ganz unmerklich. Du wirſt oben leben, und ich unten. Derjenige aber, der unten
               bleibt, bemerkt die Veränderung immer zuerſt. Ich \strikeout{\textcolor{gray}{b}} habe die Empfindung, daß Du \strikeout{mir} mir langſam
               entrückt wirſt, und daß ich Dir nicht nach kann. Ich denke \strikeout{\textcolor{gray}{no}ch} mir, daß ich ein Stadium in Deinem Daſein war, daß
               ſich Dein Leben von mir weg weiter entwickelt: denn mein Leben \strikeout{ent} entwickelt ſich {\pb}nicht, und ich bleibe ſtehen. Ich meine, daß Du mich nicht mehr brauchſt, und daß
               meine Rolle \label{K_L02758-8v}\edtext{\textsc{\begin{otherlanguage}{french}auprès de ta personne\end{otherlanguage}}}{\lemma{\textnormal{\emph{auprès de ta personne}}}\Cendnote{\textnormal{französisch: im Bezug auf Deine
                  Person}}}\label{K_L02758-8h} ausgeſpielt iſt. Ich ſehe Dich weit, weit weg von mir. Schreib’ mir,
               was Du willſt, ich kann mir nicht helfen: ich \uline{ſehe}
               Dich eben ſo. Ich weiß, daß Du die größten Kraftanſtrengungen machen wirſt, um mich
               mit Dir zu nehmen; aber ich weiß, daß {\pb}keine Kraft
               da nützen kann, weil es ein \uline{Geſetz} iſt, daß ich
               zurückbleiben muß.\pend
           \pstart
           Ich drücke das Alles ſchlecht aus. Es iſt heut wieder
               ein ſchlimmer Tag. Ich ſitze mit ſchwerem Kopfe da, und habe mich eine Nacht
               ſchlaflos herumgewälzt, in Seelenqualen. Die Arbeit habe ich ſatt. Habs wieder einmal
               mit dem Leben verſuchen wollen. Oh, was für eine Sehnſucht ich danach habe, nach dem
               heißen, lebendigen {\pb}Leben! Nicht vorwärtskommen,
               gut! Der Ehrgeiz und das Alles iſt doch nur künſtlich! Aber leben! Und da iſt ein
               ſüßes \label{K_L02758-111v}\edtext{Kind\pwindex{?? [Junge Frau, in die Goldmann Dezember 1895 verliebt ist] @\textsc{?? [Junge Frau, in die Goldmann Dezember 1895 verliebt ist]}|pwv}}{\lemma{\textnormal{\emph{Kind}}}\Cendnote{\textnormal{nicht identifiziert}}}\label{K_L02758-111h}, die der
               liebe Herrgott für mich geſchaffen hat\substVorne{}\textsuperscript{.}\substDazwischen{},\substHinten{}{ }\textsc{Grisette\pwindex{?? [Junge Frau, in die Goldmann Dezember 1895 verliebt ist] @\textsc{?? [Junge Frau, in die Goldmann Dezember 1895 verliebt ist]}|pwv}} oder ſo etwas. Aber ſie kann mich nicht lieben, weil ich nicht jung bin und
               kein feuriger Liebhaber. Und da es nun nichts wird und da alle Sehnſucht wieder
               einmal vergeblich war, entdecke ich, daß ich im Innern ſtets eine Angſt davor {\pb}gehabt habe, es könne doch wahr werden und mir doch
                  gelingen! {\dotsfour}\pend
           \pstart
           Grüß’ Dich Gott, mein lieber Freund!\pend
           \pstart
           Dein {\\[\baselineskip]}treuer {\\[\baselineskip]}\spacefill\mbox{Paul Goldmnn}\pend
           \leftskip=0em{}\pstart
           \noindent{}Schreib’ bald!\pend
           
         
         \endnumbering\mylabel{h}\end{ledgroupsized}  \newcommand{\dateiname}{L02758}\newcommand{\titel}{Paul Goldmann an Arthur Schnitzler, 5. 12. [1895]}\newcommand{\editorInnen}{Martin Anton Müller und Laura Untner}%% latex-leseansicht-abspann.tex
%% Abspann für die Leseansicht.
%% Der Schalter \ifkorrekturansicht ist bereits durch den Vorspann gesetzt.

%% latex-abspann.tex
%% Gemeinsamer Abspann für Korrekturansicht und Leseansicht.
%% Setzt den Schalter \ifkorrekturansicht voraus (gesetzt in den
%% einbindenden Dateien latex-korrekturansicht-abspann.tex bzw.
%% latex-leseansicht-abspann.tex).
%% ---------------------------------------------------------------

\normalsize

% Das esempio-Environment wird nur in der Leseansicht benötigt
\ifkorrekturansicht\else
\newenvironment{esempio}[3]%
{
    \vspace{1.5ex}
    \rlap{\underline{#1}}
    \par
    \setlength{\parindent}{0cm}
    \nopagebreak
    \leftskip=#2cm
    \rightskip=#3cm
}
{
    \par
}
\fi

\doendnotes{C}
\bigskip
\vfill

\clearpage

\footnotesize

\ifkorrekturansicht
  \lohead{\textsc{register}}
\fi

% theindex-Environment neu definieren ohne reledmac
\makeatletter
\renewenvironment{theindex}{%
  \ifkorrekturansicht
    \section*{\indexname}%
  \else
    \subsubsection*{Index der erwähnten Entitäten}%
  \fi
  \setlength{\parindent}{0pt}%
  \setlength{\parskip}{0pt plus 0.3pt}%
  \let\item\@idxitem
}{%
  \ifkorrekturansicht\clearpage\fi
}
\makeatother

\IfFileExists{\jobname-pw.ind}{\input{\jobname-pw.ind}}{}

% Quellenangabe nur in der Leseansicht
\ifkorrekturansicht\else
% Fallback-Definitionen, falls die .tex-Datei \titel etc. nicht gesetzt hat
\providecommand{\titel}{}
\providecommand{\editorInnen}{}
\providecommand{\dateiname}{\jobname}

\vspace{3cm}

\vfill

\footnotesize
\textsc{Quelle}: \titel. Herausgegeben von {\editorInnen}. In: \emph{Arthur Schnitzler: Briefwechsel mit Autorinnen und Autoren}.
 Digitale Edition, https://schnitzler-briefe.acdh.oeaw.ac.at/{\dateiname}.html (Stand \today)
\fi

\end{document}


      