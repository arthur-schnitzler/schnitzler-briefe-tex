%% latex-leseansicht-vorspann.tex
%% Vorspann für die Leseansicht.
%% Lädt die gemeinsame Datei latex-vorspann.tex mit nicht gesetztem Schalter.

\newif\ifkorrekturansicht
\korrekturansichtfalse

\input{../tex-inputs/latex-vorspann}


\section[Paul Goldmann an Arthur Schnitzler, 5. 12. [1895]]{L02758 Paul Goldmann an Arthur Schnitzler, 5. 12. [1895]}
\nopagebreak\mylabel{L02758v}
\rehead{ }\normalsize\beginnumbering\briefempfaengerindex{Schnitzler, Arthur@\textsc{Schnitzler, Arthur}!zzzGoldmann, Paul@\emph{von Paul Goldmann}!1895-12-051@{5. 12. [1895]}|(be}
\toendnotes[C]{\smallbreak\pagebreak[2]}
\correspDesc{Versand  durch Paul Goldmann am 5. 12. [1895] in Paris
\newline{}Erhalt  durch Arthur Schnitzler im Zeitraum [6. 12. 1895
                  – 10. 12. 1895?] in Wien}\toendnotes[C]{\smallbreak}
\Standort{DLA, A:Schnitzler, HS.NZ85.1.3165.}
\physDesc{Brief, 4 Blätter, 16 Seiten, 6803 Zeichen
\newline{}Handschrift: blaue Tinte, deutsche Kurrent
\newline{}Schnitzler: 1) mit Bleistift das Jahr »95« vermerkt  2) mit rotem Buntstift acht Unterstreichungen und eine seitliche Markierung}\toendnotes[C]{\smallbreak}
\pstart
           {\pb}\textcolor{gray}{\textbf{\textbf{Frankfurter Zeitung\orgindex{Frankfurter Zeitung@Frankfurter Zeitung|pw}}}}\pend
           
\pstart
           \textcolor{gray}{\textbf{(\begin{otherlanguage}{french}Gazette de Francfort\end{otherlanguage}\orgindex{Frankfurter Zeitung@Frankfurter Zeitung|pw}).}}\pend
           
\pstart
           \textcolor{gray}{\textbf{\textbf{\begin{otherlanguage}{french}Fondateur M. L.
                              Sonnemann\pwindex{Sonnemann, Leopold 29.\,10.\,1831 Höchberg – 30.\,10.\,1909 Frankfurt am Main@\textsc{Sonnemann, Leopold} (29.\,10.\,1831 Höchberg – 30.\,10.\,1909 Frankfurt am Main), \emph{Journalist, Herausgeber}|pw}\end{otherlanguage}.}}}\pend
           
\pstart
           \begin{otherlanguage}{french}\textcolor{gray}{\textbf{Journal politique, financier,}}\end{otherlanguage}\pend
           
\pstart
           \begin{otherlanguage}{french}\textcolor{gray}{\textbf{commercial et littéraire.}}\end{otherlanguage}\pend
           
\pstart
           \begin{otherlanguage}{french}\textcolor{gray}{\textbf{\textbf{Paraissant trois fois par jour.}}}\end{otherlanguage}\pend
           
\pstart
           \begin{otherlanguage}{french}\textcolor{gray}{\textbf{\textbf{Bureau à Paris\oindex{Paris@\textbf{Paris}, \emph{Hauptstadt}|pw}:}}}\end{otherlanguage}\pend
           
\pstart
           \begin{otherlanguage}{french}\textcolor{gray}{\textbf{\textbf{24. Rue Feydeau\oindex{rue Feydeau@\textbf{rue Feydeau}, \emph{Straße}|pw}.}}}\end{otherlanguage}\hfill \textsc{Paris\oindex{Paris@\textbf{Paris}, \emph{Hauptstadt}|pw}}, 5. December.\pend
           
\pstart\center{}Mein lieber Freund,\pend\vspace{0.5em}
\pstart
           In Angelegenheit der Aufführung von »Liebelei\pwindex{Schnitzler, Arthur 15.\,5.\,1862 Wien – 21.\,10.\,1931 ebd.@\textsc{Schnitzler, Arthur} (15.\,5.\,1862 Wien – 21.\,10.\,1931 ebd.), \emph{Schriftsteller, Mediziner}!Liebelei. Schauspiel in drei Akten@\strich\emph{Liebelei. Schauspiel in drei Akten}|pw}«
               in \textsc{Paris\oindex{Paris@\textbf{Paris}, \emph{Hauptstadt}|pw}} habe ich geſtern einen Schritt gethan, den ich
               längſt thun wollte. Ich war bei \textsc{Jean Thorel\pwindex{Thorel, Jean 11.\,9.\,1859 Éragny – 20.\,8.\,1916 Enghien-les-Bains@\textsc{Thorel, Jean} (11.\,9.\,1859 Éragny – 20.\,8.\,1916 Enghien-les-Bains), \emph{Übersetzer, Dramatiker}|pw}}, deſſen Namen Du gewiß kennſt. Sehr braver u. gewiſſenhafter Menſch\pwindex{Thorel, Jean 11.\,9.\,1859 Éragny – 20.\,8.\,1916 Enghien-les-Bains@\textsc{Thorel, Jean} (11.\,9.\,1859 Éragny – 20.\,8.\,1916 Enghien-les-Bains), \emph{Übersetzer, Dramatiker}|pwv}, wenig Künſtler, großer Freund \textsc{Hauptmann\pwindex{Hauptmann, Gerhart 15.\,11.\,1862 Szczawno-Zdrój – 6.\,6.\,1946 Jagniątków@\textsc{Hauptmann, Gerhart} (15.\,11.\,1862 Szczawno-Zdrój – 6.\,6.\,1946 Jagniątków), \emph{Schriftsteller}|pw}}s\pwindex{Thorel, Jean 11.\,9.\,1859 Éragny – 20.\,8.\,1916 Enghien-les-Bains@\textsc{Thorel, Jean} (11.\,9.\,1859 Éragny – 20.\,8.\,1916 Enghien-les-Bains), \emph{Übersetzer, Dramatiker}|pwv}, von dem er die »Weber\pwindex{Hauptmann, Gerhart 15.\,11.\,1862 Szczawno-Zdrój – 6.\,6.\,1946 Jagniątków@\textsc{Hauptmann, Gerhart} (15.\,11.\,1862 Szczawno-Zdrój – 6.\,6.\,1946 Jagniątków), \emph{Schriftsteller}!Weber. Schauspiel aus den vierziger Jahren@\strich\emph{Die Weber. Schauspiel aus den vierziger Jahren}|pw}« u. »\textsc{Hannele\pwindex{Hauptmann, Gerhart 15.\,11.\,1862 Szczawno-Zdrój – 6.\,6.\,1946 Jagniątków@\textsc{Hauptmann, Gerhart} (15.\,11.\,1862 Szczawno-Zdrój – 6.\,6.\,1946 Jagniątków), \emph{Schriftsteller}!Hanneles Himmelfahrt. Traumdichtung in zwei Teilen@\strich\emph{Hanneles Himmelfahrt. Traumdichtung in zwei Teilen}|pw}}« für die Pariſ\oindex{Paris@\textbf{Paris}, \emph{Hauptstadt}|pw}er Aufführung überſetzt hat,
                  \textsc{Intimus\pwindex{Thorel, Jean 11.\,9.\,1859 Éragny – 20.\,8.\,1916 Enghien-les-Bains@\textsc{Thorel, Jean} (11.\,9.\,1859 Éragny – 20.\,8.\,1916 Enghien-les-Bains), \emph{Übersetzer, Dramatiker}|pwv}}{ }\strikeout{\textcolor{gray}{v}} von \textsc{Antoine\pwindex{Antoine, André 31.\,1.\,1858 Limoges – 23.\,10.\,1943 Le Pouliguen@\textsc{Antoine, André} (31.\,1.\,1858 Limoges – 23.\,10.\,1943 Le Pouliguen), \emph{Theaterleiter, Schauspieler}|pw}}{ }\textsc{etc.} Ich habe ihm von Deinem Stück\pwindex{Schnitzler, Arthur 15.\,5.\,1862 Wien – 21.\,10.\,1931 ebd.@\textsc{Schnitzler, Arthur} (15.\,5.\,1862 Wien – 21.\,10.\,1931 ebd.), \emph{Schriftsteller, Mediziner}!Liebelei. Schauspiel in drei Akten@\strich\emph{Liebelei. Schauspiel in drei Akten}|pwv} geſprochen, \label{K_L02758-1v}\edtext{\begin{otherlanguage}{french}\textsc{il est très – emballé là-dessus}\end{otherlanguage}}{\lemma{\textnormal{\emph{il … là-dessus}}}\Cendnote{\textnormal{französisch: er ist sehr dafür
                  eingenommen}}}\label{K_L02758-1}, will es gern \label{K_L02758-2v}\edtext{überſetzen}{\lemma{\textnormal{\emph{übersetzen}}}\Cendnote{\textnormal{Die Übersetzung wurde,
                  obzwar mit einer Summe von 500 Francs bezahlt, nie fertiggestellt. Am
                     16. 6. 1910 setzte Schnitzler{ }Jean Thorel\pwindex{Thorel, Jean 11.\,9.\,1859 Éragny – 20.\,8.\,1916 Enghien-les-Bains@\textsc{Thorel, Jean} (11.\,9.\,1859 Éragny – 20.\,8.\,1916 Enghien-les-Bains), \emph{Übersetzer, Dramatiker}|pwk} davon in Kenntnis, dass er sich
                  nach vierzehn Jahren nicht mehr an frühere Abmachungen gebunden fühle und er
                  nunmehr über das Recht, \emph{Liebelei}\pwindex{Schnitzler, Arthur 15.\,5.\,1862 Wien – 21.\,10.\,1931 ebd.@\textsc{Schnitzler, Arthur} (15.\,5.\,1862 Wien – 21.\,10.\,1931 ebd.), \emph{Schriftsteller, Mediziner}!Liebelei. Schauspiel in drei Akten@\strich\emph{Liebelei. Schauspiel in drei Akten}|pwk} übersetzen
                  zu lassen und auf die Bühne zu bringen, wieder frei verfüge (\emph{Deutsches Literaturarchiv Marbach},
                  HS.1985.1.2069).}}}\label{K_L02758-2}, unter der Bedingung freilich, daß es zur
               Aufführung {\pb}kommt, will Schritte zur Aufführung bei
               ernſten Theatern thun, verlangt aber baldige Einſendung des Buch\pwindex{Schnitzler, Arthur 15.\,5.\,1862 Wien – 21.\,10.\,1931 ebd.@\textsc{Schnitzler, Arthur} (15.\,5.\,1862 Wien – 21.\,10.\,1931 ebd.), \emph{Schriftsteller, Mediziner}!Liebelei. Schauspiel in drei Akten@\strich\emph{Liebelei. Schauspiel in drei Akten}|pwv}es, im \strikeout{Druk} Druck oder auch im Manuſcript. Wenn es irgend geht,{ }ſende ihm die Sache\pwindex{Schnitzler, Arthur 15.\,5.\,1862 Wien – 21.\,10.\,1931 ebd.@\textsc{Schnitzler, Arthur} (15.\,5.\,1862 Wien – 21.\,10.\,1931 ebd.), \emph{Schriftsteller, Mediziner}!Liebelei. Schauspiel in drei Akten@\strich\emph{Liebelei. Schauspiel in drei Akten}|pwv}, mit einem artigen
               Briefe, deutſch geſchrieben, worin Du Dich entſchuldigſt, daß Du wegen mangelnder
               franzöſiſcher Stylgewandtheit ihm nicht franzöſiſch{ }ſchreibſt. Er wird keine
               glänzende Überſetzung machen; eine gute franzöſiſche Überſetzung bekommſt Du
               überhaupt nicht, da alle überſetzenden Franzoſen mehr oder minder plumpe Handwerker{ }ſind; aber von Allen, die ich kenne, {\pb}wird er die
               Sache noch am Wenigſten verhunzen. Damit erledigt{ }ſich wohl von{ }ſelbſt der Brief des
               jungen \label{K_L02758-3v}\edtext{Mann\pwindex{Riaz, Henri de 1871 Lyon – 1951 Lausanne@\textsc{Riaz, Henri de} (1871 Lyon – 1951 Lausanne), \emph{Dichter}|pwv}es}{\lemma{\textnormal{\emph{Mannes}}}\Cendnote{\textnormal{Henry de Riaz\pwindex{Riaz, Henri de 1871 Lyon – 1951 Lausanne@\textsc{Riaz, Henri de} (1871 Lyon – 1951 Lausanne), \emph{Dichter}|pwk}; von ihm finden sich drei
                  Briefe aus dem Zeitraum 1895–1896 im Nachlass Schnitzlers.}}}\label{K_L02758-3} aus \textsc{Lyon\oindex{Lyon@\textbf{Lyon}|pw}}, der mir{ }ſonſt{ }ſehr gefällt und{ }ſehr ehrlich zu{ }ſein{ }ſcheint. Aber ich habe
               mich nach ihm erkundigt, kein Menſch kennt den Namen,{ }ſelbſt die \textsc{Lyon\oindex{Lyon@\textbf{Lyon}|pw}er} Journaliſten nicht. \strikeout{D\textcolor{gray}{rum}} Drum iſts wohl beſſer,{ }ſich nicht aufs Unſichere einzulaſſen und lieber den
               geraden Weg, d. h. einen bekannten Überſetzer\pwindex{Thorel, Jean 11.\,9.\,1859 Éragny – 20.\,8.\,1916 Enghien-les-Bains@\textsc{Thorel, Jean} (11.\,9.\,1859 Éragny – 20.\,8.\,1916 Enghien-les-Bains), \emph{Übersetzer, Dramatiker}|pwv} zu wählen. Entſchuldige, daß ich den Brief{ }ſolange behalten. Aber
               wüßteſt Du, was Alles in meinen Kopfe rumort hat,{ }ſeitdem!\pend
           
\pstart
           {\pb}Haſt Du an \textsc{Aubry\pwindex{Aubry, Georges †~1923@\textsc{Aubry, Georges} (†~1923), \emph{Redakteur}|pw}} oder Frau\pwindex{Aubry, [MMe. Georges] @\textsc{Aubry, [MMe. Georges]}, \emph{Übersetzerin}|pwv}
               geſchrieben?\pend
           
\pstart
           Die kürzlich zurückgeſandten Druckſachen haben mich intereſſirt, wie alles Übrige.
                  \label{K_L02758-4v}\edtext{\textsc{Wolter\pwindex{Baumfeld, Moriz 6.\,10.\,1868 Wien – 4.\,3.\,1913 New York City@\textsc{Baumfeld, Moriz} (6.\,10.\,1868 Wien – 4.\,3.\,1913 New York City), \emph{Journalist, Rechtswissenschaftler, Theaterleiter}!Bei Charlotte Wolter@\strich\emph{Bei Charlotte Wolter}|pwuv}\pwindex{Wolter, Charlotte 1.\,3.\,1834 Köln – 14.\,6.\,1897 Wien@\textsc{Wolter, Charlotte} (1.\,3.\,1834 Köln – 14.\,6.\,1897 Wien), \emph{Schauspielerin}|pw}}}{\lemma{\textnormal{\emph{Wolter}}}\Cendnote{\textnormal{Wahrscheinlich folgende \emph{home story}, die in Schnitzlers Zeitungsausschnittsammlung an der \emph{University of Exeter} aufbewahrt wird (5. Liebelei, box 10/1): Moriz Baumfeld\pwindex{Baumfeld, Moriz 6.\,10.\,1868 Wien – 4.\,3.\,1913 New York City@\textsc{Baumfeld, Moriz} (6.\,10.\,1868 Wien – 4.\,3.\,1913 New York City), \emph{Journalist, Rechtswissenschaftler, Theaterleiter}|pwk}: \emph{Bei Charlotte Wolter}\pwindex{Baumfeld, Moriz 6.\,10.\,1868 Wien – 4.\,3.\,1913 New York City@\textsc{Baumfeld, Moriz} (6.\,10.\,1868 Wien – 4.\,3.\,1913 New York City), \emph{Journalist, Rechtswissenschaftler, Theaterleiter}!Bei Charlotte Wolter@\strich\emph{Bei Charlotte Wolter}|pwk}. In: \emph{Extrapost}\pwindex{Extrapost. Unparteiische Montags-Zeitung@\emph{Extrapost. Unparteiische Montags-Zeitung}|pwk}, Jg. 14, Nr. 718, 21. 10. 1895, S. 1–2. Darin erzählt Charlotte Wolter\pwindex{Wolter, Charlotte 1.\,3.\,1834 Köln – 14.\,6.\,1897 Wien@\textsc{Wolter, Charlotte} (1.\,3.\,1834 Köln – 14.\,6.\,1897 Wien), \emph{Schauspielerin}|pwk}, dass sie nach einem Jahr erstmals wieder
                  im Theater war und das Pech hatte, \emph{Liebelei}\pwindex{Schnitzler, Arthur 15.\,5.\,1862 Wien – 21.\,10.\,1931 ebd.@\textsc{Schnitzler, Arthur} (15.\,5.\,1862 Wien – 21.\,10.\,1931 ebd.), \emph{Schriftsteller, Mediziner}!Liebelei. Schauspiel in drei Akten@\strich\emph{Liebelei. Schauspiel in drei Akten}|pwk}
                  zu sehen – eine, wie sie fand, völlig kunstlose Arbeit\pwindex{Schnitzler, Arthur 15.\,5.\,1862 Wien – 21.\,10.\,1931 ebd.@\textsc{Schnitzler, Arthur} (15.\,5.\,1862 Wien – 21.\,10.\,1931 ebd.), \emph{Schriftsteller, Mediziner}!Liebelei. Schauspiel in drei Akten@\strich\emph{Liebelei. Schauspiel in drei Akten}|pwkv}.}}}\label{K_L02758-4}, die dumme Gans\pwindex{Wolter, Charlotte 1.\,3.\,1834 Köln – 14.\,6.\,1897 Wien@\textsc{Wolter, Charlotte} (1.\,3.\,1834 Köln – 14.\,6.\,1897 Wien), \emph{Schauspielerin}|pwv}, hat mich beluſtigt, \label{K_L02758-5v}\edtext{\textsc{Ludassy\pwindex{Gans-Ludassy, Julius von 13.\,4.\,1858 Wien – 30.\,9.\,1922 ebd.@\textsc{Gans-Ludassy, Julius von} (13.\,4.\,1858 Wien – 30.\,9.\,1922 ebd.), \emph{Schriftsteller, Journalist, Herausgeber}!Burgtheater. »Rechte der Seele«, Schauspiel in einem Acte von Giuseppe Giacosa: deutsch von Otto Eisenschitz. »Liebelei«, Schauspiel in drei Acten von Arthur Schnitzler. Beide zum erstenmale aufgeführt am 9. October 1895@\strich\emph{Burgtheater. »Rechte der Seele«, Schauspiel in einem Acte von Giuseppe Giacosa: deutsch von Otto Eisenschitz. »Liebelei«, Schauspiel in drei Acten von Arthur Schnitzler. Beide zum erstenmale aufgeführt am 9. October 1895}|pwuv}\pwindex{Gans-Ludassy, Julius von 13.\,4.\,1858 Wien – 30.\,9.\,1922 ebd.@\textsc{Gans-Ludassy, Julius von} (13.\,4.\,1858 Wien – 30.\,9.\,1922 ebd.), \emph{Schriftsteller, Journalist, Herausgeber}|pw}}}{\lemma{\textnormal{\emph{Ludassy}}}\Cendnote{\textnormal{Es könnte sich um den Nachtrag der
                  früheren Kritik\pwindex{Gans-Ludassy, Julius von 13.\,4.\,1858 Wien – 30.\,9.\,1922 ebd.@\textsc{Gans-Ludassy, Julius von} (13.\,4.\,1858 Wien – 30.\,9.\,1922 ebd.), \emph{Schriftsteller, Journalist, Herausgeber}!Burgtheater. »Rechte der Seele«, Schauspiel in einem Acte von Giuseppe Giacosa: deutsch von Otto Eisenschitz. »Liebelei«, Schauspiel in drei Acten von Arthur Schnitzler. Beide zum erstenmale aufgeführt am 9. October 1895@\strich\emph{Burgtheater. »Rechte der Seele«, Schauspiel in einem Acte von Giuseppe Giacosa: deutsch von Otto Eisenschitz. »Liebelei«, Schauspiel in drei Acten von Arthur Schnitzler. Beide zum erstenmale aufgeführt am 9. October 1895}|pwkv} handeln: L\pwindex{Gans-Ludassy, Julius von 13.\,4.\,1858 Wien – 30.\,9.\,1922 ebd.@\textsc{Gans-Ludassy, Julius von} (13.\,4.\,1858 Wien – 30.\,9.\,1922 ebd.), \emph{Schriftsteller, Journalist, Herausgeber}|pwkv} [ = Julius von Gans-Ludassy\pwindex{Gans-Ludassy, Julius von 13.\,4.\,1858 Wien – 30.\,9.\,1922 ebd.@\textsc{Gans-Ludassy, Julius von} (13.\,4.\,1858 Wien – 30.\,9.\,1922 ebd.), \emph{Schriftsteller, Journalist, Herausgeber}|pwk}]: \emph{Burgtheater. »Rechte der Seele«, Schauspiel in einem Acte
                        von Giuseppe Giacosa: deutsch von Otto Eisenschitz. »Liebelei«, Schauspiel
                        in drei Acten von Arthur Schnitzler. Beide zum erstenmale aufgeführt am
                        9. October 1895}\pwindex{Gans-Ludassy, Julius von 13.\,4.\,1858 Wien – 30.\,9.\,1922 ebd.@\textsc{Gans-Ludassy, Julius von} (13.\,4.\,1858 Wien – 30.\,9.\,1922 ebd.), \emph{Schriftsteller, Journalist, Herausgeber}!Burgtheater. »Rechte der Seele«, Schauspiel in einem Acte von Giuseppe Giacosa: deutsch von Otto Eisenschitz. »Liebelei«, Schauspiel in drei Acten von Arthur Schnitzler. Beide zum erstenmale aufgeführt am 9. October 1895@\strich\emph{Burgtheater. »Rechte der Seele«, Schauspiel in einem Acte von Giuseppe Giacosa: deutsch von Otto Eisenschitz. »Liebelei«, Schauspiel in drei Acten von Arthur Schnitzler. Beide zum erstenmale aufgeführt am 9. October 1895}|pwk}. In: \emph{Wiener Allgemeine
                        Zeitung}\pwindex{Wiener Allgemeine Zeitung@\emph{Wiener Allgemeine Zeitung}|pwk}, Nr. 5282, 11. 10. 1895,
                     S. 2–3.}}}\label{K_L02758-5} mag \strikeout{\textcolor{gray}{d}} ich gar nicht – auch Einer, der mit dem Erfolge geht und Dich bei der erſten
               Schwierigkeit im Stich laſſen wird. Die kleine \label{K_L02758-6v}\edtext{Parodie\pwindex{Parodie auf Liebelei / Schnitzler]@\emph{[Parodie auf Liebelei / Schnitzler]}|pwv}}{\lemma{\textnormal{\emph{Parodie}}}\Cendnote{\textnormal{Eventuell der ungezeichnete Text\pwindex{Aus dem Tagebuch einer Weltdame@\emph{Aus dem Tagebuch einer Weltdame}|pwkv}: \emph{Aus dem Tagebuch einer Weltdame}\pwindex{Aus dem Tagebuch einer Weltdame@\emph{Aus dem Tagebuch einer Weltdame}|pwk}. In: \emph{Wiener Caricaturen}\pwindex{Wiener Caricaturen@\emph{Wiener Caricaturen}|pwk}, Jg. 15, Nr. 42,
                        20. 10. 1895, S. 2–3. Nicht so sehr eine Parodie, als
                  eine Satire: Geschildert wird aus der Perspektive einer eher simplen »Dame von
                  Welt«, wie junge Mädchen nicht durch den Besuch von \emph{Liebelei}\pwindex{Schnitzler, Arthur 15.\,5.\,1862 Wien – 21.\,10.\,1931 ebd.@\textsc{Schnitzler, Arthur} (15.\,5.\,1862 Wien – 21.\,10.\,1931 ebd.), \emph{Schriftsteller, Mediziner}!Liebelei. Schauspiel in drei Akten@\strich\emph{Liebelei. Schauspiel in drei Akten}|pwk}, sondern durch Gespräche in der »stillen
                     Häuslichkeit« in sittliche Gefahr geraten.}}}\label{K_L02758-6} iſt nicht übel
               gemacht. Daß \label{K_L02758-7v}\edtext{\textsc{Granichstaedten\pwindex{Granichstaedten, Emil 8.\,7.\,1847 Wien – 2.\,7.\,1904 Berlin@\textsc{Granichstaedten, Emil} (8.\,7.\,1847 Wien – 2.\,7.\,1904 Berlin), \emph{Journalist, Rechtswissenschaftler}!Deutsches Volkstheater. (»Ein Regentag«, Charakterbild von J. J. David.)@\strich\emph{Deutsches Volkstheater. (»Ein Regentag«, Charakterbild von J. J. David.)}|pwv}\pwindex{Granichstaedten, Emil 8.\,7.\,1847 Wien – 2.\,7.\,1904 Berlin@\textsc{Granichstaedten, Emil} (8.\,7.\,1847 Wien – 2.\,7.\,1904 Berlin), \emph{Journalist, Rechtswissenschaftler}|pw}}}{\lemma{\textnormal{\emph{Granichstaedten}}}\Cendnote{\textnormal{Bezug womöglich auf diese Stelle:
                     »Werden alle die Redlichen, welche das Glück hatten, an Schnitzler’s ›Liebelei\pwindex{Schnitzler, Arthur 15.\,5.\,1862 Wien – 21.\,10.\,1931 ebd.@\textsc{Schnitzler, Arthur} (15.\,5.\,1862 Wien – 21.\,10.\,1931 ebd.), \emph{Schriftsteller, Mediziner}!Liebelei. Schauspiel in drei Akten@\strich\emph{Liebelei. Schauspiel in drei Akten}|pw}‹ Gefallen zu finden, nun auch für David\pwindex{David, Jakob Julius 6.\,2.\,1859 Hranice – 20.\,11.\,1906 Wien@\textsc{David, Jakob Julius} (6.\,2.\,1859 Hranice – 20.\,11.\,1906 Wien), \emph{Schriftsteller, Journalist}|pw}’s ›Ein Regentag\pwindex{David, Jakob Julius 6.\,2.\,1859 Hranice – 20.\,11.\,1906 Wien@\textsc{David, Jakob Julius} (6.\,2.\,1859 Hranice – 20.\,11.\,1906 Wien), \emph{Schriftsteller, Journalist}!Regentag. Charakterbild@\strich\emph{Ein Regentag. Charakterbild}|pw}‹ das Wort ergreifen und das Lob eines Dichters singen,
                     der sein Werk aus seiner Seele geholt und mit der Beredtsamkeit seines Herzens
                     geschmückt hat? – Mag es gelten, daß man jedes Streben mit Wohlwollen fördern
                     soll. Aber warum offenbart sich dieses Wohlwollen nicht gleich beglückend und
                     gleich allgemein und kräftig bei dem armen Poeten, der nicht die Zeit hat, so
                     viele gewiß redliche Freunde gewiß redlich zu gewinnen, der nicht in der Lage
                     ist, auch in der Gesellschaft als interessanter junger Mann eine Stellung zu
                     haben? Nicht darin liegt die Gefährlichkeit der Camaraderie, daß sie kleine
                     Talente aufbläht, sondern darin, daß sie damit echten Talenten den Weg
                     erschwert, wol auch versperrt. Es ist so leicht, ein ›lieber Kerl‹ zu sein, und
                     die ›lieben Kerle‹ wissen gar nicht, wie viel himmelschreiendes Unrecht sie
                     täglich verschulden.« Emil Granichstaedten\pwindex{Granichstaedten, Emil 8.\,7.\,1847 Wien – 2.\,7.\,1904 Berlin@\textsc{Granichstaedten, Emil} (8.\,7.\,1847 Wien – 2.\,7.\,1904 Berlin), \emph{Journalist, Rechtswissenschaftler}|pwk}: \emph{Deutsches Volkstheater. (»Ein Regentag«, Charakterbild von
                        J. J. David)}\pwindex{Granichstaedten, Emil 8.\,7.\,1847 Wien – 2.\,7.\,1904 Berlin@\textsc{Granichstaedten, Emil} (8.\,7.\,1847 Wien – 2.\,7.\,1904 Berlin), \emph{Journalist, Rechtswissenschaftler}!Deutsches Volkstheater. (»Ein Regentag«, Charakterbild von J. J. David.)@\strich\emph{Deutsches Volkstheater. (»Ein Regentag«, Charakterbild von J. J. David.)}|pwk}. In: \emph{Die Presse}\pwindex{Presse@\emph{Die Presse}|pwk},
                     Jg. 48, Nr. 283, 15. 10. 1895, S. 1–2, hier:
                     S. 2. }}}\label{K_L02758-7}{ }\substVorne{}\textsuperscript{jed\textcolor{gray}{e}}\substDazwischen{}jede\substHinten{} nur irgend mögliche Gemeinheit begeht, iſt{ }ſelbſtverſtändlich. Du haſt
               Recht, Dich nicht dabei aufzuhalten. Weiterſchreiben iſt die beſte {\pb}Antwort. Zum Haſſen und zum Bekämpfen{ }ſolcher
               perſönlicher Widerſacher haben nur die unproductiven Leute Zeit\substVorne{}\textsuperscript{,}\substDazwischen{}.\substHinten{}{ }\strikeout{wie z. B.} Nur den \textsc{Bahr\pwindex{Bahr, Hermann 19.\,7.\,1863 Linz – 15.\,1.\,1934 München@\textsc{Bahr, Hermann} (19.\,7.\,1863 Linz – 15.\,1.\,1934 München), \emph{Schriftsteller, Kritiker}|pw}} würde ich an Deiner Stelle doch einſalzen. Das iſt nämlich eine Maßnahme von
               Hygiene des alltäglichen Lebens. Der Burſch\pwindex{Bahr, Hermann 19.\,7.\,1863 Linz – 15.\,1.\,1934 München@\textsc{Bahr, Hermann} (19.\,7.\,1863 Linz – 15.\,1.\,1934 München), \emph{Schriftsteller, Kritiker}|pwv} darf Dir nicht mehr ins Haus, es muß ein deutlicher
               und klarer Bruch zwiſchen Dir und ihm{ }ſein. Was haſt Du ihm auf das infame \label{K_L02758-8v}\edtext{Billet}{\lemma{\textnormal{\emph{Billet}}}\Cendnote{\textnormal{Gemeint ist die herzliche Gratulation, trotz der mehr als
                  distanzierten Kritik\pwindex{Bahr, Hermann 19.\,7.\,1863 Linz – 15.\,1.\,1934 München@\textsc{Bahr, Hermann} (19.\,7.\,1863 Linz – 15.\,1.\,1934 München), \emph{Schriftsteller, Kritiker}!Burgtheater (Liebelei, Schauspiel in drei Acten von Arthur Schnitzler. Rechte der Seele, Schauspiel in einem Act von Guiseppe Giacosa. Zum ersten Mal aufgeführt am 9. October)@\strich\emph{Burgtheater (Liebelei, Schauspiel in drei Acten von Arthur Schnitzler. Rechte der Seele, Schauspiel in einem Act von Guiseppe Giacosa. Zum ersten Mal aufgeführt am 9. October)}|pwkv} von
                     \emph{Liebelei}\pwindex{Schnitzler, Arthur 15.\,5.\,1862 Wien – 21.\,10.\,1931 ebd.@\textsc{Schnitzler, Arthur} (15.\,5.\,1862 Wien – 21.\,10.\,1931 ebd.), \emph{Schriftsteller, Mediziner}!Liebelei. Schauspiel in drei Akten@\strich\emph{Liebelei. Schauspiel in drei Akten}|pwk}, XXXX Auszeichnungsfehler: Dokument L00505 nicht gefunden.}}}\label{K_L02758-8} geantwortet, das er Dir nach{ }ſeiner Kritik\pwindex{Bahr, Hermann 19.\,7.\,1863 Linz – 15.\,1.\,1934 München@\textsc{Bahr, Hermann} (19.\,7.\,1863 Linz – 15.\,1.\,1934 München), \emph{Schriftsteller, Kritiker}!Burgtheater (Liebelei, Schauspiel in drei Acten von Arthur Schnitzler. Rechte der Seele, Schauspiel in einem Act von Guiseppe Giacosa. Zum ersten Mal aufgeführt am 9. October)@\strich\emph{Burgtheater (Liebelei, Schauspiel in drei Acten von Arthur Schnitzler. Rechte der Seele, Schauspiel in einem Act von Guiseppe Giacosa. Zum ersten Mal aufgeführt am 9. October)}|pwv}{ }{\pb}zu{ }ſchreiben die Frechheit h\substVorne{}\textsuperscript{\textcolor{gray}{e}}\substDazwischen{}a\substHinten{}tte?\pend
           
\pstart
           \label{K_L02758-9v}\edtext{\textsc{Bergers\pwindex{Berger, Alfred von 30.\,4.\,1853 Wien – 24.\,8.\,1912 ebd.@\textsc{Berger, Alfred von} (30.\,4.\,1853 Wien – 24.\,8.\,1912 ebd.), \emph{Schriftsteller, Journalist, Theaterleiter}|pw}}{ }Feuilleton\pwindex{Berger, Alfred von 30.\,4.\,1853 Wien – 24.\,8.\,1912 ebd.@\textsc{Berger, Alfred von} (30.\,4.\,1853 Wien – 24.\,8.\,1912 ebd.), \emph{Schriftsteller, Journalist, Theaterleiter}!Burgtheater [Rechte der Seele, Liebelei]@\strich\emph{Burgtheater [Rechte der Seele, Liebelei]}|pwv}}{\lemma{\textnormal{\emph{Bergers Feuilleton}}}\Cendnote{\textnormal{Alfred Freiherr von Berger\pwindex{Berger, Alfred von 30.\,4.\,1853 Wien – 24.\,8.\,1912 ebd.@\textsc{Berger, Alfred von} (30.\,4.\,1853 Wien – 24.\,8.\,1912 ebd.), \emph{Schriftsteller, Journalist, Theaterleiter}|pwk}: \emph{Burgtheater}\pwindex{Berger, Alfred von 30.\,4.\,1853 Wien – 24.\,8.\,1912 ebd.@\textsc{Berger, Alfred von} (30.\,4.\,1853 Wien – 24.\,8.\,1912 ebd.), \emph{Schriftsteller, Journalist, Theaterleiter}!Burgtheater [Rechte der Seele, Liebelei]@\strich\emph{Burgtheater [Rechte der Seele, Liebelei]}|pwk}. In: \emph{Montags-Revue}\pwindex{Montags-Revue. Wochenschrift für Politik, Finanzen, Kunst und Literatur@\emph{Montags-Revue. Wochenschrift für Politik, Finanzen, Kunst und Literatur}|pwk}, Jg. 26, Nr. 41, 14. 10. 1895, S. 1–4.}}}\label{K_L02758-9} haſt Du mir leider nicht
               geſchickt.\pend
           
\pstart
           Daran, daß die Leute Deinen Erfolg Deinen Freunden und Beziehungen zuſchreiben, wirſt
               Du Dich gewöhnen müſſen. Das Geſindel \strikeout{d} kann doch
               nicht rückhaltslos loben; irgend etwas Geringſchätzendes müſſen{ }ſie einfließen
               laſſen. So haben{ }ſie das gefunden. Beim nächſten Erfolg werden{ }ſie{ }ſchon auf etwas
               Neues kommen. Das Alles hat aber nicht die geringſte Bedeutung, {\pb}und mit all’ ihrer Gemeinheit, vorn herum oder
               hinten herum, können{ }ſie Dir nichts Weſentliches \strikeout{rauben.} rauben.\pend
           
\pstart
           \textsc{Herzl\pwindex{Herzl, Theodor 2.\,5.\,1860 Budapest – 3.\,7.\,1904 Edlach@\textsc{Herzl, Theodor} (2.\,5.\,1860 Budapest – 3.\,7.\,1904 Edlach), \emph{Schriftsteller, Journalist}|pw}} war bei mir und{ }ſagte über Dich \strikeout{wohl\textcolor{gray}{×}} wohlwollend: »Der iſt jetzt der größte Dichter von Wien\oindex{Wien@\textbf{Wien}, \emph{Verwaltungsgebiet}|pw}«. Auch dieſen wirſt Du bald auf der Gegenſeite finden. Oh
               was für ein widerliches Subject\pwindex{Herzl, Theodor 2.\,5.\,1860 Budapest – 3.\,7.\,1904 Edlach@\textsc{Herzl, Theodor} (2.\,5.\,1860 Budapest – 3.\,7.\,1904 Edlach), \emph{Schriftsteller, Journalist}|pwv}! Ich habe nicht die Kraft \strikeout{verhehlt,
                  ihn} gehabt, ihm diesmal den abſtoßenden Eindruck zu verbergen, den er mir
               machte.\pend
           
\pstart
           {\pb}Auch \textsc{Sudermann\pwindex{Sudermann, Hermann 30.\,9.\,1857 Macikai – 21.\,11.\,1928 Berlin@\textsc{Sudermann, Hermann} (30.\,9.\,1857 Macikai – 21.\,11.\,1928 Berlin), \emph{Schriftsteller}|pw}} iſt mir nicht{ }ſympathiſch. Freilich iſt er zu Dir anders\textcolor{gray}{,}
               wie zu mir. Aber dieſe{ }ſeine Einfachheit \strikeout{iſt eine} iſt
               eine gemachte; und er iſt{ }ſogar eitel darauf, der{ }ſchöne Mann zu{ }ſein. Auch bin ich
               überzeugt, bei \strikeout{Fra} Frauen{ }ſpielt er den Räthſelhaften
               und Dämoniſchen.\pend
           
\pstart
           Haſt Du nun wirklich die »Liebelei\pwindex{Schnitzler, Arthur 15.\,5.\,1862 Wien – 21.\,10.\,1931 ebd.@\textsc{Schnitzler, Arthur} (15.\,5.\,1862 Wien – 21.\,10.\,1931 ebd.), \emph{Schriftsteller, Mediziner}!Liebelei. Schauspiel in drei Akten@\strich\emph{Liebelei. Schauspiel in drei Akten}|pw}« für Dich
               umgearbeitet? Und was macht das neue Stück\pwindex{Schnitzler, Arthur 15.\,5.\,1862 Wien – 21.\,10.\,1931 ebd.@\textsc{Schnitzler, Arthur} (15.\,5.\,1862 Wien – 21.\,10.\,1931 ebd.), \emph{Schriftsteller, Mediziner}!Freiwild. Schauspiel in 3 Akten@\strich\emph{Freiwild. Schauspiel in 3 Akten}|pwv}? Werde ich es im Manuſkript zu{ }ſehen
               bekommen, auf {\pb}einen Tag, wie immer? Und was \label{K_L02758-10v}\edtext{ſchreibſt Du{ }ſonſt}{\lemma{\textnormal{\emph{schreibst Du sonst}}}\Cendnote{\textnormal{Schnitzler arbeitete an \emph{Freiwild}\pwindex{Schnitzler, Arthur 15.\,5.\,1862 Wien – 21.\,10.\,1931 ebd.@\textsc{Schnitzler, Arthur} (15.\,5.\,1862 Wien – 21.\,10.\,1931 ebd.), \emph{Schriftsteller, Mediziner}!Freiwild. Schauspiel in 3 Akten@\strich\emph{Freiwild. Schauspiel in 3 Akten}|pwk}, einem Schauspiel, mit dem er zu diesem Zeitpunkt
                  sehr unzufrieden war (vgl. A. S.: \emph{Tagebuch}, 2. 12. 1895). Am 5. 12. 1895 begann er zudem die Erzählung \emph{Die Frau des Weisen}\pwindex{Schnitzler, Arthur 15.\,5.\,1862 Wien – 21.\,10.\,1931 ebd.@\textsc{Schnitzler, Arthur} (15.\,5.\,1862 Wien – 21.\,10.\,1931 ebd.), \emph{Schriftsteller, Mediziner}!Frau des Weisen. Erzählung@\strich\emph{Die Frau des Weisen. Erzählung}|pwk} neu.}}}\label{K_L02758-10}? Und wie und mit wem lebſt
               Du? Was macht die große Tragödin\pwindex{Sandrock, Adele 19.\,8.\,1863 Rotterdam – 30.\,8.\,1937 Berlin@\textsc{Sandrock, Adele} (19.\,8.\,1863 Rotterdam – 30.\,8.\,1937 Berlin), \emph{Schauspielerin}|pwv}? Wie lange wird die »Liebelei\pwindex{Schnitzler, Arthur 15.\,5.\,1862 Wien – 21.\,10.\,1931 ebd.@\textsc{Schnitzler, Arthur} (15.\,5.\,1862 Wien – 21.\,10.\,1931 ebd.), \emph{Schriftsteller, Mediziner}!Liebelei. Schauspiel in drei Akten@\strich\emph{Liebelei. Schauspiel in drei Akten}|pw}«
               noch geſpielt werden? Der Erfolg iſt phänomenal. Haſt Du viel Geld verdient? Und das{ }ſparſt Du doch hoffentlich? Haſt Du die{ }ſechs \strikeout{E}{ }\label{K_L02758-11v}\edtext{Ausſchnitte}{\lemma{\textnormal{\emph{Ausschnitte}}}\Cendnote{\textnormal{Beilage nicht erhalten. Eventuell Teile der bis
                     28. 11. 1895 in acht Folgen abgedruckten Übersetzung\pwindex{Schnitzler, Arthur 15.\,5.\,1862 Wien – 21.\,10.\,1931 ebd.@\textsc{Schnitzler, Arthur} (15.\,5.\,1862 Wien – 21.\,10.\,1931 ebd.), \emph{Schriftsteller, Mediziner}!petite comédie. Mœurs viennois@\strich\emph{La petite comédie. Mœurs viennois}|pwkv} von \emph{Die kleine Komödie}\pwindex{Schnitzler, Arthur 15.\,5.\,1862 Wien – 21.\,10.\,1931 ebd.@\textsc{Schnitzler, Arthur} (15.\,5.\,1862 Wien – 21.\,10.\,1931 ebd.), \emph{Schriftsteller, Mediziner}!kleine Komödie@\strich\emph{Die kleine Komödie}|pwk}, \emph{La
                     petite comédie}\pwindex{Schnitzler, Arthur 15.\,5.\,1862 Wien – 21.\,10.\,1931 ebd.@\textsc{Schnitzler, Arthur} (15.\,5.\,1862 Wien – 21.\,10.\,1931 ebd.), \emph{Schriftsteller, Mediziner}!petite comédie. Mœurs viennois@\strich\emph{La petite comédie. Mœurs viennois}|pwk}. }}}\label{K_L02758-11} aus der »\textsc{Liberté\pwindex{Liberté@\emph{La Liberté}|pw}}« erhalten, die ich Dir{ }ſenden ließ? Was macht die Frau \textsc{Lou Andreas\pwindex{Andreas-Salomé, Lou 12.\,2.\,1861 Sankt Petersburg – 5.\,2.\,1937 Göttingen@\textsc{Andreas-Salomé, Lou} (12.\,2.\,1861 Sankt Petersburg – 5.\,2.\,1937 Göttingen), \emph{Schriftstellerin}|pw}}? Was {\pb}macht \textsc{Richard\pwindex{Beer-Hofmann, Richard 11.\,7.\,1866 Wien – 26.\,9.\,1945 New York City@\textsc{Beer-Hofmann, Richard} (11.\,7.\,1866 Wien – 26.\,9.\,1945 New York City), \emph{Schriftsteller}|pw}}? Arbeitet er? Wird was von ihm erſcheinen? {\dotsfive}\pend
           
\pstart
           Wir Zwei! In einem Deiner Briefe befindet{ }ſich eine lange und rührende Stelle
               darüber, die mich jetzt beim Wiederleſen nicht weniger bewegt, als beim \strikeout{A\textcolor{gray}{×}f}
               erſten Mal. Es iſt lieb, daß Du Dir{ }ſolche Mühe gibſt, mir die{ }ſchlimmen Dinge
               auszureden. Sprechen muß ich Dir davon, denn ich bin Dir Ehrlichkeit{ }ſchuldig. Von
               Dir aus iſt gewiß nichts zu befürchten. Du wirſt {\pb}Dich nicht ändern, was auch kommen mag, und wirſt einfach und treu bleiben. Aber in
               mir{ }ſitzt das Übel. Ich habe die Empfindung – und{ }ſie kehrt immer wieder, trotz allen
               Ankämpfens dagegen – daß Du mir auf einmal ferner gerückt biſt, als je, daß Du und
               ich jetzt auf zwei ganz verſchiedenen Lebensgefilden{ }ſtehen, die weiter auseinander
               liegen, als \strikeout{\textcolor{gray}{fe}}{ }Wien\oindex{Wien@\textbf{Wien}, \emph{Verwaltungsgebiet}|pw} und \textsc{Paris\oindex{Paris@\textbf{Paris}, \emph{Hauptstadt}|pw}}, und \strikeout{w} durch etwas Weiteres getrennt{ }ſind, als
               durch einen Raum von fünf Jahren. Du und ich, \strikeout{w} wir
               werden jetzt zwei {\pb}verſchiedene Leben führen. Das
                  \strikeout{\textcolor{gray}{×}} kommt nicht plötzlich, aber ganz \strikeout{all} allmälig,
               ganz unmerklich. Du wirſt oben leben, und ich unten. Derjenige aber, der unten
               bleibt, bemerkt die Veränderung immer zuerſt. Ich \strikeout{\textcolor{gray}{b}} habe die Empfindung, daß Du \strikeout{mir} mir langſam
               entrückt wirſt, und daß ich Dir nicht nach kann. Ich denke \strikeout{\textcolor{gray}{no}ch} mir, daß ich ein Stadium in Deinem Daſein war, daß{ }ſich Dein Leben von mir weg weiter entwickelt: denn mein Leben \strikeout{ent} entwickelt{ }ſich {\pb}nicht, und ich bleibe{ }ſtehen. Ich meine, daß Du mich nicht mehr brauchſt, und daß
               meine Rolle \label{K_L02758-12v}\edtext{\textsc{\begin{otherlanguage}{french}auprès de ta personne\end{otherlanguage}}}{\lemma{\textnormal{\emph{auprès de ta personne}}}\Cendnote{\textnormal{französisch: im Bezug auf Deine
                  Person}}}\label{K_L02758-12} ausgeſpielt iſt. Ich{ }ſehe Dich weit, weit weg von mir. Schreib’ mir,
               was Du willſt, ich kann mir nicht helfen: ich \uline{ſehe}
               Dich eben{ }ſo. Ich weiß, daß Du die größten Kraftanſtrengungen machen wirſt, um mich
               mit Dir zu nehmen; aber ich weiß, daß {\pb}keine Kraft
               da nützen kann, weil es ein \uline{Geſetz} iſt, daß ich
               zurückbleiben muß.\pend
           
\pstart
           Ich drücke das Alles{ }ſchlecht aus. Es iſt heut wieder
               ein{ }ſchlimmer Tag. Ich{ }ſitze mit{ }ſchwerem Kopfe da, und habe mich eine Nacht{ }ſchlaflos herumgewälzt, in Seelenqualen. Die Arbeit habe ich{ }ſatt. Habs wieder einmal
               mit dem Leben verſuchen wollen. Oh, was für eine Sehnſucht ich danach habe, nach dem
               heißen, lebendigen {\pb}Leben! Nicht vorwärtskommen,
               gut! Der Ehrgeiz und das Alles iſt doch nur künſtlich! Aber leben! Und da iſt ein{ }ſüßes \label{K_L02758-13v}\edtext{Kind\pwindex{?? [Junge Frau, in die Goldmann Dezember 1895 verliebt ist] @\textsc{?? [Junge Frau, in die Goldmann Dezember 1895 verliebt ist]}|pwv}}{\lemma{\textnormal{\emph{Kind}}}\Cendnote{\textnormal{nicht identifiziert}}}\label{K_L02758-13}, die der
               liebe Herrgott für mich geſchaffen hat\substVorne{}\textsuperscript{.}\substDazwischen{},\substHinten{}{ }\textsc{Grisette\pwindex{?? [Junge Frau, in die Goldmann Dezember 1895 verliebt ist] @\textsc{?? [Junge Frau, in die Goldmann Dezember 1895 verliebt ist]}|pwv}} oder{ }ſo etwas. Aber{ }ſie kann mich nicht lieben, weil ich nicht jung bin und
               kein feuriger Liebhaber. Und da es nun nichts wird und da alle Sehnſucht wieder
               einmal vergeblich war, entdecke ich, daß ich im Innern{ }ſtets eine Angſt davor {\pb}gehabt habe, es könne doch wahr werden und mir doch
                  gelingen! {\dotsfour}\pend
           
\pstart
           Grüß’ Dich Gott, mein lieber Freund!\pend
           
\pstart
           Dein {\\[\baselineskip]}treuer {\\[\baselineskip]}\spacefill\mbox{Paul Goldmnn}\pend
           \leftskip=0em{}
\pstart
           \noindent{}Schreib’ bald!\pend
           \selectlanguage{ngerman}\endnumbering\briefempfaengerindex{Schnitzler, Arthur@\textsc{Schnitzler, Arthur}!zzzGoldmann, Paul@\emph{von Paul Goldmann}!1895-12-051@{5. 12. [1895]}|)be}\mylabel{L02758h}  \newcommand{\dateiname}{L02758}\newcommand{\titel}{Paul Goldmann an Arthur Schnitzler, 5. 12. [1895]}\newcommand{\editorInnen}{Martin Anton Müller und Laura Untner}%% latex-leseansicht-abspann.tex
%% Abspann für die Leseansicht.
%% Der Schalter \ifkorrekturansicht ist bereits durch den Vorspann gesetzt.

%% latex-abspann.tex
%% Gemeinsamer Abspann für Korrekturansicht und Leseansicht.
%% Setzt den Schalter \ifkorrekturansicht voraus (gesetzt in den
%% einbindenden Dateien latex-korrekturansicht-abspann.tex bzw.
%% latex-leseansicht-abspann.tex).
%% ---------------------------------------------------------------

\normalsize

% Das esempio-Environment wird nur in der Leseansicht benötigt
\ifkorrekturansicht\else
\newenvironment{esempio}[3]%
{
    \vspace{1.5ex}
    \rlap{\underline{#1}}
    \par
    \setlength{\parindent}{0cm}
    \nopagebreak
    \leftskip=#2cm
    \rightskip=#3cm
}
{
    \par
}
\fi

\doendnotes{C}
\bigskip
\vfill

\clearpage

\footnotesize

\ifkorrekturansicht
  \lohead{\textsc{register}}
\fi

% theindex-Environment neu definieren ohne reledmac
\makeatletter
\renewenvironment{theindex}{%
  \ifkorrekturansicht
    \section*{\indexname}%
  \else
    \subsubsection*{Index der erwähnten Entitäten}%
  \fi
  \setlength{\parindent}{0pt}%
  \setlength{\parskip}{0pt plus 0.3pt}%
  \let\item\@idxitem
}{%
  \ifkorrekturansicht\clearpage\fi
}
\makeatother

\IfFileExists{\jobname-pw.ind}{\input{\jobname-pw.ind}}{}

% Quellenangabe nur in der Leseansicht
\ifkorrekturansicht\else
% Fallback-Definitionen, falls die .tex-Datei \titel etc. nicht gesetzt hat
\providecommand{\titel}{}
\providecommand{\editorInnen}{}
\providecommand{\dateiname}{\jobname}

\vspace{3cm}

\vfill

\footnotesize
\textsc{Quelle}: \titel. Herausgegeben von {\editorInnen}. In: \emph{Arthur Schnitzler: Briefwechsel mit Autorinnen und Autoren}.
 Digitale Edition, https://schnitzler-briefe.acdh.oeaw.ac.at/{\dateiname}.html (Stand \today)
\fi

\end{document}


