%% latex-leseansicht-vorspann.tex
%% Vorspann für die Leseansicht.
%% Lädt die gemeinsame Datei latex-vorspann.tex mit nicht gesetztem Schalter.

\newif\ifkorrekturansicht
\korrekturansichtfalse

\input{../tex-inputs/latex-vorspann}


         
         \renewcommand{\erwaehntePersonen}{Personen: Richard Beer-Hofmann, Friedrich Michael Fels, Hugo von Hofmannsthal, Hugo August von Hofmannsthal, Anna von Hofmannsthal, Felix Salten}
         \renewcommand{\erwaehnteOrte}{Orte: Böhmen, München, Nürnberg, Strobl, Strohgasse, Wien, Znaim}
         \renewcommand{\erwaehnteWerke}{}
               \section[Hugo von Hofmannsthal an Arthur Schnitzler, {[}9. 9. 1893{]}]{ Hugo von Hofmannsthal an Arthur Schnitzler, {[}9. 9. 1893{]}}\nopagebreak\mylabel{v}\rehead{ }\begin{ledgroupsized}[t]{13cm}\normalsize\beginnumbering\briefempfaengerindex{Schnitzler, Arthur@\textsc{Schnitzler, Arthur}!zzzHofmannsthal, Hugo von@\emph{von Hugo von Hofmannsthal}!1893-09-091@{{[}9. 9. 1893{]}}|(be} \toendnotes[C]{\smallbreak\pagebreak[2]} \Standort{CUL, Schnitzler, B 43.}
\physDesc{Brief, 1 Blatt, 4 Seiten, 1347 Zeichen
\newline{}Handschrift: schwarze Tinte, deutsche Kurrent
\newline{}Schnitzler: mit Bleistift datiert: »9/9 93« und nummeriert: »57« }\buchAbdrucke{\weitereDrucke{1) Hugo von Hofmannsthal: \emph{Briefe. 1890–1901}. Berlin: \emph{S. Fischer} 1935, S. 88–89.} \weitereDrucke{2) Hugo von Hofmannsthal, Arthur Schnitzler: \emph{Briefwechsel}. Hg. Therese Nickl und Heinrich Schnitzler. Frankfurt am Main: \emph{S. Fischer} 1964, S. 45–46.} }\toendnotes[C]{\smallbreak}\pstart
           \raggedleft{}{\pb}\textsc{Strobl}\oindex{Strobl@\textbf{Strobl}|pw}\pend
           \pstart{}mein lieber Arthur!\pend\pstart
           Schönheit und Leben! Iſt Ihnen das nicht aufgefallen, daſs einem das Leben ſo ganz
               beſonders gut gefällt und man ganz genau weiß, wie es ausſchaut und ſchmeckt, wenn
               man eben momentan innerlich müſſig iſt und eigentlich nicht lebt? Wie Euer\pwindex{Salten, Felix 06.09.1869 – 08.10.1945@\textsc{Salten, Felix} (06.09.1869 – 08.10.1945), \emph{Schriftsteller, Journalist}|pwv} Brief gekommen iſt, der
               »launige« Brief mit dieſen 2 großen Worten, iſt es mir ein bischen vorgekommen, wie
               wenn ich an einem Tiſch ſäße und wirklich gegeſſen hätte und vor mir lägen in
               unappetitlicher Realität {\pb}Krebsſchalen, Hühnerknochen und Pfirſichkerne{\dots} Ihr aber
               ſitzt vor einem wunderſchönen Stilleben mit roten Languſten, goldrothen Weintrauben
               und bunten Truthühnern. Um es zu eſſen, muſs man es rupfen und ſieden und ſchälen und
               ſchneiden und kauen und dann iſt es gar nicht mehr ſchön!\pend
           \pstart
           Und doch gehört’s zum Eſſen und nicht zum Anſchauen. Es – ich meine das Leben.\pend
           \pstart
           Ich bleibe alſo hier bis zum 11\textsuperscript{ten}; dann mit den Eltern\pwindex{Hofmannsthal, Hugo August von 21.12.1841 – 08.12.1915@\textsc{Hofmannsthal, Hugo August von} (21.12.1841 – 08.12.1915), \emph{Bankdirektor}|pw}\pwindex{Hofmannsthal, Anna von 27.01.1849 – 22.03.1904@\textsc{Hofmannsthal, Anna von} (27.01.1849 – 22.03.1904)|pw} nach
                  {\pb}München\oindex{Muenchen@\textbf{München}|pw} u. Nürnberg\oindex{Nuernberg@\textbf{Nürnberg}|pw}; dann vielleicht zur Jagd nach Böhmen\oindex{Boehmen@\textbf{Böhmen}|pw}.\pend
           \pstart
           Jedenfalls bin ich Ende September bei Euch.\pend
           \pstart
           Dieſer Tage iſt die 8\textsuperscript{te}, letzte Rate von 12 fl. an Fels\pwindex{Fels, Friedrich Michael *~1864@\textsc{Fels, Friedrich Michael} (*~1864), \emph{Journalist}|pw} (III \textsc{Strohgasse 3}\oindex{Strohgasse@\textbf{Strohgasse}|pw}) fällig; ich weiß nicht, ob Sie dazu nur 5 fl oder mehr ſchulden; da ich aber
               momentan kein Geld habe und Richard\pwindex{Beer-Hofmann, Richard 1866-07-11 – 1945-09-26@\textsc{Beer-Hofmann, Richard} (1866-07-11 – 1945-09-26), \emph{Schriftsteller}|pw} nicht da
               iſt, ſo bitte ſchicken Sie ihm \uuline{12} fl. mit dem Vermerk
               »letzte Rate.«\pend
           \pstart
           {\pb}Wiſſen Sie die Nummer von Richard\pwindex{Beer-Hofmann, Richard 1866-07-11 – 1945-09-26@\textsc{Beer-Hofmann, Richard} (1866-07-11 – 1945-09-26), \emph{Schriftsteller}|pw}’s Regiment (Znaim\oindex{Znaim@\textbf{Znaim}|pw})?\pend
           \pstart
           Servus{\\[\baselineskip]}\spacefill\mbox{Loris.}\pend
           \leftskip=0em{}\pstart
           \noindent{}Bitte bald ſchreiben! Wo iſt \textsc{Salten}\pwindex{Salten, Felix 06.09.1869 – 08.10.1945@\textsc{Salten, Felix} (06.09.1869 – 08.10.1945), \emph{Schriftsteller, Journalist}|pw}?\pend
           
         
         \endnumbering\mylabel{h}\end{ledgroupsized}  \newcommand{\dateiname}{L00261}\newcommand{\titel}{Hugo von Hofmannsthal an Arthur Schnitzler, [9. 9. 1893]}\newcommand{\editorInnen}{Martin Anton Müller und Gerd-Hermann Susen}%% latex-leseansicht-abspann.tex
%% Abspann für die Leseansicht.
%% Der Schalter \ifkorrekturansicht ist bereits durch den Vorspann gesetzt.

%% latex-abspann.tex
%% Gemeinsamer Abspann für Korrekturansicht und Leseansicht.
%% Setzt den Schalter \ifkorrekturansicht voraus (gesetzt in den
%% einbindenden Dateien latex-korrekturansicht-abspann.tex bzw.
%% latex-leseansicht-abspann.tex).
%% ---------------------------------------------------------------

\normalsize

% Das esempio-Environment wird nur in der Leseansicht benötigt
\ifkorrekturansicht\else
\newenvironment{esempio}[3]%
{
    \vspace{1.5ex}
    \rlap{\underline{#1}}
    \par
    \setlength{\parindent}{0cm}
    \nopagebreak
    \leftskip=#2cm
    \rightskip=#3cm
}
{
    \par
}
\fi

\doendnotes{C}
\bigskip
\vfill

\clearpage

\footnotesize

\ifkorrekturansicht
  \lohead{\textsc{register}}
\fi

% theindex-Environment neu definieren ohne reledmac
\makeatletter
\renewenvironment{theindex}{%
  \ifkorrekturansicht
    \section*{\indexname}%
  \else
    \subsubsection*{Index der erwähnten Entitäten}%
  \fi
  \setlength{\parindent}{0pt}%
  \setlength{\parskip}{0pt plus 0.3pt}%
  \let\item\@idxitem
}{%
  \ifkorrekturansicht\clearpage\fi
}
\makeatother

\IfFileExists{\jobname-pw.ind}{\input{\jobname-pw.ind}}{}

% Quellenangabe nur in der Leseansicht
\ifkorrekturansicht\else
% Fallback-Definitionen, falls die .tex-Datei \titel etc. nicht gesetzt hat
\providecommand{\titel}{}
\providecommand{\editorInnen}{}
\providecommand{\dateiname}{\jobname}

\vspace{3cm}

\vfill

\footnotesize
\textsc{Quelle}: \titel. Herausgegeben von {\editorInnen}. In: \emph{Arthur Schnitzler: Briefwechsel mit Autorinnen und Autoren}.
 Digitale Edition, https://schnitzler-briefe.acdh.oeaw.ac.at/{\dateiname}.html (Stand \today)
\fi

\end{document}


      