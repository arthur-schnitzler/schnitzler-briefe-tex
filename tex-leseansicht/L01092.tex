%% latex-korrekturansicht-vorspann.tex
%% Vorspann für die Korrekturansicht.
%% Lädt die gemeinsame Datei latex-vorspann.tex mit gesetztem Schalter.

\newif\ifkorrekturansicht
\korrekturansichttrue

\input{../tex-inputs/latex-vorspann}


\section[Hugo von Hofmannsthal an Arthur Schnitzler, {[}17. 1. 1901{]}]{L01092 Hugo von Hofmannsthal an Arthur Schnitzler, {[}17. 1. 1901{]}}
\nopagebreak\mylabel{L01092v}
\rehead{ }\normalsize\beginnumbering\briefempfaengerindex{Schnitzler, Arthur@\textsc{Schnitzler, Arthur}!zzzHofmannsthal, Hugo von@\emph{von Hugo von Hofmannsthal}!1901-01-171@{17. 1. 1901}|(be}
\toendnotes[C]{\smallbreak\pagebreak[2]}\Standort{CUL, Schnitzler, B 43.}
\physDesc{Brief, 1 Blatt, 3 Seiten, 674 Zeichen
\newline{}Handschrift: schwarze Tinte, deutsche Kurrent
\newline{}Schnitzler: mit Bleistift datiert: »17/1 901.« 
\newline{}Ordnung: 1) mit Bleistift von unbekannter Hand nummeriert: »\strikeout{190}«  2) mit Bleistift von unbekannter Hand nummeriert:
                                    »183«}
\buchAbdrucke{\weitereDrucke{Hugo von Hofmannsthal, Arthur Schnitzler: \emph{Briefwechsel}. Frankfurt am Main: \emph{S. Fischer} 1964, S. 146.} }\toendnotes[C]{\smallbreak}
\pstart{}{\pb}lieber,\pend\vspace{0.5em}
\pstart
           falls Sie dem kranken Schriftſteller \label{K_L01092-1v}\edtext{Hans
               Wagner\pwindex{Wagner, Hanns @\textsc{Wagner, Hanns}, \emph{Schriftsteller/Schriftstellerin}|pw} keins von Ihren Büchern geſchickt}{\lemma{\textnormal{\emph{Hans … geſchickt}}}\Cendnote{\textnormal{Hanns
                  Wagner\pwindex{Wagner, Hanns @\textsc{Wagner, Hanns}, \emph{Schriftsteller/Schriftstellerin}|pwk} hatte sich zu diesem Zeitpunkt bereits in einem mit 15. 1. 1901 datierten
                  Brief direkt an Schnitzler gewandt. Dieser leistete der Bitte nach
                  Schriften Folge. Am 22. 1. 1901 bekam er von Wagner\pwindex{Wagner, Hanns @\textsc{Wagner, Hanns}, \emph{Schriftsteller/Schriftstellerin}|pwk} ein Dankschreiben für die Zusendung von \emph{Die Frau des Weisen}\pwindex{Frau des Weisen. Novelletten@\emph{Die Frau des Weisen. Novelletten}|pwk} (\emph{CUL}, B 320).}}}\label{K_L01092-1} haben, so thuen Sie es bitte doch
               noch; er hat mir einen ſo merkwürdigen ergreifenden Dankbrief geſchrieben, Geld will
               er abſolut nicht, aber die Freude, die er über Bücher hat, {\pb}iſt ſehr rührend und man kann ſich
               ſeinen Zuſtand ganz gut vorſtellen.\pend
           
\pstart
           Er ist gewiſs ein Dichter, d. h. ein Menſch mit einem Fieber der Phantaſie, ſowie
                  »mein Freund Y.\pwindex{Mein Freund Ypsilon. Aus den Papieren eines Arztes@\emph{Mein Freund Ypsilon. Aus den Papieren eines Arztes}|pw}«\pend
           
\pstart
           Wahrſcheinlich iſt natürlich das was er ſchreibt, gar nichts werth. Auf
               Wiederſehen!\pend
           
\pstart
           {\pb}Von Herzen Ihr{\\[\baselineskip]}\spacefill\mbox{Hugo}\pend
           \leftskip=0em{}
\pstart
           \noindent{}An die Frau Berthe{ }\textsc{Garlan}\pwindex{Frau Bertha Garlan. Roman@\emph{Frau Bertha Garlan. Roman}|pw} hab ich mich gleich beim Aufwachen mit Freude erinnert.\pend
           
\pstart
           Der arme Menſch\pwindex{Wagner, Hanns @\textsc{Wagner, Hanns}, \emph{Schriftsteller/Schriftstellerin}|pwv} iſt im
                     Eliſabethſpital\oindex{Kaiserin-Elisabeth-Spital@\textbf{Kaiserin-Elisabeth-Spital}, \emph{Krankenhaus (K.KKH)}|pw}{\\}Pavillon III{\\}Saal 3{\\}Bett 26.\pend
           \selectlanguage{ngerman}\endnumbering\briefempfaengerindex{Schnitzler, Arthur@\textsc{Schnitzler, Arthur}!zzzHofmannsthal, Hugo von@\emph{von Hugo von Hofmannsthal}!1901-01-171@{17. 1. 1901}|)be}\mylabel{L01092h}  \normalsize

\doendnotes{C}
\bigskip
\vfill

\clearpage

\footnotesize

\lohead{\textsc{register}}

% Definiere theindex-Environment komplett neu ohne reledmac
\makeatletter
\renewenvironment{theindex}{%
  \section*{\indexname}%
  \setlength{\parindent}{0pt}%
  \setlength{\parskip}{0pt plus 0.3pt}%
  \let\item\@idxitem
}{%
  \clearpage
}
\makeatother

\IfFileExists{\jobname-pw.ind}{\input{\jobname-pw.ind}}{}

\end{document}

      