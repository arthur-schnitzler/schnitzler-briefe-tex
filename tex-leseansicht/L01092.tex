%% latex-leseansicht-vorspann.tex
%% Vorspann für die Leseansicht.
%% Lädt die gemeinsame Datei latex-vorspann.tex mit nicht gesetztem Schalter.

\newif\ifkorrekturansicht
\korrekturansichtfalse

\input{../tex-inputs/latex-vorspann}


\section[Hugo von Hofmannsthal an Arthur Schnitzler, {[}17. 1. 1901{]}]{L01092 Hugo von Hofmannsthal an Arthur Schnitzler, {[}17. 1. 1901{]}}
\nopagebreak\mylabel{L01092v}
\rehead{ }\normalsize\beginnumbering\briefempfaengerindex{Schnitzler, Arthur@\textsc{Schnitzler, Arthur}!zzzHofmannsthal, Hugo von@\emph{von Hugo von Hofmannsthal}!1901-01-171@{17. 1. 1901}|(be}
\toendnotes[C]{\smallbreak\pagebreak[2]}
\correspDesc{Versand  durch Hugo von Hofmannsthal am 17. 1. 1901 \textbf{Ort fehlend} 
\newline{}Erhalt  durch Arthur Schnitzler im Zeitraum [17. 1. 1901
                  – 21. 1. 1901?] in Wien}\toendnotes[C]{\smallbreak}
\Standort{CUL, Schnitzler, B 43.}
\physDesc{Brief, 1 Blatt, 3 Seiten, 674 Zeichen
\newline{}Handschrift: schwarze Tinte, deutsche Kurrent
\newline{}Schnitzler: mit Bleistift datiert: »17/1 901.« 
\newline{}Ordnung: 1) mit Bleistift von unbekannter Hand nummeriert: »\strikeout{190}«  2) mit Bleistift von unbekannter Hand nummeriert:
                                    »183«}
\buchAbdrucke{\weitereDrucke{Hugo von Hofmannsthal, Arthur Schnitzler: \emph{Briefwechsel}. Herausgegeben von Therese Nickl und Heinrich Schnitzler. Frankfurt am Main: \emph{S. Fischer} 1964, S. 146.} }\toendnotes[C]{\smallbreak}
\pstart{}{\pb}lieber,\pend\vspace{0.5em}
\pstart
           falls Sie dem kranken Schriftſteller \label{K_L01092-1v}\edtext{Hans
               Wagner\pwindex{Wagner, Hanns @\textsc{Wagner, Hanns}, \emph{Schriftsteller}|pw} keins von Ihren Büchern geſchickt}{\lemma{\textnormal{\emph{Hans … geschickt}}}\Cendnote{\textnormal{Hanns
                  Wagner\pwindex{Wagner, Hanns @\textsc{Wagner, Hanns}, \emph{Schriftsteller}|pwk} hatte sich zu diesem Zeitpunkt bereits in einem mit 15. 1. 1901 datierten
                  Brief direkt an Schnitzler gewandt. Dieser leistete der Bitte nach
                  Schriften Folge. Am 22. 1. 1901 bekam er von Wagner\pwindex{Wagner, Hanns @\textsc{Wagner, Hanns}, \emph{Schriftsteller}|pwk} ein Dankschreiben für die Zusendung von \emph{Die Frau des Weisen}\pwindex{Schnitzler, Arthur 15.\,5.\,1862 Wien – 21.\,10.\,1931 ebd.@\textsc{Schnitzler, Arthur} (15.\,5.\,1862 Wien – 21.\,10.\,1931 ebd.), \emph{Schriftsteller, Mediziner}!Frau des Weisen. Novelletten@\strich\emph{Die Frau des Weisen. Novelletten}|pwk} (\emph{CUL}, B 320).}}}\label{K_L01092-1} haben, so thuen Sie es bitte doch
               noch; er hat mir einen{ }ſo merkwürdigen ergreifenden Dankbrief geſchrieben, Geld will
               er abſolut nicht, aber die Freude, die er über Bücher hat, {\pb}iſt{ }ſehr rührend und man kann{ }ſich{ }ſeinen Zuſtand ganz gut vorſtellen.\pend
           
\pstart
           Er ist gewiſs ein Dichter, d. h. ein Menſch mit einem Fieber der Phantaſie,{ }ſowie
                  »mein Freund Y.\pwindex{Schnitzler, Arthur 15.\,5.\,1862 Wien – 21.\,10.\,1931 ebd.@\textsc{Schnitzler, Arthur} (15.\,5.\,1862 Wien – 21.\,10.\,1931 ebd.), \emph{Schriftsteller, Mediziner}!Mein Freund Ypsilon. Aus den Papieren eines Arztes@\strich\emph{Mein Freund Ypsilon. Aus den Papieren eines Arztes}|pw}«\pend
           
\pstart
           Wahrſcheinlich iſt natürlich das was er{ }ſchreibt, gar nichts werth. Auf
               Wiederſehen!\pend
           
\pstart
           {\pb}Von Herzen Ihr{\\[\baselineskip]}\spacefill\mbox{Hugo}\pend
           \leftskip=0em{}
\pstart
           \noindent{}An die Frau Berthe{ }\textsc{Garlan}\pwindex{Schnitzler, Arthur 15.\,5.\,1862 Wien – 21.\,10.\,1931 ebd.@\textsc{Schnitzler, Arthur} (15.\,5.\,1862 Wien – 21.\,10.\,1931 ebd.), \emph{Schriftsteller, Mediziner}!Frau Bertha Garlan. Roman@\strich\emph{Frau Bertha Garlan. Roman}|pw} hab ich mich gleich beim Aufwachen mit Freude erinnert.\pend
           
\pstart
           Der arme Menſch\pwindex{Wagner, Hanns @\textsc{Wagner, Hanns}, \emph{Schriftsteller}|pwv} iſt im
                     Eliſabethſpital\oindex{Wien@\textbf{Wien}!XV., Rudolfsheim-Fünfhaus@\textbf{XV., Rudolfsheim-Fünfhaus}!Kaiserin-Elisabeth-Spital@\textbf{Kaiserin-Elisabeth-Spital}, \emph{Krankenhaus}|pw}{\\}Pavillon III{\\}Saal 3{\\}Bett 26.\pend
           \selectlanguage{ngerman}\endnumbering\briefempfaengerindex{Schnitzler, Arthur@\textsc{Schnitzler, Arthur}!zzzHofmannsthal, Hugo von@\emph{von Hugo von Hofmannsthal}!1901-01-171@{17. 1. 1901}|)be}\mylabel{L01092h}  \newcommand{\dateiname}{L01092}\newcommand{\titel}{Hugo von Hofmannsthal an Arthur Schnitzler, [17. 1. 1901]}\newcommand{\editorInnen}{Martin Anton Müller und Gerd-Hermann Susen}%% latex-leseansicht-abspann.tex
%% Abspann für die Leseansicht.
%% Der Schalter \ifkorrekturansicht ist bereits durch den Vorspann gesetzt.

%% latex-abspann.tex
%% Gemeinsamer Abspann für Korrekturansicht und Leseansicht.
%% Setzt den Schalter \ifkorrekturansicht voraus (gesetzt in den
%% einbindenden Dateien latex-korrekturansicht-abspann.tex bzw.
%% latex-leseansicht-abspann.tex).
%% ---------------------------------------------------------------

\normalsize

% Das esempio-Environment wird nur in der Leseansicht benötigt
\ifkorrekturansicht\else
\newenvironment{esempio}[3]%
{
    \vspace{1.5ex}
    \rlap{\underline{#1}}
    \par
    \setlength{\parindent}{0cm}
    \nopagebreak
    \leftskip=#2cm
    \rightskip=#3cm
}
{
    \par
}
\fi

\doendnotes{C}
\bigskip
\vfill

\clearpage

\footnotesize

\ifkorrekturansicht
  \lohead{\textsc{register}}
\fi

% theindex-Environment neu definieren ohne reledmac
\makeatletter
\renewenvironment{theindex}{%
  \ifkorrekturansicht
    \section*{\indexname}%
  \else
    \subsubsection*{Index der erwähnten Entitäten}%
  \fi
  \setlength{\parindent}{0pt}%
  \setlength{\parskip}{0pt plus 0.3pt}%
  \let\item\@idxitem
}{%
  \ifkorrekturansicht\clearpage\fi
}
\makeatother

\IfFileExists{\jobname-pw.ind}{\input{\jobname-pw.ind}}{}

% Quellenangabe nur in der Leseansicht
\ifkorrekturansicht\else
% Fallback-Definitionen, falls die .tex-Datei \titel etc. nicht gesetzt hat
\providecommand{\titel}{}
\providecommand{\editorInnen}{}
\providecommand{\dateiname}{\jobname}

\vspace{3cm}

\vfill

\footnotesize
\textsc{Quelle}: \titel. Herausgegeben von {\editorInnen}. In: \emph{Arthur Schnitzler: Briefwechsel mit Autorinnen und Autoren}.
 Digitale Edition, https://schnitzler-briefe.acdh.oeaw.ac.at/{\dateiname}.html (Stand \today)
\fi

\end{document}


