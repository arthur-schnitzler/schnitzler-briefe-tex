\input{../tex-inputs/latex-pdf-vorspann}
\begin{center}
            \textcolor{red}{ENTWURF. ENTZIFFERUNG NOCH NICHT KORREKTURGELESEN}
                      \end{center}
            
               \section[Hermann Bahr an Arthur Schnitzler, {[}8.? 5. 1895{]}]{ Hermann Bahr an Arthur Schnitzler, {[}8.? 5. 1895{]}}\nopagebreak\mylabel{v}\rehead{ }\begin{ledgroupsized}[t]{13cm}\normalsize\beginnumbering\briefempfaengerindex{Schnitzler, Arthur@\textsc{Schnitzler, Arthur}!zzzBahr, Hermann@\emph{von Hermann Bahr}!1895-05-081@{{[}8.? 5. 1895{]}}|(be} \toendnotes[C]{\smallbreak\pagebreak[2]} \Standort{CUL, Schnitzler, B 5b.}
\physDesc{Brief, 1 Blatt, 1 Seite
\newline{}Handschrift: schwarze Tinte, deutsche Kurrent
\newline{}Schnitzler: mit Bleistift datiert: »\textcolor{gray}{8}/5 95« \newline{}Ordnung: 1) mit rotem Buntstift von unbekannter Hand nummeriert: »27« 2) mit Bleistift von unbekannter Hand nummeriert: »27«}\buchAbdrucke{\weitereDrucke{Hermann Bahr, Arthur Schnitzler: \emph{Briefwechsel, Aufzeichnungen, Dokumente (1891–1931)}. Hg. Kurt Ifkovits und Martin Anton Müller. Göttingen: \emph{Wallstein} 2018, S. 101.} }\toendnotes[C]{\smallbreak}\pstart
           \noindent{}{\pb}\textcolor{gray}{\textbf{»Die
                        Zeit\orgindex{Zeit. Wiener Wochenschrift@Die Zeit. Wiener Wochenschrift|pw}«}}\hfill \textcolor{gray}{\textbf{\textbf{Wien\oindex{Wien@\textbf{Wien}|pw}}, den }}8/10 \textcolor{gray}{\textbf{189}}\pend
           \pstart
           \textcolor{gray}{\textbf{Wiener Wochenſchrift}}\hfill \textcolor{gray}{\textbf{IX/3, Günthergaſſe 1\oindex{Guenthergasse@\textbf{Günthergasse}|pw}.}}\pend
           \pstart
           \textcolor{gray}{\textbf{\textbf{Herausgeber}:}}{\\}\textcolor{gray}{\textbf{Profeſſor Dr. I. Singer\pwindex{Singer, Isidor 16.01.1857 – 08.12.1927@\textsc{Singer, Isidor} (16.01.1857 – 08.12.1927), \emph{Journalist, Herausgeber, Soziologe}|pw}, Hermann Bahr\pwindex{Bahr, Hermann 19.07.1863 – 15.01.1934@\textsc{Bahr, Hermann} (19.07.1863 – 15.01.1934), \emph{Schriftsteller, Kritiker}|pw},
                        Dr. Heinrich Kanner\pwindex{Kanner, Heinrich 09.11.1864 – 15.02.1930@\textsc{Kanner, Heinrich} (09.11.1864 – 15.02.1930), \emph{Publizist}|pw}.}}\pend
           \pstart
           \textcolor{gray}{\textbf{Telephon Nr. 6415.}}\pend
           \pstart\center{}Lieber Thuri!\pend\pstart
           Herzlichen Dank für Deine lieben \label{K_L00438_1v}\edtext{Wünſche}{\lemma{\textnormal{\emph{Wünſche}}}\Cendnote{\textnormal{nicht überliefert. Schnitzler\pwindex{Schnitzler, Arthur 15.05.1862 – 21.10.1931@\textsc{Schnitzler, Arthur} (15.05.1862 – 21.10.1931), \emph{Schriftsteller, Mediziner}|pwk} dürfte auf die Meldung des
                  Abendblatts der \emph{Neuen Freien Presse}\orgindex{Neue Freie Presse@Neue Freie Presse|pwk} vom
                        6. 5. 1895, S. 1 (oder eine vergleichbare
                  Zeitungsnotiz) reagiert haben: »Gestern hat im Rathhause\oindex{Rathaus@\textbf{Rathaus}|pw} die Civiltrauung des Schriftstellers Hermann \so{Bahr}\pwindex{Bahr, Hermann 19.07.1863 – 15.01.1934@\textsc{Bahr, Hermann} (19.07.1863 – 15.01.1934), \emph{Schriftsteller, Kritiker}|pw} mit Fräulein Rosa \so{Joël}\pwindex{Bahr, Rosa 26.10.1871 – 17.02.1940@\textsc{Bahr, Rosa} (26.10.1871 – 17.02.1940), \emph{Schauspielerin}|pw}{ }stattgefunden. Beistände des Bräutigams waren Herr Adalbert v. \so{Goldschmidt}\pwindex{Goldschmidt, Adalbert von 05.05.1848 – 21.12.1906@\textsc{Goldschmidt, Adalbert von} (05.05.1848 – 21.12.1906), \emph{Schriftsteller, Komponist}|pw} und Herr Dr. Heinrich \so{Müller}\pwindex{Mueller, Heinrich 1859 – 1932-06-06@\textsc{Müller, Heinrich} (1859 – 1932-06-06), \emph{Rechtswissenschaftler}|pw}.« Bahr\pwindex{Bahr, Hermann 19.07.1863 – 15.01.1934@\textsc{Bahr, Hermann} (19.07.1863 – 15.01.1934), \emph{Schriftsteller, Kritiker}|pwk} lebte mit ihr bis
                  zur Jahrhundertwende in gemeinsamem Haushalt. 1909 wurde die
                  Scheidung erwirkt.}}}\label{K_L00438_1h} von\pend
           \pstart
           Deinem alten{\\[\baselineskip]}\spacefill\mbox{Hermann}\pend
           \leftskip=0em{}\endnumbering\briefempfaengerindex{Schnitzler, Arthur@\textsc{Schnitzler, Arthur}!zzzBahr, Hermann@\emph{von Hermann Bahr}!1895-05-081@{{[}8.? 5. 1895{]}}|)be}\mylabel{h}\end{ledgroupsized}  \newcommand{\dateiname}{L00438}\newcommand{\titel}{Hermann Bahr an Arthur Schnitzler, [8.? 5. 1895]}\newcommand{\editorInnen}{ Kurt Ifkovits,  Martin Anton Müller}\input{../tex-inputs/latex-pdf-abspann}
      