%% latex-korrekturansicht-vorspann.tex
%% Vorspann für die Korrekturansicht.
%% Lädt die gemeinsame Datei latex-vorspann.tex mit gesetztem Schalter.

\newif\ifkorrekturansicht
\korrekturansichttrue

\input{../tex-inputs/latex-vorspann}


\section[Hermann Bahr an Arthur Schnitzler, {[}8.? 5. 1895{]}]{L00438 Hermann Bahr an Arthur Schnitzler, {[}8.? 5. 1895{]}}
\nopagebreak\mylabel{L00438v}
\rehead{ }\normalsize\beginnumbering\briefempfaengerindex{Schnitzler, Arthur@\textsc{Schnitzler, Arthur}!zzzBahr, Hermann@\emph{von Hermann Bahr}!1895-05-081@{{[}8.? 5. 1895{]}}|(be}
\toendnotes[C]{\smallbreak\pagebreak[2]}\Standort{CUL, Schnitzler, B 5b.}
\physDesc{Brief, 1 Blatt, 1 Seite, 74 Zeichen
\newline{}Handschrift: schwarze Tinte, deutsche Kurrent
\newline{}Schnitzler: mit Bleistift datiert: »\textcolor{gray}{8}/5 95« 
\newline{}Ordnung: 1) mit rotem Buntstift von unbekannter Hand nummeriert:
                                    »27«  2) mit Bleistift von unbekannter Hand nummeriert:
                                    »27«}
\buchAbdrucke{\weitereDrucke{Hermann Bahr, Arthur Schnitzler: \emph{Briefwechsel, Aufzeichnungen, Dokumente (1891–1931)}. Göttingen: \emph{Wallstein} 2018, S. 101.} }\toendnotes[C]{\smallbreak}
\pstart
           {\pb}\textcolor{gray}{\textbf{»Die Zeit\orgindex{Zeit. Wiener Wochenschrift@Die Zeit. Wiener Wochenschrift|pw}«}}\hfill \textcolor{gray}{\textbf{\textbf{Wien\oindex{Wien@\textbf{Wien}, \emph{A.ADM2}|pw}}, den .......... 189}}\pend
           
\pstart
           \textcolor{gray}{\textbf{Wiener Wochenſchrift}}\hfill \textcolor{gray}{\textbf{IX/3, Günthergaſſe 1\oindex{Guenthergasse@\textbf{Günthergasse}, \emph{Straße (K.STR)}|pw}.}}\pend
           
\pstart
           \textcolor{gray}{\textbf{\textbf{Herausgeber}:}}{\\}\textcolor{gray}{\textbf{Profeſſor Dr. I. Singer\pwindex{Singer, Isidor 16.01.1857 – 08.12.1927@\textsc{Singer, Isidor} (16.01.1857 – 08.12.1927), \emph{Journalist/Journalistin, Herausgeber/Herausgeberin, Soziologe/Soziologin}|pw}, Hermann Bahr\pwindex{Bahr, Hermann 19.07.1863 – 15.01.1934@\textsc{Bahr, Hermann} (19.07.1863 – 15.01.1934), \emph{Schriftsteller/Schriftstellerin, Kritiker/Kritikerin}|pw},
                        Dr. Heinrich Kanner\pwindex{Kanner, Heinrich 09.11.1864 – 15.02.1930@\textsc{Kanner, Heinrich} (09.11.1864 – 15.02.1930), \emph{Herausgeber/Herausgeberin, Publizist/Publizistin}|pw}.}}\pend
           
\pstart
           \textcolor{gray}{\textbf{Telephon Nr. 6415.}}\pend
           
\pstart\center{}Lieber Thuri!\pend\vspace{0.5em}
\pstart
           Herzlichen Dank für Deine lieben \label{K_L00438-1v}\edtext{Wünſche}{\lemma{\textnormal{\emph{Wünſche}}}\Cendnote{\textnormal{Das Gratulationsschreiben ist nicht  überliefert. Schnitzler dürfte auf die Meldung des
                  Abendblatts der \emph{Neuen Freien Presse}\orgindex{Neue Freie Presse@Neue Freie Presse|pwk} vom
                        6. 5. 1895, S. 1 (oder eine vergleichbare Zeitungsnotiz)
                  reagiert haben: »Gestern hat im Rathhause\oindex{Rathaus [Wien]@\textbf{Rathaus [Wien]}, \emph{S.ADMF}|pw} die Civiltrauung des Schriftstellers Hermann \so{Bahr}\pwindex{Bahr, Hermann 19.07.1863 – 15.01.1934@\textsc{Bahr, Hermann} (19.07.1863 – 15.01.1934), \emph{Schriftsteller/Schriftstellerin, Kritiker/Kritikerin}|pw} mit Fräulein Rosa \so{Joël}\pwindex{Bahr, Rosa 26.10.1871 – 17.02.1940@\textsc{Bahr, Rosa} (26.10.1871 – 17.02.1940), \emph{Schauspieler/Schauspielerin}|pw}{ }stattgefunden. Beistände des Bräutigams waren
                     Herr Adalbert v. \so{Goldschmidt}\pwindex{Goldschmidt, Adalbert von 1848-05-05 – 1906-12-21@\textsc{Goldschmidt, Adalbert von} (1848-05-05 – 1906-12-21), \emph{Schriftsteller/Schriftstellerin, Komponist/Komponistin}|pw} und Herr Dr. Heinrich \so{Müller}\pwindex{Mueller, Heinrich 1859 – 1932-06-06@\textsc{Müller, Heinrich} (1859 – 1932-06-06), \emph{Rechtswissenschaftler/Rechtswissenschaftlerin}|pw}.« Bahr\pwindex{Bahr, Hermann 19.07.1863 – 15.01.1934@\textsc{Bahr, Hermann} (19.07.1863 – 15.01.1934), \emph{Schriftsteller/Schriftstellerin, Kritiker/Kritikerin}|pwk} lebte mit seiner ersten Frau bis
                  zur Jahrhundertwende in gemeinsamem Haushalt. 1909 wurde die
                  Scheidung erwirkt.}}}\label{K_L00438-1} von\pend
           
\pstart
           Deinem alten{\\[\baselineskip]}\spacefill\mbox{Hermann}\pend
           \leftskip=0em{}\selectlanguage{ngerman}\endnumbering\briefempfaengerindex{Schnitzler, Arthur@\textsc{Schnitzler, Arthur}!zzzBahr, Hermann@\emph{von Hermann Bahr}!1895-05-081@{{[}8.? 5. 1895{]}}|)be}\mylabel{L00438h}  \normalsize

\doendnotes{C}
\bigskip
\vfill

\clearpage

\footnotesize

\lohead{\textsc{register}}

% Definiere theindex-Environment komplett neu ohne reledmac
\makeatletter
\renewenvironment{theindex}{%
  \section*{\indexname}%
  \setlength{\parindent}{0pt}%
  \setlength{\parskip}{0pt plus 0.3pt}%
  \let\item\@idxitem
}{%
  \clearpage
}
\makeatother

\IfFileExists{\jobname-pw.ind}{\input{\jobname-pw.ind}}{}

\end{document}

      