%% latex-leseansicht-vorspann.tex
%% Vorspann für die Leseansicht.
%% Lädt die gemeinsame Datei latex-vorspann.tex mit nicht gesetztem Schalter.

\newif\ifkorrekturansicht
\korrekturansichtfalse

\input{../tex-inputs/latex-vorspann}


\section[Hermann Bahr an Arthur Schnitzler, {{[}}8.? 5. 1895{{]}}]{L00438 Hermann Bahr an Arthur Schnitzler, {[}8.? 5. 1895{]}}
\nopagebreak\mylabel{L00438v}
\rehead{ }\normalsize\beginnumbering\briefempfaengerindex{Schnitzler, Arthur@\textsc{Schnitzler, Arthur}!zzzBahr, Hermann@\emph{von Hermann Bahr}!1895-05-081@{{[}8.? 5. 1895{]}}|(be}
\toendnotes[C]{\smallbreak\pagebreak[2]}
\correspDesc{Versand  durch Hermann Bahr am [8.? 5. 1895] in Wien
\newline{}Erhalt  durch Arthur Schnitzler im Zeitraum [8. 5. 1895
                  – 12. 5. 1895?] in Wien}\toendnotes[C]{\smallbreak}
\Standort{CUL, Schnitzler, B 5b.}
\physDesc{Brief, 1 Blatt, 1 Seite, 74 Zeichen
\newline{}Handschrift: schwarze Tinte, deutsche Kurrent
\newline{}Schnitzler: mit Bleistift datiert: »\textcolor{gray}{8}/5 95« 
\newline{}Ordnung: 1) mit rotem Buntstift von unbekannter Hand nummeriert:
                                    »27«  2) mit Bleistift von unbekannter Hand nummeriert:
                                    »27«}
\buchAbdrucke{\weitereDrucke{Hermann Bahr, Arthur Schnitzler: \emph{Briefwechsel, Aufzeichnungen, Dokumente (1891–1931)}. Herausgegeben von Kurt Ifkovits und Martin Anton Müller. Göttingen: \emph{Wallstein} 2018, S. 101.} }\toendnotes[C]{\smallbreak}
\pstart
           {\pb}\textcolor{gray}{\textbf{»Die Zeit\orgindex{Zeit. Wiener Wochenschrift@Die Zeit. Wiener Wochenschrift|pw}«}}\hfill \textcolor{gray}{\textbf{\textbf{Wien\oindex{Wien@\textbf{Wien}, \emph{Verwaltungsgebiet}|pw}}, den .......... 189}}\pend
           
\pstart
           \textcolor{gray}{\textbf{Wiener Wochenſchrift}}\hfill \textcolor{gray}{\textbf{IX/3, Günthergaſſe 1\oindex{Wien@\textbf{Wien}!IX., Alsergrund@\textbf{IX., Alsergrund}!Günthergasse@\textbf{Günthergasse}, \emph{Straße}|pw}.}}\pend
           
\pstart
           \textcolor{gray}{\textbf{\textbf{Herausgeber}:}}{\\}\textcolor{gray}{\textbf{Profeſſor Dr. I. Singer\pwindex{Singer, Isidor 16.\,1.\,1857 Budapest – 8.\,12.\,1927 Wien@\textsc{Singer, Isidor} (16.\,1.\,1857 Budapest – 8.\,12.\,1927 Wien), \emph{Journalist, Herausgeber, Soziologe}|pw}, Hermann Bahr\pwindex{Bahr, Hermann 19.\,7.\,1863 Linz – 15.\,1.\,1934 München@\textsc{Bahr, Hermann} (19.\,7.\,1863 Linz – 15.\,1.\,1934 München), \emph{Schriftsteller, Kritiker}|pw},
                        Dr. Heinrich Kanner\pwindex{Kanner, Heinrich 9.\,11.\,1864 Galați – 15.\,2.\,1930 Wien@\textsc{Kanner, Heinrich} (9.\,11.\,1864 Galați – 15.\,2.\,1930 Wien), \emph{Herausgeber, Publizist}|pw}.}}\pend
           
\pstart
           \textcolor{gray}{\textbf{Telephon Nr. 6415.}}\pend
           
\pstart\center{}Lieber Thuri!\pend\vspace{0.5em}
\pstart
           Herzlichen Dank für Deine lieben \label{K_L00438-1v}\edtext{Wünſche}{\lemma{\textnormal{\emph{Wünsche}}}\Cendnote{\textnormal{Das Gratulationsschreiben ist nicht  überliefert. Schnitzler dürfte auf die Meldung des
                  Abendblatts der \emph{Neuen Freien Presse}\orgindex{Neue Freie Presse@Neue Freie Presse|pwk} vom
                        6. 5. 1895, S. 1 (oder eine vergleichbare Zeitungsnotiz)
                  reagiert haben: »Gestern hat im Rathhause\oindex{Wien@\textbf{Wien}!I., Innere Stadt@\textbf{I., Innere Stadt}!Rathaus [Wien]@\textbf{Rathaus [Wien]}, \emph{Verwaltungsgebäude}|pw} die Civiltrauung des Schriftstellers Hermann \so{Bahr}\pwindex{Bahr, Hermann 19.\,7.\,1863 Linz – 15.\,1.\,1934 München@\textsc{Bahr, Hermann} (19.\,7.\,1863 Linz – 15.\,1.\,1934 München), \emph{Schriftsteller, Kritiker}|pw} mit Fräulein Rosa \so{Joël}\pwindex{Bahr, Rosa 26.\,10.\,1871 Prag – 17.\,2.\,1940 Berlin@\textsc{Bahr, Rosa} (26.\,10.\,1871 Prag – 17.\,2.\,1940 Berlin), \emph{Schauspielerin}|pw}{ }stattgefunden. Beistände des Bräutigams waren
                     Herr Adalbert v. \so{Goldschmidt}\pwindex{Goldschmidt, Adalbert von 5.\,5.\,1848 Wien – 21.\,12.\,1906 ebd.@\textsc{Goldschmidt, Adalbert von} (5.\,5.\,1848 Wien – 21.\,12.\,1906 ebd.), \emph{Schriftsteller, Komponist}|pw} und Herr Dr. Heinrich \so{Müller}\pwindex{Müller, Heinrich 1859 – 6.\,6.\,1932 Hadersdorf@\textsc{Müller, Heinrich} (1859 – 6.\,6.\,1932 Hadersdorf), \emph{Rechtswissenschaftler}|pw}.« Bahr\pwindex{Bahr, Hermann 19.\,7.\,1863 Linz – 15.\,1.\,1934 München@\textsc{Bahr, Hermann} (19.\,7.\,1863 Linz – 15.\,1.\,1934 München), \emph{Schriftsteller, Kritiker}|pwk} lebte mit seiner ersten Frau bis
                  zur Jahrhundertwende in gemeinsamem Haushalt. 1909 wurde die
                  Scheidung erwirkt.}}}\label{K_L00438-1} von\pend
           
\pstart
           Deinem alten{\\[\baselineskip]}\spacefill\mbox{Hermann}\pend
           \leftskip=0em{}\selectlanguage{ngerman}\endnumbering\briefempfaengerindex{Schnitzler, Arthur@\textsc{Schnitzler, Arthur}!zzzBahr, Hermann@\emph{von Hermann Bahr}!1895-05-081@{{[}8.? 5. 1895{]}}|)be}\mylabel{L00438h}  \newcommand{\dateiname}{L00438}\newcommand{\titel}{Hermann Bahr an Arthur Schnitzler, [8.? 5. 1895]}\newcommand{\editorInnen}{Herausgegeben von Martin Anton Müller}%% latex-leseansicht-abspann.tex
%% Abspann für die Leseansicht.
%% Der Schalter \ifkorrekturansicht ist bereits durch den Vorspann gesetzt.

%% latex-abspann.tex
%% Gemeinsamer Abspann für Korrekturansicht und Leseansicht.
%% Setzt den Schalter \ifkorrekturansicht voraus (gesetzt in den
%% einbindenden Dateien latex-korrekturansicht-abspann.tex bzw.
%% latex-leseansicht-abspann.tex).
%% ---------------------------------------------------------------

\normalsize

% Das esempio-Environment wird nur in der Leseansicht benötigt
\ifkorrekturansicht\else
\newenvironment{esempio}[3]%
{
    \vspace{1.5ex}
    \rlap{\underline{#1}}
    \par
    \setlength{\parindent}{0cm}
    \nopagebreak
    \leftskip=#2cm
    \rightskip=#3cm
}
{
    \par
}
\fi

\doendnotes{C}
\bigskip
\vfill

\clearpage

\footnotesize

\ifkorrekturansicht
  \lohead{\textsc{register}}
\fi

% theindex-Environment neu definieren ohne reledmac
\makeatletter
\renewenvironment{theindex}{%
  \ifkorrekturansicht
    \section*{\indexname}%
  \else
    \subsubsection*{Index der erwähnten Entitäten}%
  \fi
  \setlength{\parindent}{0pt}%
  \setlength{\parskip}{0pt plus 0.3pt}%
  \let\item\@idxitem
}{%
  \ifkorrekturansicht\clearpage\fi
}
\makeatother

\IfFileExists{\jobname-pw.ind}{\input{\jobname-pw.ind}}{}

% Quellenangabe nur in der Leseansicht
\ifkorrekturansicht\else
% Fallback-Definitionen, falls die .tex-Datei \titel etc. nicht gesetzt hat
\providecommand{\titel}{}
\providecommand{\editorInnen}{}
\providecommand{\dateiname}{\jobname}

\vspace{3cm}

\vfill

\footnotesize
\textsc{Quelle}: \titel. Herausgegeben von {\editorInnen}. In: \emph{Arthur Schnitzler: Briefwechsel mit Autorinnen und Autoren}.
 Digitale Edition, https://schnitzler-briefe.acdh.oeaw.ac.at/{\dateiname}.html (Stand \today)
\fi

\end{document}


