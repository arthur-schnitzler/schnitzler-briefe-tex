%% latex-korrekturansicht-vorspann.tex
%% Vorspann für die Korrekturansicht.
%% Lädt die gemeinsame Datei latex-vorspann.tex mit gesetztem Schalter.

\newif\ifkorrekturansicht
\korrekturansichttrue

\input{../tex-inputs/latex-vorspann}


\section[Hugo von Hofmannsthal an Arthur Schnitzler, 1. 7. 1904]{L01414 Hugo von Hofmannsthal an Arthur Schnitzler, 1. 7. 1904}
\nopagebreak\mylabel{L01414v}
\rehead{ }\normalsize\beginnumbering\briefempfaengerindex{Schnitzler, Arthur@\textsc{Schnitzler, Arthur}!zzzHofmannsthal, Hugo von@\emph{von Hugo von Hofmannsthal}!1904-07-012@{1. 7. 1904}|(be}
\toendnotes[C]{\smallbreak\pagebreak[2]}\Standort{CUL, Schnitzler, B 43.}
\physDesc{Postkarte, 262 Zeichen
\newline{}Handschrift: 1) schwarze Tinte, deutsche Kurrent\hspace{1em}2) schwarze Tinte, lateinische Kurrent (\noindent{}Adresse)\hspace{1em}
\newline{}Versand: 1) Stempel: »\nobreak{}\oindex{Rodaun@\textbf{Rodaun}, \emph{A.ADM4}|pwk}Rodaun, \textcolor{gray}{1. 7. 04}\nobreak{}«.   2) Stempel: »\nobreak{}\oindex{XVIII., Waehring@\textbf{XVIII., Währing}, \emph{A.ADM3}|pwk}18/1 Wien, 2. 7. 04, 8.V, Bestellt\nobreak{}«. 
\newline{}Schnitzler: mit Bleistift datiert: »2. 7 904« 
\newline{}Ordnung: 1) mit Bleistift von unbekannter Hand nummeriert: »\strikeout{236}«  2) mit Bleistift von unbekannter Hand nummeriert:
                                    »227«}
\buchAbdrucke{\weitereDrucke{Hugo von Hofmannsthal, Arthur Schnitzler: \emph{Briefwechsel}. Frankfurt am Main: \emph{S. Fischer} 1964, S. 190.} }\toendnotes[C]{\smallbreak}\pstart{}{\pb}Herrn D\textsuperscript{r} Arthur Schnitzler\pend{}\pstart{}Wien\oindex{Wien@\textbf{Wien}, \emph{A.ADM2}|pw}\pend{}\pstart{}XVIII Spöttelgasse 7\oindex{Edmund-Weiss-Gasse 7@\textbf{Edmund-Weiß-Gasse 7}, \emph{Wohngebäude (K.WHS)}|pw}\pend{}{\bigskip}\vspace{1em}
\pstart
           \raggedleft{}{\pb}\label{K_L01414-1v}\edtext{Samstag}{\lemma{\textnormal{\emph{Samstag}}}\Cendnote{\textnormal{Schreibirrtum, da die Karte an
                        einem Samstag um 8 Uhr früh zugestellt wurde.}}}\label{K_L01414-1}.\pend
           \vspace{0.5em}
\pstart
           Also Mittwoch, \label{K_L01414-2v}\edtext{\textsc{cher jaune}}{\lemma{\textnormal{\emph{cher jaune}}}\Cendnote{\textnormal{Französisch: lieber Gelber; vgl. Hugo von Hofmannsthal an Arthur Schnitzler, 28. 6. 1904.
               }}}\label{K_L01414-2}, wenn
               es nicht abſurdes Wetter macht.\pend
           
\pstart
           O.\pwindex{Schnitzler, Olga 17.01.1882 – 13.01.1970@\textsc{Schnitzler, Olga} (17.01.1882 – 13.01.1970), \emph{Schauspieler/Schauspielerin, Sänger/Sängerin}|pw}{ }ſoll ſchön üben. \textsc{\label{K_L01414-3v}\edtext{Leisenbogh}{\lemma{\textnormal{\emph{Leisenbogh}}}\Cendnote{\textnormal{Er bezieht sich bereits auf den
                        Erstdruck, \emph{Die neue Rundschau}\pwindex{neue Rundschau@\emph{Die neue Rundschau}|pwk}, Jg. 15, H. 7,
                              Juli 1904, S. 829–842. Am 11. 4. 1904 hatte er es bereits
                        mündlich vorgetragen bekommen.}}}\label{K_L01414-3}}\pwindex{Schicksal des Freiherrn von Leisenbohg. Novellette@\emph{Das Schicksal des Freiherrn von Leisenbohg. Novellette}|pw} iſt gut, durchaus angenehm, durchaus fein, ſollte nur um ein Etwas mehr
               Intenſität in der Groteskerie haben.\pend
           \pstart Ihr \spacefill\mbox{Hugo}\pend{}\selectlanguage{ngerman}\endnumbering\briefempfaengerindex{Schnitzler, Arthur@\textsc{Schnitzler, Arthur}!zzzHofmannsthal, Hugo von@\emph{von Hugo von Hofmannsthal}!1904-07-012@{1. 7. 1904}|)be}\mylabel{L01414h}  \normalsize

\doendnotes{C}
\bigskip
\vfill

\clearpage

\footnotesize

\lohead{\textsc{register}}

% Definiere theindex-Environment komplett neu ohne reledmac
\makeatletter
\renewenvironment{theindex}{%
  \section*{\indexname}%
  \setlength{\parindent}{0pt}%
  \setlength{\parskip}{0pt plus 0.3pt}%
  \let\item\@idxitem
}{%
  \clearpage
}
\makeatother

\IfFileExists{\jobname-pw.ind}{\input{\jobname-pw.ind}}{}

\end{document}

      