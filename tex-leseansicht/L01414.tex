%% latex-leseansicht-vorspann.tex
%% Vorspann für die Leseansicht.
%% Lädt die gemeinsame Datei latex-vorspann.tex mit nicht gesetztem Schalter.

\newif\ifkorrekturansicht
\korrekturansichtfalse

\input{../tex-inputs/latex-vorspann}


         
         \renewcommand{\erwaehntePersonen}{Personen: Hugo von Hofmannsthal, Olga Schnitzler}
         \renewcommand{\erwaehnteOrte}{Orte: Edmund-Weiß-Gasse 7, Rodaun, Wien, XVIII., Währing}
         \renewcommand{\erwaehnteWerke}{Werke: Das Schicksal des Freiherrn von Leisenbohg. Novellette, Die neue Rundschau}
               \section[Hugo von Hofmannsthal an Arthur Schnitzler, 1. 7. 1904]{ Hugo von Hofmannsthal an Arthur Schnitzler, 1. 7. 1904}\nopagebreak\mylabel{v}\rehead{ }\begin{ledgroupsized}[t]{13cm}\normalsize\beginnumbering\briefempfaengerindex{Schnitzler, Arthur@\textsc{Schnitzler, Arthur}!zzzHofmannsthal, Hugo von@\emph{von Hugo von Hofmannsthal}!1904-07-012@{1. 7. 1904}|(be} \toendnotes[C]{\smallbreak\pagebreak[2]} \Standort{CUL, Schnitzler, B 43.}
\physDesc{Postkarte, 262 Zeichen
\newline{}Handschrift: 1) schwarze Tinte, deutsche Kurrent\hspace{1em}2) schwarze Tinte, lateinische Kurrent (\noindent{}Adresse)\hspace{1em}
\newline{}Versand: 1) Stempel: »\nobreak{}\oindex{Rodaun@\textbf{Rodaun}|pwk}Rodaun, \textcolor{gray}{1. 7. 04}\nobreak{}«.   2) Stempel: »\nobreak{}\oindex{XVIII., Waehring@\textbf{XVIII., Währing}|pwk}18/1 Wien, 2. 7. 04, 8.V, Bestellt\nobreak{}«. 
\newline{}Schnitzler: mit Bleistift datiert: »2. 7 904« 
\newline{}Ordnung: 1) mit Bleistift von unbekannter Hand nummeriert: »\strikeout{236}«  2) mit Bleistift von unbekannter Hand nummeriert:
                                    »227«}\buchAbdrucke{\weitereDrucke{Hugo von Hofmannsthal, Arthur Schnitzler: \emph{Briefwechsel}. Hg. Therese Nickl und Heinrich Schnitzler. Frankfurt am Main: \emph{S. Fischer} 1964, S. 190.} }\toendnotes[C]{\smallbreak}\pstart{}{\pb}Herrn D\textsuperscript{r} Arthur Schnitzler\pend{}\pstart{}Wien\oindex{Wien@\textbf{Wien}|pw}\pend{}\pstart{}XVIII Spöttelgasse 7\oindex{Edmund-Weiss-Gasse 7@\textbf{Edmund-Weiß-Gasse 7}|pw}\pend{}{\bigskip}\pstart
           \raggedleft{}{\pb}\label{K_L01414-1v}\edtext{Samstag}{\lemma{\textnormal{\emph{Samstag}}}\Cendnote{\textnormal{Schreibirrtum, da die Karte an
                        einem Samstag um 8 Uhr früh zugestellt wurde.}}}\label{K_L01414-1h}.\pend
           \pstart
           Also Mittwoch, \label{K_L01414-2v}\edtext{\textsc{cher jaune}}{\lemma{\textnormal{\emph{cher jaune}}}\Cendnote{\textnormal{Französisch: lieber Gelber; vgl. Hugo von Hofmannsthal an Arthur Schnitzler, 28. 6. 1904.
               }}}\label{K_L01414-2h}, wenn
               es nicht abſurdes Wetter macht.\pend
           \pstart
           O.\pwindex{Schnitzler, Olga 17.01.1882 – 13.01.1970@\textsc{Schnitzler, Olga} (17.01.1882 – 13.01.1970), \emph{Schauspielerin, Sängerin}|pw}{ }ſoll ſchön üben. \textsc{\label{K_L01414-3v}\edtext{Leisenbogh}{\lemma{\textnormal{\emph{Leisenbogh}}}\Cendnote{\textnormal{Er bezieht sich bereits auf den
                        Erstdruck, \emph{Die neue Rundschau}\pwindex{?? Werk@Nicht ermittelte Verfasserinnen und Verfasser!neue Rundschau1904@\emph{Die neue Rundschau} {[}1904{]}|pwk}, Jg. 15, H. 7,
                              Juli 1904, S. 829–842. Am 11. 4. 1904 hatte er es bereits
                        mündlich vorgetragen bekommen.}}}\label{K_L01414-3h}}\pwindex{Schnitzler, Arthur 15.05.1862 – 21.10.1931@\textsc{Schnitzler, Arthur} (15.05.1862 – 21.10.1931), \emph{Schriftsteller, Mediziner}!Schicksal des Freiherrn von Leisenbohg. Novellette01. 07. 1904@\strich\emph{Das Schicksal des Freiherrn von Leisenbohg. Novellette} {[}01. 07. 1904{]}|pw} iſt gut, durchaus angenehm, durchaus fein, ſollte nur um ein Etwas mehr
               Intenſität in der Groteskerie haben.\pend
           \pstart Ihr \spacefill\mbox{Hugo}\pend{}
         
         \endnumbering\mylabel{h}\end{ledgroupsized}  \newcommand{\dateiname}{L01414}\newcommand{\titel}{Hugo von Hofmannsthal an Arthur Schnitzler, 1. 7. 1904}\newcommand{\editorInnen}{Martin Anton Müller und Gerd-Hermann Susen}%% latex-leseansicht-abspann.tex
%% Abspann für die Leseansicht.
%% Der Schalter \ifkorrekturansicht ist bereits durch den Vorspann gesetzt.

%% latex-abspann.tex
%% Gemeinsamer Abspann für Korrekturansicht und Leseansicht.
%% Setzt den Schalter \ifkorrekturansicht voraus (gesetzt in den
%% einbindenden Dateien latex-korrekturansicht-abspann.tex bzw.
%% latex-leseansicht-abspann.tex).
%% ---------------------------------------------------------------

\normalsize

% Das esempio-Environment wird nur in der Leseansicht benötigt
\ifkorrekturansicht\else
\newenvironment{esempio}[3]%
{
    \vspace{1.5ex}
    \rlap{\underline{#1}}
    \par
    \setlength{\parindent}{0cm}
    \nopagebreak
    \leftskip=#2cm
    \rightskip=#3cm
}
{
    \par
}
\fi

\doendnotes{C}
\bigskip
\vfill

\clearpage

\footnotesize

\ifkorrekturansicht
  \lohead{\textsc{register}}
\fi

% theindex-Environment neu definieren ohne reledmac
\makeatletter
\renewenvironment{theindex}{%
  \ifkorrekturansicht
    \section*{\indexname}%
  \else
    \subsubsection*{Index der erwähnten Entitäten}%
  \fi
  \setlength{\parindent}{0pt}%
  \setlength{\parskip}{0pt plus 0.3pt}%
  \let\item\@idxitem
}{%
  \ifkorrekturansicht\clearpage\fi
}
\makeatother

\IfFileExists{\jobname-pw.ind}{\input{\jobname-pw.ind}}{}

% Quellenangabe nur in der Leseansicht
\ifkorrekturansicht\else
% Fallback-Definitionen, falls die .tex-Datei \titel etc. nicht gesetzt hat
\providecommand{\titel}{}
\providecommand{\editorInnen}{}
\providecommand{\dateiname}{\jobname}

\vspace{3cm}

\vfill

\footnotesize
\textsc{Quelle}: \titel. Herausgegeben von {\editorInnen}. In: \emph{Arthur Schnitzler: Briefwechsel mit Autorinnen und Autoren}.
 Digitale Edition, https://schnitzler-briefe.acdh.oeaw.ac.at/{\dateiname}.html (Stand \today)
\fi

\end{document}


      