%% latex-korrekturansicht-vorspann.tex
%% Vorspann für die Korrekturansicht.
%% Lädt die gemeinsame Datei latex-vorspann.tex mit gesetztem Schalter.

\newif\ifkorrekturansicht
\korrekturansichttrue

\input{../tex-inputs/latex-vorspann}


\section[Richard Beer-Hofmann an Arthur Schnitzler, 28. 6. 1901]{L01137 Richard Beer-Hofmann an Arthur Schnitzler, 28. 6. 1901}
\nopagebreak\mylabel{L01137v}
\rehead{ }\normalsize\beginnumbering\briefempfaengerindex{Schnitzler, Arthur@\textsc{Schnitzler, Arthur}!zzzBeer-Hofmann, Richard@\emph{von Richard Beer-Hofmann}!1901-06-281@{28. 6. 1901}|(be}
\toendnotes[C]{\smallbreak\pagebreak[2]}\Standort{CUL, Schnitzler, B 8.}
\physDesc{Brief, 1 Blatt, 2 Seiten, 700 Zeichen
\newline{}Handschrift: blauer Buntstift, lateinische Kurrent
\newline{}Ordnung: mit Bleistift von unbekannter Hand nummeriert:
                                    »163« }
\buchAbdrucke{\weitereDrucke{Arthur Schnitzler, Richard Beer-Hofmann: \emph{Briefwechsel 1891–1931}. Wien, Zürich: \emph{Europaverlag} 1992, S. 152.} }\toendnotes[C]{\smallbreak}
\pstart
           \raggedleft{}{\pb}Pörtschach\oindex{Poertschach am Woerthersee@\textbf{Pörtschach am Wörthersee}, \emph{P.PPL}|pw}{ }28/VI 1901\pend
           \vspace{0.5em}
\pstart
           Lieber Arthur! Es war Zeit daß Sie von Sich hören ließen. Ich wußte
               nur durch die N. Fr Pr\orgindex{Neue Freie Presse@Neue Freie Presse|pw}\pwindex{Kleine Chronik@\emph{Kleine Chronik}|pwv} daß Sie in Tirol\oindex{Tirol@\textbf{Tirol}, \emph{A.ADM1}|pw} sind. Ich war – um mir
               Heiterkeit zu holen – 3 Tage in Venedig\oindex{Venedig@\textbf{Venedig}, \emph{P.PPLA}|pw},
               gleichzeitig mit Hugo\pwindex{Hofmannsthal, Hugo von 1874-02-01 – 1929-07-15@\textsc{Hofmannsthal, Hugo von} (1874-02-01 – 1929-07-15), \emph{Schriftsteller/Schriftstellerin}|pw}, doch wußten wir von
               einander nichts, und erst als ich zurückkam erfuhr ich daß er auch dort war. Ich habe
               mir aber keine Heiterkeit aus Venedig\oindex{Venedig@\textbf{Venedig}, \emph{P.PPLA}|pw} geholt.\pend
           
\pstart
           {\pb}Ich möchte wissen wann Sie
               herkommen, und ob und wann Paul\pwindex{Goldmann, Paul 31.01.1865 – 25.09.1935@\textsc{Goldmann, Paul} (31.01.1865 – 25.09.1935), \emph{Schriftsteller/Schriftstellerin, Journalist/Journalistin}|pw} hieherko{\geminationm}t. Ludassy\pwindex{Gans-Ludassy, Julius von 13.04.1858 – 30.09.1922@\textsc{Gans-Ludassy, Julius von} (13.04.1858 – 30.09.1922), \emph{Schriftsteller/Schriftstellerin, Journalist/Journalistin, Herausgeber/Herausgeberin}|pw} und
                  Alexander Engel\pwindex{Engel, Alexander 10.04.1868 – 17.11.1940@\textsc{Engel, Alexander} (10.04.1868 – 17.11.1940), \emph{Schriftsteller/Schriftstellerin, Journalist/Journalistin}|pw} habe ich hier gesprochen. –
                  L.\pwindex{Gans-Ludassy, Julius von 13.04.1858 – 30.09.1922@\textsc{Gans-Ludassy, Julius von} (13.04.1858 – 30.09.1922), \emph{Schriftsteller/Schriftstellerin, Journalist/Journalistin, Herausgeber/Herausgeberin}|pw} erklärte es unsicher daß Sie kämen. Hirschfeld (Robert)\pwindex{Hirschfeld, Robert 17.09.1857 – 02.04.1914@\textsc{Hirschfeld, Robert} (17.09.1857 – 02.04.1914), \emph{Journalist/Journalistin, Musikkritiker/Musikkritikerin}|pw} hat uns besucht. Was ist
               mit Salten\pwindex{Salten, Felix 06.09.1869 – 08.10.1945@\textsc{Salten, Felix} (06.09.1869 – 08.10.1945), \emph{Schriftsteller/Schriftstellerin, Journalist/Journalistin, Chefredakteur/Chefredakteurin}|pw} und seinem bodenständigen Brettl\orgindex{Jung-Wiener Theater zum Lieben Augustin@Jung-Wiener Theater zum Lieben Augustin|pwv}; aber wichtiger: Was ist
               mit Ihnen? Ist Salzburg\oindex{Salzburg@\textbf{Salzburg}, \emph{A.ADM2}|pw} noch immer gegen Versti{\geminationm}ung gut? Von Herzen\pend
           \pstart Ihr \spacefill\mbox{Richard}\pend{}\selectlanguage{ngerman}\endnumbering\briefempfaengerindex{Schnitzler, Arthur@\textsc{Schnitzler, Arthur}!zzzBeer-Hofmann, Richard@\emph{von Richard Beer-Hofmann}!1901-06-281@{28. 6. 1901}|)be}\mylabel{L01137h}  \normalsize

\doendnotes{C}
\bigskip
\vfill

\clearpage

\footnotesize

\lohead{\textsc{register}}

% Definiere theindex-Environment komplett neu ohne reledmac
\makeatletter
\renewenvironment{theindex}{%
  \section*{\indexname}%
  \setlength{\parindent}{0pt}%
  \setlength{\parskip}{0pt plus 0.3pt}%
  \let\item\@idxitem
}{%
  \clearpage
}
\makeatother

\IfFileExists{\jobname-pw.ind}{\input{\jobname-pw.ind}}{}

\end{document}

      