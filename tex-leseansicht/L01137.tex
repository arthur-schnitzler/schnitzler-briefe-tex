\input{../tex-inputs/latex-pdf-vorspann}
\begin{center}
            \textcolor{red}{ENTWURF. ENTZIFFERUNG NOCH NICHT KORREKTURGELESEN}
                      \end{center}
            
               \section[Richard Beer-Hofmann an Arthur Schnitzler, 28. 6. 1901]{ Richard Beer-Hofmann an Arthur Schnitzler,
               28. 6. 1901}\nopagebreak\mylabel{v}\rehead{ }\begin{ledgroupsized}[t]{13cm}\normalsize\beginnumbering\briefempfaengerindex{Schnitzler, Arthur@\textsc{Schnitzler, Arthur}!zzzBeer-Hofmann, Richard@\emph{von Richard Beer-Hofmann}!1901-06-281@{28. 6. 1901}|(be} \toendnotes[C]{\smallbreak\pagebreak[2]} \Standort{CUL, Schnitzler, B 8.}
\physDesc{Brief, 1 Blatt, 2 Seiten
\newline{}Handschrift: blauer Buntstift, lateinische Kurrent\newline{}Ordnung: mit Bleistift von unbekannter Hand nummeriert: »163« }\buchAbdrucke{\weitereDrucke{Arthur Schnitzler, Richard Beer-Hofmann: \emph{Briefwechsel 1891–1931}. Hg. Konstanze Fliedl. Wien, Zürich: \emph{Europaverlag} 1992, S. 152.} }\toendnotes[C]{\smallbreak}\pstart
           \raggedleft{}{\pb}Pörtschach\oindex{Poertschach@\textbf{Pörtschach}|pw}{ }28/VI 1901\pend
           \pstart
           Lieber Arthur! Es war Zeit daß Sie von Sich hören ließen. Ich wußte
               nur durch die N. Fr Pr\orgindex{Neue Freie Presse@Neue Freie Presse|pw}\pwindex{?? Werk@Nicht ermittelte Verfasserinnen und Verfasser!Kleine Chronik21.6.1901 – 21.6.1901@\emph{Kleine Chronik} {[}21.6.1901 – 21.6.1901{]}|pwv} daß Sie in Tirol\oindex{Tirol@\textbf{Tirol}|pw} sind. Ich war – um mir
               Heiterkeit zu holen – 3 Tage in Venedig\oindex{Venedig@\textbf{Venedig}|pw},
               gleichzeitig mit Hugo\pwindex{Hofmannsthal, Hugo von 01.02.1874 – 15.07.1929@\textsc{Hofmannsthal, Hugo von} (01.02.1874 – 15.07.1929), \emph{Schriftsteller}|pw}, doch wußten wir von
               einander nichts, und erst als ich zurückkam erfuhr ich daß er auch dort war. Ich habe
               mir aber keine Heiterkeit aus Venedig\oindex{Venedig@\textbf{Venedig}|pw} geholt.\pend
           \pstart
           {\pb}Ich möchte wissen wann Sie
               herkommen, und ob und wann Paul\pwindex{Goldmann, Paul 31.01.1865 – 25.09.1935@\textsc{Goldmann, Paul} (31.01.1865 – 25.09.1935), \emph{Schriftsteller, Journalist}|pw} hieherko{\geminationm}t. Ludassy\pwindex{Gans-Ludassy, Julius von 13.04.1858 – 30.09.1922@\textsc{Gans-Ludassy, Julius von} (13.04.1858 – 30.09.1922), \emph{Schriftsteller, Journalist}|pw} und Alexander Engel\pwindex{Engel, Alexander 10.04.1868 – 17.11.1940@\textsc{Engel, Alexander} (10.04.1868 – 17.11.1940), \emph{Schriftsteller, Journalist}|pw} habe ich hier gesprochen. – L.\pwindex{Gans-Ludassy, Julius von 13.04.1858 – 30.09.1922@\textsc{Gans-Ludassy, Julius von} (13.04.1858 – 30.09.1922), \emph{Schriftsteller, Journalist}|pw} erklärte es unsicher daß Sie kämen. Hirschfeld (Robert)\pwindex{Hirschfeld, Robert 17.09.1857 – 02.04.1914@\textsc{Hirschfeld, Robert} (17.09.1857 – 02.04.1914), \emph{Journalist, Musikkritiker}|pw} hat uns besucht. Was ist mit
                  Salten\pwindex{Salten, Felix 06.09.1869 – 08.10.1945@\textsc{Salten, Felix} (06.09.1869 – 08.10.1945), \emph{Schriftsteller, Journalist}|pw} und seinem bodenständigen Brettl\oindex{Jung-Wiener Theater zum Lieben Augustin@\textbf{Jung-Wiener Theater zum Lieben Augustin}|pwv}; aber wichtiger: Was ist mit
               Ihnen? Ist Salzburg\oindex{Salzburg@\textbf{Salzburg}|pw} noch immer gegen Versti{\geminationm}ung gut? Von Herzen\pend
           \pstart Ihr \spacefill\mbox{Richard}\pend{}\endnumbering\briefempfaengerindex{Schnitzler, Arthur@\textsc{Schnitzler, Arthur}!zzzBeer-Hofmann, Richard@\emph{von Richard Beer-Hofmann}!1901-06-281@{28. 6. 1901}|)be}\mylabel{h}\end{ledgroupsized}  \newcommand{\dateiname}{L01137}\newcommand{\titel}{Richard Beer-Hofmann an Arthur Schnitzler, 28. 6. 1901}\newcommand{\editorInnen}{Martin Anton Müller und Gerd-Hermann Susen}\input{../tex-inputs/latex-pdf-abspann}
      