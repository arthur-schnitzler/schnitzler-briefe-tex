%% latex-leseansicht-vorspann.tex
%% Vorspann für die Leseansicht.
%% Lädt die gemeinsame Datei latex-vorspann.tex mit nicht gesetztem Schalter.

\newif\ifkorrekturansicht
\korrekturansichtfalse

\input{../tex-inputs/latex-vorspann}


\section[Albert Ehrenstein an Arthur Schnitzler, 24. 1. 1909]{L01826 Albert Ehrenstein an Arthur Schnitzler, 24. 1. 1909}
\nopagebreak\mylabel{L01826v}
\rehead{ }\normalsize\beginnumbering\briefempfaengerindex{Schnitzler, Arthur@\textsc{Schnitzler, Arthur}!zzzEhrenstein, Albert@\emph{von Albert Ehrenstein}!1909-01-241@{24. 1. 1909}|(be}
\toendnotes[C]{\smallbreak\pagebreak[2]}
\correspDesc{Versand  durch Albert Ehrenstein am 24. 1. 1909 in Wien
\newline{}Erhalt  durch Arthur Schnitzler im Zeitraum [24. 1. 1909
                  – 28. 1. 1909?] in Wien}\toendnotes[C]{\smallbreak}
\Standort{CUL, Schnitzler, B 30.}
\physDesc{Brief, 1 Blatt, 4 Seiten, 2852 Zeichen
\newline{}Handschrift: schwarze Tinte, deutsche Kurrent
\newline{}Schnitzler: mit Bleistift beschriftet: »\textsc{Ehrenst\textcolor{gray}{e}in}« }
\buchAbdrucke{\weitereDrucke{Albert Ehrenstein: \emph{Briefe}. Herausgegeben von Hanni Mittelmann. München: \emph{Boer} 1989, S. 25–26 (Werke, 1).} }\toendnotes[C]{\smallbreak}
\pstart
           
\pstart
           {\pb}XVI. \textsc{Ottakringerstr} 114\oindex{Wien@\textbf{Wien}!XVI., Ottakring@\textbf{XVI., Ottakring}!Ottakringer Straße@\textbf{Ottakringer Straße}, \emph{Straße}|pw}\oindex{Wien@\textbf{Wien}!XVII., Hernals@\textbf{XVII., Hernals}!Ottakringer Straße@\textbf{Ottakringer Straße}, \emph{Straße}|pw}.\pend
           
\pstart
           \raggedleft{}24. I. 09\pend
           \pend
           
\pstart{}\textsc{Sehr geehrter Herr Doktor!}\pend\vspace{0.5em}
\pstart
           Ihr geſchätztes Schreiben habe ich erhalten, und{ }ſo angenehm es mir auch war, daß
               Sie,{ }ſehr geehrter Herr Doktor,{ }ſich{ }ſo{ }ſchnell der Mühe unterzogen, mein armes Märchen\pwindex{Ehrenstein, Albert 23.\,12.\,1886 Wien – 8.\,4.\,1950 New York City@\textsc{Ehrenstein, Albert} (23.\,12.\,1886 Wien – 8.\,4.\,1950 New York City), \emph{Schriftsteller}!Tai-Gin@\strich\emph{Tai-Gin}|pwv} zu leſen, die übrigen
               Empfindungen, die mich nach der Lektüre Ihres werten Briefes beſeelten, waren von
               Freude weit entfernt. Wenig geneigt, mich mit dem »Manche freilich müſſen unten{ }ſterben\pwindex{Hofmannsthal, Hugo von 1.\,2.\,1874 Wien – 15.\,7.\,1929 Rodaun@\textsc{Hofmannsthal, Hugo von} (1.\,2.\,1874 Wien – 15.\,7.\,1929 Rodaun), \emph{Schriftsteller}!Manche freilich@\strich\emph{Manche freilich}|pwv}« zufrieden zu geben,
               wähnte ich naiv, im äußerſten Falle würden Sie,{ }ſehr geehrter Herr Doktor, mich nicht
               direkt empfehlen, sondern durch Herrn v. Hofma{\geminationn}stal\pwindex{Hofmannsthal, Hugo von 1.\,2.\,1874 Wien – 15.\,7.\,1929 Rodaun@\textsc{Hofmannsthal, Hugo von} (1.\,2.\,1874 Wien – 15.\,7.\,1929 Rodaun), \emph{Schriftsteller}|pw}. Wenn dies nicht{ }ſein mag, ich nicht durch
               übermäßige Inanſpruchnahme beläſtige, auch nicht {\pb}ſonſtwie unwillentlich mir Ihre Ungnade
               zugezogen habe, müßte ich, der Not gehorchend, nicht dem eignen Triebe, ein oder zwei
               in ihrer Harmloſigkeit entwaffnende hiſtoriſche Novellen wieder aufnehmen, die
               vielleicht für die Neue Freie Preſſe\orgindex{Neue Freie Presse@Neue Freie Presse|pw} nicht ganz
               ungeeignet{ }ſein dürften. Ich erkühne mich keineswegs, Ihnen,{ }ſehr geehrter Herr
               Doktor, neuerdings die angreifende Lektüre irgend einer meiner Mittelmäßigkeiten
               zumuten zu wollen, von denen ich übrigens letzthin loyalerweiſe die denkbar kleinſte
               Doſis überſandte. Bin ich auch leider lange nicht{ }ſoweit, eine Befürwortung irgend
               einer meiner Arbeiten um ihrer{ }ſelbſt willen erbitten zu können, hoffe ich dennoch
               dereinſt halbwegs Erſprießliches zu verfaſſen. Nicht meine Sachen, {\pb}ſondern mich möchte ich gerne an eine
               reſpektable hieſige Zeitung empfohlen{ }ſehen. Es iſt gewiß bedauerlich, daß die
               Menſchen noch{ }ſo vieler Umſtände bedürfen und nicht bereits dabei angelangt{ }ſind,
               Schriftſtellern die Keime ihrer Werke aus den Gehirnen zu extrahieren und
               Dichtmaſchinen zur Ausbrütung zu übergeben. Bis dahin werden eben meinesgleichen
               immer an den guten Glauben appellieren müſſen und dies tue ich denn auch, nicht ohne
               eine{ }ſanfte Betrübnis über mein{ }ſäumiges Wachstum. – Herr Camill Hofmann\pwindex{Hoffmann, Camill 31.\,10.\,1878 Kolín – 1.\,10.\,1944 Konzentrationslager Auschwitz-Birkenau@\textsc{Hoffmann, Camill} (31.\,10.\,1878 Kolín – 1.\,10.\,1944 Konzentrationslager Auschwitz-Birkenau), \emph{Schriftsteller, Journalist}|pw}, dem mich zu empfehlen Sie,{ }ſehr geehrter Herr
               Doktor, die Güte hatten, äußerte{ }ſich ebenſo liebenswürdig als unverdient anerkennend
               über meine Arbeiten, lehnte{ }ſie gleichwohl ab, in einer {\pb}mir unbegreiflichen Rückſicht auf das
               Publikum der »Zeit\pwindex{Zeit@\emph{Die Zeit}|pw}«, die er eigentümlicherweiſe
               als Familienblatt bezeichnete. Der »Erdgeist\orgindex{Erdgeist@Erdgeist|pw}«, an
               den Herr Hofmann\pwindex{Hoffmann, Camill 31.\,10.\,1878 Kolín – 1.\,10.\,1944 Konzentrationslager Auschwitz-Birkenau@\textsc{Hoffmann, Camill} (31.\,10.\,1878 Kolín – 1.\,10.\,1944 Konzentrationslager Auschwitz-Birkenau), \emph{Schriftsteller, Journalist}|pw} meine Skizzen weiterzugeben
               die Freundlichkeit hatte, ließ es an mich kalt laſſenden Lobeserhebungen nicht
               fehlen,{ }ſcheint aber ähnliche Bedenken zu tragen, Realeres für mich zu tun. – Indem
               ich mir bewußt bin, Ihnen,{ }ſehr geehrter Herr Doktor, niemals für all das, was Sie an
               mir getan, danken zu können, möchte ich erſuchen, es nicht übel nehmen zu wollen, daß
               ich,{ }ſo{ }ſchwer es mir auch fiel, noch einmal u. gewiß nicht ohne zwingende Gründe,
               mit der Bitte um eine Empfehlung an Sie heranzutreten genötigt bin. Hochachtungsvoll
               ergebenſt Ihr Sie,{ }ſehr geehrter Herr Doktor, verehrender\pend
           \pstart \spacefill\mbox{Albert Ehrenstein.}\pend{}\selectlanguage{ngerman}\endnumbering\briefempfaengerindex{Schnitzler, Arthur@\textsc{Schnitzler, Arthur}!zzzEhrenstein, Albert@\emph{von Albert Ehrenstein}!1909-01-241@{24. 1. 1909}|)be}\mylabel{L01826h}  \newcommand{\dateiname}{L01826}\newcommand{\titel}{Albert Ehrenstein an Arthur Schnitzler, 24. 1. 1909}\newcommand{\editorInnen}{Martin Anton Müller und Gerd-Hermann Susen}%% latex-leseansicht-abspann.tex
%% Abspann für die Leseansicht.
%% Der Schalter \ifkorrekturansicht ist bereits durch den Vorspann gesetzt.

%% latex-abspann.tex
%% Gemeinsamer Abspann für Korrekturansicht und Leseansicht.
%% Setzt den Schalter \ifkorrekturansicht voraus (gesetzt in den
%% einbindenden Dateien latex-korrekturansicht-abspann.tex bzw.
%% latex-leseansicht-abspann.tex).
%% ---------------------------------------------------------------

\normalsize

% Das esempio-Environment wird nur in der Leseansicht benötigt
\ifkorrekturansicht\else
\newenvironment{esempio}[3]%
{
    \vspace{1.5ex}
    \rlap{\underline{#1}}
    \par
    \setlength{\parindent}{0cm}
    \nopagebreak
    \leftskip=#2cm
    \rightskip=#3cm
}
{
    \par
}
\fi

\doendnotes{C}
\bigskip
\vfill

\clearpage

\footnotesize

\ifkorrekturansicht
  \lohead{\textsc{register}}
\fi

% theindex-Environment neu definieren ohne reledmac
\makeatletter
\renewenvironment{theindex}{%
  \ifkorrekturansicht
    \section*{\indexname}%
  \else
    \subsubsection*{Index der erwähnten Entitäten}%
  \fi
  \setlength{\parindent}{0pt}%
  \setlength{\parskip}{0pt plus 0.3pt}%
  \let\item\@idxitem
}{%
  \ifkorrekturansicht\clearpage\fi
}
\makeatother

\IfFileExists{\jobname-pw.ind}{\input{\jobname-pw.ind}}{}

% Quellenangabe nur in der Leseansicht
\ifkorrekturansicht\else
% Fallback-Definitionen, falls die .tex-Datei \titel etc. nicht gesetzt hat
\providecommand{\titel}{}
\providecommand{\editorInnen}{}
\providecommand{\dateiname}{\jobname}

\vspace{3cm}

\vfill

\footnotesize
\textsc{Quelle}: \titel. Herausgegeben von {\editorInnen}. In: \emph{Arthur Schnitzler: Briefwechsel mit Autorinnen und Autoren}.
 Digitale Edition, https://schnitzler-briefe.acdh.oeaw.ac.at/{\dateiname}.html (Stand \today)
\fi

\end{document}


