%% latex-leseansicht-vorspann.tex
%% Vorspann für die Leseansicht.
%% Lädt die gemeinsame Datei latex-vorspann.tex mit nicht gesetztem Schalter.

\newif\ifkorrekturansicht
\korrekturansichtfalse

\input{../tex-inputs/latex-vorspann}


\section[Arthur Schnitzler an Hugo von Hofmannsthal, 29. 6. 1896]{L00557 Arthur Schnitzler an Hugo von Hofmannsthal, 29. 6. 1896}
\nopagebreak\mylabel{L00557v}
\rehead{ }\normalsize\beginnumbering\briefempfaengerindex{Hofmannsthal, Hugo von@\textsc{Hofmannsthal, Hugo von}!zzzSchnitzler, Arthur@\emph{von Arthur Schnitzler}!1896-06-291@{29. 6. 1896}|(be}
\toendnotes[C]{\smallbreak\pagebreak[2]}
\correspDesc{Versand  durch Arthur Schnitzler am 29. 6. 1896 in Wien
\newline{}Erhalt  durch Hugo von Hofmannsthal im Zeitraum [29. 6. 1896
                  – 3. 7. 1896?] in Wien}\toendnotes[C]{\smallbreak}
\Standort{FDH, Hs-30885,50.}
\physDesc{Brief, 1 Blatt, 4 Seiten, 1200 Zeichen
\newline{}Handschrift: schwarze Tinte, deutsche Kurrent}
\buchAbdrucke{\weitereDrucke{Hugo von Hofmannsthal, Arthur Schnitzler: \emph{Briefwechsel}. Herausgegeben von Therese Nickl und Heinrich Schnitzler. Frankfurt am Main: \emph{S. Fischer} 1964, S. 68–69.} }\toendnotes[C]{\smallbreak}
\pstart
           \raggedleft{}{\pb}Wien\oindex{Wien@\textbf{Wien}, \emph{Verwaltungsgebiet}|pw}{ }29. Juni 96\pend
           \vspace{0.5em}
\pstart
           Mein lieber Hugo, ich lege Ihnen einen Zettel bei, da{ }ſteht drauf,
               wo ich für Briefe zu erreichen bin, u. bis wann. In Wien\oindex{Wien@\textbf{Wien}, \emph{Verwaltungsgebiet}|pw} bin ich noch bis zum Freitag (ſpäteſtens)
                  (3. Juli). –\pend
           
\pstart
           Ich wollte eben niederſchreiben, daſs ich mich »freue« u. habe gezögert, weil die
               Freude nicht ganz rein iſt. Es iſt, durch heftigeres Erklin{\pb}gen \label{K_L00557-1v}\edtext{früherer Lebensbeziehungen}{\lemma{\textnormal{\emph{früherer Lebensbeziehungen}}}\Cendnote{\textnormal{In den vorangehenden Tagen stand Schnitzler in
                  Kontakt mit Olga Waissnix\pwindex{Waissnix, Olga 3.\,11.\,1862 Wien – 4.\,11.\,1897 ebd.@\textsc{Waissnix, Olga} (3.\,11.\,1862 Wien – 4.\,11.\,1897 ebd.), \emph{Hotelière}|pwk} und Marie Glümer\pwindex{Glümer, Marie 3.\,7.\,1867 Wien – 16.\,11.\,1925 München@\textsc{Glümer, Marie} (3.\,7.\,1867 Wien – 16.\,11.\,1925 München), \emph{Schauspielerin}|pwk}.}}}\label{K_L00557-1}, in der letzten Zeit
               wieder manche Unruhe in mich gekommen, die in manchen Stunden, beſonders Abendſtunden
               allein auf dem Land,{ }ſchmerzlich bewegt. Nun weiſs ich nicht, ob{ }ſich das da oben
               gänzlich beruhigen wird oder ob nicht vielleicht noch dunklere Traurigkeit ko{\geminationm}en mag. Ich leide gewiſs an {\pb}einer gewiſſen \introOben{}(\introOben{}ſentimentalen\introOben{}!)\introOben{}
               Ueberempfindlichkeit für gewiſſe Begriffe, wie Ferne, Einſamkeit, und Vergangen. Das
               hängt wohl mit meine\textcolor{gray}{n} mangelnden Fähigkeit\textcolor{gray}{en}{ }\introOben{}abzuſchließen\introOben{} zusa{\geminationm}en.
               Abzuſchließen, in jedem Sinn. Fehler meines Lebens und meiner Kunſt{ }ſind daraus zu
               erklären.\pend
           
\pstart
           – Das Stück\pwindex{Schnitzler, Arthur 15.\,5.\,1862 Wien – 21.\,10.\,1931 ebd.@\textsc{Schnitzler, Arthur} (15.\,5.\,1862 Wien – 21.\,10.\,1931 ebd.), \emph{Schriftsteller, Mediziner}!Freiwild. Schauspiel in 3 Akten@\strich\emph{Freiwild. Schauspiel in 3 Akten}|pwv} reiſt natürlich
               mit; {\pb}iſt Ihnen noch was dazu eingefallen?\pend
           
\pstart
           – Iſt das eine\pwindex{Hofmannsthal, Hugo von 1.\,2.\,1874 Wien – 15.\,7.\,1929 Rodaun@\textsc{Hofmannsthal, Hugo von} (1.\,2.\,1874 Wien – 15.\,7.\,1929 Rodaun), \emph{Schriftsteller}!Geschichte der beiden Liebespaare@\strich\emph{Geschichte der beiden Liebespaare}|pwv} Ihrer \label{K_L00557-2v}\edtext{Soldatengeſchichten}{\lemma{\textnormal{\emph{Soldatengeschichten}}}\Cendnote{\textnormal{Mehrere Texte aus dieser Zeit spielen im
                  Milieu des Militärs.}}}\label{K_L00557-2}, die Sie{ }ſchreiben? –\pend
           
\pstart
           Sie hören{ }ſehr bald von mir u. laſſen mich wohl auch nicht lang ohne Nachricht.
               Empfehlen Sie mich Ihren Eltern\pwindex{Hofmannsthal, Hugo August von 21.\,12.\,1841 Wien – 8.\,12.\,1915 ebd.@\textsc{Hofmannsthal, Hugo August von} (21.\,12.\,1841 Wien – 8.\,12.\,1915 ebd.), \emph{Bankdirektor}|pwv}\pwindex{Hofmannsthal, Anna von 27.\,1.\,1849 Wien – 22.\,3.\,1904 Sanatorium Fürth@\textsc{Hofmannsthal, Anna von} (27.\,1.\,1849 Wien – 22.\,3.\,1904 Sanatorium Fürth)|pwv}. Seien Sie herzlich gegrüßt.\pend
           \pstart Ihr \spacefill\mbox{Arthur}\pend{}\selectlanguage{ngerman}\endnumbering\briefempfaengerindex{Hofmannsthal, Hugo von@\textsc{Hofmannsthal, Hugo von}!zzzSchnitzler, Arthur@\emph{von Arthur Schnitzler}!1896-06-291@{29. 6. 1896}|)be}\mylabel{L00557h}  \newcommand{\dateiname}{L00557}\newcommand{\titel}{Arthur Schnitzler an Hugo von Hofmannsthal, 29. 6. 1896}\newcommand{\editorInnen}{Martin Anton Müller und Gerd-Hermann Susen}%% latex-leseansicht-abspann.tex
%% Abspann für die Leseansicht.
%% Der Schalter \ifkorrekturansicht ist bereits durch den Vorspann gesetzt.

%% latex-abspann.tex
%% Gemeinsamer Abspann für Korrekturansicht und Leseansicht.
%% Setzt den Schalter \ifkorrekturansicht voraus (gesetzt in den
%% einbindenden Dateien latex-korrekturansicht-abspann.tex bzw.
%% latex-leseansicht-abspann.tex).
%% ---------------------------------------------------------------

\normalsize

% Das esempio-Environment wird nur in der Leseansicht benötigt
\ifkorrekturansicht\else
\newenvironment{esempio}[3]%
{
    \vspace{1.5ex}
    \rlap{\underline{#1}}
    \par
    \setlength{\parindent}{0cm}
    \nopagebreak
    \leftskip=#2cm
    \rightskip=#3cm
}
{
    \par
}
\fi

\doendnotes{C}
\bigskip
\vfill

\clearpage

\footnotesize

\ifkorrekturansicht
  \lohead{\textsc{register}}
\fi

% theindex-Environment neu definieren ohne reledmac
\makeatletter
\renewenvironment{theindex}{%
  \ifkorrekturansicht
    \section*{\indexname}%
  \else
    \subsubsection*{Index der erwähnten Entitäten}%
  \fi
  \setlength{\parindent}{0pt}%
  \setlength{\parskip}{0pt plus 0.3pt}%
  \let\item\@idxitem
}{%
  \ifkorrekturansicht\clearpage\fi
}
\makeatother

\IfFileExists{\jobname-pw.ind}{\input{\jobname-pw.ind}}{}

% Quellenangabe nur in der Leseansicht
\ifkorrekturansicht\else
% Fallback-Definitionen, falls die .tex-Datei \titel etc. nicht gesetzt hat
\providecommand{\titel}{}
\providecommand{\editorInnen}{}
\providecommand{\dateiname}{\jobname}

\vspace{3cm}

\vfill

\footnotesize
\textsc{Quelle}: \titel. Herausgegeben von {\editorInnen}. In: \emph{Arthur Schnitzler: Briefwechsel mit Autorinnen und Autoren}.
 Digitale Edition, https://schnitzler-briefe.acdh.oeaw.ac.at/{\dateiname}.html (Stand \today)
\fi

\end{document}


