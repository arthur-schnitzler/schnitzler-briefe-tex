\input{../tex-inputs/latex-pdf-vorspann}
\begin{center}
            \textcolor{red}{ENTWURF. ENTZIFFERUNG NOCH NICHT KORREKTURGELESEN}
                      \end{center}
            
               \section[Arthur Schnitzler an Richard Beer-Hofmann, {[}Mitte Januar 1907?{]}]{ Arthur Schnitzler an Richard Beer-Hofmann, {[}Mitte Januar 1907?{]}}\nopagebreak\mylabel{v}\rehead{ }\begin{ledgroupsized}[t]{13cm}\normalsize\beginnumbering\briefempfaengerindex{Beer-Hofmann, Richard@\textsc{Beer-Hofmann, Richard}!zzzSchnitzler, Arthur@\emph{von Arthur Schnitzler}!1907-01-151@{{[}Mitte Januar 1907?{]}}|(be} \toendnotes[C]{\smallbreak\pagebreak[2]} \Standort{YCGL, MSS 31.}
\physDesc{Brief, 1 Blatt, 2 Seiten, Umschlag
\newline{}Handschrift: Bleistift, deutsche Kurrent\newline{}Versand: ohne postalischen Übermittlungsvermerk }\toendnotes[C]{\smallbreak}\pstart{}{\pb}\textcolor{gray}{\textbf{Dr. Arthur Schnitzler}}\pend{}\pstart{}\textcolor{gray}{\textbf{Wien, XVIII. Spoettelgasse 7\oindex{Edmund-Weiss-Gasse@\textbf{Edmund-Weiß-Gasse}|pw}.}}\pend{}{\bigskip}\pstart{}{\pb}Hrn \textsc{Dr. Richard
                     Beer-Hofmann}\pend{}\pstart{}Wien XVIII\oindex{XVIII., Waehring@\textbf{XVIII., Währing}|pw}\pend{}\pstart{}\textsc{Hasenauerstraße}\oindex{Hasenauerstrasse@\textbf{Hasenauerstraße}|pw}\pend{}{\bigskip}\pstart
           \noindent{}{\pb}\textcolor{gray}{\textbf{Dr. Arthur Schnitzler}}{\\}\textcolor{gray}{\textbf{Wien, XVIII. Spoettelgasse 7\oindex{Edmund-Weiss-Gasse@\textbf{Edmund-Weiß-Gasse}|pw}.}}\pend
           \pstart{}lieber Richard,\pend\pstart
           \label{K_L01652_1v}\edtext{Ende Feber}{\lemma{\textnormal{\emph{Ende Feber}}}\Cendnote{\textnormal{Die Veranstaltung fand bereits am 10. 2. 1907 statt.}}}\label{K_L01652_1h} ſoll ein
               Vortragsabend für die \uline{jüdiſchen Waiſen} ſtattfinden;
               es iſt der einzige Fall, in dem ich heuer zugeſagt habe; außer mir ſollen \textsc{Wasserma{\geminationn}\pwindex{Wassermann, Jakob 10.03.1873 – 01.01.1934@\textsc{Wassermann, Jakob} (10.03.1873 – 01.01.1934), \emph{Schriftsteller}|pw}} u \textsc{Salten}\pwindex{Salten, Felix 06.09.1869 – 08.10.1945@\textsc{Salten, Felix} (06.09.1869 – 08.10.1945), \emph{Schriftsteller, Journalist}|pw} leſen – vielleicht beſtimmt {\pb}Sie der gute Zweck
                  \label{K_L01652_2v}\edtext{mitzuthun}{\lemma{\textnormal{\emph{mitzuthun}}}\Cendnote{\textnormal{Das Korrespondenzstück ist undatiert. Da die bis zum [14. 1. 1907] mögliche Teilnahme Hofmannsthal\pwindex{Hofmannsthal, Hugo von 01.02.1874 – 15.07.1929@\textsc{Hofmannsthal, Hugo von} (01.02.1874 – 15.07.1929), \emph{Schriftsteller}|pwk}s keine Erwähnung findet, wird das
                  Korrespondenzstück danach angesiedelt. Beer-Hofmann\pwindex{Beer-Hofmann, Richard 11.07.1866 – 26.09.1945@\textsc{Beer-Hofmann, Richard} (11.07.1866 – 26.09.1945), \emph{Schriftsteller}|pwk} nahm teil.}}}\label{K_L01652_2h}.\pend
           \pstart
           Herzlichſt{\\[\baselineskip]}Ihr{\\[\baselineskip]}\spacefill\mbox{A.}\pend
           \leftskip=0em{}\endnumbering\briefempfaengerindex{Beer-Hofmann, Richard@\textsc{Beer-Hofmann, Richard}!zzzSchnitzler, Arthur@\emph{von Arthur Schnitzler}!1907-01-151@{{[}Mitte Januar 1907?{]}}|)be}\mylabel{h}\end{ledgroupsized}  \newcommand{\dateiname}{L01652}\newcommand{\titel}{Arthur Schnitzler an Richard Beer-Hofmann, [Mitte Januar 1907?]}\newcommand{\editorInnen}{Martin Anton Müller und Gerd-Hermann Susen}\input{../tex-inputs/latex-pdf-abspann}
      