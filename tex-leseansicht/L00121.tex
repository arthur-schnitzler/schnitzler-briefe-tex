%% latex-leseansicht-vorspann.tex
%% Vorspann für die Leseansicht.
%% Lädt die gemeinsame Datei latex-vorspann.tex mit nicht gesetztem Schalter.

\newif\ifkorrekturansicht
\korrekturansichtfalse

\input{../tex-inputs/latex-vorspann}


\section[Hugo von Hofmannsthal an Arthur Schnitzler, 7. 9. [1892]]{L00121 Hugo von Hofmannsthal an Arthur Schnitzler, 7. 9. [1892]}
\nopagebreak\mylabel{L00121v}
\rehead{ }\normalsize\beginnumbering\briefempfaengerindex{Schnitzler, Arthur@\textsc{Schnitzler, Arthur}!zzzHofmannsthal, Hugo von@\emph{von Hugo von Hofmannsthal}!1892-09-071@{7. 9. [1892]}|(be}
\toendnotes[C]{\smallbreak\pagebreak[2]}
\correspDesc{Versand  durch Hugo von Hofmannsthal am 7. 9. [1892] in Lélex
\newline{}Erhalt  durch Arthur Schnitzler im Zeitraum [8. 9. 1892
                  – 12. 9. 1892?] in Bad Ischl}\toendnotes[C]{\smallbreak}
\Standort{CUL, Schnitzler, B 43.}
\physDesc{Brief, 1 Blatt, 3 Seiten, 1014 Zeichen
\newline{}Handschrift: blaue Tinte, deutsche Kurrent
\newline{}Schnitzler: mit Bleistift nummeriert: »\strikeout{30}
                                 31« und die Jahreszahl ergänzt: »92« }
\buchAbdrucke{\weitereDrucke{1) Hugo von Hofmannsthal, Arthur Schnitzler: \emph{Briefwechsel}. Herausgegeben von Therese Nickl und Heinrich Schnitzler. Frankfurt am Main: \emph{S. Fischer} 1964, S. 28–29.} \weitereDrucke{2) Hermann Bahr, Arthur Schnitzler: \emph{Briefwechsel, Aufzeichnungen, Dokumente (1891–1931)}. Herausgegeben von Kurt Ifkovits und Martin Anton Müller. Göttingen: \emph{Wallstein} 2018, S. 27.} }\toendnotes[C]{\smallbreak}
\pstart
           {\pb}\textsc{Lélex. (Ain)}\oindex{Lélex@\textbf{Lélex}|pw}\hfill \textsc{7. sept.}\pend
           \vspace{0.5em}
\pstart
           Fünf Stunden von der Eiſenbahn. Keine Zeitung. Kühe. \textsc{Monsieur le
                  curé qui fait des enfants aux jolies paysannes.} Der Gendarm: \textsc{Pandore\pwindex{\textcolor{red}{\textsuperscript{XXXX indx1}}!Pandore@\strich\emph{Pandore}|pw}}. Die alten Fliegenſchimmel des Wirths: \textsc{Pyrame et Thisbé\pwindex{\textcolor{red}{\textsuperscript{XXXX indx1}}!Midsommer nights dreame@\strich\emph{A Midsommer nights dreame}|pwv}}. Die Hauskatze: \textsc{Madeleine}. Der Nachttopf: \textsc{Monsieur Jules}.\pend
           
\pstart
           \centering{}– – – –\pend
           
\pstart
           Lange grüne Hochplateaus mit Farrnkraut und Jurakalk; dahinter der große See und der
                  \textsc{Montblanc\oindex{Mont Blanc@\textbf{Mont Blanc}, \emph{Berg}|pw}} und Herr \textsc{Edouard Rod\pwindex{Rod, Édouard 31.\,3.\,1857 Nyon – 1910 Grasse@\textsc{Rod, Édouard} (31.\,3.\,1857 Nyon – 1910 Grasse), \emph{Schriftsteller}|pw}}\pwindex{Rod, Édouard 31.\,3.\,1857 Nyon – 1910 Grasse@\textsc{Rod, Édouard} (31.\,3.\,1857 Nyon – 1910 Grasse), \emph{Schriftsteller}|pw}.\pend
           
\pstart
           {\pb}Gang der Handlung: Ich werde
               behandelt, wie der \label{K_L00121-1v}\edtext{kleine Dauphin\pwindex{Bourbon, Louis Charles de 27.\,3.\,1785 Versailles – 8.\,6.\,1795 Paris@\textsc{Bourbon, Louis Charles de} (27.\,3.\,1785 Versailles – 8.\,6.\,1795 Paris), \emph{Dauphin}|pwv} beim böſen
               Schuſter \textsc{Simon}\pwindex{Simon, Antoine 21.\,10.\,1736 Troyes – 1794 Paris@\textsc{Simon, Antoine} (21.\,10.\,1736 Troyes – 1794 Paris), \emph{Schuster, Revolutionär}|pw}}{\lemma{\textnormal{\emph{kleine … Simon}}}\Cendnote{\textnormal{1793 wurde der ehemalige Thronfolger Louis Charles de Bourbon\pwindex{Bourbon, Louis Charles de 27.\,3.\,1785 Versailles – 8.\,6.\,1795 Paris@\textsc{Bourbon, Louis Charles de} (27.\,3.\,1785 Versailles – 8.\,6.\,1795 Paris), \emph{Dauphin}|pwk} dem Schuster Alain Simon\pwindex{Simon, Antoine 21.\,10.\,1736 Troyes – 1794 Paris@\textsc{Simon, Antoine} (21.\,10.\,1736 Troyes – 1794 Paris), \emph{Schuster, Revolutionär}|pwk} zur ›Erziehung‹ überantwortet.}}}\label{K_L00121-1}. Man giebt
               mir mehr grüne und gelbe \label{K_L00121-2v}\edtext{Chartreuſe}{\lemma{\textnormal{\emph{Chartreuse}}}\Cendnote{\textnormal{Kräuterlikör}}}\label{K_L00121-2} zu
               trinken, als einem Steinklopfer, und dann muſs ich Lieder im Patois lernen und{ }ſingen, z. B.\pend
           \stanza{}\label{K_L00121-3v}\edtext{\textsc{Z’ame les bouguettes}\newverse{}\textsc{Et les matafans}\newverse{}\textsc{Et les dsones feuilles}\newverse{}\textsc{Qu’ont lo tétés blancs!}\newverse{}– – – – –\newverse{}\hspace*{2.5em}(unanständig)}{\lemma{\textnormal{\emph{Z’ame … blancs!– – – – –(unanständig)}}}\Cendnote{\textnormal{Es handelt sich um ein Lied, mit dem
                  nach Bougettes (einer herausgebackene Speise aus Ei, Mehl und Kartoffeln) und
                  Matafans (einer dem Crêpe verwandten, herausgebackenen Speise aus Mehl und
                  Kartoffeln) verlangt wurde. Die letzten beiden Verse besagen, dass der Sänger
                  zudem eine Vorliebe für weiße Brüste besitzt.}}}\label{K_L00121-3}\stanzaend{}
\pstart
           \textsc{{\pb}\label{K_L00121-4v}\edtext{Voilà ce qu’on ap\strikeout{p}elle se dépayser}{\lemma{\textnormal{\emph{Voilà … dépayser}}}\Cendnote{\textnormal{sinngemäß: Das heißt es, sich in ein fremdes Land zu
                     begeben.}}}\label{K_L00121-4}}; \label{LL417-1v}ſiehe Hermann Bahr\pwindex{Bahr, Hermann 19.\,7.\,1863 Linz – 15.\,1.\,1934 München@\textsc{Bahr, Hermann} (19.\,7.\,1863 Linz – 15.\,1.\,1934 München), \emph{Schriftsteller, Kritiker}|pw}, ges. Werke, \textsc{passim}\label{LL417-1h} »über die rechte Art in fremden Ländern zu reiſen«. Dienstag beginnt
               eigentlich meine Reiſe in die
                  Provinzen des mittäglichen Frankreich\oindex{Frankreich@\textbf{Frankreich}|pw}\pwindex{\textcolor{red}{\textsuperscript{XXXX indx1}}!Reise in die mittäglichen Provinzen von Frankreich@\strich\emph{Reise in die mittäglichen Provinzen von Frankreich}|pwv}.\pend
           
\pstart
           Schreiben Sie, bitte, zwiſchen 10. und 16. nach \textsc{Arles, Bouches-du-Rhône}\oindex{Arles@\textbf{Arles}, \emph{Region}|pw}{ }\textsc{poste rest.}\pend
           
\pstart
           \uline{\textsc{via Buchs\oindex{Buchs@\textbf{Buchs}, \emph{Hauptstadt}|pw}{ }Genève\oindex{Genf@\textbf{Genf}|pw}}}\pend
           \pstart \spacefill\mbox{Hugo.}\pend{}\selectlanguage{ngerman}\endnumbering\briefempfaengerindex{Schnitzler, Arthur@\textsc{Schnitzler, Arthur}!zzzHofmannsthal, Hugo von@\emph{von Hugo von Hofmannsthal}!1892-09-071@{7. 9. [1892]}|)be}\mylabel{L00121h}  \newcommand{\dateiname}{L00121}\newcommand{\titel}{Hugo von Hofmannsthal an Arthur Schnitzler, 7. 9. [1892]}\newcommand{\editorInnen}{Herausgegeben von Martin Anton Müller}%% latex-leseansicht-abspann.tex
%% Abspann für die Leseansicht.
%% Der Schalter \ifkorrekturansicht ist bereits durch den Vorspann gesetzt.

%% latex-abspann.tex
%% Gemeinsamer Abspann für Korrekturansicht und Leseansicht.
%% Setzt den Schalter \ifkorrekturansicht voraus (gesetzt in den
%% einbindenden Dateien latex-korrekturansicht-abspann.tex bzw.
%% latex-leseansicht-abspann.tex).
%% ---------------------------------------------------------------

\normalsize

% Das esempio-Environment wird nur in der Leseansicht benötigt
\ifkorrekturansicht\else
\newenvironment{esempio}[3]%
{
    \vspace{1.5ex}
    \rlap{\underline{#1}}
    \par
    \setlength{\parindent}{0cm}
    \nopagebreak
    \leftskip=#2cm
    \rightskip=#3cm
}
{
    \par
}
\fi

\doendnotes{C}
\bigskip
\vfill

\clearpage

\footnotesize

\ifkorrekturansicht
  \lohead{\textsc{register}}
\fi

% theindex-Environment neu definieren ohne reledmac
\makeatletter
\renewenvironment{theindex}{%
  \ifkorrekturansicht
    \section*{\indexname}%
  \else
    \subsubsection*{Index der erwähnten Entitäten}%
  \fi
  \setlength{\parindent}{0pt}%
  \setlength{\parskip}{0pt plus 0.3pt}%
  \let\item\@idxitem
}{%
  \ifkorrekturansicht\clearpage\fi
}
\makeatother

\IfFileExists{\jobname-pw.ind}{\input{\jobname-pw.ind}}{}

% Quellenangabe nur in der Leseansicht
\ifkorrekturansicht\else
% Fallback-Definitionen, falls die .tex-Datei \titel etc. nicht gesetzt hat
\providecommand{\titel}{}
\providecommand{\editorInnen}{}
\providecommand{\dateiname}{\jobname}

\vspace{3cm}

\vfill

\footnotesize
\textsc{Quelle}: \titel. Herausgegeben von {\editorInnen}. In: \emph{Arthur Schnitzler: Briefwechsel mit Autorinnen und Autoren}.
 Digitale Edition, https://schnitzler-briefe.acdh.oeaw.ac.at/{\dateiname}.html (Stand \today)
\fi

\end{document}


