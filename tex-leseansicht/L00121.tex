%% latex-korrekturansicht-vorspann.tex
%% Vorspann für die Korrekturansicht.
%% Lädt die gemeinsame Datei latex-vorspann.tex mit gesetztem Schalter.

\newif\ifkorrekturansicht
\korrekturansichttrue

\input{../tex-inputs/latex-vorspann}


\section[Hugo von Hofmannsthal an Arthur Schnitzler, 7. 9. {[}1892{]}]{L00121 Hugo von Hofmannsthal an Arthur Schnitzler, 7. 9. {[}1892{]}}
\nopagebreak\mylabel{L00121v}
\rehead{ }\normalsize\beginnumbering\briefempfaengerindex{Schnitzler, Arthur@\textsc{Schnitzler, Arthur}!zzzHofmannsthal, Hugo von@\emph{von Hugo von Hofmannsthal}!1892-09-071@{7. 9. {[}1892{]}}|(be}
\toendnotes[C]{\smallbreak\pagebreak[2]}\Standort{CUL, Schnitzler, B 43.}
\physDesc{Brief, 1 Blatt, 3 Seiten, 1014 Zeichen
\newline{}Handschrift: blaue Tinte, deutsche Kurrent
\newline{}Schnitzler: mit Bleistift nummeriert: »\strikeout{30}
                                 31« und die Jahreszahl ergänzt: »92« }
\buchAbdrucke{\weitereDrucke{1) Hugo von Hofmannsthal, Arthur Schnitzler: \emph{Briefwechsel}. Frankfurt am Main: \emph{S. Fischer} 1964, S. 28–29.} \weitereDrucke{2) Hermann Bahr, Arthur Schnitzler: \emph{Briefwechsel, Aufzeichnungen, Dokumente (1891–1931)}. Göttingen: \emph{Wallstein} 2018, S. 27.} }\toendnotes[C]{\smallbreak}
\pstart
           {\pb}\textsc{Lélex. (Ain)}\oindex{Lelex@\textbf{Lélex}, \emph{P.PPL}|pw}\hfill \textsc{7. sept.}\pend
           \vspace{0.5em}
\pstart
           Fünf Stunden von der Eiſenbahn. Keine Zeitung. Kühe. \textsc{Monsieur le
                  curé qui fait des enfants aux jolies paysannes.} Der Gendarm: \textsc{Pandore\pwindex{Pandore@\emph{Pandore}|pw}}. Die alten Fliegenſchimmel des Wirths: \textsc{Pyrame et Thisbé\pwindex{Midsommer nights dreame@\emph{A Midsommer nights dreame}|pwv}}. Die Hauskatze: \textsc{Madeleine}. Der Nachttopf: \textsc{Monsieur Jules}.\pend
           
\pstart
           \centering{}– – – –\pend
           
\pstart
           Lange grüne Hochplateaus mit Farrnkraut und Jurakalk; dahinter der große See und der
                  \textsc{Montblanc\oindex{Mont Blanc@\textbf{Mont Blanc}, \emph{T.MT}|pw}} und Herr \textsc{Edouard Rod\pwindex{Rod, Edouard 1857-03-31 – 1910@\textsc{Rod, Édouard} (1857-03-31 – 1910), \emph{Schriftsteller/Schriftstellerin}|pw}}\pwindex{Rod, Edouard 1857-03-31 – 1910@\textsc{Rod, Édouard} (1857-03-31 – 1910), \emph{Schriftsteller/Schriftstellerin}|pw}.\pend
           
\pstart
           {\pb}Gang der Handlung: Ich werde
               behandelt, wie der \label{K_L00121-1v}\edtext{kleine Dauphin\pwindex{Bourbon, Louis Charles de 1785-03-27 – 1795-06-08@\textsc{Bourbon, Louis Charles de} (1785-03-27 – 1795-06-08), \emph{Dauphin/Dauphine}|pwv} beim böſen
               Schuſter \textsc{Simon}\pwindex{Simon, Antoine 1736-10-21 – 1794@\textsc{Simon, Antoine} (1736-10-21 – 1794), \emph{Schuster/Schusterin, Revolutionär/Revolutionärin}|pw}}{\lemma{\textnormal{\emph{kleine … Simon}}}\Cendnote{\textnormal{1793 wurde der ehemalige Thronfolger Louis Charles de Bourbon\pwindex{Bourbon, Louis Charles de 1785-03-27 – 1795-06-08@\textsc{Bourbon, Louis Charles de} (1785-03-27 – 1795-06-08), \emph{Dauphin/Dauphine}|pwk} dem Schuster Alain Simon\pwindex{Simon, Antoine 1736-10-21 – 1794@\textsc{Simon, Antoine} (1736-10-21 – 1794), \emph{Schuster/Schusterin, Revolutionär/Revolutionärin}|pwk} zur ›Erziehung‹ überantwortet.}}}\label{K_L00121-1}. Man giebt
               mir mehr grüne und gelbe \label{K_L00121-2v}\edtext{Chartreuſe}{\lemma{\textnormal{\emph{Chartreuſe}}}\Cendnote{\textnormal{Kräuterlikör}}}\label{K_L00121-2} zu
               trinken, als einem Steinklopfer, und dann muſs ich Lieder im Patois lernen und
               ſingen, z. B.\pend
           \stanza{}\label{K_L00121-3v}\edtext{\textsc{Z’ame les bouguettes}\textsc{Et les matafans}\textsc{Et les dsones feuilles}\textsc{Qu’ont lo tétés blancs!}– – – – –\hspace*{2.5em}(unanständig)}{\lemma{\textnormal{\emph{Z’ame … blancs!– – – – –(unanständig)}}}\Cendnote{\textnormal{Es handelt sich um ein Lied, mit dem
                  nach Bougettes (einer herausgebackene Speise aus Ei, Mehl und Kartoffeln) und
                  Matafans (einer dem Crêpe verwandten, herausgebackenen Speise aus Mehl und
                  Kartoffeln) verlangt wurde. Die letzten beiden Verse besagen, dass der Sänger
                  zudem eine Vorliebe für weiße Brüste besitzt.}}}\label{K_L00121-3}\stanzaend{}
\pstart
           \textsc{{\pb}\label{K_L00121-4v}\edtext{Voilà ce qu’on ap\strikeout{p}elle se dépayser}{\lemma{\textnormal{\emph{Voilà … dépayser}}}\Cendnote{\textnormal{sinngemäß: Das heißt es, sich in ein fremdes Land zu
                     begeben.}}}\label{K_L00121-4}}; \label{LL417-1v}ſiehe Hermann Bahr\pwindex{Bahr, Hermann 19.07.1863 – 15.01.1934@\textsc{Bahr, Hermann} (19.07.1863 – 15.01.1934), \emph{Schriftsteller/Schriftstellerin, Kritiker/Kritikerin}|pw}, ges. Werke, \textsc{passim}\label{LL417-1h} »über die rechte Art in fremden Ländern zu reiſen«. Dienstag beginnt
               eigentlich meine Reiſe in die
                  Provinzen des mittäglichen Frankreich\oindex{Frankreich@\textbf{Frankreich}, \emph{A.PCLI}|pw}\pwindex{Reise in die mittaeglichen Provinzen von Frankreich@\emph{Reise in die mittäglichen Provinzen von Frankreich}|pwv}.\pend
           
\pstart
           Schreiben Sie, bitte, zwiſchen 10. und 16. nach \textsc{Arles, Bouches-du-Rhône}\oindex{Arles@\textbf{Arles}, \emph{A.ADM4}|pw}{ }\textsc{poste rest.}\pend
           
\pstart
           \uline{\textsc{via Buchs\oindex{Buchs@\textbf{Buchs}, \emph{P.PPLA3}|pw}{ }Genève\oindex{Genf@\textbf{Genf}, \emph{P.PPLA}|pw}}}\pend
           \pstart \spacefill\mbox{Hugo.}\pend{}\selectlanguage{ngerman}\endnumbering\briefempfaengerindex{Schnitzler, Arthur@\textsc{Schnitzler, Arthur}!zzzHofmannsthal, Hugo von@\emph{von Hugo von Hofmannsthal}!1892-09-071@{7. 9. {[}1892{]}}|)be}\mylabel{L00121h}  \normalsize

\doendnotes{C}
\bigskip
\vfill

\clearpage

\footnotesize

\lohead{\textsc{register}}

% Definiere theindex-Environment komplett neu ohne reledmac
\makeatletter
\renewenvironment{theindex}{%
  \section*{\indexname}%
  \setlength{\parindent}{0pt}%
  \setlength{\parskip}{0pt plus 0.3pt}%
  \let\item\@idxitem
}{%
  \clearpage
}
\makeatother

\IfFileExists{\jobname-pw.ind}{\input{\jobname-pw.ind}}{}

\end{document}

      