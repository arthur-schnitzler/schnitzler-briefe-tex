%% latex-korrekturansicht-vorspann.tex
%% Vorspann für die Korrekturansicht.
%% Lädt die gemeinsame Datei latex-vorspann.tex mit gesetztem Schalter.

\newif\ifkorrekturansicht
\korrekturansichttrue

\input{../tex-inputs/latex-vorspann}


\section[ Paul Goldmann an Arthur Schnitzler, 20. 11. {[}1900{]}]{L02939 Paul Goldmann an Arthur Schnitzler, 20. 11. {[}1900{]}}
\nopagebreak\mylabel{L02939v}
\rehead{ }\normalsize\beginnumbering\briefempfaengerindex{Schnitzler, Arthur@\textsc{Schnitzler, Arthur}!zzzGoldmann, Paul@\emph{von Paul Goldmann}!1900-11-201@{20. 11. {[}1900{]}}|(be}
\toendnotes[C]{\smallbreak\pagebreak[2]}\Standort{DLA, A:Schnitzler, HS.NZ85.1.3170.}
\physDesc{Brief, 1 Blatt, 2 Seiten, 701 Zeichen
\newline{}Handschrift: blaue Tinte, deutsche Kurrent
\newline{}Schnitzler: 1) mit Bleistift das Jahr »900« vermerkt  2) mit rotem Buntstift eine Unterstreichung}\toendnotes[C]{\smallbreak}
\pstart
           {\pb}Berlin\oindex{Berlin@\textbf{Berlin}, \emph{P.PPLC}|pw}, 20. November.\hfill \textcolor{gray}{\textbf{DESSAUERSTRASSE 19}}\oindex{Dessauer Strasse@\textbf{Dessauer Straße}, \emph{Straße (K.STR)}|pw}\pend
           
\pstart{}Mein lieber Freund,\pend\vspace{0.5em}
\pstart
           Deine Breslau\oindex{Breslau@\textbf{Breslau}, \emph{P.PPLA}|pw}er \textsc{Première\pwindex{Schleier der Beatrice. Schauspiel in fuenf Akten@\emph{Der Schleier der Beatrice. Schauspiel in fünf Akten}|pwv}} iſt, wie ich höre, \label{K_L02939-1v}\edtext{verſchoben}{\lemma{\textnormal{\emph{verſchoben}}}\Cendnote{\textnormal{Siehe Paul Goldmann an Arthur Schnitzler, 12. 11. [1900].
               }}}\label{K_L02939-1}, und ich kann Dir daher nochmals Glück auf den Weg wünſchen. Vergiß nicht,
               wenn es irgend geht, mir am Sonntag ein paar Worte zu
               telegraphiren! Dann kommſt Du hoffentlich nach Berlin\oindex{Berlin@\textbf{Berlin}, \emph{P.PPLC}|pw}. Ich hatte eigentlich gehofft, Du würdeſt ſchon voher auf einige Tage
               herkommen. Bitte, ſteige {\pb}doch diesmal nicht in dem
               ungünſtig \strikeout{gel} und entfernt gelegenen \textsc{Hôtel Continental\oindex{Hotel Continental [Berlin]@\textbf{Hotel Continental [Berlin]}, \emph{Hotel (K.HTL)}|pw}} ab, ſondern in dem auch ſonſt weit angenehmeren und auch vornehmeren \label{K_L02939-2v}\edtext{\textsc{Palast-Hotel\oindex{Palasthotel Berlin@\textbf{Palasthotel Berlin}, \emph{Hotel (K.HTL)}|pw}}}{\lemma{\textnormal{\emph{Palast-Hotel}}}\Cendnote{\textnormal{Am 28. 11. 1900 speiste Schnitzler unmittelbar vor seiner Abreise aus Berlin\oindex{Berlin@\textbf{Berlin}, \emph{P.PPLC}|pwk} im Hôtel
                     Continental\oindex{Hotel Continental [Berlin]@\textbf{Hotel Continental [Berlin]}, \emph{Hotel (K.HTL)}|pwk}. Das kann als Indiz genommen, dass Schnitzler sich nicht an Goldmanns\pwindex{Goldmann, Paul 31.01.1865 – 25.09.1935@\textsc{Goldmann, Paul} (31.01.1865 – 25.09.1935), \emph{Schriftsteller/Schriftstellerin, Journalist/Journalistin}|pwk} Rat hielt und die ganze Zeit über
                  im Hôtel Continental\oindex{Hotel Continental [Berlin]@\textbf{Hotel Continental [Berlin]}, \emph{Hotel (K.HTL)}|pwk}
                  wohnte.}}}\label{K_L02939-2}, das fünf Minuten von m\substVorne{}\textsuperscript{ir}\substDazwischen{}einer\substHinten{} Wohnung entfernt liegt.\pend
           
\pstart
           Viele treue Grüße! {\\[\baselineskip]}Dein {\\[\baselineskip]}\spacefill\mbox{Paul Goldmn}\pend
           \leftskip=0em{}
\pstart
           \noindent{}Sage doch dieſem Schurken, dem \textsc{Richard\pwindex{Beer-Hofmann, Richard 1866-07-11 – 1945-09-26@\textsc{Beer-Hofmann, Richard} (1866-07-11 – 1945-09-26), \emph{Schriftsteller/Schriftstellerin}|pw}}, er ſoll mir die \label{K_L02939-3v}\edtext{\uline{Photographien} von unſerer Reiſe}{\lemma{\textnormal{\emph{Photographien … Reiſe}}}\Cendnote{\textnormal{Siehe Paul Goldmann an Arthur Schnitzler, 16. 6. [1900].
                  }}}\label{K_L02939-3} ſchicken!\pend
           \selectlanguage{ngerman}\endnumbering\briefempfaengerindex{Schnitzler, Arthur@\textsc{Schnitzler, Arthur}!zzzGoldmann, Paul@\emph{von Paul Goldmann}!1900-11-201@{20. 11. {[}1900{]}}|)be}\mylabel{L02939h}  \normalsize

\doendnotes{C}
\bigskip
\vfill

\clearpage

\footnotesize

\lohead{\textsc{register}}

% Definiere theindex-Environment komplett neu ohne reledmac
\makeatletter
\renewenvironment{theindex}{%
  \section*{\indexname}%
  \setlength{\parindent}{0pt}%
  \setlength{\parskip}{0pt plus 0.3pt}%
  \let\item\@idxitem
}{%
  \clearpage
}
\makeatother

\IfFileExists{\jobname-pw.ind}{\input{\jobname-pw.ind}}{}

\end{document}

      