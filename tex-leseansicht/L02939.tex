%% latex-leseansicht-vorspann.tex
%% Vorspann für die Leseansicht.
%% Lädt die gemeinsame Datei latex-vorspann.tex mit nicht gesetztem Schalter.

\newif\ifkorrekturansicht
\korrekturansichtfalse

\input{../tex-inputs/latex-vorspann}


\section[ Paul Goldmann an Arthur Schnitzler, 20. 11. [1900]]{L02939 Paul Goldmann an Arthur Schnitzler,  20. 11. [1900]}
\nopagebreak\mylabel{L02939v}
\rehead{ }\normalsize\beginnumbering\briefempfaengerindex{Schnitzler, Arthur@\textsc{Schnitzler, Arthur}!zzzGoldmann, Paul@\emph{von Paul Goldmann}!1900-11-201@{20. 11. [1900]}|(be}
\toendnotes[C]{\smallbreak\pagebreak[2]}
\correspDesc{Versand  durch Paul Goldmann am 20. 11. [1900] in Berlin
\newline{}Erhalt  durch Arthur Schnitzler im Zeitraum [21. 11. 1900 – 22. 11. 1900?] in Wien?}\toendnotes[C]{\smallbreak}
\Standort{DLA, A:Schnitzler, HS.NZ85.1.3170.}
\physDesc{Brief, 1 Blatt, 2 Seiten, 701 Zeichen
\newline{}Handschrift: blaue Tinte, deutsche Kurrent
\newline{}Schnitzler: 1) mit Bleistift das Jahr »900« vermerkt  2) mit rotem Buntstift eine Unterstreichung}\toendnotes[C]{\smallbreak}
\pstart
           {\pb}Berlin\oindex{Berlin@\textbf{Berlin}, \emph{Hauptstadt}|pw}, 20. November.\hfill \textcolor{gray}{\textbf{DESSAUERSTRASSE 19}}\oindex{Dessauer Straße@\textbf{Dessauer Straße}, \emph{Straße}|pw}\pend
           
\pstart{}Mein lieber Freund,\pend\vspace{0.5em}
\pstart
           Deine Breslau\oindex{Breslau@\textbf{Breslau}|pw}er \textsc{Première\pwindex{Schnitzler, Arthur 15.\,5.\,1862 Wien – 21.\,10.\,1931 ebd.@\textsc{Schnitzler, Arthur} (15.\,5.\,1862 Wien – 21.\,10.\,1931 ebd.), \emph{Schriftsteller, Mediziner}!Schleier der Beatrice. Schauspiel in fünf Akten@\strich\emph{Der Schleier der Beatrice. Schauspiel in fünf Akten}|pwv}} iſt, wie ich höre, \label{K_L02939-1v}\edtext{verſchoben}{\lemma{\textnormal{\emph{verschoben}}}\Cendnote{\textnormal{Siehe XXXX Auszeichnungsfehler: Dokument L02938 nicht gefunden.
               }}}\label{K_L02939-1}, und ich kann Dir daher nochmals Glück auf den Weg wünſchen. Vergiß nicht,
               wenn es irgend geht, mir am Sonntag ein paar Worte zu
               telegraphiren! Dann kommſt Du hoffentlich nach Berlin\oindex{Berlin@\textbf{Berlin}, \emph{Hauptstadt}|pw}. Ich hatte eigentlich gehofft, Du würdeſt{ }ſchon voher auf einige Tage
               herkommen. Bitte,{ }ſteige {\pb}doch diesmal nicht in dem
               ungünſtig \strikeout{gel} und entfernt gelegenen \textsc{Hôtel Continental\oindex{Hotel Continental [Berlin]@\textbf{Hotel Continental [Berlin]}, \emph{Hotel}|pw}} ab,{ }ſondern in dem auch{ }ſonſt weit angenehmeren und auch vornehmeren \label{K_L02939-2v}\edtext{\textsc{Palast-Hotel\oindex{Palasthotel Berlin@\textbf{Palasthotel Berlin}, \emph{Hotel}|pw}}}{\lemma{\textnormal{\emph{Palast-Hotel}}}\Cendnote{\textnormal{Am 28. 11. 1900 speiste Schnitzler unmittelbar vor seiner Abreise aus Berlin\oindex{Berlin@\textbf{Berlin}, \emph{Hauptstadt}|pwk} im Hôtel
                     Continental\oindex{Hotel Continental [Berlin]@\textbf{Hotel Continental [Berlin]}, \emph{Hotel}|pwk}. Das kann als Indiz genommen, dass Schnitzler sich nicht an Goldmanns\pwindex{Goldmann, Paul 31.\,1.\,1865 Breslau – 25.\,9.\,1935 Wien@\textsc{Goldmann, Paul} (31.\,1.\,1865 Breslau – 25.\,9.\,1935 Wien), \emph{Schriftsteller, Journalist}|pwk} Rat hielt und die ganze Zeit über
                  im Hôtel Continental\oindex{Hotel Continental [Berlin]@\textbf{Hotel Continental [Berlin]}, \emph{Hotel}|pwk}
                  wohnte.}}}\label{K_L02939-2}, das fünf Minuten von m\substVorne{}\textsuperscript{ir}\substDazwischen{}einer\substHinten{} Wohnung entfernt liegt.\pend
           
\pstart
           Viele treue Grüße! {\\[\baselineskip]}Dein {\\[\baselineskip]}\spacefill\mbox{Paul Goldmn}\pend
           \leftskip=0em{}
\pstart
           \noindent{}Sage doch dieſem Schurken, dem \textsc{Richard\pwindex{Beer-Hofmann, Richard 11.\,7.\,1866 Wien – 26.\,9.\,1945 New York City@\textsc{Beer-Hofmann, Richard} (11.\,7.\,1866 Wien – 26.\,9.\,1945 New York City), \emph{Schriftsteller}|pw}}, er{ }ſoll mir die \label{K_L02939-3v}\edtext{\uline{Photographien} von unſerer Reiſe}{\lemma{\textnormal{\emph{Photographien … Reise}}}\Cendnote{\textnormal{Siehe XXXX Auszeichnungsfehler: Dokument L02920 nicht gefunden.
                  }}}\label{K_L02939-3}{ }ſchicken!\pend
           \selectlanguage{ngerman}\endnumbering\briefempfaengerindex{Schnitzler, Arthur@\textsc{Schnitzler, Arthur}!zzzGoldmann, Paul@\emph{von Paul Goldmann}!1900-11-201@{20. 11. [1900]}|)be}\mylabel{L02939h}  \newcommand{\dateiname}{L02939}\newcommand{\titel}{Paul Goldmann an Arthur Schnitzler, 20. 11. [1900]}\newcommand{\editorInnen}{Martin Anton Müller und Laura Untner}%% latex-leseansicht-abspann.tex
%% Abspann für die Leseansicht.
%% Der Schalter \ifkorrekturansicht ist bereits durch den Vorspann gesetzt.

%% latex-abspann.tex
%% Gemeinsamer Abspann für Korrekturansicht und Leseansicht.
%% Setzt den Schalter \ifkorrekturansicht voraus (gesetzt in den
%% einbindenden Dateien latex-korrekturansicht-abspann.tex bzw.
%% latex-leseansicht-abspann.tex).
%% ---------------------------------------------------------------

\normalsize

% Das esempio-Environment wird nur in der Leseansicht benötigt
\ifkorrekturansicht\else
\newenvironment{esempio}[3]%
{
    \vspace{1.5ex}
    \rlap{\underline{#1}}
    \par
    \setlength{\parindent}{0cm}
    \nopagebreak
    \leftskip=#2cm
    \rightskip=#3cm
}
{
    \par
}
\fi

\doendnotes{C}
\bigskip
\vfill

\clearpage

\footnotesize

\ifkorrekturansicht
  \lohead{\textsc{register}}
\fi

% theindex-Environment neu definieren ohne reledmac
\makeatletter
\renewenvironment{theindex}{%
  \ifkorrekturansicht
    \section*{\indexname}%
  \else
    \subsubsection*{Index der erwähnten Entitäten}%
  \fi
  \setlength{\parindent}{0pt}%
  \setlength{\parskip}{0pt plus 0.3pt}%
  \let\item\@idxitem
}{%
  \ifkorrekturansicht\clearpage\fi
}
\makeatother

\IfFileExists{\jobname-pw.ind}{\input{\jobname-pw.ind}}{}

% Quellenangabe nur in der Leseansicht
\ifkorrekturansicht\else
% Fallback-Definitionen, falls die .tex-Datei \titel etc. nicht gesetzt hat
\providecommand{\titel}{}
\providecommand{\editorInnen}{}
\providecommand{\dateiname}{\jobname}

\vspace{3cm}

\vfill

\footnotesize
\textsc{Quelle}: \titel. Herausgegeben von {\editorInnen}. In: \emph{Arthur Schnitzler: Briefwechsel mit Autorinnen und Autoren}.
 Digitale Edition, https://schnitzler-briefe.acdh.oeaw.ac.at/{\dateiname}.html (Stand \today)
\fi

\end{document}


