%% latex-leseansicht-vorspann.tex
%% Vorspann für die Leseansicht.
%% Lädt die gemeinsame Datei latex-vorspann.tex mit nicht gesetztem Schalter.

\newif\ifkorrekturansicht
\korrekturansichtfalse

\input{../tex-inputs/latex-vorspann}


         
         \renewcommand{\erwaehntePersonen}{Personen: Richard Beer-Hofmann}
         \renewcommand{\erwaehnteOrte}{Orte: Berlin, Breslau, Dessauer Straße, Hotel Continental (Berlin), Palasthotel Berlin, Wien}
         \renewcommand{\erwaehnteWerke}{Werke: Der Schleier der Beatrice. Schauspiel in fünf Akten}
               \section[ Paul Goldmann an Arthur Schnitzler, 20. 11. {[}1900{]}]{ Paul Goldmann an Arthur Schnitzler, 20. 11. {[}1900{]}}\nopagebreak\mylabel{v}\rehead{ }\begin{ledgroupsized}[t]{13cm}\normalsize\beginnumbering \toendnotes[C]{\smallbreak\pagebreak[2]} \Standort{DLA, A:Schnitzler, HS.NZ85.1.3170.}
\physDesc{Brief, 1 Blatt, 2 Seiten, 701 Zeichen
\newline{}Handschrift: blaue Tinte, deutsche Kurrent
\newline{}Schnitzler: 1) mit Bleistift das Jahr »900« vermerkt  2) mit rotem Buntstift eine Unterstreichung}\toendnotes[C]{\smallbreak}\pstart
           \noindent{}{\pb}Berlin\oindex{Berlin@\textbf{Berlin}|pw}, 20. November.\hfill \textcolor{gray}{\textbf{DESSAUERSTRASSE 19}}\oindex{Dessauer Strasse@\textbf{Dessauer Straße}|pw}\pend
           \pstart{}Mein lieber Freund,\pend\pstart
           Deine Breslau\oindex{Breslau@\textbf{Breslau}|pw}er \textsc{Première\pwindex{Schnitzler, Arthur 15.05.1862 – 21.10.1931@\textsc{Schnitzler, Arthur} (15.05.1862 – 21.10.1931), \emph{Schriftsteller, Mediziner}!Schleier der Beatrice. Schauspiel in fuenf Akten1900-12-01@\strich\emph{Der Schleier der Beatrice. Schauspiel in fünf Akten} {[}1900-12-01{]}|pwv}} iſt, wie ich höre, \label{K_L02939-1v}\edtext{verſchoben}{\lemma{\textnormal{\emph{verſchoben}}}\Cendnote{\textnormal{siehe Paul Goldmann an Arthur Schnitzler, 12. 11. [1900]}}}\label{K_L02939-1h}, und ich kann Dir daher nochmals Glück auf den Weg wünſchen. Vergiß nicht,
               wenn es irgend geht, mir am Sonntag ein paar Worte zu
               telegraphiren! Dann kommſt Du hoffentlich nach Berlin\oindex{Berlin@\textbf{Berlin}|pw}. Ich hatte eigentlich gehofft, Du würdeſt ſchon voher auf einige Tage
               herkommen. Bitte, ſteige {\pb}doch diesmal nicht in dem
               ungünſtig \strikeout{gel} und entfernt gelegenen \textsc{Hôtel Continental\oindex{Hotel Continental (Berlin)@\textbf{Hotel Continental (Berlin)}|pw}} ab, ſondern in dem auch ſonſt weit angenehmeren und auch vornehmeren \label{K_L02939-2v}\edtext{\textsc{Palast-Hotel\oindex{Palasthotel Berlin@\textbf{Palasthotel Berlin}|pw}}}{\lemma{\textnormal{\emph{Palast-Hotel}}}\Cendnote{\textnormal{Am 28. 11. 1900 speiste Schnitzler\pwindex{Schnitzler, Arthur 15.05.1862 – 21.10.1931@\textsc{Schnitzler, Arthur} (15.05.1862 – 21.10.1931), \emph{Schriftsteller, Mediziner}|pwk} unmittelbar vor seiner Abreise aus Berlin\oindex{Berlin@\textbf{Berlin}|pwk} im Hôtel
                     Continental\oindex{Hotel Continental (Berlin)@\textbf{Hotel Continental (Berlin)}|pwk}, was als Indiz genommen werden kann, dass er sich nicht an Goldmann\pwindex{Goldmann, Paul 31.01.1865 – 25.09.1935@\textsc{Goldmann, Paul} (31.01.1865 – 25.09.1935), \emph{Schriftsteller, Journalist}|pwk}s Rat hielt und die ganze Zeit über
                  in diesem Hotel\oindex{Hotel Continental (Berlin)@\textbf{Hotel Continental (Berlin)}|pwkv}
                  wohnte.}}}\label{K_L02939-2h}, das fünf Minuten von m\substVorne{}\textsuperscript{ir}\substDazwischen{}einer\substHinten{} Wohnung entfernt liegt.\pend
           \pstart
           Viele treue Grüße! {\\[\baselineskip]}Dein {\\[\baselineskip]}\spacefill\mbox{Paul Goldmn}\pend
           \leftskip=0em{}\pstart
           \noindent{}Sage doch dieſem Schurken, dem \textsc{Richard\pwindex{Beer-Hofmann, Richard 1866-07-11 – 1945-09-26@\textsc{Beer-Hofmann, Richard} (1866-07-11 – 1945-09-26), \emph{Schriftsteller}|pw}}, er ſoll mir die \label{K_L02939-3v}\edtext{\uline{Photographien} von unſerer Reiſe}{\lemma{\textnormal{\emph{Photographien … Reiſe}}}\Cendnote{\textnormal{siehe Paul Goldmann an Arthur Schnitzler, 16. 6. [1900]}}}\label{K_L02939-3h} ſchicken!\pend
           
         
         \endnumbering\mylabel{h}\end{ledgroupsized}  \newcommand{\dateiname}{L02939}\newcommand{\titel}{Paul Goldmann an Arthur Schnitzler, 20. 11. [1900]}\newcommand{\editorInnen}{Martin Anton Müller und Laura Untner}%% latex-leseansicht-abspann.tex
%% Abspann für die Leseansicht.
%% Der Schalter \ifkorrekturansicht ist bereits durch den Vorspann gesetzt.

%% latex-abspann.tex
%% Gemeinsamer Abspann für Korrekturansicht und Leseansicht.
%% Setzt den Schalter \ifkorrekturansicht voraus (gesetzt in den
%% einbindenden Dateien latex-korrekturansicht-abspann.tex bzw.
%% latex-leseansicht-abspann.tex).
%% ---------------------------------------------------------------

\normalsize

% Das esempio-Environment wird nur in der Leseansicht benötigt
\ifkorrekturansicht\else
\newenvironment{esempio}[3]%
{
    \vspace{1.5ex}
    \rlap{\underline{#1}}
    \par
    \setlength{\parindent}{0cm}
    \nopagebreak
    \leftskip=#2cm
    \rightskip=#3cm
}
{
    \par
}
\fi

\doendnotes{C}
\bigskip
\vfill

\clearpage

\footnotesize

\ifkorrekturansicht
  \lohead{\textsc{register}}
\fi

% theindex-Environment neu definieren ohne reledmac
\makeatletter
\renewenvironment{theindex}{%
  \ifkorrekturansicht
    \section*{\indexname}%
  \else
    \subsubsection*{Index der erwähnten Entitäten}%
  \fi
  \setlength{\parindent}{0pt}%
  \setlength{\parskip}{0pt plus 0.3pt}%
  \let\item\@idxitem
}{%
  \ifkorrekturansicht\clearpage\fi
}
\makeatother

\IfFileExists{\jobname-pw.ind}{\input{\jobname-pw.ind}}{}

% Quellenangabe nur in der Leseansicht
\ifkorrekturansicht\else
% Fallback-Definitionen, falls die .tex-Datei \titel etc. nicht gesetzt hat
\providecommand{\titel}{}
\providecommand{\editorInnen}{}
\providecommand{\dateiname}{\jobname}

\vspace{3cm}

\vfill

\footnotesize
\textsc{Quelle}: \titel. Herausgegeben von {\editorInnen}. In: \emph{Arthur Schnitzler: Briefwechsel mit Autorinnen und Autoren}.
 Digitale Edition, https://schnitzler-briefe.acdh.oeaw.ac.at/{\dateiname}.html (Stand \today)
\fi

\end{document}


      