%% latex-leseansicht-vorspann.tex
%% Vorspann für die Leseansicht.
%% Lädt die gemeinsame Datei latex-vorspann.tex mit nicht gesetztem Schalter.

\newif\ifkorrekturansicht
\korrekturansichtfalse

\input{../tex-inputs/latex-vorspann}


\section[Arthur Schnitzler an Theodor Herzl, 5. 7. 1894]{L03908 Arthur Schnitzler an Theodor Herzl, 5. 7. 1894}
\nopagebreak\mylabel{L03908v}
\rehead{ }\normalsize\beginnumbering\briefempfaengerindex{Herzl, Theodor@\textsc{Herzl, Theodor}!zzzSchnitzler, Arthur@\emph{von Arthur Schnitzler}!1894-07-051@{5. 7. 1894}|(be}
\toendnotes[C]{\smallbreak\pagebreak[2]}
\correspDesc{Versand  durch Arthur Schnitzler am 5. 7. 1894 in Wien
\newline{}Erhalt  durch Theodor Herzl in Wien}\toendnotes[C]{\smallbreak}
\Standort{Jerusalem, Central Zionist Archives, H1:1924-13.}
\physDesc{,  Blätter,  Seiten
\newline{}Handschrift: , deutsche Kurrent}
\buchAbdrucke{\weitereDrucke{Arthur Schnitzler: \emph{Briefe 1875–1912}. Herausgegeben von Therese Nickl und Heinrich Schnitzler. Frankfurt am Main: \emph{S. Fischer} 1981, S. 228–229.} }\toendnotes[C]{\smallbreak}
\pstart
           {\pb}\textsc{IX.
                     Frankgasse 1\oindex{Wien@\textbf{Wien}!IX., Alsergrund@\textbf{IX., Alsergrund}!Frankgasse 1@\textbf{Frankgasse 1}, \emph{Wohngebäude}|pw}.}\pend
           
\pstart{}Verehrter Freund!\pend\vspace{0.5em}
\pstart
           Ihre freundlichen Worte haben meine Mutter\pwindex{Schnitzler, Louise 8.\,7.\,1840 Kőszeg – 9.\,9.\,1911 Wien@\textsc{Schnitzler, Louise} (8.\,7.\,1840 Kőszeg – 9.\,9.\,1911 Wien)|pwv} und meinen Bruder\pwindex{Schnitzler, Julius 13.\,7.\,1865 Wien – 29.\,6.\,1939 ebd.@\textsc{Schnitzler, Julius} (13.\,7.\,1865 Wien – 29.\,6.\,1939 ebd.), \emph{Chirurg}|pwv}{ }ſehr erfreut, und ich danke Ihnen in ihrem und meinem Namen aufs wärmſte. –\pend
           
\pstart
           Daſs ich Ihnen nichts von meinen Sachen{ }ſchicke, nach denen Sie{ }ſich in{ }ſo
               liebenswürdiger Weiſe erkundigen, liegt wirklich weniger an mir als an den Verlegern,
               die{ }ſich noch {\pb}immer{ }ſehr lang bitten laſſen, bevor{ }ſie was
               von mir drucken. Nun, im Herbſt erſcheint eine Novelle\pwindex{Schnitzler, Arthur 15.\,5.\,1862 Wien – 21.\,10.\,1931 ebd.@\textsc{Schnitzler, Arthur} (15.\,5.\,1862 Wien – 21.\,10.\,1931 ebd.), \emph{Schriftsteller, Mediziner}!Sterben. Novelle@\strich\emph{Sterben. Novelle}|pwv} von mir bei \textsc{Fischer\orgindex{S. Fischer Verlag@S. Fischer Verlag|pw}}, und ich will mir alle Mühe geben, anderes, das nun{ }ſchon fertig im Pult liegt,
               raſcher in die Oeffentlichkeit zu befördern, als es mir bisher zu gelingen pflegte.
               Im ganzen darf ich{ }ſagen, dß ich in den letzten Monaten nicht {\pb}ſehr nachläßig war, daſs mir mancherlei einfällt und dß ich
               zuweilen die Empfindung habe, daſs ich manches von dieſem Mancherlei werde zu gutem
               Ende führen können. –\pend
           
\pstart
           Ich zweifle nicht, dß mein Freund Paul\pwindex{Goldmann, Paul 31.\,1.\,1865 Breslau – 25.\,9.\,1935 Wien@\textsc{Goldmann, Paul} (31.\,1.\,1865 Breslau – 25.\,9.\,1935 Wien), \emph{Schriftsteller, Journalist}|pw} Ihnen
               meine Grüße an Sie, die ich den Briefen an ihn häufig beifüge, regelmäßig beſtellt,
               und {\pb}Ihnen auch manchmal{ }ſagt – was{ }ſich mündlich und durch
               einen Dritten beſſer{ }ſagen läßt als in einem Brief, wo es einen süßlich faden
               Beigeſchmack von Höflichkeit oder gar Förmlichkeit beko{\geminationm}t – nemlich daß ich das
               wenige, was mir von Ihnen zugänglich iſt,{ }ſtets mit wahrhaftem Genuße leſe. Beſonders
               im Laufe des letzten Jahres haben Sie einige kleine Kunſtwerke {\pb}von Feu{[}i{]}lletons geſchaffen, die nicht
               mit den Zeitungen selbst verwehen dürften. Sie wiſſen das{ }ſelbſt und man darf es
               Ihnen wohl{ }ſo unbefangen ins Geſicht{ }ſagen wie eine Grobheit. – Und die Bühne? Iſt
               Ihre Luſt zum dramatiſchen gänzlich durch den Ekel erſtickt worden? Wie oft hab’ ich
               in dieſem Winter an Ihre \label{K_L03908-11v}\edtext{ſchönen und
               wahren Worte}{\lemma{\textnormal{\emph{schönen und
               wahren Worte}}}\Cendnote{\textnormal{XXXX Auszeichnungsfehler: Dokument L03823 nicht gefunden.}}}\label{K_L03908-11} denken müſſen, die Sie mir lang vor der Aufführung
               meines »Märchens\pwindex{Schnitzler, Arthur 15.\,5.\,1862 Wien – 21.\,10.\,1931 ebd.@\textsc{Schnitzler, Arthur} (15.\,5.\,1862 Wien – 21.\,10.\,1931 ebd.), \emph{Schriftsteller, Mediziner}!Märchen. Schauspiel in drei Aufzügen@\strich\emph{Das Märchen. Schauspiel in drei Aufzügen}|pw}« geſchrieben haben. {\pb}Ich habe von allem zu koſten beko{\geminationm}en, was die Aufführung
               eines Stückes verletzendes bringen kann: wie irgend einer ka{\geminationn} ich mitreden, wenn von
               der Albernheit des Dichters, der Verlegenheit der Komödianten und der vergnügten
               Gefälligkeit der Recenſenten geſprochen wird, – wobei ich vom Publikum gänzlich{ }ſchweigen will, das albern, verlogenen und gefällig iſt. – Es iſt nicht
               anzunehmen, daſs ich anders reden würde, wenn {\pb}ich zufällig
               einen Erfolg gehabt hätte, nur{ }ſetzte ich hinzu: Trotzdem{\dotsfour}{ }\textsc{etc.} – und es klänge großartiger. –\pend
           
\pstart
           – Wenn Sie in Auſſee\oindex{Bad Aussee@\textbf{Bad Aussee}, \emph{Hauptstadt}|pw}{ }ſein werden,{ }ſo hoffe
               ich die Freude zu haben Sie zu{ }ſehen, da ich im Auguſt meine Mama\pwindex{Schnitzler, Louise 8.\,7.\,1840 Kőszeg – 9.\,9.\,1911 Wien@\textsc{Schnitzler, Louise} (8.\,7.\,1840 Kőszeg – 9.\,9.\,1911 Wien)|pwv} in Iſchl\oindex{Bad Ischl@\textbf{Bad Ischl}|pw} beſuchen
               werde. Vielleicht laſſen Sie aber bis dahin noch ein freundliches Wort von{ }ſich
               hören. Haben Sie die Güte mich Ihrer w. Frau Gemahlin\pwindex{Herzl, Julie 1.\,2.\,1868 Budapest – 10.\,11.\,1907 Wien@\textsc{Herzl, Julie} (1.\,2.\,1868 Budapest – 10.\,11.\,1907 Wien)|pwv} beſtens zu empfehlen, und{ }ſeien Sie, mein\pend
           
\pstart
           {\pb}lieber Freund, aufs herzlichſte{\\[\baselineskip]}gegrüßt. Ihr
                  \spacefill\mbox{ArthurSchnitzler}\pend
           \leftskip=0em{}
\pstart
           Wien\oindex{Wien@\textbf{Wien}, \emph{Verwaltungsgebiet}|pw},
                  5. 7. 94\pend
           \selectlanguage{ngerman}\endnumbering\briefempfaengerindex{Herzl, Theodor@\textsc{Herzl, Theodor}!zzzSchnitzler, Arthur@\emph{von Arthur Schnitzler}!1894-07-051@{5. 7. 1894}|)be}\mylabel{L03908h}
\begin{anhang}
\end{anhang}\newcommand{\dateiname}{L03908}\newcommand{\titel}{Arthur Schnitzler an Theodor Herzl, 5. 7. 1894}\newcommand{\editorInnen}{Herausgegeben von Jahnke, SelmaMüller, Martin Anton}%% latex-leseansicht-abspann.tex
%% Abspann für die Leseansicht.
%% Der Schalter \ifkorrekturansicht ist bereits durch den Vorspann gesetzt.

%% latex-abspann.tex
%% Gemeinsamer Abspann für Korrekturansicht und Leseansicht.
%% Setzt den Schalter \ifkorrekturansicht voraus (gesetzt in den
%% einbindenden Dateien latex-korrekturansicht-abspann.tex bzw.
%% latex-leseansicht-abspann.tex).
%% ---------------------------------------------------------------

\normalsize

% Das esempio-Environment wird nur in der Leseansicht benötigt
\ifkorrekturansicht\else
\newenvironment{esempio}[3]%
{
    \vspace{1.5ex}
    \rlap{\underline{#1}}
    \par
    \setlength{\parindent}{0cm}
    \nopagebreak
    \leftskip=#2cm
    \rightskip=#3cm
}
{
    \par
}
\fi

\doendnotes{C}
\bigskip
\vfill

\clearpage

\footnotesize

\ifkorrekturansicht
  \lohead{\textsc{register}}
\fi

% theindex-Environment neu definieren ohne reledmac
\makeatletter
\renewenvironment{theindex}{%
  \ifkorrekturansicht
    \section*{\indexname}%
  \else
    \subsubsection*{Index der erwähnten Entitäten}%
  \fi
  \setlength{\parindent}{0pt}%
  \setlength{\parskip}{0pt plus 0.3pt}%
  \let\item\@idxitem
}{%
  \ifkorrekturansicht\clearpage\fi
}
\makeatother

\IfFileExists{\jobname-pw.ind}{\input{\jobname-pw.ind}}{}

% Quellenangabe nur in der Leseansicht
\ifkorrekturansicht\else
% Fallback-Definitionen, falls die .tex-Datei \titel etc. nicht gesetzt hat
\providecommand{\titel}{}
\providecommand{\editorInnen}{}
\providecommand{\dateiname}{\jobname}

\vspace{3cm}

\vfill

\footnotesize
\textsc{Quelle}: \titel. Herausgegeben von {\editorInnen}. In: \emph{Arthur Schnitzler: Briefwechsel mit Autorinnen und Autoren}.
 Digitale Edition, https://schnitzler-briefe.acdh.oeaw.ac.at/{\dateiname}.html (Stand \today)
\fi

\end{document}


