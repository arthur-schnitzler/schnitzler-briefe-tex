%% latex-korrekturansicht-vorspann.tex
%% Vorspann für die Korrekturansicht.
%% Lädt die gemeinsame Datei latex-vorspann.tex mit gesetztem Schalter.

\newif\ifkorrekturansicht
\korrekturansichttrue

\input{../tex-inputs/latex-vorspann}


\section[Richard Beer-Hofmann an Arthur Schnitzler, 28. 7. 1896]{L00570 Richard Beer-Hofmann an Arthur Schnitzler, 28. 7. 1896}
\nopagebreak\mylabel{L00570v}
\rehead{ }\normalsize\beginnumbering\briefempfaengerindex{Schnitzler, Arthur@\textsc{Schnitzler, Arthur}!zzzBeer-Hofmann, Richard@\emph{von Richard Beer-Hofmann}!1896-07-281@{28. 7. 1896}|(be}
\toendnotes[C]{\smallbreak\pagebreak[2]}\Standort{CUL, Schnitzler, B 8.}
\physDesc{Brief, 1 Blatt, 4 Seiten, 921 Zeichen
\newline{}Handschrift: Bleistift, lateinische Kurrent
\newline{}Schnitzler: mit Bleistift am Beginn des Briefes datiert: »28/7 96« 
\newline{}Ordnung: mit Bleistift von unbekannter Hand nummeriert:
                                    »78« }
\buchAbdrucke{\weitereDrucke{Arthur Schnitzler, Richard Beer-Hofmann: \emph{Briefwechsel 1891–1931}. Wien, Zürich: \emph{Europaverlag} 1992, S. 93–94.} }
\pstart
           \noindent{}{\pb}Lieber Arthur! Es ist
               infam.\pend
           
\pstart
           \uline{Klampenborg}\oindex{Klampenborg@\textbf{Klampenborg}, \emph{P.PPL}|pw} wegen Eleganz ausgeschlossen\pend
           
\pstart
           \uline{Skodsborg}\oindex{Skodsborg@\textbf{Skodsborg}, \emph{P.PPL}|pw} sehr voll und vermutlich geräuschvoll\pend
           
\pstart
           Also \uline{Vedbaek}\oindex{Vedbæk@\textbf{Vedbæk}, \emph{P.PPL}|pw} (10 Minuten weiter als Klampenborg\oindex{Klampenborg@\textbf{Klampenborg}, \emph{P.PPL}|pw}.)\pend
           
\pstart
           das ist bescheiden billig – für \substVorne{}\textsuperscript{g}\substDazwischen{}e\substHinten{}in Zi{\geminationm}er mit 2 Betten und Pension für 2 Personen
               10 Kronen, aber das Zi{\geminationm}er wird erst {\pb}Samstag oder Sonntag frei, und ich bin also noch
               unentschlossen was tun. Ko{\geminationm}en Sie daher lieber \uline{direkt}{ }Kopenhagen\oindex{Kopenhagen@\textbf{Kopenhagen}, \emph{P.PPLC}|pw} und entweder bin ich noch dort und
               wir berathen gemeinsam, oder ich bin schon wo und ko{\geminationm}e
               Sie abholen nach Kopenhagen\oindex{Kopenhagen@\textbf{Kopenhagen}, \emph{P.PPLC}|pw}. –\pend
           
\pstart
           {\pb}Vedbaek\oindex{Vedbæk@\textbf{Vedbæk}, \emph{P.PPL}|pw}, das weiteste, ist von Kopenhagen\oindex{Kopenhagen@\textbf{Kopenhagen}, \emph{P.PPLC}|pw} 1 Stunde 10 Minuten mit dem Schiff.
               Wo treffen Sie mit Paul\pwindex{Goldmann, Paul 31.01.1865 – 25.09.1935@\textsc{Goldmann, Paul} (31.01.1865 – 25.09.1935), \emph{Schriftsteller/Schriftstellerin, Journalist/Journalistin}|pw} zusa{\geminationm}en\pend
           
\pstart
           Wann ko{\geminationm}en Sie (\uline{genau})\pend
           
\pstart
           Brandes\pwindex{Brandes, Georg 04.02.1842 – 19.02.1927@\textsc{Brandes, Georg} (04.02.1842 – 19.02.1927)|pw} ko{\geminationm}t
               morgen vom Land und fährt übermorgen weg, ich hoffe ihn zu sprechen. Vielleicht ist
               schon Brief von Ihnen da. {\pb}Ich war
               nämlich gestern nicht bei der Post, und gehe erst jetzt hin. Herrlich sind nur die
               Bäder hier. \uline{König von Dänemark}\oindex{Hotel Kongen af Danmark@\textbf{Hotel Kongen af Danmark}, \emph{Hotel (K.HTL)}|pw} wohne ich.\pend
           
\pstart
           Herzlichst{\\[\baselineskip]}Ihr{\\[\baselineskip]}\spacefill\mbox{Richard}\pend
           \leftskip=0em{}
\pstart
           28/VII 96{ }Kopenhagen\oindex{Kopenhagen@\textbf{Kopenhagen}, \emph{P.PPLC}|pw}\pend
           \selectlanguage{ngerman}\endnumbering\briefempfaengerindex{Schnitzler, Arthur@\textsc{Schnitzler, Arthur}!zzzBeer-Hofmann, Richard@\emph{von Richard Beer-Hofmann}!1896-07-281@{28. 7. 1896}|)be}\mylabel{L00570h}  \normalsize

\doendnotes{C}
\bigskip
\vfill

\clearpage

\footnotesize

\lohead{\textsc{register}}

% Definiere theindex-Environment komplett neu ohne reledmac
\makeatletter
\renewenvironment{theindex}{%
  \section*{\indexname}%
  \setlength{\parindent}{0pt}%
  \setlength{\parskip}{0pt plus 0.3pt}%
  \let\item\@idxitem
}{%
  \clearpage
}
\makeatother

\IfFileExists{\jobname-pw.ind}{\input{\jobname-pw.ind}}{}

\end{document}

      