%% latex-leseansicht-vorspann.tex
%% Vorspann für die Leseansicht.
%% Lädt die gemeinsame Datei latex-vorspann.tex mit nicht gesetztem Schalter.

\newif\ifkorrekturansicht
\korrekturansichtfalse

\input{../tex-inputs/latex-vorspann}


         
         \renewcommand{\erwaehntePersonen}{Personen: Richard Beer-Hofmann, Georg Brandes, Paul Goldmann}
         \renewcommand{\erwaehnteOrte}{Orte: Hotel König von Dänemark, Klampenborg, Kopenhagen, Skodsborg, Vedbæk}
         \renewcommand{\erwaehnteWerke}{}
               \section[Richard Beer-Hofmann an Arthur Schnitzler, 28. 7. 1896]{ Richard Beer-Hofmann an Arthur Schnitzler, 28. 7. 1896}\nopagebreak\mylabel{v}\rehead{ }\begin{ledgroupsized}[t]{13cm}\normalsize\beginnumbering\briefempfaengerindex{Schnitzler, Arthur@\textsc{Schnitzler, Arthur}!zzzBeer-Hofmann, Richard@\emph{von Richard Beer-Hofmann}!1896-07-281@{28. 7. 1896}|(be} \toendnotes[C]{\smallbreak\pagebreak[2]} \Standort{CUL, Schnitzler, B 8.}
\physDesc{Brief, 1 Blatt, 4 Seiten, 921 Zeichen
\newline{}Handschrift: Bleistift, lateinische Kurrent
\newline{}Schnitzler: mit Bleistift am Beginn des Briefes datiert: »28/7 96« 
\newline{}Ordnung: mit Bleistift von unbekannter Hand nummeriert:
                                    »78« }\buchAbdrucke{\weitereDrucke{Arthur Schnitzler, Richard Beer-Hofmann: \emph{Briefwechsel 1891–1931}. Hg. Konstanze Fliedl. Wien, Zürich: \emph{Europaverlag} 1992, S. 93–94.} }\pstart
           \noindent{}{\pb}Lieber Arthur! Es ist
               infam.\pend
           \pstart
           \uline{Klampenborg}\oindex{Klampenborg@\textbf{Klampenborg}|pw} wegen Eleganz ausgeschlossen\pend
           \pstart
           \uline{Skodsborg}\oindex{Skodsborg@\textbf{Skodsborg}|pw} sehr voll und vermutlich geräuschvoll\pend
           \pstart
           Also \uline{Vedbaek}\oindex{Vedbæk@\textbf{Vedbæk}|pw} (10 Minuten weiter als Klampenborg\oindex{Klampenborg@\textbf{Klampenborg}|pw}.)\pend
           \pstart
           das ist bescheiden billig – für \substVorne{}\textsuperscript{g}\substDazwischen{}e\substHinten{}in Zi{\geminationm}er mit 2 Betten und Pension für 2 Personen
               10 Kronen, aber das Zi{\geminationm}er wird erst {\pb}Samstag oder Sonntag frei, und ich bin also noch
               unentschlossen was tun. Ko{\geminationm}en Sie daher lieber \uline{direkt}{ }Kopenhagen\oindex{Kopenhagen@\textbf{Kopenhagen}|pw} und entweder bin ich noch dort und
               wir berathen gemeinsam, oder ich bin schon wo und ko{\geminationm}e
               Sie abholen nach Kopenhagen\oindex{Kopenhagen@\textbf{Kopenhagen}|pw}. –\pend
           \pstart
           {\pb}Vedbaek\oindex{Vedbæk@\textbf{Vedbæk}|pw}, das weiteste, ist von Kopenhagen\oindex{Kopenhagen@\textbf{Kopenhagen}|pw} 1 Stunde 10 Minuten mit dem Schiff.
               Wo treffen Sie mit Paul\pwindex{Goldmann, Paul 31.01.1865 – 25.09.1935@\textsc{Goldmann, Paul} (31.01.1865 – 25.09.1935), \emph{Schriftsteller, Journalist}|pw} zusa{\geminationm}en\pend
           \pstart
           Wann ko{\geminationm}en Sie (\uline{genau})\pend
           \pstart
           Brandes\pwindex{Brandes, Georg 04.02.1842 – 19.02.1927@\textsc{Brandes, Georg} (04.02.1842 – 19.02.1927)|pw} ko{\geminationm}t
               morgen vom Land und fährt übermorgen weg, ich hoffe ihn zu sprechen. Vielleicht ist
               schon Brief von Ihnen da. {\pb}Ich war
               nämlich gestern nicht bei der Post, und gehe erst jetzt hin. Herrlich sind nur die
               Bäder hier. \uline{König von Dänemark}\oindex{Hotel Koenig von Daenemark@\textbf{Hotel König von Dänemark}|pw} wohne ich.\pend
           \pstart
           Herzlichst{\\[\baselineskip]}Ihr{\\[\baselineskip]}\spacefill\mbox{Richard}\pend
           \leftskip=0em{}\pstart
           28/VII 96{ }Kopenhagen\oindex{Kopenhagen@\textbf{Kopenhagen}|pw}\pend
           
         
         \endnumbering\mylabel{h}\end{ledgroupsized}  \newcommand{\dateiname}{L00570}\newcommand{\titel}{Richard Beer-Hofmann an Arthur Schnitzler, 28. 7. 1896}\newcommand{\editorInnen}{Martin Anton Müller und Gerd-Hermann Susen}%% latex-leseansicht-abspann.tex
%% Abspann für die Leseansicht.
%% Der Schalter \ifkorrekturansicht ist bereits durch den Vorspann gesetzt.

%% latex-abspann.tex
%% Gemeinsamer Abspann für Korrekturansicht und Leseansicht.
%% Setzt den Schalter \ifkorrekturansicht voraus (gesetzt in den
%% einbindenden Dateien latex-korrekturansicht-abspann.tex bzw.
%% latex-leseansicht-abspann.tex).
%% ---------------------------------------------------------------

\normalsize

% Das esempio-Environment wird nur in der Leseansicht benötigt
\ifkorrekturansicht\else
\newenvironment{esempio}[3]%
{
    \vspace{1.5ex}
    \rlap{\underline{#1}}
    \par
    \setlength{\parindent}{0cm}
    \nopagebreak
    \leftskip=#2cm
    \rightskip=#3cm
}
{
    \par
}
\fi

\doendnotes{C}
\bigskip
\vfill

\clearpage

\footnotesize

\ifkorrekturansicht
  \lohead{\textsc{register}}
\fi

% theindex-Environment neu definieren ohne reledmac
\makeatletter
\renewenvironment{theindex}{%
  \ifkorrekturansicht
    \section*{\indexname}%
  \else
    \subsubsection*{Index der erwähnten Entitäten}%
  \fi
  \setlength{\parindent}{0pt}%
  \setlength{\parskip}{0pt plus 0.3pt}%
  \let\item\@idxitem
}{%
  \ifkorrekturansicht\clearpage\fi
}
\makeatother

\IfFileExists{\jobname-pw.ind}{\input{\jobname-pw.ind}}{}

% Quellenangabe nur in der Leseansicht
\ifkorrekturansicht\else
% Fallback-Definitionen, falls die .tex-Datei \titel etc. nicht gesetzt hat
\providecommand{\titel}{}
\providecommand{\editorInnen}{}
\providecommand{\dateiname}{\jobname}

\vspace{3cm}

\vfill

\footnotesize
\textsc{Quelle}: \titel. Herausgegeben von {\editorInnen}. In: \emph{Arthur Schnitzler: Briefwechsel mit Autorinnen und Autoren}.
 Digitale Edition, https://schnitzler-briefe.acdh.oeaw.ac.at/{\dateiname}.html (Stand \today)
\fi

\end{document}


      