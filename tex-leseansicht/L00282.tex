%% latex-korrekturansicht-vorspann.tex
%% Vorspann für die Korrekturansicht.
%% Lädt die gemeinsame Datei latex-vorspann.tex mit gesetztem Schalter.

\newif\ifkorrekturansicht
\korrekturansichttrue

\input{../tex-inputs/latex-vorspann}


\section[Arthur Schnitzler an Wilhelm Bölsche, {[}12.? 11. 1893{]}]{L00282 Arthur Schnitzler an Wilhelm Bölsche, {[}12.? 11. 1893{]}}
\nopagebreak\mylabel{L00282v}
\rehead{ }\normalsize\beginnumbering\briefempfaengerindex{Boelsche, Wilhelm@\textsc{Bölsche, Wilhelm}!zzzSchnitzler, Arthur@\emph{von Arthur Schnitzler}!1893-11-121@{{[}12.? 11. 1893{]}}|(be}
\toendnotes[C]{\smallbreak\pagebreak[2]}\Standort{Wrocław, Biblioteka Uniwersytecka, Böl.Pis 1771.}
\physDesc{Brief, 1 Blatt, 2 Seiten, 535 Zeichen (Briefpapier mit Trauerrand)
\newline{}Handschrift: schwarze Tinte, deutsche Kurrent}
\buchAbdrucke{\weitereDrucke{1) \emph{Germanica Wratislaviensia} (1987) Nr. 77, S. 465.} \weitereDrucke{2) Wilhelm Bölsche: \emph{Briefwechsel. Mit Autoren der Freien Bühne}. Berlin: \emph{Weidler} 2010, S. 694.} }\toendnotes[C]{\smallbreak}
\pstart
           \raggedleft{}{\pb}\textsc{IX. \label{K_L00282-1v}\edtext{Frankgasse 1}{\lemma{\textnormal{\emph{Frankgasse 1}}}\Cendnote{\textnormal{Die
                           Übersiedlung in sein neues Zuhause fand am 14. 11. 1893
                           statt. Die Antwort Bölsches\pwindex{Boelsche, Wilhelm 02.01.1861 – 31.08.1939@\textsc{Bölsche, Wilhelm} (02.01.1861 – 31.08.1939), \emph{Schriftsteller/Schriftstellerin, Publizist/Publizistin}|pwk}, der
                           den Brief aus Friedrichshagen\oindex{Friedrichshagen@\textbf{Friedrichshagen}, \emph{P.PPLX}|pwk} nach
                              Zürich\oindex{Zuerich@\textbf{Zürich}, \emph{P.PPLA}|pwk} nachgesandt bekam, stammt
                           vom 16. 11. 1893. Aufgrund der Verzögerung durch die Post
                           ist der 12. 11. 1893 als Absendetag plausibel.}}}\label{K_L00282-1}. }\oindex{Frankgasse 1@\textbf{Frankgasse 1}, \emph{Wohngebäude (K.WHS)}|pw}\pend
           
\pstart{}Sehr geehrter Herr Doktor,\pend\vspace{0.5em}
\pstart
           ich habe Das Märchen\pwindex{Maerchen. Schauspiel in drei Aufzuegen@\emph{Das Märchen. Schauspiel in drei Aufzügen}|pw} vor \label{K_L00282-2v}\edtext{etwa 3 Monaten}{\lemma{\textnormal{\emph{etwa 3 Monaten}}}\Cendnote{\textnormal{Am 25. 7. 1893, siehe Arthur Schnitzler an Samuel Fischer, 25. 7. 1893.
               }}}\label{K_L00282-2} Ihrer Aufforderung nach an den Verleger \textsc{Hrn Fischer\pwindex{Fischer, Samuel 24.12.1859 – 15.10.1934@\textsc{Fischer, Samuel} (24.12.1859 – 15.10.1934), \emph{Verleger/Verlegerin}|pw}} geſandt. Seither habe ich 3mal verſucht, von dieſem Herrn eine Antwort zu
               erhalten – leider vergebens.\pend
           
\pstart
           Ich muſs mich doch weiter an den Redakteur\orgindex{Neue Rundschau, Neue Deutsche Rundschau, Freie Buehne@Neue Rundschau, Neue Deutsche Rundschau, Freie Bühne|pwv} wenden, {\pb}und erſuche Sie, die
               Beantwortung meiner Fragen oder die Rückſendung meines Manuscripts umſo ſchleuniger
               veranlaſſen zu wollen, als die Aufführung des Stückes \label{K_L00282-3v}\edtext{in etwa 14 Tagen}{\lemma{\textnormal{\emph{in etwa 14 Tagen}}}\Cendnote{\textnormal{am 1. 12. 1893}}}\label{K_L00282-3} im Dtſch. Volkstheater\oindex{Volkstheater@\textbf{Volkstheater}, \emph{Theater (K.THE)}|pw}{ }ſtattfindet.\pend
           
\pstart
           Mit ausgezeichneter Hochachtung{\\[\baselineskip]}\spacefill\mbox{Dr Arthur Schnitzler}\pend
           \leftskip=0em{}\selectlanguage{ngerman}\endnumbering\briefempfaengerindex{Boelsche, Wilhelm@\textsc{Bölsche, Wilhelm}!zzzSchnitzler, Arthur@\emph{von Arthur Schnitzler}!1893-11-121@{{[}12.? 11. 1893{]}}|)be}\mylabel{L00282h}  \normalsize

\doendnotes{C}
\bigskip
\vfill

\clearpage

\footnotesize

\lohead{\textsc{register}}

% Definiere theindex-Environment komplett neu ohne reledmac
\makeatletter
\renewenvironment{theindex}{%
  \section*{\indexname}%
  \setlength{\parindent}{0pt}%
  \setlength{\parskip}{0pt plus 0.3pt}%
  \let\item\@idxitem
}{%
  \clearpage
}
\makeatother

\IfFileExists{\jobname-pw.ind}{\input{\jobname-pw.ind}}{}

\end{document}

      