%% latex-leseansicht-vorspann.tex
%% Vorspann für die Leseansicht.
%% Lädt die gemeinsame Datei latex-vorspann.tex mit nicht gesetztem Schalter.

\newif\ifkorrekturansicht
\korrekturansichtfalse

\input{../tex-inputs/latex-vorspann}


               \section[Arthur Schnitzler an Wilhelm Bölsche, {[}12.? 11. 1893{]}]{ Arthur Schnitzler an Wilhelm Bölsche, {[}12.? 11. 1893{]}}\nopagebreak\mylabel{v}\rehead{ }\begin{ledgroupsized}[t]{13cm}\normalsize\beginnumbering\briefempfaengerindex{Boelsche, Wilhelm@\textsc{Bölsche, Wilhelm}!zzzSchnitzler, Arthur@\emph{von Arthur Schnitzler}!1893-11-141@{{[}12.? 11. 1893{]}}|(be} \toendnotes[C]{\smallbreak\pagebreak[2]} \Standort{Wrocław, Biblioteka Uniwersytecka, Böl.Pis 1771.}
\physDesc{Brief, 1 Blatt (Briefpapier mit Trauerrand), 2 Seiten
\newline{}Handschrift: schwarze Tinte, deutsche Kurrent}\buchAbdrucke{\weitereDrucke{1) Alois Woldan: \emph{Arthur Schnitzler – Briefe an Wilhelm Bölsche.} In: \emph{Germanica Wratislaviensia} (1987) Nr. 77, S. 465.} \weitereDrucke{2) Wilhelm Bölsche: \emph{Briefwechsel. Mit Autoren der Freien Bühne}. Hg. Gerd-Hermann Susen. Berlin: \emph{Weidler} 2010, S. 694 (Werke und Briefe. Wissenschaftliche Ausgabe, Briefe I).} }\toendnotes[C]{\smallbreak}\pstart
           \raggedleft{}{\pb}\textsc{IX. \label{K_L00282_1v}\edtext{Frankgasse}{\lemma{\textnormal{\emph{Frankgasse}}}\Cendnote{\textnormal{Die Übersiedlung
                           in sein neues Zuhause fand am 14. 11. 1893 statt. Die Antwort Bölsches\pwindex{Boelsche, Wilhelm 02.01.1861 – 31.08.1939@\textsc{Bölsche, Wilhelm} (02.01.1861 – 31.08.1939), \emph{Schriftsteller, Publizist}|pwk}, der den Brief aus Friedrichshagen\oindex{Friedrichshagen@\textbf{Friedrichshagen}|pwk} nach Zürich\oindex{Zuerich@\textbf{Zürich}|pwk} nachgesandt bekam, stammt vom
                              16. 11. 1893. Aufgrund der Verzögerung durch
                           die Post ist der 12. 11. 1893 als Absendetag
                           plausibel.}}}\label{K_L00282_1h}}\oindex{Frankgasse@\textbf{Frankgasse}|pw}\pend
           \pstart{}Sehr geehrter Herr Doktor,\pend\pstart
           ich habe das Das Märchen\pwindex{Schnitzler, Arthur 15.05.1862 – 21.10.1931@\textsc{Schnitzler, Arthur} (15.05.1862 – 21.10.1931), \emph{Schriftsteller, Mediziner}!Maerchen. Schauspiel in drei Aufzuegen1891 – 1891@\strich\emph{Das Märchen. Schauspiel in drei Aufzügen} {[}1891 – 1891{]}|pw} vor \label{K_L00282_2v}\edtext{etwa 3 Monaten}{\lemma{\textnormal{\emph{etwa 3 Monaten}}}\Cendnote{\textnormal{am 25. 7. 1893, Arthur Schnitzler an Samuel Fischer, 25. 7. 1893}}}\label{K_L00282_2h} Ihrer Aufforderung nach an den Verleger \textsc{Hrn Fischer\pwindex{Fischer, Samuel 24.12.1859 – 15.10.1934@\textsc{Fischer, Samuel} (24.12.1859 – 15.10.1934), \emph{Verleger}|pw}} geſandt. Seither habe ich 3mal verſucht, von dieſem Herrn eine Antwort zu
               erhalten – leider vergebens.\pend
           \pstart
           Ich muſs mich doch weiter an den Redakteur\orgindex{Neue Rundschau, Neue Deutsche Rundschau, Freie Buehne@Neue Rundschau, Neue Deutsche Rundschau, Freie Bühne|pwv} wenden, {\pb}und erſuche Sie, die
               Beantwortung meiner Fragen oder die Rückſendung meines Manuscripts umſo ſchleuniger
               veranlaſſen zu wollen, als die Aufführung des Stückes \label{K_L00282_3v}\edtext{in etwa 14 Tagen}{\lemma{\textnormal{\emph{in etwa 14 Tagen}}}\Cendnote{\textnormal{am
                     1. 12. 1893}}}\label{K_L00282_3h} im
                  Dtſch. Volkstheater\oindex{Volkstheater@\textbf{Volkstheater}|pw}{ }ſtattfindet.\pend
           \pstart
           Mit ausgezeichneter Hochachtung{\\[\baselineskip]}\spacefill\mbox{Dr Arthur Schnitzler}\pend
           \leftskip=0em{}          \endnumbering\briefempfaengerindex{Boelsche, Wilhelm@\textsc{Bölsche, Wilhelm}!zzzSchnitzler, Arthur@\emph{von Arthur Schnitzler}!1893-11-141@{{[}12.? 11. 1893{]}}|)be}\mylabel{h}\end{ledgroupsized}  \newcommand{\dateiname}{L00282}\newcommand{\titel}{Arthur Schnitzler an Wilhelm Bölsche, [12.? 11. 1893]}\newcommand{\editorInnen}{Martin Anton Müller und Gerd-Hermann Susen}
            \footnotesize
\begin{ledgroupsized}[t]{11.5cm}
\doendnotes{C}
\end{ledgroupsized}
         %% latex-leseansicht-abspann.tex
%% Abspann für die Leseansicht.
%% Der Schalter \ifkorrekturansicht ist bereits durch den Vorspann gesetzt.

%% latex-abspann.tex
%% Gemeinsamer Abspann für Korrekturansicht und Leseansicht.
%% Setzt den Schalter \ifkorrekturansicht voraus (gesetzt in den
%% einbindenden Dateien latex-korrekturansicht-abspann.tex bzw.
%% latex-leseansicht-abspann.tex).
%% ---------------------------------------------------------------

\normalsize

% Das esempio-Environment wird nur in der Leseansicht benötigt
\ifkorrekturansicht\else
\newenvironment{esempio}[3]%
{
    \vspace{1.5ex}
    \rlap{\underline{#1}}
    \par
    \setlength{\parindent}{0cm}
    \nopagebreak
    \leftskip=#2cm
    \rightskip=#3cm
}
{
    \par
}
\fi

\doendnotes{C}
\bigskip
\vfill

\clearpage

\footnotesize

\ifkorrekturansicht
  \lohead{\textsc{register}}
\fi

% theindex-Environment neu definieren ohne reledmac
\makeatletter
\renewenvironment{theindex}{%
  \ifkorrekturansicht
    \section*{\indexname}%
  \else
    \subsubsection*{Index der erwähnten Entitäten}%
  \fi
  \setlength{\parindent}{0pt}%
  \setlength{\parskip}{0pt plus 0.3pt}%
  \let\item\@idxitem
}{%
  \ifkorrekturansicht\clearpage\fi
}
\makeatother

\IfFileExists{\jobname-pw.ind}{\input{\jobname-pw.ind}}{}

% Quellenangabe nur in der Leseansicht
\ifkorrekturansicht\else
% Fallback-Definitionen, falls die .tex-Datei \titel etc. nicht gesetzt hat
\providecommand{\titel}{}
\providecommand{\editorInnen}{}
\providecommand{\dateiname}{\jobname}

\vspace{3cm}

\vfill

\footnotesize
\textsc{Quelle}: \titel. Herausgegeben von {\editorInnen}. In: \emph{Arthur Schnitzler: Briefwechsel mit Autorinnen und Autoren}.
 Digitale Edition, https://schnitzler-briefe.acdh.oeaw.ac.at/{\dateiname}.html (Stand \today)
\fi

\end{document}


      