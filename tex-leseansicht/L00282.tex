%% latex-leseansicht-vorspann.tex
%% Vorspann für die Leseansicht.
%% Lädt die gemeinsame Datei latex-vorspann.tex mit nicht gesetztem Schalter.

\newif\ifkorrekturansicht
\korrekturansichtfalse

\input{../tex-inputs/latex-vorspann}


\section[Arthur Schnitzler an Wilhelm Bölsche, {{[}}12.? 11. 1893{{]}}]{L00282 Arthur Schnitzler an Wilhelm Bölsche, {[}12.? 11. 1893{]}}
\nopagebreak\mylabel{L00282v}
\rehead{ }\normalsize\beginnumbering\briefempfaengerindex{Bölsche, Wilhelm@\textsc{Bölsche, Wilhelm}!zzzSchnitzler, Arthur@\emph{von Arthur Schnitzler}!1893-11-121@{{[}12.? 11. 1893{]}}|(be}
\toendnotes[C]{\smallbreak\pagebreak[2]}
\correspDesc{Versand  durch Arthur Schnitzler am [12.? 11. 1893] in Wien
\newline{}Erhalt  durch Wilhelm Bölsche im Zeitraum [13. 11. 1893 – 16. 11. 1893?] in Berlin}\toendnotes[C]{\smallbreak}
\Standort{Wrocław, Biblioteka Uniwersytecka, Böl.Pis 1771.}
\physDesc{Brief, 1 Blatt, 2 Seiten, 535 Zeichen (Briefpapier mit Trauerrand)
\newline{}Handschrift: schwarze Tinte, deutsche Kurrent}
\buchAbdrucke{\weitereDrucke{1) Alois Woldan: \emph{Arthur Schnitzler – Briefe an Wilhelm Bölsche.} In: \emph{Germanica Wratislaviensia} (1987) Nr. 77, S. 465.} \weitereDrucke{2) Wilhelm Bölsche: \emph{Briefwechsel. Mit Autoren der Freien Bühne}. Herausgegeben von Gerd-Hermann Susen. Berlin: \emph{Weidler} 2010, S. 694 (Werke und Briefe. Wissenschaftliche Ausgabe, Briefe I).} }\toendnotes[C]{\smallbreak}
\pstart
           \raggedleft{}{\pb}\textsc{IX. \label{K_L00282-1v}\edtext{Frankgasse 1}{\lemma{\textnormal{\emph{Frankgasse 1}}}\Cendnote{\textnormal{Die
                           Übersiedlung in sein neues Zuhause fand am 14. 11. 1893
                           statt. Die Antwort Bölsches\pwindex{Bölsche, Wilhelm 2.\,1.\,1861 Köln – 31.\,8.\,1939 Szklarska Poręba@\textsc{Bölsche, Wilhelm} (2.\,1.\,1861 Köln – 31.\,8.\,1939 Szklarska Poręba), \emph{Schriftsteller, Publizist}|pwk}, der
                           den Brief aus Friedrichshagen\oindex{Friedrichshagen@\textbf{Friedrichshagen}, \emph{Ehemaliger Ort}|pwk} nach
                              Zürich\oindex{Zürich@\textbf{Zürich}|pwk} nachgesandt bekam, stammt
                           vom 16. 11. 1893. Aufgrund der Verzögerung durch die Post
                           ist der 12. 11. 1893 als Absendetag plausibel.}}}\label{K_L00282-1}.}\oindex{Wien@\textbf{Wien}!IX., Alsergrund@\textbf{IX., Alsergrund}!Frankgasse 1@\textbf{Frankgasse 1}, \emph{Wohngebäude}|pw}\pend
           
\pstart{}Sehr geehrter Herr Doktor,\pend\vspace{0.5em}
\pstart
           ich habe Das Märchen\pwindex{Schnitzler, Arthur 15.\,5.\,1862 Wien – 21.\,10.\,1931 ebd.@\textsc{Schnitzler, Arthur} (15.\,5.\,1862 Wien – 21.\,10.\,1931 ebd.), \emph{Schriftsteller, Mediziner}!Märchen. Schauspiel in drei Aufzügen@\strich\emph{Das Märchen. Schauspiel in drei Aufzügen}|pw} vor \label{K_L00282-2v}\edtext{etwa 3 Monaten}{\lemma{\textnormal{\emph{etwa 3 Monaten}}}\Cendnote{\textnormal{Am 25. 7. 1893, siehe XXXX Auszeichnungsfehler: Dokument L00242 nicht gefunden.
               }}}\label{K_L00282-2} Ihrer Aufforderung nach an den Verleger \textsc{Hrn Fischer\pwindex{Fischer, Samuel 24.\,12.\,1859 Liptovský Mikuláš – 15.\,10.\,1934 Berlin@\textsc{Fischer, Samuel} (24.\,12.\,1859 Liptovský Mikuláš – 15.\,10.\,1934 Berlin), \emph{Verleger}|pw}} geſandt. Seither habe ich 3mal verſucht, von dieſem Herrn eine Antwort zu
               erhalten – leider vergebens.\pend
           
\pstart
           Ich muſs mich doch weiter an den Redakteur\orgindex{Neue Rundschau, Neue Deutsche Rundschau, Freie Bühne@Neue Rundschau, Neue Deutsche Rundschau, Freie Bühne|pwv} wenden, {\pb}und erſuche Sie, die
               Beantwortung meiner Fragen oder die Rückſendung meines Manuscripts umſo{ }ſchleuniger
               veranlaſſen zu wollen, als die Aufführung des Stückes \label{K_L00282-3v}\edtext{in etwa 14 Tagen}{\lemma{\textnormal{\emph{in etwa 14 Tagen}}}\Cendnote{\textnormal{am 1. 12. 1893}}}\label{K_L00282-3} im Dtſch. Volkstheater\oindex{Wien@\textbf{Wien}!VII., Neubau@\textbf{VII., Neubau}!Volkstheater@\textbf{Volkstheater}, \emph{Theater}|pw}{ }ſtattfindet.\pend
           
\pstart
           Mit ausgezeichneter Hochachtung{\\[\baselineskip]}\spacefill\mbox{Dr Arthur Schnitzler}\pend
           \leftskip=0em{}\selectlanguage{ngerman}\endnumbering\briefempfaengerindex{Bölsche, Wilhelm@\textsc{Bölsche, Wilhelm}!zzzSchnitzler, Arthur@\emph{von Arthur Schnitzler}!1893-11-121@{{[}12.? 11. 1893{]}}|)be}\mylabel{L00282h}  \newcommand{\dateiname}{L00282}\newcommand{\titel}{Arthur Schnitzler an Wilhelm Bölsche, [12.? 11. 1893]}\newcommand{\editorInnen}{Martin Anton Müller und Gerd-Hermann Susen}%% latex-leseansicht-abspann.tex
%% Abspann für die Leseansicht.
%% Der Schalter \ifkorrekturansicht ist bereits durch den Vorspann gesetzt.

%% latex-abspann.tex
%% Gemeinsamer Abspann für Korrekturansicht und Leseansicht.
%% Setzt den Schalter \ifkorrekturansicht voraus (gesetzt in den
%% einbindenden Dateien latex-korrekturansicht-abspann.tex bzw.
%% latex-leseansicht-abspann.tex).
%% ---------------------------------------------------------------

\normalsize

% Das esempio-Environment wird nur in der Leseansicht benötigt
\ifkorrekturansicht\else
\newenvironment{esempio}[3]%
{
    \vspace{1.5ex}
    \rlap{\underline{#1}}
    \par
    \setlength{\parindent}{0cm}
    \nopagebreak
    \leftskip=#2cm
    \rightskip=#3cm
}
{
    \par
}
\fi

\doendnotes{C}
\bigskip
\vfill

\clearpage

\footnotesize

\ifkorrekturansicht
  \lohead{\textsc{register}}
\fi

% theindex-Environment neu definieren ohne reledmac
\makeatletter
\renewenvironment{theindex}{%
  \ifkorrekturansicht
    \section*{\indexname}%
  \else
    \subsubsection*{Index der erwähnten Entitäten}%
  \fi
  \setlength{\parindent}{0pt}%
  \setlength{\parskip}{0pt plus 0.3pt}%
  \let\item\@idxitem
}{%
  \ifkorrekturansicht\clearpage\fi
}
\makeatother

\IfFileExists{\jobname-pw.ind}{\input{\jobname-pw.ind}}{}

% Quellenangabe nur in der Leseansicht
\ifkorrekturansicht\else
% Fallback-Definitionen, falls die .tex-Datei \titel etc. nicht gesetzt hat
\providecommand{\titel}{}
\providecommand{\editorInnen}{}
\providecommand{\dateiname}{\jobname}

\vspace{3cm}

\vfill

\footnotesize
\textsc{Quelle}: \titel. Herausgegeben von {\editorInnen}. In: \emph{Arthur Schnitzler: Briefwechsel mit Autorinnen und Autoren}.
 Digitale Edition, https://schnitzler-briefe.acdh.oeaw.ac.at/{\dateiname}.html (Stand \today)
\fi

\end{document}


