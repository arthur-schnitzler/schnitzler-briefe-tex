%% latex-korrekturansicht-vorspann.tex
%% Vorspann für die Korrekturansicht.
%% Lädt die gemeinsame Datei latex-vorspann.tex mit gesetztem Schalter.

\newif\ifkorrekturansicht
\korrekturansichttrue

\input{../tex-inputs/latex-vorspann}


\section[Stefan Zweig an Arthur Schnitzler, 26. 4. 1916]{L03658 Stefan Zweig an Arthur Schnitzler, 26. 4. 1916}
\nopagebreak\mylabel{L03658v}
\rehead{ }\normalsize\beginnumbering\briefempfaengerindex{Schnitzler, Arthur@\textsc{Schnitzler, Arthur}!zzzZweig, Stefan@\emph{von Stefan Zweig}!1916-04-261@{26. 4. 1916}|(be}
\toendnotes[C]{\smallbreak\pagebreak[2]}\Standort{CUL, Schnitzler, B 118.}
\physDesc{Briefkarte, 1 Blatt, 2 Seiten, 431 Zeichen
\newline{}Handschrift: lila Tinte, lateinische Kurrent}
\buchAbdrucke{\weitereDrucke{Stefan Zweig: \emph{Briefwechsel mit Hermann Bahr, Sigmund Freud, Rainer Maria
                        Rilke und Arthur Schnitzler}. Frankfurt am Main: \emph{S. Fischer} 1987, S. 399.} }\toendnotes[C]{\smallbreak}
\pstart
           {\pb}\textcolor{gray}{\textbf{SZ}}\hfill 26. IV 1916\pend
           
\pstart
           \raggedleft{}\textcolor{gray}{\textbf{VIII. KOCHGASSE 8\oindex{Kochgasse 8@\textbf{Kochgasse 8}, \emph{Wohngebäude (K.WHS)}|pw}.}}\pend
           \vspace{0.5em}
\pstart
           Lieber verehrter Herr Doktor, ich wollte es Ihnen seit langem sagen,
               dass Sie es nicht falsch verstehen mögen, wenn ich mich gar nicht bei Ihnen zeigte
               und anfragte – ich habe mich in die \label{K_L03658-1v}\edtext{Nähe
                  Wiens\oindex{Haselbrunnerstrasse 12@\textbf{Haselbrunnerstraße 12}, \emph{Wohngebäude (K.WHS)}|pwv}}{\lemma{\textnormal{\emph{Nähe
                  Wiens}}}\Cendnote{\textnormal{Zweig\pwindex{Zweig, Stefan 28.11.1881 – 23.02.1942@\textsc{Zweig, Stefan} (28.11.1881 – 23.02.1942), \emph{Schriftsteller/Schriftstellerin}|pwk} hatte sich zwei Pavillons\oindex{Haselbrunnerstrasse 12@\textbf{Haselbrunnerstraße 12}, \emph{Wohngebäude (K.WHS)}|pwkv} 
                      in Kalksburg\oindex{Kalksburg@\textbf{Kalksburg}, \emph{A.ADM4}|pwk} gemietet, die er mit seiner Frau\pwindex{Zweig, Friderike Maria 1882-12-04 – 1971-01-18@\textsc{Zweig, Friderike Maria} (1882-12-04 – 1971-01-18), \emph{Schriftsteller/Schriftstellerin}|pwkv} in der warmen
                  Jahreszeit bewohnte.}}}\label{K_L03658-1} zurückgezogen, um von meinem zerstückelten Leben den armen Rest für
               Arbeit nutzen zu können. Umsomehr freue ich mich, \label{K_L03658-2v}\edtext{Ihre liebe Frau\pwindex{Schnitzler, Olga 17.01.1882 – 13.01.1970@\textsc{Schnitzler, Olga} (17.01.1882 – 13.01.1970), \emph{Schauspieler/Schauspielerin, Sänger/Sängerin}|pwv}}{\lemma{\textnormal{\emph{Ihre liebe Frau}}}\Cendnote{\textnormal{Am 29. 4. 1916 sang Olga\pwindex{Zweig, Friderike Maria 1882-12-04 – 1971-01-18@\textsc{Zweig, Friderike Maria} (1882-12-04 – 1971-01-18), \emph{Schriftsteller/Schriftstellerin}|pwk} ein Wohltätigkeitskonzert
                     in einem Hörsaal der Allgemeinen Poliklinik.\oindex{Allgemeine Poliklinik [neues Gebaeude]@\textbf{Allgemeine Poliklinik [neues Gebäude]}, \emph{Krankenhaus (K.KKH)}|pwk}}}}\label{K_L03658-2}{ }Samstag zu hören und hoffentlich {\pb}Sie auch sehen zu dürfen. In Verehrung
               getreu Ihr\pend
           \pstart \spacefill\mbox{Stefan Zweig}\pend{}\selectlanguage{ngerman}\endnumbering\briefempfaengerindex{Schnitzler, Arthur@\textsc{Schnitzler, Arthur}!zzzZweig, Stefan@\emph{von Stefan Zweig}!1916-04-261@{26. 4. 1916}|)be}\mylabel{L03658h}
\begin{anhang}
\end{anhang}\normalsize

\doendnotes{C}
\bigskip
\vfill

\clearpage

\footnotesize

\lohead{\textsc{register}}

% Definiere theindex-Environment komplett neu ohne reledmac
\makeatletter
\renewenvironment{theindex}{%
  \section*{\indexname}%
  \setlength{\parindent}{0pt}%
  \setlength{\parskip}{0pt plus 0.3pt}%
  \let\item\@idxitem
}{%
  \clearpage
}
\makeatother

\IfFileExists{\jobname-pw.ind}{\input{\jobname-pw.ind}}{}

\end{document}

      