%% latex-leseansicht-vorspann.tex
%% Vorspann für die Leseansicht.
%% Lädt die gemeinsame Datei latex-vorspann.tex mit nicht gesetztem Schalter.

\newif\ifkorrekturansicht
\korrekturansichtfalse

\input{../tex-inputs/latex-vorspann}


\section[Stefan Zweig an Arthur Schnitzler, 26. 4. 1916]{L03658 Stefan Zweig an Arthur Schnitzler, 26. 4. 1916}
\nopagebreak\mylabel{L03658v}
\rehead{ }\normalsize\beginnumbering\briefempfaengerindex{Schnitzler, Arthur@\textsc{Schnitzler, Arthur}!zzzZweig, Stefan@\emph{von Stefan Zweig}!1916-04-261@{26. 4. 1916}|(be}
\toendnotes[C]{\smallbreak\pagebreak[2]}
\correspDesc{Versand  durch Stefan Zweig am 26. 4. 1916 in Wien
\newline{}Erhalt  durch Arthur Schnitzler im Zeitraum [27. 4. 1916
                  – 1. 5. 1916?] in Wien}\toendnotes[C]{\smallbreak}
\Standort{CUL, Schnitzler, B 118.}
\physDesc{Briefkarte, 431 Zeichen
\newline{}Handschrift: blaue Tinte, lateinische Kurrent
\newline{}Schnitzler: mit rotem Buntstift eine Unterstreichung }
\buchAbdrucke{\weitereDrucke{Stefan Zweig: \emph{Briefwechsel mit Hermann Bahr, Sigmund Freud, Rainer Maria
                        Rilke und Arthur Schnitzler}. Herausgegeben von Jeffrey B. Berlin, Hans-Ulrich Lindken und Donald A. Prater. Frankfurt am Main: \emph{S. Fischer} 1987, S. 399.} }\toendnotes[C]{\smallbreak}
\pstart
           \raggedleft{}{\pb}26. IV 1916\pend
           
\pstart
           \textcolor{gray}{\textbf{SZ}}\hfill \textcolor{gray}{\textbf{VIII. KOCHGASSE 8\oindex{Wien@\textbf{Wien}!VIII., Josefstadt@\textbf{VIII., Josefstadt}!Kochgasse 8@\textbf{Kochgasse 8}, \emph{Wohngebäude}|pw}.}}\pend
           \vspace{0.5em}
\pstart
           Lieber verehrter Herr Doktor, ich wollte es Ihnen seit langem sagen,
               dass Sie es nicht falsch verstehen mögen, wenn ich mich gar nicht bei Ihnen zeigte
               und anfragte – ich habe mich in die \label{K_L03658-1v}\edtext{Nähe Wiens\oindex{Wien@\textbf{Wien}!XXIII., Liesing@\textbf{XXIII., Liesing}!Haselbrunnerstraße 12@\textbf{Haselbrunnerstraße 12}, \emph{Wohngebäude}|pwv}}{\lemma{\textnormal{\emph{Nähe Wiens}}}\Cendnote{\textnormal{Zweig\pwindex{Zweig, Stefan 28.\,11.\,1881 Wien – 23.\,2.\,1942 Petrópolis@\textsc{Zweig, Stefan} (28.\,11.\,1881 Wien – 23.\,2.\,1942 Petrópolis), \emph{Schriftsteller}|pwk} mietete 1916 und 1917 zwei Pavillons\oindex{Wien@\textbf{Wien}!XXIII., Liesing@\textbf{XXIII., Liesing}!Haselbrunnerstraße 12@\textbf{Haselbrunnerstraße 12}, \emph{Wohngebäude}|pwkv} in Kalksburg\oindex{Wien@\textbf{Wien}!XXIII., Liesing@\textbf{XXIII., Liesing}!Kalksburg@\textbf{Kalksburg}, \emph{Region}|pwk}, die er mit seiner Lebensgefährtin Friderike Winternitz\pwindex{Zweig, Friderike Maria 4.\,12.\,1882 Wien – 18.\,1.\,1971 Stamford@\textsc{Zweig, Friderike Maria} (4.\,12.\,1882 Wien – 18.\,1.\,1971 Stamford), \emph{Schriftstellerin}|pwk} in der warmen Jahreszeit
                  bewohnte.}}}\label{K_L03658-1} zurückgezogen, um von meinem zerstückelten Leben den armen Rest
               für Arbeit nützen zu können. Umsomehr freue ich mich, \label{K_L03658-2v}\edtext{Ihre liebe Frau\pwindex{Schnitzler, Olga 17.\,1.\,1882 Wien – 13.\,1.\,1970 Lugano@\textsc{Schnitzler, Olga} (17.\,1.\,1882 Wien – 13.\,1.\,1970 Lugano), \emph{Schauspielerin, Sängerin}|pwv}}{\lemma{\textnormal{\emph{Ihre liebe Frau}}}\Cendnote{\textnormal{Am 29. 4. 1916 sang Olga\pwindex{Zweig, Friderike Maria 4.\,12.\,1882 Wien – 18.\,1.\,1971 Stamford@\textsc{Zweig, Friderike Maria} (4.\,12.\,1882 Wien – 18.\,1.\,1971 Stamford), \emph{Schriftstellerin}|pwk} ein Wohltätigkeitskonzert\eventindex{Allgemeine Poliklinik [neues Gebäude]@\textbf{Allgemeine Poliklinik [neues Gebäude]}!Gesangskonzert von Olga Schnitzler, 29.4.1916@Gesangskonzert von Olga Schnitzler, 29.4.1916|pwkv} in einem Hörsaal der Allgemeinen Poliklinik.\oindex{Wien@\textbf{Wien}!IX., Alsergrund@\textbf{IX., Alsergrund}!Allgemeine Poliklinik [neues Gebäude]@\textbf{Allgemeine Poliklinik [neues Gebäude]}, \emph{Krankenhaus}|pwk}}}}\label{K_L03658-2}{ }Samstag\eventindex{Allgemeine Poliklinik [neues Gebäude]@\textbf{Allgemeine Poliklinik [neues Gebäude]}!Gesangskonzert von Olga Schnitzler, 29.4.1916@Gesangskonzert von Olga Schnitzler, 29.4.1916|pwv} zu hören und hoffentlich {\pb}Sie auch sehen zu dürfen. In Verehrung
               getreu Ihr\pend
           \pstart \spacefill\mbox{Stefan Zweig}\pend{}\selectlanguage{ngerman}\endnumbering\briefempfaengerindex{Schnitzler, Arthur@\textsc{Schnitzler, Arthur}!zzzZweig, Stefan@\emph{von Stefan Zweig}!1916-04-261@{26. 4. 1916}|)be}\mylabel{L03658h}  \newcommand{\dateiname}{L03658}\newcommand{\titel}{Stefan Zweig an Arthur Schnitzler, 26. 4. 1916}\newcommand{\editorInnen}{Selma Jahnke und Martin Anton Müller}%% latex-leseansicht-abspann.tex
%% Abspann für die Leseansicht.
%% Der Schalter \ifkorrekturansicht ist bereits durch den Vorspann gesetzt.

%% latex-abspann.tex
%% Gemeinsamer Abspann für Korrekturansicht und Leseansicht.
%% Setzt den Schalter \ifkorrekturansicht voraus (gesetzt in den
%% einbindenden Dateien latex-korrekturansicht-abspann.tex bzw.
%% latex-leseansicht-abspann.tex).
%% ---------------------------------------------------------------

\normalsize

% Das esempio-Environment wird nur in der Leseansicht benötigt
\ifkorrekturansicht\else
\newenvironment{esempio}[3]%
{
    \vspace{1.5ex}
    \rlap{\underline{#1}}
    \par
    \setlength{\parindent}{0cm}
    \nopagebreak
    \leftskip=#2cm
    \rightskip=#3cm
}
{
    \par
}
\fi

\doendnotes{C}
\bigskip
\vfill

\clearpage

\footnotesize

\ifkorrekturansicht
  \lohead{\textsc{register}}
\fi

% theindex-Environment neu definieren ohne reledmac
\makeatletter
\renewenvironment{theindex}{%
  \ifkorrekturansicht
    \section*{\indexname}%
  \else
    \subsubsection*{Index der erwähnten Entitäten}%
  \fi
  \setlength{\parindent}{0pt}%
  \setlength{\parskip}{0pt plus 0.3pt}%
  \let\item\@idxitem
}{%
  \ifkorrekturansicht\clearpage\fi
}
\makeatother

\IfFileExists{\jobname-pw.ind}{\input{\jobname-pw.ind}}{}

% Quellenangabe nur in der Leseansicht
\ifkorrekturansicht\else
% Fallback-Definitionen, falls die .tex-Datei \titel etc. nicht gesetzt hat
\providecommand{\titel}{}
\providecommand{\editorInnen}{}
\providecommand{\dateiname}{\jobname}

\vspace{3cm}

\vfill

\footnotesize
\textsc{Quelle}: \titel. Herausgegeben von {\editorInnen}. In: \emph{Arthur Schnitzler: Briefwechsel mit Autorinnen und Autoren}.
 Digitale Edition, https://schnitzler-briefe.acdh.oeaw.ac.at/{\dateiname}.html (Stand \today)
\fi

\end{document}


