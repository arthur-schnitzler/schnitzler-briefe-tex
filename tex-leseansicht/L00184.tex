%% latex-korrekturansicht-vorspann.tex
%% Vorspann für die Korrekturansicht.
%% Lädt die gemeinsame Datei latex-vorspann.tex mit gesetztem Schalter.

\newif\ifkorrekturansicht
\korrekturansichttrue

\input{../tex-inputs/latex-vorspann}


\section[Karl Kraus an Arthur Schnitzler, 4. 3. 1893]{L00184 Karl Kraus an Arthur Schnitzler, 4. 3. 1893}
\nopagebreak\mylabel{L00184v}
\rehead{ }\normalsize\beginnumbering\briefempfaengerindex{Schnitzler, Arthur@\textsc{Schnitzler, Arthur}!zzzKraus, Karl@\emph{von Karl Kraus}!1893-03-041@{4. 3. 1893}|(be}
\toendnotes[C]{\smallbreak\pagebreak[2]}\Standort{CUL, Schnitzler, B 55.}
\physDesc{Postkarte, 655 Zeichen
\newline{}Handschrift: schwarze Tinte, deutsche Kurrent
\newline{}Versand: 1) Stempel: »\nobreak{}\oindex{Berlin@\textbf{Berlin}, \emph{P.PPLC}|pwk}Berlin S. O. 26, 4. 3. 93, 7–8 N\nobreak{}«.   2) Stempel: »\nobreak{}\oindex{Opatija@\textbf{Opatija}, \emph{P.PPLA2}|pwk}Abbazia, 6/3 \textcolor{gray}{93}\nobreak{}«. }
\buchAbdrucke{\weitereDrucke{1) \emph{Literatur und Kritik}, Bd. 49, Oktober 1970, S. 515–516.} \weitereDrucke{2) Hermann Bahr, Arthur Schnitzler: \emph{Briefwechsel, Aufzeichnungen, Dokumente (1891–1931)}. Göttingen: \emph{Wallstein} 2018, S. 34.} }\toendnotes[C]{\smallbreak}\pstart{}{\pb}Herrn\pend{}\pstart{}D\textsuperscript{r.} Arthur Schnitzler\pend{}\pstart{}Abbazia\oindex{Hotel Guarnero@\textbf{Hotel Guarnero}, \emph{Hotel (K.HTL)}|pw} / (Curort)\pend{}\pstart{}Quisisina\oindex{Pension Quisisana@\textbf{Pension Quisisana}, \emph{Hotel (K.HTL)}|pw}\pend{}{\bigskip}\vspace{1em}
\pstart
           {\pb}Berlin\oindex{Berlin@\textbf{Berlin}, \emph{P.PPLC}|pw}, 4/3 93.\pend
           
\pstart{}Lieber kleiner Doctor!\pend\vspace{0.5em}
\pstart
           Ich dank Ihnen ſehr für Ihr liebes Schreiben. Mitte der nächſten Woche bin ich wieder
               in Wien\oindex{Wien@\textbf{Wien}, \emph{A.ADM2}|pw} (über Leipzig\oindex{Leipzig@\textbf{Leipzig}, \emph{P.PPLA3}|pw} u Prag\oindex{Prag@\textbf{Prag}, \emph{A.ADM1}|pw}).\pend
           
\pstart
           Ich vergaß damals \uline{Loris}\pwindex{Hofmannsthal, Hugo von 1874-02-01 – 1929-07-15@\textsc{Hofmannsthal, Hugo von} (1874-02-01 – 1929-07-15), \emph{Schriftsteller/Schriftstellerin}|pw} zu grüßen. Bitte, tragen Sie das nach, wenn Sie ihm ſchreiben. \label{K_L00184-1v}\edtext{Duße\pwindex{Duse, Eleonora 03.10.1858 – 21.04.1924@\textsc{Duse, Eleonora} (03.10.1858 – 21.04.1924), \emph{Schauspieler/Schauspielerin}|pw}}{\lemma{\textnormal{\emph{Duße}}}\Cendnote{\textnormal{Warum der Austausch über die
                  Schauspielerin zu diesem Zeitpunkt stattfindet, ist unklar. Schnitzler hatte Eleonora
                     Duse\pwindex{Duse, Eleonora 03.10.1858 – 21.04.1924@\textsc{Duse, Eleonora} (03.10.1858 – 21.04.1924), \emph{Schauspieler/Schauspielerin}|pwk} bereits zehn Monate zuvor gesehen: »17.5. Theaterausstellung\oindex{Internationales Ausstellungstheater im k.k. Prater@\textbf{Internationales Ausstellungstheater im k.k. Prater}, \emph{Theater (K.THE)}|pw}? Sardou\pwindex{Sardou, Victorien 07.09.1831 – 08.11.1908@\textsc{Sardou, Victorien} (07.09.1831 – 08.11.1908), \emph{Schriftsteller/Schriftstellerin}|pw}: Fernande\pwindex{Fernanda. Commedia in 4 atti@\emph{Fernanda. Commedia in 4 atti}|pw}. (Duse\pwindex{Duse, Eleonora 03.10.1858 – 21.04.1924@\textsc{Duse, Eleonora} (03.10.1858 – 21.04.1924), \emph{Schauspieler/Schauspielerin}|pw}).« (\emph{Theaterbesuche}, \emph{Cambridge University
                        Library}, Schnitzler, A 179a; nicht im \emph{Tagebuch}\pwindex{Tagebuch@\emph{Tagebuch}|pwk}). Zwei Tage später sah er sie noch in Ibsens\pwindex{Ibsen, Henrik 20.03.1828 – 23.05.1906@\textsc{Ibsen, Henrik} (20.03.1828 – 23.05.1906), \emph{Schriftsteller/Schriftstellerin}|pwk}{ }\emph{Nora}\pwindex{Nora oder ein Puppenheim. Schauspiel in drei Akten@\emph{Nora oder ein Puppenheim. Schauspiel in drei Akten}|pwk}. In Berlin\oindex{Berlin@\textbf{Berlin}, \emph{P.PPLC}|pwk} hingegen trat sie im Dezember 1892 zum
                  ersten Mal auf, ein zweites Gastspiel fand ein Jahr später statt.}}}\label{K_L00184-1} vor der
                  \uline{Wolter}\pwindex{Wolter, Charlotte 01.03.1834 – 14.06.1897@\textsc{Wolter, Charlotte} (01.03.1834 – 14.06.1897), \emph{Schauspieler/Schauspielerin}|pw}? Jemine! \label{K_L00184-2v}\edtext{Wengraf\pwindex{Wengraf, Edmund 09.01.1860 – 08.12.1933@\textsc{Wengraf, Edmund} (09.01.1860 – 08.12.1933), \emph{Schriftsteller/Schriftstellerin, Journalist/Journalistin, Kaufmann/Kauffrau}|pw} verriſs ſie}{\lemma{\textnormal{\emph{Wengraf verriſs ſie}}}\Cendnote{\textnormal{unklar, möglicherweise keine publizierte Aussage}}}\label{K_L00184-2}, \label{K_L00184-3v}\edtext{Bahr\pwindex{Bahr, Hermann 19.07.1863 – 15.01.1934@\textsc{Bahr, Hermann} (19.07.1863 – 15.01.1934), \emph{Schriftsteller/Schriftstellerin, Kritiker/Kritikerin}|pw} hob ſie in alle Himmel}{\lemma{\textnormal{\emph{Bahr … Himmel}}}\Cendnote{\textnormal{Bahr\pwindex{Bahr, Hermann 19.07.1863 – 15.01.1934@\textsc{Bahr, Hermann} (19.07.1863 – 15.01.1934), \emph{Schriftsteller/Schriftstellerin, Kritiker/Kritikerin}|pwk} rezensierte die Wien\oindex{Wien@\textbf{Wien}, \emph{A.ADM2}|pwk}er Gastspiele nicht. Es dürfte sich also um eine
                  Anspielung auf das Feuilleton \emph{Eleonora Duse}\pwindex{Eleonora Duse@\emph{Eleonora Duse}|pwk}
                  vom 9. 5. 1891 (\emph{Frankfurter Zeitung}\pwindex{Frankfurter Zeitung@\emph{Frankfurter Zeitung}|pwk}, Jg. 35, Nr. 129, 1.
                     Morgenblatt, S. 1–2) oder auf den Abdruck in der \emph{Russischen Reise}\pwindex{Russische Reise@\emph{Russische Reise}|pwk} (S. 116–125) handeln, womit die
                  deutschsprachige Duse\pwindex{Duse, Eleonora 03.10.1858 – 21.04.1924@\textsc{Duse, Eleonora} (03.10.1858 – 21.04.1924), \emph{Schauspieler/Schauspielerin}|pwk}-Rezeption eingeleitet
                  wurde.}}}\label{K_L00184-3} – beides ſpricht gegen ſie. Aber \uline{Ihre} Worte machen mich ſtutzen. »Wollen mal ſehen, was ſich machen läſst«
               Ich bin gewiss der Letzte, der der Frau nicht ihr Recht widerfahren läſst. Leben Sie
               recht wohl, \label{K_L00184-4v}\edtext{ertrinken Sie mir
                  nicht}{\lemma{\textnormal{\emph{ertrinken Sie mir
                  nicht}}}\Cendnote{\textnormal{Schnitzler machte vom 4. 3. 1893 bis
                     zum 11. 3. 1893 an der Adria\oindex{Adriatisches Meer@\textbf{Adriatisches Meer}, \emph{Meer (N.MER)}|pwk} Urlaub.}}}\label{K_L00184-4} u ſeien Sie mir
               herzlichſt gegrüßt\hspace*{3.5em}Ihr \spacefill\mbox{KarlKraus}\pend
           
\pstart
           \noindent{}\label{T_L00184-1v}\edtext{Buſſe\pwindex{Busse, Carl 12.11.1872 – 04.12.1918@\textsc{Busse, Carl} (12.11.1872 – 04.12.1918), \emph{Schriftsteller/Schriftstellerin}|pw} dankt u. grüßt herzlichſt.}{\lemma{\textnormal{\emph{Buſſe … herzlichſt.}}}\Cendnote{\textnormal{in der oberen rechten Ecke}}}\label{T_L00184-1}\pend
           \selectlanguage{ngerman}\endnumbering\briefempfaengerindex{Schnitzler, Arthur@\textsc{Schnitzler, Arthur}!zzzKraus, Karl@\emph{von Karl Kraus}!1893-03-041@{4. 3. 1893}|)be}\mylabel{L00184h}  \normalsize

\doendnotes{C}
\bigskip
\vfill

\clearpage

\footnotesize

\lohead{\textsc{register}}

% Definiere theindex-Environment komplett neu ohne reledmac
\makeatletter
\renewenvironment{theindex}{%
  \section*{\indexname}%
  \setlength{\parindent}{0pt}%
  \setlength{\parskip}{0pt plus 0.3pt}%
  \let\item\@idxitem
}{%
  \clearpage
}
\makeatother

\IfFileExists{\jobname-pw.ind}{\input{\jobname-pw.ind}}{}

\end{document}

      