%% latex-leseansicht-vorspann.tex
%% Vorspann für die Leseansicht.
%% Lädt die gemeinsame Datei latex-vorspann.tex mit nicht gesetztem Schalter.

\newif\ifkorrekturansicht
\korrekturansichtfalse

\input{../tex-inputs/latex-vorspann}


\section[Karl Kraus an Arthur Schnitzler, 4. 3. 1893]{L00184 Karl Kraus an Arthur Schnitzler, 4. 3. 1893}
\nopagebreak\mylabel{L00184v}
\rehead{ }\normalsize\beginnumbering\briefempfaengerindex{Schnitzler, Arthur@\textsc{Schnitzler, Arthur}!zzzKraus, Karl@\emph{von Karl Kraus}!1893-03-041@{4. 3. 1893}|(be}
\toendnotes[C]{\smallbreak\pagebreak[2]}
\correspDesc{Versand  durch Karl Kraus am 4. 3. 1893 in Berlin
\newline{}Erhalt  durch Arthur Schnitzler am 6. 3. 93 in Opatija}\toendnotes[C]{\smallbreak}
\Standort{CUL, Schnitzler, B 55.}
\physDesc{Postkarte, 655 Zeichen
\newline{}Handschrift: schwarze Tinte, deutsche Kurrent
\newline{}Versand: 1) Stempel: »\nobreak{}\oindex{Berlin@\textbf{Berlin}, \emph{Hauptstadt}|pwk}Berlin S. O. 26, 4. 3. 93, 7–8 N\nobreak{}«.   2) Stempel: »\nobreak{}\oindex{Opatija@\textbf{Opatija}, \emph{Hauptstadt}|pwk}Abbazia, 6/3 \textcolor{gray}{93}\nobreak{}«. }
\buchAbdrucke{\weitereDrucke{1) \emph{Karl Kraus und Arthur Schnitzler. Eine Dokumentation.}Herausgegeben von Reinhard Urbach In: \emph{Literatur und Kritik}, Bd. 49, Oktober 1970, S. 515–516.} \weitereDrucke{2) Hermann Bahr, Arthur Schnitzler: \emph{Briefwechsel, Aufzeichnungen, Dokumente (1891–1931)}. Herausgegeben von Kurt Ifkovits und Martin Anton Müller. Göttingen: \emph{Wallstein} 2018, S. 34.} }\toendnotes[C]{\smallbreak}\pstart{}{\pb}Herrn\pend{}\pstart{}D\textsuperscript{r.} Arthur Schnitzler\pend{}\pstart{}Abbazia\oindex{Hotel Guarnero@\textbf{Hotel Guarnero}, \emph{Hotel}|pw} / (Curort)\pend{}\pstart{}Quisisina\oindex{Pension Quisisana@\textbf{Pension Quisisana}, \emph{Hotel}|pw}\pend{}{\bigskip}\vspace{1em}
\pstart
           {\pb}Berlin\oindex{Berlin@\textbf{Berlin}, \emph{Hauptstadt}|pw}, 4/3 93.\pend
           
\pstart{}Lieber kleiner Doctor!\pend\vspace{0.5em}
\pstart
           Ich dank Ihnen{ }ſehr für Ihr liebes Schreiben. Mitte der nächſten Woche bin ich wieder
               in Wien\oindex{Wien@\textbf{Wien}, \emph{Verwaltungsgebiet}|pw} (über Leipzig\oindex{Leipzig@\textbf{Leipzig}, \emph{Hauptstadt}|pw} u Prag\oindex{Prag@\textbf{Prag}, \emph{Land}|pw}).\pend
           
\pstart
           Ich vergaß damals \uline{Loris}\pwindex{Hofmannsthal, Hugo von 1.\,2.\,1874 Wien – 15.\,7.\,1929 Rodaun@\textsc{Hofmannsthal, Hugo von} (1.\,2.\,1874 Wien – 15.\,7.\,1929 Rodaun), \emph{Schriftsteller}|pw} zu grüßen. Bitte, tragen Sie das nach, wenn Sie ihm{ }ſchreiben. \label{K_L00184-1v}\edtext{Duße\pwindex{Duse, Eleonora 3.\,10.\,1858 Vigevano – 21.\,4.\,1924 Pittsburgh@\textsc{Duse, Eleonora} (3.\,10.\,1858 Vigevano – 21.\,4.\,1924 Pittsburgh), \emph{Schauspielerin}|pw}}{\lemma{\textnormal{\emph{Duße}}}\Cendnote{\textnormal{Warum der Austausch über die
                  Schauspielerin zu diesem Zeitpunkt stattfindet, ist unklar. Schnitzler hatte Eleonora
                     Duse\pwindex{Duse, Eleonora 3.\,10.\,1858 Vigevano – 21.\,4.\,1924 Pittsburgh@\textsc{Duse, Eleonora} (3.\,10.\,1858 Vigevano – 21.\,4.\,1924 Pittsburgh), \emph{Schauspielerin}|pwk} bereits zehn Monate zuvor gesehen: »17.5. Theaterausstellung\oindex{Wien@\textbf{Wien}!II., Leopoldstadt@\textbf{II., Leopoldstadt}!Internationales Ausstellungstheater im k.k. Prater@\textbf{Internationales Ausstellungstheater im k.k. Prater}, \emph{Theater}|pw}? Sardou\pwindex{Sardou, Victorien 7.\,9.\,1831 Paris – 8.\,11.\,1908 ebd.@\textsc{Sardou, Victorien} (7.\,9.\,1831 Paris – 8.\,11.\,1908 ebd.), \emph{Schriftsteller}|pw}: Fernande\pwindex{Sardou, Victorien 7.\,9.\,1831 Paris – 8.\,11.\,1908 ebd.@\textsc{Sardou, Victorien} (7.\,9.\,1831 Paris – 8.\,11.\,1908 ebd.), \emph{Schriftsteller}!Fernanda. Commedia in 4 atti@\strich\emph{Fernanda. Commedia in 4 atti}|pw}. (Duse\pwindex{Duse, Eleonora 3.\,10.\,1858 Vigevano – 21.\,4.\,1924 Pittsburgh@\textsc{Duse, Eleonora} (3.\,10.\,1858 Vigevano – 21.\,4.\,1924 Pittsburgh), \emph{Schauspielerin}|pw}).« (\emph{Theaterbesuche}, \emph{Cambridge University
                        Library}, Schnitzler, A 179a; nicht im \emph{Tagebuch}\pwindex{Schnitzler, Arthur 15.\,5.\,1862 Wien – 21.\,10.\,1931 ebd.@\textsc{Schnitzler, Arthur} (15.\,5.\,1862 Wien – 21.\,10.\,1931 ebd.), \emph{Schriftsteller, Mediziner}!Tagebuch@\strich\emph{Tagebuch}|pwk}). Zwei Tage später sah er sie noch in Ibsens\pwindex{Ibsen, Henrik 20.\,3.\,1828 Skien – 23.\,5.\,1906 Oslo@\textsc{Ibsen, Henrik} (20.\,3.\,1828 Skien – 23.\,5.\,1906 Oslo), \emph{Schriftsteller}|pwk}{ }\emph{Nora}\pwindex{Ibsen, Henrik 20.\,3.\,1828 Skien – 23.\,5.\,1906 Oslo@\textsc{Ibsen, Henrik} (20.\,3.\,1828 Skien – 23.\,5.\,1906 Oslo), \emph{Schriftsteller}!Nora oder ein Puppenheim. Schauspiel in drei Akten@\strich\emph{Nora oder ein Puppenheim. Schauspiel in drei Akten}|pwk}. In Berlin\oindex{Berlin@\textbf{Berlin}, \emph{Hauptstadt}|pwk} hingegen trat sie im Dezember 1892 zum
                  ersten Mal auf, ein zweites Gastspiel fand ein Jahr später statt.}}}\label{K_L00184-1} vor der
                  \uline{Wolter}\pwindex{Wolter, Charlotte 1.\,3.\,1834 Köln – 14.\,6.\,1897 Wien@\textsc{Wolter, Charlotte} (1.\,3.\,1834 Köln – 14.\,6.\,1897 Wien), \emph{Schauspielerin}|pw}? Jemine! \label{K_L00184-2v}\edtext{Wengraf\pwindex{Wengraf, Edmund 9.\,1.\,1860 Mikulov – 8.\,12.\,1933 Wien@\textsc{Wengraf, Edmund} (9.\,1.\,1860 Mikulov – 8.\,12.\,1933 Wien), \emph{Schriftsteller, Journalist, Kaufmann}|pw} verriſs{ }ſie}{\lemma{\textnormal{\emph{Wengraf verriss sie}}}\Cendnote{\textnormal{unklar, möglicherweise keine publizierte Aussage}}}\label{K_L00184-2}, \label{K_L00184-3v}\edtext{Bahr\pwindex{Bahr, Hermann 19.\,7.\,1863 Linz – 15.\,1.\,1934 München@\textsc{Bahr, Hermann} (19.\,7.\,1863 Linz – 15.\,1.\,1934 München), \emph{Schriftsteller, Kritiker}|pw} hob{ }ſie in alle Himmel}{\lemma{\textnormal{\emph{Bahr … Himmel}}}\Cendnote{\textnormal{Bahr\pwindex{Bahr, Hermann 19.\,7.\,1863 Linz – 15.\,1.\,1934 München@\textsc{Bahr, Hermann} (19.\,7.\,1863 Linz – 15.\,1.\,1934 München), \emph{Schriftsteller, Kritiker}|pwk} rezensierte die Wien\oindex{Wien@\textbf{Wien}, \emph{Verwaltungsgebiet}|pwk}er Gastspiele nicht. Es dürfte sich also um eine
                  Anspielung auf das Feuilleton \emph{Eleonora Duse}\pwindex{Bahr, Hermann 19.\,7.\,1863 Linz – 15.\,1.\,1934 München@\textsc{Bahr, Hermann} (19.\,7.\,1863 Linz – 15.\,1.\,1934 München), \emph{Schriftsteller, Kritiker}!Eleonora Duse@\strich\emph{Eleonora Duse}|pwk}
                  vom 9. 5. 1891 (\emph{Frankfurter Zeitung}\pwindex{Frankfurter Zeitung@\emph{Frankfurter Zeitung}|pwk}, Jg. 35, Nr. 129, 1.
                     Morgenblatt, S. 1–2) oder auf den Abdruck in der \emph{Russischen Reise}\pwindex{Bahr, Hermann 19.\,7.\,1863 Linz – 15.\,1.\,1934 München@\textsc{Bahr, Hermann} (19.\,7.\,1863 Linz – 15.\,1.\,1934 München), \emph{Schriftsteller, Kritiker}!Russische Reise@\strich\emph{Russische Reise}|pwk} (S. 116–125) handeln, womit die
                  deutschsprachige Duse\pwindex{Duse, Eleonora 3.\,10.\,1858 Vigevano – 21.\,4.\,1924 Pittsburgh@\textsc{Duse, Eleonora} (3.\,10.\,1858 Vigevano – 21.\,4.\,1924 Pittsburgh), \emph{Schauspielerin}|pwk}-Rezeption eingeleitet
                  wurde.}}}\label{K_L00184-3} – beides{ }ſpricht gegen{ }ſie. Aber \uline{Ihre} Worte machen mich{ }ſtutzen. »Wollen mal{ }ſehen, was{ }ſich machen läſst«
               Ich bin gewiss der Letzte, der der Frau nicht ihr Recht widerfahren läſst. Leben Sie
               recht wohl, \label{K_L00184-4v}\edtext{ertrinken Sie mir
                  nicht}{\lemma{\textnormal{\emph{ertrinken Sie mir
                  nicht}}}\Cendnote{\textnormal{Schnitzler machte vom 4. 3. 1893 bis
                     zum 11. 3. 1893 an der Adria\oindex{Adriatisches Meer@\textbf{Adriatisches Meer}|pwk} Urlaub.}}}\label{K_L00184-4} u{ }ſeien Sie mir
               herzlichſt gegrüßt\hspace*{3.5em}Ihr \spacefill\mbox{KarlKraus}\pend
           
\pstart
           \noindent{}\label{T_L00184-1v}\edtext{Buſſe\pwindex{Busse, Carl 12.\,11.\,1872 Międzychód – 4.\,12.\,1918 Berlin@\textsc{Busse, Carl} (12.\,11.\,1872 Międzychód – 4.\,12.\,1918 Berlin), \emph{Schriftsteller}|pw} dankt u. grüßt herzlichſt.}{\lemma{\textnormal{\emph{Busse … herzlichst.}}}\Cendnote{\textnormal{in der oberen rechten Ecke}}}\label{T_L00184-1}\pend
           \selectlanguage{ngerman}\endnumbering\briefempfaengerindex{Schnitzler, Arthur@\textsc{Schnitzler, Arthur}!zzzKraus, Karl@\emph{von Karl Kraus}!1893-03-041@{4. 3. 1893}|)be}\mylabel{L00184h}  \newcommand{\dateiname}{L00184}\newcommand{\titel}{Karl Kraus an Arthur Schnitzler, 4. 3. 1893}\newcommand{\editorInnen}{Herausgegeben von Martin Anton Müller}%% latex-leseansicht-abspann.tex
%% Abspann für die Leseansicht.
%% Der Schalter \ifkorrekturansicht ist bereits durch den Vorspann gesetzt.

%% latex-abspann.tex
%% Gemeinsamer Abspann für Korrekturansicht und Leseansicht.
%% Setzt den Schalter \ifkorrekturansicht voraus (gesetzt in den
%% einbindenden Dateien latex-korrekturansicht-abspann.tex bzw.
%% latex-leseansicht-abspann.tex).
%% ---------------------------------------------------------------

\normalsize

% Das esempio-Environment wird nur in der Leseansicht benötigt
\ifkorrekturansicht\else
\newenvironment{esempio}[3]%
{
    \vspace{1.5ex}
    \rlap{\underline{#1}}
    \par
    \setlength{\parindent}{0cm}
    \nopagebreak
    \leftskip=#2cm
    \rightskip=#3cm
}
{
    \par
}
\fi

\doendnotes{C}
\bigskip
\vfill

\clearpage

\footnotesize

\ifkorrekturansicht
  \lohead{\textsc{register}}
\fi

% theindex-Environment neu definieren ohne reledmac
\makeatletter
\renewenvironment{theindex}{%
  \ifkorrekturansicht
    \section*{\indexname}%
  \else
    \subsubsection*{Index der erwähnten Entitäten}%
  \fi
  \setlength{\parindent}{0pt}%
  \setlength{\parskip}{0pt plus 0.3pt}%
  \let\item\@idxitem
}{%
  \ifkorrekturansicht\clearpage\fi
}
\makeatother

\IfFileExists{\jobname-pw.ind}{\input{\jobname-pw.ind}}{}

% Quellenangabe nur in der Leseansicht
\ifkorrekturansicht\else
% Fallback-Definitionen, falls die .tex-Datei \titel etc. nicht gesetzt hat
\providecommand{\titel}{}
\providecommand{\editorInnen}{}
\providecommand{\dateiname}{\jobname}

\vspace{3cm}

\vfill

\footnotesize
\textsc{Quelle}: \titel. Herausgegeben von {\editorInnen}. In: \emph{Arthur Schnitzler: Briefwechsel mit Autorinnen und Autoren}.
 Digitale Edition, https://schnitzler-briefe.acdh.oeaw.ac.at/{\dateiname}.html (Stand \today)
\fi

\end{document}


