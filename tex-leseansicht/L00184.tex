%% latex-leseansicht-vorspann.tex
%% Vorspann für die Leseansicht.
%% Lädt die gemeinsame Datei latex-vorspann.tex mit nicht gesetztem Schalter.

\newif\ifkorrekturansicht
\korrekturansichtfalse

\input{../tex-inputs/latex-vorspann}


         
         \renewcommand{\erwaehntePersonen}{Personen: Hermann Bahr, Carl Busse, Eleonora Duse, Hugo von Hofmannsthal, Henrik Ibsen, Victorien Sardou, Edmund Wengraf, Charlotte Wolter}
         \renewcommand{\erwaehnteOrte}{Orte: Berlin, Hotel Guarnero, Internationales Ausstellungstheater im k.k. Prater, Leipzig, Opatija, Pension Quisisana, Prag, Wien}
         \renewcommand{\erwaehnteWerke}{Werke: Eleonora Duse, Fernande, Frankfurter Zeitung, Nora oder ein Puppenheim, Russische Reise, Tagebuch}
               \section[Karl Kraus an Arthur Schnitzler, 4. 3. 1893]{ Karl Kraus an Arthur Schnitzler, 4. 3. 1893}\nopagebreak\mylabel{v}\rehead{ }\begin{ledgroupsized}[t]{13cm}\normalsize\beginnumbering \toendnotes[C]{\smallbreak\pagebreak[2]} \Standort{CUL, Schnitzler, B 55.}
\physDesc{Postkarte, 655 Zeichen
\newline{}Handschrift: schwarze Tinte, deutsche Kurrent
\newline{}Versand: 1) Stempel: »\nobreak{}\oindex{Berlin@\textbf{Berlin}|pwk}Berlin S. O. 26, 4. 3. 93, 7–8 N\nobreak{}«.   2) Stempel: »\nobreak{}\oindex{Opatija@\textbf{Opatija}|pwk}Abbazia, 6/3 \textcolor{gray}{93}\nobreak{}«. }\buchAbdrucke{\weitereDrucke{1) \emph{Karl Kraus und Arthur Schnitzler. Eine Dokumentation.} Hg. Reinhard Urbach. In: \emph{Literatur und Kritik}, Bd. 49, Oktober 1970, S. 515–516.} \weitereDrucke{2) Hermann Bahr, Arthur Schnitzler: \emph{Briefwechsel, Aufzeichnungen, Dokumente (1891–1931)}. Hg. Kurt Ifkovits und Martin Anton Müller. Göttingen: \emph{Wallstein} 2018, S. 34.} }\toendnotes[C]{\smallbreak}\pstart{}{\pb}Herrn\pend{}\pstart{}D\textsuperscript{r.} Arthur Schnitzler\pend{}\pstart{}Abbazia\oindex{Hotel Guarnero@\textbf{Hotel Guarnero}|pw} / (Curort)\pend{}\pstart{}Quisisina\oindex{Pension Quisisana@\textbf{Pension Quisisana}|pw}\pend{}{\bigskip}\pstart
           {\pb}Berlin\oindex{Berlin@\textbf{Berlin}|pw}, 4/3 93.\pend
           \pstart{}Lieber kleiner Doctor!\pend\pstart
           Ich dank Ihnen ſehr für Ihr liebes Schreiben. Mitte der nächſten Woche bin ich wieder
               in Wien\oindex{Wien@\textbf{Wien}|pw} (über Leipzig\oindex{Leipzig@\textbf{Leipzig}|pw} u Prag\oindex{Prag@\textbf{Prag}|pw}).\pend
           \pstart
           Ich vergaß damals \uline{Loris}\pwindex{Hofmannsthal, Hugo von 1874-02-01 – 1929-07-15@\textsc{Hofmannsthal, Hugo von} (1874-02-01 – 1929-07-15), \emph{Schriftsteller}|pw} zu grüßen. Bitte, tragen Sie das nach, wenn Sie ihm ſchreiben. \label{K_L00184_1v}\edtext{Duße\pwindex{Duse, Eleonora 03.10.1858 – 21.04.1924@\textsc{Duse, Eleonora} (03.10.1858 – 21.04.1924), \emph{Schauspielerin}|pw}}{\lemma{\textnormal{\emph{Duße}}}\Cendnote{\textnormal{Warum der Austausch über die
                  Schauspielerin zu diesem Zeitpunkt stattfindet, ist unklar. Schnitzler\pwindex{Schnitzler, Arthur 15.05.1862 – 21.10.1931@\textsc{Schnitzler, Arthur} (15.05.1862 – 21.10.1931), \emph{Schriftsteller, Mediziner}|pwk} hatte Eleonora
                     Duse\pwindex{Duse, Eleonora 03.10.1858 – 21.04.1924@\textsc{Duse, Eleonora} (03.10.1858 – 21.04.1924), \emph{Schauspielerin}|pwk} bereits zehn Monate zuvor gesehen: »17.5. Theaterausstellung\oindex{Internationales Ausstellungstheater im k.k. Prater@\textbf{Internationales Ausstellungstheater im k.k. Prater}|pw}? Sardou\pwindex{Sardou, Victorien 07.09.1831 – 08.11.1908@\textsc{Sardou, Victorien} (07.09.1831 – 08.11.1908), \emph{Schriftsteller}|pw}: Fernande\pwindex{Sardou, Victorien 07.09.1831 – 08.11.1908@\textsc{Sardou, Victorien} (07.09.1831 – 08.11.1908), \emph{Schriftsteller}!Fernande1844@\strich\emph{Fernande} {[}1844{]}|pw}. (Duse\pwindex{Duse, Eleonora 03.10.1858 – 21.04.1924@\textsc{Duse, Eleonora} (03.10.1858 – 21.04.1924), \emph{Schauspielerin}|pw}).« (\emph{Theaterbesuche}, \emph{Cambridge University
                        Library}, Schnitzler, A 179a; nicht im \emph{Tagebuch}\pwindex{Schnitzler, Arthur 15.05.1862 – 21.10.1931@\textsc{Schnitzler, Arthur} (15.05.1862 – 21.10.1931), \emph{Schriftsteller, Mediziner}!Tagebuch1981 – 2000@\strich\emph{Tagebuch} {[}1981 – 2000{]}|pwk}). Zwei Tage später sah er sie noch in Ibsen\pwindex{Ibsen, Henrik 20.03.1828 – 23.05.1906@\textsc{Ibsen, Henrik} (20.03.1828 – 23.05.1906), \emph{Schriftsteller}|pwk}s \emph{Nora}\pwindex{Ibsen, Henrik 20.03.1828 – 23.05.1906@\textsc{Ibsen, Henrik} (20.03.1828 – 23.05.1906), \emph{Schriftsteller}!Nora oder ein Puppenheim1879@\strich\emph{Nora oder ein Puppenheim} {[}1879{]}|pwk}. In Berlin\oindex{Berlin@\textbf{Berlin}|pwk} hingegen trat sie im Dezember 1892 zum
                  ersten Mal auf, ein zweites Gastspiel fand ein Jahr später statt.}}}\label{K_L00184_1h} vor der
                  \uline{Wolter}\pwindex{Wolter, Charlotte 01.03.1834 – 14.06.1897@\textsc{Wolter, Charlotte} (01.03.1834 – 14.06.1897), \emph{Schauspielerin}|pw}? Jemine! \label{K_L00184_2v}\edtext{Wengraf\pwindex{Wengraf, Edmund 09.01.1860 – 08.12.1933@\textsc{Wengraf, Edmund} (09.01.1860 – 08.12.1933), \emph{Journalist}|pw} verriſs ſie}{\lemma{\textnormal{\emph{Wengraf verriſs ſie}}}\Cendnote{\textnormal{unklar, möglicherweise keine publizierte Aussage}}}\label{K_L00184_2h}, \label{K_L00184_3v}\edtext{Bahr\pwindex{Bahr, Hermann 19.07.1863 – 15.01.1934@\textsc{Bahr, Hermann} (19.07.1863 – 15.01.1934), \emph{Schriftsteller, Kritiker}|pw} hob ſie in alle Himmel}{\lemma{\textnormal{\emph{Bahr … Himmel}}}\Cendnote{\textnormal{Bahr\pwindex{Bahr, Hermann 19.07.1863 – 15.01.1934@\textsc{Bahr, Hermann} (19.07.1863 – 15.01.1934), \emph{Schriftsteller, Kritiker}|pwk} rezensierte die Wien\oindex{Wien@\textbf{Wien}|pwk}er Gastspiele nicht. Es dürfte sich also um eine
                  Anspielung auf das Feuilleton \emph{Eleonora Duse}\pwindex{Bahr, Hermann 19.07.1863 – 15.01.1934@\textsc{Bahr, Hermann} (19.07.1863 – 15.01.1934), \emph{Schriftsteller, Kritiker}!Eleonora Duse09. 05. 1891@\strich\emph{Eleonora Duse} {[}09. 05. 1891{]}|pwk}
                  vom 9. 5. 1891 (\emph{Frankfurter Zeitung}\pwindex{?? Werk@Nicht ermittelte Verfasserinnen und Verfasser!Frankfurter Zeitung1856 – 1943@\emph{Frankfurter Zeitung} {[}1856 – 1943{]}|pwk}, Jg. 35, Nr. 129, 1.
                     Morgenblatt, S. 1–2) oder auf den Abdruck in der \emph{Russischen Reise}\pwindex{Bahr, Hermann 19.07.1863 – 15.01.1934@\textsc{Bahr, Hermann} (19.07.1863 – 15.01.1934), \emph{Schriftsteller, Kritiker}!Russische Reise1891@\strich\emph{Russische Reise} {[}1891{]}|pwk} (S. 116–125) handeln, womit die
                  deutschsprachige Duse\pwindex{Duse, Eleonora 03.10.1858 – 21.04.1924@\textsc{Duse, Eleonora} (03.10.1858 – 21.04.1924), \emph{Schauspielerin}|pwk}-Rezeption eingeleitet
                  wurde.}}}\label{K_L00184_3h} – beides ſpricht gegen ſie. Aber \uline{Ihre} Worte machen mich ſtutzen. »Wollen mal ſehen, was ſich machen läſst«
               Ich bin gewiss der Letzte, der der Frau nicht ihr Recht widerfahren läſst. Leben Sie
               recht wohl, \label{K_L00184_4v}\edtext{ertrinken Sie mir
                  nicht}{\lemma{\textnormal{\emph{ertrinken Sie mir
                  nicht}}}\Cendnote{\textnormal{Schnitzler\pwindex{Schnitzler, Arthur 15.05.1862 – 21.10.1931@\textsc{Schnitzler, Arthur} (15.05.1862 – 21.10.1931), \emph{Schriftsteller, Mediziner}|pwk} urlaubte vom 1. bis
                  zum 11. 3. an der Adria.}}}\label{K_L00184_4h} u ſeien Sie mir
               herzlichſt gegrüßt\hspace*{3.5em}Ihr \spacefill\mbox{KarlKraus}\pend
           \pstart
           \noindent{}\label{T_L00184_1v}\edtext{Buſſe\pwindex{Busse, Carl 12.11.1872 – 04.12.1918@\textsc{Busse, Carl} (12.11.1872 – 04.12.1918), \emph{Schriftsteller}|pw} dankt u. grüßt herzlichſt.}{\lemma{\textnormal{\emph{Buſſe … herzlichſt.}}}\Cendnote{\textnormal{in der oberen rechten Ecke}}}\label{T_L00184_1h}\pend
           
         
         \endnumbering\mylabel{h}\end{ledgroupsized}  \newcommand{\dateiname}{L00184}\newcommand{\titel}{Karl Kraus an Arthur Schnitzler, 4. 3. 1893}\newcommand{\editorInnen}{ Martin Anton Müller und Gerd-Hermann Susen}%% latex-leseansicht-abspann.tex
%% Abspann für die Leseansicht.
%% Der Schalter \ifkorrekturansicht ist bereits durch den Vorspann gesetzt.

%% latex-abspann.tex
%% Gemeinsamer Abspann für Korrekturansicht und Leseansicht.
%% Setzt den Schalter \ifkorrekturansicht voraus (gesetzt in den
%% einbindenden Dateien latex-korrekturansicht-abspann.tex bzw.
%% latex-leseansicht-abspann.tex).
%% ---------------------------------------------------------------

\normalsize

% Das esempio-Environment wird nur in der Leseansicht benötigt
\ifkorrekturansicht\else
\newenvironment{esempio}[3]%
{
    \vspace{1.5ex}
    \rlap{\underline{#1}}
    \par
    \setlength{\parindent}{0cm}
    \nopagebreak
    \leftskip=#2cm
    \rightskip=#3cm
}
{
    \par
}
\fi

\doendnotes{C}
\bigskip
\vfill

\clearpage

\footnotesize

\ifkorrekturansicht
  \lohead{\textsc{register}}
\fi

% theindex-Environment neu definieren ohne reledmac
\makeatletter
\renewenvironment{theindex}{%
  \ifkorrekturansicht
    \section*{\indexname}%
  \else
    \subsubsection*{Index der erwähnten Entitäten}%
  \fi
  \setlength{\parindent}{0pt}%
  \setlength{\parskip}{0pt plus 0.3pt}%
  \let\item\@idxitem
}{%
  \ifkorrekturansicht\clearpage\fi
}
\makeatother

\IfFileExists{\jobname-pw.ind}{\input{\jobname-pw.ind}}{}

% Quellenangabe nur in der Leseansicht
\ifkorrekturansicht\else
% Fallback-Definitionen, falls die .tex-Datei \titel etc. nicht gesetzt hat
\providecommand{\titel}{}
\providecommand{\editorInnen}{}
\providecommand{\dateiname}{\jobname}

\vspace{3cm}

\vfill

\footnotesize
\textsc{Quelle}: \titel. Herausgegeben von {\editorInnen}. In: \emph{Arthur Schnitzler: Briefwechsel mit Autorinnen und Autoren}.
 Digitale Edition, https://schnitzler-briefe.acdh.oeaw.ac.at/{\dateiname}.html (Stand \today)
\fi

\end{document}


      