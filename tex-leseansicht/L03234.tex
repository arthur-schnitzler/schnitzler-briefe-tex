%% latex-leseansicht-vorspann.tex
%% Vorspann für die Leseansicht.
%% Lädt die gemeinsame Datei latex-vorspann.tex mit nicht gesetztem Schalter.

\newif\ifkorrekturansicht
\korrekturansichtfalse

\input{../tex-inputs/latex-vorspann}


         
         \renewcommand{\erwaehntePersonen}{Personen: Siegfried Jacobsohn}
         \renewcommand{\erwaehnteInstitutionen}{Institutionen: Neue Freie Presse}
         \renewcommand{\erwaehnteOrte}{Orte: Berlin, Hotel Continental (Berlin)}
         \renewcommand{\erwaehnteWerke}{Werke: Zwischenspiel. Komödie in drei Akten}
               \section[Paul Goldmann an Arthur Schnitzler, 20. 11. {[}1905{]}]{ Paul Goldmann an Arthur Schnitzler, 20. 11. {[}1905{]}}\nopagebreak\mylabel{v}\rehead{ }\begin{ledgroupsized}[t]{13cm}\normalsize\beginnumbering \toendnotes[C]{\smallbreak\pagebreak[2]} \Standort{DLA, A:Schnitzler, HS.NZ85.1.3175.}
\physDesc{Visitenkarte
\newline{}Handschrift: Bleistift, deutsche Kurrent\newline{}Versand: Stempel: »\nobreak{}\oindex{Hotel Continental (Berlin)@\textbf{Hotel Continental (Berlin)}|pwk}Continental Hotel Berlin, {[}N{]}ov 20, 1\textsubscript{56}AM\nobreak{}«.  
\newline{}Schnitzler: mit Bleistift das Datum »20/11 {[}1{]}90\textcolor{gray}{5}« vermerkt }\toendnotes[C]{\smallbreak}\pstart{}{\pb}Lieber Freund,\pend\pstart
           Ich habe heut nach verſchiedenen Richtungen vergeblich
               nach Dir telephonirt u. Dich jetzt ebenſo vergeblich im \label{K-L03234-1v}\edtext{Hotel\oindex{Hotel Continental (Berlin)@\textbf{Hotel Continental (Berlin)}|pw}}{\lemma{\textnormal{\emph{Hotel}}}\Cendnote{\textnormal{Am 20. 11. 1905 hatte Schnitzler\pwindex{Schnitzler, Arthur 15.05.1862 – 21.10.1931@\textsc{Schnitzler, Arthur} (15.05.1862 – 21.10.1931), \emph{Schriftsteller, Mediziner}|pwk}
                  einer Probe von \emph{Zwischenspiel}\pwindex{Schnitzler, Arthur 15.05.1862 – 21.10.1931@\textsc{Schnitzler, Arthur} (15.05.1862 – 21.10.1931), \emph{Schriftsteller, Mediziner}!Zwischenspiel. Komoedie in drei Akten1905-10-12@\strich\emph{Zwischenspiel. Komödie in drei Akten} {[}1905-10-12{]}|pwk} beigewohnt, den
                  Nachmittag und Abend hatte er mit Siegfried
                     Jacobsohn\pwindex{Jacobsohn, Siegfried 28.01.1881 – 03.12.1926@\textsc{Jacobsohn, Siegfried} (28.01.1881 – 03.12.1926), \emph{Journalist, Kritiker, Publizist}|pwk} verbracht. Siehe A. S.: \emph{Tagebuch}, 20. 11. 1905.}}}\label{K-L03234-1h} geſucht. Heut habe ich
               leider keine Zeit mehr. Wenn Du {\pb}aber morgen um 7 Uhr{ }\introOben{}abends\introOben{}{ }\label{K-L03234-2v}\edtext{bei mir vorbeikommen }{\lemma{\textnormal{\emph{bei mir vorbeikommen }}}\Cendnote{\textnormal{Schnitzler\pwindex{Schnitzler, Arthur 15.05.1862 – 21.10.1931@\textsc{Schnitzler, Arthur} (15.05.1862 – 21.10.1931), \emph{Schriftsteller, Mediziner}|pwk} traf Goldmann\pwindex{Goldmann, Paul 31.01.1865 – 25.09.1935@\textsc{Goldmann, Paul} (31.01.1865 – 25.09.1935), \emph{Schriftsteller, Journalist}|pwk} am 21. 11. 1905.}}}\label{K-L03234-2h} könnteſt, würde ich mich ſehr
               freuen, Dir die Hand zu drücken. Kannſt Du nicht kommen, ſo erbitte ich morgen{ }zwiſchen 6 u 7 Uhr abend\textcolor{gray}{s} telephoniſche
               Verſtändigung.\pend
           \pstart Herzlichen Gruß! \pend{}\pstart
           \centering{}\textcolor{gray}{\textbf{D\textsuperscript{r}
                  Paul Goldmann }}\pend
           \pstart
           \noindent{}\raggedleft{}\textcolor{gray}{\textbf{»Neue Freie Presse\orgindex{Neue Freie Presse@Neue Freie Presse|pw}.«}}\pend
           
         
         \endnumbering\mylabel{h}\end{ledgroupsized}  \newcommand{\dateiname}{L03234}\newcommand{\titel}{Paul Goldmann an Arthur Schnitzler, 20. 11. [1905]}\newcommand{\editorInnen}{Martin Anton Müller und Laura Untner}%% latex-leseansicht-abspann.tex
%% Abspann für die Leseansicht.
%% Der Schalter \ifkorrekturansicht ist bereits durch den Vorspann gesetzt.

%% latex-abspann.tex
%% Gemeinsamer Abspann für Korrekturansicht und Leseansicht.
%% Setzt den Schalter \ifkorrekturansicht voraus (gesetzt in den
%% einbindenden Dateien latex-korrekturansicht-abspann.tex bzw.
%% latex-leseansicht-abspann.tex).
%% ---------------------------------------------------------------

\normalsize

% Das esempio-Environment wird nur in der Leseansicht benötigt
\ifkorrekturansicht\else
\newenvironment{esempio}[3]%
{
    \vspace{1.5ex}
    \rlap{\underline{#1}}
    \par
    \setlength{\parindent}{0cm}
    \nopagebreak
    \leftskip=#2cm
    \rightskip=#3cm
}
{
    \par
}
\fi

\doendnotes{C}
\bigskip
\vfill

\clearpage

\footnotesize

\ifkorrekturansicht
  \lohead{\textsc{register}}
\fi

% theindex-Environment neu definieren ohne reledmac
\makeatletter
\renewenvironment{theindex}{%
  \ifkorrekturansicht
    \section*{\indexname}%
  \else
    \subsubsection*{Index der erwähnten Entitäten}%
  \fi
  \setlength{\parindent}{0pt}%
  \setlength{\parskip}{0pt plus 0.3pt}%
  \let\item\@idxitem
}{%
  \ifkorrekturansicht\clearpage\fi
}
\makeatother

\IfFileExists{\jobname-pw.ind}{\input{\jobname-pw.ind}}{}

% Quellenangabe nur in der Leseansicht
\ifkorrekturansicht\else
% Fallback-Definitionen, falls die .tex-Datei \titel etc. nicht gesetzt hat
\providecommand{\titel}{}
\providecommand{\editorInnen}{}
\providecommand{\dateiname}{\jobname}

\vspace{3cm}

\vfill

\footnotesize
\textsc{Quelle}: \titel. Herausgegeben von {\editorInnen}. In: \emph{Arthur Schnitzler: Briefwechsel mit Autorinnen und Autoren}.
 Digitale Edition, https://schnitzler-briefe.acdh.oeaw.ac.at/{\dateiname}.html (Stand \today)
\fi

\end{document}


      