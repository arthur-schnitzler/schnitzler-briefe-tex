%% latex-leseansicht-vorspann.tex
%% Vorspann für die Leseansicht.
%% Lädt die gemeinsame Datei latex-vorspann.tex mit nicht gesetztem Schalter.

\newif\ifkorrekturansicht
\korrekturansichtfalse

\input{../tex-inputs/latex-vorspann}


\section[Arthur Schnitzler an Robert Adam, 10. 4. 1928]{L02498 Arthur Schnitzler an Robert Adam, 10. 4. 1928}
\nopagebreak\mylabel{L02498v}
\rehead{ }\normalsize\beginnumbering\briefempfaengerindex{Adam, Robert@\textsc{Adam, Robert}!zzzSchnitzler, Arthur@\emph{von Arthur Schnitzler}!1928-04-101@{10. 4. 1928}|(be}
\toendnotes[C]{\smallbreak\pagebreak[2]}
\correspDesc{Versand  durch Arthur Schnitzler am 10. 4. 1928 in Wien
\newline{}Erhalt  durch Robert Adam im Zeitraum [10. 4. 1928
                  – 14. 4. 1928?] in Wien}\toendnotes[C]{\smallbreak}
\Standort{DLA, 96.34.2/30.}
\physDesc{Brief, 1 Blatt, 1 Seite, Kuvert, 1652 Zeichen
\newline{}Schreibmaschine
\newline{}Handschrift: Bleistift (\noindent{}Korrekturen, Unterschrift)}\Standort{DLA, A:Schnitzler, 85.1.1621.}
\physDesc{Brief, Durchschlag, 1 Blatt, 1 Seite, Kuvert, 1652 Zeichen
\newline{}Schreibmaschine
\newline{}Handschrift: Bleistift, lateinische Kurrent (\noindent{}Vermerk: »Pollak«, »Priv« und
                                    »Kritik«)}
\buchAbdrucke{\weitereDrucke{Arthur Schnitzler: \emph{Briefe 1913–1931}. Herausgegeben von Peter Michael Braunwarth, Richard Miklin, Susanne Pertlik und Heinrich Schnitzler. Frankfurt am Main: \emph{S. Fischer} 1984, S. 540–541.} }\toendnotes[C]{\smallbreak}\pstart{}{\pb}\textcolor{gray}{\textbf{D\textsuperscript{R}{ }ARTHUR SCHNITZLER}}\pend{}\pstart{}\textcolor{gray}{\textbf{WIEN, XVIII.}}\oindex{XVIII., Währing@\textbf{XVIII., Währing}, \emph{Verwaltungsgebiet}|pw}{ }\textcolor{gray}{\textbf{{ }STERNWARTESTRASSE 71}}\oindex{Wien@\textbf{Wien}!XVIII., Währing@\textbf{XVIII., Währing}!Sternwartestraße 71@\textbf{Sternwartestraße 71}, \emph{Wohngebäude}|pw}.\pend{}{\bigskip}\pstart{}{\pb}Herrn Hofrat Dr. Robert Adam Pollak,\pend{}\pstart{}Wien XII\oindex{XII., Meidling@\textbf{XII., Meidling}, \emph{Verwaltungsgebiet}|pw}.\pend{}\pstart{}Meidlinger Hauptstr. 56\oindex{Wien@\textbf{Wien}!XII., Meidling@\textbf{XII., Meidling}!Meidlinger Hauptstraße@\textbf{Meidlinger Hauptstraße}, \emph{Straße}|pw}.\pend{}{\bigskip}\vspace{1em}
\pstart
           
\pstart
           {\pb}\textcolor{gray}{\textbf{D\textsuperscript{R} ARTHUR SCHNITZLER}}\pend
           
\pstart
           \raggedleft{}10. 4. 1928.\pend
           \pend
           
\pstart
           \textcolor{gray}{\textbf{WIEN, XVIII. STERNWARTESTRASSE 71\oindex{Wien@\textbf{Wien}!XVIII., Währing@\textbf{XVIII., Währing}!Sternwartestraße 71@\textbf{Sternwartestraße 71}, \emph{Wohngebäude}|pw}}}.\pend
           
\pstart{}Verehrtester Herr Hofrat.\pend\vspace{0.5em}
\pstart
           Ihr neues Werk, die Märchenkomödie\pwindex{Adam, Robert 20.\,4.\,1877 Wien – 16.\,10.\,1961 Baden bei Wien@\textsc{Adam, Robert} (20.\,4.\,1877 Wien – 16.\,10.\,1961 Baden bei Wien), \emph{Schriftsteller, Richter}!Märchenkomödie@\strich\emph{Märchenkomödie}|pw}, habe ich mit
               vielem Interesse, aber doch mit einer manchmal absinkenden Teilnahme an den Vorgängen
               des Stücks gelesen. Ich konnte mich in den Stil nicht ganz hineinfinden; das
               politische und das poetische Element scheinen mir nicht durchaus zur Harmonie
               gediehen. Mit dem Wundervogel vermochte ich – ob ich ihn nun allegorisch, symbolisch
               oder phantastisch zu nehmen suchte – nichts Rechtes anzufangen, und eine letzte
               Klarheit, auf die man gerade nach der entschiedenen und vornehmen, geistigen Haltung
               Ihrer Komödie Anspruch zu erheben sich gedrungen fühlte, blieb am Ende doch aus. Im
               Einzelnen gibt es ja, wie selbstverständlich, manches Amüsante, viel Feines und auch
               allerlei Herbes (was mir besonders zusagte)\substVorne{}\textsuperscript{. D}\substDazwischen{}, d\substHinten{}ie Verse knitteln, insbesondere wo sie sich humoristisch gebärden, sehr
               gewandt an Ohr und Sinn vorbei. Sicher würde auch manche Szene auf der Bühne ihre
               Wirkung tun, – trotzdem kann ich begreifen, dass die Theater nach einem Drama\pwindex{Adam, Robert 20.\,4.\,1877 Wien – 16.\,10.\,1961 Baden bei Wien@\textsc{Adam, Robert} (20.\,4.\,1877 Wien – 16.\,10.\,1961 Baden bei Wien), \emph{Schriftsteller, Richter}!Märchenkomödie@\strich\emph{Märchenkomödie}|pwv}, das trotz \strikeout{seines} und vielleicht wegen seines beträchtlichen,
               nicht so sehr künstlerischen als geistigen Niveaus, einen äusseren Erfolg etwas
               unsicher erscheinen lässt, in diesen Zeiten nicht eben lüstern sind.\pend
           
\pstart
           Mich, verehrter Herr Hofrat, hat es jedesfalls sehr gefreut Ihnen endlich wieder,
               vorläufig auf eine so mittelbare Weise zu begegnen; – nach meiner Rückkehr von einer
               Reise, die ich übermorgen antrete, hoffe ich Sie auch persönlich wieder bei mir
               begrüssen zu dürfen.\pend
           
\pstart
           Ihr herzlich ergebener{\\[\baselineskip]}\spacefill\mbox{{[}hs.:{]} Arthur Schnitzler}\pend
           \leftskip=0em{}
\pstart
           \noindent{}{[}ms.:{]} Herrn Hofrat Dr. Adam Robert Pollak,{\\}Wien\oindex{Wien@\textbf{Wien}, \emph{Verwaltungsgebiet}|pw}.\pend
           \selectlanguage{ngerman}\endnumbering\briefempfaengerindex{Adam, Robert@\textsc{Adam, Robert}!zzzSchnitzler, Arthur@\emph{von Arthur Schnitzler}!1928-04-101@{10. 4. 1928}|)be}\mylabel{L02498h}  \newcommand{\dateiname}{L02498}\newcommand{\titel}{Arthur Schnitzler an Robert Adam, 10. 4. 1928}\newcommand{\editorInnen}{Martin Anton Müller und Gerd-Hermann Susen}%% latex-leseansicht-abspann.tex
%% Abspann für die Leseansicht.
%% Der Schalter \ifkorrekturansicht ist bereits durch den Vorspann gesetzt.

%% latex-abspann.tex
%% Gemeinsamer Abspann für Korrekturansicht und Leseansicht.
%% Setzt den Schalter \ifkorrekturansicht voraus (gesetzt in den
%% einbindenden Dateien latex-korrekturansicht-abspann.tex bzw.
%% latex-leseansicht-abspann.tex).
%% ---------------------------------------------------------------

\normalsize

% Das esempio-Environment wird nur in der Leseansicht benötigt
\ifkorrekturansicht\else
\newenvironment{esempio}[3]%
{
    \vspace{1.5ex}
    \rlap{\underline{#1}}
    \par
    \setlength{\parindent}{0cm}
    \nopagebreak
    \leftskip=#2cm
    \rightskip=#3cm
}
{
    \par
}
\fi

\doendnotes{C}
\bigskip
\vfill

\clearpage

\footnotesize

\ifkorrekturansicht
  \lohead{\textsc{register}}
\fi

% theindex-Environment neu definieren ohne reledmac
\makeatletter
\renewenvironment{theindex}{%
  \ifkorrekturansicht
    \section*{\indexname}%
  \else
    \subsubsection*{Index der erwähnten Entitäten}%
  \fi
  \setlength{\parindent}{0pt}%
  \setlength{\parskip}{0pt plus 0.3pt}%
  \let\item\@idxitem
}{%
  \ifkorrekturansicht\clearpage\fi
}
\makeatother

\IfFileExists{\jobname-pw.ind}{\input{\jobname-pw.ind}}{}

% Quellenangabe nur in der Leseansicht
\ifkorrekturansicht\else
% Fallback-Definitionen, falls die .tex-Datei \titel etc. nicht gesetzt hat
\providecommand{\titel}{}
\providecommand{\editorInnen}{}
\providecommand{\dateiname}{\jobname}

\vspace{3cm}

\vfill

\footnotesize
\textsc{Quelle}: \titel. Herausgegeben von {\editorInnen}. In: \emph{Arthur Schnitzler: Briefwechsel mit Autorinnen und Autoren}.
 Digitale Edition, https://schnitzler-briefe.acdh.oeaw.ac.at/{\dateiname}.html (Stand \today)
\fi

\end{document}


