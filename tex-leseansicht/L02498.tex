%% latex-korrekturansicht-vorspann.tex
%% Vorspann für die Korrekturansicht.
%% Lädt die gemeinsame Datei latex-vorspann.tex mit gesetztem Schalter.

\newif\ifkorrekturansicht
\korrekturansichttrue

\input{../tex-inputs/latex-vorspann}


\section[Arthur Schnitzler an Robert Adam, 10. 4. 1928]{L02498 Arthur Schnitzler an Robert Adam, 10. 4. 1928}
\nopagebreak\mylabel{L02498v}
\rehead{ }\normalsize\beginnumbering\briefempfaengerindex{Adam, Robert@\textsc{Adam, Robert}!zzzSchnitzler, Arthur@\emph{von Arthur Schnitzler}!1928-04-101@{10. 4. 1928}|(be}
\toendnotes[C]{\smallbreak\pagebreak[2]}\Standort{DLA, 96.34.2/30.}
\physDesc{Brief, 1 Blatt, 1 Seite, Umschlag, 1652 Zeichen
\newline{}Schreibmaschine
\newline{}Handschrift: Bleistift (\noindent{}Korrekturen, Unterschrift)}\Standort{DLA, A:Schnitzler, 85.1.1621.}
\physDesc{Brief, Durchschlag1 Blatt, 1 Seite, Umschlag, 1652 Zeichen
\newline{}Schreibmaschine
\newline{}Handschrift: Bleistift, lateinische Kurrent (\noindent{}Vermerk: »Pollak«, »Priv« und
                                    »Kritik«)}
\buchAbdrucke{\weitereDrucke{Arthur Schnitzler: \emph{Briefe 1913–1931}. Frankfurt am Main: \emph{S. Fischer} 1984, S. 540–541.} }\toendnotes[C]{\smallbreak}\pstart{}{\pb}\textcolor{gray}{\textbf{D\textsuperscript{R}{ }ARTHUR SCHNITZLER}}\pend{}\pstart{}\textcolor{gray}{\textbf{WIEN, XVIII.}}\oindex{XVIII., Waehring@\textbf{XVIII., Währing}, \emph{A.ADM3}|pw}{ }\textcolor{gray}{\textbf{{ }STERNWARTESTRASSE 71}}\oindex{Sternwartestrasse 71@\textbf{Sternwartestraße 71}, \emph{Wohngebäude (K.WHS)}|pw}.\pend{}{\bigskip}\pstart{}{\pb}Herrn Hofrat Dr. Robert Adam Pollak,\pend{}\pstart{}Wien XII\oindex{XII., Meidling@\textbf{XII., Meidling}, \emph{A.ADM3}|pw}.\pend{}\pstart{}Meidlinger Hauptstr. 56\oindex{Meidlinger Hauptstrasse@\textbf{Meidlinger Hauptstraße}, \emph{Straße (K.STR)}|pw}.\pend{}{\bigskip}\vspace{1em}
\pstart
           
\pstart
           {\pb}\textcolor{gray}{\textbf{D\textsuperscript{R} ARTHUR SCHNITZLER}}\pend
           
\pstart
           \raggedleft{}10. 4. 1928.\pend
           \pend
           
\pstart
           \textcolor{gray}{\textbf{WIEN, XVIII. STERNWARTESTRASSE 71\oindex{Sternwartestrasse 71@\textbf{Sternwartestraße 71}, \emph{Wohngebäude (K.WHS)}|pw}}}.\pend
           
\pstart{}Verehrtester Herr Hofrat.\pend\vspace{0.5em}
\pstart
           Ihr neues Werk, die Märchenkomödie\pwindex{Maerchenkomoedie@\emph{Märchenkomödie}|pw}, habe ich mit
               vielem Interesse, aber doch mit einer manchmal absinkenden Teilnahme an den Vorgängen
               des Stücks gelesen. Ich konnte mich in den Stil nicht ganz hineinfinden; das
               politische und das poetische Element scheinen mir nicht durchaus zur Harmonie
               gediehen. Mit dem Wundervogel vermochte ich – ob ich ihn nun allegorisch, symbolisch
               oder phantastisch zu nehmen suchte – nichts Rechtes anzufangen, und eine letzte
               Klarheit, auf die man gerade nach der entschiedenen und vornehmen, geistigen Haltung
               Ihrer Komödie Anspruch zu erheben sich gedrungen fühlte, blieb am Ende doch aus. Im
               Einzelnen gibt es ja, wie selbstverständlich, manches Amüsante, viel Feines und auch
               allerlei Herbes (was mir besonders zusagte)\substVorne{}\textsuperscript{. D}\substDazwischen{}, d\substHinten{}ie Verse knitteln, insbesondere wo sie sich humoristisch gebärden, sehr
               gewandt an Ohr und Sinn vorbei. Sicher würde auch manche Szene auf der Bühne ihre
               Wirkung tun, – trotzdem kann ich begreifen, dass die Theater nach einem Drama\pwindex{Maerchenkomoedie@\emph{Märchenkomödie}|pwv}, das trotz \strikeout{seines} und vielleicht wegen seines beträchtlichen,
               nicht so sehr künstlerischen als geistigen Niveaus, einen äusseren Erfolg etwas
               unsicher erscheinen lässt, in diesen Zeiten nicht eben lüstern sind.\pend
           
\pstart
           Mich, verehrter Herr Hofrat, hat es jedesfalls sehr gefreut Ihnen endlich wieder,
               vorläufig auf eine so mittelbare Weise zu begegnen; – nach meiner Rückkehr von einer
               Reise, die ich übermorgen antrete, hoffe ich Sie auch persönlich wieder bei mir
               begrüssen zu dürfen.\pend
           
\pstart
           Ihr herzlich ergebener{\\[\baselineskip]}\spacefill\mbox{{[}hs.:{]} Arthur Schnitzler}\pend
           \leftskip=0em{}
\pstart
           \noindent{}{[}ms.:{]} Herrn Hofrat Dr. Adam Robert Pollak,{\\}Wien\oindex{Wien@\textbf{Wien}, \emph{A.ADM2}|pw}.\pend
           \selectlanguage{ngerman}\endnumbering\briefempfaengerindex{Adam, Robert@\textsc{Adam, Robert}!zzzSchnitzler, Arthur@\emph{von Arthur Schnitzler}!1928-04-101@{10. 4. 1928}|)be}\mylabel{L02498h}  \normalsize

\doendnotes{C}
\bigskip
\vfill

\clearpage

\footnotesize

\lohead{\textsc{register}}

% Definiere theindex-Environment komplett neu ohne reledmac
\makeatletter
\renewenvironment{theindex}{%
  \section*{\indexname}%
  \setlength{\parindent}{0pt}%
  \setlength{\parskip}{0pt plus 0.3pt}%
  \let\item\@idxitem
}{%
  \clearpage
}
\makeatother

\IfFileExists{\jobname-pw.ind}{\input{\jobname-pw.ind}}{}

\end{document}

      