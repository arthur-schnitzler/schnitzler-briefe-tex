%% latex-leseansicht-vorspann.tex
%% Vorspann für die Leseansicht.
%% Lädt die gemeinsame Datei latex-vorspann.tex mit nicht gesetztem Schalter.

\newif\ifkorrekturansicht
\korrekturansichtfalse

\input{../tex-inputs/latex-vorspann}


         
         \newcommand{\erwaehntePersonen}{Personen: Olga Schnitzler}
         \newcommand{\erwaehnteOrte}{Orte: Aigen, Edmund-Weiß-Gasse, München, Salzburg, Wien, XVIII., Währing, Zell am Ziller}
         \newcommand{\erwaehnteWerke}{
               \section[Hermann Bahr und Anna Bahr-Mildenburg an Arthur Schnitzler, 25. 8. 1909]{ Hermann Bahr und Anna Bahr-Mildenburg an Arthur Schnitzler,
               25. 8. 1909}\nopagebreak\mylabel{v}\rehead{ }\begin{ledgroupsized}[t]{13cm}\normalsize\beginnumbering \toendnotes[C]{\smallbreak\pagebreak[2]} \Standort{CUL, Schnitzler, B 5b.}
\physDesc{Bildpostkarte
\newline{}Handschrift Hermann Bahr: 1) Bleistift, deutsche Kurrent\hspace{1em}2) Bleistift, lateinische Kurrent (\noindent{}Adresse)\hspace{1em}\newline{}Handschrift Anna Bahr-Mildenburg: schwarze Tinte, lateinische Kurrent\newline{}Versand: 1) Stempel: »\nobreak{}\oindex{Zell am Ziller@\textbf{Zell am Ziller}|pwk}Zell am Ziller\nobreak{}«.   2) als Beilage zu einem Brief von Olga\pwindex{Schnitzler, Olga 17.01.1882 – 13.01.1970@\textsc{Schnitzler, Olga} (17.01.1882 – 13.01.1970), \emph{Schauspielerin, Sängerin}|pw} an Schnitzler übermittelt. Mit Tinte von ihrer Hand
                                 ergänzt: »\textsc{Bevor ich den Brief schliesse, kommt diese
                                       Karte!}«
\newline{}Schnitzler: mit Bleistift ergänzt »Bahr« \newline{}Ordnung: mit Bleistift von unbekannter Hand nummeriert:
                                    »159« }\buchAbdrucke{\weitereDrucke{Hermann Bahr, Arthur Schnitzler: \emph{Briefwechsel, Aufzeichnungen, Dokumente (1891–1931)}. Hg. Kurt Ifkovits und Martin Anton Müller. Göttingen: \emph{Wallstein} 2018, S. 423.} }\toendnotes[C]{\smallbreak}\pstart{}{\pb}D\textsuperscript{r} Arthur
                  Schnitzler\pend{}\pstart{}Wien XVIII\oindex{XVIII., Waehring@\textbf{XVIII., Währing}|pw}\pend{}\pstart{}Spöttelgasse 7\oindex{Edmund-Weiss-Gasse@\textbf{Edmund-Weiß-Gasse}|pw}\pend{}{\bigskip}\pstart
           \noindent{}\centering{}\textcolor{gray}{\textbf{{\pb}Gruss von Zell im Zillerthal!\oindex{Zell am Ziller@\textbf{Zell am Ziller}|pw}}}\pend
           \pstart
           \centering{}{\pb}25. 8. 09\pend
           \pstart
           Dich und Deine liebe Frau\pwindex{Schnitzler, Olga 17.01.1882 – 13.01.1970@\textsc{Schnitzler, Olga} (17.01.1882 – 13.01.1970), \emph{Schauspielerin, Sängerin}|pwv}
               grüßen herzlichſt\pend
           \pstart \spacefill\mbox{Hermann und {[}hs. Bahr-Mildenburg:{]} Anna \label{K_L01868_1v}\edtext{Bahr-Mildenburg}{\lemma{\textnormal{\emph{Bahr-Mildenburg}}}\Cendnote{\textnormal{Die Hochzeit, nach der sie den Doppelnamen führte, hatte am 22. 8. 1909 in Aigen\oindex{Aigen@\textbf{Aigen}|pwk} (heute: Stadtteil von Salzburg\oindex{Salzburg@\textbf{Salzburg}|pwk}) stattgefunden.}}}\label{K_L01868_1h}}\pend{}
         
         \endnumbering\mylabel{h}\end{ledgroupsized}  \newcommand{\dateiname}{L01868}\newcommand{\titel}{Hermann Bahr und Anna Bahr-Mildenburg an Arthur Schnitzler, 25. 8. 1909}\newcommand{\editorInnen}{ Kurt Ifkovits,  Martin Anton Müller}%% latex-leseansicht-abspann.tex
%% Abspann für die Leseansicht.
%% Der Schalter \ifkorrekturansicht ist bereits durch den Vorspann gesetzt.

%% latex-abspann.tex
%% Gemeinsamer Abspann für Korrekturansicht und Leseansicht.
%% Setzt den Schalter \ifkorrekturansicht voraus (gesetzt in den
%% einbindenden Dateien latex-korrekturansicht-abspann.tex bzw.
%% latex-leseansicht-abspann.tex).
%% ---------------------------------------------------------------

\normalsize

% Das esempio-Environment wird nur in der Leseansicht benötigt
\ifkorrekturansicht\else
\newenvironment{esempio}[3]%
{
    \vspace{1.5ex}
    \rlap{\underline{#1}}
    \par
    \setlength{\parindent}{0cm}
    \nopagebreak
    \leftskip=#2cm
    \rightskip=#3cm
}
{
    \par
}
\fi

\doendnotes{C}
\bigskip
\vfill

\clearpage

\footnotesize

\ifkorrekturansicht
  \lohead{\textsc{register}}
\fi

% theindex-Environment neu definieren ohne reledmac
\makeatletter
\renewenvironment{theindex}{%
  \ifkorrekturansicht
    \section*{\indexname}%
  \else
    \subsubsection*{Index der erwähnten Entitäten}%
  \fi
  \setlength{\parindent}{0pt}%
  \setlength{\parskip}{0pt plus 0.3pt}%
  \let\item\@idxitem
}{%
  \ifkorrekturansicht\clearpage\fi
}
\makeatother

\IfFileExists{\jobname-pw.ind}{\input{\jobname-pw.ind}}{}

% Quellenangabe nur in der Leseansicht
\ifkorrekturansicht\else
% Fallback-Definitionen, falls die .tex-Datei \titel etc. nicht gesetzt hat
\providecommand{\titel}{}
\providecommand{\editorInnen}{}
\providecommand{\dateiname}{\jobname}

\vspace{3cm}

\vfill

\footnotesize
\textsc{Quelle}: \titel. Herausgegeben von {\editorInnen}. In: \emph{Arthur Schnitzler: Briefwechsel mit Autorinnen und Autoren}.
 Digitale Edition, https://schnitzler-briefe.acdh.oeaw.ac.at/{\dateiname}.html (Stand \today)
\fi

\end{document}


      