%% latex-korrekturansicht-vorspann.tex
%% Vorspann für die Korrekturansicht.
%% Lädt die gemeinsame Datei latex-vorspann.tex mit gesetztem Schalter.

\newif\ifkorrekturansicht
\korrekturansichttrue

\input{../tex-inputs/latex-vorspann}


\section[Hermann Bahr und Anna Bahr-Mildenburg an Arthur Schnitzler, 25. 8. 1909]{L01868 Hermann Bahr und Anna Bahr-Mildenburg an Arthur Schnitzler,
               25. 8. 1909}
\nopagebreak\mylabel{L01868v}
\rehead{ }\normalsize\beginnumbering\briefempfaengerindex{Schnitzler, Arthur@\textsc{Schnitzler, Arthur}!zzzBahr-Mildenburg, Anna@\emph{von Anna Bahr-Mildenburg}!1909-08-251@{25. 8. 1909}|(be}\briefempfaengerindex{Schnitzler, Arthur@\textsc{Schnitzler, Arthur}!zzzBahr, Hermann@\emph{von Hermann Bahr}!1909-08-251@{25. 8. 1909}|(be}
\toendnotes[C]{\smallbreak\pagebreak[2]}\Standort{CUL, Schnitzler, B 5b.}
\physDesc{Bildpostkarte, 123 Zeichen
\newline{}Handschrift Hermann Bahr: 1) Bleistift, deutsche Kurrent\hspace{1em}2) Bleistift, lateinische Kurrent (\noindent{}Adresse)\hspace{1em}
\newline{}Handschrift Anna Bahr-Mildenburg: schwarze Tinte, lateinische Kurrent
\newline{}Versand: 1) Stempel: »\nobreak{}\oindex{Zell am Ziller@\textbf{Zell am Ziller}, \emph{A.ADM3}|pwk}Zell am Ziller\nobreak{}«.   2) als Beilage zu einem Brief von Olga\pwindex{Schnitzler, Olga 17.01.1882 – 13.01.1970@\textsc{Schnitzler, Olga} (17.01.1882 – 13.01.1970), \emph{Schauspieler/Schauspielerin, Sänger/Sängerin}|pw} an Schnitzler übermittelt. Mit Tinte von ihrer Hand
                                 ergänzt: »\textsc{Bevor ich den Brief schliesse, kommt diese
                                       Karte!}«
\newline{}Schnitzler: mit Bleistift ergänzt »Bahr« 
\newline{}Ordnung: mit Bleistift von unbekannter Hand nummeriert:
                                    »159« }
\buchAbdrucke{\weitereDrucke{Hermann Bahr, Arthur Schnitzler: \emph{Briefwechsel, Aufzeichnungen, Dokumente (1891–1931)}. Göttingen: \emph{Wallstein} 2018, S. 423.} }\toendnotes[C]{\smallbreak}\pstart{}{\pb}D\textsuperscript{r} Arthur
                  Schnitzler\pend{}\pstart{}Wien XVIII\oindex{XVIII., Waehring@\textbf{XVIII., Währing}, \emph{A.ADM3}|pw}\pend{}\pstart{}Spöttelgasse 7\oindex{Edmund-Weiss-Gasse 7@\textbf{Edmund-Weiß-Gasse 7}, \emph{Wohngebäude (K.WHS)}|pw}\pend{}{\bigskip}
\pstart
           \noindent{}\centering{}{\pb}\textcolor{gray}{\textbf{Gruss von Zell im Zillerthal!\oindex{Zell am Ziller@\textbf{Zell am Ziller}, \emph{A.ADM3}|pw}}}\pend
           \vspace{1em}
\pstart
           \centering{}{\pb}25. 8. 09\pend
           \vspace{0.5em}
\pstart
           Dich und Deine liebe Frau\pwindex{Schnitzler, Olga 17.01.1882 – 13.01.1970@\textsc{Schnitzler, Olga} (17.01.1882 – 13.01.1970), \emph{Schauspieler/Schauspielerin, Sänger/Sängerin}|pwv}
               grüßen herzlichſt\pend
           \pstart \spacefill\mbox{Hermann und {[}hs. :{]} Anna \label{K_L01868-1v}\edtext{Bahr-Mildenburg}{\lemma{\textnormal{\emph{Bahr-Mildenburg}}}\Cendnote{\textnormal{Die Hochzeit, nach der sie den Doppelnamen führte, hatte am 22. 8. 1909 in Aigen\oindex{Aigen@\textbf{Aigen}, \emph{P.PPLX}|pwk} (heute: Stadtteil von Salzburg\oindex{Salzburg@\textbf{Salzburg}, \emph{A.ADM2}|pwk}) stattgefunden.}}}\label{K_L01868-1}}\pend{}\selectlanguage{ngerman}\endnumbering\briefempfaengerindex{Schnitzler, Arthur@\textsc{Schnitzler, Arthur}!zzzBahr-Mildenburg, Anna@\emph{von Anna Bahr-Mildenburg}!1909-08-251@{25. 8. 1909}|)be}\briefempfaengerindex{Schnitzler, Arthur@\textsc{Schnitzler, Arthur}!zzzBahr, Hermann@\emph{von Hermann Bahr}!1909-08-251@{25. 8. 1909}|)be}\mylabel{L01868h}  \normalsize

\doendnotes{C}
\bigskip
\vfill

\clearpage

\footnotesize

\lohead{\textsc{register}}

% Definiere theindex-Environment komplett neu ohne reledmac
\makeatletter
\renewenvironment{theindex}{%
  \section*{\indexname}%
  \setlength{\parindent}{0pt}%
  \setlength{\parskip}{0pt plus 0.3pt}%
  \let\item\@idxitem
}{%
  \clearpage
}
\makeatother

\IfFileExists{\jobname-pw.ind}{\input{\jobname-pw.ind}}{}

\end{document}

      