%% latex-korrekturansicht-vorspann.tex
%% Vorspann für die Korrekturansicht.
%% Lädt die gemeinsame Datei latex-vorspann.tex mit gesetztem Schalter.

\newif\ifkorrekturansicht
\korrekturansichttrue

\input{../tex-inputs/latex-vorspann}


\section[Hugo von Hofmannsthal an Arthur Schnitzler, {[}15. 11. 1912{]}]{L02098 Hugo von Hofmannsthal an Arthur Schnitzler, {[}15. 11. 1912{]}}
\nopagebreak\mylabel{L02098v}
\rehead{ }\normalsize\beginnumbering\briefempfaengerindex{Schnitzler, Arthur@\textsc{Schnitzler, Arthur}!zzzHofmannsthal, Hugo von@\emph{von Hugo von Hofmannsthal}!1912-11-151@{{[}15. 11. 1912{]}}|(be}
\toendnotes[C]{\smallbreak\pagebreak[2]}\Standort{CUL, Schnitzler, B 43.}
\physDesc{Brief, 1 Blatt, 2 Seiten, 290 Zeichen
\newline{}Handschrift: schwarze Tinte, deutsche Kurrent
\newline{}Schnitzler: mit Bleistift datiert: »15/12 912« und beschriftet: »\textsc{Hugo}« 
\newline{}Ordnung: 1) mit Bleistift von unbekannter Hand nummeriert: »\strikeout{322}«  2) mit Bleistift von unbekannter Hand nummeriert:
                                    »345«}
\buchAbdrucke{\weitereDrucke{Hugo von Hofmannsthal, Arthur Schnitzler: \emph{Briefwechsel}. Frankfurt am Main: \emph{S. Fischer} 1964, S. 279.} }\toendnotes[C]{\smallbreak}
\pstart{}{\pb}mein guter Arthur \pend\vspace{0.5em}
\pstart
           meine Zeilen über das Hauptmann\pwindex{Hauptmann, Gerhart 15.11.1862 – 06.06.1946@\textsc{Hauptmann, Gerhart} (15.11.1862 – 06.06.1946), \emph{Schriftsteller/Schriftstellerin}|pw}banquett sind
               ganz gegenſtandslos. Man hat mir in dieſer Sache eine echt \uline{wien\oindex{Wien@\textbf{Wien}, \emph{A.ADM2}|pw}er}{ }\label{K_L02098-1v}\edtext{Ungezogenheit}{\lemma{\textnormal{\emph{Ungezogenheit}}}\Cendnote{\textnormal{Vgl. A. S.: \emph{Tagebuch}, 15. 11. 1912.
               }}}\label{K_L02098-1} gemacht und ich gehe ſelber nicht hin. Zu Hauptmanns\pwindex{Hauptmann, Gerhart 15.11.1862 – 06.06.1946@\textsc{Hauptmann, Gerhart} (15.11.1862 – 06.06.1946), \emph{Schriftsteller/Schriftstellerin}|pw}{ }\label{K_L02098-2v}\edtext{Vortrag}{\lemma{\textnormal{\emph{Vortrag}}}\Cendnote{\textnormal{Am 17. 11. 1912, vor dem Bankett; Schnitzler dürfte nur zum Bankett gegangen sein.}}}\label{K_L02098-2} gehe {\pb}ich aber, da ich ihn ſehr gern
               habe.\pend
           
\pstart
           Hoffentlich ſieht man ſich jetzt.\pend
           
\pstart
           Von Herzen Ihr{\\[\baselineskip]}\spacefill\mbox{Hugo.}\pend
           \leftskip=0em{}\selectlanguage{ngerman}\endnumbering\briefempfaengerindex{Schnitzler, Arthur@\textsc{Schnitzler, Arthur}!zzzHofmannsthal, Hugo von@\emph{von Hugo von Hofmannsthal}!1912-11-151@{{[}15. 11. 1912{]}}|)be}\mylabel{L02098h}  \normalsize

\doendnotes{C}
\bigskip
\vfill

\clearpage

\footnotesize

\lohead{\textsc{register}}

% Definiere theindex-Environment komplett neu ohne reledmac
\makeatletter
\renewenvironment{theindex}{%
  \section*{\indexname}%
  \setlength{\parindent}{0pt}%
  \setlength{\parskip}{0pt plus 0.3pt}%
  \let\item\@idxitem
}{%
  \clearpage
}
\makeatother

\IfFileExists{\jobname-pw.ind}{\input{\jobname-pw.ind}}{}

\end{document}

      