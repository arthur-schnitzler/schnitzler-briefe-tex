%% latex-korrekturansicht-vorspann.tex
%% Vorspann für die Korrekturansicht.
%% Lädt die gemeinsame Datei latex-vorspann.tex mit gesetztem Schalter.

\newif\ifkorrekturansicht
\korrekturansichttrue

\input{../tex-inputs/latex-vorspann}


\section[Robert Adam an Arthur Schnitzler, 7. 5. 1913]{L02136 Robert Adam an Arthur Schnitzler, 7. 5. 1913}
\nopagebreak\mylabel{L02136v}
\rehead{ }\normalsize\beginnumbering\briefempfaengerindex{Schnitzler, Arthur@\textsc{Schnitzler, Arthur}!zzzAdam, Robert@\emph{von Robert Adam}!1913-05-071@{7. 5. 1913}|(be}
\toendnotes[C]{\smallbreak\pagebreak[2]}\Standort{DLA, A:Schnitzler, HS.NZ85.1.4230,7.}
\physDesc{Brief, 1 Blatt, 2 Seiten, 701 Zeichen
\newline{}Handschrift: schwarze Tinte, deutsche Kurrent
\newline{}Schnitzler: mit Bleistift beschriftet: »\textsc{Adam}« }\Standort{Wien, Österreichische Nationalbibliothek, Cod.ser. 52.266, 166.}
\physDesc{handschriftliche Abschrift2 Blätter, 2 Seiten, 701 Zeichen
\newline{}Handschrift: schwarze Tinte, Gabelsberger Kurzschrift}\Standort{Wien, Österreichische Nationalbibliothek, Cod.ser. 52.266, 166.}
\physDesc{maschinenschriftliche Abschrift1 Blatt, 1 Seite, 701 Zeichen
\newline{}Schreibmaschine}\toendnotes[C]{\smallbreak}
\pstart
           \raggedleft{}{\pb}Ziſtersdorf\oindex{Zistersdorf@\textbf{Zistersdorf}, \emph{A.ADM3}|pw}, am 7. Mai 1913. \pend
           
\pstart{}Hochverehrter Herr Doktor!\pend\vspace{0.5em}
\pstart
           Nehmen Sie meinen herzlichen Dank für die freundlichen Zeilen, welche die Rückſendung
               des Manuſkripts\pwindex{Fatme@\emph{Fatme}|pwv}
               begleiteten.\pend
           
\pstart
           Trotz ihrer kann ich die Befürchtung nicht abwehren, daß meine krauſe und, wie ich
               einſehe, mißratene Studie Ihren Beifall nicht gefunden habe. Ich begreife ſehr gut,
               daß ſie Ihren Künſtlerſinn, deſſen wunderbare Reife ich zuletzt in der Frau Beate\pwindex{Frau Beate und ihr Sohn. Novelle@\emph{Frau Beate und ihr Sohn. Novelle}|pw} bewundern durfte, geradezu beleidigt
               haben muß.\pend
           
\pstart
           Vielleicht iſt es mir noch vergönnt, künftighin wieder einmal mit einem
               ausgeglichenen Produkt vor Sie hinzutreten.\pend
           
\pstart
           Genehmigen Sie, hochverehrter Herr {\pb}Doktor, den
               Ausdruck meiner unbegrenzten Verehrung und meines Dankes!\pend
           
\pstart
           Ihr ergebener{\\[\baselineskip]}\spacefill\mbox{Robert Adam}\pend
           \leftskip=0em{}\selectlanguage{ngerman}\endnumbering\briefempfaengerindex{Schnitzler, Arthur@\textsc{Schnitzler, Arthur}!zzzAdam, Robert@\emph{von Robert Adam}!1913-05-071@{7. 5. 1913}|)be}\mylabel{L02136h}  \normalsize

\doendnotes{C}
\bigskip
\vfill

\clearpage

\footnotesize

\lohead{\textsc{register}}

% Definiere theindex-Environment komplett neu ohne reledmac
\makeatletter
\renewenvironment{theindex}{%
  \section*{\indexname}%
  \setlength{\parindent}{0pt}%
  \setlength{\parskip}{0pt plus 0.3pt}%
  \let\item\@idxitem
}{%
  \clearpage
}
\makeatother

\IfFileExists{\jobname-pw.ind}{\input{\jobname-pw.ind}}{}

\end{document}

      