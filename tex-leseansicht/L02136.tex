%% latex-leseansicht-vorspann.tex
%% Vorspann für die Leseansicht.
%% Lädt die gemeinsame Datei latex-vorspann.tex mit nicht gesetztem Schalter.

\newif\ifkorrekturansicht
\korrekturansichtfalse

\input{../tex-inputs/latex-vorspann}


         
         \renewcommand{\erwaehntePersonen}{Personen: Robert Adam}
         \renewcommand{\erwaehnteOrte}{Orte: Wien, Zistersdorf}
         \renewcommand{\erwaehnteWerke}{Werke: Fatme, Frau Beate und ihr Sohn. Novelle}
               \section[Robert Adam an Arthur Schnitzler, 7. 5. 1913]{ Robert Adam an Arthur Schnitzler, 7. 5. 1913}\nopagebreak\mylabel{v}\rehead{ }\begin{ledgroupsized}[t]{13cm}\normalsize\beginnumbering\briefempfaengerindex{Schnitzler, Arthur@\textsc{Schnitzler, Arthur}!zzzAdam, Robert@\emph{von Robert Adam}!1913-05-071@{7. 5. 1913}|(be} \toendnotes[C]{\smallbreak\pagebreak[2]} \Standort{DLA, A:Schnitzler, HS.NZ85.1.4230,7.}
\physDesc{Brief, 1 Blatt, 2 Seiten, 701 Zeichen
\newline{}Handschrift: schwarze Tinte, deutsche Kurrent
\newline{}Schnitzler: mit Bleistift beschriftet: »\textsc{Adam}« }\Standort{Wien, Österreichische Nationalbibliothek, Cod.ser. 52.266, 166.}
\physDesc{handschriftliche Abschrift, 2 Blätter, 2 Seiten
\newline{}Handschrift: schwarze Tinte, Gabelsberger Kurzschrift}\Standort{Wien, Österreichische Nationalbibliothek, Cod.ser. 52.266, 166.}
\physDesc{maschinenschriftliche Abschrift, 1 Blatt, 1 Seite
\newline{}Schreibmaschine}\toendnotes[C]{\smallbreak}\pstart
           \raggedleft{}{\pb}Ziſtersdorf\oindex{Zistersdorf@\textbf{Zistersdorf}|pw}, am 7. Mai 1913. \pend
           \pstart{}Hochverehrter Herr Doktor!\pend\pstart
           Nehmen Sie meinen herzlichen Dank für die freundlichen Zeilen, welche die Rückſendung
               des Manuſkripts\pwindex{Adam, Robert 20.04.1877 – 16.10.1961@\textsc{Adam, Robert} (20.04.1877 – 16.10.1961), \emph{Schriftsteller, Richter}!FatmeNone@\strich\emph{Fatme} {[}None{]}|pwv}
               begleiteten.\pend
           \pstart
           Trotz ihrer kann ich die Befürchtung nicht abwehren, daß meine krauſe und, wie ich
               einſehe, mißratene Studie Ihren Beifall nicht gefunden habe. Ich begreife ſehr gut,
               daß ſie Ihren Künſtlerſinn, deſſen wunderbare Reife ich zuletzt in der Frau Beate\pwindex{Schnitzler, Arthur 15.05.1862 – 21.10.1931@\textsc{Schnitzler, Arthur} (15.05.1862 – 21.10.1931), \emph{Schriftsteller, Mediziner}!Frau Beate und ihr Sohn. Novelle1.2.1913 – 1.4.1913@\strich\emph{Frau Beate und ihr Sohn. Novelle} {[}1.2.1913 – 1.4.1913{]}|pw} bewundern durfte, geradezu beleidigt
               haben muß.\pend
           \pstart
           Vielleicht iſt es mir noch vergönnt, künftighin wieder einmal mit einem
               ausgeglichenen Produkt vor Sie hinzutreten.\pend
           \pstart
           Genehmigen Sie, hochverehrter Herr {\pb}Doktor, den
               Ausdruck meiner unbegrenzten Verehrung und meines Dankes!\pend
           \pstart
           Ihr ergebener{\\[\baselineskip]}\spacefill\mbox{Robert Adam}\pend
           \leftskip=0em{}
         
         \endnumbering\mylabel{h}\end{ledgroupsized}  \newcommand{\dateiname}{L02136}\newcommand{\titel}{Robert Adam an Arthur Schnitzler, 7. 5. 1913}\newcommand{\editorInnen}{Martin Anton Müller und Gerd-Hermann Susen}%% latex-leseansicht-abspann.tex
%% Abspann für die Leseansicht.
%% Der Schalter \ifkorrekturansicht ist bereits durch den Vorspann gesetzt.

%% latex-abspann.tex
%% Gemeinsamer Abspann für Korrekturansicht und Leseansicht.
%% Setzt den Schalter \ifkorrekturansicht voraus (gesetzt in den
%% einbindenden Dateien latex-korrekturansicht-abspann.tex bzw.
%% latex-leseansicht-abspann.tex).
%% ---------------------------------------------------------------

\normalsize

% Das esempio-Environment wird nur in der Leseansicht benötigt
\ifkorrekturansicht\else
\newenvironment{esempio}[3]%
{
    \vspace{1.5ex}
    \rlap{\underline{#1}}
    \par
    \setlength{\parindent}{0cm}
    \nopagebreak
    \leftskip=#2cm
    \rightskip=#3cm
}
{
    \par
}
\fi

\doendnotes{C}
\bigskip
\vfill

\clearpage

\footnotesize

\ifkorrekturansicht
  \lohead{\textsc{register}}
\fi

% theindex-Environment neu definieren ohne reledmac
\makeatletter
\renewenvironment{theindex}{%
  \ifkorrekturansicht
    \section*{\indexname}%
  \else
    \subsubsection*{Index der erwähnten Entitäten}%
  \fi
  \setlength{\parindent}{0pt}%
  \setlength{\parskip}{0pt plus 0.3pt}%
  \let\item\@idxitem
}{%
  \ifkorrekturansicht\clearpage\fi
}
\makeatother

\IfFileExists{\jobname-pw.ind}{\input{\jobname-pw.ind}}{}

% Quellenangabe nur in der Leseansicht
\ifkorrekturansicht\else
% Fallback-Definitionen, falls die .tex-Datei \titel etc. nicht gesetzt hat
\providecommand{\titel}{}
\providecommand{\editorInnen}{}
\providecommand{\dateiname}{\jobname}

\vspace{3cm}

\vfill

\footnotesize
\textsc{Quelle}: \titel. Herausgegeben von {\editorInnen}. In: \emph{Arthur Schnitzler: Briefwechsel mit Autorinnen und Autoren}.
 Digitale Edition, https://schnitzler-briefe.acdh.oeaw.ac.at/{\dateiname}.html (Stand \today)
\fi

\end{document}


      