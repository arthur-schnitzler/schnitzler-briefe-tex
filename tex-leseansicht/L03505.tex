%% latex-leseansicht-vorspann.tex
%% Vorspann für die Leseansicht.
%% Lädt die gemeinsame Datei latex-vorspann.tex mit nicht gesetztem Schalter.

\newif\ifkorrekturansicht
\korrekturansichtfalse

\input{../tex-inputs/latex-vorspann}


\section[ Felix Salten an Arthur Schnitzler, 31. 7. 1909]{L03505 Felix Salten an Arthur Schnitzler,  31. 7. 1909}
\nopagebreak\mylabel{L03505v}
\rehead{ }\normalsize\beginnumbering\briefempfaengerindex{Schnitzler, Arthur@\textsc{Schnitzler, Arthur}!zzzSalten, Felix@\emph{von Felix Salten}!1909-07-312@{31. 7. 1909}|(be}
\toendnotes[C]{\smallbreak\pagebreak[2]}
\correspDesc{Versand  durch Felix Salten am 31. 7. 1909 in Höhlenstein
\newline{}Erhalt  durch Arthur Schnitzler im Zeitraum [1. 8. 1909
                  – 5. 8. 1909?] in Edlach}\toendnotes[C]{\smallbreak}
\Standort{CUL, Schnitzler, B 89, B 1.}
\physDesc{Bildpostkarte, 398 Zeichen
\newline{}Handschrift: schwarze Tinte, lateinische Kurrent
\newline{}Versand: Stempel: »\nobreak{}\oindex{Höhlenstein@\textbf{Höhlenstein}|pwk}Höhlenstein\nobreak{}«.  
\newline{}Schnitzler: mit Bleistift Vermerk: »\textsc{Salten}« 
\newline{}Ordnung: mit Bleistift von unbekannter Hand nummeriert: »255« }\toendnotes[C]{\smallbreak}\pstart{}{\pb}Herrn D\textsuperscript{r} Arthur Schnitzler\pend{}\pstart{}Edlach \textsuperscript{b}/Reichenau\oindex{Edlach@\textbf{Edlach}|pw}\pend{}\pstart{}Nied. Öst.\oindex{Niederösterreich@\textbf{Niederösterreich}, \emph{Land}|pw}\pend{}\pstart{}Südbahn\orgindex{Südbahnstrecke@Südbahnstrecke|pw}\pend{}{\bigskip}
\pstart
           \noindent{}\centering{}{\pb}{[}Fotografie von Paul\pwindex{Salten, Paul 11.\,8.\,1903 Wien – 8.\,5.\,1937 ebd.@\textsc{Salten, Paul} (11.\,8.\,1903 Wien – 8.\,5.\,1937 ebd.), \emph{Filmcutter}|pw} und Anna Salten\pwindex{Rehmann, Anna Katharina 18.\,8.\,1904 Wien – 27.\,3.\,1977 Zürich@\textsc{Rehmann, Anna Katharina} (18.\,8.\,1904 Wien – 27.\,3.\,1977 Zürich), \emph{Schauspielerin, Übersetzerin}|pw} in
                  Tracht{]}\pend
           \vspace{1em}
\pstart
           \noindent{}{\pb}Lieber, wir
               sind etwa \label{K_L03505-1v}\edtext{gegen den 10. Aug.}{\lemma{\textnormal{\emph{gegen den 10. Aug.}}}\Cendnote{\textnormal{Die nächste im \emph{Tagebuch}\pwindex{Schnitzler, Arthur 15.\,5.\,1862 Wien – 21.\,10.\,1931 ebd.@\textsc{Schnitzler, Arthur} (15.\,5.\,1862 Wien – 21.\,10.\,1931 ebd.), \emph{Schriftsteller, Mediziner}!Tagebuch@\strich\emph{Tagebuch}|pwk} belegte
                  Begegnung – gemeinsames Tennisspiel – fand am 6. 9. 1909 statt.}}}\label{K_L03505-1} schon in Wien\oindex{Wien@\textbf{Wien}, \emph{Verwaltungsgebiet}|pw}. Vielleicht
               komme ich – wenns Ihnen recht ist – einmal zum Tennis hinaus, ehe wir nach Kaltenleutgeben\oindex{Kaltenleutgeben@\textbf{Kaltenleutgeben}, \emph{Hauptstadt}|pw} gehen. – Das »umseitige« Bild
               ist mit dem neuen Apparat von mir gemacht. Viele herzliche Grüße von uns\pwindex{Salten, Ottilie 7.\,3.\,1868 Prag – 22.\,6.\,1942 Zürich@\textsc{Salten, Ottilie} (7.\,3.\,1868 Prag – 22.\,6.\,1942 Zürich), \emph{Schauspielerin}|pwv} zu Ihnen
                  allen, \substVorne{}\textsuperscript{und}\substDazwischen{}Ihr\substHinten{}{ }\spacefill\mbox{Salten}\pend
           
\pstart
           Landro\oindex{Höhlenstein@\textbf{Höhlenstein}|pw}, 31. VII. 09\pend
           \selectlanguage{ngerman}\endnumbering\briefempfaengerindex{Schnitzler, Arthur@\textsc{Schnitzler, Arthur}!zzzSalten, Felix@\emph{von Felix Salten}!1909-07-312@{31. 7. 1909}|)be}\mylabel{L03505h}  \newcommand{\dateiname}{L03505}\newcommand{\titel}{Felix Salten an Arthur Schnitzler, 31. 7. 1909}\newcommand{\editorInnen}{Martin Anton Müller und Laura Untner}%% latex-leseansicht-abspann.tex
%% Abspann für die Leseansicht.
%% Der Schalter \ifkorrekturansicht ist bereits durch den Vorspann gesetzt.

%% latex-abspann.tex
%% Gemeinsamer Abspann für Korrekturansicht und Leseansicht.
%% Setzt den Schalter \ifkorrekturansicht voraus (gesetzt in den
%% einbindenden Dateien latex-korrekturansicht-abspann.tex bzw.
%% latex-leseansicht-abspann.tex).
%% ---------------------------------------------------------------

\normalsize

% Das esempio-Environment wird nur in der Leseansicht benötigt
\ifkorrekturansicht\else
\newenvironment{esempio}[3]%
{
    \vspace{1.5ex}
    \rlap{\underline{#1}}
    \par
    \setlength{\parindent}{0cm}
    \nopagebreak
    \leftskip=#2cm
    \rightskip=#3cm
}
{
    \par
}
\fi

\doendnotes{C}
\bigskip
\vfill

\clearpage

\footnotesize

\ifkorrekturansicht
  \lohead{\textsc{register}}
\fi

% theindex-Environment neu definieren ohne reledmac
\makeatletter
\renewenvironment{theindex}{%
  \ifkorrekturansicht
    \section*{\indexname}%
  \else
    \subsubsection*{Index der erwähnten Entitäten}%
  \fi
  \setlength{\parindent}{0pt}%
  \setlength{\parskip}{0pt plus 0.3pt}%
  \let\item\@idxitem
}{%
  \ifkorrekturansicht\clearpage\fi
}
\makeatother

\IfFileExists{\jobname-pw.ind}{\input{\jobname-pw.ind}}{}

% Quellenangabe nur in der Leseansicht
\ifkorrekturansicht\else
% Fallback-Definitionen, falls die .tex-Datei \titel etc. nicht gesetzt hat
\providecommand{\titel}{}
\providecommand{\editorInnen}{}
\providecommand{\dateiname}{\jobname}

\vspace{3cm}

\vfill

\footnotesize
\textsc{Quelle}: \titel. Herausgegeben von {\editorInnen}. In: \emph{Arthur Schnitzler: Briefwechsel mit Autorinnen und Autoren}.
 Digitale Edition, https://schnitzler-briefe.acdh.oeaw.ac.at/{\dateiname}.html (Stand \today)
\fi

\end{document}


