%% latex-leseansicht-vorspann.tex
%% Vorspann für die Leseansicht.
%% Lädt die gemeinsame Datei latex-vorspann.tex mit nicht gesetztem Schalter.

\newif\ifkorrekturansicht
\korrekturansichtfalse

\input{../tex-inputs/latex-vorspann}

\begin{center}
            \textcolor{red}{ENTWURF, NICHT FERTIG KORRIGIERT}
                      \end{center}
            
         
         \renewcommand{\erwaehntePersonen}{Personen: Anna Katharina Rehmann, Paul Salten}
         \renewcommand{\erwaehnteInstitutionen}{Institutionen: Südbahnstrecke}
         \renewcommand{\erwaehnteOrte}{Orte: Edlach, Höhlenstein, Kaltenleutgeben, Niederösterreich, Wien}
         \renewcommand{\erwaehnteWerke}{}
               \section[Felix Salten an Arthur Schnitzler, 31. 7. 1909]{ Felix Salten an Arthur Schnitzler, 31. 7. 1909}\nopagebreak\mylabel{v}\rehead{ }\begin{ledgroupsized}[t]{13cm}\normalsize\beginnumbering \toendnotes[C]{\smallbreak\pagebreak[2]} \Standort{CUL, Schnitzler, B 89, B 1.}
\physDesc{Bildpostkarte, 396 Zeichen
\newline{}Handschrift: schwarze Tinte, lateinische Kurrent
\newline{}Versand: Stempel: »\nobreak{}\oindex{Hoehlenstein@\textbf{Höhlenstein}|pwk}\textcolor{gray}{Höhlenstein}\nobreak{}«.  
\newline{}Schnitzler: mit Bleistift Vermerk: »\textsc{Salten}« 
\newline{}Ordnung: mit Bleistift von unbekannter Hand nummeriert:
                                    »255« }\pstart{}{\pb}Herrn D\textsuperscript{r} Arthur Schnitzler\pend{}\pstart{}Edlach \textcolor{gray}{b/} Reichenau\oindex{Edlach@\textbf{Edlach}|pw}\pend{}\pstart{}Nied. Öst.\oindex{Niederoesterreich@\textbf{Niederösterreich}|pw}{ }Südbahn\orgindex{Suedbahnstrecke@Südbahnstrecke|pw}\pend{}{\bigskip}\pstart
           \noindent{}\centering{}{\pb}{[}Fotografie Paul\pwindex{Salten, Paul 11.08.1903 – 08.05.1937@\textsc{Salten, Paul} (11.08.1903 – 08.05.1937), \emph{Filmcutter}|pw} und Anna Salten\pwindex{Rehmann, Anna Katharina 18.08.1904 – 27.03.1977@\textsc{Rehmann, Anna Katharina} (18.08.1904 – 27.03.1977), \emph{Schauspielerin}|pw} in
                     Tracht{]}\pend
           \pstart
           Lieber, wir sind etwa gegen den 10. Aug. schon in Wien\oindex{Wien@\textbf{Wien}|pw}. Vielleicht komme ich – wenns Ihnen recht ist – einmal zum
               Tennis hinaus, ehe wir nach Kaltenleutgeben\oindex{Kaltenleutgeben@\textbf{Kaltenleutgeben}|pw}
               gehen. – Das »umseitige« Bild ist mit dem neuen Apparat von mir gemacht. Viele
               herzliche Grüße von uns zu Ihnen allen\textcolor{gray}{;}\pend
           \pstart Ihr \spacefill\mbox{Salten}\pend{}\pstart
           Landro\oindex{Hoehlenstein@\textbf{Höhlenstein}|pw}, 31. VII. 09\pend
           
         
         \endnumbering\mylabel{h}\end{ledgroupsized}\begin{anhang}\end{anhang}\newcommand{\dateiname}{L03505}\newcommand{\titel}{Felix Salten an Arthur Schnitzler, 31. 7. 1909}\newcommand{\editorInnen}{Martin Anton Müller und Laura Untner}%% latex-leseansicht-abspann.tex
%% Abspann für die Leseansicht.
%% Der Schalter \ifkorrekturansicht ist bereits durch den Vorspann gesetzt.

%% latex-abspann.tex
%% Gemeinsamer Abspann für Korrekturansicht und Leseansicht.
%% Setzt den Schalter \ifkorrekturansicht voraus (gesetzt in den
%% einbindenden Dateien latex-korrekturansicht-abspann.tex bzw.
%% latex-leseansicht-abspann.tex).
%% ---------------------------------------------------------------

\normalsize

% Das esempio-Environment wird nur in der Leseansicht benötigt
\ifkorrekturansicht\else
\newenvironment{esempio}[3]%
{
    \vspace{1.5ex}
    \rlap{\underline{#1}}
    \par
    \setlength{\parindent}{0cm}
    \nopagebreak
    \leftskip=#2cm
    \rightskip=#3cm
}
{
    \par
}
\fi

\doendnotes{C}
\bigskip
\vfill

\clearpage

\footnotesize

\ifkorrekturansicht
  \lohead{\textsc{register}}
\fi

% theindex-Environment neu definieren ohne reledmac
\makeatletter
\renewenvironment{theindex}{%
  \ifkorrekturansicht
    \section*{\indexname}%
  \else
    \subsubsection*{Index der erwähnten Entitäten}%
  \fi
  \setlength{\parindent}{0pt}%
  \setlength{\parskip}{0pt plus 0.3pt}%
  \let\item\@idxitem
}{%
  \ifkorrekturansicht\clearpage\fi
}
\makeatother

\IfFileExists{\jobname-pw.ind}{\input{\jobname-pw.ind}}{}

% Quellenangabe nur in der Leseansicht
\ifkorrekturansicht\else
% Fallback-Definitionen, falls die .tex-Datei \titel etc. nicht gesetzt hat
\providecommand{\titel}{}
\providecommand{\editorInnen}{}
\providecommand{\dateiname}{\jobname}

\vspace{3cm}

\vfill

\footnotesize
\textsc{Quelle}: \titel. Herausgegeben von {\editorInnen}. In: \emph{Arthur Schnitzler: Briefwechsel mit Autorinnen und Autoren}.
 Digitale Edition, https://schnitzler-briefe.acdh.oeaw.ac.at/{\dateiname}.html (Stand \today)
\fi

\end{document}


      