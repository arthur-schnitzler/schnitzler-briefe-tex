%% latex-korrekturansicht-vorspann.tex
%% Vorspann für die Korrekturansicht.
%% Lädt die gemeinsame Datei latex-vorspann.tex mit gesetztem Schalter.

\newif\ifkorrekturansicht
\korrekturansichttrue

\input{../tex-inputs/latex-vorspann}


\section[ Felix Salten an Arthur Schnitzler, 31. 7. 1909]{L03505 Felix Salten an Arthur Schnitzler, 31. 7. 1909}
\nopagebreak\mylabel{L03505v}
\rehead{ }\normalsize\beginnumbering\briefempfaengerindex{Schnitzler, Arthur@\textsc{Schnitzler, Arthur}!zzzSalten, Felix@\emph{von Felix Salten}!1909-07-312@{31. 7. 1909}|(be}
\toendnotes[C]{\smallbreak\pagebreak[2]}\Standort{CUL, Schnitzler, B 89, B 1.}
\physDesc{Bildpostkarte, 398 Zeichen
\newline{}Handschrift: schwarze Tinte, lateinische Kurrent
\newline{}Versand: Stempel: »\nobreak{}\oindex{Hoehlenstein@\textbf{Höhlenstein}, \emph{P.PPLQ}|pwk}Höhlenstein\nobreak{}«.  
\newline{}Schnitzler: mit Bleistift Vermerk: »\textsc{Salten}« 
\newline{}Ordnung: mit Bleistift von unbekannter Hand nummeriert: »255« }\toendnotes[C]{\smallbreak}\pstart{}{\pb}Herrn D\textsuperscript{r} Arthur Schnitzler\pend{}\pstart{}Edlach \textsuperscript{b}/Reichenau\oindex{Edlach@\textbf{Edlach}, \emph{P.PPL}|pw}\pend{}\pstart{}Nied. Öst.\oindex{Niederoesterreich@\textbf{Niederösterreich}, \emph{A.ADM1}|pw}\pend{}\pstart{}Südbahn\orgindex{Suedbahnstrecke@Südbahnstrecke|pw}\pend{}{\bigskip}
\pstart
           \noindent{}\centering{}{\pb}{[}Fotografie von Paul\pwindex{Salten, Paul 11.08.1903 – 08.05.1937@\textsc{Salten, Paul} (11.08.1903 – 08.05.1937), \emph{Filmcutter/Filmcutterin}|pw} und Anna Salten\pwindex{Rehmann, Anna Katharina 18.08.1904 – 27.03.1977@\textsc{Rehmann, Anna Katharina} (18.08.1904 – 27.03.1977), \emph{Schauspieler/Schauspielerin, Übersetzer/Übersetzerin}|pw} in
                  Tracht{]}\pend
           \vspace{1em}
\pstart
           \noindent{}{\pb}Lieber, wir
               sind etwa \label{K_L03505-1v}\edtext{gegen den 10. Aug.}{\lemma{\textnormal{\emph{gegen den 10. Aug.}}}\Cendnote{\textnormal{Die nächste im \emph{Tagebuch}\pwindex{Tagebuch@\emph{Tagebuch}|pwk} belegte
                  Begegnung – gemeinsames Tennisspiel – fand am 6. 9. 1909 statt.}}}\label{K_L03505-1} schon in Wien\oindex{Wien@\textbf{Wien}, \emph{A.ADM2}|pw}. Vielleicht
               komme ich – wenns Ihnen recht ist – einmal zum Tennis hinaus, ehe wir nach Kaltenleutgeben\oindex{Kaltenleutgeben@\textbf{Kaltenleutgeben}, \emph{P.PPLA3}|pw} gehen. – Das »umseitige« Bild
               ist mit dem neuen Apparat von mir gemacht. Viele herzliche Grüße von uns\pwindex{Salten, Ottilie 07.03.1868 – 22.06.1942@\textsc{Salten, Ottilie} (07.03.1868 – 22.06.1942), \emph{Schauspieler/Schauspielerin}|pwv} zu Ihnen
                  allen, \substVorne{}\textsuperscript{und}\substDazwischen{}Ihr\substHinten{}{ }\spacefill\mbox{Salten}\pend
           
\pstart
           Landro\oindex{Hoehlenstein@\textbf{Höhlenstein}, \emph{P.PPLQ}|pw}, 31. VII. 09\pend
           \selectlanguage{ngerman}\endnumbering\briefempfaengerindex{Schnitzler, Arthur@\textsc{Schnitzler, Arthur}!zzzSalten, Felix@\emph{von Felix Salten}!1909-07-312@{31. 7. 1909}|)be}\mylabel{L03505h}  \normalsize

\doendnotes{C}
\bigskip
\vfill

\clearpage

\footnotesize

\lohead{\textsc{register}}

% Definiere theindex-Environment komplett neu ohne reledmac
\makeatletter
\renewenvironment{theindex}{%
  \section*{\indexname}%
  \setlength{\parindent}{0pt}%
  \setlength{\parskip}{0pt plus 0.3pt}%
  \let\item\@idxitem
}{%
  \clearpage
}
\makeatother

\IfFileExists{\jobname-pw.ind}{\input{\jobname-pw.ind}}{}

\end{document}

      