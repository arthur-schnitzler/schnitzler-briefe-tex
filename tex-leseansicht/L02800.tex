%% latex-leseansicht-vorspann.tex
%% Vorspann für die Leseansicht.
%% Lädt die gemeinsame Datei latex-vorspann.tex mit nicht gesetztem Schalter.

\newif\ifkorrekturansicht
\korrekturansichtfalse

\input{../tex-inputs/latex-vorspann}


         
         \newcommand{\erwaehntePersonen}{Personen: Hermann Bahr, Richard Beer-Hofmann, Georg Brandes, Leopold Sonnemann}
         \newcommand{\erwaehnteInstitutionen}{Institutionen: An der schönen blauen Donau, Frankfurter Zeitung, Rütten & Loening}
         \newcommand{\erwaehnteOrte}{Orte: Frankfurt am Main, Paris, Wien, rue Feydeau}
         \newcommand{\erwaehnteWerke}{Werke: Frankfurter Zeitung, Freiwild. Schauspiel in 3 Akten, Kleines Feuilleton. [»Jung-Wien.«], Menschen und Werke. Essays}
               \section[ Paul Goldmann an Arthur Schnitzler, 8. 1. {[}1897{]}]{ Paul Goldmann an Arthur Schnitzler, 8. 1. {[}1897{]}}\nopagebreak\mylabel{v}\rehead{ }\begin{ledgroupsized}[t]{13cm}\normalsize\beginnumbering \toendnotes[C]{\smallbreak\pagebreak[2]} \Standort{DLA, A:Schnitzler, HS.NZ85.1.3167.}
\physDesc{Brief, 1 Blatt, 3 Seiten
\newline{}Handschrift: blaue Tinte, deutsche Kurrent
\newline{}Schnitzler: 1) mit Bleistift das Jahr »97« vermerkt  2) mit rotem Buntstift zwei Unterstreichungen}\toendnotes[C]{\smallbreak}\pstart
           \noindent{}{\pb}\textcolor{gray}{\textbf{\textbf{Frankfurter Zeitung\orgindex{Frankfurter Zeitung@Frankfurter Zeitung|pw}}}}\pend
           \pstart
           \textcolor{gray}{\textbf{(\begin{otherlanguage}{french}Gazette de Francfort\end{otherlanguage}\orgindex{Frankfurter Zeitung@Frankfurter Zeitung|pw}).}}\pend
           \pstart
           \textcolor{gray}{\textbf{\textbf{\begin{otherlanguage}{french}Fondateur M.\end{otherlanguage}{ }L. Sonnemann\pwindex{Sonnemann, Leopold 1831-10-29 – 1909-10-30@\textsc{Sonnemann, Leopold} (1831-10-29 – 1909-10-30), \emph{Journalist, Herausgeber}|pw}.}}}\pend
           \pstart
           \begin{otherlanguage}{french}\textcolor{gray}{\textbf{Journal politique, financier,}}\end{otherlanguage}\pend
           \pstart
           \begin{otherlanguage}{french}\textcolor{gray}{\textbf{commercial et littéraire.}}\end{otherlanguage}\pend
           \pstart
           \begin{otherlanguage}{french}\textcolor{gray}{\textbf{\textbf{Paraissant trois fois par jour.}}}\end{otherlanguage}\hfill \textsc{Paris\oindex{Paris@\textbf{Paris}|pw}}, 8. Januar.\pend
           \pstart
           \begin{otherlanguage}{french}\textcolor{gray}{\textbf{\textbf{Bureau à Paris\oindex{Paris@\textbf{Paris}|pw}}}}\end{otherlanguage}\pend
           \pstart
           \begin{otherlanguage}{french}\textcolor{gray}{\textbf{\textbf{24. Rue Feydeau\oindex{rue Feydeau@\textbf{rue Feydeau}|pw}.}}}\end{otherlanguage}\pend
           \pstart\center{}Mein lieber Freund,\pend\pstart
           Da ich nicht weiß, ob Du nicht \label{K_L02800-1v}\edtext{beifolgende Notiz\pwindex{Kleines Feuilleton. [»Jung-Wien.«]1897-01-07@\emph{Kleines Feuilleton. [»Jung-Wien.«]} {[}1897-01-07{]}|pwv}}{\lemma{\textnormal{\emph{beifolgende Notiz}}}\Cendnote{\textnormal{Beilage nicht erhalten. Es handelte sich
                  um eine Notiz\pwindex{Kleines Feuilleton. [»Jung-Wien.«]1897-01-07@\emph{Kleines Feuilleton. [»Jung-Wien.«]} {[}1897-01-07{]}|pwkv} im kleinen
                  Feuilleton der \emph{Frankfurter Zeitung}\pwindex{?? Werk@Nicht ermittelte Verfasserinnen und Verfasser!Frankfurter Zeitung1856 – 1943@\emph{Frankfurter Zeitung} {[}1856 – 1943{]}|pwk}, in der
                     Goldmann\pwindex{Goldmann, Paul 31.01.1865 – 25.09.1935@\textsc{Goldmann, Paul} (31.01.1865 – 25.09.1935), \emph{Schriftsteller, Journalist}|pwk} argumentierte, dass Hermann Bahr\pwindex{Bahr, Hermann 19.07.1863 – 15.01.1934@\textsc{Bahr, Hermann} (19.07.1863 – 15.01.1934), \emph{Schriftsteller, Kritiker}|pwk} nicht der Begründer von
                     Jung-Wien\oindex{Wien@\textbf{Wien}|pwk} sei (siehe Paul Goldmann an Arthur Schnitzler, 2. [1.? 1897]). Ebenso betonte er die Bedeutung der
                  Zeitschrift \emph{An der schönen blauen Donau}\orgindex{der schoenen blauen Donau@An der schönen blauen Donau|pwk}, für
                  die er früher selbst arbeitete. Vgl. [O. V.\pwindex{Goldmann, Paul 31.01.1865 – 25.09.1935@\textsc{Goldmann, Paul} (31.01.1865 – 25.09.1935), \emph{Schriftsteller, Journalist}|pwkv}] [=Paul
                        Goldmann\pwindex{Goldmann, Paul 31.01.1865 – 25.09.1935@\textsc{Goldmann, Paul} (31.01.1865 – 25.09.1935), \emph{Schriftsteller, Journalist}|pwk}]: \emph{Kleines Feuilleton.
                        [»Jung-Wien.«]}\pwindex{Kleines Feuilleton. [»Jung-Wien.«]1897-01-07@\emph{Kleines Feuilleton. [»Jung-Wien.«]} {[}1897-01-07{]}|pwk}. In: \emph{Frankfurter
                        Zeitung}\pwindex{?? Werk@Nicht ermittelte Verfasserinnen und Verfasser!Frankfurter Zeitung1856 – 1943@\emph{Frankfurter Zeitung} {[}1856 – 1943{]}|pwk}, Jg. 41, Nr. 7, 7. 1. 1897,
                     Zweites Morgenblatt, S. 1 [eckige Klammern i. O.].}}}\label{K_L02800-1h} in der Frankfurter Zeitung\pwindex{?? Werk@Nicht ermittelte Verfasserinnen und Verfasser!Frankfurter Zeitung1856 – 1943@\emph{Frankfurter Zeitung} {[}1856 – 1943{]}|pw} überſehen haſt, ſchicke ich
               ſie Dir der Sicherheit halber. Sie iſt natürlich von mir geſchrieben; aber da \textsc{Bahr\pwindex{Bahr, Hermann 19.07.1863 – 15.01.1934@\textsc{Bahr, Hermann} (19.07.1863 – 15.01.1934), \emph{Schriftsteller, Kritiker}|pw}} an eine Vereinbarung zwiſchen Dir und mir glauben würde und ſich wahrſcheinlich
               an Dir bei der erſten Gelegenheit rächen würde, halte ich es für beſſer, ihm
               einſtweilen nichts von meiner Autorſchaft zu ſagen. \strikeout{\textcolor{gray}{E}}{ }{\pb}Einmal mußte man doch gegen den Schwindel\strikeout{s} proteſtiren, den der Kerl treibt.\pend
           \pstart
           Von \textsc{Brandes\pwindex{Brandes, Georg 04.02.1842 – 19.02.1927@\textsc{Brandes, Georg} (04.02.1842 – 19.02.1927)|pw}} erhielt ich dieſer Tage einen \label{K_L02800-2v}\edtext{Brief}{\lemma{\textnormal{\emph{Brief}}}\Cendnote{\textnormal{Schnitzler\pwindex{Schnitzler, Arthur 15.05.1862 – 21.10.1931@\textsc{Schnitzler, Arthur} (15.05.1862 – 21.10.1931), \emph{Schriftsteller, Mediziner}|pwk} reagierte, indem er Brandes\pwindex{Brandes, Georg 04.02.1842 – 19.02.1927@\textsc{Brandes, Georg} (04.02.1842 – 19.02.1927)|pwk} am 11. 1. 1897 einen freundlichen Brief schrieb und ihm
                     \emph{Freiwild}\pwindex{Schnitzler, Arthur 15.05.1862 – 21.10.1931@\textsc{Schnitzler, Arthur} (15.05.1862 – 21.10.1931), \emph{Schriftsteller, Mediziner}!Freiwild. Schauspiel in 3 Akten1896@\strich\emph{Freiwild. Schauspiel in 3 Akten} {[}1896{]}|pwk} (noch als Manuskript) zukommen
                  ließ.}}}\label{K_L02800-2h}, den ich Dir ſchicken werde, ſobald ich ihn beantwortet habe. Er
               ſchreibt unter Anderem:\pend
           \pstart
           »\begin{otherlanguage}{french}\textsc{À propos}\end{otherlanguage}, meinem Verſprechen getreu ſandte ich an Herrn \textsc{Hofmann-Beer\pwindex{Beer-Hofmann, Richard 1866-07-11 – 1945-09-26@\textsc{Beer-Hofmann, Richard} (1866-07-11 – 1945-09-26), \emph{Schriftsteller}|pw}} meine \strikeout{\textcolor{gray}{m}} neue Sammlung \label{K_L02800-21v}\edtext{\textsc{Essais\pwindex{Brandes, Georg 04.02.1842 – 19.02.1927@\textsc{Brandes, Georg} (04.02.1842 – 19.02.1927)!Menschen und Werke. Essays1894@\strich\emph{Menschen und Werke. Essays} {[}1894{]}|pwv}}}{\lemma{\textnormal{\emph{Essais}}}\Cendnote{\textnormal{Georg Brandes\pwindex{Brandes, Georg 04.02.1842 – 19.02.1927@\textsc{Brandes, Georg} (04.02.1842 – 19.02.1927)|pwk}: \emph{Menschen und Werke. Essays}\pwindex{Brandes, Georg 04.02.1842 – 19.02.1927@\textsc{Brandes, Georg} (04.02.1842 – 19.02.1927)!Menschen und Werke. Essays1894@\strich\emph{Menschen und Werke. Essays} {[}1894{]}|pwk}. Frankfurt am Main\oindex{Frankfurt am Main@\textbf{Frankfurt am Main}|pwk}: \emph{Literarische Anstalt Rütten {\kaufmannsund}
                        Loening}\orgindex{Ruetten und Loening@Rütten {\kaufmannsund}  Loening|pwk} 1894. Am 14. 1. 1897 schrieb Beer-Hofmann\pwindex{Beer-Hofmann, Richard 1866-07-11 – 1945-09-26@\textsc{Beer-Hofmann, Richard} (1866-07-11 – 1945-09-26), \emph{Schriftsteller}|pwk} an Brandes\pwindex{Brandes, Georg 04.02.1842 – 19.02.1927@\textsc{Brandes, Georg} (04.02.1842 – 19.02.1927)|pwk} unter anderem Folgendes: »Arthur\pwindex{Schnitzler, Arthur 15.05.1862 – 21.10.1931@\textsc{Schnitzler, Arthur} (15.05.1862 – 21.10.1931), \emph{Schriftsteller, Mediziner}|pw} und ich sprechen oft von Ihnen,
                     und in den Briefen von Paul Goldmann\pwindex{Goldmann, Paul 31.01.1865 – 25.09.1935@\textsc{Goldmann, Paul} (31.01.1865 – 25.09.1935), \emph{Schriftsteller, Journalist}|pw}
                     kehrt Ihr Nahmen immer wieder. Besonders freut es mich, dass Sie\pwindex{Brandes, Georg 04.02.1842 – 19.02.1927@\textsc{Brandes, Georg} (04.02.1842 – 19.02.1927)|pwv} und Paul\pwindex{Goldmann, Paul 31.01.1865 – 25.09.1935@\textsc{Goldmann, Paul} (31.01.1865 – 25.09.1935), \emph{Schriftsteller, Journalist}|pw} einander manchmal schreiben. Er ist
                     ein Mensch von Klugheit und Güte.–« (Richard Beer-Hofmann\pwindex{Beer-Hofmann, Richard 1866-07-11 – 1945-09-26@\textsc{Beer-Hofmann, Richard} (1866-07-11 – 1945-09-26), \emph{Schriftsteller}|pwk}: \emph{Briefe. 1895–1945}. Hg. u. kommentiert v. Alexander Košenina.
                     Oldenburg: \emph{Igel}{ }1999, S. 9–10. (\emph{Große Richard
                        Beer-Hofmann-Ausgabe in sechs Bänden}. Hg. v. Günter Helmes, Michael
                     M. Schardt und Andreas Thomasberger, 7 / Erster Supplementband)}}}\label{K_L02800-21h},
               er hat mir aber mit keiner Silbe geantwortet. Auch \textsc{Schnitzler} vergißt mich, ſandte mir nicht ſein Schauſpiel\pwindex{Schnitzler, Arthur 15.05.1862 – 21.10.1931@\textsc{Schnitzler, Arthur} (15.05.1862 – 21.10.1931), \emph{Schriftsteller, Mediziner}!Freiwild. Schauspiel in 3 Akten1896@\strich\emph{Freiwild. Schauspiel in 3 Akten} {[}1896{]}|pwv}.«\pend
           \pstart
           {\pb}Du wirſt dem Manne\pwindex{Brandes, Georg 04.02.1842 – 19.02.1927@\textsc{Brandes, Georg} (04.02.1842 – 19.02.1927)|pwv} gewiß raſch ſchreiben. Aber auch \textsc{Richard\pwindex{Beer-Hofmann, Richard 1866-07-11 – 1945-09-26@\textsc{Beer-Hofmann, Richard} (1866-07-11 – 1945-09-26), \emph{Schriftsteller}|pw}} ſollte ihm antworten. Das Nicht-Schreiben iſt ein Verfahren, das ſich nur im
               Verkehr mit Freunden bewährt, das aber ſeine Unzuträglichkeiten hat, wenn man es auch
               gegenüber Fremden \strikeout{an\textcolor{gray}{w}} anwendet.\pend
           \pstart
           Viele herzliche Grüße an Dich und \textsc{Richard\pwindex{Beer-Hofmann, Richard 1866-07-11 – 1945-09-26@\textsc{Beer-Hofmann, Richard} (1866-07-11 – 1945-09-26), \emph{Schriftsteller}|pw}}!\pend
           \pstart
           Dein treuer {\\[\baselineskip]}\spacefill\mbox{Paul Goldmann.}\pend
           \leftskip=0em{}
         
         \endnumbering\mylabel{h}\end{ledgroupsized}  \newcommand{\dateiname}{L02800}\newcommand{\titel}{Paul Goldmann an Arthur Schnitzler, 8. 1. [1897]}\newcommand{\editorInnen}{Martin Anton Müller und Laura Untner}%% latex-leseansicht-abspann.tex
%% Abspann für die Leseansicht.
%% Der Schalter \ifkorrekturansicht ist bereits durch den Vorspann gesetzt.

%% latex-abspann.tex
%% Gemeinsamer Abspann für Korrekturansicht und Leseansicht.
%% Setzt den Schalter \ifkorrekturansicht voraus (gesetzt in den
%% einbindenden Dateien latex-korrekturansicht-abspann.tex bzw.
%% latex-leseansicht-abspann.tex).
%% ---------------------------------------------------------------

\normalsize

% Das esempio-Environment wird nur in der Leseansicht benötigt
\ifkorrekturansicht\else
\newenvironment{esempio}[3]%
{
    \vspace{1.5ex}
    \rlap{\underline{#1}}
    \par
    \setlength{\parindent}{0cm}
    \nopagebreak
    \leftskip=#2cm
    \rightskip=#3cm
}
{
    \par
}
\fi

\doendnotes{C}
\bigskip
\vfill

\clearpage

\footnotesize

\ifkorrekturansicht
  \lohead{\textsc{register}}
\fi

% theindex-Environment neu definieren ohne reledmac
\makeatletter
\renewenvironment{theindex}{%
  \ifkorrekturansicht
    \section*{\indexname}%
  \else
    \subsubsection*{Index der erwähnten Entitäten}%
  \fi
  \setlength{\parindent}{0pt}%
  \setlength{\parskip}{0pt plus 0.3pt}%
  \let\item\@idxitem
}{%
  \ifkorrekturansicht\clearpage\fi
}
\makeatother

\IfFileExists{\jobname-pw.ind}{\input{\jobname-pw.ind}}{}

% Quellenangabe nur in der Leseansicht
\ifkorrekturansicht\else
% Fallback-Definitionen, falls die .tex-Datei \titel etc. nicht gesetzt hat
\providecommand{\titel}{}
\providecommand{\editorInnen}{}
\providecommand{\dateiname}{\jobname}

\vspace{3cm}

\vfill

\footnotesize
\textsc{Quelle}: \titel. Herausgegeben von {\editorInnen}. In: \emph{Arthur Schnitzler: Briefwechsel mit Autorinnen und Autoren}.
 Digitale Edition, https://schnitzler-briefe.acdh.oeaw.ac.at/{\dateiname}.html (Stand \today)
\fi

\end{document}


      