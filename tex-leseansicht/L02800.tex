%% latex-korrekturansicht-vorspann.tex
%% Vorspann für die Korrekturansicht.
%% Lädt die gemeinsame Datei latex-vorspann.tex mit gesetztem Schalter.

\newif\ifkorrekturansicht
\korrekturansichttrue

\input{../tex-inputs/latex-vorspann}


\section[ Paul Goldmann an Arthur Schnitzler, 8. 1. {[}1897{]}]{L02800 Paul Goldmann an Arthur Schnitzler, 8. 1. {[}1897{]}}
\nopagebreak\mylabel{L02800v}
\rehead{ }\normalsize\beginnumbering\briefempfaengerindex{Schnitzler, Arthur@\textsc{Schnitzler, Arthur}!zzzGoldmann, Paul@\emph{von Paul Goldmann}!1897-01-081@{8. 1. {[}1897{]}}|(be}
\toendnotes[C]{\smallbreak\pagebreak[2]}\Standort{DLA, A:Schnitzler, HS.NZ85.1.3167.}
\physDesc{Brief, 1 Blatt, 3 Seiten, 1149 Zeichen
\newline{}Handschrift: blaue Tinte, deutsche Kurrent
\newline{}Schnitzler: 1) mit Bleistift das Jahr »97« vermerkt  2) mit rotem Buntstift zwei Unterstreichungen}\toendnotes[C]{\smallbreak}
\pstart
           {\pb}\textcolor{gray}{\textbf{\textbf{Frankfurter Zeitung\orgindex{Frankfurter Zeitung@Frankfurter Zeitung|pw}}}}\pend
           
\pstart
           \textcolor{gray}{\textbf{(\begin{otherlanguage}{french}Gazette de Francfort\end{otherlanguage}\orgindex{Frankfurter Zeitung@Frankfurter Zeitung|pw}).}}\pend
           
\pstart
           \textcolor{gray}{\textbf{\textbf{\begin{otherlanguage}{french}Fondateur M.\end{otherlanguage}{ }L. Sonnemann\pwindex{Sonnemann, Leopold 1831-10-29 – 1909-10-30@\textsc{Sonnemann, Leopold} (1831-10-29 – 1909-10-30), \emph{Journalist/Journalistin, Herausgeber/Herausgeberin}|pw}.}}}\pend
           
\pstart
           \begin{otherlanguage}{french}\textcolor{gray}{\textbf{Journal politique, financier,}}\end{otherlanguage}\pend
           
\pstart
           \begin{otherlanguage}{french}\textcolor{gray}{\textbf{commercial et littéraire.}}\end{otherlanguage}\pend
           
\pstart
           \begin{otherlanguage}{french}\textcolor{gray}{\textbf{\textbf{Paraissant trois fois par jour.}}}\end{otherlanguage}\hfill \textsc{Paris\oindex{Paris@\textbf{Paris}, \emph{P.PPLC}|pw}}, 8. Januar.\pend
           
\pstart
           \begin{otherlanguage}{french}\textcolor{gray}{\textbf{\textbf{Bureau à Paris\oindex{Paris@\textbf{Paris}, \emph{P.PPLC}|pw}}}}\end{otherlanguage}\pend
           
\pstart
           \begin{otherlanguage}{french}\textcolor{gray}{\textbf{\textbf{24. Rue Feydeau\oindex{rue Feydeau@\textbf{rue Feydeau}, \emph{Straße (K.STR)}|pw}.}}}\end{otherlanguage}\pend
           
\pstart\center{}Mein lieber Freund,\pend\vspace{0.5em}
\pstart
           Da ich nicht weiß, ob Du nicht \label{K_L02800-1v}\edtext{beifolgende Notiz\pwindex{Kleines Feuilleton. [»Jung-Wien.«]@\emph{Kleines Feuilleton. [»Jung-Wien.«]}|pwv}}{\lemma{\textnormal{\emph{beifolgende Notiz}}}\Cendnote{\textnormal{Beilage nicht erhalten. Es handelte sich
                  um eine Notiz\pwindex{Kleines Feuilleton. [»Jung-Wien.«]@\emph{Kleines Feuilleton. [»Jung-Wien.«]}|pwkv} im kleinen
                  Feuilleton der \emph{Frankfurter Zeitung}\pwindex{Frankfurter Zeitung@\emph{Frankfurter Zeitung}|pwk}, in der
                     Goldmann\pwindex{Goldmann, Paul 31.01.1865 – 25.09.1935@\textsc{Goldmann, Paul} (31.01.1865 – 25.09.1935), \emph{Schriftsteller/Schriftstellerin, Journalist/Journalistin}|pwk} argumentierte, dass Hermann Bahr\pwindex{Bahr, Hermann 19.07.1863 – 15.01.1934@\textsc{Bahr, Hermann} (19.07.1863 – 15.01.1934), \emph{Schriftsteller/Schriftstellerin, Kritiker/Kritikerin}|pwk} nicht der Begründer von
                     Jung-Wien\oindex{Wien@\textbf{Wien}, \emph{A.ADM2}|pwk} sei (siehe Paul Goldmann an Arthur Schnitzler, 2. [1.? 1897]). Ebenso betonte er die Bedeutung der
                  Zeitschrift \emph{An der schönen blauen Donau}\orgindex{der schoenen blauen Donau@An der schönen blauen Donau|pwk}, für
                  die er früher selbst gearbeitet hatte. Vgl. [Paul
                        Goldmann\pwindex{Goldmann, Paul 31.01.1865 – 25.09.1935@\textsc{Goldmann, Paul} (31.01.1865 – 25.09.1935), \emph{Schriftsteller/Schriftstellerin, Journalist/Journalistin}|pwk}]: \emph{Kleines Feuilleton.
                        (»Jung-Wien.«)}\pwindex{Kleines Feuilleton. [»Jung-Wien.«]@\emph{Kleines Feuilleton. [»Jung-Wien.«]}|pwk}. In: \emph{Frankfurter
                        Zeitung}\pwindex{Frankfurter Zeitung@\emph{Frankfurter Zeitung}|pwk}, Jg. 41, Nr. 7, 7. 1. 1897,
                     Zweites Morgenblatt, S. 1 (im Original ist der Titel mit eckigen Klammern versehen).}}}\label{K_L02800-1} in der Frankfurter Zeitung\pwindex{Frankfurter Zeitung@\emph{Frankfurter Zeitung}|pw} überſehen haſt, ſchicke ich
               ſie Dir der Sicherheit halber. Sie iſt natürlich von mir geſchrieben; aber da \textsc{Bahr\pwindex{Bahr, Hermann 19.07.1863 – 15.01.1934@\textsc{Bahr, Hermann} (19.07.1863 – 15.01.1934), \emph{Schriftsteller/Schriftstellerin, Kritiker/Kritikerin}|pw}} an eine Vereinbarung zwiſchen Dir und mir glauben würde und ſich wahrſcheinlich
               an Dir bei der erſten Gelegenheit rächen würde, halte ich es für beſſer, ihm
               einſtweilen nichts von meiner Autorſchaft zu ſagen. \strikeout{\textcolor{gray}{E}}{ }{\pb}Einmal mußte man doch gegen den Schwindel\strikeout{s} proteſtiren, den der Kerl treibt.\pend
           
\pstart
           Von \textsc{Brandes\pwindex{Brandes, Georg 04.02.1842 – 19.02.1927@\textsc{Brandes, Georg} (04.02.1842 – 19.02.1927)|pw}} erhielt ich dieſer Tage einen \label{K_L02800-2v}\edtext{Brief}{\lemma{\textnormal{\emph{Brief}}}\Cendnote{\textnormal{Schnitzler reagierte, indem er Brandes\pwindex{Brandes, Georg 04.02.1842 – 19.02.1927@\textsc{Brandes, Georg} (04.02.1842 – 19.02.1927)|pwk} am 11. 1. 1897 einen freundlichen Brief schrieb und ihm
                     \emph{Freiwild}\pwindex{Freiwild. Schauspiel in 3 Akten@\emph{Freiwild. Schauspiel in 3 Akten}|pwk} (noch als Manuskript) zukommen
                  ließ, Arthur Schnitzler an Georg Brandes, 11. 1. 1897.}}}\label{K_L02800-2}, den ich Dir ſchicken werde, ſobald ich ihn beantwortet habe. Er
               ſchreibt unter Anderem:\pend
           
\pstart
           »\begin{otherlanguage}{french}\textsc{À propos}\end{otherlanguage}, meinem Verſprechen getreu ſandte ich an Herrn \textsc{Hofmann-Beer\pwindex{Beer-Hofmann, Richard 1866-07-11 – 1945-09-26@\textsc{Beer-Hofmann, Richard} (1866-07-11 – 1945-09-26), \emph{Schriftsteller/Schriftstellerin}|pw}} meine \strikeout{\textcolor{gray}{m}} neue Sammlung \label{K_L02800-3v}\edtext{\textsc{Essais\pwindex{Menschen und Werke. Essays@\emph{Menschen und Werke. Essays}|pwv}}}{\lemma{\textnormal{\emph{Essais}}}\Cendnote{\textnormal{Georg Brandes\pwindex{Brandes, Georg 04.02.1842 – 19.02.1927@\textsc{Brandes, Georg} (04.02.1842 – 19.02.1927)|pwk}: \emph{Menschen und Werke. Essays}\pwindex{Menschen und Werke. Essays@\emph{Menschen und Werke. Essays}|pwk}. Frankfurt am Main\oindex{Frankfurt am Main@\textbf{Frankfurt am Main}, \emph{P.PPLA3}|pwk}: \emph{Literarische Anstalt Rütten {\kaufmannsund}
                        Loening}\orgindex{Ruetten und Loening@Rütten {\kaufmannsund}  Loening|pwk}1894. Am 14. 1. 1897 schrieb Beer-Hofmann\pwindex{Beer-Hofmann, Richard 1866-07-11 – 1945-09-26@\textsc{Beer-Hofmann, Richard} (1866-07-11 – 1945-09-26), \emph{Schriftsteller/Schriftstellerin}|pwk} an Brandes\pwindex{Brandes, Georg 04.02.1842 – 19.02.1927@\textsc{Brandes, Georg} (04.02.1842 – 19.02.1927)|pwk} unter anderem Folgendes: »Arthur und ich sprechen oft von Ihnen,
                     und in den Briefen von Paul Goldmann\pwindex{Goldmann, Paul 31.01.1865 – 25.09.1935@\textsc{Goldmann, Paul} (31.01.1865 – 25.09.1935), \emph{Schriftsteller/Schriftstellerin, Journalist/Journalistin}|pw}
                     kehrt Ihr Nahmen immer wieder. Besonders freut es mich, dass Sie\pwindex{Brandes, Georg 04.02.1842 – 19.02.1927@\textsc{Brandes, Georg} (04.02.1842 – 19.02.1927)|pwv} und Paul\pwindex{Goldmann, Paul 31.01.1865 – 25.09.1935@\textsc{Goldmann, Paul} (31.01.1865 – 25.09.1935), \emph{Schriftsteller/Schriftstellerin, Journalist/Journalistin}|pw} einander manchmal schreiben. Er ist
                     ein Mensch von Klugheit und Güte. –« (Richard Beer-Hofmann\pwindex{Beer-Hofmann, Richard 1866-07-11 – 1945-09-26@\textsc{Beer-Hofmann, Richard} (1866-07-11 – 1945-09-26), \emph{Schriftsteller/Schriftstellerin}|pwk}: \emph{Briefe. 1895–1945}. Herausgegeben und kommentiert von Alexander Košenina.
                     Oldenburg: \emph{Igel}{ }1999, S. 9–10 (\emph{Große Richard
                        Beer-Hofmann-Ausgabe in sechs Bänden}. Herausgegeben von Günter Helmes, Michael
                     M. Schardt und Andreas Thomasberger, 7 / Erster Supplementband).}}}\label{K_L02800-3}, er hat mir aber mit keiner Silbe geantwortet. Auch \textsc{Schnitzler} vergißt mich, ſandte mir nicht ſein Schauſpiel\pwindex{Freiwild. Schauspiel in 3 Akten@\emph{Freiwild. Schauspiel in 3 Akten}|pwv}.«\pend
           
\pstart
           {\pb}Du wirſt dem Manne\pwindex{Brandes, Georg 04.02.1842 – 19.02.1927@\textsc{Brandes, Georg} (04.02.1842 – 19.02.1927)|pwv} gewiß raſch ſchreiben. Aber auch \textsc{Richard\pwindex{Beer-Hofmann, Richard 1866-07-11 – 1945-09-26@\textsc{Beer-Hofmann, Richard} (1866-07-11 – 1945-09-26), \emph{Schriftsteller/Schriftstellerin}|pw}} ſollte ihm antworten. Das Nicht-Schreiben iſt ein Verfahren, das ſich nur im
               Verkehr mit Freunden bewährt, das aber ſeine Unzuträglichkeiten hat, wenn man es auch
               gegenüber Fremden \strikeout{an\textcolor{gray}{w}} anwendet.\pend
           
\pstart
           Viele herzliche Grüße an Dich und \textsc{Richard\pwindex{Beer-Hofmann, Richard 1866-07-11 – 1945-09-26@\textsc{Beer-Hofmann, Richard} (1866-07-11 – 1945-09-26), \emph{Schriftsteller/Schriftstellerin}|pw}}!\pend
           
\pstart
           Dein treuer {\\[\baselineskip]}\spacefill\mbox{Paul Goldmann.}\pend
           \leftskip=0em{}\selectlanguage{ngerman}\endnumbering\briefempfaengerindex{Schnitzler, Arthur@\textsc{Schnitzler, Arthur}!zzzGoldmann, Paul@\emph{von Paul Goldmann}!1897-01-081@{8. 1. {[}1897{]}}|)be}\mylabel{L02800h}  \normalsize

\doendnotes{C}
\bigskip
\vfill

\clearpage

\footnotesize

\lohead{\textsc{register}}

% Definiere theindex-Environment komplett neu ohne reledmac
\makeatletter
\renewenvironment{theindex}{%
  \section*{\indexname}%
  \setlength{\parindent}{0pt}%
  \setlength{\parskip}{0pt plus 0.3pt}%
  \let\item\@idxitem
}{%
  \clearpage
}
\makeatother

\IfFileExists{\jobname-pw.ind}{\input{\jobname-pw.ind}}{}

\end{document}

      