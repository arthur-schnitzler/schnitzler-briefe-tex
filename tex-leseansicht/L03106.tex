%% latex-korrekturansicht-vorspann.tex
%% Vorspann für die Korrekturansicht.
%% Lädt die gemeinsame Datei latex-vorspann.tex mit gesetztem Schalter.

\newif\ifkorrekturansicht
\korrekturansichttrue

\input{../tex-inputs/latex-vorspann}


\section[Felix Salten an Arthur Schnitzler, {[}28. 9. 1891?{]}]{L03106 Felix Salten an Arthur Schnitzler, {[}28. 9. 1891?{]}}
\nopagebreak\mylabel{L03106v}
\rehead{ }\normalsize\beginnumbering\briefempfaengerindex{Schnitzler, Arthur@\textsc{Schnitzler, Arthur}!zzzSalten, Felix@\emph{von Felix Salten}!1891-09-281@{{[}28. 9. 1891?{]}}|(be}
\toendnotes[C]{\smallbreak\pagebreak[2]}\Standort{CUL, Schnitzler, B 89, A 1.}
\physDesc{Visitenkarte, 167 Zeichen
\newline{}Handschrift: Bleistift, lateinische Kurrent
\newline{}Schnitzler: mit Bleistift datiert: »Sept. 91« 
\newline{}Ordnung: mit Bleistift von unbekannter Hand nummeriert: »7a« }\toendnotes[C]{\smallbreak}
\pstart
           \noindent{}{\pb}Lieber Freund! Verzeihen Sie, dass ich \label{K_L03106-1v}\edtext{heute}{\lemma{\textnormal{\emph{heute}}}\Cendnote{\textnormal{Die
                  Datierung der Visitenkarte in den September 1891, die Schnitzler vorgenommen hat, kann weiter eingegrenzt werden. Die
                  Zahl der Tage, die Salten\pwindex{Salten, Felix 06.09.1869 – 08.10.1945@\textsc{Salten, Felix} (06.09.1869 – 08.10.1945), \emph{Schriftsteller/Schriftstellerin, Journalist/Journalistin, Chefredakteur/Chefredakteurin}|pwk} und Schnitzler in diesem Monat am selben Ort
                  waren, ist gering, da der eine frühestens ab 14. 9. 1891 in Wien\oindex{Wien@\textbf{Wien}, \emph{A.ADM2}|pwk} war und sich der
                  andere zwischen 19. 9. 1891 und 26. 9. 1891 in Deutschland\oindex{Deutschland@\textbf{Deutschland}, \emph{A.PCLI}|pwk}
                  aufhielt. Berücksichtigt man auch, dass es zu einem Treffen am
                     Vormittag in einer größeren Runde gekommen sein muss, bietet sich
                  mit Schnitzlers{ }\emph{Tagebuch}\pwindex{Tagebuch@\emph{Tagebuch}|pwk} nur ein Treffen im Theaterausschuss der \emph{Freien Bühne}\orgindex{»Freie Buehne« Verein fuer moderne Literatur@»Freie Bühne« Verein für moderne Literatur|pwk} an, das am 28. 9. 1891
                  stattfand.}}}\label{K_L03106-1} so ohne Gruss verschwunden bin. Das kam wegen der kleinen
                  \label{K_L03106-2v}\edtext{C.\pwindex{C. @\textsc{C.}|pw}}{\lemma{\textnormal{\emph{C.}}}\Cendnote{\textnormal{nicht ermittelt}}}\label{K_L03106-2}\pend
           
\pstart
           Ich bin um 10 im Kremser\oindex{Cafe Kremser@\textbf{Café Kremser}, \emph{Kaffeehaus (K.KAF)}|pw}, wo ich Sie
               gar gerne \label{K_L03106-3v}\edtext{sehen {\pb}möchte}{\lemma{\textnormal{\emph{sehen möchte}}}\Cendnote{\textnormal{Ein solcher Kaffeehausbesuch ist nicht in Schnitzlers{ }\emph{Tagebuch}\pwindex{Tagebuch@\emph{Tagebuch}|pwk}
                  erwähnt.}}}\label{K_L03106-3}\pend
           \pstart Herzlich Ihr\pend{}
\pstart
           \centering{}\textcolor{gray}{\textbf{FELIX SALTEN}}\pend
           
\pstart
           \raggedleft{}\textcolor{gray}{\textbf{IX., BERGGASSE 13\oindex{Berggasse@\textbf{Berggasse}, \emph{Straße (K.STR)}|pw}.}}\pend
           \selectlanguage{ngerman}\endnumbering\briefempfaengerindex{Schnitzler, Arthur@\textsc{Schnitzler, Arthur}!zzzSalten, Felix@\emph{von Felix Salten}!1891-09-281@{{[}28. 9. 1891?{]}}|)be}\mylabel{L03106h}  \normalsize

\doendnotes{C}
\bigskip
\vfill

\clearpage

\footnotesize

\lohead{\textsc{register}}

% Definiere theindex-Environment komplett neu ohne reledmac
\makeatletter
\renewenvironment{theindex}{%
  \section*{\indexname}%
  \setlength{\parindent}{0pt}%
  \setlength{\parskip}{0pt plus 0.3pt}%
  \let\item\@idxitem
}{%
  \clearpage
}
\makeatother

\IfFileExists{\jobname-pw.ind}{\input{\jobname-pw.ind}}{}

\end{document}

      