%% latex-leseansicht-vorspann.tex
%% Vorspann für die Leseansicht.
%% Lädt die gemeinsame Datei latex-vorspann.tex mit nicht gesetztem Schalter.

\newif\ifkorrekturansicht
\korrekturansichtfalse

\input{../tex-inputs/latex-vorspann}

\begin{center}
            \textcolor{red}{ENTWURF, NICHT FERTIG KORRIGIERT}
                      \end{center}
            
         
         \renewcommand{\erwaehntePersonen}{Personen:  C.}
         \renewcommand{\erwaehnteInstitutionen}{Institutionen: »Freie Bühne« Verein für moderne Literatur}
         \renewcommand{\erwaehnteOrte}{Orte: Berggasse, Café Kremser, Deutschland, Wien}
         \renewcommand{\erwaehnteWerke}{Werke: Tagebuch}
               \section[Felix Salten an Arthur Schnitzler, {[}28. 9. 1891?{]}]{ Felix Salten an Arthur Schnitzler, {[}28. 9. 1891?{]}}\nopagebreak\mylabel{v}\rehead{ }\begin{ledgroupsized}[t]{13cm}\normalsize\beginnumbering \toendnotes[C]{\smallbreak\pagebreak[2]} \Standort{CUL, Schnitzler, B 89, A 1.}
\physDesc{Karte, 169 Zeichen
\newline{}Handschrift: Bleistift, lateinische Kurrent
\newline{}Schnitzler: mit Bleistift datiert: »Sept. 91« 
\newline{}Ordnung: mit Bleistift von unbekannter Hand nummeriert:
                                    »7a« }\toendnotes[C]{\smallbreak}\pstart
           \noindent{}{\pb}Lieber Freund! Verzeihen Sie, dass ich \label{K_L03106-1v}\edtext{heute so ohne Gruss}{\lemma{\textnormal{\emph{heute so ohne Gruss}}}\Cendnote{\textnormal{Die Zahl der Tage, die Salten\pwindex{Salten, Felix 06.09.1869 – 08.10.1945@\textsc{Salten, Felix} (06.09.1869 – 08.10.1945), \emph{Schriftsteller, Journalist}|pwk} und Schnitzler\pwindex{Schnitzler, Arthur 15.05.1862 – 21.10.1931@\textsc{Schnitzler, Arthur} (15.05.1862 – 21.10.1931), \emph{Schriftsteller, Mediziner}|pwk} im
                     September 1891 am gleichen Ort sind, ist, gering, da der eine
                  frühestens ab 14. 9. 1891 in Wien\oindex{Wien@\textbf{Wien}|pwk},
                  der andere aber zwischen 19. 9. 1891 und 26. 9. 1891 in Deutschland\oindex{Deutschland@\textbf{Deutschland}|pwk}. Berücksichtigt man auch, dass es zu einem Treffen am Vormittag
                  in einer größeren Runde gekommen sein muss, bietet sich mit Schnitzler\pwindex{Schnitzler, Arthur 15.05.1862 – 21.10.1931@\textsc{Schnitzler, Arthur} (15.05.1862 – 21.10.1931), \emph{Schriftsteller, Mediziner}|pwk}s \emph{Tagebuch}\pwindex{Schnitzler, Arthur 15.05.1862 – 21.10.1931@\textsc{Schnitzler, Arthur} (15.05.1862 – 21.10.1931), \emph{Schriftsteller, Mediziner}!Tagebuch1981 – 2000@\strich\emph{Tagebuch} {[}1981 – 2000{]}|pwk}
                  nur ein Treffen im Theaterausschuss der \emph{Freien
                     Bühne}\orgindex{»Freie Buehne« Verein fuer moderne Literatur@»Freie Bühne« Verein für moderne Literatur|pwk} an, das am 28. 9. 1891 stattfand.}}}\label{K_L03106-1h} verschwunden bin. Das kam wegen
               der kleinen C.\pwindex{C. @\textsc{C.}|pw}\pend
           \pstart
           Ich bin um 10 im Kremser\oindex{Cafe Kremser@\textbf{Café Kremser}|pw}, wo ich Sie
               gar gerne sehen {\pb}möchte. \pend
           \pstart  Herzlich Ihr \pend{}\pstart
           \centering{}\textcolor{gray}{\textbf{FELIX SALTEN}}\pend
           \pstart
           \noindent{}\raggedleft{}\textcolor{gray}{\textbf{IX., BERGGASSE 13\oindex{Berggasse@\textbf{Berggasse}|pw}. }}\pend
           
         
         \endnumbering\mylabel{h}\end{ledgroupsized}\begin{anhang}\end{anhang}\newcommand{\dateiname}{L03106}\newcommand{\titel}{Felix Salten an Arthur Schnitzler, [28. 9. 1891?]}\newcommand{\editorInnen}{Martin Anton Müller und Laura Untner}%% latex-leseansicht-abspann.tex
%% Abspann für die Leseansicht.
%% Der Schalter \ifkorrekturansicht ist bereits durch den Vorspann gesetzt.

%% latex-abspann.tex
%% Gemeinsamer Abspann für Korrekturansicht und Leseansicht.
%% Setzt den Schalter \ifkorrekturansicht voraus (gesetzt in den
%% einbindenden Dateien latex-korrekturansicht-abspann.tex bzw.
%% latex-leseansicht-abspann.tex).
%% ---------------------------------------------------------------

\normalsize

% Das esempio-Environment wird nur in der Leseansicht benötigt
\ifkorrekturansicht\else
\newenvironment{esempio}[3]%
{
    \vspace{1.5ex}
    \rlap{\underline{#1}}
    \par
    \setlength{\parindent}{0cm}
    \nopagebreak
    \leftskip=#2cm
    \rightskip=#3cm
}
{
    \par
}
\fi

\doendnotes{C}
\bigskip
\vfill

\clearpage

\footnotesize

\ifkorrekturansicht
  \lohead{\textsc{register}}
\fi

% theindex-Environment neu definieren ohne reledmac
\makeatletter
\renewenvironment{theindex}{%
  \ifkorrekturansicht
    \section*{\indexname}%
  \else
    \subsubsection*{Index der erwähnten Entitäten}%
  \fi
  \setlength{\parindent}{0pt}%
  \setlength{\parskip}{0pt plus 0.3pt}%
  \let\item\@idxitem
}{%
  \ifkorrekturansicht\clearpage\fi
}
\makeatother

\IfFileExists{\jobname-pw.ind}{\input{\jobname-pw.ind}}{}

% Quellenangabe nur in der Leseansicht
\ifkorrekturansicht\else
% Fallback-Definitionen, falls die .tex-Datei \titel etc. nicht gesetzt hat
\providecommand{\titel}{}
\providecommand{\editorInnen}{}
\providecommand{\dateiname}{\jobname}

\vspace{3cm}

\vfill

\footnotesize
\textsc{Quelle}: \titel. Herausgegeben von {\editorInnen}. In: \emph{Arthur Schnitzler: Briefwechsel mit Autorinnen und Autoren}.
 Digitale Edition, https://schnitzler-briefe.acdh.oeaw.ac.at/{\dateiname}.html (Stand \today)
\fi

\end{document}


      