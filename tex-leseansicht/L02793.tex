%% latex-korrekturansicht-vorspann.tex
%% Vorspann für die Korrekturansicht.
%% Lädt die gemeinsame Datei latex-vorspann.tex mit gesetztem Schalter.

\newif\ifkorrekturansicht
\korrekturansichttrue

\input{../tex-inputs/latex-vorspann}


\section[ Felix Salten und Amanda Zsolnay an Arthur Schnitzler, 15. 6. 1923]{L02793 Felix Salten und Amanda Zsolnay an Arthur Schnitzler, 15. 6. 1923}
\nopagebreak\mylabel{L02793v}
\rehead{ }\normalsize\beginnumbering\briefempfaengerindex{Schnitzler, Arthur@\textsc{Schnitzler, Arthur}!zzzSalten, Felix@\emph{von Felix Salten}!1923-06-152@{15. 6. 1923}|(be}
\toendnotes[C]{\smallbreak\pagebreak[2]}\Standort{CUL, Schnitzler, B 89, B 2.}
\physDesc{Bildpostkarte, 688 Zeichen
\newline{}Handschrift Felix Salten: schwarze Tinte, lateinische Kurrent
\newline{}Handschrift Amanda von Zsolnay: schwarze Tinte, deutsche Kurrent
\newline{}Ordnung: 1) mit Bleistift von Frieda Pollak\pwindex{Pollak, Frieda 08.12.1881 – 13.07.1937@\textsc{Pollak, Frieda} (08.12.1881 – 13.07.1937), \emph{Sekretär/Sekretärin}|pw} (?) mit
                                 dem Buchstaben »A« (Abgeschrieben/Abschrift)
                                 gekennzeichnet  2) mit Bleistift von unbekannter Hand nummeriert: »293«}\toendnotes[C]{\smallbreak}
\pstart
           \noindent{}\centering{}{\pb}{[}Schloss Zsolnay\oindex{Schloss Csáky@\textbf{Schloss Csáky}, \emph{Schloss (K.SLS)}|pw}{]}\pend
           \vspace{1em}
\pstart
           \raggedleft{}{\pb}Oberufer\oindex{Prievoz@\textbf{Prievoz}, \emph{P.PPL}|pw}{ }15. 6. 23\pend
           
\pstart{}Lieber,\pend\vspace{0.5em}
\pstart
           am Dienstag (19.) komme ich nach Wien\oindex{Wien@\textbf{Wien}, \emph{A.ADM2}|pw}, weil ich ins Theater mu\textcolor{gray}{ß}. Am Mittwoch fahre ich wieder hierher\oindex{Prievoz@\textbf{Prievoz}, \emph{P.PPL}|pwv}, wo wir sehr schöne \uline{stille} Tage haben. Wollen Sie nicht \label{K_L02793-1v}\edtext{am Mittwoch
               mit mir kommen}{\lemma{\textnormal{\emph{am … kommen}}}\Cendnote{\textnormal{Dazu kam es nicht.}}}\label{K_L02793-1}?
               Und sei’s auch nur überm Tag. Das wäre reizend. Sie können Donnerstag{ }Mittag wieder in Wien\oindex{Wien@\textbf{Wien}, \emph{A.ADM2}|pw} sein, wenns
               nicht anders geht. Bitte um ein Wort in die Cottagegasse\oindex{Cottagegasse@\textbf{Cottagegasse}, \emph{Straße (K.STR)}|pw}.\pend
           \pstart Herzlichst Ihr \spacefill\mbox{Salten}\pend{}\selectlanguage{ngerman}\vspace{1em}{\vspace{1\baselineskip}}
\pstart{}{[}hs. :{]} Verehrter Herr Doktor,\pend\vspace{0.5em}
\pstart
           obwohl ich überzeugt bin, daß unſer Freund \textsc{Salten} Ihnen
               meine Einladung mit ſoviel Wärme und Herzlichkeit übermittelt hat, wie ſie gemeint
               iſt, möchte ich Ihnen doch gerne ſelbſt ſagen, wie ſehr {\pb}wir uns darauf freuen, Sie bei
               uns zu begrüßen.\pend
           
\pstart
           Tauſend herzliche Grüße {\\[\baselineskip]}\spacefill\mbox{Andy Zsolnay}\pend
           \leftskip=0em{}\selectlanguage{ngerman}\endnumbering\briefempfaengerindex{Schnitzler, Arthur@\textsc{Schnitzler, Arthur}!zzzSalten, Felix@\emph{von Felix Salten}!1923-06-152@{15. 6. 1923}|)be}\mylabel{L02793h}  \normalsize

\doendnotes{C}
\bigskip
\vfill

\clearpage

\footnotesize

\lohead{\textsc{register}}

% Definiere theindex-Environment komplett neu ohne reledmac
\makeatletter
\renewenvironment{theindex}{%
  \section*{\indexname}%
  \setlength{\parindent}{0pt}%
  \setlength{\parskip}{0pt plus 0.3pt}%
  \let\item\@idxitem
}{%
  \clearpage
}
\makeatother

\IfFileExists{\jobname-pw.ind}{\input{\jobname-pw.ind}}{}

\end{document}

      