%% latex-leseansicht-vorspann.tex
%% Vorspann für die Leseansicht.
%% Lädt die gemeinsame Datei latex-vorspann.tex mit nicht gesetztem Schalter.

\newif\ifkorrekturansicht
\korrekturansichtfalse

\input{../tex-inputs/latex-vorspann}

\begin{center}
            \textcolor{red}{ENTWURF, NICHT FERTIG KORRIGIERT}
                      \end{center}
            
         
         \renewcommand{\erwaehntePersonen}{Personen: Frieda Pollak}
         \renewcommand{\erwaehnteOrte}{Orte: Cottagegasse, Prievoz, Schloss Csáky, Wien}
         \renewcommand{\erwaehnteWerke}{}
               \section[Felix Salten an Arthur Schnitzler, 15. 6. 1923]{ Felix Salten an Arthur Schnitzler, 15. 6. 1923}\nopagebreak\mylabel{v}\rehead{ }\begin{ledgroupsized}[t]{13cm}\normalsize\beginnumbering \toendnotes[C]{\smallbreak\pagebreak[2]} \Standort{CUL, Schnitzler, B 89, B 2.}
\physDesc{Bildpostkarte, 693 Zeichen
\newline{}Handschrift Felix Salten: schwarze Tinte, lateinische Kurrent\newline{}Handschrift Amanda von Zsolnay: schwarze Tinte, deutsche Kurrent
\newline{}Ordnung: 1) mit Bleistift von Frieda
                                    Pollak\pwindex{Pollak, Frieda 08.12.1881 – 13.07.1937@\textsc{Pollak, Frieda} (08.12.1881 – 13.07.1937), \emph{Sekretärin}|pw} (?) mit dem Buchstaben »A«
                                 (Abgeschrieben/Abschrift) gekennzeichnet  2) mit Bleistift von unbekannter Hand nummeriert:
                                    »293«}\pstart
           \noindent{}\centering{}{\pb}{[}Schloss
                     Zsolnay\oindex{Schloss Csáky@\textbf{Schloss Csáky}|pw}{]}\pend
           \pstart
           {\pb}Oberufer\oindex{Prievoz@\textbf{Prievoz}|pw}, 15. 6. 23\pend
           \pstart{}Lieber,\pend\pstart
           am Dienstag (19.) komme ich nach Wien\oindex{Wien@\textbf{Wien}|pw}, weil ich ins Theater muß. Am Mittwoch fahre ich wieder
               hierher, wo wir sehr schöne \uline{stille} Tage haben. Wollen
               Sie nicht am Mittwoch mit mir kommen? Und sei’s auch nur über’n Tag. Das
               wäre reizend. Sie können Donnerstag Mittag wieder in Wien\oindex{Wien@\textbf{Wien}|pw} sein, wenns nicht anders geht. Bitte um ein Wort in die Cottagegasse\oindex{Cottagegasse@\textbf{Cottagegasse}|pw}. \pend
           \pstart Herzlichst Ihr \spacefill\mbox{Salten}\pend{}{\bigskip}\pstart{}{[}hs. Zsolnay:{]} Verehrter Herr Doktor,\pend\pstart
           obwohl ich überzeugt bin, daß unſer Freund \textsc{Salten} Ihnen
               meine Einladung mit ſoviel Wärme und Herzlichkeit übermittelt hat, wie ſie gemeint
               iſt, möchte ich Ihnen doch gern ſelbſt ſagen, wie ſehr {\pb}wir uns darauf freuen, Sie bei
               uns zu begrüßen. \pend
           \pstart
           Tauſend herzliche Grüße {\\[\baselineskip]}\spacefill\mbox{Andy Zsolnay}\pend
           \leftskip=0em{}
         
         \endnumbering\mylabel{h}\end{ledgroupsized}\begin{anhang}\end{anhang}\newcommand{\dateiname}{L02793}\newcommand{\titel}{Felix Salten an Arthur Schnitzler, 15. 6. 1923}\newcommand{\editorInnen}{Martin Anton Müller und Laura Untner}%% latex-leseansicht-abspann.tex
%% Abspann für die Leseansicht.
%% Der Schalter \ifkorrekturansicht ist bereits durch den Vorspann gesetzt.

%% latex-abspann.tex
%% Gemeinsamer Abspann für Korrekturansicht und Leseansicht.
%% Setzt den Schalter \ifkorrekturansicht voraus (gesetzt in den
%% einbindenden Dateien latex-korrekturansicht-abspann.tex bzw.
%% latex-leseansicht-abspann.tex).
%% ---------------------------------------------------------------

\normalsize

% Das esempio-Environment wird nur in der Leseansicht benötigt
\ifkorrekturansicht\else
\newenvironment{esempio}[3]%
{
    \vspace{1.5ex}
    \rlap{\underline{#1}}
    \par
    \setlength{\parindent}{0cm}
    \nopagebreak
    \leftskip=#2cm
    \rightskip=#3cm
}
{
    \par
}
\fi

\doendnotes{C}
\bigskip
\vfill

\clearpage

\footnotesize

\ifkorrekturansicht
  \lohead{\textsc{register}}
\fi

% theindex-Environment neu definieren ohne reledmac
\makeatletter
\renewenvironment{theindex}{%
  \ifkorrekturansicht
    \section*{\indexname}%
  \else
    \subsubsection*{Index der erwähnten Entitäten}%
  \fi
  \setlength{\parindent}{0pt}%
  \setlength{\parskip}{0pt plus 0.3pt}%
  \let\item\@idxitem
}{%
  \ifkorrekturansicht\clearpage\fi
}
\makeatother

\IfFileExists{\jobname-pw.ind}{\input{\jobname-pw.ind}}{}

% Quellenangabe nur in der Leseansicht
\ifkorrekturansicht\else
% Fallback-Definitionen, falls die .tex-Datei \titel etc. nicht gesetzt hat
\providecommand{\titel}{}
\providecommand{\editorInnen}{}
\providecommand{\dateiname}{\jobname}

\vspace{3cm}

\vfill

\footnotesize
\textsc{Quelle}: \titel. Herausgegeben von {\editorInnen}. In: \emph{Arthur Schnitzler: Briefwechsel mit Autorinnen und Autoren}.
 Digitale Edition, https://schnitzler-briefe.acdh.oeaw.ac.at/{\dateiname}.html (Stand \today)
\fi

\end{document}


      