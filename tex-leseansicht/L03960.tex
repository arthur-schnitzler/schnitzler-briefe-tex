%% latex-leseansicht-vorspann.tex
%% Vorspann für die Leseansicht.
%% Lädt die gemeinsame Datei latex-vorspann.tex mit nicht gesetztem Schalter.

\newif\ifkorrekturansicht
\korrekturansichtfalse

\input{../tex-inputs/latex-vorspann}


\section[Arthur Schnitzler an Berta Zuckerkandl, 8. 10. 1925]{L03960 Arthur Schnitzler an Berta Zuckerkandl, 8. 10. 1925}
\nopagebreak\mylabel{L03960v}
\rehead{ }\normalsize\beginnumbering\briefempfaengerindex{Zuckerkandl, Berta@\textsc{Zuckerkandl, Berta}!zzzSchnitzler, Arthur@\emph{von Arthur Schnitzler}!1925-10-081@{8. 10. 1925}|(be}
\toendnotes[C]{\smallbreak\pagebreak[2]}
\correspDesc{Versand  durch Arthur Schnitzler am 8. 10. 1925 in Wien
\newline{}Erhalt  durch Berta Zuckerkandl im Zeitraum [9. 10. 1925
                  – 13. 10. 1925?] in Paris}\toendnotes[C]{\smallbreak}
\Standort{DLA, HS.1985.1.2282.}
\physDesc{Brief, Durchschlag, 1 Blatt, 2 Seiten, 2468 Zeichen
\newline{}Schreibmaschine
\newline{}Handschrift: roter Buntstift, lateinische Kurrent (\noindent{}beschriftet: »\uline{Zuckerkandl}« und »\uline{Frankreich.}«, zwanzig Unterstreichungen)}\toendnotes[C]{\smallbreak}
\pstart
           \raggedleft{}{\pb}8. 10. 1925.\pend
           
\pstart{}Liebe und verehrte Frau Hofrätin.\pend\vspace{0.5em}
\pstart
           Ehe ich für 8–14 Tage \label{K_L03960-1v}\edtext{nach Berlin\oindex{Berlin@\textbf{Berlin}, \emph{Hauptstadt}|pw}}{\lemma{\textnormal{\emph{nach Berlin}}}\Cendnote{\textnormal{Schnitzler hielt sich vom 12. 10. 1925 bis zum
                     20. 10. 1925 in
                     Berlin\oindex{Berlin@\textbf{Berlin}, \emph{Hauptstadt}|pwk} auf, wo sein Sohn Heinrich\pwindex{Schnitzler, Heinrich 9.\,8.\,1902 Hinterbrühl – 12.\,7.\,1982 Wien@\textsc{Schnitzler, Heinrich} (9.\,8.\,1902 Hinterbrühl – 12.\,7.\,1982 Wien), \emph{Regisseur, Schauspieler}|pwk} im Schillertheater\oindex{Schiller-Theater@\textbf{Schiller-Theater}, \emph{Theater}|pwk} in einer Inszenierung der \emph{Liebelei}\pwindex{Schnitzler, Arthur 15. 5. 1862 Wien – 21. 10. 1931 ebd.@\textsc{Schnitzler, Arthur} (15. 5. 1862 Wien – 21. 10. 1931 ebd.), \emph{Schriftsteller, Mediziner}!Liebelei. Schauspiel in drei Akten@\strich\emph{Liebelei. Schauspiel in drei Akten}|pwk} mitwirkte.}}}\label{K_L03960-1} abreise, möchte ich ein paar Worte über meine
                  französischen\oindex{Frankreich@\textbf{Frankreich}|pw} Angelegenheiten schreiben und
               mich zugleich bei Ihnen nach dem Stand der Dinge erkundigen. Hat sich in der \label{K_L03960-2v}\edtext{Sache Gemier\pwindex{Gémier, Firmin 21.\,2.\,1865 Aubervilliers – 26.\,11.\,1933 Paris@\textsc{Gémier, Firmin} (21.\,2.\,1865 Aubervilliers – 26.\,11.\,1933 Paris), \emph{Theaterleiter, Schauspieler, Drehbuchautor}|pw}}{\lemma{\textnormal{\emph{Sache Gemier}}}\Cendnote{\textnormal{Der Theaterleiter Firmin Gémier\pwindex{Gémier, Firmin 21.\,2.\,1865 Aubervilliers – 26.\,11.\,1933 Paris@\textsc{Gémier, Firmin} (21.\,2.\,1865 Aubervilliers – 26.\,11.\,1933 Paris), \emph{Theaterleiter, Schauspieler, Drehbuchautor}|pwk} hatte die Inszenierung eines Schauspiels von
                     Schnitzler erwogen, vgl. A. S.: \emph{Tagebuch}, 29. 5. 1925.}}}\label{K_L03960-2} etwas
               gerührt? Ist das »Weite Land\pwindex{Schnitzler, Arthur 15. 5. 1862 Wien – 21. 10. 1931 ebd.@\textsc{Schnitzler, Arthur} (15. 5. 1862 Wien – 21. 10. 1931 ebd.), \emph{Schriftsteller, Mediziner}!weite Land. Tragikomödie in fünf Akten@\strich\emph{Das weite Land. Tragikomödie in fünf Akten}|pw}« engültig erledigt?
               Hat Lenormand\pwindex{Lenormand, Henri-René 3.\,5.\,1882 Paris – 16.\,2.\,1951 ebd.@\textsc{Lenormand, Henri-René} (3.\,5.\,1882 Paris – 16.\,2.\,1951 ebd.), \emph{Schriftsteller}|pw} zu der französischen\oindex{Frankreich@\textbf{Frankreich}|pw}{ }Uebersetzung\pwindex{Schnitzler, Arthur 15. 5. 1862 Wien – 21. 10. 1931 ebd.@\textsc{Schnitzler, Arthur} (15. 5. 1862 Wien – 21. 10. 1931 ebd.), \emph{Schriftsteller, Mediziner}!Amourette. Pièce en trois actes. Adaptée de Arthur Schnitzler@\strich\emph{Amourette. Pièce en trois actes. Adaptée de Arthur Schnitzler}|pwv} der »Liebelei\pwindex{Schnitzler, Arthur 15. 5. 1862 Wien – 21. 10. 1931 ebd.@\textsc{Schnitzler, Arthur} (15. 5. 1862 Wien – 21. 10. 1931 ebd.), \emph{Schriftsteller, Mediziner}!Liebelei. Schauspiel in drei Akten@\strich\emph{Liebelei. Schauspiel in drei Akten}|pw}«, die ich \label{K_L03960-3v}\edtext{ihm übersandt}{\lemma{\textnormal{\emph{ihm übersandt}}}\Cendnote{\textnormal{Die
                  Übersetzung wurde übersandt mit dem Brief von Arthur Schnitzler an Henri-René
                     Lenormand\pwindex{Lenormand, Henri-René 3.\,5.\,1882 Paris – 16.\,2.\,1951 ebd.@\textsc{Lenormand, Henri-René} (3.\,5.\,1882 Paris – 16.\,2.\,1951 ebd.), \emph{Schriftsteller}|pwk}, 30. 7. 1925, \emph{Deutsches Literaturarchiv Marbach},
                  HS.1985.1.1280.}}}\label{K_L03960-3} habe, sich irgendwie geäussert? Bestehen Chancen
               hinsichtlich der »Liebelei\pwindex{Schnitzler, Arthur 15. 5. 1862 Wien – 21. 10. 1931 ebd.@\textsc{Schnitzler, Arthur} (15. 5. 1862 Wien – 21. 10. 1931 ebd.), \emph{Schriftsteller, Mediziner}!Liebelei. Schauspiel in drei Akten@\strich\emph{Liebelei. Schauspiel in drei Akten}|pw}«?\pend
           
\pstart
           Indess habe ich von Herrn Rémon\pwindex{Rémon, Maurice 27.\,11.\,1861 Paris – 20.\,6.\,1945 Mérignac@\textsc{Rémon, Maurice} (27.\,11.\,1861 Paris – 20.\,6.\,1945 Mérignac), \emph{Übersetzer}|pw}, der Ihnen ja
               schon bekannt ist, eine \label{K_L03960-4v}\edtext{Anfrage}{\lemma{\textnormal{\emph{Anfrage}}}\Cendnote{\textnormal{nicht überliefert}}}\label{K_L03960-4} sowohl
               hinsichtlich der »Literatur\pwindex{Schnitzler, Arthur 15. 5. 1862 Wien – 21. 10. 1931 ebd.@\textsc{Schnitzler, Arthur} (15. 5. 1862 Wien – 21. 10. 1931 ebd.), \emph{Schriftsteller, Mediziner}!Literatur@\strich\emph{Literatur}|pw}«, \label{K_L03960-5v}\edtext{die schon längst übersetzt ist}{\lemma{\textnormal{\emph{die … ist}}}\Cendnote{\textnormal{\emph{Littérature. Comédie en en act}\pwindex{Schnitzler, Arthur 15. 5. 1862 Wien – 21. 10. 1931 ebd.@\textsc{Schnitzler, Arthur} (15. 5. 1862 Wien – 21. 10. 1931 ebd.), \emph{Schriftsteller, Mediziner}!Littérature. Comédie en en act@\strich\emph{Littérature. Comédie en en act}|pwk}. In: \emph{La Revue bleue. La Revue politique et
                        littéraire}\pwindex{Revue politique et littéraire@\emph{La Revue politique et littéraire}|pwk}, Jahrgang 52, 1. Semester, Nr. 1, 3. 1. 1914, S. 11–16;
                     Nr. 2, S. 44–50.}}}\label{K_L03960-5} und hinsichtlich des »Kakadu\pwindex{Schnitzler, Arthur 15. 5. 1862 Wien – 21. 10. 1931 ebd.@\textsc{Schnitzler, Arthur} (15. 5. 1862 Wien – 21. 10. 1931 ebd.), \emph{Schriftsteller, Mediziner}!grüne Kakadu. Groteske in einem Akt@\strich\emph{Der grüne Kakadu. Groteske in einem Akt}|pw}\pwindex{Schnitzler, Arthur 15. 5. 1862 Wien – 21. 10. 1931 ebd.@\textsc{Schnitzler, Arthur} (15. 5. 1862 Wien – 21. 10. 1931 ebd.), \emph{Schriftsteller, Mediziner}!Au Perroquet Vert@\strich\emph{Au Perroquet Vert}|pw}« erhalten (der seinerzeit \label{K_L03960-6v}\edtext{bei Antoine\orgindex{Théâtre Antoine@Théâtre Antoine|pw} aufgeführt}{\lemma{\textnormal{\emph{bei Antoine aufgeführt}}}\Cendnote{\textnormal{ Die Die Premiere von \emph{Au
                        Perroquet Vert}\pwindex{Schnitzler, Arthur 15. 5. 1862 Wien – 21. 10. 1931 ebd.@\textsc{Schnitzler, Arthur} (15. 5. 1862 Wien – 21. 10. 1931 ebd.), \emph{Schriftsteller, Mediziner}!Au Perroquet Vert@\strich\emph{Au Perroquet Vert}|pwk}\eventindex{Théâtre Antoine-Simone Berriau@\textbf{Théâtre Antoine-Simone Berriau}!Premiere von Au Perroquet Vert, 7.11.1903@Premiere von Au Perroquet Vert, 7.11.1903|pwk} fand am 7. 11. 1903 am \emph{Théâtre Antoine}\orgindex{Théâtre Antoine@Théâtre Antoine|pwk} statt. }}}\label{K_L03960-6} wurde). Ich habe daran
               gedacht, ob man nicht eventuell »Kakadu\pwindex{Schnitzler, Arthur 15. 5. 1862 Wien – 21. 10. 1931 ebd.@\textsc{Schnitzler, Arthur} (15. 5. 1862 Wien – 21. 10. 1931 ebd.), \emph{Schriftsteller, Mediziner}!grüne Kakadu. Groteske in einem Akt@\strich\emph{Der grüne Kakadu. Groteske in einem Akt}|pw}«
               zusammen mit »Liebelei\pwindex{Schnitzler, Arthur 15. 5. 1862 Wien – 21. 10. 1931 ebd.@\textsc{Schnitzler, Arthur} (15. 5. 1862 Wien – 21. 10. 1931 ebd.), \emph{Schriftsteller, Mediziner}!Liebelei. Schauspiel in drei Akten@\strich\emph{Liebelei. Schauspiel in drei Akten}|pw}« oder »Literatur\pwindex{Schnitzler, Arthur 15. 5. 1862 Wien – 21. 10. 1931 ebd.@\textsc{Schnitzler, Arthur} (15. 5. 1862 Wien – 21. 10. 1931 ebd.), \emph{Schriftsteller, Mediziner}!Literatur@\strich\emph{Literatur}|pw}« mit »Liebelei\pwindex{Schnitzler, Arthur 15. 5. 1862 Wien – 21. 10. 1931 ebd.@\textsc{Schnitzler, Arthur} (15. 5. 1862 Wien – 21. 10. 1931 ebd.), \emph{Schriftsteller, Mediziner}!Liebelei. Schauspiel in drei Akten@\strich\emph{Liebelei. Schauspiel in drei Akten}|pw}« zur Aufführung bringen könnte. Sollte es Ihre Zeit erlauben, möchten
               Sie sich vielleicht mit Herrn Rémon\pwindex{Rémon, Maurice 27.\,11.\,1861 Paris – 20.\,6.\,1945 Mérignac@\textsc{Rémon, Maurice} (27.\,11.\,1861 Paris – 20.\,6.\,1945 Mérignac), \emph{Übersetzer}|pw}, Paris XVII, 10, rue Daubigny\oindex{10, Rue Daubigny@\textbf{10, Rue Daubigny}, \emph{Wohngebäude}|pw}, in Verbindung
               setzen?\pend
           
\pstart
           Auch wegen »Fräulein Else\pwindex{Schnitzler, Arthur 15. 5. 1862 Wien – 21. 10. 1931 ebd.@\textsc{Schnitzler, Arthur} (15. 5. 1862 Wien – 21. 10. 1931 ebd.), \emph{Schriftsteller, Mediziner}!Fräulein Else@\strich\emph{Fräulein Else}|pw}« habe ich einige
               Anfragen bekommen, ohne dass nach irgend einer Seite sehr bestimmte Aussichten
               bestünden. Auch die Herausgabe von »Casanovas
                  Heimfahrt\pwindex{Schnitzler, Arthur 15. 5. 1862 Wien – 21. 10. 1931 ebd.@\textsc{Schnitzler, Arthur} (15. 5. 1862 Wien – 21. 10. 1931 ebd.), \emph{Schriftsteller, Mediziner}!Casanovas Heimfahrt@\strich\emph{Casanovas Heimfahrt}|pw}«, für die M. Nathan\pwindex{Nathan, Nicolas @\textsc{Nathan, Nicolas}, \emph{Übersetzer}|pw} sogar
               schon Vorschuss gezahlt hat, zieht sich aus nicht ganz durchsichtigen Gründen länger
               hin, als ich vermuten konnte.\pend
           
\pstart
           Die neue \label{K_L03960-7v}\edtext{Uebersetzung\pwindex{Schnitzler, Arthur 15. 5. 1862 Wien – 21. 10. 1931 ebd.@\textsc{Schnitzler, Arthur} (15. 5. 1862 Wien – 21. 10. 1931 ebd.), \emph{Schriftsteller, Mediziner}!Mourir. Roman [1925]@\strich\emph{Mourir. Roman [1925]}|pwv} von »Sterben\pwindex{Schnitzler, Arthur 15. 5. 1862 Wien – 21. 10. 1931 ebd.@\textsc{Schnitzler, Arthur} (15. 5. 1862 Wien – 21. 10. 1931 ebd.), \emph{Schriftsteller, Mediziner}!Sterben. Novelle@\strich\emph{Sterben. Novelle}|pw}«}{\lemma{\textnormal{\emph{Uebersetzung von »Sterben«}}}\Cendnote{\textnormal{Arthur Schnitzler: \emph{Mourir. Roman}\pwindex{Schnitzler, Arthur 15. 5. 1862 Wien – 21. 10. 1931 ebd.@\textsc{Schnitzler, Arthur} (15. 5. 1862 Wien – 21. 10. 1931 ebd.), \emph{Schriftsteller, Mediziner}!Mourir. Roman [1925]@\strich\emph{Mourir. Roman [1925]}|pwk}. Traduit par Alzir Hella\pwindex{Hella, Alzir 30.\,12.\,1881 Vieux Condé – 14.\,7.\,1953 Paris@\textsc{Hella, Alzir} (30.\,12.\,1881 Vieux Condé – 14.\,7.\,1953 Paris), \emph{Übersetzer}|pwk} et O.
                        Bournac\pwindex{Bournac, Olivier 13.\,8.\,1885 Saint-Amans-du-Pech – Anfang Januar 1931 Toulon@\textsc{Bournac, Olivier} (13.\,8.\,1885 Saint-Amans-du-Pech – Anfang Januar 1931 Toulon), \emph{Schriftsteller, Übersetzer}|pwk}. Avant-propos de Maurice
                        Scheyer\pwindex{Scheyer, Moriz 27.\,12.\,1886 Focşani – 29.\,3.\,1949 Belvès@\textsc{Scheyer, Moriz} (27.\,12.\,1886 Focşani – 29.\,3.\,1949 Belvès), \emph{Schriftsteller, Journalist}|pwk}. Paris: \emph{F. Rieder}{ }1925.}}}\label{K_L03960-7} soll in Frankreich\oindex{Frankreich@\textbf{Frankreich}|pw} Erfolg
               gehabt {\pb}haben; ich selbst habe nichts darüber zu lesen
               bekommen.\pend
           
\pstart
           Ich habe einen ganz schönen und ziemlich reichhaltigen Sommer verlebt, war in den Dolomiten\oindex{Dolomiten@\textbf{Dolomiten}, \emph{Gebirge}|pw}, im Engadin\oindex{Engadin@\textbf{Engadin}, \emph{Tal}|pw}, in Forte dei Marmi\oindex{Forte dei Marmi@\textbf{Forte dei Marmi}, \emph{Hauptstadt}|pw} (am tyrännischen Meer\oindex{Tyrrhenisches Meer@\textbf{Tyrrhenisches Meer}|pw}), in Florenz\oindex{Florenz@\textbf{Florenz}|pw} und Venedig\oindex{Venedig@\textbf{Venedig}|pw}, Olga\pwindex{Schnitzler, Olga 17.\,1.\,1882 Wien – 13.\,1.\,1970 Lugano@\textsc{Schnitzler, Olga} (17.\,1.\,1882 Wien – 13.\,1.\,1970 Lugano), \emph{Schauspielerin, Sängerin}|pw} mit Lili\pwindex{Cappellini, Lili 13.\,9.\,1909 Wien – 26.\,7.\,1928 Venedig@\textsc{Cappellini, Lili} (13.\,9.\,1909 Wien – 26.\,7.\,1928 Venedig)|pw} sind noch dort. Alma\pwindex{Mahler-Werfel, Alma Maria 31.\,8.\,1879 Wien – 11.\,12.\,1964 New York City@\textsc{Mahler-Werfel, Alma Maria} (31.\,8.\,1879 Wien – 11.\,12.\,1964 New York City)|pw} habe ich
               vortrefflich aussehend und eigentlich in guter Stimmung gefunden.\pend
           
\pstart
           Zwischen 20. und 25. Oktober hoffe ich wieder in Wien\oindex{Wien@\textbf{Wien}, \emph{Verwaltungsgebiet}|pw} zu sein. Nach Berlin\oindex{Berlin@\textbf{Berlin}, \emph{Hauptstadt}|pw} fahre ich hauptsächlich wegen Heini\pwindex{Schnitzler, Heinrich 9.\,8.\,1902 Hinterbrühl – 12.\,7.\,1982 Wien@\textsc{Schnitzler, Heinrich} (9.\,8.\,1902 Hinterbrühl – 12.\,7.\,1982 Wien), \emph{Regisseur, Schauspieler}|pw} (der nächstens den Theodor in der »Liebelei\pwindex{Schnitzler, Arthur 15. 5. 1862 Wien – 21. 10. 1931 ebd.@\textsc{Schnitzler, Arthur} (15. 5. 1862 Wien – 21. 10. 1931 ebd.), \emph{Schriftsteller, Mediziner}!Liebelei. Schauspiel in drei Akten@\strich\emph{Liebelei. Schauspiel in drei Akten}|pw}« spielt); auch soll die »Komödie
                  der Verführung\pwindex{Schnitzler, Arthur 15. 5. 1862 Wien – 21. 10. 1931 ebd.@\textsc{Schnitzler, Arthur} (15. 5. 1862 Wien – 21. 10. 1931 ebd.), \emph{Schriftsteller, Mediziner}!Komödie der Verführung. In drei Akten@\strich\emph{Komödie der Verführung. In drei Akten}|pw}« bei Barnowsky\pwindex{Barnowsky, Victor 10.\,9.\,1875 Berlin – 9.\,8.\,1952 New York City@\textsc{Barnowsky, Victor} (10.\,9.\,1875 Berlin – 9.\,8.\,1952 New York City), \emph{Theaterleiter, Regisseur, Schauspieler}|pw}
               aufgeführt werden, aber wie die Situation sich bisher gestaltet, insbesondere
               hinsichtlich Besetzung etc., werde ich zum mindesten einen Aufschub verlangen.\pend
           
\pstart
           \label{K_L03960-8v}\edtext{Wann kommen Sie wieder}{\lemma{\textnormal{\emph{Wann kommen Sie wieder}}}\Cendnote{\textnormal{Das erste Treffen zwischen Schnitzler und Zuckerkandl\pwindex{Zuckerkandl, Berta 13.\,4.\,1864 Wien – 16.\,10.\,1945 Paris@\textsc{Zuckerkandl, Berta} (13.\,4.\,1864 Wien – 16.\,10.\,1945 Paris), \emph{Schriftstellerin, Journalistin, Übersetzerin}|pwk} nach ihrer Reise fand laut \emph{Tagebuch}\pwindex{Schnitzler, Arthur 15. 5. 1862 Wien – 21. 10. 1931 ebd.@\textsc{Schnitzler, Arthur} (15. 5. 1862 Wien – 21. 10. 1931 ebd.), \emph{Schriftsteller, Mediziner}!Tagebuch@\strich\emph{Tagebuch}|pwk} am 4. 12. 1925 statt.}}}\label{K_L03960-8} nach Wien\oindex{Wien@\textbf{Wien}, \emph{Verwaltungsgebiet}|pw} zu{[}rü{]}ck, liebste Frau Hofrätin? Ich
               hoffe, Sie befinden sich wohl und werde sehr froh sein, wenn ich bald etwas von Ihnen
               höre. In Berlin\oindex{Berlin@\textbf{Berlin}, \emph{Hauptstadt}|pw} werde ich voraussichtlich im
               Hotel Esplanade\oindex{Hotel Esplanade [Berlin]@\textbf{Hotel Esplanade [Berlin]}, \emph{Hotel}|pw} wohnen. Darf ich bitten mich
               Ihrer verehrten Frau Schwester\pwindex{Clemenceau, Sophie 25.\,5.\,1862 – 24.\,9.\,1937@\textsc{Clemenceau, Sophie} (25.\,5.\,1862 – 24.\,9.\,1937)|pwv}, sowie auch Lenormand\pwindex{Lenormand, Henri-René 3.\,5.\,1882 Paris – 16.\,2.\,1951 ebd.@\textsc{Lenormand, Henri-René} (3.\,5.\,1882 Paris – 16.\,2.\,1951 ebd.), \emph{Schriftsteller}|pw}, Paul Geraldy\pwindex{Géraldy, Paul 6.\,3.\,1885 Paris – 9.\,3.\,1983 Neuilly-sur-Seine@\textsc{Géraldy, Paul} (6.\,3.\,1885 Paris – 9.\,3.\,1983 Neuilly-sur-Seine), \emph{Schriftsteller}|pw} und Frau\pwindex{Géraldy, Paul 6.\,3.\,1885 Paris – 9.\,3.\,1983 Neuilly-sur-Seine@\textsc{Géraldy, Paul} (6.\,3.\,1885 Paris – 9.\,3.\,1983 Neuilly-sur-Seine), \emph{Schriftsteller}|pwv} und wer sich sonst meine freundlich in
                  Paris\oindex{Paris@\textbf{Paris}, \emph{Hauptstadt}|pw} erinnert, bestens zu empfehlen.\pend
           
\pstart
           Mit den herzlicnsten Grüssen{\\[\baselineskip]} Ihr\pend
           \leftskip=0em{}{\vspace{1\baselineskip}}
\pstart
           \noindent{}Frau Hofrätin Berta Zuckerkandl,{\\}Paris\oindex{Paris@\textbf{Paris}, \emph{Hauptstadt}|pw}.\pend
           \selectlanguage{ngerman}\endnumbering\briefempfaengerindex{Zuckerkandl, Berta@\textsc{Zuckerkandl, Berta}!zzzSchnitzler, Arthur@\emph{von Arthur Schnitzler}!1925-10-081@{8. 10. 1925}|)be}\mylabel{L03960h}
\begin{anhang}
\end{anhang}\newcommand{\dateiname}{L03960}\newcommand{\titel}{Arthur Schnitzler an Berta Zuckerkandl, 8. 10. 1925}\newcommand{\editorInnen}{Herausgegeben von Jahnke, SelmaMüller, Martin Anton}%% latex-leseansicht-abspann.tex
%% Abspann für die Leseansicht.
%% Der Schalter \ifkorrekturansicht ist bereits durch den Vorspann gesetzt.

%% latex-abspann.tex
%% Gemeinsamer Abspann für Korrekturansicht und Leseansicht.
%% Setzt den Schalter \ifkorrekturansicht voraus (gesetzt in den
%% einbindenden Dateien latex-korrekturansicht-abspann.tex bzw.
%% latex-leseansicht-abspann.tex).
%% ---------------------------------------------------------------

\normalsize

% Das esempio-Environment wird nur in der Leseansicht benötigt
\ifkorrekturansicht\else
\newenvironment{esempio}[3]%
{
    \vspace{1.5ex}
    \rlap{\underline{#1}}
    \par
    \setlength{\parindent}{0cm}
    \nopagebreak
    \leftskip=#2cm
    \rightskip=#3cm
}
{
    \par
}
\fi

\doendnotes{C}
\bigskip
\vfill

\clearpage

\footnotesize

\ifkorrekturansicht
  \lohead{\textsc{register}}
\fi

% theindex-Environment neu definieren ohne reledmac
\makeatletter
\renewenvironment{theindex}{%
  \ifkorrekturansicht
    \section*{\indexname}%
  \else
    \subsubsection*{Index der erwähnten Entitäten}%
  \fi
  \setlength{\parindent}{0pt}%
  \setlength{\parskip}{0pt plus 0.3pt}%
  \let\item\@idxitem
}{%
  \ifkorrekturansicht\clearpage\fi
}
\makeatother

\IfFileExists{\jobname-pw.ind}{\input{\jobname-pw.ind}}{}

% Quellenangabe nur in der Leseansicht
\ifkorrekturansicht\else
% Fallback-Definitionen, falls die .tex-Datei \titel etc. nicht gesetzt hat
\providecommand{\titel}{}
\providecommand{\editorInnen}{}
\providecommand{\dateiname}{\jobname}

\vspace{3cm}

\vfill

\footnotesize
\textsc{Quelle}: \titel. Herausgegeben von {\editorInnen}. In: \emph{Arthur Schnitzler: Briefwechsel mit Autorinnen und Autoren}.
 Digitale Edition, https://schnitzler-briefe.acdh.oeaw.ac.at/{\dateiname}.html (Stand \today)
\fi

\end{document}


