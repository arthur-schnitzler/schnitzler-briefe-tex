%% latex-leseansicht-vorspann.tex
%% Vorspann für die Leseansicht.
%% Lädt die gemeinsame Datei latex-vorspann.tex mit nicht gesetztem Schalter.

\newif\ifkorrekturansicht
\korrekturansichtfalse

\input{../tex-inputs/latex-vorspann}


         
         \renewcommand{\erwaehntePersonen}{Personen: Hugo von Hofmannsthal, Felix Salten, Ottilie Salten, Olga Schnitzler, Heinrich Schnitzler}
         \renewcommand{\erwaehnteOrte}{Orte: Wien}
         \renewcommand{\erwaehnteWerke}{}
               \section[ Felix Salten an Arthur Schnitzler, {[}12. 10. 1903{]}]{ Felix Salten an Arthur Schnitzler, {[}12. 10. 1903{]}}\nopagebreak\mylabel{v}\rehead{ }\begin{ledgroupsized}[t]{13cm}\normalsize\beginnumbering\briefempfaengerindex{Schnitzler, Arthur@\textsc{Schnitzler, Arthur}!zzzSalten, Felix@\emph{von Felix Salten}!1903-10-122@{{[}12. 10. 1903{]}}|(be} \toendnotes[C]{\smallbreak\pagebreak[2]} \Standort{CUL, Schnitzler, B 89, A 2.}
\physDesc{Brief, 1 Blatt, 2 Seiten, 1657 Zeichen
\newline{}Handschrift: Bleistift, lateinische Kurrent
\newline{}Schnitzler: mit Bleistift auf den »12/10 903« datiert und »\textsc{Salten}« vermerkt 
\newline{}Ordnung: mit Bleistift von unbekannter Hand nummeriert: »172« }\toendnotes[C]{\smallbreak}\pstart
           \raggedleft{}{\pb}Montag.\pend
           \pstart
           Lieber, ich wollte allerdings morgen
               lesen, bin aber so in Anspruch genommen, dass ichs vorderhand \label{K_L03347-1v}\edtext{auf Freitag
                  laßen}{\lemma{\textnormal{\emph{auf Freitag
                  laßen}}}\Cendnote{\textnormal{auch dieser Termin wurde
                  verschoben, vgl. Felix Salten an Arthur Schnitzler, 14. [10. 1903]}}}\label{K_L03347-1h} muß, wovon ich H.\pwindex{Hofmannsthal, Hugo von 1874-02-01 – 1929-07-15@\textsc{Hofmannsthal, Hugo von} (1874-02-01 – 1929-07-15), \emph{Schriftsteller}|pw}{ }heute verständigt habe. Ich wußte das schon gestern, sonst hätte ich Ihnen gestern geschrieben. Gerne kommen wir zu Ihnen, wenn Sie uns einen Tag
               vorschlagen. Aber dass man einander entgleitet, hat andere Ursachen. Denn wiewol ich
               sehr beschäftigt bin, fände ich doch Zeit genug, an dem Verkehr des \label{K_L03347-2v}\edtext{alten Kreises}{\lemma{\textnormal{\emph{alten Kreises}}}\Cendnote{\textnormal{Zur Bestimmung, wen Salten\pwindex{Salten, Felix 06.09.1869 – 08.10.1945@\textsc{Salten, Felix} (06.09.1869 – 08.10.1945), \emph{Schriftsteller, Journalist}|pwk} für den »Kreis« hält, vgl. A. S.: \emph{Tagebuch}, 9. 10. 1891. Um die Jahrhundertwende, mitverursacht
                  durch die jeweiligen Familiengründungen und Übersiedlungen in Wien\oindex{Wien@\textbf{Wien}|pwk}er Außenbezirke, dürften die bis dahin von selbst
                  ergebenden Zusammentreffen in Kaffeehäusern seltener geworden sein. In seinem
                  Antwortschreiben (Arthur Schnitzler an Felix Salten, 12. 10. [1903]) bestätigt das
                     Schnitzler\pwindex{Schnitzler, Arthur 15.05.1862 – 21.10.1931@\textsc{Schnitzler, Arthur} (15.05.1862 – 21.10.1931), \emph{Schriftsteller, Mediziner}|pwk}, indem er argumentiert, dass
                  es die Gruppe als Ganzes nicht mehr gebe. Es läge also nicht daran, dass Salten\pwindex{Salten, Felix 06.09.1869 – 08.10.1945@\textsc{Salten, Felix} (06.09.1869 – 08.10.1945), \emph{Schriftsteller, Journalist}|pwk} ausgeschlossen wäre. }}}\label{K_L03347-2h} theil zu
               nehmen. Dieser geht jedoch seit langem ohne mich vor sich. Was Sie heute zum ersten Mal bemerken, und als höchst ärgerlich
               bezeichnen, dass habe ich so oft und oft constatirt, dass ich schon aufgehört habe,
               es zu beobachten. So wenig ich das herbeigeführt habe, so wenig innere und äußere
               Eignung besitze ich, das heute noch zu ändern. Es fällt mir auch nicht im Mindesten
               ein, die Dinge zu einer absolut nutzlosen Discussion zu stellen, und \uline{bitte Sie ernstlich davon abzusehen}. Nur hätte ich
               Ihre Bemerkung mit einer ähnlichen quittiren müßen, und das erscheint mir unmöglich,
               weil es meinerseits nicht aufrichtig wäre. So hab ich Ihnen lieber gleich gesagt, was
               ich seit langem denke, ohne damit das geringste zu bezwecken. Reden hilft ja in
               solchen Fällen nichts, – es beseitigt nur \textcolor{gray}{U}nklarheiten. Und ich
               hätte, wenn ich nicht dadurch die Situation selbst weiter im Unklaren gelaßen hätte,
               sicher auch weiter nichts gesagt.\pend
           \pstart
           {\pb}Was die Vorlesung betrifft,
               bitte ich Sie sehr, sich für Freitag frei zu halten,
               oder, wenn dieser Tag nicht geht, es mir gleich zu schreiben\textcolor{gray}{.}{ }H.\pwindex{Hofmannsthal, Hugo von 1874-02-01 – 1929-07-15@\textsc{Hofmannsthal, Hugo von} (1874-02-01 – 1929-07-15), \emph{Schriftsteller}|pw} möchte, dass wir dann punct 5.
               beginnen, weil er um ½ 11 fort muß.\pend
           \pstart
           Mit herzlichsten Grüßen von uns Beiden\pwindex{Salten, Ottilie 07.03.1868 – 22.06.1942@\textsc{Salten, Ottilie} (07.03.1868 – 22.06.1942), \emph{Schauspielerin}|pwv} an Ihre Frau\pwindex{Schnitzler, Olga 17.01.1882 – 13.01.1970@\textsc{Schnitzler, Olga} (17.01.1882 – 13.01.1970), \emph{Schauspielerin, Sängerin}|pwv},
               den kl. Buben\pwindex{Schnitzler, Heinrich 09.08.1902 – 12.07.1982@\textsc{Schnitzler, Heinrich} (09.08.1902 – 12.07.1982), \emph{Regisseur, Schauspieler}|pwv} und Sie
               {\\}Ihr {\\}\spacefill\mbox{Salten}\pend
           
         
         \endnumbering\mylabel{h}\end{ledgroupsized}  \newcommand{\dateiname}{L03347}\newcommand{\titel}{Felix Salten an Arthur Schnitzler, [12. 10. 1903]}\newcommand{\editorInnen}{Martin Anton Müller und Laura Untner}%% latex-leseansicht-abspann.tex
%% Abspann für die Leseansicht.
%% Der Schalter \ifkorrekturansicht ist bereits durch den Vorspann gesetzt.

%% latex-abspann.tex
%% Gemeinsamer Abspann für Korrekturansicht und Leseansicht.
%% Setzt den Schalter \ifkorrekturansicht voraus (gesetzt in den
%% einbindenden Dateien latex-korrekturansicht-abspann.tex bzw.
%% latex-leseansicht-abspann.tex).
%% ---------------------------------------------------------------

\normalsize

% Das esempio-Environment wird nur in der Leseansicht benötigt
\ifkorrekturansicht\else
\newenvironment{esempio}[3]%
{
    \vspace{1.5ex}
    \rlap{\underline{#1}}
    \par
    \setlength{\parindent}{0cm}
    \nopagebreak
    \leftskip=#2cm
    \rightskip=#3cm
}
{
    \par
}
\fi

\doendnotes{C}
\bigskip
\vfill

\clearpage

\footnotesize

\ifkorrekturansicht
  \lohead{\textsc{register}}
\fi

% theindex-Environment neu definieren ohne reledmac
\makeatletter
\renewenvironment{theindex}{%
  \ifkorrekturansicht
    \section*{\indexname}%
  \else
    \subsubsection*{Index der erwähnten Entitäten}%
  \fi
  \setlength{\parindent}{0pt}%
  \setlength{\parskip}{0pt plus 0.3pt}%
  \let\item\@idxitem
}{%
  \ifkorrekturansicht\clearpage\fi
}
\makeatother

\IfFileExists{\jobname-pw.ind}{\input{\jobname-pw.ind}}{}

% Quellenangabe nur in der Leseansicht
\ifkorrekturansicht\else
% Fallback-Definitionen, falls die .tex-Datei \titel etc. nicht gesetzt hat
\providecommand{\titel}{}
\providecommand{\editorInnen}{}
\providecommand{\dateiname}{\jobname}

\vspace{3cm}

\vfill

\footnotesize
\textsc{Quelle}: \titel. Herausgegeben von {\editorInnen}. In: \emph{Arthur Schnitzler: Briefwechsel mit Autorinnen und Autoren}.
 Digitale Edition, https://schnitzler-briefe.acdh.oeaw.ac.at/{\dateiname}.html (Stand \today)
\fi

\end{document}


      