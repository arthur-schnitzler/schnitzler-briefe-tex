%% latex-korrekturansicht-vorspann.tex
%% Vorspann für die Korrekturansicht.
%% Lädt die gemeinsame Datei latex-vorspann.tex mit gesetztem Schalter.

\newif\ifkorrekturansicht
\korrekturansichttrue

\input{../tex-inputs/latex-vorspann}


\section[Arthur Schnitzler an Hugo von Hofmannsthal, {[}15. 5. 1894?{]}]{L00324 Arthur Schnitzler an Hugo von Hofmannsthal, {[}15. 5. 1894?{]}}
\nopagebreak\mylabel{L00324v}
\rehead{ }\normalsize\beginnumbering\briefempfaengerindex{Hofmannsthal, Hugo von@\textsc{Hofmannsthal, Hugo von}!zzzSchnitzler, Arthur@\emph{von Arthur Schnitzler}!1894-05-152@{{[}15. 5. 1894?{]}}|(be}
\toendnotes[C]{\smallbreak\pagebreak[2]}\Standort{FDH, Hs-30885,31.}
\physDesc{Briefkarte, 318 Zeichen
\newline{}Handschrift: schwarze Tinte, deutsche Kurrent}
\buchAbdrucke{\weitereDrucke{Hugo von Hofmannsthal, Arthur Schnitzler: \emph{Briefwechsel}. Frankfurt am Main: \emph{S. Fischer} 1964, S. 32.} }\toendnotes[C]{\smallbreak}
\pstart
           \noindent{}{\pb}Lieber Hugo!{ }Fels\pwindex{Fels, Friedrich Michael *~1864@\textsc{Fels, Friedrich Michael} (*~1864), \emph{Journalist/Journalistin}|pw} hat ſich wieder gemeldet. Können Sie im
               Lauf \label{K_L00324-1v}\edtext{dieſes Monats}{\lemma{\textnormal{\emph{dieſes Monats}}}\Cendnote{\textnormal{Die Einordnung des undatierten Korrespondenzstücks bereitet
                  Probleme. Der Februar 1893, in dem die Hilfe für Fels\pwindex{Fels, Friedrich Michael *~1864@\textsc{Fels, Friedrich Michael} (*~1864), \emph{Journalist/Journalistin}|pwk} zentral in der Korrespondenz ist, scheint sich durch
                  die Mitteilung der Wohnadresse in der Exnerstraße\oindex{Kruetznergasse@\textbf{Krütznergasse}, \emph{Straße (K.STR)}|pwk} auszuschließen, da Hofmannsthal\pwindex{Hofmannsthal, Hugo von 1874-02-01 – 1929-07-15@\textsc{Hofmannsthal, Hugo von} (1874-02-01 – 1929-07-15), \emph{Schriftsteller/Schriftstellerin}|pwk} am 9. 2. 1893 explizit nach der Adresse fragt, dieses
                  Korrespondenzstück aber nicht die Antwort darauf ist. Hingegen kann der Brief
                     Schnitzlers an Beer-Hofmann\pwindex{Beer-Hofmann, Richard 1866-07-11 – 1945-09-26@\textsc{Beer-Hofmann, Richard} (1866-07-11 – 1945-09-26), \emph{Schriftsteller/Schriftstellerin}|pwk} vom 15. 5. 1894 – in dem er um Hilfe für Fels\pwindex{Fels, Friedrich Michael *~1864@\textsc{Fels, Friedrich Michael} (*~1864), \emph{Journalist/Journalistin}|pwk} bittet und dessen Adresse mitteilt, als Hinweis
                  genommen werden, dass auch dieses Korrespondenzstück an diesem Tag verfasst
                  worden ist.}}}\label{K_L00324-1} noch was thun, ſo wäre es ihm, ja auch mir recht angenehm. Er wohnt,
               für alle Fälle ſei es Ihnen mitgetheilt, \textsc{XVIII. Exnerstraße 3\oindex{Kruetznergasse@\textbf{Krütznergasse}, \emph{Straße (K.STR)}|pw}}. Es ſcheint wirklich, dß er vom nächſten Monat {\pb}an
               nicht auf uns mehr angewieſen ſein wird.\pend
           
\pstart
           Herzliche Grüße.{\\[\baselineskip]}Ihr \spacefill\mbox{Arthur}\pend
           \leftskip=0em{}\selectlanguage{ngerman}\endnumbering\briefempfaengerindex{Hofmannsthal, Hugo von@\textsc{Hofmannsthal, Hugo von}!zzzSchnitzler, Arthur@\emph{von Arthur Schnitzler}!1894-05-152@{{[}15. 5. 1894?{]}}|)be}\mylabel{L00324h}  \normalsize

\doendnotes{C}
\bigskip
\vfill

\clearpage

\footnotesize

\lohead{\textsc{register}}

% Definiere theindex-Environment komplett neu ohne reledmac
\makeatletter
\renewenvironment{theindex}{%
  \section*{\indexname}%
  \setlength{\parindent}{0pt}%
  \setlength{\parskip}{0pt plus 0.3pt}%
  \let\item\@idxitem
}{%
  \clearpage
}
\makeatother

\IfFileExists{\jobname-pw.ind}{\input{\jobname-pw.ind}}{}

\end{document}

      