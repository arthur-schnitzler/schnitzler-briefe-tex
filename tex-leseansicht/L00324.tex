%% latex-leseansicht-vorspann.tex
%% Vorspann für die Leseansicht.
%% Lädt die gemeinsame Datei latex-vorspann.tex mit nicht gesetztem Schalter.

\newif\ifkorrekturansicht
\korrekturansichtfalse

\input{../tex-inputs/latex-vorspann}


         
         \renewcommand{\erwaehntePersonen}{Personen: Richard Beer-Hofmann, Friedrich Michael Fels, Hugo von Hofmannsthal}
         \renewcommand{\erwaehnteOrte}{Orte: Krütznergasse, Wien}
         \renewcommand{\erwaehnteWerke}{}
               \section[Arthur Schnitzler an Hugo von Hofmannsthal, {[}15. 5. 1894?{]}]{ Arthur Schnitzler an Hugo von Hofmannsthal, {[}15. 5. 1894?{]}}\nopagebreak\mylabel{v}\rehead{ }\begin{ledgroupsized}[t]{13cm}\normalsize\beginnumbering \toendnotes[C]{\smallbreak\pagebreak[2]} \Standort{FDH, Hs-30885,31.}
\physDesc{Briefkarte, 317 Zeichen
\newline{}Handschrift: schwarze Tinte, deutsche Kurrent}\buchAbdrucke{\weitereDrucke{Hugo von Hofmannsthal, Arthur Schnitzler: \emph{Briefwechsel}. Hg. Therese Nickl und Heinrich Schnitzler. Frankfurt am Main: \emph{S. Fischer} 1964, S. 32.} }\toendnotes[C]{\smallbreak}\pstart
           \noindent{}{\pb}Lieber Hugo! Fels\pwindex{Fels, Friedrich Michael *~1864@\textsc{Fels, Friedrich Michael} (*~1864), \emph{Journalist}|pw} hat ſich wieder gemeldet. Können Sie im
               Lauf \label{K_L00324_1v}\edtext{dieſes Monats}{\lemma{\textnormal{\emph{dieſes Monats}}}\Cendnote{\textnormal{Die Einordnung des undatierten Stückes ist
                  schwierig. Der Februar 1893, in dem die Hilfe für Fels\pwindex{Fels, Friedrich Michael *~1864@\textsc{Fels, Friedrich Michael} (*~1864), \emph{Journalist}|pwk} zentral in der Korrespondenz ist, scheint sich durch
                  die Mitteilung der Wohnadresse in der Exnerstraße\oindex{Kruetznergasse@\textbf{Krütznergasse}|pwk} auszuschließen, da Hofmannsthal\pwindex{Hofmannsthal, Hugo von 1874-02-01 – 1929-07-15@\textsc{Hofmannsthal, Hugo von} (1874-02-01 – 1929-07-15), \emph{Schriftsteller}|pwk} am 9. 2. 1893 explizit nach der Adresse fragt, dieses
                  Korrespondenzstück aber nicht die Antwort darauf ist. Hingegen können der Brief
                     Schnitzler\pwindex{Schnitzler, Arthur 15.05.1862 – 21.10.1931@\textsc{Schnitzler, Arthur} (15.05.1862 – 21.10.1931), \emph{Schriftsteller, Mediziner}|pwk}s an Beer-Hofmann\pwindex{Beer-Hofmann, Richard 1866-07-11 – 1945-09-26@\textsc{Beer-Hofmann, Richard} (1866-07-11 – 1945-09-26), \emph{Schriftsteller}|pwk} vom 15. 5. 1894 – in dem er um Hilfe für Fels\pwindex{Fels, Friedrich Michael *~1864@\textsc{Fels, Friedrich Michael} (*~1864), \emph{Journalist}|pwk} bittet und dessen Adresse mitteilt, als Hinweis
                  genommen werden, dass auch dieses Korrespondenzstück an diesem Tag verfasst
                  ist.}}}\label{K_L00324_1h} noch was thun, ſo wäre es ihm, ja auch mir recht angenehm. Er wohnt,
               für alle Fälle ſei es Ihnen mitgetheilt, \textsc{XVIII. Exnerstraße 3\oindex{Kruetznergasse@\textbf{Krütznergasse}|pw}}. Es ſcheint wirklich, dß er vom nächſten Monat {\pb}an
               nicht auf uns mehr angewieſen ſein wird.\pend
           \pstart
           Herzliche Grüße.{\\[\baselineskip]}Ihr \spacefill\mbox{Arthur}\pend
           \leftskip=0em{}
         
         \endnumbering\mylabel{h}\end{ledgroupsized}  \newcommand{\dateiname}{L00324}\newcommand{\titel}{Arthur Schnitzler an Hugo von Hofmannsthal, [15. 5. 1894?]}\newcommand{\editorInnen}{Martin Anton Müller und Gerd-Hermann Susen}%% latex-leseansicht-abspann.tex
%% Abspann für die Leseansicht.
%% Der Schalter \ifkorrekturansicht ist bereits durch den Vorspann gesetzt.

%% latex-abspann.tex
%% Gemeinsamer Abspann für Korrekturansicht und Leseansicht.
%% Setzt den Schalter \ifkorrekturansicht voraus (gesetzt in den
%% einbindenden Dateien latex-korrekturansicht-abspann.tex bzw.
%% latex-leseansicht-abspann.tex).
%% ---------------------------------------------------------------

\normalsize

% Das esempio-Environment wird nur in der Leseansicht benötigt
\ifkorrekturansicht\else
\newenvironment{esempio}[3]%
{
    \vspace{1.5ex}
    \rlap{\underline{#1}}
    \par
    \setlength{\parindent}{0cm}
    \nopagebreak
    \leftskip=#2cm
    \rightskip=#3cm
}
{
    \par
}
\fi

\doendnotes{C}
\bigskip
\vfill

\clearpage

\footnotesize

\ifkorrekturansicht
  \lohead{\textsc{register}}
\fi

% theindex-Environment neu definieren ohne reledmac
\makeatletter
\renewenvironment{theindex}{%
  \ifkorrekturansicht
    \section*{\indexname}%
  \else
    \subsubsection*{Index der erwähnten Entitäten}%
  \fi
  \setlength{\parindent}{0pt}%
  \setlength{\parskip}{0pt plus 0.3pt}%
  \let\item\@idxitem
}{%
  \ifkorrekturansicht\clearpage\fi
}
\makeatother

\IfFileExists{\jobname-pw.ind}{\input{\jobname-pw.ind}}{}

% Quellenangabe nur in der Leseansicht
\ifkorrekturansicht\else
% Fallback-Definitionen, falls die .tex-Datei \titel etc. nicht gesetzt hat
\providecommand{\titel}{}
\providecommand{\editorInnen}{}
\providecommand{\dateiname}{\jobname}

\vspace{3cm}

\vfill

\footnotesize
\textsc{Quelle}: \titel. Herausgegeben von {\editorInnen}. In: \emph{Arthur Schnitzler: Briefwechsel mit Autorinnen und Autoren}.
 Digitale Edition, https://schnitzler-briefe.acdh.oeaw.ac.at/{\dateiname}.html (Stand \today)
\fi

\end{document}


      