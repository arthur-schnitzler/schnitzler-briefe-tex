%% latex-leseansicht-vorspann.tex
%% Vorspann für die Leseansicht.
%% Lädt die gemeinsame Datei latex-vorspann.tex mit nicht gesetztem Schalter.

\newif\ifkorrekturansicht
\korrekturansichtfalse

\input{../tex-inputs/latex-vorspann}


\section[Arthur Schnitzler an Hugo von Hofmannsthal, {{[}}15. 5. 1894?{{]}}]{L00324 Arthur Schnitzler an Hugo von Hofmannsthal, {[}15. 5. 1894?{]}}
\nopagebreak\mylabel{L00324v}
\rehead{ }\normalsize\beginnumbering\briefempfaengerindex{Hofmannsthal, Hugo von@\textsc{Hofmannsthal, Hugo von}!zzzSchnitzler, Arthur@\emph{von Arthur Schnitzler}!1894-05-152@{{[}15. 5. 1894?{]}}|(be}
\toendnotes[C]{\smallbreak\pagebreak[2]}
\correspDesc{Versand  durch Arthur Schnitzler am [15. 5. 1894?] in Wien
\newline{}Erhalt  durch Hugo von Hofmannsthal im Zeitraum [15. 5. 1894
                  – 19. 5. 1894?] in Wien}\toendnotes[C]{\smallbreak}
\Standort{FDH, Hs-30885,31.}
\physDesc{Briefkarte, 318 Zeichen
\newline{}Handschrift: schwarze Tinte, deutsche Kurrent}
\buchAbdrucke{\weitereDrucke{Hugo von Hofmannsthal, Arthur Schnitzler: \emph{Briefwechsel}. Herausgegeben von Therese Nickl und Heinrich Schnitzler. Frankfurt am Main: \emph{S. Fischer} 1964, S. 32.} }\toendnotes[C]{\smallbreak}
\pstart
           \noindent{}{\pb}Lieber Hugo!{ }Fels\pwindex{Fels, Friedrich Michael *~1864 Bad Dürkheim@\textsc{Fels, Friedrich Michael} (*~1864 Bad Dürkheim), \emph{Journalist}|pw} hat{ }ſich wieder gemeldet. Können Sie im
               Lauf \label{K_L00324-1v}\edtext{dieſes Monats}{\lemma{\textnormal{\emph{dieses Monats}}}\Cendnote{\textnormal{Die Einordnung des undatierten Korrespondenzstücks bereitet
                  Probleme. Der Februar 1893, in dem die Hilfe für Fels\pwindex{Fels, Friedrich Michael *~1864 Bad Dürkheim@\textsc{Fels, Friedrich Michael} (*~1864 Bad Dürkheim), \emph{Journalist}|pwk} zentral in der Korrespondenz ist, scheint sich durch
                  die Mitteilung der Wohnadresse in der Exnerstraße\oindex{Wien@\textbf{Wien}!XVIII., Währing@\textbf{XVIII., Währing}!Krütznergasse@\textbf{Krütznergasse}, \emph{Straße}|pwk} auszuschließen, da Hofmannsthal\pwindex{Hofmannsthal, Hugo von 1.\,2.\,1874 Wien – 15.\,7.\,1929 Rodaun@\textsc{Hofmannsthal, Hugo von} (1.\,2.\,1874 Wien – 15.\,7.\,1929 Rodaun), \emph{Schriftsteller}|pwk} am XXXX Auszeichnungsfehler: Dokument L00174 nicht gefunden explizit nach der Adresse fragt, dieses
                  Korrespondenzstück aber nicht die Antwort darauf ist. Hingegen kann der Brief
                     Schnitzlers an Beer-Hofmann\pwindex{Beer-Hofmann, Richard 11.\,7.\,1866 Wien – 26.\,9.\,1945 New York City@\textsc{Beer-Hofmann, Richard} (11.\,7.\,1866 Wien – 26.\,9.\,1945 New York City), \emph{Schriftsteller}|pwk} vom XXXX Auszeichnungsfehler: Dokument L00323 nicht gefunden – in dem er um Hilfe für Fels\pwindex{Fels, Friedrich Michael *~1864 Bad Dürkheim@\textsc{Fels, Friedrich Michael} (*~1864 Bad Dürkheim), \emph{Journalist}|pwk} bittet und dessen Adresse mitteilt, als Hinweis
                  genommen werden, dass auch dieses Korrespondenzstück an diesem Tag verfasst
                  worden ist.}}}\label{K_L00324-1} noch was thun,{ }ſo wäre es ihm, ja auch mir recht angenehm. Er wohnt,
               für alle Fälle{ }ſei es Ihnen mitgetheilt, \textsc{XVIII. Exnerstraße 3\oindex{Wien@\textbf{Wien}!XVIII., Währing@\textbf{XVIII., Währing}!Krütznergasse@\textbf{Krütznergasse}, \emph{Straße}|pw}}. Es{ }ſcheint wirklich, dß er vom nächſten Monat {\pb}an
               nicht auf uns mehr angewieſen{ }ſein wird.\pend
           
\pstart
           Herzliche Grüße.{\\[\baselineskip]}Ihr \spacefill\mbox{Arthur}\pend
           \leftskip=0em{}\selectlanguage{ngerman}\endnumbering\briefempfaengerindex{Hofmannsthal, Hugo von@\textsc{Hofmannsthal, Hugo von}!zzzSchnitzler, Arthur@\emph{von Arthur Schnitzler}!1894-05-152@{{[}15. 5. 1894?{]}}|)be}\mylabel{L00324h}  \newcommand{\dateiname}{L00324}\newcommand{\titel}{Arthur Schnitzler an Hugo von Hofmannsthal, [15. 5. 1894?]}\newcommand{\editorInnen}{Martin Anton Müller und Gerd-Hermann Susen}%% latex-leseansicht-abspann.tex
%% Abspann für die Leseansicht.
%% Der Schalter \ifkorrekturansicht ist bereits durch den Vorspann gesetzt.

%% latex-abspann.tex
%% Gemeinsamer Abspann für Korrekturansicht und Leseansicht.
%% Setzt den Schalter \ifkorrekturansicht voraus (gesetzt in den
%% einbindenden Dateien latex-korrekturansicht-abspann.tex bzw.
%% latex-leseansicht-abspann.tex).
%% ---------------------------------------------------------------

\normalsize

% Das esempio-Environment wird nur in der Leseansicht benötigt
\ifkorrekturansicht\else
\newenvironment{esempio}[3]%
{
    \vspace{1.5ex}
    \rlap{\underline{#1}}
    \par
    \setlength{\parindent}{0cm}
    \nopagebreak
    \leftskip=#2cm
    \rightskip=#3cm
}
{
    \par
}
\fi

\doendnotes{C}
\bigskip
\vfill

\clearpage

\footnotesize

\ifkorrekturansicht
  \lohead{\textsc{register}}
\fi

% theindex-Environment neu definieren ohne reledmac
\makeatletter
\renewenvironment{theindex}{%
  \ifkorrekturansicht
    \section*{\indexname}%
  \else
    \subsubsection*{Index der erwähnten Entitäten}%
  \fi
  \setlength{\parindent}{0pt}%
  \setlength{\parskip}{0pt plus 0.3pt}%
  \let\item\@idxitem
}{%
  \ifkorrekturansicht\clearpage\fi
}
\makeatother

\IfFileExists{\jobname-pw.ind}{\input{\jobname-pw.ind}}{}

% Quellenangabe nur in der Leseansicht
\ifkorrekturansicht\else
% Fallback-Definitionen, falls die .tex-Datei \titel etc. nicht gesetzt hat
\providecommand{\titel}{}
\providecommand{\editorInnen}{}
\providecommand{\dateiname}{\jobname}

\vspace{3cm}

\vfill

\footnotesize
\textsc{Quelle}: \titel. Herausgegeben von {\editorInnen}. In: \emph{Arthur Schnitzler: Briefwechsel mit Autorinnen und Autoren}.
 Digitale Edition, https://schnitzler-briefe.acdh.oeaw.ac.at/{\dateiname}.html (Stand \today)
\fi

\end{document}


