%% latex-korrekturansicht-vorspann.tex
%% Vorspann für die Korrekturansicht.
%% Lädt die gemeinsame Datei latex-vorspann.tex mit gesetztem Schalter.

\newif\ifkorrekturansicht
\korrekturansichttrue

\input{../tex-inputs/latex-vorspann}


\section[ Arthur Schnitzler an Felix Salten, 6. 5. 1925]{L03021 Arthur Schnitzler an Felix Salten, 6. 5. 1925}
\nopagebreak\mylabel{L03021v}
\rehead{ }\normalsize\beginnumbering\briefempfaengerindex{Salten, Felix@\textsc{Salten, Felix}!zzzSchnitzler, Arthur@\emph{von Arthur Schnitzler}!1925-05-061@{6. 5. 1925}|(be}
\toendnotes[C]{\smallbreak\pagebreak[2]}\Standort{Wienbibliothek im Rathaus, ZPH 1681, 2.1.516.}
\physDesc{Brief, 1 Blatt, 2 Seiten, 925 Zeichen
\newline{}Handschrift: schwarze Tinte, lateinische Kurrent
\newline{}Ordnung: mit Bleistift von unbekannter Hand nummeriert: »4« }
\buchAbdrucke{\weitereDrucke{Arthur Schnitzler: \emph{Briefe 1913–1931}. Frankfurt am Main: \emph{S. Fischer} 1984, S. 406–407.} }\toendnotes[C]{\smallbreak}
\pstart
           \raggedleft{}{\pb}Wien\oindex{Wien@\textbf{Wien}, \emph{A.ADM2}|pw}{ }6. 5. 1925\pend
           \vspace{0.5em}
\pstart
           lieber, ich danke Ihnen von Herzen für Ihr wunderbares \label{K_L03021-1v}\edtext{Palaestina\oindex{Palaestina@\textbf{Palästina}, \emph{A.PCLS}|pw}-Buch\pwindex{Neue Menschen auf alter Erde. Eine Palaestinafahrt@\emph{Neue Menschen auf alter Erde. Eine Palästinafahrt}|pwv}}{\lemma{\textnormal{\emph{Palaestina-Buch}}}\Cendnote{\textnormal{Siehe Felix Salten: Widmungsexemplar Neue Menschen auf alter Erde für Arthur
               Schnitzler, 30. 4. 1925.
               }}}\label{K_L03021-1}; es
               ergreift mich sehr – nicht nur durch die Eindringlichkeit der mitgetheilten
               Thatsachen, und die meisterhafte Darstellung; – sondern auch, und ganz besonders als
               menschliches Bekenntnis eines klaren Verstandes und einer leidenschaftlichen Seele (man
               könnte vielleicht noch besser sagen: eines leidenschaftlichen Verstandes u einer
               klaren Seele.) Dieses Buch\pwindex{Neue Menschen auf alter Erde. Eine Palaestinafahrt@\emph{Neue Menschen auf alter Erde. Eine Palästinafahrt}|pwv}
               muſs ein starkes Echo, weit über literarische Kreise hinaus finden, und weit über
               jüdische; – es ist ein politisches Buch im guten Sinn – denn es ist beinahe ein
                  staatsmä{\geminationn}isches. Und ich glaube, wer sich weder für
               Literatur, noch {\pb}für Politik interessirt –
               wer einfach ein Reise- und Abenteuerbuch darin \strikeout{\textcolor{gray}{finden}} suchen wollte – er wird ein höchst fesselndes und amusantes \strikeout{\textcolor{gray}{darin}} finden. Das müssen Sie schon
               auch noch hinnehmen.\pend
           
\pstart
           Nochmals, Danke; und die herzlichsten Grüße {\\[\baselineskip]}Ihr {\\[\baselineskip]}\spacefill\mbox{ArthurSchnitzler}\pend
           \leftskip=0em{}\selectlanguage{ngerman}\endnumbering\briefempfaengerindex{Salten, Felix@\textsc{Salten, Felix}!zzzSchnitzler, Arthur@\emph{von Arthur Schnitzler}!1925-05-061@{6. 5. 1925}|)be}\mylabel{L03021h}  \normalsize

\doendnotes{C}
\bigskip
\vfill

\clearpage

\footnotesize

\lohead{\textsc{register}}

% Definiere theindex-Environment komplett neu ohne reledmac
\makeatletter
\renewenvironment{theindex}{%
  \section*{\indexname}%
  \setlength{\parindent}{0pt}%
  \setlength{\parskip}{0pt plus 0.3pt}%
  \let\item\@idxitem
}{%
  \clearpage
}
\makeatother

\IfFileExists{\jobname-pw.ind}{\input{\jobname-pw.ind}}{}

\end{document}

      