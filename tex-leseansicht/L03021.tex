%% latex-leseansicht-vorspann.tex
%% Vorspann für die Leseansicht.
%% Lädt die gemeinsame Datei latex-vorspann.tex mit nicht gesetztem Schalter.

\newif\ifkorrekturansicht
\korrekturansichtfalse

\input{../tex-inputs/latex-vorspann}


         
         \renewcommand{\erwaehntePersonen}{Personen: Ottilie Salten, Felix Salten}
         \renewcommand{\erwaehnteOrte}{Orte: Czernowitz, Palästina, Wien}
         \renewcommand{\erwaehnteWerke}{Werke: Neue Menschen auf alter Erde. Eine Palästinafahrt}
               \section[ Arthur Schnitzler an Felix Salten, 6. 5. 1925]{ Arthur Schnitzler an Felix Salten, 6. 5. 1925}\nopagebreak\mylabel{v}\rehead{ }\begin{ledgroupsized}[t]{13cm}\normalsize\beginnumbering\briefempfaengerindex{Salten, Felix@\textsc{Salten, Felix}!zzzSchnitzler, Arthur@\emph{von Arthur Schnitzler}!1925-05-061@{6. 5. 1925}|(be} \toendnotes[C]{\smallbreak\pagebreak[2]} \Standort{Wienbibliothek im Rathaus, ZPH 1681, 2.1.516.}
\physDesc{Brief, 1 Blatt, 2 Seiten, 925 Zeichen
\newline{}Handschrift: schwarze Tinte, lateinische Kurrent
\newline{}Ordnung: mit Bleistift von unbekannter Hand nummeriert: »4« }\buchAbdrucke{\weitereDrucke{Arthur Schnitzler: \emph{Briefe 1913–1931}. Hg. Peter Michael Braunwarth, Richard Miklin, Susanne Pertlik und Heinrich Schnitzler. Frankfurt am Main: \emph{S. Fischer} 1984, S. 406–407.} }\toendnotes[C]{\smallbreak}\pstart
           \raggedleft{}{\pb}Wien\oindex{Wien@\textbf{Wien}|pw}{ }6. 5. 1925\pend
           \pstart
           lieber, ich danke Ihnen von Herzen für Ihr wunderbares \label{K_L03021-1v}\edtext{Palaestina\oindex{Palaestina@\textbf{Palästina}|pw}-Buch\pwindex{Salten, Felix 06.09.1869 – 08.10.1945@\textsc{Salten, Felix} (06.09.1869 – 08.10.1945), \emph{Schriftsteller, Journalist, Chefredakteur}!Neue Menschen auf alter Erde. Eine Palaestinafahrt1925@\strich\emph{Neue Menschen auf alter Erde. Eine Palästinafahrt} {[}1925{]}|pwv}}{\lemma{\textnormal{\emph{Palaestina-Buch}}}\Cendnote{\textnormal{Siehe Felix Salten: Widmungsexemplar Neue Menschen auf alter Erde für Arthur
               Schnitzler, 30. 4. 1925.
               }}}\label{K_L03021-1h}; es
               ergreift mich sehr – nicht nur durch die Eindringlichkeit der mitgetheilten
               Thatsachen, und die meisterhafte Darstellung; – sondern auch, und ganz besonders als
               menschliches Bekenntnis eines klaren Verstandes und einer leidenschaftlichen Seele (man
               könnte vielleicht noch besser sagen: eines leidenschaftlichen Verstandes u einer
               klaren Seele.) Dieses Buch\pwindex{Salten, Felix 06.09.1869 – 08.10.1945@\textsc{Salten, Felix} (06.09.1869 – 08.10.1945), \emph{Schriftsteller, Journalist, Chefredakteur}!Neue Menschen auf alter Erde. Eine Palaestinafahrt1925@\strich\emph{Neue Menschen auf alter Erde. Eine Palästinafahrt} {[}1925{]}|pwv}
               muſs ein starkes Echo, weit über literarische Kreise hinaus finden, und weit über
               jüdische; – es ist ein politisches Buch im guten Sinn – denn es ist beinahe ein
                  staatsmä{\geminationn}isches. Und ich glaube, wer sich weder für
               Literatur, noch {\pb}für Politik interessirt –
               wer einfach ein Reise- und Abenteuerbuch darin \strikeout{\textcolor{gray}{finden}} suchen wollte – er wird ein höchst fesselndes und amusantes \strikeout{\textcolor{gray}{darin}} finden. Das müssen Sie schon
               auch noch hinnehmen.\pend
           \pstart
           Nochmals, Danke; und die herzlichsten Grüße {\\[\baselineskip]}Ihr {\\[\baselineskip]}\spacefill\mbox{ArthurSchnitzler}\pend
           \leftskip=0em{}
         
         \endnumbering\mylabel{h}\end{ledgroupsized}  \newcommand{\dateiname}{L03021}\newcommand{\titel}{Arthur Schnitzler an Felix Salten, 6. 5. 1925}\newcommand{\editorInnen}{Martin Anton Müller und Laura Untner}%% latex-leseansicht-abspann.tex
%% Abspann für die Leseansicht.
%% Der Schalter \ifkorrekturansicht ist bereits durch den Vorspann gesetzt.

%% latex-abspann.tex
%% Gemeinsamer Abspann für Korrekturansicht und Leseansicht.
%% Setzt den Schalter \ifkorrekturansicht voraus (gesetzt in den
%% einbindenden Dateien latex-korrekturansicht-abspann.tex bzw.
%% latex-leseansicht-abspann.tex).
%% ---------------------------------------------------------------

\normalsize

% Das esempio-Environment wird nur in der Leseansicht benötigt
\ifkorrekturansicht\else
\newenvironment{esempio}[3]%
{
    \vspace{1.5ex}
    \rlap{\underline{#1}}
    \par
    \setlength{\parindent}{0cm}
    \nopagebreak
    \leftskip=#2cm
    \rightskip=#3cm
}
{
    \par
}
\fi

\doendnotes{C}
\bigskip
\vfill

\clearpage

\footnotesize

\ifkorrekturansicht
  \lohead{\textsc{register}}
\fi

% theindex-Environment neu definieren ohne reledmac
\makeatletter
\renewenvironment{theindex}{%
  \ifkorrekturansicht
    \section*{\indexname}%
  \else
    \subsubsection*{Index der erwähnten Entitäten}%
  \fi
  \setlength{\parindent}{0pt}%
  \setlength{\parskip}{0pt plus 0.3pt}%
  \let\item\@idxitem
}{%
  \ifkorrekturansicht\clearpage\fi
}
\makeatother

\IfFileExists{\jobname-pw.ind}{\input{\jobname-pw.ind}}{}

% Quellenangabe nur in der Leseansicht
\ifkorrekturansicht\else
% Fallback-Definitionen, falls die .tex-Datei \titel etc. nicht gesetzt hat
\providecommand{\titel}{}
\providecommand{\editorInnen}{}
\providecommand{\dateiname}{\jobname}

\vspace{3cm}

\vfill

\footnotesize
\textsc{Quelle}: \titel. Herausgegeben von {\editorInnen}. In: \emph{Arthur Schnitzler: Briefwechsel mit Autorinnen und Autoren}.
 Digitale Edition, https://schnitzler-briefe.acdh.oeaw.ac.at/{\dateiname}.html (Stand \today)
\fi

\end{document}


      