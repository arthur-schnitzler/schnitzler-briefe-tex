%% latex-leseansicht-vorspann.tex
%% Vorspann für die Leseansicht.
%% Lädt die gemeinsame Datei latex-vorspann.tex mit nicht gesetztem Schalter.

\newif\ifkorrekturansicht
\korrekturansichtfalse

\input{../tex-inputs/latex-vorspann}


         
         \renewcommand{\erwaehntePersonen}{Personen: Berta Czegka, Felix Salten, Friedrich von Schiller}
         \renewcommand{\erwaehnteOrte}{Orte: Edmund-Weiß-Gasse 7, Wien}
         \renewcommand{\erwaehnteWerke}{Werke: Die Zeit, Schiller-Feier, Schiller-Zeit 1805 * 1905, Zum großen Wurstel. Burleske in einem Akt}
               \section[ Arthur Schnitzler an Felix Salten, 11. 4. 1905]{ Arthur Schnitzler an Felix Salten, 11. 4. 1905}\nopagebreak\mylabel{v}\rehead{ }\begin{ledgroupsized}[t]{13cm}\normalsize\beginnumbering\briefempfaengerindex{Salten, Felix@\textsc{Salten, Felix}!zzzSchnitzler, Arthur@\emph{von Arthur Schnitzler}!1905-04-111@{11. 4. 1905}|(be} \toendnotes[C]{\smallbreak\pagebreak[2]} \Standort{Wienbibliothek im Rathaus, ZPH 1681, 2.1.516.}
\physDesc{Brief, 1 Blatt, 3 Seiten, 595 Zeichen
\newline{}Handschrift: schwarze Tinte, deutsche Kurrent
\newline{}Ordnung: mit Bleistift von unbekannter Hand Nummerierung der Doppelseiten des
                                 Konvoluts: »26«–»27« }\buchAbdrucke{\weitereDrucke{Arthur Schnitzler: \emph{Briefe 1875–1912}. Hg. Therese Nickl und Heinrich Schnitzler. Frankfurt am Main: \emph{S. Fischer} 1981, S. 513.} }\toendnotes[C]{\smallbreak}\pstart
           \noindent{}\textcolor{gray}{\textbf{{\pb}Dr. Arthur Schnitzler}}\hfill 11. 4. 905\pend
           \pstart
           \textcolor{gray}{\textbf{Wien, XVIII. Spoettelgasse 7\oindex{Edmund-Weiss-Gasse 7@\textbf{Edmund-Weiß-Gasse 7}|pw}.}}\pend
           \pstart
           lieber, hiebei etliche \label{K_L02998-1v}\edtext{Diſtichen\pwindex{Schnitzler, Arthur 15.05.1862 – 21.10.1931@\textsc{Schnitzler, Arthur} (15.05.1862 – 21.10.1931), \emph{Schriftsteller, Mediziner}!Schiller-Feier23. 4. 1905@\strich\emph{Schiller-Feier} {[}23. 4. 1905{]}|pwv} für Ihre Schiller\pwindex{Schiller, Friedrich von 10.11.1759 – 09.05.1805@\textsc{Schiller, Friedrich von} (10.11.1759 – 09.05.1805), \emph{Schriftsteller, Historiker, Philosoph}|pw}nummer\pwindex{Schiller-Zeit 1805 * 19051905-04-23@\emph{Schiller-Zeit 1805 * 1905} {[}1905-04-23{]}|pw}}{\lemma{\textnormal{\emph{Diſtichen … Schillernummer}}}\Cendnote{\textnormal{Arthur Schnitzler\pwindex{Schnitzler, Arthur 15.05.1862 – 21.10.1931@\textsc{Schnitzler, Arthur} (15.05.1862 – 21.10.1931), \emph{Schriftsteller, Mediziner}|pwk}: \emph{Schiller-Feier}\pwindex{Schnitzler, Arthur 15.05.1862 – 21.10.1931@\textsc{Schnitzler, Arthur} (15.05.1862 – 21.10.1931), \emph{Schriftsteller, Mediziner}!Schiller-Feier23. 4. 1905@\strich\emph{Schiller-Feier} {[}23. 4. 1905{]}|pwk}. In: \emph{Die
                        Zeit}\pwindex{Zeit1902-09-27 – 1919@\emph{Die Zeit} {[}1902-09-27 – 1919{]}|pwk}, Jg. 4, Nr. 926, 23. 4. 1905,
                     Beilage: \emph{Die Schiller-Zeit}\pwindex{Schiller-Zeit 1805 * 19051905-04-23@\emph{Schiller-Zeit 1805 * 1905} {[}1905-04-23{]}|pwk}, S. VI.
                  Siehe A. S.: \emph{»Das Zeitlose ist von kürzester Dauer«}, Schiller-Feier, 23. 4. 1905.}}}\label{K_L02998-1h}, wenn Sie ſie
               brauchen können. –\pend
           \pstart
           Werden Sie den \label{K_L02998-2v}\edtext{Wurſtelſpaſs\pwindex{Schnitzler, Arthur 15.05.1862 – 21.10.1931@\textsc{Schnitzler, Arthur} (15.05.1862 – 21.10.1931), \emph{Schriftsteller, Mediziner}!Zum grossen Wurstel. Burleske in einem Akt08. 03. 1901@\strich\emph{Zum großen Wurstel. Burleske in einem Akt} {[}08. 03. 1901{]}|pwv}}{\lemma{\textnormal{\emph{Wurſtelſpaſs}}}\Cendnote{\textnormal{Siehe Arthur Schnitzler an Felix Salten, 8. 2. 1905.
               }}}\label{K_L02998-2h} zu Oſtern bringen? Ich ſchlug Ihnen bei Zuſendg
               vor, Bilder dazu machen zu laſſen und wollte mit dem ev. Illuſtrator\pwindex{Czegka, Berta 30.07.1880 – 04.11.1954@\textsc{Czegka, Berta} (30.07.1880 – 04.11.1954), \emph{Malerin}|pwv} ſelbſt reden. Vielleicht haben
               Sie die Stelle überleſen, ſti{\geminationm}en aber jetzt {\pb}der Bilder\substVorne{}\textsuperscript{\textcolor{gray}{illu}}\substDazwischen{}idee\substHinten{} bei, in welchem Fall man die Sache bis \uline{Pfingſten} laſſen könnte?–\pend
           \pstart
           Die \uline{Correcturen} erhalte ich doch in jedem Falle?–\pend
           \pstart
           Herzlichſt {\\[\baselineskip]}Ihr {\\[\baselineskip]}\spacefill\mbox{A.}\pend
           \leftskip=0em{}\pstart
           Iſt es zu viel verlangt, wenn ich Sie bitte mir auch eine Correctur der Diſtichen\pwindex{Schnitzler, Arthur 15.05.1862 – 21.10.1931@\textsc{Schnitzler, Arthur} (15.05.1862 – 21.10.1931), \emph{Schriftsteller, Mediziner}!Schiller-Feier23. 4. 1905@\strich\emph{Schiller-Feier} {[}23. 4. 1905{]}|pwv} ſchicken zu laſſen?
               In Verſen leiſten die Setzer {\pb}manchmal
               ſeltſames.\pend
           
         
         \endnumbering\mylabel{h}\end{ledgroupsized}  \newcommand{\dateiname}{L02998}\newcommand{\titel}{Arthur Schnitzler an Felix Salten, 11. 4. 1905}\newcommand{\editorInnen}{Martin Anton Müller und Laura Untner}%% latex-leseansicht-abspann.tex
%% Abspann für die Leseansicht.
%% Der Schalter \ifkorrekturansicht ist bereits durch den Vorspann gesetzt.

%% latex-abspann.tex
%% Gemeinsamer Abspann für Korrekturansicht und Leseansicht.
%% Setzt den Schalter \ifkorrekturansicht voraus (gesetzt in den
%% einbindenden Dateien latex-korrekturansicht-abspann.tex bzw.
%% latex-leseansicht-abspann.tex).
%% ---------------------------------------------------------------

\normalsize

% Das esempio-Environment wird nur in der Leseansicht benötigt
\ifkorrekturansicht\else
\newenvironment{esempio}[3]%
{
    \vspace{1.5ex}
    \rlap{\underline{#1}}
    \par
    \setlength{\parindent}{0cm}
    \nopagebreak
    \leftskip=#2cm
    \rightskip=#3cm
}
{
    \par
}
\fi

\doendnotes{C}
\bigskip
\vfill

\clearpage

\footnotesize

\ifkorrekturansicht
  \lohead{\textsc{register}}
\fi

% theindex-Environment neu definieren ohne reledmac
\makeatletter
\renewenvironment{theindex}{%
  \ifkorrekturansicht
    \section*{\indexname}%
  \else
    \subsubsection*{Index der erwähnten Entitäten}%
  \fi
  \setlength{\parindent}{0pt}%
  \setlength{\parskip}{0pt plus 0.3pt}%
  \let\item\@idxitem
}{%
  \ifkorrekturansicht\clearpage\fi
}
\makeatother

\IfFileExists{\jobname-pw.ind}{\input{\jobname-pw.ind}}{}

% Quellenangabe nur in der Leseansicht
\ifkorrekturansicht\else
% Fallback-Definitionen, falls die .tex-Datei \titel etc. nicht gesetzt hat
\providecommand{\titel}{}
\providecommand{\editorInnen}{}
\providecommand{\dateiname}{\jobname}

\vspace{3cm}

\vfill

\footnotesize
\textsc{Quelle}: \titel. Herausgegeben von {\editorInnen}. In: \emph{Arthur Schnitzler: Briefwechsel mit Autorinnen und Autoren}.
 Digitale Edition, https://schnitzler-briefe.acdh.oeaw.ac.at/{\dateiname}.html (Stand \today)
\fi

\end{document}


      