%% latex-korrekturansicht-vorspann.tex
%% Vorspann für die Korrekturansicht.
%% Lädt die gemeinsame Datei latex-vorspann.tex mit gesetztem Schalter.

\newif\ifkorrekturansicht
\korrekturansichttrue

\input{../tex-inputs/latex-vorspann}


\section[Arthur Schnitzler an Hugo von Hofmannsthal, 7. 6. 1901]{L01125 Arthur Schnitzler an Hugo von Hofmannsthal, 7. 6. 1901}
\nopagebreak\mylabel{L01125v}
\rehead{ }\normalsize\beginnumbering\briefempfaengerindex{Hofmannsthal, Hugo von@\textsc{Hofmannsthal, Hugo von}!zzzSchnitzler, Arthur@\emph{von Arthur Schnitzler}!1901-06-071@{7. 6. 1901}|(be}
\toendnotes[C]{\smallbreak\pagebreak[2]}\Standort{FDH, Hs-30885,94.}
\physDesc{Brief, 1 Blatt, 2 Seiten, 493 Zeichen
\newline{}Handschrift: schwarze Tinte, deutsche Kurrent
\newline{}Ordnung: mit Bleistift von Schnitzler mutmaßlich bei der Durchsicht der Korrespondenz
                                    1929 datiert: »7/6 901« }
\buchAbdrucke{\weitereDrucke{Hugo von Hofmannsthal, Arthur Schnitzler: \emph{Briefwechsel}. Frankfurt am Main: \emph{S. Fischer} 1964, S. 146.} }\toendnotes[C]{\smallbreak}
\pstart{}{\pb}Mein lieber Hugo, \pend\vspace{0.5em}
\pstart
           Sie erinnern ſich vielleicht dieſer kleinen Kaſſette oder wie Sies nennen wollen, aus
                  Salzburg\oindex{Salzburg@\textbf{Salzburg}, \emph{A.ADM2}|pw}. Ich möchte gern, daſs Sie irgendwo in
               der Rodauner Villa\oindex{Hofmannsthal-Schloessl@\textbf{Hofmannsthal-Schlössl}, \emph{Schloss (K.SLS)}|pw} einen Platz fänden ſie
               hinzuſtellen und ſich dabei manchmal jenes Salzburg\oindex{Salzburg@\textbf{Salzburg}, \emph{A.ADM2}|pw}er \label{K_L01125-1v}\edtext{Tags beim \textsc{Svatek}\orgindex{Wenzel Swatek@Wenzel Swatek|pw}}{\lemma{\textnormal{\emph{Tags beim Svatek}}}\Cendnote{\textnormal{Siehe A. S.: \emph{Tagebuch}, 12. 8. 1900.
               }}}\label{K_L01125-1} erinnern; und {\pb}andrer Tage auch. Adieu alſo und
               auf ein ſchönes Wiederſehn, ſpäteſtens zu Anfang des Herbſtes.\pend
           
\pstart
           Grüßen Sie Gerty\pwindex{Hofmannsthal, Gertrude von 16.03.1880 – 09.11.1959@\textsc{Hofmannsthal, Gertrude von} (16.03.1880 – 09.11.1959)|pw}, ich brauche Ihnen beiden
               nicht erſt zu ſagen, wie viel Glück ich Ihnen wünſche.\pend
           
\pstart
           Immer Ihr{\\[\baselineskip]}\spacefill\mbox{Arthur}\pend
           \leftskip=0em{}
\pstart
           Wien\oindex{Wien@\textbf{Wien}, \emph{A.ADM2}|pw}{ }7. Juni 901.\pend
           \selectlanguage{ngerman}\endnumbering\briefempfaengerindex{Hofmannsthal, Hugo von@\textsc{Hofmannsthal, Hugo von}!zzzSchnitzler, Arthur@\emph{von Arthur Schnitzler}!1901-06-071@{7. 6. 1901}|)be}\mylabel{L01125h}  \normalsize

\doendnotes{C}
\bigskip
\vfill

\clearpage

\footnotesize

\lohead{\textsc{register}}

% Definiere theindex-Environment komplett neu ohne reledmac
\makeatletter
\renewenvironment{theindex}{%
  \section*{\indexname}%
  \setlength{\parindent}{0pt}%
  \setlength{\parskip}{0pt plus 0.3pt}%
  \let\item\@idxitem
}{%
  \clearpage
}
\makeatother

\IfFileExists{\jobname-pw.ind}{\input{\jobname-pw.ind}}{}

\end{document}

      