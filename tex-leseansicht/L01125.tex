%% latex-leseansicht-vorspann.tex
%% Vorspann für die Leseansicht.
%% Lädt die gemeinsame Datei latex-vorspann.tex mit nicht gesetztem Schalter.

\newif\ifkorrekturansicht
\korrekturansichtfalse

\input{../tex-inputs/latex-vorspann}


\section[Arthur Schnitzler an Hugo von Hofmannsthal, 7. 6. 1901]{L01125 Arthur Schnitzler an Hugo von Hofmannsthal, 7. 6. 1901}
\nopagebreak\mylabel{L01125v}
\rehead{ }\normalsize\beginnumbering\briefempfaengerindex{Hofmannsthal, Hugo von@\textsc{Hofmannsthal, Hugo von}!zzzSchnitzler, Arthur@\emph{von Arthur Schnitzler}!1901-06-071@{7. 6. 1901}|(be}
\toendnotes[C]{\smallbreak\pagebreak[2]}
\correspDesc{Versand  durch Arthur Schnitzler am 7. 6. 1901 in Wien
\newline{}Erhalt  durch Hugo von Hofmannsthal im Zeitraum [7. 6. 1901
                  – 11. 6. 1901?] in Wien}\toendnotes[C]{\smallbreak}
\Standort{FDH, Hs-30885,94.}
\physDesc{Brief, 1 Blatt, 2 Seiten, 493 Zeichen
\newline{}Handschrift: schwarze Tinte, deutsche Kurrent
\newline{}Ordnung: mit Bleistift von Schnitzler mutmaßlich bei der Durchsicht der Korrespondenz
                                    1929 datiert: »7/6 901« }
\buchAbdrucke{\weitereDrucke{Hugo von Hofmannsthal, Arthur Schnitzler: \emph{Briefwechsel}. Herausgegeben von Therese Nickl und Heinrich Schnitzler. Frankfurt am Main: \emph{S. Fischer} 1964, S. 146.} }\toendnotes[C]{\smallbreak}
\pstart{}{\pb}Mein lieber Hugo,\pend\vspace{0.5em}
\pstart
           Sie erinnern{ }ſich vielleicht dieſer kleinen Kaſſette oder wie Sies nennen wollen, aus
                  Salzburg\oindex{Salzburg@\textbf{Salzburg}, \emph{Verwaltungsgebiet}|pw}. Ich möchte gern, daſs Sie irgendwo in
               der Rodauner Villa\oindex{Wien@\textbf{Wien}!XXIII., Liesing@\textbf{XXIII., Liesing}!Hofmannsthal-Schlössl@\textbf{Hofmannsthal-Schlössl}, \emph{Schloss}|pw} einen Platz fänden{ }ſie
               hinzuſtellen und{ }ſich dabei manchmal jenes Salzburg\oindex{Salzburg@\textbf{Salzburg}, \emph{Verwaltungsgebiet}|pw}er \label{K_L01125-1v}\edtext{Tags beim \textsc{Svatek}\orgindex{Wenzel Swatek@Wenzel Swatek|pw}}{\lemma{\textnormal{\emph{Tags beim Svatek}}}\Cendnote{\textnormal{Siehe A. S.: \emph{Tagebuch}, 12. 8. 1900.
               }}}\label{K_L01125-1} erinnern; und {\pb}andrer Tage auch. Adieu alſo und
               auf ein{ }ſchönes Wiederſehn,{ }ſpäteſtens zu Anfang des Herbſtes.\pend
           
\pstart
           Grüßen Sie Gerty\pwindex{Hofmannsthal, Gertrude von 16.\,3.\,1880 Wien – 9.\,11.\,1959 Paddington@\textsc{Hofmannsthal, Gertrude von} (16.\,3.\,1880 Wien – 9.\,11.\,1959 Paddington)|pw}, ich brauche Ihnen beiden
               nicht erſt zu{ }ſagen, wie viel Glück ich Ihnen wünſche.\pend
           
\pstart
           Immer Ihr{\\[\baselineskip]}\spacefill\mbox{Arthur}\pend
           \leftskip=0em{}
\pstart
           Wien\oindex{Wien@\textbf{Wien}, \emph{Verwaltungsgebiet}|pw}{ }7. Juni 901.\pend
           \selectlanguage{ngerman}\endnumbering\briefempfaengerindex{Hofmannsthal, Hugo von@\textsc{Hofmannsthal, Hugo von}!zzzSchnitzler, Arthur@\emph{von Arthur Schnitzler}!1901-06-071@{7. 6. 1901}|)be}\mylabel{L01125h}  \newcommand{\dateiname}{L01125}\newcommand{\titel}{Arthur Schnitzler an Hugo von Hofmannsthal, 7. 6. 1901}\newcommand{\editorInnen}{Martin Anton Müller und Gerd-Hermann Susen}%% latex-leseansicht-abspann.tex
%% Abspann für die Leseansicht.
%% Der Schalter \ifkorrekturansicht ist bereits durch den Vorspann gesetzt.

%% latex-abspann.tex
%% Gemeinsamer Abspann für Korrekturansicht und Leseansicht.
%% Setzt den Schalter \ifkorrekturansicht voraus (gesetzt in den
%% einbindenden Dateien latex-korrekturansicht-abspann.tex bzw.
%% latex-leseansicht-abspann.tex).
%% ---------------------------------------------------------------

\normalsize

% Das esempio-Environment wird nur in der Leseansicht benötigt
\ifkorrekturansicht\else
\newenvironment{esempio}[3]%
{
    \vspace{1.5ex}
    \rlap{\underline{#1}}
    \par
    \setlength{\parindent}{0cm}
    \nopagebreak
    \leftskip=#2cm
    \rightskip=#3cm
}
{
    \par
}
\fi

\doendnotes{C}
\bigskip
\vfill

\clearpage

\footnotesize

\ifkorrekturansicht
  \lohead{\textsc{register}}
\fi

% theindex-Environment neu definieren ohne reledmac
\makeatletter
\renewenvironment{theindex}{%
  \ifkorrekturansicht
    \section*{\indexname}%
  \else
    \subsubsection*{Index der erwähnten Entitäten}%
  \fi
  \setlength{\parindent}{0pt}%
  \setlength{\parskip}{0pt plus 0.3pt}%
  \let\item\@idxitem
}{%
  \ifkorrekturansicht\clearpage\fi
}
\makeatother

\IfFileExists{\jobname-pw.ind}{\input{\jobname-pw.ind}}{}

% Quellenangabe nur in der Leseansicht
\ifkorrekturansicht\else
% Fallback-Definitionen, falls die .tex-Datei \titel etc. nicht gesetzt hat
\providecommand{\titel}{}
\providecommand{\editorInnen}{}
\providecommand{\dateiname}{\jobname}

\vspace{3cm}

\vfill

\footnotesize
\textsc{Quelle}: \titel. Herausgegeben von {\editorInnen}. In: \emph{Arthur Schnitzler: Briefwechsel mit Autorinnen und Autoren}.
 Digitale Edition, https://schnitzler-briefe.acdh.oeaw.ac.at/{\dateiname}.html (Stand \today)
\fi

\end{document}


