%% latex-leseansicht-vorspann.tex
%% Vorspann für die Leseansicht.
%% Lädt die gemeinsame Datei latex-vorspann.tex mit nicht gesetztem Schalter.

\newif\ifkorrekturansicht
\korrekturansichtfalse

\input{../tex-inputs/latex-vorspann}


\section[Theodor Herzl an Arthur Schnitzler, 28. 2. 1902]{L03899 Theodor Herzl an Arthur Schnitzler, 28. 2. 1902}
\nopagebreak\mylabel{L03899v}
\rehead{ }\normalsize\beginnumbering\briefempfaengerindex{Schnitzler, Arthur@\textsc{Schnitzler, Arthur}!zzzHerzl, Theodor@\emph{von Theodor Herzl}!1902-02-282@{28. 2. 1902}|(be}
\toendnotes[C]{\smallbreak\pagebreak[2]}
\correspDesc{Versand  durch Theodor Herzl am 28. 2. 1902 in Wien
\newline{}Erhalt  durch Arthur Schnitzler im Zeitraum [28. 2. 1902
                  – 3. 3. 1902?] in Wien}\toendnotes[C]{\smallbreak}
\buchAlsQuelle{Theodor Herzl: \emph{Briefe Ende August 1900 – Ende Dezember 1902}. Bearbeitet von Barbara Schäfer in Zusammenarbeit mit Sofia Gelmann, Chaya Harel und Ines Rubin. Berlin, Frankfurt am Main, Wien: \emph{Propyläen} 1993, S. 444 (Briefe und Tagebücher. Herausgegeben von Alex Bein, Hermann Greive, Moshe Schaerf, Julius H. Schoeps und Johannes Wachten, 6).}\toendnotes[C]{\smallbreak}
\pstart
           {\pb}\textcolor{gray}{\textbf{NEUE FREIE PRESSE\orgindex{Neue Freie Presse@Neue Freie Presse|pw}. }}\hfill \label{T_L03899-1v}\edtext{28. II. 1902.}{\lemma{\textnormal{\emph{28. II. 1902.}}}\Cendnote{\textnormal{Die Datumszeile und der Hinweis, dass der Briefkopf der
                           \emph{Neuen Freien Presse}\orgindex{Neue Freie Presse@Neue Freie Presse|pwk} Verwendung fand,
                        findet sich in der ansonsten durch Normalisierungen unzuverlässigeren
                        Abschrift \emph{Central Zionist Archives Jerusalem},
                           H1:2539-3, S. 9.}}}\label{T_L03899-1}\pend
           
\pstart
           \textcolor{gray}{\textbf{\textsc{Redaction}:}}\pend
           
\pstart
           \textcolor{gray}{\textbf{WIEN\oindex{Wien@\textbf{Wien}, \emph{Verwaltungsgebiet}|pw}}}\pend
           
\pstart
           \textcolor{gray}{\textbf{Kolowratring, Fichtegasse Nr. 11\oindex{Wien@\textbf{Wien}!I., Innere Stadt@\textbf{I., Innere Stadt}!Fichtegasse 11@\textbf{Fichtegasse 11}, \emph{Gebäude}|pw}.}}\pend
           
\pstart{}Lieber Dr. Schnitzler,\pend\vspace{0.5em}
\pstart
           \label{K_L03899-1v}\edtext{Ostern}{\lemma{\textnormal{\emph{Ostern}}}\Cendnote{\textnormal{In diesem Jahr fiel der Ostersonntag auf
                  den 7. 4. 1901.}}}\label{K_L03899-1} ist vor der Thür u. ich lade Sie ein. Wenn
               mich mein ermüdetes Feuilletonredacteursgedächtnis nicht täuscht, haben Sie mir
               Weihnachten etwas\pwindex{Schnitzler, Arthur 15.\,5.\,1862 Wien – 21.\,10.\,1931 ebd.@\textsc{Schnitzler, Arthur} (15.\,5.\,1862 Wien – 21.\,10.\,1931 ebd.), \emph{Schriftsteller, Mediziner}!Dämmerseele@\strich\emph{Dämmerseele}|pwv} für Ostern \label{K_L03899-2v}\edtext{versprochen}{\lemma{\textnormal{\emph{versprochen}}}\Cendnote{\textnormal{Vgl. XXXX Auszeichnungsfehler: Dokument L03765 nicht gefunden.}}}\label{K_L03899-2}. Es kann
               eine Geschichte\pwindex{Schnitzler, Arthur 15.\,5.\,1862 Wien – 21.\,10.\,1931 ebd.@\textsc{Schnitzler, Arthur} (15.\,5.\,1862 Wien – 21.\,10.\,1931 ebd.), \emph{Schriftsteller, Mediziner}!Dämmerseele@\strich\emph{Dämmerseele}|pwv}, Plauderei oder einactiges Stück sein.\pend
           
\pstart
           Ihr Jawort bald erwartend mit den besten Grüssen{\\[\baselineskip]} Ihr \spacefill\mbox{Herzl}\pend
           \leftskip=0em{}\selectlanguage{ngerman}\endnumbering\briefempfaengerindex{Schnitzler, Arthur@\textsc{Schnitzler, Arthur}!zzzHerzl, Theodor@\emph{von Theodor Herzl}!1902-02-282@{28. 2. 1902}|)be}\mylabel{L03899h}
\begin{anhang}
\end{anhang}\newcommand{\dateiname}{L03899}\newcommand{\titel}{Theodor Herzl an Arthur Schnitzler, 28. 2. 1902}\newcommand{\editorInnen}{Selma Jahnke und Martin Anton Müller}%% latex-leseansicht-abspann.tex
%% Abspann für die Leseansicht.
%% Der Schalter \ifkorrekturansicht ist bereits durch den Vorspann gesetzt.

%% latex-abspann.tex
%% Gemeinsamer Abspann für Korrekturansicht und Leseansicht.
%% Setzt den Schalter \ifkorrekturansicht voraus (gesetzt in den
%% einbindenden Dateien latex-korrekturansicht-abspann.tex bzw.
%% latex-leseansicht-abspann.tex).
%% ---------------------------------------------------------------

\normalsize

% Das esempio-Environment wird nur in der Leseansicht benötigt
\ifkorrekturansicht\else
\newenvironment{esempio}[3]%
{
    \vspace{1.5ex}
    \rlap{\underline{#1}}
    \par
    \setlength{\parindent}{0cm}
    \nopagebreak
    \leftskip=#2cm
    \rightskip=#3cm
}
{
    \par
}
\fi

\doendnotes{C}
\bigskip
\vfill

\clearpage

\footnotesize

\ifkorrekturansicht
  \lohead{\textsc{register}}
\fi

% theindex-Environment neu definieren ohne reledmac
\makeatletter
\renewenvironment{theindex}{%
  \ifkorrekturansicht
    \section*{\indexname}%
  \else
    \subsubsection*{Index der erwähnten Entitäten}%
  \fi
  \setlength{\parindent}{0pt}%
  \setlength{\parskip}{0pt plus 0.3pt}%
  \let\item\@idxitem
}{%
  \ifkorrekturansicht\clearpage\fi
}
\makeatother

\IfFileExists{\jobname-pw.ind}{\input{\jobname-pw.ind}}{}

% Quellenangabe nur in der Leseansicht
\ifkorrekturansicht\else
% Fallback-Definitionen, falls die .tex-Datei \titel etc. nicht gesetzt hat
\providecommand{\titel}{}
\providecommand{\editorInnen}{}
\providecommand{\dateiname}{\jobname}

\vspace{3cm}

\vfill

\footnotesize
\textsc{Quelle}: \titel. Herausgegeben von {\editorInnen}. In: \emph{Arthur Schnitzler: Briefwechsel mit Autorinnen und Autoren}.
 Digitale Edition, https://schnitzler-briefe.acdh.oeaw.ac.at/{\dateiname}.html (Stand \today)
\fi

\end{document}


