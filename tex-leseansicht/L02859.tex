%% latex-leseansicht-vorspann.tex
%% Vorspann für die Leseansicht.
%% Lädt die gemeinsame Datei latex-vorspann.tex mit nicht gesetztem Schalter.

\newif\ifkorrekturansicht
\korrekturansichtfalse

\input{../tex-inputs/latex-vorspann}


\section[ Paul Goldmann an Arthur Schnitzler, 28. 9. 1898]{L02859 Paul Goldmann an Arthur Schnitzler,  28. 9. 1898}
\nopagebreak\mylabel{L02859v}
\rehead{ }\normalsize\beginnumbering\briefempfaengerindex{Schnitzler, Arthur@\textsc{Schnitzler, Arthur}!zzzGoldmann, Paul@\emph{von Paul Goldmann}!1898-09-281@{28. 9. 1898}|(be}
\toendnotes[C]{\smallbreak\pagebreak[2]}
\correspDesc{Versand  durch Paul Goldmann am 28. 9. 1898 in Tianjin
\newline{}Übermittlung  am 28. 9. 1898 in Tianjin
\newline{}Übermittlung  am 4. 10. 1898 in Yokohama
\newline{}Erhalt  durch Arthur Schnitzler am 10. 11. 1898 in Wien}\toendnotes[C]{\smallbreak}
\Standort{DLA, A:Schnitzler, HS.NZ85.1.3168.}
\physDesc{Postkarte, 261 Zeichen
\newline{}Handschrift: schwarze Tinte, deutsche Kurrent
\newline{}Versand: 1) Stempel: »\nobreak{}\oindex{Tianjin@\textbf{Tianjin}|pwk}Tientsin, 28\textcolor{gray}{/}9 98, Kaiserl. deutsche
                                          Postagentur\orgindex{Deutsche Post in China@Deutsche Post in China|pw}\nobreak{}«.   2) Stempel: »\nobreak{}\oindex{Yokohama@\textbf{Yokohama}|pwk}Yok{[}ohama{]}, \oindex{Marseille@\textbf{Marseille}|pwk}Marseille, 4 Oct. 98, L. N. N\textsuperscript{o} 10\nobreak{}«.  3) Stempel: »\nobreak{}\oindex{IX., Alsergrund@\textbf{IX., Alsergrund}, \emph{Verwaltungsgebiet}|pwk}Wien 3/9 72, 10. 11. 98, 8. V, Bestellt\nobreak{}«. 
\newline{}Schnitzler: mit Bleistift das Jahr »98« vermerkt }\toendnotes[C]{\smallbreak}\pstart{}\textsc{{\pb}Austria\oindex{Österreich@\textbf{Österreich}|pw}.}\pend{}\pstart{}\textsc{Herrn}\pend{}\pstart{}\textsc{Dr. Arthur Schnitzler}\pend{}\pstart{}\textsc{Wien\oindex{Wien@\textbf{Wien}, \emph{Verwaltungsgebiet}|pw}}\pend{}\pstart{}\textsc{IX. Frankgaſse 1\oindex{Wien@\textbf{Wien}!IX., Alsergrund@\textbf{IX., Alsergrund}!Frankgasse 1@\textbf{Frankgasse 1}, \emph{Wohngebäude}|pw}.}\pend{}{\bigskip}\vspace{1em}
\pstart
           {\pb}\textsc{Tientsin\oindex{Tianjin@\textbf{Tianjin}|pw}}, 28. September.\pend
           \vspace{0.5em}
\pstart
           Vielen Dank, lieber Freund, für Deine Karte aus
                  \label{K_L02859-1v}\edtext{Genf\oindex{Genf@\textbf{Genf}|pw}}{\lemma{\textnormal{\emph{Genf}}}\Cendnote{\textnormal{Schnitzler hielt sich vom 16. 8. 1898 bis zum 18. 8. 1898 in Genf\oindex{Genf@\textbf{Genf}|pwk} auf und traf dort an den ersten beiden
                  Tagen seine Schwester Gisela\pwindex{Hajek, Gisela 20.\,12.\,1867 Wien – 3.\,2.\,1953 Cambridge@\textsc{Hajek, Gisela} (20.\,12.\,1867 Wien – 3.\,2.\,1953 Cambridge)|pwk} und seinen
                  Schwager Markus Hajek\pwindex{Hajek, Markus 25.\,11.\,1861 Vršac – 4.\,4.\,1941 London@\textsc{Hajek, Markus} (25.\,11.\,1861 Vršac – 4.\,4.\,1941 London), \emph{Mediziner, Laryngologe}|pwk} sowie Hugo von Hofmannsthal\pwindex{Hofmannsthal, Hugo von 1.\,2.\,1874 Wien – 15.\,7.\,1929 Rodaun@\textsc{Hofmannsthal, Hugo von} (1.\,2.\,1874 Wien – 15.\,7.\,1929 Rodaun), \emph{Schriftsteller}|pwk}.}}}\label{K_L02859-1}! Bitte, auch
               Deiner Frau Schweſter\pwindex{Hajek, Gisela 20.\,12.\,1867 Wien – 3.\,2.\,1953 Cambridge@\textsc{Hajek, Gisela} (20.\,12.\,1867 Wien – 3.\,2.\,1953 Cambridge)|pwv}, Deinem
               Herrn Schwager\pwindex{Hajek, Markus 25.\,11.\,1861 Vršac – 4.\,4.\,1941 London@\textsc{Hajek, Markus} (25.\,11.\,1861 Vršac – 4.\,4.\,1941 London), \emph{Mediziner, Laryngologe}|pwv} und Herrn von
                  \textsc{Hoffmansthal\pwindex{Hofmannsthal, Hugo von 1.\,2.\,1874 Wien – 15.\,7.\,1929 Rodaun@\textsc{Hofmannsthal, Hugo von} (1.\,2.\,1874 Wien – 15.\,7.\,1929 Rodaun), \emph{Schriftsteller}|pw}} für die freundlichen Grüße zu \strikeout{danken!} danken!\pend
           
\pstart
           Dein{\\[\baselineskip]}\spacefill\mbox{P. G.}\pend
           \leftskip=0em{}\selectlanguage{ngerman}\endnumbering\briefempfaengerindex{Schnitzler, Arthur@\textsc{Schnitzler, Arthur}!zzzGoldmann, Paul@\emph{von Paul Goldmann}!1898-09-281@{28. 9. 1898}|)be}\mylabel{L02859h}  \newcommand{\dateiname}{L02859}\newcommand{\titel}{Paul Goldmann an Arthur Schnitzler, 28. 9. 1898}\newcommand{\editorInnen}{Martin Anton Müller und Laura Untner}%% latex-leseansicht-abspann.tex
%% Abspann für die Leseansicht.
%% Der Schalter \ifkorrekturansicht ist bereits durch den Vorspann gesetzt.

%% latex-abspann.tex
%% Gemeinsamer Abspann für Korrekturansicht und Leseansicht.
%% Setzt den Schalter \ifkorrekturansicht voraus (gesetzt in den
%% einbindenden Dateien latex-korrekturansicht-abspann.tex bzw.
%% latex-leseansicht-abspann.tex).
%% ---------------------------------------------------------------

\normalsize

% Das esempio-Environment wird nur in der Leseansicht benötigt
\ifkorrekturansicht\else
\newenvironment{esempio}[3]%
{
    \vspace{1.5ex}
    \rlap{\underline{#1}}
    \par
    \setlength{\parindent}{0cm}
    \nopagebreak
    \leftskip=#2cm
    \rightskip=#3cm
}
{
    \par
}
\fi

\doendnotes{C}
\bigskip
\vfill

\clearpage

\footnotesize

\ifkorrekturansicht
  \lohead{\textsc{register}}
\fi

% theindex-Environment neu definieren ohne reledmac
\makeatletter
\renewenvironment{theindex}{%
  \ifkorrekturansicht
    \section*{\indexname}%
  \else
    \subsubsection*{Index der erwähnten Entitäten}%
  \fi
  \setlength{\parindent}{0pt}%
  \setlength{\parskip}{0pt plus 0.3pt}%
  \let\item\@idxitem
}{%
  \ifkorrekturansicht\clearpage\fi
}
\makeatother

\IfFileExists{\jobname-pw.ind}{\input{\jobname-pw.ind}}{}

% Quellenangabe nur in der Leseansicht
\ifkorrekturansicht\else
% Fallback-Definitionen, falls die .tex-Datei \titel etc. nicht gesetzt hat
\providecommand{\titel}{}
\providecommand{\editorInnen}{}
\providecommand{\dateiname}{\jobname}

\vspace{3cm}

\vfill

\footnotesize
\textsc{Quelle}: \titel. Herausgegeben von {\editorInnen}. In: \emph{Arthur Schnitzler: Briefwechsel mit Autorinnen und Autoren}.
 Digitale Edition, https://schnitzler-briefe.acdh.oeaw.ac.at/{\dateiname}.html (Stand \today)
\fi

\end{document}


