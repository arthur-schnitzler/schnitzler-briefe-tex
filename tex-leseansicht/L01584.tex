%% latex-leseansicht-vorspann.tex
%% Vorspann für die Leseansicht.
%% Lädt die gemeinsame Datei latex-vorspann.tex mit nicht gesetztem Schalter.

\newif\ifkorrekturansicht
\korrekturansichtfalse

\input{../tex-inputs/latex-vorspann}


\section[Charlotte Ehrenstein an Arthur Schnitzler, {[}Mitte Februar 1906?{]}]{L01584 Charlotte Ehrenstein an Arthur Schnitzler, {[}Mitte Februar 1906?{]}}
\nopagebreak\mylabel{L01584v}
\rehead{ }\normalsize\beginnumbering\briefempfaengerindex{Schnitzler, Arthur@\textsc{Schnitzler, Arthur}!zzzEhrenstein, Charlotte@\emph{von Charlotte Ehrenstein}!1906-02-151@{{[}Mitte Februar 1906?{]}}|(be}
\toendnotes[C]{\smallbreak\pagebreak[2]}
\correspDesc{Versand  durch Charlotte Ehrenstein am [Mitte Februar 1906?] in Wien
\newline{}Erhalt  durch Arthur Schnitzler im Zeitraum [15. 2. 1906
                  – 19. 2. 1906?] in Wien}\toendnotes[C]{\smallbreak}
\Standort{DLA, A:Schnitzler, HS.NZ85.1.2837,2.}
\physDesc{Brief, 1 Blatt, 2 Seiten, 819 Zeichen
\newline{}Handschrift: Bleistift, deutsche Kurrent
\newline{}Schnitzler: mit Bleistift beschriftet: »\textsc{Ehrenstein}« }\toendnotes[C]{\smallbreak}
\pstart
           {\pb}\textsc{Hochwohlgeb. Herrn Dr. Arthur Schnitzler}.\pend
           
\pstart\center{}Sehr geehrter Herr Doctor!\pend\vspace{0.5em}
\pstart
           Heute darf ich über das Befinden meines l. Albert\pwindex{Ehrenstein, Albert 23.\,12.\,1886 Wien – 8.\,4.\,1950 New York City@\textsc{Ehrenstein, Albert} (23.\,12.\,1886 Wien – 8.\,4.\,1950 New York City), \emph{Schriftsteller}|pw}{ }ſchon recht Befriedigendes berichten. \label{K_L01584-1v}\edtext{Vor einigen Tagen}{\lemma{\textnormal{\emph{Vor einigen Tagen}}}\Cendnote{\textnormal{Das letzte mit Gewissheit zu datierende Korrespondenzstück
                  stammt vom XXXX Auszeichnungsfehler: Dokument L01579 nicht gefunden.
                  Entsprechend des anzunehmenden Krankheitsverlaufs dürfte dieses Schreiben wenige
                  Wochen danach abgefasst worden sein.}}}\label{K_L01584-1} war Dr Kornfeld\pwindex{Kornfeld, Sigmund 21.\,4.\,1859 Golčův Jeníkov – 15.\,4.\,1927 Wien@\textsc{Kornfeld, Sigmund} (21.\,4.\,1859 Golčův Jeníkov – 15.\,4.\,1927 Wien), \emph{Psychiater}|pw} hier, u. erlaubte ihm, da er Zuſtand und Ausſehen befriedigend
               fand, Albert\pwindex{Ehrenstein, Albert 23.\,12.\,1886 Wien – 8.\,4.\,1950 New York City@\textsc{Ehrenstein, Albert} (23.\,12.\,1886 Wien – 8.\,4.\,1950 New York City), \emph{Schriftsteller}|pw} nahm während{ }ſeiner Krankheit
               fünf Kilo an Gewicht zu, täglich von 3–5 Nachmittags das Bett zu verlaſſen. Auch über{ }ſein weiteres Studium{ }ſprach er mit ihm, er{ }ſchlägt Albert\pwindex{Ehrenstein, Albert 23.\,12.\,1886 Wien – 8.\,4.\,1950 New York City@\textsc{Ehrenstein, Albert} (23.\,12.\,1886 Wien – 8.\,4.\,1950 New York City), \emph{Schriftsteller}|pw}en das Mittelſchulprofeſſor-Studium vor, Geographie, Geſchichte und
               Deutſch oder Naturgeſchichte, da er {\pb}meint, das
               Doctorat in Medicin für Albert\pwindex{Ehrenstein, Albert 23.\,12.\,1886 Wien – 8.\,4.\,1950 New York City@\textsc{Ehrenstein, Albert} (23.\,12.\,1886 Wien – 8.\,4.\,1950 New York City), \emph{Schriftsteller}|pw}{ }ſchwer zu erringen{ }ſein würde. Und nun bitte ich,
               mir zu verzeihen, wenn ich außer mit meinem Heutigem, noch mit der Bitte um Ihre
               Meinung beläſtige, da{ }ſie uns allen{ }ſehr maßgebend iſt, vor allen aber, Ihrer, Sie\pend
           
\pstart
           verehrenden{\\[\baselineskip]}\spacefill\mbox{Charlotte Ehrenſtein}\pend
           \leftskip=0em{}\selectlanguage{ngerman}\endnumbering\briefempfaengerindex{Schnitzler, Arthur@\textsc{Schnitzler, Arthur}!zzzEhrenstein, Charlotte@\emph{von Charlotte Ehrenstein}!1906-02-151@{{[}Mitte Februar 1906?{]}}|)be}\mylabel{L01584h}  \newcommand{\dateiname}{L01584}\newcommand{\titel}{Charlotte Ehrenstein an Arthur Schnitzler, [Mitte Februar 1906?]}\newcommand{\editorInnen}{Martin Anton Müller und Gerd-Hermann Susen}%% latex-leseansicht-abspann.tex
%% Abspann für die Leseansicht.
%% Der Schalter \ifkorrekturansicht ist bereits durch den Vorspann gesetzt.

%% latex-abspann.tex
%% Gemeinsamer Abspann für Korrekturansicht und Leseansicht.
%% Setzt den Schalter \ifkorrekturansicht voraus (gesetzt in den
%% einbindenden Dateien latex-korrekturansicht-abspann.tex bzw.
%% latex-leseansicht-abspann.tex).
%% ---------------------------------------------------------------

\normalsize

% Das esempio-Environment wird nur in der Leseansicht benötigt
\ifkorrekturansicht\else
\newenvironment{esempio}[3]%
{
    \vspace{1.5ex}
    \rlap{\underline{#1}}
    \par
    \setlength{\parindent}{0cm}
    \nopagebreak
    \leftskip=#2cm
    \rightskip=#3cm
}
{
    \par
}
\fi

\doendnotes{C}
\bigskip
\vfill

\clearpage

\footnotesize

\ifkorrekturansicht
  \lohead{\textsc{register}}
\fi

% theindex-Environment neu definieren ohne reledmac
\makeatletter
\renewenvironment{theindex}{%
  \ifkorrekturansicht
    \section*{\indexname}%
  \else
    \subsubsection*{Index der erwähnten Entitäten}%
  \fi
  \setlength{\parindent}{0pt}%
  \setlength{\parskip}{0pt plus 0.3pt}%
  \let\item\@idxitem
}{%
  \ifkorrekturansicht\clearpage\fi
}
\makeatother

\IfFileExists{\jobname-pw.ind}{\input{\jobname-pw.ind}}{}

% Quellenangabe nur in der Leseansicht
\ifkorrekturansicht\else
% Fallback-Definitionen, falls die .tex-Datei \titel etc. nicht gesetzt hat
\providecommand{\titel}{}
\providecommand{\editorInnen}{}
\providecommand{\dateiname}{\jobname}

\vspace{3cm}

\vfill

\footnotesize
\textsc{Quelle}: \titel. Herausgegeben von {\editorInnen}. In: \emph{Arthur Schnitzler: Briefwechsel mit Autorinnen und Autoren}.
 Digitale Edition, https://schnitzler-briefe.acdh.oeaw.ac.at/{\dateiname}.html (Stand \today)
\fi

\end{document}


