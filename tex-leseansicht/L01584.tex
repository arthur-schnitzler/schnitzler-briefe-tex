%% latex-korrekturansicht-vorspann.tex
%% Vorspann für die Korrekturansicht.
%% Lädt die gemeinsame Datei latex-vorspann.tex mit gesetztem Schalter.

\newif\ifkorrekturansicht
\korrekturansichttrue

\input{../tex-inputs/latex-vorspann}


\section[Charlotte Ehrenstein an Arthur Schnitzler, {[}Mitte Februar 1906?{]}]{L01584 Charlotte Ehrenstein an Arthur Schnitzler, {[}Mitte Februar
               1906?{]}}
\nopagebreak\mylabel{L01584v}
\rehead{ }\normalsize\beginnumbering\briefempfaengerindex{Schnitzler, Arthur@\textsc{Schnitzler, Arthur}!zzzEhrenstein, Charlotte@\emph{von Charlotte Ehrenstein}!1906-02-151@{{[}Mitte Februar 1906?{]}}|(be}
\toendnotes[C]{\smallbreak\pagebreak[2]}\Standort{DLA, A:Schnitzler, HS.NZ85.1.2837,2.}
\physDesc{Brief, 1 Blatt, 2 Seiten, 819 Zeichen
\newline{}Handschrift: Bleistift, deutsche Kurrent
\newline{}Schnitzler: mit Bleistift beschriftet: »\textsc{Ehrenstein}« }\toendnotes[C]{\smallbreak}
\pstart
           {\pb}\textsc{Hochwohlgeb. Herrn Dr. Arthur Schnitzler}.\pend
           
\pstart\center{}Sehr geehrter Herr Doctor!\pend\vspace{0.5em}
\pstart
           Heute darf ich über das Befinden meines l. Albert\pwindex{Ehrenstein, Albert 23.12.1886 – 08.04.1950@\textsc{Ehrenstein, Albert} (23.12.1886 – 08.04.1950), \emph{Schriftsteller/Schriftstellerin}|pw}{ }ſchon recht Befriedigendes berichten. \label{K_L01584-1v}\edtext{Vor einigen Tagen}{\lemma{\textnormal{\emph{Vor einigen Tagen}}}\Cendnote{\textnormal{Das letzte mit Gewissheit zu datierende Korrespondenzstück
                  stammt vom 29. 1. 1906.
                  Entsprechend des anzunehmenden Krankheitsverlaufs dürfte dieses Schreiben wenige
                  Wochen danach abgefasst worden sein.}}}\label{K_L01584-1} war Dr Kornfeld\pwindex{Kornfeld, Sigmund 21.04.1859 – 15.04.1927@\textsc{Kornfeld, Sigmund} (21.04.1859 – 15.04.1927), \emph{Psychiater/Psychiaterin}|pw} hier, u. erlaubte ihm, da er Zuſtand und Ausſehen befriedigend
               fand, Albert\pwindex{Ehrenstein, Albert 23.12.1886 – 08.04.1950@\textsc{Ehrenstein, Albert} (23.12.1886 – 08.04.1950), \emph{Schriftsteller/Schriftstellerin}|pw} nahm während ſeiner Krankheit
               fünf Kilo an Gewicht zu, täglich von 3–5 Nachmittags das Bett zu verlaſſen. Auch über
               ſein weiteres Studium ſprach er mit ihm, er ſchlägt Albert\pwindex{Ehrenstein, Albert 23.12.1886 – 08.04.1950@\textsc{Ehrenstein, Albert} (23.12.1886 – 08.04.1950), \emph{Schriftsteller/Schriftstellerin}|pw}en das Mittelſchulprofeſſor-Studium vor, Geographie, Geſchichte und
               Deutſch oder Naturgeſchichte, da er {\pb}meint, das
               Doctorat in Medicin für Albert\pwindex{Ehrenstein, Albert 23.12.1886 – 08.04.1950@\textsc{Ehrenstein, Albert} (23.12.1886 – 08.04.1950), \emph{Schriftsteller/Schriftstellerin}|pw}{ }ſchwer zu erringen ſein würde. Und nun bitte ich,
               mir zu verzeihen, wenn ich außer mit meinem Heutigem, noch mit der Bitte um Ihre
               Meinung beläſtige, da ſie uns allen ſehr maßgebend iſt, vor allen aber, Ihrer, Sie \pend
           
\pstart
           verehrenden{\\[\baselineskip]}\spacefill\mbox{Charlotte Ehrenſtein}\pend
           \leftskip=0em{}\selectlanguage{ngerman}\endnumbering\briefempfaengerindex{Schnitzler, Arthur@\textsc{Schnitzler, Arthur}!zzzEhrenstein, Charlotte@\emph{von Charlotte Ehrenstein}!1906-02-151@{{[}Mitte Februar 1906?{]}}|)be}\mylabel{L01584h}  \normalsize

\doendnotes{C}
\bigskip
\vfill

\clearpage

\footnotesize

\lohead{\textsc{register}}

% Definiere theindex-Environment komplett neu ohne reledmac
\makeatletter
\renewenvironment{theindex}{%
  \section*{\indexname}%
  \setlength{\parindent}{0pt}%
  \setlength{\parskip}{0pt plus 0.3pt}%
  \let\item\@idxitem
}{%
  \clearpage
}
\makeatother

\IfFileExists{\jobname-pw.ind}{\input{\jobname-pw.ind}}{}

\end{document}

      