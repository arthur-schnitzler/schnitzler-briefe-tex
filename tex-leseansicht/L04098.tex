%% latex-leseansicht-vorspann.tex
%% Vorspann für die Leseansicht.
%% Lädt die gemeinsame Datei latex-vorspann.tex mit nicht gesetztem Schalter.

\newif\ifkorrekturansicht
\korrekturansichtfalse

\input{../tex-inputs/latex-vorspann}


\section[Arthur Schnitzler an Gustav Schwarzkopf, 12. 8. 1893]{L04098 Arthur Schnitzler an Gustav Schwarzkopf, 12. 8. 1893}
\nopagebreak\mylabel{L04098v}
\rehead{ }\normalsize\beginnumbering\briefempfaengerindex{Schwarzkopf, Gustav@\textsc{Schwarzkopf, Gustav}!zzzSchnitzler, Arthur@\emph{von Arthur Schnitzler}!1893-08-121@{12. 8. 1893}|(be}
\toendnotes[C]{\smallbreak\pagebreak[2]}
\correspDesc{Versand  durch Arthur Schnitzler am 12. 8. 1893 in Wien
\newline{}Erhalt  durch Gustav Schwarzkopf am 12. 8. 1893 in Mödling}\toendnotes[C]{\smallbreak}
\Standort{CUL, Schnitzler, B 96.}
\physDesc{Kartenbrief, 377 Zeichen
\newline{}Handschrift: Bleistift, deutsche Kurrent
\newline{}Versand: 1) Stempel: »\nobreak{}\oindex{IX., Alsergrund@\textbf{IX., Alsergrund}, \emph{Verwaltungsgebiet}|pwk}Wien 9/3 72, 12 8 9\textcolor{gray}{3}, 2–3N\nobreak{}«.   2) Stempel: »\nobreak{}\oindex{Mödling@\textbf{Mödling}, \emph{Hauptstadt}|pwk}Mödling Stadt, 12 8 9\textcolor{gray}{3}, 7A\nobreak{}«. }\toendnotes[C]{\smallbreak}\pstart{}{\pb}Herrn \textsc{Gustav
                  Schwarzkopf}\pend{}\pstart{}\textsc{Mödling}\oindex{Mödling@\textbf{Mödling}, \emph{Hauptstadt}|pw}{ }\textcolor{gray}{\textbf{in}}{ }\textsc{Brühl}\oindex{Vorderbrühl@\textbf{Vorderbrühl}|pw},\pend{}\pstart{}Reſtaurant Schönberger\oindex{Zum goldenen Stern@\textbf{Zum goldenen Stern}, \emph{Lokal}|pw}.\pend{}{\bigskip}\vspace{1em}
\pstart
           \noindent{}{\pb}Verehrter Herr Schwarzkopf, ich fahre zum  Eſſen \substVorne{}\textsuperscript{heut}\substDazwischen{}\label{K_L04098-1v}\edtext{So{\geminationn}tag}{\lemma{\textnormal{\emph{Sonntag}}}\Cendnote{\textnormal{Vgl. A. S.: \emph{Tagebuch}, 13. 8. 1893.
                     }}}\label{K_L04098-1}\substHinten{} nach Baden\oindex{Baden bei Wien@\textbf{Baden bei Wien}, \emph{Hauptstadt}|pw} (Einladung\pwindex{Schlesinger, Julius 30.\,4.\,1837 Budapest – 22.\,3.\,1927 Wien@\textsc{Schlesinger, Julius} (30.\,4.\,1837 Budapest – 22.\,3.\,1927 Wien), \emph{Bankier}|pwv}\pwindex{Schlesinger, Therese 28.\,7.\,1846 Wien – 29.\,12.\,1935 ebd.@\textsc{Schlesinger, Therese} (28.\,7.\,1846 Wien – 29.\,12.\,1935 ebd.)|pwv}\pwindex{Berger, Else 20.\,10.\,1874 Wien – 24.\,11.\,1956 ebd.@\textsc{Berger, Else} (20.\,10.\,1874 Wien – 24.\,11.\,1956 ebd.)|pwv}), weiſs alſo nicht, ob ich reſp. wann ich nach der Brühl\oindex{Vorderbrühl@\textbf{Vorderbrühl}|pw} ko{\geminationm}e. Sie
               laſſen ſich jedenfalls in nichts ſtören, wenn ich bitten darf; ich begnüge mich für
               heute, Sie und Ihren Bruder\pwindex{Schwarzkopf, Max 12.\,6.\,1857 Wien – 14.\,4.\,1928 ebd.@\textsc{Schwarzkopf, Max} (12.\,6.\,1857 Wien – 14.\,4.\,1928 ebd.), \emph{Rechtsanwalt}|pwv}
               herzlich zu grüßen.\pend
           
\pstart
           Ihr ſtets ergebner{\\[\baselineskip]}\spacefill\mbox{Arthur Schnitzler}\pend
           \leftskip=0em{}
\pstart
           \uline{12. 8. 93.}\pend
           \selectlanguage{ngerman}\endnumbering\briefempfaengerindex{Schwarzkopf, Gustav@\textsc{Schwarzkopf, Gustav}!zzzSchnitzler, Arthur@\emph{von Arthur Schnitzler}!1893-08-121@{12. 8. 1893}|)be}\mylabel{L04098h}
\begin{anhang}
\end{anhang}\newcommand{\dateiname}{L04098}\newcommand{\titel}{Arthur Schnitzler an Gustav Schwarzkopf, 12. 8. 1893}\newcommand{\editorInnen}{Herausgegeben von Jahnke, SelmaMüller, Martin Anton}%% latex-leseansicht-abspann.tex
%% Abspann für die Leseansicht.
%% Der Schalter \ifkorrekturansicht ist bereits durch den Vorspann gesetzt.

%% latex-abspann.tex
%% Gemeinsamer Abspann für Korrekturansicht und Leseansicht.
%% Setzt den Schalter \ifkorrekturansicht voraus (gesetzt in den
%% einbindenden Dateien latex-korrekturansicht-abspann.tex bzw.
%% latex-leseansicht-abspann.tex).
%% ---------------------------------------------------------------

\normalsize

% Das esempio-Environment wird nur in der Leseansicht benötigt
\ifkorrekturansicht\else
\newenvironment{esempio}[3]%
{
    \vspace{1.5ex}
    \rlap{\underline{#1}}
    \par
    \setlength{\parindent}{0cm}
    \nopagebreak
    \leftskip=#2cm
    \rightskip=#3cm
}
{
    \par
}
\fi

\doendnotes{C}
\bigskip
\vfill

\clearpage

\footnotesize

\ifkorrekturansicht
  \lohead{\textsc{register}}
\fi

% theindex-Environment neu definieren ohne reledmac
\makeatletter
\renewenvironment{theindex}{%
  \ifkorrekturansicht
    \section*{\indexname}%
  \else
    \subsubsection*{Index der erwähnten Entitäten}%
  \fi
  \setlength{\parindent}{0pt}%
  \setlength{\parskip}{0pt plus 0.3pt}%
  \let\item\@idxitem
}{%
  \ifkorrekturansicht\clearpage\fi
}
\makeatother

\IfFileExists{\jobname-pw.ind}{\input{\jobname-pw.ind}}{}

% Quellenangabe nur in der Leseansicht
\ifkorrekturansicht\else
% Fallback-Definitionen, falls die .tex-Datei \titel etc. nicht gesetzt hat
\providecommand{\titel}{}
\providecommand{\editorInnen}{}
\providecommand{\dateiname}{\jobname}

\vspace{3cm}

\vfill

\footnotesize
\textsc{Quelle}: \titel. Herausgegeben von {\editorInnen}. In: \emph{Arthur Schnitzler: Briefwechsel mit Autorinnen und Autoren}.
 Digitale Edition, https://schnitzler-briefe.acdh.oeaw.ac.at/{\dateiname}.html (Stand \today)
\fi

\end{document}


