%% latex-leseansicht-vorspann.tex
%% Vorspann für die Leseansicht.
%% Lädt die gemeinsame Datei latex-vorspann.tex mit nicht gesetztem Schalter.

\newif\ifkorrekturansicht
\korrekturansichtfalse

\input{../tex-inputs/latex-vorspann}


         
         \renewcommand{\erwaehntePersonen}{Personen: Otto Brahm, Paul Goldmann, Raphael Löwenfeld, Walther Rathenau, Olga Schnitzler, Elisabeth Steinrück}
         \renewcommand{\erwaehnteInstitutionen}{Institutionen: S. Hirzel Verlag (Leipzig), Schiller-Theater}
         \renewcommand{\erwaehnteOrte}{Orte: Berlin, Dessauer Straße, Kaltenleutgeben, Leipzig, Schweiz, Südtirol, Tirol, Wien}
         \renewcommand{\erwaehnteWerke}{Werke: Der Schleier der Beatrice. Schauspiel in fünf Akten, Impressionen}
               \section[ Paul Goldmann an Arthur Schnitzler, 25. 7. {[}1902{]}]{ Paul Goldmann an Arthur Schnitzler, 25. 7. {[}1902{]}}\nopagebreak\mylabel{v}\rehead{ }\begin{ledgroupsized}[t]{13cm}\normalsize\beginnumbering\briefempfaengerindex{Schnitzler, Arthur@\textsc{Schnitzler, Arthur}!zzzGoldmann, Paul@\emph{von Paul Goldmann}!1902-07-251@{25. 7. {[}1902{]}}|(be} \toendnotes[C]{\smallbreak\pagebreak[2]} \Standort{DLA, A:Schnitzler, HS.NZ85.1.3172.}
\physDesc{Brief, 1 Blatt, 4 Seiten, 1196 Zeichen
\newline{}Handschrift: blaue Tinte, deutsche Kurrent
\newline{}Schnitzler: mit Bleistift das Jahr »902« vermerkt }\toendnotes[C]{\smallbreak}\pstart
           \noindent{}\raggedleft{}{\pb}\textcolor{gray}{\textbf{DESSAUERSTRASSE 19}}\oindex{Dessauer Strasse@\textbf{Dessauer Straße}|pw}\pend
           \pstart
           Berlin\oindex{Berlin@\textbf{Berlin}|pw}, 25. Juli.\pend
           \pstart\center{}Mein lieber Freund,\pend\pstart
           Nach langem Schwanken habe ich mich entſchloſſen, in die Schweiz\oindex{Schweiz@\textbf{Schweiz}|pw} zu gehen. Ich komme alſo nicht über Wien\oindex{Wien@\textbf{Wien}|pw}. Der Wien\oindex{Wien@\textbf{Wien}|pw}er Aufenthalt
               hat mir zu \label{K_L03214-1v}\edtext{Pfingſten}{\lemma{\textnormal{\emph{Pfingſten}}}\Cendnote{\textnormal{siehe Paul Goldmann an Arthur Schnitzler, 5. 5. [1902]}}}\label{K_L03214-1h} gar nicht gut gethan; ich \strikeout{kam} bin ſehr
               angegriffen zurückgekehrt. Nach Tirol\oindex{Tirol@\textbf{Tirol}|pw}\oindex{Suedtirol@\textbf{Südtirol}|pw}
               gehe ich nicht, weil ich fürchte, dort zu viel Bekannte zu treffen und in ein
               ermüdendes geſellſchaftliches {\pb}\substVorne{}\textsuperscript{Treiben}{\allowbreak}\substDazwischen{}Treiben\substHinten{} hineinzugerathen. Ich will einmal ein paar Wochen lang ganz der Ruhe leben
               und es ſogar mit der Einſamkeit verſuchen. Vielleicht thut dieſe meinen gequälten
               Nerven gut.\pend
           \pstart
           Es thut mir unendlich leid, daß ich durch dieſe Änderung meiner Reiſepläne auch der
               Freude verluſtig gehe, Dich wiederzuſehen. Ich rechne aber ſehr darauf, daß die
                  \label{K_L03214-2v}\edtext{»\textsc{Beatrice\pwindex{Schnitzler, Arthur 15.05.1862 – 21.10.1931@\textsc{Schnitzler, Arthur} (15.05.1862 – 21.10.1931), \emph{Schriftsteller, Mediziner}!Schleier der Beatrice. Schauspiel in fuenf Akten1900-12-01@\strich\emph{Der Schleier der Beatrice. Schauspiel in fünf Akten} {[}1900-12-01{]}|pw}}«-Angelegenheit}{\lemma{\textnormal{\emph{»Beatrice«-Angelegenheit}}}\Cendnote{\textnormal{siehe Paul Goldmann an Arthur Schnitzler, 14. 7. [1902]}}}\label{K_L03214-2h}{ }{\pb}Dich ſchon im Anfang des Winters nach Berlin\oindex{Berlin@\textbf{Berlin}|pw} führen wird. Hat \label{K_L03214-3v}\edtext{\textsc{Brahm\pwindex{Brahm, Otto 05.02.1856 – 28.11.1912@\textsc{Brahm, Otto} (05.02.1856 – 28.11.1912), \emph{Theaterleiter, Regisseur}|pw}}}{\lemma{\textnormal{\emph{Brahm}}}\Cendnote{\textnormal{Vgl. \emph{Der Briefwechsel Arthur Schnitzler — Otto
                        Brahm}. Vollständige Ausgabe. Herausgegeben, eingeleitet und
                     erläutert von Oskar Seidlin. Tübingen:
                        \emph{Niemeyer}{ }1975, S. 126–127.}}}\label{K_L03214-3h} geantwortet? Und in
               welchem Sinne? \textsc{Dr. Löwenfeld\pwindex{Loewenfeld, Raphael 11.02.1854 – 28.12.1910@\textsc{Löwenfeld, Raphael} (11.02.1854 – 28.12.1910), \emph{Theaterleiter}|pw}}, vom »Schillertheater\orgindex{Schiller-Theater@Schiller-Theater|pw}«, iſt in Kaltenleutgeben\oindex{Kaltenleutgeben@\textbf{Kaltenleutgeben}|pw}; und wenn Du mit \textsc{Brahm\pwindex{Brahm, Otto 05.02.1856 – 28.11.1912@\textsc{Brahm, Otto} (05.02.1856 – 28.11.1912), \emph{Theaterleiter, Regisseur}|pw}} nicht einig wirſt (was ich aber hoffe) kannſt Du gleich mit ihm verhandeln.\pend
           \pstart
           Ich bleibe noch etwa acht Tage hier\oindex{Berlin@\textbf{Berlin}|pwv} und hoffe, von Dir bald zu hören. {\pb}Grüß\textcolor{gray}{e} mir \textsc{Olga\pwindex{Schnitzler, Olga 17.01.1882 – 13.01.1970@\textsc{Schnitzler, Olga} (17.01.1882 – 13.01.1970), \emph{Schauspielerin, Sängerin}|pw}} und \textsc{Liesl\pwindex{Steinrueck, Elisabeth 19.11.1885 – 07.04.1920@\textsc{Steinrück, Elisabeth} (19.11.1885 – 07.04.1920)|pw}} und ſei Du ſelbſt vielmals und von Herzen gegrüßt von {\\[\baselineskip]}Deinem getreuen {\\[\baselineskip]}\spacefill\mbox{Paul Goldmn}\pend
           \leftskip=0em{}\pstart
           \noindent{}Lies das Buch \label{K_L03214-4v}\edtext{»Impreſſionen\pwindex{Rathenau, Walther 29.09.1867 – 24.06.1922@\textsc{Rathenau, Walther} (29.09.1867 – 24.06.1922), \emph{Politiker, Industrieller}!Impressionen1902@\strich\emph{Impressionen} {[}1902{]}|pw}« von \textsc{Walther Rathenau\pwindex{Rathenau, Walther 29.09.1867 – 24.06.1922@\textsc{Rathenau, Walther} (29.09.1867 – 24.06.1922), \emph{Politiker, Industrieller}|pw}}}{\lemma{\textnormal{\emph{»Impreſſionen« … Rathenau}}}\Cendnote{\textnormal{Walter Rathenau\pwindex{Rathenau, Walther 29.09.1867 – 24.06.1922@\textsc{Rathenau, Walther} (29.09.1867 – 24.06.1922), \emph{Politiker, Industrieller}|pwk}: \emph{Impressionen}\pwindex{Rathenau, Walther 29.09.1867 – 24.06.1922@\textsc{Rathenau, Walther} (29.09.1867 – 24.06.1922), \emph{Politiker, Industrieller}!Impressionen1902@\strich\emph{Impressionen} {[}1902{]}|pwk}. Leipzig\oindex{Leipzig@\textbf{Leipzig}|pwk}: \emph{S. Hirzel}\orgindex{S. Hirzel Verlag (Leipzig)@S. Hirzel Verlag (Leipzig)|pwk}{ }1902. Eine Lektüre durch Schnitzler\pwindex{Schnitzler, Arthur 15.05.1862 – 21.10.1931@\textsc{Schnitzler, Arthur} (15.05.1862 – 21.10.1931), \emph{Schriftsteller, Mediziner}|pwk}
                     ist nicht bekannt.}}}\label{K_L03214-4h}.\pend
           
         
         \endnumbering\mylabel{h}\end{ledgroupsized}  \newcommand{\dateiname}{L03214}\newcommand{\titel}{Paul Goldmann an Arthur Schnitzler, 25. 7. [1902]}\newcommand{\editorInnen}{Martin Anton Müller und Laura Untner}%% latex-leseansicht-abspann.tex
%% Abspann für die Leseansicht.
%% Der Schalter \ifkorrekturansicht ist bereits durch den Vorspann gesetzt.

%% latex-abspann.tex
%% Gemeinsamer Abspann für Korrekturansicht und Leseansicht.
%% Setzt den Schalter \ifkorrekturansicht voraus (gesetzt in den
%% einbindenden Dateien latex-korrekturansicht-abspann.tex bzw.
%% latex-leseansicht-abspann.tex).
%% ---------------------------------------------------------------

\normalsize

% Das esempio-Environment wird nur in der Leseansicht benötigt
\ifkorrekturansicht\else
\newenvironment{esempio}[3]%
{
    \vspace{1.5ex}
    \rlap{\underline{#1}}
    \par
    \setlength{\parindent}{0cm}
    \nopagebreak
    \leftskip=#2cm
    \rightskip=#3cm
}
{
    \par
}
\fi

\doendnotes{C}
\bigskip
\vfill

\clearpage

\footnotesize

\ifkorrekturansicht
  \lohead{\textsc{register}}
\fi

% theindex-Environment neu definieren ohne reledmac
\makeatletter
\renewenvironment{theindex}{%
  \ifkorrekturansicht
    \section*{\indexname}%
  \else
    \subsubsection*{Index der erwähnten Entitäten}%
  \fi
  \setlength{\parindent}{0pt}%
  \setlength{\parskip}{0pt plus 0.3pt}%
  \let\item\@idxitem
}{%
  \ifkorrekturansicht\clearpage\fi
}
\makeatother

\IfFileExists{\jobname-pw.ind}{\input{\jobname-pw.ind}}{}

% Quellenangabe nur in der Leseansicht
\ifkorrekturansicht\else
% Fallback-Definitionen, falls die .tex-Datei \titel etc. nicht gesetzt hat
\providecommand{\titel}{}
\providecommand{\editorInnen}{}
\providecommand{\dateiname}{\jobname}

\vspace{3cm}

\vfill

\footnotesize
\textsc{Quelle}: \titel. Herausgegeben von {\editorInnen}. In: \emph{Arthur Schnitzler: Briefwechsel mit Autorinnen und Autoren}.
 Digitale Edition, https://schnitzler-briefe.acdh.oeaw.ac.at/{\dateiname}.html (Stand \today)
\fi

\end{document}


      