%% latex-leseansicht-vorspann.tex
%% Vorspann für die Leseansicht.
%% Lädt die gemeinsame Datei latex-vorspann.tex mit nicht gesetztem Schalter.

\newif\ifkorrekturansicht
\korrekturansichtfalse

\input{../tex-inputs/latex-vorspann}

\begin{center}
            \textcolor{red}{ENTWURF, NICHT FERTIG KORRIGIERT}
                      \end{center}
            
         
         \renewcommand{\erwaehntePersonen}{Personen: Otto Brahm, Olga Schnitzler}
         \renewcommand{\erwaehnteOrte}{Orte: Berlin, Dessauer Straße, Wien}
         \renewcommand{\erwaehnteWerke}{Werke: Der Schleier der Beatrice. Schauspiel in fünf Akten}
               \section[ Paul Goldmann an Arthur Schnitzler, 25. 7. {[}1902{]}]{ Paul Goldmann an Arthur Schnitzler, 25. 7. {[}1902{]}}\nopagebreak\mylabel{v}\rehead{ }\begin{ledgroupsized}[t]{13cm}\normalsize\beginnumbering \toendnotes[C]{\smallbreak\pagebreak[2]} \Standort{DLA, A:Schnitzler, HS.NZ85.1.3172.}
\physDesc{Brief, 1 Blatt, 4 Seiten
\newline{}Handschrift: blaue Tinte, deutsche Kurrent
\newline{}Schnitzler: mit Bleistift das Jahr »{[}1{]}902«
                                            vermerkt }\pstart
           \noindent{}\raggedleft{}{\pb}\textcolor{gray}{\textbf{DESSAUERSTRASSE 19}}\oindex{Dessauer Strasse@\textbf{Dessauer Straße}|pw}\pend
           \pstart
           Berlin\oindex{Berlin@\textbf{Berlin}|pw}, 25. Juli.\pend
           \pstart\center{}Mein lieber Freund,\pend\pstart
           Nach langem Schwanken habe ich mich entſchloſſen, in die Schweiz\textcolor{red}{\textsuperscript{\textbf{KEY}}} zu gehen. Ich komme alſo nicht über Wien\textcolor{red}{\textsuperscript{\textbf{KEY}}}. Der Wien\oindex{Wien@\textbf{Wien}|pw}er Aufenthalt hat mir
                    zu Pfingſten gar nicht gut gethan; ich \strikeout{kam} bin ſehr angegriffen zurückgekehrt. Nach Tirol\textcolor{red}{\textsuperscript{\textbf{KEY}}} gehe ich nicht, weil ich fürchte, dort zu viel Bekannte zu
                    treffen und in ein ermüdendes geſellſchaftliches {\pb}Treiben,\strikeout{Treiben} hineinzugerathen. Ich will einmal ein
                    paar Wochen lang ganz der Ruhe leben und es ſogar mit der Einſamkeit verſuchen.
                    Vielleicht thut dieſe meinen gequälten Nerven gut. \pend
           \pstart
           Es thut mir unendlich leid, daß ich durch dieſe Änderung meiner Reiſepläne auch
                    der Freude verluſtig gehe, Dich wiederzuſehen. Ich rechne aber ſehr darauf, daß
                    die »\textsc{Beatrice\pwindex{Schnitzler, Arthur 15.05.1862 – 21.10.1931@\textsc{Schnitzler, Arthur} (15.05.1862 – 21.10.1931), \emph{Schriftsteller, Mediziner}!Schleier der Beatrice. Schauspiel in fuenf Akten1900-12-01@\strich\emph{Der Schleier der Beatrice. Schauspiel in fünf Akten} {[}1900-12-01{]}|pw}}«-Angelegenheit {\pb} Dich ſchon
                        \textcolor{gray}{am} Anfang des Winters nach Berlin\oindex{Berlin@\textbf{Berlin}|pw} führen wird. Hat \textsc{Brahm\pwindex{Brahm, Otto 05.02.1856 – 28.11.1912@\textsc{Brahm, Otto} (05.02.1856 – 28.11.1912), \emph{Theaterleiter, Regisseur}|pw}} geantwortet? Und in welchem Sinne? \textsc{Dr. Löwenfeld\textcolor{red}{\textsuperscript{\textbf{KEY}}}}\textcolor{red}{\textsuperscript{\textbf{KEY}}}, vom »Schillertheater\textcolor{red}{\textsuperscript{\textbf{KEY}}}«,
                    iſt in Kaltenleutgeben\textcolor{red}{\textsuperscript{\textbf{KEY}}}; und wenn Du mit \textsc{Brahm\pwindex{Brahm, Otto 05.02.1856 – 28.11.1912@\textsc{Brahm, Otto} (05.02.1856 – 28.11.1912), \emph{Theaterleiter, Regisseur}|pw}} nicht einig wirſt (was ich aber hoffe) kannſt Du gleich mit ihm
                    verhandeln. \pend
           \pstart
           Ich bleibe noch etwa acht Tage hier\textcolor{red}{\textsuperscript{\textbf{KEY}}} und hoffe,
                    von Dir bald zu hören. {\pb}
                        Grüß\textcolor{gray}{e} mir \textsc{Olga\pwindex{Schnitzler, Olga 17.01.1882 – 13.01.1970@\textsc{Schnitzler, Olga} (17.01.1882 – 13.01.1970), \emph{Schauspielerin, Sängerin}|pw}} und \textsc{Liesl\textcolor{red}{\textsuperscript{\textbf{KEY}}}} und ſei Du ſelbſt vielmals und von Herzen gegrüßt von {\\[\baselineskip]}Deinem
                    getreuen\pend
           \leftskip=0em{}\pstart
           {\\[\baselineskip]}\spacefill\mbox{Paul Goldmn}\pend
           \leftskip=0em{}\pstart
           Lies das Buch »Impreſſionen\textcolor{red}{\textsuperscript{\textbf{KEY}}}«von \textsc{Walther Rathenau\textcolor{red}{\textsuperscript{\textbf{KEY}}}}. \pend
           
         
         \endnumbering\mylabel{h}\end{ledgroupsized}\begin{anhang}\end{anhang}\newcommand{\dateiname}{L03214}\newcommand{\titel}{Paul Goldmann an Arthur Schnitzler, 25. 7. [1902]}\newcommand{\editorInnen}{Martin Anton Müller und Laura Untner}%% latex-leseansicht-abspann.tex
%% Abspann für die Leseansicht.
%% Der Schalter \ifkorrekturansicht ist bereits durch den Vorspann gesetzt.

%% latex-abspann.tex
%% Gemeinsamer Abspann für Korrekturansicht und Leseansicht.
%% Setzt den Schalter \ifkorrekturansicht voraus (gesetzt in den
%% einbindenden Dateien latex-korrekturansicht-abspann.tex bzw.
%% latex-leseansicht-abspann.tex).
%% ---------------------------------------------------------------

\normalsize

% Das esempio-Environment wird nur in der Leseansicht benötigt
\ifkorrekturansicht\else
\newenvironment{esempio}[3]%
{
    \vspace{1.5ex}
    \rlap{\underline{#1}}
    \par
    \setlength{\parindent}{0cm}
    \nopagebreak
    \leftskip=#2cm
    \rightskip=#3cm
}
{
    \par
}
\fi

\doendnotes{C}
\bigskip
\vfill

\clearpage

\footnotesize

\ifkorrekturansicht
  \lohead{\textsc{register}}
\fi

% theindex-Environment neu definieren ohne reledmac
\makeatletter
\renewenvironment{theindex}{%
  \ifkorrekturansicht
    \section*{\indexname}%
  \else
    \subsubsection*{Index der erwähnten Entitäten}%
  \fi
  \setlength{\parindent}{0pt}%
  \setlength{\parskip}{0pt plus 0.3pt}%
  \let\item\@idxitem
}{%
  \ifkorrekturansicht\clearpage\fi
}
\makeatother

\IfFileExists{\jobname-pw.ind}{\input{\jobname-pw.ind}}{}

% Quellenangabe nur in der Leseansicht
\ifkorrekturansicht\else
% Fallback-Definitionen, falls die .tex-Datei \titel etc. nicht gesetzt hat
\providecommand{\titel}{}
\providecommand{\editorInnen}{}
\providecommand{\dateiname}{\jobname}

\vspace{3cm}

\vfill

\footnotesize
\textsc{Quelle}: \titel. Herausgegeben von {\editorInnen}. In: \emph{Arthur Schnitzler: Briefwechsel mit Autorinnen und Autoren}.
 Digitale Edition, https://schnitzler-briefe.acdh.oeaw.ac.at/{\dateiname}.html (Stand \today)
\fi

\end{document}


      