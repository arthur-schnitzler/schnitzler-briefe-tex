%% latex-leseansicht-vorspann.tex
%% Vorspann für die Leseansicht.
%% Lädt die gemeinsame Datei latex-vorspann.tex mit nicht gesetztem Schalter.

\newif\ifkorrekturansicht
\korrekturansichtfalse

\input{../tex-inputs/latex-vorspann}


         
         \renewcommand{\erwaehntePersonen}{Personen: Hermann Bahr, Richard Beer-Hofmann, Édouard Drumont, Ralph Waldo Emerson, Florentine Galliny, Johann Wolfgang von Goethe, Paul Goldmann, Cornelius Herz, Heinrich Kanner, Adele Sandrock, Julius Schnitzler, Helene Schnitzler, Louise Schnitzler, Leopold Sonnemann}
         \renewcommand{\erwaehnteInstitutionen}{Institutionen: Burgtheater, Deutsches Theater Berlin, Frankfurter Zeitung}
         \renewcommand{\erwaehnteOrte}{Orte: Berlin, Deutschland, Frankreich, München, Panama, Paris, Wien, rue Feydeau}
         \renewcommand{\erwaehnteWerke}{Werke: Anatol, Die Zeit. Wiener Wochenschrift, Emerson, Feuilleton. Literatur [Anatol], Feuilleton. Literatur [Sterben], Journal des débats. Politiques et littéraires, Le Courrier français, Liebelei. Schauspiel in drei Akten, Phonographie de l’amour, Sterben. Novelle, Tagebuch, Wiener Abendpost}
               \section[Paul Goldmann an Arthur Schnitzler, 2. 3. {[}1895{]}]{ Paul Goldmann an Arthur Schnitzler, 2. 3. {[}1895{]}}\nopagebreak\mylabel{v}\rehead{ }\begin{ledgroupsized}[t]{13cm}\normalsize\beginnumbering\briefempfaengerindex{Schnitzler, Arthur@\textsc{Schnitzler, Arthur}!zzzGoldmann, Paul@\emph{von Paul Goldmann}!1895-03-021@{2. 3. {[}1895{]}}|(be} \toendnotes[C]{\smallbreak\pagebreak[2]} \Standort{DLA, A:Schnitzler, HS.NZ85.1.3165.}
\physDesc{Brief, 3 Blätter, 11 Seiten, 5253 Zeichen
\newline{}Handschrift: schwarze Tinte, deutsche Kurrent
\newline{}Schnitzler: 1) mit Bleistift das Jahr »95« vermerkt  2) mit rotem Buntstift fünf Unterstreichungen}\toendnotes[C]{\smallbreak}\pstart
           \noindent{}{\pb}\textcolor{gray}{\textbf{\textbf{Frankfurter Zeitung\orgindex{Frankfurter Zeitung@Frankfurter Zeitung|pw}}}}\pend
           \pstart
           \textcolor{gray}{\textbf{(\begin{otherlanguage}{french}Gazette de Francfort\end{otherlanguage}\orgindex{Frankfurter Zeitung@Frankfurter Zeitung|pw}). }}\pend
           \pstart
           \textcolor{gray}{\textbf{\textbf{\begin{otherlanguage}{french}Fondateur M. L.
                              Sonnemann\pwindex{Sonnemann, Leopold 1831-10-29 – 1909-10-30@\textsc{Sonnemann, Leopold} (1831-10-29 – 1909-10-30), \emph{Journalist, Herausgeber}|pw}\end{otherlanguage}.}}}\pend
           \pstart
           \begin{otherlanguage}{french}\textcolor{gray}{\textbf{Journal politique, financier,}}\end{otherlanguage}\hfill \textsc{Paris\oindex{Paris@\textbf{Paris}|pw}}, 2. März.\pend
           \pstart
           \begin{otherlanguage}{french}\textcolor{gray}{\textbf{commercial et littéraire.}}\end{otherlanguage}\pend
           \pstart
           \begin{otherlanguage}{french}\textcolor{gray}{\textbf{\textbf{Paraissant trois fois par jour.}}}\end{otherlanguage}\pend
           \pstart
           \begin{otherlanguage}{french}\textcolor{gray}{\textbf{\textbf{Bureaux à Paris\oindex{Paris@\textbf{Paris}|pw}:}}}\end{otherlanguage}\pend
           \pstart
           \begin{otherlanguage}{french}\textcolor{gray}{\textbf{\textbf{24. Rue Feydeau\oindex{rue Feydeau@\textbf{rue Feydeau}|pw}.}}}\end{otherlanguage}\pend
           \pstart{}Mein lieber Freund,\pend\pstart
           Nun geht es mir langſam wieder beſſer, und ich kann Dir ſchreiben. Als Folge der
               allgemeinen Krankheit hat ſich ein hartnäckiges \label{K_L02729-1v}\edtext{Augenübel}{\lemma{\textnormal{\emph{Augenübel}}}\Cendnote{\textnormal{Syphilis hat eine Entzündung des Auges als mögliche sekundäre Folge.}}}\label{K_L02729-1h}
               ergeben. Es kam zum zweiten Male bereits und hält diesmal lange Wochen vor. Da ich
               meinen Beruf nicht ausſetzen kann, ſollte ich alles Schreiben und Leſen auf das
               unerläßlich Berufliche beſchränken. Da blieb alſo für Briefe nichts übrig. Auch war
               es nicht gut möglich, meinen armen dummen Kopf zu einem andern Gedanken zu bringen
               als zu dem an die Krankheit. Was der Beruf eiſern {\pb}erzwang, \strikeout{ging} ging noch. Sonſt aber ſaß ich da,
               Tage und Nächte, und hörte alle Geſpenſter meines unglückſeeligen Lebens um mich
               ſtreichen. Das wird ſchlimm enden, liebſter Freund.\pend
           \pstart
           Nun laß’ Dich von Herzen beglückwünſchen zur \label{K_L02729-2v}\edtext{Annahme}{\lemma{\textnormal{\emph{Annahme}}}\Cendnote{\textnormal{Am 15. 2. 1895 hatte
                     Schnitzler\pwindex{Schnitzler, Arthur 15.05.1862 – 21.10.1931@\textsc{Schnitzler, Arthur} (15.05.1862 – 21.10.1931), \emph{Schriftsteller, Mediziner}|pwk} die Nachricht, dass \emph{Liebelei}\pwindex{Schnitzler, Arthur 15.05.1862 – 21.10.1931@\textsc{Schnitzler, Arthur} (15.05.1862 – 21.10.1931), \emph{Schriftsteller, Mediziner}!Liebelei. Schauspiel in drei Akten1895-10-09@\strich\emph{Liebelei. Schauspiel in drei Akten} {[}1895-10-09{]}|pwk} am \emph{Deutschen Theater}\orgindex{Deutsches Theater Berlin@Deutsches Theater Berlin|pwk} in Berlin\oindex{Berlin@\textbf{Berlin}|pwk}
                  angenommen worden war, erhalten. Premiere feierte das Stück\pwindex{Schnitzler, Arthur 15.05.1862 – 21.10.1931@\textsc{Schnitzler, Arthur} (15.05.1862 – 21.10.1931), \emph{Schriftsteller, Mediziner}!Liebelei. Schauspiel in drei Akten1895-10-09@\strich\emph{Liebelei. Schauspiel in drei Akten} {[}1895-10-09{]}|pwkv} dort am 4. 2. 1896.}}}\label{K_L02729-2h} im »Deutſchen Theater\orgindex{Deutsches Theater Berlin@Deutsches Theater Berlin|pw}«. \strikeout{\textcolor{gray}{Ve}} Das iſt, in Bezug auf den Vertrieb am deutſch\oindex{Deutschland@\textbf{Deutschland}|pwv}en Markt, womöglich noch beſſer, als das Burgtheater\orgindex{Burgtheater@Burgtheater|pw}. Von Berlin\oindex{Berlin@\textbf{Berlin}|pw}
               aus kommt man direkt in die deutſch\oindex{Deutschland@\textbf{Deutschland}|pwv}e Literatur. Das Alles ſind ſo ſchöne Erfolge; und wenn ich ſehe, wie
               man ſonſt Erfolge davonträgt, und wie Du dazu kommſt: ohne Conceſſion, ohne die \strikeout{le\textcolor{gray}{i}ſ} leiſeſte Nacken-Beugung, {\pb}ruhig und ehrlich und Dir ſelbſt getreu – ſo gibt
               mir das ein recht ſtolzes Bild, und es iſt beinahe noch ſchöner als Dein Stück\pwindex{Schnitzler, Arthur 15.05.1862 – 21.10.1931@\textsc{Schnitzler, Arthur} (15.05.1862 – 21.10.1931), \emph{Schriftsteller, Mediziner}!Liebelei. Schauspiel in drei Akten1895-10-09@\strich\emph{Liebelei. Schauspiel in drei Akten} {[}1895-10-09{]}|pwv}. \strikeout{Ob} Daß die geniale Dame\pwindex{Sandrock, Adele 1863-08-19 – 1937-08-30@\textsc{Sandrock, Adele} (1863-08-19 – 1937-08-30), \emph{Schauspielerin}|pwv}{ }\label{K_L02729-3v}\edtext{keine Schwierigkeiten mehr}{\lemma{\textnormal{\emph{keine … mehr}}}\Cendnote{\textnormal{Adele Sandrock\pwindex{Sandrock, Adele 1863-08-19 – 1937-08-30@\textsc{Sandrock, Adele} (1863-08-19 – 1937-08-30), \emph{Schauspielerin}|pwk} schien zwar keine Drohungen
                  im Hinblick auf die Aufführung von \emph{Liebelei}\pwindex{Schnitzler, Arthur 15.05.1862 – 21.10.1931@\textsc{Schnitzler, Arthur} (15.05.1862 – 21.10.1931), \emph{Schriftsteller, Mediziner}!Liebelei. Schauspiel in drei Akten1895-10-09@\strich\emph{Liebelei. Schauspiel in drei Akten} {[}1895-10-09{]}|pwk}
                  am \emph{Burgtheater}\orgindex{Burgtheater@Burgtheater|pwk} mehr gemacht zu haben, bemühte
                  sich jedoch immer noch täglich um Schnitzlers\pwindex{Schnitzler, Arthur 15.05.1862 – 21.10.1931@\textsc{Schnitzler, Arthur} (15.05.1862 – 21.10.1931), \emph{Schriftsteller, Mediziner}|pwk} Zuneigung.}}}\label{K_L02729-3h} macht, iſt gut. Sie wird wohl wieder anfangen;
               aber ſie kann nichts mehr verderben, und wenn \strikeout{\textcolor{gray}{i}ch} ihr auch alle Teufel der Hölle im Leibe ſäßen. Ob das
                  Burgtheater\orgindex{Burgtheater@Burgtheater|pw} das Stück\pwindex{Schnitzler, Arthur 15.05.1862 – 21.10.1931@\textsc{Schnitzler, Arthur} (15.05.1862 – 21.10.1931), \emph{Schriftsteller, Mediziner}!Liebelei. Schauspiel in drei Akten1895-10-09@\strich\emph{Liebelei. Schauspiel in drei Akten} {[}1895-10-09{]}|pwv} jetzt oder in der nächſten Saiſon
               ſpielt, iſt völlig gleichgiltig. Dir zuliebe möchte ich wünſchen, daß es bald wäre.
               Mir wäre es lieber, ich hätte Dich noch ein halbes Jahr unaufgeführt. Der \textsc{Schnitzler}, der \label{K_L02729-4v}\edtext{»zum
                  klangvollſten Namenkreis moderner {\pb}Schriftſteller
                  gehört\pwindex{Feuilleton. Literatur [Sterben]1895-02-09@\emph{Feuilleton. Literatur [Sterben]} {[}1895-02-09{]}|pwv}«}{\lemma{\textnormal{\emph{»zum … gehört«}}}\Cendnote{\textnormal{Das Zitat stammt aus
                  einer Kritik\pwindex{Feuilleton. Literatur [Sterben]1895-02-09@\emph{Feuilleton. Literatur [Sterben]} {[}1895-02-09{]}|pwkv} zu \emph{Sterben}\pwindex{Schnitzler, Arthur 15.05.1862 – 21.10.1931@\textsc{Schnitzler, Arthur} (15.05.1862 – 21.10.1931), \emph{Schriftsteller, Mediziner}!Sterben. Novelle1894-10-01 – 1894-12-01@\strich\emph{Sterben. Novelle} {[}1894-10-01 – 1894-12-01{]}|pwk}: Bruno Walden\pwindex{Galliny, Florentine 24.06.1845 – 19.07.1913@\textsc{Galliny, Florentine} (24.06.1845 – 19.07.1913), \emph{Schriftstellerin, Journalistin}|pwkv} [ = Florentine Galliny\pwindex{Galliny, Florentine 24.06.1845 – 19.07.1913@\textsc{Galliny, Florentine} (24.06.1845 – 19.07.1913), \emph{Schriftstellerin, Journalistin}|pwk}]: \emph{Feuilleton. Literatur}\pwindex{Feuilleton. Literatur [Sterben]1895-02-09@\emph{Feuilleton. Literatur [Sterben]} {[}1895-02-09{]}|pwk}. In: \emph{Wiener Abendpost}\pwindex{?? Werk@Nicht ermittelte Verfasserinnen und Verfasser!Wiener Abendpost1.7.1863 – 31.12.1921@\emph{Wiener Abendpost} {[}1.7.1863 – 31.12.1921{]}|pwk}, Jg. 192, Nr. 33, 9. 2. 1895, S. 5–6, hier: S. 5.}}}\label{K_L02729-4h},
               kommt mir recht kalt und fremd vor. Aber welch’ eine ſchöne Kritik\pwindex{Feuilleton. Literatur [Sterben]1895-02-09@\emph{Feuilleton. Literatur [Sterben]} {[}1895-02-09{]}|pwv}, dieſer \textsc{Bruno Walden\pwindex{Galliny, Florentine 24.06.1845 – 19.07.1913@\textsc{Galliny, Florentine} (24.06.1845 – 19.07.1913), \emph{Schriftstellerin, Journalistin}|pwv}}. Da iſt einmal Einer\pwindex{Galliny, Florentine 24.06.1845 – 19.07.1913@\textsc{Galliny, Florentine} (24.06.1845 – 19.07.1913), \emph{Schriftstellerin, Journalistin}|pwv},
               der Dich nach Verdienſt würdigt. Der Erfolg iſt umſo größer, als der Ochs\pwindex{Galliny, Florentine 24.06.1845 – 19.07.1913@\textsc{Galliny, Florentine} (24.06.1845 – 19.07.1913), \emph{Schriftstellerin, Journalistin}|pwv} – oder \strikeout{die Gans\pwindex{Galliny, Florentine 24.06.1845 – 19.07.1913@\textsc{Galliny, Florentine} (24.06.1845 – 19.07.1913), \emph{Schriftstellerin, Journalistin}|pwv} –} die Gans\pwindex{Galliny, Florentine 24.06.1845 – 19.07.1913@\textsc{Galliny, Florentine} (24.06.1845 – 19.07.1913), \emph{Schriftstellerin, Journalistin}|pwv} – ſich ſo im \label{K_L02729-5v}\edtext{Urtheil\pwindex{Feuilleton. Literatur [Anatol]1893-08-03@\emph{Feuilleton. Literatur [Anatol]} {[}1893-08-03{]}|pwv} über \textsc{Anatol\pwindex{Schnitzler, Arthur 15.05.1862 – 21.10.1931@\textsc{Schnitzler, Arthur} (15.05.1862 – 21.10.1931), \emph{Schriftsteller, Mediziner}!Anatol1892-10-29@\strich\emph{Anatol} {[}1892-10-29{]}|pw}} vergriffen}{\lemma{\textnormal{\emph{Urtheil … vergriffen}}}\Cendnote{\textnormal{Vgl. Paul Goldmann an Arthur Schnitzler, 8. 8. 1893.
               }}}\label{K_L02729-5h} hat. Auch dazu laß’ Dich von Herzen beglückwünſchen! Und Dank für die
               Überſendung. Es hat mir große Freude gemacht, den Artikel\pwindex{Feuilleton. Literatur [Sterben]1895-02-09@\emph{Feuilleton. Literatur [Sterben]} {[}1895-02-09{]}|pwv} – er iſt überdies ſchön geſchrieben – zu leſen.\pend
           \pstart
           Jedesmal noch ärgere ich mich über den \label{K_L02729-6v}\edtext{Titel}{\lemma{\textnormal{\emph{Titel}}}\Cendnote{\textnormal{Vgl. Paul Goldmann an Arthur Schnitzler, 31. 12. [1894].
               }}}\label{K_L02729-6h} »Liebelei\pwindex{Schnitzler, Arthur 15.05.1862 – 21.10.1931@\textsc{Schnitzler, Arthur} (15.05.1862 – 21.10.1931), \emph{Schriftsteller, Mediziner}!Liebelei. Schauspiel in drei Akten1895-10-09@\strich\emph{Liebelei. Schauspiel in drei Akten} {[}1895-10-09{]}|pw}«. Wenn Du wüßteſt, wie garſtig
               er \strikeout{kli} klingt und wie er das Werk\pwindex{Schnitzler, Arthur 15.05.1862 – 21.10.1931@\textsc{Schnitzler, Arthur} (15.05.1862 – 21.10.1931), \emph{Schriftsteller, Mediziner}!Liebelei. Schauspiel in drei Akten1895-10-09@\strich\emph{Liebelei. Schauspiel in drei Akten} {[}1895-10-09{]}|pwv} verkleinert! {\pb}Daß Du Dir ſo gar nichts ſagen laſſen willſt! Warum
               nicht »Eine Liebſchaft\pwindex{Schnitzler, Arthur 15.05.1862 – 21.10.1931@\textsc{Schnitzler, Arthur} (15.05.1862 – 21.10.1931), \emph{Schriftsteller, Mediziner}!Liebelei. Schauspiel in drei Akten1895-10-09@\strich\emph{Liebelei. Schauspiel in drei Akten} {[}1895-10-09{]}|pwv}«?\pend
           \pstart
           Möchte wiſſen, was Du ſchreibſt und lieſt. Ich leſe gar nicht mehr. Ich habe es
               aufgegeben, – ſtrebe nicht mehr mit – laſſe mich ſinken.\pend
           \pstart
           Und wie lebſt Du? Still oder innerlich bewegt? Gehen neue Dinge vor? Bitte, ſchreib’
               mir ein wenig, wie Du lebſt.\pend
           \pstart
           Und was macht \textsc{Richard\pwindex{Beer-Hofmann, Richard 1866-07-11 – 1945-09-26@\textsc{Beer-Hofmann, Richard} (1866-07-11 – 1945-09-26), \emph{Schriftsteller}|pw}}? Schreibt natürlich keine Zeile? Aber gedenkt {\pb}er wenigſtens ſeines Verſprechens nach \textsc{Paris\oindex{Paris@\textbf{Paris}|pw}} zu kommen?\pend
           \pstart
           \textsc{Bahr\pwindex{Bahr, Hermann 19.07.1863 – 15.01.1934@\textsc{Bahr, Hermann} (19.07.1863 – 15.01.1934), \emph{Schriftsteller, Kritiker}|pw}} haſſe ich mehr und mehr. Welch’ ein Schwindler\pwindex{Bahr, Hermann 19.07.1863 – 15.01.1934@\textsc{Bahr, Hermann} (19.07.1863 – 15.01.1934), \emph{Schriftsteller, Kritiker}|pwv}! Welch’ ein \textsc{Charlatan\pwindex{Bahr, Hermann 19.07.1863 – 15.01.1934@\textsc{Bahr, Hermann} (19.07.1863 – 15.01.1934), \emph{Schriftsteller, Kritiker}|pwv}}! Ein Mann\pwindex{Bahr, Hermann 19.07.1863 – 15.01.1934@\textsc{Bahr, Hermann} (19.07.1863 – 15.01.1934), \emph{Schriftsteller, Kritiker}|pwv}, der nach
               Geſetzen und Strömungen geht in der Literatur, – der dem Publikum einreden will, man
               könne ſo eine Art exakte Literatur-Forſchung treiben, während es doch da nur
               Individualitäten gibt, alſo Zufälliges, Unberechenbares, Geheimnißvolles. Und gerade
               die ſieht er und verſteht er nicht, der Urtheilsloſe\pwindex{Bahr, Hermann 19.07.1863 – 15.01.1934@\textsc{Bahr, Hermann} (19.07.1863 – 15.01.1934), \emph{Schriftsteller, Kritiker}|pwv}. Nicht einen Neuen hat er in der »Zeit\pwindex{Zeit. Wiener Wochenschrift1894 – 1904@\emph{Die Zeit. Wiener Wochenschrift} {[}1894 – 1904{]}|pw}« heraufgebracht, {\pb}und ich bin überzeugt, es gäbe Manchen in Wien\oindex{Wien@\textbf{Wien}|pw} zu finden. Aber immer nur \textsc{Bahr\pwindex{Bahr, Hermann 19.07.1863 – 15.01.1934@\textsc{Bahr, Hermann} (19.07.1863 – 15.01.1934), \emph{Schriftsteller, Kritiker}|pw}} – \textsc{Bahr\pwindex{Bahr, Hermann 19.07.1863 – 15.01.1934@\textsc{Bahr, Hermann} (19.07.1863 – 15.01.1934), \emph{Schriftsteller, Kritiker}|pw}} über Theater und \textsc{Bahr\pwindex{Bahr, Hermann 19.07.1863 – 15.01.1934@\textsc{Bahr, Hermann} (19.07.1863 – 15.01.1934), \emph{Schriftsteller, Kritiker}|pw}} über Kunſt – \textsc{Bahr\pwindex{Bahr, Hermann 19.07.1863 – 15.01.1934@\textsc{Bahr, Hermann} (19.07.1863 – 15.01.1934), \emph{Schriftsteller, Kritiker}|pw}} über \label{K_L02729-7v}\edtext{\textsc{Emerson}\pwindex{Emerson, Ralph Waldo 25.05.1803 – 27.04.1882@\textsc{Emerson, Ralph Waldo} (25.05.1803 – 27.04.1882), \emph{Schriftsteller}|pw}\pwindex{Bahr, Hermann 19.07.1863 – 15.01.1934@\textsc{Bahr, Hermann} (19.07.1863 – 15.01.1934), \emph{Schriftsteller, Kritiker}!Emerson1894-12-29@\strich\emph{Emerson} {[}1894-12-29{]}|pw}}{\lemma{\textnormal{\emph{Emerson}}}\Cendnote{\textnormal{Hermann Bahr\pwindex{Bahr, Hermann 19.07.1863 – 15.01.1934@\textsc{Bahr, Hermann} (19.07.1863 – 15.01.1934), \emph{Schriftsteller, Kritiker}|pwk}: \emph{Emerson}\pwindex{Bahr, Hermann 19.07.1863 – 15.01.1934@\textsc{Bahr, Hermann} (19.07.1863 – 15.01.1934), \emph{Schriftsteller, Kritiker}!Emerson1894-12-29@\strich\emph{Emerson} {[}1894-12-29{]}|pwk}. In: \emph{Die
                        Zeit}\pwindex{Zeit. Wiener Wochenschrift1894 – 1904@\emph{Die Zeit. Wiener Wochenschrift} {[}1894 – 1904{]}|pwk}, Bd. 1, H. 13, 29. 12. 1894, S. 199.}}}\label{K_L02729-7h}
               und \textsc{Bahr\pwindex{Bahr, Hermann 19.07.1863 – 15.01.1934@\textsc{Bahr, Hermann} (19.07.1863 – 15.01.1934), \emph{Schriftsteller, Kritiker}|pw}}{ }\label{K_L02729-8v}\edtext{über \textsc{Goethe\pwindex{Goethe, Johann Wolfgang von 1749-08-28 – 1832-03-22@\textsc{Goethe, Johann Wolfgang von} (1749-08-28 – 1832-03-22), \emph{Schriftsteller}|pw}}}{\lemma{\textnormal{\emph{über Goethe}}}\Cendnote{\textnormal{Die Stelle bezieht sich nicht auf einen
                  spezifischen Text, sondern die regelmäßige Erwähnung Goethes\pwindex{Goethe, Johann Wolfgang von 1749-08-28 – 1832-03-22@\textsc{Goethe, Johann Wolfgang von} (1749-08-28 – 1832-03-22), \emph{Schriftsteller}|pwk} in Bahrs\pwindex{Bahr, Hermann 19.07.1863 – 15.01.1934@\textsc{Bahr, Hermann} (19.07.1863 – 15.01.1934), \emph{Schriftsteller, Kritiker}|pwk}
                  Texten.}}}\label{K_L02729-8h}. Und immer »modern«! Jetzt hat er heraus, daß das Alte modern iſt.
               Darum muß man alſo jetzt ſich mit dem Alten beſchäftigen. Alles nach Außen und nichts
               von Innen. Der Pinſel\pwindex{Bahr, Hermann 19.07.1863 – 15.01.1934@\textsc{Bahr, Hermann} (19.07.1863 – 15.01.1934), \emph{Schriftsteller, Kritiker}|pwv}!\pend
           \pstart
           \textsc{\label{K_L02729-9v}\edtext{Kanner\pwindex{Kanner, Heinrich 09.11.1864 – 15.02.1930@\textsc{Kanner, Heinrich} (09.11.1864 – 15.02.1930), \emph{Herausgeber, Publizist}|pw}}{\lemma{\textnormal{\emph{Kanner}}}\Cendnote{\textnormal{Im \emph{Tagebuch}\pwindex{Schnitzler, Arthur 15.05.1862 – 21.10.1931@\textsc{Schnitzler, Arthur} (15.05.1862 – 21.10.1931), \emph{Schriftsteller, Mediziner}!Tagebuch1981 – 2000@\strich\emph{Tagebuch} {[}1981 – 2000{]}|pwk} von Schnitzler\pwindex{Schnitzler, Arthur 15.05.1862 – 21.10.1931@\textsc{Schnitzler, Arthur} (15.05.1862 – 21.10.1931), \emph{Schriftsteller, Mediziner}|pwk} wird er
                     in dieser Zeit nicht erwähnt und auch sonst ist nur eine Begegnung
                     festgehalten.}}}\label{K_L02729-9h}} aber iſt herrlich in der »Zeit\pwindex{Zeit. Wiener Wochenschrift1894 – 1904@\emph{Die Zeit. Wiener Wochenschrift} {[}1894 – 1904{]}|pw}«. Feſt,
               klar und ſcharf. Ein männlicher Geiſt\pwindex{Kanner, Heinrich 09.11.1864 – 15.02.1930@\textsc{Kanner, Heinrich} (09.11.1864 – 15.02.1930), \emph{Herausgeber, Publizist}|pwv}\textcolor{gray}{!} Siehſt Du ihn manchmal? Wie ſtehſt Du mit ihm?\pend
           \pstart
           {\pb}Daß Du mich im Sommer doch treffen willſt, iſt lieb
               von Dir. Vielleicht daß ich alſo doch nach der Kur auf ein paar Tage nach \textsc{Muenchen\oindex{Muenchen@\textbf{München}|pw}} kann. Ich möchte Dich ja ſo gern ſehen und ſprechen. Nach \textsc{Paris\oindex{Paris@\textbf{Paris}|pw}} könnteſt Du nicht auf 14 Tage kommen?\pend
           \pstart
           Zeitungsartikel ſende ich Dir heut nicht. \strikeout{Ich habe} Es hat keine intereſſanten gegeben; habe auch
               wenig leſen dürfen. Intereſſiren ſie Dich überhaupt? Dann macht es mir eine Freude,
               weiterzuſammeln.\pend
           \pstart
           {\pb}Was Du über \textsc{Drumont\pwindex{Drumont, Edouard 1844-05-03 – 1917-02-05@\textsc{Drumont, Édouard} (1844-05-03 – 1917-02-05), \emph{Journalist, Rassentheoretiker}|pw}} ſchreibſt, iſt im Weſentlichen richtig. Aber ſo ganz blos literariſch iſt ſein
               dämoniſcher Juden-Typus doch nicht. In \label{K_L02729-10v}\edtext{\textsc{Cornelius Herz\pwindex{Herz, Cornelius 1845-09-03 – 1898-06-06@\textsc{Herz, Cornelius} (1845-09-03 – 1898-06-06), \emph{Politiker, Industrieller, Unternehmer}|pw}}}{\lemma{\textnormal{\emph{Cornelius Herz}}}\Cendnote{\textnormal{Édouard Drumont\pwindex{Drumont, Edouard 1844-05-03 – 1917-02-05@\textsc{Drumont, Édouard} (1844-05-03 – 1917-02-05), \emph{Journalist, Rassentheoretiker}|pwk} war ein fran\oindex{Frankreich@\textbf{Frankreich}|pwkv}zösischer Antisemit\pwindex{Drumont, Edouard 1844-05-03 – 1917-02-05@\textsc{Drumont, Édouard} (1844-05-03 – 1917-02-05), \emph{Journalist, Rassentheoretiker}|pwkv}, der die Idee einer
                  entarteten, degenerierten jüdischen ›Rasse‹ propagierte. Er übte unter anderem im
                  Rahmen des Panama\oindex{Panama@\textbf{Panama}|pwk}-Skandals, in den auch Cornelius Herz\pwindex{Herz, Cornelius 1845-09-03 – 1898-06-06@\textsc{Herz, Cornelius} (1845-09-03 – 1898-06-06), \emph{Politiker, Industrieller, Unternehmer}|pwk} verwickelt war,
                  antisemitische Korruptionskritik.}}}\label{K_L02729-10h} iſt er zum Theil wahr geworden. Gewiß \textsc{Drumont\pwindex{Drumont, Edouard 1844-05-03 – 1917-02-05@\textsc{Drumont, Édouard} (1844-05-03 – 1917-02-05), \emph{Journalist, Rassentheoretiker}|pw}} iſt ſtark \label{K_L02729-11v}\edtext{\textsc{monoman}}{\lemma{\textnormal{\emph{monoman}}}\Cendnote{\textnormal{eine Zwangsvorstellung oder fixe Idee
                  haben}}}\label{K_L02729-11h}. Aber er iſt der beſte Kenner\pwindex{Drumont, Edouard 1844-05-03 – 1917-02-05@\textsc{Drumont, Édouard} (1844-05-03 – 1917-02-05), \emph{Journalist, Rassentheoretiker}|pwv} der heutigen Pariſ\oindex{Paris@\textbf{Paris}|pw}er Corruption. Was dem Draußenſtehenden darin \strikeout{\textcolor{gray}{d}} wahnſinnig ſcheint, iſt oft blos wahr. Und in allen Pariſ\oindex{Paris@\textbf{Paris}|pw}er Corruptionen ſteckt der Jude. Es iſt ein infames
               Geſindel. In dieſem \textsc{Babylon\oindex{Paris@\textbf{Paris}|pwv}}{ }{\pb}iſt \textsc{Drumont\pwindex{Drumont, Edouard 1844-05-03 – 1917-02-05@\textsc{Drumont, Édouard} (1844-05-03 – 1917-02-05), \emph{Journalist, Rassentheoretiker}|pw}} der Mann\pwindex{Drumont, Edouard 1844-05-03 – 1917-02-05@\textsc{Drumont, Édouard} (1844-05-03 – 1917-02-05), \emph{Journalist, Rassentheoretiker}|pwv}, der das
               flammende \label{K_L02729-12v}\edtext{\textsc{Mene Tekel}}{\lemma{\textnormal{\emph{Mene Tekel}}}\Cendnote{\textnormal{Warnung}}}\label{K_L02729-12h} ſchreibt. Als \strikeout{Cor}{ }Corruptions-Epiker\pwindex{Drumont, Edouard 1844-05-03 – 1917-02-05@\textsc{Drumont, Édouard} (1844-05-03 – 1917-02-05), \emph{Journalist, Rassentheoretiker}|pwv} muß man
               ihn ernſt nehmen; ſonſt iſt er eitel und verrückt.\pend
           \pstart
           Ich ſende Dir »\textsc{Les Phonographies de\strikeout{’} l’Amour\pwindex{Phonographie de l amour1895@\emph{Phonographie de l’amour} {[}1895{]}|pw}}«. Eine amüſante kleine Unanſtändigkeit.\pend
           \pstart
           Bekommſt Du noch das »\textsc{Journal\pwindex{?? Werk@Nicht ermittelte Verfasserinnen und Verfasser!Journal des debats. Politiques et litteraires1789 – 1944@\emph{Journal des débats. Politiques et littéraires} {[}1789 – 1944{]}|pwv}}«? Möchteſt Du ein anderes Blatt? Bekommt Ihr den »\textsc{Courrier \strikeout{de}
                     Français\pwindex{?? Werk@Nicht ermittelte Verfasserinnen und Verfasser!Le Courrier français1884 – 1909@\emph{Le Courrier français} {[}1884 – 1909{]}|pw}}«? Kann ich Dir ſonſt etwas in \textsc{Paris\oindex{Paris@\textbf{Paris}|pw}} beſorgen?\pend
           \pstart
           {\pb}Denk’ Dir: Deinem Bruder\pwindex{Schnitzler, Julius 13.07.1865 – 29.06.1939@\textsc{Schnitzler, Julius} (13.07.1865 – 29.06.1939), \emph{Mediziner}|pwv} und Schwägerin\pwindex{Schnitzler, Helene 16.07.1871 – September 1941@\textsc{Schnitzler, Helene} (16.07.1871 – September 1941)|pwv} habe ich noch nicht für das
               entzückende \label{K_L02729-13v}\edtext{Bild}{\lemma{\textnormal{\emph{Bild}}}\Cendnote{\textnormal{Siehe Paul Goldmann an Arthur Schnitzler, 5. 1. [1895].
               }}}\label{K_L02729-13h} gedankt, an dem ich täglich meine Freude habe. Sag’ ihnen, daß ich augenkrank
               war, – bitte – und daß ich ihnen nächſtens ſchreibe. Grüße ſie Beide\pwindex{Schnitzler, Julius 13.07.1865 – 29.06.1939@\textsc{Schnitzler, Julius} (13.07.1865 – 29.06.1939), \emph{Mediziner}|pwv}\pwindex{Schnitzler, Helene 16.07.1871 – September 1941@\textsc{Schnitzler, Helene} (16.07.1871 – September 1941)|pwv} recht herzlich.\pend
           \pstart
           Bitte, empfiehl’ mich Deiner Frau Mama\pwindex{Schnitzler, Louise 1840-07-08 – 1911-09-09@\textsc{Schnitzler, Louise} (1840-07-08 – 1911-09-09)|pwv}.\pend
           \pstart
           Sei herzlichſt und in Treue begrüßt! Nun höre ich hoffentlich bald von Dir. Aber
               antworte einmal auf alle Fragen (ausnahmsweiſe!) Dein\pend
           \pstart \spacefill\mbox{Paul Goldmann}\pend{}
         
         \endnumbering\mylabel{h}\end{ledgroupsized}  \newcommand{\dateiname}{L02729}\newcommand{\titel}{Paul Goldmann an Arthur Schnitzler, 2. 3. [1895]}\newcommand{\editorInnen}{Martin Anton Müller und Laura Untner}%% latex-leseansicht-abspann.tex
%% Abspann für die Leseansicht.
%% Der Schalter \ifkorrekturansicht ist bereits durch den Vorspann gesetzt.

%% latex-abspann.tex
%% Gemeinsamer Abspann für Korrekturansicht und Leseansicht.
%% Setzt den Schalter \ifkorrekturansicht voraus (gesetzt in den
%% einbindenden Dateien latex-korrekturansicht-abspann.tex bzw.
%% latex-leseansicht-abspann.tex).
%% ---------------------------------------------------------------

\normalsize

% Das esempio-Environment wird nur in der Leseansicht benötigt
\ifkorrekturansicht\else
\newenvironment{esempio}[3]%
{
    \vspace{1.5ex}
    \rlap{\underline{#1}}
    \par
    \setlength{\parindent}{0cm}
    \nopagebreak
    \leftskip=#2cm
    \rightskip=#3cm
}
{
    \par
}
\fi

\doendnotes{C}
\bigskip
\vfill

\clearpage

\footnotesize

\ifkorrekturansicht
  \lohead{\textsc{register}}
\fi

% theindex-Environment neu definieren ohne reledmac
\makeatletter
\renewenvironment{theindex}{%
  \ifkorrekturansicht
    \section*{\indexname}%
  \else
    \subsubsection*{Index der erwähnten Entitäten}%
  \fi
  \setlength{\parindent}{0pt}%
  \setlength{\parskip}{0pt plus 0.3pt}%
  \let\item\@idxitem
}{%
  \ifkorrekturansicht\clearpage\fi
}
\makeatother

\IfFileExists{\jobname-pw.ind}{\input{\jobname-pw.ind}}{}

% Quellenangabe nur in der Leseansicht
\ifkorrekturansicht\else
% Fallback-Definitionen, falls die .tex-Datei \titel etc. nicht gesetzt hat
\providecommand{\titel}{}
\providecommand{\editorInnen}{}
\providecommand{\dateiname}{\jobname}

\vspace{3cm}

\vfill

\footnotesize
\textsc{Quelle}: \titel. Herausgegeben von {\editorInnen}. In: \emph{Arthur Schnitzler: Briefwechsel mit Autorinnen und Autoren}.
 Digitale Edition, https://schnitzler-briefe.acdh.oeaw.ac.at/{\dateiname}.html (Stand \today)
\fi

\end{document}


      