%% latex-leseansicht-vorspann.tex
%% Vorspann für die Leseansicht.
%% Lädt die gemeinsame Datei latex-vorspann.tex mit nicht gesetztem Schalter.

\newif\ifkorrekturansicht
\korrekturansichtfalse

\input{../tex-inputs/latex-vorspann}


\section[Arthur Schnitzler an Richard Beer-Hofmann, 5. 6. 1897]{L00682 Arthur Schnitzler an Richard Beer-Hofmann, 5. 6. 1897}
\nopagebreak\mylabel{L00682v}
\rehead{ }\normalsize\beginnumbering\briefempfaengerindex{Beer-Hofmann, Richard@\textsc{Beer-Hofmann, Richard}!zzzSchnitzler, Arthur@\emph{von Arthur Schnitzler}!1897-06-051@{5. 6. 1897}|(be}
\toendnotes[C]{\smallbreak\pagebreak[2]}
\correspDesc{Versand  durch Arthur Schnitzler am 5. 6. 1897 in Wien
\newline{}Erhalt  durch Richard Beer-Hofmann am 6. 6. 1897 in Bad Ischl}\toendnotes[C]{\smallbreak}
\Standort{YCGL, MSS 31.}
\physDesc{Brief, 1 Blatt, 4 Seiten, Kuvert, 1404 Zeichen
\newline{}Handschrift: schwarze Tinte, deutsche Kurrent
\newline{}Versand: 1) Stempel: »\nobreak{}\oindex{I., Innere Stadt@\textbf{I., Innere Stadt}, \emph{Verwaltungsgebiet}|pwk}Wien 1/1, 5. 6. 97, 8–9N\nobreak{}«.   2) Stempel: »\nobreak{}\oindex{Bad Ischl@\textbf{Bad Ischl}|pwk}Ischl, 6. 6. 97, 8–9V\nobreak{}«. }
\buchAbdrucke{\weitereDrucke{Arthur Schnitzler, Richard Beer-Hofmann: \emph{Briefwechsel 1891–1931}. Herausgegeben von Konstanze Fliedl. Wien, Zürich: \emph{Europaverlag} 1992, S. 107–108.} }\toendnotes[C]{\smallbreak}\pstart{}{\pb}Dr Arthur Schnitzler Wien IX. Frankgaſſe 1\oindex{Wien@\textbf{Wien}!IX., Alsergrund@\textbf{IX., Alsergrund}!Frankgasse 1@\textbf{Frankgasse 1}, \emph{Wohngebäude}|pw}.\pend{}{\bigskip}\pstart{}{\pb}Herrn \textsc{Dr. Richard
                     Beer-Hofmann}\pend{}\pstart{}\textsc{Ischl\oindex{Bad Ischl@\textbf{Bad Ischl}|pw}}\pend{}\pstart{}\textsc{Egelmoos 22}\oindex{Eglmoosgasse@\textbf{Eglmoosgasse}, \emph{Bezirk}|pw}.\pend{}{\bigskip}\vspace{1em}
\pstart
           \raggedleft{}{\pb}5. 6. 97{\\}Wien\oindex{Wien@\textbf{Wien}, \emph{Verwaltungsgebiet}|pw}.\pend
           \vspace{0.5em}
\pstart
           Lieber Richard, es hat mir leid gethan, Sie nicht mehr in Wien\oindex{Wien@\textbf{Wien}, \emph{Verwaltungsgebiet}|pw} zu finden. Ich bin in keiner guten Sti{\geminationm}ung, durch mein fortwährendes Ohrenklingen recht{ }ſehr
               enervirt. Trotzdem will ich zu arbeiten verſuchen. Das{ }ſcheint mir überhaupt ein
               miſerables Zeichen, daſs uns alles gleich (entſchuldg Sie das »uns«) ein Hindernis
               fürs {\pb}Schaffen (entſchuldigen Sie das »Schaffen«)
               bedeutet. – Eine Bitte an Sie. We{\geminationn} Sie dieſer Tage
               einmal gar nichts zu thun haben, keine Novelle zu{ }ſchreiben, keine Radpartie zu
               machen,{ }ſo gehen Sie zum Leopold\pwindex{Petter, Leopold 17.\,11.\,1850 Bad Ischl – 3.\,7.\,1917 ebd.@\textsc{Petter, Leopold} (17.\,11.\,1850 Bad Ischl – 3.\,7.\,1917 ebd.), \emph{Hotelier}|pw}\oindex{Hotel und Pension Rudolfshöhe (Leopold Petter)@\textbf{Hotel und Pension Rudolfshöhe (Leopold Petter)}, \emph{Hotel}|pw}. Wir brauchen vom 1. Juli an zwei Zimmer\oindex{Hotel und Pension Rudolfshöhe (Leopold Petter)@\textbf{Hotel und Pension Rudolfshöhe (Leopold Petter)}, \emph{Hotel}|pwv}. Und zwar: Mama\pwindex{Schnitzler, Louise 8.\,7.\,1840 Kőszeg – 9.\,9.\,1911 Wien@\textsc{Schnitzler, Louise} (8.\,7.\,1840 Kőszeg – 9.\,9.\,1911 Wien)|pwv} ein großes,{ }ſo gelegen, wie das, was{ }ſie in frühern
               Jahren hatte, mit einem Bett, in das\footnote{\noindent{}(nicht ins Bett)} man aber noch ein zweites Bett hinein{ }ſtellen kann. Ich ein kleineres Zimmer,
                  \uline{nur nicht sonnig}!, Blick auf den Wald oder Wieſen,
               im{ }ſelben Gebäude wie Mama\pwindex{Schnitzler, Louise 8.\,7.\,1840 Kőszeg – 9.\,9.\,1911 Wien@\textsc{Schnitzler, Louise} (8.\,7.\,1840 Kőszeg – 9.\,9.\,1911 Wien)|pwv}.
               Event. gleiches Stockwerk, aber \uline{ja nicht nebenan}!
               Lieber ein anderes Stockwerk eigentlich. Nur keines von den ekelhaften weißen
               Gſchnaszimmern zu 10 fl., die Herr Leopold\pwindex{Petter, Leopold 17.\,11.\,1850 Bad Ischl – 3.\,7.\,1917 ebd.@\textsc{Petter, Leopold} (17.\,11.\,1850 Bad Ischl – 3.\,7.\,1917 ebd.), \emph{Hotelier}|pw} vor
                  {\pb}zwei Jahren erfunden hat. – (Vielleicht auch ko{\geminationm} ich schon \uline{vor} dem
                  1. Juli.) –\pend
           
\pstart
           Wie gehts Paula\pwindex{Beer-Hofmann, Paula 25.\,2.\,1879 Wien – 30.\,10.\,1939 Zürich@\textsc{Beer-Hofmann, Paula} (25.\,2.\,1879 Wien – 30.\,10.\,1939 Zürich)|pw}? Grüßen Sie{ }ſie von mir.\pend
           
\pstart
           Schreiben Sie mir auch, was Sie machen. Wie behagt Ihnen das \textsc{Bicycle}?–\pend
           
\pstart
           Von G. Hirſchf.s\pwindex{Hirschfeld, Georg 11.\,2.\,1873 Berlin – 17.\,1.\,1942 München@\textsc{Hirschfeld, Georg} (11.\,2.\,1873 Berlin – 17.\,1.\,1942 München), \emph{Schriftsteller}|pw}{ }Stück\pwindex{Hirschfeld, Georg 11.\,2.\,1873 Berlin – 17.\,1.\,1942 München@\textsc{Hirschfeld, Georg} (11.\,2.\,1873 Berlin – 17.\,1.\,1942 München), \emph{Schriftsteller}!Agnes Jordan. Schauspiel in fünf Akten@\strich\emph{Agnes Jordan. Schauspiel in fünf Akten}|pwv} höre ich ja ausnehmend{ }ſchönes. –\pend
           
\pstart
           Hoffentlich ist Ihnen die Commiſſion nicht unangenehm.\pend
           \pstart Herzlichſt Ihr \spacefill\mbox{Arthur.}\pend{}\selectlanguage{ngerman}\endnumbering\briefempfaengerindex{Beer-Hofmann, Richard@\textsc{Beer-Hofmann, Richard}!zzzSchnitzler, Arthur@\emph{von Arthur Schnitzler}!1897-06-051@{5. 6. 1897}|)be}\mylabel{L00682h}  \newcommand{\dateiname}{L00682}\newcommand{\titel}{Arthur Schnitzler an Richard Beer-Hofmann, 5. 6. 1897}\newcommand{\editorInnen}{Martin Anton Müller und Gerd-Hermann Susen}%% latex-leseansicht-abspann.tex
%% Abspann für die Leseansicht.
%% Der Schalter \ifkorrekturansicht ist bereits durch den Vorspann gesetzt.

%% latex-abspann.tex
%% Gemeinsamer Abspann für Korrekturansicht und Leseansicht.
%% Setzt den Schalter \ifkorrekturansicht voraus (gesetzt in den
%% einbindenden Dateien latex-korrekturansicht-abspann.tex bzw.
%% latex-leseansicht-abspann.tex).
%% ---------------------------------------------------------------

\normalsize

% Das esempio-Environment wird nur in der Leseansicht benötigt
\ifkorrekturansicht\else
\newenvironment{esempio}[3]%
{
    \vspace{1.5ex}
    \rlap{\underline{#1}}
    \par
    \setlength{\parindent}{0cm}
    \nopagebreak
    \leftskip=#2cm
    \rightskip=#3cm
}
{
    \par
}
\fi

\doendnotes{C}
\bigskip
\vfill

\clearpage

\footnotesize

\ifkorrekturansicht
  \lohead{\textsc{register}}
\fi

% theindex-Environment neu definieren ohne reledmac
\makeatletter
\renewenvironment{theindex}{%
  \ifkorrekturansicht
    \section*{\indexname}%
  \else
    \subsubsection*{Index der erwähnten Entitäten}%
  \fi
  \setlength{\parindent}{0pt}%
  \setlength{\parskip}{0pt plus 0.3pt}%
  \let\item\@idxitem
}{%
  \ifkorrekturansicht\clearpage\fi
}
\makeatother

\IfFileExists{\jobname-pw.ind}{\input{\jobname-pw.ind}}{}

% Quellenangabe nur in der Leseansicht
\ifkorrekturansicht\else
% Fallback-Definitionen, falls die .tex-Datei \titel etc. nicht gesetzt hat
\providecommand{\titel}{}
\providecommand{\editorInnen}{}
\providecommand{\dateiname}{\jobname}

\vspace{3cm}

\vfill

\footnotesize
\textsc{Quelle}: \titel. Herausgegeben von {\editorInnen}. In: \emph{Arthur Schnitzler: Briefwechsel mit Autorinnen und Autoren}.
 Digitale Edition, https://schnitzler-briefe.acdh.oeaw.ac.at/{\dateiname}.html (Stand \today)
\fi

\end{document}


