%% latex-leseansicht-vorspann.tex
%% Vorspann für die Leseansicht.
%% Lädt die gemeinsame Datei latex-vorspann.tex mit nicht gesetztem Schalter.

\newif\ifkorrekturansicht
\korrekturansichtfalse

\input{../tex-inputs/latex-vorspann}


               \section[Arthur Schnitzler an Richard Beer-Hofmann, 5. 6. 1897]{ Arthur Schnitzler an Richard Beer-Hofmann, 5. 6. 1897}\nopagebreak\mylabel{v}\rehead{ }\begin{ledgroupsized}[t]{13cm}\normalsize\beginnumbering\briefempfaengerindex{Beer-Hofmann, Richard@\textsc{Beer-Hofmann, Richard}!zzzSchnitzler, Arthur@\emph{von Arthur Schnitzler}!1897-06-051@{5. 6. 1897}|(be} \toendnotes[C]{\smallbreak\pagebreak[2]} \Standort{YCGL, MSS 31.}
\physDesc{Brief, 1 Blatt (Briefpapier mit Trauerrand), 4 Seiten, Umschlag
\newline{}Handschrift: schwarze Tinte, deutsche Kurrent\newline{}Versand: 1) Stempel: »\nobreak{}\oindex{I., Innere Stadt@\textbf{I., Innere Stadt}|pwk}Wien 1/1, 5. 6. 97, 8–9N\nobreak{}«.  2) Stempel: »\nobreak{}\oindex{Bad Ischl@\textbf{Bad Ischl}|pwk}Ischl, 6. 6. 97, 8–9V\nobreak{}«. }\buchAbdrucke{\weitereDrucke{Arthur Schnitzler, Richard Beer-Hofmann: \emph{Briefwechsel 1891–1931}. Hg. Konstanze Fliedl. Wien, Zürich: \emph{Europaverlag} 1992, S. 107–108.} }\toendnotes[C]{\smallbreak}\pstart{}{\pb}Dr Arthur Schnitzler Wien IX. Frankgaſſe 1\oindex{Frankgasse@\textbf{Frankgasse}|pw}.\pend{}{\bigskip}\pstart{}{\pb}Herrn \textsc{Dr. Richard
                     Beer-Hofmann}\pend{}\pstart{}\textsc{Ischl\oindex{Bad Ischl@\textbf{Bad Ischl}|pw}}\pend{}\pstart{}\textsc{Egelmoos 22}\oindex{Eglmoosgasse@\textbf{Eglmoosgasse}|pw}.\pend{}{\bigskip}\pstart
           \raggedleft{}{\pb}5. 6. 97{\\}Wien\oindex{Wien@\textbf{Wien}|pw}. \pend
           \pstart
           Lieber Richard, es hat mir leid gethan, Sie nicht mehr in Wien\oindex{Wien@\textbf{Wien}|pw} zu finden. Ich bin in keiner guten Sti{\geminationm}ung, durch mein fortwährendes Ohrenklingen recht ſehr
               enervirt. Trotzdem will ich zu arbeiten verſuchen. Das ſcheint mir überhaupt ein
               miſerables Zeichen, daſs uns alles gleich (entſchuldg Sie das »uns«) ein Hindernis
               fürs {\pb}Schaffen (entſchuldigen Sie das »Schaffen«)
               bedeutet. – Eine Bitte an Sie. We{\geminationn} Sie dieſer Tage
               einmal gar nichts zu thun haben, keine Novelle zu ſchreiben, keine Radpartie zu
               machen, ſo gehen Sie zum Leopold\pwindex{Petter, Leopold 17.11.1850 – 03.07.1917@\textsc{Petter, Leopold} (17.11.1850 – 03.07.1917), \emph{Hotelier}|pw}\oindex{Hotel und Pension Rudolfshoehe (Leopold Petter)@\textbf{Hotel und Pension Rudolfshöhe (Leopold Petter)}|pw}. Wir brauchen vom 1. Juli an zwei Zimmer\oindex{Hotel und Pension Rudolfshoehe (Leopold Petter)@\textbf{Hotel und Pension Rudolfshöhe (Leopold Petter)}|pwv}. Und zwar: Mama\pwindex{Schnitzler, Louise 08.07.1840 – 09.09.1911@\textsc{Schnitzler, Louise} (08.07.1840 – 09.09.1911)|pwv} ein großes, ſo gelegen, wie das, was ſie in frühern
               Jahren hatte, mit einem Bett, in das\footnote{\noindent{}(nicht ins Bett)} man aber noch ein zweites Bett hinein ſtellen kann. Ich ein kleineres
               Zimmer, \uline{nur nicht sonnig}!, Blick auf den Wald oder
               Wieſen, im ſelben Gebäude wie Mama\pwindex{Schnitzler, Louise 08.07.1840 – 09.09.1911@\textsc{Schnitzler, Louise} (08.07.1840 – 09.09.1911)|pwv}. Event. gleiches Stockwerk, aber \uline{ja nicht
                  nebenan}! Lieber ein anderes Stockwerk eigentlich. Nur keines von den
               ekelhaften weißen Gſchnaszimmern zu 10 fl., die Herr Leopold\pwindex{Petter, Leopold 17.11.1850 – 03.07.1917@\textsc{Petter, Leopold} (17.11.1850 – 03.07.1917), \emph{Hotelier}|pw} vor {\pb}zwei Jahren erfunden hat. –
               (Vielleicht auch ko{\geminationm} ich schon \uline{vor} dem 1. Juli.) –\pend
           \pstart
           Wie gehts Paula\pwindex{Beer-Hofmann, Paula 25.02.1879 – 30.10.1939@\textsc{Beer-Hofmann, Paula} (25.02.1879 – 30.10.1939)|pw}? Grüßen Sie ſie von mir.\pend
           \pstart
           Schreiben Sie mir auch, was Sie machen. Wie behagt Ihnen das \textsc{Bicycle}?–\pend
           \pstart
           Von G. Hirſchf.\pwindex{Hirschfeld, Georg 11.02.1873 – 17.01.1942@\textsc{Hirschfeld, Georg} (11.02.1873 – 17.01.1942), \emph{Schriftsteller}|pw}s Stück\pwindex{Hirschfeld, Georg 11.02.1873 – 17.01.1942@\textsc{Hirschfeld, Georg} (11.02.1873 – 17.01.1942), \emph{Schriftsteller}!Agnes Jordan. Schauspiel in fuenf Akten1897@\strich\emph{Agnes Jordan. Schauspiel in fünf Akten} {[}1897{]}|pwv} höre ich ja ausnehmend ſchönes. –\pend
           \pstart
           Hoffentlich ist Ihnen die Commiſſion nicht unangenehm.\pend
           \pstart Herzlichſt Ihr \spacefill\mbox{Arthur.}\pend{}\endnumbering\briefempfaengerindex{Beer-Hofmann, Richard@\textsc{Beer-Hofmann, Richard}!zzzSchnitzler, Arthur@\emph{von Arthur Schnitzler}!1897-06-051@{5. 6. 1897}|)be}\mylabel{h}\end{ledgroupsized}  \newcommand{\dateiname}{L00682}\newcommand{\titel}{Arthur Schnitzler an Richard Beer-Hofmann, 5. 6. 1897}\newcommand{\editorInnen}{Martin Anton Müller und Gerd-Hermann Susen}%% latex-leseansicht-abspann.tex
%% Abspann für die Leseansicht.
%% Der Schalter \ifkorrekturansicht ist bereits durch den Vorspann gesetzt.

%% latex-abspann.tex
%% Gemeinsamer Abspann für Korrekturansicht und Leseansicht.
%% Setzt den Schalter \ifkorrekturansicht voraus (gesetzt in den
%% einbindenden Dateien latex-korrekturansicht-abspann.tex bzw.
%% latex-leseansicht-abspann.tex).
%% ---------------------------------------------------------------

\normalsize

% Das esempio-Environment wird nur in der Leseansicht benötigt
\ifkorrekturansicht\else
\newenvironment{esempio}[3]%
{
    \vspace{1.5ex}
    \rlap{\underline{#1}}
    \par
    \setlength{\parindent}{0cm}
    \nopagebreak
    \leftskip=#2cm
    \rightskip=#3cm
}
{
    \par
}
\fi

\doendnotes{C}
\bigskip
\vfill

\clearpage

\footnotesize

\ifkorrekturansicht
  \lohead{\textsc{register}}
\fi

% theindex-Environment neu definieren ohne reledmac
\makeatletter
\renewenvironment{theindex}{%
  \ifkorrekturansicht
    \section*{\indexname}%
  \else
    \subsubsection*{Index der erwähnten Entitäten}%
  \fi
  \setlength{\parindent}{0pt}%
  \setlength{\parskip}{0pt plus 0.3pt}%
  \let\item\@idxitem
}{%
  \ifkorrekturansicht\clearpage\fi
}
\makeatother

\IfFileExists{\jobname-pw.ind}{\input{\jobname-pw.ind}}{}

% Quellenangabe nur in der Leseansicht
\ifkorrekturansicht\else
% Fallback-Definitionen, falls die .tex-Datei \titel etc. nicht gesetzt hat
\providecommand{\titel}{}
\providecommand{\editorInnen}{}
\providecommand{\dateiname}{\jobname}

\vspace{3cm}

\vfill

\footnotesize
\textsc{Quelle}: \titel. Herausgegeben von {\editorInnen}. In: \emph{Arthur Schnitzler: Briefwechsel mit Autorinnen und Autoren}.
 Digitale Edition, https://schnitzler-briefe.acdh.oeaw.ac.at/{\dateiname}.html (Stand \today)
\fi

\end{document}


      