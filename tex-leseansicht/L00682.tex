%% latex-korrekturansicht-vorspann.tex
%% Vorspann für die Korrekturansicht.
%% Lädt die gemeinsame Datei latex-vorspann.tex mit gesetztem Schalter.

\newif\ifkorrekturansicht
\korrekturansichttrue

\input{../tex-inputs/latex-vorspann}


\section[Arthur Schnitzler an Richard Beer-Hofmann, 5. 6. 1897]{L00682 Arthur Schnitzler an Richard Beer-Hofmann, 5. 6. 1897}
\nopagebreak\mylabel{L00682v}
\rehead{ }\normalsize\beginnumbering\briefempfaengerindex{Beer-Hofmann, Richard@\textsc{Beer-Hofmann, Richard}!zzzSchnitzler, Arthur@\emph{von Arthur Schnitzler}!1897-06-051@{5. 6. 1897}|(be}
\toendnotes[C]{\smallbreak\pagebreak[2]}\Standort{YCGL, MSS 31.}
\physDesc{Brief, 1 Blatt, 4 Seiten, Umschlag, 1404 Zeichen
\newline{}Handschrift: schwarze Tinte, deutsche Kurrent
\newline{}Versand: 1) Stempel: »\nobreak{}\oindex{I., Innere Stadt@\textbf{I., Innere Stadt}, \emph{A.ADM3}|pwk}Wien 1/1, 5. 6. 97, 8–9N\nobreak{}«.   2) Stempel: »\nobreak{}\oindex{Bad Ischl@\textbf{Bad Ischl}, \emph{P.PPL}|pwk}Ischl, 6. 6. 97, 8–9V\nobreak{}«. }
\buchAbdrucke{\weitereDrucke{Arthur Schnitzler, Richard Beer-Hofmann: \emph{Briefwechsel 1891–1931}. Wien, Zürich: \emph{Europaverlag} 1992, S. 107–108.} }\toendnotes[C]{\smallbreak}\pstart{}{\pb}Dr Arthur Schnitzler Wien IX. Frankgaſſe 1\oindex{Frankgasse 1@\textbf{Frankgasse 1}, \emph{Wohngebäude (K.WHS)}|pw}.\pend{}{\bigskip}\pstart{}{\pb}Herrn \textsc{Dr. Richard
                     Beer-Hofmann}\pend{}\pstart{}\textsc{Ischl\oindex{Bad Ischl@\textbf{Bad Ischl}, \emph{P.PPL}|pw}}\pend{}\pstart{}\textsc{Egelmoos 22}\oindex{Eglmoosgasse@\textbf{Eglmoosgasse}, \emph{Bezirk (A.BZK)}|pw}.\pend{}{\bigskip}\vspace{1em}
\pstart
           \raggedleft{}{\pb}5. 6. 97{\\}Wien\oindex{Wien@\textbf{Wien}, \emph{A.ADM2}|pw}. \pend
           \vspace{0.5em}
\pstart
           Lieber Richard, es hat mir leid gethan, Sie nicht mehr in Wien\oindex{Wien@\textbf{Wien}, \emph{A.ADM2}|pw} zu finden. Ich bin in keiner guten Sti{\geminationm}ung, durch mein fortwährendes Ohrenklingen recht ſehr
               enervirt. Trotzdem will ich zu arbeiten verſuchen. Das ſcheint mir überhaupt ein
               miſerables Zeichen, daſs uns alles gleich (entſchuldg Sie das »uns«) ein Hindernis
               fürs {\pb}Schaffen (entſchuldigen Sie das »Schaffen«)
               bedeutet. – Eine Bitte an Sie. We{\geminationn} Sie dieſer Tage
               einmal gar nichts zu thun haben, keine Novelle zu ſchreiben, keine Radpartie zu
               machen, ſo gehen Sie zum Leopold\pwindex{Petter, Leopold 17.11.1850 – 03.07.1917@\textsc{Petter, Leopold} (17.11.1850 – 03.07.1917), \emph{Hotelier/Hotelière}|pw}\oindex{Hotel und Pension Rudolfshoehe (Leopold Petter)@\textbf{Hotel und Pension Rudolfshöhe (Leopold Petter)}, \emph{Hotel (K.HTL)}|pw}. Wir brauchen vom 1. Juli an zwei Zimmer\oindex{Hotel und Pension Rudolfshoehe (Leopold Petter)@\textbf{Hotel und Pension Rudolfshöhe (Leopold Petter)}, \emph{Hotel (K.HTL)}|pwv}. Und zwar: Mama\pwindex{Schnitzler, Louise 1840-07-08 – 1911-09-09@\textsc{Schnitzler, Louise} (1840-07-08 – 1911-09-09)|pwv} ein großes, ſo gelegen, wie das, was ſie in frühern
               Jahren hatte, mit einem Bett, in das\noindent{}(nicht ins Bett) man aber noch ein zweites Bett hinein ſtellen kann. Ich ein kleineres Zimmer,
                  \uline{nur nicht sonnig}!, Blick auf den Wald oder Wieſen,
               im ſelben Gebäude wie Mama\pwindex{Schnitzler, Louise 1840-07-08 – 1911-09-09@\textsc{Schnitzler, Louise} (1840-07-08 – 1911-09-09)|pwv}.
               Event. gleiches Stockwerk, aber \uline{ja nicht nebenan}!
               Lieber ein anderes Stockwerk eigentlich. Nur keines von den ekelhaften weißen
               Gſchnaszimmern zu 10 fl., die Herr Leopold\pwindex{Petter, Leopold 17.11.1850 – 03.07.1917@\textsc{Petter, Leopold} (17.11.1850 – 03.07.1917), \emph{Hotelier/Hotelière}|pw} vor
                  {\pb}zwei Jahren erfunden hat. – (Vielleicht auch ko{\geminationm} ich schon \uline{vor} dem
                  1. Juli.) –\pend
           
\pstart
           Wie gehts Paula\pwindex{Beer-Hofmann, Paula 25.02.1879 – 30.10.1939@\textsc{Beer-Hofmann, Paula} (25.02.1879 – 30.10.1939)|pw}? Grüßen Sie ſie von mir.\pend
           
\pstart
           Schreiben Sie mir auch, was Sie machen. Wie behagt Ihnen das \textsc{Bicycle}?–\pend
           
\pstart
           Von G. Hirſchf.s\pwindex{Hirschfeld, Georg 11.02.1873 – 17.01.1942@\textsc{Hirschfeld, Georg} (11.02.1873 – 17.01.1942), \emph{Schriftsteller/Schriftstellerin}|pw}{ }Stück\pwindex{Agnes Jordan. Schauspiel in fuenf Akten@\emph{Agnes Jordan. Schauspiel in fünf Akten}|pwv} höre ich ja ausnehmend ſchönes. –\pend
           
\pstart
           Hoffentlich ist Ihnen die Commiſſion nicht unangenehm.\pend
           \pstart Herzlichſt Ihr \spacefill\mbox{Arthur.}\pend{}\selectlanguage{ngerman}\endnumbering\briefempfaengerindex{Beer-Hofmann, Richard@\textsc{Beer-Hofmann, Richard}!zzzSchnitzler, Arthur@\emph{von Arthur Schnitzler}!1897-06-051@{5. 6. 1897}|)be}\mylabel{L00682h}  \normalsize

\doendnotes{C}
\bigskip
\vfill

\clearpage

\footnotesize

\lohead{\textsc{register}}

% Definiere theindex-Environment komplett neu ohne reledmac
\makeatletter
\renewenvironment{theindex}{%
  \section*{\indexname}%
  \setlength{\parindent}{0pt}%
  \setlength{\parskip}{0pt plus 0.3pt}%
  \let\item\@idxitem
}{%
  \clearpage
}
\makeatother

\IfFileExists{\jobname-pw.ind}{\input{\jobname-pw.ind}}{}

\end{document}

      