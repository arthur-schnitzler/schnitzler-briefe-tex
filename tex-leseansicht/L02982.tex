%% latex-leseansicht-vorspann.tex
%% Vorspann für die Leseansicht.
%% Lädt die gemeinsame Datei latex-vorspann.tex mit nicht gesetztem Schalter.

\newif\ifkorrekturansicht
\korrekturansichtfalse

\input{../tex-inputs/latex-vorspann}


\section[ Arthur Schnitzler an Felix Salten, 28. [9.] 1903]{L02982 Arthur Schnitzler an Felix Salten,  28. [9.] 1903}
\nopagebreak\mylabel{L02982v}
\rehead{ }\normalsize\beginnumbering\briefempfaengerindex{Salten, Felix@\textsc{Salten, Felix}!zzzSchnitzler, Arthur@\emph{von Arthur Schnitzler}!1903-09-281@{28. [9.] 1903}|(be}
\toendnotes[C]{\smallbreak\pagebreak[2]}
\correspDesc{Versand  durch Arthur Schnitzler am 28. [9.] 1903 in Wien
\newline{}Erhalt  durch Felix Salten im Zeitraum [28. 9. 1903
                  – 1. 10. 1903?] in Wien}\toendnotes[C]{\smallbreak}
\Standort{Wienbibliothek im Rathaus, ZPH 1681, 2.1.516.}
\physDesc{Brief, 1 Blatt, 2 Seiten, 400 Zeichen
\newline{}Handschrift: Bleistift, deutsche Kurrent
\newline{}Ordnung: mit Bleistift von unbekannter Hand nummeriert: »21« }\toendnotes[C]{\smallbreak}
\pstart
           \raggedleft{}{\pb}\textsc{Wien, XVIII\oindex{XVIII., Währing@\textbf{XVIII., Währing}, \emph{Verwaltungsgebiet}|pw}}{ }\textsc{Spöttelg. 7}\oindex{Wien@\textbf{Wien}!XVIII., Währing@\textbf{XVIII., Währing}!Edmund-Weiß-Gasse 7@\textbf{Edmund-Weiß-Gasse 7}, \emph{Wohngebäude}|pw}.\pend
           
\pstart
           \raggedleft{}28. \textcolor{gray}{9}. 903\pend
           \vspace{0.5em}
\pstart
           lieber, Ihrer freundlichen \label{K_L02982-1v}\edtext{Zuſage}{\lemma{\textnormal{\emph{Zusage}}}\Cendnote{\textnormal{Siehe XXXX Auszeichnungsfehler: Dokument L03342 nicht gefunden.
               }}}\label{K_L02982-1} vertrauend hatte ich an Frau B.\pwindex{Mewes-Béha, Emilie @\textsc{Mewes-Béha, Emilie}, \emph{Schriftstellerin, Übersetzerin}|pw}
               geſchrieben dſs ihre \label{K_L02982-2v}\edtext{Skizze\pwindex{Mewes-Béha, Emilie @\textsc{Mewes-Béha, Emilie}, \emph{Schriftstellerin, Übersetzerin}!Studie@\strich\emph{Studie}|pwv}}{\lemma{\textnormal{\emph{Skizze}}}\Cendnote{\textnormal{E. Mewes-Béha\pwindex{Mewes-Béha, Emilie @\textsc{Mewes-Béha, Emilie}, \emph{Schriftstellerin, Übersetzerin}|pwk}: \emph{Studie}\pwindex{Mewes-Béha, Emilie @\textsc{Mewes-Béha, Emilie}, \emph{Schriftstellerin, Übersetzerin}!Studie@\strich\emph{Studie}|pwk}. In: \emph{Die
                        Zeit}\pwindex{Zeit@\emph{Die Zeit}|pwk}, Jg. 2, Nr. 364, 4. 10. 1903, Die
                     Sonntags-Zeit, S. 2–3.}}}\label{K_L02982-2} beſtimmt am geſtrigen So{\geminationn}tag erſcheint;\pend
           
\pstart
           bitte theilen Sie mir doch mit, ob{ }ſie im nächſten So{\geminationn}tagsheft\pwindex{Zeit@\emph{Die Zeit}|pwv}{ }ſicher gedruckt
               wird.\pend
           
\pstart
           {\pb}In Ihrem \label{K_L02982-3v}\edtext{Geburtstagsfeuilleton\pwindex{Salten, Felix 6.\,9.\,1869 Budapest – 8.\,10.\,1945 Zürich@\textsc{Salten, Felix} (6.\,9.\,1869 Budapest – 8.\,10.\,1945 Zürich), \emph{Schriftsteller, Journalist, Chefredakteur}!Unser Geburtstag@\strich\emph{Unser Geburtstag}|pwv}}{\lemma{\textnormal{\emph{Geburtstagsfeuilleton}}}\Cendnote{\textnormal{Anlässlich des einjährigen Erscheinens
                  der Tageszeitung \emph{Zeit}\pwindex{Zeit@\emph{Die Zeit}|pwk} erschien: Felix Salten\pwindex{Salten, Felix 6.\,9.\,1869 Budapest – 8.\,10.\,1945 Zürich@\textsc{Salten, Felix} (6.\,9.\,1869 Budapest – 8.\,10.\,1945 Zürich), \emph{Schriftsteller, Journalist, Chefredakteur}|pwk}: \emph{Unser Geburtstag}\pwindex{Salten, Felix 6.\,9.\,1869 Budapest – 8.\,10.\,1945 Zürich@\textsc{Salten, Felix} (6.\,9.\,1869 Budapest – 8.\,10.\,1945 Zürich), \emph{Schriftsteller, Journalist, Chefredakteur}!Unser Geburtstag@\strich\emph{Unser Geburtstag}|pwk}. In: \emph{Die Zeit}\pwindex{Zeit@\emph{Die Zeit}|pwk}, Jg. 2, Nr. 357, 27. 9. 1903,
                     S. 1–3.}}}\label{K_L02982-3}{ }ſtecken die Elemente zu einer \label{K_L02982-55v}\edtext{Tragikomödie des Journalismus}{\lemma{\textnormal{\emph{Tragikomödie des Journalismus}}}\Cendnote{\textnormal{Schnitzler selbst trug
                  sich seit mindestens 10. 8. 1901 mit dem Plan eines Theaterstückes, das im Journalismus
                  angesiedelt war. Am 25. 11. 1903 begann er eine erste Niederschrift, woraus sich \emph{Fink und Fliederbusch}\pwindex{Schnitzler, Arthur 15.\,5.\,1862 Wien – 21.\,10.\,1931 ebd.@\textsc{Schnitzler, Arthur} (15.\,5.\,1862 Wien – 21.\,10.\,1931 ebd.), \emph{Schriftsteller, Mediziner}!Fink und Fliederbusch. Komödie in drei Akten@\strich\emph{Fink und Fliederbusch. Komödie in drei Akten}|pwk} entwickelte. }}}\label{K_L02982-55}.
               Was macht übrigens Ihr 
               \label{K_L02982-66v}\edtext{Journaliſtenſtück\pwindex{Salten, Felix 6.\,9.\,1869 Budapest – 8.\,10.\,1945 Zürich@\textsc{Salten, Felix} (6.\,9.\,1869 Budapest – 8.\,10.\,1945 Zürich), \emph{Schriftsteller, Journalist, Chefredakteur}!?? [Journalistenstück]@\strich\emph{?? [Journalistenstück]}|pwv}}{\lemma{\textnormal{\emph{Journalistenstück}}}\Cendnote{\textnormal{Das »Journaliſtenſtück\pwindex{Salten, Felix 6.\,9.\,1869 Budapest – 8.\,10.\,1945 Zürich@\textsc{Salten, Felix} (6.\,9.\,1869 Budapest – 8.\,10.\,1945 Zürich), \emph{Schriftsteller, Journalist, Chefredakteur}!?? [Journalistenstück]@\strich\emph{?? [Journalistenstück]}|pwv}« konnte nicht identifiziert werden.}}}\label{K_L02982-66} und der \label{K_L02982-4v}\edtext{Schrei\pwindex{Salten, Felix 6.\,9.\,1869 Budapest – 8.\,10.\,1945 Zürich@\textsc{Salten, Felix} (6.\,9.\,1869 Budapest – 8.\,10.\,1945 Zürich), \emph{Schriftsteller, Journalist, Chefredakteur}!Schrei der Liebe. Novelle@\strich\emph{Der Schrei der Liebe. Novelle}|pw}}{\lemma{\textnormal{\emph{Schrei}}}\Cendnote{\textnormal{Der \emph{Schrei der Liebe}\pwindex{Salten, Felix 6.\,9.\,1869 Budapest – 8.\,10.\,1945 Zürich@\textsc{Salten, Felix} (6.\,9.\,1869 Budapest – 8.\,10.\,1945 Zürich), \emph{Schriftsteller, Journalist, Chefredakteur}!Schrei der Liebe. Novelle@\strich\emph{Der Schrei der Liebe. Novelle}|pwk} stand kurz vor Fertigstellung. Vgl. A. S.: \emph{Tagebuch}, 21. 10. 1903.}}}\label{K_L02982-4}?\pend
           
\pstart
           Herzlichſt Ihr {\\[\baselineskip]}\spacefill\mbox{A.}\pend
           \leftskip=0em{}\selectlanguage{ngerman}\endnumbering\briefempfaengerindex{Salten, Felix@\textsc{Salten, Felix}!zzzSchnitzler, Arthur@\emph{von Arthur Schnitzler}!1903-09-281@{28. [9.] 1903}|)be}\mylabel{L02982h}  \newcommand{\dateiname}{L02982}\newcommand{\titel}{Arthur Schnitzler an Felix Salten, 28. [9.] 1903}\newcommand{\editorInnen}{Martin Anton Müller und Laura Untner}%% latex-leseansicht-abspann.tex
%% Abspann für die Leseansicht.
%% Der Schalter \ifkorrekturansicht ist bereits durch den Vorspann gesetzt.

%% latex-abspann.tex
%% Gemeinsamer Abspann für Korrekturansicht und Leseansicht.
%% Setzt den Schalter \ifkorrekturansicht voraus (gesetzt in den
%% einbindenden Dateien latex-korrekturansicht-abspann.tex bzw.
%% latex-leseansicht-abspann.tex).
%% ---------------------------------------------------------------

\normalsize

% Das esempio-Environment wird nur in der Leseansicht benötigt
\ifkorrekturansicht\else
\newenvironment{esempio}[3]%
{
    \vspace{1.5ex}
    \rlap{\underline{#1}}
    \par
    \setlength{\parindent}{0cm}
    \nopagebreak
    \leftskip=#2cm
    \rightskip=#3cm
}
{
    \par
}
\fi

\doendnotes{C}
\bigskip
\vfill

\clearpage

\footnotesize

\ifkorrekturansicht
  \lohead{\textsc{register}}
\fi

% theindex-Environment neu definieren ohne reledmac
\makeatletter
\renewenvironment{theindex}{%
  \ifkorrekturansicht
    \section*{\indexname}%
  \else
    \subsubsection*{Index der erwähnten Entitäten}%
  \fi
  \setlength{\parindent}{0pt}%
  \setlength{\parskip}{0pt plus 0.3pt}%
  \let\item\@idxitem
}{%
  \ifkorrekturansicht\clearpage\fi
}
\makeatother

\IfFileExists{\jobname-pw.ind}{\input{\jobname-pw.ind}}{}

% Quellenangabe nur in der Leseansicht
\ifkorrekturansicht\else
% Fallback-Definitionen, falls die .tex-Datei \titel etc. nicht gesetzt hat
\providecommand{\titel}{}
\providecommand{\editorInnen}{}
\providecommand{\dateiname}{\jobname}

\vspace{3cm}

\vfill

\footnotesize
\textsc{Quelle}: \titel. Herausgegeben von {\editorInnen}. In: \emph{Arthur Schnitzler: Briefwechsel mit Autorinnen und Autoren}.
 Digitale Edition, https://schnitzler-briefe.acdh.oeaw.ac.at/{\dateiname}.html (Stand \today)
\fi

\end{document}


