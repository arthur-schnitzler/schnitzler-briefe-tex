%% latex-leseansicht-vorspann.tex
%% Vorspann für die Leseansicht.
%% Lädt die gemeinsame Datei latex-vorspann.tex mit nicht gesetztem Schalter.

\newif\ifkorrekturansicht
\korrekturansichtfalse

\input{../tex-inputs/latex-vorspann}

\begin{center}
            \textcolor{red}{ENTWURF, NICHT FERTIG KORRIGIERT}
                      \end{center}
            
         
         \renewcommand{\erwaehntePersonen}{Personen: Emilie Mewes-Béha, Felix Salten}
         \renewcommand{\erwaehnteOrte}{Orte: Edmund-Weiß-Gasse, Wien, XVIII., Währing}
         \renewcommand{\erwaehnteWerke}{Werke: Der Schrei der Liebe. Novelle, Die Zeit, Studie, Unser Geburtstag}
               \section[Arthur Schnitzler an Felix Salten, 28. {[}9.{]} 1903]{ Arthur Schnitzler an Felix Salten, 28. {[}9.{]} 1903}\nopagebreak\mylabel{v}\rehead{ }\begin{ledgroupsized}[t]{13cm}\normalsize\beginnumbering \toendnotes[C]{\smallbreak\pagebreak[2]} \Standort{Wienbibliothek im Rathaus, ZPH 1681, 2.1.516.}
\physDesc{Brief, 1 Blatt, 2 Seiten
\newline{}Handschrift: Bleistift, deutsche Kurrent\newline{}Ordnung: mit Bleistift von unbekannter Hand Nummerierung der ungeraden Seiten:
                                 »21« }\toendnotes[C]{\smallbreak}\pstart
           \noindent{}\raggedleft{}{\pb}\textsc{Wien, XVIII\oindex{XVIII., Waehring@\textbf{XVIII., Währing}|pw}}{ }\textsc{Spöttelg. 7}\oindex{Edmund-Weiss-Gasse@\textbf{Edmund-Weiß-Gasse}|pw}. \pend
           \pstart
           \raggedleft{}28. \textcolor{gray}{9}. 903\pend
           \pstart
           lieber, Ihrer freundlichen Zuſage vertrauend, hatte ich an Frau B.\pwindex{Mewes-Beha, Emilie @\textsc{Mewes-Béha, Emilie}, \emph{Schriftstellerin, Übersetzerin}|pw} geſchrieben daſs ihre \label{K_L02982-1v}\edtext{Skizze\pwindex{Mewes-Beha, Emilie @\textsc{Mewes-Béha, Emilie}, \emph{Schriftstellerin, Übersetzerin}!Studie1903-10-04@\strich\emph{Studie} {[}1903-10-04{]}|pwv}}{\lemma{\textnormal{\emph{Skizze}}}\Cendnote{\textnormal{E. Mewes-Béha\pwindex{Mewes-Beha, Emilie @\textsc{Mewes-Béha, Emilie}, \emph{Schriftstellerin, Übersetzerin}|pwk}: \emph{Studie}\pwindex{Mewes-Beha, Emilie @\textsc{Mewes-Béha, Emilie}, \emph{Schriftstellerin, Übersetzerin}!Studie1903-10-04@\strich\emph{Studie} {[}1903-10-04{]}|pwk}. In: \emph{Die
                        Zeit}\pwindex{Zeit1902-09-27 – 1919@\emph{Die Zeit} {[}1902-09-27 – 1919{]}|pwk}, Jg. 2, Nr. 364, 4. 10. 1903, Die Sonntags-Zeit,
                     S. 2–3.}}}\label{K_L02982-1h} beſtimmt am geſtrigen So{\geminationn}tag
               erſcheint; bitte theilen Sie mir doch mit, ob ſie im nächſten So{\geminationn}tagsheft ſicher gedruckt wird. \pend
           \pstart
           {\pb}In Ihrem \label{K_L02982-22v}\edtext{Geburtstagsfeuilleton\pwindex{Salten, Felix 06.09.1869 – 08.10.1945@\textsc{Salten, Felix} (06.09.1869 – 08.10.1945), \emph{Schriftsteller, Journalist}!Unser Geburtstag1903-09-27@\strich\emph{Unser Geburtstag} {[}1903-09-27{]}|pwv}}{\lemma{\textnormal{\emph{Geburtstagsfeuilleton}}}\Cendnote{\textnormal{Felix Salten\pwindex{Salten, Felix 06.09.1869 – 08.10.1945@\textsc{Salten, Felix} (06.09.1869 – 08.10.1945), \emph{Schriftsteller, Journalist}|pwk}: \emph{Unser Geburtstag}\pwindex{Salten, Felix 06.09.1869 – 08.10.1945@\textsc{Salten, Felix} (06.09.1869 – 08.10.1945), \emph{Schriftsteller, Journalist}!Unser Geburtstag1903-09-27@\strich\emph{Unser Geburtstag} {[}1903-09-27{]}|pwk}. In: \emph{Die Zeit}\pwindex{Zeit1902-09-27 – 1919@\emph{Die Zeit} {[}1902-09-27 – 1919{]}|pwk}, Jg. 2, Nr. 357, 27. 9. 1903,
                  S. 1–3.}}}\label{K_L02982-22h} ſtecken die Elemente zu einer Tragikomödie des Journalismus.
               Was macht übrigens Ihr Journaliſtenſtück\textcolor{red}{\textsuperscript{\textbf{KEY}}} und der Schrei\pwindex{Salten, Felix 06.09.1869 – 08.10.1945@\textsc{Salten, Felix} (06.09.1869 – 08.10.1945), \emph{Schriftsteller, Journalist}!Schrei der Liebe. Novelle1904-10-22@\strich\emph{Der Schrei der Liebe. Novelle} {[}1904-10-22{]}|pw}? \pend
           \pstart
           Herzlichſt Ihr {\\[\baselineskip]}\spacefill\mbox{A.}\pend
           \leftskip=0em{}
         
         \endnumbering\mylabel{h}\end{ledgroupsized}\begin{anhang}\end{anhang}\newcommand{\dateiname}{L02982}\newcommand{\titel}{Arthur Schnitzler an Felix Salten, 28. [9.] 1903}\newcommand{\editorInnen}{Martin Anton Müller und Laura Untner}%% latex-leseansicht-abspann.tex
%% Abspann für die Leseansicht.
%% Der Schalter \ifkorrekturansicht ist bereits durch den Vorspann gesetzt.

%% latex-abspann.tex
%% Gemeinsamer Abspann für Korrekturansicht und Leseansicht.
%% Setzt den Schalter \ifkorrekturansicht voraus (gesetzt in den
%% einbindenden Dateien latex-korrekturansicht-abspann.tex bzw.
%% latex-leseansicht-abspann.tex).
%% ---------------------------------------------------------------

\normalsize

% Das esempio-Environment wird nur in der Leseansicht benötigt
\ifkorrekturansicht\else
\newenvironment{esempio}[3]%
{
    \vspace{1.5ex}
    \rlap{\underline{#1}}
    \par
    \setlength{\parindent}{0cm}
    \nopagebreak
    \leftskip=#2cm
    \rightskip=#3cm
}
{
    \par
}
\fi

\doendnotes{C}
\bigskip
\vfill

\clearpage

\footnotesize

\ifkorrekturansicht
  \lohead{\textsc{register}}
\fi

% theindex-Environment neu definieren ohne reledmac
\makeatletter
\renewenvironment{theindex}{%
  \ifkorrekturansicht
    \section*{\indexname}%
  \else
    \subsubsection*{Index der erwähnten Entitäten}%
  \fi
  \setlength{\parindent}{0pt}%
  \setlength{\parskip}{0pt plus 0.3pt}%
  \let\item\@idxitem
}{%
  \ifkorrekturansicht\clearpage\fi
}
\makeatother

\IfFileExists{\jobname-pw.ind}{\input{\jobname-pw.ind}}{}

% Quellenangabe nur in der Leseansicht
\ifkorrekturansicht\else
% Fallback-Definitionen, falls die .tex-Datei \titel etc. nicht gesetzt hat
\providecommand{\titel}{}
\providecommand{\editorInnen}{}
\providecommand{\dateiname}{\jobname}

\vspace{3cm}

\vfill

\footnotesize
\textsc{Quelle}: \titel. Herausgegeben von {\editorInnen}. In: \emph{Arthur Schnitzler: Briefwechsel mit Autorinnen und Autoren}.
 Digitale Edition, https://schnitzler-briefe.acdh.oeaw.ac.at/{\dateiname}.html (Stand \today)
\fi

\end{document}


      