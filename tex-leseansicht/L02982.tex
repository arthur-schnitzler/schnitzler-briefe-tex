%% latex-korrekturansicht-vorspann.tex
%% Vorspann für die Korrekturansicht.
%% Lädt die gemeinsame Datei latex-vorspann.tex mit gesetztem Schalter.

\newif\ifkorrekturansicht
\korrekturansichttrue

\input{../tex-inputs/latex-vorspann}


\section[ Arthur Schnitzler an Felix Salten, 28. {[}9.{]} 1903]{L02982 Arthur Schnitzler an Felix Salten, 28. {[}9.{]} 1903}
\nopagebreak\mylabel{L02982v}
\rehead{ }\normalsize\beginnumbering\briefempfaengerindex{Salten, Felix@\textsc{Salten, Felix}!zzzSchnitzler, Arthur@\emph{von Arthur Schnitzler}!1903-09-281@{28. {[}9.{]} 1903}|(be}
\toendnotes[C]{\smallbreak\pagebreak[2]}\Standort{Wienbibliothek im Rathaus, ZPH 1681, 2.1.516.}
\physDesc{Brief, 1 Blatt, 2 Seiten, 400 Zeichen
\newline{}Handschrift: Bleistift, deutsche Kurrent
\newline{}Ordnung: mit Bleistift von unbekannter Hand nummeriert: »21« }\toendnotes[C]{\smallbreak}
\pstart
           \raggedleft{}{\pb}\textsc{Wien, XVIII\oindex{XVIII., Waehring@\textbf{XVIII., Währing}, \emph{A.ADM3}|pw}}{ }\textsc{Spöttelg. 7}\oindex{Edmund-Weiss-Gasse 7@\textbf{Edmund-Weiß-Gasse 7}, \emph{Wohngebäude (K.WHS)}|pw}. \pend
           
\pstart
           \raggedleft{}28. \textcolor{gray}{9}. 903\pend
           \vspace{0.5em}
\pstart
           lieber, Ihrer freundlichen \label{K_L02982-1v}\edtext{Zuſage}{\lemma{\textnormal{\emph{Zuſage}}}\Cendnote{\textnormal{Siehe Felix Salten an Arthur Schnitzler, 11. 8. 1903.
               }}}\label{K_L02982-1} vertrauend hatte ich an Frau B.\pwindex{Mewes-Beha, Emilie @\textsc{Mewes-Béha, Emilie}, \emph{Schriftsteller/Schriftstellerin, Übersetzer/Übersetzerin}|pw}
               geſchrieben dſs ihre \label{K_L02982-2v}\edtext{Skizze\pwindex{Studie@\emph{Studie}|pwv}}{\lemma{\textnormal{\emph{Skizze}}}\Cendnote{\textnormal{E. Mewes-Béha\pwindex{Mewes-Beha, Emilie @\textsc{Mewes-Béha, Emilie}, \emph{Schriftsteller/Schriftstellerin, Übersetzer/Übersetzerin}|pwk}: \emph{Studie}\pwindex{Studie@\emph{Studie}|pwk}. In: \emph{Die
                        Zeit}\pwindex{Zeit@\emph{Die Zeit}|pwk}, Jg. 2, Nr. 364, 4. 10. 1903, Die
                     Sonntags-Zeit, S. 2–3.}}}\label{K_L02982-2} beſtimmt am geſtrigen So{\geminationn}tag erſcheint;\pend
           
\pstart
           bitte theilen Sie mir doch mit, ob ſie im nächſten So{\geminationn}tagsheft\pwindex{Zeit@\emph{Die Zeit}|pwv} ſicher gedruckt
               wird.\pend
           
\pstart
           {\pb}In Ihrem \label{K_L02982-3v}\edtext{Geburtstagsfeuilleton\pwindex{Unser Geburtstag@\emph{Unser Geburtstag}|pwv}}{\lemma{\textnormal{\emph{Geburtstagsfeuilleton}}}\Cendnote{\textnormal{Anlässlich des einjährigen Erscheinens
                  der Tageszeitung \emph{Zeit}\pwindex{Zeit@\emph{Die Zeit}|pwk} erschien: Felix Salten\pwindex{Salten, Felix 06.09.1869 – 08.10.1945@\textsc{Salten, Felix} (06.09.1869 – 08.10.1945), \emph{Schriftsteller/Schriftstellerin, Journalist/Journalistin, Chefredakteur/Chefredakteurin}|pwk}: \emph{Unser Geburtstag}\pwindex{Unser Geburtstag@\emph{Unser Geburtstag}|pwk}. In: \emph{Die Zeit}\pwindex{Zeit@\emph{Die Zeit}|pwk}, Jg. 2, Nr. 357, 27. 9. 1903,
                     S. 1–3.}}}\label{K_L02982-3} ſtecken die Elemente zu einer \label{K_L02982-55v}\edtext{Tragikomödie des Journalismus}{\lemma{\textnormal{\emph{Tragikomödie des Journalismus}}}\Cendnote{\textnormal{Schnitzler selbst trug
                  sich seit mindestens 10. 8. 1901 mit dem Plan eines Theaterstückes, das im Journalismus
                  angesiedelt war. Am 25. 11. 1903 begann er eine erste Niederschrift, woraus sich \emph{Fink und Fliederbusch}\pwindex{Fink und Fliederbusch. Komoedie in drei Akten@\emph{Fink und Fliederbusch. Komödie in drei Akten}|pwk} entwickelte. }}}\label{K_L02982-55}.
               Was macht übrigens Ihr 
               \label{K_L02982-66v}\edtext{Journaliſtenſtück\pwindex{?? [Journalistenstueck]@\emph{?? [Journalistenstück]}|pwv}}{\lemma{\textnormal{\emph{Journaliſtenſtück}}}\Cendnote{\textnormal{Das »Journaliſtenſtück\pwindex{?? [Journalistenstueck]@\emph{?? [Journalistenstück]}|pwv}« konnte nicht identifiziert werden.}}}\label{K_L02982-66} und der \label{K_L02982-4v}\edtext{Schrei\pwindex{Schrei der Liebe. Novelle@\emph{Der Schrei der Liebe. Novelle}|pw}}{\lemma{\textnormal{\emph{Schrei}}}\Cendnote{\textnormal{Der \emph{Schrei der Liebe}\pwindex{Schrei der Liebe. Novelle@\emph{Der Schrei der Liebe. Novelle}|pwk} stand kurz vor Fertigstellung. Vgl. A. S.: \emph{Tagebuch}, 21. 10. 1903.}}}\label{K_L02982-4}?\pend
           
\pstart
           Herzlichſt Ihr {\\[\baselineskip]}\spacefill\mbox{A.}\pend
           \leftskip=0em{}\selectlanguage{ngerman}\endnumbering\briefempfaengerindex{Salten, Felix@\textsc{Salten, Felix}!zzzSchnitzler, Arthur@\emph{von Arthur Schnitzler}!1903-09-281@{28. {[}9.{]} 1903}|)be}\mylabel{L02982h}  \normalsize

\doendnotes{C}
\bigskip
\vfill

\clearpage

\footnotesize

\lohead{\textsc{register}}

% Definiere theindex-Environment komplett neu ohne reledmac
\makeatletter
\renewenvironment{theindex}{%
  \section*{\indexname}%
  \setlength{\parindent}{0pt}%
  \setlength{\parskip}{0pt plus 0.3pt}%
  \let\item\@idxitem
}{%
  \clearpage
}
\makeatother

\IfFileExists{\jobname-pw.ind}{\input{\jobname-pw.ind}}{}

\end{document}

      