%% latex-leseansicht-vorspann.tex
%% Vorspann für die Leseansicht.
%% Lädt die gemeinsame Datei latex-vorspann.tex mit nicht gesetztem Schalter.

\newif\ifkorrekturansicht
\korrekturansichtfalse

\input{../tex-inputs/latex-vorspann}


         
         \renewcommand{\erwaehntePersonen}{Personen:  ?? [Studienkollege von Albert Ehrenstein], Raoul Auernheimer, Zahir ad-Din Muhammad Babur, Napoleon Bonaparte, August Fournier, Eugen Oberhummer}
         \renewcommand{\erwaehnteInstitutionen}{Institutionen: Neue Freie Presse}
         \renewcommand{\erwaehnteOrte}{Orte: Edlach, Galizien, Ottakringerstraße, Polen, Ungarn, Wien}
         \renewcommand{\erwaehnteWerke}{Werke: ?? [Dissertation], Baburnama, Die Lage in Ungarn (Siebenbürgen und Serbien ausgenommen) im Jahre 1790, Helena, Napoleon I. Eine Biographie, Tod des Zehir eddin Muhammed Baber}
               \section[Albert Ehrenstein an Arthur Schnitzler, 13. 7. 1909]{ Albert Ehrenstein an Arthur Schnitzler, 13. 7. 1909}\nopagebreak\mylabel{v}\rehead{ }\begin{ledgroupsized}[t]{13cm}\normalsize\beginnumbering \toendnotes[C]{\smallbreak\pagebreak[2]} \Standort{CUL, Schnitzler, B 30.}
\physDesc{Brief, 4 Blätter, 4 Seiten, 5077 Zeichen (Paginierung)
\newline{}Handschrift: schwarze Tinte, deutsche Kurrent
\newline{}Schnitzler: mit Bleistift beschriftet: »\textsc{Ehrenstein}« }\buchAbdrucke{\weitereDrucke{Albert Ehrenstein: \emph{Briefe}. Hg. Hanni Mittelmann. München: \emph{Boer} 1989, S. 29–31 (Werke, 1).} }\toendnotes[C]{\smallbreak}\pstart
           {\pb}\textsc{Wien, XVI. Ottakringerstr.} 114\oindex{Ottakringerstrasse@\textbf{Ottakringerstraße}|pw}.\hfill 13. Juli 09.\pend
           \pstart{}\textsc{Sehr geehrter Herr Doktor!}\pend\pstart
           Ihr freundlicher Brief gab mir gerade jetzt einigen Troſt. Mein Geſchichtsprofeſſor\pwindex{Fournier, August 19.06.1850 – 18.05.1920@\textsc{Fournier, August} (19.06.1850 – 18.05.1920), \emph{Historiker}|pwv} nämlich, mit einem
               ewigen Bronchialkatarrh behaftet und daher außerordentlich ſekant, hat mir die Ehre
               erwieſen, mir meine Diſſertation\pwindex{Ehrenstein, Albert 23.12.1886 – 08.04.1950@\textsc{Ehrenstein, Albert} (23.12.1886 – 08.04.1950), \emph{Schriftsteller}!Lage in Ungarn (Siebenbuergen und Serbien ausgenommen) im Jahre
                  17901910@\strich\emph{Die Lage in Ungarn (Siebenbürgen und Serbien ausgenommen) im Jahre 1790} {[}1910{]}|pwv} zur gänzlichen Umarbeitung zurückzugeben. Hätte der gute Mann\pwindex{Fournier, August 19.06.1850 – 18.05.1920@\textsc{Fournier, August} (19.06.1850 – 18.05.1920), \emph{Historiker}|pwv} bei dieſer Abweiſung
               imponierendes Sachverſtändnis dokumentiert, ſo wäre dawider wohl nichts einzuwenden
               geweſen. Aber das war nicht allzuſehr der Fall. Eine übergroße und malitiöſe
               Empfindlichkeit modernerem und zugreifenderem Ausdruck und Satzbau gegenüber
               verführte ihn ſogar dazu, mir faſt auf jeder Seite Mängel ſtiliſtiſcher Natur
               nachweiſen zu wollen. Wozu erſtens der Verfaſſer\pwindex{Fournier, August 19.06.1850 – 18.05.1920@\textsc{Fournier, August} (19.06.1850 – 18.05.1920), \emph{Historiker}|pwv} des langweiligsten Napoleon\pwindex{Bonaparte, Napoleon 15.08.1769 – 21.05.1821@\textsc{Bonaparte, Napoleon} (15.08.1769 – 21.05.1821), \emph{Kaiser}|pw}buches\pwindex{Fournier, August 19.06.1850 – 18.05.1920@\textsc{Fournier, August} (19.06.1850 – 18.05.1920), \emph{Historiker}!Napoleon I. Eine Biographie1886@\strich\emph{Napoleon I. Eine Biographie} {[}1886{]}|pwv} nicht das Recht hatte,
               zweitens – und das iſt die komiſche Seite der Affaire – habe ich einem galiziſchen\oindex{Galizien@\textbf{Galizien}|pw}{ }Kollegen\pwindex{?? [Studienkollege von Albert Ehrenstein] *~1909@\textsc{?? [Studienkollege von Albert Ehrenstein]} (*~1909)|pwv}, der nicht gut
               Deutſch kann, ſeine Arbeit durchgeſehen und die gröbſten Verſtöße darin korrigiert.
               Bei dem hat der Hofrat\pwindex{Fournier, August 19.06.1850 – 18.05.1920@\textsc{Fournier, August} (19.06.1850 – 18.05.1920), \emph{Historiker}|pwv}
               merkwürdigerweiſe wenig Stilwidrigkeiten zu regiſtrieren gehabt. Warum? Weil ich dem
                  Polen\oindex{Polen@\textbf{Polen}|pw} den Tric angeraten hatte, dem Profeſſor\pwindex{Fournier, August 19.06.1850 – 18.05.1920@\textsc{Fournier, August} (19.06.1850 – 18.05.1920), \emph{Historiker}|pwv} von vornherein
               weiszumachen, er werde ſeine Diſſertation polniſch\oindex{Polen@\textbf{Polen}|pw} drucken laſſen. Da begann des Profeſſor\pwindex{Fournier, August 19.06.1850 – 18.05.1920@\textsc{Fournier, August} (19.06.1850 – 18.05.1920), \emph{Historiker}|pwv}s Eigenliebe und Nationalgefühl zu funktionieren.
               Eine aus ſeinem, einem Deutſchen Seminar hervorgegangene Abhandlung ſollte anderswo,
               in einer slawiſchen Sprache erſcheinen? Lieber veranlaßte er – was beabſichtigt war –
               die Drucklegung des Manuſkriptes in Deutſcher Sprache, {\pb}hatte an dem von ihm empfohlenen Werke\pwindex{?? [Studienkollege von Albert Ehrenstein] *~1909@\textsc{?? [Studienkollege von Albert Ehrenstein]} (*~1909)!?? [Dissertation]1909@\strich\emph{?? [Dissertation]} {[}1909{]}|pwv} (von dem er übrigens auch
               nicht viel verſteht) wenig zu bekritteln und prüfte den Polen\pwindex{?? [Studienkollege von Albert Ehrenstein] *~1909@\textsc{?? [Studienkollege von Albert Ehrenstein]} (*~1909)|pwv} nicht, ſondern plauſchte mit ihm beim
               Rigoroſum. Unglücklicherweiſe kann ich nicht magyariſch\oindex{Ungarn@\textbf{Ungarn}|pw} und daher nicht mit dem magyariſchen\oindex{Ungarn@\textbf{Ungarn}|pw} Erſcheinen meines ungariſche\oindex{Ungarn@\textbf{Ungarn}|pw} Verhältniſſe gloſſierenden Elaborates\pwindex{Ehrenstein, Albert 23.12.1886 – 08.04.1950@\textsc{Ehrenstein, Albert} (23.12.1886 – 08.04.1950), \emph{Schriftsteller}!Lage in Ungarn (Siebenbuergen und Serbien ausgenommen) im Jahre
                  17901910@\strich\emph{Die Lage in Ungarn (Siebenbürgen und Serbien ausgenommen) im Jahre 1790} {[}1910{]}|pwv} dienen.\pend
           \pstart
           Obgleich die Umarbeitung nur 3 Wochen in Anſspruch nahm, wurde ich, da es nur
               3 Lehramtsprüfungstermine im Jahr gibt und ich einen durch die Nichtannahme meiner
               Diſſertation verſäumen mußte, aus meiner Bahn geworfen, ich kann meinen
               urſprünglichen Plan nicht ausführen, werde um ein halbes Jahr ſpäter mit dem
               lächerlichen Namen- und Zahlenkram fertig werden, und außerdem – ich hatte ſchon
                  1908 keine Ferien – gibt es auch heuer keine Erholung für mich. Im
                  Oktober wird meine Abhandlung\pwindex{\textcolor{red}{\textsuperscript{XXXX1 indx}}!HelenaNone@\strich\emph{Helena} {[}None{]}|pwv} in ihrer neuen Form zenſiert. Mich noch weiterhin von dem Profeſſor\pwindex{Fournier, August 19.06.1850 – 18.05.1920@\textsc{Fournier, August} (19.06.1850 – 18.05.1920), \emph{Historiker}|pwv} wie einen
               Schuldigen behandeln zu laſſen, habe ich keine Luſt. Es iſt kaum ein Verbrechen, wenn
               man ſich einen biſſigen Hofrat mit einem Stückchen Wurſt vom Leibe hält, ebenſowenig
               halte ich es für korrupt, im Regen einen Schirm aufzuſpannen. Aus dieſer
               Weltanſchauung heraus muß ich es mit Freude begrüßen, wenn Sie, ſehr geehrter Herr
               Doktor, die Liebenswürdigkeit beſäßen, Herrn Auernheimer\pwindex{Auernheimer, Raoul 15.04.1876 – 06.01.1948@\textsc{Auernheimer, Raoul} (15.04.1876 – 06.01.1948), \emph{Schriftsteller, Journalist, Kritiker}|pw} gegenüber ein paar Worte über mich fallen zu laſſen. Ich möchte
               nämlich dann gern Ende Juli Herrn Auernheimer\pwindex{Auernheimer, Raoul 15.04.1876 – 06.01.1948@\textsc{Auernheimer, Raoul} (15.04.1876 – 06.01.1948), \emph{Schriftsteller, Journalist, Kritiker}|pw} eine Notiz über die im Erſcheinen begriffene Diſſertation\pwindex{?? [Studienkollege von Albert Ehrenstein] *~1909@\textsc{?? [Studienkollege von Albert Ehrenstein]} (*~1909)!?? [Dissertation]1909@\strich\emph{?? [Dissertation]} {[}1909{]}|pwv} jenes galiziſchen\oindex{Galizien@\textbf{Galizien}|pw}{ }Kollegen\pwindex{?? [Studienkollege von Albert Ehrenstein] *~1909@\textsc{?? [Studienkollege von Albert Ehrenstein]} (*~1909)|pwv}{ }{\pb}ſowie meinen Baber\pwindex{Ehrenstein, Albert 23.12.1886 – 08.04.1950@\textsc{Ehrenstein, Albert} (23.12.1886 – 08.04.1950), \emph{Schriftsteller}!Tod des Zehir eddin Muhammed Baber1912@\strich\emph{Tod des Zehir eddin Muhammed Baber} {[}1912{]}|pw} einſenden. Kurze Kritiken über Belletriſtiker einſchicken,
               was mir Auernheimer\pwindex{Auernheimer, Raoul 15.04.1876 – 06.01.1948@\textsc{Auernheimer, Raoul} (15.04.1876 – 06.01.1948), \emph{Schriftsteller, Journalist, Kritiker}|pw} geſtattete, mag ich nicht;
               ich ſehne mich nicht danach, mich mit irgendwelchen Literaten durch Tauſchhandel zu
               verfreunden, in meiner gegenwärtigen Stimmung würde ich übrigens ſelbſt den Herrgott
               zu diskreditieren verſuchen, und das eine wie das andere darf doch eigentlich nur
               einer, der durch eigene Schöpfungen öffentlich einen gewiſſen Befähigungsnachweis
               erbracht hat. Die Notiz über die von ihm empfohlene Diſſertation\pwindex{?? [Studienkollege von Albert Ehrenstein] *~1909@\textsc{?? [Studienkollege von Albert Ehrenstein]} (*~1909)!?? [Dissertation]1909@\strich\emph{?? [Dissertation]} {[}1909{]}|pwv} würde den Hiſtoriker umgänglicher machen, der
                  Baber\pwindex{Ehrenstein, Albert 23.12.1886 – 08.04.1950@\textsc{Ehrenstein, Albert} (23.12.1886 – 08.04.1950), \emph{Schriftsteller}!Tod des Zehir eddin Muhammed Baber1912@\strich\emph{Tod des Zehir eddin Muhammed Baber} {[}1912{]}|pw} – den ich ſonſt in aller Eile
               anderweitig unterzubringen das gefährliche und bei meinem Mangel an Beziehungen auch
               ausſichtsloſe Wagnis unternehmen müßte – würde ihm imponieren, den Geographieprofeſſor\pwindex{Oberhummer, Eugen 29.03.1859 – 04.05.1944@\textsc{Oberhummer, Eugen} (29.03.1859 – 04.05.1944), \emph{Geograf}|pwv}, der uns die Memoiren\pwindex{Babur, Zahir ad-Din Muhammad 14.02.1483 – 26.12.1530@\textsc{Babur, Zahir ad-Din Muhammad} (14.02.1483 – 26.12.1530), \emph{Regent}!BaburnamaNone@\strich\emph{Baburnama} {[}None{]}|pwv} dieſes Regenten\pwindex{Babur, Zahir ad-Din Muhammad 14.02.1483 – 26.12.1530@\textsc{Babur, Zahir ad-Din Muhammad} (14.02.1483 – 26.12.1530), \emph{Regent}|pwv} namhaft machte,
               freuen. Daher, um ſozuſagen als Reſpektsperſon wenigſtens Chikanen zu entgehen, wäre
               es mir wirklich ſehr angenehm, wenn Herr Auernheimer\pwindex{Auernheimer, Raoul 15.04.1876 – 06.01.1948@\textsc{Auernheimer, Raoul} (15.04.1876 – 06.01.1948), \emph{Schriftsteller, Journalist, Kritiker}|pw} nicht (wie im Feber) ſich ausſchließlich darauf
               beſchränkte, in meinen Manuſkripten hin und wieder einen Beiſtrich anzubringen, was
               mich beluſtigte, oder ab und zu ein »Sehr ſchön« hinzuſchreiben, was mich ärgerte.
               Heute noch würde es mich freuen und mir in vieler Beziehung helfen, wenn die Preſſe\orgindex{Neue Freie Presse@Neue Freie Presse|pw} oder ſonſt ein Blatt mich lancierte, in ein
               bis zwei Jahren, wenn ich einen Poſten habe, wird es mir ſehr gleichgültig ſein, ob
               mein Name in einer Zeitung ſteht, oder ob ich ihn mit dem Spazierſtock auf einen in
               der Sonne zerrinnenden Schneehaufen ſchreibe. {\pb}Die Ehre iſt ſchließlich ſchon jetzt nicht
               gar ſo überwältigend. Und ſpäter, wenn ich einmal bekannt ſein werde – ich bin
               ſchrecklich rachſüchtig – würden die Zeitungen zunächſt doch nichts anderes von mir
               bekommen als die von ihnen ſelbſt abgelehnten Sachen. Den Luxus, derartige Prinzipien
                  \introOben{}zu\introOben{} beſitzen zu glauben, kann ich mir ja jetzt noch
               getroſt geſtatten.\pend
           \pstart
           Indem ich zwar auf eine gnädige Erfüllung meiner \introOben{}unbeſcheidenen\introOben{} Wünſche hoffe, nichtsdeſtoweniger auch auf eine ſtrenge
               Kritik meiner novelliſtiſchen Taſtverſuche und moraliſchen Grundsätze gefaßt mache,
               verbleibe ich hochachtungsvoll\pend
           \pstart
           Ihr ergebenſter{\\[\baselineskip]}\spacefill\mbox{Albert Ehrenstein.}\pend
           \leftskip=0em{}
         
         \endnumbering\mylabel{h}\end{ledgroupsized}  \newcommand{\dateiname}{L01857}\newcommand{\titel}{Albert Ehrenstein an Arthur Schnitzler, 13. 7. 1909}\newcommand{\editorInnen}{Martin Anton Müller und Gerd-Hermann Susen}%% latex-leseansicht-abspann.tex
%% Abspann für die Leseansicht.
%% Der Schalter \ifkorrekturansicht ist bereits durch den Vorspann gesetzt.

%% latex-abspann.tex
%% Gemeinsamer Abspann für Korrekturansicht und Leseansicht.
%% Setzt den Schalter \ifkorrekturansicht voraus (gesetzt in den
%% einbindenden Dateien latex-korrekturansicht-abspann.tex bzw.
%% latex-leseansicht-abspann.tex).
%% ---------------------------------------------------------------

\normalsize

% Das esempio-Environment wird nur in der Leseansicht benötigt
\ifkorrekturansicht\else
\newenvironment{esempio}[3]%
{
    \vspace{1.5ex}
    \rlap{\underline{#1}}
    \par
    \setlength{\parindent}{0cm}
    \nopagebreak
    \leftskip=#2cm
    \rightskip=#3cm
}
{
    \par
}
\fi

\doendnotes{C}
\bigskip
\vfill

\clearpage

\footnotesize

\ifkorrekturansicht
  \lohead{\textsc{register}}
\fi

% theindex-Environment neu definieren ohne reledmac
\makeatletter
\renewenvironment{theindex}{%
  \ifkorrekturansicht
    \section*{\indexname}%
  \else
    \subsubsection*{Index der erwähnten Entitäten}%
  \fi
  \setlength{\parindent}{0pt}%
  \setlength{\parskip}{0pt plus 0.3pt}%
  \let\item\@idxitem
}{%
  \ifkorrekturansicht\clearpage\fi
}
\makeatother

\IfFileExists{\jobname-pw.ind}{\input{\jobname-pw.ind}}{}

% Quellenangabe nur in der Leseansicht
\ifkorrekturansicht\else
% Fallback-Definitionen, falls die .tex-Datei \titel etc. nicht gesetzt hat
\providecommand{\titel}{}
\providecommand{\editorInnen}{}
\providecommand{\dateiname}{\jobname}

\vspace{3cm}

\vfill

\footnotesize
\textsc{Quelle}: \titel. Herausgegeben von {\editorInnen}. In: \emph{Arthur Schnitzler: Briefwechsel mit Autorinnen und Autoren}.
 Digitale Edition, https://schnitzler-briefe.acdh.oeaw.ac.at/{\dateiname}.html (Stand \today)
\fi

\end{document}


      