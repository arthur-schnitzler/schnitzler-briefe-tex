%% latex-leseansicht-vorspann.tex
%% Vorspann für die Leseansicht.
%% Lädt die gemeinsame Datei latex-vorspann.tex mit nicht gesetztem Schalter.

\newif\ifkorrekturansicht
\korrekturansichtfalse

\input{../tex-inputs/latex-vorspann}


\section[Paul Goldmann an Arthur Schnitzler, 31. 12. [1894]]{L02630 Paul Goldmann an Arthur Schnitzler, 31. 12. [1894]}
\nopagebreak\mylabel{L02630v}
\rehead{ }\normalsize\beginnumbering\briefempfaengerindex{Schnitzler, Arthur@\textsc{Schnitzler, Arthur}!zzzGoldmann, Paul@\emph{von Paul Goldmann}!1894-12-312@{31. 12. [1894]}|(be}
\toendnotes[C]{\smallbreak\pagebreak[2]}
\correspDesc{Versand  durch Paul Goldmann am 31. 12. [1894] in Paris
\newline{}Erhalt  durch Arthur Schnitzler im Zeitraum [1. 1. 1895
                  – 5. 1. 1895?] in Wien}\toendnotes[C]{\smallbreak}
\Standort{DLA, A:Schnitzler, HS.NZ85.1.3164.}
\physDesc{Brief, 3 Blätter, 11 Seiten, 5140 Zeichen
\newline{}Handschrift: schwarze Tinte, deutsche Kurrent
\newline{}Schnitzler: 1) mit Bleistift auf dem ersten Blatt die Jahreszahl »94« vermerkt  2) mit rotem Buntstift sieben Unterstreichungen}\toendnotes[C]{\smallbreak}
\pstart
           {\pb}\textcolor{gray}{\textbf{Frankfurter Zeitung\orgindex{Frankfurter Zeitung@Frankfurter Zeitung|pw}}}\hfill \textsc{Paris\oindex{Paris@\textbf{Paris}, \emph{Hauptstadt}|pw}}, 31. December.\pend
           
\pstart
           \textcolor{gray}{\textbf{(Gazette de
                     Francfort\orgindex{Frankfurter Zeitung@Frankfurter Zeitung|pw}).}}\pend
           
\pstart
           \textcolor{gray}{\textbf{Fondateur \textbf{M. L. Sonnemann\pwindex{Sonnemann, Leopold 29.\,10.\,1831 Höchberg – 30.\,10.\,1909 Frankfurt am Main@\textsc{Sonnemann, Leopold} (29.\,10.\,1831 Höchberg – 30.\,10.\,1909 Frankfurt am Main), \emph{Journalist, Herausgeber}|pw}}.}}\pend
           
\pstart
           \textcolor{gray}{\textbf{\begin{otherlanguage}{french}Journal politique, financier,\end{otherlanguage}}}\pend
           
\pstart
           \textcolor{gray}{\textbf{\begin{otherlanguage}{french}commercial et littéraire.\end{otherlanguage}}}\pend
           
\pstart
           \textcolor{gray}{\textbf{\begin{otherlanguage}{french}\textbf{Paraissant trois fois par jour.}\end{otherlanguage}}}\pend
           
\pstart
           \textcolor{gray}{\textbf{\begin{otherlanguage}{french}\textbf{Bureau à Paris\oindex{Paris@\textbf{Paris}, \emph{Hauptstadt}|pw}:}\end{otherlanguage}}}\pend
           
\pstart
           \textcolor{gray}{\textbf{\begin{otherlanguage}{french}24. Rue Feydeau\oindex{rue Feydeau@\textbf{rue Feydeau}, \emph{Straße}|pw}.\end{otherlanguage}}}\pend
           
\pstart{}Mein lieber Freund,\pend\vspace{0.5em}
\pstart
           das{ }ſind recht erfreuliche Nachrichten, – unberufen! – die Dein Brief bringt. \label{K_L02630-1v}\edtext{\textsc{Speidel\pwindex{Speidel, Ludwig 11.\,4.\,1830 Ulm – 3.\,2.\,1906 Wien@\textsc{Speidel, Ludwig} (11.\,4.\,1830 Ulm – 3.\,2.\,1906 Wien), \emph{Journalist, Kritiker}|pw}}}{\lemma{\textnormal{\emph{Speidel}}}\Cendnote{\textnormal{Zum positiven Urteil Ludwig Speidels\pwindex{Speidel, Ludwig 11.\,4.\,1830 Ulm – 3.\,2.\,1906 Wien@\textsc{Speidel, Ludwig} (11.\,4.\,1830 Ulm – 3.\,2.\,1906 Wien), \emph{Journalist, Kritiker}|pwk} über \emph{Liebelei}\pwindex{Schnitzler, Arthur 15.\,5.\,1862 Wien – 21.\,10.\,1931 ebd.@\textsc{Schnitzler, Arthur} (15.\,5.\,1862 Wien – 21.\,10.\,1931 ebd.), \emph{Schriftsteller, Mediziner}!Liebelei. Schauspiel in drei Akten@\strich\emph{Liebelei. Schauspiel in drei Akten}|pwk}{ }vgl. A. S.: \emph{Tagebuch}, 14. 12. 1894, 17. 12. 1894 und 18. 12. 1894.
               }}}\label{K_L02630-1} beſonders iſt eine förmliche Überraſchung. Der Mann, der \substVorne{}\textsuperscript{\textcolor{gray}{×}\-\textcolor{gray}{×}}\substDazwischen{}bei\substHinten{} der Lampe nach Mitternacht über Deinem Stücke\pwindex{Schnitzler, Arthur 15.\,5.\,1862 Wien – 21.\,10.\,1931 ebd.@\textsc{Schnitzler, Arthur} (15.\,5.\,1862 Wien – 21.\,10.\,1931 ebd.), \emph{Schriftsteller, Mediziner}!Liebelei. Schauspiel in drei Akten@\strich\emph{Liebelei. Schauspiel in drei Akten}|pwv}{ }ſitzt, wird mir beinahe{ }ſympathiſch. \strikeout{H} Sollten wir ihm vielleicht Unrecht gethan haben? Er
               war gegen das Neue; aber hat es denn viel Neues gegeben? Und haben wir nicht am Ende
               das Neue mit uns verwechſelt, die wir neu waren? Das Urtheil, das er über Dich fällt,{ }ſpricht{ }ſehr zu Ehren {\pb}ſeines Kunſtverſtändniſſes.
               Nun kann es doch unmöglich mehr fehlen. Wo{ }ſoviel Mächtige dafür{ }ſind, wird das
               Theater-Geſindel nichts mehr ausrichten können. Daß B.\pwindex{Burckhard, Max Eugen 14.\,7.\,1854 Korneuburg – 16.\,3.\,1912 Wien@\textsc{Burckhard, Max Eugen} (14.\,7.\,1854 Korneuburg – 16.\,3.\,1912 Wien), \emph{Schriftsteller, Rechtswissenschaftler, Theaterleiter}|pw} Dich \label{K_L02630-2v}\edtext{beſucht}{\lemma{\textnormal{\emph{besucht}}}\Cendnote{\textnormal{Vgl. A. S.: \emph{Tagebuch}, 18. 12. 1894.
               }}}\label{K_L02630-2}, imponirt mir beſonders. Welchen Weg haſt Du durchlaufen \strikeout{zwiſchen} von drei Jahren bis auf heut! Mir
               kommt{ }ſo vor, als{ }ſei jetzt nur noch ein tüchtiger Ruck zu geben, und dann am Ziel!
               Wenn{ }ſich die \textsc{Sandrock\pwindex{Sandrock, Adele 19.\,8.\,1863 Rotterdam – 30.\,8.\,1937 Berlin@\textsc{Sandrock, Adele} (19.\,8.\,1863 Rotterdam – 30.\,8.\,1937 Berlin), \emph{Schauspielerin}|pw}} vom \label{K_L02630-3v}\edtext{Volkstheater\orgindex{Volkstheater@Volkstheater|pw} jetzt{ }ſchon losmachen}{\lemma{\textnormal{\emph{Volkstheater … losmachen}}}\Cendnote{\textnormal{Der Wechsel von Adele Sandrock\pwindex{Sandrock, Adele 19.\,8.\,1863 Rotterdam – 30.\,8.\,1937 Berlin@\textsc{Sandrock, Adele} (19.\,8.\,1863 Rotterdam – 30.\,8.\,1937 Berlin), \emph{Schauspielerin}|pwk} ans \emph{Burgtheater}\orgindex{Burgtheater@Burgtheater|pwk} war schon im Sommer 1894 für die
                  Saison 1895/1896 fixiert worden. Durch neuerliche
                  Verhandlungen fand der Übertritt bereits zum 1. 2. 1895 statt. Sie
                  war also in Erwartung ihrer Verfügbarkeit für die Rolle der Christine\pwindex{Schnitzler, Arthur 15.\,5.\,1862 Wien – 21.\,10.\,1931 ebd.@\textsc{Schnitzler, Arthur} (15.\,5.\,1862 Wien – 21.\,10.\,1931 ebd.), \emph{Schriftsteller, Mediziner}!Liebelei. Schauspiel in drei Akten@\strich\emph{Liebelei. Schauspiel in drei Akten}|pwkv} vorgesehen.}}}\label{K_L02630-3} könnte,{ }ſo
               wäre es wohl gut (Warum{ }ſpielt übrigens die \textsc{Hohenfels\pwindex{Hohenfels, Stella 16.\,4.\,1857 Florenz – 21.\,2.\,1920 Wien@\textsc{Hohenfels, Stella} (16.\,4.\,1857 Florenz – 21.\,2.\,1920 Wien), \emph{Schauspielerin}|pw}} nicht die Rolle?). Wenn nicht,{ }ſo warteſt Du ruhig bis zum nächſten Jahr. Der
                  \label{K_L02630-4v}\edtext{Titel »Liebelei\pwindex{Schnitzler, Arthur 15.\,5.\,1862 Wien – 21.\,10.\,1931 ebd.@\textsc{Schnitzler, Arthur} (15.\,5.\,1862 Wien – 21.\,10.\,1931 ebd.), \emph{Schriftsteller, Mediziner}!Liebelei. Schauspiel in drei Akten@\strich\emph{Liebelei. Schauspiel in drei Akten}|pw}« mißfällt}{\lemma{\textnormal{\emph{Titel … mißfällt}}}\Cendnote{\textnormal{Ein Erfolgsstück des Jahres 1893 war \emph{Das arme Mädel}\pwindex{Lindau, Karl 26.\,11.\,1853 Wien – 15.\,1.\,1934 ebd.@\textsc{Lindau, Karl} (26.\,11.\,1853 Wien – 15.\,1.\,1934 ebd.), \emph{Schriftsteller, Schauspieler, Librettist}!arme Mädel@\strich\emph{Das arme Mädel}|pwk}\pwindex{Krenn, Leopold 6.\,12.\,1850 Wien – 2.\,10.\,1930 ebd.@\textsc{Krenn, Leopold} (6.\,12.\,1850 Wien – 2.\,10.\,1930 ebd.), \emph{Schriftsteller, Beamter}!arme Mädel@\strich\emph{Das arme Mädel}|pwk} von Karl Lindau\pwindex{Lindau, Karl 26.\,11.\,1853 Wien – 15.\,1.\,1934 ebd.@\textsc{Lindau, Karl} (26.\,11.\,1853 Wien – 15.\,1.\,1934 ebd.), \emph{Schriftsteller, Schauspieler, Librettist}|pwk} und
                     Leopold Krenn\pwindex{Krenn, Leopold 6.\,12.\,1850 Wien – 2.\,10.\,1930 ebd.@\textsc{Krenn, Leopold} (6.\,12.\,1850 Wien – 2.\,10.\,1930 ebd.), \emph{Schriftsteller, Beamter}|pwk}. Das dürfte Schnitzler gezwungen haben, Ersatz für seinen
                  Arbeitstitel »Armes Mädl« zu suchen, mit dem Goldmann\pwindex{Goldmann, Paul 31.\,1.\,1865 Breslau – 25.\,9.\,1935 Wien@\textsc{Goldmann, Paul} (31.\,1.\,1865 Breslau – 25.\,9.\,1935 Wien), \emph{Schriftsteller, Journalist}|pwk} bis dahin vertraut war. In einem Interview nahm Schnitzler{ }1912 dazu Stellung: »Hätte das Stück nicht den Titel ›Liebelei\pwindex{Schnitzler, Arthur 15.\,5.\,1862 Wien – 21.\,10.\,1931 ebd.@\textsc{Schnitzler, Arthur} (15.\,5.\,1862 Wien – 21.\,10.\,1931 ebd.), \emph{Schriftsteller, Mediziner}!Liebelei. Schauspiel in drei Akten@\strich\emph{Liebelei. Schauspiel in drei Akten}|pw}‹, also die Bezeichnung für das
                     leichte, flüchtige Gefühl bar jeder Verantwortung, das ein junger Mann hegt,
                     dem ein sorgenbelastetes, ernsthaft verliebtes Mädchen gegenübersteht, sondern
                     hieße, sagen wir ›Die große Liebe der Christine‹ – also die Bezeichnung für das
                     Gefühl des Mädchens –, so hätte das Publikum dem Stück ganz gewiß nicht
                     dasselbe Interesse entgegengebracht, wie beim Titel ›Liebelei\pwindex{Schnitzler, Arthur 15.\,5.\,1862 Wien – 21.\,10.\,1931 ebd.@\textsc{Schnitzler, Arthur} (15.\,5.\,1862 Wien – 21.\,10.\,1931 ebd.), \emph{Schriftsteller, Mediziner}!Liebelei. Schauspiel in drei Akten@\strich\emph{Liebelei. Schauspiel in drei Akten}|pw}‹.« (Ifj. B. Gy [ = Georg Ruttkay]\pwindex{Ruttkay, Georg 1.\,6.\,1890 Budapest – 18.\,10.\,1955 Wien@\textsc{Ruttkay, Georg} (1.\,6.\,1890 Budapest – 18.\,10.\,1955 Wien), \emph{Schriftsteller, Journalist}|pwk}: \emph{Schnitzler Arthurnál}. In: \emph{Az est}\pwindex{Az Est@\emph{Az Est}|pwk}, Jg. 3, Nr. 112, 10.\,5.\,1912, S. 8. Siehe A. S.: \emph{»Das Zeitlose ist von kürzester Dauer«}, Ifj. B. Gy [= Georg Ruttkay]: Schnitzler Arthurnál, 10. 5. 1912);
                     siehe XXXX Auszeichnungsfehler: Dokument L02726 nicht gefunden.
               }}}\label{K_L02630-4} mir. {\pb}Er klingt maniriert, unliterariſch und
               verkleinert die Arbeit. Ich möchte, daß Du auf die kleine \textsc{Nuance} verzichteſt und einfach gerade heraus »Eine Liebſchaft«{ }ſagſt. Das
               klingt mehr nach bürgerlichem Drama. Und nun werde ich endlich ungeduldig. Alle Welt
               hat{ }ſchon über dem Stücke\pwindex{Schnitzler, Arthur 15.\,5.\,1862 Wien – 21.\,10.\,1931 ebd.@\textsc{Schnitzler, Arthur} (15.\,5.\,1862 Wien – 21.\,10.\,1931 ebd.), \emph{Schriftsteller, Mediziner}!Liebelei. Schauspiel in drei Akten@\strich\emph{Liebelei. Schauspiel in drei Akten}|pwv}
               geſeſſen, mit \strikeout{B} Bangen und ohne. Ich weiß allerlei
               Urtheile und kenne es{ }ſelber noch nicht. Könnteſt Du es mir nicht auf wenige Tage
               zugänglich machen? Ich leſe es in einem Tage aus und{ }ſchicke es{ }ſofort zurück. Bitte,
               bitte, mach’ es irgendwie möglich; Du kannſt Dir denken, wie geſpannt {\pb}ich bin. Die Spannung wächſt mit jeder neuen
               Nachricht. Nun muß ichs endlich kennen lernen, zum Teufel auch! Und, nicht wahr,{ }ſobald Cenſur und Intendanz geſprochen haben, theilſt Du mir{ }ſofort das Reſultat mit?
               Schreib’ mir auch, ob die Frankf. Ztg.\orgindex{Frankfurter Zeitung@Frankfurter Zeitung|pw} etwas
               darüber bringen{ }ſoll. Einſtweilen beglückwünſche ich Dich von Herzen zu den
               bisherigen guten Reſultaten{[}.{]}{ }\textsc{Speidel\pwindex{Speidel, Ludwig 11.\,4.\,1830 Ulm – 3.\,2.\,1906 Wien@\textsc{Speidel, Ludwig} (11.\,4.\,1830 Ulm – 3.\,2.\,1906 Wien), \emph{Journalist, Kritiker}|pw}} iſt bereits der halbe Erfolg. Ich freue mich{ }ſehr{\dotsfive}\pend
           
\pstart
           In einem der nächſten Hefte des »\textsc{Mercure de France\orgindex{Mercure de France@Mercure de France|pw}}« kommt ein \label{K_L02630-5v}\edtext{Aufſatz\pwindex{Albert, Henri 16.\,11.\,1869 Niederbronn-les-Bains – 3.\,8.\,1921 Straßburg@\textsc{Albert, Henri} (16.\,11.\,1869 Niederbronn-les-Bains – 3.\,8.\,1921 Straßburg), \emph{Journalist, Kritiker, Übersetzer}!Jeunes Viennois@\strich\emph{Les Jeunes Viennois}|pwv}}{\lemma{\textnormal{\emph{Aufsatz}}}\Cendnote{\textnormal{Der Text erschien mit einer gewissen
                  Verzögerung in einer anderen Zeitschrift: Henri Albert\pwindex{Albert, Henri 16.\,11.\,1869 Niederbronn-les-Bains – 3.\,8.\,1921 Straßburg@\textsc{Albert, Henri} (16.\,11.\,1869 Niederbronn-les-Bains – 3.\,8.\,1921 Straßburg), \emph{Journalist, Kritiker, Übersetzer}|pwk}: \emph{Les Jeunes Viennois}\pwindex{Albert, Henri 16.\,11.\,1869 Niederbronn-les-Bains – 3.\,8.\,1921 Straßburg@\textsc{Albert, Henri} (16.\,11.\,1869 Niederbronn-les-Bains – 3.\,8.\,1921 Straßburg), \emph{Journalist, Kritiker, Übersetzer}!Jeunes Viennois@\strich\emph{Les Jeunes Viennois}|pwk}. In: \emph{Revue des revues}\pwindex{Revue des Revues@\emph{Revue des Revues}|pwk}, Bd. 13, 1. 4. 1895,
                     S. 8–13.}}}\label{K_L02630-5} von \textsc{Albert\pwindex{Albert, Henri 16.\,11.\,1869 Niederbronn-les-Bains – 3.\,8.\,1921 Straßburg@\textsc{Albert, Henri} (16.\,11.\,1869 Niederbronn-les-Bains – 3.\,8.\,1921 Straßburg), \emph{Journalist, Kritiker, Übersetzer}|pw}} über Euch. Leider hat er mich nicht um Rath {\pb}beim Schreiben gefragt. Es{ }ſtehen alſo offenbar einige Stiefel drin. Aber die
               Hauptſache iſt doch, daß etwas geſchrieben wird. Auch will er nächſtens etwas von Dir
               überſetzen. Wie macht{ }ſich der literariſche und buchhändleriſche Erfolg von »Sterben\pwindex{Schnitzler, Arthur 15.\,5.\,1862 Wien – 21.\,10.\,1931 ebd.@\textsc{Schnitzler, Arthur} (15.\,5.\,1862 Wien – 21.\,10.\,1931 ebd.), \emph{Schriftsteller, Mediziner}!Sterben. Novelle@\strich\emph{Sterben. Novelle}|pw}«?\pend
           
\pstart
           Was hört man von der »Zeit\orgindex{Zeit. Wiener Wochenschrift@Die Zeit. Wiener Wochenschrift|pw}«? Wie geht{ }ſie und wie
               gefällt{ }ſie?\pend
           
\pstart
           Gern will ich Dir die Frankf. Ztg.\orgindex{Frankfurter Zeitung@Frankfurter Zeitung|pw}{ }ſchicken, wenn
               ich etwas darin habe. Aber ich habe kaum mehr etwas drin. Kann {\pb}mich nicht mehr zum Schreiben aufraffen. Es liegen
               Centnerlaſten auf mir. Die Krankheit, die nicht heilen will – Ihr Ärzte{ }ſeid nichts
               als menſchenfreundliche Lügner – die Vereinſamung, die Heimatloſigkeit, das Gefühl
               des Zurückbleibens, die Verlotterung. Wie ich aus \textsc{Ischl\oindex{Bad Ischl@\textbf{Bad Ischl}|pw}} zurückkam, wollte ich eine Rieſen-Anſtrengung machen. Die iſt mißlungen, und
               nun laſſe ich mich{ }ſinken und leiſte nur mehr wenig Widerſtand. Ich leſe nicht ein
               Mal mehr ein Buch zu Ende; und wenn die Reue kommt,{ }ſo{ }ſlüchte ich mich in Politik
               und Depeſchen hinein.\pend
           
\pstart
           {\pb}Den Brief an Frl. \textsc{Sandrock\pwindex{Sandrock, Adele 19.\,8.\,1863 Rotterdam – 30.\,8.\,1937 Berlin@\textsc{Sandrock, Adele} (19.\,8.\,1863 Rotterdam – 30.\,8.\,1937 Berlin), \emph{Schauspielerin}|pw}} habe ich endlich geſchrieben. Es war keine Kleinigkeit. Ich{ }ſollte meine
               Anſicht über das Leben mittheilen. Das iſt nicht leicht, wenn man viel zu thun hat.
               Ich habe ein idiotiſches Zeug abgeſchickt, \textsc{\label{K_L02630-6v}\edtext{mais enfin}{\lemma{\textnormal{\emph{mais enfin}}}\Cendnote{\textnormal{französisch: aber
                     zuletzt}}}\label{K_L02630-6}}, ich habe geantwortet.\pend
           
\pstart
           Ich möchte ein wenig wiſſen, wie Du lebſt? Geſellſchaft? Freundſchaſt? Abenteuer?\pend
           
\pstart
           \label{K_L02630-7v}\edtext{\textsc{Bahr\pwindex{Bahr, Hermann 19.\,7.\,1863 Linz – 15.\,1.\,1934 München@\textsc{Bahr, Hermann} (19.\,7.\,1863 Linz – 15.\,1.\,1934 München), \emph{Schriftsteller, Kritiker}|pw}} hat mich neulich in{ }ſehr liebenswürdiger Weiſe citirt}{\lemma{\textnormal{\emph{Bahr … citirt}}}\Cendnote{\textnormal{Sein Text\pwindex{Bahr, Hermann 19.\,7.\,1863 Linz – 15.\,1.\,1934 München@\textsc{Bahr, Hermann} (19.\,7.\,1863 Linz – 15.\,1.\,1934 München), \emph{Schriftsteller, Kritiker}!Camille Mauclair@\strich\emph{Camille Mauclair}|pwkv} beginnt mit: »Als ich diesen Mai
                     in Paris\oindex{Paris@\textbf{Paris}, \emph{Hauptstadt}|pw} mit Paul Goldmann\pwindex{Goldmann, Paul 31.\,1.\,1865 Breslau – 25.\,9.\,1935 Wien@\textsc{Goldmann, Paul} (31.\,1.\,1865 Breslau – 25.\,9.\,1935 Wien), \emph{Schriftsteller, Journalist}|pw}, dem Correspondenten\pwindex{Goldmann, Paul 31.\,1.\,1865 Breslau – 25.\,9.\,1935 Wien@\textsc{Goldmann, Paul} (31.\,1.\,1865 Breslau – 25.\,9.\,1935 Wien), \emph{Schriftsteller, Journalist}|pwv} der Frankfurter Zeitung\orgindex{Frankfurter Zeitung@Frankfurter Zeitung|pw}, plauderte und um jeden Preis ein neues Talent
                     wissen wollte, sagte er mir: ›Ein Talent? Ein neues Talent? Ein ernstes,
                     sicheres, wirkliches Talent? Nicht bloß so eine geschwinde und vergängliche
                     Erfindung der Journale von heute auf morgen? Das ist schwer. Da ist jetzt wohl
                     niemand als Camille Mauclair\pwindex{Mauclair, Camille 29.\,11.\,1872 Paris – 23.\,4.\,1945 ebd.@\textsc{Mauclair, Camille} (29.\,11.\,1872 Paris – 23.\,4.\,1945 ebd.), \emph{Schriftsteller}|pw}. Sonst wüßte
                     ich keinen. Er hat freilich eigentlich noch nichts geschrieben; aber alle
                     hoffen viel von ihm. Er verspricht mehr, als er bis jetzt gehalten hätte; aber
                     er scheint mir sicher. Stellen Sie sich etwa, ins Paris\oindex{Paris@\textbf{Paris}, \emph{Hauptstadt}|pw}erische übersetzt, Ihren kleinen Hofmannsthal\pwindex{Hofmannsthal, Hugo von 1.\,2.\,1874 Wien – 15.\,7.\,1929 Rodaun@\textsc{Hofmannsthal, Hugo von} (1.\,2.\,1874 Wien – 15.\,7.\,1929 Rodaun), \emph{Schriftsteller}|pw} vor.‹« (Hermann Bahr\pwindex{Bahr, Hermann 19.\,7.\,1863 Linz – 15.\,1.\,1934 München@\textsc{Bahr, Hermann} (19.\,7.\,1863 Linz – 15.\,1.\,1934 München), \emph{Schriftsteller, Kritiker}|pwk}: \emph{Camille Mauclair}\pwindex{Bahr, Hermann 19.\,7.\,1863 Linz – 15.\,1.\,1934 München@\textsc{Bahr, Hermann} (19.\,7.\,1863 Linz – 15.\,1.\,1934 München), \emph{Schriftsteller, Kritiker}!Camille Mauclair@\strich\emph{Camille Mauclair}|pwk}. In: \emph{Die Zeit}\pwindex{Zeit. Wiener Wochenschrift@\emph{Die Zeit. Wiener Wochenschrift}|pwk}, Bd. 1, H. 10, 8. 12. 1894,
                  S. 154–155.)}}}\label{K_L02630-7}. Warum hat er das gethan?\pend
           
\pstart
           Ich mache mir Vorwürſe, daß ich Dich zum Abonnement auf das {\pb}»Journal\orgindex{Le Journal@Le Journal|pwv}« aufgefordert habe. Es wird niederträchtig{ }ſchlecht. Vielleicht
               verſuchſt Du es fortan mit der Abendausgabe des »\textsc{Journal des Débats\orgindex{Journal des débats@Journal des débats|pw}}«. Die politiſchen Artikel brauchſt Du ja nicht zu leſen; aber es{ }ſind köſtliche
                  \textsc{\label{K_L02630-8v}\edtext{chroniqueurs}{\lemma{\textnormal{\emph{chroniqueurs}}}\Cendnote{\textnormal{französisch: Kolumnisten}}}\label{K_L02630-8}} darin, höhere literariſche Leute: \textsc{Hallays\pwindex{Hallays, André 16.\,3.\,1859 Paris – 30.\,3.\,1930 ebd.@\textsc{Hallays, André} (16.\,3.\,1859 Paris – 30.\,3.\,1930 ebd.), \emph{Journalist, Kunstkritiker, Jurist}|pw}, Bazin\pwindex{Bazin, René 26.\,12.\,1853 Angers – 20.\,7.\,1932 Paris@\textsc{Bazin, René} (26.\,12.\,1853 Angers – 20.\,7.\,1932 Paris), \emph{Schriftsteller}|pw}, Filon\pwindex{Filon, Augustin 28.\,11.\,1841 Paris – 13.\,6.\,1916 Croydon@\textsc{Filon, Augustin} (28.\,11.\,1841 Paris – 13.\,6.\,1916 Croydon), \emph{Schriftsteller}|pw}, Lemaître\pwindex{Lemaître, Jules 27.\,4.\,1853 Vennecy – 4.\,8.\,1914 Tavers@\textsc{Lemaître, Jules} (27.\,4.\,1853 Vennecy – 4.\,8.\,1914 Tavers), \emph{Schriftsteller, Librettist}|pw}{ }}\textsc{etc}. Willſt Du, daß ichs Dir abonnire? Noch habe ich \textsc{30 Francs 30 ct.}, die Du beharrlich todtſchweigſt. Hat \textsc{Richard\pwindex{Beer-Hofmann, Richard 11.\,7.\,1866 Wien – 26.\,9.\,1945 New York City@\textsc{Beer-Hofmann, Richard} (11.\,7.\,1866 Wien – 26.\,9.\,1945 New York City), \emph{Schriftsteller}|pw}} den »\label{K_L02630-9v}\edtext{Courrier Français\pwindex{Le Courrier français@\emph{Le Courrier français}|pw}}{\lemma{\textnormal{\emph{Courrier Français}}}\Cendnote{\textnormal{illustrierte Satirezeitschrift\pwindex{Le Courrier français@\emph{Le Courrier français}|pwkv}, die zwischen
                     1884 und 1914 erschien}}}\label{K_L02630-9}« abonnirt? Sonſt{ }ſchicke ich ihn Dir. Anbei{ }ſchicke ich Dir wieder ein paar \label{K_L02630-10v}\edtext{Artikel}{\lemma{\textnormal{\emph{Artikel}}}\Cendnote{\textnormal{Die Beilagen sind nicht
               überliefert.}}}\label{K_L02630-10}, Kraut und Rüben durcheinander. \textsc{Drumont\pwindex{Drumont, Édouard 3.\,5.\,1844 Paris – 5.\,2.\,1917 ebd.@\textsc{Drumont, Édouard} (3.\,5.\,1844 Paris – 5.\,2.\,1917 ebd.), \emph{Journalist, Rassentheoretiker}|pw}} iſt ein großer {\pb}Polemiſt, nur{ }ſtark irrſinnig.
               In Bezug auf Juden und Deutſche leidet er an Verfolgungswahn. Aber in erſterer
               Beziehung beginnt der Irrſinn doch erſt nach einer weiten Grenze; Vieles
               Unglaubliche, was er über jüdiſche Corruption{ }ſchreibt, iſt wahr. Auch iſt er
               größenwahnſinnig und kommt{ }ſich thatſächlich als gottgeſandter Meſſias vor.
               Anderſeits gibt ihm aber gerade nur dieſer Wahnſinn die ungeheure Kraft, mit der er
               manchmal dreinſchlägt.\pend
           
\pstart
           {\pb}\textsc{Sokal\pwindex{Sokal, Clemens *~21.\,1.\,1867 Lviv@\textsc{Sokal, Clemens} (*~21.\,1.\,1867 Lviv), \emph{Journalist, Rechtsanwalt}|pw}} war bei mir; er gefällt mir gut. Scheint ein geſcheiter und ernſter Menſch zu{ }ſein{\dotsfour}\pend
           
\pstart
           Ich wünſche Dir von Herzen Glück zum neuen Jahr. Mir ahnt, daß das Jahr
                  1895 wichtig für Dich werden wird. Sieht es nicht vertrauenerweckend
               aus? Mit{ }ſeiner runden Fünfheiten!\pend
           
\pstart
           Was aber auch geſchehen mag, Gutes oder Allerbeſtes, wir bleiben die Alten, nicht
               wahr?\pend
           
\pstart
           Herzlichſt und in Treue Dein{\\[\baselineskip]}\spacefill\mbox{Paul Goldmann\textcolor{gray}{.}}\pend
           \leftskip=0em{}
\pstart
           \noindent{}{\pb}Bitte, empfiehl’ mich Deiner Frau Mutter\pwindex{Schnitzler, Louise 8.\,7.\,1840 Kőszeg – 9.\,9.\,1911 Wien@\textsc{Schnitzler, Louise} (8.\,7.\,1840 Kőszeg – 9.\,9.\,1911 Wien)|pwv} und richte \label{T_L02630-1v}\edtext{ihr}{\lemma{\textnormal{\emph{ihr}}}\Cendnote{\textnormal{Goldmann schreibt
                        »Ihr«.}}}\label{T_L02630-1} meine ergebenſten Neujahrs-Wünſche aus.\pend
           
\pstart
           Was lieſt Du jetzt?\pend
           \selectlanguage{ngerman}\endnumbering\briefempfaengerindex{Schnitzler, Arthur@\textsc{Schnitzler, Arthur}!zzzGoldmann, Paul@\emph{von Paul Goldmann}!1894-12-312@{31. 12. [1894]}|)be}\mylabel{L02630h}  \newcommand{\dateiname}{L02630}\newcommand{\titel}{Paul Goldmann an Arthur Schnitzler, 31. 12. [1894]}\newcommand{\editorInnen}{Martin Anton Müller und Laura Untner}%% latex-leseansicht-abspann.tex
%% Abspann für die Leseansicht.
%% Der Schalter \ifkorrekturansicht ist bereits durch den Vorspann gesetzt.

%% latex-abspann.tex
%% Gemeinsamer Abspann für Korrekturansicht und Leseansicht.
%% Setzt den Schalter \ifkorrekturansicht voraus (gesetzt in den
%% einbindenden Dateien latex-korrekturansicht-abspann.tex bzw.
%% latex-leseansicht-abspann.tex).
%% ---------------------------------------------------------------

\normalsize

% Das esempio-Environment wird nur in der Leseansicht benötigt
\ifkorrekturansicht\else
\newenvironment{esempio}[3]%
{
    \vspace{1.5ex}
    \rlap{\underline{#1}}
    \par
    \setlength{\parindent}{0cm}
    \nopagebreak
    \leftskip=#2cm
    \rightskip=#3cm
}
{
    \par
}
\fi

\doendnotes{C}
\bigskip
\vfill

\clearpage

\footnotesize

\ifkorrekturansicht
  \lohead{\textsc{register}}
\fi

% theindex-Environment neu definieren ohne reledmac
\makeatletter
\renewenvironment{theindex}{%
  \ifkorrekturansicht
    \section*{\indexname}%
  \else
    \subsubsection*{Index der erwähnten Entitäten}%
  \fi
  \setlength{\parindent}{0pt}%
  \setlength{\parskip}{0pt plus 0.3pt}%
  \let\item\@idxitem
}{%
  \ifkorrekturansicht\clearpage\fi
}
\makeatother

\IfFileExists{\jobname-pw.ind}{\input{\jobname-pw.ind}}{}

% Quellenangabe nur in der Leseansicht
\ifkorrekturansicht\else
% Fallback-Definitionen, falls die .tex-Datei \titel etc. nicht gesetzt hat
\providecommand{\titel}{}
\providecommand{\editorInnen}{}
\providecommand{\dateiname}{\jobname}

\vspace{3cm}

\vfill

\footnotesize
\textsc{Quelle}: \titel. Herausgegeben von {\editorInnen}. In: \emph{Arthur Schnitzler: Briefwechsel mit Autorinnen und Autoren}.
 Digitale Edition, https://schnitzler-briefe.acdh.oeaw.ac.at/{\dateiname}.html (Stand \today)
\fi

\end{document}


