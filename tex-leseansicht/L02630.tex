%% latex-korrekturansicht-vorspann.tex
%% Vorspann für die Korrekturansicht.
%% Lädt die gemeinsame Datei latex-vorspann.tex mit gesetztem Schalter.

\newif\ifkorrekturansicht
\korrekturansichttrue

\input{../tex-inputs/latex-vorspann}


\section[Paul Goldmann an Arthur Schnitzler, 31. 12. {[}1894{]}]{L02630 Paul Goldmann an Arthur Schnitzler, 31. 12. {[}1894{]}}
\nopagebreak\mylabel{L02630v}
\rehead{ }\normalsize\beginnumbering\briefempfaengerindex{Schnitzler, Arthur@\textsc{Schnitzler, Arthur}!zzzGoldmann, Paul@\emph{von Paul Goldmann}!1894-12-312@{31. 12. {[}1894{]}}|(be}
\toendnotes[C]{\smallbreak\pagebreak[2]}\Standort{DLA, A:Schnitzler, HS.NZ85.1.3164.}
\physDesc{Brief, 3 Blätter, 11 Seiten, 5140 Zeichen
\newline{}Handschrift: schwarze Tinte, deutsche Kurrent
\newline{}Schnitzler: 1) mit Bleistift auf dem ersten Blatt die Jahreszahl »94« vermerkt  2) mit rotem Buntstift sieben Unterstreichungen}\toendnotes[C]{\smallbreak}
\pstart
           {\pb}\textcolor{gray}{\textbf{Frankfurter Zeitung\orgindex{Frankfurter Zeitung@Frankfurter Zeitung|pw}}}\hfill \textsc{Paris\oindex{Paris@\textbf{Paris}, \emph{P.PPLC}|pw}}, 31. December.\pend
           
\pstart
           \textcolor{gray}{\textbf{(Gazette de
                     Francfort\orgindex{Frankfurter Zeitung@Frankfurter Zeitung|pw}).}}\pend
           
\pstart
           \textcolor{gray}{\textbf{Fondateur \textbf{M. L. Sonnemann\pwindex{Sonnemann, Leopold 1831-10-29 – 1909-10-30@\textsc{Sonnemann, Leopold} (1831-10-29 – 1909-10-30), \emph{Journalist/Journalistin, Herausgeber/Herausgeberin}|pw}}.}}\pend
           
\pstart
           \textcolor{gray}{\textbf{\begin{otherlanguage}{french}Journal politique, financier,\end{otherlanguage}}}\pend
           
\pstart
           \textcolor{gray}{\textbf{\begin{otherlanguage}{french}commercial et littéraire.\end{otherlanguage}}}\pend
           
\pstart
           \textcolor{gray}{\textbf{\begin{otherlanguage}{french}\textbf{Paraissant trois fois par jour.}\end{otherlanguage}}}\pend
           
\pstart
           \textcolor{gray}{\textbf{\begin{otherlanguage}{french}\textbf{Bureau à Paris\oindex{Paris@\textbf{Paris}, \emph{P.PPLC}|pw}:}\end{otherlanguage}}}\pend
           
\pstart
           \textcolor{gray}{\textbf{\begin{otherlanguage}{french}24. Rue Feydeau\oindex{rue Feydeau@\textbf{rue Feydeau}, \emph{Straße (K.STR)}|pw}.\end{otherlanguage}}}\pend
           
\pstart{}Mein lieber Freund,\pend\vspace{0.5em}
\pstart
           das ſind recht erfreuliche Nachrichten, – unberufen! – die Dein Brief bringt. \label{K_L02630-1v}\edtext{\textsc{Speidel\pwindex{Speidel, Ludwig 1830-04-11 – 1906-02-03@\textsc{Speidel, Ludwig} (1830-04-11 – 1906-02-03), \emph{Journalist/Journalistin, Kritiker/Kritikerin}|pw}}}{\lemma{\textnormal{\emph{Speidel}}}\Cendnote{\textnormal{Zum positiven Urteil Ludwig Speidels\pwindex{Speidel, Ludwig 1830-04-11 – 1906-02-03@\textsc{Speidel, Ludwig} (1830-04-11 – 1906-02-03), \emph{Journalist/Journalistin, Kritiker/Kritikerin}|pwk} über \emph{Liebelei}\pwindex{Liebelei. Schauspiel in drei Akten@\emph{Liebelei. Schauspiel in drei Akten}|pwk}{ }vgl. A. S.: \emph{Tagebuch}, 14. 12. 1894, 17. 12. 1894 und 18. 12. 1894.
               }}}\label{K_L02630-1} beſonders iſt eine förmliche Überraſchung. Der Mann, der \substVorne{}\textsuperscript{\textcolor{gray}{×}\-\textcolor{gray}{×}}\substDazwischen{}bei\substHinten{} der Lampe nach Mitternacht über Deinem Stücke\pwindex{Liebelei. Schauspiel in drei Akten@\emph{Liebelei. Schauspiel in drei Akten}|pwv} ſitzt, wird mir beinahe ſympathiſch. \strikeout{H} Sollten wir ihm vielleicht Unrecht gethan haben? Er
               war gegen das Neue; aber hat es denn viel Neues gegeben? Und haben wir nicht am Ende
               das Neue mit uns verwechſelt, die wir neu waren? Das Urtheil, das er über Dich fällt,
               ſpricht ſehr zu Ehren {\pb}ſeines Kunſtverſtändniſſes.
               Nun kann es doch unmöglich mehr fehlen. Wo ſoviel Mächtige dafür ſind, wird das
               Theater-Geſindel nichts mehr ausrichten können. Daß B.\pwindex{Burckhard, Max Eugen 14.07.1854 – 16.03.1912@\textsc{Burckhard, Max Eugen} (14.07.1854 – 16.03.1912), \emph{Schriftsteller/Schriftstellerin, Rechtswissenschaftler/Rechtswissenschaftlerin, Theaterleiter/Theaterleiterin}|pw} Dich \label{K_L02630-2v}\edtext{beſucht}{\lemma{\textnormal{\emph{beſucht}}}\Cendnote{\textnormal{Vgl. A. S.: \emph{Tagebuch}, 18. 12. 1894.
               }}}\label{K_L02630-2}, imponirt mir beſonders. Welchen Weg haſt Du durchlaufen \strikeout{zwiſchen} von drei Jahren bis auf heut! Mir
               kommt ſo vor, als ſei jetzt nur noch ein tüchtiger Ruck zu geben, und dann am Ziel!
               Wenn ſich die \textsc{Sandrock\pwindex{Sandrock, Adele 1863-08-19 – 1937-08-30@\textsc{Sandrock, Adele} (1863-08-19 – 1937-08-30), \emph{Schauspieler/Schauspielerin}|pw}} vom \label{K_L02630-3v}\edtext{Volkstheater\orgindex{Volkstheater@Volkstheater|pw} jetzt ſchon losmachen}{\lemma{\textnormal{\emph{Volkstheater … losmachen}}}\Cendnote{\textnormal{Der Wechsel von Adele Sandrock\pwindex{Sandrock, Adele 1863-08-19 – 1937-08-30@\textsc{Sandrock, Adele} (1863-08-19 – 1937-08-30), \emph{Schauspieler/Schauspielerin}|pwk} ans \emph{Burgtheater}\orgindex{Burgtheater@Burgtheater|pwk} war schon im Sommer 1894 für die
                  Saison 1895/1896 fixiert worden. Durch neuerliche
                  Verhandlungen fand der Übertritt bereits zum 1. 2. 1895 statt. Sie
                  war also in Erwartung ihrer Verfügbarkeit für die Rolle der Christine\pwindex{Liebelei. Schauspiel in drei Akten@\emph{Liebelei. Schauspiel in drei Akten}|pwkv} vorgesehen.}}}\label{K_L02630-3} könnte, ſo
               wäre es wohl gut (Warum ſpielt übrigens die \textsc{Hohenfels\pwindex{Hohenfels, Stella 16.04.1857 – 21.02.1920@\textsc{Hohenfels, Stella} (16.04.1857 – 21.02.1920), \emph{Schauspieler/Schauspielerin}|pw}} nicht die Rolle?). Wenn nicht, ſo warteſt Du ruhig bis zum nächſten Jahr. Der
                  \label{K_L02630-4v}\edtext{Titel »Liebelei\pwindex{Liebelei. Schauspiel in drei Akten@\emph{Liebelei. Schauspiel in drei Akten}|pw}« mißfällt}{\lemma{\textnormal{\emph{Titel … mißfällt}}}\Cendnote{\textnormal{Ein Erfolgsstück des Jahres 1893 war \emph{Das arme Mädel}\pwindex{arme Maedel@\emph{Das arme Mädel}|pwk} von Karl Lindau\pwindex{Lindau, Karl 26.11.1853 – 15.01.1934@\textsc{Lindau, Karl} (26.11.1853 – 15.01.1934), \emph{Schriftsteller/Schriftstellerin, Schauspieler/Schauspielerin, Librettist/Librettistin}|pwk} und
                     Leopold Krenn\pwindex{Krenn, Leopold 06.12.1850 – 02.10.1930@\textsc{Krenn, Leopold} (06.12.1850 – 02.10.1930), \emph{Schriftsteller/Schriftstellerin, Beamter/Beamte}|pwk}. Das dürfte Schnitzler gezwungen haben, Ersatz für seinen
                  Arbeitstitel »Armes Mädl« zu suchen, mit dem Goldmann\pwindex{Goldmann, Paul 31.01.1865 – 25.09.1935@\textsc{Goldmann, Paul} (31.01.1865 – 25.09.1935), \emph{Schriftsteller/Schriftstellerin, Journalist/Journalistin}|pwk} bis dahin vertraut war. In einem Interview nahm Schnitzler{ }1912 dazu Stellung: »Hätte das Stück nicht den Titel ›Liebelei\pwindex{Liebelei. Schauspiel in drei Akten@\emph{Liebelei. Schauspiel in drei Akten}|pw}‹, also die Bezeichnung für das
                     leichte, flüchtige Gefühl bar jeder Verantwortung, das ein junger Mann hegt,
                     dem ein sorgenbelastetes, ernsthaft verliebtes Mädchen gegenübersteht, sondern
                     hieße, sagen wir ›Die große Liebe der Christine‹ – also die Bezeichnung für das
                     Gefühl des Mädchens –, so hätte das Publikum dem Stück ganz gewiß nicht
                     dasselbe Interesse entgegengebracht, wie beim Titel ›Liebelei\pwindex{Liebelei. Schauspiel in drei Akten@\emph{Liebelei. Schauspiel in drei Akten}|pw}‹.« (Ifj. B. Gy [ = Georg Ruttkay]\pwindex{Ruttkay, Georg 01.06.1890 – 18.10.1955@\textsc{Ruttkay, Georg} (01.06.1890 – 18.10.1955), \emph{Schriftsteller/Schriftstellerin, Journalist/Journalistin}|pwk}: \emph{Schnitzler Arthurnál}. In: \emph{Az est}\pwindex{Az Est@\emph{Az Est}|pwk}, Jg. 3, Nr. 112, 10. 5. 1912, S. 8. Siehe A. S.: \emph{»Das Zeitlose ist von kürzester Dauer«}, Ifj. B. Gy [= Georg Ruttkay]: Schnitzler Arthurnál, 10. 5. 1912);
                     siehe Paul Goldmann an Arthur Schnitzler, 5. 1. [1895].
               }}}\label{K_L02630-4} mir. {\pb}Er klingt maniriert, unliterariſch und
               verkleinert die Arbeit. Ich möchte, daß Du auf die kleine \textsc{Nuance} verzichteſt und einfach gerade heraus »Eine Liebſchaft« ſagſt. Das
               klingt mehr nach bürgerlichem Drama. Und nun werde ich endlich ungeduldig. Alle Welt
               hat ſchon über dem Stücke\pwindex{Liebelei. Schauspiel in drei Akten@\emph{Liebelei. Schauspiel in drei Akten}|pwv}
               geſeſſen, mit \strikeout{B} Bangen und ohne. Ich weiß allerlei
               Urtheile und kenne es ſelber noch nicht. Könnteſt Du es mir nicht auf wenige Tage
               zugänglich machen? Ich leſe es in einem Tage aus und ſchicke es ſofort zurück. Bitte,
               bitte, mach’ es irgendwie möglich; Du kannſt Dir denken, wie geſpannt {\pb}ich bin. Die Spannung wächſt mit jeder neuen
               Nachricht. Nun muß ichs endlich kennen lernen, zum Teufel auch! Und, nicht wahr,
               ſobald Cenſur und Intendanz geſprochen haben, theilſt Du mir ſofort das Reſultat mit?
               Schreib’ mir auch, ob die Frankf. Ztg.\orgindex{Frankfurter Zeitung@Frankfurter Zeitung|pw} etwas
               darüber bringen ſoll. Einſtweilen beglückwünſche ich Dich von Herzen zu den
               bisherigen guten Reſultaten{[}.{]}{ }\textsc{Speidel\pwindex{Speidel, Ludwig 1830-04-11 – 1906-02-03@\textsc{Speidel, Ludwig} (1830-04-11 – 1906-02-03), \emph{Journalist/Journalistin, Kritiker/Kritikerin}|pw}} iſt bereits der halbe Erfolg. Ich freue mich ſehr{\dotsfive}\pend
           
\pstart
           In einem der nächſten Hefte des »\textsc{Mercure de France\orgindex{Mercure de France@Mercure de France|pw}}« kommt ein \label{K_L02630-5v}\edtext{Aufſatz\pwindex{Jeunes Viennois@\emph{Les Jeunes Viennois}|pwv}}{\lemma{\textnormal{\emph{Aufſatz}}}\Cendnote{\textnormal{Der Text erschien mit einer gewissen
                  Verzögerung in einer anderen Zeitschrift: Henri Albert\pwindex{Albert, Henri 1869-11-16 – 1921-08-03@\textsc{Albert, Henri} (1869-11-16 – 1921-08-03), \emph{Journalist/Journalistin, Kritiker/Kritikerin, Übersetzer/Übersetzerin}|pwk}: \emph{Les Jeunes Viennois}\pwindex{Jeunes Viennois@\emph{Les Jeunes Viennois}|pwk}. In: \emph{Revue des revues}\pwindex{Revue des Revues@\emph{Revue des Revues}|pwk}, Bd. 13, 1. 4. 1895,
                     S. 8–13.}}}\label{K_L02630-5} von \textsc{Albert\pwindex{Albert, Henri 1869-11-16 – 1921-08-03@\textsc{Albert, Henri} (1869-11-16 – 1921-08-03), \emph{Journalist/Journalistin, Kritiker/Kritikerin, Übersetzer/Übersetzerin}|pw}} über Euch. Leider hat er mich nicht um Rath {\pb}beim Schreiben gefragt. Es ſtehen alſo offenbar einige Stiefel drin. Aber die
               Hauptſache iſt doch, daß etwas geſchrieben wird. Auch will er nächſtens etwas von Dir
               überſetzen. Wie macht ſich der literariſche und buchhändleriſche Erfolg von »Sterben\pwindex{Sterben. Novelle@\emph{Sterben. Novelle}|pw}«?\pend
           
\pstart
           Was hört man von der »Zeit\orgindex{Zeit. Wiener Wochenschrift@Die Zeit. Wiener Wochenschrift|pw}«? Wie geht ſie und wie
               gefällt ſie?\pend
           
\pstart
           Gern will ich Dir die Frankf. Ztg.\orgindex{Frankfurter Zeitung@Frankfurter Zeitung|pw} ſchicken, wenn
               ich etwas darin habe. Aber ich habe kaum mehr etwas drin. Kann {\pb}mich nicht mehr zum Schreiben aufraffen. Es liegen
               Centnerlaſten auf mir. Die Krankheit, die nicht heilen will – Ihr Ärzte ſeid nichts
               als menſchenfreundliche Lügner – die Vereinſamung, die Heimatloſigkeit, das Gefühl
               des Zurückbleibens, die Verlotterung. Wie ich aus \textsc{Ischl\oindex{Bad Ischl@\textbf{Bad Ischl}, \emph{P.PPL}|pw}} zurückkam, wollte ich eine Rieſen-Anſtrengung machen. Die iſt mißlungen, und
               nun laſſe ich mich ſinken und leiſte nur mehr wenig Widerſtand. Ich leſe nicht ein
               Mal mehr ein Buch zu Ende; und wenn die Reue kommt, ſo ſlüchte ich mich in Politik
               und Depeſchen hinein.\pend
           
\pstart
           {\pb}Den Brief an Frl. \textsc{Sandrock\pwindex{Sandrock, Adele 1863-08-19 – 1937-08-30@\textsc{Sandrock, Adele} (1863-08-19 – 1937-08-30), \emph{Schauspieler/Schauspielerin}|pw}} habe ich endlich geſchrieben. Es war keine Kleinigkeit. Ich ſollte meine
               Anſicht über das Leben mittheilen. Das iſt nicht leicht, wenn man viel zu thun hat.
               Ich habe ein idiotiſches Zeug abgeſchickt, \textsc{\label{K_L02630-6v}\edtext{mais enfin}{\lemma{\textnormal{\emph{mais enfin}}}\Cendnote{\textnormal{französisch: aber
                     zuletzt}}}\label{K_L02630-6}}, ich habe geantwortet.\pend
           
\pstart
           Ich möchte ein wenig wiſſen, wie Du lebſt? Geſellſchaft? Freundſchaſt? Abenteuer?\pend
           
\pstart
           \label{K_L02630-7v}\edtext{\textsc{Bahr\pwindex{Bahr, Hermann 19.07.1863 – 15.01.1934@\textsc{Bahr, Hermann} (19.07.1863 – 15.01.1934), \emph{Schriftsteller/Schriftstellerin, Kritiker/Kritikerin}|pw}} hat mich neulich in ſehr liebenswürdiger Weiſe citirt}{\lemma{\textnormal{\emph{Bahr … citirt}}}\Cendnote{\textnormal{Sein Text\pwindex{Camille Mauclair@\emph{Camille Mauclair}|pwkv} beginnt mit: »Als ich diesen Mai
                     in Paris\oindex{Paris@\textbf{Paris}, \emph{P.PPLC}|pw} mit Paul Goldmann\pwindex{Goldmann, Paul 31.01.1865 – 25.09.1935@\textsc{Goldmann, Paul} (31.01.1865 – 25.09.1935), \emph{Schriftsteller/Schriftstellerin, Journalist/Journalistin}|pw}, dem Correspondenten\pwindex{Goldmann, Paul 31.01.1865 – 25.09.1935@\textsc{Goldmann, Paul} (31.01.1865 – 25.09.1935), \emph{Schriftsteller/Schriftstellerin, Journalist/Journalistin}|pwv} der Frankfurter Zeitung\orgindex{Frankfurter Zeitung@Frankfurter Zeitung|pw}, plauderte und um jeden Preis ein neues Talent
                     wissen wollte, sagte er mir: ›Ein Talent? Ein neues Talent? Ein ernstes,
                     sicheres, wirkliches Talent? Nicht bloß so eine geschwinde und vergängliche
                     Erfindung der Journale von heute auf morgen? Das ist schwer. Da ist jetzt wohl
                     niemand als Camille Mauclair\pwindex{Mauclair, Camille 1872-11-29 – 1945-04-23@\textsc{Mauclair, Camille} (1872-11-29 – 1945-04-23), \emph{Schriftsteller/Schriftstellerin}|pw}. Sonst wüßte
                     ich keinen. Er hat freilich eigentlich noch nichts geschrieben; aber alle
                     hoffen viel von ihm. Er verspricht mehr, als er bis jetzt gehalten hätte; aber
                     er scheint mir sicher. Stellen Sie sich etwa, ins Paris\oindex{Paris@\textbf{Paris}, \emph{P.PPLC}|pw}erische übersetzt, Ihren kleinen Hofmannsthal\pwindex{Hofmannsthal, Hugo von 1874-02-01 – 1929-07-15@\textsc{Hofmannsthal, Hugo von} (1874-02-01 – 1929-07-15), \emph{Schriftsteller/Schriftstellerin}|pw} vor.‹« (Hermann Bahr\pwindex{Bahr, Hermann 19.07.1863 – 15.01.1934@\textsc{Bahr, Hermann} (19.07.1863 – 15.01.1934), \emph{Schriftsteller/Schriftstellerin, Kritiker/Kritikerin}|pwk}: \emph{Camille Mauclair}\pwindex{Camille Mauclair@\emph{Camille Mauclair}|pwk}. In: \emph{Die Zeit}\pwindex{Zeit. Wiener Wochenschrift@\emph{Die Zeit. Wiener Wochenschrift}|pwk}, Bd. 1, H. 10, 8. 12. 1894,
                  S. 154–155.)}}}\label{K_L02630-7}. Warum hat er das gethan?\pend
           
\pstart
           Ich mache mir Vorwürſe, daß ich Dich zum Abonnement auf das {\pb}»Journal\orgindex{Le Journal@Le Journal|pwv}« aufgefordert habe. Es wird niederträchtig ſchlecht. Vielleicht
               verſuchſt Du es fortan mit der Abendausgabe des »\textsc{Journal des Débats\orgindex{Journal des debats@Journal des débats|pw}}«. Die politiſchen Artikel brauchſt Du ja nicht zu leſen; aber es ſind köſtliche
                  \textsc{\label{K_L02630-8v}\edtext{chroniqueurs}{\lemma{\textnormal{\emph{chroniqueurs}}}\Cendnote{\textnormal{französisch: Kolumnisten}}}\label{K_L02630-8}} darin, höhere literariſche Leute: \textsc{Hallays\pwindex{Hallays, Andre 1859-03-16 – 1930-03-30@\textsc{Hallays, André} (1859-03-16 – 1930-03-30), \emph{Journalist/Journalistin, Kunstkritiker/Kunstkritikerin, Jurist/Juristin}|pw}, Bazin\pwindex{Bazin, Rene 1853-12-26 – 1932-07-20@\textsc{Bazin, René} (1853-12-26 – 1932-07-20), \emph{Schriftsteller/Schriftstellerin}|pw}, Filon\pwindex{Filon, Augustin 1841-11-28 – 1916-06-13@\textsc{Filon, Augustin} (1841-11-28 – 1916-06-13), \emph{Schriftsteller/Schriftstellerin}|pw}, Lemaître\pwindex{Lemaître, Jules 1853-04-27 – 1914-08-04@\textsc{Lemaître, Jules} (1853-04-27 – 1914-08-04), \emph{Schriftsteller/Schriftstellerin, Librettist/Librettistin}|pw}{ }}\textsc{etc}. Willſt Du, daß ichs Dir abonnire? Noch habe ich \textsc{30 Francs 30 ct.}, die Du beharrlich todtſchweigſt. Hat \textsc{Richard\pwindex{Beer-Hofmann, Richard 1866-07-11 – 1945-09-26@\textsc{Beer-Hofmann, Richard} (1866-07-11 – 1945-09-26), \emph{Schriftsteller/Schriftstellerin}|pw}} den »\label{K_L02630-9v}\edtext{Courrier Français\pwindex{Le Courrier français@\emph{Le Courrier français}|pw}}{\lemma{\textnormal{\emph{Courrier Français}}}\Cendnote{\textnormal{illustrierte Satirezeitschrift\pwindex{Le Courrier français@\emph{Le Courrier français}|pwkv}, die zwischen
                     1884 und 1914 erschien}}}\label{K_L02630-9}« abonnirt? Sonſt
               ſchicke ich ihn Dir. Anbei ſchicke ich Dir wieder ein paar \label{K_L02630-10v}\edtext{Artikel}{\lemma{\textnormal{\emph{Artikel}}}\Cendnote{\textnormal{Die Beilagen sind nicht
               überliefert.}}}\label{K_L02630-10}, Kraut und Rüben durcheinander. \textsc{Drumont\pwindex{Drumont, Edouard 1844-05-03 – 1917-02-05@\textsc{Drumont, Édouard} (1844-05-03 – 1917-02-05), \emph{Journalist/Journalistin, Rassentheoretiker/Rassentheoretikerin}|pw}} iſt ein großer {\pb}Polemiſt, nur ſtark irrſinnig.
               In Bezug auf Juden und Deutſche leidet er an Verfolgungswahn. Aber in erſterer
               Beziehung beginnt der Irrſinn doch erſt nach einer weiten Grenze; Vieles
               Unglaubliche, was er über jüdiſche Corruption ſchreibt, iſt wahr. Auch iſt er
               größenwahnſinnig und kommt ſich thatſächlich als gottgeſandter Meſſias vor.
               Anderſeits gibt ihm aber gerade nur dieſer Wahnſinn die ungeheure Kraft, mit der er
               manchmal dreinſchlägt.\pend
           
\pstart
           {\pb}\textsc{Sokal\pwindex{Sokal, Clemens *~21.01.1867@\textsc{Sokal, Clemens} (*~21.01.1867), \emph{Journalist/Journalistin, Rechtsanwalt/Rechtsanwältin}|pw}} war bei mir; er gefällt mir gut. Scheint ein geſcheiter und ernſter Menſch zu
                  ſein{\dotsfour}\pend
           
\pstart
           Ich wünſche Dir von Herzen Glück zum neuen Jahr. Mir ahnt, daß das Jahr
                  1895 wichtig für Dich werden wird. Sieht es nicht vertrauenerweckend
               aus? Mit ſeiner runden Fünfheiten!\pend
           
\pstart
           Was aber auch geſchehen mag, Gutes oder Allerbeſtes, wir bleiben die Alten, nicht
               wahr?\pend
           
\pstart
           Herzlichſt und in Treue Dein{\\[\baselineskip]}\spacefill\mbox{Paul Goldmann\textcolor{gray}{.}}\pend
           \leftskip=0em{}
\pstart
           \noindent{}{\pb}Bitte, empfiehl’ mich Deiner Frau Mutter\pwindex{Schnitzler, Louise 1840-07-08 – 1911-09-09@\textsc{Schnitzler, Louise} (1840-07-08 – 1911-09-09)|pwv} und richte \label{T_L02630-1v}\edtext{ihr}{\lemma{\textnormal{\emph{ihr}}}\Cendnote{\textnormal{Goldmann schreibt
                        »Ihr«.}}}\label{T_L02630-1} meine ergebenſten Neujahrs-Wünſche aus.\pend
           
\pstart
           Was lieſt Du jetzt?\pend
           \selectlanguage{ngerman}\endnumbering\briefempfaengerindex{Schnitzler, Arthur@\textsc{Schnitzler, Arthur}!zzzGoldmann, Paul@\emph{von Paul Goldmann}!1894-12-312@{31. 12. {[}1894{]}}|)be}\mylabel{L02630h}  \normalsize

\doendnotes{C}
\bigskip
\vfill

\clearpage

\footnotesize

\lohead{\textsc{register}}

% Definiere theindex-Environment komplett neu ohne reledmac
\makeatletter
\renewenvironment{theindex}{%
  \section*{\indexname}%
  \setlength{\parindent}{0pt}%
  \setlength{\parskip}{0pt plus 0.3pt}%
  \let\item\@idxitem
}{%
  \clearpage
}
\makeatother

\IfFileExists{\jobname-pw.ind}{\input{\jobname-pw.ind}}{}

\end{document}

      