%% latex-leseansicht-vorspann.tex
%% Vorspann für die Leseansicht.
%% Lädt die gemeinsame Datei latex-vorspann.tex mit nicht gesetztem Schalter.

\newif\ifkorrekturansicht
\korrekturansichtfalse

\input{../tex-inputs/latex-vorspann}


         
         \renewcommand{\erwaehntePersonen}{Personen: Richard Beer-Hofmann, Theodor Entsch, Karl Kraus, Gilbert Otto Neumann-Hofer, Fedor von Zobeltitz}
         \renewcommand{\erwaehnteInstitutionen}{Institutionen: Lessing-Theater, Magazin für die Literatur des Auslandes}
         \renewcommand{\erwaehnteOrte}{Orte: Bad Ischl, Ramsauer Garni Café, Wien}
         \renewcommand{\erwaehnteWerke}{Werke: Abschiedssouper, Anatol, Camelias, Das Kind, Das Märchen. Schauspiel in drei Aufzügen, Magazin für die Literatur des Auslandes, Novellen, Ohne Geläut, Wiener Dichter, [Meldung: Märchen im Lessingtheater]}
               \section[Karl Kraus an Arthur Schnitzler, 12. 8. 1893]{ Karl Kraus an Arthur Schnitzler, 12. 8. 1893}\nopagebreak\mylabel{v}\rehead{ }\begin{ledgroupsized}[t]{13cm}\normalsize\beginnumbering\briefempfaengerindex{Schnitzler, Arthur@\textsc{Schnitzler, Arthur}!zzzKraus, Karl@\emph{von Karl Kraus}!1893-08-124@{12. 8. 1893}|(be} \toendnotes[C]{\smallbreak\pagebreak[2]} \Standort{CUL, Schnitzler, B 55.}
\physDesc{Brief, 1 Blatt, 4 Seiten, 1877 Zeichen
\newline{}Handschrift: schwarze Tinte, deutsche Kurrent}\buchAbdrucke{\weitereDrucke{\emph{Karl Kraus und Arthur Schnitzler. Eine Dokumentation.} Hg. Reinhard Urbach. In: \emph{Literatur und Kritik}, Bd. 49, Oktober 1970, S. 520.} }\toendnotes[C]{\smallbreak}\pstart
           {\pb}Ischl, Ramsauer\oindex{Ramsauer Garni Cafe@\textbf{Ramsauer Garni Café}|pw},
                  12. 8. 93.\pend
           \pstart
           Liebſter Doktor! Eben holte ich mir von der Post den Brief u. beeile
               mich, Ihnen auf Ihr Schreiben zu antworten: ich bin über die Auskunft des Herrn Entſch\pwindex{Entsch, Theodor 17.01.1853 – 19.12.1913@\textsc{Entsch, Theodor} (17.01.1853 – 19.12.1913), \emph{Verleger, Theateragent}|pw} ganz paff – es iſt mir \uline{nie im Traume eingefallen}, dem Magazin\orgindex{Magazin fuer die Literatur des Auslandes@Magazin für die Literatur des Auslandes|pw} eine derartige aus der Luft gegriffene \label{K_L00255-1v}\edtext{Mittheilung\pwindex{?? Werk@Nicht ermittelte Verfasserinnen und Verfasser!Meldung: Maerchen im Lessingtheater]22. 07. 1893@\emph{[Meldung: Märchen im Lessingtheater]} {[}22. 07. 1893{]}|pwv}}{\lemma{\textnormal{\emph{Mittheilung}}}\Cendnote{\textnormal{Auf S. 469 der Nr. 29 vom
                     22. 7. 1893{ }stand: »Am Lessingtheater\orgindex{Lessing-Theater@Lessing-Theater|pw} kommen ferner noch im Laufe des Sommers ein Drama von
                        \so{Fedor von Zobeltitz}\pwindex{Zobeltitz, Fedor von 05.10.1857 – 10.02.1934@\textsc{Zobeltitz, Fedor von} (05.10.1857 – 10.02.1934), \emph{Schriftsteller}|pw}: ›Ohne Geläut\pwindex{Zobeltitz, Fedor von 05.10.1857 – 10.02.1934@\textsc{Zobeltitz, Fedor von} (05.10.1857 – 10.02.1934), \emph{Schriftsteller}!Ohne Gelaeut1894@\strich\emph{Ohne Geläut} {[}1894{]}|pw}‹ und ein dreiaktiges
                     Schauspiel von \textsc{Dr}. Arthur \so{Schnitzler}\pwindex{Schnitzler, Arthur 15.05.1862 – 21.10.1931@\textsc{Schnitzler, Arthur} (15.05.1862 – 21.10.1931), \emph{Schriftsteller, Mediziner}|pw} in Wien\oindex{Wien@\textbf{Wien}|pw}: ›Das Märchen\pwindex{Schnitzler, Arthur 15.05.1862 – 21.10.1931@\textsc{Schnitzler, Arthur} (15.05.1862 – 21.10.1931), \emph{Schriftsteller, Mediziner}!Maerchen. Schauspiel in drei Aufzuegen1893-12-01@\strich\emph{Das Märchen. Schauspiel in drei Aufzügen} {[}1893-12-01{]}|pw}‹, zur Aufführung.«}}}\label{K_L00255-1h} zu machen –
               das wäre dann eine höchſt abgeſchmackte Fopperei \introOben{}von\introOben{} meiner
               Seite geweſen, wenn ich Ihnen dann »freudig \uline{überraſcht}« das Blatt ſchicken konnte: »Sehen Sie, da ſteht was über das
                  ›Märchen\pwindex{Schnitzler, Arthur 15.05.1862 – 21.10.1931@\textsc{Schnitzler, Arthur} (15.05.1862 – 21.10.1931), \emph{Schriftsteller, Mediziner}!Maerchen. Schauspiel in drei Aufzuegen1893-12-01@\strich\emph{Das Märchen. Schauspiel in drei Aufzügen} {[}1893-12-01{]}|pw}‹ drin!« Wie geſagt, liebſter Herr
               Doktor, \uuline{nie und nimmer} würde mir ſoetwas einfallen,
               ich habe \uuline{nie} (Sie wiſſen ja, bei {\pb}Abſchiedssouper\pwindex{Schnitzler, Arthur 15.05.1862 – 21.10.1931@\textsc{Schnitzler, Arthur} (15.05.1862 – 21.10.1931), \emph{Schriftsteller, Mediziner}!Abschiedssouper1892@\strich\emph{Abschiedssouper} {[}1892{]}|pw} habe ich Sie zu erst brieflich
               befragt) Herrn Neumann-Hofer\pwindex{Neumann-Hofer, Gilbert Otto 04.02.1857 – 14.04.1941@\textsc{Neumann-Hofer, Gilbert Otto} (04.02.1857 – 14.04.1941), \emph{Kritiker, Theaterleiter}|pw} den
               Aufführungstermin Ihres Märchen\pwindex{Schnitzler, Arthur 15.05.1862 – 21.10.1931@\textsc{Schnitzler, Arthur} (15.05.1862 – 21.10.1931), \emph{Schriftsteller, Mediziner}!Maerchen. Schauspiel in drei Aufzuegen1893-12-01@\strich\emph{Das Märchen. Schauspiel in drei Aufzügen} {[}1893-12-01{]}|pw} geſchrieben: das
               wäre doch meinerſeits eine recht ungeſchickte Reklame für Sie geweſen. Das Ganze muſs
               unbedingt auf einem \uline{Irrthum} beruhen, vielleicht
               erklärt es ſich daraus, daſs ich einmal – Sie haben’s ja geleſen – im Magazin\pwindex{?? Werk@Nicht ermittelte Verfasserinnen und Verfasser!Magazin fuer die Literatur des Auslandes1832 – 1915@\emph{Magazin für die Literatur des Auslandes} {[}1832 – 1915{]}|pw} gelegentlich der Anatol\pwindex{Schnitzler, Arthur 15.05.1862 – 21.10.1931@\textsc{Schnitzler, Arthur} (15.05.1862 – 21.10.1931), \emph{Schriftsteller, Mediziner}!Anatol1892-10-29@\strich\emph{Anatol} {[}1892-10-29{]}|pw}-recenſion\pwindex{Kraus, Karl 28.04.1874 – 12.06.1936@\textsc{Kraus, Karl} (28.04.1874 – 12.06.1936), \emph{Schriftsteller, Publizist}!Wiener Dichter06. 05. 1893@\strich\emph{Wiener Dichter} {[}06. 05. 1893{]}|pwv} auch Ihr \uline{Märchen}\pwindex{Schnitzler, Arthur 15.05.1862 – 21.10.1931@\textsc{Schnitzler, Arthur} (15.05.1862 – 21.10.1931), \emph{Schriftsteller, Mediziner}!Maerchen. Schauspiel in drei Aufzuegen1893-12-01@\strich\emph{Das Märchen. Schauspiel in drei Aufzügen} {[}1893-12-01{]}|pw} als beachtenswertes Schauspiel erwähnte.\pend
           \pstart
           Mir iſt die ganze Sache \uuline{ſehr peinlich}, glauben Sie
               mir! {\pb}Jawohl, wenn Sie mir ſelbſt den \strikeout{I} Inhalt dieſer vielbeſprochenen Märchen\pwindex{Schnitzler, Arthur 15.05.1862 – 21.10.1931@\textsc{Schnitzler, Arthur} (15.05.1862 – 21.10.1931), \emph{Schriftsteller, Mediziner}!Maerchen. Schauspiel in drei Aufzuegen1893-12-01@\strich\emph{Das Märchen. Schauspiel in drei Aufzügen} {[}1893-12-01{]}|pw}notiz\pwindex{?? Werk@Nicht ermittelte Verfasserinnen und Verfasser!Meldung: Maerchen im Lessingtheater]22. 07. 1893@\emph{[Meldung: Märchen im Lessingtheater]} {[}22. 07. 1893{]}|pwv} geſagt hätten, mit
               Vergnügen \uuline{hätte} ich, um Ihnen zu dienen, dem Magazin\orgindex{Magazin fuer die Literatur des Auslandes@Magazin für die Literatur des Auslandes|pw} die Notiz mitgetheilt – aber ſo – wie
               werde ich ſo plump ſein, ſo etwas aus der Luft zu greifen oder aus dem Finger zu
               zutzeln und dann Ihnen das Heft mit »freudig–überraschter« Miene noch zu\introOben{}zu\introOben{}senden? Ich bitte Sie, mir nicht böſe zu ſein, daſs ich
               Ihnen (\uuline{unverſchuldet}!) ſolche Unannehmlichkeiten
               bereite – aber mich ſelbſt {\pb}berührt die
               Angelegenheit noch \uuline{viel} unangenehmer. \uuline{Selbſtverſtändlich}{ }ſchreibe ich ſofort dem Magazin\orgindex{Magazin fuer die Literatur des Auslandes@Magazin für die Literatur des Auslandes|pw} u. erſuche um Aufklärung. Der \introOben{}Entſch\pwindex{Entsch, Theodor 17.01.1853 – 19.12.1913@\textsc{Entsch, Theodor} (17.01.1853 – 19.12.1913), \emph{Verleger, Theateragent}|pw}\introOben{}brief liegt bei. Ich bin mit den herzlichſten Grüßen Ihr\pend
           \pstart \spacefill\mbox{KarlKraus.}\pend{}\pstart
           \noindent{}\uuline{NB.} um von freundlicheren Sachen zu ſprechen: Beer Hofmanns\pwindex{Beer-Hofmann, Richard 1866-07-11 – 1945-09-26@\textsc{Beer-Hofmann, Richard} (1866-07-11 – 1945-09-26), \emph{Schriftsteller}|pw} »Kind\pwindex{Beer-Hofmann, Richard 1866-07-11 – 1945-09-26@\textsc{Beer-Hofmann, Richard} (1866-07-11 – 1945-09-26), \emph{Schriftsteller}!Kind1893@\strich\emph{Das Kind} {[}1893{]}|pwv}« iſt ein prächtiger, geſunder
                  Bengel. Der grauſame Vater will es – \label{K_L00255-2v}\edtext{verlegen}{\lemma{\textnormal{\emph{verlegen}}}\Cendnote{\textnormal{Richard Beer-Hofmann\pwindex{Beer-Hofmann, Richard 1866-07-11 – 1945-09-26@\textsc{Beer-Hofmann, Richard} (1866-07-11 – 1945-09-26), \emph{Schriftsteller}|pwk}: \emph{Novellen}\pwindex{Beer-Hofmann, Richard 1866-07-11 – 1945-09-26@\textsc{Beer-Hofmann, Richard} (1866-07-11 – 1945-09-26), \emph{Schriftsteller}!Novellen1. 12. 1893@\strich\emph{Novellen} {[}1. 12. 1893{]}|pwk}. Berlin: \emph{Freund {\kaufmannsund} Jeckel}{ }1893. Darin enthalten sind die Prosatexte \emph{Das Kind}\pwindex{Beer-Hofmann, Richard 1866-07-11 – 1945-09-26@\textsc{Beer-Hofmann, Richard} (1866-07-11 – 1945-09-26), \emph{Schriftsteller}!Kind1893@\strich\emph{Das Kind} {[}1893{]}|pwk} und \emph{Camelias}\pwindex{Beer-Hofmann, Richard 1866-07-11 – 1945-09-26@\textsc{Beer-Hofmann, Richard} (1866-07-11 – 1945-09-26), \emph{Schriftsteller}!Camelias1893@\strich\emph{Camelias} {[}1893{]}|pwk}. Das genaue Erscheinungsdatum war Anfang Dezember
                        1893.}}}\label{K_L00255-2h} laſſen.\pend
           
         
         \endnumbering\mylabel{h}\end{ledgroupsized}  \newcommand{\dateiname}{L00255}\newcommand{\titel}{Karl Kraus an Arthur Schnitzler, 12. 8. 1893}\newcommand{\editorInnen}{Martin Anton Müller und Gerd-Hermann Susen}%% latex-leseansicht-abspann.tex
%% Abspann für die Leseansicht.
%% Der Schalter \ifkorrekturansicht ist bereits durch den Vorspann gesetzt.

%% latex-abspann.tex
%% Gemeinsamer Abspann für Korrekturansicht und Leseansicht.
%% Setzt den Schalter \ifkorrekturansicht voraus (gesetzt in den
%% einbindenden Dateien latex-korrekturansicht-abspann.tex bzw.
%% latex-leseansicht-abspann.tex).
%% ---------------------------------------------------------------

\normalsize

% Das esempio-Environment wird nur in der Leseansicht benötigt
\ifkorrekturansicht\else
\newenvironment{esempio}[3]%
{
    \vspace{1.5ex}
    \rlap{\underline{#1}}
    \par
    \setlength{\parindent}{0cm}
    \nopagebreak
    \leftskip=#2cm
    \rightskip=#3cm
}
{
    \par
}
\fi

\doendnotes{C}
\bigskip
\vfill

\clearpage

\footnotesize

\ifkorrekturansicht
  \lohead{\textsc{register}}
\fi

% theindex-Environment neu definieren ohne reledmac
\makeatletter
\renewenvironment{theindex}{%
  \ifkorrekturansicht
    \section*{\indexname}%
  \else
    \subsubsection*{Index der erwähnten Entitäten}%
  \fi
  \setlength{\parindent}{0pt}%
  \setlength{\parskip}{0pt plus 0.3pt}%
  \let\item\@idxitem
}{%
  \ifkorrekturansicht\clearpage\fi
}
\makeatother

\IfFileExists{\jobname-pw.ind}{\input{\jobname-pw.ind}}{}

% Quellenangabe nur in der Leseansicht
\ifkorrekturansicht\else
% Fallback-Definitionen, falls die .tex-Datei \titel etc. nicht gesetzt hat
\providecommand{\titel}{}
\providecommand{\editorInnen}{}
\providecommand{\dateiname}{\jobname}

\vspace{3cm}

\vfill

\footnotesize
\textsc{Quelle}: \titel. Herausgegeben von {\editorInnen}. In: \emph{Arthur Schnitzler: Briefwechsel mit Autorinnen und Autoren}.
 Digitale Edition, https://schnitzler-briefe.acdh.oeaw.ac.at/{\dateiname}.html (Stand \today)
\fi

\end{document}


      