%% latex-korrekturansicht-vorspann.tex
%% Vorspann für die Korrekturansicht.
%% Lädt die gemeinsame Datei latex-vorspann.tex mit gesetztem Schalter.

\newif\ifkorrekturansicht
\korrekturansichttrue

\input{../tex-inputs/latex-vorspann}


\section[Karl Kraus an Arthur Schnitzler, 12. 8. 1893]{L00255 Karl Kraus an Arthur Schnitzler, 12. 8. 1893}
\nopagebreak\mylabel{L00255v}
\rehead{ }\normalsize\beginnumbering\briefempfaengerindex{Schnitzler, Arthur@\textsc{Schnitzler, Arthur}!zzzKraus, Karl@\emph{von Karl Kraus}!1893-08-124@{12. 8. 1893}|(be}
\toendnotes[C]{\smallbreak\pagebreak[2]}\Standort{CUL, Schnitzler, B 55.}
\physDesc{Brief, 1 Blatt, 4 Seiten, 1877 Zeichen
\newline{}Handschrift: schwarze Tinte, deutsche Kurrent}
\buchAbdrucke{\weitereDrucke{\emph{Literatur und Kritik}, Bd. 49, Oktober 1970, S. 520.} }\toendnotes[C]{\smallbreak}
\pstart
           {\pb}Ischl, Ramsauer\oindex{Ramsauer Garni Cafe@\textbf{Ramsauer Garni Café}, \emph{Hotel (K.HTL)}|pw},
                  12. 8. 93.\pend
           \vspace{0.5em}
\pstart
           Liebſter Doktor! Eben holte ich mir von der Post den Brief u. beeile
               mich, Ihnen auf Ihr Schreiben zu antworten: ich bin über die Auskunft des Herrn Entſch\pwindex{Entsch, Theodor 17.01.1853 – 19.12.1913@\textsc{Entsch, Theodor} (17.01.1853 – 19.12.1913), \emph{Verleger/Verlegerin, Theateragent/Theateragentin}|pw} ganz paff – es iſt mir \uline{nie im Traume eingefallen}, dem Magazin\orgindex{Magazin fuer die Literatur des Auslandes@Magazin für die Literatur des Auslandes|pw} eine derartige aus der Luft gegriffene \label{K_L00255-1v}\edtext{Mittheilung\pwindex{Meldung: Maerchen im Lessingtheater]@\emph{[Meldung: Märchen im Lessingtheater]}|pwv}}{\lemma{\textnormal{\emph{Mittheilung}}}\Cendnote{\textnormal{Auf S. 469 der Nr. 29 vom
                     22. 7. 1893{ }stand: »Am Lessingtheater\orgindex{Lessing-Theater@Lessing-Theater|pw} kommen ferner noch im Laufe des Sommers ein Drama von
                        \so{Fedor von Zobeltitz}\pwindex{Zobeltitz, Fedor von 05.10.1857 – 10.02.1934@\textsc{Zobeltitz, Fedor von} (05.10.1857 – 10.02.1934), \emph{Schriftsteller/Schriftstellerin}|pw}: ›Ohne Geläut\pwindex{Ohne Gelaeut@\emph{Ohne Geläut}|pw}‹ und ein dreiaktiges
                     Schauspiel von \textsc{Dr}. Arthur \so{Schnitzler} in Wien\oindex{Wien@\textbf{Wien}, \emph{A.ADM2}|pw}: ›Das Märchen\pwindex{Maerchen. Schauspiel in drei Aufzuegen@\emph{Das Märchen. Schauspiel in drei Aufzügen}|pw}‹, zur Aufführung.«}}}\label{K_L00255-1} zu machen –
               das wäre dann eine höchſt abgeſchmackte Fopperei \introOben{}von\introOben{} meiner
               Seite geweſen, wenn ich Ihnen dann »freudig \uline{überraſcht}« das Blatt ſchicken konnte: »Sehen Sie, da ſteht was über das
                  ›Märchen\pwindex{Maerchen. Schauspiel in drei Aufzuegen@\emph{Das Märchen. Schauspiel in drei Aufzügen}|pw}‹ drin!« Wie geſagt, liebſter Herr
               Doktor, \uuline{nie und nimmer} würde mir ſoetwas einfallen,
               ich habe \uuline{nie} (Sie wiſſen ja, bei {\pb}Abſchiedssouper\pwindex{Abschiedssouper@\emph{Abschiedssouper}|pw} habe ich Sie zu erst brieflich
               befragt) Herrn Neumann-Hofer\pwindex{Neumann-Hofer, Gilbert Otto 04.02.1857 – 14.04.1941@\textsc{Neumann-Hofer, Gilbert Otto} (04.02.1857 – 14.04.1941), \emph{Kritiker/Kritikerin, Theaterleiter/Theaterleiterin}|pw} den
               Aufführungstermin Ihres Märchen\pwindex{Maerchen. Schauspiel in drei Aufzuegen@\emph{Das Märchen. Schauspiel in drei Aufzügen}|pw} geſchrieben: das
               wäre doch meinerſeits eine recht ungeſchickte Reklame für Sie geweſen. Das Ganze muſs
               unbedingt auf einem \uline{Irrthum} beruhen, vielleicht
               erklärt es ſich daraus, daſs ich einmal – Sie haben’s ja geleſen – im Magazin\pwindex{Magazin fuer die Literatur des Auslandes@\emph{Magazin für die Literatur des Auslandes}|pw} gelegentlich der Anatol\pwindex{Anatol@\emph{Anatol}|pw}-recenſion\pwindex{Wiener Dichter@\emph{Wiener Dichter}|pwv} auch Ihr \uline{Märchen}\pwindex{Maerchen. Schauspiel in drei Aufzuegen@\emph{Das Märchen. Schauspiel in drei Aufzügen}|pw} als beachtenswertes Schauspiel erwähnte.\pend
           
\pstart
           Mir iſt die ganze Sache \uuline{ſehr peinlich}, glauben Sie
               mir! {\pb}Jawohl, wenn Sie mir ſelbſt den \strikeout{I} Inhalt dieſer vielbeſprochenen Märchen\pwindex{Maerchen. Schauspiel in drei Aufzuegen@\emph{Das Märchen. Schauspiel in drei Aufzügen}|pw}notiz\pwindex{Meldung: Maerchen im Lessingtheater]@\emph{[Meldung: Märchen im Lessingtheater]}|pwv} geſagt hätten, mit
               Vergnügen \uuline{hätte} ich, um Ihnen zu dienen, dem Magazin\orgindex{Magazin fuer die Literatur des Auslandes@Magazin für die Literatur des Auslandes|pw} die Notiz mitgetheilt – aber ſo – wie
               werde ich ſo plump ſein, ſo etwas aus der Luft zu greifen oder aus dem Finger zu
               zutzeln und dann Ihnen das Heft mit »freudig–überraschter« Miene noch zu\introOben{}zu\introOben{}senden? Ich bitte Sie, mir nicht böſe zu ſein, daſs ich
               Ihnen (\uuline{unverſchuldet}!) ſolche Unannehmlichkeiten
               bereite – aber mich ſelbſt {\pb}berührt die
               Angelegenheit noch \uuline{viel} unangenehmer. \uuline{Selbſtverſtändlich}{ }ſchreibe ich ſofort dem Magazin\orgindex{Magazin fuer die Literatur des Auslandes@Magazin für die Literatur des Auslandes|pw} u. erſuche um Aufklärung. Der \introOben{}Entſch\pwindex{Entsch, Theodor 17.01.1853 – 19.12.1913@\textsc{Entsch, Theodor} (17.01.1853 – 19.12.1913), \emph{Verleger/Verlegerin, Theateragent/Theateragentin}|pw}\introOben{}brief liegt bei. Ich bin mit den herzlichſten Grüßen Ihr\pend
           \pstart \spacefill\mbox{KarlKraus.}\pend{}
\pstart
           \noindent{}\uuline{NB.} um von freundlicheren Sachen zu ſprechen: Beer Hofmanns\pwindex{Beer-Hofmann, Richard 1866-07-11 – 1945-09-26@\textsc{Beer-Hofmann, Richard} (1866-07-11 – 1945-09-26), \emph{Schriftsteller/Schriftstellerin}|pw} »Kind\pwindex{Kind@\emph{Das Kind}|pwv}« iſt ein prächtiger, geſunder
                  Bengel. Der grauſame Vater will es – \label{K_L00255-2v}\edtext{verlegen}{\lemma{\textnormal{\emph{verlegen}}}\Cendnote{\textnormal{Richard Beer-Hofmann\pwindex{Beer-Hofmann, Richard 1866-07-11 – 1945-09-26@\textsc{Beer-Hofmann, Richard} (1866-07-11 – 1945-09-26), \emph{Schriftsteller/Schriftstellerin}|pwk}: \emph{Novellen}\pwindex{Novellen@\emph{Novellen}|pwk}. Berlin: \emph{Freund {\kaufmannsund} Jeckel}{ }1893. Darin enthalten sind die Prosatexte \emph{Das Kind}\pwindex{Kind@\emph{Das Kind}|pwk} und \emph{Camelias}\pwindex{Camelias@\emph{Camelias}|pwk}. Das genaue Erscheinungsdatum war Anfang Dezember
                        1893.}}}\label{K_L00255-2} laſſen.\pend
           \selectlanguage{ngerman}\endnumbering\briefempfaengerindex{Schnitzler, Arthur@\textsc{Schnitzler, Arthur}!zzzKraus, Karl@\emph{von Karl Kraus}!1893-08-124@{12. 8. 1893}|)be}\mylabel{L00255h}  \normalsize

\doendnotes{C}
\bigskip
\vfill

\clearpage

\footnotesize

\lohead{\textsc{register}}

% Definiere theindex-Environment komplett neu ohne reledmac
\makeatletter
\renewenvironment{theindex}{%
  \section*{\indexname}%
  \setlength{\parindent}{0pt}%
  \setlength{\parskip}{0pt plus 0.3pt}%
  \let\item\@idxitem
}{%
  \clearpage
}
\makeatother

\IfFileExists{\jobname-pw.ind}{\input{\jobname-pw.ind}}{}

\end{document}

      