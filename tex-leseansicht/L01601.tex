%% latex-leseansicht-vorspann.tex
%% Vorspann für die Leseansicht.
%% Lädt die gemeinsame Datei latex-vorspann.tex mit nicht gesetztem Schalter.

\newif\ifkorrekturansicht
\korrekturansichtfalse

\input{../tex-inputs/latex-vorspann}


         
         \renewcommand{\erwaehntePersonen}{Personen: Hermann Bahr, Felix Salten, Olga Schnitzler}
         \renewcommand{\erwaehnteOrte}{Orte: Berlin, Ober Sankt Veit, Veitlissengasse, Venedig, Wien}
         \renewcommand{\erwaehnteWerke}{Werke: Der Faun. Ein Akt}
               \section[Hermann Bahr an Arthur Schnitzler, 21. 6. 1906]{ Hermann Bahr an Arthur Schnitzler, 21. 6. 1906}\nopagebreak\mylabel{v}\rehead{ }\begin{ledgroupsized}[t]{13cm}\normalsize\beginnumbering \toendnotes[C]{\smallbreak\pagebreak[2]} \Standort{CUL, Schnitzler, B 5b.}
\physDesc{Brief, 1 Blatt, 2 Seiten, 572 Zeichen
\newline{}Handschrift: blaue Tinte, deutsche Kurrent
\newline{}Ordnung: mit Bleistift von unbekannter Hand nummeriert:
                                    »139« }\buchAbdrucke{\weitereDrucke{Hermann Bahr, Arthur Schnitzler: \emph{Briefwechsel, Aufzeichnungen, Dokumente (1891–1931)}. Hg. Kurt Ifkovits und Martin Anton Müller. Göttingen: \emph{Wallstein} 2018, S. 378–379.} }\toendnotes[C]{\smallbreak}\pstart
           \raggedleft{}{\pb}Wien XIII/\textsubscript{7}\oindex{Ober Sankt Veit@\textbf{Ober Sankt Veit}|pw}{\\}21. 6. 06\pend
           \pstart\center{}Lieber Artur!\pend\pstart
           Ich wollte immer noch zu Dir, war aber die letzte Zeit ſo gehetzt, daß es nie ging.
               Den »\label{K_L01601-1v}\edtext{Faun\pwindex{Bahr, Hermann 19.07.1863 – 15.01.1934@\textsc{Bahr, Hermann} (19.07.1863 – 15.01.1934), \emph{Schriftsteller, Kritiker}!Faun. Ein Akt1906-10-04 – 1906-11-08@\strich\emph{Der Faun. Ein Akt} {[}1906-10-04 – 1906-11-08{]}|pw}}{\lemma{\textnormal{\emph{Faun}}}\Cendnote{\textnormal{fertiggestellt am
                     5. 6. 1906 (Bahr: \emph{Tagebücher, Skizzenhefte,
                        Notizbücher} V,16)}}}\label{K_L01601-1h}« haſt Du wol bekommen. Ich möchte
               gern gelegentlich ein durchaus aufrichtiges, rückſichtsloſes Wort von Dir darüber
               hören. Und dann bitte ich Dich, es, wenn Dus geleſen haſt, an Salten\pwindex{Salten, Felix 06.09.1869 – 08.10.1945@\textsc{Salten, Felix} (06.09.1869 – 08.10.1945), \emph{Schriftsteller, Journalist}|pw} nach Berlin\oindex{Berlin@\textbf{Berlin}|pw} zu
               ſchicken. Ich fahre \label{K_L01601-2v}\edtext{morgen nach Venedig\oindex{Venedig@\textbf{Venedig}|pw}}{\lemma{\textnormal{\emph{morgen nach Venedig}}}\Cendnote{\textnormal{Bahr\pwindex{Bahr, Hermann 19.07.1863 – 15.01.1934@\textsc{Bahr, Hermann} (19.07.1863 – 15.01.1934), \emph{Schriftsteller, Kritiker}|pwk} fuhr am 23. 6. 1906 und
                  blieb bis Ende Juli.}}}\label{K_L01601-2h}. Nachrichten an meine Wiener {\pb}Adresse\oindex{Veitlissengasse@\textbf{Veitlissengasse}|pw} kommen mir
               immer nach. Vielleicht könnten wir uns im Auguſt irgendwo treffen. Grüß Deine Frau\pwindex{Schnitzler, Olga 17.01.1882 – 13.01.1970@\textsc{Schnitzler, Olga} (17.01.1882 – 13.01.1970), \emph{Schauspielerin, Sängerin}|pwv} herzlichſt und nimm die
               beſten Wünſche für einen frohen Sommer von \pend
           \pstart
           Deinem alten{\\[\baselineskip]}\spacefill\mbox{Hermann}\pend
           \leftskip=0em{}
         
         \endnumbering\mylabel{h}\end{ledgroupsized}  \newcommand{\dateiname}{L01601}\newcommand{\titel}{Hermann Bahr an Arthur Schnitzler, 21. 6. 1906}\newcommand{\editorInnen}{ Kurt Ifkovits,  Martin Anton Müller}%% latex-leseansicht-abspann.tex
%% Abspann für die Leseansicht.
%% Der Schalter \ifkorrekturansicht ist bereits durch den Vorspann gesetzt.

%% latex-abspann.tex
%% Gemeinsamer Abspann für Korrekturansicht und Leseansicht.
%% Setzt den Schalter \ifkorrekturansicht voraus (gesetzt in den
%% einbindenden Dateien latex-korrekturansicht-abspann.tex bzw.
%% latex-leseansicht-abspann.tex).
%% ---------------------------------------------------------------

\normalsize

% Das esempio-Environment wird nur in der Leseansicht benötigt
\ifkorrekturansicht\else
\newenvironment{esempio}[3]%
{
    \vspace{1.5ex}
    \rlap{\underline{#1}}
    \par
    \setlength{\parindent}{0cm}
    \nopagebreak
    \leftskip=#2cm
    \rightskip=#3cm
}
{
    \par
}
\fi

\doendnotes{C}
\bigskip
\vfill

\clearpage

\footnotesize

\ifkorrekturansicht
  \lohead{\textsc{register}}
\fi

% theindex-Environment neu definieren ohne reledmac
\makeatletter
\renewenvironment{theindex}{%
  \ifkorrekturansicht
    \section*{\indexname}%
  \else
    \subsubsection*{Index der erwähnten Entitäten}%
  \fi
  \setlength{\parindent}{0pt}%
  \setlength{\parskip}{0pt plus 0.3pt}%
  \let\item\@idxitem
}{%
  \ifkorrekturansicht\clearpage\fi
}
\makeatother

\IfFileExists{\jobname-pw.ind}{\input{\jobname-pw.ind}}{}

% Quellenangabe nur in der Leseansicht
\ifkorrekturansicht\else
% Fallback-Definitionen, falls die .tex-Datei \titel etc. nicht gesetzt hat
\providecommand{\titel}{}
\providecommand{\editorInnen}{}
\providecommand{\dateiname}{\jobname}

\vspace{3cm}

\vfill

\footnotesize
\textsc{Quelle}: \titel. Herausgegeben von {\editorInnen}. In: \emph{Arthur Schnitzler: Briefwechsel mit Autorinnen und Autoren}.
 Digitale Edition, https://schnitzler-briefe.acdh.oeaw.ac.at/{\dateiname}.html (Stand \today)
\fi

\end{document}


      