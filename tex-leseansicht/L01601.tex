%% latex-korrekturansicht-vorspann.tex
%% Vorspann für die Korrekturansicht.
%% Lädt die gemeinsame Datei latex-vorspann.tex mit gesetztem Schalter.

\newif\ifkorrekturansicht
\korrekturansichttrue

\input{../tex-inputs/latex-vorspann}


\section[Hermann Bahr an Arthur Schnitzler, 21. 6. 1906]{L01601 Hermann Bahr an Arthur Schnitzler, 21. 6. 1906}
\nopagebreak\mylabel{L01601v}
\rehead{ }\normalsize\beginnumbering\briefempfaengerindex{Schnitzler, Arthur@\textsc{Schnitzler, Arthur}!zzzBahr, Hermann@\emph{von Hermann Bahr}!1906-06-211@{21. 6. 1906}|(be}
\toendnotes[C]{\smallbreak\pagebreak[2]}\Standort{CUL, Schnitzler, B 5b.}
\physDesc{Brief, 1 Blatt, 2 Seiten, 572 Zeichen
\newline{}Handschrift: blaue Tinte, deutsche Kurrent
\newline{}Ordnung: mit Bleistift von unbekannter Hand nummeriert:
                                    »139« }
\buchAbdrucke{\weitereDrucke{Hermann Bahr, Arthur Schnitzler: \emph{Briefwechsel, Aufzeichnungen, Dokumente (1891–1931)}. Göttingen: \emph{Wallstein} 2018, S. 378–379.} }\toendnotes[C]{\smallbreak}
\pstart
           \raggedleft{}{\pb}Wien XIII/\textsubscript{7}\oindex{Ober Sankt Veit@\textbf{Ober Sankt Veit}, \emph{P.PPLX}|pw}{\\}21. 6. 06\pend
           
\pstart\center{}Lieber Artur!\pend\vspace{0.5em}
\pstart
           Ich wollte immer noch zu Dir, war aber die letzte Zeit ſo gehetzt, daß es nie ging.
               Den »\label{K_L01601-1v}\edtext{Faun\pwindex{Faun. Ein Akt@\emph{Der Faun. Ein Akt}|pw}}{\lemma{\textnormal{\emph{Faun}}}\Cendnote{\textnormal{fertiggestellt am
                     5. 6. 1906 (Bahr: \emph{Tagebücher, Skizzenhefte,
                        Notizbücher} V,16.)}}}\label{K_L01601-1}« haſt Du wol bekommen. Ich möchte
               gern gelegentlich ein durchaus aufrichtiges, rückſichtsloſes Wort von Dir darüber
               hören. Und dann bitte ich Dich, es, wenn Dus geleſen haſt, an Salten\pwindex{Salten, Felix 06.09.1869 – 08.10.1945@\textsc{Salten, Felix} (06.09.1869 – 08.10.1945), \emph{Schriftsteller/Schriftstellerin, Journalist/Journalistin, Chefredakteur/Chefredakteurin}|pw} nach Berlin\oindex{Berlin@\textbf{Berlin}, \emph{P.PPLC}|pw} zu
               ſchicken. Ich fahre \label{K_L01601-2v}\edtext{morgen nach Venedig\oindex{Venedig@\textbf{Venedig}, \emph{P.PPLA}|pw}}{\lemma{\textnormal{\emph{morgen nach Venedig}}}\Cendnote{\textnormal{Bahr\pwindex{Bahr, Hermann 19.07.1863 – 15.01.1934@\textsc{Bahr, Hermann} (19.07.1863 – 15.01.1934), \emph{Schriftsteller/Schriftstellerin, Kritiker/Kritikerin}|pwk} fuhr am 23. 6. 1906 und
                  blieb bis Ende Juli.}}}\label{K_L01601-2}. Nachrichten an meine Wiener {\pb}Adresse\oindex{Veitlissengasse@\textbf{Veitlissengasse}, \emph{Straße (K.STR)}|pw} kommen mir
               immer nach. Vielleicht könnten wir uns im Auguſt irgendwo treffen. Grüß Deine Frau\pwindex{Schnitzler, Olga 17.01.1882 – 13.01.1970@\textsc{Schnitzler, Olga} (17.01.1882 – 13.01.1970), \emph{Schauspieler/Schauspielerin, Sänger/Sängerin}|pwv} herzlichſt und nimm die
               beſten Wünſche für einen frohen Sommer von \pend
           
\pstart
           Deinem alten{\\[\baselineskip]}\spacefill\mbox{Hermann}\pend
           \leftskip=0em{}\selectlanguage{ngerman}\endnumbering\briefempfaengerindex{Schnitzler, Arthur@\textsc{Schnitzler, Arthur}!zzzBahr, Hermann@\emph{von Hermann Bahr}!1906-06-211@{21. 6. 1906}|)be}\mylabel{L01601h}  \normalsize

\doendnotes{C}
\bigskip
\vfill

\clearpage

\footnotesize

\lohead{\textsc{register}}

% Definiere theindex-Environment komplett neu ohne reledmac
\makeatletter
\renewenvironment{theindex}{%
  \section*{\indexname}%
  \setlength{\parindent}{0pt}%
  \setlength{\parskip}{0pt plus 0.3pt}%
  \let\item\@idxitem
}{%
  \clearpage
}
\makeatother

\IfFileExists{\jobname-pw.ind}{\input{\jobname-pw.ind}}{}

\end{document}

      