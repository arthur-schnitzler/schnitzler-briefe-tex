%% latex-korrekturansicht-vorspann.tex
%% Vorspann für die Korrekturansicht.
%% Lädt die gemeinsame Datei latex-vorspann.tex mit gesetztem Schalter.

\newif\ifkorrekturansicht
\korrekturansichttrue

\input{../tex-inputs/latex-vorspann}


\section[Olga Schnitzler an Anna Bahr-Mildenburg, 11. 5. 1921]{L02367 Olga Schnitzler an Anna Bahr-Mildenburg, 11. 5. 1921}
\nopagebreak\mylabel{L02367v}
\rehead{ }\normalsize\beginnumbering\briefempfaengerindex{Bahr-Mildenburg, Anna@\textsc{Bahr-Mildenburg, Anna}!zzzSchnitzler, Olga@\emph{von Olga Schnitzler}!1921-05-111@{11. 5. 1921}|(be}
\toendnotes[C]{\smallbreak\pagebreak[2]}\Standort{TMW, HS AM 31276 BaM.}
\physDesc{Brief, 1 Blatt, 2 Seiten, 582 Zeichen
\newline{}Handschrift: schwarze Tinte, lateinische Kurrent}
\buchAbdrucke{\weitereDrucke{1) Arthur Schnitzler: \emph{The Letters of Arthur Schnitzler to Hermann Bahr}. Chapel Hill: \emph{The University of North Carolina Press} 1978, S. 116.} \weitereDrucke{2) Hermann Bahr, Arthur Schnitzler: \emph{Briefwechsel, Aufzeichnungen, Dokumente (1891–1931)}. Göttingen: \emph{Wallstein} 2018, S. 541–542.} }\toendnotes[C]{\smallbreak}
\pstart{}{\pb}Meine liebe und hochverehrte gnädige Frau,\pend\vspace{0.5em}
\pstart
           soeben erst erfahre ich von D\textsuperscript{r}{ }\label{K_L02367-1v}\edtext{Knappe\pwindex{Knappe, Heinrich 1887-09-28 – 1980-10-12@\textsc{Knappe, Heinrich} (1887-09-28 – 1980-10-12), \emph{Dirigent/Dirigentin}|pw}}{\lemma{\textnormal{\emph{Knappe}}}\Cendnote{\textnormal{Korrepetitor von Anna Bahr-Mildenburg\pwindex{Bahr-Mildenburg, Anna 29.11.1872 – 27.01.1947@\textsc{Bahr-Mildenburg, Anna} (29.11.1872 – 27.01.1947), \emph{Sänger/Sängerin}|pwk}}}}\label{K_L02367-1}, was für \label{K_L02367-2v}\edtext{schreckliche Wochen}{\lemma{\textnormal{\emph{schreckliche Wochen}}}\Cendnote{\textnormal{Am 17. 4. 1921 war Anna Bahr-Mildenburgs\pwindex{Bahr-Mildenburg, Anna 29.11.1872 – 27.01.1947@\textsc{Bahr-Mildenburg, Anna} (29.11.1872 – 27.01.1947), \emph{Sänger/Sängerin}|pwk}
                  Mutter Anna Bellschan von Mildenburg\pwindex{Bellschan von Mildenburg, Anna 12.7.1837 – 17.4.1921@\textsc{Bellschan von Mildenburg, Anna} (12.7.1837 – 17.4.1921)|pwk} in Klagenfurt\oindex{Klagenfurt@\textbf{Klagenfurt}, \emph{P.PPLA}|pwk} gestorben.}}}\label{K_L02367-2} Sie hatten, – ich hatte ja
               keine Ahnung! Ich war selbst krank und hab mich vor lauter Kummer ganz in meine vier
               Wände verkrochen,– nun war Arthur eine Woche
               bei mir, er ist heute früh abgereist, und ich glaube, an freundlichere Zeiten und
               besseres Verstehen zwischen uns.\pend
           
\pstart
           Nehmen Sie diese Blumen, liebe gnädige Frau, als ein Zeichen meiner innigsten
               Verehrung für Sie entgegen,– und glauben Sie an die herzlichste Anteilnahme{\pb}\pend
           
\pstart
           Ihrer aufrichtig ergebenen{\\[\baselineskip]}\spacefill\mbox{Olga Schnitzler.}\pend
           \leftskip=0em{}
\pstart
           \noindent{}11. Mai 21. \pend
           \selectlanguage{ngerman}\endnumbering\briefempfaengerindex{Bahr-Mildenburg, Anna@\textsc{Bahr-Mildenburg, Anna}!zzzSchnitzler, Olga@\emph{von Olga Schnitzler}!1921-05-111@{11. 5. 1921}|)be}\mylabel{L02367h}  \normalsize

\doendnotes{C}
\bigskip
\vfill

\clearpage

\footnotesize

\lohead{\textsc{register}}

% Definiere theindex-Environment komplett neu ohne reledmac
\makeatletter
\renewenvironment{theindex}{%
  \section*{\indexname}%
  \setlength{\parindent}{0pt}%
  \setlength{\parskip}{0pt plus 0.3pt}%
  \let\item\@idxitem
}{%
  \clearpage
}
\makeatother

\IfFileExists{\jobname-pw.ind}{\input{\jobname-pw.ind}}{}

\end{document}

      