%% latex-leseansicht-vorspann.tex
%% Vorspann für die Leseansicht.
%% Lädt die gemeinsame Datei latex-vorspann.tex mit nicht gesetztem Schalter.

\newif\ifkorrekturansicht
\korrekturansichtfalse

\input{../tex-inputs/latex-vorspann}


\section[Olga Schnitzler an Anna Bahr-Mildenburg, 11. 5. 1921]{L02367 Olga Schnitzler an Anna Bahr-Mildenburg, 11. 5. 1921}
\nopagebreak\mylabel{L02367v}
\rehead{ }\normalsize\beginnumbering\briefempfaengerindex{Bahr-Mildenburg, Anna@\textsc{Bahr-Mildenburg, Anna}!zzzSchnitzler, Olga@\emph{von Olga Schnitzler}!1921-05-111@{11. 5. 1921}|(be}
\toendnotes[C]{\smallbreak\pagebreak[2]}
\correspDesc{Versand  durch Olga Schnitzler am 11. 5. 1921 in München
\newline{}Erhalt  durch Anna Bahr-Mildenburg im Zeitraum [12. 5. 1921
                  – 16. 5. 1921?] in Salzburg}\toendnotes[C]{\smallbreak}
\Standort{TMW, HS AM 31276 BaM.}
\physDesc{Brief, 1 Blatt, 2 Seiten, 582 Zeichen
\newline{}Handschrift: schwarze Tinte, lateinische Kurrent}
\buchAbdrucke{\weitereDrucke{1) Arthur Schnitzler: \emph{The Letters of Arthur Schnitzler to Hermann Bahr}. Edited, annotated, and with an introduction, by Donald G. Daviau. Chapel Hill: \emph{The University of North Carolina Press} 1978, S. 116 (University of North Carolina studies in the Germanic languages
                        and literatures, 89).} \weitereDrucke{2) Hermann Bahr, Arthur Schnitzler: \emph{Briefwechsel, Aufzeichnungen, Dokumente (1891–1931)}. Herausgegeben von Kurt Ifkovits und Martin Anton Müller. Göttingen: \emph{Wallstein} 2018, S. 541–542.} }\toendnotes[C]{\smallbreak}
\pstart{}{\pb}Meine liebe und hochverehrte gnädige Frau,\pend\vspace{0.5em}
\pstart
           soeben erst erfahre ich von D\textsuperscript{r}{ }\label{K_L02367-1v}\edtext{Knappe\pwindex{Knappe, Heinrich 28.\,9.\,1887 Bamberg – 12.\,10.\,1980 München@\textsc{Knappe, Heinrich} (28.\,9.\,1887 Bamberg – 12.\,10.\,1980 München), \emph{Dirigent}|pw}}{\lemma{\textnormal{\emph{Knappe}}}\Cendnote{\textnormal{Korrepetitor von Anna Bahr-Mildenburg\pwindex{Bahr-Mildenburg, Anna 29.\,11.\,1872 Wien – 27.\,1.\,1947 ebd.@\textsc{Bahr-Mildenburg, Anna} (29.\,11.\,1872 Wien – 27.\,1.\,1947 ebd.), \emph{Sängerin}|pwk}}}}\label{K_L02367-1}, was für \label{K_L02367-2v}\edtext{schreckliche Wochen}{\lemma{\textnormal{\emph{schreckliche Wochen}}}\Cendnote{\textnormal{Am 17. 4. 1921 war Anna Bahr-Mildenburgs\pwindex{Bahr-Mildenburg, Anna 29.\,11.\,1872 Wien – 27.\,1.\,1947 ebd.@\textsc{Bahr-Mildenburg, Anna} (29.\,11.\,1872 Wien – 27.\,1.\,1947 ebd.), \emph{Sängerin}|pwk}
                  Mutter Anna Bellschan von Mildenburg\pwindex{Bellschan von Mildenburg, Anna 12.\,7.\,1837 Ellwangen – 17.\,4.\,1921 Klagenfurt@\textsc{Bellschan von Mildenburg, Anna} (12.\,7.\,1837 Ellwangen – 17.\,4.\,1921 Klagenfurt)|pwk} in Klagenfurt\oindex{Klagenfurt@\textbf{Klagenfurt}|pwk} gestorben.}}}\label{K_L02367-2} Sie hatten, – ich hatte ja
               keine Ahnung! Ich war selbst krank und hab mich vor lauter Kummer ganz in meine vier
               Wände verkrochen,– nun war Arthur eine Woche
               bei mir, er ist heute früh abgereist, und ich glaube, an freundlichere Zeiten und
               besseres Verstehen zwischen uns.\pend
           
\pstart
           Nehmen Sie diese Blumen, liebe gnädige Frau, als ein Zeichen meiner innigsten
               Verehrung für Sie entgegen,– und glauben Sie an die herzlichste Anteilnahme\pend
           
\pstart
           {\pb}Ihrer aufrichtig ergebenen{\\[\baselineskip]}\spacefill\mbox{Olga Schnitzler.}\pend
           \leftskip=0em{}
\pstart
           \noindent{}11. Mai 21.\pend
           \selectlanguage{ngerman}\endnumbering\briefempfaengerindex{Bahr-Mildenburg, Anna@\textsc{Bahr-Mildenburg, Anna}!zzzSchnitzler, Olga@\emph{von Olga Schnitzler}!1921-05-111@{11. 5. 1921}|)be}\mylabel{L02367h}  \newcommand{\dateiname}{L02367}\newcommand{\titel}{Olga Schnitzler an Anna Bahr-Mildenburg, 11. 5. 1921}\newcommand{\editorInnen}{Herausgegeben von Martin Anton Müller}%% latex-leseansicht-abspann.tex
%% Abspann für die Leseansicht.
%% Der Schalter \ifkorrekturansicht ist bereits durch den Vorspann gesetzt.

%% latex-abspann.tex
%% Gemeinsamer Abspann für Korrekturansicht und Leseansicht.
%% Setzt den Schalter \ifkorrekturansicht voraus (gesetzt in den
%% einbindenden Dateien latex-korrekturansicht-abspann.tex bzw.
%% latex-leseansicht-abspann.tex).
%% ---------------------------------------------------------------

\normalsize

% Das esempio-Environment wird nur in der Leseansicht benötigt
\ifkorrekturansicht\else
\newenvironment{esempio}[3]%
{
    \vspace{1.5ex}
    \rlap{\underline{#1}}
    \par
    \setlength{\parindent}{0cm}
    \nopagebreak
    \leftskip=#2cm
    \rightskip=#3cm
}
{
    \par
}
\fi

\doendnotes{C}
\bigskip
\vfill

\clearpage

\footnotesize

\ifkorrekturansicht
  \lohead{\textsc{register}}
\fi

% theindex-Environment neu definieren ohne reledmac
\makeatletter
\renewenvironment{theindex}{%
  \ifkorrekturansicht
    \section*{\indexname}%
  \else
    \subsubsection*{Index der erwähnten Entitäten}%
  \fi
  \setlength{\parindent}{0pt}%
  \setlength{\parskip}{0pt plus 0.3pt}%
  \let\item\@idxitem
}{%
  \ifkorrekturansicht\clearpage\fi
}
\makeatother

\IfFileExists{\jobname-pw.ind}{\input{\jobname-pw.ind}}{}

% Quellenangabe nur in der Leseansicht
\ifkorrekturansicht\else
% Fallback-Definitionen, falls die .tex-Datei \titel etc. nicht gesetzt hat
\providecommand{\titel}{}
\providecommand{\editorInnen}{}
\providecommand{\dateiname}{\jobname}

\vspace{3cm}

\vfill

\footnotesize
\textsc{Quelle}: \titel. Herausgegeben von {\editorInnen}. In: \emph{Arthur Schnitzler: Briefwechsel mit Autorinnen und Autoren}.
 Digitale Edition, https://schnitzler-briefe.acdh.oeaw.ac.at/{\dateiname}.html (Stand \today)
\fi

\end{document}


