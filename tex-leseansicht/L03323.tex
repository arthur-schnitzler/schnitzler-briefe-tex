%% latex-leseansicht-vorspann.tex
%% Vorspann für die Leseansicht.
%% Lädt die gemeinsame Datei latex-vorspann.tex mit nicht gesetztem Schalter.

\newif\ifkorrekturansicht
\korrekturansichtfalse

\input{../tex-inputs/latex-vorspann}


         
         \renewcommand{\erwaehntePersonen}{Personen: Felix Salten}
         \renewcommand{\erwaehnteInstitutionen}{Institutionen: Burgtheater}
         \renewcommand{\erwaehnteOrte}{Orte: Burgtheater, Mödling, Wien}
         \renewcommand{\erwaehnteWerke}{Werke: (Burgtheater.) [Das Komplott von Gustav Triesch], Das Komplott. Lustspiel in vier Akten, Tagebuch, Wiener Allgemeine Zeitung}
               \section[ Felix Salten an Arthur Schnitzler, {[}19. 2. 1902{]}]{ Felix Salten an Arthur Schnitzler, {[}19. 2. 1902{]}}\nopagebreak\mylabel{v}\rehead{ }\begin{ledgroupsized}[t]{13cm}\normalsize\beginnumbering\briefempfaengerindex{Schnitzler, Arthur@\textsc{Schnitzler, Arthur}!zzzSalten, Felix@\emph{von Felix Salten}!1902-02-191@{{[}19. 2. 1902{]}}|(be} \toendnotes[C]{\smallbreak\pagebreak[2]} \Standort{CUL, Schnitzler, B 89, A 2.}
\physDesc{Brief, 1 Blatt, 2 Seiten, 302 Zeichen
\newline{}Handschrift: Bleistift, lateinische Kurrent
\newline{}Schnitzler: mit Bleistift datiert: »19/2 902« 
\newline{}Ordnung: mit Bleistift von unbekannter Hand nummeriert: »147« }\toendnotes[C]{\smallbreak}\pstart
           \noindent{}{\pb}Lieber Arthur, entschuldigen Sie, dass ich \label{K_L03323-1v}\edtext{gestern}{\lemma{\textnormal{\emph{gestern}}}\Cendnote{\textnormal{Schnitzler\pwindex{Schnitzler, Arthur 15.05.1862 – 21.10.1931@\textsc{Schnitzler, Arthur} (15.05.1862 – 21.10.1931), \emph{Schriftsteller, Mediziner}|pwk} pendelte in
                  diesen Tagen zwischen Wien\oindex{Wien@\textbf{Wien}|pwk} und Mödling\oindex{Moedling@\textbf{Mödling}|pwk}. Für den 18. 2. 1902 gibt es
                  keinen Eintrag im \emph{Tagebuch}\pwindex{\textcolor{red}{\textsuperscript{XXXX1 indx}}!Tagebuch1981 – 2000@\strich\emph{Tagebuch} {[}Hrsg., 1981 – 2000{]}|pwk}. Dieser Brief kann
                  als Indiz genommen werden, dass sich Schnitzler\pwindex{Schnitzler, Arthur 15.05.1862 – 21.10.1931@\textsc{Schnitzler, Arthur} (15.05.1862 – 21.10.1931), \emph{Schriftsteller, Mediziner}|pwk} an diesem Tag in Wien\oindex{Wien@\textbf{Wien}|pwk}
                  aufhielt.}}}\label{K_L03323-1h} nicht kam. Ich hatte eine abscheuliche Scene, die eben anfing,
               als ich fortgehen wollte (ohne damit in Zusa{\geminationm}enhang zu
               stehen) und die in aller Lieblichkeit {\pb}bis 12\textsuperscript{h} dauerte.\pend
           \pstart
           Ich bin \label{K_L03323-2v}\edtext{Morgen{ }nach dem Burgth.\oindex{Burgtheater@\textbf{Burgtheater}|pw}}{\lemma{\textnormal{\emph{Morgen nach dem Burgth.}}}\Cendnote{\textnormal{Zumindest
                  partiell erlaubt das die Verifizierung der Datierung. Am
                     18. 2. 1902 hatte \emph{Das
                     Komplott. Lustspiel in vier Akten}\pwindex{\textcolor{red}{\textsuperscript{XXXX1 indx}}!Komplott. Lustspiel in vier Akten1902-02-18@\strich\emph{Das Komplott. Lustspiel in vier Akten} {[}1902-02-18{]}|pwk} am \emph{Burgtheater}\orgindex{Burgtheater@Burgtheater|pwk} Uraufführung. Salten\pwindex{Salten, Felix 06.09.1869 – 08.10.1945@\textsc{Salten, Felix} (06.09.1869 – 08.10.1945), \emph{Schriftsteller, Journalist, Chefredakteur}|pwk}
                  verriss sie (f. s.\pwindex{Salten, Felix 06.09.1869 – 08.10.1945@\textsc{Salten, Felix} (06.09.1869 – 08.10.1945), \emph{Schriftsteller, Journalist, Chefredakteur}|pwk}:
                        \emph{(Burgtheater.)}\pwindex{(Burgtheater.) [Das Komplott von Gustav Triesch]1902-02-20@\emph{(Burgtheater.) [Das Komplott von Gustav Triesch]} {[}1902-02-20{]}|pwk} In: \emph{Wiener Allgemeine Zeitung. 6-Uhr-Blatt}\pwindex{Wiener Allgemeine Zeitung1.3.1880 – 11.2.1934@\emph{Wiener Allgemeine Zeitung} {[}1.3.1880 – 11.2.1934{]}|pwk}, Nr. 7.183,
                        20. 2. 1902, S. 2.). Es ist also unwahrscheinlich,
                  dass er sich das Stück ein zweites Mal ansah. Entsprechend ist eine ansonsten in der Korrespondenz durchaus
                  vorkommende Umdatierung des Schreibens um einen Tag früher oder später hier nicht wahrscheinlich, da 
                  er unter diesen Umständen neuerlich \emph{Das
                     Komplott}\pwindex{\textcolor{red}{\textsuperscript{XXXX1 indx}}!Komplott. Lustspiel in vier Akten1902-02-18@\strich\emph{Das Komplott. Lustspiel in vier Akten} {[}1902-02-18{]}|pwk} gesehen hätte.}}}\label{K_L03323-2h} im Caféhaus.
               Vielleicht sind Sie \label{K_L03323-3v}\edtext{dort}{\lemma{\textnormal{\emph{dort}}}\Cendnote{\textnormal{Einen Kaffeehausbesuch am  20. 2. 1902
                  kann mit Schnitzler\pwindex{Schnitzler, Arthur 15.05.1862 – 21.10.1931@\textsc{Schnitzler, Arthur} (15.05.1862 – 21.10.1931), \emph{Schriftsteller, Mediziner}|pwk}s \emph{Tagebuch}\pwindex{\textcolor{red}{\textsuperscript{XXXX1 indx}}!Tagebuch1981 – 2000@\strich\emph{Tagebuch} {[}Hrsg., 1981 – 2000{]}|pwk} nicht belegt werden.}}}\label{K_L03323-3h}?\pend
           \pstart
           Herzlichst {\\[\baselineskip]}Ihr {\\[\baselineskip]}\spacefill\mbox{Salten}\pend
           \leftskip=0em{}
         
         \endnumbering\mylabel{h}\end{ledgroupsized}  \newcommand{\dateiname}{L03323}\newcommand{\titel}{Felix Salten an Arthur Schnitzler, [19. 2. 1902]}\newcommand{\editorInnen}{Martin Anton Müller und Laura Untner}%% latex-leseansicht-abspann.tex
%% Abspann für die Leseansicht.
%% Der Schalter \ifkorrekturansicht ist bereits durch den Vorspann gesetzt.

%% latex-abspann.tex
%% Gemeinsamer Abspann für Korrekturansicht und Leseansicht.
%% Setzt den Schalter \ifkorrekturansicht voraus (gesetzt in den
%% einbindenden Dateien latex-korrekturansicht-abspann.tex bzw.
%% latex-leseansicht-abspann.tex).
%% ---------------------------------------------------------------

\normalsize

% Das esempio-Environment wird nur in der Leseansicht benötigt
\ifkorrekturansicht\else
\newenvironment{esempio}[3]%
{
    \vspace{1.5ex}
    \rlap{\underline{#1}}
    \par
    \setlength{\parindent}{0cm}
    \nopagebreak
    \leftskip=#2cm
    \rightskip=#3cm
}
{
    \par
}
\fi

\doendnotes{C}
\bigskip
\vfill

\clearpage

\footnotesize

\ifkorrekturansicht
  \lohead{\textsc{register}}
\fi

% theindex-Environment neu definieren ohne reledmac
\makeatletter
\renewenvironment{theindex}{%
  \ifkorrekturansicht
    \section*{\indexname}%
  \else
    \subsubsection*{Index der erwähnten Entitäten}%
  \fi
  \setlength{\parindent}{0pt}%
  \setlength{\parskip}{0pt plus 0.3pt}%
  \let\item\@idxitem
}{%
  \ifkorrekturansicht\clearpage\fi
}
\makeatother

\IfFileExists{\jobname-pw.ind}{\input{\jobname-pw.ind}}{}

% Quellenangabe nur in der Leseansicht
\ifkorrekturansicht\else
% Fallback-Definitionen, falls die .tex-Datei \titel etc. nicht gesetzt hat
\providecommand{\titel}{}
\providecommand{\editorInnen}{}
\providecommand{\dateiname}{\jobname}

\vspace{3cm}

\vfill

\footnotesize
\textsc{Quelle}: \titel. Herausgegeben von {\editorInnen}. In: \emph{Arthur Schnitzler: Briefwechsel mit Autorinnen und Autoren}.
 Digitale Edition, https://schnitzler-briefe.acdh.oeaw.ac.at/{\dateiname}.html (Stand \today)
\fi

\end{document}


      