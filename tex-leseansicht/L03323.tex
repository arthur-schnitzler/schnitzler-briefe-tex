%% latex-korrekturansicht-vorspann.tex
%% Vorspann für die Korrekturansicht.
%% Lädt die gemeinsame Datei latex-vorspann.tex mit gesetztem Schalter.

\newif\ifkorrekturansicht
\korrekturansichttrue

\input{../tex-inputs/latex-vorspann}


\section[ Felix Salten an Arthur Schnitzler, {[}19. 2. 1902{]}]{L03323 Felix Salten an Arthur Schnitzler, {[}19. 2. 1902{]}}
\nopagebreak\mylabel{L03323v}
\rehead{ }\normalsize\beginnumbering\briefempfaengerindex{Schnitzler, Arthur@\textsc{Schnitzler, Arthur}!zzzSalten, Felix@\emph{von Felix Salten}!1902-02-191@{{[}19. 2. 1902{]}}|(be}
\toendnotes[C]{\smallbreak\pagebreak[2]}\Standort{CUL, Schnitzler, B 89, A 2.}
\physDesc{Brief, 1 Blatt, 2 Seiten, 302 Zeichen
\newline{}Handschrift: Bleistift, lateinische Kurrent
\newline{}Schnitzler: mit Bleistift datiert: »19/2 902« 
\newline{}Ordnung: mit Bleistift von unbekannter Hand nummeriert: »147« }\toendnotes[C]{\smallbreak}
\pstart
           \noindent{}{\pb}Lieber Arthur, entschuldigen Sie, dass ich \label{K_L03323-1v}\edtext{gestern}{\lemma{\textnormal{\emph{gestern}}}\Cendnote{\textnormal{Schnitzler pendelte in
                  diesen Tagen zwischen Wien\oindex{Wien@\textbf{Wien}, \emph{A.ADM2}|pwk} und Mödling\oindex{Moedling@\textbf{Mödling}, \emph{P.PPLA3}|pwk}. Für den 18. 2. 1902 gibt es
                  keinen Eintrag im \emph{Tagebuch}\pwindex{Tagebuch@\emph{Tagebuch}|pwk}. Dieser Brief kann
                  als Indiz genommen werden, dass sich Schnitzler an diesem Tag in Wien\oindex{Wien@\textbf{Wien}, \emph{A.ADM2}|pwk}
                  aufhielt.}}}\label{K_L03323-1} nicht kam. Ich hatte eine abscheuliche Scene, die eben anfing,
               als ich fortgehen wollte (ohne damit in Zusa{\geminationm}enhang zu
               stehen) und die in aller Lieblichkeit {\pb}bis 12\textsuperscript{h} dauerte.\pend
           
\pstart
           Ich bin \label{K_L03323-2v}\edtext{Morgen{ }nach dem Burgth.\oindex{Burgtheater@\textbf{Burgtheater}, \emph{S.THTR}|pw}}{\lemma{\textnormal{\emph{Morgen nach dem Burgth.}}}\Cendnote{\textnormal{Zumindest
                  partiell erlaubt das die Verifizierung der Datierung. Am
                     18. 2. 1902 hatte \emph{Das
                     Komplott. Lustspiel in vier Akten}\pwindex{Komplott. Lustspiel in vier Akten@\emph{Das Komplott. Lustspiel in vier Akten}|pwk} am \emph{Burgtheater}\orgindex{Burgtheater@Burgtheater|pwk} Uraufführung. Salten\pwindex{Salten, Felix 06.09.1869 – 08.10.1945@\textsc{Salten, Felix} (06.09.1869 – 08.10.1945), \emph{Schriftsteller/Schriftstellerin, Journalist/Journalistin, Chefredakteur/Chefredakteurin}|pwk}
                  verriss sie (f. s.\pwindex{Salten, Felix 06.09.1869 – 08.10.1945@\textsc{Salten, Felix} (06.09.1869 – 08.10.1945), \emph{Schriftsteller/Schriftstellerin, Journalist/Journalistin, Chefredakteur/Chefredakteurin}|pwk}:
                        \emph{(Burgtheater.)}\pwindex{(Burgtheater.) [Das Komplott von Gustav Triesch]@\emph{(Burgtheater.) [Das Komplott von Gustav Triesch]}|pwk} In: \emph{Wiener Allgemeine Zeitung. 6-Uhr-Blatt}\pwindex{Wiener Allgemeine Zeitung@\emph{Wiener Allgemeine Zeitung}|pwk}, Nr. 7183,
                        20. 2. 1902, S. 2). Es ist also unwahrscheinlich,
                  dass er sich das Stück ein zweites Mal ansah. Entsprechend ist eine ansonsten in der Korrespondenz durchaus
                  vorkommende Umdatierung des Schreibens um einen Tag früher oder später hier nicht wahrscheinlich, da 
                  er unter diesen Umständen neuerlich \emph{Das
                     Komplott}\pwindex{Komplott. Lustspiel in vier Akten@\emph{Das Komplott. Lustspiel in vier Akten}|pwk} gesehen hätte.}}}\label{K_L03323-2} im Caféhaus.
               Vielleicht sind Sie \label{K_L03323-3v}\edtext{dort}{\lemma{\textnormal{\emph{dort}}}\Cendnote{\textnormal{Ein Kaffeehausbesuch am  20. 2. 1902
                  kann mit Schnitzlers{ }\emph{Tagebuch}\pwindex{Tagebuch@\emph{Tagebuch}|pwk} nicht belegt werden.}}}\label{K_L03323-3}?\pend
           
\pstart
           Herzlichst {\\[\baselineskip]}Ihr {\\[\baselineskip]}\spacefill\mbox{Salten}\pend
           \leftskip=0em{}\selectlanguage{ngerman}\endnumbering\briefempfaengerindex{Schnitzler, Arthur@\textsc{Schnitzler, Arthur}!zzzSalten, Felix@\emph{von Felix Salten}!1902-02-191@{{[}19. 2. 1902{]}}|)be}\mylabel{L03323h}  \normalsize

\doendnotes{C}
\bigskip
\vfill

\clearpage

\footnotesize

\lohead{\textsc{register}}

% Definiere theindex-Environment komplett neu ohne reledmac
\makeatletter
\renewenvironment{theindex}{%
  \section*{\indexname}%
  \setlength{\parindent}{0pt}%
  \setlength{\parskip}{0pt plus 0.3pt}%
  \let\item\@idxitem
}{%
  \clearpage
}
\makeatother

\IfFileExists{\jobname-pw.ind}{\input{\jobname-pw.ind}}{}

\end{document}

      