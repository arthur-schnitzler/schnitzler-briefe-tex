\input{../tex-inputs/latex-pdf-vorspann}
\begin{center}
            \textcolor{red}{ENTWURF. ENTZIFFERUNG NOCH NICHT KORREKTURGELESEN}
                      \end{center}
            
               \section[Paul Goldmann an Arthur Schnitzler, 20. 3. 1899]{ Paul Goldmann an Arthur Schnitzler, 20. 3. 1899}\nopagebreak\mylabel{v}\rehead{ }\begin{ledgroupsized}[t]{13cm}\normalsize\beginnumbering\briefempfaengerindex{Schnitzler, Arthur@\textsc{Schnitzler, Arthur}!zzzGoldmann, Paul@\emph{von Paul Goldmann}!1899-03-201@{20. 3. 1899}|(be} \toendnotes[C]{\smallbreak\pagebreak[2]} \Standort{DLA, A:Schnitzler, HS.NZ85.1.3169.}
\physDesc{Telegramm, 1 Blatt, 2 Seiten
\newline{}maschinell\newline{}Versand: 1) Stempel: »\nobreak{}\oindex{IX., Alsergrund@\textbf{IX., Alsergrund}|pwk}\textcolor{gray}{W}ien 9/2, 20 III 99, 11 50V\nobreak{}«.  2) Stempel: »\nobreak{}20 3 1899, Ulrich\pwindex{Ulrich @\textsc{Ulrich}, \emph{Telegrafenbeamter/Telegrafenbeamtin}|pw}.\nobreak{}«. \newline{}Ordnung: beschnitten }\pstart{}{\pb}neuntbezirk frankgasze 1\oindex{Frankgasse@\textbf{Frankgasse}|pw}.+\pend{}{\bigskip}\pstart
           \noindent{}\centering{}{\pb}v frankfurtmain\oindex{Frankfurt am Main@\textbf{Frankfurt am Main}|pw}
               928 38 20/3{ }9/55,– m =\pend
           \pstart
           \noindent{}tief erschuettert druecke ich dir die hand im innigsten bejlejd. es ist furchtbar und
               ich finde keine worte. und doch darfst du selbst jetzt nicht glauben dass alles zu
               ende ist.\pend
           \pstart \spacefill\mbox{goldmann}\pend{}\endnumbering\briefempfaengerindex{Schnitzler, Arthur@\textsc{Schnitzler, Arthur}!zzzGoldmann, Paul@\emph{von Paul Goldmann}!1899-03-201@{20. 3. 1899}|)be}\mylabel{h}\end{ledgroupsized}\begin{anhang}\end{anhang}\newcommand{\dateiname}{L02680}\newcommand{\titel}{Paul Goldmann an Arthur Schnitzler, 20. 3. 1899}\newcommand{\editorInnen}{Martin Anton Müller und Laura Untner}\input{../tex-inputs/latex-pdf-abspann}
      