%% latex-leseansicht-vorspann.tex
%% Vorspann für die Leseansicht.
%% Lädt die gemeinsame Datei latex-vorspann.tex mit nicht gesetztem Schalter.

\newif\ifkorrekturansicht
\korrekturansichtfalse

\input{../tex-inputs/latex-vorspann}


\section[Arthur Schnitzler an Richard Beer-Hofmann, 12. 7. 1910]{L01945 Arthur Schnitzler an Richard Beer-Hofmann, 12. 7. 1910}
\nopagebreak\mylabel{L01945v}
\rehead{ }\normalsize\beginnumbering\briefempfaengerindex{Beer-Hofmann, Richard@\textsc{Beer-Hofmann, Richard}!zzzSchnitzler, Arthur@\emph{von Arthur Schnitzler}!1910-07-121@{12. 7. 1910}|(be}
\toendnotes[C]{\smallbreak\pagebreak[2]}
\correspDesc{Versand  durch Arthur Schnitzler am 12. 7. 1910 in Wien
\newline{}Erhalt  durch Richard Beer-Hofmann im Zeitraum [13. 7. 1910
                  – 17. 7. 1910?] in Bad Ischl}\toendnotes[C]{\smallbreak}
\Standort{CUL, Schnitzler, B 8.1, S. 137.}
\physDesc{Brief, maschinenschriftliche Abschrift, 1 Blatt, 1 Seite, 1513 Zeichen
\newline{}Schreibmaschine
\newline{}Ordnung: von unbekannter Hand als Briefnummer 297 gekennzeichnet }
\buchAbdrucke{\weitereDrucke{Arthur Schnitzler, Richard Beer-Hofmann: \emph{Briefwechsel 1891–1931}. Herausgegeben von Konstanze Fliedl. Wien, Zürich: \emph{Europaverlag} 1992, S. 210–211.} }\toendnotes[C]{\smallbreak}
\pstart
           \raggedleft{}{\pb}Wien\oindex{Wien@\textbf{Wien}, \emph{Verwaltungsgebiet}|pw}, 12. 7. 1910.\pend
           \vspace{0.5em}
\pstart
           Mein lieber Richard, wir waren ein paar Tage auf dem Semmering\oindex{Semmering@\textbf{Semmering}, \emph{Verwaltungsgebiet}|pw} – Mama\pwindex{Schnitzler, Louise 8.\,7.\,1840 Kőszeg – 9.\,9.\,1911 Wien@\textsc{Schnitzler, Louise} (8.\,7.\,1840 Kőszeg – 9.\,9.\,1911 Wien)|pwv}’s Geburtstag, englische
                  Verwandte\pwindex{Markbreiter, Felix 20.\,11.\,1855 Wien – 15.\,9.\,1914 London@\textsc{Markbreiter, Felix} (20.\,11.\,1855 Wien – 15.\,9.\,1914 London), \emph{Kaufmann}|pwv}\pwindex{Markbreiter, Amelia Margaret 1887 London – 6.\,10.\,1954 ebd.@\textsc{Markbreiter, Amelia Margaret} (1887 London – 6.\,10.\,1954 ebd.)|pwv}\pwindex{Markbreiter, Andrée Marie 15.\,9.\,1888 London – 16.\,11.\,1971 Worthing@\textsc{Markbreiter, Andrée Marie} (15.\,9.\,1888 London – 16.\,11.\,1971 Worthing)|pwv}\pwindex{Markbreiter, Julie 9.\,11.\,1862 Rechnitz – 24.\,2.\,1938 London@\textsc{Markbreiter, Julie} (9.\,11.\,1862 Rechnitz – 24.\,2.\,1938 London)|pwv}, Brahm\pwindex{Brahm, Otto 5.\,2.\,1856 Hamburg – 28.\,11.\,1912 Berlin@\textsc{Brahm, Otto} (5.\,2.\,1856 Hamburg – 28.\,11.\,1912 Berlin), \emph{Theaterleiter, Regisseur}|pw}, Kainz\pwindex{Kainz, Josef 2.\,1.\,1858 Mosonmagyaróvár – 20.\,9.\,1910 Wien@\textsc{Kainz, Josef} (2.\,1.\,1858 Mosonmagyaróvár – 20.\,9.\,1910 Wien), \emph{Schauspieler}|pw} – und Ihr Brief erwartete mich, als ich unsere schon in
               Zerstörung begriffene Wohnung\oindex{Wien@\textbf{Wien}!XVIII., Währing@\textbf{XVIII., Währing}!Edmund-Weiß-Gasse 7@\textbf{Edmund-Weiß-Gasse 7}, \emph{Wohngebäude}|pwv}
               wieder betrat. Ich freu mich sehr, dass Sie das Stück\pwindex{Schnitzler, Arthur 15.\,5.\,1862 Wien – 21.\,10.\,1931 ebd.@\textsc{Schnitzler, Arthur} (15.\,5.\,1862 Wien – 21.\,10.\,1931 ebd.), \emph{Schriftsteller, Mediziner}!weite Land. Tragikomödie in fünf Akten@\strich\emph{Das weite Land. Tragikomödie in fünf Akten}|pwv} gut finden und glaube auch gern Ihrer Voraussage,
               dass ich noch Freude an meiner Tragikomödie\pwindex{Schnitzler, Arthur 15.\,5.\,1862 Wien – 21.\,10.\,1931 ebd.@\textsc{Schnitzler, Arthur} (15.\,5.\,1862 Wien – 21.\,10.\,1931 ebd.), \emph{Schriftsteller, Mediziner}!weite Land. Tragikomödie in fünf Akten@\strich\emph{Das weite Land. Tragikomödie in fünf Akten}|pwv} haben werde – nur bin ich nicht sicher, ob das schon bei
               Gelegenheit der ersten Aufführung sein wird {\dotstwo} was
               ebensowohl mit Publikumspsychologie als mit Schauspielerconstellation zusammenhängt.
               Ueber all dies, – auch über die Liebe der Genia\pwindex{Schnitzler, Arthur 15.\,5.\,1862 Wien – 21.\,10.\,1931 ebd.@\textsc{Schnitzler, Arthur} (15.\,5.\,1862 Wien – 21.\,10.\,1931 ebd.), \emph{Schriftsteller, Mediziner}!weite Land. Tragikomödie in fünf Akten@\strich\emph{Das weite Land. Tragikomödie in fünf Akten}|pwv}’s zu den Hofreiters\pwindex{Schnitzler, Arthur 15.\,5.\,1862 Wien – 21.\,10.\,1931 ebd.@\textsc{Schnitzler, Arthur} (15.\,5.\,1862 Wien – 21.\,10.\,1931 ebd.), \emph{Schriftsteller, Mediziner}!weite Land. Tragikomödie in fünf Akten@\strich\emph{Das weite Land. Tragikomödie in fünf Akten}|pwv} (die vorkommt! öfters als die zu edlern
               Exemplaren!) näheres, hoffentlich, noch in diesem Sommer. Vorläufig bin ich etwas
               gerührt und fast etwas beschämt, dass Sie mir einen so langen und schönen Brief
               geschrieben haben. (Wenn es aber als Ausrede benützt werden soll, dass Sie im »Traum«
               nicht weiter gekommen sind, so wasch ich meine Hände in Unschuld.) Morgen kommen
               meine Bücher in die Sternwartestrasse\oindex{Wien@\textbf{Wien}!XVIII., Währing@\textbf{XVIII., Währing}!Sternwartestraße 71@\textbf{Sternwartestraße 71}, \emph{Wohngebäude}|pw}; und wir
               hoffen Samstag oder Sonntag zum ersten Mal drüben zu schlafen. Ihr Mirjam\pwindex{Beer-Hofmann, Mirjam 4.\,9.\,1897 Wien – 24.\,12.\,1984 New York City@\textsc{Beer-Hofmann, Mirjam} (4.\,9.\,1897 Wien – 24.\,12.\,1984 New York City)|pw}-Gedicht\pwindex{Beer-Hofmann, Richard 11.\,7.\,1866 Wien – 26.\,9.\,1945 New York City@\textsc{Beer-Hofmann, Richard} (11.\,7.\,1866 Wien – 26.\,9.\,1945 New York City), \emph{Schriftsteller}!Schlaflied für Mirjam@\strich\emph{Schlaflied für Mirjam}|pwv} (für dessen Sendung ich
               herzlich danke) kann ich jetzt von der braven Frieda\pwindex{Pollak, Frieda 8.\,12.\,1881 Wien – 13.\,7.\,1937 ebd.@\textsc{Pollak, Frieda} (8.\,12.\,1881 Wien – 13.\,7.\,1937 ebd.), \emph{Sekretärin}|pw} nicht abschreiben lassen, weil sie in Alt-Aussee\oindex{Altaussee@\textbf{Altaussee}, \emph{Verwaltungsgebiet}|pw}{ }Salzberggasse 46\oindex{Salzbergstraße [Altaussee]@\textbf{Salzbergstraße [Altaussee]}, \emph{Straße}|pw} lebt, ohne Schreibmaschine.
               Aber ich will nächste Woche, wenn wir so weit sind, ihre Vertreterin\pwindex{Hoffmann, Grethe @\textsc{Hoffmann, Grethe}, \emph{Schauspielerin, Schreiberin}|pwv} kommen lassen.\pend
           
\pstart
           Und wie geht es Ihnen? Sind Sie mit Wohnung und allem übrigen zufrieden? Und Paula\pwindex{Beer-Hofmann, Paula 25.\,2.\,1879 Wien – 30.\,10.\,1939 Zürich@\textsc{Beer-Hofmann, Paula} (25.\,2.\,1879 Wien – 30.\,10.\,1939 Zürich)|pw}? Und die Kinder\pwindex{Beer-Hofmann, Gabriel 9.\,1.\,1901 Wien – 24.\,3.\,1971 St Albans@\textsc{Beer-Hofmann, Gabriel} (9.\,1.\,1901 Wien – 24.\,3.\,1971 St Albans), \emph{Schriftsteller, Filmagent}|pwv}\pwindex{Beer-Hofmann, Mirjam 4.\,9.\,1897 Wien – 24.\,12.\,1984 New York City@\textsc{Beer-Hofmann, Mirjam} (4.\,9.\,1897 Wien – 24.\,12.\,1984 New York City)|pwv}\pwindex{Beer-Hofmann, Naëmah 20.\,12.\,1898 Wien – 10.\,11.\,1971 New York City@\textsc{Beer-Hofmann, Naëmah} (20.\,12.\,1898 Wien – 10.\,11.\,1971 New York City)|pwv}?\pend
           
\pstart
           Wir grüssen Euch alle vielmals.\pend
           \pstart Herzlichst Ihr \spacefill\mbox{Arthur.}\pend{}
\pstart
           \noindent{}(nach Ischl\oindex{Bad Ischl@\textbf{Bad Ischl}|pw})\pend
           \selectlanguage{ngerman}\endnumbering\briefempfaengerindex{Beer-Hofmann, Richard@\textsc{Beer-Hofmann, Richard}!zzzSchnitzler, Arthur@\emph{von Arthur Schnitzler}!1910-07-121@{12. 7. 1910}|)be}\mylabel{L01945h}  \newcommand{\dateiname}{L01945}\newcommand{\titel}{Arthur Schnitzler an Richard Beer-Hofmann, 12. 7. 1910}\newcommand{\editorInnen}{Martin Anton Müller und Gerd-Hermann Susen}%% latex-leseansicht-abspann.tex
%% Abspann für die Leseansicht.
%% Der Schalter \ifkorrekturansicht ist bereits durch den Vorspann gesetzt.

%% latex-abspann.tex
%% Gemeinsamer Abspann für Korrekturansicht und Leseansicht.
%% Setzt den Schalter \ifkorrekturansicht voraus (gesetzt in den
%% einbindenden Dateien latex-korrekturansicht-abspann.tex bzw.
%% latex-leseansicht-abspann.tex).
%% ---------------------------------------------------------------

\normalsize

% Das esempio-Environment wird nur in der Leseansicht benötigt
\ifkorrekturansicht\else
\newenvironment{esempio}[3]%
{
    \vspace{1.5ex}
    \rlap{\underline{#1}}
    \par
    \setlength{\parindent}{0cm}
    \nopagebreak
    \leftskip=#2cm
    \rightskip=#3cm
}
{
    \par
}
\fi

\doendnotes{C}
\bigskip
\vfill

\clearpage

\footnotesize

\ifkorrekturansicht
  \lohead{\textsc{register}}
\fi

% theindex-Environment neu definieren ohne reledmac
\makeatletter
\renewenvironment{theindex}{%
  \ifkorrekturansicht
    \section*{\indexname}%
  \else
    \subsubsection*{Index der erwähnten Entitäten}%
  \fi
  \setlength{\parindent}{0pt}%
  \setlength{\parskip}{0pt plus 0.3pt}%
  \let\item\@idxitem
}{%
  \ifkorrekturansicht\clearpage\fi
}
\makeatother

\IfFileExists{\jobname-pw.ind}{\input{\jobname-pw.ind}}{}

% Quellenangabe nur in der Leseansicht
\ifkorrekturansicht\else
% Fallback-Definitionen, falls die .tex-Datei \titel etc. nicht gesetzt hat
\providecommand{\titel}{}
\providecommand{\editorInnen}{}
\providecommand{\dateiname}{\jobname}

\vspace{3cm}

\vfill

\footnotesize
\textsc{Quelle}: \titel. Herausgegeben von {\editorInnen}. In: \emph{Arthur Schnitzler: Briefwechsel mit Autorinnen und Autoren}.
 Digitale Edition, https://schnitzler-briefe.acdh.oeaw.ac.at/{\dateiname}.html (Stand \today)
\fi

\end{document}


