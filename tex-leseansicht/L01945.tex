%% latex-leseansicht-vorspann.tex
%% Vorspann für die Leseansicht.
%% Lädt die gemeinsame Datei latex-vorspann.tex mit nicht gesetztem Schalter.

\newif\ifkorrekturansicht
\korrekturansichtfalse

\input{../tex-inputs/latex-vorspann}


         
         \renewcommand{\erwaehntePersonen}{Personen: Richard Beer-Hofmann, Mirjam Beer-Hofmann, Paula Beer-Hofmann, Gabriel Beer-Hofmann, Naëmah Beer-Hofmann, Otto Brahm, Grethe Hoffmann, Josef Kainz, Felix Markbreiter, Amelia Margaret Markbreiter, Andrée Marie Markbreiter, Julie Markbreiter, Frieda Pollak, Louise Schnitzler}
         \renewcommand{\erwaehnteOrte}{Orte: Altaussee, Bad Ischl, Edmund-Weiß-Gasse, Salzbergstraße, Semmering, Sternwartestraße, Wien}
         \renewcommand{\erwaehnteWerke}{Werke: Das weite Land. Tragikomödie in fünf Akten, Schlaflied für Mirjam}
               \section[Arthur Schnitzler an Richard Beer-Hofmann, 12. 7. 1910]{ Arthur Schnitzler an Richard Beer-Hofmann, 12. 7. 1910}\nopagebreak\mylabel{v}\rehead{ }\begin{ledgroupsized}[t]{13cm}\normalsize\beginnumbering \toendnotes[C]{\smallbreak\pagebreak[2]} \Standort{CUL, Schnitzler, B 8.1, S. 137.}
\physDesc{Brief, maschinenschriftliche Abschrift, 1 Blatt, 1 Seite, 1513 Zeichen
\newline{}Schreibmaschine
\newline{}Ordnung: von unbekannter Hand als Briefnummer 297 gekennzeichnet }\buchAbdrucke{\weitereDrucke{Arthur Schnitzler, Richard Beer-Hofmann: \emph{Briefwechsel 1891–1931}. Hg. Konstanze Fliedl. Wien, Zürich: \emph{Europaverlag} 1992, S. 210–211.} }\toendnotes[C]{\smallbreak}\pstart
           \raggedleft{}{\pb}Wien\oindex{Wien@\textbf{Wien}|pw}, 12. 7. 1910.\pend
           \pstart
           Mein lieber Richard, wir waren ein paar Tage auf dem Semmering\oindex{Semmering@\textbf{Semmering}|pw} – Mama\pwindex{Schnitzler, Louise 1840-07-08 – 1911-09-09@\textsc{Schnitzler, Louise} (1840-07-08 – 1911-09-09)|pwv}’s Geburtstag, englische
                  Verwandte\pwindex{Markbreiter, Felix 20.11.1855 – 15.09.1914@\textsc{Markbreiter, Felix} (20.11.1855 – 15.09.1914), \emph{Kaufmann}|pwv}\pwindex{Markbreiter, Amelia Margaret 1887 – 06.10.1954@\textsc{Markbreiter, Amelia Margaret} (1887 – 06.10.1954)|pwv}\pwindex{Markbreiter, Andree Marie 15.09.1888 – 16.11.1971@\textsc{Markbreiter, Andrée Marie} (15.09.1888 – 16.11.1971)|pwv}\pwindex{Markbreiter, Julie 9.11.1862 – 24.02.1938@\textsc{Markbreiter, Julie} (9.11.1862 – 24.02.1938)|pwv}, Brahm\pwindex{Brahm, Otto 05.02.1856 – 28.11.1912@\textsc{Brahm, Otto} (05.02.1856 – 28.11.1912), \emph{Theaterleiter, Regisseur}|pw}, Kainz\pwindex{Kainz, Josef 02.01.1858 – 20.09.1910@\textsc{Kainz, Josef} (02.01.1858 – 20.09.1910), \emph{Schauspieler}|pw} – und Ihr Brief erwartete mich, als ich unsere schon in
               Zerstörung begriffene Wohnung\oindex{XXXX Ortsangabe fehlt|pwv}
               wieder betrat. Ich freu mich sehr, dass Sie das Stück\pwindex{Schnitzler, Arthur 15.05.1862 – 21.10.1931@\textsc{Schnitzler, Arthur} (15.05.1862 – 21.10.1931), \emph{Schriftsteller, Mediziner}!weite Land. Tragikomoedie in fuenf Akten1910-10-20@\strich\emph{Das weite Land. Tragikomödie in fünf Akten} {[}1910-10-20{]}|pwv} gut finden und glaube auch gern Ihrer Voraussage,
               dass ich noch Freude an meiner Tragikomödie\pwindex{Schnitzler, Arthur 15.05.1862 – 21.10.1931@\textsc{Schnitzler, Arthur} (15.05.1862 – 21.10.1931), \emph{Schriftsteller, Mediziner}!weite Land. Tragikomoedie in fuenf Akten1910-10-20@\strich\emph{Das weite Land. Tragikomödie in fünf Akten} {[}1910-10-20{]}|pwv} haben werde – nur bin ich nicht sicher, ob das schon bei
               Gelegenheit der ersten Aufführung sein wird {\dotstwo} was
               ebensowohl mit Publikumspsychologie als mit Schauspielerconstellation zusammenhängt.
               Ueber all dies, – auch über die Liebe der Genia\pwindex{Schnitzler, Arthur 15.05.1862 – 21.10.1931@\textsc{Schnitzler, Arthur} (15.05.1862 – 21.10.1931), \emph{Schriftsteller, Mediziner}!weite Land. Tragikomoedie in fuenf Akten1910-10-20@\strich\emph{Das weite Land. Tragikomödie in fünf Akten} {[}1910-10-20{]}|pwv}’s zu den Hofreiter\pwindex{Schnitzler, Arthur 15.05.1862 – 21.10.1931@\textsc{Schnitzler, Arthur} (15.05.1862 – 21.10.1931), \emph{Schriftsteller, Mediziner}!weite Land. Tragikomoedie in fuenf Akten1910-10-20@\strich\emph{Das weite Land. Tragikomödie in fünf Akten} {[}1910-10-20{]}|pwv}s (die vorkommt! öfters als die zu edlern
               Exemplaren!) näheres, hoffentlich, noch in diesem Sommer. Vorläufig bin ich etwas
               gerührt und fast etwas beschämt, dass Sie mir einen so langen und schönen Brief
               geschrieben haben. (Wenn es aber als Ausrede benützt werden soll, dass Sie im »Traum«
               nicht weiter gekommen sind, so wasch ich meine Hände in Unschuld.) Morgen kommen
               meine Bücher in die Sternwartestrasse\oindex{XXXX Ortsangabe fehlt|pw}; und wir
               hoffen Samstag oder Sonntag zum ersten Mal drüben zu schlafen. Ihr Mirjam\pwindex{Beer-Hofmann, Mirjam 04.09.1897 – 24.12.1984@\textsc{Beer-Hofmann, Mirjam} (04.09.1897 – 24.12.1984)|pw}-Gedicht\pwindex{Beer-Hofmann, Richard 1866-07-11 – 1945-09-26@\textsc{Beer-Hofmann, Richard} (1866-07-11 – 1945-09-26), \emph{Schriftsteller}!Schlaflied fuer Mirjam15. 11. 1898@\strich\emph{Schlaflied für Mirjam} {[}15. 11. 1898{]}|pwv} (für dessen Sendung ich
               herzlich danke) kann ich jetzt von der braven Frieda\pwindex{Pollak, Frieda 08.12.1881 – 13.07.1937@\textsc{Pollak, Frieda} (08.12.1881 – 13.07.1937), \emph{Sekretärin}|pw} nicht abschreiben lassen, weil sie in Alt-Aussee\oindex{Altaussee@\textbf{Altaussee}|pw}{ }Salzberggasse 46\oindex{Salzbergstrasse@\textbf{Salzbergstraße}|pw} lebt, ohne Schreibmaschine.
               Aber ich will nächste Woche, wenn wir so weit sind, ihre Vertreterin\pwindex{Hoffmann, Grethe @\textsc{Hoffmann, Grethe}, \emph{Schauspielerin, Schreiberin}|pwv} kommen lassen.\pend
           \pstart
           Und wie geht es Ihnen? Sind Sie mit Wohnung und allem übrigen zufrieden? Und Paula\pwindex{Beer-Hofmann, Paula 25.02.1879 – 30.10.1939@\textsc{Beer-Hofmann, Paula} (25.02.1879 – 30.10.1939)|pw}? Und die Kinder\pwindex{Beer-Hofmann, Gabriel 09.01.1901 – 24.03.1971@\textsc{Beer-Hofmann, Gabriel} (09.01.1901 – 24.03.1971), \emph{Schriftsteller, Filmagent}|pwv}\pwindex{Beer-Hofmann, Mirjam 04.09.1897 – 24.12.1984@\textsc{Beer-Hofmann, Mirjam} (04.09.1897 – 24.12.1984)|pwv}\pwindex{Beer-Hofmann, Naemah 20.12.1898 – 10.11.1971@\textsc{Beer-Hofmann, Naëmah} (20.12.1898 – 10.11.1971)|pwv}?\pend
           \pstart
           Wir grüssen Euch alle vielmals.\pend
           \pstart Herzlichst Ihr \spacefill\mbox{Arthur.}\pend{}\pstart
           \noindent{}(nach Ischl\oindex{Bad Ischl@\textbf{Bad Ischl}|pw})\pend
           
         
         \endnumbering\mylabel{h}\end{ledgroupsized}  \newcommand{\dateiname}{L01945}\newcommand{\titel}{Arthur Schnitzler an Richard Beer-Hofmann, 12. 7. 1910}\newcommand{\editorInnen}{Martin Anton Müller und Gerd-Hermann Susen}%% latex-leseansicht-abspann.tex
%% Abspann für die Leseansicht.
%% Der Schalter \ifkorrekturansicht ist bereits durch den Vorspann gesetzt.

%% latex-abspann.tex
%% Gemeinsamer Abspann für Korrekturansicht und Leseansicht.
%% Setzt den Schalter \ifkorrekturansicht voraus (gesetzt in den
%% einbindenden Dateien latex-korrekturansicht-abspann.tex bzw.
%% latex-leseansicht-abspann.tex).
%% ---------------------------------------------------------------

\normalsize

% Das esempio-Environment wird nur in der Leseansicht benötigt
\ifkorrekturansicht\else
\newenvironment{esempio}[3]%
{
    \vspace{1.5ex}
    \rlap{\underline{#1}}
    \par
    \setlength{\parindent}{0cm}
    \nopagebreak
    \leftskip=#2cm
    \rightskip=#3cm
}
{
    \par
}
\fi

\doendnotes{C}
\bigskip
\vfill

\clearpage

\footnotesize

\ifkorrekturansicht
  \lohead{\textsc{register}}
\fi

% theindex-Environment neu definieren ohne reledmac
\makeatletter
\renewenvironment{theindex}{%
  \ifkorrekturansicht
    \section*{\indexname}%
  \else
    \subsubsection*{Index der erwähnten Entitäten}%
  \fi
  \setlength{\parindent}{0pt}%
  \setlength{\parskip}{0pt plus 0.3pt}%
  \let\item\@idxitem
}{%
  \ifkorrekturansicht\clearpage\fi
}
\makeatother

\IfFileExists{\jobname-pw.ind}{\input{\jobname-pw.ind}}{}

% Quellenangabe nur in der Leseansicht
\ifkorrekturansicht\else
% Fallback-Definitionen, falls die .tex-Datei \titel etc. nicht gesetzt hat
\providecommand{\titel}{}
\providecommand{\editorInnen}{}
\providecommand{\dateiname}{\jobname}

\vspace{3cm}

\vfill

\footnotesize
\textsc{Quelle}: \titel. Herausgegeben von {\editorInnen}. In: \emph{Arthur Schnitzler: Briefwechsel mit Autorinnen und Autoren}.
 Digitale Edition, https://schnitzler-briefe.acdh.oeaw.ac.at/{\dateiname}.html (Stand \today)
\fi

\end{document}


      