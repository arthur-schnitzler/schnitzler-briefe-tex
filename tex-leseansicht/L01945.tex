%% latex-korrekturansicht-vorspann.tex
%% Vorspann für die Korrekturansicht.
%% Lädt die gemeinsame Datei latex-vorspann.tex mit gesetztem Schalter.

\newif\ifkorrekturansicht
\korrekturansichttrue

\input{../tex-inputs/latex-vorspann}


\section[Arthur Schnitzler an Richard Beer-Hofmann, 12. 7. 1910]{L01945 Arthur Schnitzler an Richard Beer-Hofmann, 12. 7. 1910}
\nopagebreak\mylabel{L01945v}
\rehead{ }\normalsize\beginnumbering\briefempfaengerindex{Beer-Hofmann, Richard@\textsc{Beer-Hofmann, Richard}!zzzSchnitzler, Arthur@\emph{von Arthur Schnitzler}!1910-07-121@{12. 7. 1910}|(be}
\toendnotes[C]{\smallbreak\pagebreak[2]}\Standort{CUL, Schnitzler, B 8.1, S. 137.}
\physDesc{Brief, maschinenschriftliche Abschrift1 Blatt, 1 Seite, 1513 Zeichen
\newline{}Schreibmaschine
\newline{}Ordnung: von unbekannter Hand als Briefnummer 297 gekennzeichnet }
\buchAbdrucke{\weitereDrucke{Arthur Schnitzler, Richard Beer-Hofmann: \emph{Briefwechsel 1891–1931}. Wien, Zürich: \emph{Europaverlag} 1992, S. 210–211.} }\toendnotes[C]{\smallbreak}
\pstart
           \raggedleft{}{\pb}Wien\oindex{Wien@\textbf{Wien}, \emph{A.ADM2}|pw}, 12. 7. 1910.\pend
           \vspace{0.5em}
\pstart
           Mein lieber Richard, wir waren ein paar Tage auf dem Semmering\oindex{Semmering@\textbf{Semmering}, \emph{A.ADM3}|pw} – Mama\pwindex{Schnitzler, Louise 1840-07-08 – 1911-09-09@\textsc{Schnitzler, Louise} (1840-07-08 – 1911-09-09)|pwv}’s Geburtstag, englische
                  Verwandte\pwindex{Markbreiter, Felix 20.11.1855 – 15.09.1914@\textsc{Markbreiter, Felix} (20.11.1855 – 15.09.1914), \emph{Kaufmann/Kauffrau}|pwv}\pwindex{Markbreiter, Amelia Margaret 1887 – 06.10.1954@\textsc{Markbreiter, Amelia Margaret} (1887 – 06.10.1954)|pwv}\pwindex{Markbreiter, Andree Marie 15.09.1888 – 16.11.1971@\textsc{Markbreiter, Andrée Marie} (15.09.1888 – 16.11.1971)|pwv}\pwindex{Markbreiter, Julie 9.11.1862 – 24.02.1938@\textsc{Markbreiter, Julie} (9.11.1862 – 24.02.1938)|pwv}, Brahm\pwindex{Brahm, Otto 05.02.1856 – 28.11.1912@\textsc{Brahm, Otto} (05.02.1856 – 28.11.1912), \emph{Theaterleiter/Theaterleiterin, Regisseur/Regisseurin}|pw}, Kainz\pwindex{Kainz, Josef 02.01.1858 – 20.09.1910@\textsc{Kainz, Josef} (02.01.1858 – 20.09.1910), \emph{Schauspieler/Schauspielerin}|pw} – und Ihr Brief erwartete mich, als ich unsere schon in
               Zerstörung begriffene Wohnung\oindex{Edmund-Weiss-Gasse 7@\textbf{Edmund-Weiß-Gasse 7}, \emph{Wohngebäude (K.WHS)}|pwv}
               wieder betrat. Ich freu mich sehr, dass Sie das Stück\pwindex{weite Land. Tragikomoedie in fuenf Akten@\emph{Das weite Land. Tragikomödie in fünf Akten}|pwv} gut finden und glaube auch gern Ihrer Voraussage,
               dass ich noch Freude an meiner Tragikomödie\pwindex{weite Land. Tragikomoedie in fuenf Akten@\emph{Das weite Land. Tragikomödie in fünf Akten}|pwv} haben werde – nur bin ich nicht sicher, ob das schon bei
               Gelegenheit der ersten Aufführung sein wird {\dotstwo} was
               ebensowohl mit Publikumspsychologie als mit Schauspielerconstellation zusammenhängt.
               Ueber all dies, – auch über die Liebe der Genia\pwindex{weite Land. Tragikomoedie in fuenf Akten@\emph{Das weite Land. Tragikomödie in fünf Akten}|pwv}’s zu den Hofreiters\pwindex{weite Land. Tragikomoedie in fuenf Akten@\emph{Das weite Land. Tragikomödie in fünf Akten}|pwv} (die vorkommt! öfters als die zu edlern
               Exemplaren!) näheres, hoffentlich, noch in diesem Sommer. Vorläufig bin ich etwas
               gerührt und fast etwas beschämt, dass Sie mir einen so langen und schönen Brief
               geschrieben haben. (Wenn es aber als Ausrede benützt werden soll, dass Sie im »Traum«
               nicht weiter gekommen sind, so wasch ich meine Hände in Unschuld.) Morgen kommen
               meine Bücher in die Sternwartestrasse\oindex{Sternwartestrasse 71@\textbf{Sternwartestraße 71}, \emph{Wohngebäude (K.WHS)}|pw}; und wir
               hoffen Samstag oder Sonntag zum ersten Mal drüben zu schlafen. Ihr Mirjam\pwindex{Beer-Hofmann, Mirjam 04.09.1897 – 24.12.1984@\textsc{Beer-Hofmann, Mirjam} (04.09.1897 – 24.12.1984)|pw}-Gedicht\pwindex{Schlaflied fuer Mirjam@\emph{Schlaflied für Mirjam}|pwv} (für dessen Sendung ich
               herzlich danke) kann ich jetzt von der braven Frieda\pwindex{Pollak, Frieda 08.12.1881 – 13.07.1937@\textsc{Pollak, Frieda} (08.12.1881 – 13.07.1937), \emph{Sekretär/Sekretärin}|pw} nicht abschreiben lassen, weil sie in Alt-Aussee\oindex{Altaussee@\textbf{Altaussee}, \emph{A.ADM3}|pw}{ }Salzberggasse 46\oindex{Salzbergstrasse [Altaussee]@\textbf{Salzbergstraße [Altaussee]}, \emph{Straße (K.STR)}|pw} lebt, ohne Schreibmaschine.
               Aber ich will nächste Woche, wenn wir so weit sind, ihre Vertreterin\pwindex{Hoffmann, Grethe @\textsc{Hoffmann, Grethe}, \emph{Schauspieler/Schauspielerin, Schreiber/Schreiberin}|pwv} kommen lassen.\pend
           
\pstart
           Und wie geht es Ihnen? Sind Sie mit Wohnung und allem übrigen zufrieden? Und Paula\pwindex{Beer-Hofmann, Paula 25.02.1879 – 30.10.1939@\textsc{Beer-Hofmann, Paula} (25.02.1879 – 30.10.1939)|pw}? Und die Kinder\pwindex{Beer-Hofmann, Gabriel 09.01.1901 – 24.03.1971@\textsc{Beer-Hofmann, Gabriel} (09.01.1901 – 24.03.1971), \emph{Schriftsteller/Schriftstellerin, Filmagent/Filmagentin}|pwv}\pwindex{Beer-Hofmann, Mirjam 04.09.1897 – 24.12.1984@\textsc{Beer-Hofmann, Mirjam} (04.09.1897 – 24.12.1984)|pwv}\pwindex{Beer-Hofmann, Naemah 20.12.1898 – 10.11.1971@\textsc{Beer-Hofmann, Naëmah} (20.12.1898 – 10.11.1971)|pwv}?\pend
           
\pstart
           Wir grüssen Euch alle vielmals.\pend
           \pstart Herzlichst Ihr \spacefill\mbox{Arthur.}\pend{}
\pstart
           \noindent{}(nach Ischl\oindex{Bad Ischl@\textbf{Bad Ischl}, \emph{P.PPL}|pw})\pend
           \selectlanguage{ngerman}\endnumbering\briefempfaengerindex{Beer-Hofmann, Richard@\textsc{Beer-Hofmann, Richard}!zzzSchnitzler, Arthur@\emph{von Arthur Schnitzler}!1910-07-121@{12. 7. 1910}|)be}\mylabel{L01945h}  \normalsize

\doendnotes{C}
\bigskip
\vfill

\clearpage

\footnotesize

\lohead{\textsc{register}}

% Definiere theindex-Environment komplett neu ohne reledmac
\makeatletter
\renewenvironment{theindex}{%
  \section*{\indexname}%
  \setlength{\parindent}{0pt}%
  \setlength{\parskip}{0pt plus 0.3pt}%
  \let\item\@idxitem
}{%
  \clearpage
}
\makeatother

\IfFileExists{\jobname-pw.ind}{\input{\jobname-pw.ind}}{}

\end{document}

      