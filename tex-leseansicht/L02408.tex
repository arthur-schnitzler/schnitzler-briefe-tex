%% latex-leseansicht-vorspann.tex
%% Vorspann für die Leseansicht.
%% Lädt die gemeinsame Datei latex-vorspann.tex mit nicht gesetztem Schalter.

\newif\ifkorrekturansicht
\korrekturansichtfalse

\input{../tex-inputs/latex-vorspann}


\section[Hugo Hofmannsthal an Arthur Schnitzler, 18. 1. 1924]{L02408 Hugo Hofmannsthal an Arthur Schnitzler, 18. 1. 1924}
\nopagebreak\mylabel{L02408v}
\rehead{ }\normalsize\beginnumbering\briefempfaengerindex{Schnitzler, Arthur@\textsc{Schnitzler, Arthur}!zzzHofmannsthal, Hugo von@\emph{von Hugo von Hofmannsthal}!1924-01-181@{18. 1. 1924}|(be}
\toendnotes[C]{\smallbreak\pagebreak[2]}
\correspDesc{Versand  durch Hugo von Hofmannsthal am 18. 1. 1924 in Rodaun
\newline{}Erhalt  durch Arthur Schnitzler im Zeitraum [19. 1. 1924
                  – 23. 1. 1924?] in Wien}\toendnotes[C]{\smallbreak}
\Standort{CUL, Schnitzler, B 43.}
\physDesc{Postkarte, 898 Zeichen
\newline{}Handschrift: schwarze Tinte, lateinische Kurrent
\newline{}Versand: Stempel: »\nobreak{}\oindex{Wien@\textbf{Wien}!XXIII., Liesing@\textbf{XXIII., Liesing}!Rodaun@\textbf{Rodaun}, \emph{Region}|pwk}\textcolor{gray}{Rod}{[}aun{]}\nobreak{}«.  
\newline{}Ordnung: 1) mit Bleistift von unbekannter Hand nummeriert: »\strikeout{384}«  2) mit Bleistift von unbekannter Hand nummeriert:
                                    »373«}
\buchAbdrucke{\weitereDrucke{Hugo von Hofmannsthal, Arthur Schnitzler: \emph{Briefwechsel}. Herausgegeben von Therese Nickl und Heinrich Schnitzler. Frankfurt am Main: \emph{S. Fischer} 1964, S. 298.} }\toendnotes[C]{\smallbreak}\pstart{}{\pb}Herrn D\textsuperscript{r} Arthur Schnitzler\pend{}\pstart{}Wien\oindex{Wien@\textbf{Wien}, \emph{Verwaltungsgebiet}|pw}\pend{}\pstart{}XVIII Sternwartestrasse 71\oindex{Wien@\textbf{Wien}!XVIII., Währing@\textbf{XVIII., Währing}!Sternwartestraße 71@\textbf{Sternwartestraße 71}, \emph{Wohngebäude}|pw}\pend{}{\bigskip}\vspace{1em}
\pstart
           \textcolor{gray}{\textbf{{\pb}\textsc{Rodaun}\oindex{Wien@\textbf{Wien}!XXIII., Liesing@\textbf{XXIII., Liesing}!Rodaun@\textbf{Rodaun}, \emph{Region}|pw}}}\hfill 18 I 24.\pend
           
\pstart
           \textcolor{gray}{\textbf{B. WIEN\oindex{Wien@\textbf{Wien}, \emph{Verwaltungsgebiet}|pw}}}\pend
           
\pstart{}mein lieber Arthur\pend\vspace{0.5em}
\pstart
           um unser \label{K_L02408-1v}\edtext{Gespräch}{\lemma{\textnormal{\emph{Gespräch}}}\Cendnote{\textnormal{Vgl. A. S.: \emph{Tagebuch}, 11. 1. 1924.
               }}}\label{K_L02408-1} noch für mich allein zu verlängern, wollte ich gestern abends die »Große Scene\pwindex{Schnitzler, Arthur 15.\,5.\,1862 Wien – 21.\,10.\,1931 ebd.@\textsc{Schnitzler, Arthur} (15.\,5.\,1862 Wien – 21.\,10.\,1931 ebd.), \emph{Schriftsteller, Mediziner}!Große Szene@\strich\emph{Große Szene}|pw}« lesen – aber ich muss durch ein
               Versehen seinerzeit diesen Band (Comödie der
                  Worte\pwindex{Schnitzler, Arthur 15.\,5.\,1862 Wien – 21.\,10.\,1931 ebd.@\textsc{Schnitzler, Arthur} (15.\,5.\,1862 Wien – 21.\,10.\,1931 ebd.), \emph{Schriftsteller, Mediziner}!Komödie der Worte. Drei Einakter@\strich\emph{Komödie der Worte. Drei Einakter}|pw}) {\pb}nicht beko{\geminationm}en haben! Haben Sie vielleicht ein entbehrliches
               Exemplar? Nämlich auch in meinen Bänden \introOben{}Ihrer\introOben{}{ }\label{K_L02408-2v}\edtext{ges. Theaterstücke\pwindex{Schnitzler, Arthur 15.\,5.\,1862 Wien – 21.\,10.\,1931 ebd.@\textsc{Schnitzler, Arthur} (15.\,5.\,1862 Wien – 21.\,10.\,1931 ebd.), \emph{Schriftsteller, Mediziner}!Gesammelte Werke@\strich\emph{Gesammelte Werke}|pwv}}{\lemma{\textnormal{\emph{ges. Theaterstücke}}}\Cendnote{\textnormal{1912 waren \emph{Die gesammelten
                     Werke}\pwindex{Schnitzler, Arthur 15.\,5.\,1862 Wien – 21.\,10.\,1931 ebd.@\textsc{Schnitzler, Arthur} (15.\,5.\,1862 Wien – 21.\,10.\,1931 ebd.), \emph{Schriftsteller, Mediziner}!Gesammelte Werke@\strich\emph{Gesammelte Werke}|pwk} mit vier Bänden \emph{Die
                     Theaterstücke}\pwindex{Schnitzler, Arthur 15.\,5.\,1862 Wien – 21.\,10.\,1931 ebd.@\textsc{Schnitzler, Arthur} (15.\,5.\,1862 Wien – 21.\,10.\,1931 ebd.), \emph{Schriftsteller, Mediziner}!Theaterstücke@\strich\emph{Die Theaterstücke}|pwk} erschienen. Anlässlich des 60. Geburtstages wurde 1922 die
                  Ausgabe um einen Ergänzungsband\pwindex{Schnitzler, Arthur 15.\,5.\,1862 Wien – 21.\,10.\,1931 ebd.@\textsc{Schnitzler, Arthur} (15.\,5.\,1862 Wien – 21.\,10.\,1931 ebd.), \emph{Schriftsteller, Mediziner}!Theaterstücke. Ergänzungsband V@\strich\emph{Die Theaterstücke. Ergänzungsband V}|pwkv} erweitert, der die Stücke seit 1912 umfasste.
                     \emph{Die gesammelten Werke}\pwindex{Schnitzler, Arthur 15.\,5.\,1862 Wien – 21.\,10.\,1931 ebd.@\textsc{Schnitzler, Arthur} (15.\,5.\,1862 Wien – 21.\,10.\,1931 ebd.), \emph{Schriftsteller, Mediziner}!Gesammelte Werke@\strich\emph{Gesammelte Werke}|pwk} sind nicht in Hofmannsthals\pwindex{Hofmannsthal, Hugo von 1.\,2.\,1874 Wien – 15.\,7.\,1929 Rodaun@\textsc{Hofmannsthal, Hugo von} (1.\,2.\,1874 Wien – 15.\,7.\,1929 Rodaun), \emph{Schriftsteller}|pwk} Nachlass mit seiner Bibliothek
                  überliefert.}}}\label{K_L02408-2} deren ich 4 habe, finde ich diese Einacterreihe nicht! – Zum
               Ersatz habe ich da{\geminationn} das »Weite Land\pwindex{Schnitzler, Arthur 15.\,5.\,1862 Wien – 21.\,10.\,1931 ebd.@\textsc{Schnitzler, Arthur} (15.\,5.\,1862 Wien – 21.\,10.\,1931 ebd.), \emph{Schriftsteller, Mediziner}!weite Land. Tragikomödie in fünf Akten@\strich\emph{Das weite Land. Tragikomödie in fünf Akten}|pw}« gelesen u. mit sehr großem Eindruck. Sie haben damals offenbar
               alles \uline{Detail} sehr \label{K_L02408-3v}\edtext{eindrucksvoll vorgelesen}{\lemma{\textnormal{\emph{eindrucksvoll vorgelesen}}}\Cendnote{\textnormal{Sofern sie stattgefunden hat, lässt sich diese Lesung nicht
                  datieren.}}}\label{K_L02408-3}, auf der Bühne habe ich es nie gesehen, u. so war mir nicht
               gegenwärtig gewesen, wie sehr dieses complexe Ganze durch die erstaunliche Gestalt
               des Hofreiter\pwindex{Schnitzler, Arthur 15.\,5.\,1862 Wien – 21.\,10.\,1931 ebd.@\textsc{Schnitzler, Arthur} (15.\,5.\,1862 Wien – 21.\,10.\,1931 ebd.), \emph{Schriftsteller, Mediziner}!weite Land. Tragikomödie in fünf Akten@\strich\emph{Das weite Land. Tragikomödie in fünf Akten}|pwv}{ }\uline{zusa{\geminationm}engehalten} wird. –
               Das \label{T_L02408-1v}\edtext{ganze genre gehört nur Ihnen, u.
               ist höchst interressant}{\lemma{\textnormal{\emph{ganze … interressant}}}\Cendnote{\textnormal{quer am linken
                  Rand}}}\label{T_L02408-1}\pend
           
\pstart
           \label{T_L02408-2v}\edtext{Von Herzen Ihr{\\[\baselineskip]}\spacefill\mbox{Hugo.}}{\lemma{\textnormal{\emph{Von Herzen IhrHugo.}}}\Cendnote{\textnormal{Grußformel quer am rechten Rand}}}\label{T_L02408-2}\pend
           \leftskip=0em{}
\pstart
           \noindent{}\label{T_L02408-3v}\edtext{PS. Eben finde ich B\textsuperscript{d} V\pwindex{Schnitzler, Arthur 15.\,5.\,1862 Wien – 21.\,10.\,1931 ebd.@\textsc{Schnitzler, Arthur} (15.\,5.\,1862 Wien – 21.\,10.\,1931 ebd.), \emph{Schriftsteller, Mediziner}!Theaterstücke. Ergänzungsband V@\strich\emph{Die Theaterstücke. Ergänzungsband V}|pw} der Theaterstücke! Er war
                     verstellt.}{\lemma{\textnormal{\emph{PS. … verstellt.}}}\Cendnote{\textnormal{quer am linken Rand der
                     ersten Seite}}}\label{T_L02408-3}\pend
           \selectlanguage{ngerman}\endnumbering\briefempfaengerindex{Schnitzler, Arthur@\textsc{Schnitzler, Arthur}!zzzHofmannsthal, Hugo von@\emph{von Hugo von Hofmannsthal}!1924-01-181@{18. 1. 1924}|)be}\mylabel{L02408h}  \newcommand{\dateiname}{L02408}\newcommand{\titel}{Hugo Hofmannsthal an Arthur Schnitzler, 18. 1. 1924}\newcommand{\editorInnen}{Martin Anton Müller und Gerd-Hermann Susen}%% latex-leseansicht-abspann.tex
%% Abspann für die Leseansicht.
%% Der Schalter \ifkorrekturansicht ist bereits durch den Vorspann gesetzt.

%% latex-abspann.tex
%% Gemeinsamer Abspann für Korrekturansicht und Leseansicht.
%% Setzt den Schalter \ifkorrekturansicht voraus (gesetzt in den
%% einbindenden Dateien latex-korrekturansicht-abspann.tex bzw.
%% latex-leseansicht-abspann.tex).
%% ---------------------------------------------------------------

\normalsize

% Das esempio-Environment wird nur in der Leseansicht benötigt
\ifkorrekturansicht\else
\newenvironment{esempio}[3]%
{
    \vspace{1.5ex}
    \rlap{\underline{#1}}
    \par
    \setlength{\parindent}{0cm}
    \nopagebreak
    \leftskip=#2cm
    \rightskip=#3cm
}
{
    \par
}
\fi

\doendnotes{C}
\bigskip
\vfill

\clearpage

\footnotesize

\ifkorrekturansicht
  \lohead{\textsc{register}}
\fi

% theindex-Environment neu definieren ohne reledmac
\makeatletter
\renewenvironment{theindex}{%
  \ifkorrekturansicht
    \section*{\indexname}%
  \else
    \subsubsection*{Index der erwähnten Entitäten}%
  \fi
  \setlength{\parindent}{0pt}%
  \setlength{\parskip}{0pt plus 0.3pt}%
  \let\item\@idxitem
}{%
  \ifkorrekturansicht\clearpage\fi
}
\makeatother

\IfFileExists{\jobname-pw.ind}{\input{\jobname-pw.ind}}{}

% Quellenangabe nur in der Leseansicht
\ifkorrekturansicht\else
% Fallback-Definitionen, falls die .tex-Datei \titel etc. nicht gesetzt hat
\providecommand{\titel}{}
\providecommand{\editorInnen}{}
\providecommand{\dateiname}{\jobname}

\vspace{3cm}

\vfill

\footnotesize
\textsc{Quelle}: \titel. Herausgegeben von {\editorInnen}. In: \emph{Arthur Schnitzler: Briefwechsel mit Autorinnen und Autoren}.
 Digitale Edition, https://schnitzler-briefe.acdh.oeaw.ac.at/{\dateiname}.html (Stand \today)
\fi

\end{document}


