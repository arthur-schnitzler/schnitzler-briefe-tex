%% latex-korrekturansicht-vorspann.tex
%% Vorspann für die Korrekturansicht.
%% Lädt die gemeinsame Datei latex-vorspann.tex mit gesetztem Schalter.

\newif\ifkorrekturansicht
\korrekturansichttrue

\input{../tex-inputs/latex-vorspann}


\section[Hugo Hofmannsthal an Arthur Schnitzler, 18. 1. 1924]{L02408 Hugo Hofmannsthal an Arthur Schnitzler, 18. 1. 1924}
\nopagebreak\mylabel{L02408v}
\rehead{ }\normalsize\beginnumbering\briefempfaengerindex{Schnitzler, Arthur@\textsc{Schnitzler, Arthur}!zzzHofmannsthal, Hugo von@\emph{von Hugo von Hofmannsthal}!1924-01-181@{18. 1. 1924}|(be}
\toendnotes[C]{\smallbreak\pagebreak[2]}\Standort{CUL, Schnitzler, B 43.}
\physDesc{Postkarte, 898 Zeichen
\newline{}Handschrift: schwarze Tinte, lateinische Kurrent
\newline{}Versand: Stempel: »\nobreak{}\oindex{Rodaun@\textbf{Rodaun}, \emph{A.ADM4}|pwk}\textcolor{gray}{Rod}{[}aun{]}\nobreak{}«.  
\newline{}Ordnung: 1) mit Bleistift von unbekannter Hand nummeriert: »\strikeout{384}«  2) mit Bleistift von unbekannter Hand nummeriert:
                                    »373«}
\buchAbdrucke{\weitereDrucke{Hugo von Hofmannsthal, Arthur Schnitzler: \emph{Briefwechsel}. Frankfurt am Main: \emph{S. Fischer} 1964, S. 298.} }\toendnotes[C]{\smallbreak}\pstart{}{\pb}Herrn D\textsuperscript{r} Arthur Schnitzler\pend{}\pstart{}Wien\oindex{Wien@\textbf{Wien}, \emph{A.ADM2}|pw}\pend{}\pstart{}XVIII Sternwartestrasse 71\oindex{Sternwartestrasse 71@\textbf{Sternwartestraße 71}, \emph{Wohngebäude (K.WHS)}|pw}\pend{}{\bigskip}\vspace{1em}
\pstart
           \textcolor{gray}{\textbf{{\pb}\textsc{Rodaun}\oindex{Rodaun@\textbf{Rodaun}, \emph{A.ADM4}|pw}}}\hfill 18 I 24.\pend
           
\pstart
           \textcolor{gray}{\textbf{B. WIEN\oindex{Wien@\textbf{Wien}, \emph{A.ADM2}|pw}}}\pend
           
\pstart{}mein lieber Arthur\pend\vspace{0.5em}
\pstart
           um unser \label{K_L02408-1v}\edtext{Gespräch}{\lemma{\textnormal{\emph{Gespräch}}}\Cendnote{\textnormal{Vgl. A. S.: \emph{Tagebuch}, 11. 1. 1924.
               }}}\label{K_L02408-1} noch für mich allein zu verlängern, wollte ich gestern abends die »Große Scene\pwindex{Grosse Szene@\emph{Große Szene}|pw}« lesen – aber ich muss durch ein
               Versehen seinerzeit diesen Band (Comödie der
                  Worte\pwindex{Komoedie der Worte. Drei Einakter@\emph{Komödie der Worte. Drei Einakter}|pw}) {\pb}nicht beko{\geminationm}en haben! Haben Sie vielleicht ein entbehrliches
               Exemplar? Nämlich auch in meinen Bänden \introOben{}Ihrer\introOben{}{ }\label{K_L02408-2v}\edtext{ges. Theaterstücke\pwindex{Gesammelte Werke@\emph{Gesammelte Werke}|pwv}}{\lemma{\textnormal{\emph{ges. Theaterstücke}}}\Cendnote{\textnormal{1912 waren \emph{Die gesammelten
                     Werke}\pwindex{Gesammelte Werke@\emph{Gesammelte Werke}|pwk} mit vier Bänden \emph{Die
                     Theaterstücke}\pwindex{Theaterstuecke@\emph{Die Theaterstücke}|pwk} erschienen. Anlässlich des 60. Geburtstages wurde 1922 die
                  Ausgabe um einen Ergänzungsband\pwindex{Theaterstuecke. Ergaenzungsband V@\emph{Die Theaterstücke. Ergänzungsband V}|pwkv} erweitert, der die Stücke seit 1912 umfasste.
                     \emph{Die gesammelten Werke}\pwindex{Gesammelte Werke@\emph{Gesammelte Werke}|pwk} sind nicht in Hofmannsthals\pwindex{Hofmannsthal, Hugo von 1874-02-01 – 1929-07-15@\textsc{Hofmannsthal, Hugo von} (1874-02-01 – 1929-07-15), \emph{Schriftsteller/Schriftstellerin}|pwk} Nachlass mit seiner Bibliothek
                  überliefert.}}}\label{K_L02408-2} deren ich 4 habe, finde ich diese Einacterreihe nicht! – Zum
               Ersatz habe ich da{\geminationn} das »Weite Land\pwindex{weite Land. Tragikomoedie in fuenf Akten@\emph{Das weite Land. Tragikomödie in fünf Akten}|pw}« gelesen u. mit sehr großem Eindruck. Sie haben damals offenbar
               alles \uline{Detail} sehr \label{K_L02408-3v}\edtext{eindrucksvoll vorgelesen}{\lemma{\textnormal{\emph{eindrucksvoll vorgelesen}}}\Cendnote{\textnormal{Sofern sie stattgefunden hat, lässt sich diese Lesung nicht
                  datieren.}}}\label{K_L02408-3}, auf der Bühne habe ich es nie gesehen, u. so war mir nicht
               gegenwärtig gewesen, wie sehr dieses complexe Ganze durch die erstaunliche Gestalt
               des Hofreiter\pwindex{weite Land. Tragikomoedie in fuenf Akten@\emph{Das weite Land. Tragikomödie in fünf Akten}|pwv}{ }\uline{zusa{\geminationm}engehalten} wird. –
               Das \label{T_L02408-1v}\edtext{ganze genre gehört nur Ihnen, u.
               ist höchst interressant}{\lemma{\textnormal{\emph{ganze … interressant}}}\Cendnote{\textnormal{quer am linken
                  Rand}}}\label{T_L02408-1}\pend
           
\pstart
           \label{T_L02408-2v}\edtext{Von Herzen Ihr{\\[\baselineskip]}\spacefill\mbox{Hugo.}}{\lemma{\textnormal{\emph{Von Herzen IhrHugo.}}}\Cendnote{\textnormal{Grußformel quer am rechten Rand}}}\label{T_L02408-2}\pend
           \leftskip=0em{}
\pstart
           \noindent{}\label{T_L02408-3v}\edtext{PS. Eben finde ich B\textsuperscript{d} V\pwindex{Theaterstuecke. Ergaenzungsband V@\emph{Die Theaterstücke. Ergänzungsband V}|pw} der Theaterstücke! Er war
                     verstellt.}{\lemma{\textnormal{\emph{PS. … verstellt.}}}\Cendnote{\textnormal{quer am linken Rand der
                     ersten Seite}}}\label{T_L02408-3}\pend
           \selectlanguage{ngerman}\endnumbering\briefempfaengerindex{Schnitzler, Arthur@\textsc{Schnitzler, Arthur}!zzzHofmannsthal, Hugo von@\emph{von Hugo von Hofmannsthal}!1924-01-181@{18. 1. 1924}|)be}\mylabel{L02408h}  \normalsize

\doendnotes{C}
\bigskip
\vfill

\clearpage

\footnotesize

\lohead{\textsc{register}}

% Definiere theindex-Environment komplett neu ohne reledmac
\makeatletter
\renewenvironment{theindex}{%
  \section*{\indexname}%
  \setlength{\parindent}{0pt}%
  \setlength{\parskip}{0pt plus 0.3pt}%
  \let\item\@idxitem
}{%
  \clearpage
}
\makeatother

\IfFileExists{\jobname-pw.ind}{\input{\jobname-pw.ind}}{}

\end{document}

      