%% latex-leseansicht-vorspann.tex
%% Vorspann für die Leseansicht.
%% Lädt die gemeinsame Datei latex-vorspann.tex mit nicht gesetztem Schalter.

\newif\ifkorrekturansicht
\korrekturansichtfalse

\input{../tex-inputs/latex-vorspann}


               \section[Hugo Hofmannsthal an Arthur Schnitzler, 18. 1. 1924]{ Hugo Hofmannsthal an Arthur Schnitzler, 18. 1. 1924}\nopagebreak\mylabel{v}\rehead{ }\begin{ledgroupsized}[t]{13cm}\normalsize\beginnumbering\briefempfaengerindex{Schnitzler, Arthur@\textsc{Schnitzler, Arthur}!zzzHofmannsthal, Hugo von@\emph{von Hugo von Hofmannsthal}!1924-01-181@{18. 1. 1924}|(be} \toendnotes[C]{\smallbreak\pagebreak[2]} \Standort{CUL, Schnitzler, B 43.}
\physDesc{Postkarte
\newline{}Handschrift: schwarze Tinte, lateinische Kurrent\newline{}Versand: Stempel: »\nobreak{}\oindex{Rodaun@\textbf{Rodaun}|pwk}\textcolor{gray}{Rod}{[}aun{]}\nobreak{}«.  \newline{}Ordnung: 1) mit Bleistift von unbekannter Hand nummeriert: »\strikeout{384}« 2) mit Bleistift von unbekannter Hand nummeriert:
                                    »373«}\buchAbdrucke{\weitereDrucke{Hugo von Hofmannsthal, Arthur Schnitzler: \emph{Briefwechsel}. Hg. Therese Nickl und Heinrich Schnitzler. Frankfurt am Main: \emph{S. Fischer} 1964, S. 298.} }\toendnotes[C]{\smallbreak}\pstart{}{\pb}Herrn D\textsuperscript{r} Arthur Schnitzler\pend{}\pstart{}Wien\oindex{Wien@\textbf{Wien}|pw}\pend{}\pstart{}XVIII Sternwartestrasse 71\oindex{Sternwartestrasse@\textbf{Sternwartestraße}|pw}\pend{}{\bigskip}\pstart
           \noindent{}\textcolor{gray}{\textbf{{\pb}\textsc{Rodaun}\oindex{Rodaun@\textbf{Rodaun}|pw}}}\hfill 18 I 24.\pend
           \pstart
           \textcolor{gray}{\textbf{B. WIEN\oindex{Wien@\textbf{Wien}|pw}}}\pend
           \pstart{}mein lieber Arthur\pend\pstart
           um unser \label{K_L02408_1v}\edtext{Gespräch}{\lemma{\textnormal{\emph{Gespräch}}}\Cendnote{\textnormal{vgl. A. S.: \emph{Tagebuch}, 11. 1. 1924}}}\label{K_L02408_1h} noch für mich allein zu verlängern, wollte ich gestern abends die »Große Scene\pwindex{Schnitzler, Arthur 15.05.1862 – 21.10.1931@\textsc{Schnitzler, Arthur} (15.05.1862 – 21.10.1931), \emph{Schriftsteller, Mediziner}!Grosse Szene1915@\strich\emph{Große Szene} {[}1915{]}|pw}« lesen – aber ich muss durch ein
               Versehen seinerzeit diesen Band (Comödie der Worte\pwindex{Schnitzler, Arthur 15.05.1862 – 21.10.1931@\textsc{Schnitzler, Arthur} (15.05.1862 – 21.10.1931), \emph{Schriftsteller, Mediziner}!Komoedie der Worte. Drei Einakter1915@\strich\emph{Komödie der Worte. Drei Einakter} {[}1915{]}|pw})
                  {\pb}nicht beko{\geminationm}en haben! Haben Sie vielleicht ein entbehrliches
               Exemplar? Nämlich auch in meinen Bänden \introOben{}Ihrer\introOben{}{ }\label{K_L02408_2v}\edtext{ges. Theaterstücke\pwindex{Schnitzler, Arthur 15.05.1862 – 21.10.1931@\textsc{Schnitzler, Arthur} (15.05.1862 – 21.10.1931), \emph{Schriftsteller, Mediziner}!Gesammelte Werke1912 – 1922@\strich\emph{Gesammelte Werke} {[}1912 – 1922{]}|pwv}}{\lemma{\textnormal{\emph{ges. Theaterstücke}}}\Cendnote{\textnormal{1912 erschien \emph{Die gesammelten Werke}\pwindex{Schnitzler, Arthur 15.05.1862 – 21.10.1931@\textsc{Schnitzler, Arthur} (15.05.1862 – 21.10.1931), \emph{Schriftsteller, Mediziner}!Gesammelte Werke1912 – 1922@\strich\emph{Gesammelte Werke} {[}1912 – 1922{]}|pwk}
                  mit vier Bänden \emph{Die Theaterstücke}\pwindex{Schnitzler, Arthur 15.05.1862 – 21.10.1931@\textsc{Schnitzler, Arthur} (15.05.1862 – 21.10.1931), \emph{Schriftsteller, Mediziner}!Theaterstuecke1912 – 1922@\strich\emph{Die Theaterstücke} {[}1912 – 1922{]}|pwk}. Anlässlich des
                  60. Geburtstages wurde 1922 die Ausgabe um einen Ergänzungsband\pwindex{Schnitzler, Arthur 15.05.1862 – 21.10.1931@\textsc{Schnitzler, Arthur} (15.05.1862 – 21.10.1931), \emph{Schriftsteller, Mediziner}!Theaterstuecke. Ergaenzungsband V1922 – 1922@\strich\emph{Die Theaterstücke. Ergänzungsband V} {[}1922 – 1922{]}|pwkv} erweitert, der die Stücke
                  seit 1912 umfasste. \emph{Die gesammelten
                     Werke}\pwindex{Schnitzler, Arthur 15.05.1862 – 21.10.1931@\textsc{Schnitzler, Arthur} (15.05.1862 – 21.10.1931), \emph{Schriftsteller, Mediziner}!Gesammelte Werke1912 – 1922@\strich\emph{Gesammelte Werke} {[}1912 – 1922{]}|pwk} sind nicht in Hofmannsthal\pwindex{Hofmannsthal, Hugo von 01.02.1874 – 15.07.1929@\textsc{Hofmannsthal, Hugo von} (01.02.1874 – 15.07.1929), \emph{Schriftsteller}|pwk}s
                  Nachlass mit seiner Bibliothek überliefert.}}}\label{K_L02408_2h} deren ich 4 habe, finde ich
               diese Einacterreihe nicht! – Zum Ersatz habe ich da{\geminationn} das
                  »Weite Land\pwindex{Schnitzler, Arthur 15.05.1862 – 21.10.1931@\textsc{Schnitzler, Arthur} (15.05.1862 – 21.10.1931), \emph{Schriftsteller, Mediziner}!weite Land. Tragikomoedie in fuenf Akten1910-10-20@\strich\emph{Das weite Land. Tragikomödie in fünf Akten} {[}1910-10-20{]}|pw}« gelesen u. mit sehr großem
               Eindruck. Sie haben damals offenbar alles \uline{Detail} sehr
                  \label{K_L02408_3v}\edtext{eindrucksvoll vorgelesen}{\lemma{\textnormal{\emph{eindrucksvoll vorgelesen}}}\Cendnote{\textnormal{Sofern sie stattgefunden hat, lässt sich
                  diese Lesung nicht datieren.}}}\label{K_L02408_3h}, auf der Bühne habe ich es nie gesehen, u. so
               war mir nicht gegenwärtig gewesen, wie sehr dieses complexe Ganze durch die
               erstaunliche Gestalt des Hofreiter\pwindex{Schnitzler, Arthur 15.05.1862 – 21.10.1931@\textsc{Schnitzler, Arthur} (15.05.1862 – 21.10.1931), \emph{Schriftsteller, Mediziner}!weite Land. Tragikomoedie in fuenf Akten1910-10-20@\strich\emph{Das weite Land. Tragikomödie in fünf Akten} {[}1910-10-20{]}|pwv}{ }\uline{zusa{\geminationm}engehalten} wird. –
               Das \label{T_L02408_1v}\edtext{ganze genre gehört nur Ihnen, u.
               ist höchst interressant}{\lemma{\textnormal{\emph{ganze … interressant}}}\Cendnote{\textnormal{quer am linken
                  Rand}}}\label{T_L02408_1h}\pend
           \pstart
           \label{T_L02408_2v}\edtext{Von Herzen Ihr{\\[\baselineskip]}\spacefill\mbox{Hugo.}}{\lemma{\textnormal{\emph{Von Herzen IhrHugo.}}}\Cendnote{\textnormal{Grußformel quer am rechten Rand}}}\label{T_L02408_2h}\pend
           \leftskip=0em{}\pstart
           \noindent{}\label{T_L02408_3v}\edtext{PS. Eben finde ich B\textsuperscript{d} V\pwindex{Schnitzler, Arthur 15.05.1862 – 21.10.1931@\textsc{Schnitzler, Arthur} (15.05.1862 – 21.10.1931), \emph{Schriftsteller, Mediziner}!Theaterstuecke. Ergaenzungsband V1922 – 1922@\strich\emph{Die Theaterstücke. Ergänzungsband V} {[}1922 – 1922{]}|pw} der Theaterstücke! Er war
                     verstellt.}{\lemma{\textnormal{\emph{PS. … verstellt.}}}\Cendnote{\textnormal{quer am linken Rand der
                     ersten Seite}}}\label{T_L02408_3h}\pend
           \endnumbering\briefempfaengerindex{Schnitzler, Arthur@\textsc{Schnitzler, Arthur}!zzzHofmannsthal, Hugo von@\emph{von Hugo von Hofmannsthal}!1924-01-181@{18. 1. 1924}|)be}\mylabel{h}\end{ledgroupsized}  \newcommand{\dateiname}{L02408}\newcommand{\titel}{Hugo Hofmannsthal an Arthur Schnitzler, 18. 1. 1924}\newcommand{\editorInnen}{Martin Anton Müller und Gerd-Hermann Susen}
            \footnotesize
\begin{ledgroupsized}[t]{11.5cm}
\doendnotes{C}
\end{ledgroupsized}
         %% latex-leseansicht-abspann.tex
%% Abspann für die Leseansicht.
%% Der Schalter \ifkorrekturansicht ist bereits durch den Vorspann gesetzt.

%% latex-abspann.tex
%% Gemeinsamer Abspann für Korrekturansicht und Leseansicht.
%% Setzt den Schalter \ifkorrekturansicht voraus (gesetzt in den
%% einbindenden Dateien latex-korrekturansicht-abspann.tex bzw.
%% latex-leseansicht-abspann.tex).
%% ---------------------------------------------------------------

\normalsize

% Das esempio-Environment wird nur in der Leseansicht benötigt
\ifkorrekturansicht\else
\newenvironment{esempio}[3]%
{
    \vspace{1.5ex}
    \rlap{\underline{#1}}
    \par
    \setlength{\parindent}{0cm}
    \nopagebreak
    \leftskip=#2cm
    \rightskip=#3cm
}
{
    \par
}
\fi

\doendnotes{C}
\bigskip
\vfill

\clearpage

\footnotesize

\ifkorrekturansicht
  \lohead{\textsc{register}}
\fi

% theindex-Environment neu definieren ohne reledmac
\makeatletter
\renewenvironment{theindex}{%
  \ifkorrekturansicht
    \section*{\indexname}%
  \else
    \subsubsection*{Index der erwähnten Entitäten}%
  \fi
  \setlength{\parindent}{0pt}%
  \setlength{\parskip}{0pt plus 0.3pt}%
  \let\item\@idxitem
}{%
  \ifkorrekturansicht\clearpage\fi
}
\makeatother

\IfFileExists{\jobname-pw.ind}{\input{\jobname-pw.ind}}{}

% Quellenangabe nur in der Leseansicht
\ifkorrekturansicht\else
% Fallback-Definitionen, falls die .tex-Datei \titel etc. nicht gesetzt hat
\providecommand{\titel}{}
\providecommand{\editorInnen}{}
\providecommand{\dateiname}{\jobname}

\vspace{3cm}

\vfill

\footnotesize
\textsc{Quelle}: \titel. Herausgegeben von {\editorInnen}. In: \emph{Arthur Schnitzler: Briefwechsel mit Autorinnen und Autoren}.
 Digitale Edition, https://schnitzler-briefe.acdh.oeaw.ac.at/{\dateiname}.html (Stand \today)
\fi

\end{document}


      