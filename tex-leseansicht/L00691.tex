%% latex-leseansicht-vorspann.tex
%% Vorspann für die Leseansicht.
%% Lädt die gemeinsame Datei latex-vorspann.tex mit nicht gesetztem Schalter.

\newif\ifkorrekturansicht
\korrekturansichtfalse

\input{../tex-inputs/latex-vorspann}


               \section[Max Burckhard an Arthur Schnitzler, {[}23. 6. 1897{]}]{ Max Burckhard an Arthur Schnitzler, {[}23. 6. 1897{]}}\nopagebreak\mylabel{v}\rehead{ }\begin{ledgroupsized}[t]{13cm}\normalsize\beginnumbering\briefempfaengerindex{Schnitzler, Arthur@\textsc{Schnitzler, Arthur}!zzzBurckhard, Max Eugen@\emph{von Max Eugen Burckhard}!1897-06-232@{{[}23. 6. 1897{]}}|(be} \toendnotes[C]{\smallbreak\pagebreak[2]} \Standort{CUL, Schnitzler, B 20.}
\physDesc{Brief, 1 Blatt, 1 Seite
\newline{}Handschrift: schwarze Tinte, deutsche Kurrent
\newline{}Schnitzler: mit Bleistift datiert: »23/6 97« \newline{}Ordnung: mit Bleistift von unbekannter Hand nummeriert:
                                        »9« }\toendnotes[C]{\smallbreak}\pstart{}{\pb}Sehr verehrter lieber Herr
                        Doctor!\pend\pstart
           Ich bin ſo frei im Auftrage des Autors beiliegendes \label{K_L00691_1v}\edtext{Stück\pwindex{ s Katherl. Volksstueck in fuenf Aufzuegen1897-11-25@\emph{’s Katherl. Volksstück in fünf Aufzügen} {[}1897-11-25{]}|pwuv}\pwindex{Burckhard, Max Eugen 14.07.1854 – 16.03.1912@\textsc{Burckhard, Max Eugen} (14.07.1854 – 16.03.1912), \emph{Schriftsteller, Rechtswissenschaftler, Theaterleiter}!Buergermeisterwahl. Laendliche Comoedie in vier Acten1897-11-20@\strich\emph{Die Bürgermeisterwahl. Ländliche Comödie in vier Acten} {[}1897-11-20{]}|pwuv}}{\lemma{\textnormal{\emph{Stück}}}\Cendnote{\textnormal{Es könnte sich um \emph{Die Bürgermeisterwahl}\pwindex{Burckhard, Max Eugen 14.07.1854 – 16.03.1912@\textsc{Burckhard, Max Eugen} (14.07.1854 – 16.03.1912), \emph{Schriftsteller, Rechtswissenschaftler, Theaterleiter}!Buergermeisterwahl. Laendliche Comoedie in vier Acten1897-11-20@\strich\emph{Die Bürgermeisterwahl. Ländliche Comödie in vier Acten} {[}1897-11-20{]}|pwk} oder \emph{s’Katherl}\pwindex{ s Katherl. Volksstueck in fuenf Aufzuegen1897-11-25@\emph{’s Katherl. Volksstück in fünf Aufzügen} {[}1897-11-25{]}|pwk} von Max Burckhard\pwindex{Burckhard, Max Eugen 14.07.1854 – 16.03.1912@\textsc{Burckhard, Max Eugen} (14.07.1854 – 16.03.1912), \emph{Schriftsteller, Rechtswissenschaftler, Theaterleiter}|pwk}
                        handeln, die am 20. respektive 25. 11. 1897 in Wien\oindex{Wien@\textbf{Wien}|pwk} ihre Uraufführung erlebten.}}}\label{K_L00691_1h} zu
                    überſenden.\pend
           \pstart
           In herzlicher Verehrung{\\[\baselineskip]}\spacefill\mbox{D\textsuperscript{r}Burckhard}\pend
           \leftskip=0em{}\endnumbering\briefempfaengerindex{Schnitzler, Arthur@\textsc{Schnitzler, Arthur}!zzzBurckhard, Max Eugen@\emph{von Max Eugen Burckhard}!1897-06-232@{{[}23. 6. 1897{]}}|)be}\mylabel{h}\end{ledgroupsized}  \newcommand{\dateiname}{L00691}\newcommand{\titel}{Max Burckhard an Arthur Schnitzler, [23. 6. 1897]}\newcommand{\editorInnen}{Martin Anton Müller und Gerd-Hermann Susen}%% latex-leseansicht-abspann.tex
%% Abspann für die Leseansicht.
%% Der Schalter \ifkorrekturansicht ist bereits durch den Vorspann gesetzt.

%% latex-abspann.tex
%% Gemeinsamer Abspann für Korrekturansicht und Leseansicht.
%% Setzt den Schalter \ifkorrekturansicht voraus (gesetzt in den
%% einbindenden Dateien latex-korrekturansicht-abspann.tex bzw.
%% latex-leseansicht-abspann.tex).
%% ---------------------------------------------------------------

\normalsize

% Das esempio-Environment wird nur in der Leseansicht benötigt
\ifkorrekturansicht\else
\newenvironment{esempio}[3]%
{
    \vspace{1.5ex}
    \rlap{\underline{#1}}
    \par
    \setlength{\parindent}{0cm}
    \nopagebreak
    \leftskip=#2cm
    \rightskip=#3cm
}
{
    \par
}
\fi

\doendnotes{C}
\bigskip
\vfill

\clearpage

\footnotesize

\ifkorrekturansicht
  \lohead{\textsc{register}}
\fi

% theindex-Environment neu definieren ohne reledmac
\makeatletter
\renewenvironment{theindex}{%
  \ifkorrekturansicht
    \section*{\indexname}%
  \else
    \subsubsection*{Index der erwähnten Entitäten}%
  \fi
  \setlength{\parindent}{0pt}%
  \setlength{\parskip}{0pt plus 0.3pt}%
  \let\item\@idxitem
}{%
  \ifkorrekturansicht\clearpage\fi
}
\makeatother

\IfFileExists{\jobname-pw.ind}{\input{\jobname-pw.ind}}{}

% Quellenangabe nur in der Leseansicht
\ifkorrekturansicht\else
% Fallback-Definitionen, falls die .tex-Datei \titel etc. nicht gesetzt hat
\providecommand{\titel}{}
\providecommand{\editorInnen}{}
\providecommand{\dateiname}{\jobname}

\vspace{3cm}

\vfill

\footnotesize
\textsc{Quelle}: \titel. Herausgegeben von {\editorInnen}. In: \emph{Arthur Schnitzler: Briefwechsel mit Autorinnen und Autoren}.
 Digitale Edition, https://schnitzler-briefe.acdh.oeaw.ac.at/{\dateiname}.html (Stand \today)
\fi

\end{document}


      