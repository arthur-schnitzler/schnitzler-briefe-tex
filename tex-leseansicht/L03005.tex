%% latex-leseansicht-vorspann.tex
%% Vorspann für die Leseansicht.
%% Lädt die gemeinsame Datei latex-vorspann.tex mit nicht gesetztem Schalter.

\newif\ifkorrekturansicht
\korrekturansichtfalse

\input{../tex-inputs/latex-vorspann}


\section[ Arthur Schnitzler an Felix Salten, 16. 5. 1906]{L03005 Arthur Schnitzler an Felix Salten,  16. 5. 1906}
\nopagebreak\mylabel{L03005v}
\rehead{ }\normalsize\beginnumbering\briefempfaengerindex{Salten, Felix@\textsc{Salten, Felix}!zzzSchnitzler, Arthur@\emph{von Arthur Schnitzler}!1906-05-161@{16. 5. 1906}|(be}
\toendnotes[C]{\smallbreak\pagebreak[2]}
\correspDesc{Versand  durch Arthur Schnitzler am 16. 5. 1906 in Wien
\newline{}Erhalt  durch Felix Salten im Zeitraum [17. 5. 1906
                  – 21. 5. 1906?] in Berlin}\toendnotes[C]{\smallbreak}
\Standort{Wienbibliothek im Rathaus, ZPH 1681, 2.1.516.}
\physDesc{Brief, 2 Blätter, 8 Seiten, 4277 Zeichen
\newline{}Handschrift: schwarze Tinte, deutsche Kurrent
\newline{}Ordnung: mit Bleistift von unbekannter Hand Nummerierung der Doppelseiten des
                                 Konvoluts: »12«–»15« }\toendnotes[C]{\smallbreak}
\pstart
           {\pb}\textcolor{gray}{\textbf{Dr. Arthur Schnitzler}}\hfill 16. Mai 906\pend
           
\pstart
           \textcolor{gray}{\textbf{Wien XVIII. Spoettelgasse 7\oindex{Wien@\textbf{Wien}!XVIII., Währing@\textbf{XVIII., Währing}!Edmund-Weiß-Gasse 7@\textbf{Edmund-Weiß-Gasse 7}, \emph{Wohngebäude}|pw}.}}\pend
           \vspace{0.5em}
\pstart
           lieber, beim Nachhauſeko{\geminationm}en aus \label{K_L03005-1v}\edtext{Theater\oindex{Wien@\textbf{Wien}!VI., Mariahilf@\textbf{VI., Mariahilf}!Theater an der Wien@\textbf{Theater an der Wien}, \emph{Theater}|pwv} und Hotel\oindex{Wien@\textbf{Wien}!I., Innere Stadt@\textbf{I., Innere Stadt}!Meissl {\kaufmannsund} Schadn@\textbf{Meissl {\kaufmannsund} Schadn}, \emph{Hotel}|pwv}}{\lemma{\textnormal{\emph{Theater und Hotel}}}\Cendnote{\textnormal{Siehe A. S.: \emph{Tagebuch}, 15. 5. 1906.
               }}}\label{K_L03005-1} hab \label{T_L03005-1v}\edtext{ich}{\lemma{\textnormal{\emph{ich}}}\Cendnote{\textnormal{In der Vorlage steht: »ich ich«.}}}\label{T_L03005-1} Ihren
               kurzen aber klingenden \label{K_L03005-2v}\edtext{Brief}{\lemma{\textnormal{\emph{Brief}}}\Cendnote{\textnormal{XXXX Auszeichnungsfehler: Dokument L03474 nicht gefunden.
               }}}\label{K_L03005-2} vorgefunden und mich{ }ſehr damit gefreut. Es mußte für mich freilich nicht
               gerade der Einſ. Weg\pwindex{Schnitzler, Arthur 15.\,5.\,1862 Wien – 21.\,10.\,1931 ebd.@\textsc{Schnitzler, Arthur} (15.\,5.\,1862 Wien – 21.\,10.\,1931 ebd.), \emph{Schriftsteller, Mediziner}!einsame Weg. Schauspiel in fünf Akten@\strich\emph{Der einsame Weg. Schauspiel in fünf Akten}|pw} kommen, um mich Ihr
               Fernſein{ }ſchmerzlich empfinden zu laſſen. Der Abend{ }geſtern\eventindex{Theater an der Wien@\textbf{Theater an der Wien}!Aufführung von Der einsame Weg, 15.5.1906@Aufführung von Der einsame Weg, 15.5.1906|pwv} iſt überraſchend gut ausgefallen: jedenfalls
               war er äußerlich der{ }ſtärkſte Erfolg meiner Theaterlaufbahn. Völlige Stu{\geminationm}heit nach dem erſten Akt\pwindex{Schnitzler, Arthur 15.\,5.\,1862 Wien – 21.\,10.\,1931 ebd.@\textsc{Schnitzler, Arthur} (15.\,5.\,1862 Wien – 21.\,10.\,1931 ebd.), \emph{Schriftsteller, Mediziner}!einsame Weg. Schauspiel in fünf Akten@\strich\emph{Der einsame Weg. Schauspiel in fünf Akten}|pwv}, wahre »Stürme« nach 2.\pwindex{Schnitzler, Arthur 15.\,5.\,1862 Wien – 21.\,10.\,1931 ebd.@\textsc{Schnitzler, Arthur} (15.\,5.\,1862 Wien – 21.\,10.\,1931 ebd.), \emph{Schriftsteller, Mediziner}!einsame Weg. Schauspiel in fünf Akten@\strich\emph{Der einsame Weg. Schauspiel in fünf Akten}|pwv}, 3.\pwindex{Schnitzler, Arthur 15.\,5.\,1862 Wien – 21.\,10.\,1931 ebd.@\textsc{Schnitzler, Arthur} (15.\,5.\,1862 Wien – 21.\,10.\,1931 ebd.), \emph{Schriftsteller, Mediziner}!einsame Weg. Schauspiel in fünf Akten@\strich\emph{Der einsame Weg. Schauspiel in fünf Akten}|pwv}, gedämpft nach dem 4\pwindex{Schnitzler, Arthur 15.\,5.\,1862 Wien – 21.\,10.\,1931 ebd.@\textsc{Schnitzler, Arthur} (15.\,5.\,1862 Wien – 21.\,10.\,1931 ebd.), \emph{Schriftsteller, Mediziner}!einsame Weg. Schauspiel in fünf Akten@\strich\emph{Der einsame Weg. Schauspiel in fünf Akten}|pwv},
               wieder{ }ſehr{ }ſtark {\pb}nach dem 5. Akt\pwindex{Schnitzler, Arthur 15.\,5.\,1862 Wien – 21.\,10.\,1931 ebd.@\textsc{Schnitzler, Arthur} (15.\,5.\,1862 Wien – 21.\,10.\,1931 ebd.), \emph{Schriftsteller, Mediziner}!einsame Weg. Schauspiel in fünf Akten@\strich\emph{Der einsame Weg. Schauspiel in fünf Akten}|pwv}. Baſſermann\pwindex{Bassermann, Albert 7.\,9.\,1867 Mannheim – 15.\,5.\,1952 Atlantischer Ozean@\textsc{Bassermann, Albert} (7.\,9.\,1867 Mannheim – 15.\,5.\,1952 Atlantischer Ozean), \emph{Schauspieler}|pw} anfangs etwas bläßlich, am Schluſs unvergleichlich. \label{K_L03005-3v}\edtext{Reicher\pwindex{Reicher, Emanuel 18.\,6.\,1849 Bochnia – 15.\,5.\,1924 Berlin@\textsc{Reicher, Emanuel} (18.\,6.\,1849 Bochnia – 15.\,5.\,1924 Berlin), \emph{Schauspieler}|pw} hat mich in gewiſſem Sinne angenehm
               enttäuſcht}{\lemma{\textnormal{\emph{Reicher … enttäuscht}}}\Cendnote{\textnormal{Vgl. XXXX Auszeichnungsfehler: Dokument L03419 nicht gefunden.
               }}}\label{K_L03005-3}. Im ganzen war er wohl unerträglich genug; aber die Leiſtung als ganzes
               war von einer gewiſſen Geſchloſſenheit,{ }ſo daſs man einen mehr menſchlichen als
               künſtleriſchen Widerwillen gegen die \label{K_L03005-4v}\edtext{Figur\pwindex{Schnitzler, Arthur 15.\,5.\,1862 Wien – 21.\,10.\,1931 ebd.@\textsc{Schnitzler, Arthur} (15.\,5.\,1862 Wien – 21.\,10.\,1931 ebd.), \emph{Schriftsteller, Mediziner}!einsame Weg. Schauspiel in fünf Akten@\strich\emph{Der einsame Weg. Schauspiel in fünf Akten}|pwv}}{\lemma{\textnormal{\emph{Figur}}}\Cendnote{\textnormal{Albert Bassermann\pwindex{Bassermann, Albert 7.\,9.\,1867 Mannheim – 15.\,5.\,1952 Atlantischer Ozean@\textsc{Bassermann, Albert} (7.\,9.\,1867 Mannheim – 15.\,5.\,1952 Atlantischer Ozean), \emph{Schauspieler}|pwk} spielte den Stephan von Sala\pwindex{Schnitzler, Arthur 15.\,5.\,1862 Wien – 21.\,10.\,1931 ebd.@\textsc{Schnitzler, Arthur} (15.\,5.\,1862 Wien – 21.\,10.\,1931 ebd.), \emph{Schriftsteller, Mediziner}!einsame Weg. Schauspiel in fünf Akten@\strich\emph{Der einsame Weg. Schauspiel in fünf Akten}|pwkv}.}}}\label{K_L03005-4}
               kriegte. – Seltſam{ }ſind doch Dramenſchicksale. Eine{ }ſolche Aufnahme \label{K_L03005-5v}\edtext{in Berlin\oindex{Berlin@\textbf{Berlin}, \emph{Hauptstadt}|pw} vor 2 ½ Jahren}{\lemma{\textnormal{\emph{in Berlin vor 2 ½ Jahren}}}\Cendnote{\textnormal{Uraufführung von \emph{Der einsame Weg}\pwindex{Schnitzler, Arthur 15.\,5.\,1862 Wien – 21.\,10.\,1931 ebd.@\textsc{Schnitzler, Arthur} (15.\,5.\,1862 Wien – 21.\,10.\,1931 ebd.), \emph{Schriftsteller, Mediziner}!einsame Weg. Schauspiel in fünf Akten@\strich\emph{Der einsame Weg. Schauspiel in fünf Akten}|pwk}\eventindex{Deutsches Theater Berlin@\textbf{Deutsches Theater Berlin}!Uraufführung von Der einsame Weg, 13.2.1904@Uraufführung von Der einsame Weg, 13.2.1904|pwk} am Deutschen Theater Berlin\oindex{Deutsches Theater Berlin@\textbf{Deutsches Theater Berlin}, \emph{Theater}|pwk} am 13. 2. 1904}}}\label{K_L03005-5} – und Ihre Profezeihung wäre erfüllt geweſen.\pend
           
\pstart
           – Den \label{K_L03005-6v}\edtext{Rehberg\pwindex{Salten, Felix 6.\,9.\,1869 Budapest – 8.\,10.\,1945 Zürich@\textsc{Salten, Felix} (6.\,9.\,1869 Budapest – 8.\,10.\,1945 Zürich), \emph{Schriftsteller, Journalist, Chefredakteur}!Herr Wenzel auf Rehberg. Novelle@\strich\emph{Herr Wenzel auf Rehberg. Novelle}|pw} hab ich in der Hinterbrühl\oindex{Hinterbrühl@\textbf{Hinterbrühl}, \emph{Hauptstadt}|pw}}{\lemma{\textnormal{\emph{Rehberg … Hinterbrühl}}}\Cendnote{\textnormal{Siehe A. S.: \emph{Tagebuch}, 8. 5. 1906.
               }}}\label{K_L03005-6} geleſen, wo wir höchſt angenehme \label{K_L03005-7v}\edtext{acht Tage im Hotel Radetzky\oindex{Hotel Radetzky@\textbf{Hotel Radetzky}, \emph{Hotel}|pw}}{\lemma{\textnormal{\emph{acht … Radetzky}}}\Cendnote{\textnormal{vom 7. 5. 1906 bis zum 14. 5. 1906}}}\label{K_L03005-7}{ }{\pb}gewohnt und \textsc{tennis}
               geſpielt haben (Einmal mit \label{K_L03005-8v}\edtext{Hugo\pwindex{Hofmannsthal, Hugo von 1.\,2.\,1874 Wien – 15.\,7.\,1929 Rodaun@\textsc{Hofmannsthal, Hugo von} (1.\,2.\,1874 Wien – 15.\,7.\,1929 Rodaun), \emph{Schriftsteller}|pw}, den ich im \textsc{\begin{otherlanguage}{english}single set\end{otherlanguage}} 6:4 ſchlug}{\lemma{\textnormal{\emph{Hugo, … set 6:4 schlug}}}\Cendnote{\textnormal{Siehe A. S.: \emph{Tagebuch}, 11. 5. 1906.
               }}}\label{K_L03005-8}!) – Es iſt ein glänzendes Ding\pwindex{Salten, Felix 6.\,9.\,1869 Budapest – 8.\,10.\,1945 Zürich@\textsc{Salten, Felix} (6.\,9.\,1869 Budapest – 8.\,10.\,1945 Zürich), \emph{Schriftsteller, Journalist, Chefredakteur}!Herr Wenzel auf Rehberg. Novelle@\strich\emph{Herr Wenzel auf Rehberg. Novelle}|pwv}, und es gibt vielleicht im ganzen darin nur 3–5 Stellen, bei denen mir
               im Stil irgend was wie ein falſcher Ton erſcheint. Doch möcht ichs, nach einem
               Zwiſchenraum von ein paar Wochen, noch einmal leſen, um mich{ }ſelber nachzuprüfen.
               Hingegen{ }ſage ich{ }ſchon heute mit Entſchiedenheit,
               daſs ich den vorletzten Abſatz\pwindex{Salten, Felix 6.\,9.\,1869 Budapest – 8.\,10.\,1945 Zürich@\textsc{Salten, Felix} (6.\,9.\,1869 Budapest – 8.\,10.\,1945 Zürich), \emph{Schriftsteller, Journalist, Chefredakteur}!Herr Wenzel auf Rehberg. Novelle@\strich\emph{Herr Wenzel auf Rehberg. Novelle}|pwv}
               fortwünſchte. Hier werden Zuſa{\geminationm}enhänge mit einer meinen
               Geſchmack{ }ſtörenden Deutlichkeit aufgezeigt; \strikeout{die}
                  Zuſa{\geminationm}enhänge, die im {\pb}Gang der Geſchichte\pwindex{Salten, Felix 6.\,9.\,1869 Budapest – 8.\,10.\,1945 Zürich@\textsc{Salten, Felix} (6.\,9.\,1869 Budapest – 8.\,10.\,1945 Zürich), \emph{Schriftsteller, Journalist, Chefredakteur}!Herr Wenzel auf Rehberg. Novelle@\strich\emph{Herr Wenzel auf Rehberg. Novelle}|pwv}{ }\strikeout{wirklich} für jeden erſichtlich werden, der in
               anſtändiger Weiſe zu leſen verſteht, und mir erſchien daher dieſer ganze Abſatz\pwindex{Salten, Felix 6.\,9.\,1869 Budapest – 8.\,10.\,1945 Zürich@\textsc{Salten, Felix} (6.\,9.\,1869 Budapest – 8.\,10.\,1945 Zürich), \emph{Schriftsteller, Journalist, Chefredakteur}!Herr Wenzel auf Rehberg. Novelle@\strich\emph{Herr Wenzel auf Rehberg. Novelle}|pwv} wie eine
                  Reverenz vor den oberflächlichen, die ihnen nicht gebührt. Ich
               habe mich natürlich auch gefragt, ob dieſer Rückblick vielleicht als Ergänzung zum
               Charakterbild des Erzählers\pwindex{Salten, Felix 6.\,9.\,1869 Budapest – 8.\,10.\,1945 Zürich@\textsc{Salten, Felix} (6.\,9.\,1869 Budapest – 8.\,10.\,1945 Zürich), \emph{Schriftsteller, Journalist, Chefredakteur}!Herr Wenzel auf Rehberg. Novelle@\strich\emph{Herr Wenzel auf Rehberg. Novelle}|pwv}
               Ihnen unerläßlich{ }ſcheinen mochte – doch find ich daſs die etwas neuen Züge\strikeout{n} höchſtens
               im Sinne philoſophiſcher Altersveränderungen zu deuten wären, die mit dem
               köſtlich-fertigen Chronik-Rehberg\pwindex{Salten, Felix 6.\,9.\,1869 Budapest – 8.\,10.\,1945 Zürich@\textsc{Salten, Felix} (6.\,9.\,1869 Budapest – 8.\,10.\,1945 Zürich), \emph{Schriftsteller, Journalist, Chefredakteur}!Herr Wenzel auf Rehberg. Novelle@\strich\emph{Herr Wenzel auf Rehberg. Novelle}|pw}, den Sie
               geſtalteten, nichts weiter zu thun haben. Auch wirkt {\pb}die Stelle\pwindex{Salten, Felix 6.\,9.\,1869 Budapest – 8.\,10.\,1945 Zürich@\textsc{Salten, Felix} (6.\,9.\,1869 Budapest – 8.\,10.\,1945 Zürich), \emph{Schriftsteller, Journalist, Chefredakteur}!Herr Wenzel auf Rehberg. Novelle@\strich\emph{Herr Wenzel auf Rehberg. Novelle}|pwv}, wo Rehberg\pwindex{Salten, Felix 6.\,9.\,1869 Budapest – 8.\,10.\,1945 Zürich@\textsc{Salten, Felix} (6.\,9.\,1869 Budapest – 8.\,10.\,1945 Zürich), \emph{Schriftsteller, Journalist, Chefredakteur}!Herr Wenzel auf Rehberg. Novelle@\strich\emph{Herr Wenzel auf Rehberg. Novelle}|pwv} zum Selbſtankläger wird »Und da{\geminationn} hat mich dies Treiben{ }ſo
                  weit von meinem Worte fortgeriſſen\pwindex{Salten, Felix 6.\,9.\,1869 Budapest – 8.\,10.\,1945 Zürich@\textsc{Salten, Felix} (6.\,9.\,1869 Budapest – 8.\,10.\,1945 Zürich), \emph{Schriftsteller, Journalist, Chefredakteur}!Herr Wenzel auf Rehberg. Novelle@\strich\emph{Herr Wenzel auf Rehberg. Novelle}|pwv}{ }\textsc{etc}« keineswegs bezwingend wahr. Weder{ }ſubjectiv noch
               objektiv. – Ich würde daher \label{K_L03005-9v}\edtext{in der Buchausgabe\pwindex{Salten, Felix 6.\,9.\,1869 Budapest – 8.\,10.\,1945 Zürich@\textsc{Salten, Felix} (6.\,9.\,1869 Budapest – 8.\,10.\,1945 Zürich), \emph{Schriftsteller, Journalist, Chefredakteur}!Herr Wenzel auf Rehberg. Novelle@\strich\emph{Herr Wenzel auf Rehberg. Novelle}|pwv} von dem Abſatz\pwindex{Salten, Felix 6.\,9.\,1869 Budapest – 8.\,10.\,1945 Zürich@\textsc{Salten, Felix} (6.\,9.\,1869 Budapest – 8.\,10.\,1945 Zürich), \emph{Schriftsteller, Journalist, Chefredakteur}!Herr Wenzel auf Rehberg. Novelle@\strich\emph{Herr Wenzel auf Rehberg. Novelle}|pwv} nur die erſten Zeilen{ }ſtehen laſſen bei »als der Kaiſer
                  gegen ihn geweſen\pwindex{Salten, Felix 6.\,9.\,1869 Budapest – 8.\,10.\,1945 Zürich@\textsc{Salten, Felix} (6.\,9.\,1869 Budapest – 8.\,10.\,1945 Zürich), \emph{Schriftsteller, Journalist, Chefredakteur}!Herr Wenzel auf Rehberg. Novelle@\strich\emph{Herr Wenzel auf Rehberg. Novelle}|pwv}« – oder nicht einmal die}{\lemma{\textnormal{\emph{in … die}}}\Cendnote{\textnormal{Salten\pwindex{Salten, Felix 6.\,9.\,1869 Budapest – 8.\,10.\,1945 Zürich@\textsc{Salten, Felix} (6.\,9.\,1869 Budapest – 8.\,10.\,1945 Zürich), \emph{Schriftsteller, Journalist, Chefredakteur}|pwk} übernahm Schnitzlers Vorschläge für die 1907 bei \emph{S. Fischer}\orgindex{S. Fischer Verlag@S. Fischer Verlag|pwk} erschienene
                  Buchausgabe von \emph{Herr Wenzel auf Rehberg}\pwindex{Salten, Felix 6.\,9.\,1869 Budapest – 8.\,10.\,1945 Zürich@\textsc{Salten, Felix} (6.\,9.\,1869 Budapest – 8.\,10.\,1945 Zürich), \emph{Schriftsteller, Journalist, Chefredakteur}!Herr Wenzel auf Rehberg. Novelle@\strich\emph{Herr Wenzel auf Rehberg. Novelle}|pwk}
                  nicht.}}}\label{K_L03005-9} – und ruhig auf den letzten Abſatz übergehen. –\pend
           
\pstart
           Ihr \label{K_L03005-10v}\edtext{Berlin\oindex{Berlin@\textbf{Berlin}, \emph{Hauptstadt}|pw}er Feu{[}i{]}lleton\pwindex{Salten, Felix 6.\,9.\,1869 Budapest – 8.\,10.\,1945 Zürich@\textsc{Salten, Felix} (6.\,9.\,1869 Budapest – 8.\,10.\,1945 Zürich), \emph{Schriftsteller, Journalist, Chefredakteur}!fremde Stadt. Thema mit Variationen@\strich\emph{Die fremde Stadt. Thema mit Variationen}|pwv}}{\lemma{\textnormal{\emph{Berliner Feuilleton}}}\Cendnote{\textnormal{Felix Salten\pwindex{Salten, Felix 6.\,9.\,1869 Budapest – 8.\,10.\,1945 Zürich@\textsc{Salten, Felix} (6.\,9.\,1869 Budapest – 8.\,10.\,1945 Zürich), \emph{Schriftsteller, Journalist, Chefredakteur}|pwk}: \emph{Die fremde Stadt. Thema mit Variationen}\pwindex{Salten, Felix 6.\,9.\,1869 Budapest – 8.\,10.\,1945 Zürich@\textsc{Salten, Felix} (6.\,9.\,1869 Budapest – 8.\,10.\,1945 Zürich), \emph{Schriftsteller, Journalist, Chefredakteur}!fremde Stadt. Thema mit Variationen@\strich\emph{Die fremde Stadt. Thema mit Variationen}|pwk}. In: \emph{Die Zeit}\pwindex{Zeit@\emph{Die Zeit}|pwk}, Jg. 5, Nr. 1304, 13. 5. 1906, Morgenblatt, S. 1–3.}}}\label{K_L03005-10} in
               der Zeit\pwindex{Zeit@\emph{Die Zeit}|pw} hab ich mit Ergriffenheit geleſen. Sind
                  {\pb}Sie nun{ }ſchon an der \textsc{Herzl\pwindex{Herzl, Theodor 2.\,5.\,1860 Budapest – 3.\,7.\,1904 Edlach@\textsc{Herzl, Theodor} (2.\,5.\,1860 Budapest – 3.\,7.\,1904 Edlach), \emph{Schriftsteller, Journalist}|pw}}-Biographie? Und welches{ }ſind die größern Sachen, die Sie componiren? – Die
                  \label{K_L03005-11v}\edtext{Wartburg\oindex{Wartburg@\textbf{Wartburg}, \emph{Burg}|pw}erreiſe}{\lemma{\textnormal{\emph{Wartburgerreise}}}\Cendnote{\textnormal{Siehe XXXX Auszeichnungsfehler: Dokument L03423 nicht gefunden.
               }}}\label{K_L03005-11} war ein Ausflug zum Vergnügen oder{ }ſonſt was? – Wie{ }ſtehts mit \label{K_L03005-12v}\edtext{Spanien\oindex{Spanien@\textbf{Spanien}|pw}}{\lemma{\textnormal{\emph{Spanien}}}\Cendnote{\textnormal{Siehe XXXX Auszeichnungsfehler: Dokument L03422 nicht gefunden.
               }}}\label{K_L03005-12}? – Unser Kinderarzt Dr \textsc{Pollak\pwindex{Pollak, Jacob 7.\,2.\,1860 Černá Hora – 25.\,3.\,1941 Wien@\textsc{Pollak, Jacob} (7.\,2.\,1860 Černá Hora – 25.\,3.\,1941 Wien), \emph{Mediziner}|pw}} theilt mir mit, dſs Heringsdorf\oindex{Heringsdorf@\textbf{Heringsdorf}, \emph{Hauptstadt}|pw} u
               beſonders \textsc{Swinemünde\oindex{Świnoujście@\textbf{Świnoujście}, \emph{Hauptstadt}|pw}} enorm gelſengeplagt{ }ſind.\footnote{\noindent{}Er war in Sw.\oindex{Świnoujście@\textbf{Świnoujście}, \emph{Hauptstadt}|pw}} Erkundg Sie{ }ſich doch gut, eh Sie miethen. –\pend
           
\pstart
           Eben bekam ich von Ludaſſy\pwindex{Gans-Ludassy, Julius von 13.\,4.\,1858 Wien – 30.\,9.\,1922 ebd.@\textsc{Gans-Ludassy, Julius von} (13.\,4.\,1858 Wien – 30.\,9.\,1922 ebd.), \emph{Schriftsteller, Journalist, Herausgeber}|pw} eine Gratul-Karte
               zum geſtrigen Erfolg\pwindex{Schnitzler, Arthur 15.\,5.\,1862 Wien – 21.\,10.\,1931 ebd.@\textsc{Schnitzler, Arthur} (15.\,5.\,1862 Wien – 21.\,10.\,1931 ebd.), \emph{Schriftsteller, Mediziner}!einsame Weg. Schauspiel in fünf Akten@\strich\emph{Der einsame Weg. Schauspiel in fünf Akten}|pwv}. Seine
                  Frau\pwindex{Gans-Ludassy, Olga von 5.\,6.\,1867 Wien – 18.\,8.\,1948 Islip@\textsc{Gans-Ludassy, Olga von} (5.\,6.\,1867 Wien – 18.\,8.\,1948 Islip)|pwv} hat eben eine{ }ſchwere Lungenentzündg durchgemacht, und ich muſs{ }ſie \label{K_L03005-13v}\edtext{nächſtens beſuchen}{\lemma{\textnormal{\emph{nächstens besuchen}}}\Cendnote{\textnormal{Siehe A. S.: \emph{Tagebuch}, 2. 6. 1906.
               }}}\label{K_L03005-13}. So wär es mir{ }ſehr lieb, {\pb}we{\geminationn} Sie mir raſch nur mit 2 Worten \strikeout{mit}{ }ſagten, wie nun eigentlich Ihre \label{K_L03005-14v}\edtext{Prozeſsſache}{\lemma{\textnormal{\emph{Prozesssache}}}\Cendnote{\textnormal{Siehe XXXX Auszeichnungsfehler: Dokument L03415 nicht gefunden.
               }}}\label{K_L03005-14}{ }ſteht? –\pend
           
\pstart
           Frl Erl\pwindex{Erl, Dora @\textsc{Erl, Dora}, \emph{Schauspielerin, Gesangspädagogin}|pw} iſt ab nach Dresden\oindex{Dresden@\textbf{Dresden}|pw} (vorläufg ohne beſti{\geminationm}tes
                  Engagement){[}.{]}{ }\textsc{Tennis} regelmäßig \textsc{Kaufma{\geminationn}\pwindex{Kaufmann, Arthur 4.\,4.\,1872 Iași – 25.\,7.\,1938 Wien@\textsc{Kaufmann, Arthur} (4.\,4.\,1872 Iași – 25.\,7.\,1938 Wien), \emph{Rechtswissenschaftler, Privatgelehrte, Privatier}|pw}}, manchmal \textsc{Speidels\pwindex{Speidel, Felix 2.\,7.\,1875 Stuttgart – 3.\,10.\,1952 Unterach am Attersee@\textsc{Speidel, Felix} (2.\,7.\,1875 Stuttgart – 3.\,10.\,1952 Unterach am Attersee), \emph{Schriftsteller, Verleger}|pw}\pwindex{Speidel-Haeberle, Else 11.\,7.\,1877 Stuttgart – 21.\,7.\,1937 Augustenfeld@\textsc{Speidel-Haeberle, Else} (11.\,7.\,1877 Stuttgart – 21.\,7.\,1937 Augustenfeld), \emph{Schauspielerin}|pw}} (er\pwindex{Speidel, Felix 2.\,7.\,1875 Stuttgart – 3.\,10.\,1952 Unterach am Attersee@\textsc{Speidel, Felix} (2.\,7.\,1875 Stuttgart – 3.\,10.\,1952 Unterach am Attersee), \emph{Schriftsteller, Verleger}|pwv} kam erſt jüngſt aus
                  Griechenland\oindex{Griechenland@\textbf{Griechenland}|pw} zurück). –\pend
           
\pstart
           – \label{K_L03005-15v}\edtext{Richard\pwindex{Beer-Hofmann, Richard 11.\,7.\,1866 Wien – 26.\,9.\,1945 New York City@\textsc{Beer-Hofmann, Richard} (11.\,7.\,1866 Wien – 26.\,9.\,1945 New York City), \emph{Schriftsteller}|pw} war einmal bei uns in der Hinterbrühl\oindex{Hinterbrühl@\textbf{Hinterbrühl}, \emph{Hauptstadt}|pw}, mit Paula\pwindex{Beer-Hofmann, Paula 25.\,2.\,1879 Wien – 30.\,10.\,1939 Zürich@\textsc{Beer-Hofmann, Paula} (25.\,2.\,1879 Wien – 30.\,10.\,1939 Zürich)|pw} u Mirjam\pwindex{Beer-Hofmann, Mirjam 4.\,9.\,1897 Wien – 24.\,12.\,1984 New York City@\textsc{Beer-Hofmann, Mirjam} (4.\,9.\,1897 Wien – 24.\,12.\,1984 New York City)|pw}}{\lemma{\textnormal{\emph{Richard … Mirjam}}}\Cendnote{\textnormal{Siehe A. S.: \emph{Tagebuch}, 12. 5. 1906.
               }}}\label{K_L03005-15};{ }ſehr erfüllt von{ }ſeinem \label{K_L03005-16v}\edtext{Fünfabend Stück\pwindex{Beer-Hofmann, Richard 11.\,7.\,1866 Wien – 26.\,9.\,1945 New York City@\textsc{Beer-Hofmann, Richard} (11.\,7.\,1866 Wien – 26.\,9.\,1945 New York City), \emph{Schriftsteller}!Historie von König David. Ein Zyklus@\strich\emph{Die Historie von König David. Ein Zyklus}|pwv}}{\lemma{\textnormal{\emph{Fünfabend Stück}}}\Cendnote{\textnormal{der Dramenzyklus \emph{Die Historie von König David}\pwindex{Beer-Hofmann, Richard 11.\,7.\,1866 Wien – 26.\,9.\,1945 New York City@\textsc{Beer-Hofmann, Richard} (11.\,7.\,1866 Wien – 26.\,9.\,1945 New York City), \emph{Schriftsteller}!Historie von König David. Ein Zyklus@\strich\emph{Die Historie von König David. Ein Zyklus}|pwk}}}}\label{K_L03005-16}. Erfülltſein iſt doch der neidenswertheſte Zuſtand von allen; – we{\geminationn} nicht die Verpflichtungsgefühle{ }ſich einſtellen – die
               oft trügeriſch{ }ſind, we{\geminationn}{ }ſie{ }ſich auf uns{ }ſelbſt, und
               immer we{\geminationn}{ }ſie{ }ſich auf die Welt (ſowohl »Mit« als
               »Nach«) {\pb}beziehen. Dies iſt eine Wahrheit.
               Sollte es aber nicht wahrere Wahrheiten geben?\pend
           
\pstart
           – Wir haben ein neues Fräulein, angenehm jüdiſch, Anna Loew\pwindex{Loew, Anna *~11.\,4.\,1888 Ješín@\textsc{Loew, Anna} (*~11.\,4.\,1888 Ješín), \emph{Kinderbetreuerin, Dienstbotin}|pw} betitelt, und wegen einer Halsentzündg in Hinterbrühl\oindex{Hinterbrühl@\textbf{Hinterbrühl}, \emph{Hauptstadt}|pw} zurückgeblieben. Sie hat einen Bruder, \textsc{Johann Loew\pwindex{Loew, Johann @\textsc{Loew, Johann}, \emph{Arbeiterführer}|pw}}, Arbeiterführer, und{ }ſo bekam ich plötzlich aus Brüſſel\oindex{Brüssel@\textbf{Brüssel}, \emph{Hauptstadt}|pw} eine, \textsc{resp.} zwei waterlo\oindex{Waterloo@\textbf{Waterloo}|pw}hende Karten, von \textsc{Johann Loew\pwindex{Loew, Johann @\textsc{Loew, Johann}, \emph{Arbeiterführer}|pw}} und \textsc{Lotte Pohl-Glas\pwindex{Pohl-Glas, Charlotte 1.\,1.\,1873 Wien – 15.\,2.\,1944 Zürich@\textsc{Pohl-Glas, Charlotte} (1.\,1.\,1873 Wien – 15.\,2.\,1944 Zürich), \emph{Schriftstellerin, Politikerin, Sozialistin}|pw}}. Wer die Zuſa{\geminationm}enhänge begreift, lebt ewig.\pend
           
\pstart
           Dies wünſcht Ihnen, nebſt vielen herzlichen Güßen für Sie und die Ihren von uns
               allen. {\\[\baselineskip]}Ihr {\\[\baselineskip]}\spacefill\mbox{Arthur}\pend
           \leftskip=0em{}
\pstart
           \noindent{}Richard\pwindex{Beer-Hofmann, Richard 11.\,7.\,1866 Wien – 26.\,9.\,1945 New York City@\textsc{Beer-Hofmann, Richard} (11.\,7.\,1866 Wien – 26.\,9.\,1945 New York City), \emph{Schriftsteller}|pw} hat zwei{ }ſchöne \label{K_L03005-17v}\edtext{Gedichte\pwindex{Beer-Hofmann, Richard 11.\,7.\,1866 Wien – 26.\,9.\,1945 New York City@\textsc{Beer-Hofmann, Richard} (11.\,7.\,1866 Wien – 26.\,9.\,1945 New York City), \emph{Schriftsteller}!einsame Weg@\strich\emph{Der einsame Weg}|pw}\pwindex{Beer-Hofmann, Richard 11.\,7.\,1866 Wien – 26.\,9.\,1945 New York City@\textsc{Beer-Hofmann, Richard} (11.\,7.\,1866 Wien – 26.\,9.\,1945 New York City), \emph{Schriftsteller}!Altern@\strich\emph{Altern}|pw} geſchrieben, eins »Der einſame Weg\pwindex{Beer-Hofmann, Richard 11.\,7.\,1866 Wien – 26.\,9.\,1945 New York City@\textsc{Beer-Hofmann, Richard} (11.\,7.\,1866 Wien – 26.\,9.\,1945 New York City), \emph{Schriftsteller}!einsame Weg@\strich\emph{Der einsame Weg}|pw}}{\lemma{\textnormal{\emph{Gedichte … Weg}}}\Cendnote{\textnormal{Siehe XXXX Auszeichnungsfehler: Dokument L01597 nicht gefunden.
                  }}}\label{K_L03005-17}« – ein andres »Altern\pwindex{Beer-Hofmann, Richard 11.\,7.\,1866 Wien – 26.\,9.\,1945 New York City@\textsc{Beer-Hofmann, Richard} (11.\,7.\,1866 Wien – 26.\,9.\,1945 New York City), \emph{Schriftsteller}!Altern@\strich\emph{Altern}|pw}«, 1 an mich, 1 an \textsc{Kerr\pwindex{Kerr, Alfred 25.\,12.\,1867 Breslau – 12.\,10.\,1948 Hamburg@\textsc{Kerr, Alfred} (25.\,12.\,1867 Breslau – 12.\,10.\,1948 Hamburg), \emph{Schriftsteller, Kritiker}|pw}}.\pend
           \selectlanguage{ngerman}\endnumbering\briefempfaengerindex{Salten, Felix@\textsc{Salten, Felix}!zzzSchnitzler, Arthur@\emph{von Arthur Schnitzler}!1906-05-161@{16. 5. 1906}|)be}\mylabel{L03005h}  \newcommand{\dateiname}{L03005}\newcommand{\titel}{Arthur Schnitzler an Felix Salten, 16. 5. 1906}\newcommand{\editorInnen}{Martin Anton Müller und Laura Untner}%% latex-leseansicht-abspann.tex
%% Abspann für die Leseansicht.
%% Der Schalter \ifkorrekturansicht ist bereits durch den Vorspann gesetzt.

%% latex-abspann.tex
%% Gemeinsamer Abspann für Korrekturansicht und Leseansicht.
%% Setzt den Schalter \ifkorrekturansicht voraus (gesetzt in den
%% einbindenden Dateien latex-korrekturansicht-abspann.tex bzw.
%% latex-leseansicht-abspann.tex).
%% ---------------------------------------------------------------

\normalsize

% Das esempio-Environment wird nur in der Leseansicht benötigt
\ifkorrekturansicht\else
\newenvironment{esempio}[3]%
{
    \vspace{1.5ex}
    \rlap{\underline{#1}}
    \par
    \setlength{\parindent}{0cm}
    \nopagebreak
    \leftskip=#2cm
    \rightskip=#3cm
}
{
    \par
}
\fi

\doendnotes{C}
\bigskip
\vfill

\clearpage

\footnotesize

\ifkorrekturansicht
  \lohead{\textsc{register}}
\fi

% theindex-Environment neu definieren ohne reledmac
\makeatletter
\renewenvironment{theindex}{%
  \ifkorrekturansicht
    \section*{\indexname}%
  \else
    \subsubsection*{Index der erwähnten Entitäten}%
  \fi
  \setlength{\parindent}{0pt}%
  \setlength{\parskip}{0pt plus 0.3pt}%
  \let\item\@idxitem
}{%
  \ifkorrekturansicht\clearpage\fi
}
\makeatother

\IfFileExists{\jobname-pw.ind}{\input{\jobname-pw.ind}}{}

% Quellenangabe nur in der Leseansicht
\ifkorrekturansicht\else
% Fallback-Definitionen, falls die .tex-Datei \titel etc. nicht gesetzt hat
\providecommand{\titel}{}
\providecommand{\editorInnen}{}
\providecommand{\dateiname}{\jobname}

\vspace{3cm}

\vfill

\footnotesize
\textsc{Quelle}: \titel. Herausgegeben von {\editorInnen}. In: \emph{Arthur Schnitzler: Briefwechsel mit Autorinnen und Autoren}.
 Digitale Edition, https://schnitzler-briefe.acdh.oeaw.ac.at/{\dateiname}.html (Stand \today)
\fi

\end{document}


