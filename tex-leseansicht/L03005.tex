%% latex-korrekturansicht-vorspann.tex
%% Vorspann für die Korrekturansicht.
%% Lädt die gemeinsame Datei latex-vorspann.tex mit gesetztem Schalter.

\newif\ifkorrekturansicht
\korrekturansichttrue

\input{../tex-inputs/latex-vorspann}


\section[ Arthur Schnitzler an Felix Salten, 16. 5. 1906]{L03005 Arthur Schnitzler an Felix Salten, 16. 5. 1906}
\nopagebreak\mylabel{L03005v}
\rehead{ }\normalsize\beginnumbering\briefempfaengerindex{Salten, Felix@\textsc{Salten, Felix}!zzzSchnitzler, Arthur@\emph{von Arthur Schnitzler}!1906-05-161@{16. 5. 1906}|(be}
\toendnotes[C]{\smallbreak\pagebreak[2]}\Standort{Wienbibliothek im Rathaus, ZPH 1681, 2.1.516.}
\physDesc{Brief, 2 Blätter, 8 Seiten, 4277 Zeichen
\newline{}Handschrift: schwarze Tinte, deutsche Kurrent
\newline{}Ordnung: mit Bleistift von unbekannter Hand Nummerierung der Doppelseiten des
                                 Konvoluts: »12«–»15« }\toendnotes[C]{\smallbreak}
\pstart
           {\pb}\textcolor{gray}{\textbf{Dr. Arthur Schnitzler}}\hfill 16. Mai 906\pend
           
\pstart
           \textcolor{gray}{\textbf{Wien XVIII. Spoettelgasse 7\oindex{Edmund-Weiss-Gasse 7@\textbf{Edmund-Weiß-Gasse 7}, \emph{Wohngebäude (K.WHS)}|pw}.}}\pend
           \vspace{0.5em}
\pstart
           lieber, beim Nachhauſeko{\geminationm}en aus \label{K_L03005-1v}\edtext{Theater\oindex{Theater an der Wien@\textbf{Theater an der Wien}, \emph{Theater (K.THE)}|pwv} und Hotel\oindex{Meissl {\kaufmannsund} Schadn@\textbf{Meissl {\kaufmannsund} Schadn}, \emph{Hotel (K.HTL)}|pwv}}{\lemma{\textnormal{\emph{Theater und Hotel}}}\Cendnote{\textnormal{Siehe A. S.: \emph{Tagebuch}, 15. 5. 1906.
               }}}\label{K_L03005-1} hab \label{T_L03005-1v}\edtext{ich}{\lemma{\textnormal{\emph{ich}}}\Cendnote{\textnormal{In der Vorlage steht: »ich ich«.}}}\label{T_L03005-1} Ihren
               kurzen aber klingenden \label{K_L03005-2v}\edtext{Brief}{\lemma{\textnormal{\emph{Brief}}}\Cendnote{\textnormal{Felix Salten an Arthur Schnitzler, 14. 5. 1906.
               }}}\label{K_L03005-2} vorgefunden und mich ſehr damit gefreut. Es mußte für mich freilich nicht
               gerade der Einſ. Weg\pwindex{einsame Weg. Schauspiel in fuenf Akten@\emph{Der einsame Weg. Schauspiel in fünf Akten}|pw} kommen, um mich Ihr
               Fernſein ſchmerzlich empfinden zu laſſen. Der Abend{ }geſtern iſt überraſchend gut ausgefallen: jedenfalls
               war er äußerlich der ſtärkſte Erfolg meiner Theaterlaufbahn. Völlige Stu{\geminationm}heit nach dem erſten Akt\pwindex{einsame Weg. Schauspiel in fuenf Akten@\emph{Der einsame Weg. Schauspiel in fünf Akten}|pwv}, wahre »Stürme« nach 2.\pwindex{einsame Weg. Schauspiel in fuenf Akten@\emph{Der einsame Weg. Schauspiel in fünf Akten}|pwv}, 3.\pwindex{einsame Weg. Schauspiel in fuenf Akten@\emph{Der einsame Weg. Schauspiel in fünf Akten}|pwv}, gedämpft nach dem 4\pwindex{einsame Weg. Schauspiel in fuenf Akten@\emph{Der einsame Weg. Schauspiel in fünf Akten}|pwv},
               wieder ſehr ſtark {\pb}nach dem 5. Akt\pwindex{einsame Weg. Schauspiel in fuenf Akten@\emph{Der einsame Weg. Schauspiel in fünf Akten}|pwv}. Baſſermann\pwindex{Bassermann, Albert 07.09.1867 – 15.05.1952@\textsc{Bassermann, Albert} (07.09.1867 – 15.05.1952), \emph{Schauspieler/Schauspielerin}|pw} anfangs etwas bläßlich, am Schluſs unvergleichlich. \label{K_L03005-3v}\edtext{Reicher\pwindex{Reicher, Emanuel 18.06.1849 – 15.05.1924@\textsc{Reicher, Emanuel} (18.06.1849 – 15.05.1924), \emph{Schauspieler/Schauspielerin}|pw} hat mich in gewiſſem Sinne angenehm
               enttäuſcht}{\lemma{\textnormal{\emph{Reicher … enttäuſcht}}}\Cendnote{\textnormal{Vgl. Felix Salten u. a. an Arthur Schnitzler, 19. 4. 1906.
               }}}\label{K_L03005-3}. Im ganzen war er wohl unerträglich genug; aber die Leiſtung als ganzes
               war von einer gewiſſen Geſchloſſenheit, ſo daſs man einen mehr menſchlichen als
               künſtleriſchen Widerwillen gegen die \label{K_L03005-4v}\edtext{Figur\pwindex{einsame Weg. Schauspiel in fuenf Akten@\emph{Der einsame Weg. Schauspiel in fünf Akten}|pwv}}{\lemma{\textnormal{\emph{Figur}}}\Cendnote{\textnormal{Albert Bassermann\pwindex{Bassermann, Albert 07.09.1867 – 15.05.1952@\textsc{Bassermann, Albert} (07.09.1867 – 15.05.1952), \emph{Schauspieler/Schauspielerin}|pwk} spielte den Stephan von Sala\pwindex{einsame Weg. Schauspiel in fuenf Akten@\emph{Der einsame Weg. Schauspiel in fünf Akten}|pwkv}.}}}\label{K_L03005-4}
               kriegte. – Seltſam ſind doch Dramenſchicksale. Eine ſolche Aufnahme \label{K_L03005-5v}\edtext{in Berlin\oindex{Berlin@\textbf{Berlin}, \emph{P.PPLC}|pw} vor 2 ½ Jahren}{\lemma{\textnormal{\emph{in Berlin vor 2 ½ Jahren}}}\Cendnote{\textnormal{Uraufführung von \emph{Der einsame Weg}\pwindex{einsame Weg. Schauspiel in fuenf Akten@\emph{Der einsame Weg. Schauspiel in fünf Akten}|pwk} am Deutschen Theater Berlin\oindex{Deutsches Theater Berlin@\textbf{Deutsches Theater Berlin}, \emph{Theater (K.THE)}|pwk} am 13. 2. 1904}}}\label{K_L03005-5} – und Ihre Profezeihung wäre erfüllt geweſen.\pend
           
\pstart
           – Den \label{K_L03005-6v}\edtext{Rehberg\pwindex{Herr Wenzel auf Rehberg. Novelle@\emph{Herr Wenzel auf Rehberg. Novelle}|pw} hab ich in der Hinterbrühl\oindex{Hinterbruehl@\textbf{Hinterbrühl}, \emph{P.PPLA3}|pw}}{\lemma{\textnormal{\emph{Rehberg … Hinterbrühl}}}\Cendnote{\textnormal{Siehe A. S.: \emph{Tagebuch}, 8. 5. 1906.
               }}}\label{K_L03005-6} geleſen, wo wir höchſt angenehme \label{K_L03005-7v}\edtext{acht Tage im Hotel Radetzky\oindex{Hotel Radetzky@\textbf{Hotel Radetzky}, \emph{Hotel (K.HTL)}|pw}}{\lemma{\textnormal{\emph{acht … Radetzky}}}\Cendnote{\textnormal{vom 7. 5. 1906 bis zum 14. 5. 1906}}}\label{K_L03005-7}{ }{\pb}gewohnt und \textsc{tennis}
               geſpielt haben (Einmal mit \label{K_L03005-8v}\edtext{Hugo\pwindex{Hofmannsthal, Hugo von 1874-02-01 – 1929-07-15@\textsc{Hofmannsthal, Hugo von} (1874-02-01 – 1929-07-15), \emph{Schriftsteller/Schriftstellerin}|pw}, den ich im \textsc{\begin{otherlanguage}{english}single set\end{otherlanguage}} 6:4 ſchlug}{\lemma{\textnormal{\emph{Hugo, … set 6:4 ſchlug}}}\Cendnote{\textnormal{Siehe A. S.: \emph{Tagebuch}, 11. 5. 1906.
               }}}\label{K_L03005-8}!) – Es iſt ein glänzendes Ding\pwindex{Herr Wenzel auf Rehberg. Novelle@\emph{Herr Wenzel auf Rehberg. Novelle}|pwv}, und es gibt vielleicht im ganzen darin nur 3–5 Stellen, bei denen mir
               im Stil irgend was wie ein falſcher Ton erſcheint. Doch möcht ichs, nach einem
               Zwiſchenraum von ein paar Wochen, noch einmal leſen, um mich ſelber nachzuprüfen.
               Hingegen ſage ich ſchon heute mit Entſchiedenheit,
               daſs ich den vorletzten Abſatz\pwindex{Herr Wenzel auf Rehberg. Novelle@\emph{Herr Wenzel auf Rehberg. Novelle}|pwv}
               fortwünſchte. Hier werden Zuſa{\geminationm}enhänge mit einer meinen
               Geſchmack ſtörenden Deutlichkeit aufgezeigt; \strikeout{die}
                  Zuſa{\geminationm}enhänge, die im {\pb}Gang der Geſchichte\pwindex{Herr Wenzel auf Rehberg. Novelle@\emph{Herr Wenzel auf Rehberg. Novelle}|pwv}{ }\strikeout{wirklich} für jeden erſichtlich werden, der in
               anſtändiger Weiſe zu leſen verſteht, und mir erſchien daher dieſer ganze Abſatz\pwindex{Herr Wenzel auf Rehberg. Novelle@\emph{Herr Wenzel auf Rehberg. Novelle}|pwv} wie eine
                  Reverenz vor den oberflächlichen, die ihnen nicht gebührt. Ich
               habe mich natürlich auch gefragt, ob dieſer Rückblick vielleicht als Ergänzung zum
               Charakterbild des Erzählers\pwindex{Herr Wenzel auf Rehberg. Novelle@\emph{Herr Wenzel auf Rehberg. Novelle}|pwv}
               Ihnen unerläßlich ſcheinen mochte – doch find ich daſs die etwas neuen Züge\strikeout{n} höchſtens
               im Sinne philoſophiſcher Altersveränderungen zu deuten wären, die mit dem
               köſtlich-fertigen Chronik-Rehberg\pwindex{Herr Wenzel auf Rehberg. Novelle@\emph{Herr Wenzel auf Rehberg. Novelle}|pw}, den Sie
               geſtalteten, nichts weiter zu thun haben. Auch wirkt {\pb}die Stelle\pwindex{Herr Wenzel auf Rehberg. Novelle@\emph{Herr Wenzel auf Rehberg. Novelle}|pwv}, wo Rehberg\pwindex{Herr Wenzel auf Rehberg. Novelle@\emph{Herr Wenzel auf Rehberg. Novelle}|pwv} zum Selbſtankläger wird »Und da{\geminationn} hat mich dies Treiben ſo
                  weit von meinem Worte fortgeriſſen\pwindex{Herr Wenzel auf Rehberg. Novelle@\emph{Herr Wenzel auf Rehberg. Novelle}|pwv}{ }\textsc{etc}« keineswegs bezwingend wahr. Weder ſubjectiv noch
               objektiv. – Ich würde daher \label{K_L03005-9v}\edtext{in der Buchausgabe\pwindex{Herr Wenzel auf Rehberg. Novelle@\emph{Herr Wenzel auf Rehberg. Novelle}|pwv} von dem Abſatz\pwindex{Herr Wenzel auf Rehberg. Novelle@\emph{Herr Wenzel auf Rehberg. Novelle}|pwv} nur die erſten Zeilen
               ſtehen laſſen bei »als der Kaiſer
                  gegen ihn geweſen\pwindex{Herr Wenzel auf Rehberg. Novelle@\emph{Herr Wenzel auf Rehberg. Novelle}|pwv}« – oder nicht einmal die}{\lemma{\textnormal{\emph{in … die}}}\Cendnote{\textnormal{Salten\pwindex{Salten, Felix 06.09.1869 – 08.10.1945@\textsc{Salten, Felix} (06.09.1869 – 08.10.1945), \emph{Schriftsteller/Schriftstellerin, Journalist/Journalistin, Chefredakteur/Chefredakteurin}|pwk} übernahm Schnitzlers Vorschläge für die 1907 bei \emph{S. Fischer}\orgindex{S. Fischer Verlag@S. Fischer Verlag|pwk} erschienene
                  Buchausgabe von \emph{Herr Wenzel auf Rehberg}\pwindex{Herr Wenzel auf Rehberg. Novelle@\emph{Herr Wenzel auf Rehberg. Novelle}|pwk}
                  nicht.}}}\label{K_L03005-9} – und ruhig auf den letzten Abſatz übergehen. –\pend
           
\pstart
           Ihr \label{K_L03005-10v}\edtext{Berlin\oindex{Berlin@\textbf{Berlin}, \emph{P.PPLC}|pw}er Feu{[}i{]}lleton\pwindex{fremde Stadt. Thema mit Variationen@\emph{Die fremde Stadt. Thema mit Variationen}|pwv}}{\lemma{\textnormal{\emph{Berliner Feuilleton}}}\Cendnote{\textnormal{Felix Salten\pwindex{Salten, Felix 06.09.1869 – 08.10.1945@\textsc{Salten, Felix} (06.09.1869 – 08.10.1945), \emph{Schriftsteller/Schriftstellerin, Journalist/Journalistin, Chefredakteur/Chefredakteurin}|pwk}: \emph{Die fremde Stadt. Thema mit Variationen}\pwindex{fremde Stadt. Thema mit Variationen@\emph{Die fremde Stadt. Thema mit Variationen}|pwk}. In: \emph{Die Zeit}\pwindex{Zeit@\emph{Die Zeit}|pwk}, Jg. 5, Nr. 1304, 13. 5. 1906, Morgenblatt, S. 1–3.}}}\label{K_L03005-10} in
               der Zeit\pwindex{Zeit@\emph{Die Zeit}|pw} hab ich mit Ergriffenheit geleſen. Sind
                  {\pb}Sie nun ſchon an der \textsc{Herzl\pwindex{Herzl, Theodor 1860-05-02 – 1904-07-03@\textsc{Herzl, Theodor} (1860-05-02 – 1904-07-03), \emph{Schriftsteller/Schriftstellerin, Journalist/Journalistin}|pw}}-Biographie? Und welches ſind die größern Sachen, die Sie componiren? – Die
                  \label{K_L03005-11v}\edtext{Wartburg\oindex{Wartburg@\textbf{Wartburg}, \emph{S.CSTL}|pw}erreiſe}{\lemma{\textnormal{\emph{Wartburgerreiſe}}}\Cendnote{\textnormal{Siehe Felix Salten, Paul Lindau und Marie Barthel an Arthur
               Schnitzler, 9. 5. 1906.
               }}}\label{K_L03005-11} war ein Ausflug zum Vergnügen oder ſonſt was? – Wie ſtehts mit \label{K_L03005-12v}\edtext{Spanien\oindex{Spanien@\textbf{Spanien}, \emph{A.PCLI}|pw}}{\lemma{\textnormal{\emph{Spanien}}}\Cendnote{\textnormal{Siehe Felix Salten an Arthur Schnitzler, 1. 5. 1906.
               }}}\label{K_L03005-12}? – Unser Kinderarzt Dr \textsc{Pollak\pwindex{Pollak, Jacob 07.02.1860 – 25.03.1941@\textsc{Pollak, Jacob} (07.02.1860 – 25.03.1941), \emph{Mediziner/Medizinerin}|pw}} theilt mir mit, dſs Heringsdorf\oindex{Heringsdorf@\textbf{Heringsdorf}, \emph{P.PPLA4}|pw} u
               beſonders \textsc{Swinemünde\oindex{Swinoujście@\textbf{Świnoujście}, \emph{P.PPLA2}|pw}} enorm gelſengeplagt ſind.\noindent{}Er war in Sw.\oindex{Swinoujście@\textbf{Świnoujście}, \emph{P.PPLA2}|pw} Erkundg Sie ſich doch gut, eh Sie miethen. –\pend
           
\pstart
           Eben bekam ich von Ludaſſy\pwindex{Gans-Ludassy, Julius von 13.04.1858 – 30.09.1922@\textsc{Gans-Ludassy, Julius von} (13.04.1858 – 30.09.1922), \emph{Schriftsteller/Schriftstellerin, Journalist/Journalistin, Herausgeber/Herausgeberin}|pw} eine Gratul-Karte
               zum geſtrigen Erfolg\pwindex{einsame Weg. Schauspiel in fuenf Akten@\emph{Der einsame Weg. Schauspiel in fünf Akten}|pwv}. Seine
                  Frau\pwindex{Gans-Ludassy, Olga von 05.06.1867 – 1948-08-18@\textsc{Gans-Ludassy, Olga von} (05.06.1867 – 1948-08-18)|pwv} hat eben eine
               ſchwere Lungenentzündg durchgemacht, und ich muſs ſie \label{K_L03005-13v}\edtext{nächſtens beſuchen}{\lemma{\textnormal{\emph{nächſtens beſuchen}}}\Cendnote{\textnormal{Siehe A. S.: \emph{Tagebuch}, 2. 6. 1906.
               }}}\label{K_L03005-13}. So wär es mir ſehr lieb, {\pb}we{\geminationn} Sie mir raſch nur mit 2 Worten \strikeout{mit} ſagten, wie nun eigentlich Ihre \label{K_L03005-14v}\edtext{Prozeſsſache}{\lemma{\textnormal{\emph{Prozeſsſache}}}\Cendnote{\textnormal{Siehe Felix Salten an Arthur Schnitzler, 9. 3. 1906.
               }}}\label{K_L03005-14} ſteht? –\pend
           
\pstart
           Frl Erl\pwindex{Erl, Dora @\textsc{Erl, Dora}, \emph{Schauspieler/Schauspielerin, Gesangspädagoge/Gesangspädagogin}|pw} iſt ab nach Dresden\oindex{Dresden@\textbf{Dresden}, \emph{P.PPLA}|pw} (vorläufg ohne beſti{\geminationm}tes
                  Engagement){[}.{]}{ }\textsc{Tennis} regelmäßig \textsc{Kaufma{\geminationn}\pwindex{Kaufmann, Arthur 04.04.1872 – 25.07.1938@\textsc{Kaufmann, Arthur} (04.04.1872 – 25.07.1938), \emph{Rechtswissenschaftler/Rechtswissenschaftlerin, Privatgelehrte/Privatgelehrte, Privatier/Privatière}|pw}}, manchmal \textsc{Speidels\pwindex{Speidel, Felix 02.07.1875 – 1952-10-03@\textsc{Speidel, Felix} (02.07.1875 – 1952-10-03), \emph{Schriftsteller/Schriftstellerin, Verleger/Verlegerin}|pw}\pwindex{Speidel-Haeberle, Else 11.07.1877 – 21.07.1937@\textsc{Speidel-Haeberle, Else} (11.07.1877 – 21.07.1937), \emph{Schauspieler/Schauspielerin}|pw}} (er\pwindex{Speidel, Felix 02.07.1875 – 1952-10-03@\textsc{Speidel, Felix} (02.07.1875 – 1952-10-03), \emph{Schriftsteller/Schriftstellerin, Verleger/Verlegerin}|pwv} kam erſt jüngſt aus
                  Griechenland\oindex{Griechenland@\textbf{Griechenland}, \emph{A.PCLI}|pw} zurück). –\pend
           
\pstart
           – \label{K_L03005-15v}\edtext{Richard\pwindex{Beer-Hofmann, Richard 1866-07-11 – 1945-09-26@\textsc{Beer-Hofmann, Richard} (1866-07-11 – 1945-09-26), \emph{Schriftsteller/Schriftstellerin}|pw} war einmal bei uns in der Hinterbrühl\oindex{Hinterbruehl@\textbf{Hinterbrühl}, \emph{P.PPLA3}|pw}, mit Paula\pwindex{Beer-Hofmann, Paula 25.02.1879 – 30.10.1939@\textsc{Beer-Hofmann, Paula} (25.02.1879 – 30.10.1939)|pw} u Mirjam\pwindex{Beer-Hofmann, Mirjam 04.09.1897 – 24.12.1984@\textsc{Beer-Hofmann, Mirjam} (04.09.1897 – 24.12.1984)|pw}}{\lemma{\textnormal{\emph{Richard … Mirjam}}}\Cendnote{\textnormal{Siehe A. S.: \emph{Tagebuch}, 12. 5. 1906.
               }}}\label{K_L03005-15}; ſehr erfüllt von ſeinem \label{K_L03005-16v}\edtext{Fünfabend Stück\pwindex{Historie von Koenig David. Ein Zyklus@\emph{Die Historie von König David. Ein Zyklus}|pwv}}{\lemma{\textnormal{\emph{Fünfabend Stück}}}\Cendnote{\textnormal{der Dramenzyklus \emph{Die Historie von König David}\pwindex{Historie von Koenig David. Ein Zyklus@\emph{Die Historie von König David. Ein Zyklus}|pwk}}}}\label{K_L03005-16}. Erfülltſein iſt doch der neidenswertheſte Zuſtand von allen; – we{\geminationn} nicht die Verpflichtungsgefühle ſich einſtellen – die
               oft trügeriſch ſind, we{\geminationn} ſie ſich auf uns ſelbſt, und
               immer we{\geminationn} ſie ſich auf die Welt (ſowohl »Mit« als
               »Nach«) {\pb}beziehen. Dies iſt eine Wahrheit.
               Sollte es aber nicht wahrere Wahrheiten geben?\pend
           
\pstart
           – Wir haben ein neues Fräulein, angenehm jüdiſch, Anna Loew\pwindex{Loew, Anna *~11.04.1888@\textsc{Loew, Anna} (*~11.04.1888), \emph{Kinderbetreuer/Kinderbetreuerin, Dienstbote/Dienstbotin}|pw} betitelt, und wegen einer Halsentzündg in Hinterbrühl\oindex{Hinterbruehl@\textbf{Hinterbrühl}, \emph{P.PPLA3}|pw} zurückgeblieben. Sie hat einen Bruder, \textsc{Johann Loew\pwindex{Loew, Johann @\textsc{Loew, Johann}, \emph{Arbeiterführer/Arbeiterführerin}|pw}}, Arbeiterführer, und ſo bekam ich plötzlich aus Brüſſel\oindex{Bruessel@\textbf{Brüssel}, \emph{P.PPLC}|pw} eine, \textsc{resp.} zwei waterlo\oindex{Waterloo@\textbf{Waterloo}, \emph{P.PPL}|pw}hende Karten, von \textsc{Johann Loew\pwindex{Loew, Johann @\textsc{Loew, Johann}, \emph{Arbeiterführer/Arbeiterführerin}|pw}} und \textsc{Lotte Pohl-Glas\pwindex{Pohl-Glas, Charlotte 1873-01-01 – 1944-02-15@\textsc{Pohl-Glas, Charlotte} (1873-01-01 – 1944-02-15), \emph{Schriftsteller/Schriftstellerin, Politiker/Politikerin, Sozialist/Sozialistin}|pw}}. Wer die Zuſa{\geminationm}enhänge begreift, lebt ewig.\pend
           
\pstart
           Dies wünſcht Ihnen, nebſt vielen herzlichen Güßen für Sie und die Ihren von uns
               allen. {\\[\baselineskip]}Ihr {\\[\baselineskip]}\spacefill\mbox{Arthur}\pend
           \leftskip=0em{}
\pstart
           \noindent{}Richard\pwindex{Beer-Hofmann, Richard 1866-07-11 – 1945-09-26@\textsc{Beer-Hofmann, Richard} (1866-07-11 – 1945-09-26), \emph{Schriftsteller/Schriftstellerin}|pw} hat zwei ſchöne \label{K_L03005-17v}\edtext{Gedichte\pwindex{einsame Weg@\emph{Der einsame Weg}|pw}\pwindex{Altern@\emph{Altern}|pw} geſchrieben, eins »Der einſame Weg\pwindex{einsame Weg@\emph{Der einsame Weg}|pw}}{\lemma{\textnormal{\emph{Gedichte … Weg}}}\Cendnote{\textnormal{Siehe Richard Beer-Hofmann an Arthur Schnitzler, [15.?] 5. 1906.
                  }}}\label{K_L03005-17}« – ein andres »Altern\pwindex{Altern@\emph{Altern}|pw}«, 1 an mich, 1 an \textsc{Kerr\pwindex{Kerr, Alfred 25.12.1867 – 12.10.1948@\textsc{Kerr, Alfred} (25.12.1867 – 12.10.1948), \emph{Schriftsteller/Schriftstellerin, Kritiker/Kritikerin}|pw}}.\pend
           \selectlanguage{ngerman}\endnumbering\briefempfaengerindex{Salten, Felix@\textsc{Salten, Felix}!zzzSchnitzler, Arthur@\emph{von Arthur Schnitzler}!1906-05-161@{16. 5. 1906}|)be}\mylabel{L03005h}  \normalsize

\doendnotes{C}
\bigskip
\vfill

\clearpage

\footnotesize

\lohead{\textsc{register}}

% Definiere theindex-Environment komplett neu ohne reledmac
\makeatletter
\renewenvironment{theindex}{%
  \section*{\indexname}%
  \setlength{\parindent}{0pt}%
  \setlength{\parskip}{0pt plus 0.3pt}%
  \let\item\@idxitem
}{%
  \clearpage
}
\makeatother

\IfFileExists{\jobname-pw.ind}{\input{\jobname-pw.ind}}{}

\end{document}

      