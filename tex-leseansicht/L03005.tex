%% latex-leseansicht-vorspann.tex
%% Vorspann für die Leseansicht.
%% Lädt die gemeinsame Datei latex-vorspann.tex mit nicht gesetztem Schalter.

\newif\ifkorrekturansicht
\korrekturansichtfalse

\input{../tex-inputs/latex-vorspann}

\begin{center}
            \textcolor{red}{ENTWURF, NICHT FERTIG KORRIGIERT}
                      \end{center}
            
         
         \newcommand{\erwaehntePersonen}{Personen: Albert Bassermann, Richard Beer-Hofmann, Dora Erl, Julius von Gans-Ludassy, Hugo von Hofmannsthal, Arthur Kaufmann, Anna Loew, Johann Loew, Charlotte Pohl-Glas, Jacob Pollak, Emanuel Reicher, Felix Salten}
         \newcommand{\erwaehnteInstitutionen}{}
         \newcommand{\erwaehnteOrte}{Orte: Berlin, Brüssel, Dresden, Griechenland, Heringsdorf, Hinterbrühl, Hotel Radetzky, Spanien, Wartburg, Waterloo, Wien, Świnoujście}
         \newcommand{\erwaehnteWerke}{
               \section[Arthur Schnitzler an Felix Salten, 16. 5. 1906]{ Arthur Schnitzler an Felix Salten, 16. 5. 1906}\nopagebreak\mylabel{v}\rehead{ }\begin{ledgroupsized}[t]{13cm}\normalsize\beginnumbering \toendnotes[C]{\smallbreak\pagebreak[2]} \Standort{Wienbibliothek im Rathaus, ZPH 1681, 2.1.516.}
\physDesc{
\newline{}Handschrift: , deutsche Kurrent}\toendnotes[C]{\smallbreak}\pstart
           \noindent{}{\pb}\textcolor{gray}{\textbf{Dr. Arthur Schnitzler}}\hfill 16. Mai 906\pend
           \pstart
           \textcolor{gray}{\textbf{Wien XVIII.
                        Spoettelgasse 7\oindex{XXXX Ortsangabe fehlt|pw}.}}\pend
           \pstart
           lieber, beim Nachhauſeko{\geminationm}en aus Theater
               und Hotel hab \label{T_L03005-1v}\edtext{ich}{\lemma{\textnormal{\emph{ich}}}\Cendnote{\textnormal{in der Vorlage steht »ich ich«}}}\label{T_L03005-1h} Ihren
               kurzen aber klingenden Brief vorgefunden und mich ſehr damit gefreut. Es mußte für
               mich freilich nicht gerade der Einſ. Weg\textcolor{red}{\textsuperscript{\textbf{KEY}}} kommen, um mich
               Ihr Fernſein ſchmerzlich empfinden zu laſſen. Der Abend geſtern iſt überraſchend gut
               ausgefallen: jedenfalls war er äußerlich der ſtärkſte Erfolg meiner Theaterlaufbahn.
               Völlige Stu{\geminationm}heit nach dem erſten Akt, wahre »Stürme«
               nach 2., 3., gedämpft nach dem 4{[}.{]}, wieder ſehr ſtark {\pb}nach dem 5. Akt. Baſſermann\pwindex{Bassermann, Albert 07.09.1867 – 15.05.1952@\textsc{Bassermann, Albert} (07.09.1867 – 15.05.1952), \emph{Schauspieler}|pw} anfangs etwas bläßlich, am Schluſs
                  unvergleichlich.Reicher\pwindex{Reicher, Emanuel 18.06.1849 – 15.05.1924@\textsc{Reicher, Emanuel} (18.06.1849 – 15.05.1924), \emph{Schauspieler}|pw} hat mich in
               gewiſſem Sinne angenehm enttäuſcht. Im ganzen war er wohl unerträglich genug; aber
               die Leiſtung als ganzes war von einer gewiſſen Geſchloſſenheit, ſo daſs man einen
               mehr menſchlichen als künſtleriſchen Widerwillen gegen die Figur kriegte.– Seltſam
               ſind doch Dramenſchicksale. Eine ſolche Aufnahme inBerlin\oindex{Berlin@\textbf{Berlin}|pw} vor 2½ Jahren – und Ihre Profezeihung wäre erfüllt geweſen. – Dem Rehberg\textcolor{red}{\textsuperscript{\textbf{KEY}}} hab ich in der Hinterbrühl\oindex{Hinterbruehl@\textbf{Hinterbrühl}|pw} geleſen, wo wir höchſt angenehme acht Tage im Hotel Radetzky\oindex{Hotel Radetzky@\textbf{Hotel Radetzky}|pw}{\pb}gewohnt und \textsc{tennis} geſpielt haben (Einmal mit Hugo\pwindex{Hofmannsthal, Hugo von 1874-02-01 – 1929-07-15@\textsc{Hofmannsthal, Hugo von} (1874-02-01 – 1929-07-15), \emph{Schriftsteller}|pw}, den ich im \textsc{single} set 6:4 ſchlug!) – Es iſt
               ein glänzendes Ding, und es gibt vielleicht im ganzen darin nur 3–5 Stellen, bei
               denen mir im Stil irgend was wie ein falſcher Ton erſcheint. Doch möcht ichs, nach
               einem Zwiſchenraum von ein paar Wochen, noch einmal leſen, um mich ſelber
               nachzuprüfen. Hingegen ſage ich ſchon heute mit Entſchiedenheit, daſs ich den
               vorletzten Abſatz fortwünſchte. Hier weden Zuſa{\geminationm}enhänge
               mit einer meinen Geſchmack ſtörenden Deutlichkeit aufgezeigt;\strikeout{die} Zuſa{\geminationm}enhänge, die im
               {\pb}Gang der Geſchichte \strikeout{wirklich} für jeden erſichtlich werden, der in
               anſtändiger Weiſe zu leſen verſteht, und mir erſchien daher dieſer ganze Abſatz wie
               eine Referenz vor den oberflächlichen, die ihnen nicht gebührt. Ich habe mich
               natürlich auch gefragt, ob dieſer Rückblick vielleicht als Ergänzung zum
               Charakterbild des Erzählers Ihnen unerläßlich ſcheinen mochte – doch find ich daſs
               die etwa neuen Züge höchſtens um Sinne philoſophiſcher Altersverän derungen zu deuten
               wären, die mit dem köſtlich-fertigen Chronik-Rehberg\textcolor{red}{\textsuperscript{\textbf{KEY}}}, den
               Sie geſtalteten, nichts weiter zu thun haben. Auch wirkt {\pb}die Stelle, wo Rehberg\textcolor{red}{\textsuperscript{\textbf{KEY}}} zum Selbſtankläger wird »Und da{\geminationn} hat mich dies\textcolor{red}{\textsuperscript{\textbf{KEY}}}Treiben ſo weit von meinem Worte fortgeriſſen\textcolor{red}{\textsuperscript{\textbf{KEY}}}\textsc{etc}« keineswegs bezwingend wahr. Weder ſubjectiv noch
               objektiv.– Ich würde daher in der Buchausgabe von dem Abſatz nur die erſten Zeilen
               ſtehen laſſen bei »als der Kaiſer gegen ihn
                  geweſen\textcolor{red}{\textsuperscript{\textbf{KEY}}}« – oder nicht einmal die – und ruhig auf den letzten Abſatz
               übergehen.– \pend
           \pstart
           Ihr Berlin\oindex{Berlin@\textbf{Berlin}|pw}er Feu{[}i{]}lleton\textcolor{red}{\textsuperscript{\textbf{KEY}}} in der Zeit\textcolor{red}{\textsuperscript{\textbf{KEY}}} hab ich mit Ergriffenheit geleſen. Sind {\pb}Sie nun ſchon an der \textsc{Herzl\textcolor{red}{\textsuperscript{\textbf{KEY}}}}-Biographie? Und welches ſind die größten Sachen, die Sie componiren? – Die Wartburger\oindex{Wartburg@\textbf{Wartburg}|pw}reiſe war ein Ausflug zum Vergnügen
               oder ſonſt was?– Wie ſtehts mit Spanien\oindex{Spanien@\textbf{Spanien}|pw}?– Unser
               Kinderarzt Dr \textsc{Pollak\pwindex{Pollak, Jacob 07.02.1860 – 25.03.1941@\textsc{Pollak, Jacob} (07.02.1860 – 25.03.1941), \emph{Mediziner}|pw}} theilt mir mit, dſs Heringsdorf\oindex{Heringsdorf@\textbf{Heringsdorf}|pw} u
               beſonders \textsc{Swinemünde\oindex{Swinoujście@\textbf{Świnoujście}|pw}} enorm gelſengeplagt ſind\footnote{\noindent{}Er war in Sw.\oindex{Swinoujście@\textbf{Świnoujście}|pw}} Erkundg Sie ſich doch gut, eh Sie miethen.– Eben bekam ich von Ludaſſy\pwindex{Gans-Ludassy, Julius von 13.04.1858 – 30.09.1922@\textsc{Gans-Ludassy, Julius von} (13.04.1858 – 30.09.1922), \emph{Schriftsteller, Journalist, Herausgeber}|pw} eine Gratul- karte zum geſtrigen
               Erfolg. Seine Frau\textcolor{red}{\textsuperscript{\textbf{KEY}}} hat eben eine
               ſchwere Lungenentzündg durchgemacht, und ich muſs ſie nächſtens beſuchen. So wär es
               mir ſehr lieb, {\pb}we{\geminationn} Sie mir raſch nur mit 2 Worten \strikeout{mit} ſagten, wie nun eigentlich Ihre Prozeſsſache ſteht. \pend
           \pstart
           Frl Erl\pwindex{Erl, Dora @\textsc{Erl, Dora}, \emph{Schauspielerin, Gesangspädagogin}|pw} iſt ab nach Dresden\oindex{Dresden@\textbf{Dresden}|pw} (vorläufg ohne beſti{\geminationm}tes
               Engagement) \textsc{Tennis} regelmäßig \textsc{Kaufma{\geminationn}\pwindex{Kaufmann, Arthur 04.04.1872 – 25.07.1938@\textsc{Kaufmann, Arthur} (04.04.1872 – 25.07.1938), \emph{Wissenschaftler, Privatgelehrte, Privatier}|pw}}, manchmal \textsc{Speidels\textcolor{red}{\textsuperscript{\textbf{KEY}}}} (er kam erſt jüngſt aus Griechenland\oindex{Griechenland@\textbf{Griechenland}|pw}
               zurück).– \pend
           \pstart
           – Richard\pwindex{Beer-Hofmann, Richard 1866-07-11 – 1945-09-26@\textsc{Beer-Hofmann, Richard} (1866-07-11 – 1945-09-26), \emph{Schriftsteller}|pw} war einmal bei uns in der Hinterbrühl\oindex{Hinterbruehl@\textbf{Hinterbrühl}|pw}, mit Paula\textcolor{red}{\textsuperscript{\textbf{KEY}}} u Mirjam\textcolor{red}{\textsuperscript{\textbf{KEY}}}; ſehr erfüllt von ſeinem Fünfabend-\textcolor{red}{\textsuperscript{\textbf{KEY}}}Stück\textcolor{red}{\textsuperscript{\textbf{KEY}}}. Erfülltſein iſt doch der
               neidenswertheſte Zuſtand von allen; – we{\geminationn} nicht die
               Verpflchtungsgefühle ſich einſtellen – die oft trügeriſch ſind, we{\geminationn} ſie ſich auf uns ſelbſt, und immer we{\geminationn} ſie ſich auf die Welt (ſowohl »Mit« als »Nach«) {\pb}beziehen. Dies iſt eine Wahrheit.
               Sollte es aber nicht wahrere Wahrheiten geben? \pend
           \pstart
           – Wir haben ein neues Fräulein, angenehm jüdiſch, Anna Loew\pwindex{Loew, Anna *~11.04.1888@\textsc{Loew, Anna} (*~11.04.1888), \emph{Kinderbetreuerin, Dienstbotin}|pw} betitelt, und wegen einer Halsentzündg in Hinterbrühl\oindex{Hinterbruehl@\textbf{Hinterbrühl}|pw} zurückgeblieben. Sie hat einen Bruder,\textsc{Johann Loew\pwindex{Loew, Johann @\textsc{Loew, Johann}, \emph{Arbeiterführer}|pw}}, Arbeiterführer, und ſo bekam ich plötzlich aus Brüſſel\oindex{Bruessel@\textbf{Brüssel}|pw} eine, \textsc{resp.} zwei waterlo\oindex{Waterloo@\textbf{Waterloo}|pw}hende Karten, von \textsc{Johann Loew\pwindex{Loew, Johann @\textsc{Loew, Johann}, \emph{Arbeiterführer}|pw}} und \textsc{Lotte Pohl-Glas\pwindex{Pohl-Glas, Charlotte 1873-01-01 – 1944-02-15@\textsc{Pohl-Glas, Charlotte} (1873-01-01 – 1944-02-15), \emph{Schriftstellerin, Politikerin, Sozialistin}|pw}}. Wer die Zuſa{\geminationm}enhänge begreift, lebt ewig. \pend
           \pstart
            Dies wünſcht Ihnen, nebſt vielen herz {\\}lichen Güßen für Sie und die Ihren
               {\\}von uns allen.\pend
           \pstart
           Ihr {\\[\baselineskip]}\spacefill\mbox{Arthur}\pend
           \leftskip=0em{}\pstart
           Richard\pwindex{Beer-Hofmann, Richard 1866-07-11 – 1945-09-26@\textsc{Beer-Hofmann, Richard} (1866-07-11 – 1945-09-26), \emph{Schriftsteller}|pw} hat zwei ſchöne Gedichte geſchrieben,
               eins »Der einſame Weg\textcolor{red}{\textsuperscript{\textbf{KEY}}}« u ein andres »Altern\textcolor{red}{\textsuperscript{\textbf{KEY}}}«, 1 an mich, 1 an \textsc{Kerr\textcolor{red}{\textsuperscript{\textbf{KEY}}}}. \pend
           
         
         \endnumbering\mylabel{h}\end{ledgroupsized}\begin{anhang}\end{anhang}\newcommand{\dateiname}{L03005}\newcommand{\titel}{Arthur Schnitzler an Felix Salten, 16. 5. 1906}\newcommand{\editorInnen}{Martin Anton Müller und Laura Untner}%% latex-leseansicht-abspann.tex
%% Abspann für die Leseansicht.
%% Der Schalter \ifkorrekturansicht ist bereits durch den Vorspann gesetzt.

%% latex-abspann.tex
%% Gemeinsamer Abspann für Korrekturansicht und Leseansicht.
%% Setzt den Schalter \ifkorrekturansicht voraus (gesetzt in den
%% einbindenden Dateien latex-korrekturansicht-abspann.tex bzw.
%% latex-leseansicht-abspann.tex).
%% ---------------------------------------------------------------

\normalsize

% Das esempio-Environment wird nur in der Leseansicht benötigt
\ifkorrekturansicht\else
\newenvironment{esempio}[3]%
{
    \vspace{1.5ex}
    \rlap{\underline{#1}}
    \par
    \setlength{\parindent}{0cm}
    \nopagebreak
    \leftskip=#2cm
    \rightskip=#3cm
}
{
    \par
}
\fi

\doendnotes{C}
\bigskip
\vfill

\clearpage

\footnotesize

\ifkorrekturansicht
  \lohead{\textsc{register}}
\fi

% theindex-Environment neu definieren ohne reledmac
\makeatletter
\renewenvironment{theindex}{%
  \ifkorrekturansicht
    \section*{\indexname}%
  \else
    \subsubsection*{Index der erwähnten Entitäten}%
  \fi
  \setlength{\parindent}{0pt}%
  \setlength{\parskip}{0pt plus 0.3pt}%
  \let\item\@idxitem
}{%
  \ifkorrekturansicht\clearpage\fi
}
\makeatother

\IfFileExists{\jobname-pw.ind}{\input{\jobname-pw.ind}}{}

% Quellenangabe nur in der Leseansicht
\ifkorrekturansicht\else
% Fallback-Definitionen, falls die .tex-Datei \titel etc. nicht gesetzt hat
\providecommand{\titel}{}
\providecommand{\editorInnen}{}
\providecommand{\dateiname}{\jobname}

\vspace{3cm}

\vfill

\footnotesize
\textsc{Quelle}: \titel. Herausgegeben von {\editorInnen}. In: \emph{Arthur Schnitzler: Briefwechsel mit Autorinnen und Autoren}.
 Digitale Edition, https://schnitzler-briefe.acdh.oeaw.ac.at/{\dateiname}.html (Stand \today)
\fi

\end{document}


      