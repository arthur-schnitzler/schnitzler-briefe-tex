%% latex-leseansicht-vorspann.tex
%% Vorspann für die Leseansicht.
%% Lädt die gemeinsame Datei latex-vorspann.tex mit nicht gesetztem Schalter.

\newif\ifkorrekturansicht
\korrekturansichtfalse

\input{../tex-inputs/latex-vorspann}


         
         \renewcommand{\erwaehntePersonen}{Personen: Aura Hertwig, Ottilie Salten, Felix Salten}
         \renewcommand{\erwaehnteInstitutionen}{Institutionen: Ullstein Verlag}
         \renewcommand{\erwaehnteOrte}{Orte: Berlin, Wien}
         \renewcommand{\erwaehnteWerke}{Werke: Schnitzler [Griff an den Bart, 1905]}
               \section[ Arthur Schnitzler an Ottilie Salten, 14. 1. 1906]{ Arthur Schnitzler an Ottilie Salten, 14. 1. 1906}\nopagebreak\mylabel{v}\rehead{ }\begin{ledgroupsized}[t]{13cm}\normalsize\beginnumbering\briefempfaengerindex{Salten, Ottilie@\textsc{Salten, Ottilie}!zzzSchnitzler, Arthur@\emph{von Arthur Schnitzler}!1906-01-141@{14. 1. 1906}|(be} \toendnotes[C]{\smallbreak\pagebreak[2]} \Standort{Wienbibliothek im Rathaus, Nachlass Salten, ZPH 1681, 17.3.11.11.40.6.}
\physDesc{Fotografie, 79 Zeichen
\newline{}Handschrift: schwarze Tinte, deutsche Kurrent
\newline{}Editorischer Hinweis: mit rotem Buntstift auf der Fotografie die handschriftliche Signatur »\textsc{Aura Hertwig}\pwindex{Hertwig, Aura 06.06.1861 – 28.09.1944@\textsc{Hertwig, Aura} (06.06.1861 – 28.09.1944), \emph{Fotografin}|pw}« und »1905« }\toendnotes[C]{\smallbreak}\begin{figure}[H]\centering\includegraphics[width=10cm]{../tex-inputs/img/ZPH1681_Box_17_3_11_11_40_5_0001_1.jpg}\end{figure}\pstart
           \noindent{}{\pb}\textsc{Frau Otti Salten}{ }{\\} zur \label{K_L03027-1v}\edtext{freundlichen
                     Erinnerung}{\lemma{\textnormal{\emph{freundlichen
                     Erinnerung}}}\Cendnote{\textnormal{Es handelte sich um das
                     Abschiedstreffen, weil Salten\pwindex{Salten, Felix 06.09.1869 – 08.10.1945@\textsc{Salten, Felix} (06.09.1869 – 08.10.1945), \emph{Schriftsteller, Journalist, Chefredakteur}|pwk} eine
                     Stellung beim \emph{Ullstein-Konzern}\orgindex{Ullstein Verlag@Ullstein Verlag|pwk} in Berlin\oindex{Berlin@\textbf{Berlin}|pwk} übernahm, siehe Felix Salten an Arthur Schnitzler, 29. 1. 1906. »Abd. Salten’s\pwindex{Salten, Felix 06.09.1869 – 08.10.1945@\textsc{Salten, Felix} (06.09.1869 – 08.10.1945), \emph{Schriftsteller, Journalist, Chefredakteur}|pw}\pwindex{Salten, Ottilie 07.03.1868 – 22.06.1942@\textsc{Salten, Ottilie} (07.03.1868 – 22.06.1942), \emph{Schauspielerin}|pw} bei uns. […]
                        Photographien angeschaut, ihm und ihr etliche gegeben. – Auf meine schrieb
                        ich ihm ›M. l. F. S.\pwindex{Salten, Felix 06.09.1869 – 08.10.1945@\textsc{Salten, Felix} (06.09.1869 – 08.10.1945), \emph{Schriftsteller, Journalist, Chefredakteur}|pw} nach 15
                        Jahren für alle weitern in Freundschaft herzlichst
                        A. S.‹ –« A. S.: \emph{Tagebuch}, 14. 1. 1906.
                  }}}\label{K_L03027-1h}\pwindex{Hertwig, Aura 06.06.1861 – 28.09.1944@\textsc{Hertwig, Aura} (06.06.1861 – 28.09.1944), \emph{Fotografin}!Schnitzler [Griff an den Bart, 1905]1905-11-25@\strich\emph{Schnitzler [Griff an den Bart, 1905]} {[}1905-11-25{]}|pwv}\pend
           \pstart Herzlichſt \spacefill\mbox{Arth Schnitzler}\pend{}\pstart
           14. 1. 906.\pend
           
         
         \endnumbering\mylabel{h}\end{ledgroupsized}  \newcommand{\dateiname}{L03027}\newcommand{\titel}{Arthur Schnitzler an Ottilie Salten, 14. 1. 1906}\newcommand{\editorInnen}{Martin Anton Müller und Laura Untner}%% latex-leseansicht-abspann.tex
%% Abspann für die Leseansicht.
%% Der Schalter \ifkorrekturansicht ist bereits durch den Vorspann gesetzt.

%% latex-abspann.tex
%% Gemeinsamer Abspann für Korrekturansicht und Leseansicht.
%% Setzt den Schalter \ifkorrekturansicht voraus (gesetzt in den
%% einbindenden Dateien latex-korrekturansicht-abspann.tex bzw.
%% latex-leseansicht-abspann.tex).
%% ---------------------------------------------------------------

\normalsize

% Das esempio-Environment wird nur in der Leseansicht benötigt
\ifkorrekturansicht\else
\newenvironment{esempio}[3]%
{
    \vspace{1.5ex}
    \rlap{\underline{#1}}
    \par
    \setlength{\parindent}{0cm}
    \nopagebreak
    \leftskip=#2cm
    \rightskip=#3cm
}
{
    \par
}
\fi

\doendnotes{C}
\bigskip
\vfill

\clearpage

\footnotesize

\ifkorrekturansicht
  \lohead{\textsc{register}}
\fi

% theindex-Environment neu definieren ohne reledmac
\makeatletter
\renewenvironment{theindex}{%
  \ifkorrekturansicht
    \section*{\indexname}%
  \else
    \subsubsection*{Index der erwähnten Entitäten}%
  \fi
  \setlength{\parindent}{0pt}%
  \setlength{\parskip}{0pt plus 0.3pt}%
  \let\item\@idxitem
}{%
  \ifkorrekturansicht\clearpage\fi
}
\makeatother

\IfFileExists{\jobname-pw.ind}{\input{\jobname-pw.ind}}{}

% Quellenangabe nur in der Leseansicht
\ifkorrekturansicht\else
% Fallback-Definitionen, falls die .tex-Datei \titel etc. nicht gesetzt hat
\providecommand{\titel}{}
\providecommand{\editorInnen}{}
\providecommand{\dateiname}{\jobname}

\vspace{3cm}

\vfill

\footnotesize
\textsc{Quelle}: \titel. Herausgegeben von {\editorInnen}. In: \emph{Arthur Schnitzler: Briefwechsel mit Autorinnen und Autoren}.
 Digitale Edition, https://schnitzler-briefe.acdh.oeaw.ac.at/{\dateiname}.html (Stand \today)
\fi

\end{document}


      