%% latex-leseansicht-vorspann.tex
%% Vorspann für die Leseansicht.
%% Lädt die gemeinsame Datei latex-vorspann.tex mit nicht gesetztem Schalter.

\newif\ifkorrekturansicht
\korrekturansichtfalse

\input{../tex-inputs/latex-vorspann}


               \section[Albert Ehrenstein an Arthur Schnitzler, 20. 10. 1908]{ Albert Ehrenstein an Arthur Schnitzler, 20. 10. 1908}\nopagebreak\mylabel{v}\rehead{ }\begin{ledgroupsized}[t]{13cm}\normalsize\beginnumbering\briefempfaengerindex{Schnitzler, Arthur@\textsc{Schnitzler, Arthur}!zzzEhrenstein, Albert@\emph{von Albert Ehrenstein}!1908-10-201@{20. 10. 1908}|(be} \toendnotes[C]{\smallbreak\pagebreak[2]} \Standort{CUL, Schnitzler, B 30.}
\physDesc{Brief, 1 Blatt, 2 Seiten
\newline{}Handschrift: schwarze Tinte, lateinische Kurrent
\newline{}Schnitzler: mit Bleistift beschriftet: »\textsc{Ehrenst\textcolor{gray}{e}i\textcolor{gray}{n}}« }\toendnotes[C]{\smallbreak}\pstart
           {\pb}Wien XVI. Ottakringerstr 114\oindex{Ottakringerstrasse@\textbf{Ottakringerstraße}|pw}\hfill 20. Okt. 08\pend
           \pstart{}Sehr geehrter Herr Doktor! \pend\pstart
           Am 10. Oktober, um ½ 5\textsuperscript{h} nachmittags war ich so rücksichtslos, bei Ihnen, sehr geehrter Herr
                    Doktor, ein in braunes Packpapier geschlagenes Manuscript\pwindex{Ehrenstein, Albert 23.12.1886 – 08.04.1950@\textsc{Ehrenstein, Albert} (23.12.1886 – 08.04.1950), \emph{Schriftsteller}!Seltene Gaeste1991 – 1991@\strich\emph{Seltene Gäste} {[}1991 – 1991{]}|pwuv} nebst inliegendem Briefe zu
                    hinterlassen. Da ich keine zweite Abschrift besitze, an jenem Tage im ganzen
                    Hause ein gewaltiger Rauch herrschte, die Sächelchen für mich einen gewissen
                    Affektionswert besitzen, würde es mir sehr angenehm sein, Wenn
                    Sie, sehr geehrter Herr Doktor, {\pb}mir den Empfang oder Nichtempfang
                    des unerfreulichen Packetes bestätigen zu wollen die Liebenswürdigkeit hätten.
                    Indem ich um Entschuldigung für diese Störung bitte, verbleibe ich
                    Hochachtungsvoll ergebenst Ihr Sie, sehr geehrter Herr Doktor verehrender\pend
           \pstart \spacefill\mbox{Albert Ehrenstein}\pend{}\pstart
           \noindent{}(XVI. Ottakringerstr 114\oindex{Ottakringerstrasse@\textbf{Ottakringerstraße}|pw}.)\pend
                     \endnumbering\briefempfaengerindex{Schnitzler, Arthur@\textsc{Schnitzler, Arthur}!zzzEhrenstein, Albert@\emph{von Albert Ehrenstein}!1908-10-201@{20. 10. 1908}|)be}\mylabel{h}\end{ledgroupsized}  \newcommand{\dateiname}{L01793}\newcommand{\titel}{Albert Ehrenstein an Arthur Schnitzler, 20. 10. 1908}\newcommand{\editorInnen}{Martin Anton Müller und Gerd-Hermann Susen}%% latex-leseansicht-abspann.tex
%% Abspann für die Leseansicht.
%% Der Schalter \ifkorrekturansicht ist bereits durch den Vorspann gesetzt.

%% latex-abspann.tex
%% Gemeinsamer Abspann für Korrekturansicht und Leseansicht.
%% Setzt den Schalter \ifkorrekturansicht voraus (gesetzt in den
%% einbindenden Dateien latex-korrekturansicht-abspann.tex bzw.
%% latex-leseansicht-abspann.tex).
%% ---------------------------------------------------------------

\normalsize

% Das esempio-Environment wird nur in der Leseansicht benötigt
\ifkorrekturansicht\else
\newenvironment{esempio}[3]%
{
    \vspace{1.5ex}
    \rlap{\underline{#1}}
    \par
    \setlength{\parindent}{0cm}
    \nopagebreak
    \leftskip=#2cm
    \rightskip=#3cm
}
{
    \par
}
\fi

\doendnotes{C}
\bigskip
\vfill

\clearpage

\footnotesize

\ifkorrekturansicht
  \lohead{\textsc{register}}
\fi

% theindex-Environment neu definieren ohne reledmac
\makeatletter
\renewenvironment{theindex}{%
  \ifkorrekturansicht
    \section*{\indexname}%
  \else
    \subsubsection*{Index der erwähnten Entitäten}%
  \fi
  \setlength{\parindent}{0pt}%
  \setlength{\parskip}{0pt plus 0.3pt}%
  \let\item\@idxitem
}{%
  \ifkorrekturansicht\clearpage\fi
}
\makeatother

\IfFileExists{\jobname-pw.ind}{\input{\jobname-pw.ind}}{}

% Quellenangabe nur in der Leseansicht
\ifkorrekturansicht\else
% Fallback-Definitionen, falls die .tex-Datei \titel etc. nicht gesetzt hat
\providecommand{\titel}{}
\providecommand{\editorInnen}{}
\providecommand{\dateiname}{\jobname}

\vspace{3cm}

\vfill

\footnotesize
\textsc{Quelle}: \titel. Herausgegeben von {\editorInnen}. In: \emph{Arthur Schnitzler: Briefwechsel mit Autorinnen und Autoren}.
 Digitale Edition, https://schnitzler-briefe.acdh.oeaw.ac.at/{\dateiname}.html (Stand \today)
\fi

\end{document}


      