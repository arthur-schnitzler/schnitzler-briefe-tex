%% latex-leseansicht-vorspann.tex
%% Vorspann für die Leseansicht.
%% Lädt die gemeinsame Datei latex-vorspann.tex mit nicht gesetztem Schalter.

\newif\ifkorrekturansicht
\korrekturansichtfalse

\input{../tex-inputs/latex-vorspann}

\begin{center}
            \textcolor{red}{ENTWURF. ENTZIFFERUNG NOCH NICHT KORREKTURGELESEN}
                      \end{center}
            
               \section[Hugo von Hofmannsthal an Arthur Schnitzler, 24. 4. {[}1897{]}]{ Hugo von Hofmannsthal an Arthur Schnitzler, 24. 4. {[}1897{]}}\nopagebreak\mylabel{v}\rehead{ }\begin{ledgroupsized}[t]{13cm}\normalsize\beginnumbering\briefempfaengerindex{Schnitzler, Arthur@\textsc{Schnitzler, Arthur}!zzzHofmannsthal, Hugo von@\emph{von Hugo von Hofmannsthal}!1897-04-241@{24. 4. {[}1897{]}}|(be} \toendnotes[C]{\smallbreak\pagebreak[2]} \Standort{CUL, Schnitzler, B 43.}
\physDesc{Brief, 2 Blätter, 5 Seiten
\newline{}Handschrift: 1) schwarze Tinte, deutsche Kurrent\hspace{1em}2) Bleistift, deutsche Kurrent (\noindent{}ab »Eben kommt«)\hspace{1em}
\newline{}Schnitzler: mit Bleistift die Jahreszahl ergänzt: »97« \newline{}Ordnung: 1) mit Bleistift von unbekannter Hand nummeriert: »88« und paginiert 1–2 2) mit Bleistift von unbekannter Hand nummeriert: »87«}\buchAbdrucke{\weitereDrucke{Hugo von Hofmannsthal, Arthur Schnitzler: \emph{Briefwechsel}. Hg. Therese Nickl und Heinrich Schnitzler. Frankfurt am Main: \emph{S. Fischer} 1964, S. 80.} }\toendnotes[C]{\smallbreak}\pstart
           \raggedleft{}{\pb}Wien\oindex{Wien@\textbf{Wien}|pw}{ }24\textsuperscript{ten} April\pend
           \pstart{}mein lieber Arthur\pend\pstart
           zuerſt kommt eine dumme Geſchichte, dann anderes. Die »Mimi\pwindex{Mimi1.4.1897 – 1.4.1897@\emph{Mimi} {[}1.4.1897 – 1.4.1897{]}|pw}« von der Clara
                        Loeb\pwindex{Pollaczek, Clara Katharina 15.01.1875 – 22.07.1951@\textsc{Pollaczek, Clara Katharina} (15.01.1875 – 22.07.1951), \emph{Schriftstellerin}|pw}{ }ſteht ſeit 10 Tagen in der »Freien Bühne\pwindex{Freie Buehne fuer den Entwickelungskampf der Zeit1892 – 1893@\emph{Freie Bühne für den Entwickelungskampf der Zeit}|pw}«, natürlich iſt es herausgekommen von wem es
                    iſt.\pend
           \pstart
           Zum Theil hat die Minnie B.\pwindex{Benedict, Marianne 01.01.1848 – 12.05.1930@\textsc{Benedict, Marianne} (01.01.1848 – 12.05.1930)|pw} einen recht
                    überflüſſigen Tratſch angefangen (komiſch muſs ſich das alles in Paris\oindex{Paris@\textbf{Paris}|pw}{ }{\pb}anhören) andersſeits hat
                    jemand recht gemeiner den Eltern Loeb\pwindex{Loeb, Louis 29.06.1842 – 06.06.1921@\textsc{Loeb, Louis} (29.06.1842 – 06.06.1921), \emph{Bankier}|pw}\pwindex{Loeb, Regina 1850 – 5.2.1918@\textsc{Loeb, Regina} (1850 – 5.2.1918)|pw}
                    einen anonymen Brief geſchrieben, kurz heute Früh läſst mich die Mutter\pwindex{Loeb, Regina 1850 – 5.2.1918@\textsc{Loeb, Regina} (1850 – 5.2.1918)|pwv} bitten hinzukommen.
                    Die Clara\pwindex{Pollaczek, Clara Katharina 15.01.1875 – 22.07.1951@\textsc{Pollaczek, Clara Katharina} (15.01.1875 – 22.07.1951), \emph{Schriftstellerin}|pw} war nicht zu ſehen, die Anna\pwindex{Epstein, Anna 6.3.1877 – 16.3.1943@\textsc{Epstein, Anna} (6.3.1877 – 16.3.1943)|pw} und die Mutter\pwindex{Loeb, Regina 1850 – 5.2.1918@\textsc{Loeb, Regina} (1850 – 5.2.1918)|pwv} verweint wie bei einem Leichenbegängnis, der
                        Vater\pwindex{Loeb, Louis 29.06.1842 – 06.06.1921@\textsc{Loeb, Louis} (29.06.1842 – 06.06.1921), \emph{Bankier}|pwv} ganz blaſs und
                    mit zitternder Stimme. Das weitere iſt unintereſſant; ich glaube daſs ich ſie
                    doch ein biſſel herumgekriegt {\pb}habe; \uuline{Ihre} active Theilnahme hab ich
                    verſchwiegen, weil die Mutter\pwindex{Loeb, Regina 1850 – 5.2.1918@\textsc{Loeb, Regina} (1850 – 5.2.1918)|pwv} ohnehin eine ſchlechte moraliſche Meinung von Ihnen hat,
                    während ich doch ſo brav und anſtändig bin. (Hoch!)\pend
           \pstart
           Zum Schluſs waren ſie faſt gerührt über mich und vielleicht laſſen \strikeout{S}ſie mich noch die Männer für die Mädeln ausſuchen.
                    Von Ihnen aber will ich nur zweierlei: 1.) wenn irgend jemand bei Ihnen anfragt
                    (bei der rätſelhaften Stellung, die {\pb}die Minnie\pwindex{Benedict, Marianne 01.01.1848 – 12.05.1930@\textsc{Benedict, Marianne} (01.01.1848 – 12.05.1930)|pw} zu der Geſchichte hat, iſt alles möglich) ſo
                    wiſſen Sie einfach nicht, wer die Verfaſſerin\pwindex{Pollaczek, Clara Katharina 15.01.1875 – 22.07.1951@\textsc{Pollaczek, Clara Katharina} (15.01.1875 – 22.07.1951), \emph{Schriftstellerin}|pwv} iſt.\pend
           \pstart
           2.) Sie müſſen ſo gut ſein, ſofort an Fiſcher\pwindex{Fischer, Samuel 24.12.1859 – 15.10.1934@\textsc{Fischer, Samuel} (24.12.1859 – 15.10.1934), \emph{Verleger}|pw}{ }ſchreiben, daſs der Druck des Buches
                    unterbleibt und er das Manuſcript umgehend an mich zurück ſchicken ſoll. Sie
                    müſſen das von Ihrem Verleger\pwindex{Fischer, Samuel 24.12.1859 – 15.10.1934@\textsc{Fischer, Samuel} (24.12.1859 – 15.10.1934), \emph{Verleger}|pwv} als perſönliche Gefälligkeit verlangen. Ich habe es den Eltern\pwindex{Loeb, Louis 29.06.1842 – 06.06.1921@\textsc{Loeb, Louis} (29.06.1842 – 06.06.1921), \emph{Bankier}|pwv}\pwindex{Loeb, Regina 1850 – 5.2.1918@\textsc{Loeb, Regina} (1850 – 5.2.1918)|pwv} beſtimmt
                    verſprochen, mir zu liebe tut er es aber vielleicht nicht, weil {\pb}es ihm etwa unbequem iſt.
                    Alſo bitte, \uline{ſofort}.\pend
           \pstart
           \centering{}Das Andere.\pend
           \pstart
           \noindent{}was eſſen Sie in Paris\oindex{Paris@\textbf{Paris}|pw}{ }ſtatt des gemiſchten Hausbrotes?\pend
           \pstart
           \noindent{}\label{T_L00669_1v}\edtext{Eben kommt Hirschfeld\pwindex{Hirschfeld, Georg 11.02.1873 – 17.01.1942@\textsc{Hirschfeld, Georg} (11.02.1873 – 17.01.1942), \emph{Schriftsteller}|pw}.}{\lemma{\textnormal{\emph{Eben kommt Hirschfeld.}}}\Cendnote{\textnormal{ab
                        hier Bleistift.}}}\label{T_L00669_1h}\pend
           \pstart
           Muſs für heute ſchließen.\pend
           \pstart
           Grüße Goldmann\pwindex{Goldmann, Paul 31.01.1865 – 25.09.1935@\textsc{Goldmann, Paul} (31.01.1865 – 25.09.1935), \emph{Schriftsteller, Journalist}|pw}.\pend
           \pstart
           Ihr{\\[\baselineskip]}\spacefill\mbox{Hugo}\pend
           \leftskip=0em{}\endnumbering\briefempfaengerindex{Schnitzler, Arthur@\textsc{Schnitzler, Arthur}!zzzHofmannsthal, Hugo von@\emph{von Hugo von Hofmannsthal}!1897-04-241@{24. 4. {[}1897{]}}|)be}\mylabel{h}\end{ledgroupsized}  \newcommand{\dateiname}{L00669}\newcommand{\titel}{Hugo von Hofmannsthal an Arthur Schnitzler, 24. 4. [1897]}\newcommand{\editorInnen}{Martin Anton Müller und Gerd-Hermann Susen}%% latex-leseansicht-abspann.tex
%% Abspann für die Leseansicht.
%% Der Schalter \ifkorrekturansicht ist bereits durch den Vorspann gesetzt.

%% latex-abspann.tex
%% Gemeinsamer Abspann für Korrekturansicht und Leseansicht.
%% Setzt den Schalter \ifkorrekturansicht voraus (gesetzt in den
%% einbindenden Dateien latex-korrekturansicht-abspann.tex bzw.
%% latex-leseansicht-abspann.tex).
%% ---------------------------------------------------------------

\normalsize

% Das esempio-Environment wird nur in der Leseansicht benötigt
\ifkorrekturansicht\else
\newenvironment{esempio}[3]%
{
    \vspace{1.5ex}
    \rlap{\underline{#1}}
    \par
    \setlength{\parindent}{0cm}
    \nopagebreak
    \leftskip=#2cm
    \rightskip=#3cm
}
{
    \par
}
\fi

\doendnotes{C}
\bigskip
\vfill

\clearpage

\footnotesize

\ifkorrekturansicht
  \lohead{\textsc{register}}
\fi

% theindex-Environment neu definieren ohne reledmac
\makeatletter
\renewenvironment{theindex}{%
  \ifkorrekturansicht
    \section*{\indexname}%
  \else
    \subsubsection*{Index der erwähnten Entitäten}%
  \fi
  \setlength{\parindent}{0pt}%
  \setlength{\parskip}{0pt plus 0.3pt}%
  \let\item\@idxitem
}{%
  \ifkorrekturansicht\clearpage\fi
}
\makeatother

\IfFileExists{\jobname-pw.ind}{\input{\jobname-pw.ind}}{}

% Quellenangabe nur in der Leseansicht
\ifkorrekturansicht\else
% Fallback-Definitionen, falls die .tex-Datei \titel etc. nicht gesetzt hat
\providecommand{\titel}{}
\providecommand{\editorInnen}{}
\providecommand{\dateiname}{\jobname}

\vspace{3cm}

\vfill

\footnotesize
\textsc{Quelle}: \titel. Herausgegeben von {\editorInnen}. In: \emph{Arthur Schnitzler: Briefwechsel mit Autorinnen und Autoren}.
 Digitale Edition, https://schnitzler-briefe.acdh.oeaw.ac.at/{\dateiname}.html (Stand \today)
\fi

\end{document}


      