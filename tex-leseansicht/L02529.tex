%% latex-leseansicht-vorspann.tex
%% Vorspann für die Leseansicht.
%% Lädt die gemeinsame Datei latex-vorspann.tex mit nicht gesetztem Schalter.

\newif\ifkorrekturansicht
\korrekturansichtfalse

\input{../tex-inputs/latex-vorspann}


               \section[Gerty von Hofmannsthal an Arthur Schnitzler, 2. 1. 1930]{ Gerty von Hofmannsthal an Arthur Schnitzler, 2. 1. 1930}\nopagebreak\mylabel{v}\rehead{ }\begin{ledgroupsized}[t]{13cm}\normalsize\beginnumbering\briefempfaengerindex{Schnitzler, Arthur@\textsc{Schnitzler, Arthur}!zzzHofmannsthal, Gertrude von@\emph{von Gertrude von Hofmannsthal}!1930-01-021@{2. 1. 1930}|(be} \toendnotes[C]{\smallbreak\pagebreak[2]} \Standort{CUL, Schnitzler, B 43.}
\physDesc{Bildpostkarte
\newline{}Handschrift: blaue Tinte, lateinische Kurrent\newline{}Versand: Stempel: »\nobreak{}\oindex{Heidelberg@\textbf{Heidelberg}|pwk}Heidelberg, \textcolor{gray}{3. 1.} 30, 11–12V\nobreak{}«.  
\newline{}Schnitzler: mit rotem Buntstift beschriftet »\textsc{H\textcolor{gray}{ofm}}« }\toendnotes[C]{\smallbreak}\pstart{}{\pb}Herrn\pend{}\pstart{}Arthur Schnitzler\pend{}\pstart{}Wien XVIII\oindex{XVIII., Waehring@\textbf{XVIII., Währing}|pw}\pend{}\pstart{}Sternwartestrasse 71\oindex{Sternwartestrasse@\textbf{Sternwartestraße}|pw}\pend{}{\bigskip}\pstart
           \noindent{}\centering{}{\pb}{[}Heidelberg\oindex{Heidelberg@\textbf{Heidelberg}|pw}{]}\pend
           \pstart
           \raggedleft{}{\pb}Heidelberg\oindex{Heidelberg@\textbf{Heidelberg}|pw} d. 2/I 30\pend
           \pstart
           Lieber Arthur, ich konnte leider \label{K_L02529_1v}\edtext{neulich}{\lemma{\textnormal{\emph{neulich}}}\Cendnote{\textnormal{Offensichtlich war sie bei der
                        Uraufführung von \emph{Im Spiel der Sommerlüfte}\pwindex{Schnitzler, Arthur 15.05.1862 – 21.10.1931@\textsc{Schnitzler, Arthur} (15.05.1862 – 21.10.1931), \emph{Schriftsteller, Mediziner}!Im Spiel der Sommerluefte. In drei Aufzuegen1929-12-21 – 1929-12-21@\strich\emph{Im Spiel der Sommerlüfte. In drei Aufzügen} {[}1929-12-21 – 1929-12-21{]}|pwk}
                        am 21. 12. 1929.}}}\label{K_L02529_1h} nicht mehr sagen \uline{wie}{ }\uline{sehr} mich das Stück\pwindex{Schnitzler, Arthur 15.05.1862 – 21.10.1931@\textsc{Schnitzler, Arthur} (15.05.1862 – 21.10.1931), \emph{Schriftsteller, Mediziner}!Im Spiel der Sommerluefte. In drei Aufzuegen1929-12-21 – 1929-12-21@\strich\emph{Im Spiel der Sommerlüfte. In drei Aufzügen} {[}1929-12-21 – 1929-12-21{]}|pw} inter\textcolor{gray}{essiert} hat, wie ausgezeichnet die
                    Darstellung war! Danke Ihnen sehr!\pend
           \pstart
           Alles Liebe von uns allen und herzl. Grüsse{\\[\baselineskip]}Ihre{\\[\baselineskip]}\spacefill\mbox{Gerty}\pend
           \leftskip=0em{}\endnumbering\briefempfaengerindex{Schnitzler, Arthur@\textsc{Schnitzler, Arthur}!zzzHofmannsthal, Gertrude von@\emph{von Gertrude von Hofmannsthal}!1930-01-021@{2. 1. 1930}|)be}\mylabel{h}\end{ledgroupsized}  \newcommand{\dateiname}{L02529}\newcommand{\titel}{Gerty von Hofmannsthal an Arthur Schnitzler, 2. 1. 1930}\newcommand{\editorInnen}{Martin Anton Müller und Gerd-Hermann Susen}%% latex-leseansicht-abspann.tex
%% Abspann für die Leseansicht.
%% Der Schalter \ifkorrekturansicht ist bereits durch den Vorspann gesetzt.

%% latex-abspann.tex
%% Gemeinsamer Abspann für Korrekturansicht und Leseansicht.
%% Setzt den Schalter \ifkorrekturansicht voraus (gesetzt in den
%% einbindenden Dateien latex-korrekturansicht-abspann.tex bzw.
%% latex-leseansicht-abspann.tex).
%% ---------------------------------------------------------------

\normalsize

% Das esempio-Environment wird nur in der Leseansicht benötigt
\ifkorrekturansicht\else
\newenvironment{esempio}[3]%
{
    \vspace{1.5ex}
    \rlap{\underline{#1}}
    \par
    \setlength{\parindent}{0cm}
    \nopagebreak
    \leftskip=#2cm
    \rightskip=#3cm
}
{
    \par
}
\fi

\doendnotes{C}
\bigskip
\vfill

\clearpage

\footnotesize

\ifkorrekturansicht
  \lohead{\textsc{register}}
\fi

% theindex-Environment neu definieren ohne reledmac
\makeatletter
\renewenvironment{theindex}{%
  \ifkorrekturansicht
    \section*{\indexname}%
  \else
    \subsubsection*{Index der erwähnten Entitäten}%
  \fi
  \setlength{\parindent}{0pt}%
  \setlength{\parskip}{0pt plus 0.3pt}%
  \let\item\@idxitem
}{%
  \ifkorrekturansicht\clearpage\fi
}
\makeatother

\IfFileExists{\jobname-pw.ind}{\input{\jobname-pw.ind}}{}

% Quellenangabe nur in der Leseansicht
\ifkorrekturansicht\else
% Fallback-Definitionen, falls die .tex-Datei \titel etc. nicht gesetzt hat
\providecommand{\titel}{}
\providecommand{\editorInnen}{}
\providecommand{\dateiname}{\jobname}

\vspace{3cm}

\vfill

\footnotesize
\textsc{Quelle}: \titel. Herausgegeben von {\editorInnen}. In: \emph{Arthur Schnitzler: Briefwechsel mit Autorinnen und Autoren}.
 Digitale Edition, https://schnitzler-briefe.acdh.oeaw.ac.at/{\dateiname}.html (Stand \today)
\fi

\end{document}


      