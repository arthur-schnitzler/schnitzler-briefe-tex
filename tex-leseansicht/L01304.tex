%% latex-korrekturansicht-vorspann.tex
%% Vorspann für die Korrekturansicht.
%% Lädt die gemeinsame Datei latex-vorspann.tex mit gesetztem Schalter.

\newif\ifkorrekturansicht
\korrekturansichttrue

\input{../tex-inputs/latex-vorspann}


\section[Arthur Schnitzler an Hermann Bahr, 19. 7. 1903]{L01304 Arthur Schnitzler an Hermann Bahr, 19. 7. 1903}
\nopagebreak\mylabel{L01304v}
\rehead{ }\normalsize\beginnumbering\briefempfaengerindex{Bahr, Hermann@\textsc{Bahr, Hermann}!zzzSchnitzler, Arthur@\emph{von Arthur Schnitzler}!1903-07-191@{19. 7. 1903}|(be}
\toendnotes[C]{\smallbreak\pagebreak[2]}\Standort{TMW, FS PK277797 alt.}
\physDesc{Fotografie, 1 Blatt, 1 Seite, 137 Zeichen (Foto Aura Hertwig\pwindex{Hertwig, Aura 06.06.1861 – 28.09.1944@\textsc{Hertwig, Aura} (06.06.1861 – 28.09.1944), \emph{Fotograf/Fotografin}|pw}{ }Berlin\oindex{Berlin@\textbf{Berlin}, \emph{P.PPLC}|pw}{ }1903)
\newline{}Handschrift: schwarze Tinte, deutsche Kurrent
\newline{}Zusatz: Passepartoutreste weisen auf eine frühere Rahmung }
\buchAbdrucke{\weitereDrucke{Hermann Bahr, Arthur Schnitzler: \emph{Briefwechsel, Aufzeichnungen, Dokumente (1891–1931)}. Göttingen: \emph{Wallstein} 2018, S. 268.} }\toendnotes[C]{\smallbreak}\begin{figure}[H]\centering\includegraphics[width=9cm]{../tex-inputs/img/FS_PK266826_V.jpg}\end{figure}\vspace{1em}
\pstart
           \noindent{}{\pb}Erinner dich, wie oft
               du ſchon alt warſt, und freu dich, wie oft du noch jung ſein wirſt!\pwindex{Arthur Schnitzler [Halbprofil 1903]@\emph{Arthur Schnitzler [Halbprofil 1903]}|pwv}\pend
           
\pstart
           \label{K_L01304-1v}\edtext{Zum 19. Juli 1903}{\lemma{\textnormal{\emph{Zum 19. Juli 1903}}}\Cendnote{\textnormal{Bahrs\pwindex{Bahr, Hermann 19.07.1863 – 15.01.1934@\textsc{Bahr, Hermann} (19.07.1863 – 15.01.1934), \emph{Schriftsteller/Schriftstellerin, Kritiker/Kritikerin}|pwk} 40. Geburtstag.}}}\label{K_L01304-1}{\\}meinem lieben Hermann\pend
           \pstart \spacefill\mbox{Arthur Schn}\pend{}\selectlanguage{ngerman}\endnumbering\briefempfaengerindex{Bahr, Hermann@\textsc{Bahr, Hermann}!zzzSchnitzler, Arthur@\emph{von Arthur Schnitzler}!1903-07-191@{19. 7. 1903}|)be}\mylabel{L01304h}  \normalsize

\doendnotes{C}
\bigskip
\vfill

\clearpage

\footnotesize

\lohead{\textsc{register}}

% Definiere theindex-Environment komplett neu ohne reledmac
\makeatletter
\renewenvironment{theindex}{%
  \section*{\indexname}%
  \setlength{\parindent}{0pt}%
  \setlength{\parskip}{0pt plus 0.3pt}%
  \let\item\@idxitem
}{%
  \clearpage
}
\makeatother

\IfFileExists{\jobname-pw.ind}{\input{\jobname-pw.ind}}{}

\end{document}

      