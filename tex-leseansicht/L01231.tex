%% latex-korrekturansicht-vorspann.tex
%% Vorspann für die Korrekturansicht.
%% Lädt die gemeinsame Datei latex-vorspann.tex mit gesetztem Schalter.

\newif\ifkorrekturansicht
\korrekturansichttrue

\input{../tex-inputs/latex-vorspann}


\section[Arthur Schnitzler an Hermann Bahr, 13. 7. 1902]{L01231 Arthur Schnitzler an Hermann Bahr, 13. 7. 1902}
\nopagebreak\mylabel{L01231v}
\rehead{ }\normalsize\beginnumbering\briefempfaengerindex{Bahr, Hermann@\textsc{Bahr, Hermann}!zzzSchnitzler, Arthur@\emph{von Arthur Schnitzler}!1902-07-131@{13. 7. 1902}|(be}
\toendnotes[C]{\smallbreak\pagebreak[2]}\Standort{TMW, HS AM 23352 Ba.}
\physDesc{Brief, 1 Blatt, 2 Seiten, 456 Zeichen
\newline{}Handschrift: schwarze Tinte, deutsche Kurrent
\newline{}Bahr: das Wort »gutächtlich« mit rotem Buntstift
                                 unterstrichen und mit »?« versehen 
\newline{}Ordnung: Lochung }
\buchAbdrucke{\weitereDrucke{1) Arthur Schnitzler: \emph{The Letters of Arthur Schnitzler to Hermann Bahr}. Chapel Hill: \emph{The University of North Carolina Press} 1978, S. 76.} \weitereDrucke{2) Hermann Bahr, Arthur Schnitzler: \emph{Briefwechsel, Aufzeichnungen, Dokumente (1891–1931)}. Göttingen: \emph{Wallstein} 2018, S. 241.} }
\pstart{}{\pb}mein lieber
                  Hermann,\pend\vspace{0.5em}
\pstart
           es war von allem Anfang an meine Abſicht, der »Verpflichtg« mich gutächtlich zu
               äußern, nur negativ nachzukommen und ſchrieb dir eben, hauptſächlich, um dir falls du
               irgendwelchen ſpez. Wunsch hätteſt, gefällig zu ſein. Ich habe jetzt, wohl auch in
               deinem Sinn geantwortet, dſs ich keinerlei Anlaſs u Neigung habe mich um das
               Einkommen {\pb}von anderen
               Leuten zu kümmern u deshalb \textsc{etc etc.} –\pend
           
\pstart
           Auf baldg Wiederſehn,{\\[\baselineskip]}herzlichſt dein{\\[\baselineskip]}\spacefill\mbox{Arthur}\pend
           \leftskip=0em{}
\pstart
           13. 7. 902\pend
           \selectlanguage{ngerman}\endnumbering\briefempfaengerindex{Bahr, Hermann@\textsc{Bahr, Hermann}!zzzSchnitzler, Arthur@\emph{von Arthur Schnitzler}!1902-07-131@{13. 7. 1902}|)be}\mylabel{L01231h}  \normalsize

\doendnotes{C}
\bigskip
\vfill

\clearpage

\footnotesize

\lohead{\textsc{register}}

% Definiere theindex-Environment komplett neu ohne reledmac
\makeatletter
\renewenvironment{theindex}{%
  \section*{\indexname}%
  \setlength{\parindent}{0pt}%
  \setlength{\parskip}{0pt plus 0.3pt}%
  \let\item\@idxitem
}{%
  \clearpage
}
\makeatother

\IfFileExists{\jobname-pw.ind}{\input{\jobname-pw.ind}}{}

\end{document}

      