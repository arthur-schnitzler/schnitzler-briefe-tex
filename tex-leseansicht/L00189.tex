%% latex-korrekturansicht-vorspann.tex
%% Vorspann für die Korrekturansicht.
%% Lädt die gemeinsame Datei latex-vorspann.tex mit gesetztem Schalter.

\newif\ifkorrekturansicht
\korrekturansichttrue

\input{../tex-inputs/latex-vorspann}


\section[Arthur Schnitzler an Richard Beer-Hofmann, {[}14. 3. 1893?{]}]{L00189 Arthur Schnitzler an Richard Beer-Hofmann, {[}14. 3. 1893?{]}}
\nopagebreak\mylabel{L00189v}
\rehead{ }\normalsize\beginnumbering\briefempfaengerindex{Beer-Hofmann, Richard@\textsc{Beer-Hofmann, Richard}!zzzSchnitzler, Arthur@\emph{von Arthur Schnitzler}!1893-03-141@{{[}14. 3. 1893?{]}}|(be}
\toendnotes[C]{\smallbreak\pagebreak[2]}\Standort{YCGL, MSS 31.}
\physDesc{Visitenkarte, , Umschlag, 285 Zeichen
\newline{}Handschrift: Bleistift, deutsche Kurrent
\newline{}Versand: ohne postalischen Übermittlungsvermerk }\toendnotes[C]{\smallbreak}\pstart{}{\pb}\damage{Hr}
                  n 
                  \textsc{Dr. Rich. Beer-Hofmann}\pend{}\pstart{}\textsc{Wien}\oindex{Wien@\textbf{Wien}, \emph{A.ADM2}|pw}\pend{}\pstart{}\textsc{I Wollzeile 15}\oindex{Wollzeile@\textbf{Wollzeile}, \emph{Straße (K.STR)}|pw}
                  .
               \pend{}{\bigskip}\vspace{1em}
\pstart
           \noindent{}{\pb}Lieber Richard!
                Es wäre 
               \uline{ſehr}
                hübſch,
                  we
               {\geminationn}
                Sie heute Abend auf dieſen Sitz (neben mir) ins
                  
               \label{K_L00189-1v}\edtext{
               Concert
               }{\lemma{\textnormal{\emph{
               Concert
               }}}\Cendnote{\textnormal{
                  Das Korrespondenzstück ist undatiert. Aber nur für den 
                  14. 3. 1893
                   lassen
                  sich die beiden hier erwähnten Besuche, bei 
                  Alexander Singer\pwindex{Singer, Alexander 16.11.1841 – 30.11.1906@\textsc{Singer, Alexander} (16.11.1841 – 30.11.1906), \emph{Herausgeber/Herausgeberin, Administrator/Administratorin}|pwk}
                   und in einem abendlichen Konzert, nachweisen. Es handelt
                     sich um einen Auftritt des 
                  \emph{Rosé-Quartetts}\orgindex{Rose-Quartett@Rosé-Quartett|pwk}
                  . (
                  \emph{Cambridge University Library}
                     , Schnitzler,
                  A 179
                  
                  .)
               }}}\label{K_L00189-1}
                gehen wollten. Sollten Sie 
               \uline{abſolut}
                nicht 
               {\pb}
               Luſt haben, ſo ſenden Sie
                  
               \introOben{}mir\introOben{}
                ihn 
               \strikeout{\textcolor{gray}{gef}\textcolor{gray}{×}\-\textcolor{gray}{×}\-\textcolor{gray}{×}\-\textcolor{gray}{×}\-\textcolor{gray}{×}\-\textcolor{gray}{×}\-\textcolor{gray}{×}\-\textcolor{gray}{×}\-\textcolor{gray}{×}\-\textcolor{gray}{×}\-\textcolor{gray}{×}\-\textcolor{gray}{×}\-\textcolor{gray}{×}\-\textcolor{gray}{×}\-\textcolor{gray}{×}
                   ja?
               }
                – Aber ich hoffe,
               Sie ko
               {\geminationm}
               en.
            \pend
           
\pstart
           \centering{}\textcolor{gray}{\textbf{
                  D
                  \textsuperscript{r}
                   Arthur Schnitzler
               }}\pend
           
\pstart
           
               Vorher bin ich bei 
               \textsc{Singer\pwindex{Singer, Alexander 16.11.1841 – 30.11.1906@\textsc{Singer, Alexander} (16.11.1841 – 30.11.1906), \emph{Herausgeber/Herausgeberin, Administrator/Administratorin}|pw}}
               , vielleicht Sie auch?
            \pend
           \selectlanguage{ngerman}\endnumbering\briefempfaengerindex{Beer-Hofmann, Richard@\textsc{Beer-Hofmann, Richard}!zzzSchnitzler, Arthur@\emph{von Arthur Schnitzler}!1893-03-141@{{[}14. 3. 1893?{]}}|)be}\mylabel{L00189h}  \normalsize

\doendnotes{C}
\bigskip
\vfill

\clearpage

\footnotesize

\lohead{\textsc{register}}

% Definiere theindex-Environment komplett neu ohne reledmac
\makeatletter
\renewenvironment{theindex}{%
  \section*{\indexname}%
  \setlength{\parindent}{0pt}%
  \setlength{\parskip}{0pt plus 0.3pt}%
  \let\item\@idxitem
}{%
  \clearpage
}
\makeatother

\IfFileExists{\jobname-pw.ind}{\input{\jobname-pw.ind}}{}

\end{document}

      