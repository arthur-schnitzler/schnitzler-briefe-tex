%% latex-leseansicht-vorspann.tex
%% Vorspann für die Leseansicht.
%% Lädt die gemeinsame Datei latex-vorspann.tex mit nicht gesetztem Schalter.

\newif\ifkorrekturansicht
\korrekturansichtfalse

\input{../tex-inputs/latex-vorspann}


         
         \newcommand{\erwaehntePersonen}{Personen: }
         \newcommand{\erwaehnteInstitutionen}{}
         \newcommand{\erwaehnteOrte}{}
         \newcommand{\erwaehnteWerke}{
               \section[Hugo von Hofmannsthal an Arthur Schnitzler, {[}18. 5. 1898{]}]{ Hugo von Hofmannsthal an Arthur Schnitzler, {[}18. 5. 1898{]}}\nopagebreak\mylabel{v}\rehead{ }\begin{ledgroupsized}[t]{13cm}\normalsize\beginnumbering \toendnotes[C]{\smallbreak\pagebreak[2]} \Standort{CUL, Schnitzler, B 43.}
\physDesc{Brief, 1 Blatt, 2 Seiten
\newline{}Handschrift: Bleistift, deutsche Kurrent
\newline{}Schnitzler: mit Bleistift datiert: »Mai 98« \newline{}Ordnung: 1) mit Bleistift von unbekannter Hand nummeriert:
                                 »114«  2) mit Bleistift von unbekannter Hand nummeriert:
                                    »117«}\buchAbdrucke{\weitereDrucke{Hugo von Hofmannsthal, Arthur Schnitzler: \emph{Briefwechsel}. Hg. Therese Nickl und Heinrich Schnitzler. Frankfurt am Main: \emph{S. Fischer} 1964, S. 101–102.} }\toendnotes[C]{\smallbreak}\pstart{}{\pb}lieber Arthur!\pend\pstart
           ich hätt Sie ſo gern geſehen.\pend
           \pstart
           Ich hab ſchrecklich wenig Zeit wegen der Prüfung. \label{K_L00796_1v}\edtext{Morgen}{\lemma{\textnormal{\emph{Morgen}}}\Cendnote{\textnormal{Dieser
                  Hinweis lässt den Brief am Mittwoch nach der Premiere von \emph{Madonna Dianora}\textcolor{red}{\textsuperscript{XXXX indx}} zeitlich einordnen.}}}\label{K_L00796_1h}{ }Do{\geminationn}erstag abend werd ich beſtimmt um
                  ¾ 11 im Arkadencafé\oindex{XXXX Ortsangabe fehlt|pw}{ }ſein, ich hoff Sie ſind dort. Über die \label{K_L00796_2v}\edtext{Première\textcolor{red}{\textsuperscript{XXXX indx}}}{\lemma{\textnormal{\emph{Première}}}\Cendnote{\textnormal{Als \emph{Madonna
                     Dianora}\textcolor{red}{\textsuperscript{XXXX indx}} hatte Hofmannsthal\pwindex{\textcolor{red}{\textsuperscript{XXXX1 indx}}|pwk}s \emph{Die Frau im Fenster}\textcolor{red}{\textsuperscript{XXXX indx}} am 15. 5. 1898
                  als öffentliche Matinée der Berlin\oindex{XXXX Ortsangabe fehlt|pwk}er \emph{Freien Bühne}XXXX ORGangabe fehlt am Deutschen Theater\oindex{XXXX Ortsangabe fehlt|pwk} die Uraufführung erlebt.}}}\label{K_L00796_2h} iſt natürlich nur
               mündlich zu reden.\pend
           \pstart
           Es iſt mir ein biſſel zuwider, daſs die W\textsuperscript{r}\oindex{XXXX Ortsangabe fehlt|pw} Zeitungen gar keine Telegra{\geminationm}e haben. Schiff\pwindex{\textcolor{red}{\textsuperscript{XXXX1 indx}}|pw} wird zudem nicht {\pb}ſehr freundlich ſein.\pend
           \pstart
           Könnte nicht Salten\pwindex{\textcolor{red}{\textsuperscript{XXXX1 indx}}|pw} etwas bringen, etwa einen
                  \label{K_L00796_3v}\edtext{Auszug}{\lemma{\textnormal{\emph{Auszug}}}\Cendnote{\textnormal{Im \emph{Berliner Börsen-Courier}XXXX ORGangabe fehlt
                  erschien keine Besprechung, sehr wohl aber im \emph{Berliner Tageblatt}\textcolor{red}{\textsuperscript{XXXX indx}}: F. E.\pwindex{\textcolor{red}{\textsuperscript{XXXX1 indx}}|pwk} (=Fritz
                        Engel\pwindex{\textcolor{red}{\textsuperscript{XXXX1 indx}}|pwk}): \emph{»Freie Bühne«}\textcolor{red}{\textsuperscript{XXXX indx}}. In: \emph{Berliner Tageblatt}\textcolor{red}{\textsuperscript{XXXX indx}}, Jg. 27, Nr. 245,
                     Montags-Ausgabe, 16. 5. 1898, S. 2.}}}\label{K_L00796_3h} aus dem \textsc{Börsencourier}XXXX ORGangabe fehlt oder ſonſt woher, ich würde ihm die Ausſchnitte natürlich auch ſchicken.
               Vielleicht fragen Sie ihn telephoniſch oder ſonſt.\pend
           \pstart
           Herzlich Ihr{\\[\baselineskip]}\spacefill\mbox{Hugo}\pend
           \leftskip=0em{}
         
         \endnumbering\mylabel{h}\end{ledgroupsized}  \newcommand{\dateiname}{L00796}\newcommand{\titel}{Hugo von Hofmannsthal an Arthur Schnitzler, [18. 5. 1898]}\newcommand{\editorInnen}{Martin Anton Müller und Gerd-Hermann Susen}%% latex-leseansicht-abspann.tex
%% Abspann für die Leseansicht.
%% Der Schalter \ifkorrekturansicht ist bereits durch den Vorspann gesetzt.

%% latex-abspann.tex
%% Gemeinsamer Abspann für Korrekturansicht und Leseansicht.
%% Setzt den Schalter \ifkorrekturansicht voraus (gesetzt in den
%% einbindenden Dateien latex-korrekturansicht-abspann.tex bzw.
%% latex-leseansicht-abspann.tex).
%% ---------------------------------------------------------------

\normalsize

% Das esempio-Environment wird nur in der Leseansicht benötigt
\ifkorrekturansicht\else
\newenvironment{esempio}[3]%
{
    \vspace{1.5ex}
    \rlap{\underline{#1}}
    \par
    \setlength{\parindent}{0cm}
    \nopagebreak
    \leftskip=#2cm
    \rightskip=#3cm
}
{
    \par
}
\fi

\doendnotes{C}
\bigskip
\vfill

\clearpage

\footnotesize

\ifkorrekturansicht
  \lohead{\textsc{register}}
\fi

% theindex-Environment neu definieren ohne reledmac
\makeatletter
\renewenvironment{theindex}{%
  \ifkorrekturansicht
    \section*{\indexname}%
  \else
    \subsubsection*{Index der erwähnten Entitäten}%
  \fi
  \setlength{\parindent}{0pt}%
  \setlength{\parskip}{0pt plus 0.3pt}%
  \let\item\@idxitem
}{%
  \ifkorrekturansicht\clearpage\fi
}
\makeatother

\IfFileExists{\jobname-pw.ind}{\input{\jobname-pw.ind}}{}

% Quellenangabe nur in der Leseansicht
\ifkorrekturansicht\else
% Fallback-Definitionen, falls die .tex-Datei \titel etc. nicht gesetzt hat
\providecommand{\titel}{}
\providecommand{\editorInnen}{}
\providecommand{\dateiname}{\jobname}

\vspace{3cm}

\vfill

\footnotesize
\textsc{Quelle}: \titel. Herausgegeben von {\editorInnen}. In: \emph{Arthur Schnitzler: Briefwechsel mit Autorinnen und Autoren}.
 Digitale Edition, https://schnitzler-briefe.acdh.oeaw.ac.at/{\dateiname}.html (Stand \today)
\fi

\end{document}


      