%% latex-leseansicht-vorspann.tex
%% Vorspann für die Leseansicht.
%% Lädt die gemeinsame Datei latex-vorspann.tex mit nicht gesetztem Schalter.

\newif\ifkorrekturansicht
\korrekturansichtfalse

\input{../tex-inputs/latex-vorspann}


\section[Hugo von Hofmannsthal an Arthur Schnitzler, {[}18. 5. 1898{]}]{L00796 Hugo von Hofmannsthal an Arthur Schnitzler, {[}18. 5. 1898{]}}
\nopagebreak\mylabel{L00796v}
\rehead{ }\normalsize\beginnumbering\briefempfaengerindex{Schnitzler, Arthur@\textsc{Schnitzler, Arthur}!zzzHofmannsthal, Hugo von@\emph{von Hugo von Hofmannsthal}!1898-05-181@{{[}18. 5. 1898{]}}|(be}
\toendnotes[C]{\smallbreak\pagebreak[2]}
\correspDesc{Versand  durch Hugo von Hofmannsthal am [18. 5. 1898] in Wien
\newline{}Erhalt  durch Arthur Schnitzler im Zeitraum [18. 5. 1898
                  – 22. 5. 1898?] in Wien}\toendnotes[C]{\smallbreak}
\Standort{CUL, Schnitzler, B 43.}
\physDesc{Brief, 1 Blatt, 2 Seiten, 562 Zeichen
\newline{}Handschrift: Bleistift, deutsche Kurrent
\newline{}Schnitzler: mit Bleistift datiert: »Mai 98« 
\newline{}Ordnung: 1) mit Bleistift von unbekannter Hand nummeriert:
                                    »114«  2) mit Bleistift von unbekannter Hand nummeriert:
                                    »117«}
\buchAbdrucke{\weitereDrucke{Hugo von Hofmannsthal, Arthur Schnitzler: \emph{Briefwechsel}. Herausgegeben von Therese Nickl und Heinrich Schnitzler. Frankfurt am Main: \emph{S. Fischer} 1964, S. 101–102.} }\toendnotes[C]{\smallbreak}
\pstart{}{\pb}lieber Arthur!\pend\vspace{0.5em}
\pstart
           ich hätt Sie{ }ſo gern geſehen.\pend
           
\pstart
           Ich hab{ }ſchrecklich wenig Zeit wegen der Prüfung. \label{K_L00796-1v}\edtext{Morgen}{\lemma{\textnormal{\emph{Morgen}}}\Cendnote{\textnormal{Dieser
                  Hinweis lässt den Brief am Mittwoch nach der Premiere von \emph{Madonna Dianora}\pwindex{Hofmannsthal, Hugo von 1.\,2.\,1874 Wien – 15.\,7.\,1929 Rodaun@\textsc{Hofmannsthal, Hugo von} (1.\,2.\,1874 Wien – 15.\,7.\,1929 Rodaun), \emph{Schriftsteller}!Frau im Fenster@\strich\emph{Die Frau im Fenster}|pwk} zeitlich einordnen.}}}\label{K_L00796-1}{ }Do{\geminationn}erstag abend werd ich beſtimmt um
                  ¾ 11 im Arkadencafé\oindex{Wien@\textbf{Wien}!I., Innere Stadt@\textbf{I., Innere Stadt}!Café Arkaden@\textbf{Café Arkaden}, \emph{Kaffeehaus}|pw}{ }ſein, ich hoff Sie{ }ſind dort. Über die \label{K_L00796-2v}\edtext{Première\pwindex{Hofmannsthal, Hugo von 1.\,2.\,1874 Wien – 15.\,7.\,1929 Rodaun@\textsc{Hofmannsthal, Hugo von} (1.\,2.\,1874 Wien – 15.\,7.\,1929 Rodaun), \emph{Schriftsteller}!Frau im Fenster@\strich\emph{Die Frau im Fenster}|pwv}}{\lemma{\textnormal{\emph{Première}}}\Cendnote{\textnormal{Als \emph{Madonna Dianora}\pwindex{Hofmannsthal, Hugo von 1.\,2.\,1874 Wien – 15.\,7.\,1929 Rodaun@\textsc{Hofmannsthal, Hugo von} (1.\,2.\,1874 Wien – 15.\,7.\,1929 Rodaun), \emph{Schriftsteller}!Frau im Fenster@\strich\emph{Die Frau im Fenster}|pwk} hatte Hofmannsthals\pwindex{Hofmannsthal, Hugo von 1.\,2.\,1874 Wien – 15.\,7.\,1929 Rodaun@\textsc{Hofmannsthal, Hugo von} (1.\,2.\,1874 Wien – 15.\,7.\,1929 Rodaun), \emph{Schriftsteller}|pwk}{ }\emph{Die Frau im Fenster}\pwindex{Hofmannsthal, Hugo von 1.\,2.\,1874 Wien – 15.\,7.\,1929 Rodaun@\textsc{Hofmannsthal, Hugo von} (1.\,2.\,1874 Wien – 15.\,7.\,1929 Rodaun), \emph{Schriftsteller}!Frau im Fenster@\strich\emph{Die Frau im Fenster}|pwk} am
                     15. 5. 1898 als öffentliche Matinée der Berlin\oindex{Berlin@\textbf{Berlin}, \emph{Hauptstadt}|pwk}er \emph{Freien Bühne}\orgindex{Freie Bühne@Freie Bühne|pwk} im
                   Deutschen Theater\oindex{Deutsches Theater Berlin@\textbf{Deutsches Theater Berlin}, \emph{Theater}|pwk} die Uraufführung\eventindex{Deutsches Theater Berlin@\textbf{Deutsches Theater Berlin}!Uraufführung von Madonna Dianora, 15.5.1898@Uraufführung von Madonna Dianora, 15.5.1898|pwkv}
                  erlebt.}}}\label{K_L00796-2} iſt natürlich nur mündlich zu reden.\pend
           
\pstart
           Es iſt mir ein biſſel zuwider, daſs die W\textsuperscript{r}\oindex{Wien@\textbf{Wien}, \emph{Verwaltungsgebiet}|pw} Zeitungen gar keine Telegra{\geminationm}e haben. Schiff\pwindex{Schiff, Emil 30.\,5.\,1849 Roudnice nad Labem – 23.\,1.\,1899 Berlin@\textsc{Schiff, Emil} (30.\,5.\,1849 Roudnice nad Labem – 23.\,1.\,1899 Berlin), \emph{Journalist}|pw} wird zudem nicht {\pb}ſehr freundlich{ }ſein.\pend
           
\pstart
           Könnte nicht Salten\pwindex{Salten, Felix 6.\,9.\,1869 Budapest – 8.\,10.\,1945 Zürich@\textsc{Salten, Felix} (6.\,9.\,1869 Budapest – 8.\,10.\,1945 Zürich), \emph{Schriftsteller, Journalist, Chefredakteur}|pw} etwas bringen, etwa einen
                  \label{K_L00796-3v}\edtext{Auszug}{\lemma{\textnormal{\emph{Auszug}}}\Cendnote{\textnormal{Im \emph{Berliner Börsen-Courier}\orgindex{Berliner Börsen-Courier@Berliner Börsen-Courier|pwk}
                  erschien keine Besprechung, sehr wohl aber im \emph{Berliner Tageblatt}\pwindex{Berliner Tageblatt@\emph{Berliner Tageblatt}|pwk}: F. E.\pwindex{Engel, Fritz 16.\,2.\,1867 Breslau – 3.\,2.\,1935 Berlin@\textsc{Engel, Fritz} (16.\,2.\,1867 Breslau – 3.\,2.\,1935 Berlin), \emph{Journalist}|pwk} [ =Fritz Engel\pwindex{Engel, Fritz 16.\,2.\,1867 Breslau – 3.\,2.\,1935 Berlin@\textsc{Engel, Fritz} (16.\,2.\,1867 Breslau – 3.\,2.\,1935 Berlin), \emph{Journalist}|pwk}]: \emph{»Freie Bühne«}\pwindex{?? [Rezension von Madonna Dianora]@\emph{?? [Rezension von Madonna Dianora]}|pwk}. In:
                        \emph{Berliner Tageblatt}\pwindex{Berliner Tageblatt@\emph{Berliner Tageblatt}|pwk}, Jg. 27, Nr. 245,
                     Montags-Ausgabe, 16. 5. 1898, S. 2.}}}\label{K_L00796-3} aus dem \textsc{Börsencourier}\orgindex{Berliner Börsen-Courier@Berliner Börsen-Courier|pw} oder{ }ſonſt woher, ich würde ihm die Ausſchnitte natürlich auch{ }ſchicken.
               Vielleicht fragen Sie ihn telephoniſch oder{ }ſonſt.\pend
           
\pstart
           Herzlich Ihr{\\[\baselineskip]}\spacefill\mbox{Hugo}\pend
           \leftskip=0em{}\selectlanguage{ngerman}\endnumbering\briefempfaengerindex{Schnitzler, Arthur@\textsc{Schnitzler, Arthur}!zzzHofmannsthal, Hugo von@\emph{von Hugo von Hofmannsthal}!1898-05-181@{{[}18. 5. 1898{]}}|)be}\mylabel{L00796h}  \newcommand{\dateiname}{L00796}\newcommand{\titel}{Hugo von Hofmannsthal an Arthur Schnitzler, [18. 5. 1898]}\newcommand{\editorInnen}{Martin Anton Müller und Gerd-Hermann Susen}%% latex-leseansicht-abspann.tex
%% Abspann für die Leseansicht.
%% Der Schalter \ifkorrekturansicht ist bereits durch den Vorspann gesetzt.

%% latex-abspann.tex
%% Gemeinsamer Abspann für Korrekturansicht und Leseansicht.
%% Setzt den Schalter \ifkorrekturansicht voraus (gesetzt in den
%% einbindenden Dateien latex-korrekturansicht-abspann.tex bzw.
%% latex-leseansicht-abspann.tex).
%% ---------------------------------------------------------------

\normalsize

% Das esempio-Environment wird nur in der Leseansicht benötigt
\ifkorrekturansicht\else
\newenvironment{esempio}[3]%
{
    \vspace{1.5ex}
    \rlap{\underline{#1}}
    \par
    \setlength{\parindent}{0cm}
    \nopagebreak
    \leftskip=#2cm
    \rightskip=#3cm
}
{
    \par
}
\fi

\doendnotes{C}
\bigskip
\vfill

\clearpage

\footnotesize

\ifkorrekturansicht
  \lohead{\textsc{register}}
\fi

% theindex-Environment neu definieren ohne reledmac
\makeatletter
\renewenvironment{theindex}{%
  \ifkorrekturansicht
    \section*{\indexname}%
  \else
    \subsubsection*{Index der erwähnten Entitäten}%
  \fi
  \setlength{\parindent}{0pt}%
  \setlength{\parskip}{0pt plus 0.3pt}%
  \let\item\@idxitem
}{%
  \ifkorrekturansicht\clearpage\fi
}
\makeatother

\IfFileExists{\jobname-pw.ind}{\input{\jobname-pw.ind}}{}

% Quellenangabe nur in der Leseansicht
\ifkorrekturansicht\else
% Fallback-Definitionen, falls die .tex-Datei \titel etc. nicht gesetzt hat
\providecommand{\titel}{}
\providecommand{\editorInnen}{}
\providecommand{\dateiname}{\jobname}

\vspace{3cm}

\vfill

\footnotesize
\textsc{Quelle}: \titel. Herausgegeben von {\editorInnen}. In: \emph{Arthur Schnitzler: Briefwechsel mit Autorinnen und Autoren}.
 Digitale Edition, https://schnitzler-briefe.acdh.oeaw.ac.at/{\dateiname}.html (Stand \today)
\fi

\end{document}


