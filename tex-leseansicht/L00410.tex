\input{../tex-inputs/latex-pdf-vorspann}
\begin{center}
            \textcolor{red}{ENTWURF. ENTZIFFERUNG NOCH NICHT KORREKTURGELESEN}
                      \end{center}
            
               \section[Ferdinand von Saar an Arthur Schnitzler, 13. 12. 1894]{ Ferdinand von Saar an Arthur Schnitzler,
                    13. 12. 1894}\nopagebreak\mylabel{v}\rehead{ }\begin{ledgroupsized}[t]{13cm}\normalsize\beginnumbering\briefempfaengerindex{Schnitzler, Arthur@\textsc{Schnitzler, Arthur}!zzzSaar, Ferdinand von@\emph{von Ferdinand von Saar}!1894-12-131@{13. 12. 1894}|(be} \toendnotes[C]{\smallbreak\pagebreak[2]} \Standort{CUL, Schnitzler, B 88.}
\physDesc{Brief, 1 Blatt, 2 Seiten
\newline{}Handschrift: blaue Tinte, deutsche Kurrent
\newline{}Schnitzler: mit Bleistift nummeriert: »3« }\toendnotes[C]{\smallbreak}\pstart
           \raggedleft{}{\pb}\textsc{Raitz} in Mähren\oindex{Rájec-Jestřebí@\textbf{Rájec-Jestřebí}|pw}, 13\textsuperscript{t} Decbr. 1894.\pend
           \pstart{}Sehr geehrter Herr Doctor!\pend\pstart
           Haben Sie Dank für die freundlich auszeichnende Überſendung Ihrer neueſten Novelle\pwindex{Schnitzler, Arthur 15.05.1862 – 21.10.1931@\textsc{Schnitzler, Arthur} (15.05.1862 – 21.10.1931), \emph{Schriftsteller, Mediziner}!Sterben. Novelle1.10.1894 – 1.12.1894@\strich\emph{Sterben. Novelle} {[}1.10.1894 – 1.12.1894{]}|pw}, die ich nunmehr an zwei ſtillen Abenden
                    geleſen. Bewunderungswürdig iſt die Kunſt – oder beſſer geſagt die Wahrheit, mit
                    der Sie die Seelenqualen des hinſterbenden Felix\pwindex{Schnitzler, Arthur 15.05.1862 – 21.10.1931@\textsc{Schnitzler, Arthur} (15.05.1862 – 21.10.1931), \emph{Schriftsteller, Mediziner}!Sterben. Novelle1.10.1894 – 1.12.1894@\strich\emph{Sterben. Novelle} {[}1.10.1894 – 1.12.1894{]}|pwv}, den allmäligen Loslöſungsprozeß der Geliebten ſchildern.
                    Aber hätten Sie nicht dieſes pſychologiſche Duett (oder wenn Sie wollen Terzett)
                    vielſtimmiger machen, nicht einige Handlung und Verwicklung dazu erfinden
                    können? Gerade \uline{das} wollte ich nicht! werden Sie
                    ausrufen. Und dann haben Sie auch recht. Es muß, es darf ja nicht ein Werk wie
                    das {\pb}andere ſein, und da Sie ſchon ſo
                    viel Abwechſlungsvolles gebracht haben, ſo wird dieſes peinvolle Machtſtück in
                    ſeiner knapp umrahmten Düſterkeit \introOben{}auch\introOben{} den richtigen
                    Platz in der Reihe Ihrer Schriften finden, allwo es ſeine eigenthümliche Wirkung
                    ganz und voll ausüben kann.\pend
           \pstart
           Ich ſelbſt bin jetzt auch beſchäftigt – und zwar mit allerlei. Wollen ſehen, was
                    dabei herauskommt!\pend
           \pstart
           Es grüßt Sie herzlich und mit aufrichtiger
                        Hochſchätzung{\\[\baselineskip]}Ihr{\\[\baselineskip]}\spacefill\mbox{Ferdinand von Saar}\pend
           \leftskip=0em{}\endnumbering\briefempfaengerindex{Schnitzler, Arthur@\textsc{Schnitzler, Arthur}!zzzSaar, Ferdinand von@\emph{von Ferdinand von Saar}!1894-12-131@{13. 12. 1894}|)be}\mylabel{h}\end{ledgroupsized}  \newcommand{\dateiname}{L00410}\newcommand{\titel}{Ferdinand von Saar an Arthur Schnitzler, 13. 12. 1894}\newcommand{\editorInnen}{Martin Anton Müller und Gerd-Hermann Susen}\input{../tex-inputs/latex-pdf-abspann}
      