%% latex-korrekturansicht-vorspann.tex
%% Vorspann für die Korrekturansicht.
%% Lädt die gemeinsame Datei latex-vorspann.tex mit gesetztem Schalter.

\newif\ifkorrekturansicht
\korrekturansichttrue

\input{../tex-inputs/latex-vorspann}


\section[Peter Altenberg an Arthur Schnitzler, {[}zwischen April und Oktober 1912{]}]{L02061 Peter Altenberg an Arthur Schnitzler, {[}zwischen April und Oktober
               1912{]}}
\nopagebreak\mylabel{L02061v}
\rehead{ }\normalsize\beginnumbering\briefempfaengerindex{Schnitzler, Arthur@\textsc{Schnitzler, Arthur}!zzzAltenberg, Peter@\emph{von Peter Altenberg}!1912-10-191@{{[}zwischen April und Oktober
                  1912{]}}|(be}
\toendnotes[C]{\smallbreak\pagebreak[2]}\Standort{DLA, A:Schnitzler, HS.NZ85.1.2342, S. 9.}
\physDesc{Karte, maschinenschriftliche Abschrift1 Blatt, 1 Seite, 127 Zeichen
\newline{}Schreibmaschine}\toendnotes[C]{\smallbreak}
\pstart
           \raggedleft{}{\pb}Semmering\oindex{Semmering@\textbf{Semmering}, \emph{A.ADM3}|pw}.\pend
           \vspace{0.5em}
\pstart
           Lieber Dr. Arthur Schnitzler, bitte, schenken Sie mir für meine
               12-jährige Heilige \label{K_L02061-1v}\edtext{Klara Panhans\pwindex{Panhans, Klara 09.04.1900 – 08.04.1921@\textsc{Panhans, Klara} (09.04.1900 – 08.04.1921), \emph{Hotelierssohn/Hotelierstochter}|pw}}{\lemma{\textnormal{\emph{Klara Panhans}}}\Cendnote{\textnormal{Am 9. 4. 1912 wurde die
                  Hotelierstochter 12 Jahre alt, am 17. 10. 1912 fand der Bruch statt,
                  sodass dieses Korrespondenzstück in der Zwischenzeit abgefasst worden sein muss.}}}\label{K_L02061-1}
               ein Autogramm! Ihr\pend
           \pstart \spacefill\mbox{P. A.}\pend{}\selectlanguage{ngerman}\endnumbering\briefempfaengerindex{Schnitzler, Arthur@\textsc{Schnitzler, Arthur}!zzzAltenberg, Peter@\emph{von Peter Altenberg}!1912-04-091@{{[}zwischen April und Oktober
                  1912{]}}|)be}\mylabel{L02061h}  \normalsize

\doendnotes{C}
\bigskip
\vfill

\clearpage

\footnotesize

\lohead{\textsc{register}}

% Definiere theindex-Environment komplett neu ohne reledmac
\makeatletter
\renewenvironment{theindex}{%
  \section*{\indexname}%
  \setlength{\parindent}{0pt}%
  \setlength{\parskip}{0pt plus 0.3pt}%
  \let\item\@idxitem
}{%
  \clearpage
}
\makeatother

\IfFileExists{\jobname-pw.ind}{\input{\jobname-pw.ind}}{}

\end{document}

      