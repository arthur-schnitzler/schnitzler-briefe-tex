%% latex-leseansicht-vorspann.tex
%% Vorspann für die Leseansicht.
%% Lädt die gemeinsame Datei latex-vorspann.tex mit nicht gesetztem Schalter.

\newif\ifkorrekturansicht
\korrekturansichtfalse

\input{../tex-inputs/latex-vorspann}

\begin{center}
            \textcolor{red}{ENTWURF. ENTZIFFERUNG NOCH NICHT KORREKTURGELESEN}
                      \end{center}
            
               \section[Arthur Schnitzler an Hugo Hofmannsthal, 15. 1. 1923]{ Arthur Schnitzler an Hugo Hofmannsthal, 15. 1. 1923}\nopagebreak\mylabel{v}\rehead{ }\begin{ledgroupsized}[t]{13cm}\normalsize\beginnumbering\briefempfaengerindex{Hofmannsthal, Hugo von@\textsc{Hofmannsthal, Hugo von}!zzzSchnitzler, Arthur@\emph{von Arthur Schnitzler}!1923-01-151@{15. 1. 1923}|(be} \toendnotes[C]{\smallbreak\pagebreak[2]} \buchAlsQuelle{Hugo von Hofmannsthal, Arthur Schnitzler: \emph{Briefwechsel}. Hg. Therese Nickl und Heinrich Schnitzler. Frankfurt am Main: \emph{S. Fischer} 1964, S. 296–297.}\buchAbdrucke{\weitereDrucke{Arthur Schnitzler: \emph{Briefe 1913–1931}. Hg. Peter Michael Braunwarth, Richard Miklin, Susanne Pertlik und Heinrich Schnitzler. Frankfurt am Main: \emph{S. Fischer} 1984, S. 301–302.} }\toendnotes[C]{\smallbreak}\pstart
           \noindent{}{\pb}{[}\label{K_L02395_1v}\edtext{Maschinenschrift}{\lemma{\textnormal{\emph{Maschinenschrift}}}\Cendnote{\textnormal{Das originale Typoskript ist nicht
                        auffindbar.}}}\label{K_L02395_1h}{]}\pend
           \pstart
           \raggedleft{}15. 1. 1923\pend
           \pstart{}Mein lieber Hugo. \pend\pstart
           Sie wissen vielleicht, daß die »Beatrice\pwindex{Schnitzler, Arthur 15.05.1862 – 21.10.1931@\textsc{Schnitzler, Arthur} (15.05.1862 – 21.10.1931), \emph{Schriftsteller, Mediziner}!Schleier der Beatrice. Schauspiel in fuenf Akten1900-12-01 – 1900-12-01@\strich\emph{Der Schleier der Beatrice. Schauspiel in fünf Akten} {[}1900-12-01 – 1900-12-01{]}|pw}« von Heinrich Noren\pwindex{Noren, Heinrich 05.01.1861 – 06.06.1928@\textsc{Noren, Heinrich} (05.01.1861 – 06.06.1928), \emph{Komponist, Violinist}|pw} komponiert worden ist. Auf mein
               Ersuchen die Partitur anzusehen, resp. sich Teile aus der Oper\pwindex{Schnitzler, Arthur 15.05.1862 – 21.10.1931@\textsc{Schnitzler, Arthur} (15.05.1862 – 21.10.1931), \emph{Schriftsteller, Mediziner}!Schleier der Beatrice. Schauspiel in fuenf Akten1900-12-01 – 1900-12-01@\strich\emph{Der Schleier der Beatrice. Schauspiel in fünf Akten} {[}1900-12-01 – 1900-12-01{]}|pwv} von Noren\pwindex{Noren, Heinrich 05.01.1861 – 06.06.1928@\textsc{Noren, Heinrich} (05.01.1861 – 06.06.1928), \emph{Komponist, Violinist}|pw}{ }selbst (der einen höchst geachteten musikalischen
               Namen besitzt) vorspielen zu lassen, erwiderte mir Richard Strauss\pwindex{Strauss, Richard 11.06.1864 – 08.09.1949@\textsc{Strauss, Richard} (11.06.1864 – 08.09.1949), \emph{Theaterleiter, Komponist, Dirigent}|pw}, daß die Oper\orgindex{Staatsoper@Staatsoper|pw} überhaupt
               nicht daran denken könne Uraufführungen zu bringen – aus hauptsächlich materiellen,
               aber gewiß plausiblen Gründen. Es gibt vielleicht Fälle, in denen man von diesem
               Prinzip abgehen könnte, es scheint ja auch, daß es manchmal geschieht. Ich selbst
               konnte natürlich in meinem Falle nicht insistieren, obwohl gerade er am ehesten Anlaß
               gäbe von jenem Prinzip wenigstens insoweit abzuweichen, als die Direktion der Oper\orgindex{Staatsoper@Staatsoper|pw} immerhin den Versuch riskieren könnte, das Werk
               kennen zu lernen. Warum ich das Ihnen erzähle, lieber Hugo? Weil mir neulich Noren\pwindex{Noren, Heinrich 05.01.1861 – 06.06.1928@\textsc{Noren, Heinrich} (05.01.1861 – 06.06.1928), \emph{Komponist, Violinist}|pw}{ }schreibt, und weil Bruno Walter\pwindex{Walter, Bruno 15.09.1876 – 17.02.1962@\textsc{Walter, Bruno} (15.09.1876 – 17.02.1962), \emph{Theaterleiter, Komponist, Dirigent}|pw} gleichfalls behauptet, daß Sie der einzige Mensch wären, der
               auf Strauss\pwindex{Strauss, Richard 11.06.1864 – 08.09.1949@\textsc{Strauss, Richard} (11.06.1864 – 08.09.1949), \emph{Theaterleiter, Komponist, Dirigent}|pw} oder Schalk\pwindex{Schalk, Franz 27.05.1863 – 03.09.1931@\textsc{Schalk, Franz} (27.05.1863 – 03.09.1931), \emph{Theaterleiter, Dirigent}|pw} oder auf sie Beide in dem Sinne einwirken könnte, daß diese zum
               mindesten von der Existenz des in Frage stehenden Werkes Notiz nähmen, der vielleicht
               sogar (dies sind Bruno Walter\pwindex{Walter, Bruno 15.09.1876 – 17.02.1962@\textsc{Walter, Bruno} (15.09.1876 – 17.02.1962), \emph{Theaterleiter, Komponist, Dirigent}|pw}s Worte) auf die
               Absurdität hinweisen dürfte, die nicht nur dem Komponisten darin zu liegen scheint,
               daß die Wiener Oper\orgindex{Staatsoper@Staatsoper|pw} ein sozusagen von zwei Österreichern\oindex{Oesterreich@\textbf{Österreich}|pw} verfaßtes Werk, und von nicht ganz
               unbekannten überdies, nicht nur nicht zu eventueller Uraufführung in Erwägung ziehen,
               sondern vorläufig sogar eine Prüfung lieber vermeiden möchte. Auch ich fühle etwas
               von der Absurdität, die in Strauss\pwindex{Strauss, Richard 11.06.1864 – 08.09.1949@\textsc{Strauss, Richard} (11.06.1864 – 08.09.1949), \emph{Theaterleiter, Komponist, Dirigent}|pw}ens Vorgehen
               steckt (mit Schalk\pwindex{Schalk, Franz 27.05.1863 – 03.09.1931@\textsc{Schalk, Franz} (27.05.1863 – 03.09.1931), \emph{Theaterleiter, Dirigent}|pw} habe ich nicht gesprochen, er
               weiß vielleicht von der Existenz der Oper\pwindex{Schnitzler, Arthur 15.05.1862 – 21.10.1931@\textsc{Schnitzler, Arthur} (15.05.1862 – 21.10.1931), \emph{Schriftsteller, Mediziner}!Schleier der Beatrice. Schauspiel in fuenf Akten1900-12-01 – 1900-12-01@\strich\emph{Der Schleier der Beatrice. Schauspiel in fünf Akten} {[}1900-12-01 – 1900-12-01{]}|pwv} bis heute gar nichts); trotzdem hätte ich Sie in der Sache nicht
               bemüht, wenn ich es nicht allzu schwer fände Heinrich
                  Noren\pwindex{Noren, Heinrich 05.01.1861 – 06.06.1928@\textsc{Noren, Heinrich} (05.01.1861 – 06.06.1928), \emph{Komponist, Violinist}|pw} die Erfüllung eines Wunsches zu verweigern, die ihm die Erfüllung
               seines wesent{\pb}lichern – die Aufführung seiner
                  Oper\pwindex{Schnitzler, Arthur 15.05.1862 – 21.10.1931@\textsc{Schnitzler, Arthur} (15.05.1862 – 21.10.1931), \emph{Schriftsteller, Mediziner}!Schleier der Beatrice. Schauspiel in fuenf Akten1900-12-01 – 1900-12-01@\strich\emph{Der Schleier der Beatrice. Schauspiel in fünf Akten} {[}1900-12-01 – 1900-12-01{]}|pwv} in Wien\oindex{Wien@\textbf{Wien}|pw} – in die Nähe zu rücken scheint. Ich weiß weder, ob Sie,
               lieber Hugo, Gelegenheit, noch ob Sie Lust haben sich mit dieser Sache in irgend
               einer Form zu befassen. Vielleicht sprechen wir bald einmal darüber, wenn Sie wieder
               nach Wien\oindex{Wien@\textbf{Wien}|pw} hereinkommen. Es wäre ja überhaupt schon
               Zeit, daß man sich wieder einmal sieht und spricht. Ich habe Ihnen noch nicht einmal
               zum \label{K_L02395_2v}\edtext{Erfolg des »Großen Welttheaters\pwindex{Hofmannsthal, Hugo von 01.02.1874 – 15.07.1929@\textsc{Hofmannsthal, Hugo von} (01.02.1874 – 15.07.1929), \emph{Schriftsteller}!Salzburger grosse Welttheater1922@\strich\emph{Das Salzburger große Welttheater} {[}1922{]}|pw}«}{\lemma{\textnormal{\emph{Erfolg … Welttheaters«}}}\Cendnote{\textnormal{Die Uraufführung fand am 12. 8. 1922 in der Kollegienkirche in Salzburg\oindex{Kollegienkirche (Salzburg)@\textbf{Kollegienkirche (Salzburg)}|pwk} statt. Regie führte Max Reinhardt\pwindex{Reinhardt, Max 09.09.1873 – 30.10.1943@\textsc{Reinhardt, Max} (09.09.1873 – 30.10.1943), \emph{Theaterleiter, Regisseur, Schauspieler}|pwk}.}}}\label{K_L02395_2h} gratuliert und nicht
               gesagt, wie schön Ihre beiden Artikel\pwindex{Hofmannsthal, Hugo von 01.02.1874 – 15.07.1929@\textsc{Hofmannsthal, Hugo von} (01.02.1874 – 15.07.1929), \emph{Schriftsteller}!Vienna Letter1.8.1922 – 1.8.1922@\strich\emph{Vienna Letter} {[}1.8.1922 – 1.8.1922{]}|pwv}\pwindex{Hofmannsthal, Hugo von 01.02.1874 – 15.07.1929@\textsc{Hofmannsthal, Hugo von} (01.02.1874 – 15.07.1929), \emph{Schriftsteller}!Vienna Letter1.10.1922 – 1.10.1922@\strich\emph{Vienna Letter} {[}1.10.1922 – 1.10.1922{]}|pwv} im »Dial\pwindex{Dial1840 – 1929@\emph{The Dial}|pw}« (nicht nur der über mich\pwindex{Hofmannsthal, Hugo von 01.02.1874 – 15.07.1929@\textsc{Hofmannsthal, Hugo von} (01.02.1874 – 15.07.1929), \emph{Schriftsteller}!Vienna Letter1.8.1922 – 1.8.1922@\strich\emph{Vienna Letter} {[}1.8.1922 – 1.8.1922{]}|pwv}) waren.\pend
           \pstart Seien Sie herzlichst gegrüßt \spacefill\mbox{\textcolor{gray}{A. S.}}\pend{}\endnumbering\briefempfaengerindex{Hofmannsthal, Hugo von@\textsc{Hofmannsthal, Hugo von}!zzzSchnitzler, Arthur@\emph{von Arthur Schnitzler}!1923-01-151@{15. 1. 1923}|)be}\mylabel{h}\end{ledgroupsized}  \newcommand{\dateiname}{L02395}\newcommand{\titel}{Arthur Schnitzler an Hugo Hofmannsthal, 15. 1. 1923}\newcommand{\editorInnen}{Martin Anton Müller und Gerd-Hermann Susen}%% latex-leseansicht-abspann.tex
%% Abspann für die Leseansicht.
%% Der Schalter \ifkorrekturansicht ist bereits durch den Vorspann gesetzt.

%% latex-abspann.tex
%% Gemeinsamer Abspann für Korrekturansicht und Leseansicht.
%% Setzt den Schalter \ifkorrekturansicht voraus (gesetzt in den
%% einbindenden Dateien latex-korrekturansicht-abspann.tex bzw.
%% latex-leseansicht-abspann.tex).
%% ---------------------------------------------------------------

\normalsize

% Das esempio-Environment wird nur in der Leseansicht benötigt
\ifkorrekturansicht\else
\newenvironment{esempio}[3]%
{
    \vspace{1.5ex}
    \rlap{\underline{#1}}
    \par
    \setlength{\parindent}{0cm}
    \nopagebreak
    \leftskip=#2cm
    \rightskip=#3cm
}
{
    \par
}
\fi

\doendnotes{C}
\bigskip
\vfill

\clearpage

\footnotesize

\ifkorrekturansicht
  \lohead{\textsc{register}}
\fi

% theindex-Environment neu definieren ohne reledmac
\makeatletter
\renewenvironment{theindex}{%
  \ifkorrekturansicht
    \section*{\indexname}%
  \else
    \subsubsection*{Index der erwähnten Entitäten}%
  \fi
  \setlength{\parindent}{0pt}%
  \setlength{\parskip}{0pt plus 0.3pt}%
  \let\item\@idxitem
}{%
  \ifkorrekturansicht\clearpage\fi
}
\makeatother

\IfFileExists{\jobname-pw.ind}{\input{\jobname-pw.ind}}{}

% Quellenangabe nur in der Leseansicht
\ifkorrekturansicht\else
% Fallback-Definitionen, falls die .tex-Datei \titel etc. nicht gesetzt hat
\providecommand{\titel}{}
\providecommand{\editorInnen}{}
\providecommand{\dateiname}{\jobname}

\vspace{3cm}

\vfill

\footnotesize
\textsc{Quelle}: \titel. Herausgegeben von {\editorInnen}. In: \emph{Arthur Schnitzler: Briefwechsel mit Autorinnen und Autoren}.
 Digitale Edition, https://schnitzler-briefe.acdh.oeaw.ac.at/{\dateiname}.html (Stand \today)
\fi

\end{document}


      