%% latex-leseansicht-vorspann.tex
%% Vorspann für die Leseansicht.
%% Lädt die gemeinsame Datei latex-vorspann.tex mit nicht gesetztem Schalter.

\newif\ifkorrekturansicht
\korrekturansichtfalse

\input{../tex-inputs/latex-vorspann}


\section[Arthur Schnitzler an Hugo Hofmannsthal, 15. 1. 1923]{L02395 Arthur Schnitzler an Hugo Hofmannsthal, 15. 1. 1923}
\nopagebreak\mylabel{L02395v}
\rehead{ }\normalsize\beginnumbering\briefempfaengerindex{Hofmannsthal, Hugo von@\textsc{Hofmannsthal, Hugo von}!zzzSchnitzler, Arthur@\emph{von Arthur Schnitzler}!1923-01-151@{15. 1. 1923}|(be}
\toendnotes[C]{\smallbreak\pagebreak[2]}
\correspDesc{Versand  durch Arthur Schnitzler am 15. 1. 1923 in Wien
\newline{}Erhalt  durch Hugo von Hofmannsthal im Zeitraum [16. 1. 1923
                  – 20. 1. 1923?] in Rodaun}\toendnotes[C]{\smallbreak}
\buchAlsQuelle{Hugo von Hofmannsthal, Arthur Schnitzler: \emph{Briefwechsel}. Herausgegeben von Therese Nickl und Heinrich Schnitzler. Frankfurt am Main: \emph{S. Fischer} 1964, S. 296–297.}
\buchAbdrucke{\weitereDrucke{Arthur Schnitzler: \emph{Briefe 1913–1931}. Herausgegeben von Peter Michael Braunwarth, Richard Miklin, Susanne Pertlik und Heinrich Schnitzler. Frankfurt am Main: \emph{S. Fischer} 1984, S. 301–302.} }\toendnotes[C]{\smallbreak}
\pstart
           {\pb}{[}\label{K_L02395-1v}\edtext{Maschinenschrift}{\lemma{\textnormal{\emph{Maschinenschrift}}}\Cendnote{\textnormal{Das originale Typoskript ist nicht
                        auffindbar.}}}\label{K_L02395-1}{]}\pend
           
\pstart
           \raggedleft{}15. 1. 1923\pend
           
\pstart{}Mein lieber Hugo.\pend\vspace{0.5em}
\pstart
           Sie wissen vielleicht, daß die »Beatrice\pwindex{Schnitzler, Arthur 15.\,5.\,1862 Wien – 21.\,10.\,1931 ebd.@\textsc{Schnitzler, Arthur} (15.\,5.\,1862 Wien – 21.\,10.\,1931 ebd.), \emph{Schriftsteller, Mediziner}!Schleier der Beatrice. Schauspiel in fünf Akten@\strich\emph{Der Schleier der Beatrice. Schauspiel in fünf Akten}|pw}\pwindex{Noren, Heinrich G. 5.\,1.\,1861 Graz – 6.\,6.\,1928 Oberhof [Kreuth]@\textsc{Noren, Heinrich G.} (5.\,1.\,1861 Graz – 6.\,6.\,1928 Oberhof [Kreuth]), \emph{Komponist, Violinist, Geiger}!Schleier der Beatrice. Oper@\strich\emph{Der Schleier der Beatrice. Oper}|pw}« von
                  Heinrich Noren\pwindex{Noren, Heinrich G. 5.\,1.\,1861 Graz – 6.\,6.\,1928 Oberhof [Kreuth]@\textsc{Noren, Heinrich G.} (5.\,1.\,1861 Graz – 6.\,6.\,1928 Oberhof [Kreuth]), \emph{Komponist, Violinist, Geiger}|pw} komponiert worden ist. Auf
               mein Ersuchen die Partitur anzusehen, resp. sich Teile aus der Oper\pwindex{Noren, Heinrich G. 5.\,1.\,1861 Graz – 6.\,6.\,1928 Oberhof [Kreuth]@\textsc{Noren, Heinrich G.} (5.\,1.\,1861 Graz – 6.\,6.\,1928 Oberhof [Kreuth]), \emph{Komponist, Violinist, Geiger}!Schleier der Beatrice. Oper@\strich\emph{Der Schleier der Beatrice. Oper}|pwv} von Noren\pwindex{Noren, Heinrich G. 5.\,1.\,1861 Graz – 6.\,6.\,1928 Oberhof [Kreuth]@\textsc{Noren, Heinrich G.} (5.\,1.\,1861 Graz – 6.\,6.\,1928 Oberhof [Kreuth]), \emph{Komponist, Violinist, Geiger}|pw}{ }selbst (der einen höchst geachteten musikalischen
               Namen besitzt) vorspielen zu lassen, erwiderte mir Richard Strauss\pwindex{Strauss, Richard 11.\,6.\,1864 München – 8.\,9.\,1949 Garmisch-Partenkirchen@\textsc{Strauss, Richard} (11.\,6.\,1864 München – 8.\,9.\,1949 Garmisch-Partenkirchen), \emph{Theaterleiter, Komponist, Dirigent}|pw}, daß die Oper\orgindex{Staatsoper@Staatsoper|pw} überhaupt
               nicht daran denken könne Uraufführungen zu bringen – aus hauptsächlich materiellen,
               aber gewiß plausiblen Gründen. Es gibt vielleicht Fälle, in denen man von diesem
               Prinzip abgehen könnte, es scheint ja auch, daß es manchmal geschieht. Ich selbst
               konnte natürlich in meinem Falle nicht insistieren, obwohl gerade er am ehesten Anlaß
               gäbe von jenem Prinzip wenigstens insoweit abzuweichen, als die Direktion der Oper\orgindex{Staatsoper@Staatsoper|pw} immerhin den Versuch riskieren könnte, das
               Werk kennen zu lernen. Warum ich das Ihnen erzähle, lieber Hugo? Weil mir neulich Noren\pwindex{Noren, Heinrich G. 5.\,1.\,1861 Graz – 6.\,6.\,1928 Oberhof [Kreuth]@\textsc{Noren, Heinrich G.} (5.\,1.\,1861 Graz – 6.\,6.\,1928 Oberhof [Kreuth]), \emph{Komponist, Violinist, Geiger}|pw}{ }schreibt, und weil Bruno Walter\pwindex{Walter, Bruno 15.\,9.\,1876 Berlin – 17.\,2.\,1962 Beverly Hills@\textsc{Walter, Bruno} (15.\,9.\,1876 Berlin – 17.\,2.\,1962 Beverly Hills), \emph{Theaterleiter, Komponist, Dirigent}|pw} gleichfalls behauptet, daß Sie der einzige Mensch
               wären, der auf Strauss\pwindex{Strauss, Richard 11.\,6.\,1864 München – 8.\,9.\,1949 Garmisch-Partenkirchen@\textsc{Strauss, Richard} (11.\,6.\,1864 München – 8.\,9.\,1949 Garmisch-Partenkirchen), \emph{Theaterleiter, Komponist, Dirigent}|pw} oder Schalk\pwindex{Schalk, Franz 27.\,5.\,1863 Wien – 3.\,9.\,1931 Edlach@\textsc{Schalk, Franz} (27.\,5.\,1863 Wien – 3.\,9.\,1931 Edlach), \emph{Theaterleiter, Dirigent}|pw} oder auf sie Beide in dem Sinne
               einwirken könnte, daß diese zum mindesten von der Existenz des in Frage stehenden
               Werkes\pwindex{Noren, Heinrich G. 5.\,1.\,1861 Graz – 6.\,6.\,1928 Oberhof [Kreuth]@\textsc{Noren, Heinrich G.} (5.\,1.\,1861 Graz – 6.\,6.\,1928 Oberhof [Kreuth]), \emph{Komponist, Violinist, Geiger}!Schleier der Beatrice. Oper@\strich\emph{Der Schleier der Beatrice. Oper}|pwv} Notiz nähmen, der vielleicht sogar (dies sind Bruno Walters\pwindex{Walter, Bruno 15.\,9.\,1876 Berlin – 17.\,2.\,1962 Beverly Hills@\textsc{Walter, Bruno} (15.\,9.\,1876 Berlin – 17.\,2.\,1962 Beverly Hills), \emph{Theaterleiter, Komponist, Dirigent}|pw} Worte) auf die Absurdität hinweisen dürfte, die
               nicht nur dem Komponisten darin zu liegen scheint, daß die Wiener Oper\orgindex{Staatsoper@Staatsoper|pw} ein sozusagen von zwei Österreichern\oindex{Österreich@\textbf{Österreich}|pw} verfaßtes Werk\pwindex{Noren, Heinrich G. 5.\,1.\,1861 Graz – 6.\,6.\,1928 Oberhof [Kreuth]@\textsc{Noren, Heinrich G.} (5.\,1.\,1861 Graz – 6.\,6.\,1928 Oberhof [Kreuth]), \emph{Komponist, Violinist, Geiger}!Schleier der Beatrice. Oper@\strich\emph{Der Schleier der Beatrice. Oper}|pwv}, und von nicht ganz unbekannten
               überdies, nicht nur nicht zu eventueller Uraufführung in Erwägung ziehen, sondern
               vorläufig sogar eine Prüfung lieber vermeiden möchte. Auch ich fühle etwas von der
               Absurdität, die in Strauss\pwindex{Strauss, Richard 11.\,6.\,1864 München – 8.\,9.\,1949 Garmisch-Partenkirchen@\textsc{Strauss, Richard} (11.\,6.\,1864 München – 8.\,9.\,1949 Garmisch-Partenkirchen), \emph{Theaterleiter, Komponist, Dirigent}|pw}ens Vorgehen steckt
               (mit Schalk\pwindex{Schalk, Franz 27.\,5.\,1863 Wien – 3.\,9.\,1931 Edlach@\textsc{Schalk, Franz} (27.\,5.\,1863 Wien – 3.\,9.\,1931 Edlach), \emph{Theaterleiter, Dirigent}|pw} habe ich nicht gesprochen, er weiß
               vielleicht von der Existenz der Oper\pwindex{Noren, Heinrich G. 5.\,1.\,1861 Graz – 6.\,6.\,1928 Oberhof [Kreuth]@\textsc{Noren, Heinrich G.} (5.\,1.\,1861 Graz – 6.\,6.\,1928 Oberhof [Kreuth]), \emph{Komponist, Violinist, Geiger}!Schleier der Beatrice. Oper@\strich\emph{Der Schleier der Beatrice. Oper}|pwv} bis heute gar nichts); trotzdem hätte ich Sie in der Sache nicht
               bemüht, wenn ich es nicht allzu schwer fände Heinrich Noren\pwindex{Noren, Heinrich G. 5.\,1.\,1861 Graz – 6.\,6.\,1928 Oberhof [Kreuth]@\textsc{Noren, Heinrich G.} (5.\,1.\,1861 Graz – 6.\,6.\,1928 Oberhof [Kreuth]), \emph{Komponist, Violinist, Geiger}|pw} die Erfüllung eines Wunsches zu verweigern, die ihm die
               Erfüllung seines wesent{\pb}lichern – die
               Aufführung seiner Oper\pwindex{Noren, Heinrich G. 5.\,1.\,1861 Graz – 6.\,6.\,1928 Oberhof [Kreuth]@\textsc{Noren, Heinrich G.} (5.\,1.\,1861 Graz – 6.\,6.\,1928 Oberhof [Kreuth]), \emph{Komponist, Violinist, Geiger}!Schleier der Beatrice. Oper@\strich\emph{Der Schleier der Beatrice. Oper}|pwv} in Wien\oindex{Wien@\textbf{Wien}, \emph{Verwaltungsgebiet}|pw} – in die Nähe zu rücken scheint. Ich weiß
               weder, ob Sie, lieber Hugo, Gelegenheit, noch ob Sie Lust haben sich mit dieser Sache
               in irgend einer Form zu befassen. Vielleicht sprechen wir bald einmal darüber, wenn
               Sie wieder nach Wien\oindex{Wien@\textbf{Wien}, \emph{Verwaltungsgebiet}|pw} hereinkommen. Es wäre ja
               überhaupt schon Zeit, daß man sich wieder einmal sieht und spricht. Ich habe Ihnen
               noch nicht einmal zum \label{K_L02395-2v}\edtext{Erfolg des »Großen Welttheaters\pwindex{Hofmannsthal, Hugo von 1.\,2.\,1874 Wien – 15.\,7.\,1929 Rodaun@\textsc{Hofmannsthal, Hugo von} (1.\,2.\,1874 Wien – 15.\,7.\,1929 Rodaun), \emph{Schriftsteller}!Salzburger große Welttheater@\strich\emph{Das Salzburger große Welttheater}|pw}«}{\lemma{\textnormal{\emph{Erfolg … Welttheaters«}}}\Cendnote{\textnormal{Die Uraufführung\eventindex{Salzburg@\textbf{Salzburg}!Uraufführung von Das Salzburger große Welttheater, 12.8.1922@Uraufführung von Das Salzburger große Welttheater, 12.8.1922|pwkv} fand am 12. 8. 1922 in der Kollegienkirche in Salzburg\oindex{Kollegienkirche (Salzburg)@\textbf{Kollegienkirche (Salzburg)}, \emph{Kirche}|pwk} statt. Regie
                  führte Max Reinhardt\pwindex{Reinhardt, Max 9.\,9.\,1873 Baden bei Wien – 30.\,10.\,1943 New York City@\textsc{Reinhardt, Max} (9.\,9.\,1873 Baden bei Wien – 30.\,10.\,1943 New York City), \emph{Theaterleiter, Regisseur, Schauspieler}|pwk}.}}}\label{K_L02395-2} gratuliert und
               nicht gesagt, wie schön Ihre beiden Artikel\pwindex{Hofmannsthal, Hugo von 1.\,2.\,1874 Wien – 15.\,7.\,1929 Rodaun@\textsc{Hofmannsthal, Hugo von} (1.\,2.\,1874 Wien – 15.\,7.\,1929 Rodaun), \emph{Schriftsteller}!Vienna Letter@\strich\emph{Vienna Letter}|pwv}\pwindex{Hofmannsthal, Hugo von 1.\,2.\,1874 Wien – 15.\,7.\,1929 Rodaun@\textsc{Hofmannsthal, Hugo von} (1.\,2.\,1874 Wien – 15.\,7.\,1929 Rodaun), \emph{Schriftsteller}!Vienna Letter@\strich\emph{Vienna Letter}|pwv} im »Dial\pwindex{Dial@\emph{The Dial}|pw}«
               (nicht nur der über mich\pwindex{Hofmannsthal, Hugo von 1.\,2.\,1874 Wien – 15.\,7.\,1929 Rodaun@\textsc{Hofmannsthal, Hugo von} (1.\,2.\,1874 Wien – 15.\,7.\,1929 Rodaun), \emph{Schriftsteller}!Vienna Letter@\strich\emph{Vienna Letter}|pwv})
               waren.\pend
           \pstart Seien Sie herzlichst gegrüßt \spacefill\mbox{\textcolor{gray}{A. S.}}\pend{}\selectlanguage{ngerman}\endnumbering\briefempfaengerindex{Hofmannsthal, Hugo von@\textsc{Hofmannsthal, Hugo von}!zzzSchnitzler, Arthur@\emph{von Arthur Schnitzler}!1923-01-151@{15. 1. 1923}|)be}\mylabel{L02395h}  \newcommand{\dateiname}{L02395}\newcommand{\titel}{Arthur Schnitzler an Hugo Hofmannsthal, 15. 1. 1923}\newcommand{\editorInnen}{Martin Anton Müller und Gerd-Hermann Susen}%% latex-leseansicht-abspann.tex
%% Abspann für die Leseansicht.
%% Der Schalter \ifkorrekturansicht ist bereits durch den Vorspann gesetzt.

%% latex-abspann.tex
%% Gemeinsamer Abspann für Korrekturansicht und Leseansicht.
%% Setzt den Schalter \ifkorrekturansicht voraus (gesetzt in den
%% einbindenden Dateien latex-korrekturansicht-abspann.tex bzw.
%% latex-leseansicht-abspann.tex).
%% ---------------------------------------------------------------

\normalsize

% Das esempio-Environment wird nur in der Leseansicht benötigt
\ifkorrekturansicht\else
\newenvironment{esempio}[3]%
{
    \vspace{1.5ex}
    \rlap{\underline{#1}}
    \par
    \setlength{\parindent}{0cm}
    \nopagebreak
    \leftskip=#2cm
    \rightskip=#3cm
}
{
    \par
}
\fi

\doendnotes{C}
\bigskip
\vfill

\clearpage

\footnotesize

\ifkorrekturansicht
  \lohead{\textsc{register}}
\fi

% theindex-Environment neu definieren ohne reledmac
\makeatletter
\renewenvironment{theindex}{%
  \ifkorrekturansicht
    \section*{\indexname}%
  \else
    \subsubsection*{Index der erwähnten Entitäten}%
  \fi
  \setlength{\parindent}{0pt}%
  \setlength{\parskip}{0pt plus 0.3pt}%
  \let\item\@idxitem
}{%
  \ifkorrekturansicht\clearpage\fi
}
\makeatother

\IfFileExists{\jobname-pw.ind}{\input{\jobname-pw.ind}}{}

% Quellenangabe nur in der Leseansicht
\ifkorrekturansicht\else
% Fallback-Definitionen, falls die .tex-Datei \titel etc. nicht gesetzt hat
\providecommand{\titel}{}
\providecommand{\editorInnen}{}
\providecommand{\dateiname}{\jobname}

\vspace{3cm}

\vfill

\footnotesize
\textsc{Quelle}: \titel. Herausgegeben von {\editorInnen}. In: \emph{Arthur Schnitzler: Briefwechsel mit Autorinnen und Autoren}.
 Digitale Edition, https://schnitzler-briefe.acdh.oeaw.ac.at/{\dateiname}.html (Stand \today)
\fi

\end{document}


