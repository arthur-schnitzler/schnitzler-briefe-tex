%% latex-leseansicht-vorspann.tex
%% Vorspann für die Leseansicht.
%% Lädt die gemeinsame Datei latex-vorspann.tex mit nicht gesetztem Schalter.

\newif\ifkorrekturansicht
\korrekturansichtfalse

\input{../tex-inputs/latex-vorspann}


\section[Olga und Arthur Schnitzler, Hugo und Gerty Hofmannsthal an Max Mell, {[}18.? 7. 1907{]}]{L01693 Olga und Arthur Schnitzler, Hugo und Gerty Hofmannsthal an Max Mell, {[}18.? 7. 1907{]}}
\nopagebreak\mylabel{L01693v}
\rehead{ }\normalsize\beginnumbering\briefempfaengerindex{Mell, Max@\textsc{Mell, Max}!zzzHofmannsthal, Gertrude von@\emph{von Gertrude von Hofmannsthal}!1907-07-181@{{[}18.? 7. 1907{]}}|(be}\briefempfaengerindex{Mell, Max@\textsc{Mell, Max}!zzzHofmannsthal, Hugo von@\emph{von Hugo von Hofmannsthal}!1907-07-181@{{[}18.? 7. 1907{]}}|(be}\briefempfaengerindex{Mell, Max@\textsc{Mell, Max}!zzzSchnitzler, Olga@\emph{von Olga Schnitzler}!1907-07-181@{{[}18.? 7. 1907{]}}|(be}\briefempfaengerindex{Mell, Max@\textsc{Mell, Max}!zzzSchnitzler, Arthur@\emph{von Arthur Schnitzler}!1907-07-181@{{[}18.? 7. 1907{]}}|(be}
\toendnotes[C]{\smallbreak\pagebreak[2]}
\correspDesc{Versand  durch Arthur Schnitzler, Olga Schnitzler, Hugo von Hofmannsthal, Gerty von Hofmannsthal am [18.? 7. 1907] in Wien
\newline{}Erhalt  durch Max Mell im Zeitraum [18. 7. 1907
                  – 22. 7. 1907?] in Wien}\toendnotes[C]{\smallbreak}
\Standort{Wienbibliothek im Rathaus, H.I.N.-207636.}
\physDesc{Bildpostkarte, 260 Zeichen
\newline{}Handschrift Arthur Schnitzler: Bleistift, deutsche Kurrent
\newline{}Handschrift Olga Schnitzler: Bleistift, lateinische Kurrent
\newline{}Handschrift Gertrude von Hofmannsthal: Bleistift, lateinische Kurrent
\newline{}Handschrift Hugo von Hofmannsthal: Bleistift, deutsche Kurrent}\pstart{}{\pb}{[}hs. Schnitzler:{]} \textsc{Herrn Max Mell}\pend{}\pstart{}\textsc{Wien II/\textsubscript{2}}\oindex{II., Leopoldstadt@\textbf{II., Leopoldstadt}, \emph{Verwaltungsgebiet}|pw}\pend{}\pstart{}\textsc{Wittelsbachstrasse 5}\oindex{Wien@\textbf{Wien}!II., Leopoldstadt@\textbf{II., Leopoldstadt}!Wittelsbachstraße@\textbf{Wittelsbachstraße}, \emph{Straße}|pw}.\pend{}{\bigskip}
\pstart
           \noindent{}\centering{}{\pb}\textcolor{gray}{\textbf{Bruneck, Pustertal (Tirol)\oindex{Bruneck@\textbf{Bruneck}, \emph{Hauptstadt}|pw}}}\pend
           \vspace{1em}
\pstart
           \noindent{}\centering{}{[}hs. Schnitzler:{]} Wir{ }ſind in \textsc{Welsberg}-Waldbrunn\oindex{Wildbad Waldbrunn@\textbf{Wildbad Waldbrunn}, \emph{Spa}|pw}\pend
           
\pstart
           {\pb}{[}hs. Hofmannsthal:{]} \textsc{Herzliche Grüsse}{ }\spacefill\mbox{Gerty Hofmannsthal}\pend
           
\pstart
           {[}hs. Schnitzler:{]} Herzliche Grüße Ihnen und Fräulein \textsc{Mary}\pwindex{Mell, Maria 12.\,7.\,1885 Maribor – 29.\,10.\,1954 Wien@\textsc{Mell, Maria} (12.\,7.\,1885 Maribor – 29.\,10.\,1954 Wien), \emph{Schauspielerin}|pw}! Ihr{ }\spacefill\mbox{ArthSchnitzler}\pend
           
\pstart
           {[}hs. Schnitzler:{]} \textsc{Auch von mir!}{ }\spacefill\mbox{OlgaSchnitzler}\pend
           
\pstart
           {[}hs. Hofmannsthal:{]} Herzlich Gruſs. Wir{ }ſind in wenigen Tagen in Rodaun\oindex{Wien@\textbf{Wien}!XXIII., Liesing@\textbf{XXIII., Liesing}!Rodaun@\textbf{Rodaun}, \emph{Region}|pw}.{ }\spacefill\mbox{HvHofmannsthal}\pend
           \selectlanguage{ngerman}\endnumbering\briefempfaengerindex{Mell, Max@\textsc{Mell, Max}!zzzHofmannsthal, Gertrude von@\emph{von Gertrude von Hofmannsthal}!1907-07-181@{{[}18.? 7. 1907{]}}|)be}\briefempfaengerindex{Mell, Max@\textsc{Mell, Max}!zzzHofmannsthal, Hugo von@\emph{von Hugo von Hofmannsthal}!1907-07-181@{{[}18.? 7. 1907{]}}|)be}\briefempfaengerindex{Mell, Max@\textsc{Mell, Max}!zzzSchnitzler, Olga@\emph{von Olga Schnitzler}!1907-07-181@{{[}18.? 7. 1907{]}}|)be}\briefempfaengerindex{Mell, Max@\textsc{Mell, Max}!zzzSchnitzler, Arthur@\emph{von Arthur Schnitzler}!1907-07-181@{{[}18.? 7. 1907{]}}|)be}\mylabel{L01693h}  \newcommand{\dateiname}{L01693}\newcommand{\titel}{Olga und Arthur Schnitzler, Hugo und Gerty Hofmannsthal an Max Mell, [18.? 7. 1907]}\newcommand{\editorInnen}{Martin Anton Müller und Gerd-Hermann Susen}%% latex-leseansicht-abspann.tex
%% Abspann für die Leseansicht.
%% Der Schalter \ifkorrekturansicht ist bereits durch den Vorspann gesetzt.

%% latex-abspann.tex
%% Gemeinsamer Abspann für Korrekturansicht und Leseansicht.
%% Setzt den Schalter \ifkorrekturansicht voraus (gesetzt in den
%% einbindenden Dateien latex-korrekturansicht-abspann.tex bzw.
%% latex-leseansicht-abspann.tex).
%% ---------------------------------------------------------------

\normalsize

% Das esempio-Environment wird nur in der Leseansicht benötigt
\ifkorrekturansicht\else
\newenvironment{esempio}[3]%
{
    \vspace{1.5ex}
    \rlap{\underline{#1}}
    \par
    \setlength{\parindent}{0cm}
    \nopagebreak
    \leftskip=#2cm
    \rightskip=#3cm
}
{
    \par
}
\fi

\doendnotes{C}
\bigskip
\vfill

\clearpage

\footnotesize

\ifkorrekturansicht
  \lohead{\textsc{register}}
\fi

% theindex-Environment neu definieren ohne reledmac
\makeatletter
\renewenvironment{theindex}{%
  \ifkorrekturansicht
    \section*{\indexname}%
  \else
    \subsubsection*{Index der erwähnten Entitäten}%
  \fi
  \setlength{\parindent}{0pt}%
  \setlength{\parskip}{0pt plus 0.3pt}%
  \let\item\@idxitem
}{%
  \ifkorrekturansicht\clearpage\fi
}
\makeatother

\IfFileExists{\jobname-pw.ind}{\input{\jobname-pw.ind}}{}

% Quellenangabe nur in der Leseansicht
\ifkorrekturansicht\else
% Fallback-Definitionen, falls die .tex-Datei \titel etc. nicht gesetzt hat
\providecommand{\titel}{}
\providecommand{\editorInnen}{}
\providecommand{\dateiname}{\jobname}

\vspace{3cm}

\vfill

\footnotesize
\textsc{Quelle}: \titel. Herausgegeben von {\editorInnen}. In: \emph{Arthur Schnitzler: Briefwechsel mit Autorinnen und Autoren}.
 Digitale Edition, https://schnitzler-briefe.acdh.oeaw.ac.at/{\dateiname}.html (Stand \today)
\fi

\end{document}


