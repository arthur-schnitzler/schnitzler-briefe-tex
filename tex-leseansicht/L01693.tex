%% latex-korrekturansicht-vorspann.tex
%% Vorspann für die Korrekturansicht.
%% Lädt die gemeinsame Datei latex-vorspann.tex mit gesetztem Schalter.

\newif\ifkorrekturansicht
\korrekturansichttrue

\input{../tex-inputs/latex-vorspann}


\section[Olga und Arthur Schnitzler, Hugo und Gerty Hofmannsthal an Max Mell, {[}18.? 7. 1907{]}]{L01693 Olga und Arthur Schnitzler, Hugo und Gerty Hofmannsthal an Max Mell,
               {[}18.? 7. 1907{]}}
\nopagebreak\mylabel{L01693v}
\rehead{ }\normalsize\beginnumbering\briefempfaengerindex{Mell, Max@\textsc{Mell, Max}!zzzHofmannsthal, Gertrude von@\emph{von Gertrude von Hofmannsthal}!1907-07-181@{{[}18.? 7. 1907{]}}|(be}\briefempfaengerindex{Mell, Max@\textsc{Mell, Max}!zzzHofmannsthal, Hugo von@\emph{von Hugo von Hofmannsthal}!1907-07-181@{{[}18.? 7. 1907{]}}|(be}\briefempfaengerindex{Mell, Max@\textsc{Mell, Max}!zzzSchnitzler, Olga@\emph{von Olga Schnitzler}!1907-07-181@{{[}18.? 7. 1907{]}}|(be}\briefempfaengerindex{Mell, Max@\textsc{Mell, Max}!zzzSchnitzler, Arthur@\emph{von Arthur Schnitzler}!1907-07-181@{{[}18.? 7. 1907{]}}|(be}
\toendnotes[C]{\smallbreak\pagebreak[2]}\Standort{Wienbibliothek im Rathaus, H.I.N.-207636.}
\physDesc{Bildpostkarte, 260 Zeichen
\newline{}Handschrift Arthur Schnitzler: Bleistift, deutsche Kurrent
\newline{}Handschrift Olga Schnitzler: Bleistift, lateinische Kurrent
\newline{}Handschrift Gertrude von Hofmannsthal: Bleistift, lateinische Kurrent
\newline{}Handschrift Hugo von Hofmannsthal: Bleistift, deutsche Kurrent}\pstart{}{\pb}{[}hs. :{]} \textsc{Herrn Max Mell}\pend{}\pstart{}\textsc{Wien II/\textsubscript{2}}\oindex{II., Leopoldstadt@\textbf{II., Leopoldstadt}, \emph{A.ADM3}|pw}\pend{}\pstart{}\textsc{Wittelsbachstrasse 5}\oindex{Wittelsbachstrasse@\textbf{Wittelsbachstraße}, \emph{Straße (K.STR)}|pw}.\pend{}{\bigskip}
\pstart
           \noindent{}\centering{}{\pb}\textcolor{gray}{\textbf{Bruneck, Pustertal (Tirol)\oindex{Bruneck@\textbf{Bruneck}, \emph{P.PPLA3}|pw}}}\pend
           \vspace{1em}
\pstart
           \noindent{}\centering{}{[}hs. :{]} Wir ſind in \textsc{Welsberg}-Waldbrunn\oindex{Wildbad Waldbrunn@\textbf{Wildbad Waldbrunn}, \emph{S.SPA}|pw}\pend
           
\pstart
           {\pb}{[}hs. :{]} \textsc{Herzliche Grüsse}{ }\spacefill\mbox{Gerty Hofmannsthal}\pend
           
\pstart
           {[}hs. :{]} Herzliche Grüße Ihnen und Fräulein \textsc{Mary}\pwindex{Mell, Maria 12.07.1885 – 29.10.1954@\textsc{Mell, Maria} (12.07.1885 – 29.10.1954), \emph{Schauspieler/Schauspielerin}|pw}! Ihr{ }\spacefill\mbox{ArthSchnitzler}\pend
           
\pstart
           {[}hs. :{]} \textsc{Auch von mir!}{ }\spacefill\mbox{OlgaSchnitzler}\pend
           
\pstart
           {[}hs. :{]} Herzlich Gruſs. Wir ſind in wenigen Tagen in Rodaun\oindex{Rodaun@\textbf{Rodaun}, \emph{A.ADM4}|pw}.{ }\spacefill\mbox{HvHofmannsthal}\pend
           \selectlanguage{ngerman}\endnumbering\briefempfaengerindex{Mell, Max@\textsc{Mell, Max}!zzzHofmannsthal, Gertrude von@\emph{von Gertrude von Hofmannsthal}!1907-07-181@{{[}18.? 7. 1907{]}}|)be}\briefempfaengerindex{Mell, Max@\textsc{Mell, Max}!zzzHofmannsthal, Hugo von@\emph{von Hugo von Hofmannsthal}!1907-07-181@{{[}18.? 7. 1907{]}}|)be}\briefempfaengerindex{Mell, Max@\textsc{Mell, Max}!zzzSchnitzler, Olga@\emph{von Olga Schnitzler}!1907-07-181@{{[}18.? 7. 1907{]}}|)be}\briefempfaengerindex{Mell, Max@\textsc{Mell, Max}!zzzSchnitzler, Arthur@\emph{von Arthur Schnitzler}!1907-07-181@{{[}18.? 7. 1907{]}}|)be}\mylabel{L01693h}  \normalsize

\doendnotes{C}
\bigskip
\vfill

\clearpage

\footnotesize

\lohead{\textsc{register}}

% Definiere theindex-Environment komplett neu ohne reledmac
\makeatletter
\renewenvironment{theindex}{%
  \section*{\indexname}%
  \setlength{\parindent}{0pt}%
  \setlength{\parskip}{0pt plus 0.3pt}%
  \let\item\@idxitem
}{%
  \clearpage
}
\makeatother

\IfFileExists{\jobname-pw.ind}{\input{\jobname-pw.ind}}{}

\end{document}

      