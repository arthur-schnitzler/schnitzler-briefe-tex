%% latex-leseansicht-vorspann.tex
%% Vorspann für die Leseansicht.
%% Lädt die gemeinsame Datei latex-vorspann.tex mit nicht gesetztem Schalter.

\newif\ifkorrekturansicht
\korrekturansichtfalse

\input{../tex-inputs/latex-vorspann}


\section[Arthur Schnitzler an Gustav Schwarzkopf, 6. 10. 1895]{L04119 Arthur Schnitzler an Gustav Schwarzkopf, 6. 10. 1895}
\nopagebreak\mylabel{L04119v}
\rehead{ }\normalsize\beginnumbering\briefempfaengerindex{Schwarzkopf, Gustav@\textsc{Schwarzkopf, Gustav}!zzzSchnitzler, Arthur@\emph{von Arthur Schnitzler}!1895-10-061@{6. 10. 1895}|(be}
\toendnotes[C]{\smallbreak\pagebreak[2]}
\correspDesc{Versand  durch Arthur Schnitzler am 6. 10. 1895 in Wien
\newline{}Erhalt  durch Gustav Schwarzkopf im Zeitraum [6. 10. 1895 – 9. 10. 1895?] in Wien}\toendnotes[C]{\smallbreak}
\Standort{CUL, Schnitzler, B 96.}
\physDesc{Briefkarte, 305 Zeichen
\newline{}Handschrift: Bleistift, deutsche Kurrent}\toendnotes[C]{\smallbreak}
\pstart
           \noindent{}{\pb}lieber Freund, ich habe für Sie einen
            Sitz\eventindex{Burgtheater@\textbf{Burgtheater}!Uraufführung von Liebelei, Premiere von Rechte der Seele, 9.10.1895@Uraufführung von Liebelei, Premiere von Rechte der Seele, 9.10.1895|pwv} beim Kassier \textsc{Zelzer\pwindex{Zelzer, Georg 1850? – 3.\,1.\,1903 Wien@\textsc{Zelzer, Georg} (1850? – 3.\,1.\,1903 Wien), \emph{Kassier}|pw}}\textsc{reserviren}
      laſſen, Parterre oder 3. Gallerie,
      was Ihnen ja eben recht iſt, we{\geminationn}
      ich Sie neulich richtig verſtanden habe. Vielleicht fragen
               {\pb}Sie Dinſtag Nachmittag nach,
      ich hoffe zuverſichtlich, daſs
      er bereit iſt\pend
           
\pstart
           Herzlichen Gruſs von{\\[\baselineskip]} Ihrem \spacefill\mbox{ArthS}\pend
           \leftskip=0em{}
\pstart
           6. 10. 95\pend
           \selectlanguage{ngerman}\endnumbering\briefempfaengerindex{Schwarzkopf, Gustav@\textsc{Schwarzkopf, Gustav}!zzzSchnitzler, Arthur@\emph{von Arthur Schnitzler}!1895-10-061@{6. 10. 1895}|)be}\mylabel{L04119h}
\begin{anhang}
\end{anhang}\newcommand{\dateiname}{L04119}\newcommand{\titel}{Arthur Schnitzler an Gustav Schwarzkopf, 6. 10. 1895}\newcommand{\editorInnen}{Herausgegeben von Jahnke, SelmaMüller, Martin Anton}%% latex-leseansicht-abspann.tex
%% Abspann für die Leseansicht.
%% Der Schalter \ifkorrekturansicht ist bereits durch den Vorspann gesetzt.

%% latex-abspann.tex
%% Gemeinsamer Abspann für Korrekturansicht und Leseansicht.
%% Setzt den Schalter \ifkorrekturansicht voraus (gesetzt in den
%% einbindenden Dateien latex-korrekturansicht-abspann.tex bzw.
%% latex-leseansicht-abspann.tex).
%% ---------------------------------------------------------------

\normalsize

% Das esempio-Environment wird nur in der Leseansicht benötigt
\ifkorrekturansicht\else
\newenvironment{esempio}[3]%
{
    \vspace{1.5ex}
    \rlap{\underline{#1}}
    \par
    \setlength{\parindent}{0cm}
    \nopagebreak
    \leftskip=#2cm
    \rightskip=#3cm
}
{
    \par
}
\fi

\doendnotes{C}
\bigskip
\vfill

\clearpage

\footnotesize

\ifkorrekturansicht
  \lohead{\textsc{register}}
\fi

% theindex-Environment neu definieren ohne reledmac
\makeatletter
\renewenvironment{theindex}{%
  \ifkorrekturansicht
    \section*{\indexname}%
  \else
    \subsubsection*{Index der erwähnten Entitäten}%
  \fi
  \setlength{\parindent}{0pt}%
  \setlength{\parskip}{0pt plus 0.3pt}%
  \let\item\@idxitem
}{%
  \ifkorrekturansicht\clearpage\fi
}
\makeatother

\IfFileExists{\jobname-pw.ind}{\input{\jobname-pw.ind}}{}

% Quellenangabe nur in der Leseansicht
\ifkorrekturansicht\else
% Fallback-Definitionen, falls die .tex-Datei \titel etc. nicht gesetzt hat
\providecommand{\titel}{}
\providecommand{\editorInnen}{}
\providecommand{\dateiname}{\jobname}

\vspace{3cm}

\vfill

\footnotesize
\textsc{Quelle}: \titel. Herausgegeben von {\editorInnen}. In: \emph{Arthur Schnitzler: Briefwechsel mit Autorinnen und Autoren}.
 Digitale Edition, https://schnitzler-briefe.acdh.oeaw.ac.at/{\dateiname}.html (Stand \today)
\fi

\end{document}


