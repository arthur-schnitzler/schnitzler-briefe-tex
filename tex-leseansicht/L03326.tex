%% latex-leseansicht-vorspann.tex
%% Vorspann für die Leseansicht.
%% Lädt die gemeinsame Datei latex-vorspann.tex mit nicht gesetztem Schalter.

\newif\ifkorrekturansicht
\korrekturansichtfalse

\input{../tex-inputs/latex-vorspann}


\section[ Felix Salten an Arthur Schnitzler, 12. 3. 1902]{L03326 Felix Salten an Arthur Schnitzler,  12. 3. 1902}
\nopagebreak\mylabel{L03326v}
\rehead{ }\normalsize\beginnumbering\briefempfaengerindex{Schnitzler, Arthur@\textsc{Schnitzler, Arthur}!zzzSalten, Felix@\emph{von Felix Salten}!1902-03-121@{12. 3. 1902}|(be}
\toendnotes[C]{\smallbreak\pagebreak[2]}
\correspDesc{Versand  durch Felix Salten am 12. 3. 1902 in Wien
\newline{}Erhalt  durch Arthur Schnitzler im Zeitraum [12. 3. 1902
                  – 15. 3. 1902?] in Wien}\toendnotes[C]{\smallbreak}
\Standort{CUL, Schnitzler, B 89, A 2.}
\physDesc{Brief, 1 Blatt, 2 Seiten, 819 Zeichen
\newline{}Handschrift: blaue Tinte, lateinische Kurrent
\newline{}Ordnung: mit Bleistift von unbekannter Hand nummeriert: »150« }\toendnotes[C]{\smallbreak}
\pstart
           \raggedleft{}{\pb}den 12. März 02.\pend
           \vspace{0.5em}
\pstart
           Lieber – mit der \label{K_L03326-1v}\edtext{»Zeit\orgindex{Zeit@Die Zeit|pw}«}{\lemma{\textnormal{\emph{»Zeit«}}}\Cendnote{\textnormal{Die Herausgeber Heinrich Kanner\pwindex{Kanner, Heinrich 9.\,11.\,1864 Galați – 15.\,2.\,1930 Wien@\textsc{Kanner, Heinrich} (9.\,11.\,1864 Galați – 15.\,2.\,1930 Wien), \emph{Herausgeber, Publizist}|pwk} und Isidor Singer\pwindex{Singer, Isidor 16.\,1.\,1857 Budapest – 8.\,12.\,1927 Wien@\textsc{Singer, Isidor} (16.\,1.\,1857 Budapest – 8.\,12.\,1927 Wien), \emph{Journalist, Herausgeber, Soziologe}|pwk} planten, die Wochenschrift mit
                  diesem Titel um eine gleichnamige Tageszeitung zu erweitern. Diese erschien ab
                     27. 9. 1902. Bis dahin verfasste Salten\pwindex{Salten, Felix 6.\,9.\,1869 Budapest – 8.\,10.\,1945 Zürich@\textsc{Salten, Felix} (6.\,9.\,1869 Budapest – 8.\,10.\,1945 Zürich), \emph{Schriftsteller, Journalist, Chefredakteur}|pwk} noch unter dem Pseudonym »Martin Finder« Beiträge für die Wochenschrift\orgindex{Zeit. Wiener Wochenschrift@Die Zeit. Wiener Wochenschrift|pwkv}. Im Hinblick
                  auf Schnitzler könnte sich Salten\pwindex{Salten, Felix 6.\,9.\,1869 Budapest – 8.\,10.\,1945 Zürich@\textsc{Salten, Felix} (6.\,9.\,1869 Budapest – 8.\,10.\,1945 Zürich), \emph{Schriftsteller, Journalist, Chefredakteur}|pwk} auf eine mögliche Publikation bezogen haben. Am
                     26. 7. 1902 erschien in der Zeit \emph{Andreas Thameyers letzter Brief}\pwindex{Schnitzler, Arthur 15.\,5.\,1862 Wien – 21.\,10.\,1931 ebd.@\textsc{Schnitzler, Arthur} (15.\,5.\,1862 Wien – 21.\,10.\,1931 ebd.), \emph{Schriftsteller, Mediziner}!Andreas Thameyers letzter Brief@\strich\emph{Andreas Thameyers letzter Brief}|pwk} (Jg. 32, Nr. 408,
                     S. 63–64).}}}\label{K_L03326-1} bin ich noch lange nicht fertig, und in ernsten
               Verhandlungen eben wegen der Feuilletonredaction. Diese Unterhandlungen werden
               voraussichtlich, – da sie ein negatives Resultat während der ersten Unterredungen
               nicht hatten – bis gegen Ende April dauern, und läßt
               sich heute trotz alledem ihr Ausgang nicht einmal annähernd voraussagen. Sollte aber
               irgend ein Ergebnis früher eintreten, dann theile ich es Ihnen gewiss sogleich mit.
               Im Übrigen – ich brauche das wol nicht zu sagen – soll diese Mittheilung Sie in
               keiner Weise \label{K_L03326-2v}\edtext{beeinflußen}{\lemma{\textnormal{\emph{beeinflußen}}}\Cendnote{\textnormal{Am 6. 3. 1902 schrieb Schnitzler an einer ersten Fassung von \emph{Dämmerseele}\pwindex{Schnitzler, Arthur 15.\,5.\,1862 Wien – 21.\,10.\,1931 ebd.@\textsc{Schnitzler, Arthur} (15.\,5.\,1862 Wien – 21.\,10.\,1931 ebd.), \emph{Schriftsteller, Mediziner}!Dämmerseele@\strich\emph{Dämmerseele}|pwk}. Möglicherweise überlegte er den Text Salten\pwindex{Salten, Felix 6.\,9.\,1869 Budapest – 8.\,10.\,1945 Zürich@\textsc{Salten, Felix} (6.\,9.\,1869 Budapest – 8.\,10.\,1945 Zürich), \emph{Schriftsteller, Journalist, Chefredakteur}|pwk} zur Publikation anzuvertrauen, sofern
                  dessen Verhältnis mit der \emph{Zeit}\orgindex{Zeit@Die Zeit|pwk} mittlerweile fixiert gewesen wäre. Anstatt auf diese Weise erschien der Text am 18. 5. 1902 in der \emph{Neuen Freien Presse}\pwindex{Neue Freie Presse@\emph{Neue Freie Presse}|pwk} (Arthur Schnitzler: \emph{Dämmerseele}\pwindex{Schnitzler, Arthur 15.\,5.\,1862 Wien – 21.\,10.\,1931 ebd.@\textsc{Schnitzler, Arthur} (15.\,5.\,1862 Wien – 21.\,10.\,1931 ebd.), \emph{Schriftsteller, Mediziner}!Dämmerseele@\strich\emph{Dämmerseele}|pwk}. In: \emph{Neue
                        Freie Presse}\pwindex{Neue Freie Presse@\emph{Neue Freie Presse}|pwk}, Nr. 13.553, 18.\,5.\,1902,
                     Morgenblatt, Pfingstbeilage, S. 31–33). }}}\label{K_L03326-2}.\pend
           
\pstart
           Ich bin seit heute außer Bett, gehe morgen ins Burgtheater\oindex{Wien@\textbf{Wien}!I., Innere Stadt@\textbf{I., Innere Stadt}!Burgtheater@\textbf{Burgtheater}, \emph{Theater}|pw}
               und möchte Sie jedenfalls bald gerne sprechen. Kann aber Abends nicht
               ausgehen. Vielleicht entschließen Sie sich, dieser Tage nach {\pb}dem \label{K_L03326-3v}\edtext{Nachtmahl zu mir zu kommen? Samstag? od. Freitag}{\lemma{\textnormal{\emph{Nachtmahl … Freitag}}}\Cendnote{\textnormal{Schnitzler kam am Freitag, dem 14. 3. 1902.}}}\label{K_L03326-3}?\pend
           
\pstart
           herzlichst {\\[\baselineskip]}Ihr {\\[\baselineskip]}\spacefill\mbox{Salten}\pend
           \leftskip=0em{}\selectlanguage{ngerman}\endnumbering\briefempfaengerindex{Schnitzler, Arthur@\textsc{Schnitzler, Arthur}!zzzSalten, Felix@\emph{von Felix Salten}!1902-03-121@{12. 3. 1902}|)be}\mylabel{L03326h}  \newcommand{\dateiname}{L03326}\newcommand{\titel}{Felix Salten an Arthur Schnitzler, 12. 3. 1902}\newcommand{\editorInnen}{Martin Anton Müller und Laura Untner}%% latex-leseansicht-abspann.tex
%% Abspann für die Leseansicht.
%% Der Schalter \ifkorrekturansicht ist bereits durch den Vorspann gesetzt.

%% latex-abspann.tex
%% Gemeinsamer Abspann für Korrekturansicht und Leseansicht.
%% Setzt den Schalter \ifkorrekturansicht voraus (gesetzt in den
%% einbindenden Dateien latex-korrekturansicht-abspann.tex bzw.
%% latex-leseansicht-abspann.tex).
%% ---------------------------------------------------------------

\normalsize

% Das esempio-Environment wird nur in der Leseansicht benötigt
\ifkorrekturansicht\else
\newenvironment{esempio}[3]%
{
    \vspace{1.5ex}
    \rlap{\underline{#1}}
    \par
    \setlength{\parindent}{0cm}
    \nopagebreak
    \leftskip=#2cm
    \rightskip=#3cm
}
{
    \par
}
\fi

\doendnotes{C}
\bigskip
\vfill

\clearpage

\footnotesize

\ifkorrekturansicht
  \lohead{\textsc{register}}
\fi

% theindex-Environment neu definieren ohne reledmac
\makeatletter
\renewenvironment{theindex}{%
  \ifkorrekturansicht
    \section*{\indexname}%
  \else
    \subsubsection*{Index der erwähnten Entitäten}%
  \fi
  \setlength{\parindent}{0pt}%
  \setlength{\parskip}{0pt plus 0.3pt}%
  \let\item\@idxitem
}{%
  \ifkorrekturansicht\clearpage\fi
}
\makeatother

\IfFileExists{\jobname-pw.ind}{\input{\jobname-pw.ind}}{}

% Quellenangabe nur in der Leseansicht
\ifkorrekturansicht\else
% Fallback-Definitionen, falls die .tex-Datei \titel etc. nicht gesetzt hat
\providecommand{\titel}{}
\providecommand{\editorInnen}{}
\providecommand{\dateiname}{\jobname}

\vspace{3cm}

\vfill

\footnotesize
\textsc{Quelle}: \titel. Herausgegeben von {\editorInnen}. In: \emph{Arthur Schnitzler: Briefwechsel mit Autorinnen und Autoren}.
 Digitale Edition, https://schnitzler-briefe.acdh.oeaw.ac.at/{\dateiname}.html (Stand \today)
\fi

\end{document}


