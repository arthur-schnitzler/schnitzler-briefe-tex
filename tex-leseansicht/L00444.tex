%% latex-leseansicht-vorspann.tex
%% Vorspann für die Leseansicht.
%% Lädt die gemeinsame Datei latex-vorspann.tex mit nicht gesetztem Schalter.

\newif\ifkorrekturansicht
\korrekturansichtfalse

\input{../tex-inputs/latex-vorspann}

\begin{center}
            \textcolor{red}{ENTWURF. ENTZIFFERUNG NOCH NICHT KORREKTURGELESEN}
                      \end{center}
            
               \section[Friedrich M. Fels an Arthur Schnitzler, 20. 5. 1895]{ Friedrich M. Fels an Arthur Schnitzler, 20. 5. 1895}\nopagebreak\mylabel{v}\rehead{ }\begin{ledgroupsized}[t]{13cm}\normalsize\beginnumbering\briefempfaengerindex{Schnitzler, Arthur@\textsc{Schnitzler, Arthur}!zzzFels, Friedrich Michael@\emph{von Friedrich Michael Fels}!1895-05-201@{20. 5. 1895}|(be} \toendnotes[C]{\smallbreak\pagebreak[2]} \Standort{DLA, A:Schnitzler, HS.NZ85.1.2956.}
\physDesc{Kartenbrief
\newline{}Handschrift: schwarze Tinte, lateinische Kurrent\newline{}Versand: 1) Stempel: »\nobreak{}\oindex{I., Innere Stadt@\textbf{I., Innere Stadt}|pwk}Wien 1/1, 20. 5. 95, 1–2N\nobreak{}«.  2) Stempel: »\nobreak{}\oindex{IX., Alsergrund@\textbf{IX., Alsergrund}|pwk}Wien 9/3, 20. 5. 95, 3.N, Bestellt\nobreak{}«. 
\newline{}Schnitzler: mit Bleistift datiert: »23/4 95« und nummeriert: »22« }\toendnotes[C]{\smallbreak}\pstart{}{\pb}Herrn\pend{}\pstart{}Dr. Arthur Schnitzler\pend{}\pstart{}Wien\oindex{Wien@\textbf{Wien}|pw}\pend{}\pstart{}IX, Frankgaſse \damage{1}\oindex{Frankgasse@\textbf{Frankgasse}|pw}\pend{}{\bigskip}\pstart
           \noindent{}{\pb}Lieber Dr Schnitzler!  Sie sagten mir neulich, Sie wollten mit Beer-Hofma{\geminationn}\pwindex{Beer-Hofmann, Richard 11.07.1866 – 26.09.1945@\textsc{Beer-Hofmann, Richard} (11.07.1866 – 26.09.1945), \emph{Schriftsteller}|pw} reden wegen eines Anzugs; falls Sie es nicht gethan haben, darf ich jetzt wohl
               daran eri{\geminationn}ern. Es ist sehr langweilig, seine Hose jeden
               Morgen, da man sie anzieht, flicken zu müſsen. – Haben Sie das Buch\pwindex{Groeger, Fanny 12.01.1869 – 07.04.1936@\textsc{Gröger, Fanny} (12.01.1869 – 07.04.1936), \emph{Schriftstellerin}!Adhimukti1895 – 1895@\strich\emph{Adhimukti} {[}1895 – 1895{]}|pwv} der Fa{\geminationn}y Gröger\pwindex{Groeger, Fanny 12.01.1869 – 07.04.1936@\textsc{Gröger, Fanny} (12.01.1869 – 07.04.1936), \emph{Schriftstellerin}|pw}{ }ſchon gesehen, oder besitzen Sie es gar? We{\geminationn} ja, darf ich Sie später auf ein paar Tage darum
                  \damage{bi}tten? – Mit Hirschfeld\pwindex{Hirschfeld, Robert 17.09.1857 – 02.04.1914@\textsc{Hirschfeld, Robert} (17.09.1857 – 02.04.1914), \emph{Journalist, Musikkritiker}|pw} habe ich nicht
               gesprochen. Doch werde ich dieser Tage zu ihm gehen, um ihm ein neues Feuilleton zu
               bringen; da{\geminationn} erfahre ich wohl auch, ob aus Ossiacher See\oindex{Ossiacher See@\textbf{Ossiacher See}|pw} etwas wird. – Beiläufig: Sie müſsen
               ja ganz hochmütig geworden sein. \label{K_L00444_1v}\edtext{150 frcs für Übersetzungsrecht\pwindex{Schnitzler, Arthur 15.05.1862 – 21.10.1931@\textsc{Schnitzler, Arthur} (15.05.1862 – 21.10.1931), \emph{Schriftsteller, Mediziner}!Sterben. Novelle1.10.1894 – 1.12.1894@\strich\emph{Sterben. Novelle} {[}1.10.1894 – 1.12.1894{]}|pwv}}{\lemma{\textnormal{\emph{150 frcs für Übersetzungsrecht}}}\Cendnote{\textnormal{Für die französische Übersetzung von \emph{Sterben}\pwindex{Schnitzler, Arthur 15.05.1862 – 21.10.1931@\textsc{Schnitzler, Arthur} (15.05.1862 – 21.10.1931), \emph{Schriftsteller, Mediziner}!Sterben. Novelle1.10.1894 – 1.12.1894@\strich\emph{Sterben. Novelle} {[}1.10.1894 – 1.12.1894{]}|pwk} vgl. den Antrag durch Raoul Bourse\pwindex{Bourse, Raoul *~14.04.1871@\textsc{Bourse, Raoul} (*~14.04.1871), \emph{Kaufmann}|pwk} (A. S.: \emph{Tagebuch}, 1. 5. 1895), die Übersetzung erfolgte durch Gaspard Vallette\pwindex{Vallette, Gaspard 13.5.1865 – 6.8.1911@\textsc{Vallette, Gaspard} (13.5.1865 – 6.8.1911), \emph{Journalist, Übersetzer}|pwk}.}}}\label{K_L00444_1h} – so was hätten Sie sich so bald
               nicht träumen laſsen.\pend
           \pstart
           Herzl. Gruſs und Dank{\\[\baselineskip]}\spacefill\mbox{F.}\pend
           \leftskip=0em{}\pstart
           \noindent{}Wien XVIII, Währinger-Gürtel 154 part. Th. 9\oindex{Waehringer Guertel@\textbf{Währinger Gürtel}|pw}\pend
           \endnumbering\briefempfaengerindex{Schnitzler, Arthur@\textsc{Schnitzler, Arthur}!zzzFels, Friedrich Michael@\emph{von Friedrich Michael Fels}!1895-05-201@{20. 5. 1895}|)be}\mylabel{h}\end{ledgroupsized}  \newcommand{\dateiname}{L00444}\newcommand{\titel}{Friedrich M. Fels an Arthur Schnitzler, 20. 5. 1895}\newcommand{\editorInnen}{Martin Anton Müller und Gerd-Hermann Susen}%% latex-leseansicht-abspann.tex
%% Abspann für die Leseansicht.
%% Der Schalter \ifkorrekturansicht ist bereits durch den Vorspann gesetzt.

%% latex-abspann.tex
%% Gemeinsamer Abspann für Korrekturansicht und Leseansicht.
%% Setzt den Schalter \ifkorrekturansicht voraus (gesetzt in den
%% einbindenden Dateien latex-korrekturansicht-abspann.tex bzw.
%% latex-leseansicht-abspann.tex).
%% ---------------------------------------------------------------

\normalsize

% Das esempio-Environment wird nur in der Leseansicht benötigt
\ifkorrekturansicht\else
\newenvironment{esempio}[3]%
{
    \vspace{1.5ex}
    \rlap{\underline{#1}}
    \par
    \setlength{\parindent}{0cm}
    \nopagebreak
    \leftskip=#2cm
    \rightskip=#3cm
}
{
    \par
}
\fi

\doendnotes{C}
\bigskip
\vfill

\clearpage

\footnotesize

\ifkorrekturansicht
  \lohead{\textsc{register}}
\fi

% theindex-Environment neu definieren ohne reledmac
\makeatletter
\renewenvironment{theindex}{%
  \ifkorrekturansicht
    \section*{\indexname}%
  \else
    \subsubsection*{Index der erwähnten Entitäten}%
  \fi
  \setlength{\parindent}{0pt}%
  \setlength{\parskip}{0pt plus 0.3pt}%
  \let\item\@idxitem
}{%
  \ifkorrekturansicht\clearpage\fi
}
\makeatother

\IfFileExists{\jobname-pw.ind}{\input{\jobname-pw.ind}}{}

% Quellenangabe nur in der Leseansicht
\ifkorrekturansicht\else
% Fallback-Definitionen, falls die .tex-Datei \titel etc. nicht gesetzt hat
\providecommand{\titel}{}
\providecommand{\editorInnen}{}
\providecommand{\dateiname}{\jobname}

\vspace{3cm}

\vfill

\footnotesize
\textsc{Quelle}: \titel. Herausgegeben von {\editorInnen}. In: \emph{Arthur Schnitzler: Briefwechsel mit Autorinnen und Autoren}.
 Digitale Edition, https://schnitzler-briefe.acdh.oeaw.ac.at/{\dateiname}.html (Stand \today)
\fi

\end{document}


      