%% latex-korrekturansicht-vorspann.tex
%% Vorspann für die Korrekturansicht.
%% Lädt die gemeinsame Datei latex-vorspann.tex mit gesetztem Schalter.

\newif\ifkorrekturansicht
\korrekturansichttrue

\input{../tex-inputs/latex-vorspann}


\section[Friedrich M. Fels an Arthur Schnitzler, 20. 5. 1895]{L00444 Friedrich M. Fels an Arthur Schnitzler, 20. 5. 1895}
\nopagebreak\mylabel{L00444v}
\rehead{ }\normalsize\beginnumbering\briefempfaengerindex{Schnitzler, Arthur@\textsc{Schnitzler, Arthur}!zzzFels, Friedrich Michael@\emph{von Friedrich Michael Fels}!1895-05-201@{20. 5. 1895}|(be}
\toendnotes[C]{\smallbreak\pagebreak[2]}\Standort{DLA, A:Schnitzler, HS.NZ85.1.2956.}
\physDesc{Kartenbrief, 818 Zeichen
\newline{}Handschrift: schwarze Tinte, lateinische Kurrent
\newline{}Versand: 1) Stempel: »\nobreak{}\oindex{I., Innere Stadt@\textbf{I., Innere Stadt}, \emph{A.ADM3}|pwk}Wien 1/1, 20. 5. 95, 1–2N\nobreak{}«.   2) Stempel: »\nobreak{}\oindex{IX., Alsergrund@\textbf{IX., Alsergrund}, \emph{A.ADM3}|pwk}Wien 9/3, 20. 5. 95, 3.N, Bestellt\nobreak{}«. 
\newline{}Schnitzler: mit Bleistift datiert: »23/4 95« und nummeriert: »22« }\toendnotes[C]{\smallbreak}\pstart{}{\pb}Herrn\pend{}\pstart{}Dr. Arthur Schnitzler\pend{}\pstart{}Wien\oindex{Wien@\textbf{Wien}, \emph{A.ADM2}|pw}\pend{}\pstart{}IX, Frankgaſse \damage{1}\oindex{Frankgasse 1@\textbf{Frankgasse 1}, \emph{Wohngebäude (K.WHS)}|pw}\pend{}{\bigskip}\vspace{1em}
\pstart
           \noindent{}{\pb}Lieber Dr Schnitzler!  Sie sagten mir neulich, Sie wollten mit Beer-Hofma{\geminationn}\pwindex{Beer-Hofmann, Richard 1866-07-11 – 1945-09-26@\textsc{Beer-Hofmann, Richard} (1866-07-11 – 1945-09-26), \emph{Schriftsteller/Schriftstellerin}|pw} reden wegen eines Anzugs; falls Sie es nicht gethan haben, darf ich jetzt wohl
               daran eri{\geminationn}ern. Es ist sehr langweilig, seine Hose jeden
               Morgen, da man sie anzieht, flicken zu müſsen. – Haben Sie das Buch\pwindex{Adhimukti@\emph{Adhimukti}|pwv} der Fa{\geminationn}y Gröger\pwindex{Groeger, Fanny 12.01.1869 – 07.04.1936@\textsc{Gröger, Fanny} (12.01.1869 – 07.04.1936), \emph{Schriftsteller/Schriftstellerin}|pw}{ }ſchon gesehen, oder besitzen Sie es gar? We{\geminationn} ja, darf ich Sie später auf ein paar Tage darum
                  \damage{bi}tten? – Mit Hirschfeld\pwindex{Hirschfeld, Robert 17.09.1857 – 02.04.1914@\textsc{Hirschfeld, Robert} (17.09.1857 – 02.04.1914), \emph{Journalist/Journalistin, Musikkritiker/Musikkritikerin}|pw} habe ich nicht
               gesprochen. Doch werde ich dieser Tage zu ihm gehen, um ihm ein neues Feuilleton zu
               bringen; da{\geminationn} erfahre ich wohl auch, ob aus Ossiacher See\oindex{Ossiacher See@\textbf{Ossiacher See}, \emph{See (N.SEE)}|pw} etwas wird. – Beiläufig: Sie
               müſsen ja ganz hochmütig geworden sein. \label{K_L00444-1v}\edtext{150 frcs für Übersetzungsrecht\pwindex{Sterben. Novelle@\emph{Sterben. Novelle}|pwv}}{\lemma{\textnormal{\emph{150 frcs für Übersetzungsrecht}}}\Cendnote{\textnormal{Für die französische Übersetzung von \emph{Sterben}\pwindex{Sterben. Novelle@\emph{Sterben. Novelle}|pwk} vgl. den Antrag durch Raoul Bourse\pwindex{Bourse, Raoul *~14.04.1871@\textsc{Bourse, Raoul} (*~14.04.1871), \emph{Kaufmann/Kauffrau}|pwk} (A. S.: \emph{Tagebuch}, 1. 5. 1895), die Übersetzung erfolgte
                  durch Gaspard Vallette\pwindex{Vallette, Gaspard 13.5.1865 – 6.8.1911@\textsc{Vallette, Gaspard} (13.5.1865 – 6.8.1911), \emph{Journalist/Journalistin, Übersetzer/Übersetzerin}|pwk}.}}}\label{K_L00444-1} – so was
               hätten Sie sich so bald nicht träumen laſsen.\pend
           
\pstart
           Herzl. Gruſs und Dank{\\[\baselineskip]}\spacefill\mbox{F.}\pend
           \leftskip=0em{}
\pstart
           \noindent{}Wien XVIII, Währinger-Gürtel 154 part. Th. 9\oindex{Waehringer Guertel@\textbf{Währinger Gürtel}, \emph{Straße (K.STR)}|pw}\pend
           \selectlanguage{ngerman}\endnumbering\briefempfaengerindex{Schnitzler, Arthur@\textsc{Schnitzler, Arthur}!zzzFels, Friedrich Michael@\emph{von Friedrich Michael Fels}!1895-05-201@{20. 5. 1895}|)be}\mylabel{L00444h}  \normalsize

\doendnotes{C}
\bigskip
\vfill

\clearpage

\footnotesize

\lohead{\textsc{register}}

% Definiere theindex-Environment komplett neu ohne reledmac
\makeatletter
\renewenvironment{theindex}{%
  \section*{\indexname}%
  \setlength{\parindent}{0pt}%
  \setlength{\parskip}{0pt plus 0.3pt}%
  \let\item\@idxitem
}{%
  \clearpage
}
\makeatother

\IfFileExists{\jobname-pw.ind}{\input{\jobname-pw.ind}}{}

\end{document}

      