%% latex-leseansicht-vorspann.tex
%% Vorspann für die Leseansicht.
%% Lädt die gemeinsame Datei latex-vorspann.tex mit nicht gesetztem Schalter.

\newif\ifkorrekturansicht
\korrekturansichtfalse

\input{../tex-inputs/latex-vorspann}


\section[Friedrich M. Fels an Arthur Schnitzler, 20. 5. 1895]{L00444 Friedrich M. Fels an Arthur Schnitzler, 20. 5. 1895}
\nopagebreak\mylabel{L00444v}
\rehead{ }\normalsize\beginnumbering\briefempfaengerindex{Schnitzler, Arthur@\textsc{Schnitzler, Arthur}!zzzFels, Friedrich Michael@\emph{von Friedrich Michael Fels}!1895-05-201@{20. 5. 1895}|(be}
\toendnotes[C]{\smallbreak\pagebreak[2]}
\correspDesc{Versand  durch Friedrich M. Fels am 20. 5. 1895 in Wien
\newline{}Erhalt  durch Arthur Schnitzler am 20. 5. 1895 in Wien}\toendnotes[C]{\smallbreak}
\Standort{DLA, A:Schnitzler, HS.NZ85.1.2956.}
\physDesc{Kartenbrief, 818 Zeichen
\newline{}Handschrift: schwarze Tinte, lateinische Kurrent
\newline{}Versand: 1) Stempel: »\nobreak{}\oindex{I., Innere Stadt@\textbf{I., Innere Stadt}, \emph{Verwaltungsgebiet}|pwk}Wien 1/1, 20. 5. 95, 1–2N\nobreak{}«.   2) Stempel: »\nobreak{}\oindex{IX., Alsergrund@\textbf{IX., Alsergrund}, \emph{Verwaltungsgebiet}|pwk}Wien 9/3, 20. 5. 95, 3.N, Bestellt\nobreak{}«. 
\newline{}Schnitzler: mit Bleistift datiert: »23/4 95« und nummeriert: »22« }\toendnotes[C]{\smallbreak}\pstart{}{\pb}Herrn\pend{}\pstart{}Dr. Arthur Schnitzler\pend{}\pstart{}Wien\oindex{Wien@\textbf{Wien}, \emph{Verwaltungsgebiet}|pw}\pend{}\pstart{}IX, Frankgaſse \damage{1}\oindex{Wien@\textbf{Wien}!IX., Alsergrund@\textbf{IX., Alsergrund}!Frankgasse 1@\textbf{Frankgasse 1}, \emph{Wohngebäude}|pw}\pend{}{\bigskip}\vspace{1em}
\pstart
           \noindent{}{\pb}Lieber Dr Schnitzler!  Sie sagten mir neulich, Sie wollten mit Beer-Hofma{\geminationn}\pwindex{Beer-Hofmann, Richard 11.\,7.\,1866 Wien – 26.\,9.\,1945 New York City@\textsc{Beer-Hofmann, Richard} (11.\,7.\,1866 Wien – 26.\,9.\,1945 New York City), \emph{Schriftsteller}|pw} reden wegen eines Anzugs; falls Sie es nicht gethan haben, darf ich jetzt wohl
               daran eri{\geminationn}ern. Es ist sehr langweilig, seine Hose jeden
               Morgen, da man sie anzieht, flicken zu müſsen. – Haben Sie das Buch\pwindex{Gröger, Fanny 12.\,1.\,1869 Wien – 7.\,4.\,1936 ebd.@\textsc{Gröger, Fanny} (12.\,1.\,1869 Wien – 7.\,4.\,1936 ebd.), \emph{Schriftstellerin}!Adhimukti@\strich\emph{Adhimukti}|pwv} der Fa{\geminationn}y Gröger\pwindex{Gröger, Fanny 12.\,1.\,1869 Wien – 7.\,4.\,1936 ebd.@\textsc{Gröger, Fanny} (12.\,1.\,1869 Wien – 7.\,4.\,1936 ebd.), \emph{Schriftstellerin}|pw}{ }ſchon gesehen, oder besitzen Sie es gar? We{\geminationn} ja, darf ich Sie später auf ein paar Tage darum
                  \damage{bi}tten? – Mit Hirschfeld\pwindex{Hirschfeld, Robert 17.\,9.\,1857 Žďár nad Sázavou – 2.\,4.\,1914 Salzburg@\textsc{Hirschfeld, Robert} (17.\,9.\,1857 Žďár nad Sázavou – 2.\,4.\,1914 Salzburg), \emph{Journalist, Musikkritiker}|pw} habe ich nicht
               gesprochen. Doch werde ich dieser Tage zu ihm gehen, um ihm ein neues Feuilleton zu
               bringen; da{\geminationn} erfahre ich wohl auch, ob aus Ossiacher See\oindex{Ossiacher See@\textbf{Ossiacher See}, \emph{See}|pw} etwas wird. – Beiläufig: Sie
               müſsen ja ganz hochmütig geworden sein. \label{K_L00444-1v}\edtext{150 frcs für Übersetzungsrecht\pwindex{Schnitzler, Arthur 15.\,5.\,1862 Wien – 21.\,10.\,1931 ebd.@\textsc{Schnitzler, Arthur} (15.\,5.\,1862 Wien – 21.\,10.\,1931 ebd.), \emph{Schriftsteller, Mediziner}!Sterben. Novelle@\strich\emph{Sterben. Novelle}|pwv}}{\lemma{\textnormal{\emph{150 frcs für Übersetzungsrecht}}}\Cendnote{\textnormal{Für die französische Übersetzung von \emph{Sterben}\pwindex{Schnitzler, Arthur 15.\,5.\,1862 Wien – 21.\,10.\,1931 ebd.@\textsc{Schnitzler, Arthur} (15.\,5.\,1862 Wien – 21.\,10.\,1931 ebd.), \emph{Schriftsteller, Mediziner}!Sterben. Novelle@\strich\emph{Sterben. Novelle}|pwk} vgl. den Antrag durch Raoul Bourse\pwindex{Bourse, Raoul *~14.\,4.\,1871 Paris@\textsc{Bourse, Raoul} (*~14.\,4.\,1871 Paris), \emph{Kaufmann}|pwk} (A. S.: \emph{Tagebuch}, 1. 5. 1895), die Übersetzung erfolgte
                  durch Gaspard Vallette\pwindex{Vallette, Gaspard 13.\,5.\,1865 Jussy – 6.\,8.\,1911 La Tène@\textsc{Vallette, Gaspard} (13.\,5.\,1865 Jussy – 6.\,8.\,1911 La Tène), \emph{Journalist, Übersetzer}|pwk}.}}}\label{K_L00444-1} – so was
               hätten Sie sich so bald nicht träumen laſsen.\pend
           
\pstart
           Herzl. Gruſs und Dank{\\[\baselineskip]}\spacefill\mbox{F.}\pend
           \leftskip=0em{}
\pstart
           \noindent{}Wien XVIII, Währinger-Gürtel 154 part. Th. 9\oindex{Wien@\textbf{Wien}!IX., Alsergrund@\textbf{IX., Alsergrund}!Währinger Gürtel@\textbf{Währinger Gürtel}, \emph{Straße}|pw}\oindex{Wien@\textbf{Wien}!XVIII., Währing@\textbf{XVIII., Währing}!Währinger Gürtel@\textbf{Währinger Gürtel}, \emph{Straße}|pw}\pend
           \selectlanguage{ngerman}\endnumbering\briefempfaengerindex{Schnitzler, Arthur@\textsc{Schnitzler, Arthur}!zzzFels, Friedrich Michael@\emph{von Friedrich Michael Fels}!1895-05-201@{20. 5. 1895}|)be}\mylabel{L00444h}  \newcommand{\dateiname}{L00444}\newcommand{\titel}{Friedrich M. Fels an Arthur Schnitzler, 20. 5. 1895}\newcommand{\editorInnen}{Martin Anton Müller und Gerd-Hermann Susen}%% latex-leseansicht-abspann.tex
%% Abspann für die Leseansicht.
%% Der Schalter \ifkorrekturansicht ist bereits durch den Vorspann gesetzt.

%% latex-abspann.tex
%% Gemeinsamer Abspann für Korrekturansicht und Leseansicht.
%% Setzt den Schalter \ifkorrekturansicht voraus (gesetzt in den
%% einbindenden Dateien latex-korrekturansicht-abspann.tex bzw.
%% latex-leseansicht-abspann.tex).
%% ---------------------------------------------------------------

\normalsize

% Das esempio-Environment wird nur in der Leseansicht benötigt
\ifkorrekturansicht\else
\newenvironment{esempio}[3]%
{
    \vspace{1.5ex}
    \rlap{\underline{#1}}
    \par
    \setlength{\parindent}{0cm}
    \nopagebreak
    \leftskip=#2cm
    \rightskip=#3cm
}
{
    \par
}
\fi

\doendnotes{C}
\bigskip
\vfill

\clearpage

\footnotesize

\ifkorrekturansicht
  \lohead{\textsc{register}}
\fi

% theindex-Environment neu definieren ohne reledmac
\makeatletter
\renewenvironment{theindex}{%
  \ifkorrekturansicht
    \section*{\indexname}%
  \else
    \subsubsection*{Index der erwähnten Entitäten}%
  \fi
  \setlength{\parindent}{0pt}%
  \setlength{\parskip}{0pt plus 0.3pt}%
  \let\item\@idxitem
}{%
  \ifkorrekturansicht\clearpage\fi
}
\makeatother

\IfFileExists{\jobname-pw.ind}{\input{\jobname-pw.ind}}{}

% Quellenangabe nur in der Leseansicht
\ifkorrekturansicht\else
% Fallback-Definitionen, falls die .tex-Datei \titel etc. nicht gesetzt hat
\providecommand{\titel}{}
\providecommand{\editorInnen}{}
\providecommand{\dateiname}{\jobname}

\vspace{3cm}

\vfill

\footnotesize
\textsc{Quelle}: \titel. Herausgegeben von {\editorInnen}. In: \emph{Arthur Schnitzler: Briefwechsel mit Autorinnen und Autoren}.
 Digitale Edition, https://schnitzler-briefe.acdh.oeaw.ac.at/{\dateiname}.html (Stand \today)
\fi

\end{document}


