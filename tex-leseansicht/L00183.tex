%% latex-leseansicht-vorspann.tex
%% Vorspann für die Leseansicht.
%% Lädt die gemeinsame Datei latex-vorspann.tex mit nicht gesetztem Schalter.

\newif\ifkorrekturansicht
\korrekturansichtfalse

\input{../tex-inputs/latex-vorspann}


         
         \newcommand{\erwaehntePersonen}{Personen: Richard Beer-Hofmann, Carl Busse, Felix Dörmann, Georg Fischer, Robert Fischer, Gerhart Hauptmann, Felix Salten, Richard Specht, Leo Van-Jung, Boris Van-Jung}
         \newcommand{\erwaehnteInstitutionen}{Institutionen: Freie Bühne, Lessing-Theater, Magazin für die Literatur des Auslandes}
         \newcommand{\erwaehnteOrte}{Orte: Berlin, Café Griensteidl, Grillparzerstraße, I., Innere Stadt, Neues Theater, Schultheiß, Waldemarstraße, Wien}
         \newcommand{\erwaehnteWerke}{Werke: Das Märchen. Schauspiel in drei Aufzügen, Die Weber. Schauspiel aus den vierziger Jahren, Magazin für die Literatur des Auslandes, Wiener Lyriker}
               \section[Karl Kraus an Arthur Schnitzler, 27. 2. 1893]{ Karl Kraus an Arthur Schnitzler, 27. 2. 1893}\nopagebreak\mylabel{v}\rehead{ }\begin{ledgroupsized}[t]{13cm}\normalsize\beginnumbering \toendnotes[C]{\smallbreak\pagebreak[2]} \Standort{CUL, Schnitzler, B 55.}
\physDesc{Postkarte
\newline{}Handschrift: Bleistift, deutsche Kurrent\newline{}Versand: 1) Stempel: »\nobreak{}\oindex{Berlin@\textbf{Berlin}|pwk}Berlin. N.W. 66, 27/02 93, 3–4 N\nobreak{}«.   2) Stempel: »\nobreak{}Wien 1/1, 28. 2. 93, 5–6½ N\nobreak{}«. }\buchAbdrucke{\weitereDrucke{\emph{Karl Kraus und Arthur Schnitzler. Eine Dokumentation.} Hg. Reinhard Urbach. In: \emph{Literatur und Kritik}, Bd. 49, Oktober 1970, S. 515.} }\toendnotes[C]{\smallbreak}\pstart{}{\pb}Herrn Schriftſteller\pend{}\pstart{}D\textsuperscript{r} Arthur Schnitzler,\pend{}\pstart{}Wien I\oindex{I., Innere Stadt@\textbf{I., Innere Stadt}|pw}\pend{}\pstart{}Grillparzerſtr 7\oindex{Grillparzerstrasse@\textbf{Grillparzerstraße}|pw}\pend{}{\bigskip}\pstart
           {\pb}Berlin\oindex{Berlin@\textbf{Berlin}|pw}, Montag, \textcolor{gray}{2}7/2 93, Restaurant Schultheiß\oindex{Schultheiss@\textbf{Schultheiß}|pw}.\pend
           \pstart
           Liebſter Doctor! Mir geht’s hier famos! Geſtern war Matinée im
                        »Neuen Theater\oindex{Neues Theater@\textbf{Neues Theater}|pw}«: »Freie Bühne\orgindex{Freie Buehne@Freie Bühne|pw}« – \uuline{Weber}\pwindex{Hauptmann, Gerhart 15.11.1862 – 06.06.1946@\textsc{Hauptmann, Gerhart} (15.11.1862 – 06.06.1946), \emph{Schriftsteller}!Weber. Schauspiel aus den vierziger Jahren1892@\strich\emph{Die Weber. Schauspiel aus den vierziger Jahren} {[}1892{]}|pw}! \uuline{Colossaler} Erfolg. Hauptmann\pwindex{Hauptmann, Gerhart 15.11.1862 – 06.06.1946@\textsc{Hauptmann, Gerhart} (15.11.1862 – 06.06.1946), \emph{Schriftsteller}|pw} war ganz glückſeelig. Im »Magazin\orgindex{Magazin fuer die Literatur des Auslandes@Magazin für die Literatur des Auslandes|pw}« (25. Feber) iſt von mir ein
                        \label{K_L00183_1v}\edtext{Artikel\pwindex{Kraus, Karl 28.04.1874 – 12.06.1936@\textsc{Kraus, Karl} (28.04.1874 – 12.06.1936), \emph{Schriftsteller, Publizist}!Wiener Lyriker25. 02. 1893@\strich\emph{Wiener Lyriker} {[}25. 02. 1893{]}|pwv}}{\lemma{\textnormal{\emph{Artikel}}}\Cendnote{\textnormal{Karl Kraus\pwindex{Kraus, Karl 28.04.1874 – 12.06.1936@\textsc{Kraus, Karl} (28.04.1874 – 12.06.1936), \emph{Schriftsteller, Publizist}|pwk}: \emph{Wiener Lyriker. »Sensationen« von Felix Dörmann (Wien: L. Weiß) und »Gedichte« von Richard
                        Specht (München: Seitz {\kaufmannsund}
                        Schauer)}\pwindex{Kraus, Karl 28.04.1874 – 12.06.1936@\textsc{Kraus, Karl} (28.04.1874 – 12.06.1936), \emph{Schriftsteller, Publizist}!Wiener Lyriker25. 02. 1893@\strich\emph{Wiener Lyriker} {[}25. 02. 1893{]}|pwk}. In: \emph{Das Magazin für
                                Litteratur}\pwindex{?? Werk@Nicht ermittelte Verfasserinnen und Verfasser!Magazin fuer die Literatur des Auslandes1832 – 1915@\emph{Magazin für die Literatur des Auslandes} {[}1832 – 1915{]}|pwk}, Jg. 62, Nr. 8, 25. 1. 1893,
                            S. 128.}}}\label{K_L00183_1h} über Dörmann\pwindex{Doermann, Felix 29.05.1870 – 26.10.1928@\textsc{Dörmann, Felix} (29.05.1870 – 26.10.1928), \emph{Schriftsteller}|pw} und Specht\pwindex{Specht, Richard 07.12.1870 – 18.03.1932@\textsc{Specht, Richard} (07.12.1870 – 18.03.1932), \emph{Schriftsteller, Journalist, Kritiker}|pw}. Jetzt geh ich mir das Honorar
                    eincaſſieren.\pend
           \pstart
           Ach, in Berlin\oindex{Berlin@\textbf{Berlin}|pw} ist’s herrlich!! Grüßen Sie mir den \uline{Salten}\pwindex{Salten, Felix 06.09.1869 – 08.10.1945@\textsc{Salten, Felix} (06.09.1869 – 08.10.1945), \emph{Schriftsteller, Journalist}|pw} u D\textsuperscript{r}{ }\uline{Beer-Hofmann}\pwindex{Beer-Hofmann, Richard 1866-07-11 – 1945-09-26@\textsc{Beer-Hofmann, Richard} (1866-07-11 – 1945-09-26), \emph{Schriftsteller}|pw}; Dörmann\pwindex{Doermann, Felix 29.05.1870 – 26.10.1928@\textsc{Dörmann, Felix} (29.05.1870 – 26.10.1928), \emph{Schriftsteller}|pw}, Fannjungs\pwindex{Van-Jung, Leo 15.10.1866 – 02.07.1939@\textsc{Van-Jung, Leo} (15.10.1866 – 02.07.1939), \emph{Gesangspädagoge, Mathematiker}|pw}\pwindex{Van-Jung, Boris 15.10.1872 – 03.10.1899@\textsc{Van-Jung, Boris} (15.10.1872 – 03.10.1899), \emph{Mediziner}|pw}, Fiſcher\pwindex{Fischer, Georg 11.01.1866 – 17.01.1934@\textsc{Fischer, Georg} (11.01.1866 – 17.01.1934)|pwu}\pwindex{Fischer, Robert 07.10.1860 – 27.05.1939@\textsc{Fischer, Robert} (07.10.1860 – 27.05.1939), \emph{Rechtsanwalt}|pwu} etc. ganz
                        Grienſteidl\oindex{Cafe Griensteidl@\textbf{Café Griensteidl}|pw}. Ja, wenn ich hier Ihr
                        »\uline{Märchen}\pwindex{Schnitzler, Arthur 15.05.1862 – 21.10.1931@\textsc{Schnitzler, Arthur} (15.05.1862 – 21.10.1931), \emph{Schriftsteller, Mediziner}!Maerchen. Schauspiel in drei Aufzuegen1893-12-01@\strich\emph{Das Märchen. Schauspiel in drei Aufzügen} {[}1893-12-01{]}|pw}« im Leſſingtheater\orgindex{Lessing-Theater@Lessing-Theater|pw}{ }ſehen könnte!
                    Viele Grüße\pend
           \pstart Ihr \spacefill\mbox{Karl Kraus}\pend{}\pstart
           \noindent{}\strikeout{p. A.}{ }Berlin S. O. Waldemarstr 3\oindex{Waldemarstrasse@\textbf{Waldemarstraße}|pw}/\textsuperscript{II} p. A. Carl
                            Buſſe\pwindex{Busse, Carl 12.11.1872 – 04.12.1918@\textsc{Busse, Carl} (12.11.1872 – 04.12.1918), \emph{Schriftsteller}|pw}. Schreiben Sie bald!\pend
           
         
         \endnumbering\mylabel{h}\end{ledgroupsized}  \newcommand{\dateiname}{L00183}\newcommand{\titel}{Karl Kraus an Arthur Schnitzler, 27. 2. 1893}\newcommand{\editorInnen}{Martin Anton Müller und Gerd-Hermann Susen}%% latex-leseansicht-abspann.tex
%% Abspann für die Leseansicht.
%% Der Schalter \ifkorrekturansicht ist bereits durch den Vorspann gesetzt.

%% latex-abspann.tex
%% Gemeinsamer Abspann für Korrekturansicht und Leseansicht.
%% Setzt den Schalter \ifkorrekturansicht voraus (gesetzt in den
%% einbindenden Dateien latex-korrekturansicht-abspann.tex bzw.
%% latex-leseansicht-abspann.tex).
%% ---------------------------------------------------------------

\normalsize

% Das esempio-Environment wird nur in der Leseansicht benötigt
\ifkorrekturansicht\else
\newenvironment{esempio}[3]%
{
    \vspace{1.5ex}
    \rlap{\underline{#1}}
    \par
    \setlength{\parindent}{0cm}
    \nopagebreak
    \leftskip=#2cm
    \rightskip=#3cm
}
{
    \par
}
\fi

\doendnotes{C}
\bigskip
\vfill

\clearpage

\footnotesize

\ifkorrekturansicht
  \lohead{\textsc{register}}
\fi

% theindex-Environment neu definieren ohne reledmac
\makeatletter
\renewenvironment{theindex}{%
  \ifkorrekturansicht
    \section*{\indexname}%
  \else
    \subsubsection*{Index der erwähnten Entitäten}%
  \fi
  \setlength{\parindent}{0pt}%
  \setlength{\parskip}{0pt plus 0.3pt}%
  \let\item\@idxitem
}{%
  \ifkorrekturansicht\clearpage\fi
}
\makeatother

\IfFileExists{\jobname-pw.ind}{\input{\jobname-pw.ind}}{}

% Quellenangabe nur in der Leseansicht
\ifkorrekturansicht\else
% Fallback-Definitionen, falls die .tex-Datei \titel etc. nicht gesetzt hat
\providecommand{\titel}{}
\providecommand{\editorInnen}{}
\providecommand{\dateiname}{\jobname}

\vspace{3cm}

\vfill

\footnotesize
\textsc{Quelle}: \titel. Herausgegeben von {\editorInnen}. In: \emph{Arthur Schnitzler: Briefwechsel mit Autorinnen und Autoren}.
 Digitale Edition, https://schnitzler-briefe.acdh.oeaw.ac.at/{\dateiname}.html (Stand \today)
\fi

\end{document}


      