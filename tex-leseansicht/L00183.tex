%% latex-leseansicht-vorspann.tex
%% Vorspann für die Leseansicht.
%% Lädt die gemeinsame Datei latex-vorspann.tex mit nicht gesetztem Schalter.

\newif\ifkorrekturansicht
\korrekturansichtfalse

\input{../tex-inputs/latex-vorspann}


\section[Karl Kraus an Arthur Schnitzler, 27. 2. 1893]{L00183 Karl Kraus an Arthur Schnitzler, 27. 2. 1893}
\nopagebreak\mylabel{L00183v}
\rehead{ }\normalsize\beginnumbering\briefempfaengerindex{Schnitzler, Arthur@\textsc{Schnitzler, Arthur}!zzzKraus, Karl@\emph{von Karl Kraus}!1893-02-271@{27. 2. 1893}|(be}
\toendnotes[C]{\smallbreak\pagebreak[2]}
\correspDesc{Versand  durch Karl Kraus am 27. 2. 1893 in Berlin
\newline{}Erhalt  durch Arthur Schnitzler im Zeitraum [28. 2. 1893
                  – 4. 3. 1893?] in Wien}\toendnotes[C]{\smallbreak}
\Standort{CUL, Schnitzler, B 55.}
\physDesc{Postkarte, 643 Zeichen
\newline{}Handschrift: Bleistift, deutsche Kurrent
\newline{}Versand: 1) Stempel: »\nobreak{}\oindex{Berlin@\textbf{Berlin}, \emph{Hauptstadt}|pwk}Berlin. N.W. 66, 27/02 93, 3–4 N\nobreak{}«.   2) Stempel: »\nobreak{}\oindex{Wien@\textbf{Wien}, \emph{Verwaltungsgebiet}|pwk}Wien 1/1, 28. 2. 93, 5–6½ N\nobreak{}«. }
\buchAbdrucke{\weitereDrucke{\emph{Karl Kraus und Arthur Schnitzler. Eine Dokumentation.}Herausgegeben von Reinhard Urbach In: \emph{Literatur und Kritik}, Bd. 49, Oktober 1970, S. 515.} }\toendnotes[C]{\smallbreak}\pstart{}{\pb}Herrn Schriftſteller\pend{}\pstart{}D\textsuperscript{r} Arthur Schnitzler,\pend{}\pstart{}Wien I\oindex{I., Innere Stadt@\textbf{I., Innere Stadt}, \emph{Verwaltungsgebiet}|pw}\pend{}\pstart{}Grillparzerſtr 7\oindex{Wien@\textbf{Wien}!I., Innere Stadt@\textbf{I., Innere Stadt}!Grillparzerstraße@\textbf{Grillparzerstraße}, \emph{Straße}|pw}\pend{}{\bigskip}\vspace{1em}
\pstart
           {\pb}Berlin\oindex{Berlin@\textbf{Berlin}, \emph{Hauptstadt}|pw}, Montag,
                     \textcolor{gray}{2}7/2 93, Restaurant
                     Schultheiß\oindex{Schultheiß@\textbf{Schultheiß}, \emph{Gastgewerbegebäude}|pw}.\pend
           \vspace{0.5em}
\pstart
           Liebſter Doctor! Mir geht’s hier famos! Geſtern war Matinée im »Neuen Theater\oindex{Neues Theater@\textbf{Neues Theater}, \emph{Theater}|pw}«: »Freie Bühne\orgindex{Freie Bühne@Freie Bühne|pw}« – \uuline{Weber}\pwindex{Hauptmann, Gerhart 15.\,11.\,1862 Szczawno-Zdrój – 6.\,6.\,1946 Jagniątków@\textsc{Hauptmann, Gerhart} (15.\,11.\,1862 Szczawno-Zdrój – 6.\,6.\,1946 Jagniątków), \emph{Schriftsteller}!Weber. Schauspiel aus den vierziger Jahren@\strich\emph{Die Weber. Schauspiel aus den vierziger Jahren}|pw}! \uuline{Colossaler} Erfolg. Hauptmann\pwindex{Hauptmann, Gerhart 15.\,11.\,1862 Szczawno-Zdrój – 6.\,6.\,1946 Jagniątków@\textsc{Hauptmann, Gerhart} (15.\,11.\,1862 Szczawno-Zdrój – 6.\,6.\,1946 Jagniątków), \emph{Schriftsteller}|pw} war ganz glückſeelig. Im »Magazin\orgindex{Magazin für die Literatur des Auslandes@Magazin für die Literatur des Auslandes|pw}« (25. Feber) iſt von mir ein \label{K_L00183-1v}\edtext{Artikel\pwindex{Kraus, Karl 28.\,4.\,1874 Jičín – 12.\,6.\,1936 Wien@\textsc{Kraus, Karl} (28.\,4.\,1874 Jičín – 12.\,6.\,1936 Wien), \emph{Schriftsteller, Publizist, Schriftsteller}!Wiener Lyriker@\strich\emph{Wiener Lyriker}|pwv}}{\lemma{\textnormal{\emph{Artikel}}}\Cendnote{\textnormal{Karl Kraus\pwindex{Kraus, Karl 28.\,4.\,1874 Jičín – 12.\,6.\,1936 Wien@\textsc{Kraus, Karl} (28.\,4.\,1874 Jičín – 12.\,6.\,1936 Wien), \emph{Schriftsteller, Publizist, Schriftsteller}|pwk}: \emph{Wiener Lyriker. »Sensationen« von Felix Dörmann (Wien: L. Weiß)
                        und »Gedichte« von Richard Specht (München: Seitz {\kaufmannsund} Schauer)}\pwindex{Kraus, Karl 28.\,4.\,1874 Jičín – 12.\,6.\,1936 Wien@\textsc{Kraus, Karl} (28.\,4.\,1874 Jičín – 12.\,6.\,1936 Wien), \emph{Schriftsteller, Publizist, Schriftsteller}!Wiener Lyriker@\strich\emph{Wiener Lyriker}|pwk}. In: \emph{Das Magazin für Litteratur}\pwindex{Magazin für die Literatur des Auslandes@\emph{Magazin für die Literatur des Auslandes}|pwk}, Jg. 62, Nr. 8,
                        25. 1. 1893, S. 128.
               }}}\label{K_L00183-1} über Dörmann\pwindex{Dörmann, Felix 29.\,5.\,1870 Wien – 26.\,10.\,1928 ebd.@\textsc{Dörmann, Felix} (29.\,5.\,1870 Wien – 26.\,10.\,1928 ebd.), \emph{Schriftsteller}|pw} und Specht\pwindex{Specht, Richard 7.\,12.\,1870 Wien – 18.\,3.\,1932 ebd.@\textsc{Specht, Richard} (7.\,12.\,1870 Wien – 18.\,3.\,1932 ebd.), \emph{Schriftsteller, Journalist, Kritiker}|pw}. Jetzt geh ich mir das Honorar eincaſſieren.\pend
           
\pstart
           Ach, in Berlin\oindex{Berlin@\textbf{Berlin}, \emph{Hauptstadt}|pw} ist’s herrlich!! Grüßen Sie mir
               den \uline{Salten}\pwindex{Salten, Felix 6.\,9.\,1869 Budapest – 8.\,10.\,1945 Zürich@\textsc{Salten, Felix} (6.\,9.\,1869 Budapest – 8.\,10.\,1945 Zürich), \emph{Schriftsteller, Journalist, Chefredakteur}|pw} u D\textsuperscript{r}{ }\uline{Beer-Hofmann}\pwindex{Beer-Hofmann, Richard 11.\,7.\,1866 Wien – 26.\,9.\,1945 New York City@\textsc{Beer-Hofmann, Richard} (11.\,7.\,1866 Wien – 26.\,9.\,1945 New York City), \emph{Schriftsteller}|pw}; Dörmann\pwindex{Dörmann, Felix 29.\,5.\,1870 Wien – 26.\,10.\,1928 ebd.@\textsc{Dörmann, Felix} (29.\,5.\,1870 Wien – 26.\,10.\,1928 ebd.), \emph{Schriftsteller}|pw}, Fannjungs\pwindex{Van-Jung, Leo 15.\,10.\,1866 Odessa – 2.\,7.\,1939 Riga@\textsc{Van-Jung, Leo} (15.\,10.\,1866 Odessa – 2.\,7.\,1939 Riga), \emph{Gesangspädagoge, Mathematiker}|pw}\pwindex{Van-Jung, Boris 15.\,10.\,1872 Odessa – 3.\,10.\,1899 Wien@\textsc{Van-Jung, Boris} (15.\,10.\,1872 Odessa – 3.\,10.\,1899 Wien), \emph{Mediziner}|pw}, Fiſcher\pwindex{Fischer, Georg 11.\,1.\,1866 Lomnice – 17.\,1.\,1934 Wien@\textsc{Fischer, Georg} (11.\,1.\,1866 Lomnice – 17.\,1.\,1934 Wien)|pwu}\pwindex{Fischer, Robert 7.\,10.\,1860 Lomnice – 27.\,5.\,1939 Wien@\textsc{Fischer, Robert} (7.\,10.\,1860 Lomnice – 27.\,5.\,1939 Wien), \emph{Rechtsanwalt}|pwu} etc. ganz Grienſteidl\oindex{Wien@\textbf{Wien}!I., Innere Stadt@\textbf{I., Innere Stadt}!Café Griensteidl@\textbf{Café Griensteidl}, \emph{Kaffeehaus}|pw}. Ja, wenn ich hier Ihr »\uline{Märchen}\pwindex{Schnitzler, Arthur 15.\,5.\,1862 Wien – 21.\,10.\,1931 ebd.@\textsc{Schnitzler, Arthur} (15.\,5.\,1862 Wien – 21.\,10.\,1931 ebd.), \emph{Schriftsteller, Mediziner}!Märchen. Schauspiel in drei Aufzügen@\strich\emph{Das Märchen. Schauspiel in drei Aufzügen}|pw}« im Leſſingtheater\orgindex{Lessing-Theater@Lessing-Theater|pw}{ }ſehen könnte! Viele Grüße\pend
           \pstart Ihr \spacefill\mbox{Karl Kraus}\pend{}
\pstart
           \noindent{}\strikeout{p. A.}{ }Berlin S. O. Waldemarstr 3\oindex{Waldemarstraße@\textbf{Waldemarstraße}, \emph{Straße}|pw}/\textsuperscript{II} p. A. Carl
                     Buſſe\pwindex{Busse, Carl 12.\,11.\,1872 Międzychód – 4.\,12.\,1918 Berlin@\textsc{Busse, Carl} (12.\,11.\,1872 Międzychód – 4.\,12.\,1918 Berlin), \emph{Schriftsteller}|pw}. Schreiben Sie bald!\pend
           \selectlanguage{ngerman}\endnumbering\briefempfaengerindex{Schnitzler, Arthur@\textsc{Schnitzler, Arthur}!zzzKraus, Karl@\emph{von Karl Kraus}!1893-02-271@{27. 2. 1893}|)be}\mylabel{L00183h}  \newcommand{\dateiname}{L00183}\newcommand{\titel}{Karl Kraus an Arthur Schnitzler, 27. 2. 1893}\newcommand{\editorInnen}{Martin Anton Müller und Gerd-Hermann Susen}%% latex-leseansicht-abspann.tex
%% Abspann für die Leseansicht.
%% Der Schalter \ifkorrekturansicht ist bereits durch den Vorspann gesetzt.

%% latex-abspann.tex
%% Gemeinsamer Abspann für Korrekturansicht und Leseansicht.
%% Setzt den Schalter \ifkorrekturansicht voraus (gesetzt in den
%% einbindenden Dateien latex-korrekturansicht-abspann.tex bzw.
%% latex-leseansicht-abspann.tex).
%% ---------------------------------------------------------------

\normalsize

% Das esempio-Environment wird nur in der Leseansicht benötigt
\ifkorrekturansicht\else
\newenvironment{esempio}[3]%
{
    \vspace{1.5ex}
    \rlap{\underline{#1}}
    \par
    \setlength{\parindent}{0cm}
    \nopagebreak
    \leftskip=#2cm
    \rightskip=#3cm
}
{
    \par
}
\fi

\doendnotes{C}
\bigskip
\vfill

\clearpage

\footnotesize

\ifkorrekturansicht
  \lohead{\textsc{register}}
\fi

% theindex-Environment neu definieren ohne reledmac
\makeatletter
\renewenvironment{theindex}{%
  \ifkorrekturansicht
    \section*{\indexname}%
  \else
    \subsubsection*{Index der erwähnten Entitäten}%
  \fi
  \setlength{\parindent}{0pt}%
  \setlength{\parskip}{0pt plus 0.3pt}%
  \let\item\@idxitem
}{%
  \ifkorrekturansicht\clearpage\fi
}
\makeatother

\IfFileExists{\jobname-pw.ind}{\input{\jobname-pw.ind}}{}

% Quellenangabe nur in der Leseansicht
\ifkorrekturansicht\else
% Fallback-Definitionen, falls die .tex-Datei \titel etc. nicht gesetzt hat
\providecommand{\titel}{}
\providecommand{\editorInnen}{}
\providecommand{\dateiname}{\jobname}

\vspace{3cm}

\vfill

\footnotesize
\textsc{Quelle}: \titel. Herausgegeben von {\editorInnen}. In: \emph{Arthur Schnitzler: Briefwechsel mit Autorinnen und Autoren}.
 Digitale Edition, https://schnitzler-briefe.acdh.oeaw.ac.at/{\dateiname}.html (Stand \today)
\fi

\end{document}


