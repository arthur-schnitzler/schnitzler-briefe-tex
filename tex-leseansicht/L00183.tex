%% latex-korrekturansicht-vorspann.tex
%% Vorspann für die Korrekturansicht.
%% Lädt die gemeinsame Datei latex-vorspann.tex mit gesetztem Schalter.

\newif\ifkorrekturansicht
\korrekturansichttrue

\input{../tex-inputs/latex-vorspann}


\section[Karl Kraus an Arthur Schnitzler, 27. 2. 1893]{L00183 Karl Kraus an Arthur Schnitzler, 27. 2. 1893}
\nopagebreak\mylabel{L00183v}
\rehead{ }\normalsize\beginnumbering\briefempfaengerindex{Schnitzler, Arthur@\textsc{Schnitzler, Arthur}!zzzKraus, Karl@\emph{von Karl Kraus}!1893-02-271@{27. 2. 1893}|(be}
\toendnotes[C]{\smallbreak\pagebreak[2]}\Standort{CUL, Schnitzler, B 55.}
\physDesc{Postkarte, 643 Zeichen
\newline{}Handschrift: Bleistift, deutsche Kurrent
\newline{}Versand: 1) Stempel: »\nobreak{}\oindex{Berlin@\textbf{Berlin}, \emph{P.PPLC}|pwk}Berlin. N.W. 66, 27/02 93, 3–4 N\nobreak{}«.   2) Stempel: »\nobreak{}Wien 1/1, 28. 2. 93, 5–6½ N\nobreak{}«. }
\buchAbdrucke{\weitereDrucke{\emph{Literatur und Kritik}, Bd. 49, Oktober 1970, S. 515.} }\toendnotes[C]{\smallbreak}\pstart{}{\pb}Herrn Schriftſteller\pend{}\pstart{}D\textsuperscript{r} Arthur Schnitzler,\pend{}\pstart{}Wien I\oindex{I., Innere Stadt@\textbf{I., Innere Stadt}, \emph{A.ADM3}|pw}\pend{}\pstart{}Grillparzerſtr 7\oindex{Grillparzerstrasse@\textbf{Grillparzerstraße}, \emph{R.ST}|pw}\pend{}{\bigskip}\vspace{1em}
\pstart
           {\pb}Berlin\oindex{Berlin@\textbf{Berlin}, \emph{P.PPLC}|pw}, Montag,
                     \textcolor{gray}{2}7/2 93, Restaurant
                     Schultheiß\oindex{Schultheiss@\textbf{Schultheiß}, \emph{Gastgewerbegebäude (K.GGW)}|pw}.\pend
           \vspace{0.5em}
\pstart
           Liebſter Doctor! Mir geht’s hier famos! Geſtern war Matinée im »Neuen Theater\oindex{Neues Theater@\textbf{Neues Theater}, \emph{Theater (K.THE)}|pw}«: »Freie Bühne\orgindex{Freie Buehne@Freie Bühne|pw}« – \uuline{Weber}\pwindex{Weber. Schauspiel aus den vierziger Jahren@\emph{Die Weber. Schauspiel aus den vierziger Jahren}|pw}! \uuline{Colossaler} Erfolg. Hauptmann\pwindex{Hauptmann, Gerhart 15.11.1862 – 06.06.1946@\textsc{Hauptmann, Gerhart} (15.11.1862 – 06.06.1946), \emph{Schriftsteller/Schriftstellerin}|pw} war ganz glückſeelig. Im »Magazin\orgindex{Magazin fuer die Literatur des Auslandes@Magazin für die Literatur des Auslandes|pw}« (25. Feber) iſt von mir ein \label{K_L00183-1v}\edtext{Artikel\pwindex{Wiener Lyriker@\emph{Wiener Lyriker}|pwv}}{\lemma{\textnormal{\emph{Artikel}}}\Cendnote{\textnormal{Karl Kraus\pwindex{Kraus, Karl 28.04.1874 – 12.06.1936@\textsc{Kraus, Karl} (28.04.1874 – 12.06.1936), \emph{Schriftsteller/Schriftstellerin, Publizist/Publizistin, Schriftsteller/Schriftstellerin}|pwk}: \emph{Wiener Lyriker. »Sensationen« von Felix Dörmann (Wien: L. Weiß)
                        und »Gedichte« von Richard Specht (München: Seitz {\kaufmannsund} Schauer)}\pwindex{Wiener Lyriker@\emph{Wiener Lyriker}|pwk}. In: \emph{Das Magazin für Litteratur}\pwindex{Magazin fuer die Literatur des Auslandes@\emph{Magazin für die Literatur des Auslandes}|pwk}, Jg. 62, Nr. 8,
                        25. 1. 1893, S. 128.
               }}}\label{K_L00183-1} über Dörmann\pwindex{Doermann, Felix 29.05.1870 – 26.10.1928@\textsc{Dörmann, Felix} (29.05.1870 – 26.10.1928), \emph{Schriftsteller/Schriftstellerin}|pw} und Specht\pwindex{Specht, Richard 07.12.1870 – 18.03.1932@\textsc{Specht, Richard} (07.12.1870 – 18.03.1932), \emph{Schriftsteller/Schriftstellerin, Journalist/Journalistin, Kritiker/Kritikerin}|pw}. Jetzt geh ich mir das Honorar eincaſſieren.\pend
           
\pstart
           Ach, in Berlin\oindex{Berlin@\textbf{Berlin}, \emph{P.PPLC}|pw} ist’s herrlich!! Grüßen Sie mir
               den \uline{Salten}\pwindex{Salten, Felix 06.09.1869 – 08.10.1945@\textsc{Salten, Felix} (06.09.1869 – 08.10.1945), \emph{Schriftsteller/Schriftstellerin, Journalist/Journalistin, Chefredakteur/Chefredakteurin}|pw} u D\textsuperscript{r}{ }\uline{Beer-Hofmann}\pwindex{Beer-Hofmann, Richard 1866-07-11 – 1945-09-26@\textsc{Beer-Hofmann, Richard} (1866-07-11 – 1945-09-26), \emph{Schriftsteller/Schriftstellerin}|pw}; Dörmann\pwindex{Doermann, Felix 29.05.1870 – 26.10.1928@\textsc{Dörmann, Felix} (29.05.1870 – 26.10.1928), \emph{Schriftsteller/Schriftstellerin}|pw}, Fannjungs\pwindex{Van-Jung, Leo 15.10.1866 – 02.07.1939@\textsc{Van-Jung, Leo} (15.10.1866 – 02.07.1939), \emph{Gesangspädagoge/Gesangspädagogin, Mathematiker/Mathematikerin}|pw}\pwindex{Van-Jung, Boris 15.10.1872 – 03.10.1899@\textsc{Van-Jung, Boris} (15.10.1872 – 03.10.1899), \emph{Mediziner/Medizinerin}|pw}, Fiſcher\pwindex{Fischer, Georg 11.01.1866 – 17.01.1934@\textsc{Fischer, Georg} (11.01.1866 – 17.01.1934)|pwu}\pwindex{Fischer, Robert 07.10.1860 – 27.05.1939@\textsc{Fischer, Robert} (07.10.1860 – 27.05.1939), \emph{Rechtsanwalt/Rechtsanwältin}|pwu} etc. ganz Grienſteidl\oindex{Cafe Griensteidl@\textbf{Café Griensteidl}, \emph{Kaffeehaus (K.KAF)}|pw}. Ja, wenn ich hier Ihr »\uline{Märchen}\pwindex{Maerchen. Schauspiel in drei Aufzuegen@\emph{Das Märchen. Schauspiel in drei Aufzügen}|pw}« im Leſſingtheater\orgindex{Lessing-Theater@Lessing-Theater|pw}{ }ſehen könnte! Viele Grüße\pend
           \pstart Ihr \spacefill\mbox{Karl Kraus}\pend{}
\pstart
           \noindent{}\strikeout{p. A.}{ }Berlin S. O. Waldemarstr 3\oindex{Waldemarstrasse@\textbf{Waldemarstraße}, \emph{Straße (K.STR)}|pw}/\textsuperscript{II} p. A. Carl
                     Buſſe\pwindex{Busse, Carl 12.11.1872 – 04.12.1918@\textsc{Busse, Carl} (12.11.1872 – 04.12.1918), \emph{Schriftsteller/Schriftstellerin}|pw}. Schreiben Sie bald!\pend
           \selectlanguage{ngerman}\endnumbering\briefempfaengerindex{Schnitzler, Arthur@\textsc{Schnitzler, Arthur}!zzzKraus, Karl@\emph{von Karl Kraus}!1893-02-271@{27. 2. 1893}|)be}\mylabel{L00183h}  \normalsize

\doendnotes{C}
\bigskip
\vfill

\clearpage

\footnotesize

\lohead{\textsc{register}}

% Definiere theindex-Environment komplett neu ohne reledmac
\makeatletter
\renewenvironment{theindex}{%
  \section*{\indexname}%
  \setlength{\parindent}{0pt}%
  \setlength{\parskip}{0pt plus 0.3pt}%
  \let\item\@idxitem
}{%
  \clearpage
}
\makeatother

\IfFileExists{\jobname-pw.ind}{\input{\jobname-pw.ind}}{}

\end{document}

      