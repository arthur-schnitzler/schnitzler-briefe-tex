%% latex-leseansicht-vorspann.tex
%% Vorspann für die Leseansicht.
%% Lädt die gemeinsame Datei latex-vorspann.tex mit nicht gesetztem Schalter.

\newif\ifkorrekturansicht
\korrekturansichtfalse

\input{../tex-inputs/latex-vorspann}

\begin{center}
            \textcolor{red}{ENTWURF. ENTZIFFERUNG NOCH NICHT KORREKTURGELESEN}
                      \end{center}
            
               \section[Hugo von Hofmannsthal an Arthur Schnitzler mit Beilage Christiane Thun an Hofmannsthal, {[}25. 5. 1907{]}]{ Hugo von Hofmannsthal an Arthur Schnitzler mit Beilage Christiane Thun
               an Hofmannsthal, {[}25. 5. 1907{]}}\nopagebreak\mylabel{v}\rehead{ }\begin{ledgroupsized}[t]{13cm}\normalsize\beginnumbering\briefempfaengerindex{Schnitzler, Arthur@\textsc{Schnitzler, Arthur}!zzzHofmannsthal, Hugo von@\emph{von Hugo von Hofmannsthal}!1907-05-251@{{[}25. 5. 1907{]}}|(be} \toendnotes[C]{\smallbreak\pagebreak[2]} \Standort{CUL, Schnitzler, B 43.}
\physDesc{Brief, 1 Blatt, 4 Seiten
\newline{}Handschrift: schwarze Tinte, deutsche Kurrent\newline{}Beilage: Christine Thun-Salm\pwindex{Thun-Hohenstein-Salm-Reifferscheidt, Christiane von 12.06.1859 – 06.08.1935@\textsc{Thun-Hohenstein-Salm-Reifferscheidt, Christiane von} (12.06.1859 – 06.08.1935), \emph{Schriftstellerin}|pw}: Briefkarte, schwarze Tinte, Lateinschrift 
\newline{}Schnitzler: mit Bleistift datiert: »25/5 907« \newline{}Ordnung: 1) mit Bleistift von unbekannter Hand nummeriert: »\strikeout{279}« 2) mit Bleistift von unbekannter Hand nummeriert: »277«}\buchAbdrucke{\weitereDrucke{Hugo von Hofmannsthal, Arthur Schnitzler: \emph{Briefwechsel}. Hg. Therese Nickl und Heinrich Schnitzler. Frankfurt am Main: \emph{S. Fischer} 1964, S. 228.} }\toendnotes[C]{\smallbreak}\pstart
           \raggedleft{}{\pb}Samstag\pend
           \pstart{}mein lieber Arthur\pend\pstart
           habe Brahm\pwindex{Brahm, Otto 05.02.1856 – 28.11.1912@\textsc{Brahm, Otto} (05.02.1856 – 28.11.1912), \emph{Theaterleiter, Regisseur}|pw} das Original vorgewieſen:
               2975 Mark. Er bezahlt. Reiſe heute Abend, zunächſt \textsc{Ravenna}\oindex{Ravenna@\textbf{Ravenna}|pw}, dann \textsc{Umbrien}\oindex{Umbrien@\textbf{Umbrien}|pw}. Hoffe ich finde Sie noch in Wien\oindex{Wien@\textbf{Wien}|pw} oder nahe Wien\oindex{Wien@\textbf{Wien}|pw} gegen
                     10\textsuperscript{ten} July. Ich empfinde es \uline{ſehr}{ }ſchmerzlich wie
               ſelten man ſich sieht. –\pend
           \pstart
           {\pb}Schicke Ihnen dieſen Brief der
               Gräfin Thun\pwindex{Thun-Hohenstein-Salm-Reifferscheidt, Christiane von 12.06.1859 – 06.08.1935@\textsc{Thun-Hohenstein-Salm-Reifferscheidt, Christiane von} (12.06.1859 – 06.08.1935), \emph{Schriftstellerin}|pw}, geſchrieben noch nachdem ſie mir
               damals Adieu (für immer) geſagt hatte, weil es Sie wahrſcheinlich freuen wird, wie
               herzlich ſie in einem ſolchen Moment des letzten Überblicks Ihrer gedenkt.\hspace*{1.5em}Wenn ſie davon kommt – es {\pb}ſcheint Hoffnung zu ſein –
               trotzdem die Operation \uline{ſehr}{ }\uline{ſchwer} war – ſo beſuchen Sie ſie
               vielleicht im Sanatorium\oindex{Sanatorium Loew@\textbf{Sanatorium Loew}|pwv}, oder
               ſchicken ihr vielleicht die Dä{\geminationm}erſeelen\pwindex{Schnitzler, Arthur 15.05.1862 – 21.10.1931@\textsc{Schnitzler, Arthur} (15.05.1862 – 21.10.1931), \emph{Schriftsteller, Mediziner}!Daemmerseelen. Novellen1907@\strich\emph{Dämmerseelen. Novellen} {[}1907{]}|pw}, die ſie noch nicht kennt.\pend
           \pstart
           Adieu. Ich freue mich von Herzen auf den Roman\pwindex{Schnitzler, Arthur 15.05.1862 – 21.10.1931@\textsc{Schnitzler, Arthur} (15.05.1862 – 21.10.1931), \emph{Schriftsteller, Mediziner}!Weg ins Freie. Roman1.1.1908 – 1.6.1908@\strich\emph{Der Weg ins Freie. Roman} {[}1.1.1908 – 1.6.1908{]}|pwv}, das Stück\pwindex{Schnitzler, Arthur 15.05.1862 – 21.10.1931@\textsc{Schnitzler, Arthur} (15.05.1862 – 21.10.1931), \emph{Schriftsteller, Mediziner}!Wort. Tragikomoedie in fuenf Akten1966@\strich\emph{Das Wort. Tragikomödie in fünf Akten} {[}1966{]}|pwv}, auf
               alles was Sie machen. {\pb}Denn ich
               habe noch nie eines Ihrer Bücher ohne tiefe Mitfreude wieder in die Hand geno{\geminationm}en.\pend
           \pstart
           Adieu.{\\[\baselineskip]}Ihr\spacefill\mbox{Hugo.}\pend
           \leftskip=0em{}{\bigskip}\pstart
           \raggedleft{}{\pb}{[}hs. Thun-Hohenstein-Salm-Reifferscheidt:{]} 21. 5. 1907\pend
           \pstart
           \raggedleft{}Wien, Sanatorium Löw\oindex{Sanatorium Loew@\textbf{Sanatorium Loew}|pw}.\pend
           \pstart
           Ich habe mich sehr gefreut, Sie heute noch zu sehen. Nachdem Sie bei mir waren, bin
               ich ins Sanatorium\oindex{Sanatorium Loew@\textbf{Sanatorium Loew}|pwv} gefahren. Es
               scheint hier sehr voll zu sein, {\kaufmannsund} ich habe ein
               Schandloch auf die Gasse hinaus. –\pend
           \pstart
           Im besten Fall 4 Wochen hier zu sitzen ist eine abscheuliche Aussicht!\pend
           \pstart
           Leben Sie wohl! Sagen Sie Ihrer Frau\pwindex{Hofmannsthal, Gertrude von 16.03.1880 – 09.11.1959@\textsc{Hofmannsthal, Gertrude von} (16.03.1880 – 09.11.1959)|pwv} viel Liebes von mir {\kaufmannsund} seien Sie herzlich
               von mir gegrüsst!\pend
           \pstart
           {\pb}Danke noch für alle Ihre
               Freundschaft! Ich habe auch für Sie immer sehr viel Freundschaft gehabt.\pend
           \pstart
           Möge es Ihnen gut gehen! Das wünscht Ihnen von
                  Herzen{\\[\baselineskip]}\spacefill\mbox{ChristThunSalm}\pend
           \leftskip=0em{}\pstart
           \noindent{}Wenn Sie Dtr. Arthur Schnitzler sehen, dann bitte grüssen Sie ihn herzlich von
                  mir!\pend
           \endnumbering\briefempfaengerindex{Schnitzler, Arthur@\textsc{Schnitzler, Arthur}!zzzHofmannsthal, Hugo von@\emph{von Hugo von Hofmannsthal}!1907-05-251@{{[}25. 5. 1907{]}}|)be}\mylabel{h}\end{ledgroupsized}  \newcommand{\dateiname}{L01678}\newcommand{\titel}{Hugo von Hofmannsthal an Arthur Schnitzler mit Beilage Christiane Thun an Hofmannsthal, [25. 5. 1907]}\newcommand{\editorInnen}{Martin Anton Müller und Gerd-Hermann Susen}%% latex-leseansicht-abspann.tex
%% Abspann für die Leseansicht.
%% Der Schalter \ifkorrekturansicht ist bereits durch den Vorspann gesetzt.

%% latex-abspann.tex
%% Gemeinsamer Abspann für Korrekturansicht und Leseansicht.
%% Setzt den Schalter \ifkorrekturansicht voraus (gesetzt in den
%% einbindenden Dateien latex-korrekturansicht-abspann.tex bzw.
%% latex-leseansicht-abspann.tex).
%% ---------------------------------------------------------------

\normalsize

% Das esempio-Environment wird nur in der Leseansicht benötigt
\ifkorrekturansicht\else
\newenvironment{esempio}[3]%
{
    \vspace{1.5ex}
    \rlap{\underline{#1}}
    \par
    \setlength{\parindent}{0cm}
    \nopagebreak
    \leftskip=#2cm
    \rightskip=#3cm
}
{
    \par
}
\fi

\doendnotes{C}
\bigskip
\vfill

\clearpage

\footnotesize

\ifkorrekturansicht
  \lohead{\textsc{register}}
\fi

% theindex-Environment neu definieren ohne reledmac
\makeatletter
\renewenvironment{theindex}{%
  \ifkorrekturansicht
    \section*{\indexname}%
  \else
    \subsubsection*{Index der erwähnten Entitäten}%
  \fi
  \setlength{\parindent}{0pt}%
  \setlength{\parskip}{0pt plus 0.3pt}%
  \let\item\@idxitem
}{%
  \ifkorrekturansicht\clearpage\fi
}
\makeatother

\IfFileExists{\jobname-pw.ind}{\input{\jobname-pw.ind}}{}

% Quellenangabe nur in der Leseansicht
\ifkorrekturansicht\else
% Fallback-Definitionen, falls die .tex-Datei \titel etc. nicht gesetzt hat
\providecommand{\titel}{}
\providecommand{\editorInnen}{}
\providecommand{\dateiname}{\jobname}

\vspace{3cm}

\vfill

\footnotesize
\textsc{Quelle}: \titel. Herausgegeben von {\editorInnen}. In: \emph{Arthur Schnitzler: Briefwechsel mit Autorinnen und Autoren}.
 Digitale Edition, https://schnitzler-briefe.acdh.oeaw.ac.at/{\dateiname}.html (Stand \today)
\fi

\end{document}


      