%% latex-korrekturansicht-vorspann.tex
%% Vorspann für die Korrekturansicht.
%% Lädt die gemeinsame Datei latex-vorspann.tex mit gesetztem Schalter.

\newif\ifkorrekturansicht
\korrekturansichttrue

\input{../tex-inputs/latex-vorspann}


\section[Hugo von Hofmannsthal an Arthur Schnitzler mit Beilage Christiane Thun an Hofmannsthal, {[}25. 5. 1907{]}]{L01678 Hugo von Hofmannsthal an Arthur Schnitzler mit Beilage Christiane Thun
               an Hofmannsthal, {[}25. 5. 1907{]}}
\nopagebreak\mylabel{L01678v}
\rehead{ }\normalsize\beginnumbering\briefempfaengerindex{Schnitzler, Arthur@\textsc{Schnitzler, Arthur}!zzzHofmannsthal, Hugo von@\emph{von Hugo von Hofmannsthal}!1907-05-251@{{[}25. 5. 1907{]}}|(be}
\toendnotes[C]{\smallbreak\pagebreak[2]}\Standort{CUL, Schnitzler, B 43.}
\physDesc{Brief, 1 Blatt, 4 Seiten, 1512 Zeichen
\newline{}Handschrift: schwarze Tinte, deutsche Kurrent
\newline{}Beilage: Christine Thun-Salm\pwindex{Thun-Hohenstein-Salm-Reifferscheidt, Christiane von 12.06.1859 – 06.08.1935@\textsc{Thun-Hohenstein-Salm-Reifferscheidt, Christiane von} (12.06.1859 – 06.08.1935), \emph{Schriftsteller/Schriftstellerin}|pw}:
                                 Briefkarte, schwarze Tinte, Lateinschrift 
\newline{}Schnitzler: mit Bleistift datiert: »25/5 907« 
\newline{}Ordnung: 1) mit Bleistift von unbekannter Hand nummeriert: »\strikeout{279}«  2) mit Bleistift von unbekannter Hand nummeriert:
                                    »277«}
\buchAbdrucke{\weitereDrucke{Hugo von Hofmannsthal, Arthur Schnitzler: \emph{Briefwechsel}. Frankfurt am Main: \emph{S. Fischer} 1964, S. 228.} }\toendnotes[C]{\smallbreak}
\pstart
           \raggedleft{}{\pb}Samstag\pend
           
\pstart{}mein lieber Arthur\pend\vspace{0.5em}
\pstart
           habe Brahm\pwindex{Brahm, Otto 05.02.1856 – 28.11.1912@\textsc{Brahm, Otto} (05.02.1856 – 28.11.1912), \emph{Theaterleiter/Theaterleiterin, Regisseur/Regisseurin}|pw} das Original vorgewieſen:
               2975 Mark. Er bezahlt. Reiſe heute Abend, zunächſt \textsc{Ravenna}\oindex{Ravenna@\textbf{Ravenna}, \emph{P.PPLA2}|pw}, dann \textsc{Umbrien}\oindex{Umbrien@\textbf{Umbrien}, \emph{A.ADM1}|pw}. Hoffe ich finde Sie noch in Wien\oindex{Wien@\textbf{Wien}, \emph{A.ADM2}|pw} oder nahe
                  Wien\oindex{Wien@\textbf{Wien}, \emph{A.ADM2}|pw} gegen 10\textsuperscript{ten} July. Ich empfinde es \uline{ſehr}{ }ſchmerzlich wie ſelten man ſich sieht. –\pend
           
\pstart
           {\pb}Schicke Ihnen dieſen Brief der
               Gräfin Thun\pwindex{Thun-Hohenstein-Salm-Reifferscheidt, Christiane von 12.06.1859 – 06.08.1935@\textsc{Thun-Hohenstein-Salm-Reifferscheidt, Christiane von} (12.06.1859 – 06.08.1935), \emph{Schriftsteller/Schriftstellerin}|pw}, geſchrieben noch nachdem ſie mir
               damals Adieu (für immer) geſagt hatte, weil es Sie wahrſcheinlich freuen wird, wie
               herzlich ſie in einem ſolchen Moment des letzten Überblicks Ihrer gedenkt.\hspace*{1.5em}Wenn ſie davon kommt – es {\pb}ſcheint Hoffnung zu ſein –
               trotzdem die Operation \uline{ſehr}{ }\uline{ſchwer} war – ſo beſuchen Sie ſie vielleicht im Sanatorium\oindex{Sanatorium Loew@\textbf{Sanatorium Loew}, \emph{Sanatorium (K.SAN)}|pwv}, oder ſchicken ihr
               vielleicht die Dä{\geminationm}erſeelen\pwindex{Daemmerseelen. Novellen@\emph{Dämmerseelen. Novellen}|pw}, die ſie noch nicht kennt.\pend
           
\pstart
           Adieu. Ich freue mich von Herzen auf den Roman\pwindex{Weg ins Freie. Roman@\emph{Der Weg ins Freie. Roman}|pwv}, das Stück\pwindex{Wort. Tragikomoedie in fuenf Akten@\emph{Das Wort. Tragikomödie in fünf Akten}|pwv}, auf alles was Sie machen. {\pb}Denn ich habe noch nie eines Ihrer
               Bücher ohne tiefe Mitfreude wieder in die Hand geno{\geminationm}en.\pend
           
\pstart
           Adieu.{\\[\baselineskip]}Ihr\spacefill\mbox{Hugo.}\pend
           \leftskip=0em{}\selectlanguage{ngerman}\vspace{1em}{\vspace{1\baselineskip}}
\pstart
           \raggedleft{}{\pb}{[}hs. :{]} 21. 5. 1907\pend
           
\pstart
           \raggedleft{}Wien, Sanatorium Löw\oindex{Sanatorium Loew@\textbf{Sanatorium Loew}, \emph{Sanatorium (K.SAN)}|pw}.\pend
           \vspace{0.5em}
\pstart
           Ich habe mich sehr gefreut, Sie heute noch zu sehen. Nachdem Sie bei mir waren, bin
               ich ins Sanatorium\oindex{Sanatorium Loew@\textbf{Sanatorium Loew}, \emph{Sanatorium (K.SAN)}|pwv} gefahren. Es
               scheint hier sehr voll zu sein, {\kaufmannsund} ich habe ein
               Schandloch auf die Gasse hinaus. –\pend
           
\pstart
           Im besten Fall 4 Wochen hier zu sitzen ist eine abscheuliche Aussicht!\pend
           
\pstart
           Leben Sie wohl! Sagen Sie Ihrer Frau\pwindex{Hofmannsthal, Gertrude von 16.03.1880 – 09.11.1959@\textsc{Hofmannsthal, Gertrude von} (16.03.1880 – 09.11.1959)|pwv} viel Liebes von mir {\kaufmannsund} seien Sie herzlich
               von mir gegrüsst!\pend
           
\pstart
           {\pb}Danke noch für alle Ihre
               Freundschaft! Ich habe auch für Sie immer sehr viel Freundschaft gehabt.\pend
           
\pstart
           Möge es Ihnen gut gehen! Das wünscht Ihnen von Herzen{\\[\baselineskip]}\spacefill\mbox{ChristThunSalm}\pend
           \leftskip=0em{}
\pstart
           \noindent{}Wenn Sie Dtr. Arthur Schnitzler sehen, dann bitte grüssen Sie ihn herzlich von
                  mir!\pend
           \selectlanguage{ngerman}\endnumbering\briefempfaengerindex{Schnitzler, Arthur@\textsc{Schnitzler, Arthur}!zzzHofmannsthal, Hugo von@\emph{von Hugo von Hofmannsthal}!1907-05-251@{{[}25. 5. 1907{]}}|)be}\mylabel{L01678h}  \normalsize

\doendnotes{C}
\bigskip
\vfill

\clearpage

\footnotesize

\lohead{\textsc{register}}

% Definiere theindex-Environment komplett neu ohne reledmac
\makeatletter
\renewenvironment{theindex}{%
  \section*{\indexname}%
  \setlength{\parindent}{0pt}%
  \setlength{\parskip}{0pt plus 0.3pt}%
  \let\item\@idxitem
}{%
  \clearpage
}
\makeatother

\IfFileExists{\jobname-pw.ind}{\input{\jobname-pw.ind}}{}

\end{document}

      