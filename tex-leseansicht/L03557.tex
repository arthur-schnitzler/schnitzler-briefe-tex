%% latex-leseansicht-vorspann.tex
%% Vorspann für die Leseansicht.
%% Lädt die gemeinsame Datei latex-vorspann.tex mit nicht gesetztem Schalter.

\newif\ifkorrekturansicht
\korrekturansichtfalse

\input{../tex-inputs/latex-vorspann}

\begin{center}
            \textcolor{red}{ENTWURF, NICHT FERTIG KORRIGIERT}
                      \end{center}
            
         
         \renewcommand{\erwaehntePersonen}{Personen: Lili Cappellini, Paul Goldmann, Gerhart Hauptmann, Olga Schnitzler, Heinrich Schnitzler, Louise Wolff}
         \renewcommand{\erwaehnteOrte}{Orte: Berghof, Berlin, Brijuni, Dresden, Kroatien, Landshut, Leipzig, Prag, Salzkammergut, Sternwartestraße 71, Unterach am Attersee, Weimar, Wien}
         \renewcommand{\erwaehnteWerke}{Werke: Eine Gerhart Hauptmann-Première in Lauchstedt. (»Gabriel Schillings Flucht.«), Gabriel Schillings Flucht. Drama, Neue Freie Presse, Tagebuch}
               \section[ Felix Salten an Arthur Schnitzler, 2. 7. 1912]{ Felix Salten an Arthur Schnitzler, 2. 7. 1912}\nopagebreak\mylabel{v}\rehead{ }\begin{ledgroupsized}[t]{13cm}\normalsize\beginnumbering \toendnotes[C]{\smallbreak\pagebreak[2]} \Standort{CUL, Schnitzler, B 89, B 2.}
\physDesc{Bildpostkarte, 810 Zeichen
\newline{}Handschrift: schwarze Tinte, lateinische Kurrent
\newline{}Versand: Stempel: »\nobreak{}\oindex{Unterach am Attersee@\textbf{Unterach am Attersee}|pwk}Unterach
                                       \textcolor{gray}{Atterse}e a, 2. VII. 12\nobreak{}«.  
\newline{}Ordnung: mit Bleistift von unbekannter Hand nummeriert: »272« }\toendnotes[C]{\smallbreak}\pstart{}{\pb}Herrn D\textsuperscript{r} Arthur Schnitzler\pend{}\pstart{}Wien\oindex{Wien@\textbf{Wien}|pw}\pend{}\pstart{}XVIII. Sternwartestraße 71\oindex{Sternwartestrasse 71@\textbf{Sternwartestraße 71}|pw}\pend{}{\bigskip}\pstart
           \noindent{}\centering{}{\pb}\textcolor{gray}{\textbf{Salzkammergut\oindex{Salzkammergut@\textbf{Salzkammergut}|pw}. Berghof\oindex{Berghof@\textbf{Berghof}|pw} bei Unterach\oindex{Unterach am Attersee@\textbf{Unterach am Attersee}|pw}.}}\pend
           \pstart
           {\pb}Vielen Dank für die \label{K_L03557-1v}\edtext{Prag\oindex{Prag@\textbf{Prag}|pw}er Karte}{\lemma{\textnormal{\emph{Prager Karte}}}\Cendnote{\textnormal{Schnitzler\pwindex{Schnitzler, Arthur 15.05.1862 – 21.10.1931@\textsc{Schnitzler, Arthur} (15.05.1862 – 21.10.1931), \emph{Schriftsteller, Mediziner}|pwk} war am 14. 6. 1912 in Prag\oindex{Prag@\textbf{Prag}|pwk} gewesen.}}}\label{K_L03557-1h}. Ich bin vorgestern über Landshut\oindex{Landshut@\textbf{Landshut}|pw}, Leipzig\oindex{Leipzig@\textbf{Leipzig}|pw}, Weimar\oindex{Weimar@\textbf{Weimar}|pw}, Berlin\oindex{Berlin@\textbf{Berlin}|pw} u. Dresden\oindex{Dresden@\textbf{Dresden}|pw} wieder hier\oindex{Unterach am Attersee@\textbf{Unterach am Attersee}|pwv} gelandet. War drei Wochen fort, und freue mich jetzt,
               wieder hier zu sein. Wenn gehen Sie nach \label{K_L03557-2v}\edtext{Brioni\oindex{Brijuni@\textbf{Brijuni}|pw}}{\lemma{\textnormal{\emph{Brioni}}}\Cendnote{\textnormal{Schnitzler\pwindex{Schnitzler, Arthur 15.05.1862 – 21.10.1931@\textsc{Schnitzler, Arthur} (15.05.1862 – 21.10.1931), \emph{Schriftsteller, Mediziner}|pwk} kam am 21. 7. 1912 in Brijuni\oindex{Brijuni@\textbf{Brijuni}|pwk} im heutigen Kroatien\oindex{Kroatien@\textbf{Kroatien}|pwk} an und reiste am 24. 8. 1912 wieder ab.}}}\label{K_L03557-2h}? Sie haben, glaub’
               ich, sehr gut gewählt damit. Denn hier regnet es sich wieder tüchtig ein, und möchte
               ein nasser Sommer werden. Wie geht es Frau Olga\pwindex{Schnitzler, Olga 17.01.1882 – 13.01.1970@\textsc{Schnitzler, Olga} (17.01.1882 – 13.01.1970), \emph{Schauspielerin, Sängerin}|pw}
               und den Kinder\pwindex{Schnitzler, Heinrich 09.08.1902 – 12.07.1982@\textsc{Schnitzler, Heinrich} (09.08.1902 – 12.07.1982), \emph{Regisseur, Schauspieler}|pwv}\pwindex{Cappellini, Lili 13.09.1909 – 26.07.1928@\textsc{Cappellini, Lili} (13.09.1909 – 26.07.1928)|pwv}n? In
                  Berlin\oindex{Berlin@\textbf{Berlin}|pw} hörte ich, Frau Wolf\pwindex{Wolff, Louise 25.03.1855 – 25.06.1935@\textsc{Wolff, Louise} (25.03.1855 – 25.06.1935), \emph{Konzertagentin}|pwu} sei verreist gewesen, und habe durch
               Krankheitsfälle in der Familie böse Zeiten gehabt; wolle aber Ihrer Frau\pwindex{Schnitzler, Olga 17.01.1882 – 13.01.1970@\textsc{Schnitzler, Olga} (17.01.1882 – 13.01.1970), \emph{Schauspielerin, Sängerin}|pwv} nun endlich schreiben. Über Landshut\oindex{Landshut@\textbf{Landshut}|pw}{ }\textcolor{gray}{etc}. wäre viel zu erzählen. Ihrem \label{K_L03557-3v}\edtext{Urteil über das Stück\pwindex{Hauptmann, Gerhart 15.11.1862 – 06.06.1946@\textsc{Hauptmann, Gerhart} (15.11.1862 – 06.06.1946), \emph{Schriftsteller}!Gabriel Schillings Flucht. Drama01. 01. 1912@\strich\emph{Gabriel Schillings Flucht. Drama} {[}01. 01. 1912{]}|pwv}}{\lemma{\textnormal{\emph{Urteil über das Stück}}}\Cendnote{\textnormal{Salten\pwindex{Salten, Felix 06.09.1869 – 08.10.1945@\textsc{Salten, Felix} (06.09.1869 – 08.10.1945), \emph{Schriftsteller, Journalist}|pwk} dürfte sich in Folge auf die zuletzt
                  erschienene Theaterkritik von Paul Goldmann\pwindex{Goldmann, Paul 31.01.1865 – 25.09.1935@\textsc{Goldmann, Paul} (31.01.1865 – 25.09.1935), \emph{Schriftsteller, Journalist}|pwk}
                  bezogen haben, die eine Aufführung von Gerhart
                     Hauptmann\pwindex{Hauptmann, Gerhart 15.11.1862 – 06.06.1946@\textsc{Hauptmann, Gerhart} (15.11.1862 – 06.06.1946), \emph{Schriftsteller}|pwk}s \emph{Gabriel Schillings Flucht}\pwindex{Hauptmann, Gerhart 15.11.1862 – 06.06.1946@\textsc{Hauptmann, Gerhart} (15.11.1862 – 06.06.1946), \emph{Schriftsteller}!Gabriel Schillings Flucht. Drama01. 01. 1912@\strich\emph{Gabriel Schillings Flucht. Drama} {[}01. 01. 1912{]}|pwk}
                  behandelte: Paul Goldmann\pwindex{Goldmann, Paul 31.01.1865 – 25.09.1935@\textsc{Goldmann, Paul} (31.01.1865 – 25.09.1935), \emph{Schriftsteller, Journalist}|pwk}: \emph{Eine Gerhart Hauptmann-Première in Lauchstedt. (»Gabriel
                        Schillings Flucht.«)}\pwindex{Goldmann, Paul 31.01.1865 – 25.09.1935@\textsc{Goldmann, Paul} (31.01.1865 – 25.09.1935), \emph{Schriftsteller, Journalist}!Eine Gerhart Hauptmann-Premiere in Lauchstedt. (»Gabriel Schillings
                  Flucht.«)1912-06-27@\strich\emph{Eine Gerhart Hauptmann-Première in Lauchstedt. (»Gabriel Schillings Flucht.«)} {[}1912-06-27{]}|pwk}. In: \emph{Neue Freie
                        Presse}\pwindex{Neue Freie Presse1864 – 1939@\emph{Neue Freie Presse} {[}1864 – 1939{]}|pwk}, Nr. 17.185, 27. 6. 1912,
                     Morgenblatt, S. 1–4. Für den 2. 2. 1912 führt Schnitzler\pwindex{Schnitzler, Arthur 15.05.1862 – 21.10.1931@\textsc{Schnitzler, Arthur} (15.05.1862 – 21.10.1931), \emph{Schriftsteller, Mediziner}|pwk}s \emph{Tagebuch}\pwindex{\textcolor{red}{\textsuperscript{XXXX1 indx}}!Tagebuch1981 – 2000@\strich\emph{Tagebuch} {[}Hrsg., 1981 – 2000{]}|pwk}
                  eine Diskussion mit Salten\pwindex{Salten, Felix 06.09.1869 – 08.10.1945@\textsc{Salten, Felix} (06.09.1869 – 08.10.1945), \emph{Schriftsteller, Journalist}|pwk} über das Stück\pwindex{Hauptmann, Gerhart 15.11.1862 – 06.06.1946@\textsc{Hauptmann, Gerhart} (15.11.1862 – 06.06.1946), \emph{Schriftsteller}!Gabriel Schillings Flucht. Drama01. 01. 1912@\strich\emph{Gabriel Schillings Flucht. Drama} {[}01. 01. 1912{]}|pwkv} an.}}}\label{K_L03557-3h} bin ich ein
               wenig näher geko{\geminationm}en, seit ich es auf der Bühne sah. Paul Goldmann\pwindex{Goldmann, Paul 31.01.1865 – 25.09.1935@\textsc{Goldmann, Paul} (31.01.1865 – 25.09.1935), \emph{Schriftsteller, Journalist}|pw} war wieder »fein\pwindex{Goldmann, Paul 31.01.1865 – 25.09.1935@\textsc{Goldmann, Paul} (31.01.1865 – 25.09.1935), \emph{Schriftsteller, Journalist}!Eine Gerhart Hauptmann-Premiere in Lauchstedt. (»Gabriel Schillings
                  Flucht.«)1912-06-27@\strich\emph{Eine Gerhart Hauptmann-Première in Lauchstedt. (»Gabriel Schillings Flucht.«)} {[}1912-06-27{]}|pwv}«!\pend
           \pstart
           Alles Herzlichste von uns allen Sie alle! Ihr {\\[\baselineskip]}\spacefill\mbox{Salten}\pend
           \leftskip=0em{}\pstart
           Berghof\oindex{Berghof@\textbf{Berghof}|pw}, 2. Juli 12\pend
           
         
         \endnumbering\mylabel{h}\end{ledgroupsized}  \newcommand{\dateiname}{L03557}\newcommand{\titel}{Felix Salten an Arthur Schnitzler, 2. 7. 1912}\newcommand{\editorInnen}{Martin Anton Müller und Laura Untner}%% latex-leseansicht-abspann.tex
%% Abspann für die Leseansicht.
%% Der Schalter \ifkorrekturansicht ist bereits durch den Vorspann gesetzt.

%% latex-abspann.tex
%% Gemeinsamer Abspann für Korrekturansicht und Leseansicht.
%% Setzt den Schalter \ifkorrekturansicht voraus (gesetzt in den
%% einbindenden Dateien latex-korrekturansicht-abspann.tex bzw.
%% latex-leseansicht-abspann.tex).
%% ---------------------------------------------------------------

\normalsize

% Das esempio-Environment wird nur in der Leseansicht benötigt
\ifkorrekturansicht\else
\newenvironment{esempio}[3]%
{
    \vspace{1.5ex}
    \rlap{\underline{#1}}
    \par
    \setlength{\parindent}{0cm}
    \nopagebreak
    \leftskip=#2cm
    \rightskip=#3cm
}
{
    \par
}
\fi

\doendnotes{C}
\bigskip
\vfill

\clearpage

\footnotesize

\ifkorrekturansicht
  \lohead{\textsc{register}}
\fi

% theindex-Environment neu definieren ohne reledmac
\makeatletter
\renewenvironment{theindex}{%
  \ifkorrekturansicht
    \section*{\indexname}%
  \else
    \subsubsection*{Index der erwähnten Entitäten}%
  \fi
  \setlength{\parindent}{0pt}%
  \setlength{\parskip}{0pt plus 0.3pt}%
  \let\item\@idxitem
}{%
  \ifkorrekturansicht\clearpage\fi
}
\makeatother

\IfFileExists{\jobname-pw.ind}{\input{\jobname-pw.ind}}{}

% Quellenangabe nur in der Leseansicht
\ifkorrekturansicht\else
% Fallback-Definitionen, falls die .tex-Datei \titel etc. nicht gesetzt hat
\providecommand{\titel}{}
\providecommand{\editorInnen}{}
\providecommand{\dateiname}{\jobname}

\vspace{3cm}

\vfill

\footnotesize
\textsc{Quelle}: \titel. Herausgegeben von {\editorInnen}. In: \emph{Arthur Schnitzler: Briefwechsel mit Autorinnen und Autoren}.
 Digitale Edition, https://schnitzler-briefe.acdh.oeaw.ac.at/{\dateiname}.html (Stand \today)
\fi

\end{document}


      