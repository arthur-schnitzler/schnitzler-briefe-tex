%% latex-leseansicht-vorspann.tex
%% Vorspann für die Leseansicht.
%% Lädt die gemeinsame Datei latex-vorspann.tex mit nicht gesetztem Schalter.

\newif\ifkorrekturansicht
\korrekturansichtfalse

\input{../tex-inputs/latex-vorspann}


\section[ Felix Salten an Arthur Schnitzler, 2. 7. 1912]{L03557 Felix Salten an Arthur Schnitzler,  2. 7. 1912}
\nopagebreak\mylabel{L03557v}
\rehead{ }\normalsize\beginnumbering\briefempfaengerindex{Schnitzler, Arthur@\textsc{Schnitzler, Arthur}!zzzSalten, Felix@\emph{von Felix Salten}!1912-07-021@{2. 7. 1912}|(be}
\toendnotes[C]{\smallbreak\pagebreak[2]}
\correspDesc{Versand  durch Felix Salten am 2. 7. 1912 in Unterach am Attersee
\newline{}Erhalt  durch Arthur Schnitzler im Zeitraum [3. 7. 1912
                  – 7. 7. 1912?] in Wien}\toendnotes[C]{\smallbreak}
\Standort{CUL, Schnitzler, B 89, B 2.}
\physDesc{Bildpostkarte, 812 Zeichen
\newline{}Handschrift: schwarze Tinte, lateinische Kurrent
\newline{}Versand: Stempel: »\nobreak{}\oindex{Unterach am Attersee@\textbf{Unterach am Attersee}|pwk}Unterach
                                          \textcolor{gray}{Atters}e\textcolor{gray}{e}, 2. VII. 12\nobreak{}«.  
\newline{}Ordnung: mit Bleistift von unbekannter Hand nummeriert: »272« }\toendnotes[C]{\smallbreak}\pstart{}{\pb}Herrn D\textsuperscript{r} Arthur Schnitzler\pend{}\pstart{}Wien\oindex{Wien@\textbf{Wien}, \emph{Verwaltungsgebiet}|pw}\pend{}\pstart{}XVIII. Sternwartestraße 71\oindex{Wien@\textbf{Wien}!XVIII., Währing@\textbf{XVIII., Währing}!Sternwartestraße 71@\textbf{Sternwartestraße 71}, \emph{Wohngebäude}|pw}\pend{}{\bigskip}
\pstart
           \noindent{}\centering{}{\pb}\textcolor{gray}{\textbf{Salzkammergut\oindex{Salzkammergut@\textbf{Salzkammergut}, \emph{Region}|pw}. Berghof\oindex{Berghof@\textbf{Berghof}, \emph{Wohngebäude}|pw} bei Unterach\oindex{Unterach am Attersee@\textbf{Unterach am Attersee}|pw}.}}\pend
           \vspace{1em}
\pstart
           \noindent{}{\pb}Vielen Dank für die \label{K_L03557-1v}\edtext{Prag\oindex{Prag@\textbf{Prag}, \emph{Land}|pw}er Karte}{\lemma{\textnormal{\emph{Prager Karte}}}\Cendnote{\textnormal{Schnitzler hielt sich am 14. 6. 1912 für einen
                  Tag in Prag\oindex{Prag@\textbf{Prag}, \emph{Land}|pwk} auf.}}}\label{K_L03557-1}. Ich bin vorgestern über Landshut\oindex{Landshut@\textbf{Landshut}, \emph{Hauptstadt}|pw}, Leipzig\oindex{Leipzig@\textbf{Leipzig}, \emph{Hauptstadt}|pw}, Weimar\oindex{Weimar@\textbf{Weimar}, \emph{Verwaltungsgebiet}|pw}, Berlin\oindex{Berlin@\textbf{Berlin}, \emph{Hauptstadt}|pw} u. Dresden\oindex{Dresden@\textbf{Dresden}|pw} wieder hier\oindex{Unterach am Attersee@\textbf{Unterach am Attersee}|pwv} gelandet. War drei Wochen fort, und freue mich jetzt,
               wieder hier zu sein. Wenn gehen Sie nach \label{K_L03557-2v}\edtext{Brioni\oindex{Brijuni@\textbf{Brijuni}|pw}}{\lemma{\textnormal{\emph{Brioni}}}\Cendnote{\textnormal{Schnitzler reiste mit seiner Familie am 20. 7. 1912
                  aus Wien\oindex{Wien@\textbf{Wien}, \emph{Verwaltungsgebiet}|pwk} ab und war am nächsten Tag in Brijuni\oindex{Brijuni@\textbf{Brijuni}|pwk}.
                  Hier blieben sie den ganzen Sommer bis zum 24. 8. 1912.}}}\label{K_L03557-2}? Sie haben, glaub’ ich,
               sehr gut gewählt damit. Denn hier regnet es sich wieder tüchtig ein, und möchte ein
               nasser Sommer werden. Wie geht es Frau Olga\pwindex{Schnitzler, Olga 17.\,1.\,1882 Wien – 13.\,1.\,1970 Lugano@\textsc{Schnitzler, Olga} (17.\,1.\,1882 Wien – 13.\,1.\,1970 Lugano), \emph{Schauspielerin, Sängerin}|pw} und
               den Kinder\pwindex{Schnitzler, Heinrich 9.\,8.\,1902 Hinterbrühl – 12.\,7.\,1982 Wien@\textsc{Schnitzler, Heinrich} (9.\,8.\,1902 Hinterbrühl – 12.\,7.\,1982 Wien), \emph{Regisseur, Schauspieler}|pwv}\pwindex{Cappellini, Lili 13.\,9.\,1909 Wien – 26.\,7.\,1928 Venedig@\textsc{Cappellini, Lili} (13.\,9.\,1909 Wien – 26.\,7.\,1928 Venedig)|pwv}n? In Berlin\oindex{Berlin@\textbf{Berlin}, \emph{Hauptstadt}|pw} hörte ich, Frau Wolf\pwindex{Wolff, Louise 25.\,3.\,1855 Brünn – 25.\,6.\,1935 Berlin@\textsc{Wolff, Louise} (25.\,3.\,1855 Brünn – 25.\,6.\,1935 Berlin), \emph{Konzertagentin}|pwu} sei verreist gewesen, und habe durch
               Krankheitsfälle in der Familie böse Zeiten gehabt; wolle aber Ihrer Frau\pwindex{Schnitzler, Olga 17.\,1.\,1882 Wien – 13.\,1.\,1970 Lugano@\textsc{Schnitzler, Olga} (17.\,1.\,1882 Wien – 13.\,1.\,1970 Lugano), \emph{Schauspielerin, Sängerin}|pwv} nun endlich schreiben. Über Landshut\oindex{Landshut@\textbf{Landshut}, \emph{Hauptstadt}|pw}{ }\textcolor{gray}{etc}. wäre viel zu erzählen. Ihrem \label{K_L03557-3v}\edtext{Urteil über das Stück\pwindex{Hauptmann, Gerhart 15.\,11.\,1862 Szczawno-Zdrój – 6.\,6.\,1946 Jagniątków@\textsc{Hauptmann, Gerhart} (15.\,11.\,1862 Szczawno-Zdrój – 6.\,6.\,1946 Jagniątków), \emph{Schriftsteller}!Gabriel Schillings Flucht. Drama@\strich\emph{Gabriel Schillings Flucht. Drama}|pwv}}{\lemma{\textnormal{\emph{Urteil über das Stück}}}\Cendnote{\textnormal{Salten\pwindex{Salten, Felix 6.\,9.\,1869 Budapest – 8.\,10.\,1945 Zürich@\textsc{Salten, Felix} (6.\,9.\,1869 Budapest – 8.\,10.\,1945 Zürich), \emph{Schriftsteller, Journalist, Chefredakteur}|pwk} dürfte sich in Folge auf die zuletzt
                  erschienene Theaterkritik von Paul Goldmann\pwindex{Goldmann, Paul 31.\,1.\,1865 Breslau – 25.\,9.\,1935 Wien@\textsc{Goldmann, Paul} (31.\,1.\,1865 Breslau – 25.\,9.\,1935 Wien), \emph{Schriftsteller, Journalist}|pwk}
                  bezogen haben, die eine Aufführung von Gerhart
                     Hauptmanns\pwindex{Hauptmann, Gerhart 15.\,11.\,1862 Szczawno-Zdrój – 6.\,6.\,1946 Jagniątków@\textsc{Hauptmann, Gerhart} (15.\,11.\,1862 Szczawno-Zdrój – 6.\,6.\,1946 Jagniątków), \emph{Schriftsteller}|pwk}{ }\emph{Gabriel Schillings Flucht}\pwindex{Hauptmann, Gerhart 15.\,11.\,1862 Szczawno-Zdrój – 6.\,6.\,1946 Jagniątków@\textsc{Hauptmann, Gerhart} (15.\,11.\,1862 Szczawno-Zdrój – 6.\,6.\,1946 Jagniątków), \emph{Schriftsteller}!Gabriel Schillings Flucht. Drama@\strich\emph{Gabriel Schillings Flucht. Drama}|pwk}
                  behandelte: Paul Goldmann\pwindex{Goldmann, Paul 31.\,1.\,1865 Breslau – 25.\,9.\,1935 Wien@\textsc{Goldmann, Paul} (31.\,1.\,1865 Breslau – 25.\,9.\,1935 Wien), \emph{Schriftsteller, Journalist}|pwk}: \emph{Eine Gerhart Hauptmann-Première in Lauchstedt. (»Gabriel
                        Schillings Flucht.«)}\pwindex{Goldmann, Paul 31.\,1.\,1865 Breslau – 25.\,9.\,1935 Wien@\textsc{Goldmann, Paul} (31.\,1.\,1865 Breslau – 25.\,9.\,1935 Wien), \emph{Schriftsteller, Journalist}!Eine Gerhart Hauptmann-Première in Lauchstedt. (»Gabriel Schillings Flucht.«)@\strich\emph{Eine Gerhart Hauptmann-Première in Lauchstedt. (»Gabriel Schillings Flucht.«)}|pwk}. In: \emph{Neue Freie
                        Presse}\pwindex{Neue Freie Presse@\emph{Neue Freie Presse}|pwk}, Nr. 17.185, 27. 6. 1912,
                     Morgenblatt, S. 1–4. Für den 2. 2. 1912 führt Schnitzlers{ }\emph{Tagebuch}\pwindex{Schnitzler, Arthur 15.\,5.\,1862 Wien – 21.\,10.\,1931 ebd.@\textsc{Schnitzler, Arthur} (15.\,5.\,1862 Wien – 21.\,10.\,1931 ebd.), \emph{Schriftsteller, Mediziner}!Tagebuch@\strich\emph{Tagebuch}|pwk}
                  eine Diskussion mit Salten\pwindex{Salten, Felix 6.\,9.\,1869 Budapest – 8.\,10.\,1945 Zürich@\textsc{Salten, Felix} (6.\,9.\,1869 Budapest – 8.\,10.\,1945 Zürich), \emph{Schriftsteller, Journalist, Chefredakteur}|pwk} über das Stück\pwindex{Hauptmann, Gerhart 15.\,11.\,1862 Szczawno-Zdrój – 6.\,6.\,1946 Jagniątków@\textsc{Hauptmann, Gerhart} (15.\,11.\,1862 Szczawno-Zdrój – 6.\,6.\,1946 Jagniątków), \emph{Schriftsteller}!Gabriel Schillings Flucht. Drama@\strich\emph{Gabriel Schillings Flucht. Drama}|pwkv} an.}}}\label{K_L03557-3} bin ich ein
               wenig näher geko{\geminationm}en, seit ich es auf der Bühne sah. Paul Goldmann\pwindex{Goldmann, Paul 31.\,1.\,1865 Breslau – 25.\,9.\,1935 Wien@\textsc{Goldmann, Paul} (31.\,1.\,1865 Breslau – 25.\,9.\,1935 Wien), \emph{Schriftsteller, Journalist}|pw} war wieder »fein\pwindex{Goldmann, Paul 31.\,1.\,1865 Breslau – 25.\,9.\,1935 Wien@\textsc{Goldmann, Paul} (31.\,1.\,1865 Breslau – 25.\,9.\,1935 Wien), \emph{Schriftsteller, Journalist}!Eine Gerhart Hauptmann-Première in Lauchstedt. (»Gabriel Schillings Flucht.«)@\strich\emph{Eine Gerhart Hauptmann-Première in Lauchstedt. (»Gabriel Schillings Flucht.«)}|pwv}«!\pend
           
\pstart
           Alles Herzlichste von uns allen Sie alle! Ihr {\\[\baselineskip]}\spacefill\mbox{Salten}\pend
           \leftskip=0em{}
\pstart
           Berghof\oindex{Berghof@\textbf{Berghof}, \emph{Wohngebäude}|pw}, 2. Juli 12\pend
           \selectlanguage{ngerman}\endnumbering\briefempfaengerindex{Schnitzler, Arthur@\textsc{Schnitzler, Arthur}!zzzSalten, Felix@\emph{von Felix Salten}!1912-07-021@{2. 7. 1912}|)be}\mylabel{L03557h}  \newcommand{\dateiname}{L03557}\newcommand{\titel}{Felix Salten an Arthur Schnitzler, 2. 7. 1912}\newcommand{\editorInnen}{Martin Anton Müller und Laura Untner}%% latex-leseansicht-abspann.tex
%% Abspann für die Leseansicht.
%% Der Schalter \ifkorrekturansicht ist bereits durch den Vorspann gesetzt.

%% latex-abspann.tex
%% Gemeinsamer Abspann für Korrekturansicht und Leseansicht.
%% Setzt den Schalter \ifkorrekturansicht voraus (gesetzt in den
%% einbindenden Dateien latex-korrekturansicht-abspann.tex bzw.
%% latex-leseansicht-abspann.tex).
%% ---------------------------------------------------------------

\normalsize

% Das esempio-Environment wird nur in der Leseansicht benötigt
\ifkorrekturansicht\else
\newenvironment{esempio}[3]%
{
    \vspace{1.5ex}
    \rlap{\underline{#1}}
    \par
    \setlength{\parindent}{0cm}
    \nopagebreak
    \leftskip=#2cm
    \rightskip=#3cm
}
{
    \par
}
\fi

\doendnotes{C}
\bigskip
\vfill

\clearpage

\footnotesize

\ifkorrekturansicht
  \lohead{\textsc{register}}
\fi

% theindex-Environment neu definieren ohne reledmac
\makeatletter
\renewenvironment{theindex}{%
  \ifkorrekturansicht
    \section*{\indexname}%
  \else
    \subsubsection*{Index der erwähnten Entitäten}%
  \fi
  \setlength{\parindent}{0pt}%
  \setlength{\parskip}{0pt plus 0.3pt}%
  \let\item\@idxitem
}{%
  \ifkorrekturansicht\clearpage\fi
}
\makeatother

\IfFileExists{\jobname-pw.ind}{\input{\jobname-pw.ind}}{}

% Quellenangabe nur in der Leseansicht
\ifkorrekturansicht\else
% Fallback-Definitionen, falls die .tex-Datei \titel etc. nicht gesetzt hat
\providecommand{\titel}{}
\providecommand{\editorInnen}{}
\providecommand{\dateiname}{\jobname}

\vspace{3cm}

\vfill

\footnotesize
\textsc{Quelle}: \titel. Herausgegeben von {\editorInnen}. In: \emph{Arthur Schnitzler: Briefwechsel mit Autorinnen und Autoren}.
 Digitale Edition, https://schnitzler-briefe.acdh.oeaw.ac.at/{\dateiname}.html (Stand \today)
\fi

\end{document}


