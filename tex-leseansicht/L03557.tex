%% latex-korrekturansicht-vorspann.tex
%% Vorspann für die Korrekturansicht.
%% Lädt die gemeinsame Datei latex-vorspann.tex mit gesetztem Schalter.

\newif\ifkorrekturansicht
\korrekturansichttrue

\input{../tex-inputs/latex-vorspann}


\section[ Felix Salten an Arthur Schnitzler, 2. 7. 1912]{L03557 Felix Salten an Arthur Schnitzler, 2. 7. 1912}
\nopagebreak\mylabel{L03557v}
\rehead{ }\normalsize\beginnumbering\briefempfaengerindex{Schnitzler, Arthur@\textsc{Schnitzler, Arthur}!zzzSalten, Felix@\emph{von Felix Salten}!1912-07-021@{2. 7. 1912}|(be}
\toendnotes[C]{\smallbreak\pagebreak[2]}\Standort{CUL, Schnitzler, B 89, B 2.}
\physDesc{Bildpostkarte, 812 Zeichen
\newline{}Handschrift: schwarze Tinte, lateinische Kurrent
\newline{}Versand: Stempel: »\nobreak{}\oindex{Unterach am Attersee@\textbf{Unterach am Attersee}, \emph{P.PPL}|pwk}Unterach
                                          \textcolor{gray}{Atters}e\textcolor{gray}{e}, 2. VII. 12\nobreak{}«.  
\newline{}Ordnung: mit Bleistift von unbekannter Hand nummeriert: »272« }\toendnotes[C]{\smallbreak}\pstart{}{\pb}Herrn D\textsuperscript{r} Arthur Schnitzler\pend{}\pstart{}Wien\oindex{Wien@\textbf{Wien}, \emph{A.ADM2}|pw}\pend{}\pstart{}XVIII. Sternwartestraße 71\oindex{Sternwartestrasse 71@\textbf{Sternwartestraße 71}, \emph{Wohngebäude (K.WHS)}|pw}\pend{}{\bigskip}
\pstart
           \noindent{}\centering{}{\pb}\textcolor{gray}{\textbf{Salzkammergut\oindex{Salzkammergut@\textbf{Salzkammergut}, \emph{L.RGN}|pw}. Berghof\oindex{Berghof@\textbf{Berghof}, \emph{Wohngebäude (K.WHS)}|pw} bei Unterach\oindex{Unterach am Attersee@\textbf{Unterach am Attersee}, \emph{P.PPL}|pw}.}}\pend
           \vspace{1em}
\pstart
           \noindent{}{\pb}Vielen Dank für die \label{K_L03557-1v}\edtext{Prag\oindex{Prag@\textbf{Prag}, \emph{A.ADM1}|pw}er Karte}{\lemma{\textnormal{\emph{Prager Karte}}}\Cendnote{\textnormal{Schnitzler hielt sich am 14. 6. 1912 für einen
                  Tag in Prag\oindex{Prag@\textbf{Prag}, \emph{A.ADM1}|pwk} auf.}}}\label{K_L03557-1}. Ich bin vorgestern über Landshut\oindex{Landshut@\textbf{Landshut}, \emph{P.PPLA2}|pw}, Leipzig\oindex{Leipzig@\textbf{Leipzig}, \emph{P.PPLA3}|pw}, Weimar\oindex{Weimar@\textbf{Weimar}, \emph{A.ADM3}|pw}, Berlin\oindex{Berlin@\textbf{Berlin}, \emph{P.PPLC}|pw} u. Dresden\oindex{Dresden@\textbf{Dresden}, \emph{P.PPLA}|pw} wieder hier\oindex{Unterach am Attersee@\textbf{Unterach am Attersee}, \emph{P.PPL}|pwv} gelandet. War drei Wochen fort, und freue mich jetzt,
               wieder hier zu sein. Wenn gehen Sie nach \label{K_L03557-2v}\edtext{Brioni\oindex{Brijuni@\textbf{Brijuni}, \emph{P.PPL}|pw}}{\lemma{\textnormal{\emph{Brioni}}}\Cendnote{\textnormal{Schnitzler reiste mit seiner Familie am 20. 7. 1912
                  aus Wien\oindex{Wien@\textbf{Wien}, \emph{A.ADM2}|pwk} ab und war am nächsten Tag in Brijuni\oindex{Brijuni@\textbf{Brijuni}, \emph{P.PPL}|pwk}.
                  Hier blieben sie den ganzen Sommer bis zum 24. 8. 1912.}}}\label{K_L03557-2}? Sie haben, glaub’ ich,
               sehr gut gewählt damit. Denn hier regnet es sich wieder tüchtig ein, und möchte ein
               nasser Sommer werden. Wie geht es Frau Olga\pwindex{Schnitzler, Olga 17.01.1882 – 13.01.1970@\textsc{Schnitzler, Olga} (17.01.1882 – 13.01.1970), \emph{Schauspieler/Schauspielerin, Sänger/Sängerin}|pw} und
               den Kinder\pwindex{Schnitzler, Heinrich 09.08.1902 – 12.07.1982@\textsc{Schnitzler, Heinrich} (09.08.1902 – 12.07.1982), \emph{Regisseur/Regisseurin, Schauspieler/Schauspielerin}|pwv}\pwindex{Cappellini, Lili 13.09.1909 – 26.07.1928@\textsc{Cappellini, Lili} (13.09.1909 – 26.07.1928)|pwv}n? In Berlin\oindex{Berlin@\textbf{Berlin}, \emph{P.PPLC}|pw} hörte ich, Frau Wolf\pwindex{Wolff, Louise 25.03.1855 – 25.06.1935@\textsc{Wolff, Louise} (25.03.1855 – 25.06.1935), \emph{Konzertagent/Konzertagentin}|pwu} sei verreist gewesen, und habe durch
               Krankheitsfälle in der Familie böse Zeiten gehabt; wolle aber Ihrer Frau\pwindex{Schnitzler, Olga 17.01.1882 – 13.01.1970@\textsc{Schnitzler, Olga} (17.01.1882 – 13.01.1970), \emph{Schauspieler/Schauspielerin, Sänger/Sängerin}|pwv} nun endlich schreiben. Über Landshut\oindex{Landshut@\textbf{Landshut}, \emph{P.PPLA2}|pw}{ }\textcolor{gray}{etc}. wäre viel zu erzählen. Ihrem \label{K_L03557-3v}\edtext{Urteil über das Stück\pwindex{Gabriel Schillings Flucht. Drama@\emph{Gabriel Schillings Flucht. Drama}|pwv}}{\lemma{\textnormal{\emph{Urteil über das Stück}}}\Cendnote{\textnormal{Salten\pwindex{Salten, Felix 06.09.1869 – 08.10.1945@\textsc{Salten, Felix} (06.09.1869 – 08.10.1945), \emph{Schriftsteller/Schriftstellerin, Journalist/Journalistin, Chefredakteur/Chefredakteurin}|pwk} dürfte sich in Folge auf die zuletzt
                  erschienene Theaterkritik von Paul Goldmann\pwindex{Goldmann, Paul 31.01.1865 – 25.09.1935@\textsc{Goldmann, Paul} (31.01.1865 – 25.09.1935), \emph{Schriftsteller/Schriftstellerin, Journalist/Journalistin}|pwk}
                  bezogen haben, die eine Aufführung von Gerhart
                     Hauptmanns\pwindex{Hauptmann, Gerhart 15.11.1862 – 06.06.1946@\textsc{Hauptmann, Gerhart} (15.11.1862 – 06.06.1946), \emph{Schriftsteller/Schriftstellerin}|pwk}{ }\emph{Gabriel Schillings Flucht}\pwindex{Gabriel Schillings Flucht. Drama@\emph{Gabriel Schillings Flucht. Drama}|pwk}
                  behandelte: Paul Goldmann\pwindex{Goldmann, Paul 31.01.1865 – 25.09.1935@\textsc{Goldmann, Paul} (31.01.1865 – 25.09.1935), \emph{Schriftsteller/Schriftstellerin, Journalist/Journalistin}|pwk}: \emph{Eine Gerhart Hauptmann-Première in Lauchstedt. (»Gabriel
                        Schillings Flucht.«)}\pwindex{Eine Gerhart Hauptmann-Premiere in Lauchstedt. (»Gabriel Schillings Flucht.«)@\emph{Eine Gerhart Hauptmann-Première in Lauchstedt. (»Gabriel Schillings Flucht.«)}|pwk}. In: \emph{Neue Freie
                        Presse}\pwindex{Neue Freie Presse@\emph{Neue Freie Presse}|pwk}, Nr. 17.185, 27. 6. 1912,
                     Morgenblatt, S. 1–4. Für den 2. 2. 1912 führt Schnitzlers{ }\emph{Tagebuch}\pwindex{Tagebuch@\emph{Tagebuch}|pwk}
                  eine Diskussion mit Salten\pwindex{Salten, Felix 06.09.1869 – 08.10.1945@\textsc{Salten, Felix} (06.09.1869 – 08.10.1945), \emph{Schriftsteller/Schriftstellerin, Journalist/Journalistin, Chefredakteur/Chefredakteurin}|pwk} über das Stück\pwindex{Gabriel Schillings Flucht. Drama@\emph{Gabriel Schillings Flucht. Drama}|pwkv} an.}}}\label{K_L03557-3} bin ich ein
               wenig näher geko{\geminationm}en, seit ich es auf der Bühne sah. Paul Goldmann\pwindex{Goldmann, Paul 31.01.1865 – 25.09.1935@\textsc{Goldmann, Paul} (31.01.1865 – 25.09.1935), \emph{Schriftsteller/Schriftstellerin, Journalist/Journalistin}|pw} war wieder »fein\pwindex{Eine Gerhart Hauptmann-Premiere in Lauchstedt. (»Gabriel Schillings Flucht.«)@\emph{Eine Gerhart Hauptmann-Première in Lauchstedt. (»Gabriel Schillings Flucht.«)}|pwv}«!\pend
           
\pstart
           Alles Herzlichste von uns allen Sie alle! Ihr {\\[\baselineskip]}\spacefill\mbox{Salten}\pend
           \leftskip=0em{}
\pstart
           Berghof\oindex{Berghof@\textbf{Berghof}, \emph{Wohngebäude (K.WHS)}|pw}, 2. Juli 12\pend
           \selectlanguage{ngerman}\endnumbering\briefempfaengerindex{Schnitzler, Arthur@\textsc{Schnitzler, Arthur}!zzzSalten, Felix@\emph{von Felix Salten}!1912-07-021@{2. 7. 1912}|)be}\mylabel{L03557h}  \normalsize

\doendnotes{C}
\bigskip
\vfill

\clearpage

\footnotesize

\lohead{\textsc{register}}

% Definiere theindex-Environment komplett neu ohne reledmac
\makeatletter
\renewenvironment{theindex}{%
  \section*{\indexname}%
  \setlength{\parindent}{0pt}%
  \setlength{\parskip}{0pt plus 0.3pt}%
  \let\item\@idxitem
}{%
  \clearpage
}
\makeatother

\IfFileExists{\jobname-pw.ind}{\input{\jobname-pw.ind}}{}

\end{document}

      