%% latex-leseansicht-vorspann.tex
%% Vorspann für die Leseansicht.
%% Lädt die gemeinsame Datei latex-vorspann.tex mit nicht gesetztem Schalter.

\newif\ifkorrekturansicht
\korrekturansichtfalse

\input{../tex-inputs/latex-vorspann}


         
         \newcommand{\erwaehntePersonen}{Personen: Lisa Clarus, Peter Rotenstern, Olga Schnitzler}
         \newcommand{\erwaehnteInstitutionen}{}
         \newcommand{\erwaehnteOrte}{Orte: Russland, Wien}
         \newcommand{\erwaehnteWerke}{
               \section[Hermann Bahr an Arthur Schnitzler, 27. 9. 1906]{ Hermann Bahr an Arthur Schnitzler, 27. 9. 1906}\nopagebreak\mylabel{v}\rehead{ }\begin{ledgroupsized}[t]{13cm}\normalsize\beginnumbering \toendnotes[C]{\smallbreak\pagebreak[2]} \Standort{CUL, Schnitzler, B 5b.}
\physDesc{Brief, 1 Blatt, 2 Seiten
\newline{}Handschrift Lisa Clarus: blaue Tinte, lateinische Kurrent\newline{}Handschrift Hermann Bahr: blaue Tinte\newline{}Ordnung: mit Bleistift von unbekannter Hand nummeriert:
                                    »141« }\buchAbdrucke{\weitereDrucke{Hermann Bahr, Arthur Schnitzler: \emph{Briefwechsel, Aufzeichnungen, Dokumente (1891–1931)}. Hg. Kurt Ifkovits und Martin Anton Müller. Göttingen: \emph{Wallstein} 2018, S. 381–382.} }\toendnotes[C]{\smallbreak}\pstart
           \raggedleft{}{\pb}27. 9. 06.\pend
           \pstart\center{}Lieber Arthur!\pend\pstart
           Verzeihe, dass ich dictiere, aber mich macht das Mechanische des Schreibens
               schrecklich nervös.\pend
           \pstart
           Ich bleibe bis zum 1. November noch in Wien\oindex{Wien@\textbf{Wien}|pw} und
               möchte nun sehr gern Ende der nächsten Woche, oder Anfang der übernächsten Woche
               einmal Vormittag zu Dir kommen. Vielleicht bestimmst Du mir einen Tag, der Dir
               passt.\pend
           \pstart
           Und noch etwas: Du hast einen russischen\oindex{Russland@\textbf{Russland}|pw}{ }{\pb}Uebersetzer\pwindex{Rotenstern, Peter 10.01.1868 – 1944@\textsc{Rotenstern, Peter} (10.01.1868 – 1944), \emph{Journalist, Übersetzer}|pwv}, der sich auch
               einmal an mich gewendet hat, ich habe aber seinen Namen und seine Adresse vergessen.
               Kannst Du mir diese schreiben?\pend
           \pstart
           Mit vielen Grüssen an Frau Olga\pwindex{Schnitzler, Olga 17.01.1882 – 13.01.1970@\textsc{Schnitzler, Olga} (17.01.1882 – 13.01.1970), \emph{Schauspielerin, Sängerin}|pw}{\\[\baselineskip]}herzlichst{\\[\baselineskip]}Dein{\\[\baselineskip]}\spacefill\mbox{{[}hs. Bahr:{]} HermannBahr}\pend
           \leftskip=0em{}
         
         \endnumbering\mylabel{h}\end{ledgroupsized}  \newcommand{\dateiname}{L01630}\newcommand{\titel}{Hermann Bahr an Arthur Schnitzler, 27. 9. 1906}\newcommand{\editorInnen}{ Kurt Ifkovits,  Martin Anton Müller}%% latex-leseansicht-abspann.tex
%% Abspann für die Leseansicht.
%% Der Schalter \ifkorrekturansicht ist bereits durch den Vorspann gesetzt.

%% latex-abspann.tex
%% Gemeinsamer Abspann für Korrekturansicht und Leseansicht.
%% Setzt den Schalter \ifkorrekturansicht voraus (gesetzt in den
%% einbindenden Dateien latex-korrekturansicht-abspann.tex bzw.
%% latex-leseansicht-abspann.tex).
%% ---------------------------------------------------------------

\normalsize

% Das esempio-Environment wird nur in der Leseansicht benötigt
\ifkorrekturansicht\else
\newenvironment{esempio}[3]%
{
    \vspace{1.5ex}
    \rlap{\underline{#1}}
    \par
    \setlength{\parindent}{0cm}
    \nopagebreak
    \leftskip=#2cm
    \rightskip=#3cm
}
{
    \par
}
\fi

\doendnotes{C}
\bigskip
\vfill

\clearpage

\footnotesize

\ifkorrekturansicht
  \lohead{\textsc{register}}
\fi

% theindex-Environment neu definieren ohne reledmac
\makeatletter
\renewenvironment{theindex}{%
  \ifkorrekturansicht
    \section*{\indexname}%
  \else
    \subsubsection*{Index der erwähnten Entitäten}%
  \fi
  \setlength{\parindent}{0pt}%
  \setlength{\parskip}{0pt plus 0.3pt}%
  \let\item\@idxitem
}{%
  \ifkorrekturansicht\clearpage\fi
}
\makeatother

\IfFileExists{\jobname-pw.ind}{\input{\jobname-pw.ind}}{}

% Quellenangabe nur in der Leseansicht
\ifkorrekturansicht\else
% Fallback-Definitionen, falls die .tex-Datei \titel etc. nicht gesetzt hat
\providecommand{\titel}{}
\providecommand{\editorInnen}{}
\providecommand{\dateiname}{\jobname}

\vspace{3cm}

\vfill

\footnotesize
\textsc{Quelle}: \titel. Herausgegeben von {\editorInnen}. In: \emph{Arthur Schnitzler: Briefwechsel mit Autorinnen und Autoren}.
 Digitale Edition, https://schnitzler-briefe.acdh.oeaw.ac.at/{\dateiname}.html (Stand \today)
\fi

\end{document}


      