%% latex-leseansicht-vorspann.tex
%% Vorspann für die Leseansicht.
%% Lädt die gemeinsame Datei latex-vorspann.tex mit nicht gesetztem Schalter.

\newif\ifkorrekturansicht
\korrekturansichtfalse

\input{../tex-inputs/latex-vorspann}


         
         \newcommand{\erwaehntePersonen}{Personen: Hermann Bahr, Paula Dehmel, Heinrich Schnitzler}
         \newcommand{\erwaehnteOrte}{Orte: Wien}
         \newcommand{\erwaehnteWerke}{
               \section[Arthur Schnitzler an Hermann Bahr, 3. 10. 1902]{ Arthur Schnitzler an Hermann Bahr, 3. 10. 1902}\nopagebreak\mylabel{v}\rehead{ }\begin{ledgroupsized}[t]{13cm}\normalsize\beginnumbering \toendnotes[C]{\smallbreak\pagebreak[2]} \Standort{TMW, HS AM 23353 Ba.}
\physDesc{Brief, 1 Blatt, 2 Seiten
\newline{}Handschrift: schwarze Tinte, deutsche Kurrent
\newline{}Bahr: Blattecken vermutlich beim Brieföffnen beschädigt \newline{}Ordnung: Lochung }\buchAbdrucke{\weitereDrucke{1) \emph{3. 10. 1902.} In: Arthur Schnitzler: \emph{The Letters of Arthur Schnitzler to Hermann Bahr}. Edited, annotated, and with an introduction, by Donald G.
                        Daviau. Chapel Hill: \emph{The University of North Carolina Press} 1978, S. 76 (University of North Carolina studies in the Germanic languages
                        and literatures, 89).} \weitereDrucke{2) Hermann Bahr, Arthur Schnitzler: \emph{Briefwechsel, Aufzeichnungen, Dokumente (1891–1931)}. Hg. Kurt Ifkovits und Martin Anton Müller. Göttingen: \emph{Wallstein} 2018, S. 243.} }\toendnotes[C]{\smallbreak}\pstart
           \raggedleft{}{\pb}Wien\oindex{Wien@\textbf{Wien}|pw}, 3. 10. 902\pend
           \pstart
           lieber Hermann, zu einem einmaligen Beitrag, der natürlich die Höhe
               einer Monatsrate überſchreiten und gelegentlich auch wiederholt werden könnte, bin
               ich gern bereit – zur Auszahlung einer monatlichen noch ſo kleinen Rente wünſche ich
               mich nicht zu verpflichten.\pend
           \pstart
           Da man über meine Vermögensverhältniſſe, die allerdings niemanden angehen, \strikeout{übrigens}{ }ſonderbare Anſichten zu hegen ſcheint, die mir
               manchmal unbequem werden, {\pb}bitte ich dich, die
               freundliche Briefſchreiberin\pwindex{Dehmel, Paula 1862-12-31 – 1918-07-08@\textsc{Dehmel, Paula} (1862-12-31 – 1918-07-08), \emph{Schriftstellerin}|pwv} zu
               belehren, daſs mein Einkommen aus meinem »Vermögen« zwiſchen 7 und 800 Gulden
               jährlich ſchwankt und ich im übrigen auf den Ertrag meiner Feder angewieſen bin. (Und
               dir iſt es ja wohl bekannt, daſs ich \label{K_L01237_1v}\edtext{nicht für mich allein\pwindex{Schnitzler, Heinrich 09.08.1902 – 12.07.1982@\textsc{Schnitzler, Heinrich} (09.08.1902 – 12.07.1982), \emph{Regisseur, Schauspieler}|pwv} zu
                  ſorgen}{\lemma{\textnormal{\emph{nicht … ſorgen}}}\Cendnote{\textnormal{Am 9. 8. 1902 war der Sohn Heinrich\pwindex{Schnitzler, Heinrich 09.08.1902 – 12.07.1982@\textsc{Schnitzler, Heinrich} (09.08.1902 – 12.07.1982), \emph{Regisseur, Schauspieler}|pwk}
                  auf die Welt gekommen.}}}\label{K_L01237_1h} habe.)\pend
           \pstart
           Herzlichen Gruſs, und auf ſehr baldiges Wiederſehen.{\\[\baselineskip]}Dein{\\[\baselineskip]}\spacefill\mbox{Arthur Sch}\pend
           \leftskip=0em{}
         
         \endnumbering\mylabel{h}\end{ledgroupsized}  \newcommand{\dateiname}{L01237}\newcommand{\titel}{Arthur Schnitzler an Hermann Bahr, 3. 10. 1902}\newcommand{\editorInnen}{ Kurt Ifkovits,  Martin Anton Müller}%% latex-leseansicht-abspann.tex
%% Abspann für die Leseansicht.
%% Der Schalter \ifkorrekturansicht ist bereits durch den Vorspann gesetzt.

%% latex-abspann.tex
%% Gemeinsamer Abspann für Korrekturansicht und Leseansicht.
%% Setzt den Schalter \ifkorrekturansicht voraus (gesetzt in den
%% einbindenden Dateien latex-korrekturansicht-abspann.tex bzw.
%% latex-leseansicht-abspann.tex).
%% ---------------------------------------------------------------

\normalsize

% Das esempio-Environment wird nur in der Leseansicht benötigt
\ifkorrekturansicht\else
\newenvironment{esempio}[3]%
{
    \vspace{1.5ex}
    \rlap{\underline{#1}}
    \par
    \setlength{\parindent}{0cm}
    \nopagebreak
    \leftskip=#2cm
    \rightskip=#3cm
}
{
    \par
}
\fi

\doendnotes{C}
\bigskip
\vfill

\clearpage

\footnotesize

\ifkorrekturansicht
  \lohead{\textsc{register}}
\fi

% theindex-Environment neu definieren ohne reledmac
\makeatletter
\renewenvironment{theindex}{%
  \ifkorrekturansicht
    \section*{\indexname}%
  \else
    \subsubsection*{Index der erwähnten Entitäten}%
  \fi
  \setlength{\parindent}{0pt}%
  \setlength{\parskip}{0pt plus 0.3pt}%
  \let\item\@idxitem
}{%
  \ifkorrekturansicht\clearpage\fi
}
\makeatother

\IfFileExists{\jobname-pw.ind}{\input{\jobname-pw.ind}}{}

% Quellenangabe nur in der Leseansicht
\ifkorrekturansicht\else
% Fallback-Definitionen, falls die .tex-Datei \titel etc. nicht gesetzt hat
\providecommand{\titel}{}
\providecommand{\editorInnen}{}
\providecommand{\dateiname}{\jobname}

\vspace{3cm}

\vfill

\footnotesize
\textsc{Quelle}: \titel. Herausgegeben von {\editorInnen}. In: \emph{Arthur Schnitzler: Briefwechsel mit Autorinnen und Autoren}.
 Digitale Edition, https://schnitzler-briefe.acdh.oeaw.ac.at/{\dateiname}.html (Stand \today)
\fi

\end{document}


      