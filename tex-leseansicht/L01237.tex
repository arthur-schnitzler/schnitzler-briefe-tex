%% latex-korrekturansicht-vorspann.tex
%% Vorspann für die Korrekturansicht.
%% Lädt die gemeinsame Datei latex-vorspann.tex mit gesetztem Schalter.

\newif\ifkorrekturansicht
\korrekturansichttrue

\input{../tex-inputs/latex-vorspann}


\section[Arthur Schnitzler an Hermann Bahr, 3. 10. 1902]{L01237 Arthur Schnitzler an Hermann Bahr, 3. 10. 1902}
\nopagebreak\mylabel{L01237v}
\rehead{ }\normalsize\beginnumbering\briefempfaengerindex{Bahr, Hermann@\textsc{Bahr, Hermann}!zzzSchnitzler, Arthur@\emph{von Arthur Schnitzler}!1902-10-031@{3. 10. 1902}|(be}
\toendnotes[C]{\smallbreak\pagebreak[2]}\Standort{TMW, HS AM 23353 Ba.}
\physDesc{Brief, 1 Blatt, 2 Seiten, 784 Zeichen
\newline{}Handschrift: schwarze Tinte, deutsche Kurrent
\newline{}Bahr: Blattecken vermutlich beim Brieföffnen beschädigt 
\newline{}Ordnung: Lochung }
\buchAbdrucke{\weitereDrucke{1) Arthur Schnitzler: \emph{The Letters of Arthur Schnitzler to Hermann Bahr}. Chapel Hill: \emph{The University of North Carolina Press} 1978, S. 76.} \weitereDrucke{2) Hermann Bahr, Arthur Schnitzler: \emph{Briefwechsel, Aufzeichnungen, Dokumente (1891–1931)}. Göttingen: \emph{Wallstein} 2018, S. 243.} }\toendnotes[C]{\smallbreak}
\pstart
           \raggedleft{}{\pb}Wien\oindex{Wien@\textbf{Wien}, \emph{A.ADM2}|pw}, 3. 10. 902\pend
           \vspace{0.5em}
\pstart
           lieber Hermann, zu einem einmaligen Beitrag, der natürlich die Höhe
               einer Monatsrate überſchreiten und gelegentlich auch wiederholt werden könnte, bin
               ich gern bereit – zur Auszahlung einer monatlichen noch ſo kleinen Rente wünſche ich
               mich nicht zu verpflichten.\pend
           
\pstart
           Da man über meine Vermögensverhältniſſe, die allerdings niemanden angehen, \strikeout{übrigens}{ }ſonderbare Anſichten zu hegen ſcheint, die mir
               manchmal unbequem werden, {\pb}bitte ich dich, die
               freundliche Briefſchreiberin\pwindex{Dehmel, Paula 1862-12-31 – 1918-07-08@\textsc{Dehmel, Paula} (1862-12-31 – 1918-07-08), \emph{Schriftsteller/Schriftstellerin}|pwv}
               zu belehren, daſs mein Einkommen aus meinem »Vermögen« zwiſchen 7 und 800 Gulden
               jährlich ſchwankt und ich im übrigen auf den Ertrag meiner Feder angewieſen bin. (Und
               dir iſt es ja wohl bekannt, daſs ich \label{K_L01237-1v}\edtext{nicht für mich allein\pwindex{Schnitzler, Heinrich 09.08.1902 – 12.07.1982@\textsc{Schnitzler, Heinrich} (09.08.1902 – 12.07.1982), \emph{Regisseur/Regisseurin, Schauspieler/Schauspielerin}|pwv} zu
                  ſorgen}{\lemma{\textnormal{\emph{nicht … ſorgen}}}\Cendnote{\textnormal{Am 9. 8. 1902 war der Sohn Heinrich\pwindex{Schnitzler, Heinrich 09.08.1902 – 12.07.1982@\textsc{Schnitzler, Heinrich} (09.08.1902 – 12.07.1982), \emph{Regisseur/Regisseurin, Schauspieler/Schauspielerin}|pwk}
                  auf die Welt gekommen.}}}\label{K_L01237-1} habe.)\pend
           
\pstart
           Herzlichen Gruſs, und auf ſehr baldiges Wiederſehen.{\\[\baselineskip]}Dein{\\[\baselineskip]}\spacefill\mbox{Arthur Sch}\pend
           \leftskip=0em{}\selectlanguage{ngerman}\endnumbering\briefempfaengerindex{Bahr, Hermann@\textsc{Bahr, Hermann}!zzzSchnitzler, Arthur@\emph{von Arthur Schnitzler}!1902-10-031@{3. 10. 1902}|)be}\mylabel{L01237h}  \normalsize

\doendnotes{C}
\bigskip
\vfill

\clearpage

\footnotesize

\lohead{\textsc{register}}

% Definiere theindex-Environment komplett neu ohne reledmac
\makeatletter
\renewenvironment{theindex}{%
  \section*{\indexname}%
  \setlength{\parindent}{0pt}%
  \setlength{\parskip}{0pt plus 0.3pt}%
  \let\item\@idxitem
}{%
  \clearpage
}
\makeatother

\IfFileExists{\jobname-pw.ind}{\input{\jobname-pw.ind}}{}

\end{document}

      