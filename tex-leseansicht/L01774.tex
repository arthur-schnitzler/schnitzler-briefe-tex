%% latex-korrekturansicht-vorspann.tex
%% Vorspann für die Korrekturansicht.
%% Lädt die gemeinsame Datei latex-vorspann.tex mit gesetztem Schalter.

\newif\ifkorrekturansicht
\korrekturansichttrue

\input{../tex-inputs/latex-vorspann}


\section[Max Burckhard an Arthur Schnitzler, 7. 6. 1908]{L01774 Max Burckhard an Arthur Schnitzler, 7. 6. 1908}
\nopagebreak\mylabel{L01774v}
\rehead{ }\normalsize\beginnumbering\briefempfaengerindex{Schnitzler, Arthur@\textsc{Schnitzler, Arthur}!zzzBurckhard, Max Eugen@\emph{von Max Eugen Burckhard}!1908-06-071@{7. 6. 1908}|(be}
\toendnotes[C]{\smallbreak\pagebreak[2]}\Standort{CUL, Schnitzler, B 20.}
\physDesc{Brief, 1 Blatt, 1 Seite, 801 Zeichen
\newline{}Handschrift: schwarze Tinte, deutsche Kurrent
\newline{}Ordnung: mit Bleistift von unbekannter Hand nummeriert:
                                    »22« }\toendnotes[C]{\smallbreak}
\pstart
           {\pb}\textcolor{gray}{\textbf{D\textsuperscript{r.} Max Burckhard}}\hfill \textcolor{gray}{\textbf{Wien, IX. Porzellangasse 48\oindex{Porzellangasse@\textbf{Porzellangasse}, \emph{Straße (K.STR)}|pw}{ }..........}}\pend
           
\pstart
           \raggedleft{}\textcolor{gray}{\textbf{St. Gilgen\oindex{St. Gilgen@\textbf{St. Gilgen}, \emph{A.ADM3}|pw}}}{ }7. 6. 08.\pend
           
\pstart{}Lieber, ſehr verehrter Herr Doctor!\pend\vspace{0.5em}
\pstart
           Ich ſage Ihnen herzlichſten Dank für die freundliche Zuſendung Ihres eben
               erſchienenen Romans\pwindex{Weg ins Freie. Roman@\emph{Der Weg ins Freie. Roman}|pwv}. Gegen
               meine Principien hatte ich die »Fortſetzungen« bereits in der Rundſchau\pwindex{neue Rundschau@\emph{Die neue Rundschau}|pw} geleſen, da mich ſchon die erſte Nu{\geminationm}er hiezu verleitet hatte: den Schluß aber hatte ich
               noch nicht erhalten, denn die Entfernung von Wien\oindex{Wien@\textbf{Wien}, \emph{A.ADM2}|pw}
               nach Gilgen\oindex{St. Gilgen@\textbf{St. Gilgen}, \emph{A.ADM3}|pw} iſt lang und mein Buchhändler und die
               Poſt ſind langſam. Mich hat ſo Vieles in dem Buche\pwindex{Weg ins Freie. Roman@\emph{Der Weg ins Freie. Roman}|pwv} tief bewegt, daſs ich es nicht mit ein paar Zeilen
               zum Ausdruck bringen könnte.\pend
           
\pstart
           Ko{\geminationm}en Sie nicht heuer nach Jahrhunderten wieder nach St Gilgen\oindex{St. Gilgen@\textbf{St. Gilgen}, \emph{A.ADM3}|pw}? Ich war leider, da ich im Herbſt und
               nach Weihnachten in Wien\oindex{Wien@\textbf{Wien}, \emph{A.ADM2}|pw} war, beidemal unwohl und
               konnte daher meinen Vorſatz, Sie aufzuſuchen nicht ausführen.\pend
           
\pstart
           Herzlichſt mit Handkuſs an die verehrte gnädige Frau\pwindex{Schnitzler, Olga 17.01.1882 – 13.01.1970@\textsc{Schnitzler, Olga} (17.01.1882 – 13.01.1970), \emph{Schauspieler/Schauspielerin, Sänger/Sängerin}|pwv}\pend
           
\pstart
           Ihr{\\[\baselineskip]}\spacefill\mbox{D\textsuperscript{r}Burckhard}\pend
           \leftskip=0em{}\selectlanguage{ngerman}\endnumbering\briefempfaengerindex{Schnitzler, Arthur@\textsc{Schnitzler, Arthur}!zzzBurckhard, Max Eugen@\emph{von Max Eugen Burckhard}!1908-06-071@{7. 6. 1908}|)be}\mylabel{L01774h}  \normalsize

\doendnotes{C}
\bigskip
\vfill

\clearpage

\footnotesize

\lohead{\textsc{register}}

% Definiere theindex-Environment komplett neu ohne reledmac
\makeatletter
\renewenvironment{theindex}{%
  \section*{\indexname}%
  \setlength{\parindent}{0pt}%
  \setlength{\parskip}{0pt plus 0.3pt}%
  \let\item\@idxitem
}{%
  \clearpage
}
\makeatother

\IfFileExists{\jobname-pw.ind}{\input{\jobname-pw.ind}}{}

\end{document}

      