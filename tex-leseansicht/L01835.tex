%% latex-leseansicht-vorspann.tex
%% Vorspann für die Leseansicht.
%% Lädt die gemeinsame Datei latex-vorspann.tex mit nicht gesetztem Schalter.

\newif\ifkorrekturansicht
\korrekturansichtfalse

\input{../tex-inputs/latex-vorspann}


\section[Albert Ehrenstein an Arthur Schnitzler, 27. 3. 1909]{L01835 Albert Ehrenstein an Arthur Schnitzler, 27. 3. 1909}
\nopagebreak\mylabel{L01835v}
\rehead{ }\normalsize\beginnumbering\briefempfaengerindex{Schnitzler, Arthur@\textsc{Schnitzler, Arthur}!zzzEhrenstein, Albert@\emph{von Albert Ehrenstein}!1909-03-271@{27. 3. 1909}|(be}
\toendnotes[C]{\smallbreak\pagebreak[2]}
\correspDesc{Versand  durch Albert Ehrenstein am 27. 3. 1909 in Wien
\newline{}Erhalt  durch Arthur Schnitzler im Zeitraum [27. 3. 1909
                  – 31. 3. 1909?] in Wien}\toendnotes[C]{\smallbreak}
\Standort{CUL, Schnitzler, B 30.}
\physDesc{Brief, 1 Blatt, 3 Seiten, 1492 Zeichen
\newline{}Handschrift: schwarze Tinte, deutsche Kurrent
\newline{}Schnitzler: mit Bleistift beschriftet: »\textsc{Ehrenst\textcolor{gray}{ein}}« }
\buchAbdrucke{\weitereDrucke{Albert Ehrenstein: \emph{Briefe}. Herausgegeben von Hanni Mittelmann. München: \emph{Boer} 1989, S. 27 (Werke, 1).} }\toendnotes[C]{\smallbreak}
\pstart
           
\pstart
           {\pb}XVI \textsc{Ottakringerstr}
                        114\oindex{Wien@\textbf{Wien}!XVI., Ottakring@\textbf{XVI., Ottakring}!Ottakringer Straße@\textbf{Ottakringer Straße}, \emph{Straße}|pw}\oindex{Wien@\textbf{Wien}!XVII., Hernals@\textbf{XVII., Hernals}!Ottakringer Straße@\textbf{Ottakringer Straße}, \emph{Straße}|pw}.\pend
           
\pstart
           \raggedleft{}27 III. 09.\pend
           \pend
           
\pstart{}Sehr geehrter Herr Doktor,\pend\vspace{0.5em}
\pstart
           gerne möchte ich pflichtſchuldigſt einen ausführlichen Bericht erſtatten über meine
               »Besuche« bei den Herren Geld- und Schreibheimers\pwindex{Auernheimer, Raoul 15.\,4.\,1876 Wien – 6.\,1.\,1948 Oakland@\textsc{Auernheimer, Raoul} (15.\,4.\,1876 Wien – 6.\,1.\,1948 Oakland), \emph{Schriftsteller, Journalist, Kritiker}|pwv}\pwindex{Oppenheimer, Felix von 20.\,2.\,1874 Wien – 15.\,11.\,1938 ebd.@\textsc{Oppenheimer, Felix von} (20.\,2.\,1874 Wien – 15.\,11.\,1938 ebd.), \emph{Schriftsteller, Soziologe, Mäzen}|pwv}. Es liegen bei mir aus verſchiedenen Jahren Briefe
               an Sie,{ }ſehr geehrter Herr Doktor, die ich nicht abſchickte, fröhlich-ergebene und
               verärgerte, Geſchäftsbriefe und{ }ſolche vornehmeren Charakters. Auch diesmal verfaßte
               ich eine Menge mehr, minder gewundener Schreiben. Sie gerieten aber wie jene anderen
               im Format zu groß, und (ich{ }ſage es \label{K_L01835-1v}\edtext{\textsc{pro privata Augustissimi notitia}}{\lemma{\textnormal{\emph{pro … notitia}}}\Cendnote{\textnormal{lateinisch: zur persönlichen
                  Kenntnisnahme des Herrschers}}}\label{K_L01835-1}) {\pb}inhaltlich bargen{ }ſie Dinge, die weder für die genannten Herren noch für mich
               beſonders schmeichelhaft waren. Wenn eine getreue Schilderung des mir Widerfahrenen
               für Sie,{ }ſehr geehrter Herr Doktor, Intereſſe haben{ }ſollte, würden Sie mich aufs Neue
               verbinden, indem Sie mir geſtatten, Ihnen einmal mündlich über meine Erfahrungen im
               Lande der Ariſtokratoiden und Zeitungsleute Rede zu{ }ſtehen. Starke pſychiſche
               Depreſſionen, hervorgerufen durch das empfangsfeindliche Benehmen der Herren Gloſſy\pwindex{Glossy, Karl 7.\,3.\,1848 Wien – 9.\,9.\,1937 ebd.@\textsc{Glossy, Karl} (7.\,3.\,1848 Wien – 9.\,9.\,1937 ebd.), \emph{Schriftsteller, Museumsleiter, Zensurbeirat}|pw}, Auern-\pwindex{Auernheimer, Raoul 15.\,4.\,1876 Wien – 6.\,1.\,1948 Oakland@\textsc{Auernheimer, Raoul} (15.\,4.\,1876 Wien – 6.\,1.\,1948 Oakland), \emph{Schriftsteller, Journalist, Kritiker}|pw} und Oppenheimer\pwindex{Oppenheimer, Felix von 20.\,2.\,1874 Wien – 15.\,11.\,1938 ebd.@\textsc{Oppenheimer, Felix von} (20.\,2.\,1874 Wien – 15.\,11.\,1938 ebd.), \emph{Schriftsteller, Soziologe, Mäzen}|pw}, und {\pb}nicht zumindeſt durch meine altbewährten
               Ungeſchicklichkeiten, die leider auch auf Sie,{ }ſehr geehrter Herr Doktor, Bezug
               haben, Bitterkeit und Rachſucht, wie Demut und übertriebene Sucht gerecht zu{ }ſein,
               machen die Abfaſſung eines vernünftigen Briefes zur Unmöglichkeit Ihrem Ihnen,{ }ſehr
               geehrter Herr Doktor, nun auch noch für recht merkwürdige tragikomiſche Erlebniſſe
               dankbaren, ergebenſten\pend
           \pstart \spacefill\mbox{Albert Ehrenstein.}\pend{}\selectlanguage{ngerman}\endnumbering\briefempfaengerindex{Schnitzler, Arthur@\textsc{Schnitzler, Arthur}!zzzEhrenstein, Albert@\emph{von Albert Ehrenstein}!1909-03-271@{27. 3. 1909}|)be}\mylabel{L01835h}  \newcommand{\dateiname}{L01835}\newcommand{\titel}{Albert Ehrenstein an Arthur Schnitzler, 27. 3. 1909}\newcommand{\editorInnen}{Martin Anton Müller und Gerd-Hermann Susen}%% latex-leseansicht-abspann.tex
%% Abspann für die Leseansicht.
%% Der Schalter \ifkorrekturansicht ist bereits durch den Vorspann gesetzt.

%% latex-abspann.tex
%% Gemeinsamer Abspann für Korrekturansicht und Leseansicht.
%% Setzt den Schalter \ifkorrekturansicht voraus (gesetzt in den
%% einbindenden Dateien latex-korrekturansicht-abspann.tex bzw.
%% latex-leseansicht-abspann.tex).
%% ---------------------------------------------------------------

\normalsize

% Das esempio-Environment wird nur in der Leseansicht benötigt
\ifkorrekturansicht\else
\newenvironment{esempio}[3]%
{
    \vspace{1.5ex}
    \rlap{\underline{#1}}
    \par
    \setlength{\parindent}{0cm}
    \nopagebreak
    \leftskip=#2cm
    \rightskip=#3cm
}
{
    \par
}
\fi

\doendnotes{C}
\bigskip
\vfill

\clearpage

\footnotesize

\ifkorrekturansicht
  \lohead{\textsc{register}}
\fi

% theindex-Environment neu definieren ohne reledmac
\makeatletter
\renewenvironment{theindex}{%
  \ifkorrekturansicht
    \section*{\indexname}%
  \else
    \subsubsection*{Index der erwähnten Entitäten}%
  \fi
  \setlength{\parindent}{0pt}%
  \setlength{\parskip}{0pt plus 0.3pt}%
  \let\item\@idxitem
}{%
  \ifkorrekturansicht\clearpage\fi
}
\makeatother

\IfFileExists{\jobname-pw.ind}{\input{\jobname-pw.ind}}{}

% Quellenangabe nur in der Leseansicht
\ifkorrekturansicht\else
% Fallback-Definitionen, falls die .tex-Datei \titel etc. nicht gesetzt hat
\providecommand{\titel}{}
\providecommand{\editorInnen}{}
\providecommand{\dateiname}{\jobname}

\vspace{3cm}

\vfill

\footnotesize
\textsc{Quelle}: \titel. Herausgegeben von {\editorInnen}. In: \emph{Arthur Schnitzler: Briefwechsel mit Autorinnen und Autoren}.
 Digitale Edition, https://schnitzler-briefe.acdh.oeaw.ac.at/{\dateiname}.html (Stand \today)
\fi

\end{document}


