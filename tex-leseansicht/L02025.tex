%% latex-korrekturansicht-vorspann.tex
%% Vorspann für die Korrekturansicht.
%% Lädt die gemeinsame Datei latex-vorspann.tex mit gesetztem Schalter.

\newif\ifkorrekturansicht
\korrekturansichttrue

\input{../tex-inputs/latex-vorspann}


\section[Max Burckhard an Arthur Schnitzler, 23. 8. 1911]{L02025 Max Burckhard an Arthur Schnitzler, 23. 8. 1911}
\nopagebreak\mylabel{L02025v}
\rehead{ }\normalsize\beginnumbering\briefempfaengerindex{Schnitzler, Arthur@\textsc{Schnitzler, Arthur}!zzzBurckhard, Max Eugen@\emph{von Max Eugen Burckhard}!1911-08-231@{23. 8. 1911}|(be}
\toendnotes[C]{\smallbreak\pagebreak[2]}\Standort{CUL, Schnitzler, B 20.}
\physDesc{Brief, 1 Blatt, 1 Seite, 1130 Zeichen
\newline{}Schreibmaschine
\newline{}Handschrift: schwarze Tinte (\noindent{}Unterschrift)
\newline{}Schnitzler: mit rotem Buntstift eine Unterstreichung 
\newline{}Ordnung: mit Bleistift von unbekannter Hand nummeriert:
                                    »28« }\toendnotes[C]{\smallbreak}
\pstart
           {\pb}\textcolor{gray}{\textbf{\textsc{D\textsuperscript{r.} Max Burckhard}}}\hfill \textcolor{gray}{\textbf{Wien, I. Lichtenfelsgasse 7\oindex{Lichtenfelsgasse@\textbf{Lichtenfelsgasse}, \emph{Straße (K.STR)}|pw}}}\pend
           
\pstart
           \raggedleft{}\textcolor{gray}{\textbf{St. Gilgen\oindex{St. Gilgen@\textbf{St. Gilgen}, \emph{A.ADM3}|pw}}}{ }23. 8. 11.\pend
           
\pstart{}Sehr verehrter lieber Herr Doctor!\pend\vspace{0.5em}
\pstart
           Herzlichsten Dank für die Zusendung des »weiten
                  Landes\pwindex{weite Land. Tragikomoedie in fuenf Akten@\emph{Das weite Land. Tragikomödie in fünf Akten}|pw}«, das mich natürlich, wie alles von Ihnen sehr interessiert hat und
               das auch durch die Personen sehr stark auf mich gewirkt hat. Freilich hat es mich
               jetzt sehr traurig ergriffen, da das Vorbild\pwindex{Christomannos, Theodor 31.07.1854 – 30.01.1911@\textsc{Christomannos, Theodor} (31.07.1854 – 30.01.1911), \emph{Politiker/Politikerin, Rechtswissenschaftler/Rechtswissenschaftlerin, Rechtsanwalt/Rechtsanwältin}|pwv} Dr. Aigners\pwindex{weite Land. Tragikomoedie in fuenf Akten@\emph{Das weite Land. Tragikomödie in fünf Akten}|pwv} inzwischen von uns gegangen ist, und ich diesem
               prächtigen Menschen von Herzen zugethan war. Ich habe übrigens zufällig noch eine
               andere gute Bekannte in dem Stück gefunden (wenn auch Sie sie vielleicht gar nicht
               als dieselbe Person kennen); im Leben hat sich nemlich die »kritische Scene« zwischen
                  Erna\pwindex{weite Land. Tragikomoedie in fuenf Akten@\emph{Das weite Land. Tragikomödie in fünf Akten}|pwv} und Türk\pwindex{weite Land. Tragikomoedie in fuenf Akten@\emph{Das weite Land. Tragikomödie in fünf Akten}|pwv} (unter welchem Spitznamen Ihnen wol
                  Christomanos\pwindex{Christomannos, Theodor 31.07.1854 – 30.01.1911@\textsc{Christomannos, Theodor} (31.07.1854 – 30.01.1911), \emph{Politiker/Politikerin, Rechtswissenschaftler/Rechtswissenschaftlerin, Rechtsanwalt/Rechtsanwältin}|pw} auch bekannt worden sein wird)
               abgespielt. Jedenfalls glich sie Erna\pwindex{weite Land. Tragikomoedie in fuenf Akten@\emph{Das weite Land. Tragikomödie in fünf Akten}|pwv}{ }sehr in ihrer Art und obwol wir uns nur sehr selten
               sprachen, waren wir doch sehr gut (»im guten Sinne«). Inzwischen wird sie wol auch
               älter geworden sein, was ja bekanntlich den Menschen gewöhnlich nicht zum Vorteil
               gereicht.\pend
           
\pstart
           Sehr leid war es mir, daß ich heuer nicht mehr von Ihrer Anwesenheit haben konnte.
               Mit Handkuss an die verehrte gnädige Frau\pwindex{Schnitzler, Olga 17.01.1882 – 13.01.1970@\textsc{Schnitzler, Olga} (17.01.1882 – 13.01.1970), \emph{Schauspieler/Schauspielerin, Sänger/Sängerin}|pwv} und herzlichsten Grüßen Ihr treu ergebener\pend
           \pstart \spacefill\mbox{{[}hs.:{]} D\textsuperscript{r}Burckhard}\pend{}\selectlanguage{ngerman}\endnumbering\briefempfaengerindex{Schnitzler, Arthur@\textsc{Schnitzler, Arthur}!zzzBurckhard, Max Eugen@\emph{von Max Eugen Burckhard}!1911-08-231@{23. 8. 1911}|)be}\mylabel{L02025h}  \normalsize

\doendnotes{C}
\bigskip
\vfill

\clearpage

\footnotesize

\lohead{\textsc{register}}

% Definiere theindex-Environment komplett neu ohne reledmac
\makeatletter
\renewenvironment{theindex}{%
  \section*{\indexname}%
  \setlength{\parindent}{0pt}%
  \setlength{\parskip}{0pt plus 0.3pt}%
  \let\item\@idxitem
}{%
  \clearpage
}
\makeatother

\IfFileExists{\jobname-pw.ind}{\input{\jobname-pw.ind}}{}

\end{document}

      