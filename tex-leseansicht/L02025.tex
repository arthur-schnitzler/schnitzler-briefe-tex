%% latex-leseansicht-vorspann.tex
%% Vorspann für die Leseansicht.
%% Lädt die gemeinsame Datei latex-vorspann.tex mit nicht gesetztem Schalter.

\newif\ifkorrekturansicht
\korrekturansichtfalse

\input{../tex-inputs/latex-vorspann}


\section[Max Burckhard an Arthur Schnitzler, 23. 8. 1911]{L02025 Max Burckhard an Arthur Schnitzler, 23. 8. 1911}
\nopagebreak\mylabel{L02025v}
\rehead{ }\normalsize\beginnumbering\briefempfaengerindex{Schnitzler, Arthur@\textsc{Schnitzler, Arthur}!zzzBurckhard, Max Eugen@\emph{von Max Eugen Burckhard}!1911-08-231@{23. 8. 1911}|(be}
\toendnotes[C]{\smallbreak\pagebreak[2]}
\correspDesc{Versand  durch Max Burckhard am 23. 8. 1911 in St. Gilgen
\newline{}Erhalt  durch Arthur Schnitzler im Zeitraum [24. 8. 1911
                  – 28. 8. 1911?] in Wien}\toendnotes[C]{\smallbreak}
\Standort{CUL, Schnitzler, B 20.}
\physDesc{Brief, 1 Blatt, 1 Seite, 1130 Zeichen
\newline{}Schreibmaschine
\newline{}Handschrift: schwarze Tinte (\noindent{}Unterschrift)
\newline{}Schnitzler: mit rotem Buntstift eine Unterstreichung 
\newline{}Ordnung: mit Bleistift von unbekannter Hand nummeriert:
                                    »28« }\toendnotes[C]{\smallbreak}
\pstart
           {\pb}\textcolor{gray}{\textbf{\textsc{D\textsuperscript{r.} Max Burckhard}}}\hfill \textcolor{gray}{\textbf{Wien, I. Lichtenfelsgasse 7\oindex{Wien@\textbf{Wien}!I., Innere Stadt@\textbf{I., Innere Stadt}!Lichtenfelsgasse@\textbf{Lichtenfelsgasse}, \emph{Straße}|pw}}}\pend
           
\pstart
           \raggedleft{}\textcolor{gray}{\textbf{St. Gilgen\oindex{St. Gilgen@\textbf{St. Gilgen}, \emph{Verwaltungsgebiet}|pw}}}{ }23. 8. 11.\pend
           
\pstart{}Sehr verehrter lieber Herr Doctor!\pend\vspace{0.5em}
\pstart
           Herzlichsten Dank für die Zusendung des »weiten
                  Landes\pwindex{Schnitzler, Arthur 15.\,5.\,1862 Wien – 21.\,10.\,1931 ebd.@\textsc{Schnitzler, Arthur} (15.\,5.\,1862 Wien – 21.\,10.\,1931 ebd.), \emph{Schriftsteller, Mediziner}!weite Land. Tragikomödie in fünf Akten@\strich\emph{Das weite Land. Tragikomödie in fünf Akten}|pw}«, das mich natürlich, wie alles von Ihnen sehr interessiert hat und
               das auch durch die Personen sehr stark auf mich gewirkt hat. Freilich hat es mich
               jetzt sehr traurig ergriffen, da das Vorbild\pwindex{Christomannos, Theodor 31.\,7.\,1854 Wien – 30.\,1.\,1911 Meran@\textsc{Christomannos, Theodor} (31.\,7.\,1854 Wien – 30.\,1.\,1911 Meran), \emph{Politiker, Rechtswissenschaftler, Rechtsanwalt}|pwv} Dr. Aigners\pwindex{Schnitzler, Arthur 15.\,5.\,1862 Wien – 21.\,10.\,1931 ebd.@\textsc{Schnitzler, Arthur} (15.\,5.\,1862 Wien – 21.\,10.\,1931 ebd.), \emph{Schriftsteller, Mediziner}!weite Land. Tragikomödie in fünf Akten@\strich\emph{Das weite Land. Tragikomödie in fünf Akten}|pwv} inzwischen von uns gegangen ist, und ich diesem
               prächtigen Menschen von Herzen zugethan war. Ich habe übrigens zufällig noch eine
               andere gute Bekannte in dem Stück gefunden (wenn auch Sie sie vielleicht gar nicht
               als dieselbe Person kennen); im Leben hat sich nemlich die »kritische Scene« zwischen
                  Erna\pwindex{Schnitzler, Arthur 15.\,5.\,1862 Wien – 21.\,10.\,1931 ebd.@\textsc{Schnitzler, Arthur} (15.\,5.\,1862 Wien – 21.\,10.\,1931 ebd.), \emph{Schriftsteller, Mediziner}!weite Land. Tragikomödie in fünf Akten@\strich\emph{Das weite Land. Tragikomödie in fünf Akten}|pwv} und Türk\pwindex{Schnitzler, Arthur 15.\,5.\,1862 Wien – 21.\,10.\,1931 ebd.@\textsc{Schnitzler, Arthur} (15.\,5.\,1862 Wien – 21.\,10.\,1931 ebd.), \emph{Schriftsteller, Mediziner}!weite Land. Tragikomödie in fünf Akten@\strich\emph{Das weite Land. Tragikomödie in fünf Akten}|pwv} (unter welchem Spitznamen Ihnen wol
                  Christomanos\pwindex{Christomannos, Theodor 31.\,7.\,1854 Wien – 30.\,1.\,1911 Meran@\textsc{Christomannos, Theodor} (31.\,7.\,1854 Wien – 30.\,1.\,1911 Meran), \emph{Politiker, Rechtswissenschaftler, Rechtsanwalt}|pw} auch bekannt worden sein wird)
               abgespielt. Jedenfalls glich sie Erna\pwindex{Schnitzler, Arthur 15.\,5.\,1862 Wien – 21.\,10.\,1931 ebd.@\textsc{Schnitzler, Arthur} (15.\,5.\,1862 Wien – 21.\,10.\,1931 ebd.), \emph{Schriftsteller, Mediziner}!weite Land. Tragikomödie in fünf Akten@\strich\emph{Das weite Land. Tragikomödie in fünf Akten}|pwv}{ }sehr in ihrer Art und obwol wir uns nur sehr selten
               sprachen, waren wir doch sehr gut (»im guten Sinne«). Inzwischen wird sie wol auch
               älter geworden sein, was ja bekanntlich den Menschen gewöhnlich nicht zum Vorteil
               gereicht.\pend
           
\pstart
           Sehr leid war es mir, daß ich heuer nicht mehr von Ihrer Anwesenheit haben konnte.
               Mit Handkuss an die verehrte gnädige Frau\pwindex{Schnitzler, Olga 17.\,1.\,1882 Wien – 13.\,1.\,1970 Lugano@\textsc{Schnitzler, Olga} (17.\,1.\,1882 Wien – 13.\,1.\,1970 Lugano), \emph{Schauspielerin, Sängerin}|pwv} und herzlichsten Grüßen Ihr treu ergebener\pend
           \pstart \spacefill\mbox{{[}hs.:{]} D\textsuperscript{r}Burckhard}\pend{}\selectlanguage{ngerman}\endnumbering\briefempfaengerindex{Schnitzler, Arthur@\textsc{Schnitzler, Arthur}!zzzBurckhard, Max Eugen@\emph{von Max Eugen Burckhard}!1911-08-231@{23. 8. 1911}|)be}\mylabel{L02025h}  \newcommand{\dateiname}{L02025}\newcommand{\titel}{Max Burckhard an Arthur Schnitzler, 23. 8. 1911}\newcommand{\editorInnen}{Martin Anton Müller und Gerd-Hermann Susen}%% latex-leseansicht-abspann.tex
%% Abspann für die Leseansicht.
%% Der Schalter \ifkorrekturansicht ist bereits durch den Vorspann gesetzt.

%% latex-abspann.tex
%% Gemeinsamer Abspann für Korrekturansicht und Leseansicht.
%% Setzt den Schalter \ifkorrekturansicht voraus (gesetzt in den
%% einbindenden Dateien latex-korrekturansicht-abspann.tex bzw.
%% latex-leseansicht-abspann.tex).
%% ---------------------------------------------------------------

\normalsize

% Das esempio-Environment wird nur in der Leseansicht benötigt
\ifkorrekturansicht\else
\newenvironment{esempio}[3]%
{
    \vspace{1.5ex}
    \rlap{\underline{#1}}
    \par
    \setlength{\parindent}{0cm}
    \nopagebreak
    \leftskip=#2cm
    \rightskip=#3cm
}
{
    \par
}
\fi

\doendnotes{C}
\bigskip
\vfill

\clearpage

\footnotesize

\ifkorrekturansicht
  \lohead{\textsc{register}}
\fi

% theindex-Environment neu definieren ohne reledmac
\makeatletter
\renewenvironment{theindex}{%
  \ifkorrekturansicht
    \section*{\indexname}%
  \else
    \subsubsection*{Index der erwähnten Entitäten}%
  \fi
  \setlength{\parindent}{0pt}%
  \setlength{\parskip}{0pt plus 0.3pt}%
  \let\item\@idxitem
}{%
  \ifkorrekturansicht\clearpage\fi
}
\makeatother

\IfFileExists{\jobname-pw.ind}{\input{\jobname-pw.ind}}{}

% Quellenangabe nur in der Leseansicht
\ifkorrekturansicht\else
% Fallback-Definitionen, falls die .tex-Datei \titel etc. nicht gesetzt hat
\providecommand{\titel}{}
\providecommand{\editorInnen}{}
\providecommand{\dateiname}{\jobname}

\vspace{3cm}

\vfill

\footnotesize
\textsc{Quelle}: \titel. Herausgegeben von {\editorInnen}. In: \emph{Arthur Schnitzler: Briefwechsel mit Autorinnen und Autoren}.
 Digitale Edition, https://schnitzler-briefe.acdh.oeaw.ac.at/{\dateiname}.html (Stand \today)
\fi

\end{document}


