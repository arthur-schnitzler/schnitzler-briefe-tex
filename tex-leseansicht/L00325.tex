%% latex-leseansicht-vorspann.tex
%% Vorspann für die Leseansicht.
%% Lädt die gemeinsame Datei latex-vorspann.tex mit nicht gesetztem Schalter.

\newif\ifkorrekturansicht
\korrekturansichtfalse

\input{../tex-inputs/latex-vorspann}

\begin{center}
            \textcolor{red}{ENTWURF. ENTZIFFERUNG NOCH NICHT KORREKTURGELESEN}
                      \end{center}
            
               \section[Lou Andreas-Salomé an Arthur Schnitzler, 15. 5. 1894]{ Lou Andreas-Salomé an Arthur Schnitzler, 15. 5. 1894}\nopagebreak\mylabel{v}\rehead{ }\begin{ledgroupsized}[t]{13cm}\normalsize\beginnumbering\briefempfaengerindex{Schnitzler, Arthur@\textsc{Schnitzler, Arthur}!zzzAndreas-Salome, Lou@\emph{von Lou Andreas-Salomé}!1894-05-153@{15. 5. 1894}|(be} \toendnotes[C]{\smallbreak\pagebreak[2]} \Standort{CUL, Schnitzler, B 3.}
\physDesc{Brief, 2 Blätter, 6 Seiten
\newline{}Handschrift: schwarze Tinte, deutsche Kurrent
\newline{}Schnitzler: mit Bleistift beschriftet »\textsc{Lou Andrea-Salomé}« und datiert: »15/5 94« \newline{}Ordnung: mit rotem Buntstift von unbekannter Hand nummeriert:
                                        »1.« }\pstart{}{\pb}Sehr geehrter Herr,\pend\pstart
           ich habe kürzlich erſt Ihren »\textsc{Anatol}\pwindex{Schnitzler, Arthur 15.05.1862 – 21.10.1931@\textsc{Schnitzler, Arthur} (15.05.1862 – 21.10.1931), \emph{Schriftsteller, Mediziner}!Anatol1892-10-29 – 1892-10-29@\strich\emph{Anatol} {[}1892-10-29 – 1892-10-29{]}|pw}« kennen gelernt und, Dank der Freundlichkeit des Herrn \textsc{D\textsuperscript{r} }\textsc{Goldmann}\pwindex{Goldmann, Paul 31.01.1865 – 25.09.1935@\textsc{Goldmann, Paul} (31.01.1865 – 25.09.1935), \emph{Schriftsteller, Journalist}|pw}, darauf auch die beiden Manuſkripte »\textsc{Eine überspannte Person}\pwindex{Schnitzler, Arthur 15.05.1862 – 21.10.1931@\textsc{Schnitzler, Arthur} (15.05.1862 – 21.10.1931), \emph{Schriftsteller, Mediziner}!ueberspannte Person18. 04. 1896@\strich\emph{Die überspannte Person} {[}18. 04. 1896{]}|pw}« und »\textsc{Halb zwei}\pwindex{Schnitzler, Arthur 15.05.1862 – 21.10.1931@\textsc{Schnitzler, Arthur} (15.05.1862 – 21.10.1931), \emph{Schriftsteller, Mediziner}!Halb Zwei01. 04. 1897@\strich\emph{Halb Zwei} {[}01. 04. 1897{]}|pw}« leſen dürfen. Das war ein großer Genuß, ſo groß, wie ihn nur die
                    echteſten Bücher geben. Wenn man ſich hinterher darüber klar zu werden verſucht,
                    was ihn in jedem einzelnen Fall bedingt hat, ſo ſteht man überraſcht vor der
                    Fülle von Talent, die zuſammmenſtrömen mußte, um dieſe feinen Sachen zu
                    ſchaffen. Denn es iſt eine Verbindung von Geiſt, Geſtaltungskraft und
                    dichteriſcher Stimmung in ihnen, wie ſie gewiß ſelten vorkommt. Und doch iſt es
                    nicht einmal dies, was ich am meiſten {\pb}daran bewundere, ſondern daß es gelang, etwas an ſich Gehaltvolles mit ſo
                    unvergleichlich leichter und zarter Hand zu formen, daß es in den Feinheiten der
                    graziöſen Form gleichſam verflüchtigt wird. Man erhält, wie im Tanz, das Gefühl
                    der aufgehobenen Schwere eines Gegenſtandes. Und dennoch bleibt der Eindruck des
                    Gehaltvollen, Inhaltvollen, nach beendeter Lektüre beſtehen, ja er verſtärkt
                    ſich noch, indem man die einzelnen Scenen unwillkürlich noch vorwärts und
                    rückwärts weiterſpinnt, als handle es ſich um ein geſchautes Stück wirklichen
                    Lebens mit offenen Perſpektiven nach beiden Seiten. Im »\textsc{Anatol}\pwindex{Schnitzler, Arthur 15.05.1862 – 21.10.1931@\textsc{Schnitzler, Arthur} (15.05.1862 – 21.10.1931), \emph{Schriftsteller, Mediziner}!Anatol1892-10-29 – 1892-10-29@\strich\emph{Anatol} {[}1892-10-29 – 1892-10-29{]}|pw}« gilt dies am meiſten von »\textsc{Weihnachtseinkäufe}\pwindex{Schnitzler, Arthur 15.05.1862 – 21.10.1931@\textsc{Schnitzler, Arthur} (15.05.1862 – 21.10.1931), \emph{Schriftsteller, Mediziner}!Weihnachts-Einkaeufe24. 12. 1891@\strich\emph{Weihnachts-Einkäufe} {[}24. 12. 1891{]}|pw}« und »\textsc{Denksteine}\pwindex{Schnitzler, Arthur 15.05.1862 – 21.10.1931@\textsc{Schnitzler, Arthur} (15.05.1862 – 21.10.1931), \emph{Schriftsteller, Mediziner}!Denksteine15. 05. 1891@\strich\emph{Denksteine} {[}15. 05. 1891{]}|pw}«, und im höchſten Grade von den beiden Manuſkripten, die, meiner
                    Empfindung nach, den »\textsc{Anatol}\pwindex{Schnitzler, Arthur 15.05.1862 – 21.10.1931@\textsc{Schnitzler, Arthur} (15.05.1862 – 21.10.1931), \emph{Schriftsteller, Mediziner}!Anatol1892-10-29 – 1892-10-29@\strich\emph{Anatol} {[}1892-10-29 – 1892-10-29{]}|pw}« übertreffen. Das eine derſelben, »\textsc{Eine überspannte Person}\pwindex{Schnitzler, Arthur 15.05.1862 – 21.10.1931@\textsc{Schnitzler, Arthur} (15.05.1862 – 21.10.1931), \emph{Schriftsteller, Mediziner}!ueberspannte Person18. 04. 1896@\strich\emph{Die überspannte Person} {[}18. 04. 1896{]}|pw}«, war mir auch noch beſonders merkwürdig wegen der Art, wie hier die
                    Frau von den Frauen {\pb}in allen übrigen
                    Einaktern angehoben wird, und wegen der ironiſchen Beleuchtung die, ſchon vom
                    vortrefflichen Titel aus, hier auf den Mann fällt. Es wäre intereſſant, dieſes
                    kleine Drama nach einer beſtimmten Seite hin in Vergleich zu ziehen mit
                        »\textsc{Ein Märchen}\pwindex{Schnitzler, Arthur 15.05.1862 – 21.10.1931@\textsc{Schnitzler, Arthur} (15.05.1862 – 21.10.1931), \emph{Schriftsteller, Mediziner}!Maerchen. Schauspiel in drei Aufzuegen1891 – 1891@\strich\emph{Das Märchen. Schauspiel in drei Aufzügen} {[}1891 – 1891{]}|pw}«, welches ja wahrhaftig ebenſo gut heißen könnte: »\textsc{Ein überspannter Mann},« – und zwar \uline{ohne} ironiſchen Nebenklang im Titel. Wird man nicht
                    davon frappirt, wie einfach, ſelbſtverſtändlich und natürlich das Gefühl in der
                    »überſpannten« Frau, und wie gänzlich verdreht und verbildet es dagegen im
                    überſpannten Mann iſt? Mann und Frau, ſo einander gegenübergeſtellt, nehmen ſich
                    faſt wie Krankheit und Geſundheit aus. Und verräth es nicht etwas, \strikeout{d}wenn ein Autor, um die Frau in ihrer tiefern
                    Liebesempfindung zu ſchildern, nur auf das Nächſte, Natürlichſte zurückzugreifen
                    braucht, während er im gleichen Fall beim Mann ſogleich in {\pb}eine ganze Wirrniß von zwieſpältigen
                    verzwickten und widerſpruchsvollen Empfindungen hineingeräth? Auf mich hat das
                        »\textsc{Märchen}\pwindex{Schnitzler, Arthur 15.05.1862 – 21.10.1931@\textsc{Schnitzler, Arthur} (15.05.1862 – 21.10.1931), \emph{Schriftsteller, Mediziner}!Maerchen. Schauspiel in drei Aufzuegen1891 – 1891@\strich\emph{Das Märchen. Schauspiel in drei Aufzügen} {[}1891 – 1891{]}|pw}« weit ſchwächer gewirkt als der »\textsc{Anatol}\pwindex{Schnitzler, Arthur 15.05.1862 – 21.10.1931@\textsc{Schnitzler, Arthur} (15.05.1862 – 21.10.1931), \emph{Schriftsteller, Mediziner}!Anatol1892-10-29 – 1892-10-29@\strich\emph{Anatol} {[}1892-10-29 – 1892-10-29{]}|pw}« und es kam mir vor, als ſei eine viel geringere poetiſche und
                    plaſtiſche Kraft darin lebendig, aber der Grund kann auch ſein, daß ich Ihren
                    Märchenhelden abſolut nicht leiden mag und deshalb dem Autor Unrecht thue.
                    Auffallend iſt es, wie ſchlecht der Mann überhaupt in Ihren Dichtungen wegkommt,
                    – ſo ſchlecht, daß man verſucht iſt, an ein klein wenig Verläumdung zu glauben.
                    Gleichviel ob er ſich als der verhältnißmäßig Bravere oder Böſere giebt, – immer
                    iſt er, neben der Frau, der Unintereſſantere. Alle dieſe Frauen ſind ihm, und
                    wäre es auch nur in der Unſchuld ihrer Nichtsnutzigkeit, irgendwie überlegen.
                    Eine wunderliche Sorte von Selbſtverleugnung \introOben{}des Autors\introOben{}
                    liegt in faſt jedem Strich, mit dem der Mann den Frauen gegenüber geſchildert
                    iſt, {\pb}– wer den Mann ſo ſchildert,
                    räumt der Frau damit den Platz. Ich kann in den von Ihnen gewählten Fällen die
                    Richtigkeit Ihrer Darſtellung in dieſem Punkt nicht recht beurtheilen, aber
                    natürlich bin ich, als Frau, außerordentlich bereit, ihr ohne Weiteres jede nur
                    denkbare Lebenswahrheit zuzugeſtehn. –\pend
           \pstart
           Sie werden gewiß etwas verwundert ſein, wenn dieſer gänzlich überflüſſige Brief
                    Ihnen zukommt, doch das hat Ihr Freund, Herr \textsc{D\textsuperscript{r} }\textsc{Goldmann}\pwindex{Goldmann, Paul 31.01.1865 – 25.09.1935@\textsc{Goldmann, Paul} (31.01.1865 – 25.09.1935), \emph{Schriftsteller, Journalist}|pw}, ganz und gar auf ſeinem Gewiſſen. Ich hätte ſonſt vielleicht
                    beſcheidentlich den Mund gehalten, da es nach meiner Erfahrung nur wenig oder
                    gar keine Freude macht, Stimmen aus dem Publikum über Arbeiten zu vernehmen, die
                    einem doch an's Herz gewachſen ſind, wenn ſie was taugen. Nur die paar ſeltenen
                    Menſchen, die man liebt oder die man fürchtet, ſollte man darüber hören. Denn
                    das, was man am liebſten hat, theilt man ja {\pb}nicht leicht und nicht gern mit
                    vielen Andern, und noch weniger gern läßt man es von Andern analyſiren und
                    begucken, ganz einerlei ob Lob oder Tadel dabei herauskommt.\pend
           \pstart
           In jedem Fall aber wollte dieſe Schreiberei Ihnen herzlichen Dank ſagen für
                    gute Stunden.{\\[\baselineskip]}\spacefill\mbox{Lou Andreas-Salomé.}\pend
           \leftskip=0em{}\pstart
           Paris\oindex{Paris@\textbf{Paris}|pw}, 15. V. 94.
                    \pend
           \endnumbering\briefempfaengerindex{Schnitzler, Arthur@\textsc{Schnitzler, Arthur}!zzzAndreas-Salome, Lou@\emph{von Lou Andreas-Salomé}!1894-05-153@{15. 5. 1894}|)be}\mylabel{h}\end{ledgroupsized}  \newcommand{\dateiname}{L00325}\newcommand{\titel}{Lou Andreas-Salomé an Arthur Schnitzler, 15. 5. 1894}\newcommand{\editorInnen}{Martin Anton Müller und Gerd-Hermann Susen}%% latex-leseansicht-abspann.tex
%% Abspann für die Leseansicht.
%% Der Schalter \ifkorrekturansicht ist bereits durch den Vorspann gesetzt.

%% latex-abspann.tex
%% Gemeinsamer Abspann für Korrekturansicht und Leseansicht.
%% Setzt den Schalter \ifkorrekturansicht voraus (gesetzt in den
%% einbindenden Dateien latex-korrekturansicht-abspann.tex bzw.
%% latex-leseansicht-abspann.tex).
%% ---------------------------------------------------------------

\normalsize

% Das esempio-Environment wird nur in der Leseansicht benötigt
\ifkorrekturansicht\else
\newenvironment{esempio}[3]%
{
    \vspace{1.5ex}
    \rlap{\underline{#1}}
    \par
    \setlength{\parindent}{0cm}
    \nopagebreak
    \leftskip=#2cm
    \rightskip=#3cm
}
{
    \par
}
\fi

\doendnotes{C}
\bigskip
\vfill

\clearpage

\footnotesize

\ifkorrekturansicht
  \lohead{\textsc{register}}
\fi

% theindex-Environment neu definieren ohne reledmac
\makeatletter
\renewenvironment{theindex}{%
  \ifkorrekturansicht
    \section*{\indexname}%
  \else
    \subsubsection*{Index der erwähnten Entitäten}%
  \fi
  \setlength{\parindent}{0pt}%
  \setlength{\parskip}{0pt plus 0.3pt}%
  \let\item\@idxitem
}{%
  \ifkorrekturansicht\clearpage\fi
}
\makeatother

\IfFileExists{\jobname-pw.ind}{\input{\jobname-pw.ind}}{}

% Quellenangabe nur in der Leseansicht
\ifkorrekturansicht\else
% Fallback-Definitionen, falls die .tex-Datei \titel etc. nicht gesetzt hat
\providecommand{\titel}{}
\providecommand{\editorInnen}{}
\providecommand{\dateiname}{\jobname}

\vspace{3cm}

\vfill

\footnotesize
\textsc{Quelle}: \titel. Herausgegeben von {\editorInnen}. In: \emph{Arthur Schnitzler: Briefwechsel mit Autorinnen und Autoren}.
 Digitale Edition, https://schnitzler-briefe.acdh.oeaw.ac.at/{\dateiname}.html (Stand \today)
\fi

\end{document}


      