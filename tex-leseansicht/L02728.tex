%% latex-leseansicht-vorspann.tex
%% Vorspann für die Leseansicht.
%% Lädt die gemeinsame Datei latex-vorspann.tex mit nicht gesetztem Schalter.

\newif\ifkorrekturansicht
\korrekturansichtfalse

\input{../tex-inputs/latex-vorspann}


         
         \renewcommand{\erwaehntePersonen}{Personen: Hermann Bahr, Max Eugen Burckhard, Alfred Gold, Paul Goldmann, Josef Rosengart, Felix Salten, Adele Sandrock, Leopold Sonnemann}
         \renewcommand{\erwaehnteInstitutionen}{Institutionen: Burgtheater, Frankfurter Zeitung}
         \renewcommand{\erwaehnteOrte}{Orte: Bad Tölz, Bayern, Burgtheater, Frankreich, Paris, Wien, rue Feydeau}
         \renewcommand{\erwaehnteWerke}{Werke: Arthur Schnitzler: Sterben, Die Zeit. Wiener Wochenschrift, Liebelei. Schauspiel in drei Akten, Sterben. Novelle, Tagebuch}
               \section[Paul Goldmann an Arthur Schnitzler, 6. 2. {[}1895{]}]{ Paul Goldmann an Arthur Schnitzler, 6. 2. {[}1895{]}}\nopagebreak\mylabel{v}\rehead{ }\begin{ledgroupsized}[t]{13cm}\normalsize\beginnumbering \toendnotes[C]{\smallbreak\pagebreak[2]} \Standort{DLA, A:Schnitzler, HS.NZ85.1.3165.}
\physDesc{Brief, 2 Blätter, 7 Seiten, 2897 Zeichen
\newline{}Handschrift: schwarze Tinte, deutsche Kurrent
\newline{}Schnitzler: 1) mit Bleistift das Jahr »95« vermerkt  2) mit rotem Buntstift eine Unterstreichung}\toendnotes[C]{\smallbreak}\pstart
           \noindent{}{\pb}\textcolor{gray}{\textbf{\textbf{Frankfurter Zeitung\orgindex{Frankfurter Zeitung@Frankfurter Zeitung|pw}}}}\pend
           \pstart
           \textcolor{gray}{\textbf{(\begin{otherlanguage}{french}Gazette de Francfort\end{otherlanguage}\orgindex{Frankfurter Zeitung@Frankfurter Zeitung|pw}).}}\hfill \textsc{Paris}\oindex{Paris@\textbf{Paris}|pw}, 6. Februar.\pend
           \pstart
           \textcolor{gray}{\textbf{\textbf{\begin{otherlanguage}{french}Fondateur M. L.
                              Sonnemann\pwindex{Sonnemann, Leopold 1831-10-29 – 1909-10-30@\textsc{Sonnemann, Leopold} (1831-10-29 – 1909-10-30), \emph{Journalist, Herausgeber}|pw}\end{otherlanguage}.}}}\pend
           \pstart
           \begin{otherlanguage}{french}\textcolor{gray}{\textbf{Journal politique, financier,}}\end{otherlanguage}\pend
           \pstart
           \begin{otherlanguage}{french}\textcolor{gray}{\textbf{commercial et littéraire.}}\end{otherlanguage}\pend
           \pstart
           \begin{otherlanguage}{french}\textcolor{gray}{\textbf{\textbf{Paraissant trois fois par jour.}}}\end{otherlanguage}\pend
           \pstart
           \begin{otherlanguage}{french}\textcolor{gray}{\textbf{\textbf{Bureau à Paris\oindex{Paris@\textbf{Paris}|pw}:}}}\end{otherlanguage}\pend
           \pstart
           \begin{otherlanguage}{french}\textcolor{gray}{\textbf{\textbf{24. Rue Feydeau\oindex{rue Feydeau@\textbf{rue Feydeau}|pw}.}}}\end{otherlanguage}\pend
           \pstart\center{}Mein lieber Freund,\pend\pstart
           Ich hätte Dir Deinen Brief gern umgehend beantwortet, hatte aber gerade ausnahmsweis
               viel zu thun und komme nun erſt heut zur Antwort.\pend
           \pstart
           Was Du mir da ſchreibſt, aus einer Aufregung und Verſtimmung heraus, die noch an
               jedem Worte haften geblieben iſt, hat mich recht ſehr geſchmerzt. Freilich nur in dem
               Sinne, daß es mir unendlich leid thut, Dich inmitten all’ dieſer \label{K_L02728-1v}\edtext{Widerwärtigkeiten}{\lemma{\textnormal{\emph{Widerwärtigkeiten}}}\Cendnote{\textnormal{Wie Schnitzler\pwindex{Schnitzler, Arthur 15.05.1862 – 21.10.1931@\textsc{Schnitzler, Arthur} (15.05.1862 – 21.10.1931), \emph{Schriftsteller, Mediziner}|pwk} in
                  seinem \emph{Tagebuch}\pwindex{Schnitzler, Arthur 15.05.1862 – 21.10.1931@\textsc{Schnitzler, Arthur} (15.05.1862 – 21.10.1931), \emph{Schriftsteller, Mediziner}!Tagebuch1981 – 2000@\strich\emph{Tagebuch} {[}1981 – 2000{]}|pwk} ausführlich dokumentierte,
                  machte ihm in dieser Zeit vor allem die Beziehung zu Adele Sandrock\pwindex{Sandrock, Adele 1863-08-19 – 1937-08-30@\textsc{Sandrock, Adele} (1863-08-19 – 1937-08-30), \emph{Schauspielerin}|pwk} zu schaffen. Die Schauspielerin\pwindex{Sandrock, Adele 1863-08-19 – 1937-08-30@\textsc{Sandrock, Adele} (1863-08-19 – 1937-08-30), \emph{Schauspielerin}|pwkv}, mit der er – neben
                  anderen – ein Verhältnis führte, ging ein Verhältnis mit Felix Salten\pwindex{Salten, Felix 06.09.1869 – 08.10.1945@\textsc{Salten, Felix} (06.09.1869 – 08.10.1945), \emph{Schriftsteller, Journalist}|pwk} ein, nicht zuletzt um ihn eifersüchtig zu
                  machen. Als Schnitzler\pwindex{Schnitzler, Arthur 15.05.1862 – 21.10.1931@\textsc{Schnitzler, Arthur} (15.05.1862 – 21.10.1931), \emph{Schriftsteller, Mediziner}|pwk} die Beziehung
                  beendete, drohte Sandrock\pwindex{Sandrock, Adele 1863-08-19 – 1937-08-30@\textsc{Sandrock, Adele} (1863-08-19 – 1937-08-30), \emph{Schauspielerin}|pwk}, sich das Leben
                  zu nehmen. Er fürchtete auch, sie würde versuchen, die \emph{Liebelei}\pwindex{Schnitzler, Arthur 15.05.1862 – 21.10.1931@\textsc{Schnitzler, Arthur} (15.05.1862 – 21.10.1931), \emph{Schriftsteller, Mediziner}!Liebelei. Schauspiel in drei Akten1895-10-09@\strich\emph{Liebelei. Schauspiel in drei Akten} {[}1895-10-09{]}|pwk} vom \emph{Burgtheater}\orgindex{Burgtheater@Burgtheater|pwk} wieder abzusetzen. Laut Hermann Bahr\pwindex{Bahr, Hermann 19.07.1863 – 15.01.1934@\textsc{Bahr, Hermann} (19.07.1863 – 15.01.1934), \emph{Schriftsteller, Kritiker}|pwk} soll Sandrock\pwindex{Sandrock, Adele 1863-08-19 – 1937-08-30@\textsc{Sandrock, Adele} (1863-08-19 – 1937-08-30), \emph{Schauspielerin}|pwk} sogar
                  das Stück\pwindex{Schnitzler, Arthur 15.05.1862 – 21.10.1931@\textsc{Schnitzler, Arthur} (15.05.1862 – 21.10.1931), \emph{Schriftsteller, Mediziner}!Liebelei. Schauspiel in drei Akten1895-10-09@\strich\emph{Liebelei. Schauspiel in drei Akten} {[}1895-10-09{]}|pwkv} und ihre Rolle,
                  jene der Christine\pwindex{Schnitzler, Arthur 15.05.1862 – 21.10.1931@\textsc{Schnitzler, Arthur} (15.05.1862 – 21.10.1931), \emph{Schriftsteller, Mediziner}!Liebelei. Schauspiel in drei Akten1895-10-09@\strich\emph{Liebelei. Schauspiel in drei Akten} {[}1895-10-09{]}|pwkv}, auch
                  vor Max Eugen Burckhard\pwindex{Burckhard, Max Eugen 14.07.1854 – 16.03.1912@\textsc{Burckhard, Max Eugen} (14.07.1854 – 16.03.1912), \emph{Schriftsteller, Wissenschaftler, Theaterleiter}|pwk}, dem Leiter\pwindex{Burckhard, Max Eugen 14.07.1854 – 16.03.1912@\textsc{Burckhard, Max Eugen} (14.07.1854 – 16.03.1912), \emph{Schriftsteller, Wissenschaftler, Theaterleiter}|pwkv} des \emph{Burgtheater}\orgindex{Burgtheater@Burgtheater|pwk}s, schlechtgeredet und versucht haben, die
                  Aufführung des Stück\pwindex{Schnitzler, Arthur 15.05.1862 – 21.10.1931@\textsc{Schnitzler, Arthur} (15.05.1862 – 21.10.1931), \emph{Schriftsteller, Mediziner}!Liebelei. Schauspiel in drei Akten1895-10-09@\strich\emph{Liebelei. Schauspiel in drei Akten} {[}1895-10-09{]}|pwkv}s
                  hinauszuschieben, um Schnitzler\pwindex{Schnitzler, Arthur 15.05.1862 – 21.10.1931@\textsc{Schnitzler, Arthur} (15.05.1862 – 21.10.1931), \emph{Schriftsteller, Mediziner}|pwk}s
                  Aufmerksamkeit und Zuneigung zu erhalten. Bei der Uraufführung am 9. 10. 1895 am Burgtheater\oindex{Burgtheater@\textbf{Burgtheater}|pwk} spielte Sandrock\pwindex{Sandrock, Adele 1863-08-19 – 1937-08-30@\textsc{Sandrock, Adele} (1863-08-19 – 1937-08-30), \emph{Schauspielerin}|pwk} in der Hauptrolle.}}}\label{K_L02728-1h} zu wiſſen. {\pb}Um das Endreſultat\pwindex{Schnitzler, Arthur 15.05.1862 – 21.10.1931@\textsc{Schnitzler, Arthur} (15.05.1862 – 21.10.1931), \emph{Schriftsteller, Mediziner}!Liebelei. Schauspiel in drei Akten1895-10-09@\strich\emph{Liebelei. Schauspiel in drei Akten} {[}1895-10-09{]}|pwv} machen ſie mich nicht im Mindeſten bekümmert. Ich
               ſehe die Dinge von fern an, wie aus den Wolken. Da ſehe ich denn ein Schiff, das
               unaufhaltſam dem Ziele zufährt. Die einzelnen Zickzacklinien des Kurſes ſehe ich
               nicht. Ich ſehe nur, daß es vorwärts geht, nicht zurück – daß es nicht zurückgehen
               kann. Ein paar intriguante Weibsbild\pwindex{Sandrock, Adele 1863-08-19 – 1937-08-30@\textsc{Sandrock, Adele} (1863-08-19 – 1937-08-30), \emph{Schauspielerin}|pwv}er ſollen Dein Werk\pwindex{Schnitzler, Arthur 15.05.1862 – 21.10.1931@\textsc{Schnitzler, Arthur} (15.05.1862 – 21.10.1931), \emph{Schriftsteller, Mediziner}!Liebelei. Schauspiel in drei Akten1895-10-09@\strich\emph{Liebelei. Schauspiel in drei Akten} {[}1895-10-09{]}|pwv}{ }\strikeout{a\textcolor{gray}{n}} aufhalten, das mit der Kraft Deines Talentes dem Ziele zuſtrebt? Der Gedanke
               macht mich heiter, ſo unſinnig iſt er. {\pb}Und ich
               verliere meine Heiterkeit nur, wenn ich Deinen Brief wieder vornehme und Deine
               Verſtimmung herausleſe, die ich Dir gern erſpart wüßte. Aber ſchön! Du kämpfſt. Wer
               kämpft nicht? Und vergleiche Dein glückliches Loos, für ein hohes Ziel kämpfen zu
               dürfen, mit dem Anderer, mit dem meinen zum Beiſpiel, der ich mit Widerwärtigkeiten
               und tauſend Verhängiſſen ringen muß, nicht um hinaufzugelangen, wie Du, ſondern um
               nicht tiefer zu fallen, als ich ſchon ſtehe.\pend
           \pstart
           {\pb}Hab’ Geduld, mein lieber Freund! Sei ruhig und laß’
               die Dinge gehen, wie ſie gehen. Das Entſcheidende iſt bereits geſchehen: Du haſt ein
               ſchönes Stück\pwindex{Schnitzler, Arthur 15.05.1862 – 21.10.1931@\textsc{Schnitzler, Arthur} (15.05.1862 – 21.10.1931), \emph{Schriftsteller, Mediziner}!Liebelei. Schauspiel in drei Akten1895-10-09@\strich\emph{Liebelei. Schauspiel in drei Akten} {[}1895-10-09{]}|pwv}
                  geſchrieben{[}.{]} Alles Übrige iſt vollſtändig gleichgiltig. \strikeout{Laß\textcolor{gray}{’}} Laß’ Dich alſo nicht erregen. Blick’ weit hinaus in die Zukunft, laß’ Dich vom
               Tage nicht unterkriegen und vertrau’ auf Dich, wie ich auf Dich vertraue.\pend
           \pstart
           Das iſt freilich Alles recht vag und allgemein. Ich wünſchte, ich wüßte \strikeout{Nah} Näheres oder könnte gar bei Dir ſein, um {\pb}die Dinge im Einzelnen mit durchzuleben. Du ſollſt
               aber jedenfalls nicht glauben, daß Du mir ſchreiben mußt. Ich verſtehe es, daß Du
               wenig Stimmung zu Briefen findeſt, und warte ſchon meine Zeit ab. Nur möchte ich
               wiſſen, wann ungefähr die Aufführung\pwindex{Schnitzler, Arthur 15.05.1862 – 21.10.1931@\textsc{Schnitzler, Arthur} (15.05.1862 – 21.10.1931), \emph{Schriftsteller, Mediziner}!Liebelei. Schauspiel in drei Akten1895-10-09@\strich\emph{Liebelei. Schauspiel in drei Akten} {[}1895-10-09{]}|pwv} ſein wird; und wenn ſie dann iſt, möchte ich mir am nächſten Morgen eine Depeſche über das Reſultat
               erbitten.\pend
           \pstart
           Iſt \textsc{Bahr\pwindex{Bahr, Hermann 19.07.1863 – 15.01.1934@\textsc{Bahr, Hermann} (19.07.1863 – 15.01.1934), \emph{Schriftsteller, Kritiker}|pw}} nicht mit {\pb}unter denen, gegen die Du zu
               kämpfen haſt? Die \label{K_L02728-2v}\edtext{Kritik\pwindex{Arthur Schnitzler: Sterben1895-01-05@\emph{Arthur Schnitzler: Sterben} {[}1895-01-05{]}|pwv}}{\lemma{\textnormal{\emph{Kritik}}}\Cendnote{\textnormal{A. G.\pwindex{Gold, Alfred 28.06.1874 – 24.10.1958@\textsc{Gold, Alfred} (28.06.1874 – 24.10.1958), \emph{Schriftsteller, Journalist, Händler}|pwkv} [ = Alfred Gold\pwindex{Gold, Alfred 28.06.1874 – 24.10.1958@\textsc{Gold, Alfred} (28.06.1874 – 24.10.1958), \emph{Schriftsteller, Journalist, Händler}|pwk}]: Arthur Schnitzler: Sterben\pwindex{Arthur Schnitzler: Sterben1895-01-05@\emph{Arthur Schnitzler: Sterben} {[}1895-01-05{]}|pwkv}. In: \emph{Die Zeit}\pwindex{Zeit. Wiener Wochenschrift1894 – 1904@\emph{Die Zeit. Wiener Wochenschrift} {[}1894 – 1904{]}|pwk}, Bd. 2, Nr. 14, 5. 1. 1895, S. 14.}}}\label{K_L02728-2h} über »Sterben\pwindex{Schnitzler, Arthur 15.05.1862 – 21.10.1931@\textsc{Schnitzler, Arthur} (15.05.1862 – 21.10.1931), \emph{Schriftsteller, Mediziner}!Sterben. Novelle1894-10-01 – 1894-12-01@\strich\emph{Sterben. Novelle} {[}1894-10-01 – 1894-12-01{]}|pw}« in der »Zeit\pwindex{Zeit. Wiener Wochenschrift1894 – 1904@\emph{Die Zeit. Wiener Wochenschrift} {[}1894 – 1904{]}|pw}« war ebenſo dumm als \label{K_L02728-3v}\edtext{beſchmockt}{\lemma{\textnormal{\emph{beſchmockt}}}\Cendnote{\textnormal{pejorativ: auf Wirkung,
                  Effekt bedacht}}}\label{K_L02728-3h}.\pend
           \pstart
           Ich ſandte Dir dieſer Tage ein paar fran\oindex{Frankreich@\textbf{Frankreich}|pwv}zöſiſche \label{K_L02728-4v}\edtext{Zeitungsartikel}{\lemma{\textnormal{\emph{Zeitungsartikel}}}\Cendnote{\textnormal{nicht
                  überliefert}}}\label{K_L02728-4h}. Du findeſt darunter vielleicht Manches, das Dich zerſtreut.
               Kann ich Dir ſonſt was aus \textsc{Paris\oindex{Paris@\textbf{Paris}|pw}} ſchicken? Das Geſcheiteſte wäre, Du ließeſt den ganzen Kram in Wien\oindex{Wien@\textbf{Wien}|pw} im Stich und kämeſt auf vierzehn Tage hierher. Das würde
               Dir gut thun!\pend
           \pstart
           {\pb}Im Sommer werden wir uns \label{K_L02728-5v}\edtext{kaum ſehen}{\lemma{\textnormal{\emph{kaum ſehen}}}\Cendnote{\textnormal{Trotz
                     Goldmann\pwindex{Goldmann, Paul 31.01.1865 – 25.09.1935@\textsc{Goldmann, Paul} (31.01.1865 – 25.09.1935), \emph{Schriftsteller, Journalist}|pwk}s Kuraufenthalt in Bad Tölz\oindex{Bad Toelz@\textbf{Bad Tölz}|pwk} sahen sich die beiden zwischen 28. 8. 1895 und 6. 9. 1895 in Bayern\oindex{Bayern@\textbf{Bayern}|pwk}.}}}\label{K_L02728-5h} können. Ich werde \label{K_L02728-6v}\edtext{krank}{\lemma{\textnormal{\emph{krank}}}\Cendnote{\textnormal{siehe Paul Goldmann an Arthur Schnitzler, 2. 3. [1895]}}}\label{K_L02728-6h} und kränker, und mein Schwager\pwindex{Rosengart, Josef 1860-02-08 – 1927-08-04@\textsc{Rosengart, Josef} (1860-02-08 – 1927-08-04), \emph{Arzt}|pwv} beſteht darauf, daß ich während meines Urlaubs eine Kur gebrauche,
               vielleicht in \textsc{Toelz\oindex{Bad Toelz@\textbf{Bad Tölz}|pw}}, im bai\oindex{Bayern@\textbf{Bayern}|pwv}riſchen
               Hochgebirge.\pend
           \pstart
           Grüß’ Dich Gott, mein lieber Freund, und ſei guten Muths!\pend
           \pstart
           Dein {\\[\baselineskip]}treuer {\\[\baselineskip]}\spacefill\mbox{Paul Goldmann}\pend
           \leftskip=0em{}
         
         \endnumbering\mylabel{h}\end{ledgroupsized}  \newcommand{\dateiname}{L02728}\newcommand{\titel}{Paul Goldmann an Arthur Schnitzler, 6. 2. [1895]}\newcommand{\editorInnen}{Martin Anton Müller und Laura Untner}%% latex-leseansicht-abspann.tex
%% Abspann für die Leseansicht.
%% Der Schalter \ifkorrekturansicht ist bereits durch den Vorspann gesetzt.

%% latex-abspann.tex
%% Gemeinsamer Abspann für Korrekturansicht und Leseansicht.
%% Setzt den Schalter \ifkorrekturansicht voraus (gesetzt in den
%% einbindenden Dateien latex-korrekturansicht-abspann.tex bzw.
%% latex-leseansicht-abspann.tex).
%% ---------------------------------------------------------------

\normalsize

% Das esempio-Environment wird nur in der Leseansicht benötigt
\ifkorrekturansicht\else
\newenvironment{esempio}[3]%
{
    \vspace{1.5ex}
    \rlap{\underline{#1}}
    \par
    \setlength{\parindent}{0cm}
    \nopagebreak
    \leftskip=#2cm
    \rightskip=#3cm
}
{
    \par
}
\fi

\doendnotes{C}
\bigskip
\vfill

\clearpage

\footnotesize

\ifkorrekturansicht
  \lohead{\textsc{register}}
\fi

% theindex-Environment neu definieren ohne reledmac
\makeatletter
\renewenvironment{theindex}{%
  \ifkorrekturansicht
    \section*{\indexname}%
  \else
    \subsubsection*{Index der erwähnten Entitäten}%
  \fi
  \setlength{\parindent}{0pt}%
  \setlength{\parskip}{0pt plus 0.3pt}%
  \let\item\@idxitem
}{%
  \ifkorrekturansicht\clearpage\fi
}
\makeatother

\IfFileExists{\jobname-pw.ind}{\input{\jobname-pw.ind}}{}

% Quellenangabe nur in der Leseansicht
\ifkorrekturansicht\else
% Fallback-Definitionen, falls die .tex-Datei \titel etc. nicht gesetzt hat
\providecommand{\titel}{}
\providecommand{\editorInnen}{}
\providecommand{\dateiname}{\jobname}

\vspace{3cm}

\vfill

\footnotesize
\textsc{Quelle}: \titel. Herausgegeben von {\editorInnen}. In: \emph{Arthur Schnitzler: Briefwechsel mit Autorinnen und Autoren}.
 Digitale Edition, https://schnitzler-briefe.acdh.oeaw.ac.at/{\dateiname}.html (Stand \today)
\fi

\end{document}


      