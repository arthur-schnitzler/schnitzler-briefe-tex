%% latex-leseansicht-vorspann.tex
%% Vorspann für die Leseansicht.
%% Lädt die gemeinsame Datei latex-vorspann.tex mit nicht gesetztem Schalter.

\newif\ifkorrekturansicht
\korrekturansichtfalse

\input{../tex-inputs/latex-vorspann}


\section[Paul Goldmann an Arthur Schnitzler, 6. 2. {[}1895{]}]{L02728 Paul Goldmann an Arthur Schnitzler, 6. 2. [1895]}
\nopagebreak\mylabel{L02728v}
\rehead{ }\normalsize\beginnumbering\briefempfaengerindex{Schnitzler, Arthur@\textsc{Schnitzler, Arthur}!zzzGoldmann, Paul@\emph{von Paul Goldmann}!1895-02-061@{6. 2. [1895]}|(be}
\toendnotes[C]{\smallbreak\pagebreak[2]}
\correspDesc{Versand  durch Paul Goldmann am 6. 2. [1895] in Paris
\newline{}Erhalt  durch Arthur Schnitzler im Zeitraum [7. 2. 1895
                  – 11. 2. 1895?] in Wien}\toendnotes[C]{\smallbreak}
\Standort{DLA, A:Schnitzler, HS.NZ85.1.3165.}
\physDesc{Brief, 2 Blätter, 7 Seiten, 2897 Zeichen
\newline{}Handschrift: schwarze Tinte, deutsche Kurrent
\newline{}Schnitzler: 1) mit Bleistift das Jahr »95« vermerkt  2) mit rotem Buntstift eine Unterstreichung}\toendnotes[C]{\smallbreak}
\pstart
           {\pb}\textcolor{gray}{\textbf{\textbf{Frankfurter Zeitung\orgindex{Frankfurter Zeitung@Frankfurter Zeitung|pw}}}}\pend
           
\pstart
           \textcolor{gray}{\textbf{(\begin{otherlanguage}{french}Gazette de Francfort\end{otherlanguage}\orgindex{Frankfurter Zeitung@Frankfurter Zeitung|pw}).}}\hfill \textsc{Paris}\oindex{Paris@\textbf{Paris}, \emph{Hauptstadt}|pw}, 6. Februar.\pend
           
\pstart
           \textcolor{gray}{\textbf{\textbf{\begin{otherlanguage}{french}Fondateur M. L.
                              Sonnemann\pwindex{Sonnemann, Leopold 29.\,10.\,1831 Höchberg – 30.\,10.\,1909 Frankfurt am Main@\textsc{Sonnemann, Leopold} (29.\,10.\,1831 Höchberg – 30.\,10.\,1909 Frankfurt am Main), \emph{Journalist, Herausgeber}|pw}\end{otherlanguage}.}}}\pend
           
\pstart
           \begin{otherlanguage}{french}\textcolor{gray}{\textbf{Journal politique, financier,}}\end{otherlanguage}\pend
           
\pstart
           \begin{otherlanguage}{french}\textcolor{gray}{\textbf{commercial et littéraire.}}\end{otherlanguage}\pend
           
\pstart
           \begin{otherlanguage}{french}\textcolor{gray}{\textbf{\textbf{Paraissant trois fois par jour.}}}\end{otherlanguage}\pend
           
\pstart
           \begin{otherlanguage}{french}\textcolor{gray}{\textbf{\textbf{Bureau à Paris\oindex{Paris@\textbf{Paris}, \emph{Hauptstadt}|pw}:}}}\end{otherlanguage}\pend
           
\pstart
           \begin{otherlanguage}{french}\textcolor{gray}{\textbf{\textbf{24. Rue Feydeau\oindex{rue Feydeau@\textbf{rue Feydeau}, \emph{Straße}|pw}.}}}\end{otherlanguage}\pend
           
\pstart\center{}Mein lieber Freund,\pend\vspace{0.5em}
\pstart
           Ich hätte Dir Deinen Brief gern umgehend beantwortet, hatte aber gerade ausnahmsweis
               viel zu thun und komme nun erſt heut zur Antwort.\pend
           
\pstart
           Was Du mir da{ }ſchreibſt, aus einer Aufregung und Verſtimmung heraus, die noch an
               jedem Worte haften geblieben iſt, hat mich recht{ }ſehr geſchmerzt. Freilich nur in dem
               Sinne, daß es mir unendlich leid thut, Dich inmitten all’ dieſer \label{K_L02728-1v}\edtext{Widerwärtigkeiten}{\lemma{\textnormal{\emph{Widerwärtigkeiten}}}\Cendnote{\textnormal{Wie Schnitzler in
                  seinem \emph{Tagebuch}\pwindex{Schnitzler, Arthur 15.\,5.\,1862 Wien – 21.\,10.\,1931 ebd.@\textsc{Schnitzler, Arthur} (15.\,5.\,1862 Wien – 21.\,10.\,1931 ebd.), \emph{Schriftsteller, Mediziner}!Tagebuch@\strich\emph{Tagebuch}|pwk} ausführlich dokumentierte,
                  machte ihm in dieser Zeit vor allem die Beziehung zu Adele Sandrock\pwindex{Sandrock, Adele 19.\,8.\,1863 Rotterdam – 30.\,8.\,1937 Berlin@\textsc{Sandrock, Adele} (19.\,8.\,1863 Rotterdam – 30.\,8.\,1937 Berlin), \emph{Schauspielerin}|pwk} zu schaffen. Die Schauspielerin\pwindex{Sandrock, Adele 19.\,8.\,1863 Rotterdam – 30.\,8.\,1937 Berlin@\textsc{Sandrock, Adele} (19.\,8.\,1863 Rotterdam – 30.\,8.\,1937 Berlin), \emph{Schauspielerin}|pwkv}, mit der er – neben
                  anderen – ein Verhältnis führte, ging ein Verhältnis mit Felix Salten\pwindex{Salten, Felix 6.\,9.\,1869 Budapest – 8.\,10.\,1945 Zürich@\textsc{Salten, Felix} (6.\,9.\,1869 Budapest – 8.\,10.\,1945 Zürich), \emph{Schriftsteller, Journalist, Chefredakteur}|pwk} ein, nicht zuletzt, um ihn eifersüchtig zu
                  machen. Als Schnitzler die Beziehung
                  beendete, drohte Sandrock\pwindex{Sandrock, Adele 19.\,8.\,1863 Rotterdam – 30.\,8.\,1937 Berlin@\textsc{Sandrock, Adele} (19.\,8.\,1863 Rotterdam – 30.\,8.\,1937 Berlin), \emph{Schauspielerin}|pwk}, sich das Leben
                  zu nehmen. Er fürchtete auch, sie würde versuchen, \emph{Liebelei}\pwindex{Schnitzler, Arthur 15.\,5.\,1862 Wien – 21.\,10.\,1931 ebd.@\textsc{Schnitzler, Arthur} (15.\,5.\,1862 Wien – 21.\,10.\,1931 ebd.), \emph{Schriftsteller, Mediziner}!Liebelei. Schauspiel in drei Akten@\strich\emph{Liebelei. Schauspiel in drei Akten}|pwk} vom \emph{Burgtheater}\orgindex{Burgtheater@Burgtheater|pwk} wieder abzusetzen. Laut Hermann Bahr\pwindex{Bahr, Hermann 19.\,7.\,1863 Linz – 15.\,1.\,1934 München@\textsc{Bahr, Hermann} (19.\,7.\,1863 Linz – 15.\,1.\,1934 München), \emph{Schriftsteller, Kritiker}|pwk} soll Sandrock\pwindex{Sandrock, Adele 19.\,8.\,1863 Rotterdam – 30.\,8.\,1937 Berlin@\textsc{Sandrock, Adele} (19.\,8.\,1863 Rotterdam – 30.\,8.\,1937 Berlin), \emph{Schauspielerin}|pwk} sogar
                  das Stück\pwindex{Schnitzler, Arthur 15.\,5.\,1862 Wien – 21.\,10.\,1931 ebd.@\textsc{Schnitzler, Arthur} (15.\,5.\,1862 Wien – 21.\,10.\,1931 ebd.), \emph{Schriftsteller, Mediziner}!Liebelei. Schauspiel in drei Akten@\strich\emph{Liebelei. Schauspiel in drei Akten}|pwkv} und ihre Rolle,
                  jene der Christine\pwindex{Schnitzler, Arthur 15.\,5.\,1862 Wien – 21.\,10.\,1931 ebd.@\textsc{Schnitzler, Arthur} (15.\,5.\,1862 Wien – 21.\,10.\,1931 ebd.), \emph{Schriftsteller, Mediziner}!Liebelei. Schauspiel in drei Akten@\strich\emph{Liebelei. Schauspiel in drei Akten}|pwkv}, auch
                  vor Max Eugen Burckhard\pwindex{Burckhard, Max Eugen 14.\,7.\,1854 Korneuburg – 16.\,3.\,1912 Wien@\textsc{Burckhard, Max Eugen} (14.\,7.\,1854 Korneuburg – 16.\,3.\,1912 Wien), \emph{Schriftsteller, Rechtswissenschaftler, Theaterleiter}|pwk}, dem Leiter\pwindex{Burckhard, Max Eugen 14.\,7.\,1854 Korneuburg – 16.\,3.\,1912 Wien@\textsc{Burckhard, Max Eugen} (14.\,7.\,1854 Korneuburg – 16.\,3.\,1912 Wien), \emph{Schriftsteller, Rechtswissenschaftler, Theaterleiter}|pwkv} des \emph{Burgtheaters}\orgindex{Burgtheater@Burgtheater|pwk}, schlechtgeredet und versucht haben, die
                  Aufführung des Stücks\pwindex{Schnitzler, Arthur 15.\,5.\,1862 Wien – 21.\,10.\,1931 ebd.@\textsc{Schnitzler, Arthur} (15.\,5.\,1862 Wien – 21.\,10.\,1931 ebd.), \emph{Schriftsteller, Mediziner}!Liebelei. Schauspiel in drei Akten@\strich\emph{Liebelei. Schauspiel in drei Akten}|pwkv}
                  hinauszuschieben, um Schnitzlers Aufmerksamkeit und Zuneigung zu erhalten. Bei der Uraufführung\eventindex{Burgtheater@\textbf{Burgtheater}!Uraufführung von Liebelei, Premiere von Rechte der Seele, 9.10.1895@Uraufführung von Liebelei, Premiere von Rechte der Seele, 9.10.1895|pwkv} am 9. 10. 1895 am Burgtheater\oindex{Wien@\textbf{Wien}!I., Innere Stadt@\textbf{I., Innere Stadt}!Burgtheater@\textbf{Burgtheater}, \emph{Theater}|pwk} spielte Sandrock\pwindex{Sandrock, Adele 19.\,8.\,1863 Rotterdam – 30.\,8.\,1937 Berlin@\textsc{Sandrock, Adele} (19.\,8.\,1863 Rotterdam – 30.\,8.\,1937 Berlin), \emph{Schauspielerin}|pwk} in der Hauptrolle.}}}\label{K_L02728-1} zu wiſſen. {\pb}Um das Endreſultat\pwindex{Schnitzler, Arthur 15.\,5.\,1862 Wien – 21.\,10.\,1931 ebd.@\textsc{Schnitzler, Arthur} (15.\,5.\,1862 Wien – 21.\,10.\,1931 ebd.), \emph{Schriftsteller, Mediziner}!Liebelei. Schauspiel in drei Akten@\strich\emph{Liebelei. Schauspiel in drei Akten}|pwv} machen{ }ſie mich nicht im Mindeſten bekümmert. Ich{ }ſehe die Dinge von fern an, wie aus den Wolken. Da{ }ſehe ich denn ein Schiff, das
               unaufhaltſam dem Ziele zufährt. Die einzelnen Zickzacklinien des Kurſes{ }ſehe ich
               nicht. Ich{ }ſehe nur, daß es vorwärts geht, nicht zurück – daß es nicht zurückgehen
               kann. Ein paar intriguante Weibsbild\pwindex{Sandrock, Adele 19.\,8.\,1863 Rotterdam – 30.\,8.\,1937 Berlin@\textsc{Sandrock, Adele} (19.\,8.\,1863 Rotterdam – 30.\,8.\,1937 Berlin), \emph{Schauspielerin}|pwv}er{ }ſollen Dein Werk\pwindex{Schnitzler, Arthur 15.\,5.\,1862 Wien – 21.\,10.\,1931 ebd.@\textsc{Schnitzler, Arthur} (15.\,5.\,1862 Wien – 21.\,10.\,1931 ebd.), \emph{Schriftsteller, Mediziner}!Liebelei. Schauspiel in drei Akten@\strich\emph{Liebelei. Schauspiel in drei Akten}|pwv}{ }\strikeout{a\textcolor{gray}{n}} aufhalten, das mit der Kraft Deines Talentes dem Ziele zuſtrebt? Der Gedanke
               macht mich heiter,{ }ſo unſinnig iſt er. {\pb}Und ich
               verliere meine Heiterkeit nur, wenn ich Deinen Brief wieder vornehme und Deine
               Verſtimmung herausleſe, die ich Dir gern erſpart wüßte. Aber{ }ſchön! Du kämpfſt. Wer
               kämpft nicht? Und vergleiche Dein glückliches Loos, für ein hohes Ziel kämpfen zu
               dürfen, mit dem Anderer, mit dem meinen zum Beiſpiel, der ich mit Widerwärtigkeiten
               und tauſend Verhängiſſen ringen muß, nicht um hinaufzugelangen, wie Du,{ }ſondern um
               nicht tiefer zu fallen, als ich{ }ſchon{ }ſtehe.\pend
           
\pstart
           {\pb}Hab’ Geduld, mein lieber Freund! Sei ruhig und laß’
               die Dinge gehen, wie{ }ſie gehen. Das Entſcheidende iſt bereits geſchehen: Du haſt ein{ }ſchönes Stück\pwindex{Schnitzler, Arthur 15.\,5.\,1862 Wien – 21.\,10.\,1931 ebd.@\textsc{Schnitzler, Arthur} (15.\,5.\,1862 Wien – 21.\,10.\,1931 ebd.), \emph{Schriftsteller, Mediziner}!Liebelei. Schauspiel in drei Akten@\strich\emph{Liebelei. Schauspiel in drei Akten}|pwv}
                  geſchrieben{[}.{]} Alles Übrige iſt vollſtändig gleichgiltig. \strikeout{Laß\textcolor{gray}{’}} Laß’ Dich alſo nicht erregen. Blick’ weit hinaus in die Zukunft, laß’ Dich vom
               Tage nicht unterkriegen und vertrau’ auf Dich, wie ich auf Dich vertraue.\pend
           
\pstart
           Das iſt freilich Alles recht vag und allgemein. Ich wünſchte, ich wüßte \strikeout{Nah} Näheres oder könnte gar bei Dir{ }ſein, um {\pb}die Dinge im Einzelnen mit durchzuleben. Du{ }ſollſt
               aber jedenfalls nicht glauben, daß Du mir{ }ſchreiben mußt. Ich verſtehe es, daß Du
               wenig Stimmung zu Briefen findeſt, und warte{ }ſchon meine Zeit ab. Nur möchte ich
               wiſſen, wann ungefähr die Aufführung\pwindex{Schnitzler, Arthur 15.\,5.\,1862 Wien – 21.\,10.\,1931 ebd.@\textsc{Schnitzler, Arthur} (15.\,5.\,1862 Wien – 21.\,10.\,1931 ebd.), \emph{Schriftsteller, Mediziner}!Liebelei. Schauspiel in drei Akten@\strich\emph{Liebelei. Schauspiel in drei Akten}|pwv}{ }ſein wird; und wenn{ }ſie dann iſt, möchte ich mir am nächſten Morgen eine Depeſche über das Reſultat
               erbitten.\pend
           
\pstart
           Iſt \textsc{Bahr\pwindex{Bahr, Hermann 19.\,7.\,1863 Linz – 15.\,1.\,1934 München@\textsc{Bahr, Hermann} (19.\,7.\,1863 Linz – 15.\,1.\,1934 München), \emph{Schriftsteller, Kritiker}|pw}} nicht mit {\pb}unter denen, gegen die Du zu
               kämpfen haſt? Die \label{K_L02728-2v}\edtext{Kritik\pwindex{Gold, Alfred 28.\,6.\,1874 Wien – 24.\,10.\,1958 New York City@\textsc{Gold, Alfred} (28.\,6.\,1874 Wien – 24.\,10.\,1958 New York City), \emph{Schriftsteller, Journalist, Kunsthändler}!Arthur Schnitzler: Sterben@\strich\emph{Arthur Schnitzler: Sterben}|pwv}}{\lemma{\textnormal{\emph{Kritik}}}\Cendnote{\textnormal{A. G.\pwindex{Gold, Alfred 28.\,6.\,1874 Wien – 24.\,10.\,1958 New York City@\textsc{Gold, Alfred} (28.\,6.\,1874 Wien – 24.\,10.\,1958 New York City), \emph{Schriftsteller, Journalist, Kunsthändler}|pwkv} [ = Alfred Gold\pwindex{Gold, Alfred 28.\,6.\,1874 Wien – 24.\,10.\,1958 New York City@\textsc{Gold, Alfred} (28.\,6.\,1874 Wien – 24.\,10.\,1958 New York City), \emph{Schriftsteller, Journalist, Kunsthändler}|pwk}]: \emph{Arthur Schnitzler: Sterben}\pwindex{Gold, Alfred 28.\,6.\,1874 Wien – 24.\,10.\,1958 New York City@\textsc{Gold, Alfred} (28.\,6.\,1874 Wien – 24.\,10.\,1958 New York City), \emph{Schriftsteller, Journalist, Kunsthändler}!Arthur Schnitzler: Sterben@\strich\emph{Arthur Schnitzler: Sterben}|pwk}. In: \emph{Die Zeit}\pwindex{Zeit. Wiener Wochenschrift@\emph{Die Zeit. Wiener Wochenschrift}|pwk}, Bd. 2, Nr. 14, 5. 1. 1895, S. 14.}}}\label{K_L02728-2} über »Sterben\pwindex{Schnitzler, Arthur 15.\,5.\,1862 Wien – 21.\,10.\,1931 ebd.@\textsc{Schnitzler, Arthur} (15.\,5.\,1862 Wien – 21.\,10.\,1931 ebd.), \emph{Schriftsteller, Mediziner}!Sterben. Novelle@\strich\emph{Sterben. Novelle}|pw}« in der »Zeit\pwindex{Zeit. Wiener Wochenschrift@\emph{Die Zeit. Wiener Wochenschrift}|pw}« war ebenſo dumm als \label{K_L02728-3v}\edtext{beſchmockt}{\lemma{\textnormal{\emph{beschmockt}}}\Cendnote{\textnormal{pejorativ: auf Wirkung,
                  Effekt bedacht}}}\label{K_L02728-3}.\pend
           
\pstart
           Ich{ }ſandte Dir dieſer Tage ein paar fran\oindex{Frankreich@\textbf{Frankreich}|pwv}zöſiſche \label{K_L02728-4v}\edtext{Zeitungsartikel}{\lemma{\textnormal{\emph{Zeitungsartikel}}}\Cendnote{\textnormal{nicht
                  überliefert}}}\label{K_L02728-4}. Du findeſt darunter vielleicht Manches, das Dich zerſtreut.
               Kann ich Dir{ }ſonſt was aus \textsc{Paris\oindex{Paris@\textbf{Paris}, \emph{Hauptstadt}|pw}}{ }ſchicken? Das Geſcheiteſte wäre, Du ließeſt den ganzen Kram in Wien\oindex{Wien@\textbf{Wien}, \emph{Verwaltungsgebiet}|pw} im Stich und kämeſt auf vierzehn Tage hierher. Das würde
               Dir gut thun!\pend
           
\pstart
           {\pb}Im Sommer werden wir uns \label{K_L02728-5v}\edtext{kaum{ }ſehen}{\lemma{\textnormal{\emph{kaum sehen}}}\Cendnote{\textnormal{Trotz
                     Goldmanns\pwindex{Goldmann, Paul 31.\,1.\,1865 Breslau – 25.\,9.\,1935 Wien@\textsc{Goldmann, Paul} (31.\,1.\,1865 Breslau – 25.\,9.\,1935 Wien), \emph{Schriftsteller, Journalist}|pwk} Kuraufenthalt in Bad Tölz\oindex{Bad Tölz@\textbf{Bad Tölz}, \emph{Hauptstadt}|pwk} sahen sich die beiden zwischen 28. 8. 1895 und 6. 9. 1895 in Bayern\oindex{Bayern@\textbf{Bayern}, \emph{Land}|pwk}.}}}\label{K_L02728-5} können. Ich werde \label{K_L02728-6v}\edtext{krank}{\lemma{\textnormal{\emph{krank}}}\Cendnote{\textnormal{Siehe XXXX Auszeichnungsfehler: Dokument L02729 nicht gefunden.
               }}}\label{K_L02728-6} und kränker, und mein Schwager\pwindex{Rosengart, Josef 8.\,2.\,1860 Laupheim – 4.\,8.\,1927 Frankfurt am Main@\textsc{Rosengart, Josef} (8.\,2.\,1860 Laupheim – 4.\,8.\,1927 Frankfurt am Main), \emph{Arzt}|pwv} beſteht darauf, daß ich während meines Urlaubs eine Kur gebrauche,
               vielleicht in \textsc{Toelz\oindex{Bad Tölz@\textbf{Bad Tölz}, \emph{Hauptstadt}|pw}}, im bai\oindex{Bayern@\textbf{Bayern}, \emph{Land}|pwv}riſchen
               Hochgebirge.\pend
           
\pstart
           Grüß’ Dich Gott, mein lieber Freund, und{ }ſei guten Muths!\pend
           
\pstart
           Dein {\\[\baselineskip]}treuer {\\[\baselineskip]}\spacefill\mbox{Paul Goldmann}\pend
           \leftskip=0em{}\selectlanguage{ngerman}\endnumbering\briefempfaengerindex{Schnitzler, Arthur@\textsc{Schnitzler, Arthur}!zzzGoldmann, Paul@\emph{von Paul Goldmann}!1895-02-061@{6. 2. [1895]}|)be}\mylabel{L02728h}  \newcommand{\dateiname}{L02728}\newcommand{\titel}{Paul Goldmann an Arthur Schnitzler, 6. 2. [1895]}\newcommand{\editorInnen}{Martin Anton Müller und Laura Untner}%% latex-leseansicht-abspann.tex
%% Abspann für die Leseansicht.
%% Der Schalter \ifkorrekturansicht ist bereits durch den Vorspann gesetzt.

%% latex-abspann.tex
%% Gemeinsamer Abspann für Korrekturansicht und Leseansicht.
%% Setzt den Schalter \ifkorrekturansicht voraus (gesetzt in den
%% einbindenden Dateien latex-korrekturansicht-abspann.tex bzw.
%% latex-leseansicht-abspann.tex).
%% ---------------------------------------------------------------

\normalsize

% Das esempio-Environment wird nur in der Leseansicht benötigt
\ifkorrekturansicht\else
\newenvironment{esempio}[3]%
{
    \vspace{1.5ex}
    \rlap{\underline{#1}}
    \par
    \setlength{\parindent}{0cm}
    \nopagebreak
    \leftskip=#2cm
    \rightskip=#3cm
}
{
    \par
}
\fi

\doendnotes{C}
\bigskip
\vfill

\clearpage

\footnotesize

\ifkorrekturansicht
  \lohead{\textsc{register}}
\fi

% theindex-Environment neu definieren ohne reledmac
\makeatletter
\renewenvironment{theindex}{%
  \ifkorrekturansicht
    \section*{\indexname}%
  \else
    \subsubsection*{Index der erwähnten Entitäten}%
  \fi
  \setlength{\parindent}{0pt}%
  \setlength{\parskip}{0pt plus 0.3pt}%
  \let\item\@idxitem
}{%
  \ifkorrekturansicht\clearpage\fi
}
\makeatother

\IfFileExists{\jobname-pw.ind}{\input{\jobname-pw.ind}}{}

% Quellenangabe nur in der Leseansicht
\ifkorrekturansicht\else
% Fallback-Definitionen, falls die .tex-Datei \titel etc. nicht gesetzt hat
\providecommand{\titel}{}
\providecommand{\editorInnen}{}
\providecommand{\dateiname}{\jobname}

\vspace{3cm}

\vfill

\footnotesize
\textsc{Quelle}: \titel. Herausgegeben von {\editorInnen}. In: \emph{Arthur Schnitzler: Briefwechsel mit Autorinnen und Autoren}.
 Digitale Edition, https://schnitzler-briefe.acdh.oeaw.ac.at/{\dateiname}.html (Stand \today)
\fi

\end{document}


