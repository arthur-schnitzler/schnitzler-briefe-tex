%% latex-leseansicht-vorspann.tex
%% Vorspann für die Leseansicht.
%% Lädt die gemeinsame Datei latex-vorspann.tex mit nicht gesetztem Schalter.

\newif\ifkorrekturansicht
\korrekturansichtfalse

\input{../tex-inputs/latex-vorspann}


\section[Arthur Schnitzler und Felix Salten an Gustav Schwarzkopf, 27. 8. 1893]{L04097 Arthur Schnitzler und Felix Salten an Gustav Schwarzkopf, 27. 8. 1893}
\nopagebreak\mylabel{L04097v}
\rehead{ }\normalsize\beginnumbering\briefempfaengerindex{Schwarzkopf, Gustav@\textsc{Schwarzkopf, Gustav}!zzzSalten, Felix@\emph{von Felix Salten}!1893-08-272@{27. 8. 1893}|(be}\briefempfaengerindex{Schwarzkopf, Gustav@\textsc{Schwarzkopf, Gustav}!zzzSchnitzler, Arthur@\emph{von Arthur Schnitzler}!1893-08-272@{27. 8. 1893}|(be}
\toendnotes[C]{\smallbreak\pagebreak[2]}
\correspDesc{Versand  durch Arthur Schnitzler, Felix Salten am 27. 8. 1893 in Pörtschach am Wörthersee
\newline{}Erhalt  durch Gustav Schwarzkopf am 28. 8. 1893 in Wien}\toendnotes[C]{\smallbreak}
\Standort{CUL, Schnitzler, B 96.}
\physDesc{Kartenbrief, 576 Zeichen
\newline{}Handschrift Arthur Schnitzler: Bleistift, deutsche Kurrent
\newline{}Handschrift Felix Salten: Bleistift, lateinische Kurrent
\newline{}Versand: 1) Stempel: »\nobreak{}\oindex{Pörtschach am Wörthersee@\textbf{Pörtschach am Wörthersee}|pwk}Pörtschach am See, 27 8 93\nobreak{}«.   2) Stempel: »\nobreak{}\oindex{I., Innere Stadt@\textbf{I., Innere Stadt}, \emph{Verwaltungsgebiet}|pwk}{[}Wien 1{]}/1 1, 28 8. 93, 9–10½V., Bestellt\nobreak{}«. }\toendnotes[C]{\smallbreak}\pstart{}{\pb}Herrn \textsc{Gustav
                     Schwarzkopf}\pend{}\pstart{}Schriftſteller\pend{}\pstart{}\textcolor{gray}{\textbf{in}}{ }\textsc{Wien}\oindex{Wien@\textbf{Wien}, \emph{Verwaltungsgebiet}|pw}\pend{}\pstart{}I. Tiefer Graben 23\oindex{Wien@\textbf{Wien}!I., Innere Stadt@\textbf{I., Innere Stadt}!Tiefer Graben 23@\textbf{Tiefer Graben 23}, \emph{Wohngebäude}|pw}.\pend{}{\bigskip}\vspace{1em}
\pstart
           \raggedleft{}{\pb}27. 8. 93\pend
           \vspace{0.5em}
\pstart
           Verehrteſter Freund,{ }wir\pwindex{Salten, Felix 6.\,9.\,1869 Budapest – 8.\,10.\,1945 Zürich@\textsc{Salten, Felix} (6.\,9.\,1869 Budapest – 8.\,10.\,1945 Zürich), \emph{Schriftsteller, Journalist, Chefredakteur}|pw} ſind in Pörtſchach\oindex{Pörtschach am Wörthersee@\textbf{Pörtschach am Wörthersee}|pw}. Mir ko{\geminationm}en aus \textsc{Pieve di Cadore\oindex{Pieve di Cadore@\textbf{Pieve di Cadore}, \emph{Verwaltungsgebiet}|pw}} – wie, das klingt?– Wir fahren auf dem Bicycle; zuweilen auch auf der Bahn. Wir
               haben ſchönes Wetter, zuweilen auch Regen. Immer haben wir guten Appetit, nicht i{\geminationm}er gutes Eſſen. In \textsc{Cadore\oindex{Pieve di Cadore@\textbf{Pieve di Cadore}, \emph{Verwaltungsgebiet}|pw}} wurde Tizian\pwindex{Tizian zwischen 1488 und 1490 Pieve di Cadore – 27.\,8.\,1576 Venedig@\textsc{Tizian} (zwischen 1488 und 1490 Pieve di Cadore – 27.\,8.\,1576 Venedig), \emph{Maler}|pw} geboren; aber das Eſſen
               iſt ſchlecht; in \textsc{Pörtſchach\oindex{Pörtschach am Wörthersee@\textbf{Pörtschach am Wörthersee}|pw}} wurde nicht einmal \textsc{Bératon\pwindex{Bératon, Ferry 6.\,12.\,1859 Wien – 11.\,2.\,1900 Venedig@\textsc{Bératon, Ferry} (6.\,12.\,1859 Wien – 11.\,2.\,1900 Venedig), \emph{Schriftsteller, Journalist, Maler}|pw}} geboren, aber das Eſſen iſt gut. –\pend
           \pstart Herzlich Ihr \spacefill\mbox{ArthSchnitzler}\pend{}
\pstart
           \noindent{}Grüßen Sie Ihre Brüder\pwindex{Schwarzkopf, Max 12.\,6.\,1857 Wien – 14.\,4.\,1928 ebd.@\textsc{Schwarzkopf, Max} (12.\,6.\,1857 Wien – 14.\,4.\,1928 ebd.), \emph{Rechtsanwalt}|pwv}\pwindex{Schwarzkopf, Rudolf 25.\,5.\,1861 Wien – 13.\,10.\,1893 Meran@\textsc{Schwarzkopf, Rudolf} (25.\,5.\,1861 Wien – 13.\,10.\,1893 Meran), \emph{Schriftsteller}|pwv}\pwindex{Schwarzkopf, Emil 17.\,9.\,1851 Wien – 28.\,1.\,1928 ebd.@\textsc{Schwarzkopf, Emil} (17.\,9.\,1851 Wien – 28.\,1.\,1928 ebd.), \emph{Übersetzer, Komponist, Musiklehrer}|pwv}.\pend
           \selectlanguage{ngerman}\vspace{1em}
\pstart
           \noindent{}{\pb}{[}hs. Salten:{]} \label{T_L04097-1v}\edtext{Schnitzler hat den Brief geschloßen, nehmen Sie daher
               meinen Gruß hier}{\lemma{\textnormal{\emph{Schnitzler … hier}}}\Cendnote{\textnormal{Saltens\pwindex{Salten, Felix 6.\,9.\,1869 Budapest – 8.\,10.\,1945 Zürich@\textsc{Salten, Felix} (6.\,9.\,1869 Budapest – 8.\,10.\,1945 Zürich), \emph{Schriftsteller, Journalist, Chefredakteur}|pwk} Text auf der Rückseite des Kartenbriefs.}}}\label{T_L04097-1}\pend
           
\pstart
           Ihr{\\[\baselineskip]}\spacefill\mbox{Salten}\pend
           \leftskip=0em{}\selectlanguage{ngerman}\endnumbering\briefempfaengerindex{Schwarzkopf, Gustav@\textsc{Schwarzkopf, Gustav}!zzzSalten, Felix@\emph{von Felix Salten}!1893-08-272@{27. 8. 1893}|)be}\briefempfaengerindex{Schwarzkopf, Gustav@\textsc{Schwarzkopf, Gustav}!zzzSchnitzler, Arthur@\emph{von Arthur Schnitzler}!1893-08-272@{27. 8. 1893}|)be}\mylabel{L04097h}
\begin{anhang}
\end{anhang}\newcommand{\dateiname}{L04097}\newcommand{\titel}{Arthur Schnitzler und Felix Salten an Gustav Schwarzkopf, 27. 8. 1893}\newcommand{\editorInnen}{Herausgegeben von Jahnke, SelmaMüller, Martin Anton}%% latex-leseansicht-abspann.tex
%% Abspann für die Leseansicht.
%% Der Schalter \ifkorrekturansicht ist bereits durch den Vorspann gesetzt.

%% latex-abspann.tex
%% Gemeinsamer Abspann für Korrekturansicht und Leseansicht.
%% Setzt den Schalter \ifkorrekturansicht voraus (gesetzt in den
%% einbindenden Dateien latex-korrekturansicht-abspann.tex bzw.
%% latex-leseansicht-abspann.tex).
%% ---------------------------------------------------------------

\normalsize

% Das esempio-Environment wird nur in der Leseansicht benötigt
\ifkorrekturansicht\else
\newenvironment{esempio}[3]%
{
    \vspace{1.5ex}
    \rlap{\underline{#1}}
    \par
    \setlength{\parindent}{0cm}
    \nopagebreak
    \leftskip=#2cm
    \rightskip=#3cm
}
{
    \par
}
\fi

\doendnotes{C}
\bigskip
\vfill

\clearpage

\footnotesize

\ifkorrekturansicht
  \lohead{\textsc{register}}
\fi

% theindex-Environment neu definieren ohne reledmac
\makeatletter
\renewenvironment{theindex}{%
  \ifkorrekturansicht
    \section*{\indexname}%
  \else
    \subsubsection*{Index der erwähnten Entitäten}%
  \fi
  \setlength{\parindent}{0pt}%
  \setlength{\parskip}{0pt plus 0.3pt}%
  \let\item\@idxitem
}{%
  \ifkorrekturansicht\clearpage\fi
}
\makeatother

\IfFileExists{\jobname-pw.ind}{\input{\jobname-pw.ind}}{}

% Quellenangabe nur in der Leseansicht
\ifkorrekturansicht\else
% Fallback-Definitionen, falls die .tex-Datei \titel etc. nicht gesetzt hat
\providecommand{\titel}{}
\providecommand{\editorInnen}{}
\providecommand{\dateiname}{\jobname}

\vspace{3cm}

\vfill

\footnotesize
\textsc{Quelle}: \titel. Herausgegeben von {\editorInnen}. In: \emph{Arthur Schnitzler: Briefwechsel mit Autorinnen und Autoren}.
 Digitale Edition, https://schnitzler-briefe.acdh.oeaw.ac.at/{\dateiname}.html (Stand \today)
\fi

\end{document}


