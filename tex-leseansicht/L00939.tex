%% latex-korrekturansicht-vorspann.tex
%% Vorspann für die Korrekturansicht.
%% Lädt die gemeinsame Datei latex-vorspann.tex mit gesetztem Schalter.

\newif\ifkorrekturansicht
\korrekturansichttrue

\input{../tex-inputs/latex-vorspann}


\section[Arthur Schnitzler an Richard Beer-Hofmann, 13. 7. 1899]{L00939 Arthur Schnitzler an Richard Beer-Hofmann, 13. 7. 1899}
\nopagebreak\mylabel{L00939v}
\rehead{ }\normalsize\beginnumbering\briefempfaengerindex{Beer-Hofmann, Richard@\textsc{Beer-Hofmann, Richard}!zzzSchnitzler, Arthur@\emph{von Arthur Schnitzler}!1899-07-131@{13. 7. 1899}|(be}
\toendnotes[C]{\smallbreak\pagebreak[2]}\Standort{YCGL, MSS 31.}
\physDesc{Brief, 1 Blatt, 2 Seiten, Umschlag, 542 Zeichen
\newline{}Handschrift: Bleistift, deutsche Kurrent
\newline{}Versand: 1) Stempel: »\nobreak{}\oindex{IX., Alsergrund@\textbf{IX., Alsergrund}, \emph{A.ADM3}|pwk}Wien 9/3, 13. 7. 99, 12–1 N\nobreak{}«.   2) Stempel: »\nobreak{}\oindex{Seeboden@\textbf{Seeboden}, \emph{A.ADM3}|pwk}{[}Seebod{]}en, 1\textcolor{gray}{4}. 7. {[}9{]}9\nobreak{}«. }
\buchAbdrucke{\weitereDrucke{Arthur Schnitzler, Richard Beer-Hofmann: \emph{Briefwechsel 1891–1931}. Wien, Zürich: \emph{Europaverlag} 1992, S. 132.} }\pstart{}{\pb}Herrn \textsc{Dr. Richard
                     Beer-Hofmann}\pend{}\pstart{}\textsc{Seeboden}\oindex{Seeboden@\textbf{Seeboden}, \emph{A.ADM3}|pw}\pend{}\pstart{}am \textsc{Millstätter}ſee\oindex{Millstaetter See@\textbf{Millstätter See}, \emph{See (N.SEE)}|pw}\pend{}\pstart{}\textsc{Villa Platzer}\oindex{Villa Platzer@\textbf{Villa Platzer}, \emph{Gebäude (K.GBD)}|pw}\pend{}{\bigskip}\vspace{1em}
\pstart
           \raggedleft{}{\pb}\textsc{Wien}\oindex{Wien@\textbf{Wien}, \emph{A.ADM2}|pw}. 13/7 99\pend
           \vspace{0.5em}
\pstart
           lieber Richard, wenn Sie ſchon nichts von ſich ſchreiben zu wollen,
               bitte ſehr, es handelt ſich auch ein bischen um mich. Sie ſchreiben heute, wir ſind
               »vielleicht« ein paar Tage zuſa{\geminationm}en, während doch ſowohl
               von einer gemeinſchaftlichen Tour als von einem möglichen kurzen Aufenthalt
               meinerſeits in Millſtadt\oindex{Millstatt@\textbf{Millstatt}, \emph{A.ADM3}|pw} die Rede war. Alſo
               ſchreiben Sie freundlichſt wenigſtens ſo viel von ſich, daſs ich mich danach richten
               kann.\pend
           
\pstart
           Noch \uline{hieher}; aber gleich.\pend
           \pstart Herzlichen Gruſs. Ihr \spacefill\mbox{Arth}\pend{}\selectlanguage{ngerman}\endnumbering\briefempfaengerindex{Beer-Hofmann, Richard@\textsc{Beer-Hofmann, Richard}!zzzSchnitzler, Arthur@\emph{von Arthur Schnitzler}!1899-07-131@{13. 7. 1899}|)be}\mylabel{L00939h}  \normalsize

\doendnotes{C}
\bigskip
\vfill

\clearpage

\footnotesize

\lohead{\textsc{register}}

% Definiere theindex-Environment komplett neu ohne reledmac
\makeatletter
\renewenvironment{theindex}{%
  \section*{\indexname}%
  \setlength{\parindent}{0pt}%
  \setlength{\parskip}{0pt plus 0.3pt}%
  \let\item\@idxitem
}{%
  \clearpage
}
\makeatother

\IfFileExists{\jobname-pw.ind}{\input{\jobname-pw.ind}}{}

\end{document}

      