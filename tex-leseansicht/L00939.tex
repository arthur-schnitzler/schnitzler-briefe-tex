%% latex-leseansicht-vorspann.tex
%% Vorspann für die Leseansicht.
%% Lädt die gemeinsame Datei latex-vorspann.tex mit nicht gesetztem Schalter.

\newif\ifkorrekturansicht
\korrekturansichtfalse

\input{../tex-inputs/latex-vorspann}


\section[Arthur Schnitzler an Richard Beer-Hofmann, 13. 7. 1899]{L00939 Arthur Schnitzler an Richard Beer-Hofmann, 13. 7. 1899}
\nopagebreak\mylabel{L00939v}
\rehead{ }\normalsize\beginnumbering\briefempfaengerindex{Beer-Hofmann, Richard@\textsc{Beer-Hofmann, Richard}!zzzSchnitzler, Arthur@\emph{von Arthur Schnitzler}!1899-07-131@{13. 7. 1899}|(be}
\toendnotes[C]{\smallbreak\pagebreak[2]}
\correspDesc{Versand  durch Arthur Schnitzler am 13. 7. 1899 in Wien
\newline{}Erhalt  durch Richard Beer-Hofmann am 14. 7. 1899 in Seeboden}\toendnotes[C]{\smallbreak}
\Standort{YCGL, MSS 31.}
\physDesc{Brief, 1 Blatt, 2 Seiten, Kuvert, 542 Zeichen
\newline{}Handschrift: Bleistift, deutsche Kurrent
\newline{}Versand: 1) Stempel: »\nobreak{}\oindex{IX., Alsergrund@\textbf{IX., Alsergrund}, \emph{Verwaltungsgebiet}|pwk}Wien 9/3, 13. 7. 99, 12–1 N\nobreak{}«.   2) Stempel: »\nobreak{}\oindex{Seeboden am Millstättersee@\textbf{Seeboden am Millstättersee}|pwk}{[}Seebod{]}en, 1\textcolor{gray}{4}. 7. {[}9{]}9\nobreak{}«. }
\buchAbdrucke{\weitereDrucke{Arthur Schnitzler, Richard Beer-Hofmann: \emph{Briefwechsel 1891–1931}. Herausgegeben von Konstanze Fliedl. Wien, Zürich: \emph{Europaverlag} 1992, S. 132.} }\pstart{}{\pb}Herrn \textsc{Dr. Richard
                     Beer-Hofmann}\pend{}\pstart{}\textsc{Seeboden}\oindex{Seeboden am Millstättersee@\textbf{Seeboden am Millstättersee}|pw}\pend{}\pstart{}am \textsc{Millstätter}ſee\oindex{Millstätter See@\textbf{Millstätter See}, \emph{See}|pw}\pend{}\pstart{}\textsc{Villa Platzer}\oindex{Villa Platzer@\textbf{Villa Platzer}, \emph{Gebäude}|pw}\pend{}{\bigskip}\vspace{1em}
\pstart
           \raggedleft{}{\pb}\textsc{Wien}\oindex{Wien@\textbf{Wien}, \emph{Verwaltungsgebiet}|pw}. 13/7 99\pend
           \vspace{0.5em}
\pstart
           lieber Richard, wenn Sie{ }ſchon nichts von{ }ſich{ }ſchreiben zu wollen,
               bitte{ }ſehr, es handelt{ }ſich auch ein bischen um mich. Sie{ }ſchreiben heute, wir{ }ſind
               »vielleicht« ein paar Tage zuſa{\geminationm}en, während doch{ }ſowohl
               von einer gemeinſchaftlichen Tour als von einem möglichen kurzen Aufenthalt
               meinerſeits in Millſtadt\oindex{Millstatt@\textbf{Millstatt}, \emph{Verwaltungsgebiet}|pw} die Rede war. Alſo{ }ſchreiben Sie freundlichſt wenigſtens{ }ſo viel von{ }ſich, daſs ich mich danach richten
               kann.\pend
           
\pstart
           Noch \uline{hieher}; aber gleich.\pend
           \pstart Herzlichen Gruſs. Ihr \spacefill\mbox{Arth}\pend{}\selectlanguage{ngerman}\endnumbering\briefempfaengerindex{Beer-Hofmann, Richard@\textsc{Beer-Hofmann, Richard}!zzzSchnitzler, Arthur@\emph{von Arthur Schnitzler}!1899-07-131@{13. 7. 1899}|)be}\mylabel{L00939h}  \newcommand{\dateiname}{L00939}\newcommand{\titel}{Arthur Schnitzler an Richard Beer-Hofmann, 13. 7. 1899}\newcommand{\editorInnen}{Martin Anton Müller und Gerd-Hermann Susen}%% latex-leseansicht-abspann.tex
%% Abspann für die Leseansicht.
%% Der Schalter \ifkorrekturansicht ist bereits durch den Vorspann gesetzt.

%% latex-abspann.tex
%% Gemeinsamer Abspann für Korrekturansicht und Leseansicht.
%% Setzt den Schalter \ifkorrekturansicht voraus (gesetzt in den
%% einbindenden Dateien latex-korrekturansicht-abspann.tex bzw.
%% latex-leseansicht-abspann.tex).
%% ---------------------------------------------------------------

\normalsize

% Das esempio-Environment wird nur in der Leseansicht benötigt
\ifkorrekturansicht\else
\newenvironment{esempio}[3]%
{
    \vspace{1.5ex}
    \rlap{\underline{#1}}
    \par
    \setlength{\parindent}{0cm}
    \nopagebreak
    \leftskip=#2cm
    \rightskip=#3cm
}
{
    \par
}
\fi

\doendnotes{C}
\bigskip
\vfill

\clearpage

\footnotesize

\ifkorrekturansicht
  \lohead{\textsc{register}}
\fi

% theindex-Environment neu definieren ohne reledmac
\makeatletter
\renewenvironment{theindex}{%
  \ifkorrekturansicht
    \section*{\indexname}%
  \else
    \subsubsection*{Index der erwähnten Entitäten}%
  \fi
  \setlength{\parindent}{0pt}%
  \setlength{\parskip}{0pt plus 0.3pt}%
  \let\item\@idxitem
}{%
  \ifkorrekturansicht\clearpage\fi
}
\makeatother

\IfFileExists{\jobname-pw.ind}{\input{\jobname-pw.ind}}{}

% Quellenangabe nur in der Leseansicht
\ifkorrekturansicht\else
% Fallback-Definitionen, falls die .tex-Datei \titel etc. nicht gesetzt hat
\providecommand{\titel}{}
\providecommand{\editorInnen}{}
\providecommand{\dateiname}{\jobname}

\vspace{3cm}

\vfill

\footnotesize
\textsc{Quelle}: \titel. Herausgegeben von {\editorInnen}. In: \emph{Arthur Schnitzler: Briefwechsel mit Autorinnen und Autoren}.
 Digitale Edition, https://schnitzler-briefe.acdh.oeaw.ac.at/{\dateiname}.html (Stand \today)
\fi

\end{document}


