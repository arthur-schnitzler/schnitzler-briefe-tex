%% latex-korrekturansicht-vorspann.tex
%% Vorspann für die Korrekturansicht.
%% Lädt die gemeinsame Datei latex-vorspann.tex mit gesetztem Schalter.

\newif\ifkorrekturansicht
\korrekturansichttrue

\input{../tex-inputs/latex-vorspann}


\section[Hermann Bahr an Arthur Schnitzler, 9. 2. 1921]{L02361 Hermann Bahr an Arthur Schnitzler, 9. 2. 1921}
\nopagebreak\mylabel{L02361v}
\rehead{ }\normalsize\beginnumbering\briefempfaengerindex{Schnitzler, Arthur@\textsc{Schnitzler, Arthur}!zzzBahr, Hermann@\emph{von Hermann Bahr}!1921-02-091@{9. 2. 1921}|(be}
\toendnotes[C]{\smallbreak\pagebreak[2]}\Standort{CUL, Schnitzler, B 5b.}
\physDesc{Postkarte, 634 Zeichen
\newline{}Handschrift: schwarze Tinte, deutsche Kurrent
\newline{}Versand: Stempel: »\nobreak{}\oindex{Salzburg@\textbf{Salzburg}, \emph{A.ADM2}|pwk}Salzburg, 10. II. {[}1921{]}\nobreak{}«.  
\newline{}Schnitzler: mit Bleistift Vermerk: »\textsc{A}«, vermutlich für »Abzuschreiben«/»Abschrift« 
\newline{}Ordnung: mit Bleistift von unbekannter Hand nummeriert:
                                    »184« }
\buchAbdrucke{\weitereDrucke{Hermann Bahr, Arthur Schnitzler: \emph{Briefwechsel, Aufzeichnungen, Dokumente (1891–1931)}. Göttingen: \emph{Wallstein} 2018, S. 540.} }\toendnotes[C]{\smallbreak}\pstart{}{\pb}Herrn D\textsuperscript{r} Arthur
                  Schnitzler\pend{}\pstart{}\textsc{Wien XVIII\oindex{XVIII., Waehring@\textbf{XVIII., Währing}, \emph{A.ADM3}|pw}}\pend{}\pstart{}Sternwarteſtraße 71\oindex{Wien@\textbf{Wien}, \emph{A.ADM2}|pw}\pend{}{\bigskip}\vspace{1em}
\pstart
           \raggedleft{}{\pb}9. 2. 21\pend
           
\pstart{}Lieber Arthur!\pend\vspace{0.5em}
\pstart
           Herzlichſten Dank für Deinen lieben Brief! Aber als er kam, war mein für das Journal\pwindex{Neues Wiener Journal@\emph{Neues Wiener Journal}|pw}{ }\label{K_L02361-1v}\edtext{vom 20.}{\lemma{\textnormal{\emph{vom 20.}}}\Cendnote{\textnormal{Hermann Bahr\pwindex{Bahr, Hermann 19.07.1863 – 15.01.1934@\textsc{Bahr, Hermann} (19.07.1863 – 15.01.1934), \emph{Schriftsteller/Schriftstellerin, Kritiker/Kritikerin}|pwk}: \emph{Tagebuch. 30. Januar, 1. Februar und 3. Februar}\pwindex{Tagebuch. 30. Januar [1921]@\emph{Tagebuch. 30. Januar [1921]}|pwk}. In: \emph{Neues Wiener Journal}\pwindex{Neues Wiener Journal@\emph{Neues Wiener Journal}|pwk}, Jg. 29, Nr. 9803,
                        20. 2. 1921, S. 6.}}}\label{K_L02361-1} beſtimmtes Tagebuch\pwindex{Tagebuch [Kolumne im Neuen Wiener Journal]@\emph{Tagebuch [Kolumne im Neuen Wiener Journal]}|pw}{ }ſchon abgegangen. Wenns irgend geht, hoff ich aber
               dennoch des verehrten Mannes\pwindex{Popper-Lynkeus, Josef 21.02.1838 – 22.12.1921@\textsc{Popper-Lynkeus, Josef} (21.02.1838 – 22.12.1921), \emph{Schriftsteller/Schriftstellerin}|pwv}
               u. ſeines Geburtstags zu gedenken, wenn auch \label{K_L02361-2v}\edtext{\textsc{post festum}}{\lemma{\textnormal{\emph{post festum}}}\Cendnote{\textnormal{Im \emph{Tagebuch. 20. Februar}\pwindex{Tagebuch. 20. Februar [1921]@\emph{Tagebuch. 20. Februar [1921]}|pwk} (damit den falschen Tag aus Schnitzlers Brief übernehmend), erschienen am 13. 3. 1921 (\emph{Neues Wiener Journal}\pwindex{Neues Wiener Journal@\emph{Neues Wiener Journal}|pwk}, Jg. 29, Nr. 9824,
                     S. 7).}}}\label{K_L02361-2}. – Ich leſe jetzt Deinen Namen ſo oft – erinnerſt Du Dich
               denn, daß ich der erſte war, der »Reigen\pwindex{Reigen. Zehn Dialoge@\emph{Reigen. Zehn Dialoge}|pw}«
               öffentlich vorleſen wollte, ja ſogar bis zu \label{K_L02361-3v}\edtext{Körber\pwindex{Koerber, Ernest von 06.11.1850 – 05.03.1919@\textsc{Koerber, Ernest von} (06.11.1850 – 05.03.1919), \emph{Politiker/Politikerin}|pw}{ }ſelber ging}{\lemma{\textnormal{\emph{Körber ſelber ging}}}\Cendnote{\textnormal{Siehe Hermann Bahr, Arthur Schnitzler: \emph{Briefwechsel, Aufzeichnungen, Dokumente (1891–1931)}, Hermann Bahr beim Ministerpräsidenten, 5. 11. 1903.}}}\label{K_L02361-3}, um es durchzuſetzen, leider vergebens? – Ich
               wäre ſehr froh, Dich bald einmal endlich wiederzuſehen!\pend
           
\pstart
           Dich u. die Deinen\pwindex{Schnitzler, Olga 17.01.1882 – 13.01.1970@\textsc{Schnitzler, Olga} (17.01.1882 – 13.01.1970), \emph{Schauspieler/Schauspielerin, Sänger/Sängerin}|pwv}\pwindex{Schnitzler, Heinrich 09.08.1902 – 12.07.1982@\textsc{Schnitzler, Heinrich} (09.08.1902 – 12.07.1982), \emph{Regisseur/Regisseurin, Schauspieler/Schauspielerin}|pwv}\pwindex{Cappellini, Lili 13.09.1909 – 26.07.1928@\textsc{Cappellini, Lili} (13.09.1909 – 26.07.1928)|pwv} herzlichſt grüßend{\\[\baselineskip]}Dein
                  alter\spacefill\mbox{Hermann}\pend
           \leftskip=0em{}\selectlanguage{ngerman}\endnumbering\briefempfaengerindex{Schnitzler, Arthur@\textsc{Schnitzler, Arthur}!zzzBahr, Hermann@\emph{von Hermann Bahr}!1921-02-091@{9. 2. 1921}|)be}\mylabel{L02361h}  \normalsize

\doendnotes{C}
\bigskip
\vfill

\clearpage

\footnotesize

\lohead{\textsc{register}}

% Definiere theindex-Environment komplett neu ohne reledmac
\makeatletter
\renewenvironment{theindex}{%
  \section*{\indexname}%
  \setlength{\parindent}{0pt}%
  \setlength{\parskip}{0pt plus 0.3pt}%
  \let\item\@idxitem
}{%
  \clearpage
}
\makeatother

\IfFileExists{\jobname-pw.ind}{\input{\jobname-pw.ind}}{}

\end{document}

      