%% latex-leseansicht-vorspann.tex
%% Vorspann für die Leseansicht.
%% Lädt die gemeinsame Datei latex-vorspann.tex mit nicht gesetztem Schalter.

\newif\ifkorrekturansicht
\korrekturansichtfalse

\input{../tex-inputs/latex-vorspann}


\section[Hermann Bahr an Arthur Schnitzler, 9. 2. 1921]{L02361 Hermann Bahr an Arthur Schnitzler, 9. 2. 1921}
\nopagebreak\mylabel{L02361v}
\rehead{ }\normalsize\beginnumbering\briefempfaengerindex{Schnitzler, Arthur@\textsc{Schnitzler, Arthur}!zzzBahr, Hermann@\emph{von Hermann Bahr}!1921-02-091@{9. 2. 1921}|(be}
\toendnotes[C]{\smallbreak\pagebreak[2]}
\correspDesc{Versand  durch Hermann Bahr am 9. 2. 1921 in Salzburg
\newline{}Erhalt  durch Arthur Schnitzler am 10. II. [1921] in Wien}\toendnotes[C]{\smallbreak}
\Standort{CUL, Schnitzler, B 5b.}
\physDesc{Postkarte, 634 Zeichen
\newline{}Handschrift: schwarze Tinte, deutsche Kurrent
\newline{}Versand: Stempel: »\nobreak{}\oindex{Salzburg@\textbf{Salzburg}, \emph{Verwaltungsgebiet}|pwk}Salzburg, 10. II. {[}1921{]}\nobreak{}«.  
\newline{}Schnitzler: mit Bleistift Vermerk: »\textsc{A}«, vermutlich für »Abzuschreiben«/»Abschrift« 
\newline{}Ordnung: mit Bleistift von unbekannter Hand nummeriert:
                                    »184« }
\buchAbdrucke{\weitereDrucke{Hermann Bahr, Arthur Schnitzler: \emph{Briefwechsel, Aufzeichnungen, Dokumente (1891–1931)}. Herausgegeben von Kurt Ifkovits und Martin Anton Müller. Göttingen: \emph{Wallstein} 2018, S. 540.} }\toendnotes[C]{\smallbreak}\pstart{}{\pb}Herrn D\textsuperscript{r} Arthur
                  Schnitzler\pend{}\pstart{}\textsc{Wien XVIII\oindex{XVIII., Währing@\textbf{XVIII., Währing}, \emph{Verwaltungsgebiet}|pw}}\pend{}\pstart{}Sternwarteſtraße 71\oindex{Wien@\textbf{Wien}, \emph{Verwaltungsgebiet}|pw}\pend{}{\bigskip}\vspace{1em}
\pstart
           \raggedleft{}{\pb}9. 2. 21\pend
           
\pstart{}Lieber Arthur!\pend\vspace{0.5em}
\pstart
           Herzlichſten Dank für Deinen lieben Brief! Aber als er kam, war mein für das Journal\pwindex{Neues Wiener Journal@\emph{Neues Wiener Journal}|pw}{ }\label{K_L02361-1v}\edtext{vom 20.}{\lemma{\textnormal{\emph{vom 20.}}}\Cendnote{\textnormal{Hermann Bahr\pwindex{Bahr, Hermann 19.\,7.\,1863 Linz – 15.\,1.\,1934 München@\textsc{Bahr, Hermann} (19.\,7.\,1863 Linz – 15.\,1.\,1934 München), \emph{Schriftsteller, Kritiker}|pwk}: \emph{Tagebuch. 30. Januar, 1. Februar und 3. Februar}\pwindex{Bahr, Hermann 19.\,7.\,1863 Linz – 15.\,1.\,1934 München@\textsc{Bahr, Hermann} (19.\,7.\,1863 Linz – 15.\,1.\,1934 München), \emph{Schriftsteller, Kritiker}!Tagebuch. 30. Januar [1921]@\strich\emph{Tagebuch. 30. Januar [1921]}|pwk}. In: \emph{Neues Wiener Journal}\pwindex{Neues Wiener Journal@\emph{Neues Wiener Journal}|pwk}, Jg. 29, Nr. 9803,
                        20. 2. 1921, S. 6.}}}\label{K_L02361-1} beſtimmtes Tagebuch\pwindex{Bahr, Hermann 19.\,7.\,1863 Linz – 15.\,1.\,1934 München@\textsc{Bahr, Hermann} (19.\,7.\,1863 Linz – 15.\,1.\,1934 München), \emph{Schriftsteller, Kritiker}!Tagebuch [Kolumne im Neuen Wiener Journal]@\strich\emph{Tagebuch [Kolumne im Neuen Wiener Journal]}|pw}{ }ſchon abgegangen. Wenns irgend geht, hoff ich aber
               dennoch des verehrten Mannes\pwindex{Popper-Lynkeus, Josef 21.\,2.\,1838 Kolín – 22.\,12.\,1921 Woltergasse 2a@\textsc{Popper-Lynkeus, Josef} (21.\,2.\,1838 Kolín – 22.\,12.\,1921 Woltergasse 2a), \emph{Schriftsteller}|pwv}
               u.{ }ſeines Geburtstags zu gedenken, wenn auch \label{K_L02361-2v}\edtext{\textsc{post festum}}{\lemma{\textnormal{\emph{post festum}}}\Cendnote{\textnormal{Im \emph{Tagebuch. 20. Februar}\pwindex{Bahr, Hermann 19.\,7.\,1863 Linz – 15.\,1.\,1934 München@\textsc{Bahr, Hermann} (19.\,7.\,1863 Linz – 15.\,1.\,1934 München), \emph{Schriftsteller, Kritiker}!Tagebuch. 20. Februar [1921]@\strich\emph{Tagebuch. 20. Februar [1921]}|pwk} (damit den falschen Tag aus Schnitzlers Brief übernehmend), erschienen am 13. 3. 1921 (\emph{Neues Wiener Journal}\pwindex{Neues Wiener Journal@\emph{Neues Wiener Journal}|pwk}, Jg. 29, Nr. 9824,
                     S. 7).}}}\label{K_L02361-2}. – Ich leſe jetzt Deinen Namen{ }ſo oft – erinnerſt Du Dich
               denn, daß ich der erſte war, der »Reigen\pwindex{Schnitzler, Arthur 15.\,5.\,1862 Wien – 21.\,10.\,1931 ebd.@\textsc{Schnitzler, Arthur} (15.\,5.\,1862 Wien – 21.\,10.\,1931 ebd.), \emph{Schriftsteller, Mediziner}!Reigen. Zehn Dialoge@\strich\emph{Reigen. Zehn Dialoge}|pw}«
               öffentlich vorleſen wollte, ja{ }ſogar bis zu \label{K_L02361-3v}\edtext{Körber\pwindex{Koerber, Ernest von 6.\,11.\,1850 Trient – 5.\,3.\,1919 Baden bei Wien@\textsc{Koerber, Ernest von} (6.\,11.\,1850 Trient – 5.\,3.\,1919 Baden bei Wien), \emph{Politiker}|pw}{ }ſelber ging}{\lemma{\textnormal{\emph{Körber selber ging}}}\Cendnote{\textnormal{Siehe Hermann Bahr, Arthur Schnitzler: \emph{Briefwechsel, Aufzeichnungen, Dokumente (1891–1931)}, Hermann Bahr beim Ministerpräsidenten, 5. 11. 1903.}}}\label{K_L02361-3}, um es durchzuſetzen, leider vergebens? – Ich
               wäre{ }ſehr froh, Dich bald einmal endlich wiederzuſehen!\pend
           
\pstart
           Dich u. die Deinen\pwindex{Schnitzler, Olga 17.\,1.\,1882 Wien – 13.\,1.\,1970 Lugano@\textsc{Schnitzler, Olga} (17.\,1.\,1882 Wien – 13.\,1.\,1970 Lugano), \emph{Schauspielerin, Sängerin}|pwv}\pwindex{Schnitzler, Heinrich 9.\,8.\,1902 Hinterbrühl – 12.\,7.\,1982 Wien@\textsc{Schnitzler, Heinrich} (9.\,8.\,1902 Hinterbrühl – 12.\,7.\,1982 Wien), \emph{Regisseur, Schauspieler}|pwv}\pwindex{Cappellini, Lili 13.\,9.\,1909 Wien – 26.\,7.\,1928 Venedig@\textsc{Cappellini, Lili} (13.\,9.\,1909 Wien – 26.\,7.\,1928 Venedig)|pwv} herzlichſt grüßend{\\[\baselineskip]}Dein
                  alter\spacefill\mbox{Hermann}\pend
           \leftskip=0em{}\selectlanguage{ngerman}\endnumbering\briefempfaengerindex{Schnitzler, Arthur@\textsc{Schnitzler, Arthur}!zzzBahr, Hermann@\emph{von Hermann Bahr}!1921-02-091@{9. 2. 1921}|)be}\mylabel{L02361h}  \newcommand{\dateiname}{L02361}\newcommand{\titel}{Hermann Bahr an Arthur Schnitzler, 9. 2. 1921}\newcommand{\editorInnen}{Herausgegeben von Martin Anton Müller}%% latex-leseansicht-abspann.tex
%% Abspann für die Leseansicht.
%% Der Schalter \ifkorrekturansicht ist bereits durch den Vorspann gesetzt.

%% latex-abspann.tex
%% Gemeinsamer Abspann für Korrekturansicht und Leseansicht.
%% Setzt den Schalter \ifkorrekturansicht voraus (gesetzt in den
%% einbindenden Dateien latex-korrekturansicht-abspann.tex bzw.
%% latex-leseansicht-abspann.tex).
%% ---------------------------------------------------------------

\normalsize

% Das esempio-Environment wird nur in der Leseansicht benötigt
\ifkorrekturansicht\else
\newenvironment{esempio}[3]%
{
    \vspace{1.5ex}
    \rlap{\underline{#1}}
    \par
    \setlength{\parindent}{0cm}
    \nopagebreak
    \leftskip=#2cm
    \rightskip=#3cm
}
{
    \par
}
\fi

\doendnotes{C}
\bigskip
\vfill

\clearpage

\footnotesize

\ifkorrekturansicht
  \lohead{\textsc{register}}
\fi

% theindex-Environment neu definieren ohne reledmac
\makeatletter
\renewenvironment{theindex}{%
  \ifkorrekturansicht
    \section*{\indexname}%
  \else
    \subsubsection*{Index der erwähnten Entitäten}%
  \fi
  \setlength{\parindent}{0pt}%
  \setlength{\parskip}{0pt plus 0.3pt}%
  \let\item\@idxitem
}{%
  \ifkorrekturansicht\clearpage\fi
}
\makeatother

\IfFileExists{\jobname-pw.ind}{\input{\jobname-pw.ind}}{}

% Quellenangabe nur in der Leseansicht
\ifkorrekturansicht\else
% Fallback-Definitionen, falls die .tex-Datei \titel etc. nicht gesetzt hat
\providecommand{\titel}{}
\providecommand{\editorInnen}{}
\providecommand{\dateiname}{\jobname}

\vspace{3cm}

\vfill

\footnotesize
\textsc{Quelle}: \titel. Herausgegeben von {\editorInnen}. In: \emph{Arthur Schnitzler: Briefwechsel mit Autorinnen und Autoren}.
 Digitale Edition, https://schnitzler-briefe.acdh.oeaw.ac.at/{\dateiname}.html (Stand \today)
\fi

\end{document}


