%% latex-leseansicht-vorspann.tex
%% Vorspann für die Leseansicht.
%% Lädt die gemeinsame Datei latex-vorspann.tex mit nicht gesetztem Schalter.

\newif\ifkorrekturansicht
\korrekturansichtfalse

\input{../tex-inputs/latex-vorspann}


         
         \renewcommand{\erwaehntePersonen}{Personen: Hermann Bahr, Ernest von Koerber, Josef Popper-Lynkeus, Olga Schnitzler, Heinrich Schnitzler, Lili Schnitzler}
         \renewcommand{\erwaehnteOrte}{Orte: Salzburg, Wien, XVIII., Währing}
         \renewcommand{\erwaehnteWerke}{Werke: Neues Wiener Journal, Reigen. Zehn Dialoge, Tagebuch [Kolumne im Neuen Wiener Journal], Tagebuch. 20. Februar [1921], Tagebuch. 30. Januar [1921]}
               \section[Hermann Bahr an Arthur Schnitzler, 9. 2. 1921]{ Hermann Bahr an Arthur Schnitzler, 9. 2. 1921}\nopagebreak\mylabel{v}\rehead{ }\begin{ledgroupsized}[t]{13cm}\normalsize\beginnumbering\briefempfaengerindex{Schnitzler, Arthur@\textsc{Schnitzler, Arthur}!zzzBahr, Hermann@\emph{von Hermann Bahr}!1921-02-091@{9. 2. 1921}|(be} \toendnotes[C]{\smallbreak\pagebreak[2]} \Standort{CUL, Schnitzler, B 5b.}
\physDesc{Postkarte, 634 Zeichen
\newline{}Handschrift: schwarze Tinte, deutsche Kurrent
\newline{}Versand: Stempel: »\nobreak{}\oindex{Salzburg@\textbf{Salzburg}|pwk}Salzburg, 10. II. {[}1921{]}\nobreak{}«.  
\newline{}Schnitzler: mit Bleistift Vermerk: »\textsc{A}«, vermutlich für »Abzuschreiben«/»Abschrift« 
\newline{}Ordnung: mit Bleistift von unbekannter Hand nummeriert:
                                    »184« }\buchAbdrucke{\weitereDrucke{Hermann Bahr, Arthur Schnitzler: \emph{Briefwechsel, Aufzeichnungen, Dokumente (1891–1931)}. Hg. Kurt Ifkovits und Martin Anton Müller. Göttingen: \emph{Wallstein} 2018, S. 540.} }\toendnotes[C]{\smallbreak}\pstart{}{\pb}Herrn D\textsuperscript{r} Arthur
                  Schnitzler\pend{}\pstart{}\textsc{Wien XVIII\oindex{XVIII., Waehring@\textbf{XVIII., Währing}|pw}}\pend{}\pstart{}Sternwarteſtraße 71\oindex{Wien@\textbf{Wien}|pw}\pend{}{\bigskip}\pstart
           \raggedleft{}{\pb}9. 2. 21\pend
           \pstart{}Lieber Arthur!\pend\pstart
           Herzlichſten Dank für Deinen lieben Brief! Aber als er kam, war mein für das Journal\pwindex{Neues Wiener Journal1893 – 1939@\emph{Neues Wiener Journal} {[}1893 – 1939{]}|pw}{ }\label{K_L02361-1v}\edtext{vom 20.}{\lemma{\textnormal{\emph{vom 20.}}}\Cendnote{\textnormal{Hermann Bahr\pwindex{Bahr, Hermann 19.07.1863 – 15.01.1934@\textsc{Bahr, Hermann} (19.07.1863 – 15.01.1934), \emph{Schriftsteller, Kritiker}|pwk}: \emph{Tagebuch. 30. Januar, 1. Februar und 3. Februar}\pwindex{Bahr, Hermann 19.07.1863 – 15.01.1934@\textsc{Bahr, Hermann} (19.07.1863 – 15.01.1934), \emph{Schriftsteller, Kritiker}!Tagebuch. 30. Januar [1921]20. 02. 1921@\strich\emph{Tagebuch. 30. Januar [1921]} {[}20. 02. 1921{]}|pwk}. In: \emph{Neues Wiener Journal}\pwindex{Neues Wiener Journal1893 – 1939@\emph{Neues Wiener Journal} {[}1893 – 1939{]}|pwk}, Jg. 29, Nr. 9803,
                        20. 2. 1921, S. 6.}}}\label{K_L02361-1h} beſtimmtes Tagebuch\pwindex{Bahr, Hermann 19.07.1863 – 15.01.1934@\textsc{Bahr, Hermann} (19.07.1863 – 15.01.1934), \emph{Schriftsteller, Kritiker}!Tagebuch [Kolumne im Neuen Wiener Journal]24.12.1916 – 1931@\strich\emph{Tagebuch [Kolumne im Neuen Wiener Journal]} {[}24.12.1916 – 1931{]}|pw}{ }ſchon abgegangen. Wenns irgend geht, hoff ich aber
               dennoch des verehrten Mannes\pwindex{Popper-Lynkeus, Josef 21.02.1838 – 22.12.1921@\textsc{Popper-Lynkeus, Josef} (21.02.1838 – 22.12.1921), \emph{Schriftsteller}|pwv}
               u. ſeines Geburtstags zu gedenken, wenn auch \label{K_L02361-2v}\edtext{\textsc{post festum}}{\lemma{\textnormal{\emph{post festum}}}\Cendnote{\textnormal{Im \emph{Tagebuch. 20. Februar}\pwindex{Bahr, Hermann 19.07.1863 – 15.01.1934@\textsc{Bahr, Hermann} (19.07.1863 – 15.01.1934), \emph{Schriftsteller, Kritiker}!Tagebuch. 20. Februar [1921]13. 03. 1921@\strich\emph{Tagebuch. 20. Februar [1921]} {[}13. 03. 1921{]}|pwk} (damit den falschen Tag aus Schnitzlers\pwindex{Schnitzler, Arthur 15.05.1862 – 21.10.1931@\textsc{Schnitzler, Arthur} (15.05.1862 – 21.10.1931), \emph{Schriftsteller, Mediziner}|pwk} Brief übernehmend), erschienen am 13. 3. 1921 (\emph{Neues Wiener Journal}\pwindex{Neues Wiener Journal1893 – 1939@\emph{Neues Wiener Journal} {[}1893 – 1939{]}|pwk}, Jg. 29, Nr. 9824,
                     S. 7).}}}\label{K_L02361-2h}. – Ich leſe jetzt Deinen Namen ſo oft – erinnerſt Du Dich
               denn, daß ich der erſte war, der »Reigen\pwindex{Schnitzler, Arthur 15.05.1862 – 21.10.1931@\textsc{Schnitzler, Arthur} (15.05.1862 – 21.10.1931), \emph{Schriftsteller, Mediziner}!Reigen. Zehn Dialoge1900@\strich\emph{Reigen. Zehn Dialoge} {[}1900{]}|pw}«
               öffentlich vorleſen wollte, ja ſogar bis zu \label{K_L02361-3v}\edtext{Körber\pwindex{Koerber, Ernest von 06.11.1850 – 05.03.1919@\textsc{Koerber, Ernest von} (06.11.1850 – 05.03.1919), \emph{Politiker}|pw}{ }ſelber ging}{\lemma{\textnormal{\emph{Körber ſelber ging}}}\Cendnote{\textnormal{Vgl. \emph{Briefwechsel}
                  Bahr/Schnitzler 276.}}}\label{K_L02361-3h}, um es durchzuſetzen, leider vergebens? – Ich
               wäre ſehr froh, Dich bald einmal endlich wiederzuſehen!\pend
           \pstart
           Dich u. die Deinen\pwindex{Schnitzler, Olga 17.01.1882 – 13.01.1970@\textsc{Schnitzler, Olga} (17.01.1882 – 13.01.1970), \emph{Schauspielerin, Sängerin}|pwv}\pwindex{Schnitzler, Heinrich 09.08.1902 – 12.07.1982@\textsc{Schnitzler, Heinrich} (09.08.1902 – 12.07.1982), \emph{Regisseur, Schauspieler}|pwv}\pwindex{Schnitzler, Lili 13.09.1909 – 26.07.1928@\textsc{Schnitzler, Lili} (13.09.1909 – 26.07.1928)|pwv} herzlichſt grüßend{\\[\baselineskip]}Dein
                  alter\spacefill\mbox{Hermann}\pend
           \leftskip=0em{}
         
         \endnumbering\mylabel{h}\end{ledgroupsized}  \newcommand{\dateiname}{L02361}\newcommand{\titel}{Hermann Bahr an Arthur Schnitzler, 9. 2. 1921}\newcommand{\editorInnen}{ Kurt Ifkovits,  Martin Anton Müller}%% latex-leseansicht-abspann.tex
%% Abspann für die Leseansicht.
%% Der Schalter \ifkorrekturansicht ist bereits durch den Vorspann gesetzt.

%% latex-abspann.tex
%% Gemeinsamer Abspann für Korrekturansicht und Leseansicht.
%% Setzt den Schalter \ifkorrekturansicht voraus (gesetzt in den
%% einbindenden Dateien latex-korrekturansicht-abspann.tex bzw.
%% latex-leseansicht-abspann.tex).
%% ---------------------------------------------------------------

\normalsize

% Das esempio-Environment wird nur in der Leseansicht benötigt
\ifkorrekturansicht\else
\newenvironment{esempio}[3]%
{
    \vspace{1.5ex}
    \rlap{\underline{#1}}
    \par
    \setlength{\parindent}{0cm}
    \nopagebreak
    \leftskip=#2cm
    \rightskip=#3cm
}
{
    \par
}
\fi

\doendnotes{C}
\bigskip
\vfill

\clearpage

\footnotesize

\ifkorrekturansicht
  \lohead{\textsc{register}}
\fi

% theindex-Environment neu definieren ohne reledmac
\makeatletter
\renewenvironment{theindex}{%
  \ifkorrekturansicht
    \section*{\indexname}%
  \else
    \subsubsection*{Index der erwähnten Entitäten}%
  \fi
  \setlength{\parindent}{0pt}%
  \setlength{\parskip}{0pt plus 0.3pt}%
  \let\item\@idxitem
}{%
  \ifkorrekturansicht\clearpage\fi
}
\makeatother

\IfFileExists{\jobname-pw.ind}{\input{\jobname-pw.ind}}{}

% Quellenangabe nur in der Leseansicht
\ifkorrekturansicht\else
% Fallback-Definitionen, falls die .tex-Datei \titel etc. nicht gesetzt hat
\providecommand{\titel}{}
\providecommand{\editorInnen}{}
\providecommand{\dateiname}{\jobname}

\vspace{3cm}

\vfill

\footnotesize
\textsc{Quelle}: \titel. Herausgegeben von {\editorInnen}. In: \emph{Arthur Schnitzler: Briefwechsel mit Autorinnen und Autoren}.
 Digitale Edition, https://schnitzler-briefe.acdh.oeaw.ac.at/{\dateiname}.html (Stand \today)
\fi

\end{document}


      