%% latex-leseansicht-vorspann.tex
%% Vorspann für die Leseansicht.
%% Lädt die gemeinsame Datei latex-vorspann.tex mit nicht gesetztem Schalter.

\newif\ifkorrekturansicht
\korrekturansichtfalse

\input{../tex-inputs/latex-vorspann}


         
         \newcommand{\erwaehntePersonen}{Personen: }
         \newcommand{\erwaehnteInstitutionen}{}
         \newcommand{\erwaehnteOrte}{}
         \newcommand{\erwaehnteWerke}{
               \section[Hermann Bahr an Arthur Schnitzler, 24. 1. 1898]{ Hermann Bahr an Arthur Schnitzler, 24. 1. 1898}\nopagebreak\mylabel{v}\rehead{ }\begin{ledgroupsized}[t]{13cm}\normalsize\beginnumbering \toendnotes[C]{\smallbreak\pagebreak[2]} \Standort{CUL, Schnitzler, B 5b.}
\physDesc{Brief, 1 Blatt, 2 Seiten
\newline{}Handschrift: schwarze Tinte, deutsche Kurrent\newline{}Ordnung: mit Bleistift von unbekannter Hand nummeriert: »58« }\buchAbdrucke{\weitereDrucke{Hermann Bahr, Arthur Schnitzler: \emph{Briefwechsel, Aufzeichnungen, Dokumente (1891–1931)}. Hg. Kurt Ifkovits und Martin Anton Müller. Göttingen: \emph{Wallstein} 2018, S. 160.} }\toendnotes[C]{\smallbreak}\pstart
           \noindent{}{\pb}\textcolor{gray}{\textbf{»Die ZeitXXXX ORGangabe fehlt«}}\hfill \textcolor{gray}{\textbf{\textbf{Wien\oindex{XXXX Ortsangabe fehlt|pw}}, den }}24 Januar \textcolor{gray}{\textbf{189}}8\pend
           \pstart
           \textcolor{gray}{\textbf{Wiener Wochenſchrift}}\hfill \textcolor{gray}{\textbf{IX/3, Günthergaſſe 1\oindex{XXXX Ortsangabe fehlt|pw}.}}\pend
           \pstart
           \textcolor{gray}{\textbf{\textbf{Herausgeber}:}}{\\}\textcolor{gray}{\textbf{Profeſſor Dr. I. Singer\pwindex{\textcolor{red}{\textsuperscript{XXXX1 indx}}|pw},
                        Hermann Bahr\pwindex{\textcolor{red}{\textsuperscript{XXXX1 indx}}|pw}, Dr. Heinrich Kanner\pwindex{\textcolor{red}{\textsuperscript{XXXX1 indx}}|pw}.}}\pend
           \pstart
           \textcolor{gray}{\textbf{Telephon Nr. 6415.}}\pend
           \pstart\center{}Lieber Arthur!\pend\pstart
           Ich bitte Dich, an einem Abschiedsabend für Burckhard\pwindex{\textcolor{red}{\textsuperscript{XXXX1 indx}}|pw} am 2. Februar theilzunehmen – ganz intim, jeder zahlt sein
               Couvert, wahrſcheinlich bei Sacher\oindex{XXXX Ortsangabe fehlt|pw}, etwa 40 Personen,
                  Saar\pwindex{\textcolor{red}{\textsuperscript{XXXX1 indx}}|pw}, Speidel\pwindex{\textcolor{red}{\textsuperscript{XXXX1 indx}}|pw}, Julius \textsc{Bauer}\pwindex{\textcolor{red}{\textsuperscript{XXXX1 indx}}|pw}, Groß\pwindex{\textcolor{red}{\textsuperscript{XXXX1 indx}}|pw}, Karlweis\pwindex{\textcolor{red}{\textsuperscript{XXXX1 indx}}|pw}, Chiavacci\pwindex{\textcolor{red}{\textsuperscript{XXXX1 indx}}|pw}, \textsc{Ebermann}\pwindex{\textcolor{red}{\textsuperscript{XXXX1 indx}}|pw}, einige Maler, Bukovics\pwindex{\textcolor{red}{\textsuperscript{XXXX1 indx}}|pw}, Gettke\pwindex{\textcolor{red}{\textsuperscript{XXXX1 indx}}|pw}, Baron \textsc{Berger}\pwindex{\textcolor{red}{\textsuperscript{XXXX1 indx}}|pw} uſw uſw. Hoffentlich biſt Du {\pb}dabei und
               ſchreibst baldigſt ein Ja\pend
           \pstart
           Deinem alten{\\[\baselineskip]}\spacefill\mbox{Hermann Bahr}\pend
           \leftskip=0em{}\pstart
           \textcolor{gray}{\textbf{\label{T_L00765_1v}\edtext{Alle für »Die ZeitXXXX ORGangabe fehlt« beſtimmten Zuſchriften und Sendungen
                  ſind an die Redaction der »ZeitXXXX ORGangabe fehlt« und \textbf{nicht} an die Perſon eines der Herausgeber oder Mitarbeiter zu
                     richten.}{\lemma{\textnormal{\emph{Alle … richten.}}}\Cendnote{\textnormal{am unteren
                     Rand der ersten Seite}}}\label{T_L00765_1h}}}\pend
           
         
         \endnumbering\mylabel{h}\end{ledgroupsized}  \newcommand{\dateiname}{L00765}\newcommand{\titel}{Hermann Bahr an Arthur Schnitzler, 24. 1. 1898}\newcommand{\editorInnen}{ Kurt Ifkovits,  Martin Anton Müller}%% latex-leseansicht-abspann.tex
%% Abspann für die Leseansicht.
%% Der Schalter \ifkorrekturansicht ist bereits durch den Vorspann gesetzt.

%% latex-abspann.tex
%% Gemeinsamer Abspann für Korrekturansicht und Leseansicht.
%% Setzt den Schalter \ifkorrekturansicht voraus (gesetzt in den
%% einbindenden Dateien latex-korrekturansicht-abspann.tex bzw.
%% latex-leseansicht-abspann.tex).
%% ---------------------------------------------------------------

\normalsize

% Das esempio-Environment wird nur in der Leseansicht benötigt
\ifkorrekturansicht\else
\newenvironment{esempio}[3]%
{
    \vspace{1.5ex}
    \rlap{\underline{#1}}
    \par
    \setlength{\parindent}{0cm}
    \nopagebreak
    \leftskip=#2cm
    \rightskip=#3cm
}
{
    \par
}
\fi

\doendnotes{C}
\bigskip
\vfill

\clearpage

\footnotesize

\ifkorrekturansicht
  \lohead{\textsc{register}}
\fi

% theindex-Environment neu definieren ohne reledmac
\makeatletter
\renewenvironment{theindex}{%
  \ifkorrekturansicht
    \section*{\indexname}%
  \else
    \subsubsection*{Index der erwähnten Entitäten}%
  \fi
  \setlength{\parindent}{0pt}%
  \setlength{\parskip}{0pt plus 0.3pt}%
  \let\item\@idxitem
}{%
  \ifkorrekturansicht\clearpage\fi
}
\makeatother

\IfFileExists{\jobname-pw.ind}{\input{\jobname-pw.ind}}{}

% Quellenangabe nur in der Leseansicht
\ifkorrekturansicht\else
% Fallback-Definitionen, falls die .tex-Datei \titel etc. nicht gesetzt hat
\providecommand{\titel}{}
\providecommand{\editorInnen}{}
\providecommand{\dateiname}{\jobname}

\vspace{3cm}

\vfill

\footnotesize
\textsc{Quelle}: \titel. Herausgegeben von {\editorInnen}. In: \emph{Arthur Schnitzler: Briefwechsel mit Autorinnen und Autoren}.
 Digitale Edition, https://schnitzler-briefe.acdh.oeaw.ac.at/{\dateiname}.html (Stand \today)
\fi

\end{document}


      