%% latex-leseansicht-vorspann.tex
%% Vorspann für die Leseansicht.
%% Lädt die gemeinsame Datei latex-vorspann.tex mit nicht gesetztem Schalter.

\newif\ifkorrekturansicht
\korrekturansichtfalse

\input{../tex-inputs/latex-vorspann}


\section[Felix Salten an Arthur Schnitzler, 17. 8. 1892]{L03112 Felix Salten an Arthur Schnitzler, 17. 8. 1892}
\nopagebreak\mylabel{L03112v}
\rehead{ }\normalsize\beginnumbering\briefempfaengerindex{Schnitzler, Arthur@\textsc{Schnitzler, Arthur}!zzzSalten, Felix@\emph{von Felix Salten}!1892-08-172@{17. 8. 1892}|(be}
\toendnotes[C]{\smallbreak\pagebreak[2]}
\correspDesc{Versand  durch Felix Salten am 17. 8. 1892 in Unterach am Attersee
\newline{}Erhalt  durch Arthur Schnitzler am 19. 8. 1892 in Wien}\toendnotes[C]{\smallbreak}
\Standort{CUL, Schnitzler, B 89, A 1.}
\physDesc{Brief, handschriftliche Abschrift. 2 Blätter, 5 Seiten, 2156 Zeichen
\newline{}HandschriftX2  : Bleistift, deutsche Kurrent
\newline{}Ordnung: mit Bleistift von unbekannter Hand nummeriert: »16« }\toendnotes[C]{\smallbreak}
\pstart
           \centering{}{\pb}(Brief von F. S.–), Unterach\oindex{Unterach am Attersee@\textbf{Unterach am Attersee}|pw}, 17/8. 1892\pend
           
\pstart
           \centering{}\label{K_L03112-1v}\edtext{Abſchrift}{\lemma{\textnormal{\emph{Abschrift}}}\Cendnote{\textnormal{Siehe A. S.: \emph{Tagebuch}, 1. 3. 1907.
                     Möglicherweise stellt diese frühe Abschrift ein Initialmoment dar, auf den hin
                        Schnitzler begann, seine jeweilige
                     Sekretärin mit Abschriften seiner wichtigsten Korrespondenzen zu
                     beauftragen.}}}\label{K_L03112-1} (1/3 907.)\pend
           \vspace{0.5em}{\vspace{1\baselineskip}}
\pstart
           Verehrteſter! Ich bin durch das was ich die ganzen
               Tage hier durchlebt, wirklich für mein \label{K_L03112-2v}\edtext{Vergehen}{\lemma{\textnormal{\emph{Vergehen}}}\Cendnote{\textnormal{Siehe XXXX Auszeichnungsfehler: Dokument L03186 nicht gefunden. Schnitzler kommentierte den Erhalt dieses Briefes am 19. 8. 1892 im \emph{Tagebuch}\pwindex{Schnitzler, Arthur 15.\,5.\,1862 Wien – 21.\,10.\,1931 ebd.@\textsc{Schnitzler, Arthur} (15.\,5.\,1862 Wien – 21.\,10.\,1931 ebd.), \emph{Schriftsteller, Mediziner}!Tagebuch@\strich\emph{Tagebuch}|pwk}: »Von S.\pwindex{Salten, Felix 6.\,9.\,1869 Budapest – 8.\,10.\,1945 Zürich@\textsc{Salten, Felix} (6.\,9.\,1869 Budapest – 8.\,10.\,1945 Zürich), \emph{Schriftsteller, Journalist, Chefredakteur}|pw} zerknirschter Brief, allerdings erst auf dringende
                     Aufforderung.«}}}\label{K_L03112-2} hart geſtraft, und nicht zuletzt iſt es Ihre
               Güte, die mich faſt ganz zu Boden drückt. Glauben Sie mir – und Sie \uline{können} mir \uline{jetzt}
               glauben, – ich{ }ſtehe vor mir{ }ſelber wie vor einem Rätſel! Ich will{ }ſehr kurz{ }ſein,
               Ihnen keine Phraſen machen. Erlaſſen Sie mir bitte, ein detailliertes Geſtänd{\pb}nis. Nehmen Sie als
               Wahrheit an, dſs ich \uline{Alles} wieder gut machen werde u.
               es i{\geminationm}er wollte, dſs aber nicht Alles, was Sie mir jetzt
               zuſchreiben, auf mein Kerbholz ko{\geminationm}t. Könnte ich Ihnen{ }ſagen, wie ich gelebt, wie meine häuslichen Umſtände waren, Sie würden manches
               begreifen, vielleicht auch mehr als ich{ }ſelbſt davon begreifen kann.\pend
           
\pstart
           Ich weiſs, dſs ich nun bei jedem andern Menſchen das Vertrauen verloren hätte, allein
               ich weiſs auch, dſs ich{ }ſelbſt bei Ihnen nicht {\pb}auf das »frühere Verhältnis«
               hoffen darf, allein das Eine will ich Ihnen{ }ſagen, dſs mir jetzt zu trauen iſt wie
               nur irgend Einem, dſs ich auch gute Keime in mir trage, die nicht vernichtet werden{ }ſollen, u daſs{ }ſolange ich denken u fühlen ka{\geminationn} mein
               Geiſt u meine Seele unzerbrüchlich Ihnen zu eigen bleibt.\pend
           
\pstart
           Es mag das erſtgradig klingen, doch ko{\geminationm}t es mir zu{ }ſehr
               aus tiefinnerſtem erſchüttertem Gemüth, als dſs ich es{ }ſtiliſiren könnte.\pend
           
\pstart
           Ich mache keinen Verſuch der Entſchuldigung, keinen Ihre Vertraulichkeit wieder zu
                  er{\pb}langen, allein ich
               erſehne den Tag, an dem Sie mich wieder genug{ }ſchätzen, um meine Freundſchaft zu
               erproben.\pend
           
\pstart
           Verzeihen Sie dſs dieſer Brief auf{ }ſich warten lieſs. Solange ich ganz verzweifelt
                  war{[},{]} konnte ich Ihnen nicht{ }ſchreiben, – ich hatte auch
               andres im Sinne, nun bin ich wieder etwas gefaſſter, u es bleibt mir nur die eine
               Bitte, daſs das Geſchehene zwiſchen uns an keinen Dritten verlaute. Ich habe zwar
               kein Recht darauf, allein ich ka{\geminationn} mirs noch erwerben.
               Ich bitte Sie um nichts als mir zu{ }ſchreiben, ob das{ }ſo{ }ſein{ }ſoll, oder ob ein {\pb}Dritter bereits darum
               weiſs\pend
           
\pstart
           Werden Sie mir das mittheilen?\pend
           
\pstart
           Ich bleibe indeſſen ich ihrer Antwort harre, wie man nur je einen Brief voll
               Sorge u Aufregung erwartet, {\\[\baselineskip]}\uline{Ihr}{ }\spacefill\mbox{Felix Salten}\pend
           \leftskip=0em{}
\pstart
           \noindent{}Unterach\oindex{Unterach am Attersee@\textbf{Unterach am Attersee}|pw}\pend
           
\pstart
           17/VIII 92\pend
           \selectlanguage{ngerman}\endnumbering\briefempfaengerindex{Schnitzler, Arthur@\textsc{Schnitzler, Arthur}!zzzSalten, Felix@\emph{von Felix Salten}!1892-08-172@{17. 8. 1892}|)be}\mylabel{L03112h}  \newcommand{\dateiname}{L03112}\newcommand{\titel}{Felix Salten an Arthur Schnitzler, 17. 8. 1892}\newcommand{\editorInnen}{Martin Anton Müller und Laura Untner}%% latex-leseansicht-abspann.tex
%% Abspann für die Leseansicht.
%% Der Schalter \ifkorrekturansicht ist bereits durch den Vorspann gesetzt.

%% latex-abspann.tex
%% Gemeinsamer Abspann für Korrekturansicht und Leseansicht.
%% Setzt den Schalter \ifkorrekturansicht voraus (gesetzt in den
%% einbindenden Dateien latex-korrekturansicht-abspann.tex bzw.
%% latex-leseansicht-abspann.tex).
%% ---------------------------------------------------------------

\normalsize

% Das esempio-Environment wird nur in der Leseansicht benötigt
\ifkorrekturansicht\else
\newenvironment{esempio}[3]%
{
    \vspace{1.5ex}
    \rlap{\underline{#1}}
    \par
    \setlength{\parindent}{0cm}
    \nopagebreak
    \leftskip=#2cm
    \rightskip=#3cm
}
{
    \par
}
\fi

\doendnotes{C}
\bigskip
\vfill

\clearpage

\footnotesize

\ifkorrekturansicht
  \lohead{\textsc{register}}
\fi

% theindex-Environment neu definieren ohne reledmac
\makeatletter
\renewenvironment{theindex}{%
  \ifkorrekturansicht
    \section*{\indexname}%
  \else
    \subsubsection*{Index der erwähnten Entitäten}%
  \fi
  \setlength{\parindent}{0pt}%
  \setlength{\parskip}{0pt plus 0.3pt}%
  \let\item\@idxitem
}{%
  \ifkorrekturansicht\clearpage\fi
}
\makeatother

\IfFileExists{\jobname-pw.ind}{\input{\jobname-pw.ind}}{}

% Quellenangabe nur in der Leseansicht
\ifkorrekturansicht\else
% Fallback-Definitionen, falls die .tex-Datei \titel etc. nicht gesetzt hat
\providecommand{\titel}{}
\providecommand{\editorInnen}{}
\providecommand{\dateiname}{\jobname}

\vspace{3cm}

\vfill

\footnotesize
\textsc{Quelle}: \titel. Herausgegeben von {\editorInnen}. In: \emph{Arthur Schnitzler: Briefwechsel mit Autorinnen und Autoren}.
 Digitale Edition, https://schnitzler-briefe.acdh.oeaw.ac.at/{\dateiname}.html (Stand \today)
\fi

\end{document}


