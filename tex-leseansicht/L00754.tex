%% latex-korrekturansicht-vorspann.tex
%% Vorspann für die Korrekturansicht.
%% Lädt die gemeinsame Datei latex-vorspann.tex mit gesetztem Schalter.

\newif\ifkorrekturansicht
\korrekturansichttrue

\input{../tex-inputs/latex-vorspann}


\section[Georg Brandes an Arthur Schnitzler, 28. 12. 1897]{L00754 Georg Brandes an Arthur Schnitzler, 28. 12. 1897}
\nopagebreak\mylabel{L00754v}
\rehead{ }\normalsize\beginnumbering\briefempfaengerindex{Schnitzler, Arthur@\textsc{Schnitzler, Arthur}!zzzBrandes, Georg@\emph{von Georg Brandes}!1897-12-281@{28. 12. 1897}|(be}
\toendnotes[C]{\smallbreak\pagebreak[2]}\Standort{CUL, Schnitzler, B 17.}
\physDesc{Brief, 1 Blatt, 1 Seite, 423 Zeichen
\newline{}Handschrift: blaue Tinte, lateinische Kurrent
\newline{}Ordnung: mit Bleistift von unbekannter Hand nummeriert:
                                 »7« }
\buchAbdrucke{\weitereDrucke{Georg Brandes, Arthur Schnitzler: \emph{Ein Briefwechsel}. Bern: \emph{Francke} 1956, S. 65.} }
\pstart
           \raggedleft{}{\pb}Kopenhagen\oindex{Kopenhagen@\textbf{Kopenhagen}, \emph{P.PPLC}|pw}{\\}28 Dec. 97\pend
           
\pstart\center{}Lieber Herr Doctor\pend\vspace{0.5em}
\pstart
           Ich werde im Anfang von Januar von hier reisen und vielleicht gegen die Mitte des
               Monats \introOben{}auf der Reise südwärts\introOben{} in Wien\oindex{Wien@\textbf{Wien}, \emph{A.ADM2}|pw} ankommen können. Ob ich die Stadt einige Tage besuche,
               hängt zum Theil davon ab, ob ich Sie und Herrn Hofmann-Beer\pwindex{Beer-Hofmann, Richard 1866-07-11 – 1945-09-26@\textsc{Beer-Hofmann, Richard} (1866-07-11 – 1945-09-26), \emph{Schriftsteller/Schriftstellerin}|pw} in Wien\oindex{Wien@\textbf{Wien}, \emph{A.ADM2}|pw} treffen werde; ich
               kenne nur wenige Personen dort.\pend
           
\pstart
           Um eine Zeile bittet deshalb{\\[\baselineskip]}Ihr ergebener{\\[\baselineskip]}\spacefill\mbox{Georg Brandes}\pend
           \leftskip=0em{}
\pstart
           \noindent{}Nennen Sie mir ein gutes und angenehmes Hotel.\pend
           \selectlanguage{ngerman}\endnumbering\briefempfaengerindex{Schnitzler, Arthur@\textsc{Schnitzler, Arthur}!zzzBrandes, Georg@\emph{von Georg Brandes}!1897-12-281@{28. 12. 1897}|)be}\mylabel{L00754h}  \normalsize

\doendnotes{C}
\bigskip
\vfill

\clearpage

\footnotesize

\lohead{\textsc{register}}

% Definiere theindex-Environment komplett neu ohne reledmac
\makeatletter
\renewenvironment{theindex}{%
  \section*{\indexname}%
  \setlength{\parindent}{0pt}%
  \setlength{\parskip}{0pt plus 0.3pt}%
  \let\item\@idxitem
}{%
  \clearpage
}
\makeatother

\IfFileExists{\jobname-pw.ind}{\input{\jobname-pw.ind}}{}

\end{document}

      