%% latex-leseansicht-vorspann.tex
%% Vorspann für die Leseansicht.
%% Lädt die gemeinsame Datei latex-vorspann.tex mit nicht gesetztem Schalter.

\newif\ifkorrekturansicht
\korrekturansichtfalse

\input{../tex-inputs/latex-vorspann}


         
         \renewcommand{\erwaehntePersonen}{Personen: Hermann Bahr, Felix Salten}
         \renewcommand{\erwaehnteOrte}{Orte: Café Arkaden, Wien}
         \renewcommand{\erwaehnteWerke}{}
               \section[ Felix Salten an Arthur Schnitzler, {[}7. 2. 1895{]}]{ Felix Salten an Arthur Schnitzler, {[}7. 2. 1895{]}}\nopagebreak\mylabel{v}\rehead{ }\begin{ledgroupsized}[t]{13cm}\normalsize\beginnumbering\briefempfaengerindex{Schnitzler, Arthur@\textsc{Schnitzler, Arthur}!zzzSalten, Felix@\emph{von Felix Salten}!1895-02-071@{{[}7. 2. 1895{]}}|(be} \toendnotes[C]{\smallbreak\pagebreak[2]} \Standort{CUL, Schnitzler, B 89, A 1.}
\physDesc{Brief, 1 Blatt, 1 Seite, 440 Zeichen
\newline{}Handschrift: Bleistift, lateinische Kurrent
\newline{}Schnitzler: mit Bleistift datiert: »7/2 95« 
\newline{}Ordnung: mit Bleistift von unbekannter Hand nummeriert: »51« }\buchAbdrucke{\weitereDrucke{Hermann Bahr, Arthur Schnitzler: \emph{Briefwechsel, Aufzeichnungen, Dokumente (1891–1931)}. Hg. Kurt Ifkovits und Martin Anton Müller. Göttingen: \emph{Wallstein} 2018, S. 96.} }\toendnotes[C]{\smallbreak}\pstart
           \noindent{}{\pb}\label{K_L03150-1v}\edtext{L. F.}{\lemma{\textnormal{\emph{L. F.}}}\Cendnote{\textnormal{Lieber Freund}}}\label{K_L03150-1h} Von Bahr\pwindex{Bahr, Hermann 19.07.1863 – 15.01.1934@\textsc{Bahr, Hermann} (19.07.1863 – 15.01.1934), \emph{Schriftsteller, Kritiker}|pw} noch lange aufgehalten, kam ich leider zu
               spät ins Caféhaus. Ich bedaure das am meisten, weil ich gewünscht hätte, mich gleich
               mit Ihnen \label{K_L03150-2v}\edtext{auseinanderzusetzen}{\lemma{\textnormal{\emph{auseinanderzusetzen}}}\Cendnote{\textnormal{Ein senkrechter Strich nach
                     »ausein« könnte darauf hindeuten, dass Salten\pwindex{Salten, Felix 06.09.1869 – 08.10.1945@\textsc{Salten, Felix} (06.09.1869 – 08.10.1945), \emph{Schriftsteller, Journalist}|pwk} hier nachträglich eine Trennung des Wortes andeuten
                  wollte.}}}\label{K_L03150-2h}. Es wäre mir sehr werthvoll, wenn ich Sie \uline{jetzt gleich} sprechen könnte, oder zu Mittag. Wollen Sie
               nicht \introOben{}jetzt\introOben{} auf einem Sprung \label{K_L03150-3v}\edtext{ins Arcadencafé\oindex{Cafe Arkaden@\textbf{Café Arkaden}|pw}
                  kommen}{\lemma{\textnormal{\emph{ins Arcadencafé
                  kommen}}}\Cendnote{\textnormal{Ein solches Treffen ist nicht belegt.}}}\label{K_L03150-3h}? Ich
               würde die Sache nur höchst ungern auf \substVorne{}\textsuperscript{n}\substDazwischen{}N\substHinten{}achmittag verschoben sehen, da mir für N. M. noch
               vieles zu thun \substVorne{}\textsuperscript{\textcolor{gray}{u}}\substDazwischen{}b\substHinten{}leibt.\pend
           \pstart
           Ihr treuer {\\[\baselineskip]}\spacefill\mbox{Salten}\pend
           \leftskip=0em{}
         
         \endnumbering\mylabel{h}\end{ledgroupsized}  \newcommand{\dateiname}{L03150}\newcommand{\titel}{Felix Salten an Arthur Schnitzler, [7. 2. 1895]}\newcommand{\editorInnen}{Martin Anton Müller und Laura Untner}%% latex-leseansicht-abspann.tex
%% Abspann für die Leseansicht.
%% Der Schalter \ifkorrekturansicht ist bereits durch den Vorspann gesetzt.

%% latex-abspann.tex
%% Gemeinsamer Abspann für Korrekturansicht und Leseansicht.
%% Setzt den Schalter \ifkorrekturansicht voraus (gesetzt in den
%% einbindenden Dateien latex-korrekturansicht-abspann.tex bzw.
%% latex-leseansicht-abspann.tex).
%% ---------------------------------------------------------------

\normalsize

% Das esempio-Environment wird nur in der Leseansicht benötigt
\ifkorrekturansicht\else
\newenvironment{esempio}[3]%
{
    \vspace{1.5ex}
    \rlap{\underline{#1}}
    \par
    \setlength{\parindent}{0cm}
    \nopagebreak
    \leftskip=#2cm
    \rightskip=#3cm
}
{
    \par
}
\fi

\doendnotes{C}
\bigskip
\vfill

\clearpage

\footnotesize

\ifkorrekturansicht
  \lohead{\textsc{register}}
\fi

% theindex-Environment neu definieren ohne reledmac
\makeatletter
\renewenvironment{theindex}{%
  \ifkorrekturansicht
    \section*{\indexname}%
  \else
    \subsubsection*{Index der erwähnten Entitäten}%
  \fi
  \setlength{\parindent}{0pt}%
  \setlength{\parskip}{0pt plus 0.3pt}%
  \let\item\@idxitem
}{%
  \ifkorrekturansicht\clearpage\fi
}
\makeatother

\IfFileExists{\jobname-pw.ind}{\input{\jobname-pw.ind}}{}

% Quellenangabe nur in der Leseansicht
\ifkorrekturansicht\else
% Fallback-Definitionen, falls die .tex-Datei \titel etc. nicht gesetzt hat
\providecommand{\titel}{}
\providecommand{\editorInnen}{}
\providecommand{\dateiname}{\jobname}

\vspace{3cm}

\vfill

\footnotesize
\textsc{Quelle}: \titel. Herausgegeben von {\editorInnen}. In: \emph{Arthur Schnitzler: Briefwechsel mit Autorinnen und Autoren}.
 Digitale Edition, https://schnitzler-briefe.acdh.oeaw.ac.at/{\dateiname}.html (Stand \today)
\fi

\end{document}


      