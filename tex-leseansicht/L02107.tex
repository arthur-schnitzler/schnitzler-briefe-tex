%% latex-korrekturansicht-vorspann.tex
%% Vorspann für die Korrekturansicht.
%% Lädt die gemeinsame Datei latex-vorspann.tex mit gesetztem Schalter.

\newif\ifkorrekturansicht
\korrekturansichttrue

\input{../tex-inputs/latex-vorspann}


\section[Hermann Bahr an Arthur Schnitzler, 7. 12. 1912]{L02107 Hermann Bahr an Arthur Schnitzler, 7. 12. 1912}
\nopagebreak\mylabel{L02107v}
\rehead{ }\normalsize\beginnumbering\briefempfaengerindex{Schnitzler, Arthur@\textsc{Schnitzler, Arthur}!zzzBahr, Hermann@\emph{von Hermann Bahr}!1912-12-071@{7. 12. 1912}|(be}
\toendnotes[C]{\smallbreak\pagebreak[2]}\Standort{CUL, Schnitzler, B 5b.}
\physDesc{Brief, 1 Blatt, 1 Seite, 886 Zeichen
\newline{}Handschrift: schwarze Tinte, deutsche Kurrent
\newline{}Ordnung: mit Bleistift von unbekannter Hand nummeriert:
                                    »175« und ergänzt: »\textsc{Bahr}« }
\buchAbdrucke{\weitereDrucke{Hermann Bahr, Arthur Schnitzler: \emph{Briefwechsel, Aufzeichnungen, Dokumente (1891–1931)}. Göttingen: \emph{Wallstein} 2018, S. 479.} }\toendnotes[C]{\smallbreak}
\pstart
           {\pb}\textcolor{gray}{\textbf{GRAND HOTEL DE L’EUROPE\oindex{Grand Hotel de L Europe, G. Jung@\textbf{Grand Hotel de L’Europe, G. Jung}, \emph{Hotel (K.HTL)}|pw}}}\pend
           
\pstart
           \textcolor{gray}{\textbf{G. JUNG\pwindex{Jung, Georg 1866 – 1934@\textsc{Jung, Georg} (1866 – 1934), \emph{Hotelbesitzer/Hotelbesitzerin}|pw}}}\pend
           
\pstart
           \raggedleft{}\textcolor{gray}{\textbf{Salzburg\oindex{Salzburg@\textbf{Salzburg}, \emph{A.ADM2}|pw}, }}{ }7. 12. 12\pend
           
\pstart\center{}Lieber Arthur!\pend\vspace{0.5em}
\pstart
           Ich war ſechs Wochen unterwegs, jeden Abend in einer anderen Stadt auf dem »Brettl«,
               ſo komm ich nun hier erſt dazu, Deinen lieben Brief zu beantworten. An Altenberg\pwindex{Altenberg, Peter 09.03.1859 – 08.01.1919@\textsc{Altenberg, Peter} (09.03.1859 – 08.01.1919), \emph{Schriftsteller/Schriftstellerin}|pw} kann ich mich nicht beteiligen. Ich tu
               nach meinem Gefühl genug für andere, für anonyme Armut, die mich braucht und ohne
               mich ſich keinen Rat wüßte, während der Betrag, den ich dem guten Peter\pwindex{Altenberg, Peter 09.03.1859 – 08.01.1919@\textsc{Altenberg, Peter} (09.03.1859 – 08.01.1919), \emph{Schriftsteller/Schriftstellerin}|pw} geben könnte, für ihn nichts bedeuten würde und er
               tauſendfach Gelegenheit hat, ſich ihn zu beſchaffen. Misverſteh mich \substVorne{}\textsuperscript{\textcolor{gray}{ſ}\textcolor{gray}{×}\-\textcolor{gray}{×}}\substDazwischen{}ni\substHinten{}cht: ich ſchätze Altenberg\pwindex{Altenberg, Peter 09.03.1859 – 08.01.1919@\textsc{Altenberg, Peter} (09.03.1859 – 08.01.1919), \emph{Schriftsteller/Schriftstellerin}|pw} als Dichter
               ſehr, aber als »Armen« gar nicht, auf dieſem Gebiet leiſten andere viel mehr.\pend
           
\pstart
           Ich freue mich ſehr über alle Deine Erfolge und habe das gute Gefühl, daß Du nun »in
               Fülle« haſt, was Du Dir je gewünſcht. Möge es Dir ſo bleiben! Und auch Deiner lieben
                  Frau\pwindex{Schnitzler, Olga 17.01.1882 – 13.01.1970@\textsc{Schnitzler, Olga} (17.01.1882 – 13.01.1970), \emph{Schauspieler/Schauspielerin, Sänger/Sängerin}|pwv} und den Kindern\pwindex{Schnitzler, Heinrich 09.08.1902 – 12.07.1982@\textsc{Schnitzler, Heinrich} (09.08.1902 – 12.07.1982), \emph{Regisseur/Regisseurin, Schauspieler/Schauspielerin}|pw}\pwindex{Cappellini, Lili 13.09.1909 – 26.07.1928@\textsc{Cappellini, Lili} (13.09.1909 – 26.07.1928)|pw} wünſch ich immer alles
               Beſte!\pend
           
\pstart
           Mit den ſchönſten Grüßen von uns Beiden\pwindex{Bahr-Mildenburg, Anna 29.11.1872 – 27.01.1947@\textsc{Bahr-Mildenburg, Anna} (29.11.1872 – 27.01.1947), \emph{Sänger/Sängerin}|pwv}{\\[\baselineskip]}Dein alter{\\[\baselineskip]}\spacefill\mbox{Hermann}\pend
           \leftskip=0em{}\selectlanguage{ngerman}\endnumbering\briefempfaengerindex{Schnitzler, Arthur@\textsc{Schnitzler, Arthur}!zzzBahr, Hermann@\emph{von Hermann Bahr}!1912-12-071@{7. 12. 1912}|)be}\mylabel{L02107h}  \normalsize

\doendnotes{C}
\bigskip
\vfill

\clearpage

\footnotesize

\lohead{\textsc{register}}

% Definiere theindex-Environment komplett neu ohne reledmac
\makeatletter
\renewenvironment{theindex}{%
  \section*{\indexname}%
  \setlength{\parindent}{0pt}%
  \setlength{\parskip}{0pt plus 0.3pt}%
  \let\item\@idxitem
}{%
  \clearpage
}
\makeatother

\IfFileExists{\jobname-pw.ind}{\input{\jobname-pw.ind}}{}

\end{document}

      