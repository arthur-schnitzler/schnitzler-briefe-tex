%% latex-leseansicht-vorspann.tex
%% Vorspann für die Leseansicht.
%% Lädt die gemeinsame Datei latex-vorspann.tex mit nicht gesetztem Schalter.

\newif\ifkorrekturansicht
\korrekturansichtfalse

\input{../tex-inputs/latex-vorspann}


         
         \newcommand{\erwaehntePersonen}{Personen: Albert Ehrenstein}
         \newcommand{\erwaehnteOrte}{Orte: Ottakringerstraße, Semmering, Südbahnhotel, Wien, XIV., Penzing}
         \newcommand{\erwaehnteWerke}{
               \section[Arthur Schnitzler an Albert Ehrenstein, 26. 1. 1909]{ Arthur Schnitzler an Albert Ehrenstein,
                    26. 1. 1909}\nopagebreak\mylabel{v}\rehead{ }\begin{ledgroupsized}[t]{13cm}\normalsize\beginnumbering \toendnotes[C]{\smallbreak\pagebreak[2]} \Standort{Jerusalem, The National Library of Israel, ARC. Ms. Var. 306 1 118.}
\physDesc{Bildpostkarte
\newline{}Handschrift: Bleistift, deutsche Kurrent\newline{}Versand: Stempel: »\nobreak{}\oindex{Semmering@\textbf{Semmering}|pwk}Semmering, 2{[}6. 1. 190{]}9\nobreak{}«.  \newline{}Ordnung: mit Bleistift von unbekannter Hand
                                    nummeriert: »9« }\pstart{}{\pb}Hrn \textsc{Albert Ehrenstein}\pend{}\pstart{}\textsc{Wien XIV\oindex{XIV., Penzing@\textbf{XIV., Penzing}|pw}}\pend{}\pstart{}\textsc{Ottakringerstr. 114\oindex{Ottakringerstrasse@\textbf{Ottakringerstraße}|pw}}\pend{}{\bigskip}\pstart
           \noindent{}\centering{}\textcolor{gray}{\textbf{{\pb}Semmering.\hspace*{1.5em}Südbahnhotel.\oindex{Suedbahnhotel@\textbf{Südbahnhotel}|pw}}}\pend
           \pstart
           {\pb}Zu mündlicher Erkärung in Wien\oindex{Wien@\textbf{Wien}|pw}, etwa in 10–14 Tagen, gern bereit.\pend
           \pstart
           Beſtens grüßend{\\[\baselineskip]}bis dahin Ihr{\\[\baselineskip]}\spacefill\mbox{Arth. Sch.}\pend
           \leftskip=0em{}\pstart
           26. 1. 09.\pend
           
         
         \endnumbering\mylabel{h}\end{ledgroupsized}  \newcommand{\dateiname}{L01828}\newcommand{\titel}{Arthur Schnitzler an Albert Ehrenstein, 26. 1. 1909}\newcommand{\editorInnen}{Martin Anton Müller und Gerd-Hermann Susen}%% latex-leseansicht-abspann.tex
%% Abspann für die Leseansicht.
%% Der Schalter \ifkorrekturansicht ist bereits durch den Vorspann gesetzt.

%% latex-abspann.tex
%% Gemeinsamer Abspann für Korrekturansicht und Leseansicht.
%% Setzt den Schalter \ifkorrekturansicht voraus (gesetzt in den
%% einbindenden Dateien latex-korrekturansicht-abspann.tex bzw.
%% latex-leseansicht-abspann.tex).
%% ---------------------------------------------------------------

\normalsize

% Das esempio-Environment wird nur in der Leseansicht benötigt
\ifkorrekturansicht\else
\newenvironment{esempio}[3]%
{
    \vspace{1.5ex}
    \rlap{\underline{#1}}
    \par
    \setlength{\parindent}{0cm}
    \nopagebreak
    \leftskip=#2cm
    \rightskip=#3cm
}
{
    \par
}
\fi

\doendnotes{C}
\bigskip
\vfill

\clearpage

\footnotesize

\ifkorrekturansicht
  \lohead{\textsc{register}}
\fi

% theindex-Environment neu definieren ohne reledmac
\makeatletter
\renewenvironment{theindex}{%
  \ifkorrekturansicht
    \section*{\indexname}%
  \else
    \subsubsection*{Index der erwähnten Entitäten}%
  \fi
  \setlength{\parindent}{0pt}%
  \setlength{\parskip}{0pt plus 0.3pt}%
  \let\item\@idxitem
}{%
  \ifkorrekturansicht\clearpage\fi
}
\makeatother

\IfFileExists{\jobname-pw.ind}{\input{\jobname-pw.ind}}{}

% Quellenangabe nur in der Leseansicht
\ifkorrekturansicht\else
% Fallback-Definitionen, falls die .tex-Datei \titel etc. nicht gesetzt hat
\providecommand{\titel}{}
\providecommand{\editorInnen}{}
\providecommand{\dateiname}{\jobname}

\vspace{3cm}

\vfill

\footnotesize
\textsc{Quelle}: \titel. Herausgegeben von {\editorInnen}. In: \emph{Arthur Schnitzler: Briefwechsel mit Autorinnen und Autoren}.
 Digitale Edition, https://schnitzler-briefe.acdh.oeaw.ac.at/{\dateiname}.html (Stand \today)
\fi

\end{document}


      