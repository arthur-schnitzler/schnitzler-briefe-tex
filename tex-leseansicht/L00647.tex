%% latex-korrekturansicht-vorspann.tex
%% Vorspann für die Korrekturansicht.
%% Lädt die gemeinsame Datei latex-vorspann.tex mit gesetztem Schalter.

\newif\ifkorrekturansicht
\korrekturansichttrue

\input{../tex-inputs/latex-vorspann}


\section[Arthur Schnitzler an Hermann Bahr, 26. 2. 1897]{L00647 Arthur Schnitzler an Hermann Bahr, 26. 2. 1897}
\nopagebreak\mylabel{L00647v}
\rehead{ }\normalsize\beginnumbering\briefempfaengerindex{Bahr, Hermann@\textsc{Bahr, Hermann}!zzzSchnitzler, Arthur@\emph{von Arthur Schnitzler}!1897-02-262@{26. 2. 1897}|(be}
\toendnotes[C]{\smallbreak\pagebreak[2]}\Standort{TMW, HS AM 23328 Ba.}
\physDesc{Brief, 1 Blatt, 1 Seite, 203 Zeichen
\newline{}Handschrift: schwarze Tinte, deutsche Kurrent}
\buchAbdrucke{\weitereDrucke{1) Arthur Schnitzler: \emph{The Letters of Arthur Schnitzler to Hermann Bahr}. Chapel Hill: \emph{The University of North Carolina Press} 1978, S. 60.} \weitereDrucke{2) Hermann Bahr, Arthur Schnitzler: \emph{Briefwechsel, Aufzeichnungen, Dokumente (1891–1931)}. Göttingen: \emph{Wallstein} 2018, S. 136.} }
\pstart
           \noindent{}{\pb}Lieber Hermann, es iſt ſehr freundlich dſs du dich ſelbſt bemüht u
               mir die Karte ſelbſt geſchickt haſt. Erlaube mir gleich meine Schuld abzutragen.\pend
           
\pstart
           Mit herzlichem Dank und Gruſs{\\[\baselineskip]}dein \spacefill\mbox{ArthurSch}\pend
           \leftskip=0em{}
\pstart
           Wien\oindex{Wien@\textbf{Wien}, \emph{A.ADM2}|pw}, 26. 2. 97.\pend
           \selectlanguage{ngerman}\endnumbering\briefempfaengerindex{Bahr, Hermann@\textsc{Bahr, Hermann}!zzzSchnitzler, Arthur@\emph{von Arthur Schnitzler}!1897-02-262@{26. 2. 1897}|)be}\mylabel{L00647h}  \normalsize

\doendnotes{C}
\bigskip
\vfill

\clearpage

\footnotesize

\lohead{\textsc{register}}

% Definiere theindex-Environment komplett neu ohne reledmac
\makeatletter
\renewenvironment{theindex}{%
  \section*{\indexname}%
  \setlength{\parindent}{0pt}%
  \setlength{\parskip}{0pt plus 0.3pt}%
  \let\item\@idxitem
}{%
  \clearpage
}
\makeatother

\IfFileExists{\jobname-pw.ind}{\input{\jobname-pw.ind}}{}

\end{document}

      