%% latex-leseansicht-vorspann.tex
%% Vorspann für die Leseansicht.
%% Lädt die gemeinsame Datei latex-vorspann.tex mit nicht gesetztem Schalter.

\newif\ifkorrekturansicht
\korrekturansichtfalse

\input{../tex-inputs/latex-vorspann}


         
         \renewcommand{\erwaehntePersonen}{Personen: Rudolf Münger}
         \renewcommand{\erwaehnteInstitutionen}{Institutionen: Stadttheater Bern}
         \renewcommand{\erwaehnteOrte}{Orte: Bern, IX., Alsergrund, Wien, Österreich}
         \renewcommand{\erwaehnteWerke}{Werke: Der grüne Kakadu. Groteske in einem Akt}
               \section[Joseph Victor Widmann an Arthur Schnitzler, 9. 5. 1903]{ Joseph Victor Widmann an Arthur Schnitzler, 9. 5. 1903}\nopagebreak\mylabel{v}\rehead{ }\begin{ledgroupsized}[t]{13cm}\normalsize\beginnumbering \toendnotes[C]{\smallbreak\pagebreak[2]} \Standort{CUL, Schnitzler, B 113.}
\physDesc{Bildpostkarte, 258 Zeichen
\newline{}Handschrift: schwarze Tinte, deutsche Kurrent
\newline{}Versand: 1) Stempel: »\nobreak{}\oindex{Bern@\textbf{Bern}|pwk}Bern Brf. Exp., 10. V. 03., 11\nobreak{}«.   2) Stempel: »\nobreak{}\oindex{IX., Alsergrund@\textbf{IX., Alsergrund}|pwk}Wien 9/3, 11. 5. 03, 5N, Bestellt\nobreak{}«.  3) mit Bleistift von unbekannter Hand zur Adresse ergänzt:
                                    »IX/3«
\newline{}Zusatz: auf dem Motiv im Vordergrund eine Illustration von Rudolf Münger\pwindex{Muenger, Rudolf 1862-11-10 – 1929-09-17@\textsc{Münger, Rudolf} (1862-11-10 – 1929-09-17), \emph{Maler, Grafiker}|pw} zu Schnitzlers
                                    Der grüne Kakadu\pwindex{Schnitzler, Arthur 15.05.1862 – 21.10.1931@\textsc{Schnitzler, Arthur} (15.05.1862 – 21.10.1931), \emph{Schriftsteller, Mediziner}!gruene Kakadu. Groteske in einem Akt1. 3. 1899@\strich\emph{Der grüne Kakadu. Groteske in einem Akt} {[}1. 3. 1899{]}|pw}\pwindex{Schnitzler, Arthur 15.05.1862 – 21.10.1931@\textsc{Schnitzler, Arthur} (15.05.1862 – 21.10.1931), \emph{Schriftsteller, Mediziner}!gruene Kakadu. Groteske in einem Akt1. 3. 1899@\strich\emph{Der grüne Kakadu. Groteske in einem Akt} {[}1. 3. 1899{]}|pw}, der Vogel
                                 und der Bildrahmen grün koloriert }\toendnotes[C]{\smallbreak}\pstart{}{\pb}\textsc{Herrn D\textsuperscript{r} Arthur Schnitzler}\pend{}\pstart{}Dichter in\pend{}\pstart{}\textsc{Wien}\oindex{Wien@\textbf{Wien}|pw}.\pend{}\pstart{}(\textsc{Österreich\oindex{Oesterreich@\textbf{Österreich}|pw}}.)\pend{}{\bigskip}\pstart
           \noindent{}\centering{}\textcolor{gray}{\textbf{{\pb}⋅1903⋅
                     Theater-Bazar}}\pend
           \pstart
           Da ſehn Sie, verehrteſter Herr, wie man in Bern\oindex{Bern@\textbf{Bern}|pw}{ }Sie liebt u. ke{\geminationn}t und
               ſchätzt. Und natürlich aufführt, ſobald das neue Theater\orgindex{Stadttheater Bern@Stadttheater Bern|pwv} in dieſem Herbſt ſeine Hallen
               öffnet.\pend
           \pstart
           Mit höflichem Gruß{\\[\baselineskip]}\spacefill\mbox{J. V. Widmann}\pend
           \leftskip=0em{}\pstart
           \textcolor{gray}{\textbf{BERN\oindex{Bern@\textbf{Bern}|pw}, den}}{ }9. Mai 1903.\pend
           
         
         \endnumbering\mylabel{h}\end{ledgroupsized}  \newcommand{\dateiname}{L01288}\newcommand{\titel}{Joseph Victor Widmann an Arthur Schnitzler, 9. 5. 1903}\newcommand{\editorInnen}{Martin Anton Müller und Gerd-Hermann Susen}%% latex-leseansicht-abspann.tex
%% Abspann für die Leseansicht.
%% Der Schalter \ifkorrekturansicht ist bereits durch den Vorspann gesetzt.

%% latex-abspann.tex
%% Gemeinsamer Abspann für Korrekturansicht und Leseansicht.
%% Setzt den Schalter \ifkorrekturansicht voraus (gesetzt in den
%% einbindenden Dateien latex-korrekturansicht-abspann.tex bzw.
%% latex-leseansicht-abspann.tex).
%% ---------------------------------------------------------------

\normalsize

% Das esempio-Environment wird nur in der Leseansicht benötigt
\ifkorrekturansicht\else
\newenvironment{esempio}[3]%
{
    \vspace{1.5ex}
    \rlap{\underline{#1}}
    \par
    \setlength{\parindent}{0cm}
    \nopagebreak
    \leftskip=#2cm
    \rightskip=#3cm
}
{
    \par
}
\fi

\doendnotes{C}
\bigskip
\vfill

\clearpage

\footnotesize

\ifkorrekturansicht
  \lohead{\textsc{register}}
\fi

% theindex-Environment neu definieren ohne reledmac
\makeatletter
\renewenvironment{theindex}{%
  \ifkorrekturansicht
    \section*{\indexname}%
  \else
    \subsubsection*{Index der erwähnten Entitäten}%
  \fi
  \setlength{\parindent}{0pt}%
  \setlength{\parskip}{0pt plus 0.3pt}%
  \let\item\@idxitem
}{%
  \ifkorrekturansicht\clearpage\fi
}
\makeatother

\IfFileExists{\jobname-pw.ind}{\input{\jobname-pw.ind}}{}

% Quellenangabe nur in der Leseansicht
\ifkorrekturansicht\else
% Fallback-Definitionen, falls die .tex-Datei \titel etc. nicht gesetzt hat
\providecommand{\titel}{}
\providecommand{\editorInnen}{}
\providecommand{\dateiname}{\jobname}

\vspace{3cm}

\vfill

\footnotesize
\textsc{Quelle}: \titel. Herausgegeben von {\editorInnen}. In: \emph{Arthur Schnitzler: Briefwechsel mit Autorinnen und Autoren}.
 Digitale Edition, https://schnitzler-briefe.acdh.oeaw.ac.at/{\dateiname}.html (Stand \today)
\fi

\end{document}


      