%% latex-korrekturansicht-vorspann.tex
%% Vorspann für die Korrekturansicht.
%% Lädt die gemeinsame Datei latex-vorspann.tex mit gesetztem Schalter.

\newif\ifkorrekturansicht
\korrekturansichttrue

\input{../tex-inputs/latex-vorspann}


\section[Joseph Victor Widmann an Arthur Schnitzler, 9. 5. 1903]{L01288 Joseph Victor Widmann an Arthur Schnitzler, 9. 5. 1903}
\nopagebreak\mylabel{L01288v}
\rehead{ }\normalsize\beginnumbering\briefempfaengerindex{Schnitzler, Arthur@\textsc{Schnitzler, Arthur}!zzzWidmann, Joseph Victor@\emph{von Joseph Victor Widmann}!1903-05-091@{9. 5. 1903}|(be}
\toendnotes[C]{\smallbreak\pagebreak[2]}\Standort{CUL, Schnitzler, B 113.}
\physDesc{Bildpostkarte, 258 Zeichen
\newline{}Handschrift: schwarze Tinte, deutsche Kurrent
\newline{}Versand: 1) Stempel: »\nobreak{}\oindex{Bern@\textbf{Bern}, \emph{P.PPLC}|pwk}Bern Brf. Exp., 10. V. 03., 11\nobreak{}«.   2) Stempel: »\nobreak{}\oindex{IX., Alsergrund@\textbf{IX., Alsergrund}, \emph{A.ADM3}|pwk}Wien 9/3, 11. 5. 03, 5N, Bestellt\nobreak{}«.  3) mit Bleistift von unbekannter Hand zur Adresse ergänzt:
                                 »IX/3«
\newline{}Zusatz: auf dem Motiv im Vordergrund eine Illustration von Rudolf Münger\pwindex{Muenger, Rudolf 1862-11-10 – 1929-09-17@\textsc{Münger, Rudolf} (1862-11-10 – 1929-09-17), \emph{Maler/Malerin, Grafiker/Grafikerin}|pw} zu Schnitzlers
                                    Der grüne Kakadu\pwindex{gruene Kakadu. Groteske in einem Akt@\emph{Der grüne Kakadu. Groteske in einem Akt}|pw}, der Vogel
                                 und der Bildrahmen grün koloriert }\toendnotes[C]{\smallbreak}\pstart{}{\pb}\textsc{Herrn D\textsuperscript{r} Arthur Schnitzler}\pend{}\pstart{}Dichter in\pend{}\pstart{}\textsc{Wien}\oindex{Wien@\textbf{Wien}, \emph{A.ADM2}|pw}.\pend{}\pstart{}(\textsc{Österreich\oindex{Oesterreich@\textbf{Österreich}, \emph{A.PCLI}|pw}}.)\pend{}{\bigskip}
\pstart
           \noindent{}\centering{}{\pb}\textcolor{gray}{\textbf{⋅1903⋅ Theater-Bazar}}\pend
           \vspace{1em}
\pstart
           \noindent{}{\pb}Da ſehn Sie, verehrteſter Herr, wie man
               in Bern\oindex{Bern@\textbf{Bern}, \emph{P.PPLC}|pw}{ }Sie liebt u. ke{\geminationn}t und
               ſchätzt. Und natürlich aufführt, ſobald das neue Theater\orgindex{Stadttheater Bern@Stadttheater Bern|pwv} in dieſem Herbſt ſeine Hallen
               öffnet.\pend
           
\pstart
           Mit höflichem Gruß{\\[\baselineskip]}\spacefill\mbox{J. V. Widmann}\pend
           \leftskip=0em{}
\pstart
           \textcolor{gray}{\textbf{BERN\oindex{Bern@\textbf{Bern}, \emph{P.PPLC}|pw}, den}}{ }9. Mai 1903.\pend
           \selectlanguage{ngerman}\endnumbering\briefempfaengerindex{Schnitzler, Arthur@\textsc{Schnitzler, Arthur}!zzzWidmann, Joseph Victor@\emph{von Joseph Victor Widmann}!1903-05-091@{9. 5. 1903}|)be}\mylabel{L01288h}  \normalsize

\doendnotes{C}
\bigskip
\vfill

\clearpage

\footnotesize

\lohead{\textsc{register}}

% Definiere theindex-Environment komplett neu ohne reledmac
\makeatletter
\renewenvironment{theindex}{%
  \section*{\indexname}%
  \setlength{\parindent}{0pt}%
  \setlength{\parskip}{0pt plus 0.3pt}%
  \let\item\@idxitem
}{%
  \clearpage
}
\makeatother

\IfFileExists{\jobname-pw.ind}{\input{\jobname-pw.ind}}{}

\end{document}

      