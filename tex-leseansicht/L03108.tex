%% latex-leseansicht-vorspann.tex
%% Vorspann für die Leseansicht.
%% Lädt die gemeinsame Datei latex-vorspann.tex mit nicht gesetztem Schalter.

\newif\ifkorrekturansicht
\korrekturansichtfalse

\input{../tex-inputs/latex-vorspann}

\begin{center}
            \textcolor{red}{ENTWURF, NICHT FERTIG KORRIGIERT}
                      \end{center}
            
         \renewcommand{\erwaehnteOrte}{Orte: Café Kremser, Wien}
         \renewcommand{\erwaehnteWerke}{}
               \section[Felix Salten an Arthur Schnitzler, {[}31. 3. 1892{]}]{ Felix Salten an Arthur Schnitzler, {[}31. 3. 1892{]}}\nopagebreak\mylabel{v}\rehead{ }\begin{ledgroupsized}[t]{13cm}\normalsize\beginnumbering \toendnotes[C]{\smallbreak\pagebreak[2]} \Standort{CUL, Schnitzler, B 89, A 1.}
\physDesc{Brief, 1 Blatt, 4 Seiten
\newline{}Handschrift: schwarze Tinte, lateinische Kurrent
\newline{}Schnitzler: mit Bleistift datiert: »31/3 92« }\pstart
           \noindent{}{\pb}Lieber Arthur! Soeben bin ich für immer von der »schönsten Pflicht
               des Bürgers« freigesprochen worden, und mir ist, als hätte ich eben mich selbst zum
               Geschenk erhalten. Ich bin in einer so guten, leichten Stimmung, dass ich meine, man
               hätte mir in der Welt kein schöneres Präsent machen können. Der Aufenthalt {\pb}im Aussenlokale mitten unter diesen Anderen ist etwas Entsetzliches. Man
               ist wie diese hier, und wird als dasselbe angesehen und behandelt wie der
               vertrottelte Schuster, besoffene Maurergeselle, arrogante Commis ec. ec. 1529, – der
               Schuster – 1530 – der Maurergehilfe, – 1531 – ich, 1532 – der Commis u. s. w. aber
               man kann niemandem einen Vorwurf daraus machen, der Staat richtet {\pb}sich
               hierin nach der Natur, die ja für uns nicht die Ehre hat, – Sie wissen schon, und die
                  \uline{uns} weder ein längeres Leben noch andere Nerven
               gibt. Der Maurergehilfe lebt sicher länger als ich, und der Commis wird mich
               vermutlich mit meiner Geliebten betrügen, weil er eine vielversprechendere Nase hat
               als ich. Auf der Herreise habe ich eine kleine Novelle erlebt, reizend sage {\pb}ich Ihnen. Ganz ohne Handlung,
               denn das Rendezvous auf der Kettenbrücke werde ich heute N. M. kaum einhalten. Es ist
               nicht mehr nothwendig. Ich kenn’ sie schon, also – abtreten.\pend
           \pstart
           Leben Sie wol. Vielleicht erst Samstag{ }Abend{ }\uline{Café Kremser\oindex{Cafe Kremser@\textbf{Café Kremser}|pw}}\pend
           \pstart
           Herzlich Ihr {\\[\baselineskip]}\spacefill\mbox{Felix Salten}\pend
           \leftskip=0em{}
         
         \endnumbering\mylabel{h}\end{ledgroupsized}\begin{anhang}\end{anhang}\newcommand{\dateiname}{L03108}\newcommand{\titel}{Felix Salten an Arthur Schnitzler, [31. 3. 1892]}\newcommand{\editorInnen}{Martin Anton Müller und Laura Untner}%% latex-leseansicht-abspann.tex
%% Abspann für die Leseansicht.
%% Der Schalter \ifkorrekturansicht ist bereits durch den Vorspann gesetzt.

%% latex-abspann.tex
%% Gemeinsamer Abspann für Korrekturansicht und Leseansicht.
%% Setzt den Schalter \ifkorrekturansicht voraus (gesetzt in den
%% einbindenden Dateien latex-korrekturansicht-abspann.tex bzw.
%% latex-leseansicht-abspann.tex).
%% ---------------------------------------------------------------

\normalsize

% Das esempio-Environment wird nur in der Leseansicht benötigt
\ifkorrekturansicht\else
\newenvironment{esempio}[3]%
{
    \vspace{1.5ex}
    \rlap{\underline{#1}}
    \par
    \setlength{\parindent}{0cm}
    \nopagebreak
    \leftskip=#2cm
    \rightskip=#3cm
}
{
    \par
}
\fi

\doendnotes{C}
\bigskip
\vfill

\clearpage

\footnotesize

\ifkorrekturansicht
  \lohead{\textsc{register}}
\fi

% theindex-Environment neu definieren ohne reledmac
\makeatletter
\renewenvironment{theindex}{%
  \ifkorrekturansicht
    \section*{\indexname}%
  \else
    \subsubsection*{Index der erwähnten Entitäten}%
  \fi
  \setlength{\parindent}{0pt}%
  \setlength{\parskip}{0pt plus 0.3pt}%
  \let\item\@idxitem
}{%
  \ifkorrekturansicht\clearpage\fi
}
\makeatother

\IfFileExists{\jobname-pw.ind}{\input{\jobname-pw.ind}}{}

% Quellenangabe nur in der Leseansicht
\ifkorrekturansicht\else
% Fallback-Definitionen, falls die .tex-Datei \titel etc. nicht gesetzt hat
\providecommand{\titel}{}
\providecommand{\editorInnen}{}
\providecommand{\dateiname}{\jobname}

\vspace{3cm}

\vfill

\footnotesize
\textsc{Quelle}: \titel. Herausgegeben von {\editorInnen}. In: \emph{Arthur Schnitzler: Briefwechsel mit Autorinnen und Autoren}.
 Digitale Edition, https://schnitzler-briefe.acdh.oeaw.ac.at/{\dateiname}.html (Stand \today)
\fi

\end{document}


      