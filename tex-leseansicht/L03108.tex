%% latex-leseansicht-vorspann.tex
%% Vorspann für die Leseansicht.
%% Lädt die gemeinsame Datei latex-vorspann.tex mit nicht gesetztem Schalter.

\newif\ifkorrekturansicht
\korrekturansichtfalse

\input{../tex-inputs/latex-vorspann}


         
         \renewcommand{\erwaehntePersonen}{Personen:  ?? [Partnerin von Felix Salten, Ende März 1892], Felix Salten}
         \renewcommand{\erwaehnteOrte}{Orte: Café Kremser, Kettenbrücke, Wien}
         \renewcommand{\erwaehnteWerke}{Werke: Tagebuch}
               \section[Felix Salten an Arthur Schnitzler, {[}31. 3. 1892{]}]{ Felix Salten an Arthur Schnitzler, {[}31. 3. 1892{]}}\nopagebreak\mylabel{v}\rehead{ }\begin{ledgroupsized}[t]{13cm}\normalsize\beginnumbering\briefempfaengerindex{Schnitzler, Arthur@\textsc{Schnitzler, Arthur}!zzzSalten, Felix@\emph{von Felix Salten}!1892-03-311@{{[}31. 3. 1892{]}}|(be} \toendnotes[C]{\smallbreak\pagebreak[2]} \Standort{CUL, Schnitzler, B 89, A 1.}
\physDesc{Brief, 1 Blatt, 4 Seiten, 1313 Zeichen
\newline{}Handschrift: schwarze Tinte, lateinische Kurrent
\newline{}Schnitzler: mit Bleistift datiert: »31/3 92« 
\newline{}Ordnung: mit Bleistift von unbekannter Hand nummeriert: »9« }\toendnotes[C]{\smallbreak}\pstart
           \noindent{}{\pb}Lieber Arthur! Soeben bin ich für immer von der
                  \label{K_L03108-1v}\edtext{»schönsten Pflicht des
                  Bürgers«}{\lemma{\textnormal{\emph{»schönsten … Bürgers«}}}\Cendnote{\textnormal{Wehrdienst}}}\label{K_L03108-1h}
               freigesprochen worden, und mir ist, als hätte ich eben mich selbst zum Geschenk
               erhalten. Ich bin in einer so guten, leichten Stimmung, dass ich meine, man hätte mir
               in der Welt kein schöneres Präsent machen können. Der Aufenthalt {\pb}im Assentlokale mitten
               unter diesen Anderen ist etwas \substVorne{}\textsuperscript{e}\substDazwischen{}E\substHinten{}ntsetzliches. Man ist wie diese hier, und wird als
               dasselbe angesehen und behandelt wie der vertrottelte Schuster, besoffene
               Maurergeselle, arrogante Commis ec. ec. 1529, – der Schuster – 1530 – der
               Maurergehilfe, – 1531 – ich, 1532 – der Commis u. s. w. aber man kann niemandem einen
               Vorwurf daraus machen, der Staat richtet {\pb}sich hierin nach der
               Natur, die ja für uns nicht die Ehre hat, – Sie wissen schon, und die \uline{uns} weder ein längeres Leben noch andere Nerven
                  gibt\textcolor{gray}{.}– Der Maurergehilfe leb\substVorne{}\textsuperscript{\textcolor{gray}{st}}\substDazwischen{}t\substHinten{} sicher länger als ich, und der Commis wird mich vermutlich mit meiner
               Geliebten betrügen, weil er eine vielversprechendere Nase hat als ich.\pend
           \pstart
           Auf der Herreise habe ich eine kleine Novelle erlebt, reizend sage {\pb}ich Ihnen. Ganz ohne Handlung,
               denn das Rendezvous auf der Kettenbrücke\oindex{Kettenbruecke@\textbf{Kettenbrücke}|pw} werde
               ich heute N. M. kaum einhalten. Es ist nicht mehr nothwendig. Ich kenn’
                  \label{K_L03108-2v}\edtext{sie\pwindex{?? [Partnerin von Felix Salten, Ende Maerz 1892] @\textsc{?? [Partnerin von Felix Salten, Ende März 1892]}|pwv}}{\lemma{\textnormal{\emph{sie}}}\Cendnote{\textnormal{nicht ermittelt}}}\label{K_L03108-2h} schon, also –
               abtreten.\pend
           \pstart
           Leben Sie wol. Vielleicht erst \label{K_L03108-3v}\edtext{Samstag{ }Abend{ }\uline{Café Kremser\oindex{Cafe Kremser@\textbf{Café Kremser}|pw}}}{\lemma{\textnormal{\emph{Samstag … Kremser}}}\Cendnote{\textnormal{Ein Aufenthalt Schnitzlers\pwindex{Schnitzler, Arthur 15.05.1862 – 21.10.1931@\textsc{Schnitzler, Arthur} (15.05.1862 – 21.10.1931), \emph{Schriftsteller, Mediziner}|pwk} an diesem Tag im Café Kremser\oindex{Cafe Kremser@\textbf{Café Kremser}|pwk} ist nicht im \emph{Tagebuch}\pwindex{\textcolor{red}{\textsuperscript{XXXX1 indx}}!Tagebuch1981 – 2000@\strich\emph{Tagebuch} {[}Hrsg., 1981 – 2000{]}|pwk} erwähnt.}}}\label{K_L03108-3h}\pend
           \pstart
           Herzlich Ihr {\\[\baselineskip]}\spacefill\mbox{Felix Salten.}\pend
           \leftskip=0em{}
         
         \endnumbering\mylabel{h}\end{ledgroupsized}  \newcommand{\dateiname}{L03108}\newcommand{\titel}{Felix Salten an Arthur Schnitzler, [31. 3. 1892]}\newcommand{\editorInnen}{Martin Anton Müller und Laura Untner}%% latex-leseansicht-abspann.tex
%% Abspann für die Leseansicht.
%% Der Schalter \ifkorrekturansicht ist bereits durch den Vorspann gesetzt.

%% latex-abspann.tex
%% Gemeinsamer Abspann für Korrekturansicht und Leseansicht.
%% Setzt den Schalter \ifkorrekturansicht voraus (gesetzt in den
%% einbindenden Dateien latex-korrekturansicht-abspann.tex bzw.
%% latex-leseansicht-abspann.tex).
%% ---------------------------------------------------------------

\normalsize

% Das esempio-Environment wird nur in der Leseansicht benötigt
\ifkorrekturansicht\else
\newenvironment{esempio}[3]%
{
    \vspace{1.5ex}
    \rlap{\underline{#1}}
    \par
    \setlength{\parindent}{0cm}
    \nopagebreak
    \leftskip=#2cm
    \rightskip=#3cm
}
{
    \par
}
\fi

\doendnotes{C}
\bigskip
\vfill

\clearpage

\footnotesize

\ifkorrekturansicht
  \lohead{\textsc{register}}
\fi

% theindex-Environment neu definieren ohne reledmac
\makeatletter
\renewenvironment{theindex}{%
  \ifkorrekturansicht
    \section*{\indexname}%
  \else
    \subsubsection*{Index der erwähnten Entitäten}%
  \fi
  \setlength{\parindent}{0pt}%
  \setlength{\parskip}{0pt plus 0.3pt}%
  \let\item\@idxitem
}{%
  \ifkorrekturansicht\clearpage\fi
}
\makeatother

\IfFileExists{\jobname-pw.ind}{\input{\jobname-pw.ind}}{}

% Quellenangabe nur in der Leseansicht
\ifkorrekturansicht\else
% Fallback-Definitionen, falls die .tex-Datei \titel etc. nicht gesetzt hat
\providecommand{\titel}{}
\providecommand{\editorInnen}{}
\providecommand{\dateiname}{\jobname}

\vspace{3cm}

\vfill

\footnotesize
\textsc{Quelle}: \titel. Herausgegeben von {\editorInnen}. In: \emph{Arthur Schnitzler: Briefwechsel mit Autorinnen und Autoren}.
 Digitale Edition, https://schnitzler-briefe.acdh.oeaw.ac.at/{\dateiname}.html (Stand \today)
\fi

\end{document}


      