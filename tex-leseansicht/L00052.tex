%% latex-leseansicht-vorspann.tex
%% Vorspann für die Leseansicht.
%% Lädt die gemeinsame Datei latex-vorspann.tex mit nicht gesetztem Schalter.

\newif\ifkorrekturansicht
\korrekturansichtfalse

\input{../tex-inputs/latex-vorspann}


         
         \renewcommand{\erwaehntePersonen}{Personen: Oskar Blumenthal, Emanuel Reicher}
         \renewcommand{\erwaehnteInstitutionen}{Institutionen: Emanuel Reicher’s Deutsche Gastspielgesellschaft, Internationale Ausstellung für Musik und Theaterwesen, Lessing-Theater}
         \renewcommand{\erwaehnteOrte}{Orte: Berlin, Kapelle-Ufer, Wien}
         \renewcommand{\erwaehnteWerke}{Werke: Das Märchen. Schauspiel in drei Aufzügen}
               \section[Oscar Blumenthal an Arthur Schnitzler, 15. 12. 1891]{ Oscar Blumenthal an Arthur Schnitzler, 15. 12. 1891}\nopagebreak\mylabel{v}\rehead{ }\begin{ledgroupsized}[t]{13cm}\normalsize\beginnumbering\briefempfaengerindex{Schnitzler, Arthur@\textsc{Schnitzler, Arthur}!zzzBlumenthal, Oskar@\emph{von Oskar Blumenthal}!1891-12-151@{15. 12. 1891}|(be} \toendnotes[C]{\smallbreak\pagebreak[2]} \Standort{CUL, Schnitzler, B 15.}
\physDesc{Brief, 1 Blatt, 2 Seiten, 1490 Zeichen
\newline{}Handschrift Schreibkraft: schwarze Tinte, deutsche Kurrent\newline{}Handschrift Oskar Blumenthal: schwarze Tinte, deutsche Kurrent
\newline{}Schnitzler: 1) mit Bleistift beschriftet: »\textsc{Blumenthal}«  2) mit rotem Buntstift eine Unterstreichung und nummeriert:
                                    »1«
\newline{}Ordnung: mit Bleistift von unbekannter Hand nummeriert:
                                 »1« }\toendnotes[C]{\smallbreak}\pstart
           \noindent{}\centering{}{\pb}\textcolor{gray}{\textbf{LESSING-THEATER\orgindex{Lessing-Theater@Lessing-Theater|pw}}}\pend
           \pstart
           \noindent{}\centering{}\textcolor{gray}{\textbf{Director:}}{\\}\textcolor{gray}{\textbf{Dr. Oscar Blumenthal.}}\pend
           \pstart
           \noindent{}\raggedleft{}\textcolor{gray}{\textbf{Berlin N.W.\oindex{Berlin@\textbf{Berlin}|pw}, den}}{ }15. Dezember \textcolor{gray}{\textbf{189}}1.{\\}\textcolor{gray}{\textbf{Friedrich-Carl-Ufer\oindex{Kapelle-Ufer@\textbf{Kapelle-Ufer}|pw}}}.\pend
           \pstart\center{}Sehr geehrter Herr!\pend\pstart
           Ihr Schauſpiel »Das Märchen\pwindex{Schnitzler, Arthur 15.05.1862 – 21.10.1931@\textsc{Schnitzler, Arthur} (15.05.1862 – 21.10.1931), \emph{Schriftsteller, Mediziner}!Maerchen. Schauspiel in drei Aufzuegen1893-12-01@\strich\emph{Das Märchen. Schauspiel in drei Aufzügen} {[}1893-12-01{]}|pw}« habe ich mit allem
               Intereſſe geleſen und bin auch bereit, es in der nächſten Saiſon mit Herrn \textsc{Emanuel Reicher\pwindex{Reicher, Emanuel 18.06.1849 – 15.05.1924@\textsc{Reicher, Emanuel} (18.06.1849 – 15.05.1924), \emph{Schauspieler}|pw}} in der ſchwierigen und leider auch recht unerfreulichen Rolle des Fedor Denner\pwindex{Schnitzler, Arthur 15.05.1862 – 21.10.1931@\textsc{Schnitzler, Arthur} (15.05.1862 – 21.10.1931), \emph{Schriftsteller, Mediziner}!Maerchen. Schauspiel in drei Aufzuegen1893-12-01@\strich\emph{Das Märchen. Schauspiel in drei Aufzügen} {[}1893-12-01{]}|pwv} zur Darſtellung zu
               bringen. Sehr wünſche ich aber, daß Sie die Zwiſchenzeit benutzten, um das Stück noch
               recht liebevoll auszubauen und es von einer Ueberfracht von Reflexionen und
               Nebenſcenen zu befreien, um dafür die Hauptmomente deſto wirkſamer und plaſtiſcher
               hervortreten zu laſſen. Nicht überall iſt es Ihnen gelungen, aus der didaktiſchen
               Betrachtung heraus Ihren Stoff in Leben und Anſchauung umzuſetzen. Beſonders im
               zweiten Akte macht ſich die Mattheit des dramatiſchen Pulsſchlages geltend, obwohl
               gerade hier in der Begegnung zwiſchen Fedor\pwindex{Schnitzler, Arthur 15.05.1862 – 21.10.1931@\textsc{Schnitzler, Arthur} (15.05.1862 – 21.10.1931), \emph{Schriftsteller, Mediziner}!Maerchen. Schauspiel in drei Aufzuegen1893-12-01@\strich\emph{Das Märchen. Schauspiel in drei Aufzügen} {[}1893-12-01{]}|pwv} und Friedrich Witte\pwindex{Schnitzler, Arthur 15.05.1862 – 21.10.1931@\textsc{Schnitzler, Arthur} (15.05.1862 – 21.10.1931), \emph{Schriftsteller, Mediziner}!Maerchen. Schauspiel in drei Aufzuegen1893-12-01@\strich\emph{Das Märchen. Schauspiel in drei Aufzügen} {[}1893-12-01{]}|pwv} Gelegenheit genug geboten iſt, den Stoff zu einer großen
               dramatiſchen Erregung emporzuführen.\pend
           \pstart
           {\pb}Ich habe Herrn Reicher\pwindex{Reicher, Emanuel 18.06.1849 – 15.05.1924@\textsc{Reicher, Emanuel} (18.06.1849 – 15.05.1924), \emph{Schauspieler}|pw} den Rath gegeben, unter der Vorausſetzung Ihres
               Einverſtändniſſes, das Stück zunächſt einmal mit ſeiner Gaſtſpielgeſellſchaft\orgindex{Emanuel Reicher s Deutsche Gastspielgesellschaft@Emanuel Reicher’s Deutsche Gastspielgesellschaft|pw} auf der \label{K_L00052-1v}\edtext{Muſik- und Theaterausſtellung\orgindex{Internationale Ausstellung fuer Musik und Theaterwesen@Internationale Ausstellung für Musik und Theaterwesen|pw}}{\lemma{\textnormal{\emph{Muſik- und Theaterausſtellung}}}\Cendnote{\textnormal{Die \emph{Internationale Ausstellung für Musik- und Theaterwesen}\orgindex{Internationale Ausstellung fuer Musik und Theaterwesen@Internationale Ausstellung für Musik und Theaterwesen|pwk} dauerte vom
                     7. 5. bis zum 9. 10. 1892.}}}\label{K_L00052-1h} zur Darſtellung zu
               bringen. Sie werden dann in perſönlichen Berathungen mit dem ſehr intelligenten und
               urtheilsklaren Darſteller\pwindex{Reicher, Emanuel 18.06.1849 – 15.05.1924@\textsc{Reicher, Emanuel} (18.06.1849 – 15.05.1924), \emph{Schauspieler}|pwv}
               vielleicht die beſte Gelegenheit finden, die Schwächen des Werkes\pwindex{Schnitzler, Arthur 15.05.1862 – 21.10.1931@\textsc{Schnitzler, Arthur} (15.05.1862 – 21.10.1931), \emph{Schriftsteller, Mediziner}!Maerchen. Schauspiel in drei Aufzuegen1893-12-01@\strich\emph{Das Märchen. Schauspiel in drei Aufzügen} {[}1893-12-01{]}|pwv} zu beſeitigen, welchem ich wegen der
               darin zu Tage tretenden Kunſt der Charakteriſtik und Tiefe der Lebensbeobachtung
               einen vollen Bühnenerfolg gern bereitet ſehen möchte.\pend
           \pstart
           Hochachtungsvoll{\\[\baselineskip]}\spacefill\mbox{{[}hs. Blumenthal:{]} Dr. Osc. Blumenthal}\pend
           \leftskip=0em{}
         
         \endnumbering\mylabel{h}\end{ledgroupsized}  \newcommand{\dateiname}{L00052}\newcommand{\titel}{Oscar Blumenthal an Arthur Schnitzler, 15. 12. 1891}\newcommand{\editorInnen}{Martin Anton Müller und Gerd-Hermann Susen}%% latex-leseansicht-abspann.tex
%% Abspann für die Leseansicht.
%% Der Schalter \ifkorrekturansicht ist bereits durch den Vorspann gesetzt.

%% latex-abspann.tex
%% Gemeinsamer Abspann für Korrekturansicht und Leseansicht.
%% Setzt den Schalter \ifkorrekturansicht voraus (gesetzt in den
%% einbindenden Dateien latex-korrekturansicht-abspann.tex bzw.
%% latex-leseansicht-abspann.tex).
%% ---------------------------------------------------------------

\normalsize

% Das esempio-Environment wird nur in der Leseansicht benötigt
\ifkorrekturansicht\else
\newenvironment{esempio}[3]%
{
    \vspace{1.5ex}
    \rlap{\underline{#1}}
    \par
    \setlength{\parindent}{0cm}
    \nopagebreak
    \leftskip=#2cm
    \rightskip=#3cm
}
{
    \par
}
\fi

\doendnotes{C}
\bigskip
\vfill

\clearpage

\footnotesize

\ifkorrekturansicht
  \lohead{\textsc{register}}
\fi

% theindex-Environment neu definieren ohne reledmac
\makeatletter
\renewenvironment{theindex}{%
  \ifkorrekturansicht
    \section*{\indexname}%
  \else
    \subsubsection*{Index der erwähnten Entitäten}%
  \fi
  \setlength{\parindent}{0pt}%
  \setlength{\parskip}{0pt plus 0.3pt}%
  \let\item\@idxitem
}{%
  \ifkorrekturansicht\clearpage\fi
}
\makeatother

\IfFileExists{\jobname-pw.ind}{\input{\jobname-pw.ind}}{}

% Quellenangabe nur in der Leseansicht
\ifkorrekturansicht\else
% Fallback-Definitionen, falls die .tex-Datei \titel etc. nicht gesetzt hat
\providecommand{\titel}{}
\providecommand{\editorInnen}{}
\providecommand{\dateiname}{\jobname}

\vspace{3cm}

\vfill

\footnotesize
\textsc{Quelle}: \titel. Herausgegeben von {\editorInnen}. In: \emph{Arthur Schnitzler: Briefwechsel mit Autorinnen und Autoren}.
 Digitale Edition, https://schnitzler-briefe.acdh.oeaw.ac.at/{\dateiname}.html (Stand \today)
\fi

\end{document}


      