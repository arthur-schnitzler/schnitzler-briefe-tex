%% latex-leseansicht-vorspann.tex
%% Vorspann für die Leseansicht.
%% Lädt die gemeinsame Datei latex-vorspann.tex mit nicht gesetztem Schalter.

\newif\ifkorrekturansicht
\korrekturansichtfalse

\input{../tex-inputs/latex-vorspann}


\section[Theodor Herzl an Arthur Schnitzler, {[}zwischen dem 20.  und 23.?{]} 1. 1895]{L03846 Theodor Herzl an Arthur Schnitzler, [zwischen dem 20.  und 23.?] 1. 1895}
\nopagebreak\mylabel{L03846v}
\rehead{ }\normalsize\beginnumbering\briefempfaengerindex{Schnitzler, Arthur@\textsc{Schnitzler, Arthur}!zzzHerzl, Theodor@\emph{von Theodor Herzl}!1895-01-233@{[zwischen 20.  und 23.?] 1. 1895}|(be}
\toendnotes[C]{\smallbreak\pagebreak[2]}
\correspDesc{Versand  durch Theodor Herzl im Zeitraum [zwischen 20.  und
                  23.?] 1. 1895 in Paris
\newline{}Erhalt  durch Arthur Schnitzler in Wien}\toendnotes[C]{\smallbreak}
\Standort{CUL, Schnitzler, B 39.}
\physDesc{Brief, 1 Blatt, 2 Seiten, 820 Zeichen, Fragment (unteres Drittel von S. 1 Textverlust durch Abschneiden)
\newline{}Handschrift: schwarze Tinte, lateinische Kurrent
\newline{}Schnitzler: mit Bleistift datiert: »Ende Januar
                                 95« 
\newline{}Ordnung: mit Bleistift von unbekannter Hand nummeriert: »25« }
\buchAbdrucke{\weitereDrucke{Theodor Herzl: \emph{Briefe und
                        autobiographische Notizen 1866–1895}. Bearbeitet von Johannes Wachten in Zusammenarbeit mit Chaya Harel, Daisy Tycho und Manfred Winkler. Berlin, Frankfurt am Main, Wien: \emph{Propyläen} 1983, S. 570–571 (Briefe und Tagebücher. Herausgegeben von Alex Bein, Hermann Greive, Moshe Schaerf, Julius H. Schoeps und Johannes Wachten, 1).} }\toendnotes[C]{\smallbreak}
\pstart
           {\pb}\textcolor{gray}{\textbf{NOUVELLE PRESSE LIBRE}}\orgindex{Neue Freie Presse@Neue Freie Presse|pw}\hfill \textcolor{gray}{\textbf{8, RUE DE MONCEAU }}\oindex{8, rue de Monceau@\textbf{8, rue de Monceau}, \emph{Wohngebäude}|pw}\pend
           
\pstart
           \textcolor{gray}{\textbf{D\textsuperscript{r}{ }TH. HERZL}}\pend
           
\pstart{}Mein lieber Freund!\pend\vspace{0.5em}
\pstart
           Sie haben richtig vermuthet, dass mir die Annahme Ihres Stücks\pwindex{Schnitzler, Arthur 15.\,5.\,1862 Wien – 21.\,10.\,1931 ebd.@\textsc{Schnitzler, Arthur} (15.\,5.\,1862 Wien – 21.\,10.\,1931 ebd.), \emph{Schriftsteller, Mediziner}!Liebelei. Schauspiel in drei Akten@\strich\emph{Liebelei. Schauspiel in drei Akten}|pwv} im Burgtheater\orgindex{Burgtheater@Burgtheater|pw} viel Freude machen wird. Bravo! schon jetzt. Ich möchte gern
               Näheres darüber hören. Schreiben Sie ausführlich. Alles \label{K_L03846-1v}\edtext{Albert Betreffende}{\lemma{\textnormal{\emph{Albert Betreffende}}}\Cendnote{\textnormal{Um das Inkongito seiner Verfasserschaft am Schauspiels \emph{Das neue Ghetto}\pwindex{Herzl, Theodor 2.\,5.\,1860 Budapest – 3.\,7.\,1904 Edlach@\textsc{Herzl, Theodor} (2.\,5.\,1860 Budapest – 3.\,7.\,1904 Edlach), \emph{Schriftsteller, Journalist}!neue Ghetto. Schauspiel in vier Acten@\strich\emph{Das neue Ghetto. Schauspiel in vier Acten}|pwk}, das Herzl\pwindex{Herzl, Theodor 2.\,5.\,1860 Budapest – 3.\,7.\,1904 Edlach@\textsc{Herzl, Theodor} (2.\,5.\,1860 Budapest – 3.\,7.\,1904 Edlach), \emph{Schriftsteller, Journalist}|pwk} unter dem Pseudonym Albert Schnabel von Schnitzler bei verschiedenen Berliner\oindex{Berlin@\textbf{Berlin}, \emph{Hauptstadt}|pwk} Theatern einreichen ließ, abzusichern, hatte er Schnitzler angewiesen, Nachrichten zu dieser
                  Angelegenheit postlagernd an ein Postbüro zu senden und sie in der normalen
                  Korrespondenz nur verklausuliert zu berühren, siehe XXXX Auszeichnungsfehler: Dokument L03836 nicht gefunden.}}}\label{K_L03846-1} natürlich an Albert.\pend
           
\pstart
           Ihren letzten so lieben \label{K_L03846-2v}\edtext{Brief}{\lemma{\textnormal{\emph{Brief}}}\Cendnote{\textnormal{XXXX14.1.1895}}}\label{K_L03846-2} konnte ich wegen des
                  \label{K_L03846-3v}\edtext{Krisenrummels}{\lemma{\textnormal{\emph{Krisenrummels}}}\Cendnote{\textnormal{Mitte Januar 1895 traten zunächst das franzöische\oindex{Frankreich@\textbf{Frankreich}|pwk} Kabinett unter der Leitung von Premierminister Charles Dupuy\pwindex{Dupuy, Charles 5.\,11.\,1851 Le Puy-en-Velay – 23.\,7.\,1923 Ille-sur-Têt@\textsc{Dupuy, Charles} (5.\,11.\,1851 Le Puy-en-Velay – 23.\,7.\,1923 Ille-sur-Têt)|pwk} und dann der Staatspräsident
                     Jean Casimir-Perier\pwindex{Casimir-Perier, Jean 8.\,11.\,1847 Paris – 11.\,3.\,1907 ebd.@\textsc{Casimir-Perier, Jean} (8.\,11.\,1847 Paris – 11.\,3.\,1907 ebd.), \emph{Politiker, Präsident}|pwk} zurück. Herzl\pwindex{Herzl, Theodor 2.\,5.\,1860 Budapest – 3.\,7.\,1904 Edlach@\textsc{Herzl, Theodor} (2.\,5.\,1860 Budapest – 3.\,7.\,1904 Edlach), \emph{Schriftsteller, Journalist}|pwk} berichtete ab dem
                     15. 1. 1895 täglich für die \emph{Neue
                     Freie Presse}\pwindex{Neue Freie Presse@\emph{Neue Freie Presse}|pwk} ausführlich über die Entwicklungen, beginnend mit dem Artikel \emph{Der Sturz des französischen Ministeriums}\pwindex{Herzl, Theodor 2.\,5.\,1860 Budapest – 3.\,7.\,1904 Edlach@\textsc{Herzl, Theodor} (2.\,5.\,1860 Budapest – 3.\,7.\,1904 Edlach), \emph{Schriftsteller, Journalist}!Sturz des französischen Ministeriums@\strich\emph{Der Sturz des französischen Ministeriums}|pwk}.
                     In: \emph{Neue Freie Presse}\pwindex{Neue Freie Presse@\emph{Neue Freie Presse}|pwk}, Nr. 10.918,
                        15. 1. 1895, Morgenblatt, S. 1–2.}}}\label{K_L03846-3} noch nicht
               beantworten. Auch \label{K_L03846-4v}\edtext{heute}{\lemma{\textnormal{\emph{heute}}}\Cendnote{\textnormal{Der nicht datierte Brief kann frühestens am
                     20. 1. 1895 abgefasst worden sein, da er auf die Nachricht aus Schnitzlers Brief vom refXXXX19.1.1895
                  reagiert, dass die \emph{Liebelei}\pwindex{Schnitzler, Arthur 15.\,5.\,1862 Wien – 21.\,10.\,1931 ebd.@\textsc{Schnitzler, Arthur} (15.\,5.\,1862 Wien – 21.\,10.\,1931 ebd.), \emph{Schriftsteller, Mediziner}!Liebelei. Schauspiel in drei Akten@\strich\emph{Liebelei. Schauspiel in drei Akten}|pwk} am \emph{Burgtheater}\orgindex{Burgtheater@Burgtheater|pwk} angenommen sei. Da er den Auftrag zur
                  Versendung und vorherigen Abschrift eines Briefes an die Direktion des \emph{Deutschen Theaters}\orgindex{Deutsches Theater Berlin@Deutsches Theater Berlin|pwk} bis zum
                     25. 1. 1895 enthält, ist eine Abfassung nicht nach dem
                     23. 1. 1895 anzunehmen.}}}\label{K_L03846-4} nur ein Wort. Ich bitte Sie, zwei
               Tage vor Ablauf der Frist – also am 25\textsuperscript{ten}? – von der Hand\pwindex{?? [Schreibkraft für Arthur Schnitzler] @\textsc{?? [Schreibkraft für Arthur Schnitzler]}|pwuv} die den \label{K_L03846-5v}\edtext{Begleitbrief}{\lemma{\textnormal{\emph{Begleitbrief}}}\Cendnote{\textnormal{Siehe XXXX Auszeichnungsfehler: Dokument L03843 nicht gefunden.}}}\label{K_L03846-5} schrieb,
               folgenden \label{T_L03846-1v}\edtext{recommandirten Brief}{\lemma{\textnormal{\emph{recommandirten Brief}}}\Cendnote{\textnormal{Das untere Drittel der Seite, das den
                  abzuschreibenden Brief enthielt, wurde herausgeschnitten – vermutlich durch Schnitzler, um es der Schreibkraft\pwindex{?? [Schreibkraft für Arthur Schnitzler] @\textsc{?? [Schreibkraft für Arthur Schnitzler]}|pwuvk} weiterzugeben.}}}\label{T_L03846-1}{ }\damage{\textcolor{gray}{×}\-\textcolor{gray}{×}\-\textcolor{gray}{×}\-\textcolor{gray}{×}\-\textcolor{gray}{×}\-\textcolor{gray}{×}\-\textcolor{gray}{×}\-\textcolor{gray}{×}\-\textcolor{gray}{×}\-\textcolor{gray}{×}\-\textcolor{gray}{×}\-\textcolor{gray}{×}}{ }\damage{abschreiben z}u lassen {[}:{]}\pend
           
\pstart
           \damage{\textcolor{gray}{[9 Zeilen Textverlust{]} }}\pend
           
\pstart
           {\pb}Ich hab’s nicht viel anders erwartet
               und bin nur ganz unwesentlich enttäuscht. Jedenfalls lasse ich mich vom Unverstand
               (oder Verstand!?) der Directoren\pwindex{Blumenthal, Oskar 13.\,3.\,1852 Berlin – 24.\,4.\,1917 ebd.@\textsc{Blumenthal, Oskar} (13.\,3.\,1852 Berlin – 24.\,4.\,1917 ebd.), \emph{Schriftsteller, Journalist, Theaterleiter}|pwv}\pwindex{Brahm, Otto 5.\,2.\,1856 Hamburg – 28.\,11.\,1912 Berlin@\textsc{Brahm, Otto} (5.\,2.\,1856 Hamburg – 28.\,11.\,1912 Berlin), \emph{Theaterleiter, Regisseur}|pwv}\pwindex{Schlenther, Paul 20.\,8.\,1854 Chernyakhovsk – 30.\,4.\,1916 Berlin@\textsc{Schlenther, Paul} (20.\,8.\,1854 Chernyakhovsk – 30.\,4.\,1916 Berlin), \emph{Schriftsteller, Kritiker, Theaterleiter}|pwv}{ }\strikeout{machen} nicht irr und zaghaft machen. Ich gehe meinen
               Weg weiter. Meinen neuen Weg! Darin liegt auch etwas Seliges.\pend
           
\pstart
           Herzliche Grüsse von Ihrem guten Freund {\\[\baselineskip]}\spacefill\mbox{Herzl}\pend
           \leftskip=0em{}\selectlanguage{ngerman}\endnumbering\briefempfaengerindex{Schnitzler, Arthur@\textsc{Schnitzler, Arthur}!zzzHerzl, Theodor@\emph{von Theodor Herzl}!1895-01-203@{[zwischen 20.  und 23.?] 1. 1895}|)be}\mylabel{L03846h}
\begin{anhang}
\end{anhang}\newcommand{\dateiname}{L03846}\newcommand{\titel}{Theodor Herzl an Arthur Schnitzler, [zwischen dem 20.  und 23.?] 1. 1895}\newcommand{\editorInnen}{Selma Jahnke und Martin Anton Müller}%% latex-leseansicht-abspann.tex
%% Abspann für die Leseansicht.
%% Der Schalter \ifkorrekturansicht ist bereits durch den Vorspann gesetzt.

%% latex-abspann.tex
%% Gemeinsamer Abspann für Korrekturansicht und Leseansicht.
%% Setzt den Schalter \ifkorrekturansicht voraus (gesetzt in den
%% einbindenden Dateien latex-korrekturansicht-abspann.tex bzw.
%% latex-leseansicht-abspann.tex).
%% ---------------------------------------------------------------

\normalsize

% Das esempio-Environment wird nur in der Leseansicht benötigt
\ifkorrekturansicht\else
\newenvironment{esempio}[3]%
{
    \vspace{1.5ex}
    \rlap{\underline{#1}}
    \par
    \setlength{\parindent}{0cm}
    \nopagebreak
    \leftskip=#2cm
    \rightskip=#3cm
}
{
    \par
}
\fi

\doendnotes{C}
\bigskip
\vfill

\clearpage

\footnotesize

\ifkorrekturansicht
  \lohead{\textsc{register}}
\fi

% theindex-Environment neu definieren ohne reledmac
\makeatletter
\renewenvironment{theindex}{%
  \ifkorrekturansicht
    \section*{\indexname}%
  \else
    \subsubsection*{Index der erwähnten Entitäten}%
  \fi
  \setlength{\parindent}{0pt}%
  \setlength{\parskip}{0pt plus 0.3pt}%
  \let\item\@idxitem
}{%
  \ifkorrekturansicht\clearpage\fi
}
\makeatother

\IfFileExists{\jobname-pw.ind}{\input{\jobname-pw.ind}}{}

% Quellenangabe nur in der Leseansicht
\ifkorrekturansicht\else
% Fallback-Definitionen, falls die .tex-Datei \titel etc. nicht gesetzt hat
\providecommand{\titel}{}
\providecommand{\editorInnen}{}
\providecommand{\dateiname}{\jobname}

\vspace{3cm}

\vfill

\footnotesize
\textsc{Quelle}: \titel. Herausgegeben von {\editorInnen}. In: \emph{Arthur Schnitzler: Briefwechsel mit Autorinnen und Autoren}.
 Digitale Edition, https://schnitzler-briefe.acdh.oeaw.ac.at/{\dateiname}.html (Stand \today)
\fi

\end{document}


