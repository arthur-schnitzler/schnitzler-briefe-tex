%% latex-leseansicht-vorspann.tex
%% Vorspann für die Leseansicht.
%% Lädt die gemeinsame Datei latex-vorspann.tex mit nicht gesetztem Schalter.

\newif\ifkorrekturansicht
\korrekturansichtfalse

\input{../tex-inputs/latex-vorspann}


\section[Arthur Schnitzler an Gustav Schwarzkopf, {[}18. 1. 1900?{]}]{L04174 Arthur Schnitzler an Gustav Schwarzkopf, {[}18. 1. 1900?{]}}
\nopagebreak\mylabel{L04174v}
\rehead{ }\normalsize\beginnumbering\briefempfaengerindex{Schwarzkopf, Gustav@\textsc{Schwarzkopf, Gustav}!zzzSchnitzler, Arthur@\emph{von Arthur Schnitzler}!1900-01-181@{{[}18. 1. 1900?{]}}|(be}
\toendnotes[C]{\smallbreak\pagebreak[2]}
\correspDesc{Versand  durch Arthur Schnitzler am [18. 1. 1900?] in Wien
\newline{}Erhalt  durch Gustav Schwarzkopf im Zeitraum [18. 1. 1900 –
                  19. 1. 1900?] in Wien}\toendnotes[C]{\smallbreak}
\Standort{CUL, Schnitzler, B 96.}
\physDesc{Visitenkarte, 241 Zeichen
\newline{}Handschrift: Bleistift, deutsche Kurrent}\toendnotes[C]{\smallbreak}
\pstart
           \noindent{}{\pb}lieber Guſtav, beifolgd ein Sitz zu dem morgigen angeblich \label{K_L04174-1v}\edtext{literariſchen Abend\eventindex{Theater in der Josefstadt@\textbf{Theater in der Josefstadt}!Literarischer Abend (Premiere von Die Frage an das Schicksal, Gläubiger), 19.1.1900@Literarischer Abend (Premiere von Die Frage an das Schicksal, Gläubiger), 19.1.1900|pwv}}{\lemma{\textnormal{\emph{literarischen Abend}}}\Cendnote{\textnormal{Als »literarischer Abend\eventindex{Theater in der Josefstadt@\textbf{Theater in der Josefstadt}!Literarischer Abend (Premiere von Die Frage an das Schicksal, Gläubiger), 19.1.1900@Literarischer Abend (Premiere von Die Frage an das Schicksal, Gläubiger), 19.1.1900|pwkv}« wurde die Aufführung von \emph{Die Frage an das Schicksal}\pwindex{Schnitzler, Arthur 15. 5. 1862 Wien – 21. 10. 1931 ebd.@\textsc{Schnitzler, Arthur} (15. 5. 1862 Wien – 21. 10. 1931 ebd.), \emph{Schriftsteller, Mediziner}!Frage an das Schicksal@\strich\emph{Die Frage an das Schicksal}|pwk} (von Arthur Schnitzler) und \emph{Gläubiger}\pwindex{Strindberg, August 22.\,1.\,1849 Stockholm – 14.\,5.\,1912 ebd.@\textsc{Strindberg, August} (22.\,1.\,1849 Stockholm – 14.\,5.\,1912 ebd.), \emph{Schriftsteller}!Gläubiger. Schauspiel in einem Act@\strich\emph{Gläubiger. Schauspiel in einem Act}|pwk} von August
                     Strindberg\pwindex{Strindberg, August 22.\,1.\,1849 Stockholm – 14.\,5.\,1912 ebd.@\textsc{Strindberg, August} (22.\,1.\,1849 Stockholm – 14.\,5.\,1912 ebd.), \emph{Schriftsteller}|pwk} am 19. 1. 1900 im Theater in der Josefstadt\oindex{Wien@\textbf{Wien}!VIII., Josefstadt@\textbf{VIII., Josefstadt}!Theater in der Josefstadt@\textbf{Theater in der Josefstadt}, \emph{Theater}|pwk} beworben. Schnitzler war explizit nicht bei der
                  Aufführung (vgl. A. S.: \emph{Tagebuch}, 19. 1. 1900),
                  dafür aber am Vortag 18. 1. 1900
                  bei der Generalprobe\eventindex{Theater in der Josefstadt@\textbf{Theater in der Josefstadt}!Generalprobe von Frage an das Schicksal, 18.1.1900@Generalprobe von Frage an das Schicksal, 18.1.1900|pwkv} gewesen.}}}\label{K_L04174-1}. Ich bin (vielleicht, – we{\geminationn}
               faſt gewiſs,) nachher \label{K_L04174-2v}\edtext{im Café \textsc{Kaiserhof\oindex{Wien@\textbf{Wien}!I., Innere Stadt@\textbf{I., Innere Stadt}!Café Kaiserhof (Inh. Johann Wortner) [Wien]@\textbf{Café Kaiserhof (Inh. Johann Wortner) [Wien]}, \emph{Kaffeehaus}|pw}}}{\lemma{\textnormal{\emph{im Café Kaiserhof}}}\Cendnote{\textnormal{Der Besuch im Kaffeehaus findet sich nicht im \emph{Tagebuch}\pwindex{Schnitzler, Arthur 15. 5. 1862 Wien – 21. 10. 1931 ebd.@\textsc{Schnitzler, Arthur} (15. 5. 1862 Wien – 21. 10. 1931 ebd.), \emph{Schriftsteller, Mediziner}!Tagebuch@\strich\emph{Tagebuch}|pwk}.}}}\label{K_L04174-2} aber erst ziemlich ſpät.\pend
           
\pstart
           Herzlichſt{\\[\baselineskip]} Ihr{\\[\baselineskip]}\spacefill\mbox{Arthur}\pend
           \leftskip=0em{}
\pstart
           \noindent{}(Im Theater\oindex{Wien@\textbf{Wien}!VIII., Josefstadt@\textbf{VIII., Josefstadt}!Theater in der Josefstadt@\textbf{Theater in der Josefstadt}, \emph{Theater}|pwv} bin ich nicht,
                     {\pb}habe heute die Generalprobe\eventindex{Theater in der Josefstadt@\textbf{Theater in der Josefstadt}!Generalprobe von Frage an das Schicksal, 18.1.1900@Generalprobe von Frage an das Schicksal, 18.1.1900|pwv} geſehn)\pend
           
\pstart
           \centering{}\textcolor{gray}{\textbf{ D\textsuperscript{r} Arthur Schnitzler}}\pend
           
\pstart
           \raggedleft{}\textcolor{gray}{\textbf{Wien}}\oindex{Wien@\textbf{Wien}, \emph{Verwaltungsgebiet}|pw}\pend
           \selectlanguage{ngerman}\endnumbering\briefempfaengerindex{Schwarzkopf, Gustav@\textsc{Schwarzkopf, Gustav}!zzzSchnitzler, Arthur@\emph{von Arthur Schnitzler}!1900-01-181@{{[}18. 1. 1900?{]}}|)be}\mylabel{L04174h}
\begin{anhang}
\end{anhang}\newcommand{\dateiname}{L04174}\newcommand{\titel}{Arthur Schnitzler an Gustav Schwarzkopf, [18. 1. 1900?]}\newcommand{\editorInnen}{Herausgegeben von Jahnke, SelmaMüller, Martin Anton}%% latex-leseansicht-abspann.tex
%% Abspann für die Leseansicht.
%% Der Schalter \ifkorrekturansicht ist bereits durch den Vorspann gesetzt.

%% latex-abspann.tex
%% Gemeinsamer Abspann für Korrekturansicht und Leseansicht.
%% Setzt den Schalter \ifkorrekturansicht voraus (gesetzt in den
%% einbindenden Dateien latex-korrekturansicht-abspann.tex bzw.
%% latex-leseansicht-abspann.tex).
%% ---------------------------------------------------------------

\normalsize

% Das esempio-Environment wird nur in der Leseansicht benötigt
\ifkorrekturansicht\else
\newenvironment{esempio}[3]%
{
    \vspace{1.5ex}
    \rlap{\underline{#1}}
    \par
    \setlength{\parindent}{0cm}
    \nopagebreak
    \leftskip=#2cm
    \rightskip=#3cm
}
{
    \par
}
\fi

\doendnotes{C}
\bigskip
\vfill

\clearpage

\footnotesize

\ifkorrekturansicht
  \lohead{\textsc{register}}
\fi

% theindex-Environment neu definieren ohne reledmac
\makeatletter
\renewenvironment{theindex}{%
  \ifkorrekturansicht
    \section*{\indexname}%
  \else
    \subsubsection*{Index der erwähnten Entitäten}%
  \fi
  \setlength{\parindent}{0pt}%
  \setlength{\parskip}{0pt plus 0.3pt}%
  \let\item\@idxitem
}{%
  \ifkorrekturansicht\clearpage\fi
}
\makeatother

\IfFileExists{\jobname-pw.ind}{\input{\jobname-pw.ind}}{}

% Quellenangabe nur in der Leseansicht
\ifkorrekturansicht\else
% Fallback-Definitionen, falls die .tex-Datei \titel etc. nicht gesetzt hat
\providecommand{\titel}{}
\providecommand{\editorInnen}{}
\providecommand{\dateiname}{\jobname}

\vspace{3cm}

\vfill

\footnotesize
\textsc{Quelle}: \titel. Herausgegeben von {\editorInnen}. In: \emph{Arthur Schnitzler: Briefwechsel mit Autorinnen und Autoren}.
 Digitale Edition, https://schnitzler-briefe.acdh.oeaw.ac.at/{\dateiname}.html (Stand \today)
\fi

\end{document}


