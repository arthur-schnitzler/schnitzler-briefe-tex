%% latex-korrekturansicht-vorspann.tex
%% Vorspann für die Korrekturansicht.
%% Lädt die gemeinsame Datei latex-vorspann.tex mit gesetztem Schalter.

\newif\ifkorrekturansicht
\korrekturansichttrue

\input{../tex-inputs/latex-vorspann}


\section[Thomas Mann an Arthur Schnitzler, 22. 5. 1913]{L02139 Thomas Mann an Arthur Schnitzler, 22. 5. 1913}
\nopagebreak\mylabel{L02139v}
\rehead{ }\normalsize\beginnumbering\briefempfaengerindex{Schnitzler, Arthur@\textsc{Schnitzler, Arthur}!zzzMann, Thomas@\emph{von Thomas Mann}!1913-05-221@{22. 5. 1913}|(be}
\toendnotes[C]{\smallbreak\pagebreak[2]}\Standort{CUL, Schnitzler, B 67.}
\physDesc{Brief, 1 Blatt, 3 Seiten, 1462 Zeichen
\newline{}Handschrift: schwarze Tinte, deutsche Kurrent
\newline{}Schnitzler: 1) mit Bleistift beschriftet: »\textsc{Thomas Mann}«  2) mit rotem Buntstift eine Unterstreichung}
\buchAbdrucke{\weitereDrucke{1) Thomas Mann: \emph{Briefe 1889–1936}. Frankfurt am Main: \emph{S. Fischer} 1961, S. 102.} \weitereDrucke{2) \emph{Modern Austrian Literature}, Jg. 7 (1974) Nr. 1/2, S. 16–17.} }\toendnotes[C]{\smallbreak}
\pstart
           \raggedleft{}{\pb}\textcolor{gray}{\textbf{BAD TÖLZ\oindex{Bad Toelz@\textbf{Bad Tölz}, \emph{P.PPLA3}|pw}, DEN}}{ }22. Mai 1913.\pend
           
\pstart
           \raggedleft{}\textcolor{gray}{\textbf{LANDHAUS THOMAS MANN.\oindex{Thomas Mann Villa@\textbf{Thomas Mann Villa}, \emph{Wohngebäude (K.WHS)}|pw}}}\pend
           
\pstart{}Verehrter Herr Doctor:\pend\vspace{0.5em}
\pstart
           Ihre wundervolle Sommergeſchichte\pwindex{Frau Beate und ihr Sohn. Novelle@\emph{Frau Beate und ihr Sohn. Novelle}|pwv}, von der mir ein Exemplar in Ihrem gütigen Auftrage
               zugeſandt wurde, habe ich geſtern Abend in großer Bewegung beendigt. Sie wird mich
               noch lange feſthalten und beſchäftigen. Die heutige Kunſt verſteht ſich ja im Ganzen
               nicht ſchlecht auf »Stimmung«; aber einen Fall, wo Stimmung ſich dermaßen
               unerbittlich, fürchterlich, verhängnishaft verdichtet, wie hier bei Ihnen, – den gibt
               es, glaube ich, auch heute {\pb}nicht zum
               zweiten Mal. Ich werde nicht müde, auch bei geſchloſſenem Buche die Dichtigkeit und
               magiſche Unzerreißbarkeit dieſes erotiſchen Kunſt- und Schickſalsgeſpinſtes zu prüfen
               und zu bewundern und bitte Ihnen meinen tiefen Reſpekt ausdrücken zu dürfen vor Ihrer
               großen Zaubermacht. Der Schluß geht mir beſtändig nach. Trotz feinſter,
               vielfältigſter Vorbereitung – iſt er möglich ſo oder iſt er es nicht? Auf jeden Fall
               iſt er überwältigend ſchön.\pend
           
\pstart
           Ich habe die Überraſchung, zu ſehen, daß mein »Tod in
                  Venedig\pwindex{Tod in Venedig@\emph{Der Tod in Venedig}|pw}«, bei deſſen Herſtellung ich {\pb}auf garnichts hoffte, ſehr warm
               aufgenommen wird. Bis auf einen giftigen Angriff\pwindex{Tagebuch@\emph{Tagebuch}|pwv} des Herrn Kerr\pwindex{Kerr, Alfred 25.12.1867 – 12.10.1948@\textsc{Kerr, Alfred} (25.12.1867 – 12.10.1948), \emph{Schriftsteller/Schriftstellerin, Kritiker/Kritikerin}|pw}, hinter deſſen tänzeriſchem Pamphletchen\pwindex{Tagebuch@\emph{Tagebuch}|pwv} gegen mich ſich freilich viel Charakter-Elend
               verbirgt, habe ich faſt nur ſehr Ehrenvolles darüber gehört. Und daß die erſte
               Beruhigung vom Autor der »Frau Beate\pwindex{Frau Beate und ihr Sohn. Novelle@\emph{Frau Beate und ihr Sohn. Novelle}|pw}« kam,
               darüber bin ich nun wieder beſonders glücklich.\pend
           
\pstart
           Mit den beſten Empfehlungen an Sie und Ihre Gattin\pwindex{Schnitzler, Olga 17.01.1882 – 13.01.1970@\textsc{Schnitzler, Olga} (17.01.1882 – 13.01.1970), \emph{Schauspieler/Schauspielerin, Sänger/Sängerin}|pwv}, verehrter Herr Doctor,{\\[\baselineskip]}Ihr ergebenſter{\\[\baselineskip]}\spacefill\mbox{Thomas Mann.}\pend
           \leftskip=0em{}\selectlanguage{ngerman}\endnumbering\briefempfaengerindex{Schnitzler, Arthur@\textsc{Schnitzler, Arthur}!zzzMann, Thomas@\emph{von Thomas Mann}!1913-05-221@{22. 5. 1913}|)be}\mylabel{L02139h}  \normalsize

\doendnotes{C}
\bigskip
\vfill

\clearpage

\footnotesize

\lohead{\textsc{register}}

% Definiere theindex-Environment komplett neu ohne reledmac
\makeatletter
\renewenvironment{theindex}{%
  \section*{\indexname}%
  \setlength{\parindent}{0pt}%
  \setlength{\parskip}{0pt plus 0.3pt}%
  \let\item\@idxitem
}{%
  \clearpage
}
\makeatother

\IfFileExists{\jobname-pw.ind}{\input{\jobname-pw.ind}}{}

\end{document}

      