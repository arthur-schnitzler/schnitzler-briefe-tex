%% latex-leseansicht-vorspann.tex
%% Vorspann für die Leseansicht.
%% Lädt die gemeinsame Datei latex-vorspann.tex mit nicht gesetztem Schalter.

\newif\ifkorrekturansicht
\korrekturansichtfalse

\input{../tex-inputs/latex-vorspann}


\section[Thomas Mann an Arthur Schnitzler, 22. 5. 1913]{L02139 Thomas Mann an Arthur Schnitzler, 22. 5. 1913}
\nopagebreak\mylabel{L02139v}
\rehead{ }\normalsize\beginnumbering\briefempfaengerindex{Schnitzler, Arthur@\textsc{Schnitzler, Arthur}!zzzMann, Thomas@\emph{von Thomas Mann}!1913-05-221@{22. 5. 1913}|(be}
\toendnotes[C]{\smallbreak\pagebreak[2]}
\correspDesc{Versand  durch Thomas Mann am 22. 5. 1913 in Bad Tölz
\newline{}Erhalt  durch Arthur Schnitzler im Zeitraum [23. 5. 1913
                  – 27. 5. 1913?] in Wien}\toendnotes[C]{\smallbreak}
\Standort{CUL, Schnitzler, B 67.}
\physDesc{Brief, 1 Blatt, 3 Seiten, 1462 Zeichen
\newline{}Handschrift: schwarze Tinte, deutsche Kurrent
\newline{}Schnitzler: 1) mit Bleistift beschriftet: »\textsc{Thomas Mann}«  2) mit rotem Buntstift eine Unterstreichung}
\buchAbdrucke{\weitereDrucke{1) Thomas Mann: \emph{Briefe 1889–1936}. Herausgegeben von Erika Mann. Frankfurt am Main: \emph{S. Fischer} 1961, S. 102.} \weitereDrucke{2) Hertha Krotkoff: \emph{Arthur Schnitzler – Thomas Mann: Briefe.} In: \emph{Modern Austrian Literature}, Jg. 7 (1974) Nr. 1/2, S. 16–17.} }\toendnotes[C]{\smallbreak}
\pstart
           \raggedleft{}{\pb}\textcolor{gray}{\textbf{BAD TÖLZ\oindex{Bad Tölz@\textbf{Bad Tölz}, \emph{Hauptstadt}|pw}, DEN}}{ }22. Mai 1913.\pend
           
\pstart
           \raggedleft{}\textcolor{gray}{\textbf{LANDHAUS THOMAS MANN\oindex{Thomas Mann Villa@\textbf{Thomas Mann Villa}, \emph{Wohngebäude}|pw}.}}\pend
           
\pstart{}Verehrter Herr Doctor:\pend\vspace{0.5em}
\pstart
           Ihre wundervolle Sommergeſchichte\pwindex{Schnitzler, Arthur 15.\,5.\,1862 Wien – 21.\,10.\,1931 ebd.@\textsc{Schnitzler, Arthur} (15.\,5.\,1862 Wien – 21.\,10.\,1931 ebd.), \emph{Schriftsteller, Mediziner}!Frau Beate und ihr Sohn. Novelle@\strich\emph{Frau Beate und ihr Sohn. Novelle}|pwv}, von der mir ein Exemplar in Ihrem gütigen Auftrage
               zugeſandt wurde, habe ich geſtern Abend in großer Bewegung beendigt. Sie wird mich
               noch lange feſthalten und beſchäftigen. Die heutige Kunſt verſteht{ }ſich ja im Ganzen
               nicht{ }ſchlecht auf »Stimmung«; aber einen Fall, wo Stimmung{ }ſich dermaßen
               unerbittlich, fürchterlich, verhängnishaft verdichtet, wie hier bei Ihnen, – den gibt
               es, glaube ich, auch heute {\pb}nicht zum
               zweiten Mal. Ich werde nicht müde, auch bei geſchloſſenem Buche die Dichtigkeit und
               magiſche Unzerreißbarkeit dieſes erotiſchen Kunſt- und Schickſalsgeſpinſtes zu prüfen
               und zu bewundern und bitte Ihnen meinen tiefen Reſpekt ausdrücken zu dürfen vor Ihrer
               großen Zaubermacht. Der Schluß geht mir beſtändig nach. Trotz feinſter,
               vielfältigſter Vorbereitung – iſt er möglich{ }ſo oder iſt er es nicht? Auf jeden Fall
               iſt er überwältigend{ }ſchön.\pend
           
\pstart
           Ich habe die Überraſchung, zu{ }ſehen, daß mein »Tod in
                  Venedig\pwindex{Mann, Thomas 6.\,6.\,1875 Lübeck – 12.\,8.\,1955 Zürich@\textsc{Mann, Thomas} (6.\,6.\,1875 Lübeck – 12.\,8.\,1955 Zürich), \emph{Schriftsteller}!Tod in Venedig@\strich\emph{Der Tod in Venedig}|pw}«, bei deſſen Herſtellung ich {\pb}auf garnichts hoffte,{ }ſehr warm
               aufgenommen wird. Bis auf einen giftigen Angriff\pwindex{Kerr, Alfred 25.\,12.\,1867 Breslau – 12.\,10.\,1948 Hamburg@\textsc{Kerr, Alfred} (25.\,12.\,1867 Breslau – 12.\,10.\,1948 Hamburg), \emph{Schriftsteller, Kritiker}!Tagebuch@\strich\emph{Tagebuch}|pwv} des Herrn Kerr\pwindex{Kerr, Alfred 25.\,12.\,1867 Breslau – 12.\,10.\,1948 Hamburg@\textsc{Kerr, Alfred} (25.\,12.\,1867 Breslau – 12.\,10.\,1948 Hamburg), \emph{Schriftsteller, Kritiker}|pw}, hinter deſſen tänzeriſchem Pamphletchen\pwindex{Kerr, Alfred 25.\,12.\,1867 Breslau – 12.\,10.\,1948 Hamburg@\textsc{Kerr, Alfred} (25.\,12.\,1867 Breslau – 12.\,10.\,1948 Hamburg), \emph{Schriftsteller, Kritiker}!Tagebuch@\strich\emph{Tagebuch}|pwv} gegen mich{ }ſich freilich viel Charakter-Elend
               verbirgt, habe ich faſt nur{ }ſehr Ehrenvolles darüber gehört. Und daß die erſte
               Beruhigung vom Autor der »Frau Beate\pwindex{Schnitzler, Arthur 15.\,5.\,1862 Wien – 21.\,10.\,1931 ebd.@\textsc{Schnitzler, Arthur} (15.\,5.\,1862 Wien – 21.\,10.\,1931 ebd.), \emph{Schriftsteller, Mediziner}!Frau Beate und ihr Sohn. Novelle@\strich\emph{Frau Beate und ihr Sohn. Novelle}|pw}« kam,
               darüber bin ich nun wieder beſonders glücklich.\pend
           
\pstart
           Mit den beſten Empfehlungen an Sie und Ihre Gattin\pwindex{Schnitzler, Olga 17.\,1.\,1882 Wien – 13.\,1.\,1970 Lugano@\textsc{Schnitzler, Olga} (17.\,1.\,1882 Wien – 13.\,1.\,1970 Lugano), \emph{Schauspielerin, Sängerin}|pwv}, verehrter Herr Doctor,{\\[\baselineskip]}Ihr ergebenſter{\\[\baselineskip]}\spacefill\mbox{Thomas Mann.}\pend
           \leftskip=0em{}\selectlanguage{ngerman}\endnumbering\briefempfaengerindex{Schnitzler, Arthur@\textsc{Schnitzler, Arthur}!zzzMann, Thomas@\emph{von Thomas Mann}!1913-05-221@{22. 5. 1913}|)be}\mylabel{L02139h}  \newcommand{\dateiname}{L02139}\newcommand{\titel}{Thomas Mann an Arthur Schnitzler, 22. 5. 1913}\newcommand{\editorInnen}{Martin Anton Müller und Gerd-Hermann Susen}%% latex-leseansicht-abspann.tex
%% Abspann für die Leseansicht.
%% Der Schalter \ifkorrekturansicht ist bereits durch den Vorspann gesetzt.

%% latex-abspann.tex
%% Gemeinsamer Abspann für Korrekturansicht und Leseansicht.
%% Setzt den Schalter \ifkorrekturansicht voraus (gesetzt in den
%% einbindenden Dateien latex-korrekturansicht-abspann.tex bzw.
%% latex-leseansicht-abspann.tex).
%% ---------------------------------------------------------------

\normalsize

% Das esempio-Environment wird nur in der Leseansicht benötigt
\ifkorrekturansicht\else
\newenvironment{esempio}[3]%
{
    \vspace{1.5ex}
    \rlap{\underline{#1}}
    \par
    \setlength{\parindent}{0cm}
    \nopagebreak
    \leftskip=#2cm
    \rightskip=#3cm
}
{
    \par
}
\fi

\doendnotes{C}
\bigskip
\vfill

\clearpage

\footnotesize

\ifkorrekturansicht
  \lohead{\textsc{register}}
\fi

% theindex-Environment neu definieren ohne reledmac
\makeatletter
\renewenvironment{theindex}{%
  \ifkorrekturansicht
    \section*{\indexname}%
  \else
    \subsubsection*{Index der erwähnten Entitäten}%
  \fi
  \setlength{\parindent}{0pt}%
  \setlength{\parskip}{0pt plus 0.3pt}%
  \let\item\@idxitem
}{%
  \ifkorrekturansicht\clearpage\fi
}
\makeatother

\IfFileExists{\jobname-pw.ind}{\input{\jobname-pw.ind}}{}

% Quellenangabe nur in der Leseansicht
\ifkorrekturansicht\else
% Fallback-Definitionen, falls die .tex-Datei \titel etc. nicht gesetzt hat
\providecommand{\titel}{}
\providecommand{\editorInnen}{}
\providecommand{\dateiname}{\jobname}

\vspace{3cm}

\vfill

\footnotesize
\textsc{Quelle}: \titel. Herausgegeben von {\editorInnen}. In: \emph{Arthur Schnitzler: Briefwechsel mit Autorinnen und Autoren}.
 Digitale Edition, https://schnitzler-briefe.acdh.oeaw.ac.at/{\dateiname}.html (Stand \today)
\fi

\end{document}


