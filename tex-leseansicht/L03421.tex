%% latex-leseansicht-vorspann.tex
%% Vorspann für die Leseansicht.
%% Lädt die gemeinsame Datei latex-vorspann.tex mit nicht gesetztem Schalter.

\newif\ifkorrekturansicht
\korrekturansichtfalse

\input{../tex-inputs/latex-vorspann}


\section[ Felix Salten an Arthur Schnitzler, 22. – 23. 4. 1906]{L03421 Felix Salten an Arthur Schnitzler,  22. – 23. 4. 1906}
\nopagebreak\mylabel{L03421v}
\rehead{ }\normalsize\beginnumbering\briefempfaengerindex{Schnitzler, Arthur@\textsc{Schnitzler, Arthur}!zzzSalten, Felix@\emph{von Felix Salten}!1906-04-231@{22. – 23. 4. 1906}|(be}
\toendnotes[C]{\smallbreak\pagebreak[2]}
\correspDesc{Versand  durch Felix Salten im Zeitraum 22. – 23. 4. 1906 in Berlin
\newline{}Erhalt  durch Arthur Schnitzler im Zeitraum 24. 4. 1906 – 27. 4. 1906 in Wien}\toendnotes[C]{\smallbreak}
\Standort{CUL, Schnitzler, B 89, B 1.}
\physDesc{Brief, 1 Blatt, 2 Seiten, 3329 Zeichen
\newline{}Handschrift: schwarze Tinte, lateinische Kurrent
\newline{}Ordnung: mit Bleistift von unbekannter Hand nummeriert: »211« }\toendnotes[C]{\smallbreak}
\pstart
           \raggedleft{}{\pb}Berlin\oindex{Berlin@\textbf{Berlin}, \emph{Hauptstadt}|pw}, 22. IV. 06\pend
           \vspace{0.5em}
\pstart
           Lieber, eben, da ich mich hinsetzen will, um Ihnen zu schreiben,
               kommt Ihre zweite Depesche. Ich bin nun einigermaßen in Verlegenheit. Denn wie leicht
               kann Brahm\pwindex{Brahm, Otto 5.\,2.\,1856 Hamburg – 28.\,11.\,1912 Berlin@\textsc{Brahm, Otto} (5.\,2.\,1856 Hamburg – 28.\,11.\,1912 Berlin), \emph{Theaterleiter, Regisseur}|pw} meinen unverlangten \label{K_L03421-1v}\edtext{Rath}{\lemma{\textnormal{\emph{Rath}}}\Cendnote{\textnormal{Siehe XXXX Auszeichnungsfehler: Dokument L03420 nicht gefunden.
               }}}\label{K_L03421-1} ablehnen; kann ihn, was mir noch weniger lieb wäre, missdeuten, und als die
               Sucht, »dreinzureden« auffassen. Ganz abgesehen davon, dass ich ja garnicht weiss, ob
                  Brahm\pwindex{Brahm, Otto 5.\,2.\,1856 Hamburg – 28.\,11.\,1912 Berlin@\textsc{Brahm, Otto} (5.\,2.\,1856 Hamburg – 28.\,11.\,1912 Berlin), \emph{Theaterleiter, Regisseur}|pw} auf mein Urteil auch nur das Mindeste
               gibt. Und ausserdem habe ich, als wir \label{K_L03421-2v}\edtext{nach der Vorstellung\pwindex{Schnitzler, Arthur 15.\,5.\,1862 Wien – 21.\,10.\,1931 ebd.@\textsc{Schnitzler, Arthur} (15.\,5.\,1862 Wien – 21.\,10.\,1931 ebd.), \emph{Schriftsteller, Mediziner}!einsame Weg. Schauspiel in fünf Akten@\strich\emph{Der einsame Weg. Schauspiel in fünf Akten}|pw}}{\lemma{\textnormal{\emph{nach der Vorstellung}}}\Cendnote{\textnormal{Siehe XXXX Auszeichnungsfehler: Dokument L03419 nicht gefunden.
               }}}\label{K_L03421-2} beisammen waren, zu merken geglaubt, dass Brahm\pwindex{Brahm, Otto 5.\,2.\,1856 Hamburg – 28.\,11.\,1912 Berlin@\textsc{Brahm, Otto} (5.\,2.\,1856 Hamburg – 28.\,11.\,1912 Berlin), \emph{Theaterleiter, Regisseur}|pw} (vielleicht aus Theaterpolitik) Reichers\pwindex{Reicher, Emanuel 18.\,6.\,1849 Bochnia – 15.\,5.\,1924 Berlin@\textsc{Reicher, Emanuel} (18.\,6.\,1849 Bochnia – 15.\,5.\,1924 Berlin), \emph{Schauspieler}|pw}{ }Julian\pwindex{Schnitzler, Arthur 15.\,5.\,1862 Wien – 21.\,10.\,1931 ebd.@\textsc{Schnitzler, Arthur} (15.\,5.\,1862 Wien – 21.\,10.\,1931 ebd.), \emph{Schriftsteller, Mediziner}!einsame Weg. Schauspiel in fünf Akten@\strich\emph{Der einsame Weg. Schauspiel in fünf Akten}|pwv} über
               den von Rittner\pwindex{Rittner, Rudolf 30.\,6.\,1869 Bílý Potok – 4.\,2.\,1943 ebd.@\textsc{Rittner, Rudolf} (30.\,6.\,1869 Bílý Potok – 4.\,2.\,1943 ebd.), \emph{Theaterleiter, Schauspieler}|pw} zu stellen geneigt ist. Ich
               kann mich ja darin irren. Jedenfalls erleichtert es die Situation nicht, denn ich
               habe Rittner\pwindex{Rittner, Rudolf 30.\,6.\,1869 Bílý Potok – 4.\,2.\,1943 ebd.@\textsc{Rittner, Rudolf} (30.\,6.\,1869 Bílý Potok – 4.\,2.\,1943 ebd.), \emph{Theaterleiter, Schauspieler}|pw} in dieser Rolle nicht gesehen.
               Wie immer er aber auch gewesen sein mag, er war sicherlich besser als Reicher\pwindex{Reicher, Emanuel 18.\,6.\,1849 Bochnia – 15.\,5.\,1924 Berlin@\textsc{Reicher, Emanuel} (18.\,6.\,1849 Bochnia – 15.\,5.\,1924 Berlin), \emph{Schauspieler}|pw}. Einfach aus dem Grund, weil es
               unmöglich ist, schlechter zu sein als Herr Reicher\pwindex{Reicher, Emanuel 18.\,6.\,1849 Bochnia – 15.\,5.\,1924 Berlin@\textsc{Reicher, Emanuel} (18.\,6.\,1849 Bochnia – 15.\,5.\,1924 Berlin), \emph{Schauspieler}|pw} war. (Dieser Satz könnte von Goldmann\pwindex{Goldmann, Paul 31.\,1.\,1865 Breslau – 25.\,9.\,1935 Wien@\textsc{Goldmann, Paul} (31.\,1.\,1865 Breslau – 25.\,9.\,1935 Wien), \emph{Schriftsteller, Journalist}|pw} sein; ist aber gleichwol richtig) Um Rittner ist doch stets ein
               Hauch von der Fülle der Erlebnisse. Auch ein leiser Hauch von Einsamkeit ist jetzt
               mehr und mehr um ihn. Rittner\pwindex{Rittner, Rudolf 30.\,6.\,1869 Bílý Potok – 4.\,2.\,1943 ebd.@\textsc{Rittner, Rudolf} (30.\,6.\,1869 Bílý Potok – 4.\,2.\,1943 ebd.), \emph{Theaterleiter, Schauspieler}|pw} ist doch auf
               eine glaubhafte Art von Verliebtheit umgeben, von allerlei Karessen, und das Parfum
               vieler Frauen haftet gleichsam in seinen Kleidern. Wenn nun alle diese Dinge welk und
               herbstlich werden, dann haben sie, wie es der Julian\pwindex{Schnitzler, Arthur 15.\,5.\,1862 Wien – 21.\,10.\,1931 ebd.@\textsc{Schnitzler, Arthur} (15.\,5.\,1862 Wien – 21.\,10.\,1931 ebd.), \emph{Schriftsteller, Mediziner}!einsame Weg. Schauspiel in fünf Akten@\strich\emph{Der einsame Weg. Schauspiel in fünf Akten}|pwv} braucht{[},{]} jene Melancholie,
               deren besondere Schattierung eben ein Goldton ist, ein verblaßender, vormals aber –
               das sieht man noch genau – üppiger und leuchtender Goldton. Von solchen {\pb}Dingen ist bei Reicher\pwindex{Reicher, Emanuel 18.\,6.\,1849 Bochnia – 15.\,5.\,1924 Berlin@\textsc{Reicher, Emanuel} (18.\,6.\,1849 Bochnia – 15.\,5.\,1924 Berlin), \emph{Schauspieler}|pw} nichts zu spüren. Er ist ganz und gar bürgerlich. Hat
               leider den Moment versäumt, Kinder zu zeugen, mit denen er jetzt \label{K_L03421-3v}\edtext{Schabbes}{\lemma{\textnormal{\emph{Schabbes}}}\Cendnote{\textnormal{Sabbat}}}\label{K_L03421-3} machen oder den \label{K_L03421-4v}\edtext{Seder-Abend}{\lemma{\textnormal{\emph{Seder-Abend}}}\Cendnote{\textnormal{Abendessen am Vorabend des Pessach-Festes}}}\label{K_L03421-4} halten könnte. Mir wäre, wie ich
               gewiss nicht erst zu sagen brauche, auch der jüdische Julian\pwindex{Schnitzler, Arthur 15.\,5.\,1862 Wien – 21.\,10.\,1931 ebd.@\textsc{Schnitzler, Arthur} (15.\,5.\,1862 Wien – 21.\,10.\,1931 ebd.), \emph{Schriftsteller, Mediziner}!einsame Weg. Schauspiel in fünf Akten@\strich\emph{Der einsame Weg. Schauspiel in fünf Akten}|pwv} recht, wenn es nur eben ein Julian\pwindex{Schnitzler, Arthur 15.\,5.\,1862 Wien – 21.\,10.\,1931 ebd.@\textsc{Schnitzler, Arthur} (15.\,5.\,1862 Wien – 21.\,10.\,1931 ebd.), \emph{Schriftsteller, Mediziner}!einsame Weg. Schauspiel in fünf Akten@\strich\emph{Der einsame Weg. Schauspiel in fünf Akten}|pwv} wäre: etwa Adalbert Goldschmidt\pwindex{Goldschmidt, Adalbert von 5.\,5.\,1848 Wien – 21.\,12.\,1906 ebd.@\textsc{Goldschmidt, Adalbert von} (5.\,5.\,1848 Wien – 21.\,12.\,1906 ebd.), \emph{Schriftsteller, Komponist}|pw}, der ja den jüdischen und zugleich einen
                  Daudet\pwindex{Daudet, Alphonse 13.\,5.\,1840 Nîmes – 16.\,11.\,1897 Paris@\textsc{Daudet, Alphonse} (13.\,5.\,1840 Nîmes – 16.\,11.\,1897 Paris), \emph{Schriftsteller}|pw}’schen Einschlag hat. Allein Reicher\pwindex{Reicher, Emanuel 18.\,6.\,1849 Bochnia – 15.\,5.\,1924 Berlin@\textsc{Reicher, Emanuel} (18.\,6.\,1849 Bochnia – 15.\,5.\,1924 Berlin), \emph{Schauspieler}|pw} ist trocken, und erscheint höchstens
               als verkrachter Familienvater. – – –\pend
           \selectlanguage{ngerman}\vspace{1em}
\pstart
           Montag.\pend
           \vspace{0.5em}
\pstart
           Gestern wurde ich durch Besuche (die Leute machen hier
               unaufhörlich Besuche) unterbrochen. \label{K_L03421-5v}\edtext{Abends taf ich zufällig Rittner\pwindex{Rittner, Rudolf 30.\,6.\,1869 Bílý Potok – 4.\,2.\,1943 ebd.@\textsc{Rittner, Rudolf} (30.\,6.\,1869 Bílý Potok – 4.\,2.\,1943 ebd.), \emph{Theaterleiter, Schauspieler}|pw}. Er ist nicht abgeneigt, den Julian\pwindex{Schnitzler, Arthur 15.\,5.\,1862 Wien – 21.\,10.\,1931 ebd.@\textsc{Schnitzler, Arthur} (15.\,5.\,1862 Wien – 21.\,10.\,1931 ebd.), \emph{Schriftsteller, Mediziner}!einsame Weg. Schauspiel in fünf Akten@\strich\emph{Der einsame Weg. Schauspiel in fünf Akten}|pwv} in Wien\oindex{Wien@\textbf{Wien}, \emph{Verwaltungsgebiet}|pw} zu
               spielen. Oder genauer: »im Prinzip nicht dagegen}{\lemma{\textnormal{\emph{Abends … dagegen}}}\Cendnote{\textnormal{Diese Stelle übermittelte Schnitzler an Brahm\pwindex{Brahm, Otto 5.\,2.\,1856 Hamburg – 28.\,11.\,1912 Berlin@\textsc{Brahm, Otto} (5.\,2.\,1856 Hamburg – 28.\,11.\,1912 Berlin), \emph{Theaterleiter, Regisseur}|pwk}: »Vor
                     ein paar Tagen traf Rittner\pwindex{Rittner, Rudolf 30.\,6.\,1869 Bílý Potok – 4.\,2.\,1943 ebd.@\textsc{Rittner, Rudolf} (30.\,6.\,1869 Bílý Potok – 4.\,2.\,1943 ebd.), \emph{Theaterleiter, Schauspieler}|pw}{ }Salten\pwindex{Salten, Felix 6.\,9.\,1869 Budapest – 8.\,10.\,1945 Zürich@\textsc{Salten, Felix} (6.\,9.\,1869 Budapest – 8.\,10.\,1945 Zürich), \emph{Schriftsteller, Journalist, Chefredakteur}|pw} auf
                     der Straße und äußerte sich gesprächsweise zu ihm ›er sei im Prinzip nicht
                     dagegen, den Julian\pwindex{Schnitzler, Arthur 15.\,5.\,1862 Wien – 21.\,10.\,1931 ebd.@\textsc{Schnitzler, Arthur} (15.\,5.\,1862 Wien – 21.\,10.\,1931 ebd.), \emph{Schriftsteller, Mediziner}!einsame Weg. Schauspiel in fünf Akten@\strich\emph{Der einsame Weg. Schauspiel in fünf Akten}|pwv} in
                  Wien\oindex{Wien@\textbf{Wien}, \emph{Verwaltungsgebiet}|pw} zu spielen{\dots}‹« \emph{Der Briefwechsel Arthur Schnitzler – Otto Brahm}.
                     Vollständige Ausgabe. Herausgegeben, eingeleitet und erläutert von Oskar
                     Seidlin. Tübingen: \emph{Niemeyer}{ }1975, S. 227.}}}\label{K_L03421-5}{[}«{]}. Als ich ihm sagte, \uline{Sie} hätten keineswegs darauf bestanden, dass er den
                  \label{K_L03421-6v}\edtext{Forstadjunkten\pwindex{Schnitzler, Arthur 15.\,5.\,1862 Wien – 21.\,10.\,1931 ebd.@\textsc{Schnitzler, Arthur} (15.\,5.\,1862 Wien – 21.\,10.\,1931 ebd.), \emph{Schriftsteller, Mediziner}!Ruf des Lebens. Schauspiel in drei Akten@\strich\emph{Der Ruf des Lebens. Schauspiel in drei Akten}|pwv}}{\lemma{\textnormal{\emph{Forstadjunkten}}}\Cendnote{\textnormal{Bei der deutschsprachigen Uraufführung
                  von \emph{Der Ruf des Lebens}\pwindex{Schnitzler, Arthur 15.\,5.\,1862 Wien – 21.\,10.\,1931 ebd.@\textsc{Schnitzler, Arthur} (15.\,5.\,1862 Wien – 21.\,10.\,1931 ebd.), \emph{Schriftsteller, Mediziner}!Ruf des Lebens. Schauspiel in drei Akten@\strich\emph{Der Ruf des Lebens. Schauspiel in drei Akten}|pwk} am 24. 2. 1906 im Lessing-Theater\oindex{Lessing-Theater@\textbf{Lessing-Theater}, \emph{Theater}|pwk} gab Rittner\pwindex{Rittner, Rudolf 30.\,6.\,1869 Bílý Potok – 4.\,2.\,1943 ebd.@\textsc{Rittner, Rudolf} (30.\,6.\,1869 Bílý Potok – 4.\,2.\,1943 ebd.), \emph{Theaterleiter, Schauspieler}|pwk} den Forstadjunkten Eduard Rainer\pwindex{Schnitzler, Arthur 15.\,5.\,1862 Wien – 21.\,10.\,1931 ebd.@\textsc{Schnitzler, Arthur} (15.\,5.\,1862 Wien – 21.\,10.\,1931 ebd.), \emph{Schriftsteller, Mediziner}!Ruf des Lebens. Schauspiel in drei Akten@\strich\emph{Der Ruf des Lebens. Schauspiel in drei Akten}|pwkv}.}}}\label{K_L03421-6} gibt, und hätten
               ihm sein Versagen auch nicht übelgenommen, war er erfreut. Er meint nur, es wird für
                  Brahm\pwindex{Brahm, Otto 5.\,2.\,1856 Hamburg – 28.\,11.\,1912 Berlin@\textsc{Brahm, Otto} (5.\,2.\,1856 Hamburg – 28.\,11.\,1912 Berlin), \emph{Theaterleiter, Regisseur}|pw} schwer sein, Reicher\pwindex{Reicher, Emanuel 18.\,6.\,1849 Bochnia – 15.\,5.\,1924 Berlin@\textsc{Reicher, Emanuel} (18.\,6.\,1849 Bochnia – 15.\,5.\,1924 Berlin), \emph{Schauspieler}|pw} die Rolle abzunehmen, und die für Rittner\pwindex{Rittner, Rudolf 30.\,6.\,1869 Bílý Potok – 4.\,2.\,1943 ebd.@\textsc{Rittner, Rudolf} (30.\,6.\,1869 Bílý Potok – 4.\,2.\,1943 ebd.), \emph{Theaterleiter, Schauspieler}|pw} nötigen Proben abzuhalten. Außerdem wird Brahm\pwindex{Brahm, Otto 5.\,2.\,1856 Hamburg – 28.\,11.\,1912 Berlin@\textsc{Brahm, Otto} (5.\,2.\,1856 Hamburg – 28.\,11.\,1912 Berlin), \emph{Theaterleiter, Regisseur}|pw} es nicht gerne sehen, wenn Rittner\pwindex{Rittner, Rudolf 30.\,6.\,1869 Bílý Potok – 4.\,2.\,1943 ebd.@\textsc{Rittner, Rudolf} (30.\,6.\,1869 Bílý Potok – 4.\,2.\,1943 ebd.), \emph{Theaterleiter, Schauspieler}|pw} über seine Garantie kommt. Die beträgt
               für Wien\oindex{Wien@\textbf{Wien}, \emph{Verwaltungsgebiet}|pw} 12 Abende, welche mit »Elga\pwindex{\textcolor{red}{\textsuperscript{XXXX indx1}}!Elga@\strich\emph{Elga}|pw}« gedeckt scheinen. Ist er im »Einsamen Weg\pwindex{Schnitzler, Arthur 15.\,5.\,1862 Wien – 21.\,10.\,1931 ebd.@\textsc{Schnitzler, Arthur} (15.\,5.\,1862 Wien – 21.\,10.\,1931 ebd.), \emph{Schriftsteller, Mediziner}!einsame Weg. Schauspiel in fünf Akten@\strich\emph{Der einsame Weg. Schauspiel in fünf Akten}|pw}« tätig, muß dann Brahm\pwindex{Brahm, Otto 5.\,2.\,1856 Hamburg – 28.\,11.\,1912 Berlin@\textsc{Brahm, Otto} (5.\,2.\,1856 Hamburg – 28.\,11.\,1912 Berlin), \emph{Theaterleiter, Regisseur}|pw} das Plus zahlen, was er – wie Sie wissen – überhaupt,
               und im Fall Rittner\pwindex{Rittner, Rudolf 30.\,6.\,1869 Bílý Potok – 4.\,2.\,1943 ebd.@\textsc{Rittner, Rudolf} (30.\,6.\,1869 Bílý Potok – 4.\,2.\,1943 ebd.), \emph{Theaterleiter, Schauspieler}|pw} erst recht lieber
               vermeidet.\pend
           
\pstart
           Was soll ich, nach Ihrer Meinung, tun? Dass ich mit Vergnügen zu allem bereit bin,
               brauche ich nicht erst zu sagen. Erwägen Sie, was ich Ihnen wegen mir u. Brahm\pwindex{Brahm, Otto 5.\,2.\,1856 Hamburg – 28.\,11.\,1912 Berlin@\textsc{Brahm, Otto} (5.\,2.\,1856 Hamburg – 28.\,11.\,1912 Berlin), \emph{Theaterleiter, Regisseur}|pw} sagte, und denken Sie nach, wie man es
               machen könnte, dass ich bei Brahm\pwindex{Brahm, Otto 5.\,2.\,1856 Hamburg – 28.\,11.\,1912 Berlin@\textsc{Brahm, Otto} (5.\,2.\,1856 Hamburg – 28.\,11.\,1912 Berlin), \emph{Theaterleiter, Regisseur}|pw} nicht eine
               Unannehmlichkeit erfahre. Soll ich vielleicht Elias\pwindex{Elias, Julius 12.\,7.\,1861 Hoya – 2.\,7.\,1927 Berlin@\textsc{Elias, Julius} (12.\,7.\,1861 Hoya – 2.\,7.\,1927 Berlin), \emph{Übersetzer, Publizist}|pw} zu ihm schicken? Das will ich auf alle Fälle gleich tun.\pend
           
\pstart
           Eben kommt wieder Besuch. (Die Leute machen hier unaufhörlich Besuche) Ich will aber,
               dass der Brief heute abgeht. Also viele herzlichste
               Grüße von uns\pwindex{Salten, Ottilie 7.\,3.\,1868 Prag – 22.\,6.\,1942 Zürich@\textsc{Salten, Ottilie} (7.\,3.\,1868 Prag – 22.\,6.\,1942 Zürich), \emph{Schauspielerin}|pwv} an Sie Beide\pwindex{Schnitzler, Olga 17.\,1.\,1882 Wien – 13.\,1.\,1970 Lugano@\textsc{Schnitzler, Olga} (17.\,1.\,1882 Wien – 13.\,1.\,1970 Lugano), \emph{Schauspielerin, Sängerin}|pwv}. {\\}Ihr
                  \spacefill\mbox{Salten}\pend
           
\pstart
           \noindent{}\label{K_L03421-7v}\edtext{\textcolor{gray}{N}B.}{\lemma{\textnormal{\emph{NB.}}}\Cendnote{\textnormal{nota bene;
                     lateinisch: merke wohl}}}\label{K_L03421-7}{ }Jacobsohn\pwindex{Jacobsohn, Siegfried 28.\,1.\,1881 Berlin – 3.\,12.\,1926 ebd.@\textsc{Jacobsohn, Siegfried} (28.\,1.\,1881 Berlin – 3.\,12.\,1926 ebd.), \emph{Journalist, Kritiker, Publizist}|pw}{ }\label{K_L03421-8v}\edtext{tobt\pwindex{Jacobsohn, Siegfried 28.\,1.\,1881 Berlin – 3.\,12.\,1926 ebd.@\textsc{Jacobsohn, Siegfried} (28.\,1.\,1881 Berlin – 3.\,12.\,1926 ebd.), \emph{Journalist, Kritiker, Publizist}!einsame Weg@\strich\emph{Der einsame Weg}|pwv}}{\lemma{\textnormal{\emph{tobt}}}\Cendnote{\textnormal{[Siegfried Jacobsohn\pwindex{Jacobsohn, Siegfried 28.\,1.\,1881 Berlin – 3.\,12.\,1926 ebd.@\textsc{Jacobsohn, Siegfried} (28.\,1.\,1881 Berlin – 3.\,12.\,1926 ebd.), \emph{Journalist, Kritiker, Publizist}|pwk}]: \emph{Der einsame Weg}\pwindex{Jacobsohn, Siegfried 28.\,1.\,1881 Berlin – 3.\,12.\,1926 ebd.@\textsc{Jacobsohn, Siegfried} (28.\,1.\,1881 Berlin – 3.\,12.\,1926 ebd.), \emph{Journalist, Kritiker, Publizist}!einsame Weg@\strich\emph{Der einsame Weg}|pwk}. In: \emph{Die Schaubühne}\pwindex{Schaubühne@\emph{Die Schaubühne}|pwk}, Jg. 2, Nr. 17, 26. 4. 1906, S. 487–491.}}}\label{K_L03421-8} ja auch
                  gegen Reicher\pwindex{Reicher, Emanuel 18.\,6.\,1849 Bochnia – 15.\,5.\,1924 Berlin@\textsc{Reicher, Emanuel} (18.\,6.\,1849 Bochnia – 15.\,5.\,1924 Berlin), \emph{Schauspieler}|pw}!\pend
           \selectlanguage{ngerman}\endnumbering\briefempfaengerindex{Schnitzler, Arthur@\textsc{Schnitzler, Arthur}!zzzSalten, Felix@\emph{von Felix Salten}!1@{22. – 23. 4. 1906}|)be}\mylabel{L03421h}  \newcommand{\dateiname}{L03421}\newcommand{\titel}{Felix Salten an Arthur Schnitzler, 22. – 23. 4. 1906}\newcommand{\editorInnen}{Martin Anton Müller und Laura Untner}%% latex-leseansicht-abspann.tex
%% Abspann für die Leseansicht.
%% Der Schalter \ifkorrekturansicht ist bereits durch den Vorspann gesetzt.

%% latex-abspann.tex
%% Gemeinsamer Abspann für Korrekturansicht und Leseansicht.
%% Setzt den Schalter \ifkorrekturansicht voraus (gesetzt in den
%% einbindenden Dateien latex-korrekturansicht-abspann.tex bzw.
%% latex-leseansicht-abspann.tex).
%% ---------------------------------------------------------------

\normalsize

% Das esempio-Environment wird nur in der Leseansicht benötigt
\ifkorrekturansicht\else
\newenvironment{esempio}[3]%
{
    \vspace{1.5ex}
    \rlap{\underline{#1}}
    \par
    \setlength{\parindent}{0cm}
    \nopagebreak
    \leftskip=#2cm
    \rightskip=#3cm
}
{
    \par
}
\fi

\doendnotes{C}
\bigskip
\vfill

\clearpage

\footnotesize

\ifkorrekturansicht
  \lohead{\textsc{register}}
\fi

% theindex-Environment neu definieren ohne reledmac
\makeatletter
\renewenvironment{theindex}{%
  \ifkorrekturansicht
    \section*{\indexname}%
  \else
    \subsubsection*{Index der erwähnten Entitäten}%
  \fi
  \setlength{\parindent}{0pt}%
  \setlength{\parskip}{0pt plus 0.3pt}%
  \let\item\@idxitem
}{%
  \ifkorrekturansicht\clearpage\fi
}
\makeatother

\IfFileExists{\jobname-pw.ind}{\input{\jobname-pw.ind}}{}

% Quellenangabe nur in der Leseansicht
\ifkorrekturansicht\else
% Fallback-Definitionen, falls die .tex-Datei \titel etc. nicht gesetzt hat
\providecommand{\titel}{}
\providecommand{\editorInnen}{}
\providecommand{\dateiname}{\jobname}

\vspace{3cm}

\vfill

\footnotesize
\textsc{Quelle}: \titel. Herausgegeben von {\editorInnen}. In: \emph{Arthur Schnitzler: Briefwechsel mit Autorinnen und Autoren}.
 Digitale Edition, https://schnitzler-briefe.acdh.oeaw.ac.at/{\dateiname}.html (Stand \today)
\fi

\end{document}


