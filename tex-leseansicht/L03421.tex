%% latex-leseansicht-vorspann.tex
%% Vorspann für die Leseansicht.
%% Lädt die gemeinsame Datei latex-vorspann.tex mit nicht gesetztem Schalter.

\newif\ifkorrekturansicht
\korrekturansichtfalse

\input{../tex-inputs/latex-vorspann}

\begin{center}
            \textcolor{red}{ENTWURF, NICHT FERTIG KORRIGIERT}
                      \end{center}
            
         
         \renewcommand{\erwaehntePersonen}{Personen: Otto Brahm, Alphonse Daudet, Julius Elias, Paul Goldmann, Adalbert von Goldschmidt, Siegfried Jacobsohn, Emanuel Reicher, Rudolf Rittner, Ottilie Salten, Olga Schnitzler}
         \renewcommand{\erwaehnteOrte}{Orte: Berlin, Lessing-Theater, Wien}
         \renewcommand{\erwaehnteWerke}{Werke: Der Ruf des Lebens. Schauspiel in drei Akten, Der einsame Weg, Der einsame Weg. Schauspiel in fünf Akten, Die Schaubühne, Elga}
               \section[ Felix Salten an Arthur Schnitzler, 22.–23. 4. 1906]{ Felix Salten an Arthur Schnitzler, 22.–23. 4. 1906}\nopagebreak\mylabel{v}\rehead{ }\begin{ledgroupsized}[t]{13cm}\normalsize\beginnumbering \toendnotes[C]{\smallbreak\pagebreak[2]} \Standort{CUL, Schnitzler, B 89, B 1.}
\physDesc{Brief, 1 Blatt, 2 Seiten, 3328 Zeichen
\newline{}Handschrift: schwarze Tinte, lateinische Kurrent
\newline{}Ordnung: mit Bleistift von unbekannter Hand nummeriert: »211« }\toendnotes[C]{\smallbreak}\pstart
           \raggedleft{}{\pb}Berlin\oindex{Berlin@\textbf{Berlin}|pw}, 22. IV. 06\pend
           \pstart
           Lieber, eben, da ich mich hinsetzen will, um Ihnen zu schreiben,
               kommt Ihre zweite Depesche. Ich bin nun einigermaßen in Verlegenheit. Denn wie leicht
               kann Brahm\pwindex{Brahm, Otto 05.02.1856 – 28.11.1912@\textsc{Brahm, Otto} (05.02.1856 – 28.11.1912), \emph{Theaterleiter, Regisseur}|pw} meinen unverlangten \label{K_L03421-1v}\edtext{Rath}{\lemma{\textnormal{\emph{Rath}}}\Cendnote{\textnormal{siehe Felix Salten an Arthur Schnitzler, 21. 4. [1906]}}}\label{K_L03421-1h} ablehnen; kann ihn, was mir noch weniger lieb wäre, missdeuten, und als die
               Sucht, »dreinzureden« auffassen. Ganz abgesehen davon, dass ich ja garnicht weiss, ob
                  Brahm\pwindex{Brahm, Otto 05.02.1856 – 28.11.1912@\textsc{Brahm, Otto} (05.02.1856 – 28.11.1912), \emph{Theaterleiter, Regisseur}|pw} auf mein Urteil auch nur das Mindeste
               gibt. Und ausserdem habe ich, als wir \label{K_L03421-2v}\edtext{nach der Vorstellung\pwindex{Schnitzler, Arthur 15.05.1862 – 21.10.1931@\textsc{Schnitzler, Arthur} (15.05.1862 – 21.10.1931), \emph{Schriftsteller, Mediziner}!einsame Weg. Schauspiel in fuenf Akten1904@\strich\emph{Der einsame Weg. Schauspiel in fünf Akten} {[}1904{]}|pw}}{\lemma{\textnormal{\emph{nach der Vorstellung}}}\Cendnote{\textnormal{siehe Felix Salten u. a. an Arthur Schnitzler, 19. 4. 1906}}}\label{K_L03421-2h} beisammen waren, zu merken geglaubt, dass Brahm\pwindex{Brahm, Otto 05.02.1856 – 28.11.1912@\textsc{Brahm, Otto} (05.02.1856 – 28.11.1912), \emph{Theaterleiter, Regisseur}|pw} (vielleicht aus Theaterpolitik) Reicher\pwindex{Reicher, Emanuel 18.06.1849 – 15.05.1924@\textsc{Reicher, Emanuel} (18.06.1849 – 15.05.1924), \emph{Schauspieler}|pw}s Julian\pwindex{Schnitzler, Arthur 15.05.1862 – 21.10.1931@\textsc{Schnitzler, Arthur} (15.05.1862 – 21.10.1931), \emph{Schriftsteller, Mediziner}!einsame Weg. Schauspiel in fuenf Akten1904@\strich\emph{Der einsame Weg. Schauspiel in fünf Akten} {[}1904{]}|pwv} über
               den von Rittner\pwindex{Rittner, Rudolf 30.06.1869 – 04.02.1943@\textsc{Rittner, Rudolf} (30.06.1869 – 04.02.1943), \emph{Theaterleiter, Schauspieler}|pw} zu stellen geneigt ist. Ich
               kann mich ja darin irren. Jedenfalls erleichtert es die Situation nicht, denn ich
               habe Rittner\pwindex{Rittner, Rudolf 30.06.1869 – 04.02.1943@\textsc{Rittner, Rudolf} (30.06.1869 – 04.02.1943), \emph{Theaterleiter, Schauspieler}|pw} in dieser Rolle nicht gesehen.
               Wie immer er aber auch gewesen sein mag, er war sicherlich besser als Reicher\pwindex{Reicher, Emanuel 18.06.1849 – 15.05.1924@\textsc{Reicher, Emanuel} (18.06.1849 – 15.05.1924), \emph{Schauspieler}|pw}. Einfach aus dem Grund, weil es
               unmöglich ist, schlechter zu sein als Herr Reicher\pwindex{Reicher, Emanuel 18.06.1849 – 15.05.1924@\textsc{Reicher, Emanuel} (18.06.1849 – 15.05.1924), \emph{Schauspieler}|pw} war. (Dieser Satz könnte von Goldmann\pwindex{Goldmann, Paul 31.01.1865 – 25.09.1935@\textsc{Goldmann, Paul} (31.01.1865 – 25.09.1935), \emph{Schriftsteller, Journalist}|pw} sein; ist aber gleichwol richtig) Um Rittner ist doch stets ein
               Hauch von der Fülle der Erlebnisse. Auch ein leiser Hauch von Einsamkeit ist jetzt
               mehr und mehr um ihn. Rittner\pwindex{Rittner, Rudolf 30.06.1869 – 04.02.1943@\textsc{Rittner, Rudolf} (30.06.1869 – 04.02.1943), \emph{Theaterleiter, Schauspieler}|pw} ist doch auf
               eine glaubhafte Art von Verliebtheit umgeben, von allerlei Karessen, und das Parfum
               vieler Frauen haftet gleichsam in seinen Kleidern. Wenn nun alle diese Dinge welk und
               herbstlich werden, dann haben sie, wie es der Julian\pwindex{Schnitzler, Arthur 15.05.1862 – 21.10.1931@\textsc{Schnitzler, Arthur} (15.05.1862 – 21.10.1931), \emph{Schriftsteller, Mediziner}!einsame Weg. Schauspiel in fuenf Akten1904@\strich\emph{Der einsame Weg. Schauspiel in fünf Akten} {[}1904{]}|pwv} braucht{[},{]} jene Melancholie,
               deren besondere Schattierung eben ein Goldton ist, ein verblaßender, vormals aber –
               das sieht man noch genau – üppiger und leuchtender Goldton. Von solchen {\pb}Dingen ist bei Reicher\pwindex{Reicher, Emanuel 18.06.1849 – 15.05.1924@\textsc{Reicher, Emanuel} (18.06.1849 – 15.05.1924), \emph{Schauspieler}|pw} nichts zu spüren. Er ist ganz und gar bürgerlich. Hat
               leider den Moment versäumt, Kinder zu zeugen, mit denen er jetzt \label{K_L03421-3v}\edtext{Schabbes}{\lemma{\textnormal{\emph{Schabbes}}}\Cendnote{\textnormal{Sabbat}}}\label{K_L03421-3h} machen oder den \label{K_L03421-4v}\edtext{Seder-Abend}{\lemma{\textnormal{\emph{Seder-Abend}}}\Cendnote{\textnormal{Abendessen am Vorabend des Pessach-Festes}}}\label{K_L03421-4h} halten könnte. Mir wäre, wie ich
               gewiss nicht erst zu sagen brauche, auch der jüdische Julian\pwindex{Schnitzler, Arthur 15.05.1862 – 21.10.1931@\textsc{Schnitzler, Arthur} (15.05.1862 – 21.10.1931), \emph{Schriftsteller, Mediziner}!einsame Weg. Schauspiel in fuenf Akten1904@\strich\emph{Der einsame Weg. Schauspiel in fünf Akten} {[}1904{]}|pwv} recht, wenn es nur eben ein Julian\pwindex{Schnitzler, Arthur 15.05.1862 – 21.10.1931@\textsc{Schnitzler, Arthur} (15.05.1862 – 21.10.1931), \emph{Schriftsteller, Mediziner}!einsame Weg. Schauspiel in fuenf Akten1904@\strich\emph{Der einsame Weg. Schauspiel in fünf Akten} {[}1904{]}|pwv} wäre: etwa Adalbert Goldschmidt\pwindex{Goldschmidt, Adalbert von 1848-05-05 – 1906-12-21@\textsc{Goldschmidt, Adalbert von} (1848-05-05 – 1906-12-21), \emph{Schriftsteller, Komponist}|pw}, der ja den jüdischen und zugleich einen
                  Daudet\pwindex{Daudet, Alphonse 13.05.1840 – 16.11.1897@\textsc{Daudet, Alphonse} (13.05.1840 – 16.11.1897), \emph{Schriftsteller}|pw}’schen Einschlag hat. Allein Reicher\pwindex{Reicher, Emanuel 18.06.1849 – 15.05.1924@\textsc{Reicher, Emanuel} (18.06.1849 – 15.05.1924), \emph{Schauspieler}|pw} ist trocken, und erscheint höchstens
               als verkrachter Familienvater. – – –\pend
           \pstart
           Montag.\pend
           \pstart
           Gestern wurde ich durch Besuche (die Leute machen hier
               unaufhörlich Besuche) unterbrochen. Abends taf ich zufällig Rittner\pwindex{Rittner, Rudolf 30.06.1869 – 04.02.1943@\textsc{Rittner, Rudolf} (30.06.1869 – 04.02.1943), \emph{Theaterleiter, Schauspieler}|pw}. Er ist nicht abgeneigt, den Julian\pwindex{Schnitzler, Arthur 15.05.1862 – 21.10.1931@\textsc{Schnitzler, Arthur} (15.05.1862 – 21.10.1931), \emph{Schriftsteller, Mediziner}!einsame Weg. Schauspiel in fuenf Akten1904@\strich\emph{Der einsame Weg. Schauspiel in fünf Akten} {[}1904{]}|pwv} in Wien\oindex{Wien@\textbf{Wien}|pw} zu spielen. Oder genauer: »im Prinzip nicht
                  dagegen{[}«{]}. Als ich ihm sagte, \uline{Sie} hätten keineswegs darauf bestanden, dass er den \label{K_L03421-5v}\edtext{Forstadjunkten\pwindex{Schnitzler, Arthur 15.05.1862 – 21.10.1931@\textsc{Schnitzler, Arthur} (15.05.1862 – 21.10.1931), \emph{Schriftsteller, Mediziner}!Ruf des Lebens. Schauspiel in drei Akten1906-02-20@\strich\emph{Der Ruf des Lebens. Schauspiel in drei Akten} {[}1906-02-20{]}|pwv}}{\lemma{\textnormal{\emph{Forstadjunkten}}}\Cendnote{\textnormal{Bei der deutschsprachigen Uraufführung
                  von \emph{Der Ruf des Lebens}\pwindex{Schnitzler, Arthur 15.05.1862 – 21.10.1931@\textsc{Schnitzler, Arthur} (15.05.1862 – 21.10.1931), \emph{Schriftsteller, Mediziner}!Ruf des Lebens. Schauspiel in drei Akten1906-02-20@\strich\emph{Der Ruf des Lebens. Schauspiel in drei Akten} {[}1906-02-20{]}|pwk} am 24. 2. 1906 im Lessing-Theater\oindex{Lessing-Theater@\textbf{Lessing-Theater}|pwk} gab Rittner\pwindex{Rittner, Rudolf 30.06.1869 – 04.02.1943@\textsc{Rittner, Rudolf} (30.06.1869 – 04.02.1943), \emph{Theaterleiter, Schauspieler}|pwk} den Forstadjunkten Eduard Rainer\pwindex{Schnitzler, Arthur 15.05.1862 – 21.10.1931@\textsc{Schnitzler, Arthur} (15.05.1862 – 21.10.1931), \emph{Schriftsteller, Mediziner}!Ruf des Lebens. Schauspiel in drei Akten1906-02-20@\strich\emph{Der Ruf des Lebens. Schauspiel in drei Akten} {[}1906-02-20{]}|pwkv}.}}}\label{K_L03421-5h} gibt, und hätten
               ihm sein Versagen auch nicht übelgenommen, war er erfreut. Er meint nur, es wird für
                  Brahm\pwindex{Brahm, Otto 05.02.1856 – 28.11.1912@\textsc{Brahm, Otto} (05.02.1856 – 28.11.1912), \emph{Theaterleiter, Regisseur}|pw} schwer sein, Reicher\pwindex{Reicher, Emanuel 18.06.1849 – 15.05.1924@\textsc{Reicher, Emanuel} (18.06.1849 – 15.05.1924), \emph{Schauspieler}|pw} die Rolle abzunehmen, und die für Rittner\pwindex{Rittner, Rudolf 30.06.1869 – 04.02.1943@\textsc{Rittner, Rudolf} (30.06.1869 – 04.02.1943), \emph{Theaterleiter, Schauspieler}|pw} nötigen Proben abzuhalten. Außerdem wird Brahm\pwindex{Brahm, Otto 05.02.1856 – 28.11.1912@\textsc{Brahm, Otto} (05.02.1856 – 28.11.1912), \emph{Theaterleiter, Regisseur}|pw} es nicht gerne sehen, wenn Rittner\pwindex{Rittner, Rudolf 30.06.1869 – 04.02.1943@\textsc{Rittner, Rudolf} (30.06.1869 – 04.02.1943), \emph{Theaterleiter, Schauspieler}|pw} über seine Garantie kommt. Die beträgt
               für Wien\oindex{Wien@\textbf{Wien}|pw} 12 Abende, welche mit »Elga\pwindex{\textcolor{red}{\textsuperscript{XXXX1 indx}}!Elga1905@\strich\emph{Elga} {[}1905{]}|pw}« gedeckt scheinen. Ist er im »Einsamen Weg\pwindex{Schnitzler, Arthur 15.05.1862 – 21.10.1931@\textsc{Schnitzler, Arthur} (15.05.1862 – 21.10.1931), \emph{Schriftsteller, Mediziner}!einsame Weg. Schauspiel in fuenf Akten1904@\strich\emph{Der einsame Weg. Schauspiel in fünf Akten} {[}1904{]}|pw}« tätig, muß dann Brahm\pwindex{Brahm, Otto 05.02.1856 – 28.11.1912@\textsc{Brahm, Otto} (05.02.1856 – 28.11.1912), \emph{Theaterleiter, Regisseur}|pw} das Plus zahlen, was er – wie Sie wissen – überhaupt,
               und im Fall Rittner\pwindex{Rittner, Rudolf 30.06.1869 – 04.02.1943@\textsc{Rittner, Rudolf} (30.06.1869 – 04.02.1943), \emph{Theaterleiter, Schauspieler}|pw} erst recht lieber
               vermeidet.\pend
           \pstart
           Was soll ich, nach Ihrer Meinung, tun? Dass ich mit Vergnügen zu allem bereit bin,
               brauche ich nicht erst zu sagen. Erwägen Sie, was ich Ihnen wegen mir u. Brahm\pwindex{Brahm, Otto 05.02.1856 – 28.11.1912@\textsc{Brahm, Otto} (05.02.1856 – 28.11.1912), \emph{Theaterleiter, Regisseur}|pw} sagte, und denken Sie nach, wie man es
               machen könnte, dass ich bei Brahm\pwindex{Brahm, Otto 05.02.1856 – 28.11.1912@\textsc{Brahm, Otto} (05.02.1856 – 28.11.1912), \emph{Theaterleiter, Regisseur}|pw} nicht eine
               Unannehmlichkeit erfahre. Soll ich vielleicht Elias\pwindex{Elias, Julius 12.07.1861 – 02.07.1927@\textsc{Elias, Julius} (12.07.1861 – 02.07.1927), \emph{Übersetzer, Publizist}|pw} zu ihm schicken? Das will ich auf alle Fälle gleich tun.\pend
           \pstart
           Eben kommt wieder Besuch. (Die Leute machen hier unaufhörlich Besuche) Ich will aber,
               dass der Brief heute abgeht. Also viele herzlichste
               Grüße von uns\pwindex{Salten, Ottilie 07.03.1868 – 22.06.1942@\textsc{Salten, Ottilie} (07.03.1868 – 22.06.1942), \emph{Schauspielerin}|pwv} an Sie Beide\pwindex{Schnitzler, Olga 17.01.1882 – 13.01.1970@\textsc{Schnitzler, Olga} (17.01.1882 – 13.01.1970), \emph{Schauspielerin, Sängerin}|pwv}. {\\}Ihr
                  \spacefill\mbox{Salten}\pend
           \pstart
           \noindent{}\label{K_L03421-7v}\edtext{\textcolor{gray}{N}B.}{\lemma{\textnormal{\emph{NB.}}}\Cendnote{\textnormal{nota bene; lateinisch: merke
                     wohl}}}\label{K_L03421-7h}{ }Jacobsohn\pwindex{Jacobsohn, Siegfried 28.01.1881 – 03.12.1926@\textsc{Jacobsohn, Siegfried} (28.01.1881 – 03.12.1926), \emph{Journalist, Kritiker, Publizist}|pw}{ }\label{K_L03421-6v}\edtext{tobt\pwindex{Jacobsohn, Siegfried 28.01.1881 – 03.12.1926@\textsc{Jacobsohn, Siegfried} (28.01.1881 – 03.12.1926), \emph{Journalist, Kritiker, Publizist}!einsame Weg1906-04-26@\strich\emph{Der einsame Weg} {[}1906-04-26{]}|pwv}}{\lemma{\textnormal{\emph{tobt}}}\Cendnote{\textnormal{[Siegfried Jacobsohn\pwindex{Jacobsohn, Siegfried 28.01.1881 – 03.12.1926@\textsc{Jacobsohn, Siegfried} (28.01.1881 – 03.12.1926), \emph{Journalist, Kritiker, Publizist}|pwk}]: \emph{Der einsame Weg}\pwindex{Jacobsohn, Siegfried 28.01.1881 – 03.12.1926@\textsc{Jacobsohn, Siegfried} (28.01.1881 – 03.12.1926), \emph{Journalist, Kritiker, Publizist}!einsame Weg1906-04-26@\strich\emph{Der einsame Weg} {[}1906-04-26{]}|pwk}. In: \emph{Die Schaubühne}\pwindex{Schaubuehne7.9.1905 – 1993@\emph{Die Schaubühne} {[}7.9.1905 – 1993{]}|pwk}, Jg. 2, Nr. 17, 26. 4. 1906, S. 487–491.}}}\label{K_L03421-6h} ja auch
                  gegen Reicher\pwindex{Reicher, Emanuel 18.06.1849 – 15.05.1924@\textsc{Reicher, Emanuel} (18.06.1849 – 15.05.1924), \emph{Schauspieler}|pw}!\pend
           
         
         \endnumbering\mylabel{h}\end{ledgroupsized}  \newcommand{\dateiname}{L03421}\newcommand{\titel}{Felix Salten an Arthur Schnitzler, 22.–23. 4. 1906}\newcommand{\editorInnen}{Martin Anton Müller und Laura Untner}%% latex-leseansicht-abspann.tex
%% Abspann für die Leseansicht.
%% Der Schalter \ifkorrekturansicht ist bereits durch den Vorspann gesetzt.

%% latex-abspann.tex
%% Gemeinsamer Abspann für Korrekturansicht und Leseansicht.
%% Setzt den Schalter \ifkorrekturansicht voraus (gesetzt in den
%% einbindenden Dateien latex-korrekturansicht-abspann.tex bzw.
%% latex-leseansicht-abspann.tex).
%% ---------------------------------------------------------------

\normalsize

% Das esempio-Environment wird nur in der Leseansicht benötigt
\ifkorrekturansicht\else
\newenvironment{esempio}[3]%
{
    \vspace{1.5ex}
    \rlap{\underline{#1}}
    \par
    \setlength{\parindent}{0cm}
    \nopagebreak
    \leftskip=#2cm
    \rightskip=#3cm
}
{
    \par
}
\fi

\doendnotes{C}
\bigskip
\vfill

\clearpage

\footnotesize

\ifkorrekturansicht
  \lohead{\textsc{register}}
\fi

% theindex-Environment neu definieren ohne reledmac
\makeatletter
\renewenvironment{theindex}{%
  \ifkorrekturansicht
    \section*{\indexname}%
  \else
    \subsubsection*{Index der erwähnten Entitäten}%
  \fi
  \setlength{\parindent}{0pt}%
  \setlength{\parskip}{0pt plus 0.3pt}%
  \let\item\@idxitem
}{%
  \ifkorrekturansicht\clearpage\fi
}
\makeatother

\IfFileExists{\jobname-pw.ind}{\input{\jobname-pw.ind}}{}

% Quellenangabe nur in der Leseansicht
\ifkorrekturansicht\else
% Fallback-Definitionen, falls die .tex-Datei \titel etc. nicht gesetzt hat
\providecommand{\titel}{}
\providecommand{\editorInnen}{}
\providecommand{\dateiname}{\jobname}

\vspace{3cm}

\vfill

\footnotesize
\textsc{Quelle}: \titel. Herausgegeben von {\editorInnen}. In: \emph{Arthur Schnitzler: Briefwechsel mit Autorinnen und Autoren}.
 Digitale Edition, https://schnitzler-briefe.acdh.oeaw.ac.at/{\dateiname}.html (Stand \today)
\fi

\end{document}


      