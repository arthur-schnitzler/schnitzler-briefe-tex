%% latex-leseansicht-vorspann.tex
%% Vorspann für die Leseansicht.
%% Lädt die gemeinsame Datei latex-vorspann.tex mit nicht gesetztem Schalter.

\newif\ifkorrekturansicht
\korrekturansichtfalse

\input{../tex-inputs/latex-vorspann}


         
         \renewcommand{\erwaehntePersonen}{Personen: Hugo von Hofmannsthal}
         \renewcommand{\erwaehnteInstitutionen}{Institutionen: S. Fischer Verlag, Theater in der Josefstadt, Zentralstelle der Bühnen-Autoren und -Verleger}
         \renewcommand{\erwaehnteOrte}{Orte: Deutschland, Salzburg, Stallburggasse, Tschechoslowakei, Wien, Österreich}
         \renewcommand{\erwaehnteWerke}{Werke: Der Schwierige. Lustspiel in drei Akten, Jedermann. Das Spiel vom Sterben des reichen Mannes}
               \section[Gerty von Hofmannsthal an Arthur Schnitzler, 23. 11. 1929]{ Gerty von Hofmannsthal an Arthur Schnitzler, 23. 11. 1929}\nopagebreak\mylabel{v}\rehead{ }\begin{ledgroupsized}[t]{13cm}\normalsize\beginnumbering \toendnotes[C]{\smallbreak\pagebreak[2]} \Standort{CUL, Schnitzler, B 43.}
\physDesc{Brief, 1 Blatt, 1 Seite, 1477 Zeichen
\newline{}Schreibmaschine
\newline{}Handschrift: schwarze Tinte, lateinische Kurrent (\noindent{}Angabe der Straße, Unterschrift)
\newline{}Schnitzler: mit rotem Buntstift beschriftet: »\textsc{Hofm}« und zwei Unterstreichungen vorgenommen }\toendnotes[C]{\smallbreak}\pstart
           \raggedleft{}{\pb}Wien\oindex{Wien@\textbf{Wien}|pw} d 23/IX 29\pend
           \pstart
           \raggedleft{}{[}hs.:{]} I Stallburggasse 2\oindex{Stallburggasse@\textbf{Stallburggasse}|pw}\pend
           \pstart
           {[}ms.:{]} Lieber Arthur, darf ich Sie heute um einen Rat fragen in einer
               geschäftlichen Angelegenheit: Die Zentralstelle der
                  Bühnenautoren und Verleger\orgindex{Zentralstelle der Buehnen-Autoren und -Verleger@Zentralstelle der Bühnen-Autoren und -Verleger|pw} reclamiert eine 3{\%}tige
               Tantiemenabgabe aus Eingängen aus Oest{[}e{]}rreich\oindex{Oesterreich@\textbf{Österreich}|pw} und C.S.R.\oindex{Tschechoslowakei@\textbf{Tschechoslowakei}|pw}\pend
           \pstart
           Ich weiss dass auch Hugo\pwindex{Hofmannsthal, Hugo von 1874-02-01 – 1929-07-15@\textsc{Hofmannsthal, Hugo von} (1874-02-01 – 1929-07-15), \emph{Schriftsteller}|pw} dies tat wenn es sich
               um ein Werk wie Jedermann\pwindex{Hofmannsthal, Hugo von 1874-02-01 – 1929-07-15@\textsc{Hofmannsthal, Hugo von} (1874-02-01 – 1929-07-15), \emph{Schriftsteller}!Jedermann. Das Spiel vom Sterben des reichen Mannes1911@\strich\emph{Jedermann. Das Spiel vom Sterben des reichen Mannes} {[}1911{]}|pw} gehandelt hat welches
               er für Oesterreich\oindex{Oesterreich@\textbf{Österreich}|pw} selbst zum Vertrieb hatte
               und ich weiss auch dass er voriges Jahr im Mai für die Aufführungen in
                  Salzburg\oindex{Salzburg@\textbf{Salzburg}|pw} dem Verein 120 \label{K_L02525_1v}\edtext{S.}{\lemma{\textnormal{\emph{S.}}}\Cendnote{\textnormal{Schilling}}}\label{K_L02525_1h} anwies (was unter uns gesagt keine 3{\%} der Einnahmen war) Da die heurigen Einnahmen doch eine
               ziemliche Höhe hatten und auch die Josefstadt\orgindex{Theater in der Josefstadt@Theater in der Josefstadt|pw} den
                  Schwierigen\pwindex{Hofmannsthal, Hugo von 1874-02-01 – 1929-07-15@\textsc{Hofmannsthal, Hugo von} (1874-02-01 – 1929-07-15), \emph{Schriftsteller}!Schwierige. Lustspiel in drei Akten1921@\strich\emph{Der Schwierige. Lustspiel in drei Akten} {[}1921{]}|pw} direct mit mir abrechnete so
               wären 3{\%}{ }\so{ehrlich} abgerechnet doch ganz viel.\pend
           \pstart
           Nun habe ich bei Fischer\orgindex{S. Fischer Verlag@S. Fischer Verlag|pw} nachgesehen und gesehen
               dass er in Deutschland\oindex{Deutschland@\textbf{Deutschland}|pw} immer 2{\%} bei den \label{T_L02525_1v}\edtext{Abrechnungen}{\lemma{\textnormal{\emph{Abrechnungen}}}\Cendnote{\textnormal{Sie schreibt:
                     »Abrechrenungen«}}}\label{T_L02525_1h} abzieht. Warum also 3{\%} hier? Ferner ob Sie glauben dass ich nach unten abrunden
               kann in der Berechnung, oder ob der Verein\orgindex{Zentralstelle der Buehnen-Autoren und -Verleger@Zentralstelle der Bühnen-Autoren und -Verleger|pwv} das Recht hat nachzuforschen wie viel tatsächlich die Einnahmen
               waren. Ich verstehe ja gar nicht die Rechte, die dieser Verein\orgindex{Zentralstelle der Buehnen-Autoren und -Verleger@Zentralstelle der Bühnen-Autoren und -Verleger|pwv} hat, und welche Vorteile man
               wiederum hat wenn man ihm angehört – aber vielleicht muss das eben sein, sonst würde
               Fischer ja auch nicht die Percente gleich automatisch zahlen.\pend
           \pstart
           Also meine Frage: muss ich \so{ehrlich} sein?\pend
           \pstart
           2/ ist 3{\%} berechtigt?\pend
           \pstart
           Ich schreibe dies, weil mein Telephon so schnell abschnappt. Aber wenn Sie so lieb
               sind mich anzurufen und mir die Antwort sagen R 23757, (am besten zwischen
                  ½10–11), so wäre ich sehr dankbar\pend
           \pstart
           \label{T_L02525_2v}\edtext{Herzlichst}{\lemma{\textnormal{\emph{Herzlichst}}}\Cendnote{\textnormal{Sie schreibt: »Herzlchst«}}}\label{T_L02525_2h}{\\[\baselineskip]}Ihre{\\[\baselineskip]}\spacefill\mbox{{[}hs.:{]} Gerty}\pend
           \leftskip=0em{}
         
         \endnumbering\mylabel{h}\end{ledgroupsized}  \newcommand{\dateiname}{L02525}\newcommand{\titel}{Gerty von Hofmannsthal an Arthur Schnitzler, 23. 11. 1929}\newcommand{\editorInnen}{Martin Anton Müller und Gerd-Hermann Susen}%% latex-leseansicht-abspann.tex
%% Abspann für die Leseansicht.
%% Der Schalter \ifkorrekturansicht ist bereits durch den Vorspann gesetzt.

%% latex-abspann.tex
%% Gemeinsamer Abspann für Korrekturansicht und Leseansicht.
%% Setzt den Schalter \ifkorrekturansicht voraus (gesetzt in den
%% einbindenden Dateien latex-korrekturansicht-abspann.tex bzw.
%% latex-leseansicht-abspann.tex).
%% ---------------------------------------------------------------

\normalsize

% Das esempio-Environment wird nur in der Leseansicht benötigt
\ifkorrekturansicht\else
\newenvironment{esempio}[3]%
{
    \vspace{1.5ex}
    \rlap{\underline{#1}}
    \par
    \setlength{\parindent}{0cm}
    \nopagebreak
    \leftskip=#2cm
    \rightskip=#3cm
}
{
    \par
}
\fi

\doendnotes{C}
\bigskip
\vfill

\clearpage

\footnotesize

\ifkorrekturansicht
  \lohead{\textsc{register}}
\fi

% theindex-Environment neu definieren ohne reledmac
\makeatletter
\renewenvironment{theindex}{%
  \ifkorrekturansicht
    \section*{\indexname}%
  \else
    \subsubsection*{Index der erwähnten Entitäten}%
  \fi
  \setlength{\parindent}{0pt}%
  \setlength{\parskip}{0pt plus 0.3pt}%
  \let\item\@idxitem
}{%
  \ifkorrekturansicht\clearpage\fi
}
\makeatother

\IfFileExists{\jobname-pw.ind}{\input{\jobname-pw.ind}}{}

% Quellenangabe nur in der Leseansicht
\ifkorrekturansicht\else
% Fallback-Definitionen, falls die .tex-Datei \titel etc. nicht gesetzt hat
\providecommand{\titel}{}
\providecommand{\editorInnen}{}
\providecommand{\dateiname}{\jobname}

\vspace{3cm}

\vfill

\footnotesize
\textsc{Quelle}: \titel. Herausgegeben von {\editorInnen}. In: \emph{Arthur Schnitzler: Briefwechsel mit Autorinnen und Autoren}.
 Digitale Edition, https://schnitzler-briefe.acdh.oeaw.ac.at/{\dateiname}.html (Stand \today)
\fi

\end{document}


      