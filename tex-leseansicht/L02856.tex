%% latex-leseansicht-vorspann.tex
%% Vorspann für die Leseansicht.
%% Lädt die gemeinsame Datei latex-vorspann.tex mit nicht gesetztem Schalter.

\newif\ifkorrekturansicht
\korrekturansichtfalse

\input{../tex-inputs/latex-vorspann}


         
         \renewcommand{\erwaehntePersonen}{Personen: Richard Beer-Hofmann, Paul Goldmann, Leo Van-Jung}
         \renewcommand{\erwaehnteInstitutionen}{Institutionen: Weinlaube}
         \renewcommand{\erwaehnteOrte}{Orte: Baby Tower Hill, Beach Hotel, Frankgasse, Hong Kong, Klosterneuburg, Wien, Yantai, Österreich}
         \renewcommand{\erwaehnteWerke}{}
               \section[ Paul Goldmann an Arthur Schnitzler, 24. 8. 1898]{ Paul Goldmann an Arthur Schnitzler, 24. 8. 1898}\nopagebreak\mylabel{v}\rehead{ }\begin{ledgroupsized}[t]{13cm}\normalsize\beginnumbering\briefempfaengerindex{Schnitzler, Arthur@\textsc{Schnitzler, Arthur}!zzzGoldmann, Paul@\emph{von Paul Goldmann}!1898-08-242@{24. 8. 1898}|(be} \toendnotes[C]{\smallbreak\pagebreak[2]} \Standort{DLA, A:Schnitzler, HS.NZ85.1.3168.}
\physDesc{Bildpostkarte, 239 Zeichen
\newline{}Handschrift: 1) blaue Tinte, deutsche Kurrent\hspace{1em}2) blaue Tinte, lateinische Kurrent (\noindent{}Adresse)\hspace{1em}
\newline{}Versand: 1) Stempel: »\nobreak{}\oindex{Yantai@\textbf{Yantai}|pwk}Chefoo, 25 Aug 98\nobreak{}«.   2) Stempel: »\nobreak{}\oindex{Hong Kong@\textbf{Hong Kong}|pwk}{[}Hong Ko{]}ng, Sp 4 98\nobreak{}«.  3) Stempel: »\nobreak{}\oindex{XXXX Ortsangabe fehlt|pwk}Wien 9/3 72, 8. 10. 98, 1. N, Bestellt\nobreak{}«. 
\newline{}Schnitzler: mit Bleistift das Jahr »98« vermerkt }\toendnotes[C]{\smallbreak}\pstart{}{\pb}\begin{otherlanguage}{english}Aust{[}ria{]}\oindex{Oesterreich@\textbf{Österreich}|pw}\end{otherlanguage}\pend{}\pstart{}Herrn Dr. Arthur Schnitzler\pend{}\pstart{}Wien\oindex{Wien@\textbf{Wien}|pw}\pend{}\pstart{}IX. Frankgaſse 1\oindex{XXXX Ortsangabe fehlt|pw}.\pend{}{\bigskip}\pstart
           \noindent{}\centering{}{\pb}\textcolor{gray}{\textbf{Chefoo China\oindex{Yantai@\textbf{Yantai}|pw}.}}\pend
           \pstart
           \noindent{}\textcolor{gray}{\textbf{Beach Hotel\oindex{Beach Hotel@\textbf{Beach Hotel}|pw}.}}\pend
           \pstart
           \textcolor{gray}{\textbf{Baby Tower Hill\oindex{Baby Tower Hill@\textbf{Baby Tower Hill}|pw}.}}\pend
           \pstart
           \raggedleft{}24. Auguſt.\pend
           \pstart{}Mein lieber Freund,\pend\pstart
           Ich hoffe, Du biſt von Deiner \label{K_L02856-1v}\edtext{Reiſe}{\lemma{\textnormal{\emph{Reiſe}}}\Cendnote{\textnormal{Siehe Paul Goldmann an Arthur Schnitzler, 16. 5. 1898.
               }}}\label{K_L02856-1h} geſund zurückgekehrt. Ich wünſchte, daß ich auch ſchon wieder daheim wäre!
               Viele Grüße Dir, \textsc{Richard\pwindex{Beer-Hofmann, Richard 1866-07-11 – 1945-09-26@\textsc{Beer-Hofmann, Richard} (1866-07-11 – 1945-09-26), \emph{Schriftsteller}|pw}} u. \textsc{Leo\pwindex{Van-Jung, Leo 15.10.1866 – 02.07.1939@\textsc{Van-Jung, Leo} (15.10.1866 – 02.07.1939), \emph{Gesangspädagoge, Mathematiker}|pw}}!\pend
           \pstart
           Dein {\\[\baselineskip]}\spacefill\mbox{Paul Goldmnn}\pend
           \leftskip=0em{}\pstart
           \noindent{}\textcolor{gray}{\textbf{Weinlaube\orgindex{Weinlaube@Weinlaube|pw}, Klosterneuburg\oindex{Klosterneuburg@\textbf{Klosterneuburg}|pw}.}}\pend
           
         
         \endnumbering\mylabel{h}\end{ledgroupsized}  \newcommand{\dateiname}{L02856}\newcommand{\titel}{Paul Goldmann an Arthur Schnitzler, 24. 8. 1898}\newcommand{\editorInnen}{Martin Anton Müller und Laura Untner}%% latex-leseansicht-abspann.tex
%% Abspann für die Leseansicht.
%% Der Schalter \ifkorrekturansicht ist bereits durch den Vorspann gesetzt.

%% latex-abspann.tex
%% Gemeinsamer Abspann für Korrekturansicht und Leseansicht.
%% Setzt den Schalter \ifkorrekturansicht voraus (gesetzt in den
%% einbindenden Dateien latex-korrekturansicht-abspann.tex bzw.
%% latex-leseansicht-abspann.tex).
%% ---------------------------------------------------------------

\normalsize

% Das esempio-Environment wird nur in der Leseansicht benötigt
\ifkorrekturansicht\else
\newenvironment{esempio}[3]%
{
    \vspace{1.5ex}
    \rlap{\underline{#1}}
    \par
    \setlength{\parindent}{0cm}
    \nopagebreak
    \leftskip=#2cm
    \rightskip=#3cm
}
{
    \par
}
\fi

\doendnotes{C}
\bigskip
\vfill

\clearpage

\footnotesize

\ifkorrekturansicht
  \lohead{\textsc{register}}
\fi

% theindex-Environment neu definieren ohne reledmac
\makeatletter
\renewenvironment{theindex}{%
  \ifkorrekturansicht
    \section*{\indexname}%
  \else
    \subsubsection*{Index der erwähnten Entitäten}%
  \fi
  \setlength{\parindent}{0pt}%
  \setlength{\parskip}{0pt plus 0.3pt}%
  \let\item\@idxitem
}{%
  \ifkorrekturansicht\clearpage\fi
}
\makeatother

\IfFileExists{\jobname-pw.ind}{\input{\jobname-pw.ind}}{}

% Quellenangabe nur in der Leseansicht
\ifkorrekturansicht\else
% Fallback-Definitionen, falls die .tex-Datei \titel etc. nicht gesetzt hat
\providecommand{\titel}{}
\providecommand{\editorInnen}{}
\providecommand{\dateiname}{\jobname}

\vspace{3cm}

\vfill

\footnotesize
\textsc{Quelle}: \titel. Herausgegeben von {\editorInnen}. In: \emph{Arthur Schnitzler: Briefwechsel mit Autorinnen und Autoren}.
 Digitale Edition, https://schnitzler-briefe.acdh.oeaw.ac.at/{\dateiname}.html (Stand \today)
\fi

\end{document}


      