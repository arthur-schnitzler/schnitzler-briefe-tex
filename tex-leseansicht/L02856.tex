%% latex-korrekturansicht-vorspann.tex
%% Vorspann für die Korrekturansicht.
%% Lädt die gemeinsame Datei latex-vorspann.tex mit gesetztem Schalter.

\newif\ifkorrekturansicht
\korrekturansichttrue

\input{../tex-inputs/latex-vorspann}


\section[ Paul Goldmann an Arthur Schnitzler, 24. 8. 1898]{L02856 Paul Goldmann an Arthur Schnitzler, 24. 8. 1898}
\nopagebreak\mylabel{L02856v}
\rehead{ }\normalsize\beginnumbering\briefempfaengerindex{Schnitzler, Arthur@\textsc{Schnitzler, Arthur}!zzzGoldmann, Paul@\emph{von Paul Goldmann}!1898-08-242@{24. 8. 1898}|(be}
\toendnotes[C]{\smallbreak\pagebreak[2]}\Standort{DLA, A:Schnitzler, HS.NZ85.1.3168.}
\physDesc{Bildpostkarte, 239 Zeichen
\newline{}Handschrift: 1) blaue Tinte, deutsche Kurrent\hspace{1em}2) blaue Tinte, lateinische Kurrent (\noindent{}Adresse)\hspace{1em}
\newline{}Versand: 1) Stempel: »\nobreak{}\oindex{Yantai@\textbf{Yantai}, \emph{Besiedelter Ort (A.BSO)}|pwk}Chefoo, 25 Aug 98\nobreak{}«.   2) Stempel: »\nobreak{}\oindex{Hong Kong@\textbf{Hong Kong}, \emph{P.PPLC}|pwk}{[}Hong Ko{]}ng, Sp 4 98\nobreak{}«.  3) Stempel: »\nobreak{}\oindex{IX., Alsergrund@\textbf{IX., Alsergrund}, \emph{A.ADM3}|pwk}Wien 9/3 72, 8. 10. 98, 1. N, Bestellt\nobreak{}«. 
\newline{}Schnitzler: mit Bleistift das Jahr »98« vermerkt }\toendnotes[C]{\smallbreak}\pstart{}{\pb}\begin{otherlanguage}{english}Aust{[}ria{]}\oindex{Oesterreich@\textbf{Österreich}, \emph{A.PCLI}|pw}\end{otherlanguage}\pend{}\pstart{}Herrn Dr. Arthur Schnitzler\pend{}\pstart{}Wien\oindex{Wien@\textbf{Wien}, \emph{A.ADM2}|pw}\pend{}\pstart{}IX. Frankgaſse 1\oindex{Frankgasse 1@\textbf{Frankgasse 1}, \emph{Wohngebäude (K.WHS)}|pw}.\pend{}{\bigskip}
\pstart
           \noindent{}\centering{}{\pb}\textcolor{gray}{\textbf{Chefoo China\oindex{Yantai@\textbf{Yantai}, \emph{Besiedelter Ort (A.BSO)}|pw}.}}\pend
           
\pstart
           \textcolor{gray}{\textbf{Beach Hotel\oindex{Beach Hotel@\textbf{Beach Hotel}, \emph{Hotel (K.HTL)}|pw}.}}\pend
           
\pstart
           \textcolor{gray}{\textbf{Baby Tower Hill\oindex{Baby Tower Hill@\textbf{Baby Tower Hill}, \emph{Erhebung}|pw}.}}\pend
           \vspace{1em}
\pstart
           \raggedleft{}{\pb}24. Auguſt.\pend
           
\pstart{}Mein lieber Freund,\pend\vspace{0.5em}
\pstart
           Ich hoffe, Du biſt von Deiner \label{K_L02856-1v}\edtext{Reiſe}{\lemma{\textnormal{\emph{Reiſe}}}\Cendnote{\textnormal{Siehe Paul Goldmann an Arthur Schnitzler, 16. 5. 1898.
               }}}\label{K_L02856-1} geſund zurückgekehrt. Ich wünſchte, daß ich auch ſchon wieder daheim wäre!
               Viele Grüße Dir, \textsc{Richard\pwindex{Beer-Hofmann, Richard 1866-07-11 – 1945-09-26@\textsc{Beer-Hofmann, Richard} (1866-07-11 – 1945-09-26), \emph{Schriftsteller/Schriftstellerin}|pw}} u. \textsc{Leo\pwindex{Van-Jung, Leo 15.10.1866 – 02.07.1939@\textsc{Van-Jung, Leo} (15.10.1866 – 02.07.1939), \emph{Gesangspädagoge/Gesangspädagogin, Mathematiker/Mathematikerin}|pw}}!\pend
           
\pstart
           Dein {\\[\baselineskip]}\spacefill\mbox{Paul Goldmnn}\pend
           \leftskip=0em{}
\pstart
           \noindent{}\textcolor{gray}{\textbf{Weinlaube\orgindex{Weinlaube@Weinlaube|pw}, Klosterneuburg\oindex{Klosterneuburg@\textbf{Klosterneuburg}, \emph{P.PPLA3}|pw}.}}\pend
           \selectlanguage{ngerman}\endnumbering\briefempfaengerindex{Schnitzler, Arthur@\textsc{Schnitzler, Arthur}!zzzGoldmann, Paul@\emph{von Paul Goldmann}!1898-08-242@{24. 8. 1898}|)be}\mylabel{L02856h}  \normalsize

\doendnotes{C}
\bigskip
\vfill

\clearpage

\footnotesize

\lohead{\textsc{register}}

% Definiere theindex-Environment komplett neu ohne reledmac
\makeatletter
\renewenvironment{theindex}{%
  \section*{\indexname}%
  \setlength{\parindent}{0pt}%
  \setlength{\parskip}{0pt plus 0.3pt}%
  \let\item\@idxitem
}{%
  \clearpage
}
\makeatother

\IfFileExists{\jobname-pw.ind}{\input{\jobname-pw.ind}}{}

\end{document}

      