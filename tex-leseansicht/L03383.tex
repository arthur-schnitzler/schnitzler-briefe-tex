%% latex-korrekturansicht-vorspann.tex
%% Vorspann für die Korrekturansicht.
%% Lädt die gemeinsame Datei latex-vorspann.tex mit gesetztem Schalter.

\newif\ifkorrekturansicht
\korrekturansichttrue

\input{../tex-inputs/latex-vorspann}


\section[ Paul Goldmann an Arthur Schnitzler, 11. 8. 1903]{L03383 Paul Goldmann an Arthur Schnitzler, 11. 8. 1903}
\nopagebreak\mylabel{L03383v}
\rehead{ }\normalsize\beginnumbering\briefempfaengerindex{Schnitzler, Arthur@\textsc{Schnitzler, Arthur}!zzzGoldmann, Paul@\emph{von Paul Goldmann}!1903-08-112@{11. 8. 1903}|(be}
\toendnotes[C]{\smallbreak\pagebreak[2]}\Standort{DLA, A:Schnitzler, HS.NZ85.1.3173.}
\physDesc{Postkarte, 379 Zeichen
\newline{}Handschrift: 1) blaue Tinte, deutsche Kurrent\hspace{1em}2) blaue Tinte, lateinische Kurrent (\noindent{}Adresse)\hspace{1em}
\newline{}Versand: Stempel: »\nobreak{}Wien 1/1 15 r, 11 VIII 03, 6 30N\nobreak{}«. Stempel: »\nobreak{}\oindex{IX., Alsergrund@\textbf{IX., Alsergrund}, \emph{A.ADM3}|pwk}Wien 9/2 71 r, 11 VIII 03, 7 20N\nobreak{}«.  
\newline{}Schnitzler: mit Bleistift datiert: »11. 8 903.« }\toendnotes[C]{\smallbreak}\pstart{}{\pb}Herrn\pend{}\pstart{}Dr. Arthur Schnitzler\pend{}\pstart{}IX. Frankgaſse 1\oindex{Frankgasse 1@\textbf{Frankgasse 1}, \emph{Wohngebäude (K.WHS)}|pw}\pend{}\pstart{}Wien\oindex{Wien@\textbf{Wien}, \emph{A.ADM2}|pw}\pend{}{\bigskip}\vspace{1em}
\pstart
           {\pb}Dienſtag{ }Abend.\pend
           
\pstart{}Mein lieber Freund,\pend\vspace{0.5em}
\pstart
           Ich \label{K_L03383-1v}\edtext{fahre}{\lemma{\textnormal{\emph{fahre}}}\Cendnote{\textnormal{Siehe Paul Goldmann an Arthur Schnitzler, 27. 6. [1903].
               }}}\label{K_L03383-1}{ }heut{ }Abend, habe morgen{ }Nachmittag in Franzensfeſte\oindex{Franzensfeste@\textbf{Franzensfeste}, \emph{A.ADM3}|pw}{ }Rendezvous\pwindex{Rottenberg, Theodore 1875-09-07 – 1945-04-05@\textsc{Rottenberg, Theodore} (1875-09-07 – 1945-04-05)|pwv}, dürfte morgen{ }Abend in \textsc{Klausen\oindex{Klausen [Suedtirol]@\textbf{Klausen [Südtirol]}, \emph{Besiedelter Ort (A.BSO)}|pw}} (\textsc{Hotel Lamm\oindex{Gasthof zum Lamm@\textbf{Gasthof zum Lamm}, \emph{Gastgewerbegebäude (K.GGW)}|pw}}) ſein. Donnerſtag ſende ich Dir Nachricht nach
                  \textsc{Bozen\oindex{Bozen@\textbf{Bozen}, \emph{P.PPLA2}|pw}}. Du kannft mir bis Donnerſtag{ }früh Nachricht nach \textsc{Klausen\oindex{Klausen [Suedtirol]@\textbf{Klausen [Südtirol]}, \emph{Besiedelter Ort (A.BSO)}|pw}} ſenden, wenn Du willſt.\pend
           
\pstart
           Leb’ wohl! Glückliche Reiſe! Tauſend Dank! Grüße! {\\[\baselineskip]}Dein \spacefill\mbox{Paul
                  Goldm}\pend
           \leftskip=0em{}\selectlanguage{ngerman}\endnumbering\briefempfaengerindex{Schnitzler, Arthur@\textsc{Schnitzler, Arthur}!zzzGoldmann, Paul@\emph{von Paul Goldmann}!1903-08-112@{11. 8. 1903}|)be}\mylabel{L03383h}  \normalsize

\doendnotes{C}
\bigskip
\vfill

\clearpage

\footnotesize

\lohead{\textsc{register}}

% Definiere theindex-Environment komplett neu ohne reledmac
\makeatletter
\renewenvironment{theindex}{%
  \section*{\indexname}%
  \setlength{\parindent}{0pt}%
  \setlength{\parskip}{0pt plus 0.3pt}%
  \let\item\@idxitem
}{%
  \clearpage
}
\makeatother

\IfFileExists{\jobname-pw.ind}{\input{\jobname-pw.ind}}{}

\end{document}

      