%% latex-korrekturansicht-vorspann.tex
%% Vorspann für die Korrekturansicht.
%% Lädt die gemeinsame Datei latex-vorspann.tex mit gesetztem Schalter.

\newif\ifkorrekturansicht
\korrekturansichttrue

\input{../tex-inputs/latex-vorspann}


\section[ Felix Salten an Arthur Schnitzler, {[}zwischen 12. – 29. 2. 1896{]}]{L03168 Felix Salten an Arthur Schnitzler, {[}zwischen 12. – 29. 2. 1896{]}}
\nopagebreak\mylabel{L03168v}
\rehead{ }\normalsize\beginnumbering\briefempfaengerindex{Schnitzler, Arthur@\textsc{Schnitzler, Arthur}!zzzSalten, Felix@\emph{von Felix Salten}!1896-02-291@{{[}zwischen 12. – 29. 2. 1896{]}}|(be}
\toendnotes[C]{\smallbreak\pagebreak[2]}\Standort{CUL, Schnitzler, B 89, A 1.}
\physDesc{Brief, 1 Blatt, 2 Seiten, 737 Zeichen
\newline{}Handschrift: Bleistift, lateinische Kurrent
\newline{}Schnitzler: mit Bleistift datiert: »Feber 96.« 
\newline{}Ordnung: mit Bleistift von unbekannter Hand nummeriert: »68a« }\toendnotes[C]{\smallbreak}
\pstart
           \raggedleft{}{\pb}½ 3 Uhr\pend
           \vspace{0.5em}
\pstart
           Lieber Arthur! verzeihen Sie, dass ich Sie \label{K_L03168-1v}\edtext{wecken}{\lemma{\textnormal{\emph{wecken}}}\Cendnote{\textnormal{Dadurch
                  wird Schnitzlers Datierung auf »Feber 96« näher einschränkbar: Schnitzler
                  kam erst am 11. 2. 1896 aus Berlin\oindex{Berlin@\textbf{Berlin}, \emph{P.PPLC}|pwk} zurück,
                  hätte also davor nicht so schnell handeln können.}}}\label{K_L03168-1} laße. Aber ich fand heute{ }Nachts, als ich nach Hause kam{[},{]} den \label{K_L03168-2v}\edtext{inliegenden Brief}{\lemma{\textnormal{\emph{inliegenden Brief}}}\Cendnote{\textnormal{Beilage nicht erhalten}}}\label{K_L03168-2}. Lesen Sie ihn, – er erklärt
               Ihnen die Situation, und helfen Sie mir. Ich kann Ihnen sagen, dass es Niemanden
               gibt, den ich um diese Stunde um \uline{das} bitten könnte.
               Ich gebe Ihnen mein Wort, dass Sie die Hälfte bis \substVorne{}\textsuperscript{\textcolor{gray}{×}}\substDazwischen{}3\substHinten{} Uhr Nachmittags zurück haben, und die andere Hälfte bis
                  Dienstag{ }{\pb}\uline{um 5 Uhr}. Ich sage nichts weiter dazu. Wenn Sie den inliegenden Brief gelesen haben,
               werden Sie begreifen, wie mir zu \substVorne{}\textsuperscript{m}\substDazwischen{}M\substHinten{}uthe ist, und ich hoffe, Sie zweifeln \uline{gewiss}{ }\uline{nicht} daran, dass ich Ihnen das Geld auf die Stunde
               zurückerstatte. Ich kanns. Was ich \uline{nicht} kann, ist,
               es mir jetzt bis zur angegebenen Stunde verschaffen.\pend
           
\pstart
           Herzlich {\\[\baselineskip]}Ihr {\\[\baselineskip]}\spacefill\mbox{Salten}\pend
           \leftskip=0em{}\selectlanguage{ngerman}\endnumbering\briefempfaengerindex{Schnitzler, Arthur@\textsc{Schnitzler, Arthur}!zzzSalten, Felix@\emph{von Felix Salten}!1896-02-121@{{[}zwischen 12. – 29. 2. 1896{]}}|)be}\mylabel{L03168h}  \normalsize

\doendnotes{C}
\bigskip
\vfill

\clearpage

\footnotesize

\lohead{\textsc{register}}

% Definiere theindex-Environment komplett neu ohne reledmac
\makeatletter
\renewenvironment{theindex}{%
  \section*{\indexname}%
  \setlength{\parindent}{0pt}%
  \setlength{\parskip}{0pt plus 0.3pt}%
  \let\item\@idxitem
}{%
  \clearpage
}
\makeatother

\IfFileExists{\jobname-pw.ind}{\input{\jobname-pw.ind}}{}

\end{document}

      