%% latex-leseansicht-vorspann.tex
%% Vorspann für die Leseansicht.
%% Lädt die gemeinsame Datei latex-vorspann.tex mit nicht gesetztem Schalter.

\newif\ifkorrekturansicht
\korrekturansichtfalse

\input{../tex-inputs/latex-vorspann}


         \renewcommand{\erwaehnteOrte}{Orte: Berlin, Wien}
         \renewcommand{\erwaehnteWerke}{}
               \section[ Felix Salten an Arthur Schnitzler, {[}zwischen 12.–29. 2. 1896{]}]{ Felix Salten an Arthur Schnitzler, {[}zwischen 12.–29. 2. 1896{]}}\nopagebreak\mylabel{v}\rehead{ }\begin{ledgroupsized}[t]{13cm}\normalsize\beginnumbering \toendnotes[C]{\smallbreak\pagebreak[2]} \Standort{CUL, Schnitzler, B 89, A 1.}
\physDesc{Brief, 1 Blatt, 2 Seiten, 737 Zeichen
\newline{}Handschrift: Bleistift, lateinische Kurrent
\newline{}Schnitzler: mit Bleistift datiert: »Feber 96.« 
\newline{}Ordnung: mit Bleistift von unbekannter Hand nummeriert: »68a« }\toendnotes[C]{\smallbreak}\pstart
           \raggedleft{}{\pb}½ 3 Uhr\pend
           \pstart
           Lieber Arthur! verzeihen Sie, dass ich Sie \label{K_L03168-1v}\edtext{wecken}{\lemma{\textnormal{\emph{wecken}}}\Cendnote{\textnormal{Dadurch
                  wird Schnitzler\pwindex{Schnitzler, Arthur 15.05.1862 – 21.10.1931@\textsc{Schnitzler, Arthur} (15.05.1862 – 21.10.1931), \emph{Schriftsteller, Mediziner}|pwk}s Datierung auf »Feber 96« näher einschränkbar: Schnitzler\pwindex{Schnitzler, Arthur 15.05.1862 – 21.10.1931@\textsc{Schnitzler, Arthur} (15.05.1862 – 21.10.1931), \emph{Schriftsteller, Mediziner}|pwk}
                  kam erst am 11. 2. 1896 aus Berlin\oindex{Berlin@\textbf{Berlin}|pwk} zurück,
                  hätte also davor nicht so schnell handeln können.}}}\label{K_L03168-1h} laße. Aber ich fand heute{ }Nachts, als ich nach Hause kam{[},{]} den \label{K_L03168-2v}\edtext{inliegenden Brief}{\lemma{\textnormal{\emph{inliegenden Brief}}}\Cendnote{\textnormal{Beilage nicht erhalten}}}\label{K_L03168-2h}. Lesen Sie ihn, — er erklärt
               Ihnen die Situation, und helfen Sie mir. Ich kann Ihnen sagen, dass es Niemanden
               gibt, den ich um diese Stunde um \uline{das} bitten könnte.
               Ich gebe Ihnen mein Wort, dass Sie die Hälfte bis \substVorne{}\textsuperscript{\textcolor{gray}{×}}\substDazwischen{}3\substHinten{} Uhr Nachmittags zurück haben, und die andere Hälfte bis
                  Dienstag{ }{\pb}\uline{um 5 Uhr}. Ich sage nichts weiter dazu. Wenn Sie den inliegenden Brief gelesen haben,
               werden Sie begreifen, wie mir zu \substVorne{}\textsuperscript{m}\substDazwischen{}M\substHinten{}uthe ist, und ich hoffe, Sie zweifeln \uline{gewiss}{ }\uline{nicht} daran, dass ich Ihnen das Geld auf die Stunde
               zurückerstatte. Ich kanns. Was ich \uline{nicht} kann, ist,
               es mir jetzt bis zur angegebenen Stunde verschaffen.\pend
           \pstart
           Herzlich {\\[\baselineskip]}Ihr {\\[\baselineskip]}\spacefill\mbox{Salten}\pend
           \leftskip=0em{}
         
         \endnumbering\mylabel{h}\end{ledgroupsized}  \newcommand{\dateiname}{L03168}\newcommand{\titel}{Felix Salten an Arthur Schnitzler, [zwischen 12.–29. 2. 1896]}\newcommand{\editorInnen}{Martin Anton Müller und Laura Untner}%% latex-leseansicht-abspann.tex
%% Abspann für die Leseansicht.
%% Der Schalter \ifkorrekturansicht ist bereits durch den Vorspann gesetzt.

%% latex-abspann.tex
%% Gemeinsamer Abspann für Korrekturansicht und Leseansicht.
%% Setzt den Schalter \ifkorrekturansicht voraus (gesetzt in den
%% einbindenden Dateien latex-korrekturansicht-abspann.tex bzw.
%% latex-leseansicht-abspann.tex).
%% ---------------------------------------------------------------

\normalsize

% Das esempio-Environment wird nur in der Leseansicht benötigt
\ifkorrekturansicht\else
\newenvironment{esempio}[3]%
{
    \vspace{1.5ex}
    \rlap{\underline{#1}}
    \par
    \setlength{\parindent}{0cm}
    \nopagebreak
    \leftskip=#2cm
    \rightskip=#3cm
}
{
    \par
}
\fi

\doendnotes{C}
\bigskip
\vfill

\clearpage

\footnotesize

\ifkorrekturansicht
  \lohead{\textsc{register}}
\fi

% theindex-Environment neu definieren ohne reledmac
\makeatletter
\renewenvironment{theindex}{%
  \ifkorrekturansicht
    \section*{\indexname}%
  \else
    \subsubsection*{Index der erwähnten Entitäten}%
  \fi
  \setlength{\parindent}{0pt}%
  \setlength{\parskip}{0pt plus 0.3pt}%
  \let\item\@idxitem
}{%
  \ifkorrekturansicht\clearpage\fi
}
\makeatother

\IfFileExists{\jobname-pw.ind}{\input{\jobname-pw.ind}}{}

% Quellenangabe nur in der Leseansicht
\ifkorrekturansicht\else
% Fallback-Definitionen, falls die .tex-Datei \titel etc. nicht gesetzt hat
\providecommand{\titel}{}
\providecommand{\editorInnen}{}
\providecommand{\dateiname}{\jobname}

\vspace{3cm}

\vfill

\footnotesize
\textsc{Quelle}: \titel. Herausgegeben von {\editorInnen}. In: \emph{Arthur Schnitzler: Briefwechsel mit Autorinnen und Autoren}.
 Digitale Edition, https://schnitzler-briefe.acdh.oeaw.ac.at/{\dateiname}.html (Stand \today)
\fi

\end{document}


      