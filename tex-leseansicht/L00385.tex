%% latex-korrekturansicht-vorspann.tex
%% Vorspann für die Korrekturansicht.
%% Lädt die gemeinsame Datei latex-vorspann.tex mit gesetztem Schalter.

\newif\ifkorrekturansicht
\korrekturansichttrue

\input{../tex-inputs/latex-vorspann}


\section[Richard Beer-Hofmann an Arthur Schnitzler, 18. 10. 1894]{L00385 Richard Beer-Hofmann an Arthur Schnitzler, 18. 10. 1894}
\nopagebreak\mylabel{L00385v}
\rehead{ }\normalsize\beginnumbering\briefempfaengerindex{Schnitzler, Arthur@\textsc{Schnitzler, Arthur}!zzzBeer-Hofmann, Richard@\emph{von Richard Beer-Hofmann}!1894-10-182@{18. 10. 1894}|(be}
\toendnotes[C]{\smallbreak\pagebreak[2]}\Standort{CUL, Schnitzler, B 8.}
\physDesc{Postkarte, 287 Zeichen
\newline{}Handschrift: Bleistift, lateinische Kurrent
\newline{}Versand: 1) Stempel: »\nobreak{}\oindex{Neapel@\textbf{Neapel}, \emph{P.PPLA}|pwk}Napoli Ferrovia, 18 10–94, 4 M\nobreak{}«.   2) Stempel: »\nobreak{}\oindex{IX., Alsergrund@\textbf{IX., Alsergrund}, \emph{A.ADM3}|pwk}Wien 9/3, 20. 10. 94, 8.V, Bestellt\nobreak{}«. 
\newline{}Schnitzler: mit Bleistift nummeriert: »43« und beschriftet: »\noindent{}R.{ / }Mz\pwindex{Gluemer, Marie 03.07.1867 – 16.11.1925@\textsc{Glümer, Marie} (03.07.1867 – 16.11.1925), \emph{Schauspieler/Schauspielerin}|pw}{ / }D.\pwindex{Sandrock, Adele 1863-08-19 – 1937-08-30@\textsc{Sandrock, Adele} (1863-08-19 – 1937-08-30), \emph{Schauspieler/Schauspielerin}|pw}« }\pstart{}{\pb}\textcolor{gray}{\textbf{A}}n Herrn D\textsuperscript{r} Arthur
                  Schnitzler\pend{}\pstart{}Wien\oindex{Wien@\textbf{Wien}, \emph{A.ADM2}|pw}\pend{}\pstart{}IX Frankgasse 1\oindex{Frankgasse 1@\textbf{Frankgasse 1}, \emph{Wohngebäude (K.WHS)}|pw}\pend{}\pstart{}Austria\oindex{Oesterreich@\textbf{Österreich}, \emph{A.PCLI}|pw}\pend{}{\bigskip}\vspace{1em}
\pstart
           \noindent{}{\pb}Lieber Arthur! II. Nu{\geminationm}er der Zukunft\pwindex{Zukunft@\emph{Die Zukunft}|pw} erhalten, erste nicht; bitte I u III \uline{Neapel}\oindex{Neapel@\textbf{Neapel}, \emph{P.PPLA}|pw}{ }\uline{a posta ferma} oder Hôtel Hassler\oindex{Hôtel Hassler@\textbf{Hôtel Hassler}, \emph{Hotel (K.HTL)}|pw} zu adressiren. Möchte von \uline{Ihnen}{ }Schmetterlingsschlacht\pwindex{Schmetterlingsschlacht. Komoedie in 4 Akten@\emph{Die Schmetterlingsschlacht. Komödie in 4 Akten}|pw} »hören«. Ich bin
                  5. Nov in Wien\oindex{Wien@\textbf{Wien}, \emph{A.ADM2}|pw}.\pend
           
\pstart
           Herzlichst Ihr{\\[\baselineskip]}\spacefill\mbox{Richard}\pend
           \leftskip=0em{}
\pstart
           Mittwoch. \substVorne{}\textsuperscript{M}\substDazwischen{}A\substHinten{}bends. Neapel\oindex{Neapel@\textbf{Neapel}, \emph{P.PPLA}|pw}.\pend
           \selectlanguage{ngerman}\endnumbering\briefempfaengerindex{Schnitzler, Arthur@\textsc{Schnitzler, Arthur}!zzzBeer-Hofmann, Richard@\emph{von Richard Beer-Hofmann}!1894-10-182@{18. 10. 1894}|)be}\mylabel{L00385h}  \normalsize

\doendnotes{C}
\bigskip
\vfill

\clearpage

\footnotesize

\lohead{\textsc{register}}

% Definiere theindex-Environment komplett neu ohne reledmac
\makeatletter
\renewenvironment{theindex}{%
  \section*{\indexname}%
  \setlength{\parindent}{0pt}%
  \setlength{\parskip}{0pt plus 0.3pt}%
  \let\item\@idxitem
}{%
  \clearpage
}
\makeatother

\IfFileExists{\jobname-pw.ind}{\input{\jobname-pw.ind}}{}

\end{document}

      