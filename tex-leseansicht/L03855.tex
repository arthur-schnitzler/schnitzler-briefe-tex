%% latex-leseansicht-vorspann.tex
%% Vorspann für die Leseansicht.
%% Lädt die gemeinsame Datei latex-vorspann.tex mit nicht gesetztem Schalter.

\newif\ifkorrekturansicht
\korrekturansichtfalse

\input{../tex-inputs/latex-vorspann}


\section[Theodor Herzl an Arthur Schnitzler, 30. 3. 1895]{L03855 Theodor Herzl an Arthur Schnitzler, 30. 3. 1895}
\nopagebreak\mylabel{L03855v}
\rehead{ }\normalsize\beginnumbering\briefempfaengerindex{Schnitzler, Arthur@\textsc{Schnitzler, Arthur}!zzzHerzl, Theodor@\emph{von Theodor Herzl}!1895-03-302@{30. 3. 1895}|(be}
\toendnotes[C]{\smallbreak\pagebreak[2]}
\correspDesc{Versand  durch Theodor Herzl am 30. 3. 1895 in Wien
\newline{}Erhalt  durch Arthur Schnitzler im Zeitraum [30. 3. 1895
                  – 2. 4. 1895?] in Wien}\toendnotes[C]{\smallbreak}
\Standort{CUL, Schnitzler, B 39.}
\physDesc{Brief, 1 Blatt, 1 Seite, 251 Zeichen
\newline{}Handschrift: schwarze Tinte, lateinische Kurrent
\newline{}Ordnung: mit Bleistift von unbekannter Hand nummeriert:»34« }
\buchAbdrucke{\weitereDrucke{Theodor Herzl: \emph{Briefe und autobiographische Notizen 1866–1895}. Bearbeitet von Johannes Wachten in Zusammenarbeit mit Chaya Harel, Daisy Tycho und Manfred Winkler. Berlin, Frankfurt am Main, Wien: \emph{Propyläen} 1983, S. 580 (Briefe und Tagebücher. Herausgegeben von Alex Bein, Hermann Greive, Moshe Schaerf, Julius H. Schoeps und Johannes Wachten, 1).} }\toendnotes[C]{\smallbreak}
\pstart
           {\pb}\textcolor{gray}{\textbf{NEUE FREIE PRESSE\orgindex{Neue Freie Presse@Neue Freie Presse|pw}. }}\pend
           
\pstart
           \textcolor{gray}{\textbf{\textsc{Redaction}:}}\pend
           
\pstart
           \textcolor{gray}{\textbf{WIEN\oindex{Wien@\textbf{Wien}, \emph{Verwaltungsgebiet}|pw}}}\pend
           
\pstart
           \textcolor{gray}{\textbf{Kolowratring, Fichtegasse Nr. 11\oindex{Wien@\textbf{Wien}!I., Innere Stadt@\textbf{I., Innere Stadt}!Fichtegasse 11@\textbf{Fichtegasse 11}, \emph{Gebäude}|pw}.}}\hfill 30 März 95\pend
           
\pstart{}Lieber Freund!\pend\vspace{0.5em}
\pstart
           Nachmittag{ }zwischen 4 u. 6 komme ich zu Ihnen.\pend
           
\pstart
           Kann ich das nicht{[},{]} so telephonire ich Ihnen die \label{K_L03855-1v}\edtext{Logennummer}{\lemma{\textnormal{\emph{Logennummer}}}\Cendnote{\textnormal{Schnitzler vermerkt im \emph{Tagebuch}\pwindex{Schnitzler, Arthur 15.\,5.\,1862 Wien – 21.\,10.\,1931 ebd.@\textsc{Schnitzler, Arthur} (15.\,5.\,1862 Wien – 21.\,10.\,1931 ebd.), \emph{Schriftsteller, Mediziner}!Tagebuch@\strich\emph{Tagebuch}|pwk} den Besuch der Aufführung\eventindex{Carl-Theater@\textbf{Carl-Theater}!Aufführung von Die Brillanten-Königin, 30.3.1895@Aufführung von Die Brillanten-Königin, 30.3.1895|pwkv} der Operette \emph{die Billanten-Königin}\pwindex{\textcolor{red}{\textsuperscript{XXXX indx1}}!Brillanten-Königin. Operette in drei Acten@\strich\emph{Die Brillanten-Königin. Operette in drei Acten}|pwk}\pwindex{\textcolor{red}{\textsuperscript{XXXX indx1}}!Brillanten-Königin. Operette in drei Acten@\strich\emph{Die Brillanten-Königin. Operette in drei Acten}|pwk} und ein anschließendes Abendessen mit Herzl\pwindex{Herzl, Theodor 2.\,5.\,1860 Budapest – 3.\,7.\,1904 Edlach@\textsc{Herzl, Theodor} (2.\,5.\,1860 Budapest – 3.\,7.\,1904 Edlach), \emph{Schriftsteller, Journalist}|pwk}, vgl. A. S.: \emph{Tagebuch}, 30. 3. 1895.}}}\label{K_L03855-1}.\pend
           
\pstart
           Sollte von M. G.\pwindex{Müller-Guttenbrunn, Adam 22.\,10.\,1852 Zăbrani – 5.\,1.\,1923 Wien@\textsc{Müller-Guttenbrunn, Adam} (22.\,10.\,1852 Zăbrani – 5.\,1.\,1923 Wien), \emph{Schriftsteller, Theaterleiter, Beamter}|pwv}{ }\label{K_L03855-2v}\edtext{Brief
               über d. G.\pwindex{Herzl, Theodor 2.\,5.\,1860 Budapest – 3.\,7.\,1904 Edlach@\textsc{Herzl, Theodor} (2.\,5.\,1860 Budapest – 3.\,7.\,1904 Edlach), \emph{Schriftsteller, Journalist}!neue Ghetto. Schauspiel in vier Acten@\strich\emph{Das neue Ghetto. Schauspiel in vier Acten}|pw}}{\lemma{\textnormal{\emph{Brief
               über d. G.}}}\Cendnote{\textnormal{Herzl\pwindex{Herzl, Theodor 2.\,5.\,1860 Budapest – 3.\,7.\,1904 Edlach@\textsc{Herzl, Theodor} (2.\,5.\,1860 Budapest – 3.\,7.\,1904 Edlach), \emph{Schriftsteller, Journalist}|pwk} dürfte zu diesem
                  Zeitpunkt den Brief Schnitzlers vom XXXX Auszeichnungsfehler: Dokument L03927 nicht gefunden
                  noch nicht erhalten haben.}}}\label{K_L03855-2} da sein, so bitte ich mir ihn \uline{nicht} zu schicken, sondern ins Theater\oindex{Wien@\textbf{Wien}!II., Leopoldstadt@\textbf{II., Leopoldstadt}!Carl-Theater@\textbf{Carl-Theater}, \emph{Theater}|pw}\pwindex{\textcolor{red}{\textsuperscript{XXXX indx1}}!Brillanten-Königin. Operette in drei Acten@\strich\emph{Die Brillanten-Königin. Operette in drei Acten}|pwv}\pwindex{\textcolor{red}{\textsuperscript{XXXX indx1}}!Brillanten-Königin. Operette in drei Acten@\strich\emph{Die Brillanten-Königin. Operette in drei Acten}|pwv}\eventindex{Carl-Theater@\textbf{Carl-Theater}!Aufführung von Die Brillanten-Königin, 30.3.1895@Aufführung von Die Brillanten-Königin, 30.3.1895|pwv} zu bringen\pend
           
\pstart
           Herzlich Ihr{\\[\baselineskip]}\spacefill\mbox{Th H.}\pend
           \leftskip=0em{}\selectlanguage{ngerman}\endnumbering\briefempfaengerindex{Schnitzler, Arthur@\textsc{Schnitzler, Arthur}!zzzHerzl, Theodor@\emph{von Theodor Herzl}!1895-03-302@{30. 3. 1895}|)be}\mylabel{L03855h}
\begin{anhang}
\end{anhang}\newcommand{\dateiname}{L03855}\newcommand{\titel}{Theodor Herzl an Arthur Schnitzler, 30. 3. 1895}\newcommand{\editorInnen}{Selma Jahnke und Martin Anton Müller}%% latex-leseansicht-abspann.tex
%% Abspann für die Leseansicht.
%% Der Schalter \ifkorrekturansicht ist bereits durch den Vorspann gesetzt.

%% latex-abspann.tex
%% Gemeinsamer Abspann für Korrekturansicht und Leseansicht.
%% Setzt den Schalter \ifkorrekturansicht voraus (gesetzt in den
%% einbindenden Dateien latex-korrekturansicht-abspann.tex bzw.
%% latex-leseansicht-abspann.tex).
%% ---------------------------------------------------------------

\normalsize

% Das esempio-Environment wird nur in der Leseansicht benötigt
\ifkorrekturansicht\else
\newenvironment{esempio}[3]%
{
    \vspace{1.5ex}
    \rlap{\underline{#1}}
    \par
    \setlength{\parindent}{0cm}
    \nopagebreak
    \leftskip=#2cm
    \rightskip=#3cm
}
{
    \par
}
\fi

\doendnotes{C}
\bigskip
\vfill

\clearpage

\footnotesize

\ifkorrekturansicht
  \lohead{\textsc{register}}
\fi

% theindex-Environment neu definieren ohne reledmac
\makeatletter
\renewenvironment{theindex}{%
  \ifkorrekturansicht
    \section*{\indexname}%
  \else
    \subsubsection*{Index der erwähnten Entitäten}%
  \fi
  \setlength{\parindent}{0pt}%
  \setlength{\parskip}{0pt plus 0.3pt}%
  \let\item\@idxitem
}{%
  \ifkorrekturansicht\clearpage\fi
}
\makeatother

\IfFileExists{\jobname-pw.ind}{\input{\jobname-pw.ind}}{}

% Quellenangabe nur in der Leseansicht
\ifkorrekturansicht\else
% Fallback-Definitionen, falls die .tex-Datei \titel etc. nicht gesetzt hat
\providecommand{\titel}{}
\providecommand{\editorInnen}{}
\providecommand{\dateiname}{\jobname}

\vspace{3cm}

\vfill

\footnotesize
\textsc{Quelle}: \titel. Herausgegeben von {\editorInnen}. In: \emph{Arthur Schnitzler: Briefwechsel mit Autorinnen und Autoren}.
 Digitale Edition, https://schnitzler-briefe.acdh.oeaw.ac.at/{\dateiname}.html (Stand \today)
\fi

\end{document}


