%% latex-korrekturansicht-vorspann.tex
%% Vorspann für die Korrekturansicht.
%% Lädt die gemeinsame Datei latex-vorspann.tex mit gesetztem Schalter.

\newif\ifkorrekturansicht
\korrekturansichttrue

\input{../tex-inputs/latex-vorspann}


\section[Arthur Schnitzler an Richard Beer-Hofmann, 26. 6. 1900]{L01050 Arthur Schnitzler an Richard Beer-Hofmann, 26. 6. 1900}
\nopagebreak\mylabel{L01050v}
\rehead{ }\normalsize\beginnumbering\briefempfaengerindex{Beer-Hofmann, Richard@\textsc{Beer-Hofmann, Richard}!zzzSchnitzler, Arthur@\emph{von Arthur Schnitzler}!1900-06-261@{26. 6. 1900}|(be}
\toendnotes[C]{\smallbreak\pagebreak[2]}\Standort{YCGL, MSS 31.}
\physDesc{Briefkarte, , Umschlag, 334 Zeichen
\newline{}Handschrift: Bleistift, deutsche Kurrent
\newline{}Versand: 1) Stempel: »\nobreak{}\oindex{IX., Alsergrund@\textbf{IX., Alsergrund}, \emph{A.ADM3}|pwk}Wien 9/3, 26. 6. 00, 4–5N\nobreak{}«.   2) Stempel: »\nobreak{}\oindex{Altaussee@\textbf{Altaussee}, \emph{A.ADM3}|pwk}Alt-Aussee, 27 6 {[}00{]}\nobreak{}«. }
\buchAbdrucke{\weitereDrucke{Arthur Schnitzler, Richard Beer-Hofmann: \emph{Briefwechsel 1891–1931}. Wien, Zürich: \emph{Europaverlag} 1992, S. 147.} }\pstart{}{\pb}Herrn \textsc{Dr. Rich.
                     Beer-Hofmann}\pend{}\pstart{}\textsc{Altaussee\oindex{Altaussee@\textbf{Altaussee}, \emph{A.ADM3}|pw}.}\pend{}{\bigskip}\vspace{1em}
\pstart
           \noindent{}{\pb}Lieber Richard, bitte beſtellen Sie mir
               bei Bru{\geminationn}thaler\oindex{Gasthaus Brunnthaler@\textbf{Gasthaus Brunnthaler}, \emph{Hotel (K.HTL)}|pw} ein
                  Zi{\geminationm}er. Ich beabſichtige Donnerstag früh
               nach Iſchl\oindex{Bad Ischl@\textbf{Bad Ischl}, \emph{P.PPL}|pw} zu fahren, dort Hirſchf.s\pwindex{Petersen, Elly 26.02.1874 – 29.12.1965@\textsc{Petersen, Elly} (26.02.1874 – 29.12.1965), \emph{Schriftsteller/Schriftstellerin}|pw}\pwindex{Hirschfeld, Georg 11.02.1873 – 17.01.1942@\textsc{Hirschfeld, Georg} (11.02.1873 – 17.01.1942), \emph{Schriftsteller/Schriftstellerin}|pw} zu beſuchen u zu übernachten, \uuline{Freitag}{ }\uline{früh} ſpäteſtens Mittag in Altauſſee\oindex{Altaussee@\textbf{Altaussee}, \emph{A.ADM3}|pw}{ }{\pb}zu ſein wo ich bis So{\geminationn}tag früh bleibe. Weiteres mündlich.\pend
           
\pstart
           Auf Wiederſehn. Herzlichſt{\\[\baselineskip]}Ihr \spacefill\mbox{Arthur}\pend
           \leftskip=0em{}
\pstart
           26 6 900.\pend
           \selectlanguage{ngerman}\endnumbering\briefempfaengerindex{Beer-Hofmann, Richard@\textsc{Beer-Hofmann, Richard}!zzzSchnitzler, Arthur@\emph{von Arthur Schnitzler}!1900-06-261@{26. 6. 1900}|)be}\mylabel{L01050h}  \normalsize

\doendnotes{C}
\bigskip
\vfill

\clearpage

\footnotesize

\lohead{\textsc{register}}

% Definiere theindex-Environment komplett neu ohne reledmac
\makeatletter
\renewenvironment{theindex}{%
  \section*{\indexname}%
  \setlength{\parindent}{0pt}%
  \setlength{\parskip}{0pt plus 0.3pt}%
  \let\item\@idxitem
}{%
  \clearpage
}
\makeatother

\IfFileExists{\jobname-pw.ind}{\input{\jobname-pw.ind}}{}

\end{document}

      