\input{../tex-inputs/latex-pdf-vorspann}
\begin{center}
            \textcolor{red}{ENTWURF. ENTZIFFERUNG NOCH NICHT KORREKTURGELESEN}
                      \end{center}
            
               \section[Arthur Schnitzler an Richard Beer-Hofmann, 25. 8. 1907]{ Arthur Schnitzler an Richard Beer-Hofmann, 25. 8. 1907}\nopagebreak\mylabel{v}\rehead{ }\begin{ledgroupsized}[t]{13cm}\normalsize\beginnumbering\briefempfaengerindex{Beer-Hofmann, Richard@\textsc{Beer-Hofmann, Richard}!zzzSchnitzler, Arthur@\emph{von Arthur Schnitzler}!1907-08-251@{25. 8. 1907}|(be} \toendnotes[C]{\smallbreak\pagebreak[2]} \Standort{YCGL, MSS 31.}
\physDesc{Brief, 1 Blatt, 3 Seiten, Umschlag (auf der Rückseite Fotografien von »Hôtel Pension BavariaMeran-Obermais« und »Hôtel Pension Wildbad-Waldbrunn, Pustertal (Tirol)«.)
\newline{}Handschrift: Bleistift, deutsche Kurrent\newline{}Versand: 1) Stempel: »\nobreak{}\oindex{Welsberg-Taisten@\textbf{Welsberg-Taisten}|pwk}Wels{[}berg{]}, 25. 7. 07\nobreak{}«.  2) Stempel: »\nobreak{}\oindex{Velden@\textbf{Velden}|pwk}{\pb}Velden am
                                       Wörtherseee, 26. VIII. 07, VII\nobreak{}«. 
\newline{}Beer-Hofmann: mit Bleistift den Zeitpunkt der Beantwortung festgehalten:
               »Be 29/VIII 07« }\buchAbdrucke{\weitereDrucke{Arthur Schnitzler, Richard Beer-Hofmann: \emph{Briefwechsel 1891–1931}. Hg. Konstanze Fliedl. Wien, Zürich: \emph{Europaverlag} 1992, S. 183–184.} }\toendnotes[C]{\smallbreak}\pstart{}{\pb}\textcolor{gray}{\textbf{HÔTEL {\kaufmannsund} PENSION
                           BAVARIA\oindex{Hotel {\kaufmannsund} Pension Bavaria@\textbf{Hotel {\kaufmannsund} Pension Bavaria}|pw}{ }MERAN-OBERMAIS\oindex{Obermais@\textbf{Obermais}|pw}}}\pend{}\pstart{}\textcolor{gray}{\textbf{HÔTEL {\kaufmannsund} PENSION WILDBAD WALDBRUNN\oindex{Wildbad Waldbrunn@\textbf{Wildbad Waldbrunn}|pw}, PUSTERTAL\oindex{Pustertal@\textbf{Pustertal}|pw}}}\pend{}\pstart{}\textcolor{gray}{\textbf{BESITZER: JOS. BÖHM\pwindex{Boehm, Josef @\textsc{Böhm, Josef}, \emph{Hotelbesitzer}|pw}}}\pend{}{\bigskip}\pstart{}\textsc{Herrn Dr. Rich}\pend{}\pstart{}\textsc{Beer-Hofmann}\pend{}\pstart{}\textsc{Velden}\oindex{Velden@\textbf{Velden}|pw}\pend{}\pstart{}\textsc{am Wörthersee}\oindex{Woerthersee@\textbf{Wörthersee}|pw}\pend{}\pstart{}\textsc{Etabl. Wahliss}\oindex{Etablissement Ernst Wahliss@\textbf{Etablissement Ernst Wahliss}|pw}\pend{}{\bigskip}\pstart
           \raggedleft{}{\pb}\textsc{Welsb.-Waldbr.}\oindex{Wildbad Waldbrunn@\textbf{Wildbad Waldbrunn}|pw}{ }25. 8 907\pend
           \pstart{}lieber Richard,\pend\pstart
           wir fahren morgen 26. von hier fort. (Heini\pwindex{Schnitzler, Heinrich 09.08.1902 – 12.07.1982@\textsc{Schnitzler, Heinrich} (09.08.1902 – 12.07.1982), \emph{Regisseur, Schauspieler}|pw} direct nach Wien\oindex{Wien@\textbf{Wien}|pw}.) Olga\pwindex{Schnitzler, Olga 17.01.1882 – 13.01.1970@\textsc{Schnitzler, Olga} (17.01.1882 – 13.01.1970), \emph{Schauspielerin, Sängerin}|pw} u ich (höchſtwahrſcheinlich) Waidbruck\oindex{Ponte Gardena@\textbf{Ponte Gardena}|pw}, übernachten. Da{\geminationn} durchs Grödner Thal\oindex{Val Badia@\textbf{Val Badia}|pw}, Grödnerjoch\oindex{Groedner Joch@\textbf{Grödner Joch}|pw}, \textsc{Corvara\oindex{Corvara@\textbf{Corvara}|pw} – Arabba\oindex{Arabba@\textbf{Arabba}|pw}} – Pordoj\oindex{Pordoijoch@\textbf{Pordoijoch}|pw}, – Vigo\oindex{Vigo di Fassa@\textbf{Vigo di Fassa}|pw} – \textsc{Karer}ſee\oindex{Karersee@\textbf{Karersee}|pw} – \textsc{Bozen}\oindex{Bozen@\textbf{Bozen}|pw}. Manches zu Fuſs, manches zu Wagen; Modificationen nicht ausgeſchloſſen.
               Höchſtwahrſcheinlich Samſtag oder So{\geminationn}tag ſind wir in Bozen\oindex{Bozen@\textbf{Bozen}|pw}, wollen aber \introOben{}von\introOben{}
               dort direct nach Meran\oindex{Meran@\textbf{Meran}|pw} weiter, um dort, in {\pb}dem uns lebhaft empfohlenen parkumgebenen \textsc{Palasthotel}\oindex{Palasthotel Meran@\textbf{Palasthotel Meran}|pw} etwa 8 Tage zu bleiben. Da{\geminationn} wohl –(vielleicht mit
               Aufenthalt in \textsc{Bozen}\oindex{Bozen@\textbf{Bozen}|pw} (\textsc{Mendel}\oindex{Mendelpass@\textbf{Mendelpass}|pw}) – I{\geminationn}sbruck\oindex{Innsbruck@\textbf{Innsbruck}|pw}
                  (\label{K_L01702_1v}\edtext{\textsc{Theoderich}\pwindex{Theoderich der Grosse 451/456 – 30.08.0526@\textsc{Theoderich der Große} (451/456 – 30.08.0526), \emph{König, Regent}|pw}\oindex{Hofkirche@\textbf{Hofkirche}|pwv}}{\lemma{\textnormal{\emph{Theoderich}}}\Cendnote{\textnormal{Die Statue Theoderichs des Großen\pwindex{Theoderich der Grosse 451/456 – 30.08.0526@\textsc{Theoderich der Große} (451/456 – 30.08.0526), \emph{König, Regent}|pwk} von Peter
                     Vischer\pwindex{Vischer, Peter um 1455 – 07.01.1529@\textsc{Vischer, Peter} (um 1455 – 07.01.1529), \emph{Bildhauer}|pwk} (dem Älteren) in der Innsbrucker
                     Hofkirche\oindex{Hofkirche@\textbf{Hofkirche}|pwk} spielt in Schnitzler\pwindex{Schnitzler, Arthur 15.05.1862 – 21.10.1931@\textsc{Schnitzler, Arthur} (15.05.1862 – 21.10.1931), \emph{Schriftsteller, Mediziner}|pwk}s
                  Erzählung \emph{Dämmerseele}\pwindex{Schnitzler, Arthur 15.05.1862 – 21.10.1931@\textsc{Schnitzler, Arthur} (15.05.1862 – 21.10.1931), \emph{Schriftsteller, Mediziner}!Fremde18.5.1902 – 18.5.1902@\strich\emph{Die Fremde} {[}18.5.1902 – 18.5.1902{]}|pwk} eine zentrale Rolle.}}}\label{K_L01702_1h}) – Salzburg\oindex{Salzburg@\textbf{Salzburg}|pw} (Salzburg\oindex{Salzburg@\textbf{Salzburg}|pw})–) nach Wien\oindex{Wien@\textbf{Wien}|pw}. –\pend
           \pstart
           Ein Telegra{\geminationm}{ }\substVorne{}\textsuperscript{Freitag}{\allowbreak}\substDazwischen{}Do{\geminationn}erſtag\substHinten{}{ }\textsc{Karersee\oindex{Karersee@\textbf{Karersee}|pw} post rest} hat Chance uns zu
               erreichen. Vielleicht entſchließen Sie ſich auch zu Meran\oindex{Meran@\textbf{Meran}|pw}. Gardasee\oindex{Lago di Garda@\textbf{Lago di Garda}|pw} werden wir diesmal
                  (diesmal{\dots}!) wohl nicht mitnehmen. In dem \textsc{Karer}ſee\oindex{Karersee@\textbf{Karersee}|pw}telegr. vermerken Sie vielleicht, we{\geminationn}s geht, wo Sie Samſtag{ }{\pb}u So{\geminationn}tag\textsc{telegr.} zu erreichen ſind. –\pend
           \pstart
           \textsc{Goldmann}\pwindex{Goldmann, Paul 31.01.1865 – 25.09.1935@\textsc{Goldmann, Paul} (31.01.1865 – 25.09.1935), \emph{Schriftsteller, Journalist}|pw} iſt mit Mutter\pwindex{Goldmann, Clementine 1842-05-15 – 1924-02-24@\textsc{Goldmann, Clementine} (1842-05-15 – 1924-02-24)|pwv} in \textsc{Gossensass}\oindex{Gossensass@\textbf{Gossensass}|pw} verblieben, wie er mir auf mündl Weg ausrichten lieſs. –\pend
           \pstart
           Leben Sie wohl, ſeien Sie Alle gegrüßt und laſſen Sie mich hoffen, dſs wir einander
               bald wiederſehen.\pend
           \pstart
           Herzlichſt Ihr{\\[\baselineskip]}\spacefill\mbox{A.}\pend
           \leftskip=0em{}\endnumbering\briefempfaengerindex{Beer-Hofmann, Richard@\textsc{Beer-Hofmann, Richard}!zzzSchnitzler, Arthur@\emph{von Arthur Schnitzler}!1907-08-251@{25. 8. 1907}|)be}\mylabel{h}\end{ledgroupsized}  \newcommand{\dateiname}{L01702}\newcommand{\titel}{Arthur Schnitzler an Richard Beer-Hofmann, 25. 8. 1907}\newcommand{\editorInnen}{Martin Anton Müller und Gerd-Hermann Susen}\input{../tex-inputs/latex-pdf-abspann}
      