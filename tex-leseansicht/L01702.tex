%% latex-korrekturansicht-vorspann.tex
%% Vorspann für die Korrekturansicht.
%% Lädt die gemeinsame Datei latex-vorspann.tex mit gesetztem Schalter.

\newif\ifkorrekturansicht
\korrekturansichttrue

\input{../tex-inputs/latex-vorspann}


\section[Arthur Schnitzler an Richard Beer-Hofmann, 25. 8. 1907]{L01702 Arthur Schnitzler an Richard Beer-Hofmann, 25. 8. 1907}
\nopagebreak\mylabel{L01702v}
\rehead{ }\normalsize\beginnumbering\briefempfaengerindex{Beer-Hofmann, Richard@\textsc{Beer-Hofmann, Richard}!zzzSchnitzler, Arthur@\emph{von Arthur Schnitzler}!1907-08-251@{25. 8. 1907}|(be}
\toendnotes[C]{\smallbreak\pagebreak[2]}\Standort{YCGL, MSS 31.}
\physDesc{Brief, 1 Blatt, 3 Seiten, Umschlag, 1158 Zeichen (Rückseite des Umschlags mit Fotografien von »Hôtel {\kaufmannsund} Pension Bavaria\oindex{Hotel {\kaufmannsund} Pension Bavaria@\textbf{Hotel {\kaufmannsund} Pension Bavaria}, \emph{Hotel (K.HTL)}|pw}{ }Meran-Obermais\oindex{Obermais@\textbf{Obermais}, \emph{Bezirk (A.BZK)}|pw}« und »Hôtel {\kaufmannsund} Pension
                                       Wildbad-Waldbrunn\oindex{Wildbad Waldbrunn@\textbf{Wildbad Waldbrunn}, \emph{S.SPA}|pw}, Pustertal (Tirol)\oindex{Pustertal@\textbf{Pustertal}, \emph{T.VAL}|pw}«)
\newline{}Handschrift: Bleistift, deutsche Kurrent
\newline{}Versand: 1) Stempel: »\nobreak{}\oindex{Welsberg-Taisten@\textbf{Welsberg-Taisten}, \emph{A.ADM3}|pwk}Wels{[}berg{]}, 25. 7. 07\nobreak{}«.   2) Stempel: »\nobreak{}\oindex{Velden am Woerthersee@\textbf{Velden am Wörthersee}, \emph{P.PPL}|pwk}{\pb}Velden am
                                       Wörtherseee, 26. VIII. 07, VII\nobreak{}«. 
\newline{}Beer-Hofmann: mit Bleistift den Zeitpunkt der Beantwortung festgehalten:
                                    »Be 29/VIII 07« }
\buchAbdrucke{\weitereDrucke{Arthur Schnitzler, Richard Beer-Hofmann: \emph{Briefwechsel 1891–1931}. Wien, Zürich: \emph{Europaverlag} 1992, S. 183–184.} }\toendnotes[C]{\smallbreak}\pstart{}{\pb}\textcolor{gray}{\textbf{HÔTEL {\kaufmannsund}
                           PENSION BAVARIA\oindex{Hotel {\kaufmannsund} Pension Bavaria@\textbf{Hotel {\kaufmannsund} Pension Bavaria}, \emph{Hotel (K.HTL)}|pw}{ }MERAN-OBERMAIS\oindex{Obermais@\textbf{Obermais}, \emph{Bezirk (A.BZK)}|pw}}}\pend{}\pstart{}\textcolor{gray}{\textbf{HÔTEL {\kaufmannsund} PENSION WILDBAD WALDBRUNN\oindex{Wildbad Waldbrunn@\textbf{Wildbad Waldbrunn}, \emph{S.SPA}|pw}, PUSTERTAL\oindex{Pustertal@\textbf{Pustertal}, \emph{T.VAL}|pw}}}\pend{}\pstart{}\textcolor{gray}{\textbf{BESITZER: JOS. BÖHM\pwindex{Boehm, Josef @\textsc{Böhm, Josef}, \emph{Hotelbesitzer/Hotelbesitzerin}|pw}}}\pend{}{\bigskip}\pstart{}\textsc{Herrn Dr. Rich}\pend{}\pstart{}\textsc{Beer-Hofmann}\pend{}\pstart{}\textsc{Velden}\oindex{Velden am Woerthersee@\textbf{Velden am Wörthersee}, \emph{P.PPL}|pw}\pend{}\pstart{}\textsc{am Wörthersee}\oindex{Woerthersee@\textbf{Wörthersee}, \emph{H.LK}|pw}\pend{}\pstart{}\textsc{Etabl. Wahliss}\oindex{Etablissement Ernst Wahliss@\textbf{Etablissement Ernst Wahliss}, \emph{Hotel (K.HTL)}|pw}\pend{}{\bigskip}\vspace{1em}
\pstart
           \raggedleft{}{\pb}\textsc{Welsb.-Waldbr.}\oindex{Wildbad Waldbrunn@\textbf{Wildbad Waldbrunn}, \emph{S.SPA}|pw}{ }25. 8 907\pend
           
\pstart{}lieber Richard,\pend\vspace{0.5em}
\pstart
           wir fahren morgen 26. von hier fort. (Heini\pwindex{Schnitzler, Heinrich 09.08.1902 – 12.07.1982@\textsc{Schnitzler, Heinrich} (09.08.1902 – 12.07.1982), \emph{Regisseur/Regisseurin, Schauspieler/Schauspielerin}|pw} direct nach Wien\oindex{Wien@\textbf{Wien}, \emph{A.ADM2}|pw}.) Olga\pwindex{Schnitzler, Olga 17.01.1882 – 13.01.1970@\textsc{Schnitzler, Olga} (17.01.1882 – 13.01.1970), \emph{Schauspieler/Schauspielerin, Sänger/Sängerin}|pw} u ich (höchſtwahrſcheinlich) Waidbruck\oindex{Ponte Gardena@\textbf{Ponte Gardena}, \emph{A.ADM3}|pw}, übernachten. Da{\geminationn} durchs Grödner
                  Thal\oindex{Val Badia@\textbf{Val Badia}, \emph{T.VAL}|pw}, Grödnerjoch\oindex{Groedner Joch@\textbf{Grödner Joch}, \emph{Berg (N.BRG)}|pw}, \textsc{Corvara\oindex{Corvara@\textbf{Corvara}, \emph{P.PPLA3}|pw} – Arabba\oindex{Arabba@\textbf{Arabba}, \emph{P.PPL}|pw}} – Pordoj\oindex{Pordoijoch@\textbf{Pordoijoch}, \emph{Berg (N.BRG)}|pw}, – Vigo\oindex{Vigo di Fassa@\textbf{Vigo di Fassa}, \emph{P.PPLA3}|pw} – \textsc{Karer}ſee\oindex{Karersee@\textbf{Karersee}, \emph{See (N.SEE)}|pw} – \textsc{Bozen}\oindex{Bozen@\textbf{Bozen}, \emph{P.PPLA2}|pw}. Manches zu Fuſs, manches zu Wagen; Modificationen nicht ausgeſchloſſen.
               Höchſtwahrſcheinlich Samſtag oder So{\geminationn}tag ſind wir in Bozen\oindex{Bozen@\textbf{Bozen}, \emph{P.PPLA2}|pw}, wollen aber \introOben{}von\introOben{}
               dort direct nach Meran\oindex{Meran@\textbf{Meran}, \emph{P.PPLA3}|pw} weiter, um dort, in {\pb}dem uns lebhaft empfohlenen parkumgebenen \textsc{Palasthotel}\oindex{Palasthotel Meran@\textbf{Palasthotel Meran}, \emph{Hotel (K.HTL)}|pw} etwa 8 Tage zu bleiben. Da{\geminationn} wohl –(vielleicht mit
               Aufenthalt in \textsc{Bozen}\oindex{Bozen@\textbf{Bozen}, \emph{P.PPLA2}|pw} (\textsc{Mendel}\oindex{Mendelpass@\textbf{Mendelpass}, \emph{Pass (N.PAS)}|pw}) – I{\geminationn}sbruck\oindex{Innsbruck@\textbf{Innsbruck}, \emph{A.ADM2}|pw}
                  (\label{K_L01702-1v}\edtext{\textsc{Theoderich}\pwindex{Theoderich der Grosse 451/456 – 30.08.0526@\textsc{Theoderich der Große} (451/456 – 30.08.0526), \emph{König/Königin, Regent/Regentin}|pw}\oindex{Hofkirche@\textbf{Hofkirche}, \emph{Kirche (K.KRC)}|pwv}}{\lemma{\textnormal{\emph{Theoderich}}}\Cendnote{\textnormal{Die Statue Theoderichs des Großen\pwindex{Theoderich der Grosse 451/456 – 30.08.0526@\textsc{Theoderich der Große} (451/456 – 30.08.0526), \emph{König/Königin, Regent/Regentin}|pwk} von Peter Vischer\pwindex{Vischer, Peter um 1455 – 07.01.1529@\textsc{Vischer, Peter} (um 1455 – 07.01.1529), \emph{Bildhauer/Bildhauerin}|pwk} (dem Älteren) in der Innsbrucker Hofkirche\oindex{Hofkirche@\textbf{Hofkirche}, \emph{Kirche (K.KRC)}|pwk} spielt in Schnitzlers Erzählung \emph{Dämmerseele}\pwindex{Daemmerseele@\emph{Dämmerseele}|pwk}
                  eine zentrale Rolle.}}}\label{K_L01702-1}) – Salzburg\oindex{Salzburg@\textbf{Salzburg}, \emph{A.ADM2}|pw} (Salzburg\oindex{Salzburg@\textbf{Salzburg}, \emph{A.ADM2}|pw})–) nach Wien\oindex{Wien@\textbf{Wien}, \emph{A.ADM2}|pw}. –\pend
           
\pstart
           Ein Telegra{\geminationm}{ }\substVorne{}\textsuperscript{Freitag}\substDazwischen{}Do{\geminationn}erſtag\substHinten{}{ }\textsc{Karersee\oindex{Karersee@\textbf{Karersee}, \emph{See (N.SEE)}|pw} post rest} hat Chance uns zu
               erreichen. Vielleicht entſchließen Sie ſich auch zu Meran\oindex{Meran@\textbf{Meran}, \emph{P.PPLA3}|pw}. Gardasee\oindex{Lago di Garda@\textbf{Lago di Garda}, \emph{See (N.SEE)}|pw} werden wir diesmal
                  (diesmal{\dots}!) wohl nicht mitnehmen. In dem \textsc{Karer}ſee\oindex{Karersee@\textbf{Karersee}, \emph{See (N.SEE)}|pw}telegr. vermerken Sie vielleicht, we{\geminationn}s geht, wo Sie Samſtag{ }{\pb}u So{\geminationn}tag\textsc{telegr.} zu erreichen ſind. –\pend
           
\pstart
           \textsc{Goldmann}\pwindex{Goldmann, Paul 31.01.1865 – 25.09.1935@\textsc{Goldmann, Paul} (31.01.1865 – 25.09.1935), \emph{Schriftsteller/Schriftstellerin, Journalist/Journalistin}|pw} iſt mit Mutter\pwindex{Goldmann, Clementine 1842-05-15 – 1924-02-24@\textsc{Goldmann, Clementine} (1842-05-15 – 1924-02-24)|pwv} in \textsc{Gossensass}\oindex{Gossensass@\textbf{Gossensaß}, \emph{P.PPLA3}|pw} verblieben, wie er mir auf mündl Weg ausrichten lieſs. –\pend
           
\pstart
           Leben Sie wohl, ſeien Sie Alle gegrüßt und laſſen Sie mich hoffen, dſs wir einander
               bald wiederſehen.\pend
           
\pstart
           Herzlichſt Ihr{\\[\baselineskip]}\spacefill\mbox{A.}\pend
           \leftskip=0em{}\selectlanguage{ngerman}\endnumbering\briefempfaengerindex{Beer-Hofmann, Richard@\textsc{Beer-Hofmann, Richard}!zzzSchnitzler, Arthur@\emph{von Arthur Schnitzler}!1907-08-251@{25. 8. 1907}|)be}\mylabel{L01702h}  \normalsize

\doendnotes{C}
\bigskip
\vfill

\clearpage

\footnotesize

\lohead{\textsc{register}}

% Definiere theindex-Environment komplett neu ohne reledmac
\makeatletter
\renewenvironment{theindex}{%
  \section*{\indexname}%
  \setlength{\parindent}{0pt}%
  \setlength{\parskip}{0pt plus 0.3pt}%
  \let\item\@idxitem
}{%
  \clearpage
}
\makeatother

\IfFileExists{\jobname-pw.ind}{\input{\jobname-pw.ind}}{}

\end{document}

      