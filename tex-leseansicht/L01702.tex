%% latex-leseansicht-vorspann.tex
%% Vorspann für die Leseansicht.
%% Lädt die gemeinsame Datei latex-vorspann.tex mit nicht gesetztem Schalter.

\newif\ifkorrekturansicht
\korrekturansichtfalse

\input{../tex-inputs/latex-vorspann}


\section[Arthur Schnitzler an Richard Beer-Hofmann, 25. 8. 1907]{L01702 Arthur Schnitzler an Richard Beer-Hofmann, 25. 8. 1907}
\nopagebreak\mylabel{L01702v}
\rehead{ }\normalsize\beginnumbering\briefempfaengerindex{Beer-Hofmann, Richard@\textsc{Beer-Hofmann, Richard}!zzzSchnitzler, Arthur@\emph{von Arthur Schnitzler}!1907-08-251@{25. 8. 1907}|(be}
\toendnotes[C]{\smallbreak\pagebreak[2]}
\correspDesc{Versand  durch Arthur Schnitzler am 25. 8. 1907 in Welsberg-Taisten
\newline{}Erhalt  durch Richard Beer-Hofmann am 26. 8. 1907 in Velden am Wörthersee}\toendnotes[C]{\smallbreak}
\Standort{YCGL, MSS 31.}
\physDesc{Brief, 1 Blatt, 3 Seiten, Kuvert, 1158 Zeichen (Rückseite des Umschlags mit Fotografien von »Hôtel {\kaufmannsund} Pension Bavaria\oindex{Hotel {\kaufmannsund} Pension Bavaria@\textbf{Hotel {\kaufmannsund} Pension Bavaria}, \emph{Hotel}|pw}{ }Meran-Obermais\oindex{Obermais@\textbf{Obermais}, \emph{Bezirk}|pw}« und »Hôtel {\kaufmannsund} Pension
                                       Wildbad-Waldbrunn\oindex{Wildbad Waldbrunn@\textbf{Wildbad Waldbrunn}, \emph{Spa}|pw}, Pustertal (Tirol)\oindex{Pustertal@\textbf{Pustertal}, \emph{Tal}|pw}«)
\newline{}Handschrift: Bleistift, deutsche Kurrent
\newline{}Versand: 1) Stempel: »\nobreak{}\oindex{Welsberg-Taisten@\textbf{Welsberg-Taisten}, \emph{Verwaltungsgebiet}|pwk}Wels{[}berg{]}, 25. 7. 07\nobreak{}«.   2) Stempel: »\nobreak{}\oindex{Velden am Wörthersee@\textbf{Velden am Wörthersee}|pwk}{\pb}Velden am
                                       Wörtherseee, 26. VIII. 07, VII\nobreak{}«. 
\newline{}Beer-Hofmann: mit Bleistift den Zeitpunkt der Beantwortung festgehalten:
                                    »Be 29/VIII 07« }
\buchAbdrucke{\weitereDrucke{Arthur Schnitzler, Richard Beer-Hofmann: \emph{Briefwechsel 1891–1931}. Herausgegeben von Konstanze Fliedl. Wien, Zürich: \emph{Europaverlag} 1992, S. 183–184.} }\toendnotes[C]{\smallbreak}\pstart{}{\pb}\textcolor{gray}{\textbf{HÔTEL {\kaufmannsund}
                           PENSION BAVARIA\oindex{Hotel {\kaufmannsund} Pension Bavaria@\textbf{Hotel {\kaufmannsund} Pension Bavaria}, \emph{Hotel}|pw}{ }MERAN-OBERMAIS\oindex{Obermais@\textbf{Obermais}, \emph{Bezirk}|pw}}}\pend{}\pstart{}\textcolor{gray}{\textbf{HÔTEL {\kaufmannsund} PENSION WILDBAD WALDBRUNN\oindex{Wildbad Waldbrunn@\textbf{Wildbad Waldbrunn}, \emph{Spa}|pw}, PUSTERTAL\oindex{Pustertal@\textbf{Pustertal}, \emph{Tal}|pw}}}\pend{}\pstart{}\textcolor{gray}{\textbf{BESITZER: JOS. BÖHM\pwindex{Böhm, Josef @\textsc{Böhm, Josef}, \emph{Hotelbesitzer}|pw}}}\pend{}{\bigskip}\pstart{}\textsc{Herrn Dr. Rich}\pend{}\pstart{}\textsc{Beer-Hofmann}\pend{}\pstart{}\textsc{Velden}\oindex{Velden am Wörthersee@\textbf{Velden am Wörthersee}|pw}\pend{}\pstart{}\textsc{am Wörthersee}\oindex{Wörthersee@\textbf{Wörthersee}, \emph{See}|pw}\pend{}\pstart{}\textsc{Etabl. Wahliss}\oindex{Etablissement Ernst Wahliss@\textbf{Etablissement Ernst Wahliss}, \emph{Hotel}|pw}\pend{}{\bigskip}\vspace{1em}
\pstart
           \raggedleft{}{\pb}\textsc{Welsb.-Waldbr.}\oindex{Wildbad Waldbrunn@\textbf{Wildbad Waldbrunn}, \emph{Spa}|pw}{ }25. 8 907\pend
           
\pstart{}lieber Richard,\pend\vspace{0.5em}
\pstart
           wir fahren morgen 26. von hier fort. (Heini\pwindex{Schnitzler, Heinrich 9.\,8.\,1902 Hinterbrühl – 12.\,7.\,1982 Wien@\textsc{Schnitzler, Heinrich} (9.\,8.\,1902 Hinterbrühl – 12.\,7.\,1982 Wien), \emph{Regisseur, Schauspieler}|pw} direct nach Wien\oindex{Wien@\textbf{Wien}, \emph{Verwaltungsgebiet}|pw}.) Olga\pwindex{Schnitzler, Olga 17.\,1.\,1882 Wien – 13.\,1.\,1970 Lugano@\textsc{Schnitzler, Olga} (17.\,1.\,1882 Wien – 13.\,1.\,1970 Lugano), \emph{Schauspielerin, Sängerin}|pw} u ich (höchſtwahrſcheinlich) Waidbruck\oindex{Ponte Gardena@\textbf{Ponte Gardena}, \emph{Verwaltungsgebiet}|pw}, übernachten. Da{\geminationn} durchs Grödner
                  Thal\oindex{Val Badia@\textbf{Val Badia}, \emph{Tal}|pw}, Grödnerjoch\oindex{Grödner Joch@\textbf{Grödner Joch}, \emph{Berg}|pw}, \textsc{Corvara\oindex{Corvara@\textbf{Corvara}, \emph{Hauptstadt}|pw} – Arabba\oindex{Arabba@\textbf{Arabba}|pw}} – Pordoj\oindex{Pordoijoch@\textbf{Pordoijoch}, \emph{Berg}|pw}, – Vigo\oindex{Vigo di Fassa@\textbf{Vigo di Fassa}, \emph{Hauptstadt}|pw} – \textsc{Karer}ſee\oindex{Karersee@\textbf{Karersee}, \emph{See}|pw} – \textsc{Bozen}\oindex{Bozen@\textbf{Bozen}, \emph{Hauptstadt}|pw}. Manches zu Fuſs, manches zu Wagen; Modificationen nicht ausgeſchloſſen.
               Höchſtwahrſcheinlich Samſtag oder So{\geminationn}tag{ }ſind wir in Bozen\oindex{Bozen@\textbf{Bozen}, \emph{Hauptstadt}|pw}, wollen aber \introOben{}von\introOben{}
               dort direct nach Meran\oindex{Meran@\textbf{Meran}, \emph{Hauptstadt}|pw} weiter, um dort, in {\pb}dem uns lebhaft empfohlenen parkumgebenen \textsc{Palasthotel}\oindex{Palasthotel Meran@\textbf{Palasthotel Meran}, \emph{Hotel}|pw} etwa 8 Tage zu bleiben. Da{\geminationn} wohl –(vielleicht mit
               Aufenthalt in \textsc{Bozen}\oindex{Bozen@\textbf{Bozen}, \emph{Hauptstadt}|pw} (\textsc{Mendel}\oindex{Mendelpass@\textbf{Mendelpass}, \emph{Pass}|pw}) – I{\geminationn}sbruck\oindex{Innsbruck@\textbf{Innsbruck}, \emph{Verwaltungsgebiet}|pw}
                  (\label{K_L01702-1v}\edtext{\textsc{Theoderich}\pwindex{Theoderich der Große 451/456 Ungarn – 30.\,8.\,526 Ravenna@\textsc{Theoderich der Große} (451/456 Ungarn – 30.\,8.\,526 Ravenna), \emph{König, Regent}|pw}\oindex{Hofkirche@\textbf{Hofkirche}, \emph{Kirche}|pwv}}{\lemma{\textnormal{\emph{Theoderich}}}\Cendnote{\textnormal{Die Statue Theoderichs des Großen\pwindex{Theoderich der Große 451/456 Ungarn – 30.\,8.\,526 Ravenna@\textsc{Theoderich der Große} (451/456 Ungarn – 30.\,8.\,526 Ravenna), \emph{König, Regent}|pwk} von Peter Vischer\pwindex{Vischer, Peter um 1455 Nürnberg – 7.\,1.\,1529 ebd.@\textsc{Vischer, Peter} (um 1455 Nürnberg – 7.\,1.\,1529 ebd.), \emph{Bildhauer}|pwk} (dem Älteren) in der Innsbrucker Hofkirche\oindex{Hofkirche@\textbf{Hofkirche}, \emph{Kirche}|pwk} spielt in Schnitzlers Erzählung \emph{Dämmerseele}\pwindex{Schnitzler, Arthur 15.\,5.\,1862 Wien – 21.\,10.\,1931 ebd.@\textsc{Schnitzler, Arthur} (15.\,5.\,1862 Wien – 21.\,10.\,1931 ebd.), \emph{Schriftsteller, Mediziner}!Dämmerseele@\strich\emph{Dämmerseele}|pwk}
                  eine zentrale Rolle.}}}\label{K_L01702-1}) – Salzburg\oindex{Salzburg@\textbf{Salzburg}, \emph{Verwaltungsgebiet}|pw} (Salzburg\oindex{Salzburg@\textbf{Salzburg}, \emph{Verwaltungsgebiet}|pw})–) nach Wien\oindex{Wien@\textbf{Wien}, \emph{Verwaltungsgebiet}|pw}. –\pend
           
\pstart
           Ein Telegra{\geminationm}{ }\substVorne{}\textsuperscript{Freitag}\substDazwischen{}Do{\geminationn}erſtag\substHinten{}{ }\textsc{Karersee\oindex{Karersee@\textbf{Karersee}, \emph{See}|pw} post rest} hat Chance uns zu
               erreichen. Vielleicht entſchließen Sie{ }ſich auch zu Meran\oindex{Meran@\textbf{Meran}, \emph{Hauptstadt}|pw}. Gardasee\oindex{Lago di Garda@\textbf{Lago di Garda}, \emph{See}|pw} werden wir diesmal
                  (diesmal{\dots}!) wohl nicht mitnehmen. In dem \textsc{Karer}ſee\oindex{Karersee@\textbf{Karersee}, \emph{See}|pw}telegr. vermerken Sie vielleicht, we{\geminationn}s geht, wo Sie Samſtag{ }{\pb}u So{\geminationn}tag\textsc{telegr.} zu erreichen{ }ſind. –\pend
           
\pstart
           \textsc{Goldmann}\pwindex{Goldmann, Paul 31.\,1.\,1865 Breslau – 25.\,9.\,1935 Wien@\textsc{Goldmann, Paul} (31.\,1.\,1865 Breslau – 25.\,9.\,1935 Wien), \emph{Schriftsteller, Journalist}|pw} iſt mit Mutter\pwindex{Goldmann, Clementine 15.\,5.\,1842 Breslau – 24.\,2.\,1924 Frankfurt am Main@\textsc{Goldmann, Clementine} (15.\,5.\,1842 Breslau – 24.\,2.\,1924 Frankfurt am Main)|pwv} in \textsc{Gossensass}\oindex{Gossensaß@\textbf{Gossensaß}, \emph{Hauptstadt}|pw} verblieben, wie er mir auf mündl Weg ausrichten lieſs. –\pend
           
\pstart
           Leben Sie wohl,{ }ſeien Sie Alle gegrüßt und laſſen Sie mich hoffen, dſs wir einander
               bald wiederſehen.\pend
           
\pstart
           Herzlichſt Ihr{\\[\baselineskip]}\spacefill\mbox{A.}\pend
           \leftskip=0em{}\selectlanguage{ngerman}\endnumbering\briefempfaengerindex{Beer-Hofmann, Richard@\textsc{Beer-Hofmann, Richard}!zzzSchnitzler, Arthur@\emph{von Arthur Schnitzler}!1907-08-251@{25. 8. 1907}|)be}\mylabel{L01702h}  \newcommand{\dateiname}{L01702}\newcommand{\titel}{Arthur Schnitzler an Richard Beer-Hofmann, 25. 8. 1907}\newcommand{\editorInnen}{Martin Anton Müller und Gerd-Hermann Susen}%% latex-leseansicht-abspann.tex
%% Abspann für die Leseansicht.
%% Der Schalter \ifkorrekturansicht ist bereits durch den Vorspann gesetzt.

%% latex-abspann.tex
%% Gemeinsamer Abspann für Korrekturansicht und Leseansicht.
%% Setzt den Schalter \ifkorrekturansicht voraus (gesetzt in den
%% einbindenden Dateien latex-korrekturansicht-abspann.tex bzw.
%% latex-leseansicht-abspann.tex).
%% ---------------------------------------------------------------

\normalsize

% Das esempio-Environment wird nur in der Leseansicht benötigt
\ifkorrekturansicht\else
\newenvironment{esempio}[3]%
{
    \vspace{1.5ex}
    \rlap{\underline{#1}}
    \par
    \setlength{\parindent}{0cm}
    \nopagebreak
    \leftskip=#2cm
    \rightskip=#3cm
}
{
    \par
}
\fi

\doendnotes{C}
\bigskip
\vfill

\clearpage

\footnotesize

\ifkorrekturansicht
  \lohead{\textsc{register}}
\fi

% theindex-Environment neu definieren ohne reledmac
\makeatletter
\renewenvironment{theindex}{%
  \ifkorrekturansicht
    \section*{\indexname}%
  \else
    \subsubsection*{Index der erwähnten Entitäten}%
  \fi
  \setlength{\parindent}{0pt}%
  \setlength{\parskip}{0pt plus 0.3pt}%
  \let\item\@idxitem
}{%
  \ifkorrekturansicht\clearpage\fi
}
\makeatother

\IfFileExists{\jobname-pw.ind}{\input{\jobname-pw.ind}}{}

% Quellenangabe nur in der Leseansicht
\ifkorrekturansicht\else
% Fallback-Definitionen, falls die .tex-Datei \titel etc. nicht gesetzt hat
\providecommand{\titel}{}
\providecommand{\editorInnen}{}
\providecommand{\dateiname}{\jobname}

\vspace{3cm}

\vfill

\footnotesize
\textsc{Quelle}: \titel. Herausgegeben von {\editorInnen}. In: \emph{Arthur Schnitzler: Briefwechsel mit Autorinnen und Autoren}.
 Digitale Edition, https://schnitzler-briefe.acdh.oeaw.ac.at/{\dateiname}.html (Stand \today)
\fi

\end{document}


