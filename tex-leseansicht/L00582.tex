%% latex-leseansicht-vorspann.tex
%% Vorspann für die Leseansicht.
%% Lädt die gemeinsame Datei latex-vorspann.tex mit nicht gesetztem Schalter.

\newif\ifkorrekturansicht
\korrekturansichtfalse

\input{../tex-inputs/latex-vorspann}


\section[Arthur Schnitzler an Hugo von Hofmannsthal, 2. 9. 1896]{L00582 Arthur Schnitzler an Hugo von Hofmannsthal, 2. 9. 1896}
\nopagebreak\mylabel{L00582v}
\rehead{ }\normalsize\beginnumbering\briefempfaengerindex{Hofmannsthal, Hugo von@\textsc{Hofmannsthal, Hugo von}!zzzSchnitzler, Arthur@\emph{von Arthur Schnitzler}!1896-09-021@{2. 9. 1896}|(be}
\toendnotes[C]{\smallbreak\pagebreak[2]}
\correspDesc{Versand  durch Arthur Schnitzler am 2. 9. 1896 in Wien
\newline{}Erhalt  durch Hugo von Hofmannsthal im Zeitraum [2. 9. 1896
                  – 6. 9. 1896?] in Wien}\toendnotes[C]{\smallbreak}
\Standort{FDH, Hs-30885,52.}
\physDesc{Brief, 1 Blatt, 4 Seiten, 1278 Zeichen
\newline{}Handschrift: schwarze Tinte, deutsche Kurrent}
\buchAbdrucke{\weitereDrucke{Hugo von Hofmannsthal, Arthur Schnitzler: \emph{Briefwechsel}. Herausgegeben von Therese Nickl und Heinrich Schnitzler. Frankfurt am Main: \emph{S. Fischer} 1964, S. 74–75.} }\toendnotes[C]{\smallbreak}
\pstart
           \raggedleft{}{\pb}Wien\oindex{Wien@\textbf{Wien}, \emph{Verwaltungsgebiet}|pw}{ }2. 9. 96.\pend
           
\pstart{}Lieber Hugo,\pend\vspace{0.5em}
\pstart
           Ihren{ }ſo gemeinſchaftlichen\pwindex{Schaffgotsch, Hermine von 25.\,11.\,1871 Wien – 25.\,11.\,1928 Purgstall@\textsc{Schaffgotsch, Hermine von} (25.\,11.\,1871 Wien – 25.\,11.\,1928 Purgstall)|pwv}
               Brief hab ich in Berlin\oindex{Berlin@\textbf{Berlin}, \emph{Hauptstadt}|pw} beko{\geminationm}en und hab mich{ }ſehr darüber gefreut. Sind Sie noch in
                  Altausſee\oindex{Altaussee@\textbf{Altaussee}, \emph{Verwaltungsgebiet}|pw}? Jedenfalls{ }ſende ich Ihnen dahin
               meine herzlichſten Grüße und hoffe Sie bald in Wien\oindex{Wien@\textbf{Wien}, \emph{Verwaltungsgebiet}|pw} zu{ }ſehn. Ich war in Berlin\oindex{Berlin@\textbf{Berlin}, \emph{Hauptstadt}|pw}{ }{\pb}4 Tage; das bis zur Unkenntlichkeit umgearbeitete Stück\pwindex{Schnitzler, Arthur 15.\,5.\,1862 Wien – 21.\,10.\,1931 ebd.@\textsc{Schnitzler, Arthur} (15.\,5.\,1862 Wien – 21.\,10.\,1931 ebd.), \emph{Schriftsteller, Mediziner}!Freiwild. Schauspiel in 3 Akten@\strich\emph{Freiwild. Schauspiel in 3 Akten}|pwv} hab ich dem Brahm\pwindex{Brahm, Otto 5.\,2.\,1856 Hamburg – 28.\,11.\,1912 Berlin@\textsc{Brahm, Otto} (5.\,2.\,1856 Hamburg – 28.\,11.\,1912 Berlin), \emph{Theaterleiter, Regisseur}|pw} vorgeleſen, der es, nicht ohne
               ausgeſprochenes Vergnügen, gleich angeno{\geminationm}en hat. Er
               wollte es{ }ſchon im September aufführen, wogegen ich mich wehre; wohl mit
               Erfolg. –\pend
           
\pstart
           Auch in München\oindex{München@\textbf{München}|pw} war ich 2 Tage, und{ }ſeit
                  \label{K_L00582-1v}\edtext{Samstag{ }Früh}{\lemma{\textnormal{\emph{Samstag Früh}}}\Cendnote{\textnormal{29. 8. 1896}}}\label{K_L00582-1} bin ich wieder zu Hauſe, wo ich eben einen {\pb}der
               wildeſten Schnupfen durchlebe. So kann ich nicht mit der nötigen Geiſtesfriſche auf
               die Vierzeiler antworten, obwohl ich mehr als dreifachen Sinn darin erkannt zu haben
               glaube.\pend
           
\pstart
           Daſs ich Ihre Novelle\pwindex{Hofmannsthal, Hugo von 1.\,2.\,1874 Wien – 15.\,7.\,1929 Rodaun@\textsc{Hofmannsthal, Hugo von} (1.\,2.\,1874 Wien – 15.\,7.\,1929 Rodaun), \emph{Schriftsteller}!Geschichte der beiden Liebespaare@\strich\emph{Geschichte der beiden Liebespaare}|pwv} nicht
               hören{ }ſoll, beleidigt mich – nur Richard\pwindex{Beer-Hofmann, Richard 11.\,7.\,1866 Wien – 26.\,9.\,1945 New York City@\textsc{Beer-Hofmann, Richard} (11.\,7.\,1866 Wien – 26.\,9.\,1945 New York City), \emph{Schriftsteller}|pw}{ }ſoll das \label{K_L00582-2v}\edtext{Vorrecht}{\lemma{\textnormal{\emph{Vorrecht}}}\Cendnote{\textnormal{Hofmannsthal\pwindex{Hofmannsthal, Hugo von 1.\,2.\,1874 Wien – 15.\,7.\,1929 Rodaun@\textsc{Hofmannsthal, Hugo von} (1.\,2.\,1874 Wien – 15.\,7.\,1929 Rodaun), \emph{Schriftsteller}|pwk} hatte \emph{Geschichte der beiden Liebespaare}\pwindex{Hofmannsthal, Hugo von 1.\,2.\,1874 Wien – 15.\,7.\,1929 Rodaun@\textsc{Hofmannsthal, Hugo von} (1.\,2.\,1874 Wien – 15.\,7.\,1929 Rodaun), \emph{Schriftsteller}!Geschichte der beiden Liebespaare@\strich\emph{Geschichte der beiden Liebespaare}|pwk} nach harter Kritik von Beer-Hofmann\pwindex{Beer-Hofmann, Richard 11.\,7.\,1866 Wien – 26.\,9.\,1945 New York City@\textsc{Beer-Hofmann, Richard} (11.\,7.\,1866 Wien – 26.\,9.\,1945 New York City), \emph{Schriftsteller}|pwk} zurückgelegt.}}}\label{K_L00582-2} haben,
               Sachen zu leſen, die Sie nicht für gelungen halten?\pend
           
\pstart
           Ich wollte, es käme mir einmal {\pb}was von Ihnen vor Augen
               mit{ }ſchönen jungen Fehlern!\pend
           
\pstart
           Wie ko{\geminationm}en Sie plötzlich aufs Theaterſpielen? Ich war
               ganz erſchüttert!\pend
           
\pstart
           – Aber Zuſa{\geminationm}enſein werden wir hoffentlich oft – und ohne
               das, was Sie »Halbwahres« ne{\geminationn}en, was aber was ganz
               andres iſt.\pend
           
\pstart
           Wüßt ich nur ganz genau was! In \textsc{Upsala}\oindex{Uppsala@\textbf{Uppsala}, \emph{Region}|pw} hab ich drüber nachgedacht – \uline{wirklich} in \textsc{Upsala}\oindex{Uppsala@\textbf{Uppsala}, \emph{Region}|pw}! –\pend
           \pstart Herzliche Grüße! Ihr \spacefill\mbox{Arthur}\pend{}\selectlanguage{ngerman}\endnumbering\briefempfaengerindex{Hofmannsthal, Hugo von@\textsc{Hofmannsthal, Hugo von}!zzzSchnitzler, Arthur@\emph{von Arthur Schnitzler}!1896-09-021@{2. 9. 1896}|)be}\mylabel{L00582h}  \newcommand{\dateiname}{L00582}\newcommand{\titel}{Arthur Schnitzler an Hugo von Hofmannsthal, 2. 9. 1896}\newcommand{\editorInnen}{Martin Anton Müller und Gerd-Hermann Susen}%% latex-leseansicht-abspann.tex
%% Abspann für die Leseansicht.
%% Der Schalter \ifkorrekturansicht ist bereits durch den Vorspann gesetzt.

%% latex-abspann.tex
%% Gemeinsamer Abspann für Korrekturansicht und Leseansicht.
%% Setzt den Schalter \ifkorrekturansicht voraus (gesetzt in den
%% einbindenden Dateien latex-korrekturansicht-abspann.tex bzw.
%% latex-leseansicht-abspann.tex).
%% ---------------------------------------------------------------

\normalsize

% Das esempio-Environment wird nur in der Leseansicht benötigt
\ifkorrekturansicht\else
\newenvironment{esempio}[3]%
{
    \vspace{1.5ex}
    \rlap{\underline{#1}}
    \par
    \setlength{\parindent}{0cm}
    \nopagebreak
    \leftskip=#2cm
    \rightskip=#3cm
}
{
    \par
}
\fi

\doendnotes{C}
\bigskip
\vfill

\clearpage

\footnotesize

\ifkorrekturansicht
  \lohead{\textsc{register}}
\fi

% theindex-Environment neu definieren ohne reledmac
\makeatletter
\renewenvironment{theindex}{%
  \ifkorrekturansicht
    \section*{\indexname}%
  \else
    \subsubsection*{Index der erwähnten Entitäten}%
  \fi
  \setlength{\parindent}{0pt}%
  \setlength{\parskip}{0pt plus 0.3pt}%
  \let\item\@idxitem
}{%
  \ifkorrekturansicht\clearpage\fi
}
\makeatother

\IfFileExists{\jobname-pw.ind}{\input{\jobname-pw.ind}}{}

% Quellenangabe nur in der Leseansicht
\ifkorrekturansicht\else
% Fallback-Definitionen, falls die .tex-Datei \titel etc. nicht gesetzt hat
\providecommand{\titel}{}
\providecommand{\editorInnen}{}
\providecommand{\dateiname}{\jobname}

\vspace{3cm}

\vfill

\footnotesize
\textsc{Quelle}: \titel. Herausgegeben von {\editorInnen}. In: \emph{Arthur Schnitzler: Briefwechsel mit Autorinnen und Autoren}.
 Digitale Edition, https://schnitzler-briefe.acdh.oeaw.ac.at/{\dateiname}.html (Stand \today)
\fi

\end{document}


