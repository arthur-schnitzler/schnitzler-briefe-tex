%% latex-korrekturansicht-vorspann.tex
%% Vorspann für die Korrekturansicht.
%% Lädt die gemeinsame Datei latex-vorspann.tex mit gesetztem Schalter.

\newif\ifkorrekturansicht
\korrekturansichttrue

\input{../tex-inputs/latex-vorspann}


\section[Arthur Schnitzler an Hugo von Hofmannsthal, 2. 9. 1896]{L00582 Arthur Schnitzler an Hugo von Hofmannsthal, 2. 9. 1896}
\nopagebreak\mylabel{L00582v}
\rehead{ }\normalsize\beginnumbering\briefempfaengerindex{Hofmannsthal, Hugo von@\textsc{Hofmannsthal, Hugo von}!zzzSchnitzler, Arthur@\emph{von Arthur Schnitzler}!1896-09-021@{2. 9. 1896}|(be}
\toendnotes[C]{\smallbreak\pagebreak[2]}\Standort{FDH, Hs-30885,52.}
\physDesc{Brief, 1 Blatt, 4 Seiten, 1278 Zeichen
\newline{}Handschrift: schwarze Tinte, deutsche Kurrent}
\buchAbdrucke{\weitereDrucke{Hugo von Hofmannsthal, Arthur Schnitzler: \emph{Briefwechsel}. Frankfurt am Main: \emph{S. Fischer} 1964, S. 74–75.} }\toendnotes[C]{\smallbreak}
\pstart
           \raggedleft{}{\pb}Wien\oindex{Wien@\textbf{Wien}, \emph{A.ADM2}|pw}{ }2. 9. 96.\pend
           
\pstart{}Lieber Hugo,\pend\vspace{0.5em}
\pstart
           Ihren ſo gemeinſchaftlichen\pwindex{Schaffgotsch, Hermine von 25.11.1871 – 25.11.1928@\textsc{Schaffgotsch, Hermine von} (25.11.1871 – 25.11.1928)|pwv}
               Brief hab ich in Berlin\oindex{Berlin@\textbf{Berlin}, \emph{P.PPLC}|pw} beko{\geminationm}en und hab mich ſehr darüber gefreut. Sind Sie noch in
                  Altausſee\oindex{Altaussee@\textbf{Altaussee}, \emph{A.ADM3}|pw}? Jedenfalls ſende ich Ihnen dahin
               meine herzlichſten Grüße und hoffe Sie bald in Wien\oindex{Wien@\textbf{Wien}, \emph{A.ADM2}|pw} zu ſehn. Ich war in Berlin\oindex{Berlin@\textbf{Berlin}, \emph{P.PPLC}|pw}{ }{\pb}4 Tage; das bis zur Unkenntlichkeit umgearbeitete Stück\pwindex{Freiwild. Schauspiel in 3 Akten@\emph{Freiwild. Schauspiel in 3 Akten}|pwv} hab ich dem Brahm\pwindex{Brahm, Otto 05.02.1856 – 28.11.1912@\textsc{Brahm, Otto} (05.02.1856 – 28.11.1912), \emph{Theaterleiter/Theaterleiterin, Regisseur/Regisseurin}|pw} vorgeleſen, der es, nicht ohne
               ausgeſprochenes Vergnügen, gleich angeno{\geminationm}en hat. Er
               wollte es ſchon im September aufführen, wogegen ich mich wehre; wohl mit
               Erfolg. –\pend
           
\pstart
           Auch in München\oindex{Muenchen@\textbf{München}, \emph{P.PPLA}|pw} war ich 2 Tage, und ſeit
                  \label{K_L00582-1v}\edtext{SamstagFrüh}{\lemma{\textnormal{\emph{SamstagFrüh}}}\Cendnote{\textnormal{29. 8. 1896}}}\label{K_L00582-1} bin ich wieder zu Hauſe, wo ich eben einen {\pb}der
               wildeſten Schnupfen durchlebe. So kann ich nicht mit der nötigen Geiſtesfriſche auf
               die Vierzeiler antworten, obwohl ich mehr als dreifachen Sinn darin erkannt zu haben
               glaube.\pend
           
\pstart
           Daſs ich Ihre Novelle\pwindex{Geschichte der beiden Liebespaare@\emph{Geschichte der beiden Liebespaare}|pwv} nicht
               hören ſoll, beleidigt mich – nur Richard\pwindex{Beer-Hofmann, Richard 1866-07-11 – 1945-09-26@\textsc{Beer-Hofmann, Richard} (1866-07-11 – 1945-09-26), \emph{Schriftsteller/Schriftstellerin}|pw}{ }ſoll das \label{K_L00582-2v}\edtext{Vorrecht}{\lemma{\textnormal{\emph{Vorrecht}}}\Cendnote{\textnormal{Hofmannsthal\pwindex{Hofmannsthal, Hugo von 1874-02-01 – 1929-07-15@\textsc{Hofmannsthal, Hugo von} (1874-02-01 – 1929-07-15), \emph{Schriftsteller/Schriftstellerin}|pwk} hatte \emph{Geschichte der beiden Liebespaare}\pwindex{Geschichte der beiden Liebespaare@\emph{Geschichte der beiden Liebespaare}|pwk} nach harter Kritik von Beer-Hofmann\pwindex{Beer-Hofmann, Richard 1866-07-11 – 1945-09-26@\textsc{Beer-Hofmann, Richard} (1866-07-11 – 1945-09-26), \emph{Schriftsteller/Schriftstellerin}|pwk} zurückgelegt.}}}\label{K_L00582-2} haben,
               Sachen zu leſen, die Sie nicht für gelungen halten?\pend
           
\pstart
           Ich wollte, es käme mir einmal {\pb}was von Ihnen vor Augen
               mit ſchönen jungen Fehlern!\pend
           
\pstart
           Wie ko{\geminationm}en Sie plötzlich aufs Theaterſpielen? Ich war
               ganz erſchüttert!\pend
           
\pstart
           – Aber Zuſa{\geminationm}enſein werden wir hoffentlich oft – und ohne
               das, was Sie »Halbwahres« ne{\geminationn}en, was aber was ganz
               andres iſt.\pend
           
\pstart
           Wüßt ich nur ganz genau was! In \textsc{Upsala}\oindex{Uppsala@\textbf{Uppsala}, \emph{A.ADM4}|pw} hab ich drüber nachgedacht – \uline{wirklich} in \textsc{Upsala}\oindex{Uppsala@\textbf{Uppsala}, \emph{A.ADM4}|pw}! –\pend
           \pstart Herzliche Grüße! Ihr \spacefill\mbox{Arthur}\pend{}\selectlanguage{ngerman}\endnumbering\briefempfaengerindex{Hofmannsthal, Hugo von@\textsc{Hofmannsthal, Hugo von}!zzzSchnitzler, Arthur@\emph{von Arthur Schnitzler}!1896-09-021@{2. 9. 1896}|)be}\mylabel{L00582h}  \normalsize

\doendnotes{C}
\bigskip
\vfill

\clearpage

\footnotesize

\lohead{\textsc{register}}

% Definiere theindex-Environment komplett neu ohne reledmac
\makeatletter
\renewenvironment{theindex}{%
  \section*{\indexname}%
  \setlength{\parindent}{0pt}%
  \setlength{\parskip}{0pt plus 0.3pt}%
  \let\item\@idxitem
}{%
  \clearpage
}
\makeatother

\IfFileExists{\jobname-pw.ind}{\input{\jobname-pw.ind}}{}

\end{document}

      