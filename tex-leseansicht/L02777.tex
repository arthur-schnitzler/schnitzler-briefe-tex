%% latex-leseansicht-vorspann.tex
%% Vorspann für die Leseansicht.
%% Lädt die gemeinsame Datei latex-vorspann.tex mit nicht gesetztem Schalter.

\newif\ifkorrekturansicht
\korrekturansichtfalse

\input{../tex-inputs/latex-vorspann}


         
         \renewcommand{\erwaehntePersonen}{Personen: Paul Goldmann, Christian Schefer, Leopold Sonnemann, Richard Wagner}
         \renewcommand{\erwaehnteInstitutionen}{Institutionen: Frankfurter Zeitung, Nouvelle Revue}
         \renewcommand{\erwaehnteOrte}{Orte: Berlin, Deutschland, Dänemark, Frankreich, Hamburg, Kopenhagen, Melun, Norwegen, Paris, Rue Doré, Schweiz, Stavanger, Wien, rue Feydeau}
         \renewcommand{\erwaehnteWerke}{Werke: Frankfurter Zeitung, La Nouvelle Revue, Neue Deutsche Rundschau, Un jeune écrivain viennois: M. Arthur Schnitzler}
               \section[ Paul Goldmann an Arthur Schnitzler, 15. 6. {[}1896{]}]{ Paul Goldmann an Arthur Schnitzler, 15. 6. {[}1896{]}}\nopagebreak\mylabel{v}\rehead{ }\begin{ledgroupsized}[t]{13cm}\normalsize\beginnumbering\briefempfaengerindex{Schnitzler, Arthur@\textsc{Schnitzler, Arthur}!zzzGoldmann, Paul@\emph{von Paul Goldmann}!1896-06-151@{15. 6. {[}1896{]}}|(be} \toendnotes[C]{\smallbreak\pagebreak[2]} \Standort{DLA, A:Schnitzler, HS.NZ85.1.3166.}
\physDesc{Brief, 1 Blatt, 3 Seiten, 3173 Zeichen
\newline{}Handschrift: blaue Tinte, deutsche Kurrent
\newline{}Beilage: maschinenschriftlicher Brief: 1 Blatt, 2 Seiten, mit
                                 handschriftlicher Unterschrift in schwarzer Tinte 
\newline{}Schnitzler: mit Bleistift das Jahr »96« vermerkt }\toendnotes[C]{\smallbreak}\pstart
           \noindent{}{\pb}\textcolor{gray}{\textbf{\textbf{Frankfurter Zeitung\orgindex{Frankfurter Zeitung@Frankfurter Zeitung|pw}}}}\pend
           \pstart
           \textcolor{gray}{\textbf{(\begin{otherlanguage}{french}Gazette de Francfort\end{otherlanguage}\orgindex{Frankfurter Zeitung@Frankfurter Zeitung|pw}).}}\pend
           \pstart
           \textcolor{gray}{\textbf{\textbf{\begin{otherlanguage}{french}Fondateur M.\end{otherlanguage}{ }L. Sonnemann\pwindex{Sonnemann, Leopold 1831-10-29 – 1909-10-30@\textsc{Sonnemann, Leopold} (1831-10-29 – 1909-10-30), \emph{Journalist, Herausgeber}|pw}.}}}\pend
           \pstart
           \begin{otherlanguage}{french}\textcolor{gray}{\textbf{Journal\pwindex{?? Werk@Nicht ermittelte Verfasserinnen und Verfasser!Frankfurter Zeitung1856 – 1943@\emph{Frankfurter Zeitung} {[}1856 – 1943{]}|pwv} politique,
                        financier,}}\end{otherlanguage}\pend
           \pstart
           \begin{otherlanguage}{french}\textcolor{gray}{\textbf{commercial et littéraire.}}\end{otherlanguage}\pend
           \pstart
           \begin{otherlanguage}{french}\textcolor{gray}{\textbf{\textbf{Paraissant trois fois par jour.}}}\end{otherlanguage}\pend
           \pstart
           \begin{otherlanguage}{french}\textcolor{gray}{\textbf{\textbf{Bureau à Paris\oindex{Paris@\textbf{Paris}|pw}}}}\end{otherlanguage}\hfill \textsc{Paris\oindex{Paris@\textbf{Paris}|pw}}, 15. Juni.\pend
           \pstart
           \begin{otherlanguage}{french}\textcolor{gray}{\textbf{\textbf{24. Rue Feydeau\oindex{rue Feydeau@\textbf{rue Feydeau}|pw}.}}}\end{otherlanguage}\pend
           \pstart\center{}Mein lieber Freund,\pend\pstart
           Anbei erhältſt Du die »\textsc{Nouvelle Revue\pwindex{?? Werk@Nicht ermittelte Verfasserinnen und Verfasser!Nouvelle Revue1879 – 1940@\emph{La Nouvelle Revue} {[}1879 – 1940{]}|pw}}« mit dem \label{K_L02777-1v}\edtext{Artikel\pwindex{Schefer, Christian 1866-07-14 – Februar 1944@\textsc{Schefer, Christian} (1866-07-14 – Februar 1944), \emph{Journalist, Lehrer}!Un jeune ecrivain viennois: M. Arthur Schnitzler1896-06-15@\strich\emph{Un jeune écrivain viennois: M. Arthur Schnitzler} {[}1896-06-15{]}|pwv}}{\lemma{\textnormal{\emph{Artikel}}}\Cendnote{\textnormal{Christian Schefer\pwindex{Schefer, Christian 1866-07-14 – Februar 1944@\textsc{Schefer, Christian} (1866-07-14 – Februar 1944), \emph{Journalist, Lehrer}|pwk}: \emph{Un jeune écrivain viennois: M. Arthur Schnitzler}\pwindex{Schefer, Christian 1866-07-14 – Februar 1944@\textsc{Schefer, Christian} (1866-07-14 – Februar 1944), \emph{Journalist, Lehrer}!Un jeune ecrivain viennois: M. Arthur Schnitzler1896-06-15@\strich\emph{Un jeune écrivain viennois: M. Arthur Schnitzler} {[}1896-06-15{]}|pwk}. In:
                        \emph{La Nouvelle Revue}\pwindex{?? Werk@Nicht ermittelte Verfasserinnen und Verfasser!Nouvelle Revue1879 – 1940@\emph{La Nouvelle Revue} {[}1879 – 1940{]}|pwk}, Jg. 18, Nr. 100,
                        Mai–Juni 1896,
                     S. 855–859.}}}\label{K_L02777-1h} über Dich. Die Eindrücke ſind nicht ſtichhaltig, aber
               ich finde den Artikel\pwindex{Schefer, Christian 1866-07-14 – Februar 1944@\textsc{Schefer, Christian} (1866-07-14 – Februar 1944), \emph{Journalist, Lehrer}!Un jeune ecrivain viennois: M. Arthur Schnitzler1896-06-15@\strich\emph{Un jeune écrivain viennois: M. Arthur Schnitzler} {[}1896-06-15{]}|pwv} ſehr
               liebenswürdig, beſonders mit Rückſicht auf die Stelle\pwindex{Schefer, Christian 1866-07-14 – Februar 1944@\textsc{Schefer, Christian} (1866-07-14 – Februar 1944), \emph{Journalist, Lehrer}!Un jeune ecrivain viennois: M. Arthur Schnitzler1896-06-15@\strich\emph{Un jeune écrivain viennois: M. Arthur Schnitzler} {[}1896-06-15{]}|pwv}, wo er \strikeout{ſich} ſteht,
               denn ſonſt iſt man dort ſehr gegen alles Deutſche. Auch den Brief von \textsc{M. Christian Schefer\pwindex{Schefer, Christian 1866-07-14 – Februar 1944@\textsc{Schefer, Christian} (1866-07-14 – Februar 1944), \emph{Journalist, Lehrer}|pw}} lege ich bei; ſeine Adreſſe ſteht oben; nur mußt Du ſchreiben \textsc{Melun, \label{K_L02777-2v}\edtext{\begin{otherlanguage}{french}près\end{otherlanguage}}{\lemma{\textnormal{\emph{près}}}\Cendnote{\textnormal{französisch: nahe}}}\label{K_L02777-2h}
                     Paris\oindex{Melun@\textbf{Melun}|pw}}. Du dankſt ihm wohl mit einigen artigen Worten. {\pb}Wenn Du willſt, kannſt Du Dich auch gegen die
               Einwände rechtfertigen. Das wird ihm ſehr ſchmeicheln. Schreib ihm deutſch und
               entſchuldige Dich, daß Du nicht des Franzöſiſchen mächtig genug biſt, um ihm in
               ſeiner Sprache zu ſchreiben{\dotsfive}\pend
           \pstart
           Mit meiner Zuſage betreffs des Rendezvous in \label{K_L02777-3v}\edtext{Dänemark\oindex{Daenemark@\textbf{Dänemark}|pw}}{\lemma{\textnormal{\emph{Dänemark}}}\Cendnote{\textnormal{siehe Paul Goldmann an Arthur Schnitzler, 29. 4. [1896]}}}\label{K_L02777-3h} bin ich leichtſinnig geweſen. Ich habe nicht an die Koſten gedacht. Nach
               eingezogenen Erkundigungen ſtellt ſich die Eiſenbahn-Reiſe \textsc{Paris\oindex{Paris@\textbf{Paris}|pw} – Kopenhagen\oindex{Kopenhagen@\textbf{Kopenhagen}|pw} – Berlin\oindex{Berlin@\textbf{Berlin}|pw} – Paris\oindex{Paris@\textbf{Paris}|pw}} allein {\pb}auf über 230 \textsc{Francs}, mit allen Rundreiſe-Ermäßigungen. Das geht über meine Kräfte. So
               werde ich wohl \substVorne{}\textsuperscript{\textcolor{gray}{lzu}}\substDazwischen{}zu\substHinten{} meinem anfänglichen Project einer Reiſe nach der Schweiz\oindex{Schweiz@\textbf{Schweiz}|pw} zurückkehren müſſen, wo ich in einer Nacht hinkann,
               und wir werden uns in dieſem Jahre wohl kaum ſehen.\pend
           \pstart
           Wie gehts, liebſter Freund?\pend
           \pstart
           Wann trittſt Du Deine \label{K_L02777-4v}\edtext{Fahrt nach
                  Norden}{\lemma{\textnormal{\emph{Fahrt nach
                  Norden}}}\Cendnote{\textnormal{Schnitzler\pwindex{Schnitzler, Arthur 15.05.1862 – 21.10.1931@\textsc{Schnitzler, Arthur} (15.05.1862 – 21.10.1931), \emph{Schriftsteller, Mediziner}|pwk} kam am 4. 7. 1896 nach Hamburg\oindex{Hamburg@\textbf{Hamburg}|pwk}, wo er sich einschiffte. Am 9. 7. 1896 erreichte
                  er den »Norden« – er langte in Stavanger\oindex{Stavanger@\textbf{Stavanger}|pwk} (Norwegen\oindex{Norwegen@\textbf{Norwegen}|pwk}) an.}}}\label{K_L02777-4h}
               an?\pend
           \pstart
           Von Herzen Dein {\\[\baselineskip]}\spacefill\mbox{Paul Goldmann}\pend
           \leftskip=0em{}{\bigskip}\pstart
           \raggedleft{}{\pb}{[}ms.:{]} MELUN\oindex{Melun@\textbf{Melun}|pw}, 12 rue Doré\oindex{Rue Dore@\textbf{Rue Doré}|pw}, ce
                     mercredi.\pend
           \pstart{}\begin{otherlanguage}{french}Mon cher Monsieur,\end{otherlanguage}\pend\pstart
           \begin{otherlanguage}{french}\label{K_L02777-5v}\edtext{J’ai bien des excuses à vous faire
                  pour ne vous pas avoir renvoyé plus tôt, le numéro\pwindex{Neue Deutsche Rundschau1894-01-01 – 1903-12-31@\emph{Neue Deutsche Rundschau} {[}1894-01-01 – 1903-12-31{]}|pwv} de la Freie
                     Bühne\pwindex{Neue Deutsche Rundschau1894-01-01 – 1903-12-31@\emph{Neue Deutsche Rundschau} {[}1894-01-01 – 1903-12-31{]}|pw} que je mets à la poste en même temps que cette lettre. Je viens
                  d’être assez souffrant pendant plusieurs jours; sachant cela, j’espère que vous ne
                  m’en voudrez pas de mon inexactitude. – J’ai demandé à Nouvelle Revue\orgindex{Nouvelle Revue@Nouvelle Revue|pw} de vous faire parvenir, en épreuves
                  corrigées, deux ou trois exemplaires de la chronique\pwindex{Schefer, Christian 1866-07-14 – Februar 1944@\textsc{Schefer, Christian} (1866-07-14 – Februar 1944), \emph{Journalist, Lehrer}!Un jeune ecrivain viennois: M. Arthur Schnitzler1896-06-15@\strich\emph{Un jeune écrivain viennois: M. Arthur Schnitzler} {[}1896-06-15{]}|pwv} que nous allons publier sur M. Schnitzler. Vous
                  allez, je pense, les recevoir. J’ai supposé, que si vous connaissiez quelque journal\pwindex{?? Werk@Nicht ermittelte Verfasserinnen und Verfasser!Nouvelle Revue1879 – 1940@\emph{La Nouvelle Revue} {[}1879 – 1940{]}|pwv} ami de M.
                  Schnitzler, il vous serait agréable de pouvoir lui faire parvenir ce article\pwindex{Schefer, Christian 1866-07-14 – Februar 1944@\textsc{Schefer, Christian} (1866-07-14 – Februar 1944), \emph{Journalist, Lehrer}!Un jeune ecrivain viennois: M. Arthur Schnitzler1896-06-15@\strich\emph{Un jeune écrivain viennois: M. Arthur Schnitzler} {[}1896-06-15{]}|pwv} avant sa
                  publication. Ce n’est pas que l’article\pwindex{Schefer, Christian 1866-07-14 – Februar 1944@\textsc{Schefer, Christian} (1866-07-14 – Februar 1944), \emph{Journalist, Lehrer}!Un jeune ecrivain viennois: M. Arthur Schnitzler1896-06-15@\strich\emph{Un jeune écrivain viennois: M. Arthur Schnitzler} {[}1896-06-15{]}|pwv} soit aussi important que je l’eusse souhaité, mais enfin, c’est
                  le premier qui parait en France\oindex{Frankreich@\textbf{Frankreich}|pw}. D’autre part,
                  si j’ai fait, çà et là, les quelques réserves que me dictait mon désir d’être
                  parfaitement sincère, je pense néanmoins que vous ne serez pas mécontent de la
                  manière dont j’ai parlé de votre ami.}{\lemma{\textnormal{\emph{J’ai … ami.}}}\Cendnote{\textnormal{französisch: Mein lieber Herr, ich muss mich bei Ihnen entschuldigen,
                        dass ich Ihnen die Nummer\pwindex{Neue Deutsche Rundschau1894-01-01 – 1903-12-31@\emph{Neue Deutsche Rundschau} {[}1894-01-01 – 1903-12-31{]}|pwv} der Freien Bühne\pwindex{Neue Deutsche Rundschau1894-01-01 – 1903-12-31@\emph{Neue Deutsche Rundschau} {[}1894-01-01 – 1903-12-31{]}|pw}, die
                        ich zusammen mit diesem Brief auf die Post gebe, nicht früher
                        zurückgeschickt habe. Ich war gerade mehrere Tage lang ziemlich krank; das
                        wissend, hoffe ich, dass Sie mir mein Fehlverhalten nicht übel nehmen. – Ich
                        habe die Nouvelle Revue\orgindex{Nouvelle Revue@Nouvelle Revue|pw} gebeten, Ihnen
                        zwei oder drei Exemplare der Besprechung\pwindex{Schefer, Christian 1866-07-14 – Februar 1944@\textsc{Schefer, Christian} (1866-07-14 – Februar 1944), \emph{Journalist, Lehrer}!Un jeune ecrivain viennois: M. Arthur Schnitzler1896-06-15@\strich\emph{Un jeune écrivain viennois: M. Arthur Schnitzler} {[}1896-06-15{]}|pwv}, die wir über Herrn Schnitzler\pwindex{Schnitzler, Arthur 15.05.1862 – 21.10.1931@\textsc{Schnitzler, Arthur} (15.05.1862 – 21.10.1931), \emph{Schriftsteller, Mediziner}|pw} veröffentlichen werden, in korrigierten
                        Abzügen zukommen zu lassen. Ich denke, Sie werden sie erhalten. Ich habe
                        angenommen, dass Sie, wenn Sie eine Zeitung\pwindex{?? Werk@Nicht ermittelte Verfasserinnen und Verfasser!Nouvelle Revue1879 – 1940@\emph{La Nouvelle Revue} {[}1879 – 1940{]}|pw} kennen, die Herrn Schnitzler\pwindex{Schnitzler, Arthur 15.05.1862 – 21.10.1931@\textsc{Schnitzler, Arthur} (15.05.1862 – 21.10.1931), \emph{Schriftsteller, Mediziner}|pw} freundschaftlich zugetan ist, diesen Artikel\pwindex{Schefer, Christian 1866-07-14 – Februar 1944@\textsc{Schefer, Christian} (1866-07-14 – Februar 1944), \emph{Journalist, Lehrer}!Un jeune ecrivain viennois: M. Arthur Schnitzler1896-06-15@\strich\emph{Un jeune écrivain viennois: M. Arthur Schnitzler} {[}1896-06-15{]}|pwv} vor seiner
                        Veröffentlichung an diese übermitteln könnten. Nicht, dass der Artikel\pwindex{Schefer, Christian 1866-07-14 – Februar 1944@\textsc{Schefer, Christian} (1866-07-14 – Februar 1944), \emph{Journalist, Lehrer}!Un jeune ecrivain viennois: M. Arthur Schnitzler1896-06-15@\strich\emph{Un jeune écrivain viennois: M. Arthur Schnitzler} {[}1896-06-15{]}|pwv} so wichtig
                        wäre, wie ich es mir gewünscht hätte, aber es ist doch der erste, der in Frankreich\oindex{Frankreich@\textbf{Frankreich}|pw} erscheint. Ich habe zwar hier
                        und da ein paar Vorbehalte gemacht, die mir mein Wunsch nach vollkommener
                        Aufrichtigkeit diktierte, aber ich denke, dass Sie mit der Art und Weise,
                        wie ich über Ihren Freund\pwindex{Schnitzler, Arthur 15.05.1862 – 21.10.1931@\textsc{Schnitzler, Arthur} (15.05.1862 – 21.10.1931), \emph{Schriftsteller, Mediziner}|pwv} gesprochen habe, nicht unzufrieden sein werden.}}}\label{K_L02777-5h}\end{otherlanguage}\pend
           \pstart
           \label{K_L02777-6v}\edtext{\begin{otherlanguage}{french}J’ai réflechi de nouveau à tout ce que vous avez bien {\pb}voulu me dire l’autre jour, et je vais en
                  faire mon profit. Me voici, toutefois, obligé, à ma grande confusion, de vous
                  importuner encore d’une demande de renseignements. Vous m’avez signalé, les drames
                  italiens qui se jouent en Allemagne: serait abuser de votre complaisance que vous
                  prier de m’indiquer un ou deux titres? D’autre part, vous m’avez parlé des \uline{littérateurs} qui ont imité Wagner\pwindex{Wagner, Richard 22.05.1813 – 13.02.1883@\textsc{Wagner, Richard} (22.05.1813 – 13.02.1883), \emph{Komponist}|pw} et de ceux qui, ont jugé à propos, d’assassiner
                  leurs contemporains à l’aide du Stabreim: à ce propos là, encore, un ou deux noms
                  ou titres, me rempliraient de joie.\end{otherlanguage}}{\lemma{\textnormal{\emph{J’ai … joie.}}}\Cendnote{\textnormal{französisch: Ich habe noch
                     einmal über alles nachgedacht, was Sie mir damals gesagt haben und ich werde
                     meinen Nutzen daraus ziehen. Ich bin jedoch zu meiner großen Verwirrung
                     gezwungen, Sie erneut mit der Bitte um Rat zu behelligen. Sie haben mich auf
                     italienische Dramen hingewiesen, die in Deutschland\oindex{Deutschland@\textbf{Deutschland}|pw} aufgeführt werden: Wäre es ein Missbrauch Ihrer
                     Gefälligkeit, wenn ich Sie bitten würde, mir einen oder zwei Titel zu nennen?
                     Andererseits haben Sie mir von den \uline{Literaten}
                     erzählt, die Wagner\pwindex{Wagner, Richard 22.05.1813 – 13.02.1883@\textsc{Wagner, Richard} (22.05.1813 – 13.02.1883), \emph{Komponist}|pw} nachgeahmt haben,
                     und von denen, die es für angebracht hielten, ihre Zeitgenossen mit Hilfe des
                     Stabreims zu morden: auch hier würden mich ein oder zwei Namen oder Titel mit
                     Freude erfüllen.}}}\label{K_L02777-6h}\pend
           \pstart
           \label{K_L02777-7v}\edtext{\begin{otherlanguage}{french}Encore toutes mes excuses pour mon indiscrétion, et en même
                  temps que pour mes nouveaux remerciements pour les précieux renseignements que
                  vous m’avez fournis déjà, veuillez, je vous prie, Mon cher Monsieur, agréer
                  l’expression de mes sentiments les plus distingués.\end{otherlanguage}}{\lemma{\textnormal{\emph{Encore … distingués.}}}\Cendnote{\textnormal{französisch: Neuerlich bitte ich
                     um Entschuldigung für meine Unaufmerksamkeit und sende gleichzeitig erneut Dank
                     für die wertvollen Informationen, die Sie mir bereits gegeben haben. Bitte
                     nehmen Sie, mein lieber Herr, den Ausdruck meiner vornehmsten Gefühle
                     entgegen.}}}\label{K_L02777-7h}\pend
           \pstart {[}hs. Schefer:{]} \spacefill\mbox{Christian Schefer\pwindex{Schefer, Christian 1866-07-14 – Februar 1944@\textsc{Schefer, Christian} (1866-07-14 – Februar 1944), \emph{Journalist, Lehrer}|pw}.}\pend{}
         
         \endnumbering\mylabel{h}\end{ledgroupsized}  \newcommand{\dateiname}{L02777}\newcommand{\titel}{Paul Goldmann an Arthur Schnitzler, 15. 6. [1896]}\newcommand{\editorInnen}{Martin Anton Müller und Laura Untner}%% latex-leseansicht-abspann.tex
%% Abspann für die Leseansicht.
%% Der Schalter \ifkorrekturansicht ist bereits durch den Vorspann gesetzt.

%% latex-abspann.tex
%% Gemeinsamer Abspann für Korrekturansicht und Leseansicht.
%% Setzt den Schalter \ifkorrekturansicht voraus (gesetzt in den
%% einbindenden Dateien latex-korrekturansicht-abspann.tex bzw.
%% latex-leseansicht-abspann.tex).
%% ---------------------------------------------------------------

\normalsize

% Das esempio-Environment wird nur in der Leseansicht benötigt
\ifkorrekturansicht\else
\newenvironment{esempio}[3]%
{
    \vspace{1.5ex}
    \rlap{\underline{#1}}
    \par
    \setlength{\parindent}{0cm}
    \nopagebreak
    \leftskip=#2cm
    \rightskip=#3cm
}
{
    \par
}
\fi

\doendnotes{C}
\bigskip
\vfill

\clearpage

\footnotesize

\ifkorrekturansicht
  \lohead{\textsc{register}}
\fi

% theindex-Environment neu definieren ohne reledmac
\makeatletter
\renewenvironment{theindex}{%
  \ifkorrekturansicht
    \section*{\indexname}%
  \else
    \subsubsection*{Index der erwähnten Entitäten}%
  \fi
  \setlength{\parindent}{0pt}%
  \setlength{\parskip}{0pt plus 0.3pt}%
  \let\item\@idxitem
}{%
  \ifkorrekturansicht\clearpage\fi
}
\makeatother

\IfFileExists{\jobname-pw.ind}{\input{\jobname-pw.ind}}{}

% Quellenangabe nur in der Leseansicht
\ifkorrekturansicht\else
% Fallback-Definitionen, falls die .tex-Datei \titel etc. nicht gesetzt hat
\providecommand{\titel}{}
\providecommand{\editorInnen}{}
\providecommand{\dateiname}{\jobname}

\vspace{3cm}

\vfill

\footnotesize
\textsc{Quelle}: \titel. Herausgegeben von {\editorInnen}. In: \emph{Arthur Schnitzler: Briefwechsel mit Autorinnen und Autoren}.
 Digitale Edition, https://schnitzler-briefe.acdh.oeaw.ac.at/{\dateiname}.html (Stand \today)
\fi

\end{document}


      