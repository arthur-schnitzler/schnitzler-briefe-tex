%% latex-leseansicht-vorspann.tex
%% Vorspann für die Leseansicht.
%% Lädt die gemeinsame Datei latex-vorspann.tex mit nicht gesetztem Schalter.

\newif\ifkorrekturansicht
\korrekturansichtfalse

\input{../tex-inputs/latex-vorspann}


               \section[Hermann Bahr an Arthur Schnitzler, 20. 2. {[}1901{]}]{ Hermann Bahr an Arthur Schnitzler, 20. 2. {[}1901{]}}\nopagebreak\mylabel{v}\rehead{ }\begin{ledgroupsized}[t]{13cm}\normalsize\beginnumbering\briefempfaengerindex{Schnitzler, Arthur@\textsc{Schnitzler, Arthur}!zzzBahr, Hermann@\emph{von Hermann Bahr}!1901-02-201@{20. 2. 1901}|(be} \toendnotes[C]{\smallbreak\pagebreak[2]} \Standort{CUL, Schnitzler, B 5b.}
\physDesc{Brief, 1 Blatt, 2 Seiten
\newline{}Handschrift: schwarze Tinte, deutsche Kurrent
\newline{}Schnitzler: mit Bleistift die Jahreszahl »901.« ergänzt \newline{}Ordnung: mit Bleistift von unbekannter Hand nummeriert:
                                    »75« }\buchAbdrucke{\weitereDrucke{Hermann Bahr, Arthur Schnitzler: \emph{Briefwechsel, Aufzeichnungen, Dokumente (1891–1931)}. Hg. Kurt Ifkovits und Martin Anton Müller. Göttingen: \emph{Wallstein} 2018, S. 197.} }\toendnotes[C]{\smallbreak}\pstart
           \noindent{}\centering{}{\pb}\textcolor{gray}{\textbf{Redaktion des Neuen Wiener Tagblatt\orgindex{Neues Wiener Tagblatt@Neues Wiener Tagblatt|pw}}}\pend
           \pstart
           \noindent{}\centering{}\textcolor{gray}{\textbf{\textsc{Wien, I., Rothenturmstrasse,
                        Steyrerhof\oindex{Steyrerhof@\textbf{Steyrerhof}|pw}.}}}\pend
           \pstart
           \noindent{}\centering{}\textcolor{gray}{\textbf{Telegramm-Adresse: Tagblatt\orgindex{Neues Wiener Tagblatt@Neues Wiener Tagblatt|pw},
                        Steyrerhof, Wien\oindex{Steyrerhof@\textbf{Steyrerhof}|pw}. – Telephon Nr. 384.
                     Staats-Telephon Nr. 36.}}\pend
           \pstart
           20. Febr.\pend
           \pstart\center{}Lieber Arthur!\pend\pstart
           Ich habe, in einer zu meinem Kraus\pwindex{Kraus, Karl 28.04.1874 – 12.06.1936@\textsc{Kraus, Karl} (28.04.1874 – 12.06.1936), \emph{Schriftsteller, Publizist}|pw}-Proceß
               gehörenden Angelegenheit, \uline{dringendſt} mit Dir, ſo bald
               als irgend möglich, \substVorne{}\textsuperscript{\textcolor{gray}{mi}}\substDazwischen{}zu\substHinten{}{ }ſprechen und bitte Dich deshalb, mich morgen, ſo
               bald Du aufgeſtanden biſt, telephoniſch (an \textsc{Bukovics}\pwindex{Bukovics, Emerich von 28.02.1844 – 04.07.1905@\textsc{Bukovics, Emerich von} (28.02.1844 – 04.07.1905), \emph{Journalist, Theaterleiter}|pw}, \label{K_L01098_1v}\edtext{Ober St. Veiter\oindex{Ober Sankt Veit@\textbf{Ober Sankt Veit}|pw} Wohnung}{\lemma{\textnormal{\emph{Ober St. Veiter Wohnung}}}\Cendnote{\textnormal{Eigentlich ein Haus. Auf einem Teil des ursprünglich zu diesem Haus
                  gehörenden Grundstücks hatte Bahr\pwindex{Bahr, Hermann 19.07.1863 – 15.01.1934@\textsc{Bahr, Hermann} (19.07.1863 – 15.01.1934), \emph{Schriftsteller, Kritiker}|pwk} sein Haus
                  errichtet.}}}\label{K_L01098_1h}) wiſſen zu laſſen, wann und wo ich Dich treffen kann. Ich bin
               auf Dein Aviſo parat, ſofort \label{K_L01098_2v}\edtext{nach Wien\oindex{Wien@\textbf{Wien}|pw} zu fahren}{\lemma{\textnormal{\emph{nach Wien zu fahren}}}\Cendnote{\textnormal{Ober Sankt Veit\oindex{Ober Sankt Veit@\textbf{Ober Sankt Veit}|pwk} war bis zur Eingliederung in Wien\oindex{Wien@\textbf{Wien}|pwk}{ }1892 eine eigenständige Gemeinde, was sich in dieser Aussage
                  offensichtlich tradiert.}}}\label{K_L01098_2h} u. eine Stunde ſpäter überall zu ſein, wo es Dir
               paßt. Nur bitte, beſtimmt vor vier {\pb}Uhr und, wenn
               es irgendwie früher angeht, je früher, deſto beſſer.\pend
           \pstart
           Verzeih die Störung{\\[\baselineskip]}Deinem{\\[\baselineskip]}herzlich grüßenden{\\[\baselineskip]}\spacefill\mbox{HermannBahr}\pend
           \leftskip=0em{}\endnumbering\briefempfaengerindex{Schnitzler, Arthur@\textsc{Schnitzler, Arthur}!zzzBahr, Hermann@\emph{von Hermann Bahr}!1901-02-201@{20. 2. 1901}|)be}\mylabel{h}\end{ledgroupsized}  \newcommand{\dateiname}{L01098}\newcommand{\titel}{Hermann Bahr an Arthur Schnitzler, 20. 2. [1901]}\newcommand{\editorInnen}{ Kurt Ifkovits,  Martin Anton Müller}
            \footnotesize
\begin{ledgroupsized}[t]{11.5cm}
\doendnotes{C}
\end{ledgroupsized}
         %% latex-leseansicht-abspann.tex
%% Abspann für die Leseansicht.
%% Der Schalter \ifkorrekturansicht ist bereits durch den Vorspann gesetzt.

%% latex-abspann.tex
%% Gemeinsamer Abspann für Korrekturansicht und Leseansicht.
%% Setzt den Schalter \ifkorrekturansicht voraus (gesetzt in den
%% einbindenden Dateien latex-korrekturansicht-abspann.tex bzw.
%% latex-leseansicht-abspann.tex).
%% ---------------------------------------------------------------

\normalsize

% Das esempio-Environment wird nur in der Leseansicht benötigt
\ifkorrekturansicht\else
\newenvironment{esempio}[3]%
{
    \vspace{1.5ex}
    \rlap{\underline{#1}}
    \par
    \setlength{\parindent}{0cm}
    \nopagebreak
    \leftskip=#2cm
    \rightskip=#3cm
}
{
    \par
}
\fi

\doendnotes{C}
\bigskip
\vfill

\clearpage

\footnotesize

\ifkorrekturansicht
  \lohead{\textsc{register}}
\fi

% theindex-Environment neu definieren ohne reledmac
\makeatletter
\renewenvironment{theindex}{%
  \ifkorrekturansicht
    \section*{\indexname}%
  \else
    \subsubsection*{Index der erwähnten Entitäten}%
  \fi
  \setlength{\parindent}{0pt}%
  \setlength{\parskip}{0pt plus 0.3pt}%
  \let\item\@idxitem
}{%
  \ifkorrekturansicht\clearpage\fi
}
\makeatother

\IfFileExists{\jobname-pw.ind}{\input{\jobname-pw.ind}}{}

% Quellenangabe nur in der Leseansicht
\ifkorrekturansicht\else
% Fallback-Definitionen, falls die .tex-Datei \titel etc. nicht gesetzt hat
\providecommand{\titel}{}
\providecommand{\editorInnen}{}
\providecommand{\dateiname}{\jobname}

\vspace{3cm}

\vfill

\footnotesize
\textsc{Quelle}: \titel. Herausgegeben von {\editorInnen}. In: \emph{Arthur Schnitzler: Briefwechsel mit Autorinnen und Autoren}.
 Digitale Edition, https://schnitzler-briefe.acdh.oeaw.ac.at/{\dateiname}.html (Stand \today)
\fi

\end{document}


      