%% latex-korrekturansicht-vorspann.tex
%% Vorspann für die Korrekturansicht.
%% Lädt die gemeinsame Datei latex-vorspann.tex mit gesetztem Schalter.

\newif\ifkorrekturansicht
\korrekturansichttrue

\input{../tex-inputs/latex-vorspann}


\section[Hermann Bahr an Arthur Schnitzler, 20. 2. {[}1901{]}]{L01098 Hermann Bahr an Arthur Schnitzler, 20. 2. {[}1901{]}}
\nopagebreak\mylabel{L01098v}
\rehead{ }\normalsize\beginnumbering\briefempfaengerindex{Schnitzler, Arthur@\textsc{Schnitzler, Arthur}!zzzBahr, Hermann@\emph{von Hermann Bahr}!1901-02-201@{20. 2. 1901}|(be}
\toendnotes[C]{\smallbreak\pagebreak[2]}\Standort{CUL, Schnitzler, B 5b.}
\physDesc{Brief, 1 Blatt, 2 Seiten, 574 Zeichen
\newline{}Handschrift: schwarze Tinte, deutsche Kurrent
\newline{}Schnitzler: mit Bleistift die Jahreszahl »901.« ergänzt 
\newline{}Ordnung: mit Bleistift von unbekannter Hand nummeriert:
                                    »75« }
\buchAbdrucke{\weitereDrucke{Hermann Bahr, Arthur Schnitzler: \emph{Briefwechsel, Aufzeichnungen, Dokumente (1891–1931)}. Göttingen: \emph{Wallstein} 2018, S. 197.} }\toendnotes[C]{\smallbreak}
\pstart
           \centering{}{\pb}\textcolor{gray}{\textbf{Redaktion des Neuen Wiener Tagblatt\orgindex{Neues Wiener Tagblatt@Neues Wiener Tagblatt|pw}}}\pend
           
\pstart
           \centering{}\textcolor{gray}{\textbf{\textsc{Wien, I., Rothenturmstrasse,
                        Steyrerhof\oindex{Steyrerhof@\textbf{Steyrerhof}, \emph{Gebäude (K.GBD)}|pw}.}}}\pend
           
\pstart
           \centering{}\textcolor{gray}{\textbf{Telegramm-Adresse: Tagblatt\orgindex{Neues Wiener Tagblatt@Neues Wiener Tagblatt|pw}, Steyrerhof, Wien\oindex{Steyrerhof@\textbf{Steyrerhof}, \emph{Gebäude (K.GBD)}|pw}. –
                     Telephon Nr. 384. Staats-Telephon Nr. 36.}}\pend
           
\pstart
           20. Febr.\pend
           
\pstart\center{}Lieber Arthur!\pend\vspace{0.5em}
\pstart
           Ich habe, in einer zu meinem Kraus\pwindex{Kraus, Karl 28.04.1874 – 12.06.1936@\textsc{Kraus, Karl} (28.04.1874 – 12.06.1936), \emph{Schriftsteller/Schriftstellerin, Publizist/Publizistin, Schriftsteller/Schriftstellerin}|pw}-Proceß
               gehörenden Angelegenheit, \uline{dringendſt} mit Dir, ſo bald
               als irgend möglich, \substVorne{}\textsuperscript{\textcolor{gray}{mi}}\substDazwischen{}zu\substHinten{}{ }ſprechen und bitte Dich deshalb, mich morgen, ſo
               bald Du aufgeſtanden biſt, telephoniſch (an \textsc{Bukovics}\pwindex{Bukovics, Emerich von 28.02.1844 – 04.07.1905@\textsc{Bukovics, Emerich von} (28.02.1844 – 04.07.1905), \emph{Journalist/Journalistin, Theaterleiter/Theaterleiterin}|pw}, \label{K_L01098-1v}\edtext{Ober St. Veiter\oindex{Ober Sankt Veit@\textbf{Ober Sankt Veit}, \emph{P.PPLX}|pw} Wohnung}{\lemma{\textnormal{\emph{Ober St. Veiter Wohnung}}}\Cendnote{\textnormal{Es war ein Wohnhaus. Auf einem Teil des ursprünglich zu diesem
                  Haus gehörenden Grundstücks hatte Bahr\pwindex{Bahr, Hermann 19.07.1863 – 15.01.1934@\textsc{Bahr, Hermann} (19.07.1863 – 15.01.1934), \emph{Schriftsteller/Schriftstellerin, Kritiker/Kritikerin}|pwk} seine kleine
                  Villa errichtet.}}}\label{K_L01098-1}) wiſſen zu laſſen, wann und wo ich Dich treffen kann. Ich
               bin auf Dein Aviſo parat, ſofort \label{K_L01098-2v}\edtext{nach
                  Wien\oindex{Wien@\textbf{Wien}, \emph{A.ADM2}|pw} zu fahren}{\lemma{\textnormal{\emph{nach
                  Wien zu fahren}}}\Cendnote{\textnormal{Ober Sankt Veit\oindex{Ober Sankt Veit@\textbf{Ober Sankt Veit}, \emph{P.PPLX}|pwk} war bis zur Eingliederung in
                     Wien\oindex{Wien@\textbf{Wien}, \emph{A.ADM2}|pwk}{ }1892 eine eigenständige Gemeinde, was sich in dieser Aussage
                  offensichtlich tradiert.}}}\label{K_L01098-2} u. eine Stunde ſpäter überall zu ſein, wo es Dir
               paßt. Nur bitte, beſtimmt vor vier {\pb}Uhr und, wenn
               es irgendwie früher angeht, je früher, deſto beſſer.\pend
           
\pstart
           Verzeih die Störung{\\[\baselineskip]}Deinem{\\[\baselineskip]}herzlich grüßenden{\\[\baselineskip]}\spacefill\mbox{HermannBahr}\pend
           \leftskip=0em{}\selectlanguage{ngerman}\endnumbering\briefempfaengerindex{Schnitzler, Arthur@\textsc{Schnitzler, Arthur}!zzzBahr, Hermann@\emph{von Hermann Bahr}!1901-02-201@{20. 2. 1901}|)be}\mylabel{L01098h}  \normalsize

\doendnotes{C}
\bigskip
\vfill

\clearpage

\footnotesize

\lohead{\textsc{register}}

% Definiere theindex-Environment komplett neu ohne reledmac
\makeatletter
\renewenvironment{theindex}{%
  \section*{\indexname}%
  \setlength{\parindent}{0pt}%
  \setlength{\parskip}{0pt plus 0.3pt}%
  \let\item\@idxitem
}{%
  \clearpage
}
\makeatother

\IfFileExists{\jobname-pw.ind}{\input{\jobname-pw.ind}}{}

\end{document}

      