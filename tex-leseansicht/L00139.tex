%% latex-korrekturansicht-vorspann.tex
%% Vorspann für die Korrekturansicht.
%% Lädt die gemeinsame Datei latex-vorspann.tex mit gesetztem Schalter.

\newif\ifkorrekturansicht
\korrekturansichttrue

\input{../tex-inputs/latex-vorspann}


\section[Arthur Schnitzler an Hugo von Hofmannsthal, 24. 11. 1892]{L00139 Arthur Schnitzler an Hugo von Hofmannsthal, 24. 11. 1892}
\nopagebreak\mylabel{L00139v}
\rehead{ }\normalsize\beginnumbering\briefempfaengerindex{Hofmannsthal, Hugo von@\textsc{Hofmannsthal, Hugo von}!zzzSchnitzler, Arthur@\emph{von Arthur Schnitzler}!1892-11-241@{24. 11. 1892}|(be}
\toendnotes[C]{\smallbreak\pagebreak[2]}\Standort{FDH, Hs-30885,27.}
\physDesc{Brief, 1 Blatt, 3 Seiten, 694 Zeichen
\newline{}Handschrift: Bleistift, deutsche Kurrent
\newline{}Ordnung: mit Bleistift von Schnitzler mutmaßlich während der Durchsicht der Briefe
                                    1929 am oberen Rand der ersten Seite datiert: »24/11 92« }
\buchAbdrucke{\weitereDrucke{1) Hugo von Hofmannsthal, Arthur Schnitzler: \emph{Briefwechsel}. Frankfurt am Main: \emph{S. Fischer} 1964, S. 31–32.} \weitereDrucke{2) Hermann Bahr, Arthur Schnitzler: \emph{Briefwechsel, Aufzeichnungen, Dokumente (1891–1931)}. Göttingen: \emph{Wallstein} 2018.} }\toendnotes[C]{\smallbreak}
\pstart{}{\pb}Lieber Loris,\pend\vspace{0.5em}
\pstart
           ſehr wahr! – Und wie denken Sie z. B. darüber, für einen Abend der Woche ſtatt des
                  Pfob\oindex{Cafe Pfob@\textbf{Café Pfob}, \emph{Kaffeehaus (K.KAF)}|pw} ein anderes Café zu beſti{\geminationm}en, in dem \uline{nur}{ }\uline{wir} zuſa{\geminationm}en ko{\geminationm}en? – Und eventuell Bahr\pwindex{Bahr, Hermann 19.07.1863 – 15.01.1934@\textsc{Bahr, Hermann} (19.07.1863 – 15.01.1934), \emph{Schriftsteller/Schriftstellerin, Kritiker/Kritikerin}|pw}. Ich wiederhole übrigens, was ich Ihnen ſchon \label{K_L00139-1v}\edtext{neulich geſchrieben}{\lemma{\textnormal{\emph{neulich geſchrieben}}}\Cendnote{\textnormal{Siehe Arthur Schnitzler an Hugo von Hofmannsthal, 9. 11. 1892.}}}\label{K_L00139-1}, daſs ich nämlich ſehr {\pb}unangenehm
               enttäuſcht bin, auch heuer ſo wenig mit Ihnen zuſa{\geminationm}en zu
                  ko{\geminationm}en.\pend
           
\pstart
           Beſti{\geminationm}en Sie Abend, beſti{\geminationm}en Sie Caféhaus – und beſti{\geminationm}en Sie \substVorne{}\textsuperscript{und}\substDazwischen{}vielleicht\substHinten{} auch Bahr\pwindex{Bahr, Hermann 19.07.1863 – 15.01.1934@\textsc{Bahr, Hermann} (19.07.1863 – 15.01.1934), \emph{Schriftsteller/Schriftstellerin, Kritiker/Kritikerin}|pw}, einmal hinzuko{\geminationm}en.\pend
           
\pstart
           \label{K_L00139-2v}\edtext{So{\geminationn}tag
               alſo bei mir}{\lemma{\textnormal{\emph{Sonntag
               alſo bei mir}}}\Cendnote{\textnormal{Am 27. 11. 1892 ist lediglich der Besuch Hofmannsthals\pwindex{Hofmannsthal, Hugo von 1874-02-01 – 1929-07-15@\textsc{Hofmannsthal, Hugo von} (1874-02-01 – 1929-07-15), \emph{Schriftsteller/Schriftstellerin}|pwk} in Schnitzlers{ }\emph{Tagebuch}\pwindex{Tagebuch@\emph{Tagebuch}|pwk}
                  erwähnt.}}}\label{K_L00139-2}, für alle Fälle? – Ich möchte mir den Vorſchlag erlauben, daſs Sie
                  {\pb}Ihre \textsc{psychol.}{ }\label{K_L00139-3v}\edtext{Novellette\pwindex{Age of Innocence@\emph{Age of Innocence}|pwv}}{\lemma{\textnormal{\emph{Novellette}}}\Cendnote{\textnormal{\emph{Age of Innocence}\pwindex{Age of Innocence@\emph{Age of Innocence}|pwk} (postum veröffentlicht
                     1930)}}}\label{K_L00139-3} (die von der \textsc{Freien Bühne}\pwindex{Freie Buehne fuer den Entwickelungskampf der Zeit@\emph{Freie Bühne für den Entwickelungskampf der Zeit}|pw} refüſirt wurde) vorleſen. Ich glaube, daſs weder \textsc{Richard}\pwindex{Beer-Hofmann, Richard 1866-07-11 – 1945-09-26@\textsc{Beer-Hofmann, Richard} (1866-07-11 – 1945-09-26), \emph{Schriftsteller/Schriftstellerin}|pw} noch \textsc{Salten}\pwindex{Salten, Felix 06.09.1869 – 08.10.1945@\textsc{Salten, Felix} (06.09.1869 – 08.10.1945), \emph{Schriftsteller/Schriftstellerin, Journalist/Journalistin, Chefredakteur/Chefredakteurin}|pw} dieſelbe kennen. –\pend
           
\pstart
           Herzlich der Ihre{\\[\baselineskip]}\spacefill\mbox{Arthur}\pend
           \leftskip=0em{}
\pstart
           Wien\oindex{Wien@\textbf{Wien}, \emph{A.ADM2}|pw}{ }24. XI. 92.\pend
           \selectlanguage{ngerman}\endnumbering\briefempfaengerindex{Hofmannsthal, Hugo von@\textsc{Hofmannsthal, Hugo von}!zzzSchnitzler, Arthur@\emph{von Arthur Schnitzler}!1892-11-241@{24. 11. 1892}|)be}\mylabel{L00139h}  \normalsize

\doendnotes{C}
\bigskip
\vfill

\clearpage

\footnotesize

\lohead{\textsc{register}}

% Definiere theindex-Environment komplett neu ohne reledmac
\makeatletter
\renewenvironment{theindex}{%
  \section*{\indexname}%
  \setlength{\parindent}{0pt}%
  \setlength{\parskip}{0pt plus 0.3pt}%
  \let\item\@idxitem
}{%
  \clearpage
}
\makeatother

\IfFileExists{\jobname-pw.ind}{\input{\jobname-pw.ind}}{}

\end{document}

      