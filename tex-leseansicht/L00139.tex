%% latex-leseansicht-vorspann.tex
%% Vorspann für die Leseansicht.
%% Lädt die gemeinsame Datei latex-vorspann.tex mit nicht gesetztem Schalter.

\newif\ifkorrekturansicht
\korrekturansichtfalse

\input{../tex-inputs/latex-vorspann}


         
         \renewcommand{\erwaehntePersonen}{Personen: Hermann Bahr, Richard Beer-Hofmann, Hugo von Hofmannsthal, Felix Salten}
         \renewcommand{\erwaehnteOrte}{Orte: Café Pfob, Wien}
         \renewcommand{\erwaehnteWerke}{Werke: Age of Innocence, Freie Bühne für den Entwickelungskampf der Zeit, Tagebuch}
               \section[Arthur Schnitzler an Hugo von Hofmannsthal, 24. 11. 1892]{ Arthur Schnitzler an Hugo von Hofmannsthal, 24. 11. 1892}\nopagebreak\mylabel{v}\rehead{ }\begin{ledgroupsized}[t]{13cm}\normalsize\beginnumbering \toendnotes[C]{\smallbreak\pagebreak[2]} \Standort{FDH, Hs-30885,27.}
\physDesc{Brief, 1 Blatt, 3 Seiten
\newline{}Handschrift: Bleistift, deutsche Kurrent\newline{}Ordnung: mit Bleistift von Schnitzler mutmaßlich während der Durchsicht
                                 der Briefe 1929 am oberen Rand der ersten Seite
                                 datiert: »24/11 92« }\buchAbdrucke{\weitereDrucke{1) Hugo von Hofmannsthal, Arthur Schnitzler: \emph{Briefwechsel}. Hg. Therese Nickl und Heinrich Schnitzler. Frankfurt am Main: \emph{S. Fischer} 1964, S. 31–32.} \weitereDrucke{2) Hermann Bahr, Arthur Schnitzler: \emph{Briefwechsel, Aufzeichnungen, Dokumente (1891–1931)}. Hg. Kurt Ifkovits und Martin Anton Müller. Göttingen: \emph{Wallstein} 2018.} }\toendnotes[C]{\smallbreak}\pstart{}{\pb}Lieber Loris,\pend\pstart
           ſehr wahr! – Und wie denken Sie z. B. darüber, für einen Abend der Woche ſtatt des
                  Pfob\oindex{Cafe Pfob@\textbf{Café Pfob}|pw} ein anderes Café zu beſti{\geminationm}en, in dem \uline{nur}{ }\uline{wir} zuſa{\geminationm}en ko{\geminationm}en? – Und eventuell Bahr\pwindex{Bahr, Hermann 19.07.1863 – 15.01.1934@\textsc{Bahr, Hermann} (19.07.1863 – 15.01.1934), \emph{Schriftsteller, Kritiker}|pw}. Ich wiederhole übrigens, was ich Ihnen ſchon \label{K_L00139_1v}\edtext{neulich geſchrieben}{\lemma{\textnormal{\emph{neulich geſchrieben}}}\Cendnote{\textnormal{am 9. 11. 1892 (\emph{Briefwechsel} Hofmannsthal/Schnitzler
                  31).}}}\label{K_L00139_1h}, daſs ich nämlich ſehr {\pb}unangenehm
               enttäuſcht bin, auch heuer ſo wenig mit Ihnen zuſa{\geminationm}en zu
                  ko{\geminationm}en.\pend
           \pstart
           Beſti{\geminationm}en Sie Abend, beſti{\geminationm}en Sie Caféhaus – und beſti{\geminationm}en Sie \substVorne{}\textsuperscript{und}\substDazwischen{}vielleicht\substHinten{} auch Bahr\pwindex{Bahr, Hermann 19.07.1863 – 15.01.1934@\textsc{Bahr, Hermann} (19.07.1863 – 15.01.1934), \emph{Schriftsteller, Kritiker}|pw}, einmal hinzuko{\geminationm}en.\pend
           \pstart
           \label{K_L00139_2v}\edtext{So{\geminationn}tag
               alſo bei mir}{\lemma{\textnormal{\emph{Sonntag
               alſo bei mir}}}\Cendnote{\textnormal{Am 27. 11. 1892 ist lediglich der Besuch Hofmannsthals\pwindex{Hofmannsthal, Hugo von 1874-02-01 – 1929-07-15@\textsc{Hofmannsthal, Hugo von} (1874-02-01 – 1929-07-15), \emph{Schriftsteller}|pwk} in Schnitzler\pwindex{Schnitzler, Arthur 15.05.1862 – 21.10.1931@\textsc{Schnitzler, Arthur} (15.05.1862 – 21.10.1931), \emph{Schriftsteller, Mediziner}|pwk}s \emph{Tagebuch}\pwindex{Schnitzler, Arthur 15.05.1862 – 21.10.1931@\textsc{Schnitzler, Arthur} (15.05.1862 – 21.10.1931), \emph{Schriftsteller, Mediziner}!Tagebuch1981 – 2000@\strich\emph{Tagebuch} {[}1981 – 2000{]}|pwk} erwähnt.}}}\label{K_L00139_2h},
               für alle Fälle? – Ich möchte mir den Vorſchlag erlauben, daſs Sie {\pb}Ihre \textsc{psychol.}{ }\label{K_L00139_3v}\edtext{Novellette\pwindex{Hofmannsthal, Hugo von 1874-02-01 – 1929-07-15@\textsc{Hofmannsthal, Hugo von} (1874-02-01 – 1929-07-15), \emph{Schriftsteller}!Age of Innocence1930@\strich\emph{Age of Innocence} {[}1930{]}|pwv}}{\lemma{\textnormal{\emph{Novellette}}}\Cendnote{\textnormal{\emph{Age of Innocence}\pwindex{Hofmannsthal, Hugo von 1874-02-01 – 1929-07-15@\textsc{Hofmannsthal, Hugo von} (1874-02-01 – 1929-07-15), \emph{Schriftsteller}!Age of Innocence1930@\strich\emph{Age of Innocence} {[}1930{]}|pwk} (postum veröffentlicht
                     1930).}}}\label{K_L00139_3h} (die von der \textsc{Freien Bühne}\pwindex{Freie Buehne fuer den Entwickelungskampf der Zeit1892 – 1893@\emph{Freie Bühne für den Entwickelungskampf der Zeit} {[}1892 – 1893{]}|pw} refüſirt wurde) vorleſen. Ich glaube, daſs weder \textsc{Richard}\pwindex{Beer-Hofmann, Richard 1866-07-11 – 1945-09-26@\textsc{Beer-Hofmann, Richard} (1866-07-11 – 1945-09-26), \emph{Schriftsteller}|pw} noch \textsc{Salten}\pwindex{Salten, Felix 06.09.1869 – 08.10.1945@\textsc{Salten, Felix} (06.09.1869 – 08.10.1945), \emph{Schriftsteller, Journalist}|pw} dieſelbe kennen. –\pend
           \pstart
           Herzlich der Ihre{\\[\baselineskip]}\spacefill\mbox{Arthur}\pend
           \leftskip=0em{}\pstart
           Wien\oindex{Wien@\textbf{Wien}|pw}{ }24. XI. 92.\pend
           
         
         \endnumbering\mylabel{h}\end{ledgroupsized}  \newcommand{\dateiname}{L00139}\newcommand{\titel}{Arthur Schnitzler an Hugo von Hofmannsthal, 24. 11. 1892}\newcommand{\editorInnen}{ Martin Anton Müller und Gerd-Hermann Susen}%% latex-leseansicht-abspann.tex
%% Abspann für die Leseansicht.
%% Der Schalter \ifkorrekturansicht ist bereits durch den Vorspann gesetzt.

%% latex-abspann.tex
%% Gemeinsamer Abspann für Korrekturansicht und Leseansicht.
%% Setzt den Schalter \ifkorrekturansicht voraus (gesetzt in den
%% einbindenden Dateien latex-korrekturansicht-abspann.tex bzw.
%% latex-leseansicht-abspann.tex).
%% ---------------------------------------------------------------

\normalsize

% Das esempio-Environment wird nur in der Leseansicht benötigt
\ifkorrekturansicht\else
\newenvironment{esempio}[3]%
{
    \vspace{1.5ex}
    \rlap{\underline{#1}}
    \par
    \setlength{\parindent}{0cm}
    \nopagebreak
    \leftskip=#2cm
    \rightskip=#3cm
}
{
    \par
}
\fi

\doendnotes{C}
\bigskip
\vfill

\clearpage

\footnotesize

\ifkorrekturansicht
  \lohead{\textsc{register}}
\fi

% theindex-Environment neu definieren ohne reledmac
\makeatletter
\renewenvironment{theindex}{%
  \ifkorrekturansicht
    \section*{\indexname}%
  \else
    \subsubsection*{Index der erwähnten Entitäten}%
  \fi
  \setlength{\parindent}{0pt}%
  \setlength{\parskip}{0pt plus 0.3pt}%
  \let\item\@idxitem
}{%
  \ifkorrekturansicht\clearpage\fi
}
\makeatother

\IfFileExists{\jobname-pw.ind}{\input{\jobname-pw.ind}}{}

% Quellenangabe nur in der Leseansicht
\ifkorrekturansicht\else
% Fallback-Definitionen, falls die .tex-Datei \titel etc. nicht gesetzt hat
\providecommand{\titel}{}
\providecommand{\editorInnen}{}
\providecommand{\dateiname}{\jobname}

\vspace{3cm}

\vfill

\footnotesize
\textsc{Quelle}: \titel. Herausgegeben von {\editorInnen}. In: \emph{Arthur Schnitzler: Briefwechsel mit Autorinnen und Autoren}.
 Digitale Edition, https://schnitzler-briefe.acdh.oeaw.ac.at/{\dateiname}.html (Stand \today)
\fi

\end{document}


      