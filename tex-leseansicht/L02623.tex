%% latex-korrekturansicht-vorspann.tex
%% Vorspann für die Korrekturansicht.
%% Lädt die gemeinsame Datei latex-vorspann.tex mit gesetztem Schalter.

\newif\ifkorrekturansicht
\korrekturansichttrue

\input{../tex-inputs/latex-vorspann}


\section[Paul Goldmann an Arthur Schnitzler, 1. 6. {[}1894{]}]{L02623 Paul Goldmann an Arthur Schnitzler, 1. 6. {[}1894{]}}
\nopagebreak\mylabel{L02623v}
\rehead{ }\normalsize\beginnumbering\briefempfaengerindex{Schnitzler, Arthur@\textsc{Schnitzler, Arthur}!zzzGoldmann, Paul@\emph{von Paul Goldmann}!1894-06-012@{1. 6. {[}1894{]}}|(be}
\toendnotes[C]{\smallbreak\pagebreak[2]}\Standort{DLA, A:Schnitzler, HS.NZ85.1.3164.}
\physDesc{Brief, 2 Blätter, 6 Seiten, 3193 Zeichen
\newline{}Handschrift: schwarze Tinte, deutsche Kurrent
\newline{}Schnitzler: 1) mit Bleistift auf dem ersten Blatt die Jahreszahl »94« vermerkt  2) mit rotem Buntstift vier Unterstreichungen}\toendnotes[C]{\smallbreak}
\pstart
           {\pb}\textcolor{gray}{\textbf{Frankfurter Zeitung\orgindex{Frankfurter Zeitung@Frankfurter Zeitung|pw}}}\hfill \textsc{Paris\oindex{Paris@\textbf{Paris}, \emph{P.PPLC}|pw}}, 1. Juni.\pend
           
\pstart
           \textcolor{gray}{\textbf{(Gazette de
                     Francfort\orgindex{Frankfurter Zeitung@Frankfurter Zeitung|pw}).}}\pend
           
\pstart
           \textcolor{gray}{\textbf{Fondateur \textbf{M. L. Sonnemann\pwindex{Sonnemann, Leopold 1831-10-29 – 1909-10-30@\textsc{Sonnemann, Leopold} (1831-10-29 – 1909-10-30), \emph{Journalist/Journalistin, Herausgeber/Herausgeberin}|pw}}.}}\pend
           
\pstart
           \textcolor{gray}{\textbf{\begin{otherlanguage}{french}Journal politique, financier,\end{otherlanguage}}}\pend
           
\pstart
           \textcolor{gray}{\textbf{\begin{otherlanguage}{french}commercial et littéraire.\end{otherlanguage}}}\pend
           
\pstart
           \textcolor{gray}{\textbf{\begin{otherlanguage}{french}\textbf{Paraissant trois fois par jour.}\end{otherlanguage}}}\pend
           
\pstart
           \textcolor{gray}{\textbf{\begin{otherlanguage}{french}\textbf{Bureau à Paris\oindex{Paris@\textbf{Paris}, \emph{P.PPLC}|pw}:}\end{otherlanguage}}}\pend
           
\pstart
           \textcolor{gray}{\textbf{\begin{otherlanguage}{french}24. Rue Feydeau\oindex{rue Feydeau@\textbf{rue Feydeau}, \emph{Straße (K.STR)}|pw}.\end{otherlanguage}}}\pend
           
\pstart\center{}Mein lieber Freund,\pend\vspace{0.5em}
\pstart
           \textsc{Hermann Bahr}\pwindex{Bahr, Hermann 19.07.1863 – 15.01.1934@\textsc{Bahr, Hermann} (19.07.1863 – 15.01.1934), \emph{Schriftsteller/Schriftstellerin, Kritiker/Kritikerin}|pw} iſt alſo doch bei mir geweſen; aber ich wünſchte, es wäre lieber nicht
               geſchehen. Er hat mir einen abſcheulichen Eindruck gemacht, – ein Intriguant, ein
               Jeſuit – und wenn, wie dies wahrſcheinlich, ſeine Geſinnung der meinigen gleicht, ſo
               ſind wir, mit einem herzlichen Händedruck, als erklärte Feinde geſchieden. Der Mann\pwindex{Bahr, Hermann 19.07.1863 – 15.01.1934@\textsc{Bahr, Hermann} (19.07.1863 – 15.01.1934), \emph{Schriftsteller/Schriftstellerin, Kritiker/Kritikerin}|pwv} hat mir in der kurzen
               Zeit ſeines Hier-Seins mehr Stänkereien angerichtet, als ſonſt irgend Einer, hat mich
               aus meiner Sicherheit {\pb}gebracht und mich durch
               allerlei Perfidie erregt und verſtimmt. Es wäre zu weitläufig, das hier zu erzählen;
               der Mensch\pwindex{Bahr, Hermann 19.07.1863 – 15.01.1934@\textsc{Bahr, Hermann} (19.07.1863 – 15.01.1934), \emph{Schriftsteller/Schriftstellerin, Kritiker/Kritikerin}|pwv}, der hier mit
               einem infamen Pack von Reportern niedrigſter Sorte verkehrt, hat ſich dort allerlei
               Verleumdungen über mich geholt, die er mir, mit liebenswürdigem Wohlwollen, wieder
               erzählt hat. Ich berühre das nur, um Dich davor zu warnen, irgendwelchen
               freundſchaftlichen Referaten aus dieſer Quelle Glauben zu ſchenken. Der Grund,
               weshalb ich mich heut an Dich wende, iſt ein \strikeout{b}
               anderer. Er liegt in Einigem, was mir der Herr\pwindex{Bahr, Hermann 19.07.1863 – 15.01.1934@\textsc{Bahr, Hermann} (19.07.1863 – 15.01.1934), \emph{Schriftsteller/Schriftstellerin, Kritiker/Kritikerin}|pwv} über Euch geſagt hat. Zunächſt ſelbſtverſtändlich ſpielt
               er ſich als den eigentlichen Förderer und {\pb}Inſpirator der \label{K_L02623-1v}\edtext{Wien\oindex{Wien@\textbf{Wien}, \emph{A.ADM2}|pw}er Literatur-Strömung}{\lemma{\textnormal{\emph{Wiener Literatur-Strömung}}}\Cendnote{\textnormal{Bei »Jung Wien« handelte es sich um eine losen Verbund von
                  Autoren ohne gemeinsames Programm. Unter diesem Namen agierte kurze Zeit ein
                  Verein, der sich zumindest zwischen 17. 3. 1891 und 5. 5. 1891 wöchentlich traf. Einen Anspruch auf
                  Popularisierung der neuen Strömung und damit auch auf eine Rolle als ihr Ausformer
                  konnte Bahr\pwindex{Bahr, Hermann 19.07.1863 – 15.01.1934@\textsc{Bahr, Hermann} (19.07.1863 – 15.01.1934), \emph{Schriftsteller/Schriftstellerin, Kritiker/Kritikerin}|pwk} damit begründen, dass er in
                  einem dreiteiligen Feuilleton, \emph{Das junge
                     Österreich}\pwindex{junge Oesterreich@\emph{Das junge Österreich}|pwk}, das zuerst am 20. 9. 1893, am
                     27. 9. 1893 und am 7. 10. 1893 in der \emph{Deutschen Zeitung}\pwindex{Deutsche Zeitung@\emph{Deutsche Zeitung}|pwk} erschienen war, erstmals eine gemeinsame
                  Sichtung unternommen hatte (Jg. 23, Nr. 7806, Morgen-Ausgabe, S. 1–2; Nr. 7813,
                     Morgen-Ausgabe, S. 1–3; Nr. 7823, Morgen-Ausgabe, S. 1–3). Im Folgejahr
                  nahm er es in die Zusammenstellung von Texten \emph{Studien zur Kritik der Moderne}\pwindex{Studien zur Kritik der Moderne@\emph{Studien zur Kritik der Moderne}|pwk} (Frankfurt am Main: \emph{Literarische Anstalt Rütten {\kaufmannsund} Loening}\orgindex{Ruetten und Loening@Rütten {\kaufmannsund}  Loening|pwk}) auf. Das »Euch« dürfte dabei auf die bleibendsten dieser Autoren gemünzt
                  sein, die privat in regelmäßigem Umgang mit Schnitzler standen, vor allem Richard
                     Beer-Hofmann\pwindex{Beer-Hofmann, Richard 1866-07-11 – 1945-09-26@\textsc{Beer-Hofmann, Richard} (1866-07-11 – 1945-09-26), \emph{Schriftsteller/Schriftstellerin}|pwk}, Hugo von Hofmannsthal\pwindex{Hofmannsthal, Hugo von 1874-02-01 – 1929-07-15@\textsc{Hofmannsthal, Hugo von} (1874-02-01 – 1929-07-15), \emph{Schriftsteller/Schriftstellerin}|pwk}
                  und Felix Salten\pwindex{Salten, Felix 06.09.1869 – 08.10.1945@\textsc{Salten, Felix} (06.09.1869 – 08.10.1945), \emph{Schriftsteller/Schriftstellerin, Journalist/Journalistin, Chefredakteur/Chefredakteurin}|pwk}.}}}\label{K_L02623-1} auf. Zu gleicher
               Zeit hat er über jeden von Euch bei aller ſcheinbaren Anerkennung irgend ein
               herabſetzendes Wort, ſo daß von der Wien\oindex{Wien@\textbf{Wien}, \emph{A.ADM2}|pw}er
               Literatur eigentlich als vollgiltig nur Hermann \textsc{Bahr}\pwindex{Bahr, Hermann 19.07.1863 – 15.01.1934@\textsc{Bahr, Hermann} (19.07.1863 – 15.01.1934), \emph{Schriftsteller/Schriftstellerin, Kritiker/Kritikerin}|pw} übrig bleibt. Selbſt die Leute ſeiner eigenen Revüe\pwindex{Zeit. Wiener Wochenschrift@\emph{Die Zeit. Wiener Wochenschrift}|pwv} drückt er herunter. \textsc{Kanner\pwindex{Kanner, Heinrich 09.11.1864 – 15.02.1930@\textsc{Kanner, Heinrich} (09.11.1864 – 15.02.1930), \emph{Herausgeber/Herausgeberin, Publizist/Publizistin}|pw}}{ }\strikeout{iſt} wird ſich nach ſeiner Darſtellung mit der
               Adminiſtration befaſſen; und wenn \strikeout{n} man \textsc{Kanner\pwindex{Kanner, Heinrich 09.11.1864 – 15.02.1930@\textsc{Kanner, Heinrich} (09.11.1864 – 15.02.1930), \emph{Herausgeber/Herausgeberin, Publizist/Publizistin}|pw}} nur aus ſeinen Reden kennt, ſo muß man ihn für nichts als für einen Kaſſier
               halten, während doch in Wahrheit \textsc{Kanner\pwindex{Kanner, Heinrich 09.11.1864 – 15.02.1930@\textsc{Kanner, Heinrich} (09.11.1864 – 15.02.1930), \emph{Herausgeber/Herausgeberin, Publizist/Publizistin}|pw}} der \strikeout{Ein} Einzige iſt, der für die {\pb}\textsc{Revue\pwindex{Zeit. Wiener Wochenschrift@\emph{Die Zeit. Wiener Wochenschrift}|pwv}} Zukunfts-Hoffnungen rechtfertigt. Nun aber zu Euch zurück. Ich möchte Dich
               bitten, mir mit ein paar Worten etwas über das Verhältniß von \textsc{Hermann Bahr\pwindex{Bahr, Hermann 19.07.1863 – 15.01.1934@\textsc{Bahr, Hermann} (19.07.1863 – 15.01.1934), \emph{Schriftsteller/Schriftstellerin, Kritiker/Kritikerin}|pw}} zu Eurem Kreiſe zu ſagen. Insbeſondere möchte ich wiſſen, ob zwiſchen ihm und
                  \label{K_L02623-2v}\edtext{\textsc{Loris\pwindex{Hofmannsthal, Hugo von 1874-02-01 – 1929-07-15@\textsc{Hofmannsthal, Hugo von} (1874-02-01 – 1929-07-15), \emph{Schriftsteller/Schriftstellerin}|pw}} wirklich jene intime Freundſchaft}{\lemma{\textnormal{\emph{Loris … Freundſchaft}}}\Cendnote{\textnormal{Ohne Schnitzlers Antwort zu kennen, finden
                  sich in seinem \emph{Tagebuch}\pwindex{Tagebuch@\emph{Tagebuch}|pwk} doch mehrfach
                  Aussagen, die die bestehende Nähe zwischen Bahr\pwindex{Bahr, Hermann 19.07.1863 – 15.01.1934@\textsc{Bahr, Hermann} (19.07.1863 – 15.01.1934), \emph{Schriftsteller/Schriftstellerin, Kritiker/Kritikerin}|pwk} und Hofmannsthal\pwindex{Hofmannsthal, Hugo von 1874-02-01 – 1929-07-15@\textsc{Hofmannsthal, Hugo von} (1874-02-01 – 1929-07-15), \emph{Schriftsteller/Schriftstellerin}|pwk} kritisch
                  beurteilen, beispielsweise A. S.: \emph{Tagebuch}, 6. 11. 1895,
                  aber auch Goldmann\pwindex{Goldmann, Paul 31.01.1865 – 25.09.1935@\textsc{Goldmann, Paul} (31.01.1865 – 25.09.1935), \emph{Schriftsteller/Schriftstellerin, Journalist/Journalistin}|pwk} beschäftigte das Thema
                  länger, vgl. A. S.: \emph{Tagebuch}, 26. 8. 1895.
               }}}\label{K_L02623-2} beſteht, \strikeout{die} wie er vorgibt; ob er wirklich
               berechtigt iſt, ſich als den »Erzieher\pwindex{Bahr, Hermann 19.07.1863 – 15.01.1934@\textsc{Bahr, Hermann} (19.07.1863 – 15.01.1934), \emph{Schriftsteller/Schriftstellerin, Kritiker/Kritikerin}|pwv}« von \textsc{Loris\pwindex{Hofmannsthal, Hugo von 1874-02-01 – 1929-07-15@\textsc{Hofmannsthal, Hugo von} (1874-02-01 – 1929-07-15), \emph{Schriftsteller/Schriftstellerin}|pw}} aufzuſpielen, wie er das thut \textsc{etc}. Bitte, ſchreib’
               mir bald; denn das Alles quält mich ſehr ſeit geſtern{ }Abend. Ich will Dir nicht ſagen, warum, ſondern Deine Antwort
               abwarten.\pend
           
\pstart
           Herzlichſt und in Treue {\\[\baselineskip]}Dein \spacefill\mbox{Paul Goldmann.}\pend
           \leftskip=0em{}
\pstart
           \noindent{}{\pb}Ja ſo, entſchuldige, in meiner Erregung hätte ich
                  beinahe Deine Angelegenheiten vergeſſen. Der Verleger \textsc{Albert Langen}\pwindex{Langen, Albert 1869-07-08 – 1909-04-30@\textsc{Langen, Albert} (1869-07-08 – 1909-04-30), \emph{Verleger/Verlegerin}|pw} iſt ein reicher junger Menſch, der ſich zum Verleger gemacht hat, um mit
                  Literatur protzen zu können. Der Menſch\pwindex{Langen, Albert 1869-07-08 – 1909-04-30@\textsc{Langen, Albert} (1869-07-08 – 1909-04-30), \emph{Verleger/Verlegerin}|pwv} iſt idiotiſch urtheilslos, \strikeout{und} verlogen und betrügeriſch. Er iſt von dem halb wahnſinnigen \textsc{Gretor\pwindex{Gretor, Willy 1868-07-16 – 1923-07-31@\textsc{Gretor, Willy} (1868-07-16 – 1923-07-31), \emph{Maler/Malerin, Kunstagent/Kunstagentin, Kunsthändler/Kunsthändlerin}|pw}} beeinflußt, von dem ich Dir im vorigen Sommer erzählt. Ich rathe Dir
                  dringend, Dich \label{K_L02623-3v}\edtext{mit dem Burſchen in
                  nichts {\pb}einzulaſſen}{\lemma{\textnormal{\emph{mit … einzulaſſen}}}\Cendnote{\textnormal{In Langens\pwindex{Langen, Albert 1869-07-08 – 1909-04-30@\textsc{Langen, Albert} (1869-07-08 – 1909-04-30), \emph{Verleger/Verlegerin}|pwk}{ }\emph{Simplicissimus}\pwindex{Simplicissimus@\emph{Simplicissimus}|pwk} erschien nur knapp zwei
                     Jahre später, am 18. 4. 1896, Schnitzlers Einakter \emph{Die überspannte Person}\pwindex{ueberspannte Person@\emph{Die überspannte Person}|pwk}.}}}\label{K_L02623-3}.\pend
           
\pstart
           Deine \label{K_L02623-4v}\edtext{Novelle\pwindex{Sterben. Novelle@\emph{Sterben. Novelle}|pwuv}}{\lemma{\textnormal{\emph{Novelle}}}\Cendnote{\textnormal{Es dürfte sich um die Buchausgabe von
                        \emph{Sterben}\pwindex{Sterben. Novelle@\emph{Sterben. Novelle}|pwk} handeln. Fedor Mamroth\pwindex{Mamroth, Fedor 21.02.1851 – 25.06.1907@\textsc{Mamroth, Fedor} (21.02.1851 – 25.06.1907), \emph{Journalist/Journalistin, Kritiker/Kritikerin}|pwk} hatte im Vorjahr den Abdruck abgelehnt,
                        vgl. Fedor Mamroth an Arthur Schnitzler, 4. 6. 1893. Am
                        4. 12. 1894 wurde die Novelle in der \emph{Frankfurter Zeitung}\pwindex{Frankfurter Zeitung@\emph{Frankfurter Zeitung}|pwk}{ }rezensiert\pwindex{Belletristische Rundschau@\emph{Belletristische Rundschau}|pwkv}, vgl. Arthur Schnitzler an Fedor Mamroth, 7. 12. 1894.}}}\label{K_L02623-4} ſollſt Du
                  natürlich ſofort der Frankf. Ztg.\orgindex{Frankfurter Zeitung@Frankfurter Zeitung|pw}
                  ſchicken.\pend
           
\pstart
           Wenn Du nur eine Ahnung hätteſt, wie mich alle »äußeren Umſtände Deiner Exiſtenz«
                  intereſſieren. Vor Allem: haſt Du materielle Sorgen?\pend
           
\pstart
           Glückliche Reiſe und frohe Stimmung für die Reiſe! Such’ Dir in \label{K_L02623-5v}\edtext{\textsc{Muenchen\oindex{Muenchen@\textbf{München}, \emph{P.PPLA}|pw}}}{\lemma{\textnormal{\emph{Muenchen}}}\Cendnote{\textnormal{Vom 2. 6. 1894 bis 8. 6. 1894 hielt
                     sich Schnitzler in München\oindex{Muenchen@\textbf{München}, \emph{P.PPLA}|pwk} auf.}}}\label{K_L02623-5} in
                  einem der kleinen Seiten-Cabinete der \textsc{Pinakothek\oindex{Alte Pinakothek@\textbf{Alte Pinakothek}, \emph{Museum (K.MUS)}|pw}} den kleinen \textsc{Altdorfer\pwindex{Altdorfer, Albrecht 1480 – 1538-02-12@\textsc{Altdorfer, Albrecht} (1480 – 1538-02-12), \emph{Maler/Malerin, Kupferstecher/Kupferstecherin, Baumeister/Baumeisterin}|pw}\pwindex{Laubwald mit dem heiligen Georg@\emph{Laubwald mit dem heiligen Georg}|pwv}}{ }\strikeout{de} auf, welcher einen grünen, grünen Wald
                  darſtellt, worin ein putziger kleiner Ritter einen Drachen bekämpft! Das iſt eines
                  meiner Lieblingsbilder: Deutſch und märchenhaft.\pend
           \selectlanguage{ngerman}\endnumbering\briefempfaengerindex{Schnitzler, Arthur@\textsc{Schnitzler, Arthur}!zzzGoldmann, Paul@\emph{von Paul Goldmann}!1894-06-012@{1. 6. {[}1894{]}}|)be}\mylabel{L02623h}  \normalsize

\doendnotes{C}
\bigskip
\vfill

\clearpage

\footnotesize

\lohead{\textsc{register}}

% Definiere theindex-Environment komplett neu ohne reledmac
\makeatletter
\renewenvironment{theindex}{%
  \section*{\indexname}%
  \setlength{\parindent}{0pt}%
  \setlength{\parskip}{0pt plus 0.3pt}%
  \let\item\@idxitem
}{%
  \clearpage
}
\makeatother

\IfFileExists{\jobname-pw.ind}{\input{\jobname-pw.ind}}{}

\end{document}

      