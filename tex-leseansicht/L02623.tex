%% latex-leseansicht-vorspann.tex
%% Vorspann für die Leseansicht.
%% Lädt die gemeinsame Datei latex-vorspann.tex mit nicht gesetztem Schalter.

\newif\ifkorrekturansicht
\korrekturansichtfalse

\input{../tex-inputs/latex-vorspann}


\section[Paul Goldmann an Arthur Schnitzler, 1. 6. [1894]]{L02623 Paul Goldmann an Arthur Schnitzler, 1. 6. [1894]}
\nopagebreak\mylabel{L02623v}
\rehead{ }\normalsize\beginnumbering\briefempfaengerindex{Schnitzler, Arthur@\textsc{Schnitzler, Arthur}!zzzGoldmann, Paul@\emph{von Paul Goldmann}!1894-06-012@{1. 6. [1894]}|(be}
\toendnotes[C]{\smallbreak\pagebreak[2]}
\correspDesc{Versand  durch Paul Goldmann am 1. 6. [1894] in Paris
\newline{}Erhalt  durch Arthur Schnitzler im Zeitraum [2. 6. 1894
                  – 6. 6. 1894?] in Wien}\toendnotes[C]{\smallbreak}
\Standort{DLA, A:Schnitzler, HS.NZ85.1.3164.}
\physDesc{Brief, 2 Blätter, 6 Seiten, 3193 Zeichen
\newline{}Handschrift: schwarze Tinte, deutsche Kurrent
\newline{}Schnitzler: 1) mit Bleistift auf dem ersten Blatt die Jahreszahl »94« vermerkt  2) mit rotem Buntstift vier Unterstreichungen}\toendnotes[C]{\smallbreak}
\pstart
           {\pb}\textcolor{gray}{\textbf{Frankfurter Zeitung\orgindex{Frankfurter Zeitung@Frankfurter Zeitung|pw}}}\hfill \textsc{Paris\oindex{Paris@\textbf{Paris}, \emph{Hauptstadt}|pw}}, 1. Juni.\pend
           
\pstart
           \textcolor{gray}{\textbf{(Gazette de
                     Francfort\orgindex{Frankfurter Zeitung@Frankfurter Zeitung|pw}).}}\pend
           
\pstart
           \textcolor{gray}{\textbf{Fondateur \textbf{M. L. Sonnemann\pwindex{Sonnemann, Leopold 29.\,10.\,1831 Höchberg – 30.\,10.\,1909 Frankfurt am Main@\textsc{Sonnemann, Leopold} (29.\,10.\,1831 Höchberg – 30.\,10.\,1909 Frankfurt am Main), \emph{Journalist, Herausgeber}|pw}}.}}\pend
           
\pstart
           \textcolor{gray}{\textbf{\begin{otherlanguage}{french}Journal politique, financier,\end{otherlanguage}}}\pend
           
\pstart
           \textcolor{gray}{\textbf{\begin{otherlanguage}{french}commercial et littéraire.\end{otherlanguage}}}\pend
           
\pstart
           \textcolor{gray}{\textbf{\begin{otherlanguage}{french}\textbf{Paraissant trois fois par jour.}\end{otherlanguage}}}\pend
           
\pstart
           \textcolor{gray}{\textbf{\begin{otherlanguage}{french}\textbf{Bureau à Paris\oindex{Paris@\textbf{Paris}, \emph{Hauptstadt}|pw}:}\end{otherlanguage}}}\pend
           
\pstart
           \textcolor{gray}{\textbf{\begin{otherlanguage}{french}24. Rue Feydeau\oindex{rue Feydeau@\textbf{rue Feydeau}, \emph{Straße}|pw}.\end{otherlanguage}}}\pend
           
\pstart\center{}Mein lieber Freund,\pend\vspace{0.5em}
\pstart
           \textsc{Hermann Bahr}\pwindex{Bahr, Hermann 19.\,7.\,1863 Linz – 15.\,1.\,1934 München@\textsc{Bahr, Hermann} (19.\,7.\,1863 Linz – 15.\,1.\,1934 München), \emph{Schriftsteller, Kritiker}|pw} iſt alſo doch bei mir geweſen; aber ich wünſchte, es wäre lieber nicht
               geſchehen. Er hat mir einen abſcheulichen Eindruck gemacht, – ein Intriguant, ein
               Jeſuit – und wenn, wie dies wahrſcheinlich,{ }ſeine Geſinnung der meinigen gleicht,{ }ſo{ }ſind wir, mit einem herzlichen Händedruck, als erklärte Feinde geſchieden. Der Mann\pwindex{Bahr, Hermann 19.\,7.\,1863 Linz – 15.\,1.\,1934 München@\textsc{Bahr, Hermann} (19.\,7.\,1863 Linz – 15.\,1.\,1934 München), \emph{Schriftsteller, Kritiker}|pwv} hat mir in der kurzen
               Zeit{ }ſeines Hier-Seins mehr Stänkereien angerichtet, als{ }ſonſt irgend Einer, hat mich
               aus meiner Sicherheit {\pb}gebracht und mich durch
               allerlei Perfidie erregt und verſtimmt. Es wäre zu weitläufig, das hier zu erzählen;
               der Mensch\pwindex{Bahr, Hermann 19.\,7.\,1863 Linz – 15.\,1.\,1934 München@\textsc{Bahr, Hermann} (19.\,7.\,1863 Linz – 15.\,1.\,1934 München), \emph{Schriftsteller, Kritiker}|pwv}, der hier mit
               einem infamen Pack von Reportern niedrigſter Sorte verkehrt, hat{ }ſich dort allerlei
               Verleumdungen über mich geholt, die er mir, mit liebenswürdigem Wohlwollen, wieder
               erzählt hat. Ich berühre das nur, um Dich davor zu warnen, irgendwelchen
               freundſchaftlichen Referaten aus dieſer Quelle Glauben zu{ }ſchenken. Der Grund,
               weshalb ich mich heut an Dich wende, iſt ein \strikeout{b}
               anderer. Er liegt in Einigem, was mir der Herr\pwindex{Bahr, Hermann 19.\,7.\,1863 Linz – 15.\,1.\,1934 München@\textsc{Bahr, Hermann} (19.\,7.\,1863 Linz – 15.\,1.\,1934 München), \emph{Schriftsteller, Kritiker}|pwv} über Euch geſagt hat. Zunächſt{ }ſelbſtverſtändlich{ }ſpielt
               er{ }ſich als den eigentlichen Förderer und {\pb}Inſpirator der \label{K_L02623-1v}\edtext{Wien\oindex{Wien@\textbf{Wien}, \emph{Verwaltungsgebiet}|pw}er Literatur-Strömung}{\lemma{\textnormal{\emph{Wiener Literatur-Strömung}}}\Cendnote{\textnormal{Bei »Jung Wien« handelte es sich um eine losen Verbund von
                  Autoren ohne gemeinsames Programm. Unter diesem Namen agierte kurze Zeit ein
                  Verein, der sich zumindest zwischen 17. 3. 1891 und 5. 5. 1891 wöchentlich traf. Einen Anspruch auf
                  Popularisierung der neuen Strömung und damit auch auf eine Rolle als ihr Ausformer
                  konnte Bahr\pwindex{Bahr, Hermann 19.\,7.\,1863 Linz – 15.\,1.\,1934 München@\textsc{Bahr, Hermann} (19.\,7.\,1863 Linz – 15.\,1.\,1934 München), \emph{Schriftsteller, Kritiker}|pwk} damit begründen, dass er in
                  einem dreiteiligen Feuilleton, \emph{Das junge
                     Österreich}\pwindex{Bahr, Hermann 19.\,7.\,1863 Linz – 15.\,1.\,1934 München@\textsc{Bahr, Hermann} (19.\,7.\,1863 Linz – 15.\,1.\,1934 München), \emph{Schriftsteller, Kritiker}!junge Österreich@\strich\emph{Das junge Österreich}|pwk}, das zuerst am 20. 9. 1893, am
                     27. 9. 1893 und am 7. 10. 1893 in der \emph{Deutschen Zeitung}\pwindex{Deutsche Zeitung@\emph{Deutsche Zeitung}|pwk} erschienen war, erstmals eine gemeinsame
                  Sichtung unternommen hatte (Jg. 23, Nr. 7806, Morgen-Ausgabe, S. 1–2; Nr. 7813,
                     Morgen-Ausgabe, S. 1–3; Nr. 7823, Morgen-Ausgabe, S. 1–3). Im Folgejahr
                  nahm er es in die Zusammenstellung von Texten \emph{Studien zur Kritik der Moderne}\pwindex{Bahr, Hermann 19.\,7.\,1863 Linz – 15.\,1.\,1934 München@\textsc{Bahr, Hermann} (19.\,7.\,1863 Linz – 15.\,1.\,1934 München), \emph{Schriftsteller, Kritiker}!Studien zur Kritik der Moderne@\strich\emph{Studien zur Kritik der Moderne}|pwk} (Frankfurt am Main: \emph{Literarische Anstalt Rütten {\kaufmannsund} Loening}\orgindex{Rütten und Loening@Rütten {\kaufmannsund}  Loening|pwk}) auf. Das »Euch« dürfte dabei auf die bleibendsten dieser Autoren gemünzt
                  sein, die privat in regelmäßigem Umgang mit Schnitzler standen, vor allem Richard
                     Beer-Hofmann\pwindex{Beer-Hofmann, Richard 11.\,7.\,1866 Wien – 26.\,9.\,1945 New York City@\textsc{Beer-Hofmann, Richard} (11.\,7.\,1866 Wien – 26.\,9.\,1945 New York City), \emph{Schriftsteller}|pwk}, Hugo von Hofmannsthal\pwindex{Hofmannsthal, Hugo von 1.\,2.\,1874 Wien – 15.\,7.\,1929 Rodaun@\textsc{Hofmannsthal, Hugo von} (1.\,2.\,1874 Wien – 15.\,7.\,1929 Rodaun), \emph{Schriftsteller}|pwk}
                  und Felix Salten\pwindex{Salten, Felix 6.\,9.\,1869 Budapest – 8.\,10.\,1945 Zürich@\textsc{Salten, Felix} (6.\,9.\,1869 Budapest – 8.\,10.\,1945 Zürich), \emph{Schriftsteller, Journalist, Chefredakteur}|pwk}.}}}\label{K_L02623-1} auf. Zu gleicher
               Zeit hat er über jeden von Euch bei aller{ }ſcheinbaren Anerkennung irgend ein
               herabſetzendes Wort,{ }ſo daß von der Wien\oindex{Wien@\textbf{Wien}, \emph{Verwaltungsgebiet}|pw}er
               Literatur eigentlich als vollgiltig nur Hermann \textsc{Bahr}\pwindex{Bahr, Hermann 19.\,7.\,1863 Linz – 15.\,1.\,1934 München@\textsc{Bahr, Hermann} (19.\,7.\,1863 Linz – 15.\,1.\,1934 München), \emph{Schriftsteller, Kritiker}|pw} übrig bleibt. Selbſt die Leute{ }ſeiner eigenen Revüe\pwindex{Zeit. Wiener Wochenschrift@\emph{Die Zeit. Wiener Wochenschrift}|pwv} drückt er herunter. \textsc{Kanner\pwindex{Kanner, Heinrich 9.\,11.\,1864 Galați – 15.\,2.\,1930 Wien@\textsc{Kanner, Heinrich} (9.\,11.\,1864 Galați – 15.\,2.\,1930 Wien), \emph{Herausgeber, Publizist}|pw}}{ }\strikeout{iſt} wird{ }ſich nach{ }ſeiner Darſtellung mit der
               Adminiſtration befaſſen; und wenn \strikeout{n} man \textsc{Kanner\pwindex{Kanner, Heinrich 9.\,11.\,1864 Galați – 15.\,2.\,1930 Wien@\textsc{Kanner, Heinrich} (9.\,11.\,1864 Galați – 15.\,2.\,1930 Wien), \emph{Herausgeber, Publizist}|pw}} nur aus{ }ſeinen Reden kennt,{ }ſo muß man ihn für nichts als für einen Kaſſier
               halten, während doch in Wahrheit \textsc{Kanner\pwindex{Kanner, Heinrich 9.\,11.\,1864 Galați – 15.\,2.\,1930 Wien@\textsc{Kanner, Heinrich} (9.\,11.\,1864 Galați – 15.\,2.\,1930 Wien), \emph{Herausgeber, Publizist}|pw}} der \strikeout{Ein} Einzige iſt, der für die {\pb}\textsc{Revue\pwindex{Zeit. Wiener Wochenschrift@\emph{Die Zeit. Wiener Wochenschrift}|pwv}} Zukunfts-Hoffnungen rechtfertigt. Nun aber zu Euch zurück. Ich möchte Dich
               bitten, mir mit ein paar Worten etwas über das Verhältniß von \textsc{Hermann Bahr\pwindex{Bahr, Hermann 19.\,7.\,1863 Linz – 15.\,1.\,1934 München@\textsc{Bahr, Hermann} (19.\,7.\,1863 Linz – 15.\,1.\,1934 München), \emph{Schriftsteller, Kritiker}|pw}} zu Eurem Kreiſe zu{ }ſagen. Insbeſondere möchte ich wiſſen, ob zwiſchen ihm und
                  \label{K_L02623-2v}\edtext{\textsc{Loris\pwindex{Hofmannsthal, Hugo von 1.\,2.\,1874 Wien – 15.\,7.\,1929 Rodaun@\textsc{Hofmannsthal, Hugo von} (1.\,2.\,1874 Wien – 15.\,7.\,1929 Rodaun), \emph{Schriftsteller}|pw}} wirklich jene intime Freundſchaft}{\lemma{\textnormal{\emph{Loris … Freundschaft}}}\Cendnote{\textnormal{Ohne Schnitzlers Antwort zu kennen, finden
                  sich in seinem \emph{Tagebuch}\pwindex{Schnitzler, Arthur 15.\,5.\,1862 Wien – 21.\,10.\,1931 ebd.@\textsc{Schnitzler, Arthur} (15.\,5.\,1862 Wien – 21.\,10.\,1931 ebd.), \emph{Schriftsteller, Mediziner}!Tagebuch@\strich\emph{Tagebuch}|pwk} doch mehrfach
                  Aussagen, die die bestehende Nähe zwischen Bahr\pwindex{Bahr, Hermann 19.\,7.\,1863 Linz – 15.\,1.\,1934 München@\textsc{Bahr, Hermann} (19.\,7.\,1863 Linz – 15.\,1.\,1934 München), \emph{Schriftsteller, Kritiker}|pwk} und Hofmannsthal\pwindex{Hofmannsthal, Hugo von 1.\,2.\,1874 Wien – 15.\,7.\,1929 Rodaun@\textsc{Hofmannsthal, Hugo von} (1.\,2.\,1874 Wien – 15.\,7.\,1929 Rodaun), \emph{Schriftsteller}|pwk} kritisch
                  beurteilen, beispielsweise A. S.: \emph{Tagebuch}, 6. 11. 1895,
                  aber auch Goldmann\pwindex{Goldmann, Paul 31.\,1.\,1865 Breslau – 25.\,9.\,1935 Wien@\textsc{Goldmann, Paul} (31.\,1.\,1865 Breslau – 25.\,9.\,1935 Wien), \emph{Schriftsteller, Journalist}|pwk} beschäftigte das Thema
                  länger, vgl. A. S.: \emph{Tagebuch}, 26. 8. 1895.
               }}}\label{K_L02623-2} beſteht, \strikeout{die} wie er vorgibt; ob er wirklich
               berechtigt iſt,{ }ſich als den »Erzieher\pwindex{Bahr, Hermann 19.\,7.\,1863 Linz – 15.\,1.\,1934 München@\textsc{Bahr, Hermann} (19.\,7.\,1863 Linz – 15.\,1.\,1934 München), \emph{Schriftsteller, Kritiker}|pwv}« von \textsc{Loris\pwindex{Hofmannsthal, Hugo von 1.\,2.\,1874 Wien – 15.\,7.\,1929 Rodaun@\textsc{Hofmannsthal, Hugo von} (1.\,2.\,1874 Wien – 15.\,7.\,1929 Rodaun), \emph{Schriftsteller}|pw}} aufzuſpielen, wie er das thut \textsc{etc}. Bitte,{ }ſchreib’
               mir bald; denn das Alles quält mich{ }ſehr{ }ſeit geſtern{ }Abend. Ich will Dir nicht{ }ſagen, warum,{ }ſondern Deine Antwort
               abwarten.\pend
           
\pstart
           Herzlichſt und in Treue {\\[\baselineskip]}Dein \spacefill\mbox{Paul Goldmann.}\pend
           \leftskip=0em{}
\pstart
           \noindent{}{\pb}Ja{ }ſo, entſchuldige, in meiner Erregung hätte ich
                  beinahe Deine Angelegenheiten vergeſſen. Der Verleger \textsc{Albert Langen}\pwindex{Langen, Albert 8.\,7.\,1869 Antwerpen – 30.\,4.\,1909 München@\textsc{Langen, Albert} (8.\,7.\,1869 Antwerpen – 30.\,4.\,1909 München), \emph{Verleger}|pw} iſt ein reicher junger Menſch, der{ }ſich zum Verleger gemacht hat, um mit
                  Literatur protzen zu können. Der Menſch\pwindex{Langen, Albert 8.\,7.\,1869 Antwerpen – 30.\,4.\,1909 München@\textsc{Langen, Albert} (8.\,7.\,1869 Antwerpen – 30.\,4.\,1909 München), \emph{Verleger}|pwv} iſt idiotiſch urtheilslos, \strikeout{und} verlogen und betrügeriſch. Er iſt von dem halb wahnſinnigen \textsc{Gretor\pwindex{Gretor, Willy 16.\,7.\,1868 Kaliningrad – 31.\,7.\,1923 Kopenhagen@\textsc{Gretor, Willy} (16.\,7.\,1868 Kaliningrad – 31.\,7.\,1923 Kopenhagen), \emph{Maler, Kunstagent, Kunsthändler}|pw}} beeinflußt, von dem ich Dir im vorigen Sommer erzählt. Ich rathe Dir
                  dringend, Dich \label{K_L02623-3v}\edtext{mit dem Burſchen in
                  nichts {\pb}einzulaſſen}{\lemma{\textnormal{\emph{mit … einzulassen}}}\Cendnote{\textnormal{In Langens\pwindex{Langen, Albert 8.\,7.\,1869 Antwerpen – 30.\,4.\,1909 München@\textsc{Langen, Albert} (8.\,7.\,1869 Antwerpen – 30.\,4.\,1909 München), \emph{Verleger}|pwk}{ }\emph{Simplicissimus}\pwindex{Simplicissimus@\emph{Simplicissimus}|pwk} erschien nur knapp zwei
                     Jahre später, am 18. 4. 1896, Schnitzlers Einakter \emph{Die überspannte Person}\pwindex{Schnitzler, Arthur 15.\,5.\,1862 Wien – 21.\,10.\,1931 ebd.@\textsc{Schnitzler, Arthur} (15.\,5.\,1862 Wien – 21.\,10.\,1931 ebd.), \emph{Schriftsteller, Mediziner}!überspannte Person@\strich\emph{Die überspannte Person}|pwk}.}}}\label{K_L02623-3}.\pend
           
\pstart
           Deine \label{K_L02623-4v}\edtext{Novelle\pwindex{Schnitzler, Arthur 15.\,5.\,1862 Wien – 21.\,10.\,1931 ebd.@\textsc{Schnitzler, Arthur} (15.\,5.\,1862 Wien – 21.\,10.\,1931 ebd.), \emph{Schriftsteller, Mediziner}!Sterben. Novelle@\strich\emph{Sterben. Novelle}|pwuv}}{\lemma{\textnormal{\emph{Novelle}}}\Cendnote{\textnormal{Es dürfte sich um die Buchausgabe von
                        \emph{Sterben}\pwindex{Schnitzler, Arthur 15.\,5.\,1862 Wien – 21.\,10.\,1931 ebd.@\textsc{Schnitzler, Arthur} (15.\,5.\,1862 Wien – 21.\,10.\,1931 ebd.), \emph{Schriftsteller, Mediziner}!Sterben. Novelle@\strich\emph{Sterben. Novelle}|pwk} handeln. Fedor Mamroth\pwindex{Mamroth, Fedor 21.\,2.\,1851 Breslau – 25.\,6.\,1907 Frankfurt am Main@\textsc{Mamroth, Fedor} (21.\,2.\,1851 Breslau – 25.\,6.\,1907 Frankfurt am Main), \emph{Journalist, Kritiker}|pwk} hatte im Vorjahr den Abdruck abgelehnt,
                        vgl. XXXX Auszeichnungsfehler: Dokument L00216 nicht gefunden. Am
                        4. 12. 1894 wurde die Novelle in der \emph{Frankfurter Zeitung}\pwindex{Frankfurter Zeitung@\emph{Frankfurter Zeitung}|pwk}{ }rezensiert\pwindex{\textcolor{red}{\textsuperscript{XXXX indx1}}!Belletristische Rundschau@\strich\emph{Belletristische Rundschau}|pwkv}, vgl. XXXX Auszeichnungsfehler: Dokument L00409 nicht gefunden.}}}\label{K_L02623-4}{ }ſollſt Du
                  natürlich{ }ſofort der Frankf. Ztg.\orgindex{Frankfurter Zeitung@Frankfurter Zeitung|pw}{ }ſchicken.\pend
           
\pstart
           Wenn Du nur eine Ahnung hätteſt, wie mich alle »äußeren Umſtände Deiner Exiſtenz«
                  intereſſieren. Vor Allem: haſt Du materielle Sorgen?\pend
           
\pstart
           Glückliche Reiſe und frohe Stimmung für die Reiſe! Such’ Dir in \label{K_L02623-5v}\edtext{\textsc{Muenchen\oindex{München@\textbf{München}|pw}}}{\lemma{\textnormal{\emph{Muenchen}}}\Cendnote{\textnormal{Vom 2. 6. 1894 bis 8. 6. 1894 hielt
                     sich Schnitzler in München\oindex{München@\textbf{München}|pwk} auf.}}}\label{K_L02623-5} in
                  einem der kleinen Seiten-Cabinete der \textsc{Pinakothek\oindex{Alte Pinakothek@\textbf{Alte Pinakothek}, \emph{Museum}|pw}} den kleinen \textsc{Altdorfer\pwindex{Altdorfer, Albrecht 1480 – 12.\,2.\,1538@\textsc{Altdorfer, Albrecht} (1480 – 12.\,2.\,1538), \emph{Maler, Kupferstecher, Baumeister}|pw}\pwindex{Altdorfer, Albrecht 1480 – 12.\,2.\,1538@\textsc{Altdorfer, Albrecht} (1480 – 12.\,2.\,1538), \emph{Maler, Kupferstecher, Baumeister}!Laubwald mit dem heiligen Georg@\strich\emph{Laubwald mit dem heiligen Georg}|pwv}}{ }\strikeout{de} auf, welcher einen grünen, grünen Wald
                  darſtellt, worin ein putziger kleiner Ritter einen Drachen bekämpft! Das iſt eines
                  meiner Lieblingsbilder: Deutſch und märchenhaft.\pend
           \selectlanguage{ngerman}\endnumbering\briefempfaengerindex{Schnitzler, Arthur@\textsc{Schnitzler, Arthur}!zzzGoldmann, Paul@\emph{von Paul Goldmann}!1894-06-012@{1. 6. [1894]}|)be}\mylabel{L02623h}  \newcommand{\dateiname}{L02623}\newcommand{\titel}{Paul Goldmann an Arthur Schnitzler, 1. 6. [1894]}\newcommand{\editorInnen}{Martin Anton Müller und Laura Untner}%% latex-leseansicht-abspann.tex
%% Abspann für die Leseansicht.
%% Der Schalter \ifkorrekturansicht ist bereits durch den Vorspann gesetzt.

%% latex-abspann.tex
%% Gemeinsamer Abspann für Korrekturansicht und Leseansicht.
%% Setzt den Schalter \ifkorrekturansicht voraus (gesetzt in den
%% einbindenden Dateien latex-korrekturansicht-abspann.tex bzw.
%% latex-leseansicht-abspann.tex).
%% ---------------------------------------------------------------

\normalsize

% Das esempio-Environment wird nur in der Leseansicht benötigt
\ifkorrekturansicht\else
\newenvironment{esempio}[3]%
{
    \vspace{1.5ex}
    \rlap{\underline{#1}}
    \par
    \setlength{\parindent}{0cm}
    \nopagebreak
    \leftskip=#2cm
    \rightskip=#3cm
}
{
    \par
}
\fi

\doendnotes{C}
\bigskip
\vfill

\clearpage

\footnotesize

\ifkorrekturansicht
  \lohead{\textsc{register}}
\fi

% theindex-Environment neu definieren ohne reledmac
\makeatletter
\renewenvironment{theindex}{%
  \ifkorrekturansicht
    \section*{\indexname}%
  \else
    \subsubsection*{Index der erwähnten Entitäten}%
  \fi
  \setlength{\parindent}{0pt}%
  \setlength{\parskip}{0pt plus 0.3pt}%
  \let\item\@idxitem
}{%
  \ifkorrekturansicht\clearpage\fi
}
\makeatother

\IfFileExists{\jobname-pw.ind}{\input{\jobname-pw.ind}}{}

% Quellenangabe nur in der Leseansicht
\ifkorrekturansicht\else
% Fallback-Definitionen, falls die .tex-Datei \titel etc. nicht gesetzt hat
\providecommand{\titel}{}
\providecommand{\editorInnen}{}
\providecommand{\dateiname}{\jobname}

\vspace{3cm}

\vfill

\footnotesize
\textsc{Quelle}: \titel. Herausgegeben von {\editorInnen}. In: \emph{Arthur Schnitzler: Briefwechsel mit Autorinnen und Autoren}.
 Digitale Edition, https://schnitzler-briefe.acdh.oeaw.ac.at/{\dateiname}.html (Stand \today)
\fi

\end{document}


