%% latex-korrekturansicht-vorspann.tex
%% Vorspann für die Korrekturansicht.
%% Lädt die gemeinsame Datei latex-vorspann.tex mit gesetztem Schalter.

\newif\ifkorrekturansicht
\korrekturansichttrue

\input{../tex-inputs/latex-vorspann}


\section[ Paul Goldmann an Arthur Schnitzler, 29. 9. {[}1899{]}]{L02888 Paul Goldmann an Arthur Schnitzler, 29. 9. {[}1899{]}}
\nopagebreak\mylabel{L02888v}
\rehead{ }\normalsize\beginnumbering\briefempfaengerindex{Schnitzler, Arthur@\textsc{Schnitzler, Arthur}!zzzGoldmann, Paul@\emph{von Paul Goldmann}!1899-09-293@{29. 9. {[}1899{]}}|(be}
\toendnotes[C]{\smallbreak\pagebreak[2]}\Standort{DLA, A:Schnitzler, HS.NZ85.1.3169.}
\physDesc{Brief, 1 Blatt, 2 Seiten, 1014 Zeichen
\newline{}Handschrift: blaue Tinte, deutsche Kurrent
\newline{}Schnitzler: mit Bleistift das Jahr »99« vermerkt }\toendnotes[C]{\smallbreak}
\pstart
           {\pb}\textcolor{gray}{\textbf{\begin{otherlanguage}{french}Florence\oindex{Florenz@\textbf{Florenz}, \emph{P.PPLA}|pw} – Hôtel Pension Barbensi\oindex{Hôtel Pension Barbensi@\textbf{Hôtel Pension Barbensi}, \emph{Hotel (K.HTL)}|pw}\end{otherlanguage}}}\pend
           
\pstart
           Lung’Arno Guicciardini\oindex{Lungarno Guicciardini@\textbf{Lungarno Guicciardini}, \emph{Straße (K.STR)}|pw}\hfill Florenz\oindex{Florenz@\textbf{Florenz}, \emph{P.PPLA}|pw}, 29. September.\pend
           
\pstart
           \textcolor{gray}{\textbf{G. ZANETTA\pwindex{Zanetta, G. @\textsc{Zanetta, G.}, \emph{Hotelbesitzer/Hotelbesitzerin}|pw}{ }{\kaufmannsund} C.\textsuperscript{i}}}\pend
           
\pstart{}Mein lieber Freund,\pend\vspace{0.5em}
\pstart
           Es regnet in Florenz\oindex{Florenz@\textbf{Florenz}, \emph{P.PPLA}|pw}, wie in \textsc{Wiesbaden\oindex{Wiesbaden@\textbf{Wiesbaden}, \emph{P.PPLA}|pw}}. Auch ſonſt komme ich \strikeout{\textcolor{gray}{vol}} vorläufig nicht recht \strikeout{\textcolor{gray}{in}} auf den Geſchmack. Ich hatte gemeint, mitten in die \textsc{Renaissance}-Stimmung hineinzugerathen, und finde \strikeout{\textcolor{gray}{i}} eine italien\oindex{Italien@\textbf{Italien}, \emph{A.PCLI}|pwv}iſche Provinzſtadt\oindex{Florenz@\textbf{Florenz}, \emph{P.PPLA}|pwv}, in der ſich faſt
               alles Schöne in den Sammlungen befindet. Allerdings, der herrliche \strikeout{D\textcolor{gray}{om}}{ }Dom\oindex{Kathedrale von Florenz@\textbf{Kathedrale von Florenz}, \emph{Kirche (K.KRC)}|pw}. Aber ich hatte erwartet, auf
                  jede\textcolor{gray}{m} Schritt italien\oindex{Italien@\textbf{Italien}, \emph{A.PCLI}|pwv}iſches Mittelalter zu finden, und bin nun etwas
               enttäuſcht. Die Sammlungen freilich ſind überwältigend. \textsc{Botticelli\pwindex{Botticelli, Sandro 1445-03-01 – 1510-05-17@\textsc{Botticelli, Sandro} (1445-03-01 – 1510-05-17), \emph{Maler/Malerin}|pw}}, \textsc{Rafael\pwindex{Raffaello Sanzio da Urbino 28. 3. oder 6. 4. 1483 – 6.04.1520@\textsc{Raffaello Sanzio da Urbino} (28. 3. oder 6. 4. 1483 – 6.04.1520), \emph{Maler/Malerin}|pw}} (jawohl, \textsc{Rafael\pwindex{Raffaello Sanzio da Urbino 28. 3. oder 6. 4. 1483 – 6.04.1520@\textsc{Raffaello Sanzio da Urbino} (28. 3. oder 6. 4. 1483 – 6.04.1520), \emph{Maler/Malerin}|pw}}!). Aber als \uline{Städte} ſind, ſoweit ich bisher
               urtheilen kann, Mailand\oindex{Mailand@\textbf{Mailand}, \emph{P.PPLA}|pw}, \textsc{Bologna\oindex{Bologna@\textbf{Bologna}, \emph{P.PPLA}|pw}} oder gar Venedig\oindex{Venedig@\textbf{Venedig}, \emph{P.PPLA}|pw} viel ſchöner.\pend
           
\pstart
           {\pb}Mach’ Dir in \label{K_L02888-1v}\edtext{Berlin\oindex{Berlin@\textbf{Berlin}, \emph{P.PPLC}|pw}}{\lemma{\textnormal{\emph{Berlin}}}\Cendnote{\textnormal{Schnitzler reiste am 3. 10. 1899 von Wiesbaden\oindex{Wiesbaden@\textbf{Wiesbaden}, \emph{P.PPLA}|pwk} nach Berlin\oindex{Berlin@\textbf{Berlin}, \emph{P.PPLC}|pwk} und blieb dort bis zum 11. 10. 1899.}}}\label{K_L02888-1} ein paar gute Tage!\pend
           
\pstart
           In Wien\oindex{Wien@\textbf{Wien}, \emph{A.ADM2}|pw} ſollſt Du mich nicht \label{K_L02888-2v}\edtext{erwarten}{\lemma{\textnormal{\emph{erwarten}}}\Cendnote{\textnormal{Goldmann\pwindex{Goldmann, Paul 31.01.1865 – 25.09.1935@\textsc{Goldmann, Paul} (31.01.1865 – 25.09.1935), \emph{Schriftsteller/Schriftstellerin, Journalist/Journalistin}|pwk} kam am 13. 10. 1899 nach Wien\oindex{Wien@\textbf{Wien}, \emph{A.ADM2}|pwk} und blieb bis zum 21. 10. 1899.}}}\label{K_L02888-2}.
               Ich käme gern, das brauche ich Dir wohl nicht zu ſagen. Aber die Entfernung ſchreckt
               mich. Die lange Eiſenbahnreiſe von Frankfurt\oindex{Frankfurt am Main@\textbf{Frankfurt am Main}, \emph{P.PPLA3}|pw}{ }hierher\oindex{Florenz@\textbf{Florenz}, \emph{P.PPLA}|pwv} ſteckt mir heut noch in den Gliedern. Und dann langt ſicherlich mein
               Geld nicht.\pend
           
\pstart
           Schreib’ mir wieder hierher\oindex{Florenz@\textbf{Florenz}, \emph{P.PPLA}|pwv}{ }\begin{otherlanguage}{french}\textsc{poste restante}\end{otherlanguage}!\pend
           
\pstart
           Viele treue Grüße! {\\[\baselineskip]}Dein {\\[\baselineskip]}\spacefill\mbox{Paul Goldmnn}\pend
           \leftskip=0em{}\selectlanguage{ngerman}\endnumbering\briefempfaengerindex{Schnitzler, Arthur@\textsc{Schnitzler, Arthur}!zzzGoldmann, Paul@\emph{von Paul Goldmann}!1899-09-293@{29. 9. {[}1899{]}}|)be}\mylabel{L02888h}  \normalsize

\doendnotes{C}
\bigskip
\vfill

\clearpage

\footnotesize

\lohead{\textsc{register}}

% Definiere theindex-Environment komplett neu ohne reledmac
\makeatletter
\renewenvironment{theindex}{%
  \section*{\indexname}%
  \setlength{\parindent}{0pt}%
  \setlength{\parskip}{0pt plus 0.3pt}%
  \let\item\@idxitem
}{%
  \clearpage
}
\makeatother

\IfFileExists{\jobname-pw.ind}{\input{\jobname-pw.ind}}{}

\end{document}

      