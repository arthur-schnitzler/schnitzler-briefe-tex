%% latex-leseansicht-vorspann.tex
%% Vorspann für die Leseansicht.
%% Lädt die gemeinsame Datei latex-vorspann.tex mit nicht gesetztem Schalter.

\newif\ifkorrekturansicht
\korrekturansichtfalse

\input{../tex-inputs/latex-vorspann}

\begin{center}
            \textcolor{red}{ENTWURF, NICHT FERTIG KORRIGIERT}
                      \end{center}
            
         
         \newcommand{\erwaehntePersonen}{Personen: Sandro Botticelli,  Raffaello Sanzio da Urbino, G. Zanetta}
         \newcommand{\erwaehnteInstitutionen}{}
         \newcommand{\erwaehnteOrte}{Orte: Berlin, Bologna, Florenz, Frankfurt am Main, Hôtel Pension Barbensi, Italien, Kathedrale von Florenz, Lungarno Guicciardini, Mailand, Venedig, Wien, Wiesbaden}
         \newcommand{\erwaehnteWerke}{
               \section[ Paul Goldmann an Arthur Schnitzler, 29. 9. {[}1899{]}]{ Paul Goldmann an Arthur Schnitzler, 29. 9. {[}1899{]}}\nopagebreak\mylabel{v}\rehead{ }\begin{ledgroupsized}[t]{13cm}\normalsize\beginnumbering \toendnotes[C]{\smallbreak\pagebreak[2]} \Standort{DLA, A:Schnitzler, HS.NZ85.1.3169.}
\physDesc{Brief, 1 Blatt, 2 Seiten
\newline{}Handschrift: blaue Tinte, deutsche Kurrent
\newline{}Schnitzler: mit Bleistift das Jahr »99« vermerkt }\toendnotes[C]{\smallbreak}\pstart
           \noindent{}{\pb}\textcolor{gray}{\textbf{\begin{otherlanguage}{french}Florence\oindex{Florenz@\textbf{Florenz}|pw} – Hôtel Pension Barbensi\oindex{Hôtel Pension Barbensi@\textbf{Hôtel Pension Barbensi}|pw}\end{otherlanguage}}}\pend
           \pstart
           Lung’Arno Guicciardini\oindex{Lungarno Guicciardini@\textbf{Lungarno Guicciardini}|pw}\hfill Florenz\oindex{Florenz@\textbf{Florenz}|pw}, 29. September.\pend
           \pstart
           \textcolor{gray}{\textbf{G. ZANETTA\pwindex{Zanetta, G. @\textsc{Zanetta, G.}, \emph{Hotelbesitzer/Hotelbesitzerin}|pw}{ }{\kaufmannsund} C.\textsuperscript{i}}}\pend
           \pstart{}Mein lieber Freund,\pend\pstart
           Es regnet in Florenz\oindex{Florenz@\textbf{Florenz}|pw}, wie in \textsc{Wiesbaden\oindex{Wiesbaden@\textbf{Wiesbaden}|pw}}. Auch ſonſt komme ich \strikeout{\textcolor{gray}{vol}} vorläufig nicht recht \strikeout{\textcolor{gray}{in}} auf den Geſchmack. Ich hatte gemeint, mitten in die \textsc{Renaissance}-Stimmung hineinzugerathen, und finde \strikeout{\textcolor{gray}{i}} eine italien\oindex{Italien@\textbf{Italien}|pwv}iſche Provinzſtadt\oindex{Florenz@\textbf{Florenz}|pwv}, in der ſich faſt
               alles Schöne in den Sammlungen befindet. Allerdings, der herrliche \strikeout{D\textcolor{gray}{om}}{ }Dom\oindex{Kathedrale von Florenz@\textbf{Kathedrale von Florenz}|pw}. Aber ich hatte erwartet, auf jeden
               Schritt italien\oindex{Italien@\textbf{Italien}|pwv}iſches
               Mittelalter zu finden, und bin nun etwas enttäuſcht. Die Sammlungen freilich ſind
               überwältigend. \textsc{Botticelli\pwindex{Botticelli, Sandro 1445-03-01 – 1510-05-17@\textsc{Botticelli, Sandro} (1445-03-01 – 1510-05-17), \emph{Maler}|pw}}, \textsc{Rafael\pwindex{Raffaello Sanzio da Urbino 28. 3. oder 6. 4. 1483 – 6.04.1520@\textsc{Raffaello Sanzio da Urbino} (28. 3. oder 6. 4. 1483 – 6.04.1520), \emph{Maler}|pw}} (jawohl, \textsc{Rafael\pwindex{Raffaello Sanzio da Urbino 28. 3. oder 6. 4. 1483 – 6.04.1520@\textsc{Raffaello Sanzio da Urbino} (28. 3. oder 6. 4. 1483 – 6.04.1520), \emph{Maler}|pw}}!). Aber als \uline{Städte} ſind, ſoweit ich bisher
               urtheilen kann, Mailand\oindex{Mailand@\textbf{Mailand}|pw}, \textsc{Bologna\oindex{Bologna@\textbf{Bologna}|pw}} oder gar Venedig\oindex{Venedig@\textbf{Venedig}|pw} viel ſchöner.\pend
           \pstart
           {\pb}Mach’ Dir in \label{K_L02888-1v}\edtext{Berlin\oindex{Berlin@\textbf{Berlin}|pw}}{\lemma{\textnormal{\emph{Berlin}}}\Cendnote{\textnormal{Schnitzler\pwindex{Schnitzler, Arthur 15.05.1862 – 21.10.1931@\textsc{Schnitzler, Arthur} (15.05.1862 – 21.10.1931), \emph{Schriftsteller, Mediziner}|pwk} reiste am 3. 10. 1899 von Wiesbaden\oindex{Wiesbaden@\textbf{Wiesbaden}|pwk} nach Berlin\oindex{Berlin@\textbf{Berlin}|pwk} und blieb dort bis zum 11. 10. 1899.}}}\label{K_L02888-1h}
               ein paar gute Tage!\pend
           \pstart
           In Wien\oindex{Wien@\textbf{Wien}|pw} ſollſt Du mich nicht \label{K_L02888-2v}\edtext{erwarten}{\lemma{\textnormal{\emph{erwarten}}}\Cendnote{\textnormal{Goldmann\pwindex{Goldmann, Paul 31.01.1865 – 25.09.1935@\textsc{Goldmann, Paul} (31.01.1865 – 25.09.1935), \emph{Schriftsteller, Journalist}|pwk} kam am 13. 10. 1899 nach Wien\oindex{Wien@\textbf{Wien}|pwk} und blieb bis zum 21. 10. 1899.}}}\label{K_L02888-2h}.
               Ich käme gern, das brauche ich Dir wohl nicht zu ſagen. Aber die Entfernung ſchreckt
               mich. Die lange Eiſenbahnreiſe von Frankfurt\oindex{Frankfurt am Main@\textbf{Frankfurt am Main}|pw}{ }hierher\oindex{Florenz@\textbf{Florenz}|pwv} ſteckt mir heut noch in den Gliedern. Und dann langt ſicherlich mein
               Geld nicht.\pend
           \pstart
           Schreib’ mir wieder hierher\oindex{Florenz@\textbf{Florenz}|pwv}{ }\begin{otherlanguage}{french}\textsc{poste restante}\end{otherlanguage}!\pend
           \pstart
           Viele treue Grüße! {\\[\baselineskip]}Dein {\\[\baselineskip]}\spacefill\mbox{Paul Goldmnn}\pend
           \leftskip=0em{}
         
         \endnumbering\mylabel{h}\end{ledgroupsized}  \newcommand{\dateiname}{L02888}\newcommand{\titel}{Paul Goldmann an Arthur Schnitzler, 29. 9. [1899]}\newcommand{\editorInnen}{Martin Anton Müller und Laura Untner}%% latex-leseansicht-abspann.tex
%% Abspann für die Leseansicht.
%% Der Schalter \ifkorrekturansicht ist bereits durch den Vorspann gesetzt.

%% latex-abspann.tex
%% Gemeinsamer Abspann für Korrekturansicht und Leseansicht.
%% Setzt den Schalter \ifkorrekturansicht voraus (gesetzt in den
%% einbindenden Dateien latex-korrekturansicht-abspann.tex bzw.
%% latex-leseansicht-abspann.tex).
%% ---------------------------------------------------------------

\normalsize

% Das esempio-Environment wird nur in der Leseansicht benötigt
\ifkorrekturansicht\else
\newenvironment{esempio}[3]%
{
    \vspace{1.5ex}
    \rlap{\underline{#1}}
    \par
    \setlength{\parindent}{0cm}
    \nopagebreak
    \leftskip=#2cm
    \rightskip=#3cm
}
{
    \par
}
\fi

\doendnotes{C}
\bigskip
\vfill

\clearpage

\footnotesize

\ifkorrekturansicht
  \lohead{\textsc{register}}
\fi

% theindex-Environment neu definieren ohne reledmac
\makeatletter
\renewenvironment{theindex}{%
  \ifkorrekturansicht
    \section*{\indexname}%
  \else
    \subsubsection*{Index der erwähnten Entitäten}%
  \fi
  \setlength{\parindent}{0pt}%
  \setlength{\parskip}{0pt plus 0.3pt}%
  \let\item\@idxitem
}{%
  \ifkorrekturansicht\clearpage\fi
}
\makeatother

\IfFileExists{\jobname-pw.ind}{\input{\jobname-pw.ind}}{}

% Quellenangabe nur in der Leseansicht
\ifkorrekturansicht\else
% Fallback-Definitionen, falls die .tex-Datei \titel etc. nicht gesetzt hat
\providecommand{\titel}{}
\providecommand{\editorInnen}{}
\providecommand{\dateiname}{\jobname}

\vspace{3cm}

\vfill

\footnotesize
\textsc{Quelle}: \titel. Herausgegeben von {\editorInnen}. In: \emph{Arthur Schnitzler: Briefwechsel mit Autorinnen und Autoren}.
 Digitale Edition, https://schnitzler-briefe.acdh.oeaw.ac.at/{\dateiname}.html (Stand \today)
\fi

\end{document}


      