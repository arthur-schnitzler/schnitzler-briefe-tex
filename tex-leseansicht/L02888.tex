%% latex-leseansicht-vorspann.tex
%% Vorspann für die Leseansicht.
%% Lädt die gemeinsame Datei latex-vorspann.tex mit nicht gesetztem Schalter.

\newif\ifkorrekturansicht
\korrekturansichtfalse

\input{../tex-inputs/latex-vorspann}


\section[ Paul Goldmann an Arthur Schnitzler, 29. 9. {[}1899{]}]{L02888 Paul Goldmann an Arthur Schnitzler,  29. 9. [1899]}
\nopagebreak\mylabel{L02888v}
\rehead{ }\normalsize\beginnumbering\briefempfaengerindex{Schnitzler, Arthur@\textsc{Schnitzler, Arthur}!zzzGoldmann, Paul@\emph{von Paul Goldmann}!1899-09-294@{29. 9. [1899]}|(be}
\toendnotes[C]{\smallbreak\pagebreak[2]}
\correspDesc{Versand  durch Paul Goldmann am 29. 9. [1899] in Florenz
\newline{}Erhalt  durch Arthur Schnitzler im Zeitraum [30. 9. 1899
                  – 4. 10. 1899?] in Wiesbaden}\toendnotes[C]{\smallbreak}
\Standort{DLA, A:Schnitzler, HS.NZ85.1.3169.}
\physDesc{Brief, 1 Blatt, 2 Seiten, 1014 Zeichen
\newline{}Handschrift: blaue Tinte, deutsche Kurrent
\newline{}Schnitzler: mit Bleistift das Jahr »99« vermerkt }\toendnotes[C]{\smallbreak}
\pstart
           {\pb}\textcolor{gray}{\textbf{\begin{otherlanguage}{french}Florence\oindex{Florenz@\textbf{Florenz}|pw} – Hôtel Pension Barbensi\oindex{Hôtel Pension Barbensi@\textbf{Hôtel Pension Barbensi}, \emph{Hotel}|pw}\end{otherlanguage}}}\pend
           
\pstart
           Lung’Arno Guicciardini\oindex{Lungarno Guicciardini@\textbf{Lungarno Guicciardini}, \emph{Straße}|pw}\hfill Florenz\oindex{Florenz@\textbf{Florenz}|pw}, 29. September.\pend
           
\pstart
           \textcolor{gray}{\textbf{G. ZANETTA\pwindex{Zanetta, G. @\textsc{Zanetta, G.}, \emph{Hotelbesitzer/Hotelbesitzerin}|pw}{ }{\kaufmannsund} C.\textsuperscript{i}}}\pend
           
\pstart{}Mein lieber Freund,\pend\vspace{0.5em}
\pstart
           Es regnet in Florenz\oindex{Florenz@\textbf{Florenz}|pw}, wie in \textsc{Wiesbaden\oindex{Wiesbaden@\textbf{Wiesbaden}|pw}}. Auch{ }ſonſt komme ich \strikeout{\textcolor{gray}{vol}} vorläufig nicht recht \strikeout{\textcolor{gray}{in}} auf den Geſchmack. Ich hatte gemeint, mitten in die \textsc{Renaissance}-Stimmung hineinzugerathen, und finde \strikeout{\textcolor{gray}{i}} eine italien\oindex{Italien@\textbf{Italien}|pwv}iſche Provinzſtadt\oindex{Florenz@\textbf{Florenz}|pwv}, in der{ }ſich faſt
               alles Schöne in den Sammlungen befindet. Allerdings, der herrliche \strikeout{D\textcolor{gray}{om}}{ }Dom\oindex{Cattedrale di Santa Maria del Fiore@\textbf{Cattedrale di Santa Maria del Fiore}, \emph{Kirche}|pw}. Aber ich hatte erwartet, auf
                  jede\textcolor{gray}{m} Schritt italien\oindex{Italien@\textbf{Italien}|pwv}iſches Mittelalter zu finden, und bin nun etwas
               enttäuſcht. Die Sammlungen freilich{ }ſind überwältigend. \textsc{Botticelli\pwindex{Botticelli, Sandro 1.\,3.\,1445 Florenz – 17.\,5.\,1510 ebd.@\textsc{Botticelli, Sandro} (1.\,3.\,1445 Florenz – 17.\,5.\,1510 ebd.), \emph{Maler}|pw}}, \textsc{Rafael\pwindex{Raffaello Sanzio da Urbino 28. 3. oder 6. 4. 1483 Urbino – 6.\,4.\,1520 Rom@\textsc{Raffaello Sanzio da Urbino} (28. 3. oder 6. 4. 1483 Urbino – 6.\,4.\,1520 Rom), \emph{Maler}|pw}} (jawohl, \textsc{Rafael\pwindex{Raffaello Sanzio da Urbino 28. 3. oder 6. 4. 1483 Urbino – 6.\,4.\,1520 Rom@\textsc{Raffaello Sanzio da Urbino} (28. 3. oder 6. 4. 1483 Urbino – 6.\,4.\,1520 Rom), \emph{Maler}|pw}}!). Aber als \uline{Städte}{ }ſind,{ }ſoweit ich bisher
               urtheilen kann, Mailand\oindex{Mailand@\textbf{Mailand}|pw}, \textsc{Bologna\oindex{Bologna@\textbf{Bologna}|pw}} oder gar Venedig\oindex{Venedig@\textbf{Venedig}|pw} viel{ }ſchöner.\pend
           
\pstart
           {\pb}Mach’ Dir in \label{K_L02888-1v}\edtext{Berlin\oindex{Berlin@\textbf{Berlin}, \emph{Hauptstadt}|pw}}{\lemma{\textnormal{\emph{Berlin}}}\Cendnote{\textnormal{Schnitzler reiste am 3. 10. 1899 von Wiesbaden\oindex{Wiesbaden@\textbf{Wiesbaden}|pwk} nach Berlin\oindex{Berlin@\textbf{Berlin}, \emph{Hauptstadt}|pwk} und blieb dort bis zum 11. 10. 1899.}}}\label{K_L02888-1} ein paar gute Tage!\pend
           
\pstart
           In Wien\oindex{Wien@\textbf{Wien}, \emph{Verwaltungsgebiet}|pw}{ }ſollſt Du mich nicht \label{K_L02888-2v}\edtext{erwarten}{\lemma{\textnormal{\emph{erwarten}}}\Cendnote{\textnormal{Goldmann\pwindex{Goldmann, Paul 31.\,1.\,1865 Breslau – 25.\,9.\,1935 Wien@\textsc{Goldmann, Paul} (31.\,1.\,1865 Breslau – 25.\,9.\,1935 Wien), \emph{Schriftsteller, Journalist}|pwk} kam am 13. 10. 1899 nach Wien\oindex{Wien@\textbf{Wien}, \emph{Verwaltungsgebiet}|pwk} und blieb bis zum 21. 10. 1899.}}}\label{K_L02888-2}.
               Ich käme gern, das brauche ich Dir wohl nicht zu{ }ſagen. Aber die Entfernung{ }ſchreckt
               mich. Die lange Eiſenbahnreiſe von Frankfurt\oindex{Frankfurt am Main@\textbf{Frankfurt am Main}, \emph{Hauptstadt}|pw}{ }hierher\oindex{Florenz@\textbf{Florenz}|pwv}{ }ſteckt mir heut noch in den Gliedern. Und dann langt{ }ſicherlich mein
               Geld nicht.\pend
           
\pstart
           Schreib’ mir wieder hierher\oindex{Florenz@\textbf{Florenz}|pwv}{ }\begin{otherlanguage}{french}\textsc{poste restante}\end{otherlanguage}!\pend
           
\pstart
           Viele treue Grüße! {\\[\baselineskip]}Dein {\\[\baselineskip]}\spacefill\mbox{Paul Goldmnn}\pend
           \leftskip=0em{}\selectlanguage{ngerman}\endnumbering\briefempfaengerindex{Schnitzler, Arthur@\textsc{Schnitzler, Arthur}!zzzGoldmann, Paul@\emph{von Paul Goldmann}!1899-09-294@{29. 9. [1899]}|)be}\mylabel{L02888h}  \newcommand{\dateiname}{L02888}\newcommand{\titel}{Paul Goldmann an Arthur Schnitzler, 29. 9. [1899]}\newcommand{\editorInnen}{Martin Anton Müller und Laura Untner}%% latex-leseansicht-abspann.tex
%% Abspann für die Leseansicht.
%% Der Schalter \ifkorrekturansicht ist bereits durch den Vorspann gesetzt.

%% latex-abspann.tex
%% Gemeinsamer Abspann für Korrekturansicht und Leseansicht.
%% Setzt den Schalter \ifkorrekturansicht voraus (gesetzt in den
%% einbindenden Dateien latex-korrekturansicht-abspann.tex bzw.
%% latex-leseansicht-abspann.tex).
%% ---------------------------------------------------------------

\normalsize

% Das esempio-Environment wird nur in der Leseansicht benötigt
\ifkorrekturansicht\else
\newenvironment{esempio}[3]%
{
    \vspace{1.5ex}
    \rlap{\underline{#1}}
    \par
    \setlength{\parindent}{0cm}
    \nopagebreak
    \leftskip=#2cm
    \rightskip=#3cm
}
{
    \par
}
\fi

\doendnotes{C}
\bigskip
\vfill

\clearpage

\footnotesize

\ifkorrekturansicht
  \lohead{\textsc{register}}
\fi

% theindex-Environment neu definieren ohne reledmac
\makeatletter
\renewenvironment{theindex}{%
  \ifkorrekturansicht
    \section*{\indexname}%
  \else
    \subsubsection*{Index der erwähnten Entitäten}%
  \fi
  \setlength{\parindent}{0pt}%
  \setlength{\parskip}{0pt plus 0.3pt}%
  \let\item\@idxitem
}{%
  \ifkorrekturansicht\clearpage\fi
}
\makeatother

\IfFileExists{\jobname-pw.ind}{\input{\jobname-pw.ind}}{}

% Quellenangabe nur in der Leseansicht
\ifkorrekturansicht\else
% Fallback-Definitionen, falls die .tex-Datei \titel etc. nicht gesetzt hat
\providecommand{\titel}{}
\providecommand{\editorInnen}{}
\providecommand{\dateiname}{\jobname}

\vspace{3cm}

\vfill

\footnotesize
\textsc{Quelle}: \titel. Herausgegeben von {\editorInnen}. In: \emph{Arthur Schnitzler: Briefwechsel mit Autorinnen und Autoren}.
 Digitale Edition, https://schnitzler-briefe.acdh.oeaw.ac.at/{\dateiname}.html (Stand \today)
\fi

\end{document}


