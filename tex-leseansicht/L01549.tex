%% latex-korrekturansicht-vorspann.tex
%% Vorspann für die Korrekturansicht.
%% Lädt die gemeinsame Datei latex-vorspann.tex mit gesetztem Schalter.

\newif\ifkorrekturansicht
\korrekturansichttrue

\input{../tex-inputs/latex-vorspann}


\section[Arthur Schnitzler an Hermann Bahr, 18. 9. 1905]{L01549 Arthur Schnitzler an Hermann Bahr, 18. 9. 1905}
\nopagebreak\mylabel{L01549v}
\rehead{ }\normalsize\beginnumbering\briefempfaengerindex{Bahr, Hermann@\textsc{Bahr, Hermann}!zzzSchnitzler, Arthur@\emph{von Arthur Schnitzler}!1905-09-181@{18. 9. 1905}|(be}
\toendnotes[C]{\smallbreak\pagebreak[2]}\Standort{TMW, HS AM 23377 Ba.}
\physDesc{Kartenbrief, 864 Zeichen
\newline{}Handschrift: schwarze Tinte, deutsche Kurrent
\newline{}Versand: 1) Stempel: »\nobreak{}Wien, 19. IX. 05\nobreak{}«.   2) Stempel: »\nobreak{}\oindex{XIII., Hietzing@\textbf{XIII., Hietzing}, \emph{A.ADM3}|pwk}Wien 13/7, 19. 9. 05\nobreak{}«. 
\newline{}Ordnung: Lochung }
\buchAbdrucke{\weitereDrucke{1) Arthur Schnitzler: \emph{The Letters of Arthur Schnitzler to Hermann Bahr}. Chapel Hill: \emph{The University of North Carolina Press} 1978, S. 91.} \weitereDrucke{2) Hermann Bahr, Arthur Schnitzler: \emph{Briefwechsel, Aufzeichnungen, Dokumente (1891–1931)}. Göttingen: \emph{Wallstein} 2018, S. 353.} }\toendnotes[C]{\smallbreak}\pstart{}{\pb}\textcolor{gray}{\textbf{Dr. Arthur Schnitzler}}\pend{}\pstart{}\textcolor{gray}{\textbf{Wien XVIII. Spoettelgasse 7\oindex{Edmund-Weiss-Gasse 7@\textbf{Edmund-Weiß-Gasse 7}, \emph{Wohngebäude (K.WHS)}|pw}.}}\pend{}{\bigskip}\pstart{}\textsc{Herrn Hermann Bahr}\pend{}\pstart{}\textsc{Wien Ober St Veit\oindex{Ober Sankt Veit@\textbf{Ober Sankt Veit}, \emph{P.PPLX}|pw}}\pend{}\pstart{}\textsc{Veitlissengasse\oindex{Veitlissengasse@\textbf{Veitlissengasse}, \emph{Straße (K.STR)}|pw}}\pend{}{\bigskip}\vspace{1em}
\pstart
           \raggedleft{}{\pb}18/9 905\pend
           \vspace{0.5em}
\pstart
           lieber Hermann, herzlichen Dank für deinen Brief. Es iſt mir ſehr
               wahrſcheinlich, daſs du in deinem Bedenken gegen den 2. Akt\pwindex{Ruf des Lebens. Schauspiel in drei Akten@\emph{Der Ruf des Lebens. Schauspiel in drei Akten}|pwv} recht haſt – vielleicht ſpricht sogar \uline{dafür}, dſs er beim Vorleſen i{\geminationm}er am ſtärkſten wirkte. Ob es aber in der Oekonomie
               gerade dieſes Stückes (ſo wie es mir eben eingefallen ist) \introOben{}möglich \substVorne{}\textsuperscript{iſt}\substDazwischen{}u\substHinten{} geſta\textcolor{gray}{t}tet iſt\introOben{} die Figuren dieſes Aktes, deren
                  (we{\geminationn} ich den Ausdruck erfinden darf) Fernhaftigkeit
               nicht allein im Unvermögen des Autors begründet liegt, realer zu machen, das iſt die
               Frage. (Bisher hat von allen Figuren immer der \damage{Ob}erſt am stärkſten gewirkt. Nun ja, gewirkt.)\pend
           
\pstart
           \label{K_L01549-1v}\edtext{Freitag fahr ich vielleicht auf 3–6
               Tage fort}{\lemma{\textnormal{\emph{Freitag … fort}}}\Cendnote{\textnormal{Schnitzler fuhr tatsächlich am
                  Freitag, den 22. 9. 1905 auf den Semmering\oindex{Semmering@\textbf{Semmering}, \emph{A.ADM3}|pwk} und kehrte am Donnerstag, den 26. 9. 1905
                  zurück.}}}\label{K_L01549-1}; aber da{\geminationn} muſs man ſich doch wirklich
               endlich, endlich ſehn. Das \textsc{Mscrpt}\pwindex{Ruf des Lebens. Schauspiel in drei Akten@\emph{Der Ruf des Lebens. Schauspiel in drei Akten}|pwv}{ }ſchicke mir gelegentlich, da ich nur 1 Ex. daheim
               habe, u das wieder fortſchicken muſs. – \pend
           
\pstart
           Herzlichſt dein{\\[\baselineskip]}\spacefill\mbox{A.}\pend
           \leftskip=0em{}\selectlanguage{ngerman}\endnumbering\briefempfaengerindex{Bahr, Hermann@\textsc{Bahr, Hermann}!zzzSchnitzler, Arthur@\emph{von Arthur Schnitzler}!1905-09-181@{18. 9. 1905}|)be}\mylabel{L01549h}  \normalsize

\doendnotes{C}
\bigskip
\vfill

\clearpage

\footnotesize

\lohead{\textsc{register}}

% Definiere theindex-Environment komplett neu ohne reledmac
\makeatletter
\renewenvironment{theindex}{%
  \section*{\indexname}%
  \setlength{\parindent}{0pt}%
  \setlength{\parskip}{0pt plus 0.3pt}%
  \let\item\@idxitem
}{%
  \clearpage
}
\makeatother

\IfFileExists{\jobname-pw.ind}{\input{\jobname-pw.ind}}{}

\end{document}

      