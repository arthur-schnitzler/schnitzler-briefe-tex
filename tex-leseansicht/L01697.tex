%% latex-korrekturansicht-vorspann.tex
%% Vorspann für die Korrekturansicht.
%% Lädt die gemeinsame Datei latex-vorspann.tex mit gesetztem Schalter.

\newif\ifkorrekturansicht
\korrekturansichttrue

\input{../tex-inputs/latex-vorspann}


\section[Richard Beer-Hofmann an Arthur Schnitzler, 30. 7. 1907]{L01697 Richard Beer-Hofmann an Arthur Schnitzler, 30. 7. 1907}
\nopagebreak\mylabel{L01697v}
\rehead{ }\normalsize\beginnumbering\briefempfaengerindex{Schnitzler, Arthur@\textsc{Schnitzler, Arthur}!zzzBeer-Hofmann, Richard@\emph{von Richard Beer-Hofmann}!1907-07-301@{30. 7. 1907}|(be}
\toendnotes[C]{\smallbreak\pagebreak[2]}\Standort{CUL, Schnitzler, B 8.}
\physDesc{Brief, 1 Blatt, 2 Seiten, 758 Zeichen (Briefpapier mit Trauerrand)
\newline{}Handschrift: schwarze Tinte, lateinische Kurrent
\newline{}Ordnung: mit Bleistift von unbekannter Hand nummeriert:
                                    »210« }
\buchAbdrucke{\weitereDrucke{Arthur Schnitzler, Richard Beer-Hofmann: \emph{Briefwechsel 1891–1931}. Wien, Zürich: \emph{Europaverlag} 1992, S. 182.} }\toendnotes[C]{\smallbreak}
\pstart
           \raggedleft{}{\pb}Maria Schutz\oindex{Maria Schutz@\textbf{Maria Schutz}, \emph{P.PPL}|pw}{ }30./VII 07.\pend
           \vspace{0.5em}
\pstart
           Lieber Arthur! Zwischen 14. u. 19. August,
               wollen wir von Wien\oindex{Wien@\textbf{Wien}, \emph{A.ADM2}|pw} abreisen das ergiebt, mit der
               Woche Kärnten\oindex{Kaernten@\textbf{Kärnten}, \emph{A.ADM1}|pw}, ein passiren des Pustertales\oindex{Pustertal@\textbf{Pustertal}, \emph{T.VAL}|pw} zwischen 23.–28. August.\pend
           
\pstart
           Wir sind aber müde, verprügelt, keine übermässig heitere Gesellschaft, und ich glaube
               nur mit Vorsicht zu gebrauchen wenn wir nicht wider unsern Willen andere versti{\geminationm}en sollen.\pend
           
\pstart
           {\pb}Freilich hoffe ich auf bessere
               Tage; wenn noch ein wenig Elastisches in uns ist, müssen wir wol nach so vieler
               Depression doch irgendeinmal wieder aufschnellen.\pend
           
\pstart
           Einen Brief an Hugo\pwindex{Hofmannsthal, Hugo von 1874-02-01 – 1929-07-15@\textsc{Hofmannsthal, Hugo von} (1874-02-01 – 1929-07-15), \emph{Schriftsteller/Schriftstellerin}|pw} habe ich dieser Tage nach
                  Waldbrunn\oindex{Wildbad Waldbrunn@\textbf{Wildbad Waldbrunn}, \emph{S.SPA}|pw} geschickt; fragen Sie, bitte,
               gelegentlich nach, \label{K_L01697-1v}\edtext{ob er
               nachgeschickt}{\lemma{\textnormal{\emph{ob er
               nachgeschickt}}}\Cendnote{\textnormal{Der Brief an Hofmannsthal\pwindex{Hofmannsthal, Hugo von 1874-02-01 – 1929-07-15@\textsc{Hofmannsthal, Hugo von} (1874-02-01 – 1929-07-15), \emph{Schriftsteller/Schriftstellerin}|pwk} wurde 
                  diesem nachgeschickt. Abdruck in: Hugo von Hofmannsthal\pwindex{Hofmannsthal, Hugo von 1874-02-01 – 1929-07-15@\textsc{Hofmannsthal, Hugo von} (1874-02-01 – 1929-07-15), \emph{Schriftsteller/Schriftstellerin}|pwk}, Richard Beer-Hofmann\pwindex{Beer-Hofmann, Richard 1866-07-11 – 1945-09-26@\textsc{Beer-Hofmann, Richard} (1866-07-11 – 1945-09-26), \emph{Schriftsteller/Schriftstellerin}|pwk}:
                     \emph{Briefwechsel}. Herausgegeben von Eugene Weber. Frankfurt am Main:
                     \emph{S. Fischer}{ }1972, S. 130.}}}\label{K_L01697-1} wurde.\pend
           
\pstart
           Sie verständigen mich von Ihren Reise- oder Abreiseplänen?\pend
           \pstart Herzlichst Ihr \spacefill\mbox{Richard}\pend{}
\pstart
           An Frau Olga\pwindex{Schnitzler, Olga 17.01.1882 – 13.01.1970@\textsc{Schnitzler, Olga} (17.01.1882 – 13.01.1970), \emph{Schauspieler/Schauspielerin, Sänger/Sängerin}|pw} von uns Beiden herzliche
               Grüsse.\pend
           \selectlanguage{ngerman}\endnumbering\briefempfaengerindex{Schnitzler, Arthur@\textsc{Schnitzler, Arthur}!zzzBeer-Hofmann, Richard@\emph{von Richard Beer-Hofmann}!1907-07-301@{30. 7. 1907}|)be}\mylabel{L01697h}  \normalsize

\doendnotes{C}
\bigskip
\vfill

\clearpage

\footnotesize

\lohead{\textsc{register}}

% Definiere theindex-Environment komplett neu ohne reledmac
\makeatletter
\renewenvironment{theindex}{%
  \section*{\indexname}%
  \setlength{\parindent}{0pt}%
  \setlength{\parskip}{0pt plus 0.3pt}%
  \let\item\@idxitem
}{%
  \clearpage
}
\makeatother

\IfFileExists{\jobname-pw.ind}{\input{\jobname-pw.ind}}{}

\end{document}

      