%% latex-leseansicht-vorspann.tex
%% Vorspann für die Leseansicht.
%% Lädt die gemeinsame Datei latex-vorspann.tex mit nicht gesetztem Schalter.

\newif\ifkorrekturansicht
\korrekturansichtfalse

\input{../tex-inputs/latex-vorspann}


\section[Richard Beer-Hofmann an Arthur Schnitzler, 30. 7. 1907]{L01697 Richard Beer-Hofmann an Arthur Schnitzler, 30. 7. 1907}
\nopagebreak\mylabel{L01697v}
\rehead{ }\normalsize\beginnumbering\briefempfaengerindex{Schnitzler, Arthur@\textsc{Schnitzler, Arthur}!zzzBeer-Hofmann, Richard@\emph{von Richard Beer-Hofmann}!1907-07-301@{30. 7. 1907}|(be}
\toendnotes[C]{\smallbreak\pagebreak[2]}
\correspDesc{Versand  durch Richard Beer-Hofmann am 30. 7. 1907 in Maria Schutz
\newline{}Erhalt  durch Arthur Schnitzler im Zeitraum [31. 7. 1907
                  – 4. 8. 1907?] in Welsberg-Taisten}\toendnotes[C]{\smallbreak}
\Standort{CUL, Schnitzler, B 8.}
\physDesc{Brief, 1 Blatt, 2 Seiten, 758 Zeichen (Briefpapier mit Trauerrand)
\newline{}Handschrift: schwarze Tinte, lateinische Kurrent
\newline{}Ordnung: mit Bleistift von unbekannter Hand nummeriert:
                                    »210« }
\buchAbdrucke{\weitereDrucke{Arthur Schnitzler, Richard Beer-Hofmann: \emph{Briefwechsel 1891–1931}. Herausgegeben von Konstanze Fliedl. Wien, Zürich: \emph{Europaverlag} 1992, S. 182.} }\toendnotes[C]{\smallbreak}
\pstart
           \raggedleft{}{\pb}Maria Schutz\oindex{Maria Schutz@\textbf{Maria Schutz}|pw}{ }30./VII 07.\pend
           \vspace{0.5em}
\pstart
           Lieber Arthur! Zwischen 14. u. 19. August,
               wollen wir von Wien\oindex{Wien@\textbf{Wien}, \emph{Verwaltungsgebiet}|pw} abreisen das ergiebt, mit der
               Woche Kärnten\oindex{Kärnten@\textbf{Kärnten}, \emph{Land}|pw}, ein passiren des Pustertales\oindex{Pustertal@\textbf{Pustertal}, \emph{Tal}|pw} zwischen 23.–28. August.\pend
           
\pstart
           Wir sind aber müde, verprügelt, keine übermässig heitere Gesellschaft, und ich glaube
               nur mit Vorsicht zu gebrauchen wenn wir nicht wider unsern Willen andere versti{\geminationm}en sollen.\pend
           
\pstart
           {\pb}Freilich hoffe ich auf bessere
               Tage; wenn noch ein wenig Elastisches in uns ist, müssen wir wol nach so vieler
               Depression doch irgendeinmal wieder aufschnellen.\pend
           
\pstart
           Einen Brief an Hugo\pwindex{Hofmannsthal, Hugo von 1.\,2.\,1874 Wien – 15.\,7.\,1929 Rodaun@\textsc{Hofmannsthal, Hugo von} (1.\,2.\,1874 Wien – 15.\,7.\,1929 Rodaun), \emph{Schriftsteller}|pw} habe ich dieser Tage nach
                  Waldbrunn\oindex{Wildbad Waldbrunn@\textbf{Wildbad Waldbrunn}, \emph{Spa}|pw} geschickt; fragen Sie, bitte,
               gelegentlich nach, \label{K_L01697-1v}\edtext{ob er
               nachgeschickt}{\lemma{\textnormal{\emph{ob er
               nachgeschickt}}}\Cendnote{\textnormal{Der Brief an Hofmannsthal\pwindex{Hofmannsthal, Hugo von 1.\,2.\,1874 Wien – 15.\,7.\,1929 Rodaun@\textsc{Hofmannsthal, Hugo von} (1.\,2.\,1874 Wien – 15.\,7.\,1929 Rodaun), \emph{Schriftsteller}|pwk} wurde 
                  diesem nachgeschickt. Abdruck in: Hugo von Hofmannsthal\pwindex{Hofmannsthal, Hugo von 1.\,2.\,1874 Wien – 15.\,7.\,1929 Rodaun@\textsc{Hofmannsthal, Hugo von} (1.\,2.\,1874 Wien – 15.\,7.\,1929 Rodaun), \emph{Schriftsteller}|pwk}, Richard Beer-Hofmann\pwindex{Beer-Hofmann, Richard 11.\,7.\,1866 Wien – 26.\,9.\,1945 New York City@\textsc{Beer-Hofmann, Richard} (11.\,7.\,1866 Wien – 26.\,9.\,1945 New York City), \emph{Schriftsteller}|pwk}:
                     \emph{Briefwechsel}. Herausgegeben von Eugene Weber. Frankfurt am Main:
                     \emph{S. Fischer}{ }1972, S. 130.}}}\label{K_L01697-1} wurde.\pend
           
\pstart
           Sie verständigen mich von Ihren Reise- oder Abreiseplänen?\pend
           \pstart Herzlichst Ihr \spacefill\mbox{Richard}\pend{}
\pstart
           An Frau Olga\pwindex{Schnitzler, Olga 17.\,1.\,1882 Wien – 13.\,1.\,1970 Lugano@\textsc{Schnitzler, Olga} (17.\,1.\,1882 Wien – 13.\,1.\,1970 Lugano), \emph{Schauspielerin, Sängerin}|pw} von uns Beiden herzliche
               Grüsse.\pend
           \selectlanguage{ngerman}\endnumbering\briefempfaengerindex{Schnitzler, Arthur@\textsc{Schnitzler, Arthur}!zzzBeer-Hofmann, Richard@\emph{von Richard Beer-Hofmann}!1907-07-301@{30. 7. 1907}|)be}\mylabel{L01697h}  \newcommand{\dateiname}{L01697}\newcommand{\titel}{Richard Beer-Hofmann an Arthur Schnitzler, 30. 7. 1907}\newcommand{\editorInnen}{Martin Anton Müller und Gerd-Hermann Susen}%% latex-leseansicht-abspann.tex
%% Abspann für die Leseansicht.
%% Der Schalter \ifkorrekturansicht ist bereits durch den Vorspann gesetzt.

%% latex-abspann.tex
%% Gemeinsamer Abspann für Korrekturansicht und Leseansicht.
%% Setzt den Schalter \ifkorrekturansicht voraus (gesetzt in den
%% einbindenden Dateien latex-korrekturansicht-abspann.tex bzw.
%% latex-leseansicht-abspann.tex).
%% ---------------------------------------------------------------

\normalsize

% Das esempio-Environment wird nur in der Leseansicht benötigt
\ifkorrekturansicht\else
\newenvironment{esempio}[3]%
{
    \vspace{1.5ex}
    \rlap{\underline{#1}}
    \par
    \setlength{\parindent}{0cm}
    \nopagebreak
    \leftskip=#2cm
    \rightskip=#3cm
}
{
    \par
}
\fi

\doendnotes{C}
\bigskip
\vfill

\clearpage

\footnotesize

\ifkorrekturansicht
  \lohead{\textsc{register}}
\fi

% theindex-Environment neu definieren ohne reledmac
\makeatletter
\renewenvironment{theindex}{%
  \ifkorrekturansicht
    \section*{\indexname}%
  \else
    \subsubsection*{Index der erwähnten Entitäten}%
  \fi
  \setlength{\parindent}{0pt}%
  \setlength{\parskip}{0pt plus 0.3pt}%
  \let\item\@idxitem
}{%
  \ifkorrekturansicht\clearpage\fi
}
\makeatother

\IfFileExists{\jobname-pw.ind}{\input{\jobname-pw.ind}}{}

% Quellenangabe nur in der Leseansicht
\ifkorrekturansicht\else
% Fallback-Definitionen, falls die .tex-Datei \titel etc. nicht gesetzt hat
\providecommand{\titel}{}
\providecommand{\editorInnen}{}
\providecommand{\dateiname}{\jobname}

\vspace{3cm}

\vfill

\footnotesize
\textsc{Quelle}: \titel. Herausgegeben von {\editorInnen}. In: \emph{Arthur Schnitzler: Briefwechsel mit Autorinnen und Autoren}.
 Digitale Edition, https://schnitzler-briefe.acdh.oeaw.ac.at/{\dateiname}.html (Stand \today)
\fi

\end{document}


