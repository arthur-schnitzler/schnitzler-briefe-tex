%% latex-korrekturansicht-vorspann.tex
%% Vorspann für die Korrekturansicht.
%% Lädt die gemeinsame Datei latex-vorspann.tex mit gesetztem Schalter.

\newif\ifkorrekturansicht
\korrekturansichttrue

\input{../tex-inputs/latex-vorspann}


\section[Robert Adam an Arthur Schnitzler, 15. 11. 1918]{L02311 Robert Adam an Arthur Schnitzler, 15. 11. 1918}
\nopagebreak\mylabel{L02311v}
\rehead{ }\normalsize\beginnumbering\briefempfaengerindex{Schnitzler, Arthur@\textsc{Schnitzler, Arthur}!zzzAdam, Robert@\emph{von Robert Adam}!1918-11-151@{15. 11. 1918}|(be}
\toendnotes[C]{\smallbreak\pagebreak[2]}\Standort{CUL, Schnitzler, B 1.}
\physDesc{Brief, 1 Blatt, 3 Seiten, 1922 Zeichen
\newline{}Handschrift: schwarze Tinte, deutsche Kurrent
\newline{}Schnitzler: 1) mit Bleistift beschriftet: »\textsc{Adam}«  2) mit rotem Buntstift drei Unterstreichungen
\newline{}Ordnung: von unbekannter Hand nummeriert: »9« }\Standort{Wien, Österreichische Nationalbibliothek, Cod.ser. 52.269, 225.}
\physDesc{Brief, maschinenschriftliche Abschrift1 Blatt, 1 Seite, 1922 Zeichen
\newline{}Schreibmaschine}\toendnotes[C]{\smallbreak}
\pstart
           \raggedleft{}{\pb}Wien\oindex{Wien@\textbf{Wien}, \emph{A.ADM2}|pw}, am 15. November 1918\pend
           
\pstart\center{}Hochverehrter Herr Doktor!\pend\vspace{0.5em}
\pstart
           Ich habe geſtern, ſofort nach Erhalt Ihres Schreibens, beide Stücke – den »Fremden\pwindex{Fremde@\emph{Der Fremde}|pw}« und »Yppl\pwindex{Yppl. Idylle in fuenf Akten@\emph{Yppl. Idylle in fünf Akten}|pw}« beim Deutſchen Volkstheater\oindex{Volkstheater@\textbf{Volkstheater}, \emph{Theater (K.THE)}|pw}
               eingereicht, und zwar zu Händen des Dramaturgen D\textsuperscript{r}{ }Glücksmann\pwindex{Gluecksmann, Heinrich 08.07.1864 – 01.03.1943@\textsc{Glücksmann, Heinrich} (08.07.1864 – 01.03.1943), \emph{Schriftsteller/Schriftstellerin, Journalist/Journalistin, Dramaturg/Dramaturgin}|pw}, dem ich einen kurzen an die
               Direktion gerichteten Brief mit Berufung auf Ihre mündliche Empfehlung übergab; in
               dieſem Schreiben wies ich darauf hin, daß es mit dem Stil des »Fremden\pwindex{Fremde@\emph{Der Fremde}|pw}« vereinbar wäre, wenn die Perſonen – wie auf Uhde\pwindex{Uhde, Fritz von 22.05.1848 – 25.02.1911@\textsc{Uhde, Fritz von} (22.05.1848 – 25.02.1911), \emph{Maler/Malerin}|pw}’ſchen Bildern – in modernen oder
               halbmodernen Koſtümen er{\pb}ſcheinen, daß
               daher die Koſtümfrage kaum Schwierigkeiten bereiten dürfte. Heute vormittags wollte
               ich beim Direktor\pwindex{Bernau, Alfred 06.03.1879 – 20.08.1950@\textsc{Bernau, Alfred} (06.03.1879 – 20.08.1950), \emph{Theaterleiter/Theaterleiterin, Schauspieler/Schauspielerin}|pw} vorſprechen, traf ihn aber
               nicht an und hinterließ meine Karte, wobei ich den Sekretär erſuchte, darauf
               aufmerkſam zu machen, daß die Stücke bereits eingereicht ſeien.\pend
           
\pstart
           Nun muß ich die Dinge ihren Lauf gehen laſſen und ſehe der Entſcheidung mit oft
               erprobtem Fatalismus entgegen. Hätte ich diesmal nicht wieder Pech, ſo wär’s ein
               Wunder! –\pend
           
\pstart
           Die letzten Tage, die uns die Republik und mir damit die Erfüllung langjähriger
               Träume gebracht haben, habe ich in größter Erregung durchlebt, von der auch eine
               ziemlich geſchmackloſe \label{K_L02311-1v}\edtext{Kundgebung}{\lemma{\textnormal{\emph{Kundgebung}}}\Cendnote{\textnormal{[O. V.]: \emph{Ein Richter für die Republik}\pwindex{Richter fuer die Republik@\emph{Ein Richter für die Republik}|pwk}.
                     In: \emph{Wiener Allgemeine Zeitung}\pwindex{Wiener Allgemeine Zeitung@\emph{Wiener Allgemeine Zeitung}|pwk}, Nr. 12.169,
                        12. 11. 1918, 6 Uhr-Blatt, S. 1: »An der Türe des Verhandlungssaales IV beim Bezirksgericht Josefstadt\orgindex{Bezirksgericht Wien Josefstadt@Bezirksgericht Wien Josefstadt|pw} war heute folgende \so{Kundmachung} auf einem halben Kanzleibogen zu lesen:{ / }›Am Tage, da die demokratische Republik und der Anschluß an Deutschland\oindex{Deutschland@\textbf{Deutschland}, \emph{A.PCLI}|pw} verkündigt wird, \so{will ich keine Strafurteile zu fällen haben. Die Strafverhandlungen werden daher nicht stattfinden. Es lebe die Republik!}{ / }12. November 1918.\hspace*{1.5em}
                        Landesgerichtsrat Dr. Pollak‹«.}}}\label{K_L02311-1} zeugt, die ich am Tage der Proklamation verbrach und {\pb}die ihren Weg in die Blätter gefunden hat
               (wie ich höre ſogar in’s \label{K_L02311-2v}\edtext{Prager Tagblatt\orgindex{Prager Tagblatt@Prager Tagblatt|pw}}{\lemma{\textnormal{\emph{Prager Tagblatt}}}\Cendnote{\textnormal{[O. V.]: \emph{Kein Strafurteil an dem ersten Tag
                        der Republik}\pwindex{Kein Strafurteil an dem ersten Tag der Republik@\emph{Kein Strafurteil an dem ersten Tag der Republik}|pwk}. In: \emph{Prager Tagblatt}\pwindex{Prager Tagblatt@\emph{Prager Tagblatt}|pwk},
                     Jg. 43, Nr. 264, 13. 11. 1918, Morgen-Ausgabe,
               S. 3.}}}\label{K_L02311-2}; dies iſt ſchließlich in Anbetracht der Eigentümlichkeit der Prag\oindex{Prag@\textbf{Prag}, \emph{A.ADM1}|pw}er Pſyche nichts Verwunderliches). Ich tröſte
               mich mit einem Spruch: »Begeiſterung macht Schmöcke aus uns allen«. – Ich habe auch
               die furchtbare Panik vor dem Parlament miterlebt und weiß jetzt, wie einem zumute
               iſt, wenn man wehrlos im Maſchinengewehrfeuer zu ſtehen vermeint. Es waren ganz
               entſetzliche und ſehr intereſſante Minuten. –\pend
           
\pstart
           Ich danke Ihnen herzlich für Ihre liebenswürdige Verwendung und gebe in Anbetracht
               derſelben, trotz allem Kleinmut, die Hoffnung nicht auf, diesmal doch einen
               Durchbruch zu erzielen.\pend
           
\pstart
           Mit beſten Grüßen Ihr ergebener\pend
           \pstart \spacefill\mbox{D\textsuperscript{r}RAdam}\pend{}\selectlanguage{ngerman}\endnumbering\briefempfaengerindex{Schnitzler, Arthur@\textsc{Schnitzler, Arthur}!zzzAdam, Robert@\emph{von Robert Adam}!1918-11-151@{15. 11. 1918}|)be}\mylabel{L02311h}  \normalsize

\doendnotes{C}
\bigskip
\vfill

\clearpage

\footnotesize

\lohead{\textsc{register}}

% Definiere theindex-Environment komplett neu ohne reledmac
\makeatletter
\renewenvironment{theindex}{%
  \section*{\indexname}%
  \setlength{\parindent}{0pt}%
  \setlength{\parskip}{0pt plus 0.3pt}%
  \let\item\@idxitem
}{%
  \clearpage
}
\makeatother

\IfFileExists{\jobname-pw.ind}{\input{\jobname-pw.ind}}{}

\end{document}

      