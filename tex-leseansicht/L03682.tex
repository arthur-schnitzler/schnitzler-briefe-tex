%% latex-korrekturansicht-vorspann.tex
%% Vorspann für die Korrekturansicht.
%% Lädt die gemeinsame Datei latex-vorspann.tex mit gesetztem Schalter.

\newif\ifkorrekturansicht
\korrekturansichttrue

\input{../tex-inputs/latex-vorspann}


\section[Stefan Zweig an Arthur Schnitzler, 7. 1. 192{[}8?{]}]{L03682 Stefan Zweig an Arthur Schnitzler, 7. 1. 192{[}8?{]}}
\nopagebreak\mylabel{L03682v}
\rehead{ }\normalsize\beginnumbering\briefempfaengerindex{Schnitzler, Arthur@\textsc{Schnitzler, Arthur}!zzzZweig, Stefan@\emph{von Stefan Zweig}!1928-01-071@{7. 1. 192{[}8?{]}}|(be}
\toendnotes[C]{\smallbreak\pagebreak[2]}\Standort{DLA, A:Schnitzler, HS.2009.87.}
\physDesc{XXXX PHYSDESC FEHLER}
\buchAbdrucke{\weitereDrucke{1) Stefan Zweig: \emph{Briefe an Freunde}. Frankfurt am Main: \emph{S. Fischer} 1978, S. 175–177.} \weitereDrucke{2) Stefan Zweig: \emph{Briefwechsel mit Hermann Bahr, Sigmund Freud, Rainer Maria
                        Rilke und Arthur Schnitzler}. Frankfurt am Main: \emph{S. Fischer} 1987, S. 432–434.} }\toendnotes[C]{\smallbreak}
\pstart
           \raggedleft{}{\pb}7. 1. \label{K_L03682-1v}\edtext{1927}{\lemma{\textnormal{\emph{1927}}}\Cendnote{\textnormal{Schreibirrtum Zweigs\pwindex{Zweig, Stefan 28.11.1881 – 23.02.1942@\textsc{Zweig, Stefan} (28.11.1881 – 23.02.1942), \emph{Schriftsteller/Schriftstellerin}|pwk}, wie sich aus
                      dem Inhalt und dem Antwortschreiben Schnitzlers vom 16. 1. 1928 ergibt.}}}\label{K_L03682-1}\pend
           
\pstart
           \raggedleft{}Kapuzinerberg 5\oindex{Paschinger Schloessl@\textbf{Paschinger Schlössl}, \emph{Wohngebäude (K.WHS)}|pw}\pend
           
\pstart
           \raggedleft{}Salzburg\oindex{Salzburg@\textbf{Salzburg}, \emph{A.ADM2}|pw}\pend
           \vspace{0.5em}
\pstart
           Lieber verehrter Herr Doktor, Ihr Buch\pwindex{Geist im Wort und der Geist in der Tat@\emph{Der Geist im Wort und der Geist in der Tat}|pwv} war mir eine grosse Freude und eine besonders
               persönliche: ich habe immer das Gefühl gehabt, als wüsste man zu wenig von Ihrer
               innern Geistigkeit, ihrer Gefühlswärme und dem Ernst hinter ihrem Lächeln. Wer einmal
               den Menschen heiter kommt, scheint verwirkt zu haben, für seriös im strengen Sinne zu
               gelten, als ob nicht gerade das Spielhafte immer Erlösung von einem tiefen innern
               Ernst bedeutete: Sie haben nur zu recht, dass die Wenigsten eigentlich von Ihnen
               hinter Ihrem Ruhme wissen. Zu diesen zu zählen war immer mein Stolz. Das Einzige, was
               mich an diesen Sprüchen\pwindex{Geist im Wort und der Geist in der Tat@\emph{Der Geist im Wort und der Geist in der Tat}|pwv} ein wenig
               verdross, war, um goethisch\pwindex{Goethe, Johann Wolfgang von 1749-08-28 – 1832-03-22@\textsc{Goethe, Johann Wolfgang von} (1749-08-28 – 1832-03-22), \emph{Schriftsteller/Schriftstellerin}|pwv} zu reden
               »das Buch des Unmuts«, nämlich dass Sie den Kleingeistigen die Freude machen, zu
               zeigen, dass Mückenstiche Sie manchmal ärgerten. Zu viel Ehre! Wer wie Sie auf einem
               Werke steht, kann herabsehen; Verachtung zu zeigen, verrät eine vorangegangene
               Entrüstung und die hätten Sie niemals an solchen engen Deutungen erfahren sollen.
               Notwendigerweise hält sich der lockere Geist am Äusseren, aus Faulheit, in die Tiefe
               zu dringen, er klammert sich an einen Begriff und der ist Ihnen durch das Deminutiv
               der »Liebelei\pwindex{Liebelei. Schauspiel in drei Akten@\emph{Liebelei. Schauspiel in drei Akten}|pwv}« von anfangs an taxfrei
               verliehen worden. Lassen Sie der Zeit ihre Zeit und Sie werden selbst noch die
               Wandlung erfahren, dieselbe die allen Österreichern\oindex{Oesterreich@\textbf{Österreich}, \emph{A.PCLI}|pw}
               allmählich bewilligt wurde, sehr unwillig zwar aber dann umso dauerhafter. Aber Ihr
               Buch\pwindex{Geist im Wort und der Geist in der Tat@\emph{Der Geist im Wort und der Geist in der Tat}|pwv} war fördernd für ein ernsteres
               Anschaun, ein Sich besinnen dieser Gleichgiltigkeit, die ich für Sie empörter
               empfinde als Sie selbst: Ihre hohe Haltung, der nicht im schulmässigen wohl aber viel
               intensiveren Sinne sittliche Ernst Ihres Werks waren für mich immer vorbildlich und
               werden es dauernd bleiben, denn immer wieder steht Ihr neues Schaffen auf einer neuen
               Stufe, andern Ausblick eröffnend und gleichsam tiefere Quellen aufdeutend. Ich
               erwarte mir gerade von diesen Ihren reifsten Jahren noch unendlich viel und da Sies
               nie getan haben, werden Sie mich auch in dieser liebevollen Erwartung nicht
               enttäuschen.\pend
           
\pstart
           Von mir darf ich nichts sagen als dass ein neues Drei-Meisterbuch\pwindex{Drei Dichter ihres Lebens. Casanova – Stendhal – Tolstoi@\emph{Drei Dichter ihres Lebens. Casanova – Stendhal – Tolstoi}|pwv} das meiner eigenen Arbeit wie ein Klotz im
               Wege gelegen, bald fortgerollt sein wird und ich wieder dem Erfinderischen mich
               nähern kann. Inzwischen fiel mir eine kleine Komödie\pwindex{Quiproquo. Komoedie in drei Akten@\emph{Quiproquo. Komödie in drei Akten}|pwv} ein, die zu schreiben ich allein zu träge bin; aber schon in
               Gedanken mit Heiterkeiten zu spielen, entlastet. Ich glaube man kann sich nur von
               einer Arbeit in der andern erholen oder wenigstens im Spiel mit neuen Plänen und
               Möglichkeiten. Möge jeder Tag Ihnen freudig und erfüllt sein. Wer verdient dies
               Bedeutsamste wenn nicht Sie? \pend
           
\pstart
           Innigst Ihnen getreu{\\[\baselineskip]}Ihr \spacefill\mbox{Stefan Zweig}\pend
           \leftskip=0em{}\selectlanguage{ngerman}\endnumbering\briefempfaengerindex{Schnitzler, Arthur@\textsc{Schnitzler, Arthur}!zzzZweig, Stefan@\emph{von Stefan Zweig}!1928-01-071@{7. 1. 192{[}8?{]}}|)be}\mylabel{L03682h}
\begin{anhang}
\end{anhang}\normalsize

\doendnotes{C}
\bigskip
\vfill

\clearpage

\footnotesize

\lohead{\textsc{register}}

% Definiere theindex-Environment komplett neu ohne reledmac
\makeatletter
\renewenvironment{theindex}{%
  \section*{\indexname}%
  \setlength{\parindent}{0pt}%
  \setlength{\parskip}{0pt plus 0.3pt}%
  \let\item\@idxitem
}{%
  \clearpage
}
\makeatother

\IfFileExists{\jobname-pw.ind}{\input{\jobname-pw.ind}}{}

\end{document}

      