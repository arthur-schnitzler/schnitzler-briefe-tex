%% latex-leseansicht-vorspann.tex
%% Vorspann für die Leseansicht.
%% Lädt die gemeinsame Datei latex-vorspann.tex mit nicht gesetztem Schalter.

\newif\ifkorrekturansicht
\korrekturansichtfalse

\input{../tex-inputs/latex-vorspann}


\section[Stefan Zweig an Arthur Schnitzler, 7. 1. 192{[}8?{]}]{L03682 Stefan Zweig an Arthur Schnitzler, 7. 1. 192[8?]}
\nopagebreak\mylabel{L03682v}
\rehead{ }\normalsize\beginnumbering\briefempfaengerindex{Schnitzler, Arthur@\textsc{Schnitzler, Arthur}!zzzZweig, Stefan@\emph{von Stefan Zweig}!1928-01-071@{7. 1. 192[8?]}|(be}
\toendnotes[C]{\smallbreak\pagebreak[2]}
\correspDesc{Versand  durch Stefan Zweig am 7. 1. 192[8?] in Salzburg
\newline{}Erhalt  durch Arthur Schnitzler im Zeitraum [8. 1. 1928
                  – 12. 1. 1928?] in Wien}\toendnotes[C]{\smallbreak}
\Standort{DLA, A:Schnitzler, HS.2009.87.}
\physDesc{Brief, 1 Blatt, 2 Seiten, 2631 Zeichen
\newline{}Handschrift: lila Tinte, lateinische Kurrent
\newline{}Schnitzler: mit rotem Buntstift 13 Unterstreichungen und Vermerk: »\noindent{}\textsc{Zweig}{ / }\textsc{Aph}{[}orismen{]}« }
\buchAbdrucke{\weitereDrucke{1) Stefan Zweig: \emph{Briefe an Freunde}. Herausgegeben von Richard Friedenthal. Frankfurt am Main: \emph{S. Fischer} 1978, S. 175–177.} \weitereDrucke{2) Stefan Zweig: \emph{Briefwechsel mit Hermann Bahr, Sigmund Freud, Rainer Maria
                        Rilke und Arthur Schnitzler}. Herausgegeben von Jeffrey B. Berlin, Hans-Ulrich Lindken und Donald A. Prater. Frankfurt am Main: \emph{S. Fischer} 1987, S. 432–434.} }\toendnotes[C]{\smallbreak}
\pstart
           {\pb}\textcolor{gray}{\textbf{SZ}}\hfill 7. 1. \label{K_L03682-1v}\edtext{1927}{\lemma{\textnormal{\emph{1927}}}\Cendnote{\textnormal{Schreibirrtum Zweigs\pwindex{Zweig, Stefan 28.\,11.\,1881 Wien – 23.\,2.\,1942 Petrópolis@\textsc{Zweig, Stefan} (28.\,11.\,1881 Wien – 23.\,2.\,1942 Petrópolis), \emph{Schriftsteller}|pwk}, wie sich aus dem Inhalt und dem
                           Antwortschreiben Schnitzlers vom
                              XXXX Auszeichnungsfehler: Dokument L03741 nicht gefunden
                           ergibt.}}}\label{K_L03682-1}\pend
           
\pstart
           \raggedleft{}\textcolor{gray}{\textbf{KAPUZINERBERG{ }5\oindex{Paschinger Schlössl@\textbf{Paschinger Schlössl}, \emph{Wohngebäude}|pw}}}\pend
           
\pstart
           \raggedleft{}\textcolor{gray}{\textbf{SALZBURG\oindex{Salzburg@\textbf{Salzburg}, \emph{Verwaltungsgebiet}|pw}}}\pend
           \vspace{0.5em}
\pstart
           Lieber verehrter Herr Doktor, Ihr Buch\pwindex{Schnitzler, Arthur 15.\,5.\,1862 Wien – 21.\,10.\,1931 ebd.@\textsc{Schnitzler, Arthur} (15.\,5.\,1862 Wien – 21.\,10.\,1931 ebd.), \emph{Schriftsteller, Mediziner}!Geist im Wort und der Geist in der Tat@\strich\emph{Der Geist im Wort und der Geist in der Tat}|pwv} war mir eine grosse Freude und eine besonders
               persönliche: ich habe immer das Gefühl gehabt, als wüsste man zu wenig von Ihrer
               innern Geistigkeit, ihrer Gefühlswärme und dem Ernst hinter ihrem Lächeln. Wer einmal
               den Menschen heiter kommt, scheint verwirkt zu haben, für seriös im strengen Sinne zu
               gelten, als ob nicht gerade das Spielhafte immer Erlösung von einem tiefen innern
               Ernst bedeutete: Sie haben nur zu recht, dass die Wenigsten eigentlich von Ihnen
               hinter Ihrem Ruhme wissen. Zu diesen zu zählen war immer mein Stolz. Das Einzige, was
               mich an diesen Sprüchen\pwindex{Schnitzler, Arthur 15.\,5.\,1862 Wien – 21.\,10.\,1931 ebd.@\textsc{Schnitzler, Arthur} (15.\,5.\,1862 Wien – 21.\,10.\,1931 ebd.), \emph{Schriftsteller, Mediziner}!Geist im Wort und der Geist in der Tat@\strich\emph{Der Geist im Wort und der Geist in der Tat}|pwv} ein
               wenig verdross, war, um goethisch\pwindex{Goethe, Johann Wolfgang von 28.\,8.\,1749 Frankfurt am Main – 22.\,3.\,1832 Weimar@\textsc{Goethe, Johann Wolfgang von} (28.\,8.\,1749 Frankfurt am Main – 22.\,3.\,1832 Weimar), \emph{Schriftsteller}|pwv} zu reden »das Buch des Unmuts«, nämlich dass Sie den
               Kleingeistigen die Freude machen, zu zeigen, dass Mückenstiche Sie manchmal ärgerten.
               Zu viel Ehre! Wer wie Sie auf einem Werke steht, kann herabsehen; Verachtung zu
               zeigen, verrät eine vorangegangene Entrüstung und die hätten Sie niemals an solchen
               engen Deutungen erfahren sollen. Notwendigerweise hält sich der lockere Geist am
               Äusseren, aus Faulheit, in die Tiefe zu dringen, er klammert sich an einen Begriff
               und der ist Ihnen durch das Deminutiv der »Liebel\uline{ei}\pwindex{Schnitzler, Arthur 15.\,5.\,1862 Wien – 21.\,10.\,1931 ebd.@\textsc{Schnitzler, Arthur} (15.\,5.\,1862 Wien – 21.\,10.\,1931 ebd.), \emph{Schriftsteller, Mediziner}!Liebelei. Schauspiel in drei Akten@\strich\emph{Liebelei. Schauspiel in drei Akten}|pwv}« von anfangs an taxfrei verliehen worden. Lassen Sie
               der Zeit ihre Zeit und Sie werden selbst noch die Wandlung erfahren, die{\pb}selbe die allen Österreichern\oindex{Österreich@\textbf{Österreich}|pw} allmählich bewilligt wurde, sehr unwillig zwar aber dann umso
               dauerhafter. Aber Ihr Buch\pwindex{Schnitzler, Arthur 15.\,5.\,1862 Wien – 21.\,10.\,1931 ebd.@\textsc{Schnitzler, Arthur} (15.\,5.\,1862 Wien – 21.\,10.\,1931 ebd.), \emph{Schriftsteller, Mediziner}!Geist im Wort und der Geist in der Tat@\strich\emph{Der Geist im Wort und der Geist in der Tat}|pwv} war
               fördernd für ein ernsteres Anschaun, ein Sich besinnen dieser Gleichgiltigkeit, die
               ich für Sie empörter empfinde als Sie selbst: Ihre hohe Haltung, der nicht im
               schulmässigen wohl aber viel intensiveren Sinne sittliche Ernst Ihres Werks waren für
               mich immer vorbildlich und werden es dauernd bleiben, denn immer wieder steht Ihr
               neues Schaffen auf einer neuen Stufe, andern Ausblick eröffnend und gleichsam tiefere
               Quellen aufdeutend. Ich erwarte mir gerade von diesen Ihren reifsten Jahren noch
               unendlich viel und da Sies nie getan haben, werden Sie mich auch in dieser
               liebevollen Erwartung nicht enttäuschen.\pend
           
\pstart
           Von mir darf ich nichts sagen als dass ein neues Drei-Meisterbuch\pwindex{Zweig, Stefan 28.\,11.\,1881 Wien – 23.\,2.\,1942 Petrópolis@\textsc{Zweig, Stefan} (28.\,11.\,1881 Wien – 23.\,2.\,1942 Petrópolis), \emph{Schriftsteller}!Drei Dichter ihres Lebens. Casanova – Stendhal – Tolstoi@\strich\emph{Drei Dichter ihres Lebens. Casanova – Stendhal – Tolstoi}|pwv} das meiner eigenen Arbeit wie ein Klotz im
               Wege gelegen, bald fortgerollt sein wird und ich wieder dem Erfinderischen mich
               nähern kann. Inzwischen fiel mir eine kleine Komödie\pwindex{Zweig, Stefan 28.\,11.\,1881 Wien – 23.\,2.\,1942 Petrópolis@\textsc{Zweig, Stefan} (28.\,11.\,1881 Wien – 23.\,2.\,1942 Petrópolis), \emph{Schriftsteller}!Quiproquo. Komödie in drei Akten@\strich\emph{Quiproquo. Komödie in drei Akten}|pwv}\pwindex{\textcolor{red}{\textsuperscript{XXXX indx1}}!Quiproquo. Komödie in drei Akten@\strich\emph{Quiproquo. Komödie in drei Akten}|pwv} ein, die zu schreiben ich allein zu träge bin; aber
               schon in Gedanken mit Heiterkeiten zu spielen, entlastet. Ich glaube man kann sich
               nur von einer Arbeit in der andern erholen oder wenigstens im Spiel mit neuen Plänen
               und Möglichkeiten.\pend
           
\pstart
           Möge jeder Tag Ihnen freudig und erfüllt sein! Wer verdient dies Bedeutsamste wenn
               nicht Sie?\pend
           \pstart Innigst Ihnen getreu Ihr \spacefill\mbox{Stefan Zweig}\pend{}\selectlanguage{ngerman}\endnumbering\briefempfaengerindex{Schnitzler, Arthur@\textsc{Schnitzler, Arthur}!zzzZweig, Stefan@\emph{von Stefan Zweig}!1928-01-071@{7. 1. 192[8?]}|)be}\mylabel{L03682h}  \newcommand{\dateiname}{L03682}\newcommand{\titel}{Stefan Zweig an Arthur Schnitzler, 7. 1. 192[8?]}\newcommand{\editorInnen}{Selma Jahnke und Martin Anton Müller}%% latex-leseansicht-abspann.tex
%% Abspann für die Leseansicht.
%% Der Schalter \ifkorrekturansicht ist bereits durch den Vorspann gesetzt.

%% latex-abspann.tex
%% Gemeinsamer Abspann für Korrekturansicht und Leseansicht.
%% Setzt den Schalter \ifkorrekturansicht voraus (gesetzt in den
%% einbindenden Dateien latex-korrekturansicht-abspann.tex bzw.
%% latex-leseansicht-abspann.tex).
%% ---------------------------------------------------------------

\normalsize

% Das esempio-Environment wird nur in der Leseansicht benötigt
\ifkorrekturansicht\else
\newenvironment{esempio}[3]%
{
    \vspace{1.5ex}
    \rlap{\underline{#1}}
    \par
    \setlength{\parindent}{0cm}
    \nopagebreak
    \leftskip=#2cm
    \rightskip=#3cm
}
{
    \par
}
\fi

\doendnotes{C}
\bigskip
\vfill

\clearpage

\footnotesize

\ifkorrekturansicht
  \lohead{\textsc{register}}
\fi

% theindex-Environment neu definieren ohne reledmac
\makeatletter
\renewenvironment{theindex}{%
  \ifkorrekturansicht
    \section*{\indexname}%
  \else
    \subsubsection*{Index der erwähnten Entitäten}%
  \fi
  \setlength{\parindent}{0pt}%
  \setlength{\parskip}{0pt plus 0.3pt}%
  \let\item\@idxitem
}{%
  \ifkorrekturansicht\clearpage\fi
}
\makeatother

\IfFileExists{\jobname-pw.ind}{\input{\jobname-pw.ind}}{}

% Quellenangabe nur in der Leseansicht
\ifkorrekturansicht\else
% Fallback-Definitionen, falls die .tex-Datei \titel etc. nicht gesetzt hat
\providecommand{\titel}{}
\providecommand{\editorInnen}{}
\providecommand{\dateiname}{\jobname}

\vspace{3cm}

\vfill

\footnotesize
\textsc{Quelle}: \titel. Herausgegeben von {\editorInnen}. In: \emph{Arthur Schnitzler: Briefwechsel mit Autorinnen und Autoren}.
 Digitale Edition, https://schnitzler-briefe.acdh.oeaw.ac.at/{\dateiname}.html (Stand \today)
\fi

\end{document}


