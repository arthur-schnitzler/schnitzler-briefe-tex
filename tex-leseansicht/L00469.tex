%% latex-korrekturansicht-vorspann.tex
%% Vorspann für die Korrekturansicht.
%% Lädt die gemeinsame Datei latex-vorspann.tex mit gesetztem Schalter.

\newif\ifkorrekturansicht
\korrekturansichttrue

\input{../tex-inputs/latex-vorspann}


\section[Arthur Schnitzler an Richard Beer-Hofmann, {[}5. 8. 1895?{]}]{L00469 Arthur Schnitzler an Richard Beer-Hofmann, {[}5. 8. 1895?{]}}
\nopagebreak\mylabel{L00469v}
\rehead{ }\normalsize\beginnumbering\briefempfaengerindex{Beer-Hofmann, Richard@\textsc{Beer-Hofmann, Richard}!zzzSchnitzler, Arthur@\emph{von Arthur Schnitzler}!1895-08-051@{{[}5. 8. 1895?{]}}|(be}
\toendnotes[C]{\smallbreak\pagebreak[2]}\Standort{YCGL, MSS 31.}
\physDesc{Briefkarte, 260 Zeichen
\newline{}Handschrift: Bleistift, deutsche Kurrent}\toendnotes[C]{\smallbreak}
\pstart
           \noindent{}{\pb}Lieber Richard!{ }\label{K_L00469-1v}\edtext{\textsc{Salten}\pwindex{Salten, Felix 06.09.1869 – 08.10.1945@\textsc{Salten, Felix} (06.09.1869 – 08.10.1945), \emph{Schriftsteller/Schriftstellerin, Journalist/Journalistin, Chefredakteur/Chefredakteurin}|pw}}{\lemma{\textnormal{\emph{Salten}}}\Cendnote{\textnormal{Das verwendete Papier (und die
                  Einordnung zwischen die anderen Korrespondenzstücke im Archiv) deuten auf
                     1895. Aus dem Inhalt geht hervor, dass die Kommunikation außerhalb
                  von Wien\oindex{Wien@\textbf{Wien}, \emph{A.ADM2}|pwk}{ }stattfindet (»angeko{\geminationm}en«). Das reduziert die durch das \emph{Tagebuch}\pwindex{Tagebuch@\emph{Tagebuch}|pwk} möglichen Daten auf 5. 8. 1895 und 16. 8. 1895. Beim
                  zweiten Termin kündigt Salten\pwindex{Salten, Felix 06.09.1869 – 08.10.1945@\textsc{Salten, Felix} (06.09.1869 – 08.10.1945), \emph{Schriftsteller/Schriftstellerin, Journalist/Journalistin, Chefredakteur/Chefredakteurin}|pwk} aber an, einen
                  späteren Zug zu nehmen. Auch dürfte sich Beer-Hofmann\pwindex{Beer-Hofmann, Richard 1866-07-11 – 1945-09-26@\textsc{Beer-Hofmann, Richard} (1866-07-11 – 1945-09-26), \emph{Schriftsteller/Schriftstellerin}|pwk} zu diesem Zeitpunkt nicht in Ischl\oindex{Bad Ischl@\textbf{Bad Ischl}, \emph{P.PPL}|pwk} aufgehalten haben, was den 5. 8. 1895 wahrscheinlich
                  macht. Im Theater hat Schnitzler an diesem
                  Tag \emph{Zwei glückliche Tage}\pwindex{Zwei glueckliche Tage@\emph{Zwei glückliche Tage}|pwk} gesehen.}}}\label{K_L00469-1} iſt erſt
               kurz vor 1 hier angeko{\geminationm}en. – Haben Sie
               ſchon einen Sitz für mich geno{\geminationm}en, ſo geh ich natürlich
               ins Theater – nicht – nicht. – Für alle Fälle laſſen Sie mir was ſagen. {\pb}Iſts Ihnen recht, ko{\geminationm} ich
               mit \textsc{S.}\pwindex{Salten, Felix 06.09.1869 – 08.10.1945@\textsc{Salten, Felix} (06.09.1869 – 08.10.1945), \emph{Schriftsteller/Schriftstellerin, Journalist/Journalistin, Chefredakteur/Chefredakteurin}|pw} zwiſchen 5 u 6 zu Ihnen.\pend
           
\pstart
           Herzlich{\\[\baselineskip]}Ihr \spacefill\mbox{Arth}\pend
           \leftskip=0em{}\selectlanguage{ngerman}\endnumbering\briefempfaengerindex{Beer-Hofmann, Richard@\textsc{Beer-Hofmann, Richard}!zzzSchnitzler, Arthur@\emph{von Arthur Schnitzler}!1895-08-051@{{[}5. 8. 1895?{]}}|)be}\mylabel{L00469h}  \normalsize

\doendnotes{C}
\bigskip
\vfill

\clearpage

\footnotesize

\lohead{\textsc{register}}

% Definiere theindex-Environment komplett neu ohne reledmac
\makeatletter
\renewenvironment{theindex}{%
  \section*{\indexname}%
  \setlength{\parindent}{0pt}%
  \setlength{\parskip}{0pt plus 0.3pt}%
  \let\item\@idxitem
}{%
  \clearpage
}
\makeatother

\IfFileExists{\jobname-pw.ind}{\input{\jobname-pw.ind}}{}

\end{document}

      