%% latex-leseansicht-vorspann.tex
%% Vorspann für die Leseansicht.
%% Lädt die gemeinsame Datei latex-vorspann.tex mit nicht gesetztem Schalter.

\newif\ifkorrekturansicht
\korrekturansichtfalse

\input{../tex-inputs/latex-vorspann}


         
         \renewcommand{\erwaehntePersonen}{Personen: Richard Beer-Hofmann, Felix Salten}
         \renewcommand{\erwaehnteOrte}{Orte: Bad Ischl, Wien}
         \renewcommand{\erwaehnteWerke}{Werke: Tagebuch, Zwei glückliche Tage}
               \section[Arthur Schnitzler an Richard Beer-Hofmann, {[}5. 8. 1895?{]}]{ Arthur Schnitzler an Richard Beer-Hofmann, {[}5. 8. 1895?{]}}\nopagebreak\mylabel{v}\rehead{ }\begin{ledgroupsized}[t]{13cm}\normalsize\beginnumbering \toendnotes[C]{\smallbreak\pagebreak[2]} \Standort{YCGL, MSS 31.}
\physDesc{Briefkarte
\newline{}Handschrift: Bleistift, deutsche Kurrent}\toendnotes[C]{\smallbreak}\pstart
           \noindent{}{\pb}Lieber Richard! \label{K_L00469_1v}\edtext{\textsc{Salten}\pwindex{Salten, Felix 06.09.1869 – 08.10.1945@\textsc{Salten, Felix} (06.09.1869 – 08.10.1945), \emph{Schriftsteller, Journalist}|pw}}{\lemma{\textnormal{\emph{Salten}}}\Cendnote{\textnormal{Das verwendete Papier (und die
                  Einordnung zwischen die anderen Korrespondenzstücke im Archiv) deuten auf
                     1895. Aus dem Inhalt geht hervor, dass die Kommunikation
                  außerhalb von Wien\oindex{Wien@\textbf{Wien}|pwk}{ }stattfindet (»angeko{\geminationm}en«). Das reduziert die durch das \emph{Tagebuch}\pwindex{Schnitzler, Arthur 15.05.1862 – 21.10.1931@\textsc{Schnitzler, Arthur} (15.05.1862 – 21.10.1931), \emph{Schriftsteller, Mediziner}!Tagebuch1981 – 2000@\strich\emph{Tagebuch} {[}1981 – 2000{]}|pwk} möglichen Daten auf 5. 8. 1895 und 16. 8. 1895. Beim zweiten
                  Termin kündigt Salten\pwindex{Salten, Felix 06.09.1869 – 08.10.1945@\textsc{Salten, Felix} (06.09.1869 – 08.10.1945), \emph{Schriftsteller, Journalist}|pwk} aber an, einen späteren
                  Zug zu nehmen. Auch dürfte sich Beer-Hofmann\pwindex{Beer-Hofmann, Richard 1866-07-11 – 1945-09-26@\textsc{Beer-Hofmann, Richard} (1866-07-11 – 1945-09-26), \emph{Schriftsteller}|pwk}
                  zu diesem Zeitpunkt nicht in Ischl\oindex{Bad Ischl@\textbf{Bad Ischl}|pwk} aufgehalten
                  haben, was den 5. 8. 1895 wahrscheinlich macht. Im Theater sieht Schnitzler\pwindex{Schnitzler, Arthur 15.05.1862 – 21.10.1931@\textsc{Schnitzler, Arthur} (15.05.1862 – 21.10.1931), \emph{Schriftsteller, Mediziner}|pwk} an diesem Tag \emph{Zwei glückliche Tage}\pwindex{\textcolor{red}{\textsuperscript{XXXX1 indx}}!Zwei glueckliche Tage1892@\strich\emph{Zwei glückliche Tage} {[}1892{]}|pwk}\pwindex{\textcolor{red}{\textsuperscript{XXXX1 indx}}!Zwei glueckliche Tage1892@\strich\emph{Zwei glückliche Tage} {[}1892{]}|pwk}.}}}\label{K_L00469_1h} iſt erſt kurz vor 1
               hier angeko{\geminationm}en. – Haben Sie ſchon einen Sitz für mich
                  geno{\geminationm}en, ſo geh ich natürlich ins Theater – nicht –
               nicht. – Für alle Fälle laſſen Sie mir was ſagen. {\pb}Iſts Ihnen recht, ko{\geminationm} ich mit \textsc{S.}\pwindex{Salten, Felix 06.09.1869 – 08.10.1945@\textsc{Salten, Felix} (06.09.1869 – 08.10.1945), \emph{Schriftsteller, Journalist}|pw} zwiſchen 5 u 6 zu Ihnen.\pend
           \pstart
           Herzlich{\\[\baselineskip]}Ihr \spacefill\mbox{Arth}\pend
           \leftskip=0em{}
         
         \endnumbering\mylabel{h}\end{ledgroupsized}  \newcommand{\dateiname}{L00469}\newcommand{\titel}{Arthur Schnitzler an Richard Beer-Hofmann, [5. 8. 1895?]}\newcommand{\editorInnen}{Martin Anton Müller und Gerd-Hermann Susen}%% latex-leseansicht-abspann.tex
%% Abspann für die Leseansicht.
%% Der Schalter \ifkorrekturansicht ist bereits durch den Vorspann gesetzt.

%% latex-abspann.tex
%% Gemeinsamer Abspann für Korrekturansicht und Leseansicht.
%% Setzt den Schalter \ifkorrekturansicht voraus (gesetzt in den
%% einbindenden Dateien latex-korrekturansicht-abspann.tex bzw.
%% latex-leseansicht-abspann.tex).
%% ---------------------------------------------------------------

\normalsize

% Das esempio-Environment wird nur in der Leseansicht benötigt
\ifkorrekturansicht\else
\newenvironment{esempio}[3]%
{
    \vspace{1.5ex}
    \rlap{\underline{#1}}
    \par
    \setlength{\parindent}{0cm}
    \nopagebreak
    \leftskip=#2cm
    \rightskip=#3cm
}
{
    \par
}
\fi

\doendnotes{C}
\bigskip
\vfill

\clearpage

\footnotesize

\ifkorrekturansicht
  \lohead{\textsc{register}}
\fi

% theindex-Environment neu definieren ohne reledmac
\makeatletter
\renewenvironment{theindex}{%
  \ifkorrekturansicht
    \section*{\indexname}%
  \else
    \subsubsection*{Index der erwähnten Entitäten}%
  \fi
  \setlength{\parindent}{0pt}%
  \setlength{\parskip}{0pt plus 0.3pt}%
  \let\item\@idxitem
}{%
  \ifkorrekturansicht\clearpage\fi
}
\makeatother

\IfFileExists{\jobname-pw.ind}{\input{\jobname-pw.ind}}{}

% Quellenangabe nur in der Leseansicht
\ifkorrekturansicht\else
% Fallback-Definitionen, falls die .tex-Datei \titel etc. nicht gesetzt hat
\providecommand{\titel}{}
\providecommand{\editorInnen}{}
\providecommand{\dateiname}{\jobname}

\vspace{3cm}

\vfill

\footnotesize
\textsc{Quelle}: \titel. Herausgegeben von {\editorInnen}. In: \emph{Arthur Schnitzler: Briefwechsel mit Autorinnen und Autoren}.
 Digitale Edition, https://schnitzler-briefe.acdh.oeaw.ac.at/{\dateiname}.html (Stand \today)
\fi

\end{document}


      