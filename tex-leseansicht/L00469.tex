%% latex-leseansicht-vorspann.tex
%% Vorspann für die Leseansicht.
%% Lädt die gemeinsame Datei latex-vorspann.tex mit nicht gesetztem Schalter.

\newif\ifkorrekturansicht
\korrekturansichtfalse

\input{../tex-inputs/latex-vorspann}


\section[Arthur Schnitzler an Richard Beer-Hofmann, {[}5. 8. 1895?{]}]{L00469 Arthur Schnitzler an Richard Beer-Hofmann, {[}5. 8. 1895?{]}}
\nopagebreak\mylabel{L00469v}
\rehead{ }\normalsize\beginnumbering\briefempfaengerindex{Beer-Hofmann, Richard@\textsc{Beer-Hofmann, Richard}!zzzSchnitzler, Arthur@\emph{von Arthur Schnitzler}!1895-08-051@{{[}5. 8. 1895?{]}}|(be}
\toendnotes[C]{\smallbreak\pagebreak[2]}
\correspDesc{Versand  durch Arthur Schnitzler am [5. 8. 1895?] in Bad Ischl
\newline{}Erhalt  durch Richard Beer-Hofmann im Zeitraum [5. 8. 1895
                  – 9. 8. 1895?] in Bad Ischl}\toendnotes[C]{\smallbreak}
\Standort{YCGL, MSS 31.}
\physDesc{Briefkarte, 260 Zeichen
\newline{}Handschrift: Bleistift, deutsche Kurrent}\toendnotes[C]{\smallbreak}
\pstart
           \noindent{}{\pb}Lieber Richard!{ }\label{K_L00469-1v}\edtext{\textsc{Salten}\pwindex{Salten, Felix 6.\,9.\,1869 Budapest – 8.\,10.\,1945 Zürich@\textsc{Salten, Felix} (6.\,9.\,1869 Budapest – 8.\,10.\,1945 Zürich), \emph{Schriftsteller, Journalist, Chefredakteur}|pw}}{\lemma{\textnormal{\emph{Salten}}}\Cendnote{\textnormal{Das verwendete Papier (und die
                  Einordnung zwischen die anderen Korrespondenzstücke im Archiv) deuten auf
                     1895. Aus dem Inhalt geht hervor, dass die Kommunikation außerhalb
                  von Wien\oindex{Wien@\textbf{Wien}, \emph{Verwaltungsgebiet}|pwk}{ }stattfindet (»angeko{\geminationm}en«). Das reduziert die durch das \emph{Tagebuch}\pwindex{Schnitzler, Arthur 15.\,5.\,1862 Wien – 21.\,10.\,1931 ebd.@\textsc{Schnitzler, Arthur} (15.\,5.\,1862 Wien – 21.\,10.\,1931 ebd.), \emph{Schriftsteller, Mediziner}!Tagebuch@\strich\emph{Tagebuch}|pwk} möglichen Daten auf 5. 8. 1895 und 16. 8. 1895. Beim
                  zweiten Termin kündigt Salten\pwindex{Salten, Felix 6.\,9.\,1869 Budapest – 8.\,10.\,1945 Zürich@\textsc{Salten, Felix} (6.\,9.\,1869 Budapest – 8.\,10.\,1945 Zürich), \emph{Schriftsteller, Journalist, Chefredakteur}|pwk} aber an, einen
                  späteren Zug zu nehmen. Auch dürfte sich Beer-Hofmann\pwindex{Beer-Hofmann, Richard 11.\,7.\,1866 Wien – 26.\,9.\,1945 New York City@\textsc{Beer-Hofmann, Richard} (11.\,7.\,1866 Wien – 26.\,9.\,1945 New York City), \emph{Schriftsteller}|pwk} zu diesem Zeitpunkt nicht in Ischl\oindex{Bad Ischl@\textbf{Bad Ischl}|pwk} aufgehalten haben, was den 5. 8. 1895 wahrscheinlich
                  macht. Im Theater hat Schnitzler an diesem
                  Tag \emph{Zwei glückliche Tage}\pwindex{\textcolor{red}{\textsuperscript{XXXX indx1}}!Zwei glückliche Tage@\strich\emph{Zwei glückliche Tage}|pwk}\pwindex{\textcolor{red}{\textsuperscript{XXXX indx1}}!Zwei glückliche Tage@\strich\emph{Zwei glückliche Tage}|pwk} gesehen.}}}\label{K_L00469-1} iſt erſt
               kurz vor 1 hier angeko{\geminationm}en. – Haben Sie{ }ſchon einen Sitz für mich geno{\geminationm}en,{ }ſo geh ich natürlich
               ins Theater – nicht – nicht. – Für alle Fälle laſſen Sie mir was{ }ſagen. {\pb}Iſts Ihnen recht, ko{\geminationm} ich
               mit \textsc{S.}\pwindex{Salten, Felix 6.\,9.\,1869 Budapest – 8.\,10.\,1945 Zürich@\textsc{Salten, Felix} (6.\,9.\,1869 Budapest – 8.\,10.\,1945 Zürich), \emph{Schriftsteller, Journalist, Chefredakteur}|pw} zwiſchen 5 u 6 zu Ihnen.\pend
           
\pstart
           Herzlich{\\[\baselineskip]}Ihr \spacefill\mbox{Arth}\pend
           \leftskip=0em{}\selectlanguage{ngerman}\endnumbering\briefempfaengerindex{Beer-Hofmann, Richard@\textsc{Beer-Hofmann, Richard}!zzzSchnitzler, Arthur@\emph{von Arthur Schnitzler}!1895-08-051@{{[}5. 8. 1895?{]}}|)be}\mylabel{L00469h}  \newcommand{\dateiname}{L00469}\newcommand{\titel}{Arthur Schnitzler an Richard Beer-Hofmann, [5. 8. 1895?]}\newcommand{\editorInnen}{Martin Anton Müller und Gerd-Hermann Susen}%% latex-leseansicht-abspann.tex
%% Abspann für die Leseansicht.
%% Der Schalter \ifkorrekturansicht ist bereits durch den Vorspann gesetzt.

%% latex-abspann.tex
%% Gemeinsamer Abspann für Korrekturansicht und Leseansicht.
%% Setzt den Schalter \ifkorrekturansicht voraus (gesetzt in den
%% einbindenden Dateien latex-korrekturansicht-abspann.tex bzw.
%% latex-leseansicht-abspann.tex).
%% ---------------------------------------------------------------

\normalsize

% Das esempio-Environment wird nur in der Leseansicht benötigt
\ifkorrekturansicht\else
\newenvironment{esempio}[3]%
{
    \vspace{1.5ex}
    \rlap{\underline{#1}}
    \par
    \setlength{\parindent}{0cm}
    \nopagebreak
    \leftskip=#2cm
    \rightskip=#3cm
}
{
    \par
}
\fi

\doendnotes{C}
\bigskip
\vfill

\clearpage

\footnotesize

\ifkorrekturansicht
  \lohead{\textsc{register}}
\fi

% theindex-Environment neu definieren ohne reledmac
\makeatletter
\renewenvironment{theindex}{%
  \ifkorrekturansicht
    \section*{\indexname}%
  \else
    \subsubsection*{Index der erwähnten Entitäten}%
  \fi
  \setlength{\parindent}{0pt}%
  \setlength{\parskip}{0pt plus 0.3pt}%
  \let\item\@idxitem
}{%
  \ifkorrekturansicht\clearpage\fi
}
\makeatother

\IfFileExists{\jobname-pw.ind}{\input{\jobname-pw.ind}}{}

% Quellenangabe nur in der Leseansicht
\ifkorrekturansicht\else
% Fallback-Definitionen, falls die .tex-Datei \titel etc. nicht gesetzt hat
\providecommand{\titel}{}
\providecommand{\editorInnen}{}
\providecommand{\dateiname}{\jobname}

\vspace{3cm}

\vfill

\footnotesize
\textsc{Quelle}: \titel. Herausgegeben von {\editorInnen}. In: \emph{Arthur Schnitzler: Briefwechsel mit Autorinnen und Autoren}.
 Digitale Edition, https://schnitzler-briefe.acdh.oeaw.ac.at/{\dateiname}.html (Stand \today)
\fi

\end{document}


