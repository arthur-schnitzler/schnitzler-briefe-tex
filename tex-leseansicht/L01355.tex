%% latex-korrekturansicht-vorspann.tex
%% Vorspann für die Korrekturansicht.
%% Lädt die gemeinsame Datei latex-vorspann.tex mit gesetztem Schalter.

\newif\ifkorrekturansicht
\korrekturansichttrue

\input{../tex-inputs/latex-vorspann}


\section[Max Burckhard an Arthur Schnitzler, 2. 1. 1904]{L01355 Max Burckhard an Arthur Schnitzler, 2. 1. 1904}
\nopagebreak\mylabel{L01355v}
\rehead{ }\normalsize\beginnumbering\briefempfaengerindex{Schnitzler, Arthur@\textsc{Schnitzler, Arthur}!zzzBurckhard, Max Eugen@\emph{von Max Eugen Burckhard}!1904-01-021@{2. 1. 1904}|(be}
\toendnotes[C]{\smallbreak\pagebreak[2]}\Standort{CUL, Schnitzler, B 20.}
\physDesc{Brief, 1 Blatt, 2 Seiten, 486 Zeichen
\newline{}Handschrift: schwarze Tinte, deutsche Kurrent
\newline{}Schnitzler: mit Bleistift beschriftet: »\textsc{Burckhard}« 
\newline{}Ordnung: mit Bleistift von unbekannter Hand nummeriert:
                                    »13« }\toendnotes[C]{\smallbreak}
\pstart
           \raggedleft{}{\pb}Wien\oindex{Wien@\textbf{Wien}, \emph{A.ADM2}|pw}{ }2. 1. 04\pend
           
\pstart{}Lieber verehrter Herr Schnitzler!\pend\vspace{0.5em}
\pstart
           Ich danke Ihnen herzlich für Ihre lieben Zeilen, mit denen Sie mir eine wirkliche
               Freude bereitet haben, da ſie mir in gleicher Weiſe als \label{K_L01355-1v}\edtext{Zeugen Ihres Urteiles}{\lemma{\textnormal{\emph{Zeugen Ihres Urteiles}}}\Cendnote{\textnormal{ungeklärt}}}\label{K_L01355-1} und Ihrer freundlichen Geſinnung wertvoll ſind.
               Seit wir nicht mehr im gleichen Hauſe wohnen, ſollten wir uns eigentlich nach den
               Naturgeſetzen öfter ſehen – aber {\pb}auch die
               Naturgeſetze ſind trügeriſch, beſonders wenn man ſie ſchon verkehrt annimmt.\pend
           
\pstart
           Herzlichſt Ihr\hspace*{1.5em}ergebener{\\[\baselineskip]}\spacefill\mbox{D\textsuperscript{r}Burckhard}\pend
           \leftskip=0em{}\selectlanguage{ngerman}\endnumbering\briefempfaengerindex{Schnitzler, Arthur@\textsc{Schnitzler, Arthur}!zzzBurckhard, Max Eugen@\emph{von Max Eugen Burckhard}!1904-01-021@{2. 1. 1904}|)be}\mylabel{L01355h}  \normalsize

\doendnotes{C}
\bigskip
\vfill

\clearpage

\footnotesize

\lohead{\textsc{register}}

% Definiere theindex-Environment komplett neu ohne reledmac
\makeatletter
\renewenvironment{theindex}{%
  \section*{\indexname}%
  \setlength{\parindent}{0pt}%
  \setlength{\parskip}{0pt plus 0.3pt}%
  \let\item\@idxitem
}{%
  \clearpage
}
\makeatother

\IfFileExists{\jobname-pw.ind}{\input{\jobname-pw.ind}}{}

\end{document}

      