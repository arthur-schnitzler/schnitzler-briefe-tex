%% latex-korrekturansicht-vorspann.tex
%% Vorspann für die Korrekturansicht.
%% Lädt die gemeinsame Datei latex-vorspann.tex mit gesetztem Schalter.

\newif\ifkorrekturansicht
\korrekturansichttrue

\input{../tex-inputs/latex-vorspann}


\section[ Paul Goldmann an Arthur Schnitzler, 3. 12. {[}1900{]}]{L02943 Paul Goldmann an Arthur Schnitzler, 3. 12. {[}1900{]}}
\nopagebreak\mylabel{L02943v}
\rehead{ }\normalsize\beginnumbering\briefempfaengerindex{Schnitzler, Arthur@\textsc{Schnitzler, Arthur}!zzzGoldmann, Paul@\emph{von Paul Goldmann}!1900-12-031@{3. 12. {[}1900{]}}|(be}
\toendnotes[C]{\smallbreak\pagebreak[2]}\Standort{DLA, A:Schnitzler, HS.NZ85.1.3170.}
\physDesc{Brief, 1 Blatt, 1 Seite, 219 Zeichen
\newline{}Handschrift: blaue Tinte, deutsche Kurrent
\newline{}Beilage: ein Zeitungsartikel, beschnitten und aufgeklebt 
\newline{}Schnitzler: 1) mit Bleistift das Jahr »900« vermerkt  2) mit rotem Buntstift eine seitliche Markierung neben der
                                 Begrüßungsformel}\toendnotes[C]{\smallbreak}
\pstart
           \raggedleft{}{\pb}Berlin\oindex{Berlin@\textbf{Berlin}, \emph{P.PPLC}|pw}, 3. December.\pend
           
\pstart\center{}Mein lieber Freund,\pend\vspace{0.5em}
\pstart
           Das \label{K_L02943-1v}\edtext{Telegramm\pwindex{Man telegraphirt uns aus Breslau…]@\emph{[Man telegraphirt uns aus Breslau…]}|pwv}}{\lemma{\textnormal{\emph{Telegramm}}}\Cendnote{\textnormal{[Erich Freund\pwindex{Freund, Erich 1866-08-13 – 1940@\textsc{Freund, Erich} (1866-08-13 – 1940), \emph{Kritiker/Kritikerin, Musikjournalist/Musikjournalistin}|pwk}]: \emph{Theater- und Kunstnachrichten [Telegramm]}\pwindex{Man telegraphirt uns aus Breslau…]@\emph{[Man telegraphirt uns aus Breslau…]}|pwk}.
                     In: \emph{Neue Freie Presse}\pwindex{Neue Freie Presse@\emph{Neue Freie Presse}|pwk}, Nr. 13.031, 2. 12. 1900, Morgenblatt, S. 10. Siehe auch
                     Paul Goldmann an Arthur Schnitzler, 9. 12. [1900].}}}\label{K_L02943-1} des \textsc{Dr. Freund\pwindex{Freund, Erich 1866-08-13 – 1940@\textsc{Freund, Erich} (1866-08-13 – 1940), \emph{Kritiker/Kritikerin, Musikjournalist/Musikjournalistin}|pw}} in der N. Fr. Pr.\pwindex{Neue Freie Presse@\emph{Neue Freie Presse}|pw} iſt blödſinnig. Offenbar
               ſind auch Streichungen erfolgt. Beifolgender \label{K_L02943-2v}\edtext{Ausſchnitt\pwindex{Theater und Musik [Schleier der Beatrice]@\emph{Theater und Musik [Schleier der Beatrice]}|pwv}}{\lemma{\textnormal{\emph{Ausſchnitt}}}\Cendnote{\textnormal{[O. V.]: \emph{Theater und Musik}\pwindex{Theater und Musik [Schleier der Beatrice]@\emph{Theater und Musik [Schleier der Beatrice]}|pwk}. In: \emph{Vossische Zeitung}\pwindex{Vossische Zeitung@\emph{Vossische Zeitung}|pwk}, Nr. 565, 3. 12. 1900, Abend-Ausgabe, S. [7].}}}\label{K_L02943-2}
               iſt aus der Voſſiſchen Zeitung\pwindex{Vossische Zeitung@\emph{Vossische Zeitung}|pw}. Viele Grüße!\pend
           
\pstart
           Dein {\\[\baselineskip]}\spacefill\mbox{Paul Goldmann.}\pend
           \leftskip=0em{}{\vspace{1\baselineskip}}
\pstart
           \textcolor{gray}{\textbf{– Das neue Drama\pwindex{Schleier der Beatrice. Schauspiel in fuenf Akten@\emph{Der Schleier der Beatrice. Schauspiel in fünf Akten}|pwv} von \textbf{Arthur Schnitzler, »Der Schleier der Beatrice\pwindex{Schleier der Beatrice. Schauspiel in fuenf Akten@\emph{Der Schleier der Beatrice. Schauspiel in fünf Akten}|pw}«,} deſſen Einreichung
                  beim Hofburgtheater\orgindex{Burgtheater@Burgtheater|pw} im letzten Frühjahr zu
                  einem Konflikt des Dichters mit Direktor \textsc{Dr.}{ }Schlenther\pwindex{Schlenther, Paul 20.08.1854 – 30.04.1916@\textsc{Schlenther, Paul} (20.08.1854 – 30.04.1916), \emph{Schriftsteller/Schriftstellerin, Kritiker/Kritikerin, Theaterleiter/Theaterleiterin}|pw} Anlaß gegeben hat, wurde am
                     Sonnabend im \so{Lobe-Theater}\oindex{Lobe-Theater@\textbf{Lobe-Theater}, \emph{Theater (K.THE)}|pw} zu \so{Breslau}\oindex{Breslau@\textbf{Breslau}, \emph{P.PPLA}|pw} zum erſten Male aufgeführt. Der äußere Erfolg des Stück\pwindex{Schleier der Beatrice. Schauspiel in fuenf Akten@\emph{Der Schleier der Beatrice. Schauspiel in fünf Akten}|pwv}es wurde durch die wenig gute
                  Aufführung ſtark beeinträchtigt. Das Stück\pwindex{Schleier der Beatrice. Schauspiel in fuenf Akten@\emph{Der Schleier der Beatrice. Schauspiel in fünf Akten}|pwv} ſelbſt erzielte bei ausverkauftem Hauſe\oindex{Lobe-Theater@\textbf{Lobe-Theater}, \emph{Theater (K.THE)}|pwv} eine große Wirkung. Man meldet uns
                  darüber aus Breslau\oindex{Breslau@\textbf{Breslau}, \emph{P.PPLA}|pw}: »Schnitzlers Stück\pwindex{Schleier der Beatrice. Schauspiel in fuenf Akten@\emph{Der Schleier der Beatrice. Schauspiel in fünf Akten}|pwv} iſt ein
                  farbenglühendes Gemälde aus der Hochrenaiſſancezeit und faßt die Tragik zweier
                  hochgeſtimmter Charaktere in der unbewußten Tragödie einer Mädchenſeele zuſammen.
                  Das Stück\pwindex{Schleier der Beatrice. Schauspiel in fuenf Akten@\emph{Der Schleier der Beatrice. Schauspiel in fünf Akten}|pwv} ſteigert ſich in
                  der dramatiſchen Wirkung von Akt zu Akt, und das ſichtlich lebhaft intereſſirte
                  Publikum bereitete dem anweſenden Dichter einen ſich ſtetig ſteigernden großen
                  Erfolg.«}}\pend
           \selectlanguage{ngerman}\endnumbering\briefempfaengerindex{Schnitzler, Arthur@\textsc{Schnitzler, Arthur}!zzzGoldmann, Paul@\emph{von Paul Goldmann}!1900-12-031@{3. 12. {[}1900{]}}|)be}\mylabel{L02943h}  \normalsize

\doendnotes{C}
\bigskip
\vfill

\clearpage

\footnotesize

\lohead{\textsc{register}}

% Definiere theindex-Environment komplett neu ohne reledmac
\makeatletter
\renewenvironment{theindex}{%
  \section*{\indexname}%
  \setlength{\parindent}{0pt}%
  \setlength{\parskip}{0pt plus 0.3pt}%
  \let\item\@idxitem
}{%
  \clearpage
}
\makeatother

\IfFileExists{\jobname-pw.ind}{\input{\jobname-pw.ind}}{}

\end{document}

      