%% latex-leseansicht-vorspann.tex
%% Vorspann für die Leseansicht.
%% Lädt die gemeinsame Datei latex-vorspann.tex mit nicht gesetztem Schalter.

\newif\ifkorrekturansicht
\korrekturansichtfalse

\input{../tex-inputs/latex-vorspann}


         
         \renewcommand{\erwaehntePersonen}{Personen: Erich Freund, Paul Goldmann, Paul Schlenther}
         \renewcommand{\erwaehnteInstitutionen}{Institutionen: Burgtheater}
         \renewcommand{\erwaehnteOrte}{Orte: Berlin, Breslau, Lobe-Theater, Wien}
         \renewcommand{\erwaehnteWerke}{Werke: Der Schleier der Beatrice. Schauspiel in fünf Akten, Neue Freie Presse, Theater und Musik [Schleier der Beatrice], Vossische Zeitung, [Man telegraphirt uns aus Breslau…]}
               \section[ Paul Goldmann an Arthur Schnitzler, 3. 12. {[}1900{]}]{ Paul Goldmann an Arthur Schnitzler, 3. 12. {[}1900{]}}\nopagebreak\mylabel{v}\rehead{ }\begin{ledgroupsized}[t]{13cm}\normalsize\beginnumbering \toendnotes[C]{\smallbreak\pagebreak[2]} \Standort{DLA, A:Schnitzler, HS.NZ85.1.3170.}
\physDesc{Brief, 1 Blatt, 1 Seite, 219 Zeichen
\newline{}Handschrift: blaue Tinte, deutsche Kurrent
\newline{}Beilage: ein Zeitungsartikel, beschnitten und aufgeklebt 
\newline{}Schnitzler: 1) mit Bleistift das Jahr »900« vermerkt  2) mit rotem Buntstift eine seitliche Markierung neben der
                                 Begrüßungsformel}\toendnotes[C]{\smallbreak}\pstart
           \raggedleft{}{\pb}Berlin\oindex{Berlin@\textbf{Berlin}|pw}, 3. December.\pend
           \pstart\center{}Mein lieber Freund,\pend\pstart
           Das \label{K_L02943-1v}\edtext{Telegramm\pwindex{Man telegraphirt uns aus Breslau…]1900-12-02@\emph{[Man telegraphirt uns aus Breslau…]} {[}1900-12-02{]}|pwv}}{\lemma{\textnormal{\emph{Telegramm}}}\Cendnote{\textnormal{[Erich Freund\pwindex{Freund, Erich 1866-08-13 – 1940@\textsc{Freund, Erich} (1866-08-13 – 1940), \emph{Kritiker, Journalist}|pwk}]: \emph{Theater- und Kunstnachrichten [Telegramm]}\pwindex{Man telegraphirt uns aus Breslau…]1900-12-02@\emph{[Man telegraphirt uns aus Breslau…]} {[}1900-12-02{]}|pwk}.
                     In: \emph{Neue Freie Presse}\pwindex{Neue Freie Presse1864 – 1939@\emph{Neue Freie Presse} {[}1864 – 1939{]}|pwk}, Nr. 13031, 2. 12. 1900, Morgenblatt, S. 10. Siehe auch
                     Paul Goldmann an Arthur Schnitzler, 9. 12. [1900].}}}\label{K_L02943-1h} des \textsc{Dr. Freund\pwindex{Freund, Erich 1866-08-13 – 1940@\textsc{Freund, Erich} (1866-08-13 – 1940), \emph{Kritiker, Journalist}|pw}} in der N. Fr. Pr.\pwindex{Neue Freie Presse1864 – 1939@\emph{Neue Freie Presse} {[}1864 – 1939{]}|pw} iſt blödſinnig. Offenbar
               ſind auch Streichungen erfolgt. Beifolgender \label{K_L02943-2v}\edtext{Ausſchnitt\pwindex{?? Werk@Nicht ermittelte Verfasserinnen und Verfasser!Theater und Musik [Schleier der Beatrice]1900-12-03@\emph{Theater und Musik [Schleier der Beatrice]} {[}1900-12-03{]}|pwv}}{\lemma{\textnormal{\emph{Ausſchnitt}}}\Cendnote{\textnormal{[O. V.]: \emph{Theater und Musik}\pwindex{?? Werk@Nicht ermittelte Verfasserinnen und Verfasser!Theater und Musik [Schleier der Beatrice]1900-12-03@\emph{Theater und Musik [Schleier der Beatrice]} {[}1900-12-03{]}|pwk}. In: \emph{Vossische Zeitung}\pwindex{?? Werk@Nicht ermittelte Verfasserinnen und Verfasser!Vossische Zeitung1617 – 1934@\emph{Vossische Zeitung} {[}1617 – 1934{]}|pwk}, Nr. 565, 3. 12. 1900, Abend-Ausgabe, S. [7].}}}\label{K_L02943-2h}
               iſt aus der Voſſiſchen Zeitung\pwindex{?? Werk@Nicht ermittelte Verfasserinnen und Verfasser!Vossische Zeitung1617 – 1934@\emph{Vossische Zeitung} {[}1617 – 1934{]}|pw}. Viele Grüße!\pend
           \pstart
           Dein {\\[\baselineskip]}\spacefill\mbox{Paul Goldmann.}\pend
           \leftskip=0em{}{\bigskip}\pstart
           \noindent{}\textcolor{gray}{\textbf{– Das neue Drama\pwindex{Schnitzler, Arthur 15.05.1862 – 21.10.1931@\textsc{Schnitzler, Arthur} (15.05.1862 – 21.10.1931), \emph{Schriftsteller, Mediziner}!Schleier der Beatrice. Schauspiel in fuenf Akten1900-12-01@\strich\emph{Der Schleier der Beatrice. Schauspiel in fünf Akten} {[}1900-12-01{]}|pwv} von \textbf{Arthur Schnitzler, »Der Schleier der Beatrice\pwindex{Schnitzler, Arthur 15.05.1862 – 21.10.1931@\textsc{Schnitzler, Arthur} (15.05.1862 – 21.10.1931), \emph{Schriftsteller, Mediziner}!Schleier der Beatrice. Schauspiel in fuenf Akten1900-12-01@\strich\emph{Der Schleier der Beatrice. Schauspiel in fünf Akten} {[}1900-12-01{]}|pw}«,} deſſen Einreichung
                  beim Hofburgtheater\orgindex{Burgtheater@Burgtheater|pw} im letzten Frühjahr zu
                  einem Konflikt des Dichters mit Direktor \textsc{Dr.}{ }Schlenther\pwindex{Schlenther, Paul 20.08.1854 – 30.04.1916@\textsc{Schlenther, Paul} (20.08.1854 – 30.04.1916), \emph{Schriftsteller, Kritiker, Theaterleiter}|pw} Anlaß gegeben hat, wurde am
                     Sonnabend im \so{Lobe-Theater}\oindex{Lobe-Theater@\textbf{Lobe-Theater}|pw} zu \so{Breslau}\oindex{Breslau@\textbf{Breslau}|pw} zum erſten Male aufgeführt. Der äußere Erfolg des Stück\pwindex{Schnitzler, Arthur 15.05.1862 – 21.10.1931@\textsc{Schnitzler, Arthur} (15.05.1862 – 21.10.1931), \emph{Schriftsteller, Mediziner}!Schleier der Beatrice. Schauspiel in fuenf Akten1900-12-01@\strich\emph{Der Schleier der Beatrice. Schauspiel in fünf Akten} {[}1900-12-01{]}|pwv}es wurde durch die wenig gute
                  Aufführung ſtark beeinträchtigt. Das Stück\pwindex{Schnitzler, Arthur 15.05.1862 – 21.10.1931@\textsc{Schnitzler, Arthur} (15.05.1862 – 21.10.1931), \emph{Schriftsteller, Mediziner}!Schleier der Beatrice. Schauspiel in fuenf Akten1900-12-01@\strich\emph{Der Schleier der Beatrice. Schauspiel in fünf Akten} {[}1900-12-01{]}|pwv} ſelbſt erzielte bei ausverkauftem Hauſe\oindex{Lobe-Theater@\textbf{Lobe-Theater}|pwv} eine große Wirkung. Man meldet uns
                  darüber aus Breslau\oindex{Breslau@\textbf{Breslau}|pw}: »Schnitzlers Stück\pwindex{Schnitzler, Arthur 15.05.1862 – 21.10.1931@\textsc{Schnitzler, Arthur} (15.05.1862 – 21.10.1931), \emph{Schriftsteller, Mediziner}!Schleier der Beatrice. Schauspiel in fuenf Akten1900-12-01@\strich\emph{Der Schleier der Beatrice. Schauspiel in fünf Akten} {[}1900-12-01{]}|pwv} iſt ein
                  farbenglühendes Gemälde aus der Hochrenaiſſancezeit und faßt die Tragik zweier
                  hochgeſtimmter Charaktere in der unbewußten Tragödie einer Mädchenſeele zuſammen.
                  Das Stück\pwindex{Schnitzler, Arthur 15.05.1862 – 21.10.1931@\textsc{Schnitzler, Arthur} (15.05.1862 – 21.10.1931), \emph{Schriftsteller, Mediziner}!Schleier der Beatrice. Schauspiel in fuenf Akten1900-12-01@\strich\emph{Der Schleier der Beatrice. Schauspiel in fünf Akten} {[}1900-12-01{]}|pwv} ſteigert ſich in
                  der dramatiſchen Wirkung von Akt zu Akt, und das ſichtlich lebhaft intereſſirte
                  Publikum bereitete dem anweſenden Dichter einen ſich ſtetig ſteigernden großen
                  Erfolg.«}}\pend
           
         
         \endnumbering\mylabel{h}\end{ledgroupsized}  \newcommand{\dateiname}{L02943}\newcommand{\titel}{Paul Goldmann an Arthur Schnitzler, 3. 12. [1900]}\newcommand{\editorInnen}{Martin Anton Müller und Laura Untner}%% latex-leseansicht-abspann.tex
%% Abspann für die Leseansicht.
%% Der Schalter \ifkorrekturansicht ist bereits durch den Vorspann gesetzt.

%% latex-abspann.tex
%% Gemeinsamer Abspann für Korrekturansicht und Leseansicht.
%% Setzt den Schalter \ifkorrekturansicht voraus (gesetzt in den
%% einbindenden Dateien latex-korrekturansicht-abspann.tex bzw.
%% latex-leseansicht-abspann.tex).
%% ---------------------------------------------------------------

\normalsize

% Das esempio-Environment wird nur in der Leseansicht benötigt
\ifkorrekturansicht\else
\newenvironment{esempio}[3]%
{
    \vspace{1.5ex}
    \rlap{\underline{#1}}
    \par
    \setlength{\parindent}{0cm}
    \nopagebreak
    \leftskip=#2cm
    \rightskip=#3cm
}
{
    \par
}
\fi

\doendnotes{C}
\bigskip
\vfill

\clearpage

\footnotesize

\ifkorrekturansicht
  \lohead{\textsc{register}}
\fi

% theindex-Environment neu definieren ohne reledmac
\makeatletter
\renewenvironment{theindex}{%
  \ifkorrekturansicht
    \section*{\indexname}%
  \else
    \subsubsection*{Index der erwähnten Entitäten}%
  \fi
  \setlength{\parindent}{0pt}%
  \setlength{\parskip}{0pt plus 0.3pt}%
  \let\item\@idxitem
}{%
  \ifkorrekturansicht\clearpage\fi
}
\makeatother

\IfFileExists{\jobname-pw.ind}{\input{\jobname-pw.ind}}{}

% Quellenangabe nur in der Leseansicht
\ifkorrekturansicht\else
% Fallback-Definitionen, falls die .tex-Datei \titel etc. nicht gesetzt hat
\providecommand{\titel}{}
\providecommand{\editorInnen}{}
\providecommand{\dateiname}{\jobname}

\vspace{3cm}

\vfill

\footnotesize
\textsc{Quelle}: \titel. Herausgegeben von {\editorInnen}. In: \emph{Arthur Schnitzler: Briefwechsel mit Autorinnen und Autoren}.
 Digitale Edition, https://schnitzler-briefe.acdh.oeaw.ac.at/{\dateiname}.html (Stand \today)
\fi

\end{document}


      