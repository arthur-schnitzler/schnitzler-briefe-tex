%% latex-korrekturansicht-vorspann.tex
%% Vorspann für die Korrekturansicht.
%% Lädt die gemeinsame Datei latex-vorspann.tex mit gesetztem Schalter.

\newif\ifkorrekturansicht
\korrekturansichttrue

\input{../tex-inputs/latex-vorspann}


\section[ Felix Salten an Arthur Schnitzler, 22. 7. 1912]{L03558 Felix Salten an Arthur Schnitzler, 22. 7. 1912}
\nopagebreak\mylabel{L03558v}
\rehead{ }\normalsize\beginnumbering\briefempfaengerindex{Schnitzler, Arthur@\textsc{Schnitzler, Arthur}!zzzSalten, Felix@\emph{von Felix Salten}!1912-07-221@{22. 7. 1912}|(be}
\toendnotes[C]{\smallbreak\pagebreak[2]}\Standort{CUL, Schnitzler, B 89, B 2.}
\physDesc{Brief, 1 Blatt, 1 Seite, 2052 Zeichen
\newline{}Handschrift: schwarze Tinte, lateinische Kurrent
\newline{}Schnitzler: mit rotem Buntstift eine Unterstreichung 
\newline{}Ordnung: mit Bleistift von unbekannter Hand nummeriert: »273« }\toendnotes[C]{\smallbreak}
\pstart
           {\pb}\textcolor{gray}{\textbf{\textsc{Felix Salten}}}\hfill Berghof\oindex{Berghof@\textbf{Berghof}, \emph{Wohngebäude (K.WHS)}|pw}, 22. VII. 12\pend
           
\pstart{}Lieber,\pend\vspace{0.5em}
\pstart
           Sie sind nun wol schon \label{K_L03558-1v}\edtext{in Brioni\oindex{Brijuni@\textbf{Brijuni}, \emph{P.PPL}|pw}}{\lemma{\textnormal{\emph{in Brioni}}}\Cendnote{\textnormal{Vgl. Felix Salten an Arthur Schnitzler, 2. 7. 1912.
               }}}\label{K_L03558-1} und haben dort gewiß all die schöne Sonne, die uns seit drei Tagen hier\oindex{Unterach am Attersee@\textbf{Unterach am Attersee}, \emph{P.PPL}|pwv} fehlt. Hier gibt’s Sturm,
               Gewitter und Regen. Man muß im Zimmer sitzen, aber das fördert meine Arbeit nicht.
               Wenn wir schönes Wetter haben und am Vormittag Ausflüge machen, bringe ich weit mehr
               zustande. Der graue Himmmel macht mich kaput.\pend
           
\pstart
           Was Sie mir über Ihren »Bernhardi\pwindex{Professor Bernhardi. Komoedie in fuenf Akten@\emph{Professor Bernhardi. Komödie in fünf Akten}|pw}« schreiben,
               hab’ ich garnicht anders erwartet. Ich verstehe es so gut, dass Sie garnicht anders
               verfahren können. Das Stück\pwindex{Professor Bernhardi. Komoedie in fuenf Akten@\emph{Professor Bernhardi. Komödie in fünf Akten}|pwv}
               ist nun da, es ist ein lebendiges Wesen, hat seine Notwendigkeit und seine Mission,
               und es wäre gerade für Sie unmöglich, ihm diese Existenz nun wieder zu nehmen. Ich
               kann es mir sehr lebhaft denken, dass Sie es als die schlimmere Eventualität
               empfinden, \label{K_L03558-2v}\edtext{das Stück\pwindex{Professor Bernhardi. Komoedie in fuenf Akten@\emph{Professor Bernhardi. Komödie in fünf Akten}|pwv} vorsichtig zurückzuhalten}{\lemma{\textnormal{\emph{das … zurückzuhalten}}}\Cendnote{\textnormal{Die Einreichung von
                     \emph{Professor Bernhardi}\pwindex{Professor Bernhardi. Komoedie in fuenf Akten@\emph{Professor Bernhardi. Komödie in fünf Akten}|pwk} bei der Zensurbehörde\orgindex{K. u. k. Zensurstelle@K. u. k. Zensurstelle|pwkv} stand bevor. Das Stück wurde nicht zugelassen.}}}\label{K_L03558-2}, statt es seinen
               Weg gehen und sein Schicksal haben zu laßen. Deswegen werden Sie es gewiß verstehen,
               dass ich fürs erste doch den Versuch machte, Sie zur Vorsicht zu bewegen. Von uns
               beiden müßte ich (oder sonst ein anderer Ihrer Freunde) die Bedenken haben, und Sie
               den Mut. Umgekehrt wär’s weniger angenehm, und ich muß sagen, in der jungen
               Geschichte dieses Stück\pwindex{Professor Bernhardi. Komoedie in fuenf Akten@\emph{Professor Bernhardi. Komödie in fünf Akten}|pwv}es
               möchte ich weder für jetzt, noch für alles, was eben noch kommt, unsere Discussion
               über den Gefährlichkeitspunkt nicht missen. Ich hoffe übrigens,
                  das{[}s{]} ich in meiner Besorgnis zu schwarz gesehen habe, und
               dass auch hier alles anders kommen wird, als man sich’s erwartet.\pend
           
\pstart
           Wir leben hier ziemlich still. Fischers\pwindex{Fischer, Samuel 24.12.1859 – 15.10.1934@\textsc{Fischer, Samuel} (24.12.1859 – 15.10.1934), \emph{Verleger/Verlegerin}|pw}\pwindex{Fischer, Hedwig 08.09.1871 – 11.04.1952@\textsc{Fischer, Hedwig} (08.09.1871 – 11.04.1952)|pw} sind seit einer Woche da. Goldmark\pwindex{Goldmark, Karl 30.05.1830 – 02.01.1915@\textsc{Goldmark, Karl} (30.05.1830 – 02.01.1915), \emph{Komponist/Komponistin, Dirigent/Dirigentin, Bratschist/Bratschistin}|pw} seit sieben Wochen. Er ist mir mit seinen dreiundachtzig Jahren
               bewunderungswürdig. Er lernt französisch, spielt Klavier, komponirt, flirtet, und hat
               in allem einen so verklärten Egoismus, dass man wirklich so was wie Größe empfindet.
               Ich entledige mich mir einiger Muß-Arbeiten, und denke, im Herbst zu wichtigeren
               Plänen zu gelangen. Alle sind wol, und warten auf gutes Wetter. Laßen Sie uns wißen,
               wie es Ihnen allen geht, wie sie auf Brioni\oindex{Brijuni@\textbf{Brijuni}, \emph{P.PPL}|pw}
               leben, und seien Sie mit Frau Olga\pwindex{Schnitzler, Olga 17.01.1882 – 13.01.1970@\textsc{Schnitzler, Olga} (17.01.1882 – 13.01.1970), \emph{Schauspieler/Schauspielerin, Sänger/Sängerin}|pw} und den Kinder\pwindex{Schnitzler, Heinrich 09.08.1902 – 12.07.1982@\textsc{Schnitzler, Heinrich} (09.08.1902 – 12.07.1982), \emph{Regisseur/Regisseurin, Schauspieler/Schauspielerin}|pwv}\pwindex{Cappellini, Lili 13.09.1909 – 26.07.1928@\textsc{Cappellini, Lili} (13.09.1909 – 26.07.1928)|pwv}n von uns
               allen herzlichst gegrüßt –\pend
           \pstart Ihr \spacefill\mbox{Salten}\pend{}\selectlanguage{ngerman}\endnumbering\briefempfaengerindex{Schnitzler, Arthur@\textsc{Schnitzler, Arthur}!zzzSalten, Felix@\emph{von Felix Salten}!1912-07-221@{22. 7. 1912}|)be}\mylabel{L03558h}  \normalsize

\doendnotes{C}
\bigskip
\vfill

\clearpage

\footnotesize

\lohead{\textsc{register}}

% Definiere theindex-Environment komplett neu ohne reledmac
\makeatletter
\renewenvironment{theindex}{%
  \section*{\indexname}%
  \setlength{\parindent}{0pt}%
  \setlength{\parskip}{0pt plus 0.3pt}%
  \let\item\@idxitem
}{%
  \clearpage
}
\makeatother

\IfFileExists{\jobname-pw.ind}{\input{\jobname-pw.ind}}{}

\end{document}

      