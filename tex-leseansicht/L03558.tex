%% latex-leseansicht-vorspann.tex
%% Vorspann für die Leseansicht.
%% Lädt die gemeinsame Datei latex-vorspann.tex mit nicht gesetztem Schalter.

\newif\ifkorrekturansicht
\korrekturansichtfalse

\input{../tex-inputs/latex-vorspann}

\begin{center}
            \textcolor{red}{ENTWURF, NICHT FERTIG KORRIGIERT}
                      \end{center}
            
         
         \renewcommand{\erwaehntePersonen}{Personen: Lili Cappellini, Samuel Fischer, Hedwig Fischer, Karl Goldmark, Felix Salten, Olga Schnitzler, Heinrich Schnitzler}
         \renewcommand{\erwaehnteInstitutionen}{Institutionen: K. u. k. Zensurstelle}
         \renewcommand{\erwaehnteOrte}{Orte: Berghof, Brijuni, Unterach am Attersee}
         \renewcommand{\erwaehnteWerke}{Werke: Professor Bernhardi. Komödie in fünf Akten}
               \section[ Felix Salten an Arthur Schnitzler, 22. 7. 1912]{ Felix Salten an Arthur Schnitzler, 22. 7. 1912}\nopagebreak\mylabel{v}\rehead{ }\begin{ledgroupsized}[t]{13cm}\normalsize\beginnumbering \toendnotes[C]{\smallbreak\pagebreak[2]} \Standort{CUL, Schnitzler, B 89, B 2.}
\physDesc{Brief, 1 Blatt, 1 Seite, 2044 Zeichen
\newline{}Handschrift: schwarze Tinte, lateinische Kurrent
\newline{}Schnitzler: mit rotem Buntstift eine Unterstreichung 
\newline{}Ordnung: mit Bleistift von unbekannter Hand nummeriert: »273« }\toendnotes[C]{\smallbreak}\pstart
           \noindent{}{\pb}\textcolor{gray}{\textbf{\textsc{Felix Salten}}}\hfill Berghof\oindex{Berghof@\textbf{Berghof}|pw}, 22. VII. 12\pend
           \pstart{}Lieber,\pend\pstart
           Sie sind nun wol schon \label{K_L03558-1v}\edtext{in Brioni\oindex{Brijuni@\textbf{Brijuni}|pw}}{\lemma{\textnormal{\emph{in Brioni}}}\Cendnote{\textnormal{siehe Felix Salten an Arthur Schnitzler, 2. 7. 1912}}}\label{K_L03558-1h} und haben dort gewiß all die schöne Sonne, die uns seit drei Tagen hier\oindex{Unterach am Attersee@\textbf{Unterach am Attersee}|pwv} fehlt. Hier gibt’s Sturm,
               Gewitter und Regen. Man muß im Zimmer sitzen, aber das fördert meine Arbeit nicht.
               Wenn wir schönes Wetter haben und am Vormittag Ausflüge machen, bringe ich weit mehr
               zustande. Der graue Himmmel macht mich kaput.\pend
           \pstart
           Was Sie mir über Ihren »Bernhardi\pwindex{Schnitzler, Arthur 15.05.1862 – 21.10.1931@\textsc{Schnitzler, Arthur} (15.05.1862 – 21.10.1931), \emph{Schriftsteller, Mediziner}!Professor Bernhardi. Komoedie in fuenf Akten1912@\strich\emph{Professor Bernhardi. Komödie in fünf Akten} {[}1912{]}|pw}« schreiben,
               hab’ ich garnicht anders erwartet. Ich verstehe es so gut, dass Sie garnicht anders
               verfahren können. Das Stück\pwindex{Schnitzler, Arthur 15.05.1862 – 21.10.1931@\textsc{Schnitzler, Arthur} (15.05.1862 – 21.10.1931), \emph{Schriftsteller, Mediziner}!Professor Bernhardi. Komoedie in fuenf Akten1912@\strich\emph{Professor Bernhardi. Komödie in fünf Akten} {[}1912{]}|pwv}
               ist nun da, es ist ein lebendiges Wesen, hat seine Notwendigkeit und seine Mission,
               und es wäre gerade für Sie unmöglich, ihm diese Existenz nun wieder zu nehmen. Ich
               kann es mir sehr lebhaft denken, dass Sie es als die schlimmere Eventualität
               empfinden, \label{K_L03558-2v}\edtext{das Stück\pwindex{Schnitzler, Arthur 15.05.1862 – 21.10.1931@\textsc{Schnitzler, Arthur} (15.05.1862 – 21.10.1931), \emph{Schriftsteller, Mediziner}!Professor Bernhardi. Komoedie in fuenf Akten1912@\strich\emph{Professor Bernhardi. Komödie in fünf Akten} {[}1912{]}|pwv} vorsichtig zurückzuhalten}{\lemma{\textnormal{\emph{das … zurückzuhalten}}}\Cendnote{\textnormal{Bezug auf die bevorstehende Einreichung von
                     \emph{Professor Bernhardi}\pwindex{Schnitzler, Arthur 15.05.1862 – 21.10.1931@\textsc{Schnitzler, Arthur} (15.05.1862 – 21.10.1931), \emph{Schriftsteller, Mediziner}!Professor Bernhardi. Komoedie in fuenf Akten1912@\strich\emph{Professor Bernhardi. Komödie in fünf Akten} {[}1912{]}|pwk} bei der Zensurbehörde\orgindex{K. u. k. Zensurstelle@K. u. k. Zensurstelle|pwkv}?}}}\label{K_L03558-2h}, statt es seinen
               Weg gehen und sein Schicksal haben zu laßen. Deswegen werden Sie es gewiß verstehen,
               dass ich fürs erste doch den Versuch machte, Sie zur Vorsicht zu bewegen. Von uns
               beiden müßte ich (oder sonst ein anderer Ihrer Freunde) die Bedenken haben, und Sie
               den Mut. Umgekehrt wär’s weniger angenehm, und ich muß sagen, in der jungen
               Geschichte dieses Stück\pwindex{Schnitzler, Arthur 15.05.1862 – 21.10.1931@\textsc{Schnitzler, Arthur} (15.05.1862 – 21.10.1931), \emph{Schriftsteller, Mediziner}!Professor Bernhardi. Komoedie in fuenf Akten1912@\strich\emph{Professor Bernhardi. Komödie in fünf Akten} {[}1912{]}|pwv}es
               möchte ich weder für jetzt, noch für alles, was eben noch kommt, unsere Discussion
               über den Gefährlichkeitspunkt nicht missen. Ich hoffe übigens,
                  das{[}s{]} ich in meiner Besorgnis zu schwarz gesehen habe, und
               dass auch \textcolor{gray}{nun} alles anders kommen wird, als man sich’s
               erwartet.\pend
           \pstart
           Wir leben hier ziemlich still. Fischers\pwindex{Fischer, Samuel 24.12.1859 – 15.10.1934@\textsc{Fischer, Samuel} (24.12.1859 – 15.10.1934), \emph{Verleger}|pw}\pwindex{Fischer, Hedwig 08.09.1871 – 11.04.1952@\textsc{Fischer, Hedwig} (08.09.1871 – 11.04.1952)|pw} sind seit einer Woche da. Goldmark\pwindex{Goldmark, Karl 30.05.1830 – 02.01.1915@\textsc{Goldmark, Karl} (30.05.1830 – 02.01.1915), \emph{Komponist, Dirigent, Bratschist}|pw} seit sieben Wochen. Er ist mit seinen dreiundachtzig Jahren
               bewunderungswürdig. Er lernt französisch, spielt Klavier, komponirt, flirtet, und hat
               in allem einen so verklärten Egoismus, dass man wirklich so was wie Größe empfindet.
               Ich entledige mich mir einiger Muß-Arbeiten, und denke, im Herbst zu wichtigeren
               Plänen zu gelangen. Alle sind wol, und warten auf gutes Wetter. Laßen Sie uns wißen,
               wie es Ihnen allen geht, wie sie auf Brioni\oindex{Brijuni@\textbf{Brijuni}|pw}
               leben, und seien Sie mit Frau Olga\pwindex{Schnitzler, Olga 17.01.1882 – 13.01.1970@\textsc{Schnitzler, Olga} (17.01.1882 – 13.01.1970), \emph{Schauspielerin, Sängerin}|pw} und den Kinder\pwindex{Schnitzler, Heinrich 09.08.1902 – 12.07.1982@\textsc{Schnitzler, Heinrich} (09.08.1902 – 12.07.1982), \emph{Regisseur, Schauspieler}|pwv}\pwindex{Cappellini, Lili 13.09.1909 – 26.07.1928@\textsc{Cappellini, Lili} (13.09.1909 – 26.07.1928)|pwv}n von uns
               allen herzlichst gegrüßt –\pend
           \pstart Ihr \spacefill\mbox{Salten}\pend{}
         
         \endnumbering\mylabel{h}\end{ledgroupsized}  \newcommand{\dateiname}{L03558}\newcommand{\titel}{Felix Salten an Arthur Schnitzler, 22. 7. 1912}\newcommand{\editorInnen}{Martin Anton Müller und Laura Untner}%% latex-leseansicht-abspann.tex
%% Abspann für die Leseansicht.
%% Der Schalter \ifkorrekturansicht ist bereits durch den Vorspann gesetzt.

%% latex-abspann.tex
%% Gemeinsamer Abspann für Korrekturansicht und Leseansicht.
%% Setzt den Schalter \ifkorrekturansicht voraus (gesetzt in den
%% einbindenden Dateien latex-korrekturansicht-abspann.tex bzw.
%% latex-leseansicht-abspann.tex).
%% ---------------------------------------------------------------

\normalsize

% Das esempio-Environment wird nur in der Leseansicht benötigt
\ifkorrekturansicht\else
\newenvironment{esempio}[3]%
{
    \vspace{1.5ex}
    \rlap{\underline{#1}}
    \par
    \setlength{\parindent}{0cm}
    \nopagebreak
    \leftskip=#2cm
    \rightskip=#3cm
}
{
    \par
}
\fi

\doendnotes{C}
\bigskip
\vfill

\clearpage

\footnotesize

\ifkorrekturansicht
  \lohead{\textsc{register}}
\fi

% theindex-Environment neu definieren ohne reledmac
\makeatletter
\renewenvironment{theindex}{%
  \ifkorrekturansicht
    \section*{\indexname}%
  \else
    \subsubsection*{Index der erwähnten Entitäten}%
  \fi
  \setlength{\parindent}{0pt}%
  \setlength{\parskip}{0pt plus 0.3pt}%
  \let\item\@idxitem
}{%
  \ifkorrekturansicht\clearpage\fi
}
\makeatother

\IfFileExists{\jobname-pw.ind}{\input{\jobname-pw.ind}}{}

% Quellenangabe nur in der Leseansicht
\ifkorrekturansicht\else
% Fallback-Definitionen, falls die .tex-Datei \titel etc. nicht gesetzt hat
\providecommand{\titel}{}
\providecommand{\editorInnen}{}
\providecommand{\dateiname}{\jobname}

\vspace{3cm}

\vfill

\footnotesize
\textsc{Quelle}: \titel. Herausgegeben von {\editorInnen}. In: \emph{Arthur Schnitzler: Briefwechsel mit Autorinnen und Autoren}.
 Digitale Edition, https://schnitzler-briefe.acdh.oeaw.ac.at/{\dateiname}.html (Stand \today)
\fi

\end{document}


      