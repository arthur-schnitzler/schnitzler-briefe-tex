\input{../tex-inputs/latex-pdf-vorspann}
\begin{center}
            \textcolor{red}{ENTWURF. ENTZIFFERUNG NOCH NICHT KORREKTURGELESEN}
                      \end{center}
            
               \section[Hermann Bahr an Arthur Schnitzler, {[}19. 9. 1893?{]}]{ Hermann Bahr an Arthur Schnitzler, {[}19. 9. 1893?{]}}\nopagebreak\mylabel{v}\rehead{ }\begin{ledgroupsized}[t]{13cm}\normalsize\beginnumbering\briefempfaengerindex{Schnitzler, Arthur@\textsc{Schnitzler, Arthur}!zzzBahr, Hermann@\emph{von Hermann Bahr}!1893-09-191@{{[}19. 9. 1893?{]}}|(be} \toendnotes[C]{\smallbreak\pagebreak[2]} \Standort{CUL, Schnitzler, B 5b.}
\physDesc{Brief, 1 Blatt, 1 Seite
\newline{}Handschrift: Bleistift, deutsche Kurrent\newline{}Ordnung: 1) oberer und der linker Seitenrand
                           beschnitten 2) mit rotem Buntstift von unbekannter Hand nummeriert: »10«3) mit Bleistift von unbekannter Hand nummeriert: »10«}\buchAbdrucke{\weitereDrucke{Hermann Bahr, Arthur Schnitzler: \emph{Briefwechsel, Aufzeichnungen, Dokumente (1891–1931)}. Hg. Kurt Ifkovits und Martin Anton Müller. Göttingen: \emph{Wallstein} 2018, S. 37.} }\pstart\center{}{\pb}Lieber Freund!\pend\pstart
           Ich möchte Sie gern ein bischen ſprechen. Könnten Sie morgen Dienſtag um 4 Uhr Daheim
               ſein?\pend
           \pstart
           Herzlichſt{\\[\baselineskip]}\spacefill\mbox{HermannBahr}\pend
           \leftskip=0em{}\endnumbering\briefempfaengerindex{Schnitzler, Arthur@\textsc{Schnitzler, Arthur}!zzzBahr, Hermann@\emph{von Hermann Bahr}!1893-09-191@{{[}19. 9. 1893?{]}}|)be}\mylabel{h}\end{ledgroupsized}  \newcommand{\dateiname}{L00265}\newcommand{\titel}{Hermann Bahr an Arthur Schnitzler, [19. 9. 1893?]}\newcommand{\editorInnen}{ Kurt Ifkovits,  Martin Anton Müller}\input{../tex-inputs/latex-pdf-abspann}
      