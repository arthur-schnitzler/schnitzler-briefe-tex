\input{../tex-inputs/latex-pdf-vorspann}
\begin{center}
            \textcolor{red}{ENTWURF. ENTZIFFERUNG NOCH NICHT KORREKTURGELESEN}
                      \end{center}
            
               \section[Arthur Schnitzler an Edith Brandes, 22. 7. 1901]{ Arthur Schnitzler an Edith Brandes, 22. 7. 1901}\nopagebreak\mylabel{v}\rehead{ }\begin{ledgroupsized}[t]{13cm}\normalsize\beginnumbering\briefempfaengerindex{Philipp, Edith@\textsc{Philipp, Edith}!zzzSchnitzler, Arthur@\emph{von Arthur Schnitzler}!1901-07-222@{22. 7. 1901}|(be} \toendnotes[C]{\smallbreak\pagebreak[2]} \buchAlsQuelle{Georg Brandes, Arthur Schnitzler: \emph{Ein Briefwechsel}. Hg. Kurt Bergel. Bern: \emph{Francke} 1956, S. 90.}\stanza{}{\pb}In offner Flut auf leichtem Kahn\newverse{}An frischem Sommertage!\newverse{}Die Mücken, die dir wehgethan\newverse{}Sind nur des Ufers Plage.\stanzaend{}\pstart 22. 7. 01{ }\spacefill\mbox{Arthur Schnitzler}\pend{}\endnumbering\briefempfaengerindex{Philipp, Edith@\textsc{Philipp, Edith}!zzzSchnitzler, Arthur@\emph{von Arthur Schnitzler}!1901-07-222@{22. 7. 1901}|)be}\mylabel{h}\end{ledgroupsized}  \newcommand{\dateiname}{L01151}\newcommand{\titel}{Arthur Schnitzler an Edith Brandes, 22. 7. 1901}\newcommand{\editorInnen}{Martin Anton Müller und Gerd-Hermann Susen}\input{../tex-inputs/latex-pdf-abspann}
      