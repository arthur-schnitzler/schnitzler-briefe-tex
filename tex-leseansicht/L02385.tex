%% latex-leseansicht-vorspann.tex
%% Vorspann für die Leseansicht.
%% Lädt die gemeinsame Datei latex-vorspann.tex mit nicht gesetztem Schalter.

\newif\ifkorrekturansicht
\korrekturansichtfalse

\input{../tex-inputs/latex-vorspann}


\section[Arthur Schnitzler an Hermann Bahr, 6. 6. 1922]{L02385 Arthur Schnitzler an Hermann Bahr, 6. 6. 1922}
\nopagebreak\mylabel{L02385v}
\rehead{ }\normalsize\beginnumbering\briefempfaengerindex{Bahr, Hermann@\textsc{Bahr, Hermann}!zzzSchnitzler, Arthur@\emph{von Arthur Schnitzler}!1922-06-061@{6. 6. 1922}|(be}
\toendnotes[C]{\smallbreak\pagebreak[2]}
\correspDesc{Versand  durch Arthur Schnitzler am 6. 6. 1922 in Wien
\newline{}Erhalt  durch Hermann Bahr am 7. VI. 22 in München}\toendnotes[C]{\smallbreak}
\Standort{TMW, HS AM 60137 Ba.}
\physDesc{Postkarte, 474 Zeichen
\newline{}Handschrift: schwarze Tinte, deutsche Kurrent
\newline{}Versand: 1) Stempel: »\nobreak{}\oindex{Wien@\textbf{Wien}, \emph{Verwaltungsgebiet}|pwk}Wien, 7. VI. 22, 8\nobreak{}«.   2) mit Bleistift von unbekannter Hand Ergänzung der Adresse:
                                    »NW 18«, die erste Ziffer überschrieben mit:
                                    »3«}
\buchAbdrucke{\weitereDrucke{1) \emph{6. 6. 1922, Abschrift.} In: Arthur Schnitzler: \emph{The Letters of Arthur Schnitzler to Hermann Bahr}. Edited, annotated, and with an introduction, by Donald G. Daviau. Chapel Hill: \emph{The University of North Carolina Press} 1978, S. 116 (University of North Carolina studies in the Germanic languages
                        and literatures, 89).} \weitereDrucke{2) Hermann Bahr, Arthur Schnitzler: \emph{Briefwechsel, Aufzeichnungen, Dokumente (1891–1931)}. Herausgegeben von Kurt Ifkovits und Martin Anton Müller. Göttingen: \emph{Wallstein} 2018, S. 561.} }\toendnotes[C]{\smallbreak}\pstart{}{\pb}\textsc{A. S.}\pend{}\pstart{}Wien XVIII\oindex{XVIII., Währing@\textbf{XVIII., Währing}, \emph{Verwaltungsgebiet}|pw}\pend{}\pstart{}\textsc{Sternwstr 71\oindex{Wien@\textbf{Wien}!XVIII., Währing@\textbf{XVIII., Währing}!Sternwartestraße 71@\textbf{Sternwartestraße 71}, \emph{Wohngebäude}|pw}}\pend{}{\bigskip}\pstart{}Herr Hermann Bahr\pend{}\pstart{}München\oindex{München@\textbf{München}|pw}\pend{}\pstart{}Barerſtraße.\oindex{Barerstraße@\textbf{Barerstraße}, \emph{Straße}|pw}\pend{}{\bigskip}\vspace{1em}
\pstart
           \raggedleft{}{\pb}Wien\oindex{Wien@\textbf{Wien}, \emph{Verwaltungsgebiet}|pw}, 6. 6. 22\pend
           \vspace{0.5em}
\pstart
           Mein lieber Hermann, laß dir vorläufig auf dieſem Weg für die
               ausführlichen, freundſchaftlichen warmherzigen Grüße\pwindex{Bahr, Hermann 19.\,7.\,1863 Linz – 15.\,1.\,1934 München@\textsc{Bahr, Hermann} (19.\,7.\,1863 Linz – 15.\,1.\,1934 München), \emph{Schriftsteller, Kritiker}!Brief an Arthur Schnitzler@\strich\emph{Brief an Arthur Schnitzler}|pwv}\pwindex{Arthur Schnitzler. Zu seinem sechzigsten Geburtstag (15. Mai 1922)@\emph{Arthur Schnitzler. Zu seinem sechzigsten Geburtstag (15. Mai 1922)}|pwv}{ }\substVorne{}\textsuperscript{ſ}\substDazwischen{}d\substHinten{}anken, die du mir durch die Zeitungen zu meinem Geburtstag geſandt hast. In
               dieſem So{\geminationm}er hoffe ich zuverſichtlich dir
               endlich wieder die Hand drücken zu kö{\geminationn}en. Ich
               \textcolor{gray}{nehme} an, du bleibſt vorläufig in München\oindex{München@\textbf{München}|pw}, {\pb}ich komme wohl durch
               und darf dich aufſuchen!\pend
           
\pstart
           Mit tauſen{[}d{]} Grüßen,{\\[\baselineskip]}Dein getreuer{\\[\baselineskip]}\spacefill\mbox{Arthur}\pend
           \leftskip=0em{}\selectlanguage{ngerman}\endnumbering\briefempfaengerindex{Bahr, Hermann@\textsc{Bahr, Hermann}!zzzSchnitzler, Arthur@\emph{von Arthur Schnitzler}!1922-06-061@{6. 6. 1922}|)be}\mylabel{L02385h}  \newcommand{\dateiname}{L02385}\newcommand{\titel}{Arthur Schnitzler an Hermann Bahr, 6. 6. 1922}\newcommand{\editorInnen}{Herausgegeben von Martin Anton Müller}%% latex-leseansicht-abspann.tex
%% Abspann für die Leseansicht.
%% Der Schalter \ifkorrekturansicht ist bereits durch den Vorspann gesetzt.

%% latex-abspann.tex
%% Gemeinsamer Abspann für Korrekturansicht und Leseansicht.
%% Setzt den Schalter \ifkorrekturansicht voraus (gesetzt in den
%% einbindenden Dateien latex-korrekturansicht-abspann.tex bzw.
%% latex-leseansicht-abspann.tex).
%% ---------------------------------------------------------------

\normalsize

% Das esempio-Environment wird nur in der Leseansicht benötigt
\ifkorrekturansicht\else
\newenvironment{esempio}[3]%
{
    \vspace{1.5ex}
    \rlap{\underline{#1}}
    \par
    \setlength{\parindent}{0cm}
    \nopagebreak
    \leftskip=#2cm
    \rightskip=#3cm
}
{
    \par
}
\fi

\doendnotes{C}
\bigskip
\vfill

\clearpage

\footnotesize

\ifkorrekturansicht
  \lohead{\textsc{register}}
\fi

% theindex-Environment neu definieren ohne reledmac
\makeatletter
\renewenvironment{theindex}{%
  \ifkorrekturansicht
    \section*{\indexname}%
  \else
    \subsubsection*{Index der erwähnten Entitäten}%
  \fi
  \setlength{\parindent}{0pt}%
  \setlength{\parskip}{0pt plus 0.3pt}%
  \let\item\@idxitem
}{%
  \ifkorrekturansicht\clearpage\fi
}
\makeatother

\IfFileExists{\jobname-pw.ind}{\input{\jobname-pw.ind}}{}

% Quellenangabe nur in der Leseansicht
\ifkorrekturansicht\else
% Fallback-Definitionen, falls die .tex-Datei \titel etc. nicht gesetzt hat
\providecommand{\titel}{}
\providecommand{\editorInnen}{}
\providecommand{\dateiname}{\jobname}

\vspace{3cm}

\vfill

\footnotesize
\textsc{Quelle}: \titel. Herausgegeben von {\editorInnen}. In: \emph{Arthur Schnitzler: Briefwechsel mit Autorinnen und Autoren}.
 Digitale Edition, https://schnitzler-briefe.acdh.oeaw.ac.at/{\dateiname}.html (Stand \today)
\fi

\end{document}


