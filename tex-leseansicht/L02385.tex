%% latex-korrekturansicht-vorspann.tex
%% Vorspann für die Korrekturansicht.
%% Lädt die gemeinsame Datei latex-vorspann.tex mit gesetztem Schalter.

\newif\ifkorrekturansicht
\korrekturansichttrue

\input{../tex-inputs/latex-vorspann}


\section[Arthur Schnitzler an Hermann Bahr, 6. 6. 1922]{L02385 Arthur Schnitzler an Hermann Bahr, 6. 6. 1922}
\nopagebreak\mylabel{L02385v}
\rehead{ }\normalsize\beginnumbering\briefempfaengerindex{Bahr, Hermann@\textsc{Bahr, Hermann}!zzzSchnitzler, Arthur@\emph{von Arthur Schnitzler}!1922-06-061@{6. 6. 1922}|(be}
\toendnotes[C]{\smallbreak\pagebreak[2]}\Standort{TMW, HS AM 60137 Ba.}
\physDesc{Postkarte, 474 Zeichen
\newline{}Handschrift: schwarze Tinte, deutsche Kurrent
\newline{}Versand: 1) Stempel: »\nobreak{}Wien, 7. VI. 22, 8\nobreak{}«.   2) mit Bleistift von unbekannter Hand Ergänzung der Adresse:
                                    »NW 18«, die erste Ziffer überschrieben mit:
                                    »3«}
\buchAbdrucke{\weitereDrucke{1) Arthur Schnitzler: \emph{The Letters of Arthur Schnitzler to Hermann Bahr}. Chapel Hill: \emph{The University of North Carolina Press} 1978, S. 116.} \weitereDrucke{2) Hermann Bahr, Arthur Schnitzler: \emph{Briefwechsel, Aufzeichnungen, Dokumente (1891–1931)}. Göttingen: \emph{Wallstein} 2018, S. 561.} }\toendnotes[C]{\smallbreak}\pstart{}{\pb}\textsc{A. S.}\pend{}\pstart{}Wien XVIII\oindex{XVIII., Waehring@\textbf{XVIII., Währing}, \emph{A.ADM3}|pw}\pend{}\pstart{}\textsc{Sternwstr 71\oindex{Sternwartestrasse 71@\textbf{Sternwartestraße 71}, \emph{Wohngebäude (K.WHS)}|pw}}\pend{}{\bigskip}\pstart{}Herr Hermann Bahr\pend{}\pstart{}München\oindex{Muenchen@\textbf{München}, \emph{P.PPLA}|pw}\pend{}\pstart{}Barerſtraße.\oindex{Barerstrasse@\textbf{Barerstraße}, \emph{Straße (K.STR)}|pw}\pend{}{\bigskip}\vspace{1em}
\pstart
           \raggedleft{}{\pb}Wien\oindex{Wien@\textbf{Wien}, \emph{A.ADM2}|pw}, 6. 6. 22\pend
           \vspace{0.5em}
\pstart
           Mein lieber Hermann, laß dir vorläufig auf dieſem Weg für die
               ausführlichen, freundſchaftlichen warmherzigen Grüße\pwindex{Brief an Arthur Schnitzler@\emph{Brief an Arthur Schnitzler}|pwv}\pwindex{Arthur Schnitzler. Zu seinem sechzigsten Geburtstag (15. Mai 1922)@\emph{Arthur Schnitzler. Zu seinem sechzigsten Geburtstag (15. Mai 1922)}|pwv}{ }\substVorne{}\textsuperscript{ſ}\substDazwischen{}d\substHinten{}anken, die du mir durch die Zeitungen zu meinem Geburtstag geſandt hast. In
               dieſem So{\geminationm}er hoffe ich zuverſichtlich dir
               endlich wieder die Hand drücken zu kö{\geminationn}en. Ich
                  \textcolor{gray}{nehme} an, du bleibſt vorläufig in München\oindex{Muenchen@\textbf{München}, \emph{P.PPLA}|pw}, {\pb}ich komme wohl durch
               und darf dich aufſuchen!\pend
           
\pstart
           Mit tauſen{[}d{]} Grüßen,{\\[\baselineskip]}Dein getreuer{\\[\baselineskip]}\spacefill\mbox{Arthur}\pend
           \leftskip=0em{}\selectlanguage{ngerman}\endnumbering\briefempfaengerindex{Bahr, Hermann@\textsc{Bahr, Hermann}!zzzSchnitzler, Arthur@\emph{von Arthur Schnitzler}!1922-06-061@{6. 6. 1922}|)be}\mylabel{L02385h}  \normalsize

\doendnotes{C}
\bigskip
\vfill

\clearpage

\footnotesize

\lohead{\textsc{register}}

% Definiere theindex-Environment komplett neu ohne reledmac
\makeatletter
\renewenvironment{theindex}{%
  \section*{\indexname}%
  \setlength{\parindent}{0pt}%
  \setlength{\parskip}{0pt plus 0.3pt}%
  \let\item\@idxitem
}{%
  \clearpage
}
\makeatother

\IfFileExists{\jobname-pw.ind}{\input{\jobname-pw.ind}}{}

\end{document}

      