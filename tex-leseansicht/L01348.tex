%% latex-leseansicht-vorspann.tex
%% Vorspann für die Leseansicht.
%% Lädt die gemeinsame Datei latex-vorspann.tex mit nicht gesetztem Schalter.

\newif\ifkorrekturansicht
\korrekturansichtfalse

\input{../tex-inputs/latex-vorspann}


               \section[Arthur Schnitzler an Hugo von Hofmannsthal, 10. 12. 1903]{ Arthur Schnitzler an Hugo von Hofmannsthal, 10. 12. 1903}\nopagebreak\mylabel{v}\rehead{ }\begin{ledgroupsized}[t]{13cm}\normalsize\beginnumbering\briefempfaengerindex{Hofmannsthal, Hugo von@\textsc{Hofmannsthal, Hugo von}!zzzSchnitzler, Arthur@\emph{von Arthur Schnitzler}!1903-12-101@{10. 12. 1903}|(be} \toendnotes[C]{\smallbreak\pagebreak[2]} \Standort{FDH, Hs-30885,106.}
\physDesc{Brief, 2 Blätter, 6 Seiten
\newline{}Handschrift: schwarze Tinte, deutsche Kurrent\newline{}Ordnung: 1) von Schnitzler mutmaßlich bei der Durchsicht der Korrespondenz
                                    1929 mit Bleistift datiert: »910« 2) mit Bleistift von Olga
                                    Schnitzler\pwindex{Schnitzler, Olga 17.01.1882 – 13.01.1970@\textsc{Schnitzler, Olga} (17.01.1882 – 13.01.1970), \emph{Schauspielerin, Sängerin}|pw} neben der Adressangabe vermerkt: »\textsc{Irrtum: damals wohnten wir schon in der Sternwartestrasse\oindex{Sternwartestrasse@\textbf{Sternwartestraße}|pw}. O.}«, was
                                 sich auf die (falsche) nachträgliche Einordnung auf das Jahr 1910
                                 bezieht3) das zweite Blatt von unbekannter Hand mit Bleistift beschriftet:
                                    »II 10/12 910«4) mit Bleistift von unbekannter Hand nummeriert
                                    »106a«}\buchAbdrucke{\weitereDrucke{Hugo von Hofmannsthal, Arthur Schnitzler: \emph{Briefwechsel}. Hg. Therese Nickl und Heinrich Schnitzler. Frankfurt am Main: \emph{S. Fischer} 1964, S. 179–180.} }\toendnotes[C]{\smallbreak}\pstart
           \raggedleft{}{\pb}XVIII Spöttelg. 7\oindex{Edmund-Weiss-Gasse@\textbf{Edmund-Weiß-Gasse}|pw}. {\\}Wien\oindex{Wien@\textbf{Wien}|pw}{ }10. 12. 9\textcolor{gray}{03}\pend
           \pstart{}mein lieber Hugo, \pend\pstart
           Sie haben offenbar einen Brief von mir nicht beko{\geminationm}en,
               den ich an Sie vor etwa 14 Tagen, ich glaube an dem Tag wo Ihre Elektra\pwindex{Hofmannsthal, Hugo von 01.02.1874 – 15.07.1929@\textsc{Hofmannsthal, Hugo von} (01.02.1874 – 15.07.1929), \emph{Schriftsteller}!Elektra. Tragoedie in einem Aufzug1903@\strich\emph{Elektra. Tragödie in einem Aufzug} {[}1903{]}|pw} bei mir erſchien, an Sie geſchrieben habe. Das
               weſentlichſte, was dieſer Brief enthielt war die Bitte Ihre Elektra\pwindex{Hofmannsthal, Hugo von 01.02.1874 – 15.07.1929@\textsc{Hofmannsthal, Hugo von} (01.02.1874 – 15.07.1929), \emph{Schriftsteller}!Elektra. Tragoedie in einem Aufzug1903@\strich\emph{Elektra. Tragödie in einem Aufzug} {[}1903{]}|pw} an \textsc{Antoine}\pwindex{Antoine, Andre 31.01.1858 – 23.10.1943@\textsc{Antoine, André} (31.01.1858 – 23.10.1943), \emph{Theaterleiter, Schauspieler}|pw}, \textsc{resp}. an Dr \textsc{Stephan Epstein\pwindex{Epstein, Stephan 12.11.1866 – 1941@\textsc{Epstein, Stephan} (12.11.1866 – 1941), \emph{Schriftsteller, Dramaturg, Übersetzer}|pw}{ }Paris 78 rue de l’Assomption\oindex{rue de l Assomption@\textbf{rue de l’Assomption}|pw}, Antoine\pwindex{Antoine, Andre 31.01.1858 – 23.10.1943@\textsc{Antoine, André} (31.01.1858 – 23.10.1943), \emph{Theaterleiter, Schauspieler}|pw}s} Dramaturgen fürs Ausland zu ſenden, dem ich
               neulich \strikeout{darüber} über das Stück\pwindex{Hofmannsthal, Hugo von 01.02.1874 – 15.07.1929@\textsc{Hofmannsthal, Hugo von} (01.02.1874 – 15.07.1929), \emph{Schriftsteller}!Elektra. Tragoedie in einem Aufzug1903@\strich\emph{Elektra. Tragödie in einem Aufzug} {[}1903{]}|pwv} kurz berichtet habe.\pend
           \pstart
           {\pb}Daſs \introOben{}B.\introOben{} Garlan\pwindex{Schnitzler, Arthur 15.05.1862 – 21.10.1931@\textsc{Schnitzler, Arthur} (15.05.1862 – 21.10.1931), \emph{Schriftsteller, Mediziner}!Frau Bertha Garlan. Roman15.1.1901 – 15.3.1901@\strich\emph{Frau Bertha Garlan. Roman} {[}15.1.1901 – 15.3.1901{]}|pw} beim zweiten Leſen ſo angenehm auf Sie
               wirkte, freut mich ſehr – ich hab es ſeit dem Erſcheinen nicht wieder geleſen wie ich
               es (we{\geminationn} mich nicht äußerliche Gründe zu einer
               wiederholten Lectüre nöthigen) mit allen meinen gedruckten Sachen halte. Daher weiſs
               ich auch ſeit etwa 8 Jahren nichts mehr von »Sterben\pwindex{Schnitzler, Arthur 15.05.1862 – 21.10.1931@\textsc{Schnitzler, Arthur} (15.05.1862 – 21.10.1931), \emph{Schriftsteller, Mediziner}!Sterben. Novelle1.10.1894 – 1.12.1894@\strich\emph{Sterben. Novelle} {[}1.10.1894 – 1.12.1894{]}|pw}«. Es sta{\geminationm}t aus der Zeit, wo mich der
               »Fall« mehr intereſſirt hat als die Menſchen, und ich denke das meiſte aus dieſer
               Epoche muſs wie luftlos wirken. Dieſe Sachen – ich hab es neulich wieder am »\textsc{Jour de {\pb}gloire}\pwindex{Schnitzler, Arthur 15.05.1862 – 21.10.1931@\textsc{Schnitzler, Arthur} (15.05.1862 – 21.10.1931), \emph{Schriftsteller, Mediziner}!Ehrentag15. 01. 1898@\strich\emph{Der Ehrentag} {[}15. 01. 1898{]}|pw}« \substVorne{}\textsuperscript{g}\substDazwischen{}e\substHinten{}rfahren, wirken in anſtändiger franzöſiſcher Übertragung beſſer als in meinem
               Deutſch. Die reine Tendenz des Erzählens iſt dem romaniſchen Sprachgeiſt eingeboren,
               während es im deutſchen gleichſam wie gegen die Natur wirkt, wenn die Mittheilung von
               Thatſachen der Seele und Menſchlichkeit entbehrt. Die umgekehrte Probe kann man
               machen, we{\geminationn} man irgend eine kurze \textsc{Maupassant}\pwindex{Maupassant, Guy de 05.08.1850 – 07.07.1893@\textsc{Maupassant, Guy de} (05.08.1850 – 07.07.1893), \emph{Schriftsteller}|pw} Geſchichte die franzöſiſch noch lange nicht ſchwach wirkt, in deutſcher
               Ueberſetzung lieſt.\pend
           \pstart
           – Immerhin hab ich die Empfindg daſs {\pb}meine Technik der
               inneren Entwicklung meiner Production noch nicht nachgekommen iſt – was mir übrigens
               nicht bange macht. Es iſt jetzt in mir wieder ſo eine Neigung Sachen nur anzufangen
               und zu ſkizziren wie in der Zeit, die der Anatol\pwindex{Schnitzler, Arthur 15.05.1862 – 21.10.1931@\textsc{Schnitzler, Arthur} (15.05.1862 – 21.10.1931), \emph{Schriftsteller, Mediziner}!Anatol1892-10-29 – 1892-10-29@\strich\emph{Anatol} {[}1892-10-29 – 1892-10-29{]}|pw}-Epoche vorherging. Am meiſten beſchäftige ich mich jetzt mit einer Art
               von Komödie\pwindex{Schnitzler, Arthur 15.05.1862 – 21.10.1931@\textsc{Schnitzler, Arthur} (15.05.1862 – 21.10.1931), \emph{Schriftsteller, Mediziner}!Fink und Fliederbusch. Komoedie in drei Akten1917@\strich\emph{Fink und Fliederbusch. Komödie in drei Akten} {[}1917{]}|pwv} und bin innerlich
                  \strikeout{von dem Roman\pwindex{Schnitzler, Arthur 15.05.1862 – 21.10.1931@\textsc{Schnitzler, Arthur} (15.05.1862 – 21.10.1931), \emph{Schriftsteller, Mediziner}!Weg ins Freie. Roman1.1.1908 – 1.6.1908@\strich\emph{Der Weg ins Freie. Roman} {[}1.1.1908 – 1.6.1908{]}|pwv}} am meiſten von dem Roman\pwindex{Schnitzler, Arthur 15.05.1862 – 21.10.1931@\textsc{Schnitzler, Arthur} (15.05.1862 – 21.10.1931), \emph{Schriftsteller, Mediziner}!Weg ins Freie. Roman1.1.1908 – 1.6.1908@\strich\emph{Der Weg ins Freie. Roman} {[}1.1.1908 – 1.6.1908{]}|pwv}
               erfüllt, den ich im Frühjahr begonnen, den aber fortzuſetzen ich nicht in genügend
               reiner Sti{\geminationm}ung mich befinde.\pend
           \pstart
           In Concerte gehen wir nicht ſelten, ins Theater beinahe nie, aus perſönlichen {\pb}Gründen waren wir bei der \label{K_L01348_1v}\edtext{\textsc{Novella d’Andrea}\pwindex{\textcolor{red}{\textsuperscript{XXXX1 indx}}!Novella DAndrea1903@\strich\emph{Novella d’Andrea} {[}1903{]}|pw}}{\lemma{\textnormal{\emph{Novella d’Andrea}}}\Cendnote{\textnormal{siehe A. S.: \emph{Tagebuch}, 21. 11. 1903}}}\label{K_L01348_1h} – und ich hab es nicht ohne Bitterkeit empfunden, daſs ich den Kainz\pwindex{Kainz, Josef 02.01.1858 – 20.09.1910@\textsc{Kainz, Josef} (02.01.1858 – 20.09.1910), \emph{Schauspieler}|pw} nie werde den Sala\pwindex{Schnitzler, Arthur 15.05.1862 – 21.10.1931@\textsc{Schnitzler, Arthur} (15.05.1862 – 21.10.1931), \emph{Schriftsteller, Mediziner}!einsame Weg. Schauspiel in fuenf Akten1904@\strich\emph{Der einsame Weg. Schauspiel in fünf Akten} {[}1904{]}|pwv}{ }ſpielen \strikeout{k}{ }ſehen. Denn das Burgtheater\orgindex{Burgtheater@Burgtheater|pw}, wie Herr Schlenther\pwindex{Schlenther, Paul 20.08.1854 – 30.04.1916@\textsc{Schlenther, Paul} (20.08.1854 – 30.04.1916), \emph{Schriftsteller, Kritiker, Theaterleiter}|pw} an Fiſcher\pwindex{Fischer, Samuel 24.12.1859 – 15.10.1934@\textsc{Fischer, Samuel} (24.12.1859 – 15.10.1934), \emph{Verleger}|pw} geſchrieben, »reflectirt nicht« auf dieſes
                  Stück\pwindex{Schnitzler, Arthur 15.05.1862 – 21.10.1931@\textsc{Schnitzler, Arthur} (15.05.1862 – 21.10.1931), \emph{Schriftsteller, Mediziner}!einsame Weg. Schauspiel in fuenf Akten1904@\strich\emph{Der einsame Weg. Schauspiel in fünf Akten} {[}1904{]}|pwv}. Brahm\pwindex{Brahm, Otto 05.02.1856 – 28.11.1912@\textsc{Brahm, Otto} (05.02.1856 – 28.11.1912), \emph{Theaterleiter, Regisseur}|pw} gegenüber (was Sie ja wohl wiſſen dürften) hat sich Schl.\pwindex{Schlenther, Paul 20.08.1854 – 30.04.1916@\textsc{Schlenther, Paul} (20.08.1854 – 30.04.1916), \emph{Schriftsteller, Kritiker, Theaterleiter}|pw} über das Stück\pwindex{Schnitzler, Arthur 15.05.1862 – 21.10.1931@\textsc{Schnitzler, Arthur} (15.05.1862 – 21.10.1931), \emph{Schriftsteller, Mediziner}!einsame Weg. Schauspiel in fuenf Akten1904@\strich\emph{Der einsame Weg. Schauspiel in fünf Akten} {[}1904{]}|pwv}{ }ſehr misfällig geäußert; ſcheint es aber, wie Brahm\pwindex{Brahm, Otto 05.02.1856 – 28.11.1912@\textsc{Brahm, Otto} (05.02.1856 – 28.11.1912), \emph{Theaterleiter, Regisseur}|pw}{ }ſagt, ganz oberflächlich – und wie ich überzeugt
               bin – mit böſem Willen geleſen zu haben.\pend
           \pstart
           Und nun, wann ſieht man ſich wieder? Wie wär es, Montag oder
                  Mittwoch{ }Abend in dem Hietzinger
                  Restaurant\oindex{Ottakringer Braeu@\textbf{Ottakringer Bräu}|pwv}? Schrei{\pb}ben Sie mir, wann es Ihnen
               beſſer paſſt und ob auch Ihre Frau\pwindex{Hofmannsthal, Gertrude von 16.03.1880 – 09.11.1959@\textsc{Hofmannsthal, Gertrude von} (16.03.1880 – 09.11.1959)|pwv} mitkommt.\pend
           \pstart
           Und Richard\pwindex{Beer-Hofmann, Richard 11.07.1866 – 26.09.1945@\textsc{Beer-Hofmann, Richard} (11.07.1866 – 26.09.1945), \emph{Schriftsteller}|pw}? Ich höre u ſehe nichts von ihm. –
               Sobald das Wetter ein bischen angenehmer wird, kommen wir gern nach Rodaun\oindex{Rodaun@\textbf{Rodaun}|pw}.\pend
           \pstart
           \label{K_L01348_2v}\edtext{Das andere}{\lemma{\textnormal{\emph{Das andere}}}\Cendnote{\textnormal{vgl. Hugo von Hofmannsthal an Arthur Schnitzler, 8. 12. [1903]}}}\label{K_L01348_2h}, das ich bald bekomme, iſt wohl das gerettete
                     \textsc{Venedig}\pwindex{Hofmannsthal, Hugo von 01.02.1874 – 15.07.1929@\textsc{Hofmannsthal, Hugo von} (01.02.1874 – 15.07.1929), \emph{Schriftsteller}!gerettete Venedig. Trauerspiel in fuenf Aufzuegen1905@\strich\emph{Das gerettete Venedig. Trauerspiel in fünf Aufzügen} {[}1905{]}|pw}? –\pend
           \pstart
           Leben Sie wohl. Herzlichſt\hspace*{1.5em}Ihr{\\[\baselineskip]}\spacefill\mbox{A.}\pend
           \leftskip=0em{}\endnumbering\briefempfaengerindex{Hofmannsthal, Hugo von@\textsc{Hofmannsthal, Hugo von}!zzzSchnitzler, Arthur@\emph{von Arthur Schnitzler}!1903-12-101@{10. 12. 1903}|)be}\mylabel{h}\end{ledgroupsized}  \newcommand{\dateiname}{L01348}\newcommand{\titel}{Arthur Schnitzler an Hugo von Hofmannsthal, 10. 12. 1903}\newcommand{\editorInnen}{Martin Anton Müller und Gerd-Hermann Susen}%% latex-leseansicht-abspann.tex
%% Abspann für die Leseansicht.
%% Der Schalter \ifkorrekturansicht ist bereits durch den Vorspann gesetzt.

%% latex-abspann.tex
%% Gemeinsamer Abspann für Korrekturansicht und Leseansicht.
%% Setzt den Schalter \ifkorrekturansicht voraus (gesetzt in den
%% einbindenden Dateien latex-korrekturansicht-abspann.tex bzw.
%% latex-leseansicht-abspann.tex).
%% ---------------------------------------------------------------

\normalsize

% Das esempio-Environment wird nur in der Leseansicht benötigt
\ifkorrekturansicht\else
\newenvironment{esempio}[3]%
{
    \vspace{1.5ex}
    \rlap{\underline{#1}}
    \par
    \setlength{\parindent}{0cm}
    \nopagebreak
    \leftskip=#2cm
    \rightskip=#3cm
}
{
    \par
}
\fi

\doendnotes{C}
\bigskip
\vfill

\clearpage

\footnotesize

\ifkorrekturansicht
  \lohead{\textsc{register}}
\fi

% theindex-Environment neu definieren ohne reledmac
\makeatletter
\renewenvironment{theindex}{%
  \ifkorrekturansicht
    \section*{\indexname}%
  \else
    \subsubsection*{Index der erwähnten Entitäten}%
  \fi
  \setlength{\parindent}{0pt}%
  \setlength{\parskip}{0pt plus 0.3pt}%
  \let\item\@idxitem
}{%
  \ifkorrekturansicht\clearpage\fi
}
\makeatother

\IfFileExists{\jobname-pw.ind}{\input{\jobname-pw.ind}}{}

% Quellenangabe nur in der Leseansicht
\ifkorrekturansicht\else
% Fallback-Definitionen, falls die .tex-Datei \titel etc. nicht gesetzt hat
\providecommand{\titel}{}
\providecommand{\editorInnen}{}
\providecommand{\dateiname}{\jobname}

\vspace{3cm}

\vfill

\footnotesize
\textsc{Quelle}: \titel. Herausgegeben von {\editorInnen}. In: \emph{Arthur Schnitzler: Briefwechsel mit Autorinnen und Autoren}.
 Digitale Edition, https://schnitzler-briefe.acdh.oeaw.ac.at/{\dateiname}.html (Stand \today)
\fi

\end{document}


      