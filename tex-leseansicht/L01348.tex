%% latex-leseansicht-vorspann.tex
%% Vorspann für die Leseansicht.
%% Lädt die gemeinsame Datei latex-vorspann.tex mit nicht gesetztem Schalter.

\newif\ifkorrekturansicht
\korrekturansichtfalse

\input{../tex-inputs/latex-vorspann}


\section[Arthur Schnitzler an Hugo von Hofmannsthal, 10. 12. 1903]{L01348 Arthur Schnitzler an Hugo von Hofmannsthal, 10. 12. 1903}
\nopagebreak\mylabel{L01348v}
\rehead{ }\normalsize\beginnumbering\briefempfaengerindex{Hofmannsthal, Hugo von@\textsc{Hofmannsthal, Hugo von}!zzzSchnitzler, Arthur@\emph{von Arthur Schnitzler}!1903-12-101@{10. 12. 1903}|(be}
\toendnotes[C]{\smallbreak\pagebreak[2]}
\correspDesc{Versand  durch Arthur Schnitzler am 10. 12. 1903 in Wien
\newline{}Erhalt  durch Hugo von Hofmannsthal im Zeitraum [10. 12. 1903 – 14. 12. 1903?] in Wien}\toendnotes[C]{\smallbreak}
\Standort{FDH, Hs-30885,106.}
\physDesc{Brief, 2 Blätter, 6 Seiten, 2886 Zeichen
\newline{}Handschrift: schwarze Tinte, deutsche Kurrent
\newline{}Ordnung: 1) mit Bleistift von Schnitzler mutmaßlich bei der Durchsicht der Korrespondenz
                                    1929 datiert: »910«  2) mit Bleistift von Olga
                                 Schnitzler\pwindex{Schnitzler, Olga 17.\,1.\,1882 Wien – 13.\,1.\,1970 Lugano@\textsc{Schnitzler, Olga} (17.\,1.\,1882 Wien – 13.\,1.\,1970 Lugano), \emph{Schauspielerin, Sängerin}|pw} neben der Adressangabe vermerkt: »\textsc{Irrtum: damals wohnten wir schon in der Sternwartestrasse\oindex{Wien@\textbf{Wien}!XVIII., Währing@\textbf{XVIII., Währing}!Sternwartestraße 71@\textbf{Sternwartestraße 71}, \emph{Wohngebäude}|pw}. O.}«,
                                 was sich auf die (falsche) nachträgliche Einordnung auf das Jahr
                                 1910 bezieht 3) das zweite Blatt von unbekannter Hand mit Bleistift beschriftet:
                                    »II 10/12 910« 4) mit Bleistift von unbekannter Hand nummeriert
                                    »106a«}
\buchAbdrucke{\weitereDrucke{Hugo von Hofmannsthal, Arthur Schnitzler: \emph{Briefwechsel}. Herausgegeben von Therese Nickl und Heinrich Schnitzler. Frankfurt am Main: \emph{S. Fischer} 1964, S. 179–180.} }\toendnotes[C]{\smallbreak}
\pstart
           \raggedleft{}{\pb}XVIII Spöttelg. 7\oindex{Wien@\textbf{Wien}!XVIII., Währing@\textbf{XVIII., Währing}!Edmund-Weiß-Gasse 7@\textbf{Edmund-Weiß-Gasse 7}, \emph{Wohngebäude}|pw}. {\\}Wien\oindex{Wien@\textbf{Wien}, \emph{Verwaltungsgebiet}|pw}{ }10. 12. 9\textcolor{gray}{03}\pend
           
\pstart{}mein lieber Hugo,\pend\vspace{0.5em}
\pstart
           Sie haben offenbar einen Brief von mir nicht beko{\geminationm}en,
               den ich an Sie vor etwa 14 Tagen, ich glaube an dem Tag wo Ihre Elektra\pwindex{Hofmannsthal, Hugo von 1.\,2.\,1874 Wien – 15.\,7.\,1929 Rodaun@\textsc{Hofmannsthal, Hugo von} (1.\,2.\,1874 Wien – 15.\,7.\,1929 Rodaun), \emph{Schriftsteller}!Elektra. Tragödie in einem Aufzug@\strich\emph{Elektra. Tragödie in einem Aufzug}|pw} bei mir erſchien, an Sie geſchrieben habe. Das
               weſentlichſte, was dieſer Brief enthielt war die Bitte Ihre Elektra\pwindex{Hofmannsthal, Hugo von 1.\,2.\,1874 Wien – 15.\,7.\,1929 Rodaun@\textsc{Hofmannsthal, Hugo von} (1.\,2.\,1874 Wien – 15.\,7.\,1929 Rodaun), \emph{Schriftsteller}!Elektra. Tragödie in einem Aufzug@\strich\emph{Elektra. Tragödie in einem Aufzug}|pw} an \textsc{Antoine}\pwindex{Antoine, André 31.\,1.\,1858 Limoges – 23.\,10.\,1943 Le Pouliguen@\textsc{Antoine, André} (31.\,1.\,1858 Limoges – 23.\,10.\,1943 Le Pouliguen), \emph{Theaterleiter, Schauspieler}|pw}, \textsc{resp}. an Dr \textsc{Stephan Epstein\pwindex{Epstein, Stephan 12.\,11.\,1866 Warschau – 1941 Paris@\textsc{Epstein, Stephan} (12.\,11.\,1866 Warschau – 1941 Paris), \emph{Schriftsteller, Dramaturg, Übersetzer}|pw}{ }Paris 78 rue de l’Assomption\oindex{rue de l’Assomption@\textbf{rue de l’Assomption}, \emph{Straße}|pw}, Antoines\pwindex{Antoine, André 31.\,1.\,1858 Limoges – 23.\,10.\,1943 Le Pouliguen@\textsc{Antoine, André} (31.\,1.\,1858 Limoges – 23.\,10.\,1943 Le Pouliguen), \emph{Theaterleiter, Schauspieler}|pw}} Dramaturgen fürs Ausland zu{ }ſenden, dem ich neulich \strikeout{darüber} über das Stück\pwindex{Hofmannsthal, Hugo von 1.\,2.\,1874 Wien – 15.\,7.\,1929 Rodaun@\textsc{Hofmannsthal, Hugo von} (1.\,2.\,1874 Wien – 15.\,7.\,1929 Rodaun), \emph{Schriftsteller}!Elektra. Tragödie in einem Aufzug@\strich\emph{Elektra. Tragödie in einem Aufzug}|pwv} kurz berichtet habe.\pend
           
\pstart
           {\pb}Daſs \introOben{}B.\introOben{} Garlan\pwindex{Schnitzler, Arthur 15.\,5.\,1862 Wien – 21.\,10.\,1931 ebd.@\textsc{Schnitzler, Arthur} (15.\,5.\,1862 Wien – 21.\,10.\,1931 ebd.), \emph{Schriftsteller, Mediziner}!Frau Bertha Garlan. Roman@\strich\emph{Frau Bertha Garlan. Roman}|pw} beim zweiten Leſen{ }ſo angenehm auf Sie
               wirkte, freut mich{ }ſehr – ich hab es{ }ſeit dem Erſcheinen nicht wieder geleſen wie ich
               es (we{\geminationn} mich nicht äußerliche Gründe zu einer
               wiederholten Lectüre nöthigen) mit allen meinen gedruckten Sachen halte. Daher weiſs
               ich auch{ }ſeit etwa 8 Jahren nichts mehr von »Sterben\pwindex{Schnitzler, Arthur 15.\,5.\,1862 Wien – 21.\,10.\,1931 ebd.@\textsc{Schnitzler, Arthur} (15.\,5.\,1862 Wien – 21.\,10.\,1931 ebd.), \emph{Schriftsteller, Mediziner}!Sterben. Novelle@\strich\emph{Sterben. Novelle}|pw}«. Es sta{\geminationm}t aus der Zeit, wo mich der
               »Fall« mehr intereſſirt hat als die Menſchen, und ich denke das meiſte aus dieſer
               Epoche muſs wie luftlos wirken. Dieſe Sachen – ich hab es neulich wieder am »\textsc{Jour de {\pb}gloire}\pwindex{Schnitzler, Arthur 15.\,5.\,1862 Wien – 21.\,10.\,1931 ebd.@\textsc{Schnitzler, Arthur} (15.\,5.\,1862 Wien – 21.\,10.\,1931 ebd.), \emph{Schriftsteller, Mediziner}!Ehrentag@\strich\emph{Der Ehrentag}|pw}« \substVorne{}\textsuperscript{g}\substDazwischen{}e\substHinten{}rfahren, wirken in anſtändiger franzöſiſcher Übertragung beſſer als in meinem
               Deutſch. Die reine Tendenz des Erzählens iſt dem romaniſchen Sprachgeiſt eingeboren,
               während es im deutſchen gleichſam wie gegen die Natur wirkt, wenn die Mittheilung von
               Thatſachen der Seele und Menſchlichkeit entbehrt. Die umgekehrte Probe kann man
               machen, we{\geminationn} man irgend eine kurze \textsc{Maupassant}\pwindex{Maupassant, Guy de 5.\,8.\,1850 Tourville-sur-Arques – 7.\,7.\,1893 Paris@\textsc{Maupassant, Guy de} (5.\,8.\,1850 Tourville-sur-Arques – 7.\,7.\,1893 Paris), \emph{Schriftsteller}|pw} Geſchichte die franzöſiſch noch lange nicht{ }ſchwach wirkt, in deutſcher
               Ueberſetzung lieſt.\pend
           
\pstart
           – Immerhin hab ich die Empfindg daſs {\pb}meine Technik der
               inneren Entwicklung meiner Production noch nicht nachgekommen iſt – was mir übrigens
               nicht bange macht. Es iſt jetzt in mir wieder{ }ſo eine Neigung Sachen nur anzufangen
               und zu{ }ſkizziren wie in der Zeit, die der Anatol\pwindex{Schnitzler, Arthur 15.\,5.\,1862 Wien – 21.\,10.\,1931 ebd.@\textsc{Schnitzler, Arthur} (15.\,5.\,1862 Wien – 21.\,10.\,1931 ebd.), \emph{Schriftsteller, Mediziner}!Anatol@\strich\emph{Anatol}|pw}-Epoche vorherging. Am meiſten beſchäftige ich mich jetzt mit einer Art
               von Komödie\pwindex{Schnitzler, Arthur 15.\,5.\,1862 Wien – 21.\,10.\,1931 ebd.@\textsc{Schnitzler, Arthur} (15.\,5.\,1862 Wien – 21.\,10.\,1931 ebd.), \emph{Schriftsteller, Mediziner}!Fink und Fliederbusch. Komödie in drei Akten@\strich\emph{Fink und Fliederbusch. Komödie in drei Akten}|pwv} und bin innerlich
                  \strikeout{von dem Roman\pwindex{Schnitzler, Arthur 15.\,5.\,1862 Wien – 21.\,10.\,1931 ebd.@\textsc{Schnitzler, Arthur} (15.\,5.\,1862 Wien – 21.\,10.\,1931 ebd.), \emph{Schriftsteller, Mediziner}!Weg ins Freie. Roman@\strich\emph{Der Weg ins Freie. Roman}|pwv}} am meiſten von dem Roman\pwindex{Schnitzler, Arthur 15.\,5.\,1862 Wien – 21.\,10.\,1931 ebd.@\textsc{Schnitzler, Arthur} (15.\,5.\,1862 Wien – 21.\,10.\,1931 ebd.), \emph{Schriftsteller, Mediziner}!Weg ins Freie. Roman@\strich\emph{Der Weg ins Freie. Roman}|pwv} erfüllt, den ich im Frühjahr begonnen, den aber fortzuſetzen ich nicht
               in genügend reiner Sti{\geminationm}ung mich befinde.\pend
           
\pstart
           In Concerte gehen wir nicht{ }ſelten, ins Theater beinahe nie, aus perſönlichen {\pb}Gründen waren wir bei der \label{K_L01348-1v}\edtext{\textsc{Novella d’Andrea}\pwindex{\textcolor{red}{\textsuperscript{XXXX indx1}}!Novella d’Andrea@\strich\emph{Novella d’Andrea}|pw}}{\lemma{\textnormal{\emph{Novella d’Andrea}}}\Cendnote{\textnormal{Siehe A. S.: \emph{Tagebuch}, 21. 11. 1903.
               }}}\label{K_L01348-1} – und ich hab es nicht ohne Bitterkeit empfunden, daſs ich den Kainz\pwindex{Kainz, Josef 2.\,1.\,1858 Mosonmagyaróvár – 20.\,9.\,1910 Wien@\textsc{Kainz, Josef} (2.\,1.\,1858 Mosonmagyaróvár – 20.\,9.\,1910 Wien), \emph{Schauspieler}|pw} nie werde den Sala\pwindex{Schnitzler, Arthur 15.\,5.\,1862 Wien – 21.\,10.\,1931 ebd.@\textsc{Schnitzler, Arthur} (15.\,5.\,1862 Wien – 21.\,10.\,1931 ebd.), \emph{Schriftsteller, Mediziner}!einsame Weg. Schauspiel in fünf Akten@\strich\emph{Der einsame Weg. Schauspiel in fünf Akten}|pwv}{ }ſpielen \strikeout{k}{ }ſehen. Denn das Burgtheater\orgindex{Burgtheater@Burgtheater|pw}, wie Herr Schlenther\pwindex{Schlenther, Paul 20.\,8.\,1854 Chernyakhovsk – 30.\,4.\,1916 Berlin@\textsc{Schlenther, Paul} (20.\,8.\,1854 Chernyakhovsk – 30.\,4.\,1916 Berlin), \emph{Schriftsteller, Kritiker, Theaterleiter}|pw} an
                  Fiſcher\pwindex{Fischer, Samuel 24.\,12.\,1859 Liptovský Mikuláš – 15.\,10.\,1934 Berlin@\textsc{Fischer, Samuel} (24.\,12.\,1859 Liptovský Mikuláš – 15.\,10.\,1934 Berlin), \emph{Verleger}|pw} geſchrieben, »reflectirt nicht« auf
               dieſes Stück\pwindex{Schnitzler, Arthur 15.\,5.\,1862 Wien – 21.\,10.\,1931 ebd.@\textsc{Schnitzler, Arthur} (15.\,5.\,1862 Wien – 21.\,10.\,1931 ebd.), \emph{Schriftsteller, Mediziner}!einsame Weg. Schauspiel in fünf Akten@\strich\emph{Der einsame Weg. Schauspiel in fünf Akten}|pwv}. Brahm\pwindex{Brahm, Otto 5.\,2.\,1856 Hamburg – 28.\,11.\,1912 Berlin@\textsc{Brahm, Otto} (5.\,2.\,1856 Hamburg – 28.\,11.\,1912 Berlin), \emph{Theaterleiter, Regisseur}|pw} gegenüber (was Sie ja wohl wiſſen
               dürften) hat sich Schl.\pwindex{Schlenther, Paul 20.\,8.\,1854 Chernyakhovsk – 30.\,4.\,1916 Berlin@\textsc{Schlenther, Paul} (20.\,8.\,1854 Chernyakhovsk – 30.\,4.\,1916 Berlin), \emph{Schriftsteller, Kritiker, Theaterleiter}|pw} über das Stück\pwindex{Schnitzler, Arthur 15.\,5.\,1862 Wien – 21.\,10.\,1931 ebd.@\textsc{Schnitzler, Arthur} (15.\,5.\,1862 Wien – 21.\,10.\,1931 ebd.), \emph{Schriftsteller, Mediziner}!einsame Weg. Schauspiel in fünf Akten@\strich\emph{Der einsame Weg. Schauspiel in fünf Akten}|pwv}{ }ſehr misfällig geäußert;{ }ſcheint es aber, wie Brahm\pwindex{Brahm, Otto 5.\,2.\,1856 Hamburg – 28.\,11.\,1912 Berlin@\textsc{Brahm, Otto} (5.\,2.\,1856 Hamburg – 28.\,11.\,1912 Berlin), \emph{Theaterleiter, Regisseur}|pw}{ }ſagt, ganz oberflächlich – und wie ich überzeugt
               bin – mit böſem Willen geleſen zu haben.\pend
           
\pstart
           Und nun, wann{ }ſieht man{ }ſich wieder? Wie wär es, Montag oder
                  Mittwoch{ }Abend in dem Hietzinger
                  Restaurant\oindex{Wien@\textbf{Wien}!XIII., Hietzing@\textbf{XIII., Hietzing}!Ottakringer Bräu@\textbf{Ottakringer Bräu}, \emph{Bierhaus}|pwv}? Schrei{\pb}ben Sie mir, wann es Ihnen
               beſſer paſſt und ob auch Ihre Frau\pwindex{Hofmannsthal, Gertrude von 16.\,3.\,1880 Wien – 9.\,11.\,1959 Paddington@\textsc{Hofmannsthal, Gertrude von} (16.\,3.\,1880 Wien – 9.\,11.\,1959 Paddington)|pwv} mitkommt.\pend
           
\pstart
           Und Richard\pwindex{Beer-Hofmann, Richard 11.\,7.\,1866 Wien – 26.\,9.\,1945 New York City@\textsc{Beer-Hofmann, Richard} (11.\,7.\,1866 Wien – 26.\,9.\,1945 New York City), \emph{Schriftsteller}|pw}? Ich höre u{ }ſehe nichts von ihm. –
               Sobald das Wetter ein bischen angenehmer wird, kommen wir gern nach Rodaun\oindex{Wien@\textbf{Wien}!XXIII., Liesing@\textbf{XXIII., Liesing}!Rodaun@\textbf{Rodaun}, \emph{Region}|pw}.\pend
           
\pstart
           \label{K_L01348-2v}\edtext{Das andere}{\lemma{\textnormal{\emph{Das andere}}}\Cendnote{\textnormal{Vgl. XXXX Auszeichnungsfehler: Dokument L01347 nicht gefunden.
               }}}\label{K_L01348-2}, das ich bald bekomme, iſt wohl das gerettete
                     \textsc{Venedig}\pwindex{Hofmannsthal, Hugo von 1.\,2.\,1874 Wien – 15.\,7.\,1929 Rodaun@\textsc{Hofmannsthal, Hugo von} (1.\,2.\,1874 Wien – 15.\,7.\,1929 Rodaun), \emph{Schriftsteller}!gerettete Venedig. Trauerspiel in fünf Aufzügen@\strich\emph{Das gerettete Venedig. Trauerspiel in fünf Aufzügen}|pw}? –\pend
           
\pstart
           Leben Sie wohl. Herzlichſt\hspace*{1.5em}Ihr{\\[\baselineskip]}\spacefill\mbox{A.}\pend
           \leftskip=0em{}\selectlanguage{ngerman}\endnumbering\briefempfaengerindex{Hofmannsthal, Hugo von@\textsc{Hofmannsthal, Hugo von}!zzzSchnitzler, Arthur@\emph{von Arthur Schnitzler}!1903-12-101@{10. 12. 1903}|)be}\mylabel{L01348h}  \newcommand{\dateiname}{L01348}\newcommand{\titel}{Arthur Schnitzler an Hugo von Hofmannsthal, 10. 12. 1903}\newcommand{\editorInnen}{Martin Anton Müller und Gerd-Hermann Susen}%% latex-leseansicht-abspann.tex
%% Abspann für die Leseansicht.
%% Der Schalter \ifkorrekturansicht ist bereits durch den Vorspann gesetzt.

%% latex-abspann.tex
%% Gemeinsamer Abspann für Korrekturansicht und Leseansicht.
%% Setzt den Schalter \ifkorrekturansicht voraus (gesetzt in den
%% einbindenden Dateien latex-korrekturansicht-abspann.tex bzw.
%% latex-leseansicht-abspann.tex).
%% ---------------------------------------------------------------

\normalsize

% Das esempio-Environment wird nur in der Leseansicht benötigt
\ifkorrekturansicht\else
\newenvironment{esempio}[3]%
{
    \vspace{1.5ex}
    \rlap{\underline{#1}}
    \par
    \setlength{\parindent}{0cm}
    \nopagebreak
    \leftskip=#2cm
    \rightskip=#3cm
}
{
    \par
}
\fi

\doendnotes{C}
\bigskip
\vfill

\clearpage

\footnotesize

\ifkorrekturansicht
  \lohead{\textsc{register}}
\fi

% theindex-Environment neu definieren ohne reledmac
\makeatletter
\renewenvironment{theindex}{%
  \ifkorrekturansicht
    \section*{\indexname}%
  \else
    \subsubsection*{Index der erwähnten Entitäten}%
  \fi
  \setlength{\parindent}{0pt}%
  \setlength{\parskip}{0pt plus 0.3pt}%
  \let\item\@idxitem
}{%
  \ifkorrekturansicht\clearpage\fi
}
\makeatother

\IfFileExists{\jobname-pw.ind}{\input{\jobname-pw.ind}}{}

% Quellenangabe nur in der Leseansicht
\ifkorrekturansicht\else
% Fallback-Definitionen, falls die .tex-Datei \titel etc. nicht gesetzt hat
\providecommand{\titel}{}
\providecommand{\editorInnen}{}
\providecommand{\dateiname}{\jobname}

\vspace{3cm}

\vfill

\footnotesize
\textsc{Quelle}: \titel. Herausgegeben von {\editorInnen}. In: \emph{Arthur Schnitzler: Briefwechsel mit Autorinnen und Autoren}.
 Digitale Edition, https://schnitzler-briefe.acdh.oeaw.ac.at/{\dateiname}.html (Stand \today)
\fi

\end{document}


