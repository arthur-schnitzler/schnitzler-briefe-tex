%% latex-korrekturansicht-vorspann.tex
%% Vorspann für die Korrekturansicht.
%% Lädt die gemeinsame Datei latex-vorspann.tex mit gesetztem Schalter.

\newif\ifkorrekturansicht
\korrekturansichttrue

\input{../tex-inputs/latex-vorspann}


\section[Arthur Schnitzler an Hugo von Hofmannsthal, 10. 12. 1903]{L01348 Arthur Schnitzler an Hugo von Hofmannsthal, 10. 12. 1903}
\nopagebreak\mylabel{L01348v}
\rehead{ }\normalsize\beginnumbering\briefempfaengerindex{Hofmannsthal, Hugo von@\textsc{Hofmannsthal, Hugo von}!zzzSchnitzler, Arthur@\emph{von Arthur Schnitzler}!1903-12-101@{10. 12. 1903}|(be}
\toendnotes[C]{\smallbreak\pagebreak[2]}\Standort{FDH, Hs-30885,106.}
\physDesc{Brief, 2 Blätter, 6 Seiten, 2886 Zeichen
\newline{}Handschrift: schwarze Tinte, deutsche Kurrent
\newline{}Ordnung: 1) mit Bleistift von Schnitzler mutmaßlich bei der Durchsicht der Korrespondenz
                                    1929 datiert: »910«  2) mit Bleistift von Olga
                                 Schnitzler\pwindex{Schnitzler, Olga 17.01.1882 – 13.01.1970@\textsc{Schnitzler, Olga} (17.01.1882 – 13.01.1970), \emph{Schauspieler/Schauspielerin, Sänger/Sängerin}|pw} neben der Adressangabe vermerkt: »\textsc{Irrtum: damals wohnten wir schon in der Sternwartestrasse\oindex{Sternwartestrasse 71@\textbf{Sternwartestraße 71}, \emph{Wohngebäude (K.WHS)}|pw}. O.}«,
                                 was sich auf die (falsche) nachträgliche Einordnung auf das Jahr
                                 1910 bezieht 3) das zweite Blatt von unbekannter Hand mit Bleistift beschriftet:
                                    »II 10/12 910« 4) mit Bleistift von unbekannter Hand nummeriert
                                    »106a«}
\buchAbdrucke{\weitereDrucke{Hugo von Hofmannsthal, Arthur Schnitzler: \emph{Briefwechsel}. Frankfurt am Main: \emph{S. Fischer} 1964, S. 179–180.} }\toendnotes[C]{\smallbreak}
\pstart
           \raggedleft{}{\pb}XVIII Spöttelg. 7\oindex{Edmund-Weiss-Gasse 7@\textbf{Edmund-Weiß-Gasse 7}, \emph{Wohngebäude (K.WHS)}|pw}. {\\}Wien\oindex{Wien@\textbf{Wien}, \emph{A.ADM2}|pw}{ }10. 12. 9\textcolor{gray}{03}\pend
           
\pstart{}mein lieber Hugo, \pend\vspace{0.5em}
\pstart
           Sie haben offenbar einen Brief von mir nicht beko{\geminationm}en,
               den ich an Sie vor etwa 14 Tagen, ich glaube an dem Tag wo Ihre Elektra\pwindex{Elektra. Tragoedie in einem Aufzug@\emph{Elektra. Tragödie in einem Aufzug}|pw} bei mir erſchien, an Sie geſchrieben habe. Das
               weſentlichſte, was dieſer Brief enthielt war die Bitte Ihre Elektra\pwindex{Elektra. Tragoedie in einem Aufzug@\emph{Elektra. Tragödie in einem Aufzug}|pw} an \textsc{Antoine}\pwindex{Antoine, Andre 1858-01-31 – 1943-10-23@\textsc{Antoine, André} (1858-01-31 – 1943-10-23), \emph{Theaterleiter/Theaterleiterin, Schauspieler/Schauspielerin}|pw}, \textsc{resp}. an Dr \textsc{Stephan Epstein\pwindex{Epstein, Stephan 12.11.1866 – 1941@\textsc{Epstein, Stephan} (12.11.1866 – 1941), \emph{Schriftsteller/Schriftstellerin, Dramaturg/Dramaturgin, Übersetzer/Übersetzerin}|pw}{ }Paris 78 rue de l’Assomption\oindex{rue de l Assomption@\textbf{rue de l’Assomption}, \emph{Straße (K.STR)}|pw}, Antoines\pwindex{Antoine, Andre 1858-01-31 – 1943-10-23@\textsc{Antoine, André} (1858-01-31 – 1943-10-23), \emph{Theaterleiter/Theaterleiterin, Schauspieler/Schauspielerin}|pw}} Dramaturgen fürs Ausland zu
               ſenden, dem ich neulich \strikeout{darüber} über das Stück\pwindex{Elektra. Tragoedie in einem Aufzug@\emph{Elektra. Tragödie in einem Aufzug}|pwv} kurz berichtet habe.\pend
           
\pstart
           {\pb}Daſs \introOben{}B.\introOben{} Garlan\pwindex{Frau Bertha Garlan. Roman@\emph{Frau Bertha Garlan. Roman}|pw} beim zweiten Leſen ſo angenehm auf Sie
               wirkte, freut mich ſehr – ich hab es ſeit dem Erſcheinen nicht wieder geleſen wie ich
               es (we{\geminationn} mich nicht äußerliche Gründe zu einer
               wiederholten Lectüre nöthigen) mit allen meinen gedruckten Sachen halte. Daher weiſs
               ich auch ſeit etwa 8 Jahren nichts mehr von »Sterben\pwindex{Sterben. Novelle@\emph{Sterben. Novelle}|pw}«. Es sta{\geminationm}t aus der Zeit, wo mich der
               »Fall« mehr intereſſirt hat als die Menſchen, und ich denke das meiſte aus dieſer
               Epoche muſs wie luftlos wirken. Dieſe Sachen – ich hab es neulich wieder am »\textsc{Jour de {\pb}gloire}\pwindex{Ehrentag@\emph{Der Ehrentag}|pw}« \substVorne{}\textsuperscript{g}\substDazwischen{}e\substHinten{}rfahren, wirken in anſtändiger franzöſiſcher Übertragung beſſer als in meinem
               Deutſch. Die reine Tendenz des Erzählens iſt dem romaniſchen Sprachgeiſt eingeboren,
               während es im deutſchen gleichſam wie gegen die Natur wirkt, wenn die Mittheilung von
               Thatſachen der Seele und Menſchlichkeit entbehrt. Die umgekehrte Probe kann man
               machen, we{\geminationn} man irgend eine kurze \textsc{Maupassant}\pwindex{Maupassant, Guy de 05.08.1850 – 07.07.1893@\textsc{Maupassant, Guy de} (05.08.1850 – 07.07.1893), \emph{Schriftsteller/Schriftstellerin}|pw} Geſchichte die franzöſiſch noch lange nicht ſchwach wirkt, in deutſcher
               Ueberſetzung lieſt.\pend
           
\pstart
           – Immerhin hab ich die Empfindg daſs {\pb}meine Technik der
               inneren Entwicklung meiner Production noch nicht nachgekommen iſt – was mir übrigens
               nicht bange macht. Es iſt jetzt in mir wieder ſo eine Neigung Sachen nur anzufangen
               und zu ſkizziren wie in der Zeit, die der Anatol\pwindex{Anatol@\emph{Anatol}|pw}-Epoche vorherging. Am meiſten beſchäftige ich mich jetzt mit einer Art
               von Komödie\pwindex{Fink und Fliederbusch. Komoedie in drei Akten@\emph{Fink und Fliederbusch. Komödie in drei Akten}|pwv} und bin innerlich
                  \strikeout{von dem Roman\pwindex{Weg ins Freie. Roman@\emph{Der Weg ins Freie. Roman}|pwv}} am meiſten von dem Roman\pwindex{Weg ins Freie. Roman@\emph{Der Weg ins Freie. Roman}|pwv} erfüllt, den ich im Frühjahr begonnen, den aber fortzuſetzen ich nicht
               in genügend reiner Sti{\geminationm}ung mich befinde.\pend
           
\pstart
           In Concerte gehen wir nicht ſelten, ins Theater beinahe nie, aus perſönlichen {\pb}Gründen waren wir bei der \label{K_L01348-1v}\edtext{\textsc{Novella d’Andrea}\pwindex{Novella DAndrea@\emph{Novella d’Andrea}|pw}}{\lemma{\textnormal{\emph{Novella d’Andrea}}}\Cendnote{\textnormal{Siehe A. S.: \emph{Tagebuch}, 21. 11. 1903.
               }}}\label{K_L01348-1} – und ich hab es nicht ohne Bitterkeit empfunden, daſs ich den Kainz\pwindex{Kainz, Josef 02.01.1858 – 20.09.1910@\textsc{Kainz, Josef} (02.01.1858 – 20.09.1910), \emph{Schauspieler/Schauspielerin}|pw} nie werde den Sala\pwindex{einsame Weg. Schauspiel in fuenf Akten@\emph{Der einsame Weg. Schauspiel in fünf Akten}|pwv}{ }ſpielen \strikeout{k}{ }ſehen. Denn das Burgtheater\orgindex{Burgtheater@Burgtheater|pw}, wie Herr Schlenther\pwindex{Schlenther, Paul 20.08.1854 – 30.04.1916@\textsc{Schlenther, Paul} (20.08.1854 – 30.04.1916), \emph{Schriftsteller/Schriftstellerin, Kritiker/Kritikerin, Theaterleiter/Theaterleiterin}|pw} an
                  Fiſcher\pwindex{Fischer, Samuel 24.12.1859 – 15.10.1934@\textsc{Fischer, Samuel} (24.12.1859 – 15.10.1934), \emph{Verleger/Verlegerin}|pw} geſchrieben, »reflectirt nicht« auf
               dieſes Stück\pwindex{einsame Weg. Schauspiel in fuenf Akten@\emph{Der einsame Weg. Schauspiel in fünf Akten}|pwv}. Brahm\pwindex{Brahm, Otto 05.02.1856 – 28.11.1912@\textsc{Brahm, Otto} (05.02.1856 – 28.11.1912), \emph{Theaterleiter/Theaterleiterin, Regisseur/Regisseurin}|pw} gegenüber (was Sie ja wohl wiſſen
               dürften) hat sich Schl.\pwindex{Schlenther, Paul 20.08.1854 – 30.04.1916@\textsc{Schlenther, Paul} (20.08.1854 – 30.04.1916), \emph{Schriftsteller/Schriftstellerin, Kritiker/Kritikerin, Theaterleiter/Theaterleiterin}|pw} über das Stück\pwindex{einsame Weg. Schauspiel in fuenf Akten@\emph{Der einsame Weg. Schauspiel in fünf Akten}|pwv}{ }ſehr misfällig geäußert; ſcheint es aber, wie Brahm\pwindex{Brahm, Otto 05.02.1856 – 28.11.1912@\textsc{Brahm, Otto} (05.02.1856 – 28.11.1912), \emph{Theaterleiter/Theaterleiterin, Regisseur/Regisseurin}|pw}{ }ſagt, ganz oberflächlich – und wie ich überzeugt
               bin – mit böſem Willen geleſen zu haben.\pend
           
\pstart
           Und nun, wann ſieht man ſich wieder? Wie wär es, Montag oder
                  Mittwoch{ }Abend in dem Hietzinger
                  Restaurant\oindex{Ottakringer Braeu@\textbf{Ottakringer Bräu}, \emph{Bierhaus (K.BIR)}|pwv}? Schrei{\pb}ben Sie mir, wann es Ihnen
               beſſer paſſt und ob auch Ihre Frau\pwindex{Hofmannsthal, Gertrude von 16.03.1880 – 09.11.1959@\textsc{Hofmannsthal, Gertrude von} (16.03.1880 – 09.11.1959)|pwv} mitkommt.\pend
           
\pstart
           Und Richard\pwindex{Beer-Hofmann, Richard 1866-07-11 – 1945-09-26@\textsc{Beer-Hofmann, Richard} (1866-07-11 – 1945-09-26), \emph{Schriftsteller/Schriftstellerin}|pw}? Ich höre u ſehe nichts von ihm. –
               Sobald das Wetter ein bischen angenehmer wird, kommen wir gern nach Rodaun\oindex{Rodaun@\textbf{Rodaun}, \emph{A.ADM4}|pw}.\pend
           
\pstart
           \label{K_L01348-2v}\edtext{Das andere}{\lemma{\textnormal{\emph{Das andere}}}\Cendnote{\textnormal{Vgl. Hugo von Hofmannsthal an Arthur Schnitzler, 8. 12. [1903].
               }}}\label{K_L01348-2}, das ich bald bekomme, iſt wohl das gerettete
                     \textsc{Venedig}\pwindex{gerettete Venedig. Trauerspiel in fuenf Aufzuegen@\emph{Das gerettete Venedig. Trauerspiel in fünf Aufzügen}|pw}? –\pend
           
\pstart
           Leben Sie wohl. Herzlichſt\hspace*{1.5em}Ihr{\\[\baselineskip]}\spacefill\mbox{A.}\pend
           \leftskip=0em{}\selectlanguage{ngerman}\endnumbering\briefempfaengerindex{Hofmannsthal, Hugo von@\textsc{Hofmannsthal, Hugo von}!zzzSchnitzler, Arthur@\emph{von Arthur Schnitzler}!1903-12-101@{10. 12. 1903}|)be}\mylabel{L01348h}  \normalsize

\doendnotes{C}
\bigskip
\vfill

\clearpage

\footnotesize

\lohead{\textsc{register}}

% Definiere theindex-Environment komplett neu ohne reledmac
\makeatletter
\renewenvironment{theindex}{%
  \section*{\indexname}%
  \setlength{\parindent}{0pt}%
  \setlength{\parskip}{0pt plus 0.3pt}%
  \let\item\@idxitem
}{%
  \clearpage
}
\makeatother

\IfFileExists{\jobname-pw.ind}{\input{\jobname-pw.ind}}{}

\end{document}

      