%% latex-korrekturansicht-vorspann.tex
%% Vorspann für die Korrekturansicht.
%% Lädt die gemeinsame Datei latex-vorspann.tex mit gesetztem Schalter.

\newif\ifkorrekturansicht
\korrekturansichttrue

\input{../tex-inputs/latex-vorspann}


\section[ Paul Goldmann an Arthur Schnitzler, 17. 8. 1908]{L03465 Paul Goldmann an Arthur Schnitzler, 17. 8. 1908}
\nopagebreak\mylabel{L03465v}
\rehead{ }\normalsize\beginnumbering\briefempfaengerindex{Schnitzler, Arthur@\textsc{Schnitzler, Arthur}!zzzGoldmann, Paul@\emph{von Paul Goldmann}!1908-08-171@{17. 8. 1908}|(be}
\toendnotes[C]{\smallbreak\pagebreak[2]}\Standort{DLA, A:Schnitzler, HS.NZ85.1.3175.}
\physDesc{Bildpostkarte, 320 Zeichen
\newline{}Handschrift: 1) schwarze Tinte, deutsche Kurrent\hspace{1em}2) schwarze Tinte, lateinische Kurrent (\noindent{}Adresse)\hspace{1em}
\newline{}Versand: Stempel: »\nobreak{}\oindex{Marienbad@\textbf{Marienbad}, \emph{P.PPL}|pwk}Marienbad 1, 17. VIII. 08, 1\nobreak{}«.  }\toendnotes[C]{\smallbreak}\pstart{}{\pb}Herrn\pend{}\pstart{}Dr. Arthur Schnitzler\pend{}\pstart{}Wien\oindex{Wien@\textbf{Wien}, \emph{A.ADM2}|pw}\pend{}\pstart{}XVIII. Spöttelgaſse 7\oindex{Edmund-Weiss-Gasse 7@\textbf{Edmund-Weiß-Gasse 7}, \emph{Wohngebäude (K.WHS)}|pw}.\pend{}{\bigskip}
\pstart
           \noindent{}\centering{}{\pb}\textcolor{gray}{\textbf{Marienbad\oindex{Marienbad@\textbf{Marienbad}, \emph{P.PPL}|pw}. Kreuzbrunn\oindex{Kreuzbrunnen@\textbf{Kreuzbrunnen}, \emph{Monument (K.MON)}|pw} Colonnade.}}\pend
           \vspace{1em}
\pstart
           {\pb}17. 8. 08.\pend
           
\pstart{}Lieber Freund,\pend\vspace{0.5em}
\pstart
           Meine Frau\pwindex{Goldmann, Eva Marie 27.10.1877 – 02.11.1937@\textsc{Goldmann, Eva Marie} (27.10.1877 – 02.11.1937)|pwv} u. ich danken Dir
               herzlich für Deine Karte u. ſenden Deiner Frau\pwindex{Schnitzler, Olga 17.01.1882 – 13.01.1970@\textsc{Schnitzler, Olga} (17.01.1882 – 13.01.1970), \emph{Schauspieler/Schauspielerin, Sänger/Sängerin}|pwv} u. Dir herzliche Grüße! Hier\oindex{Marienbad@\textbf{Marienbad}, \emph{P.PPL}|pwv} gießt es ununterbrochen. Es tut mir
               leid, daß ich nicht auch dieſes Jahr nach \label{K_L03465-1v}\edtext{Tirol\oindex{Tirol@\textbf{Tirol}, \emph{A.ADM1}|pw}\oindex{Suedtirol@\textbf{Südtirol}, \emph{A.ADM2}|pw}}{\lemma{\textnormal{\emph{Tirol}}}\Cendnote{\textnormal{Schnitzler hielt sich im Sommer 1908 in Südtirol\oindex{Suedtirol@\textbf{Südtirol}, \emph{A.ADM2}|pwk}
                  auf.}}}\label{K_L03465-1} gegangen bin.\pend
           
\pstart
           Kommſt du dieſen Winter nach \label{K_L03465-2v}\edtext{Berlin\oindex{Berlin@\textbf{Berlin}, \emph{P.PPLC}|pw}}{\lemma{\textnormal{\emph{Berlin}}}\Cendnote{\textnormal{Schnitzler war erst Jahre später wieder in
                  Berlin, zwischen 22. 2. 1911 und 28. 2. 1911.}}}\label{K_L03465-2}?\pend
           \pstart Dein \spacefill\mbox{Paul Goldmann}\pend{}\selectlanguage{ngerman}\endnumbering\briefempfaengerindex{Schnitzler, Arthur@\textsc{Schnitzler, Arthur}!zzzGoldmann, Paul@\emph{von Paul Goldmann}!1908-08-171@{17. 8. 1908}|)be}\mylabel{L03465h}  \normalsize

\doendnotes{C}
\bigskip
\vfill

\clearpage

\footnotesize

\lohead{\textsc{register}}

% Definiere theindex-Environment komplett neu ohne reledmac
\makeatletter
\renewenvironment{theindex}{%
  \section*{\indexname}%
  \setlength{\parindent}{0pt}%
  \setlength{\parskip}{0pt plus 0.3pt}%
  \let\item\@idxitem
}{%
  \clearpage
}
\makeatother

\IfFileExists{\jobname-pw.ind}{\input{\jobname-pw.ind}}{}

\end{document}

      