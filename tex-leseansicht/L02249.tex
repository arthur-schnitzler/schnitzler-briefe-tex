%% latex-leseansicht-vorspann.tex
%% Vorspann für die Leseansicht.
%% Lädt die gemeinsame Datei latex-vorspann.tex mit nicht gesetztem Schalter.

\newif\ifkorrekturansicht
\korrekturansichtfalse

\input{../tex-inputs/latex-vorspann}


\section[Robert Adam an Arthur Schnitzler, 9. 12. 1916]{L02249 Robert Adam an Arthur Schnitzler, 9. 12. 1916}
\nopagebreak\mylabel{L02249v}
\rehead{ }\normalsize\beginnumbering\briefempfaengerindex{Schnitzler, Arthur@\textsc{Schnitzler, Arthur}!zzzAdam, Robert@\emph{von Robert Adam}!1916-12-091@{9. 12. 1916}|(be}
\toendnotes[C]{\smallbreak\pagebreak[2]}
\correspDesc{Versand  durch Robert Adam am 9. 12. 1916 in Wien
\newline{}Erhalt  durch Arthur Schnitzler im Zeitraum [9. 12. 1916
                  – 13. 12. 1916?] in Wien}\toendnotes[C]{\smallbreak}
\Standort{DLA, A:Schnitzler, HS.NZ85.1.4230,17.}
\physDesc{Brief, 1 Blatt, 3 Seiten, 1172 Zeichen
\newline{}Handschrift: schwarze Tinte, deutsche Kurrent
\newline{}Schnitzler: 1) mit Bleistift beschriftet: »\textsc{Adam}«  2) mit rotem Buntstift zwei Unterstreichungen}\Standort{Wien, Österreichische Nationalbibliothek, Cod.ser. 52.263, 182.}
\physDesc{Brief, maschinenschriftliche Abschrift, 1 Blatt, 1 Seite, 1172 Zeichen
\newline{}Schreibmaschine}\toendnotes[C]{\smallbreak}
\pstart
           \raggedleft{}{\pb}Wien\oindex{Wien@\textbf{Wien}, \emph{Verwaltungsgebiet}|pw}, am 9. Dezember 1916.\pend
           
\pstart{}Hochverehrter Herr Doktor!\pend\vspace{0.5em}
\pstart
           Ich teile Ihnen – natürlich{ }ſehr erſtaunt – mit, daß ich heute einen Brief des Hr.
               Oberregiſſeurs \textsc{Steinrück}\pwindex{Steinrück, Albert 20.\,5.\,1872 Wetterburg – 11.\,2.\,1929 Berlin@\textsc{Steinrück, Albert} (20.\,5.\,1872 Wetterburg – 11.\,2.\,1929 Berlin), \emph{Schauspieler}|pw} erhielt: der »Neidhard\pwindex{Adam, Robert 20.\,4.\,1877 Wien – 16.\,10.\,1961 Baden bei Wien@\textsc{Adam, Robert} (20.\,4.\,1877 Wien – 16.\,10.\,1961 Baden bei Wien), \emph{Schriftsteller, Richter}!Neidhard@\strich\emph{Neidhard}|pw}« habe{ }ſein
               ehrliches Intereſſe erweckt und er bedaure es unendlich, daß er{ }ſeiner monſtröſen
               Form wegen nicht zu einer Aufführung geeignet{ }ſei; er rate mir zu einer Überarbeitung
               unter herzhaften Strichen, wodurch ein wirkſames Werk zuſtande käme. Dieſes{ }ſoll ich
               direkt an den Dramaturgen D\textsuperscript{r}{ }Gutherz\pwindex{Gutherz, Gerhard 7.\,9.\,1877 Wien – 21.\,3.\,1942 München@\textsc{Gutherz, Gerhard} (7.\,9.\,1877 Wien – 21.\,3.\,1942 München), \emph{Schriftsteller, Dramaturg}|pw}{ }ſenden und dürfte mich auf ihn be{\pb}rufen, auch darauf, daß er{ }ſich für die Rolle \uline{ſehr} intereſſiere. Den \textsc{Alî ibn Bekkâr}\pwindex{Adam, Robert 20.\,4.\,1877 Wien – 16.\,10.\,1961 Baden bei Wien@\textsc{Adam, Robert} (20.\,4.\,1877 Wien – 16.\,10.\,1961 Baden bei Wien), \emph{Schriftsteller, Richter}!Geschichte des Alî ibn Bekkâr mit Schams an-Nahâr@\strich\emph{Die Geschichte des Alî ibn Bekkâr mit Schams an-Nahâr}|pw} hielte er für »nicht hinreißend«.\pend
           
\pstart
           Ich habe natürlich umgehend erwidert, daß ich mich{ }ſofort an die Herſtellung eines
                  Bühnen-Neidhard\pwindex{Adam, Robert 20.\,4.\,1877 Wien – 16.\,10.\,1961 Baden bei Wien@\textsc{Adam, Robert} (20.\,4.\,1877 Wien – 16.\,10.\,1961 Baden bei Wien), \emph{Schriftsteller, Richter}!Neidhard@\strich\emph{Neidhard}|pw} machen würde, und zugleich
               das Manuſkript des »Fremden\pwindex{Adam, Robert 20.\,4.\,1877 Wien – 16.\,10.\,1961 Baden bei Wien@\textsc{Adam, Robert} (20.\,4.\,1877 Wien – 16.\,10.\,1961 Baden bei Wien), \emph{Schriftsteller, Richter}!Fremde@\strich\emph{Der Fremde}|pw}« beigeſchloſſen. Ich
               bin{ }ſehr begierig, ob \textsc{Steinrück}\pwindex{Steinrück, Albert 20.\,5.\,1872 Wetterburg – 11.\,2.\,1929 Berlin@\textsc{Steinrück, Albert} (20.\,5.\,1872 Wetterburg – 11.\,2.\,1929 Berlin), \emph{Schauspieler}|pw} meinem Peſſimismus \label{K_L02249-1v}\edtext{\textsc{quoad}}{\lemma{\textnormal{\emph{quoad}}}\Cendnote{\textnormal{lateinisch: insofern}}}\label{K_L02249-1}
               Bühnenwirkſamkeit Recht geben wird oder Ihrer dem Stücke günſtigeren Anſicht (die ich
               ihm mitteilte).\pend
           
\pstart
           Nochmals herzlichen Dank, hochverehrter Herr Doktor! Jetzt heißt’s an die Neidhard\pwindex{Adam, Robert 20.\,4.\,1877 Wien – 16.\,10.\,1961 Baden bei Wien@\textsc{Adam, Robert} (20.\,4.\,1877 Wien – 16.\,10.\,1961 Baden bei Wien), \emph{Schriftsteller, Richter}!Neidhard@\strich\emph{Neidhard}|pw}-Arbeit gehen: ach, wenn Sie wüßten, in
               welchem atembeengenden Wuſt von Be{\pb}tätigungen und
               unerfüllten Pflichten ich{ }ſtecke!\pend
           
\pstart
           Ihr{\\[\baselineskip]}\spacefill\mbox{Robert Adam}\pend
           \leftskip=0em{}\selectlanguage{ngerman}\endnumbering\briefempfaengerindex{Schnitzler, Arthur@\textsc{Schnitzler, Arthur}!zzzAdam, Robert@\emph{von Robert Adam}!1916-12-091@{9. 12. 1916}|)be}\mylabel{L02249h}  \newcommand{\dateiname}{L02249}\newcommand{\titel}{Robert Adam an Arthur Schnitzler, 9. 12. 1916}\newcommand{\editorInnen}{Martin Anton Müller und Gerd-Hermann Susen}%% latex-leseansicht-abspann.tex
%% Abspann für die Leseansicht.
%% Der Schalter \ifkorrekturansicht ist bereits durch den Vorspann gesetzt.

%% latex-abspann.tex
%% Gemeinsamer Abspann für Korrekturansicht und Leseansicht.
%% Setzt den Schalter \ifkorrekturansicht voraus (gesetzt in den
%% einbindenden Dateien latex-korrekturansicht-abspann.tex bzw.
%% latex-leseansicht-abspann.tex).
%% ---------------------------------------------------------------

\normalsize

% Das esempio-Environment wird nur in der Leseansicht benötigt
\ifkorrekturansicht\else
\newenvironment{esempio}[3]%
{
    \vspace{1.5ex}
    \rlap{\underline{#1}}
    \par
    \setlength{\parindent}{0cm}
    \nopagebreak
    \leftskip=#2cm
    \rightskip=#3cm
}
{
    \par
}
\fi

\doendnotes{C}
\bigskip
\vfill

\clearpage

\footnotesize

\ifkorrekturansicht
  \lohead{\textsc{register}}
\fi

% theindex-Environment neu definieren ohne reledmac
\makeatletter
\renewenvironment{theindex}{%
  \ifkorrekturansicht
    \section*{\indexname}%
  \else
    \subsubsection*{Index der erwähnten Entitäten}%
  \fi
  \setlength{\parindent}{0pt}%
  \setlength{\parskip}{0pt plus 0.3pt}%
  \let\item\@idxitem
}{%
  \ifkorrekturansicht\clearpage\fi
}
\makeatother

\IfFileExists{\jobname-pw.ind}{\input{\jobname-pw.ind}}{}

% Quellenangabe nur in der Leseansicht
\ifkorrekturansicht\else
% Fallback-Definitionen, falls die .tex-Datei \titel etc. nicht gesetzt hat
\providecommand{\titel}{}
\providecommand{\editorInnen}{}
\providecommand{\dateiname}{\jobname}

\vspace{3cm}

\vfill

\footnotesize
\textsc{Quelle}: \titel. Herausgegeben von {\editorInnen}. In: \emph{Arthur Schnitzler: Briefwechsel mit Autorinnen und Autoren}.
 Digitale Edition, https://schnitzler-briefe.acdh.oeaw.ac.at/{\dateiname}.html (Stand \today)
\fi

\end{document}


