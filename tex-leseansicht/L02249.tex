%% latex-korrekturansicht-vorspann.tex
%% Vorspann für die Korrekturansicht.
%% Lädt die gemeinsame Datei latex-vorspann.tex mit gesetztem Schalter.

\newif\ifkorrekturansicht
\korrekturansichttrue

\input{../tex-inputs/latex-vorspann}


\section[Robert Adam an Arthur Schnitzler, 9. 12. 1916]{L02249 Robert Adam an Arthur Schnitzler, 9. 12. 1916}
\nopagebreak\mylabel{L02249v}
\rehead{ }\normalsize\beginnumbering\briefempfaengerindex{Schnitzler, Arthur@\textsc{Schnitzler, Arthur}!zzzAdam, Robert@\emph{von Robert Adam}!1916-12-091@{9. 12. 1916}|(be}
\toendnotes[C]{\smallbreak\pagebreak[2]}\Standort{DLA, A:Schnitzler, HS.NZ85.1.4230,17.}
\physDesc{Brief, 1 Blatt, 3 Seiten, 1172 Zeichen
\newline{}Handschrift: schwarze Tinte, deutsche Kurrent
\newline{}Schnitzler: 1) mit Bleistift beschriftet: »\textsc{Adam}«  2) mit rotem Buntstift zwei Unterstreichungen}\Standort{Wien, Österreichische Nationalbibliothek, Cod.ser. 52.263, 182.}
\physDesc{Brief, maschinenschriftliche Abschrift1 Blatt, 1 Seite, 1172 Zeichen
\newline{}Schreibmaschine}\toendnotes[C]{\smallbreak}
\pstart
           \raggedleft{}{\pb}Wien\oindex{Wien@\textbf{Wien}, \emph{A.ADM2}|pw}, am 9. Dezember
                  1916.\pend
           
\pstart{}Hochverehrter Herr Doktor!\pend\vspace{0.5em}
\pstart
           Ich teile Ihnen – natürlich ſehr erſtaunt – mit, daß ich heute einen Brief des Hr.
               Oberregiſſeurs \textsc{Steinrück}\pwindex{Steinrueck, Albert 20.05.1872 – 11.02.1929@\textsc{Steinrück, Albert} (20.05.1872 – 11.02.1929), \emph{Schauspieler/Schauspielerin}|pw} erhielt: der »Neidhard\pwindex{Neidhard@\emph{Neidhard}|pw}« habe ſein
               ehrliches Intereſſe erweckt und er bedaure es unendlich, daß er ſeiner monſtröſen
               Form wegen nicht zu einer Aufführung geeignet ſei; er rate mir zu einer Überarbeitung
               unter herzhaften Strichen, wodurch ein wirkſames Werk zuſtande käme. Dieſes ſoll ich
               direkt an den Dramaturgen D\textsuperscript{r}{ }Gutherz\pwindex{Gutherz, Gerhard 07.09.1877 – 21.03.1942@\textsc{Gutherz, Gerhard} (07.09.1877 – 21.03.1942), \emph{Schriftsteller/Schriftstellerin, Dramaturg/Dramaturgin}|pw}{ }ſenden und dürfte mich auf ihn be{\pb}rufen, auch darauf, daß er ſich für die Rolle \uline{ſehr} intereſſiere. Den \textsc{Alî ibn Bekkâr}\pwindex{Geschichte des Alî ibn Bekkâr mit Schams an-Nahâr@\emph{Die Geschichte des Alî ibn Bekkâr mit Schams an-Nahâr}|pw} hielte er für »nicht hinreißend«.\pend
           
\pstart
           Ich habe natürlich umgehend erwidert, daß ich mich ſofort an die Herſtellung eines
                  Bühnen-Neidhard\pwindex{Neidhard@\emph{Neidhard}|pw} machen würde, und zugleich
               das Manuſkript des »Fremden\pwindex{Fremde@\emph{Der Fremde}|pw}« beigeſchloſſen. Ich
               bin ſehr begierig, ob \textsc{Steinrück}\pwindex{Steinrueck, Albert 20.05.1872 – 11.02.1929@\textsc{Steinrück, Albert} (20.05.1872 – 11.02.1929), \emph{Schauspieler/Schauspielerin}|pw} meinem Peſſimismus \label{K_L02249-1v}\edtext{\textsc{quoad}}{\lemma{\textnormal{\emph{quoad}}}\Cendnote{\textnormal{lateinisch: insofern}}}\label{K_L02249-1}
               Bühnenwirkſamkeit Recht geben wird oder Ihrer dem Stücke günſtigeren Anſicht (die ich
               ihm mitteilte).\pend
           
\pstart
           Nochmals herzlichen Dank, hochverehrter Herr Doktor! Jetzt heißt’s an die Neidhard\pwindex{Neidhard@\emph{Neidhard}|pw}-Arbeit gehen: ach, wenn Sie wüßten, in
               welchem atembeengenden Wuſt von Be{\pb}tätigungen und
               unerfüllten Pflichten ich ſtecke!\pend
           
\pstart
           Ihr{\\[\baselineskip]}\spacefill\mbox{Robert Adam}\pend
           \leftskip=0em{}\selectlanguage{ngerman}\endnumbering\briefempfaengerindex{Schnitzler, Arthur@\textsc{Schnitzler, Arthur}!zzzAdam, Robert@\emph{von Robert Adam}!1916-12-091@{9. 12. 1916}|)be}\mylabel{L02249h}  \normalsize

\doendnotes{C}
\bigskip
\vfill

\clearpage

\footnotesize

\lohead{\textsc{register}}

% Definiere theindex-Environment komplett neu ohne reledmac
\makeatletter
\renewenvironment{theindex}{%
  \section*{\indexname}%
  \setlength{\parindent}{0pt}%
  \setlength{\parskip}{0pt plus 0.3pt}%
  \let\item\@idxitem
}{%
  \clearpage
}
\makeatother

\IfFileExists{\jobname-pw.ind}{\input{\jobname-pw.ind}}{}

\end{document}

      