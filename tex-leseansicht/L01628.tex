\input{../tex-inputs/latex-pdf-vorspann}
\begin{center}
            \textcolor{red}{ENTWURF. ENTZIFFERUNG NOCH NICHT KORREKTURGELESEN}
                      \end{center}
            
               \section[Hugo von Hofmannsthal an Arthur Schnitzler, {[}12. 9. 1906{]}]{ Hugo von Hofmannsthal an Arthur Schnitzler, {[}12. 9. 1906{]}}\nopagebreak\mylabel{v}\rehead{ }\begin{ledgroupsized}[t]{13cm}\normalsize\beginnumbering\briefempfaengerindex{Schnitzler, Arthur@\textsc{Schnitzler, Arthur}!zzzHofmannsthal, Hugo von@\emph{von Hugo von Hofmannsthal}!1906-09-122@{{[}12. 9. 1906{]}}|(be} \toendnotes[C]{\smallbreak\pagebreak[2]} \Standort{CUL, Schnitzler, B 43.}
\physDesc{Brief, 1 Blatt, 1 Seite
\newline{}Handschrift: schwarze Tinte, lateinische Kurrent
\newline{}Schnitzler: mit Bleistift datiert: »12/9 906« \newline{}Ordnung: 1) mit Bleistift von unbekannter Hand nummeriert: »\strikeout{264}« 2) mit Bleistift von unbekannter Hand nummeriert: »266«}\buchAbdrucke{\weitereDrucke{Hugo von Hofmannsthal, Arthur Schnitzler: \emph{Briefwechsel}. Hg. Therese Nickl und Heinrich Schnitzler. Frankfurt am Main: \emph{S. Fischer} 1964, S. 223.} }\pstart
           \centering{}{\pb}Mittwoch\pend
           \pstart
           Kommen direct; freuen uns. Bitten durch Rosenbaum\pwindex{Rostler, Karl 20.04.1872 – 31.01.1940@\textsc{Rostler, Karl} (20.04.1872 – 31.01.1940), \emph{Hotelportier}|pw}
               für morgen Donnerstag{ }7\textsuperscript{h} abends zweibettiges Zimmer möglichst
               billig, neues Gebäude, höheres Stockwerk, \uline{nach rückwärts}, bestellen.\pend
           \pstart Herzlich\spacefill\mbox{Hugo.}\pend{}\endnumbering\briefempfaengerindex{Schnitzler, Arthur@\textsc{Schnitzler, Arthur}!zzzHofmannsthal, Hugo von@\emph{von Hugo von Hofmannsthal}!1906-09-122@{{[}12. 9. 1906{]}}|)be}\mylabel{h}\end{ledgroupsized}  \newcommand{\dateiname}{L01628}\newcommand{\titel}{Hugo von Hofmannsthal an Arthur Schnitzler, [12. 9. 1906]}\newcommand{\editorInnen}{Martin Anton Müller und Gerd-Hermann Susen}\input{../tex-inputs/latex-pdf-abspann}
      