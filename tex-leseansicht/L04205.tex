%% latex-leseansicht-vorspann.tex
%% Vorspann für die Leseansicht.
%% Lädt die gemeinsame Datei latex-vorspann.tex mit nicht gesetztem Schalter.

\newif\ifkorrekturansicht
\korrekturansichtfalse

\input{../tex-inputs/latex-vorspann}


\section[Arthur Schnitzler an Gustav Schwarzkopf, {{[}}4. 5. 1903?{{]}}]{L04205 Arthur Schnitzler an Gustav Schwarzkopf, {[}4. 5. 1903?{]}}
\nopagebreak\mylabel{L04205v}
\rehead{ }\normalsize\beginnumbering\briefempfaengerindex{Schwarzkopf, Gustav@\textsc{Schwarzkopf, Gustav}!zzzSchnitzler, Arthur@\emph{von Arthur Schnitzler}!1903-05-041@{{[}4. 5. 1903?{]}}|(be}
\toendnotes[C]{\smallbreak\pagebreak[2]}
\correspDesc{Versand  durch Arthur Schnitzler am [4. 5. 1903?] in Wien
\newline{}Erhalt  durch Gustav Schwarzkopf im Zeitraum [4. 5. 1903 – 7. 5. 1903?] in Wien}\toendnotes[C]{\smallbreak}
\Standort{CUL, Schnitzler, B 96.}
\physDesc{Brief, 1 Blatt, 2 Seiten, 248 Zeichen
\newline{}Handschrift: schwarze Tinte, deutsche Kurrent}\toendnotes[C]{\smallbreak}
\pstart{}{\pb}lieber Guſtav,\pend\vspace{0.5em}
\pstart
           we{\geminationn} Sie \label{K_L04205-1v}\edtext{morgen Dinſtag}{\lemma{\textnormal{\emph{morgen Dinstag}}}\Cendnote{\textnormal{Vgl. A. S.: \emph{Tagebuch}, 5. 5. 1903.
            }}}\label{K_L04205-1}
      Abend gegen 6 in
      der Gentzgaſſe\oindex{Wien@\textbf{Wien}!XVIII., Währing@\textbf{XVIII., Währing}!Gentzgasse 110@\textbf{Gentzgasse 110}, \emph{Wohngebäude}|pw} erſcheinen
      wollen, werden Sie nicht
      nur Olga\pwindex{Schnitzler, Olga 17.\,1.\,1882 Wien – 13.\,1.\,1970 Lugano@\textsc{Schnitzler, Olga} (17.\,1.\,1882 Wien – 13.\,1.\,1970 Lugano), \emph{Schauspielerin, Sängerin}|pw} u mich, ſondern 
      wieder einmal einen
            Paul\pwindex{Marx, Paul 21.\,7.\,1879 Wien – 30.\,10.\,1956 ebd.@\textsc{Marx, Paul} (21.\,7.\,1879 Wien – 30.\,10.\,1956 ebd.), \emph{Regisseur, Schauspieler}|pw}, diesmal den
               {\pb}\label{K_L04205-2v}\edtext{\textcolor{gray}{dünnen}}{\lemma{\textnormal{\emph{dünnen}}}\Cendnote{\textnormal{In Unterscheidung
                  zum beleibteren Paul Goldmann\pwindex{Goldmann, Paul 31.\,1.\,1865 Breslau – 25.\,9.\,1935 Wien@\textsc{Goldmann, Paul} (31.\,1.\,1865 Breslau – 25.\,9.\,1935 Wien), \emph{Schriftsteller, Journalist}|pwk}.}}}\label{K_L04205-2} ſehr erfreue\textcolor{gray}{n.}\pend
           
\pstart
           Herzlichſt{\\[\baselineskip]} Ihr{\\[\baselineskip]}\spacefill\mbox{Arthur}\pend
           \leftskip=0em{}
\pstart
           \noindent{}Natürlich ſind Sie geladen,
      mit uns zu nachtmahlen.\pend
           \selectlanguage{ngerman}\endnumbering\briefempfaengerindex{Schwarzkopf, Gustav@\textsc{Schwarzkopf, Gustav}!zzzSchnitzler, Arthur@\emph{von Arthur Schnitzler}!1903-05-041@{{[}4. 5. 1903?{]}}|)be}\mylabel{L04205h}
\begin{anhang}
\end{anhang}\newcommand{\dateiname}{L04205}\newcommand{\titel}{Arthur Schnitzler an Gustav Schwarzkopf, [4. 5. 1903?]}\newcommand{\editorInnen}{Herausgegeben von Jahnke, SelmaMüller, Martin Anton}%% latex-leseansicht-abspann.tex
%% Abspann für die Leseansicht.
%% Der Schalter \ifkorrekturansicht ist bereits durch den Vorspann gesetzt.

%% latex-abspann.tex
%% Gemeinsamer Abspann für Korrekturansicht und Leseansicht.
%% Setzt den Schalter \ifkorrekturansicht voraus (gesetzt in den
%% einbindenden Dateien latex-korrekturansicht-abspann.tex bzw.
%% latex-leseansicht-abspann.tex).
%% ---------------------------------------------------------------

\normalsize

% Das esempio-Environment wird nur in der Leseansicht benötigt
\ifkorrekturansicht\else
\newenvironment{esempio}[3]%
{
    \vspace{1.5ex}
    \rlap{\underline{#1}}
    \par
    \setlength{\parindent}{0cm}
    \nopagebreak
    \leftskip=#2cm
    \rightskip=#3cm
}
{
    \par
}
\fi

\doendnotes{C}
\bigskip
\vfill

\clearpage

\footnotesize

\ifkorrekturansicht
  \lohead{\textsc{register}}
\fi

% theindex-Environment neu definieren ohne reledmac
\makeatletter
\renewenvironment{theindex}{%
  \ifkorrekturansicht
    \section*{\indexname}%
  \else
    \subsubsection*{Index der erwähnten Entitäten}%
  \fi
  \setlength{\parindent}{0pt}%
  \setlength{\parskip}{0pt plus 0.3pt}%
  \let\item\@idxitem
}{%
  \ifkorrekturansicht\clearpage\fi
}
\makeatother

\IfFileExists{\jobname-pw.ind}{\input{\jobname-pw.ind}}{}

% Quellenangabe nur in der Leseansicht
\ifkorrekturansicht\else
% Fallback-Definitionen, falls die .tex-Datei \titel etc. nicht gesetzt hat
\providecommand{\titel}{}
\providecommand{\editorInnen}{}
\providecommand{\dateiname}{\jobname}

\vspace{3cm}

\vfill

\footnotesize
\textsc{Quelle}: \titel. Herausgegeben von {\editorInnen}. In: \emph{Arthur Schnitzler: Briefwechsel mit Autorinnen und Autoren}.
 Digitale Edition, https://schnitzler-briefe.acdh.oeaw.ac.at/{\dateiname}.html (Stand \today)
\fi

\end{document}


