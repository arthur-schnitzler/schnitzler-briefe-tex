%% latex-korrekturansicht-vorspann.tex
%% Vorspann für die Korrekturansicht.
%% Lädt die gemeinsame Datei latex-vorspann.tex mit gesetztem Schalter.

\newif\ifkorrekturansicht
\korrekturansichttrue

\input{../tex-inputs/latex-vorspann}


\section[Hugo von Hofmannsthal an Arthur Schnitzler, {[}1. 6. 1905{]}]{L01522 Hugo von Hofmannsthal an Arthur Schnitzler, {[}1. 6. 1905{]}}
\nopagebreak\mylabel{L01522v}
\rehead{ }\normalsize\beginnumbering\briefempfaengerindex{Schnitzler, Arthur@\textsc{Schnitzler, Arthur}!zzzHofmannsthal, Hugo von@\emph{von Hugo von Hofmannsthal}!1905-06-011@{{[}1. 6. 1905{]}}|(be}
\toendnotes[C]{\smallbreak\pagebreak[2]}\Standort{CUL, Schnitzler, B 43.}
\physDesc{Brief, 1 Blatt, 1 Seite, 171 Zeichen
\newline{}Handschrift: schwarze Tinte, deutsche Kurrent
\newline{}Schnitzler: mit Bleistift datiert: »Juni 905« 
\newline{}Ordnung: 1) mit Bleistift von unbekannter Hand nummeriert:
                                    »252«  2) mit Bleistift von unbekannter Hand nummeriert:
                                    »254a«}
\buchAbdrucke{\weitereDrucke{Hugo von Hofmannsthal, Arthur Schnitzler: \emph{Briefwechsel}. Frankfurt am Main: \emph{S. Fischer} 1964, S. 211.} }\toendnotes[C]{\smallbreak}
\pstart
           \raggedleft{}{\pb}Do{\geminationn}erstg\pend
           \vspace{0.5em}
\pstart
           Müſſen ausgerechnet Samstag{ }\label{K_L01522-1v}\edtext{Sommernachtstraum\pwindex{Sommernachtstraum. Komoedie in fuenf Aufzuegen@\emph{Ein Sommernachtstraum. Komödie in fünf Aufzügen}|pw}}{\lemma{\textnormal{\emph{Sommernachtstraum}}}\Cendnote{\textnormal{Sie besuchten ein Gastspiel des \emph{Kleinen}\orgindex{Kleines Theater@Kleines Theater|pwk} und des \emph{Neuen Theaters}\orgindex{Neues Theater@Neues Theater|pwk} im Theater an der Wien\oindex{Theater an der Wien@\textbf{Theater an der Wien}, \emph{Theater (K.THE)}|pwk}
                  am 3. 6. 1905. Schnitzler hatte
                  bereits eine frühere Aufführung besucht, vgl. A. S.: \emph{Tagebuch}, 20. 5. 1905.}}}\label{K_L01522-1} gehen. Erklärung mündlich. Erbitten
               morgen Freitag Depeſche ob \textsc{rendezvous}{ }7\textsuperscript{h} morgen Freitag möglich. Andernfalls Montag??\pend
           \pstart \spacefill\mbox{Hugo.}\pend{}\selectlanguage{ngerman}\endnumbering\briefempfaengerindex{Schnitzler, Arthur@\textsc{Schnitzler, Arthur}!zzzHofmannsthal, Hugo von@\emph{von Hugo von Hofmannsthal}!1905-06-011@{{[}1. 6. 1905{]}}|)be}\mylabel{L01522h}  \normalsize

\doendnotes{C}
\bigskip
\vfill

\clearpage

\footnotesize

\lohead{\textsc{register}}

% Definiere theindex-Environment komplett neu ohne reledmac
\makeatletter
\renewenvironment{theindex}{%
  \section*{\indexname}%
  \setlength{\parindent}{0pt}%
  \setlength{\parskip}{0pt plus 0.3pt}%
  \let\item\@idxitem
}{%
  \clearpage
}
\makeatother

\IfFileExists{\jobname-pw.ind}{\input{\jobname-pw.ind}}{}

\end{document}

      