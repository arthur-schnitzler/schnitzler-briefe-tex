%% latex-leseansicht-vorspann.tex
%% Vorspann für die Leseansicht.
%% Lädt die gemeinsame Datei latex-vorspann.tex mit nicht gesetztem Schalter.

\newif\ifkorrekturansicht
\korrekturansichtfalse

\input{../tex-inputs/latex-vorspann}


         
         \renewcommand{\erwaehntePersonen}{Personen: Raoul Auernheimer, Richard Beer-Hofmann, Alfred Eckmann, Albert Renkin, Karl Schönherr, Franz Werfel}
         \renewcommand{\erwaehnteInstitutionen}{Institutionen: Bundestheaterkassen, Burgtheater, Ministerium für Unterricht}
         \renewcommand{\erwaehnteOrte}{Orte: Wien}
         \renewcommand{\erwaehnteWerke}{}
               \section[Arthur Schnitzler an Richard Beer-Hofmann, 21. 1. 1924]{ Arthur Schnitzler an Richard Beer-Hofmann, 21. 1. 1924}\nopagebreak\mylabel{v}\rehead{ }\begin{ledgroupsized}[t]{13cm}\normalsize\beginnumbering \toendnotes[C]{\smallbreak\pagebreak[2]} \Standort{YCGL, MSS 31.}
\physDesc{Brief, 1 Blatt, 1 Seite
\newline{}Schreibmaschine
\newline{}Handschrift: Bleistift, lateinische Kurrent (\noindent{}Anrede und Schlussformel)\newline{}Beilagen: 1) maschinschriftlicher Durchschlag: 1 Blatt, 2 Seiten  2) maschinschriftlicher Durchschlag: 1 Blatt, 2 Seiten, mit
                                 handschriftlichen Korrekturen in schwarzer Tinte\newline{}Ordnung: 1) mit Bleistift von unbekannter Hand auf der ersten Beilage die
                                 Zugehörigkeit festgehalten: »(zu
                                    21. 1. 24)«  2) mit Bleistift von unbekannter Hand auf der zweiten Beilage die
                                 Zugehörigkeit festgehalten: »Beilage zum Brief an
                                    Beer-Hofmann (21. 1. 24)« und die
                                 fehlendende Unterschrift in eckiger Klammer ergänzt: »Arthur
                                    Schnitzler«}\toendnotes[C]{\smallbreak}\pstart
           \raggedleft{}{\pb}Wien\oindex{Wien@\textbf{Wien}|pw}, 21. 1. 1924.\pend
           \pstart\center{}{[}hs.:{]} lieber Richard,\pend\pstart
           {[}ms.:{]} Beifolgend lege ich zwei Blätter bei, das eine
               enthält die Antwort des Bundestheaterkommissärs\pwindex{Renkin, Albert 11.10.1878 – 20.12.1962@\textsc{Renkin, Albert} (11.10.1878 – 20.12.1962), \emph{Leitender Beamter}|pwv} auf unser erstes an die Staatstheaterkasse\orgindex{Bundestheaterkassen@Bundestheaterkassen|pw} gerichtetes Schreiben, das zweite Blatt die Erwiderung,
               die ich dem Bundestheaterkommissär\pwindex{Renkin, Albert 11.10.1878 – 20.12.1962@\textsc{Renkin, Albert} (11.10.1878 – 20.12.1962), \emph{Leitender Beamter}|pwv} zuzusenden vorschlage. Wenn Sie sich damit
               einverstanden erklären, ersuche ich um Unterzeichnung und Rücksendung an mich.\pend
           \pstart
           {[}hs.:{]} Herzlichst{\\[\baselineskip]}Ihr{\\[\baselineskip]}\spacefill\mbox{A.}\pend
           \leftskip=0em{}{\bigskip}\pstart
           \noindent{}\raggedleft{}{\pb}Bundesministerium für Unterricht\orgindex{Ministerium fuer Unterricht@Ministerium für Unterricht|pw}\pend
           \pstart
           \noindent{}\raggedleft{}Bundestheater-Kommissär\pwindex{Renkin, Albert 11.10.1878 – 20.12.1962@\textsc{Renkin, Albert} (11.10.1878 – 20.12.1962), \emph{Leitender Beamter}|pwv}\pend
           \pstart
           \noindent{}Zahl 2831/1923\hfill 3. Jänner 1924.\pend
           \pstart
           Herrn\pend
           \leftskip=3em{}\pstart
           \noindent{}Dr. Raoul Auernheimer\pwindex{Auernheimer, Raoul 15.04.1876 – 06.01.1948@\textsc{Auernheimer, Raoul} (15.04.1876 – 06.01.1948), \emph{Schriftsteller, Journalist, Kritiker}|pw}\pend
           \leftskip=0em{}\leftskip=3em{}\pstart
           \uline{Wien\oindex{Wien@\textbf{Wien}|pw}.}\pend
           \leftskip=0em{}\pstart
           Auf die Zuschrift vom 20. Dezember 1923 beehrt sich der Bundestheater-Kommissär\pwindex{Renkin, Albert 11.10.1878 – 20.12.1962@\textsc{Renkin, Albert} (11.10.1878 – 20.12.1962), \emph{Leitender Beamter}|pwv}
               mitzuteilen, dass die gesonderte Aufstellung von Kasseneinnahmen und
               Abonnementsquoten in den Tantiemenabrechnungen der früheren Jahre auf Grund damals
               üblicher Tantiemenverträge erfolgte. Eine derartige Trennung macht jedoch dermalen
               einerseits der gegenwärtig in Verwendung stehende Tantiemenvertrag, nach welchem der
               Tantiemenberechnung einheitlich die aus den Tageseingängen und den
               Abonnementsvergütungen sich ergebende Summe zu Grunde gelegt wird, andererseits die
               gegenüber früher geänderte Art der Verrechnung der Abonnementsbeträge überflüssig,
                  {\pb}indem sie nicht mehr eine den Durchschnitt
               darstellende \so{fixe} Abonnementsquote, sondern die
               Abonnementsbeträge in ihrer vollen Höhe in die Einnahmen einbezogen werden.\pend
           \pstart
           Betreffs der Frage bezüglich der Lustbarkeitssteuer und eventueller sonstiger Abgaben
               wolle zur Kenntnis genommen werden, dass von den Gesammteinnahmen die
               Pensionszuschläge und die Lustbarkeitssteuer in Abzug gebracht und von der so
               verbleibenden Einnahmensumme die Tantiemen berechnet werden. Andere Abzüge finden
               nicht statt.\pend
           \pstart
           Der Bundestheater-Kommissär\pwindex{Renkin, Albert 11.10.1878 – 20.12.1962@\textsc{Renkin, Albert} (11.10.1878 – 20.12.1962), \emph{Leitender Beamter}|pwv}
               ersucht, die Herren Dr. Beer-Hofmann, Dr. ARTHUR Schnitzler,
               Dr. Karl Schönherr\pwindex{Schoenherr, Karl 24.02.1867 – 15.03.1943@\textsc{Schönherr, Karl} (24.02.1867 – 15.03.1943), \emph{Schriftsteller, Mediziner}|pw} und Franz Werfel\pwindex{Werfel, Franz 10.09.1890 – 26.08.1945@\textsc{Werfel, Franz} (10.09.1890 – 26.08.1945), \emph{Schriftsteller}|pw} hievon in Kenntnis zu setzen.\pend
           \pstart
           Für den Bundestheater-Kommissär\pwindex{Renkin, Albert 11.10.1878 – 20.12.1962@\textsc{Renkin, Albert} (11.10.1878 – 20.12.1962), \emph{Leitender Beamter}|pwv}: \strikeout{Dr Ernst}{\\}\spacefill\mbox{Dr. Eckmann\pwindex{Eckmann, Alfred 05.01.1880 – 25.03.1948@\textsc{Eckmann, Alfred} (05.01.1880 – 25.03.1948), \emph{Ministerialbeamter}|pw}}\pend
           {\bigskip}\pstart
           \raggedleft{}{\pb}21. 1. 1924.\pend
           \pstart
           An den\pend
           \leftskip=3em{}\pstart
           \noindent{}Bundestheater-Kommissär\pwindex{Renkin, Albert 11.10.1878 – 20.12.1962@\textsc{Renkin, Albert} (11.10.1878 – 20.12.1962), \emph{Leitender Beamter}|pwv}\pend
           \leftskip=0em{}\leftskip=3em{}\pstart
           Bundesministerium für Unterricht\orgindex{Ministerium fuer Unterricht@Ministerium für Unterricht|pw}\pend
           \leftskip=0em{}\pstart
           \noindent{}Zahl 2831/1923.\hfill \uline{Wien\oindex{Wien@\textbf{Wien}|pw}.}\pend
           \pstart
           Die Beantwortung unserer an die Staatstheaterkasse\orgindex{Bundestheaterkassen@Bundestheaterkassen|pw}
               gerichtete Anfrage bestätigen wir dankend und erlauben uns Folgendes zu bemerken.\pend
           \pstart
           Die Bestimmungen über die Tantiemenauszahlung resp. -Verrechnung erscheinen in den
               gegenwärtigen Verträgen gegenüber den früheren, die keine eigentlichen Verträge,
               sondern sogenannte Tantiemenreverse waren, kaum geändert. Doch da nach jenen früheren
               Verträgen eine fixe Abonnementsquote galt, jetzt aber, wie der Herr Bundestheaterkommissär\pwindex{Renkin, Albert 11.10.1878 – 20.12.1962@\textsc{Renkin, Albert} (11.10.1878 – 20.12.1962), \emph{Leitender Beamter}|pwv} schreibt, die
               Abonnementsbeträge in ihrer vollen Höhe in die Einnahmen einbezogen werden, so wäre
               gerade jetzt eine getrennte Aufstellung von Tageseinnahmen und Abonnementsquote
               vorzuziehen; wie ja auch früher in den Tantiemenabrechnungen für den Autor bei jeder
               Vorstellung die Tageseinnahme und die fixe Abonnementsquote getrennt figurierten.\pend
           \pstart
           Da ja auch der Burgtheaterdirektion\orgindex{Burgtheater@Burgtheater|pw} allabendlich
               eine nach Abonnementsquote und Tageseinnahme getrennte Verrechnung vorgelegt wird,
               erwächst für die Kassagebahrung nicht die geringste Schwierigkeit oder Mühe dadurch
               dass sie, wie es eben früher der Fall war, den Autoren die gleiche Verrechnung zu{\pb}gänglich machte.\pend
           \pstart
           Zu der Frage \substVorne{}\textsuperscript{der}\substDazwischen{}eines\substHinten{} Pensionsabz\substVorne{}\textsuperscript{ü}\substDazwischen{}u\substHinten{}g\substVorne{}\textsuperscript{e}\substDazwischen{}es\substHinten{} von den Tantiemen\strikeout{von den Tantiemen}, \substVorne{}\textsuperscript{die}\substDazwischen{}der\substHinten{} unseres Wissens an anderen Theatern \strikeout{von den
                  Tantiemen} nicht stattfinde\substVorne{}\textsuperscript{n}\substDazwischen{}t\substHinten{}, behalten wir uns eine Aeusserung vor, sobald wir über die Höhe der
               Pensionszuschläge und Höhe der Lustbarkeitssteuer den bereits in unserem vorigen
               Schreiben erbetenen Aufschluss erhalten haben.\pend
           
         
         \endnumbering\mylabel{h}\end{ledgroupsized}  \newcommand{\dateiname}{L02409}\newcommand{\titel}{Arthur Schnitzler an Richard Beer-Hofmann, 21. 1. 1924}\newcommand{\editorInnen}{Martin Anton Müller und Gerd-Hermann Susen}%% latex-leseansicht-abspann.tex
%% Abspann für die Leseansicht.
%% Der Schalter \ifkorrekturansicht ist bereits durch den Vorspann gesetzt.

%% latex-abspann.tex
%% Gemeinsamer Abspann für Korrekturansicht und Leseansicht.
%% Setzt den Schalter \ifkorrekturansicht voraus (gesetzt in den
%% einbindenden Dateien latex-korrekturansicht-abspann.tex bzw.
%% latex-leseansicht-abspann.tex).
%% ---------------------------------------------------------------

\normalsize

% Das esempio-Environment wird nur in der Leseansicht benötigt
\ifkorrekturansicht\else
\newenvironment{esempio}[3]%
{
    \vspace{1.5ex}
    \rlap{\underline{#1}}
    \par
    \setlength{\parindent}{0cm}
    \nopagebreak
    \leftskip=#2cm
    \rightskip=#3cm
}
{
    \par
}
\fi

\doendnotes{C}
\bigskip
\vfill

\clearpage

\footnotesize

\ifkorrekturansicht
  \lohead{\textsc{register}}
\fi

% theindex-Environment neu definieren ohne reledmac
\makeatletter
\renewenvironment{theindex}{%
  \ifkorrekturansicht
    \section*{\indexname}%
  \else
    \subsubsection*{Index der erwähnten Entitäten}%
  \fi
  \setlength{\parindent}{0pt}%
  \setlength{\parskip}{0pt plus 0.3pt}%
  \let\item\@idxitem
}{%
  \ifkorrekturansicht\clearpage\fi
}
\makeatother

\IfFileExists{\jobname-pw.ind}{\input{\jobname-pw.ind}}{}

% Quellenangabe nur in der Leseansicht
\ifkorrekturansicht\else
% Fallback-Definitionen, falls die .tex-Datei \titel etc. nicht gesetzt hat
\providecommand{\titel}{}
\providecommand{\editorInnen}{}
\providecommand{\dateiname}{\jobname}

\vspace{3cm}

\vfill

\footnotesize
\textsc{Quelle}: \titel. Herausgegeben von {\editorInnen}. In: \emph{Arthur Schnitzler: Briefwechsel mit Autorinnen und Autoren}.
 Digitale Edition, https://schnitzler-briefe.acdh.oeaw.ac.at/{\dateiname}.html (Stand \today)
\fi

\end{document}


      