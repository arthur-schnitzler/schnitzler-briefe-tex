%% latex-leseansicht-vorspann.tex
%% Vorspann für die Leseansicht.
%% Lädt die gemeinsame Datei latex-vorspann.tex mit nicht gesetztem Schalter.

\newif\ifkorrekturansicht
\korrekturansichtfalse

\input{../tex-inputs/latex-vorspann}


\section[Friedrich und Lili Waerndorfer und Hermann Bahr an Arthur Schnitzler, 12. 3. 1904]{L01381 Friedrich und Lili Waerndorfer und Hermann Bahr an Arthur Schnitzler, 12. 3. 1904}
\nopagebreak\mylabel{L01381v}
\rehead{ }\normalsize\beginnumbering\briefempfaengerindex{Schnitzler, Arthur@\textsc{Schnitzler, Arthur}!zzzBahr, Hermann@\emph{von Hermann Bahr}!1904-03-121@{12. 3. 1904}|(be}\briefempfaengerindex{Schnitzler, Arthur@\textsc{Schnitzler, Arthur}!zzzWaerndorfer, Lili@\emph{von Lili Waerndorfer}!1904-03-121@{12. 3. 1904}|(be}\briefempfaengerindex{Schnitzler, Arthur@\textsc{Schnitzler, Arthur}!zzzWärndorfer, Friedrich@\emph{von Friedrich Wärndorfer}!1904-03-121@{12. 3. 1904}|(be}
\toendnotes[C]{\smallbreak\pagebreak[2]}
\correspDesc{Versand  durch Friedrich Waerndorfer, Lili Waerndorfer, Hermann Bahr am 12. 3. 1904 in Dubrovnik
\newline{}Erhalt  durch Arthur Schnitzler am 15. 3. 04 in Wien}\toendnotes[C]{\smallbreak}
\Standort{CUL, Schnitzler, B 5b.}
\physDesc{Bildpostkarte, 108 Zeichen
\newline{}Handschrift Hermann Bahr: schwarze Tinte, deutsche Kurrent
\newline{}Handschrift Lili Waerndorfer: schwarze Tinte
\newline{}Handschrift Friedrich Wärndorfer: schwarze Tinte, lateinische Kurrent
\newline{}Versand: 1) Stempel: »\nobreak{}\oindex{Dubrovnik@\textbf{Dubrovnik}|pwk}Dubrovnik Ragusa, 12. 3{[}.{]} 04\nobreak{}«.   2) Stempel: »\nobreak{}Bestellt, \oindex{Wien@\textbf{Wien}, \emph{Verwaltungsgebiet}|pwk}Wien, 15. 3. 04, V\nobreak{}«. 
\newline{}Schnitzler: mit Bleistift datiert: »15. 3. 901« 
\newline{}Ordnung: mit Bleistift von unbekannter Hand nummeriert:
                                    »113a« }
\buchAbdrucke{\weitereDrucke{Hermann Bahr, Arthur Schnitzler: \emph{Briefwechsel, Aufzeichnungen, Dokumente (1891–1931)}. Herausgegeben von Kurt Ifkovits und Martin Anton Müller. Göttingen: \emph{Wallstein} 2018, S. 305.} }\pstart{}{\pb}Dr. ARTHUR\pend{}\pstart{}SCHNITZLER\pend{}\pstart{}XVIII SPÖTTELGASSE 7\oindex{Wien@\textbf{Wien}!XVIII., Währing@\textbf{XVIII., Währing}!Edmund-Weiß-Gasse 7@\textbf{Edmund-Weiß-Gasse 7}, \emph{Wohngebäude}|pw}\pend{}\pstart{}WIEN\oindex{Wien@\textbf{Wien}, \emph{Verwaltungsgebiet}|pw}\pend{}{\bigskip}
\pstart
           \noindent{}\centering{}{\pb}\textcolor{gray}{\textbf{L’Ombla – Cempresata\oindex{Ombla@\textbf{Ombla}, \emph{Fluss}|pw}}}\pend
           \vspace{1em}
\pstart
           \noindent{}\spacefill\mbox{{\pb}{[}hs. Waerndorfer:{]} Lili +}\pend
           
\pstart
           {[}hs. Wärndorfer:{]} Dein \spacefill\mbox{FWaerndorfer}\pend
           \selectlanguage{ngerman}\vspace{1em}
\pstart
           \noindent{}{[}hs. Bahr:{]} Herzlichſt\pend
           
\pstart
           Deine Frau beſtens grüßend{\\[\baselineskip]}\spacefill\mbox{Hermann}\pend
           \leftskip=0em{}\selectlanguage{ngerman}\endnumbering\briefempfaengerindex{Schnitzler, Arthur@\textsc{Schnitzler, Arthur}!zzzBahr, Hermann@\emph{von Hermann Bahr}!1904-03-121@{12. 3. 1904}|)be}\briefempfaengerindex{Schnitzler, Arthur@\textsc{Schnitzler, Arthur}!zzzWaerndorfer, Lili@\emph{von Lili Waerndorfer}!1904-03-121@{12. 3. 1904}|)be}\briefempfaengerindex{Schnitzler, Arthur@\textsc{Schnitzler, Arthur}!zzzWärndorfer, Friedrich@\emph{von Friedrich Wärndorfer}!1904-03-121@{12. 3. 1904}|)be}\mylabel{L01381h}  \newcommand{\dateiname}{L01381}\newcommand{\titel}{Friedrich und Lili Waerndorfer und Hermann Bahr an Arthur Schnitzler, 12. 3. 1904}\newcommand{\editorInnen}{Herausgegeben von Martin Anton Müller}%% latex-leseansicht-abspann.tex
%% Abspann für die Leseansicht.
%% Der Schalter \ifkorrekturansicht ist bereits durch den Vorspann gesetzt.

%% latex-abspann.tex
%% Gemeinsamer Abspann für Korrekturansicht und Leseansicht.
%% Setzt den Schalter \ifkorrekturansicht voraus (gesetzt in den
%% einbindenden Dateien latex-korrekturansicht-abspann.tex bzw.
%% latex-leseansicht-abspann.tex).
%% ---------------------------------------------------------------

\normalsize

% Das esempio-Environment wird nur in der Leseansicht benötigt
\ifkorrekturansicht\else
\newenvironment{esempio}[3]%
{
    \vspace{1.5ex}
    \rlap{\underline{#1}}
    \par
    \setlength{\parindent}{0cm}
    \nopagebreak
    \leftskip=#2cm
    \rightskip=#3cm
}
{
    \par
}
\fi

\doendnotes{C}
\bigskip
\vfill

\clearpage

\footnotesize

\ifkorrekturansicht
  \lohead{\textsc{register}}
\fi

% theindex-Environment neu definieren ohne reledmac
\makeatletter
\renewenvironment{theindex}{%
  \ifkorrekturansicht
    \section*{\indexname}%
  \else
    \subsubsection*{Index der erwähnten Entitäten}%
  \fi
  \setlength{\parindent}{0pt}%
  \setlength{\parskip}{0pt plus 0.3pt}%
  \let\item\@idxitem
}{%
  \ifkorrekturansicht\clearpage\fi
}
\makeatother

\IfFileExists{\jobname-pw.ind}{\input{\jobname-pw.ind}}{}

% Quellenangabe nur in der Leseansicht
\ifkorrekturansicht\else
% Fallback-Definitionen, falls die .tex-Datei \titel etc. nicht gesetzt hat
\providecommand{\titel}{}
\providecommand{\editorInnen}{}
\providecommand{\dateiname}{\jobname}

\vspace{3cm}

\vfill

\footnotesize
\textsc{Quelle}: \titel. Herausgegeben von {\editorInnen}. In: \emph{Arthur Schnitzler: Briefwechsel mit Autorinnen und Autoren}.
 Digitale Edition, https://schnitzler-briefe.acdh.oeaw.ac.at/{\dateiname}.html (Stand \today)
\fi

\end{document}


