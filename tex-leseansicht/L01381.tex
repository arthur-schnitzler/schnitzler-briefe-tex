%% latex-korrekturansicht-vorspann.tex
%% Vorspann für die Korrekturansicht.
%% Lädt die gemeinsame Datei latex-vorspann.tex mit gesetztem Schalter.

\newif\ifkorrekturansicht
\korrekturansichttrue

\input{../tex-inputs/latex-vorspann}


\section[Friedrich und Lili Waerndorfer und Hermann Bahr an Arthur Schnitzler, 12. 3. 1904]{L01381 Friedrich und Lili Waerndorfer und Hermann Bahr an Arthur Schnitzler,
               12. 3. 1904}
\nopagebreak\mylabel{L01381v}
\rehead{ }\normalsize\beginnumbering\briefempfaengerindex{Schnitzler, Arthur@\textsc{Schnitzler, Arthur}!zzzBahr, Hermann@\emph{von Hermann Bahr}!1904-03-121@{12. 3. 1904}|(be}\briefempfaengerindex{Schnitzler, Arthur@\textsc{Schnitzler, Arthur}!zzzWaerndorfer, Lili@\emph{von Lili Waerndorfer}!1904-03-121@{12. 3. 1904}|(be}\briefempfaengerindex{Schnitzler, Arthur@\textsc{Schnitzler, Arthur}!zzzWaerndorfer, Friedrich@\emph{von Friedrich Wärndorfer}!1904-03-121@{12. 3. 1904}|(be}
\toendnotes[C]{\smallbreak\pagebreak[2]}\Standort{CUL, Schnitzler, B 5b.}
\physDesc{Bildpostkarte, 108 Zeichen
\newline{}Handschrift Hermann Bahr: schwarze Tinte, deutsche Kurrent
\newline{}Handschrift Lili Waerndorfer: schwarze Tinte
\newline{}Handschrift Friedrich Wärndorfer: schwarze Tinte, lateinische Kurrent
\newline{}Versand: 1) Stempel: »\nobreak{}\oindex{Dubrovnik@\textbf{Dubrovnik}, \emph{P.PPLA}|pwk}Dubrovnik Ragusa, 12. 3{[}.{]} 04\nobreak{}«.   2) Stempel: »\nobreak{}Bestellt, Wien, 15. 3. 04, V\nobreak{}«. 
\newline{}Schnitzler: mit Bleistift datiert: »15. 3. 901« 
\newline{}Ordnung: mit Bleistift von unbekannter Hand nummeriert:
                                    »113a« }
\buchAbdrucke{\weitereDrucke{Hermann Bahr, Arthur Schnitzler: \emph{Briefwechsel, Aufzeichnungen, Dokumente (1891–1931)}. Göttingen: \emph{Wallstein} 2018, S. 305.} }\pstart{}{\pb}Dr. ARTHUR\pend{}\pstart{}SCHNITZLER\pend{}\pstart{}XVIII SPÖTTELGASSE 7\oindex{Edmund-Weiss-Gasse 7@\textbf{Edmund-Weiß-Gasse 7}, \emph{Wohngebäude (K.WHS)}|pw}\pend{}\pstart{}WIEN\oindex{Wien@\textbf{Wien}, \emph{A.ADM2}|pw}\pend{}{\bigskip}
\pstart
           \noindent{}\centering{}{\pb}\textcolor{gray}{\textbf{L’Ombla – Cempresata\oindex{Ombla@\textbf{Ombla}, \emph{Fluss (N.FLS)}|pw}}}\pend
           \vspace{1em}
\pstart
           \noindent{}\spacefill\mbox{{\pb}{[}hs. :{]} Lili +}\pend
           
\pstart
           {[}hs. :{]} Dein \spacefill\mbox{FWaerndorfer}\pend
           \selectlanguage{ngerman}\vspace{1em}
\pstart
           \noindent{}{[}hs. :{]} Herzlichſt\pend
           
\pstart
           Deine Frau beſtens grüßend{\\[\baselineskip]}\spacefill\mbox{Hermann}\pend
           \leftskip=0em{}\selectlanguage{ngerman}\endnumbering\briefempfaengerindex{Schnitzler, Arthur@\textsc{Schnitzler, Arthur}!zzzBahr, Hermann@\emph{von Hermann Bahr}!1904-03-121@{12. 3. 1904}|)be}\briefempfaengerindex{Schnitzler, Arthur@\textsc{Schnitzler, Arthur}!zzzWaerndorfer, Lili@\emph{von Lili Waerndorfer}!1904-03-121@{12. 3. 1904}|)be}\briefempfaengerindex{Schnitzler, Arthur@\textsc{Schnitzler, Arthur}!zzzWaerndorfer, Friedrich@\emph{von Friedrich Wärndorfer}!1904-03-121@{12. 3. 1904}|)be}\mylabel{L01381h}  \normalsize

\doendnotes{C}
\bigskip
\vfill

\clearpage

\footnotesize

\lohead{\textsc{register}}

% Definiere theindex-Environment komplett neu ohne reledmac
\makeatletter
\renewenvironment{theindex}{%
  \section*{\indexname}%
  \setlength{\parindent}{0pt}%
  \setlength{\parskip}{0pt plus 0.3pt}%
  \let\item\@idxitem
}{%
  \clearpage
}
\makeatother

\IfFileExists{\jobname-pw.ind}{\input{\jobname-pw.ind}}{}

\end{document}

      