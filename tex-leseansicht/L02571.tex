%% latex-leseansicht-vorspann.tex
%% Vorspann für die Leseansicht.
%% Lädt die gemeinsame Datei latex-vorspann.tex mit nicht gesetztem Schalter.

\newif\ifkorrekturansicht
\korrekturansichtfalse

\input{../tex-inputs/latex-vorspann}


               \section[Therese Rie-Andro an Arthur Schnitzler, 27. 1. 1913]{ Therese Rie-Andro an Arthur Schnitzler, 27. 1. 1913}\nopagebreak\mylabel{v}\rehead{ }\begin{ledgroupsized}[t]{13cm}\normalsize\beginnumbering\briefempfaengerindex{Schnitzler, Arthur@\textsc{Schnitzler, Arthur}!zzzRie, Therese@\emph{von Therese Rie}!1913-01-271@{27. 1. 1913}|(be} \toendnotes[C]{\smallbreak\pagebreak[2]} \Standort{DLA, A:Schnitzler, 85.1.4310.}
\physDesc{Brief, 1 Blatt, 2 Seiten
\newline{}Handschrift: blaue Tinte, lateinische Kurrent
\newline{}Schnitzler: 1) mit Bleistift beschriftet: »\textsc{Andro}« 2) mit rotem Buntstift zwei Unterstreichungen}\toendnotes[C]{\smallbreak}\pstart
           \raggedleft{}{\pb}Wien\oindex{Wien@\textbf{Wien}|pw}, d. 27. Januar 13.\pend
           \pstart
           \raggedleft{}IV, Schönburgſtr. 48\oindex{Schoenburgstrasse@\textbf{Schönburgstraße}|pw}\pend
           \pstart{}Verehrter Herr Doktor,\pend\pstart
           Auf der Rückreise von Berlin\oindex{Berlin@\textbf{Berlin}|pw} las ich den »Weg ins Freie\pwindex{Schnitzler, Arthur 15.05.1862 – 21.10.1931@\textsc{Schnitzler, Arthur} (15.05.1862 – 21.10.1931), \emph{Schriftsteller, Mediziner}!Weg ins Freie. Roman1.1.1908 – 1.6.1908@\strich\emph{Der Weg ins Freie. Roman} {[}1.1.1908 – 1.6.1908{]}|pw}« so ungefähr zum sechsten Mal und wie
               jedesmal bei diesem merkwürdig reichen Buche fielen mir eine Menge neue, nicht
               erfaßte Dinge auf, diesmal besonders im letzten Teil. Dabei stieß ich auch auf eine
               kleine Bemerkung über Melot\pwindex{\textcolor{red}{\textsuperscript{XXXX1 indx}}!Tristan und Isolde1865@\strich\emph{Tristan und Isolde} {[}1865{]}|pwv}, den
               von einem zweiten Sänger \substVorne{}\textsuperscript{D}\substDazwischen{}d\substHinten{}argestellt zu sehen \label{K_L02571-2v}\edtext{Georg\pwindex{Schnitzler, Arthur 15.05.1862 – 21.10.1931@\textsc{Schnitzler, Arthur} (15.05.1862 – 21.10.1931), \emph{Schriftsteller, Mediziner}!Weg ins Freie. Roman1.1.1908 – 1.6.1908@\strich\emph{Der Weg ins Freie. Roman} {[}1.1.1908 – 1.6.1908{]}|pwv} sich ärgert}{\lemma{\textnormal{\emph{Georg sich ärgert}}}\Cendnote{\textnormal{»und gar nicht einverstanden war
                        er damit, daß Melot, durch dessen Hand Tristan sterben mußte, hier von einem
                        Sänger zweiten Ranges dargestellt wurde, wie übrigens beinahe überall.\pwindex{Schnitzler, Arthur 15.05.1862 – 21.10.1931@\textsc{Schnitzler, Arthur} (15.05.1862 – 21.10.1931), \emph{Schriftsteller, Mediziner}!Weg ins Freie. Roman1.1.1908 – 1.6.1908@\strich\emph{Der Weg ins Freie. Roman} {[}1.1.1908 – 1.6.1908{]}|pwv}« (neuntes Kapitel).}}}\label{K_L02571-2h}. Da fiel mir ein, daß Sie sich
               für Pfitzner\pwindex{Pfitzner, Hans 05.05.1869 – 22.05.1949@\textsc{Pfitzner, Hans} (05.05.1869 – 22.05.1949), \emph{Komponist}|pw} interessieren und daß von ihm ein
               feiner geistvoller Aufsatz\pwindex{Pfitzner, Hans 05.05.1869 – 22.05.1949@\textsc{Pfitzner, Hans} (05.05.1869 – 22.05.1949), \emph{Komponist}!Buehnentradition1906-01-01@\strich\emph{Bühnentradition} {[}1906-01-01{]}|pwv}
               existiert, der ausführlich das begründet, was Sie \introOben{}in ganz
                  ähnlicher Auffassung\introOben{} in einem Satze andeuten. Ich grabe ihn also aus meinem
               Bücherschrank aus und schicke ihn an Sie – vielleicht kennen Sie ihn nicht und es
               macht {\pb}Ihnen Vergnügen, ihn zu lesen.\pend
           \pstart
           Vom Palestrina\pwindex{Pfitzner, Hans 05.05.1869 – 22.05.1949@\textsc{Pfitzner, Hans} (05.05.1869 – 22.05.1949), \emph{Komponist}!Palestrina. Musikalische Legende in drei Akten1912@\strich\emph{Palestrina. Musikalische Legende in drei Akten} {[}1912{]}|pw} weiß ich seit diesem So{\geminationm}er, wo ich Pf.\pwindex{Pfitzner, Hans 05.05.1869 – 22.05.1949@\textsc{Pfitzner, Hans} (05.05.1869 – 22.05.1949), \emph{Komponist}|pw} in
                  Leipzig\oindex{Leipzig@\textbf{Leipzig}|pw} traf, nicht mehr viel, außer daß der
               1. Akt auch musikalisch fertig iſt. Weiter wird er wol inzwischen auch nicht geko{\geminationm}en sein, da er ja leider als Operndirektor\orgindex{Oper Strassburg@Oper Straßburg|pwv} tätig iſt – leider, da wir ja
               nichts davon haben; für die Straßburg\oindex{Strassburg@\textbf{Straßburg}|pw}er mag’s ja
               ganz hübsch sein.\pend
           \pstart
           Noch will ich Sie von zweien Ihrer Werke grüßen: vom »Professor Bernhardi\pwindex{Schnitzler, Arthur 15.05.1862 – 21.10.1931@\textsc{Schnitzler, Arthur} (15.05.1862 – 21.10.1931), \emph{Schriftsteller, Mediziner}!Professor Bernhardi. Komoedie in fuenf Akten1912@\strich\emph{Professor Bernhardi. Komödie in fünf Akten} {[}1912{]}|pw}«, von dem ich durch einen Zufall aber nur die erſten
               zwei Akte hörte; und vom »Schleier der Pierrette\pwindex{Schnitzler, Arthur 15.05.1862 – 21.10.1931@\textsc{Schnitzler, Arthur} (15.05.1862 – 21.10.1931), \emph{Schriftsteller, Mediziner}!Schleier der Pierrette1910-01-22@\strich\emph{Der Schleier der Pierrette} {[}1910-01-22{]}|pw}«,
               den ich in Dresden\oindex{Dresden@\textbf{Dresden}|pw}, bei der \label{K_L02571-1v}\edtext{Generalprobe}{\lemma{\textnormal{\emph{Generalprobe}}}\Cendnote{\textnormal{\emph{Tante Simona}\pwindex{Dohnányi, Ernst von 27.07.1877 – 09.02.1960@\textsc{Dohnányi, Ernst von} (27.07.1877 – 09.02.1960), \emph{Komponist, Pianist}!Tante Simona. Komische Oper in einem Akt1913-01-22@\strich\emph{Tante Simona. Komische Oper in einem Akt} {[}Vertonung, 1913-01-22{]}|pwk} hatte am 22. 1. 1913
                  Uraufführung und wurde gemeinsam mit \emph{Schleier der
                     Pierrette}\pwindex{Schnitzler, Arthur 15.05.1862 – 21.10.1931@\textsc{Schnitzler, Arthur} (15.05.1862 – 21.10.1931), \emph{Schriftsteller, Mediziner}!Schleier der Pierrette1910-01-22@\strich\emph{Der Schleier der Pierrette} {[}1910-01-22{]}|pwk} gegeben. Entsprechend ist die Generalprobe einen oder zwei Tage
                  davor anzusetzen.}}}\label{K_L02571-1h} von Dóhnanyi\pwindex{Dohnányi, Ernst von 27.07.1877 – 09.02.1960@\textsc{Dohnányi, Ernst von} (27.07.1877 – 09.02.1960), \emph{Komponist, Pianist}|pw}s neuer
                  Oper\pwindex{Dohnányi, Ernst von 27.07.1877 – 09.02.1960@\textsc{Dohnányi, Ernst von} (27.07.1877 – 09.02.1960), \emph{Komponist, Pianist}!Tante Simona. Komische Oper in einem Akt1913-01-22@\strich\emph{Tante Simona. Komische Oper in einem Akt} {[}Vertonung, 1913-01-22{]}|pwv} zu sehen bekam.\pend
           \pstart
           In alter herzlicher Bewunderung{\\[\baselineskip]}\spacefill\mbox{L. Andro.}{\\[\baselineskip]}(Therese Rie.) \pend
           \leftskip=0em{}\endnumbering\briefempfaengerindex{Schnitzler, Arthur@\textsc{Schnitzler, Arthur}!zzzRie, Therese@\emph{von Therese Rie}!1913-01-271@{27. 1. 1913}|)be}\mylabel{h}\end{ledgroupsized}  \newcommand{\dateiname}{L02571}\newcommand{\titel}{Therese Rie-Andro an Arthur Schnitzler, 27. 1. 1913}\newcommand{\editorInnen}{Martin Anton Müller und Gerd-Hermann Susen}%% latex-leseansicht-abspann.tex
%% Abspann für die Leseansicht.
%% Der Schalter \ifkorrekturansicht ist bereits durch den Vorspann gesetzt.

%% latex-abspann.tex
%% Gemeinsamer Abspann für Korrekturansicht und Leseansicht.
%% Setzt den Schalter \ifkorrekturansicht voraus (gesetzt in den
%% einbindenden Dateien latex-korrekturansicht-abspann.tex bzw.
%% latex-leseansicht-abspann.tex).
%% ---------------------------------------------------------------

\normalsize

% Das esempio-Environment wird nur in der Leseansicht benötigt
\ifkorrekturansicht\else
\newenvironment{esempio}[3]%
{
    \vspace{1.5ex}
    \rlap{\underline{#1}}
    \par
    \setlength{\parindent}{0cm}
    \nopagebreak
    \leftskip=#2cm
    \rightskip=#3cm
}
{
    \par
}
\fi

\doendnotes{C}
\bigskip
\vfill

\clearpage

\footnotesize

\ifkorrekturansicht
  \lohead{\textsc{register}}
\fi

% theindex-Environment neu definieren ohne reledmac
\makeatletter
\renewenvironment{theindex}{%
  \ifkorrekturansicht
    \section*{\indexname}%
  \else
    \subsubsection*{Index der erwähnten Entitäten}%
  \fi
  \setlength{\parindent}{0pt}%
  \setlength{\parskip}{0pt plus 0.3pt}%
  \let\item\@idxitem
}{%
  \ifkorrekturansicht\clearpage\fi
}
\makeatother

\IfFileExists{\jobname-pw.ind}{\input{\jobname-pw.ind}}{}

% Quellenangabe nur in der Leseansicht
\ifkorrekturansicht\else
% Fallback-Definitionen, falls die .tex-Datei \titel etc. nicht gesetzt hat
\providecommand{\titel}{}
\providecommand{\editorInnen}{}
\providecommand{\dateiname}{\jobname}

\vspace{3cm}

\vfill

\footnotesize
\textsc{Quelle}: \titel. Herausgegeben von {\editorInnen}. In: \emph{Arthur Schnitzler: Briefwechsel mit Autorinnen und Autoren}.
 Digitale Edition, https://schnitzler-briefe.acdh.oeaw.ac.at/{\dateiname}.html (Stand \today)
\fi

\end{document}


      