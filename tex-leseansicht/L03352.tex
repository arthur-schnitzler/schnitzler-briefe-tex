%% latex-leseansicht-vorspann.tex
%% Vorspann für die Leseansicht.
%% Lädt die gemeinsame Datei latex-vorspann.tex mit nicht gesetztem Schalter.

\newif\ifkorrekturansicht
\korrekturansichtfalse

\input{../tex-inputs/latex-vorspann}


\section[ Felix Salten an Arthur Schnitzler, 27. 11. 1903]{L03352 Felix Salten an Arthur Schnitzler,  27. 11. 1903}
\nopagebreak\mylabel{L03352v}
\rehead{ }\normalsize\beginnumbering\briefempfaengerindex{Schnitzler, Arthur@\textsc{Schnitzler, Arthur}!zzzSalten, Felix@\emph{von Felix Salten}!1903-11-271@{27. 11. 1903}|(be}
\toendnotes[C]{\smallbreak\pagebreak[2]}
\correspDesc{Versand  durch Felix Salten am 27. 11. 1903 in Wien
\newline{}Erhalt  durch Arthur Schnitzler im Zeitraum [27. 11. 1903 – 29. 11. 1903?] in Wien}\toendnotes[C]{\smallbreak}
\Standort{CUL, Schnitzler, B 89, A 2.}
\physDesc{Brief, 1 Blatt, 1 Seite, 362 Zeichen
\newline{}Handschrift: Bleistift, lateinische Kurrent
\newline{}Ordnung: mit Bleistift von unbekannter Hand nummeriert: »178« }\toendnotes[C]{\smallbreak}
\pstart
           \raggedleft{}{\pb}Am Kahlenberg\oindex{Wien@\textbf{Wien}!XIX., Döbling@\textbf{XIX., Döbling}!Kahlenberg@\textbf{Kahlenberg}, \emph{Berg}|pw}, 27 XI. 03\pend
           \vspace{0.5em}
\pstart
           Lieber, ich bin doch nicht nach Waidhofen\oindex{Waidhofen an der Ybbs@\textbf{Waidhofen an der Ybbs}, \emph{Hauptstadt}|pw} sondern lieber hier herauf, wo es wunderschön und ganz still ist.
               Gedenke mir diesen Berg\oindex{Wien@\textbf{Wien}!XIX., Döbling@\textbf{XIX., Döbling}!Kahlenberg@\textbf{Kahlenberg}, \emph{Berg}|pwv} jetzt
               als meinen Privat-Semmering\oindex{Semmering@\textbf{Semmering}, \emph{Verwaltungsgebiet}|pw} anzuschaffen. Herzl.
               Dank für Ihre Wolmeinung über meinen \label{K_L03352-1v}\edtext{Klimt\pwindex{Klimt, Gustav 14.\,7.\,1862 Wien – 6.\,2.\,1918 ebd.@\textsc{Klimt, Gustav} (14.\,7.\,1862 Wien – 6.\,2.\,1918 ebd.), \emph{Maler}|pw}-Aufsatz\pwindex{Salten, Felix 6.\,9.\,1869 Budapest – 8.\,10.\,1945 Zürich@\textsc{Salten, Felix} (6.\,9.\,1869 Budapest – 8.\,10.\,1945 Zürich), \emph{Schriftsteller, Journalist, Chefredakteur}!Gustav Klimt. Gelegentliche Anmerkungen@\strich\emph{Gustav Klimt. Gelegentliche Anmerkungen}|pwv}}{\lemma{\textnormal{\emph{Klimt-Aufsatz}}}\Cendnote{\textnormal{Felix Salten\pwindex{Salten, Felix 6.\,9.\,1869 Budapest – 8.\,10.\,1945 Zürich@\textsc{Salten, Felix} (6.\,9.\,1869 Budapest – 8.\,10.\,1945 Zürich), \emph{Schriftsteller, Journalist, Chefredakteur}|pwk}: \emph{Gustav Klimt. Gelegentliche Anmerkungen}\pwindex{Salten, Felix 6.\,9.\,1869 Budapest – 8.\,10.\,1945 Zürich@\textsc{Salten, Felix} (6.\,9.\,1869 Budapest – 8.\,10.\,1945 Zürich), \emph{Schriftsteller, Journalist, Chefredakteur}!Gustav Klimt. Gelegentliche Anmerkungen@\strich\emph{Gustav Klimt. Gelegentliche Anmerkungen}|pwk}. Buchschmuck
                     von Bertold Löffler\pwindex{Löffler, Bertold 28.\,9.\,1874 Liberec – 23.\,3.\,1960 Wien@\textsc{Löffler, Bertold} (28.\,9.\,1874 Liberec – 23.\,3.\,1960 Wien), \emph{Maler, Grafiker}|pwk}. Wien\oindex{Wien@\textbf{Wien}, \emph{Verwaltungsgebiet}|pwk}, Leipzig\oindex{Leipzig@\textbf{Leipzig}, \emph{Hauptstadt}|pwk}: \emph{Wiener Verlag}\orgindex{Wiener Verlag@Wiener Verlag|pwk}{ }1903.}}}\label{K_L03352-1}. \label{K_L03352-2v}\edtext{Nächstens ziehe
               ich mich hierher mit Schlenther\pwindex{Schlenther, Paul 20.\,8.\,1854 Chernyakhovsk – 30.\,4.\,1916 Berlin@\textsc{Schlenther, Paul} (20.\,8.\,1854 Chernyakhovsk – 30.\,4.\,1916 Berlin), \emph{Schriftsteller, Kritiker, Theaterleiter}|pw} zurück}{\lemma{\textnormal{\emph{Nächstens … zurück}}}\Cendnote{\textnormal{nicht nachgewiesen; an Ostern 1904
                  plante er ebenfalls am Kahlenberg\oindex{Wien@\textbf{Wien}!XIX., Döbling@\textbf{XIX., Döbling}!Kahlenberg@\textbf{Kahlenberg}, \emph{Berg}|pwk} Tage zu verbringen, vgl. XXXX Auszeichnungsfehler: Dokument L03394 nicht gefunden.}}}\label{K_L03352-2}, und hoffe Sie
               noch besser zu bedienen.\pend
           \pstart herzlichst Ihr \spacefill\mbox{S.}\pend{}\selectlanguage{ngerman}\endnumbering\briefempfaengerindex{Schnitzler, Arthur@\textsc{Schnitzler, Arthur}!zzzSalten, Felix@\emph{von Felix Salten}!1903-11-271@{27. 11. 1903}|)be}\mylabel{L03352h}  \newcommand{\dateiname}{L03352}\newcommand{\titel}{Felix Salten an Arthur Schnitzler, 27. 11. 1903}\newcommand{\editorInnen}{Martin Anton Müller und Laura Untner}%% latex-leseansicht-abspann.tex
%% Abspann für die Leseansicht.
%% Der Schalter \ifkorrekturansicht ist bereits durch den Vorspann gesetzt.

%% latex-abspann.tex
%% Gemeinsamer Abspann für Korrekturansicht und Leseansicht.
%% Setzt den Schalter \ifkorrekturansicht voraus (gesetzt in den
%% einbindenden Dateien latex-korrekturansicht-abspann.tex bzw.
%% latex-leseansicht-abspann.tex).
%% ---------------------------------------------------------------

\normalsize

% Das esempio-Environment wird nur in der Leseansicht benötigt
\ifkorrekturansicht\else
\newenvironment{esempio}[3]%
{
    \vspace{1.5ex}
    \rlap{\underline{#1}}
    \par
    \setlength{\parindent}{0cm}
    \nopagebreak
    \leftskip=#2cm
    \rightskip=#3cm
}
{
    \par
}
\fi

\doendnotes{C}
\bigskip
\vfill

\clearpage

\footnotesize

\ifkorrekturansicht
  \lohead{\textsc{register}}
\fi

% theindex-Environment neu definieren ohne reledmac
\makeatletter
\renewenvironment{theindex}{%
  \ifkorrekturansicht
    \section*{\indexname}%
  \else
    \subsubsection*{Index der erwähnten Entitäten}%
  \fi
  \setlength{\parindent}{0pt}%
  \setlength{\parskip}{0pt plus 0.3pt}%
  \let\item\@idxitem
}{%
  \ifkorrekturansicht\clearpage\fi
}
\makeatother

\IfFileExists{\jobname-pw.ind}{\input{\jobname-pw.ind}}{}

% Quellenangabe nur in der Leseansicht
\ifkorrekturansicht\else
% Fallback-Definitionen, falls die .tex-Datei \titel etc. nicht gesetzt hat
\providecommand{\titel}{}
\providecommand{\editorInnen}{}
\providecommand{\dateiname}{\jobname}

\vspace{3cm}

\vfill

\footnotesize
\textsc{Quelle}: \titel. Herausgegeben von {\editorInnen}. In: \emph{Arthur Schnitzler: Briefwechsel mit Autorinnen und Autoren}.
 Digitale Edition, https://schnitzler-briefe.acdh.oeaw.ac.at/{\dateiname}.html (Stand \today)
\fi

\end{document}


