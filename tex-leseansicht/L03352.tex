%% latex-leseansicht-vorspann.tex
%% Vorspann für die Leseansicht.
%% Lädt die gemeinsame Datei latex-vorspann.tex mit nicht gesetztem Schalter.

\newif\ifkorrekturansicht
\korrekturansichtfalse

\input{../tex-inputs/latex-vorspann}

\begin{center}
            \textcolor{red}{ENTWURF, NICHT FERTIG KORRIGIERT}
                      \end{center}
            
         
         \renewcommand{\erwaehntePersonen}{Personen: Gustav Klimt, Paul Schlenther}
         \renewcommand{\erwaehnteOrte}{Orte: Kahlenberg, Semmering, Waidhofen an der Ybbs, Wien}
         \renewcommand{\erwaehnteWerke}{Werke: Gustav Klimt. Gelegentliche Anmerkungen}
               \section[Felix Salten an Arthur Schnitzler, 27. 11. 1903]{ Felix Salten an Arthur Schnitzler, 27. 11. 1903}\nopagebreak\mylabel{v}\rehead{ }\begin{ledgroupsized}[t]{13cm}\normalsize\beginnumbering \toendnotes[C]{\smallbreak\pagebreak[2]} \Standort{CUL, Schnitzler, B 89, A 2.}
\physDesc{Brief, 1 Blatt, 1 Seite
\newline{}Handschrift: Bleistift, lateinische Kurrent\newline{}Ordnung: mit Bleistift von unbekannter Hand nummeriert:
                                    »178« }\toendnotes[C]{\smallbreak}\pstart
           \raggedleft{}{\pb}Am Kahlenberg\oindex{Kahlenberg@\textbf{Kahlenberg}|pw}, 27 XI. 03\pend
           \pstart
           Lieber, ich bin doch nicht nach Waidhofen\oindex{Waidhofen an der Ybbs@\textbf{Waidhofen an der Ybbs}|pw}, sondern lieber hier herauf, wo es wunderschön und ganz still ist.
               Gedenke mir diesen Berg jetzt als meinen Privat-Semmering\oindex{Semmering@\textbf{Semmering}|pw} anzuschaffen. Herzl. Dank für Ihre Wolmeinung über meinen Klimt\pwindex{Klimt, Gustav 14.07.1862 – 06.02.1918@\textsc{Klimt, Gustav} (14.07.1862 – 06.02.1918), \emph{Bildender Künstler}|pw}-Aufsatz\pwindex{Salten, Felix 06.09.1869 – 08.10.1945@\textsc{Salten, Felix} (06.09.1869 – 08.10.1945), \emph{Schriftsteller, Journalist}!Gustav Klimt. Gelegentliche Anmerkungen1903-11-21@\strich\emph{Gustav Klimt. Gelegentliche Anmerkungen} {[}1903-11-21{]}|pwv}. Nächstens ziehe ich
               mich hierher mit Schlenther\pwindex{Schlenther, Paul 20.08.1854 – 30.04.1916@\textsc{Schlenther, Paul} (20.08.1854 – 30.04.1916), \emph{Schriftsteller, Kritiker, Theaterleiter}|pw} zurück, und hoffe
               Sie noch besser zu bedienen, \pend
           \pstart herzlichst Ihr \spacefill\mbox{S.}\pend{}
         
         \endnumbering\mylabel{h}\end{ledgroupsized}\begin{anhang}\end{anhang}\newcommand{\dateiname}{L03352}\newcommand{\titel}{Felix Salten an Arthur Schnitzler, 27. 11. 1903}\newcommand{\editorInnen}{Martin Anton Müller und Laura Untner}%% latex-leseansicht-abspann.tex
%% Abspann für die Leseansicht.
%% Der Schalter \ifkorrekturansicht ist bereits durch den Vorspann gesetzt.

%% latex-abspann.tex
%% Gemeinsamer Abspann für Korrekturansicht und Leseansicht.
%% Setzt den Schalter \ifkorrekturansicht voraus (gesetzt in den
%% einbindenden Dateien latex-korrekturansicht-abspann.tex bzw.
%% latex-leseansicht-abspann.tex).
%% ---------------------------------------------------------------

\normalsize

% Das esempio-Environment wird nur in der Leseansicht benötigt
\ifkorrekturansicht\else
\newenvironment{esempio}[3]%
{
    \vspace{1.5ex}
    \rlap{\underline{#1}}
    \par
    \setlength{\parindent}{0cm}
    \nopagebreak
    \leftskip=#2cm
    \rightskip=#3cm
}
{
    \par
}
\fi

\doendnotes{C}
\bigskip
\vfill

\clearpage

\footnotesize

\ifkorrekturansicht
  \lohead{\textsc{register}}
\fi

% theindex-Environment neu definieren ohne reledmac
\makeatletter
\renewenvironment{theindex}{%
  \ifkorrekturansicht
    \section*{\indexname}%
  \else
    \subsubsection*{Index der erwähnten Entitäten}%
  \fi
  \setlength{\parindent}{0pt}%
  \setlength{\parskip}{0pt plus 0.3pt}%
  \let\item\@idxitem
}{%
  \ifkorrekturansicht\clearpage\fi
}
\makeatother

\IfFileExists{\jobname-pw.ind}{\input{\jobname-pw.ind}}{}

% Quellenangabe nur in der Leseansicht
\ifkorrekturansicht\else
% Fallback-Definitionen, falls die .tex-Datei \titel etc. nicht gesetzt hat
\providecommand{\titel}{}
\providecommand{\editorInnen}{}
\providecommand{\dateiname}{\jobname}

\vspace{3cm}

\vfill

\footnotesize
\textsc{Quelle}: \titel. Herausgegeben von {\editorInnen}. In: \emph{Arthur Schnitzler: Briefwechsel mit Autorinnen und Autoren}.
 Digitale Edition, https://schnitzler-briefe.acdh.oeaw.ac.at/{\dateiname}.html (Stand \today)
\fi

\end{document}


      