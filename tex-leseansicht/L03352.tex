%% latex-korrekturansicht-vorspann.tex
%% Vorspann für die Korrekturansicht.
%% Lädt die gemeinsame Datei latex-vorspann.tex mit gesetztem Schalter.

\newif\ifkorrekturansicht
\korrekturansichttrue

\input{../tex-inputs/latex-vorspann}


\section[ Felix Salten an Arthur Schnitzler, 27. 11. 1903]{L03352 Felix Salten an Arthur Schnitzler, 27. 11. 1903}
\nopagebreak\mylabel{L03352v}
\rehead{ }\normalsize\beginnumbering\briefempfaengerindex{Schnitzler, Arthur@\textsc{Schnitzler, Arthur}!zzzSalten, Felix@\emph{von Felix Salten}!1903-11-271@{27. 11. 1903}|(be}
\toendnotes[C]{\smallbreak\pagebreak[2]}\Standort{CUL, Schnitzler, B 89, A 2.}
\physDesc{Brief, 1 Blatt, 1 Seite, 362 Zeichen
\newline{}Handschrift: Bleistift, lateinische Kurrent
\newline{}Ordnung: mit Bleistift von unbekannter Hand nummeriert: »178« }\toendnotes[C]{\smallbreak}
\pstart
           \raggedleft{}{\pb}Am Kahlenberg\oindex{Kahlenberg@\textbf{Kahlenberg}, \emph{T.MT}|pw}, 27 XI. 03\pend
           \vspace{0.5em}
\pstart
           Lieber, ich bin doch nicht nach Waidhofen\oindex{Waidhofen an der Ybbs@\textbf{Waidhofen an der Ybbs}, \emph{P.PPLA3}|pw} sondern lieber hier herauf, wo es wunderschön und ganz still ist.
               Gedenke mir diesen Berg\oindex{Kahlenberg@\textbf{Kahlenberg}, \emph{T.MT}|pwv} jetzt
               als meinen Privat-Semmering\oindex{Semmering@\textbf{Semmering}, \emph{A.ADM3}|pw} anzuschaffen. Herzl.
               Dank für Ihre Wolmeinung über meinen \label{K_L03352-1v}\edtext{Klimt\pwindex{Klimt, Gustav 14.07.1862 – 06.02.1918@\textsc{Klimt, Gustav} (14.07.1862 – 06.02.1918), \emph{Maler/Malerin}|pw}-Aufsatz\pwindex{Gustav Klimt. Gelegentliche Anmerkungen@\emph{Gustav Klimt. Gelegentliche Anmerkungen}|pwv}}{\lemma{\textnormal{\emph{Klimt-Aufsatz}}}\Cendnote{\textnormal{Felix Salten\pwindex{Salten, Felix 06.09.1869 – 08.10.1945@\textsc{Salten, Felix} (06.09.1869 – 08.10.1945), \emph{Schriftsteller/Schriftstellerin, Journalist/Journalistin, Chefredakteur/Chefredakteurin}|pwk}: \emph{Gustav Klimt. Gelegentliche Anmerkungen}\pwindex{Gustav Klimt. Gelegentliche Anmerkungen@\emph{Gustav Klimt. Gelegentliche Anmerkungen}|pwk}. Buchschmuck
                     von Bertold Löffler\pwindex{Loeffler, Bertold 28.09.1874 – 23.03.1960@\textsc{Löffler, Bertold} (28.09.1874 – 23.03.1960), \emph{Maler/Malerin, Grafiker/Grafikerin}|pwk}. Wien\oindex{Wien@\textbf{Wien}, \emph{A.ADM2}|pwk}, Leipzig\oindex{Leipzig@\textbf{Leipzig}, \emph{P.PPLA3}|pwk}: \emph{Wiener Verlag}\orgindex{Wiener Verlag@Wiener Verlag|pwk}{ }1903.}}}\label{K_L03352-1}. \label{K_L03352-2v}\edtext{Nächstens ziehe
               ich mich hierher mit Schlenther\pwindex{Schlenther, Paul 20.08.1854 – 30.04.1916@\textsc{Schlenther, Paul} (20.08.1854 – 30.04.1916), \emph{Schriftsteller/Schriftstellerin, Kritiker/Kritikerin, Theaterleiter/Theaterleiterin}|pw} zurück}{\lemma{\textnormal{\emph{Nächstens … zurück}}}\Cendnote{\textnormal{nicht nachgewiesen; an Ostern 1904
                  plante er ebenfalls am Kahlenberg\oindex{Kahlenberg@\textbf{Kahlenberg}, \emph{T.MT}|pwk} Tage zu verbringen, vgl. Felix Salten an Arthur Schnitzler, 30. 3. 1904.}}}\label{K_L03352-2}, und hoffe Sie
               noch besser zu bedienen.\pend
           \pstart herzlichst Ihr \spacefill\mbox{S.}\pend{}\selectlanguage{ngerman}\endnumbering\briefempfaengerindex{Schnitzler, Arthur@\textsc{Schnitzler, Arthur}!zzzSalten, Felix@\emph{von Felix Salten}!1903-11-271@{27. 11. 1903}|)be}\mylabel{L03352h}  \normalsize

\doendnotes{C}
\bigskip
\vfill

\clearpage

\footnotesize

\lohead{\textsc{register}}

% Definiere theindex-Environment komplett neu ohne reledmac
\makeatletter
\renewenvironment{theindex}{%
  \section*{\indexname}%
  \setlength{\parindent}{0pt}%
  \setlength{\parskip}{0pt plus 0.3pt}%
  \let\item\@idxitem
}{%
  \clearpage
}
\makeatother

\IfFileExists{\jobname-pw.ind}{\input{\jobname-pw.ind}}{}

\end{document}

      