%% latex-korrekturansicht-vorspann.tex
%% Vorspann für die Korrekturansicht.
%% Lädt die gemeinsame Datei latex-vorspann.tex mit gesetztem Schalter.

\newif\ifkorrekturansicht
\korrekturansichttrue

\input{../tex-inputs/latex-vorspann}


\section[Hugo von Hofmannsthal an Arthur Schnitzler, 12. 8. 1893]{L00254 Hugo von Hofmannsthal an Arthur Schnitzler, 12. 8. 1893}
\nopagebreak\mylabel{L00254v}
\rehead{ }\normalsize\beginnumbering\briefempfaengerindex{Schnitzler, Arthur@\textsc{Schnitzler, Arthur}!zzzHofmannsthal, Hugo von@\emph{von Hugo von Hofmannsthal}!1893-08-123@{12. 8. 1893}|(be}
\toendnotes[C]{\smallbreak\pagebreak[2]}\Standort{CUL, Schnitzler, B 43.}
\physDesc{Brief, 1 Blatt, 2 Seiten, Umschlag, 692 Zeichen
\newline{}Handschrift: schwarze Tinte, deutsche Kurrent
\newline{}Versand: Stempel: »\nobreak{}\oindex{I., Innere Stadt@\textbf{I., Innere Stadt}, \emph{A.ADM3}|pwk}Wien 1/1, 13 8. 93, 9–10½V., Bestellt\nobreak{}«.  
\newline{}Schnitzler: mit Bleistift die Umschlagrückseite datiert: »12/8 93« 
\newline{}Ordnung: 1) mit Bleistift von unbekannter Hand nummeriert:
                                    »56«  2) von unbekannter Hand die Umschlagsrückseite nummeriert:
                                    »56a«}
\buchAbdrucke{\weitereDrucke{1) Hugo von Hofmannsthal: \emph{Briefe. 1890–1901}. Berlin: \emph{S. Fischer} 1935, S. 90.} \weitereDrucke{2) Hugo von Hofmannsthal, Arthur Schnitzler: \emph{Briefwechsel}. Frankfurt am Main: \emph{S. Fischer} 1964, S. 44.} }\toendnotes[C]{\smallbreak}\pstart{}{\pb}Hofmannsthal\pend{}\pstart{}stud iur.\pend{}{\bigskip}\vspace{1em}
\pstart
           \raggedleft{}{\pb}Strobl\oindex{Strobl@\textbf{Strobl}, \emph{A.ADM3}|pw}{ }12 VIII 93.\pend
           
\pstart{}mein lieber Arthur.\pend\vspace{0.5em}
\pstart
           Vielen Dank für Ihre 2 lieben Briefe. Ich arbeite \uuline{nichts}; ich befinde mich ſehr wohl. Ich ſpiele \textsc{Tennis}, \label{K_L00254-1v}\edtext{\textsc{Macao}}{\lemma{\textnormal{\emph{Macao}}}\Cendnote{\textnormal{Kartenspiel}}}\label{K_L00254-1}, fahre, ſchwimme und
               habe keine zuſammenhängenden Gedanken. Ich bin \uuline{kein}
               Poet (Dichter, Schriftſteller, merkwürdiger Menſch \textsc{etc})
               ſondern höchſtens \pend
           
\pstart
           Ihr guter Freund{\\[\baselineskip]}\spacefill\mbox{Hugo.}\pend
           \leftskip=0em{}
\pstart
           \noindent{}\uuline{\edtext{Wo}{\Cendnote{dreifach unterstrichen}}} iſt \textsc{Salten\pwindex{Salten, Felix 06.09.1869 – 08.10.1945@\textsc{Salten, Felix} (06.09.1869 – 08.10.1945), \emph{Schriftsteller/Schriftstellerin, Journalist/Journalistin, Chefredakteur/Chefredakteurin}|pw}}?! Sie ſchreiben er iſt »unten«.\pend
           
\pstart
           \raggedleft{}umdrehen!!\pend
           
\pstart
           {\pb}Im \label{K_L00254-2v}\edtext{September komme ich jedenfalls nach Salzburg\oindex{Salzburg@\textbf{Salzburg}, \emph{A.ADM2}|pw}}{\lemma{\textnormal{\emph{September … Salzburg}}}\Cendnote{\textnormal{Vgl. Paul Goldmann an Arthur Schnitzler, 8. 8. 1893; dazu kam es
                     nicht.}}}\label{K_L00254-2}. Übrigens kann ich jeden Tag in \uline{2} Stunden hinfahren. Ein \textsc{rendez vous} mit Goldmann\pwindex{Goldmann, Paul 31.01.1865 – 25.09.1935@\textsc{Goldmann, Paul} (31.01.1865 – 25.09.1935), \emph{Schriftsteller/Schriftstellerin, Journalist/Journalistin}|pw} wäre mir natürlich eine große
                  Freude.\pend
           \selectlanguage{ngerman}\vspace{1em}
\pstart
           \noindent{}{\pb}\label{T_L00254-1v}\edtext{Es ist eine Gemeinheit, zu sagen, dass
               ich mit »meinem Flämmchen« die Umgebung erleuchten soll, weil es geheissen hat, mit
               einem \uuline{ganz kleinen} Flämm\damage{chen.}}{\lemma{\textnormal{\emph{Es … Flämmchen.}}}\Cendnote{\textnormal{auf der Rückseite des Umschlags}}}\label{T_L00254-1}\pend
           \selectlanguage{ngerman}\endnumbering\briefempfaengerindex{Schnitzler, Arthur@\textsc{Schnitzler, Arthur}!zzzHofmannsthal, Hugo von@\emph{von Hugo von Hofmannsthal}!1893-08-123@{12. 8. 1893}|)be}\mylabel{L00254h}  \normalsize

\doendnotes{C}
\bigskip
\vfill

\clearpage

\footnotesize

\lohead{\textsc{register}}

% Definiere theindex-Environment komplett neu ohne reledmac
\makeatletter
\renewenvironment{theindex}{%
  \section*{\indexname}%
  \setlength{\parindent}{0pt}%
  \setlength{\parskip}{0pt plus 0.3pt}%
  \let\item\@idxitem
}{%
  \clearpage
}
\makeatother

\IfFileExists{\jobname-pw.ind}{\input{\jobname-pw.ind}}{}

\end{document}

      