\input{../tex-inputs/latex-pdf-vorspann}
\begin{center}
            \textcolor{red}{ENTWURF. ENTZIFFERUNG NOCH NICHT KORREKTURGELESEN}
                      \end{center}
            
               \section[Hugo von Hofmannsthal an Arthur Schnitzler, 12. 8. 1893]{ Hugo von Hofmannsthal an Arthur Schnitzler, 12. 8. 1893}\nopagebreak\mylabel{v}\rehead{ }\begin{ledgroupsized}[t]{13cm}\normalsize\beginnumbering\briefempfaengerindex{Schnitzler, Arthur@\textsc{Schnitzler, Arthur}!zzzHofmannsthal, Hugo von@\emph{von Hugo von Hofmannsthal}!1893-08-123@{12. 8. 1893}|(be} \toendnotes[C]{\smallbreak\pagebreak[2]} \Standort{CUL, Schnitzler, B 43.}
\physDesc{Brief, 1 Blatt, 2 Seiten, Umschlag
\newline{}Handschrift: schwarze Tinte, deutsche Kurrent\newline{}Versand: Stempel: »\nobreak{}\oindex{I., Innere Stadt@\textbf{I., Innere Stadt}|pwk}Wien 1/1, 13 8. 93, 9–10½V., Bestellt\nobreak{}«.  
\newline{}Schnitzler: mit Bleistift die Umschlagrückseite datiert: »12/8 93« \newline{}Ordnung: 1) mit Bleistift von unbekannter Hand nummeriert: »56« 2) von unbekannter Hand die Umschlagsrückseite nummeriert:
                                        »56a«}\buchAbdrucke{\weitereDrucke{1) Hugo von Hofmannsthal: \emph{Briefe. 1890–1901}. Berlin: \emph{S. Fischer} 1935, S. 90.} \weitereDrucke{2) Hugo von Hofmannsthal, Arthur Schnitzler: \emph{Briefwechsel}. Hg. Therese Nickl und Heinrich Schnitzler. Frankfurt am Main: \emph{S. Fischer} 1964, S. 44.} }\toendnotes[C]{\smallbreak}\pstart{}{\pb}Hofmannsthal\pend{}\pstart{}stud iur.\pend{}{\bigskip}\pstart
           \raggedleft{}{\pb}Strobl\oindex{Strobl@\textbf{Strobl}|pw}{ }12 VIII 93.\pend
           \pstart{}mein lieber Arthur.\pend\pstart
           Vielen Dank für Ihre 2 lieben Briefe. Ich arbeite \uuline{nichts}; ich befinde mich ſehr wohl. Ich ſpiele \textsc{Tennis}, \label{K_L00254_1v}\edtext{\textsc{Macao}}{\lemma{\textnormal{\emph{Macao}}}\Cendnote{\textnormal{Kartenspiel}}}\label{K_L00254_1h}, fahre, ſchwimme
                    und habe keine zuſammenhängenden Gedanken. Ich bin \uuline{kein} Poet (Dichter, Schriftſteller, merkwürdiger Menſch \textsc{etc}) ſondern höchſtens \pend
           \pstart
           Ihr guter Freund{\\[\baselineskip]}\spacefill\mbox{Hugo.}\pend
           \leftskip=0em{}\pstart
           \noindent{}\uuline{\edtext{Wo}{\Cendnote{dreifach unterstrichen}}} iſt \textsc{Salten\pwindex{Salten, Felix 06.09.1869 – 08.10.1945@\textsc{Salten, Felix} (06.09.1869 – 08.10.1945), \emph{Schriftsteller, Journalist}|pw}}?! Sie ſchreiben er iſt »unten«.\pend
           \pstart
           \raggedleft{}umdrehen!!\pend
           \pstart
           {\pb}Im
                            September komme ich jedenfalls nach Salzburg\oindex{Salzburg@\textbf{Salzburg}|pw}. Übrigens kann ich jeden Tag in \uline{2} Stunden hinfahren. Ein \textsc{rendez vous} mit Goldmann\pwindex{Goldmann, Paul 31.01.1865 – 25.09.1935@\textsc{Goldmann, Paul} (31.01.1865 – 25.09.1935), \emph{Schriftsteller, Journalist}|pw} wäre mir natürlich eine große Freude.\pend
           \pstart
           \noindent{}{\pb}\label{T_L00254-1v}\edtext{Es ist eine Gemeinheit, zu
                    sagen, dass ich mit »meinem Flämmchen« die Umgebung erleuchten soll, weil es
                    geheissen hat, mit einem \uuline{ganz kleinen}
                    Flämm\damage{chen.}}{\lemma{\textnormal{\emph{Es … Flämmchen.}}}\Cendnote{\textnormal{auf der Rückseite des Umschlags}}}\label{T_L00254-1h}\pend
           \endnumbering\briefempfaengerindex{Schnitzler, Arthur@\textsc{Schnitzler, Arthur}!zzzHofmannsthal, Hugo von@\emph{von Hugo von Hofmannsthal}!1893-08-123@{12. 8. 1893}|)be}\mylabel{h}\end{ledgroupsized}  \newcommand{\dateiname}{L00254}\newcommand{\titel}{Hugo von Hofmannsthal an Arthur Schnitzler, 12. 8. 1893}\newcommand{\editorInnen}{Martin Anton Müller und Gerd-Hermann Susen}\input{../tex-inputs/latex-pdf-abspann}
      