%% latex-korrekturansicht-vorspann.tex
%% Vorspann für die Korrekturansicht.
%% Lädt die gemeinsame Datei latex-vorspann.tex mit gesetztem Schalter.

\newif\ifkorrekturansicht
\korrekturansichttrue

\input{../tex-inputs/latex-vorspann}


\section[Hugo von Hofmannsthal an Arthur Schnitzler, 19. 9. {[}1909{]}]{L01877 Hugo von Hofmannsthal an Arthur Schnitzler, 19. 9. {[}1909{]}}
\nopagebreak\mylabel{L01877v}
\rehead{ }\normalsize\beginnumbering\briefempfaengerindex{Schnitzler, Arthur@\textsc{Schnitzler, Arthur}!zzzHofmannsthal, Hugo von@\emph{von Hugo von Hofmannsthal}!1909-09-191@{19. 9. {[}1909{]}}|(be}
\toendnotes[C]{\smallbreak\pagebreak[2]}\Standort{CUL, Schnitzler, B 43.}
\physDesc{Brief, 1 Blatt, 4 Seiten, 1374 Zeichen
\newline{}Handschrift: schwarze Tinte, deutsche Kurrent
\newline{}Schnitzler: mit Bleistift datiert: »19/X 909.« und beschriftet: »\textsc{Hofma{\geminationn}sthal}« 
\newline{}Ordnung: 1) mit Bleistift von unbekannter Hand nummeriert:
                                    »309«  2) mit Bleistift von unbekannter Hand nummeriert:
                                    »307«}
\buchAbdrucke{\weitereDrucke{Hugo von Hofmannsthal, Arthur Schnitzler: \emph{Briefwechsel}. Frankfurt am Main: \emph{S. Fischer} 1964, S. 246.} }\toendnotes[C]{\smallbreak}
\pstart
           
\pstart
           {\pb}19 IX.\pend
           
\pstart
           \raggedleft{}\textsc{Aussee Obertressen} 14\oindex{Obertressen@\textbf{Obertressen}, \emph{P.PPL}|pw}.\pend
           \pend
           
\pstart{}mein guter lieber Arthur \pend\vspace{0.5em}
\pstart
           ich freue mich von ganzem Herzen daſs Ihr ein zweites Kind\pwindex{Cappellini, Lili 13.09.1909 – 26.07.1928@\textsc{Cappellini, Lili} (13.09.1909 – 26.07.1928)|pwv} habt. Ich kann mir denken daſs Sie es
               ſich im Stillen gewünſcht haben, und es iſt zu nett von Olga\pwindex{Schnitzler, Olga 17.01.1882 – 13.01.1970@\textsc{Schnitzler, Olga} (17.01.1882 – 13.01.1970), \emph{Schauspieler/Schauspielerin, Sänger/Sängerin}|pw}, daſs Sie es Ihnen ſofort geſchenkt hat. Ja, ja, die {\pb}eigenen Frauen ſind doch etwas
               ſehr nettes und vielleicht noch netter als die Frauen der Andern, was meinen Sie, Sie
               geübter \textsc{roué, emeritierter Anatol\pwindex{Anatol@\emph{Anatol}|pwv} etc}., Sie Julian Fichtner\pwindex{einsame Weg. Schauspiel in fuenf Akten@\emph{Der einsame Weg. Schauspiel in fünf Akten}|pwv}, Waldemar von Sala\pwindex{einsame Weg. Schauspiel in fuenf Akten@\emph{Der einsame Weg. Schauspiel in fünf Akten}|pwv} – nein der Sala\pwindex{einsame Weg. Schauspiel in fuenf Akten@\emph{Der einsame Weg. Schauspiel in fünf Akten}|pwv} bin ja ich!\pend
           
\pstart
           Kurz, ich freue mich ſehr, daſs für \textsc{Heini}\pwindex{Schnitzler, Heinrich 09.08.1902 – 12.07.1982@\textsc{Schnitzler, Heinrich} (09.08.1902 – 12.07.1982), \emph{Regisseur/Regisseurin, Schauspieler/Schauspielerin}|pw} der einſame Weg\pwindex{einsame Weg. Schauspiel in fuenf Akten@\emph{Der einsame Weg. Schauspiel in fünf Akten}|pwv} nun zu
               Ende iſt und eine kleine {\pb}Dämmerſeele\pwindex{Daemmerseele@\emph{Dämmerseele}|pwv} ihm Geſellſchaft
               leiſten wird, die ſich hoffentlich bald zu einer griechiſchen Tänzerin\pwindex{griechische Taenzerin. Novellette@\emph{Die griechische Tänzerin. Novellette}|pwv} entwickelt.\pend
           
\pstart
           Ich hab Sie ſehr lieb, mein lieber Arthur, und auch Ihre Arbeiten habe ich ſehr lieb,
               das gehört ja dazu. – Von dieſen allen hat mir aber die letzte: »Brüderlein \textsc{Medardus} Hiergeiſt\pwindex{junge Medardus. Dramatische Historie in einem Vorspiel und fuenf Aufzuegen@\emph{Der junge Medardus. Dramatische Historie in einem Vorspiel und fünf Aufzügen}|pw}« den
               allerſchwächſten Eindruck gemacht, ſowohl die Geſtalten als die Fabel. {\pb}Kommt das vielleicht daher, weil
               ich beides nicht kenne?\pend
           
\pstart
           Ich habe eine Spieloper\pwindex{Rosenkavalier@\emph{Der Rosenkavalier}|pwv}
               gemacht, die glaub ich hübſch iſt. (Nicht ſo hübſch wie der tapfere Caſſian\pwindex{tapfere Cassian. Puppenspiel in einem Akt@\emph{Der tapfere Cassian. Puppenspiel in einem Akt}|pw}) Und ferner bilde ich mir in den letzten Tagen
               ſtark ein daſs ich meine (äußerſt ſehr veränderte) Florindocomödie\pwindex{Cristinas Heimreise. Komoedie@\emph{Cristinas Heimreise. Komödie}|pwv} in den nächſten Wochen fertig kriegen
               werde. Ich werde mich zu dieſem Zweck etwas iſolieren, vielleicht in München\oindex{Muenchen@\textbf{München}, \emph{P.PPLA}|pw} oder ſo.\hspace*{1.5em}Auf ein gutes Wiederſehen und vieles \uline{ſehr herzliche}
               an Olga\pwindex{Schnitzler, Olga 17.01.1882 – 13.01.1970@\textsc{Schnitzler, Olga} (17.01.1882 – 13.01.1970), \emph{Schauspieler/Schauspielerin, Sänger/Sängerin}|pw}.\pend
           \pstart Ihr \spacefill\mbox{Arthur}\pend{}\selectlanguage{ngerman}\endnumbering\briefempfaengerindex{Schnitzler, Arthur@\textsc{Schnitzler, Arthur}!zzzHofmannsthal, Hugo von@\emph{von Hugo von Hofmannsthal}!1909-09-191@{19. 9. {[}1909{]}}|)be}\mylabel{L01877h}  \normalsize

\doendnotes{C}
\bigskip
\vfill

\clearpage

\footnotesize

\lohead{\textsc{register}}

% Definiere theindex-Environment komplett neu ohne reledmac
\makeatletter
\renewenvironment{theindex}{%
  \section*{\indexname}%
  \setlength{\parindent}{0pt}%
  \setlength{\parskip}{0pt plus 0.3pt}%
  \let\item\@idxitem
}{%
  \clearpage
}
\makeatother

\IfFileExists{\jobname-pw.ind}{\input{\jobname-pw.ind}}{}

\end{document}

      