%% latex-leseansicht-vorspann.tex
%% Vorspann für die Leseansicht.
%% Lädt die gemeinsame Datei latex-vorspann.tex mit nicht gesetztem Schalter.

\newif\ifkorrekturansicht
\korrekturansichtfalse

\input{../tex-inputs/latex-vorspann}


         
         \renewcommand{\erwaehntePersonen}{Personen: Hugo von Hofmannsthal, Lili Schnitzler, Olga Schnitzler, Heinrich Schnitzler}
         \renewcommand{\erwaehnteOrte}{Orte: Bad Aussee, München, Obertressen, Wien}
         \renewcommand{\erwaehnteWerke}{Werke: Anatol, Cristinas Heimreise. Komödie, Der Rosenkavalier, Der einsame Weg. Schauspiel in fünf Akten, Der junge Medardus. Dramatische Historie in einem Vorspiel und fünf Aufzügen, Der tapfere Cassian. Puppenspiel in einem Akt, Die griechische Tänzerin. Novellette, Dämmerseele}
               \section[Hugo von Hofmannsthal an Arthur Schnitzler, 19. 9. {[}1909{]}]{ Hugo von Hofmannsthal an Arthur Schnitzler, 19. 9. {[}1909{]}}\nopagebreak\mylabel{v}\rehead{ }\begin{ledgroupsized}[t]{13cm}\normalsize\beginnumbering\briefempfaengerindex{Schnitzler, Arthur@\textsc{Schnitzler, Arthur}!zzzHofmannsthal, Hugo von@\emph{von Hugo von Hofmannsthal}!1909-09-191@{19. 9. {[}1909{]}}|(be} \toendnotes[C]{\smallbreak\pagebreak[2]} \Standort{CUL, Schnitzler, B 43.}
\physDesc{Brief, 1 Blatt, 4 Seiten, 1374 Zeichen
\newline{}Handschrift: schwarze Tinte, deutsche Kurrent
\newline{}Schnitzler: mit Bleistift datiert: »19/X 909.« und beschriftet: »\textsc{Hofma{\geminationn}sthal}« 
\newline{}Ordnung: 1) mit Bleistift von unbekannter Hand nummeriert:
                                    »309«  2) mit Bleistift von unbekannter Hand nummeriert:
                                    »307«}\buchAbdrucke{\weitereDrucke{Hugo von Hofmannsthal, Arthur Schnitzler: \emph{Briefwechsel}. Hg. Therese Nickl und Heinrich Schnitzler. Frankfurt am Main: \emph{S. Fischer} 1964, S. 246.} }\toendnotes[C]{\smallbreak}\pstart
           {\pb}19 IX.\hfill \textsc{Aussee Obertressen} 14\oindex{Obertressen@\textbf{Obertressen}|pw}.\pend
           \pstart{}mein guter lieber Arthur \pend\pstart
           ich freue mich von ganzem Herzen daſs Ihr ein zweites Kind\pwindex{Schnitzler, Lili 13.09.1909 – 26.07.1928@\textsc{Schnitzler, Lili} (13.09.1909 – 26.07.1928)|pwv} habt. Ich kann mir denken daſs Sie es
               ſich im Stillen gewünſcht haben, und es iſt zu nett von Olga\pwindex{Schnitzler, Olga 17.01.1882 – 13.01.1970@\textsc{Schnitzler, Olga} (17.01.1882 – 13.01.1970), \emph{Schauspielerin, Sängerin}|pw}, daſs Sie es Ihnen ſofort geſchenkt hat. Ja, ja, die {\pb}eigenen Frauen ſind doch etwas
               ſehr nettes und vielleicht noch netter als die Frauen der Andern, was meinen Sie, Sie
               geübter \textsc{roué, emeritierter Anatol\pwindex{Schnitzler, Arthur 15.05.1862 – 21.10.1931@\textsc{Schnitzler, Arthur} (15.05.1862 – 21.10.1931), \emph{Schriftsteller, Mediziner}!Anatol1892-10-29@\strich\emph{Anatol} {[}1892-10-29{]}|pwv} etc}., Sie Julian Fichtner\pwindex{Schnitzler, Arthur 15.05.1862 – 21.10.1931@\textsc{Schnitzler, Arthur} (15.05.1862 – 21.10.1931), \emph{Schriftsteller, Mediziner}!einsame Weg. Schauspiel in fuenf Akten1904@\strich\emph{Der einsame Weg. Schauspiel in fünf Akten} {[}1904{]}|pwv}, Waldemar von Sala\pwindex{Schnitzler, Arthur 15.05.1862 – 21.10.1931@\textsc{Schnitzler, Arthur} (15.05.1862 – 21.10.1931), \emph{Schriftsteller, Mediziner}!einsame Weg. Schauspiel in fuenf Akten1904@\strich\emph{Der einsame Weg. Schauspiel in fünf Akten} {[}1904{]}|pwv} – nein der Sala\pwindex{Schnitzler, Arthur 15.05.1862 – 21.10.1931@\textsc{Schnitzler, Arthur} (15.05.1862 – 21.10.1931), \emph{Schriftsteller, Mediziner}!einsame Weg. Schauspiel in fuenf Akten1904@\strich\emph{Der einsame Weg. Schauspiel in fünf Akten} {[}1904{]}|pwv} bin ja ich!\pend
           \pstart
           Kurz, ich freue mich ſehr, daſs für \textsc{Heini}\pwindex{Schnitzler, Heinrich 09.08.1902 – 12.07.1982@\textsc{Schnitzler, Heinrich} (09.08.1902 – 12.07.1982), \emph{Regisseur, Schauspieler}|pw} der einſame Weg\pwindex{Schnitzler, Arthur 15.05.1862 – 21.10.1931@\textsc{Schnitzler, Arthur} (15.05.1862 – 21.10.1931), \emph{Schriftsteller, Mediziner}!einsame Weg. Schauspiel in fuenf Akten1904@\strich\emph{Der einsame Weg. Schauspiel in fünf Akten} {[}1904{]}|pwv} nun zu
               Ende iſt und eine kleine {\pb}Dämmerſeele\pwindex{Schnitzler, Arthur 15.05.1862 – 21.10.1931@\textsc{Schnitzler, Arthur} (15.05.1862 – 21.10.1931), \emph{Schriftsteller, Mediziner}!Daemmerseele18. 5. 1902@\strich\emph{Dämmerseele} {[}18. 5. 1902{]}|pwv} ihm Geſellſchaft
               leiſten wird, die ſich hoffentlich bald zu einer griechiſchen Tänzerin\pwindex{Schnitzler, Arthur 15.05.1862 – 21.10.1931@\textsc{Schnitzler, Arthur} (15.05.1862 – 21.10.1931), \emph{Schriftsteller, Mediziner}!griechische Taenzerin. Novellette28. 09. 1902@\strich\emph{Die griechische Tänzerin. Novellette} {[}28. 09. 1902{]}|pwv} entwickelt.\pend
           \pstart
           Ich hab Sie ſehr lieb, mein lieber Arthur, und auch Ihre Arbeiten habe ich ſehr lieb,
               das gehört ja dazu. – Von dieſen allen hat mir aber die letzte: »Brüderlein \textsc{Medardus} Hiergeiſt\pwindex{Schnitzler, Arthur 15.05.1862 – 21.10.1931@\textsc{Schnitzler, Arthur} (15.05.1862 – 21.10.1931), \emph{Schriftsteller, Mediziner}!junge Medardus. Dramatische Historie in einem Vorspiel und fuenf
                  Aufzuegen1910-10-26@\strich\emph{Der junge Medardus. Dramatische Historie in einem Vorspiel und fünf Aufzügen} {[}1910-10-26{]}|pw}« den
               allerſchwächſten Eindruck gemacht, ſowohl die Geſtalten als die Fabel. {\pb}Kommt das vielleicht daher, weil
               ich beides nicht kenne?\pend
           \pstart
           Ich habe eine Spieloper\pwindex{Hofmannsthal, Hugo von 1874-02-01 – 1929-07-15@\textsc{Hofmannsthal, Hugo von} (1874-02-01 – 1929-07-15), \emph{Schriftsteller}!Rosenkavalier1911@\strich\emph{Der Rosenkavalier} {[}1911{]}|pwv}
               gemacht, die glaub ich hübſch iſt. (Nicht ſo hübſch wie der tapfere Caſſian\pwindex{Schnitzler, Arthur 15.05.1862 – 21.10.1931@\textsc{Schnitzler, Arthur} (15.05.1862 – 21.10.1931), \emph{Schriftsteller, Mediziner}!tapfere Cassian. Puppenspiel in einem Akt01. 02. 1904@\strich\emph{Der tapfere Cassian. Puppenspiel in einem Akt} {[}01. 02. 1904{]}|pw}) Und ferner bilde ich mir in den letzten Tagen
               ſtark ein daſs ich meine (äußerſt ſehr veränderte) Florindocomödie\pwindex{Hofmannsthal, Hugo von 1874-02-01 – 1929-07-15@\textsc{Hofmannsthal, Hugo von} (1874-02-01 – 1929-07-15), \emph{Schriftsteller}!Cristinas Heimreise. Komoedie11. 2. 1910@\strich\emph{Cristinas Heimreise. Komödie} {[}11. 2. 1910{]}|pwv} in den nächſten Wochen fertig kriegen
               werde. Ich werde mich zu dieſem Zweck etwas iſolieren, vielleicht in München\oindex{Muenchen@\textbf{München}|pw} oder ſo.\hspace*{1.5em}Auf ein gutes Wiederſehen und vieles \uline{ſehr herzliche}
               an Olga\pwindex{Schnitzler, Olga 17.01.1882 – 13.01.1970@\textsc{Schnitzler, Olga} (17.01.1882 – 13.01.1970), \emph{Schauspielerin, Sängerin}|pw}.\pend
           \pstart Ihr \spacefill\mbox{Arthur}\pend{}
         
         \endnumbering\mylabel{h}\end{ledgroupsized}  \newcommand{\dateiname}{L01877}\newcommand{\titel}{Hugo von Hofmannsthal an Arthur Schnitzler, 19. 9. [1909]}\newcommand{\editorInnen}{Martin Anton Müller und Gerd-Hermann Susen}%% latex-leseansicht-abspann.tex
%% Abspann für die Leseansicht.
%% Der Schalter \ifkorrekturansicht ist bereits durch den Vorspann gesetzt.

%% latex-abspann.tex
%% Gemeinsamer Abspann für Korrekturansicht und Leseansicht.
%% Setzt den Schalter \ifkorrekturansicht voraus (gesetzt in den
%% einbindenden Dateien latex-korrekturansicht-abspann.tex bzw.
%% latex-leseansicht-abspann.tex).
%% ---------------------------------------------------------------

\normalsize

% Das esempio-Environment wird nur in der Leseansicht benötigt
\ifkorrekturansicht\else
\newenvironment{esempio}[3]%
{
    \vspace{1.5ex}
    \rlap{\underline{#1}}
    \par
    \setlength{\parindent}{0cm}
    \nopagebreak
    \leftskip=#2cm
    \rightskip=#3cm
}
{
    \par
}
\fi

\doendnotes{C}
\bigskip
\vfill

\clearpage

\footnotesize

\ifkorrekturansicht
  \lohead{\textsc{register}}
\fi

% theindex-Environment neu definieren ohne reledmac
\makeatletter
\renewenvironment{theindex}{%
  \ifkorrekturansicht
    \section*{\indexname}%
  \else
    \subsubsection*{Index der erwähnten Entitäten}%
  \fi
  \setlength{\parindent}{0pt}%
  \setlength{\parskip}{0pt plus 0.3pt}%
  \let\item\@idxitem
}{%
  \ifkorrekturansicht\clearpage\fi
}
\makeatother

\IfFileExists{\jobname-pw.ind}{\input{\jobname-pw.ind}}{}

% Quellenangabe nur in der Leseansicht
\ifkorrekturansicht\else
% Fallback-Definitionen, falls die .tex-Datei \titel etc. nicht gesetzt hat
\providecommand{\titel}{}
\providecommand{\editorInnen}{}
\providecommand{\dateiname}{\jobname}

\vspace{3cm}

\vfill

\footnotesize
\textsc{Quelle}: \titel. Herausgegeben von {\editorInnen}. In: \emph{Arthur Schnitzler: Briefwechsel mit Autorinnen und Autoren}.
 Digitale Edition, https://schnitzler-briefe.acdh.oeaw.ac.at/{\dateiname}.html (Stand \today)
\fi

\end{document}


      