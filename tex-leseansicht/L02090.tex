%% latex-leseansicht-vorspann.tex
%% Vorspann für die Leseansicht.
%% Lädt die gemeinsame Datei latex-vorspann.tex mit nicht gesetztem Schalter.

\newif\ifkorrekturansicht
\korrekturansichtfalse

\input{../tex-inputs/latex-vorspann}


               \section[Arthur Schnitzler an Hermann Bahr, 25. 9. 1912]{ Arthur Schnitzler an Hermann Bahr, 25. 9. 1912}\nopagebreak\mylabel{v}\rehead{ }\begin{ledgroupsized}[t]{13cm}\normalsize\beginnumbering\briefempfaengerindex{Bahr, Hermann@\textsc{Bahr, Hermann}!zzzSchnitzler, Arthur@\emph{von Arthur Schnitzler}!1912-09-251@{{[}25./26.{]} 9. 1912}|(be} \toendnotes[C]{\smallbreak\pagebreak[2]} \Standort{TMW, HS AM 60162 Ba.}
\physDesc{Bildpostkarte
\newline{}Handschrift: schwarze Tinte, deutsche Kurrent\newline{}Versand: 1) Stempel: »\nobreak{}\oindex{I., Innere Stadt@\textbf{I., Innere Stadt}|pwk}1/1 Wien 8, 25. IX 12, {[}3{]}–4\nobreak{}«.  2) mit Bleistift von unbekannter Hand die ursprüngliche
                                 Adressierung gestrichen und ausgebessert zu: »Semmering Villa Mauthner\oindex{Villa Mauthner-Markhof@\textbf{Villa Mauthner-Markhof}|pw}«\newline{}Ordnung: Lochung \newline{}Zusatz: Postkartenmotiv mit Olga\pwindex{Schnitzler, Olga 17.01.1882 – 13.01.1970@\textsc{Schnitzler, Olga} (17.01.1882 – 13.01.1970), \emph{Schauspielerin, Sängerin}|pw} und
                                    Heinrich\pwindex{Schnitzler, Heinrich 09.08.1902 – 12.07.1982@\textsc{Schnitzler, Heinrich} (09.08.1902 – 12.07.1982), \emph{Regisseur, Schauspieler}|pw} links vor dem Haus
                                 und Schnitzler und Lili\pwindex{Schnitzler, Lili 13.09.1909 – 26.07.1928@\textsc{Schnitzler, Lili} (13.09.1909 – 26.07.1928)|pw} auf dem
                                 Söller }\buchAbdrucke{\weitereDrucke{1) \emph{26. 9. 1912, Abschrift.} In: Arthur Schnitzler: \emph{The Letters of Arthur Schnitzler to Hermann Bahr}. Edited, annotated, and with an introduction, by Donald G.
                        Daviau. Chapel Hill: \emph{The University of North Carolina Press} 1978, S. 109 (University of North Carolina studies in the Germanic languages
                        and literatures, 89).} \weitereDrucke{2) Hermann Bahr, Arthur Schnitzler: \emph{Briefwechsel, Aufzeichnungen, Dokumente (1891–1931)}. Hg. Kurt Ifkovits und Martin Anton Müller. Göttingen: \emph{Wallstein} 2018, S. 477.} }\toendnotes[C]{\smallbreak}\pstart{}{\pb}Herrn \pend{}\pstart{}\textsc{Hermann Bahr}\pend{}\pstart{}Wien Ober St. Veit\oindex{Ober Sankt Veit@\textbf{Ober Sankt Veit}|pw}\pend{}\pstart{}Veitliſſengaſſe\oindex{Veitlissengasse@\textbf{Veitlissengasse}|pw}\pend{}{\bigskip}\pstart
           \noindent{}\centering{}\textcolor{gray}{\textbf{{\pb}Wien, XVIII, Sternwartestr. 71\oindex{Sternwartestrasse@\textbf{Sternwartestraße}|pw}.}}\pend
           \pstart
           herzlichen Dank, lieber Hermann für dein neues \label{K_L02090_1v}\edtext{B\damage{uc}h\pwindex{Bahr, Hermann 19.07.1863 – 15.01.1934@\textsc{Bahr, Hermann} (19.07.1863 – 15.01.1934), \emph{Schriftsteller, Kritiker}!Inventur1912@\strich\emph{Inventur} {[}1912{]}|pwv}}{\lemma{\textnormal{\emph{Buch}}}\Cendnote{\textnormal{Hermann Bahr\pwindex{Bahr, Hermann 19.07.1863 – 15.01.1934@\textsc{Bahr, Hermann} (19.07.1863 – 15.01.1934), \emph{Schriftsteller, Kritiker}|pwk}: \emph{Inventur}\pwindex{Bahr, Hermann 19.07.1863 – 15.01.1934@\textsc{Bahr, Hermann} (19.07.1863 – 15.01.1934), \emph{Schriftsteller, Kritiker}!Inventur1912@\strich\emph{Inventur} {[}1912{]}|pwk}. Berlin: \emph{S. Fischer}\orgindex{S. Fischer Verlag@S. Fischer Verlag|pwk}{ }1912.}}}\label{K_L02090_1h} u viele Grüße. Ob die dich \damage{tre}ffen werden, weiſs ich nicht – de{\geminationn} niemand
               weiſs \label{K_L02090_2v}\edtext{wo du biſt}{\lemma{\textnormal{\emph{wo du biſt}}}\Cendnote{\textnormal{Bahr\pwindex{Bahr, Hermann 19.07.1863 – 15.01.1934@\textsc{Bahr, Hermann} (19.07.1863 – 15.01.1934), \emph{Schriftsteller, Kritiker}|pwk} war zumindest seit Mitte
                     September mehrere Wochen in der Villa
                     Mautner\oindex{Villa Mauthner-Markhof@\textbf{Villa Mauthner-Markhof}|pwk}.}}}\label{K_L02090_2h}. So ſei denn der Findigkeit der Poſt {\pb}vertraut. Au\damage{f} bald!\pend
           \pstart Dein \spacefill\mbox{Arthur }\pend{}\pstart
           \label{K_L02090_3v}\edtext{26/9 1912}{\lemma{\textnormal{\emph{26/9 1912}}}\Cendnote{\textnormal{Der Poststempel widerspricht der
                        Datierung.}}}\label{K_L02090_3h}\pend
                     \endnumbering\briefempfaengerindex{Bahr, Hermann@\textsc{Bahr, Hermann}!zzzSchnitzler, Arthur@\emph{von Arthur Schnitzler}!1912-09-251@{{[}25./26.{]} 9. 1912}|)be}\mylabel{h}\end{ledgroupsized}  \newcommand{\dateiname}{L02090}\newcommand{\titel}{Arthur Schnitzler an Hermann Bahr, 25. 9. 1912}\newcommand{\editorInnen}{ Kurt Ifkovits,  Martin Anton Müller}
            \footnotesize
\begin{ledgroupsized}[t]{11.5cm}
\doendnotes{C}
\end{ledgroupsized}
         %% latex-leseansicht-abspann.tex
%% Abspann für die Leseansicht.
%% Der Schalter \ifkorrekturansicht ist bereits durch den Vorspann gesetzt.

%% latex-abspann.tex
%% Gemeinsamer Abspann für Korrekturansicht und Leseansicht.
%% Setzt den Schalter \ifkorrekturansicht voraus (gesetzt in den
%% einbindenden Dateien latex-korrekturansicht-abspann.tex bzw.
%% latex-leseansicht-abspann.tex).
%% ---------------------------------------------------------------

\normalsize

% Das esempio-Environment wird nur in der Leseansicht benötigt
\ifkorrekturansicht\else
\newenvironment{esempio}[3]%
{
    \vspace{1.5ex}
    \rlap{\underline{#1}}
    \par
    \setlength{\parindent}{0cm}
    \nopagebreak
    \leftskip=#2cm
    \rightskip=#3cm
}
{
    \par
}
\fi

\doendnotes{C}
\bigskip
\vfill

\clearpage

\footnotesize

\ifkorrekturansicht
  \lohead{\textsc{register}}
\fi

% theindex-Environment neu definieren ohne reledmac
\makeatletter
\renewenvironment{theindex}{%
  \ifkorrekturansicht
    \section*{\indexname}%
  \else
    \subsubsection*{Index der erwähnten Entitäten}%
  \fi
  \setlength{\parindent}{0pt}%
  \setlength{\parskip}{0pt plus 0.3pt}%
  \let\item\@idxitem
}{%
  \ifkorrekturansicht\clearpage\fi
}
\makeatother

\IfFileExists{\jobname-pw.ind}{\input{\jobname-pw.ind}}{}

% Quellenangabe nur in der Leseansicht
\ifkorrekturansicht\else
% Fallback-Definitionen, falls die .tex-Datei \titel etc. nicht gesetzt hat
\providecommand{\titel}{}
\providecommand{\editorInnen}{}
\providecommand{\dateiname}{\jobname}

\vspace{3cm}

\vfill

\footnotesize
\textsc{Quelle}: \titel. Herausgegeben von {\editorInnen}. In: \emph{Arthur Schnitzler: Briefwechsel mit Autorinnen und Autoren}.
 Digitale Edition, https://schnitzler-briefe.acdh.oeaw.ac.at/{\dateiname}.html (Stand \today)
\fi

\end{document}


      