%% latex-korrekturansicht-vorspann.tex
%% Vorspann für die Korrekturansicht.
%% Lädt die gemeinsame Datei latex-vorspann.tex mit gesetztem Schalter.

\newif\ifkorrekturansicht
\korrekturansichttrue

\input{../tex-inputs/latex-vorspann}


\section[Arthur Schnitzler an Hermann Bahr, 25. 9. 1912]{L02090 Arthur Schnitzler an Hermann Bahr, 25. 9. 1912}
\nopagebreak\mylabel{L02090v}
\rehead{ }\normalsize\beginnumbering\briefempfaengerindex{Bahr, Hermann@\textsc{Bahr, Hermann}!zzzSchnitzler, Arthur@\emph{von Arthur Schnitzler}!1912-09-251@{{[}25./26.{]} 9. 1912}|(be}
\toendnotes[C]{\smallbreak\pagebreak[2]}\Standort{TMW, HS AM 60162 Ba.}
\physDesc{Bildpostkarte, 259 Zeichen
\newline{}Handschrift: schwarze Tinte, deutsche Kurrent
\newline{}Versand: 1) Stempel: »\nobreak{}\oindex{I., Innere Stadt@\textbf{I., Innere Stadt}, \emph{A.ADM3}|pwk}1/1 Wien 8, 25. IX 12, {[}3{]}–4\nobreak{}«.   2) mit Bleistift von unbekannter Hand die ursprüngliche
                                 Adressierung gestrichen und ausgebessert zu: »Semmering Villa Mauthner\oindex{Villa Mauthner-Markhof@\textbf{Villa Mauthner-Markhof}, \emph{Wohngebäude (K.WHS)}|pw}«
\newline{}Ordnung: Lochung 
\newline{}Zusatz: Postkartenmotiv mit Olga\pwindex{Schnitzler, Olga 17.01.1882 – 13.01.1970@\textsc{Schnitzler, Olga} (17.01.1882 – 13.01.1970), \emph{Schauspieler/Schauspielerin, Sänger/Sängerin}|pw}
                                 und Heinrich\pwindex{Schnitzler, Heinrich 09.08.1902 – 12.07.1982@\textsc{Schnitzler, Heinrich} (09.08.1902 – 12.07.1982), \emph{Regisseur/Regisseurin, Schauspieler/Schauspielerin}|pw} links vor dem
                                 Haus und Schnitzler und Lili\pwindex{Cappellini, Lili 13.09.1909 – 26.07.1928@\textsc{Cappellini, Lili} (13.09.1909 – 26.07.1928)|pw}
                                 auf dem Söller }
\buchAbdrucke{\weitereDrucke{1) Arthur Schnitzler: \emph{The Letters of Arthur Schnitzler to Hermann Bahr}. Chapel Hill: \emph{The University of North Carolina Press} 1978, S. 109.} \weitereDrucke{2) Hermann Bahr, Arthur Schnitzler: \emph{Briefwechsel, Aufzeichnungen, Dokumente (1891–1931)}. Göttingen: \emph{Wallstein} 2018, S. 477.} }\toendnotes[C]{\smallbreak}\pstart{}{\pb}Herrn \pend{}\pstart{}\textsc{Hermann Bahr}\pend{}\pstart{}Wien Ober St. Veit\oindex{Ober Sankt Veit@\textbf{Ober Sankt Veit}, \emph{P.PPLX}|pw}\pend{}\pstart{}Veitliſſengaſſe\oindex{Veitlissengasse@\textbf{Veitlissengasse}, \emph{Straße (K.STR)}|pw}\pend{}{\bigskip}
\pstart
           \noindent{}\centering{}{\pb}\textcolor{gray}{\textbf{Wien, XVIII, Sternwartestr. 71\oindex{Sternwartestrasse 71@\textbf{Sternwartestraße 71}, \emph{Wohngebäude (K.WHS)}|pw}.}}\pend
           \vspace{1em}
\pstart
           \noindent{}{\pb}herzlichen Dank, lieber Hermann für dein neues \label{K_L02090-1v}\edtext{B\damage{uc}h\pwindex{Inventur@\emph{Inventur}|pwv}}{\lemma{\textnormal{\emph{Buch}}}\Cendnote{\textnormal{Hermann Bahr\pwindex{Bahr, Hermann 19.07.1863 – 15.01.1934@\textsc{Bahr, Hermann} (19.07.1863 – 15.01.1934), \emph{Schriftsteller/Schriftstellerin, Kritiker/Kritikerin}|pwk}: \emph{Inventur}\pwindex{Inventur@\emph{Inventur}|pwk}. Berlin: \emph{S. Fischer}\orgindex{S. Fischer Verlag@S. Fischer Verlag|pwk}{ }1912.}}}\label{K_L02090-1} u viele Grüße. Ob die dich \damage{tre}ffen werden, weiſs ich nicht – de{\geminationn} niemand
               weiſs \label{K_L02090-2v}\edtext{wo du biſt}{\lemma{\textnormal{\emph{wo du biſt}}}\Cendnote{\textnormal{Bahr\pwindex{Bahr, Hermann 19.07.1863 – 15.01.1934@\textsc{Bahr, Hermann} (19.07.1863 – 15.01.1934), \emph{Schriftsteller/Schriftstellerin, Kritiker/Kritikerin}|pwk} war zumindest seit Mitte
                     September mehrere Wochen in der Villa
                     Mautner\oindex{Villa Mauthner-Markhof@\textbf{Villa Mauthner-Markhof}, \emph{Wohngebäude (K.WHS)}|pwk}.}}}\label{K_L02090-2}. So ſei denn der Findigkeit der Poſt {\pb}vertraut. Au\damage{f} bald!\pend
           \pstart Dein \spacefill\mbox{Arthur}\pend{}
\pstart
           \label{K_L02090-3v}\edtext{26/9 1912}{\lemma{\textnormal{\emph{26/9 1912}}}\Cendnote{\textnormal{Der Poststempel widerspricht der
                        Datierung.}}}\label{K_L02090-3}\pend
           \selectlanguage{ngerman}\endnumbering\briefempfaengerindex{Bahr, Hermann@\textsc{Bahr, Hermann}!zzzSchnitzler, Arthur@\emph{von Arthur Schnitzler}!1912-09-251@{{[}25./26.{]} 9. 1912}|)be}\mylabel{L02090h}  \normalsize

\doendnotes{C}
\bigskip
\vfill

\clearpage

\footnotesize

\lohead{\textsc{register}}

% Definiere theindex-Environment komplett neu ohne reledmac
\makeatletter
\renewenvironment{theindex}{%
  \section*{\indexname}%
  \setlength{\parindent}{0pt}%
  \setlength{\parskip}{0pt plus 0.3pt}%
  \let\item\@idxitem
}{%
  \clearpage
}
\makeatother

\IfFileExists{\jobname-pw.ind}{\input{\jobname-pw.ind}}{}

\end{document}

      