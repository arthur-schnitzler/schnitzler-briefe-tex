%% latex-leseansicht-vorspann.tex
%% Vorspann für die Leseansicht.
%% Lädt die gemeinsame Datei latex-vorspann.tex mit nicht gesetztem Schalter.

\newif\ifkorrekturansicht
\korrekturansichtfalse

\input{../tex-inputs/latex-vorspann}


\section[ Paul Goldmann an Arthur Schnitzler, 26. 9. [1896]]{L02786 Paul Goldmann an Arthur Schnitzler,  26. 9. [1896]}
\nopagebreak\mylabel{L02786v}
\rehead{ }\normalsize\beginnumbering\briefempfaengerindex{Schnitzler, Arthur@\textsc{Schnitzler, Arthur}!zzzGoldmann, Paul@\emph{von Paul Goldmann}!1896-09-261@{26. 9. [1896]}|(be}
\toendnotes[C]{\smallbreak\pagebreak[2]}
\correspDesc{Versand  durch Paul Goldmann am 26. 9. [1896] in Paris
\newline{}Erhalt  durch Arthur Schnitzler im Zeitraum [27. 9. 1896
                  – 1. 10. 1896?] in Wien}\toendnotes[C]{\smallbreak}
\Standort{DLA, A:Schnitzler, HS.NZ85.1.3166.}
\physDesc{Brief, 2 Blätter, 7 Seiten, 5697 Zeichen
\newline{}Handschrift: blaue Tinte, deutsche Kurrent
\newline{}Beilagen: 1) Thorel\pwindex{Thorel, Jean 11.\,9.\,1859 Éragny – 20.\,8.\,1916 Enghien-les-Bains@\textsc{Thorel, Jean} (11.\,9.\,1859 Éragny – 20.\,8.\,1916 Enghien-les-Bains), \emph{Übersetzer, Dramatiker}|pw}: handschriftlicher
                                 Brief: 1 Blatt, 1 Seite, schwarze Tinte, lateinische Kurrent; mit
                                 Bleistift von Schnitzler
                                 datiert: »Sept 96«  2) Nansen\pwindex{Nansen, Peter 20.\,1.\,1861 Kopenhagen – 31.\,7.\,1918 Mariager@\textsc{Nansen, Peter} (20.\,1.\,1861 Kopenhagen – 31.\,7.\,1918 Mariager), \emph{Schriftsteller, Journalist, Verleger}|pw}: handschriftlicher
                                 Brief: 1 Blatt, 4 Seiten, schwarze Tinte, lateinische
                                 Kurrent
\newline{}Schnitzler: 1) mit Bleistift das Jahr »96« vermerkt  2) mit rotem Buntstift sechs Unterstreichungen}\toendnotes[C]{\smallbreak}
\pstart
           {\pb}\textcolor{gray}{\textbf{\textbf{Frankfurter Zeitung\orgindex{Frankfurter Zeitung@Frankfurter Zeitung|pw}}}}\pend
           
\pstart
           \textcolor{gray}{\textbf{(\begin{otherlanguage}{french}Gazette de Francfort\end{otherlanguage}\orgindex{Frankfurter Zeitung@Frankfurter Zeitung|pw}).}}\pend
           
\pstart
           \textcolor{gray}{\textbf{\textbf{\begin{otherlanguage}{french}Fondateur M.\end{otherlanguage}{ }L. Sonnemann\pwindex{Sonnemann, Leopold 29.\,10.\,1831 Höchberg – 30.\,10.\,1909 Frankfurt am Main@\textsc{Sonnemann, Leopold} (29.\,10.\,1831 Höchberg – 30.\,10.\,1909 Frankfurt am Main), \emph{Journalist, Herausgeber}|pw}.}}}\pend
           
\pstart
           \begin{otherlanguage}{french}\textcolor{gray}{\textbf{Journal\pwindex{Frankfurter Zeitung@\emph{Frankfurter Zeitung}|pwv} politique,
                        financier,}}\end{otherlanguage}\pend
           
\pstart
           \begin{otherlanguage}{french}\textcolor{gray}{\textbf{commercial et littéraire.}}\end{otherlanguage}\pend
           
\pstart
           \begin{otherlanguage}{french}\textcolor{gray}{\textbf{\textbf{Paraissant trois fois par jour.}}}\end{otherlanguage}\hfill \textsc{Paris\oindex{Paris@\textbf{Paris}, \emph{Hauptstadt}|pw}}, 26. September.\pend
           
\pstart
           \begin{otherlanguage}{french}\textcolor{gray}{\textbf{\textbf{Bureau à Paris\oindex{Paris@\textbf{Paris}, \emph{Hauptstadt}|pw}}}}\end{otherlanguage}\pend
           
\pstart
           \begin{otherlanguage}{french}\textcolor{gray}{\textbf{\textbf{24. Rue Feydeau\oindex{rue Feydeau@\textbf{rue Feydeau}, \emph{Straße}|pw}.}}}\end{otherlanguage}\pend
           
\pstart\center{}Mein lieber Freund,\pend\vspace{0.5em}
\pstart
           Ich beſtätige Dir den Empfang der 500 \textsc{Francs}, die ich
               gleich an \textsc{Thorel\pwindex{Thorel, Jean 11.\,9.\,1859 Éragny – 20.\,8.\,1916 Enghien-les-Bains@\textsc{Thorel, Jean} (11.\,9.\,1859 Éragny – 20.\,8.\,1916 Enghien-les-Bains), \emph{Übersetzer, Dramatiker}|pw}} weitergeben will. Anbei ein Brief von ihm.\pend
           
\pstart
           Ich füge ferner einen Brief von \textsc{Nansen\pwindex{Nansen, Peter 20.\,1.\,1861 Kopenhagen – 31.\,7.\,1918 Mariager@\textsc{Nansen, Peter} (20.\,1.\,1861 Kopenhagen – 31.\,7.\,1918 Mariager), \emph{Schriftsteller, Journalist, Verleger}|pw}{ }\strikeout{bei}} bei, den ich dieſer Tage erhielt, nachdem ich{ }ſeiner Frau\pwindex{Nansen, Betty 19.\,3.\,1873 Kopenhagen – 15.\,3.\,1943 ebd.@\textsc{Nansen, Betty} (19.\,3.\,1873 Kopenhagen – 15.\,3.\,1943 ebd.), \emph{Theaterleiterin, Schauspielerin}|pwv} franzöſiſche \label{K_L02786-1v}\edtext{\textsc{\begin{otherlanguage}{french}chansons\end{otherlanguage}}}{\lemma{\textnormal{\emph{chansons}}}\Cendnote{\textnormal{französisch: Lieder}}}\label{K_L02786-1} geſchickt.
               Ihr{ }ſolltet dem Manne\pwindex{Nansen, Peter 20.\,1.\,1861 Kopenhagen – 31.\,7.\,1918 Mariager@\textsc{Nansen, Peter} (20.\,1.\,1861 Kopenhagen – 31.\,7.\,1918 Mariager), \emph{Schriftsteller, Journalist, Verleger}|pwv} einen
                  \label{K_L02786-2v}\edtext{Gruß{ }ſchreiben}{\lemma{\textnormal{\emph{Gruß schreiben}}}\Cendnote{\textnormal{In Folge schrieb Schnitzler am 28. 9. 1896 an
                     Peter Nansen\pwindex{Nansen, Peter 20.\,1.\,1861 Kopenhagen – 31.\,7.\,1918 Mariager@\textsc{Nansen, Peter} (20.\,1.\,1861 Kopenhagen – 31.\,7.\,1918 Mariager), \emph{Schriftsteller, Journalist, Verleger}|pwk}. Siehe \emph{Peter Nansen – Arthur Schnitzler. Ein Briefwechsel zweier
                        Geistesverwandter}. Herausgegeben, kommentiert und mit einem Nachwort
                     versehen von Karin Bang. Roskilde: \emph{Zentrum für
                        österreichisch-nordische Kulturstudien}{ }2003, S. 5–6 (Småskrifter fra CØNK / Kleine Schriften
                     von ZÖNK 9).
               }}}\label{K_L02786-2}, denke ich.\pend
           
\pstart
           Es thut mir von Herzen leid, daß Dich die Wien\oindex{Wien@\textbf{Wien}, \emph{Verwaltungsgebiet}|pw}er
               Nervoſitäten wieder haben. Gibts denn {\pb}gar kein
               Mittel dagegen? Geh’ doch auf ein paar Wochen nach dem Süden!\pend
           
\pstart
           \label{K_L02786-3v}\edtext{Was hörſt Du aus Berlin\oindex{Berlin@\textbf{Berlin}, \emph{Hauptstadt}|pw} über Dein Stück\pwindex{Schnitzler, Arthur 15.\,5.\,1862 Wien – 21.\,10.\,1931 ebd.@\textsc{Schnitzler, Arthur} (15.\,5.\,1862 Wien – 21.\,10.\,1931 ebd.), \emph{Schriftsteller, Mediziner}!Freiwild. Schauspiel in 3 Akten@\strich\emph{Freiwild. Schauspiel in 3 Akten}|pwv}}{\lemma{\textnormal{\emph{Was … Stück}}}\Cendnote{\textnormal{Goldmann\pwindex{Goldmann, Paul 31.\,1.\,1865 Breslau – 25.\,9.\,1935 Wien@\textsc{Goldmann, Paul} (31.\,1.\,1865 Breslau – 25.\,9.\,1935 Wien), \emph{Schriftsteller, Journalist}|pwk} wollte wissen, wie die
                  Vorbereitungen zur Uraufführung\eventindex{Deutsches Theater Berlin@\textbf{Deutsches Theater Berlin}!Uraufführung von Freiwild, 3.11.1896@Uraufführung von Freiwild, 3.11.1896|pwkv} von \emph{Freiwild}\pwindex{Schnitzler, Arthur 15.\,5.\,1862 Wien – 21.\,10.\,1931 ebd.@\textsc{Schnitzler, Arthur} (15.\,5.\,1862 Wien – 21.\,10.\,1931 ebd.), \emph{Schriftsteller, Mediziner}!Freiwild. Schauspiel in 3 Akten@\strich\emph{Freiwild. Schauspiel in 3 Akten}|pwk}
                  vorangingen. Vgl. \emph{Der Briefwechsel Arthur Schnitzler – Otto
                        Brahm}. Vollständige Ausgabe. Herausgegeben, eingeleitet und
                     erläutert von Oskar Seidlin. Tübingen: \emph{Niemeyer}{ }1975, S. 14–28. }}}\label{K_L02786-3}? Daß es Dir zuwider iſt, verſteht{ }ſich von{ }ſelbſt. Das iſt die natürliche Reaction gegen die ungeheure Arbeit, die Du
               darauf verwandt haſt.\pend
           
\pstart
           Dieſer Tage war ein \textsc{Arthur Holitscher\pwindex{Holitscher, Arthur 22.\,8.\,1869 Budapest – 14.\,10.\,1941 Genf@\textsc{Holitscher, Arthur} (22.\,8.\,1869 Budapest – 14.\,10.\,1941 Genf), \emph{Schriftsteller}|pw}} bei mir. Was iſt das? Er hat zunächſt gegen{ }ſich, daß er von \textsc{Bahr\pwindex{Bahr, Hermann 19.\,7.\,1863 Linz – 15.\,1.\,1934 München@\textsc{Bahr, Hermann} (19.\,7.\,1863 Linz – 15.\,1.\,1934 München), \emph{Schriftsteller, Kritiker}|pw}} empfohlen wird. Auch{ }ſonſt{ }ſieht er mehr nach einem Lausbuben aus, als nach
               irgend etwas Anderem.\pend
           
\pstart
           Der \textsc{Schiller\pwindex{Schiller, Friedrich von 10.\,11.\,1759 Marbach am Neckar – 9.\,5.\,1805 Weimar@\textsc{Schiller, Friedrich von} (10.\,11.\,1759 Marbach am Neckar – 9.\,5.\,1805 Weimar), \emph{Schriftsteller, Historiker, Philosoph}|pw}-Goethesche\pwindex{Goethe, Johann Wolfgang von 28.\,8.\,1749 Frankfurt am Main – 22.\,3.\,1832 Weimar@\textsc{Goethe, Johann Wolfgang von} (28.\,8.\,1749 Frankfurt am Main – 22.\,3.\,1832 Weimar), \emph{Schriftsteller}|pw}}{ }Briefwechſel\pwindex{Schiller, Friedrich von 10.\,11.\,1759 Marbach am Neckar – 9.\,5.\,1805 Weimar@\textsc{Schiller, Friedrich von} (10.\,11.\,1759 Marbach am Neckar – 9.\,5.\,1805 Weimar), \emph{Schriftsteller, Historiker, Philosoph}!Briefwechsel zwischen Schiller und Goethe@\strich\emph{Briefwechsel zwischen Schiller und Goethe}|pw}\pwindex{Goethe, Johann Wolfgang von 28.\,8.\,1749 Frankfurt am Main – 22.\,3.\,1832 Weimar@\textsc{Goethe, Johann Wolfgang von} (28.\,8.\,1749 Frankfurt am Main – 22.\,3.\,1832 Weimar), \emph{Schriftsteller}!Briefwechsel zwischen Schiller und Goethe@\strich\emph{Briefwechsel zwischen Schiller und Goethe}|pw} macht mich{ }ſehr {\pb}nervös. Dieſe Leute, die{ }ſich über nichts als über
               Bücher und{ }ſonſtiges Literariſches{ }ſchreiben! Dieſes unerträglich Gönnerhafte von
               Seiten \textsc{Goethes\pwindex{Goethe, Johann Wolfgang von 28.\,8.\,1749 Frankfurt am Main – 22.\,3.\,1832 Weimar@\textsc{Goethe, Johann Wolfgang von} (28.\,8.\,1749 Frankfurt am Main – 22.\,3.\,1832 Weimar), \emph{Schriftsteller}|pw}}, der den vornehmen Herrn
               gegenüber dem Profeſſor{ }ſpielt (»Mein Wertheſter«, »werther Mann«) und gegenüber dem
               Mann in kleinen Verhältniſſen mit{ }ſeinen Reiſen renommirt, \strikeout{ſ} mit{ }ſeinem Reitpferde (»Ein Ritt von Weimar\oindex{Weimar@\textbf{Weimar}, \emph{Verwaltungsgebiet}|pw} nach \textsc{Jena\oindex{Jena@\textbf{Jena}, \emph{Hauptstadt}|pw}} wird mir gut thun«) \textsc{etc}. Und dieſes nicht minder
               unerträgliche Sich-Geehrt-Fühlen von Seiten \textsc{Schillers\pwindex{Schiller, Friedrich von 10.\,11.\,1759 Marbach am Neckar – 9.\,5.\,1805 Weimar@\textsc{Schiller, Friedrich von} (10.\,11.\,1759 Marbach am Neckar – 9.\,5.\,1805 Weimar), \emph{Schriftsteller, Historiker, Philosoph}|pw}}! Eigentlich drückt{ }ſich nur
                  \textsc{Goethe\pwindex{Goethe, Johann Wolfgang von 28.\,8.\,1749 Frankfurt am Main – 22.\,3.\,1832 Weimar@\textsc{Goethe, Johann Wolfgang von} (28.\,8.\,1749 Frankfurt am Main – 22.\,3.\,1832 Weimar), \emph{Schriftsteller}|pw}} frei aus in dieſer Correſpondenz, {\pb}bei \textsc{Schiller\pwindex{Schiller, Friedrich von 10.\,11.\,1759 Marbach am Neckar – 9.\,5.\,1805 Weimar@\textsc{Schiller, Friedrich von} (10.\,11.\,1759 Marbach am Neckar – 9.\,5.\,1805 Weimar), \emph{Schriftsteller, Historiker, Philosoph}|pw}} merkt man immer die Gedrücktheit. An ihm{ }ſieht man, was für ein
               kleinbürgerlicher \strikeout{a} armer Kerl doch ein deutſcher
               Dichter iſt! Nein, ein Briefwechſel iſt nur erfreulich zwiſchen zwei Gleichſtehenden.
               Ich finde den unſeren viel intereſſanter, als das, was ich bisher von dem zwiſchen
                  \textsc{Goethe\pwindex{Goethe, Johann Wolfgang von 28.\,8.\,1749 Frankfurt am Main – 22.\,3.\,1832 Weimar@\textsc{Goethe, Johann Wolfgang von} (28.\,8.\,1749 Frankfurt am Main – 22.\,3.\,1832 Weimar), \emph{Schriftsteller}|pw}} und \textsc{Schiller\pwindex{Schiller, Friedrich von 10.\,11.\,1759 Marbach am Neckar – 9.\,5.\,1805 Weimar@\textsc{Schiller, Friedrich von} (10.\,11.\,1759 Marbach am Neckar – 9.\,5.\,1805 Weimar), \emph{Schriftsteller, Historiker, Philosoph}|pw}} kenne.\pend
           
\pstart
           Was mit \textsc{Dreyfus\pwindex{Dreyfus, Alfred 9.\,10.\,1859 Mulhouse – 12.\,7.\,1935 Paris@\textsc{Dreyfus, Alfred} (9.\,10.\,1859 Mulhouse – 12.\,7.\,1935 Paris), \emph{Militär}|pw}} weiter wird, fragſt Du? Gar nichts. Der Mann\pwindex{Dreyfus, Alfred 9.\,10.\,1859 Mulhouse – 12.\,7.\,1935 Paris@\textsc{Dreyfus, Alfred} (9.\,10.\,1859 Mulhouse – 12.\,7.\,1935 Paris), \emph{Militär}|pwv} bleibt, wo er iſt, und wird unſchuldig gemordet, wenn
               nicht ein Wunder geſchieht. Die Enthüllungen der Preſſe, welche den {\pb}unerhörten Blödſinn bewieſen, auf dem die Anklage
               aufgebaut iſt, werden hier als niederſchmetternde Schuldbeweiſe betrachtet. Meine
                  \label{K_L02786-4v}\edtext{Artikel\pwindex{Goldmann, Paul 31.\,1.\,1865 Breslau – 25.\,9.\,1935 Wien@\textsc{Goldmann, Paul} (31.\,1.\,1865 Breslau – 25.\,9.\,1935 Wien), \emph{Schriftsteller, Journalist}!Enthüllungen über die Affaire Dreyfus@\strich\emph{Die Enthüllungen über die Affaire Dreyfus}|pwv}}{\lemma{\textnormal{\emph{Artikel}}}\Cendnote{\textnormal{G.\pwindex{Goldmann, Paul 31.\,1.\,1865 Breslau – 25.\,9.\,1935 Wien@\textsc{Goldmann, Paul} (31.\,1.\,1865 Breslau – 25.\,9.\,1935 Wien), \emph{Schriftsteller, Journalist}|pwk} [ = Paul Goldmann\pwindex{Goldmann, Paul 31.\,1.\,1865 Breslau – 25.\,9.\,1935 Wien@\textsc{Goldmann, Paul} (31.\,1.\,1865 Breslau – 25.\,9.\,1935 Wien), \emph{Schriftsteller, Journalist}|pwk}]: \emph{Die Enthüllungen über
                        die Affaire Dreyfus}\pwindex{Goldmann, Paul 31.\,1.\,1865 Breslau – 25.\,9.\,1935 Wien@\textsc{Goldmann, Paul} (31.\,1.\,1865 Breslau – 25.\,9.\,1935 Wien), \emph{Schriftsteller, Journalist}!Enthüllungen über die Affaire Dreyfus@\strich\emph{Die Enthüllungen über die Affaire Dreyfus}|pwk}. In: \emph{Frankfurter
                        Zeitung}\pwindex{Frankfurter Zeitung@\emph{Frankfurter Zeitung}|pwk}, Jg. 41, Nr. 258, 16. 9. 1896, Erstes Morgenblatt,
                     S. 1. Seither war nur eine ungezeichnete Notiz\pwindex{Meldung aus Paris. Brief von Lucie Dreyfus]@\emph{[Meldung aus Paris. Brief von Lucie Dreyfus]}|pwkv} mit einem Brief der Gattin Lucie Dreyfus\pwindex{Dreyfus, Lucie 23.\,8.\,1869 Chatou – 14.\,12.\,1945 Paris@\textsc{Dreyfus, Lucie} (23.\,8.\,1869 Chatou – 14.\,12.\,1945 Paris), \emph{Aktivistin}|pwk} erschienen (\emph{Frankfurter Zeitung}\pwindex{Meldung aus Paris. Brief von Lucie Dreyfus]@\emph{[Meldung aus Paris. Brief von Lucie Dreyfus]}|pwk}, Jg. 41, Nr. 261,
                        19. 9. 1896, Abendblatt, S. 2). Eventuell spielte Goldmann\pwindex{Goldmann, Paul 31.\,1.\,1865 Breslau – 25.\,9.\,1935 Wien@\textsc{Goldmann, Paul} (31.\,1.\,1865 Breslau – 25.\,9.\,1935 Wien), \emph{Schriftsteller, Journalist}|pwk} auf frühere Artikel an, die er seit
                  dem ersten Urteil gegen Alfred Dreyfus\pwindex{Dreyfus, Alfred 9.\,10.\,1859 Mulhouse – 12.\,7.\,1935 Paris@\textsc{Dreyfus, Alfred} (9.\,10.\,1859 Mulhouse – 12.\,7.\,1935 Paris), \emph{Militär}|pwk} im
                     Dezember 1894 publiziert hatte.}}}\label{K_L02786-4} haben nur
               den \uline{einen} Erfolg gehabt, daß{ }ſie \uline{mir} geſchadet haben. Nicht nur daß ich in der Preſſe öffentlich
               beſchimpft worden bin – auch meine fran\oindex{Frankreich@\textbf{Frankreich}|pwv}zöſiſchen Freunde haben mich mit Vorwürfen überſchüttet: »Was geht Sie
               dieſe Geſchichte an? Niemand wird mehr mit Ihnen verkehren können« \textsc{etc.} Wenn mich ein guter {\pb}\label{K_L02786-5v}\edtext{Bekannter in einer
                  Redactionsſtube}{\lemma{\textnormal{\emph{Bekannter … Redactionsstube}}}\Cendnote{\textnormal{nicht
                  identifiziert}}}\label{K_L02786-5} vertheidigen will,{ }ſo wird ihm geantwortet: »Fragen Sie ihn
               nur, welchen Grad er in der deutſch\oindex{Deutschland@\textbf{Deutschland}|pwv}en Reſerve einnimmt« \textsc{etc}. Mangels weiteren
               Materials habe ich natürlich die Campagne einſtellen müſſen. Sobald es aber wieder
               losgeht – und es wird wieder losgehen – fange auch ich wieder an. Es kann mir{ }ſehr{ }ſchlecht dabei gehen – aber das iſt \strikeout{ja} mir
               gleichgiltig. Das iſt ja gerade das Schöne in unſerem Metier, daß {\pb}man die Unſchuldigen vertheidigen und die Schwachen{ }ſchützen kann. \textsc{Don Quixote\pwindex{\textcolor{red}{\textsuperscript{XXXX indx1}}!Don Quijote@\strich\emph{Don Quijote}|pwv}} iſt ein herrliches Vorbild für einen \strikeout{Jou}
               Journaliſten.\pend
           
\pstart
           Wie iſts mit \textsc{Ebermann\pwindex{Ebermann, Leo 16.\,7.\,1863 Draganovka – 9.\,10.\,1914 Wien@\textsc{Ebermann, Leo} (16.\,7.\,1863 Draganovka – 9.\,10.\,1914 Wien), \emph{Schriftsteller, Journalist, Rechtswissenschaftler}|pw}} gegangen? Ich höre, man hat ihn als \label{K_L02786-6v}\edtext{zweiten \textsc{Grillparzer\pwindex{Grillparzer, Franz 15.\,1.\,1791 Wien – 21.\,1.\,1872 ebd.@\textsc{Grillparzer, Franz} (15.\,1.\,1791 Wien – 21.\,1.\,1872 ebd.), \emph{Schriftsteller, Beamter}|pw}}}{\lemma{\textnormal{\emph{zweiten Grillparzer}}}\Cendnote{\textnormal{wohl wegen der mit Werken Grillparzers\pwindex{Grillparzer, Franz 15.\,1.\,1791 Wien – 21.\,1.\,1872 ebd.@\textsc{Grillparzer, Franz} (15.\,1.\,1791 Wien – 21.\,1.\,1872 ebd.), \emph{Schriftsteller, Beamter}|pwk} vergleichbaren Antikisierung in
                  der \emph{Athenerin}\pwindex{Ebermann, Leo 16.\,7.\,1863 Draganovka – 9.\,10.\,1914 Wien@\textsc{Ebermann, Leo} (16.\,7.\,1863 Draganovka – 9.\,10.\,1914 Wien), \emph{Schriftsteller, Journalist, Rechtswissenschaftler}!Athenerin. Drama in drei Aufzügen@\strich\emph{Die Athenerin. Drama in drei Aufzügen}|pwk}}}}\label{K_L02786-6} begrüßt. Und was iſt das für ein Schwindel mit dem \label{K_L02786-7v}\edtext{in Berlin\oindex{Berlin@\textbf{Berlin}, \emph{Hauptstadt}|pw} aufgeführten
                  Stücke\pwindex{Bahr, Hermann 19.\,7.\,1863 Linz – 15.\,1.\,1934 München@\textsc{Bahr, Hermann} (19.\,7.\,1863 Linz – 15.\,1.\,1934 München), \emph{Schriftsteller, Kritiker}!Juana. Drama@\strich\emph{Juana. Drama}|pwv} von \textsc{Bahr\pwindex{Bahr, Hermann 19.\,7.\,1863 Linz – 15.\,1.\,1934 München@\textsc{Bahr, Hermann} (19.\,7.\,1863 Linz – 15.\,1.\,1934 München), \emph{Schriftsteller, Kritiker}|pw}}}{\lemma{\textnormal{\emph{in … Bahr}}}\Cendnote{\textnormal{Bahrs\pwindex{Bahr, Hermann 19.\,7.\,1863 Linz – 15.\,1.\,1934 München@\textsc{Bahr, Hermann} (19.\,7.\,1863 Linz – 15.\,1.\,1934 München), \emph{Schriftsteller, Kritiker}|pwk} Einakter \emph{Juana}\pwindex{Bahr, Hermann 19.\,7.\,1863 Linz – 15.\,1.\,1934 München@\textsc{Bahr, Hermann} (19.\,7.\,1863 Linz – 15.\,1.\,1934 München), \emph{Schriftsteller, Kritiker}!Juana. Drama@\strich\emph{Juana. Drama}|pwk} war am 22. 9. 1896
                  am \emph{Neuen Theater}\orgindex{Neues Theater@Neues Theater|pwk} in Berlin\oindex{Berlin@\textbf{Berlin}, \emph{Hauptstadt}|pwk} uraufgeführt worden. Goldmanns\pwindex{Goldmann, Paul 31.\,1.\,1865 Breslau – 25.\,9.\,1935 Wien@\textsc{Goldmann, Paul} (31.\,1.\,1865 Breslau – 25.\,9.\,1935 Wien), \emph{Schriftsteller, Journalist}|pwk} Vorwurf des »Schwindel« bezieht sich darauf,
                  dass Bahr\pwindex{Bahr, Hermann 19.\,7.\,1863 Linz – 15.\,1.\,1934 München@\textsc{Bahr, Hermann} (19.\,7.\,1863 Linz – 15.\,1.\,1934 München), \emph{Schriftsteller, Kritiker}|pwk} nur als Übersetzer am
                  Theaterzettel stand, als Autorname aber Alejandro Lanza\pwindex{Bahr, Hermann 19.\,7.\,1863 Linz – 15.\,1.\,1934 München@\textsc{Bahr, Hermann} (19.\,7.\,1863 Linz – 15.\,1.\,1934 München), \emph{Schriftsteller, Kritiker}|pwkv} vermerkt war. Bereits die ersten
                  Besprechungen des Stück\pwindex{Bahr, Hermann 19.\,7.\,1863 Linz – 15.\,1.\,1934 München@\textsc{Bahr, Hermann} (19.\,7.\,1863 Linz – 15.\,1.\,1934 München), \emph{Schriftsteller, Kritiker}!Juana. Drama@\strich\emph{Juana. Drama}|pwkv}es
                  konnten berichten, dass es sich dabei um ein Pseudonym Bahrs\pwindex{Bahr, Hermann 19.\,7.\,1863 Linz – 15.\,1.\,1934 München@\textsc{Bahr, Hermann} (19.\,7.\,1863 Linz – 15.\,1.\,1934 München), \emph{Schriftsteller, Kritiker}|pwk} handelte.}}}\label{K_L02786-7}?\pend
           
\pstart
           Grüß’ Dich Gott!\pend
           
\pstart
           Schreib’ bald!\pend
           
\pstart
           Dein treuer {\\[\baselineskip]}\spacefill\mbox{Paul Goldmann.}\pend
           \leftskip=0em{}
\pstart
           \noindent{}Empfiehl’ mich der \label{K_L02786-8v}\edtext{geheimnißvollen
                     Dame\pwindex{Reinhard, Marie 13.\,3.\,1871 Wien – 18.\,3.\,1899 ebd.@\textsc{Reinhard, Marie} (13.\,3.\,1871 Wien – 18.\,3.\,1899 ebd.), \emph{Gesangspädagogin}|pwuv}}{\lemma{\textnormal{\emph{geheimnißvollen
                     Dame}}}\Cendnote{\textnormal{Siehe XXXX Auszeichnungsfehler: Dokument L02684 nicht gefunden.
                  }}}\label{K_L02786-8}!\pend
           \selectlanguage{ngerman}\vspace{1em}{\vspace{1\baselineskip}}
\pstart
           \raggedleft{}{\pb}{[}hs. Thorel:{]} \begin{otherlanguage}{french}12 rue de Milan\oindex{Rue de Milan@\textbf{Rue de Milan}, \emph{Straße}|pw}{\\}\label{K_L02786-9v}\edtext{jeudi}{\lemma{\textnormal{\emph{jeudi}}}\Cendnote{\textnormal{französisch: Donnerstag. Der Brief
                        könnte demnach vom Vortag, dem 25. 9. 1896,
                        stammen.}}}\label{K_L02786-9}.\end{otherlanguage}\pend
           
\pstart\center{}\begin{otherlanguage}{french}Cher monsieur Goldmann,\end{otherlanguage}\pend\vspace{0.5em}
\pstart
           \label{K_L02786-10v}\edtext{\begin{otherlanguage}{french}Je suis en plein travail – j’ai déjà presque fini le premier
                     acte\pwindex{Schnitzler, Arthur 15.\,5.\,1862 Wien – 21.\,10.\,1931 ebd.@\textsc{Schnitzler, Arthur} (15.\,5.\,1862 Wien – 21.\,10.\,1931 ebd.), \emph{Schriftsteller, Mediziner}!Amourette. Pièce en trois actes. Adaptée de Arthur Schnitzler@\strich\emph{Amourette. Pièce en trois actes. Adaptée de Arthur Schnitzler}|pwv} – j’aurais voulu
                  vous le montrer, mais mes dates de voyage et de passage à Paris\oindex{Paris@\textbf{Paris}, \emph{Hauptstadt}|pw} ont été un peu brouillées, et je depars tout à l’heure
                  pour Auxerre\oindex{Auxerre@\textbf{Auxerre}, \emph{Hauptstadt}|pw} où je resterai une huitaine de
                  jours.\end{otherlanguage}}{\lemma{\textnormal{\emph{Je … jours.}}}\Cendnote{\textnormal{französisch: Ich bin mitten in der
                  Arbeit – ich habe den ersten Akt\pwindex{Schnitzler, Arthur 15.\,5.\,1862 Wien – 21.\,10.\,1931 ebd.@\textsc{Schnitzler, Arthur} (15.\,5.\,1862 Wien – 21.\,10.\,1931 ebd.), \emph{Schriftsteller, Mediziner}!Amourette. Pièce en trois actes. Adaptée de Arthur Schnitzler@\strich\emph{Amourette. Pièce en trois actes. Adaptée de Arthur Schnitzler}|pwkv} schon fast fertig – ich hätte ihn Ihnen gerne gezeigt, aber meine
                  Reise- und Aufenthaltsdaten in Paris\oindex{Paris@\textbf{Paris}, \emph{Hauptstadt}|pwk} sind ein
                  wenig durcheinander geraten, und ich fahre umgehend nach Auxerre\oindex{Auxerre@\textbf{Auxerre}, \emph{Hauptstadt}|pwk}, wo ich etwa acht Tage bleiben werde.}}}\label{K_L02786-10}\pend
           
\pstart
           \label{K_L02786-11v}\edtext{\begin{otherlanguage}{french}Sitôt rentré, je vous verrai, et je terminerai.\end{otherlanguage}}{\lemma{\textnormal{\emph{Sitôt … terminerai.}}}\Cendnote{\textnormal{französisch: Sobald ich zurück bin,
                  werde ich Sie sehen, und es beenden.}}}\label{K_L02786-11}\pend
           
\pstart
           \label{K_L02786-12v}\edtext{\begin{otherlanguage}{french}\textcolor{gray}{A mesure que je la pénètre davantage}, je me rends de plus en
                  plus compte combien c’est exquis, cette petite pièce\pwindex{Schnitzler, Arthur 15.\,5.\,1862 Wien – 21.\,10.\,1931 ebd.@\textsc{Schnitzler, Arthur} (15.\,5.\,1862 Wien – 21.\,10.\,1931 ebd.), \emph{Schriftsteller, Mediziner}!Liebelei. Schauspiel in drei Akten@\strich\emph{Liebelei. Schauspiel in drei Akten}|pwv}; et, avec cela, d’une habileté consommée. Et nous
                  aurons fait là un joli cadeau aux Paris\oindex{Paris@\textbf{Paris}, \emph{Hauptstadt}|pw}iens.\end{otherlanguage}}{\lemma{\textnormal{\emph{A … Parisiens.}}}\Cendnote{\textnormal{französisch: Umso weiter ich vordringe,
                  desto mehr merke ich, wie besonders dieses kleine Stück\pwindex{Schnitzler, Arthur 15.\,5.\,1862 Wien – 21.\,10.\,1931 ebd.@\textsc{Schnitzler, Arthur} (15.\,5.\,1862 Wien – 21.\,10.\,1931 ebd.), \emph{Schriftsteller, Mediziner}!Liebelei. Schauspiel in drei Akten@\strich\emph{Liebelei. Schauspiel in drei Akten}|pwkv} ist; und wie geschickt es gemacht
                  ist. Und wir werden den Paris\oindex{Paris@\textbf{Paris}, \emph{Hauptstadt}|pwk}ern ein schönes
                  Geschenk machen.}}}\label{K_L02786-12}\pend
           
\pstart
           \label{K_L02786-13v}\edtext{\begin{otherlanguage}{french}Bien à vous\end{otherlanguage}}{\lemma{\textnormal{\emph{Bien à vous}}}\Cendnote{\textnormal{französisch: Der Ihre}}}\label{K_L02786-13}{ }{\\[\baselineskip]}\spacefill\mbox{Jean Thorel\pwindex{Thorel, Jean 11.\,9.\,1859 Éragny – 20.\,8.\,1916 Enghien-les-Bains@\textsc{Thorel, Jean} (11.\,9.\,1859 Éragny – 20.\,8.\,1916 Enghien-les-Bains), \emph{Übersetzer, Dramatiker}|pw}}\pend
           \leftskip=0em{}\selectlanguage{ngerman}\vspace{1em}{\vspace{1\baselineskip}}
\pstart
           \raggedleft{}{\pb}{[}hs. Nansen:{]} Kopenhagen\oindex{Kopenhagen@\textbf{Kopenhagen}, \emph{Hauptstadt}|pw}{ }20 Sept. 96\pend
           
\pstart{}Lieber Herr Goldmann!\pend\vspace{0.5em}
\pstart
           Wenn ich nicht eher geschrieben habe, ist der Grund meine Manieristische Furcht für
               die deutsche Sprache. Oft habe ich \introOben{}an\introOben{} Ihnen gedacht, an
               Ihnen und Ihren Freunden. Ja, lieber Herr, Freundschaft und Sympathie kann man sich
               nicht verklaren. Vom ersten Tag’, ich Sie sah, habe ich Sie lieb, und ich hoffe, wie
               Sie, dass unsre Freundschaft in aller Zukunft dauern wird – {\pb}auch wenn ich ein schlechter Briefschreiber bin.\pend
           
\pstart
           Ich vergesse aber ganz meinen Dank \strikeout{\textcolor{gray}{z}} und den meiner Frau\pwindex{Nansen, Betty 19.\,3.\,1873 Kopenhagen – 15.\,3.\,1943 ebd.@\textsc{Nansen, Betty} (19.\,3.\,1873 Kopenhagen – 15.\,3.\,1943 ebd.), \emph{Theaterleiterin, Schauspielerin}|pwv} zu
               bringen für die Zusendung der franzoesi{[}s{]}chen Chansons. Meine Frau\pwindex{Nansen, Betty 19.\,3.\,1873 Kopenhagen – 15.\,3.\,1943 ebd.@\textsc{Nansen, Betty} (19.\,3.\,1873 Kopenhagen – 15.\,3.\,1943 ebd.), \emph{Theaterleiterin, Schauspielerin}|pwv} freut sich sehr sie zu
               singen – ich sie zu hören.\pend
           
\pstart
           Ich bin jetzt Subscribent der Frankf. Zeitung\pwindex{Frankfurter Zeitung@\emph{Frankfurter Zeitung}|pw}{ }\strikeout{\textcolor{gray}{g}} und habe neulich da eine ausgezeichnete \introOben{}Dreyfus\pwindex{Dreyfus, Alfred 9.\,10.\,1859 Mulhouse – 12.\,7.\,1935 Paris@\textsc{Dreyfus, Alfred} (9.\,10.\,1859 Mulhouse – 12.\,7.\,1935 Paris), \emph{Militär}|pwv}-\introOben{}Feuilleton\pwindex{Goldmann, Paul 31.\,1.\,1865 Breslau – 25.\,9.\,1935 Wien@\textsc{Goldmann, Paul} (31.\,1.\,1865 Breslau – 25.\,9.\,1935 Wien), \emph{Schriftsteller, Journalist}!Enthüllungen über die Affaire Dreyfus@\strich\emph{Die Enthüllungen über die Affaire Dreyfus}|pwv} von Ihnen gelesen. Das ist das beste, was ich
               von dieser merkwürdigen Sache gelesen.\pend
           
\pstart
           (Ich schreibe so undeutlich {[}um{]} meine Sprachfehler zu
               verbergen)\pend
           \selectlanguage{ngerman}\vspace{1em}
\pstart
           \noindent{}– – Ich wurde gestern in meinem {\pb}Schreiben unterbrochen und setze jetzt fort, d. 21.{ }\label{K_L02786-14v}\edtext{hujus}{\lemma{\textnormal{\emph{hujus}}}\Cendnote{\textnormal{lateinisch: von diesem [Monat]}}}\label{K_L02786-14}.\pend
           
\pstart
           Meine Frau\pwindex{Nansen, Betty 19.\,3.\,1873 Kopenhagen – 15.\,3.\,1943 ebd.@\textsc{Nansen, Betty} (19.\,3.\,1873 Kopenhagen – 15.\,3.\,1943 ebd.), \emph{Theaterleiterin, Schauspielerin}|pwv} hat
                  i{[}n{]} diesen Tagen im königlichen
                  Theater\orgindex{Det Kongelige Teater@Det Kongelige Teater|pw} ihre Entrée gehabt mit grossem Erfolg. In einer kleinen Ibsen\pwindex{Ibsen, Henrik 20.\,3.\,1828 Skien – 23.\,5.\,1906 Oslo@\textsc{Ibsen, Henrik} (20.\,3.\,1828 Skien – 23.\,5.\,1906 Oslo), \emph{Schriftsteller}|pw}-Rolle. Frl. Bernick\pwindex{Ibsen, Henrik 20.\,3.\,1828 Skien – 23.\,5.\,1906 Oslo@\textsc{Ibsen, Henrik} (20.\,3.\,1828 Skien – 23.\,5.\,1906 Oslo), \emph{Schriftsteller}!Samfundets Støtter. Skuespil i fire Akter@\strich\emph{Samfundets Støtter. Skuespil i fire Akter}|pwv} in »Stützen der Gesellschaft\pwindex{Ibsen, Henrik 20.\,3.\,1828 Skien – 23.\,5.\,1906 Oslo@\textsc{Ibsen, Henrik} (20.\,3.\,1828 Skien – 23.\,5.\,1906 Oslo), \emph{Schriftsteller}!Samfundets Støtter. Skuespil i fire Akter@\strich\emph{Samfundets Støtter. Skuespil i fire Akter}|pw}«.\pend
           
\pstart
           Dieses Jahr werde ich deutsch\oindex{Deutschland@\textbf{Deutschland}|pwv}er
               Journalist. Der vortreffliche Herr Fischer\pwindex{Fischer, Samuel 24.\,12.\,1859 Liptovský Mikuláš – 15.\,10.\,1934 Berlin@\textsc{Fischer, Samuel} (24.\,12.\,1859 Liptovský Mikuláš – 15.\,10.\,1934 Berlin), \emph{Verleger}|pw} hat
               mich engagiert vier Briefe vom Norden\pwindex{Nansen, Peter 20.\,1.\,1861 Kopenhagen – 31.\,7.\,1918 Mariager@\textsc{Nansen, Peter} (20.\,1.\,1861 Kopenhagen – 31.\,7.\,1918 Mariager), \emph{Schriftsteller, Journalist, Verleger}!Brief aus dem Norden@\strich\emph{Brief aus dem Norden}|pwv}\pwindex{Nansen, Peter 20.\,1.\,1861 Kopenhagen – 31.\,7.\,1918 Mariager@\textsc{Nansen, Peter} (20.\,1.\,1861 Kopenhagen – 31.\,7.\,1918 Mariager), \emph{Schriftsteller, Journalist, Verleger}!Briefe aus dem Norden. II. Das Kopenhagener Theater@\strich\emph{Briefe aus dem Norden. II. Das Kopenhagener Theater}|pwv} in »Neue deutsche
                  Rundschau\pwindex{Neue Deutsche Rundschau@\emph{Neue Deutsche Rundschau}|pw}« zu schreiben. Den \label{K_L02786-15v}\edtext{ersten Brief}{\lemma{\textnormal{\emph{ersten Brief}}}\Cendnote{\textnormal{Peter Nansen\pwindex{Nansen, Peter 20.\,1.\,1861 Kopenhagen – 31.\,7.\,1918 Mariager@\textsc{Nansen, Peter} (20.\,1.\,1861 Kopenhagen – 31.\,7.\,1918 Mariager), \emph{Schriftsteller, Journalist, Verleger}|pwk}: \emph{Brief aus dem Norden}\pwindex{Nansen, Peter 20.\,1.\,1861 Kopenhagen – 31.\,7.\,1918 Mariager@\textsc{Nansen, Peter} (20.\,1.\,1861 Kopenhagen – 31.\,7.\,1918 Mariager), \emph{Schriftsteller, Journalist, Verleger}!Brief aus dem Norden@\strich\emph{Brief aus dem Norden}|pwk}. In: \emph{Neue Deutsche Rundschau}\pwindex{Neue Deutsche Rundschau@\emph{Neue Deutsche Rundschau}|pwk}, Jg. 7 (1896),
                     Oktober, S. 1028–1033. Der nächste Brief\pwindex{Nansen, Peter 20.\,1.\,1861 Kopenhagen – 31.\,7.\,1918 Mariager@\textsc{Nansen, Peter} (20.\,1.\,1861 Kopenhagen – 31.\,7.\,1918 Mariager), \emph{Schriftsteller, Journalist, Verleger}!Briefe aus dem Norden. II. Das Kopenhagener Theater@\strich\emph{Briefe aus dem Norden. II. Das Kopenhagener Theater}|pwkv} erschien im März-Heft\pwindex{Neue Deutsche Rundschau@\emph{Neue Deutsche Rundschau}|pwkv}{ }1897.}}}\label{K_L02786-15} habe ich schon fertig. Der kommt im
               October-Hefte.\pend
           
\pstart
           Sie schreiben natürlich oft an Herrn Schnitzler und Beer-Hofmann\pwindex{Beer-Hofmann, Richard 11.\,7.\,1866 Wien – 26.\,9.\,1945 New York City@\textsc{Beer-Hofmann, Richard} (11.\,7.\,1866 Wien – 26.\,9.\,1945 New York City), \emph{Schriftsteller}|pw}. Sagen – bitte – den zwei liebenswertesten Menschen\pwindex{Beer-Hofmann, Richard 11.\,7.\,1866 Wien – 26.\,9.\,1945 New York City@\textsc{Beer-Hofmann, Richard} (11.\,7.\,1866 Wien – 26.\,9.\,1945 New York City), \emph{Schriftsteller}|pwv}, dass sie mir nicht böse sein
               dürfen, weil sie nichts von mir noch gehört haben. Sie wissen ja alle {\pb}Drei\pwindex{Beer-Hofmann, Richard 11.\,7.\,1866 Wien – 26.\,9.\,1945 New York City@\textsc{Beer-Hofmann, Richard} (11.\,7.\,1866 Wien – 26.\,9.\,1945 New York City), \emph{Schriftsteller}|pwv} den legitimen Grund
               meiner Stummheit.\pend
           
\pstart
           Ach – könnten Sie nur alle Drei\pwindex{Beer-Hofmann, Richard 11.\,7.\,1866 Wien – 26.\,9.\,1945 New York City@\textsc{Beer-Hofmann, Richard} (11.\,7.\,1866 Wien – 26.\,9.\,1945 New York City), \emph{Schriftsteller}|pwv} recht oft \strikeout{\textcolor{gray}{ein}} Abendvisiten machen und mit uns plaudern und lachen und bisweilen – weil es
               auch gut ist – ein bischen sentimental sein.\pend
           
\pstart
           Lieber Freund – ich sende Ihnen alle meine besten Grüsse und meine Frau\pwindex{Nansen, Betty 19.\,3.\,1873 Kopenhagen – 15.\,3.\,1943 ebd.@\textsc{Nansen, Betty} (19.\,3.\,1873 Kopenhagen – 15.\,3.\,1943 ebd.), \emph{Theaterleiterin, Schauspielerin}|pwv} fügt ihre Grüsse zu den meinigen.\pend
           
\pstart
           Vergessen Sie uns nicht \substVorne{}\textsuperscript{z}\substDazwischen{}u\substHinten{}nd schreiben Sie bald wieder.{\\[\baselineskip]} Ihr ergebener{\\[\baselineskip]}\spacefill\mbox{Peter Nansen}\pend
           \leftskip=0em{}\selectlanguage{ngerman}\endnumbering\briefempfaengerindex{Schnitzler, Arthur@\textsc{Schnitzler, Arthur}!zzzGoldmann, Paul@\emph{von Paul Goldmann}!1896-09-261@{26. 9. [1896]}|)be}\mylabel{L02786h}  \newcommand{\dateiname}{L02786}\newcommand{\titel}{Paul Goldmann an Arthur Schnitzler, 26. 9. [1896]}\newcommand{\editorInnen}{Martin Anton Müller und Laura Untner}%% latex-leseansicht-abspann.tex
%% Abspann für die Leseansicht.
%% Der Schalter \ifkorrekturansicht ist bereits durch den Vorspann gesetzt.

%% latex-abspann.tex
%% Gemeinsamer Abspann für Korrekturansicht und Leseansicht.
%% Setzt den Schalter \ifkorrekturansicht voraus (gesetzt in den
%% einbindenden Dateien latex-korrekturansicht-abspann.tex bzw.
%% latex-leseansicht-abspann.tex).
%% ---------------------------------------------------------------

\normalsize

% Das esempio-Environment wird nur in der Leseansicht benötigt
\ifkorrekturansicht\else
\newenvironment{esempio}[3]%
{
    \vspace{1.5ex}
    \rlap{\underline{#1}}
    \par
    \setlength{\parindent}{0cm}
    \nopagebreak
    \leftskip=#2cm
    \rightskip=#3cm
}
{
    \par
}
\fi

\doendnotes{C}
\bigskip
\vfill

\clearpage

\footnotesize

\ifkorrekturansicht
  \lohead{\textsc{register}}
\fi

% theindex-Environment neu definieren ohne reledmac
\makeatletter
\renewenvironment{theindex}{%
  \ifkorrekturansicht
    \section*{\indexname}%
  \else
    \subsubsection*{Index der erwähnten Entitäten}%
  \fi
  \setlength{\parindent}{0pt}%
  \setlength{\parskip}{0pt plus 0.3pt}%
  \let\item\@idxitem
}{%
  \ifkorrekturansicht\clearpage\fi
}
\makeatother

\IfFileExists{\jobname-pw.ind}{\input{\jobname-pw.ind}}{}

% Quellenangabe nur in der Leseansicht
\ifkorrekturansicht\else
% Fallback-Definitionen, falls die .tex-Datei \titel etc. nicht gesetzt hat
\providecommand{\titel}{}
\providecommand{\editorInnen}{}
\providecommand{\dateiname}{\jobname}

\vspace{3cm}

\vfill

\footnotesize
\textsc{Quelle}: \titel. Herausgegeben von {\editorInnen}. In: \emph{Arthur Schnitzler: Briefwechsel mit Autorinnen und Autoren}.
 Digitale Edition, https://schnitzler-briefe.acdh.oeaw.ac.at/{\dateiname}.html (Stand \today)
\fi

\end{document}


