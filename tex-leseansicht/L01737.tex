%% latex-korrekturansicht-vorspann.tex
%% Vorspann für die Korrekturansicht.
%% Lädt die gemeinsame Datei latex-vorspann.tex mit gesetztem Schalter.

\newif\ifkorrekturansicht
\korrekturansichttrue

\input{../tex-inputs/latex-vorspann}


\section[Richard Beer-Hofmann an Arthur Schnitzler, {[}7. 12. 1907{]}]{L01737 Richard Beer-Hofmann an Arthur Schnitzler, {[}7. 12. 1907{]}}
\nopagebreak\mylabel{L01737v}
\rehead{ }\normalsize\beginnumbering\briefempfaengerindex{Schnitzler, Arthur@\textsc{Schnitzler, Arthur}!zzzBeer-Hofmann, Richard@\emph{von Richard Beer-Hofmann}!1907-12-071@{{[}7. 12. 1907{]}}|(be}
\toendnotes[C]{\smallbreak\pagebreak[2]}\Standort{CUL, Schnitzler, B 8.}
\physDesc{Visitenkarte, 141 Zeichen
\newline{}Handschrift: Bleistift, lateinische Kurrent
\newline{}Schnitzler: mit Bleistift datiert: »? 907 17/12« 
\newline{}Ordnung: mit Bleistift von unbekannter Hand nummeriert:
                                    »176« }
\buchAbdrucke{\weitereDrucke{Arthur Schnitzler, Richard Beer-Hofmann: \emph{Briefwechsel 1891–1931}. Wien, Zürich: \emph{Europaverlag} 1992, S. 187.} }\toendnotes[C]{\smallbreak}
\pstart
           \noindent{}{\pb}Lieber Arthur!{ }Montag um halbsechs – wenn es Ihnen so recht ist. Bitte
               sagen Sie dem Mädchen\pwindex{?? [Haushaltshilfe von Beer-Hofmann] 1907 – 1907@\textsc{?? [Haushaltshilfe von Beer-Hofmann]} (1907 – 1907)|pwv}{ }{\pb}das die Karte bringt, wie es Frau
                  Olga\pwindex{Schnitzler, Olga 17.01.1882 – 13.01.1970@\textsc{Schnitzler, Olga} (17.01.1882 – 13.01.1970), \emph{Schauspieler/Schauspielerin, Sänger/Sängerin}|pw} geht.\pend
           
\pstart
           Herzlichst{\\[\baselineskip]}\spacefill\mbox{R}\pend
           \leftskip=0em{}
\pstart
           \noindent{}\centering{}\textcolor{gray}{\textbf{RICHARD{ }\strikeout{BEER-HOFMANN}}}\pend
           \selectlanguage{ngerman}\endnumbering\briefempfaengerindex{Schnitzler, Arthur@\textsc{Schnitzler, Arthur}!zzzBeer-Hofmann, Richard@\emph{von Richard Beer-Hofmann}!1907-12-071@{{[}7. 12. 1907{]}}|)be}\mylabel{L01737h}  \normalsize

\doendnotes{C}
\bigskip
\vfill

\clearpage

\footnotesize

\lohead{\textsc{register}}

% Definiere theindex-Environment komplett neu ohne reledmac
\makeatletter
\renewenvironment{theindex}{%
  \section*{\indexname}%
  \setlength{\parindent}{0pt}%
  \setlength{\parskip}{0pt plus 0.3pt}%
  \let\item\@idxitem
}{%
  \clearpage
}
\makeatother

\IfFileExists{\jobname-pw.ind}{\input{\jobname-pw.ind}}{}

\end{document}

      