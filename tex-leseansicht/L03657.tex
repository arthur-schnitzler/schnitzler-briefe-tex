%% latex-leseansicht-vorspann.tex
%% Vorspann für die Leseansicht.
%% Lädt die gemeinsame Datei latex-vorspann.tex mit nicht gesetztem Schalter.

\newif\ifkorrekturansicht
\korrekturansichtfalse

\input{../tex-inputs/latex-vorspann}


\section[Stefan Zweig an Arthur Schnitzler, 18. 1. 1916]{L03657 Stefan Zweig an Arthur Schnitzler, 18. 1. 1916}
\nopagebreak\mylabel{L03657v}
\rehead{ }\normalsize\beginnumbering\briefempfaengerindex{Schnitzler, Arthur@\textsc{Schnitzler, Arthur}!zzzZweig, Stefan@\emph{von Stefan Zweig}!1916-01-181@{18. 1. 1916}|(be}
\toendnotes[C]{\smallbreak\pagebreak[2]}
\correspDesc{Versand  durch Stefan Zweig am 18. 1. 1916 in Wien
\newline{}Erhalt  durch Arthur Schnitzler im Zeitraum [19. 1. 1916
                  – 23. 1. 1916?] in Wien}\toendnotes[C]{\smallbreak}
\Standort{CUL, Schnitzler, B 118.}
\physDesc{Brief, 1 Blatt, 2 Seiten, 639 Zeichen
\newline{}Handschrift: schwarze Tinte, lateinische Kurrent
\newline{}Schnitzler: 1) mit Bleistift »\textsc{Zweig}«  2) mit rotem Buntstift eine Unterstreichung}
\buchAbdrucke{\weitereDrucke{1) Stefan Zweig: \emph{Briefwechsel mit Hermann Bahr, Sigmund Freud, Rainer Maria
                        Rilke und Arthur Schnitzler}. Herausgegeben von Jeffrey B. Berlin, Hans-Ulrich Lindken und Donald A. Prater. Frankfurt am Main: \emph{S. Fischer} 1987, S. 397–398.} \weitereDrucke{2) Stefan Zweig: \emph{Briefe. Bd. II: 1914–1919}. Herausgegeben von Knut Beck, Jeffrey B. Berlin und Natascha Weschenbach-Feggeler. Frankfurt am Main: \emph{S. Fischer} 1998, S. 100–101.} }\toendnotes[C]{\smallbreak}
\pstart
           \raggedleft{}{\pb}18. Januar 1916\pend
           
\pstart
           \textcolor{gray}{\textbf{SZ}}\hfill \textcolor{gray}{\textbf{VIII. KOCHGASSE\oindex{Wien@\textbf{Wien}!VIII., Josefstadt@\textbf{VIII., Josefstadt}!Kochgasse 8@\textbf{Kochgasse 8}, \emph{Wohngebäude}|pw}}}\pend
           
\pstart
           \raggedleft{}\textcolor{gray}{\textbf{WIEN\oindex{Wien@\textbf{Wien}, \emph{Verwaltungsgebiet}|pw},}}\pend
           
\pstart{}Lieber verehrter Herr Doktor,\pend\vspace{0.5em}
\pstart
           darf ich wieder einmal \label{K_L03657-1v}\edtext{zu Ihnen
                  kommen}{\lemma{\textnormal{\emph{zu Ihnen
                  kommen}}}\Cendnote{\textnormal{Vgl. A. S.: \emph{Tagebuch}, 21. 1. 1916.}}}\label{K_L03657-1}? Oder mögen Sie Menschen jetzt nicht
               sehen? Ich würde auch dies verstehn – die Worte und Gespräche werden einem manchmal
               jetzt verhasst, man weiss wie nutzlos wie unwissend \substVorne{}\textsuperscript{S}\substDazwischen{}s\substHinten{}ie sind.\pend
           
\pstart
           Ich möchte bei dieser Gelegenheit auch Ihren Rat \label{K_L03657-2v}\edtext{in Sachen Rilkes\pwindex{Rilke, Rainer Maria 4.\,12.\,1875 Prag – 29.\,12.\,1926 Montreux@\textsc{Rilke, Rainer Maria} (4.\,12.\,1875 Prag – 29.\,12.\,1926 Montreux), \emph{Schriftsteller}|pw}}{\lemma{\textnormal{\emph{in Sachen Rilkes}}}\Cendnote{\textnormal{Beim Treffen am 21. 1. 1916
                  unterbreitete Zweig\pwindex{Zweig, Stefan 28.\,11.\,1881 Wien – 23.\,2.\,1942 Petrópolis@\textsc{Zweig, Stefan} (28.\,11.\,1881 Wien – 23.\,2.\,1942 Petrópolis), \emph{Schriftsteller}|pwk}{ }Schnitzler den Vorschlag einer Eingabe beim
                  zuständigen Minister. Als Folge der Aktivitäten Zweigs\pwindex{Zweig, Stefan 28.\,11.\,1881 Wien – 23.\,2.\,1942 Petrópolis@\textsc{Zweig, Stefan} (28.\,11.\,1881 Wien – 23.\,2.\,1942 Petrópolis), \emph{Schriftsteller}|pwk} wurde Rilke\pwindex{Rilke, Rainer Maria 4.\,12.\,1875 Prag – 29.\,12.\,1926 Montreux@\textsc{Rilke, Rainer Maria} (4.\,12.\,1875 Prag – 29.\,12.\,1926 Montreux), \emph{Schriftsteller}|pwk} nach der
                  Grundausbildung zu ihm ins \emph{Kriegsarchiv}\orgindex{Kriegsarchiv@Kriegsarchiv|pwk} versetzt. }}}\label{K_L03657-2} erbitten, der eingerückt
               ist und der (aus vielen Gründen) sehr leidet. Vielleicht könnten Wir durch eine
               gemeinsame Initiative ihm helfen. Und {\pb}wer verdient es, wenn nicht er?\pend
           
\pstart
           Getreulichst (mit vielen Grüssen an Ihre liebe Frau\pwindex{Schnitzler, Olga 17.\,1.\,1882 Wien – 13.\,1.\,1970 Lugano@\textsc{Schnitzler, Olga} (17.\,1.\,1882 Wien – 13.\,1.\,1970 Lugano), \emph{Schauspielerin, Sängerin}|pwv} und Sie){\\[\baselineskip]} Ihr{\\[\baselineskip]}\spacefill\mbox{Stefan Zweig}\pend
           \leftskip=0em{}
\pstart
           \noindent{}\uline{P. S.} Ich bin (ausser Mittwoch)
                  immer frei, nachmittags oder abends.\pend
           \selectlanguage{ngerman}\endnumbering\briefempfaengerindex{Schnitzler, Arthur@\textsc{Schnitzler, Arthur}!zzzZweig, Stefan@\emph{von Stefan Zweig}!1916-01-181@{18. 1. 1916}|)be}\mylabel{L03657h}  \newcommand{\dateiname}{L03657}\newcommand{\titel}{Stefan Zweig an Arthur Schnitzler, 18. 1. 1916}\newcommand{\editorInnen}{Selma Jahnke und Martin Anton Müller}%% latex-leseansicht-abspann.tex
%% Abspann für die Leseansicht.
%% Der Schalter \ifkorrekturansicht ist bereits durch den Vorspann gesetzt.

%% latex-abspann.tex
%% Gemeinsamer Abspann für Korrekturansicht und Leseansicht.
%% Setzt den Schalter \ifkorrekturansicht voraus (gesetzt in den
%% einbindenden Dateien latex-korrekturansicht-abspann.tex bzw.
%% latex-leseansicht-abspann.tex).
%% ---------------------------------------------------------------

\normalsize

% Das esempio-Environment wird nur in der Leseansicht benötigt
\ifkorrekturansicht\else
\newenvironment{esempio}[3]%
{
    \vspace{1.5ex}
    \rlap{\underline{#1}}
    \par
    \setlength{\parindent}{0cm}
    \nopagebreak
    \leftskip=#2cm
    \rightskip=#3cm
}
{
    \par
}
\fi

\doendnotes{C}
\bigskip
\vfill

\clearpage

\footnotesize

\ifkorrekturansicht
  \lohead{\textsc{register}}
\fi

% theindex-Environment neu definieren ohne reledmac
\makeatletter
\renewenvironment{theindex}{%
  \ifkorrekturansicht
    \section*{\indexname}%
  \else
    \subsubsection*{Index der erwähnten Entitäten}%
  \fi
  \setlength{\parindent}{0pt}%
  \setlength{\parskip}{0pt plus 0.3pt}%
  \let\item\@idxitem
}{%
  \ifkorrekturansicht\clearpage\fi
}
\makeatother

\IfFileExists{\jobname-pw.ind}{\input{\jobname-pw.ind}}{}

% Quellenangabe nur in der Leseansicht
\ifkorrekturansicht\else
% Fallback-Definitionen, falls die .tex-Datei \titel etc. nicht gesetzt hat
\providecommand{\titel}{}
\providecommand{\editorInnen}{}
\providecommand{\dateiname}{\jobname}

\vspace{3cm}

\vfill

\footnotesize
\textsc{Quelle}: \titel. Herausgegeben von {\editorInnen}. In: \emph{Arthur Schnitzler: Briefwechsel mit Autorinnen und Autoren}.
 Digitale Edition, https://schnitzler-briefe.acdh.oeaw.ac.at/{\dateiname}.html (Stand \today)
\fi

\end{document}


