%% latex-korrekturansicht-vorspann.tex
%% Vorspann für die Korrekturansicht.
%% Lädt die gemeinsame Datei latex-vorspann.tex mit gesetztem Schalter.

\newif\ifkorrekturansicht
\korrekturansichttrue

\input{../tex-inputs/latex-vorspann}


\section[Stefan Zweig an Arthur Schnitzler, 18. 1. 1916]{L03657 Stefan Zweig an Arthur Schnitzler, 18. 1. 1916}
\nopagebreak\mylabel{L03657v}
\rehead{ }\normalsize\beginnumbering\briefempfaengerindex{Schnitzler, Arthur@\textsc{Schnitzler, Arthur}!zzzZweig, Stefan@\emph{von Stefan Zweig}!1916-01-181@{18. 1. 1916}|(be}
\toendnotes[C]{\smallbreak\pagebreak[2]}\Standort{CUL, Schnitzler, B 118.}
\physDesc{Brief, 1 Blatt, 2 Seiten, 638 Zeichen
\newline{}Handschrift: schwarze Tinte, lateinische Kurrent
\newline{}Schnitzler: 1) mit Bleistift »\textsc{Zweig}«  2) mit rotem Buntstift eine Unterstreichung}
\buchAbdrucke{\weitereDrucke{1) Stefan Zweig: \emph{Briefwechsel mit Hermann Bahr, Sigmund Freud, Rainer Maria
                        Rilke und Arthur Schnitzler}. Frankfurt am Main: \emph{S. Fischer} 1987, S. 397–398.} \weitereDrucke{2) Stefan Zweig: \emph{Briefe. Bd. II: 1914–1919}. Frankfurt am Main: \emph{S. Fischer} 1998, S. 100–101.} }\toendnotes[C]{\smallbreak}
\pstart
           {\pb}\textcolor{gray}{\textbf{SZ}}\hfill 18. Januar 1916\pend
           
\pstart
           \raggedleft{}\textcolor{gray}{\textbf{VIII. KOCHGASSE\oindex{Kochgasse 8@\textbf{Kochgasse 8}, \emph{Wohngebäude (K.WHS)}|pw}}}\pend
           
\pstart
           \raggedleft{}\textcolor{gray}{\textbf{WIEN\oindex{Wien@\textbf{Wien}, \emph{A.ADM2}|pw},}}\pend
           
\pstart{}Lieber verehrter Herr Doktor,\pend\vspace{0.5em}
\pstart
           darf ich wieder einmal \label{K_L03657-1v}\edtext{zu Ihnen
                  kommen}{\lemma{\textnormal{\emph{zu Ihnen
                  kommen}}}\Cendnote{\textnormal{vgl. A. S.: \emph{Tagebuch}, 21. 1. 1916.}}}\label{K_L03657-1}? Oder
               mögen Sie Menschen jetzt nicht sehen. Ich würde auch dies verstehn – die Worte und
               Gespräche werden einem manchmal jetzt verhasst, man weiss, wie nutzlos wie unwissend \substVorne{}\textsuperscript{S}\substDazwischen{}s\substHinten{}ie sind.\pend
           
\pstart
           Ich möchte bei dieser Gelegenheit auch Ihren Rat \label{K_L03657-2v}\edtext{in Sachen Rilkes\pwindex{Rilke, Rainer Maria 04.12.1875 – 29.12.1926@\textsc{Rilke, Rainer Maria} (04.12.1875 – 29.12.1926), \emph{Schriftsteller/Schriftstellerin}|pw}}{\lemma{\textnormal{\emph{in Sachen Rilkes}}}\Cendnote{\textnormal{Beim Treffen am 21. 1. 1916
                  unterbreitete Zweig\pwindex{Zweig, Stefan 28.11.1881 – 23.02.1942@\textsc{Zweig, Stefan} (28.11.1881 – 23.02.1942), \emph{Schriftsteller/Schriftstellerin}|pwk}{ }Schnitzler den Vorschlag
                  einer Eingabe beim zuständigen Minister. Also Folge der Aktivitäten Zweigs\pwindex{Zweig, Stefan 28.11.1881 – 23.02.1942@\textsc{Zweig, Stefan} (28.11.1881 – 23.02.1942), \emph{Schriftsteller/Schriftstellerin}|pwk} wurde Rilke\pwindex{Rilke, Rainer Maria 04.12.1875 – 29.12.1926@\textsc{Rilke, Rainer Maria} (04.12.1875 – 29.12.1926), \emph{Schriftsteller/Schriftstellerin}|pwk} nach der Grundausbildung zu Zweig\pwindex{Zweig, Stefan 28.11.1881 – 23.02.1942@\textsc{Zweig, Stefan} (28.11.1881 – 23.02.1942), \emph{Schriftsteller/Schriftstellerin}|pwk} ins \emph{Kriegsarchiv}\orgindex{Kriegsarchiv@Kriegsarchiv|pwk} versetzt. }}}\label{K_L03657-2} erbitten, der eingerückt ist und der (aus
               vielen Gründen) sehr leidet. Vielleicht könnten Wir durch eine gemeinsame Initiative
               ihm helfen. Und {\pb}wer verdient es, wenn
               nicht er?\pend
           
\pstart
           Getreulichst (mit vielen Grüssen an Ihre liebe Frau\pwindex{Schnitzler, Olga 17.01.1882 – 13.01.1970@\textsc{Schnitzler, Olga} (17.01.1882 – 13.01.1970), \emph{Schauspieler/Schauspielerin, Sänger/Sängerin}|pwv} und Sie){\\[\baselineskip]} Ihr \spacefill\mbox{Stefan Zweig}\pend
           \leftskip=0em{}
\pstart
           \noindent{}\uline{P.S.} Ich bin (ausser Mittwoch) immer
                  frei.{\\}nachmittags oder abends.\pend
           \selectlanguage{ngerman}\endnumbering\briefempfaengerindex{Schnitzler, Arthur@\textsc{Schnitzler, Arthur}!zzzZweig, Stefan@\emph{von Stefan Zweig}!1916-01-181@{18. 1. 1916}|)be}\mylabel{L03657h}
\begin{anhang}
\end{anhang}\normalsize

\doendnotes{C}
\bigskip
\vfill

\clearpage

\footnotesize

\lohead{\textsc{register}}

% Definiere theindex-Environment komplett neu ohne reledmac
\makeatletter
\renewenvironment{theindex}{%
  \section*{\indexname}%
  \setlength{\parindent}{0pt}%
  \setlength{\parskip}{0pt plus 0.3pt}%
  \let\item\@idxitem
}{%
  \clearpage
}
\makeatother

\IfFileExists{\jobname-pw.ind}{\input{\jobname-pw.ind}}{}

\end{document}

      