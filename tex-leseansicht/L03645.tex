%% latex-leseansicht-vorspann.tex
%% Vorspann für die Leseansicht.
%% Lädt die gemeinsame Datei latex-vorspann.tex mit nicht gesetztem Schalter.

\newif\ifkorrekturansicht
\korrekturansichtfalse

\input{../tex-inputs/latex-vorspann}


\section[Stefan Zweig an Arthur Schnitzler, {[}7.?{]} 12. 1914]{L03645 Stefan Zweig an Arthur Schnitzler, [7.?] 12. 1914}
\nopagebreak\mylabel{L03645v}
\rehead{ }\normalsize\beginnumbering\briefempfaengerindex{Schnitzler, Arthur@\textsc{Schnitzler, Arthur}!zzzZweig, Stefan@\emph{von Stefan Zweig}!1914-12-071@{[7.?] 12. 1914}|(be}
\toendnotes[C]{\smallbreak\pagebreak[2]}
\correspDesc{Versand  durch Stefan Zweig am [7.?] 12. 1914 in Wien
\newline{}Erhalt  durch Arthur Schnitzler im Zeitraum [7. 12. 1914 – 10. 12. 1914?] in Wien}\toendnotes[C]{\smallbreak}
\Standort{CUL, Schnitzler, B 118.}
\physDesc{Bildpostkarte, 438 Zeichen
\newline{}Handschrift: blaue Tinte, lateinische Kurrent
\newline{}Versand: Stempel: »\nobreak{}\oindex{Wien@\textbf{Wien}, \emph{Verwaltungsgebiet}|pwk}Wien, \textcolor{gray}{7.} XII. \textcolor{gray}{1}4, 2\nobreak{}«.  }
\buchAbdrucke{\weitereDrucke{Stefan Zweig: \emph{Briefwechsel mit Hermann Bahr, Sigmund Freud, Rainer Maria
                        Rilke und Arthur Schnitzler}. Herausgegeben von Jeffrey B. Berlin, Hans-Ulrich Lindken und Donald A. Prater. Frankfurt am Main: \emph{S. Fischer} 1987, S. 388.} }\toendnotes[C]{\smallbreak}\pstart{}{\pb}D\textsuperscript{r} Arthur
                  Schnitzler\pend{}\pstart{}Wien – Cottage\oindex{Wien@\textbf{Wien}!XVIII., Währing@\textbf{XVIII., Währing}!Währinger Cottage@\textbf{Währinger Cottage}, \emph{Teil eines besiedelten Ortes}|pw}\pend{}\pstart{}Sternwartestrasse 71\oindex{Wien@\textbf{Wien}!XVIII., Währing@\textbf{XVIII., Währing}!Sternwartestraße 71@\textbf{Sternwartestraße 71}, \emph{Wohngebäude}|pw}\pend{}{\bigskip}
\pstart
           \noindent{}\textcolor{gray}{\textbf{{\pb}GUSTINUS AMBROSI\pwindex{Ambrosi, Gustinus 24.\,2.\,1893 Eisenstadt – 30.\,6.\,1975 Wien@\textsc{Ambrosi, Gustinus} (24.\,2.\,1893 Eisenstadt – 30.\,6.\,1975 Wien), \emph{Schriftsteller, Bildhauer, Philosoph}|pw}}}\pend
           
\pstart
           \textcolor{gray}{\textbf{BÜSTE STEFAN ZWEIG\pwindex{Ambrosi, Gustinus 24.\,2.\,1893 Eisenstadt – 30.\,6.\,1975 Wien@\textsc{Ambrosi, Gustinus} (24.\,2.\,1893 Eisenstadt – 30.\,6.\,1975 Wien), \emph{Schriftsteller, Bildhauer, Philosoph}!Stefan Zweig@\strich\emph{Stefan Zweig}|pw}}}\pend
           \vspace{1em}
\pstart{}{\pb}Verehrter Herr Doktor,\pend\vspace{0.5em}
\pstart
           ich komme \label{K_L03645-1v}\edtext{Donnerstag}{\lemma{\textnormal{\emph{Donnerstag}}}\Cendnote{\textnormal{Vgl. A. S.: \emph{Tagebuch}, 10. 12. 1914.}}}\label{K_L03645-1} freudigst und pünktlichst. Diese
               Karte stellt ein Werk\pwindex{Ambrosi, Gustinus 24.\,2.\,1893 Eisenstadt – 30.\,6.\,1975 Wien@\textsc{Ambrosi, Gustinus} (24.\,2.\,1893 Eisenstadt – 30.\,6.\,1975 Wien), \emph{Schriftsteller, Bildhauer, Philosoph}!Stefan Zweig@\strich\emph{Stefan Zweig}|pwv} des
               wirklich genialen taubstummen Bildhauers Ambrosi\pwindex{Ambrosi, Gustinus 24.\,2.\,1893 Eisenstadt – 30.\,6.\,1975 Wien@\textsc{Ambrosi, Gustinus} (24.\,2.\,1893 Eisenstadt – 30.\,6.\,1975 Wien), \emph{Schriftsteller, Bildhauer, Philosoph}|pw} dar, der in diesem Jahre bei Gerhardt Hauptmann\pwindex{Hauptmann, Gerhart 15.\,11.\,1862 Szczawno-Zdrój – 6.\,6.\,1946 Jagniątków@\textsc{Hauptmann, Gerhart} (15.\,11.\,1862 Szczawno-Zdrój – 6.\,6.\,1946 Jagniątków), \emph{Schriftsteller}|pw} ein wundervolles Portrait\pwindex{Ambrosi, Gustinus 24.\,2.\,1893 Eisenstadt – 30.\,6.\,1975 Wien@\textsc{Ambrosi, Gustinus} (24.\,2.\,1893 Eisenstadt – 30.\,6.\,1975 Wien), \emph{Schriftsteller, Bildhauer, Philosoph}!Gerhart Hauptmann@\strich\emph{Gerhart Hauptmann}|pwv} machte und keinen sehnlichern \label{K_L03645-2v}\edtext{Wunsch}{\lemma{\textnormal{\emph{Wunsch}}}\Cendnote{\textnormal{Schnitzler
               dürfte auf den Wunsch nicht reagiert haben. In seinem Nachlass ist nur der Durchschlag eines 
               Schreibens an Ambrosi\pwindex{Ambrosi, Gustinus 24.\,2.\,1893 Eisenstadt – 30.\,6.\,1975 Wien@\textsc{Ambrosi, Gustinus} (24.\,2.\,1893 Eisenstadt – 30.\,6.\,1975 Wien), \emph{Schriftsteller, Bildhauer, Philosoph}|pwk} ein Jahrzehnt später überliefert (20. 11. 1924, \emph{Deutsches Literaturarchiv Marbach}, HS.1985.1.242).}}}\label{K_L03645-2} als den: Sie
               möchten ihm einmal 2 x 2 Stunden widmen, dass er auch die Ihre schaffen könnte.\pend
           \pstart Mit vielen Grüssen Ihr getreuer \spacefill\mbox{Stefan Zweig}\pend{}\selectlanguage{ngerman}\endnumbering\briefempfaengerindex{Schnitzler, Arthur@\textsc{Schnitzler, Arthur}!zzzZweig, Stefan@\emph{von Stefan Zweig}!1914-12-071@{[7.?] 12. 1914}|)be}\mylabel{L03645h}  \newcommand{\dateiname}{L03645}\newcommand{\titel}{Stefan Zweig an Arthur Schnitzler, [7.?] 12. 1914}\newcommand{\editorInnen}{Selma Jahnke und Martin Anton Müller}%% latex-leseansicht-abspann.tex
%% Abspann für die Leseansicht.
%% Der Schalter \ifkorrekturansicht ist bereits durch den Vorspann gesetzt.

%% latex-abspann.tex
%% Gemeinsamer Abspann für Korrekturansicht und Leseansicht.
%% Setzt den Schalter \ifkorrekturansicht voraus (gesetzt in den
%% einbindenden Dateien latex-korrekturansicht-abspann.tex bzw.
%% latex-leseansicht-abspann.tex).
%% ---------------------------------------------------------------

\normalsize

% Das esempio-Environment wird nur in der Leseansicht benötigt
\ifkorrekturansicht\else
\newenvironment{esempio}[3]%
{
    \vspace{1.5ex}
    \rlap{\underline{#1}}
    \par
    \setlength{\parindent}{0cm}
    \nopagebreak
    \leftskip=#2cm
    \rightskip=#3cm
}
{
    \par
}
\fi

\doendnotes{C}
\bigskip
\vfill

\clearpage

\footnotesize

\ifkorrekturansicht
  \lohead{\textsc{register}}
\fi

% theindex-Environment neu definieren ohne reledmac
\makeatletter
\renewenvironment{theindex}{%
  \ifkorrekturansicht
    \section*{\indexname}%
  \else
    \subsubsection*{Index der erwähnten Entitäten}%
  \fi
  \setlength{\parindent}{0pt}%
  \setlength{\parskip}{0pt plus 0.3pt}%
  \let\item\@idxitem
}{%
  \ifkorrekturansicht\clearpage\fi
}
\makeatother

\IfFileExists{\jobname-pw.ind}{\input{\jobname-pw.ind}}{}

% Quellenangabe nur in der Leseansicht
\ifkorrekturansicht\else
% Fallback-Definitionen, falls die .tex-Datei \titel etc. nicht gesetzt hat
\providecommand{\titel}{}
\providecommand{\editorInnen}{}
\providecommand{\dateiname}{\jobname}

\vspace{3cm}

\vfill

\footnotesize
\textsc{Quelle}: \titel. Herausgegeben von {\editorInnen}. In: \emph{Arthur Schnitzler: Briefwechsel mit Autorinnen und Autoren}.
 Digitale Edition, https://schnitzler-briefe.acdh.oeaw.ac.at/{\dateiname}.html (Stand \today)
\fi

\end{document}


