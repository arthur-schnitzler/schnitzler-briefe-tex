%% latex-korrekturansicht-vorspann.tex
%% Vorspann für die Korrekturansicht.
%% Lädt die gemeinsame Datei latex-vorspann.tex mit gesetztem Schalter.

\newif\ifkorrekturansicht
\korrekturansichttrue

\input{../tex-inputs/latex-vorspann}


\section[Stefan Zweig an Arthur Schnitzler, {[}7.?{]} 12. 1914]{L03645 Stefan Zweig an Arthur Schnitzler, {[}7.?{]} 12. 1914}
\nopagebreak\mylabel{L03645v}
\rehead{ }\normalsize\beginnumbering\briefempfaengerindex{Schnitzler, Arthur@\textsc{Schnitzler, Arthur}!zzzZweig, Stefan@\emph{von Stefan Zweig}!1914-12-071@{{[}7.?{]} 12. 1914}|(be}
\toendnotes[C]{\smallbreak\pagebreak[2]}\Standort{CUL, Schnitzler, B 118.}
\physDesc{Bildpostkarte, 439 Zeichen
\newline{}Handschrift: blaue Tinte, lateinische Kurrent
\newline{}Versand: Stempel: »\nobreak{}Wien, \textcolor{gray}{7.} XII. \textcolor{gray}{1}4, 2\nobreak{}«.  }
\buchAbdrucke{\weitereDrucke{Stefan Zweig: \emph{Briefwechsel mit Hermann Bahr, Sigmund Freud, Rainer Maria
                        Rilke und Arthur Schnitzler}. Frankfurt am Main: \emph{S. Fischer} 1987, S. 388.} }\toendnotes[C]{\smallbreak}\pstart{}{\pb}D\textsuperscript{r} Artur
                  Schnitzler\pend{}\pstart{}Wien – Cottage\oindex{Waehringer Cottage@\textbf{Währinger Cottage}, \emph{Teil eines besiedelten Ortes (A.BSOX)}|pw}\pend{}\pstart{}Sternwartestrasse 71\oindex{Sternwartestrasse 71@\textbf{Sternwartestraße 71}, \emph{Wohngebäude (K.WHS)}|pw}\pend{}{\bigskip}
\pstart
           \noindent{}\textcolor{gray}{\textbf{{\pb}GUSTINUS AMBROSI\pwindex{Ambrosi, Gustinus 24.02.1893 – 30.06.1975@\textsc{Ambrosi, Gustinus} (24.02.1893 – 30.06.1975), \emph{Schriftsteller/Schriftstellerin, Bildhauer/Bildhauerin, Philosoph/Philosophin}|pw}}}\pend
           
\pstart
           \textcolor{gray}{\textbf{BÜSTE STEFAN ZWEIG\pwindex{Stefan Zweig@\emph{Stefan Zweig}|pw}}}\pend
           \vspace{1em}
\pstart{}{\pb}Verehrter Herr Doktor,\pend\vspace{0.5em}
\pstart
           ich komme \label{K_L03645-1v}\edtext{Donnerstag}{\lemma{\textnormal{\emph{Donnerstag}}}\Cendnote{\textnormal{Vgl. A. S.: \emph{Tagebuch}, 10. 12. 1914.}}}\label{K_L03645-1} freudigst und pünktlichst. Diese
               Karte stellt ein Werk\pwindex{Stefan Zweig@\emph{Stefan Zweig}|pwv} des
               wirklich genialen taubstummen Bildhauers Ambrosi\pwindex{Ambrosi, Gustinus 24.02.1893 – 30.06.1975@\textsc{Ambrosi, Gustinus} (24.02.1893 – 30.06.1975), \emph{Schriftsteller/Schriftstellerin, Bildhauer/Bildhauerin, Philosoph/Philosophin}|pw} dar, der in diesem Jahre bei Gerhardt Hauptmann\pwindex{Hauptmann, Gerhart 15.11.1862 – 06.06.1946@\textsc{Hauptmann, Gerhart} (15.11.1862 – 06.06.1946), \emph{Schriftsteller/Schriftstellerin}|pw} ein wundervolles Portrait\pwindex{Gerhart Hauptmann@\emph{Gerhart Hauptmann}|pwv} machte und keinen sehnlichern Wunsch als den: Sie
               möchten ihm einmal 2 x 2 Stunden widmen, dass er auch die Ihre schaffen könnte.\pend
           \pstart Mit vielen Grüssen Ihr getreuer \spacefill\mbox{Stefan Zweig}\pend{}\selectlanguage{ngerman}\endnumbering\briefempfaengerindex{Schnitzler, Arthur@\textsc{Schnitzler, Arthur}!zzzZweig, Stefan@\emph{von Stefan Zweig}!1914-12-071@{{[}7.?{]} 12. 1914}|)be}\mylabel{L03645h}
\begin{anhang}
\end{anhang}\normalsize

\doendnotes{C}
\bigskip
\vfill

\clearpage

\footnotesize

\lohead{\textsc{register}}

% Definiere theindex-Environment komplett neu ohne reledmac
\makeatletter
\renewenvironment{theindex}{%
  \section*{\indexname}%
  \setlength{\parindent}{0pt}%
  \setlength{\parskip}{0pt plus 0.3pt}%
  \let\item\@idxitem
}{%
  \clearpage
}
\makeatother

\IfFileExists{\jobname-pw.ind}{\input{\jobname-pw.ind}}{}

\end{document}

      