%% latex-korrekturansicht-vorspann.tex
%% Vorspann für die Korrekturansicht.
%% Lädt die gemeinsame Datei latex-vorspann.tex mit gesetztem Schalter.

\newif\ifkorrekturansicht
\korrekturansichttrue

\input{../tex-inputs/latex-vorspann}


\section[Thomas Mann an Arthur Schnitzler, 16. 10. 1911]{L02039 Thomas Mann an Arthur Schnitzler, 16. 10. 1911}
\nopagebreak\mylabel{L02039v}
\rehead{ }\normalsize\beginnumbering\briefempfaengerindex{Schnitzler, Arthur@\textsc{Schnitzler, Arthur}!zzzMann, Thomas@\emph{von Thomas Mann}!1911-10-163@{16. 10. 1911}|(be}
\toendnotes[C]{\smallbreak\pagebreak[2]}\Standort{CUL, Schnitzler, B 67.}
\physDesc{Brief, 1 Blatt, 2 Seiten, 827 Zeichen
\newline{}Handschrift: schwarze Tinte, deutsche Kurrent
\newline{}Schnitzler: 1) mit Bleistift beschriftet: »\textsc{Mann}«  2) mit rotem Buntstift eine Unterstreichung}
\buchAbdrucke{\weitereDrucke{\emph{Modern Austrian Literature}, Jg. 7 (1974) Nr. 1/2, S. 15–16.} }\toendnotes[C]{\smallbreak}
\pstart
           \raggedleft{}{\pb}München\oindex{Muenchen@\textbf{München}, \emph{P.PPLA}|pw} den
                  16. X. 1911.\pend
           
\pstart{}Sehr verehrter Herr:\pend\vspace{0.5em}
\pstart
           Es war mir eine beſondere Freude, am Morgen nach der \label{K_L02039-1v}\edtext{Première\eventindex{Residenztheater Muenchen@\textbf{Residenztheater München}!Premiere von Das weite Land, 14.10.1911 [IV.]@Premiere von Das weite Land, 14.10.1911 [IV.]|pwv}}{\lemma{\textnormal{\emph{Première}}}\Cendnote{\textnormal{Eine der sieben gleichzeitigen \emph{Theaterpremieren von \emph{Das weite Land}\pwindex{weite Land. Tragikomoedie in fuenf Akten@\emph{Das weite Land. Tragikomödie in fünf Akten}|pwk}}\eventindex{Residenztheater Muenchen@\textbf{Residenztheater München}!Premiere von Das weite Land, 14.10.1911 [IV.]@Premiere von Das weite Land, 14.10.1911 [IV.]|pwk} am 14. 10. 1911 fand am \emph{Residenztheater
                     München}\orgindex{Residenztheater Muenchen@Residenztheater München|pwk} in München\oindex{Muenchen@\textbf{München}, \emph{P.PPLA}|pwk} statt. }}}\label{K_L02039-1},
               noch ganz erfüllt von Ihrer Kunſt, das Buch des »Weiten Landes\pwindex{weite Land. Tragikomoedie in fuenf Akten@\emph{Das weite Land. Tragikomödie in fünf Akten}|pw}« von Ihrer eigenen Hand zu empfangen. Ich danke Ihnen von
               Herzen. Ihr Stück hat hier tiefen Eindruck gemacht, der Beifall am Schluſſe ruhte
               nicht, bis der Regiſſeur\pwindex{Basil, Friedrich 16.05.1862 – 31.03.1938@\textsc{Basil, Friedrich} (16.05.1862 – 31.03.1938), \emph{Regisseur/Regisseurin, Schauspieler/Schauspielerin}|pwv} in
               Ihrem Namen gedankt hatte. Die Aufführung war recht leidlich, Steinrück\pwindex{Steinrueck, Albert 20.05.1872 – 11.02.1929@\textsc{Steinrück, Albert} (20.05.1872 – 11.02.1929), \emph{Schauspieler/Schauspielerin}|pw} in ſeiner Art meiſter{\pb}haft, wenn auch wohl nicht der Menſch, den
               Sie geſehen haben. Es fehlte die aeußere Weichheit, die zu der gefährlichen Energie
               des Mannes ſo lebensvoll kontraſtieren müßte. Dieſer letztere, der erotiſche Ernſt,
               war deſto eindrucksvoller betont. Mein Bruder\pwindex{Mann, Heinrich 27.03.1871 – 11.03.1950@\textsc{Mann, Heinrich} (27.03.1871 – 11.03.1950), \emph{Schriftsteller/Schriftstellerin}|pwv} und ich verbrachten den Reſt des Abends \introOben{}mit\introOben{} den Hauptdarſtellern. Das Telegramm »an Arthur« war
               allgemeines Herzensbedürfnis.\pend
           
\pstart
           Ihr ergebener{\\[\baselineskip]}\spacefill\mbox{Thomas Mann.}\pend
           \leftskip=0em{}\selectlanguage{ngerman}\endnumbering\briefempfaengerindex{Schnitzler, Arthur@\textsc{Schnitzler, Arthur}!zzzMann, Thomas@\emph{von Thomas Mann}!1911-10-163@{16. 10. 1911}|)be}\mylabel{L02039h}  \normalsize

\doendnotes{C}
\bigskip
\vfill

\clearpage

\footnotesize

\lohead{\textsc{register}}

% Definiere theindex-Environment komplett neu ohne reledmac
\makeatletter
\renewenvironment{theindex}{%
  \section*{\indexname}%
  \setlength{\parindent}{0pt}%
  \setlength{\parskip}{0pt plus 0.3pt}%
  \let\item\@idxitem
}{%
  \clearpage
}
\makeatother

\IfFileExists{\jobname-pw.ind}{\input{\jobname-pw.ind}}{}

\end{document}

      