%% latex-leseansicht-vorspann.tex
%% Vorspann für die Leseansicht.
%% Lädt die gemeinsame Datei latex-vorspann.tex mit nicht gesetztem Schalter.

\newif\ifkorrekturansicht
\korrekturansichtfalse

\input{../tex-inputs/latex-vorspann}


\section[Thomas Mann an Arthur Schnitzler, 16. 10. 1911]{L02039 Thomas Mann an Arthur Schnitzler, 16. 10. 1911}
\nopagebreak\mylabel{L02039v}
\rehead{ }\normalsize\beginnumbering\briefempfaengerindex{Schnitzler, Arthur@\textsc{Schnitzler, Arthur}!zzzMann, Thomas@\emph{von Thomas Mann}!1911-10-163@{16. 10. 1911}|(be}
\toendnotes[C]{\smallbreak\pagebreak[2]}
\correspDesc{Versand  durch Thomas Mann am 16. 10. 1911 in München
\newline{}Erhalt  durch Arthur Schnitzler im Zeitraum [17. 10. 1911 – 21. 10. 1911?] in Wien}\toendnotes[C]{\smallbreak}
\Standort{CUL, Schnitzler, B 67.}
\physDesc{Brief, 1 Blatt, 2 Seiten, 827 Zeichen
\newline{}Handschrift: schwarze Tinte, deutsche Kurrent
\newline{}Schnitzler: 1) mit Bleistift beschriftet: »\textsc{Mann}«  2) mit rotem Buntstift eine Unterstreichung}
\buchAbdrucke{\weitereDrucke{Hertha Krotkoff: \emph{Arthur Schnitzler – Thomas Mann: Briefe.} In: \emph{Modern Austrian Literature}, Jg. 7 (1974) Nr. 1/2, S. 15–16.} }\toendnotes[C]{\smallbreak}
\pstart
           \raggedleft{}{\pb}München\oindex{München@\textbf{München}|pw} den
                  16. X. 1911.\pend
           
\pstart{}Sehr verehrter Herr:\pend\vspace{0.5em}
\pstart
           Es war mir eine beſondere Freude, am Morgen nach der \label{K_L02039-1v}\edtext{Première\eventindex{Residenztheater München@\textbf{Residenztheater München}!Premiere von Das weite Land, 14.10.1911 [IV.]@Premiere von Das weite Land, 14.10.1911 [IV.]|pwv}}{\lemma{\textnormal{\emph{Première}}}\Cendnote{\textnormal{Eine der sieben gleichzeitigen Theaterpremieren von \emph{Das weite Land}\pwindex{Schnitzler, Arthur 15.\,5.\,1862 Wien – 21.\,10.\,1931 ebd.@\textsc{Schnitzler, Arthur} (15.\,5.\,1862 Wien – 21.\,10.\,1931 ebd.), \emph{Schriftsteller, Mediziner}!weite Land. Tragikomödie in fünf Akten@\strich\emph{Das weite Land. Tragikomödie in fünf Akten}|pwk}\eventindex{Residenztheater München@\textbf{Residenztheater München}!Premiere von Das weite Land, 14.10.1911 [IV.]@Premiere von Das weite Land, 14.10.1911 [IV.]|pwk} am 14. 10. 1911 fand am \emph{Residenztheater
                     München}\orgindex{Residenztheater München@Residenztheater München|pwk} in München\oindex{München@\textbf{München}|pwk} statt. }}}\label{K_L02039-1},
               noch ganz erfüllt von Ihrer Kunſt, das Buch des »Weiten Landes\pwindex{Schnitzler, Arthur 15.\,5.\,1862 Wien – 21.\,10.\,1931 ebd.@\textsc{Schnitzler, Arthur} (15.\,5.\,1862 Wien – 21.\,10.\,1931 ebd.), \emph{Schriftsteller, Mediziner}!weite Land. Tragikomödie in fünf Akten@\strich\emph{Das weite Land. Tragikomödie in fünf Akten}|pw}« von Ihrer eigenen Hand zu empfangen. Ich danke Ihnen von
               Herzen. Ihr Stück hat hier tiefen Eindruck gemacht, der Beifall am Schluſſe ruhte
               nicht, bis der Regiſſeur\pwindex{Basil, Friedrich 16.\,5.\,1862 Frankfurt (Oder) – 31.\,3.\,1938 München@\textsc{Basil, Friedrich} (16.\,5.\,1862 Frankfurt (Oder) – 31.\,3.\,1938 München), \emph{Regisseur, Schauspieler}|pwv} in
               Ihrem Namen gedankt hatte. Die Aufführung war recht leidlich, Steinrück\pwindex{Steinrück, Albert 20.\,5.\,1872 Wetterburg – 11.\,2.\,1929 Berlin@\textsc{Steinrück, Albert} (20.\,5.\,1872 Wetterburg – 11.\,2.\,1929 Berlin), \emph{Schauspieler}|pw} in{ }ſeiner Art meiſter{\pb}haft, wenn auch wohl nicht der Menſch, den
               Sie geſehen haben. Es fehlte die aeußere Weichheit, die zu der gefährlichen Energie
               des Mannes{ }ſo lebensvoll kontraſtieren müßte. Dieſer letztere, der erotiſche Ernſt,
               war deſto eindrucksvoller betont. Mein Bruder\pwindex{Mann, Heinrich 27.\,3.\,1871 Lübeck – 11.\,3.\,1950 Santa Monica@\textsc{Mann, Heinrich} (27.\,3.\,1871 Lübeck – 11.\,3.\,1950 Santa Monica), \emph{Schriftsteller}|pwv} und ich verbrachten den Reſt des Abends \introOben{}mit\introOben{} den Hauptdarſtellern. Das Telegramm »an Arthur« war
               allgemeines Herzensbedürfnis.\pend
           
\pstart
           Ihr ergebener{\\[\baselineskip]}\spacefill\mbox{Thomas Mann.}\pend
           \leftskip=0em{}\selectlanguage{ngerman}\endnumbering\briefempfaengerindex{Schnitzler, Arthur@\textsc{Schnitzler, Arthur}!zzzMann, Thomas@\emph{von Thomas Mann}!1911-10-163@{16. 10. 1911}|)be}\mylabel{L02039h}  \newcommand{\dateiname}{L02039}\newcommand{\titel}{Thomas Mann an Arthur Schnitzler, 16. 10. 1911}\newcommand{\editorInnen}{Martin Anton Müller und Gerd-Hermann Susen}%% latex-leseansicht-abspann.tex
%% Abspann für die Leseansicht.
%% Der Schalter \ifkorrekturansicht ist bereits durch den Vorspann gesetzt.

%% latex-abspann.tex
%% Gemeinsamer Abspann für Korrekturansicht und Leseansicht.
%% Setzt den Schalter \ifkorrekturansicht voraus (gesetzt in den
%% einbindenden Dateien latex-korrekturansicht-abspann.tex bzw.
%% latex-leseansicht-abspann.tex).
%% ---------------------------------------------------------------

\normalsize

% Das esempio-Environment wird nur in der Leseansicht benötigt
\ifkorrekturansicht\else
\newenvironment{esempio}[3]%
{
    \vspace{1.5ex}
    \rlap{\underline{#1}}
    \par
    \setlength{\parindent}{0cm}
    \nopagebreak
    \leftskip=#2cm
    \rightskip=#3cm
}
{
    \par
}
\fi

\doendnotes{C}
\bigskip
\vfill

\clearpage

\footnotesize

\ifkorrekturansicht
  \lohead{\textsc{register}}
\fi

% theindex-Environment neu definieren ohne reledmac
\makeatletter
\renewenvironment{theindex}{%
  \ifkorrekturansicht
    \section*{\indexname}%
  \else
    \subsubsection*{Index der erwähnten Entitäten}%
  \fi
  \setlength{\parindent}{0pt}%
  \setlength{\parskip}{0pt plus 0.3pt}%
  \let\item\@idxitem
}{%
  \ifkorrekturansicht\clearpage\fi
}
\makeatother

\IfFileExists{\jobname-pw.ind}{\input{\jobname-pw.ind}}{}

% Quellenangabe nur in der Leseansicht
\ifkorrekturansicht\else
% Fallback-Definitionen, falls die .tex-Datei \titel etc. nicht gesetzt hat
\providecommand{\titel}{}
\providecommand{\editorInnen}{}
\providecommand{\dateiname}{\jobname}

\vspace{3cm}

\vfill

\footnotesize
\textsc{Quelle}: \titel. Herausgegeben von {\editorInnen}. In: \emph{Arthur Schnitzler: Briefwechsel mit Autorinnen und Autoren}.
 Digitale Edition, https://schnitzler-briefe.acdh.oeaw.ac.at/{\dateiname}.html (Stand \today)
\fi

\end{document}


