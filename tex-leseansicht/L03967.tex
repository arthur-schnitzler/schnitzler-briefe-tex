%% latex-leseansicht-vorspann.tex
%% Vorspann für die Leseansicht.
%% Lädt die gemeinsame Datei latex-vorspann.tex mit nicht gesetztem Schalter.

\newif\ifkorrekturansicht
\korrekturansichtfalse

\input{../tex-inputs/latex-vorspann}


\section[Arthur Schnitzler an Berta Zuckerkandl, 25. 1. 1926]{L03967 Arthur Schnitzler an Berta Zuckerkandl, 25. 1. 1926}
\nopagebreak\mylabel{L03967v}
\rehead{ }\normalsize\beginnumbering\briefempfaengerindex{Zuckerkandl, Berta@\textsc{Zuckerkandl, Berta}!zzzSchnitzler, Arthur@\emph{von Arthur Schnitzler}!1926-01-252@{25. 1. 1926}|(be}
\toendnotes[C]{\smallbreak\pagebreak[2]}
\correspDesc{Versand  durch Arthur Schnitzler am 25. 1. 1926 in Wien
\newline{}Erhalt  durch Berta Zuckerkandl im Zeitraum [26. 1. 1926 – 30. 1. 1926?] in Paris}\toendnotes[C]{\smallbreak}
\Standort{DLA, HS.1985.1.2282.}
\physDesc{Brief, Durchschlag, 1 Blatt, 2 Seiten, 1279 Zeichen
\newline{}Schreibmaschine
\newline{}Handschrift: roter Buntstift, lateinische Kurrent (\noindent{}beschriftet: »\uline{Zuckerkandl}« und »\uline{Paris}«, fünf Unterstreichungen)}\toendnotes[C]{\smallbreak}
\pstart
           \raggedleft{}{\pb}25. 1. 1926.\pend
           
\pstart{}Verehrte Freundin.\pend\vspace{0.5em}
\pstart
           Eben kommt Ihr \label{K_L03967-1v}\edtext{Brief vom 23. Jänner}{\lemma{\textnormal{\emph{Brief vom 23. Jänner}}}\Cendnote{\textnormal{nicht überliefert}}}\label{K_L03967-1}.
               Ich habe von Frau Emma Cabire\pwindex{Cabire, Emma @\textsc{Cabire, Emma}, \emph{Übersetzerin, Redakteurin, Literaturagentin}|pw} keine andere
               Adresse als Paris 27, Rue Lemercier\oindex{27, Rue Lemercier@\textbf{27, Rue Lemercier}, \emph{Wohngebäude}|pw}. Weiss Gemier\pwindex{Gémier, Firmin 21.\,2.\,1865 Aubervilliers – 26.\,11.\,1933 Paris@\textsc{Gémier, Firmin} (21.\,2.\,1865 Aubervilliers – 26.\,11.\,1933 Paris), \emph{Theaterleiter, Schauspieler, Drehbuchautor}|pw} nicht wo sie wohnt? Oder Lenormand\pwindex{Lenormand, Henri-René 3.\,5.\,1882 Paris – 16.\,2.\,1951 ebd.@\textsc{Lenormand, Henri-René} (3.\,5.\,1882 Paris – 16.\,2.\,1951 ebd.), \emph{Schriftsteller}|pw}? Das
               Wichtigste ist ja nur, dass wir die Uebersetzung\pwindex{Schnitzler, Arthur 15. 5. 1862 Wien – 21. 10. 1931 ebd.@\textsc{Schnitzler, Arthur} (15. 5. 1862 Wien – 21. 10. 1931 ebd.), \emph{Schriftsteller, Mediziner}!Le Pays de l’âme. Drame en 5 actes@\strich\emph{Le Pays de l’âme. Drame en 5 actes}|pwv} in der Hand haben. Ist Mme. Cabire\pwindex{Cabire, Emma @\textsc{Cabire, Emma}, \emph{Übersetzerin, Redakteurin, Literaturagentin}|pw} vorläufig unauffindbar, so ist das ja nicht unsere
      Schuld und auf die perzentuelle \label{T_L03967-1v}\edtext{Verteilung}{\lemma{\textnormal{\emph{Verteilung}}}\Cendnote{\textnormal{In der Vorlage steht: »Verteilug«.}}}\label{T_L03967-1} der
      Tantièmen wird sie ja in jedem Falle eingehen.\pend
           
\pstart
           Es ist ja nun wirklich schwer
               sich in der Frage des „Reigen\pwindex{Schnitzler, Arthur 15. 5. 1862 Wien – 21. 10. 1931 ebd.@\textsc{Schnitzler, Arthur} (15. 5. 1862 Wien – 21. 10. 1931 ebd.), \emph{Schriftsteller, Mediziner}!Reigen. Zehn Dialoge@\strich\emph{Reigen. Zehn Dialoge}|pw}« zu entscheiden. \label{K_L03967-2v}\edtext{Wie ich Ihnen eben schrieb}{\lemma{\textnormal{\emph{Wie … schrieb}}}\Cendnote{\textnormal{Siehe XXXX Auszeichnungsfehler: Dokument L03965 nicht gefunden. Ob die beiden Briefe vom selben Tag mit einer Sendung abgingen oder der vorliegende, der auf einen in der Zwischenzeit eingegangenen, nicht überlieferten Brief Zuckerkandls\pwindex{Zuckerkandl, Berta 13.\,4.\,1864 Wien – 16.\,10.\,1945 Paris@\textsc{Zuckerkandl, Berta} (13.\,4.\,1864 Wien – 16.\,10.\,1945 Paris), \emph{Schriftstellerin, Journalistin, Übersetzerin}|pwk} antwortet, geschrieben wurde, als der erste schon abgeschickt war, ist nicht ersichtlich.}}}\label{K_L03967-2} die bei Stock\orgindex{Éditions Stock@Éditions Stock|pw}
               erschienene Uebersetzung\pwindex{Schnitzler, Arthur 15. 5. 1862 Wien – 21. 10. 1931 ebd.@\textsc{Schnitzler, Arthur} (15. 5. 1862 Wien – 21. 10. 1931 ebd.), \emph{Schriftsteller, Mediziner}!ronde. Dix scènes dialoguées@\strich\emph{La ronde. Dix scènes dialoguées}|pwv} ist nicht gut und
      es scheint mir in jedem Fall inopportun mit
      dem »Reigen\pwindex{Schnitzler, Arthur 15. 5. 1862 Wien – 21. 10. 1931 ebd.@\textsc{Schnitzler, Arthur} (15. 5. 1862 Wien – 21. 10. 1931 ebd.), \emph{Schriftsteller, Mediziner}!Reigen. Zehn Dialoge@\strich\emph{Reigen. Zehn Dialoge}|pw}« vor einem andern wirklichen Theatererfolg in Paris\oindex{Paris@\textbf{Paris}, \emph{Hauptstadt}|pw} herauszukommen. Die Vorteile lägen also tatsächlich auf der finanziellen Seite, die ja bei alledem unsicher bleiben. Und soll der »Reigen\pwindex{Schnitzler, Arthur 15. 5. 1862 Wien – 21. 10. 1931 ebd.@\textsc{Schnitzler, Arthur} (15. 5. 1862 Wien – 21. 10. 1931 ebd.), \emph{Schriftsteller, Mediziner}!Reigen. Zehn Dialoge@\strich\emph{Reigen. Zehn Dialoge}|pw}« überhaupt aufgeführt werden, so wird er ja bis zum Herbst
      nicht schlechter geworden sein. Somit also vorläufig nein!\pend
           
\pstart
           Nochmals herzlichsten Dank und viele Grüsse.{\\[\baselineskip]} Ihr\pend
           \leftskip=0em{}{\vspace{1\baselineskip}}
\pstart
           \noindent{}Frau Hofrätin Bertha Zuckerkandl,{\\}Paris\oindex{Paris@\textbf{Paris}, \emph{Hauptstadt}|pw}.\pend
           
\pstart
           {\pb}P. S. Da Sie ja nun so bald wieder in Wien\oindex{Wien@\textbf{Wien}, \emph{Verwaltungsgebiet}|pw} sind,
                  werden Sie wohl auch kaum Zeit haben die Sache mit Delamain\pwindex{Delamain, Maurice 28.\,4.\,1883 Jarnac – 2.\,5.\,1974 Paris@\textsc{Delamain, Maurice} (28.\,4.\,1883 Jarnac – 2.\,5.\,1974 Paris), \emph{Kritiker, Rechtsanwalt, Verleger}|pw} (Fräulein Else\pwindex{Schnitzler, Arthur 15. 5. 1862 Wien – 21. 10. 1931 ebd.@\textsc{Schnitzler, Arthur} (15. 5. 1862 Wien – 21. 10. 1931 ebd.), \emph{Schriftsteller, Mediziner}!Fräulein Else@\strich\emph{Fräulein Else}|pw}\pwindex{Schnitzler, Arthur 15. 5. 1862 Wien – 21. 10. 1931 ebd.@\textsc{Schnitzler, Arthur} (15. 5. 1862 Wien – 21. 10. 1931 ebd.), \emph{Schriftsteller, Mediziner}!Madmoiselle Else@\strich\emph{Madmoiselle Else}|pw}) in Ordnung zu
               bringen. Ich bitte Sie recht sehr, liebe und
               verehrte Freundin, hetzen Sie sich ja nicht um
               meinetwillen ab; mir eilt gar nichts.\pend
           \selectlanguage{ngerman}\endnumbering\briefempfaengerindex{Zuckerkandl, Berta@\textsc{Zuckerkandl, Berta}!zzzSchnitzler, Arthur@\emph{von Arthur Schnitzler}!1926-01-252@{25. 1. 1926}|)be}\mylabel{L03967h}
\begin{anhang}
\end{anhang}\newcommand{\dateiname}{L03967}\newcommand{\titel}{Arthur Schnitzler an Berta Zuckerkandl, 25. 1. 1926}\newcommand{\editorInnen}{Herausgegeben von Jahnke, SelmaMüller, Martin Anton}%% latex-leseansicht-abspann.tex
%% Abspann für die Leseansicht.
%% Der Schalter \ifkorrekturansicht ist bereits durch den Vorspann gesetzt.

%% latex-abspann.tex
%% Gemeinsamer Abspann für Korrekturansicht und Leseansicht.
%% Setzt den Schalter \ifkorrekturansicht voraus (gesetzt in den
%% einbindenden Dateien latex-korrekturansicht-abspann.tex bzw.
%% latex-leseansicht-abspann.tex).
%% ---------------------------------------------------------------

\normalsize

% Das esempio-Environment wird nur in der Leseansicht benötigt
\ifkorrekturansicht\else
\newenvironment{esempio}[3]%
{
    \vspace{1.5ex}
    \rlap{\underline{#1}}
    \par
    \setlength{\parindent}{0cm}
    \nopagebreak
    \leftskip=#2cm
    \rightskip=#3cm
}
{
    \par
}
\fi

\doendnotes{C}
\bigskip
\vfill

\clearpage

\footnotesize

\ifkorrekturansicht
  \lohead{\textsc{register}}
\fi

% theindex-Environment neu definieren ohne reledmac
\makeatletter
\renewenvironment{theindex}{%
  \ifkorrekturansicht
    \section*{\indexname}%
  \else
    \subsubsection*{Index der erwähnten Entitäten}%
  \fi
  \setlength{\parindent}{0pt}%
  \setlength{\parskip}{0pt plus 0.3pt}%
  \let\item\@idxitem
}{%
  \ifkorrekturansicht\clearpage\fi
}
\makeatother

\IfFileExists{\jobname-pw.ind}{\input{\jobname-pw.ind}}{}

% Quellenangabe nur in der Leseansicht
\ifkorrekturansicht\else
% Fallback-Definitionen, falls die .tex-Datei \titel etc. nicht gesetzt hat
\providecommand{\titel}{}
\providecommand{\editorInnen}{}
\providecommand{\dateiname}{\jobname}

\vspace{3cm}

\vfill

\footnotesize
\textsc{Quelle}: \titel. Herausgegeben von {\editorInnen}. In: \emph{Arthur Schnitzler: Briefwechsel mit Autorinnen und Autoren}.
 Digitale Edition, https://schnitzler-briefe.acdh.oeaw.ac.at/{\dateiname}.html (Stand \today)
\fi

\end{document}


