%% latex-korrekturansicht-vorspann.tex
%% Vorspann für die Korrekturansicht.
%% Lädt die gemeinsame Datei latex-vorspann.tex mit gesetztem Schalter.

\newif\ifkorrekturansicht
\korrekturansichttrue

\input{../tex-inputs/latex-vorspann}


\section[Hugo von Hofmannsthal und Hermine Benedict an Arthur Schnitzler, 21. {[}8. 1896{]}]{L00580 Hugo von Hofmannsthal und Hermine Benedict an Arthur Schnitzler,
               21. {[}8. 1896{]}}
\nopagebreak\mylabel{L00580v}
\rehead{ }\normalsize\beginnumbering\briefempfaengerindex{Schnitzler, Arthur@\textsc{Schnitzler, Arthur}!zzzSchaffgotsch, Hermine von@\emph{von Hermine von Schaffgotsch}!1896-08-211@{21. {[}8. 1896{]}}|(be}\briefempfaengerindex{Schnitzler, Arthur@\textsc{Schnitzler, Arthur}!zzzHofmannsthal, Hugo von@\emph{von Hugo von Hofmannsthal}!1896-08-211@{21. {[}8. 1896{]}}|(be}
\toendnotes[C]{\smallbreak\pagebreak[2]}\Standort{CUL, Schnitzler, B 43.}
\physDesc{Brief, 1 Blatt, 4 Seiten, 3330 Zeichen
\newline{}Handschrift Hugo von Hofmannsthal: schwarze Tinte, deutsche Kurrent
\newline{}Handschrift Hermine von Schaffgotsch: schwarze Tinte, deutsche Kurrent
\newline{}Schnitzler: mit Bleistift Monat und Jahr ergänzt: »Aug. 96« 
\newline{}Ordnung: mit Bleistift von unbekannter Hand nummeriert:
                                    »79« }
\buchAbdrucke{\weitereDrucke{Hugo von Hofmannsthal, Arthur Schnitzler: \emph{Briefwechsel}. Frankfurt am Main: \emph{S. Fischer} 1964, S. 72–74.} }\toendnotes[C]{\smallbreak}
\pstart
           \raggedleft{}{\pb}Alt.auſſee\oindex{Altaussee@\textbf{Altaussee}, \emph{A.ADM3}|pw}{ }21\textsuperscript{ten}\pend
           
\pstart{}lieber Arthur!\pend\vspace{0.5em}
\pstart
           {[}hs. :{]} Ihre erſtaunten Augen beim Eröffnen dieſes \label{K_L00580-1v}\edtext{Briefes}{\lemma{\textnormal{\emph{Briefes}}}\Cendnote{\textnormal{Vgl. A. S.: \emph{Tagebuch}, 26. 8. 1896.
               }}}\label{K_L00580-1}{\\}{[}hs. :{]} zu ſehen intereſſiert mich weniger als zu erfahren,
               wie Ihr vier\pwindex{Beer-Hofmann, Richard 1866-07-11 – 1945-09-26@\textsc{Beer-Hofmann, Richard} (1866-07-11 – 1945-09-26), \emph{Schriftsteller/Schriftstellerin}|pwv}\pwindex{Beer-Hofmann, Paula 25.02.1879 – 30.10.1939@\textsc{Beer-Hofmann, Paula} (25.02.1879 – 30.10.1939)|pwv}\pwindex{Goldmann, Paul 31.01.1865 – 25.09.1935@\textsc{Goldmann, Paul} (31.01.1865 – 25.09.1935), \emph{Schriftsteller/Schriftstellerin, Journalist/Journalistin}|pwv} Menſchen {\\}{[}hs. :{]} beſonders Richard\pwindex{Beer-Hofmann, Richard 1866-07-11 – 1945-09-26@\textsc{Beer-Hofmann, Richard} (1866-07-11 – 1945-09-26), \emph{Schriftsteller/Schriftstellerin}|pw} und Paula\pwindex{Beer-Hofmann, Paula 25.02.1879 – 30.10.1939@\textsc{Beer-Hofmann, Paula} (25.02.1879 – 30.10.1939)|pw}, von der man nicht
               recht weiß, {\\}{[}hs. :{]} ob ſie außer der Seekrankheit noch etwas merkwürdiges
               in Dänemark\oindex{Daenemark@\textbf{Dänemark}, \emph{A.PCLI}|pw} erlebt hat {\\}{[}hs. :{]}  (und ob das Mädchen mit dem Loch im Strumpf ſchon
               »die Epiſode« gena{\geminationn}t werden darf {\\}{[}hs. :{]} weiß man ja auch nicht) Euch befindet.\pend
           
\pstart
           Von Paul\pwindex{Goldmann, Paul 31.01.1865 – 25.09.1935@\textsc{Goldmann, Paul} (31.01.1865 – 25.09.1935), \emph{Schriftsteller/Schriftstellerin, Journalist/Journalistin}|pw} hab ich immer die Empfindung, er {\\}{[}hs. :{]} erinnert ſich auch ſo gut an die Heroinenzeit beim
                  »\textsc{Leopold}\oindex{Hotel und Pension Rudolfshoehe (Leopold Petter)@\textbf{Hotel und Pension Rudolfshöhe (Leopold Petter)}, \emph{Hotel (K.HTL)}|pw}« in \textsc{Ischl}\oindex{Bad Ischl@\textbf{Bad Ischl}, \emph{P.PPL}|pw} vor 2 Jahren \pend
           
\pstart
           {\pb}{[}hs. :{]} wie wir alle, aber gar nicht mehr ordentlich an mich
               und ich hab ihn wirklich {\\}{[}hs. :{]} nur einmal geſehen und ka{\geminationn} da- her unmöglich ſo warm empfinden wie jener Dichter.\pend
           
\pstart
           {[}hs. :{]} Ich verlange mir ſehr zu wiſſen, ob das was wir einmal
               in der Nacht nach der \textsc{Soirée}{\\}{[}hs. :{]} beſprochen, auf Wahrheit beruht – mir will ſcheinen –
               nein – 3mal Nein!! {\\}{[}hs. :{]} ich hoffe ja!: daſs Sie einmal für ein paar Wochen von
               allen inneren Gewöhnungen losgeko{\geminationm}en, {\\}{[}hs. :{]} iſt für Sie wahrſcheinlich ſehr gut, aber \introOben{}für\introOben{} das, was Sie früher beſchäftigt, recht traurig. {\\}{[}hs. :{]} Umſo beſſer! – Daſs Sie in dem zweiten Act\pwindex{Freiwild. Schauspiel in 3 Akten@\emph{Freiwild. Schauspiel in 3 Akten}|pwv} dem Mädel\pwindex{Freiwild. Schauspiel in 3 Akten@\emph{Freiwild. Schauspiel in 3 Akten}|pwv} mehr Leben gegeben haben, wird ſicher {\\}{[}hs. :{]} eine große Wirkung haben, denn wir haben ja ſchon oft
               beſprochen, daß die Christine\pwindex{Liebelei. Schauspiel in drei Akten@\emph{Liebelei. Schauspiel in drei Akten}|pwv}
               davon nicht genug habe {\\}{[}hs. :{]} und das Stück\pwindex{Freiwild. Schauspiel in 3 Akten@\emph{Freiwild. Schauspiel in 3 Akten}|pwv} braucht Rührung, ſonſt wird es trocken und
               revoltierend. Meine {\\}{[}hs. :{]} Neugierde, es zu leſen, kennt keine Grenzen, denn wenn
               man Leute nicht oft ſieht, muſs man in ihren Zeilen leſen \pend
           
\pstart
           {\pb}{[}hs. :{]} und das iſt ſchwer, denn leider drücken immer nur
               einzelne kleine Sachen das Wirkliche aus, {\\}{[}hs. :{]} während große Thaten und große Züge, die darauf
               angelegt ſind, charakteriſtiſch zu wirken, eine ganze Welt von Mißverſtändniſſen
               hervorrufen.\pend
           
\pstart
           {[}hs. :{]} Werden wir heuer endlich theaterſpielen? { }ſind wir zu jung oder zu alt dazu? Oder zu ernſt, oder {\\}{[}hs. :{]} »zu alt, um nur zu ſpielen«? Jedenfalls müſste die
               weibliche Hauptrolle diesmal nicht von Ihnen geſchrieben ſein, {\\}{[}hs. :{]} (warum?). Meine Novelle\pwindex{Geschichte der beiden Liebespaare@\emph{Geschichte der beiden Liebespaare}|pw} werden Sie nie ſehen. Nie heißt nie. Weil ſie ſo ſchlecht iſt. {\\}{[}hs. :{]} Er zeigt nicht einmal die guten Sachen herzu. Doch \uline{müſste} man ihn manchmal leſen, we{\geminationn} die Perſon undeutlich wird. {\\}{[}hs. :{]} Freilich haben meine Sachen wieder das Häßliche, daſs
               alles allzudeutlich geſagt iſt. Ob der Richard\pwindex{Beer-Hofmann, Richard 1866-07-11 – 1945-09-26@\textsc{Beer-Hofmann, Richard} (1866-07-11 – 1945-09-26), \emph{Schriftsteller/Schriftstellerin}|pw}{\\}{[}hs. :{]} wieder etwas ſchreibt, iſt, wie ich reumüthig bekenne,
               für uns \textsc{Altausseer}\oindex{Altaussee@\textbf{Altaussee}, \emph{A.ADM3}|pw} ganz intereſſant, {\\}{[}hs. :{]} ich verſuche mir manchmal vor\introOben{}zu\introOben{}ſtellen wie es wäre, wenn Sie hier wären {\\}{[}hs. :{]} und ob wir alle Drei dabei nicht \uline{viel} netter herauskämen, was ich ganz beſtimmt glaube; ſeien
               Sie \pend
           
\pstart
           {\pb}{[}hs. :{]} nicht bös, aber ich bin ſicher wir würden uns { }ſchrecklich nervös machen und beinahe ſtreiten, denn {\\}{[}hs. :{]} zwei noch ſo gute, gleichgeartete, männliche Naturen
               haben nicht die Größe nett neben einander einherzugehen {\\}{[}hs. :{]} wenn zwiſchen ihnen etwas Halbwahres beunruhigend
               herumwimmelt. Deswegen {\\}{[}hs. :{]} werden Sie doch herkommen, ſchon allein um \strikeout{J}dieſe jugendliche Behauptung von »\uuline{Halb}\uline{wahr}« zu widerlegen, {\\}{[}hs. :{]} wozu Sie ja durch Ihre oft beſprochene Überſchätzung
               der weiblichen »Individualitäten« ſo geeignet ſind. {\\}{[}hs. :{]} Glücklich der, welcher imſtande iſt, Geſtalten zu
               ſchaffen, an die er glaubt, drum laſſen Sie ſich nicht hetzen, {\\}{[}hs. :{]} ſondern glauben Sie ruhig weiter, auf das Wirkliche
               kommt’s nicht an, denn vielleicht exiſtiert es gar nicht. {\\}{[}hs. :{]} Ich glaube, wir brauchen Sie darüber nicht
               aufzuklären, Sie haben ein ſo ſtarkes Wahrheitsgefühl, {\\}{[}hs. :{]} daſs Sie auch den dreifachen Sinn dieſes Briefes
               erkannt haben werden, worüber Sie nächſtens in Wien\oindex{Wien@\textbf{Wien}, \emph{A.ADM2}|pw} mir (nur hier) Auskunft geben können.\pend
           
\pstart
           Herzlich Ihr{\\[\baselineskip]}\spacefill\mbox{Hugo.}\pend
           \leftskip=0em{}\selectlanguage{ngerman}\endnumbering\briefempfaengerindex{Schnitzler, Arthur@\textsc{Schnitzler, Arthur}!zzzSchaffgotsch, Hermine von@\emph{von Hermine von Schaffgotsch}!1896-08-211@{21. {[}8. 1896{]}}|)be}\briefempfaengerindex{Schnitzler, Arthur@\textsc{Schnitzler, Arthur}!zzzHofmannsthal, Hugo von@\emph{von Hugo von Hofmannsthal}!1896-08-211@{21. {[}8. 1896{]}}|)be}\mylabel{L00580h}  \normalsize

\doendnotes{C}
\bigskip
\vfill

\clearpage

\footnotesize

\lohead{\textsc{register}}

% Definiere theindex-Environment komplett neu ohne reledmac
\makeatletter
\renewenvironment{theindex}{%
  \section*{\indexname}%
  \setlength{\parindent}{0pt}%
  \setlength{\parskip}{0pt plus 0.3pt}%
  \let\item\@idxitem
}{%
  \clearpage
}
\makeatother

\IfFileExists{\jobname-pw.ind}{\input{\jobname-pw.ind}}{}

\end{document}

      