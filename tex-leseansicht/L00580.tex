%% latex-leseansicht-vorspann.tex
%% Vorspann für die Leseansicht.
%% Lädt die gemeinsame Datei latex-vorspann.tex mit nicht gesetztem Schalter.

\newif\ifkorrekturansicht
\korrekturansichtfalse

\input{../tex-inputs/latex-vorspann}


               \section[Hugo von Hofmannsthal und Hermine Benedict an Arthur Schnitzler, 21. {[}8. 1896{]}]{ Hugo von Hofmannsthal und Hermine Benedict an Arthur Schnitzler,
                    21. {[}8. 1896{]}}\nopagebreak\mylabel{v}\rehead{ }\begin{ledgroupsized}[t]{13cm}\normalsize\beginnumbering\briefempfaengerindex{Schnitzler, Arthur@\textsc{Schnitzler, Arthur}!zzzSchaffgotsch, Hermine von@\emph{von Hermine von Schaffgotsch}!1896-08-211@{21. {[}8. 1896{]}}|(be}\briefempfaengerindex{Schnitzler, Arthur@\textsc{Schnitzler, Arthur}!zzzHofmannsthal, Hugo von@\emph{von Hugo von Hofmannsthal}!1896-08-211@{21. {[}8. 1896{]}}|(be} \toendnotes[C]{\smallbreak\pagebreak[2]} \Standort{CUL, Schnitzler, B 43.}
\physDesc{Brief, 1 Blatt, 4 Seiten
\newline{}Handschrift Hugo von Hofmannsthal: schwarze Tinte, deutsche Kurrent\newline{}Handschrift Hermine von Schaffgotsch: schwarze Tinte, deutsche Kurrent
\newline{}Schnitzler: mit Bleistift Monat und Jahr ergänzt: »Aug. 96« \newline{}Ordnung: mit Bleistift von unbekannter Hand nummeriert:
                                        »79« }\buchAbdrucke{\weitereDrucke{Hugo von Hofmannsthal, Arthur Schnitzler: \emph{Briefwechsel}. Hg. Therese Nickl und Heinrich Schnitzler. Frankfurt am Main: \emph{S. Fischer} 1964, S. 72–74.} }\toendnotes[C]{\smallbreak}\pstart
           \raggedleft{}{\pb}Alt.auſſee\oindex{Altaussee@\textbf{Altaussee}|pw}{ }21\textsuperscript{ten}\pend
           \pstart{}lieber Arthur!\pend\pstart
           {[}hs. Schaffgotsch:{]} Ihre erſtaunten Augen beim Eröffnen dieſes \label{K_L00580_1v}\edtext{Briefes}{\lemma{\textnormal{\emph{Briefes}}}\Cendnote{\textnormal{vgl. A. S.: \emph{Tagebuch}, 26. 8. 1896}}}\label{K_L00580_1h}{\\}{[}hs. Hofmannsthal:{]} zu ſehen intereſſiert mich weniger als zu erfahren,
                    wie Ihr vier\pwindex{Beer-Hofmann, Richard 11.07.1866 – 26.09.1945@\textsc{Beer-Hofmann, Richard} (11.07.1866 – 26.09.1945), \emph{Schriftsteller}|pwv}\pwindex{Beer-Hofmann, Paula 25.02.1879 – 30.10.1939@\textsc{Beer-Hofmann, Paula} (25.02.1879 – 30.10.1939)|pwv}\pwindex{Goldmann, Paul 31.01.1865 – 25.09.1935@\textsc{Goldmann, Paul} (31.01.1865 – 25.09.1935), \emph{Schriftsteller, Journalist}|pwv}
                    Menſchen {\\}{[}hs. Schaffgotsch:{]} beſonders Richard\pwindex{Beer-Hofmann, Richard 11.07.1866 – 26.09.1945@\textsc{Beer-Hofmann, Richard} (11.07.1866 – 26.09.1945), \emph{Schriftsteller}|pw} und Paula\pwindex{Beer-Hofmann, Paula 25.02.1879 – 30.10.1939@\textsc{Beer-Hofmann, Paula} (25.02.1879 – 30.10.1939)|pw}, von der man
                    nicht recht weiß, {\\}{[}hs. Hofmannsthal:{]} ob ſie außer der Seekrankheit noch
                    etwas merkwürdiges in Dänemark\oindex{Daenemark@\textbf{Dänemark}|pw} erlebt hat
                        {\\}{[}hs. Schaffgotsch:{]}  (und ob das Mädchen mit dem Loch im Strumpf
                    ſchon »die Epiſode« gena{\geminationn}t werden darf {\\}{[}hs. Hofmannsthal:{]} weiß man ja auch nicht) Euch befindet.\pend
           \pstart
           Von Paul\pwindex{Goldmann, Paul 31.01.1865 – 25.09.1935@\textsc{Goldmann, Paul} (31.01.1865 – 25.09.1935), \emph{Schriftsteller, Journalist}|pw} hab ich immer die Empfindung, er
                        {\\}{[}hs. Schaffgotsch:{]} erinnert ſich auch ſo gut an die Heroinenzeit
                    beim »\textsc{Leopold}\oindex{Hotel und Pension Rudolfshoehe (Leopold Petter)@\textbf{Hotel und Pension Rudolfshöhe (Leopold Petter)}|pw}« in \textsc{Ischl}\oindex{Bad Ischl@\textbf{Bad Ischl}|pw} vor 2 Jahren \pend
           \pstart
           {\pb}{[}hs. Hofmannsthal:{]} wie wir alle, aber gar nicht mehr ordentlich an mich
                    und ich hab ihn wirklich {\\}{[}hs. Schaffgotsch:{]} nur einmal geſehen und ka{\geminationn} da- her unmöglich ſo warm empfinden wie jener
                    Dichter.\pend
           \pstart
           {[}hs. Hofmannsthal:{]} Ich verlange mir ſehr zu wiſſen, ob das was wir
                    einmal in der Nacht nach der \textsc{Soirée}{\\}{[}hs. Schaffgotsch:{]} beſprochen, auf Wahrheit beruht – mir will ſcheinen
                    – nein – 3mal Nein!! {\\}{[}hs. Hofmannsthal:{]} ich hoffe ja!: daſs Sie einmal
                    für ein paar Wochen von allen inneren Gewöhnungen losgeko{\geminationm}en, {\\}{[}hs. Schaffgotsch:{]} iſt für Sie
                    wahrſcheinlich ſehr gut, aber \introOben{}für\introOben{} das, was Sie früher
                    beſchäftigt, recht traurig. {\\}{[}hs. Hofmannsthal:{]} Umſo beſſer! – Daſs Sie
                    in dem zweiten Act\pwindex{Schnitzler, Arthur 15.05.1862 – 21.10.1931@\textsc{Schnitzler, Arthur} (15.05.1862 – 21.10.1931), \emph{Schriftsteller, Mediziner}!Freiwild. Schauspiel in 3 Akten1896@\strich\emph{Freiwild. Schauspiel in 3 Akten} {[}1896{]}|pwv} dem Mädel\pwindex{Schnitzler, Arthur 15.05.1862 – 21.10.1931@\textsc{Schnitzler, Arthur} (15.05.1862 – 21.10.1931), \emph{Schriftsteller, Mediziner}!Freiwild. Schauspiel in 3 Akten1896@\strich\emph{Freiwild. Schauspiel in 3 Akten} {[}1896{]}|pwv} mehr Leben gegeben
                    haben, wird ſicher {\\}{[}hs. Schaffgotsch:{]} eine große Wirkung haben, denn
                    wir haben ja ſchon oft beſprochen, daß die Christine\pwindex{Schnitzler, Arthur 15.05.1862 – 21.10.1931@\textsc{Schnitzler, Arthur} (15.05.1862 – 21.10.1931), \emph{Schriftsteller, Mediziner}!Liebelei. Schauspiel in drei Akten9. 10. 1895@\strich\emph{Liebelei. Schauspiel in drei Akten} {[}9. 10. 1895{]}|pwv} davon nicht genug habe {\\}{[}hs. Hofmannsthal:{]} und das Stück\pwindex{Schnitzler, Arthur 15.05.1862 – 21.10.1931@\textsc{Schnitzler, Arthur} (15.05.1862 – 21.10.1931), \emph{Schriftsteller, Mediziner}!Freiwild. Schauspiel in 3 Akten1896@\strich\emph{Freiwild. Schauspiel in 3 Akten} {[}1896{]}|pwv} braucht Rührung, ſonſt wird es trocken und revoltierend. Meine
                        {\\}{[}hs. Schaffgotsch:{]} Neugierde, es zu leſen, kennt keine Grenzen,
                    denn wenn man Leute nicht oft ſieht, muſs man in ihren Zeilen leſen \pend
           \pstart
           {\pb}{[}hs. Hofmannsthal:{]} und das iſt ſchwer, denn leider drücken immer nur
                    einzelne kleine Sachen das Wirkliche aus, {\\}{[}hs. Schaffgotsch:{]} während
                    große Thaten und große Züge, die darauf angelegt ſind, charakteriſtiſch zu
                    wirken, eine ganze Welt von Mißverſtändniſſen hervorrufen.\pend
           \pstart
           {[}hs. Hofmannsthal:{]} Werden wir heuer endlich theaterſpielen? { }ſind wir zu jung oder zu alt dazu? Oder zu ernſt,
                    oder {\\}{[}hs. Schaffgotsch:{]} »zu alt, um nur zu ſpielen«? Jedenfalls müſste
                    die weibliche Hauptrolle diesmal nicht von Ihnen geſchrieben ſein, {\\}{[}hs. Hofmannsthal:{]} (warum?). Meine Novelle\pwindex{Hofmannsthal, Hugo von 01.02.1874 – 15.07.1929@\textsc{Hofmannsthal, Hugo von} (01.02.1874 – 15.07.1929), \emph{Schriftsteller}!Geschichte der beiden Liebespaare1978@\strich\emph{Geschichte der beiden Liebespaare} {[}1978{]}|pw}
                    werden Sie nie ſehen. Nie heißt nie. Weil ſie ſo ſchlecht iſt. {\\}{[}hs. Schaffgotsch:{]} Er zeigt nicht einmal die guten Sachen herzu. Doch \uline{müſste} man ihn manchmal leſen, we{\geminationn} die Perſon undeutlich wird. {\\}{[}hs. Hofmannsthal:{]} Freilich haben meine Sachen wieder das Häßliche, daſs alles
                    allzudeutlich geſagt iſt. Ob der Richard\pwindex{Beer-Hofmann, Richard 11.07.1866 – 26.09.1945@\textsc{Beer-Hofmann, Richard} (11.07.1866 – 26.09.1945), \emph{Schriftsteller}|pw}{\\}{[}hs. Schaffgotsch:{]} wieder etwas ſchreibt, iſt, wie ich reumüthig
                    bekenne, für uns \textsc{Altausseer}\oindex{Altaussee@\textbf{Altaussee}|pw} ganz intereſſant, {\\}{[}hs. Hofmannsthal:{]} ich verſuche mir manchmal
                        vor\introOben{}zu\introOben{}ſtellen wie es wäre, wenn Sie hier wären
                        {\\}{[}hs. Schaffgotsch:{]} und ob wir alle Drei dabei nicht \uline{viel} netter herauskämen, was ich ganz beſtimmt glaube;
                    ſeien Sie \pend
           \pstart
           {\pb}{[}hs. Hofmannsthal:{]} nicht bös, aber ich bin ſicher wir würden uns { }ſchrecklich nervös machen und beinahe ſtreiten,
                    denn {\\}{[}hs. Schaffgotsch:{]} zwei noch ſo gute, gleichgeartete, männliche
                    Naturen haben nicht die Größe nett neben einander einherzugehen {\\}{[}hs. Hofmannsthal:{]} wenn zwiſchen ihnen etwas Halbwahres beunruhigend
                    herumwimmelt. Deswegen {\\}{[}hs. Schaffgotsch:{]} werden Sie doch herkommen,
                    ſchon allein um \strikeout{J}dieſe jugendliche Behauptung von
                        »\uuline{Halb}\uline{wahr}« zu widerlegen, {\\}{[}hs. Hofmannsthal:{]} wozu Sie ja durch Ihre oft beſprochene Überſchätzung der weiblichen
                    »Individualitäten« ſo geeignet ſind. {\\}{[}hs. Schaffgotsch:{]} Glücklich der,
                    welcher imſtande iſt, Geſtalten zu ſchaffen, an die er glaubt, drum laſſen Sie
                    ſich nicht hetzen, {\\}{[}hs. Hofmannsthal:{]} ſondern glauben Sie ruhig weiter,
                    auf das Wirkliche kommt’s nicht an, denn vielleicht exiſtiert es gar nicht.
                        {\\}{[}hs. Schaffgotsch:{]} Ich glaube, wir brauchen Sie darüber nicht
                    aufzuklären, Sie haben ein ſo ſtarkes Wahrheitsgefühl, {\\}{[}hs. Hofmannsthal:{]} daſs Sie auch den dreifachen Sinn dieſes Briefes erkannt
                    haben werden, worüber Sie nächſtens in Wien\oindex{Wien@\textbf{Wien}|pw} mir
                    (nur hier) Auskunft geben können.\pend
           \pstart
           Herzlich Ihr{\\[\baselineskip]}\spacefill\mbox{Hugo.}\pend
           \leftskip=0em{}\endnumbering\briefempfaengerindex{Schnitzler, Arthur@\textsc{Schnitzler, Arthur}!zzzSchaffgotsch, Hermine von@\emph{von Hermine von Schaffgotsch}!1896-08-211@{21. {[}8. 1896{]}}|)be}\briefempfaengerindex{Schnitzler, Arthur@\textsc{Schnitzler, Arthur}!zzzHofmannsthal, Hugo von@\emph{von Hugo von Hofmannsthal}!1896-08-211@{21. {[}8. 1896{]}}|)be}\mylabel{h}\end{ledgroupsized}  \newcommand{\dateiname}{L00580}\newcommand{\titel}{Hugo von Hofmannsthal und Hermine Benedict an Arthur Schnitzler, 21. [8. 1896]}\newcommand{\editorInnen}{Martin Anton Müller und Gerd-Hermann Susen}
            \footnotesize
\begin{ledgroupsized}[t]{11.5cm}
\doendnotes{C}
\end{ledgroupsized}
         %% latex-leseansicht-abspann.tex
%% Abspann für die Leseansicht.
%% Der Schalter \ifkorrekturansicht ist bereits durch den Vorspann gesetzt.

%% latex-abspann.tex
%% Gemeinsamer Abspann für Korrekturansicht und Leseansicht.
%% Setzt den Schalter \ifkorrekturansicht voraus (gesetzt in den
%% einbindenden Dateien latex-korrekturansicht-abspann.tex bzw.
%% latex-leseansicht-abspann.tex).
%% ---------------------------------------------------------------

\normalsize

% Das esempio-Environment wird nur in der Leseansicht benötigt
\ifkorrekturansicht\else
\newenvironment{esempio}[3]%
{
    \vspace{1.5ex}
    \rlap{\underline{#1}}
    \par
    \setlength{\parindent}{0cm}
    \nopagebreak
    \leftskip=#2cm
    \rightskip=#3cm
}
{
    \par
}
\fi

\doendnotes{C}
\bigskip
\vfill

\clearpage

\footnotesize

\ifkorrekturansicht
  \lohead{\textsc{register}}
\fi

% theindex-Environment neu definieren ohne reledmac
\makeatletter
\renewenvironment{theindex}{%
  \ifkorrekturansicht
    \section*{\indexname}%
  \else
    \subsubsection*{Index der erwähnten Entitäten}%
  \fi
  \setlength{\parindent}{0pt}%
  \setlength{\parskip}{0pt plus 0.3pt}%
  \let\item\@idxitem
}{%
  \ifkorrekturansicht\clearpage\fi
}
\makeatother

\IfFileExists{\jobname-pw.ind}{\input{\jobname-pw.ind}}{}

% Quellenangabe nur in der Leseansicht
\ifkorrekturansicht\else
% Fallback-Definitionen, falls die .tex-Datei \titel etc. nicht gesetzt hat
\providecommand{\titel}{}
\providecommand{\editorInnen}{}
\providecommand{\dateiname}{\jobname}

\vspace{3cm}

\vfill

\footnotesize
\textsc{Quelle}: \titel. Herausgegeben von {\editorInnen}. In: \emph{Arthur Schnitzler: Briefwechsel mit Autorinnen und Autoren}.
 Digitale Edition, https://schnitzler-briefe.acdh.oeaw.ac.at/{\dateiname}.html (Stand \today)
\fi

\end{document}


      