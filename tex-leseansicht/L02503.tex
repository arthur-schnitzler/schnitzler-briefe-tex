%% latex-leseansicht-vorspann.tex
%% Vorspann für die Leseansicht.
%% Lädt die gemeinsame Datei latex-vorspann.tex mit nicht gesetztem Schalter.

\newif\ifkorrekturansicht
\korrekturansichtfalse

\input{../tex-inputs/latex-vorspann}


\section[Hugo Hofmannsthal an Arthur Schnitzler, 10. 7. [1928]]{L02503 Hugo Hofmannsthal an Arthur Schnitzler, 10. 7. [1928]}
\nopagebreak\mylabel{L02503v}
\rehead{ }\normalsize\beginnumbering\briefempfaengerindex{Schnitzler, Arthur@\textsc{Schnitzler, Arthur}!zzzHofmannsthal, Hugo von@\emph{von Hugo von Hofmannsthal}!1928-07-101@{10. 7. [1928]}|(be}
\toendnotes[C]{\smallbreak\pagebreak[2]}
\correspDesc{Versand  durch Hugo von Hofmannsthal am 10. 7. [1928] in Breitenstein am Semmering
\newline{}Erhalt  durch Arthur Schnitzler im Zeitraum [11. 7. 1928
                  – 15. 7. 1928?] in Wien}\toendnotes[C]{\smallbreak}
\Standort{CUL, Schnitzler, B 43.}
\physDesc{Brief, 2 Blätter, 4 Seiten, 3266 Zeichen
\newline{}Handschrift: schwarze Tinte, lateinische Kurrent
\newline{}Schnitzler: 1) mit Bleistift datiert: »10/7 28« und beschriftet: »HvH«  2) mit rotem Buntstift mehrere Unterstreichungen
\newline{}Ordnung: 1) mit Bleistift von unbekannter Hand nummeriert: »\strikeout{371}«  2) mit Bleistift von unbekannter Hand nummeriert:
                                    »380«}
\buchAbdrucke{\weitereDrucke{Hugo von Hofmannsthal, Arthur Schnitzler: \emph{Briefwechsel}. Herausgegeben von Therese Nickl und Heinrich Schnitzler. Frankfurt am Main: \emph{S. Fischer} 1964, S. 309.} }\toendnotes[C]{\smallbreak}
\pstart
           {\pb}Haus Mahler\oindex{Haus Mahler@\textbf{Haus Mahler}, \emph{Gebäude}|pw}\hspace*{1em}Breitenstein am Semmering.\oindex{Breitenstein am Semmering@\textbf{Breitenstein am Semmering}|pw}\pend
           
\pstart
           \centering{}10\textsuperscript{ter} Juli.\pend
           \vspace{0.5em}
\pstart
           mein lieber Arthur,\hspace*{1.5em}schon seit ich das Buch gelesen habe, wollte ich
               Ihnen ein paar Worte über den Roman »Therese\pwindex{Schnitzler, Arthur 15.\,5.\,1862 Wien – 21.\,10.\,1931 ebd.@\textsc{Schnitzler, Arthur} (15.\,5.\,1862 Wien – 21.\,10.\,1931 ebd.), \emph{Schriftsteller, Mediziner}!Therese. Chronik eines Frauenlebens@\strich\emph{Therese. Chronik eines Frauenlebens}|pw}«
               sagen. Aber der letzte Monat war bei mir sehr unruhig, durch die \label{K_L02503-1v}\edtext{beiden Opernpremieren\pwindex{Hofmannsthal, Hugo von 1.\,2.\,1874 Wien – 15.\,7.\,1929 Rodaun@\textsc{Hofmannsthal, Hugo von} (1.\,2.\,1874 Wien – 15.\,7.\,1929 Rodaun), \emph{Schriftsteller}!ägyptische Helena@\strich\emph{Die ägyptische Helena}|pwv}\pwindex{\textcolor{red}{\textsuperscript{XXXX indx1}}!ägyptische Helena@\strich\emph{Die ägyptische Helena}|pwv}}{\lemma{\textnormal{\emph{beiden Opernpremieren}}}\Cendnote{\textnormal{\emph{Die ägyptische Helena}\pwindex{Hofmannsthal, Hugo von 1.\,2.\,1874 Wien – 15.\,7.\,1929 Rodaun@\textsc{Hofmannsthal, Hugo von} (1.\,2.\,1874 Wien – 15.\,7.\,1929 Rodaun), \emph{Schriftsteller}!ägyptische Helena@\strich\emph{Die ägyptische Helena}|pwk}\pwindex{\textcolor{red}{\textsuperscript{XXXX indx1}}!ägyptische Helena@\strich\emph{Die ägyptische Helena}|pwk} wurde am
                     6. 6. 1928 in Dresden\oindex{Dresden@\textbf{Dresden}|pwk}, am
                     12. 6. 1928 in Wien\oindex{Wien@\textbf{Wien}, \emph{Verwaltungsgebiet}|pwk}
                  aufgeführt.}}}\label{K_L02503-1} und verschiedenes Andere. Auch war ich dazwischen \label{K_L02503-2v}\edtext{eine Woche in Salzburg\oindex{Salzburg@\textbf{Salzburg}, \emph{Verwaltungsgebiet}|pw}}{\lemma{\textnormal{\emph{eine Woche in Salzburg}}}\Cendnote{\textnormal{vom 19. 6. 1928 bis zum
                     25. 6. 1928}}}\label{K_L02503-2}, um Reinhardt\pwindex{Reinhardt, Max 9.\,9.\,1873 Baden bei Wien – 30.\,10.\,1943 New York City@\textsc{Reinhardt, Max} (9.\,9.\,1873 Baden bei Wien – 30.\,10.\,1943 New York City), \emph{Theaterleiter, Regisseur, Schauspieler}|pw} bei einem Film\pwindex{Hofmannsthal, Hugo von 1.\,2.\,1874 Wien – 15.\,7.\,1929 Rodaun@\textsc{Hofmannsthal, Hugo von} (1.\,2.\,1874 Wien – 15.\,7.\,1929 Rodaun), \emph{Schriftsteller}!Film für Lillian Gish@\strich\emph{Film für Lillian Gish}|pwv}\pwindex{Reinhardt, Max 9.\,9.\,1873 Baden bei Wien – 30.\,10.\,1943 New York City@\textsc{Reinhardt, Max} (9.\,9.\,1873 Baden bei Wien – 30.\,10.\,1943 New York City), \emph{Theaterleiter, Regisseur, Schauspieler}!Film für Lillian Gish@\strich\emph{Film für Lillian Gish}|pwv} zu helfen, dies nur aus
               dem Grund, weil es – im Fall des Gelingens – ein Stück Geld einträgt und ich alles
               daran setzen möchte für Christiane\pwindex{Zimmer, Christiane 14.\,5.\,1902 Rodaun – 5.\,1.\,1987 New York City@\textsc{Zimmer, Christiane} (14.\,5.\,1902 Rodaun – 5.\,1.\,1987 New York City)|pw} ein kleines
               Haus in Heidelberg\oindex{Heidelberg@\textbf{Heidelberg}, \emph{Hauptstadt}|pw} zu kaufen (natürlich in den
               bescheidensten Dimensionen) – denn die Wohnverhältnisse dort sind unerträglich.\pend
           
\pstart
           Sie haben nicht auf mich gewartet, um zu hören, dass Sie in einer Epoche in der es
               sehr wenige Meister gibt, ein Meister der Erzählung sind. In allen Ihren kurzen und
               mittelgroßen Erzählungen ist ein wunderbar sicheres Maßgefühl wirksam – und dadurch,
               durch ihre schönen Maße, bleiben sie auch so schön und lebendig in der Erinnerung.
                  {\pb}Dabei ist in ihnen alles mit
               sparsamen aber sehr reinen Farben gemalt, die Abstufungen der Farbe mit dem
               sichersten Instinct hingesetzt, das Ganze ist nie grellbunt, nie aber stumpf – von
               den ungeheuren rhythmischen Vorzügen aber will ich gar nicht sprechen. Die große
               Lebenserzählung Therese\pwindex{Schnitzler, Arthur 15.\,5.\,1862 Wien – 21.\,10.\,1931 ebd.@\textsc{Schnitzler, Arthur} (15.\,5.\,1862 Wien – 21.\,10.\,1931 ebd.), \emph{Schriftsteller, Mediziner}!Therese. Chronik eines Frauenlebens@\strich\emph{Therese. Chronik eines Frauenlebens}|pw} aber hat mich besonders
               gefesselt und beschäftigt. Schon der Stoff gehört ganz nur Ihnen. Indem Sie diesen
               Stoff wählten: das Leben einer Wien\oindex{Wien@\textbf{Wien}, \emph{Verwaltungsgebiet}|pw}er Gouvernante
               – war schon eine ganze Welt hingestellt, und ein großer Reichtum von Aspecten, Sti{\geminationm}ungen, Gefühlen und gedankenhaften Halbgefühlen im
               verstehenden Leser gesichert. Ganz besonders groß aber tritt Ihr Vorzug, einem Stoff
               den Rhythmus zu geben, wodurch er Dichtung wird, hier hervor. Eben was dem stumpfen
               Leser monoton scheinen kö{\geminationn}te, dass sich sozusagen die
               Figur des Erlebnisses bis zur beabsichtigten Unzählbarkeit wiederholt, das hat Ihnen
               ermöglicht, Ihre rhythmische Kraft bis zum Zauberhaften zu entfalten. Es sind diese
               Vorzüge, die ein Kunstwerk über viele andere scheinbar ähnliche, bis zur
               Unvergleichbarkeit erheben, und die {\pb}es auf lange lebendig erhalten
               werden.\pend
           
\pstart
           Über Christianes\pwindex{Zimmer, Christiane 14.\,5.\,1902 Rodaun – 5.\,1.\,1987 New York City@\textsc{Zimmer, Christiane} (14.\,5.\,1902 Rodaun – 5.\,1.\,1987 New York City)|pw}{ }\label{K_L02503-3v}\edtext{Vermählung}{\lemma{\textnormal{\emph{Vermählung}}}\Cendnote{\textnormal{Diese
                  hatte Mitte Juni 1928 stattgefunden.}}}\label{K_L02503-3} freuen wir uns sehr. Sie
               hat ein besonders liebenswertes Wesen, einen sehr schönen loyalen Character, viel
               Verstand, aber einen menschlichen keinen frauenhaften, und gerade die subtilen Waffen
               für den Lebenskampf, die nur der Frau, je mehr Frau sie ist, umso wirksamer gegeben
               sind, sind ihr versagt. Es war vielleicht zu fürchten dass gerade der Mann, der ihren
               Wert zu erkennen bestimmt war, sich unter den Besten dieser Generation, den
               Gefallenen, befunden hätte. Aber dieser\pwindex{Zimmer, Heinrich 6.\,12.\,1890 Greifswald – 20.\,3.\,1943 New York City@\textsc{Zimmer, Heinrich} (6.\,12.\,1890 Greifswald – 20.\,3.\,1943 New York City), \emph{Indologe}|pwv} gerade, den sie nun gefunden hat, ist aus vierjährigem
               Schützengrabendasein munter und unversehrt hervorgestiegen.\pend
           
\pstart
           Ich lernte ihn diesen Winter in Heidelberg\oindex{Heidelberg@\textbf{Heidelberg}, \emph{Hauptstadt}|pw} kennen, und ich muss sagen, er gefiel mir sehr. Alles was er
               sagte, und wie er es sagte, war mir gleich sympathisch. Dabei streifte mich nicht
               einmal der Gedanke dass die {\pb}zwischen ihm und Christiane\pwindex{Zimmer, Christiane 14.\,5.\,1902 Rodaun – 5.\,1.\,1987 New York City@\textsc{Zimmer, Christiane} (14.\,5.\,1902 Rodaun – 5.\,1.\,1987 New York City)|pw} bestehende
               muntere gesprächige Freundschaft je zu etwas anderem führen könnte, als eben zu
               Freundschaft.\pend
           
\pstart
           Dass Sie, wie ich von Freunden öfters gehört habe, an Ihrem \label{K_L02503-4v}\edtext{Schwiegersohn\pwindex{Cappellini, Arnoldo 10.\,8.\,1889 Venedig – 8.\,5.\,1954 Rom@\textsc{Cappellini, Arnoldo} (10.\,8.\,1889 Venedig – 8.\,5.\,1954 Rom)|pwv}}{\lemma{\textnormal{\emph{Schwiegersohn}}}\Cendnote{\textnormal{Die Hochzeit der noch nicht 18-jährigen
                     Lili\pwindex{Cappellini, Lili 13.\,9.\,1909 Wien – 26.\,7.\,1928 Venedig@\textsc{Cappellini, Lili} (13.\,9.\,1909 Wien – 26.\,7.\,1928 Venedig)|pwk} mit dem italienischen\oindex{Italien@\textbf{Italien}|pwk} Faschisten Arnoldo Cappellini\pwindex{Cappellini, Arnoldo 10.\,8.\,1889 Venedig – 8.\,5.\,1954 Rom@\textsc{Cappellini, Arnoldo} (10.\,8.\,1889 Venedig – 8.\,5.\,1954 Rom)|pwk} hatte am 30. 6. 1927 stattgefunden.}}}\label{K_L02503-4} wirklich einen Freund
               gewonnen haben, und eine Bereicherung Ihres Lebens, nehme ich als ein gutes Omen.\pend
           
\pstart
           Ich drücke Ihnen herzlich die Hand, lieber guter Arthur.\pend
           \pstart Ihr\spacefill\mbox{Hugo.}\pend{}\selectlanguage{ngerman}\endnumbering\briefempfaengerindex{Schnitzler, Arthur@\textsc{Schnitzler, Arthur}!zzzHofmannsthal, Hugo von@\emph{von Hugo von Hofmannsthal}!1928-07-101@{10. 7. [1928]}|)be}\mylabel{L02503h}  \newcommand{\dateiname}{L02503}\newcommand{\titel}{Hugo Hofmannsthal an Arthur Schnitzler, 10. 7. [1928]}\newcommand{\editorInnen}{Martin Anton Müller und Gerd-Hermann Susen}%% latex-leseansicht-abspann.tex
%% Abspann für die Leseansicht.
%% Der Schalter \ifkorrekturansicht ist bereits durch den Vorspann gesetzt.

%% latex-abspann.tex
%% Gemeinsamer Abspann für Korrekturansicht und Leseansicht.
%% Setzt den Schalter \ifkorrekturansicht voraus (gesetzt in den
%% einbindenden Dateien latex-korrekturansicht-abspann.tex bzw.
%% latex-leseansicht-abspann.tex).
%% ---------------------------------------------------------------

\normalsize

% Das esempio-Environment wird nur in der Leseansicht benötigt
\ifkorrekturansicht\else
\newenvironment{esempio}[3]%
{
    \vspace{1.5ex}
    \rlap{\underline{#1}}
    \par
    \setlength{\parindent}{0cm}
    \nopagebreak
    \leftskip=#2cm
    \rightskip=#3cm
}
{
    \par
}
\fi

\doendnotes{C}
\bigskip
\vfill

\clearpage

\footnotesize

\ifkorrekturansicht
  \lohead{\textsc{register}}
\fi

% theindex-Environment neu definieren ohne reledmac
\makeatletter
\renewenvironment{theindex}{%
  \ifkorrekturansicht
    \section*{\indexname}%
  \else
    \subsubsection*{Index der erwähnten Entitäten}%
  \fi
  \setlength{\parindent}{0pt}%
  \setlength{\parskip}{0pt plus 0.3pt}%
  \let\item\@idxitem
}{%
  \ifkorrekturansicht\clearpage\fi
}
\makeatother

\IfFileExists{\jobname-pw.ind}{\input{\jobname-pw.ind}}{}

% Quellenangabe nur in der Leseansicht
\ifkorrekturansicht\else
% Fallback-Definitionen, falls die .tex-Datei \titel etc. nicht gesetzt hat
\providecommand{\titel}{}
\providecommand{\editorInnen}{}
\providecommand{\dateiname}{\jobname}

\vspace{3cm}

\vfill

\footnotesize
\textsc{Quelle}: \titel. Herausgegeben von {\editorInnen}. In: \emph{Arthur Schnitzler: Briefwechsel mit Autorinnen und Autoren}.
 Digitale Edition, https://schnitzler-briefe.acdh.oeaw.ac.at/{\dateiname}.html (Stand \today)
\fi

\end{document}


