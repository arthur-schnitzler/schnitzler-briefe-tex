%% latex-korrekturansicht-vorspann.tex
%% Vorspann für die Korrekturansicht.
%% Lädt die gemeinsame Datei latex-vorspann.tex mit gesetztem Schalter.

\newif\ifkorrekturansicht
\korrekturansichttrue

\input{../tex-inputs/latex-vorspann}


\section[ Felix Salten an Arthur Schnitzler, 11. 8. 1903]{L03342 Felix Salten an Arthur Schnitzler, 11. 8. 1903}
\nopagebreak\mylabel{L03342v}
\rehead{ }\normalsize\beginnumbering\briefempfaengerindex{Schnitzler, Arthur@\textsc{Schnitzler, Arthur}!zzzSalten, Felix@\emph{von Felix Salten}!1903-08-111@{11. 8. 1903}|(be}
\toendnotes[C]{\smallbreak\pagebreak[2]}\Standort{CUL, Schnitzler, B 89, A 2.}
\physDesc{Brief, 1 Blatt, 1 Seite, 604 Zeichen
\newline{}Handschrift: blaue Tinte, lateinische Kurrent
\newline{}Ordnung: mit Bleistift von unbekannter Hand nummeriert: »167« }\toendnotes[C]{\smallbreak}
\pstart
           \raggedleft{}{\pb}11. VIII. 03\pend
           \vspace{0.5em}
\pstart
           Lieber, Ihre Sendung\pwindex{Studie@\emph{Studie}|pwv} hab ich heute bei meiner Rückkehr
               vorgefunden und gleich gelesen. Es ist nichts besonderes, aber doch so,– dass man es in der Sonntags-Zeit\pwindex{Zeit@\emph{Die Zeit}|pw} einmal bringen kann, was ich denn auch mit Vergnügen thue, da
               es Ihnen offenbar sehr erwünscht ist. Hab’ ich Ihren Brief recht gelesen, so soll die
                  »\label{K_L03342-1v}\edtext{Studie\pwindex{Studie@\emph{Studie}|pw}}{\lemma{\textnormal{\emph{Studie}}}\Cendnote{\textnormal{E. Mewes-Béha\pwindex{Mewes-Beha, Emilie @\textsc{Mewes-Béha, Emilie}, \emph{Schriftsteller/Schriftstellerin, Übersetzer/Übersetzerin}|pwk}: \emph{Studie}\pwindex{Studie@\emph{Studie}|pwk}. In: \emph{Die
                        Zeit}\pwindex{Zeit@\emph{Die Zeit}|pwk}, Jg. 2, Nr. 364, 4. 10. 1903, Die
                     Sonntags-Zeit, S. 2–3.}}}\label{K_L03342-1}« erst in der zweiten Hälfte September publizirt werden. Ich habe das auf dem Mscpt
               vorgemerkt.\pend
           
\pstart
           Heute{ }Nachmittag um ¾ 2 hat meine Frau\pwindex{Salten, Ottilie 07.03.1868 – 22.06.1942@\textsc{Salten, Ottilie} (07.03.1868 – 22.06.1942), \emph{Schauspieler/Schauspielerin}|pwv} einen \label{K_L03342-2v}\edtext{Buben\pwindex{Salten, Paul 11.08.1903 – 08.05.1937@\textsc{Salten, Paul} (11.08.1903 – 08.05.1937), \emph{Filmcutter/Filmcutterin}|pwv}}{\lemma{\textnormal{\emph{Buben}}}\Cendnote{\textnormal{Paul Salten\pwindex{Salten, Paul 11.08.1903 – 08.05.1937@\textsc{Salten, Paul} (11.08.1903 – 08.05.1937), \emph{Filmcutter/Filmcutterin}|pwk}, siehe auch A. S.: \emph{Tagebuch}, 12. 8. 1903.
               }}}\label{K_L03342-2} bekommen und befindet sich sehr wol. Wir freuen uns sehr, wie Sie sich denken
               können. Wollen Sie es, bitte, an Olga\pwindex{Schnitzler, Olga 17.01.1882 – 13.01.1970@\textsc{Schnitzler, Olga} (17.01.1882 – 13.01.1970), \emph{Schauspieler/Schauspielerin, Sänger/Sängerin}|pw}
               mittheilen.\pend
           
\pstart
           Herzlichst {\\[\baselineskip]}Ihr {\\[\baselineskip]}\spacefill\mbox{Salten}\pend
           \leftskip=0em{}\selectlanguage{ngerman}\endnumbering\briefempfaengerindex{Schnitzler, Arthur@\textsc{Schnitzler, Arthur}!zzzSalten, Felix@\emph{von Felix Salten}!1903-08-111@{11. 8. 1903}|)be}\mylabel{L03342h}  \normalsize

\doendnotes{C}
\bigskip
\vfill

\clearpage

\footnotesize

\lohead{\textsc{register}}

% Definiere theindex-Environment komplett neu ohne reledmac
\makeatletter
\renewenvironment{theindex}{%
  \section*{\indexname}%
  \setlength{\parindent}{0pt}%
  \setlength{\parskip}{0pt plus 0.3pt}%
  \let\item\@idxitem
}{%
  \clearpage
}
\makeatother

\IfFileExists{\jobname-pw.ind}{\input{\jobname-pw.ind}}{}

\end{document}

      