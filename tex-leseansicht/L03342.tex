%% latex-leseansicht-vorspann.tex
%% Vorspann für die Leseansicht.
%% Lädt die gemeinsame Datei latex-vorspann.tex mit nicht gesetztem Schalter.

\newif\ifkorrekturansicht
\korrekturansichtfalse

\input{../tex-inputs/latex-vorspann}


         
         \renewcommand{\erwaehntePersonen}{Personen: Emilie Mewes-Béha, Ottilie Salten, Paul Salten, Olga Schnitzler}
         \renewcommand{\erwaehnteOrte}{Orte: Wien}
         \renewcommand{\erwaehnteWerke}{Werke: Die Zeit, Studie}
               \section[ Felix Salten an Arthur Schnitzler, 11. 8. 1903]{ Felix Salten an Arthur Schnitzler, 11. 8. 1903}\nopagebreak\mylabel{v}\rehead{ }\begin{ledgroupsized}[t]{13cm}\normalsize\beginnumbering \toendnotes[C]{\smallbreak\pagebreak[2]} \Standort{CUL, Schnitzler, B 89, A 2.}
\physDesc{Brief, 1 Blatt, 1 Seite, 604 Zeichen
\newline{}Handschrift: blaue Tinte, lateinische Kurrent
\newline{}Ordnung: mit Bleistift von unbekannter Hand nummeriert: »167« }\toendnotes[C]{\smallbreak}\pstart
           \raggedleft{}{\pb}11. VIII. 03\pend
           \pstart
           Lieber, Ihre Sendung\pwindex{Mewes-Beha, Emilie @\textsc{Mewes-Béha, Emilie}, \emph{Schriftstellerin, Übersetzerin}!Studie1903-10-04@\strich\emph{Studie} {[}1903-10-04{]}|pwv} hab ich heute bei meiner Rückkehr
               vorgefunden und gleich gelesen. Es ist nichts besonderes, aber doch so,– dass man es in der Sonntags-Zeit\pwindex{Zeit1902-09-27 – 1919@\emph{Die Zeit} {[}1902-09-27 – 1919{]}|pw} einmal bringen kann, was ich denn auch mit Vergnügen thue, da
               es Ihnen offenbar sehr erwünscht ist. Hab’ ich Ihren Brief recht gelesen, so soll die
                  »\label{K_L03342-1v}\edtext{Studie\pwindex{Mewes-Beha, Emilie @\textsc{Mewes-Béha, Emilie}, \emph{Schriftstellerin, Übersetzerin}!Studie1903-10-04@\strich\emph{Studie} {[}1903-10-04{]}|pw}}{\lemma{\textnormal{\emph{Studie}}}\Cendnote{\textnormal{E. Mewes-Béha\pwindex{Mewes-Beha, Emilie @\textsc{Mewes-Béha, Emilie}, \emph{Schriftstellerin, Übersetzerin}|pwk}: \emph{Studie}\pwindex{Mewes-Beha, Emilie @\textsc{Mewes-Béha, Emilie}, \emph{Schriftstellerin, Übersetzerin}!Studie1903-10-04@\strich\emph{Studie} {[}1903-10-04{]}|pwk}. In: \emph{Die
                        Zeit}\pwindex{Zeit1902-09-27 – 1919@\emph{Die Zeit} {[}1902-09-27 – 1919{]}|pwk}, Jg. 2, Nr. 364, 4. 10. 1903, Die
                     Sonntags-Zeit, S. 2–3.}}}\label{K_L03342-1h}« erst in der zweiten Hälfte September publizirt werden. Ich habe das auf dem Mscpt
               vorgemerkt.\pend
           \pstart
           Heute{ }Nachmittag um ¾ 2 hat meine Frau\pwindex{Salten, Ottilie 07.03.1868 – 22.06.1942@\textsc{Salten, Ottilie} (07.03.1868 – 22.06.1942), \emph{Schauspielerin}|pwv} einen \label{K_L03342-2v}\edtext{Buben\pwindex{Salten, Paul 11.08.1903 – 08.05.1937@\textsc{Salten, Paul} (11.08.1903 – 08.05.1937), \emph{Filmcutter}|pwv}}{\lemma{\textnormal{\emph{Buben}}}\Cendnote{\textnormal{Paul Salten\pwindex{Salten, Paul 11.08.1903 – 08.05.1937@\textsc{Salten, Paul} (11.08.1903 – 08.05.1937), \emph{Filmcutter}|pwk}, siehe auch A. S.: \emph{Tagebuch}, 12. 8. 1903}}}\label{K_L03342-2h} bekommen und befindet sich sehr wol. Wir freuen uns sehr, wie Sie sich denken
               können. Wollen Sie es, bitte, an Olga\pwindex{Schnitzler, Olga 17.01.1882 – 13.01.1970@\textsc{Schnitzler, Olga} (17.01.1882 – 13.01.1970), \emph{Schauspielerin, Sängerin}|pw}
               mittheilen.\pend
           \pstart
           Herzlichst {\\[\baselineskip]}Ihr {\\[\baselineskip]}\spacefill\mbox{Salten}\pend
           \leftskip=0em{}
         
         \endnumbering\mylabel{h}\end{ledgroupsized}  \newcommand{\dateiname}{L03342}\newcommand{\titel}{Felix Salten an Arthur Schnitzler, 11. 8. 1903}\newcommand{\editorInnen}{Martin Anton Müller und Laura Untner}%% latex-leseansicht-abspann.tex
%% Abspann für die Leseansicht.
%% Der Schalter \ifkorrekturansicht ist bereits durch den Vorspann gesetzt.

%% latex-abspann.tex
%% Gemeinsamer Abspann für Korrekturansicht und Leseansicht.
%% Setzt den Schalter \ifkorrekturansicht voraus (gesetzt in den
%% einbindenden Dateien latex-korrekturansicht-abspann.tex bzw.
%% latex-leseansicht-abspann.tex).
%% ---------------------------------------------------------------

\normalsize

% Das esempio-Environment wird nur in der Leseansicht benötigt
\ifkorrekturansicht\else
\newenvironment{esempio}[3]%
{
    \vspace{1.5ex}
    \rlap{\underline{#1}}
    \par
    \setlength{\parindent}{0cm}
    \nopagebreak
    \leftskip=#2cm
    \rightskip=#3cm
}
{
    \par
}
\fi

\doendnotes{C}
\bigskip
\vfill

\clearpage

\footnotesize

\ifkorrekturansicht
  \lohead{\textsc{register}}
\fi

% theindex-Environment neu definieren ohne reledmac
\makeatletter
\renewenvironment{theindex}{%
  \ifkorrekturansicht
    \section*{\indexname}%
  \else
    \subsubsection*{Index der erwähnten Entitäten}%
  \fi
  \setlength{\parindent}{0pt}%
  \setlength{\parskip}{0pt plus 0.3pt}%
  \let\item\@idxitem
}{%
  \ifkorrekturansicht\clearpage\fi
}
\makeatother

\IfFileExists{\jobname-pw.ind}{\input{\jobname-pw.ind}}{}

% Quellenangabe nur in der Leseansicht
\ifkorrekturansicht\else
% Fallback-Definitionen, falls die .tex-Datei \titel etc. nicht gesetzt hat
\providecommand{\titel}{}
\providecommand{\editorInnen}{}
\providecommand{\dateiname}{\jobname}

\vspace{3cm}

\vfill

\footnotesize
\textsc{Quelle}: \titel. Herausgegeben von {\editorInnen}. In: \emph{Arthur Schnitzler: Briefwechsel mit Autorinnen und Autoren}.
 Digitale Edition, https://schnitzler-briefe.acdh.oeaw.ac.at/{\dateiname}.html (Stand \today)
\fi

\end{document}


      