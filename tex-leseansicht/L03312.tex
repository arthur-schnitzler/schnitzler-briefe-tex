%% latex-korrekturansicht-vorspann.tex
%% Vorspann für die Korrekturansicht.
%% Lädt die gemeinsame Datei latex-vorspann.tex mit gesetztem Schalter.

\newif\ifkorrekturansicht
\korrekturansichttrue

\input{../tex-inputs/latex-vorspann}


\section[ Felix Salten an Arthur Schnitzler, 20. 8. 1900]{L03312 Felix Salten an Arthur Schnitzler, 20. 8. 1900}
\nopagebreak\mylabel{L03312v}
\rehead{ }\normalsize\beginnumbering\briefempfaengerindex{Schnitzler, Arthur@\textsc{Schnitzler, Arthur}!zzzSalten, Felix@\emph{von Felix Salten}!1900-08-201@{20. 8. 1900}|(be}
\toendnotes[C]{\smallbreak\pagebreak[2]}\Standort{CUL, Schnitzler, B 89, A 2.}
\physDesc{Postkarte, 487 Zeichen
\newline{}Handschrift: schwarze Tinte, lateinische Kurrent
\newline{}Versand: Stempel: »\nobreak{}\oindex{Strobl@\textbf{Strobl}, \emph{A.ADM3}|pwk}Strobl, 20 8 00\nobreak{}«. Stempel: »\nobreak{}\oindex{Pontresina@\textbf{Pontresina}, \emph{P.PPL}|pwk}Pontresina, 23. VIII. 00., 4\nobreak{}«. Stempel: »\nobreak{}\oindex{Pontresina@\textbf{Pontresina}, \emph{P.PPL}|pwk}Pontresina, 23. VIII. 00., XI\nobreak{}«.  
\newline{}Schnitzler: mit Bleistift datiert: »20/8 90\textcolor{gray}{0}« 
\newline{}Ordnung: mit Bleistift von unbekannter Hand nummeriert: »136« }\toendnotes[C]{\smallbreak}\pstart{}{\pb}Herrn D\textsuperscript{r} Arthur Schnitzler\pend{}\pstart{}Pontresina\oindex{Pontresina@\textbf{Pontresina}, \emph{P.PPL}|pw}\pend{}\pstart{}Poste restante.\pend{}{\bigskip}\vspace{1em}
\pstart
           \noindent{}{\pb}Lieber,{ }heute erhielt ich Ihre Carte. Ich möchte von Ischl\oindex{Bad Ischl@\textbf{Bad Ischl}, \emph{P.PPL}|pw}{ }\uline{so} fortfahren, dass ich gleichzeitig mit Ihnen in
                  \label{K_L03312-1v}\edtext{Bozen\oindex{Bozen@\textbf{Bozen}, \emph{P.PPLA2}|pw}}{\lemma{\textnormal{\emph{Bozen}}}\Cendnote{\textnormal{Schnitzler und Salten\pwindex{Salten, Felix 06.09.1869 – 08.10.1945@\textsc{Salten, Felix} (06.09.1869 – 08.10.1945), \emph{Schriftsteller/Schriftstellerin, Journalist/Journalistin, Chefredakteur/Chefredakteurin}|pwk} trafen sich am 28. 8. 1900 in Meran\oindex{Meran@\textbf{Meran}, \emph{P.PPLA3}|pwk}. Am 30. 8. 1900 fuhren sie womöglich gemeinsam weiter nach Bozen\oindex{Bozen@\textbf{Bozen}, \emph{P.PPLA2}|pwk} und Ponte Adige\oindex{Ponte Adige@\textbf{Ponte Adige}, \emph{P.PPL}|pwk}.}}}\label{K_L03312-1} bin. Bitte, sagen Sie mir also, wann Sie dort sind, –
               ungefähr wenigstens. Ferner: Ich möchte am 1.
               spätestens am 3. September in Wien\oindex{Wien@\textbf{Wien}, \emph{A.ADM2}|pw} sein. Endlich: welche Tour machen wir? \label{K_L03312-2v}\edtext{Ob. Italien\oindex{Italien@\textbf{Italien}, \emph{A.PCLI}|pw}}{\lemma{\textnormal{\emph{Ob. Italien}}}\Cendnote{\textnormal{Oberitalien\oindex{Italien@\textbf{Italien}, \emph{A.PCLI}|pwk}}}}\label{K_L03312-2} u. Venedig\oindex{Venedig@\textbf{Venedig}, \emph{P.PPLA}|pw} ist vielleicht
                  \textcolor{gray}{n}och zu heiß u. hat jetzt zu viel Mosquitos. Übrigens ist es
               mir ziemlich egal, wohin wir fahren.\pend
           
\pstart
           Auf Wiedersehen, {\\[\baselineskip]}herzl. {\\[\baselineskip]}\spacefill\mbox{Salten.}\pend
           \leftskip=0em{}\selectlanguage{ngerman}\endnumbering\briefempfaengerindex{Schnitzler, Arthur@\textsc{Schnitzler, Arthur}!zzzSalten, Felix@\emph{von Felix Salten}!1900-08-201@{20. 8. 1900}|)be}\mylabel{L03312h}  \normalsize

\doendnotes{C}
\bigskip
\vfill

\clearpage

\footnotesize

\lohead{\textsc{register}}

% Definiere theindex-Environment komplett neu ohne reledmac
\makeatletter
\renewenvironment{theindex}{%
  \section*{\indexname}%
  \setlength{\parindent}{0pt}%
  \setlength{\parskip}{0pt plus 0.3pt}%
  \let\item\@idxitem
}{%
  \clearpage
}
\makeatother

\IfFileExists{\jobname-pw.ind}{\input{\jobname-pw.ind}}{}

\end{document}

      