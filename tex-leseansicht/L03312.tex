%% latex-leseansicht-vorspann.tex
%% Vorspann für die Leseansicht.
%% Lädt die gemeinsame Datei latex-vorspann.tex mit nicht gesetztem Schalter.

\newif\ifkorrekturansicht
\korrekturansichtfalse

\input{../tex-inputs/latex-vorspann}


         
         \renewcommand{\erwaehntePersonen}{Personen: Felix Salten}
         \renewcommand{\erwaehnteOrte}{Orte: Bad Ischl, Bozen, Italien, Meran, Ponte Adige, Pontresina, Strobl, Venedig, Wien}
         \renewcommand{\erwaehnteWerke}{}
               \section[ Felix Salten an Arthur Schnitzler, 20. 8. 1900]{ Felix Salten an Arthur Schnitzler, 20. 8. 1900}\nopagebreak\mylabel{v}\rehead{ }\begin{ledgroupsized}[t]{13cm}\normalsize\beginnumbering \toendnotes[C]{\smallbreak\pagebreak[2]} \Standort{CUL, Schnitzler, B 89, A 2.}
\physDesc{Postkarte, 485 Zeichen
\newline{}Handschrift: schwarze Tinte, lateinische Kurrent
\newline{}Versand: Stempel: »\nobreak{}\oindex{Strobl@\textbf{Strobl}|pwk}Strobl, 20 8 00\nobreak{}«. Stempel: »\nobreak{}\oindex{Pontresina@\textbf{Pontresina}|pwk}Pontresina, 23. VIII. 00., 4\nobreak{}«. Stempel: »\nobreak{}\oindex{Pontresina@\textbf{Pontresina}|pwk}Pontresina, 23. VIII. 00., XI\nobreak{}«.  
\newline{}Schnitzler: mit Bleistift datiert: »20/8 90\textcolor{gray}{0}« 
\newline{}Ordnung: mit Bleistift von unbekannter Hand nummeriert: »136« }\toendnotes[C]{\smallbreak}\pstart{}{\pb}Herrn D\textsuperscript{r} Arthur Schnitzler\pend{}\pstart{}Pontresina\oindex{Pontresina@\textbf{Pontresina}|pw}\pend{}\pstart{}Poste restante.\pend{}{\bigskip}\pstart
           \noindent{}{\pb}Lieber,{ }heute erhielt ich Ihre Carte. Ich möchte von Ischl\oindex{Bad Ischl@\textbf{Bad Ischl}|pw}{ }\uline{so} fortfahren, dass ich gleichzeitig mit Ihnen in
                  \label{K_L03312-1v}\edtext{Bozen\oindex{Bozen@\textbf{Bozen}|pw}}{\lemma{\textnormal{\emph{Bozen}}}\Cendnote{\textnormal{Schnitzler\pwindex{Schnitzler, Arthur 15.05.1862 – 21.10.1931@\textsc{Schnitzler, Arthur} (15.05.1862 – 21.10.1931), \emph{Schriftsteller, Mediziner}|pwk} und Salten\pwindex{Salten, Felix 06.09.1869 – 08.10.1945@\textsc{Salten, Felix} (06.09.1869 – 08.10.1945), \emph{Schriftsteller, Journalist}|pwk} trafen sich am 28. 8. 1900 in Meran\oindex{Meran@\textbf{Meran}|pwk}. Am 30. 8. 1900 fuhren sie womöglich gemeinsam weiter nach Bozen\oindex{Bozen@\textbf{Bozen}|pwk} und Ponte Adige\oindex{Ponte Adige@\textbf{Ponte Adige}|pwk}.}}}\label{K_L03312-1h} bin. Bitte, sagen Sie mir also, wann Sie dort sind, –
               ungefähr wenigstens. Ferner: Ich möchte am 1.
               spätestens am 3. September in Wien\oindex{Wien@\textbf{Wien}|pw} sein. Endlich: welche Tour machen wir? \label{K_L03312-2v}\edtext{Ob. Italien\oindex{Italien@\textbf{Italien}|pw}}{\lemma{\textnormal{\emph{Ob. Italien}}}\Cendnote{\textnormal{Oberitalien\oindex{Italien@\textbf{Italien}|pwk}}}}\label{K_L03312-2h} u. Venedig\oindex{Venedig@\textbf{Venedig}|pw} ist vielleicht
                  \textcolor{gray}{n}och zu heiß u. hat jetzt zu viel Mosquitos. Übrigens ist es
               mir ziemlich egal, wohin wir fahren.\pend
           \pstart
           Auf Wiedersehen, {\\[\baselineskip]}herzl. {\\[\baselineskip]}\spacefill\mbox{Salten.}\pend
           \leftskip=0em{}
         
         \endnumbering\mylabel{h}\end{ledgroupsized}  \newcommand{\dateiname}{L03312}\newcommand{\titel}{Felix Salten an Arthur Schnitzler, 20. 8. 1900}\newcommand{\editorInnen}{Martin Anton Müller und Laura Untner}%% latex-leseansicht-abspann.tex
%% Abspann für die Leseansicht.
%% Der Schalter \ifkorrekturansicht ist bereits durch den Vorspann gesetzt.

%% latex-abspann.tex
%% Gemeinsamer Abspann für Korrekturansicht und Leseansicht.
%% Setzt den Schalter \ifkorrekturansicht voraus (gesetzt in den
%% einbindenden Dateien latex-korrekturansicht-abspann.tex bzw.
%% latex-leseansicht-abspann.tex).
%% ---------------------------------------------------------------

\normalsize

% Das esempio-Environment wird nur in der Leseansicht benötigt
\ifkorrekturansicht\else
\newenvironment{esempio}[3]%
{
    \vspace{1.5ex}
    \rlap{\underline{#1}}
    \par
    \setlength{\parindent}{0cm}
    \nopagebreak
    \leftskip=#2cm
    \rightskip=#3cm
}
{
    \par
}
\fi

\doendnotes{C}
\bigskip
\vfill

\clearpage

\footnotesize

\ifkorrekturansicht
  \lohead{\textsc{register}}
\fi

% theindex-Environment neu definieren ohne reledmac
\makeatletter
\renewenvironment{theindex}{%
  \ifkorrekturansicht
    \section*{\indexname}%
  \else
    \subsubsection*{Index der erwähnten Entitäten}%
  \fi
  \setlength{\parindent}{0pt}%
  \setlength{\parskip}{0pt plus 0.3pt}%
  \let\item\@idxitem
}{%
  \ifkorrekturansicht\clearpage\fi
}
\makeatother

\IfFileExists{\jobname-pw.ind}{\input{\jobname-pw.ind}}{}

% Quellenangabe nur in der Leseansicht
\ifkorrekturansicht\else
% Fallback-Definitionen, falls die .tex-Datei \titel etc. nicht gesetzt hat
\providecommand{\titel}{}
\providecommand{\editorInnen}{}
\providecommand{\dateiname}{\jobname}

\vspace{3cm}

\vfill

\footnotesize
\textsc{Quelle}: \titel. Herausgegeben von {\editorInnen}. In: \emph{Arthur Schnitzler: Briefwechsel mit Autorinnen und Autoren}.
 Digitale Edition, https://schnitzler-briefe.acdh.oeaw.ac.at/{\dateiname}.html (Stand \today)
\fi

\end{document}


      