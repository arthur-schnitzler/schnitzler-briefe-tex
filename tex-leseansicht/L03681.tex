%% latex-leseansicht-vorspann.tex
%% Vorspann für die Leseansicht.
%% Lädt die gemeinsame Datei latex-vorspann.tex mit nicht gesetztem Schalter.

\newif\ifkorrekturansicht
\korrekturansichtfalse

\input{../tex-inputs/latex-vorspann}


\section[Stefan Zweig an Arthur Schnitzler, 12. 10. 1917]{L03681 Stefan Zweig an Arthur Schnitzler, 12. 10. 1917}
\nopagebreak\mylabel{L03681v}
\rehead{ }\normalsize\beginnumbering\briefempfaengerindex{Schnitzler, Arthur@\textsc{Schnitzler, Arthur}!zzzZweig, Stefan@\emph{von Stefan Zweig}!1917-10-122@{12. 10. 1917}|(be}
\toendnotes[C]{\smallbreak\pagebreak[2]}
\correspDesc{Versand  durch Stefan Zweig am 12. 10. 1917 in Wien
\newline{}Erhalt  durch Arthur Schnitzler im Zeitraum [13. 10. 1917 – 17. 10. 1917?] in Wien}\toendnotes[C]{\smallbreak}
\Standort{CUL, Schnitzler, B 118.}
\physDesc{Briefkarte, 1183 Zeichen
\newline{}Handschrift: blaue Tinte, lateinische Kurrent
\newline{}Schnitzler: 1) mit Bleistift »\textsc{Zweig}«  2) mit rotem Buntstift eine Unterstreichung}
\buchAbdrucke{\weitereDrucke{Stefan Zweig: \emph{Briefwechsel mit Hermann Bahr, Sigmund Freud, Rainer Maria
                        Rilke und Arthur Schnitzler}. Herausgegeben von Jeffrey B. Berlin, Hans-Ulrich Lindken und Donald A. Prater. Frankfurt am Main: \emph{S. Fischer} 1987, S. 409.} }\toendnotes[C]{\smallbreak}
\pstart
           \raggedleft{}{\pb}12. October 1917\pend
           
\pstart
           \textcolor{gray}{\textbf{SZ}}\hfill \textcolor{gray}{\textbf{VIII. KOCHGASSE 8\oindex{Wien@\textbf{Wien}!VIII., Josefstadt@\textbf{VIII., Josefstadt}!Kochgasse 8@\textbf{Kochgasse 8}, \emph{Wohngebäude}|pw}.}}\pend
           \vspace{0.5em}
\pstart
           Lieber verehrter Herr Doktor, ich danke Ihnen sehr für Ihre guten
               Worte; dass dieses Werk\pwindex{Zweig, Stefan 28.\,11.\,1881 Wien – 23.\,2.\,1942 Petrópolis@\textsc{Zweig, Stefan} (28.\,11.\,1881 Wien – 23.\,2.\,1942 Petrópolis), \emph{Schriftsteller}!Jeremias. Eine dramatische Dichtung in neun Bildern@\strich\emph{Jeremias. Eine dramatische Dichtung in neun Bildern}|pwv}, eigentlich aus
               Zorn und Qual geboren, mir nun Liebe gerade der Besten gewinnt, bezeugt mir die
               Notwendigkeit dieser Erbitterung, die ich lange selbst wie eine sinnlose Verstörung
               empfand. Vielleicht hat die Verwandlung die Leidenschaft gelöst und damit auch das
               Leiden erlöst: ich fühle mich jetzt freier, so sehr ich äusserlich noch gebunden
               bin.\pend
           
\pstart
           {\pb}Es wäre nur ein menschliches Bedürfis,
               Sie und Ihre verehrte Frau Gemahlin\pwindex{Schnitzler, Olga 17.\,1.\,1882 Wien – 13.\,1.\,1970 Lugano@\textsc{Schnitzler, Olga} (17.\,1.\,1882 Wien – 13.\,1.\,1970 Lugano), \emph{Schauspielerin, Sängerin}|pwv} wieder einmal sehen zu dürfen. Aber ich lebe ganz im Ungewissen.
               Vor 6 Wochen hat das Auswärtige Amt\orgindex{Außenministerium@Außenministerium|pw} für mich um einen
               \label{K_L03681-1v}\edtext{Urlaub nach der Schweiz\oindex{Schweiz@\textbf{Schweiz}|pw}}{\lemma{\textnormal{\emph{Urlaub nach der Schweiz}}}\Cendnote{\textnormal{Zweig\pwindex{Zweig, Stefan 28.\,11.\,1881 Wien – 23.\,2.\,1942 Petrópolis@\textsc{Zweig, Stefan} (28.\,11.\,1881 Wien – 23.\,2.\,1942 Petrópolis), \emph{Schriftsteller}|pwk} bekam 
                  den gewünschten Urlaub und war ab 5. 11. 1917 von der Arbeit im \emph{Kriegsarchiv}\orgindex{Österreichisches Staatsarchiv@Österreichisches Staatsarchiv|pwk} freigestellt.
                  Eine Woche später reiste er in die Schweiz\oindex{Schweiz@\textbf{Schweiz}|pwk}, wo er – nunmehr als ständiger Mitarbeiter der
                  \emph{Neuen Freien Presse}\orgindex{Neue Freie Presse@Neue Freie Presse|pwk} – bis März 1919 blieb.}}}\label{K_L03681-1} gebeten, wo ich einige
               Vorträge halten soll. Das Kriegsministerium\orgindex{k. u. k. Kriegsministerium@k. u. k. Kriegsministerium|pw}, das jeden
               \label{K_L03681-2v}\edtext{Filmschapsel}{\lemma{\textnormal{\emph{Filmschapsel}}}\Cendnote{\textnormal{Schapsel: (intellektuelles) Leichtgewicht, Einfallspinsel}}}\label{K_L03681-2} und Operettengaukler willig entliess, hat in sechs Wochen nicht geruht,
               darauf Antwort zu geben, der Vortrag wartet auf mich und ich weiss nicht, ob ich darf
               oder nicht. Freilich: ich rühre nicht einen Finger, weil es mir zu kläglich scheint,
               nach drei Jahren Dienst um so einen Atemzug Freiheit noch bittlich zu werden: aber
               ich hänge jetzt in der Luft und weiss nicht von heute auf morgen.\pend
           \pstart Herzlich ergeben Ihr getreuer \spacefill\mbox{Stefan Zweig}\pend{}\selectlanguage{ngerman}\endnumbering\briefempfaengerindex{Schnitzler, Arthur@\textsc{Schnitzler, Arthur}!zzzZweig, Stefan@\emph{von Stefan Zweig}!1917-10-122@{12. 10. 1917}|)be}\mylabel{L03681h}  \newcommand{\dateiname}{L03681}\newcommand{\titel}{Stefan Zweig an Arthur Schnitzler, 12. 10. 1917}\newcommand{\editorInnen}{Selma Jahnke und Martin Anton Müller}%% latex-leseansicht-abspann.tex
%% Abspann für die Leseansicht.
%% Der Schalter \ifkorrekturansicht ist bereits durch den Vorspann gesetzt.

%% latex-abspann.tex
%% Gemeinsamer Abspann für Korrekturansicht und Leseansicht.
%% Setzt den Schalter \ifkorrekturansicht voraus (gesetzt in den
%% einbindenden Dateien latex-korrekturansicht-abspann.tex bzw.
%% latex-leseansicht-abspann.tex).
%% ---------------------------------------------------------------

\normalsize

% Das esempio-Environment wird nur in der Leseansicht benötigt
\ifkorrekturansicht\else
\newenvironment{esempio}[3]%
{
    \vspace{1.5ex}
    \rlap{\underline{#1}}
    \par
    \setlength{\parindent}{0cm}
    \nopagebreak
    \leftskip=#2cm
    \rightskip=#3cm
}
{
    \par
}
\fi

\doendnotes{C}
\bigskip
\vfill

\clearpage

\footnotesize

\ifkorrekturansicht
  \lohead{\textsc{register}}
\fi

% theindex-Environment neu definieren ohne reledmac
\makeatletter
\renewenvironment{theindex}{%
  \ifkorrekturansicht
    \section*{\indexname}%
  \else
    \subsubsection*{Index der erwähnten Entitäten}%
  \fi
  \setlength{\parindent}{0pt}%
  \setlength{\parskip}{0pt plus 0.3pt}%
  \let\item\@idxitem
}{%
  \ifkorrekturansicht\clearpage\fi
}
\makeatother

\IfFileExists{\jobname-pw.ind}{\input{\jobname-pw.ind}}{}

% Quellenangabe nur in der Leseansicht
\ifkorrekturansicht\else
% Fallback-Definitionen, falls die .tex-Datei \titel etc. nicht gesetzt hat
\providecommand{\titel}{}
\providecommand{\editorInnen}{}
\providecommand{\dateiname}{\jobname}

\vspace{3cm}

\vfill

\footnotesize
\textsc{Quelle}: \titel. Herausgegeben von {\editorInnen}. In: \emph{Arthur Schnitzler: Briefwechsel mit Autorinnen und Autoren}.
 Digitale Edition, https://schnitzler-briefe.acdh.oeaw.ac.at/{\dateiname}.html (Stand \today)
\fi

\end{document}


