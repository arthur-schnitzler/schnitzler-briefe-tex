%% latex-korrekturansicht-vorspann.tex
%% Vorspann für die Korrekturansicht.
%% Lädt die gemeinsame Datei latex-vorspann.tex mit gesetztem Schalter.

\newif\ifkorrekturansicht
\korrekturansichttrue

\input{../tex-inputs/latex-vorspann}


\section[Hugo von Hofmannsthal an Arthur Schnitzler, 15. 3. {[}1900{]}]{L01021 Hugo von Hofmannsthal an Arthur Schnitzler, 15. 3. {[}1900{]}}
\nopagebreak\mylabel{L01021v}
\rehead{ }\normalsize\beginnumbering\briefempfaengerindex{Schnitzler, Arthur@\textsc{Schnitzler, Arthur}!zzzHofmannsthal, Hugo von@\emph{von Hugo von Hofmannsthal}!1900-03-151@{15. 3. {[}1900{]}}|(be}
\toendnotes[C]{\smallbreak\pagebreak[2]}\Standort{CUL, Schnitzler, B 43.}
\physDesc{Brief, 2 Blätter, 8 Seiten, 2853 Zeichen
\newline{}Handschrift: schwarze Tinte, deutsche Kurrent
\newline{}Schnitzler: mit Bleistift die Jahreszahl ergänzt: »1900« 
\newline{}Ordnung: mit Bleistift von unbekannter Hand nummeriert:
                                    »160« }
\buchAbdrucke{\weitereDrucke{Hugo von Hofmannsthal, Arthur Schnitzler: \emph{Briefwechsel}. Frankfurt am Main: \emph{S. Fischer} 1964, S. 134.} }\toendnotes[C]{\smallbreak}
\pstart
           {\pb}15 März.\hfill \textsc{192 B\textsuperscript{d}
                           Haussmann\oindex{Boulevard Haussmann@\textbf{Boulevard Haussmann}, \emph{Straße (K.STR)}|pw}}\pend
           
\pstart
           \raggedleft{}\textsc{Paris}\pend
           
\pstart{}mein lieber Arthur\pend\vspace{0.5em}
\pstart
           es geht einem hier merkwürdig: ohne einzelnen Menſchen übermäßig nahe zu treten, iſt
               man doch von einem ſolchen Gewirr von Menſchen und Beſtrebungen umgeben, daſs einem
               zuhauſe und Deutſchland\oindex{Deutschland@\textbf{Deutschland}, \emph{A.PCLI}|pw} ungeheuer weit weg
               vorkommt. Für mich hat eine ſolche Suggeſtion etwas ſehr gutes: ſchon lang hab ich
               mich nicht ſo frei gefühlt, mich {\pb}nicht ſo zuſammenfaſſen können. Es fällt mir manchmal mehr ein als ich aufſchreiben
               kann: kleinere und größere Stücke, Erzählungen und anderes Phantaſtiſches. Ich hoffe,
               daſs ich wol halbwegs Abgeſchloſſenes fertig bringe.\pend
           
\pstart
           Die Stadt\oindex{Paris@\textbf{Paris}, \emph{P.PPLC}|pwv} und das \textsc{bois}\oindex{Bois de Boulogne@\textbf{Bois de Boulogne}, \emph{L.PRK}|pw}{ }ſind noch nicht ſehr hübſch; man freut ſich hier
               doppelt auf das Frühjahr, das Licht und die Blätter.\pend
           
\pstart
           \textsc{Anatole France}\pwindex{France, Anatole 16.04.1844 – 12.10.1924@\textsc{France, Anatole} (16.04.1844 – 12.10.1924), \emph{Schriftsteller/Schriftstellerin}|pw} zu ſehen, iſt recht intereſſant; es kommen {\pb}viel junge Leute zu ihm, das
               Geſpräch iſt faſt ausſchließlich politiſch, die Färbung ſocialiſtiſch. (Die
               »Geſellſchaft« iſt faſt vollſtändig nationaliſtiſch, zum Theil in einer widerwärtigen
               bornierten Weise).\pend
           
\pstart
           Eine große Freude iſt es, \textsc{Rodin}\pwindex{Rodin, Auguste 12.11.1840 – 17.11.1917@\textsc{Rodin, Auguste} (12.11.1840 – 17.11.1917), \emph{Bildhauer/Bildhauerin}|pw} in ſeinem Atelier zu beſuchen. Da iſt man in einer ganz andern, ſehr großen
               Welt. Er ſelbſt iſt von einer merkwürdigen Güte und Freundlichkeit. Ich {\pb}werde \strikeout{ihn} nächſtens auch nach \textsc{Meudon}\oindex{Meudon@\textbf{Meudon}, \emph{P.PPL}|pw} zu ihm hinausfahren.\pend
           
\pstart
           \uline{Wie heißt der kleine Ort\oindex{Villennes-sur-Seine@\textbf{Villennes-sur-Seine}, \emph{P.PPL}|pwv} am Waſſer, wo Sie einen ſo ſchönen und traurigen
                     \label{K_L01021-1v}\edtext{Abend}{\lemma{\textnormal{\emph{Abend}}}\Cendnote{\textnormal{am 21. 5. 1897, zusammen mit Marie
                        Reinhard\pwindex{Reinhard, Marie 1871-03-13 – 1899-03-18@\textsc{Reinhard, Marie} (1871-03-13 – 1899-03-18), \emph{Gesangspädagoge/Gesangspädagogin}|pwk}}}}\label{K_L01021-1} verbracht haben?} Ich denke ſehr oft daran.\pend
           
\pstart
           Ich beſchäftige mich mit Ihnen in Gedanken in einer ſehr lebhaften ſonderbaren Weiſe.
               Mir iſt unter andern ein ganz incommenſurables kleines groteſkes Stück\pwindex{Paracelsus und Dr. Schnitzler@\emph{Paracelsus und Dr. Schnitzler}|pwv} eingefallen, in welchem Sie und \textsc{Paracelsus}\pwindex{Paracelsus, Theofrastus Bombastus 1493/1494 – 24.9.1541@\textsc{Paracelsus, Theofrastus Bombastus} (1493/1494 – 24.9.1541), \emph{Mediziner/Medizinerin, Philosoph/Philosophin, Chemiker/Chemikerin}|pw} (der wirkliche, von dem ich ganz {\pb}außerordentliche Bücher hier,
               überſetzt, auszugsweise, mithabe) die Hauptfiguren ſind. Es iſt ein Stoff der mich
               merkwürdig aufregt. Wenn ich es fertig bringe, müſsten wir es beim Richard\pwindex{Beer-Hofmann, Richard 1866-07-11 – 1945-09-26@\textsc{Beer-Hofmann, Richard} (1866-07-11 – 1945-09-26), \emph{Schriftsteller/Schriftstellerin}|pw}{ }ſpielen. Ich ſpüre dabei ſehr ſtark, daſs mir an
               dem Verkehr mit Ihnen \uline{gar}{ }\uline{nichts} unfruchtbar iſt; auch nicht die kleinſte
               Sache, mit der ſich nicht in der {\pb}Erinnerung etwas anfangen ließe.\pend
           
\pstart
           Was thuen Sie? von \label{K_L01021-2v}\edtext{dieſen Tagen}{\lemma{\textnormal{\emph{dieſen Tagen}}}\Cendnote{\textnormal{Am 18. 3. 1900 jährte sich der
                  Tod Marie Reinhards\pwindex{Reinhard, Marie 1871-03-13 – 1899-03-18@\textsc{Reinhard, Marie} (1871-03-13 – 1899-03-18), \emph{Gesangspädagoge/Gesangspädagogin}|pwk}.}}}\label{K_L01021-2} jetzt gerade
               kann ich es mir ja denken, beinahe fühlen, aber nachher? woran arbeiten Sie, lieber
               Arthur?\pend
           
\pstart
           Iſt Richard\pwindex{Beer-Hofmann, Richard 1866-07-11 – 1945-09-26@\textsc{Beer-Hofmann, Richard} (1866-07-11 – 1945-09-26), \emph{Schriftsteller/Schriftstellerin}|pw} in Wien\oindex{Wien@\textbf{Wien}, \emph{A.ADM2}|pw}? Ich erwartete auf mehrere Karten lange eine Antwort, erhielt endlich
               eine ſehr {\pb}flüchtige, dürftige aus
                  Florenz\oindex{Florenz@\textbf{Florenz}, \emph{P.PPLA}|pw}.\pend
           
\pstart
           Mein Papa\pwindex{Hofmannsthal, Hugo August von 21.12.1841 – 08.12.1915@\textsc{Hofmannsthal, Hugo August von} (21.12.1841 – 08.12.1915), \emph{Bankdirektor/Bankdirektorin}|pwv} wird Ihnen in den
               nächſten Tagen ein typiertes Exemplar des kleinen Vorſpiels\pwindex{Vorspiel zur Antigone des Sophokles@\emph{Vorspiel zur Antigone des Sophokles}|pwv}{ }ſchicken, das ich für eine Berlin\oindex{Berlin@\textbf{Berlin}, \emph{P.PPLC}|pw}er Antigone\pwindex{Antigone@\emph{Antigone}|pw}-vorstellung (26\textsuperscript{ten} März)
               geſchrieben habe. Bitte \label{K_L01021-3v}\edtext{ſchicken Sie
               es mit meinen herzlichen Grüßen an Goldmann\pwindex{Goldmann, Paul 31.01.1865 – 25.09.1935@\textsc{Goldmann, Paul} (31.01.1865 – 25.09.1935), \emph{Schriftsteller/Schriftstellerin, Journalist/Journalistin}|pw}}{\lemma{\textnormal{\emph{ſchicken … Goldmann}}}\Cendnote{\textnormal{Vgl. Paul Goldmann an Arthur Schnitzler, 24. 3. [1900].
               }}}\label{K_L01021-3}. Es wäre mir natürlich angenehm {\pb}wenn er etwa in die Vorſtellung
               gehen und darüber ſchreiben würde, aber natürlich abſolut nur, wenn er Luſt hat.\pend
           
\pstart
           Ich hoffe bald einen Brief von Ihnen, ſehe \textsc{Maeterlinck}\pwindex{Maeterlinck, Maurice 29.08.1862 – 06.05.1949@\textsc{Maeterlinck, Maurice} (29.08.1862 – 06.05.1949), \emph{Schriftsteller/Schriftstellerin}|pw}{ }ſehr viel, einen überaus erfreulichen Menſchen,
               auch andere Leute, Frauen, Cocotten, Schauſpielerinnen, ſehr viele ſchlechte
               Menſchen, arbeite ſehr viel, finde endlich, daſs der Tag 24 Stunden hat und bin nie
               ſchläfrig.\pend
           \pstart Von Herzen Ihr \spacefill\mbox{Hugo.}\pend{}\selectlanguage{ngerman}\endnumbering\briefempfaengerindex{Schnitzler, Arthur@\textsc{Schnitzler, Arthur}!zzzHofmannsthal, Hugo von@\emph{von Hugo von Hofmannsthal}!1900-03-151@{15. 3. {[}1900{]}}|)be}\mylabel{L01021h}  \normalsize

\doendnotes{C}
\bigskip
\vfill

\clearpage

\footnotesize

\lohead{\textsc{register}}

% Definiere theindex-Environment komplett neu ohne reledmac
\makeatletter
\renewenvironment{theindex}{%
  \section*{\indexname}%
  \setlength{\parindent}{0pt}%
  \setlength{\parskip}{0pt plus 0.3pt}%
  \let\item\@idxitem
}{%
  \clearpage
}
\makeatother

\IfFileExists{\jobname-pw.ind}{\input{\jobname-pw.ind}}{}

\end{document}

      