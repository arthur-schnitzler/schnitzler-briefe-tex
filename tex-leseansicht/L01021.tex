%% latex-leseansicht-vorspann.tex
%% Vorspann für die Leseansicht.
%% Lädt die gemeinsame Datei latex-vorspann.tex mit nicht gesetztem Schalter.

\newif\ifkorrekturansicht
\korrekturansichtfalse

\input{../tex-inputs/latex-vorspann}


\section[Hugo von Hofmannsthal an Arthur Schnitzler, 15. 3. {[}1900{]}]{L01021 Hugo von Hofmannsthal an Arthur Schnitzler, 15. 3. [1900]}
\nopagebreak\mylabel{L01021v}
\rehead{ }\normalsize\beginnumbering\briefempfaengerindex{Schnitzler, Arthur@\textsc{Schnitzler, Arthur}!zzzHofmannsthal, Hugo von@\emph{von Hugo von Hofmannsthal}!1900-03-151@{15. 3. [1900]}|(be}
\toendnotes[C]{\smallbreak\pagebreak[2]}
\correspDesc{Versand  durch Hugo von Hofmannsthal am 15. 3. [1900] in Paris
\newline{}Erhalt  durch Arthur Schnitzler am 18. 3. 1900 in Wien}\toendnotes[C]{\smallbreak}
\Standort{CUL, Schnitzler, B 43.}
\physDesc{Brief, 2 Blätter, 8 Seiten, 2853 Zeichen
\newline{}Handschrift: schwarze Tinte, deutsche Kurrent
\newline{}Schnitzler: mit Bleistift die Jahreszahl ergänzt: »1900« 
\newline{}Ordnung: mit Bleistift von unbekannter Hand nummeriert:
                                    »160« }
\buchAbdrucke{\weitereDrucke{Hugo von Hofmannsthal, Arthur Schnitzler: \emph{Briefwechsel}. Herausgegeben von Therese Nickl und Heinrich Schnitzler. Frankfurt am Main: \emph{S. Fischer} 1964, S. 134.} }\toendnotes[C]{\smallbreak}
\pstart
           {\pb}15 März.\hfill \textsc{192 B\textsuperscript{d}
                           Haussmann\oindex{Boulevard Haussmann@\textbf{Boulevard Haussmann}, \emph{Straße}|pw}}\pend
           
\pstart
           \raggedleft{}\textsc{Paris}\pend
           
\pstart{}mein lieber Arthur\pend\vspace{0.5em}
\pstart
           es geht einem hier merkwürdig: ohne einzelnen Menſchen übermäßig nahe zu treten, iſt
               man doch von einem{ }ſolchen Gewirr von Menſchen und Beſtrebungen umgeben, daſs einem
               zuhauſe und Deutſchland\oindex{Deutschland@\textbf{Deutschland}|pw} ungeheuer weit weg
               vorkommt. Für mich hat eine{ }ſolche Suggeſtion etwas{ }ſehr gutes:{ }ſchon lang hab ich
               mich nicht{ }ſo frei gefühlt, mich {\pb}nicht{ }ſo zuſammenfaſſen können. Es fällt mir manchmal mehr ein als ich aufſchreiben
               kann: kleinere und größere Stücke, Erzählungen und anderes Phantaſtiſches. Ich hoffe,
               daſs ich wol halbwegs Abgeſchloſſenes fertig bringe.\pend
           
\pstart
           Die Stadt\oindex{Paris@\textbf{Paris}, \emph{Hauptstadt}|pwv} und das \textsc{bois}\oindex{Bois de Boulogne@\textbf{Bois de Boulogne}, \emph{Park}|pw}{ }ſind noch nicht{ }ſehr hübſch; man freut{ }ſich hier
               doppelt auf das Frühjahr, das Licht und die Blätter.\pend
           
\pstart
           \textsc{Anatole France}\pwindex{France, Anatole 16.\,4.\,1844 Paris – 12.\,10.\,1924 Saint-Cyr-sur-Loire@\textsc{France, Anatole} (16.\,4.\,1844 Paris – 12.\,10.\,1924 Saint-Cyr-sur-Loire), \emph{Schriftsteller}|pw} zu{ }ſehen, iſt recht intereſſant; es kommen {\pb}viel junge Leute zu ihm, das
               Geſpräch iſt faſt ausſchließlich politiſch, die Färbung{ }ſocialiſtiſch. (Die
               »Geſellſchaft« iſt faſt vollſtändig nationaliſtiſch, zum Theil in einer widerwärtigen
               bornierten Weise).\pend
           
\pstart
           Eine große Freude iſt es, \textsc{Rodin}\pwindex{Rodin, Auguste 12.\,11.\,1840 Paris – 17.\,11.\,1917 Meudon@\textsc{Rodin, Auguste} (12.\,11.\,1840 Paris – 17.\,11.\,1917 Meudon), \emph{Bildhauer}|pw} in{ }ſeinem Atelier zu beſuchen. Da iſt man in einer ganz andern,{ }ſehr großen
               Welt. Er{ }ſelbſt iſt von einer merkwürdigen Güte und Freundlichkeit. Ich {\pb}werde \strikeout{ihn} nächſtens auch nach \textsc{Meudon}\oindex{Meudon@\textbf{Meudon}|pw} zu ihm hinausfahren.\pend
           
\pstart
           \uline{Wie heißt der kleine Ort\oindex{Villennes-sur-Seine@\textbf{Villennes-sur-Seine}|pwv} am Waſſer, wo Sie einen{ }ſo{ }ſchönen und traurigen
                     \label{K_L01021-1v}\edtext{Abend}{\lemma{\textnormal{\emph{Abend}}}\Cendnote{\textnormal{am 21. 5. 1897, zusammen mit Marie
                        Reinhard\pwindex{Reinhard, Marie 13.\,3.\,1871 Wien – 18.\,3.\,1899 ebd.@\textsc{Reinhard, Marie} (13.\,3.\,1871 Wien – 18.\,3.\,1899 ebd.), \emph{Gesangspädagogin}|pwk}}}}\label{K_L01021-1} verbracht haben?} Ich denke{ }ſehr oft daran.\pend
           
\pstart
           Ich beſchäftige mich mit Ihnen in Gedanken in einer{ }ſehr lebhaften{ }ſonderbaren Weiſe.
               Mir iſt unter andern ein ganz incommenſurables kleines groteſkes Stück\pwindex{Hofmannsthal, Hugo von 1.\,2.\,1874 Wien – 15.\,7.\,1929 Rodaun@\textsc{Hofmannsthal, Hugo von} (1.\,2.\,1874 Wien – 15.\,7.\,1929 Rodaun), \emph{Schriftsteller}!Paracelsus und Dr. Schnitzler@\strich\emph{Paracelsus und Dr. Schnitzler}|pwv} eingefallen, in welchem Sie und \textsc{Paracelsus}\pwindex{Paracelsus, Theofrastus Bombastus 1493/1494 Egg – 24.\,9.\,1541 Salzburg@\textsc{Paracelsus, Theofrastus Bombastus} (1493/1494 Egg – 24.\,9.\,1541 Salzburg), \emph{Mediziner, Philosoph, Chemiker}|pw} (der wirkliche, von dem ich ganz {\pb}außerordentliche Bücher hier,
               überſetzt, auszugsweise, mithabe) die Hauptfiguren{ }ſind. Es iſt ein Stoff der mich
               merkwürdig aufregt. Wenn ich es fertig bringe, müſsten wir es beim Richard\pwindex{Beer-Hofmann, Richard 11.\,7.\,1866 Wien – 26.\,9.\,1945 New York City@\textsc{Beer-Hofmann, Richard} (11.\,7.\,1866 Wien – 26.\,9.\,1945 New York City), \emph{Schriftsteller}|pw}{ }ſpielen. Ich{ }ſpüre dabei{ }ſehr{ }ſtark, daſs mir an
               dem Verkehr mit Ihnen \uline{gar}{ }\uline{nichts} unfruchtbar iſt; auch nicht die kleinſte
               Sache, mit der{ }ſich nicht in der {\pb}Erinnerung etwas anfangen ließe.\pend
           
\pstart
           Was thuen Sie? von \label{K_L01021-2v}\edtext{dieſen Tagen}{\lemma{\textnormal{\emph{diesen Tagen}}}\Cendnote{\textnormal{Am 18. 3. 1900 jährte sich der
                  Tod Marie Reinhards\pwindex{Reinhard, Marie 13.\,3.\,1871 Wien – 18.\,3.\,1899 ebd.@\textsc{Reinhard, Marie} (13.\,3.\,1871 Wien – 18.\,3.\,1899 ebd.), \emph{Gesangspädagogin}|pwk}.}}}\label{K_L01021-2} jetzt gerade
               kann ich es mir ja denken, beinahe fühlen, aber nachher? woran arbeiten Sie, lieber
               Arthur?\pend
           
\pstart
           Iſt Richard\pwindex{Beer-Hofmann, Richard 11.\,7.\,1866 Wien – 26.\,9.\,1945 New York City@\textsc{Beer-Hofmann, Richard} (11.\,7.\,1866 Wien – 26.\,9.\,1945 New York City), \emph{Schriftsteller}|pw} in Wien\oindex{Wien@\textbf{Wien}, \emph{Verwaltungsgebiet}|pw}? Ich erwartete auf mehrere Karten lange eine Antwort, erhielt endlich
               eine{ }ſehr {\pb}flüchtige, dürftige aus
                  Florenz\oindex{Florenz@\textbf{Florenz}|pw}.\pend
           
\pstart
           Mein Papa\pwindex{Hofmannsthal, Hugo August von 21.\,12.\,1841 Wien – 8.\,12.\,1915 ebd.@\textsc{Hofmannsthal, Hugo August von} (21.\,12.\,1841 Wien – 8.\,12.\,1915 ebd.), \emph{Bankdirektor}|pwv} wird Ihnen in den
               nächſten Tagen ein typiertes Exemplar des kleinen Vorſpiels\pwindex{Hofmannsthal, Hugo von 1.\,2.\,1874 Wien – 15.\,7.\,1929 Rodaun@\textsc{Hofmannsthal, Hugo von} (1.\,2.\,1874 Wien – 15.\,7.\,1929 Rodaun), \emph{Schriftsteller}!Vorspiel zur Antigone des Sophokles@\strich\emph{Vorspiel zur Antigone des Sophokles}|pwv}{ }ſchicken, das ich für eine Berlin\oindex{Berlin@\textbf{Berlin}, \emph{Hauptstadt}|pw}er Antigone\pwindex{\textcolor{red}{\textsuperscript{XXXX indx1}}!Antigone@\strich\emph{Antigone}|pw}-vorstellung (26\textsuperscript{ten} März)
               geſchrieben habe. Bitte \label{K_L01021-3v}\edtext{ſchicken Sie
               es mit meinen herzlichen Grüßen an Goldmann\pwindex{Goldmann, Paul 31.\,1.\,1865 Breslau – 25.\,9.\,1935 Wien@\textsc{Goldmann, Paul} (31.\,1.\,1865 Breslau – 25.\,9.\,1935 Wien), \emph{Schriftsteller, Journalist}|pw}}{\lemma{\textnormal{\emph{schicken … Goldmann}}}\Cendnote{\textnormal{Vgl. XXXX Auszeichnungsfehler: Dokument L02908 nicht gefunden.
               }}}\label{K_L01021-3}. Es wäre mir natürlich angenehm {\pb}wenn er etwa in die Vorſtellung
               gehen und darüber{ }ſchreiben würde, aber natürlich abſolut nur, wenn er Luſt hat.\pend
           
\pstart
           Ich hoffe bald einen Brief von Ihnen,{ }ſehe \textsc{Maeterlinck}\pwindex{Maeterlinck, Maurice 29.\,8.\,1862 Gent – 6.\,5.\,1949 Nizza@\textsc{Maeterlinck, Maurice} (29.\,8.\,1862 Gent – 6.\,5.\,1949 Nizza), \emph{Schriftsteller}|pw}{ }ſehr viel, einen überaus erfreulichen Menſchen,
               auch andere Leute, Frauen, Cocotten, Schauſpielerinnen,{ }ſehr viele{ }ſchlechte
               Menſchen, arbeite{ }ſehr viel, finde endlich, daſs der Tag 24 Stunden hat und bin nie{ }ſchläfrig.\pend
           \pstart Von Herzen Ihr \spacefill\mbox{Hugo.}\pend{}\selectlanguage{ngerman}\endnumbering\briefempfaengerindex{Schnitzler, Arthur@\textsc{Schnitzler, Arthur}!zzzHofmannsthal, Hugo von@\emph{von Hugo von Hofmannsthal}!1900-03-151@{15. 3. [1900]}|)be}\mylabel{L01021h}  \newcommand{\dateiname}{L01021}\newcommand{\titel}{Hugo von Hofmannsthal an Arthur Schnitzler, 15. 3. [1900]}\newcommand{\editorInnen}{Martin Anton Müller und Gerd-Hermann Susen}%% latex-leseansicht-abspann.tex
%% Abspann für die Leseansicht.
%% Der Schalter \ifkorrekturansicht ist bereits durch den Vorspann gesetzt.

%% latex-abspann.tex
%% Gemeinsamer Abspann für Korrekturansicht und Leseansicht.
%% Setzt den Schalter \ifkorrekturansicht voraus (gesetzt in den
%% einbindenden Dateien latex-korrekturansicht-abspann.tex bzw.
%% latex-leseansicht-abspann.tex).
%% ---------------------------------------------------------------

\normalsize

% Das esempio-Environment wird nur in der Leseansicht benötigt
\ifkorrekturansicht\else
\newenvironment{esempio}[3]%
{
    \vspace{1.5ex}
    \rlap{\underline{#1}}
    \par
    \setlength{\parindent}{0cm}
    \nopagebreak
    \leftskip=#2cm
    \rightskip=#3cm
}
{
    \par
}
\fi

\doendnotes{C}
\bigskip
\vfill

\clearpage

\footnotesize

\ifkorrekturansicht
  \lohead{\textsc{register}}
\fi

% theindex-Environment neu definieren ohne reledmac
\makeatletter
\renewenvironment{theindex}{%
  \ifkorrekturansicht
    \section*{\indexname}%
  \else
    \subsubsection*{Index der erwähnten Entitäten}%
  \fi
  \setlength{\parindent}{0pt}%
  \setlength{\parskip}{0pt plus 0.3pt}%
  \let\item\@idxitem
}{%
  \ifkorrekturansicht\clearpage\fi
}
\makeatother

\IfFileExists{\jobname-pw.ind}{\input{\jobname-pw.ind}}{}

% Quellenangabe nur in der Leseansicht
\ifkorrekturansicht\else
% Fallback-Definitionen, falls die .tex-Datei \titel etc. nicht gesetzt hat
\providecommand{\titel}{}
\providecommand{\editorInnen}{}
\providecommand{\dateiname}{\jobname}

\vspace{3cm}

\vfill

\footnotesize
\textsc{Quelle}: \titel. Herausgegeben von {\editorInnen}. In: \emph{Arthur Schnitzler: Briefwechsel mit Autorinnen und Autoren}.
 Digitale Edition, https://schnitzler-briefe.acdh.oeaw.ac.at/{\dateiname}.html (Stand \today)
\fi

\end{document}


