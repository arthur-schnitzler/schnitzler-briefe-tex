%% latex-korrekturansicht-vorspann.tex
%% Vorspann für die Korrekturansicht.
%% Lädt die gemeinsame Datei latex-vorspann.tex mit gesetztem Schalter.

\newif\ifkorrekturansicht
\korrekturansichttrue

\input{../tex-inputs/latex-vorspann}


\section[Arthur Schnitzler an Hugo von Hofmannsthal, 11. 9. 1905]{L01545 Arthur Schnitzler an Hugo von Hofmannsthal, 11. 9. 1905}
\nopagebreak\mylabel{L01545v}
\rehead{ }\normalsize\beginnumbering\briefempfaengerindex{Hofmannsthal, Hugo von@\textsc{Hofmannsthal, Hugo von}!zzzSchnitzler, Arthur@\emph{von Arthur Schnitzler}!1905-09-111@{11. 9. 1905}|(be}
\toendnotes[C]{\smallbreak\pagebreak[2]}\Standort{FDH, Hs-30885,122.}
\physDesc{Brief, 1 Blatt, 4 Seiten, 1846 Zeichen
\newline{}Handschrift: schwarze Tinte, deutsche Kurrent}
\buchAbdrucke{\weitereDrucke{Hugo von Hofmannsthal, Arthur Schnitzler: \emph{Briefwechsel}. Frankfurt am Main: \emph{S. Fischer} 1964, S. 214.} }\toendnotes[C]{\smallbreak}
\pstart
           \raggedleft{}{\pb}Wien\oindex{Wien@\textbf{Wien}, \emph{A.ADM2}|pw}{ }11. 9. 905\pend
           
\pstart{}lieber Hugo,\pend\vspace{0.5em}
\pstart
           die Sache mit dem Burgtheater\orgindex{Burgtheater@Burgtheater|pw} war ungeheuer
               einfach. Brahm\pwindex{Brahm, Otto 05.02.1856 – 28.11.1912@\textsc{Brahm, Otto} (05.02.1856 – 28.11.1912), \emph{Theaterleiter/Theaterleiterin, Regisseur/Regisseurin}|pw}{ }ſchrieb mir \label{K_L01545-1v}\edtext{Ende Auguſt}{\lemma{\textnormal{\emph{Ende Auguſt}}}\Cendnote{\textnormal{Brahm\pwindex{Brahm, Otto 05.02.1856 – 28.11.1912@\textsc{Brahm, Otto} (05.02.1856 – 28.11.1912), \emph{Theaterleiter/Theaterleiterin, Regisseur/Regisseurin}|pwk} schrieb am 27. 8. 1905 (\emph{Der Briefwechsel Arthur Schnitzler – Otto Brahm}.
                  Vollständige Ausgabe. Herausgegeben, eingeleitet und erläutert von Oskar
                  Seidlin. Tübingen: \emph{Niemeyer}{ }1975, S. 187–189).}}}\label{K_L01545-1}, Schlenther\pwindex{Schlenther, Paul 20.08.1854 – 30.04.1916@\textsc{Schlenther, Paul} (20.08.1854 – 30.04.1916), \emph{Schriftsteller/Schriftstellerin, Kritiker/Kritikerin, Theaterleiter/Theaterleiterin}|pw} habe
               ihn mit der Miſſion betraut, mich zur Einſendg meines neueſten \label{T_L01545-1v}\edtext{aufzufordern}{\lemma{\textnormal{\emph{aufzufordern}}}\Cendnote{\textnormal{Er schreibt: »einzuſenden«.}}}\label{T_L01545-1}. Ich hierauf,
               nicht faul, ſchreibe Schl.\pwindex{Schlenther, Paul 20.08.1854 – 30.04.1916@\textsc{Schlenther, Paul} (20.08.1854 – 30.04.1916), \emph{Schriftsteller/Schriftstellerin, Kritiker/Kritikerin, Theaterleiter/Theaterleiterin}|pw}, daſs ich eine
               fertige Komoedie\pwindex{Zwischenspiel. Komoedie in drei Akten@\emph{Zwischenspiel. Komödie in drei Akten}|pwv}, u 2 Dramenakte\pwindex{Ruf des Lebens. Schauspiel in drei Akten@\emph{Der Ruf des Lebens. Schauspiel in drei Akten}|pwv} auf Lager ha\substVorne{}\textsuperscript{tte}\substDazwischen{}be\substHinten{}, er telegrafirt, noch fleißiger, ſoll ihm alles schicken; \introOben{}ich thu es,\introOben{} er antwortet 5 Tage drauf, die Entſcheidg über Dra{\pb}ma\pwindex{Ruf des Lebens. Schauspiel in drei Akten@\emph{Der Ruf des Lebens. Schauspiel in drei Akten}|pwv}{ }\substVorne{}\textsuperscript{laſſe}\substDazwischen{}bitte\substHinten{} er bis nach Vollendg aufſchieben zu dürfen, Komoedie\pwindex{Zwischenspiel. Komoedie in drei Akten@\emph{Zwischenspiel. Komödie in drei Akten}|pwv} nehme er an Mitte
                  October (ich hatte frühen Termin zur Beding gemacht), wolle meine
               Beſetzsvorſchläge, er ni{\geminationm}t ſie ſelben Tags ebenſo
               telegrafiſch an, und am nächſten Morgen ſteht die \label{K_L01545-2v}\edtext{Notiz}{\lemma{\textnormal{\emph{Notiz}}}\Cendnote{\textnormal{»Ende
                     Oktober geht Schnitzlers neue Komödie
                        ›\so{Zwischenspiel}\pwindex{Zwischenspiel. Komoedie in drei Akten@\emph{Zwischenspiel. Komödie in drei Akten}|pw}‹ zum erstenmal in Szene.« ([O. V.]: \emph{Aus den Theatern. Wien, 9. September}\pwindex{Aus den Theatern. Wien, 9. September@\emph{Aus den Theatern. Wien, 9. September}|pwk}. In: \emph{Neue Freie Presse}\pwindex{Neue Freie Presse@\emph{Neue Freie Presse}|pwk}, Nr. 14.744,
                        9. 9. 1905, Abendblatt, S. 4.)}}}\label{K_L01545-2} in der Zeitung. Es
                  ko{\geminationm}t hier vor Berlin\oindex{Berlin@\textbf{Berlin}, \emph{P.PPLC}|pw}; mit Brahm\pwindex{Brahm, Otto 05.02.1856 – 28.11.1912@\textsc{Brahm, Otto} (05.02.1856 – 28.11.1912), \emph{Theaterleiter/Theaterleiterin, Regisseur/Regisseurin}|pw} bin ich erſt heute
               (vor 5 Minuten kam das endgiltige Telegra{\geminationm}) einig
               geworden; Verzögerung, weil er durchaus beide Stücke\pwindex{Zwischenspiel. Komoedie in drei Akten@\emph{Zwischenspiel. Komödie in drei Akten}|pwv}\pwindex{Ruf des Lebens. Schauspiel in drei Akten@\emph{Der Ruf des Lebens. Schauspiel in drei Akten}|pwv} wollte – Mit dem \textsc{Reinhardt}theater\orgindex{Deutsches Theater Berlin@Deutsches Theater Berlin|pwv} wird ſich wahr{\pb}ſcheinlich nichts machen laſſen; was ſie mir im Lauf der
               letzten 10 Tage an (mildeſter Ausdruck) Schlampereien angethan, iſt unglaublich. Der
               letzte Scherz war, daſs ich \label{K_L01545-3v}\edtext{Mittwoch{ }ein Telegramm bekam}{\lemma{\textnormal{\emph{Mittwoch … bekam}}}\Cendnote{\textnormal{abgedruckt in: \emph{Der Briefwechsel Arthur Schnitzlers mit Max Reinhardt und
                        dessen Mitarbeitern}. Herausgegeben von Renate Wagner. Salzburg: \emph{Otto
                        Müller Verlag}{ }1971, S. 50. Den versprochenen Brief (und einen weiteren,
                  der am 12. 9. 1905 angekündigt wurde) dürfte er nicht erhalten
                  haben.}}}\label{K_L01545-3} dſs ein ausführlicher Brief auf d. Wege – und der bisher nicht da
               iſt. Es ſtand beinah ſchon feſt für mich, dſs die \textsc{Sorma}\pwindex{Sorma, Agnes 17.05.1862 – 10.02.1927@\textsc{Sorma, Agnes} (17.05.1862 – 10.02.1927), \emph{Schauspieler/Schauspielerin}|pw} die Komoedie\pwindex{Zwischenspiel. Komoedie in drei Akten@\emph{Zwischenspiel. Komödie in drei Akten}|pwv}{ }ſpielen müſſte. Über all dies mündlich
               näheres. –\pend
           
\pstart
           Wir bleiben bis nach 15. hier, wohl 20., denken da{\geminationn} auf 10 Tage fortzugehn, – Salzka{\geminationm}ergut\oindex{Salzkammergut@\textbf{Salzkammergut}, \emph{L.RGN}|pw} kaum; vielleicht nur
                  {\pb}\label{K_L01545-4v}\edtext{Semmering\oindex{Semmering@\textbf{Semmering}, \emph{A.ADM3}|pw}}{\lemma{\textnormal{\emph{Semmering}}}\Cendnote{\textnormal{Dahin fuhren sie vom 22.
                  bis zum 26. 9. 1905.}}}\label{K_L01545-4}. – Mit dem 3. Akt\pwindex{Ruf des Lebens. Schauspiel in drei Akten@\emph{Der Ruf des Lebens. Schauspiel in drei Akten}|pwv} glaub ich zu einer Art Reſultat zu
                  ko{\geminationm}en – das 3 mal einaktige des Stoffes iſt natürlich
               nicht ganz zu beſiegen, es ko{\geminationm}t im weſentlichen, was man
               auch thut, dramatiſch auf einen Schwindel heraus. Nun, das iſt unſer Metier.\pend
           
\pstart
           Ich freue mich, dſs Sie viel arbeiten, und ſehe dem nächſten \label{K_L01545-5v}\edtext{Vorleſungsabend}{\lemma{\textnormal{\emph{Vorleſungsabend}}}\Cendnote{\textnormal{Gemeint ist eine Vorlesung von Werken in privatem Kreis.}}}\label{K_L01545-5}
               mit ſchönſter Erwartung entgegen. Was hat Sie ſo raſch aus \textsc{Misurina}\oindex{Misurina@\textbf{Misurina}, \emph{P.PPL}|pw} vertrieben?\pend
           
\pstart
           Wir grüßen Sie Beide\pwindex{Schnitzler, Olga 17.01.1882 – 13.01.1970@\textsc{Schnitzler, Olga} (17.01.1882 – 13.01.1970), \emph{Schauspieler/Schauspielerin, Sänger/Sängerin}|pwv}{ }Beide\pwindex{Hofmannsthal, Gertrude von 16.03.1880 – 09.11.1959@\textsc{Hofmannsthal, Gertrude von} (16.03.1880 – 09.11.1959)|pwv}.\pend
           \pstart Herzlichst Ihr \spacefill\mbox{A.}\pend{}
\pstart
           \noindent{}\label{T_L01545-2v}\edtext{Sehen Sie Burckhard\pwindex{Burckhard, Max Eugen 14.07.1854 – 16.03.1912@\textsc{Burckhard, Max Eugen} (14.07.1854 – 16.03.1912), \emph{Schriftsteller/Schriftstellerin, Rechtswissenschaftler/Rechtswissenschaftlerin, Theaterleiter/Theaterleiterin}|pw}, grüßen Sie ihn ſehr.}{\lemma{\textnormal{\emph{Sehen … ſehr.}}}\Cendnote{\textnormal{neben der Anrede auf dem Kopf}}}\label{T_L01545-2}\pend
           \selectlanguage{ngerman}\endnumbering\briefempfaengerindex{Hofmannsthal, Hugo von@\textsc{Hofmannsthal, Hugo von}!zzzSchnitzler, Arthur@\emph{von Arthur Schnitzler}!1905-09-111@{11. 9. 1905}|)be}\mylabel{L01545h}  \normalsize

\doendnotes{C}
\bigskip
\vfill

\clearpage

\footnotesize

\lohead{\textsc{register}}

% Definiere theindex-Environment komplett neu ohne reledmac
\makeatletter
\renewenvironment{theindex}{%
  \section*{\indexname}%
  \setlength{\parindent}{0pt}%
  \setlength{\parskip}{0pt plus 0.3pt}%
  \let\item\@idxitem
}{%
  \clearpage
}
\makeatother

\IfFileExists{\jobname-pw.ind}{\input{\jobname-pw.ind}}{}

\end{document}

      