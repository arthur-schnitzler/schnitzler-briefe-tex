%% latex-leseansicht-vorspann.tex
%% Vorspann für die Leseansicht.
%% Lädt die gemeinsame Datei latex-vorspann.tex mit nicht gesetztem Schalter.

\newif\ifkorrekturansicht
\korrekturansichtfalse

\input{../tex-inputs/latex-vorspann}


\section[Arthur Schnitzler an Gustav Schwarzkopf, 1. 8. 1899]{L04134 Arthur Schnitzler an Gustav Schwarzkopf, 1. 8. 1899}
\nopagebreak\mylabel{L04134v}
\rehead{ }\normalsize\beginnumbering\briefempfaengerindex{Schwarzkopf, Gustav@\textsc{Schwarzkopf, Gustav}!zzzSchnitzler, Arthur@\emph{von Arthur Schnitzler}!1899-08-012@{1. 8. 1899}|(be}
\toendnotes[C]{\smallbreak\pagebreak[2]}
\correspDesc{Versand  durch Arthur Schnitzler am 1. 8. 1899 in Toblach
\newline{}Erhalt  durch Gustav Schwarzkopf im Zeitraum [2. 8. 1899 – 6. 8. 1899?] in Wien}\toendnotes[C]{\smallbreak}
\Standort{CUL, Schnitzler, B 96.}
\physDesc{Brief, 1 Blatt, 4 Seiten, 2313 Zeichen
\newline{}Handschrift: Bleistift, deutsche Kurrent}\toendnotes[C]{\smallbreak}
\pstart
           \raggedleft{}{\pb}\textsc{Toblach\oindex{Toblach@\textbf{Toblach}, \emph{Verwaltungsgebiet}|pw}} , 1. 8. 99.\pend
           \vspace{0.5em}
\pstart
           lieber Guſtav, machen Sie aus den 2–3 Iſchler\oindex{Bad Ischl@\textbf{Bad Ischl}|pw} Tagen 10–24, ſo will ich mich zufrieden geben – obwohl Ihr Mistrauen
               gegen die Fußpartie ziemlich ungerechtfertigt iſt. Wir wollen wirklich zu Fuß gehen,
               die Partie iſt bereits zusa{\geminationm}en geſtellt,
               und, ſoweit ſich etwas vorausſagen läßt, fangen wir (Richard\pwindex{Beer-Hofmann, Richard 11.\,7.\,1866 Wien – 26.\,9.\,1945 New York City@\textsc{Beer-Hofmann, Richard} (11.\,7.\,1866 Wien – 26.\,9.\,1945 New York City), \emph{Schriftsteller}|pw}, \textsc{Wasserm\pwindex{Wassermann, Jakob 10.\,3.\,1873 Fürth – 1.\,1.\,1934 Altaussee@\textsc{Wassermann, Jakob} (10.\,3.\,1873 Fürth – 1.\,1.\,1934 Altaussee), \emph{Schriftsteller}|pw}}, ich –) \label{K_L04134-1v}\edtext{Samſtg früh zu ſchreiten}{\lemma{\textnormal{\emph{Samstg früh zu schreiten}}}\Cendnote{\textnormal{Das wurde eingehalten, vgl. A. S.: \emph{Tagebuch}, 5. 8. 1899.}}}\label{K_L04134-1} an. Sehr mäßige Tagesleiſtungen; die Nachmittge i{\geminationm}er frei, bei \label{K_L04134-2v}\edtext{Trient\oindex{Trient@\textbf{Trient}|pw} ko{\geminationm}en wir heraus}{\lemma{\textnormal{\emph{Trient kommen wir heraus}}}\Cendnote{\textnormal{Sie erreichten Trient\oindex{Trient@\textbf{Trient}|pwk} am
                     12. 8. 1899.}}}\label{K_L04134-2}. Da{\geminationn} über Bozen\oindex{Bozen@\textbf{Bozen}, \emph{Hauptstadt}|pw} zurück, von Bozen\oindex{Bozen@\textbf{Bozen}, \emph{Hauptstadt}|pw} aus will ich
               ev. zum Theil per Rad über Brenner\oindex{Brenner@\textbf{Brenner}, \emph{Pass}|pw} nach Innsbruck\oindex{Innsbruck@\textbf{Innsbruck}, \emph{Verwaltungsgebiet}|pw}, Salzburg\oindex{Salzburg@\textbf{Salzburg}, \emph{Verwaltungsgebiet}|pw}, Iſchl\oindex{Bad Ischl@\textbf{Bad Ischl}|pw}, wo ich {\pb}\label{K_L04134-3v}\edtext{zwiſchen 18. u 20.}{\lemma{\textnormal{\emph{zwischen 18. u 20.}}}\Cendnote{\textnormal{Er langte bereits am 15. 8. 1899 in Ischl\oindex{Bad Ischl@\textbf{Bad Ischl}|pwk} an.}}}\label{K_L04134-3} ſein dürfte. Vielleicht
               treffen wir ſchon früher zuſa{\geminationm}en? in Salzburg\oindex{Salzburg@\textbf{Salzburg}, \emph{Verwaltungsgebiet}|pw}? Jedenfalls halte ich Sie ſtets auf dem Laufenden, wo
               ich bin; mir{ }ſchreiben Sie am beſten nach Wien\oindex{Wien@\textbf{Wien}, \emph{Verwaltungsgebiet}|pw}; es
               wird mir von Zeit \textcolor{gray}{zu} Zeit nachgeſchickt; u. bleibt noch immer
               am ſicherſten. – Bei \label{K_L04134-4v}\edtext{Richard\pwindex{Beer-Hofmann, Richard 11.\,7.\,1866 Wien – 26.\,9.\,1945 New York City@\textsc{Beer-Hofmann, Richard} (11.\,7.\,1866 Wien – 26.\,9.\,1945 New York City), \emph{Schriftsteller}|pw} war ich (u Waſſerm\pwindex{Wassermann, Jakob 10.\,3.\,1873 Fürth – 1.\,1.\,1934 Altaussee@\textsc{Wassermann, Jakob} (10.\,3.\,1873 Fürth – 1.\,1.\,1934 Altaussee), \emph{Schriftsteller}|pw}) vorgeſtern}{\lemma{\textnormal{\emph{Richard … vorgestern}}}\Cendnote{\textnormal{Vgl. A. S.: \emph{Tagebuch}, 30. 7. 1899.}}}\label{K_L04134-4}; er iſt eigentlich in ganz guter Stimmung, und, denken
               Sie, die Novelle\pwindex{Beer-Hofmann, Richard 11.\,7.\,1866 Wien – 26.\,9.\,1945 New York City@\textsc{Beer-Hofmann, Richard} (11.\,7.\,1866 Wien – 26.\,9.\,1945 New York City), \emph{Schriftsteller}!Tod Georgs@\strich\emph{Der Tod Georgs}|pwv} iſt fertig.
               Zu Weihnachten (\label{K_L04134-5v}\edtext{mäßigen
               Sie Ihre Heiterkeit}{\lemma{\textnormal{\emph{mäßigen … Heiterkeit}}}\Cendnote{\textnormal{Beer-Hofmann\pwindex{Beer-Hofmann, Richard 11.\,7.\,1866 Wien – 26.\,9.\,1945 New York City@\textsc{Beer-Hofmann, Richard} (11.\,7.\,1866 Wien – 26.\,9.\,1945 New York City), \emph{Schriftsteller}|pwk} brauchte notorisch lange für die Abfassung
                  seiner Werke.}}}\label{K_L04134-5} beſſer als ich es bei dieſer Mittheilg vermochte) will
               er ſeine Tragödie\pwindex{Beer-Hofmann, Richard 11.\,7.\,1866 Wien – 26.\,9.\,1945 New York City@\textsc{Beer-Hofmann, Richard} (11.\,7.\,1866 Wien – 26.\,9.\,1945 New York City), \emph{Schriftsteller}!Graf von Charolais. Ein Trauerspiel@\strich\emph{Der Graf von Charolais. Ein Trauerspiel}|pwv} beendet
               haben. – Ich ſchreibe an der mei{\pb}nen\pwindex{Schnitzler, Arthur 15. 5. 1862 Wien – 21. 10. 1931 ebd.@\textsc{Schnitzler, Arthur} (15. 5. 1862 Wien – 21. 10. 1931 ebd.), \emph{Schriftsteller, Mediziner}!Schleier der Beatrice. Schauspiel in fünf Akten@\strich\emph{Der Schleier der Beatrice. Schauspiel in fünf Akten}|pwv},– Nachmittg gibt es i{\geminationm}er 2–4 Stunden, in denen man dazu Zeit hat; auch auf
               der Fußt\textcolor{gray}{our}{ }ſoll es so gehalten werden. – Während der
               Arbeitszeit geht’s mir am beſten; dagegen wach ich heut recht oft aus
               den ſchrecklichſten Träumen auf; und an den Abenden packt es mich oft ſo – na, Sie
               gehn mir ſehr, ſehr ab. – Morgen Vormittg fahr ich nach Niederdorf\oindex{Niederdorf@\textbf{Niederdorf}, \emph{Verwaltungsgebiet}|pw}, wo die \label{K_L04134-6v}\edtext{Familie\pwindex{Reinhard, Karl 2.\,3.\,1834 Prag – 28.\,4.\,1905 Wien@\textsc{Reinhard, Karl} (2.\,3.\,1834 Prag – 28.\,4.\,1905 Wien), \emph{Geschäftsführer}|pwv}\pwindex{Reinhard, Therese 13.\,12.\,1844 Wien – 25.\,3.\,1926 ebd.@\textsc{Reinhard, Therese} (13.\,12.\,1844 Wien – 25.\,3.\,1926 ebd.)|pwv}\pwindex{Burger, Caroline 11.\,7.\,1869 Wien – 15.\,3.\,1959 ebd.@\textsc{Burger, Caroline} (11.\,7.\,1869 Wien – 15.\,3.\,1959 ebd.)|pwv}}{\lemma{\textnormal{\emph{Familie}}}\Cendnote{\textnormal{seiner
                  verstorbenen Partnerin Marie Reinhard\pwindex{Reinhard, Marie 13.\,3.\,1871 Wien – 18.\,3.\,1899 ebd.@\textsc{Reinhard, Marie} (13.\,3.\,1871 Wien – 18.\,3.\,1899 ebd.), \emph{Gesangspädagogin}|pwk}}}}\label{K_L04134-6}
               iſt (Vater\pwindex{Reinhard, Karl 2.\,3.\,1834 Prag – 28.\,4.\,1905 Wien@\textsc{Reinhard, Karl} (2.\,3.\,1834 Prag – 28.\,4.\,1905 Wien), \emph{Geschäftsführer}|pwv}, Mutter\pwindex{Reinhard, Therese 13.\,12.\,1844 Wien – 25.\,3.\,1926 ebd.@\textsc{Reinhard, Therese} (13.\,12.\,1844 Wien – 25.\,3.\,1926 ebd.)|pwv}, Schweſter\pwindex{Burger, Caroline 11.\,7.\,1869 Wien – 15.\,3.\,1959 ebd.@\textsc{Burger, Caroline} (11.\,7.\,1869 Wien – 15.\,3.\,1959 ebd.)|pwv}). – Jetzt, in dieſen heilen
               Sommertagen wird alles eigentlich ſo ganz, ganz klar. –\pend
           
\pstart
           {\pb}Von dem »\label{K_L04134-7v}\edtext{Milchreis\pwindex{Auernheimer, Raoul 15.\,4.\,1876 Wien – 6.\,1.\,1948 Oakland@\textsc{Auernheimer, Raoul} (15.\,4.\,1876 Wien – 6.\,1.\,1948 Oakland), \emph{Schriftsteller, Journalist, Kritiker}!Milchreis@\strich\emph{Der Milchreis}|pw}}{\lemma{\textnormal{\emph{Milchreis}}}\Cendnote{\textnormal{Raoul Auernheimer\pwindex{Auernheimer, Raoul 15.\,4.\,1876 Wien – 6.\,1.\,1948 Oakland@\textsc{Auernheimer, Raoul} (15.\,4.\,1876 Wien – 6.\,1.\,1948 Oakland), \emph{Schriftsteller, Journalist, Kritiker}|pwk}: \emph{Der Milchreis}\pwindex{Auernheimer, Raoul 15.\,4.\,1876 Wien – 6.\,1.\,1948 Oakland@\textsc{Auernheimer, Raoul} (15.\,4.\,1876 Wien – 6.\,1.\,1948 Oakland), \emph{Schriftsteller, Journalist, Kritiker}!Milchreis@\strich\emph{Der Milchreis}|pwk}. In: \emph{Neue Freie Presse}\pwindex{Neue Freie Presse@\emph{Neue Freie Presse}|pwk}, Nr. 12.548,
                     30. 7. 1899, Morgenblatt, S. 7–8. }}}\label{K_L04134-7}«
               waren Sie hoffentlich ebenſo entzückt wie ich. Ein \label{K_L04134-8v}\edtext{Epigone\pwindex{Auernheimer, Raoul 15.\,4.\,1876 Wien – 6.\,1.\,1948 Oakland@\textsc{Auernheimer, Raoul} (15.\,4.\,1876 Wien – 6.\,1.\,1948 Oakland), \emph{Schriftsteller, Journalist, Kritiker}|pwv}}{\lemma{\textnormal{\emph{Epigone}}}\Cendnote{\textnormal{Auernheimer\pwindex{Auernheimer, Raoul 15.\,4.\,1876 Wien – 6.\,1.\,1948 Oakland@\textsc{Auernheimer, Raoul} (15.\,4.\,1876 Wien – 6.\,1.\,1948 Oakland), \emph{Schriftsteller, Journalist, Kritiker}|pwk} und Herzl\pwindex{Herzl, Theodor 2.\,5.\,1860 Budapest – 3.\,7.\,1904 Edlach@\textsc{Herzl, Theodor} (2.\,5.\,1860 Budapest – 3.\,7.\,1904 Edlach), \emph{Schriftsteller, Journalist}|pwk} waren weitläufig verwandt, das dürfte Schnitzler aber zu diesem Zeitpunkt nicht gewusst
                  haben.}}}\label{K_L04134-8} von Herzl\pwindex{Herzl, Theodor 2.\,5.\,1860 Budapest – 3.\,7.\,1904 Edlach@\textsc{Herzl, Theodor} (2.\,5.\,1860 Budapest – 3.\,7.\,1904 Edlach), \emph{Schriftsteller, Journalist}|pw} wie es ſcheint –
               nur ohne Geiſt und noch viel affectirter. Es iſt doch wirklich einer \introOben{}von dieſe\textcolor{gray}{m} Jüngſt Wien\introOben{} ekelhafter als der
               andre; (– wie gut haben wir uns doch i{\geminationm}er zuſa{\geminationm}en ekeln können! –)\pend
           
\pstart
           — – Von der Gl.\pwindex{Glümer, Marie 3.\,7.\,1867 Wien – 16.\,11.\,1925 München@\textsc{Glümer, Marie} (3.\,7.\,1867 Wien – 16.\,11.\,1925 München), \emph{Schauspielerin}|pw} hab ich Nachricht,
               dſs ſie nach Abazzia\oindex{Opatija@\textbf{Opatija}, \emph{Hauptstadt}|pw} geht; die »\label{K_L04134-9v}\edtext{ruſſiſche Gräfin\pwindex{Elsinger, Marie *~28.\,2.\,1874 St. Pölten@\textsc{Elsinger, Marie} (*~28.\,2.\,1874 St. Pölten), \emph{Schauspielerin}|pwv}}{\lemma{\textnormal{\emph{russische Gräfin}}}\Cendnote{\textnormal{Marie Elsinger\pwindex{Elsinger, Marie *~28.\,2.\,1874 St. Pölten@\textsc{Elsinger, Marie} (*~28.\,2.\,1874 St. Pölten), \emph{Schauspielerin}|pwk} war mit
                  einem Russen\pwindex{?? [Russischer Verlobter von Marie Elsinger] @\textsc{?? [Russischer Verlobter von Marie Elsinger]}|pwkv} verlobt,
                     vgl. A. S.: \emph{Tagebuch}, 15. 7. 1899.}}}\label{K_L04134-9}«,
               hat \introOben{}mir\introOben{} einige irrſinnige So{\geminationm}erpläne mitgetheilt; ich nehme an, dſs mir heute ausWien\oindex{Wien@\textbf{Wien}, \emph{Verwaltungsgebiet}|pw} ein Brief von ihr nachgeſchickt wird. We{\geminationn}
               nicht, iſt ſie verſchollen; – und ich auch. –\pend
           
\pstart
           Leben Sie wohl, ſeien Sie mir herzlich{\\[\baselineskip]} gegrüßt,{\\[\baselineskip]} Ihr
                  \spacefill\mbox{Arthur}\pend
           \leftskip=0em{}
\pstart
           \noindent{}(daſs meine Mama\pwindex{Schnitzler, Louise 8.\,7.\,1840 Kőszeg – 9.\,9.\,1911 Wien@\textsc{Schnitzler, Louise} (8.\,7.\,1840 Kőszeg – 9.\,9.\,1911 Wien)|pwv} u Schweſter\pwindex{Hajek, Gisela 20.\,12.\,1867 Wien – 3.\,2.\,1953 Cambridge@\textsc{Hajek, Gisela} (20.\,12.\,1867 Wien – 3.\,2.\,1953 Cambridge)|pwv} hier ſind,
                  wiſſen Sie ja?–) \textsc{Wasserm.\pwindex{Wassermann, Jakob 10.\,3.\,1873 Fürth – 1.\,1.\,1934 Altaussee@\textsc{Wassermann, Jakob} (10.\,3.\,1873 Fürth – 1.\,1.\,1934 Altaussee), \emph{Schriftsteller}|pw}} hat ſich aus \introOben{}Pndſch\oindex{Pension Pundschu@\textbf{Pension Pundschu}, \emph{Hotel}|pw}.\introOben{}{ }\textsc{Seeboden\oindex{Seeboden am Millstättersee@\textbf{Seeboden am Millstättersee}|pw}}{ }Richards\pwindex{Beer-Hofmann, Richard 11.\,7.\,1866 Wien – 26.\,9.\,1945 New York City@\textsc{Beer-Hofmann, Richard} (11.\,7.\,1866 Wien – 26.\,9.\,1945 New York City), \emph{Schriftsteller}|pw} Rad mitgenommen, in deſſen
                  Speichen (buchſtäblich) Spinnweb ſteckte.\pend
           \selectlanguage{ngerman}\endnumbering\briefempfaengerindex{Schwarzkopf, Gustav@\textsc{Schwarzkopf, Gustav}!zzzSchnitzler, Arthur@\emph{von Arthur Schnitzler}!1899-08-012@{1. 8. 1899}|)be}\mylabel{L04134h}
\begin{anhang}
\end{anhang}\newcommand{\dateiname}{L04134}\newcommand{\titel}{Arthur Schnitzler an Gustav Schwarzkopf, 1. 8. 1899}\newcommand{\editorInnen}{Herausgegeben von Jahnke, SelmaMüller, Martin Anton}%% latex-leseansicht-abspann.tex
%% Abspann für die Leseansicht.
%% Der Schalter \ifkorrekturansicht ist bereits durch den Vorspann gesetzt.

%% latex-abspann.tex
%% Gemeinsamer Abspann für Korrekturansicht und Leseansicht.
%% Setzt den Schalter \ifkorrekturansicht voraus (gesetzt in den
%% einbindenden Dateien latex-korrekturansicht-abspann.tex bzw.
%% latex-leseansicht-abspann.tex).
%% ---------------------------------------------------------------

\normalsize

% Das esempio-Environment wird nur in der Leseansicht benötigt
\ifkorrekturansicht\else
\newenvironment{esempio}[3]%
{
    \vspace{1.5ex}
    \rlap{\underline{#1}}
    \par
    \setlength{\parindent}{0cm}
    \nopagebreak
    \leftskip=#2cm
    \rightskip=#3cm
}
{
    \par
}
\fi

\doendnotes{C}
\bigskip
\vfill

\clearpage

\footnotesize

\ifkorrekturansicht
  \lohead{\textsc{register}}
\fi

% theindex-Environment neu definieren ohne reledmac
\makeatletter
\renewenvironment{theindex}{%
  \ifkorrekturansicht
    \section*{\indexname}%
  \else
    \subsubsection*{Index der erwähnten Entitäten}%
  \fi
  \setlength{\parindent}{0pt}%
  \setlength{\parskip}{0pt plus 0.3pt}%
  \let\item\@idxitem
}{%
  \ifkorrekturansicht\clearpage\fi
}
\makeatother

\IfFileExists{\jobname-pw.ind}{\input{\jobname-pw.ind}}{}

% Quellenangabe nur in der Leseansicht
\ifkorrekturansicht\else
% Fallback-Definitionen, falls die .tex-Datei \titel etc. nicht gesetzt hat
\providecommand{\titel}{}
\providecommand{\editorInnen}{}
\providecommand{\dateiname}{\jobname}

\vspace{3cm}

\vfill

\footnotesize
\textsc{Quelle}: \titel. Herausgegeben von {\editorInnen}. In: \emph{Arthur Schnitzler: Briefwechsel mit Autorinnen und Autoren}.
 Digitale Edition, https://schnitzler-briefe.acdh.oeaw.ac.at/{\dateiname}.html (Stand \today)
\fi

\end{document}


