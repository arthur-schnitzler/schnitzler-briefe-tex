%% latex-leseansicht-vorspann.tex
%% Vorspann für die Leseansicht.
%% Lädt die gemeinsame Datei latex-vorspann.tex mit nicht gesetztem Schalter.

\newif\ifkorrekturansicht
\korrekturansichtfalse

\input{../tex-inputs/latex-vorspann}


\section[Arthur Schnitzler an Stefan Zweig, 27. 7. 1923]{L03750 Arthur Schnitzler an Stefan Zweig, 27. 7. 1923}
\nopagebreak\mylabel{L03750v}
\rehead{ }\normalsize\beginnumbering\briefempfaengerindex{Zweig, Stefan@\textsc{Zweig, Stefan}!zzzSchnitzler, Arthur@\emph{von Arthur Schnitzler}!1923-07-271@{27. 7. 1923}|(be}
\toendnotes[C]{\smallbreak\pagebreak[2]}
\correspDesc{Versand  durch Arthur Schnitzler am 27. 7. 1923 in Wien
\newline{}Erhalt  durch Stefan Zweig am [28.? 7. 1923] in Salzburg}\toendnotes[C]{\smallbreak}
\Standort{Jerusalem, National Library of Israel, ARC. Ms. Var. 305 1 58 Stefan Zweig Collection.}
\physDesc{Postkarte, 824 Zeichen
\newline{}Handschrift: schwarze Tinte, lateinische Kurrent
\newline{}Versand: Stempel: »\nobreak{}\oindex{XVIII., Währing@\textbf{XVIII., Währing}, \emph{Verwaltungsgebiet}|pwk}18/\textsubscript{1} Wien
                                       110, 27. VII. 23, 15\nobreak{}«.  
\newline{}Zweig: mit Bleistift datiert: »27/VII 1923« }\toendnotes[C]{\smallbreak}\pstart{}{\pb}\label{T_L03750-1v}\edtext{\textcolor{gray}{\textbf{A. S.}}}{\lemma{\textnormal{\emph{A. S.}}}\Cendnote{\textnormal{ovaler Absenderkleber}}}\label{T_L03750-1}\pend{}\pstart{}\textcolor{gray}{\textbf{WIEN, XVIII.}}\oindex{XVIII., Währing@\textbf{XVIII., Währing}, \emph{Verwaltungsgebiet}|pw}\pend{}\pstart{}\textcolor{gray}{\textbf{STERNWARTESTR. 71}}\oindex{Wien@\textbf{Wien}!XVIII., Währing@\textbf{XVIII., Währing}!Sternwartestraße 71@\textbf{Sternwartestraße 71}, \emph{Wohngebäude}|pw}\pend{}{\bigskip}\pstart{}An\pend{}\pstart{}Hrn Dr. Stefan Zweig\pend{}\pstart{}Salzburg\oindex{Salzburg@\textbf{Salzburg}, \emph{Verwaltungsgebiet}|pw}\pend{}\pstart{}Kapuzinerberg 5\oindex{Paschinger Schlössl@\textbf{Paschinger Schlössl}, \emph{Wohngebäude}|pw}\pend{}{\bigskip}\vspace{1em}
\pstart
           \raggedleft{}{\pb}27. 7. 923.\pend
           \vspace{0.5em}
\pstart
           lieber Herr Doctor, vielen vielen Dank! Wie Sie sehen bin ich noch
               (war wieder) in Wien\oindex{Wien@\textbf{Wien}, \emph{Verwaltungsgebiet}|pw}, fahre voraussichtlich Ende
               nächster Woche nach \label{K_L03750-1v}\edtext{Deutschland\oindex{Deutschland@\textbf{Deutschland}|pw} (Schwarzwald\oindex{Schwarzwald@\textbf{Schwarzwald}, \emph{Gebirge}|pw}, 
               Baden Baden\oindex{Baden-Baden@\textbf{Baden-Baden}|pw}, wo meine Kinder\pwindex{Cappellini, Lili 13.\,9.\,1909 Wien – 26.\,7.\,1928 Venedig@\textsc{Cappellini, Lili} (13.\,9.\,1909 Wien – 26.\,7.\,1928 Venedig)|pwv}\pwindex{Schnitzler, Heinrich 9.\,8.\,1902 Hinterbrühl – 12.\,7.\,1982 Wien@\textsc{Schnitzler, Heinrich} (9.\,8.\,1902 Hinterbrühl – 12.\,7.\,1982 Wien), \emph{Regisseur, Schauspieler}|pwv} bei ihrer Mutter\pwindex{Schnitzler, Olga 17.\,1.\,1882 Wien – 13.\,1.\,1970 Lugano@\textsc{Schnitzler, Olga} (17.\,1.\,1882 Wien – 13.\,1.\,1970 Lugano), \emph{Schauspielerin, Sängerin}|pwv} sind) und in die Schweiz\oindex{Schweiz@\textbf{Schweiz}|pw}}{\lemma{\textnormal{\emph{Deutschland … Schweiz}}}\Cendnote{\textnormal{Schnitzler reiste am 3. 8. 1923 von Wien\oindex{Wien@\textbf{Wien}, \emph{Verwaltungsgebiet}|pwk} nach Salzburg\oindex{Salzburg@\textbf{Salzburg}, \emph{Verwaltungsgebiet}|pwk}, wo er im Österreichischen
                                    Hof\oindex{Österreichischer Hof@\textbf{Österreichischer Hof}, \emph{Hotel}|pwk} abstieg und zwei Nächte blieb. Über Stuttgart\oindex{Stuttgart@\textbf{Stuttgart}|pwk} (eine Übernachtung) reiste er dann nach Baden-Baden\oindex{Baden-Baden@\textbf{Baden-Baden}|pwk}. Hier blieb er bis zum 14. 8. 1923, danach
                        folgte die Reise in die Schweiz\oindex{Schweiz@\textbf{Schweiz}|pwk}, vor allem
                        nach Celerina\oindex{Celerina@\textbf{Celerina}|pwk}. Am 7. 9. 1923 reiste er
                        nach Vorarlberg\oindex{Vorarlberg@\textbf{Vorarlberg}|pwk}. Am 15. 9. 1923 kam er wieder
                        in Wien\oindex{Wien@\textbf{Wien}, \emph{Verwaltungsgebiet}|pwk} an.}}}\label{K_L03750-1}. Um R. R.\pwindex{Rolland, Romain 29.\,1.\,1866 Clamecy – 30.\,12.\,1944 Vézelay@\textsc{Rolland, Romain} (29.\,1.\,1866 Clamecy – 30.\,12.\,1944 Vézelay), \emph{Schriftsteller}|pw}
               kennen zu lernen und Sie wiederzusehen, werd ich, we{\geminationn}
               nicht unvorhergesehene Hindernisse obwalten – mich gern auf der Durchreise in Salzburg\oindex{Salzburg@\textbf{Salzburg}, \emph{Verwaltungsgebiet}|pw} aufhalten – ich denke, das wäre dann
                  3., ev. 4. od 5. August. Wohnen werd ich im
                  oesterr. Hof.\oindex{Österreichischer Hof@\textbf{Österreichischer Hof}, \emph{Hotel}|pw} – und Sie in {\pb}jedem Fall vorher verständigen. (Oder raten Sie mir ein
               andres Hotel? Ist Europe\oindex{Grand Hotel de L’Europe, G. Jung@\textbf{Grand Hotel de L’Europe, G. Jung}, \emph{Hotel}|pw} erschwinglich – was bei
               kurzem Aufenthalt durch die Bahnhofnähe\oindex{Hauptbahnhof Salzburg@\textbf{Hauptbahnhof Salzburg}, \emph{Bahnhofsgebäude}|pwv} verlockend wäre!)\pend
           
\pstart
           Empfehlen Sie mich Ihrer verehrten Gattin\pwindex{Zweig, Friderike Maria 4.\,12.\,1882 Wien – 18.\,1.\,1971 Stamford@\textsc{Zweig, Friderike Maria} (4.\,12.\,1882 Wien – 18.\,1.\,1971 Stamford), \emph{Schriftstellerin}|pwv} und seien Sie sehr herzlich gegrüßt, und nochmal\textcolor{gray}{s}
               allerwärmstens bedankt von Ihrem\pend
           \pstart \spacefill\mbox{Arthur Schnitzler}\pend{}\selectlanguage{ngerman}\endnumbering\briefempfaengerindex{Zweig, Stefan@\textsc{Zweig, Stefan}!zzzSchnitzler, Arthur@\emph{von Arthur Schnitzler}!1923-07-271@{27. 7. 1923}|)be}\mylabel{L03750h}  \newcommand{\dateiname}{L03750}\newcommand{\titel}{Arthur Schnitzler an Stefan Zweig, 27. 7. 1923}\newcommand{\editorInnen}{Selma Jahnke und Martin Anton Müller}%% latex-leseansicht-abspann.tex
%% Abspann für die Leseansicht.
%% Der Schalter \ifkorrekturansicht ist bereits durch den Vorspann gesetzt.

%% latex-abspann.tex
%% Gemeinsamer Abspann für Korrekturansicht und Leseansicht.
%% Setzt den Schalter \ifkorrekturansicht voraus (gesetzt in den
%% einbindenden Dateien latex-korrekturansicht-abspann.tex bzw.
%% latex-leseansicht-abspann.tex).
%% ---------------------------------------------------------------

\normalsize

% Das esempio-Environment wird nur in der Leseansicht benötigt
\ifkorrekturansicht\else
\newenvironment{esempio}[3]%
{
    \vspace{1.5ex}
    \rlap{\underline{#1}}
    \par
    \setlength{\parindent}{0cm}
    \nopagebreak
    \leftskip=#2cm
    \rightskip=#3cm
}
{
    \par
}
\fi

\doendnotes{C}
\bigskip
\vfill

\clearpage

\footnotesize

\ifkorrekturansicht
  \lohead{\textsc{register}}
\fi

% theindex-Environment neu definieren ohne reledmac
\makeatletter
\renewenvironment{theindex}{%
  \ifkorrekturansicht
    \section*{\indexname}%
  \else
    \subsubsection*{Index der erwähnten Entitäten}%
  \fi
  \setlength{\parindent}{0pt}%
  \setlength{\parskip}{0pt plus 0.3pt}%
  \let\item\@idxitem
}{%
  \ifkorrekturansicht\clearpage\fi
}
\makeatother

\IfFileExists{\jobname-pw.ind}{\input{\jobname-pw.ind}}{}

% Quellenangabe nur in der Leseansicht
\ifkorrekturansicht\else
% Fallback-Definitionen, falls die .tex-Datei \titel etc. nicht gesetzt hat
\providecommand{\titel}{}
\providecommand{\editorInnen}{}
\providecommand{\dateiname}{\jobname}

\vspace{3cm}

\vfill

\footnotesize
\textsc{Quelle}: \titel. Herausgegeben von {\editorInnen}. In: \emph{Arthur Schnitzler: Briefwechsel mit Autorinnen und Autoren}.
 Digitale Edition, https://schnitzler-briefe.acdh.oeaw.ac.at/{\dateiname}.html (Stand \today)
\fi

\end{document}


