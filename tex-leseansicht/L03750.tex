%% latex-korrekturansicht-vorspann.tex
%% Vorspann für die Korrekturansicht.
%% Lädt die gemeinsame Datei latex-vorspann.tex mit gesetztem Schalter.

\newif\ifkorrekturansicht
\korrekturansichttrue

\input{../tex-inputs/latex-vorspann}


\section[Arthur Schnitzler an Stefan Zweig, 27. 7. 1923]{L03750 Arthur Schnitzler an Stefan Zweig, 27. 7. 1923}
\nopagebreak\mylabel{L03750v}
\rehead{ }\normalsize\beginnumbering\briefempfaengerindex{Zweig, Stefan@\textsc{Zweig, Stefan}!zzzSchnitzler, Arthur@\emph{von Arthur Schnitzler}!1923-07-271@{27. 7. 1923}|(be}
\toendnotes[C]{\smallbreak\pagebreak[2]}\Standort{Jerusalem, National Library of Israel, ARC. Ms. Var. 305 1 58 Stefan Zweig Collection.}
\physDesc{Postkarte, 1 Blatt, 2 Seiten, 825 Zeichen
\newline{}Handschrift: schwarze Tinte, lateinische Kurrent
\newline{}Versand: Stempel: »\nobreak{}\oindex{XVIII., Waehring@\textbf{XVIII., Währing}, \emph{A.ADM3}|pwk}18/\textsubscript{1} Wien
                                       110, 27. VII. 23, 15\nobreak{}«.  
\newline{}Zweig: mit Bleistift datiert: »27/VII 1923« }\toendnotes[C]{\smallbreak}\pstart{}{\pb}\label{T_L03750-1v}\edtext{\textcolor{gray}{\textbf{A. S.}}}{\lemma{\textnormal{\emph{A. S.}}}\Cendnote{\textnormal{ovaler Absenderkleber}}}\label{T_L03750-1}\pend{}\pstart{}\textcolor{gray}{\textbf{WIEN, XVIII.}}\oindex{XVIII., Waehring@\textbf{XVIII., Währing}, \emph{A.ADM3}|pw}\pend{}\pstart{}\textcolor{gray}{\textbf{STERNWARTESTR. 71}}\oindex{Sternwartestrasse 71@\textbf{Sternwartestraße 71}, \emph{Wohngebäude (K.WHS)}|pw}\pend{}{\bigskip}\pstart{}An\pend{}\pstart{}Hn. Dr. Stefan Zweig\pend{}\pstart{}Salzburg\oindex{Salzburg@\textbf{Salzburg}, \emph{A.ADM2}|pw}\pend{}\pstart{}Kapuzinerberg 5\oindex{Paschinger Schloessl@\textbf{Paschinger Schlössl}, \emph{Wohngebäude (K.WHS)}|pw}\pend{}{\bigskip}\vspace{1em}
\pstart
           \raggedleft{}{\pb}27. 7. 923\pend
           \vspace{0.5em}
\pstart
           lieber Herr Doctor, vielen vielen Dank! Wie Sie sehen bin ich noch
               (war wieder) in Wien\oindex{Wien@\textbf{Wien}, \emph{A.ADM2}|pw}, fahre voraussichtlich Ende
               nächster Woche nach Deutschland\oindex{Deutschland@\textbf{Deutschland}, \emph{A.PCLI}|pw} (Schwarzwald\oindex{Schwarzwald@\textbf{Schwarzwald}, \emph{Gebirge (N.GBR)}|pw}, \label{K_L03750-1v}\edtext{Baden Baden\oindex{Baden-Baden@\textbf{Baden-Baden}, \emph{P.PPL}|pw}}{\lemma{\textnormal{\emph{Baden Baden}}}\Cendnote{\textnormal{Schnitzler reiste am 3. 8. 1923 von Wien\oindex{Wien@\textbf{Wien}, \emph{A.ADM2}|pwk} nach Salzburg\oindex{Salzburg@\textbf{Salzburg}, \emph{A.ADM2}|pwk}, wo er im Österreichischen
                     Hof\oindex{Oesterreichischer Hof@\textbf{Österreichischer Hof}, \emph{Hotel (K.HTL)}|pwk} abstieg und zwei Nächte blieb. Über Stuttgart\oindex{Stuttgart@\textbf{Stuttgart}, \emph{P.PPLA}|pwk} (eine Übernachtung) reiste er dann nach Baden-Baden\oindex{Baden-Baden@\textbf{Baden-Baden}, \emph{P.PPL}|pwk}. Hier blieb er bis zum 14. 8. 1923, danach
                  folgte die Reise in die Schweiz\oindex{Schweiz@\textbf{Schweiz}, \emph{A.PCLI}|pwk}, vor allem
                  nach Celerina\oindex{Celerina@\textbf{Celerina}, \emph{P.PPL}|pwk}. Am 7. 9. 1923 reiste er
                  nach Vorarlberg\oindex{Vorarlberg@\textbf{Vorarlberg}, \emph{Teil eines Landes (A.LNDX)}|pwk}. Am 15. 9. 1923 kam er wieder
                  in Wien\oindex{Wien@\textbf{Wien}, \emph{A.ADM2}|pwk} an.}}}\label{K_L03750-1}, wo meine Kinder\pwindex{Cappellini, Lili 13.09.1909 – 26.07.1928@\textsc{Cappellini, Lili} (13.09.1909 – 26.07.1928)|pwv}\pwindex{Schnitzler, Heinrich 09.08.1902 – 12.07.1982@\textsc{Schnitzler, Heinrich} (09.08.1902 – 12.07.1982), \emph{Regisseur/Regisseurin, Schauspieler/Schauspielerin}|pwv} bei ihrer Mutter\pwindex{Schnitzler, Olga 17.01.1882 – 13.01.1970@\textsc{Schnitzler, Olga} (17.01.1882 – 13.01.1970), \emph{Schauspieler/Schauspielerin, Sänger/Sängerin}|pwv} sind) und in die Schweiz\oindex{Schweiz@\textbf{Schweiz}, \emph{A.PCLI}|pw}. Um R. R.\pwindex{Rolland, Romain 29.01.1866 – 30.12.1944@\textsc{Rolland, Romain} (29.01.1866 – 30.12.1944), \emph{Schriftsteller/Schriftstellerin}|pw}
               kennen zu lernen und Sie wiederzusehen, werd ich, we{\geminationn} nicht unvorhergesehene
               Hindernisse obwalten – mich gern auf der Durchreise in Salzburg\oindex{Salzburg@\textbf{Salzburg}, \emph{A.ADM2}|pw} aufhalten – ich denke, das wäre dann 3., ev.
                  4. od 5. August. Wohnen werd ich im oesterr. Hof.\oindex{Oesterreichischer Hof@\textbf{Österreichischer Hof}, \emph{Hotel (K.HTL)}|pw} – und Sie im jedem Fall vorher
               verständigen. (Oder raten Sie mir ein andres Hotel? Ist Europe\oindex{Grand Hotel de L Europe, G. Jung@\textbf{Grand Hotel de L’Europe, G. Jung}, \emph{Hotel (K.HTL)}|pw} erschwinglich – was bei kurzem Aufenthalt durch die Bahnhofnähe\oindex{Hauptbahnhof Salzburg@\textbf{Hauptbahnhof Salzburg}, \emph{Bahnhofsgebäude (K.BHF)}|pwv} verlockend
               wäre!).\pend
           
\pstart
           Empfehlen Sie mich Ihrer verehrten Gattin\pwindex{Zweig, Friderike Maria 1882-12-04 – 1971-01-18@\textsc{Zweig, Friderike Maria} (1882-12-04 – 1971-01-18), \emph{Schriftsteller/Schriftstellerin}|pwv} und seien Sie sehr herzlich gegrüßt, und nochmals allerwärmstens
               bedankt von Ihrem\pend
           \pstart \spacefill\mbox{Arthur Schnitzler}\pend{}\selectlanguage{ngerman}\endnumbering\briefempfaengerindex{Zweig, Stefan@\textsc{Zweig, Stefan}!zzzSchnitzler, Arthur@\emph{von Arthur Schnitzler}!1923-07-271@{27. 7. 1923}|)be}\mylabel{L03750h}
\begin{anhang}
\end{anhang}\normalsize

\doendnotes{C}
\bigskip
\vfill

\clearpage

\footnotesize

\lohead{\textsc{register}}

% Definiere theindex-Environment komplett neu ohne reledmac
\makeatletter
\renewenvironment{theindex}{%
  \section*{\indexname}%
  \setlength{\parindent}{0pt}%
  \setlength{\parskip}{0pt plus 0.3pt}%
  \let\item\@idxitem
}{%
  \clearpage
}
\makeatother

\IfFileExists{\jobname-pw.ind}{\input{\jobname-pw.ind}}{}

\end{document}

      