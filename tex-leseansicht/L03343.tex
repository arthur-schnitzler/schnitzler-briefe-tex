%% latex-leseansicht-vorspann.tex
%% Vorspann für die Leseansicht.
%% Lädt die gemeinsame Datei latex-vorspann.tex mit nicht gesetztem Schalter.

\newif\ifkorrekturansicht
\korrekturansichtfalse

\input{../tex-inputs/latex-vorspann}


\section[ Felix Salten an Arthur Schnitzler, 17. 9. 1903]{L03343 Felix Salten an Arthur Schnitzler,  17. 9. 1903}
\nopagebreak\mylabel{L03343v}
\rehead{ }\normalsize\beginnumbering\briefempfaengerindex{Schnitzler, Arthur@\textsc{Schnitzler, Arthur}!zzzSalten, Felix@\emph{von Felix Salten}!1903-09-172@{17. 9. 1903}|(be}
\toendnotes[C]{\smallbreak\pagebreak[2]}
\correspDesc{Versand  durch Felix Salten am 17. 9. 1903 in Wien
\newline{}Erhalt  durch Arthur Schnitzler im Zeitraum [17. 9. 1903
                  – 20. 9. 1903?] in Wien}\toendnotes[C]{\smallbreak}
\Standort{CUL, Schnitzler, B 89, A 2.}
\physDesc{Brief, 1 Blatt, 1 Seite, 408 Zeichen
\newline{}Handschrift: schwarze Tinte, lateinische Kurrent
\newline{}Ordnung: mit Bleistift von unbekannter Hand nummeriert: »168« }\toendnotes[C]{\smallbreak}
\pstart
           {\pb}\textcolor{gray}{\textbf{DIE}}\pend
           
\pstart
           \textcolor{gray}{\textbf{ZEIT\orgindex{Zeit@Die Zeit|pw}}}\hfill \textcolor{gray}{\textbf{\emph{WIEN}\oindex{Wien@\textbf{Wien}, \emph{Verwaltungsgebiet}|pw}}}{ }17. IX. 03\pend
           
\pstart
           \textcolor{gray}{\textbf{Wien\oindex{Wien@\textbf{Wien}, \emph{Verwaltungsgebiet}|pw}er Tageszeitung}}\hfill \textcolor{gray}{\textbf{\emph{I. Wipplingerstrasse 38\oindex{Wien@\textbf{Wien}!I., Innere Stadt@\textbf{I., Innere Stadt}!Wipplingerstraße@\textbf{Wipplingerstraße}, \emph{Straße}|pw}}}}\pend
           
\pstart
           \textcolor{gray}{\textbf{Herausgeber:}}\pend
           
\pstart
           \textcolor{gray}{\textbf{\textbf{Prof. Dr. I. Singer\pwindex{Singer, Isidor 16.\,1.\,1857 Budapest – 8.\,12.\,1927 Wien@\textsc{Singer, Isidor} (16.\,1.\,1857 Budapest – 8.\,12.\,1927 Wien), \emph{Journalist, Herausgeber, Soziologe}|pw}}}}\pend
           
\pstart
           \textcolor{gray}{\textbf{\textbf{Dr. Heinrich Kanner\pwindex{Kanner, Heinrich 9.\,11.\,1864 Galați – 15.\,2.\,1930 Wien@\textsc{Kanner, Heinrich} (9.\,11.\,1864 Galați – 15.\,2.\,1930 Wien), \emph{Herausgeber, Publizist}|pw}}}}\pend
           
\pstart
           \textcolor{gray}{\textbf{\textbf{Redaction}}}\pend
           
\pstart
           \textcolor{gray}{\textbf{Telegramm-Adresse: \so{Zeit}\orgindex{Zeit@Die Zeit|pw}\so{,}{ }\so{Wien}\oindex{Wien@\textbf{Wien}, \emph{Verwaltungsgebiet}|pw}}}\pend
           
\pstart
           \textcolor{gray}{\textbf{Interurbanes Telephon Nr. 15.988}}\pend
           
\pstart
           \textcolor{gray}{\textbf{= Telephone Nr. 17.040, 17.041 =}}\pend
           \vspace{0.5em}
\pstart
           Lieber, ich weiß nicht, ob Sie \label{K_L03343-1v}\edtext{noch, oder wieder in Wien\oindex{Wien@\textbf{Wien}, \emph{Verwaltungsgebiet}|pw}}{\lemma{\textnormal{\emph{noch, … Wien}}}\Cendnote{\textnormal{Schnitzler war in Wien\oindex{Wien@\textbf{Wien}, \emph{Verwaltungsgebiet}|pwk}.}}}\label{K_L03343-1} sind, und wundere mich natürlich, nichts von
               Ihnen zu hören. Otti\pwindex{Salten, Ottilie 7.\,3.\,1868 Prag – 22.\,6.\,1942 Zürich@\textsc{Salten, Ottilie} (7.\,3.\,1868 Prag – 22.\,6.\,1942 Zürich), \emph{Schauspielerin}|pw} ist noch immer nicht ganz
               wol und erholt sich nur langsam.\pend
           
\pstart
           Wenn Sie da sind, möchte ich Sie bald, in einer, die »Zeit\orgindex{Zeit@Die Zeit|pw}«
               betreffd. \label{K_L03343-2v}\edtext{Sache}{\lemma{\textnormal{\emph{Sache}}}\Cendnote{\textnormal{Siehe XXXX Auszeichnungsfehler: Dokument L03344 nicht gefunden.
               }}}\label{K_L03343-2} sprechen. Mit den schönsten Grüßen von uns Beiden\pwindex{Salten, Ottilie 7.\,3.\,1868 Prag – 22.\,6.\,1942 Zürich@\textsc{Salten, Ottilie} (7.\,3.\,1868 Prag – 22.\,6.\,1942 Zürich), \emph{Schauspielerin}|pwv} an Olga\pwindex{Schnitzler, Olga 17.\,1.\,1882 Wien – 13.\,1.\,1970 Lugano@\textsc{Schnitzler, Olga} (17.\,1.\,1882 Wien – 13.\,1.\,1970 Lugano), \emph{Schauspielerin, Sängerin}|pw}\pend
           
\pstart
           herzlich {\\[\baselineskip]}Ihr {\\[\baselineskip]}\spacefill\mbox{Salten}\pend
           \leftskip=0em{}
\pstart
           \noindent{}Ich weiß auch Ihre \label{K_L03343-3v}\edtext{neue Adreße}{\lemma{\textnormal{\emph{neue Adreße}}}\Cendnote{\textnormal{Seit 2. 9. 1903 wohnten Olga Schnitzler\pwindex{Schnitzler, Olga 17.\,1.\,1882 Wien – 13.\,1.\,1970 Lugano@\textsc{Schnitzler, Olga} (17.\,1.\,1882 Wien – 13.\,1.\,1970 Lugano), \emph{Schauspielerin, Sängerin}|pwk} und der Sohn Heinrich\pwindex{Schnitzler, Heinrich 9.\,8.\,1902 Hinterbrühl – 12.\,7.\,1982 Wien@\textsc{Schnitzler, Heinrich} (9.\,8.\,1902 Hinterbrühl – 12.\,7.\,1982 Wien), \emph{Regisseur, Schauspieler}|pwk} in der ersten gemeinsamen Wohnung in einem neu
                     errichteten Haus in der Spöttelgasse 7\oindex{Wien@\textbf{Wien}!XVIII., Währing@\textbf{XVIII., Währing}!Edmund-Weiß-Gasse 7@\textbf{Edmund-Weiß-Gasse 7}, \emph{Wohngebäude}|pwk}
                     (heute: Edmund-Weiß-Gasse\oindex{Wien@\textbf{Wien}!XVIII., Währing@\textbf{XVIII., Währing}!Edmund-Weiß-Gasse 7@\textbf{Edmund-Weiß-Gasse 7}, \emph{Wohngebäude}|pwk}) im 18. Wiener Gemeindebezirk\oindex{XVIII., Währing@\textbf{XVIII., Währing}, \emph{Verwaltungsgebiet}|pwk}. Seit 9. 9. 1903 wohnte Schnitzler ebenfalls an dieser Adresse.}}}\label{K_L03343-3} nicht, {\kaufmannsund} sende den Brief deshalb in die Franckgaße\oindex{Wien@\textbf{Wien}!IX., Alsergrund@\textbf{IX., Alsergrund}!Frankgasse 1@\textbf{Frankgasse 1}, \emph{Wohngebäude}|pw}.\pend
           \selectlanguage{ngerman}\endnumbering\briefempfaengerindex{Schnitzler, Arthur@\textsc{Schnitzler, Arthur}!zzzSalten, Felix@\emph{von Felix Salten}!1903-09-172@{17. 9. 1903}|)be}\mylabel{L03343h}  \newcommand{\dateiname}{L03343}\newcommand{\titel}{Felix Salten an Arthur Schnitzler, 17. 9. 1903}\newcommand{\editorInnen}{Martin Anton Müller und Laura Untner}%% latex-leseansicht-abspann.tex
%% Abspann für die Leseansicht.
%% Der Schalter \ifkorrekturansicht ist bereits durch den Vorspann gesetzt.

%% latex-abspann.tex
%% Gemeinsamer Abspann für Korrekturansicht und Leseansicht.
%% Setzt den Schalter \ifkorrekturansicht voraus (gesetzt in den
%% einbindenden Dateien latex-korrekturansicht-abspann.tex bzw.
%% latex-leseansicht-abspann.tex).
%% ---------------------------------------------------------------

\normalsize

% Das esempio-Environment wird nur in der Leseansicht benötigt
\ifkorrekturansicht\else
\newenvironment{esempio}[3]%
{
    \vspace{1.5ex}
    \rlap{\underline{#1}}
    \par
    \setlength{\parindent}{0cm}
    \nopagebreak
    \leftskip=#2cm
    \rightskip=#3cm
}
{
    \par
}
\fi

\doendnotes{C}
\bigskip
\vfill

\clearpage

\footnotesize

\ifkorrekturansicht
  \lohead{\textsc{register}}
\fi

% theindex-Environment neu definieren ohne reledmac
\makeatletter
\renewenvironment{theindex}{%
  \ifkorrekturansicht
    \section*{\indexname}%
  \else
    \subsubsection*{Index der erwähnten Entitäten}%
  \fi
  \setlength{\parindent}{0pt}%
  \setlength{\parskip}{0pt plus 0.3pt}%
  \let\item\@idxitem
}{%
  \ifkorrekturansicht\clearpage\fi
}
\makeatother

\IfFileExists{\jobname-pw.ind}{\input{\jobname-pw.ind}}{}

% Quellenangabe nur in der Leseansicht
\ifkorrekturansicht\else
% Fallback-Definitionen, falls die .tex-Datei \titel etc. nicht gesetzt hat
\providecommand{\titel}{}
\providecommand{\editorInnen}{}
\providecommand{\dateiname}{\jobname}

\vspace{3cm}

\vfill

\footnotesize
\textsc{Quelle}: \titel. Herausgegeben von {\editorInnen}. In: \emph{Arthur Schnitzler: Briefwechsel mit Autorinnen und Autoren}.
 Digitale Edition, https://schnitzler-briefe.acdh.oeaw.ac.at/{\dateiname}.html (Stand \today)
\fi

\end{document}


