%% latex-korrekturansicht-vorspann.tex
%% Vorspann für die Korrekturansicht.
%% Lädt die gemeinsame Datei latex-vorspann.tex mit gesetztem Schalter.

\newif\ifkorrekturansicht
\korrekturansichttrue

\input{../tex-inputs/latex-vorspann}


\section[Arthur Schnitzler an Richard Beer-Hofmann, 12. 6. 1897]{L00685 Arthur Schnitzler an Richard Beer-Hofmann, 12. 6. 1897}
\nopagebreak\mylabel{L00685v}
\rehead{ }\normalsize\beginnumbering\briefempfaengerindex{Beer-Hofmann, Richard@\textsc{Beer-Hofmann, Richard}!zzzSchnitzler, Arthur@\emph{von Arthur Schnitzler}!1897-06-121@{12. 6. 1897}|(be}
\toendnotes[C]{\smallbreak\pagebreak[2]}\Standort{YCGL, MSS 31.}
\physDesc{Brief, 2 Blätter, 8 Seiten, Umschlag, 1743 Zeichen
\newline{}Handschrift: 1) Bleistift, deutsche Kurrent\hspace{1em}2) Bleistift, lateinische Kurrent (\noindent{}Adresse)\hspace{1em}
\newline{}Versand: 1) Stempel: »\nobreak{}\oindex{IX., Alsergrund@\textbf{IX., Alsergrund}, \emph{A.ADM3}|pwk}Wien 9/3, 12. 6. 97, 5–6N\nobreak{}«.   2) Stempel: »\nobreak{}\oindex{Bad Ischl@\textbf{Bad Ischl}, \emph{P.PPL}|pwk}Ischl, 13. 6. 97, 7–8V\nobreak{}«. }
\buchAbdrucke{\weitereDrucke{Arthur Schnitzler, Richard Beer-Hofmann: \emph{Briefwechsel 1891–1931}. Wien, Zürich: \emph{Europaverlag} 1992, S. 108–109.} }\toendnotes[C]{\smallbreak}\pstart{}{\pb}Dr. Richard Beer-Hofmann\pend{}\pstart{}Ischl\oindex{Bad Ischl@\textbf{Bad Ischl}, \emph{P.PPL}|pw}\pend{}\pstart{}Egelmoos 22\oindex{Eglmoosgasse@\textbf{Eglmoosgasse}, \emph{Bezirk (A.BZK)}|pw}\pend{}\pstart{}Ober-Oesterreich\oindex{Oberoesterreich@\textbf{Oberösterreich}, \emph{A.ADM1}|pw}\pend{}{\bigskip}\vspace{1em}
\pstart
           \raggedleft{}{\pb}12. 6. 97\pend
           \vspace{0.5em}
\pstart
           Mein lieber Richard. Ich danke ſehr für Ihre Bemühung bei \textsc{Leopold}\oindex{Hotel und Pension Rudolfshoehe (Leopold Petter)@\textbf{Hotel und Pension Rudolfshöhe (Leopold Petter)}, \emph{Hotel (K.HTL)}|pw}. Wahrſcheinlich ko{\geminationm} ich früher, ſo gegen
                  27, 28. Bitte ſchaun Sie ſich da{\geminationn} im Vorüberradeln das Zi{\geminationm}er an, ob nicht alles wackelt, was in dieſem Wirtshaus {\pb}immer vorauszuſetzen iſt. Notwendig ein großer Tiſch
               (zum Schreiben.) Da meine Mama\pwindex{Schnitzler, Louise 1840-07-08 – 1911-09-09@\textsc{Schnitzler, Louise} (1840-07-08 – 1911-09-09)|pwv} eine kleine Couſine, Grethel\pwindex{Manassewitsch, Margarethe 6.11.1880 – 21.09.1940@\textsc{Manasséwitsch, Margarethe} (6.11.1880 – 21.09.1940)|pw},
               zur Begleitg hat, brauch ich gar nicht nah von ihr zu ſein. –\pend
           
\pstart
           Nun, wegen \textsc{Bayreuth}\oindex{Bayreuth@\textbf{Bayreuth}, \emph{P.PPLA2}|pw}, da müſſen Sie ſich rasch {\pb}entſchließen, aber
               nicht gleich Nein ſagen, weil es raſch ſein muſs. \textsc{Parsifal}\pwindex{Parsifal@\emph{Parsifal}|pw} iſt am 27., 28. und 30. \uline{Juli}{ }ſoweit es für mich in Betracht kommt. \uline{Ein}{ }Sitz 12 Gulden. Ich habe auch an Paul\pwindex{Goldmann, Paul 31.01.1865 – 25.09.1935@\textsc{Goldmann, Paul} (31.01.1865 – 25.09.1935), \emph{Schriftsteller/Schriftstellerin, Journalist/Journalistin}|pw} geſchrieben. Soll ich ei{\pb}nen Sitz für Sie nehmen? Am liebſten 28.
               Man bringt ihn auch i{\geminationm}er wieder los, da ein großes
               Geriſs iſt; alſo riskirt iſt nicht viel. Überhaupt! 12 Gulden – Zwei Gulden – und
               noch vier – – Und noch ſechs – Man {\pb}hält es und hat
               vier achter gegen vier zehner, da iſt doch die \textsc{Parsifal\pwindex{Parsifal@\emph{Parsifal}|pw}-Chance} eher werth. –\pend
           
\pstart
           – Ich ſpiele mich mit einem Komödienplan\pwindex{Weg ins Freie. Roman@\emph{Der Weg ins Freie. Roman}|pwv} herum {\dotsfour} aber ich fang nicht an,
               bevor die Sache von der 1. bis zur letzten Scene abſolut feſtſteht und alle {\pb}Perſonen zu einander eine wirkliche ſowohl äußerliche
               als innerliche Beziehung haben. Ich habe keine Luſt, wieder ein Stück zu ſchreiben,
               wo man Perſonen nach Belieben entfernen und dazu thun kann. – Freiwild\pwindex{Freiwild. Schauspiel in 3 Akten@\emph{Freiwild. Schauspiel in 3 Akten}|pw} in Prag\oindex{Prag@\textbf{Prag}, \emph{A.ADM1}|pw} frei{\pb}gegeben – für den Fall, daſs Bayern\oindex{Bayern@\textbf{Bayern}, \emph{A.ADM1}|pw}. Man räth mir ſehr, besonders Gustav Schwk\pwindex{Schwarzkopf, Gustav 07.11.1853 – 13.11.1939@\textsc{Schwarzkopf, Gustav} (07.11.1853 – 13.11.1939), \emph{Schriftsteller/Schriftstellerin}|pw}. Habe noch nicht geantwortet. –\pend
           
\pstart
           – Ängſtigt Sie’s »mit ahnungsvoller
                  Gegenwart\pwindex{Faust. Eine Tragoedie@\emph{Faust. Eine Tragödie}|pwv}«? – Ich ſpüre noch garnichts. –\pend
           
\pstart
           Ich freu mich ſehr auf Sie. We{\geminationn}{ }{\pb}Sie »\textsc{\so{fesch}}« ſind, ſo ko{\geminationm}en Sie mir nach Lambach\oindex{Lambach@\textbf{Lambach}, \emph{A.ADM3}|pw}, oder, billiger, nach Gmunden\oindex{Gmunden@\textbf{Gmunden}, \emph{P.PPL}|pw} entgegen auf dem Rad und wir fahren zuſa{\geminationm}en u. ſ. w.\pend
           
\pstart
           Antworten Sie mir gleich.\pend
           
\pstart
           Herzlich Ihr{\\}\spacefill\mbox{Arthur.}\pend
           \selectlanguage{ngerman}\endnumbering\briefempfaengerindex{Beer-Hofmann, Richard@\textsc{Beer-Hofmann, Richard}!zzzSchnitzler, Arthur@\emph{von Arthur Schnitzler}!1897-06-121@{12. 6. 1897}|)be}\mylabel{L00685h}  \normalsize

\doendnotes{C}
\bigskip
\vfill

\clearpage

\footnotesize

\lohead{\textsc{register}}

% Definiere theindex-Environment komplett neu ohne reledmac
\makeatletter
\renewenvironment{theindex}{%
  \section*{\indexname}%
  \setlength{\parindent}{0pt}%
  \setlength{\parskip}{0pt plus 0.3pt}%
  \let\item\@idxitem
}{%
  \clearpage
}
\makeatother

\IfFileExists{\jobname-pw.ind}{\input{\jobname-pw.ind}}{}

\end{document}

      