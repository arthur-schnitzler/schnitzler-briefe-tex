%% latex-leseansicht-vorspann.tex
%% Vorspann für die Leseansicht.
%% Lädt die gemeinsame Datei latex-vorspann.tex mit nicht gesetztem Schalter.

\newif\ifkorrekturansicht
\korrekturansichtfalse

\input{../tex-inputs/latex-vorspann}


               \section[Arthur Schnitzler an Richard Beer-Hofmann, 12. 6. 1897]{ Arthur Schnitzler an Richard Beer-Hofmann,
               12. 6. 1897}\nopagebreak\mylabel{v}\rehead{ }\begin{ledgroupsized}[t]{13cm}\normalsize\beginnumbering\briefempfaengerindex{Beer-Hofmann, Richard@\textsc{Beer-Hofmann, Richard}!zzzSchnitzler, Arthur@\emph{von Arthur Schnitzler}!1897-06-121@{12. 6. 1897}|(be} \toendnotes[C]{\smallbreak\pagebreak[2]} \Standort{YCGL, MSS 31.}
\physDesc{Brief, 2 Blätter, 8 Seiten, Umschlag
\newline{}Handschrift: Bleistift, deutsche Kurrent\newline{}Versand: 1) Stempel: »\nobreak{}\oindex{IX., Alsergrund@\textbf{IX., Alsergrund}|pwk}Wien 9/3, 12. 6. 97, 5–6N\nobreak{}«.  2) Stempel: »\nobreak{}\oindex{Bad Ischl@\textbf{Bad Ischl}|pwk}Ischl, 13. 6. 97, 7–8V\nobreak{}«. }\buchAbdrucke{\weitereDrucke{Arthur Schnitzler, Richard Beer-Hofmann: \emph{Briefwechsel 1891–1931}. Hg. Konstanze Fliedl. Wien, Zürich: \emph{Europaverlag} 1992, S. 108–109.} }\toendnotes[C]{\smallbreak}\pstart{}{\pb}\textsc{Dr. Richard Beer-Hofmann}\pend{}\pstart{}\textsc{Ischl\oindex{Bad Ischl@\textbf{Bad Ischl}|pw}}\pend{}\pstart{}\textsc{Egelmoos 22}\oindex{Eglmoosgasse@\textbf{Eglmoosgasse}|pw}\pend{}\pstart{}\textsc{Ober-Oesterreich}\oindex{Oberoesterreich@\textbf{Oberösterreich}|pw}\pend{}{\bigskip}\pstart
           \raggedleft{}{\pb}12. 6. 97\pend
           \pstart
           Mein lieber Richard. Ich danke ſehr für Ihre Bemühung bei \textsc{Leopold}\oindex{Hotel und Pension Rudolfshoehe (Leopold Petter)@\textbf{Hotel und Pension Rudolfshöhe (Leopold Petter)}|pw}. Wahrſcheinlich ko{\geminationm} ich früher, ſo gegen
                  27, 28. Bitte ſchaun Sie ſich da{\geminationn} im Vorüberradeln das Zi{\geminationm}er an, ob nicht alles wackelt, was in dieſem Wirtshaus {\pb}immer vorauszuſetzen iſt. Notwendig ein großer
               Tiſch (zum Schreiben.) Da meine Mama\pwindex{Schnitzler, Louise 08.07.1840 – 09.09.1911@\textsc{Schnitzler, Louise} (08.07.1840 – 09.09.1911)|pwv} eine kleine Couſine, Grethel\pwindex{Manassewitsch, Margarethe 6.11.1880 – 21.09.1940@\textsc{Manassewitsch, Margarethe} (6.11.1880 – 21.09.1940)|pw}, zur
               Begleitg hat, brauch ich gar nicht nah von ihr zu ſein. –\pend
           \pstart
           Nun, wegen \textsc{Bayreuth}\oindex{Bayreuth@\textbf{Bayreuth}|pw}, da müſſen Sie ſich rasch {\pb}entſchließen,
               aber nicht gleich Nein ſagen, weil es raſch ſein muſs. \textsc{Parsifal}\pwindex{\textcolor{red}{\textsuperscript{XXXX1 indx}}!Parsifal1882@\strich\emph{Parsifal} {[}1882{]}|pw} iſt am 27., 28. und 30. \uline{Juli}{ }ſoweit es für mich in Betracht kommt. \uline{Ein}{ }Sitz 12 Gulden. Ich habe auch an Paul\pwindex{Goldmann, Paul 31.01.1865 – 25.09.1935@\textsc{Goldmann, Paul} (31.01.1865 – 25.09.1935), \emph{Schriftsteller, Journalist}|pw}
               geſchrieben. Soll ich ei{\pb}nen Sitz für Sie nehmen?
               Am liebſten 28. Man bringt ihn auch i{\geminationm}er
               wieder los, da ein großes Geriſs iſt; alſo riskirt iſt nicht viel. Überhaupt!
               12 Gulden – Zwei Gulden – und noch vier – – Und noch ſechs – Man {\pb}hält es und hat vier achter gegen vier zehner, da
               iſt doch die \textsc{Parsifal\pwindex{\textcolor{red}{\textsuperscript{XXXX1 indx}}!Parsifal1882@\strich\emph{Parsifal} {[}1882{]}|pw}-Chance} eher werth. –\pend
           \pstart
           – Ich ſpiele mich mit einem Komödienplan\pwindex{Schnitzler, Arthur 15.05.1862 – 21.10.1931@\textsc{Schnitzler, Arthur} (15.05.1862 – 21.10.1931), \emph{Schriftsteller, Mediziner}!Weg ins Freie. Roman1.1.1908 – 1.6.1908@\strich\emph{Der Weg ins Freie. Roman} {[}1.1.1908 – 1.6.1908{]}|pwv} herum {\dotsfour} aber ich fang nicht an,
               bevor die Sache von der 1. bis zur letzten Scene abſolut feſtſteht und alle {\pb}Perſonen zu einander eine wirkliche ſowohl
               äußerliche als innerliche Beziehung haben. Ich habe keine Luſt, wieder ein Stück zu
               ſchreiben, wo man Perſonen nach Belieben entfernen und dazu thun kann. – Freiwild\pwindex{Schnitzler, Arthur 15.05.1862 – 21.10.1931@\textsc{Schnitzler, Arthur} (15.05.1862 – 21.10.1931), \emph{Schriftsteller, Mediziner}!Freiwild. Schauspiel in 3 Akten1896@\strich\emph{Freiwild. Schauspiel in 3 Akten} {[}1896{]}|pw} in Prag\oindex{Prag@\textbf{Prag}|pw}
                  frei{\pb}gegeben – für den Fall, daſs Bayern\oindex{Bayern@\textbf{Bayern}|pw}. Man räth mir ſehr, besonders Gustav Schwk\pwindex{Schwarzkopf, Gustav 07.11.1853 – 13.11.1939@\textsc{Schwarzkopf, Gustav} (07.11.1853 – 13.11.1939), \emph{Schriftsteller}|pw}. Habe noch nicht geantwortet. –\pend
           \pstart
           – Ängſtigt Sie’s »mit ahnungsvoller
                  Gegenwart\pwindex{\textcolor{red}{\textsuperscript{XXXX1 indx}}!Faust1790 – 1832@\strich\emph{Faust} {[}1790 – 1832{]}|pwv}«? – Ich ſpüre noch garnichts. –\pend
           \pstart
           Ich freu mich ſehr auf Sie. We{\geminationn}{ }{\pb}Sie »\textsc{\so{fesch}}« ſind, ſo ko{\geminationm}en
               Sie mir nach Lambach\oindex{Lambach@\textbf{Lambach}|pw}, oder, billiger, nach Gmunden\oindex{Gmunden@\textbf{Gmunden}|pw} entgegen auf dem Rad und wir fahren zuſa{\geminationm}en u. ſ. w.\pend
           \pstart
           Antworten Sie mir gleich.\pend
           \pstart
           Herzlich Ihr{\\}\spacefill\mbox{Arthur.}\pend
                     \endnumbering\briefempfaengerindex{Beer-Hofmann, Richard@\textsc{Beer-Hofmann, Richard}!zzzSchnitzler, Arthur@\emph{von Arthur Schnitzler}!1897-06-121@{12. 6. 1897}|)be}\mylabel{h}\end{ledgroupsized}  \newcommand{\dateiname}{L00685}\newcommand{\titel}{Arthur Schnitzler an Richard Beer-Hofmann, 12. 6. 1897}\newcommand{\editorInnen}{Martin Anton Müller und Gerd-Hermann Susen}%% latex-leseansicht-abspann.tex
%% Abspann für die Leseansicht.
%% Der Schalter \ifkorrekturansicht ist bereits durch den Vorspann gesetzt.

%% latex-abspann.tex
%% Gemeinsamer Abspann für Korrekturansicht und Leseansicht.
%% Setzt den Schalter \ifkorrekturansicht voraus (gesetzt in den
%% einbindenden Dateien latex-korrekturansicht-abspann.tex bzw.
%% latex-leseansicht-abspann.tex).
%% ---------------------------------------------------------------

\normalsize

% Das esempio-Environment wird nur in der Leseansicht benötigt
\ifkorrekturansicht\else
\newenvironment{esempio}[3]%
{
    \vspace{1.5ex}
    \rlap{\underline{#1}}
    \par
    \setlength{\parindent}{0cm}
    \nopagebreak
    \leftskip=#2cm
    \rightskip=#3cm
}
{
    \par
}
\fi

\doendnotes{C}
\bigskip
\vfill

\clearpage

\footnotesize

\ifkorrekturansicht
  \lohead{\textsc{register}}
\fi

% theindex-Environment neu definieren ohne reledmac
\makeatletter
\renewenvironment{theindex}{%
  \ifkorrekturansicht
    \section*{\indexname}%
  \else
    \subsubsection*{Index der erwähnten Entitäten}%
  \fi
  \setlength{\parindent}{0pt}%
  \setlength{\parskip}{0pt plus 0.3pt}%
  \let\item\@idxitem
}{%
  \ifkorrekturansicht\clearpage\fi
}
\makeatother

\IfFileExists{\jobname-pw.ind}{\input{\jobname-pw.ind}}{}

% Quellenangabe nur in der Leseansicht
\ifkorrekturansicht\else
% Fallback-Definitionen, falls die .tex-Datei \titel etc. nicht gesetzt hat
\providecommand{\titel}{}
\providecommand{\editorInnen}{}
\providecommand{\dateiname}{\jobname}

\vspace{3cm}

\vfill

\footnotesize
\textsc{Quelle}: \titel. Herausgegeben von {\editorInnen}. In: \emph{Arthur Schnitzler: Briefwechsel mit Autorinnen und Autoren}.
 Digitale Edition, https://schnitzler-briefe.acdh.oeaw.ac.at/{\dateiname}.html (Stand \today)
\fi

\end{document}


      