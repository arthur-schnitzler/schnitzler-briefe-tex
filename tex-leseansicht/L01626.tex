%% latex-korrekturansicht-vorspann.tex
%% Vorspann für die Korrekturansicht.
%% Lädt die gemeinsame Datei latex-vorspann.tex mit gesetztem Schalter.

\newif\ifkorrekturansicht
\korrekturansichttrue

\input{../tex-inputs/latex-vorspann}


\section[Hugo von Hofmannsthal an Arthur Schnitzler, 10. 9. {[}1906{]}]{L01626 Hugo von Hofmannsthal an Arthur Schnitzler, 10. 9. {[}1906{]}}
\nopagebreak\mylabel{L01626v}
\rehead{ }\normalsize\beginnumbering\briefempfaengerindex{Schnitzler, Arthur@\textsc{Schnitzler, Arthur}!zzzHofmannsthal, Hugo von@\emph{von Hugo von Hofmannsthal}!1906-09-101@{10. 9. {[}1906{]}}|(be}
\toendnotes[C]{\smallbreak\pagebreak[2]}\Standort{CUL, Schnitzler, B 43.}
\physDesc{Telegramm, 112 Zeichen
\newline{}maschinell
\newline{}Versand: Stempel der Telegrafenbeamtin: »M. Baumgartner\pwindex{Baumgartner, Marie 11.9.1906 – 11.9.1906@\textsc{Baumgartner, Marie} (11.9.1906 – 11.9.1906), \emph{Postbeamter/Postbeamtin}|pw}« 
\newline{}Schnitzler: mit Bleistift datiert »11/9 906« 
\newline{}Ordnung: 1) beschnitten  2) mit Bleistift von unbekannter Hand nummeriert:
                                    »265«}
\buchAbdrucke{\weitereDrucke{Hugo von Hofmannsthal, Arthur Schnitzler: \emph{Briefwechsel}. Frankfurt am Main: \emph{S. Fischer} 1964, S. 223.} }
\pstart
           {\pb}win\oindex{Wien@\textbf{Wien}, \emph{A.ADM2}|pw} fr st.
                     gilgen\oindex{St. Gilgen@\textbf{St. Gilgen}, \emph{A.ADM3}|pw} 336 17 10/9{ }11v\pend
           \vspace{0.5em}
\pstart
           ware in naechster zeit paar tage zusammensein semmering\oindex{Semmering@\textbf{Semmering}, \emph{A.ADM3}|pw} oder anderswo moeglich =\pend
           \pstart \spacefill\mbox{hugo.}\pend{}\selectlanguage{ngerman}\endnumbering\briefempfaengerindex{Schnitzler, Arthur@\textsc{Schnitzler, Arthur}!zzzHofmannsthal, Hugo von@\emph{von Hugo von Hofmannsthal}!1906-09-101@{10. 9. {[}1906{]}}|)be}\mylabel{L01626h}  \normalsize

\doendnotes{C}
\bigskip
\vfill

\clearpage

\footnotesize

\lohead{\textsc{register}}

% Definiere theindex-Environment komplett neu ohne reledmac
\makeatletter
\renewenvironment{theindex}{%
  \section*{\indexname}%
  \setlength{\parindent}{0pt}%
  \setlength{\parskip}{0pt plus 0.3pt}%
  \let\item\@idxitem
}{%
  \clearpage
}
\makeatother

\IfFileExists{\jobname-pw.ind}{\input{\jobname-pw.ind}}{}

\end{document}

      