%% latex-korrekturansicht-vorspann.tex
%% Vorspann für die Korrekturansicht.
%% Lädt die gemeinsame Datei latex-vorspann.tex mit gesetztem Schalter.

\newif\ifkorrekturansicht
\korrekturansichttrue

\input{../tex-inputs/latex-vorspann}


\section[Arthur Schnitzler an Wilhelm Bölsche, {[}Anfang September{]} 1890]{L00005 Arthur Schnitzler an Wilhelm Bölsche, {[}Anfang September{]} 1890}
\nopagebreak\mylabel{L00005v}
\rehead{ }\normalsize\beginnumbering\briefempfaengerindex{Boelsche, Wilhelm@\textsc{Bölsche, Wilhelm}!zzzSchnitzler, Arthur@\emph{von Arthur Schnitzler}!1890-09-011@{{[}Anfang September{]} 1890}|(be}
\toendnotes[C]{\smallbreak\pagebreak[2]}\Standort{Wrocław, Biblioteka Uniwersytecka, Böl.Pis 1773.}
\physDesc{Brief, 1 Blatt, 2 Seiten, 580 Zeichen
\newline{}Handschrift: schwarze Tinte, deutsche Kurrent
\newline{}Bölsche: als »Erledigt« gezeichnet }
\buchAbdrucke{\weitereDrucke{1) \emph{Germanica Wratislaviensia} (1987) Nr. 77, S. 465–466.} \weitereDrucke{2) Wilhelm Bölsche: \emph{Briefwechsel. Mit Autoren der Freien Bühne}. Berlin: \emph{Weidler} 2010, S. 667.} }\toendnotes[C]{\smallbreak}
\pstart\center{}{\pb}Sehr geehrter Herr Redakteur!\pend\vspace{0.5em}
\pstart
           Erlauben Sie mir, Ihnen beifolgende \label{K_L00005-1v}\edtext{Skizze\pwindex{Aus der Kaffeehausecke@\emph{Aus der Kaffeehausecke}|pwv}}{\lemma{\textnormal{\emph{Skizze}}}\Cendnote{\textnormal{\emph{Aus der Kaffeehausecke}\pwindex{Aus der Kaffeehausecke@\emph{Aus der Kaffeehausecke}|pwk}; Schnitzler hat sie am
                     3. 2. 1890 und
                  unmittelbar vor diesem Brief, am 29. 8. 1890, abgefasst und dann wohl gleich an Bölsche\pwindex{Boelsche, Wilhelm 02.01.1861 – 31.08.1939@\textsc{Bölsche, Wilhelm} (02.01.1861 – 31.08.1939), \emph{Schriftsteller/Schriftstellerin, Publizist/Publizistin}|pwk} geschickt. Die Skizze blieb zu
                  Lebzeiten unpubliziert.}}}\label{K_L00005-1} vorzulegen. Sie iſt raſch geleſen; ich fürchte
               kaum, Sie allzuſehr in Anſpruch zu nehmen. Vielleicht finden Sie, daß ſie ſich dem
               Rahmen Ihrer \textsc{Freien Bühne für modernes Leben}\pwindex{Freie Buehne fuer modernes Leben@\emph{Freie Bühne für modernes Leben}|pw} ohne allzu ſchli{\geminationm}en Zwang einfügen ließe – in
               dieſem Falle würde ich Sie höflichſt um Veröffentlichung derſelben erſuchen. Misfällt
               ſie Ihnen, ſehr geehrter Herr, {\pb}\damage{ha}ben Sie wohl die Güte, das kleine Heft an meine Adreſſe zurückzuſenden.\pend
           
\pstart
           Ich bin mit ausgezeichneter Hochachtung{\\[\baselineskip]}Ihr ergebner{\\[\baselineskip]}\spacefill\mbox{Dr.  med. Arthur Schnitzler}\pend
           \leftskip=0em{}
\pstart
           \noindent{}\textsc{Wien, I. Giselastraße 11\oindex{Ordination Arthur Schnitzler [Boesendorferstrasse 11]@\textbf{Ordination Arthur Schnitzler [Bösendorferstraße 11]}, \emph{Ordination}|pw}.}\pend
           \selectlanguage{ngerman}\endnumbering\briefempfaengerindex{Boelsche, Wilhelm@\textsc{Bölsche, Wilhelm}!zzzSchnitzler, Arthur@\emph{von Arthur Schnitzler}!1890-09-011@{{[}Anfang September{]} 1890}|)be}\mylabel{L00005h}  \normalsize

\doendnotes{C}
\bigskip
\vfill

\clearpage

\footnotesize

\lohead{\textsc{register}}

% Definiere theindex-Environment komplett neu ohne reledmac
\makeatletter
\renewenvironment{theindex}{%
  \section*{\indexname}%
  \setlength{\parindent}{0pt}%
  \setlength{\parskip}{0pt plus 0.3pt}%
  \let\item\@idxitem
}{%
  \clearpage
}
\makeatother

\IfFileExists{\jobname-pw.ind}{\input{\jobname-pw.ind}}{}

\end{document}

      