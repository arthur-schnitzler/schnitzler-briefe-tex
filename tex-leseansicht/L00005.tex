%% latex-leseansicht-vorspann.tex
%% Vorspann für die Leseansicht.
%% Lädt die gemeinsame Datei latex-vorspann.tex mit nicht gesetztem Schalter.

\newif\ifkorrekturansicht
\korrekturansichtfalse

\input{../tex-inputs/latex-vorspann}


               \section[Arthur Schnitzler an Wilhelm Bölsche, {[}Anfang September{]} 1890]{ Arthur Schnitzler an Wilhelm Bölsche, {[}Anfang September{]} 1890}\nopagebreak\mylabel{v}\rehead{ }\begin{ledgroupsized}[t]{13cm}\normalsize\beginnumbering\briefempfaengerindex{Boelsche, Wilhelm@\textsc{Bölsche, Wilhelm}!zzzSchnitzler, Arthur@\emph{von Arthur Schnitzler}!1890-09-011@{{[}Anfang September{]} 1890}|(be} \toendnotes[C]{\smallbreak\pagebreak[2]} \Standort{Wrocław, Biblioteka Uniwersytecka, Böl.Pis 1773.}
\physDesc{Brief, 1 Blatt, 2 Seiten
\newline{}Handschrift: schwarze Tinte, deutsche Kurrent
\newline{}Bölsche: als »Erledigt« gezeichnet }\buchAbdrucke{\weitereDrucke{1) Alois Woldan: \emph{Arthur Schnitzler – Briefe an Wilhelm Bölsche.} In: \emph{Germanica Wratislaviensia} (1987) Nr. 77, S. 465–466.} \weitereDrucke{2) Wilhelm Bölsche: \emph{Briefwechsel. Mit Autoren der Freien Bühne}. Hg. Gerd-Hermann Susen. Berlin: \emph{Weidler} 2010, S. 667 (Werke und Briefe. Wissenschaftliche Ausgabe, Briefe I).} }\toendnotes[C]{\smallbreak}\pstart\center{}{\pb}Sehr geehrter Herr Redakteur!\pend\pstart
           Erlauben Sie mir, Ihnen beifolgende \label{K_L00005_1v}\edtext{Skizze\pwindex{Schnitzler, Arthur 15.05.1862 – 21.10.1931@\textsc{Schnitzler, Arthur} (15.05.1862 – 21.10.1931), \emph{Schriftsteller, Mediziner}!Aus der Kaffeehausecke1977@\strich\emph{Aus der Kaffeehausecke} {[}1977{]}|pwv}}{\lemma{\textnormal{\emph{Skizze}}}\Cendnote{\textnormal{\emph{Aus der Kaffeehausecke}\pwindex{Schnitzler, Arthur 15.05.1862 – 21.10.1931@\textsc{Schnitzler, Arthur} (15.05.1862 – 21.10.1931), \emph{Schriftsteller, Mediziner}!Aus der Kaffeehausecke1977@\strich\emph{Aus der Kaffeehausecke} {[}1977{]}|pwk}; Schnitzler hat sie am
                     3. 2. 1890 und unmittelbar vor diesem Brief, am
                  29. 8. 1890, abgefasst und dann wohl gleich an Bölsche\pwindex{Boelsche, Wilhelm 02.01.1861 – 31.08.1939@\textsc{Bölsche, Wilhelm} (02.01.1861 – 31.08.1939), \emph{Schriftsteller, Publizist}|pwk} geschickt. Die Skizze blieb zu Lebzeiten
                  unpubliziert.}}}\label{K_L00005_1h} vorzulegen. Sie iſt raſch geleſen; ich fürchte kaum, Sie
               allzuſehr in Anſpruch zu nehmen. Vielleicht finden Sie, daß ſie ſich dem Rahmen Ihrer
                  \textsc{Freien Bühne für modernes Leben}\pwindex{Freie Buehne fuer modernes Leben1890 – 1891@\emph{Freie Bühne für modernes Leben}|pw} ohne allzu ſchli{\geminationm}en Zwang einfügen ließe – in
               dieſem Falle würde ich Sie höflichſt um Veröffentlichung derſelben erſuchen. Misfällt
               ſie Ihnen, ſehr geehrter Herr, {\pb}\damage{ha}ben Sie wohl die Güte, das kleine Heft an meine Adreſſe zurückzuſenden.\pend
           \pstart
           Ich bin mit ausgezeichneter Hochachtung{\\[\baselineskip]}Ihr ergebner{\\[\baselineskip]}\spacefill\mbox{Dr.  med. Arthur Schnitzler}\pend
           \leftskip=0em{}\pstart
           \noindent{}\textsc{Wien, I. Giselastraße 11\oindex{Boesendorferstrasse@\textbf{Bösendorferstraße}|pw}.}\pend
           \endnumbering\briefempfaengerindex{Boelsche, Wilhelm@\textsc{Bölsche, Wilhelm}!zzzSchnitzler, Arthur@\emph{von Arthur Schnitzler}!1890-09-011@{{[}Anfang September{]} 1890}|)be}\mylabel{h}\end{ledgroupsized}  \newcommand{\dateiname}{L00005}\newcommand{\titel}{Arthur Schnitzler an Wilhelm Bölsche, [Anfang September] 1890}\newcommand{\editorInnen}{Martin Anton Müller und Gerd-Hermann Susen}
            \footnotesize
\begin{ledgroupsized}[t]{11.5cm}
\doendnotes{C}
\end{ledgroupsized}
         %% latex-leseansicht-abspann.tex
%% Abspann für die Leseansicht.
%% Der Schalter \ifkorrekturansicht ist bereits durch den Vorspann gesetzt.

%% latex-abspann.tex
%% Gemeinsamer Abspann für Korrekturansicht und Leseansicht.
%% Setzt den Schalter \ifkorrekturansicht voraus (gesetzt in den
%% einbindenden Dateien latex-korrekturansicht-abspann.tex bzw.
%% latex-leseansicht-abspann.tex).
%% ---------------------------------------------------------------

\normalsize

% Das esempio-Environment wird nur in der Leseansicht benötigt
\ifkorrekturansicht\else
\newenvironment{esempio}[3]%
{
    \vspace{1.5ex}
    \rlap{\underline{#1}}
    \par
    \setlength{\parindent}{0cm}
    \nopagebreak
    \leftskip=#2cm
    \rightskip=#3cm
}
{
    \par
}
\fi

\doendnotes{C}
\bigskip
\vfill

\clearpage

\footnotesize

\ifkorrekturansicht
  \lohead{\textsc{register}}
\fi

% theindex-Environment neu definieren ohne reledmac
\makeatletter
\renewenvironment{theindex}{%
  \ifkorrekturansicht
    \section*{\indexname}%
  \else
    \subsubsection*{Index der erwähnten Entitäten}%
  \fi
  \setlength{\parindent}{0pt}%
  \setlength{\parskip}{0pt plus 0.3pt}%
  \let\item\@idxitem
}{%
  \ifkorrekturansicht\clearpage\fi
}
\makeatother

\IfFileExists{\jobname-pw.ind}{\input{\jobname-pw.ind}}{}

% Quellenangabe nur in der Leseansicht
\ifkorrekturansicht\else
% Fallback-Definitionen, falls die .tex-Datei \titel etc. nicht gesetzt hat
\providecommand{\titel}{}
\providecommand{\editorInnen}{}
\providecommand{\dateiname}{\jobname}

\vspace{3cm}

\vfill

\footnotesize
\textsc{Quelle}: \titel. Herausgegeben von {\editorInnen}. In: \emph{Arthur Schnitzler: Briefwechsel mit Autorinnen und Autoren}.
 Digitale Edition, https://schnitzler-briefe.acdh.oeaw.ac.at/{\dateiname}.html (Stand \today)
\fi

\end{document}


      