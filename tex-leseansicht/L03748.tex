%% latex-leseansicht-vorspann.tex
%% Vorspann für die Leseansicht.
%% Lädt die gemeinsame Datei latex-vorspann.tex mit nicht gesetztem Schalter.

\newif\ifkorrekturansicht
\korrekturansichtfalse

\input{../tex-inputs/latex-vorspann}


\section[Arthur Schnitzler an Stefan Zweig, 21. 2. 1927]{L03748 Arthur Schnitzler an Stefan Zweig, 21. 2. 1927}
\nopagebreak\mylabel{L03748v}
\rehead{ }\normalsize\beginnumbering\briefempfaengerindex{Zweig, Stefan@\textsc{Zweig, Stefan}!zzzSchnitzler, Arthur@\emph{von Arthur Schnitzler}!1927-02-211@{21. 2. 1927}|(be}
\toendnotes[C]{\smallbreak\pagebreak[2]}
\correspDesc{Versand  durch Arthur Schnitzler am 21. 2. 1927 in Wien
\newline{}Übermittlung  am 22. 2. 1927 in Wien
\newline{}Erhalt  durch Stefan Zweig im Zeitraum [23. 2. 1927
                  – 25. 2. 1927?] in Salzburg}\toendnotes[C]{\smallbreak}
\Standort{Jerusalem, National Library of Israel, ARC. Ms. Var. 305 1 58 Stefan Zweig Collection.}
\physDesc{Postkarte, 815 Zeichen
\newline{}Handschrift: schwarze Tinte, lateinische Kurrent
\newline{}Versand: Stempel: »\nobreak{}\oindex{XVIII., Währing@\textbf{XVIII., Währing}, \emph{Verwaltungsgebiet}|pwk}18/\textsubscript{1} Wien
                                       110, 22. II. 27, \textcolor{gray}{9}\nobreak{}«.  }\toendnotes[C]{\smallbreak}\pstart{}{\pb}\label{T_L03748-1v}\edtext{\textcolor{gray}{\textbf{A. S.}}}{\lemma{\textnormal{\emph{A. S.}}}\Cendnote{\textnormal{ovaler Absenderkleber}}}\label{T_L03748-1}\pend{}\pstart{}\textcolor{gray}{\textbf{WIEN, XVIII.}}\oindex{XVIII., Währing@\textbf{XVIII., Währing}, \emph{Verwaltungsgebiet}|pw}\pend{}\pstart{}\textcolor{gray}{\textbf{STERNWARTESTR. 71}}\oindex{Wien@\textbf{Wien}!XVIII., Währing@\textbf{XVIII., Währing}!Sternwartestraße 71@\textbf{Sternwartestraße 71}, \emph{Wohngebäude}|pw}\pend{}{\bigskip}\pstart{}Hrn Doctor Stefan Zweig\pend{}\pstart{}Salzburg\oindex{Salzburg@\textbf{Salzburg}, \emph{Verwaltungsgebiet}|pw}. \pend{}\pstart{}Kapuzinerberg 5\oindex{Paschinger Schlössl@\textbf{Paschinger Schlössl}, \emph{Wohngebäude}|pw}.\pend{}{\bigskip}\vspace{1em}
\pstart
           \raggedleft{}{\pb}Wien\oindex{Wien@\textbf{Wien}, \emph{Verwaltungsgebiet}|pw}, 21. 2. 927\pend
           \vspace{0.5em}
\pstart
           lieber und verehrter Herr Doctor, für Ihre guten und schönen Worte
               anläßlich meiner diagra{\geminationm}atischen Versuche\pwindex{Schnitzler, Arthur 15.\,5.\,1862 Wien – 21.\,10.\,1931 ebd.@\textsc{Schnitzler, Arthur} (15.\,5.\,1862 Wien – 21.\,10.\,1931 ebd.), \emph{Schriftsteller, Mediziner}!Geist im Wort und der Geist in der Tat@\strich\emph{Der Geist im Wort und der Geist in der Tat}|pwv} dank ich Ihnen herzlichst.
               Dieser Dank reist Ihnen wohl schon in den Süden nach, wo die sich in der Arbeit und
               den wahrhaft verdienten Erfolgen dieses letzten Jahres erholen und zu neuen rüsten
               werden. Ich habe indess den \label{K_L03748-1v}\edtext{Volpone\pwindex{Zweig, Stefan 28.\,11.\,1881 Wien – 23.\,2.\,1942 Petrópolis@\textsc{Zweig, Stefan} (28.\,11.\,1881 Wien – 23.\,2.\,1942 Petrópolis), \emph{Schriftsteller}!Ben Jonsons »Volpone«.  Eine lieblose Komödie in drei Akten@\strich\emph{Ben Jonsons »Volpone«. Eine lieblose Komödie in drei Akten}|pw} auch in Berlin\oindex{Berlin@\textbf{Berlin}, \emph{Hauptstadt}|pw}}{\lemma{\textnormal{\emph{Volpone auch in Berlin}}}\Cendnote{\textnormal{Schnitzler besuchte die Theateraufführung von \emph{Ben
                        Jonsons »Volpone«}\pwindex{Zweig, Stefan 28.\,11.\,1881 Wien – 23.\,2.\,1942 Petrópolis@\textsc{Zweig, Stefan} (28.\,11.\,1881 Wien – 23.\,2.\,1942 Petrópolis), \emph{Schriftsteller}!Ben Jonsons »Volpone«.  Eine lieblose Komödie in drei Akten@\strich\emph{Ben Jonsons »Volpone«. Eine lieblose Komödie in drei Akten}|pwk}\eventindex{Freie Volksbühne@\textbf{Freie Volksbühne}!Aufführung von Volpone, 30.12.1926@Aufführung von Volpone, 30.12.1926|pwk} in der Bearbeitung von Zweig\pwindex{Zweig, Stefan 28.\,11.\,1881 Wien – 23.\,2.\,1942 Petrópolis@\textsc{Zweig, Stefan} (28.\,11.\,1881 Wien – 23.\,2.\,1942 Petrópolis), \emph{Schriftsteller}|pwk} am 30. 12. 1926 in der Freien Volksbühne\oindex{Freie Volksbühne@\textbf{Freie Volksbühne}, \emph{Theater}|pwk}. }}}\label{K_L03748-1} gesehen, in einer Vorstellung, die trotz Steinrücks\pwindex{Steinrück, Albert 20.\,5.\,1872 Wetterburg – 11.\,2.\,1929 Berlin@\textsc{Steinrück, Albert} (20.\,5.\,1872 Wetterburg – 11.\,2.\,1929 Berlin), \emph{Schauspieler}|pw}{[},{]} im ganzen ungleich roher war als \label{K_L03748-2v}\edtext{die im Wiener Burgtheater\orgindex{Burgtheater@Burgtheater|pw}}{\lemma{\textnormal{\emph{die … Burgtheater}}}\Cendnote{\textnormal{Schnitzler sah die Generalprobe von \emph{Volpone}\pwindex{Zweig, Stefan 28.\,11.\,1881 Wien – 23.\,2.\,1942 Petrópolis@\textsc{Zweig, Stefan} (28.\,11.\,1881 Wien – 23.\,2.\,1942 Petrópolis), \emph{Schriftsteller}!Ben Jonsons »Volpone«.  Eine lieblose Komödie in drei Akten@\strich\emph{Ben Jonsons »Volpone«. Eine lieblose Komödie in drei Akten}|pwk}\eventindex{Burgtheater@\textbf{Burgtheater}!Generalprobe von Volpone oder Der Fuchs, 5.11.1926@Generalprobe von Volpone oder Der Fuchs, 5.11.1926|pwk} am 5. 11. 1926 im
                     Burgtheater\oindex{Wien@\textbf{Wien}!I., Innere Stadt@\textbf{I., Innere Stadt}!Burgtheater@\textbf{Burgtheater}, \emph{Theater}|pwk}. }}}\label{K_L03748-2} aber von starker
               Wirkung. Ich freue mich Ihren nächsten Werken entgegen und hoffe wir sprechen bald
                wieder miteinander – es muſs ja nicht gerade in \label{K_L03748-55v}\edtext{1600 Meter Höhe}{\lemma{\textnormal{\emph{1600 Meter Höhe}}}\Cendnote{\textnormal{Anspielung auf das gemeinsame Treffen in Zermatt\oindex{Zermatt@\textbf{Zermatt}|pwk} am 19. 8. 1926 und am Folgetag.}}}\label{K_L03748-55} sein. Kapuzinerberg\oindex{Kapuzinerberg@\textbf{Kapuzinerberg}, \emph{Berg}|pw} oder Sternwartestraße\oindex{Wien@\textbf{Wien}!XVIII., Währing@\textbf{XVIII., Währing}!Sternwartestraße@\textbf{Sternwartestraße}, \emph{Straße}|pw} werden auch zur Noth genügen\pend
           \pstart {\pb}Auf Wiedersehen also, und alles herzliche von Ihrem
                  \spacefill\mbox{Arthur Schnitzler}\pend{}\selectlanguage{ngerman}\endnumbering\briefempfaengerindex{Zweig, Stefan@\textsc{Zweig, Stefan}!zzzSchnitzler, Arthur@\emph{von Arthur Schnitzler}!1927-02-211@{21. 2. 1927}|)be}\mylabel{L03748h}  \newcommand{\dateiname}{L03748}\newcommand{\titel}{Arthur Schnitzler an Stefan Zweig, 21. 2. 1927}\newcommand{\editorInnen}{Selma Jahnke und Martin Anton Müller}%% latex-leseansicht-abspann.tex
%% Abspann für die Leseansicht.
%% Der Schalter \ifkorrekturansicht ist bereits durch den Vorspann gesetzt.

%% latex-abspann.tex
%% Gemeinsamer Abspann für Korrekturansicht und Leseansicht.
%% Setzt den Schalter \ifkorrekturansicht voraus (gesetzt in den
%% einbindenden Dateien latex-korrekturansicht-abspann.tex bzw.
%% latex-leseansicht-abspann.tex).
%% ---------------------------------------------------------------

\normalsize

% Das esempio-Environment wird nur in der Leseansicht benötigt
\ifkorrekturansicht\else
\newenvironment{esempio}[3]%
{
    \vspace{1.5ex}
    \rlap{\underline{#1}}
    \par
    \setlength{\parindent}{0cm}
    \nopagebreak
    \leftskip=#2cm
    \rightskip=#3cm
}
{
    \par
}
\fi

\doendnotes{C}
\bigskip
\vfill

\clearpage

\footnotesize

\ifkorrekturansicht
  \lohead{\textsc{register}}
\fi

% theindex-Environment neu definieren ohne reledmac
\makeatletter
\renewenvironment{theindex}{%
  \ifkorrekturansicht
    \section*{\indexname}%
  \else
    \subsubsection*{Index der erwähnten Entitäten}%
  \fi
  \setlength{\parindent}{0pt}%
  \setlength{\parskip}{0pt plus 0.3pt}%
  \let\item\@idxitem
}{%
  \ifkorrekturansicht\clearpage\fi
}
\makeatother

\IfFileExists{\jobname-pw.ind}{\input{\jobname-pw.ind}}{}

% Quellenangabe nur in der Leseansicht
\ifkorrekturansicht\else
% Fallback-Definitionen, falls die .tex-Datei \titel etc. nicht gesetzt hat
\providecommand{\titel}{}
\providecommand{\editorInnen}{}
\providecommand{\dateiname}{\jobname}

\vspace{3cm}

\vfill

\footnotesize
\textsc{Quelle}: \titel. Herausgegeben von {\editorInnen}. In: \emph{Arthur Schnitzler: Briefwechsel mit Autorinnen und Autoren}.
 Digitale Edition, https://schnitzler-briefe.acdh.oeaw.ac.at/{\dateiname}.html (Stand \today)
\fi

\end{document}


