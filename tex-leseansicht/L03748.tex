%% latex-korrekturansicht-vorspann.tex
%% Vorspann für die Korrekturansicht.
%% Lädt die gemeinsame Datei latex-vorspann.tex mit gesetztem Schalter.

\newif\ifkorrekturansicht
\korrekturansichttrue

\input{../tex-inputs/latex-vorspann}


\section[Arthur Schnitzler an Stefan Zweig, 21. 2. 1927]{L03748 Arthur Schnitzler an Stefan Zweig, 21. 2. 1927}
\nopagebreak\mylabel{L03748v}
\rehead{ }\normalsize\beginnumbering\briefempfaengerindex{Zweig, Stefan@\textsc{Zweig, Stefan}!zzzSchnitzler, Arthur@\emph{von Arthur Schnitzler}!1927-02-211@{21. 2. 1927}|(be}
\toendnotes[C]{\smallbreak\pagebreak[2]}\Standort{Jerusalem, National Library of Israel, ARC. Ms. Var. 305 1 58 Stefan Zweig Collection.}
\physDesc{Postkarte, 1 Blatt, 2 Seiten, 806 Zeichen
\newline{}Handschrift: schwarze Tinte, lateinische Kurrent
\newline{}Versand: Stempel: »\nobreak{}\oindex{XVIII., Waehring@\textbf{XVIII., Währing}, \emph{A.ADM3}|pwk}18/\textsubscript{1} Wien 110, 22. II. 27, 5\nobreak{}«.  }\toendnotes[C]{\smallbreak}\pstart{}{\pb}\label{T_L03748-1v}\edtext{\textcolor{gray}{\textbf{A. S.}}}{\lemma{\textnormal{\emph{A. S.}}}\Cendnote{\textnormal{ovaler Absenderkleber}}}\label{T_L03748-1}\pend{}\pstart{}\textcolor{gray}{\textbf{WIEN, XVIII.}}\oindex{XVIII., Waehring@\textbf{XVIII., Währing}, \emph{A.ADM3}|pw}\pend{}\pstart{}\textcolor{gray}{\textbf{STERNWARTESTR. 71}}\oindex{Sternwartestrasse 71@\textbf{Sternwartestraße 71}, \emph{Wohngebäude (K.WHS)}|pw}\pend{}{\bigskip}\pstart{}Hrn Stefan Zweig\pend{}\pstart{}Salzburg\oindex{Salzburg@\textbf{Salzburg}, \emph{A.ADM2}|pw}\pend{}\pstart{}Kapuzinerberg 5\oindex{Paschinger Schloessl@\textbf{Paschinger Schlössl}, \emph{Wohngebäude (K.WHS)}|pw}.\pend{}{\bigskip}\vspace{1em}
\pstart
           \raggedleft{}{\pb}Wien\oindex{Wien@\textbf{Wien}, \emph{A.ADM2}|pw},
                        21. 2. 927\pend
           \vspace{0.5em}
\pstart
           lieber und verehrter Herr Doctor, für Ihre guten und schönen Worte
               anläßlich meiner diagra{\geminationm}atischen
                  Versuche\pwindex{Geist im Wort und der Geist in der Tat@\emph{Der Geist im Wort und der Geist in der Tat}|pwv} dank ich Ihnen herzlichst. Dieser Dank reist Ihnen wohl schon in den
               Süden nach, wo die sich in der Arbeit und den wahrhaft verdienten Erfolgen dieses
               letzten Jahres erholen und zu neuen rüsten werden. Ich habe indess den \label{K_L03748-1v}\edtext{Volpone\pwindex{Ben Jonsons »Volpone«  Eine lieblose Komoedie in drei Akten@\emph{Ben Jonsons »Volpone« Eine lieblose Komödie in drei Akten}|pw} auch in Berlin\oindex{Berlin@\textbf{Berlin}, \emph{P.PPLC}|pw}}{\lemma{\textnormal{\emph{Volpone auch in Berlin}}}\Cendnote{\textnormal{Schnitzler besuchte die \emph{Theateraufführung von \emph{Ben Jonsons »Volpone« Eine lieblose Komödie}\pwindex{Ben Jonsons »Volpone«  Eine lieblose Komoedie in drei Akten@\emph{Ben Jonsons »Volpone« Eine lieblose Komödie in drei Akten}|pwk}}\eventindex{Freie Volksbuehne@\textbf{Freie Volksbühne}!Auffuehrung von Volpone, 30.12.1926@Aufführung von Volpone, 30.12.1926|pwk} in der Bearbeitung von Zweig\pwindex{Zweig, Stefan 28.11.1881 – 23.02.1942@\textsc{Zweig, Stefan} (28.11.1881 – 23.02.1942), \emph{Schriftsteller/Schriftstellerin}|pwk} am 30. 12. 1926 in der Freien Volksbühne\oindex{Freie Volksbuehne@\textbf{Freie Volksbühne}, \emph{Theater (K.THE)}|pwk}.
               }}}\label{K_L03748-1} gesehen, in einer Vorstellung, die
               trotz Steinrücks\pwindex{Steinrueck, Albert 20.05.1872 – 11.02.1929@\textsc{Steinrück, Albert} (20.05.1872 – 11.02.1929), \emph{Schauspieler/Schauspielerin}|pw}{[},{]} im
               ganzen ungleich roher war als \label{K_L03748-2v}\edtext{die im
               Wiener Burgtheater\orgindex{Burgtheater@Burgtheater|pw}}{\lemma{\textnormal{\emph{die … Burgtheater}}}\Cendnote{\textnormal{Schnitzler sah die \emph{Generalprobe von \emph{Volpone}\pwindex{Ben Jonsons »Volpone«  Eine lieblose Komoedie in drei Akten@\emph{Ben Jonsons »Volpone« Eine lieblose Komödie in drei Akten}|pwk}}\eventindex{Burgtheater@\textbf{Burgtheater}!Generalprobe von Volpone oder Der Fuchs, 5.11.1926@Generalprobe von Volpone oder Der Fuchs, 5.11.1926|pwk} am 5. 11. 1926 im Burgtheater\oindex{Burgtheater@\textbf{Burgtheater}, \emph{S.THTR}|pwk}.
               }}}\label{K_L03748-2} aber von starker Wirkung\pend
           
\pstart
           Ich freue mich Ihren
               nächsten Werken entgegen und hoffe wir sprechen bald  wieder miteinander – es muſs ja
               nicht gerade in 1600 Meter Höhe sein. Kapuzinerberg\oindex{Kapuzinerberg@\textbf{Kapuzinerberg}, \emph{T.MT}|pw} die Sternwartestraße\oindex{Sternwartestrasse@\textbf{Sternwartestraße}, \emph{R.ST}|pw} werden auch
               zur Noth genügen\pend
           \pstart {\pb}Auf Wiedersehen also, und alles herzliche von Ihrem
                  \spacefill\mbox{Arthur Schnitzler}\pend{}\selectlanguage{ngerman}\endnumbering\briefempfaengerindex{Zweig, Stefan@\textsc{Zweig, Stefan}!zzzSchnitzler, Arthur@\emph{von Arthur Schnitzler}!1927-02-211@{21. 2. 1927}|)be}\mylabel{L03748h}
\begin{anhang}
\end{anhang}\normalsize

\doendnotes{C}
\bigskip
\vfill

\clearpage

\footnotesize

\lohead{\textsc{register}}

% Definiere theindex-Environment komplett neu ohne reledmac
\makeatletter
\renewenvironment{theindex}{%
  \section*{\indexname}%
  \setlength{\parindent}{0pt}%
  \setlength{\parskip}{0pt plus 0.3pt}%
  \let\item\@idxitem
}{%
  \clearpage
}
\makeatother

\IfFileExists{\jobname-pw.ind}{\input{\jobname-pw.ind}}{}

\end{document}

      