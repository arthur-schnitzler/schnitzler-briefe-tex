%% latex-leseansicht-vorspann.tex
%% Vorspann für die Leseansicht.
%% Lädt die gemeinsame Datei latex-vorspann.tex mit nicht gesetztem Schalter.

\newif\ifkorrekturansicht
\korrekturansichtfalse

\input{../tex-inputs/latex-vorspann}


         
         \renewcommand{\erwaehntePersonen}{Personen: Karl Kraus, Johann Schnitzler}
         \renewcommand{\erwaehnteInstitutionen}{Institutionen: Wiener Abendpost, Wiener Zeitung}
         \renewcommand{\erwaehnteOrte}{Orte: Wien}
         \renewcommand{\erwaehnteWerke}{Werke: Regierungsrath Professor Schnitzler †}
               \section[Karl Kraus an Arthur Schnitzler, 2. 5. 1893]{ Karl Kraus an Arthur Schnitzler, 2. 5. 1893}\nopagebreak\mylabel{v}\rehead{ }\begin{ledgroupsized}[t]{13cm}\normalsize\beginnumbering\briefempfaengerindex{Schnitzler, Arthur@\textsc{Schnitzler, Arthur}!zzzKraus, Karl@\emph{von Karl Kraus}!1893-05-021@{2. 5. 1895}|(be} \toendnotes[C]{\smallbreak\pagebreak[2]} \Standort{DLA, A:Schnitzler, HS.NZ85.1.3790, S. 11.}
\physDesc{Brief, maschinenschriftliche Abschrift, 1 Blatt, 1 Seite, 272 Zeichen
\newline{}Schreibmaschine
\newline{}Handschrift: Bleistift, deutsche Kurrent (\noindent{}eine Korrektur)}\buchAbdrucke{\weitereDrucke{\emph{Karl Kraus und Arthur Schnitzler. Eine Dokumentation.} Hg. Reinhard Urbach. In: \emph{Literatur und Kritik}, Bd. 49, Oktober 1970, S. 518.} }\toendnotes[C]{\smallbreak}\pstart
           \raggedleft{}{\pb}Wien\oindex{Wien@\textbf{Wien}|pw}, 2. Mai 1893. \pend
           \pstart
           \label{K_L00206-1v}\edtext{Eben lese ich}{\lemma{\textnormal{\emph{Eben lese ich}}}\Cendnote{\textnormal{Die \emph{Wiener
                     Zeitung}\orgindex{Wiener Zeitung@Wiener Zeitung|pwk} brachte bereits wenige Stunden nach Johann Schnitzler\pwindex{Schnitzler, Johann 10.04.1835 – 02.05.1893@\textsc{Schnitzler, Johann} (10.04.1835 – 02.05.1893), \emph{Laryngologe}|pwk}s Tod in ihrer Abendausgabe \emph{Wiener Abendpost}\orgindex{Wiener Abendpost@Wiener Abendpost|pwk}, Nr. 100 vom 2. 5. 1893,
                  S. 3, eine nicht gezeichnete, kurze Todesmeldung: »\emph{Regierungsrath Professor Schnitzler †}\pwindex{?? Werk@Nicht ermittelte Verfasserinnen und Verfasser!Regierungsrath Professor Schnitzler †2. 5. 1893@\emph{Regierungsrath Professor Schnitzler †} {[}2. 5. 1893{]}|pwk}.«}}}\label{K_L00206-1h}, hochverehrter Herr Doctor, von dem schmerzlichen Ereignisse\pwindex{Schnitzler, Johann 10.04.1835 – 02.05.1893@\textsc{Schnitzler, Johann} (10.04.1835 – 02.05.1893), \emph{Laryngologe}|pwv} in Ihrer werten Familie. Nehmen Sie, verehrter,
               liebster Herr Doctor, die Versicherung meiner \uline{herzlichsten, innigsten Antheilnahme}! Ich bin mit hochachtungsvollem Grusse
               Ihr treuer\pend
           \pstart \spacefill\mbox{K. K.}\pend{}
         
         \endnumbering\mylabel{h}\end{ledgroupsized}  \newcommand{\dateiname}{L00206}\newcommand{\titel}{Karl Kraus an Arthur Schnitzler, 2. 5. 1893}\newcommand{\editorInnen}{Martin Anton Müller und Gerd-Hermann Susen}%% latex-leseansicht-abspann.tex
%% Abspann für die Leseansicht.
%% Der Schalter \ifkorrekturansicht ist bereits durch den Vorspann gesetzt.

%% latex-abspann.tex
%% Gemeinsamer Abspann für Korrekturansicht und Leseansicht.
%% Setzt den Schalter \ifkorrekturansicht voraus (gesetzt in den
%% einbindenden Dateien latex-korrekturansicht-abspann.tex bzw.
%% latex-leseansicht-abspann.tex).
%% ---------------------------------------------------------------

\normalsize

% Das esempio-Environment wird nur in der Leseansicht benötigt
\ifkorrekturansicht\else
\newenvironment{esempio}[3]%
{
    \vspace{1.5ex}
    \rlap{\underline{#1}}
    \par
    \setlength{\parindent}{0cm}
    \nopagebreak
    \leftskip=#2cm
    \rightskip=#3cm
}
{
    \par
}
\fi

\doendnotes{C}
\bigskip
\vfill

\clearpage

\footnotesize

\ifkorrekturansicht
  \lohead{\textsc{register}}
\fi

% theindex-Environment neu definieren ohne reledmac
\makeatletter
\renewenvironment{theindex}{%
  \ifkorrekturansicht
    \section*{\indexname}%
  \else
    \subsubsection*{Index der erwähnten Entitäten}%
  \fi
  \setlength{\parindent}{0pt}%
  \setlength{\parskip}{0pt plus 0.3pt}%
  \let\item\@idxitem
}{%
  \ifkorrekturansicht\clearpage\fi
}
\makeatother

\IfFileExists{\jobname-pw.ind}{\input{\jobname-pw.ind}}{}

% Quellenangabe nur in der Leseansicht
\ifkorrekturansicht\else
% Fallback-Definitionen, falls die .tex-Datei \titel etc. nicht gesetzt hat
\providecommand{\titel}{}
\providecommand{\editorInnen}{}
\providecommand{\dateiname}{\jobname}

\vspace{3cm}

\vfill

\footnotesize
\textsc{Quelle}: \titel. Herausgegeben von {\editorInnen}. In: \emph{Arthur Schnitzler: Briefwechsel mit Autorinnen und Autoren}.
 Digitale Edition, https://schnitzler-briefe.acdh.oeaw.ac.at/{\dateiname}.html (Stand \today)
\fi

\end{document}


      