%% latex-korrekturansicht-vorspann.tex
%% Vorspann für die Korrekturansicht.
%% Lädt die gemeinsame Datei latex-vorspann.tex mit gesetztem Schalter.

\newif\ifkorrekturansicht
\korrekturansichttrue

\input{../tex-inputs/latex-vorspann}


\section[Karl Kraus an Arthur Schnitzler, 2. 5. 1893]{L00206 Karl Kraus an Arthur Schnitzler, 2. 5. 1893}
\nopagebreak\mylabel{L00206v}
\rehead{ }\normalsize\beginnumbering\briefempfaengerindex{Schnitzler, Arthur@\textsc{Schnitzler, Arthur}!zzzKraus, Karl@\emph{von Karl Kraus}!1893-05-021@{2. 5. 1895}|(be}
\toendnotes[C]{\smallbreak\pagebreak[2]}\Standort{DLA, A:Schnitzler, HS.NZ85.1.3790, S. 11.}
\physDesc{Brief, maschinenschriftliche Abschrift1 Blatt, 1 Seite, 272 Zeichen
\newline{}Schreibmaschine
\newline{}Handschrift: Bleistift, deutsche Kurrent (\noindent{}eine Korrektur)}
\buchAbdrucke{\weitereDrucke{\emph{Literatur und Kritik}, Bd. 49, Oktober 1970, S. 518.} }\toendnotes[C]{\smallbreak}
\pstart
           \raggedleft{}{\pb}Wien\oindex{Wien@\textbf{Wien}, \emph{A.ADM2}|pw}, 2. Mai 1893. \pend
           \vspace{0.5em}
\pstart
           \label{K_L00206-1v}\edtext{Eben lese ich}{\lemma{\textnormal{\emph{Eben lese ich}}}\Cendnote{\textnormal{Die \emph{Wiener
                     Zeitung}\orgindex{Wiener Zeitung@Wiener Zeitung|pwk} brachte bereits wenige Stunden nach Johann Schnitzlers\pwindex{Schnitzler, Johann 10.04.1835 – 02.05.1893@\textsc{Schnitzler, Johann} (10.04.1835 – 02.05.1893), \emph{Laryngologe/Laryngologin}|pwk} Tod in ihrer Abendausgabe \emph{Wiener Abendpost}\orgindex{Wiener Abendpost@Wiener Abendpost|pwk}, Nr. 100 vom 2. 5. 1893,
                  S. 3, eine nicht gezeichnete, kurze Todesmeldung: »\emph{Regierungsrath Professor Schnitzler †}\pwindex{Regierungsrath Professor Schnitzler †@\emph{Regierungsrath Professor Schnitzler †}|pwk}.«}}}\label{K_L00206-1}, hochverehrter Herr Doctor, von dem schmerzlichen Ereignisse\pwindex{Schnitzler, Johann 10.04.1835 – 02.05.1893@\textsc{Schnitzler, Johann} (10.04.1835 – 02.05.1893), \emph{Laryngologe/Laryngologin}|pwv} in Ihrer werten Familie. Nehmen Sie, verehrter,
               liebster Herr Doctor, die Versicherung meiner \uline{herzlichsten, innigsten Antheilnahme}! Ich bin mit hochachtungsvollem Grusse
               Ihr treuer\pend
           \pstart \spacefill\mbox{K. K.}\pend{}\selectlanguage{ngerman}\endnumbering\briefempfaengerindex{Schnitzler, Arthur@\textsc{Schnitzler, Arthur}!zzzKraus, Karl@\emph{von Karl Kraus}!1893-05-021@{2. 5. 1895}|)be}\mylabel{L00206h}  \normalsize

\doendnotes{C}
\bigskip
\vfill

\clearpage

\footnotesize

\lohead{\textsc{register}}

% Definiere theindex-Environment komplett neu ohne reledmac
\makeatletter
\renewenvironment{theindex}{%
  \section*{\indexname}%
  \setlength{\parindent}{0pt}%
  \setlength{\parskip}{0pt plus 0.3pt}%
  \let\item\@idxitem
}{%
  \clearpage
}
\makeatother

\IfFileExists{\jobname-pw.ind}{\input{\jobname-pw.ind}}{}

\end{document}

      