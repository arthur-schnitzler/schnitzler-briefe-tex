%% latex-korrekturansicht-vorspann.tex
%% Vorspann für die Korrekturansicht.
%% Lädt die gemeinsame Datei latex-vorspann.tex mit gesetztem Schalter.

\newif\ifkorrekturansicht
\korrekturansichttrue

\input{../tex-inputs/latex-vorspann}


\section[Michael Georg Conrad an Arthur Schnitzler, 26. 5. 1906]{L01599 Michael Georg Conrad an Arthur Schnitzler, 26. 5. 1906}
\nopagebreak\mylabel{L01599v}
\rehead{ }\normalsize\beginnumbering\briefempfaengerindex{Schnitzler, Arthur@\textsc{Schnitzler, Arthur}!zzzConrad, Michael Georg@\emph{von Michael Georg Conrad}!1906-05-262@{26. 5. 1906}|(be}
\toendnotes[C]{\smallbreak\pagebreak[2]}\Standort{CUL, Schnitzler, B 22.}
\physDesc{Briefkarte, 315 Zeichen
\newline{}Handschrift: schwarze Tinte, deutsche Kurrent}
\pstart
           {\pb}München\oindex{Muenchen@\textbf{München}, \emph{P.PPLA}|pw}, Steinsdorfſtr. 7\oindex{Steinsdorfstrasse@\textbf{Steinsdorfstraße}, \emph{Straße (K.STR)}|pw}\pend
           
\pstart
           \raggedleft{}26. 5. 06.\pend
           \vspace{0.5em}
\pstart
           Ach, lieber Arthur Schnitzler, Sie haben mich auch ruhig ſechzig
               w\textcolor{gray}{e}r\textcolor{gray}{d}en laſſen, um mich antelegraphiren zu können. Macht Ihnen denn ſo was Freude? Mir
               nicht. Ich warne Sie, bandeln Sie nicht mit dem Altwerden an, ſagen Sie’s wenigſtens
               keiner Seele.\pend
           
\pstart
           Ihr gewitzigter Jubelgreis{\\[\baselineskip]}\spacefill\mbox{Conrad.}\pend
           \leftskip=0em{}\selectlanguage{ngerman}\endnumbering\briefempfaengerindex{Schnitzler, Arthur@\textsc{Schnitzler, Arthur}!zzzConrad, Michael Georg@\emph{von Michael Georg Conrad}!1906-05-262@{26. 5. 1906}|)be}\mylabel{L01599h}  \normalsize

\doendnotes{C}
\bigskip
\vfill

\clearpage

\footnotesize

\lohead{\textsc{register}}

% Definiere theindex-Environment komplett neu ohne reledmac
\makeatletter
\renewenvironment{theindex}{%
  \section*{\indexname}%
  \setlength{\parindent}{0pt}%
  \setlength{\parskip}{0pt plus 0.3pt}%
  \let\item\@idxitem
}{%
  \clearpage
}
\makeatother

\IfFileExists{\jobname-pw.ind}{\input{\jobname-pw.ind}}{}

\end{document}

      