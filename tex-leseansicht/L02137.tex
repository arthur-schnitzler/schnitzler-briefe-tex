%% latex-korrekturansicht-vorspann.tex
%% Vorspann für die Korrekturansicht.
%% Lädt die gemeinsame Datei latex-vorspann.tex mit gesetztem Schalter.

\newif\ifkorrekturansicht
\korrekturansichttrue

\input{../tex-inputs/latex-vorspann}


\section[Arthur Schnitzler an Robert Adam, 13. 5. 1913]{L02137 Arthur Schnitzler an Robert Adam, 13. 5. 1913}
\nopagebreak\mylabel{L02137v}
\rehead{ }\normalsize\beginnumbering\briefempfaengerindex{Adam, Robert@\textsc{Adam, Robert}!zzzSchnitzler, Arthur@\emph{von Arthur Schnitzler}!1913-05-131@{13. 5. 1913}|(be}
\toendnotes[C]{\smallbreak\pagebreak[2]}\Standort{DLA, 96.34.1/11.}
\physDesc{Brief, 1 Blatt, 2 Seiten, Umschlag, 1326 Zeichen
\newline{}Schreibmaschine
\newline{}Handschrift: schwarze Tinte, lateinische Kurrent (\noindent{}Korrekturen, Unterschrift)
\newline{}Versand: Stempel: »\nobreak{}13. {[}5.{]} 13\nobreak{}«.  }\Standort{DLA, A:Schnitzler, 85.1.1621.}
\physDesc{Brief, Durchschlag1 Blatt, 2 Seiten, Umschlag, 1326 Zeichen
\newline{}Schreibmaschine
\newline{}Handschrift: Bleistift, lateinische Kurrent (\noindent{}Beschriftung »Pollak« und
                                       »K{[}opie{]}«)}\toendnotes[C]{\smallbreak}\pstart{}{\pb}\textcolor{gray}{\textbf{Dr. Arthur Schnitzler}}\pend{}\pstart{}\textcolor{gray}{\textbf{Wien, XVIII. Sternwartestrasse 71}}\oindex{Sternwartestrasse 71@\textbf{Sternwartestraße 71}, \emph{Wohngebäude (K.WHS)}|pw}\pend{}{\bigskip}\pstart{}{\pb}Herrn Bezirksrichter \strikeout{Dr.}\pend{}\pstart{}Dr. Robert Adam-Pollak\pend{}\pstart{}\so{Zistersdorf}\oindex{Zistersdorf@\textbf{Zistersdorf}, \emph{A.ADM3}|pw}.\pend{}\pstart{}N. Oe.\oindex{Niederoesterreich@\textbf{Niederösterreich}, \emph{A.ADM1}|pw}\pend{}{\bigskip}\vspace{1em}
\pstart
           
\pstart
           {\pb}\textcolor{gray}{\textbf{Dr. Arthur Schnitzler}}\pend
           
\pstart
           \raggedleft{}13. 5. 1913.\pend
           \pend
           
\pstart
           \textcolor{gray}{\textbf{Wien XVIII. Sternwartestrasse 71\oindex{Sternwartestrasse 71@\textbf{Sternwartestraße 71}, \emph{Wohngebäude (K.WHS)}|pw}}}\pend
           
\pstart\center{}Sehr geehrter Herr Doktor.\pend\vspace{0.5em}
\pstart
           Es ist mir nicht ganz klar geworden, warum Sie glauben, dass die »Fatme\pwindex{Fatme@\emph{Fatme}|pw}« nicht meinen Beifall gefunden habe. Dass ich mich etwas
               kurz gefasst habe liegt einfach daran, dass meine Neigung zu ausführlicher
               essayistischer Behandlung im Allgemeinen eine recht geringe ist. Es kommt noch dazu,
               dass ich Ihr Stück\pwindex{Fatme@\emph{Fatme}|pwv}, das ich
               wirklich mit Vergnügen gelesen habe, gleich Ihnen doch nur als Studie und nicht als
               reines Kunstwerk auffassen kann, was ja wohl auch nicht in Ihrer Intention gelegen \substVorne{}\textsuperscript{ist}\substDazwischen{}war\substHinten{}. Bei all dem habe ich gewisse Szenen auch poetisch sehr gelungen gefunden
               und wenn mir etwas weniger behagt hat, so waren es vielleicht etliche humoristische
               Partien Ihrer Studie, die sich ein wenig unter dem Niveau des Gesamtwerkes\pwindex{Fatme@\emph{Fatme}|pwv} abzuspielen scheinen. Aber wir
               wollen nicht dogmatisch sein; wenn {\pb}es auch kein Drama \strikeout{ist} vorstellt, wenn man auch von einem höheren
               künstlerischen Standpunkt aus überhaupt nichts Rechtes damit anfangen kann, – aus dem
               Einfall als solchen und aus manchem Detail spricht ein feiner, kultivierter Geist,
               dessen Aeusserungen in welcher Form immer sie mir dargebracht werden, ich \introOben{}stets\introOben{} mit Interesse aufnehme.\pend
           
\pstart
           Mit verbindlichem Gruss{\\[\baselineskip]}Ihr sehr ergebener{\\[\baselineskip]}\spacefill\mbox{{[}hs.:{]} Arthur Schnitzler}\pend
           \leftskip=0em{}
\pstart
           \noindent{}Herrn Bezirksrichter Dr.Adam Pollak, Zistersdorf\oindex{Zistersdorf@\textbf{Zistersdorf}, \emph{A.ADM3}|pw}.\pend
           \selectlanguage{ngerman}\endnumbering\briefempfaengerindex{Adam, Robert@\textsc{Adam, Robert}!zzzSchnitzler, Arthur@\emph{von Arthur Schnitzler}!1913-05-131@{13. 5. 1913}|)be}\mylabel{L02137h}  \normalsize

\doendnotes{C}
\bigskip
\vfill

\clearpage

\footnotesize

\lohead{\textsc{register}}

% Definiere theindex-Environment komplett neu ohne reledmac
\makeatletter
\renewenvironment{theindex}{%
  \section*{\indexname}%
  \setlength{\parindent}{0pt}%
  \setlength{\parskip}{0pt plus 0.3pt}%
  \let\item\@idxitem
}{%
  \clearpage
}
\makeatother

\IfFileExists{\jobname-pw.ind}{\input{\jobname-pw.ind}}{}

\end{document}

      