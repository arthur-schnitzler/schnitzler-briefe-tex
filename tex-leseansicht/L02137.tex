%% latex-leseansicht-vorspann.tex
%% Vorspann für die Leseansicht.
%% Lädt die gemeinsame Datei latex-vorspann.tex mit nicht gesetztem Schalter.

\newif\ifkorrekturansicht
\korrekturansichtfalse

\input{../tex-inputs/latex-vorspann}


         
         \renewcommand{\erwaehntePersonen}{Personen: Robert Adam}
         \renewcommand{\erwaehnteOrte}{Orte: Niederösterreich, Sternwartestraße, Wien, Zistersdorf}
         \renewcommand{\erwaehnteWerke}{Werke: Fatme}
               \section[Arthur Schnitzler an Robert Adam, 13. 5. 1913]{ Arthur Schnitzler an Robert Adam, 13. 5. 1913}\nopagebreak\mylabel{v}\rehead{ }\begin{ledgroupsized}[t]{13cm}\normalsize\beginnumbering \toendnotes[C]{\smallbreak\pagebreak[2]} \Standort{DLA, 96.34.1/11.}
\physDesc{Brief, 1 Blatt, 2 Seiten, Umschlag, 1326 Zeichen
\newline{}Schreibmaschine
\newline{}Handschrift: schwarze Tinte, lateinische Kurrent (\noindent{}Korrekturen, Unterschrift)
\newline{}Versand: Stempel: »\nobreak{}13. {[}5.{]} 13\nobreak{}«.  }\Standort{DLA, A:Schnitzler, 85.1.1621.}
\physDesc{Brief, Durchschlag, 1 Blatt, 2 Seiten, Umschlag
\newline{}Schreibmaschine
\newline{}Handschrift: Bleistift, lateinische Kurrent (\noindent{}Beschriftung »Pollak« und
                                       »K{[}opie{]}«)}\toendnotes[C]{\smallbreak}\pstart{}{\pb}\textcolor{gray}{\textbf{Dr. Arthur Schnitzler}}\pend{}\pstart{}\textcolor{gray}{\textbf{Wien, XVIII. Sternwartestrasse 71}}\oindex{XXXX Ortsangabe fehlt|pw}\pend{}{\bigskip}\pstart{}{\pb}Herrn Bezirksrichter \strikeout{Dr.}\pend{}\pstart{}Dr. Robert Adam-Pollak\pend{}\pstart{}\so{Zistersdorf}\oindex{Zistersdorf@\textbf{Zistersdorf}|pw}.\pend{}\pstart{}N. Oe.\oindex{Niederoesterreich@\textbf{Niederösterreich}|pw}\pend{}{\bigskip}\pstart
           {\pb}\textcolor{gray}{\textbf{Dr. Arthur Schnitzler}}\hfill 13. 5. 1913.\pend
           \pstart
           \textcolor{gray}{\textbf{Wien XVIII. Sternwartestrasse 71\oindex{XXXX Ortsangabe fehlt|pw}}}\pend
           \pstart\center{}Sehr geehrter Herr Doktor.\pend\pstart
           Es ist mir nicht ganz klar geworden, warum Sie glauben, dass die »Fatme\pwindex{Adam, Robert 20.04.1877 – 16.10.1961@\textsc{Adam, Robert} (20.04.1877 – 16.10.1961), \emph{Schriftsteller, Richter}!FatmeNone@\strich\emph{Fatme} {[}None{]}|pw}« nicht meinen Beifall gefunden habe. Dass ich mich etwas
               kurz gefasst habe liegt einfach daran, dass meine Neigung zu ausführlicher
               essayistischer Behandlung im Allgemeinen eine recht geringe ist. Es kommt noch dazu,
               dass ich Ihr Stück\pwindex{Adam, Robert 20.04.1877 – 16.10.1961@\textsc{Adam, Robert} (20.04.1877 – 16.10.1961), \emph{Schriftsteller, Richter}!FatmeNone@\strich\emph{Fatme} {[}None{]}|pwv}, das ich
               wirklich mit Vergnügen gelesen habe, gleich Ihnen doch nur als Studie und nicht als
               reines Kunstwerk auffassen kann, was ja wohl auch nicht in Ihrer Intention gelegen \substVorne{}\textsuperscript{ist}\substDazwischen{}war\substHinten{}. Bei all dem habe ich gewisse Szenen auch poetisch sehr gelungen gefunden
               und wenn mir etwas weniger behagt hat, so waren es vielleicht etliche humoristische
               Partien Ihrer Studie, die sich ein wenig unter dem Niveau des Gesamtwerkes\pwindex{Adam, Robert 20.04.1877 – 16.10.1961@\textsc{Adam, Robert} (20.04.1877 – 16.10.1961), \emph{Schriftsteller, Richter}!FatmeNone@\strich\emph{Fatme} {[}None{]}|pwv} abzuspielen scheinen. Aber wir
               wollen nicht dogmatisch sein; wenn {\pb}es auch kein Drama \strikeout{ist} vorstellt, wenn man auch von einem höheren
               künstlerischen Standpunkt aus überhaupt nichts Rechtes damit anfangen kann, – aus dem
               Einfall als solchen und aus manchem Detail spricht ein feiner, kultivierter Geist,
               dessen Aeusserungen in welcher Form immer sie mir dargebracht werden, ich \introOben{}stets\introOben{} mit Interesse aufnehme.\pend
           \pstart
           Mit verbindlichem Gruss{\\[\baselineskip]}Ihr sehr ergebener{\\[\baselineskip]}\spacefill\mbox{{[}hs.:{]} Arthur Schnitzler}\pend
           \leftskip=0em{}\pstart
           \noindent{}Herrn Bezirksrichter Dr.Adam Pollak, Zistersdorf\oindex{Zistersdorf@\textbf{Zistersdorf}|pw}.\pend
           
         
         \endnumbering\mylabel{h}\end{ledgroupsized}  \newcommand{\dateiname}{L02137}\newcommand{\titel}{Arthur Schnitzler an Robert Adam, 13. 5. 1913}\newcommand{\editorInnen}{Martin Anton Müller und Gerd-Hermann Susen}%% latex-leseansicht-abspann.tex
%% Abspann für die Leseansicht.
%% Der Schalter \ifkorrekturansicht ist bereits durch den Vorspann gesetzt.

%% latex-abspann.tex
%% Gemeinsamer Abspann für Korrekturansicht und Leseansicht.
%% Setzt den Schalter \ifkorrekturansicht voraus (gesetzt in den
%% einbindenden Dateien latex-korrekturansicht-abspann.tex bzw.
%% latex-leseansicht-abspann.tex).
%% ---------------------------------------------------------------

\normalsize

% Das esempio-Environment wird nur in der Leseansicht benötigt
\ifkorrekturansicht\else
\newenvironment{esempio}[3]%
{
    \vspace{1.5ex}
    \rlap{\underline{#1}}
    \par
    \setlength{\parindent}{0cm}
    \nopagebreak
    \leftskip=#2cm
    \rightskip=#3cm
}
{
    \par
}
\fi

\doendnotes{C}
\bigskip
\vfill

\clearpage

\footnotesize

\ifkorrekturansicht
  \lohead{\textsc{register}}
\fi

% theindex-Environment neu definieren ohne reledmac
\makeatletter
\renewenvironment{theindex}{%
  \ifkorrekturansicht
    \section*{\indexname}%
  \else
    \subsubsection*{Index der erwähnten Entitäten}%
  \fi
  \setlength{\parindent}{0pt}%
  \setlength{\parskip}{0pt plus 0.3pt}%
  \let\item\@idxitem
}{%
  \ifkorrekturansicht\clearpage\fi
}
\makeatother

\IfFileExists{\jobname-pw.ind}{\input{\jobname-pw.ind}}{}

% Quellenangabe nur in der Leseansicht
\ifkorrekturansicht\else
% Fallback-Definitionen, falls die .tex-Datei \titel etc. nicht gesetzt hat
\providecommand{\titel}{}
\providecommand{\editorInnen}{}
\providecommand{\dateiname}{\jobname}

\vspace{3cm}

\vfill

\footnotesize
\textsc{Quelle}: \titel. Herausgegeben von {\editorInnen}. In: \emph{Arthur Schnitzler: Briefwechsel mit Autorinnen und Autoren}.
 Digitale Edition, https://schnitzler-briefe.acdh.oeaw.ac.at/{\dateiname}.html (Stand \today)
\fi

\end{document}


      