%% latex-leseansicht-vorspann.tex
%% Vorspann für die Leseansicht.
%% Lädt die gemeinsame Datei latex-vorspann.tex mit nicht gesetztem Schalter.

\newif\ifkorrekturansicht
\korrekturansichtfalse

\input{../tex-inputs/latex-vorspann}


\section[Stefan Zweig an Arthur Schnitzler, 15. 5. {[}1922{]}]{L03680 Stefan Zweig an Arthur Schnitzler, 15. 5. [1922]}
\nopagebreak\mylabel{L03680v}
\rehead{ }\normalsize\beginnumbering\briefempfaengerindex{Schnitzler, Arthur@\textsc{Schnitzler, Arthur}!zzzZweig, Stefan@\emph{von Stefan Zweig}!1922-05-154@{15. 5. [1922]}|(be}
\toendnotes[C]{\smallbreak\pagebreak[2]}
\correspDesc{Versand  durch Stefan Zweig am 15. 5. [1922] in Salzburg
\newline{}Erhalt  durch Arthur Schnitzler am 15. 5. [1922] in Wien}\toendnotes[C]{\smallbreak}
\Standort{DLA, A:Schnitzler, HS.NZ85.1.5577.}
\physDesc{Telegramm, 1 Blatt, 1 Seite, 181 Zeichen
\newline{}maschinell
\newline{}Handschrift: Bleistift, lateinische Kurrent (\noindent{}Streichung am Zeilenende und Ergänzung eines Buchstabens in der
                                 nächsten Zeile)
\newline{}Versand: mit Bleistift Eintragung am Vordruck: »\noindent{}\textcolor{gray}{\textbf{Aufgenommen von .......... auf Ltg. Nr. ..........
                                          am}}{ }15/5 \textcolor{gray}{\textbf{192{\dots}}}{ }\textcolor{gray}{\textbf{um {\dotsfive}
                                             Uhr{ }{\dots}M.}}{ }\textcolor{gray}{fl}{ }\textcolor{gray}{\textbf{Mitt.}}« }
\buchAbdrucke{\weitereDrucke{Stefan Zweig: \emph{Briefwechsel mit Hermann Bahr, Sigmund Freud, Rainer Maria
                        Rilke und Arthur Schnitzler}. Herausgegeben von Jeffrey B. Berlin, Hans-Ulrich Lindken und Donald A. Prater. Frankfurt am Main: \emph{S. Fischer} 1987, S. 412.} }\toendnotes[C]{\smallbreak}\pstart{}{\pb}artur schnitzler \strikeout{s}\pend{}\pstart{}\introOben{}s\introOben{}ternwartestrasze wien\oindex{Wien@\textbf{Wien}!XVIII., Währing@\textbf{XVIII., Währing}!Sternwartestraße 71@\textbf{Sternwartestraße 71}, \emph{Wohngebäude}|pw}\pend{}{\bigskip}\vspace{1em}
\pstart
           \centering{}{\pb}salzburg\oindex{Salzburg@\textbf{Salzburg}, \emph{Verwaltungsgebiet}|pw} ts 1020 21/20 15/5{ }0.10 =\pend
           \vspace{0.5em}
\pstart
           empfangen sie zu tausendfaeltigen \label{K_L03680-1v}\edtext{grueszen der liebe}{\lemma{\textnormal{\emph{grueszen der liebe}}}\Cendnote{\textnormal{Am 15. 5. 1922 wurde Schnitzler 60 Jahre alt. Das Telegramm ist
                  durch die Übermittlungszeile nur auf den Tag und Monat genau datierbar, die
                  Jahresangabe fehlt. Einen gewissen Hinweis gibt der Vordruck der Drucksache
                  (unterer Rand): »\textcolor{gray}{\textbf{Auflage 1922}}«. Die Aufbewahrung des Telegramms im Nachlass Schnitzlers zusammen mit weiteren Gratulationsschreiben zu
                  60. Geburtstag stützt diese Einordnung.}}}\label{K_L03680-1} und verehrung guetig auch die ihres
               getreuen \spacefill\mbox{stefan zweig .+}\pend
           \selectlanguage{ngerman}\endnumbering\briefempfaengerindex{Schnitzler, Arthur@\textsc{Schnitzler, Arthur}!zzzZweig, Stefan@\emph{von Stefan Zweig}!1922-05-154@{15. 5. [1922]}|)be}\mylabel{L03680h}  \newcommand{\dateiname}{L03680}\newcommand{\titel}{Stefan Zweig an Arthur Schnitzler, 15. 5. [1922]}\newcommand{\editorInnen}{Selma Jahnke und Martin Anton Müller}%% latex-leseansicht-abspann.tex
%% Abspann für die Leseansicht.
%% Der Schalter \ifkorrekturansicht ist bereits durch den Vorspann gesetzt.

%% latex-abspann.tex
%% Gemeinsamer Abspann für Korrekturansicht und Leseansicht.
%% Setzt den Schalter \ifkorrekturansicht voraus (gesetzt in den
%% einbindenden Dateien latex-korrekturansicht-abspann.tex bzw.
%% latex-leseansicht-abspann.tex).
%% ---------------------------------------------------------------

\normalsize

% Das esempio-Environment wird nur in der Leseansicht benötigt
\ifkorrekturansicht\else
\newenvironment{esempio}[3]%
{
    \vspace{1.5ex}
    \rlap{\underline{#1}}
    \par
    \setlength{\parindent}{0cm}
    \nopagebreak
    \leftskip=#2cm
    \rightskip=#3cm
}
{
    \par
}
\fi

\doendnotes{C}
\bigskip
\vfill

\clearpage

\footnotesize

\ifkorrekturansicht
  \lohead{\textsc{register}}
\fi

% theindex-Environment neu definieren ohne reledmac
\makeatletter
\renewenvironment{theindex}{%
  \ifkorrekturansicht
    \section*{\indexname}%
  \else
    \subsubsection*{Index der erwähnten Entitäten}%
  \fi
  \setlength{\parindent}{0pt}%
  \setlength{\parskip}{0pt plus 0.3pt}%
  \let\item\@idxitem
}{%
  \ifkorrekturansicht\clearpage\fi
}
\makeatother

\IfFileExists{\jobname-pw.ind}{\input{\jobname-pw.ind}}{}

% Quellenangabe nur in der Leseansicht
\ifkorrekturansicht\else
% Fallback-Definitionen, falls die .tex-Datei \titel etc. nicht gesetzt hat
\providecommand{\titel}{}
\providecommand{\editorInnen}{}
\providecommand{\dateiname}{\jobname}

\vspace{3cm}

\vfill

\footnotesize
\textsc{Quelle}: \titel. Herausgegeben von {\editorInnen}. In: \emph{Arthur Schnitzler: Briefwechsel mit Autorinnen und Autoren}.
 Digitale Edition, https://schnitzler-briefe.acdh.oeaw.ac.at/{\dateiname}.html (Stand \today)
\fi

\end{document}


