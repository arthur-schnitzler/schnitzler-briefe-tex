%% latex-korrekturansicht-vorspann.tex
%% Vorspann für die Korrekturansicht.
%% Lädt die gemeinsame Datei latex-vorspann.tex mit gesetztem Schalter.

\newif\ifkorrekturansicht
\korrekturansichttrue

\input{../tex-inputs/latex-vorspann}


\section[Stefan Zweig an Arthur Schnitzler, 15. 5. {[}1922{]}]{L03680 Stefan Zweig an Arthur Schnitzler, 15. 5. {[}1922{]}}
\nopagebreak\mylabel{L03680v}
\rehead{ }\normalsize\beginnumbering\briefempfaengerindex{Schnitzler, Arthur@\textsc{Schnitzler, Arthur}!zzzZweig, Stefan@\emph{von Stefan Zweig}!1922-05-154@{15. 5. {[}1922{]}}|(be}
\toendnotes[C]{\smallbreak\pagebreak[2]}\Standort{DLA, A:Schnitzler, HS.NZ85.1.5577.}
\physDesc{Telegramm, 1 Blatt, 1 Seite, 179 Zeichen
\newline{}maschinell
\newline{}Versand: mit Bleistift Eintragung am Vordruck: »\noindent{}\textcolor{gray}{\textbf{Aufgenommen von .......... auf Ltg. Nr. ..........
                                          am}}{ }15/5 \textcolor{gray}{\textbf{192{\dots}}}{ }\textcolor{gray}{\textbf{um {\dotsfive}
                                       Uhr{ }{\dots}M.}}{ }\textcolor{gray}{fl}\textcolor{gray}{\textbf{Mittag}}« }
\buchAbdrucke{\weitereDrucke{Stefan Zweig: \emph{Briefwechsel mit Hermann Bahr, Sigmund Freud, Rainer Maria
                        Rilke und Arthur Schnitzler}. Frankfurt am Main: \emph{S. Fischer} 1987, S. 412.} }\toendnotes[C]{\smallbreak}\pstart{}{\pb}artur schnitzler \strikeout{s}\pend{}\pstart{}\introOben{}s\introOben{}ternwartestrasze
                     wien\oindex{Sternwartestrasse 71@\textbf{Sternwartestraße 71}, \emph{Wohngebäude (K.WHS)}|pw}\pend{}{\bigskip}\vspace{1em}
\pstart
           \centering{}{\pb}salzburg\oindex{Salzburg@\textbf{Salzburg}, \emph{A.ADM2}|pw} ts 1020 21/20 15/5{ }0.10\pend
           \vspace{0.5em}
\pstart
           empfangen sie zu tausendfaeltigen \label{K_L03680-1v}\edtext{grueszen der liebe}{\lemma{\textnormal{\emph{grueszen der liebe}}}\Cendnote{\textnormal{Am 15. 5. 1922 wurde Schnitzler
            60 Jahre alt. Das Telegramm ist durch die Übermittlungszeile nur auf den Tag und Monat genau datierbar, die Jahresangabe fehlt Einen gewissen Hinweis gibt der Vordruck der Drucksache: »\textcolor{gray}{\textbf{Auflage 1922}}«. Die 
                  Aufbewahrung des Telegramms im Nachlass Schnitzlers zusammen mit weiteren Gratulationsschreiben
                  zu diesem Geburtstag stützt diese Einordnung.}}}\label{K_L03680-1} und verehrung guetig auch die
               ihres getreuen \spacefill\mbox{stefan zweig .+}\pend
           \selectlanguage{ngerman}\endnumbering\briefempfaengerindex{Schnitzler, Arthur@\textsc{Schnitzler, Arthur}!zzzZweig, Stefan@\emph{von Stefan Zweig}!1922-05-154@{15. 5. {[}1922{]}}|)be}\mylabel{L03680h}  \normalsize

\doendnotes{C}
\bigskip
\vfill

\clearpage

\footnotesize

\lohead{\textsc{register}}

% Definiere theindex-Environment komplett neu ohne reledmac
\makeatletter
\renewenvironment{theindex}{%
  \section*{\indexname}%
  \setlength{\parindent}{0pt}%
  \setlength{\parskip}{0pt plus 0.3pt}%
  \let\item\@idxitem
}{%
  \clearpage
}
\makeatother

\IfFileExists{\jobname-pw.ind}{\input{\jobname-pw.ind}}{}

\end{document}

      