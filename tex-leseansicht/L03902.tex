%% latex-leseansicht-vorspann.tex
%% Vorspann für die Leseansicht.
%% Lädt die gemeinsame Datei latex-vorspann.tex mit nicht gesetztem Schalter.

\newif\ifkorrekturansicht
\korrekturansichtfalse

\input{../tex-inputs/latex-vorspann}


\section[Arthur Schnitzler an Theodor Herzl, 21. 6. 1893]{L03902 Arthur Schnitzler an Theodor Herzl, 21. 6. 1893}
\nopagebreak\mylabel{L03902v}
\rehead{ }\normalsize\beginnumbering\briefempfaengerindex{Herzl, Theodor@\textsc{Herzl, Theodor}!zzzSchnitzler, Arthur@\emph{von Arthur Schnitzler}!1893-06-211@{21. 6. 1893}|(be}
\toendnotes[C]{\smallbreak\pagebreak[2]}
\correspDesc{Versand  durch Arthur Schnitzler am 21. 6. 1893 in Wien
\newline{}Erhalt  durch Theodor Herzl in Paris}\toendnotes[C]{\smallbreak}
\Standort{Jerusalem, Central Zionist Archives, H1:1924-7.}
\physDesc{, 2 Blätter, 8 Seiten
\newline{}Handschrift: schwarze Tinte, deutsche Kurrent
\newline{}Ordnung: mit Bleistift von unbekannter Hand nummeriert: »7« und innerhalb das Konvoluts paginiert: »25«–»28« }
\buchAbdrucke{\weitereDrucke{Arthur Schnitzler: \emph{Briefe 1875–1912}. Herausgegeben von Therese Nickl und Heinrich Schnitzler. Frankfurt am Main: \emph{S. Fischer} 1981, S. 207–209.} }\toendnotes[C]{\smallbreak}
\pstart{}{\pb}Verehrter Freund,\pend\vspace{0.5em}
\pstart
           hoffentlich iſt alles bei Ihnen wohl und die Maſern haben Ihre älteſte\pwindex{Hüft, Pauline 29.\,3.\,1890 – 8.\,9.\,1930@\textsc{Hüft, Pauline} (29.\,3.\,1890 – 8.\,9.\,1930)|pw} verſchont. Es wird also wohl nach Wien\oindex{Wien@\textbf{Wien}, \emph{Verwaltungsgebiet}|pw} gefahren? – Wie geſagt, ich wiederhole meine Bitte, daſs
               Sie mich irgendwie verſtändigen, wie lange Sie dableiben \textsc{etc
                  etc.} – Es iſt nicht unmöglich, daß ich Anfang Juli auf 10–14
               Tage mit meiner armen Mama\pwindex{Schnitzler, Louise 8.\,7.\,1840 Kőszeg – 9.\,9.\,1911 Wien@\textsc{Schnitzler, Louise} (8.\,7.\,1840 Kőszeg – 9.\,9.\,1911 Wien)|pwv}
               nach Iſchl\oindex{Bad Ischl@\textbf{Bad Ischl}|pw} gehe; im übrigen wäre ein biſſel
                  Gebirgs{\pb}luft für mich nicht gerade überflüſſig, da ich{ }ſeit den qualvollen Aufregungen des Frühjahrs an einem großentheils nervöſen Huſten
               leide, den ich nicht anbringe. Im übrigen habe ich mich jetzt auf den Sport geworfen,
               u. fahre{ }ſeit ein paar Tagen auf dem \textsc{Bicycle}. Ich{ }ſchreibe
               dieſe Zeilen mit{ }ſteifem Arm und{ }ſteifem Bein – iſt auch das letztere zu
               bemerken?–\pend
           
\pstart
           {\pb}– Ihren Prinzen aus
                  Genieland\pwindex{Herzl, Theodor 2.\,5.\,1860 Budapest – 3.\,7.\,1904 Edlach@\textsc{Herzl, Theodor} (2.\,5.\,1860 Budapest – 3.\,7.\,1904 Edlach), \emph{Schriftsteller, Journalist}!Prinzen aus Genieland. Lustspiel in 4 Akten@\strich\emph{Prinzen aus Genieland. Lustspiel in 4 Akten}|pw} hab ich hier im Carltheater\oindex{Wien@\textbf{Wien}!II., Leopoldstadt@\textbf{II., Leopoldstadt}!Carl-Theater@\textbf{Carl-Theater}, \emph{Theater}|pw}{ }geſehen\eventindex{Carl-Theater@\textbf{Carl-Theater}!Uraufführung von Prinzen in Genieland, 20.11.1891@Uraufführung von Prinzen in Genieland, 20.11.1891|pwv} – we{\geminationn}{ }ſie nicht ganz um\substVorne{}\textsuperscript{lege}\substDazwischen{}gebr\substHinten{}acht worden{ }ſind,{ }ſo{ }ſind nicht die Mörder,{ }ſondern die Prinzen{ }ſelbſt
                  dran{ }ſchuld geweſen; denn es iſt im ganzen miſerabel geſpielt worden.
               Mir war dasStück\pwindex{Herzl, Theodor 2.\,5.\,1860 Budapest – 3.\,7.\,1904 Edlach@\textsc{Herzl, Theodor} (2.\,5.\,1860 Budapest – 3.\,7.\,1904 Edlach), \emph{Schriftsteller, Journalist}!Prinzen aus Genieland. Lustspiel in 4 Akten@\strich\emph{Prinzen aus Genieland. Lustspiel in 4 Akten}|pwv}{ }ſehr{ }ſympathiſch; es lag wie ein Duft von 1840 drüber; es gehört vielleicht zu denjenigen
               Ihrer dramatiſchen Sachen, in {\pb}denen am meiſten Frühling
               iſt. Allerdings sind beträchtliche Ungleichheiten drin, und was mir am ärgerlichſten
               daran war, –{ }ſo weit mich heut meine Eri{\geminationn}erg nicht
               täuſcht, – war die zu pathetiſche u abſichtliche Manier, in welchem plötzlich die
               Grundidee (im 3. Akt\pwindex{Herzl, Theodor 2.\,5.\,1860 Budapest – 3.\,7.\,1904 Edlach@\textsc{Herzl, Theodor} (2.\,5.\,1860 Budapest – 3.\,7.\,1904 Edlach), \emph{Schriftsteller, Journalist}!Prinzen aus Genieland. Lustspiel in 4 Akten@\strich\emph{Prinzen aus Genieland. Lustspiel in 4 Akten}|pwv} glaub
               ich) \substVorne{}\textsuperscript{\textcolor{gray}{b}}\substDazwischen{}a\substHinten{}usgeſprochen wird,{ }ſtatt daſs das ganze Stück\pwindex{Herzl, Theodor 2.\,5.\,1860 Budapest – 3.\,7.\,1904 Edlach@\textsc{Herzl, Theodor} (2.\,5.\,1860 Budapest – 3.\,7.\,1904 Edlach), \emph{Schriftsteller, Journalist}!Prinzen aus Genieland. Lustspiel in 4 Akten@\strich\emph{Prinzen aus Genieland. Lustspiel in 4 Akten}|pwv} im Fortſchreiten{ }ſelbſt und in{ }ſeinen {\pb}Charakteren jene Grundidee ausſpricht. – Sehr deutlich iſt
               mir eine treffliche Charge \textsc{Knaack\pwindex{Knaack, Wilhelm 13.\,2.\,1829 Rostock – 29.\,10.\,1894@\textsc{Knaack, Wilhelm} (13.\,2.\,1829 Rostock – 29.\,10.\,1894), \emph{Schauspieler}|pw}’}s im Gedächtnis. – Da{\geminationn} die hübſche kleinbürgerliche Scene im 2. Akt, in der
               eine Violine vorko{\geminationm}t. Da{\geminationn}
               die Liebesſcene, die von Herrn \textsc{Franker\pwindex{Franker, Heinrich *~1857@\textsc{Franker, Heinrich} (*~1857)|pw}} u Frl. \textsc{Ernst\pwindex{Ernst, Carla 15.\,5.\,1867 Wien – 15.\,6.\,1925 ebd.@\textsc{Ernst, Carla} (15.\,5.\,1867 Wien – 15.\,6.\,1925 ebd.), \emph{Schauspielerin}|pw}} gar nicht übel gegeben wurde. Unangenehm war der Oberprinz – Herr \textsc{Lenor\pwindex{Lenor, Robert v. 8.\,1.\,1855 Wien – 29.\,6.\,1900 Innsbruck@\textsc{Lenor, Robert v.} (8.\,1.\,1855 Wien – 29.\,6.\,1900 Innsbruck), \emph{Schauspieler, Verwaltungsbeamter}|pw}}, der seine Rolle {\pb}mit einer{ }ſchnodderigen
               Liebenswürdigkeit gab, welche mir im Ohr und im Herzen wehthat. Ich hätte damals den
               lebhaften Wunſch, das Stück\pwindex{Herzl, Theodor 2.\,5.\,1860 Budapest – 3.\,7.\,1904 Edlach@\textsc{Herzl, Theodor} (2.\,5.\,1860 Budapest – 3.\,7.\,1904 Edlach), \emph{Schriftsteller, Journalist}!Prinzen aus Genieland. Lustspiel in 4 Akten@\strich\emph{Prinzen aus Genieland. Lustspiel in 4 Akten}|pwv} wo
               anders zu{ }ſehen, u mit mir wünſchten{ }ſich einige Vernünftige, daſs das Luſtſpiel
               nicht am Volkstheater\orgindex{Volkstheater@Volkstheater|pw} gegeben würde – \textsc{resp.} dß Sie’s nicht dem V.th.\orgindex{Volkstheater@Volkstheater|pw}{ }{\pb}eingereicht hatten. – Ihre Geschichten von Barnay\pwindex{Barnay, Ludwig 11.\,2.\,1842 Budapest – 1.\,2.\,1924@\textsc{Barnay, Ludwig} (11.\,2.\,1842 Budapest – 1.\,2.\,1924), \emph{Schriftsteller, Schauspieler, Theaterdirektor}|pw}, vom Flüchtling\pwindex{Herzl, Theodor 2.\,5.\,1860 Budapest – 3.\,7.\,1904 Edlach@\textsc{Herzl, Theodor} (2.\,5.\,1860 Budapest – 3.\,7.\,1904 Edlach), \emph{Schriftsteller, Journalist}!Flüchtling. Lustspiel in einem Aufzug@\strich\emph{Der Flüchtling. Lustspiel in einem Aufzug}|pw} u.{ }ſo w.
               haben mich intereſſirt u. gerührt. Jawohl gerührt – denn ich finde{ }ſolche Abenteuer
               des Ehrgeizes und des Talents um der Ehrlichkeit rührender als hungrige Mägen und
               manches verliebte Idyll. – Vom Volkstheater\orgindex{Volkstheater@Volkstheater|pw}
               müſſen Sie mir erzählen.\pend
           
\pstart
           {\pb}– Der Brief da trifft Sie, lieber Freund, wohl noch in Paris\oindex{Paris@\textbf{Paris}, \emph{Hauptstadt}|pw}. Nehmen Sie noch mein Glückwunsch zumGretherl\pwindex{Neumann, Margarethe 20.\,5.\,1893 Paris – 15.\,3.\,1943 Konzentrationslager Theresienstadt@\textsc{Neumann, Margarethe} (20.\,5.\,1893 Paris – 15.\,3.\,1943 Konzentrationslager Theresienstadt)|pw} entgegen und beſtellen Sie dieſelbe
               auch Ihrer Frau Gemahlin\pwindex{Herzl, Julie 1.\,2.\,1868 Budapest – 10.\,11.\,1907 Wien@\textsc{Herzl, Julie} (1.\,2.\,1868 Budapest – 10.\,11.\,1907 Wien)|pwv}, die{ }ſich vielleicht noch meiner erinnert.–\pend
           
\pstart
           Alſo auf baldiges Wiederſehen,{\\[\baselineskip]}Ihr herzlich ergebener{\\[\baselineskip]}\spacefill\mbox{ArthSchnitzler}\pend
           \leftskip=0em{}
\pstart
           21. 6. 93.\pend
           \selectlanguage{ngerman}\endnumbering\briefempfaengerindex{Herzl, Theodor@\textsc{Herzl, Theodor}!zzzSchnitzler, Arthur@\emph{von Arthur Schnitzler}!1893-06-211@{21. 6. 1893}|)be}\mylabel{L03902h}
\begin{anhang}
\end{anhang}\newcommand{\dateiname}{L03902}\newcommand{\titel}{Arthur Schnitzler an Theodor Herzl, 21. 6. 1893}\newcommand{\editorInnen}{Herausgegeben von Jahnke, SelmaMüller, Martin Anton}%% latex-leseansicht-abspann.tex
%% Abspann für die Leseansicht.
%% Der Schalter \ifkorrekturansicht ist bereits durch den Vorspann gesetzt.

%% latex-abspann.tex
%% Gemeinsamer Abspann für Korrekturansicht und Leseansicht.
%% Setzt den Schalter \ifkorrekturansicht voraus (gesetzt in den
%% einbindenden Dateien latex-korrekturansicht-abspann.tex bzw.
%% latex-leseansicht-abspann.tex).
%% ---------------------------------------------------------------

\normalsize

% Das esempio-Environment wird nur in der Leseansicht benötigt
\ifkorrekturansicht\else
\newenvironment{esempio}[3]%
{
    \vspace{1.5ex}
    \rlap{\underline{#1}}
    \par
    \setlength{\parindent}{0cm}
    \nopagebreak
    \leftskip=#2cm
    \rightskip=#3cm
}
{
    \par
}
\fi

\doendnotes{C}
\bigskip
\vfill

\clearpage

\footnotesize

\ifkorrekturansicht
  \lohead{\textsc{register}}
\fi

% theindex-Environment neu definieren ohne reledmac
\makeatletter
\renewenvironment{theindex}{%
  \ifkorrekturansicht
    \section*{\indexname}%
  \else
    \subsubsection*{Index der erwähnten Entitäten}%
  \fi
  \setlength{\parindent}{0pt}%
  \setlength{\parskip}{0pt plus 0.3pt}%
  \let\item\@idxitem
}{%
  \ifkorrekturansicht\clearpage\fi
}
\makeatother

\IfFileExists{\jobname-pw.ind}{\input{\jobname-pw.ind}}{}

% Quellenangabe nur in der Leseansicht
\ifkorrekturansicht\else
% Fallback-Definitionen, falls die .tex-Datei \titel etc. nicht gesetzt hat
\providecommand{\titel}{}
\providecommand{\editorInnen}{}
\providecommand{\dateiname}{\jobname}

\vspace{3cm}

\vfill

\footnotesize
\textsc{Quelle}: \titel. Herausgegeben von {\editorInnen}. In: \emph{Arthur Schnitzler: Briefwechsel mit Autorinnen und Autoren}.
 Digitale Edition, https://schnitzler-briefe.acdh.oeaw.ac.at/{\dateiname}.html (Stand \today)
\fi

\end{document}


