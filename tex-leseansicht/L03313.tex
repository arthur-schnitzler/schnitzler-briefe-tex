%% latex-leseansicht-vorspann.tex
%% Vorspann für die Leseansicht.
%% Lädt die gemeinsame Datei latex-vorspann.tex mit nicht gesetztem Schalter.

\newif\ifkorrekturansicht
\korrekturansichtfalse

\input{../tex-inputs/latex-vorspann}

\begin{center}
            \textcolor{red}{ENTWURF, NICHT FERTIG KORRIGIERT}
                      \end{center}
            
         
         \renewcommand{\erwaehntePersonen}{Personen: Hermann Bahr, Marie Bardi, Arnold Barkay, Otto Brahm, Emerich von Bukovics, Hugo Felix, Hugo von Hofmannsthal, Irene Mandl, Ottilie Salten, Louise Schnitzler, Sven Scholander}
         \renewcommand{\erwaehnteInstitutionen}{Institutionen: Jung-Wiener Theater zum Lieben Augustin}
         \renewcommand{\erwaehnteOrte}{Orte: Bad Aussee, Frankfurt am Main, Karlsbad, Köln, München, Paris, Prag, Salzkammergut, Stuttgart, Theater an der Wien, Wien, Wiesbaden, Wörthersee, Zürich}
         \renewcommand{\erwaehnteWerke}{Werke: Der Gemeine. Schauspiel in drei Aufzügen, Die Frau mit dem Dolche, Zum großen Wurstel. Burleske in einem Akt}
               \section[Felix Salten an Arthur Schnitzler, 12. 6. 1901]{ Felix Salten an Arthur Schnitzler, 12. 6. 1901}\nopagebreak\mylabel{v}\rehead{ }\begin{ledgroupsized}[t]{13cm}\normalsize\beginnumbering \toendnotes[C]{\smallbreak\pagebreak[2]} \Standort{CUL, Schnitzler, B 89, A 2.}
\physDesc{Brief, 1 Blatt, 2 Seiten
\newline{}Handschrift: schwarze Tinte, lateinische Kurrent\newline{}Ordnung: mit Bleistift von unbekannter Hand nummeriert:
                                    »137« }\toendnotes[C]{\smallbreak}\pstart
           \noindent{}{\pb}\textcolor{gray}{\textbf{Jung-Wiener Theater\orgindex{Jung-Wiener Theater zum Lieben Augustin@Jung-Wiener Theater zum Lieben Augustin|pw}}}\hfill \textcolor{gray}{\textbf{Wien\oindex{Wien@\textbf{Wien}|pw},}}{ }12. Juni \textcolor{gray}{\textbf{190}}1\pend
           \pstart
           \textcolor{gray}{\textbf{Zum lieben Augustin\orgindex{Jung-Wiener Theater zum Lieben Augustin@Jung-Wiener Theater zum Lieben Augustin|pw}.}}\hfill \textcolor{gray}{\textbf{(Theater a. d.
                        Wien\oindex{Theater an der Wien@\textbf{Theater an der Wien}|pw})}}\pend
           \pstart
           \textcolor{gray}{\textbf{Direction.}}\pend
           \pstart
           Lieber Freund, es thut mir leid, dass ich Sie nicht mehr gesprochen
               habe. Bis Sonntag war ich verreist, Karlsbad\oindex{Karlsbad@\textbf{Karlsbad}|pw}Prag\oindex{Prag@\textbf{Prag}|pw}. Habe in Prag\oindex{Prag@\textbf{Prag}|pw} Frl. Bardi\pwindex{Bardi, Marie *~um 1870@\textsc{Bardi, Marie} (*~um 1870), \emph{Schauspielerin}|pw} und einen hübschen
               jungen Tenor\pwindex{Barkay, Arnold 1873 – 1938@\textsc{Barkay, Arnold} (1873 – 1938), \emph{Sänger/Sängerin, Humorist/Humoristin}|pwv} engagiert, der
               die größte Ambition hat, ein Sven Skolander\pwindex{Scholander, Sven 21.04.1860 – 14.12.1936@\textsc{Scholander, Sven} (21.04.1860 – 14.12.1936), \emph{Sänger}|pw} zu
               werden. Von D\textsuperscript{r} Mandl\pwindex{Mandl, Irene 02.08.1844 – 12.05.1919@\textsc{Mandl, Irene} (02.08.1844 – 12.05.1919)|pw} haben Sie gehört, dass Otti\pwindex{Salten, Ottilie 07.03.1868 – 22.06.1942@\textsc{Salten, Ottilie} (07.03.1868 – 22.06.1942), \emph{Schauspielerin}|pw}
               operirt wurde. Das war ziemlich schrecklich, obwol die ganze Sache an sich ja nichts
               bedeutet und glücklich verlaufen ist. Ich bleibe nun ungefähr acht Tage in Wien\oindex{Wien@\textbf{Wien}|pw} und fahre dann nach München\oindex{Muenchen@\textbf{München}|pw}, zwei Tage, von dort nach Zürich\oindex{Zuerich@\textbf{Zürich}|pw}, drei Tage, (Felix\pwindex{Felix, Hugo 19.11.1866 – 25.08.1934@\textsc{Felix, Hugo} (19.11.1866 – 25.08.1934), \emph{Komponist, Chemiker}|pw}) von da nach Paris\oindex{Paris@\textbf{Paris}|pw}, zwölf-14 Tage und
               d'dann nach Köln\oindex{Koeln@\textbf{Köln}|pw}, Frankfurt\oindex{Frankfurt am Main@\textbf{Frankfurt am Main}|pw}, Wiesbaden\oindex{Wiesbaden@\textbf{Wiesbaden}|pw}, Stuttgart\oindex{Stuttgart@\textbf{Stuttgart}|pw} – Wien\oindex{Wien@\textbf{Wien}|pw}. Im
                  Juli werde ich im Salzkammergut\oindex{Salzkammergut@\textbf{Salzkammergut}|pw}
               oder am Wörthersee\oindex{Woerthersee@\textbf{Wörthersee}|pw} sein. Auch zu einer kleinen
               Radtour wäre ich bereit. Den größten Theil des August bin ich in Wien\oindex{Wien@\textbf{Wien}|pw}, mit Ausnahme einer kurzen Reise nach Prag\oindex{Prag@\textbf{Prag}|pw} und nach Aussee\oindex{Bad Aussee@\textbf{Bad Aussee}|pw}. Das
               ist alles. Ich freue mich, dass Sie ein neues Stück\pwindex{Schnitzler, Arthur 15.05.1862 – 21.10.1931@\textsc{Schnitzler, Arthur} (15.05.1862 – 21.10.1931), \emph{Schriftsteller, Mediziner}!Frau mit dem Dolche1901@\strich\emph{Die Frau mit dem Dolche} {[}1901{]}|pw} haben, und hege künstlerisch eine ganz bestimmte Erwartung daran.
               Vielleicht läßt es sich machen, dass Bukovics\pwindex{Bukovics, Emerich von 28.02.1844 – 04.07.1905@\textsc{Bukovics, Emerich von} (28.02.1844 – 04.07.1905), \emph{Journalist, Theaterleiter}|pw}
               mir die »Marionetten\pwindex{Schnitzler, Arthur 15.05.1862 – 21.10.1931@\textsc{Schnitzler, Arthur} (15.05.1862 – 21.10.1931), \emph{Schriftsteller, Mediziner}!Zum grossen Wurstel. Burleske in einem Akt08. 03. 1901@\strich\emph{Zum großen Wurstel. Burleske in einem Akt} {[}08. 03. 1901{]}|pw}« abtritt, d. h. wenn Sie
               mir das Stück\pwindex{Schnitzler, Arthur 15.05.1862 – 21.10.1931@\textsc{Schnitzler, Arthur} (15.05.1862 – 21.10.1931), \emph{Schriftsteller, Mediziner}!Zum grossen Wurstel. Burleske in einem Akt08. 03. 1901@\strich\emph{Zum großen Wurstel. Burleske in einem Akt} {[}08. 03. 1901{]}|pwv} geben wollen.
                  Schrei{\pb}ben Sie mir darüber.
                  Brahm\pwindex{Brahm, Otto 05.02.1856 – 28.11.1912@\textsc{Brahm, Otto} (05.02.1856 – 28.11.1912), \emph{Theaterleiter, Regisseur}|pw} ist, wie Sie wissen, hier. Wir sahen
               uns im Theater, ohne uns zu grüßen. Es ist mir ja sonst ganz gleichgiltig, aber ich
               bereue jetzt, dass ich mich s. Z. doch habe bereden laßen, ihm mein Stück\pwindex{Salten, Felix 06.09.1869 – 08.10.1945@\textsc{Salten, Felix} (06.09.1869 – 08.10.1945), \emph{Schriftsteller, Journalist}!Gemeine. Schauspiel in drei Aufzuegen1901@\strich\emph{Der Gemeine. Schauspiel in drei Aufzügen} {[}1901{]}|pw} einzureichen. Nun bringt er mich durch sein Benehmen in
               den peinlichen Verdacht, als sei ich ihm \uline{deshalb}
               böse. Ich bin ihm aber garnicht böse, am wenigsten deshalb. Nur sehe ich keine
               Ursache, sein unfreundliches Verhalten einzustecken. \pend
           \pstart
           Von Bahr\pwindex{Bahr, Hermann 19.07.1863 – 15.01.1934@\textsc{Bahr, Hermann} (19.07.1863 – 15.01.1934), \emph{Schriftsteller, Kritiker}|pw} erfuhr ich, dass \label{K_L03313-1v}\edtext{Hofmannsthal\pwindex{Hofmannsthal, Hugo von 1874-02-01 – 1929-07-15@\textsc{Hofmannsthal, Hugo von} (1874-02-01 – 1929-07-15), \emph{Schriftsteller}|pw}{ }Samstag geheiratet}{\lemma{\textnormal{\emph{Hofmannsthal … geheiratet}}}\Cendnote{\textnormal{am
                     8. 6. 1901}}}\label{K_L03313-1h} hat. Schreiben Sie mir, bitte, bald. Hauptsächlich wohin Sie reisen. Ich habe
               das »wir« nicht verstanden. Sind Sie mit Ihrer Mama\pwindex{Schnitzler, Louise 1840-07-08 – 1911-09-09@\textsc{Schnitzler, Louise} (1840-07-08 – 1911-09-09)|pwv}? \pend
           \pstart
           Herzlichst {\\[\baselineskip]}Ihr {\\[\baselineskip]}\spacefill\mbox{Salten}\pend
           \leftskip=0em{}
         
         \endnumbering\mylabel{h}\end{ledgroupsized}\begin{anhang}\end{anhang}\newcommand{\dateiname}{L03313}\newcommand{\titel}{Felix Salten an Arthur Schnitzler, 12. 6. 1901}\newcommand{\editorInnen}{Martin Anton Müller und Laura Untner}%% latex-leseansicht-abspann.tex
%% Abspann für die Leseansicht.
%% Der Schalter \ifkorrekturansicht ist bereits durch den Vorspann gesetzt.

%% latex-abspann.tex
%% Gemeinsamer Abspann für Korrekturansicht und Leseansicht.
%% Setzt den Schalter \ifkorrekturansicht voraus (gesetzt in den
%% einbindenden Dateien latex-korrekturansicht-abspann.tex bzw.
%% latex-leseansicht-abspann.tex).
%% ---------------------------------------------------------------

\normalsize

% Das esempio-Environment wird nur in der Leseansicht benötigt
\ifkorrekturansicht\else
\newenvironment{esempio}[3]%
{
    \vspace{1.5ex}
    \rlap{\underline{#1}}
    \par
    \setlength{\parindent}{0cm}
    \nopagebreak
    \leftskip=#2cm
    \rightskip=#3cm
}
{
    \par
}
\fi

\doendnotes{C}
\bigskip
\vfill

\clearpage

\footnotesize

\ifkorrekturansicht
  \lohead{\textsc{register}}
\fi

% theindex-Environment neu definieren ohne reledmac
\makeatletter
\renewenvironment{theindex}{%
  \ifkorrekturansicht
    \section*{\indexname}%
  \else
    \subsubsection*{Index der erwähnten Entitäten}%
  \fi
  \setlength{\parindent}{0pt}%
  \setlength{\parskip}{0pt plus 0.3pt}%
  \let\item\@idxitem
}{%
  \ifkorrekturansicht\clearpage\fi
}
\makeatother

\IfFileExists{\jobname-pw.ind}{\input{\jobname-pw.ind}}{}

% Quellenangabe nur in der Leseansicht
\ifkorrekturansicht\else
% Fallback-Definitionen, falls die .tex-Datei \titel etc. nicht gesetzt hat
\providecommand{\titel}{}
\providecommand{\editorInnen}{}
\providecommand{\dateiname}{\jobname}

\vspace{3cm}

\vfill

\footnotesize
\textsc{Quelle}: \titel. Herausgegeben von {\editorInnen}. In: \emph{Arthur Schnitzler: Briefwechsel mit Autorinnen und Autoren}.
 Digitale Edition, https://schnitzler-briefe.acdh.oeaw.ac.at/{\dateiname}.html (Stand \today)
\fi

\end{document}


      