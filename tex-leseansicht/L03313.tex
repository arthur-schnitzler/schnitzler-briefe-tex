%% latex-leseansicht-vorspann.tex
%% Vorspann für die Leseansicht.
%% Lädt die gemeinsame Datei latex-vorspann.tex mit nicht gesetztem Schalter.

\newif\ifkorrekturansicht
\korrekturansichtfalse

\input{../tex-inputs/latex-vorspann}


\section[ Felix Salten an Arthur Schnitzler, 12. 6. 1901]{L03313 Felix Salten an Arthur Schnitzler,  12. 6. 1901}
\nopagebreak\mylabel{L03313v}
\rehead{ }\normalsize\beginnumbering\briefempfaengerindex{Schnitzler, Arthur@\textsc{Schnitzler, Arthur}!zzzSalten, Felix@\emph{von Felix Salten}!1901-06-121@{12. 6. 1901}|(be}
\toendnotes[C]{\smallbreak\pagebreak[2]}
\correspDesc{Versand  durch Felix Salten am 12. 6. 1901 in Wien
\newline{}Umleitung  am 12. 6. 1901 in Wien
\newline{}Erhalt  durch Arthur Schnitzler im Zeitraum [14. 6. 1901
                  – 17. 6. 1901?] in Berchtesgaden}\toendnotes[C]{\smallbreak}
\Standort{CUL, Schnitzler, B 89, A 2.}
\physDesc{Brief, 1 Blatt, 2 Seiten, 1704 Zeichen
\newline{}Handschrift: schwarze Tinte, lateinische Kurrent
\newline{}Ordnung: mit Bleistift von unbekannter Hand nummeriert: »137« }
\buchAbdrucke{\weitereDrucke{Hermann Bahr, Arthur Schnitzler: \emph{Briefwechsel, Aufzeichnungen, Dokumente (1891–1931)}. Herausgegeben von Kurt Ifkovits und Martin Anton Müller. Göttingen: \emph{Wallstein} 2018, S. 204–205.} }\toendnotes[C]{\smallbreak}
\pstart
           {\pb}\textcolor{gray}{\textbf{Jung-Wiener Theater\orgindex{Jung-Wiener Theater zum Lieben Augustin@Jung-Wiener Theater zum Lieben Augustin|pw}}}\hfill \textcolor{gray}{\textbf{Wien\oindex{Wien@\textbf{Wien}, \emph{Verwaltungsgebiet}|pw},}}{ }12. Juni \textcolor{gray}{\textbf{190}}1\pend
           
\pstart
           \textcolor{gray}{\textbf{Zum lieben Augustin\orgindex{Jung-Wiener Theater zum Lieben Augustin@Jung-Wiener Theater zum Lieben Augustin|pw}.}}\hfill \textcolor{gray}{\textbf{(Theater a. d.
                        Wien\oindex{Wien@\textbf{Wien}!VI., Mariahilf@\textbf{VI., Mariahilf}!Theater an der Wien@\textbf{Theater an der Wien}, \emph{Theater}|pw})}}\pend
           
\pstart
           \textcolor{gray}{\textbf{Direction.}}\pend
           \vspace{0.5em}
\pstart
           Lieber Freund, es thut mir leid, dass ich Sie nicht mehr gesprochen
               habe. Bis Sonntag war ich verreist, Karlsbad\oindex{Karlsbad@\textbf{Karlsbad}|pw}{ }Prag\oindex{Prag@\textbf{Prag}, \emph{Land}|pw}. Habe in Prag\oindex{Prag@\textbf{Prag}, \emph{Land}|pw} Frl. \label{K_L03313-1v}\edtext{Bardi\pwindex{Bardi, Marie *~um 1870@\textsc{Bardi, Marie} (*~um 1870), \emph{Schauspielerin}|pw} und einen hübschen jungen Tenor\pwindex{Barkay, Arnold 1873 – 1938 Wien@\textsc{Barkay, Arnold} (1873 – 1938 Wien), \emph{Sänger, Humorist}|pwv} engagirt}{\lemma{\textnormal{\emph{Bardi … engagirt}}}\Cendnote{\textnormal{Marie Bardi\pwindex{Bardi, Marie *~um 1870@\textsc{Bardi, Marie} (*~um 1870), \emph{Schauspielerin}|pwk} und Arnold Barkay\pwindex{Barkay, Arnold 1873 – 1938 Wien@\textsc{Barkay, Arnold} (1873 – 1938 Wien), \emph{Sänger, Humorist}|pwk}. Die Engagements erfolgten im Hinblick auf
                  das \emph{Jung-Wiener Theater zum lieben Augustin}\orgindex{Jung-Wiener Theater zum Lieben Augustin@Jung-Wiener Theater zum Lieben Augustin|pwk},
                  das Salten\pwindex{Salten, Felix 6.\,9.\,1869 Budapest – 8.\,10.\,1945 Zürich@\textsc{Salten, Felix} (6.\,9.\,1869 Budapest – 8.\,10.\,1945 Zürich), \emph{Schriftsteller, Journalist, Chefredakteur}|pwk} zu dieser Zeit vorbereitete.
               }}}\label{K_L03313-1}, der die größte Ambition hat, ein \label{K_L03313-2v}\edtext{Sven Skolander\pwindex{Scholander, Sven 21.\,4.\,1860 Stockholm – 14.\,12.\,1936 Djursholm@\textsc{Scholander, Sven} (21.\,4.\,1860 Stockholm – 14.\,12.\,1936 Djursholm), \emph{Sänger}|pw}}{\lemma{\textnormal{\emph{Sven Skolander}}}\Cendnote{\textnormal{Sven Scholander\pwindex{Scholander, Sven 21.\,4.\,1860 Stockholm – 14.\,12.\,1936 Djursholm@\textsc{Scholander, Sven} (21.\,4.\,1860 Stockholm – 14.\,12.\,1936 Djursholm), \emph{Sänger}|pwk} war ein erfolgreicher schwed\oindex{Schweden@\textbf{Schweden}|pwkv}ischer Sänger.}}}\label{K_L03313-2}
               zu werden. Von D\textsuperscript{r}{ }Mandl\pwindex{Mandl, Ludwig 19.\,8.\,1862 Iași – 22.\,9.\,1937 Wien@\textsc{Mandl, Ludwig} (19.\,8.\,1862 Iași – 22.\,9.\,1937 Wien), \emph{Gynäkologe}|pw} haben Sie gehört, dass \label{K_L03313-3v}\edtext{Otti\pwindex{Salten, Ottilie 7.\,3.\,1868 Prag – 22.\,6.\,1942 Zürich@\textsc{Salten, Ottilie} (7.\,3.\,1868 Prag – 22.\,6.\,1942 Zürich), \emph{Schauspielerin}|pw} operirt}{\lemma{\textnormal{\emph{Otti operirt}}}\Cendnote{\textnormal{Ludwig Mandl\pwindex{Mandl, Ludwig 19.\,8.\,1862 Iași – 22.\,9.\,1937 Wien@\textsc{Mandl, Ludwig} (19.\,8.\,1862 Iași – 22.\,9.\,1937 Wien), \emph{Gynäkologe}|pwk} war Gynäkologe. Welcher
                  Eingriff vorgenommen wurde, konnte nicht ermittelt werden.}}}\label{K_L03313-3} wurde. Das war
               ziemlich schrecklich, obwol die ganze Sache an sich ja nichts bedeutet und glücklich
               verlaufen ist. Ich bleibe nun ungefähr acht Tage in Wien\oindex{Wien@\textbf{Wien}, \emph{Verwaltungsgebiet}|pw} und fahre dann nach München\oindex{München@\textbf{München}|pw}, zwei
               Tage, von dort nach Zürich\oindex{Zürich@\textbf{Zürich}|pw}, drei Tage, (\label{K_L03313-4v}\edtext{Felix\pwindex{Felix, Hugo 19.\,11.\,1866 Budapest – 25.\,8.\,1934 Hollywood@\textsc{Felix, Hugo} (19.\,11.\,1866 Budapest – 25.\,8.\,1934 Hollywood), \emph{Komponist, Chemiker}|pw}}{\lemma{\textnormal{\emph{Felix}}}\Cendnote{\textnormal{Hugo Felix\pwindex{Felix, Hugo 19.\,11.\,1866 Budapest – 25.\,8.\,1934 Hollywood@\textsc{Felix, Hugo} (19.\,11.\,1866 Budapest – 25.\,8.\,1934 Hollywood), \emph{Komponist, Chemiker}|pwk}}}}\label{K_L03313-4}) von da nach Paris\oindex{Paris@\textbf{Paris}, \emph{Hauptstadt}|pw}, zwölf – 14 Tage und
               d’dann nach Köln\oindex{Köln@\textbf{Köln}, \emph{Hauptstadt}|pw}, Frankfurt\oindex{Frankfurt am Main@\textbf{Frankfurt am Main}, \emph{Hauptstadt}|pw}, Wiesbaden\oindex{Wiesbaden@\textbf{Wiesbaden}|pw}, Stuttgart\oindex{Stuttgart@\textbf{Stuttgart}|pw} – Wien\oindex{Wien@\textbf{Wien}, \emph{Verwaltungsgebiet}|pw}. Im Juli werde ich im Salzkammergut\oindex{Salzkammergut@\textbf{Salzkammergut}, \emph{Region}|pw} oder am Wörthersee\oindex{Wörthersee@\textbf{Wörthersee}, \emph{See}|pw} sein.
               Auch zu einer kleinen Radtour wäre ich bereit. Den größten Theil des August bin ich in Wien\oindex{Wien@\textbf{Wien}, \emph{Verwaltungsgebiet}|pw},
               mit Ausnahme einer kurzen Reise nach Prag\oindex{Prag@\textbf{Prag}, \emph{Land}|pw} und
               nach Aussee\oindex{Bad Aussee@\textbf{Bad Aussee}, \emph{Hauptstadt}|pw}. Das ist Alles. Ich freue mich, dass
               Sie ein neues \label{K_L03313-5v}\edtext{Stück\pwindex{Schnitzler, Arthur 15.\,5.\,1862 Wien – 21.\,10.\,1931 ebd.@\textsc{Schnitzler, Arthur} (15.\,5.\,1862 Wien – 21.\,10.\,1931 ebd.), \emph{Schriftsteller, Mediziner}!einsame Weg. Schauspiel in fünf Akten@\strich\emph{Der einsame Weg. Schauspiel in fünf Akten}|pw}}{\lemma{\textnormal{\emph{Stück}}}\Cendnote{\textnormal{Siehe XXXX Auszeichnungsfehler: Dokument L03038 nicht gefunden. }}}\label{K_L03313-5} haben, und
               hege künstlerisch eine ganz bestimmte Erwartung davon. Vielleicht läßt es sich
               machen, das{[}s{]}{ }\label{K_L03313-6v}\edtext{Bukovics\pwindex{Bukovics, Emerich von 28.\,2.\,1844 Wien – 4.\,7.\,1905 ebd.@\textsc{Bukovics, Emerich von} (28.\,2.\,1844 Wien – 4.\,7.\,1905 ebd.), \emph{Journalist, Theaterleiter}|pw} mir die »Marionetten\pwindex{Schnitzler, Arthur 15.\,5.\,1862 Wien – 21.\,10.\,1931 ebd.@\textsc{Schnitzler, Arthur} (15.\,5.\,1862 Wien – 21.\,10.\,1931 ebd.), \emph{Schriftsteller, Mediziner}!Zum großen Wurstel. Burleske in einem Akt@\strich\emph{Zum großen Wurstel. Burleske in einem Akt}|pw}« abtritt}{\lemma{\textnormal{\emph{Bukovics … abtritt}}}\Cendnote{\textnormal{Gemeint ist die später als \emph{Zum großen Wurstel}\pwindex{Schnitzler, Arthur 15.\,5.\,1862 Wien – 21.\,10.\,1931 ebd.@\textsc{Schnitzler, Arthur} (15.\,5.\,1862 Wien – 21.\,10.\,1931 ebd.), \emph{Schriftsteller, Mediziner}!Zum großen Wurstel. Burleske in einem Akt@\strich\emph{Zum großen Wurstel. Burleske in einem Akt}|pwk} geführte Burleske. Zu einer Übernahme durch das \emph{Jung-Wiener Theater zum lieben Augustin}\orgindex{Jung-Wiener Theater zum Lieben Augustin@Jung-Wiener Theater zum Lieben Augustin|pwk} kam es
                  nicht. Auch am \emph{Volkstheater}\orgindex{Volkstheater@Volkstheater|pwk} wurde es nicht
                  gegeben, vgl. Hermann Bahr, Arthur Schnitzler: \emph{Briefwechsel, Aufzeichnungen, Dokumente (1891–1931)}, Arthur Schnitzler an Emerich von Bukovics, 11. 12. 1901.}}}\label{K_L03313-6},
               d. h. wenn Sie mir das Stück\pwindex{Schnitzler, Arthur 15.\,5.\,1862 Wien – 21.\,10.\,1931 ebd.@\textsc{Schnitzler, Arthur} (15.\,5.\,1862 Wien – 21.\,10.\,1931 ebd.), \emph{Schriftsteller, Mediziner}!Zum großen Wurstel. Burleske in einem Akt@\strich\emph{Zum großen Wurstel. Burleske in einem Akt}|pwv}
               geben wollen. Schrei{\pb}ben Sie mir
               darüber. Brahm\pwindex{Brahm, Otto 5.\,2.\,1856 Hamburg – 28.\,11.\,1912 Berlin@\textsc{Brahm, Otto} (5.\,2.\,1856 Hamburg – 28.\,11.\,1912 Berlin), \emph{Theaterleiter, Regisseur}|pw} ist, wie Sie wissen, hier\oindex{Wien@\textbf{Wien}, \emph{Verwaltungsgebiet}|pwv}. Wir sahen uns im Theater,
               ohne uns zu grüßen. Es ist mir ja sonst ganz gleichgiltig, aber ich bereue jetzt,
               dass ich mich \label{K_L03313-7v}\edtext{s. Z.}{\lemma{\textnormal{\emph{s. Z.}}}\Cendnote{\textnormal{seiner Zeit}}}\label{K_L03313-7} doch habe bereden laßen,
               ihm mein \label{K_L03313-8v}\edtext{Stück\pwindex{Salten, Felix 6.\,9.\,1869 Budapest – 8.\,10.\,1945 Zürich@\textsc{Salten, Felix} (6.\,9.\,1869 Budapest – 8.\,10.\,1945 Zürich), \emph{Schriftsteller, Journalist, Chefredakteur}!Gemeine. Schauspiel in drei Aufzügen@\strich\emph{Der Gemeine. Schauspiel in drei Aufzügen}|pw}}{\lemma{\textnormal{\emph{Stück}}}\Cendnote{\textnormal{ In Schnitzlers Korrespondenz mit Brahm\pwindex{Brahm, Otto 5.\,2.\,1856 Hamburg – 28.\,11.\,1912 Berlin@\textsc{Brahm, Otto} (5.\,2.\,1856 Hamburg – 28.\,11.\,1912 Berlin), \emph{Theaterleiter, Regisseur}|pwk} ist die Einreichung von \emph{Der
                     Gemeine}\pwindex{Salten, Felix 6.\,9.\,1869 Budapest – 8.\,10.\,1945 Zürich@\textsc{Salten, Felix} (6.\,9.\,1869 Budapest – 8.\,10.\,1945 Zürich), \emph{Schriftsteller, Journalist, Chefredakteur}!Gemeine. Schauspiel in drei Aufzügen@\strich\emph{Der Gemeine. Schauspiel in drei Aufzügen}|pwk} nicht thematisiert.}}}\label{K_L03313-8} einzureichen. Nun bringt er mich durch
               sein Benehmen in den peinlichen Verdacht, als sei ich ihm \uline{deshalb} böse. Ich bin ihm aber garnicht böse, am wenigsten deshalb. Nur sehe
               ich keine Ursache, sein unfreundliches Verhalten einzustecken.\pend
           
\pstart
           Von Bahr\pwindex{Bahr, Hermann 19.\,7.\,1863 Linz – 15.\,1.\,1934 München@\textsc{Bahr, Hermann} (19.\,7.\,1863 Linz – 15.\,1.\,1934 München), \emph{Schriftsteller, Kritiker}|pw} erfuhr ich, dass \label{K_L03313-9v}\edtext{Hofmannsthal\pwindex{Hofmannsthal, Hugo von 1.\,2.\,1874 Wien – 15.\,7.\,1929 Rodaun@\textsc{Hofmannsthal, Hugo von} (1.\,2.\,1874 Wien – 15.\,7.\,1929 Rodaun), \emph{Schriftsteller}|pw}{ }Samstag geheirathet}{\lemma{\textnormal{\emph{Hofmannsthal … geheirathet}}}\Cendnote{\textnormal{Hugo von Hofmannsthal\pwindex{Hofmannsthal, Hugo von 1.\,2.\,1874 Wien – 15.\,7.\,1929 Rodaun@\textsc{Hofmannsthal, Hugo von} (1.\,2.\,1874 Wien – 15.\,7.\,1929 Rodaun), \emph{Schriftsteller}|pwk} und Gertrude (Gerty) Schlesinger\pwindex{Hofmannsthal, Gertrude von 16.\,3.\,1880 Wien – 9.\,11.\,1959 Paddington@\textsc{Hofmannsthal, Gertrude von} (16.\,3.\,1880 Wien – 9.\,11.\,1959 Paddington)|pwk} heirateten am 8. 6. 1901.}}}\label{K_L03313-9} hat. Schreiben Sie mir, bitte,
               bald. Hauptsächlich, wohin Sie reisen. Ich habe das \label{K_L03313-10v}\edtext{»wir«}{\lemma{\textnormal{\emph{»wir«}}}\Cendnote{\textnormal{Schnitzler reiste mit Olga Gussmann\pwindex{Schnitzler, Olga 17.\,1.\,1882 Wien – 13.\,1.\,1970 Lugano@\textsc{Schnitzler, Olga} (17.\,1.\,1882 Wien – 13.\,1.\,1970 Lugano), \emph{Schauspielerin, Sängerin}|pwk}.}}}\label{K_L03313-10} nicht verstanden. Sind Sie mit Ihrer
                  Mama\pwindex{Schnitzler, Louise 8.\,7.\,1840 Kőszeg – 9.\,9.\,1911 Wien@\textsc{Schnitzler, Louise} (8.\,7.\,1840 Kőszeg – 9.\,9.\,1911 Wien)|pwv}?\pend
           
\pstart
           herzlichst {\\[\baselineskip]}Ihr {\\[\baselineskip]}\spacefill\mbox{Salten}\pend
           \leftskip=0em{}\selectlanguage{ngerman}\endnumbering\briefempfaengerindex{Schnitzler, Arthur@\textsc{Schnitzler, Arthur}!zzzSalten, Felix@\emph{von Felix Salten}!1901-06-121@{12. 6. 1901}|)be}\mylabel{L03313h}  \newcommand{\dateiname}{L03313}\newcommand{\titel}{Felix Salten an Arthur Schnitzler, 12. 6. 1901}\newcommand{\editorInnen}{Martin Anton Müller und Laura Untner}%% latex-leseansicht-abspann.tex
%% Abspann für die Leseansicht.
%% Der Schalter \ifkorrekturansicht ist bereits durch den Vorspann gesetzt.

%% latex-abspann.tex
%% Gemeinsamer Abspann für Korrekturansicht und Leseansicht.
%% Setzt den Schalter \ifkorrekturansicht voraus (gesetzt in den
%% einbindenden Dateien latex-korrekturansicht-abspann.tex bzw.
%% latex-leseansicht-abspann.tex).
%% ---------------------------------------------------------------

\normalsize

% Das esempio-Environment wird nur in der Leseansicht benötigt
\ifkorrekturansicht\else
\newenvironment{esempio}[3]%
{
    \vspace{1.5ex}
    \rlap{\underline{#1}}
    \par
    \setlength{\parindent}{0cm}
    \nopagebreak
    \leftskip=#2cm
    \rightskip=#3cm
}
{
    \par
}
\fi

\doendnotes{C}
\bigskip
\vfill

\clearpage

\footnotesize

\ifkorrekturansicht
  \lohead{\textsc{register}}
\fi

% theindex-Environment neu definieren ohne reledmac
\makeatletter
\renewenvironment{theindex}{%
  \ifkorrekturansicht
    \section*{\indexname}%
  \else
    \subsubsection*{Index der erwähnten Entitäten}%
  \fi
  \setlength{\parindent}{0pt}%
  \setlength{\parskip}{0pt plus 0.3pt}%
  \let\item\@idxitem
}{%
  \ifkorrekturansicht\clearpage\fi
}
\makeatother

\IfFileExists{\jobname-pw.ind}{\input{\jobname-pw.ind}}{}

% Quellenangabe nur in der Leseansicht
\ifkorrekturansicht\else
% Fallback-Definitionen, falls die .tex-Datei \titel etc. nicht gesetzt hat
\providecommand{\titel}{}
\providecommand{\editorInnen}{}
\providecommand{\dateiname}{\jobname}

\vspace{3cm}

\vfill

\footnotesize
\textsc{Quelle}: \titel. Herausgegeben von {\editorInnen}. In: \emph{Arthur Schnitzler: Briefwechsel mit Autorinnen und Autoren}.
 Digitale Edition, https://schnitzler-briefe.acdh.oeaw.ac.at/{\dateiname}.html (Stand \today)
\fi

\end{document}


