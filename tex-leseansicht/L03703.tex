%% latex-korrekturansicht-vorspann.tex
%% Vorspann für die Korrekturansicht.
%% Lädt die gemeinsame Datei latex-vorspann.tex mit gesetztem Schalter.

\newif\ifkorrekturansicht
\korrekturansichttrue

\input{../tex-inputs/latex-vorspann}


\section[Elsa Plessner an Arthur Schnitzler, 21. 9. 1896]{L03703 Elsa Plessner an Arthur Schnitzler, 21. 9. 1896}
\nopagebreak\mylabel{L03703v}
\rehead{ }\normalsize\beginnumbering\briefempfaengerindex{Schnitzler, Arthur@\textsc{Schnitzler, Arthur}!zzzPlessner, Elsa@\emph{von Elsa Plessner}!1896-09-213@{21. 9. 1896}|(be}
\toendnotes[C]{\smallbreak\pagebreak[2]}\Standort{DLA, A:Schnitzler, HS.1985.1.419.}
\physDesc{Brief,  Blätter, 3 Seiten, 2784 Zeichen
\newline{}Handschrift: , lateinische Kurrent}\toendnotes[C]{\smallbreak}
\pstart
           {\pb}I. Bäckerstrasse N\textsuperscript{o}
                     1\oindex{Baeckerstrasse 1@\textbf{Bäckerstraße 1}, \emph{Wohngebäude (K.WHS)}|pw}, den 21. 9. 96. \pend
           
\pstart{}Verehrter Herr Doctor!\pend\vspace{0.5em}
\pstart
           Mit dem Tage, der eben schließt, sind Sie zum \uline{Erzengel} avancirt. –\pend
           
\pstart
           Herzlichsten, aufrichtigsten Dank für die Geduld und Aufmerksamkeit die Sie meinen
                  \label{K_L03703-1v}\edtext{Arbeiten\pwindex{glaeserne Kaefig. Skizzen und Novellen@\emph{Der gläserne Käfig. Skizzen und Novellen}|pwv}}{\lemma{\textnormal{\emph{Arbeiten}}}\Cendnote{\textnormal{Elsa Plessners\pwindex{Plessner, Elsa 22.08.1875 – 01.05.1932@\textsc{Plessner, Elsa} (22.08.1875 – 01.05.1932), \emph{Schriftsteller/Schriftstellerin}|pwk} Band \emph{Der gläserne Käfig}\pwindex{glaeserne Kaefig. Skizzen und Novellen@\emph{Der gläserne Käfig. Skizzen und Novellen}|pwk} mit vierzehn Novellen und Skizzen
                  erschien 1901. Welche zehn Texte daraus sie in welcher Reihenfolge
                     Schnitzler mit dem vorangegangenen Brief
                  geschickt hatte, läßt sich nur zum Teil rekonstruieren. Sicher dabei waren die
                  Skizzen \emph{Warten}\pwindex{Warten@\emph{Warten}|pwk}, \emph{Der Selbstmörder}\pwindex{Leiter der Seele@\emph{Die Leiter der Seele}|pwk}, \emph{Begräbnißtag}\pwindex{Begraebnisstag@\emph{Der Begräbnißtag}|pwk}, \emph{Im Feuer geprüft}\pwindex{Im Feuer geprueft@\emph{Im Feuer geprüft}|pwk} und
                     \emph{Im Widerschein}\pwindex{Im Widerschein@\emph{Im Widerschein}|pwk}.}}}\label{K_L03703-1} zugewendet haben.
               Diese Liebenswürdigkeit, die Sie mir gegenüber so oft schon bethätigten ist so
               beispiellos, daß mir jeder Ausdruck fehlt, sie näher zu characterisieren! Sie werden
               zwar sagen: »Schlamperei! Man muß alle Ausdrücke finden!« Ich bin aber wieder so
               empörend faul, nicht lange darüber nachzudenken! »Wie gesagt« – Sie sind ein Engel in
               xter Potenz! – Geradezu fabelhaft finde ich es, daß sie die {\pb}schöne Zeit,
               die Sie zu so vielem Anderen hätten verwenden können, zur Anfertigung der \label{K_L03703-2v}\edtext{graziösen
               Excerpte}{\lemma{\textnormal{\emph{graziösen
               Excerpte}}}\Cendnote{\textnormal{Schnitzlers Antwortbrief und seine Redaktionsarbeiten an den eingesandten Texten\pwindex{glaeserne Kaefig. Skizzen und Novellen@\emph{Der gläserne Käfig. Skizzen und Novellen}|pwkv} sind nicht überliefert.}}}\label{K_L03703-2} aus meinen Meisterwerken\pwindex{glaeserne Kaefig. Skizzen und Novellen@\emph{Der gläserne Käfig. Skizzen und Novellen}|pwv} geopfert haben! Wie werde ich das \introOben{}vor\introOben{}
               der deutschen\oindex{Deutschland@\textbf{Deutschland}, \emph{A.PCLI}|pw} Literatur verantworten können? –
               Übrigens, verehrter Meister \label{K_L03703-3v}\edtext{Anatol}{\lemma{\textnormal{\emph{Anatol}}}\Cendnote{\textnormal{Bezugnahme auf Arthur Schnitzlers Einakter-Zyklus \emph{Anatol}\pwindex{Anatol@\emph{Anatol}|pwk} und den gleichnamigen
                  Protagonisten}}}\label{K_L03703-3} – Sie haben mir zu den Kopf gewaschen, daß mir alle
               Haarwurzeln weh thuen und, – – mit Recht!!! – Alle die Abscheulichkeiten, die ich
               verbrochen, haben Sie mir in einem so lieblichen Neben- und Nacheinander vor mein
               jetzt gänzlich zerschmettertes literarisches Gewissen geführt – – – \label{K_L03703-4v}\edtext{mea culpa}{\lemma{\textnormal{\emph{mea culpa}}}\Cendnote{\textnormal{latein: durch meine Schuld}}}\label{K_L03703-4}!–
               Eines aber freut mich riesig – dass No 1.\pwindex{Warten@\emph{Warten}|pwv} (jetzt »Warten\pwindex{Warten@\emph{Warten}|pw}« früher »Blätter«)
               Ihnen nun doch ein wenig gefällt! Denn das ist die einzige Arbeit, an der mir etwas
               liegt und auch – meine letzte!! Überhaupt finde {\pb}ich zu meinem großem
               Vergnügen, daß Sie alle die Arbeiten für die relativ besten erklären, die richtig
               jüngeres Datum tragen als die andern. \label{K_L03703-5v}\edtext{Der Onkel\pwindex{Onkel@\emph{Der Onkel}|pw}}{\lemma{\textnormal{\emph{Der Onkel}}}\Cendnote{\textnormal{Die hier genannten Texte \emph{Der Onkel}\pwindex{Onkel@\emph{Der Onkel}|pwk}, \emph{Sie
                     gähnt}\pwindex{Sie gaehnt@\emph{Sie gähnt}|pwk} und \emph{Eile}\pwindex{Eile@\emph{Eile}|pwk} sind nicht unter
                  diesen Titeln in den Band \emph{Der gläserne Käfig}\pwindex{glaeserne Kaefig. Skizzen und Novellen@\emph{Der gläserne Käfig. Skizzen und Novellen}|pwk}
                  aufgenommen worden. Es ist aber gut möglich, dass es sich um frühe Versionen
                  später umbenannter Texte handelt. Elsa
                     Plessne\pwindex{Plessner, Elsa 22.08.1875 – 01.05.1932@\textsc{Plessner, Elsa} (22.08.1875 – 01.05.1932), \emph{Schriftsteller/Schriftstellerin}|pwk}r betont in ihren Briefen an Schnitzler mehrfach, dass ihr schmales Werk nichts über die später
                  publizierten Texte hinaus enthalte, vgl. Elsa Plessner an Arthur Schnitzler, 12. 10. 1900.}}}\label{K_L03703-5}, das Monstrum von Geschmacklosigkeit,
               ist aus dem Jahre \uuline{93} – sowie auch »Sie gähnt\pwindex{Sie gaehnt@\emph{Sie gähnt}|pw}« ungefähr so
               alt ist. Was Sie von »Eile\pwindex{Eile@\emph{Eile}|pw}« schreiben, kann ich
               eigentlich nicht begreifen! Die zehn Skizzen\pwindex{glaeserne Kaefig. Skizzen und Novellen@\emph{Der gläserne Käfig. Skizzen und Novellen}|pwv} und das Stück\pwindex{Heimweh [dreiaktige Tragikomoedie]@\emph{Heimweh [dreiaktige Tragikomödie]}|pwv}, sowie die »freien
                  Rhythmen\pwindex{Pierettes Tagebuch [19 unveroeffentlichte Gedichte]@\emph{Pierettes Tagebuch [19 unveröffentlichte Gedichte]}|pw}{[}«{]}, die Sie seinerzeit so wüthend gemacht haben, sind meine ganze
               gesammte Production von – 9 Jahren!! – Das ist doch nicht viel? – Mir sind die alten
               Sachen zu in der Seele zuwider, daß ich am liebsten gar nichts davon mehr wissen
               wollte – soll ich da wirklich noch lange in dem alten Kehricht herumstöbern? – Wenn
               ich nicht \uuline{müsste} – so ließe ich sie wirklich nicht
               aus Tageslicht – doch so? – Ich werde die Blößen der armen Kinder nothdürftig
               bedecken, so von oben auf nach Ihren Angaben und dann – fort mit Schaden – ! Für die
               Zukunft verspreche und gelobe ich, nach Ihren Directiven anständig und ehrlich zu
               arbeiten, nichts mehr zu schleudern, und im übrigen auf mein Talent, das Sie ja »mit
               einem heitern, einem nassen Auge« anerkennen, zu bauen. – – – Darf ich mir die
               Anfrage gestatten, was ich nun betreffs Director Brahm\pwindex{Brahm, Otto 05.02.1856 – 28.11.1912@\textsc{Brahm, Otto} (05.02.1856 – 28.11.1912), \emph{Theaterleiter/Theaterleiterin, Regisseur/Regisseurin}|pw} thun soll? – ihm ein Abschrift meines Stückes\pwindex{Heimweh [dreiaktige Tragikomoedie]@\emph{Heimweh [dreiaktige Tragikomödie]}|pwv}{ }\introOben{}senden\introOben{} mit gleichzeitiger Bezugnahme auf Sie, verehrter
               Meister? – – – Oder erst nach eventueller Antwort diesbezüglich von dort an Sie? – – \pend
           \pstart Mit Dank und Verehrung grüßt \spacefill\mbox{Elsa Plessner}\pend{}\selectlanguage{ngerman}\endnumbering\briefempfaengerindex{Schnitzler, Arthur@\textsc{Schnitzler, Arthur}!zzzPlessner, Elsa@\emph{von Elsa Plessner}!1896-09-213@{21. 9. 1896}|)be}\mylabel{L03703h}
\begin{anhang}
\end{anhang}\normalsize

\doendnotes{C}
\bigskip
\vfill

\clearpage

\footnotesize

\lohead{\textsc{register}}

% Definiere theindex-Environment komplett neu ohne reledmac
\makeatletter
\renewenvironment{theindex}{%
  \section*{\indexname}%
  \setlength{\parindent}{0pt}%
  \setlength{\parskip}{0pt plus 0.3pt}%
  \let\item\@idxitem
}{%
  \clearpage
}
\makeatother

\IfFileExists{\jobname-pw.ind}{\input{\jobname-pw.ind}}{}

\end{document}

      