%% latex-korrekturansicht-vorspann.tex
%% Vorspann für die Korrekturansicht.
%% Lädt die gemeinsame Datei latex-vorspann.tex mit gesetztem Schalter.

\newif\ifkorrekturansicht
\korrekturansichttrue

\input{../tex-inputs/latex-vorspann}


\section[Stefan Zweig an Arthur Schnitzler, 26. 7. 1929]{L03675 Stefan Zweig an Arthur Schnitzler, 26. 7. 1929}
\nopagebreak\mylabel{L03675v}
\rehead{ }\normalsize\beginnumbering\briefempfaengerindex{Schnitzler, Arthur@\textsc{Schnitzler, Arthur}!zzzZweig, Stefan@\emph{von Stefan Zweig}!1929-07-261@{26. 7. 1929}|(be}
\toendnotes[C]{\smallbreak\pagebreak[2]}\Standort{CUL, Schnitzler, B 118.}
\physDesc{Briefkarte, 1 Blatt, 2 Seiten, 725 Zeichen
\newline{}Handschrift: lila Tinte, lateinische Kurrent
\newline{}Schnitzler: mit rotem Buntstift fünf Unterstreichungen }
\buchAbdrucke{\weitereDrucke{Stefan Zweig: \emph{Briefwechsel mit Hermann Bahr, Sigmund Freud, Rainer Maria
                        Rilke und Arthur Schnitzler}. Frankfurt am Main: \emph{S. Fischer} 1987, S. 444–445.} }\toendnotes[C]{\smallbreak}
\pstart
           \raggedleft{}{\pb}26. Juli 29\pend
           
\pstart
           \textcolor{gray}{\textbf{SZ}}\hfill \textcolor{gray}{\textbf{SALZBURG\oindex{Salzburg@\textbf{Salzburg}, \emph{A.ADM2}|pw}}}\pend
           
\pstart
           \raggedleft{}\textcolor{gray}{\textbf{KAPUZINERBERG 5\oindex{Paschinger Schloessl@\textbf{Paschinger Schlössl}, \emph{Wohngebäude (K.WHS)}|pw},}}\pend
           \vspace{0.5em}
\pstart
           Verehrter und lieber Herr Doktor, ich habe (hoffentlich zum
               erstenmale in unsern Beziehungen!) einen kleinen Verstoss gegen die guten Sitten
               begangen. Aber die innere Gesinnung darf da wohl Pardon erbitten. Mich kränkte es
               nämlich seit langem, dass ich nie die rechte Gelegenheit fand, meine Verehrung und
               Liebe für Sie öffentlich kund\substVorne{}\textsuperscript{ge}\substDazwischen{}zu\substHinten{}geben. So habe ich Ihren Namen auf das \label{K_L03675-1v}\edtext{Widmungsblatt}{\lemma{\textnormal{\emph{Widmungsblatt}}}\Cendnote{\textnormal{Die
                  Widmung lautet: »Arthur Schnitzler in
                     liebender Verehrung«.}}}\label{K_L03675-1} meines Fouché-Buches\pwindex{Joseph Fouche. Bildnis eines politischen Menschen@\emph{Joseph Fouché. Bildnis eines politischen Menschen}|pw} drucken {\pb}lassen,
               ohne das Geziemende zu tun: Sie voraus anzufragen, ob sie diese Widmung annehmen
               wollen. Nun, ich denke Sie werden mir diesen kleinen Verstoss verzeihen und nicht die
               Auflage einstampfen lassen, nur weil sie meine redliche Liebe zu Ihnen öffentlich
               bekennt.\pend
           
\pstart
           In Treue ergeben Ihr{\\[\baselineskip]}\spacefill\mbox{Stefan Zweig}\pend
           \leftskip=0em{}\selectlanguage{ngerman}\endnumbering\briefempfaengerindex{Schnitzler, Arthur@\textsc{Schnitzler, Arthur}!zzzZweig, Stefan@\emph{von Stefan Zweig}!1929-07-261@{26. 7. 1929}|)be}\mylabel{L03675h}
\begin{anhang}
\end{anhang}\normalsize

\doendnotes{C}
\bigskip
\vfill

\clearpage

\footnotesize

\lohead{\textsc{register}}

% Definiere theindex-Environment komplett neu ohne reledmac
\makeatletter
\renewenvironment{theindex}{%
  \section*{\indexname}%
  \setlength{\parindent}{0pt}%
  \setlength{\parskip}{0pt plus 0.3pt}%
  \let\item\@idxitem
}{%
  \clearpage
}
\makeatother

\IfFileExists{\jobname-pw.ind}{\input{\jobname-pw.ind}}{}

\end{document}

      