%% latex-leseansicht-vorspann.tex
%% Vorspann für die Leseansicht.
%% Lädt die gemeinsame Datei latex-vorspann.tex mit nicht gesetztem Schalter.

\newif\ifkorrekturansicht
\korrekturansichtfalse

\input{../tex-inputs/latex-vorspann}


\section[Stefan Zweig an Arthur Schnitzler, 26. 7. 1929]{L03675 Stefan Zweig an Arthur Schnitzler, 26. 7. 1929}
\nopagebreak\mylabel{L03675v}
\rehead{ }\normalsize\beginnumbering\briefempfaengerindex{Schnitzler, Arthur@\textsc{Schnitzler, Arthur}!zzzZweig, Stefan@\emph{von Stefan Zweig}!1929-07-261@{26. 7. 1929}|(be}
\toendnotes[C]{\smallbreak\pagebreak[2]}
\correspDesc{Versand  durch Stefan Zweig am 26. 7. 1929 in Salzburg
\newline{}Erhalt  durch Arthur Schnitzler im Zeitraum [27. 7. 1929
                  – 31. 7. 1929?] in Wien}\toendnotes[C]{\smallbreak}
\Standort{CUL, Schnitzler, B 118.}
\physDesc{Briefkarte, 723 Zeichen
\newline{}Handschrift: lila Tinte, lateinische Kurrent
\newline{}Schnitzler: mit rotem Buntstift fünf Unterstreichungen }
\buchAbdrucke{\weitereDrucke{Stefan Zweig: \emph{Briefwechsel mit Hermann Bahr, Sigmund Freud, Rainer Maria
                        Rilke und Arthur Schnitzler}. Herausgegeben von Jeffrey B. Berlin, Hans-Ulrich Lindken und Donald A. Prater. Frankfurt am Main: \emph{S. Fischer} 1987, S. 444–445.} }\toendnotes[C]{\smallbreak}
\pstart
           \raggedleft{}{\pb}26. Juli 29\pend
           
\pstart
           \textcolor{gray}{\textbf{SZ}}\hfill \textcolor{gray}{\textbf{SALZBURG\oindex{Salzburg@\textbf{Salzburg}, \emph{Verwaltungsgebiet}|pw}}}\pend
           
\pstart
           \raggedleft{}\textcolor{gray}{\textbf{KAPUZINERBERG 5\oindex{Paschinger Schlössl@\textbf{Paschinger Schlössl}, \emph{Wohngebäude}|pw}}}\pend
           \vspace{0.5em}
\pstart
           Verehrter und lieber Herr Doktor, ich habe (hoffentlich zum
               erstenmale in unsern Beziehungen!) einen kleinen Verstoss gegen die guten Sitten
               begangen. Aber die innere Gesinnung darf da wohl Pardon erbitten. Mich kränkte es
               nämlich seit langem, dass ich nie die rechte Gelegenheit fand, meine Verehrung und
               Liebe für Sie öffentlich kund\substVorne{}\textsuperscript{ge}\substDazwischen{}zu\substHinten{}geben. So habe ich Ihren Namen auf das \label{K_L03675-1v}\edtext{Widmungsblatt}{\lemma{\textnormal{\emph{Widmungsblatt}}}\Cendnote{\textnormal{Die
                  Widmung lautet: »Arthur Schnitzler in
                     liebender Verehrung«.}}}\label{K_L03675-1} meines Fouché-Buches\pwindex{Zweig, Stefan 28.\,11.\,1881 Wien – 23.\,2.\,1942 Petrópolis@\textsc{Zweig, Stefan} (28.\,11.\,1881 Wien – 23.\,2.\,1942 Petrópolis), \emph{Schriftsteller}!Joseph Fouché. Bildnis eines politischen Menschen@\strich\emph{Joseph Fouché. Bildnis eines politischen Menschen}|pw} drucken {\pb}lassen,
               ohne das Geziemende zu tun: Sie voraus anzufragen, ob sie diese Widmung annehmen
               wollen. Nun, ich denke Sie werden mir diesen kleinen Verstoss verzeihen und nicht die
               Auflage einstampfen lassen, nur weil sie meine redliche Liebe zu Ihnen öffentlich
               bekennt.\pend
           
\pstart
           In Treue ergeben Ihr{\\[\baselineskip]}\spacefill\mbox{Stefan Zweig}\pend
           \leftskip=0em{}\selectlanguage{ngerman}\endnumbering\briefempfaengerindex{Schnitzler, Arthur@\textsc{Schnitzler, Arthur}!zzzZweig, Stefan@\emph{von Stefan Zweig}!1929-07-261@{26. 7. 1929}|)be}\mylabel{L03675h}  \newcommand{\dateiname}{L03675}\newcommand{\titel}{Stefan Zweig an Arthur Schnitzler, 26. 7. 1929}\newcommand{\editorInnen}{Selma Jahnke und Martin Anton Müller}%% latex-leseansicht-abspann.tex
%% Abspann für die Leseansicht.
%% Der Schalter \ifkorrekturansicht ist bereits durch den Vorspann gesetzt.

%% latex-abspann.tex
%% Gemeinsamer Abspann für Korrekturansicht und Leseansicht.
%% Setzt den Schalter \ifkorrekturansicht voraus (gesetzt in den
%% einbindenden Dateien latex-korrekturansicht-abspann.tex bzw.
%% latex-leseansicht-abspann.tex).
%% ---------------------------------------------------------------

\normalsize

% Das esempio-Environment wird nur in der Leseansicht benötigt
\ifkorrekturansicht\else
\newenvironment{esempio}[3]%
{
    \vspace{1.5ex}
    \rlap{\underline{#1}}
    \par
    \setlength{\parindent}{0cm}
    \nopagebreak
    \leftskip=#2cm
    \rightskip=#3cm
}
{
    \par
}
\fi

\doendnotes{C}
\bigskip
\vfill

\clearpage

\footnotesize

\ifkorrekturansicht
  \lohead{\textsc{register}}
\fi

% theindex-Environment neu definieren ohne reledmac
\makeatletter
\renewenvironment{theindex}{%
  \ifkorrekturansicht
    \section*{\indexname}%
  \else
    \subsubsection*{Index der erwähnten Entitäten}%
  \fi
  \setlength{\parindent}{0pt}%
  \setlength{\parskip}{0pt plus 0.3pt}%
  \let\item\@idxitem
}{%
  \ifkorrekturansicht\clearpage\fi
}
\makeatother

\IfFileExists{\jobname-pw.ind}{\input{\jobname-pw.ind}}{}

% Quellenangabe nur in der Leseansicht
\ifkorrekturansicht\else
% Fallback-Definitionen, falls die .tex-Datei \titel etc. nicht gesetzt hat
\providecommand{\titel}{}
\providecommand{\editorInnen}{}
\providecommand{\dateiname}{\jobname}

\vspace{3cm}

\vfill

\footnotesize
\textsc{Quelle}: \titel. Herausgegeben von {\editorInnen}. In: \emph{Arthur Schnitzler: Briefwechsel mit Autorinnen und Autoren}.
 Digitale Edition, https://schnitzler-briefe.acdh.oeaw.ac.at/{\dateiname}.html (Stand \today)
\fi

\end{document}


