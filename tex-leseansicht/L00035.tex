%% latex-leseansicht-vorspann.tex
%% Vorspann für die Leseansicht.
%% Lädt die gemeinsame Datei latex-vorspann.tex mit nicht gesetztem Schalter.

\newif\ifkorrekturansicht
\korrekturansichtfalse

\input{../tex-inputs/latex-vorspann}


         
         \renewcommand{\erwaehntePersonen}{Personen:  ?? [Befreundete Frau 1],  ?? [Befreundete Frau 2], Richard von Krafft-Ebing, Felix Salten}
         \renewcommand{\erwaehnteOrte}{Orte: Bad Aussee, Pörtschach, Wien}
         \renewcommand{\erwaehnteWerke}{}
               \section[Richard Beer-Hofmann an Arthur Schnitzler, 21. 8. 1891]{ Richard Beer-Hofmann an Arthur Schnitzler, 21. 8. 1891}\nopagebreak\mylabel{v}\rehead{ }\begin{ledgroupsized}[t]{13cm}\normalsize\beginnumbering \toendnotes[C]{\smallbreak\pagebreak[2]} \Standort{CUL, Schnitzler, B 8.}
\physDesc{Briefkarte, 416 Zeichen
\newline{}Handschrift: schwarze Tinte, lateinische Kurrent
\newline{}Schnitzler: mit Bleistift nummeriert: »4.« }\buchAbdrucke{\weitereDrucke{Arthur Schnitzler, Richard Beer-Hofmann: \emph{Briefwechsel 1891–1931}. Hg. Konstanze Fliedl. Wien, Zürich: \emph{Europaverlag} 1992, S. 32.} }\toendnotes[C]{\smallbreak}\pstart\center{}{\pb}Lieber Arthur!\pend\pstart
           Zwei uns befreundete Damen\pwindex{?? [Befreundete Frau 1] @\textsc{?? [Befreundete Frau 1]}|pwv}\pwindex{?? [Befreundete Frau 2] @\textsc{?? [Befreundete Frau 2]}|pwv} – nicht aus Wien\oindex{Wien@\textbf{Wien}|pw} – wollen nach Wien\oindex{Wien@\textbf{Wien}|pw} von hier aus, um Professor Kraft-Ebing\pwindex{Krafft-Ebing, Richard von 14.08.1840 – 22.12.1902@\textsc{Krafft-Ebing, Richard von} (14.08.1840 – 22.12.1902), \emph{Sexualforscher, Psychiater}|pw} zu consultiren. Ist Kraft-Ebing\pwindex{Krafft-Ebing, Richard von 14.08.1840 – 22.12.1902@\textsc{Krafft-Ebing, Richard von} (14.08.1840 – 22.12.1902), \emph{Sexualforscher, Psychiater}|pw} aber jetzt in Wien\oindex{Wien@\textbf{Wien}|pw}? Wenn nicht, ist bekannt, wann er zurückkehrt? Bitte antworten Sie mir
               bald. Bez. meiner Wenigkeit ist noch kein Entschluss gefasst, Wien\oindex{Wien@\textbf{Wien}|pw} – Pörtschach\oindex{Poertschach@\textbf{Pörtschach}|pw} – Aussee\oindex{Bad Aussee@\textbf{Bad Aussee}|pw} – alles noch ungewiss.\pend
           \pstart
           {\pb}Was haben Sie beschlossen?\pend
           \pstart
           Grüßen Sie mir herzlich Salten\pwindex{Salten, Felix 06.09.1869 – 08.10.1945@\textsc{Salten, Felix} (06.09.1869 – 08.10.1945), \emph{Schriftsteller, Journalist}|pw}.\pend
           \pstart
           Ihr{\\[\baselineskip]}treuer{\\[\baselineskip]}\spacefill\mbox{Richard}\pend
           \leftskip=0em{}\pstart
           21. Aug. 91.\pend
           
         
         \endnumbering\mylabel{h}\end{ledgroupsized}  \newcommand{\dateiname}{L00035}\newcommand{\titel}{Richard Beer-Hofmann an Arthur Schnitzler, 21. 8. 1891}\newcommand{\editorInnen}{Martin Anton Müller und Gerd-Hermann Susen}%% latex-leseansicht-abspann.tex
%% Abspann für die Leseansicht.
%% Der Schalter \ifkorrekturansicht ist bereits durch den Vorspann gesetzt.

%% latex-abspann.tex
%% Gemeinsamer Abspann für Korrekturansicht und Leseansicht.
%% Setzt den Schalter \ifkorrekturansicht voraus (gesetzt in den
%% einbindenden Dateien latex-korrekturansicht-abspann.tex bzw.
%% latex-leseansicht-abspann.tex).
%% ---------------------------------------------------------------

\normalsize

% Das esempio-Environment wird nur in der Leseansicht benötigt
\ifkorrekturansicht\else
\newenvironment{esempio}[3]%
{
    \vspace{1.5ex}
    \rlap{\underline{#1}}
    \par
    \setlength{\parindent}{0cm}
    \nopagebreak
    \leftskip=#2cm
    \rightskip=#3cm
}
{
    \par
}
\fi

\doendnotes{C}
\bigskip
\vfill

\clearpage

\footnotesize

\ifkorrekturansicht
  \lohead{\textsc{register}}
\fi

% theindex-Environment neu definieren ohne reledmac
\makeatletter
\renewenvironment{theindex}{%
  \ifkorrekturansicht
    \section*{\indexname}%
  \else
    \subsubsection*{Index der erwähnten Entitäten}%
  \fi
  \setlength{\parindent}{0pt}%
  \setlength{\parskip}{0pt plus 0.3pt}%
  \let\item\@idxitem
}{%
  \ifkorrekturansicht\clearpage\fi
}
\makeatother

\IfFileExists{\jobname-pw.ind}{\input{\jobname-pw.ind}}{}

% Quellenangabe nur in der Leseansicht
\ifkorrekturansicht\else
% Fallback-Definitionen, falls die .tex-Datei \titel etc. nicht gesetzt hat
\providecommand{\titel}{}
\providecommand{\editorInnen}{}
\providecommand{\dateiname}{\jobname}

\vspace{3cm}

\vfill

\footnotesize
\textsc{Quelle}: \titel. Herausgegeben von {\editorInnen}. In: \emph{Arthur Schnitzler: Briefwechsel mit Autorinnen und Autoren}.
 Digitale Edition, https://schnitzler-briefe.acdh.oeaw.ac.at/{\dateiname}.html (Stand \today)
\fi

\end{document}


      