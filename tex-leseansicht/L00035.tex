%% latex-korrekturansicht-vorspann.tex
%% Vorspann für die Korrekturansicht.
%% Lädt die gemeinsame Datei latex-vorspann.tex mit gesetztem Schalter.

\newif\ifkorrekturansicht
\korrekturansichttrue

\input{../tex-inputs/latex-vorspann}


\section[Richard Beer-Hofmann an Arthur Schnitzler, 21. 8. 1891]{L00035 Richard Beer-Hofmann an Arthur Schnitzler, 21. 8. 1891}
\nopagebreak\mylabel{L00035v}
\rehead{ }\normalsize\beginnumbering\briefempfaengerindex{Schnitzler, Arthur@\textsc{Schnitzler, Arthur}!zzzBeer-Hofmann, Richard@\emph{von Richard Beer-Hofmann}!1891-08-211@{21. 8. 1891}|(be}
\toendnotes[C]{\smallbreak\pagebreak[2]}\Standort{CUL, Schnitzler, B 8.}
\physDesc{Briefkarte, 416 Zeichen
\newline{}Handschrift: schwarze Tinte, lateinische Kurrent
\newline{}Schnitzler: mit Bleistift nummeriert: »4.« }
\buchAbdrucke{\weitereDrucke{Arthur Schnitzler, Richard Beer-Hofmann: \emph{Briefwechsel 1891–1931}. Wien, Zürich: \emph{Europaverlag} 1992, S. 32.} }\toendnotes[C]{\smallbreak}
\pstart\center{}{\pb}Lieber Arthur!\pend\vspace{0.5em}
\pstart
           Zwei uns befreundete Damen\pwindex{?? [Befreundete Frau 1] @\textsc{?? [Befreundete Frau 1]}|pwv}\pwindex{?? [Befreundete Frau 2] @\textsc{?? [Befreundete Frau 2]}|pwv} – nicht aus Wien\oindex{Wien@\textbf{Wien}, \emph{A.ADM2}|pw} – wollen nach Wien\oindex{Wien@\textbf{Wien}, \emph{A.ADM2}|pw} von hier aus, um Professor Kraft-Ebing\pwindex{Krafft-Ebing, Richard von 14.08.1840 – 22.12.1902@\textsc{Krafft-Ebing, Richard von} (14.08.1840 – 22.12.1902), \emph{Sexualforscher/Sexualforscherin, Psychiater/Psychiaterin}|pw} zu consultiren. Ist Kraft-Ebing\pwindex{Krafft-Ebing, Richard von 14.08.1840 – 22.12.1902@\textsc{Krafft-Ebing, Richard von} (14.08.1840 – 22.12.1902), \emph{Sexualforscher/Sexualforscherin, Psychiater/Psychiaterin}|pw} aber jetzt in Wien\oindex{Wien@\textbf{Wien}, \emph{A.ADM2}|pw}? Wenn nicht, ist bekannt, wann er zurückkehrt? Bitte antworten Sie mir
               bald. Bez. meiner Wenigkeit ist noch kein Entschluss gefasst, Wien\oindex{Wien@\textbf{Wien}, \emph{A.ADM2}|pw} – Pörtschach\oindex{Poertschach am Woerthersee@\textbf{Pörtschach am Wörthersee}, \emph{P.PPL}|pw} – Aussee\oindex{Bad Aussee@\textbf{Bad Aussee}, \emph{P.PPLA3}|pw} – alles noch ungewiss.\pend
           
\pstart
           {\pb}Was haben Sie beschlossen?\pend
           
\pstart
           Grüßen Sie mir herzlich Salten\pwindex{Salten, Felix 06.09.1869 – 08.10.1945@\textsc{Salten, Felix} (06.09.1869 – 08.10.1945), \emph{Schriftsteller/Schriftstellerin, Journalist/Journalistin, Chefredakteur/Chefredakteurin}|pw}.\pend
           
\pstart
           Ihr{\\[\baselineskip]}treuer{\\[\baselineskip]}\spacefill\mbox{Richard}\pend
           \leftskip=0em{}
\pstart
           21. Aug. 91.\pend
           \selectlanguage{ngerman}\endnumbering\briefempfaengerindex{Schnitzler, Arthur@\textsc{Schnitzler, Arthur}!zzzBeer-Hofmann, Richard@\emph{von Richard Beer-Hofmann}!1891-08-211@{21. 8. 1891}|)be}\mylabel{L00035h}  \normalsize

\doendnotes{C}
\bigskip
\vfill

\clearpage

\footnotesize

\lohead{\textsc{register}}

% Definiere theindex-Environment komplett neu ohne reledmac
\makeatletter
\renewenvironment{theindex}{%
  \section*{\indexname}%
  \setlength{\parindent}{0pt}%
  \setlength{\parskip}{0pt plus 0.3pt}%
  \let\item\@idxitem
}{%
  \clearpage
}
\makeatother

\IfFileExists{\jobname-pw.ind}{\input{\jobname-pw.ind}}{}

\end{document}

      