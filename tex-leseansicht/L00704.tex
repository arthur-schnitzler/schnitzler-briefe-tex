%% latex-korrekturansicht-vorspann.tex
%% Vorspann für die Korrekturansicht.
%% Lädt die gemeinsame Datei latex-vorspann.tex mit gesetztem Schalter.

\newif\ifkorrekturansicht
\korrekturansichttrue

\input{../tex-inputs/latex-vorspann}


\section[Arthur Schnitzler an Richard Beer-Hofmann, {[}17. oder 18. 7. 1897?{]}]{L00704 Arthur Schnitzler an Richard Beer-Hofmann, {[}17. oder
               18. 7. 1897?{]}}
\nopagebreak\mylabel{L00704v}
\rehead{ }\normalsize\beginnumbering\briefempfaengerindex{Beer-Hofmann, Richard@\textsc{Beer-Hofmann, Richard}!zzzSchnitzler, Arthur@\emph{von Arthur Schnitzler}!1897-07-182@{{[}17. oder
                  18. 7. 1897?{]}}|(be}
\toendnotes[C]{\smallbreak\pagebreak[2]}\Standort{YCGL, MSS 31.}
\physDesc{Visitenkarte, 98 Zeichen
\newline{}Handschrift: 1) Bleistift, deutsche Kurrent\hspace{1em}2) Bleistift, lateinische Kurrent (\noindent{}Adresse)\hspace{1em}
\newline{}Zusatz: die Adresse quer am linken Rand der bedruckten Seite }\toendnotes[C]{\smallbreak}\pstart{}{\pb}Dr Beer Hofma{\geminationn}\pend{}\pstart{}Egelmoos 22\oindex{Eglmoosgasse@\textbf{Eglmoosgasse}, \emph{Bezirk (A.BZK)}|pw}\pend{}{\bigskip}\vspace{1em}
\pstart
           \noindent{}\centering{}{\pb}\textcolor{gray}{\textbf{D\textsuperscript{r} Arthur Schnitzler}}\pend
           
\pstart
           \raggedleft{}\textcolor{gray}{\textbf{Wien\oindex{Wien@\textbf{Wien}, \emph{A.ADM2}|pw}.}}\pend
           
\pstart
           {\pb}Lieber Richard, eben \label{K_L00704-1v}\edtext{Telegra{\geminationm} von Hugo\pwindex{Hofmannsthal, Hugo von 1874-02-01 – 1929-07-15@\textsc{Hofmannsthal, Hugo von} (1874-02-01 – 1929-07-15), \emph{Schriftsteller/Schriftstellerin}|pw}}{\lemma{\textnormal{\emph{Telegramm von Hugo}}}\Cendnote{\textnormal{nicht überliefert}}}\label{K_L00704-1}, er
               ist So{\geminationn}tag früh in Zell\oindex{Zell am See@\textbf{Zell am See}, \emph{P.PPLA3}|pw}.\pend
           \pstart Herzlich Ihr \spacefill\mbox{A}\pend{}\selectlanguage{ngerman}\endnumbering\briefempfaengerindex{Beer-Hofmann, Richard@\textsc{Beer-Hofmann, Richard}!zzzSchnitzler, Arthur@\emph{von Arthur Schnitzler}!1897-07-172@{{[}17. oder
                  18. 7. 1897?{]}}|)be}\mylabel{L00704h}  \normalsize

\doendnotes{C}
\bigskip
\vfill

\clearpage

\footnotesize

\lohead{\textsc{register}}

% Definiere theindex-Environment komplett neu ohne reledmac
\makeatletter
\renewenvironment{theindex}{%
  \section*{\indexname}%
  \setlength{\parindent}{0pt}%
  \setlength{\parskip}{0pt plus 0.3pt}%
  \let\item\@idxitem
}{%
  \clearpage
}
\makeatother

\IfFileExists{\jobname-pw.ind}{\input{\jobname-pw.ind}}{}

\end{document}

      