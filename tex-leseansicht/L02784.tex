%% latex-leseansicht-vorspann.tex
%% Vorspann für die Leseansicht.
%% Lädt die gemeinsame Datei latex-vorspann.tex mit nicht gesetztem Schalter.

\newif\ifkorrekturansicht
\korrekturansichtfalse

\input{../tex-inputs/latex-vorspann}


         
         \newcommand{\erwaehntePersonen}{Personen: Alphonse Allais, Peter Altenberg, George Auriol, Richard Beer-Hofmann, Paula Beer-Hofmann, Tristan Bernard, Oskar Bie, Otto Brahm, Max Eugen Burckhard, Georges Courteline, Samuel Fischer, Hedwig Fischer, Jean-Louis Forain,  Franz Joseph I. von Österreich-Ungarn, Clementine Goldmann, Ferdinand Gregorovius, Gisela Hajek, Markus Hajek, Alfred Kerr, Narcisse Lebeau, Friedrich Mitterwurzer, Peter Nansen, Betty Nansen, Maurice O’Reilly, Marie Reinhard, Rudolf Rittner, Louise Schnitzler, Julius Schnitzler, Helene Schnitzler, Leopold Sonnemann, Pierre Veber, Léon Xanrof}
         \newcommand{\erwaehnteInstitutionen}{Institutionen: Frankfurter Zeitung}
         \newcommand{\erwaehnteOrte}{Orte: Baden bei Wien, Berlin, Frankfurt am Main, Hagenbecks Tierpark, Hamburg, Kaiser-Franz-Ring, Kopenhagen, München, Paris, Skodsborg, Wien, rue Feydeau, Österreich}
         \newcommand{\erwaehnteWerke}{Werke: Frankfurter Zeitung, Freiwild. Schauspiel in 3 Akten, Liebelei. Schauspiel in drei Akten, Neue französische Humoristen, Neue französische Humoristen. Alphonse Allais, Neue französische Humoristen. Georges Courteline, Neue französische Humoristen. L. Xanrof, Neue französische Humoristen. Narcisse Lebeau, Neue französische Humoristen. Pierre Veber, Neue französische Humoristen. Tristan Bernard. – Georges Auriol. – Bill Sharp. – Maurice O’Reilly, Wie ich es sehe}
               \section[ Paul Goldmann an Arthur Schnitzler, 7. 9. {[}1896{]}]{ Paul Goldmann an Arthur Schnitzler, 7. 9. {[}1896{]}}\nopagebreak\mylabel{v}\rehead{ }\begin{ledgroupsized}[t]{13cm}\normalsize\beginnumbering \toendnotes[C]{\smallbreak\pagebreak[2]} \Standort{DLA, A:Schnitzler, HS.NZ85.1.3166.}
\physDesc{Brief, 3 Blätter, 12 Seiten
\newline{}Handschrift: schwarze Tinte, deutsche Kurrent
\newline{}Schnitzler: 1) mit Bleistift das Jahr »96« vermerkt  2) mit rotem Buntstift zwölf Unterstreichungen}\toendnotes[C]{\smallbreak}\pstart
           \noindent{}{\pb}\textcolor{gray}{\textbf{\textbf{Frankfurter Zeitung\orgindex{Frankfurter Zeitung@Frankfurter Zeitung|pw}}}}\pend
           \pstart
           \textcolor{gray}{\textbf{(\begin{otherlanguage}{french}Gazette de Francfort\end{otherlanguage}\orgindex{Frankfurter Zeitung@Frankfurter Zeitung|pw}).}}\pend
           \pstart
           \textcolor{gray}{\textbf{\textbf{\begin{otherlanguage}{french}Fondateur M.\end{otherlanguage}{ }L. Sonnemann\pwindex{Sonnemann, Leopold 1831-10-29 – 1909-10-30@\textsc{Sonnemann, Leopold} (1831-10-29 – 1909-10-30), \emph{Journalist, Herausgeber}|pw}.}}}\pend
           \pstart
           \begin{otherlanguage}{french}\textcolor{gray}{\textbf{Journal\pwindex{?? Werk@Nicht ermittelte Verfasserinnen und Verfasser!Frankfurter Zeitung1856 – 1943@\emph{Frankfurter Zeitung} {[}1856 – 1943{]}|pwv} politique,
                        financier,}}\end{otherlanguage}\pend
           \pstart
           \begin{otherlanguage}{french}\textcolor{gray}{\textbf{commercial et littéraire.}}\end{otherlanguage}\pend
           \pstart
           \begin{otherlanguage}{french}\textcolor{gray}{\textbf{\textbf{Paraissant trois fois par jour.}}}\end{otherlanguage}\pend
           \pstart
           \begin{otherlanguage}{french}\textcolor{gray}{\textbf{\textbf{Bureau à Paris\oindex{Paris@\textbf{Paris}|pw}}}}\end{otherlanguage}\pend
           \pstart
           \begin{otherlanguage}{french}\textcolor{gray}{\textbf{\textbf{24. Rue Feydeau\oindex{rue Feydeau@\textbf{rue Feydeau}|pw}.}}}\end{otherlanguage}\hfill \textsc{Berlin\oindex{Berlin@\textbf{Berlin}|pw}}, 7. September.\pend
           \pstart\center{}Mein lieber Freund,\pend\pstart
           Morgen, Dienſtag, \label{K_L02784-1v}\edtext{fahre ich heim}{\lemma{\textnormal{\emph{fahre ich heim}}}\Cendnote{\textnormal{Schnitzler\pwindex{Schnitzler, Arthur 15.05.1862 – 21.10.1931@\textsc{Schnitzler, Arthur} (15.05.1862 – 21.10.1931), \emph{Schriftsteller, Mediziner}|pwk} war bereits am 26. 8. 1896 von Berlin\oindex{Berlin@\textbf{Berlin}|pwk} über München\oindex{Muenchen@\textbf{München}|pwk} nach Wien\oindex{Wien@\textbf{Wien}|pwk} gereist, wo er am
                     29. 8. 1896
                  ankam.}}}\label{K_L02784-1h} (»heim« iſt gut!), und Dein lieber Brief iſt das letzte Angenehme,
               das mir hier widerfährt.\pend
           \pstart
           Ich freue mich, daß Du glücklich wieder in Wien\oindex{Wien@\textbf{Wien}|pw}
               biſt und dort Alles beim Rechten gefunden haſt.\pend
           \pstart
           \label{K_L02784-2v}\edtext{\textsc{Burckhardt\pwindex{Burckhard, Max Eugen 14.07.1854 – 16.03.1912@\textsc{Burckhard, Max Eugen} (14.07.1854 – 16.03.1912), \emph{Schriftsteller, Rechtswissenschaftler, Theaterleiter}|pw}}s Begeiſterung}{\lemma{\textnormal{\emph{Burckhardts Begeiſterung}}}\Cendnote{\textnormal{siehe A. S.: \emph{Tagebuch}, 4. 9. 1896}}}\label{K_L02784-2h} für Dein Stück\pwindex{Schnitzler, Arthur 15.05.1862 – 21.10.1931@\textsc{Schnitzler, Arthur} (15.05.1862 – 21.10.1931), \emph{Schriftsteller, Mediziner}!Freiwild. Schauspiel in 3 Akten1896@\strich\emph{Freiwild. Schauspiel in 3 Akten} {[}1896{]}|pwv} iſt ein
               weiteres gutes \textsc{omen}. Daß das Werk\pwindex{Schnitzler, Arthur 15.05.1862 – 21.10.1931@\textsc{Schnitzler, Arthur} (15.05.1862 – 21.10.1931), \emph{Schriftsteller, Mediziner}!Freiwild. Schauspiel in 3 Akten1896@\strich\emph{Freiwild. Schauspiel in 3 Akten} {[}1896{]}|pwv} den Theaterleuten ſo gefällt, iſt das
               ſtärkſte Zeugniß für die Theater-Wirkung, die man {\pb}davon erwarten kann. Warum B.\pwindex{Burckhard, Max Eugen 14.07.1854 – 16.03.1912@\textsc{Burckhard, Max Eugen} (14.07.1854 – 16.03.1912), \emph{Schriftsteller, Rechtswissenschaftler, Theaterleiter}|pwv} ſämmtliche noch überlebenden Perſonen des Stück\pwindex{Schnitzler, Arthur 15.05.1862 – 21.10.1931@\textsc{Schnitzler, Arthur} (15.05.1862 – 21.10.1931), \emph{Schriftsteller, Mediziner}!Freiwild. Schauspiel in 3 Akten1896@\strich\emph{Freiwild. Schauspiel in 3 Akten} {[}1896{]}|pwv}es \strikeout{\textcolor{gray}{von d}\textcolor{gray}{×}\-\textcolor{gray}{×}} umbringen will, iſt mir nicht recht begreiflich. Dieſe Abänderungs-\label{K_L02784-123v}\edtext{Vorſchläge}{\lemma{\textnormal{\emph{Vorſchläge}}}\Cendnote{\textnormal{in der Vorlage steht: »Vorſchlage«}}}\label{K_L02784-123h}
               ſind ſehr komiſch. Da wüßte ich viel beſſere: \textsc{Anna\pwindex{Schnitzler, Arthur 15.05.1862 – 21.10.1931@\textsc{Schnitzler, Arthur} (15.05.1862 – 21.10.1931), \emph{Schriftsteller, Mediziner}!Freiwild. Schauspiel in 3 Akten1896@\strich\emph{Freiwild. Schauspiel in 3 Akten} {[}1896{]}|pwv}} ſoll den Kaſſierer \textsc{Kohn\pwindex{Schnitzler, Arthur 15.05.1862 – 21.10.1931@\textsc{Schnitzler, Arthur} (15.05.1862 – 21.10.1931), \emph{Schriftsteller, Mediziner}!Freiwild. Schauspiel in 3 Akten1896@\strich\emph{Freiwild. Schauspiel in 3 Akten} {[}1896{]}|pwv}} heirathen und \textsc{Vogel\pwindex{Schnitzler, Arthur 15.05.1862 – 21.10.1931@\textsc{Schnitzler, Arthur} (15.05.1862 – 21.10.1931), \emph{Schriftsteller, Mediziner}!Freiwild. Schauspiel in 3 Akten1896@\strich\emph{Freiwild. Schauspiel in 3 Akten} {[}1896{]}|pwv}} ſoll in dem Theater-Director ſeinen verloren geglaubten Vater wiederfinden{\dotsfive}\pend
           \pstart
           Die \label{K_L02784-5v}\edtext{Äußerung des allerhöchſten Herrn\pwindex{Franz Joseph I. von Oesterreich-Ungarn 18.08.1830 – 21.11.1916@\textsc{Franz Joseph I. von Österreich-Ungarn} (18.08.1830 – 21.11.1916), \emph{Kaiser}|pwv} über »Liebelei\pwindex{Schnitzler, Arthur 15.05.1862 – 21.10.1931@\textsc{Schnitzler, Arthur} (15.05.1862 – 21.10.1931), \emph{Schriftsteller, Mediziner}!Liebelei. Schauspiel in drei Akten1895-10-09@\strich\emph{Liebelei. Schauspiel in drei Akten} {[}1895-10-09{]}|pw}«}{\lemma{\textnormal{\emph{Äußerung … »Liebelei«}}}\Cendnote{\textnormal{siehe A. S.: \emph{Tagebuch}, 5. 9. 1896}}}\label{K_L02784-5h} iſt köſtlich. Ich hoffe, Seine Majeſtät\pwindex{Franz Joseph I. von Oesterreich-Ungarn 18.08.1830 – 21.11.1916@\textsc{Franz Joseph I. von Österreich-Ungarn} (18.08.1830 – 21.11.1916), \emph{Kaiser}|pwv} verſteht vom Regieren mehr, wie von der Kunſt, {\pb}ſonſt müßte man mit großer Beſorgniß in die Zukunft
                  Öſterreich\oindex{Oesterreich@\textbf{Österreich}|pw}s blicken. \label{K_L02784-3v}\edtext{\textsc{Mitterwurzer\pwindex{Mitterwurzer, Friedrich 16.10.1844 – 13.02.1897@\textsc{Mitterwurzer, Friedrich} (16.10.1844 – 13.02.1897), \emph{Schauspieler}|pw}} iſt ſo der rechte Sau-Komödiant\pwindex{Mitterwurzer, Friedrich 16.10.1844 – 13.02.1897@\textsc{Mitterwurzer, Friedrich} (16.10.1844 – 13.02.1897), \emph{Schauspieler}|pwv}}{\lemma{\textnormal{\emph{Mitterwurzer … Sau-Komödiant}}}\Cendnote{\textnormal{siehe A. S.: \emph{Tagebuch}, 5. 9. 1896}}}\label{K_L02784-3h}. Schreib’ \strikeout{ihn} ihm einmal eine Rolle, in der
               er Erfolg hat, und er wird Dich als das erſte Genie der Welt ausſchreien.\pend
           \pstart
           Von \textsc{Richard\pwindex{Beer-Hofmann, Richard 1866-07-11 – 1945-09-26@\textsc{Beer-Hofmann, Richard} (1866-07-11 – 1945-09-26), \emph{Schriftsteller}|pw}} weiß ich Dir wenig zu ſagen. Er muß ſchon \label{K_L02784-6v}\edtext{in \textsc{Baden\oindex{Kaiser-Franz-Ring@\textbf{Kaiser-Franz-Ring}|pw}}}{\lemma{\textnormal{\emph{in Baden}}}\Cendnote{\textnormal{siehe Richard Beer-Hofmann an Arthur Schnitzler, 5. 9. 1896}}}\label{K_L02784-6h} ſein. Während der letzten Tage ſeines Hierſeins war er nervös und erging ſich
               in unangenehmen Betrachtungen über die »guten Menſchen«. \textsc{Paula\pwindex{Beer-Hofmann, Paula 25.02.1879 – 30.10.1939@\textsc{Beer-Hofmann, Paula} (25.02.1879 – 30.10.1939)|pw}} hat er {\pb}fortgeſchickt; ſie wollte natürlich
               zum Schluß durchaus noch dableiben, weil ſie bei \label{K_L02784-7v}\edtext{\textsc{Hagenbeck\oindex{Hagenbecks Tierpark@\textbf{Hagenbecks Tierpark}|pw}}}{\lemma{\textnormal{\emph{Hagenbeck}}}\Cendnote{\textnormal{Hamburg\oindex{Hamburg@\textbf{Hamburg}|pwk}er Tierpark\oindex{Hagenbecks Tierpark@\textbf{Hagenbecks Tierpark}|pwkv}}}}\label{K_L02784-7h} ſo ſchöne Affen und Raubthiere geſehen hatte.\pend
           \pstart
           Was mich anlangt, ſo ſind mir die Tage in Berlin\oindex{Berlin@\textbf{Berlin}|pw}
               recht angenehm verfloſſen. Der liebſte unter den Menſchen, die ich hier kennen
               gelernt, iſt mir Dr. \textsc{Bie\pwindex{Bie, Oskar 09.02.1864 – 21.04.1938@\textsc{Bie, Oskar} (09.02.1864 – 21.04.1938), \emph{Schriftsteller, Journalist, Redakteur}|pw}}. Er iſt ehrlich und gut. Wir verſtehen uns und haben uns wohl auch gern. \textsc{Kerr\pwindex{Kerr, Alfred 25.12.1867 – 12.10.1948@\textsc{Kerr, Alfred} (25.12.1867 – 12.10.1948), \emph{Schriftsteller, Kritiker}|pw}} mag ich weniger. Ich wittere in ihm {\pb}den
                  \label{K_L02784-8v}\edtext{\begin{otherlanguage}{french}\textsc{froid ambitieux}\end{otherlanguage}}{\lemma{\textnormal{\emph{froid ambitieux}}}\Cendnote{\textnormal{französisch: kühler Ehrgeizling}}}\label{K_L02784-8h}.
               Mit \textsc{Brahm\pwindex{Brahm, Otto 05.02.1856 – 28.11.1912@\textsc{Brahm, Otto} (05.02.1856 – 28.11.1912), \emph{Theaterleiter, Regisseur}|pw}}, \textsc{Rittner\pwindex{Rittner, Rudolf 30.06.1869 – 04.02.1943@\textsc{Rittner, Rudolf} (30.06.1869 – 04.02.1943), \emph{Theaterleiter, Schauspieler}|pw}} und \textsc{Richard\pwindex{Beer-Hofmann, Richard 1866-07-11 – 1945-09-26@\textsc{Beer-Hofmann, Richard} (1866-07-11 – 1945-09-26), \emph{Schriftsteller}|pw}} verbrachte ich einen Abend. \textsc{Rittner\pwindex{Rittner, Rudolf 30.06.1869 – 04.02.1943@\textsc{Rittner, Rudolf} (30.06.1869 – 04.02.1943), \emph{Theaterleiter, Schauspieler}|pw}} gefiel auch mir ausnehmend. \textsc{Brahm\pwindex{Brahm, Otto 05.02.1856 – 28.11.1912@\textsc{Brahm, Otto} (05.02.1856 – 28.11.1912), \emph{Theaterleiter, Regisseur}|pw}} forderte mich auf, ihm noch einmal Rendezvous für einen Abend zu geben. Ich
               hab’ es aber nicht gethan; ich glaub’ nicht, daß ihm irgend etwas an mir liegt. \textsc{Fischer\pwindex{Fischer, Samuel 24.12.1859 – 15.10.1934@\textsc{Fischer, Samuel} (24.12.1859 – 15.10.1934), \emph{Verleger}|pw}} hat ſofort \strikeout{\textcolor{gray}{×}} in mir einen ausnutzbaren Mann geſehen, hat \strikeout{mich} ſich von mir einige Stunden über \textsc{Paris\oindex{Paris@\textbf{Paris}|pw}} erzählen {\pb}laſſen, hat mich auch zum Abendeſſen
               geladen. \strikeout{Das} Die Herausgabe der Humoriſten hat er
               natürlich abgelehnt. Hingegen wird ſeine Frau\pwindex{Fischer, Hedwig 08.09.1871 – 11.04.1952@\textsc{Fischer, Hedwig} (08.09.1871 – 11.04.1952)|pwv} wohl einen oder den anderen von dieſen Leuten jetzt
               überſetzen, angeregt durch die Lectüre meiner \label{K_L02784-9v}\edtext{Feuilletons}{\lemma{\textnormal{\emph{Feuilletons}}}\Cendnote{\textnormal{Goldmann\pwindex{Goldmann, Paul 31.01.1865 – 25.09.1935@\textsc{Goldmann, Paul} (31.01.1865 – 25.09.1935), \emph{Schriftsteller, Journalist}|pwk} hat in seiner Feuilletonreihe »\emph{Neue französische Humoristen}\pwindex{Goldmann, Paul 31.01.1865 – 25.09.1935@\textsc{Goldmann, Paul} (31.01.1865 – 25.09.1935), \emph{Schriftsteller, Journalist}!Neue franzoesische Humoristen1893-09-03 – 1894@\strich\emph{Neue französische Humoristen} {[}1893-09-03 – 1894{]}|pwk}« in der \emph{Frankfurter Zeitung}\pwindex{?? Werk@Nicht ermittelte Verfasserinnen und Verfasser!Frankfurter Zeitung1856 – 1943@\emph{Frankfurter Zeitung} {[}1856 – 1943{]}|pwk} verschiedene
                  Literaturschaffende vorgestellt, jeweils mit einer kurzen Einleitung und einer
                  kleinen Übersetzung. Während die ersten Beiträge nachgewiesen werden können, muss
                  offen bleiben, wie viele Beiträge in Folge erschienen. \emph{Alphonse Allais}\pwindex{Goldmann, Paul 31.01.1865 – 25.09.1935@\textsc{Goldmann, Paul} (31.01.1865 – 25.09.1935), \emph{Schriftsteller, Journalist}!Neue franzoesische Humoristen. Alphonse Allais1893-09-03@\strich\emph{Neue französische Humoristen. Alphonse Allais} {[}1893-09-03{]}|pwk}, 3. 9. 1893; \emph{Georges Courteline}\pwindex{Goldmann, Paul 31.01.1865 – 25.09.1935@\textsc{Goldmann, Paul} (31.01.1865 – 25.09.1935), \emph{Schriftsteller, Journalist}!Neue franzoesische Humoristen. Georges Courteline1893-12-31 – 1894-01-01@\strich\emph{Neue französische Humoristen. Georges Courteline} {[}1893-12-31 – 1894-01-01{]}|pwk},
                        31. 12. 1893 und 1. 1. 1894; \emph{L. Xanrof}\pwindex{Goldmann, Paul 31.01.1865 – 25.09.1935@\textsc{Goldmann, Paul} (31.01.1865 – 25.09.1935), \emph{Schriftsteller, Journalist}!Neue franzoesische Humoristen. L. Xanrof1894-03-25@\strich\emph{Neue französische Humoristen. L. Xanrof} {[}1894-03-25{]}|pwk}, 25. 3. 1894, \emph{Pierre Veber}\pwindex{Goldmann, Paul 31.01.1865 – 25.09.1935@\textsc{Goldmann, Paul} (31.01.1865 – 25.09.1935), \emph{Schriftsteller, Journalist}!Neue franzoesische Humoristen. Pierre Veber1894-05-11 – 1894-05-13@\strich\emph{Neue französische Humoristen. Pierre Veber} {[}1894-05-11 – 1894-05-13{]}|pwk}, 11. 5. 1894
                     und 13. 5. 1894; \emph{Narcisse Lebeau}\pwindex{Goldmann, Paul 31.01.1865 – 25.09.1935@\textsc{Goldmann, Paul} (31.01.1865 – 25.09.1935), \emph{Schriftsteller, Journalist}!Neue franzoesische Humoristen. Narcisse Lebeau1894-10-05@\strich\emph{Neue französische Humoristen. Narcisse Lebeau} {[}1894-10-05{]}|pwk}, 5. 10. 1894; \emph{Tristan Bernard. – Georges Auriol. – Bill
                        Sharp. – Maurice O’Reilly}\pwindex{Goldmann, Paul 31.01.1865 – 25.09.1935@\textsc{Goldmann, Paul} (31.01.1865 – 25.09.1935), \emph{Schriftsteller, Journalist}!Neue franzoesische Humoristen. Tristan Bernard. – Georges Auriol. – Bill Sharp. – Maurice O Reilly1895-04-14 – 1895-04-17@\strich\emph{Neue französische Humoristen. Tristan Bernard. – Georges Auriol. – Bill Sharp. – Maurice O’Reilly} {[}1895-04-14 – 1895-04-17{]}|pwk}, 14. 4. 1894 und
                        17. 4. 1894. Zu Übersetzungen von diesen Autoren\pwindex{Lebeau, Narcisse 1865-04-09 – 1931-07-12@\textsc{Lebeau, Narcisse} (1865-04-09 – 1931-07-12), \emph{Schriftsteller, Humorist, Klempner}|pwkv}\pwindex{Allais, Alphonse 1854-10-20 – 28.10.1905@\textsc{Allais, Alphonse} (1854-10-20 – 28.10.1905), \emph{Schriftsteller}|pwkv}\pwindex{Courteline, Georges 25.06.1858 – 25.06.1929@\textsc{Courteline, Georges} (25.06.1858 – 25.06.1929), \emph{Schriftsteller}|pwkv}\pwindex{Xanrof, Leon 1867-12-09 – 1953-05-17@\textsc{Xanrof, Léon} (1867-12-09 – 1953-05-17), \emph{Dramatiker, Humorist}|pwkv}\pwindex{Bernard, Tristan 07.09.1866 – 07.12.1947@\textsc{Bernard, Tristan} (07.09.1866 – 07.12.1947), \emph{Schriftsteller}|pwkv}\pwindex{Auriol, George 1863-04-27 – 1938-02-06@\textsc{Auriol, George} (1863-04-27 – 1938-02-06), \emph{Maler, Lyriker}|pwkv}\pwindex{Veber, Pierre 1869-05-15 – 1942-08-20@\textsc{Veber, Pierre} (1869-05-15 – 1942-08-20), \emph{Schriftsteller}|pwkv}\pwindex{O Reilly, Maurice @\textsc{O’Reilly, Maurice}, \emph{Kabarettist, Diplomat}|pwkv} durch Hedwig
                     Fischer\pwindex{Fischer, Hedwig 08.09.1871 – 11.04.1952@\textsc{Fischer, Hedwig} (08.09.1871 – 11.04.1952)|pwk} konnte nichts gefunden werden.}}}\label{K_L02784-9h}! Das mindert nicht den
               Freundſchaftsdienſt, den Du mir haſt leiſten wollen, und ich danke Dir von ganzem
               Herzen dafür. Die \label{K_L02784-11v}\edtext{Zeichnung von \textsc{Forain\pwindex{Forain, Jean-Louis 1852-10-23 – 1931-07-11@\textsc{Forain, Jean-Louis} (1852-10-23 – 1931-07-11), \emph{Maler, Grafiker, Karikaturist}|pw}}}{\lemma{\textnormal{\emph{Zeichnung von Forain}}}\Cendnote{\textnormal{nicht ermittelt}}}\label{K_L02784-11h}{ }{\pb}konnte ich ihm nicht zeigen. Ich habe ſie dem \textsc{Richard \pwindex{Beer-Hofmann, Richard 1866-07-11 – 1945-09-26@\textsc{Beer-Hofmann, Richard} (1866-07-11 – 1945-09-26), \emph{Schriftsteller}|pw}} für Dich mitgegeben; derſelbe hat auch Deinen \textsc{Altenberg\pwindex{Altenberg, Peter 09.03.1859 – 08.01.1919@\textsc{Altenberg, Peter} (09.03.1859 – 08.01.1919), \emph{Schriftsteller}!Wie ich es sehe1896@\strich\emph{Wie ich es sehe} {[}1896{]}|pwv}\pwindex{Altenberg, Peter 09.03.1859 – 08.01.1919@\textsc{Altenberg, Peter} (09.03.1859 – 08.01.1919), \emph{Schriftsteller}|pw}}. Sag’ ihm, bitte, daß ich ihm \label{K_L02784-12v}\edtext{den \textsc{Gregorovius\pwindex{Gregorovius, Ferdinand 19.01.1821 – 01.05.1891@\textsc{Gregorovius, Ferdinand} (19.01.1821 – 01.05.1891), \emph{Schriftsteller, Historiker}|pw}}}{\lemma{\textnormal{\emph{den Gregorovius}}}\Cendnote{\textnormal{nicht ermittelt}}}\label{K_L02784-12h} ſofort nach
               meiner Ankunft in \textsc{Paris\oindex{Paris@\textbf{Paris}|pw}} ſchicken werde. Ich habe \strikeout{die} den Brief mit
               ſeiner Baden\oindex{Baden bei Wien@\textbf{Baden bei Wien}|pw}er Adreſſe verloren, und auch ſeine
                  Wien\oindex{Wien@\textbf{Wien}|pw}er Adreſſe finde ich erſt in \textsc{Paris\oindex{Paris@\textbf{Paris}|pw}}.\pend
           \pstart
           Sonſt hat mir \textsc{Berlin\oindex{Berlin@\textbf{Berlin}|pw}} beſſer gefallen, als ich erwartet. Aber lieb {\pb}gewinnen könnte ich die Stadt\oindex{Berlin@\textbf{Berlin}|pwv}
               wohl nicht. Im Großen und Ganzen macht ſie den Eindruck\strikeout{\textcolor{gray}{,}} einer raſch und billig hergeſtellten Großſtadt. Aber überall fehlt Cultur und
               Schönheit. Immerhin iſt Vieles impoſant; und die Leute ſitzen da und hören Einem \strikeout{zu, oh} ſogar zu, als ahnten ſie, daß es noch etwas
               jenſeits ihres Horizontes gibt – was mich überraſcht hat. Freilich das ſind {\pb}doch wohl flüchtige und vielleicht falſche
               Eindrücke.\pend
           \pstart
           Meine arme Mama\pwindex{Goldmann, Clementine 1842-05-15 – 1924-02-24@\textsc{Goldmann, Clementine} (1842-05-15 – 1924-02-24)|pwv} iſt geſtern unter vielen Thränen nach Frankfurt\oindex{Frankfurt am Main@\textbf{Frankfurt am Main}|pw} gefahren. Was daraus werden ſoll, weiß ich nicht.
               Einſtweilen muß ich meine \label{K_L02784-88v}\edtext{Monatsrate}{\lemma{\textnormal{\emph{Monatsrate}}}\Cendnote{\textnormal{Damit dürfte eine
                  Unterhaltszahlung für Clementine Goldmann\pwindex{Goldmann, Clementine 1842-05-15 – 1924-02-24@\textsc{Goldmann, Clementine} (1842-05-15 – 1924-02-24)|pwk}
                  gemeint sein.}}}\label{K_L02784-88h} erhöhen. Ich kanns natürlich nicht, aber ich muß es.\pend
           \pstart
           Mir grauſt vor \textsc{Paris\oindex{Paris@\textbf{Paris}|pw}} – das heißt vor der Arbeit, die \strikeout{ich} mich {\pb}dort erwartet, und auch an dieſer Arbeit iſt nur
               ſchrecklich, daß ſie ſo ganz vergeblich iſt. Ich ſehe es \strikeout{\textcolor{gray}{×}\-\textcolor{gray}{×}} klarer wie je: Alles, was ich dort arbeite, kommt nur meinem Chef\pwindex{Sonnemann, Leopold 1831-10-29 – 1909-10-30@\textsc{Sonnemann, Leopold} (1831-10-29 – 1909-10-30), \emph{Journalist, Herausgeber}|pwv} zu gute, nicht mir. All’ dieſe
               Rieſen-Anſtrengung da drüben zählt nicht, und ich müßte \strikeout{eig} noch nach dem ermüdenden Arbeitstage Zeit und Kraft finden, um das
               Eigentliche zu arbeiten, das erſt zählen würde. Unter {\pb}dieſen Umſtänden muß man müde und muthlos
               werden.\pend
           \pstart
           Grüß’ Dich Gott, mein lieber Arthur, und hab’ Dank für Deine Treue und Freundſchaft
               und für die ſchönen \label{K_L02784-13v}\edtext{Tage von \textsc{Skodsborg\oindex{Skodsborg@\textbf{Skodsborg}|pw}}}{\lemma{\textnormal{\emph{Tage von Skodsborg}}}\Cendnote{\textnormal{Nachdem Goldmann\pwindex{Goldmann, Paul 31.01.1865 – 25.09.1935@\textsc{Goldmann, Paul} (31.01.1865 – 25.09.1935), \emph{Schriftsteller, Journalist}|pwk} von Schnitzler\pwindex{Schnitzler, Arthur 15.05.1862 – 21.10.1931@\textsc{Schnitzler, Arthur} (15.05.1862 – 21.10.1931), \emph{Schriftsteller, Mediziner}|pwk}, Richard Beer-Hofmann\pwindex{Beer-Hofmann, Richard 1866-07-11 – 1945-09-26@\textsc{Beer-Hofmann, Richard} (1866-07-11 – 1945-09-26), \emph{Schriftsteller}|pwk}
                  und vermutlich auch Paula Beer-Hofmann\pwindex{Beer-Hofmann, Paula 25.02.1879 – 30.10.1939@\textsc{Beer-Hofmann, Paula} (25.02.1879 – 30.10.1939)|pwk} am
                     5. 8. 1896 in Kopenhagen\oindex{Kopenhagen@\textbf{Kopenhagen}|pwk} abgeholt wurde (vgl. A. S.: \emph{Tagebuch}, 8. 8. 1896), dürfte er bis um den 20. 8. 1896 mit ihnen in Skodsborg\oindex{Skodsborg@\textbf{Skodsborg}|pwk} gewesen sein. Am 21. 8. 1896 war er jedenfalls, wenn auch
                  womöglich nur für einen Tag, wieder in Kopenhagen\oindex{Kopenhagen@\textbf{Kopenhagen}|pwk}, zu Besuch bei Peter
                     und Betty Nansen\pwindex{Nansen, Peter 20.01.1861 – 31.07.1918@\textsc{Nansen, Peter} (20.01.1861 – 31.07.1918), \emph{Schriftsteller, Journalist, Verleger}|pwk}\pwindex{Nansen, Betty 19.3.1873 – 15.3.1943@\textsc{Nansen, Betty} (19.3.1873 – 15.3.1943), \emph{Theaterleiterin, Schauspielerin}|pwk}.}}}\label{K_L02784-13h} (nicht wahr, ſie waren ſchön?)\pend
           \pstart
           Empfiehl’ mich Deiner Frau Mutter\pwindex{Schnitzler, Louise 1840-07-08 – 1911-09-09@\textsc{Schnitzler, Louise} (1840-07-08 – 1911-09-09)|pwv}, deinem Bruder\pwindex{Schnitzler, Julius 13.07.1865 – 29.06.1939@\textsc{Schnitzler, Julius} (13.07.1865 – 29.06.1939), \emph{Chirurg}|pwv}, deiner Schwägerin\pwindex{Schnitzler, Helene 16.07.1871 – September 1941@\textsc{Schnitzler, Helene} (16.07.1871 – September 1941)|pwv}, Deiner Schweſter\pwindex{Hajek, Gisela 20.12.1867 – 03.02.1953@\textsc{Hajek, Gisela} (20.12.1867 – 03.02.1953)|pwv} und {\pb}Deinem Schwager\pwindex{Hajek, Markus 25.11.1861 – 04.04.1941@\textsc{Hajek, Markus} (25.11.1861 – 04.04.1941), \emph{Mediziner, Laryngologe}|pwv}.\pend
           \pstart
           Empfiehl’ mich auch der unbekannten \label{K_L02784-881v}\edtext{Dame\pwindex{Reinhard, Marie 1871-03-13 – 1899-03-18@\textsc{Reinhard, Marie} (1871-03-13 – 1899-03-18), \emph{Gesangspädagogin}|pwv}}{\lemma{\textnormal{\emph{Dame}}}\Cendnote{\textnormal{Vgl. den Brief von Schnitzler\pwindex{Schnitzler, Arthur 15.05.1862 – 21.10.1931@\textsc{Schnitzler, Arthur} (15.05.1862 – 21.10.1931), \emph{Schriftsteller, Mediziner}|pwk} an Marie
                     Reinhard\pwindex{Reinhard, Marie 1871-03-13 – 1899-03-18@\textsc{Reinhard, Marie} (1871-03-13 – 1899-03-18), \emph{Gesangspädagogin}|pwk}, 25. 7. 1896: »— Mit Altenberg\pwindex{Altenberg, Peter 09.03.1859 – 08.01.1919@\textsc{Altenberg, Peter} (09.03.1859 – 08.01.1919), \emph{Schriftsteller}!Wie ich es sehe1896@\strich\emph{Wie ich es sehe} {[}1896{]}|pwv}\pwindex{Altenberg, Peter 09.03.1859 – 08.01.1919@\textsc{Altenberg, Peter} (09.03.1859 – 08.01.1919), \emph{Schriftsteller}|pw} hast du ganz recht; freilich ist noch mehr zu sagen. Danke herzlich dass
                     du ihn an G.[oldmann]\pwindex{Goldmann, Paul 31.01.1865 – 25.09.1935@\textsc{Goldmann, Paul} (31.01.1865 – 25.09.1935), \emph{Schriftsteller, Journalist}|pw} geschickt; er hat
                     ihn schon zum Theil gelesen.« (\emph{Arthur Schnitzler an Marie Reinhard (1896)}. Hg.
                     Therese Nickl. In: \emph{Modern Austrian Literature}, Jg. 10
                        (1977) H. 3/4, S. 42)}}}\label{K_L02784-881h}, die mir den \textsc{Altenberg\pwindex{Altenberg, Peter 09.03.1859 – 08.01.1919@\textsc{Altenberg, Peter} (09.03.1859 – 08.01.1919), \emph{Schriftsteller}!Wie ich es sehe1896@\strich\emph{Wie ich es sehe} {[}1896{]}|pwv}\pwindex{Altenberg, Peter 09.03.1859 – 08.01.1919@\textsc{Altenberg, Peter} (09.03.1859 – 08.01.1919), \emph{Schriftsteller}|pw}} überſandt hat.\pend
           \pstart
           In Treue {\\[\baselineskip]}Dein {\\[\baselineskip]}\spacefill\mbox{Paul Goldmann}\pend
           \leftskip=0em{}\pstart
           \noindent{}Schreib’ mir bald nach \textsc{Paris\oindex{Paris@\textbf{Paris}|pw}}.\pend
           \pstart
           Wann gehſt Du \label{K_L02784-14v}\edtext{nach \textsc{Berlin\oindex{Berlin@\textbf{Berlin}|pw}}}{\lemma{\textnormal{\emph{nach Berlin}}}\Cendnote{\textnormal{Schnitzler\pwindex{Schnitzler, Arthur 15.05.1862 – 21.10.1931@\textsc{Schnitzler, Arthur} (15.05.1862 – 21.10.1931), \emph{Schriftsteller, Mediziner}|pwk} war bereits von 22. 8. 1896 bis
                        26. 8. 1896 in
                        Berlin\oindex{Berlin@\textbf{Berlin}|pwk}. Das nächste Mal war er dort
                     zwischen 26. 10. 1896 und 9. 11. 1896.}}}\label{K_L02784-14h}?\pend
           
         
         \endnumbering\mylabel{h}\end{ledgroupsized}  \newcommand{\dateiname}{L02784}\newcommand{\titel}{Paul Goldmann an Arthur Schnitzler, 7. 9. [1896]}\newcommand{\editorInnen}{Martin Anton Müller und Laura Untner}%% latex-leseansicht-abspann.tex
%% Abspann für die Leseansicht.
%% Der Schalter \ifkorrekturansicht ist bereits durch den Vorspann gesetzt.

%% latex-abspann.tex
%% Gemeinsamer Abspann für Korrekturansicht und Leseansicht.
%% Setzt den Schalter \ifkorrekturansicht voraus (gesetzt in den
%% einbindenden Dateien latex-korrekturansicht-abspann.tex bzw.
%% latex-leseansicht-abspann.tex).
%% ---------------------------------------------------------------

\normalsize

% Das esempio-Environment wird nur in der Leseansicht benötigt
\ifkorrekturansicht\else
\newenvironment{esempio}[3]%
{
    \vspace{1.5ex}
    \rlap{\underline{#1}}
    \par
    \setlength{\parindent}{0cm}
    \nopagebreak
    \leftskip=#2cm
    \rightskip=#3cm
}
{
    \par
}
\fi

\doendnotes{C}
\bigskip
\vfill

\clearpage

\footnotesize

\ifkorrekturansicht
  \lohead{\textsc{register}}
\fi

% theindex-Environment neu definieren ohne reledmac
\makeatletter
\renewenvironment{theindex}{%
  \ifkorrekturansicht
    \section*{\indexname}%
  \else
    \subsubsection*{Index der erwähnten Entitäten}%
  \fi
  \setlength{\parindent}{0pt}%
  \setlength{\parskip}{0pt plus 0.3pt}%
  \let\item\@idxitem
}{%
  \ifkorrekturansicht\clearpage\fi
}
\makeatother

\IfFileExists{\jobname-pw.ind}{\input{\jobname-pw.ind}}{}

% Quellenangabe nur in der Leseansicht
\ifkorrekturansicht\else
% Fallback-Definitionen, falls die .tex-Datei \titel etc. nicht gesetzt hat
\providecommand{\titel}{}
\providecommand{\editorInnen}{}
\providecommand{\dateiname}{\jobname}

\vspace{3cm}

\vfill

\footnotesize
\textsc{Quelle}: \titel. Herausgegeben von {\editorInnen}. In: \emph{Arthur Schnitzler: Briefwechsel mit Autorinnen und Autoren}.
 Digitale Edition, https://schnitzler-briefe.acdh.oeaw.ac.at/{\dateiname}.html (Stand \today)
\fi

\end{document}


      