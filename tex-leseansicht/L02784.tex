%% latex-leseansicht-vorspann.tex
%% Vorspann für die Leseansicht.
%% Lädt die gemeinsame Datei latex-vorspann.tex mit nicht gesetztem Schalter.

\newif\ifkorrekturansicht
\korrekturansichtfalse

\input{../tex-inputs/latex-vorspann}


\section[ Paul Goldmann an Arthur Schnitzler, 7. 9. [1896]]{L02784 Paul Goldmann an Arthur Schnitzler,  7. 9. [1896]}
\nopagebreak\mylabel{L02784v}
\rehead{ }\normalsize\beginnumbering\briefempfaengerindex{Schnitzler, Arthur@\textsc{Schnitzler, Arthur}!zzzGoldmann, Paul@\emph{von Paul Goldmann}!1896-09-072@{7. 9. [1896]}|(be}
\toendnotes[C]{\smallbreak\pagebreak[2]}
\correspDesc{Versand  durch Paul Goldmann am 7. 9. [1896] in Berlin
\newline{}Erhalt  durch Arthur Schnitzler im Zeitraum [8. 9. 1896
                  – 12. 9. 1896?] in Wien}\toendnotes[C]{\smallbreak}
\Standort{DLA, A:Schnitzler, HS.NZ85.1.3166.}
\physDesc{Brief, 3 Blätter, 12 Seiten, 4335 Zeichen
\newline{}Handschrift: schwarze Tinte, deutsche Kurrent
\newline{}Schnitzler: 1) mit Bleistift das Jahr »96« vermerkt  2) mit rotem Buntstift zwölf Unterstreichungen}\toendnotes[C]{\smallbreak}
\pstart
           {\pb}\textcolor{gray}{\textbf{\textbf{Frankfurter Zeitung\orgindex{Frankfurter Zeitung@Frankfurter Zeitung|pw}}}}\pend
           
\pstart
           \textcolor{gray}{\textbf{(\begin{otherlanguage}{french}Gazette de Francfort\end{otherlanguage}\orgindex{Frankfurter Zeitung@Frankfurter Zeitung|pw}).}}\pend
           
\pstart
           \textcolor{gray}{\textbf{\textbf{\begin{otherlanguage}{french}Fondateur M.\end{otherlanguage}{ }L. Sonnemann\pwindex{Sonnemann, Leopold 29.\,10.\,1831 Höchberg – 30.\,10.\,1909 Frankfurt am Main@\textsc{Sonnemann, Leopold} (29.\,10.\,1831 Höchberg – 30.\,10.\,1909 Frankfurt am Main), \emph{Journalist, Herausgeber}|pw}.}}}\pend
           
\pstart
           \begin{otherlanguage}{french}\textcolor{gray}{\textbf{Journal\pwindex{Frankfurter Zeitung@\emph{Frankfurter Zeitung}|pwv} politique,
                        financier,}}\end{otherlanguage}\pend
           
\pstart
           \begin{otherlanguage}{french}\textcolor{gray}{\textbf{commercial et littéraire.}}\end{otherlanguage}\pend
           
\pstart
           \begin{otherlanguage}{french}\textcolor{gray}{\textbf{\textbf{Paraissant trois fois par jour.}}}\end{otherlanguage}\pend
           
\pstart
           \begin{otherlanguage}{french}\textcolor{gray}{\textbf{\textbf{Bureau à Paris\oindex{Paris@\textbf{Paris}, \emph{Hauptstadt}|pw}}}}\end{otherlanguage}\pend
           
\pstart
           \begin{otherlanguage}{french}\textcolor{gray}{\textbf{\textbf{24. Rue Feydeau\oindex{rue Feydeau@\textbf{rue Feydeau}, \emph{Straße}|pw}.}}}\end{otherlanguage}\hfill \textsc{Berlin\oindex{Berlin@\textbf{Berlin}, \emph{Hauptstadt}|pw}}, 7. September.\pend
           
\pstart\center{}Mein lieber Freund,\pend\vspace{0.5em}
\pstart
           Morgen, Dienſtag, \label{K_L02784-1v}\edtext{fahre ich heim}{\lemma{\textnormal{\emph{fahre ich heim}}}\Cendnote{\textnormal{Schnitzler war bereits am 26. 8. 1896 von Berlin\oindex{Berlin@\textbf{Berlin}, \emph{Hauptstadt}|pwk} über München\oindex{München@\textbf{München}|pwk} nach Wien\oindex{Wien@\textbf{Wien}, \emph{Verwaltungsgebiet}|pwk} gereist, wo er am
                     29. 8. 1896
                  ankam.}}}\label{K_L02784-1} (»heim« iſt gut!), und Dein lieber Brief iſt das letzte Angenehme,
               das mir hier widerfährt.\pend
           
\pstart
           Ich freue mich, daß Du glücklich wieder in Wien\oindex{Wien@\textbf{Wien}, \emph{Verwaltungsgebiet}|pw}
               biſt und dort Alles beim Rechten gefunden haſt.\pend
           
\pstart
           \label{K_L02784-2v}\edtext{\textsc{Burckhardts\pwindex{Burckhard, Max Eugen 14.\,7.\,1854 Korneuburg – 16.\,3.\,1912 Wien@\textsc{Burckhard, Max Eugen} (14.\,7.\,1854 Korneuburg – 16.\,3.\,1912 Wien), \emph{Schriftsteller, Rechtswissenschaftler, Theaterleiter}|pw}} Begeiſterung}{\lemma{\textnormal{\emph{Burckhardts Begeisterung}}}\Cendnote{\textnormal{Siehe A. S.: \emph{Tagebuch}, 4. 9. 1896. }}}\label{K_L02784-2} für
               Dein Stück\pwindex{Schnitzler, Arthur 15.\,5.\,1862 Wien – 21.\,10.\,1931 ebd.@\textsc{Schnitzler, Arthur} (15.\,5.\,1862 Wien – 21.\,10.\,1931 ebd.), \emph{Schriftsteller, Mediziner}!Freiwild. Schauspiel in 3 Akten@\strich\emph{Freiwild. Schauspiel in 3 Akten}|pwv} iſt ein weiteres
               gutes \textsc{omen}. Daß das Werk\pwindex{Schnitzler, Arthur 15.\,5.\,1862 Wien – 21.\,10.\,1931 ebd.@\textsc{Schnitzler, Arthur} (15.\,5.\,1862 Wien – 21.\,10.\,1931 ebd.), \emph{Schriftsteller, Mediziner}!Freiwild. Schauspiel in 3 Akten@\strich\emph{Freiwild. Schauspiel in 3 Akten}|pwv} den Theaterleuten{ }ſo gefällt, iſt das{ }ſtärkſte Zeugniß
               für die Theater-Wirkung, die man {\pb}davon erwarten
               kann. Warum B.\pwindex{Burckhard, Max Eugen 14.\,7.\,1854 Korneuburg – 16.\,3.\,1912 Wien@\textsc{Burckhard, Max Eugen} (14.\,7.\,1854 Korneuburg – 16.\,3.\,1912 Wien), \emph{Schriftsteller, Rechtswissenschaftler, Theaterleiter}|pwv}{ }ſämmtliche
               noch überlebenden Perſonen des Stück\pwindex{Schnitzler, Arthur 15.\,5.\,1862 Wien – 21.\,10.\,1931 ebd.@\textsc{Schnitzler, Arthur} (15.\,5.\,1862 Wien – 21.\,10.\,1931 ebd.), \emph{Schriftsteller, Mediziner}!Freiwild. Schauspiel in 3 Akten@\strich\emph{Freiwild. Schauspiel in 3 Akten}|pwv}es \strikeout{\textcolor{gray}{von d}\textcolor{gray}{×}\-\textcolor{gray}{×}} umbringen will, iſt mir nicht recht begreiflich. Dieſe Abänderungs-\label{K_L02784-3v}\edtext{Vorſchläge}{\lemma{\textnormal{\emph{Vorschläge}}}\Cendnote{\textnormal{In der Vorlage steht: »Vorſchlage«.}}}\label{K_L02784-3}{ }ſind{ }ſehr komiſch. Da wüßte ich viel beſſere: \textsc{Anna\pwindex{Schnitzler, Arthur 15.\,5.\,1862 Wien – 21.\,10.\,1931 ebd.@\textsc{Schnitzler, Arthur} (15.\,5.\,1862 Wien – 21.\,10.\,1931 ebd.), \emph{Schriftsteller, Mediziner}!Freiwild. Schauspiel in 3 Akten@\strich\emph{Freiwild. Schauspiel in 3 Akten}|pwv}}{ }ſoll den Kaſſierer \textsc{Kohn\pwindex{Schnitzler, Arthur 15.\,5.\,1862 Wien – 21.\,10.\,1931 ebd.@\textsc{Schnitzler, Arthur} (15.\,5.\,1862 Wien – 21.\,10.\,1931 ebd.), \emph{Schriftsteller, Mediziner}!Freiwild. Schauspiel in 3 Akten@\strich\emph{Freiwild. Schauspiel in 3 Akten}|pwv}} heirathen und \textsc{Vogel\pwindex{Schnitzler, Arthur 15.\,5.\,1862 Wien – 21.\,10.\,1931 ebd.@\textsc{Schnitzler, Arthur} (15.\,5.\,1862 Wien – 21.\,10.\,1931 ebd.), \emph{Schriftsteller, Mediziner}!Freiwild. Schauspiel in 3 Akten@\strich\emph{Freiwild. Schauspiel in 3 Akten}|pwv}}{ }ſoll in dem Theater-Director{ }ſeinen verloren geglaubten Vater wiederfinden{\dotsfive}\pend
           
\pstart
           Die \label{K_L02784-4v}\edtext{Äußerung des allerhöchſten Herrn\pwindex{Franz Joseph I. von Österreich-Ungarn 18.\,8.\,1830 Wien – 21.\,11.\,1916 ebd.@\textsc{Franz Joseph I. von Österreich-Ungarn} (18.\,8.\,1830 Wien – 21.\,11.\,1916 ebd.), \emph{Kaiser}|pwv} über »Liebelei\pwindex{Schnitzler, Arthur 15.\,5.\,1862 Wien – 21.\,10.\,1931 ebd.@\textsc{Schnitzler, Arthur} (15.\,5.\,1862 Wien – 21.\,10.\,1931 ebd.), \emph{Schriftsteller, Mediziner}!Liebelei. Schauspiel in drei Akten@\strich\emph{Liebelei. Schauspiel in drei Akten}|pw}«}{\lemma{\textnormal{\emph{Äußerung … »Liebelei«}}}\Cendnote{\textnormal{Siehe A. S.: \emph{Tagebuch}, 5. 9. 1896. }}}\label{K_L02784-4} iſt
               köſtlich. Ich hoffe, Seine Majeſtät\pwindex{Franz Joseph I. von Österreich-Ungarn 18.\,8.\,1830 Wien – 21.\,11.\,1916 ebd.@\textsc{Franz Joseph I. von Österreich-Ungarn} (18.\,8.\,1830 Wien – 21.\,11.\,1916 ebd.), \emph{Kaiser}|pwv} verſteht vom Regieren mehr, wie von der Kunſt, {\pb}ſonſt müßte man mit großer Beſorgniß in die Zukunft
                  Öſterreichs\oindex{Österreich@\textbf{Österreich}|pw} blicken. \label{K_L02784-5v}\edtext{\textsc{Mitterwurzer\pwindex{Mitterwurzer, Friedrich 16.\,10.\,1844 Dresden – 13.\,2.\,1897 Wien@\textsc{Mitterwurzer, Friedrich} (16.\,10.\,1844 Dresden – 13.\,2.\,1897 Wien), \emph{Schauspieler}|pw}} iſt{ }ſo der rechte Sau-Komödiant\pwindex{Mitterwurzer, Friedrich 16.\,10.\,1844 Dresden – 13.\,2.\,1897 Wien@\textsc{Mitterwurzer, Friedrich} (16.\,10.\,1844 Dresden – 13.\,2.\,1897 Wien), \emph{Schauspieler}|pwv}}{\lemma{\textnormal{\emph{Mitterwurzer … Sau-Komödiant}}}\Cendnote{\textnormal{Siehe A. S.: \emph{Tagebuch}, 5. 9. 1896. }}}\label{K_L02784-5}.
               Schreib’ \strikeout{ihn} ihm einmal eine Rolle, in der er Erfolg
               hat, und er wird Dich als das erſte Genie der Welt ausſchreien.\pend
           
\pstart
           Von \textsc{Richard\pwindex{Beer-Hofmann, Richard 11.\,7.\,1866 Wien – 26.\,9.\,1945 New York City@\textsc{Beer-Hofmann, Richard} (11.\,7.\,1866 Wien – 26.\,9.\,1945 New York City), \emph{Schriftsteller}|pw}} weiß ich Dir wenig zu{ }ſagen. Er muß{ }ſchon \label{K_L02784-6v}\edtext{in \textsc{Baden\oindex{Kaiser-Franz-Ring@\textbf{Kaiser-Franz-Ring}, \emph{Straße}|pw}}}{\lemma{\textnormal{\emph{in Baden}}}\Cendnote{\textnormal{Siehe XXXX Auszeichnungsfehler: Dokument L00585 nicht gefunden. }}}\label{K_L02784-6}{ }ſein.
               Während der letzten Tage{ }ſeines Hierſeins war er nervös und erging{ }ſich in
               unangenehmen Betrachtungen über die »guten Menſchen«. \textsc{Paula\pwindex{Beer-Hofmann, Paula 25.\,2.\,1879 Wien – 30.\,10.\,1939 Zürich@\textsc{Beer-Hofmann, Paula} (25.\,2.\,1879 Wien – 30.\,10.\,1939 Zürich)|pw}} hat er {\pb}fortgeſchickt;{ }ſie wollte natürlich
               zum Schluß durchaus noch dableiben, weil{ }ſie bei \label{K_L02784-7v}\edtext{\textsc{Hagenbeck\oindex{Hagenbecks Tierpark@\textbf{Hagenbecks Tierpark}, \emph{Zoo}|pw}}}{\lemma{\textnormal{\emph{Hagenbeck}}}\Cendnote{\textnormal{Hamburg\oindex{Hamburg@\textbf{Hamburg}|pwk}er Tierpark\oindex{Hagenbecks Tierpark@\textbf{Hagenbecks Tierpark}, \emph{Zoo}|pwkv}}}}\label{K_L02784-7}{ }ſo{ }ſchöne Affen und Raubthiere geſehen hatte.\pend
           
\pstart
           Was mich anlangt,{ }ſo{ }ſind mir die Tage in Berlin\oindex{Berlin@\textbf{Berlin}, \emph{Hauptstadt}|pw}
               recht angenehm verfloſſen. Der liebſte unter den Menſchen, die ich hier kennen
               gelernt, iſt mir Dr. \textsc{Bie\pwindex{Bie, Oskar 9.\,2.\,1864 Breslau – 21.\,4.\,1938 Berlin@\textsc{Bie, Oskar} (9.\,2.\,1864 Breslau – 21.\,4.\,1938 Berlin), \emph{Schriftsteller, Journalist, Redakteur}|pw}}. Er iſt ehrlich und gut. Wir verſtehen uns und haben uns wohl auch gern. \textsc{Kerr\pwindex{Kerr, Alfred 25.\,12.\,1867 Breslau – 12.\,10.\,1948 Hamburg@\textsc{Kerr, Alfred} (25.\,12.\,1867 Breslau – 12.\,10.\,1948 Hamburg), \emph{Schriftsteller, Kritiker}|pw}} mag ich weniger. Ich wittere in ihm {\pb}den
                  \label{K_L02784-8v}\edtext{\begin{otherlanguage}{french}\textsc{froid ambitieux}\end{otherlanguage}}{\lemma{\textnormal{\emph{froid ambitieux}}}\Cendnote{\textnormal{französisch: kühler Ehrgeizling}}}\label{K_L02784-8}.
               Mit \textsc{Brahm\pwindex{Brahm, Otto 5.\,2.\,1856 Hamburg – 28.\,11.\,1912 Berlin@\textsc{Brahm, Otto} (5.\,2.\,1856 Hamburg – 28.\,11.\,1912 Berlin), \emph{Theaterleiter, Regisseur}|pw}}, \textsc{Rittner\pwindex{Rittner, Rudolf 30.\,6.\,1869 Bílý Potok – 4.\,2.\,1943 ebd.@\textsc{Rittner, Rudolf} (30.\,6.\,1869 Bílý Potok – 4.\,2.\,1943 ebd.), \emph{Theaterleiter, Schauspieler}|pw}} und \textsc{Richard\pwindex{Beer-Hofmann, Richard 11.\,7.\,1866 Wien – 26.\,9.\,1945 New York City@\textsc{Beer-Hofmann, Richard} (11.\,7.\,1866 Wien – 26.\,9.\,1945 New York City), \emph{Schriftsteller}|pw}} verbrachte ich einen Abend. \textsc{Rittner\pwindex{Rittner, Rudolf 30.\,6.\,1869 Bílý Potok – 4.\,2.\,1943 ebd.@\textsc{Rittner, Rudolf} (30.\,6.\,1869 Bílý Potok – 4.\,2.\,1943 ebd.), \emph{Theaterleiter, Schauspieler}|pw}} gefiel auch mir ausnehmend. \textsc{Brahm\pwindex{Brahm, Otto 5.\,2.\,1856 Hamburg – 28.\,11.\,1912 Berlin@\textsc{Brahm, Otto} (5.\,2.\,1856 Hamburg – 28.\,11.\,1912 Berlin), \emph{Theaterleiter, Regisseur}|pw}} forderte mich auf, ihm noch einmal Rendezvous für einen Abend zu geben. Ich
               hab’ es aber nicht gethan; ich glaub’ nicht, daß ihm irgend etwas an mir liegt. \textsc{Fischer\pwindex{Fischer, Samuel 24.\,12.\,1859 Liptovský Mikuláš – 15.\,10.\,1934 Berlin@\textsc{Fischer, Samuel} (24.\,12.\,1859 Liptovský Mikuláš – 15.\,10.\,1934 Berlin), \emph{Verleger}|pw}} hat{ }ſofort \strikeout{\textcolor{gray}{×}} in mir einen ausnutzbaren Mann geſehen, hat \strikeout{mich}{ }ſich von mir einige Stunden über \textsc{Paris\oindex{Paris@\textbf{Paris}, \emph{Hauptstadt}|pw}} erzählen {\pb}laſſen, hat mich auch zum Abendeſſen
               geladen. \strikeout{Das} Die Herausgabe der Humoriſten hat er
               natürlich abgelehnt. Hingegen wird{ }ſeine Frau\pwindex{Fischer, Hedwig 8.\,9.\,1871 Szczecin – 11.\,4.\,1952 Königstein im Taunus@\textsc{Fischer, Hedwig} (8.\,9.\,1871 Szczecin – 11.\,4.\,1952 Königstein im Taunus)|pwv} wohl einen oder den anderen von dieſen Leuten jetzt
               überſetzen, angeregt durch die Lectüre meiner \label{K_L02784-9v}\edtext{Feuilletons}{\lemma{\textnormal{\emph{Feuilletons}}}\Cendnote{\textnormal{Goldmann\pwindex{Goldmann, Paul 31.\,1.\,1865 Breslau – 25.\,9.\,1935 Wien@\textsc{Goldmann, Paul} (31.\,1.\,1865 Breslau – 25.\,9.\,1935 Wien), \emph{Schriftsteller, Journalist}|pwk} hat in seiner Feuilletonreihe \emph{Neue französische Humoristen}\pwindex{Goldmann, Paul 31.\,1.\,1865 Breslau – 25.\,9.\,1935 Wien@\textsc{Goldmann, Paul} (31.\,1.\,1865 Breslau – 25.\,9.\,1935 Wien), \emph{Schriftsteller, Journalist}!Neue französische Humoristen@\strich\emph{Neue französische Humoristen}|pwk} in der \emph{Frankfurter Zeitung}\pwindex{Frankfurter Zeitung@\emph{Frankfurter Zeitung}|pwk} verschiedene
                  Literaturschaffende vorgestellt, jeweils mit einer kurzen Einleitung und einer
                  kleinen Übersetzung. Während die ersten Beiträge nachgewiesen werden können, muss
                  offen bleiben, wie viele Beiträge in Folge erschienen sind. \emph{Alphonse Allais}\pwindex{Goldmann, Paul 31.\,1.\,1865 Breslau – 25.\,9.\,1935 Wien@\textsc{Goldmann, Paul} (31.\,1.\,1865 Breslau – 25.\,9.\,1935 Wien), \emph{Schriftsteller, Journalist}!Neue französische Humoristen. Alphonse Allais@\strich\emph{Neue französische Humoristen. Alphonse Allais}|pwk}, 3. 9. 1893; \emph{Georges Courteline}\pwindex{Goldmann, Paul 31.\,1.\,1865 Breslau – 25.\,9.\,1935 Wien@\textsc{Goldmann, Paul} (31.\,1.\,1865 Breslau – 25.\,9.\,1935 Wien), \emph{Schriftsteller, Journalist}!Neue französische Humoristen. Georges Courteline@\strich\emph{Neue französische Humoristen. Georges Courteline}|pwk},
                        31. 12. 1893 und 1. 1. 1894; \emph{L. Xanrof}\pwindex{Goldmann, Paul 31.\,1.\,1865 Breslau – 25.\,9.\,1935 Wien@\textsc{Goldmann, Paul} (31.\,1.\,1865 Breslau – 25.\,9.\,1935 Wien), \emph{Schriftsteller, Journalist}!Neue französische Humoristen. L. Xanrof@\strich\emph{Neue französische Humoristen. L. Xanrof}|pwk}, 25. 3. 1894, \emph{Pierre Veber}\pwindex{Goldmann, Paul 31.\,1.\,1865 Breslau – 25.\,9.\,1935 Wien@\textsc{Goldmann, Paul} (31.\,1.\,1865 Breslau – 25.\,9.\,1935 Wien), \emph{Schriftsteller, Journalist}!Neue französische Humoristen. Pierre Veber@\strich\emph{Neue französische Humoristen. Pierre Veber}|pwk}, 11. 5. 1894
                     und 13. 5. 1894; \emph{Narcisse Lebeau}\pwindex{Goldmann, Paul 31.\,1.\,1865 Breslau – 25.\,9.\,1935 Wien@\textsc{Goldmann, Paul} (31.\,1.\,1865 Breslau – 25.\,9.\,1935 Wien), \emph{Schriftsteller, Journalist}!Neue französische Humoristen. Narcisse Lebeau@\strich\emph{Neue französische Humoristen. Narcisse Lebeau}|pwk}, 5. 10. 1894; \emph{Tristan Bernard. – Georges Auriol. – Bill
                        Sharp. – Maurice O’Reilly}\pwindex{Goldmann, Paul 31.\,1.\,1865 Breslau – 25.\,9.\,1935 Wien@\textsc{Goldmann, Paul} (31.\,1.\,1865 Breslau – 25.\,9.\,1935 Wien), \emph{Schriftsteller, Journalist}!Neue französische Humoristen. Tristan Bernard. – Georges Auriol. – Bill Sharp. – Maurice O’Reilly@\strich\emph{Neue französische Humoristen. Tristan Bernard. – Georges Auriol. – Bill Sharp. – Maurice O’Reilly}|pwk}, 14. 4. 1894 und
                        17. 4. 1894. Zu Übersetzungen von diesen Autoren\pwindex{Lebeau, Narcisse 9.\,4.\,1865 – 12.\,7.\,1931@\textsc{Lebeau, Narcisse} (9.\,4.\,1865 – 12.\,7.\,1931), \emph{Schriftsteller, Humorist, Klempner}|pwkv}\pwindex{Allais, Alphonse 20.\,10.\,1854 Honfleur – 28.\,10.\,1905 Paris@\textsc{Allais, Alphonse} (20.\,10.\,1854 Honfleur – 28.\,10.\,1905 Paris), \emph{Schriftsteller}|pwkv}\pwindex{Courteline, Georges 25.\,6.\,1858 Tours – 25.\,6.\,1929 Paris@\textsc{Courteline, Georges} (25.\,6.\,1858 Tours – 25.\,6.\,1929 Paris), \emph{Schriftsteller}|pwkv}\pwindex{Xanrof, Léon 9.\,12.\,1867 Paris – 17.\,5.\,1953 ebd.@\textsc{Xanrof, Léon} (9.\,12.\,1867 Paris – 17.\,5.\,1953 ebd.), \emph{Dramatiker, Humorist}|pwkv}\pwindex{Bernard, Tristan 7.\,9.\,1866 Besançon – 7.\,12.\,1947 Paris@\textsc{Bernard, Tristan} (7.\,9.\,1866 Besançon – 7.\,12.\,1947 Paris), \emph{Schriftsteller}|pwkv}\pwindex{Auriol, George 27.\,4.\,1863 Beauvais – 6.\,2.\,1938 Paris@\textsc{Auriol, George} (27.\,4.\,1863 Beauvais – 6.\,2.\,1938 Paris), \emph{Maler, Lyriker}|pwkv}\pwindex{Veber, Pierre 15.\,5.\,1869 Paris – 20.\,8.\,1942 ebd.@\textsc{Veber, Pierre} (15.\,5.\,1869 Paris – 20.\,8.\,1942 ebd.), \emph{Schriftsteller}|pwkv}\pwindex{O’Reilly, Maurice @\textsc{O’Reilly, Maurice}, \emph{Kabarettist, Diplomat}|pwkv} durch Hedwig
                     Fischer\pwindex{Fischer, Hedwig 8.\,9.\,1871 Szczecin – 11.\,4.\,1952 Königstein im Taunus@\textsc{Fischer, Hedwig} (8.\,9.\,1871 Szczecin – 11.\,4.\,1952 Königstein im Taunus)|pwk} konnte nichts gefunden werden.}}}\label{K_L02784-9}! Das mindert nicht den
               Freundſchaftsdienſt, den Du mir haſt leiſten wollen, und ich danke Dir von ganzem
               Herzen dafür. Die \label{K_L02784-10v}\edtext{Zeichnung von \textsc{Forain\pwindex{Forain, Jean-Louis 23.\,10.\,1852 Reims – 11.\,7.\,1931 Paris@\textsc{Forain, Jean-Louis} (23.\,10.\,1852 Reims – 11.\,7.\,1931 Paris), \emph{Maler, Grafiker, Karikaturist}|pw}}}{\lemma{\textnormal{\emph{Zeichnung von Forain}}}\Cendnote{\textnormal{nicht ermittelt}}}\label{K_L02784-10}{ }{\pb}konnte ich ihm nicht zeigen. Ich habe{ }ſie dem \textsc{Richard\pwindex{Beer-Hofmann, Richard 11.\,7.\,1866 Wien – 26.\,9.\,1945 New York City@\textsc{Beer-Hofmann, Richard} (11.\,7.\,1866 Wien – 26.\,9.\,1945 New York City), \emph{Schriftsteller}|pw}} für Dich mitgegeben; derſelbe hat auch Deinen \textsc{Altenberg\pwindex{Altenberg, Peter 9.\,3.\,1859 Wien – 8.\,1.\,1919 ebd.@\textsc{Altenberg, Peter} (9.\,3.\,1859 Wien – 8.\,1.\,1919 ebd.), \emph{Schriftsteller}!Wie ich es sehe@\strich\emph{Wie ich es sehe}|pwv}\pwindex{Altenberg, Peter 9.\,3.\,1859 Wien – 8.\,1.\,1919 ebd.@\textsc{Altenberg, Peter} (9.\,3.\,1859 Wien – 8.\,1.\,1919 ebd.), \emph{Schriftsteller}|pw}}. Sag’ ihm, bitte, daß ich ihm \label{K_L02784-11v}\edtext{den \textsc{Gregorovius\pwindex{Gregorovius, Ferdinand 19.\,1.\,1821 Nidzica – 1.\,5.\,1891 München@\textsc{Gregorovius, Ferdinand} (19.\,1.\,1821 Nidzica – 1.\,5.\,1891 München), \emph{Schriftsteller, Historiker}|pw}}}{\lemma{\textnormal{\emph{den Gregorovius}}}\Cendnote{\textnormal{nicht ermittelt}}}\label{K_L02784-11}{ }ſofort nach
               meiner Ankunft in \textsc{Paris\oindex{Paris@\textbf{Paris}, \emph{Hauptstadt}|pw}}{ }ſchicken werde. Ich habe \strikeout{die} den Brief mit{ }ſeiner Baden\oindex{Baden bei Wien@\textbf{Baden bei Wien}, \emph{Hauptstadt}|pw}er Adreſſe verloren, und auch{ }ſeine
                  Wien\oindex{Wien@\textbf{Wien}, \emph{Verwaltungsgebiet}|pw}er Adreſſe finde ich erſt in \textsc{Paris\oindex{Paris@\textbf{Paris}, \emph{Hauptstadt}|pw}}.\pend
           
\pstart
           Sonſt hat mir \textsc{Berlin\oindex{Berlin@\textbf{Berlin}, \emph{Hauptstadt}|pw}} beſſer gefallen, als ich erwartet. Aber lieb {\pb}gewinnen könnte ich die Stadt\oindex{Berlin@\textbf{Berlin}, \emph{Hauptstadt}|pwv}
               wohl nicht. Im Großen und Ganzen macht{ }ſie den Eindruck\strikeout{\textcolor{gray}{,}} einer raſch und billig hergeſtellten Großſtadt. Aber überall fehlt Cultur und
               Schönheit. Immerhin iſt Vieles impoſant; und die Leute{ }ſitzen da und hören Einem \strikeout{zu, oh}{ }ſogar zu, als ahnten{ }ſie, daß es noch etwas
               jenſeits ihres Horizontes gibt – was mich überraſcht hat. Freilich das{ }ſind {\pb}doch wohl flüchtige und vielleicht falſche
               Eindrücke.\pend
           
\pstart
           Meine arme Mama\pwindex{Goldmann, Clementine 15.\,5.\,1842 Breslau – 24.\,2.\,1924 Frankfurt am Main@\textsc{Goldmann, Clementine} (15.\,5.\,1842 Breslau – 24.\,2.\,1924 Frankfurt am Main)|pwv} iſt geſtern unter vielen Thränen nach Frankfurt\oindex{Frankfurt am Main@\textbf{Frankfurt am Main}, \emph{Hauptstadt}|pw} gefahren. Was daraus werden{ }ſoll, weiß ich nicht.
               Einſtweilen muß ich meine \label{K_L02784-12v}\edtext{Monatsrate}{\lemma{\textnormal{\emph{Monatsrate}}}\Cendnote{\textnormal{Damit dürfte eine
                  Unterhaltszahlung für Clementine Goldmann\pwindex{Goldmann, Clementine 15.\,5.\,1842 Breslau – 24.\,2.\,1924 Frankfurt am Main@\textsc{Goldmann, Clementine} (15.\,5.\,1842 Breslau – 24.\,2.\,1924 Frankfurt am Main)|pwk}
                  gemeint sein.}}}\label{K_L02784-12} erhöhen. Ich kanns natürlich nicht, aber ich muß es.\pend
           
\pstart
           Mir grauſt vor \textsc{Paris\oindex{Paris@\textbf{Paris}, \emph{Hauptstadt}|pw}} – das heißt vor der Arbeit, die \strikeout{ich} mich {\pb}dort erwartet, und auch an dieſer Arbeit iſt nur{ }ſchrecklich, daß{ }ſie{ }ſo ganz vergeblich iſt. Ich{ }ſehe es \strikeout{\textcolor{gray}{×}\-\textcolor{gray}{×}} klarer wie je: Alles, was ich dort arbeite, kommt nur meinem Chef\pwindex{Sonnemann, Leopold 29.\,10.\,1831 Höchberg – 30.\,10.\,1909 Frankfurt am Main@\textsc{Sonnemann, Leopold} (29.\,10.\,1831 Höchberg – 30.\,10.\,1909 Frankfurt am Main), \emph{Journalist, Herausgeber}|pwv} zu gute, nicht mir. All’ dieſe
               Rieſen-Anſtrengung da drüben zählt nicht, und ich müßte \strikeout{eig} noch nach dem ermüdenden Arbeitstage Zeit und Kraft finden, um das
               Eigentliche zu arbeiten, das erſt zählen würde. Unter {\pb}dieſen Umſtänden muß man müde und muthlos
               werden.\pend
           
\pstart
           Grüß’ Dich Gott, mein lieber Arthur, und hab’ Dank für Deine Treue und Freundſchaft
               und für die{ }ſchönen \label{K_L02784-13v}\edtext{Tage von \textsc{Skodsborg\oindex{Skodsborg@\textbf{Skodsborg}|pw}}}{\lemma{\textnormal{\emph{Tage von Skodsborg}}}\Cendnote{\textnormal{Nachdem Goldmann\pwindex{Goldmann, Paul 31.\,1.\,1865 Breslau – 25.\,9.\,1935 Wien@\textsc{Goldmann, Paul} (31.\,1.\,1865 Breslau – 25.\,9.\,1935 Wien), \emph{Schriftsteller, Journalist}|pwk} von Schnitzler, Richard Beer-Hofmann\pwindex{Beer-Hofmann, Richard 11.\,7.\,1866 Wien – 26.\,9.\,1945 New York City@\textsc{Beer-Hofmann, Richard} (11.\,7.\,1866 Wien – 26.\,9.\,1945 New York City), \emph{Schriftsteller}|pwk}
                  und vermutlich auch Paula Beer-Hofmann\pwindex{Beer-Hofmann, Paula 25.\,2.\,1879 Wien – 30.\,10.\,1939 Zürich@\textsc{Beer-Hofmann, Paula} (25.\,2.\,1879 Wien – 30.\,10.\,1939 Zürich)|pwk} am
                     5. 8. 1896 in Kopenhagen\oindex{Kopenhagen@\textbf{Kopenhagen}, \emph{Hauptstadt}|pwk} abgeholt worden war (vgl. A. S.: \emph{Tagebuch}, 8. 8. 1896), dürfte er bis um den 20. 8. 1896 mit ihnen in Skodsborg\oindex{Skodsborg@\textbf{Skodsborg}|pwk} gewesen sein. Am 21. 8. 1896 war er jedenfalls, wenn auch
                  womöglich nur für einen Tag, wieder in Kopenhagen\oindex{Kopenhagen@\textbf{Kopenhagen}, \emph{Hauptstadt}|pwk}, zu Besuch bei Peter
                     und Betty Nansen\pwindex{Nansen, Peter 20.\,1.\,1861 Kopenhagen – 31.\,7.\,1918 Mariager@\textsc{Nansen, Peter} (20.\,1.\,1861 Kopenhagen – 31.\,7.\,1918 Mariager), \emph{Schriftsteller, Journalist, Verleger}|pwk}\pwindex{Nansen, Betty 19.\,3.\,1873 Kopenhagen – 15.\,3.\,1943 ebd.@\textsc{Nansen, Betty} (19.\,3.\,1873 Kopenhagen – 15.\,3.\,1943 ebd.), \emph{Theaterleiterin, Schauspielerin}|pwk}.}}}\label{K_L02784-13} (nicht wahr,{ }ſie waren{ }ſchön?)\pend
           
\pstart
           Empfiehl’ mich Deiner Frau Mutter\pwindex{Schnitzler, Louise 8.\,7.\,1840 Kőszeg – 9.\,9.\,1911 Wien@\textsc{Schnitzler, Louise} (8.\,7.\,1840 Kőszeg – 9.\,9.\,1911 Wien)|pwv}, deinem Bruder\pwindex{Schnitzler, Julius 13.\,7.\,1865 Wien – 29.\,6.\,1939 ebd.@\textsc{Schnitzler, Julius} (13.\,7.\,1865 Wien – 29.\,6.\,1939 ebd.), \emph{Chirurg}|pwv}, deiner Schwägerin\pwindex{Schnitzler, Helene 16.\,7.\,1871 Budapest – September 1941 Atlantischer Ozean@\textsc{Schnitzler, Helene} (16.\,7.\,1871 Budapest – September 1941 Atlantischer Ozean)|pwv}, Deiner Schweſter\pwindex{Hajek, Gisela 20.\,12.\,1867 Wien – 3.\,2.\,1953 Cambridge@\textsc{Hajek, Gisela} (20.\,12.\,1867 Wien – 3.\,2.\,1953 Cambridge)|pwv} und {\pb}Deinem Schwager\pwindex{Hajek, Markus 25.\,11.\,1861 Vršac – 4.\,4.\,1941 London@\textsc{Hajek, Markus} (25.\,11.\,1861 Vršac – 4.\,4.\,1941 London), \emph{Mediziner, Laryngologe}|pwv}.\pend
           
\pstart
           Empfiehl’ mich auch der unbekannten \label{K_L02784-14v}\edtext{Dame\pwindex{Reinhard, Marie 13.\,3.\,1871 Wien – 18.\,3.\,1899 ebd.@\textsc{Reinhard, Marie} (13.\,3.\,1871 Wien – 18.\,3.\,1899 ebd.), \emph{Gesangspädagogin}|pwv}}{\lemma{\textnormal{\emph{Dame}}}\Cendnote{\textnormal{Vgl. den Brief von Schnitzler an Marie
                     Reinhard\pwindex{Reinhard, Marie 13.\,3.\,1871 Wien – 18.\,3.\,1899 ebd.@\textsc{Reinhard, Marie} (13.\,3.\,1871 Wien – 18.\,3.\,1899 ebd.), \emph{Gesangspädagogin}|pwk}, 25. 7. 1896: »– Mit Altenberg\pwindex{Altenberg, Peter 9.\,3.\,1859 Wien – 8.\,1.\,1919 ebd.@\textsc{Altenberg, Peter} (9.\,3.\,1859 Wien – 8.\,1.\,1919 ebd.), \emph{Schriftsteller}!Wie ich es sehe@\strich\emph{Wie ich es sehe}|pwv}\pwindex{Altenberg, Peter 9.\,3.\,1859 Wien – 8.\,1.\,1919 ebd.@\textsc{Altenberg, Peter} (9.\,3.\,1859 Wien – 8.\,1.\,1919 ebd.), \emph{Schriftsteller}|pw} hast du ganz recht; freilich ist noch mehr zu sagen. Danke herzlich dass
                     du ihn an G.[oldmann]\pwindex{Goldmann, Paul 31.\,1.\,1865 Breslau – 25.\,9.\,1935 Wien@\textsc{Goldmann, Paul} (31.\,1.\,1865 Breslau – 25.\,9.\,1935 Wien), \emph{Schriftsteller, Journalist}|pw} geschickt; er hat
                     ihn schon zum Theil gelesen.« (\emph{Arthur Schnitzler an Marie Reinhard (1896)}.
                     Herausgegeben von Therese Nickl. In: \emph{Modern Austrian
                        Literature}, Jg. 10 (1977) H. 3/4, S. 42.)}}}\label{K_L02784-14},
               die mir den \textsc{Altenberg\pwindex{Altenberg, Peter 9.\,3.\,1859 Wien – 8.\,1.\,1919 ebd.@\textsc{Altenberg, Peter} (9.\,3.\,1859 Wien – 8.\,1.\,1919 ebd.), \emph{Schriftsteller}!Wie ich es sehe@\strich\emph{Wie ich es sehe}|pwv}\pwindex{Altenberg, Peter 9.\,3.\,1859 Wien – 8.\,1.\,1919 ebd.@\textsc{Altenberg, Peter} (9.\,3.\,1859 Wien – 8.\,1.\,1919 ebd.), \emph{Schriftsteller}|pw}} überſandt hat.\pend
           
\pstart
           In Treue {\\[\baselineskip]}Dein {\\[\baselineskip]}\spacefill\mbox{Paul Goldmann}\pend
           \leftskip=0em{}
\pstart
           \noindent{}Schreib’ mir bald nach \textsc{Paris\oindex{Paris@\textbf{Paris}, \emph{Hauptstadt}|pw}}.\pend
           
\pstart
           Wann gehſt Du \label{K_L02784-15v}\edtext{nach \textsc{Berlin\oindex{Berlin@\textbf{Berlin}, \emph{Hauptstadt}|pw}}}{\lemma{\textnormal{\emph{nach Berlin}}}\Cendnote{\textnormal{Schnitzler war bereits vom 22. 8. 1896 bis zum
                        26. 8. 1896 in
                        Berlin\oindex{Berlin@\textbf{Berlin}, \emph{Hauptstadt}|pwk}. Das nächste Mal war er dort
                     zwischen 26. 10. 1896 und 9. 11. 1896.}}}\label{K_L02784-15}?\pend
           \selectlanguage{ngerman}\endnumbering\briefempfaengerindex{Schnitzler, Arthur@\textsc{Schnitzler, Arthur}!zzzGoldmann, Paul@\emph{von Paul Goldmann}!1896-09-072@{7. 9. [1896]}|)be}\mylabel{L02784h}  \newcommand{\dateiname}{L02784}\newcommand{\titel}{Paul Goldmann an Arthur Schnitzler, 7. 9. [1896]}\newcommand{\editorInnen}{Martin Anton Müller und Laura Untner}%% latex-leseansicht-abspann.tex
%% Abspann für die Leseansicht.
%% Der Schalter \ifkorrekturansicht ist bereits durch den Vorspann gesetzt.

%% latex-abspann.tex
%% Gemeinsamer Abspann für Korrekturansicht und Leseansicht.
%% Setzt den Schalter \ifkorrekturansicht voraus (gesetzt in den
%% einbindenden Dateien latex-korrekturansicht-abspann.tex bzw.
%% latex-leseansicht-abspann.tex).
%% ---------------------------------------------------------------

\normalsize

% Das esempio-Environment wird nur in der Leseansicht benötigt
\ifkorrekturansicht\else
\newenvironment{esempio}[3]%
{
    \vspace{1.5ex}
    \rlap{\underline{#1}}
    \par
    \setlength{\parindent}{0cm}
    \nopagebreak
    \leftskip=#2cm
    \rightskip=#3cm
}
{
    \par
}
\fi

\doendnotes{C}
\bigskip
\vfill

\clearpage

\footnotesize

\ifkorrekturansicht
  \lohead{\textsc{register}}
\fi

% theindex-Environment neu definieren ohne reledmac
\makeatletter
\renewenvironment{theindex}{%
  \ifkorrekturansicht
    \section*{\indexname}%
  \else
    \subsubsection*{Index der erwähnten Entitäten}%
  \fi
  \setlength{\parindent}{0pt}%
  \setlength{\parskip}{0pt plus 0.3pt}%
  \let\item\@idxitem
}{%
  \ifkorrekturansicht\clearpage\fi
}
\makeatother

\IfFileExists{\jobname-pw.ind}{\input{\jobname-pw.ind}}{}

% Quellenangabe nur in der Leseansicht
\ifkorrekturansicht\else
% Fallback-Definitionen, falls die .tex-Datei \titel etc. nicht gesetzt hat
\providecommand{\titel}{}
\providecommand{\editorInnen}{}
\providecommand{\dateiname}{\jobname}

\vspace{3cm}

\vfill

\footnotesize
\textsc{Quelle}: \titel. Herausgegeben von {\editorInnen}. In: \emph{Arthur Schnitzler: Briefwechsel mit Autorinnen und Autoren}.
 Digitale Edition, https://schnitzler-briefe.acdh.oeaw.ac.at/{\dateiname}.html (Stand \today)
\fi

\end{document}


