%% latex-leseansicht-vorspann.tex
%% Vorspann für die Leseansicht.
%% Lädt die gemeinsame Datei latex-vorspann.tex mit nicht gesetztem Schalter.

\newif\ifkorrekturansicht
\korrekturansichtfalse

\input{../tex-inputs/latex-vorspann}


\section[Arthur Schnitzler an Georg Brandes, 27. 2. 1913]{L02114 Arthur Schnitzler an Georg Brandes, 27. 2. 1913}
\nopagebreak\mylabel{L02114v}
\rehead{ }\normalsize\beginnumbering\briefempfaengerindex{Brandes, Georg@\textsc{Brandes, Georg}!zzzSchnitzler, Arthur@\emph{von Arthur Schnitzler}!1913-02-271@{27. 2. 1913}|(be}
\toendnotes[C]{\smallbreak\pagebreak[2]}
\correspDesc{Versand  durch Arthur Schnitzler am 27. 2. 1913 in Wien
\newline{}Erhalt  durch Georg Brandes im Zeitraum [28. 2. 1913
                  – 4. 3. 1913?] in Kopenhagen}\toendnotes[C]{\smallbreak}
\Standort{Kopenhagen, Det Kongelige Bibliotek, Georg Brandes Arkiv, box 125.}
\physDesc{Brief, 2 Blätter, 3 Seiten, 2334 Zeichen (Seite 2 und 3 mit Schreibmaschine paginiert)
\newline{}Schreibmaschine
\newline{}Handschrift: schwarze Tinte (\noindent{}zwei Streichungen, Unterschrift)
\newline{}Ordnung: mit Bleistift von unbekannter Hand auf dem ersten Blatt
                                 nummeriert: »34.«; das zweite Blatt datiert mit: »27/2 13« }
\buchAbdrucke{\weitereDrucke{1) Georg Brandes, Arthur Schnitzler: \emph{Ein Briefwechsel}. Herausgegeben von Kurt Bergel. Bern: \emph{Francke} 1956, S. 106–107.} \weitereDrucke{2) Arthur Schnitzler: \emph{Briefe 1913–1931}. Herausgegeben von Peter Michael Braunwarth, Richard Miklin, Susanne Pertlik und Heinrich Schnitzler. Frankfurt am Main: \emph{S. Fischer} 1984, S. 12–13.} }\toendnotes[C]{\smallbreak}
\pstart
           {\pb}\textcolor{gray}{\textbf{Dr. Arthur Schnitzler}}{\\}\textcolor{gray}{\textbf{Wien XVIII. Sternwartestrasse 71\oindex{Wien@\textbf{Wien}!XVIII., Währing@\textbf{XVIII., Währing}!Sternwartestraße 71@\textbf{Sternwartestraße 71}, \emph{Wohngebäude}|pw}}}\pend
           
\pstart
           \raggedleft{}27. 2. 1913\pend
           
\pstart\center{}Lieber und verehrter Freund.\pend\vspace{0.5em}
\pstart
           Ihr \label{K_L02114-1v}\edtext{Bild\pwindex{\textcolor{red}{\textsuperscript{XXXX indx1}}!Georg Brandes]@\strich\emph{[Georg Brandes]}|pwv}}{\lemma{\textnormal{\emph{Bild}}}\Cendnote{\textnormal{Es dürfte sich um die Fotografie
                  handeln, die abgebildet ist in: Arthur Schnitzler: \emph{Sein
                        Leben · Sein Werk · Seine Zeit}. Herausgegeben von  Heinrich Schnitzler, Christian
                     Brandstätter und Reinhard Urbach. Frankfurt am Main: \emph{S. Fischer}\orgindex{S. Fischer Verlag@S. Fischer Verlag|pwk}{ }1981, S. 73.}}}\label{K_L02114-1} ist aus Paris\oindex{Paris@\textbf{Paris}, \emph{Hauptstadt}|pw} eingetroffen, es ist ausserordentlich gelungen, hat uns grosse Freude
               gemacht und wir sagen Ihnen herzlichen Dank dafür.\pend
           
\pstart
           Im »Merker\orgindex{Merker@Der Merker|pw}« habe ich eben Ihren höchst anregenden
                  \label{K_L02114-2v}\edtext{Artikel\pwindex{Brandes, Georg 4.\,2.\,1842 Kopenhagen – 19.\,2.\,1927 ebd.@\textsc{Brandes, Georg} (4.\,2.\,1842 Kopenhagen – 19.\,2.\,1927 ebd.)!Theater und Schauspiele in Deutschland@\strich\emph{Theater und Schauspiele in Deutschland}|pwv}}{\lemma{\textnormal{\emph{Artikel}}}\Cendnote{\textnormal{Georg Brandes\pwindex{Brandes, Georg 4.\,2.\,1842 Kopenhagen – 19.\,2.\,1927 ebd.@\textsc{Brandes, Georg} (4.\,2.\,1842 Kopenhagen – 19.\,2.\,1927 ebd.)|pwk}: \emph{Theater und Schauspiele in Deutschland}\pwindex{Brandes, Georg 4.\,2.\,1842 Kopenhagen – 19.\,2.\,1927 ebd.@\textsc{Brandes, Georg} (4.\,2.\,1842 Kopenhagen – 19.\,2.\,1927 ebd.)!Theater und Schauspiele in Deutschland@\strich\emph{Theater und Schauspiele in Deutschland}|pwk}. In: \emph{Der Merker}\pwindex{Merker. Österreichische Zeitschrift für Musik und Theater@\emph{Der Merker. Österreichische Zeitschrift für Musik und Theater}|pwk}, Jg. 4, H. 3,
                        1. Februar-Heft 1913, S. 95–99.}}}\label{K_L02114-2} über Theater in
                  Deutschland\oindex{Deutschland@\textbf{Deutschland}|pw} gelesen. Dass meine neue Komödie
                  »Professor Bernhardi\pwindex{Schnitzler, Arthur 15.\,5.\,1862 Wien – 21.\,10.\,1931 ebd.@\textsc{Schnitzler, Arthur} (15.\,5.\,1862 Wien – 21.\,10.\,1931 ebd.), \emph{Schriftsteller, Mediziner}!Professor Bernhardi. Komödie in fünf Akten@\strich\emph{Professor Bernhardi. Komödie in fünf Akten}|pw}« Sie so lebhaft
               interessiert hat, ist mir sehr lieb. Es ist über dieses Stück gar viel herumgeredet
               und – nicht immer bonafide – herumgeschwätzt\substVorne{}\textsuperscript{{ }worden}\substDazwischen{},\substHinten{} und auch Sie, verehrter Freund, sind wie speziell aus einer Ihrer
               Bemerkungen hervorgeht, über die Entstehungsgeschichte meines Stückes nicht ganz
               richtig informiert worden. Die Komödie\pwindex{Schnitzler, Arthur 15.\,5.\,1862 Wien – 21.\,10.\,1931 ebd.@\textsc{Schnitzler, Arthur} (15.\,5.\,1862 Wien – 21.\,10.\,1931 ebd.), \emph{Schriftsteller, Mediziner}!Professor Bernhardi. Komödie in fünf Akten@\strich\emph{Professor Bernhardi. Komödie in fünf Akten}|pwv} behandelt nicht eigentlich »ein Lebensschicksal, wie es mein Vater
               erfahren hat, der Inhalt ist vielmehr frei erfunden. Mein {\pb}Vater\pwindex{Schnitzler, Johann 10.\,4.\,1835 Nagykanizsa – 2.\,5.\,1893 Wien@\textsc{Schnitzler, Johann} (10.\,4.\,1835 Nagykanizsa – 2.\,5.\,1893 Wien), \emph{Laryngologe}|pwv} hat wohl seinerzeit,
               mit Freunden zusammen, ein Krankeninstitut\oindex{Wien@\textbf{Wien}!IX., Alsergrund@\textbf{IX., Alsergrund}!Allgemeine Poliklinik@\textbf{Allgemeine Poliklinik}, \emph{Krankenhaus}|pwv} in der Art des Elisabethinums gegründet, hat es gegen
               mancherlei Anfeindungen mit Aufgebot seiner ganzen Begabung und Tatkraft, natürlich
               nicht ohne die Mithilfe ausgezeichneter Arbeits- und Kampfgefährten, zu hoher Blüte
               gebracht und musste insbesondere gegen Schlus seines Lebens von mancher Seite Undank
               und Kränkung erfahren; – aber wenn sein Ausscheiden aus dem von ihm gegründeten
               Institut vielleicht auch Einem oder dem Andern nicht unangenehm gewesen wäre, er ist
               keineswegs »hinausintrigiert« worden, ja, \strikeout{es} ist
               sogar als Direktor des Instituts\oindex{Wien@\textbf{Wien}!IX., Alsergrund@\textbf{IX., Alsergrund}!Allgemeine Poliklinik@\textbf{Allgemeine Poliklinik}, \emph{Krankenhaus}|pwv}
               am 2. Mai 1893 gestorben. Uebrigens hat mein Titelheld, der »Professor Bernhardi\pwindex{Schnitzler, Arthur 15.\,5.\,1862 Wien – 21.\,10.\,1931 ebd.@\textsc{Schnitzler, Arthur} (15.\,5.\,1862 Wien – 21.\,10.\,1931 ebd.), \emph{Schriftsteller, Mediziner}!Professor Bernhardi. Komödie in fünf Akten@\strich\emph{Professor Bernhardi. Komödie in fünf Akten}|pwv}«, von
               meinem Vater\pwindex{Schnitzler, Johann 10.\,4.\,1835 Nagykanizsa – 2.\,5.\,1893 Wien@\textsc{Schnitzler, Johann} (10.\,4.\,1835 Nagykanizsa – 2.\,5.\,1893 Wien), \emph{Laryngologe}|pwv} nur wenige Züge
               entliehen,und auch die anderen Figuren meines Stückes sind, mit der freilich
               unerlässlichen Benützung von Wirklichkeitszügen so frei gestaltet, dass nur
               Kunstfremde, an denen es natürlich {\pb}niemals
               mangelt, hier von einem Schlüsselstück reden k\substVorne{}\textsuperscript{o}\substDazwischen{}ö\substHinten{}nnten. Meine Komödie\pwindex{Schnitzler, Arthur 15.\,5.\,1862 Wien – 21.\,10.\,1931 ebd.@\textsc{Schnitzler, Arthur} (15.\,5.\,1862 Wien – 21.\,10.\,1931 ebd.), \emph{Schriftsteller, Mediziner}!Professor Bernhardi. Komödie in fünf Akten@\strich\emph{Professor Bernhardi. Komödie in fünf Akten}|pwv}
               hat keine andere Wahrheit als die, dass sich die Handlung genau so, wie ich sie
                  erfunden\strikeout{,} habe, zugetragen haben könnte, – zum
               mindesten in Wien\oindex{Wien@\textbf{Wien}, \emph{Verwaltungsgebiet}|pw} zu Ende des vorigen
               Jahrhunderts.\pend
           
\pstart
           Ich sende Ihnen diese Zeilen nach Kopenhagen\oindex{Kopenhagen@\textbf{Kopenhagen}, \emph{Hauptstadt}|pw},
               freilich ohne zu wissen, ob Sie jetzt schon zurück sind. Man schickt Ihnen den Brief
               wohl nach, sei es nach Paris\oindex{Paris@\textbf{Paris}, \emph{Hauptstadt}|pw} oder anderswohin.
               Kommen Sie vielleicht über Wien\oindex{Wien@\textbf{Wien}, \emph{Verwaltungsgebiet}|pw}, wenn Sie
               heimreisen? Oder wo sonst werden Sie im Frühjahr sein? Es wäre schön einander einmal
               im Süden zu begegnen.\pend
           
\pstart
           Mit herzlichen Grüssen{\\[\baselineskip]}Ihr{\\[\baselineskip]}\spacefill\mbox{{[}hs.:{]} ArthurSchnitzler}\pend
           \leftskip=0em{}
\pstart
           \noindent{}{[}ms.:{]} Herrn Georg Brandes, Kopenhagen\oindex{Kopenhagen@\textbf{Kopenhagen}, \emph{Hauptstadt}|pw}.\pend
           \selectlanguage{ngerman}\endnumbering\briefempfaengerindex{Brandes, Georg@\textsc{Brandes, Georg}!zzzSchnitzler, Arthur@\emph{von Arthur Schnitzler}!1913-02-271@{27. 2. 1913}|)be}\mylabel{L02114h}  \newcommand{\dateiname}{L02114}\newcommand{\titel}{Arthur Schnitzler an Georg Brandes, 27. 2. 1913}\newcommand{\editorInnen}{Martin Anton Müller und Gerd-Hermann Susen}%% latex-leseansicht-abspann.tex
%% Abspann für die Leseansicht.
%% Der Schalter \ifkorrekturansicht ist bereits durch den Vorspann gesetzt.

%% latex-abspann.tex
%% Gemeinsamer Abspann für Korrekturansicht und Leseansicht.
%% Setzt den Schalter \ifkorrekturansicht voraus (gesetzt in den
%% einbindenden Dateien latex-korrekturansicht-abspann.tex bzw.
%% latex-leseansicht-abspann.tex).
%% ---------------------------------------------------------------

\normalsize

% Das esempio-Environment wird nur in der Leseansicht benötigt
\ifkorrekturansicht\else
\newenvironment{esempio}[3]%
{
    \vspace{1.5ex}
    \rlap{\underline{#1}}
    \par
    \setlength{\parindent}{0cm}
    \nopagebreak
    \leftskip=#2cm
    \rightskip=#3cm
}
{
    \par
}
\fi

\doendnotes{C}
\bigskip
\vfill

\clearpage

\footnotesize

\ifkorrekturansicht
  \lohead{\textsc{register}}
\fi

% theindex-Environment neu definieren ohne reledmac
\makeatletter
\renewenvironment{theindex}{%
  \ifkorrekturansicht
    \section*{\indexname}%
  \else
    \subsubsection*{Index der erwähnten Entitäten}%
  \fi
  \setlength{\parindent}{0pt}%
  \setlength{\parskip}{0pt plus 0.3pt}%
  \let\item\@idxitem
}{%
  \ifkorrekturansicht\clearpage\fi
}
\makeatother

\IfFileExists{\jobname-pw.ind}{\input{\jobname-pw.ind}}{}

% Quellenangabe nur in der Leseansicht
\ifkorrekturansicht\else
% Fallback-Definitionen, falls die .tex-Datei \titel etc. nicht gesetzt hat
\providecommand{\titel}{}
\providecommand{\editorInnen}{}
\providecommand{\dateiname}{\jobname}

\vspace{3cm}

\vfill

\footnotesize
\textsc{Quelle}: \titel. Herausgegeben von {\editorInnen}. In: \emph{Arthur Schnitzler: Briefwechsel mit Autorinnen und Autoren}.
 Digitale Edition, https://schnitzler-briefe.acdh.oeaw.ac.at/{\dateiname}.html (Stand \today)
\fi

\end{document}


