%% latex-leseansicht-vorspann.tex
%% Vorspann für die Leseansicht.
%% Lädt die gemeinsame Datei latex-vorspann.tex mit nicht gesetztem Schalter.

\newif\ifkorrekturansicht
\korrekturansichtfalse

\input{../tex-inputs/latex-vorspann}


         
         \newcommand{\erwaehntePersonen}{Personen:  ?? [Totgeborener Sohn von Arthur Schnitzler und Marie Reinhard], Richard Beer-Hofmann, Marie Reinhard, Leopold Sonnemann}
         \newcommand{\erwaehnteInstitutionen}{Institutionen: Frankfurter Zeitung}
         \newcommand{\erwaehnteOrte}{Orte: Paris, Wien, rue de la Bourse}
         \newcommand{\erwaehnteWerke}{Werke: Prestuplenie i nakazanie. Roman v 6 častjach s ňpilogom}
               \section[ Paul Goldmann an Arthur Schnitzler, 29. 9. {[}1897{]}]{ Paul Goldmann an Arthur Schnitzler, 29. 9. {[}1897{]}}\nopagebreak\mylabel{v}\rehead{ }\begin{ledgroupsized}[t]{13cm}\normalsize\beginnumbering \toendnotes[C]{\smallbreak\pagebreak[2]} \Standort{DLA, A:Schnitzler, HS.NZ85.1.3167.}
\physDesc{Brief, 2 Blätter, 6 Seiten
\newline{}Handschrift: blaue Tinte, deutsche Kurrent
\newline{}Schnitzler: mit Bleistift das Jahr »97« vermerkt }\toendnotes[C]{\smallbreak}\pstart
           \noindent{}{\pb}\textcolor{gray}{\textbf{\textbf{Frankfurter Zeitung\orgindex{Frankfurter Zeitung@Frankfurter Zeitung|pw}}}}\pend
           \pstart
           \textcolor{gray}{\textbf{(\begin{otherlanguage}{french}Gazette de Francfort\end{otherlanguage}\orgindex{Frankfurter Zeitung@Frankfurter Zeitung|pw}).}}\pend
           \pstart
           \textcolor{gray}{\textbf{\textbf{\begin{otherlanguage}{french}Fondateur M.\end{otherlanguage}{ }L. Sonnemann\pwindex{Sonnemann, Leopold 1831-10-29 – 1909-10-30@\textsc{Sonnemann, Leopold} (1831-10-29 – 1909-10-30), \emph{Journalist, Herausgeber}|pw}.}}}\pend
           \pstart
           \begin{otherlanguage}{french}\textcolor{gray}{\textbf{Journal politique, financier,}}\end{otherlanguage}\pend
           \pstart
           \begin{otherlanguage}{french}\textcolor{gray}{\textbf{commercial et littéraire.}}\end{otherlanguage}\pend
           \pstart
           \begin{otherlanguage}{french}\textcolor{gray}{\textbf{\textbf{Paraissant trois fois par jour.}}}\end{otherlanguage}\pend
           \pstart
           \begin{otherlanguage}{french}\textcolor{gray}{\textbf{\textbf{Bureau à Paris\oindex{Paris@\textbf{Paris}|pw}}}}\end{otherlanguage}\hfill \textsc{Paris\oindex{Paris@\textbf{Paris}|pw}}, 29. Sept.\pend
           \pstart
           \begin{otherlanguage}{french}\textcolor{gray}{\textbf{\textbf{10 \so{Rue de la Bourse}\oindex{rue de la Bourse@\textbf{rue de la Bourse}|pw}.}}}\end{otherlanguage}\pend
           \pstart\center{}Mein lieber Freund,\pend\pstart
           Dein Brief hat mich etwas ſpäter erreicht, da er recommandirt war. Geſtern{ }Abend habe ich ihn erſt in Händen gehabt. Deine herzzerreißende
               Schilderung hat mich tief erſchüttert. Armer, armer Freund! Und ich habe nicht einmal
               bei dir ſein und Dir mitfühlend die Hand drücken können!\pend
           \pstart
           Daß Du Dich mit Gedanken von \label{K_L02827-1v}\edtext{Schuld
               und Sühne}{\lemma{\textnormal{\emph{Schuld
               und Sühne}}}\Cendnote{\textnormal{siehe Paul Goldmann an Arthur Schnitzler, 25. 9. [1897]}}}\label{K_L02827-1h} quälen würdeſt, ahnte ich ſofort. Liebes Kind, denk’ nur einmal ruhig über
               dieſen tollen Unſinn nach. Es iſt unſer \strikeout{\textcolor{gray}{×}\-\textcolor{gray}{×}\-\textcolor{gray}{×}\-\textcolor{gray}{×}{ }\textcolor{gray}{×}\-\textcolor{gray}{×}\-\textcolor{gray}{×}\-\textcolor{gray}{×}\-\textcolor{gray}{×}\-\textcolor{gray}{×}{ }\textcolor{gray}{×}\-\textcolor{gray}{×}\-\textcolor{gray}{×}\-\textcolor{gray}{×}} verfluchtes Schreiber-{\pb}Metier, das uns die
               Manie gibt, überall Zuſammenhänge zu ſuchen. Wir leben ja davon, ich meine
               künſtleriſch, daß wir Beziehungen zwiſchen den Dingen herſtellen. Aber das iſt ja ein
               Schwindel, de\substVorne{}\textsuperscript{m}\substDazwischen{}n\substHinten{} wir dem Publicum vormachen. In Wirklichkeit gibt es keine Zuſammenhänge. Es
               iſt Alles nur ein plumpes und ungeordnetes Nebeneinander. Das wiſſen wir, wenn wir
               ehrlich ſind, beſſer als alle Anderen. Und nun ſollten wir uns gar ſelbſt damit
               betrügen? Ich bin ſonſt ein ruhig und klar denkender Mann. Und auf einmal ſoll ich
               mich zum Aber{\pb}glauben wenden, blos weil ich darin
               allerlei Vorwände finde\strikeout{\textcolor{gray}{n}}, um mich ſelbſt zu martern? Schuld und Sühne\pwindex{\textcolor{red}{\textsuperscript{XXXX1 indx}}!Prestuplenie i nakazanie. Roman v 6 castjach s ňpilogom1866@\strich\emph{Prestuplenie i nakazanie. Roman v 6 častjach s ňpilogom} {[}1866{]}|pwv} ſind literariſche \textsc{Pointen}, und
               ich verſichere Dich, das Schickſal gibt ſich nicht damit ab, Dramen zu ſchreiben.\pend
           \pstart
           Auch leugne ich aufs Entſchiedenſte, bei ſtrengſter Beurtheilung, jede Spur von
               Schuld. Du haſt zärtlich und liebevoll Alles vorbereitet für den Eintritt des Kind\pwindex{?? [Totgeborener Sohn von Arthur Schnitzler und Marie Reinhard] 1897-09-24 – 1897-09-24@\textsc{?? [Totgeborener Sohn von Arthur Schnitzler und Marie Reinhard]} (1897-09-24 – 1897-09-24)|pwv}es in die Welt. Wie ſoll
               man denn noch mehr ein Weſen lieben, das noch nicht exiſtirt? Und wo ſteht
               geſchrieben, daß Jemand, der ein Kind erwartet, aufhören {\pb}ſoll, ſein eigenes Leben zu leben? Wenn die Liebe
               der Väter auf Leben oder Nichtleben der Kinder Einfluß hätte, wie kommt es dann, daß
               zahlreiche Kinder in der Welt herumlaufen, die nicht einmal wiſſen, wer ihr Vater
                  war? {\dotsfive}\pend
           \pstart
           Daß Einem in Augenblicken des Schmerzes Manches klar wird, beſtreite ich auch. Nur in
               der Ruhe ſieht man klar, der Affekt täuſcht, und der Schmerz lügt ebenſo wie die
                  Freude{\dotsfour}\pend
           \pstart
           Wäre ich nicht ein ſo armſeliger Sklave, ſo wäre ich ſofort nach Empfang Deines
               Briefes {\pb}nach Wien\oindex{Wien@\textbf{Wien}|pw} gekommen. Inzwiſchen biſt Du ja übrigens ſicher ruhig und gefaßt
               geworden. Es iſt eine traurige Geſchichte; aber wenn man ſichs genau überlegt, wird
               doch alles Weſentliche unberührt ſein, wenn einmal der Sturm vorüber iſt. Eine
               Hoffnung hat ſich nicht erfüllt. Man wiſcht ſich die Thränen ab und hofft aufs Neue{\dotsfour}\pend
           \pstart
           Bitte, ſchreib’ mir bald, wenn auch nur drei Worte. Wiſſen möchte ich auch, ob
                  \label{K_L02827-2v}\edtext{\textsc{Richard\pwindex{Beer-Hofmann, Richard 1866-07-11 – 1945-09-26@\textsc{Beer-Hofmann, Richard} (1866-07-11 – 1945-09-26), \emph{Schriftsteller}|pw}} informirt}{\lemma{\textnormal{\emph{Richard informirt}}}\Cendnote{\textnormal{Richard Beer-Hofmann\pwindex{Beer-Hofmann, Richard 1866-07-11 – 1945-09-26@\textsc{Beer-Hofmann, Richard} (1866-07-11 – 1945-09-26), \emph{Schriftsteller}|pwk} wurde am 25. 9. 1897 von Schnitzler\pwindex{Schnitzler, Arthur 15.05.1862 – 21.10.1931@\textsc{Schnitzler, Arthur} (15.05.1862 – 21.10.1931), \emph{Schriftsteller, Mediziner}|pwk} über die Totgeburt\pwindex{?? [Totgeborener Sohn von Arthur Schnitzler und Marie Reinhard] 1897-09-24 – 1897-09-24@\textsc{?? [Totgeborener Sohn von Arthur Schnitzler und Marie Reinhard]} (1897-09-24 – 1897-09-24)|pwkv} informiert.}}}\label{K_L02827-2h} iſt.\pend
           \pstart
           Grüße Deine Freundin\pwindex{Reinhard, Marie 1871-03-13 – 1899-03-18@\textsc{Reinhard, Marie} (1871-03-13 – 1899-03-18), \emph{Gesangspädagogin}|pwv}, {\pb}die liebe, prächtige Frau\pwindex{Reinhard, Marie 1871-03-13 – 1899-03-18@\textsc{Reinhard, Marie} (1871-03-13 – 1899-03-18), \emph{Gesangspädagogin}|pwv}, die ſo \label{K_L02827-3v}\edtext{ſacht zu dulden weiß}{\lemma{\textnormal{\emph{ſacht zu dulden weiß}}}\Cendnote{\textnormal{siehe Paul Goldmann an Arthur Schnitzler, 25. 9. [1897]}}}\label{K_L02827-3h}, und ſei Du ſelbſt von ganzem Herzen gegrüßt.\pend
           \pstart
           In Treue {\\[\baselineskip]}Dein {\\[\baselineskip]}\spacefill\mbox{Paul Goldmann}\pend
           \leftskip=0em{}\pstart
           \noindent{}Ich werde natürlich die Idee nicht los, daß das Alles ſo gekommen iſt, weil es\pwindex{?? [Totgeborener Sohn von Arthur Schnitzler und Marie Reinhard] 1897-09-24 – 1897-09-24@\textsc{?? [Totgeborener Sohn von Arthur Schnitzler und Marie Reinhard]} (1897-09-24 – 1897-09-24)|pwv} meinen Namen tragen
                  ſollte.\pend
           
         
         \endnumbering\mylabel{h}\end{ledgroupsized}  \newcommand{\dateiname}{L02827}\newcommand{\titel}{Paul Goldmann an Arthur Schnitzler, 29. 9. [1897]}\newcommand{\editorInnen}{Martin Anton Müller und Laura Untner}%% latex-leseansicht-abspann.tex
%% Abspann für die Leseansicht.
%% Der Schalter \ifkorrekturansicht ist bereits durch den Vorspann gesetzt.

%% latex-abspann.tex
%% Gemeinsamer Abspann für Korrekturansicht und Leseansicht.
%% Setzt den Schalter \ifkorrekturansicht voraus (gesetzt in den
%% einbindenden Dateien latex-korrekturansicht-abspann.tex bzw.
%% latex-leseansicht-abspann.tex).
%% ---------------------------------------------------------------

\normalsize

% Das esempio-Environment wird nur in der Leseansicht benötigt
\ifkorrekturansicht\else
\newenvironment{esempio}[3]%
{
    \vspace{1.5ex}
    \rlap{\underline{#1}}
    \par
    \setlength{\parindent}{0cm}
    \nopagebreak
    \leftskip=#2cm
    \rightskip=#3cm
}
{
    \par
}
\fi

\doendnotes{C}
\bigskip
\vfill

\clearpage

\footnotesize

\ifkorrekturansicht
  \lohead{\textsc{register}}
\fi

% theindex-Environment neu definieren ohne reledmac
\makeatletter
\renewenvironment{theindex}{%
  \ifkorrekturansicht
    \section*{\indexname}%
  \else
    \subsubsection*{Index der erwähnten Entitäten}%
  \fi
  \setlength{\parindent}{0pt}%
  \setlength{\parskip}{0pt plus 0.3pt}%
  \let\item\@idxitem
}{%
  \ifkorrekturansicht\clearpage\fi
}
\makeatother

\IfFileExists{\jobname-pw.ind}{\input{\jobname-pw.ind}}{}

% Quellenangabe nur in der Leseansicht
\ifkorrekturansicht\else
% Fallback-Definitionen, falls die .tex-Datei \titel etc. nicht gesetzt hat
\providecommand{\titel}{}
\providecommand{\editorInnen}{}
\providecommand{\dateiname}{\jobname}

\vspace{3cm}

\vfill

\footnotesize
\textsc{Quelle}: \titel. Herausgegeben von {\editorInnen}. In: \emph{Arthur Schnitzler: Briefwechsel mit Autorinnen und Autoren}.
 Digitale Edition, https://schnitzler-briefe.acdh.oeaw.ac.at/{\dateiname}.html (Stand \today)
\fi

\end{document}


      