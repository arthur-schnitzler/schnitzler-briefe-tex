%% latex-korrekturansicht-vorspann.tex
%% Vorspann für die Korrekturansicht.
%% Lädt die gemeinsame Datei latex-vorspann.tex mit gesetztem Schalter.

\newif\ifkorrekturansicht
\korrekturansichttrue

\input{../tex-inputs/latex-vorspann}


\section[Paul Goldmann an Arthur Schnitzler, 27. 9. {[}1893{]}]{L02717 Paul Goldmann an Arthur Schnitzler, 27. 9. {[}1893{]}}
\nopagebreak\mylabel{L02717v}
\rehead{ }\normalsize\beginnumbering\briefempfaengerindex{Schnitzler, Arthur@\textsc{Schnitzler, Arthur}!zzzGoldmann, Paul@\emph{von Paul Goldmann}!1893-09-271@{27. 9. {[}1893{]}}|(be}
\toendnotes[C]{\smallbreak\pagebreak[2]}\Standort{DLA, A:Schnitzler, HS.NZ85.1.3163.}
\physDesc{Brief, 1 Blatt, 4 Seiten, 2724 Zeichen
\newline{}Handschrift: schwarze Tinte, deutsche Kurrent
\newline{}Schnitzler: 1) mit Bleistift das Jahr »93« vermerkt  2) mit rotem Buntstift eine Unterstreichung}\toendnotes[C]{\smallbreak}
\pstart
           {\pb}\textcolor{gray}{\textbf{\textbf{Frankfurter Zeitung\orgindex{Frankfurter Zeitung@Frankfurter Zeitung|pw}.}}}\pend
           
\pstart
           \textcolor{gray}{\textbf{\textbf{(\begin{otherlanguage}{french}Gazette de Francfort\end{otherlanguage}\orgindex{Frankfurter Zeitung@Frankfurter Zeitung|pw}.)}}}\pend
           
\pstart
           \textcolor{gray}{\textbf{\begin{otherlanguage}{french}Directeur\end{otherlanguage}{ }\textbf{M. L. Sonnemann\pwindex{Sonnemann, Leopold 1831-10-29 – 1909-10-30@\textsc{Sonnemann, Leopold} (1831-10-29 – 1909-10-30), \emph{Journalist/Journalistin, Herausgeber/Herausgeberin}|pw}.}}}\hfill \textsc{Paris\oindex{Paris@\textbf{Paris}, \emph{P.PPLC}|pw}}, 27. \strikeout{Juni.}
                        September.\pend
           
\pstart
           \begin{otherlanguage}{french}\textcolor{gray}{\textbf{Journal politique, financier,}}\end{otherlanguage}\pend
           
\pstart
           \begin{otherlanguage}{french}\textcolor{gray}{\textbf{commercial et litteraire.}}\end{otherlanguage}\pend
           
\pstart
           \begin{otherlanguage}{french}\textcolor{gray}{\textbf{\textbf{Paraissant trois fois par jour}}}\end{otherlanguage}\pend
           
\pstart
           \begin{otherlanguage}{french}\textcolor{gray}{\textbf{\textbf{Bureaux à Paris\oindex{Paris@\textbf{Paris}, \emph{P.PPLC}|pw}:}}}\end{otherlanguage}\pend
           
\pstart
           \begin{otherlanguage}{french}\textcolor{gray}{\textbf{\textbf{rue Richelieu 75\oindex{rue Richelieu@\textbf{rue Richelieu}, \emph{Straße (K.STR)}|pw}.}}}\end{otherlanguage}\pend
           
\pstart\center{}Mein lieber Arthur!\pend\vspace{0.5em}
\pstart
           Ich danke Dir für Deinen lieben Brief und für die Sendung Deiner \label{K_L02717-1v}\edtext{Bücher\pwindex{Maerchen. Schauspiel in drei Aufzuegen@\emph{Das Märchen. Schauspiel in drei Aufzügen}|pwuv}\pwindex{Anatol@\emph{Anatol}|pwuv}}{\lemma{\textnormal{\emph{Bücher}}}\Cendnote{\textnormal{Es könnte sich um Exemplare von \emph{Das Märchen}\pwindex{Maerchen. Schauspiel in drei Aufzuegen@\emph{Das Märchen. Schauspiel in drei Aufzügen}|pwk} und \emph{Anatol}\pwindex{Anatol@\emph{Anatol}|pwk} handeln, die von Goldmann\pwindex{Goldmann, Paul 31.01.1865 – 25.09.1935@\textsc{Goldmann, Paul} (31.01.1865 – 25.09.1935), \emph{Schriftsteller/Schriftstellerin, Journalist/Journalistin}|pwk} in Paris\oindex{Paris@\textbf{Paris}, \emph{P.PPLC}|pwk} bei Theatern
                  eingereicht werden sollten (vgl. Paul Goldmann an Arthur Schnitzler, 4. 11. [1893]).}}}\label{K_L02717-1}. Und noch beſonders danke ich Dir für die paar frohen
               Stunden in \textsc{Salzburg\oindex{Salzburg@\textbf{Salzburg}, \emph{A.ADM2}|pw}}. Mir hat das eine Zeit lang die Empfindung der Heimatloſigkeit genommen. Damit
               haſt Du eine gute That für einen \strikeout{\textcolor{gray}{ar}} armen Verlaſſenen gethan, und dieſes Bewußtſein ſoll Dich Deinen \label{K_L02717-2v}\edtext{Katarrh}{\lemma{\textnormal{\emph{Katarrh}}}\Cendnote{\textnormal{Entzündung von Schleimhäuten der Atmungsorgane}}}\label{K_L02717-2}
               leichter tragen laſſen, dem ich übrigens von Herzen ein baldiges Ende wünſche.\pend
           
\pstart
           In \textsc{Muenchen\oindex{Muenchen@\textbf{München}, \emph{P.PPLA}|pw}} gab es noch ein paar ſchöne Augenblicke. Es iſt eine liebe Stadt\oindex{Muenchen@\textbf{München}, \emph{P.PPLA}|pwv}, in {\pb}manchen Beziehungen ein \textsc{Wien}\oindex{Wien@\textbf{Wien}, \emph{A.ADM2}|pw}, in manchen ſogar ein beſſeres \textsc{Wien}\oindex{Wien@\textbf{Wien}, \emph{A.ADM2}|pw}. Die Hauptzeit habe ich in der \textsc{Pinakothek}\oindex{Alte Pinakothek@\textbf{Alte Pinakothek}, \emph{Museum (K.MUS)}|pwv} verbracht und mir die Augen mit Schönheit vollgeſogen – Proviant für eine
               lange, öde Reiſe. \strikeout{M\textcolor{gray}{i}t} Von meinem
                  Onkel\pwindex{Mamroth, Fedor 21.02.1851 – 25.06.1907@\textsc{Mamroth, Fedor} (21.02.1851 – 25.06.1907), \emph{Journalist/Journalistin, Kritiker/Kritikerin}|pwv} bin ich kühler
               geſchieden als je. Auch von dieſem Manne\pwindex{Mamroth, Fedor 21.02.1851 – 25.06.1907@\textsc{Mamroth, Fedor} (21.02.1851 – 25.06.1907), \emph{Journalist/Journalistin, Kritiker/Kritikerin}|pwv} ſcheint mich das Leben trennen zu wollen. Wir ſind plötzlich gereizt
               gegen einander, ſo mü\textcolor{gray}{ſ}ſen wir das zu verbergen trachten. Im
               Grunde, glaube ich, grollt wohl Einer dem Andern, daß er ihm nicht helfen kann.
               Gleiche Unproductivität, gleiche negative Schärfe, gleiche Willenloſigkeit und
               Unſtätheit auf beiden Seiten. Dieſe Erkenntniß hat mir das Herz erfrieren gemacht,
               und ſo bin ich aus \textsc{Muenchen\oindex{Muenchen@\textbf{München}, \emph{P.PPLA}|pw}} herausgefahren. Troſtloſe, endloſe Rückreiſe. {\pb}Und nun bin ich hier, und Bergeslaſten liegen mir wieder auf der Bruſt. Ich habe
               gerade heut{ }Morgen wieder eine Stunde gehabt, wo ich meinte, ich müſſe ruhig die
               Hände in den Schoß legen und auf dem Seſſel ſitzen bleiben, weil ich nicht mehr
               weiter kann. Die alte Thätigkeit widert mich an, die Leute und die Verhältniſſe hier
               ſind mir verhaßt, von allen Seiten ſtellen ſich wieder die Unmöglichkeiten in den
               Weg. Vor Allem \strikeout{h\textcolor{gray}{a}} aber habe ich \strikeout{das} die klare Erkenntniß, daß
               ich im Begriff bin, mein Leben zu verfehlen. Ich ſehe alle Fehler, ich ſehe die
               deutliche \strikeout{\textcolor{gray}{W}} Wendung meines Weſens in der falſchen Richtung, ich {\pb}habe aber nicht die Kraft, zurückzureißen. Ich frage
               mich: Was ich eigentlich auf der Welt ſoll? und ich weiß es nicht. Mir ſällt ein, daß
               ich bald dreißig bin und daß ich nichts, nichts, nichts noch geſchaffen habe; und ich
               weiß ganz genau, daß das Werk auch in Zukunft nicht kommen wird. Und ſonſt noch
               tauſenderlei. Oh pfui! {\dotsfive}\pend
           
\pstart
           Nun wollen wir ſehen, was ſich in \textsc{Paris\oindex{Paris@\textbf{Paris}, \emph{P.PPLC}|pw}} für Dich thun läßt. In \textsc{Muenchen\oindex{Muenchen@\textbf{München}, \emph{P.PPLA}|pw}} war vorläufig nichts zu machen; aber ich habe eine \label{K_L02717-3v}\edtext{Verſprechung}{\lemma{\textnormal{\emph{Verſprechung}}}\Cendnote{\textnormal{nicht rekonstruierbar}}}\label{K_L02717-3}. Nochmals: Vergiß’ nicht, mich \uuline{ſofort} zu \label{K_L02717-4v}\edtext{benachrichtigen, wenn dein Stück\pwindex{Maerchen. Schauspiel in drei Aufzuegen@\emph{Das Märchen. Schauspiel in drei Aufzügen}|pwv} zur Aufführung angeſetzt}{\lemma{\textnormal{\emph{benachrichtigen, … angeſetzt}}}\Cendnote{\textnormal{Siehe Paul Goldmann an Arthur Schnitzler, 4. 11. [1893].
               }}}\label{K_L02717-4} iſt. Sei von Herzen begrüßt, Du und die lieben Freunde!\pend
           
\pstart
           Dein {\\[\baselineskip]}\spacefill\mbox{Paul Goldmann}\pend
           \leftskip=0em{}
\pstart
           \noindent{}Zu leſen: \textsc{Barbey d’Aurevilly\pwindex{Barbey DAurevilly, Jules-Amedee 02.11.1808 – 23.04.1889@\textsc{Barbey d’Aurevilly, Jules-Amédée} (02.11.1808 – 23.04.1889), \emph{Schriftsteller/Schriftstellerin}|pw}}: \textsc{Les Diaboliques\pwindex{Diaboliques@\emph{Les Diaboliques}|pw}}.\pend
           
\pstart
           Wichtig: Denk’ an die Empfehlung, bitte. Ich bin ſo einſam hier!\pend
           
\pstart
           \label{T_L02717-1v}\edtext{Schreibe mir sehr bald!}{\lemma{\textnormal{\emph{Schreibe mir sehr bald!}}}\Cendnote{\textnormal{seitlich am linken Rand entlang des
                     Mittelfalzes}}}\label{T_L02717-1}\pend
           
\pstart
           {\pb}\label{T_L02717-2v}\edtext{\label{K_L02717-5v}\edtext{\textsc{Mandel\pwindex{Mandl, Richard 1859-05-09 – 1918-03-31@\textsc{Mandl, Richard} (1859-05-09 – 1918-03-31), \emph{Komponist/Komponistin}|pwv}}}{\lemma{\textnormal{\emph{Mandel}}}\Cendnote{\textnormal{Richard Mandl\pwindex{Mandl, Richard 1859-05-09 – 1918-03-31@\textsc{Mandl, Richard} (1859-05-09 – 1918-03-31), \emph{Komponist/Komponistin}|pwk} (nicht »Mandel\pwindex{Mandl, Richard 1859-05-09 – 1918-03-31@\textsc{Mandl, Richard} (1859-05-09 – 1918-03-31), \emph{Komponist/Komponistin}|pwv}«) war ein Komponist\pwindex{Mandl, Richard 1859-05-09 – 1918-03-31@\textsc{Mandl, Richard} (1859-05-09 – 1918-03-31), \emph{Komponist/Komponistin}|pwkv}, der zwischen 1883 und 1900 in Paris\oindex{Paris@\textbf{Paris}, \emph{P.PPLC}|pwk} lebte.
                     Am 26. 9. 1893
                     fand bei Schnitzler zu Hause eine private
                     Lesung von Werken\pwindex{Anatols Hochzeitsmorgen@\emph{Anatols Hochzeitsmorgen}|pwkv}\pwindex{Abschiedssouper@\emph{Abschiedssouper}|pwkv}\pwindex{Agonie@\emph{Agonie}|pwkv}\pwindex{Gedichte@\emph{Gedichte}|pwkv}{ }Schnitzlers statt, Mandl\pwindex{Mandl, Richard 1859-05-09 – 1918-03-31@\textsc{Mandl, Richard} (1859-05-09 – 1918-03-31), \emph{Komponist/Komponistin}|pwk} spielte eigene Kompositionen. Von diesem
                     anstehenden Treffen dürfte Schnitzler in
                     seinem letzten Brief gesprochen und dabei die Frage gestellt haben, ob
                        Goldmann\pwindex{Goldmann, Paul 31.01.1865 – 25.09.1935@\textsc{Goldmann, Paul} (31.01.1865 – 25.09.1935), \emph{Schriftsteller/Schriftstellerin, Journalist/Journalistin}|pwk} ihn kenne.}}}\label{K_L02717-5} kenne ich
                  nicht ebenſo wenig wie den \label{K_L02717-6v}\edtext{deutschen Quartettverein\orgindex{Deutscher Quartettverein in Paris@Deutscher Quartettverein in Paris|pw}}{\lemma{\textnormal{\emph{deutschen Quartettverein}}}\Cendnote{\textnormal{Der \emph{Deutsche Quartettverein in Paris}\orgindex{Deutscher Quartettverein in Paris@Deutscher Quartettverein in Paris|pwk}, von vier Musikern um
                        1850 gegründet, widmete sich ursprünglich dem Werk von Ludwig van Beethoven\pwindex{Beethoven, Ludwig van 17.12.1770 – 26.03.1827@\textsc{Beethoven, Ludwig van} (17.12.1770 – 26.03.1827), \emph{Komponist/Komponistin}|pwk}. }}}\label{K_L02717-6}. Er
                  verwechselt mich wahrſcheinlich mit meinem \label{K_L02717-7v}\edtext{Vorgänger\pwindex{Muehling, Karl 1858-09-19 – nach 1916@\textsc{Mühling, Karl} (1858-09-19 – nach 1916), \emph{Journalist/Journalistin}|pwuv}}{\lemma{\textnormal{\emph{Vorgänger}}}\Cendnote{\textnormal{Der letzte nachweisbare Paris\oindex{Paris@\textbf{Paris}, \emph{P.PPLC}|pwk}er Korrespondent\pwindex{Muehling, Karl 1858-09-19 – nach 1916@\textsc{Mühling, Karl} (1858-09-19 – nach 1916), \emph{Journalist/Journalistin}|pwkv} der \emph{Frankfurter Zeitung}\orgindex{Frankfurter Zeitung@Frankfurter Zeitung|pwk} vor Goldmann\pwindex{Goldmann, Paul 31.01.1865 – 25.09.1935@\textsc{Goldmann, Paul} (31.01.1865 – 25.09.1935), \emph{Schriftsteller/Schriftstellerin, Journalist/Journalistin}|pwk} war Karl Mühling\pwindex{Muehling, Karl 1858-09-19 – nach 1916@\textsc{Mühling, Karl} (1858-09-19 – nach 1916), \emph{Journalist/Journalistin}|pwk}
                     zwischen 1887 und 1889. Es
                     ist nicht sicher, ob Goldmann\pwindex{Goldmann, Paul 31.01.1865 – 25.09.1935@\textsc{Goldmann, Paul} (31.01.1865 – 25.09.1935), \emph{Schriftsteller/Schriftstellerin, Journalist/Journalistin}|pwk}{ }Mühling\pwindex{Muehling, Karl 1858-09-19 – nach 1916@\textsc{Mühling, Karl} (1858-09-19 – nach 1916), \emph{Journalist/Journalistin}|pwk} meinte oder ob es zwischen den
                     beiden einen weiteren Korrespondenten gegeben hatte.}}}\label{K_L02717-7}.}{\lemma{\textnormal{\emph{Mandel … Vorgänger.}}}\Cendnote{\textnormal{kopfüber am oberen Rand der ersten Seite}}}\label{T_L02717-2}\pend
           \selectlanguage{ngerman}\endnumbering\briefempfaengerindex{Schnitzler, Arthur@\textsc{Schnitzler, Arthur}!zzzGoldmann, Paul@\emph{von Paul Goldmann}!1893-09-271@{27. 9. {[}1893{]}}|)be}\mylabel{L02717h}  \normalsize

\doendnotes{C}
\bigskip
\vfill

\clearpage

\footnotesize

\lohead{\textsc{register}}

% Definiere theindex-Environment komplett neu ohne reledmac
\makeatletter
\renewenvironment{theindex}{%
  \section*{\indexname}%
  \setlength{\parindent}{0pt}%
  \setlength{\parskip}{0pt plus 0.3pt}%
  \let\item\@idxitem
}{%
  \clearpage
}
\makeatother

\IfFileExists{\jobname-pw.ind}{\input{\jobname-pw.ind}}{}

\end{document}

      