\input{../tex-inputs/latex-pdf-vorspann}
\begin{center}
            \textcolor{red}{ENTWURF. ENTZIFFERUNG NOCH NICHT KORREKTURGELESEN}
                      \end{center}
            
               \section[Hugo von Hofmannsthal an Arthur Schnitzler, 9. 11. {[}1909{]}]{ Hugo von Hofmannsthal an Arthur Schnitzler, 9. 11. {[}1909{]}}\nopagebreak\mylabel{v}\rehead{ }\begin{ledgroupsized}[t]{13cm}\normalsize\beginnumbering\briefempfaengerindex{Schnitzler, Arthur@\textsc{Schnitzler, Arthur}!zzzHofmannsthal, Hugo von@\emph{von Hugo von Hofmannsthal}!1909-11-091@{9. 11. {[}1909{]}}|(be} \toendnotes[C]{\smallbreak\pagebreak[2]} \Standort{CUL, Schnitzler, B 43.}
\physDesc{Postkarte
\newline{}Handschrift: schwarze Tinte, deutsche Kurrent\newline{}Versand: Stempel: »\nobreak{}\oindex{Rodaun@\textbf{Rodaun}|pwk}Rodaun, 10 1\textcolor{gray}{1}{ }{[}09{]}\nobreak{}«.  
\newline{}Schnitzler: mit Bleistift die Jahreszahl ergänzt: »909« und beschriftet »Hugo« \newline{}Ordnung: 1) mit Bleistift von unbekannter Hand nummeriert: »\strikeout{305}« 2) mit Bleistift von unbekannter Hand nummeriert: »312«}\buchAbdrucke{\weitereDrucke{Hugo von Hofmannsthal, Arthur Schnitzler: \emph{Briefwechsel}. Hg. Therese Nickl und Heinrich Schnitzler. Frankfurt am Main: \emph{S. Fischer} 1964, S. 247.} }\toendnotes[C]{\smallbreak}\pstart{}{\pb}\textsc{Herrn D\textsuperscript{r} Arthur Schnitzler}\pend{}\pstart{}\textsc{Wien}\oindex{Wien@\textbf{Wien}|pw}\pend{}\pstart{}\textsc{XVIII Spöttelgasse 7\oindex{Edmund-Weiss-Gasse@\textbf{Edmund-Weiß-Gasse}|pw}}\pend{}{\bigskip}\pstart
           \noindent{}\raggedleft{}Rodaun\oindex{Rodaun@\textbf{Rodaun}|pw}{ }9. XI.\pend
           \pstart
           \noindent{}lieber, gehen Sie eventuell wegen Brahm\pwindex{Brahm, Otto 05.02.1856 – 28.11.1912@\textsc{Brahm, Otto} (05.02.1856 – 28.11.1912), \emph{Theaterleiter, Regisseur}|pw} auf den Se{\geminationm}ering\oindex{Semmering@\textbf{Semmering}|pw}? Und \label{K_L01886_1v}\edtext{wann}{\lemma{\textnormal{\emph{wann}}}\Cendnote{\textnormal{Er war von 11. 11. 1909 drei Tage lang am Semmering\oindex{Semmering@\textbf{Semmering}|pwk}. Brahm\pwindex{Brahm, Otto 05.02.1856 – 28.11.1912@\textsc{Brahm, Otto} (05.02.1856 – 28.11.1912), \emph{Theaterleiter, Regisseur}|pwk} reiste am
                  selben Tag an und die beiden unternahmen viele gemeinsame Spaziergänge. Hofmannsthal\pwindex{Hofmannsthal, Hugo von 01.02.1874 – 15.07.1929@\textsc{Hofmannsthal, Hugo von} (01.02.1874 – 15.07.1929), \emph{Schriftsteller}|pwk} reiste nicht.}}}\label{K_L01886_1h}? Bitte um
               ein Wort!\pend
           \pstart \spacefill\mbox{Hugo}\pend{}\endnumbering\briefempfaengerindex{Schnitzler, Arthur@\textsc{Schnitzler, Arthur}!zzzHofmannsthal, Hugo von@\emph{von Hugo von Hofmannsthal}!1909-11-091@{9. 11. {[}1909{]}}|)be}\mylabel{h}\end{ledgroupsized}  \newcommand{\dateiname}{L01886}\newcommand{\titel}{Hugo von Hofmannsthal an Arthur Schnitzler, 9. 11. [1909]}\newcommand{\editorInnen}{Martin Anton Müller und Gerd-Hermann Susen}\input{../tex-inputs/latex-pdf-abspann}
      