%% latex-leseansicht-vorspann.tex
%% Vorspann für die Leseansicht.
%% Lädt die gemeinsame Datei latex-vorspann.tex mit nicht gesetztem Schalter.

\newif\ifkorrekturansicht
\korrekturansichtfalse

\input{../tex-inputs/latex-vorspann}


\section[ Paul Goldmann an Arthur Schnitzler, {[}4. 5. 1897{]}]{L02811 Paul Goldmann an Arthur Schnitzler,  [4. 5. 1897]}
\nopagebreak\mylabel{L02811v}
\rehead{ }\normalsize\beginnumbering\briefempfaengerindex{Schnitzler, Arthur@\textsc{Schnitzler, Arthur}!zzzGoldmann, Paul@\emph{von Paul Goldmann}!1897-05-041@{{[}4. 5. 1897{]}}|(be}
\toendnotes[C]{\smallbreak\pagebreak[2]}
\correspDesc{Versand  durch Paul Goldmann am [4. 5. 1897] in Paris
\newline{}Erhalt  durch Arthur Schnitzler im Zeitraum [4. 5. 1897
                  – 8. 5. 1897?] in Paris}\toendnotes[C]{\smallbreak}
\Standort{DLA, A:Schnitzler, HS.NZ85.1.3167.}
\physDesc{Brief, 1 Blatt, 1 Seite, 218 Zeichen
\newline{}Handschrift: schwarze Tinte, deutsche Kurrent
\newline{}Schnitzler: mit Bleistift das Datum »4/5 97« vermerkt }\toendnotes[C]{\smallbreak}
\pstart\center{}{\pb}Liebſter Freund,\pend\vspace{0.5em}
\pstart
           Furchtbare \label{K_L02811-1v}\edtext{Brand-Kataſtrophe}{\lemma{\textnormal{\emph{Brand-Katastrophe}}}\Cendnote{\textnormal{Am Abend des 4. 5. 1897 gab es
                  einen Brand im Bazar de la Charité\oindex{Bazar de la Charité@\textbf{Bazar de la Charité}, \emph{Geschäft}|pwk} in der
                     Rue Jean Goujon\oindex{Rue Jean Goujon@\textbf{Rue Jean Goujon}, \emph{Straße}|pwk}, bei dem über 100
                  Menschen ihr Leben verloren.}}}\label{K_L02811-1}. Kann nicht kommen. Vielleicht biſt Du gegen
                  11 Uhr im \label{K_L02811-2v}\edtext{\textsc{Café}}{\lemma{\textnormal{\emph{Café}}}\Cendnote{\textnormal{nicht ermittelt}}}\label{K_L02811-2} an der Ecke der
                  \textsc{R. Maubeuge\oindex{5, rue de Maubeuge@\textbf{5, rue de Maubeuge}, \emph{Wohngebäude}|pw}}. Wenn ich kann, komme ich vorbei.\pend
           
\pstart
           Herzlichſt {\\[\baselineskip]}\spacefill\mbox{P. G.}\pend
           \leftskip=0em{}
\pstart
           \noindent{}\textsc{D\textsuperscript{r} Schnitzler}\pend
           
\pstart
           \textsc{5. rue Maubeuge\oindex{5, rue de Maubeuge@\textbf{5, rue de Maubeuge}, \emph{Wohngebäude}|pw}}\pend
           
\pstart
           \textsc{chez{ }M\textsuperscript{me}{ }Hauser\pwindex{Hauser @\textsc{Hauser}, \emph{Vermieterin}|pw}}.\pend
           \selectlanguage{ngerman}\endnumbering\briefempfaengerindex{Schnitzler, Arthur@\textsc{Schnitzler, Arthur}!zzzGoldmann, Paul@\emph{von Paul Goldmann}!1897-05-041@{{[}4. 5. 1897{]}}|)be}\mylabel{L02811h}  \newcommand{\dateiname}{L02811}\newcommand{\titel}{Paul Goldmann an Arthur Schnitzler, [4. 5. 1897]}\newcommand{\editorInnen}{Martin Anton Müller und Laura Untner}%% latex-leseansicht-abspann.tex
%% Abspann für die Leseansicht.
%% Der Schalter \ifkorrekturansicht ist bereits durch den Vorspann gesetzt.

%% latex-abspann.tex
%% Gemeinsamer Abspann für Korrekturansicht und Leseansicht.
%% Setzt den Schalter \ifkorrekturansicht voraus (gesetzt in den
%% einbindenden Dateien latex-korrekturansicht-abspann.tex bzw.
%% latex-leseansicht-abspann.tex).
%% ---------------------------------------------------------------

\normalsize

% Das esempio-Environment wird nur in der Leseansicht benötigt
\ifkorrekturansicht\else
\newenvironment{esempio}[3]%
{
    \vspace{1.5ex}
    \rlap{\underline{#1}}
    \par
    \setlength{\parindent}{0cm}
    \nopagebreak
    \leftskip=#2cm
    \rightskip=#3cm
}
{
    \par
}
\fi

\doendnotes{C}
\bigskip
\vfill

\clearpage

\footnotesize

\ifkorrekturansicht
  \lohead{\textsc{register}}
\fi

% theindex-Environment neu definieren ohne reledmac
\makeatletter
\renewenvironment{theindex}{%
  \ifkorrekturansicht
    \section*{\indexname}%
  \else
    \subsubsection*{Index der erwähnten Entitäten}%
  \fi
  \setlength{\parindent}{0pt}%
  \setlength{\parskip}{0pt plus 0.3pt}%
  \let\item\@idxitem
}{%
  \ifkorrekturansicht\clearpage\fi
}
\makeatother

\IfFileExists{\jobname-pw.ind}{\input{\jobname-pw.ind}}{}

% Quellenangabe nur in der Leseansicht
\ifkorrekturansicht\else
% Fallback-Definitionen, falls die .tex-Datei \titel etc. nicht gesetzt hat
\providecommand{\titel}{}
\providecommand{\editorInnen}{}
\providecommand{\dateiname}{\jobname}

\vspace{3cm}

\vfill

\footnotesize
\textsc{Quelle}: \titel. Herausgegeben von {\editorInnen}. In: \emph{Arthur Schnitzler: Briefwechsel mit Autorinnen und Autoren}.
 Digitale Edition, https://schnitzler-briefe.acdh.oeaw.ac.at/{\dateiname}.html (Stand \today)
\fi

\end{document}


