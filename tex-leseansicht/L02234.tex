%% latex-korrekturansicht-vorspann.tex
%% Vorspann für die Korrekturansicht.
%% Lädt die gemeinsame Datei latex-vorspann.tex mit gesetztem Schalter.

\newif\ifkorrekturansicht
\korrekturansichttrue

\input{../tex-inputs/latex-vorspann}


\section[Hugo von Hofmannsthal an Arthur Schnitzler, 24. 7. {[}1916{]}]{L02234 Hugo von Hofmannsthal an Arthur Schnitzler, 24. 7. {[}1916{]}}
\nopagebreak\mylabel{L02234v}
\rehead{ }\normalsize\beginnumbering\briefempfaengerindex{Schnitzler, Arthur@\textsc{Schnitzler, Arthur}!zzzHofmannsthal, Hugo von@\emph{von Hugo von Hofmannsthal}!1916-07-241@{24. 7. {[}1916{]}}|(be}
\toendnotes[C]{\smallbreak\pagebreak[2]}\Standort{CUL, Schnitzler, B 43.}
\physDesc{Briefkarte, 573 Zeichen
\newline{}Handschrift: schwarze Tinte, deutsche Kurrent
\newline{}Schnitzler: mit Bleistift Jahreszahl und Ort ergänzt: »1916{ }\textsc{Altaussee\oindex{Altaussee@\textbf{Altaussee}, \emph{A.ADM3}|pw}}« 
\newline{}Ordnung: 1) mit Bleistift von Frieda
                                    Pollak\pwindex{Pollak, Frieda 08.12.1881 – 13.07.1937@\textsc{Pollak, Frieda} (08.12.1881 – 13.07.1937), \emph{Sekretär/Sekretärin}|pw} (?) mit dem Buchstaben »A«
                                 (Abgeschrieben/Abschrift) gekennzeichnet  2) mit Bleistift von unbekannter Hand nummeriert: »\strikeout{346}« 3) mit Bleistift von unbekannter Hand nummeriert:
                                    »355«}
\buchAbdrucke{\weitereDrucke{Hugo von Hofmannsthal, Arthur Schnitzler: \emph{Briefwechsel}. Frankfurt am Main: \emph{S. Fischer} 1964, S. 278.} }\toendnotes[C]{\smallbreak}
\pstart
           \raggedleft{}{\pb}24 VII.\pend
           
\pstart{}mein lieber Arthur\pend\vspace{0.5em}
\pstart
           ich freue mich zu denken daſs Sie Olga\pwindex{Schnitzler, Olga 17.01.1882 – 13.01.1970@\textsc{Schnitzler, Olga} (17.01.1882 – 13.01.1970), \emph{Schauspieler/Schauspielerin, Sänger/Sängerin}|pw} u. die
                  Kinder\pwindex{Schnitzler, Heinrich 09.08.1902 – 12.07.1982@\textsc{Schnitzler, Heinrich} (09.08.1902 – 12.07.1982), \emph{Regisseur/Regisseurin, Schauspieler/Schauspielerin}|pwv}\pwindex{Cappellini, Lili 13.09.1909 – 26.07.1928@\textsc{Cappellini, Lili} (13.09.1909 – 26.07.1928)|pwv} hier in
               der Nähe ſind und, wie ich denke, zufrieden.\hspace*{1.5em}Ich
               hoffe daſs ich eine Zeitlang hier bleiben u. vielleicht etwas für mich arbeiten kann
               – es iſt freilich immer ungewiſs.\hspace*{1.5em}Die Kinder\pwindex{Schnitzler, Heinrich 09.08.1902 – 12.07.1982@\textsc{Schnitzler, Heinrich} (09.08.1902 – 12.07.1982), \emph{Regisseur/Regisseurin, Schauspieler/Schauspielerin}|pwv}\pwindex{Cappellini, Lili 13.09.1909 – 26.07.1928@\textsc{Cappellini, Lili} (13.09.1909 – 26.07.1928)|pwv}{ }ſagen mir, Sie hätten {\pb}geſagt, Ihre Arbeitszeit wäre
               nachmittag bis gegen 6\textsuperscript{h}.\hspace*{1.5em}So würde ich gerne morgen etwas nach 6\textsuperscript{h} zu Ihnen ko{\geminationm}en, Gerty\pwindex{Hofmannsthal, Gertrude von 16.03.1880 – 09.11.1959@\textsc{Hofmannsthal, Gertrude von} (16.03.1880 – 09.11.1959)|pw} auch (außer Olga\pwindex{Schnitzler, Olga 17.01.1882 – 13.01.1970@\textsc{Schnitzler, Olga} (17.01.1882 – 13.01.1970), \emph{Schauspieler/Schauspielerin, Sänger/Sängerin}|pw} läſst anderes
                  ſagen)\hspace*{1.5em}Man könnte dann vielleicht zuſa{\geminationm}en herumgehen u zuſa{\geminationm}en
               beim \textsc{Seewirth}\oindex{Hotel am See@\textbf{Hotel am See}, \emph{Hotel (K.HTL)}|pw} nachtmahlen. Wenn es paſst bedarf es keiner Antwort.\pend
           
\pstart
           Der Ihre, herzlich{\\[\baselineskip]}\spacefill\mbox{Hugo.}\pend
           \leftskip=0em{}\selectlanguage{ngerman}\endnumbering\briefempfaengerindex{Schnitzler, Arthur@\textsc{Schnitzler, Arthur}!zzzHofmannsthal, Hugo von@\emph{von Hugo von Hofmannsthal}!1916-07-241@{24. 7. {[}1916{]}}|)be}\mylabel{L02234h}  \normalsize

\doendnotes{C}
\bigskip
\vfill

\clearpage

\footnotesize

\lohead{\textsc{register}}

% Definiere theindex-Environment komplett neu ohne reledmac
\makeatletter
\renewenvironment{theindex}{%
  \section*{\indexname}%
  \setlength{\parindent}{0pt}%
  \setlength{\parskip}{0pt plus 0.3pt}%
  \let\item\@idxitem
}{%
  \clearpage
}
\makeatother

\IfFileExists{\jobname-pw.ind}{\input{\jobname-pw.ind}}{}

\end{document}

      