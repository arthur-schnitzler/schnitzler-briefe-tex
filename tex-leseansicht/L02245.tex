%% latex-leseansicht-vorspann.tex
%% Vorspann für die Leseansicht.
%% Lädt die gemeinsame Datei latex-vorspann.tex mit nicht gesetztem Schalter.

\newif\ifkorrekturansicht
\korrekturansichtfalse

\input{../tex-inputs/latex-vorspann}


               \section[Arthur Schnitzler an Robert Adam, 16. 11. 1916]{ Arthur Schnitzler an Robert Adam, 16. 11. 1916}\nopagebreak\mylabel{v}\rehead{ }\begin{ledgroupsized}[t]{13cm}\normalsize\beginnumbering\briefempfaengerindex{Adam, Robert@\textsc{Adam, Robert}!zzzSchnitzler, Arthur@\emph{von Arthur Schnitzler}!1916-11-161@{16. 11. 1916}|(be} \toendnotes[C]{\smallbreak\pagebreak[2]} \Standort{DLA, 96.34.1/22.}
\physDesc{Bildpostkarte
\newline{}Handschrift: schwarze Tinte, deutsche Kurrent\newline{}Versand: Stempel: »\nobreak{}\textcolor{gray}{1}8. XI. 16\nobreak{}«.  \newline{}Zusatz: Postkartenmotiv mit Olga\pwindex{Schnitzler, Olga 17.01.1882 – 13.01.1970@\textsc{Schnitzler, Olga} (17.01.1882 – 13.01.1970), \emph{Schauspielerin, Sängerin}|pw} und Heinrich\pwindex{Schnitzler, Heinrich 09.08.1902 – 12.07.1982@\textsc{Schnitzler, Heinrich} (09.08.1902 – 12.07.1982), \emph{Regisseur, Schauspieler}|pw} links vor dem Haus und Schnitzler und Lili\pwindex{Schnitzler, Lili 13.09.1909 – 26.07.1928@\textsc{Schnitzler, Lili} (13.09.1909 – 26.07.1928)|pw} auf dem Söller }\toendnotes[C]{\smallbreak}\pstart{}{\pb}\textsc{Hrn Dr. Robert Adam Pollak}\pend{}\pstart{}Wien XII\oindex{XII., Meidling@\textbf{XII., Meidling}|pw}\pend{}\pstart{}\textsc{Meidlinger Hauptstr} 56\oindex{Meidlinger Hauptstrasse@\textbf{Meidlinger Hauptstraße}|pw}.\pend{}{\bigskip}\pstart
           \noindent{}\centering{}{\pb}\textcolor{gray}{\textbf{Wien, XVIII, Sternwartestr. 71\oindex{Sternwartestrasse@\textbf{Sternwartestraße}|pw}.}}\pend
           \pstart
           \raggedleft{}{\pb}16. 11. 916.\pend
           \pstart
           Verehrter Herr Doktor, vielen Dank für die
                        freundl\textcolor{gray}{iche} Ueberſendg der \textsc{Dumas}\pwindex{Dumas, Alexandre pere 24.07.1802 – 05.12.1870@\textsc{Dumas, Alexandre père} (24.07.1802 – 05.12.1870), \emph{Schriftsteller}|pw}{ }\textsc{Mem}\pwindex{Dumas, Alexandre pere 24.07.1802 – 05.12.1870@\textsc{Dumas, Alexandre père} (24.07.1802 – 05.12.1870), \emph{Schriftsteller}!Meine Memoiren1852 – 1856@\strich\emph{Meine Memoiren} {[}1852 – 1856{]}|pwv} – geſtern erſt bega{\geminationn} ich ſie zu leſen – und
                    hoffe ſehr Sie zu ſehen u zu ſprechen, ehe ich zu Ende gelangt \substVorne{}\textsuperscript{\textcolor{gray}{w}}\substDazwischen{}bin\substHinten{}. Sie befinden ſich wohl wieder beſſer? Oder ganz gut? Dies und
                    aehnliches wünschend mit herzlichem Gruß Ihr\pend
           \pstart \spacefill\mbox{ArthurSchnitzler}\pend{}\endnumbering\briefempfaengerindex{Adam, Robert@\textsc{Adam, Robert}!zzzSchnitzler, Arthur@\emph{von Arthur Schnitzler}!1916-11-161@{16. 11. 1916}|)be}\mylabel{h}\end{ledgroupsized}  \newcommand{\dateiname}{L02245}\newcommand{\titel}{Arthur Schnitzler an Robert Adam, 16. 11. 1916}\newcommand{\editorInnen}{Martin Anton Müller und Gerd-Hermann Susen}%% latex-leseansicht-abspann.tex
%% Abspann für die Leseansicht.
%% Der Schalter \ifkorrekturansicht ist bereits durch den Vorspann gesetzt.

%% latex-abspann.tex
%% Gemeinsamer Abspann für Korrekturansicht und Leseansicht.
%% Setzt den Schalter \ifkorrekturansicht voraus (gesetzt in den
%% einbindenden Dateien latex-korrekturansicht-abspann.tex bzw.
%% latex-leseansicht-abspann.tex).
%% ---------------------------------------------------------------

\normalsize

% Das esempio-Environment wird nur in der Leseansicht benötigt
\ifkorrekturansicht\else
\newenvironment{esempio}[3]%
{
    \vspace{1.5ex}
    \rlap{\underline{#1}}
    \par
    \setlength{\parindent}{0cm}
    \nopagebreak
    \leftskip=#2cm
    \rightskip=#3cm
}
{
    \par
}
\fi

\doendnotes{C}
\bigskip
\vfill

\clearpage

\footnotesize

\ifkorrekturansicht
  \lohead{\textsc{register}}
\fi

% theindex-Environment neu definieren ohne reledmac
\makeatletter
\renewenvironment{theindex}{%
  \ifkorrekturansicht
    \section*{\indexname}%
  \else
    \subsubsection*{Index der erwähnten Entitäten}%
  \fi
  \setlength{\parindent}{0pt}%
  \setlength{\parskip}{0pt plus 0.3pt}%
  \let\item\@idxitem
}{%
  \ifkorrekturansicht\clearpage\fi
}
\makeatother

\IfFileExists{\jobname-pw.ind}{\input{\jobname-pw.ind}}{}

% Quellenangabe nur in der Leseansicht
\ifkorrekturansicht\else
% Fallback-Definitionen, falls die .tex-Datei \titel etc. nicht gesetzt hat
\providecommand{\titel}{}
\providecommand{\editorInnen}{}
\providecommand{\dateiname}{\jobname}

\vspace{3cm}

\vfill

\footnotesize
\textsc{Quelle}: \titel. Herausgegeben von {\editorInnen}. In: \emph{Arthur Schnitzler: Briefwechsel mit Autorinnen und Autoren}.
 Digitale Edition, https://schnitzler-briefe.acdh.oeaw.ac.at/{\dateiname}.html (Stand \today)
\fi

\end{document}


      