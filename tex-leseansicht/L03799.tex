%% latex-korrekturansicht-vorspann.tex
%% Vorspann für die Korrekturansicht.
%% Lädt die gemeinsame Datei latex-vorspann.tex mit gesetztem Schalter.

\newif\ifkorrekturansicht
\korrekturansichttrue

\input{../tex-inputs/latex-vorspann}


\section[Arthur Schnitzler an Stefan Zweig, 11. 2. 1911]{L03799 Arthur Schnitzler an Stefan Zweig, 11. 2. 1911}
\nopagebreak\mylabel{L03799v}
\rehead{ }\normalsize\beginnumbering\briefempfaengerindex{Zweig, Stefan@\textsc{Zweig, Stefan}!zzzSchnitzler, Arthur@\emph{von Arthur Schnitzler}!1911-02-112@{11. 2. 1911}|(be}
\toendnotes[C]{\smallbreak\pagebreak[2]}\Standort{Jerusalem, National Library of Israel, ARC. Ms. Var. 305 1 58 Stefan Zweig Collection.}
\physDesc{Postkarte, 1 Blatt, 2 Seiten, 179 Zeichen
\newline{}Handschrift: schwarze Tinte, deutsche Kurrent
\newline{}Versand: Stempel: »\nobreak{}\oindex{I., Innere Stadt@\textbf{I., Innere Stadt}, \emph{A.ADM3}|pwk}1/\textsubscript{1} Wien 6, 11. II. 11, 4\nobreak{}«.  }\toendnotes[C]{\smallbreak}\pstart{}{\pb}\textcolor{gray}{\textbf{Dr. Arthur Schnitzler}}\pend{}\pstart{}\textcolor{gray}{\textbf{Wien XVIII.
                        Sternwartestrasse 71\oindex{Sternwartestrasse 71@\textbf{Sternwartestraße 71}, \emph{Wohngebäude (K.WHS)}|pw}}}\pend{}{\bigskip}\pstart{}Herrn Dr. \textsc{Stefan Zweig}\pend{}\pstart{}Wien I\oindex{I., Innere Stadt@\textbf{I., Innere Stadt}, \emph{A.ADM3}|pw}\pend{}\pstart{}\textsc{Kochgasse 8}\oindex{Kochgasse 8@\textbf{Kochgasse 8}, \emph{Wohngebäude (K.WHS)}|pw}\pend{}{\bigskip}\vspace{1em}
\pstart{}{\pb}lieber Herr Doctor\pend\vspace{0.5em}
\pstart
           danke ſchön! Der \label{K_L03799-1v}\edtext{Artikel\pwindex{vie litteraire à l etrangere. Arthur Schnitzler@\emph{La vie littéraire à l’étrangere. Arthur Schnitzler}|pwv}}{\lemma{\textnormal{\emph{Artikel}}}\Cendnote{\textnormal{Stanislas Rzewuski\pwindex{Rzewuski, Stanisław 1864-07-27 – 1913-05-16@\textsc{Rzewuski, Stanisław} (1864-07-27 – 1913-05-16), \emph{Dramatiker/Dramatikerin, Kritiker/Kritikerin, Schriftsteller/Schriftstellerin}|pwk}: \emph{Arthur Schnitzler}\pwindex{vie litteraire à l etrangere. Arthur Schnitzler@\emph{La vie littéraire à l’étrangere. Arthur Schnitzler}|pwk}. In: \emph{Le Figaro}\pwindex{Le Figaro@\emph{Le Figaro}|pwk}. Supplément Littéraire, Jg. 7, Nr. 5,
                        4. 2. 1911, S. 3.}}}\label{K_L03799-1} (von \textsc{Rzewuski\pwindex{Rzewuski, Stanisław 1864-07-27 – 1913-05-16@\textsc{Rzewuski, Stanisław} (1864-07-27 – 1913-05-16), \emph{Dramatiker/Dramatikerin, Kritiker/Kritikerin, Schriftsteller/Schriftstellerin}|pw}}, nicht \textsc{Wyzewa\pwindex{Wyzewa, Theodore de 1862-09-12 – 1917-04-07@\textsc{Wyzewa, Théodore de} (1862-09-12 – 1917-04-07), \emph{Schriftsteller/Schriftstellerin, Journalist/Journalistin}|pw}}) im \textsc{Figaro\pwindex{Le Figaro@\emph{Le Figaro}|pw}} iſt mir durch den \label{K_L03799-2v}\edtext{\textsc{Observer}\orgindex{Observer. Alexander Weigl s Unternehmen fuer Zeitungssausschnitte@Observer. Alexander Weigl’s Unternehmen für Zeitungssausschnitte|pw}}{\lemma{\textnormal{\emph{Observer}}}\Cendnote{\textnormal{Eine
                  Agentur für Zeitungsausschnitte. Schnitzler hatte ein Abonnement und bekam
                  Zeitungsartikel zugesandt, die ihn mehr als nur im Nebenbei erwähnten.}}}\label{K_L03799-2}
               zugegangen.\pend
           
\pstart
           Herzlichſt Ihr{\\[\baselineskip]}\spacefill\mbox{A. S.}\pend
           \leftskip=0em{}
\pstart
           11. 2. 911\pend
           \selectlanguage{ngerman}\endnumbering\briefempfaengerindex{Zweig, Stefan@\textsc{Zweig, Stefan}!zzzSchnitzler, Arthur@\emph{von Arthur Schnitzler}!1911-02-112@{11. 2. 1911}|)be}\mylabel{L03799h}  \normalsize

\doendnotes{C}
\bigskip
\vfill

\clearpage

\footnotesize

\lohead{\textsc{register}}

% Definiere theindex-Environment komplett neu ohne reledmac
\makeatletter
\renewenvironment{theindex}{%
  \section*{\indexname}%
  \setlength{\parindent}{0pt}%
  \setlength{\parskip}{0pt plus 0.3pt}%
  \let\item\@idxitem
}{%
  \clearpage
}
\makeatother

\IfFileExists{\jobname-pw.ind}{\input{\jobname-pw.ind}}{}

\end{document}

      