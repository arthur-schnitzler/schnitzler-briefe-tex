%% latex-leseansicht-vorspann.tex
%% Vorspann für die Leseansicht.
%% Lädt die gemeinsame Datei latex-vorspann.tex mit nicht gesetztem Schalter.

\newif\ifkorrekturansicht
\korrekturansichtfalse

\input{../tex-inputs/latex-vorspann}


\section[Arthur Schnitzler an Stefan Zweig, 11. 2. 1911]{L03799 Arthur Schnitzler an Stefan Zweig, 11. 2. 1911}
\nopagebreak\mylabel{L03799v}
\rehead{ }\normalsize\beginnumbering\briefempfaengerindex{Zweig, Stefan@\textsc{Zweig, Stefan}!zzzSchnitzler, Arthur@\emph{von Arthur Schnitzler}!1911-02-112@{11. 2. 1911}|(be}
\toendnotes[C]{\smallbreak\pagebreak[2]}
\correspDesc{Versand  durch Arthur Schnitzler am 11. 2. 1911 in Wien
\newline{}Erhalt  durch Stefan Zweig im Zeitraum [11. 2. 1911 – 14. 2. 1911?] in Wien}\toendnotes[C]{\smallbreak}
\Standort{Jerusalem, National Library of Israel, ARC. Ms. Var. 305 1 58 Stefan Zweig Collection.}
\physDesc{Postkarte, 179 Zeichen
\newline{}Handschrift: schwarze Tinte, deutsche Kurrent
\newline{}Versand: Stempel: »\nobreak{}\oindex{I., Innere Stadt@\textbf{I., Innere Stadt}, \emph{Verwaltungsgebiet}|pwk}1/\textsubscript{1} Wien 6, 11. II. 11, 4\nobreak{}«.  }\toendnotes[C]{\smallbreak}\pstart{}{\pb}\textcolor{gray}{\textbf{Dr. Arthur Schnitzler}}\pend{}\pstart{}\textcolor{gray}{\textbf{Wien XVIII.
                        Sternwartestrasse 71\oindex{Wien@\textbf{Wien}!XVIII., Währing@\textbf{XVIII., Währing}!Sternwartestraße 71@\textbf{Sternwartestraße 71}, \emph{Wohngebäude}|pw}}}\pend{}{\bigskip}\pstart{}Herrn Dr. \textsc{Stefan Zweig}\pend{}\pstart{}Wien I\oindex{I., Innere Stadt@\textbf{I., Innere Stadt}, \emph{Verwaltungsgebiet}|pw}\pend{}\pstart{}\textsc{Kochgasse 8}\oindex{Wien@\textbf{Wien}!VIII., Josefstadt@\textbf{VIII., Josefstadt}!Kochgasse 8@\textbf{Kochgasse 8}, \emph{Wohngebäude}|pw}\pend{}{\bigskip}\vspace{1em}
\pstart{}{\pb}lieber Herr Doctor\pend\vspace{0.5em}
\pstart
           danke{ }ſchön! Der \label{K_L03799-1v}\edtext{Artikel\pwindex{Rzewuski, Stanisław 27.\,7.\,1864 Pohrebyshche – 16.\,5.\,1913 18. arrondissement [Paris]@\textsc{Rzewuski, Stanisław} (27.\,7.\,1864 Pohrebyshche – 16.\,5.\,1913 18. arrondissement [Paris]), \emph{Dramatiker, Kritiker, Schriftsteller}!vie littéraire à l’étrangere. Arthur Schnitzler@\strich\emph{La vie littéraire à l’étrangere. Arthur Schnitzler}|pwv}}{\lemma{\textnormal{\emph{Artikel}}}\Cendnote{\textnormal{Stanislas Rzewuski\pwindex{Rzewuski, Stanisław 27.\,7.\,1864 Pohrebyshche – 16.\,5.\,1913 18. arrondissement [Paris]@\textsc{Rzewuski, Stanisław} (27.\,7.\,1864 Pohrebyshche – 16.\,5.\,1913 18. arrondissement [Paris]), \emph{Dramatiker, Kritiker, Schriftsteller}|pwk}: \emph{Arthur Schnitzler}\pwindex{Rzewuski, Stanisław 27.\,7.\,1864 Pohrebyshche – 16.\,5.\,1913 18. arrondissement [Paris]@\textsc{Rzewuski, Stanisław} (27.\,7.\,1864 Pohrebyshche – 16.\,5.\,1913 18. arrondissement [Paris]), \emph{Dramatiker, Kritiker, Schriftsteller}!vie littéraire à l’étrangere. Arthur Schnitzler@\strich\emph{La vie littéraire à l’étrangere. Arthur Schnitzler}|pwk}. In: \emph{Le Figaro}\pwindex{Le Figaro@\emph{Le Figaro}|pwk}. Supplément Littéraire, Jg. 7, Nr. 5,
                        4. 2. 1911, S. 3.}}}\label{K_L03799-1} (von \textsc{Rzewuski\pwindex{Rzewuski, Stanisław 27.\,7.\,1864 Pohrebyshche – 16.\,5.\,1913 18. arrondissement [Paris]@\textsc{Rzewuski, Stanisław} (27.\,7.\,1864 Pohrebyshche – 16.\,5.\,1913 18. arrondissement [Paris]), \emph{Dramatiker, Kritiker, Schriftsteller}|pw}}, nicht \textsc{Wyzewa\pwindex{Wyzewa, Théodore de 12.\,9.\,1862 Kalush – 7.\,4.\,1917 Paris@\textsc{Wyzewa, Théodore de} (12.\,9.\,1862 Kalush – 7.\,4.\,1917 Paris), \emph{Schriftsteller, Journalist}|pw}}) im \textsc{Figaro\pwindex{Le Figaro@\emph{Le Figaro}|pw}} iſt mir durch den \label{K_L03799-2v}\edtext{\textsc{Observer}\orgindex{Observer. Alexander Weigl’s Unternehmen für Zeitungssausschnitte@Observer. Alexander Weigl’s Unternehmen für Zeitungssausschnitte|pw}}{\lemma{\textnormal{\emph{Observer}}}\Cendnote{\textnormal{Eine
                  Agentur für Zeitungsausschnitte. Schnitzler hatte ein Abonnement und bekam
                  Zeitungsartikel zugesandt, die ihn mehr als nur im Nebenbei erwähnten.}}}\label{K_L03799-2}
               zugegangen.\pend
           
\pstart
           Herzlichſt Ihr{\\[\baselineskip]}\spacefill\mbox{A. S.}\pend
           \leftskip=0em{}
\pstart
           11. 2. 911\pend
           \selectlanguage{ngerman}\endnumbering\briefempfaengerindex{Zweig, Stefan@\textsc{Zweig, Stefan}!zzzSchnitzler, Arthur@\emph{von Arthur Schnitzler}!1911-02-112@{11. 2. 1911}|)be}\mylabel{L03799h}  \newcommand{\dateiname}{L03799}\newcommand{\titel}{Arthur Schnitzler an Stefan Zweig, 11. 2. 1911}\newcommand{\editorInnen}{Selma Jahnke und Martin Anton Müller}%% latex-leseansicht-abspann.tex
%% Abspann für die Leseansicht.
%% Der Schalter \ifkorrekturansicht ist bereits durch den Vorspann gesetzt.

%% latex-abspann.tex
%% Gemeinsamer Abspann für Korrekturansicht und Leseansicht.
%% Setzt den Schalter \ifkorrekturansicht voraus (gesetzt in den
%% einbindenden Dateien latex-korrekturansicht-abspann.tex bzw.
%% latex-leseansicht-abspann.tex).
%% ---------------------------------------------------------------

\normalsize

% Das esempio-Environment wird nur in der Leseansicht benötigt
\ifkorrekturansicht\else
\newenvironment{esempio}[3]%
{
    \vspace{1.5ex}
    \rlap{\underline{#1}}
    \par
    \setlength{\parindent}{0cm}
    \nopagebreak
    \leftskip=#2cm
    \rightskip=#3cm
}
{
    \par
}
\fi

\doendnotes{C}
\bigskip
\vfill

\clearpage

\footnotesize

\ifkorrekturansicht
  \lohead{\textsc{register}}
\fi

% theindex-Environment neu definieren ohne reledmac
\makeatletter
\renewenvironment{theindex}{%
  \ifkorrekturansicht
    \section*{\indexname}%
  \else
    \subsubsection*{Index der erwähnten Entitäten}%
  \fi
  \setlength{\parindent}{0pt}%
  \setlength{\parskip}{0pt plus 0.3pt}%
  \let\item\@idxitem
}{%
  \ifkorrekturansicht\clearpage\fi
}
\makeatother

\IfFileExists{\jobname-pw.ind}{\input{\jobname-pw.ind}}{}

% Quellenangabe nur in der Leseansicht
\ifkorrekturansicht\else
% Fallback-Definitionen, falls die .tex-Datei \titel etc. nicht gesetzt hat
\providecommand{\titel}{}
\providecommand{\editorInnen}{}
\providecommand{\dateiname}{\jobname}

\vspace{3cm}

\vfill

\footnotesize
\textsc{Quelle}: \titel. Herausgegeben von {\editorInnen}. In: \emph{Arthur Schnitzler: Briefwechsel mit Autorinnen und Autoren}.
 Digitale Edition, https://schnitzler-briefe.acdh.oeaw.ac.at/{\dateiname}.html (Stand \today)
\fi

\end{document}


