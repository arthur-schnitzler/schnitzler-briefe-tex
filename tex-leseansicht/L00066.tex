%% latex-leseansicht-vorspann.tex
%% Vorspann für die Leseansicht.
%% Lädt die gemeinsame Datei latex-vorspann.tex mit nicht gesetztem Schalter.

\newif\ifkorrekturansicht
\korrekturansichtfalse

\input{../tex-inputs/latex-vorspann}


\section[Hermann Bahr an Arthur Schnitzler, {[}1.? 2. 1892{]}]{L00066 Hermann Bahr an Arthur Schnitzler, {[}1.? 2. 1892{]}}
\nopagebreak\mylabel{L00066v}
\rehead{ }\normalsize\beginnumbering\briefempfaengerindex{Schnitzler, Arthur@\textsc{Schnitzler, Arthur}!zzzBahr, Hermann@\emph{von Hermann Bahr}!1892-02-011@{{[}1.? 2. 1892{]}}|(be}
\toendnotes[C]{\smallbreak\pagebreak[2]}
\correspDesc{Versand  durch Hermann Bahr am [1.? 2. 1892] in Wien
\newline{}Erhalt  durch Arthur Schnitzler im Zeitraum [1. 2. 1892
                  – 5. 2. 1892?] in Wien}\toendnotes[C]{\smallbreak}
\Standort{CUL, Schnitzler, B 5b.}
\physDesc{Visitenkarte, 107 Zeichen
\newline{}Handschrift: Bleistift, deutsche Kurrent
\newline{}Schnitzler: mit rotem Buntstift datiert: »92« 
\newline{}Ordnung: mit Bleistift und rotem Buntstift von unbekannter Hand
                                 nummeriert: »2« }
\buchAbdrucke{\weitereDrucke{Hermann Bahr, Arthur Schnitzler: \emph{Briefwechsel, Aufzeichnungen, Dokumente (1891–1931)}. Herausgegeben von Kurt Ifkovits und Martin Anton Müller. Göttingen: \emph{Wallstein} 2018, S. 20.} }\toendnotes[C]{\smallbreak}
\pstart
           \noindent{}\centering{}\textcolor{gray}{\textbf{{\pb}Hermann Bahr.}}\pend
           
\pstart
           bittet Sie, ihm mitzuteilen, ob er Ihnen eine Einladg {\pb}zu \label{K_L00066-1v}\edtext{\textsc{Matinee}{ }\textsc{Reicher}\pwindex{Reicher, Emanuel 18.\,6.\,1849 Bochnia – 15.\,5.\,1924 Berlin@\textsc{Reicher, Emanuel} (18.\,6.\,1849 Bochnia – 15.\,5.\,1924 Berlin), \emph{Schauspieler}|pw}}{\lemma{\textnormal{\emph{Matinee Reicher}}}\Cendnote{\textnormal{Die Matinée fand am 7. 2. 1892{ }statt; Schnitzler nahm teil.}}}\label{K_L00066-1} bei Goldſchmid\pwindex{Goldschmidt, Adalbert von 5.\,5.\,1848 Wien – 21.\,12.\,1906 ebd.@\textsc{Goldschmidt, Adalbert von} (5.\,5.\,1848 Wien – 21.\,12.\,1906 ebd.), \emph{Schriftsteller, Komponist}|pw} beſorgen{ }ſoll\pend
           
\pstart
           Herzlichſt{\\[\baselineskip]}\spacefill\mbox{H}\pend
           \leftskip=0em{}\selectlanguage{ngerman}\endnumbering\briefempfaengerindex{Schnitzler, Arthur@\textsc{Schnitzler, Arthur}!zzzBahr, Hermann@\emph{von Hermann Bahr}!1892-02-011@{{[}1.? 2. 1892{]}}|)be}\mylabel{L00066h}  \newcommand{\dateiname}{L00066}\newcommand{\titel}{Hermann Bahr an Arthur Schnitzler, [1.? 2. 1892]}\newcommand{\editorInnen}{Herausgegeben von Martin Anton Müller}%% latex-leseansicht-abspann.tex
%% Abspann für die Leseansicht.
%% Der Schalter \ifkorrekturansicht ist bereits durch den Vorspann gesetzt.

%% latex-abspann.tex
%% Gemeinsamer Abspann für Korrekturansicht und Leseansicht.
%% Setzt den Schalter \ifkorrekturansicht voraus (gesetzt in den
%% einbindenden Dateien latex-korrekturansicht-abspann.tex bzw.
%% latex-leseansicht-abspann.tex).
%% ---------------------------------------------------------------

\normalsize

% Das esempio-Environment wird nur in der Leseansicht benötigt
\ifkorrekturansicht\else
\newenvironment{esempio}[3]%
{
    \vspace{1.5ex}
    \rlap{\underline{#1}}
    \par
    \setlength{\parindent}{0cm}
    \nopagebreak
    \leftskip=#2cm
    \rightskip=#3cm
}
{
    \par
}
\fi

\doendnotes{C}
\bigskip
\vfill

\clearpage

\footnotesize

\ifkorrekturansicht
  \lohead{\textsc{register}}
\fi

% theindex-Environment neu definieren ohne reledmac
\makeatletter
\renewenvironment{theindex}{%
  \ifkorrekturansicht
    \section*{\indexname}%
  \else
    \subsubsection*{Index der erwähnten Entitäten}%
  \fi
  \setlength{\parindent}{0pt}%
  \setlength{\parskip}{0pt plus 0.3pt}%
  \let\item\@idxitem
}{%
  \ifkorrekturansicht\clearpage\fi
}
\makeatother

\IfFileExists{\jobname-pw.ind}{\input{\jobname-pw.ind}}{}

% Quellenangabe nur in der Leseansicht
\ifkorrekturansicht\else
% Fallback-Definitionen, falls die .tex-Datei \titel etc. nicht gesetzt hat
\providecommand{\titel}{}
\providecommand{\editorInnen}{}
\providecommand{\dateiname}{\jobname}

\vspace{3cm}

\vfill

\footnotesize
\textsc{Quelle}: \titel. Herausgegeben von {\editorInnen}. In: \emph{Arthur Schnitzler: Briefwechsel mit Autorinnen und Autoren}.
 Digitale Edition, https://schnitzler-briefe.acdh.oeaw.ac.at/{\dateiname}.html (Stand \today)
\fi

\end{document}


