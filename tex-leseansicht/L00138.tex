%% latex-korrekturansicht-vorspann.tex
%% Vorspann für die Korrekturansicht.
%% Lädt die gemeinsame Datei latex-vorspann.tex mit gesetztem Schalter.

\newif\ifkorrekturansicht
\korrekturansichttrue

\input{../tex-inputs/latex-vorspann}


\section[Hugo von Hofmannsthal an Arthur Schnitzler, {[}23. 11. 1892?{]}]{L00138 Hugo von Hofmannsthal an Arthur Schnitzler, {[}23. 11. 1892?{]}}
\nopagebreak\mylabel{L00138v}
\rehead{ }\normalsize\beginnumbering\briefempfaengerindex{Schnitzler, Arthur@\textsc{Schnitzler, Arthur}!zzzHofmannsthal, Hugo von@\emph{von Hugo von Hofmannsthal}!1892-11-231@{{[}23. 11. 1892?{]}}|(be}
\toendnotes[C]{\smallbreak\pagebreak[2]}\Standort{CUL, Schnitzler, B 43.}
\physDesc{Briefkarte, 487 Zeichen
\newline{}Handschrift: 1) blaue Tinte, deutsche Kurrent (\noindent{}bis »macht aber nichts.«)\hspace{1em}2) schwarze Tinte, deutsche Kurrent (\noindent{}bis »Robert E«)\hspace{1em}3) Bleistift, deutsche Kurrent (\noindent{}ab »hrhardt und Paul Horn«)\hspace{1em}
\newline{}Ordnung: mit Bleistift von unbekannter Hand nummeriert:
                                 »8« }
\buchAbdrucke{\weitereDrucke{Hugo von Hofmannsthal, Arthur Schnitzler: \emph{Briefwechsel}. Frankfurt am Main: \emph{S. Fischer} 1964, S. 31.} }\toendnotes[C]{\smallbreak}
\pstart
           \raggedleft{}{\pb}\label{K_L00138-1v}\edtext{Mittwoch}{\lemma{\textnormal{\emph{Mittwoch}}}\Cendnote{\textnormal{Die Datierung beruht auf dem Brief: Arthur Schnitzler an Hugo von Hofmannsthal, 24. 11. 1892, bei dem es sich um die Antwort auf diese Karte handeln dürfte.}}}\label{K_L00138-1}\pend
           
\pstart{}Lieber Arthur\pend\vspace{0.5em}
\pstart
           Ich ſchreibe zufällig \label{T_L00138-1v}\edtext{an Richards\pwindex{Beer-Hofmann, Richard 1866-07-11 – 1945-09-26@\textsc{Beer-Hofmann, Richard} (1866-07-11 – 1945-09-26), \emph{Schriftsteller/Schriftstellerin}|pw} Schreibtiſch}{\lemma{\textnormal{\emph{an Richards Schreibtiſch}}}\Cendnote{\textnormal{Papier und der verwendete blaue Stift entsprechen den Briefen
                     Richard Beer-Hofmanns\pwindex{Beer-Hofmann, Richard 1866-07-11 – 1945-09-26@\textsc{Beer-Hofmann, Richard} (1866-07-11 – 1945-09-26), \emph{Schriftsteller/Schriftstellerin}|pwk}.}}}\label{T_L00138-1}, das macht
               aber nichts. Ich möchte Ihnen nämlich etwas ſagen: \strikeout{wir} wir ſollten doch einmal wieder ein bischen unter uns zuſammenkommen.
                  Robert\pwindex{Ehrhart-Ehrhartstein, Robert 12.09.1870 – 11.11.1956@\textsc{Ehrhart-Ehrhartstein, Robert} (12.09.1870 – 11.11.1956), \emph{Schriftsteller/Schriftstellerin, Ministerialbeamter/Ministerialbeamte}|pw}{ }Ehrhardt\pwindex{Ehrhart-Ehrhartstein, Robert 12.09.1870 – 11.11.1956@\textsc{Ehrhart-Ehrhartstein, Robert} (12.09.1870 – 11.11.1956), \emph{Schriftsteller/Schriftstellerin, Ministerialbeamter/Ministerialbeamte}|pw} und \textsc{Paul Horn}\pwindex{Horn, Paul 13.02.1867 – 18.01.1936@\textsc{Horn, Paul} (13.02.1867 – 18.01.1936), \emph{Fabrikant/Fabrikantin}|pw} und alle ſind ja jeder in ſeiner Art ſehr nett, aber immer, das vergröbert und
               encanailliert naturgemäß Thema und Ton. Ich gehe deshalb nicht zu {\pb}Pfob\oindex{Cafe Pfob@\textbf{Café Pfob}, \emph{Kaffeehaus (K.KAF)}|pw}. Meinen Sie nicht auch? Wir haben ja sehr gut
               ohne das alles existiert. Uebrigens auf Wiedersehen Sonntag.\pend
           \pstart Ihr \spacefill\mbox{Loris}\pend{}\selectlanguage{ngerman}\endnumbering\briefempfaengerindex{Schnitzler, Arthur@\textsc{Schnitzler, Arthur}!zzzHofmannsthal, Hugo von@\emph{von Hugo von Hofmannsthal}!1892-11-231@{{[}23. 11. 1892?{]}}|)be}\mylabel{L00138h}  \normalsize

\doendnotes{C}
\bigskip
\vfill

\clearpage

\footnotesize

\lohead{\textsc{register}}

% Definiere theindex-Environment komplett neu ohne reledmac
\makeatletter
\renewenvironment{theindex}{%
  \section*{\indexname}%
  \setlength{\parindent}{0pt}%
  \setlength{\parskip}{0pt plus 0.3pt}%
  \let\item\@idxitem
}{%
  \clearpage
}
\makeatother

\IfFileExists{\jobname-pw.ind}{\input{\jobname-pw.ind}}{}

\end{document}

      