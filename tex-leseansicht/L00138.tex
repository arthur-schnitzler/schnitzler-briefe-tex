%% latex-leseansicht-vorspann.tex
%% Vorspann für die Leseansicht.
%% Lädt die gemeinsame Datei latex-vorspann.tex mit nicht gesetztem Schalter.

\newif\ifkorrekturansicht
\korrekturansichtfalse

\input{../tex-inputs/latex-vorspann}


\section[Hugo von Hofmannsthal an Arthur Schnitzler, {[}23. 11. 1892?{]}]{L00138 Hugo von Hofmannsthal an Arthur Schnitzler, {[}23. 11. 1892?{]}}
\nopagebreak\mylabel{L00138v}
\rehead{ }\normalsize\beginnumbering\briefempfaengerindex{Schnitzler, Arthur@\textsc{Schnitzler, Arthur}!zzzHofmannsthal, Hugo von@\emph{von Hugo von Hofmannsthal}!1892-11-231@{{[}23. 11. 1892?{]}}|(be}
\toendnotes[C]{\smallbreak\pagebreak[2]}
\correspDesc{Versand  durch Hugo von Hofmannsthal am [23. 11. 1892?] in Wien
\newline{}Erhalt  durch Arthur Schnitzler im Zeitraum [23. 11. 1892 – 27. 11. 1892?] in Wien}\toendnotes[C]{\smallbreak}
\Standort{CUL, Schnitzler, B 43.}
\physDesc{Briefkarte, 487 Zeichen
\newline{}Handschrift: 1) blaue Tinte, deutsche Kurrent (\noindent{}bis »macht aber nichts.«)\hspace{1em}2) schwarze Tinte, deutsche Kurrent (\noindent{}bis »Robert E«)\hspace{1em}3) Bleistift, deutsche Kurrent (\noindent{}ab »hrhardt und Paul Horn«)\hspace{1em}
\newline{}Ordnung: mit Bleistift von unbekannter Hand nummeriert:
                                 »8« }
\buchAbdrucke{\weitereDrucke{Hugo von Hofmannsthal, Arthur Schnitzler: \emph{Briefwechsel}. Herausgegeben von Therese Nickl und Heinrich Schnitzler. Frankfurt am Main: \emph{S. Fischer} 1964, S. 31.} }\toendnotes[C]{\smallbreak}
\pstart
           \raggedleft{}{\pb}\label{K_L00138-1v}\edtext{Mittwoch}{\lemma{\textnormal{\emph{Mittwoch}}}\Cendnote{\textnormal{Die Datierung beruht auf dem Brief: XXXX Auszeichnungsfehler: Dokument L00139 nicht gefunden, bei dem es sich um die Antwort auf diese Karte handeln dürfte.}}}\label{K_L00138-1}\pend
           
\pstart{}Lieber Arthur\pend\vspace{0.5em}
\pstart
           Ich{ }ſchreibe zufällig \label{T_L00138-1v}\edtext{an Richards\pwindex{Beer-Hofmann, Richard 11.\,7.\,1866 Wien – 26.\,9.\,1945 New York City@\textsc{Beer-Hofmann, Richard} (11.\,7.\,1866 Wien – 26.\,9.\,1945 New York City), \emph{Schriftsteller}|pw} Schreibtiſch}{\lemma{\textnormal{\emph{an Richards Schreibtisch}}}\Cendnote{\textnormal{Papier und der verwendete blaue Stift entsprechen den Briefen
                     Richard Beer-Hofmanns\pwindex{Beer-Hofmann, Richard 11.\,7.\,1866 Wien – 26.\,9.\,1945 New York City@\textsc{Beer-Hofmann, Richard} (11.\,7.\,1866 Wien – 26.\,9.\,1945 New York City), \emph{Schriftsteller}|pwk}.}}}\label{T_L00138-1}, das macht
               aber nichts. Ich möchte Ihnen nämlich etwas{ }ſagen: \strikeout{wir} wir{ }ſollten doch einmal wieder ein bischen unter uns zuſammenkommen.
                  Robert\pwindex{Ehrhart-Ehrhartstein, Robert 12.\,9.\,1870 Innsbruck – 11.\,11.\,1956 Baden bei Wien@\textsc{Ehrhart-Ehrhartstein, Robert} (12.\,9.\,1870 Innsbruck – 11.\,11.\,1956 Baden bei Wien), \emph{Schriftsteller, Ministerialbeamter}|pw}{ }Ehrhardt\pwindex{Ehrhart-Ehrhartstein, Robert 12.\,9.\,1870 Innsbruck – 11.\,11.\,1956 Baden bei Wien@\textsc{Ehrhart-Ehrhartstein, Robert} (12.\,9.\,1870 Innsbruck – 11.\,11.\,1956 Baden bei Wien), \emph{Schriftsteller, Ministerialbeamter}|pw} und \textsc{Paul Horn}\pwindex{Horn, Paul 13.\,2.\,1867 Wien – 18.\,1.\,1936 Menton@\textsc{Horn, Paul} (13.\,2.\,1867 Wien – 18.\,1.\,1936 Menton), \emph{Fabrikant}|pw} und alle{ }ſind ja jeder in{ }ſeiner Art{ }ſehr nett, aber immer, das vergröbert und
               encanailliert naturgemäß Thema und Ton. Ich gehe deshalb nicht zu {\pb}Pfob\oindex{Wien@\textbf{Wien}!I., Innere Stadt@\textbf{I., Innere Stadt}!Café Pfob@\textbf{Café Pfob}, \emph{Kaffeehaus}|pw}. Meinen Sie nicht auch? Wir haben ja sehr gut
               ohne das alles existiert. Uebrigens auf Wiedersehen Sonntag.\pend
           \pstart Ihr \spacefill\mbox{Loris}\pend{}\selectlanguage{ngerman}\endnumbering\briefempfaengerindex{Schnitzler, Arthur@\textsc{Schnitzler, Arthur}!zzzHofmannsthal, Hugo von@\emph{von Hugo von Hofmannsthal}!1892-11-231@{{[}23. 11. 1892?{]}}|)be}\mylabel{L00138h}  \newcommand{\dateiname}{L00138}\newcommand{\titel}{Hugo von Hofmannsthal an Arthur Schnitzler, [23. 11. 1892?]}\newcommand{\editorInnen}{Martin Anton Müller und Gerd-Hermann Susen}%% latex-leseansicht-abspann.tex
%% Abspann für die Leseansicht.
%% Der Schalter \ifkorrekturansicht ist bereits durch den Vorspann gesetzt.

%% latex-abspann.tex
%% Gemeinsamer Abspann für Korrekturansicht und Leseansicht.
%% Setzt den Schalter \ifkorrekturansicht voraus (gesetzt in den
%% einbindenden Dateien latex-korrekturansicht-abspann.tex bzw.
%% latex-leseansicht-abspann.tex).
%% ---------------------------------------------------------------

\normalsize

% Das esempio-Environment wird nur in der Leseansicht benötigt
\ifkorrekturansicht\else
\newenvironment{esempio}[3]%
{
    \vspace{1.5ex}
    \rlap{\underline{#1}}
    \par
    \setlength{\parindent}{0cm}
    \nopagebreak
    \leftskip=#2cm
    \rightskip=#3cm
}
{
    \par
}
\fi

\doendnotes{C}
\bigskip
\vfill

\clearpage

\footnotesize

\ifkorrekturansicht
  \lohead{\textsc{register}}
\fi

% theindex-Environment neu definieren ohne reledmac
\makeatletter
\renewenvironment{theindex}{%
  \ifkorrekturansicht
    \section*{\indexname}%
  \else
    \subsubsection*{Index der erwähnten Entitäten}%
  \fi
  \setlength{\parindent}{0pt}%
  \setlength{\parskip}{0pt plus 0.3pt}%
  \let\item\@idxitem
}{%
  \ifkorrekturansicht\clearpage\fi
}
\makeatother

\IfFileExists{\jobname-pw.ind}{\input{\jobname-pw.ind}}{}

% Quellenangabe nur in der Leseansicht
\ifkorrekturansicht\else
% Fallback-Definitionen, falls die .tex-Datei \titel etc. nicht gesetzt hat
\providecommand{\titel}{}
\providecommand{\editorInnen}{}
\providecommand{\dateiname}{\jobname}

\vspace{3cm}

\vfill

\footnotesize
\textsc{Quelle}: \titel. Herausgegeben von {\editorInnen}. In: \emph{Arthur Schnitzler: Briefwechsel mit Autorinnen und Autoren}.
 Digitale Edition, https://schnitzler-briefe.acdh.oeaw.ac.at/{\dateiname}.html (Stand \today)
\fi

\end{document}


