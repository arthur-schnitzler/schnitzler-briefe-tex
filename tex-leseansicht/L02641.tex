%% latex-korrekturansicht-vorspann.tex
%% Vorspann für die Korrekturansicht.
%% Lädt die gemeinsame Datei latex-vorspann.tex mit gesetztem Schalter.

\newif\ifkorrekturansicht
\korrekturansichttrue

\input{../tex-inputs/latex-vorspann}


\section[Paul Goldmann an Arthur Schnitzler, 25. 6. 1889]{L02641 Paul Goldmann an Arthur Schnitzler, 25. 6. 1889}
\nopagebreak\mylabel{L02641v}
\rehead{ }\normalsize\beginnumbering\briefempfaengerindex{Schnitzler, Arthur@\textsc{Schnitzler, Arthur}!zzzGoldmann, Paul@\emph{von Paul Goldmann}!1889-06-251@{25. 6. 1889}|(be}
\toendnotes[C]{\smallbreak\pagebreak[2]}\Standort{DLA, A:Schnitzler, HS.NZ85.1.3162.}
\physDesc{Brief, 1 Blatt, 2 Seiten, 470 Zeichen
\newline{}Handschrift: blaue Tinte, deutsche Kurrent
\newline{}Schnitzler: mit rotem Buntstift eine Unterstreichung }\toendnotes[C]{\smallbreak}
\pstart
           \centering{}{\pb}\textcolor{gray}{\textbf{\textbf{Adminiſtration: VII.
                           Seidengaſſe 7\oindex{Seidengasse@\textbf{Seidengasse}, \emph{Straße (K.STR)}|pw}} (Jos. Eberle {\kaufmannsund} Co.\orgindex{Josef Eberle Stein-, Buch und Musikaliendruckerei@Josef Eberle Stein-, Buch und Musikaliendruckerei|pw})}}\pend
           
\pstart
           \centering{}\textcolor{gray}{\textbf{An der Schönen Blauen Donau\orgindex{der schoenen blauen Donau@An der schönen blauen Donau|pw}}}\pend
           
\pstart
           \centering{}\textcolor{gray}{\textbf{Chef-Redacteur: Dr. F.
                        Mamroth\pwindex{Mamroth, Fedor 21.02.1851 – 25.06.1907@\textsc{Mamroth, Fedor} (21.02.1851 – 25.06.1907), \emph{Journalist/Journalistin, Kritiker/Kritikerin}|pw}. – Redaction: IX.,
                        Berggaſſe 31\oindex{Berggasse@\textbf{Berggasse}, \emph{Straße (K.STR)}|pw}.}}\pend
           
\pstart
           \raggedleft{}\textcolor{gray}{\textbf{Wien\oindex{Wien@\textbf{Wien}, \emph{A.ADM2}|pw}, den}}{ }25. Juni \textcolor{gray}{\textbf{18}}89.\pend
           
\pstart\center{}Sehr geehrter Herr Doctor!\pend\vspace{0.5em}
\pstart
           Herr \textsc{Dr. Spitzer\pwindex{Spitzer, Sigmund 1813-04-01 – 1894-12-26@\textsc{Spitzer, Sigmund} (1813-04-01 – 1894-12-26), \emph{Mediziner/Medizinerin, Diplomat/Diplomatin}|pwu}}, der geſtern in Wien\oindex{Wien@\textbf{Wien}, \emph{A.ADM2}|pw} war, bittet Sie und mich, morgen,
                  Mittwoch, zu ihm nach Baden\oindex{Baden bei Wien@\textbf{Baden bei Wien}, \emph{P.PPLA3}|pw} zu kommen,
               und hat mich erſucht, Sie zu verſtändigen. Ich bitte Sie daher, mir freundlichſt
                  morgen im Laufe des Vormittags
               mittheilen zu {\pb}wollen, ob es Ihnen möglich iſt,
                  morgen{ }Nachmittag mit mir \label{K_L02641-1v}\edtext{hinauszufahren}{\lemma{\textnormal{\emph{hinauszufahren}}}\Cendnote{\textnormal{Zur 
                  Identifikation siehe Paul Goldmann an Arthur Schnitzler, 6. 8. 1889.}}}\label{K_L02641-1}, und im bejahenden Falle Herrn \textsc{Dr. Spitzer\pwindex{Spitzer, Alfred @\textsc{Spitzer, Alfred}, \emph{Kaufmann/Kauffrau}|pwu}} zu verſtändigen.\pend
           
\pstart
           Ich empfehle mich Ihnen mit beſten Grüßen {\\[\baselineskip]}Hochachtungsvoll {\\[\baselineskip]}Ihr
                  ergebe\textcolor{gray}{ner}{\\[\baselineskip]}\spacefill\mbox{Dr. Goldmann}\pend
           \leftskip=0em{}\selectlanguage{ngerman}\endnumbering\briefempfaengerindex{Schnitzler, Arthur@\textsc{Schnitzler, Arthur}!zzzGoldmann, Paul@\emph{von Paul Goldmann}!1889-06-251@{25. 6. 1889}|)be}\mylabel{L02641h}  \normalsize

\doendnotes{C}
\bigskip
\vfill

\clearpage

\footnotesize

\lohead{\textsc{register}}

% Definiere theindex-Environment komplett neu ohne reledmac
\makeatletter
\renewenvironment{theindex}{%
  \section*{\indexname}%
  \setlength{\parindent}{0pt}%
  \setlength{\parskip}{0pt plus 0.3pt}%
  \let\item\@idxitem
}{%
  \clearpage
}
\makeatother

\IfFileExists{\jobname-pw.ind}{\input{\jobname-pw.ind}}{}

\end{document}

      