%% latex-korrekturansicht-vorspann.tex
%% Vorspann für die Korrekturansicht.
%% Lädt die gemeinsame Datei latex-vorspann.tex mit gesetztem Schalter.

\newif\ifkorrekturansicht
\korrekturansichttrue

\input{../tex-inputs/latex-vorspann}


\section[Arthur Schnitzler an Richard Beer-Hofmann, 28. 7. 1904]{L01420 Arthur Schnitzler an Richard Beer-Hofmann, 28. 7. 1904}
\nopagebreak\mylabel{L01420v}
\rehead{ }\normalsize\beginnumbering\briefempfaengerindex{Beer-Hofmann, Richard@\textsc{Beer-Hofmann, Richard}!zzzSchnitzler, Arthur@\emph{von Arthur Schnitzler}!1904-07-281@{28. 7. 1904}|(be}
\toendnotes[C]{\smallbreak\pagebreak[2]}\Standort{YCGL, MSS 31.}
\physDesc{Brief, 1 Blatt, 4 Seiten, Umschlag, 1029 Zeichen
\newline{}Handschrift: 1) Bleistift, deutsche Kurrent\hspace{1em}2) Bleistift, lateinische Kurrent (\noindent{}Adresse)\hspace{1em}
\newline{}Versand: 1) Stempel: »\nobreak{}\oindex{I., Innere Stadt@\textbf{I., Innere Stadt}, \emph{A.ADM3}|pwk}Wien 1/1, 28. VII. 04, 12\nobreak{}«.   2) Stempel: »\nobreak{}\oindex{Bad Aussee@\textbf{Bad Aussee}, \emph{P.PPLA3}|pwk}Aussee in Steiermark, 29 7 04\nobreak{}«. }
\buchAbdrucke{\weitereDrucke{Arthur Schnitzler, Richard Beer-Hofmann: \emph{Briefwechsel 1891–1931}. Wien, Zürich: \emph{Europaverlag} 1992, S. 164–165.} }\toendnotes[C]{\smallbreak}\pstart{}{\pb}A. Schn. XIII Spöttelg. 7\oindex{Edmund-Weiss-Gasse 7@\textbf{Edmund-Weiß-Gasse 7}, \emph{Wohngebäude (K.WHS)}|pw}\pend{}{\bigskip}\pstart{}{\pb}Dr Richard Beer-Hofma{\geminationn}\pend{}\pstart{}Markt Aussee\oindex{Bad Aussee@\textbf{Bad Aussee}, \emph{P.PPLA3}|pw}\pend{}\pstart{}Villa Frühling\oindex{Villa Fruehling@\textbf{Villa Frühling}, \emph{Gebäude (K.GBD)}|pw}\pend{}{\bigskip}\vspace{1em}
\pstart
           \raggedleft{}{\pb}28. 7. 904\pend
           \vspace{0.5em}
\pstart
           lieber Richard – ich hatte mir wirklich ſchon eingebildet – es
               könnte ein Brief ſein – aber auch für den Theaterzettel mit Gruſs und Spaſs danke ich
               Ihnen herzlich. Wir waren etwa 14 Tage \introOben{}(\introOben{}mit Mama\pwindex{Schnitzler, Louise 1840-07-08 – 1911-09-09@\textsc{Schnitzler, Louise} (1840-07-08 – 1911-09-09)|pwv}\introOben{})\introOben{} in Reichenau\oindex{Reichenau an der Rax@\textbf{Reichenau an der Rax}, \emph{A.ADM3}|pw}, ſind
                  Samſtag zurück; es war wunderſchön, {\pb}ich war im Naßwald\oindex{Nasswald@\textbf{Nasswald}, \emph{P.PPL}|pw} und endlich ſogar auf der
               Rax\oindex{Rax@\textbf{Rax}, \emph{Berg (N.BRG)}|pw}, habe etliches gearbeitet, und was meine
               Geſundheit anbelangt, ſo iſt ſie eigentlich ko{\geminationm}t mir vor
               beſſer als \uline{vor} der Gelbſucht. Nun bleiben wir
               wahrſcheinlich (\introOben{}von\introOben{} Ausflüg\textcolor{gray}{en} von ein
               paar Tagen abge{\pb}ſehen) bis Ende Auguſt
               hier und fahren da{\geminationn} vielleicht auf 10–14 Tage nach Iſchl\oindex{Bad Ischl@\textbf{Bad Ischl}, \emph{P.PPL}|pw} bei welcher Gelegenheit ich Sie hoffentlich
               ſehen und – als letzter unter den {\dots} »Näheren« das Stück\pwindex{Graf von Charolais. Ein Trauerspiel@\emph{Der Graf von Charolais. Ein Trauerspiel}|pwv} hören werde, von dem mir
                  Salten\pwindex{Salten, Felix 06.09.1869 – 08.10.1945@\textsc{Salten, Felix} (06.09.1869 – 08.10.1945), \emph{Schriftsteller/Schriftstellerin, Journalist/Journalistin, Chefredakteur/Chefredakteurin}|pw} vorgeſtern höchſt begeiſtert ſprach.
               Ich denke, {\pb}Sie ſind bald fertig? –\pend
           
\pstart
           Schreiben Sie mir bald, we{\geminationn} auch nur eine Zeile, auch
               wie es Ihnen allen geht. –\pend
           
\pstart
           Mein Balkon iſt ein Luftkurort (heute übrigens beinah ein Sturmkurort)\pend
           
\pstart
           Wir grüßen Sie Beide\noindent{}Subjekt Beide\noindent{}Objekt.\pend
           
\pstart
           Von Herzen{\\[\baselineskip]}Ihr \spacefill\mbox{A.}\pend
           \leftskip=0em{}\selectlanguage{ngerman}\endnumbering\briefempfaengerindex{Beer-Hofmann, Richard@\textsc{Beer-Hofmann, Richard}!zzzSchnitzler, Arthur@\emph{von Arthur Schnitzler}!1904-07-281@{28. 7. 1904}|)be}\mylabel{L01420h}  \normalsize

\doendnotes{C}
\bigskip
\vfill

\clearpage

\footnotesize

\lohead{\textsc{register}}

% Definiere theindex-Environment komplett neu ohne reledmac
\makeatletter
\renewenvironment{theindex}{%
  \section*{\indexname}%
  \setlength{\parindent}{0pt}%
  \setlength{\parskip}{0pt plus 0.3pt}%
  \let\item\@idxitem
}{%
  \clearpage
}
\makeatother

\IfFileExists{\jobname-pw.ind}{\input{\jobname-pw.ind}}{}

\end{document}

      