\input{../tex-inputs/latex-pdf-vorspann}
\begin{center}
            \textcolor{red}{ENTWURF. ENTZIFFERUNG NOCH NICHT KORREKTURGELESEN}
                      \end{center}
            
               \section[Arthur Schnitzler an Richard Beer-Hofmann, 28. 7. 1904]{ Arthur Schnitzler an Richard Beer-Hofmann, 28. 7. 1904}\nopagebreak\mylabel{v}\rehead{ }\begin{ledgroupsized}[t]{13cm}\normalsize\beginnumbering\briefempfaengerindex{Beer-Hofmann, Richard@\textsc{Beer-Hofmann, Richard}!zzzSchnitzler, Arthur@\emph{von Arthur Schnitzler}!1904-07-281@{28. 7. 1904}|(be} \toendnotes[C]{\smallbreak\pagebreak[2]} \Standort{YCGL, MSS 31.}
\physDesc{Brief, 1 Blatt, 4 Seiten, Umschlag
\newline{}Handschrift: Bleistift, deutsche Kurrent\newline{}Versand: 1) Stempel: »\nobreak{}\oindex{I., Innere Stadt@\textbf{I., Innere Stadt}|pwk}Wien 1/1, 28. VII. 04, 12\nobreak{}«.  2) Stempel: »\nobreak{}\oindex{Bad Aussee@\textbf{Bad Aussee}|pwk}Aussee in Steiermark, 29 7 04\nobreak{}«. }\buchAbdrucke{\weitereDrucke{Arthur Schnitzler, Richard Beer-Hofmann: \emph{Briefwechsel 1891–1931}. Hg. Konstanze Fliedl. Wien, Zürich: \emph{Europaverlag} 1992, S. 164–165.} }\toendnotes[C]{\smallbreak}\pstart{}{\pb}\textsc{A. Schn. XIII Spöttelg. 7\oindex{Edmund-Weiss-Gasse@\textbf{Edmund-Weiß-Gasse}|pw}}\pend{}{\bigskip}\pstart{}{\pb}\textsc{Dr Richard Beer-Hofma{\geminationn}}\pend{}\pstart{}\textsc{Markt Aussee\oindex{Bad Aussee@\textbf{Bad Aussee}|pw}}\pend{}\pstart{}\textsc{Villa Frühling}\oindex{Villa Fruehling@\textbf{Villa Frühling}|pw}\pend{}{\bigskip}\pstart
           \raggedleft{}{\pb}28. 7. 904\pend
           \pstart
           lieber Richard – ich hatte mir wirklich ſchon eingebildet – es
               könnte ein Brief ſein – aber auch für den Theaterzettel mit Gruſs und Spaſs danke ich
               Ihnen herzlich. Wir waren etwa 14 Tage \introOben{}(\introOben{}mit Mama\pwindex{Schnitzler, Louise 08.07.1840 – 09.09.1911@\textsc{Schnitzler, Louise} (08.07.1840 – 09.09.1911)|pwv}\introOben{})\introOben{} in Reichenau\oindex{Reichenau an der Rax@\textbf{Reichenau an der Rax}|pw}, ſind Samſtag zurück; es war wunderſchön,
                  {\pb}ich war im Naßwald\oindex{Nasswald@\textbf{Nasswald}|pw} und endlich ſogar auf der Rax,
               habe etliches gearbeitet, und was meine Geſundheit anbelangt, ſo iſt ſie eigentlich
                  ko{\geminationm}t mir vor beſſer als \uline{vor} der Gelbſucht. Nun bleiben wir wahrſcheinlich (\introOben{}von\introOben{} Ausflüg\textcolor{gray}{en} von ein paar Tagen abge{\pb}ſehen) bis Ende Auguſt hier und fahren
                  da{\geminationn} vielleicht auf 10–14 Tage nach Iſchl\oindex{Bad Ischl@\textbf{Bad Ischl}|pw} bei welcher Gelegenheit ich Sie hoffentlich ſehen und –
               als letzter unter den {\dots} »Näheren« das Stück\pwindex{Beer-Hofmann, Richard 11.07.1866 – 26.09.1945@\textsc{Beer-Hofmann, Richard} (11.07.1866 – 26.09.1945), \emph{Schriftsteller}!Graf von Charolais. Ein Trauerspiel1904-12-23 – 1904-12-23@\strich\emph{Der Graf von Charolais. Ein Trauerspiel} {[}1904-12-23 – 1904-12-23{]}|pwv} hören werde, von dem mir Salten\pwindex{Salten, Felix 06.09.1869 – 08.10.1945@\textsc{Salten, Felix} (06.09.1869 – 08.10.1945), \emph{Schriftsteller, Journalist}|pw} vorgeſtern höchſt begeiſtert ſprach. Ich denke, {\pb}Sie ſind bald fertig? –\pend
           \pstart
           Schreiben Sie mir bald, we{\geminationn} auch nur eine Zeile, auch
               wie es Ihnen allen geht. –\pend
           \pstart
           Mein Balkon iſt ein Luftkurort (heute übrigens beinah ein Sturmkurort)\pend
           \pstart
           Wir grüßen Sie Beide\footnote{\noindent{}Subjekt} Beide\footnote{\noindent{}Objekt.}\pend
           \pstart
           Von Herzen{\\[\baselineskip]}Ihr \spacefill\mbox{A.}\pend
           \leftskip=0em{}\endnumbering\briefempfaengerindex{Beer-Hofmann, Richard@\textsc{Beer-Hofmann, Richard}!zzzSchnitzler, Arthur@\emph{von Arthur Schnitzler}!1904-07-281@{28. 7. 1904}|)be}\mylabel{h}\end{ledgroupsized}  \newcommand{\dateiname}{L01420}\newcommand{\titel}{Arthur Schnitzler an Richard Beer-Hofmann, 28. 7. 1904}\newcommand{\editorInnen}{Martin Anton Müller und Gerd-Hermann Susen}\input{../tex-inputs/latex-pdf-abspann}
      