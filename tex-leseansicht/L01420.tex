%% latex-leseansicht-vorspann.tex
%% Vorspann für die Leseansicht.
%% Lädt die gemeinsame Datei latex-vorspann.tex mit nicht gesetztem Schalter.

\newif\ifkorrekturansicht
\korrekturansichtfalse

\input{../tex-inputs/latex-vorspann}


         
         \renewcommand{\erwaehntePersonen}{Personen: Richard Beer-Hofmann, Felix Salten, Louise Schnitzler}
         \renewcommand{\erwaehnteOrte}{Orte: Bad Aussee, Bad Ischl, Edmund-Weiß-Gasse, I., Innere Stadt, Nasswald, Reichenau an der Rax, Villa Frühling, Wien}
         \renewcommand{\erwaehnteWerke}{Werke: Der Graf von Charolais. Ein Trauerspiel}
               \section[Arthur Schnitzler an Richard Beer-Hofmann, 28. 7. 1904]{ Arthur Schnitzler an Richard Beer-Hofmann, 28. 7. 1904}\nopagebreak\mylabel{v}\rehead{ }\begin{ledgroupsized}[t]{13cm}\normalsize\beginnumbering \toendnotes[C]{\smallbreak\pagebreak[2]} \Standort{YCGL, MSS 31.}
\physDesc{Brief, 1 Blatt, 4 Seiten, Umschlag
\newline{}Handschrift: 1) Bleistift, deutsche Kurrent\hspace{1em}2) Bleistift, lateinische Kurrent (\noindent{}Adresse)\hspace{1em}\newline{}Versand: 1) Stempel: »\nobreak{}\oindex{I., Innere Stadt@\textbf{I., Innere Stadt}|pwk}Wien 1/1, 28. VII. 04, 12\nobreak{}«.   2) Stempel: »\nobreak{}\oindex{Bad Aussee@\textbf{Bad Aussee}|pwk}Aussee in Steiermark, 29 7 04\nobreak{}«. }\buchAbdrucke{\weitereDrucke{Arthur Schnitzler, Richard Beer-Hofmann: \emph{Briefwechsel 1891–1931}. Hg. Konstanze Fliedl. Wien, Zürich: \emph{Europaverlag} 1992, S. 164–165.} }\toendnotes[C]{\smallbreak}\pstart{}{\pb}A. Schn. XIII Spöttelg. 7\oindex{Edmund-Weiss-Gasse@\textbf{Edmund-Weiß-Gasse}|pw}\pend{}{\bigskip}\pstart{}{\pb}Dr Richard Beer-Hofma{\geminationn}\pend{}\pstart{}Markt Aussee\oindex{Bad Aussee@\textbf{Bad Aussee}|pw}\pend{}\pstart{}Villa Frühling\oindex{Villa Fruehling@\textbf{Villa Frühling}|pw}\pend{}{\bigskip}\pstart
           \raggedleft{}{\pb}28. 7. 904\pend
           \pstart
           lieber Richard – ich hatte mir wirklich ſchon eingebildet – es
               könnte ein Brief ſein – aber auch für den Theaterzettel mit Gruſs und Spaſs danke ich
               Ihnen herzlich. Wir waren etwa 14 Tage \introOben{}(\introOben{}mit Mama\pwindex{Schnitzler, Louise 1840-07-08 – 1911-09-09@\textsc{Schnitzler, Louise} (1840-07-08 – 1911-09-09)|pwv}\introOben{})\introOben{} in Reichenau\oindex{Reichenau an der Rax@\textbf{Reichenau an der Rax}|pw}, ſind
                  Samſtag zurück; es war wunderſchön, {\pb}ich war im Naßwald\oindex{Nasswald@\textbf{Nasswald}|pw} und endlich ſogar auf der
                  Rax, habe etliches gearbeitet, und was meine
               Geſundheit anbelangt, ſo iſt ſie eigentlich ko{\geminationm}t mir vor
               beſſer als \uline{vor} der Gelbſucht. Nun bleiben wir
               wahrſcheinlich (\introOben{}von\introOben{} Ausflüg\textcolor{gray}{en} von ein
               paar Tagen abge{\pb}ſehen) bis Ende Auguſt
               hier und fahren da{\geminationn} vielleicht auf 10–14 Tage nach Iſchl\oindex{Bad Ischl@\textbf{Bad Ischl}|pw} bei welcher Gelegenheit ich Sie hoffentlich
               ſehen und – als letzter unter den {\dots} »Näheren« das Stück\pwindex{Beer-Hofmann, Richard 1866-07-11 – 1945-09-26@\textsc{Beer-Hofmann, Richard} (1866-07-11 – 1945-09-26), \emph{Schriftsteller}!Graf von Charolais. Ein Trauerspiel1904-12-23@\strich\emph{Der Graf von Charolais. Ein Trauerspiel} {[}1904-12-23{]}|pwv} hören werde, von dem mir
                  Salten\pwindex{Salten, Felix 06.09.1869 – 08.10.1945@\textsc{Salten, Felix} (06.09.1869 – 08.10.1945), \emph{Schriftsteller, Journalist}|pw} vorgeſtern höchſt begeiſtert ſprach.
               Ich denke, {\pb}Sie ſind bald fertig? –\pend
           \pstart
           Schreiben Sie mir bald, we{\geminationn} auch nur eine Zeile, auch
               wie es Ihnen allen geht. –\pend
           \pstart
           Mein Balkon iſt ein Luftkurort (heute übrigens beinah ein Sturmkurort)\pend
           \pstart
           Wir grüßen Sie Beide\footnote{\noindent{}Subjekt} Beide\footnote{\noindent{}Objekt.}\pend
           \pstart
           Von Herzen{\\[\baselineskip]}Ihr \spacefill\mbox{A.}\pend
           \leftskip=0em{}
         
         \endnumbering\mylabel{h}\end{ledgroupsized}  \newcommand{\dateiname}{L01420}\newcommand{\titel}{Arthur Schnitzler an Richard Beer-Hofmann, 28. 7. 1904}\newcommand{\editorInnen}{Martin Anton Müller und Gerd-Hermann Susen}%% latex-leseansicht-abspann.tex
%% Abspann für die Leseansicht.
%% Der Schalter \ifkorrekturansicht ist bereits durch den Vorspann gesetzt.

%% latex-abspann.tex
%% Gemeinsamer Abspann für Korrekturansicht und Leseansicht.
%% Setzt den Schalter \ifkorrekturansicht voraus (gesetzt in den
%% einbindenden Dateien latex-korrekturansicht-abspann.tex bzw.
%% latex-leseansicht-abspann.tex).
%% ---------------------------------------------------------------

\normalsize

% Das esempio-Environment wird nur in der Leseansicht benötigt
\ifkorrekturansicht\else
\newenvironment{esempio}[3]%
{
    \vspace{1.5ex}
    \rlap{\underline{#1}}
    \par
    \setlength{\parindent}{0cm}
    \nopagebreak
    \leftskip=#2cm
    \rightskip=#3cm
}
{
    \par
}
\fi

\doendnotes{C}
\bigskip
\vfill

\clearpage

\footnotesize

\ifkorrekturansicht
  \lohead{\textsc{register}}
\fi

% theindex-Environment neu definieren ohne reledmac
\makeatletter
\renewenvironment{theindex}{%
  \ifkorrekturansicht
    \section*{\indexname}%
  \else
    \subsubsection*{Index der erwähnten Entitäten}%
  \fi
  \setlength{\parindent}{0pt}%
  \setlength{\parskip}{0pt plus 0.3pt}%
  \let\item\@idxitem
}{%
  \ifkorrekturansicht\clearpage\fi
}
\makeatother

\IfFileExists{\jobname-pw.ind}{\input{\jobname-pw.ind}}{}

% Quellenangabe nur in der Leseansicht
\ifkorrekturansicht\else
% Fallback-Definitionen, falls die .tex-Datei \titel etc. nicht gesetzt hat
\providecommand{\titel}{}
\providecommand{\editorInnen}{}
\providecommand{\dateiname}{\jobname}

\vspace{3cm}

\vfill

\footnotesize
\textsc{Quelle}: \titel. Herausgegeben von {\editorInnen}. In: \emph{Arthur Schnitzler: Briefwechsel mit Autorinnen und Autoren}.
 Digitale Edition, https://schnitzler-briefe.acdh.oeaw.ac.at/{\dateiname}.html (Stand \today)
\fi

\end{document}


      