%% latex-leseansicht-vorspann.tex
%% Vorspann für die Leseansicht.
%% Lädt die gemeinsame Datei latex-vorspann.tex mit nicht gesetztem Schalter.

\newif\ifkorrekturansicht
\korrekturansichtfalse

\input{../tex-inputs/latex-vorspann}

\begin{center}
            \textcolor{red}{ENTWURF, NICHT FERTIG KORRIGIERT}
                      \end{center}
            
         
         \renewcommand{\erwaehntePersonen}{Personen: Ludwig Fulda, Karl Goldmann, Elisabeth von Heyking, Edmund Friedrich Gustav von Heyking, Theodore Rottenberg, Olga Schnitzler, Heinrich Schnitzler, Friedrich Zelnik, Ida d’Albert}
         \renewcommand{\erwaehnteInstitutionen}{Institutionen: Akademisch-dramatischer Verein München, Gebrüder Paetel Verlag, Reichstag}
         \renewcommand{\erwaehnteOrte}{Orte: Berlin, China, Dessauer Straße, Deutschland, Edmund-Weiß-Gasse, Italien, München, Riva del Garda, Südtirol, Trient, Wien, XVIII., Währing}
         \renewcommand{\erwaehnteWerke}{Werke: Briefe, die ihn nicht erreichten, Briefe, die ihn nicht erreichten..., Die Tragödie des Triumphes, Fink und Fliederbusch. Komödie in drei Akten, Reigen. Zehn Dialoge, Tägliche Rundschau}
               \section[ Paul Goldmann an Arthur Schnitzler, 27. 6. {[}1903{]}]{ Paul Goldmann an Arthur Schnitzler, 27. 6. {[}1903{]}}\nopagebreak\mylabel{v}\rehead{ }\begin{ledgroupsized}[t]{13cm}\normalsize\beginnumbering \toendnotes[C]{\smallbreak\pagebreak[2]} \Standort{DLA, A:Schnitzler, HS.NZ85.1.3173.}
\physDesc{Brief, 1 Blatt, 4 Seiten, 1569 Zeichen
\newline{}Handschrift: blaue Tinte, deutsche Kurrent
\newline{}Schnitzler: 1) mit Bleistift das Jahr »{[}1{]}903« und »\textsc{Nest\textcolor{gray}{l}}« vermerkt  2) mit rotem Buntstift eine einfache und eine doppelte
                                 Unterstreichung}\toendnotes[C]{\smallbreak}\pstart
           \noindent{}\raggedleft{}{\pb}\textcolor{gray}{\textbf{DESSAUERSTRASSE 19\oindex{Dessauer Strasse@\textbf{Dessauer Straße}|pw}}}\pend
           \pstart
           Berlin\oindex{Berlin@\textbf{Berlin}|pw}, 27. Juni\pend
           \pstart{}Mein lieber Freund,\pend\pstart
           Ich habe mit den \label{K_L03375-13v}\edtext{Wahlen}{\lemma{\textnormal{\emph{Wahlen}}}\Cendnote{\textnormal{Gemeint war die \emph{Reichstag}\orgindex{Reichstag@Reichstag|pwk}swahl am 16. 6. 1903.}}}\label{K_L03375-13h} ſchrecklich viel zu thun und kann daher erſt heut Dir und \textsc{Olga\pwindex{Schnitzler, Olga 17.01.1882 – 13.01.1970@\textsc{Schnitzler, Olga} (17.01.1882 – 13.01.1970), \emph{Schauspielerin, Sängerin}|pw}} für Eure lieben Grüße von \label{K_L03375-1v}\edtext{unterwegs}{\lemma{\textnormal{\emph{unterwegs}}}\Cendnote{\textnormal{siehe Paul Goldmann an Arthur Schnitzler, 2[2?]. 5. [1903]}}}\label{K_L03375-1h} vielmals danken. Alſo im Herbſt werdet Ihr Eure kleine \label{K_L03375-2v}\edtext{Wohnung}{\lemma{\textnormal{\emph{Wohnung}}}\Cendnote{\textnormal{Am 2. 9. 1903 zogen Olga\pwindex{Schnitzler, Olga 17.01.1882 – 13.01.1970@\textsc{Schnitzler, Olga} (17.01.1882 – 13.01.1970), \emph{Schauspielerin, Sängerin}|pwk} und Heinrich\pwindex{Schnitzler, Heinrich 09.08.1902 – 12.07.1982@\textsc{Schnitzler, Heinrich} (09.08.1902 – 12.07.1982), \emph{Regisseur, Schauspieler}|pwk} in eine Wohnung in der Spöttelgasse 7\oindex{Edmund-Weiss-Gasse@\textbf{Edmund-Weiß-Gasse}|pwk} (heute Edmund-Weiß-Gasse\oindex{Edmund-Weiss-Gasse@\textbf{Edmund-Weiß-Gasse}|pwk}) im 18. Wiener Gemeindebezirk\oindex{XVIII., Waehring@\textbf{XVIII., Währing}|pwk}. Zehn Tage später, am 2. 9. 1903, zog Schnitzler\pwindex{Schnitzler, Arthur 15.05.1862 – 21.10.1931@\textsc{Schnitzler, Arthur} (15.05.1862 – 21.10.1931), \emph{Schriftsteller, Mediziner}|pwk} ein.}}}\label{K_L03375-2h} beziehen? Sie muß ſehr
               traulich und ſehr reizend ſein, nach Deiner Schilderung, und ich hoffe ſehr, daß Ihr
               darin glückliche Tage und Jahre verleben werdet.\pend
           \pstart
           Die \label{K_L03375-3v}\edtext{»Komödie\pwindex{Schnitzler, Arthur 15.05.1862 – 21.10.1931@\textsc{Schnitzler, Arthur} (15.05.1862 – 21.10.1931), \emph{Schriftsteller, Mediziner}!Fink und Fliederbusch. Komoedie in drei Akten1917@\strich\emph{Fink und Fliederbusch. Komödie in drei Akten} {[}1917{]}|pwv}«}{\lemma{\textnormal{\emph{»Komödie«}}}\Cendnote{\textnormal{\emph{Flink und Fliederbusch}\pwindex{Schnitzler, Arthur 15.05.1862 – 21.10.1931@\textsc{Schnitzler, Arthur} (15.05.1862 – 21.10.1931), \emph{Schriftsteller, Mediziner}!Fink und Fliederbusch. Komoedie in drei Akten1917@\strich\emph{Fink und Fliederbusch. Komödie in drei Akten} {[}1917{]}|pwk}, vgl. Paul Goldmann an Arthur Schnitzler, 2[2?]. 5. [1903]. }}}\label{K_L03375-3h} wird
               hoffentlich noch feſte Geſtalt annehmen. {\pb}Wenn Dich
               gar nichts Anderes reizt, ſo denke an das »Geſchäft«, das mit einem luſtigen Stück
               heut zu machen wäre. Alle Theater würden danach greifen.\pend
           \pstart
           Der \label{K_L03375-4v}\edtext{\textsc{Goldmann\pwindex{Goldmann, Karl *~26.12.1865@\textsc{Goldmann, Karl} (*~26.12.1865), \emph{Schriftsteller, Journalist}|pw}} von der »Tragödie des Triumphes\pwindex{Goldmann, Karl *~26.12.1865@\textsc{Goldmann, Karl} (*~26.12.1865), \emph{Schriftsteller, Journalist}!Tragoedie des Triumphes1903@\strich\emph{Die Tragödie des Triumphes} {[}1903{]}|pw}}{\lemma{\textnormal{\emph{Goldmann … Triumphes}}}\Cendnote{\textnormal{\emph{Die Tragödie des Triumphes}\pwindex{Goldmann, Karl *~26.12.1865@\textsc{Goldmann, Karl} (*~26.12.1865), \emph{Schriftsteller, Journalist}!Tragoedie des Triumphes1903@\strich\emph{Die Tragödie des Triumphes} {[}1903{]}|pwk} von Karl Goldmann\pwindex{Goldmann, Karl *~26.12.1865@\textsc{Goldmann, Karl} (*~26.12.1865), \emph{Schriftsteller, Journalist}|pwk} wurde am 25. 6. 1903 gemeinsam mit einzelnen Szenen aus \emph{Reigen}\pwindex{Schnitzler, Arthur 15.05.1862 – 21.10.1931@\textsc{Schnitzler, Arthur} (15.05.1862 – 21.10.1931), \emph{Schriftsteller, Mediziner}!Reigen. Zehn Dialoge1900@\strich\emph{Reigen. Zehn Dialoge} {[}1900{]}|pwk} in München\oindex{Muenchen@\textbf{München}|pwk} in einer geschlossenen Aufführung vom \emph{Akademisch-dramatischen Verein}\orgindex{Akademisch-dramatischer Verein Muenchen@Akademisch-dramatischer Verein München|pwk} gegeben. Unmittelbare Folge
                  der Aufführung von \emph{Reigen}\pwindex{Schnitzler, Arthur 15.05.1862 – 21.10.1931@\textsc{Schnitzler, Arthur} (15.05.1862 – 21.10.1931), \emph{Schriftsteller, Mediziner}!Reigen. Zehn Dialoge1900@\strich\emph{Reigen. Zehn Dialoge} {[}1900{]}|pwk}-Szenen war die
                  Auflösung des seit 1890 bestehenden Vereins. Diese Briefstelle belegt,
               dass Schnitzler\pwindex{Schnitzler, Arthur 15.05.1862 – 21.10.1931@\textsc{Schnitzler, Arthur} (15.05.1862 – 21.10.1931), \emph{Schriftsteller, Mediziner}|pwk} bereits vorab von der Inszenierung
               wusste.}}}\label{K_L03375-4h}« bin nicht ich.
               Wie man Deinen »Reigen\pwindex{Schnitzler, Arthur 15.05.1862 – 21.10.1931@\textsc{Schnitzler, Arthur} (15.05.1862 – 21.10.1931), \emph{Schriftsteller, Mediziner}!Reigen. Zehn Dialoge1900@\strich\emph{Reigen. Zehn Dialoge} {[}1900{]}|pw}« aufführen will, –
               namentlich die \strikeout{\textcolor{gray}{r}}{ }\label{K_L03375-66v}\edtext{Gedankenſtriche}{\lemma{\textnormal{\emph{Gedankenſtriche}}}\Cendnote{\textnormal{Jede der zehn Szenen von \emph{Reigen}\pwindex{Schnitzler, Arthur 15.05.1862 – 21.10.1931@\textsc{Schnitzler, Arthur} (15.05.1862 – 21.10.1931), \emph{Schriftsteller, Mediziner}!Reigen. Zehn Dialoge1900@\strich\emph{Reigen. Zehn Dialoge} {[}1900{]}|pwk} besteht aus der Kommunikation vor und nach dem
                  Geschlechtsverkehr der Dialogpartner. Der Geschlechtsverkehr selbst ist in der
                  gedruckten Ausgabe mit Gedankenstrichen markiert.}}}\label{K_L03375-66h} – darauf bin ich ſehr
               neugierig. Das Buch\pwindex{Schnitzler, Arthur 15.05.1862 – 21.10.1931@\textsc{Schnitzler, Arthur} (15.05.1862 – 21.10.1931), \emph{Schriftsteller, Mediziner}!Reigen. Zehn Dialoge1900@\strich\emph{Reigen. Zehn Dialoge} {[}1900{]}|pwv} wird auch
               hier allgemein geleſen und erregt großes Entzücken.\pend
           \pstart
           Sommerpläne habe ich noch nicht. Ich ſehe mit Schrecken meinen Urlaub herankommen.
               Mir {\pb}grauſt davor, einen Entſchluß zu faſſen – Wohin
               ſoll ich gehen? Die Welt iſt leer, und Niemand wartet auf mich.\pend
           \pstart
           Vielleicht komme ich Anfang Auguſt nach Wien\oindex{Wien@\textbf{Wien}|pw} und fahre mit Dir nach \label{K_L03375-7v}\edtext{Südtirol\oindex{Suedtirol@\textbf{Südtirol}|pw}}{\lemma{\textnormal{\emph{Südtirol}}}\Cendnote{\textnormal{Goldmann\pwindex{Goldmann, Paul 31.01.1865 – 25.09.1935@\textsc{Goldmann, Paul} (31.01.1865 – 25.09.1935), \emph{Schriftsteller, Journalist}|pwk} war von 8. 8. 1903 bis 11. 8. 1903 in Wien\oindex{Wien@\textbf{Wien}|pwk} (vgl. Paul Goldmann an Arthur Schnitzler, 7. 8. [1903] und 11. 8. 1903). Schnitzler\pwindex{Schnitzler, Arthur 15.05.1862 – 21.10.1931@\textsc{Schnitzler, Arthur} (15.05.1862 – 21.10.1931), \emph{Schriftsteller, Mediziner}|pwk} traf er am 9. 8. 1903 und 11. 8. 1903. An diesem Tag reiste Goldmann\pwindex{Goldmann, Paul 31.01.1865 – 25.09.1935@\textsc{Goldmann, Paul} (31.01.1865 – 25.09.1935), \emph{Schriftsteller, Journalist}|pwk} nach Südtirol\oindex{Suedtirol@\textbf{Südtirol}|pwk} und Italien\oindex{Italien@\textbf{Italien}|pwk}, wo er mit Theodore Rottenberg\pwindex{Rottenberg, Theodore 1875-09-07 – 1945-04-05@\textsc{Rottenberg, Theodore} (1875-09-07 – 1945-04-05)|pwk} zusammentraf, mit der es zur Versöhnung gekommen
                  war (vgl. Paul Goldmann an Arthur Schnitzler, 11. 8. 1903). 
                  Danach trafen sich die drei zumindest am
                  18. 8. 1903 in Riva del Garda\oindex{Riva del Garda@\textbf{Riva del Garda}|pwk} (Paul Goldmann und Theodore Rottenberg an Arthur
               Schnitzler, 18. 8. [1903]),
                  am Folgetag dann wieder in Trient\oindex{Trient@\textbf{Trient}|pwk}, von 
                  wo sie nach einer Übernachtung zu dritt nach Lavarone\oindex{XXXX Ortsangabe fehlt|pwk} gingen.
                  Am 21. 8. 1903 trennte sich
                  Schnitzler\pwindex{Schnitzler, Arthur 15.05.1862 – 21.10.1931@\textsc{Schnitzler, Arthur} (15.05.1862 – 21.10.1931), \emph{Schriftsteller, Mediziner}|pwk} von den Beiden und fuhr über Trient\oindex{Trient@\textbf{Trient}|pwk} wieder nach Wien\oindex{Wien@\textbf{Wien}|pwk}.}}}\label{K_L03375-7h}.\pend
           \pstart
           Die \label{K_L03375-9v}\edtext{\textsc{Fulda\pwindex{Fulda, Ludwig 15.07.1862 – 30.03.1939@\textsc{Fulda, Ludwig} (15.07.1862 – 30.03.1939), \emph{Schriftsteller, Übersetzer}|pw}\pwindex{DAlbert, Ida 05.12.1869 – 1926-10-06@\textsc{d’Albert, Ida} (05.12.1869 – 1926-10-06)|pw}}’ſche Eheſcheidung}{\lemma{\textnormal{\emph{Fulda’ſche Eheſcheidung}}}\Cendnote{\textnormal{siehe Paul Goldmann an Arthur Schnitzler, 15. 6. [1903]}}}\label{K_L03375-9h} geht ihren Gang. Sie\pwindex{DAlbert, Ida 05.12.1869 – 1926-10-06@\textsc{d’Albert, Ida} (05.12.1869 – 1926-10-06)|pwv}
               hat ihren Mann\pwindex{Fulda, Ludwig 15.07.1862 – 30.03.1939@\textsc{Fulda, Ludwig} (15.07.1862 – 30.03.1939), \emph{Schriftsteller, Übersetzer}|pwv} ſo lange
               gequält, bis er es nicht mehr aushielt\strikeout{,} und auf
               Scheidung klagte. Es iſt eine große Dummheit von ihr, daß ſie es ſo weit kommen ließ;
                  {\pb}denn ſie\pwindex{DAlbert, Ida 05.12.1869 – 1926-10-06@\textsc{d’Albert, Ida} (05.12.1869 – 1926-10-06)|pwv} wird den Sturz von der ſocialen Höhe, auf der ſie \strikeout{\textcolor{gray}{ſte}h\textcolor{gray}{t,}} bisher ſtand, doch nicht vertragen.\pend
           \pstart
           Lies: \label{K_L03375-45v}\edtext{»Briefe, die ihn nicht erreichten\pwindex{Briefe, die ihn nicht erreichten1903@\emph{Briefe, die ihn nicht erreichten} {[}1903{]}|pw}«}{\lemma{\textnormal{\emph{»Briefe, … erreichten«}}}\Cendnote{\textnormal{[Elisabeth von Heyking\pwindex{Heyking, Elisabeth von 1861-12-10 – 1925-01-04@\textsc{Heyking, Elisabeth von} (1861-12-10 – 1925-01-04), \emph{Schriftstellerin}|pwk}]: \emph{Briefe, die ihn nicht erreichten}\pwindex{Briefe, die ihn nicht erreichten1903@\emph{Briefe, die ihn nicht erreichten} {[}1903{]}|pwk}. Berlin\oindex{Berlin@\textbf{Berlin}|pwk}: \emph{Gebrüder Paetel}\orgindex{Gebrueder Paetel Verlag@Gebrüder Paetel Verlag|pwk}{ }1903, Vorabdruck in der \emph{Täglichen Rundschau}\pwindex{?? Werk@Nicht ermittelte Verfasserinnen und Verfasser!Taegliche Rundschau1881 – 1933@\emph{Tägliche Rundschau} {[}1881 – 1933{]}|pwk}{ }1902. Eine Lektüre durch Schnitzler\pwindex{Schnitzler, Arthur 15.05.1862 – 21.10.1931@\textsc{Schnitzler, Arthur} (15.05.1862 – 21.10.1931), \emph{Schriftsteller, Mediziner}|pwk} ist
                  nicht belegt. Am 14. 10. 1925 sah er die gleichnamige Verfilmung\pwindex{Zelnik, Friedrich 1885-05-17 – 1950-11-29@\textsc{Zelnik, Friedrich} (1885-05-17 – 1950-11-29), \emph{Regisseur, Schauspieler, Produzent}!Briefe, die ihn nicht erreichten...1925@\strich\emph{Briefe, die ihn nicht erreichten...} {[}1925{]}|pwkv} von Friedrich
                     Zelnik\pwindex{Zelnik, Friedrich 1885-05-17 – 1950-11-29@\textsc{Zelnik, Friedrich} (1885-05-17 – 1950-11-29), \emph{Regisseur, Schauspieler, Produzent}|pwk}.}}}\label{K_L03375-45h}. Verfaſſerin iſt die Baronin \textsc{Heyking\pwindex{Heyking, Elisabeth von 1861-12-10 – 1925-01-04@\textsc{Heyking, Elisabeth von} (1861-12-10 – 1925-01-04), \emph{Schriftstellerin}|pw}}, die Frau des ehemaligen deutſch\oindex{Deutschland@\textbf{Deutschland}|pwv}en Geſandten\pwindex{Heyking, Edmund Friedrich Gustav von 1850-03-16 – 1915-06-15@\textsc{Heyking, Edmund Friedrich Gustav von} (1850-03-16 – 1915-06-15), \emph{Diplomat}|pwv} in China\oindex{China@\textbf{China}|pw}.\pend
           \pstart
           Grüße \textsc{Olga\pwindex{Schnitzler, Olga 17.01.1882 – 13.01.1970@\textsc{Schnitzler, Olga} (17.01.1882 – 13.01.1970), \emph{Schauspielerin, Sängerin}|pw}} vielmals und ſei auch Du herzlichſt gegrüßt von Deinem {\\[\baselineskip]}\spacefill\mbox{Paul Goldm}\pend
           \leftskip=0em{}
         
         \endnumbering\mylabel{h}\end{ledgroupsized}\begin{anhang}\end{anhang}\newcommand{\dateiname}{L03375}\newcommand{\titel}{Paul Goldmann an Arthur Schnitzler, 27. 6. [1903]}\newcommand{\editorInnen}{Martin Anton Müller und Laura Untner}%% latex-leseansicht-abspann.tex
%% Abspann für die Leseansicht.
%% Der Schalter \ifkorrekturansicht ist bereits durch den Vorspann gesetzt.

%% latex-abspann.tex
%% Gemeinsamer Abspann für Korrekturansicht und Leseansicht.
%% Setzt den Schalter \ifkorrekturansicht voraus (gesetzt in den
%% einbindenden Dateien latex-korrekturansicht-abspann.tex bzw.
%% latex-leseansicht-abspann.tex).
%% ---------------------------------------------------------------

\normalsize

% Das esempio-Environment wird nur in der Leseansicht benötigt
\ifkorrekturansicht\else
\newenvironment{esempio}[3]%
{
    \vspace{1.5ex}
    \rlap{\underline{#1}}
    \par
    \setlength{\parindent}{0cm}
    \nopagebreak
    \leftskip=#2cm
    \rightskip=#3cm
}
{
    \par
}
\fi

\doendnotes{C}
\bigskip
\vfill

\clearpage

\footnotesize

\ifkorrekturansicht
  \lohead{\textsc{register}}
\fi

% theindex-Environment neu definieren ohne reledmac
\makeatletter
\renewenvironment{theindex}{%
  \ifkorrekturansicht
    \section*{\indexname}%
  \else
    \subsubsection*{Index der erwähnten Entitäten}%
  \fi
  \setlength{\parindent}{0pt}%
  \setlength{\parskip}{0pt plus 0.3pt}%
  \let\item\@idxitem
}{%
  \ifkorrekturansicht\clearpage\fi
}
\makeatother

\IfFileExists{\jobname-pw.ind}{\input{\jobname-pw.ind}}{}

% Quellenangabe nur in der Leseansicht
\ifkorrekturansicht\else
% Fallback-Definitionen, falls die .tex-Datei \titel etc. nicht gesetzt hat
\providecommand{\titel}{}
\providecommand{\editorInnen}{}
\providecommand{\dateiname}{\jobname}

\vspace{3cm}

\vfill

\footnotesize
\textsc{Quelle}: \titel. Herausgegeben von {\editorInnen}. In: \emph{Arthur Schnitzler: Briefwechsel mit Autorinnen und Autoren}.
 Digitale Edition, https://schnitzler-briefe.acdh.oeaw.ac.at/{\dateiname}.html (Stand \today)
\fi

\end{document}


      