%% latex-korrekturansicht-vorspann.tex
%% Vorspann für die Korrekturansicht.
%% Lädt die gemeinsame Datei latex-vorspann.tex mit gesetztem Schalter.

\newif\ifkorrekturansicht
\korrekturansichttrue

\input{../tex-inputs/latex-vorspann}


\section[ Paul Goldmann an Arthur Schnitzler, 27. 6. {[}1903{]}]{L03375 Paul Goldmann an Arthur Schnitzler, 27. 6. {[}1903{]}}
\nopagebreak\mylabel{L03375v}
\rehead{ }\normalsize\beginnumbering\briefempfaengerindex{Schnitzler, Arthur@\textsc{Schnitzler, Arthur}!zzzGoldmann, Paul@\emph{von Paul Goldmann}!1903-06-271@{27. 6. {[}1903{]}}|(be}
\toendnotes[C]{\smallbreak\pagebreak[2]}\Standort{DLA, A:Schnitzler, HS.NZ85.1.3173.}
\physDesc{Brief, 1 Blatt, 4 Seiten, 1571 Zeichen
\newline{}Handschrift: blaue Tinte, deutsche Kurrent
\newline{}Schnitzler: 1) mit Bleistift das Jahr »903« und »\textsc{Nest\textcolor{gray}{l}}« vermerkt  2) mit rotem Buntstift eine einfache und eine doppelte Unterstreichung}\toendnotes[C]{\smallbreak}
\pstart
           \raggedleft{}{\pb}\textcolor{gray}{\textbf{DESSAUERSTRASSE 19\oindex{Dessauer Strasse@\textbf{Dessauer Straße}, \emph{Straße (K.STR)}|pw}}}\pend
           
\pstart
           Berlin\oindex{Berlin@\textbf{Berlin}, \emph{P.PPLC}|pw}, 27. Juni\pend
           
\pstart{}Mein lieber Freund,\pend\vspace{0.5em}
\pstart
           Ich habe mit den \label{K_L03375-1v}\edtext{Wahlen}{\lemma{\textnormal{\emph{Wahlen}}}\Cendnote{\textnormal{Gemeint war die Reichstagswahl\orgindex{Reichstag@Reichstag|pwkv} am 16. 6. 1903.}}}\label{K_L03375-1} ſchrecklich viel zu thun und kann
               daher erſt heut Dir und \textsc{Olga\pwindex{Schnitzler, Olga 17.01.1882 – 13.01.1970@\textsc{Schnitzler, Olga} (17.01.1882 – 13.01.1970), \emph{Schauspieler/Schauspielerin, Sänger/Sängerin}|pw}} für Eure lieben Grüße von \label{K_L03375-2v}\edtext{unterwegs}{\lemma{\textnormal{\emph{unterwegs}}}\Cendnote{\textnormal{Siehe Paul Goldmann an Arthur Schnitzler, 2[2?]. 5. [1903]. }}}\label{K_L03375-2} vielmals
               danken. Alſo im Herbſt werdet Ihr Eure kleine \label{K_L03375-3v}\edtext{Wohnung}{\lemma{\textnormal{\emph{Wohnung}}}\Cendnote{\textnormal{Am 2. 9. 1903 zogen Olga\pwindex{Schnitzler, Olga 17.01.1882 – 13.01.1970@\textsc{Schnitzler, Olga} (17.01.1882 – 13.01.1970), \emph{Schauspieler/Schauspielerin, Sänger/Sängerin}|pwk} und Heinrich\pwindex{Schnitzler, Heinrich 09.08.1902 – 12.07.1982@\textsc{Schnitzler, Heinrich} (09.08.1902 – 12.07.1982), \emph{Regisseur/Regisseurin, Schauspieler/Schauspielerin}|pwk} in eine Wohnung in der Spöttelgasse 7\oindex{Edmund-Weiss-Gasse 7@\textbf{Edmund-Weiß-Gasse 7}, \emph{Wohngebäude (K.WHS)}|pwk} (heute Edmund-Weiß-Gasse\oindex{Edmund-Weiss-Gasse 7@\textbf{Edmund-Weiß-Gasse 7}, \emph{Wohngebäude (K.WHS)}|pwk}) im 18. Wiener
                     Gemeindebezirk\oindex{XVIII., Waehring@\textbf{XVIII., Währing}, \emph{A.ADM3}|pwk}. Zehn Tage später, am 2. 9. 1903, zog Schnitzler ein.}}}\label{K_L03375-3} beziehen? Sie muß ſehr traulich und ſehr reizend
               ſein, nach Deiner Schilderung, und ich hoffe ſehr, daß Ihr darin glückliche Tage und
               Jahre verleben werdet.\pend
           
\pstart
           Die »\label{K_L03375-4v}\edtext{Komödie\pwindex{Fink und Fliederbusch. Komoedie in drei Akten@\emph{Fink und Fliederbusch. Komödie in drei Akten}|pwv}}{\lemma{\textnormal{\emph{Komödie}}}\Cendnote{\textnormal{\emph{Flink und Fliederbusch}\pwindex{Fink und Fliederbusch. Komoedie in drei Akten@\emph{Fink und Fliederbusch. Komödie in drei Akten}|pwk}, vgl. Paul Goldmann an Arthur Schnitzler, 2[2?]. 5. [1903].}}}\label{K_L03375-4}« wird
               hoffentlich noch feſte Geſtalt annehmen. {\pb}Wenn Dich
               gar nichts Anderes reizt, ſo denke an das »Geſchäft«, das mit einem luſtigen Stück
               heut zu machen wäre. Alle Theater würden danach greifen.\pend
           
\pstart
           Der \label{K_L03375-5v}\edtext{\textsc{Goldmann\pwindex{Goldmann, Karl *~26.12.1865@\textsc{Goldmann, Karl} (*~26.12.1865), \emph{Schriftsteller/Schriftstellerin, Journalist/Journalistin}|pw}} von der »Tragödie des Triumphes\pwindex{Tragoedie des Triumphes@\emph{Die Tragödie des Triumphes}|pw}}{\lemma{\textnormal{\emph{Goldmann … Triumphes}}}\Cendnote{\textnormal{\emph{Die Tragödie des Triumphes}\pwindex{Tragoedie des Triumphes@\emph{Die Tragödie des Triumphes}|pwk} von Karl Goldmann\pwindex{Goldmann, Karl *~26.12.1865@\textsc{Goldmann, Karl} (*~26.12.1865), \emph{Schriftsteller/Schriftstellerin, Journalist/Journalistin}|pwk} wurde am 25. 6. 1903 gemeinsam mit einzelnen Szenen aus dem \emph{Reigen}\pwindex{Reigen. Zehn Dialoge@\emph{Reigen. Zehn Dialoge}|pwk} in München\oindex{Muenchen@\textbf{München}, \emph{P.PPLA}|pwk} in einer geschlossenen Aufführung des \emph{Akademisch-dramatischen Vereins}\orgindex{Akademisch-dramatischer Verein Muenchen@Akademisch-dramatischer Verein München|pwk} gegeben. Unmittelbare
                  Folge der Aufführung der \emph{Reigen}\pwindex{Reigen. Zehn Dialoge@\emph{Reigen. Zehn Dialoge}|pwk}-Szenen war
                  die Auflösung des seit 1890 bestehenden Vereins\orgindex{Akademisch-dramatischer Verein Muenchen@Akademisch-dramatischer Verein München|pwkv}. Diese Briefstelle belegt, dass
                     Schnitzler bereits vorab von der
                  Inszenierung wusste.}}}\label{K_L03375-5}« bin nicht ich. Wie man Deinen »Reigen\pwindex{Reigen. Zehn Dialoge@\emph{Reigen. Zehn Dialoge}|pw}« aufführen will, – namentlich die \strikeout{\textcolor{gray}{r}}{ }\label{K_L03375-6v}\edtext{Gedankenſtriche}{\lemma{\textnormal{\emph{Gedankenſtriche}}}\Cendnote{\textnormal{Jede der zehn Szenen im \emph{Reigen}\pwindex{Reigen. Zehn Dialoge@\emph{Reigen. Zehn Dialoge}|pwk} besteht aus Gesprächen vor und nach dem Geschlechtsverkehr der
                  Dialogpartnerinnen und -partner. Der Geschlechtsverkehr selbst ist in der
                  gedruckten Ausgabe mit Gedankenstrichen markiert.}}}\label{K_L03375-6} – darauf bin ich ſehr
               neugierig. Das Buch\pwindex{Reigen. Zehn Dialoge@\emph{Reigen. Zehn Dialoge}|pwv} wird auch
               hier allgemein geleſen und erregt großes Entzücken.\pend
           
\pstart
           Sommerpläne habe ich noch nicht. Ich ſehe mit Schrecken meinen Urlaub herankommen.
               Mir {\pb}grauſt davor, einen Entſchluß zu faſſen. Wohin
               ſoll ich gehen? Die Welt iſt leer, und Niemand wartet auf mich.\pend
           
\pstart
           Vielleicht komme ich Anfang Auguſt nach Wien\oindex{Wien@\textbf{Wien}, \emph{A.ADM2}|pw} und fahre mit Dir nach \label{K_L03375-7v}\edtext{Südtirol\oindex{Suedtirol@\textbf{Südtirol}, \emph{A.ADM2}|pw}}{\lemma{\textnormal{\emph{Südtirol}}}\Cendnote{\textnormal{Goldmann\pwindex{Goldmann, Paul 31.01.1865 – 25.09.1935@\textsc{Goldmann, Paul} (31.01.1865 – 25.09.1935), \emph{Schriftsteller/Schriftstellerin, Journalist/Journalistin}|pwk} war von 8. 8. 1903 bis 11. 8. 1903 in Wien\oindex{Wien@\textbf{Wien}, \emph{A.ADM2}|pwk} (vgl. Paul Goldmann an Arthur Schnitzler, 7. 8. [1903] und 11. 8. 1903). Schnitzler traf er am 9. 8. 1903 und 11. 8. 1903. Danach reiste Goldmann\pwindex{Goldmann, Paul 31.01.1865 – 25.09.1935@\textsc{Goldmann, Paul} (31.01.1865 – 25.09.1935), \emph{Schriftsteller/Schriftstellerin, Journalist/Journalistin}|pwk} nach Südtirol\oindex{Suedtirol@\textbf{Südtirol}, \emph{A.ADM2}|pwk} und Italien\oindex{Italien@\textbf{Italien}, \emph{A.PCLI}|pwk}, wo er mit Theodore Rottenberg\pwindex{Rottenberg, Theodore 1875-09-07 – 1945-04-05@\textsc{Rottenberg, Theodore} (1875-09-07 – 1945-04-05)|pwk} zusammentraf, mit der es zur
                  Versöhnung gekommen war (vgl. Paul Goldmann an Arthur Schnitzler, 11. 8. 1903). In Folge trafen sich die drei zumindest am 18. 8. 1903 in Riva del Garda\oindex{Riva del Garda@\textbf{Riva del Garda}, \emph{P.PPLA3}|pwk} (vgl. Paul Goldmann und Theodore Rottenberg an Arthur
               Schnitzler, 18. 8. [1903]), am Folgetag dann wieder in Trient\oindex{Trient@\textbf{Trient}, \emph{P.PPLA}|pwk}, von wo sie nach einer Übernachtung zu dritt nach Lavarone\oindex{Lavarone@\textbf{Lavarone}, \emph{A.ADM3}|pwk} gingen. Am 21. 8. 1903 trennte
                  sich Schnitzler von den beiden und fuhr über
                     Trient\oindex{Trient@\textbf{Trient}, \emph{P.PPLA}|pwk} wieder nach Wien\oindex{Wien@\textbf{Wien}, \emph{A.ADM2}|pwk}.}}}\label{K_L03375-7}.\pend
           
\pstart
           Die \label{K_L03375-8v}\edtext{\textsc{Fulda\pwindex{Fulda, Ludwig 15.07.1862 – 30.03.1939@\textsc{Fulda, Ludwig} (15.07.1862 – 30.03.1939), \emph{Schriftsteller/Schriftstellerin, Übersetzer/Übersetzerin}|pw}\pwindex{DAlbert, Ida 05.12.1869 – 1926-10-06@\textsc{d’Albert, Ida} (05.12.1869 – 1926-10-06), \emph{Schauspieler/Schauspielerin}|pw}}’ſche Eheſcheidung}{\lemma{\textnormal{\emph{Fulda’ſche Eheſcheidung}}}\Cendnote{\textnormal{Siehe Paul Goldmann an Arthur Schnitzler, 15. 6. [1903]. }}}\label{K_L03375-8} geht ihren
               Gang. Sie\pwindex{DAlbert, Ida 05.12.1869 – 1926-10-06@\textsc{d’Albert, Ida} (05.12.1869 – 1926-10-06), \emph{Schauspieler/Schauspielerin}|pwv} hat ihren Mann\pwindex{Fulda, Ludwig 15.07.1862 – 30.03.1939@\textsc{Fulda, Ludwig} (15.07.1862 – 30.03.1939), \emph{Schriftsteller/Schriftstellerin, Übersetzer/Übersetzerin}|pwv} ſo lange gequält, bis er
               es nicht mehr aushielt\strikeout{,} und auf Scheidung klagte. Es
               iſt eine große Dummheit von ihr, daß ſie es ſo weit kommen ließ; {\pb}denn ſie\pwindex{DAlbert, Ida 05.12.1869 – 1926-10-06@\textsc{d’Albert, Ida} (05.12.1869 – 1926-10-06), \emph{Schauspieler/Schauspielerin}|pwv} wird den Sturz von der ſocialen Höhe, auf der ſie \strikeout{\textcolor{gray}{ſte}h\textcolor{gray}{t,}} bisher ſtand, doch nicht vertragen.\pend
           
\pstart
           Lies: »\label{K_L03375-9v}\edtext{Briefe, die ihn nicht erreichten\pwindex{Briefe, die ihn nicht erreichten@\emph{Briefe, die ihn nicht erreichten}|pw}}{\lemma{\textnormal{\emph{Briefe, … erreichten}}}\Cendnote{\textnormal{[Elisabeth von Heyking\pwindex{Heyking, Elisabeth von 1861-12-10 – 1925-01-04@\textsc{Heyking, Elisabeth von} (1861-12-10 – 1925-01-04), \emph{Schriftsteller/Schriftstellerin}|pwk}]: \emph{Briefe, die ihn nicht erreichten}\pwindex{Briefe, die ihn nicht erreichten@\emph{Briefe, die ihn nicht erreichten}|pwk}. Berlin\oindex{Berlin@\textbf{Berlin}, \emph{P.PPLC}|pwk}: \emph{Gebrüder Paetel}\orgindex{Gebrueder Paetel Verlag@Gebrüder Paetel Verlag|pwk}{ }1903, Vorabdruck in der \emph{Täglichen Rundschau}\pwindex{Taegliche Rundschau@\emph{Tägliche Rundschau}|pwk}{ }1902. Eine Lektüre durch Schnitzler ist
                  nicht belegt. Am 14. 10. 1925 sah er die gleichnamige Verfilmung\pwindex{Briefe, die ihn nicht erreichten...@\emph{Briefe, die ihn nicht erreichten...}|pwkv} von Friedrich
                     Zelnik\pwindex{Zelnik, Friedrich 1885-05-17 – 1950-11-29@\textsc{Zelnik, Friedrich} (1885-05-17 – 1950-11-29), \emph{Regisseur/Regisseurin, Schauspieler/Schauspielerin, Produzent/Produzentin}|pwk}.}}}\label{K_L03375-9}«. Verfaſſerin iſt die Baronin \textsc{Heyking\pwindex{Heyking, Elisabeth von 1861-12-10 – 1925-01-04@\textsc{Heyking, Elisabeth von} (1861-12-10 – 1925-01-04), \emph{Schriftsteller/Schriftstellerin}|pw}}, die Frau des ehemaligen deutſch\oindex{Deutschland@\textbf{Deutschland}, \emph{A.PCLI}|pwv}en Geſandten\pwindex{Heyking, Edmund Friedrich Gustav von 1850-03-16 – 1915-06-15@\textsc{Heyking, Edmund Friedrich Gustav von} (1850-03-16 – 1915-06-15), \emph{Diplomat/Diplomatin}|pwv} in China\oindex{China@\textbf{China}, \emph{A.PCLI}|pw}.\pend
           
\pstart
           Grüße \textsc{Olga\pwindex{Schnitzler, Olga 17.01.1882 – 13.01.1970@\textsc{Schnitzler, Olga} (17.01.1882 – 13.01.1970), \emph{Schauspieler/Schauspielerin, Sänger/Sängerin}|pw}} vielmals und ſei auch Du herzlichſt gegrüßt von Deinem {\\[\baselineskip]}\spacefill\mbox{Paul Goldm}\pend
           \leftskip=0em{}\selectlanguage{ngerman}\endnumbering\briefempfaengerindex{Schnitzler, Arthur@\textsc{Schnitzler, Arthur}!zzzGoldmann, Paul@\emph{von Paul Goldmann}!1903-06-271@{27. 6. {[}1903{]}}|)be}\mylabel{L03375h}  \normalsize

\doendnotes{C}
\bigskip
\vfill

\clearpage

\footnotesize

\lohead{\textsc{register}}

% Definiere theindex-Environment komplett neu ohne reledmac
\makeatletter
\renewenvironment{theindex}{%
  \section*{\indexname}%
  \setlength{\parindent}{0pt}%
  \setlength{\parskip}{0pt plus 0.3pt}%
  \let\item\@idxitem
}{%
  \clearpage
}
\makeatother

\IfFileExists{\jobname-pw.ind}{\input{\jobname-pw.ind}}{}

\end{document}

      