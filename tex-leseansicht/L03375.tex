%% latex-leseansicht-vorspann.tex
%% Vorspann für die Leseansicht.
%% Lädt die gemeinsame Datei latex-vorspann.tex mit nicht gesetztem Schalter.

\newif\ifkorrekturansicht
\korrekturansichtfalse

\input{../tex-inputs/latex-vorspann}


\section[ Paul Goldmann an Arthur Schnitzler, 27. 6. [1903]]{L03375 Paul Goldmann an Arthur Schnitzler,  27. 6. [1903]}
\nopagebreak\mylabel{L03375v}
\rehead{ }\normalsize\beginnumbering\briefempfaengerindex{Schnitzler, Arthur@\textsc{Schnitzler, Arthur}!zzzGoldmann, Paul@\emph{von Paul Goldmann}!1903-06-271@{27. 6. [1903]}|(be}
\toendnotes[C]{\smallbreak\pagebreak[2]}
\correspDesc{Versand  durch Paul Goldmann am 27. 6. [1903] in Berlin
\newline{}Erhalt  durch Arthur Schnitzler im Zeitraum [28. 6. 1903
                  – 2. 7. 1903?] in Wien}\toendnotes[C]{\smallbreak}
\Standort{DLA, A:Schnitzler, HS.NZ85.1.3173.}
\physDesc{Brief, 1 Blatt, 4 Seiten, 1571 Zeichen
\newline{}Handschrift: blaue Tinte, deutsche Kurrent
\newline{}Schnitzler: 1) mit Bleistift das Jahr »903« und »\textsc{Nest\textcolor{gray}{l}}« vermerkt  2) mit rotem Buntstift eine einfache und eine doppelte Unterstreichung}\toendnotes[C]{\smallbreak}
\pstart
           \raggedleft{}{\pb}\textcolor{gray}{\textbf{DESSAUERSTRASSE 19\oindex{Dessauer Straße@\textbf{Dessauer Straße}, \emph{Straße}|pw}}}\pend
           
\pstart
           Berlin\oindex{Berlin@\textbf{Berlin}, \emph{Hauptstadt}|pw}, 27. Juni\pend
           
\pstart{}Mein lieber Freund,\pend\vspace{0.5em}
\pstart
           Ich habe mit den \label{K_L03375-1v}\edtext{Wahlen}{\lemma{\textnormal{\emph{Wahlen}}}\Cendnote{\textnormal{Gemeint war die Reichstagswahl\orgindex{Reichstag@Reichstag|pwkv} am 16. 6. 1903.}}}\label{K_L03375-1}{ }ſchrecklich viel zu thun und kann
               daher erſt heut Dir und \textsc{Olga\pwindex{Schnitzler, Olga 17.\,1.\,1882 Wien – 13.\,1.\,1970 Lugano@\textsc{Schnitzler, Olga} (17.\,1.\,1882 Wien – 13.\,1.\,1970 Lugano), \emph{Schauspielerin, Sängerin}|pw}} für Eure lieben Grüße von \label{K_L03375-2v}\edtext{unterwegs}{\lemma{\textnormal{\emph{unterwegs}}}\Cendnote{\textnormal{Siehe XXXX Auszeichnungsfehler: Dokument L03373 nicht gefunden. }}}\label{K_L03375-2} vielmals
               danken. Alſo im Herbſt werdet Ihr Eure kleine \label{K_L03375-3v}\edtext{Wohnung}{\lemma{\textnormal{\emph{Wohnung}}}\Cendnote{\textnormal{Am 2. 9. 1903 zogen Olga\pwindex{Schnitzler, Olga 17.\,1.\,1882 Wien – 13.\,1.\,1970 Lugano@\textsc{Schnitzler, Olga} (17.\,1.\,1882 Wien – 13.\,1.\,1970 Lugano), \emph{Schauspielerin, Sängerin}|pwk} und Heinrich\pwindex{Schnitzler, Heinrich 9.\,8.\,1902 Hinterbrühl – 12.\,7.\,1982 Wien@\textsc{Schnitzler, Heinrich} (9.\,8.\,1902 Hinterbrühl – 12.\,7.\,1982 Wien), \emph{Regisseur, Schauspieler}|pwk} in eine Wohnung in der Spöttelgasse 7\oindex{Wien@\textbf{Wien}!XVIII., Währing@\textbf{XVIII., Währing}!Edmund-Weiß-Gasse 7@\textbf{Edmund-Weiß-Gasse 7}, \emph{Wohngebäude}|pwk} (heute Edmund-Weiß-Gasse\oindex{Wien@\textbf{Wien}!XVIII., Währing@\textbf{XVIII., Währing}!Edmund-Weiß-Gasse 7@\textbf{Edmund-Weiß-Gasse 7}, \emph{Wohngebäude}|pwk}) im 18. Wiener
                     Gemeindebezirk\oindex{XVIII., Währing@\textbf{XVIII., Währing}, \emph{Verwaltungsgebiet}|pwk}. Zehn Tage später, am 2. 9. 1903, zog Schnitzler ein.}}}\label{K_L03375-3} beziehen? Sie muß{ }ſehr traulich und{ }ſehr reizend{ }ſein, nach Deiner Schilderung, und ich hoffe{ }ſehr, daß Ihr darin glückliche Tage und
               Jahre verleben werdet.\pend
           
\pstart
           Die »\label{K_L03375-4v}\edtext{Komödie\pwindex{Schnitzler, Arthur 15.\,5.\,1862 Wien – 21.\,10.\,1931 ebd.@\textsc{Schnitzler, Arthur} (15.\,5.\,1862 Wien – 21.\,10.\,1931 ebd.), \emph{Schriftsteller, Mediziner}!Fink und Fliederbusch. Komödie in drei Akten@\strich\emph{Fink und Fliederbusch. Komödie in drei Akten}|pwv}}{\lemma{\textnormal{\emph{Komödie}}}\Cendnote{\textnormal{\emph{Flink und Fliederbusch}\pwindex{Schnitzler, Arthur 15.\,5.\,1862 Wien – 21.\,10.\,1931 ebd.@\textsc{Schnitzler, Arthur} (15.\,5.\,1862 Wien – 21.\,10.\,1931 ebd.), \emph{Schriftsteller, Mediziner}!Fink und Fliederbusch. Komödie in drei Akten@\strich\emph{Fink und Fliederbusch. Komödie in drei Akten}|pwk}, vgl. XXXX Auszeichnungsfehler: Dokument L03373 nicht gefunden.}}}\label{K_L03375-4}« wird
               hoffentlich noch feſte Geſtalt annehmen. {\pb}Wenn Dich
               gar nichts Anderes reizt,{ }ſo denke an das »Geſchäft«, das mit einem luſtigen Stück
               heut zu machen wäre. Alle Theater würden danach greifen.\pend
           
\pstart
           Der \label{K_L03375-5v}\edtext{\textsc{Goldmann\pwindex{Goldmann, Karl *~26.\,12.\,1865 Bezdan@\textsc{Goldmann, Karl} (*~26.\,12.\,1865 Bezdan), \emph{Schriftsteller, Journalist}|pw}} von der »Tragödie des Triumphes\pwindex{Goldmann, Karl *~26.\,12.\,1865 Bezdan@\textsc{Goldmann, Karl} (*~26.\,12.\,1865 Bezdan), \emph{Schriftsteller, Journalist}!Tragödie des Triumphes@\strich\emph{Die Tragödie des Triumphes}|pw}}{\lemma{\textnormal{\emph{Goldmann … Triumphes}}}\Cendnote{\textnormal{\emph{Die Tragödie des Triumphes}\pwindex{Goldmann, Karl *~26.\,12.\,1865 Bezdan@\textsc{Goldmann, Karl} (*~26.\,12.\,1865 Bezdan), \emph{Schriftsteller, Journalist}!Tragödie des Triumphes@\strich\emph{Die Tragödie des Triumphes}|pwk} von Karl Goldmann\pwindex{Goldmann, Karl *~26.\,12.\,1865 Bezdan@\textsc{Goldmann, Karl} (*~26.\,12.\,1865 Bezdan), \emph{Schriftsteller, Journalist}|pwk} wurde am 25. 6. 1903 gemeinsam mit einzelnen Szenen aus dem \emph{Reigen}\pwindex{Schnitzler, Arthur 15.\,5.\,1862 Wien – 21.\,10.\,1931 ebd.@\textsc{Schnitzler, Arthur} (15.\,5.\,1862 Wien – 21.\,10.\,1931 ebd.), \emph{Schriftsteller, Mediziner}!Reigen. Zehn Dialoge@\strich\emph{Reigen. Zehn Dialoge}|pwk} in München\oindex{München@\textbf{München}|pwk} in einer geschlossenen Aufführung des \emph{Akademisch-dramatischen Vereins}\orgindex{Akademisch-dramatischer Verein München@Akademisch-dramatischer Verein München|pwk} gegeben. Unmittelbare
                  Folge der Aufführung der \emph{Reigen}\pwindex{Schnitzler, Arthur 15.\,5.\,1862 Wien – 21.\,10.\,1931 ebd.@\textsc{Schnitzler, Arthur} (15.\,5.\,1862 Wien – 21.\,10.\,1931 ebd.), \emph{Schriftsteller, Mediziner}!Reigen. Zehn Dialoge@\strich\emph{Reigen. Zehn Dialoge}|pwk}-Szenen war
                  die Auflösung des seit 1890 bestehenden Vereins\orgindex{Akademisch-dramatischer Verein München@Akademisch-dramatischer Verein München|pwkv}. Diese Briefstelle belegt, dass
                     Schnitzler bereits vorab von der
                  Inszenierung wusste.}}}\label{K_L03375-5}« bin nicht ich. Wie man Deinen »Reigen\pwindex{Schnitzler, Arthur 15.\,5.\,1862 Wien – 21.\,10.\,1931 ebd.@\textsc{Schnitzler, Arthur} (15.\,5.\,1862 Wien – 21.\,10.\,1931 ebd.), \emph{Schriftsteller, Mediziner}!Reigen. Zehn Dialoge@\strich\emph{Reigen. Zehn Dialoge}|pw}« aufführen will, – namentlich die \strikeout{\textcolor{gray}{r}}{ }\label{K_L03375-6v}\edtext{Gedankenſtriche}{\lemma{\textnormal{\emph{Gedankenstriche}}}\Cendnote{\textnormal{Jede der zehn Szenen im \emph{Reigen}\pwindex{Schnitzler, Arthur 15.\,5.\,1862 Wien – 21.\,10.\,1931 ebd.@\textsc{Schnitzler, Arthur} (15.\,5.\,1862 Wien – 21.\,10.\,1931 ebd.), \emph{Schriftsteller, Mediziner}!Reigen. Zehn Dialoge@\strich\emph{Reigen. Zehn Dialoge}|pwk} besteht aus Gesprächen vor und nach dem Geschlechtsverkehr der
                  Dialogpartnerinnen und -partner. Der Geschlechtsverkehr selbst ist in der
                  gedruckten Ausgabe mit Gedankenstrichen markiert.}}}\label{K_L03375-6} – darauf bin ich{ }ſehr
               neugierig. Das Buch\pwindex{Schnitzler, Arthur 15.\,5.\,1862 Wien – 21.\,10.\,1931 ebd.@\textsc{Schnitzler, Arthur} (15.\,5.\,1862 Wien – 21.\,10.\,1931 ebd.), \emph{Schriftsteller, Mediziner}!Reigen. Zehn Dialoge@\strich\emph{Reigen. Zehn Dialoge}|pwv} wird auch
               hier allgemein geleſen und erregt großes Entzücken.\pend
           
\pstart
           Sommerpläne habe ich noch nicht. Ich{ }ſehe mit Schrecken meinen Urlaub herankommen.
               Mir {\pb}grauſt davor, einen Entſchluß zu faſſen. Wohin{ }ſoll ich gehen? Die Welt iſt leer, und Niemand wartet auf mich.\pend
           
\pstart
           Vielleicht komme ich Anfang Auguſt nach Wien\oindex{Wien@\textbf{Wien}, \emph{Verwaltungsgebiet}|pw} und fahre mit Dir nach \label{K_L03375-7v}\edtext{Südtirol\oindex{Südtirol@\textbf{Südtirol}, \emph{Verwaltungsgebiet}|pw}}{\lemma{\textnormal{\emph{Südtirol}}}\Cendnote{\textnormal{Goldmann\pwindex{Goldmann, Paul 31.\,1.\,1865 Breslau – 25.\,9.\,1935 Wien@\textsc{Goldmann, Paul} (31.\,1.\,1865 Breslau – 25.\,9.\,1935 Wien), \emph{Schriftsteller, Journalist}|pwk} war von 8. 8. 1903 bis 11. 8. 1903 in Wien\oindex{Wien@\textbf{Wien}, \emph{Verwaltungsgebiet}|pwk} (vgl. XXXX Auszeichnungsfehler: Dokument L03382 nicht gefunden und XXXX Auszeichnungsfehler: Dokument L03383 nicht gefunden). Schnitzler traf er am 9. 8. 1903 und 11. 8. 1903. Danach reiste Goldmann\pwindex{Goldmann, Paul 31.\,1.\,1865 Breslau – 25.\,9.\,1935 Wien@\textsc{Goldmann, Paul} (31.\,1.\,1865 Breslau – 25.\,9.\,1935 Wien), \emph{Schriftsteller, Journalist}|pwk} nach Südtirol\oindex{Südtirol@\textbf{Südtirol}, \emph{Verwaltungsgebiet}|pwk} und Italien\oindex{Italien@\textbf{Italien}|pwk}, wo er mit Theodore Rottenberg\pwindex{Rottenberg, Theodore 7.\,9.\,1875 – 5.\,4.\,1945 Limburg an der Lahn@\textsc{Rottenberg, Theodore} (7.\,9.\,1875 – 5.\,4.\,1945 Limburg an der Lahn)|pwk} zusammentraf, mit der es zur
                  Versöhnung gekommen war (vgl. XXXX Auszeichnungsfehler: Dokument L03383 nicht gefunden). In Folge trafen sich die drei zumindest am 18. 8. 1903 in Riva del Garda\oindex{Riva del Garda@\textbf{Riva del Garda}, \emph{Hauptstadt}|pwk} (vgl. XXXX Auszeichnungsfehler: Dokument L03384 nicht gefunden), am Folgetag dann wieder in Trient\oindex{Trient@\textbf{Trient}|pwk}, von wo sie nach einer Übernachtung zu dritt nach Lavarone\oindex{Lavarone@\textbf{Lavarone}, \emph{Verwaltungsgebiet}|pwk} gingen. Am 21. 8. 1903 trennte
                  sich Schnitzler von den beiden und fuhr über
                     Trient\oindex{Trient@\textbf{Trient}|pwk} wieder nach Wien\oindex{Wien@\textbf{Wien}, \emph{Verwaltungsgebiet}|pwk}.}}}\label{K_L03375-7}.\pend
           
\pstart
           Die \label{K_L03375-8v}\edtext{\textsc{Fulda\pwindex{Fulda, Ludwig 15.\,7.\,1862 Frankfurt am Main – 30.\,3.\,1939 Berlin@\textsc{Fulda, Ludwig} (15.\,7.\,1862 Frankfurt am Main – 30.\,3.\,1939 Berlin), \emph{Schriftsteller, Übersetzer}|pw}\pwindex{d’Albert, Ida 5.\,12.\,1869 Wien – 6.\,10.\,1926 Berlin@\textsc{d’Albert, Ida} (5.\,12.\,1869 Wien – 6.\,10.\,1926 Berlin), \emph{Schauspielerin}|pw}}’ſche Eheſcheidung}{\lemma{\textnormal{\emph{Fulda’sche Ehescheidung}}}\Cendnote{\textnormal{Siehe XXXX Auszeichnungsfehler: Dokument L03374 nicht gefunden. }}}\label{K_L03375-8} geht ihren
               Gang. Sie\pwindex{d’Albert, Ida 5.\,12.\,1869 Wien – 6.\,10.\,1926 Berlin@\textsc{d’Albert, Ida} (5.\,12.\,1869 Wien – 6.\,10.\,1926 Berlin), \emph{Schauspielerin}|pwv} hat ihren Mann\pwindex{Fulda, Ludwig 15.\,7.\,1862 Frankfurt am Main – 30.\,3.\,1939 Berlin@\textsc{Fulda, Ludwig} (15.\,7.\,1862 Frankfurt am Main – 30.\,3.\,1939 Berlin), \emph{Schriftsteller, Übersetzer}|pwv}{ }ſo lange gequält, bis er
               es nicht mehr aushielt\strikeout{,} und auf Scheidung klagte. Es
               iſt eine große Dummheit von ihr, daß{ }ſie es{ }ſo weit kommen ließ; {\pb}denn ſie\pwindex{d’Albert, Ida 5.\,12.\,1869 Wien – 6.\,10.\,1926 Berlin@\textsc{d’Albert, Ida} (5.\,12.\,1869 Wien – 6.\,10.\,1926 Berlin), \emph{Schauspielerin}|pwv} wird den Sturz von der{ }ſocialen Höhe, auf der{ }ſie \strikeout{\textcolor{gray}{ſte}h\textcolor{gray}{t,}} bisher{ }ſtand, doch nicht vertragen.\pend
           
\pstart
           Lies: »\label{K_L03375-9v}\edtext{Briefe, die ihn nicht erreichten\pwindex{Briefe, die ihn nicht erreichten@\emph{Briefe, die ihn nicht erreichten}|pw}}{\lemma{\textnormal{\emph{Briefe, … erreichten}}}\Cendnote{\textnormal{[Elisabeth von Heyking\pwindex{Heyking, Elisabeth von 10.\,12.\,1861 Karlsruhe – 4.\,1.\,1925 Berlin@\textsc{Heyking, Elisabeth von} (10.\,12.\,1861 Karlsruhe – 4.\,1.\,1925 Berlin), \emph{Schriftstellerin}|pwk}]: \emph{Briefe, die ihn nicht erreichten}\pwindex{Briefe, die ihn nicht erreichten@\emph{Briefe, die ihn nicht erreichten}|pwk}. Berlin\oindex{Berlin@\textbf{Berlin}, \emph{Hauptstadt}|pwk}: \emph{Gebrüder Paetel}\orgindex{Gebrüder Paetel Verlag@Gebrüder Paetel Verlag|pwk}{ }1903, Vorabdruck in der \emph{Täglichen Rundschau}\pwindex{Tägliche Rundschau@\emph{Tägliche Rundschau}|pwk}{ }1902. Eine Lektüre durch Schnitzler ist
                  nicht belegt. Am 14. 10. 1925 sah er die gleichnamige Verfilmung\pwindex{Zelnik, Friedrich 17.\,5.\,1885 Czernowitz – 29.\,11.\,1950 London@\textsc{Zelnik, Friedrich} (17.\,5.\,1885 Czernowitz – 29.\,11.\,1950 London), \emph{Regisseur, Schauspieler, Produzent}!Briefe, die ihn nicht erreichten...@\strich\emph{Briefe, die ihn nicht erreichten...}|pwkv}\pwindex{Zelnik, Friedrich 17.\,5.\,1885 Czernowitz – 29.\,11.\,1950 London@\textsc{Zelnik, Friedrich} (17.\,5.\,1885 Czernowitz – 29.\,11.\,1950 London), \emph{Regisseur, Schauspieler, Produzent}!Briefe, die ihn nicht erreichten...@\strich\emph{Briefe, die ihn nicht erreichten...}|pwkv} von Friedrich
                     Zelnik\pwindex{Zelnik, Friedrich 17.\,5.\,1885 Czernowitz – 29.\,11.\,1950 London@\textsc{Zelnik, Friedrich} (17.\,5.\,1885 Czernowitz – 29.\,11.\,1950 London), \emph{Regisseur, Schauspieler, Produzent}|pwk}.}}}\label{K_L03375-9}«. Verfaſſerin iſt die Baronin \textsc{Heyking\pwindex{Heyking, Elisabeth von 10.\,12.\,1861 Karlsruhe – 4.\,1.\,1925 Berlin@\textsc{Heyking, Elisabeth von} (10.\,12.\,1861 Karlsruhe – 4.\,1.\,1925 Berlin), \emph{Schriftstellerin}|pw}}, die Frau des ehemaligen deutſch\oindex{Deutschland@\textbf{Deutschland}|pwv}en Geſandten\pwindex{Heyking, Edmund Friedrich Gustav von 16.\,3.\,1850 Lettland – 15.\,6.\,1915 Berlin@\textsc{Heyking, Edmund Friedrich Gustav von} (16.\,3.\,1850 Lettland – 15.\,6.\,1915 Berlin), \emph{Diplomat}|pwv} in China\oindex{China@\textbf{China}|pw}.\pend
           
\pstart
           Grüße \textsc{Olga\pwindex{Schnitzler, Olga 17.\,1.\,1882 Wien – 13.\,1.\,1970 Lugano@\textsc{Schnitzler, Olga} (17.\,1.\,1882 Wien – 13.\,1.\,1970 Lugano), \emph{Schauspielerin, Sängerin}|pw}} vielmals und{ }ſei auch Du herzlichſt gegrüßt von Deinem {\\[\baselineskip]}\spacefill\mbox{Paul Goldm}\pend
           \leftskip=0em{}\selectlanguage{ngerman}\endnumbering\briefempfaengerindex{Schnitzler, Arthur@\textsc{Schnitzler, Arthur}!zzzGoldmann, Paul@\emph{von Paul Goldmann}!1903-06-271@{27. 6. [1903]}|)be}\mylabel{L03375h}  \newcommand{\dateiname}{L03375}\newcommand{\titel}{Paul Goldmann an Arthur Schnitzler, 27. 6. [1903]}\newcommand{\editorInnen}{Martin Anton Müller und Laura Untner}%% latex-leseansicht-abspann.tex
%% Abspann für die Leseansicht.
%% Der Schalter \ifkorrekturansicht ist bereits durch den Vorspann gesetzt.

%% latex-abspann.tex
%% Gemeinsamer Abspann für Korrekturansicht und Leseansicht.
%% Setzt den Schalter \ifkorrekturansicht voraus (gesetzt in den
%% einbindenden Dateien latex-korrekturansicht-abspann.tex bzw.
%% latex-leseansicht-abspann.tex).
%% ---------------------------------------------------------------

\normalsize

% Das esempio-Environment wird nur in der Leseansicht benötigt
\ifkorrekturansicht\else
\newenvironment{esempio}[3]%
{
    \vspace{1.5ex}
    \rlap{\underline{#1}}
    \par
    \setlength{\parindent}{0cm}
    \nopagebreak
    \leftskip=#2cm
    \rightskip=#3cm
}
{
    \par
}
\fi

\doendnotes{C}
\bigskip
\vfill

\clearpage

\footnotesize

\ifkorrekturansicht
  \lohead{\textsc{register}}
\fi

% theindex-Environment neu definieren ohne reledmac
\makeatletter
\renewenvironment{theindex}{%
  \ifkorrekturansicht
    \section*{\indexname}%
  \else
    \subsubsection*{Index der erwähnten Entitäten}%
  \fi
  \setlength{\parindent}{0pt}%
  \setlength{\parskip}{0pt plus 0.3pt}%
  \let\item\@idxitem
}{%
  \ifkorrekturansicht\clearpage\fi
}
\makeatother

\IfFileExists{\jobname-pw.ind}{\input{\jobname-pw.ind}}{}

% Quellenangabe nur in der Leseansicht
\ifkorrekturansicht\else
% Fallback-Definitionen, falls die .tex-Datei \titel etc. nicht gesetzt hat
\providecommand{\titel}{}
\providecommand{\editorInnen}{}
\providecommand{\dateiname}{\jobname}

\vspace{3cm}

\vfill

\footnotesize
\textsc{Quelle}: \titel. Herausgegeben von {\editorInnen}. In: \emph{Arthur Schnitzler: Briefwechsel mit Autorinnen und Autoren}.
 Digitale Edition, https://schnitzler-briefe.acdh.oeaw.ac.at/{\dateiname}.html (Stand \today)
\fi

\end{document}


