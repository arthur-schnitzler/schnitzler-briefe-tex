%% latex-leseansicht-vorspann.tex
%% Vorspann für die Leseansicht.
%% Lädt die gemeinsame Datei latex-vorspann.tex mit nicht gesetztem Schalter.

\newif\ifkorrekturansicht
\korrekturansichtfalse

\input{../tex-inputs/latex-vorspann}


\section[Arthur und Olga Schnitzler an Richard und Paula Beer-Hofmann, 27. 8. 1913]{L02150 Arthur und Olga Schnitzler an Richard und Paula Beer-Hofmann, 27. 8. 1913}
\nopagebreak\mylabel{L02150v}
\rehead{ }\normalsize\beginnumbering\briefempfaengerindex{Beer-Hofmann, Paula@\textsc{Beer-Hofmann, Paula}!zzzSchnitzler, Olga@\emph{von Olga Schnitzler}!1913-08-271@{27. 8. 1913}|(be}\briefempfaengerindex{Beer-Hofmann, Paula@\textsc{Beer-Hofmann, Paula}!zzzSchnitzler, Arthur@\emph{von Arthur Schnitzler}!1913-08-271@{27. 8. 1913}|(be}\briefempfaengerindex{Beer-Hofmann, Richard@\textsc{Beer-Hofmann, Richard}!zzzSchnitzler, Olga@\emph{von Olga Schnitzler}!1913-08-271@{27. 8. 1913}|(be}\briefempfaengerindex{Beer-Hofmann, Richard@\textsc{Beer-Hofmann, Richard}!zzzSchnitzler, Arthur@\emph{von Arthur Schnitzler}!1913-08-271@{27. 8. 1913}|(be}
\toendnotes[C]{\smallbreak\pagebreak[2]}
\correspDesc{Versand  durch Arthur Schnitzler, Olga Schnitzler am 27. 8. 1913 in Maloja
\newline{}Erhalt  durch Richard Beer-Hofmann, Paula Beer-Hofmann am 29. 8. 1913 in Lido}\toendnotes[C]{\smallbreak}
\Standort{YCGL, MSS 31.}
\physDesc{Bildpostkarte, 243 Zeichen
\newline{}Handschrift Arthur Schnitzler: Bleistift, deutsche Kurrent
\newline{}Handschrift Olga Schnitzler: Bleistift, lateinische Kurrent
\newline{}Versand: 1) Stempel: »\nobreak{}\oindex{Maloja Palace@\textbf{Maloja Palace}, \emph{Hotel}|pwk}Maloja Palace Maloja
                                          (Engadin{[}){]} 1811m, Direction Schlagen{[}hauf{]}f\pwindex{Schlagenhauff, Hugo *~23.\,8.\,1876 Schwäbisch Hall@\textsc{Schlagenhauff, Hugo} (*~23.\,8.\,1876 Schwäbisch Hall), \emph{Hoteldirektor}|pw}\nobreak{}«.   2) Stempel: »\nobreak{}\oindex{Maloja@\textbf{Maloja}, \emph{Verwaltungsgebiet}|pwk}Maloja (Graubünden), 27. VIII. 13\nobreak{}«.  3) Stempel: »\nobreak{}\oindex{Venedig@\textbf{Venedig}|pwk}\textcolor{gray}{Venezia} Centro, 29. \textcolor{gray}{8}. 13\nobreak{}«. }\pstart{}{\pb}Hrn u Frau\pend{}\pstart{}\textsc{Rich. Beerhofmann}\pend{}\pstart{}aus Wien\oindex{Wien@\textbf{Wien}, \emph{Verwaltungsgebiet}|pw}, \pend{}\pstart{}d. Z. \textsc{Lido}\oindex{Lido@\textbf{Lido}|pw}\pend{}\pstart{}\textsc{Venedig\oindex{Venedig@\textbf{Venedig}|pw}}\pend{}\pstart{}\textsc{Grd Hotel des bains\oindex{Grand Hotel des Bains@\textbf{Grand Hotel des Bains}, \emph{Hotel}|pw}}\pend{}{\bigskip}
\pstart
           \noindent{}\centering{}{\pb}\textcolor{gray}{\textbf{Maloja\oindex{Maloja@\textbf{Maloja}, \emph{Verwaltungsgebiet}|pw}}}\pend
           \vspace{1em}
\pstart
           \noindent{}{\pb}Herzlichſte Grüße. Wir{ }ſind vom Engadin\oindex{Engadin@\textbf{Engadin}, \emph{Tal}|pw} ganz entzückt. Der angenehmen \textsc{Lido}\oindex{Lido@\textbf{Lido}|pw} Stunden gern gedenkend\pend
           \pstart Ihr \spacefill\mbox{A.}\pend{}
\pstart
           Maloja\oindex{Maloja@\textbf{Maloja}, \emph{Verwaltungsgebiet}|pw}{ }27/8 91\textcolor{gray}{3}\pend
           \selectlanguage{ngerman}\vspace{1em}
\pstart
           \noindent{}{[}hs. Schnitzler:{]} Hier ist die Welt besonders schön! Alles Herzliche,\pend
           \pstart Ihre \spacefill\mbox{Olga.}\pend{}\selectlanguage{ngerman}\endnumbering\briefempfaengerindex{Beer-Hofmann, Paula@\textsc{Beer-Hofmann, Paula}!zzzSchnitzler, Olga@\emph{von Olga Schnitzler}!1913-08-271@{27. 8. 1913}|)be}\briefempfaengerindex{Beer-Hofmann, Paula@\textsc{Beer-Hofmann, Paula}!zzzSchnitzler, Arthur@\emph{von Arthur Schnitzler}!1913-08-271@{27. 8. 1913}|)be}\briefempfaengerindex{Beer-Hofmann, Richard@\textsc{Beer-Hofmann, Richard}!zzzSchnitzler, Olga@\emph{von Olga Schnitzler}!1913-08-271@{27. 8. 1913}|)be}\briefempfaengerindex{Beer-Hofmann, Richard@\textsc{Beer-Hofmann, Richard}!zzzSchnitzler, Arthur@\emph{von Arthur Schnitzler}!1913-08-271@{27. 8. 1913}|)be}\mylabel{L02150h}  \newcommand{\dateiname}{L02150}\newcommand{\titel}{Arthur und Olga Schnitzler an Richard und Paula Beer-Hofmann, 27. 8. 1913}\newcommand{\editorInnen}{Martin Anton Müller und Gerd-Hermann Susen}%% latex-leseansicht-abspann.tex
%% Abspann für die Leseansicht.
%% Der Schalter \ifkorrekturansicht ist bereits durch den Vorspann gesetzt.

%% latex-abspann.tex
%% Gemeinsamer Abspann für Korrekturansicht und Leseansicht.
%% Setzt den Schalter \ifkorrekturansicht voraus (gesetzt in den
%% einbindenden Dateien latex-korrekturansicht-abspann.tex bzw.
%% latex-leseansicht-abspann.tex).
%% ---------------------------------------------------------------

\normalsize

% Das esempio-Environment wird nur in der Leseansicht benötigt
\ifkorrekturansicht\else
\newenvironment{esempio}[3]%
{
    \vspace{1.5ex}
    \rlap{\underline{#1}}
    \par
    \setlength{\parindent}{0cm}
    \nopagebreak
    \leftskip=#2cm
    \rightskip=#3cm
}
{
    \par
}
\fi

\doendnotes{C}
\bigskip
\vfill

\clearpage

\footnotesize

\ifkorrekturansicht
  \lohead{\textsc{register}}
\fi

% theindex-Environment neu definieren ohne reledmac
\makeatletter
\renewenvironment{theindex}{%
  \ifkorrekturansicht
    \section*{\indexname}%
  \else
    \subsubsection*{Index der erwähnten Entitäten}%
  \fi
  \setlength{\parindent}{0pt}%
  \setlength{\parskip}{0pt plus 0.3pt}%
  \let\item\@idxitem
}{%
  \ifkorrekturansicht\clearpage\fi
}
\makeatother

\IfFileExists{\jobname-pw.ind}{\input{\jobname-pw.ind}}{}

% Quellenangabe nur in der Leseansicht
\ifkorrekturansicht\else
% Fallback-Definitionen, falls die .tex-Datei \titel etc. nicht gesetzt hat
\providecommand{\titel}{}
\providecommand{\editorInnen}{}
\providecommand{\dateiname}{\jobname}

\vspace{3cm}

\vfill

\footnotesize
\textsc{Quelle}: \titel. Herausgegeben von {\editorInnen}. In: \emph{Arthur Schnitzler: Briefwechsel mit Autorinnen und Autoren}.
 Digitale Edition, https://schnitzler-briefe.acdh.oeaw.ac.at/{\dateiname}.html (Stand \today)
\fi

\end{document}


