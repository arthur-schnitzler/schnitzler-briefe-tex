%% latex-korrekturansicht-vorspann.tex
%% Vorspann für die Korrekturansicht.
%% Lädt die gemeinsame Datei latex-vorspann.tex mit gesetztem Schalter.

\newif\ifkorrekturansicht
\korrekturansichttrue

\input{../tex-inputs/latex-vorspann}


\section[Arthur Schnitzler an Felix Braun, 19. 4. 1918]{L02284 Arthur Schnitzler an Felix Braun, 19. 4. 1918}
\nopagebreak\mylabel{L02284v}
\rehead{ }\normalsize\beginnumbering\briefempfaengerindex{Braun, Felix@\textsc{Braun, Felix}!zzzSchnitzler, Arthur@\emph{von Arthur Schnitzler}!1918-04-191@{19. 4. 1918}|(be}
\toendnotes[C]{\smallbreak\pagebreak[2]}\Standort{Wienbibliothek im Rathaus, H.I.N.-198045.}
\physDesc{Brief, 1 Blatt, 2 Seiten, 1307 Zeichen
\newline{}Schreibmaschine
\newline{}Handschrift: schwarze Tinte, deutsche Kurrent (\noindent{}Ergänzungen, Unterstreichungen und Unterschrift)}\Standort{DLA, A:Schnitzler, HS.1985.1.447.}
\physDesc{Brief, Durchschlag2 Blätter, 2 Seiten, 1307 Zeichen
\newline{}Schreibmaschine
\newline{}Handschrift: roter Buntstift, lateinische Kurrent (\noindent{}Beschriftung »Fel Braun«)}\toendnotes[C]{\smallbreak}
\pstart
           
\pstart
           {\pb}\textcolor{gray}{\textbf{Dr. Arthur Schnitzler}}\pend
           
\pstart
           \raggedleft{}19. 4. 1918.\pend
           \pend
           
\pstart
           \textcolor{gray}{\textbf{Wien XVIII. Sternwartestrasse 71\oindex{Sternwartestrasse 71@\textbf{Sternwartestraße 71}, \emph{Wohngebäude (K.WHS)}|pw}}}\pend
           
\pstart\center{}Verehrtester Herr Felix Braun.\pend\vspace{0.5em}
\pstart
           Aus meinem Telegramm entnehmen Sie, dass meine Angelegenheit mit Fischer\orgindex{S. Fischer Verlag@S. Fischer Verlag|pw} noch immer in Schwebe ist. Es wäre immerhin doch sehr
               möglich, dass er sich das nötige Papier \uline{sowohl} für
               meine alten \uline{als} für meine neuen Sachen verschafft;
               und bei meinen persönlichen und geschäftlichen Beziehungen zu ihm schiene es mir in
               keinem Sinne richtig, anderswo anzuknüpfen, ehe ganz zwingende Gründe hiezu
               vorliegen. Darum ist es mir auch nicht möglich Ihnen \introOben{}irgend\introOben{}welche \uline{Vorschläge} zu machen, sondern ich
               will mich vorläufig damit begnügen, \strikeout{un} einige
               Anfragen an Sie zu stellen, durch deren rasche Beantwortung Sie mich sehr
               verpflichten würden.\pend
           
\pstart
           Innerhalb welcher Zeit und in wie viel Auflagen (zu tausend Exemplaren) könnte der
               Verlag Müller\orgindex{Georg Mueller Verlag@Georg Müller Verlag|pw} eine neue Novelle\pwindex{Casanovas Heimfahrt@\emph{Casanovas Heimfahrt}|pwv} (Ausdehnung {\pb}etwa wie »Badearzt Gräsler\pwindex{Doktor Graesler, Badearzt@\emph{Doktor Gräsler, Badearzt}|pw}« drucken und erscheinen lassen und zwar unter der Bedingung
               vorheriger Bezahlung, \introOben{}von\introOben{} 25 {\%} des Ladenpreises\substVorne{}\textsuperscript{,}\substDazwischen{}?\substHinten{}\strikeout{und 2.} Ferner müsste ich mir das Recht vorbehalten,
               diese Novelle\pwindex{Casanovas Heimfahrt@\emph{Casanovas Heimfahrt}|pwv} in einer
               Neuauflage meiner bei S. Fischer\orgindex{S. Fischer Verlag@S. Fischer Verlag|pw} erscheinenden
                  gesammelten Werke\pwindex{Gesammelte Werke@\emph{Gesammelte Werke}|pw}{ }\introOben{}(\introOben{}frühestens 1922\introOben{})\introOben{} auf\strikeout{zu}nehmen zu
               dürfen.\pend
           
\pstart
           Gleiches gälte für mein neues Stück\pwindex{Schwestern oder Casanova in Spa. Lustspiel in Versen@\emph{Die Schwestern oder Casanova in Spa. Lustspiel in Versen}|pwv}, das jedenfalls erst im Spätherbst oder
                  Winter erscheinen sollte.\pend
           
\pstart
           Es wird mir angenehm sein, recht bald Ihre Meinung zu vernehmen.\pend
           
\pstart
           Mit verbindlichen Grüssen{\\[\baselineskip]}Ihr sehr ergebener{\\[\baselineskip]}\spacefill\mbox{{[}hs.:{]} Arthur Schnitzler}\pend
           \leftskip=0em{}\selectlanguage{ngerman}\endnumbering\briefempfaengerindex{Braun, Felix@\textsc{Braun, Felix}!zzzSchnitzler, Arthur@\emph{von Arthur Schnitzler}!1918-04-191@{19. 4. 1918}|)be}\mylabel{L02284h}  \normalsize

\doendnotes{C}
\bigskip
\vfill

\clearpage

\footnotesize

\lohead{\textsc{register}}

% Definiere theindex-Environment komplett neu ohne reledmac
\makeatletter
\renewenvironment{theindex}{%
  \section*{\indexname}%
  \setlength{\parindent}{0pt}%
  \setlength{\parskip}{0pt plus 0.3pt}%
  \let\item\@idxitem
}{%
  \clearpage
}
\makeatother

\IfFileExists{\jobname-pw.ind}{\input{\jobname-pw.ind}}{}

\end{document}

      