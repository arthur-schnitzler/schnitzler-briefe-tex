%% latex-korrekturansicht-vorspann.tex
%% Vorspann für die Korrekturansicht.
%% Lädt die gemeinsame Datei latex-vorspann.tex mit gesetztem Schalter.

\newif\ifkorrekturansicht
\korrekturansichttrue

\input{../tex-inputs/latex-vorspann}


\section[ Paul Goldmann an Arthur Schnitzler, 5. 10. {[}1897{]}]{L02828 Paul Goldmann an Arthur Schnitzler, 5. 10. {[}1897{]}}
\nopagebreak\mylabel{L02828v}
\rehead{ }\normalsize\beginnumbering\briefempfaengerindex{Schnitzler, Arthur@\textsc{Schnitzler, Arthur}!zzzGoldmann, Paul@\emph{von Paul Goldmann}!1897-10-051@{5. 10. {[}1897{]}}|(be}
\toendnotes[C]{\smallbreak\pagebreak[2]}\Standort{DLA, A:Schnitzler, HS.NZ85.1.3167.}
\physDesc{Brief, 1 Blatt, 1 Seite, 254 Zeichen
\newline{}Handschrift: blaue Tinte, deutsche Kurrent
\newline{}Schnitzler: mit rotem Buntstift eine Unterstreichung }\toendnotes[C]{\smallbreak}
\pstart
           {\pb}\textcolor{gray}{\textbf{\textbf{Frankfurter Zeitung\orgindex{Frankfurter Zeitung@Frankfurter Zeitung|pw}}}}\pend
           
\pstart
           \textcolor{gray}{\textbf{(\begin{otherlanguage}{french}Gazette de Francfort\end{otherlanguage}\orgindex{Frankfurter Zeitung@Frankfurter Zeitung|pw}).}}\pend
           
\pstart
           \textcolor{gray}{\textbf{\textbf{\begin{otherlanguage}{french}Fondateur M.\end{otherlanguage}{ }L. Sonnemann\pwindex{Sonnemann, Leopold 1831-10-29 – 1909-10-30@\textsc{Sonnemann, Leopold} (1831-10-29 – 1909-10-30), \emph{Journalist/Journalistin, Herausgeber/Herausgeberin}|pw}.}}}\pend
           
\pstart
           \begin{otherlanguage}{french}\textcolor{gray}{\textbf{Journal politique, financier,}}\end{otherlanguage}\pend
           
\pstart
           \begin{otherlanguage}{french}\textcolor{gray}{\textbf{commercial et littéraire.}}\end{otherlanguage}\pend
           
\pstart
           \begin{otherlanguage}{french}\textcolor{gray}{\textbf{\textbf{Paraissant trois fois par jour.}}}\end{otherlanguage}\hfill \textsc{Paris\oindex{Paris@\textbf{Paris}, \emph{P.PPLC}|pw}}, 5. Oct.\pend
           
\pstart
           \begin{otherlanguage}{french}\textcolor{gray}{\textbf{\textbf{Bureau à Paris\oindex{Paris@\textbf{Paris}, \emph{P.PPLC}|pw}}}}\end{otherlanguage}\pend
           
\pstart
           \begin{otherlanguage}{french}\textcolor{gray}{\textbf{\textbf{10 \so{Rue de la Bourse}\oindex{rue de la Bourse@\textbf{rue de la Bourse}, \emph{Straße (K.STR)}|pw}.}}}\end{otherlanguage}\pend
           \vspace{0.5em}
\pstart
           Ich bitte um eine kurze Nachricht, liebſter Freund.
               Kein langer Brief, nur ein paar Worte! Du weißt ja, daß \label{K_L02828-1v}\edtext{\textsc{Richard\pwindex{Beer-Hofmann, Richard 1866-07-11 – 1945-09-26@\textsc{Beer-Hofmann, Richard} (1866-07-11 – 1945-09-26), \emph{Schriftsteller/Schriftstellerin}|pw}} nicht ſchreibt}{\lemma{\textnormal{\emph{Richard nicht ſchreibt}}}\Cendnote{\textnormal{Beer-Hofmann\pwindex{Beer-Hofmann, Richard 1866-07-11 – 1945-09-26@\textsc{Beer-Hofmann, Richard} (1866-07-11 – 1945-09-26), \emph{Schriftsteller/Schriftstellerin}|pwk} galt als notorisch
                  unzuverlässiger Briefeschreiber.}}}\label{K_L02828-1}. So muß ich mich ſchon an Dich ſelbſt
               wenden, um etwas über Dich zu wiſſen.\pend
           
\pstart
           Sei herzlichſt begrüßt!\pend
           
\pstart
           Dein {\\[\baselineskip]}\spacefill\mbox{Paul Goldmn}\pend
           \leftskip=0em{}\selectlanguage{ngerman}\endnumbering\briefempfaengerindex{Schnitzler, Arthur@\textsc{Schnitzler, Arthur}!zzzGoldmann, Paul@\emph{von Paul Goldmann}!1897-10-051@{5. 10. {[}1897{]}}|)be}\mylabel{L02828h}  \normalsize

\doendnotes{C}
\bigskip
\vfill

\clearpage

\footnotesize

\lohead{\textsc{register}}

% Definiere theindex-Environment komplett neu ohne reledmac
\makeatletter
\renewenvironment{theindex}{%
  \section*{\indexname}%
  \setlength{\parindent}{0pt}%
  \setlength{\parskip}{0pt plus 0.3pt}%
  \let\item\@idxitem
}{%
  \clearpage
}
\makeatother

\IfFileExists{\jobname-pw.ind}{\input{\jobname-pw.ind}}{}

\end{document}

      