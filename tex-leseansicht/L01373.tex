%% latex-korrekturansicht-vorspann.tex
%% Vorspann für die Korrekturansicht.
%% Lädt die gemeinsame Datei latex-vorspann.tex mit gesetztem Schalter.

\newif\ifkorrekturansicht
\korrekturansichttrue

\input{../tex-inputs/latex-vorspann}


\section[Hermann Bahr an Arthur Schnitzler, 15. 2. 1904]{L01373 Hermann Bahr an Arthur Schnitzler, 15. 2. 1904}
\nopagebreak\mylabel{L01373v}
\rehead{ }\normalsize\beginnumbering\briefempfaengerindex{Schnitzler, Arthur@\textsc{Schnitzler, Arthur}!zzzBahr, Hermann@\emph{von Hermann Bahr}!1904-02-151@{15. 2. 1904}|(be}
\toendnotes[C]{\smallbreak\pagebreak[2]}\Standort{CUL, Schnitzler, B 5b.}
\physDesc{Kartenbrief, 737 Zeichen
\newline{}Handschrift: schwarze Tinte, deutsche Kurrent
\newline{}Versand: 1) Stempel: »\nobreak{}\oindex{Opatija@\textbf{Opatija}, \emph{P.PPLA2}|pwk}Abbazia, 15. 2. 04\nobreak{}«.   2) Stempel: »\nobreak{}\oindex{XVIII., Waehring@\textbf{XVIII., Währing}, \emph{A.ADM3}|pwk}18/1 Wien, 17. 2. 04, 8.V, Bestellt\nobreak{}«. 
\newline{}Schnitzler: mit rotem Buntstift eine Unterstreichung 
\newline{}Ordnung: mit Bleistift von unbekannter Hand nummeriert:
                                    »111« }
\buchAbdrucke{\weitereDrucke{Hermann Bahr, Arthur Schnitzler: \emph{Briefwechsel, Aufzeichnungen, Dokumente (1891–1931)}. Göttingen: \emph{Wallstein} 2018, S. 300.} }\toendnotes[C]{\smallbreak}\pstart{}{\pb}Herrn \textsc{D\textsuperscript{r} Arthur Schnitzler}\pend{}\pstart{}\textsc{Wien XVIII}\oindex{XVIII., Waehring@\textbf{XVIII., Währing}, \emph{A.ADM3}|pw}\pend{}\pstart{}Spöttelgaſſe 7\oindex{Edmund-Weiss-Gasse 7@\textbf{Edmund-Weiß-Gasse 7}, \emph{Wohngebäude (K.WHS)}|pw}\pend{}{\bigskip}\vspace{1em}
\pstart
           \raggedleft{}{\pb}15. 2. 04{\\}Abbazia\oindex{Opatija@\textbf{Opatija}, \emph{P.PPLA2}|pw}{ }Hot. \textsc{Guarnero}\oindex{Hotel Guarnero@\textbf{Hotel Guarnero}, \emph{Hotel (K.HTL)}|pw}\pend
           
\pstart{}Lieber Arthur!\pend\vspace{0.5em}
\pstart
           Ich kam heut hier an und weil der Trebitſch\pwindex{Trebitsch, Siegfried 22.12.1868 – 03.06.1956@\textsc{Trebitsch, Siegfried} (22.12.1868 – 03.06.1956), \emph{Schriftsteller/Schriftstellerin, Übersetzer/Übersetzerin}|pw},
               der mir ein Telegra{\geminationm} verſprochen, es verbummelt hat,
               ließ ich mich verleiten, in den Wiener\oindex{Wien@\textbf{Wien}, \emph{A.ADM2}|pw} Zeitungen
               nachzuſehen, deren Ton aber ſo hundsgemein iſt, daß ich ihn phyſiſch nicht mehr
               vertrage. Und nun nachdem ich mich unſinnig geärgert hab, weiß ich zudem natürlich
               gar nichts: wars ein Erfolg, wars keiner? Ich weiß aber, daß das Stück\pwindex{einsame Weg. Schauspiel in fuenf Akten@\emph{Der einsame Weg. Schauspiel in fünf Akten}|pwv} zu Deinen ſchönſten und reinſten
               Arbeiten gehört, und ich mein, wir ſollten uns überhaupt nicht mehr zu Erfolgen,
               sondern zu den Werken, die uns etwas ſind, gratulieren. Mir iſt der »einſame Weg\pwindex{einsame Weg. Schauspiel in fuenf Akten@\emph{Der einsame Weg. Schauspiel in fünf Akten}|pw}« in ſeinen Hauptgeſtalten und ihrem
               Erleben sehr viel.\pend
           \pstart Herzlichſt \hspace*{1.5em}Dein \spacefill\mbox{Hermann}\pend{}\selectlanguage{ngerman}\endnumbering\briefempfaengerindex{Schnitzler, Arthur@\textsc{Schnitzler, Arthur}!zzzBahr, Hermann@\emph{von Hermann Bahr}!1904-02-151@{15. 2. 1904}|)be}\mylabel{L01373h}  \normalsize

\doendnotes{C}
\bigskip
\vfill

\clearpage

\footnotesize

\lohead{\textsc{register}}

% Definiere theindex-Environment komplett neu ohne reledmac
\makeatletter
\renewenvironment{theindex}{%
  \section*{\indexname}%
  \setlength{\parindent}{0pt}%
  \setlength{\parskip}{0pt plus 0.3pt}%
  \let\item\@idxitem
}{%
  \clearpage
}
\makeatother

\IfFileExists{\jobname-pw.ind}{\input{\jobname-pw.ind}}{}

\end{document}

      