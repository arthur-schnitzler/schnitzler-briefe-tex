%% latex-leseansicht-vorspann.tex
%% Vorspann für die Leseansicht.
%% Lädt die gemeinsame Datei latex-vorspann.tex mit nicht gesetztem Schalter.

\newif\ifkorrekturansicht
\korrekturansichtfalse

\input{../tex-inputs/latex-vorspann}


\section[Sigmund Freud an Arthur Schnitzler, 24. 3. 1926]{L03889 Sigmund Freud an Arthur Schnitzler, 24. 3. 1926}
\nopagebreak\mylabel{L03889v}
\rehead{ }\normalsize\beginnumbering\briefempfaengerindex{Schnitzler, Arthur@\textsc{Schnitzler, Arthur}!zzzFreud, Sigmund@\emph{von Sigmund Freud}!1926-03-241@{24. 3. 1926}|(be}
\toendnotes[C]{\smallbreak\pagebreak[2]}
\correspDesc{Versand  durch Sigmund Freud am 24. 3. 1926 in Wien
\newline{}Erhalt  durch Arthur Schnitzler im Zeitraum [24. 3. 1926 –
                  26. 3. 1926] in Wien}\toendnotes[C]{\smallbreak}
\Standort{Washington, DC, Library of Congress, Freud Archives, C41F8.}
\physDesc{Brief, Fotokopie, 1 Blatt, 1 Seite, 478 Zeichen
\newline{}Schreibmaschine
\newline{}Handschrift: schwarze Tinte (\noindent{}Unterschrift)
\newline{}Schnitzler: mit rotem Buntstift eine Unterstreichung 
\newline{}Zusatz: Der Verbleib des Originals ist ungeklärt. Zum Zeitpunkt der
                                 ersten Edition 1955 befand es sich im Besitz von Heinrich Schnitzler\pwindex{Schnitzler, Heinrich 9.\,8.\,1902 Hinterbrühl – 12.\,7.\,1982 Wien@\textsc{Schnitzler, Heinrich} (9.\,8.\,1902 Hinterbrühl – 12.\,7.\,1982 Wien), \emph{Regisseur, Schauspieler}|pw}. }
\buchAbdrucke{\weitereDrucke{1) Sigmund Freud: \emph{Briefe an Arthur Schnitzler.}Herausgegeben von Henry Schnitzler In: \emph{Neue deutsche Rundschau}, Jg. 66 (Januar 1955) Nr. 1, S. 99.} \weitereDrucke{2) Sigmund Freud: \emph{Sigmund Freud Edition. Digitale historisch-kritische
                        Gesamtausgabe}. Herausgegeben von Christine Diercks, Arkadi Blatow und Elisabeth Skale. (2014–2025) \url{https://www.freudedition.net/briefe/freud-sigmund/schnitzler-arthur/1926/03/24}.} }\toendnotes[C]{\smallbreak}
\pstart
           {\pb}\textcolor{gray}{\textbf{PROF. D\textsuperscript{R.} FREUD}}\hfill \textcolor{gray}{\textbf{WIEN IX., BERGGASSE 19\oindex{Wien@\textbf{Wien}!IX., Alsergrund@\textbf{IX., Alsergrund}!Berggasse 19@\textbf{Berggasse 19}, \emph{Wohngebäude}|pw}. }}\pend
           
\pstart
           \raggedleft{}24. III. 26.\pend
           
\pstart{}Verehrtester\pend\vspace{0.5em}
\pstart
           Es hat mir ausserordentlich leid getan, dass Sie unlängst einen erfolglosen Besuch
               bei mir machten. Mein Tag ist in diesem Zauberberg\oindex{Wien@\textbf{Wien}!XVIII., Währing@\textbf{XVIII., Währing}!Cottage-Sanatorium für Nerven- und Stoffwechselkranke@\textbf{Cottage-Sanatorium für Nerven- und Stoffwechselkranke}, \emph{Sanatorium}|pwv}\pwindex{\textcolor{red}{\textsuperscript{XXXX indx1}}!Zauberberg. Roman@\strich\emph{Der Zauberberg. Roman}|pwv} oder dieser Zauberhöhle\oindex{Wien@\textbf{Wien}!XVIII., Währing@\textbf{XVIII., Währing}!Cottage-Sanatorium für Nerven- und Stoffwechselkranke@\textbf{Cottage-Sanatorium für Nerven- und Stoffwechselkranke}, \emph{Sanatorium}|pwv} so kunstvoll eingeteilt, dass mir für Genüsse
               nur der Abend bleibt. Darf ich Ihnen vorschlagen, mich \label{K_L03889-1v}\edtext{heute}{\lemma{\textnormal{\emph{heute}}}\Cendnote{\textnormal{Schnitzler besuchte Freud\pwindex{Freud, Sigmund 6.\,5.\,1856 Pribor – 23.\,9.\,1939 London@\textsc{Freud, Sigmund} (6.\,5.\,1856 Pribor – 23.\,9.\,1939 London), \emph{Psychoanalytiker}|pwk} erst am 26. 3. 1926 im Cottage-Sanatorium\oindex{Wien@\textbf{Wien}!XVIII., Währing@\textbf{XVIII., Währing}!Cottage-Sanatorium für Nerven- und Stoffwechselkranke@\textbf{Cottage-Sanatorium für Nerven- und Stoffwechselkranke}, \emph{Sanatorium}|pwk}.}}}\label{K_L03889-1} nach 8 oder 8 ¼ Uhr,
               nachdem das Nachtmahl absolviert ist, auf Gedankenaustausch und Zigarre zu beehren?
               Oder Ueberbringer dieses eine andere Bestimmung mitzugeben?\pend
           
\pstart
           Mit nachbarlichem Gruss{\\[\baselineskip]} Ihr \spacefill\mbox{{[}hs.:{]} Freud}\pend
           \leftskip=0em{}\selectlanguage{ngerman}\endnumbering\briefempfaengerindex{Schnitzler, Arthur@\textsc{Schnitzler, Arthur}!zzzFreud, Sigmund@\emph{von Sigmund Freud}!1926-03-241@{24. 3. 1926}|)be}\mylabel{L03889h}
\begin{anhang}
\end{anhang}\newcommand{\dateiname}{L03889}\newcommand{\titel}{Sigmund Freud an Arthur Schnitzler, 24. 3. 1926}\newcommand{\editorInnen}{Selma Jahnke und Martin Anton Müller}%% latex-leseansicht-abspann.tex
%% Abspann für die Leseansicht.
%% Der Schalter \ifkorrekturansicht ist bereits durch den Vorspann gesetzt.

%% latex-abspann.tex
%% Gemeinsamer Abspann für Korrekturansicht und Leseansicht.
%% Setzt den Schalter \ifkorrekturansicht voraus (gesetzt in den
%% einbindenden Dateien latex-korrekturansicht-abspann.tex bzw.
%% latex-leseansicht-abspann.tex).
%% ---------------------------------------------------------------

\normalsize

% Das esempio-Environment wird nur in der Leseansicht benötigt
\ifkorrekturansicht\else
\newenvironment{esempio}[3]%
{
    \vspace{1.5ex}
    \rlap{\underline{#1}}
    \par
    \setlength{\parindent}{0cm}
    \nopagebreak
    \leftskip=#2cm
    \rightskip=#3cm
}
{
    \par
}
\fi

\doendnotes{C}
\bigskip
\vfill

\clearpage

\footnotesize

\ifkorrekturansicht
  \lohead{\textsc{register}}
\fi

% theindex-Environment neu definieren ohne reledmac
\makeatletter
\renewenvironment{theindex}{%
  \ifkorrekturansicht
    \section*{\indexname}%
  \else
    \subsubsection*{Index der erwähnten Entitäten}%
  \fi
  \setlength{\parindent}{0pt}%
  \setlength{\parskip}{0pt plus 0.3pt}%
  \let\item\@idxitem
}{%
  \ifkorrekturansicht\clearpage\fi
}
\makeatother

\IfFileExists{\jobname-pw.ind}{\input{\jobname-pw.ind}}{}

% Quellenangabe nur in der Leseansicht
\ifkorrekturansicht\else
% Fallback-Definitionen, falls die .tex-Datei \titel etc. nicht gesetzt hat
\providecommand{\titel}{}
\providecommand{\editorInnen}{}
\providecommand{\dateiname}{\jobname}

\vspace{3cm}

\vfill

\footnotesize
\textsc{Quelle}: \titel. Herausgegeben von {\editorInnen}. In: \emph{Arthur Schnitzler: Briefwechsel mit Autorinnen und Autoren}.
 Digitale Edition, https://schnitzler-briefe.acdh.oeaw.ac.at/{\dateiname}.html (Stand \today)
\fi

\end{document}


