%% latex-leseansicht-vorspann.tex
%% Vorspann für die Leseansicht.
%% Lädt die gemeinsame Datei latex-vorspann.tex mit nicht gesetztem Schalter.

\newif\ifkorrekturansicht
\korrekturansichtfalse

\input{../tex-inputs/latex-vorspann}

\begin{center}
            \textcolor{red}{ENTWURF. ENTZIFFERUNG NOCH NICHT KORREKTURGELESEN}
                      \end{center}
            
               \section[Arthur und Olga Schnitzler an Richard Beer-Hofmann, 3. 7. 1906]{ Arthur und Olga Schnitzler an Richard Beer-Hofmann,
               3. 7. 1906}\nopagebreak\mylabel{v}\rehead{ }\begin{ledgroupsized}[t]{13cm}\normalsize\beginnumbering\briefempfaengerindex{Beer-Hofmann, Richard@\textsc{Beer-Hofmann, Richard}!zzzSchnitzler, Olga@\emph{von Olga Schnitzler}!1906-07-031@{3. 7. 1906}|(be}\briefempfaengerindex{Beer-Hofmann, Richard@\textsc{Beer-Hofmann, Richard}!zzzSchnitzler, Arthur@\emph{von Arthur Schnitzler}!1906-07-031@{3. 7. 1906}|(be} \toendnotes[C]{\smallbreak\pagebreak[2]} \Standort{YCGL, MSS 31.}
\physDesc{Bildpostkarte
\newline{}Handschrift Arthur Schnitzler: Bleistift, deutsche Kurrent\newline{}Handschrift Olga Schnitzler: Bleistift\newline{}Versand: Stempel: »\nobreak{}3. 7. 06\nobreak{}«.  }\buchAbdrucke{\weitereDrucke{Arthur Schnitzler, Richard Beer-Hofmann: \emph{Briefwechsel 1891–1931}. Hg. Konstanze Fliedl. Wien, Zürich: \emph{Europaverlag} 1992, S. 178.} }\pstart{}{\pb}\textsc{Dr. Richard Beer Hofmann}\pend{}\pstart{}\textsc{Rodaun\oindex{Rodaun@\textbf{Rodaun}|pw} b/Wien\oindex{Wien@\textbf{Wien}|pw}}\pend{}\pstart{}\textsc{Liesingerstr}. 2\oindex{Liesingerstrasse@\textbf{Liesingerstraße}|pw}.\pend{}\pstart{}\textsc{Austria\oindex{Oesterreich@\textbf{Österreich}|pw}}\pend{}{\bigskip}\pstart
           \noindent{}\centering{}{\pb}\textcolor{gray}{\textbf{Hilsen fra Marienlyst\oindex{Marienlyst@\textbf{Marienlyst}|pw}.}}\pend
           \pstart
           \raggedleft{}\textsc{Marienlyst, Kurhaus}\oindex{Kurhotellet@\textbf{Kurhotellet}|pw},
                     3/7. 9\textcolor{gray}{06}\pend
           \pstart
           Wunderſchön hier. Dürften längre Zeit bleiben. Habe geſtern Brandes\pwindex{Brandes, Georg 04.02.1842 – 19.02.1927@\textsc{Brandes, Georg} (04.02.1842 – 19.02.1927)|pw} beſucht, der im \textsc{Co{\geminationm}une Hospital}\oindex{Kommunehospitalet@\textbf{Kommunehospitalet}|pw} wieder
               mit ſeiner Venenentzündung krank liegt aber friſch iſt wie je.\pend
           \pstart
           Laſſen Sie von ſich und den Ihren hören. Herzlichſt,\pend
           \pstart
           Ihr \spacefill\mbox{A.}{\\[\baselineskip]}\spacefill\mbox{{[}hs. O. Schnitzler:{]} O. S.}\pend
           \leftskip=0em{}\pstart
           {[}hs. Schnitzler:{]} \textsc{Charolais}\pwindex{Beer-Hofmann, Richard 11.07.1866 – 26.09.1945@\textsc{Beer-Hofmann, Richard} (11.07.1866 – 26.09.1945), \emph{Schriftsteller}!Graf von Charolais. Ein Trauerspiel1904-12-23 – 1904-12-23@\strich\emph{Der Graf von Charolais. Ein Trauerspiel} {[}1904-12-23 – 1904-12-23{]}|pw} wird im Herbſt in \textsc{Kopenhagen}\oindex{Kopenhagen@\textbf{Kopenhagen}|pw} aufgeführt.\pend
           \endnumbering\briefempfaengerindex{Beer-Hofmann, Richard@\textsc{Beer-Hofmann, Richard}!zzzSchnitzler, Olga@\emph{von Olga Schnitzler}!1906-07-031@{3. 7. 1906}|)be}\briefempfaengerindex{Beer-Hofmann, Richard@\textsc{Beer-Hofmann, Richard}!zzzSchnitzler, Arthur@\emph{von Arthur Schnitzler}!1906-07-031@{3. 7. 1906}|)be}\mylabel{h}\end{ledgroupsized}  \newcommand{\dateiname}{L01606}\newcommand{\titel}{Arthur und Olga Schnitzler an Richard Beer-Hofmann, 3. 7. 1906}\newcommand{\editorInnen}{Martin Anton Müller und Gerd-Hermann Susen}%% latex-leseansicht-abspann.tex
%% Abspann für die Leseansicht.
%% Der Schalter \ifkorrekturansicht ist bereits durch den Vorspann gesetzt.

%% latex-abspann.tex
%% Gemeinsamer Abspann für Korrekturansicht und Leseansicht.
%% Setzt den Schalter \ifkorrekturansicht voraus (gesetzt in den
%% einbindenden Dateien latex-korrekturansicht-abspann.tex bzw.
%% latex-leseansicht-abspann.tex).
%% ---------------------------------------------------------------

\normalsize

% Das esempio-Environment wird nur in der Leseansicht benötigt
\ifkorrekturansicht\else
\newenvironment{esempio}[3]%
{
    \vspace{1.5ex}
    \rlap{\underline{#1}}
    \par
    \setlength{\parindent}{0cm}
    \nopagebreak
    \leftskip=#2cm
    \rightskip=#3cm
}
{
    \par
}
\fi

\doendnotes{C}
\bigskip
\vfill

\clearpage

\footnotesize

\ifkorrekturansicht
  \lohead{\textsc{register}}
\fi

% theindex-Environment neu definieren ohne reledmac
\makeatletter
\renewenvironment{theindex}{%
  \ifkorrekturansicht
    \section*{\indexname}%
  \else
    \subsubsection*{Index der erwähnten Entitäten}%
  \fi
  \setlength{\parindent}{0pt}%
  \setlength{\parskip}{0pt plus 0.3pt}%
  \let\item\@idxitem
}{%
  \ifkorrekturansicht\clearpage\fi
}
\makeatother

\IfFileExists{\jobname-pw.ind}{\input{\jobname-pw.ind}}{}

% Quellenangabe nur in der Leseansicht
\ifkorrekturansicht\else
% Fallback-Definitionen, falls die .tex-Datei \titel etc. nicht gesetzt hat
\providecommand{\titel}{}
\providecommand{\editorInnen}{}
\providecommand{\dateiname}{\jobname}

\vspace{3cm}

\vfill

\footnotesize
\textsc{Quelle}: \titel. Herausgegeben von {\editorInnen}. In: \emph{Arthur Schnitzler: Briefwechsel mit Autorinnen und Autoren}.
 Digitale Edition, https://schnitzler-briefe.acdh.oeaw.ac.at/{\dateiname}.html (Stand \today)
\fi

\end{document}


      