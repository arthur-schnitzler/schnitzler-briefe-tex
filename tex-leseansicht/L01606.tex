%% latex-korrekturansicht-vorspann.tex
%% Vorspann für die Korrekturansicht.
%% Lädt die gemeinsame Datei latex-vorspann.tex mit gesetztem Schalter.

\newif\ifkorrekturansicht
\korrekturansichttrue

\input{../tex-inputs/latex-vorspann}


\section[Arthur und Olga Schnitzler an Richard Beer-Hofmann, 3. 7. 1906]{L01606 Arthur und Olga Schnitzler an Richard Beer-Hofmann, 3. 7. 1906}
\nopagebreak\mylabel{L01606v}
\rehead{ }\normalsize\beginnumbering\briefempfaengerindex{Beer-Hofmann, Richard@\textsc{Beer-Hofmann, Richard}!zzzSchnitzler, Olga@\emph{von Olga Schnitzler}!1906-07-031@{3. 7. 1906}|(be}\briefempfaengerindex{Beer-Hofmann, Richard@\textsc{Beer-Hofmann, Richard}!zzzSchnitzler, Arthur@\emph{von Arthur Schnitzler}!1906-07-031@{3. 7. 1906}|(be}
\toendnotes[C]{\smallbreak\pagebreak[2]}\Standort{YCGL, MSS 31.}
\physDesc{Bildpostkarte, 360 Zeichen
\newline{}Handschrift Arthur Schnitzler: Bleistift, deutsche Kurrent
\newline{}Handschrift Olga Schnitzler: Bleistift
\newline{}Versand: Stempel: »\nobreak{}3. 7. 06\nobreak{}«.  }
\buchAbdrucke{\weitereDrucke{Arthur Schnitzler, Richard Beer-Hofmann: \emph{Briefwechsel 1891–1931}. Wien, Zürich: \emph{Europaverlag} 1992, S. 178.} }\toendnotes[C]{\smallbreak}\pstart{}{\pb}\textsc{Dr. Richard Beer Hofmann}\pend{}\pstart{}\textsc{Rodaun\oindex{Rodaun@\textbf{Rodaun}, \emph{A.ADM4}|pw} b/Wien\oindex{Wien@\textbf{Wien}, \emph{A.ADM2}|pw}}\pend{}\pstart{}\textsc{Liesingerstr}. 2\oindex{Liesingerstrasse@\textbf{Liesingerstraße}, \emph{Straße (K.STR)}|pw}.\pend{}\pstart{}\textsc{Austria\oindex{Oesterreich@\textbf{Österreich}, \emph{A.PCLI}|pw}}\pend{}{\bigskip}
\pstart
           \noindent{}\centering{}{\pb}\textcolor{gray}{\textbf{Hilsen fra Marienlyst\oindex{Marienlyst@\textbf{Marienlyst}, \emph{S.EST}|pw}.}}\pend
           \vspace{1em}
\pstart
           \raggedleft{}{\pb}\textsc{Marienlyst, Kurhaus}\oindex{Kurhotellet@\textbf{Kurhotellet}, \emph{Hotel (K.HTL)}|pw}, 3/7. 9\textcolor{gray}{06}\pend
           \vspace{0.5em}
\pstart
           Wunderſchön hier. Dürften längre Zeit bleiben. Habe \label{K_L01606-1v}\edtext{geſtern Brandes\pwindex{Brandes, Georg 04.02.1842 – 19.02.1927@\textsc{Brandes, Georg} (04.02.1842 – 19.02.1927)|pw}
                  beſucht}{\lemma{\textnormal{\emph{geſtern Brandes
                  beſucht}}}\Cendnote{\textnormal{Vgl. A. S.: \emph{Tagebuch}, 2. 7. 1906. }}}\label{K_L01606-1}, der
               im \textsc{Co{\geminationm}une Hospital}\oindex{Kommunehospitalet@\textbf{Kommunehospitalet}, \emph{Krankenhaus (K.KKH)}|pw} wieder mit ſeiner Venenentzündung krank liegt aber friſch iſt wie je.\pend
           
\pstart
           Laſſen Sie von ſich und den Ihren hören. Herzlichſt,\pend
           
\pstart
           Ihr \spacefill\mbox{A.}{\\[\baselineskip]}\spacefill\mbox{{[}hs. :{]} O. S.}\pend
           \leftskip=0em{}
\pstart
           {[}hs. :{]} \textsc{Charolais}\pwindex{Graf von Charolais. Ein Trauerspiel@\emph{Der Graf von Charolais. Ein Trauerspiel}|pw} wird im Herbſt in \textsc{Kopenhagen}\oindex{Kopenhagen@\textbf{Kopenhagen}, \emph{P.PPLC}|pw} aufgeführt.\pend
           \selectlanguage{ngerman}\endnumbering\briefempfaengerindex{Beer-Hofmann, Richard@\textsc{Beer-Hofmann, Richard}!zzzSchnitzler, Olga@\emph{von Olga Schnitzler}!1906-07-031@{3. 7. 1906}|)be}\briefempfaengerindex{Beer-Hofmann, Richard@\textsc{Beer-Hofmann, Richard}!zzzSchnitzler, Arthur@\emph{von Arthur Schnitzler}!1906-07-031@{3. 7. 1906}|)be}\mylabel{L01606h}  \normalsize

\doendnotes{C}
\bigskip
\vfill

\clearpage

\footnotesize

\lohead{\textsc{register}}

% Definiere theindex-Environment komplett neu ohne reledmac
\makeatletter
\renewenvironment{theindex}{%
  \section*{\indexname}%
  \setlength{\parindent}{0pt}%
  \setlength{\parskip}{0pt plus 0.3pt}%
  \let\item\@idxitem
}{%
  \clearpage
}
\makeatother

\IfFileExists{\jobname-pw.ind}{\input{\jobname-pw.ind}}{}

\end{document}

      