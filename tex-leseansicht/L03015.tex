%% latex-leseansicht-vorspann.tex
%% Vorspann für die Leseansicht.
%% Lädt die gemeinsame Datei latex-vorspann.tex mit nicht gesetztem Schalter.

\newif\ifkorrekturansicht
\korrekturansichtfalse

\input{../tex-inputs/latex-vorspann}

\begin{center}
            \textcolor{red}{ENTWURF, NICHT FERTIG KORRIGIERT}
                      \end{center}
            
         
         \renewcommand{\erwaehntePersonen}{Personen: Leopold von Andrian-Werburg, Anton Bettelheim, Hedwig Bleibtreu, Felix Salten}
         \renewcommand{\erwaehnteInstitutionen}{Institutionen: Die Zeit}
         \renewcommand{\erwaehnteOrte}{Orte: Sternwartestraße 71, Wien}
         \renewcommand{\erwaehnteWerke}{Werke: Tagebuch}
               \section[ Arthur Schnitzler an Felix Salten, {[}14. 4. 1910?{]}]{ Arthur Schnitzler an Felix Salten, {[}14. 4. 1910?{]}}\nopagebreak\mylabel{v}\rehead{ }\begin{ledgroupsized}[t]{13cm}\normalsize\beginnumbering \toendnotes[C]{\smallbreak\pagebreak[2]} \Standort{Wienbibliothek im Rathaus, ZPH 1681, 2.1.516.}
\physDesc{Brief, 1 Blatt, 3 Seiten, 501 Zeichen
\newline{}Handschrift: Bleistift, deutsche Kurrent
\newline{}Ordnung: mit Bleistift von unbekannter Hand Nummerierung der Blätter des Konvoluts:
                                    »5«–»6« }\toendnotes[C]{\smallbreak}\pstart
           \noindent{}{\pb}lieber, ich weiſs nun nicht, wa{\geminationn} ich
                  \label{K_L03015-1v}\edtext{in den nächſten Tagen zu Ihnen ko{\geminationm}en}{\lemma{\textnormal{\emph{in … kommen}}}\Cendnote{\textnormal{Schnitzler\pwindex{Schnitzler, Arthur 15.05.1862 – 21.10.1931@\textsc{Schnitzler, Arthur} (15.05.1862 – 21.10.1931), \emph{Schriftsteller, Mediziner}|pwk} war am 15. 4. 1910 und am
                     21. 4. 1910 bei
                     Salten\pwindex{Salten, Felix 06.09.1869 – 08.10.1945@\textsc{Salten, Felix} (06.09.1869 – 08.10.1945), \emph{Schriftsteller, Journalist}|pwk}.}}}\label{K_L03015-1h} ka{\geminationn}; u muſs Sie nur etwas fragen: wie Ihre Sache mit der
                  »\uline{Zeit}\orgindex{Zeit@Die Zeit|pw}« ſteht. Es hat mich nemlich {\pb}\label{K_L03015-2v}\edtext{jemand\pwindex{Andrian-Werburg, Leopold von 09.05.1875 – 19.11.1951@\textsc{Andrian-Werburg, Leopold von} (09.05.1875 – 19.11.1951), \emph{Schriftsteller, Diplomat}|pwuv}\pwindex{Bettelheim, Anton 18.11.1851 – 29.03.1930@\textsc{Bettelheim, Anton} (18.11.1851 – 29.03.1930), \emph{Kritiker, Lexikograf}|pwuv}}{\lemma{\textnormal{\emph{jemand}}}\Cendnote{\textnormal{Sofern es sich um jemanden handelt, der
                  am 14. 4. 1910 im
                     \emph{Tagebuch}\pwindex{\textcolor{red}{\textsuperscript{XXXX1 indx}}!Tagebuch1981 – 2000@\strich\emph{Tagebuch} {[}Hrsg., 1981 – 2000{]}|pwk} genannt wird, könnten Leopold Andrian\pwindex{Andrian-Werburg, Leopold von 09.05.1875 – 19.11.1951@\textsc{Andrian-Werburg, Leopold von} (09.05.1875 – 19.11.1951), \emph{Schriftsteller, Diplomat}|pwk} oder Anton Bettelheim\pwindex{Bettelheim, Anton 18.11.1851 – 29.03.1930@\textsc{Bettelheim, Anton} (18.11.1851 – 29.03.1930), \emph{Kritiker, Lexikograf}|pwk} gemeint gewesen sein.}}}\label{K_L03015-2h}, den ich
               nicht nennen darf, um meine Intervention für die Stellung eines
                  Feu{[}i{]}lleton Redacteurs erſucht, u ich habe vorläufg abgelehnt,
               da ich nicht weiſs, ob Sie noch in Verhandlung {\pb}ſtehn \textsc{etc.} (Habe
               natürlich Ihren Namen nicht genannt.) Bitte ſagen Sie mir ein Wort. Was fehlt Ihnen
               eigentlich?\pend
           \pstart
           herzlichſt Ihr {\\[\baselineskip]}\spacefill\mbox{Arthur}\pend
           \leftskip=0em{}\pstart
           \noindent{}\label{K_L03015-3v}\edtext{Endlich hab ich die Villa\oindex{Sternwartestrasse 71@\textbf{Sternwartestraße 71}|pwv}}{\lemma{\textnormal{\emph{Endlich … Villa}}}\Cendnote{\textnormal{Am 14. 4. 1910 hatte Schnitzler\pwindex{Schnitzler, Arthur 15.05.1862 – 21.10.1931@\textsc{Schnitzler, Arthur} (15.05.1862 – 21.10.1931), \emph{Schriftsteller, Mediziner}|pwk} den Kaufvertrag für das bis dahin im Eigentum
                     von Hedwig Bleibtreu-Römpler\pwindex{Bleibtreu, Hedwig 23.12.1868 – 24.01.1958@\textsc{Bleibtreu, Hedwig} (23.12.1868 – 24.01.1958), \emph{Schauspielerin}|pwk} stehende
                     Haus in der Sternwartestrasse 71\oindex{Sternwartestrasse 71@\textbf{Sternwartestraße 71}|pwk}
                     unterschrieben. Damit kann das undatierte Korrespondenzstück zeitlich nach
                     vorne abgegrenzt werden. Da sich Salten\pwindex{Salten, Felix 06.09.1869 – 08.10.1945@\textsc{Salten, Felix} (06.09.1869 – 08.10.1945), \emph{Schriftsteller, Journalist}|pwk}
                     und Schnitzler\pwindex{Schnitzler, Arthur 15.05.1862 – 21.10.1931@\textsc{Schnitzler, Arthur} (15.05.1862 – 21.10.1931), \emph{Schriftsteller, Mediziner}|pwk} am Folgetag, dem 15. 4. 1910,
                     bereits ausführlich sprachen, ist auch zeitlich nach hinten eine Grenze zu
                     ziehen.}}}\label{K_L03015-3h}\pend
           
         
         \endnumbering\mylabel{h}\end{ledgroupsized}  \newcommand{\dateiname}{L03015}\newcommand{\titel}{Arthur Schnitzler an Felix Salten, [14. 4. 1910?]}\newcommand{\editorInnen}{Martin Anton Müller und Laura Untner}%% latex-leseansicht-abspann.tex
%% Abspann für die Leseansicht.
%% Der Schalter \ifkorrekturansicht ist bereits durch den Vorspann gesetzt.

%% latex-abspann.tex
%% Gemeinsamer Abspann für Korrekturansicht und Leseansicht.
%% Setzt den Schalter \ifkorrekturansicht voraus (gesetzt in den
%% einbindenden Dateien latex-korrekturansicht-abspann.tex bzw.
%% latex-leseansicht-abspann.tex).
%% ---------------------------------------------------------------

\normalsize

% Das esempio-Environment wird nur in der Leseansicht benötigt
\ifkorrekturansicht\else
\newenvironment{esempio}[3]%
{
    \vspace{1.5ex}
    \rlap{\underline{#1}}
    \par
    \setlength{\parindent}{0cm}
    \nopagebreak
    \leftskip=#2cm
    \rightskip=#3cm
}
{
    \par
}
\fi

\doendnotes{C}
\bigskip
\vfill

\clearpage

\footnotesize

\ifkorrekturansicht
  \lohead{\textsc{register}}
\fi

% theindex-Environment neu definieren ohne reledmac
\makeatletter
\renewenvironment{theindex}{%
  \ifkorrekturansicht
    \section*{\indexname}%
  \else
    \subsubsection*{Index der erwähnten Entitäten}%
  \fi
  \setlength{\parindent}{0pt}%
  \setlength{\parskip}{0pt plus 0.3pt}%
  \let\item\@idxitem
}{%
  \ifkorrekturansicht\clearpage\fi
}
\makeatother

\IfFileExists{\jobname-pw.ind}{\input{\jobname-pw.ind}}{}

% Quellenangabe nur in der Leseansicht
\ifkorrekturansicht\else
% Fallback-Definitionen, falls die .tex-Datei \titel etc. nicht gesetzt hat
\providecommand{\titel}{}
\providecommand{\editorInnen}{}
\providecommand{\dateiname}{\jobname}

\vspace{3cm}

\vfill

\footnotesize
\textsc{Quelle}: \titel. Herausgegeben von {\editorInnen}. In: \emph{Arthur Schnitzler: Briefwechsel mit Autorinnen und Autoren}.
 Digitale Edition, https://schnitzler-briefe.acdh.oeaw.ac.at/{\dateiname}.html (Stand \today)
\fi

\end{document}


      