%% latex-korrekturansicht-vorspann.tex
%% Vorspann für die Korrekturansicht.
%% Lädt die gemeinsame Datei latex-vorspann.tex mit gesetztem Schalter.

\newif\ifkorrekturansicht
\korrekturansichttrue

\input{../tex-inputs/latex-vorspann}


\section[ Arthur Schnitzler an Felix Salten, {[}14. 4. 1910?{]}]{L03015 Arthur Schnitzler an Felix Salten, {[}14. 4. 1910?{]}}
\nopagebreak\mylabel{L03015v}
\rehead{ }\normalsize\beginnumbering\briefempfaengerindex{Salten, Felix@\textsc{Salten, Felix}!zzzSchnitzler, Arthur@\emph{von Arthur Schnitzler}!1910-04-141@{{[}14. 4. 1910?{]}}|(be}
\toendnotes[C]{\smallbreak\pagebreak[2]}\Standort{Wienbibliothek im Rathaus, ZPH 1681, 2.1.516.}
\physDesc{Brief, 1 Blatt, 3 Seiten, 499 Zeichen
\newline{}Handschrift: Bleistift, deutsche Kurrent
\newline{}Ordnung: mit Bleistift von unbekannter Hand Nummerierung der Doppelseiten des
                                 Konvoluts: »5«–»6« }\toendnotes[C]{\smallbreak}
\pstart
           \noindent{}{\pb}lieber, ich weiſs nun nicht, wa{\geminationn} ich
                  \label{K_L03015-1v}\edtext{in den nächſten Tagen zu Ihnen ko{\geminationm}en}{\lemma{\textnormal{\emph{in … kommen}}}\Cendnote{\textnormal{Schnitzler war am 15. 4. 1910 und am
                     21. 4. 1910 bei
                     Salten\pwindex{Salten, Felix 06.09.1869 – 08.10.1945@\textsc{Salten, Felix} (06.09.1869 – 08.10.1945), \emph{Schriftsteller/Schriftstellerin, Journalist/Journalistin, Chefredakteur/Chefredakteurin}|pwk}.}}}\label{K_L03015-1} ka{\geminationn}, u muſs Sie nur etwas fragen: wie Ihre \label{K_L03015-2v}\edtext{Sache mit der »\uline{Zeit}\orgindex{Zeit@Die Zeit|pw}}{\lemma{\textnormal{\emph{Sache mit der »Zeit}}}\Cendnote{\textnormal{Salten\pwindex{Salten, Felix 06.09.1869 – 08.10.1945@\textsc{Salten, Felix} (06.09.1869 – 08.10.1945), \emph{Schriftsteller/Schriftstellerin, Journalist/Journalistin, Chefredakteur/Chefredakteurin}|pwk} blieb noch ziemlich genau ein weiteres Jahr bei der \emph{Zeit}\orgindex{Zeit@Die Zeit|pwk}, bevor er entlassen wurde, siehe A. S.: \emph{Tagebuch}, 23. 5. 1911. }}}\label{K_L03015-2}« ſteht.
               Es hat mich nemlich {\pb}\label{K_L03015-3v}\edtext{jemand\pwindex{Andrian-Werburg, Leopold von 09.05.1875 – 19.11.1951@\textsc{Andrian-Werburg, Leopold von} (09.05.1875 – 19.11.1951), \emph{Schriftsteller/Schriftstellerin, Diplomat/Diplomatin}|pwuv}\pwindex{Bettelheim, Anton 18.11.1851 – 29.03.1930@\textsc{Bettelheim, Anton} (18.11.1851 – 29.03.1930), \emph{Kritiker/Kritikerin, Lexikograf/Lexikografin}|pwuv}}{\lemma{\textnormal{\emph{jemand}}}\Cendnote{\textnormal{Sofern es sich um eine Person handelt,
                  die am 14. 4. 1910
                  im \emph{Tagebuch}\pwindex{Tagebuch@\emph{Tagebuch}|pwk} genannt wird, könnten Leopold Andrian\pwindex{Andrian-Werburg, Leopold von 09.05.1875 – 19.11.1951@\textsc{Andrian-Werburg, Leopold von} (09.05.1875 – 19.11.1951), \emph{Schriftsteller/Schriftstellerin, Diplomat/Diplomatin}|pwk} oder Anton Bettelheim\pwindex{Bettelheim, Anton 18.11.1851 – 29.03.1930@\textsc{Bettelheim, Anton} (18.11.1851 – 29.03.1930), \emph{Kritiker/Kritikerin, Lexikograf/Lexikografin}|pwk} gemeint sein.}}}\label{K_L03015-3}, den ich
               nicht nennen darf, um meine Intervention für die Stellung eines \textsc{Feu{[}i{]}lleton Redacteurs} erſucht, u ich habe vorläufg
               abgelehnt, da ich nicht weiſs, ob Sie noch in Verhandlung {\pb}ſtehn \textsc{etc.} (Habe
               natürlich Ihren Namen nicht genannt.) Bitte ſagen Sie mir ein Wort. Was fehlt Ihnen
               eigentlich?\pend
           
\pstart
           herzlichſt Ihr {\\[\baselineskip]}\spacefill\mbox{Arthur}\pend
           \leftskip=0em{}
\pstart
           \noindent{}\label{K_L03015-4v}\edtext{Endlich hab ich die Villa\oindex{Sternwartestrasse 71@\textbf{Sternwartestraße 71}, \emph{Wohngebäude (K.WHS)}|pwv}}{\lemma{\textnormal{\emph{Endlich … Villa}}}\Cendnote{\textnormal{Am 14. 4. 1910 hatte Schnitzler den Kaufvertrag für das bis dahin im Eigentum
                     von Hedwig Bleibtreu-Römpler\pwindex{Bleibtreu, Hedwig 23.12.1868 – 24.01.1958@\textsc{Bleibtreu, Hedwig} (23.12.1868 – 24.01.1958), \emph{Schauspieler/Schauspielerin}|pwk} stehende
                     Haus in der Sternwartestrasse 71\oindex{Sternwartestrasse 71@\textbf{Sternwartestraße 71}, \emph{Wohngebäude (K.WHS)}|pwk}
                     unterschrieben. Damit kann das undatierte Korrespondenzstück zeitlich nach
                     vorne abgegrenzt werden. Da sich Salten\pwindex{Salten, Felix 06.09.1869 – 08.10.1945@\textsc{Salten, Felix} (06.09.1869 – 08.10.1945), \emph{Schriftsteller/Schriftstellerin, Journalist/Journalistin, Chefredakteur/Chefredakteurin}|pwk}
                     und Schnitzler am Folgetag, dem 15. 4. 1910,
                     bereits ausführlich sprachen, ist auch zeitlich nach hinten eine Grenze zu
                     ziehen.}}}\label{K_L03015-4}\pend
           \selectlanguage{ngerman}\endnumbering\briefempfaengerindex{Salten, Felix@\textsc{Salten, Felix}!zzzSchnitzler, Arthur@\emph{von Arthur Schnitzler}!1910-04-141@{{[}14. 4. 1910?{]}}|)be}\mylabel{L03015h}  \normalsize

\doendnotes{C}
\bigskip
\vfill

\clearpage

\footnotesize

\lohead{\textsc{register}}

% Definiere theindex-Environment komplett neu ohne reledmac
\makeatletter
\renewenvironment{theindex}{%
  \section*{\indexname}%
  \setlength{\parindent}{0pt}%
  \setlength{\parskip}{0pt plus 0.3pt}%
  \let\item\@idxitem
}{%
  \clearpage
}
\makeatother

\IfFileExists{\jobname-pw.ind}{\input{\jobname-pw.ind}}{}

\end{document}

      