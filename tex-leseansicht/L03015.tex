%% latex-leseansicht-vorspann.tex
%% Vorspann für die Leseansicht.
%% Lädt die gemeinsame Datei latex-vorspann.tex mit nicht gesetztem Schalter.

\newif\ifkorrekturansicht
\korrekturansichtfalse

\input{../tex-inputs/latex-vorspann}


\section[ Arthur Schnitzler an Felix Salten, [14. 4. 1910?]]{L03015 Arthur Schnitzler an Felix Salten,  [14. 4. 1910?]}
\nopagebreak\mylabel{L03015v}
\rehead{ }\normalsize\beginnumbering\briefempfaengerindex{Salten, Felix@\textsc{Salten, Felix}!zzzSchnitzler, Arthur@\emph{von Arthur Schnitzler}!1910-04-141@{{[}14. 4. 1910?{]}}|(be}
\toendnotes[C]{\smallbreak\pagebreak[2]}
\correspDesc{Versand  durch Arthur Schnitzler am [14. 4. 1910?] in Wien
\newline{}Erhalt  durch Felix Salten im Zeitraum [14. 4. 1910
                  – 17. 4. 1910?] in Wien}\toendnotes[C]{\smallbreak}
\Standort{Wienbibliothek im Rathaus, ZPH 1681, 2.1.516.}
\physDesc{Brief, 1 Blatt, 3 Seiten, 499 Zeichen
\newline{}Handschrift: Bleistift, deutsche Kurrent
\newline{}Ordnung: mit Bleistift von unbekannter Hand Nummerierung der Doppelseiten des
                                 Konvoluts: »5«–»6« }\toendnotes[C]{\smallbreak}
\pstart
           \noindent{}{\pb}lieber, ich weiſs nun nicht, wa{\geminationn} ich
                  \label{K_L03015-1v}\edtext{in den nächſten Tagen zu Ihnen ko{\geminationm}en}{\lemma{\textnormal{\emph{in … kommen}}}\Cendnote{\textnormal{Schnitzler war am 15. 4. 1910 und am
                     21. 4. 1910 bei
                     Salten\pwindex{Salten, Felix 6.\,9.\,1869 Budapest – 8.\,10.\,1945 Zürich@\textsc{Salten, Felix} (6.\,9.\,1869 Budapest – 8.\,10.\,1945 Zürich), \emph{Schriftsteller, Journalist, Chefredakteur}|pwk}.}}}\label{K_L03015-1} ka{\geminationn}, u muſs Sie nur etwas fragen: wie Ihre \label{K_L03015-2v}\edtext{Sache mit der »\uline{Zeit}\orgindex{Zeit@Die Zeit|pw}}{\lemma{\textnormal{\emph{Sache mit der »Zeit}}}\Cendnote{\textnormal{Salten\pwindex{Salten, Felix 6.\,9.\,1869 Budapest – 8.\,10.\,1945 Zürich@\textsc{Salten, Felix} (6.\,9.\,1869 Budapest – 8.\,10.\,1945 Zürich), \emph{Schriftsteller, Journalist, Chefredakteur}|pwk} blieb noch ziemlich genau ein weiteres Jahr bei der \emph{Zeit}\orgindex{Zeit@Die Zeit|pwk}, bevor er entlassen wurde, siehe A. S.: \emph{Tagebuch}, 23. 5. 1911. }}}\label{K_L03015-2}«{ }ſteht.
               Es hat mich nemlich {\pb}\label{K_L03015-3v}\edtext{jemand\pwindex{Andrian-Werburg, Leopold von 9.\,5.\,1875 Berlin – 19.\,11.\,1951 Fribourg@\textsc{Andrian-Werburg, Leopold von} (9.\,5.\,1875 Berlin – 19.\,11.\,1951 Fribourg), \emph{Schriftsteller, Diplomat}|pwuv}\pwindex{Bettelheim, Anton 18.\,11.\,1851 Wien – 29.\,3.\,1930 ebd.@\textsc{Bettelheim, Anton} (18.\,11.\,1851 Wien – 29.\,3.\,1930 ebd.), \emph{Kritiker, Lexikograf}|pwuv}}{\lemma{\textnormal{\emph{jemand}}}\Cendnote{\textnormal{Sofern es sich um eine Person handelt,
                  die am 14. 4. 1910
                  im \emph{Tagebuch}\pwindex{Schnitzler, Arthur 15.\,5.\,1862 Wien – 21.\,10.\,1931 ebd.@\textsc{Schnitzler, Arthur} (15.\,5.\,1862 Wien – 21.\,10.\,1931 ebd.), \emph{Schriftsteller, Mediziner}!Tagebuch@\strich\emph{Tagebuch}|pwk} genannt wird, könnten Leopold Andrian\pwindex{Andrian-Werburg, Leopold von 9.\,5.\,1875 Berlin – 19.\,11.\,1951 Fribourg@\textsc{Andrian-Werburg, Leopold von} (9.\,5.\,1875 Berlin – 19.\,11.\,1951 Fribourg), \emph{Schriftsteller, Diplomat}|pwk} oder Anton Bettelheim\pwindex{Bettelheim, Anton 18.\,11.\,1851 Wien – 29.\,3.\,1930 ebd.@\textsc{Bettelheim, Anton} (18.\,11.\,1851 Wien – 29.\,3.\,1930 ebd.), \emph{Kritiker, Lexikograf}|pwk} gemeint sein.}}}\label{K_L03015-3}, den ich
               nicht nennen darf, um meine Intervention für die Stellung eines \textsc{Feu{[}i{]}lleton Redacteurs} erſucht, u ich habe vorläufg
               abgelehnt, da ich nicht weiſs, ob Sie noch in Verhandlung {\pb}ſtehn \textsc{etc.} (Habe
               natürlich Ihren Namen nicht genannt.) Bitte{ }ſagen Sie mir ein Wort. Was fehlt Ihnen
               eigentlich?\pend
           
\pstart
           herzlichſt Ihr {\\[\baselineskip]}\spacefill\mbox{Arthur}\pend
           \leftskip=0em{}
\pstart
           \noindent{}\label{K_L03015-4v}\edtext{Endlich hab ich die Villa\oindex{Wien@\textbf{Wien}!XVIII., Währing@\textbf{XVIII., Währing}!Sternwartestraße 71@\textbf{Sternwartestraße 71}, \emph{Wohngebäude}|pwv}}{\lemma{\textnormal{\emph{Endlich … Villa}}}\Cendnote{\textnormal{Am 14. 4. 1910 hatte Schnitzler den Kaufvertrag für das bis dahin im Eigentum
                     von Hedwig Bleibtreu-Römpler\pwindex{Bleibtreu, Hedwig 23.\,12.\,1868 Linz – 24.\,1.\,1958 Wien@\textsc{Bleibtreu, Hedwig} (23.\,12.\,1868 Linz – 24.\,1.\,1958 Wien), \emph{Schauspielerin}|pwk} stehende
                     Haus in der Sternwartestrasse 71\oindex{Wien@\textbf{Wien}!XVIII., Währing@\textbf{XVIII., Währing}!Sternwartestraße 71@\textbf{Sternwartestraße 71}, \emph{Wohngebäude}|pwk}
                     unterschrieben. Damit kann das undatierte Korrespondenzstück zeitlich nach
                     vorne abgegrenzt werden. Da sich Salten\pwindex{Salten, Felix 6.\,9.\,1869 Budapest – 8.\,10.\,1945 Zürich@\textsc{Salten, Felix} (6.\,9.\,1869 Budapest – 8.\,10.\,1945 Zürich), \emph{Schriftsteller, Journalist, Chefredakteur}|pwk}
                     und Schnitzler am Folgetag, dem 15. 4. 1910,
                     bereits ausführlich sprachen, ist auch zeitlich nach hinten eine Grenze zu
                     ziehen.}}}\label{K_L03015-4}\pend
           \selectlanguage{ngerman}\endnumbering\briefempfaengerindex{Salten, Felix@\textsc{Salten, Felix}!zzzSchnitzler, Arthur@\emph{von Arthur Schnitzler}!1910-04-141@{{[}14. 4. 1910?{]}}|)be}\mylabel{L03015h}  \newcommand{\dateiname}{L03015}\newcommand{\titel}{Arthur Schnitzler an Felix Salten, [14. 4. 1910?]}\newcommand{\editorInnen}{Martin Anton Müller und Laura Untner}%% latex-leseansicht-abspann.tex
%% Abspann für die Leseansicht.
%% Der Schalter \ifkorrekturansicht ist bereits durch den Vorspann gesetzt.

%% latex-abspann.tex
%% Gemeinsamer Abspann für Korrekturansicht und Leseansicht.
%% Setzt den Schalter \ifkorrekturansicht voraus (gesetzt in den
%% einbindenden Dateien latex-korrekturansicht-abspann.tex bzw.
%% latex-leseansicht-abspann.tex).
%% ---------------------------------------------------------------

\normalsize

% Das esempio-Environment wird nur in der Leseansicht benötigt
\ifkorrekturansicht\else
\newenvironment{esempio}[3]%
{
    \vspace{1.5ex}
    \rlap{\underline{#1}}
    \par
    \setlength{\parindent}{0cm}
    \nopagebreak
    \leftskip=#2cm
    \rightskip=#3cm
}
{
    \par
}
\fi

\doendnotes{C}
\bigskip
\vfill

\clearpage

\footnotesize

\ifkorrekturansicht
  \lohead{\textsc{register}}
\fi

% theindex-Environment neu definieren ohne reledmac
\makeatletter
\renewenvironment{theindex}{%
  \ifkorrekturansicht
    \section*{\indexname}%
  \else
    \subsubsection*{Index der erwähnten Entitäten}%
  \fi
  \setlength{\parindent}{0pt}%
  \setlength{\parskip}{0pt plus 0.3pt}%
  \let\item\@idxitem
}{%
  \ifkorrekturansicht\clearpage\fi
}
\makeatother

\IfFileExists{\jobname-pw.ind}{\input{\jobname-pw.ind}}{}

% Quellenangabe nur in der Leseansicht
\ifkorrekturansicht\else
% Fallback-Definitionen, falls die .tex-Datei \titel etc. nicht gesetzt hat
\providecommand{\titel}{}
\providecommand{\editorInnen}{}
\providecommand{\dateiname}{\jobname}

\vspace{3cm}

\vfill

\footnotesize
\textsc{Quelle}: \titel. Herausgegeben von {\editorInnen}. In: \emph{Arthur Schnitzler: Briefwechsel mit Autorinnen und Autoren}.
 Digitale Edition, https://schnitzler-briefe.acdh.oeaw.ac.at/{\dateiname}.html (Stand \today)
\fi

\end{document}


