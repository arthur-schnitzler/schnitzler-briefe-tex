%% latex-korrekturansicht-vorspann.tex
%% Vorspann für die Korrekturansicht.
%% Lädt die gemeinsame Datei latex-vorspann.tex mit gesetztem Schalter.

\newif\ifkorrekturansicht
\korrekturansichttrue

\input{../tex-inputs/latex-vorspann}


\section[ Paul Goldmann an Arthur Schnitzler, 27. 3. {[}1903{]}]{L03370 Paul Goldmann an Arthur Schnitzler, 27. 3. {[}1903{]}}
\nopagebreak\mylabel{L03370v}
\rehead{ }\normalsize\beginnumbering\briefempfaengerindex{Schnitzler, Arthur@\textsc{Schnitzler, Arthur}!zzzGoldmann, Paul@\emph{von Paul Goldmann}!1903-03-272@{27. 3. {[}1903{]}}|(be}
\toendnotes[C]{\smallbreak\pagebreak[2]}\Standort{DLA, A:Schnitzler, HS.NZ85.1.3173.}
\physDesc{Brief, 1 Blatt, 3 Seiten, 775 Zeichen
\newline{}Handschrift: blaue Tinte, deutsche Kurrent
\newline{}Schnitzler: 1) mit Bleistift das Jahr »903.« vermerkt  2) mit rotem Buntstift zwei Unterstreichungen}\toendnotes[C]{\smallbreak}
\pstart
           \raggedleft{}{\pb}\textcolor{gray}{\textbf{DESSAUERSTRASSE 19\oindex{Dessauer Strasse@\textbf{Dessauer Straße}, \emph{Straße (K.STR)}|pw}}}\pend
           
\pstart
           Berlin\oindex{Berlin@\textbf{Berlin}, \emph{P.PPLC}|pw}, 27. März.\pend
           
\pstart\center{}Mein lieber Freund,\pend\vspace{0.5em}
\pstart
           Täglich will ich Dir ſchreiben, und immer verhindert mich die Arbeit daran. Arbeit
               und Verſtimmung: ich kann mich zu gar nichs mehr aufraffen. Dein lieber Brief war mir
               eine \label{K_L03370-1v}\edtext{große Freude und
                  Herzenserleichterung}{\lemma{\textnormal{\emph{große … Herzenserleichterung}}}\Cendnote{\textnormal{Bezug auf Goldmanns\pwindex{Goldmann, Paul 31.01.1865 – 25.09.1935@\textsc{Goldmann, Paul} (31.01.1865 – 25.09.1935), \emph{Schriftsteller/Schriftstellerin, Journalist/Journalistin}|pwk} kritisches \emph{Beatrice}\pwindex{Schleier der Beatrice. Schauspiel in fuenf Akten@\emph{Der Schleier der Beatrice. Schauspiel in fünf Akten}|pwk}-Feuilleton\pwindex{Berliner Theater. (»Der Schleier der Beatrice« von Arthur Schnitzler.)@\emph{Berliner Theater. (»Der Schleier der Beatrice« von Arthur Schnitzler.)}|pwkv}, siehe Paul Goldmann an Arthur Schnitzler, 17. 3. [1903]. Schnitzler dürfte seine
                  Verärgerung über das Feuilleton\pwindex{Berliner Theater. (»Der Schleier der Beatrice« von Arthur Schnitzler.)@\emph{Berliner Theater. (»Der Schleier der Beatrice« von Arthur Schnitzler.)}|pwkv}{ }Goldmann\pwindex{Goldmann, Paul 31.01.1865 – 25.09.1935@\textsc{Goldmann, Paul} (31.01.1865 – 25.09.1935), \emph{Schriftsteller/Schriftstellerin, Journalist/Journalistin}|pwk} gegenüber noch nicht ausgedrückt
                  haben. Vgl. dazu etwa das \emph{Tagebuch}\pwindex{Tagebuch@\emph{Tagebuch}|pwk} ab dem
                     19. 3. 1903.}}}\label{K_L03370-1}. Sachlich hätte ich noch Mancherlei zu ſagen. Aber ich möchte über
               dieſes {\pb}unglückſelige Feuilleton\pwindex{Berliner Theater. (»Der Schleier der Beatrice« von Arthur Schnitzler.)@\emph{Berliner Theater. (»Der Schleier der Beatrice« von Arthur Schnitzler.)}|pwv}, das ich habe ſchreiben \uline{müſſen}, überhaupt nicht mehr reden.\pend
           
\pstart
           Heute{ }tritt\pwindex{Schleier der Beatrice@\emph{Der Schleier der Beatrice}|pwv}{ }\label{K_L03370-2v}\edtext{\textsc{Harden\pwindex{Harden, Maximilian 20.10.1861 – 30.10.1927@\textsc{Harden, Maximilian} (20.10.1861 – 30.10.1927), \emph{Schriftsteller/Schriftstellerin, Publizist/Publizistin}|pw}}}{\lemma{\textnormal{\emph{Harden}}}\Cendnote{\textnormal{M. H.\pwindex{Harden, Maximilian 20.10.1861 – 30.10.1927@\textsc{Harden, Maximilian} (20.10.1861 – 30.10.1927), \emph{Schriftsteller/Schriftstellerin, Publizist/Publizistin}|pwkv} [ = Maximilian Harden\pwindex{Harden, Maximilian 20.10.1861 – 30.10.1927@\textsc{Harden, Maximilian} (20.10.1861 – 30.10.1927), \emph{Schriftsteller/Schriftstellerin, Publizist/Publizistin}|pwk}]: \emph{Der Schleier der Beatrice}\pwindex{Schleier der Beatrice@\emph{Der Schleier der Beatrice}|pwk}. In: \emph{Die Zukunft}\pwindex{Zukunft@\emph{Die Zukunft}|pwk}, Bd. 42, 28. 3. 1903, S. 517–530.}}}\label{K_L03370-2} mit großer Wärme für die »\textsc{Beatrice\pwindex{Schleier der Beatrice. Schauspiel in fuenf Akten@\emph{Der Schleier der Beatrice. Schauspiel in fünf Akten}|pw}}« ein. Ich liebe zwar dieſe ſeine \label{K_L03370-3v}\edtext{»rhapſodiſchen«}{\lemma{\textnormal{\emph{»rhapſodiſchen«}}}\Cendnote{\textnormal{unzusammenhängend,
                  lückenhaft}}}\label{K_L03370-3} Aufſätze nicht; aber ich freue mich des ſtarken Anhängers\pwindex{Harden, Maximilian 20.10.1861 – 30.10.1927@\textsc{Harden, Maximilian} (20.10.1861 – 30.10.1927), \emph{Schriftsteller/Schriftstellerin, Publizist/Publizistin}|pwv}, der Dir und Deinem
                  Werke\pwindex{Schleier der Beatrice. Schauspiel in fuenf Akten@\emph{Der Schleier der Beatrice. Schauspiel in fünf Akten}|pwv} erwächſt.\pend
           
\pstart
           \label{K_L03370-4v}\edtext{\textsc{Salten\pwindex{Salten, Felix 06.09.1869 – 08.10.1945@\textsc{Salten, Felix} (06.09.1869 – 08.10.1945), \emph{Schriftsteller/Schriftstellerin, Journalist/Journalistin, Chefredakteur/Chefredakteurin}|pw}\pwindex{Eine kurze, aber notwendige Auseinandersetzung@\emph{Eine kurze, aber notwendige Auseinandersetzung}|pwv}} über \textsc{Schlenther\pwindex{Schlenther, Paul 20.08.1854 – 30.04.1916@\textsc{Schlenther, Paul} (20.08.1854 – 30.04.1916), \emph{Schriftsteller/Schriftstellerin, Kritiker/Kritikerin, Theaterleiter/Theaterleiterin}|pw}}}{\lemma{\textnormal{\emph{Salten über Schlenther}}}\Cendnote{\textnormal{Felix Salten\pwindex{Salten, Felix 06.09.1869 – 08.10.1945@\textsc{Salten, Felix} (06.09.1869 – 08.10.1945), \emph{Schriftsteller/Schriftstellerin, Journalist/Journalistin, Chefredakteur/Chefredakteurin}|pwk}: \emph{Eine kurze, aber notwendige Auseinandersetzung}\pwindex{Eine kurze, aber notwendige Auseinandersetzung@\emph{Eine kurze, aber notwendige Auseinandersetzung}|pwk}. In:
                        \emph{Die Zeit. Wiener Wochenschrift}\pwindex{Zeit. Wiener Wochenschrift@\emph{Die Zeit. Wiener Wochenschrift}|pwk}, Bd. 34,
                     Nr. 442, 21. 3. 1903, S. 143–145.}}}\label{K_L03370-4}{ }{\pb}hat mir und hoffentlich auch Dir ſehr wohl
               gethan.\pend
           
\pstart
           Wie geht es Dir? \textsc{Olga\pwindex{Schnitzler, Olga 17.01.1882 – 13.01.1970@\textsc{Schnitzler, Olga} (17.01.1882 – 13.01.1970), \emph{Schauspieler/Schauspielerin, Sänger/Sängerin}|pw}}? Dem Sohn\pwindex{Schnitzler, Heinrich 09.08.1902 – 12.07.1982@\textsc{Schnitzler, Heinrich} (09.08.1902 – 12.07.1982), \emph{Regisseur/Regisseurin, Schauspieler/Schauspielerin}|pwv}? Wirſt Du
                  \label{K_L03370-5v}\edtext{verreiſen}{\lemma{\textnormal{\emph{verreiſen}}}\Cendnote{\textnormal{Die nächste größere Reise ging zwischen 28. 5. 1903 und 15. 6. 1903 nach Italien\oindex{Italien@\textbf{Italien}, \emph{A.PCLI}|pwk} und Südtirol\oindex{Suedtirol@\textbf{Südtirol}, \emph{A.ADM2}|pwk}, gemeinsam mit Olga
                     Gussmann\pwindex{Schnitzler, Olga 17.01.1882 – 13.01.1970@\textsc{Schnitzler, Olga} (17.01.1882 – 13.01.1970), \emph{Schauspieler/Schauspielerin, Sänger/Sängerin}|pwk}.}}}\label{K_L03370-5}? Wann? Wohin?\pend
           
\pstart
           Sei vielmals gegrüßt von Deinem getreuen {\\[\baselineskip]}\spacefill\mbox{Paul Goldmn}\pend
           \leftskip=0em{}\selectlanguage{ngerman}\endnumbering\briefempfaengerindex{Schnitzler, Arthur@\textsc{Schnitzler, Arthur}!zzzGoldmann, Paul@\emph{von Paul Goldmann}!1903-03-272@{27. 3. {[}1903{]}}|)be}\mylabel{L03370h}  \normalsize

\doendnotes{C}
\bigskip
\vfill

\clearpage

\footnotesize

\lohead{\textsc{register}}

% Definiere theindex-Environment komplett neu ohne reledmac
\makeatletter
\renewenvironment{theindex}{%
  \section*{\indexname}%
  \setlength{\parindent}{0pt}%
  \setlength{\parskip}{0pt plus 0.3pt}%
  \let\item\@idxitem
}{%
  \clearpage
}
\makeatother

\IfFileExists{\jobname-pw.ind}{\input{\jobname-pw.ind}}{}

\end{document}

      