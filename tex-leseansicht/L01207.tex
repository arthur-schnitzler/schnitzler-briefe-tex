%% latex-leseansicht-vorspann.tex
%% Vorspann für die Leseansicht.
%% Lädt die gemeinsame Datei latex-vorspann.tex mit nicht gesetztem Schalter.

\newif\ifkorrekturansicht
\korrekturansichtfalse

\input{../tex-inputs/latex-vorspann}


         \renewcommand{\erwaehnteInstitutionen}{Institutionen: Concordia, Neues Wiener Tagblatt}
         \renewcommand{\erwaehnteOrte}{Orte: Steyrerhof, Wien}
         \renewcommand{\erwaehnteWerke}{}
               \section[Hermann Bahr an Arthur Schnitzler, 8. 3. {[}1902{]}]{ Hermann Bahr an Arthur Schnitzler, 8. 3. {[}1902{]}}\nopagebreak\mylabel{v}\rehead{ }\begin{ledgroupsized}[t]{13cm}\normalsize\beginnumbering \toendnotes[C]{\smallbreak\pagebreak[2]} \Standort{CUL, Schnitzler, B 5b.}
\physDesc{Brief, 1 Blatt, 1 Seite
\newline{}Handschrift: schwarze Tinte, deutsche Kurrent\newline{}Ordnung: mit Bleistift von unbekannter Hand nummeriert: »86« }\buchAbdrucke{\weitereDrucke{Hermann Bahr, Arthur Schnitzler: \emph{Briefwechsel, Aufzeichnungen, Dokumente (1891–1931)}. Hg. Kurt Ifkovits und Martin Anton Müller. Göttingen: \emph{Wallstein} 2018, S. 227.} }\pstart
           \noindent{}\centering{}{\pb}\textcolor{gray}{\textbf{Redaktion des Neuen Wiener
                           Tagblatt\orgindex{Neues Wiener Tagblatt@Neues Wiener Tagblatt|pw}}}\pend
           \pstart
           \noindent{}\centering{}\textcolor{gray}{\textbf{\textsc{Wien, I., Rothenturmstrasse,
                        Steyrerhof\oindex{Steyrerhof@\textbf{Steyrerhof}|pw}.}}}\pend
           \pstart
           \noindent{}\centering{}\textcolor{gray}{\textbf{Telegramm-Adresse: Tagblatt\orgindex{Neues Wiener Tagblatt@Neues Wiener Tagblatt|pw}, Steyrerhof, Wien\oindex{Steyrerhof@\textbf{Steyrerhof}|pw}. –
                     Telephon Nr. 384. Staats-Telephon Nr. 36.}}\pend
           \pstart
           8. März\pend
           \pstart\center{}Lieber Arthur!\pend\pstart
           Ein paar Mädeln quälen mich um eine Photographie von Dir mit Unterſchrift oder »ein
               paar Bücher mit Widmung«, als Gewinnſte bei einer Tombola im Concordiaclub\orgindex{Concordia@Concordia|pw} am 16. d.{ }ſchick ſie mir oder
               ſchreib mir nein, damit ich nur das Gezwitſcher los bin.\pend
           \pstart
           Herzlichſt{\\[\baselineskip]}Dein{\\[\baselineskip]}\spacefill\mbox{HermannBahr}\pend
           \leftskip=0em{}
         
         \endnumbering\mylabel{h}\end{ledgroupsized}  \newcommand{\dateiname}{L01207}\newcommand{\titel}{Hermann Bahr an Arthur Schnitzler, 8. 3. [1902]}\newcommand{\editorInnen}{ Kurt Ifkovits,  Martin Anton Müller}%% latex-leseansicht-abspann.tex
%% Abspann für die Leseansicht.
%% Der Schalter \ifkorrekturansicht ist bereits durch den Vorspann gesetzt.

%% latex-abspann.tex
%% Gemeinsamer Abspann für Korrekturansicht und Leseansicht.
%% Setzt den Schalter \ifkorrekturansicht voraus (gesetzt in den
%% einbindenden Dateien latex-korrekturansicht-abspann.tex bzw.
%% latex-leseansicht-abspann.tex).
%% ---------------------------------------------------------------

\normalsize

% Das esempio-Environment wird nur in der Leseansicht benötigt
\ifkorrekturansicht\else
\newenvironment{esempio}[3]%
{
    \vspace{1.5ex}
    \rlap{\underline{#1}}
    \par
    \setlength{\parindent}{0cm}
    \nopagebreak
    \leftskip=#2cm
    \rightskip=#3cm
}
{
    \par
}
\fi

\doendnotes{C}
\bigskip
\vfill

\clearpage

\footnotesize

\ifkorrekturansicht
  \lohead{\textsc{register}}
\fi

% theindex-Environment neu definieren ohne reledmac
\makeatletter
\renewenvironment{theindex}{%
  \ifkorrekturansicht
    \section*{\indexname}%
  \else
    \subsubsection*{Index der erwähnten Entitäten}%
  \fi
  \setlength{\parindent}{0pt}%
  \setlength{\parskip}{0pt plus 0.3pt}%
  \let\item\@idxitem
}{%
  \ifkorrekturansicht\clearpage\fi
}
\makeatother

\IfFileExists{\jobname-pw.ind}{\input{\jobname-pw.ind}}{}

% Quellenangabe nur in der Leseansicht
\ifkorrekturansicht\else
% Fallback-Definitionen, falls die .tex-Datei \titel etc. nicht gesetzt hat
\providecommand{\titel}{}
\providecommand{\editorInnen}{}
\providecommand{\dateiname}{\jobname}

\vspace{3cm}

\vfill

\footnotesize
\textsc{Quelle}: \titel. Herausgegeben von {\editorInnen}. In: \emph{Arthur Schnitzler: Briefwechsel mit Autorinnen und Autoren}.
 Digitale Edition, https://schnitzler-briefe.acdh.oeaw.ac.at/{\dateiname}.html (Stand \today)
\fi

\end{document}


      