%% latex-leseansicht-vorspann.tex
%% Vorspann für die Leseansicht.
%% Lädt die gemeinsame Datei latex-vorspann.tex mit nicht gesetztem Schalter.

\newif\ifkorrekturansicht
\korrekturansichtfalse

\input{../tex-inputs/latex-vorspann}


         
         \newcommand{\erwaehntePersonen}{Personen: Maximilian Harden, Jules Huret, Rudolf Lothar, Leopold Sonnemann}
         \newcommand{\erwaehnteInstitutionen}{Institutionen: Deutsches Theater Berlin, Frankfurter Zeitung, Le Figaro}
         \newcommand{\erwaehnteOrte}{Orte: Berlin, Burgtheater, Deutschland, Paris, Rue Drouot, Wien, rue Feydeau}
         \newcommand{\erwaehnteWerke}{Werke: Courrier des Théatres [Liebelei-Premiere Berlin], Frankfurter Zeitung, Le Figaro, Liebelei. Schauspiel in drei Akten}
               \section[Paul Goldmann an Arthur Schnitzler, 6. 2. {[}1896{]}]{ Paul Goldmann an Arthur Schnitzler, 6. 2. {[}1896{]}}\nopagebreak\mylabel{v}\rehead{ }\begin{ledgroupsized}[t]{13cm}\normalsize\beginnumbering \toendnotes[C]{\smallbreak\pagebreak[2]} \Standort{DLA, A:Schnitzler, HS.NZ85.1.3166.}
\physDesc{Brief, 2 Blätter, 7 Seiten
\newline{}Handschrift: blaue Tinte, deutsche Kurrent
\newline{}Schnitzler: 1) mit Bleistift das Jahr »96« vermerkt  2) mit rotem Buntstift eine Unterstreichung}\toendnotes[C]{\smallbreak}\pstart
           \noindent{}{\pb}\textcolor{gray}{\textbf{\textbf{Frankfurter Zeitung\orgindex{Frankfurter Zeitung@Frankfurter Zeitung|pw}}}}\pend
           \pstart
           \textcolor{gray}{\textbf{(\begin{otherlanguage}{french}Gazette de Francfort\end{otherlanguage}\orgindex{Frankfurter Zeitung@Frankfurter Zeitung|pw}).}}\pend
           \pstart
           \textcolor{gray}{\textbf{\textbf{\begin{otherlanguage}{french}Fondateur M.\end{otherlanguage}{ }L. Sonnemann\pwindex{Sonnemann, Leopold 1831-10-29 – 1909-10-30@\textsc{Sonnemann, Leopold} (1831-10-29 – 1909-10-30), \emph{Journalist, Herausgeber}|pw}.}}}\pend
           \pstart
           \begin{otherlanguage}{french}\textcolor{gray}{\textbf{Journal\pwindex{?? Werk@Nicht ermittelte Verfasserinnen und Verfasser!Frankfurter Zeitung1856 – 1943@\emph{Frankfurter Zeitung} {[}1856 – 1943{]}|pwv} politique,
                        financier,}}\end{otherlanguage}\pend
           \pstart
           \begin{otherlanguage}{french}\textcolor{gray}{\textbf{commercial et littéraire.}}\end{otherlanguage}\pend
           \pstart
           \begin{otherlanguage}{french}\textcolor{gray}{\textbf{\textbf{Paraissant trois fois par jour.}}}\end{otherlanguage}\pend
           \pstart
           \begin{otherlanguage}{french}\textcolor{gray}{\textbf{\textbf{Bureau à Paris\oindex{Paris@\textbf{Paris}|pw}:}}}\end{otherlanguage}\hfill \textsc{Paris\oindex{Paris@\textbf{Paris}|pw}}, 6. Februar.\pend
           \pstart
           \begin{otherlanguage}{french}\textcolor{gray}{\textbf{\textbf{24. Rue Feydeau\oindex{rue Feydeau@\textbf{rue Feydeau}|pw}.}}}\end{otherlanguage}\pend
           \pstart\center{}Mein lieber Freund,\pend\pstart
           Ich ſchreibe Dir nicht nach \textsc{Berlin\oindex{Berlin@\textbf{Berlin}|pw}}, weil ich nicht weiß, ob mein Brief Dich noch dort erreicht.\pend
           \pstart
           Alſo nochmals: innigen Glückwünſch! Nun biſt Du ganz und gar ein gemachter Mann.
               Selbſt dem ſkeptiſchen und kalten Berlin\oindex{Berlin@\textbf{Berlin}|pw} haſt Du
               gefallen. Jetzt wird das Stück\pwindex{Schnitzler, Arthur 15.05.1862 – 21.10.1931@\textsc{Schnitzler, Arthur} (15.05.1862 – 21.10.1931), \emph{Schriftsteller, Mediziner}!Liebelei. Schauspiel in drei Akten1895-10-09@\strich\emph{Liebelei. Schauspiel in drei Akten} {[}1895-10-09{]}|pwv}
               durch ganz Deutſchland\oindex{Deutschland@\textbf{Deutschland}|pw} gehen\strikeout{,} und Du biſt heut, in Deinen jungen Jahren, einer der
               erſten deutſchen Bühnendichter. {\pb}Das war zwar Alles
               vorauszuſehen; aber es iſt doch herrlich, daß man es \strikeout{e\textcolor{gray}{r}} erleb\strikeout{\textcolor{gray}{e}}t. Mach’ Dir keine Sorge über die Zukunft. Dein Talent wird ſich immer ſtärker
               und ſchöner entwickeln. Aber ich ſetze den, wie Du ſelbſt zugeben
                  wirſt{[},{]} etwas unwahrſcheinlichen, Fall, daß \strikeout{Du} Du fortan nur mehr lauter Stücke \textsc{à la}{ }\textsc{Rudolf Lothar\pwindex{Lothar, Rudolf 23.2.1865 – 2.10.1943@\textsc{Lothar, Rudolf} (23.2.1865 – 2.10.1943), \emph{Schriftsteller, Journalist, Theaterdirektor}|pw}} zuſtande bringſt, ſo würde ſelbſt das nichts machen. Du haſt bereits ein Werk
               geſchaffen, das {\pb}bleiben wird, und ſelbſt wenn Du
               gar nichts mehr ſchriebeſt, hätteſt Du Deinen Platz in der deutſchen Literatur
               geſichert. Ich meine alſo, Du kannſt ganz ruhig ſein\strikeout{,}
               und kannſt die Zweifel zum Teufel jagen, wenn ſie kommen.\pend
           \pstart
           Es war ſehr lieb von Dir, mir noch kurz vor der \textsc{\begin{otherlanguage}{french}Première\pwindex{Schnitzler, Arthur 15.05.1862 – 21.10.1931@\textsc{Schnitzler, Arthur} (15.05.1862 – 21.10.1931), \emph{Schriftsteller, Mediziner}!Liebelei. Schauspiel in drei Akten1895-10-09@\strich\emph{Liebelei. Schauspiel in drei Akten} {[}1895-10-09{]}|pwv}\end{otherlanguage}} zu ſchreiben. Deine Berlin\oindex{Berlin@\textbf{Berlin}|pw}er \label{K_L02767-88v}\edtext{Perſonal-Eindrücke halte ich nicht für
               ganz zutreffend. \textsc{Harden\pwindex{Harden, Maximilian 20.10.1861 – 30.10.1927@\textsc{Harden, Maximilian} (20.10.1861 – 30.10.1927), \emph{Schriftsteller, Publizist}|pw}}}{\lemma{\textnormal{\emph{Perſonal-Eindrücke … Harden}}}\Cendnote{\textnormal{siehe A. S.: \emph{Tagebuch}, 4. 2. 1896}}}\label{K_L02767-88h} mag \strikeout{eine beſtr} ein beſtrickender Menſch ſein,
               aber ein »Freier« {\pb}iſt er nicht, ſondern ein Streber
               ohne Moral und Gewiſſen. Freilich ein großes Talent. Aber vielleicht muß man ſo ſein?
               Vielleicht iſt es Kraft, wenn man ſo iſt? Die Schwachen, die hinten bleiben, kommen
               dann mit der Moral, und das iſt vielleicht ſehr albern.\pend
           \pstart
           Ich habe geſtern, mit Deiner \label{K_L02767-77v}\edtext{Depeſche}{\lemma{\textnormal{\emph{Depeſche}}}\Cendnote{\textnormal{gemeint war wohl ein Eilbrief, in dem ein positiver Bericht zur Berliner\oindex{Berlin@\textbf{Berlin}|pwk}{ }Premiere\pwindex{Schnitzler, Arthur 15.05.1862 – 21.10.1931@\textsc{Schnitzler, Arthur} (15.05.1862 – 21.10.1931), \emph{Schriftsteller, Mediziner}!Liebelei. Schauspiel in drei Akten1895-10-09@\strich\emph{Liebelei. Schauspiel in drei Akten} {[}1895-10-09{]}|pwkv} enthalten
                  war}}}\label{K_L02767-77h} in der Hand, einen Schritt beim »\textsc{Figaro\orgindex{Le Figaro@Le Figaro|pw}}« gethan, den ich mir für \strikeout{\textcolor{gray}{einen}} den entſcheidenden Moment {\pb}aufgeſpart hatte.
               Da iſt es nämlich unendlich ſchwer, \strikeout{mit} eine Notiz
               anzubringen, weil die Leute\orgindex{Le Figaro@Le Figaro|pwv} das
               Bewußtſein ihrer ungeheuren Publicität haben und gewohnt ſind, daß man \strikeout{es} ihnen zahlt. Nichtsdeſtoweniger iſt es mir gelungen,
               ein paar \label{K_L02767-1v}\edtext{Zeilen\pwindex{Courrier des Theatres [Liebelei-Premiere Berlin]1896-02-06@\emph{Courrier des Théatres [Liebelei-Premiere Berlin]} {[}1896-02-06{]}|pwv}}{\lemma{\textnormal{\emph{Zeilen}}}\Cendnote{\textnormal{[Jules Huret\pwindex{Huret, Jules 1863-04-08 – 1915-02-14@\textsc{Huret, Jules} (1863-04-08 – 1915-02-14), \emph{Journalist, Publizist, Reiseschriftsteller}|pwk}]: \emph{Courrier des Théatres}\pwindex{Courrier des Theatres [Liebelei-Premiere Berlin]1896-02-06@\emph{Courrier des Théatres [Liebelei-Premiere Berlin]} {[}1896-02-06{]}|pwk}. In: \emph{Le Figaro}\pwindex{Le Figaro1826-01-15@\emph{Le Figaro} {[}1826-01-15{]}|pwk}, Jg. 42, Nr. 37, 6. 2. 1896, S. 4: »\begin{otherlanguage}{french}De Berlin\oindex{Berlin@\textbf{Berlin}|pw}: ›Le
                           Deutsches Theater\orgindex{Deutsches Theater Berlin@Deutsches Theater Berlin|pw} vient de jouer avec
                        un grand succès la comédie \emph{Liebelei\pwindex{Schnitzler, Arthur 15.05.1862 – 21.10.1931@\textsc{Schnitzler, Arthur} (15.05.1862 – 21.10.1931), \emph{Schriftsteller, Mediziner}!Liebelei. Schauspiel in drei Akten1895-10-09@\strich\emph{Liebelei. Schauspiel in drei Akten} {[}1895-10-09{]}|pw}} (Le Badinage amoureux\pwindex{Schnitzler, Arthur 15.05.1862 – 21.10.1931@\textsc{Schnitzler, Arthur} (15.05.1862 – 21.10.1931), \emph{Schriftsteller, Mediziner}!Liebelei. Schauspiel in drei Akten1895-10-09@\strich\emph{Liebelei. Schauspiel in drei Akten} {[}1895-10-09{]}|pw}) de M. Arthur Schnitzler\pwindex{Schnitzler, Arthur 15.05.1862 – 21.10.1931@\textsc{Schnitzler, Arthur} (15.05.1862 – 21.10.1931), \emph{Schriftsteller, Mediziner}|pw}, un jeune auteur vienn\oindex{Wien@\textbf{Wien}|pwv}ois. La comédie\pwindex{Schnitzler, Arthur 15.05.1862 – 21.10.1931@\textsc{Schnitzler, Arthur} (15.05.1862 – 21.10.1931), \emph{Schriftsteller, Mediziner}!Liebelei. Schauspiel in drei Akten1895-10-09@\strich\emph{Liebelei. Schauspiel in drei Akten} {[}1895-10-09{]}|pwv}, qui raconte,
                        en trois actes tantôt gais, tantôt dramatiques, les amours d’une petite
                        grisette vienn\oindex{Wien@\textbf{Wien}|pwv}oise avec
                        un jeune homme du monde, qui vit et meurt pour une autre, a été représentée
                        au Burgtheater\oindex{Burgtheater@\textbf{Burgtheater}|pw} de Vienne\oindex{Wien@\textbf{Wien}|pw} au commencement de cette saison et y tient
                        l’affiche depuis. Le public berlin\oindex{Berlin@\textbf{Berlin}|pw}ois,
                        qui vient de ratifier le jugement de celui de Vienne\oindex{Wien@\textbf{Wien}|pw}, a fait un accueil chaleureux à l’auteur\pwindex{Schnitzler, Arthur 15.05.1862 – 21.10.1931@\textsc{Schnitzler, Arthur} (15.05.1862 – 21.10.1931), \emph{Schriftsteller, Mediziner}|pwv} de \emph{Liebelei\pwindex{Schnitzler, Arthur 15.05.1862 – 21.10.1931@\textsc{Schnitzler, Arthur} (15.05.1862 – 21.10.1931), \emph{Schriftsteller, Mediziner}!Liebelei. Schauspiel in drei Akten1895-10-09@\strich\emph{Liebelei. Schauspiel in drei Akten} {[}1895-10-09{]}|pw}}. La critique berlin\oindex{Berlin@\textbf{Berlin}|pw}oise apprécie
                        également la pièce\pwindex{Schnitzler, Arthur 15.05.1862 – 21.10.1931@\textsc{Schnitzler, Arthur} (15.05.1862 – 21.10.1931), \emph{Schriftsteller, Mediziner}!Liebelei. Schauspiel in drei Akten1895-10-09@\strich\emph{Liebelei. Schauspiel in drei Akten} {[}1895-10-09{]}|pwv} en
                        termes fort élogieux.‹\end{otherlanguage}« (»›Das Deutsche Theater\orgindex{Deutsches Theater Berlin@Deutsches Theater Berlin|pw} in
                        Berlin\oindex{Berlin@\textbf{Berlin}|pw} hat soeben mit großem Erfolg die
                     Komödie Liebelei\pwindex{Schnitzler, Arthur 15.05.1862 – 21.10.1931@\textsc{Schnitzler, Arthur} (15.05.1862 – 21.10.1931), \emph{Schriftsteller, Mediziner}!Liebelei. Schauspiel in drei Akten1895-10-09@\strich\emph{Liebelei. Schauspiel in drei Akten} {[}1895-10-09{]}|pw} des jungen Wien\oindex{Wien@\textbf{Wien}|pw}er Autors Arthur Schnitzler\pwindex{Schnitzler, Arthur 15.05.1862 – 21.10.1931@\textsc{Schnitzler, Arthur} (15.05.1862 – 21.10.1931), \emph{Schriftsteller, Mediziner}|pw} aufgeführt. Die Komödie\pwindex{Schnitzler, Arthur 15.05.1862 – 21.10.1931@\textsc{Schnitzler, Arthur} (15.05.1862 – 21.10.1931), \emph{Schriftsteller, Mediziner}!Liebelei. Schauspiel in drei Akten1895-10-09@\strich\emph{Liebelei. Schauspiel in drei Akten} {[}1895-10-09{]}|pwv}, die in drei teils heiteren,
                     teils dramatischen Akten von der Liebe eines kleinen Wien\oindex{Wien@\textbf{Wien}|pw}er Mädchens zu einem jungen Mann von Welt erzählt, der
                     für eine andere lebt und stirbt, wurde zu Beginn dieser Spielzeit im Wien\oindex{Wien@\textbf{Wien}|pw}er Burgtheater\oindex{Burgtheater@\textbf{Burgtheater}|pw} aufgeführt und steht seitdem dort auf dem Spielplan. Das
                        Berlin\oindex{Berlin@\textbf{Berlin}|pw}er Publikum, das gerade das Urteil
                     des Wien\oindex{Wien@\textbf{Wien}|pw}er Publikums bestätigt hat, hat dem
                        Autor\pwindex{Schnitzler, Arthur 15.05.1862 – 21.10.1931@\textsc{Schnitzler, Arthur} (15.05.1862 – 21.10.1931), \emph{Schriftsteller, Mediziner}|pwv} der Liebelei\pwindex{Schnitzler, Arthur 15.05.1862 – 21.10.1931@\textsc{Schnitzler, Arthur} (15.05.1862 – 21.10.1931), \emph{Schriftsteller, Mediziner}!Liebelei. Schauspiel in drei Akten1895-10-09@\strich\emph{Liebelei. Schauspiel in drei Akten} {[}1895-10-09{]}|pw} einen herzlichen Empfang bereitet.
                     Auch die Berlin\oindex{Berlin@\textbf{Berlin}|pw}er Kritiker bewerteten das
                        Stück\pwindex{Schnitzler, Arthur 15.05.1862 – 21.10.1931@\textsc{Schnitzler, Arthur} (15.05.1862 – 21.10.1931), \emph{Schriftsteller, Mediziner}!Liebelei. Schauspiel in drei Akten1895-10-09@\strich\emph{Liebelei. Schauspiel in drei Akten} {[}1895-10-09{]}|pwv} sehr
                     lobend.‹«)}}}\label{K_L02767-1h} über Dich hineinzubringen, und das hat für die Pariſ\oindex{Paris@\textbf{Paris}|pw}er Aufführungs-Projecte den größten Werth.
                  {\pb}Bitte, nimm eine Karte, adreſſire ſie an \textsc{\begin{otherlanguage}{french}M. Jules Huret\pwindex{Huret, Jules 1863-04-08 – 1915-02-14@\textsc{Huret, Jules} (1863-04-08 – 1915-02-14), \emph{Journalist, Publizist, Reiseschriftsteller}|pw} du
                        »Figaro\orgindex{Le Figaro@Le Figaro|pw}«, Rue Drouot, Paris\oindex{Rue Drouot@\textbf{Rue Drouot}|pw}\end{otherlanguage}} und ſchreibe darauf etwas wie: \label{K_L02767-8v}\edtext{\begin{otherlanguage}{french}\textsc{remercie bien vivement M. Huret\pwindex{Huret, Jules 1863-04-08 – 1915-02-14@\textsc{Huret, Jules} (1863-04-08 – 1915-02-14), \emph{Journalist, Publizist, Reiseschriftsteller}|pw} de la \strikeout{\textcolor{gray}{n}}{ }note\pwindex{Courrier des Theatres [Liebelei-Premiere Berlin]1896-02-06@\emph{Courrier des Théatres [Liebelei-Premiere Berlin]} {[}1896-02-06{]}|pwv}, qu’il a eu
                     l’amabilité d’insérer au sujet de la représentation de »Liebelei\pwindex{Schnitzler, Arthur 15.05.1862 – 21.10.1931@\textsc{Schnitzler, Arthur} (15.05.1862 – 21.10.1931), \emph{Schriftsteller, Mediziner}!Liebelei. Schauspiel in drei Akten1895-10-09@\strich\emph{Liebelei. Schauspiel in drei Akten} {[}1895-10-09{]}|pw}« à Berlin\oindex{Berlin@\textbf{Berlin}|pw}.}\end{otherlanguage}}{\lemma{\textnormal{\emph{remercie … Berlin.}}}\Cendnote{\textnormal{französisch: [Arthur Schnitzler\pwindex{Schnitzler, Arthur 15.05.1862 – 21.10.1931@\textsc{Schnitzler, Arthur} (15.05.1862 – 21.10.1931), \emph{Schriftsteller, Mediziner}|pw}] dankt Herrn Huret\pwindex{Huret, Jules 1863-04-08 – 1915-02-14@\textsc{Huret, Jules} (1863-04-08 – 1915-02-14), \emph{Journalist, Publizist, Reiseschriftsteller}|pw} herzlich für die Notiz\pwindex{Courrier des Theatres [Liebelei-Premiere Berlin]1896-02-06@\emph{Courrier des Théatres [Liebelei-Premiere Berlin]} {[}1896-02-06{]}|pwv}, die er freundlicherweise über
                     die Aufführung der Liebelei\pwindex{Schnitzler, Arthur 15.05.1862 – 21.10.1931@\textsc{Schnitzler, Arthur} (15.05.1862 – 21.10.1931), \emph{Schriftsteller, Mediziner}!Liebelei. Schauspiel in drei Akten1895-10-09@\strich\emph{Liebelei. Schauspiel in drei Akten} {[}1895-10-09{]}|pw} in Berlin\oindex{Berlin@\textbf{Berlin}|pw} eingefügt hat}}}\label{K_L02767-8h} Anbei erhältſt Du den »\textsc{Figaro}\pwindex{Le Figaro1826-01-15@\emph{Le Figaro} {[}1826-01-15{]}|pw}« (Theater-Rubrik). Ich bin ſehr {\pb}ſtolz auf
               meinen franzöſiſchen Styl.\pend
           \pstart
           Grüß Dich Gott, mein lieber Freund!\pend
           \pstart
           In Treue {\\[\baselineskip]}Dein {\\[\baselineskip]}\spacefill\mbox{Paul Goldmann.}\pend
           \leftskip=0em{}
         
         \endnumbering\mylabel{h}\end{ledgroupsized}  \newcommand{\dateiname}{L02767}\newcommand{\titel}{Paul Goldmann an Arthur Schnitzler, 6. 2. [1896]}\newcommand{\editorInnen}{Martin Anton Müller und Laura Untner}%% latex-leseansicht-abspann.tex
%% Abspann für die Leseansicht.
%% Der Schalter \ifkorrekturansicht ist bereits durch den Vorspann gesetzt.

%% latex-abspann.tex
%% Gemeinsamer Abspann für Korrekturansicht und Leseansicht.
%% Setzt den Schalter \ifkorrekturansicht voraus (gesetzt in den
%% einbindenden Dateien latex-korrekturansicht-abspann.tex bzw.
%% latex-leseansicht-abspann.tex).
%% ---------------------------------------------------------------

\normalsize

% Das esempio-Environment wird nur in der Leseansicht benötigt
\ifkorrekturansicht\else
\newenvironment{esempio}[3]%
{
    \vspace{1.5ex}
    \rlap{\underline{#1}}
    \par
    \setlength{\parindent}{0cm}
    \nopagebreak
    \leftskip=#2cm
    \rightskip=#3cm
}
{
    \par
}
\fi

\doendnotes{C}
\bigskip
\vfill

\clearpage

\footnotesize

\ifkorrekturansicht
  \lohead{\textsc{register}}
\fi

% theindex-Environment neu definieren ohne reledmac
\makeatletter
\renewenvironment{theindex}{%
  \ifkorrekturansicht
    \section*{\indexname}%
  \else
    \subsubsection*{Index der erwähnten Entitäten}%
  \fi
  \setlength{\parindent}{0pt}%
  \setlength{\parskip}{0pt plus 0.3pt}%
  \let\item\@idxitem
}{%
  \ifkorrekturansicht\clearpage\fi
}
\makeatother

\IfFileExists{\jobname-pw.ind}{\input{\jobname-pw.ind}}{}

% Quellenangabe nur in der Leseansicht
\ifkorrekturansicht\else
% Fallback-Definitionen, falls die .tex-Datei \titel etc. nicht gesetzt hat
\providecommand{\titel}{}
\providecommand{\editorInnen}{}
\providecommand{\dateiname}{\jobname}

\vspace{3cm}

\vfill

\footnotesize
\textsc{Quelle}: \titel. Herausgegeben von {\editorInnen}. In: \emph{Arthur Schnitzler: Briefwechsel mit Autorinnen und Autoren}.
 Digitale Edition, https://schnitzler-briefe.acdh.oeaw.ac.at/{\dateiname}.html (Stand \today)
\fi

\end{document}


      