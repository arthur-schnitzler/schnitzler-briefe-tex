%% latex-leseansicht-vorspann.tex
%% Vorspann für die Leseansicht.
%% Lädt die gemeinsame Datei latex-vorspann.tex mit nicht gesetztem Schalter.

\newif\ifkorrekturansicht
\korrekturansichtfalse

\input{../tex-inputs/latex-vorspann}


\section[Paul Goldmann an Arthur Schnitzler, 6. 2. {[}1896{]}]{L02767 Paul Goldmann an Arthur Schnitzler, 6. 2. [1896]}
\nopagebreak\mylabel{L02767v}
\rehead{ }\normalsize\beginnumbering\briefempfaengerindex{Schnitzler, Arthur@\textsc{Schnitzler, Arthur}!zzzGoldmann, Paul@\emph{von Paul Goldmann}!1896-02-061@{6. 2. [1896]}|(be}
\toendnotes[C]{\smallbreak\pagebreak[2]}
\correspDesc{Versand  durch Paul Goldmann am 6. 2. [1896] in Paris
\newline{}Erhalt  durch Arthur Schnitzler am [11. 2. 1896?] in Wien}\toendnotes[C]{\smallbreak}
\Standort{DLA, A:Schnitzler, HS.NZ85.1.3166.}
\physDesc{Brief, 2 Blätter, 7 Seiten, 2349 Zeichen
\newline{}Handschrift: blaue Tinte, deutsche Kurrent
\newline{}Schnitzler: 1) mit Bleistift das Jahr »96« vermerkt  2) mit rotem Buntstift eine Unterstreichung}\toendnotes[C]{\smallbreak}
\pstart
           {\pb}\textcolor{gray}{\textbf{\textbf{Frankfurter Zeitung\orgindex{Frankfurter Zeitung@Frankfurter Zeitung|pw}}}}\pend
           
\pstart
           \textcolor{gray}{\textbf{(\begin{otherlanguage}{french}Gazette de Francfort\end{otherlanguage}\orgindex{Frankfurter Zeitung@Frankfurter Zeitung|pw}).}}\pend
           
\pstart
           \textcolor{gray}{\textbf{\textbf{\begin{otherlanguage}{french}Fondateur M.\end{otherlanguage}{ }L. Sonnemann\pwindex{Sonnemann, Leopold 29.\,10.\,1831 Höchberg – 30.\,10.\,1909 Frankfurt am Main@\textsc{Sonnemann, Leopold} (29.\,10.\,1831 Höchberg – 30.\,10.\,1909 Frankfurt am Main), \emph{Journalist, Herausgeber}|pw}.}}}\pend
           
\pstart
           \begin{otherlanguage}{french}\textcolor{gray}{\textbf{Journal\pwindex{Frankfurter Zeitung@\emph{Frankfurter Zeitung}|pwv} politique,
                        financier,}}\end{otherlanguage}\pend
           
\pstart
           \begin{otherlanguage}{french}\textcolor{gray}{\textbf{commercial et littéraire.}}\end{otherlanguage}\pend
           
\pstart
           \begin{otherlanguage}{french}\textcolor{gray}{\textbf{\textbf{Paraissant trois fois par jour.}}}\end{otherlanguage}\pend
           
\pstart
           \begin{otherlanguage}{french}\textcolor{gray}{\textbf{\textbf{Bureau à Paris\oindex{Paris@\textbf{Paris}, \emph{Hauptstadt}|pw}:}}}\end{otherlanguage}\hfill \textsc{Paris\oindex{Paris@\textbf{Paris}, \emph{Hauptstadt}|pw}}, 6. Februar.\pend
           
\pstart
           \begin{otherlanguage}{french}\textcolor{gray}{\textbf{\textbf{24. Rue Feydeau\oindex{rue Feydeau@\textbf{rue Feydeau}, \emph{Straße}|pw}.}}}\end{otherlanguage}\pend
           
\pstart\center{}Mein lieber Freund,\pend\vspace{0.5em}
\pstart
           Ich{ }ſchreibe Dir nicht nach \textsc{Berlin\oindex{Berlin@\textbf{Berlin}, \emph{Hauptstadt}|pw}}, weil ich nicht weiß, ob mein Brief Dich noch dort erreicht.\pend
           
\pstart
           Alſo nochmals: innigen Glückwünſch! Nun biſt Du ganz und gar ein gemachter Mann.
               Selbſt dem{ }ſkeptiſchen und kalten Berlin\oindex{Berlin@\textbf{Berlin}, \emph{Hauptstadt}|pw} haſt Du
               gefallen. Jetzt wird das Stück\pwindex{Schnitzler, Arthur 15.\,5.\,1862 Wien – 21.\,10.\,1931 ebd.@\textsc{Schnitzler, Arthur} (15.\,5.\,1862 Wien – 21.\,10.\,1931 ebd.), \emph{Schriftsteller, Mediziner}!Liebelei. Schauspiel in drei Akten@\strich\emph{Liebelei. Schauspiel in drei Akten}|pwv}
               durch ganz Deutſchland\oindex{Deutschland@\textbf{Deutschland}|pw} gehen\strikeout{,} und Du biſt heut, in Deinen jungen Jahren, einer der
               erſten deutſchen Bühnendichter. {\pb}Das war zwar Alles
               vorauszuſehen; aber es iſt doch herrlich, daß man es \strikeout{e\textcolor{gray}{r}} erleb\strikeout{\textcolor{gray}{e}}t. Mach’ Dir keine Sorge über die Zukunft. Dein Talent wird{ }ſich immer{ }ſtärker
               und{ }ſchöner entwickeln. Aber ich{ }ſetze den, wie Du{ }ſelbſt zugeben
                  wirſt{[},{]} etwas unwahrſcheinlichen, Fall, daß \strikeout{Du} Du fortan nur mehr lauter Stücke \textsc{à la}{ }\textsc{Rudolf Lothar\pwindex{Lothar, Rudolf 23.\,2.\,1865 Budapest – 2.\,10.\,1943 ebd.@\textsc{Lothar, Rudolf} (23.\,2.\,1865 Budapest – 2.\,10.\,1943 ebd.), \emph{Schriftsteller, Journalist, Theaterdirektor}|pw}} zuſtande bringſt,{ }ſo würde{ }ſelbſt das nichts machen. Du haſt bereits ein Werk
               geſchaffen, das {\pb}bleiben wird, und{ }ſelbſt wenn Du
               gar nichts mehr{ }ſchriebeſt, hätteſt Du Deinen Platz in der deutſchen Literatur
               geſichert. Ich meine alſo, Du kannſt ganz ruhig{ }ſein\strikeout{,}
               und kannſt die Zweifel zum Teufel jagen, wenn{ }ſie kommen.\pend
           
\pstart
           Es war{ }ſehr lieb von Dir, mir noch kurz vor der \textsc{\begin{otherlanguage}{french}Première\pwindex{Schnitzler, Arthur 15.\,5.\,1862 Wien – 21.\,10.\,1931 ebd.@\textsc{Schnitzler, Arthur} (15.\,5.\,1862 Wien – 21.\,10.\,1931 ebd.), \emph{Schriftsteller, Mediziner}!Liebelei. Schauspiel in drei Akten@\strich\emph{Liebelei. Schauspiel in drei Akten}|pwv}\end{otherlanguage}} zu{ }ſchreiben. Deine Berlin\oindex{Berlin@\textbf{Berlin}, \emph{Hauptstadt}|pw}er \label{K_L02767-1v}\edtext{Perſonal-Eindrücke halte ich nicht für
               ganz zutreffend. \textsc{Harden\pwindex{Harden, Maximilian 20.\,10.\,1861 Berlin – 30.\,10.\,1927 Montana@\textsc{Harden, Maximilian} (20.\,10.\,1861 Berlin – 30.\,10.\,1927 Montana), \emph{Schriftsteller, Publizist}|pw}}}{\lemma{\textnormal{\emph{Personal-Eindrücke … Harden}}}\Cendnote{\textnormal{Siehe A. S.: \emph{Tagebuch}, 4. 2. 1896.
               }}}\label{K_L02767-1} mag \strikeout{eine beſtr} ein beſtrickender Menſch{ }ſein,
               aber ein »Freier« {\pb}iſt er nicht,{ }ſondern ein Streber
               ohne Moral und Gewiſſen. Freilich ein großes Talent. Aber vielleicht muß man{ }ſo{ }ſein?
               Vielleicht iſt es Kraft, wenn man{ }ſo iſt? Die Schwachen, die hinten bleiben, kommen
               dann mit der Moral, und das iſt vielleicht{ }ſehr albern.\pend
           
\pstart
           Ich habe geſtern, mit Deiner \label{K_L02767-2v}\edtext{Depeſche}{\lemma{\textnormal{\emph{Depesche}}}\Cendnote{\textnormal{Gemeint war wohl ein Eilbrief, in dem ein positiver Bericht zur Berliner\oindex{Berlin@\textbf{Berlin}, \emph{Hauptstadt}|pwk}{ }Premiere\pwindex{Schnitzler, Arthur 15.\,5.\,1862 Wien – 21.\,10.\,1931 ebd.@\textsc{Schnitzler, Arthur} (15.\,5.\,1862 Wien – 21.\,10.\,1931 ebd.), \emph{Schriftsteller, Mediziner}!Liebelei. Schauspiel in drei Akten@\strich\emph{Liebelei. Schauspiel in drei Akten}|pwkv} enthalten
                  war.}}}\label{K_L02767-2} in der Hand, einen Schritt beim »\textsc{Figaro\orgindex{Le Figaro@Le Figaro|pw}}« gethan, den ich mir für \strikeout{\textcolor{gray}{einen}} den entſcheidenden Moment {\pb}aufgeſpart hatte.
               Da iſt es nämlich unendlich{ }ſchwer, \strikeout{mit} eine Notiz
               anzubringen, weil die Leute\orgindex{Le Figaro@Le Figaro|pwv} das
               Bewußtſein ihrer ungeheuren Publicität haben und gewohnt{ }ſind, daß man \strikeout{es} ihnen zahlt. Nichtsdeſtoweniger iſt es mir gelungen,
               ein paar \label{K_L02767-3v}\edtext{Zeilen\pwindex{Courrier des Théâtres [Liebelei-Premiere Berlin]@\emph{Courrier des Théâtres [Liebelei-Premiere Berlin]}|pwv}}{\lemma{\textnormal{\emph{Zeilen}}}\Cendnote{\textnormal{[Jules Huret\pwindex{Huret, Jules 8.\,4.\,1863 Boulogne-sur-Mer – 14.\,2.\,1915 Paris@\textsc{Huret, Jules} (8.\,4.\,1863 Boulogne-sur-Mer – 14.\,2.\,1915 Paris), \emph{Schriftsteller, Journalist, Publizist}|pwk}]: \emph{Courrier des Théâtres}\pwindex{Courrier des Théâtres [Liebelei-Premiere Berlin]@\emph{Courrier des Théâtres [Liebelei-Premiere Berlin]}|pwk}. In: \emph{Le Figaro}\pwindex{Le Figaro@\emph{Le Figaro}|pwk}, Jg. 42, Nr. 37, 6. 2. 1896, S. 4: »\begin{otherlanguage}{french}De Berlin\oindex{Berlin@\textbf{Berlin}, \emph{Hauptstadt}|pw}: ›Le
                           Deutsches Theater\orgindex{Deutsches Theater Berlin@Deutsches Theater Berlin|pw} vient de jouer avec
                        un grand succès la comédie \emph{Liebelei\pwindex{Schnitzler, Arthur 15.\,5.\,1862 Wien – 21.\,10.\,1931 ebd.@\textsc{Schnitzler, Arthur} (15.\,5.\,1862 Wien – 21.\,10.\,1931 ebd.), \emph{Schriftsteller, Mediziner}!Liebelei. Schauspiel in drei Akten@\strich\emph{Liebelei. Schauspiel in drei Akten}|pw}} (Le Badinage amoureux\pwindex{Schnitzler, Arthur 15.\,5.\,1862 Wien – 21.\,10.\,1931 ebd.@\textsc{Schnitzler, Arthur} (15.\,5.\,1862 Wien – 21.\,10.\,1931 ebd.), \emph{Schriftsteller, Mediziner}!Liebelei. Schauspiel in drei Akten@\strich\emph{Liebelei. Schauspiel in drei Akten}|pw}) de M. Arthur Schnitzler, un jeune auteur vienn\oindex{Wien@\textbf{Wien}, \emph{Verwaltungsgebiet}|pwv}ois. La comédie\pwindex{Schnitzler, Arthur 15.\,5.\,1862 Wien – 21.\,10.\,1931 ebd.@\textsc{Schnitzler, Arthur} (15.\,5.\,1862 Wien – 21.\,10.\,1931 ebd.), \emph{Schriftsteller, Mediziner}!Liebelei. Schauspiel in drei Akten@\strich\emph{Liebelei. Schauspiel in drei Akten}|pwv}, qui raconte,
                        en trois actes tantôt gais, tantôt dramatiques, les amours d’une petite
                        grisette vienn\oindex{Wien@\textbf{Wien}, \emph{Verwaltungsgebiet}|pwv}oise avec
                        un jeune homme du monde, qui vit et meurt pour une autre, a été représentée
                        au Burgtheater\oindex{Wien@\textbf{Wien}!I., Innere Stadt@\textbf{I., Innere Stadt}!Burgtheater@\textbf{Burgtheater}, \emph{Theater}|pw} de Vienne\oindex{Wien@\textbf{Wien}, \emph{Verwaltungsgebiet}|pw} au commencement de cette saison et y tient
                        l’affiche depuis. Le public berlin\oindex{Berlin@\textbf{Berlin}, \emph{Hauptstadt}|pw}ois,
                        qui vient de ratifier le jugement de celui de Vienne\oindex{Wien@\textbf{Wien}, \emph{Verwaltungsgebiet}|pw}, a fait un accueil chaleureux à l’auteur de \emph{Liebelei\pwindex{Schnitzler, Arthur 15.\,5.\,1862 Wien – 21.\,10.\,1931 ebd.@\textsc{Schnitzler, Arthur} (15.\,5.\,1862 Wien – 21.\,10.\,1931 ebd.), \emph{Schriftsteller, Mediziner}!Liebelei. Schauspiel in drei Akten@\strich\emph{Liebelei. Schauspiel in drei Akten}|pw}}. La critique berlin\oindex{Berlin@\textbf{Berlin}, \emph{Hauptstadt}|pw}oise apprécie
                        également la pièce\pwindex{Schnitzler, Arthur 15.\,5.\,1862 Wien – 21.\,10.\,1931 ebd.@\textsc{Schnitzler, Arthur} (15.\,5.\,1862 Wien – 21.\,10.\,1931 ebd.), \emph{Schriftsteller, Mediziner}!Liebelei. Schauspiel in drei Akten@\strich\emph{Liebelei. Schauspiel in drei Akten}|pwv} en
                        termes fort élogieux.‹\end{otherlanguage}« (»›Das Deutsche Theater\orgindex{Deutsches Theater Berlin@Deutsches Theater Berlin|pw} in
                        Berlin\oindex{Berlin@\textbf{Berlin}, \emph{Hauptstadt}|pw} hat soeben mit großem Erfolg die
                     Komödie Liebelei\pwindex{Schnitzler, Arthur 15.\,5.\,1862 Wien – 21.\,10.\,1931 ebd.@\textsc{Schnitzler, Arthur} (15.\,5.\,1862 Wien – 21.\,10.\,1931 ebd.), \emph{Schriftsteller, Mediziner}!Liebelei. Schauspiel in drei Akten@\strich\emph{Liebelei. Schauspiel in drei Akten}|pw} des jungen Wien\oindex{Wien@\textbf{Wien}, \emph{Verwaltungsgebiet}|pw}er Autors Arthur Schnitzler aufgeführt. Die Komödie\pwindex{Schnitzler, Arthur 15.\,5.\,1862 Wien – 21.\,10.\,1931 ebd.@\textsc{Schnitzler, Arthur} (15.\,5.\,1862 Wien – 21.\,10.\,1931 ebd.), \emph{Schriftsteller, Mediziner}!Liebelei. Schauspiel in drei Akten@\strich\emph{Liebelei. Schauspiel in drei Akten}|pwv}, die in drei teils heiteren,
                     teils dramatischen Akten von der Liebe eines kleinen Wien\oindex{Wien@\textbf{Wien}, \emph{Verwaltungsgebiet}|pw}er Mädchens zu einem jungen Mann von Welt erzählt, der
                     für eine andere lebt und stirbt, wurde zu Beginn dieser Spielzeit im Wien\oindex{Wien@\textbf{Wien}, \emph{Verwaltungsgebiet}|pw}er Burgtheater\oindex{Wien@\textbf{Wien}!I., Innere Stadt@\textbf{I., Innere Stadt}!Burgtheater@\textbf{Burgtheater}, \emph{Theater}|pw} aufgeführt und steht seitdem dort auf dem Spielplan. Das
                        Berlin\oindex{Berlin@\textbf{Berlin}, \emph{Hauptstadt}|pw}er Publikum, das gerade das Urteil
                     des Wien\oindex{Wien@\textbf{Wien}, \emph{Verwaltungsgebiet}|pw}er Publikums bestätigt hat, hat dem
                        Autor der Liebelei\pwindex{Schnitzler, Arthur 15.\,5.\,1862 Wien – 21.\,10.\,1931 ebd.@\textsc{Schnitzler, Arthur} (15.\,5.\,1862 Wien – 21.\,10.\,1931 ebd.), \emph{Schriftsteller, Mediziner}!Liebelei. Schauspiel in drei Akten@\strich\emph{Liebelei. Schauspiel in drei Akten}|pw} einen herzlichen Empfang bereitet.
                     Auch die Berlin\oindex{Berlin@\textbf{Berlin}, \emph{Hauptstadt}|pw}er Kritiker bewerteten das
                        Stück\pwindex{Schnitzler, Arthur 15.\,5.\,1862 Wien – 21.\,10.\,1931 ebd.@\textsc{Schnitzler, Arthur} (15.\,5.\,1862 Wien – 21.\,10.\,1931 ebd.), \emph{Schriftsteller, Mediziner}!Liebelei. Schauspiel in drei Akten@\strich\emph{Liebelei. Schauspiel in drei Akten}|pwv} sehr
                     lobend.‹«)}}}\label{K_L02767-3} über Dich hineinzubringen, und das hat für die Pariſ\oindex{Paris@\textbf{Paris}, \emph{Hauptstadt}|pw}er Aufführungs-Projecte den größten Werth.
                  {\pb}Bitte, nimm eine Karte, adreſſire{ }ſie an \textsc{\begin{otherlanguage}{french}M. Jules Huret\pwindex{Huret, Jules 8.\,4.\,1863 Boulogne-sur-Mer – 14.\,2.\,1915 Paris@\textsc{Huret, Jules} (8.\,4.\,1863 Boulogne-sur-Mer – 14.\,2.\,1915 Paris), \emph{Schriftsteller, Journalist, Publizist}|pw} du
                        »Figaro\orgindex{Le Figaro@Le Figaro|pw}«, Rue Drouot, Paris\oindex{Rue Drouot@\textbf{Rue Drouot}, \emph{Straße}|pw}\end{otherlanguage}} und{ }ſchreibe darauf etwas wie: \label{K_L02767-4v}\edtext{\begin{otherlanguage}{french}\textsc{remercie bien vivement M. Huret\pwindex{Huret, Jules 8.\,4.\,1863 Boulogne-sur-Mer – 14.\,2.\,1915 Paris@\textsc{Huret, Jules} (8.\,4.\,1863 Boulogne-sur-Mer – 14.\,2.\,1915 Paris), \emph{Schriftsteller, Journalist, Publizist}|pw} de la \strikeout{\textcolor{gray}{n}}{ }note\pwindex{Courrier des Théâtres [Liebelei-Premiere Berlin]@\emph{Courrier des Théâtres [Liebelei-Premiere Berlin]}|pwv}, qu’il a eu
                     l’amabilité d’insérer au sujet de la représentation de »Liebelei\pwindex{Schnitzler, Arthur 15.\,5.\,1862 Wien – 21.\,10.\,1931 ebd.@\textsc{Schnitzler, Arthur} (15.\,5.\,1862 Wien – 21.\,10.\,1931 ebd.), \emph{Schriftsteller, Mediziner}!Liebelei. Schauspiel in drei Akten@\strich\emph{Liebelei. Schauspiel in drei Akten}|pw}« à Berlin\oindex{Berlin@\textbf{Berlin}, \emph{Hauptstadt}|pw}.}\end{otherlanguage}}{\lemma{\textnormal{\emph{remercie … Berlin.}}}\Cendnote{\textnormal{französisch: [Arthur Schnitzler] dankt Herrn Huret\pwindex{Huret, Jules 8.\,4.\,1863 Boulogne-sur-Mer – 14.\,2.\,1915 Paris@\textsc{Huret, Jules} (8.\,4.\,1863 Boulogne-sur-Mer – 14.\,2.\,1915 Paris), \emph{Schriftsteller, Journalist, Publizist}|pw} herzlich für die Notiz\pwindex{Courrier des Théâtres [Liebelei-Premiere Berlin]@\emph{Courrier des Théâtres [Liebelei-Premiere Berlin]}|pwv}, die er freundlicherweise über
                     die Aufführung der Liebelei\pwindex{Schnitzler, Arthur 15.\,5.\,1862 Wien – 21.\,10.\,1931 ebd.@\textsc{Schnitzler, Arthur} (15.\,5.\,1862 Wien – 21.\,10.\,1931 ebd.), \emph{Schriftsteller, Mediziner}!Liebelei. Schauspiel in drei Akten@\strich\emph{Liebelei. Schauspiel in drei Akten}|pw} in Berlin\oindex{Berlin@\textbf{Berlin}, \emph{Hauptstadt}|pw} eingefügt hat.}}}\label{K_L02767-4} Anbei erhältſt Du den »\textsc{Figaro}\pwindex{Le Figaro@\emph{Le Figaro}|pw}« (Theater-Rubrik). Ich bin{ }ſehr {\pb}ſtolz auf
               meinen franzöſiſchen Styl.\pend
           
\pstart
           Grüß Dich Gott, mein lieber Freund!\pend
           
\pstart
           In Treue {\\[\baselineskip]}Dein {\\[\baselineskip]}\spacefill\mbox{Paul Goldmann.}\pend
           \leftskip=0em{}\selectlanguage{ngerman}\endnumbering\briefempfaengerindex{Schnitzler, Arthur@\textsc{Schnitzler, Arthur}!zzzGoldmann, Paul@\emph{von Paul Goldmann}!1896-02-061@{6. 2. [1896]}|)be}\mylabel{L02767h}  \newcommand{\dateiname}{L02767}\newcommand{\titel}{Paul Goldmann an Arthur Schnitzler, 6. 2. [1896]}\newcommand{\editorInnen}{Martin Anton Müller und Laura Untner}%% latex-leseansicht-abspann.tex
%% Abspann für die Leseansicht.
%% Der Schalter \ifkorrekturansicht ist bereits durch den Vorspann gesetzt.

%% latex-abspann.tex
%% Gemeinsamer Abspann für Korrekturansicht und Leseansicht.
%% Setzt den Schalter \ifkorrekturansicht voraus (gesetzt in den
%% einbindenden Dateien latex-korrekturansicht-abspann.tex bzw.
%% latex-leseansicht-abspann.tex).
%% ---------------------------------------------------------------

\normalsize

% Das esempio-Environment wird nur in der Leseansicht benötigt
\ifkorrekturansicht\else
\newenvironment{esempio}[3]%
{
    \vspace{1.5ex}
    \rlap{\underline{#1}}
    \par
    \setlength{\parindent}{0cm}
    \nopagebreak
    \leftskip=#2cm
    \rightskip=#3cm
}
{
    \par
}
\fi

\doendnotes{C}
\bigskip
\vfill

\clearpage

\footnotesize

\ifkorrekturansicht
  \lohead{\textsc{register}}
\fi

% theindex-Environment neu definieren ohne reledmac
\makeatletter
\renewenvironment{theindex}{%
  \ifkorrekturansicht
    \section*{\indexname}%
  \else
    \subsubsection*{Index der erwähnten Entitäten}%
  \fi
  \setlength{\parindent}{0pt}%
  \setlength{\parskip}{0pt plus 0.3pt}%
  \let\item\@idxitem
}{%
  \ifkorrekturansicht\clearpage\fi
}
\makeatother

\IfFileExists{\jobname-pw.ind}{\input{\jobname-pw.ind}}{}

% Quellenangabe nur in der Leseansicht
\ifkorrekturansicht\else
% Fallback-Definitionen, falls die .tex-Datei \titel etc. nicht gesetzt hat
\providecommand{\titel}{}
\providecommand{\editorInnen}{}
\providecommand{\dateiname}{\jobname}

\vspace{3cm}

\vfill

\footnotesize
\textsc{Quelle}: \titel. Herausgegeben von {\editorInnen}. In: \emph{Arthur Schnitzler: Briefwechsel mit Autorinnen und Autoren}.
 Digitale Edition, https://schnitzler-briefe.acdh.oeaw.ac.at/{\dateiname}.html (Stand \today)
\fi

\end{document}


