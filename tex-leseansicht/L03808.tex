%% latex-korrekturansicht-vorspann.tex
%% Vorspann für die Korrekturansicht.
%% Lädt die gemeinsame Datei latex-vorspann.tex mit gesetztem Schalter.

\newif\ifkorrekturansicht
\korrekturansichttrue

\input{../tex-inputs/latex-vorspann}


\section[Arthur Schnitzler an Stefan Zweig, 22. 1. 1908]{L03808 Arthur Schnitzler an Stefan Zweig, 22. 1. 1908}
\nopagebreak\mylabel{L03808v}
\rehead{ }\normalsize\beginnumbering\briefempfaengerindex{Zweig, Stefan@\textsc{Zweig, Stefan}!zzzSchnitzler, Arthur@\emph{von Arthur Schnitzler}!1908-01-221@{22. 1. 1908}|(be}
\toendnotes[C]{\smallbreak\pagebreak[2]}\Standort{Jerusalem, National Library of Israel, ARC. Ms. Var. 305 1 58 Stefan Zweig Collection.}
\physDesc{Brief, 1 Blatt, 1 Seite, 346 Zeichen
\newline{}Handschrift: schwarze Tinte, lateinische Kurrent}\toendnotes[C]{\smallbreak}
\pstart
           {\pb}\textcolor{gray}{\textbf{Dr Arthur Schnitzler}}\hfill 22. 1. 908\pend
           
\pstart
           \textcolor{gray}{\textbf{Wien XVIII.
                        Spoettelgasse 7\oindex{Edmund-Weiss-Gasse@\textbf{Edmund-Weiß-Gasse}, \emph{R.ST}|pw}.}}\pend
           \vspace{0.5em}
\pstart
           verehrtester Herr Zweig,  lassen Sie mich für Ihren \label{K_L03808-1v}\edtext{Brief}{\lemma{\textnormal{\emph{Brief}}}\Cendnote{\textnormal{Stefan Zweig an Arthur Schnitzler, 15. 1. 190[8].}}}\label{K_L03808-1}, der
               mich besonders gefreut hat, herzlichst danken – und die Hoffnung aussprechen, dass
               Sie mir, aus Ihren schönen Versen und, leider nur \label{K_L03808-2v}\edtext{einzelnen Scenen des Thersites\pwindex{Tersites. Ein Trauerspiel in drei Aufzuegen@\emph{Tersites. Ein Trauerspiel in drei Aufzügen}|pw}}{\lemma{\textnormal{\emph{einzelnen … Thersites}}}\Cendnote{\textnormal{Ein teilweiser
                  Vorabdruck erschien in der \emph{Schaubühne}\pwindex{Schaubuehne@\emph{Die Schaubühne}|pwk}:
                        Stefan Zweig\pwindex{Zweig, Stefan 28.11.1881 – 23.02.1942@\textsc{Zweig, Stefan} (28.11.1881 – 23.02.1942), \emph{Schriftsteller/Schriftstellerin}|pwk}: \emph{Tersites}\pwindex{Tersites [Vorabdruck]@\emph{Tersites [Vorabdruck]}|pwk}. In: \emph{Die
                        Schaubühne}\pwindex{Schaubuehne@\emph{Die Schaubühne}|pwk}, Jg. 3, H. 33, 15. 8. 1907,
                     S. 125–130. Die vollständige Buchausgabe im \emph{Insel Verlag}\orgindex{Insel Verlag@Insel Verlag|pwk} erschien im März 1908, vgl. Arthur Schnitzler an Stefan Zweig, 23. 3. 1908.}}}\label{K_L03808-2}, längst
               beka{\geminationn}t, auch \label{K_L03808-11v}\edtext{persönlich nicht lang mehr ein Unbekannter}{\lemma{\textnormal{\emph{persönlich … Unbekannter}}}\Cendnote{\textnormal{Das erste persönliche
                  Treffen fand am 28. 5. 1908 statt.}}}\label{K_L03808-11} bleiben.\pend
           
\pstart
           Ihr aufrichtig ergebener{\\[\baselineskip]}\spacefill\mbox{Arthur Schnitzler}\pend
           \leftskip=0em{}\selectlanguage{ngerman}\endnumbering\briefempfaengerindex{Zweig, Stefan@\textsc{Zweig, Stefan}!zzzSchnitzler, Arthur@\emph{von Arthur Schnitzler}!1908-01-221@{22. 1. 1908}|)be}\mylabel{L03808h}  \normalsize

\doendnotes{C}
\bigskip
\vfill

\clearpage

\footnotesize

\lohead{\textsc{register}}

% Definiere theindex-Environment komplett neu ohne reledmac
\makeatletter
\renewenvironment{theindex}{%
  \section*{\indexname}%
  \setlength{\parindent}{0pt}%
  \setlength{\parskip}{0pt plus 0.3pt}%
  \let\item\@idxitem
}{%
  \clearpage
}
\makeatother

\IfFileExists{\jobname-pw.ind}{\input{\jobname-pw.ind}}{}

\end{document}

      