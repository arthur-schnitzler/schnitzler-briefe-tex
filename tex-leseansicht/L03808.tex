%% latex-leseansicht-vorspann.tex
%% Vorspann für die Leseansicht.
%% Lädt die gemeinsame Datei latex-vorspann.tex mit nicht gesetztem Schalter.

\newif\ifkorrekturansicht
\korrekturansichtfalse

\input{../tex-inputs/latex-vorspann}


\section[Arthur Schnitzler an Stefan Zweig, 22. 1. 1908]{L03808 Arthur Schnitzler an Stefan Zweig, 22. 1. 1908}
\nopagebreak\mylabel{L03808v}
\rehead{ }\normalsize\beginnumbering\briefempfaengerindex{Zweig, Stefan@\textsc{Zweig, Stefan}!zzzSchnitzler, Arthur@\emph{von Arthur Schnitzler}!1908-01-221@{22. 1. 1908}|(be}
\toendnotes[C]{\smallbreak\pagebreak[2]}
\correspDesc{Versand  durch Arthur Schnitzler am 22. 1. 1908 in Wien
\newline{}Erhalt  durch Stefan Zweig im Zeitraum [23. 1. 1908 – 27. 1. 1908?] in Wien}\toendnotes[C]{\smallbreak}
\Standort{Jerusalem, National Library of Israel, ARC. Ms. Var. 305 1 58 Stefan Zweig Collection.}
\physDesc{Brief, 1 Blatt, 1 Seite, 344 Zeichen
\newline{}Handschrift: schwarze Tinte, lateinische Kurrent}\toendnotes[C]{\smallbreak}
\pstart
           {\pb}\textcolor{gray}{\textbf{Dr Arthur Schnitzler}}\hfill 22. 1. 908\pend
           
\pstart
           \textcolor{gray}{\textbf{Wien XVIII.
                        Spoettelgasse 7\oindex{Wien@\textbf{Wien}!XVIII., Währing@\textbf{XVIII., Währing}!Edmund-Weiß-Gasse@\textbf{Edmund-Weiß-Gasse}, \emph{Straße}|pw}.}}\pend
           \vspace{0.5em}
\pstart
           verehrtester Herr Zweig,  lassen Sie mich für Ihren \label{K_L03808-1v}\edtext{Brief}{\lemma{\textnormal{\emph{Brief}}}\Cendnote{\textnormal{XXXX Auszeichnungsfehler: Dokument L03621 nicht gefunden.}}}\label{K_L03808-1}, der
               mich besonders gefreut hat, herzlichst danken – und die Hoffnung aussprechen, dass
               Sie mir, aus Ihren schönen Versen und, leider nur \label{K_L03808-2v}\edtext{einzelnen Scenen des Thersites\pwindex{Zweig, Stefan 28.\,11.\,1881 Wien – 23.\,2.\,1942 Petrópolis@\textsc{Zweig, Stefan} (28.\,11.\,1881 Wien – 23.\,2.\,1942 Petrópolis), \emph{Schriftsteller}!Tersites. Ein Trauerspiel in drei Aufzügen@\strich\emph{Tersites. Ein Trauerspiel in drei Aufzügen}|pw}}{\lemma{\textnormal{\emph{einzelnen … Thersites}}}\Cendnote{\textnormal{Ein teilweiser
                  Vorabdruck erschien in der \emph{Schaubühne}\pwindex{Schaubühne@\emph{Die Schaubühne}|pwk}:
                        Stefan Zweig\pwindex{Zweig, Stefan 28.\,11.\,1881 Wien – 23.\,2.\,1942 Petrópolis@\textsc{Zweig, Stefan} (28.\,11.\,1881 Wien – 23.\,2.\,1942 Petrópolis), \emph{Schriftsteller}|pwk}: \emph{Tersites}\pwindex{Zweig, Stefan 28.\,11.\,1881 Wien – 23.\,2.\,1942 Petrópolis@\textsc{Zweig, Stefan} (28.\,11.\,1881 Wien – 23.\,2.\,1942 Petrópolis), \emph{Schriftsteller}!Tersites [Vorabdruck]@\strich\emph{Tersites [Vorabdruck]}|pwk}. In: \emph{Die
                        Schaubühne}\pwindex{Schaubühne@\emph{Die Schaubühne}|pwk}, Jg. 3, H. 33, 15. 8. 1907,
                     S. 125–130. Die vollständige Buchausgabe im \emph{Insel Verlag}\orgindex{Insel Verlag@Insel Verlag|pwk} erschien im März 1908, vgl. XXXX Auszeichnungsfehler: Dokument L03809 nicht gefunden.}}}\label{K_L03808-2}, längst
               beka{\geminationn}t, auch \label{K_L03808-3v}\edtext{persönlich nicht lang mehr ein Unbekannter}{\lemma{\textnormal{\emph{persönlich … Unbekannter}}}\Cendnote{\textnormal{Das erste persönliche
                  Treffen fand am 28. 5. 1908 statt.}}}\label{K_L03808-3} bleiben.\pend
           
\pstart
           Ihr aufrichtig ergebener{\\[\baselineskip]}\spacefill\mbox{Arthur Schnitzler}\pend
           \leftskip=0em{}\selectlanguage{ngerman}\endnumbering\briefempfaengerindex{Zweig, Stefan@\textsc{Zweig, Stefan}!zzzSchnitzler, Arthur@\emph{von Arthur Schnitzler}!1908-01-221@{22. 1. 1908}|)be}\mylabel{L03808h}  \newcommand{\dateiname}{L03808}\newcommand{\titel}{Arthur Schnitzler an Stefan Zweig, 22. 1. 1908}\newcommand{\editorInnen}{Selma Jahnke und Martin Anton Müller}%% latex-leseansicht-abspann.tex
%% Abspann für die Leseansicht.
%% Der Schalter \ifkorrekturansicht ist bereits durch den Vorspann gesetzt.

%% latex-abspann.tex
%% Gemeinsamer Abspann für Korrekturansicht und Leseansicht.
%% Setzt den Schalter \ifkorrekturansicht voraus (gesetzt in den
%% einbindenden Dateien latex-korrekturansicht-abspann.tex bzw.
%% latex-leseansicht-abspann.tex).
%% ---------------------------------------------------------------

\normalsize

% Das esempio-Environment wird nur in der Leseansicht benötigt
\ifkorrekturansicht\else
\newenvironment{esempio}[3]%
{
    \vspace{1.5ex}
    \rlap{\underline{#1}}
    \par
    \setlength{\parindent}{0cm}
    \nopagebreak
    \leftskip=#2cm
    \rightskip=#3cm
}
{
    \par
}
\fi

\doendnotes{C}
\bigskip
\vfill

\clearpage

\footnotesize

\ifkorrekturansicht
  \lohead{\textsc{register}}
\fi

% theindex-Environment neu definieren ohne reledmac
\makeatletter
\renewenvironment{theindex}{%
  \ifkorrekturansicht
    \section*{\indexname}%
  \else
    \subsubsection*{Index der erwähnten Entitäten}%
  \fi
  \setlength{\parindent}{0pt}%
  \setlength{\parskip}{0pt plus 0.3pt}%
  \let\item\@idxitem
}{%
  \ifkorrekturansicht\clearpage\fi
}
\makeatother

\IfFileExists{\jobname-pw.ind}{\input{\jobname-pw.ind}}{}

% Quellenangabe nur in der Leseansicht
\ifkorrekturansicht\else
% Fallback-Definitionen, falls die .tex-Datei \titel etc. nicht gesetzt hat
\providecommand{\titel}{}
\providecommand{\editorInnen}{}
\providecommand{\dateiname}{\jobname}

\vspace{3cm}

\vfill

\footnotesize
\textsc{Quelle}: \titel. Herausgegeben von {\editorInnen}. In: \emph{Arthur Schnitzler: Briefwechsel mit Autorinnen und Autoren}.
 Digitale Edition, https://schnitzler-briefe.acdh.oeaw.ac.at/{\dateiname}.html (Stand \today)
\fi

\end{document}


