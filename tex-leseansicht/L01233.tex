%% latex-korrekturansicht-vorspann.tex
%% Vorspann für die Korrekturansicht.
%% Lädt die gemeinsame Datei latex-vorspann.tex mit gesetztem Schalter.

\newif\ifkorrekturansicht
\korrekturansichttrue

\input{../tex-inputs/latex-vorspann}


\section[Hugo von Hofmannsthal an Arthur Schnitzler, {[}23. 7. 1902{]}]{L01233 Hugo von Hofmannsthal an Arthur Schnitzler, {[}23. 7. 1902{]}}
\nopagebreak\mylabel{L01233v}
\rehead{ }\normalsize\beginnumbering\briefempfaengerindex{Schnitzler, Arthur@\textsc{Schnitzler, Arthur}!zzzHofmannsthal, Hugo von@\emph{von Hugo von Hofmannsthal}!1902-07-231@{23. 7. 1902}|(be}
\toendnotes[C]{\smallbreak\pagebreak[2]}\Standort{CUL, Schnitzler, B 43.}
\physDesc{Brief, 1 Blatt, 2 Seiten, 383 Zeichen
\newline{}Handschrift: schwarze Tinte, deutsche Kurrent
\newline{}Schnitzler: mit Bleistift datiert: »23/7 902.« 
\newline{}Ordnung: 1) mit Bleistift von unbekannter Hand nummeriert: »\strikeout{199}«  2) mit Bleistift von unbekannter Hand nummeriert:
                                    »182«}
\buchAbdrucke{\weitereDrucke{Hugo von Hofmannsthal, Arthur Schnitzler: \emph{Briefwechsel}. Frankfurt am Main: \emph{S. Fischer} 1964, S. 159.} }\toendnotes[C]{\smallbreak}
\pstart
           \noindent{}{\pb}lieber, wie kann ich
               zu Ihnen nachtmahlen kommen, wenn ich nie weiß, ob Sie \label{K_L01233-1v}\edtext{draußen\oindex{Bruehl@\textbf{Brühl}, \emph{Tal (N.TAL)}|pwv}}{\lemma{\textnormal{\emph{draußen}}}\Cendnote{\textnormal{In den Wochen vor der Geburt des Sohnes
                     Heinrich\pwindex{Schnitzler, Heinrich 09.08.1902 – 12.07.1982@\textsc{Schnitzler, Heinrich} (09.08.1902 – 12.07.1982), \emph{Regisseur/Regisseurin, Schauspieler/Schauspielerin}|pwk} pendelte Schnitzler zwischen Wien\oindex{Wien@\textbf{Wien}, \emph{A.ADM2}|pwk}
                  und Hinterbrühl\oindex{Hinterbruehl@\textbf{Hinterbrühl}, \emph{P.PPLA3}|pwk}.}}}\label{K_L01233-1} oder drinnen ſind.
               z. B. Ich käme gern morgen oder übermorgen abend, gegen 7\textsuperscript{h}. Aber ich weiß doch nicht ob Sie draußen oder drinnen
               ſind. Bitte \uline{depeſchieren} Sie mir gleich {\pb}nach Empfang dieſer Zeilen, ob Sie
               draußen oder drinnen ſind, und welchen Abend Sie mich erwarten. Von Herzen\pend
           \pstart \spacefill\mbox{Hugo.}\pend{}
\pstart
           \noindent{}Beiliegend Sacktuch.\pend
           \selectlanguage{ngerman}\endnumbering\briefempfaengerindex{Schnitzler, Arthur@\textsc{Schnitzler, Arthur}!zzzHofmannsthal, Hugo von@\emph{von Hugo von Hofmannsthal}!1902-07-231@{23. 7. 1902}|)be}\mylabel{L01233h}  \normalsize

\doendnotes{C}
\bigskip
\vfill

\clearpage

\footnotesize

\lohead{\textsc{register}}

% Definiere theindex-Environment komplett neu ohne reledmac
\makeatletter
\renewenvironment{theindex}{%
  \section*{\indexname}%
  \setlength{\parindent}{0pt}%
  \setlength{\parskip}{0pt plus 0.3pt}%
  \let\item\@idxitem
}{%
  \clearpage
}
\makeatother

\IfFileExists{\jobname-pw.ind}{\input{\jobname-pw.ind}}{}

\end{document}

      