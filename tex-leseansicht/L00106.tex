%% latex-leseansicht-vorspann.tex
%% Vorspann für die Leseansicht.
%% Lädt die gemeinsame Datei latex-vorspann.tex mit nicht gesetztem Schalter.

\newif\ifkorrekturansicht
\korrekturansichtfalse

\input{../tex-inputs/latex-vorspann}


         
         \renewcommand{\erwaehntePersonen}{Personen: Wilhelm Bölsche}
         \renewcommand{\erwaehnteOrte}{Orte: Berlin, Friedrichshagen, Peter-Hille-Straße, Wien}
         \renewcommand{\erwaehnteWerke}{Werke: Das Himmelbett}
               \section[Wilhelm Bölsche an Arthur Schnitzler, {[}24. 7. 1892{]}]{ Wilhelm Bölsche an Arthur Schnitzler, {[}24. 7. 1892{]}}\nopagebreak\mylabel{v}\rehead{ }\begin{ledgroupsized}[t]{13cm}\normalsize\beginnumbering\briefempfaengerindex{Schnitzler, Arthur@\textsc{Schnitzler, Arthur}!zzzBoelsche, Wilhelm@\emph{von Wilhelm Bölsche}!1892-07-241@{{[}24. 7. 1892{]}}|(be} \toendnotes[C]{\smallbreak\pagebreak[2]} \Standort{DLA, A:Schnitzler, HS.NZ85.1.2577,6.}
\physDesc{Brief, 1 Blatt, 2 Seiten, 837 Zeichen
\newline{}Handschrift: schwarze Tinte, deutsche Kurrent
\newline{}Schnitzler: mit Bleistift datiert: »24/7 92« 
\newline{}Ordnung: mit rotem Buntstift von unbekannter Hand nummeriert:
                                    »7« }\buchAbdrucke{\weitereDrucke{Wilhelm Bölsche: \emph{Briefwechsel. Mit Autoren der Freien Bühne}. Hg. Gerd-Hermann Susen. Berlin: \emph{Weidler} 2010, S. 682 (Werke und Briefe. Wissenschaftliche Ausgabe, Briefe I).} }\toendnotes[C]{\smallbreak}\pstart
           \raggedleft{}{\pb}Friedrichshagen\oindex{Friedrichshagen@\textbf{Friedrichshagen}|pw}{\\}b. Berlin\oindex{Berlin@\textbf{Berlin}|pw}.{\\}Wilhelmſtr. 72\oindex{Peter-Hille-Strasse@\textbf{Peter-Hille-Straße}|pw}.\pend
           \pstart\center{}Hochverehrter Herr Doktor!\pend\pstart
           Zu meinem Erſtaunen erſehe ich aus Ihrem Briefe, daß ein vor längerer Zeit ſchon an
               Sie abgeſandter Brief Sie offenbar nicht erreicht hat. Ich ſchrieb damals, daß ich
               betreffs Ihrer Novelle\pwindex{Schnitzler, Arthur 15.05.1862 – 21.10.1931@\textsc{Schnitzler, Arthur} (15.05.1862 – 21.10.1931), \emph{Schriftsteller, Mediziner}!Himmelbett1977@\strich\emph{Das Himmelbett} {[}1977{]}|pwv} etwas
                  \introOben{}in\introOben{} Zweifel ſei, ob ſie ſich für eine Zeitſchrift eigne –
               des Motivs wegen – und ſtellte Ihnen anheim, ob Sie mir nicht lieber eine andere
               dafür geben wollten. Glücklicher Weiſe – wie ich jetzt ſagen muß – legte ich in {\pb}meiner Unſchlüſſigkeit das Manuſkript nicht bei, – ich
               wollte es erſt noch von eine\substVorne{}\textsuperscript{m}\substDazwischen{}n\substHinten{} Andern leſen laſſen, um \strikeout{d} zu ſehen, ob ich
               mich nicht über die bedenkliche Wirkung täuſche. Es iſt alſo noch hier, und ich lege
               es heute bei – zugleich unter Wiederholung der Bitte um etwas Anderes. Der Stoff\pwindex{Schnitzler, Arthur 15.05.1862 – 21.10.1931@\textsc{Schnitzler, Arthur} (15.05.1862 – 21.10.1931), \emph{Schriftsteller, Mediziner}!Himmelbett1977@\strich\emph{Das Himmelbett} {[}1977{]}|pwv} iſt wirklich
               »zeitſchriftlich« unmöglich!\pend
           \pstart
           Mit herzlichem Gruß{\\[\baselineskip]}Ihr{\\[\baselineskip]}\spacefill\mbox{W. Bölsche}\pend
           \leftskip=0em{}
         
         \endnumbering\mylabel{h}\end{ledgroupsized}  \newcommand{\dateiname}{L00106}\newcommand{\titel}{Wilhelm Bölsche an Arthur Schnitzler, [24. 7. 1892]}\newcommand{\editorInnen}{Martin Anton Müller und Gerd-Hermann Susen}%% latex-leseansicht-abspann.tex
%% Abspann für die Leseansicht.
%% Der Schalter \ifkorrekturansicht ist bereits durch den Vorspann gesetzt.

%% latex-abspann.tex
%% Gemeinsamer Abspann für Korrekturansicht und Leseansicht.
%% Setzt den Schalter \ifkorrekturansicht voraus (gesetzt in den
%% einbindenden Dateien latex-korrekturansicht-abspann.tex bzw.
%% latex-leseansicht-abspann.tex).
%% ---------------------------------------------------------------

\normalsize

% Das esempio-Environment wird nur in der Leseansicht benötigt
\ifkorrekturansicht\else
\newenvironment{esempio}[3]%
{
    \vspace{1.5ex}
    \rlap{\underline{#1}}
    \par
    \setlength{\parindent}{0cm}
    \nopagebreak
    \leftskip=#2cm
    \rightskip=#3cm
}
{
    \par
}
\fi

\doendnotes{C}
\bigskip
\vfill

\clearpage

\footnotesize

\ifkorrekturansicht
  \lohead{\textsc{register}}
\fi

% theindex-Environment neu definieren ohne reledmac
\makeatletter
\renewenvironment{theindex}{%
  \ifkorrekturansicht
    \section*{\indexname}%
  \else
    \subsubsection*{Index der erwähnten Entitäten}%
  \fi
  \setlength{\parindent}{0pt}%
  \setlength{\parskip}{0pt plus 0.3pt}%
  \let\item\@idxitem
}{%
  \ifkorrekturansicht\clearpage\fi
}
\makeatother

\IfFileExists{\jobname-pw.ind}{\input{\jobname-pw.ind}}{}

% Quellenangabe nur in der Leseansicht
\ifkorrekturansicht\else
% Fallback-Definitionen, falls die .tex-Datei \titel etc. nicht gesetzt hat
\providecommand{\titel}{}
\providecommand{\editorInnen}{}
\providecommand{\dateiname}{\jobname}

\vspace{3cm}

\vfill

\footnotesize
\textsc{Quelle}: \titel. Herausgegeben von {\editorInnen}. In: \emph{Arthur Schnitzler: Briefwechsel mit Autorinnen und Autoren}.
 Digitale Edition, https://schnitzler-briefe.acdh.oeaw.ac.at/{\dateiname}.html (Stand \today)
\fi

\end{document}


      