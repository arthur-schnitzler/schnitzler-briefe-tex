%% latex-korrekturansicht-vorspann.tex
%% Vorspann für die Korrekturansicht.
%% Lädt die gemeinsame Datei latex-vorspann.tex mit gesetztem Schalter.

\newif\ifkorrekturansicht
\korrekturansichttrue

\input{../tex-inputs/latex-vorspann}


\section[Wilhelm Bölsche an Arthur Schnitzler, {[}24. 7. 1892{]}]{L00106 Wilhelm Bölsche an Arthur Schnitzler, {[}24. 7. 1892{]}}
\nopagebreak\mylabel{L00106v}
\rehead{ }\normalsize\beginnumbering\briefempfaengerindex{Schnitzler, Arthur@\textsc{Schnitzler, Arthur}!zzzBoelsche, Wilhelm@\emph{von Wilhelm Bölsche}!1892-07-241@{{[}24. 7. 1892{]}}|(be}
\toendnotes[C]{\smallbreak\pagebreak[2]}\Standort{DLA, A:Schnitzler, HS.NZ85.1.2577,6.}
\physDesc{Brief, 1 Blatt, 2 Seiten, 837 Zeichen
\newline{}Handschrift: schwarze Tinte, deutsche Kurrent
\newline{}Schnitzler: mit Bleistift datiert: »24/7 92« 
\newline{}Ordnung: mit rotem Buntstift von unbekannter Hand nummeriert:
                                    »7« }
\buchAbdrucke{\weitereDrucke{Wilhelm Bölsche: \emph{Briefwechsel. Mit Autoren der Freien Bühne}. Berlin: \emph{Weidler} 2010, S. 682.} }\toendnotes[C]{\smallbreak}
\pstart
           \raggedleft{}{\pb}Friedrichshagen\oindex{Friedrichshagen@\textbf{Friedrichshagen}, \emph{P.PPLX}|pw}{\\}b. Berlin\oindex{Berlin@\textbf{Berlin}, \emph{P.PPLC}|pw}.{\\}Wilhelmſtr. 72\oindex{Peter-Hille-Strasse@\textbf{Peter-Hille-Straße}, \emph{Straße (K.STR)}|pw}.\pend
           
\pstart\center{}Hochverehrter Herr Doktor!\pend\vspace{0.5em}
\pstart
           Zu meinem Erſtaunen erſehe ich aus Ihrem Briefe, daß ein vor längerer Zeit ſchon an
               Sie abgeſandter Brief Sie offenbar nicht erreicht hat. Ich ſchrieb damals, daß ich
               betreffs Ihrer Novelle\pwindex{Himmelbett@\emph{Das Himmelbett}|pwv} etwas
                  \introOben{}in\introOben{} Zweifel ſei, ob ſie ſich für eine Zeitſchrift eigne –
               des Motivs wegen – und ſtellte Ihnen anheim, ob Sie mir nicht lieber eine andere
               dafür geben wollten. Glücklicher Weiſe – wie ich jetzt ſagen muß – legte ich in {\pb}meiner Unſchlüſſigkeit das Manuſkript nicht bei, – ich
               wollte es erſt noch von eine\substVorne{}\textsuperscript{m}\substDazwischen{}n\substHinten{} Andern leſen laſſen, um \strikeout{d} zu ſehen, ob ich
               mich nicht über die bedenkliche Wirkung täuſche. Es iſt alſo noch hier, und ich lege
               es heute bei – zugleich unter Wiederholung der Bitte um etwas Anderes. Der Stoff\pwindex{Himmelbett@\emph{Das Himmelbett}|pwv} iſt wirklich
               »zeitſchriftlich« unmöglich!\pend
           
\pstart
           Mit herzlichem Gruß{\\[\baselineskip]}Ihr{\\[\baselineskip]}\spacefill\mbox{W. Bölsche}\pend
           \leftskip=0em{}\selectlanguage{ngerman}\endnumbering\briefempfaengerindex{Schnitzler, Arthur@\textsc{Schnitzler, Arthur}!zzzBoelsche, Wilhelm@\emph{von Wilhelm Bölsche}!1892-07-241@{{[}24. 7. 1892{]}}|)be}\mylabel{L00106h}  \normalsize

\doendnotes{C}
\bigskip
\vfill

\clearpage

\footnotesize

\lohead{\textsc{register}}

% Definiere theindex-Environment komplett neu ohne reledmac
\makeatletter
\renewenvironment{theindex}{%
  \section*{\indexname}%
  \setlength{\parindent}{0pt}%
  \setlength{\parskip}{0pt plus 0.3pt}%
  \let\item\@idxitem
}{%
  \clearpage
}
\makeatother

\IfFileExists{\jobname-pw.ind}{\input{\jobname-pw.ind}}{}

\end{document}

      