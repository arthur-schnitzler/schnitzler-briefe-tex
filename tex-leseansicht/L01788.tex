%% latex-korrekturansicht-vorspann.tex
%% Vorspann für die Korrekturansicht.
%% Lädt die gemeinsame Datei latex-vorspann.tex mit gesetztem Schalter.

\newif\ifkorrekturansicht
\korrekturansichttrue

\input{../tex-inputs/latex-vorspann}


\section[Richard Beer-Hofmann an Arthur Schnitzler, {[}zwischen 1. und 12. 9.? 1908{]}]{L01788 Richard Beer-Hofmann an Arthur Schnitzler, {[}zwischen 1. und
               12. 9.? 1908{]}}
\nopagebreak\mylabel{L01788v}
\rehead{ }\normalsize\beginnumbering\briefempfaengerindex{Schnitzler, Arthur@\textsc{Schnitzler, Arthur}!zzzBeer-Hofmann, Richard@\emph{von Richard Beer-Hofmann}!1908-09-121@{{[}zwischen 1. und
                  12. 9.? 1908{]}}|(be}
\toendnotes[C]{\smallbreak\pagebreak[2]}\Standort{CUL, Schnitzler, B 8.}
\physDesc{Bildpostkarte, 300 Zeichen
\newline{}Handschrift: schwarze Tinte, lateinische Kurrent
\newline{}Versand: nachgesandt nach »\textsc{Spöttlgasse 7 Wien XVIII\oindex{Edmund-Weiss-Gasse 7@\textbf{Edmund-Weiß-Gasse 7}, \emph{Wohngebäude (K.WHS)}|pw}}« 
\newline{}Schnitzler: mit Bleistift datiert: »Juli (?) 08« und beschriftet: »\textsc{Beerh}« 
\newline{}Ordnung: 1) mit Bleistift von unbekannter Hand nummeriert: »\strikeout{216}«  2) mit Bleistift von unbekannter Hand nummeriert:
                                    »217«}
\buchAbdrucke{\weitereDrucke{Arthur Schnitzler, Richard Beer-Hofmann: \emph{Briefwechsel 1891–1931}. Wien, Zürich: \emph{Europaverlag} 1992, S. 190.} }\toendnotes[C]{\smallbreak}\pstart{}{\pb}D\textsuperscript{r}
                  Arthur Schnitzler\pend{}\pstart{}Seis\oindex{Seis am Schlern@\textbf{Seis am Schlern}, \emph{P.PPL}|pw}\pend{}\pstart{}bei Waidbruck\oindex{Ponte Gardena@\textbf{Ponte Gardena}, \emph{A.ADM3}|pw}\pend{}\pstart{}Süd-Tirol\oindex{Suedtirol@\textbf{Südtirol}, \emph{A.ADM2}|pw}\pend{}\pstart{}Austria\oindex{Oesterreich@\textbf{Österreich}, \emph{A.PCLI}|pw}\pend{}{\bigskip}
\pstart
           \noindent{}\centering{}{\pb}\textcolor{gray}{\textbf{VENEZIA\oindex{Venedig@\textbf{Venedig}, \emph{P.PPLA}|pw}}}\pend
           \vspace{1em}
\pstart
           \centering{}{\pb}Lido\hspace*{1.5em}Hôtel des
                     Bains\oindex{Grand Hotel des Bains@\textbf{Grand Hotel des Bains}, \emph{Hotel (K.HTL)}|pw}\pend
           \vspace{0.5em}
\pstart
           {\pb}Lieber Arthur! Ich war still, da ich nicht ja{\geminationm}ern wollte. Paula\pwindex{Beer-Hofmann, Paula 25.02.1879 – 30.10.1939@\textsc{Beer-Hofmann, Paula} (25.02.1879 – 30.10.1939)|pw} hatte Hals- u. Rippenfellentzündung. Wir wollen nach 15
               hier weg, über Südtirol\oindex{Suedtirol@\textbf{Südtirol}, \emph{A.ADM2}|pw} nach Hause. Hoffentlich
                  \label{K_L01788-1v}\edtext{schickt man die Karte Ihnen
                  nach}{\lemma{\textnormal{\emph{schickt … nach}}}\Cendnote{\textnormal{Der Poststempel ist nicht
                  entzifferbar. Die Jahreszahl ist durch die Erkrankung Paulas\pwindex{Beer-Hofmann, Paula 25.02.1879 – 30.10.1939@\textsc{Beer-Hofmann, Paula} (25.02.1879 – 30.10.1939)|pwk} gesichert. Schnitzlers mit Fragezeichen versehene Angabe »Juli?« ist
                  jedoch nicht haltbar, da er sich zu dieser Zeit in Seis\oindex{Seis am Schlern@\textbf{Seis am Schlern}, \emph{P.PPL}|pwk} aufhielt, eine Nachsendung also nicht notwendig gewesen wäre. Diese
                  Nachsendung hat auch stattgefunden und Schnitzler die Karte zwei Tage vor dem 16. 9. 1908 erhalten. Damit lässt sich der Zeitraum
                  zwischen der Abreise aus Seis\oindex{Seis am Schlern@\textbf{Seis am Schlern}, \emph{P.PPL}|pwk} und der Ankunft
                  in Wien\oindex{Wien@\textbf{Wien}, \emph{A.ADM2}|pwk} am 14. 9. 1908 für den Versand der Karte
                  bestimmen.}}}\label{K_L01788-1}. \label{T_L01788-1v}\edtext{Wir grüssen Sie
                  Beide\pwindex{Schnitzler, Olga 17.01.1882 – 13.01.1970@\textsc{Schnitzler, Olga} (17.01.1882 – 13.01.1970), \emph{Schauspieler/Schauspielerin, Sänger/Sängerin}|pwv} herzlich.}{\lemma{\textnormal{\emph{Wir … herzlich.}}}\Cendnote{\textnormal{weiter quer am linken Rand}}}\label{T_L01788-1}\pend
           \pstart \spacefill\mbox{Richard}\pend{}\selectlanguage{ngerman}\endnumbering\briefempfaengerindex{Schnitzler, Arthur@\textsc{Schnitzler, Arthur}!zzzBeer-Hofmann, Richard@\emph{von Richard Beer-Hofmann}!1908-09-011@{{[}zwischen 1. und
                  12. 9.? 1908{]}}|)be}\mylabel{L01788h}  \normalsize

\doendnotes{C}
\bigskip
\vfill

\clearpage

\footnotesize

\lohead{\textsc{register}}

% Definiere theindex-Environment komplett neu ohne reledmac
\makeatletter
\renewenvironment{theindex}{%
  \section*{\indexname}%
  \setlength{\parindent}{0pt}%
  \setlength{\parskip}{0pt plus 0.3pt}%
  \let\item\@idxitem
}{%
  \clearpage
}
\makeatother

\IfFileExists{\jobname-pw.ind}{\input{\jobname-pw.ind}}{}

\end{document}

      