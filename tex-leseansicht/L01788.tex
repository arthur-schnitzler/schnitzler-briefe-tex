%% latex-leseansicht-vorspann.tex
%% Vorspann für die Leseansicht.
%% Lädt die gemeinsame Datei latex-vorspann.tex mit nicht gesetztem Schalter.

\newif\ifkorrekturansicht
\korrekturansichtfalse

\input{../tex-inputs/latex-vorspann}


\section[Richard Beer-Hofmann an Arthur Schnitzler, {{[}}zwischen 1. und 12. 9.? 1908{{]}}]{L01788 Richard Beer-Hofmann an Arthur Schnitzler, {[}zwischen 1. und 12. 9.? 1908{]}}
\nopagebreak\mylabel{L01788v}
\rehead{ }\normalsize\beginnumbering\briefempfaengerindex{Schnitzler, Arthur@\textsc{Schnitzler, Arthur}!zzzBeer-Hofmann, Richard@\emph{von Richard Beer-Hofmann}!1908-09-121@{{[}zwischen 1. und 12. 9.? 1908{]}}|(be}
\toendnotes[C]{\smallbreak\pagebreak[2]}
\correspDesc{Versand  durch Richard Beer-Hofmann im Zeitraum [zwischen 1. und
                  12. 9.? 1908] in Lido
\newline{}Umleitung  in Seis am Schlern
\newline{}Erhalt  durch Arthur Schnitzler am 14. 9. 1908 in Wien}\toendnotes[C]{\smallbreak}
\Standort{CUL, Schnitzler, B 8.}
\physDesc{Bildpostkarte, 300 Zeichen
\newline{}Handschrift: schwarze Tinte, lateinische Kurrent
\newline{}Versand: nachgesandt nach »\textsc{Spöttlgasse 7 Wien XVIII\oindex{Wien@\textbf{Wien}!XVIII., Währing@\textbf{XVIII., Währing}!Edmund-Weiß-Gasse 7@\textbf{Edmund-Weiß-Gasse 7}, \emph{Wohngebäude}|pw}}« 
\newline{}Schnitzler: mit Bleistift datiert: »Juli (?) 08« und beschriftet: »\textsc{Beerh}« 
\newline{}Ordnung: 1) mit Bleistift von unbekannter Hand nummeriert: »\strikeout{216}«  2) mit Bleistift von unbekannter Hand nummeriert:
                                    »217«}
\buchAbdrucke{\weitereDrucke{Arthur Schnitzler, Richard Beer-Hofmann: \emph{Briefwechsel 1891–1931}. Herausgegeben von Konstanze Fliedl. Wien, Zürich: \emph{Europaverlag} 1992, S. 190.} }\toendnotes[C]{\smallbreak}\pstart{}{\pb}D\textsuperscript{r}
                  Arthur Schnitzler\pend{}\pstart{}Seis\oindex{Seis am Schlern@\textbf{Seis am Schlern}|pw}\pend{}\pstart{}bei Waidbruck\oindex{Ponte Gardena@\textbf{Ponte Gardena}, \emph{Verwaltungsgebiet}|pw}\pend{}\pstart{}Süd-Tirol\oindex{Südtirol@\textbf{Südtirol}, \emph{Verwaltungsgebiet}|pw}\pend{}\pstart{}Austria\oindex{Österreich@\textbf{Österreich}|pw}\pend{}{\bigskip}
\pstart
           \noindent{}\centering{}{\pb}\textcolor{gray}{\textbf{VENEZIA\oindex{Venedig@\textbf{Venedig}|pw}}}\pend
           \vspace{1em}
\pstart
           \centering{}{\pb}Lido\hspace*{1.5em}Hôtel des
                     Bains\oindex{Grand Hotel des Bains@\textbf{Grand Hotel des Bains}, \emph{Hotel}|pw}\pend
           \vspace{0.5em}
\pstart
           {\pb}Lieber Arthur! Ich war still, da ich nicht ja{\geminationm}ern wollte. Paula\pwindex{Beer-Hofmann, Paula 25.\,2.\,1879 Wien – 30.\,10.\,1939 Zürich@\textsc{Beer-Hofmann, Paula} (25.\,2.\,1879 Wien – 30.\,10.\,1939 Zürich)|pw} hatte Hals- u. Rippenfellentzündung. Wir wollen nach 15
               hier weg, über Südtirol\oindex{Südtirol@\textbf{Südtirol}, \emph{Verwaltungsgebiet}|pw} nach Hause. Hoffentlich
                  \label{K_L01788-1v}\edtext{schickt man die Karte Ihnen
                  nach}{\lemma{\textnormal{\emph{schickt … nach}}}\Cendnote{\textnormal{Der Poststempel ist nicht
                  entzifferbar. Die Jahreszahl ist durch die Erkrankung Paulas\pwindex{Beer-Hofmann, Paula 25.\,2.\,1879 Wien – 30.\,10.\,1939 Zürich@\textsc{Beer-Hofmann, Paula} (25.\,2.\,1879 Wien – 30.\,10.\,1939 Zürich)|pwk} gesichert. Schnitzlers mit Fragezeichen versehene Angabe »Juli?« ist
                  jedoch nicht haltbar, da er sich zu dieser Zeit in Seis\oindex{Seis am Schlern@\textbf{Seis am Schlern}|pwk} aufhielt, eine Nachsendung also nicht notwendig gewesen wäre. Diese
                  Nachsendung hat auch stattgefunden und Schnitzler die Karte zwei Tage vor dem XXXX Auszeichnungsfehler: Dokument L01790 nicht gefunden erhalten. Damit lässt sich der Zeitraum
                  zwischen der Abreise aus Seis\oindex{Seis am Schlern@\textbf{Seis am Schlern}|pwk} und der Ankunft
                  in Wien\oindex{Wien@\textbf{Wien}, \emph{Verwaltungsgebiet}|pwk} am 14. 9. 1908 für den Versand der Karte
                  bestimmen.}}}\label{K_L01788-1}. \label{T_L01788-1v}\edtext{Wir grüssen Sie
                  Beide\pwindex{Schnitzler, Olga 17.\,1.\,1882 Wien – 13.\,1.\,1970 Lugano@\textsc{Schnitzler, Olga} (17.\,1.\,1882 Wien – 13.\,1.\,1970 Lugano), \emph{Schauspielerin, Sängerin}|pwv} herzlich.}{\lemma{\textnormal{\emph{Wir … herzlich.}}}\Cendnote{\textnormal{weiter quer am linken Rand}}}\label{T_L01788-1}\pend
           \pstart \spacefill\mbox{Richard}\pend{}\selectlanguage{ngerman}\endnumbering\briefempfaengerindex{Schnitzler, Arthur@\textsc{Schnitzler, Arthur}!zzzBeer-Hofmann, Richard@\emph{von Richard Beer-Hofmann}!1908-09-011@{{[}zwischen 1. und 12. 9.? 1908{]}}|)be}\mylabel{L01788h}  \newcommand{\dateiname}{L01788}\newcommand{\titel}{Richard Beer-Hofmann an Arthur Schnitzler, [zwischen 1. und 12. 9.? 1908]}\newcommand{\editorInnen}{Martin Anton Müller und Gerd-Hermann Susen}%% latex-leseansicht-abspann.tex
%% Abspann für die Leseansicht.
%% Der Schalter \ifkorrekturansicht ist bereits durch den Vorspann gesetzt.

%% latex-abspann.tex
%% Gemeinsamer Abspann für Korrekturansicht und Leseansicht.
%% Setzt den Schalter \ifkorrekturansicht voraus (gesetzt in den
%% einbindenden Dateien latex-korrekturansicht-abspann.tex bzw.
%% latex-leseansicht-abspann.tex).
%% ---------------------------------------------------------------

\normalsize

% Das esempio-Environment wird nur in der Leseansicht benötigt
\ifkorrekturansicht\else
\newenvironment{esempio}[3]%
{
    \vspace{1.5ex}
    \rlap{\underline{#1}}
    \par
    \setlength{\parindent}{0cm}
    \nopagebreak
    \leftskip=#2cm
    \rightskip=#3cm
}
{
    \par
}
\fi

\doendnotes{C}
\bigskip
\vfill

\clearpage

\footnotesize

\ifkorrekturansicht
  \lohead{\textsc{register}}
\fi

% theindex-Environment neu definieren ohne reledmac
\makeatletter
\renewenvironment{theindex}{%
  \ifkorrekturansicht
    \section*{\indexname}%
  \else
    \subsubsection*{Index der erwähnten Entitäten}%
  \fi
  \setlength{\parindent}{0pt}%
  \setlength{\parskip}{0pt plus 0.3pt}%
  \let\item\@idxitem
}{%
  \ifkorrekturansicht\clearpage\fi
}
\makeatother

\IfFileExists{\jobname-pw.ind}{\input{\jobname-pw.ind}}{}

% Quellenangabe nur in der Leseansicht
\ifkorrekturansicht\else
% Fallback-Definitionen, falls die .tex-Datei \titel etc. nicht gesetzt hat
\providecommand{\titel}{}
\providecommand{\editorInnen}{}
\providecommand{\dateiname}{\jobname}

\vspace{3cm}

\vfill

\footnotesize
\textsc{Quelle}: \titel. Herausgegeben von {\editorInnen}. In: \emph{Arthur Schnitzler: Briefwechsel mit Autorinnen und Autoren}.
 Digitale Edition, https://schnitzler-briefe.acdh.oeaw.ac.at/{\dateiname}.html (Stand \today)
\fi

\end{document}


