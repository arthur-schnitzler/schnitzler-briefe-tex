%% latex-leseansicht-vorspann.tex
%% Vorspann für die Leseansicht.
%% Lädt die gemeinsame Datei latex-vorspann.tex mit nicht gesetztem Schalter.

\newif\ifkorrekturansicht
\korrekturansichtfalse

\input{../tex-inputs/latex-vorspann}


         
         \newcommand{\erwaehntePersonen}{Personen: Paula Beer-Hofmann, Olga Schnitzler}
         \newcommand{\erwaehnteInstitutionen}{}
         \newcommand{\erwaehnteOrte}{Orte: Edmund-Weiß-Gasse, Grand Hotel des Bains, Lido, Ponte Gardena, Seis am Schlern, Südtirol, Venedig, Wien, Österreich}
         \newcommand{\erwaehnteWerke}{
               \section[Richard Beer-Hofmann an Arthur Schnitzler, {[}zwischen 1. und 12. 9.? 1908{]}]{ Richard Beer-Hofmann an Arthur Schnitzler, {[}zwischen 1. und
               12. 9.? 1908{]}}\nopagebreak\mylabel{v}\rehead{ }\begin{ledgroupsized}[t]{13cm}\normalsize\beginnumbering \toendnotes[C]{\smallbreak\pagebreak[2]} \Standort{CUL, Schnitzler, B 8.}
\physDesc{Bildpostkarte
\newline{}Handschrift: schwarze Tinte, lateinische Kurrent\newline{}Versand: nachgesandt nach »\textsc{Spöttlgasse 7 Wien XVIII\oindex{Edmund-Weiss-Gasse@\textbf{Edmund-Weiß-Gasse}|pw}}« 
\newline{}Schnitzler: mit Bleistift datiert: »Juli (?) 08« und beschriftet: »\textsc{Beerh}« \newline{}Ordnung: 1) mit Bleistift von unbekannter Hand nummeriert: »\strikeout{216}«  2) mit Bleistift von unbekannter Hand nummeriert:
                                    »217«}\buchAbdrucke{\weitereDrucke{Arthur Schnitzler, Richard Beer-Hofmann: \emph{Briefwechsel 1891–1931}. Hg. Konstanze Fliedl. Wien, Zürich: \emph{Europaverlag} 1992, S. 190.} }\toendnotes[C]{\smallbreak}\pstart{}{\pb}D\textsuperscript{r}
                  Arthur Schnitzler\pend{}\pstart{}Seis\oindex{Seis am Schlern@\textbf{Seis am Schlern}|pw}\pend{}\pstart{}bei Waidbruck\oindex{Ponte Gardena@\textbf{Ponte Gardena}|pw}\pend{}\pstart{}Süd-Tirol\oindex{Suedtirol@\textbf{Südtirol}|pw}\pend{}\pstart{}Austria\oindex{Oesterreich@\textbf{Österreich}|pw}\pend{}{\bigskip}\pstart
           \noindent{}\centering{}{\pb}\textcolor{gray}{\textbf{VENEZIA\oindex{Venedig@\textbf{Venedig}|pw}}}\pend
           \pstart
           \noindent{}\centering{}Lido\hspace*{1.5em}Hôtel des
                     Bains\oindex{Grand Hotel des Bains@\textbf{Grand Hotel des Bains}|pw}\pend
           \pstart
           {\pb}Lieber Arthur! Ich war still, da ich nicht ja{\geminationm}ern wollte. Paula\pwindex{Beer-Hofmann, Paula 25.02.1879 – 30.10.1939@\textsc{Beer-Hofmann, Paula} (25.02.1879 – 30.10.1939)|pw}
               hatte Hals- u. Rippenfellentzündung. Wir wollen nach 15 hier weg, über
                  Südtirol\oindex{Suedtirol@\textbf{Südtirol}|pw} nach Hause. Hoffentlich \label{K_L01788_1v}\edtext{schickt man die Karte Ihnen nach}{\lemma{\textnormal{\emph{schickt … nach}}}\Cendnote{\textnormal{Der Poststempel ist nicht entzifferbar. Die
                  Jahreszahl ist durch die Erkrankung Paula\pwindex{Beer-Hofmann, Paula 25.02.1879 – 30.10.1939@\textsc{Beer-Hofmann, Paula} (25.02.1879 – 30.10.1939)|pwk}s
                  gesichert. Schnitzler\pwindex{Schnitzler, Arthur 15.05.1862 – 21.10.1931@\textsc{Schnitzler, Arthur} (15.05.1862 – 21.10.1931), \emph{Schriftsteller, Mediziner}|pwk}s mit Fragezeichen
                  versehene Angabe »Juli?« ist jedoch nicht haltbar, da er sich zu
                  dieser Zeit in Seis\oindex{Seis am Schlern@\textbf{Seis am Schlern}|pwk} aufhielt, eine Nachsendung
                  also nicht notwendig gewesen wäre. Diese Nachsendung hat auch stattgefunden und Schnitzler\pwindex{Schnitzler, Arthur 15.05.1862 – 21.10.1931@\textsc{Schnitzler, Arthur} (15.05.1862 – 21.10.1931), \emph{Schriftsteller, Mediziner}|pwk} die Karte zwei Tage vor dem 16. 9. 1908 erhalten. Damit lässt
                  sich der Zeitraum zwischen der Abreise aus Seis\oindex{Seis am Schlern@\textbf{Seis am Schlern}|pwk}
                  und der Ankunft in Wien\oindex{Wien@\textbf{Wien}|pwk} am 14. 9. 1908 für den
                  Versand der Karte bestimmen.}}}\label{K_L01788_1h}. \label{T_L01788_1v}\edtext{Wir grüssen Sie Beide\pwindex{Schnitzler, Olga 17.01.1882 – 13.01.1970@\textsc{Schnitzler, Olga} (17.01.1882 – 13.01.1970), \emph{Schauspielerin, Sängerin}|pwv}
                  herzlich.}{\lemma{\textnormal{\emph{Wir … herzlich.}}}\Cendnote{\textnormal{weiter quer am linken
                  Rand}}}\label{T_L01788_1h}\pend
           \pstart \spacefill\mbox{Richard}\pend{}
         
         \endnumbering\mylabel{h}\end{ledgroupsized}  \newcommand{\dateiname}{L01788}\newcommand{\titel}{Richard Beer-Hofmann an Arthur Schnitzler, [zwischen 1. und 12. 9.? 1908]}\newcommand{\editorInnen}{Martin Anton Müller und Gerd-Hermann Susen}%% latex-leseansicht-abspann.tex
%% Abspann für die Leseansicht.
%% Der Schalter \ifkorrekturansicht ist bereits durch den Vorspann gesetzt.

%% latex-abspann.tex
%% Gemeinsamer Abspann für Korrekturansicht und Leseansicht.
%% Setzt den Schalter \ifkorrekturansicht voraus (gesetzt in den
%% einbindenden Dateien latex-korrekturansicht-abspann.tex bzw.
%% latex-leseansicht-abspann.tex).
%% ---------------------------------------------------------------

\normalsize

% Das esempio-Environment wird nur in der Leseansicht benötigt
\ifkorrekturansicht\else
\newenvironment{esempio}[3]%
{
    \vspace{1.5ex}
    \rlap{\underline{#1}}
    \par
    \setlength{\parindent}{0cm}
    \nopagebreak
    \leftskip=#2cm
    \rightskip=#3cm
}
{
    \par
}
\fi

\doendnotes{C}
\bigskip
\vfill

\clearpage

\footnotesize

\ifkorrekturansicht
  \lohead{\textsc{register}}
\fi

% theindex-Environment neu definieren ohne reledmac
\makeatletter
\renewenvironment{theindex}{%
  \ifkorrekturansicht
    \section*{\indexname}%
  \else
    \subsubsection*{Index der erwähnten Entitäten}%
  \fi
  \setlength{\parindent}{0pt}%
  \setlength{\parskip}{0pt plus 0.3pt}%
  \let\item\@idxitem
}{%
  \ifkorrekturansicht\clearpage\fi
}
\makeatother

\IfFileExists{\jobname-pw.ind}{\input{\jobname-pw.ind}}{}

% Quellenangabe nur in der Leseansicht
\ifkorrekturansicht\else
% Fallback-Definitionen, falls die .tex-Datei \titel etc. nicht gesetzt hat
\providecommand{\titel}{}
\providecommand{\editorInnen}{}
\providecommand{\dateiname}{\jobname}

\vspace{3cm}

\vfill

\footnotesize
\textsc{Quelle}: \titel. Herausgegeben von {\editorInnen}. In: \emph{Arthur Schnitzler: Briefwechsel mit Autorinnen und Autoren}.
 Digitale Edition, https://schnitzler-briefe.acdh.oeaw.ac.at/{\dateiname}.html (Stand \today)
\fi

\end{document}


      