\input{../tex-inputs/latex-pdf-vorspann}
\begin{center}
            \textcolor{red}{ENTWURF. ENTZIFFERUNG NOCH NICHT KORREKTURGELESEN}
                      \end{center}
            
               \section[Hugo August von Hofmannsthal an Arthur Schnitzler, 16. 11. 1898]{ Hugo August von Hofmannsthal an Arthur Schnitzler,
                    16. 11. 1898}\nopagebreak\mylabel{v}\rehead{ }\begin{ledgroupsized}[t]{13cm}\normalsize\beginnumbering\briefempfaengerindex{Schnitzler, Arthur@\textsc{Schnitzler, Arthur}!zzzHofmannsthal, Hugo August von@\emph{von Hugo August von Hofmannsthal}!1898-11-161@{16. 11. 1898}|(be} \toendnotes[C]{\smallbreak\pagebreak[2]} \Standort{DLA, A:Schnitzler, HS.NZ85.1.3483.}
\physDesc{Postkarte
\newline{}Handschrift: schwarze Tinte, lateinische Kurrent\newline{}Versand: 1) Stempel: »\nobreak{}\oindex{I., Innere Stadt@\textbf{I., Innere Stadt}|pwk}Wien 1/1, 16. 11. 98, 7–8V\nobreak{}«.  2) Stempel: »\nobreak{}Wien, 16. 11. 98, 9.V, Bestellt\nobreak{}«. }\pstart{}{\pb}Herrn D\textsuperscript{r} Arthur
                        Schnitzler\pend{}\pstart{}Wien\oindex{Wien@\textbf{Wien}|pw}\pend{}\pstart{}IX Frankgaße N\textsuperscript{o} 1\oindex{Frankgasse@\textbf{Frankgasse}|pw}.\pend{}{\bigskip}\pstart
           \raggedleft{}{\pb}Dienstag, 15/11 98\pend
           \pstart
           Hugo\pwindex{Hofmannsthal, Hugo von 01.02.1874 – 15.07.1929@\textsc{Hofmannsthal, Hugo von} (01.02.1874 – 15.07.1929), \emph{Schriftsteller}|pw} depeschiert erst heute daß er »aus
                    angenehmen Gründen« erst Donnerstag abends ko{\geminationm}t\pend
           \pstart
           Bestens grüßend Ihr{\\[\baselineskip]}\spacefill\mbox{D\textsuperscript{r}Hofmannsthal}\pend
           \leftskip=0em{}\endnumbering\briefempfaengerindex{Schnitzler, Arthur@\textsc{Schnitzler, Arthur}!zzzHofmannsthal, Hugo August von@\emph{von Hugo August von Hofmannsthal}!1898-11-161@{16. 11. 1898}|)be}\mylabel{h}\end{ledgroupsized}  \newcommand{\dateiname}{L00857}\newcommand{\titel}{Hugo August von Hofmannsthal an Arthur Schnitzler, 16. 11. 1898}\newcommand{\editorInnen}{Martin Anton Müller und Gerd-Hermann Susen}\input{../tex-inputs/latex-pdf-abspann}
      