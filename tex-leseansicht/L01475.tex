%% latex-leseansicht-vorspann.tex
%% Vorspann für die Leseansicht.
%% Lädt die gemeinsame Datei latex-vorspann.tex mit nicht gesetztem Schalter.

\newif\ifkorrekturansicht
\korrekturansichtfalse

\input{../tex-inputs/latex-vorspann}


         
         \renewcommand{\erwaehntePersonen}{Personen: Hermann Bahr, Albert Bassermann, Gertrud Eysoldt, Gustav Gurschner, Olga Schnitzler}
         \renewcommand{\erwaehnteOrte}{Orte: Berlin, Deutsches Theater Berlin, Edmund-Weiß-Gasse, Wien}
         \renewcommand{\erwaehnteWerke}{Werke: Arthur Schnitzler, Der Puppenspieler, Tristan und Isolde}
               \section[Arthur Schnitzler an Hermann Bahr, 5. 12. 1904]{ Arthur Schnitzler an Hermann Bahr, 5. 12. 1904}\nopagebreak\mylabel{v}\rehead{ }\begin{ledgroupsized}[t]{13cm}\normalsize\beginnumbering \toendnotes[C]{\smallbreak\pagebreak[2]} \Standort{TMW, HS AM 23369 Ba.}
\physDesc{Brief, 1 Blatt, 4 Seiten
\newline{}Handschrift: schwarze Tinte, deutsche Kurrent\newline{}Ordnung: Lochung }\buchAbdrucke{\weitereDrucke{1) Arthur Schnitzler: \emph{Briefe 1875–1912}. Hg. Therese Nickl und Heinrich Schnitzler. Frankfurt am Main: \emph{S. Fischer} 1981, S. 499.} \weitereDrucke{2) \emph{5. 12. 1904.} In: Arthur Schnitzler: \emph{The Letters of Arthur Schnitzler to Hermann Bahr}. Edited, annotated, and with an introduction, by Donald G.
                        Daviau. Chapel Hill: \emph{The University of North Carolina Press} 1978, S. 86 (University of North Carolina studies in the Germanic languages
                        and literatures, 89).} \weitereDrucke{3) Hermann Bahr, Arthur Schnitzler: \emph{Briefwechsel, Aufzeichnungen, Dokumente (1891–1931)}. Hg. Kurt Ifkovits und Martin Anton Müller. Göttingen: \emph{Wallstein} 2018, S. 327.} }\toendnotes[C]{\smallbreak}\pstart
           \noindent{}\raggedleft{}{\pb}XVIII \textsc{Spoettelg.} 7\oindex{Edmund-Weiss-Gasse@\textbf{Edmund-Weiß-Gasse}|pw}\pend
           \pstart
           \raggedleft{}\textsc{Wien\oindex{Wien@\textbf{Wien}|pw}}, 5. 12. 904\pend
           \pstart{}lieber Hermann,\pend\pstart
           dictiren u ſitzen (\label{K_L01475_1v}\edtext{Relief\pwindex{Gurschner, Gustav 28.09.1873 – 02.08.1970@\textsc{Gurschner, Gustav} (28.09.1873 – 02.08.1970), \emph{Bildhauer}!Arthur Schnitzler1905@\strich\emph{Arthur Schnitzler} {[}1905{]}|pwv}}{\lemma{\textnormal{\emph{Relief}}}\Cendnote{\textnormal{bei Gustav
                     Gurschner\pwindex{Gurschner, Gustav 28.09.1873 – 02.08.1970@\textsc{Gurschner, Gustav} (28.09.1873 – 02.08.1970), \emph{Bildhauer}|pwk}}}}\label{K_L01475_1h}) und allerlei andres haben mich abgehalten, dich aufzuſuchen und dir die
               vielen Grüße persönlich zu überbringen, die mir, am heftigſten von Frau \textsc{Eysoldt}\pwindex{Eysoldt, Gertrud 30.11.1870 – 05.01.1955@\textsc{Eysoldt, Gertrud} (30.11.1870 – 05.01.1955), \emph{Theaterleiterin, Schauspielerin}|pw}, an dich aufgetragen worden ſind. Hoffentlich können wir dich an einem Abend zu
               Beginn nächſter Woche bei uns ſehen {\pb}und bei dieſer
               Gelegenheit auch über den Weihnachtsausflug reden, zu dem große Luſt vorhanden iſt.
               (Wahrſcheinlich aber würden wir erſt nach dem in jüdischen Kreiſen ſo heiligen Abend
               abfahren.) Da wir ſchon bei den frommen Feſten halten, theile ich dir auch mit, daſs
               ich zum Nicolo den Triſtan-Auszug\pwindex{\textcolor{red}{\textsuperscript{XXXX1 indx}}!Tristan und Isolde1865@\strich\emph{Tristan und Isolde} {[}1865{]}|pw} beko{\geminationm}en habe, ihn aber {\pb}noch ſpiele wie ein
               Krampus. –\pend
           \pstart
           Laß es dir weiter wohl ſein im neu errungenen Glück der Töne – warum ſuchſt du irgend
               ein Vorgefühl darin? Eine Seligkeit hat genug \strikeout{damit}
               zu thun, wenn ſie ſich ſelbſt bedeutet. – \pend
           \pstart
           Beigeſchloſſen der »Puppenſpieler\pwindex{Schnitzler, Arthur 15.05.1862 – 21.10.1931@\textsc{Schnitzler, Arthur} (15.05.1862 – 21.10.1931), \emph{Schriftsteller, Mediziner}!Puppenspieler31. 05. 1903@\strich\emph{Der Puppenspieler} {[}31. 05. 1903{]}|pw}«, den \label{K_L01475_2v}\edtext{Baſſermann\pwindex{Bassermann, Albert 07.09.1867 – 15.05.1952@\textsc{Bassermann, Albert} (07.09.1867 – 15.05.1952), \emph{Schauspieler}|pw} in Berlin\oindex{Berlin@\textbf{Berlin}|pw}}{\lemma{\textnormal{\emph{Baſſermann in Berlin}}}\Cendnote{\textnormal{Bassermann\pwindex{Bassermann, Albert 07.09.1867 – 15.05.1952@\textsc{Bassermann, Albert} (07.09.1867 – 15.05.1952), \emph{Schauspieler}|pwk} hatte in der Uraufführung am
                     14. 9. 1903 im Deutschen Theater\oindex{Deutsches Theater Berlin@\textbf{Deutsches Theater Berlin}|pwk}
                  die Hauptrolle.}}}\label{K_L01475_2h} wundervoll gegeben haben soll. –\pend
           \pstart
           Auf Wiederſehen und herzliche Grüße {\pb}auch von meiner Frau\pwindex{Schnitzler, Olga 17.01.1882 – 13.01.1970@\textsc{Schnitzler, Olga} (17.01.1882 – 13.01.1970), \emph{Schauspielerin, Sängerin}|pwv}.{\\[\baselineskip]}Dein{\\[\baselineskip]}\spacefill\mbox{A.}\pend
           \leftskip=0em{}
         
         \endnumbering\mylabel{h}\end{ledgroupsized}  \newcommand{\dateiname}{L01475}\newcommand{\titel}{Arthur Schnitzler an Hermann Bahr, 5. 12. 1904}\newcommand{\editorInnen}{ Kurt Ifkovits,  Martin Anton Müller}%% latex-leseansicht-abspann.tex
%% Abspann für die Leseansicht.
%% Der Schalter \ifkorrekturansicht ist bereits durch den Vorspann gesetzt.

%% latex-abspann.tex
%% Gemeinsamer Abspann für Korrekturansicht und Leseansicht.
%% Setzt den Schalter \ifkorrekturansicht voraus (gesetzt in den
%% einbindenden Dateien latex-korrekturansicht-abspann.tex bzw.
%% latex-leseansicht-abspann.tex).
%% ---------------------------------------------------------------

\normalsize

% Das esempio-Environment wird nur in der Leseansicht benötigt
\ifkorrekturansicht\else
\newenvironment{esempio}[3]%
{
    \vspace{1.5ex}
    \rlap{\underline{#1}}
    \par
    \setlength{\parindent}{0cm}
    \nopagebreak
    \leftskip=#2cm
    \rightskip=#3cm
}
{
    \par
}
\fi

\doendnotes{C}
\bigskip
\vfill

\clearpage

\footnotesize

\ifkorrekturansicht
  \lohead{\textsc{register}}
\fi

% theindex-Environment neu definieren ohne reledmac
\makeatletter
\renewenvironment{theindex}{%
  \ifkorrekturansicht
    \section*{\indexname}%
  \else
    \subsubsection*{Index der erwähnten Entitäten}%
  \fi
  \setlength{\parindent}{0pt}%
  \setlength{\parskip}{0pt plus 0.3pt}%
  \let\item\@idxitem
}{%
  \ifkorrekturansicht\clearpage\fi
}
\makeatother

\IfFileExists{\jobname-pw.ind}{\input{\jobname-pw.ind}}{}

% Quellenangabe nur in der Leseansicht
\ifkorrekturansicht\else
% Fallback-Definitionen, falls die .tex-Datei \titel etc. nicht gesetzt hat
\providecommand{\titel}{}
\providecommand{\editorInnen}{}
\providecommand{\dateiname}{\jobname}

\vspace{3cm}

\vfill

\footnotesize
\textsc{Quelle}: \titel. Herausgegeben von {\editorInnen}. In: \emph{Arthur Schnitzler: Briefwechsel mit Autorinnen und Autoren}.
 Digitale Edition, https://schnitzler-briefe.acdh.oeaw.ac.at/{\dateiname}.html (Stand \today)
\fi

\end{document}


      