%% latex-leseansicht-vorspann.tex
%% Vorspann für die Leseansicht.
%% Lädt die gemeinsame Datei latex-vorspann.tex mit nicht gesetztem Schalter.

\newif\ifkorrekturansicht
\korrekturansichtfalse

\input{../tex-inputs/latex-vorspann}


\section[Arthur Schnitzler an Hermann Bahr, 5. 12. 1904]{L01475 Arthur Schnitzler an Hermann Bahr, 5. 12. 1904}
\nopagebreak\mylabel{L01475v}
\rehead{ }\normalsize\beginnumbering\briefempfaengerindex{Bahr, Hermann@\textsc{Bahr, Hermann}!zzzSchnitzler, Arthur@\emph{von Arthur Schnitzler}!1904-12-051@{5. 12. 1904}|(be}
\toendnotes[C]{\smallbreak\pagebreak[2]}
\correspDesc{Versand  durch Arthur Schnitzler am 5. 12. 1904 in Wien
\newline{}Erhalt  durch Hermann Bahr im Zeitraum [5. 12. 1904
                  – 9. 12. 1904?] \textbf{Ort fehlend} }\toendnotes[C]{\smallbreak}
\Standort{TMW, HS AM 23369 Ba.}
\physDesc{Brief, 1 Blatt, 4 Seiten, 1019 Zeichen
\newline{}Handschrift: schwarze Tinte, deutsche Kurrent
\newline{}Ordnung: Lochung }
\buchAbdrucke{\weitereDrucke{1) Arthur Schnitzler: \emph{Briefe 1875–1912}. Herausgegeben von Therese Nickl und Heinrich Schnitzler. Frankfurt am Main: \emph{S. Fischer} 1981, S. 499.} \weitereDrucke{2) \emph{5. 12. 1904.} In: Arthur Schnitzler: \emph{The Letters of Arthur Schnitzler to Hermann Bahr}. Edited, annotated, and with an introduction, by Donald G. Daviau. Chapel Hill: \emph{The University of North Carolina Press} 1978, S. 86 (University of North Carolina studies in the Germanic languages
                        and literatures, 89).} \weitereDrucke{3) Hermann Bahr, Arthur Schnitzler: \emph{Briefwechsel, Aufzeichnungen, Dokumente (1891–1931)}. Herausgegeben von Kurt Ifkovits und Martin Anton Müller. Göttingen: \emph{Wallstein} 2018, S. 327.} }\toendnotes[C]{\smallbreak}
\pstart
           \raggedleft{}{\pb}XVIII \textsc{Spoettelg.} 7\oindex{Wien@\textbf{Wien}!XVIII., Währing@\textbf{XVIII., Währing}!Edmund-Weiß-Gasse 7@\textbf{Edmund-Weiß-Gasse 7}, \emph{Wohngebäude}|pw}\pend
           
\pstart
           \raggedleft{}\textsc{Wien\oindex{Wien@\textbf{Wien}, \emph{Verwaltungsgebiet}|pw}}, 5. 12. 904\pend
           
\pstart{}lieber Hermann,\pend\vspace{0.5em}
\pstart
           dictiren u{ }ſitzen (\label{K_L01475-1v}\edtext{Relief\pwindex{Gurschner, Gustav 28.\,9.\,1873 Mühldorf – 2.\,8.\,1970 Wien@\textsc{Gurschner, Gustav} (28.\,9.\,1873 Mühldorf – 2.\,8.\,1970 Wien), \emph{Bildhauer}!Arthur Schnitzler@\strich\emph{Arthur Schnitzler}|pwv}}{\lemma{\textnormal{\emph{Relief}}}\Cendnote{\textnormal{bei Gustav Gurschner\pwindex{Gurschner, Gustav 28.\,9.\,1873 Mühldorf – 2.\,8.\,1970 Wien@\textsc{Gurschner, Gustav} (28.\,9.\,1873 Mühldorf – 2.\,8.\,1970 Wien), \emph{Bildhauer}|pwk}}}}\label{K_L01475-1}) und allerlei andres haben mich abgehalten, dich aufzuſuchen und dir die
               vielen Grüße persönlich zu überbringen, die mir, am heftigſten von Frau \textsc{Eysoldt}\pwindex{Eysoldt, Gertrud 30.\,11.\,1870 Pirna – 5.\,1.\,1955 Ohlstadt@\textsc{Eysoldt, Gertrud} (30.\,11.\,1870 Pirna – 5.\,1.\,1955 Ohlstadt), \emph{Theaterleiterin, Schauspielerin}|pw}, an dich aufgetragen worden{ }ſind. Hoffentlich können wir dich an einem Abend zu
               Beginn nächſter Woche bei uns{ }ſehen {\pb}und bei dieſer
               Gelegenheit auch über den Weihnachtsausflug reden, zu dem große Luſt vorhanden iſt.
               (Wahrſcheinlich aber würden wir erſt nach dem in jüdischen Kreiſen{ }ſo heiligen Abend
               abfahren.) Da wir{ }ſchon bei den frommen Feſten halten, theile ich dir auch mit, daſs
               ich zum Nicolo den Triſtan-Auszug\pwindex{\textcolor{red}{\textsuperscript{XXXX indx1}}!Tristan und Isolde@\strich\emph{Tristan und Isolde}|pw} beko{\geminationm}en habe, ihn aber {\pb}noch{ }ſpiele wie ein
               Krampus. –\pend
           
\pstart
           Laß es dir weiter wohl{ }ſein im neu errungenen Glück der Töne – warum{ }ſuchſt du irgend
               ein Vorgefühl darin? Eine Seligkeit hat genug \strikeout{damit}
               zu thun, wenn{ }ſie{ }ſich{ }ſelbſt bedeutet. –\pend
           
\pstart
           Beigeſchloſſen der »Puppenſpieler\pwindex{Schnitzler, Arthur 15.\,5.\,1862 Wien – 21.\,10.\,1931 ebd.@\textsc{Schnitzler, Arthur} (15.\,5.\,1862 Wien – 21.\,10.\,1931 ebd.), \emph{Schriftsteller, Mediziner}!Puppenspieler. Studie in einem Aufzuge@\strich\emph{Der Puppenspieler. Studie in einem Aufzuge}|pw}«, den \label{K_L01475-2v}\edtext{Baſſermann\pwindex{Bassermann, Albert 7.\,9.\,1867 Mannheim – 15.\,5.\,1952 Atlantischer Ozean@\textsc{Bassermann, Albert} (7.\,9.\,1867 Mannheim – 15.\,5.\,1952 Atlantischer Ozean), \emph{Schauspieler}|pw} in Berlin\oindex{Berlin@\textbf{Berlin}, \emph{Hauptstadt}|pw}}{\lemma{\textnormal{\emph{Bassermann in Berlin}}}\Cendnote{\textnormal{Bassermann\pwindex{Bassermann, Albert 7.\,9.\,1867 Mannheim – 15.\,5.\,1952 Atlantischer Ozean@\textsc{Bassermann, Albert} (7.\,9.\,1867 Mannheim – 15.\,5.\,1952 Atlantischer Ozean), \emph{Schauspieler}|pwk} hatte in der Uraufführung\eventindex{Deutsches Theater Berlin@\textbf{Deutsches Theater Berlin}!Uraufführung von Der Puppenspieler, Premiere von Das Trugbild, 12.9.1903@Uraufführung von Der Puppenspieler, Premiere von Das Trugbild, 12.9.1903|pwkv} von \emph{Der Puppenspieler}\pwindex{Schnitzler, Arthur 15.\,5.\,1862 Wien – 21.\,10.\,1931 ebd.@\textsc{Schnitzler, Arthur} (15.\,5.\,1862 Wien – 21.\,10.\,1931 ebd.), \emph{Schriftsteller, Mediziner}!Puppenspieler. Studie in einem Aufzuge@\strich\emph{Der Puppenspieler. Studie in einem Aufzuge}|pwk} am
                  12. 9. 1903 am \emph{Deutschen Theater}\orgindex{Deutsches Theater Berlin@Deutsches Theater Berlin|pwk}
                  die Hauptrolle.}}}\label{K_L01475-2} wundervoll gegeben haben soll. –\pend
           
\pstart
           Auf Wiederſehen und herzliche Grüße {\pb}auch von meiner Frau\pwindex{Schnitzler, Olga 17.\,1.\,1882 Wien – 13.\,1.\,1970 Lugano@\textsc{Schnitzler, Olga} (17.\,1.\,1882 Wien – 13.\,1.\,1970 Lugano), \emph{Schauspielerin, Sängerin}|pwv}.{\\[\baselineskip]}Dein{\\[\baselineskip]}\spacefill\mbox{A.}\pend
           \leftskip=0em{}\selectlanguage{ngerman}\endnumbering\briefempfaengerindex{Bahr, Hermann@\textsc{Bahr, Hermann}!zzzSchnitzler, Arthur@\emph{von Arthur Schnitzler}!1904-12-051@{5. 12. 1904}|)be}\mylabel{L01475h}  \newcommand{\dateiname}{L01475}\newcommand{\titel}{Arthur Schnitzler an Hermann Bahr, 5. 12. 1904}\newcommand{\editorInnen}{Herausgegeben von Martin Anton Müller}%% latex-leseansicht-abspann.tex
%% Abspann für die Leseansicht.
%% Der Schalter \ifkorrekturansicht ist bereits durch den Vorspann gesetzt.

%% latex-abspann.tex
%% Gemeinsamer Abspann für Korrekturansicht und Leseansicht.
%% Setzt den Schalter \ifkorrekturansicht voraus (gesetzt in den
%% einbindenden Dateien latex-korrekturansicht-abspann.tex bzw.
%% latex-leseansicht-abspann.tex).
%% ---------------------------------------------------------------

\normalsize

% Das esempio-Environment wird nur in der Leseansicht benötigt
\ifkorrekturansicht\else
\newenvironment{esempio}[3]%
{
    \vspace{1.5ex}
    \rlap{\underline{#1}}
    \par
    \setlength{\parindent}{0cm}
    \nopagebreak
    \leftskip=#2cm
    \rightskip=#3cm
}
{
    \par
}
\fi

\doendnotes{C}
\bigskip
\vfill

\clearpage

\footnotesize

\ifkorrekturansicht
  \lohead{\textsc{register}}
\fi

% theindex-Environment neu definieren ohne reledmac
\makeatletter
\renewenvironment{theindex}{%
  \ifkorrekturansicht
    \section*{\indexname}%
  \else
    \subsubsection*{Index der erwähnten Entitäten}%
  \fi
  \setlength{\parindent}{0pt}%
  \setlength{\parskip}{0pt plus 0.3pt}%
  \let\item\@idxitem
}{%
  \ifkorrekturansicht\clearpage\fi
}
\makeatother

\IfFileExists{\jobname-pw.ind}{\input{\jobname-pw.ind}}{}

% Quellenangabe nur in der Leseansicht
\ifkorrekturansicht\else
% Fallback-Definitionen, falls die .tex-Datei \titel etc. nicht gesetzt hat
\providecommand{\titel}{}
\providecommand{\editorInnen}{}
\providecommand{\dateiname}{\jobname}

\vspace{3cm}

\vfill

\footnotesize
\textsc{Quelle}: \titel. Herausgegeben von {\editorInnen}. In: \emph{Arthur Schnitzler: Briefwechsel mit Autorinnen und Autoren}.
 Digitale Edition, https://schnitzler-briefe.acdh.oeaw.ac.at/{\dateiname}.html (Stand \today)
\fi

\end{document}


