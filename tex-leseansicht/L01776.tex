%% latex-leseansicht-vorspann.tex
%% Vorspann für die Leseansicht.
%% Lädt die gemeinsame Datei latex-vorspann.tex mit nicht gesetztem Schalter.

\newif\ifkorrekturansicht
\korrekturansichtfalse

\input{../tex-inputs/latex-vorspann}


\section[Hugo von Hofmannsthal an Arthur Schnitzler, 12. 6. 1908]{L01776 Hugo von Hofmannsthal an Arthur Schnitzler, 12. 6. 1908}
\nopagebreak\mylabel{L01776v}
\rehead{ }\normalsize\beginnumbering\briefempfaengerindex{Schnitzler, Arthur@\textsc{Schnitzler, Arthur}!zzzHofmannsthal, Hugo von@\emph{von Hugo von Hofmannsthal}!1908-06-121@{12. 6. 1908}|(be}
\toendnotes[C]{\smallbreak\pagebreak[2]}
\correspDesc{Versand  durch Hugo von Hofmannsthal am 12. 6. 1908 in Rodaun
\newline{}Erhalt  durch Arthur Schnitzler im Zeitraum [13. 6. 1908
                  – 17. 6. 1908?] in Wien}\toendnotes[C]{\smallbreak}
\Standort{CUL, Schnitzler, B 43.}
\physDesc{Postkarte, 292 Zeichen
\newline{}Handschrift: schwarze Tinte, deutsche Kurrent
\newline{}Versand: Stempel: »\nobreak{}\oindex{Wien@\textbf{Wien}!XXIII., Liesing@\textbf{XXIII., Liesing}!Rodaun@\textbf{Rodaun}, \emph{Region}|pwk}Rodaun, 12 6 08, 9 V\nobreak{}«.  
\newline{}Schnitzler: mit Bleistift datiert: »12/6 908« 
\newline{}Ordnung: 1) mit Bleistift von unbekannter Hand nummeriert: »\strikeout{291}«  2) mit Bleistift von unbekannter Hand nummeriert:
                                    »297«}
\buchAbdrucke{\weitereDrucke{Hugo von Hofmannsthal, Arthur Schnitzler: \emph{Briefwechsel}. Herausgegeben von Therese Nickl und Heinrich Schnitzler. Frankfurt am Main: \emph{S. Fischer} 1964, S. 237.} }\toendnotes[C]{\smallbreak}\pstart{}\textsc{{\pb}Herrn Dr Arthur
                  Schnitzler}\pend{}\pstart{}\textsc{Wien\oindex{Wien@\textbf{Wien}, \emph{Verwaltungsgebiet}|pw}}\pend{}\pstart{}\textsc{XVIII Spöttelgasse 7\oindex{Wien@\textbf{Wien}!XVIII., Währing@\textbf{XVIII., Währing}!Edmund-Weiß-Gasse 7@\textbf{Edmund-Weiß-Gasse 7}, \emph{Wohngebäude}|pw}}\pend{}{\bigskip}\vspace{1em}
\pstart
           \noindent{}{\pb}Lieber,{ }Clemens Franckenſtein\pwindex{Franckenstein, Clemens von 14.\,7.\,1875 Wiesentheid – 19.\,8.\,1942 Hechendorf am Pilsensee@\textsc{Franckenstein, Clemens von} (14.\,7.\,1875 Wiesentheid – 19.\,8.\,1942 Hechendorf am Pilsensee), \emph{Theaterleiter, Komponist, Dirigent}|pw} bittet mich,{ }ſeinen Dank
               zu \substVorne{}\textsuperscript{Ü}\substDazwischen{}ü\substHinten{}bermitteln für gütige Zuſendung Ihres Buches\pwindex{Schnitzler, Arthur 15.\,5.\,1862 Wien – 21.\,10.\,1931 ebd.@\textsc{Schnitzler, Arthur} (15.\,5.\,1862 Wien – 21.\,10.\,1931 ebd.), \emph{Schriftsteller, Mediziner}!Weg ins Freie. Roman@\strich\emph{Der Weg ins Freie. Roman}|pwv}, da er Ihre Adreſſe nicht weiß.\pend
           
\pstart
           Ich hoffe, baldigſt von Ihnen zu hören, daſs Sie in der Arbeit u. zufrieden{ }ſind. Ich
               arbeite.\pend
           
\pstart
           Von Herzen.{\\[\baselineskip]}\spacefill\mbox{Hugo.}\pend
           \leftskip=0em{}\selectlanguage{ngerman}\endnumbering\briefempfaengerindex{Schnitzler, Arthur@\textsc{Schnitzler, Arthur}!zzzHofmannsthal, Hugo von@\emph{von Hugo von Hofmannsthal}!1908-06-121@{12. 6. 1908}|)be}\mylabel{L01776h}  \newcommand{\dateiname}{L01776}\newcommand{\titel}{Hugo von Hofmannsthal an Arthur Schnitzler, 12. 6. 1908}\newcommand{\editorInnen}{Martin Anton Müller und Gerd-Hermann Susen}%% latex-leseansicht-abspann.tex
%% Abspann für die Leseansicht.
%% Der Schalter \ifkorrekturansicht ist bereits durch den Vorspann gesetzt.

%% latex-abspann.tex
%% Gemeinsamer Abspann für Korrekturansicht und Leseansicht.
%% Setzt den Schalter \ifkorrekturansicht voraus (gesetzt in den
%% einbindenden Dateien latex-korrekturansicht-abspann.tex bzw.
%% latex-leseansicht-abspann.tex).
%% ---------------------------------------------------------------

\normalsize

% Das esempio-Environment wird nur in der Leseansicht benötigt
\ifkorrekturansicht\else
\newenvironment{esempio}[3]%
{
    \vspace{1.5ex}
    \rlap{\underline{#1}}
    \par
    \setlength{\parindent}{0cm}
    \nopagebreak
    \leftskip=#2cm
    \rightskip=#3cm
}
{
    \par
}
\fi

\doendnotes{C}
\bigskip
\vfill

\clearpage

\footnotesize

\ifkorrekturansicht
  \lohead{\textsc{register}}
\fi

% theindex-Environment neu definieren ohne reledmac
\makeatletter
\renewenvironment{theindex}{%
  \ifkorrekturansicht
    \section*{\indexname}%
  \else
    \subsubsection*{Index der erwähnten Entitäten}%
  \fi
  \setlength{\parindent}{0pt}%
  \setlength{\parskip}{0pt plus 0.3pt}%
  \let\item\@idxitem
}{%
  \ifkorrekturansicht\clearpage\fi
}
\makeatother

\IfFileExists{\jobname-pw.ind}{\input{\jobname-pw.ind}}{}

% Quellenangabe nur in der Leseansicht
\ifkorrekturansicht\else
% Fallback-Definitionen, falls die .tex-Datei \titel etc. nicht gesetzt hat
\providecommand{\titel}{}
\providecommand{\editorInnen}{}
\providecommand{\dateiname}{\jobname}

\vspace{3cm}

\vfill

\footnotesize
\textsc{Quelle}: \titel. Herausgegeben von {\editorInnen}. In: \emph{Arthur Schnitzler: Briefwechsel mit Autorinnen und Autoren}.
 Digitale Edition, https://schnitzler-briefe.acdh.oeaw.ac.at/{\dateiname}.html (Stand \today)
\fi

\end{document}


