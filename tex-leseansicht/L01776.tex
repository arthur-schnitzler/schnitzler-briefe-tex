%% latex-korrekturansicht-vorspann.tex
%% Vorspann für die Korrekturansicht.
%% Lädt die gemeinsame Datei latex-vorspann.tex mit gesetztem Schalter.

\newif\ifkorrekturansicht
\korrekturansichttrue

\input{../tex-inputs/latex-vorspann}


\section[Hugo von Hofmannsthal an Arthur Schnitzler, 12. 6. 1908]{L01776 Hugo von Hofmannsthal an Arthur Schnitzler, 12. 6. 1908}
\nopagebreak\mylabel{L01776v}
\rehead{ }\normalsize\beginnumbering\briefempfaengerindex{Schnitzler, Arthur@\textsc{Schnitzler, Arthur}!zzzHofmannsthal, Hugo von@\emph{von Hugo von Hofmannsthal}!1908-06-121@{12. 6. 1908}|(be}
\toendnotes[C]{\smallbreak\pagebreak[2]}\Standort{CUL, Schnitzler, B 43.}
\physDesc{Postkarte, 292 Zeichen
\newline{}Handschrift: 1) schwarze Tinte, deutsche Kurrent\hspace{1em}2) schwarze Tinte, lateinische Kurrent (\noindent{}Adresse)\hspace{1em}
\newline{}Versand: Stempel: »\nobreak{}\oindex{Rodaun@\textbf{Rodaun}, \emph{A.ADM4}|pwk}Rodaun, 12 6 08, 9 V\nobreak{}«.  
\newline{}Schnitzler: mit Bleistift datiert: »12/6 908« 
\newline{}Ordnung: 1) mit Bleistift von unbekannter Hand nummeriert: »\strikeout{291}«  2) mit Bleistift von unbekannter Hand nummeriert:
                                    »297«}
\buchAbdrucke{\weitereDrucke{Hugo von Hofmannsthal, Arthur Schnitzler: \emph{Briefwechsel}. Frankfurt am Main: \emph{S. Fischer} 1964, S. 237.} }\toendnotes[C]{\smallbreak}\pstart{}{\pb}Herrn Dr Arthur
                  Schnitzler\pend{}\pstart{}Wien\oindex{Wien@\textbf{Wien}, \emph{A.ADM2}|pw}\pend{}\pstart{}XVIII Spöttelgasse 7\oindex{Edmund-Weiss-Gasse 7@\textbf{Edmund-Weiß-Gasse 7}, \emph{Wohngebäude (K.WHS)}|pw}\pend{}{\bigskip}\vspace{1em}
\pstart
           \noindent{}{\pb}Lieber,{ }Clemens Franckenſtein\pwindex{Franckenstein, Clemens von 14.07.1875 – 19.08.1942@\textsc{Franckenstein, Clemens von} (14.07.1875 – 19.08.1942), \emph{Theaterleiter/Theaterleiterin, Komponist/Komponistin, Dirigent/Dirigentin}|pw} bittet mich, ſeinen Dank
               zu \substVorne{}\textsuperscript{Ü}\substDazwischen{}ü\substHinten{}bermitteln für gütige Zuſendung Ihres Buches\pwindex{Weg ins Freie. Roman@\emph{Der Weg ins Freie. Roman}|pwv}, da er Ihre Adreſſe nicht weiß.\pend
           
\pstart
           Ich hoffe, baldigſt von Ihnen zu hören, daſs Sie in der Arbeit u. zufrieden ſind. Ich
               arbeite.\pend
           
\pstart
           Von Herzen.{\\[\baselineskip]}\spacefill\mbox{Hugo.}\pend
           \leftskip=0em{}\selectlanguage{ngerman}\endnumbering\briefempfaengerindex{Schnitzler, Arthur@\textsc{Schnitzler, Arthur}!zzzHofmannsthal, Hugo von@\emph{von Hugo von Hofmannsthal}!1908-06-121@{12. 6. 1908}|)be}\mylabel{L01776h}  \normalsize

\doendnotes{C}
\bigskip
\vfill

\clearpage

\footnotesize

\lohead{\textsc{register}}

% Definiere theindex-Environment komplett neu ohne reledmac
\makeatletter
\renewenvironment{theindex}{%
  \section*{\indexname}%
  \setlength{\parindent}{0pt}%
  \setlength{\parskip}{0pt plus 0.3pt}%
  \let\item\@idxitem
}{%
  \clearpage
}
\makeatother

\IfFileExists{\jobname-pw.ind}{\input{\jobname-pw.ind}}{}

\end{document}

      