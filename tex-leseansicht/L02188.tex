%% latex-leseansicht-vorspann.tex
%% Vorspann für die Leseansicht.
%% Lädt die gemeinsame Datei latex-vorspann.tex mit nicht gesetztem Schalter.

\newif\ifkorrekturansicht
\korrekturansichtfalse

\input{../tex-inputs/latex-vorspann}


\section[Arthur Schnitzler an Richard Beer-Hofmann, 25. 7. 1914]{L02188 Arthur Schnitzler an Richard Beer-Hofmann, 25. 7. 1914}
\nopagebreak\mylabel{L02188v}
\rehead{ }\normalsize\beginnumbering\briefempfaengerindex{Beer-Hofmann, Richard@\textsc{Beer-Hofmann, Richard}!zzzSchnitzler, Arthur@\emph{von Arthur Schnitzler}!1914-07-251@{25. 7. 1914}|(be}
\toendnotes[C]{\smallbreak\pagebreak[2]}
\correspDesc{Versand  durch Arthur Schnitzler am 25. 7. 1914 in Celerina
\newline{}Erhalt  durch Richard Beer-Hofmann im Zeitraum [26. 7. 1914
                  – 30. 7. 1914?] in Weißenbach am Attersee}\toendnotes[C]{\smallbreak}
\Standort{YCGL, MSS 31.}
\physDesc{Bildpostkarte, 262 Zeichen
\newline{}Handschrift: Bleistift, deutsche Kurrent
\newline{}Versand: Stempel: »\nobreak{}\oindex{Celerina@\textbf{Celerina}|pwk}Celerin\textcolor{gray}{a}
                                          (\textcolor{gray}{Graubünden}), 25. VII. 14, 5\nobreak{}«.  
\newline{}Beer-Hofmann: mit blauem Buntstift den Erhalt
                                 markiert: »E« }\toendnotes[C]{\smallbreak}\pstart{}{\pb}Hrn \textsc{Dr Richard Beerhofma{\geminationn}}\pend{}\pstart{}\textsc{Weißenbach\oindex{Weißenbach am Attersee@\textbf{Weißenbach am Attersee}, \emph{Verwaltungsgebiet}|pw}.}\pend{}\pstart{}\textsc{Am Attersee\oindex{Attersee@\textbf{Attersee}, \emph{See}|pw}}\pend{}\pstart{}\textsc{Oberoesterreic{[}h{]}}\oindex{Oberösterreich@\textbf{Oberösterreich}, \emph{Land}|pw}\pend{}\pstart{}\textsc{Austria}\oindex{Österreich@\textbf{Österreich}|pw}\pend{}{\bigskip}
\pstart
           \noindent{}\centering{}{\pb}\textcolor{gray}{\textbf{Celerina\oindex{Celerina@\textbf{Celerina}|pw} gegen Pontresina\oindex{Pontresina@\textbf{Pontresina}|pw} ges.}}\pend
           \vspace{1em}
\pstart
           \noindent{}{\pb}Aus Pontresina\oindex{Pontresina@\textbf{Pontresina}|pw}
               (zu lärmen\textcolor{gray}{d} – \textsc{Hotel}{ }ſowohl als Ort)
               hieher \label{K_L02188-1v}\edtext{überſiedelt}{\lemma{\textnormal{\emph{übersiedelt}}}\Cendnote{\textnormal{Siehe A. S.: \emph{Tagebuch}, 21. 7. 1914.
               }}}\label{K_L02188-1} (\textsc{Cresta Palace, Celerina\oindex{Cresta Palace@\textbf{Cresta Palace}, \emph{Hotel}|pw}}) und höchſt befriedigt, grüßen wir Sie alle {\pb}herzlichſt. Ebenſo Gustav\pwindex{Schwarzkopf, Gustav 7.\,11.\,1853 Wien – 13.\,11.\,1939 ebd.@\textsc{Schwarzkopf, Gustav} (7.\,11.\,1853 Wien – 13.\,11.\,1939 ebd.), \emph{Schriftsteller}|pw}, der wohl
                  \textcolor{gray}{und} bei Ihnen iſt?\pend
           \pstart Ihr \spacefill\mbox{A.}\pend{}
\pstart
           25/7 914\pend
           \selectlanguage{ngerman}\endnumbering\briefempfaengerindex{Beer-Hofmann, Richard@\textsc{Beer-Hofmann, Richard}!zzzSchnitzler, Arthur@\emph{von Arthur Schnitzler}!1914-07-251@{25. 7. 1914}|)be}\mylabel{L02188h}  \newcommand{\dateiname}{L02188}\newcommand{\titel}{Arthur Schnitzler an Richard Beer-Hofmann, 25. 7. 1914}\newcommand{\editorInnen}{Martin Anton Müller und Gerd-Hermann Susen}%% latex-leseansicht-abspann.tex
%% Abspann für die Leseansicht.
%% Der Schalter \ifkorrekturansicht ist bereits durch den Vorspann gesetzt.

%% latex-abspann.tex
%% Gemeinsamer Abspann für Korrekturansicht und Leseansicht.
%% Setzt den Schalter \ifkorrekturansicht voraus (gesetzt in den
%% einbindenden Dateien latex-korrekturansicht-abspann.tex bzw.
%% latex-leseansicht-abspann.tex).
%% ---------------------------------------------------------------

\normalsize

% Das esempio-Environment wird nur in der Leseansicht benötigt
\ifkorrekturansicht\else
\newenvironment{esempio}[3]%
{
    \vspace{1.5ex}
    \rlap{\underline{#1}}
    \par
    \setlength{\parindent}{0cm}
    \nopagebreak
    \leftskip=#2cm
    \rightskip=#3cm
}
{
    \par
}
\fi

\doendnotes{C}
\bigskip
\vfill

\clearpage

\footnotesize

\ifkorrekturansicht
  \lohead{\textsc{register}}
\fi

% theindex-Environment neu definieren ohne reledmac
\makeatletter
\renewenvironment{theindex}{%
  \ifkorrekturansicht
    \section*{\indexname}%
  \else
    \subsubsection*{Index der erwähnten Entitäten}%
  \fi
  \setlength{\parindent}{0pt}%
  \setlength{\parskip}{0pt plus 0.3pt}%
  \let\item\@idxitem
}{%
  \ifkorrekturansicht\clearpage\fi
}
\makeatother

\IfFileExists{\jobname-pw.ind}{\input{\jobname-pw.ind}}{}

% Quellenangabe nur in der Leseansicht
\ifkorrekturansicht\else
% Fallback-Definitionen, falls die .tex-Datei \titel etc. nicht gesetzt hat
\providecommand{\titel}{}
\providecommand{\editorInnen}{}
\providecommand{\dateiname}{\jobname}

\vspace{3cm}

\vfill

\footnotesize
\textsc{Quelle}: \titel. Herausgegeben von {\editorInnen}. In: \emph{Arthur Schnitzler: Briefwechsel mit Autorinnen und Autoren}.
 Digitale Edition, https://schnitzler-briefe.acdh.oeaw.ac.at/{\dateiname}.html (Stand \today)
\fi

\end{document}


