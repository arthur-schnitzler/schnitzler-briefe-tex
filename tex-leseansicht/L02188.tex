%% latex-korrekturansicht-vorspann.tex
%% Vorspann für die Korrekturansicht.
%% Lädt die gemeinsame Datei latex-vorspann.tex mit gesetztem Schalter.

\newif\ifkorrekturansicht
\korrekturansichttrue

\input{../tex-inputs/latex-vorspann}


\section[Arthur Schnitzler an Richard Beer-Hofmann, 25. 7. 1914]{L02188 Arthur Schnitzler an Richard Beer-Hofmann, 25. 7. 1914}
\nopagebreak\mylabel{L02188v}
\rehead{ }\normalsize\beginnumbering\briefempfaengerindex{Beer-Hofmann, Richard@\textsc{Beer-Hofmann, Richard}!zzzSchnitzler, Arthur@\emph{von Arthur Schnitzler}!1914-07-251@{25. 7. 1914}|(be}
\toendnotes[C]{\smallbreak\pagebreak[2]}\Standort{YCGL, MSS 31.}
\physDesc{Bildpostkarte, 262 Zeichen
\newline{}Handschrift: Bleistift, deutsche Kurrent
\newline{}Versand: Stempel: »\nobreak{}\oindex{Celerina@\textbf{Celerina}, \emph{P.PPL}|pwk}Celerin\textcolor{gray}{a}
                                          (\textcolor{gray}{Graubünden}), 25. VII. 14, 5\nobreak{}«.  
\newline{}Beer-Hofmann: mit blauem Buntstift den Erhalt
                                 markiert: »E« }\toendnotes[C]{\smallbreak}\pstart{}{\pb}Hrn \textsc{Dr Richard Beerhofma{\geminationn}}\pend{}\pstart{}\textsc{Weißenbach\oindex{Weissenbach am Attersee@\textbf{Weißenbach am Attersee}, \emph{A.ADM3}|pw}.}\pend{}\pstart{}\textsc{Am Attersee\oindex{Attersee@\textbf{Attersee}, \emph{H.LK}|pw}}\pend{}\pstart{}\textsc{Oberoesterreic{[}h{]}}\oindex{Oberoesterreich@\textbf{Oberösterreich}, \emph{A.ADM1}|pw}\pend{}\pstart{}\textsc{Austria}\oindex{Oesterreich@\textbf{Österreich}, \emph{A.PCLI}|pw}\pend{}{\bigskip}
\pstart
           \noindent{}\centering{}{\pb}\textcolor{gray}{\textbf{Celerina\oindex{Celerina@\textbf{Celerina}, \emph{P.PPL}|pw} gegen Pontresina\oindex{Pontresina@\textbf{Pontresina}, \emph{P.PPL}|pw} ges.}}\pend
           \vspace{1em}
\pstart
           \noindent{}{\pb}Aus Pontresina\oindex{Pontresina@\textbf{Pontresina}, \emph{P.PPL}|pw}
               (zu lärmen\textcolor{gray}{d} – \textsc{Hotel} ſowohl als Ort)
               hieher \label{K_L02188-1v}\edtext{überſiedelt}{\lemma{\textnormal{\emph{überſiedelt}}}\Cendnote{\textnormal{Siehe A. S.: \emph{Tagebuch}, 21. 7. 1914.
               }}}\label{K_L02188-1} (\textsc{Cresta Palace, Celerina\oindex{Cresta Palace@\textbf{Cresta Palace}, \emph{Hotel (K.HTL)}|pw}}) und höchſt befriedigt, grüßen wir Sie alle {\pb}herzlichſt. Ebenſo Gustav\pwindex{Schwarzkopf, Gustav 07.11.1853 – 13.11.1939@\textsc{Schwarzkopf, Gustav} (07.11.1853 – 13.11.1939), \emph{Schriftsteller/Schriftstellerin}|pw}, der wohl
                  \textcolor{gray}{und} bei Ihnen iſt?\pend
           \pstart Ihr \spacefill\mbox{A.}\pend{}
\pstart
           25/7 914\pend
           \selectlanguage{ngerman}\endnumbering\briefempfaengerindex{Beer-Hofmann, Richard@\textsc{Beer-Hofmann, Richard}!zzzSchnitzler, Arthur@\emph{von Arthur Schnitzler}!1914-07-251@{25. 7. 1914}|)be}\mylabel{L02188h}  \normalsize

\doendnotes{C}
\bigskip
\vfill

\clearpage

\footnotesize

\lohead{\textsc{register}}

% Definiere theindex-Environment komplett neu ohne reledmac
\makeatletter
\renewenvironment{theindex}{%
  \section*{\indexname}%
  \setlength{\parindent}{0pt}%
  \setlength{\parskip}{0pt plus 0.3pt}%
  \let\item\@idxitem
}{%
  \clearpage
}
\makeatother

\IfFileExists{\jobname-pw.ind}{\input{\jobname-pw.ind}}{}

\end{document}

      