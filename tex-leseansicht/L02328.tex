%% latex-korrekturansicht-vorspann.tex
%% Vorspann für die Korrekturansicht.
%% Lädt die gemeinsame Datei latex-vorspann.tex mit gesetztem Schalter.

\newif\ifkorrekturansicht
\korrekturansichttrue

\input{../tex-inputs/latex-vorspann}


\section[Robert Adam an Arthur Schnitzler, 8. 10. 1919]{L02328 Robert Adam an Arthur Schnitzler, 8. 10. 1919}
\nopagebreak\mylabel{L02328v}
\rehead{ }\normalsize\beginnumbering\briefempfaengerindex{Schnitzler, Arthur@\textsc{Schnitzler, Arthur}!zzzAdam, Robert@\emph{von Robert Adam}!1919-10-081@{8. 10. 1919}|(be}
\toendnotes[C]{\smallbreak\pagebreak[2]}\Standort{Wien, Österreichische Nationalbibliothek, Cod. ser. 52.268, 23v.}
\physDesc{Karte, maschinenschriftliche Abschrift1 Blatt, 1 Seite, 388 Zeichen
\newline{}Schreibmaschine}\Standort{Wien, Österreichische Nationalbibliothek, Cod. ser. 52.268, 36.}
\physDesc{handschriftliche Abschrift1 Blatt, 1 Seite, 388 Zeichen
\newline{}Handschrift: schwarze Tinte, Gabelsberger Kurzschrift}
\pstart
           \raggedleft{}{\pb}8. Oktober 1919\pend
           
\pstart{}Hochverehrter Herr Doktor!\pend\vspace{0.5em}
\pstart
           Ich nehme an, dass Sie bereits nach Wien\oindex{Wien@\textbf{Wien}, \emph{A.ADM2}|pw}
               zurückgekehrt sind, und erlaube mir die Anfrage, ob ich Sie in der nächsten Zeit
               einmal besuchen könnte?\pend
           
\pstart
           Ich arbeite sehr fleissig an der Märchenkomödie\pwindex{Maerchenkomoedie@\emph{Märchenkomödie}|pw},
               selbst vewundert, dass ich noch nicht abgeschreckt bin. Vielleicht wird aus ihr doch
               noch etwas, was mich zufrieden stellt. Mit den besten Grüssen Ihr
                  \spacefill\mbox{Dr RAP}\pend
           \selectlanguage{ngerman}\endnumbering\briefempfaengerindex{Schnitzler, Arthur@\textsc{Schnitzler, Arthur}!zzzAdam, Robert@\emph{von Robert Adam}!1919-10-081@{8. 10. 1919}|)be}\mylabel{L02328h}  \normalsize

\doendnotes{C}
\bigskip
\vfill

\clearpage

\footnotesize

\lohead{\textsc{register}}

% Definiere theindex-Environment komplett neu ohne reledmac
\makeatletter
\renewenvironment{theindex}{%
  \section*{\indexname}%
  \setlength{\parindent}{0pt}%
  \setlength{\parskip}{0pt plus 0.3pt}%
  \let\item\@idxitem
}{%
  \clearpage
}
\makeatother

\IfFileExists{\jobname-pw.ind}{\input{\jobname-pw.ind}}{}

\end{document}

      