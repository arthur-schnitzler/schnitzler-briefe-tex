%% latex-leseansicht-vorspann.tex
%% Vorspann für die Leseansicht.
%% Lädt die gemeinsame Datei latex-vorspann.tex mit nicht gesetztem Schalter.

\newif\ifkorrekturansicht
\korrekturansichtfalse

\input{../tex-inputs/latex-vorspann}


\section[Karl Kraus an Arthur Schnitzler, 11. 1. 1893]{L00156 Karl Kraus an Arthur Schnitzler, 11. 1. 1893}
\nopagebreak\mylabel{L00156v}
\rehead{ }\normalsize\beginnumbering\briefempfaengerindex{Schnitzler, Arthur@\textsc{Schnitzler, Arthur}!zzzKraus, Karl@\emph{von Karl Kraus}!1893-01-111@{11. 1. 1893}|(be}
\toendnotes[C]{\smallbreak\pagebreak[2]}
\correspDesc{Versand  durch Karl Kraus am 11. 1. 1893 in Wien
\newline{}Erhalt  durch Arthur Schnitzler im Zeitraum [11. 1. 1893
                  – 15. 1. 1893?] in Wien}\toendnotes[C]{\smallbreak}
\Standort{DLA, A:Schnitzler, 69.61.}
\physDesc{Brief, 1 Blatt, 2 Seiten, 966 Zeichen
\newline{}Handschrift: schwarze Tinte, deutsche Kurrent}
\buchAbdrucke{\weitereDrucke{\emph{Karl Kraus und Arthur Schnitzler. Eine Dokumentation.}Herausgegeben von Reinhard Urbach In: \emph{Literatur und Kritik}, Bd. 49, Oktober 1970, S. 514.} }\toendnotes[C]{\smallbreak}
\pstart
           {\pb}\textcolor{gray}{\textbf{Karl Kraus}}\hfill \textcolor{gray}{\textbf{Wien\oindex{Wien@\textbf{Wien}, \emph{Verwaltungsgebiet}|pw},}}{ }11/I \textcolor{gray}{\textbf{189}}3\pend
           
\pstart
           \raggedleft{}\textcolor{gray}{\textbf{I., Maximilianstr. 13\oindex{Wien@\textbf{Wien}!I., Innere Stadt@\textbf{I., Innere Stadt}!Mahlerstraße@\textbf{Mahlerstraße}, \emph{Straße}|pw}.}}\pend
           
\pstart{}Mein guter Herr Docter!\pend\vspace{0.5em}
\pstart
           Anbei mit beſtem Danke für Ihre frdl. Bemühungen 1 Sitz neben Ihren Freunden\pwindex{Beer-Hofmann, Richard 11.\,7.\,1866 Wien – 26.\,9.\,1945 New York City@\textsc{Beer-Hofmann, Richard} (11.\,7.\,1866 Wien – 26.\,9.\,1945 New York City), \emph{Schriftsteller}|pwv}\pwindex{Hofmannsthal, Hugo von 1.\,2.\,1874 Wien – 15.\,7.\,1929 Rodaun@\textsc{Hofmannsthal, Hugo von} (1.\,2.\,1874 Wien – 15.\,7.\,1929 Rodaun), \emph{Schriftsteller}|pwv}; nur Herr Schick\pwindex{Schik, Friedrich *~6.\,9.\,1857 Wien@\textsc{Schik, Friedrich} (*~6.\,9.\,1857 Wien), \emph{Notar, Journalist, Dramaturg}|pw}{ }ſitzt ein paar Sitze vor Ihnen. Ich hatte nichts
               anderes, Doctor! Alſo Salten\pwindex{Salten, Felix 6.\,9.\,1869 Budapest – 8.\,10.\,1945 Zürich@\textsc{Salten, Felix} (6.\,9.\,1869 Budapest – 8.\,10.\,1945 Zürich), \emph{Schriftsteller, Journalist, Chefredakteur}|pw} kommt auch? Na,
               das iſt ja{ }ſehr schön! Das wird eine Hetz’ werden!! Bitte, lachen Sie mir nur nicht
               zu viel und machen Sie in der erſten Reihe ein recht freundliches Geſicht!\pend
           
\pstart
           Erſuche höflichſt, da ich 24 Stunden vor d. Vorstellung\pwindex{\textcolor{red}{\textsuperscript{XXXX indx1}}!Räuber. Ein Schauspiel@\strich\emph{Die Räuber. Ein Schauspiel}|pwv} dem Director\pwindex{Barth, Moriz von 7.\,1.\,1861 Wien – 2.\,11.\,1939 Perchtoldsdorf@\textsc{Barth, Moriz von} (7.\,1.\,1861 Wien – 2.\,11.\,1939 Perchtoldsdorf), \emph{Regisseur, Schauspieler, Beamter}|pwv} abliefern muſs, bis Freitag mittag
               den Betrag 1 fl. 20 zu{ }ſchicken. {\pb}Ein kleines Deficit dürfte
               ich haben; \uline{alle} Karten bring’ ich \uline{nicht} an!\pend
           
\pstart
           Ich bin{ }ſehr gerne bereit, eine kleine \label{K_L00156-1v}\edtext{Notiz}{\lemma{\textnormal{\emph{Notiz}}}\Cendnote{\textnormal{Diese schrieb nicht Kraus\pwindex{Kraus, Karl 28.\,4.\,1874 Jičín – 12.\,6.\,1936 Wien@\textsc{Kraus, Karl} (28.\,4.\,1874 Jičín – 12.\,6.\,1936 Wien), \emph{Schriftsteller, Publizist, Schriftsteller}|pwk}, sondern Josef Schmid-Braunfels\pwindex{Schmid-Braunfels, Josef 29.\,11.\,1871 Ryžoviště – 22.\,11.\,1911 ebd.@\textsc{Schmid-Braunfels, Josef} (29.\,11.\,1871 Ryžoviště – 22.\,11.\,1911 ebd.), \emph{Schriftsteller, Veterinärmediziner}|pwk} (\emph{Arthur Schnitzler: Anatol}\pwindex{Schmid-Braunfels, Josef 29.\,11.\,1871 Ryžoviště – 22.\,11.\,1911 ebd.@\textsc{Schmid-Braunfels, Josef} (29.\,11.\,1871 Ryžoviště – 22.\,11.\,1911 ebd.), \emph{Schriftsteller, Veterinärmediziner}!Arthur Schnitzler: Anatol@\strich\emph{Arthur Schnitzler: Anatol}|pwk}. In: \emph{Neue litterarische Blätter}\pwindex{Neue litterarische Blätter@\emph{Neue litterarische Blätter}|pwk}, Jg. 1, Nr. 7,
                        1. 4. 1893, S. 87–88).}}}\label{K_L00156-1} über Ihren »Anatol\pwindex{Schnitzler, Arthur 15.\,5.\,1862 Wien – 21.\,10.\,1931 ebd.@\textsc{Schnitzler, Arthur} (15.\,5.\,1862 Wien – 21.\,10.\,1931 ebd.), \emph{Schriftsteller, Mediziner}!Anatol@\strich\emph{Anatol}|pw}« in den »\uline{Neuen litterariſchen Blättern}\pwindex{Neue litterarische Blätter@\emph{Neue litterarische Blätter}|pw}« (Bremen\oindex{Bremen@\textbf{Bremen}|pw}, Herausgeber Franziskus Haehnel\pwindex{Haehnel, Franziskus 15.\,5.\,1864 Hamburg – 17.\,5.\,1929 ebd.@\textsc{Haehnel, Franziskus} (15.\,5.\,1864 Hamburg – 17.\,5.\,1929 ebd.), \emph{Schriftsteller}|pw}, Verlag Kühtmann\orgindex{Kühtmann@Kühtmann|pw}) zu bringen. Nur müſsten Sie einen Recensionsexemplarabgang\pwindex{Eingesandte Neuerscheinungen [Arthur Schnitzler: Anatol]@\emph{Eingesandte Neuerscheinungen [Arthur Schnitzler: Anatol]}|pwv} an dieſe Monatsblätter von \strikeout{d} Ihrem Verleger\pwindex{Fischer, Samuel 24.\,12.\,1859 Liptovský Mikuláš – 15.\,10.\,1934 Berlin@\textsc{Fischer, Samuel} (24.\,12.\,1859 Liptovský Mikuláš – 15.\,10.\,1934 Berlin), \emph{Verleger}|pwv} erwirken.\pend
           
\pstart
           Alexander Engel\pwindex{Engel, Alexander 10.\,4.\,1868 Necpaly – 17.\,11.\,1940 Wien@\textsc{Engel, Alexander} (10.\,4.\,1868 Necpaly – 17.\,11.\,1940 Wien), \emph{Schriftsteller, Journalist}|pw} dürfte in den Breslauer Monatsblättern\orgindex{Monatsblätter@Monatsblätter|pw} (Paul Barsch\pwindex{Barsch, Paul 16.\,3.\,1860 Jasienica Dolna – 3.\,8.\,1931 Tyniec nad Ślężą@\textsc{Barsch, Paul} (16.\,3.\,1860 Jasienica Dolna – 3.\,8.\,1931 Tyniec nad Ślężą), \emph{Schriftsteller, Redakteur}|pw}) bringen.\pend
           
\pstart
           Und nun herzlichen Gruß{\\[\baselineskip]} von Ihrem{ }ſehr ergebenen \spacefill\mbox{Karl
                  Kraus}\pend
           \leftskip=0em{}
\pstart
           \noindent{}Wien\oindex{Wien@\textbf{Wien}, \emph{Verwaltungsgebiet}|pw}\pend
           \selectlanguage{ngerman}\endnumbering\briefempfaengerindex{Schnitzler, Arthur@\textsc{Schnitzler, Arthur}!zzzKraus, Karl@\emph{von Karl Kraus}!1893-01-111@{11. 1. 1893}|)be}\mylabel{L00156h}  \newcommand{\dateiname}{L00156}\newcommand{\titel}{Karl Kraus an Arthur Schnitzler, 11. 1. 1893}\newcommand{\editorInnen}{Martin Anton Müller und Gerd-Hermann Susen}%% latex-leseansicht-abspann.tex
%% Abspann für die Leseansicht.
%% Der Schalter \ifkorrekturansicht ist bereits durch den Vorspann gesetzt.

%% latex-abspann.tex
%% Gemeinsamer Abspann für Korrekturansicht und Leseansicht.
%% Setzt den Schalter \ifkorrekturansicht voraus (gesetzt in den
%% einbindenden Dateien latex-korrekturansicht-abspann.tex bzw.
%% latex-leseansicht-abspann.tex).
%% ---------------------------------------------------------------

\normalsize

% Das esempio-Environment wird nur in der Leseansicht benötigt
\ifkorrekturansicht\else
\newenvironment{esempio}[3]%
{
    \vspace{1.5ex}
    \rlap{\underline{#1}}
    \par
    \setlength{\parindent}{0cm}
    \nopagebreak
    \leftskip=#2cm
    \rightskip=#3cm
}
{
    \par
}
\fi

\doendnotes{C}
\bigskip
\vfill

\clearpage

\footnotesize

\ifkorrekturansicht
  \lohead{\textsc{register}}
\fi

% theindex-Environment neu definieren ohne reledmac
\makeatletter
\renewenvironment{theindex}{%
  \ifkorrekturansicht
    \section*{\indexname}%
  \else
    \subsubsection*{Index der erwähnten Entitäten}%
  \fi
  \setlength{\parindent}{0pt}%
  \setlength{\parskip}{0pt plus 0.3pt}%
  \let\item\@idxitem
}{%
  \ifkorrekturansicht\clearpage\fi
}
\makeatother

\IfFileExists{\jobname-pw.ind}{\input{\jobname-pw.ind}}{}

% Quellenangabe nur in der Leseansicht
\ifkorrekturansicht\else
% Fallback-Definitionen, falls die .tex-Datei \titel etc. nicht gesetzt hat
\providecommand{\titel}{}
\providecommand{\editorInnen}{}
\providecommand{\dateiname}{\jobname}

\vspace{3cm}

\vfill

\footnotesize
\textsc{Quelle}: \titel. Herausgegeben von {\editorInnen}. In: \emph{Arthur Schnitzler: Briefwechsel mit Autorinnen und Autoren}.
 Digitale Edition, https://schnitzler-briefe.acdh.oeaw.ac.at/{\dateiname}.html (Stand \today)
\fi

\end{document}


