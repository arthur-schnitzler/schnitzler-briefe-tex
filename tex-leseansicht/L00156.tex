%% latex-korrekturansicht-vorspann.tex
%% Vorspann für die Korrekturansicht.
%% Lädt die gemeinsame Datei latex-vorspann.tex mit gesetztem Schalter.

\newif\ifkorrekturansicht
\korrekturansichttrue

\input{../tex-inputs/latex-vorspann}


\section[Karl Kraus an Arthur Schnitzler, 11. 1. 1893]{L00156 Karl Kraus an Arthur Schnitzler, 11. 1. 1893}
\nopagebreak\mylabel{L00156v}
\rehead{ }\normalsize\beginnumbering\briefempfaengerindex{Schnitzler, Arthur@\textsc{Schnitzler, Arthur}!zzzKraus, Karl@\emph{von Karl Kraus}!1893-01-111@{11. 1. 1893}|(be}
\toendnotes[C]{\smallbreak\pagebreak[2]}\Standort{DLA, A:Schnitzler, 69.61.}
\physDesc{Brief, 1 Blatt, 2 Seiten, 966 Zeichen
\newline{}Handschrift: schwarze Tinte, deutsche Kurrent}
\buchAbdrucke{\weitereDrucke{\emph{Literatur und Kritik}, Bd. 49, Oktober 1970, S. 514.} }\toendnotes[C]{\smallbreak}
\pstart
           {\pb}\textcolor{gray}{\textbf{Karl Kraus}}\hfill \textcolor{gray}{\textbf{Wien\oindex{Wien@\textbf{Wien}, \emph{A.ADM2}|pw},}}{ }11/I \textcolor{gray}{\textbf{189}}3\pend
           
\pstart
           \raggedleft{}\textcolor{gray}{\textbf{I., Maximilianstr. 13\oindex{Mahlerstrasse@\textbf{Mahlerstraße}, \emph{Straße (K.STR)}|pw}.}}\pend
           
\pstart{}Mein guter Herr Docter!\pend\vspace{0.5em}
\pstart
           Anbei mit beſtem Danke für Ihre frdl. Bemühungen 1 Sitz neben Ihren Freunden\pwindex{Beer-Hofmann, Richard 1866-07-11 – 1945-09-26@\textsc{Beer-Hofmann, Richard} (1866-07-11 – 1945-09-26), \emph{Schriftsteller/Schriftstellerin}|pwv}\pwindex{Hofmannsthal, Hugo von 1874-02-01 – 1929-07-15@\textsc{Hofmannsthal, Hugo von} (1874-02-01 – 1929-07-15), \emph{Schriftsteller/Schriftstellerin}|pwv}; nur Herr Schick\pwindex{Schik, Friedrich *~06.09.1857@\textsc{Schik, Friedrich} (*~06.09.1857), \emph{Journalist/Journalistin, Dramaturg/Dramaturgin}|pw}{ }ſitzt ein paar Sitze vor Ihnen. Ich hatte nichts
               anderes, Doctor! Alſo Salten\pwindex{Salten, Felix 06.09.1869 – 08.10.1945@\textsc{Salten, Felix} (06.09.1869 – 08.10.1945), \emph{Schriftsteller/Schriftstellerin, Journalist/Journalistin, Chefredakteur/Chefredakteurin}|pw} kommt auch? Na,
               das iſt ja ſehr schön! Das wird eine Hetz’ werden!! Bitte, lachen Sie mir nur nicht
               zu viel und machen Sie in der erſten Reihe ein recht freundliches Geſicht!\pend
           
\pstart
           Erſuche höflichſt, da ich 24 Stunden vor d. Vorstellung\pwindex{Raeuber. Ein Schauspiel@\emph{Die Räuber. Ein Schauspiel}|pwv} dem Director\pwindex{Barth, Moriz von 07.01.1861 – 02.11.1939@\textsc{Barth, Moriz von} (07.01.1861 – 02.11.1939), \emph{Regisseur/Regisseurin, Schauspieler/Schauspielerin, Beamter/Beamte}|pwv} abliefern muſs, bis Freitag mittag
               den Betrag 1 fl. 20 zu ſchicken. {\pb}Ein kleines Deficit dürfte
               ich haben; \uline{alle} Karten bring’ ich \uline{nicht} an!\pend
           
\pstart
           Ich bin ſehr gerne bereit, eine kleine \label{K_L00156-1v}\edtext{Notiz}{\lemma{\textnormal{\emph{Notiz}}}\Cendnote{\textnormal{Diese schrieb nicht Kraus\pwindex{Kraus, Karl 28.04.1874 – 12.06.1936@\textsc{Kraus, Karl} (28.04.1874 – 12.06.1936), \emph{Schriftsteller/Schriftstellerin, Publizist/Publizistin, Schriftsteller/Schriftstellerin}|pwk}, sondern Josef Schmid-Braunfels\pwindex{Schmid-Braunfels, Josef 29.11.1871 – 22.11.1911@\textsc{Schmid-Braunfels, Josef} (29.11.1871 – 22.11.1911), \emph{Schriftsteller/Schriftstellerin, Veterinärmediziner/Veterinärmedizinerin}|pwk} (\emph{Arthur Schnitzler: Anatol}\pwindex{Arthur Schnitzler: Anatol@\emph{Arthur Schnitzler: Anatol}|pwk}. In: \emph{Neue litterarische Blätter}\pwindex{Neue litterarische Blaetter@\emph{Neue litterarische Blätter}|pwk}, Jg. 1, Nr. 7,
                        1. 4. 1893, S. 87–88).}}}\label{K_L00156-1} über Ihren »Anatol\pwindex{Anatol@\emph{Anatol}|pw}« in den »\uline{Neuen litterariſchen Blättern}\pwindex{Neue litterarische Blaetter@\emph{Neue litterarische Blätter}|pw}« (Bremen\oindex{Bremen@\textbf{Bremen}, \emph{P.PPLA}|pw}, Herausgeber Franziskus Haehnel\pwindex{Haehnel, Franziskus 15.05.1864 – 17.05.1929@\textsc{Haehnel, Franziskus} (15.05.1864 – 17.05.1929), \emph{Schriftsteller/Schriftstellerin}|pw}, Verlag Kühtmann\orgindex{Kuehtmann@Kühtmann|pw}) zu bringen. Nur müſsten Sie einen Recensionsexemplarabgang\pwindex{Eingesandte Neuerscheinungen [Arthur Schnitzler: Anatol]@\emph{Eingesandte Neuerscheinungen [Arthur Schnitzler: Anatol]}|pwv} an dieſe Monatsblätter von \strikeout{d} Ihrem Verleger\pwindex{Fischer, Samuel 24.12.1859 – 15.10.1934@\textsc{Fischer, Samuel} (24.12.1859 – 15.10.1934), \emph{Verleger/Verlegerin}|pwv} erwirken.\pend
           
\pstart
           Alexander Engel\pwindex{Engel, Alexander 10.04.1868 – 17.11.1940@\textsc{Engel, Alexander} (10.04.1868 – 17.11.1940), \emph{Schriftsteller/Schriftstellerin, Journalist/Journalistin}|pw} dürfte in den Breslauer Monatsblättern\orgindex{Monatsblaetter@Monatsblätter|pw} (Paul Barsch\pwindex{Barsch, Paul 16.03.1860 – 03.08.1931@\textsc{Barsch, Paul} (16.03.1860 – 03.08.1931), \emph{Schriftsteller/Schriftstellerin, Redakteur/Redakteurin}|pw}) bringen.\pend
           
\pstart
           Und nun herzlichen Gruß{\\[\baselineskip]} von Ihrem ſehr ergebenen \spacefill\mbox{Karl
                  Kraus}\pend
           \leftskip=0em{}
\pstart
           \noindent{}Wien\oindex{Wien@\textbf{Wien}, \emph{A.ADM2}|pw}\pend
           \selectlanguage{ngerman}\endnumbering\briefempfaengerindex{Schnitzler, Arthur@\textsc{Schnitzler, Arthur}!zzzKraus, Karl@\emph{von Karl Kraus}!1893-01-111@{11. 1. 1893}|)be}\mylabel{L00156h}  \normalsize

\doendnotes{C}
\bigskip
\vfill

\clearpage

\footnotesize

\lohead{\textsc{register}}

% Definiere theindex-Environment komplett neu ohne reledmac
\makeatletter
\renewenvironment{theindex}{%
  \section*{\indexname}%
  \setlength{\parindent}{0pt}%
  \setlength{\parskip}{0pt plus 0.3pt}%
  \let\item\@idxitem
}{%
  \clearpage
}
\makeatother

\IfFileExists{\jobname-pw.ind}{\input{\jobname-pw.ind}}{}

\end{document}

      