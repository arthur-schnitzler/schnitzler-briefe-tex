%% latex-leseansicht-vorspann.tex
%% Vorspann für die Leseansicht.
%% Lädt die gemeinsame Datei latex-vorspann.tex mit nicht gesetztem Schalter.

\newif\ifkorrekturansicht
\korrekturansichtfalse

\input{../tex-inputs/latex-vorspann}


         
         \renewcommand{\erwaehntePersonen}{Personen: Paul Barsch, Moriz von Barth, Richard Beer-Hofmann, Alexander Engel, Samuel Fischer, Franziskus Haehnel, Hugo von Hofmannsthal, Karl Kraus, Felix Salten, Friedrich Schik, Josef Schmid-Braunfels}
         \renewcommand{\erwaehnteInstitutionen}{Institutionen: Kühtmann, Monatsblätter}
         \renewcommand{\erwaehnteOrte}{Orte: Bremen, Mahlerstraße, Wien}
         \renewcommand{\erwaehnteWerke}{Werke: Anatol, Arthur Schnitzler: Anatol, Die Räuber. Ein Schauspiel, Eingesandte Neuerscheinungen [Arthur Schnitzler: Anatol], Neue litterarische Blätter}
               \section[Karl Kraus an Arthur Schnitzler, 11. 1. 1893]{ Karl Kraus an Arthur Schnitzler, 11. 1. 1893}\nopagebreak\mylabel{v}\rehead{ }\begin{ledgroupsized}[t]{13cm}\normalsize\beginnumbering\briefempfaengerindex{Schnitzler, Arthur@\textsc{Schnitzler, Arthur}!zzzKraus, Karl@\emph{von Karl Kraus}!1893-01-111@{11. 1. 1893}|(be} \toendnotes[C]{\smallbreak\pagebreak[2]} \Standort{DLA, A:Schnitzler, 69.61.}
\physDesc{Brief, 1 Blatt, 2 Seiten, 966 Zeichen
\newline{}Handschrift: schwarze Tinte, deutsche Kurrent}\buchAbdrucke{\weitereDrucke{\emph{Karl Kraus und Arthur Schnitzler. Eine Dokumentation.} Hg. Reinhard Urbach. In: \emph{Literatur und Kritik}, Bd. 49, Oktober 1970, S. 514.} }\toendnotes[C]{\smallbreak}\pstart
           \noindent{}{\pb}\textcolor{gray}{\textbf{Karl Kraus}}\hfill \textcolor{gray}{\textbf{Wien\oindex{Wien@\textbf{Wien}|pw},}}{ }11/I \textcolor{gray}{\textbf{189}}3\pend
           \pstart
           \raggedleft{}\textcolor{gray}{\textbf{I., Maximilianstr. 13\oindex{Mahlerstrasse@\textbf{Mahlerstraße}|pw}.}}\pend
           \pstart{}Mein guter Herr Docter!\pend\pstart
           Anbei mit beſtem Danke für Ihre frdl. Bemühungen 1 Sitz neben Ihren Freunden\pwindex{Beer-Hofmann, Richard 1866-07-11 – 1945-09-26@\textsc{Beer-Hofmann, Richard} (1866-07-11 – 1945-09-26), \emph{Schriftsteller}|pwv}\pwindex{Hofmannsthal, Hugo von 1874-02-01 – 1929-07-15@\textsc{Hofmannsthal, Hugo von} (1874-02-01 – 1929-07-15), \emph{Schriftsteller}|pwv}; nur Herr Schick\pwindex{Schik, Friedrich *~06.09.1857@\textsc{Schik, Friedrich} (*~06.09.1857), \emph{Journalist, Dramaturg}|pw}{ }ſitzt ein paar Sitze vor Ihnen. Ich hatte nichts
               anderes, Doctor! Alſo Salten\pwindex{Salten, Felix 06.09.1869 – 08.10.1945@\textsc{Salten, Felix} (06.09.1869 – 08.10.1945), \emph{Schriftsteller, Journalist, Chefredakteur}|pw} kommt auch? Na,
               das iſt ja ſehr schön! Das wird eine Hetz’ werden!! Bitte, lachen Sie mir nur nicht
               zu viel und machen Sie in der erſten Reihe ein recht freundliches Geſicht!\pend
           \pstart
           Erſuche höflichſt, da ich 24 Stunden vor d. Vorstellung\pwindex{\textcolor{red}{\textsuperscript{XXXX1 indx}}!Raeuber. Ein Schauspiel1781@\strich\emph{Die Räuber. Ein Schauspiel} {[}1781{]}|pwv} dem Director\pwindex{Barth, Moriz von 07.01.1861 – 02.11.1939@\textsc{Barth, Moriz von} (07.01.1861 – 02.11.1939), \emph{Regisseur, Schauspieler, Beamter}|pwv} abliefern muſs, bis Freitag mittag
               den Betrag 1 fl. 20 zu ſchicken. {\pb}Ein kleines Deficit dürfte
               ich haben; \uline{alle} Karten bring’ ich \uline{nicht} an!\pend
           \pstart
           Ich bin ſehr gerne bereit, eine kleine \label{K_L00156-1v}\edtext{Notiz}{\lemma{\textnormal{\emph{Notiz}}}\Cendnote{\textnormal{Diese schrieb nicht Kraus\pwindex{Kraus, Karl 28.04.1874 – 12.06.1936@\textsc{Kraus, Karl} (28.04.1874 – 12.06.1936), \emph{Schriftsteller, Publizist}|pwk}, sondern Josef Schmid-Braunfels\pwindex{Schmid-Braunfels, Josef 29.11.1871 – 22.11.1911@\textsc{Schmid-Braunfels, Josef} (29.11.1871 – 22.11.1911), \emph{Schriftsteller, Veterinärmediziner}|pwk} (\emph{Arthur Schnitzler: Anatol}\pwindex{Schmid-Braunfels, Josef 29.11.1871 – 22.11.1911@\textsc{Schmid-Braunfels, Josef} (29.11.1871 – 22.11.1911), \emph{Schriftsteller, Veterinärmediziner}!Arthur Schnitzler: Anatol01. 04. 1893@\strich\emph{Arthur Schnitzler: Anatol} {[}01. 04. 1893{]}|pwk}. In: \emph{Neue litterarische Blätter}\pwindex{Neue litterarische Blaetter1893@\emph{Neue litterarische Blätter} {[}1893{]}|pwk}, Jg. 1, Nr. 7,
                        1. 4. 1893, S. 87–88).}}}\label{K_L00156-1h} über Ihren »Anatol\pwindex{Schnitzler, Arthur 15.05.1862 – 21.10.1931@\textsc{Schnitzler, Arthur} (15.05.1862 – 21.10.1931), \emph{Schriftsteller, Mediziner}!Anatol1892-10-29@\strich\emph{Anatol} {[}1892-10-29{]}|pw}« in den »\uline{Neuen litterariſchen Blättern}\pwindex{Neue litterarische Blaetter1893@\emph{Neue litterarische Blätter} {[}1893{]}|pw}« (Bremen\oindex{Bremen@\textbf{Bremen}|pw}, Herausgeber Franziskus Haehnel\pwindex{Haehnel, Franziskus 15.05.1864 – 17.05.1929@\textsc{Haehnel, Franziskus} (15.05.1864 – 17.05.1929), \emph{Schriftsteller}|pw}, Verlag Kühtmann\orgindex{Kuehtmann@Kühtmann|pw}) zu bringen. Nur müſsten Sie einen Recensionsexemplarabgang\pwindex{?? Werk@Nicht ermittelte Verfasserinnen und Verfasser!Eingesandte Neuerscheinungen [Arthur Schnitzler: Anatol]1. 3. 1893@\emph{Eingesandte Neuerscheinungen [Arthur Schnitzler: Anatol]} {[}1. 3. 1893{]}|pwv} an dieſe Monatsblätter von \strikeout{d} Ihrem Verleger\pwindex{Fischer, Samuel 24.12.1859 – 15.10.1934@\textsc{Fischer, Samuel} (24.12.1859 – 15.10.1934), \emph{Verleger}|pwv} erwirken.\pend
           \pstart
           Alexander Engel\pwindex{Engel, Alexander 10.04.1868 – 17.11.1940@\textsc{Engel, Alexander} (10.04.1868 – 17.11.1940), \emph{Schriftsteller, Journalist}|pw} dürfte in den Breslauer Monatsblättern\orgindex{Monatsblaetter@Monatsblätter|pw} (Paul Barsch\pwindex{Barsch, Paul 16.03.1860 – 03.08.1931@\textsc{Barsch, Paul} (16.03.1860 – 03.08.1931), \emph{Schriftsteller, Redakteur}|pw}) bringen.\pend
           \pstart
           Und nun herzlichen Gruß{\\[\baselineskip]} von Ihrem ſehr ergebenen \spacefill\mbox{Karl
                  Kraus}\pend
           \leftskip=0em{}\pstart
           \noindent{}Wien\oindex{Wien@\textbf{Wien}|pw}\pend
           
         
         \endnumbering\mylabel{h}\end{ledgroupsized}  \newcommand{\dateiname}{L00156}\newcommand{\titel}{Karl Kraus an Arthur Schnitzler, 11. 1. 1893}\newcommand{\editorInnen}{Martin Anton Müller und Gerd-Hermann Susen}%% latex-leseansicht-abspann.tex
%% Abspann für die Leseansicht.
%% Der Schalter \ifkorrekturansicht ist bereits durch den Vorspann gesetzt.

%% latex-abspann.tex
%% Gemeinsamer Abspann für Korrekturansicht und Leseansicht.
%% Setzt den Schalter \ifkorrekturansicht voraus (gesetzt in den
%% einbindenden Dateien latex-korrekturansicht-abspann.tex bzw.
%% latex-leseansicht-abspann.tex).
%% ---------------------------------------------------------------

\normalsize

% Das esempio-Environment wird nur in der Leseansicht benötigt
\ifkorrekturansicht\else
\newenvironment{esempio}[3]%
{
    \vspace{1.5ex}
    \rlap{\underline{#1}}
    \par
    \setlength{\parindent}{0cm}
    \nopagebreak
    \leftskip=#2cm
    \rightskip=#3cm
}
{
    \par
}
\fi

\doendnotes{C}
\bigskip
\vfill

\clearpage

\footnotesize

\ifkorrekturansicht
  \lohead{\textsc{register}}
\fi

% theindex-Environment neu definieren ohne reledmac
\makeatletter
\renewenvironment{theindex}{%
  \ifkorrekturansicht
    \section*{\indexname}%
  \else
    \subsubsection*{Index der erwähnten Entitäten}%
  \fi
  \setlength{\parindent}{0pt}%
  \setlength{\parskip}{0pt plus 0.3pt}%
  \let\item\@idxitem
}{%
  \ifkorrekturansicht\clearpage\fi
}
\makeatother

\IfFileExists{\jobname-pw.ind}{\input{\jobname-pw.ind}}{}

% Quellenangabe nur in der Leseansicht
\ifkorrekturansicht\else
% Fallback-Definitionen, falls die .tex-Datei \titel etc. nicht gesetzt hat
\providecommand{\titel}{}
\providecommand{\editorInnen}{}
\providecommand{\dateiname}{\jobname}

\vspace{3cm}

\vfill

\footnotesize
\textsc{Quelle}: \titel. Herausgegeben von {\editorInnen}. In: \emph{Arthur Schnitzler: Briefwechsel mit Autorinnen und Autoren}.
 Digitale Edition, https://schnitzler-briefe.acdh.oeaw.ac.at/{\dateiname}.html (Stand \today)
\fi

\end{document}


      