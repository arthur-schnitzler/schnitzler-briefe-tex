%% latex-leseansicht-vorspann.tex
%% Vorspann für die Leseansicht.
%% Lädt die gemeinsame Datei latex-vorspann.tex mit nicht gesetztem Schalter.

\newif\ifkorrekturansicht
\korrekturansichtfalse

\input{../tex-inputs/latex-vorspann}


\section[Arthur Schnitzler an Felix Braun, 26. 12. 1925]{L02460 Arthur Schnitzler an Felix Braun, 26. 12. 1925}
\nopagebreak\mylabel{L02460v}
\rehead{ }\normalsize\beginnumbering\briefempfaengerindex{Braun, Felix@\textsc{Braun, Felix}!zzzSchnitzler, Arthur@\emph{von Arthur Schnitzler}!1925-12-261@{26. 12. 1925}|(be}
\toendnotes[C]{\smallbreak\pagebreak[2]}
\correspDesc{Versand  durch Arthur Schnitzler am 26. 12. 1925 in Wien
\newline{}Erhalt  durch Felix Braun im Zeitraum [26. 12. 1925 – 30. 12. 1925?] in Wien}\toendnotes[C]{\smallbreak}
\Standort{Wienbibliothek im Rathaus, H.I.N.-198.049.}
\physDesc{Postkarte, 238 Zeichen
\newline{}Handschrift: schwarze Tinte, lateinische Kurrent
\newline{}Versand: Stempel: »\nobreak{}\oindex{Wien@\textbf{Wien}, \emph{Verwaltungsgebiet}|pwk}/1 Wien 88, 27. {[}X{]}II. 24, 11\nobreak{}«.  }\toendnotes[C]{\smallbreak}\pstart{}{\pb}\label{T_L02460-1v}\edtext{\textcolor{gray}{\textbf{A. S.}}}{\lemma{\textnormal{\emph{A. S.}}}\Cendnote{\textnormal{ovaler Absenderkleber}}}\label{T_L02460-1}\pend{}\pstart{}\textcolor{gray}{\textbf{WIEN, XVIII.}}\oindex{XVIII., Währing@\textbf{XVIII., Währing}, \emph{Verwaltungsgebiet}|pw}\pend{}\pstart{}\textcolor{gray}{\textbf{STERNWARTESTR. 71}}\oindex{Wien@\textbf{Wien}!XVIII., Währing@\textbf{XVIII., Währing}!Sternwartestraße 71@\textbf{Sternwartestraße 71}, \emph{Wohngebäude}|pw}\pend{}{\bigskip}\pstart{}Herrn Felix Braun\pend{}\pstart{}Wien XIX\oindex{XIX., Döbling@\textbf{XIX., Döbling}, \emph{Verwaltungsgebiet}|pw}\pend{}\pstart{}Sieveringer Hptstr 149\oindex{Wien@\textbf{Wien}!XIX., Döbling@\textbf{XIX., Döbling}!Sieveringer Straße@\textbf{Sieveringer Straße}, \emph{Straße}|pw}. \pend{}{\bigskip}\vspace{1em}
\pstart
           \raggedleft{}{\pb}26/12 925\pend
           
\pstart{}lieber Herr Felix Braun,\pend\vspace{0.5em}
\pstart
           Ihre schönen Worte über die »Frau des Richters\pwindex{Schnitzler, Arthur 15.\,5.\,1862 Wien – 21.\,10.\,1931 ebd.@\textsc{Schnitzler, Arthur} (15.\,5.\,1862 Wien – 21.\,10.\,1931 ebd.), \emph{Schriftsteller, Mediziner}!Frau des Richters. Novelle@\strich\emph{Die Frau des Richters. Novelle}|pw}«
               haben mich herzlich erfreut. Vielen Dank und die besten Wünsche und Grüße zum neuen
               Jahr.\pend
           
\pstart
           Ihr{\\[\baselineskip]}\spacefill\mbox{Arthur Schnitzler}\pend
           \leftskip=0em{}\selectlanguage{ngerman}\endnumbering\briefempfaengerindex{Braun, Felix@\textsc{Braun, Felix}!zzzSchnitzler, Arthur@\emph{von Arthur Schnitzler}!1925-12-261@{26. 12. 1925}|)be}\mylabel{L02460h}  \newcommand{\dateiname}{L02460}\newcommand{\titel}{Arthur Schnitzler an Felix Braun, 26. 12. 1925}\newcommand{\editorInnen}{Martin Anton Müller und Gerd-Hermann Susen}%% latex-leseansicht-abspann.tex
%% Abspann für die Leseansicht.
%% Der Schalter \ifkorrekturansicht ist bereits durch den Vorspann gesetzt.

%% latex-abspann.tex
%% Gemeinsamer Abspann für Korrekturansicht und Leseansicht.
%% Setzt den Schalter \ifkorrekturansicht voraus (gesetzt in den
%% einbindenden Dateien latex-korrekturansicht-abspann.tex bzw.
%% latex-leseansicht-abspann.tex).
%% ---------------------------------------------------------------

\normalsize

% Das esempio-Environment wird nur in der Leseansicht benötigt
\ifkorrekturansicht\else
\newenvironment{esempio}[3]%
{
    \vspace{1.5ex}
    \rlap{\underline{#1}}
    \par
    \setlength{\parindent}{0cm}
    \nopagebreak
    \leftskip=#2cm
    \rightskip=#3cm
}
{
    \par
}
\fi

\doendnotes{C}
\bigskip
\vfill

\clearpage

\footnotesize

\ifkorrekturansicht
  \lohead{\textsc{register}}
\fi

% theindex-Environment neu definieren ohne reledmac
\makeatletter
\renewenvironment{theindex}{%
  \ifkorrekturansicht
    \section*{\indexname}%
  \else
    \subsubsection*{Index der erwähnten Entitäten}%
  \fi
  \setlength{\parindent}{0pt}%
  \setlength{\parskip}{0pt plus 0.3pt}%
  \let\item\@idxitem
}{%
  \ifkorrekturansicht\clearpage\fi
}
\makeatother

\IfFileExists{\jobname-pw.ind}{\input{\jobname-pw.ind}}{}

% Quellenangabe nur in der Leseansicht
\ifkorrekturansicht\else
% Fallback-Definitionen, falls die .tex-Datei \titel etc. nicht gesetzt hat
\providecommand{\titel}{}
\providecommand{\editorInnen}{}
\providecommand{\dateiname}{\jobname}

\vspace{3cm}

\vfill

\footnotesize
\textsc{Quelle}: \titel. Herausgegeben von {\editorInnen}. In: \emph{Arthur Schnitzler: Briefwechsel mit Autorinnen und Autoren}.
 Digitale Edition, https://schnitzler-briefe.acdh.oeaw.ac.at/{\dateiname}.html (Stand \today)
\fi

\end{document}


