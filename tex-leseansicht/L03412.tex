%% latex-korrekturansicht-vorspann.tex
%% Vorspann für die Korrekturansicht.
%% Lädt die gemeinsame Datei latex-vorspann.tex mit gesetztem Schalter.

\newif\ifkorrekturansicht
\korrekturansichttrue

\input{../tex-inputs/latex-vorspann}


\section[ Felix Salten an Arthur Schnitzler, 18. 7. 1905]{L03412 Felix Salten an Arthur Schnitzler, 18. 7. 1905}
\nopagebreak\mylabel{L03412v}
\rehead{ }\normalsize\beginnumbering\briefempfaengerindex{Schnitzler, Arthur@\textsc{Schnitzler, Arthur}!zzzSalten, Felix@\emph{von Felix Salten}!1905-07-181@{18. 7. 1905}|(be}
\toendnotes[C]{\smallbreak\pagebreak[2]}\Standort{CUL, Schnitzler, B 89, B 1.}
\physDesc{Briefkarte, 553 Zeichen
\newline{}Handschrift: schwarze Tinte, lateinische Kurrent
\newline{}Ordnung: mit Bleistift von unbekannter Hand nummeriert: »204« }
\buchAbdrucke{\weitereDrucke{Hermann Bahr, Arthur Schnitzler: \emph{Briefwechsel, Aufzeichnungen, Dokumente (1891–1931)}. Göttingen: \emph{Wallstein} 2018, S. 346–347.} }\toendnotes[C]{\smallbreak}
\pstart
           {\pb}\textcolor{gray}{\textbf{DIE}}\pend
           
\pstart
           \textcolor{gray}{\textbf{ZEIT\orgindex{Zeit@Die Zeit|pw}}}\hfill \textcolor{gray}{\textbf{\emph{WIEN}\oindex{Wien@\textbf{Wien}, \emph{A.ADM2}|pw}}}{ }18. 7. 05\pend
           
\pstart
           \textcolor{gray}{\textbf{Wien\oindex{Wien@\textbf{Wien}, \emph{A.ADM2}|pw}er Tageszeitung}}\hfill \textcolor{gray}{\textbf{\emph{I. Wipplingerstrasse 38\oindex{Wipplingerstrasse@\textbf{Wipplingerstraße}, \emph{Straße (K.STR)}|pw}}}}\pend
           
\pstart
           \textcolor{gray}{\textbf{Herausgeber:}}\pend
           
\pstart
           \textcolor{gray}{\textbf{\textbf{Prof. Dr. I. Singer\pwindex{Singer, Isidor 16.01.1857 – 08.12.1927@\textsc{Singer, Isidor} (16.01.1857 – 08.12.1927), \emph{Journalist/Journalistin, Herausgeber/Herausgeberin, Soziologe/Soziologin}|pw}}}}\pend
           
\pstart
           \textcolor{gray}{\textbf{\textbf{Dr. Heinrich Kanner\pwindex{Kanner, Heinrich 09.11.1864 – 15.02.1930@\textsc{Kanner, Heinrich} (09.11.1864 – 15.02.1930), \emph{Herausgeber/Herausgeberin, Publizist/Publizistin}|pw}}}}\pend
           
\pstart
           \textcolor{gray}{\textbf{\textbf{Feuilleton-Redaktion}}}\pend
           \vspace{0.5em}
\pstart
           Lieber, bis jetzt waren die Kinder\pwindex{Salten, Paul 11.08.1903 – 08.05.1937@\textsc{Salten, Paul} (11.08.1903 – 08.05.1937), \emph{Filmcutter/Filmcutterin}|pw}\pwindex{Rehmann, Anna Katharina 18.08.1904 – 27.03.1977@\textsc{Rehmann, Anna Katharina} (18.08.1904 – 27.03.1977), \emph{Schauspieler/Schauspielerin, Übersetzer/Übersetzerin}|pw} krank und Paul\pwindex{Salten, Paul 11.08.1903 – 08.05.1937@\textsc{Salten, Paul} (11.08.1903 – 08.05.1937), \emph{Filmcutter/Filmcutterin}|pw} hat uns wieder viele Sorgen gemacht. Deshalb sind wir nicht abgeko{\geminationm}en. Schreiben Sie mir, ob es Ihnen passt, wenn wir
                  \label{K_L03412-1v}\edtext{Samstag nach Reichenau\oindex{Reichenau an der Rax@\textbf{Reichenau an der Rax}, \emph{A.ADM3}|pw}}{\lemma{\textnormal{\emph{Samstag nach Reichenau}}}\Cendnote{\textnormal{Dazu kam es erst am 26. 7. 1905.}}}\label{K_L03412-1} kommen, und ob
               Sie dann Lust haben (nur für diesen Fall kämen wir) am Sonntag oder Montag die \label{K_L03412-2v}\edtext{Maria Zell\oindex{Mariazell@\textbf{Mariazell}, \emph{P.PPLA3}|pw}er Partie}{\lemma{\textnormal{\emph{Maria Zeller Partie}}}\Cendnote{\textnormal{Diese fand erst am Monatsende und ohne Schnitzler statt, vgl. Felix Salten und Richard Metzl an Arthur
               Schnitzler, [30. 7. 1905?].}}}\label{K_L03412-2} mitzumachen. Ich habe auch Eisenerz\oindex{Eisenerz@\textbf{Eisenerz}, \emph{P.PPLA3}|pw} u. s. w. vor, worüber wir aber noch
               sprechen könnten. Ich denke mir: Samstag Tennis, Sonntag Tennis. Montag{ }früh od. Sonntag{ }Abds. Abfahrt nach Mzll.\oindex{Mariazell@\textbf{Mariazell}, \emph{P.PPLA3}|pw}\pend
           
\pstart
           herzliche Grüße von uns an Sie Beide\pwindex{Schnitzler, Olga 17.01.1882 – 13.01.1970@\textsc{Schnitzler, Olga} (17.01.1882 – 13.01.1970), \emph{Schauspieler/Schauspielerin, Sänger/Sängerin}|pwv}{ }{\\[\baselineskip]}Ihr {\\[\baselineskip]}\spacefill\mbox{Salten}\pend
           \leftskip=0em{}
\pstart
           \noindent{}Das \label{K_L03412-3v}\edtext{Stück\pwindex{Andere@\emph{Die Andere}|pwv} von Bahr\pwindex{Bahr, Hermann 19.07.1863 – 15.01.1934@\textsc{Bahr, Hermann} (19.07.1863 – 15.01.1934), \emph{Schriftsteller/Schriftstellerin, Kritiker/Kritikerin}|pw}}{\lemma{\textnormal{\emph{Stück von Bahr}}}\Cendnote{\textnormal{\emph{Die Andere}\pwindex{Andere@\emph{Die Andere}|pwk}, siehe Arthur Schnitzler an Hermann Bahr, 30. 7. 1905 und A. S.: \emph{Tagebuch}, 26. 7. 1905.
                  }}}\label{K_L03412-3} haben Sie erhalten?\pend
           \selectlanguage{ngerman}\endnumbering\briefempfaengerindex{Schnitzler, Arthur@\textsc{Schnitzler, Arthur}!zzzSalten, Felix@\emph{von Felix Salten}!1905-07-181@{18. 7. 1905}|)be}\mylabel{L03412h}  \normalsize

\doendnotes{C}
\bigskip
\vfill

\clearpage

\footnotesize

\lohead{\textsc{register}}

% Definiere theindex-Environment komplett neu ohne reledmac
\makeatletter
\renewenvironment{theindex}{%
  \section*{\indexname}%
  \setlength{\parindent}{0pt}%
  \setlength{\parskip}{0pt plus 0.3pt}%
  \let\item\@idxitem
}{%
  \clearpage
}
\makeatother

\IfFileExists{\jobname-pw.ind}{\input{\jobname-pw.ind}}{}

\end{document}

      