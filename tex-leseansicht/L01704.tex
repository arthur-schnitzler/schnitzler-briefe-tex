%% latex-korrekturansicht-vorspann.tex
%% Vorspann für die Korrekturansicht.
%% Lädt die gemeinsame Datei latex-vorspann.tex mit gesetztem Schalter.

\newif\ifkorrekturansicht
\korrekturansichttrue

\input{../tex-inputs/latex-vorspann}


\section[Hugo von Hofmannsthal an Arthur Schnitzler, {[}Anfang September 1907{]}]{L01704 Hugo von Hofmannsthal an Arthur Schnitzler, {[}Anfang
               September 1907{]}}
\nopagebreak\mylabel{L01704v}
\rehead{ }\normalsize\beginnumbering\briefempfaengerindex{Schnitzler, Arthur@\textsc{Schnitzler, Arthur}!zzzHofmannsthal, Hugo von@\emph{von Hugo von Hofmannsthal}!1907-09-071@{{[}Anfang
                  September 1907{]}}|(be}
\toendnotes[C]{\smallbreak\pagebreak[2]}\Standort{CUL, Schnitzler, B 43.}
\physDesc{Telegramm, 155 Zeichen
\newline{}maschinell
\newline{}Schnitzler: mit Bleistift datiert »Anf Sep 907« 
\newline{}Ordnung: 1) beschnitten  2) mit Bleistift von unbekannter Hand nummeriert:
                                    »288«}
\buchAbdrucke{\weitereDrucke{Hugo von Hofmannsthal, Arthur Schnitzler: \emph{Briefwechsel}. Frankfurt am Main: \emph{S. Fischer} 1964, S. 231.} }\toendnotes[C]{\smallbreak}
\pstart
           {\pb}m{[}e{]}ran\oindex{Meran@\textbf{Meran}, \emph{P.PPLA3}|pw} fr semmering\oindex{Semmering@\textbf{Semmering}, \emph{A.ADM3}|pw} 1+ 359 20 10 40 m\pend
           \vspace{0.5em}
\pstart
           mindestens \label{K_L01704-1v}\edtext{fuenfzehn}{\lemma{\textnormal{\emph{fuenfzehn}}}\Cendnote{\textnormal{Es handelt sich um Überlegungen, den
                  Vorabdruck von \emph{Der Weg ins Freie}\pwindex{Weg ins Freie. Roman@\emph{Der Weg ins Freie. Roman}|pwk} der
                  Zeitschrift \emph{Morgen}\orgindex{Morgen. Wochenschrift fuer deutsche Kultur@Morgen. Wochenschrift für deutsche Kultur|pwk} zu geben, an der Hofmannsthal\pwindex{Hofmannsthal, Hugo von 1874-02-01 – 1929-07-15@\textsc{Hofmannsthal, Hugo von} (1874-02-01 – 1929-07-15), \emph{Schriftsteller/Schriftstellerin}|pwk} mitarbeitete.}}}\label{K_L01704-1}
               viellejcht zwanzig da ja sonst abdruck\pwindex{Weg ins Freie. Roman@\emph{Der Weg ins Freie. Roman}|pwv}{ }rundschau\orgindex{Neue Rundschau, Neue Deutsche Rundschau, Freie Buehne@Neue Rundschau, Neue Deutsche Rundschau, Freie Bühne|pw} viel erfreulicher wuerde zwanzig
               vorschlagen herzlich\pend
           \pstart \spacefill\mbox{hugo}\pend{}\selectlanguage{ngerman}\endnumbering\briefempfaengerindex{Schnitzler, Arthur@\textsc{Schnitzler, Arthur}!zzzHofmannsthal, Hugo von@\emph{von Hugo von Hofmannsthal}!1907-09-011@{{[}Anfang
                  September 1907{]}}|)be}\mylabel{L01704h}  \normalsize

\doendnotes{C}
\bigskip
\vfill

\clearpage

\footnotesize

\lohead{\textsc{register}}

% Definiere theindex-Environment komplett neu ohne reledmac
\makeatletter
\renewenvironment{theindex}{%
  \section*{\indexname}%
  \setlength{\parindent}{0pt}%
  \setlength{\parskip}{0pt plus 0.3pt}%
  \let\item\@idxitem
}{%
  \clearpage
}
\makeatother

\IfFileExists{\jobname-pw.ind}{\input{\jobname-pw.ind}}{}

\end{document}

      