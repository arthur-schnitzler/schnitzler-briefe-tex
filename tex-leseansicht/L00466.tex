%% latex-leseansicht-vorspann.tex
%% Vorspann für die Leseansicht.
%% Lädt die gemeinsame Datei latex-vorspann.tex mit nicht gesetztem Schalter.

\newif\ifkorrekturansicht
\korrekturansichtfalse

\input{../tex-inputs/latex-vorspann}


               \section[Arthur Schnitzler an Richard Beer-Hofmann, {[}25. 7. 1895?{]}]{ Arthur Schnitzler an Richard Beer-Hofmann, {[}25. 7. 1895?{]}}\nopagebreak\mylabel{v}\rehead{ }\begin{ledgroupsized}[t]{13cm}\normalsize\beginnumbering\briefempfaengerindex{Beer-Hofmann, Richard@\textsc{Beer-Hofmann, Richard}!zzzSchnitzler, Arthur@\emph{von Arthur Schnitzler}!1895-07-251@{{[}27. 5. 1895?{]}}|(be} \toendnotes[C]{\smallbreak\pagebreak[2]} \Standort{YCGL, MSS 31.}
\physDesc{Briefkarte, Umschlag
\newline{}Handschrift: Bleistift, deutsche Kurrent\newline{}Versand: ohne postalischen Übermittlungsvermerk }\toendnotes[C]{\smallbreak}\pstart{}{\pb}Herrn \textsc{Dr Rich.
                     Beer-Hofmann}\pend{}\pstart{}\textsc{Egelmoos 22\oindex{Eglmoosgasse@\textbf{Eglmoosgasse}|pw}.}\pend{}{\bigskip}\pstart
           \noindent{}{\pb}Lieber Richard, ich ſprach
               Vormittg die Ernst’s\pwindex{Ernst, Carla 15.05.1867 – 15.6.1925@\textsc{Ernst, Carla} (15.05.1867 – 15.6.1925), \emph{Schauspielerin}|pw}\pwindex{Kohn, Katharina 18.12.1863 – 30.01.1942@\textsc{Kohn, Katharina} (18.12.1863 – 30.01.1942)|pw}, die nach \label{K_L00466_1v}\edtext{Strobl\oindex{Strobl@\textbf{Strobl}|pw}}{\lemma{\textnormal{\emph{Strobl}}}\Cendnote{\textnormal{Das erlaubt eine Datierung
                  des Korrespondenzstücks mithilfe des \emph{Tagebuchs}\pwindex{Schnitzler, Arthur 15.05.1862 – 21.10.1931@\textsc{Schnitzler, Arthur} (15.05.1862 – 21.10.1931), \emph{Schriftsteller, Mediziner}!Tagebuch1981 – 2000@\strich\emph{Tagebuch} {[}1981 – 2000{]}|pwk}.}}}\label{K_L00466_1h} fahren, – was natürlich zu nichts
               verpflichtet. Sagen Sie mir, ob Sie ſchon um ½ 5 fortgehen müſſen, ob Sie was
               beſondres vorhaben, on Sie (und ich) ins Theater gehen? We{\geminationn} nur ſpazieren gegangen {\pb}werden ſoll, möcht ich Sie lieber erſt um ½ 6
               abholen.\pend
           \pstart
           Antwort ſehr erwünſcht.\pend
           \pstart
           Herzlichen Gruß{\\[\baselineskip]}Ihr \spacefill\mbox{Arthur.}\pend
           \leftskip=0em{}          \endnumbering\briefempfaengerindex{Beer-Hofmann, Richard@\textsc{Beer-Hofmann, Richard}!zzzSchnitzler, Arthur@\emph{von Arthur Schnitzler}!1895-07-251@{{[}27. 5. 1895?{]}}|)be}\mylabel{h}\end{ledgroupsized}  \newcommand{\dateiname}{L00466}\newcommand{\titel}{Arthur Schnitzler an Richard Beer-Hofmann, [25. 7. 1895?]}\newcommand{\editorInnen}{Martin Anton Müller und Gerd-Hermann Susen}
            \footnotesize
\begin{ledgroupsized}[t]{11.5cm}
\doendnotes{C}
\end{ledgroupsized}
         %% latex-leseansicht-abspann.tex
%% Abspann für die Leseansicht.
%% Der Schalter \ifkorrekturansicht ist bereits durch den Vorspann gesetzt.

%% latex-abspann.tex
%% Gemeinsamer Abspann für Korrekturansicht und Leseansicht.
%% Setzt den Schalter \ifkorrekturansicht voraus (gesetzt in den
%% einbindenden Dateien latex-korrekturansicht-abspann.tex bzw.
%% latex-leseansicht-abspann.tex).
%% ---------------------------------------------------------------

\normalsize

% Das esempio-Environment wird nur in der Leseansicht benötigt
\ifkorrekturansicht\else
\newenvironment{esempio}[3]%
{
    \vspace{1.5ex}
    \rlap{\underline{#1}}
    \par
    \setlength{\parindent}{0cm}
    \nopagebreak
    \leftskip=#2cm
    \rightskip=#3cm
}
{
    \par
}
\fi

\doendnotes{C}
\bigskip
\vfill

\clearpage

\footnotesize

\ifkorrekturansicht
  \lohead{\textsc{register}}
\fi

% theindex-Environment neu definieren ohne reledmac
\makeatletter
\renewenvironment{theindex}{%
  \ifkorrekturansicht
    \section*{\indexname}%
  \else
    \subsubsection*{Index der erwähnten Entitäten}%
  \fi
  \setlength{\parindent}{0pt}%
  \setlength{\parskip}{0pt plus 0.3pt}%
  \let\item\@idxitem
}{%
  \ifkorrekturansicht\clearpage\fi
}
\makeatother

\IfFileExists{\jobname-pw.ind}{\input{\jobname-pw.ind}}{}

% Quellenangabe nur in der Leseansicht
\ifkorrekturansicht\else
% Fallback-Definitionen, falls die .tex-Datei \titel etc. nicht gesetzt hat
\providecommand{\titel}{}
\providecommand{\editorInnen}{}
\providecommand{\dateiname}{\jobname}

\vspace{3cm}

\vfill

\footnotesize
\textsc{Quelle}: \titel. Herausgegeben von {\editorInnen}. In: \emph{Arthur Schnitzler: Briefwechsel mit Autorinnen und Autoren}.
 Digitale Edition, https://schnitzler-briefe.acdh.oeaw.ac.at/{\dateiname}.html (Stand \today)
\fi

\end{document}


      