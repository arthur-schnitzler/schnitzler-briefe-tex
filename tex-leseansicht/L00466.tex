%% latex-korrekturansicht-vorspann.tex
%% Vorspann für die Korrekturansicht.
%% Lädt die gemeinsame Datei latex-vorspann.tex mit gesetztem Schalter.

\newif\ifkorrekturansicht
\korrekturansichttrue

\input{../tex-inputs/latex-vorspann}


\section[Arthur Schnitzler an Richard Beer-Hofmann, {[}25. 7. 1895?{]}]{L00466 Arthur Schnitzler an Richard Beer-Hofmann, {[}25. 7. 1895?{]}}
\nopagebreak\mylabel{L00466v}
\rehead{ }\normalsize\beginnumbering\briefempfaengerindex{Beer-Hofmann, Richard@\textsc{Beer-Hofmann, Richard}!zzzSchnitzler, Arthur@\emph{von Arthur Schnitzler}!1895-07-251@{{[}25. 7. 1895?{]}}|(be}
\toendnotes[C]{\smallbreak\pagebreak[2]}\Standort{YCGL, MSS 31.}
\physDesc{Briefkarte, , Umschlag, 390 Zeichen
\newline{}Handschrift: Bleistift, deutsche Kurrent
\newline{}Versand: ohne postalischen Übermittlungsvermerk }\toendnotes[C]{\smallbreak}\pstart{}{\pb}Herrn \textsc{Dr Rich.
                     Beer-Hofmann}\pend{}\pstart{}\textsc{Egelmoos 22\oindex{Eglmoosgasse@\textbf{Eglmoosgasse}, \emph{Bezirk (A.BZK)}|pw}.}\pend{}{\bigskip}\vspace{1em}
\pstart
           \noindent{}{\pb}Lieber Richard, ich ſprach Vormittg die Ernst’s\pwindex{Ernst, Carla 15.05.1867 – 15.6.1925@\textsc{Ernst, Carla} (15.05.1867 – 15.6.1925), \emph{Schauspieler/Schauspielerin}|pw}\pwindex{Kohn, Katharina 18.12.1863 – 30.01.1942@\textsc{Kohn, Katharina} (18.12.1863 – 30.01.1942)|pw}, die nach \label{K_L00466-1v}\edtext{Strobl\oindex{Strobl@\textbf{Strobl}, \emph{A.ADM3}|pw}}{\lemma{\textnormal{\emph{Strobl}}}\Cendnote{\textnormal{Das erlaubt eine Datierung des
                  Korrespondenzstücks mit Hilfe des \emph{Tagebuchs}\pwindex{Tagebuch@\emph{Tagebuch}|pwk}.}}}\label{K_L00466-1} fahren, – was natürlich zu nichts verpflichtet. Sagen Sie
               mir, ob Sie ſchon um ½ 5 fortgehen müſſen, ob Sie was beſondres
               vorhaben, on Sie (und ich) ins Theater gehen? We{\geminationn} nur
               ſpazieren gegangen {\pb}werden ſoll, möcht ich Sie lieber
               erſt um ½ 6 abholen.\pend
           
\pstart
           Antwort ſehr erwünſcht.\pend
           
\pstart
           Herzlichen Gruß{\\[\baselineskip]}Ihr \spacefill\mbox{Arthur.}\pend
           \leftskip=0em{}\selectlanguage{ngerman}\endnumbering\briefempfaengerindex{Beer-Hofmann, Richard@\textsc{Beer-Hofmann, Richard}!zzzSchnitzler, Arthur@\emph{von Arthur Schnitzler}!1895-07-251@{{[}25. 7. 1895?{]}}|)be}\mylabel{L00466h}  \normalsize

\doendnotes{C}
\bigskip
\vfill

\clearpage

\footnotesize

\lohead{\textsc{register}}

% Definiere theindex-Environment komplett neu ohne reledmac
\makeatletter
\renewenvironment{theindex}{%
  \section*{\indexname}%
  \setlength{\parindent}{0pt}%
  \setlength{\parskip}{0pt plus 0.3pt}%
  \let\item\@idxitem
}{%
  \clearpage
}
\makeatother

\IfFileExists{\jobname-pw.ind}{\input{\jobname-pw.ind}}{}

\end{document}

      