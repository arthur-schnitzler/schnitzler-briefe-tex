%% latex-leseansicht-vorspann.tex
%% Vorspann für die Leseansicht.
%% Lädt die gemeinsame Datei latex-vorspann.tex mit nicht gesetztem Schalter.

\newif\ifkorrekturansicht
\korrekturansichtfalse

\input{../tex-inputs/latex-vorspann}


\section[Arthur Schnitzler an Richard Beer-Hofmann, {{[}}25. 7. 1895?{{]}}]{L00466 Arthur Schnitzler an Richard Beer-Hofmann, {[}25. 7. 1895?{]}}
\nopagebreak\mylabel{L00466v}
\rehead{ }\normalsize\beginnumbering\briefempfaengerindex{Beer-Hofmann, Richard@\textsc{Beer-Hofmann, Richard}!zzzSchnitzler, Arthur@\emph{von Arthur Schnitzler}!1895-07-251@{{[}25. 7. 1895?{]}}|(be}
\toendnotes[C]{\smallbreak\pagebreak[2]}
\correspDesc{Versand  durch Arthur Schnitzler am [25. 7. 1895?] in Bad Ischl
\newline{}Erhalt  durch Richard Beer-Hofmann im Zeitraum [25. 7. 1895
                  – 29. 7. 1895?] in Bad Ischl}\toendnotes[C]{\smallbreak}
\Standort{YCGL, MSS 31.}
\physDesc{Briefkarte, , Kuvert, 390 Zeichen
\newline{}Handschrift: Bleistift, deutsche Kurrent
\newline{}Versand: ohne postalischen Übermittlungsvermerk }\toendnotes[C]{\smallbreak}\pstart{}{\pb}Herrn \textsc{Dr Rich.
                     Beer-Hofmann}\pend{}\pstart{}\textsc{Egelmoos 22\oindex{Eglmoosgasse@\textbf{Eglmoosgasse}, \emph{Bezirk}|pw}.}\pend{}{\bigskip}\vspace{1em}
\pstart
           \noindent{}{\pb}Lieber Richard, ich{ }ſprach Vormittg die Ernst’s\pwindex{Ernst, Carla 15.\,5.\,1867 Wien – 15.\,6.\,1925 ebd.@\textsc{Ernst, Carla} (15.\,5.\,1867 Wien – 15.\,6.\,1925 ebd.), \emph{Schauspielerin}|pw}\pwindex{Kohn, Katharina 18.\,12.\,1863 Wien – 30.\,1.\,1942 Łódź@\textsc{Kohn, Katharina} (18.\,12.\,1863 Wien – 30.\,1.\,1942 Łódź)|pw}, die nach \label{K_L00466-1v}\edtext{Strobl\oindex{Strobl@\textbf{Strobl}, \emph{Verwaltungsgebiet}|pw}}{\lemma{\textnormal{\emph{Strobl}}}\Cendnote{\textnormal{Das erlaubt eine Datierung des
                  Korrespondenzstücks mit Hilfe des \emph{Tagebuchs}\pwindex{Schnitzler, Arthur 15.\,5.\,1862 Wien – 21.\,10.\,1931 ebd.@\textsc{Schnitzler, Arthur} (15.\,5.\,1862 Wien – 21.\,10.\,1931 ebd.), \emph{Schriftsteller, Mediziner}!Tagebuch@\strich\emph{Tagebuch}|pwk}.}}}\label{K_L00466-1} fahren, – was natürlich zu nichts verpflichtet. Sagen Sie
               mir, ob Sie{ }ſchon um ½ 5 fortgehen müſſen, ob Sie was beſondres
               vorhaben, on Sie (und ich) ins Theater gehen? We{\geminationn} nur{ }ſpazieren gegangen {\pb}werden{ }ſoll, möcht ich Sie lieber
               erſt um ½ 6 abholen.\pend
           
\pstart
           Antwort{ }ſehr erwünſcht.\pend
           
\pstart
           Herzlichen Gruß{\\[\baselineskip]}Ihr \spacefill\mbox{Arthur.}\pend
           \leftskip=0em{}\selectlanguage{ngerman}\endnumbering\briefempfaengerindex{Beer-Hofmann, Richard@\textsc{Beer-Hofmann, Richard}!zzzSchnitzler, Arthur@\emph{von Arthur Schnitzler}!1895-07-251@{{[}25. 7. 1895?{]}}|)be}\mylabel{L00466h}  \newcommand{\dateiname}{L00466}\newcommand{\titel}{Arthur Schnitzler an Richard Beer-Hofmann, [25. 7. 1895?]}\newcommand{\editorInnen}{Martin Anton Müller und Gerd-Hermann Susen}%% latex-leseansicht-abspann.tex
%% Abspann für die Leseansicht.
%% Der Schalter \ifkorrekturansicht ist bereits durch den Vorspann gesetzt.

%% latex-abspann.tex
%% Gemeinsamer Abspann für Korrekturansicht und Leseansicht.
%% Setzt den Schalter \ifkorrekturansicht voraus (gesetzt in den
%% einbindenden Dateien latex-korrekturansicht-abspann.tex bzw.
%% latex-leseansicht-abspann.tex).
%% ---------------------------------------------------------------

\normalsize

% Das esempio-Environment wird nur in der Leseansicht benötigt
\ifkorrekturansicht\else
\newenvironment{esempio}[3]%
{
    \vspace{1.5ex}
    \rlap{\underline{#1}}
    \par
    \setlength{\parindent}{0cm}
    \nopagebreak
    \leftskip=#2cm
    \rightskip=#3cm
}
{
    \par
}
\fi

\doendnotes{C}
\bigskip
\vfill

\clearpage

\footnotesize

\ifkorrekturansicht
  \lohead{\textsc{register}}
\fi

% theindex-Environment neu definieren ohne reledmac
\makeatletter
\renewenvironment{theindex}{%
  \ifkorrekturansicht
    \section*{\indexname}%
  \else
    \subsubsection*{Index der erwähnten Entitäten}%
  \fi
  \setlength{\parindent}{0pt}%
  \setlength{\parskip}{0pt plus 0.3pt}%
  \let\item\@idxitem
}{%
  \ifkorrekturansicht\clearpage\fi
}
\makeatother

\IfFileExists{\jobname-pw.ind}{\input{\jobname-pw.ind}}{}

% Quellenangabe nur in der Leseansicht
\ifkorrekturansicht\else
% Fallback-Definitionen, falls die .tex-Datei \titel etc. nicht gesetzt hat
\providecommand{\titel}{}
\providecommand{\editorInnen}{}
\providecommand{\dateiname}{\jobname}

\vspace{3cm}

\vfill

\footnotesize
\textsc{Quelle}: \titel. Herausgegeben von {\editorInnen}. In: \emph{Arthur Schnitzler: Briefwechsel mit Autorinnen und Autoren}.
 Digitale Edition, https://schnitzler-briefe.acdh.oeaw.ac.at/{\dateiname}.html (Stand \today)
\fi

\end{document}


