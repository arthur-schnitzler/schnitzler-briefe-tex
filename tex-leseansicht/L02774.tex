%% latex-leseansicht-vorspann.tex
%% Vorspann für die Leseansicht.
%% Lädt die gemeinsame Datei latex-vorspann.tex mit nicht gesetztem Schalter.

\newif\ifkorrekturansicht
\korrekturansichtfalse

\input{../tex-inputs/latex-vorspann}


         
         \renewcommand{\erwaehntePersonen}{Personen: Henri Albert, Peter Altenberg, Lou Andreas-Salomé, Hermann Bahr, Samuel Fischer, Paul Goldmann, Maximilian Harden, Hugo von Hofmannsthal, Alfred Kerr, Pierre Lalo, Albert Langen, Pierre Louÿs, Friedrich Nietzsche, Leopold Sonnemann, Jean Thorel, Theodor Zasche}
         \renewcommand{\erwaehnteInstitutionen}{Institutionen: Albert Langen, Frankfurter Zeitung, L’Aube, Neue Rundschau, Neue Deutsche Rundschau, Freie Bühne, S. Fischer Verlag, Simplicissimus}
         \renewcommand{\erwaehnteOrte}{Orte: Café Griensteidl, Dänemark, Florenz, Hamburg, Köln, Norwegen, Paris, Sankt Petersburg, Schweden, Schweiz, Wien, rue Feydeau}
         \renewcommand{\erwaehnteWerke}{Werke: Amourette. Pièce en trois actes. Adaptée de Arthur Schnitzler, Aphrodite. Mœurs antiques, Arthur Schnitzler, Die Zeit. Wiener Wochenschrift, Die überspannte Person, Ein neuer Dichter, Figaro. Humoristisches Wochenblatt, Frankfurter Zeitung, Freiwild. Schauspiel in 3 Akten, Friedrich Nietzsche in seinen Werken, Liebelei. Schauspiel in drei Akten, L’Aube, Mourir. Roman, Neue Deutsche Rundschau, Simplicissimus, Tagebuch, Unter Wiener Grisetten, Wie ich es sehe}
               \section[ Paul Goldmann an Arthur Schnitzler, 17. 5. {[}1896{]}]{ Paul Goldmann an Arthur Schnitzler, 17. 5. {[}1896{]}}\nopagebreak\mylabel{v}\rehead{ }\begin{ledgroupsized}[t]{13cm}\normalsize\beginnumbering\briefempfaengerindex{Schnitzler, Arthur@\textsc{Schnitzler, Arthur}!zzzGoldmann, Paul@\emph{von Paul Goldmann}!1896-05-172@{17. 5. {[}1896{]}}|(be} \toendnotes[C]{\smallbreak\pagebreak[2]} \Standort{DLA, A:Schnitzler, HS.NZ85.1.3166.}
\physDesc{Brief, 5 Blätter, 19 Seiten, 7377 Zeichen
\newline{}Handschrift: blaue Tinte, deutsche Kurrent
\newline{}Schnitzler: 1) mit Bleistift das Jahr »96« vermerkt  2) mit rotem Buntstift dreizehn Unterstreichungen}\toendnotes[C]{\smallbreak}\pstart
           \noindent{}{\pb}\textcolor{gray}{\textbf{\textbf{Frankfurter Zeitung\orgindex{Frankfurter Zeitung@Frankfurter Zeitung|pw}}}}\pend
           \pstart
           \textcolor{gray}{\textbf{(\begin{otherlanguage}{french}Gazette de Francfort\end{otherlanguage}\orgindex{Frankfurter Zeitung@Frankfurter Zeitung|pw}).}}\pend
           \pstart
           \textcolor{gray}{\textbf{\textbf{\begin{otherlanguage}{french}Fondateur M.\end{otherlanguage}{ }L. Sonnemann\pwindex{Sonnemann, Leopold 1831-10-29 – 1909-10-30@\textsc{Sonnemann, Leopold} (1831-10-29 – 1909-10-30), \emph{Journalist, Herausgeber}|pw}.}}}\pend
           \pstart
           \begin{otherlanguage}{french}\textcolor{gray}{\textbf{Journal\pwindex{?? Werk@Nicht ermittelte Verfasserinnen und Verfasser!Frankfurter Zeitung1856 – 1943@\emph{Frankfurter Zeitung} {[}1856 – 1943{]}|pwv} politique,
                        financier,}}\end{otherlanguage}\pend
           \pstart
           \begin{otherlanguage}{french}\textcolor{gray}{\textbf{commercial et littéraire.}}\end{otherlanguage}\pend
           \pstart
           \begin{otherlanguage}{french}\textcolor{gray}{\textbf{\textbf{Paraissant trois fois par jour.}}}\end{otherlanguage}\pend
           \pstart
           \begin{otherlanguage}{french}\textcolor{gray}{\textbf{\textbf{Bureau à Paris\oindex{Paris@\textbf{Paris}|pw}:}}}\end{otherlanguage}\pend
           \pstart
           \begin{otherlanguage}{french}\textcolor{gray}{\textbf{\textbf{24. Rue Feydeau\oindex{rue Feydeau@\textbf{rue Feydeau}|pw}.}}}\end{otherlanguage}\hfill \textsc{Paris\oindex{Paris@\textbf{Paris}|pw}}, 17. Mai.\pend
           \pstart\center{}Mein lieber Freund,\pend\pstart
           1.) Nach einem flüchtigen Überſchlag von Zeit und Koſten ſehe ich, daß ich mit Dir
               werde kaum zuſammenreiſen können. Denke ſelbſt: Ich bekomme vier Wochen Urlaub und
               habe während desſelben etwa 700 \textsc{Francs} zu verzehren. Die
               Reiſe von hier über \textsc{Hamburg\oindex{Hamburg@\textbf{Hamburg}|pw}} nach Dänemark\oindex{Daenemark@\textbf{Dänemark}|pw}, Schweden\oindex{Schweden@\textbf{Schweden}|pw} und Norwegen\oindex{Norwegen@\textbf{Norwegen}|pw}{ }\strikeout{würde} und von da wieder nach \textsc{Paris\oindex{Paris@\textbf{Paris}|pw}} zurück würde allein an 500 \textsc{francs} koſten. Die
               Entfernungen {\pb}ſind außerdem groß, und ich würde
               einen guten Theil meines Urlaubs auf der Eiſenbahn verbringen. Nun ſind bei meiner
               Reiſe andere Rückſichten maßgebend, als bei Deiner. Du gehſt von Wien\oindex{Wien@\textbf{Wien}|pw} fort, um Neues zu ſehen, ich entferne mich von \textsc{Paris\oindex{Paris@\textbf{Paris}|pw}}, um auszuruhen. Endlich intereſſiren mich die \strikeout{ſkan}{ }ſkandinaviſchen Länder\oindex{Daenemark@\textbf{Dänemark}|pwv}\oindex{Schweden@\textbf{Schweden}|pwv}\oindex{Norwegen@\textbf{Norwegen}|pwv} gar wenig, und eine Reiſe nach der Schweiz\oindex{Schweiz@\textbf{Schweiz}|pw}, mit einem kleinen Abſtecher nach Florenz\oindex{Florenz@\textbf{Florenz}|pw}, wäre mir weitaus zuträglicher. Um Dich wiederzuſehen,
               bin ich freilich zu allen Conceſſionen {\pb}bereit, aber
               das ſkandinavi\oindex{Daenemark@\textbf{Dänemark}|pwv}\oindex{Schweden@\textbf{Schweden}|pwv}\oindex{Norwegen@\textbf{Norwegen}|pwv}ſche Project erweiſt ſich bei näherer Betrachtung als
               Unmöglichkeit für mich. Mach’ mir alſo, bitte, einen \label{K_L02774-1v}\edtext{anderen Vorſchlag}{\lemma{\textnormal{\emph{anderen Vorſchlag}}}\Cendnote{\textnormal{Siehe Paul Goldmann an Arthur Schnitzler, 29. 4. [1896].
               }}}\label{K_L02774-1h}. Ich gedenke, ſo zwiſchen 5. und 10. Auguſt aufzubrechen und würde meinen Urlaub als
               verfehlt betrachten, wenn ich Dich nicht ſehen könnte, worauf ich mich nun jetzt
               ſchon ſeit meinem letzten Urlaub freue.\pend
           \pstart
           2.) In Sachen von »\textsc{Mourir\pwindex{Schnitzler, Arthur 15.05.1862 – 21.10.1931@\textsc{Schnitzler, Arthur} (15.05.1862 – 21.10.1931), \emph{Schriftsteller, Mediziner}!Mourir. Roman1895-04-27 – 1895-06-01@\strich\emph{Mourir. Roman} {[}1895-04-27 – 1895-06-01{]}|pw}}« will ich demnächſt etwas thun. Gegenwärtig habe ich ſo Tauſenderlei zu
               erledigen und komme nicht {\pb}dazu, die Leute zu ſehen,
               an die ich denke. Haſt Du an \textsc{Thorel\pwindex{Thorel, Jean 1859-09-11 – 1916-08-20@\textsc{Thorel, Jean} (1859-09-11 – 1916-08-20), \emph{Übersetzer, Schriftsteller}|pw}} ein Exemplar\pwindex{Schnitzler, Arthur 15.05.1862 – 21.10.1931@\textsc{Schnitzler, Arthur} (15.05.1862 – 21.10.1931), \emph{Schriftsteller, Mediziner}!Liebelei. Schauspiel in drei Akten1895-10-09@\strich\emph{Liebelei. Schauspiel in drei Akten} {[}1895-10-09{]}|pwv}
               geſchickt?\pend
           \pstart
           3.) Ich bleibe dabei, daß ich Deine \label{K_L02774-2v}\edtext{Mitarbeiterſchaft}{\lemma{\textnormal{\emph{Mitarbeiterſchaft}}}\Cendnote{\textnormal{Siehe Paul Goldmann an Arthur Schnitzler, 29. 4. [1896].
               }}}\label{K_L02774-2h} bei \textsc{Albert Langen\pwindex{?? Werk@Nicht ermittelte Verfasserinnen und Verfasser!Simplicissimus1896-04-04 – 1944-09-13@\emph{Simplicissimus} {[}1896-04-04 – 1944-09-13{]}|pwv}\orgindex{Simplicissimus@Simplicissimus|pwv}} bedaure. \strikeout{Die} Daß Directoren, die über Dich
               ſchimpfen, trotzdem Deine Stücke aufführen, iſt richtig. Aber die Directoren ſind
                  \strikeout{\textcolor{gray}{×}\-\textcolor{gray}{×}} nicht zu umgehen. Hingegen die Sachen\pwindex{Schnitzler, Arthur 15.05.1862 – 21.10.1931@\textsc{Schnitzler, Arthur} (15.05.1862 – 21.10.1931), \emph{Schriftsteller, Mediziner}!ueberspannte Person1896-04-18@\strich\emph{Die überspannte Person} {[}1896-04-18{]}|pwv}, die bei \textsc{Langen\orgindex{Albert Langen@Albert Langen|pwv}\pwindex{Langen, Albert 1869-07-08 – 1909-04-30@\textsc{Langen, Albert} (1869-07-08 – 1909-04-30), \emph{Verleger}|pw}} erſchienen ſind, mußten nicht \strikeout{ged} gedruckt
               werden. \strikeout{Au} Auch leiſteſt Du \textsc{Langen\pwindex{Langen, Albert 1869-07-08 – 1909-04-30@\textsc{Langen, Albert} (1869-07-08 – 1909-04-30), \emph{Verleger}|pw}}{ }\strikeout{d\textcolor{gray}{e}} einen ganz beſonderen Dienſt, indem Du ihm für ſein neues Unternehmen\orgindex{Albert Langen@Albert Langen|pwv} die gegenwärtig {\pb}beſonders große Autorität Deines Namens zur
               Verfügung \strikeout{ſtell} ſtellſt. Ferner: Wenn die
               Theater-Directoren über Dich ſchimpfen, weißt Du es nicht. Bei \textsc{Langen\pwindex{Langen, Albert 1869-07-08 – 1909-04-30@\textsc{Langen, Albert} (1869-07-08 – 1909-04-30), \emph{Verleger}|pw}} weißt Du es. Und würdeſt Du einem Director Dein Stück geben, der es mit den
               Worten empfinge: »Aufführen muß ichs wohl, aber Sie können nicht deutſch ſchreiben«?
               Endlich und letz{[}t{]}lich geht es mir nicht in den Sinn, daß es in
               der Welt niemals eine Strafe für Lausbüberei geben ſoll. \textsc{Langen\pwindex{Langen, Albert 1869-07-08 – 1909-04-30@\textsc{Langen, Albert} (1869-07-08 – 1909-04-30), \emph{Verleger}|pw}} hat ſich vor {\pb}Deinen Erfolgen wie ein Lausbube
               über Dich geäußert. Jetzt ſieht er, daß er ſich verhauen hat, und Du ſendeſt ihm
               ſofort liebenswürdig Deine Manuſkripte\pwindex{Schnitzler, Arthur 15.05.1862 – 21.10.1931@\textsc{Schnitzler, Arthur} (15.05.1862 – 21.10.1931), \emph{Schriftsteller, Mediziner}!ueberspannte Person1896-04-18@\strich\emph{Die überspannte Person} {[}1896-04-18{]}|pwv}: »Bitte, mein Herr\pwindex{Langen, Albert 1869-07-08 – 1909-04-30@\textsc{Langen, Albert} (1869-07-08 – 1909-04-30), \emph{Verleger}|pwv}, wir wollen, den kleinen Irrthum berichtigen, der in unſerer
               gegenſeitigen Schätzung mit untergelaufen iſt.«\pend
           \pstart
           4.) Mit \textsc{Harden\pwindex{Harden, Maximilian 20.10.1861 – 30.10.1927@\textsc{Harden, Maximilian} (20.10.1861 – 30.10.1927), \emph{Schriftsteller, Publizist}|pw}} haſt Du vielleicht Recht; aber hüte Dich vor ihm, er iſt ein falſcher Hund. Mit
               der »Liebelei\pwindex{Schnitzler, Arthur 15.05.1862 – 21.10.1931@\textsc{Schnitzler, Arthur} (15.05.1862 – 21.10.1931), \emph{Schriftsteller, Mediziner}!Liebelei. Schauspiel in drei Akten1895-10-09@\strich\emph{Liebelei. Schauspiel in drei Akten} {[}1895-10-09{]}|pw}« iſt es Dir \uline{nicht} über Gebühr gut gegangen. {\pb}Sie\pwindex{Schnitzler, Arthur 15.05.1862 – 21.10.1931@\textsc{Schnitzler, Arthur} (15.05.1862 – 21.10.1931), \emph{Schriftsteller, Mediziner}!Liebelei. Schauspiel in drei Akten1895-10-09@\strich\emph{Liebelei. Schauspiel in drei Akten} {[}1895-10-09{]}|pwv} nimmt vielleicht einen
               geringeren Rang in Deiner Schätzung ein, weil Du ſie mit den anderen Stücken
               vergleichſt, die \uline{Du} ſchreiben könnteſt und ſchreiben
               wiſt. Aber verglichen mit den Stücken, welche die \uline{Anderen} ſchreiben, ſteht ſie im erſten Range.\pend
           \pstart
           5.) Nächſte Woche will ich \textsc{Thorel\pwindex{Thorel, Jean 1859-09-11 – 1916-08-20@\textsc{Thorel, Jean} (1859-09-11 – 1916-08-20), \emph{Übersetzer, Schriftsteller}|pw}} aufſuchen, und dann verabreden wir etwas Definitives in der Überſetzung\pwindex{Thorel, Jean 1859-09-11 – 1916-08-20@\textsc{Thorel, Jean} (1859-09-11 – 1916-08-20), \emph{Übersetzer, Schriftsteller}!Amourette. Piece en trois actes. Adaptee de Arthur Schnitzler1897@\strich\emph{Amourette. Pièce en trois actes. Adaptée de Arthur Schnitzler} {[}Übersetzung, 1897{]}|pwv}s-Angelegenheit. Günſtig ſind
               die Chancen für Aufführung ausländiſcher {\pb}Stücke an
               einem anſtändigen Theater gegenwärtig \uline{nicht}.\pend
           \pstart
           6.) Die »Freie Bühne\pwindex{Neue Deutsche Rundschau1894-01-01 – 1903-12-31@\emph{Neue Deutsche Rundschau} {[}1894-01-01 – 1903-12-31{]}|pw}« bekomme ich nie zu
               Geſicht. Könnteſt Du mir die Nummer\pwindex{Neue Deutsche Rundschau1894-01-01 – 1903-12-31@\emph{Neue Deutsche Rundschau} {[}1894-01-01 – 1903-12-31{]}|pwv} mit dem \label{K_L02774-3v}\edtext{Artikel\pwindex{Kerr, Alfred 25.12.1867 – 12.10.1948@\textsc{Kerr, Alfred} (25.12.1867 – 12.10.1948), \emph{Schriftsteller, Kritiker}!Arthur Schnitzler1896-03@\strich\emph{Arthur Schnitzler} {[}1896-03{]}|pwv}}{\lemma{\textnormal{\emph{Artikel}}}\Cendnote{\textnormal{Alfred Kerr\pwindex{Kerr, Alfred 25.12.1867 – 12.10.1948@\textsc{Kerr, Alfred} (25.12.1867 – 12.10.1948), \emph{Schriftsteller, Kritiker}|pwk}: \emph{Arthur Schnitzler}\pwindex{Kerr, Alfred 25.12.1867 – 12.10.1948@\textsc{Kerr, Alfred} (25.12.1867 – 12.10.1948), \emph{Schriftsteller, Kritiker}!Arthur Schnitzler1896-03@\strich\emph{Arthur Schnitzler} {[}1896-03{]}|pwk}. In: \emph{Neue Deutsche Rundschau (Freie Bühne)}\pwindex{Neue Deutsche Rundschau1894-01-01 – 1903-12-31@\emph{Neue Deutsche Rundschau} {[}1894-01-01 – 1903-12-31{]}|pwk}, Jg. 7, H. 3, März 1896, S. 287–292. (Die \emph{Neue Deutsche Rundschau}\orgindex{Neue Rundschau, Neue Deutsche Rundschau, Freie Buehne@Neue Rundschau, Neue Deutsche Rundschau, Freie Bühne|pwk} wurde unter dem Namen  \emph{Freie Bühne}\orgindex{Neue Rundschau, Neue Deutsche Rundschau, Freie Buehne@Neue Rundschau, Neue Deutsche Rundschau, Freie Bühne|pwk} gegründet, hieß aber seit 1894 nicht mehr so.)}}}\label{K_L02774-3h} über Dich nicht ſchicken?\pend
           \pstart
           7.) Wenn \textsc{Fischer\pwindex{Fischer, Samuel 24.12.1859 – 15.10.1934@\textsc{Fischer, Samuel} (24.12.1859 – 15.10.1934), \emph{Verleger}|pw}\orgindex{S. Fischer Verlag@S. Fischer Verlag|pw}} Dich \strikeout{\textcolor{gray}{o}} ohne Verpflichtung \label{K_L02774-4v}\edtext{honorirt}{\lemma{\textnormal{\emph{honorirt}}}\Cendnote{\textnormal{Für die erste Auflage von
                     \emph{Liebelei}\pwindex{Schnitzler, Arthur 15.05.1862 – 21.10.1931@\textsc{Schnitzler, Arthur} (15.05.1862 – 21.10.1931), \emph{Schriftsteller, Mediziner}!Liebelei. Schauspiel in drei Akten1895-10-09@\strich\emph{Liebelei. Schauspiel in drei Akten} {[}1895-10-09{]}|pwk} erhielt Schnitzler\pwindex{Schnitzler, Arthur 15.05.1862 – 21.10.1931@\textsc{Schnitzler, Arthur} (15.05.1862 – 21.10.1931), \emph{Schriftsteller, Mediziner}|pwk} vom \emph{S. Fischer
                     Verlag}\orgindex{S. Fischer Verlag@S. Fischer Verlag|pwk} 400 Mark. Vgl. A. S.: \emph{Tagebuch}, 30. 4. 1896.}}}\label{K_L02774-4h} hat, ſo geht daraus klar hervor, daß
               er Dich an ſich feſſeln will, um Dich bei Deinen ſämmtlichen nächſten Büchern
               betrügen zu können.\pend
           \pstart
           8.) Ein Menſch\pwindex{Altenberg, Peter 09.03.1859 – 08.01.1919@\textsc{Altenberg, Peter} (09.03.1859 – 08.01.1919), \emph{Schriftsteller}|pwv}, den \textsc{Bahr\pwindex{Bahr, Hermann 19.07.1863 – 15.01.1934@\textsc{Bahr, Hermann} (19.07.1863 – 15.01.1934), \emph{Schriftsteller, Kritiker}|pw}} als \label{K_L02774-5v}\edtext{»neuen Dichter«}{\lemma{\textnormal{\emph{»neuen Dichter«}}}\Cendnote{\textnormal{Peter Altenberg\pwindex{Altenberg, Peter 09.03.1859 – 08.01.1919@\textsc{Altenberg, Peter} (09.03.1859 – 08.01.1919), \emph{Schriftsteller}|pwk}. Vgl. Hermann Bahr\pwindex{Bahr, Hermann 19.07.1863 – 15.01.1934@\textsc{Bahr, Hermann} (19.07.1863 – 15.01.1934), \emph{Schriftsteller, Kritiker}|pwk}: \emph{Ein neuer Dichter}\pwindex{Bahr, Hermann 19.07.1863 – 15.01.1934@\textsc{Bahr, Hermann} (19.07.1863 – 15.01.1934), \emph{Schriftsteller, Kritiker}!neuer Dichter1896-05-02@\strich\emph{Ein neuer Dichter} {[}1896-05-02{]}|pwk}. In: \emph{Die Zeit}\pwindex{Zeit. Wiener Wochenschrift1894 – 1904@\emph{Die Zeit. Wiener Wochenschrift} {[}1894 – 1904{]}|pwk}, Bd. 7, Nr. 83, 2. 5. 1896,
                     S. 75–76.}}}\label{K_L02774-5h} ſignaliſirt, iſt bei mir ſo ſchwer {\pb}compromittirt, daß ich ihn \strikeout{\textcolor{gray}{×}} nicht mehr ohne Vorurtheil leſen kann. Immerhin würde ich gern in das \label{K_L02774-6v}\edtext{Buch\pwindex{Altenberg, Peter 09.03.1859 – 08.01.1919@\textsc{Altenberg, Peter} (09.03.1859 – 08.01.1919), \emph{Schriftsteller}!Wie ich es sehe1896@\strich\emph{Wie ich es sehe} {[}1896{]}|pwv}}{\lemma{\textnormal{\emph{Buch}}}\Cendnote{\textnormal{Peter Altenberg\pwindex{Altenberg, Peter 09.03.1859 – 08.01.1919@\textsc{Altenberg, Peter} (09.03.1859 – 08.01.1919), \emph{Schriftsteller}|pwk}: \emph{Wie ich es sehe}\pwindex{Altenberg, Peter 09.03.1859 – 08.01.1919@\textsc{Altenberg, Peter} (09.03.1859 – 08.01.1919), \emph{Schriftsteller}!Wie ich es sehe1896@\strich\emph{Wie ich es sehe} {[}1896{]}|pwk}. Berlin: \emph{S. Fischer Verlag}\orgindex{S. Fischer Verlag@S. Fischer Verlag|pwk}{ }1896.}}}\label{K_L02774-6h} hineinſchauen. Aber woher ſoll ichs bekommen? Könnteſt Du mirs
               nicht ſchicken? Nur leihweiſe, natürlich.\pend
           \pstart
           9.) Der kleine \textsc{Hugo\pwindex{Hofmannsthal, Hugo von 1874-02-01 – 1929-07-15@\textsc{Hofmannsthal, Hugo von} (1874-02-01 – 1929-07-15), \emph{Schriftsteller}|pw}} mag als Menſch charmant ſein, als Schriftſteller iſt er mir aufs Höchſte
               unſympathiſch, und er ſteht mir fern, als hätte ich ihn nie gekannt.\pend
           \pstart
           {\pb}10.) \label{K_L02774-7v}\edtext{\textsc{Bahr\pwindex{Bahr, Hermann 19.07.1863 – 15.01.1934@\textsc{Bahr, Hermann} (19.07.1863 – 15.01.1934), \emph{Schriftsteller, Kritiker}|pw}} erklärt}{\lemma{\textnormal{\emph{Bahr erklärt}}}\Cendnote{\textnormal{Siehe A. S.: \emph{Tagebuch}, 17. 4. 1896.
               }}}\label{K_L02774-7h}, Du ſeieſt ein großer Künſtler? – Was haſt Du nur in der letzten Zeit
               Schlechtes geſchrieben?\pend
           \pstart
           11.) Mit dieſer \strikeout{N\textcolor{gray}{×}} Nummer iſt in Deinem Brief die Köln\oindex{Koeln@\textbf{Köln}|pw}er
               Aufführung der »Liebelei\pwindex{Schnitzler, Arthur 15.05.1862 – 21.10.1931@\textsc{Schnitzler, Arthur} (15.05.1862 – 21.10.1931), \emph{Schriftsteller, Mediziner}!Liebelei. Schauspiel in drei Akten1895-10-09@\strich\emph{Liebelei. Schauspiel in drei Akten} {[}1895-10-09{]}|pw}« bezeichnet. Ich gehe
               zu 12 über:\pend
           \pstart
           12.) Freut mich von Herzen, daß Du mit Deinem neuen Stück\pwindex{Schnitzler, Arthur 15.05.1862 – 21.10.1931@\textsc{Schnitzler, Arthur} (15.05.1862 – 21.10.1931), \emph{Schriftsteller, Mediziner}!Freiwild. Schauspiel in 3 Akten1896@\strich\emph{Freiwild. Schauspiel in 3 Akten} {[}1896{]}|pwv} auf die rechte Bahn kommſt. Schreib’ mir nur bald, wie
                  \strikeout{es} es vorwarts rückt. Könnteſt {\pb}Du mir nicht das Manuſkript\pwindex{Schnitzler, Arthur 15.05.1862 – 21.10.1931@\textsc{Schnitzler, Arthur} (15.05.1862 – 21.10.1931), \emph{Schriftsteller, Mediziner}!Freiwild. Schauspiel in 3 Akten1896@\strich\emph{Freiwild. Schauspiel in 3 Akten} {[}1896{]}|pwv} ſchicken, wenn Dus fertig haſt?\pend
           \pstart
           13.) \textsc{Albert\pwindex{Albert, Henri 1869-11-16 – 1921-08-03@\textsc{Albert, Henri} (1869-11-16 – 1921-08-03), \emph{Journalist, Kritiker, Übersetzer}|pw}} ſehe ich kaum mehr. Er wird ein literariſcher Miſtbube (was er wohl ſtets war).
               Mich braucht er nicht mehr, und darum erklärt er, daß er ein \strikeout{Schriff}{ }Schriftſteller ſei und ich nur ein Journalist. Hat
               ganz Recht, der Mann\pwindex{Albert, Henri 1869-11-16 – 1921-08-03@\textsc{Albert, Henri} (1869-11-16 – 1921-08-03), \emph{Journalist, Kritiker, Übersetzer}|pwv}, – ich
               meine: das Publicum und auch die Standesgenoſſen denken genau ſo wie er. Was {\pb}Deine Manuſkripte anlangt, ſo reclamire ſie von ihm
               und laß’ ſie vielleicht von einem der jungen Leute, die Dein \strikeout{Stück\textcolor{gray}{e}\pwindex{Schnitzler, Arthur 15.05.1862 – 21.10.1931@\textsc{Schnitzler, Arthur} (15.05.1862 – 21.10.1931), \emph{Schriftsteller, Mediziner}!Liebelei. Schauspiel in drei Akten1895-10-09@\strich\emph{Liebelei. Schauspiel in drei Akten} {[}1895-10-09{]}|pwuv}}{ }Stück\pwindex{Schnitzler, Arthur 15.05.1862 – 21.10.1931@\textsc{Schnitzler, Arthur} (15.05.1862 – 21.10.1931), \emph{Schriftsteller, Mediziner}!Liebelei. Schauspiel in drei Akten1895-10-09@\strich\emph{Liebelei. Schauspiel in drei Akten} {[}1895-10-09{]}|pwuv} überſetzen
               wollen, zur \uline{Probe} übertragen, \strikeout{da\textcolor{gray}{m}} damit man ſieht, was ſie können.\pend
           \pstart
           14.) Von der \textsc{Andreas-Salome\pwindex{Andreas-Salome, Lou 12.02.1861 – 05.02.1937@\textsc{Andreas-Salomé, Lou} (12.02.1861 – 05.02.1937), \emph{Schriftstellerin}|pw}} höre ich nicht eine Zeile, noch ein Wort. Daß ſie \label{K_L02774-8v}\edtext{in Wien\oindex{Wien@\textbf{Wien}|pw} war}{\lemma{\textnormal{\emph{in Wien war}}}\Cendnote{\textnormal{Nach einer Reise nach St. Petersburg\oindex{Sankt Petersburg@\textbf{Sankt Petersburg}|pwk} im März 1895
                  lebte Lou Andreas-Salomé\pwindex{Andreas-Salome, Lou 12.02.1861 – 05.02.1937@\textsc{Andreas-Salomé, Lou} (12.02.1861 – 05.02.1937), \emph{Schriftstellerin}|pwk} mehrere Monate in
                     Wien\oindex{Wien@\textbf{Wien}|pwk}. Im Februar 1896 verließ sie die Stadt\oindex{Wien@\textbf{Wien}|pwkv} wieder, kehrte aber bereits im Mai zurück. Der »Stimmungswechſel« drückt sich auch
                  dadurch aus, dass sie in Schnitzlers\pwindex{Schnitzler, Arthur 15.05.1862 – 21.10.1931@\textsc{Schnitzler, Arthur} (15.05.1862 – 21.10.1931), \emph{Schriftsteller, Mediziner}|pwk}{ }\emph{Tagebuch}\pwindex{Schnitzler, Arthur 15.05.1862 – 21.10.1931@\textsc{Schnitzler, Arthur} (15.05.1862 – 21.10.1931), \emph{Schriftsteller, Mediziner}!Tagebuch1981 – 2000@\strich\emph{Tagebuch} {[}1981 – 2000{]}|pwk} am 25. 1. 1896 erwähnt wird und dann für ziemlich
                  genau zehn Jahre nicht mehr.}}}\label{K_L02774-8h}, erfahre ich erſt aus Deinem Briefe. Den
               plötzlichen Stimmungswechſel Euch gegenüber kann ich mir ſchwer {\pb}erklären. Oder doch: ſie\pwindex{Andreas-Salome, Lou 12.02.1861 – 05.02.1937@\textsc{Andreas-Salomé, Lou} (12.02.1861 – 05.02.1937), \emph{Schriftstellerin}|pwv} iſt eine ſehr launenhafte Frau. Sie
               braucht Abwechſlung in \strikeout{\textcolor{gray}{al}} ihrer Menſchen-Nahrung und zehrt nicht gern zweimal von denſelben. Sie hat mit
               Euch Alles gelebt, was ſie mit Euch leben konnte, – hat Euch Alles gegeben, was ſie
               Euch geben konnte. Daher wohl die beiderſeitige Erkältung. Feſthalten aus Moral, aus
               Treue, aus Freundſchaft {\pb}kennt ſie wohl kaum. \strikeout{Sie} Man vergißt bei ihr\pwindex{Andreas-Salome, Lou 12.02.1861 – 05.02.1937@\textsc{Andreas-Salomé, Lou} (12.02.1861 – 05.02.1937), \emph{Schriftstellerin}|pwv} immer, daß ſie eine Frau iſt, und ſie iſt doch eine.
               Solange ſie mit Einem Freund iſt, iſt ſie beſtändig – inſoweit hat ſie männlichen
               Character. Aber das Weibliche an ihr iſt, daß ſie ihre Beſtändigkeiten wechſelt.\pend
           \pstart
           15.) Dein Leben nicht intereſſant? Haha! Ich wünſchte nur, Du könnteſt vier Wochen
               das \strikeout{M\textcolor{gray}{e}} meinige leben. {\pb}Dann würde \strikeout{D\textcolor{gray}{i}} Dir Dein Leben wie ein Roman vorkommen, – wie ein ſchöner Traum. Das Unglück
               iſt nur, daß \strikeout{\textcolor{gray}{m}} wir das, was uns das Leben ſchuldig bleibt, nach den Anſprüchen berechnen, die
                  \uline{wir} an dasſelbe ſtellen, – während wir ſo rechnen
               ſollten: ſoviel gewährt es den Anderen, ſoviel mir. Dann würde faſt immer ein \textsc{Plus} herauskommen, und bei Dir ein ganz gehöriges.\pend
           \pstart
           {\pb}16.) Hier iſt eine »\label{K_L02774-9v}\edtext{Grabſchrift}{\lemma{\textnormal{\emph{Grabſchrift}}}\Cendnote{\textnormal{Vgl. A. S.: \emph{Tagebuch}, 6. 5. 1896.
               }}}\label{K_L02774-9h}« mitgetheilt in Deinem Briefe, deren Genuß mir leider nicht zugänglich iſt,
               da ein oder zwei wichtige Worte darin infolge einer unerhörten Vertauſchung von
               I-Punkten und U-Haken vollſtändig unleſerlich ſind – ſelbſt für Einen, der \strikeout{es in einem} es, wie ich, nach fünfjähriger Lectüre
               Deiner Briefe, zu einer hübſchen Fertigkeit im Hieroglyphen-Entziffern gebracht
               hat.\pend
           \pstart
           {\pb}17.) »\textsc{L’Aube\orgindex{Aube@L’Aube|pw}}« zahlt ſicher ſicher nichts, – da kannſt Du beruhigt ſein. Ich habe Deinen
                  \label{K_L02774-10v}\edtext{Namen genannt}{\lemma{\textnormal{\emph{Namen genannt}}}\Cendnote{\textnormal{keine Publikation von oder über Schnitzler\pwindex{Schnitzler, Arthur 15.05.1862 – 21.10.1931@\textsc{Schnitzler, Arthur} (15.05.1862 – 21.10.1931), \emph{Schriftsteller, Mediziner}|pwk} in \emph{L’Aube}\pwindex{?? Werk@Nicht ermittelte Verfasserinnen und Verfasser!Aube1896-04 – 1897-07@\emph{L’Aube} {[}1896-04 – 1897-07{]}|pwk} bekannt}}}\label{K_L02774-10h}, weil ich es mir zum Geſetz \strikeout{\textcolor{gray}{ma}} gemacht, jedem, der zu mir kommt und mich nach deutſcher Literatur frägt,
               zuerſt von Dir zu ſprechen. Schicke den Leuten\orgindex{Aube@L’Aube|pwv} irgend etwas Altes, was ſchon gedruckt war und wofür
               Du ſchon gezahlt worden biſt.\pend
           \pstart
           {\pb}18.) \textsc{Lalo\pwindex{Lalo, Pierre 1866-09-06 – 1943-06-09@\textsc{Lalo, Pierre} (1866-09-06 – 1943-06-09), \emph{Kritiker}|pw}} will eine Arbeit über \label{K_L02774-11v}\edtext{»\textsc{Nietzsches\pwindex{Nietzsche, Friedrich 15.10.1844 – 25.08.1900@\textsc{Nietzsche, Friedrich} (15.10.1844 – 25.08.1900), \emph{Schriftsteller, Philosoph}|pw}{ }} Einfluß auf das moderne
               deutſche Geiſtesleben«}{\lemma{\textnormal{\emph{»Nietzsches … Geiſtesleben«}}}\Cendnote{\textnormal{nicht
                  bekannt}}}\label{K_L02774-11h} machen. Welches Buch, außer dem der \textsc{Andreas-Salome\pwindex{Andreas-Salome, Lou 12.02.1861 – 05.02.1937@\textsc{Andreas-Salomé, Lou} (12.02.1861 – 05.02.1937), \emph{Schriftstellerin}|pw}\pwindex{Andreas-Salome, Lou 12.02.1861 – 05.02.1937@\textsc{Andreas-Salomé, Lou} (12.02.1861 – 05.02.1937), \emph{Schriftstellerin}!Friedrich Nietzsche in seinen Werken1894@\strich\emph{Friedrich Nietzsche in seinen Werken} {[}1894{]}|pwv}}, kann man ihm zur Lectüre empfehlen? Bitte, antworte mir – ausnahmsweiſe einmal
               – auf dieſe Frage.\pend
           \pstart
           19.) Schreib’ bald!\pend
           \pstart
           20.) Sei von ganzem Herzen gegrüßt!\pend
           \pstart
           Dein treuer {\\[\baselineskip]}\spacefill\mbox{Paul Goldmnn.}\pend
           \leftskip=0em{}\pstart
           \noindent{}{\pb}\textsc{P. S.}{ }Morgen ſende ich Dir \label{K_L02774-12v}\edtext{»\textsc{Aphrodite\pwindex{Louÿs, Pierre 10.12.1870 – 04.06.1925@\textsc{Louÿs, Pierre} (10.12.1870 – 04.06.1925), \emph{Schriftsteller}!Aphrodite. Mœurs antiques1896@\strich\emph{Aphrodite. Mœurs antiques} {[}1896{]}|pw}}« von \textsc{Pierre Louÿs\pwindex{Louÿs, Pierre 10.12.1870 – 04.06.1925@\textsc{Louÿs, Pierre} (10.12.1870 – 04.06.1925), \emph{Schriftsteller}|pw}}}{\lemma{\textnormal{\emph{»Aphrodite« … Louÿs}}}\Cendnote{\textnormal{Siehe A. S.: \emph{Lektüren}, Frankreich.
                  }}}\label{K_L02774-12h}. Schreib’ mir, wie Dirs gefällt, Aber zeig’ das Buch\pwindex{Louÿs, Pierre 10.12.1870 – 04.06.1925@\textsc{Louÿs, Pierre} (10.12.1870 – 04.06.1925), \emph{Schriftsteller}!Aphrodite. Mœurs antiques1896@\strich\emph{Aphrodite. Mœurs antiques} {[}1896{]}|pwv} weder \textsc{Bahr\pwindex{Bahr, Hermann 19.07.1863 – 15.01.1934@\textsc{Bahr, Hermann} (19.07.1863 – 15.01.1934), \emph{Schriftsteller, Kritiker}|pw}} noch einem von den \textsc{Bahr\pwindex{Bahr, Hermann 19.07.1863 – 15.01.1934@\textsc{Bahr, Hermann} (19.07.1863 – 15.01.1934), \emph{Schriftsteller, Kritiker}|pw}ischen}!\pend
           \pstart
           Der \label{K_L02774-13v}\edtext{Wien\oindex{Wien@\textbf{Wien}|pw}er »\textsc{Figaro\pwindex{?? Werk@Nicht ermittelte Verfasserinnen und Verfasser!Figaro. Humoristisches Wochenblatt1857 – 1919@\emph{Figaro. Humoristisches Wochenblatt} {[}1857 – 1919{]}|pw}}«}{\lemma{\textnormal{\emph{Wiener »Figaro«}}}\Cendnote{\textnormal{Schnitzler\pwindex{Schnitzler, Arthur 15.05.1862 – 21.10.1931@\textsc{Schnitzler, Arthur} (15.05.1862 – 21.10.1931), \emph{Schriftsteller, Mediziner}|pwk} könnte Goldmann\pwindex{Goldmann, Paul 31.01.1865 – 25.09.1935@\textsc{Goldmann, Paul} (31.01.1865 – 25.09.1935), \emph{Schriftsteller, Journalist}|pwk} auf die Zeichnung\pwindex{Zasche, Theodor 18.10.1862 – 15.11.1922@\textsc{Zasche, Theodor} (18.10.1862 – 15.11.1922), \emph{Bildender Künstler, Karikaturist}!Unter Wiener Grisetten1896-04-25@\strich\emph{Unter Wiener Grisetten} {[}1896-04-25{]}|pwkv} »\emph{Unter Wiener
                        Grisetten}\pwindex{Zasche, Theodor 18.10.1862 – 15.11.1922@\textsc{Zasche, Theodor} (18.10.1862 – 15.11.1922), \emph{Bildender Künstler, Karikaturist}!Unter Wiener Grisetten1896-04-25@\strich\emph{Unter Wiener Grisetten} {[}1896-04-25{]}|pwk}« von Theodor Zasche\pwindex{Zasche, Theodor 18.10.1862 – 15.11.1922@\textsc{Zasche, Theodor} (18.10.1862 – 15.11.1922), \emph{Bildender Künstler, Karikaturist}|pwk}
                     hingewiesen haben. Darauf wird Schnitzler\pwindex{Schnitzler, Arthur 15.05.1862 – 21.10.1931@\textsc{Schnitzler, Arthur} (15.05.1862 – 21.10.1931), \emph{Schriftsteller, Mediziner}|pwk}
                     im Café Griensteidl\oindex{Cafe Griensteidl@\textbf{Café Griensteidl}|pwk} sitzend abgebildet. Vor
                     dem Fenster des Cafés stehen zwei Frauen – »Grisetten«, die
                     darüber sprechen, dass Schnitzler\pwindex{Schnitzler, Arthur 15.05.1862 – 21.10.1931@\textsc{Schnitzler, Arthur} (15.05.1862 – 21.10.1931), \emph{Schriftsteller, Mediziner}|pwk} berühmt
                     sei, weil er sie »abgeschrieben« (als Vorlage verwendet) habe. Theodor Zasche\pwindex{Zasche, Theodor 18.10.1862 – 15.11.1922@\textsc{Zasche, Theodor} (18.10.1862 – 15.11.1922), \emph{Bildender Künstler, Karikaturist}|pwk}: \emph{Unter Wiener Grisetten}\pwindex{Zasche, Theodor 18.10.1862 – 15.11.1922@\textsc{Zasche, Theodor} (18.10.1862 – 15.11.1922), \emph{Bildender Künstler, Karikaturist}!Unter Wiener Grisetten1896-04-25@\strich\emph{Unter Wiener Grisetten} {[}1896-04-25{]}|pwk}. In: \emph{Wiener Luft. Beiblatt zum »Figaro«}\pwindex{?? Werk@Nicht ermittelte Verfasserinnen und Verfasser!Figaro. Humoristisches Wochenblatt1857 – 1919@\emph{Figaro. Humoristisches Wochenblatt} {[}1857 – 1919{]}|pwk}, Jg. 40, Nr. 17,
                           25. 4. 1896, S. [1].}}}\label{K_L02774-13h} hat
                  mich ſehr gefreut. Wie iſt Einem eigentlich zumuthe, wenn man berühmt iſt?\pend
           
         
         \endnumbering\mylabel{h}\end{ledgroupsized}  \newcommand{\dateiname}{L02774}\newcommand{\titel}{Paul Goldmann an Arthur Schnitzler, 17. 5. [1896]}\newcommand{\editorInnen}{Martin Anton Müller und Laura Untner}%% latex-leseansicht-abspann.tex
%% Abspann für die Leseansicht.
%% Der Schalter \ifkorrekturansicht ist bereits durch den Vorspann gesetzt.

%% latex-abspann.tex
%% Gemeinsamer Abspann für Korrekturansicht und Leseansicht.
%% Setzt den Schalter \ifkorrekturansicht voraus (gesetzt in den
%% einbindenden Dateien latex-korrekturansicht-abspann.tex bzw.
%% latex-leseansicht-abspann.tex).
%% ---------------------------------------------------------------

\normalsize

% Das esempio-Environment wird nur in der Leseansicht benötigt
\ifkorrekturansicht\else
\newenvironment{esempio}[3]%
{
    \vspace{1.5ex}
    \rlap{\underline{#1}}
    \par
    \setlength{\parindent}{0cm}
    \nopagebreak
    \leftskip=#2cm
    \rightskip=#3cm
}
{
    \par
}
\fi

\doendnotes{C}
\bigskip
\vfill

\clearpage

\footnotesize

\ifkorrekturansicht
  \lohead{\textsc{register}}
\fi

% theindex-Environment neu definieren ohne reledmac
\makeatletter
\renewenvironment{theindex}{%
  \ifkorrekturansicht
    \section*{\indexname}%
  \else
    \subsubsection*{Index der erwähnten Entitäten}%
  \fi
  \setlength{\parindent}{0pt}%
  \setlength{\parskip}{0pt plus 0.3pt}%
  \let\item\@idxitem
}{%
  \ifkorrekturansicht\clearpage\fi
}
\makeatother

\IfFileExists{\jobname-pw.ind}{\input{\jobname-pw.ind}}{}

% Quellenangabe nur in der Leseansicht
\ifkorrekturansicht\else
% Fallback-Definitionen, falls die .tex-Datei \titel etc. nicht gesetzt hat
\providecommand{\titel}{}
\providecommand{\editorInnen}{}
\providecommand{\dateiname}{\jobname}

\vspace{3cm}

\vfill

\footnotesize
\textsc{Quelle}: \titel. Herausgegeben von {\editorInnen}. In: \emph{Arthur Schnitzler: Briefwechsel mit Autorinnen und Autoren}.
 Digitale Edition, https://schnitzler-briefe.acdh.oeaw.ac.at/{\dateiname}.html (Stand \today)
\fi

\end{document}


      