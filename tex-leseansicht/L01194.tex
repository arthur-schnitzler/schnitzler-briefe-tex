%% latex-leseansicht-vorspann.tex
%% Vorspann für die Leseansicht.
%% Lädt die gemeinsame Datei latex-vorspann.tex mit nicht gesetztem Schalter.

\newif\ifkorrekturansicht
\korrekturansichtfalse

\input{../tex-inputs/latex-vorspann}


\section[Richard Dehmel an Arthur Schnitzler, 1. 1. 1902]{L01194 Richard Dehmel an Arthur Schnitzler, 1. 1. 1902}
\nopagebreak\mylabel{L01194v}
\rehead{ }\normalsize\beginnumbering\briefempfaengerindex{Schnitzler, Arthur@\textsc{Schnitzler, Arthur}!zzzDehmel, Richard@\emph{von Richard Dehmel}!1902-01-013@{1. 1. 1902}|(be}
\toendnotes[C]{\smallbreak\pagebreak[2]}
\correspDesc{Versand  durch Richard Dehmel am 1. 1. 1902 in Blankenese
\newline{}Erhalt  durch Arthur Schnitzler im Zeitraum [2. 1. 1902
                  – 6. 1. 1902?] in Wien}\toendnotes[C]{\smallbreak}
\Standort{Hamburg, Staats- und Universitätsbibliothek, DA:Br:D:4173.}
\physDesc{Brief, 2 Blätter, 3 Seiten, 1183 Zeichen
\newline{}Handschrift: schwarze Tinte, lateinische Kurrent
\newline{}Schnitzler: mit rotem Buntstift eine Unterstreichung 
\newline{}Zusatz: Da dieses Korrespondenzstück im Nachlass Dehmels überliefert
                                 ist, dürfte es sich um eine Abschrift des tatsächlich versandten
                                 Briefes handeln }\toendnotes[C]{\smallbreak}
\pstart
           {\pb}\textcolor{gray}{\textbf{RD}}\hfill Blankenese \textsuperscript{b}/Hamburg\oindex{Blankenese@\textbf{Blankenese}, \emph{Ehemaliger Ort}|pw}, 1. 1. 2.\pend
           
\pstart
           Verehrter Herr Schnitzler!\pend
           \vspace{0.5em}
\pstart
           Ich danke Ihnen herzlich für Ihr Buch\pwindex{Schnitzler, Arthur 15.\,5.\,1862 Wien – 21.\,10.\,1931 ebd.@\textsc{Schnitzler, Arthur} (15.\,5.\,1862 Wien – 21.\,10.\,1931 ebd.), \emph{Schriftsteller, Mediziner}!Schleier der Beatrice. Schauspiel in fünf Akten@\strich\emph{Der Schleier der Beatrice. Schauspiel in fünf Akten}|pwuv}. In Ermangelung einer Gegengabe – (aber »aufgeschoben ist
               nicht aufgehoben«) – überfalle ich Sie gleich noch mit einer Bitte. Ich will in etwa
               2 Jahren ein Kinderbuch herausgeben:\pend
           
\pstart
           \centering{}Der Buntscheck\pwindex{Buntscheck. Ein Sammelbuch herzhafter Kunst für Ohr und Auge deutscher Kinder@\emph{Der Buntscheck. Ein Sammelbuch herzhafter Kunst für Ohr und Auge deutscher Kinder}|pw},{\\}ein Sammelbuch herzhafter
               Kunst für Ohr und Auge unsrer Kinder –\pend
           
\pstart
           {\pb}würden Sie mir dazu eine einfache kurze Geschichte
               beisteuern können? Sie brauchen durchaus nicht \uline{vom}
               Kinde zu handeln, jeder andre »Stoff« ist mir sogar lieber; nur soll eben Alles ganz
               vom Kinde \uline{aus} dargestellt, also ohne sentimental\substVorne{}\textsuperscript{e}\substDazwischen{}ische\substHinten{} oder ironische Sehnsucht nach dem »verlorenen Paradiese«. Auf das Mscrpt –
               (es darf aber noch nicht gedruckt sein und darf bis 1. Oktober 190\uline{5} auch nirgendwo anders veröffentlicht werden) – kann ich bis in den
                  September dies. Js. warten; länger {\pb}nicht
               aus illustrativen Gründen. Im übrigen hat der Verleger (Schafstein {\kaufmannsund} Co.\orgindex{Schafstein und Co.@Schafstein {\kaufmannsund}  Co.|pw} in Köln\oindex{Köln@\textbf{Köln}, \emph{Hauptstadt}|pw}) mir völlig freie Hand bewilligt, sodaß ich
               für die Urheberansprüche meiner Mitarbeiter in künstlerischer wie geschäftlicher
               Hinsicht nach Gebühr eintreten kann.\pend
           
\pstart
           Mit der Bitte um baldigen Bescheid und mit meinen besten Neujahrswünschen\pend
           
\pstart
           Ihr hochachtungsvoll ergebener{\\[\baselineskip]}\spacefill\mbox{R. Dehmel.}\pend
           \leftskip=0em{}\selectlanguage{ngerman}\endnumbering\briefempfaengerindex{Schnitzler, Arthur@\textsc{Schnitzler, Arthur}!zzzDehmel, Richard@\emph{von Richard Dehmel}!1902-01-013@{1. 1. 1902}|)be}\mylabel{L01194h}  \newcommand{\dateiname}{L01194}\newcommand{\titel}{Richard Dehmel an Arthur Schnitzler, 1. 1. 1902}\newcommand{\editorInnen}{Martin Anton Müller und Gerd-Hermann Susen}%% latex-leseansicht-abspann.tex
%% Abspann für die Leseansicht.
%% Der Schalter \ifkorrekturansicht ist bereits durch den Vorspann gesetzt.

%% latex-abspann.tex
%% Gemeinsamer Abspann für Korrekturansicht und Leseansicht.
%% Setzt den Schalter \ifkorrekturansicht voraus (gesetzt in den
%% einbindenden Dateien latex-korrekturansicht-abspann.tex bzw.
%% latex-leseansicht-abspann.tex).
%% ---------------------------------------------------------------

\normalsize

% Das esempio-Environment wird nur in der Leseansicht benötigt
\ifkorrekturansicht\else
\newenvironment{esempio}[3]%
{
    \vspace{1.5ex}
    \rlap{\underline{#1}}
    \par
    \setlength{\parindent}{0cm}
    \nopagebreak
    \leftskip=#2cm
    \rightskip=#3cm
}
{
    \par
}
\fi

\doendnotes{C}
\bigskip
\vfill

\clearpage

\footnotesize

\ifkorrekturansicht
  \lohead{\textsc{register}}
\fi

% theindex-Environment neu definieren ohne reledmac
\makeatletter
\renewenvironment{theindex}{%
  \ifkorrekturansicht
    \section*{\indexname}%
  \else
    \subsubsection*{Index der erwähnten Entitäten}%
  \fi
  \setlength{\parindent}{0pt}%
  \setlength{\parskip}{0pt plus 0.3pt}%
  \let\item\@idxitem
}{%
  \ifkorrekturansicht\clearpage\fi
}
\makeatother

\IfFileExists{\jobname-pw.ind}{\input{\jobname-pw.ind}}{}

% Quellenangabe nur in der Leseansicht
\ifkorrekturansicht\else
% Fallback-Definitionen, falls die .tex-Datei \titel etc. nicht gesetzt hat
\providecommand{\titel}{}
\providecommand{\editorInnen}{}
\providecommand{\dateiname}{\jobname}

\vspace{3cm}

\vfill

\footnotesize
\textsc{Quelle}: \titel. Herausgegeben von {\editorInnen}. In: \emph{Arthur Schnitzler: Briefwechsel mit Autorinnen und Autoren}.
 Digitale Edition, https://schnitzler-briefe.acdh.oeaw.ac.at/{\dateiname}.html (Stand \today)
\fi

\end{document}


