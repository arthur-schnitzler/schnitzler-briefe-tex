%% latex-leseansicht-vorspann.tex
%% Vorspann für die Leseansicht.
%% Lädt die gemeinsame Datei latex-vorspann.tex mit nicht gesetztem Schalter.

\newif\ifkorrekturansicht
\korrekturansichtfalse

\input{../tex-inputs/latex-vorspann}


\section[Arthur Schnitzler an Stefan Zweig, 4. 9. 1914]{L03777 Arthur Schnitzler an Stefan Zweig, 4. 9. 1914}
\nopagebreak\mylabel{L03777v}
\rehead{ }\normalsize\beginnumbering\briefempfaengerindex{Zweig, Stefan@\textsc{Zweig, Stefan}!zzzSchnitzler, Arthur@\emph{von Arthur Schnitzler}!1914-09-041@{4. 9. 1914}|(be}
\toendnotes[C]{\smallbreak\pagebreak[2]}
\correspDesc{Versand  durch Arthur Schnitzler am 4. 9. 1914 in Wien
\newline{}Übermittlung  am 5. 9. 1914 in Wien
\newline{}Erhalt  durch Stefan Zweig im Zeitraum [5. 9. 1914
                  – 8. 9. 1914?] in Wien}\toendnotes[C]{\smallbreak}
\Standort{Jerusalem, National Library of Israel, ARC. Ms. Var. 305 1 58 Stefan Zweig Collection.}
\physDesc{Bildpostkarte, 544 Zeichen
\newline{}Handschrift: schwarze Tinte, deutsche Kurrent
\newline{}Versand: Stempel: »\nobreak{}\oindex{XVIII., Währing@\textbf{XVIII., Währing}, \emph{Verwaltungsgebiet}|pwk}18/\textsubscript{1} Wien
                                       110, 5. IX. 14, 9\nobreak{}«.  
\newline{}Zusatz: Postkartenmotiv mit Olga\pwindex{Schnitzler, Olga 17.\,1.\,1882 Wien – 13.\,1.\,1970 Lugano@\textsc{Schnitzler, Olga} (17.\,1.\,1882 Wien – 13.\,1.\,1970 Lugano), \emph{Schauspielerin, Sängerin}|pw}
                                 und Heinrich\pwindex{Schnitzler, Heinrich 9.\,8.\,1902 Hinterbrühl – 12.\,7.\,1982 Wien@\textsc{Schnitzler, Heinrich} (9.\,8.\,1902 Hinterbrühl – 12.\,7.\,1982 Wien), \emph{Regisseur, Schauspieler}|pw} links vor dem
                                 Haus und Schnitzler und Lili\pwindex{Cappellini, Lili 13.\,9.\,1909 Wien – 26.\,7.\,1928 Venedig@\textsc{Cappellini, Lili} (13.\,9.\,1909 Wien – 26.\,7.\,1928 Venedig)|pw}
                                 auf dem Söller }\toendnotes[C]{\smallbreak}\pstart{}{\pb}Hrn \textsc{Dr. Stefan Zweig}\pend{}\pstart{}Wien VIII\oindex{VIII., Josefstadt@\textbf{VIII., Josefstadt}, \emph{Verwaltungsgebiet}|pw}\pend{}\pstart{}\textsc{Kochgasse 8}\oindex{Wien@\textbf{Wien}!VIII., Josefstadt@\textbf{VIII., Josefstadt}!Kochgasse 8@\textbf{Kochgasse 8}, \emph{Wohngebäude}|pw}.\pend{}{\bigskip}
\pstart
           \noindent{}\centering{}{\pb}\textcolor{gray}{\textbf{Wien, XVIII, Sternwartestr. 71}}.\oindex{Wien@\textbf{Wien}!XVIII., Währing@\textbf{XVIII., Währing}!Sternwartestraße 71@\textbf{Sternwartestraße 71}, \emph{Wohngebäude}|pw}\pend
           \vspace{1em}
\pstart
           \raggedleft{}{\pb}4. 9. 14\pend
           \vspace{0.5em}
\pstart
           lieber Herr Doctor Zweig, es iſt wohl anzunehmen, dſs Ihnen Unruh\pwindex{Unruh, Fritz von 10.\,5.\,1885 Koblenz – 28.\,11.\,1970 Diez@\textsc{Unruh, Fritz von} (10.\,5.\,1885 Koblenz – 28.\,11.\,1970 Diez), \emph{Schriftsteller}|pw}{ }ſchon direct geſchrieben hat – jedenfalls
               richt ich Ihnen gerne einen herzlichen \label{K_L03777-1v}\edtext{Gruſs an Sie}{\lemma{\textnormal{\emph{Gruss an Sie}}}\Cendnote{\textnormal{Fritz v. Unruh\pwindex{Unruh, Fritz von 10.\,5.\,1885 Koblenz – 28.\,11.\,1970 Diez@\textsc{Unruh, Fritz von} (10.\,5.\,1885 Koblenz – 28.\,11.\,1970 Diez), \emph{Schriftsteller}|pwk} schrieb am
                     13. 8. 1914 an Schnitzler: »In Eile, da ich auf Patrouille fort muss. Ich bitte
                     um herzliche Grüsse an Stefan Zweig\pwindex{Zweig, Stefan 28.\,11.\,1881 Wien – 23.\,2.\,1942 Petrópolis@\textsc{Zweig, Stefan} (28.\,11.\,1881 Wien – 23.\,2.\,1942 Petrópolis), \emph{Schriftsteller}|pw} und
                     Dr. Rosenbaum\pwindex{Rosenbaum, Richard 4.\,11.\,1867 Žikov – 25.\,6.\,1942 Konzentrationslager Theresienstadt@\textsc{Rosenbaum, Richard} (4.\,11.\,1867 Žikov – 25.\,6.\,1942 Konzentrationslager Theresienstadt), \emph{Dramaturg, Verleger}|pw}. Ich werde für die lieben
                     Bundesbrüder gern mein Leben geben.« (Ulrich K. Goldsmith: \emph{Der Briefwechsel Fritz von Unruhs mit Arthur Schnitzler}.
                     In: \emph{Modern Austrian Literature}, Jg. 10,
                     Nr. 3/4, 1977, S. 95.)}}}\label{K_L03777-1} aus, der{ }ſich in
               einer Karte an mich befand, die hier (wir kamen vorgeſtern an) für mich aufbewahrt
               lagen und füge{ }ſchönſte Grüße von mir und auch von {\pb}meiner
                  Gattin\pwindex{Schnitzler, Olga 17.\,1.\,1882 Wien – 13.\,1.\,1970 Lugano@\textsc{Schnitzler, Olga} (17.\,1.\,1882 Wien – 13.\,1.\,1970 Lugano), \emph{Schauspielerin, Sängerin}|pwv} bei. Hoffentlich{ }ſehn wir Sie bald! Wollen Sie am \label{K_L03777-2v}\edtext{Montag mit uns}{\lemma{\textnormal{\emph{Montag mit uns}}}\Cendnote{\textnormal{Siehe A. S.: \emph{Tagebuch}, 7. 9. 1914.}}}\label{K_L03777-2} u Rosenbaum’s\pwindex{Rosenbaum, Richard 4.\,11.\,1867 Žikov – 25.\,6.\,1942 Konzentrationslager Theresienstadt@\textsc{Rosenbaum, Richard} (4.\,11.\,1867 Žikov – 25.\,6.\,1942 Konzentrationslager Theresienstadt), \emph{Dramaturg, Verleger}|pw}\pwindex{Rosenbaum, Kory Elisabeth 26.\,6.\,1868 Berlin – 28.\,1.\,1930 Wien@\textsc{Rosenbaum, Kory Elisabeth} (26.\,6.\,1868 Berlin – 28.\,1.\,1930 Wien), \emph{Schriftstellerin}|pw} im Freien nachtmahlen? So
               erwarten \introOben{}wir\introOben{} Sie bei uns nach 6 Uhr\pend
           
\pstart
           Wir würden uns{ }ſehr freuen\pend
           
\pstart
           Ihr{\\[\baselineskip]}\spacefill\mbox{Arthur Schnitzler}\pend
           \leftskip=0em{}\selectlanguage{ngerman}\endnumbering\briefempfaengerindex{Zweig, Stefan@\textsc{Zweig, Stefan}!zzzSchnitzler, Arthur@\emph{von Arthur Schnitzler}!1914-09-041@{4. 9. 1914}|)be}\mylabel{L03777h}  \newcommand{\dateiname}{L03777}\newcommand{\titel}{Arthur Schnitzler an Stefan Zweig, 4. 9. 1914}\newcommand{\editorInnen}{Selma Jahnke und Martin Anton Müller}%% latex-leseansicht-abspann.tex
%% Abspann für die Leseansicht.
%% Der Schalter \ifkorrekturansicht ist bereits durch den Vorspann gesetzt.

%% latex-abspann.tex
%% Gemeinsamer Abspann für Korrekturansicht und Leseansicht.
%% Setzt den Schalter \ifkorrekturansicht voraus (gesetzt in den
%% einbindenden Dateien latex-korrekturansicht-abspann.tex bzw.
%% latex-leseansicht-abspann.tex).
%% ---------------------------------------------------------------

\normalsize

% Das esempio-Environment wird nur in der Leseansicht benötigt
\ifkorrekturansicht\else
\newenvironment{esempio}[3]%
{
    \vspace{1.5ex}
    \rlap{\underline{#1}}
    \par
    \setlength{\parindent}{0cm}
    \nopagebreak
    \leftskip=#2cm
    \rightskip=#3cm
}
{
    \par
}
\fi

\doendnotes{C}
\bigskip
\vfill

\clearpage

\footnotesize

\ifkorrekturansicht
  \lohead{\textsc{register}}
\fi

% theindex-Environment neu definieren ohne reledmac
\makeatletter
\renewenvironment{theindex}{%
  \ifkorrekturansicht
    \section*{\indexname}%
  \else
    \subsubsection*{Index der erwähnten Entitäten}%
  \fi
  \setlength{\parindent}{0pt}%
  \setlength{\parskip}{0pt plus 0.3pt}%
  \let\item\@idxitem
}{%
  \ifkorrekturansicht\clearpage\fi
}
\makeatother

\IfFileExists{\jobname-pw.ind}{\input{\jobname-pw.ind}}{}

% Quellenangabe nur in der Leseansicht
\ifkorrekturansicht\else
% Fallback-Definitionen, falls die .tex-Datei \titel etc. nicht gesetzt hat
\providecommand{\titel}{}
\providecommand{\editorInnen}{}
\providecommand{\dateiname}{\jobname}

\vspace{3cm}

\vfill

\footnotesize
\textsc{Quelle}: \titel. Herausgegeben von {\editorInnen}. In: \emph{Arthur Schnitzler: Briefwechsel mit Autorinnen und Autoren}.
 Digitale Edition, https://schnitzler-briefe.acdh.oeaw.ac.at/{\dateiname}.html (Stand \today)
\fi

\end{document}


