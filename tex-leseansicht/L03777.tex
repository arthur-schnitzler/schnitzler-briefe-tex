%% latex-korrekturansicht-vorspann.tex
%% Vorspann für die Korrekturansicht.
%% Lädt die gemeinsame Datei latex-vorspann.tex mit gesetztem Schalter.

\newif\ifkorrekturansicht
\korrekturansichttrue

\input{../tex-inputs/latex-vorspann}


\section[Arthur Schnitzler an Stefan Zweig, 4. 9. 1914]{L03777 Arthur Schnitzler an Stefan Zweig, 4. 9. 1914}
\nopagebreak\mylabel{L03777v}
\rehead{ }\normalsize\beginnumbering\briefempfaengerindex{Zweig, Stefan@\textsc{Zweig, Stefan}!zzzSchnitzler, Arthur@\emph{von Arthur Schnitzler}!1914-09-041@{4. 9. 1914}|(be}
\toendnotes[C]{\smallbreak\pagebreak[2]}\Standort{Jerusalem, National Library of Israel, ARC. Ms. Var. 305 1 58 Stefan Zweig Collection.}
\physDesc{Bildpostkarte, 1 Blatt, 2 Seiten, 544 Zeichen
\newline{}Handschrift: schwarze Tinte, deutsche Kurrent
\newline{}Versand: Stempel: »\nobreak{}\oindex{XVIII., Waehring@\textbf{XVIII., Währing}, \emph{A.ADM3}|pwk}18/\textsubscript{1} Wien
                                       110, 5. IX. 14, 9\nobreak{}«.  }\toendnotes[C]{\smallbreak}\pstart{}{\pb}Hrn Dr. Stefan Zweig\pend{}\pstart{}Wien VIII\oindex{VIII., Josefstadt@\textbf{VIII., Josefstadt}, \emph{A.ADM3}|pw}\pend{}\pstart{}Kochgasse 8\oindex{Kochgasse 8@\textbf{Kochgasse 8}, \emph{Wohngebäude (K.WHS)}|pw}.\pend{}{\bigskip}
\pstart
           \noindent{}\centering{}{\pb}\textcolor{gray}{\textbf{Wien, XVIII, Sternwartestr. 71}}\oindex{Sternwartestrasse 71@\textbf{Sternwartestraße 71}, \emph{Wohngebäude (K.WHS)}|pw}\pend
           \vspace{1em}
\pstart
           \raggedleft{}{\pb}4. 9. 14\pend
           \vspace{0.5em}
\pstart
           lieber Herr Doctor Zweig, es iſt wohl anzunehmen, dſs Ihnen Unruh\pwindex{Unruh, Fritz von 10.05.1885 – 28.11.1970@\textsc{Unruh, Fritz von} (10.05.1885 – 28.11.1970), \emph{Schriftsteller/Schriftstellerin}|pw} ſchon direct geſchrieben hat – jedenfalls
               richt ich Ihnen gerne einen herzlichen \label{K_L03777-1v}\edtext{Gruſs an Sie}{\lemma{\textnormal{\emph{Gruſs an Sie}}}\Cendnote{\textnormal{Fritz v. Unruh\pwindex{Unruh, Fritz von 10.05.1885 – 28.11.1970@\textsc{Unruh, Fritz von} (10.05.1885 – 28.11.1970), \emph{Schriftsteller/Schriftstellerin}|pwk} schrieb am
                     13. 8. 1914 an Schnitzler: »In Eile, da ich auf Patrouille fort muss. Ich bitte
                     um herzliche Grüsse an Stefan Zweig\pwindex{Zweig, Stefan 28.11.1881 – 23.02.1942@\textsc{Zweig, Stefan} (28.11.1881 – 23.02.1942), \emph{Schriftsteller/Schriftstellerin}|pw} und
                     Dr. Rosenbaum\pwindex{Rosenbaum, Richard 04.11.1867 – 25.06.1942@\textsc{Rosenbaum, Richard} (04.11.1867 – 25.06.1942), \emph{Dramaturg/Dramaturgin, Verleger/Verlegerin}|pw}. Ich werde für die lieben
                     Bundesbrüder gern mein Leben geben.« (Ulrich K. Goldsmith: \emph{Der Briefwechsel Fritz von Unruhs mit Arthur Schnitzler}.
                     In: \emph{Modern Austrian Literature}, Jg. 10,
                     Nr. 3/4, 1977, S. 95.)}}}\label{K_L03777-1} aus, der ſich in
               einer Karte an mich befand, die hier (wir kamen vorgeſtern an) für mich aufbewahrt
               lagen und füge ſchönſte Grüße von mir und auch von {\pb}meiner
                  Gattin\pwindex{Schnitzler, Olga 17.01.1882 – 13.01.1970@\textsc{Schnitzler, Olga} (17.01.1882 – 13.01.1970), \emph{Schauspieler/Schauspielerin, Sänger/Sängerin}|pwv} bei. Hoffentlich
               ſehn wir Sie bald! Wollen Sie am \label{K_L03777-2v}\edtext{Montag mit uns}{\lemma{\textnormal{\emph{Montag mit uns}}}\Cendnote{\textnormal{Siehe A. S.: \emph{Tagebuch}, 7. 9. 1914.}}}\label{K_L03777-2} u Rosenbaum’s\pwindex{Rosenbaum, Richard 04.11.1867 – 25.06.1942@\textsc{Rosenbaum, Richard} (04.11.1867 – 25.06.1942), \emph{Dramaturg/Dramaturgin, Verleger/Verlegerin}|pw}\pwindex{Rosenbaum, Kory Elisabeth 26.06.1868 – 28.01.1930@\textsc{Rosenbaum, Kory Elisabeth} (26.06.1868 – 28.01.1930), \emph{Schriftsteller/Schriftstellerin}|pw} im Freien nachtmahlen? So
               erwarten \introOben{}wir\introOben{} Sie bei uns nach 6 Uhr\pend
           
\pstart
           Wir würden uns ſehr freuen\pend
           
\pstart
           Ihr{\\[\baselineskip]}\spacefill\mbox{Arthur Schnitzler}\pend
           \leftskip=0em{}\selectlanguage{ngerman}\endnumbering\briefempfaengerindex{Zweig, Stefan@\textsc{Zweig, Stefan}!zzzSchnitzler, Arthur@\emph{von Arthur Schnitzler}!1914-09-041@{4. 9. 1914}|)be}\mylabel{L03777h}
\begin{anhang}
\end{anhang}\normalsize

\doendnotes{C}
\bigskip
\vfill

\clearpage

\footnotesize

\lohead{\textsc{register}}

% Definiere theindex-Environment komplett neu ohne reledmac
\makeatletter
\renewenvironment{theindex}{%
  \section*{\indexname}%
  \setlength{\parindent}{0pt}%
  \setlength{\parskip}{0pt plus 0.3pt}%
  \let\item\@idxitem
}{%
  \clearpage
}
\makeatother

\IfFileExists{\jobname-pw.ind}{\input{\jobname-pw.ind}}{}

\end{document}

      