\input{../tex-inputs/latex-pdf-vorspann}
\begin{center}
            \textcolor{red}{ENTWURF. ENTZIFFERUNG NOCH NICHT KORREKTURGELESEN}
                      \end{center}
            
               \section[Paul Goldmann an Arthur Schnitzler, 24. 6. {[}1892{]}]{ Paul Goldmann an Arthur Schnitzler, 24. 6. {[}1892{]}}\nopagebreak\mylabel{v}\rehead{ }\begin{ledgroupsized}[t]{13cm}\normalsize\beginnumbering\briefempfaengerindex{Schnitzler, Arthur@\textsc{Schnitzler, Arthur}!zzzGoldmann, Paul@\emph{von Paul Goldmann}!1892-06-241@{24. 6. {[}1892{]}}|(be} \toendnotes[C]{\smallbreak\pagebreak[2]} \Standort{DLA, A:Schnitzler, HS.NZ85.1.3163.}
\physDesc{Brief, 1 Blatt, 4 Seiten
\newline{}Handschrift: schwarze Tinte, deutsche Kurrent
\newline{}Schnitzler: 1) mit rotem Buntstift eine Unterstreichung 2) mit Bleistift das Jahr »92« vermerkt}\toendnotes[C]{\smallbreak}\pstart
           \noindent{}{\pb}\textcolor{gray}{\textbf{Frankfurter Zeitung\orgindex{Frankfurter Zeitung@Frankfurter Zeitung|pw}.}}\pend
           \pstart
           \textcolor{gray}{\textbf{(Gazette de Francfort\orgindex{Frankfurter Zeitung@Frankfurter Zeitung|pw}.)}}\pend
           \pstart
           \textcolor{gray}{\textbf{\begin{otherlanguage}{french}Directeur\end{otherlanguage}: \textbf{M. L. Sonnemann\pwindex{Sonnemann, Leopold 1831-10-29 – 1909-10-30@\textsc{Sonnemann, Leopold} (1831-10-29 – 1909-10-30), \emph{Journalist, Herausgeber}|pw}}.}}\hfill \textsc{Paris\oindex{Paris@\textbf{Paris}|pw}}, 24. Juni.\pend
           \pstart
           \textcolor{gray}{\textbf{\begin{otherlanguage}{french}Journal politique, financier,\end{otherlanguage}}}\pend
           \pstart
           \textcolor{gray}{\textbf{\begin{otherlanguage}{french}commercial et litteraire.\end{otherlanguage}}}\pend
           \pstart
           \textcolor{gray}{\textbf{\begin{otherlanguage}{french}\textbf{Paraissant trois fois par jour}\end{otherlanguage}}}\pend
           \pstart
           \textcolor{gray}{\textbf{–}}\pend
           \pstart
           \textcolor{gray}{\textbf{\begin{otherlanguage}{french}\textbf{Bureaux à Paris\oindex{Paris@\textbf{Paris}|pw}:}\end{otherlanguage}}}\pend
           \pstart
           \textcolor{gray}{\textbf{\begin{otherlanguage}{french}\textbf{rue Richelieu 75.\oindex{rue Richelieu@\textbf{rue Richelieu}|pw}.}\end{otherlanguage}}}\pend
           \pstart
           \centering{}Mein lieber Arthur!\pend
           \pstart
           \noindent{}Ich habe \strikeout{heute}{ }\textsc{Herzl\pwindex{Herzl, Theodor 02.05.1860 – 03.07.1904@\textsc{Herzl, Theodor} (02.05.1860 – 03.07.1904), \emph{Schriftsteller, Journalist}|pw}}{ }\strikeout{h} dein Märchen\pwindex{Schnitzler, Arthur 15.05.1862 – 21.10.1931@\textsc{Schnitzler, Arthur} (15.05.1862 – 21.10.1931), \emph{Schriftsteller, Mediziner}!Maerchen. Schauspiel in drei Aufzuegen1891 – 1891@\strich\emph{Das Märchen. Schauspiel in drei Aufzügen} {[}1891 – 1891{]}|pw}
               gegeben und war heute bei ihm. Derſelbe ſprach ſich
               darüber in Worten der \label{K_L02698-1v}\edtext{Begeiſterung}{\lemma{\textnormal{\emph{Begeiſterung}}}\Cendnote{\textnormal{Am 28. 6. 1892 notierte Schnitzler\pwindex{Schnitzler, Arthur 15.05.1862 – 21.10.1931@\textsc{Schnitzler, Arthur} (15.05.1862 – 21.10.1931), \emph{Schriftsteller, Mediziner}|pwk} in seinem \emph{Tagebuch}\pwindex{Schnitzler, Arthur 15.05.1862 – 21.10.1931@\textsc{Schnitzler, Arthur} (15.05.1862 – 21.10.1931), \emph{Schriftsteller, Mediziner}!Tagebuch1981 – 2000@\strich\emph{Tagebuch} {[}1981 – 2000{]}|pwk}: »Herzl\pwindex{Herzl, Theodor 02.05.1860 – 03.07.1904@\textsc{Herzl, Theodor} (02.05.1860 – 03.07.1904), \emph{Schriftsteller, Journalist}|pw}’s begeistertes Urtheil übers Märchen\pwindex{Schnitzler, Arthur 15.05.1862 – 21.10.1931@\textsc{Schnitzler, Arthur} (15.05.1862 – 21.10.1931), \emph{Schriftsteller, Mediziner}!Maerchen. Schauspiel in drei Aufzuegen1891 – 1891@\strich\emph{Das Märchen. Schauspiel in drei Aufzügen} {[}1891 – 1891{]}|pw}, was mich lebhaft
                  freute.«}}}\label{K_L02698-1h} (wörtlich zu nehmen) aus. Er meinte, Du ſeieſt der einzige
               von uns allen Jungen – ihn inbegriffen – der ’was kann. Er meinte, du ſeieſt ein
               wahrer Dichter. Er meinte, das Ding\pwindex{Schnitzler, Arthur 15.05.1862 – 21.10.1931@\textsc{Schnitzler, Arthur} (15.05.1862 – 21.10.1931), \emph{Schriftsteller, Mediziner}!Maerchen. Schauspiel in drei Aufzuegen1891 – 1891@\strich\emph{Das Märchen. Schauspiel in drei Aufzügen} {[}1891 – 1891{]}|pwv} habe ihn ſo gepackt, daß er es in einem Zuge ausgeleſen. Er meinte,
               meinte und meinte, ich weiß nicht, was noch Alles Wunderſchönes für Dich, weil es der
               von {\pb}ſich ſelbſt eingenommenſte Menſch\pwindex{Herzl, Theodor 02.05.1860 – 03.07.1904@\textsc{Herzl, Theodor} (02.05.1860 – 03.07.1904), \emph{Schriftsteller, Journalist}|pwv}{ }Europa\oindex{Europa@\textbf{Europa}|pw}s meint. Er ſagte ſchließlich, daß er Dir
               ſofort \label{K_L02698-2v}\edtext{geſchrieben}{\lemma{\textnormal{\emph{geſchrieben}}}\Cendnote{\textnormal{Theodor Herzl\pwindex{Herzl, Theodor 02.05.1860 – 03.07.1904@\textsc{Herzl, Theodor} (02.05.1860 – 03.07.1904), \emph{Schriftsteller, Journalist}|pwk} schrieb erst am
                     29. 7. 1892 an Schnitzler (was dieser am 4. 8. 1892 im \emph{Tagebuch}\pwindex{Schnitzler, Arthur 15.05.1862 – 21.10.1931@\textsc{Schnitzler, Arthur} (15.05.1862 – 21.10.1931), \emph{Schriftsteller, Mediziner}!Tagebuch1981 – 2000@\strich\emph{Tagebuch} {[}1981 – 2000{]}|pwk} festhielt). Siehe Theodor Herzl\pwindex{Herzl, Theodor 02.05.1860 – 03.07.1904@\textsc{Herzl, Theodor} (02.05.1860 – 03.07.1904), \emph{Schriftsteller, Journalist}|pwk}: \emph{Briefe und Tagebücher}. Hg.
                     Alex Bein, Hermann Greive, Moshe Schaerf und Julius H. Schoeps. Bd. 1.:
                        \emph{Briefe und autobiographische Notizen. 1866–1895}.
                     Bearbeitet von Johannes Wachten. In Zusammenarbeit mit Chaya Harel, Daisy Tycho
                     und Manfred Winkler. Berlin,
                     Frankfurt am Main, Wien: \emph{Ullstein}\orgindex{Ullstein Verlag@Ullstein Verlag|pwk}/\emph{Propyläen}\orgindex{Propylaeen Verlag@Propyläen Verlag|pwk}{ }1983,
                     S. 498–502.}}}\label{K_L02698-2h} hätte, wenn er nicht gefürchtet hätte – \textsc{pardon}, ich
               referire wörtlich – Du ſeieſt ein Wien\oindex{Wien@\textbf{Wien}|pw}er Jüdel und
               würdeſt Dir \label{K_L02698-3v}\edtext{\textsc{parchanische}}{\lemma{\textnormal{\emph{parchanische}}}\Cendnote{\textnormal{unklar; es könnte vom jiddischen Wort
                     »parve« herrühren, und »nicht koscher«
                  bedeuten; es könnte das jiddische oder tschechische Wort für »Bastard« gemeint
                  sein.}}}\label{K_L02698-3h} Gedanken darüber machen.\pend
           \pstart
           Ich gratulire Dir herzlich zu dieſem ſchönen Erfolge Deines Talentes.\pend
           \pstart
           Das iſt das einzige Dich intereſſirende, was ich ſeit langer Zeit zu berichten
               finde.\pend
           \pstart
           Über mich laß’ mich ſchweigen. Ich verfalle und verrohe, Paris\oindex{Paris@\textbf{Paris}|pw} iſt mir widerlich, meine Stellung entſetzlich, das
               Heimweh nach Wien\oindex{Wien@\textbf{Wien}|pw}, nach Dir und all’ {\pb}den lieben Menſchen verzehrt mich. Ich bin einſam,
               zertreten und lieblos. Die Freundſchaft habe ich auch verloren, wie Du weißt. Durch
               meine Schuld, jawohl. Ich kann mich nicht mehr dazu aufſchwingen, dir ſo zu
               ſchreiben, wie ich Dir es ſchuldig wäre. Ich bin ſchon zu tief. Und ich denke, es iſt
               beſſer; ich laſſe mich langſam in die Vergeſſenheit herunterſinken.\pend
           \pstart
           Ich grüße \textsc{Richard\pwindex{Beer-Hofmann, Richard 11.07.1866 – 26.09.1945@\textsc{Beer-Hofmann, Richard} (11.07.1866 – 26.09.1945), \emph{Schriftsteller}|pw}} und \textsc{Loris}\pwindex{Hofmannsthal, Hugo von 01.02.1874 – 15.07.1929@\textsc{Hofmannsthal, Hugo von} (01.02.1874 – 15.07.1929), \emph{Schriftsteller}|pw} und
               umarme Dich von Herzen\pend
           \pstart
           Dein {\\[\baselineskip]}treuer {\\[\baselineskip]}\spacefill\mbox{Paul Goldmann.}\pend
           \leftskip=0em{}\pstart
           \noindent{}{\pb}Es ſei denn, daß Du ein Mittel wüßteſt, wie ich
                  Dich im Auguſt, wo ich wahrſcheinlich kurzen Urlaub
                  bekomme, \label{K_L02698-4v}\edtext{ſehen kann}{\lemma{\textnormal{\emph{ſehen kann}}}\Cendnote{\textnormal{Das nächste Wiedersehen fand am 17. 9. 1893
                     statt.}}}\label{K_L02698-4h}. Aber nach \textsc{Wien\oindex{Wien@\textbf{Wien}|pw}} komme ich nicht, weil ich nicht ein zweites Mal die Kraft fände, mich
                  loszureißen.\pend
           \pstart
           Meine einzige Freude iſt \textsc{Arthur Klein\pwindex{Klein, Arthur 27.11.1868 – 28.07.1943@\textsc{Klein, Arthur} (27.11.1868 – 28.07.1943)|pw}}. \textsc{Leopold Spitzer\pwindex{Spitzer, Leopold 1865 – 1913-06-28@\textsc{Spitzer, Leopold} (1865 – 1913-06-28), \emph{Journalist}|pw}}, der eine widerlich gemeine \label{K_L02698-5v}\edtext{Ladenſchwung}{\lemma{\textnormal{\emph{Ladenſchwung}}}\Cendnote{\textnormal{abwertende
                     Bezeichnung für einen Ladendiener oder Ladenjungen}}}\label{K_L02698-5h}-Seele iſt, habe ich
                  vor 14 Tagen geohrfeigt, was mich um ein Haar um meine Stellung gebracht hätte und
                  vielleicht noch bringt.\pend
           \endnumbering\briefempfaengerindex{Schnitzler, Arthur@\textsc{Schnitzler, Arthur}!zzzGoldmann, Paul@\emph{von Paul Goldmann}!1892-06-241@{24. 6. {[}1892{]}}|)be}\mylabel{h}\end{ledgroupsized}  \newcommand{\dateiname}{L02698}\newcommand{\titel}{Paul Goldmann an Arthur Schnitzler, 24. 6. [1892]}\newcommand{\editorInnen}{Martin Anton Müller und Laura Untner}\input{../tex-inputs/latex-pdf-abspann}
      