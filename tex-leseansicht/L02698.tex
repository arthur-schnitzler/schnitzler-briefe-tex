%% latex-leseansicht-vorspann.tex
%% Vorspann für die Leseansicht.
%% Lädt die gemeinsame Datei latex-vorspann.tex mit nicht gesetztem Schalter.

\newif\ifkorrekturansicht
\korrekturansichtfalse

\input{../tex-inputs/latex-vorspann}


\section[Paul Goldmann an Arthur Schnitzler, 24. 6. {[}1892{]}]{L02698 Paul Goldmann an Arthur Schnitzler, 24. 6. [1892]}
\nopagebreak\mylabel{L02698v}
\rehead{ }\normalsize\beginnumbering\briefempfaengerindex{Schnitzler, Arthur@\textsc{Schnitzler, Arthur}!zzzGoldmann, Paul@\emph{von Paul Goldmann}!1892-06-241@{24. 6. [1892]}|(be}
\toendnotes[C]{\smallbreak\pagebreak[2]}
\correspDesc{Versand  durch Paul Goldmann am 24. 6. [1892] in Paris
\newline{}Erhalt  durch Arthur Schnitzler am 28. 6. 1892 in Wien}\toendnotes[C]{\smallbreak}
\Standort{DLA, A:Schnitzler, HS.NZ85.1.3163.}
\physDesc{Brief, 1 Blatt, 4 Seiten, 1894 Zeichen
\newline{}Handschrift: schwarze Tinte, deutsche Kurrent
\newline{}Schnitzler: 1) mit rotem Buntstift eine Unterstreichung  2) mit Bleistift das Jahr »92« vermerkt}\toendnotes[C]{\smallbreak}
\pstart
           {\pb}\textcolor{gray}{\textbf{Frankfurter Zeitung\orgindex{Frankfurter Zeitung@Frankfurter Zeitung|pw}.}}\pend
           
\pstart
           \textcolor{gray}{\textbf{(Gazette de
                     Francfort\orgindex{Frankfurter Zeitung@Frankfurter Zeitung|pw}.)}}\pend
           
\pstart
           \textcolor{gray}{\textbf{\begin{otherlanguage}{french}Directeur\end{otherlanguage}: \textbf{M. L. Sonnemann\pwindex{Sonnemann, Leopold 29.\,10.\,1831 Höchberg – 30.\,10.\,1909 Frankfurt am Main@\textsc{Sonnemann, Leopold} (29.\,10.\,1831 Höchberg – 30.\,10.\,1909 Frankfurt am Main), \emph{Journalist, Herausgeber}|pw}}.}}\hfill \textsc{Paris\oindex{Paris@\textbf{Paris}, \emph{Hauptstadt}|pw}}, 24. Juni.\pend
           
\pstart
           \textcolor{gray}{\textbf{\begin{otherlanguage}{french}Journal politique, financier,\end{otherlanguage}}}\pend
           
\pstart
           \textcolor{gray}{\textbf{\begin{otherlanguage}{french}commercial et litteraire.\end{otherlanguage}}}\pend
           
\pstart
           \textcolor{gray}{\textbf{\begin{otherlanguage}{french}\textbf{Paraissant trois fois par jour}\end{otherlanguage}}}\pend
           
\pstart
           \textcolor{gray}{\textbf{\begin{otherlanguage}{french}\textbf{Bureaux à Paris\oindex{Paris@\textbf{Paris}, \emph{Hauptstadt}|pw}:}\end{otherlanguage}}}\pend
           
\pstart
           \textcolor{gray}{\textbf{\begin{otherlanguage}{french}\textbf{rue Richelieu 75\oindex{rue Richelieu@\textbf{rue Richelieu}, \emph{Straße}|pw}.}\end{otherlanguage}}}\pend
           
\pstart\center{}Mein lieber Arthur!\pend\vspace{0.5em}
\pstart
           Ich habe \strikeout{heute}{ }\textsc{Herzl\pwindex{Herzl, Theodor 2.\,5.\,1860 Budapest – 3.\,7.\,1904 Edlach@\textsc{Herzl, Theodor} (2.\,5.\,1860 Budapest – 3.\,7.\,1904 Edlach), \emph{Schriftsteller, Journalist}|pw}}{ }\strikeout{h} dein Märchen\pwindex{Schnitzler, Arthur 15.\,5.\,1862 Wien – 21.\,10.\,1931 ebd.@\textsc{Schnitzler, Arthur} (15.\,5.\,1862 Wien – 21.\,10.\,1931 ebd.), \emph{Schriftsteller, Mediziner}!Märchen. Schauspiel in drei Aufzügen@\strich\emph{Das Märchen. Schauspiel in drei Aufzügen}|pw}
               gegeben und war heute bei ihm. Derſelbe{ }ſprach{ }ſich
               darüber in Worten der \label{K_L02698-1v}\edtext{Begeiſterung}{\lemma{\textnormal{\emph{Begeisterung}}}\Cendnote{\textnormal{Am 28. 6. 1892 notierte
                     Schnitzler in seinem \emph{Tagebuch}\pwindex{Schnitzler, Arthur 15.\,5.\,1862 Wien – 21.\,10.\,1931 ebd.@\textsc{Schnitzler, Arthur} (15.\,5.\,1862 Wien – 21.\,10.\,1931 ebd.), \emph{Schriftsteller, Mediziner}!Tagebuch@\strich\emph{Tagebuch}|pwk}: »Herzl\pwindex{Herzl, Theodor 2.\,5.\,1860 Budapest – 3.\,7.\,1904 Edlach@\textsc{Herzl, Theodor} (2.\,5.\,1860 Budapest – 3.\,7.\,1904 Edlach), \emph{Schriftsteller, Journalist}|pw}’s begeistertes Urtheil übers Märchen\pwindex{Schnitzler, Arthur 15.\,5.\,1862 Wien – 21.\,10.\,1931 ebd.@\textsc{Schnitzler, Arthur} (15.\,5.\,1862 Wien – 21.\,10.\,1931 ebd.), \emph{Schriftsteller, Mediziner}!Märchen. Schauspiel in drei Aufzügen@\strich\emph{Das Märchen. Schauspiel in drei Aufzügen}|pw}, was mich lebhaft
                  freute.«}}}\label{K_L02698-1} (wörtlich zu nehmen) aus. Er meinte, Du{ }ſeieſt der einzige
               von uns allen Jungen – ihn inbegriffen – der ’was kann. Er meinte, du{ }ſeieſt ein
               wahrer Dichter. Er meinte, das Ding\pwindex{Schnitzler, Arthur 15.\,5.\,1862 Wien – 21.\,10.\,1931 ebd.@\textsc{Schnitzler, Arthur} (15.\,5.\,1862 Wien – 21.\,10.\,1931 ebd.), \emph{Schriftsteller, Mediziner}!Märchen. Schauspiel in drei Aufzügen@\strich\emph{Das Märchen. Schauspiel in drei Aufzügen}|pwv} habe ihn{ }ſo gepackt, daß er es in einem Zuge ausgeleſen. Er meinte,
               meinte und meinte, ich weiß nicht was noch Alles Wunderſchönes für Dich, weil es der
               von {\pb}ſich{ }ſelbſt eingenommenſte Menſch\pwindex{Herzl, Theodor 2.\,5.\,1860 Budapest – 3.\,7.\,1904 Edlach@\textsc{Herzl, Theodor} (2.\,5.\,1860 Budapest – 3.\,7.\,1904 Edlach), \emph{Schriftsteller, Journalist}|pwv}{ }Europas\oindex{Europa@\textbf{Europa}|pw} meint. Er{ }ſagte{ }ſchließlich, daß er Dir{ }ſofort \label{K_L02698-2v}\edtext{geſchrieben}{\lemma{\textnormal{\emph{geschrieben}}}\Cendnote{\textnormal{Schnitzler
                  nahm den Kontakt auf, siehe XXXX Auszeichnungsfehler: Dokument L03900 nicht gefunden.}}}\label{K_L02698-2} hätte, wenn er nicht gefürchtet hätte – \textsc{pardon}, ich
               referire wörtlich – Du{ }ſeieſt ein Wien\oindex{Wien@\textbf{Wien}, \emph{Verwaltungsgebiet}|pw}er Jüdel und
               würdeſt Dir \label{K_L02698-3v}\edtext{\textsc{parchanische}}{\lemma{\textnormal{\emph{parchanische}}}\Cendnote{\textnormal{Unklarer Begriff, der vom jiddischen Wort
                     ›parve‹ herrühren könnte, und ›nicht koscher‹
                  bedeutet. Es könnte aber auch das jiddische oder tschechische Wort für ›Bastard‹
                  gemeint sein.}}}\label{K_L02698-3} Gedanken darüber machen\pend
           
\pstart
           Ich gratulire Dir herzlich zu dieſem{ }ſchönen Erfolge Deines Talentes.\pend
           
\pstart
           Das iſt das einzige Dich intereſſirende, was ich{ }ſeit langer Zeit zu berichten
               finde.\pend
           
\pstart
           Über mich laß’ mich{ }ſchweigen. Ich verfalle und verrohe. Paris\oindex{Paris@\textbf{Paris}, \emph{Hauptstadt}|pw} iſt mir widerlich, meine Stellung entſetzlich, das
               Heimweh nach Wien\oindex{Wien@\textbf{Wien}, \emph{Verwaltungsgebiet}|pw}, nach Dir und all’ {\pb}den lieben Menſchen verzehrt mich. Ich bin einſam,
               zertreten und lieblos. Die Freundſchaft habe ich auch verloren, wie Du weißt. Durch
               meine Schuld, jawohl. Ich kann mich nicht mehr dazu aufſchwingen, Dir{ }ſo zu{ }ſchreiben, wie ich Dir es{ }ſchuldig wäre. Ich bin{ }ſchon zu tief. Und ich denke, es iſt
               beſſer, ich laſſe mich langſam in die Vergeſſenheit herunterſinken.\pend
           
\pstart
           Ich grüße \textsc{Richard\pwindex{Beer-Hofmann, Richard 11.\,7.\,1866 Wien – 26.\,9.\,1945 New York City@\textsc{Beer-Hofmann, Richard} (11.\,7.\,1866 Wien – 26.\,9.\,1945 New York City), \emph{Schriftsteller}|pw}} und \textsc{Loris}\pwindex{Hofmannsthal, Hugo von 1.\,2.\,1874 Wien – 15.\,7.\,1929 Rodaun@\textsc{Hofmannsthal, Hugo von} (1.\,2.\,1874 Wien – 15.\,7.\,1929 Rodaun), \emph{Schriftsteller}|pw} und umarme Dich von Herzen\pend
           
\pstart
           Dein {\\[\baselineskip]}treuer {\\[\baselineskip]}\spacefill\mbox{Paul Goldmann.}\pend
           \leftskip=0em{}
\pstart
           \noindent{}{\pb}Es{ }ſei denn, daß Du ein Mittel wüßteſt, wie ich
                  Dich im Auguſt, wo ich wahrſcheinlich
                     \textcolor{gray}{k}urzen Urlaub bekomme, \label{K_L02698-4v}\edtext{ſehen kann}{\lemma{\textnormal{\emph{sehen kann}}}\Cendnote{\textnormal{Das
                     nächste Wiedersehen fand am 17. 9. 1893 statt.}}}\label{K_L02698-4}. Aber nach \textsc{Wien\oindex{Wien@\textbf{Wien}, \emph{Verwaltungsgebiet}|pw}} komme ich nicht, weil ich nicht ein zweites Mal die Kraft fände, mich
                  loszureißen.\pend
           
\pstart
           Meine einzige Freude iſt \textsc{Arthur Klein\pwindex{Klein, Arthur 27.\,11.\,1868 Wien – 28.\,7.\,1943@\textsc{Klein, Arthur} (27.\,11.\,1868 Wien – 28.\,7.\,1943)|pw}}. \textsc{Leopold Spitzer\pwindex{Spitzer, Leopold 1865 – 28.\,6.\,1913 Wien@\textsc{Spitzer, Leopold} (1865 – 28.\,6.\,1913 Wien), \emph{Journalist}|pw}}, der eine widerlich gemeine \label{K_L02698-5v}\edtext{Ladenſchwung-Seele\pwindex{Spitzer, Leopold 1865 – 28.\,6.\,1913 Wien@\textsc{Spitzer, Leopold} (1865 – 28.\,6.\,1913 Wien), \emph{Journalist}|pwv}}{\lemma{\textnormal{\emph{Ladenschwung-Seele}}}\Cendnote{\textnormal{abwertende Bezeichnung für einen
                     Ladendiener oder Ladenjungen}}}\label{K_L02698-5} iſt, habe ich vor 14 Tagen geohrfeigt, was
                  mich um ein Haar um meine Stellung gebracht hätte und vielleicht noch bringt.\pend
           \selectlanguage{ngerman}\endnumbering\briefempfaengerindex{Schnitzler, Arthur@\textsc{Schnitzler, Arthur}!zzzGoldmann, Paul@\emph{von Paul Goldmann}!1892-06-241@{24. 6. [1892]}|)be}\mylabel{L02698h}  \newcommand{\dateiname}{L02698}\newcommand{\titel}{Paul Goldmann an Arthur Schnitzler, 24. 6. [1892]}\newcommand{\editorInnen}{Martin Anton Müller und Laura Untner}%% latex-leseansicht-abspann.tex
%% Abspann für die Leseansicht.
%% Der Schalter \ifkorrekturansicht ist bereits durch den Vorspann gesetzt.

%% latex-abspann.tex
%% Gemeinsamer Abspann für Korrekturansicht und Leseansicht.
%% Setzt den Schalter \ifkorrekturansicht voraus (gesetzt in den
%% einbindenden Dateien latex-korrekturansicht-abspann.tex bzw.
%% latex-leseansicht-abspann.tex).
%% ---------------------------------------------------------------

\normalsize

% Das esempio-Environment wird nur in der Leseansicht benötigt
\ifkorrekturansicht\else
\newenvironment{esempio}[3]%
{
    \vspace{1.5ex}
    \rlap{\underline{#1}}
    \par
    \setlength{\parindent}{0cm}
    \nopagebreak
    \leftskip=#2cm
    \rightskip=#3cm
}
{
    \par
}
\fi

\doendnotes{C}
\bigskip
\vfill

\clearpage

\footnotesize

\ifkorrekturansicht
  \lohead{\textsc{register}}
\fi

% theindex-Environment neu definieren ohne reledmac
\makeatletter
\renewenvironment{theindex}{%
  \ifkorrekturansicht
    \section*{\indexname}%
  \else
    \subsubsection*{Index der erwähnten Entitäten}%
  \fi
  \setlength{\parindent}{0pt}%
  \setlength{\parskip}{0pt plus 0.3pt}%
  \let\item\@idxitem
}{%
  \ifkorrekturansicht\clearpage\fi
}
\makeatother

\IfFileExists{\jobname-pw.ind}{\input{\jobname-pw.ind}}{}

% Quellenangabe nur in der Leseansicht
\ifkorrekturansicht\else
% Fallback-Definitionen, falls die .tex-Datei \titel etc. nicht gesetzt hat
\providecommand{\titel}{}
\providecommand{\editorInnen}{}
\providecommand{\dateiname}{\jobname}

\vspace{3cm}

\vfill

\footnotesize
\textsc{Quelle}: \titel. Herausgegeben von {\editorInnen}. In: \emph{Arthur Schnitzler: Briefwechsel mit Autorinnen und Autoren}.
 Digitale Edition, https://schnitzler-briefe.acdh.oeaw.ac.at/{\dateiname}.html (Stand \today)
\fi

\end{document}


