%% latex-korrekturansicht-vorspann.tex
%% Vorspann für die Korrekturansicht.
%% Lädt die gemeinsame Datei latex-vorspann.tex mit gesetztem Schalter.

\newif\ifkorrekturansicht
\korrekturansichttrue

\input{../tex-inputs/latex-vorspann}


\section[Paul Goldmann an Arthur Schnitzler, 24. 6. {[}1892{]}]{L02698 Paul Goldmann an Arthur Schnitzler, 24. 6. {[}1892{]}}
\nopagebreak\mylabel{L02698v}
\rehead{ }\normalsize\beginnumbering\briefempfaengerindex{Schnitzler, Arthur@\textsc{Schnitzler, Arthur}!zzzGoldmann, Paul@\emph{von Paul Goldmann}!1892-06-241@{24. 6. {[}1892{]}}|(be}
\toendnotes[C]{\smallbreak\pagebreak[2]}\Standort{DLA, A:Schnitzler, HS.NZ85.1.3163.}
\physDesc{Brief, 1 Blatt, 4 Seiten, 1894 Zeichen
\newline{}Handschrift: schwarze Tinte, deutsche Kurrent
\newline{}Schnitzler: 1) mit rotem Buntstift eine Unterstreichung  2) mit Bleistift das Jahr »92« vermerkt}\toendnotes[C]{\smallbreak}
\pstart
           {\pb}\textcolor{gray}{\textbf{Frankfurter Zeitung\orgindex{Frankfurter Zeitung@Frankfurter Zeitung|pw}.}}\pend
           
\pstart
           \textcolor{gray}{\textbf{(Gazette de
                     Francfort\orgindex{Frankfurter Zeitung@Frankfurter Zeitung|pw}.)}}\pend
           
\pstart
           \textcolor{gray}{\textbf{\begin{otherlanguage}{french}Directeur\end{otherlanguage}: \textbf{M. L. Sonnemann\pwindex{Sonnemann, Leopold 1831-10-29 – 1909-10-30@\textsc{Sonnemann, Leopold} (1831-10-29 – 1909-10-30), \emph{Journalist/Journalistin, Herausgeber/Herausgeberin}|pw}}.}}\hfill \textsc{Paris\oindex{Paris@\textbf{Paris}, \emph{P.PPLC}|pw}}, 24. Juni.\pend
           
\pstart
           \textcolor{gray}{\textbf{\begin{otherlanguage}{french}Journal politique, financier,\end{otherlanguage}}}\pend
           
\pstart
           \textcolor{gray}{\textbf{\begin{otherlanguage}{french}commercial et litteraire.\end{otherlanguage}}}\pend
           
\pstart
           \textcolor{gray}{\textbf{\begin{otherlanguage}{french}\textbf{Paraissant trois fois par jour}\end{otherlanguage}}}\pend
           
\pstart
           \textcolor{gray}{\textbf{\begin{otherlanguage}{french}\textbf{Bureaux à Paris\oindex{Paris@\textbf{Paris}, \emph{P.PPLC}|pw}:}\end{otherlanguage}}}\pend
           
\pstart
           \textcolor{gray}{\textbf{\begin{otherlanguage}{french}\textbf{rue Richelieu 75\oindex{rue Richelieu@\textbf{rue Richelieu}, \emph{Straße (K.STR)}|pw}.}\end{otherlanguage}}}\pend
           
\pstart\center{}Mein lieber Arthur!\pend\vspace{0.5em}
\pstart
           Ich habe \strikeout{heute}{ }\textsc{Herzl\pwindex{Herzl, Theodor 1860-05-02 – 1904-07-03@\textsc{Herzl, Theodor} (1860-05-02 – 1904-07-03), \emph{Schriftsteller/Schriftstellerin, Journalist/Journalistin}|pw}}{ }\strikeout{h} dein Märchen\pwindex{Maerchen. Schauspiel in drei Aufzuegen@\emph{Das Märchen. Schauspiel in drei Aufzügen}|pw}
               gegeben und war heute bei ihm. Derſelbe ſprach ſich
               darüber in Worten der \label{K_L02698-1v}\edtext{Begeiſterung}{\lemma{\textnormal{\emph{Begeiſterung}}}\Cendnote{\textnormal{Am 28. 6. 1892 notierte
                     Schnitzler in seinem \emph{Tagebuch}\pwindex{Tagebuch@\emph{Tagebuch}|pwk}: »Herzl\pwindex{Herzl, Theodor 1860-05-02 – 1904-07-03@\textsc{Herzl, Theodor} (1860-05-02 – 1904-07-03), \emph{Schriftsteller/Schriftstellerin, Journalist/Journalistin}|pw}’s begeistertes Urtheil übers Märchen\pwindex{Maerchen. Schauspiel in drei Aufzuegen@\emph{Das Märchen. Schauspiel in drei Aufzügen}|pw}, was mich lebhaft
                  freute.«}}}\label{K_L02698-1} (wörtlich zu nehmen) aus. Er meinte, Du ſeieſt der einzige
               von uns allen Jungen – ihn inbegriffen – der ’was kann. Er meinte, du ſeieſt ein
               wahrer Dichter. Er meinte, das Ding\pwindex{Maerchen. Schauspiel in drei Aufzuegen@\emph{Das Märchen. Schauspiel in drei Aufzügen}|pwv} habe ihn ſo gepackt, daß er es in einem Zuge ausgeleſen. Er meinte,
               meinte und meinte, ich weiß nicht was noch Alles Wunderſchönes für Dich, weil es der
               von {\pb}ſich ſelbſt eingenommenſte Menſch\pwindex{Herzl, Theodor 1860-05-02 – 1904-07-03@\textsc{Herzl, Theodor} (1860-05-02 – 1904-07-03), \emph{Schriftsteller/Schriftstellerin, Journalist/Journalistin}|pwv}{ }Europas\oindex{Europa@\textbf{Europa}, \emph{Kontinent (A.KNT)}|pw} meint. Er ſagte ſchließlich, daß er Dir
               ſofort \label{K_L02698-2v}\edtext{geſchrieben}{\lemma{\textnormal{\emph{geſchrieben}}}\Cendnote{\textnormal{Theodor Herzl\pwindex{Herzl, Theodor 1860-05-02 – 1904-07-03@\textsc{Herzl, Theodor} (1860-05-02 – 1904-07-03), \emph{Schriftsteller/Schriftstellerin, Journalist/Journalistin}|pwk} schrieb erst am
                     29. 7. 1892 an Schnitzler (was dieser am 4. 8. 1892 im \emph{Tagebuch}\pwindex{Tagebuch@\emph{Tagebuch}|pwk} festhielt). Siehe Theodor Herzl\pwindex{Herzl, Theodor 1860-05-02 – 1904-07-03@\textsc{Herzl, Theodor} (1860-05-02 – 1904-07-03), \emph{Schriftsteller/Schriftstellerin, Journalist/Journalistin}|pwk}: \emph{Briefe und
                        Tagebücher}. Herausgegeben von Alex Bein, Hermann Greive, Moshe Schaerf und Julius
                     H. Schoeps. Bd. 1.: \emph{Briefe und autobiographische Notizen.
                        1866–1895}. Bearbeitet von Johannes Wachten. In Zusammenarbeit mit
                     Chaya Harel, Daisy Tycho und Manfred Winkler. Berlin,
                     Frankfurt am Main, Wien: \emph{Ullstein}\orgindex{Ullstein Verlag@Ullstein Verlag|pwk}/\emph{Propyläen}\orgindex{Propylaeen Verlag@Propyläen Verlag|pwk}{ }1983, S. 498–502.
               }}}\label{K_L02698-2} hätte, wenn er nicht gefürchtet hätte – \textsc{pardon}, ich
               referire wörtlich – Du ſeieſt ein Wien\oindex{Wien@\textbf{Wien}, \emph{A.ADM2}|pw}er Jüdel und
               würdeſt Dir \label{K_L02698-3v}\edtext{\textsc{parchanische}}{\lemma{\textnormal{\emph{parchanische}}}\Cendnote{\textnormal{Unklarer Begriff, der vom jiddischen Wort
                     ›parve‹ herrühren könnte, und ›nicht koscher‹
                  bedeutet. Es könnte aber auch das jiddische oder tschechische Wort für ›Bastard‹
                  gemeint sein.}}}\label{K_L02698-3} Gedanken darüber machen\pend
           
\pstart
           Ich gratulire Dir herzlich zu dieſem ſchönen Erfolge Deines Talentes.\pend
           
\pstart
           Das iſt das einzige Dich intereſſirende, was ich ſeit langer Zeit zu berichten
               finde.\pend
           
\pstart
           Über mich laß’ mich ſchweigen. Ich verfalle und verrohe. Paris\oindex{Paris@\textbf{Paris}, \emph{P.PPLC}|pw} iſt mir widerlich, meine Stellung entſetzlich, das
               Heimweh nach Wien\oindex{Wien@\textbf{Wien}, \emph{A.ADM2}|pw}, nach Dir und all’ {\pb}den lieben Menſchen verzehrt mich. Ich bin einſam,
               zertreten und lieblos. Die Freundſchaft habe ich auch verloren, wie Du weißt. Durch
               meine Schuld, jawohl. Ich kann mich nicht mehr dazu aufſchwingen, Dir ſo zu
               ſchreiben, wie ich Dir es ſchuldig wäre. Ich bin ſchon zu tief. Und ich denke, es iſt
               beſſer, ich laſſe mich langſam in die Vergeſſenheit herunterſinken.\pend
           
\pstart
           Ich grüße \textsc{Richard\pwindex{Beer-Hofmann, Richard 1866-07-11 – 1945-09-26@\textsc{Beer-Hofmann, Richard} (1866-07-11 – 1945-09-26), \emph{Schriftsteller/Schriftstellerin}|pw}} und \textsc{Loris}\pwindex{Hofmannsthal, Hugo von 1874-02-01 – 1929-07-15@\textsc{Hofmannsthal, Hugo von} (1874-02-01 – 1929-07-15), \emph{Schriftsteller/Schriftstellerin}|pw} und umarme Dich von Herzen\pend
           
\pstart
           Dein {\\[\baselineskip]}treuer {\\[\baselineskip]}\spacefill\mbox{Paul Goldmann.}\pend
           \leftskip=0em{}
\pstart
           \noindent{}{\pb}Es ſei denn, daß Du ein Mittel wüßteſt, wie ich
                  Dich im Auguſt, wo ich wahrſcheinlich
                     \textcolor{gray}{k}urzen Urlaub bekomme, \label{K_L02698-4v}\edtext{ſehen kann}{\lemma{\textnormal{\emph{ſehen kann}}}\Cendnote{\textnormal{Das
                     nächste Wiedersehen fand am 17. 9. 1893 statt.}}}\label{K_L02698-4}. Aber nach \textsc{Wien\oindex{Wien@\textbf{Wien}, \emph{A.ADM2}|pw}} komme ich nicht, weil ich nicht ein zweites Mal die Kraft fände, mich
                  loszureißen.\pend
           
\pstart
           Meine einzige Freude iſt \textsc{Arthur Klein\pwindex{Klein, Arthur 27.11.1868 – 28.07.1943@\textsc{Klein, Arthur} (27.11.1868 – 28.07.1943)|pw}}. \textsc{Leopold Spitzer\pwindex{Spitzer, Leopold 1865 – 1913-06-28@\textsc{Spitzer, Leopold} (1865 – 1913-06-28), \emph{Journalist/Journalistin}|pw}}, der eine widerlich gemeine \label{K_L02698-5v}\edtext{Ladenſchwung-Seele\pwindex{Spitzer, Leopold 1865 – 1913-06-28@\textsc{Spitzer, Leopold} (1865 – 1913-06-28), \emph{Journalist/Journalistin}|pwv}}{\lemma{\textnormal{\emph{Ladenſchwung-Seele}}}\Cendnote{\textnormal{abwertende Bezeichnung für einen
                     Ladendiener oder Ladenjungen}}}\label{K_L02698-5} iſt, habe ich vor 14 Tagen geohrfeigt, was
                  mich um ein Haar um meine Stellung gebracht hätte und vielleicht noch bringt.\pend
           \selectlanguage{ngerman}\endnumbering\briefempfaengerindex{Schnitzler, Arthur@\textsc{Schnitzler, Arthur}!zzzGoldmann, Paul@\emph{von Paul Goldmann}!1892-06-241@{24. 6. {[}1892{]}}|)be}\mylabel{L02698h}  \normalsize

\doendnotes{C}
\bigskip
\vfill

\clearpage

\footnotesize

\lohead{\textsc{register}}

% Definiere theindex-Environment komplett neu ohne reledmac
\makeatletter
\renewenvironment{theindex}{%
  \section*{\indexname}%
  \setlength{\parindent}{0pt}%
  \setlength{\parskip}{0pt plus 0.3pt}%
  \let\item\@idxitem
}{%
  \clearpage
}
\makeatother

\IfFileExists{\jobname-pw.ind}{\input{\jobname-pw.ind}}{}

\end{document}

      