%% latex-leseansicht-vorspann.tex
%% Vorspann für die Leseansicht.
%% Lädt die gemeinsame Datei latex-vorspann.tex mit nicht gesetztem Schalter.

\newif\ifkorrekturansicht
\korrekturansichtfalse

\input{../tex-inputs/latex-vorspann}


\section[Arthur Schnitzler an Hugo von Hofmannsthal, {{[}}21. 4. 1893?{{]}}]{L00199 Arthur Schnitzler an Hugo von Hofmannsthal, {[}21. 4. 1893?{]}}
\nopagebreak\mylabel{L00199v}
\rehead{ }\normalsize\beginnumbering\briefempfaengerindex{Hofmannsthal, Hugo von@\textsc{Hofmannsthal, Hugo von}!zzzSchnitzler, Arthur@\emph{von Arthur Schnitzler}!1893-04-211@{{[}21. 4. 1893?{]}}|(be}
\toendnotes[C]{\smallbreak\pagebreak[2]}
\correspDesc{Versand  durch Arthur Schnitzler am [21. 4. 1893?] in Wien
\newline{}Erhalt  durch Hugo von Hofmannsthal im Zeitraum [21. 4. 1893 – 25. 4. 1893?] in Wien}\toendnotes[C]{\smallbreak}
\Standort{FDH, Hs-30885,39.}
\physDesc{Brief, 1 Blatt, 2 Seiten, 465 Zeichen
\newline{}Handschrift: schwarze Tinte, deutsche Kurrent
\newline{}Ordnung: mit Bleistift  von unbekannter Hand datiert: »91?«  }
\buchAbdrucke{\weitereDrucke{1) Hugo von Hofmannsthal, Arthur Schnitzler: \emph{Briefwechsel}. Herausgegeben von Therese Nickl und Heinrich Schnitzler. Frankfurt am Main: \emph{S. Fischer} 1964, S. 47–48.} \weitereDrucke{2) Hermann Bahr, Arthur Schnitzler: \emph{Briefwechsel, Aufzeichnungen, Dokumente
                                (1891–1931)}. Herausgegeben von Kurt Ifkovits und Martin Anton Müller. Göttingen: \emph{Wallstein} 2018.} }\toendnotes[C]{\smallbreak}
\pstart{}{\pb}Lieber Hugo,\pend\vspace{0.5em}
\pstart
           beifolgende Briefe, erſter \label{K_L00199-1v}\edtext{von
                        \textsc{Fels}\pwindex{Fels, Friedrich Michael *~1864 Bad Dürkheim@\textsc{Fels, Friedrich Michael} (*~1864 Bad Dürkheim), \emph{Journalist}|pw}}{\lemma{\textnormal{\emph{von
                        Fels}}}\Cendnote{\textnormal{Siehe XXXX Auszeichnungsfehler: Dokument L00198 nicht gefunden.}}}\label{K_L00199-1}
                    zweiter \label{K_L00199-2v}\edtext{von Frau \textsc{Clara Schreiber}\pwindex{Schreiber, Clara 27.\,10.\,1848 Wien – 8.\,2.\,1905 Meran@\textsc{Schreiber, Clara} (27.\,10.\,1848 Wien – 8.\,2.\,1905 Meran), \emph{Schriftstellerin}|pw}}{\lemma{\textnormal{\emph{von Frau Clara Schreiber}}}\Cendnote{\textnormal{ Sie bittet um Hilfe, Fels\pwindex{Fels, Friedrich Michael *~1864 Bad Dürkheim@\textsc{Fels, Friedrich Michael} (*~1864 Bad Dürkheim), \emph{Journalist}|pwk} habe nun seit acht Wochen sein
                        Logis nicht bezahlt und er behaupte, kein Geld zu haben. (\emph{Cambridge University Library}, Schnitzler, B 385
                        .)}}}\label{K_L00199-2}, an die ich unſern Freund empfohlen habe, die Gattin des
                    Dr. \textsc{Schreiber}\pwindex{Schreiber, Joseph 17.\,3.\,1835 Česká Lípa – 28.\,9.\,1908 Bad Aussee@\textsc{Schreiber, Joseph} (17.\,3.\,1835 Česká Lípa – 28.\,9.\,1908 Bad Aussee), \emph{Mediziner, Sanatoriumsleiter, Arzt}|pw}, Curarzt in Meran\oindex{Meran@\textbf{Meran}, \emph{Hauptstadt}|pw}, –{ }ſind auch für
                    Sie von Intereſſe. Ich bitte Sie,{ }ſich vielleicht an Bahr\pwindex{Bahr, Hermann 19.\,7.\,1863 Linz – 15.\,1.\,1934 München@\textsc{Bahr, Hermann} (19.\,7.\,1863 Linz – 15.\,1.\,1934 München), \emph{Schriftsteller, Kritiker}|pw} zu wenden, was Sie ja von uns dreien am \label{K_L00199-3v}\edtext{leichteſten}{\lemma{\textnormal{\emph{leichtesten}}}\Cendnote{\textnormal{Sie wohnten beide in der Salesianergasse 12\oindex{Wien@\textbf{Wien}!III., Landstraße@\textbf{III., Landstraße}!Salesianergasse 12@\textbf{Salesianergasse 12}, \emph{Wohngebäude}|pwk}.}}}\label{K_L00199-3} u beſten können, {\pb}und mich{ }ſo raſch als möglich von dem Ausfall Ihrer
                    Bemühungen zu unterrichten,{ }ſowie die beiden Briefe mir zurückzuſchicken.\pend
           
\pstart
           Ich bin mit herzlichen Grüßen {\\[\baselineskip]}Ihr \spacefill\mbox{Arthur}\pend
           \leftskip=0em{}\selectlanguage{ngerman}\endnumbering\briefempfaengerindex{Hofmannsthal, Hugo von@\textsc{Hofmannsthal, Hugo von}!zzzSchnitzler, Arthur@\emph{von Arthur Schnitzler}!1893-04-211@{{[}21. 4. 1893?{]}}|)be}\mylabel{L00199h}  \newcommand{\dateiname}{L00199}\newcommand{\titel}{Arthur Schnitzler an Hugo von Hofmannsthal, [21. 4. 1893?]}\newcommand{\editorInnen}{Herausgegeben von Martin Anton Müller}%% latex-leseansicht-abspann.tex
%% Abspann für die Leseansicht.
%% Der Schalter \ifkorrekturansicht ist bereits durch den Vorspann gesetzt.

%% latex-abspann.tex
%% Gemeinsamer Abspann für Korrekturansicht und Leseansicht.
%% Setzt den Schalter \ifkorrekturansicht voraus (gesetzt in den
%% einbindenden Dateien latex-korrekturansicht-abspann.tex bzw.
%% latex-leseansicht-abspann.tex).
%% ---------------------------------------------------------------

\normalsize

% Das esempio-Environment wird nur in der Leseansicht benötigt
\ifkorrekturansicht\else
\newenvironment{esempio}[3]%
{
    \vspace{1.5ex}
    \rlap{\underline{#1}}
    \par
    \setlength{\parindent}{0cm}
    \nopagebreak
    \leftskip=#2cm
    \rightskip=#3cm
}
{
    \par
}
\fi

\doendnotes{C}
\bigskip
\vfill

\clearpage

\footnotesize

\ifkorrekturansicht
  \lohead{\textsc{register}}
\fi

% theindex-Environment neu definieren ohne reledmac
\makeatletter
\renewenvironment{theindex}{%
  \ifkorrekturansicht
    \section*{\indexname}%
  \else
    \subsubsection*{Index der erwähnten Entitäten}%
  \fi
  \setlength{\parindent}{0pt}%
  \setlength{\parskip}{0pt plus 0.3pt}%
  \let\item\@idxitem
}{%
  \ifkorrekturansicht\clearpage\fi
}
\makeatother

\IfFileExists{\jobname-pw.ind}{\input{\jobname-pw.ind}}{}

% Quellenangabe nur in der Leseansicht
\ifkorrekturansicht\else
% Fallback-Definitionen, falls die .tex-Datei \titel etc. nicht gesetzt hat
\providecommand{\titel}{}
\providecommand{\editorInnen}{}
\providecommand{\dateiname}{\jobname}

\vspace{3cm}

\vfill

\footnotesize
\textsc{Quelle}: \titel. Herausgegeben von {\editorInnen}. In: \emph{Arthur Schnitzler: Briefwechsel mit Autorinnen und Autoren}.
 Digitale Edition, https://schnitzler-briefe.acdh.oeaw.ac.at/{\dateiname}.html (Stand \today)
\fi

\end{document}


