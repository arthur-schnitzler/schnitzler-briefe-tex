%% latex-leseansicht-vorspann.tex
%% Vorspann für die Leseansicht.
%% Lädt die gemeinsame Datei latex-vorspann.tex mit nicht gesetztem Schalter.

\newif\ifkorrekturansicht
\korrekturansichtfalse

\input{../tex-inputs/latex-vorspann}


         
         \renewcommand{\erwaehntePersonen}{Personen: Hermann Bahr, Friedrich Michael Fels, Hugo von Hofmannsthal, Clara Schreiber, Joseph Schreiber}
         \renewcommand{\erwaehnteInstitutionen}{Institutionen: Deutsche Zeitung}
         \renewcommand{\erwaehnteOrte}{Orte: Meran, Salesianergasse, Wien}
         \renewcommand{\erwaehnteWerke}{}
               \section[Arthur Schnitzler an Hugo von Hofmannsthal, {[}21. 4. 1893?{]}]{ Arthur Schnitzler an Hugo von Hofmannsthal, {[}21. 4. 1893?{]}}\nopagebreak\mylabel{v}\rehead{ }\begin{ledgroupsized}[t]{13cm}\normalsize\beginnumbering \toendnotes[C]{\smallbreak\pagebreak[2]} \Standort{FDH, Hs-30885,39.}
\physDesc{Brief, 1 Blatt, 2 Seiten, 465 Zeichen
\newline{}Handschrift: schwarze Tinte, deutsche Kurrent
\newline{}Ordnung: mit Bleistift von unbekannter Hand datiert: »91?« }\buchAbdrucke{\weitereDrucke{1) Hugo von Hofmannsthal, Arthur Schnitzler: \emph{Briefwechsel}. Hg. Therese Nickl und Heinrich Schnitzler. Frankfurt am Main: \emph{S. Fischer} 1964, S. 47–48.} \weitereDrucke{2) Hermann Bahr, Arthur Schnitzler: \emph{Briefwechsel, Aufzeichnungen, Dokumente (1891–1931)}. Hg. Kurt Ifkovits und Martin Anton Müller. Göttingen: \emph{Wallstein} 2018.} }\toendnotes[C]{\smallbreak}\pstart{}{\pb}Lieber Hugo,\pend\pstart
           beifolgende Briefe, erſter \label{K_L00199_1v}\edtext{von \textsc{Fels}\pwindex{Fels, Friedrich Michael *~1864@\textsc{Fels, Friedrich Michael} (*~1864), \emph{Journalist}|pw}}{\lemma{\textnormal{\emph{von Fels}}}\Cendnote{\textnormal{In einem Brief vom
                     20. 4. 1893 (\emph{Deutsches Literaturarchiv}, A:Schnitzler,
                     85.1.2956) schreibt Fels\pwindex{Fels, Friedrich Michael *~1864@\textsc{Fels, Friedrich Michael} (*~1864), \emph{Journalist}|pwk}, dass er
                  zum Monatsende nach Wien\oindex{Wien@\textbf{Wien}|pwk} und mit
                     1. 5. bei der \emph{Deutschen
                     Zeitung}\orgindex{Deutsche Zeitung@Deutsche Zeitung|pwk} beginnen könne. Er würde dann ein Drittel oder Viertel des
                  Einkommens dazu verwenden, seine Schulden in Meran\oindex{Meran@\textbf{Meran}|pwk} zu begleichen.}}}\label{K_L00199_1h} zweiter \label{K_L00199_2v}\edtext{von Frau \textsc{Clara Schreiber}\pwindex{Schreiber, Clara 27.10.1848 – 8.2.1905@\textsc{Schreiber, Clara} (27.10.1848 – 8.2.1905), \emph{Schriftstellerin}|pw}}{\lemma{\textnormal{\emph{von Frau Clara Schreiber}}}\Cendnote{\textnormal{Sie bittet um Hilfe, Fels\pwindex{Fels, Friedrich Michael *~1864@\textsc{Fels, Friedrich Michael} (*~1864), \emph{Journalist}|pwk} habe nun seit acht Wochen sein Logis nicht bezahlt und
                  er würde behaupten, kein Geld zu haben (\emph{Cambridge University Library}, Schnitzler, B
                  385).}}}\label{K_L00199_2h}, an die ich unſern Freund empfohlen habe, die Gattin des Dr.
                  \textsc{Schreiber}\pwindex{Schreiber, Joseph 17.03.1835 – 28.09.1908@\textsc{Schreiber, Joseph} (17.03.1835 – 28.09.1908), \emph{Mediziner, Mediziner}|pw}, Curarzt in Meran\oindex{Meran@\textbf{Meran}|pw}, – ſind auch für Sie
               von Intereſſe. Ich bitte Sie, ſich vielleicht an Bahr\pwindex{Bahr, Hermann 19.07.1863 – 15.01.1934@\textsc{Bahr, Hermann} (19.07.1863 – 15.01.1934), \emph{Schriftsteller, Kritiker}|pw} zu wenden, was Sie ja von uns dreien am \label{K_L00199_3v}\edtext{leichteſten}{\lemma{\textnormal{\emph{leichteſten}}}\Cendnote{\textnormal{Sie
                  wohnten beide in der Salesianergasse 12\oindex{Salesianergasse@\textbf{Salesianergasse}|pwk}.}}}\label{K_L00199_3h}
               u beſten können, {\pb}und mich ſo raſch als möglich von dem
               Ausfall Ihrer Bemühungen zu unterrichten, ſowie die beiden Briefe mir
               zurückzuſchicken.\pend
           \pstart
           Ich bin mit herzlichen Grüßen{\\[\baselineskip]}Ihr\spacefill\mbox{Arthur}\pend
           \leftskip=0em{}
         
         \endnumbering\mylabel{h}\end{ledgroupsized}  \newcommand{\dateiname}{L00199}\newcommand{\titel}{Arthur Schnitzler an Hugo von Hofmannsthal, [21. 4. 1893?]}\newcommand{\editorInnen}{ Martin Anton Müller und Gerd-Hermann Susen}%% latex-leseansicht-abspann.tex
%% Abspann für die Leseansicht.
%% Der Schalter \ifkorrekturansicht ist bereits durch den Vorspann gesetzt.

%% latex-abspann.tex
%% Gemeinsamer Abspann für Korrekturansicht und Leseansicht.
%% Setzt den Schalter \ifkorrekturansicht voraus (gesetzt in den
%% einbindenden Dateien latex-korrekturansicht-abspann.tex bzw.
%% latex-leseansicht-abspann.tex).
%% ---------------------------------------------------------------

\normalsize

% Das esempio-Environment wird nur in der Leseansicht benötigt
\ifkorrekturansicht\else
\newenvironment{esempio}[3]%
{
    \vspace{1.5ex}
    \rlap{\underline{#1}}
    \par
    \setlength{\parindent}{0cm}
    \nopagebreak
    \leftskip=#2cm
    \rightskip=#3cm
}
{
    \par
}
\fi

\doendnotes{C}
\bigskip
\vfill

\clearpage

\footnotesize

\ifkorrekturansicht
  \lohead{\textsc{register}}
\fi

% theindex-Environment neu definieren ohne reledmac
\makeatletter
\renewenvironment{theindex}{%
  \ifkorrekturansicht
    \section*{\indexname}%
  \else
    \subsubsection*{Index der erwähnten Entitäten}%
  \fi
  \setlength{\parindent}{0pt}%
  \setlength{\parskip}{0pt plus 0.3pt}%
  \let\item\@idxitem
}{%
  \ifkorrekturansicht\clearpage\fi
}
\makeatother

\IfFileExists{\jobname-pw.ind}{\input{\jobname-pw.ind}}{}

% Quellenangabe nur in der Leseansicht
\ifkorrekturansicht\else
% Fallback-Definitionen, falls die .tex-Datei \titel etc. nicht gesetzt hat
\providecommand{\titel}{}
\providecommand{\editorInnen}{}
\providecommand{\dateiname}{\jobname}

\vspace{3cm}

\vfill

\footnotesize
\textsc{Quelle}: \titel. Herausgegeben von {\editorInnen}. In: \emph{Arthur Schnitzler: Briefwechsel mit Autorinnen und Autoren}.
 Digitale Edition, https://schnitzler-briefe.acdh.oeaw.ac.at/{\dateiname}.html (Stand \today)
\fi

\end{document}


      