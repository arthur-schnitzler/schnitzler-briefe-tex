%% latex-korrekturansicht-vorspann.tex
%% Vorspann für die Korrekturansicht.
%% Lädt die gemeinsame Datei latex-vorspann.tex mit gesetztem Schalter.

\newif\ifkorrekturansicht
\korrekturansichttrue

\input{../tex-inputs/latex-vorspann}


\section[Arthur Schnitzler an Hugo von Hofmannsthal, {[}21. 4. 1893?{]}]{L00199 Arthur Schnitzler an Hugo von Hofmannsthal, {[}21. 4. 1893?{]}}
\nopagebreak\mylabel{L00199v}
\rehead{ }\normalsize\beginnumbering\briefempfaengerindex{Hofmannsthal, Hugo von@\textsc{Hofmannsthal, Hugo von}!zzzSchnitzler, Arthur@\emph{von Arthur Schnitzler}!1893-04-211@{{[}21. 4. 1893?{]}}|(be}
\toendnotes[C]{\smallbreak\pagebreak[2]}\Standort{FDH, Hs-30885,39.}
\physDesc{Brief, 1 Blatt, 2 Seiten, 465 Zeichen
\newline{}Handschrift: schwarze Tinte, deutsche Kurrent
\newline{}Ordnung: mit Bleistift 
                                 von unbekannter Hand datiert: »
                                 91?
                                 «
                               }
\buchAbdrucke{\weitereDrucke{1) Hugo von Hofmannsthal, Arthur Schnitzler: \emph{Briefwechsel}. Frankfurt am Main: \emph{S. Fischer} 1964, S. 47–48.} \weitereDrucke{2) Hermann Bahr, Arthur Schnitzler: \emph{Briefwechsel, Aufzeichnungen, Dokumente (1891–1931)}. Göttingen: \emph{Wallstein} 2018.} }\toendnotes[C]{\smallbreak}
\pstart{}{\pb}
                  Lieber Hugo,
               \pend\vspace{0.5em}
\pstart
           
               beifolgende Briefe, erſter 
               \label{K_L00199-1v}\edtext{
               von 
               \textsc{Fels}\pwindex{Fels, Friedrich Michael *~1864@\textsc{Fels, Friedrich Michael} (*~1864), \emph{Journalist/Journalistin}|pw}}{\lemma{\textnormal{\emph{
               von 
               Fels}}}\Cendnote{\textnormal{Siehe Friedrich M. Fels an Arthur Schnitzler, 20. 4. 1893
                  .
               
               }}}\label{K_L00199-1}
                zweiter 
               \label{K_L00199-2v}\edtext{
               von Frau 
               \textsc{Clara Schreiber}\pwindex{Schreiber, Clara 27.10.1848 – 8.2.1905@\textsc{Schreiber, Clara} (27.10.1848 – 8.2.1905), \emph{Schriftsteller/Schriftstellerin}|pw}}{\lemma{\textnormal{\emph{
               von Frau 
               Clara Schreiber}}}\Cendnote{\textnormal{
                  Sie bittet um Hilfe, 
                  Fels\pwindex{Fels, Friedrich Michael *~1864@\textsc{Fels, Friedrich Michael} (*~1864), \emph{Journalist/Journalistin}|pwk}
                   habe nun seit acht Wochen sein Logis nicht bezahlt und
                  er behaupte, kein Geld zu haben. (
                  \emph{Cambridge University Library}
                     , Schnitzler, B 385
                  
                  .)
               }}}\label{K_L00199-2}
               , an die ich unſern Freund empfohlen habe, die Gattin des Dr.
                  
               \textsc{Schreiber}\pwindex{Schreiber, Joseph 17.03.1835 – 28.09.1908@\textsc{Schreiber, Joseph} (17.03.1835 – 28.09.1908), \emph{Mediziner/Medizinerin, Sanatoriumsleiter/Sanatoriumsleiterin, Arzt/Ärztin}|pw}
               , Curarzt in 
               Meran\oindex{Meran@\textbf{Meran}, \emph{P.PPLA3}|pw}
               , – ſind auch für Sie
               von Intereſſe. Ich bitte Sie, ſich vielleicht an 
               Bahr\pwindex{Bahr, Hermann 19.07.1863 – 15.01.1934@\textsc{Bahr, Hermann} (19.07.1863 – 15.01.1934), \emph{Schriftsteller/Schriftstellerin, Kritiker/Kritikerin}|pw}
                zu wenden, was Sie ja von uns dreien am 
               \label{K_L00199-3v}\edtext{
               leichteſten
               }{\lemma{\textnormal{\emph{
               leichteſten
               }}}\Cendnote{\textnormal{
                  Sie
                  wohnten beide in der 
                  Salesianergasse 12\oindex{Salesianergasse 12@\textbf{Salesianergasse 12}, \emph{Wohngebäude (K.WHS)}|pwk}
                  .
               }}}\label{K_L00199-3}
               
               u beſten können, 
               {\pb}
               und mich ſo raſch als möglich von dem
               Ausfall Ihrer Bemühungen zu unterrichten, ſowie die beiden Briefe mir
               zurückzuſchicken.
            \pend
           
\pstart
           
               Ich bin mit herzlichen Grüßen
               {\\[\baselineskip]}
               Ihr
               \spacefill\mbox{Arthur}\pend
           \leftskip=0em{}\selectlanguage{ngerman}\endnumbering\briefempfaengerindex{Hofmannsthal, Hugo von@\textsc{Hofmannsthal, Hugo von}!zzzSchnitzler, Arthur@\emph{von Arthur Schnitzler}!1893-04-211@{{[}21. 4. 1893?{]}}|)be}\mylabel{L00199h}  \normalsize

\doendnotes{C}
\bigskip
\vfill

\clearpage

\footnotesize

\lohead{\textsc{register}}

% Definiere theindex-Environment komplett neu ohne reledmac
\makeatletter
\renewenvironment{theindex}{%
  \section*{\indexname}%
  \setlength{\parindent}{0pt}%
  \setlength{\parskip}{0pt plus 0.3pt}%
  \let\item\@idxitem
}{%
  \clearpage
}
\makeatother

\IfFileExists{\jobname-pw.ind}{\input{\jobname-pw.ind}}{}

\end{document}

      