%% latex-leseansicht-vorspann.tex
%% Vorspann für die Leseansicht.
%% Lädt die gemeinsame Datei latex-vorspann.tex mit nicht gesetztem Schalter.

\newif\ifkorrekturansicht
\korrekturansichtfalse

\input{../tex-inputs/latex-vorspann}


         \renewcommand{\erwaehnteInstitutionen}{Institutionen: Wiener Freie Volksbühne}
         \renewcommand{\erwaehnteOrte}{Orte: Schönbrunnerstraße, Wien}
         \renewcommand{\erwaehnteWerke}{Werke: Der Strom. Organ der Wiener Freien Volksbühne}
               \section[Engelbert Pernerstorfer und Stefan Großmann an Arthur Schnitzler, 14. 3. 1911]{ Engelbert Pernerstorfer und Stefan Großmann an Arthur Schnitzler,
                    14. 3. 1911}\nopagebreak\mylabel{v}\rehead{ }\begin{ledgroupsized}[t]{13cm}\normalsize\beginnumbering \toendnotes[C]{\smallbreak\pagebreak[2]} \Standort{CUL, Schnitzler, B 34.}
\physDesc{Brief, 1 Blatt, 1 Seite
\newline{}Schreibmaschine
\newline{}Handschrift Engelbert Pernerstorfer: schwarze Tinte\newline{}Handschrift Stefan Großmann: schwarze Tinte
\newline{}Schnitzler: 1) mit rotem Buntstift eine Unterstreichung  2) mit Bleistift beschriftet: »\textsc{Großma{\geminationn}}« und mit einer – nur unsicher lesbaren – Skizze der Antwort versehen: »\noindent{}{[}({]}bed ſehr, durch Arbeit in Anſpruch gen ein ſptr Zeitp für
                     \textcolor{gray}{Mit}arb von mir).«\newline{}Ordnung: mit Bleistift von unbekannter Hand nummeriert:
                                                »10« }\toendnotes[C]{\smallbreak}\pstart
           \noindent{}{\pb}\textcolor{gray}{\textbf{FREIE VOLKSBÜHNE\orgindex{Wiener Freie Volksbuehne@Wiener Freie Volksbühne|pw}}}\pend
           \pstart
           \textcolor{gray}{\textbf{SEKRETARIAT: V/2,
                                    SCHÖNBRUNNERSTRASSE 124\oindex{Schoenbrunnerstrasse@\textbf{Schönbrunnerstraße}|pw}}}\pend
           \pstart
           \textcolor{gray}{\textbf{Kanzleistunden: (nur an Wochentagen): Vom 1. September
                            bis 31. Mai von 9 bis 12 Uhr vormittags und von 4 bis 8 Uhr abends. Vom
                            1. Juni bis 31. August von 9 Uhr vormittags bis 4 Uhr nachmittags}}\pend
           \pstart
           \textcolor{gray}{\textbf{Telephon: Nr. 7582}}\hfill \textcolor{gray}{\textbf{Postsparkassen-Konto: Nr. 87.544}}\pend
           \pstart
           \raggedleft{}\textcolor{gray}{\textbf{WIEN\oindex{Wien@\textbf{Wien}|pw}, den}}{ }14. März \textcolor{gray}{\textbf{191}}1.\pend
           \pstart\center{}Sehr verehrter Herr!\pend\pstart
           Die Freie Volksbühne\orgindex{Wiener Freie Volksbuehne@Wiener Freie Volksbühne|pw} will ihre
                    kunst-pädagogische Tätigkeit dadurch ergänzen, dass sie ihren Mitgliedern
                    regelmässig für 10 hl eine Zeitschrift\pwindex{Strom. Organ der Wiener Freien Volksbuehne1911 – 1914@\emph{Der Strom. Organ der Wiener Freien Volksbühne} {[}1911 – 1914{]}|pwv} in die Hand gibt, die für die stille Wirkung im eigenen
                    Heim des Mitgliedes bestimmt ist. Sie würden unsere Ziele fördern, wenn Sie uns
                    schon für die ersten Hefte irgend einen Beitrag novellistischen Charakters, oder
                    auch ein Gedicht zur Verfügung stellten. Die besten Namen Deutschlands stehen
                    uns in dieser Arbeit zur Seite und wir haben die feste Hoffnung, dass auch Sie
                    verehrter Herr uns bei diesem neuen Zweige unserer Tätigkeit helfen werden.\pend
           \pstart
           Das erste Heft\pwindex{Strom. Organ der Wiener Freien Volksbuehne1911 – 1914@\emph{Der Strom. Organ der Wiener Freien Volksbühne} {[}1911 – 1914{]}|pwv} soll
                        Ende März erscheinen und aus diesem Grunde erbitten wir eine
                    umgehende Antwort unseres Briefes.\pend
           \pstart
           Mit aufrichtiger Hochschätzung{\\[\baselineskip]} sehr ergeben{\\[\baselineskip]}\spacefill\mbox{Pernerstorfer}\hspace*{1.5em}\spacefill\mbox{Stefan Großmann}\pend
           \leftskip=0em{}
         
         \endnumbering\mylabel{h}\end{ledgroupsized}  \newcommand{\dateiname}{L02013}\newcommand{\titel}{Engelbert Pernerstorfer und Stefan Großmann an Arthur Schnitzler, 14. 3. 1911}\newcommand{\editorInnen}{Martin Anton Müller und Gerd-Hermann Susen}%% latex-leseansicht-abspann.tex
%% Abspann für die Leseansicht.
%% Der Schalter \ifkorrekturansicht ist bereits durch den Vorspann gesetzt.

%% latex-abspann.tex
%% Gemeinsamer Abspann für Korrekturansicht und Leseansicht.
%% Setzt den Schalter \ifkorrekturansicht voraus (gesetzt in den
%% einbindenden Dateien latex-korrekturansicht-abspann.tex bzw.
%% latex-leseansicht-abspann.tex).
%% ---------------------------------------------------------------

\normalsize

% Das esempio-Environment wird nur in der Leseansicht benötigt
\ifkorrekturansicht\else
\newenvironment{esempio}[3]%
{
    \vspace{1.5ex}
    \rlap{\underline{#1}}
    \par
    \setlength{\parindent}{0cm}
    \nopagebreak
    \leftskip=#2cm
    \rightskip=#3cm
}
{
    \par
}
\fi

\doendnotes{C}
\bigskip
\vfill

\clearpage

\footnotesize

\ifkorrekturansicht
  \lohead{\textsc{register}}
\fi

% theindex-Environment neu definieren ohne reledmac
\makeatletter
\renewenvironment{theindex}{%
  \ifkorrekturansicht
    \section*{\indexname}%
  \else
    \subsubsection*{Index der erwähnten Entitäten}%
  \fi
  \setlength{\parindent}{0pt}%
  \setlength{\parskip}{0pt plus 0.3pt}%
  \let\item\@idxitem
}{%
  \ifkorrekturansicht\clearpage\fi
}
\makeatother

\IfFileExists{\jobname-pw.ind}{\input{\jobname-pw.ind}}{}

% Quellenangabe nur in der Leseansicht
\ifkorrekturansicht\else
% Fallback-Definitionen, falls die .tex-Datei \titel etc. nicht gesetzt hat
\providecommand{\titel}{}
\providecommand{\editorInnen}{}
\providecommand{\dateiname}{\jobname}

\vspace{3cm}

\vfill

\footnotesize
\textsc{Quelle}: \titel. Herausgegeben von {\editorInnen}. In: \emph{Arthur Schnitzler: Briefwechsel mit Autorinnen und Autoren}.
 Digitale Edition, https://schnitzler-briefe.acdh.oeaw.ac.at/{\dateiname}.html (Stand \today)
\fi

\end{document}


      