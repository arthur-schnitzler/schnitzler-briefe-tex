%% latex-leseansicht-vorspann.tex
%% Vorspann für die Leseansicht.
%% Lädt die gemeinsame Datei latex-vorspann.tex mit nicht gesetztem Schalter.

\newif\ifkorrekturansicht
\korrekturansichtfalse

\input{../tex-inputs/latex-vorspann}


\section[Hermann Bahr an Arthur Schnitzler, 12. 2. 1907]{L01656 Hermann Bahr an Arthur Schnitzler, 12. 2. 1907}
\nopagebreak\mylabel{L01656v}
\rehead{ }\normalsize\beginnumbering\briefempfaengerindex{Schnitzler, Arthur@\textsc{Schnitzler, Arthur}!zzzBahr, Hermann@\emph{von Hermann Bahr}!1907-02-121@{12. 2. 1907}|(be}
\toendnotes[C]{\smallbreak\pagebreak[2]}
\correspDesc{Versand  durch Hermann Bahr am 12. 2. 1907 in Berlin
\newline{}Erhalt  durch Arthur Schnitzler im Zeitraum [13. 2. 1907
                  – 17. 2. 1907?] in Wien}\toendnotes[C]{\smallbreak}
\Standort{CUL, Schnitzler, B 5b.}
\physDesc{Brief, 1 Blatt, 1 Seite, 493 Zeichen
\newline{}Handschrift: schwarze Tinte, deutsche Kurrent
\newline{}Ordnung: mit Bleistift von unbekannter Hand nummeriert: »144« }
\buchAbdrucke{\weitereDrucke{Hermann Bahr, Arthur Schnitzler: \emph{Briefwechsel, Aufzeichnungen, Dokumente (1891–1931)}. Herausgegeben von Kurt Ifkovits und Martin Anton Müller. Göttingen: \emph{Wallstein} 2018, S. 389.} }\toendnotes[C]{\smallbreak}
\pstart
           \raggedleft{}{\pb}Berlin NW 6 Marienstr 18\oindex{Marienstraße@\textbf{Marienstraße}, \emph{Straße}|pw}{\\}12. 2. 07\pend
           
\pstart\center{}Lieber Artur!\pend\vspace{0.5em}
\pstart
           Es iſt möglich, daß es mir gelingt, bei \label{K_L01656-1v}\edtext{Reinhardt\pwindex{Reinhardt, Max 9.\,9.\,1873 Baden bei Wien – 30.\,10.\,1943 New York City@\textsc{Reinhardt, Max} (9.\,9.\,1873 Baden bei Wien – 30.\,10.\,1943 New York City), \emph{Theaterleiter, Regisseur, Schauspieler}|pw} »Liebelei\pwindex{Schnitzler, Arthur 15.\,5.\,1862 Wien – 21.\,10.\,1931 ebd.@\textsc{Schnitzler, Arthur} (15.\,5.\,1862 Wien – 21.\,10.\,1931 ebd.), \emph{Schriftsteller, Mediziner}!Liebelei. Schauspiel in drei Akten@\strich\emph{Liebelei. Schauspiel in drei Akten}|pw}}{\lemma{\textnormal{\emph{Reinhardt »Liebelei}}}\Cendnote{\textnormal{Am
                  19. 9. 1907 hatte die Neuinszenierung von \emph{Liebelei}\pwindex{Schnitzler, Arthur 15.\,5.\,1862 Wien – 21.\,10.\,1931 ebd.@\textsc{Schnitzler, Arthur} (15.\,5.\,1862 Wien – 21.\,10.\,1931 ebd.), \emph{Schriftsteller, Mediziner}!Liebelei. Schauspiel in drei Akten@\strich\emph{Liebelei. Schauspiel in drei Akten}|pwk} in den \emph{Berliner Kammerspielen}\orgindex{Kammerspiele Berlin@Kammerspiele Berlin|pwk} Premiere. Vgl. XXXX Auszeichnungsfehler: Dokument L03513 nicht gefunden.}}}\label{K_L01656-1}« durchzuſetzen (Höflich\pwindex{Höflich, Lucie 20.\,2.\,1883 Hannover – 8.\,10.\,1956 Schmargendorf@\textsc{Höflich, Lucie} (20.\,2.\,1883 Hannover – 8.\,10.\,1956 Schmargendorf), \emph{Schauspielerin}|pw}! Pagay\pwindex{Pagay, Hans 11.\,11.\,1843 Wien – 21.\,1.\,1915 Berlin@\textsc{Pagay, Hans} (11.\,11.\,1843 Wien – 21.\,1.\,1915 Berlin), \emph{Schauspieler}|pw}!). Ich
               arbeite{ }ſehr stark daran und dränge, es gleich nach Hedda Gabler\pwindex{\textcolor{red}{\textsuperscript{XXXX indx1}}!Hedda Gabler@\strich\emph{Hedda Gabler}|pw} zu machen. Sicher iſt es noch gar nicht, Du darfſt auch noch zu
               keinem Menſchen was{ }ſagen, ich möchte aber für alle Fälle raſcheſtens ein Buch haben,
               um mir meine Inſcenierung ruhiger zu überlegen, als es{ }ſpäter geſchehen kann.\pend
           
\pstart
           In größter Eile{\\[\baselineskip]}mir vielen Grüßen an Deine Frau\pwindex{Schnitzler, Olga 17.\,1.\,1882 Wien – 13.\,1.\,1970 Lugano@\textsc{Schnitzler, Olga} (17.\,1.\,1882 Wien – 13.\,1.\,1970 Lugano), \emph{Schauspielerin, Sängerin}|pwv}{\\[\baselineskip]}herzlichſt{\\[\baselineskip]}\spacefill\mbox{Hermann}\pend
           \leftskip=0em{}\selectlanguage{ngerman}\endnumbering\briefempfaengerindex{Schnitzler, Arthur@\textsc{Schnitzler, Arthur}!zzzBahr, Hermann@\emph{von Hermann Bahr}!1907-02-121@{12. 2. 1907}|)be}\mylabel{L01656h}  \newcommand{\dateiname}{L01656}\newcommand{\titel}{Hermann Bahr an Arthur Schnitzler, 12. 2. 1907}\newcommand{\editorInnen}{Herausgegeben von Martin Anton Müller}%% latex-leseansicht-abspann.tex
%% Abspann für die Leseansicht.
%% Der Schalter \ifkorrekturansicht ist bereits durch den Vorspann gesetzt.

%% latex-abspann.tex
%% Gemeinsamer Abspann für Korrekturansicht und Leseansicht.
%% Setzt den Schalter \ifkorrekturansicht voraus (gesetzt in den
%% einbindenden Dateien latex-korrekturansicht-abspann.tex bzw.
%% latex-leseansicht-abspann.tex).
%% ---------------------------------------------------------------

\normalsize

% Das esempio-Environment wird nur in der Leseansicht benötigt
\ifkorrekturansicht\else
\newenvironment{esempio}[3]%
{
    \vspace{1.5ex}
    \rlap{\underline{#1}}
    \par
    \setlength{\parindent}{0cm}
    \nopagebreak
    \leftskip=#2cm
    \rightskip=#3cm
}
{
    \par
}
\fi

\doendnotes{C}
\bigskip
\vfill

\clearpage

\footnotesize

\ifkorrekturansicht
  \lohead{\textsc{register}}
\fi

% theindex-Environment neu definieren ohne reledmac
\makeatletter
\renewenvironment{theindex}{%
  \ifkorrekturansicht
    \section*{\indexname}%
  \else
    \subsubsection*{Index der erwähnten Entitäten}%
  \fi
  \setlength{\parindent}{0pt}%
  \setlength{\parskip}{0pt plus 0.3pt}%
  \let\item\@idxitem
}{%
  \ifkorrekturansicht\clearpage\fi
}
\makeatother

\IfFileExists{\jobname-pw.ind}{\input{\jobname-pw.ind}}{}

% Quellenangabe nur in der Leseansicht
\ifkorrekturansicht\else
% Fallback-Definitionen, falls die .tex-Datei \titel etc. nicht gesetzt hat
\providecommand{\titel}{}
\providecommand{\editorInnen}{}
\providecommand{\dateiname}{\jobname}

\vspace{3cm}

\vfill

\footnotesize
\textsc{Quelle}: \titel. Herausgegeben von {\editorInnen}. In: \emph{Arthur Schnitzler: Briefwechsel mit Autorinnen und Autoren}.
 Digitale Edition, https://schnitzler-briefe.acdh.oeaw.ac.at/{\dateiname}.html (Stand \today)
\fi

\end{document}


