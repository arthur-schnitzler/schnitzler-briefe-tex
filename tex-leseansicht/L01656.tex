%% latex-korrekturansicht-vorspann.tex
%% Vorspann für die Korrekturansicht.
%% Lädt die gemeinsame Datei latex-vorspann.tex mit gesetztem Schalter.

\newif\ifkorrekturansicht
\korrekturansichttrue

\input{../tex-inputs/latex-vorspann}


\section[Hermann Bahr an Arthur Schnitzler, 12. 2. 1907]{L01656 Hermann Bahr an Arthur Schnitzler, 12. 2. 1907}
\nopagebreak\mylabel{L01656v}
\rehead{ }\normalsize\beginnumbering\briefempfaengerindex{Schnitzler, Arthur@\textsc{Schnitzler, Arthur}!zzzBahr, Hermann@\emph{von Hermann Bahr}!1907-02-121@{12. 2. 1907}|(be}
\toendnotes[C]{\smallbreak\pagebreak[2]}\Standort{CUL, Schnitzler, B 5b.}
\physDesc{Brief, 1 Blatt, 1 Seite, 493 Zeichen
\newline{}Handschrift: schwarze Tinte, deutsche Kurrent
\newline{}Ordnung: mit Bleistift von unbekannter Hand nummeriert: »144« }
\buchAbdrucke{\weitereDrucke{Hermann Bahr, Arthur Schnitzler: \emph{Briefwechsel, Aufzeichnungen, Dokumente (1891–1931)}. Göttingen: \emph{Wallstein} 2018, S. 389.} }\toendnotes[C]{\smallbreak}
\pstart
           \raggedleft{}{\pb}Berlin NW 6 Marienstr 18\oindex{Marienstrasse@\textbf{Marienstraße}, \emph{Straße (K.STR)}|pw}{\\}12. 2. 07\pend
           
\pstart\center{}Lieber Artur!\pend\vspace{0.5em}
\pstart
           Es iſt möglich, daß es mir gelingt, bei \label{K_L01656-1v}\edtext{Reinhardt\pwindex{Reinhardt, Max 09.09.1873 – 30.10.1943@\textsc{Reinhardt, Max} (09.09.1873 – 30.10.1943), \emph{Theaterleiter/Theaterleiterin, Regisseur/Regisseurin, Schauspieler/Schauspielerin}|pw} »Liebelei\pwindex{Liebelei. Schauspiel in drei Akten@\emph{Liebelei. Schauspiel in drei Akten}|pw}}{\lemma{\textnormal{\emph{Reinhardt »Liebelei}}}\Cendnote{\textnormal{Am
                  19. 9. 1907 hatte die Neuinszenierung von \emph{Liebelei}\pwindex{Liebelei. Schauspiel in drei Akten@\emph{Liebelei. Schauspiel in drei Akten}|pwk} in den \emph{Berliner Kammerspielen}\orgindex{Kammerspiele Berlin@Kammerspiele Berlin|pwk} Premiere. Vgl. Felix Salten an Arthur Schnitzler, 15. 10. 1907.}}}\label{K_L01656-1}« durchzuſetzen (Höflich\pwindex{Hoeflich, Lucie 20.02.1883 – 08.10.1956@\textsc{Höflich, Lucie} (20.02.1883 – 08.10.1956), \emph{Schauspieler/Schauspielerin}|pw}! Pagay\pwindex{Pagay, Hans 1843-11-11 – 1915-01-21@\textsc{Pagay, Hans} (1843-11-11 – 1915-01-21), \emph{Schauspieler/Schauspielerin}|pw}!). Ich
               arbeite ſehr stark daran und dränge, es gleich nach Hedda Gabler\pwindex{Hedda Gabler@\emph{Hedda Gabler}|pw} zu machen. Sicher iſt es noch gar nicht, Du darfſt auch noch zu
               keinem Menſchen was ſagen, ich möchte aber für alle Fälle raſcheſtens ein Buch haben,
               um mir meine Inſcenierung ruhiger zu überlegen, als es ſpäter geſchehen kann.\pend
           
\pstart
           In größter Eile{\\[\baselineskip]}mir vielen Grüßen an Deine Frau\pwindex{Schnitzler, Olga 17.01.1882 – 13.01.1970@\textsc{Schnitzler, Olga} (17.01.1882 – 13.01.1970), \emph{Schauspieler/Schauspielerin, Sänger/Sängerin}|pwv}{\\[\baselineskip]}herzlichſt{\\[\baselineskip]}\spacefill\mbox{Hermann}\pend
           \leftskip=0em{}\selectlanguage{ngerman}\endnumbering\briefempfaengerindex{Schnitzler, Arthur@\textsc{Schnitzler, Arthur}!zzzBahr, Hermann@\emph{von Hermann Bahr}!1907-02-121@{12. 2. 1907}|)be}\mylabel{L01656h}  \normalsize

\doendnotes{C}
\bigskip
\vfill

\clearpage

\footnotesize

\lohead{\textsc{register}}

% Definiere theindex-Environment komplett neu ohne reledmac
\makeatletter
\renewenvironment{theindex}{%
  \section*{\indexname}%
  \setlength{\parindent}{0pt}%
  \setlength{\parskip}{0pt plus 0.3pt}%
  \let\item\@idxitem
}{%
  \clearpage
}
\makeatother

\IfFileExists{\jobname-pw.ind}{\input{\jobname-pw.ind}}{}

\end{document}

      