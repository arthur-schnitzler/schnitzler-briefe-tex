%% latex-korrekturansicht-vorspann.tex
%% Vorspann für die Korrekturansicht.
%% Lädt die gemeinsame Datei latex-vorspann.tex mit gesetztem Schalter.

\newif\ifkorrekturansicht
\korrekturansichttrue

\input{../tex-inputs/latex-vorspann}


\section[ Paul Goldmann an Arthur Schnitzler, 23. 1. {[}1900{]}]{L02903 Paul Goldmann an Arthur Schnitzler, 23. 1. {[}1900{]}}
\nopagebreak\mylabel{L02903v}
\rehead{ }\normalsize\beginnumbering\briefempfaengerindex{Schnitzler, Arthur@\textsc{Schnitzler, Arthur}!zzzGoldmann, Paul@\emph{von Paul Goldmann}!1900-01-231@{23. 1. {[}1900{]}}|(be}
\toendnotes[C]{\smallbreak\pagebreak[2]}\Standort{DLA, A:Schnitzler, HS.NZ85.1.3170.}
\physDesc{Brief, 1 Blatt, 4 Seiten, 1515 Zeichen
\newline{}Handschrift: schwarze Tinte, deutsche Kurrent
\newline{}Schnitzler: 1) mit Bleistift das Jahr »900« vermerkt  2) mit rotem Buntstift eine Unterstreichung}\toendnotes[C]{\smallbreak}
\pstart
           \centering{}{\pb}\textcolor{gray}{\textbf{\textbf{HOTEL SAXONIA\oindex{Hotel Saxonia@\textbf{Hotel Saxonia}, \emph{Hotel (K.HTL)}|pw}}}}\pend
           
\pstart
           \raggedleft{}\textcolor{gray}{\textbf{am Potsdamer Platz\oindex{Potsdamer Platz@\textbf{Potsdamer Platz}, \emph{Platz (K.PLT)}|pw} und
                        Thiergarten\oindex{Tiergarten@\textbf{Tiergarten}, \emph{P.PPLX}|pw}}}\pend
           
\pstart
           \centering{}\textcolor{gray}{\textbf{D. W. SCHRÖDER\pwindex{Schroeder, D. W. @\textsc{Schröder, D. W.}, \emph{Hotelbesitzer/Hotelbesitzerin}|pw}.}}\pend
           
\pstart
           \textcolor{gray}{\textbf{Fernsprecher:}}\pend
           
\pstart
           \textcolor{gray}{\textbf{\textbf{Amt VI. No. 2838.}}}\pend
           
\pstart
           \raggedleft{}\textcolor{gray}{\textbf{\emph{BERLIN W.}\oindex{Berlin@\textbf{Berlin}, \emph{P.PPLC}|pw}, den}}{ }23. Janua\substVorne{}\textsuperscript{\textcolor{gray}{\textbf{1}}}\substDazwischen{}r\substHinten{}\pend
           
\pstart
           \raggedleft{}\textcolor{gray}{\textbf{Königgrätzerstrasse 10\oindex{Stresemannstrasse@\textbf{Stresemannstraße}, \emph{Straße (K.STR)}|pw}.}}\pend
           
\pstart{}Mein lieber Freund,\pend\vspace{0.5em}
\pstart
           Ich danke Dir für Deinen lieben Brief. Gern hätte ich Dir längſt ſchon geſchrieben,
               habe aber unendlich wenig Zeit.\pend
           
\pstart
           Gegen Deine \label{K_L02903-1v}\edtext{Hypochondrie}{\lemma{\textnormal{\emph{Hypochondrie}}}\Cendnote{\textnormal{Zur Hypochondrie, die sich zu diesem
                  Zeitpunkt wohl primär auf Schnitzlers Otosklerose zurückführen ließ, siehe etwa A. S.: \emph{Tagebuch}, 26. 12. 1899. Schnitzler leistete
               dem Rat von Goldmann\pwindex{Goldmann, Paul 31.01.1865 – 25.09.1935@\textsc{Goldmann, Paul} (31.01.1865 – 25.09.1935), \emph{Schriftsteller/Schriftstellerin, Journalist/Journalistin}|pwk} keine Folge und verreiste 
                  nicht.}}}\label{K_L02903-1} weiß ich nur \uline{ein} Mittel: Reiſen.
               Komm’ nach Berlin\oindex{Berlin@\textbf{Berlin}, \emph{P.PPLC}|pw}! Oder geh’ nach Florenz\oindex{Florenz@\textbf{Florenz}, \emph{P.PPLA}|pw}!\pend
           
\pstart
           \strikeout{\textcolor{gray}{Bei}} In der \label{K_L02903-2v}\edtext{Paſſauer Straße\oindex{Passauerstrasse@\textbf{Passauerstraße}, \emph{Straße (K.STR)}|pw}}{\lemma{\textnormal{\emph{Paſſauer Straße}}}\Cendnote{\textnormal{Siehe Paul Goldmann, Marie Glümer, Auguste Chlum und Moritz Coschell an
               Arthur Schnitzler, 11. 1. 1900.
               }}}\label{K_L02903-2} bin ich hier und da. Sehr liebe Frauen\pwindex{Gluemer, Auguste 1862-03-16 – 1956@\textsc{Glümer, Auguste} (1862-03-16 – 1956), \emph{Lehrer/Lehrerin}|pw}\pwindex{Gluemer, Marie 03.07.1867 – 16.11.1925@\textsc{Glümer, Marie} (03.07.1867 – 16.11.1925), \emph{Schauspieler/Schauspielerin}|pw}. \strikeout{Ab} Aber was ſoll ich Dir {\pb}von ihnen\pwindex{Gluemer, Auguste 1862-03-16 – 1956@\textsc{Glümer, Auguste} (1862-03-16 – 1956), \emph{Lehrer/Lehrerin}|pwv}\pwindex{Gluemer, Marie 03.07.1867 – 16.11.1925@\textsc{Glümer, Marie} (03.07.1867 – 16.11.1925), \emph{Schauspieler/Schauspielerin}|pwv} oder von ihr\pwindex{Gluemer, Marie 03.07.1867 – 16.11.1925@\textsc{Glümer, Marie} (03.07.1867 – 16.11.1925), \emph{Schauspieler/Schauspielerin}|pwv} ſchreiben? Ich finde ſie ſehr anſtändig, ſehr gut, ſehr
               ſympathiſch. Und doch (offen geſtanden) habe ich kein rechtes inneres Intereſſe mehr
               für ſie. Das Alles iſt einmal geweſen. Vergangene Zeiten, zu denen man nicht mehr
               zurück kann. Es iſt unſere Jugend – aber unſere Jugend, die ſich nicht von der Stelle
               gerührt hat und alt geworden iſt. Wir aber ſind inzwiſchen nicht nur älter, ſondern
               auch \uline{anders} geworden.\pend
           
\pstart
           Auch über dieſe Theaterdamen-Zigeunerwirthſchaften bin ich hinausgewachſen. Es
               amüſirt mich nicht mehr\textcolor{gray}{,} es macht mich \strikeout{\textcolor{gray}{trau}} traurig. {\pb}Ich habe nur \uline{eine} Sehnſucht: geordnete Verhältniſſe, Wohlſtand, Ruhe, \label{K_L02903-3v}\edtext{\uuline{Ehe}}{\lemma{\textnormal{\emph{Ehe}}}\Cendnote{\textnormal{Trotz des häufig geäußerten Wunschs, zu heiraten, schloss Goldmann\pwindex{Goldmann, Paul 31.01.1865 – 25.09.1935@\textsc{Goldmann, Paul} (31.01.1865 – 25.09.1935), \emph{Schriftsteller/Schriftstellerin, Journalist/Journalistin}|pwk} erst 
                  1908 eine Ehe (mit Eva Marie Fränkel\pwindex{Goldmann, Eva Marie 27.10.1877 – 02.11.1937@\textsc{Goldmann, Eva Marie} (27.10.1877 – 02.11.1937)|pwk}).}}}\label{K_L02903-3}. Ich ſuche ein
               ſympathiſches, nicht allzu künſtleriſches und vermögendes Mädchen. Wenn Du eine
               weißt, kannſt Du die Parthie zuſammenbringen. Du kriegſt Prozente von der
               Mitgift.\pend
           
\pstart
           Der Wunſch, mich zu verheirathen und zu verſorgen, – noch raſch in den letzten paar
               Jahren, ehe es zu ſpät iſt, – läßt mich nicht mehr los. Mein ganzes Leben lang bin
               ich ein Arbeitsthier geweſen und habe auf Alles verzichten müſſen. Werde ich auch das
               nicht erreichen? Es ſieht {\pb}beinahe ſo aus.\pend
           
\pstart
           Schreib’ mir bald!\pend
           
\pstart
           Grüße mir den \textsc{Richard\pwindex{Beer-Hofmann, Richard 1866-07-11 – 1945-09-26@\textsc{Beer-Hofmann, Richard} (1866-07-11 – 1945-09-26), \emph{Schriftsteller/Schriftstellerin}|pw}}! (Was macht er?)\pend
           
\pstart
           Viele treue Grüße! {\\[\baselineskip]}Dein {\\[\baselineskip]}\spacefill\mbox{Paul Goldmann.}\pend
           \leftskip=0em{}\selectlanguage{ngerman}\endnumbering\briefempfaengerindex{Schnitzler, Arthur@\textsc{Schnitzler, Arthur}!zzzGoldmann, Paul@\emph{von Paul Goldmann}!1900-01-231@{23. 1. {[}1900{]}}|)be}\mylabel{L02903h}  \normalsize

\doendnotes{C}
\bigskip
\vfill

\clearpage

\footnotesize

\lohead{\textsc{register}}

% Definiere theindex-Environment komplett neu ohne reledmac
\makeatletter
\renewenvironment{theindex}{%
  \section*{\indexname}%
  \setlength{\parindent}{0pt}%
  \setlength{\parskip}{0pt plus 0.3pt}%
  \let\item\@idxitem
}{%
  \clearpage
}
\makeatother

\IfFileExists{\jobname-pw.ind}{\input{\jobname-pw.ind}}{}

\end{document}

      