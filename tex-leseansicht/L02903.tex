%% latex-leseansicht-vorspann.tex
%% Vorspann für die Leseansicht.
%% Lädt die gemeinsame Datei latex-vorspann.tex mit nicht gesetztem Schalter.

\newif\ifkorrekturansicht
\korrekturansichtfalse

\input{../tex-inputs/latex-vorspann}


\section[ Paul Goldmann an Arthur Schnitzler, 23. 1. [1900]]{L02903 Paul Goldmann an Arthur Schnitzler,  23. 1. [1900]}
\nopagebreak\mylabel{L02903v}
\rehead{ }\normalsize\beginnumbering\briefempfaengerindex{Schnitzler, Arthur@\textsc{Schnitzler, Arthur}!zzzGoldmann, Paul@\emph{von Paul Goldmann}!1900-01-231@{23. 1. [1900]}|(be}
\toendnotes[C]{\smallbreak\pagebreak[2]}
\correspDesc{Versand  durch Paul Goldmann am 23. 1. [1900] in Berlin
\newline{}Erhalt  durch Arthur Schnitzler im Zeitraum [24. 1. 1900
                  – 28. 1. 1900?] in Wien}\toendnotes[C]{\smallbreak}
\Standort{DLA, A:Schnitzler, HS.NZ85.1.3170.}
\physDesc{Brief, 1 Blatt, 4 Seiten, 1515 Zeichen
\newline{}Handschrift: schwarze Tinte, deutsche Kurrent
\newline{}Schnitzler: 1) mit Bleistift das Jahr »900« vermerkt  2) mit rotem Buntstift eine Unterstreichung}\toendnotes[C]{\smallbreak}
\pstart
           \centering{}{\pb}\textcolor{gray}{\textbf{\textbf{HOTEL SAXONIA\oindex{Hotel Saxonia@\textbf{Hotel Saxonia}, \emph{Hotel}|pw}}}}\pend
           
\pstart
           \raggedleft{}\textcolor{gray}{\textbf{am Potsdamer Platz\oindex{Potsdamer Platz@\textbf{Potsdamer Platz}, \emph{Platz}|pw} und
                        Thiergarten\oindex{Tiergarten@\textbf{Tiergarten}, \emph{Ehemaliger Ort}|pw}}}\pend
           
\pstart
           \centering{}\textcolor{gray}{\textbf{D. W. SCHRÖDER\pwindex{Schröder, D. W. @\textsc{Schröder, D. W.}, \emph{Hotelbesitzer/Hotelbesitzerin}|pw}.}}\pend
           
\pstart
           \textcolor{gray}{\textbf{Fernsprecher:}}\pend
           
\pstart
           \textcolor{gray}{\textbf{\textbf{Amt VI. No. 2838.}}}\pend
           
\pstart
           \raggedleft{}\textcolor{gray}{\textbf{\emph{BERLIN W.}\oindex{Berlin@\textbf{Berlin}, \emph{Hauptstadt}|pw}, den}}{ }23. Janua\substVorne{}\textsuperscript{\textcolor{gray}{\textbf{1}}}\substDazwischen{}r\substHinten{}\pend
           
\pstart
           \raggedleft{}\textcolor{gray}{\textbf{Königgrätzerstrasse 10\oindex{Stresemannstraße@\textbf{Stresemannstraße}, \emph{Straße}|pw}.}}\pend
           
\pstart{}Mein lieber Freund,\pend\vspace{0.5em}
\pstart
           Ich danke Dir für Deinen lieben Brief. Gern hätte ich Dir längſt{ }ſchon geſchrieben,
               habe aber unendlich wenig Zeit.\pend
           
\pstart
           Gegen Deine \label{K_L02903-1v}\edtext{Hypochondrie}{\lemma{\textnormal{\emph{Hypochondrie}}}\Cendnote{\textnormal{Zur Hypochondrie, die sich zu diesem
                  Zeitpunkt wohl primär auf Schnitzlers Otosklerose zurückführen ließ, siehe etwa A. S.: \emph{Tagebuch}, 26. 12. 1899. Schnitzler leistete
               dem Rat von Goldmann\pwindex{Goldmann, Paul 31.\,1.\,1865 Breslau – 25.\,9.\,1935 Wien@\textsc{Goldmann, Paul} (31.\,1.\,1865 Breslau – 25.\,9.\,1935 Wien), \emph{Schriftsteller, Journalist}|pwk} keine Folge und verreiste 
                  nicht.}}}\label{K_L02903-1} weiß ich nur \uline{ein} Mittel: Reiſen.
               Komm’ nach Berlin\oindex{Berlin@\textbf{Berlin}, \emph{Hauptstadt}|pw}! Oder geh’ nach Florenz\oindex{Florenz@\textbf{Florenz}|pw}!\pend
           
\pstart
           \strikeout{\textcolor{gray}{Bei}} In der \label{K_L02903-2v}\edtext{Paſſauer Straße\oindex{Passauerstraße@\textbf{Passauerstraße}, \emph{Straße}|pw}}{\lemma{\textnormal{\emph{Passauer Straße}}}\Cendnote{\textnormal{Siehe XXXX Auszeichnungsfehler: Dokument L02902 nicht gefunden.
               }}}\label{K_L02903-2} bin ich hier und da. Sehr liebe Frauen\pwindex{Glümer, Auguste 16.\,3.\,1862 Wien – 1956@\textsc{Glümer, Auguste} (16.\,3.\,1862 Wien – 1956), \emph{Lehrerin}|pw}\pwindex{Glümer, Marie 3.\,7.\,1867 Wien – 16.\,11.\,1925 München@\textsc{Glümer, Marie} (3.\,7.\,1867 Wien – 16.\,11.\,1925 München), \emph{Schauspielerin}|pw}. \strikeout{Ab} Aber was{ }ſoll ich Dir {\pb}von ihnen\pwindex{Glümer, Auguste 16.\,3.\,1862 Wien – 1956@\textsc{Glümer, Auguste} (16.\,3.\,1862 Wien – 1956), \emph{Lehrerin}|pwv}\pwindex{Glümer, Marie 3.\,7.\,1867 Wien – 16.\,11.\,1925 München@\textsc{Glümer, Marie} (3.\,7.\,1867 Wien – 16.\,11.\,1925 München), \emph{Schauspielerin}|pwv} oder von ihr\pwindex{Glümer, Marie 3.\,7.\,1867 Wien – 16.\,11.\,1925 München@\textsc{Glümer, Marie} (3.\,7.\,1867 Wien – 16.\,11.\,1925 München), \emph{Schauspielerin}|pwv}{ }ſchreiben? Ich finde{ }ſie{ }ſehr anſtändig,{ }ſehr gut,{ }ſehr{ }ſympathiſch. Und doch (offen geſtanden) habe ich kein rechtes inneres Intereſſe mehr
               für{ }ſie. Das Alles iſt einmal geweſen. Vergangene Zeiten, zu denen man nicht mehr
               zurück kann. Es iſt unſere Jugend – aber unſere Jugend, die{ }ſich nicht von der Stelle
               gerührt hat und alt geworden iſt. Wir aber{ }ſind inzwiſchen nicht nur älter,{ }ſondern
               auch \uline{anders} geworden.\pend
           
\pstart
           Auch über dieſe Theaterdamen-Zigeunerwirthſchaften bin ich hinausgewachſen. Es
               amüſirt mich nicht mehr\textcolor{gray}{,} es macht mich \strikeout{\textcolor{gray}{trau}} traurig. {\pb}Ich habe nur \uline{eine} Sehnſucht: geordnete Verhältniſſe, Wohlſtand, Ruhe, \label{K_L02903-3v}\edtext{\uuline{Ehe}}{\lemma{\textnormal{\emph{Ehe}}}\Cendnote{\textnormal{Trotz des häufig geäußerten Wunschs, zu heiraten, schloss Goldmann\pwindex{Goldmann, Paul 31.\,1.\,1865 Breslau – 25.\,9.\,1935 Wien@\textsc{Goldmann, Paul} (31.\,1.\,1865 Breslau – 25.\,9.\,1935 Wien), \emph{Schriftsteller, Journalist}|pwk} erst 
                  1908 eine Ehe (mit Eva Marie Fränkel\pwindex{Goldmann, Eva Marie 27.\,10.\,1877 Wien – 2.\,11.\,1937 ebd.@\textsc{Goldmann, Eva Marie} (27.\,10.\,1877 Wien – 2.\,11.\,1937 ebd.)|pwk}).}}}\label{K_L02903-3}. Ich{ }ſuche ein{ }ſympathiſches, nicht allzu künſtleriſches und vermögendes Mädchen. Wenn Du eine
               weißt, kannſt Du die Parthie zuſammenbringen. Du kriegſt Prozente von der
               Mitgift.\pend
           
\pstart
           Der Wunſch, mich zu verheirathen und zu verſorgen, – noch raſch in den letzten paar
               Jahren, ehe es zu{ }ſpät iſt, – läßt mich nicht mehr los. Mein ganzes Leben lang bin
               ich ein Arbeitsthier geweſen und habe auf Alles verzichten müſſen. Werde ich auch das
               nicht erreichen? Es{ }ſieht {\pb}beinahe{ }ſo aus.\pend
           
\pstart
           Schreib’ mir bald!\pend
           
\pstart
           Grüße mir den \textsc{Richard\pwindex{Beer-Hofmann, Richard 11.\,7.\,1866 Wien – 26.\,9.\,1945 New York City@\textsc{Beer-Hofmann, Richard} (11.\,7.\,1866 Wien – 26.\,9.\,1945 New York City), \emph{Schriftsteller}|pw}}! (Was macht er?)\pend
           
\pstart
           Viele treue Grüße! {\\[\baselineskip]}Dein {\\[\baselineskip]}\spacefill\mbox{Paul Goldmann.}\pend
           \leftskip=0em{}\selectlanguage{ngerman}\endnumbering\briefempfaengerindex{Schnitzler, Arthur@\textsc{Schnitzler, Arthur}!zzzGoldmann, Paul@\emph{von Paul Goldmann}!1900-01-231@{23. 1. [1900]}|)be}\mylabel{L02903h}  \newcommand{\dateiname}{L02903}\newcommand{\titel}{Paul Goldmann an Arthur Schnitzler, 23. 1. [1900]}\newcommand{\editorInnen}{Martin Anton Müller und Laura Untner}%% latex-leseansicht-abspann.tex
%% Abspann für die Leseansicht.
%% Der Schalter \ifkorrekturansicht ist bereits durch den Vorspann gesetzt.

%% latex-abspann.tex
%% Gemeinsamer Abspann für Korrekturansicht und Leseansicht.
%% Setzt den Schalter \ifkorrekturansicht voraus (gesetzt in den
%% einbindenden Dateien latex-korrekturansicht-abspann.tex bzw.
%% latex-leseansicht-abspann.tex).
%% ---------------------------------------------------------------

\normalsize

% Das esempio-Environment wird nur in der Leseansicht benötigt
\ifkorrekturansicht\else
\newenvironment{esempio}[3]%
{
    \vspace{1.5ex}
    \rlap{\underline{#1}}
    \par
    \setlength{\parindent}{0cm}
    \nopagebreak
    \leftskip=#2cm
    \rightskip=#3cm
}
{
    \par
}
\fi

\doendnotes{C}
\bigskip
\vfill

\clearpage

\footnotesize

\ifkorrekturansicht
  \lohead{\textsc{register}}
\fi

% theindex-Environment neu definieren ohne reledmac
\makeatletter
\renewenvironment{theindex}{%
  \ifkorrekturansicht
    \section*{\indexname}%
  \else
    \subsubsection*{Index der erwähnten Entitäten}%
  \fi
  \setlength{\parindent}{0pt}%
  \setlength{\parskip}{0pt plus 0.3pt}%
  \let\item\@idxitem
}{%
  \ifkorrekturansicht\clearpage\fi
}
\makeatother

\IfFileExists{\jobname-pw.ind}{\input{\jobname-pw.ind}}{}

% Quellenangabe nur in der Leseansicht
\ifkorrekturansicht\else
% Fallback-Definitionen, falls die .tex-Datei \titel etc. nicht gesetzt hat
\providecommand{\titel}{}
\providecommand{\editorInnen}{}
\providecommand{\dateiname}{\jobname}

\vspace{3cm}

\vfill

\footnotesize
\textsc{Quelle}: \titel. Herausgegeben von {\editorInnen}. In: \emph{Arthur Schnitzler: Briefwechsel mit Autorinnen und Autoren}.
 Digitale Edition, https://schnitzler-briefe.acdh.oeaw.ac.at/{\dateiname}.html (Stand \today)
\fi

\end{document}


