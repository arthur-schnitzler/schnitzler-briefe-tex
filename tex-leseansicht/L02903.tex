%% latex-leseansicht-vorspann.tex
%% Vorspann für die Leseansicht.
%% Lädt die gemeinsame Datei latex-vorspann.tex mit nicht gesetztem Schalter.

\newif\ifkorrekturansicht
\korrekturansichtfalse

\input{../tex-inputs/latex-vorspann}

\begin{center}
            \textcolor{red}{ENTWURF, NICHT FERTIG KORRIGIERT}
                      \end{center}
            
         
         \renewcommand{\erwaehntePersonen}{Personen: Richard Beer-Hofmann, Marie Glümer, Eva Marie Goldmann, D. W. Schröder}
         \renewcommand{\erwaehnteOrte}{Orte: Berlin, Florenz, Hotel Saxonia, Passauerstraße, Potsdamer Platz, Stresemannstraße, Tiergarten, Wien}
         \renewcommand{\erwaehnteWerke}{
               \section[ Paul Goldmann an Arthur Schnitzler, 23. 1. {[}1900{]}]{ Paul Goldmann an Arthur Schnitzler, 23. 1. {[}1900{]}}\nopagebreak\mylabel{v}\rehead{ }\begin{ledgroupsized}[t]{13cm}\normalsize\beginnumbering \toendnotes[C]{\smallbreak\pagebreak[2]} \Standort{DLA, A:Schnitzler, HS.NZ85.1.3170.}
\physDesc{Brief, 1 Blatt, 4 Seiten
\newline{}Handschrift: schwarze Tinte, deutsche Kurrent
\newline{}Schnitzler: 1) mit Bleistift das Jahr »{[}1{]}900« vermerkt  2) mit rotem Buntstift eine Unterstreichung}\toendnotes[C]{\smallbreak}\pstart{}{\pb}\textcolor{gray}{\textbf{\textbf{HOTEL SAXONIA\oindex{Hotel Saxonia@\textbf{Hotel Saxonia}|pw}}}}\pend{}\pstart{}\textcolor{gray}{\textbf{am Potsdamer Platz\oindex{Potsdamer Platz@\textbf{Potsdamer Platz}|pw} und
                        Thiergarten\oindex{Tiergarten@\textbf{Tiergarten}|pw}}}\pend{}\pstart{}\textcolor{gray}{\textbf{D. W. SCHRÖDER\pwindex{Schroeder, D. W. @\textsc{Schröder, D. W.}, \emph{Hotelbesitzer/Hotelbesitzerin}|pw}.}}\pend{}\pstart{}\textcolor{gray}{\textbf{Fernsprecher:}}\pend{}\pstart{}\textcolor{gray}{\textbf{\textbf{Amt VI. No. 2838.}}}\pend{}{\bigskip}\pstart
           \raggedleft{}\textcolor{gray}{\textbf{Berlin W.\oindex{Berlin@\textbf{Berlin}|pw}, den}}{ }23. Janua\substVorne{}\textsuperscript{\textcolor{gray}{\textbf{1}}}\substDazwischen{}r\substHinten{}{ }\textcolor{gray}{\textbf{Königgrätzerstrasse 10\oindex{Stresemannstrasse@\textbf{Stresemannstraße}|pw}.}}\pend
           \pstart{}Mein lieber \substVorne{}\textsuperscript{\textcolor{gray}{×}}\substDazwischen{}F\substHinten{}reund,\pend\pstart
           Ich danke Dir für Deinen lieben Brief. Gern hätte ich Dir längſt ſchon geſchrieben,
               habe aber unendlich wenig Zeit.\pend
           \pstart
           Gegen Deine \label{K_L02903-1v}\edtext{Hypochondrie}{\lemma{\textnormal{\emph{Hypochondrie}}}\Cendnote{\textnormal{Zur Hypochondrie, die sich zu diesem
                  Zeitpunkt wohl primär auf Schnitzler\pwindex{Schnitzler, Arthur 15.05.1862 – 21.10.1931@\textsc{Schnitzler, Arthur} (15.05.1862 – 21.10.1931), \emph{Schriftsteller, Mediziner}|pwk}s
                  Otosklerose zurückführen lässt, siehe etwa A. S.: \emph{Tagebuch}, 26. 12. 1899. Schnitzler\pwindex{Schnitzler, Arthur 15.05.1862 – 21.10.1931@\textsc{Schnitzler, Arthur} (15.05.1862 – 21.10.1931), \emph{Schriftsteller, Mediziner}|pwk}
                  verreiste auf Goldmann\pwindex{Goldmann, Paul 31.01.1865 – 25.09.1935@\textsc{Goldmann, Paul} (31.01.1865 – 25.09.1935), \emph{Schriftsteller, Journalist}|pwk}s Vorschlag hin
                  nicht.}}}\label{K_L02903-1h} weiß ich nur \uline{ein} Mittel: Reiſen.
               Komm’ nach Berlin\oindex{Berlin@\textbf{Berlin}|pw}! Oder geh’ nach Florenz\oindex{Florenz@\textbf{Florenz}|pw}!\pend
           \pstart
           \strikeout{\textcolor{gray}{Bei}} In der \label{K_L02903-2v}\edtext{Paſſauer Straße\oindex{Passauerstrasse@\textbf{Passauerstraße}|pw}}{\lemma{\textnormal{\emph{Paſſauer Straße}}}\Cendnote{\textnormal{siehe Paul Goldmann, Marie Glümer, Auguste Chlum und Moritz Coschell an
               Arthur Schnitzler, 11. 1. 1900}}}\label{K_L02903-2h} bin ich hier und da. Sehr liebe Frauen. \strikeout{Ab}
               Aber was ſoll ich Dir {\pb}von ihnen oder von ihr\pwindex{Gluemer, Marie 03.07.1867 – 16.11.1925@\textsc{Glümer, Marie} (03.07.1867 – 16.11.1925), \emph{Schauspielerin}|pwuv} ſchreiben?
               Ich finde ſie ſehr anſtändig, ſehr gut, ſehr ſympathiſch. Und doch (offen geſtanden)
               habe ich kein rechtes inneres Intereſſe mehr für ſie. Das Alles iſt einmal geweſen.
               Vergangene Zeiten, zu denen man nicht mehr zurück kann. Es iſt unſere Jugend – aber
               unſere Jugend, die ſich nicht von der Stelle gerührt hat und alt geworden iſt. Wir
               aber ſind inzwiſchen nicht nur älter, ſondern auch \uline{anders} geworden.\pend
           \pstart
           Auch über dieſe Theaterdamenzigeunerwirthſchaften bin ich hinausgewachſen. Es amüſirt
               mich nicht mehr\textcolor{gray}{,} es macht mich \strikeout{\textcolor{gray}{trau}} traurig. {\pb}Ich habe nur \uline{eine} Sehnſucht: geordnete Verhältniſſe, Wohlſtand, Ruhe, \label{K_L02903-4v}\edtext{\uuline{Ehe}}{\lemma{\textnormal{\emph{Ehe}}}\Cendnote{\textnormal{Erst 1908 ging
                     Goldmann\pwindex{Goldmann, Paul 31.01.1865 – 25.09.1935@\textsc{Goldmann, Paul} (31.01.1865 – 25.09.1935), \emph{Schriftsteller, Journalist}|pwk} mit Eva Marie Goldmann\pwindex{Goldmann, Eva Marie 27.10.1877 – 02.11.1937@\textsc{Goldmann, Eva Marie} (27.10.1877 – 02.11.1937)|pwk}, geb. Fränkel\pwindex{Goldmann, Eva Marie 27.10.1877 – 02.11.1937@\textsc{Goldmann, Eva Marie} (27.10.1877 – 02.11.1937)|pwkv}, ein Ehebündnis ein.}}}\label{K_L02903-4h}: Ich
               ſuche ein ſympathiſches, nicht allzu künſtleriſches und vermögendes Mädchen. Wenn Du
               eine weißt, kannſt Du die Parthie zuſammenbringen. Du kriegſt Prozente von der
               Mitgift.\pend
           \pstart
           Der Wunſch, mich zu verheirathen und zu verſorgen, – noch raſch in den letzten paar
               Jahren, ehe es zu ſpät iſt, – läßt mich nicht mehr los. Mein ganzes Leben lang bin
               ich ein Arbeitsthier geweſen und habe auf Alles verzichten müſſen. Werde ich auch das
               nicht erreichen? Es ſieht {\pb}beinahe ſo aus.\pend
           \pstart
           Schreib’ mir bald!\pend
           \pstart
           Grüße mir den \textsc{Richard\pwindex{Beer-Hofmann, Richard 1866-07-11 – 1945-09-26@\textsc{Beer-Hofmann, Richard} (1866-07-11 – 1945-09-26), \emph{Schriftsteller}|pw}}! (Was macht er?)\pend
           \pstart
           Viele treue Grüße! {\\[\baselineskip]}Dein {\\[\baselineskip]}\spacefill\mbox{Paul Goldmann.}\pend
           \leftskip=0em{}
         
         \endnumbering\mylabel{h}\end{ledgroupsized}\begin{anhang}\end{anhang}\newcommand{\dateiname}{L02903}\newcommand{\titel}{Paul Goldmann an Arthur Schnitzler, 23. 1. [1900]}\newcommand{\editorInnen}{Martin Anton Müller und Laura Untner}%% latex-leseansicht-abspann.tex
%% Abspann für die Leseansicht.
%% Der Schalter \ifkorrekturansicht ist bereits durch den Vorspann gesetzt.

%% latex-abspann.tex
%% Gemeinsamer Abspann für Korrekturansicht und Leseansicht.
%% Setzt den Schalter \ifkorrekturansicht voraus (gesetzt in den
%% einbindenden Dateien latex-korrekturansicht-abspann.tex bzw.
%% latex-leseansicht-abspann.tex).
%% ---------------------------------------------------------------

\normalsize

% Das esempio-Environment wird nur in der Leseansicht benötigt
\ifkorrekturansicht\else
\newenvironment{esempio}[3]%
{
    \vspace{1.5ex}
    \rlap{\underline{#1}}
    \par
    \setlength{\parindent}{0cm}
    \nopagebreak
    \leftskip=#2cm
    \rightskip=#3cm
}
{
    \par
}
\fi

\doendnotes{C}
\bigskip
\vfill

\clearpage

\footnotesize

\ifkorrekturansicht
  \lohead{\textsc{register}}
\fi

% theindex-Environment neu definieren ohne reledmac
\makeatletter
\renewenvironment{theindex}{%
  \ifkorrekturansicht
    \section*{\indexname}%
  \else
    \subsubsection*{Index der erwähnten Entitäten}%
  \fi
  \setlength{\parindent}{0pt}%
  \setlength{\parskip}{0pt plus 0.3pt}%
  \let\item\@idxitem
}{%
  \ifkorrekturansicht\clearpage\fi
}
\makeatother

\IfFileExists{\jobname-pw.ind}{\input{\jobname-pw.ind}}{}

% Quellenangabe nur in der Leseansicht
\ifkorrekturansicht\else
% Fallback-Definitionen, falls die .tex-Datei \titel etc. nicht gesetzt hat
\providecommand{\titel}{}
\providecommand{\editorInnen}{}
\providecommand{\dateiname}{\jobname}

\vspace{3cm}

\vfill

\footnotesize
\textsc{Quelle}: \titel. Herausgegeben von {\editorInnen}. In: \emph{Arthur Schnitzler: Briefwechsel mit Autorinnen und Autoren}.
 Digitale Edition, https://schnitzler-briefe.acdh.oeaw.ac.at/{\dateiname}.html (Stand \today)
\fi

\end{document}


      