%% latex-korrekturansicht-vorspann.tex
%% Vorspann für die Korrekturansicht.
%% Lädt die gemeinsame Datei latex-vorspann.tex mit gesetztem Schalter.

\newif\ifkorrekturansicht
\korrekturansichttrue

\input{../tex-inputs/latex-vorspann}


\section[Felix Braun an Arthur Schnitzler, 21. 4. 1918]{L02285 Felix Braun an Arthur Schnitzler, 21. 4. 1918}
\nopagebreak\mylabel{L02285v}
\rehead{ }\normalsize\beginnumbering\briefempfaengerindex{Schnitzler, Arthur@\textsc{Schnitzler, Arthur}!zzzBraun, Felix@\emph{von Felix Braun}!1918-04-211@{21. 4. 1918}|(be}
\toendnotes[C]{\smallbreak\pagebreak[2]}\Standort{DLA, A:Schnitzler, HS.NZ85.1.2604,1.}
\physDesc{Brief, 1 Blatt, 4 Seiten, 1454 Zeichen
\newline{}Handschrift: schwarze Tinte, deutsche Kurrent
\newline{}Schnitzler: 1) auf der ersten Seite mit Bleistift beschriftet: »\textsc{Braun}«  2) mit rotem Buntstift eine Unterstreichung}\toendnotes[C]{\smallbreak}
\pstart
           \centering{}{\pb}\textcolor{gray}{\textbf{GEORG MÜLLER VERLAG\orgindex{Georg Mueller Verlag@Georg Müller Verlag|pw}, MÜNCHEN\oindex{Muenchen@\textbf{München}, \emph{P.PPLA}|pw}{ }UND{ }BERLIN\oindex{Berlin@\textbf{Berlin}, \emph{P.PPLC}|pw}}}\pend
           
\pstart
           \textcolor{gray}{\textbf{TELEPHON 32043 ⋅ GIROKONTO BEI DER ALLG. ELSÄSSISCHEN BANKGESELLSCHAFT\orgindex{Allgemeine elsaessische Bankgesellschaft@Allgemeine elsässische Bankgesellschaft|pw}, FILIALE MAINZ\oindex{Mainz@\textbf{Mainz}, \emph{P.PPLA}|pw}}}\pend
           
\pstart
           \raggedleft{}\textcolor{gray}{\textbf{MÜNCHEN\oindex{Muenchen@\textbf{München}, \emph{P.PPLA}|pw}, DEN}}{ }21. IV. \textcolor{gray}{\textbf{191}}8\pend
           
\pstart
           \raggedleft{}\textcolor{gray}{\textbf{ELISABETHSTRASSE 26\oindex{Elisabethstrasse [Muenchen]@\textbf{Elisabethstraße [München]}, \emph{Straße (K.STR)}|pw}}}\pend
           
\pstart{}Verehrter Herr Doktor!\pend\vspace{0.5em}
\pstart
           Ich erhielt heute Ihren Expreß-Brief und habe ſogleich mit dem Chef des Verlags,
               Herrn Dr. Kauffmann\pwindex{Kauffmann, Arthur I. 1887-06-11 – 1983@\textsc{Kauffmann, Arthur I.} (1887-06-11 – 1983), \emph{Verleger/Verlegerin, Kunsthändler/Kunsthändlerin}|pw}, geſprochen, in deſſen
               Auftrag ich das folgende mitteilen kann:\pend
           
\pstart
           Der Verlag\orgindex{Georg Mueller Verlag@Georg Müller Verlag|pwv} würde die Novelle\pwindex{Casanovas Heimfahrt@\emph{Casanovas Heimfahrt}|pwv} ſofort drucken und zwar
               in einer Auflage von 8–10.000 Exemplaren; wenn Papier vorhanden ſein ſollte,
               eventuell mehr. Was den Prozentſatz anbelangt, ſo möchte man ſich erſt nach einer
               genauen Kalkulation darüber ausſprechen, da noch niemals 25 {\%}
               gezahlt wurden. Mit der ſpäteren Aufnahme dieſer Bücher in Ihre Geſammelten Werke\pwindex{Gesammelte Werke@\emph{Gesammelte Werke}|pw} iſt man einverſtanden. Für das Stück\pwindex{Schwestern oder Casanova in Spa. Lustspiel in Versen@\emph{Die Schwestern oder Casanova in Spa. Lustspiel in Versen}|pwv} gilt das gleiche, nur
               würde man dieſes in einer geringeren Auflage drucken.\pend
           
\pstart
           Daß man ſich hier außerordentlich freuen würde, wenn es gelänge, Ihre neuen Bücher
               zum Verlag\orgindex{Georg Mueller Verlag@Georg Müller Verlag|pwv} zu bekommen, muß ich
               gewiß nicht erſt ſagen. Man iſt ſchon über die Möglichkeit hoch erfreut. Hoffentlich
               realiſiert ſie ſich auch.\pend
           
\pstart
           {\pb}Mir perſönlich erlauben Sie, verehrter Herr Doktor,
               Ihnen zu ſagen, wie ſehr es mich erfreut hat, Sie an meinem letzten Tag in Wien\oindex{Wien@\textbf{Wien}, \emph{A.ADM2}|pw} noch geſehen und geſprochen zu haben. Dies
               ſchöne Abſchiedsfeſt bei Frau Waſſermann\pwindex{Wassermann, Julie 05.12.1876 – April 1963@\textsc{Wassermann, Julie} (05.12.1876 – April 1963), \emph{Schriftsteller/Schriftstellerin}|pw} hat
               mir den langgehegten Wunſch, einmal mit Ihnen zuſammen zu treffen, erfüllt. Ich danke
               Ihnen herzlich, daß Sie gekommen ſind, und bitte Sie, den Ausdruck aufrichtiger
               Verehrung anzunehmen von Ihrem ergebenen\pend
           \pstart \spacefill\mbox{Felix Braun}\pend{}
\pstart
           \noindent{}P.S.{\\}Ihrer Frau
                     Gemahlin\pwindex{Schnitzler, Olga 17.01.1882 – 13.01.1970@\textsc{Schnitzler, Olga} (17.01.1882 – 13.01.1970), \emph{Schauspieler/Schauspielerin, Sänger/Sängerin}|pwv}, der ich mich beſtens empfehle, bitte ich zu ſagen, daß ich das
                  Paket beim Hotelportier (Schottenhamel\oindex{Hotel Schottenhamel@\textbf{Hotel Schottenhamel}, \emph{Hotel (K.HTL)}|pw})
                  hinterlegt habe.\pend
           \selectlanguage{ngerman}\endnumbering\briefempfaengerindex{Schnitzler, Arthur@\textsc{Schnitzler, Arthur}!zzzBraun, Felix@\emph{von Felix Braun}!1918-04-211@{21. 4. 1918}|)be}\mylabel{L02285h}  \normalsize

\doendnotes{C}
\bigskip
\vfill

\clearpage

\footnotesize

\lohead{\textsc{register}}

% Definiere theindex-Environment komplett neu ohne reledmac
\makeatletter
\renewenvironment{theindex}{%
  \section*{\indexname}%
  \setlength{\parindent}{0pt}%
  \setlength{\parskip}{0pt plus 0.3pt}%
  \let\item\@idxitem
}{%
  \clearpage
}
\makeatother

\IfFileExists{\jobname-pw.ind}{\input{\jobname-pw.ind}}{}

\end{document}

      