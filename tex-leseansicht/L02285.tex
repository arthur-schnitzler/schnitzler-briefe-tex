%% latex-leseansicht-vorspann.tex
%% Vorspann für die Leseansicht.
%% Lädt die gemeinsame Datei latex-vorspann.tex mit nicht gesetztem Schalter.

\newif\ifkorrekturansicht
\korrekturansichtfalse

\input{../tex-inputs/latex-vorspann}


\section[Felix Braun an Arthur Schnitzler, 21. 4. 1918]{L02285 Felix Braun an Arthur Schnitzler, 21. 4. 1918}
\nopagebreak\mylabel{L02285v}
\rehead{ }\normalsize\beginnumbering\briefempfaengerindex{Schnitzler, Arthur@\textsc{Schnitzler, Arthur}!zzzBraun, Felix@\emph{von Felix Braun}!1918-04-211@{21. 4. 1918}|(be}
\toendnotes[C]{\smallbreak\pagebreak[2]}
\correspDesc{Versand  durch Felix Braun am 21. 4. 1918 in München
\newline{}Erhalt  durch Arthur Schnitzler im Zeitraum [22. 4. 1918
                  – 26. 4. 1918?] in Wien}\toendnotes[C]{\smallbreak}
\Standort{DLA, A:Schnitzler, HS.NZ85.1.2604,1.}
\physDesc{Brief, 1 Blatt, 4 Seiten, 1454 Zeichen
\newline{}Handschrift: schwarze Tinte, deutsche Kurrent
\newline{}Schnitzler: 1) auf der ersten Seite mit Bleistift beschriftet: »\textsc{Braun}«  2) mit rotem Buntstift eine Unterstreichung}\toendnotes[C]{\smallbreak}
\pstart
           \centering{}{\pb}\textcolor{gray}{\textbf{GEORG MÜLLER VERLAG\orgindex{Georg Müller Verlag@Georg Müller Verlag|pw}, MÜNCHEN\oindex{München@\textbf{München}|pw}{ }UND{ }BERLIN\oindex{Berlin@\textbf{Berlin}, \emph{Hauptstadt}|pw}}}\pend
           
\pstart
           \textcolor{gray}{\textbf{TELEPHON 32043 ⋅ GIROKONTO BEI DER ALLG. ELSÄSSISCHEN BANKGESELLSCHAFT\orgindex{Allgemeine elsässische Bankgesellschaft@Allgemeine elsässische Bankgesellschaft|pw}, FILIALE MAINZ\oindex{Mainz@\textbf{Mainz}|pw}}}\pend
           
\pstart
           \raggedleft{}\textcolor{gray}{\textbf{MÜNCHEN\oindex{München@\textbf{München}|pw}, DEN}}{ }21. IV. \textcolor{gray}{\textbf{191}}8\pend
           
\pstart
           \raggedleft{}\textcolor{gray}{\textbf{ELISABETHSTRASSE 26\oindex{Elisabethstraße [München]@\textbf{Elisabethstraße [München]}, \emph{Straße}|pw}}}\pend
           
\pstart{}Verehrter Herr Doktor!\pend\vspace{0.5em}
\pstart
           Ich erhielt heute Ihren Expreß-Brief und habe{ }ſogleich mit dem Chef des Verlags,
               Herrn Dr. Kauffmann\pwindex{Kauffmann, Arthur I. 11.\,6.\,1887 Stuttgart – 1983@\textsc{Kauffmann, Arthur I.} (11.\,6.\,1887 Stuttgart – 1983), \emph{Verleger, Kunsthändler}|pw}, geſprochen, in deſſen
               Auftrag ich das folgende mitteilen kann:\pend
           
\pstart
           Der Verlag\orgindex{Georg Müller Verlag@Georg Müller Verlag|pwv} würde die Novelle\pwindex{Schnitzler, Arthur 15.\,5.\,1862 Wien – 21.\,10.\,1931 ebd.@\textsc{Schnitzler, Arthur} (15.\,5.\,1862 Wien – 21.\,10.\,1931 ebd.), \emph{Schriftsteller, Mediziner}!Casanovas Heimfahrt@\strich\emph{Casanovas Heimfahrt}|pwv}{ }ſofort drucken und zwar
               in einer Auflage von 8–10.000 Exemplaren; wenn Papier vorhanden{ }ſein{ }ſollte,
               eventuell mehr. Was den Prozentſatz anbelangt,{ }ſo möchte man{ }ſich erſt nach einer
               genauen Kalkulation darüber ausſprechen, da noch niemals 25 {\%}
               gezahlt wurden. Mit der{ }ſpäteren Aufnahme dieſer Bücher in Ihre Geſammelten Werke\pwindex{Schnitzler, Arthur 15.\,5.\,1862 Wien – 21.\,10.\,1931 ebd.@\textsc{Schnitzler, Arthur} (15.\,5.\,1862 Wien – 21.\,10.\,1931 ebd.), \emph{Schriftsteller, Mediziner}!Gesammelte Werke@\strich\emph{Gesammelte Werke}|pw} iſt man einverſtanden. Für das Stück\pwindex{Schnitzler, Arthur 15.\,5.\,1862 Wien – 21.\,10.\,1931 ebd.@\textsc{Schnitzler, Arthur} (15.\,5.\,1862 Wien – 21.\,10.\,1931 ebd.), \emph{Schriftsteller, Mediziner}!Schwestern oder Casanova in Spa. Lustspiel in Versen@\strich\emph{Die Schwestern oder Casanova in Spa. Lustspiel in Versen}|pwv} gilt das gleiche, nur
               würde man dieſes in einer geringeren Auflage drucken.\pend
           
\pstart
           Daß man{ }ſich hier außerordentlich freuen würde, wenn es gelänge, Ihre neuen Bücher
               zum Verlag\orgindex{Georg Müller Verlag@Georg Müller Verlag|pwv} zu bekommen, muß ich
               gewiß nicht erſt{ }ſagen. Man iſt{ }ſchon über die Möglichkeit hoch erfreut. Hoffentlich
               realiſiert{ }ſie{ }ſich auch.\pend
           
\pstart
           {\pb}Mir perſönlich erlauben Sie, verehrter Herr Doktor,
               Ihnen zu{ }ſagen, wie{ }ſehr es mich erfreut hat, Sie an meinem letzten Tag in Wien\oindex{Wien@\textbf{Wien}, \emph{Verwaltungsgebiet}|pw} noch geſehen und geſprochen zu haben. Dies{ }ſchöne Abſchiedsfeſt bei Frau Waſſermann\pwindex{Wassermann, Julie 5.\,12.\,1876 Wien – April 1963 Zürich@\textsc{Wassermann, Julie} (5.\,12.\,1876 Wien – April 1963 Zürich), \emph{Schriftstellerin}|pw} hat
               mir den langgehegten Wunſch, einmal mit Ihnen zuſammen zu treffen, erfüllt. Ich danke
               Ihnen herzlich, daß Sie gekommen{ }ſind, und bitte Sie, den Ausdruck aufrichtiger
               Verehrung anzunehmen von Ihrem ergebenen\pend
           \pstart \spacefill\mbox{Felix Braun}\pend{}
\pstart
           \noindent{}P.S.{\\}Ihrer Frau
                     Gemahlin\pwindex{Schnitzler, Olga 17.\,1.\,1882 Wien – 13.\,1.\,1970 Lugano@\textsc{Schnitzler, Olga} (17.\,1.\,1882 Wien – 13.\,1.\,1970 Lugano), \emph{Schauspielerin, Sängerin}|pwv}, der ich mich beſtens empfehle, bitte ich zu{ }ſagen, daß ich das
                  Paket beim Hotelportier (Schottenhamel\oindex{Hotel Schottenhamel@\textbf{Hotel Schottenhamel}, \emph{Hotel}|pw})
                  hinterlegt habe.\pend
           \selectlanguage{ngerman}\endnumbering\briefempfaengerindex{Schnitzler, Arthur@\textsc{Schnitzler, Arthur}!zzzBraun, Felix@\emph{von Felix Braun}!1918-04-211@{21. 4. 1918}|)be}\mylabel{L02285h}  \newcommand{\dateiname}{L02285}\newcommand{\titel}{Felix Braun an Arthur Schnitzler, 21. 4. 1918}\newcommand{\editorInnen}{Martin Anton Müller und Gerd-Hermann Susen}%% latex-leseansicht-abspann.tex
%% Abspann für die Leseansicht.
%% Der Schalter \ifkorrekturansicht ist bereits durch den Vorspann gesetzt.

%% latex-abspann.tex
%% Gemeinsamer Abspann für Korrekturansicht und Leseansicht.
%% Setzt den Schalter \ifkorrekturansicht voraus (gesetzt in den
%% einbindenden Dateien latex-korrekturansicht-abspann.tex bzw.
%% latex-leseansicht-abspann.tex).
%% ---------------------------------------------------------------

\normalsize

% Das esempio-Environment wird nur in der Leseansicht benötigt
\ifkorrekturansicht\else
\newenvironment{esempio}[3]%
{
    \vspace{1.5ex}
    \rlap{\underline{#1}}
    \par
    \setlength{\parindent}{0cm}
    \nopagebreak
    \leftskip=#2cm
    \rightskip=#3cm
}
{
    \par
}
\fi

\doendnotes{C}
\bigskip
\vfill

\clearpage

\footnotesize

\ifkorrekturansicht
  \lohead{\textsc{register}}
\fi

% theindex-Environment neu definieren ohne reledmac
\makeatletter
\renewenvironment{theindex}{%
  \ifkorrekturansicht
    \section*{\indexname}%
  \else
    \subsubsection*{Index der erwähnten Entitäten}%
  \fi
  \setlength{\parindent}{0pt}%
  \setlength{\parskip}{0pt plus 0.3pt}%
  \let\item\@idxitem
}{%
  \ifkorrekturansicht\clearpage\fi
}
\makeatother

\IfFileExists{\jobname-pw.ind}{\input{\jobname-pw.ind}}{}

% Quellenangabe nur in der Leseansicht
\ifkorrekturansicht\else
% Fallback-Definitionen, falls die .tex-Datei \titel etc. nicht gesetzt hat
\providecommand{\titel}{}
\providecommand{\editorInnen}{}
\providecommand{\dateiname}{\jobname}

\vspace{3cm}

\vfill

\footnotesize
\textsc{Quelle}: \titel. Herausgegeben von {\editorInnen}. In: \emph{Arthur Schnitzler: Briefwechsel mit Autorinnen und Autoren}.
 Digitale Edition, https://schnitzler-briefe.acdh.oeaw.ac.at/{\dateiname}.html (Stand \today)
\fi

\end{document}


