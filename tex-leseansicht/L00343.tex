%% latex-korrekturansicht-vorspann.tex
%% Vorspann für die Korrekturansicht.
%% Lädt die gemeinsame Datei latex-vorspann.tex mit gesetztem Schalter.

\newif\ifkorrekturansicht
\korrekturansichttrue

\input{../tex-inputs/latex-vorspann}


\section[Arthur Schnitzler an Richard Beer-Hofmann, 2. 7. 1894]{L00343 Arthur Schnitzler an Richard Beer-Hofmann, 2. 7. 1894}
\nopagebreak\mylabel{L00343v}
\rehead{ }\normalsize\beginnumbering\briefempfaengerindex{Beer-Hofmann, Richard@\textsc{Beer-Hofmann, Richard}!zzzSchnitzler, Arthur@\emph{von Arthur Schnitzler}!1894-07-021@{2. 7. 1894}|(be}
\toendnotes[C]{\smallbreak\pagebreak[2]}\Standort{YCGL, MSS 31.}
\physDesc{Brief, 2 Blätter, 7 Seiten, Umschlag, 1489 Zeichen
\newline{}Handschrift: Bleistift, deutsche Kurrent
\newline{}Versand: 1) Stempel: »\nobreak{}\oindex{IX., Alsergrund@\textbf{IX., Alsergrund}, \emph{A.ADM3}|pwk}Wien 9/3, 2. 7. 94\nobreak{}«.   2) Stempel: »\nobreak{}\oindex{Bad Ischl@\textbf{Bad Ischl}, \emph{P.PPL}|pwk}Ischl, 3. 7. 94, 7 F\nobreak{}«. }
\buchAbdrucke{\weitereDrucke{1) Arthur Schnitzler, Richard Beer-Hofmann: \emph{Briefwechsel 1891–1931}. Wien, Zürich: \emph{Europaverlag} 1992, S. 56–57.} \weitereDrucke{2) Hermann Bahr, Arthur Schnitzler: \emph{Briefwechsel, Aufzeichnungen, Dokumente (1891–1931)}. Göttingen: \emph{Wallstein} 2018.} }\toendnotes[C]{\smallbreak}\pstart{}{\pb}Herrn \textsc{Dr. Rich.
                     Beer-Hofmann}\pend{}\pstart{}\textsc{Ischl\oindex{Bad Ischl@\textbf{Bad Ischl}, \emph{P.PPL}|pw}}\pend{}\pstart{}\textsc{Egelmoos 22.\oindex{Eglmoosgasse@\textbf{Eglmoosgasse}, \emph{Bezirk (A.BZK)}|pw}}\pend{}{\bigskip}\vspace{1em}
\pstart{}{\pb}Lieber Richard,\pend\vspace{0.5em}
\pstart
           das \textsc{Cachenez} hoffentlich nach Wunſch besorgt. \textsc{Stoll}\orgindex{Stoll und Uhlig@Stoll {\kaufmannsund}  Uhlig|pw}{ }ſchickt’s noch heute, ni{\geminationm}t es auf Verlangen auch wieder zurück; ich finde es
               ſehr ſchön, was keine Suggeſtion ſein ſoll. –\pend
           
\pstart
           {\pb}Gratulation ſchicken Sie in die Frankgaſſe\oindex{Frankgasse 1@\textbf{Frankgasse 1}, \emph{Wohngebäude (K.WHS)}|pw}, und, \uline{wenn Sie die Braut\pwindex{Schnitzler, Helene 16.07.1871 – September 1941@\textsc{Schnitzler, Helene} (16.07.1871 – September 1941)|pwv} kennen}, auch
               auf den Lobkowitzplatz\oindex{Lobkowitzplatz@\textbf{Lobkowitzplatz}, \emph{Platz (K.PLT)}|pw}. –\pend
           
\pstart
           Ich dürfte 13., 14., 15. nach Iſchl\oindex{Bad Ischl@\textbf{Bad Ischl}, \emph{P.PPL}|pw} ko{\geminationm}en, bleibe
               bis 20. und denke da{\geminationn} mit Ihnen u \textsc{Bahr}\pwindex{Bahr, Hermann 19.07.1863 – 15.01.1934@\textsc{Bahr, Hermann} (19.07.1863 – 15.01.1934), \emph{Schriftsteller/Schriftstellerin, Kritiker/Kritikerin}|pw}, der uns abholt, nach \textsc{Salzburg\oindex{Salzburg@\textbf{Salzburg}, \emph{A.ADM2}|pw}} zu fahren, {\pb}wohin auch Hugo\pwindex{Hofmannsthal, Hugo von 1874-02-01 – 1929-07-15@\textsc{Hofmannsthal, Hugo von} (1874-02-01 – 1929-07-15), \emph{Schriftsteller/Schriftstellerin}|pw} von der \textsc{Fusch\oindex{Bad Fusch@\textbf{Bad Fusch}, \emph{A.ADM3}|pw}} aus ko{\geminationm}en wird. Ich denke, ſo iſt’s gut? –\pend
           
\pstart
           Hugo\pwindex{Hofmannsthal, Hugo von 1874-02-01 – 1929-07-15@\textsc{Hofmannsthal, Hugo von} (1874-02-01 – 1929-07-15), \emph{Schriftsteller/Schriftstellerin}|pw} war Freitag früh auf der Durchreiſe von
               der Saleſianergaſſe\oindex{Salesianergasse 12@\textbf{Salesianergasse 12}, \emph{Wohngebäude (K.WHS)}|pw} nach Döbling\oindex{XIX., Doebling@\textbf{XIX., Döbling}, \emph{A.ADM3}|pw} bei mir. –\pend
           
\pstart
           Was macht der Götterliebling\pwindex{Tod Georgs@\emph{Der Tod Georgs}|pw}? – Ich bin nicht
                  un{\pb}fleißig. Paul
                  Schulz\pwindex{Schulz, Paul 1860-07-01 – 1919-01-31@\textsc{Schulz, Paul} (1860-07-01 – 1919-01-31), \emph{Ministerialbeamter/Ministerialbeamte, Beamter/Beamte}|pw} und die Kapper’s\pwindex{Kapper, Friedrich 21.04.1861 – 22.07.1939@\textsc{Kapper, Friedrich} (21.04.1861 – 22.07.1939), \emph{Mediziner/Medizinerin}|pw}\pwindex{Kapper, Adele 25.01.1870 – 1941@\textsc{Kapper, Adele} (25.01.1870 – 1941)|pw}
               laſſen Sie nur alle wie ſie ſind – wenn wir alle Menſchen ändern könnten wie wir
               wollen, ſo würden ſie uns – ſchrecklich zuwider werden. (Denken Sie nicht drüber
               nach; es iſt ausſichtslos. Der obige Satz ist nemlich {\pb}in mannigfacher Weiſe zu beenden.)\pend
           
\pstart
           Neulich waren \textsc{Fels}\pwindex{Fels, Friedrich Michael *~1864@\textsc{Fels, Friedrich Michael} (*~1864), \emph{Journalist/Journalistin}|pw} und \textsc{Korff}\pwindex{Korff, Heinrich von 05.06.1868 – 18.08.1938@\textsc{Korff, Heinrich von} (05.06.1868 – 18.08.1938), \emph{Journalist/Journalistin}|pw} auf einmal bei mir. –\pend
           
\pstart
           Ich zerbreche mir den Kopf, warum Sie mir geſchrieben haben; ob wegen Kapper\pwindex{Kapper, Friedrich 21.04.1861 – 22.07.1939@\textsc{Kapper, Friedrich} (21.04.1861 – 22.07.1939), \emph{Mediziner/Medizinerin}|pw} oder wegen Schulz\pwindex{Schulz, Paul 1860-07-01 – 1919-01-31@\textsc{Schulz, Paul} (1860-07-01 – 1919-01-31), \emph{Ministerialbeamter/Ministerialbeamte, Beamter/Beamte}|pw} oder wegen meines Bruders\pwindex{Schnitzler, Julius 13.07.1865 – 29.06.1939@\textsc{Schnitzler, Julius} (13.07.1865 – 29.06.1939), \emph{Chirurg/Chirurgin}|pwv}? – Einen Augenblick hatte ich nemlich den
               ſchändlichen Ver{\pb}dacht, dß – das ſchwarze, ſchwere,
               weiche, matte Cachenez – Ihres Briefes »erste Schuld und Urſach« wäre. (Ko{\geminationm}t nirgends vor. Wenn man ſich ſchämt, macht man
               Anführungszeichen.)\pend
           
\pstart
           Leben Sie wohl. Ich freue {\pb}mich nicht aufs Siegeln,
               obwohl ich mehr Grund dazu habe wie Sie. –\pend
           
\pstart
           Schreiben Sie mir bald wieder. Herzlichen Gruß{\\[\baselineskip]}Ihr{\\[\baselineskip]}\spacefill\mbox{Arthur}\pend
           \leftskip=0em{}
\pstart
           2. Juli 94. \textsc{Wien\oindex{Wien@\textbf{Wien}, \emph{A.ADM2}|pw}}\pend
           \selectlanguage{ngerman}\endnumbering\briefempfaengerindex{Beer-Hofmann, Richard@\textsc{Beer-Hofmann, Richard}!zzzSchnitzler, Arthur@\emph{von Arthur Schnitzler}!1894-07-021@{2. 7. 1894}|)be}\mylabel{L00343h}  \normalsize

\doendnotes{C}
\bigskip
\vfill

\clearpage

\footnotesize

\lohead{\textsc{register}}

% Definiere theindex-Environment komplett neu ohne reledmac
\makeatletter
\renewenvironment{theindex}{%
  \section*{\indexname}%
  \setlength{\parindent}{0pt}%
  \setlength{\parskip}{0pt plus 0.3pt}%
  \let\item\@idxitem
}{%
  \clearpage
}
\makeatother

\IfFileExists{\jobname-pw.ind}{\input{\jobname-pw.ind}}{}

\end{document}

      