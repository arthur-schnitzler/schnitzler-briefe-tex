%% latex-leseansicht-vorspann.tex
%% Vorspann für die Leseansicht.
%% Lädt die gemeinsame Datei latex-vorspann.tex mit nicht gesetztem Schalter.

\newif\ifkorrekturansicht
\korrekturansichtfalse

\input{../tex-inputs/latex-vorspann}


\section[Arthur Schnitzler an Richard Beer-Hofmann, 2. 7. 1894]{L00343 Arthur Schnitzler an Richard Beer-Hofmann, 2. 7. 1894}
\nopagebreak\mylabel{L00343v}
\rehead{ }\normalsize\beginnumbering\briefempfaengerindex{Beer-Hofmann, Richard@\textsc{Beer-Hofmann, Richard}!zzzSchnitzler, Arthur@\emph{von Arthur Schnitzler}!1894-07-021@{2. 7. 1894}|(be}
\toendnotes[C]{\smallbreak\pagebreak[2]}
\correspDesc{Versand  durch Arthur Schnitzler am 2. 7. 1894 in Wien
\newline{}Erhalt  durch Richard Beer-Hofmann am 3. 7. 94 in Bad Ischl}\toendnotes[C]{\smallbreak}
\Standort{YCGL, MSS 31.}
\physDesc{Brief, 2 Blätter, 7 Seiten, Kuvert, 1489 Zeichen
\newline{}Handschrift: Bleistift, deutsche Kurrent
\newline{}Versand: 1) Stempel: »\nobreak{}\oindex{IX., Alsergrund@\textbf{IX., Alsergrund}, \emph{Verwaltungsgebiet}|pwk}Wien 9/3, 2. 7. 94\nobreak{}«.   2) Stempel: »\nobreak{}\oindex{Bad Ischl@\textbf{Bad Ischl}|pwk}Ischl, 3. 7. 94, 7 F\nobreak{}«. }
\buchAbdrucke{\weitereDrucke{1) Arthur Schnitzler, Richard Beer-Hofmann: \emph{Briefwechsel 1891–1931}. Herausgegeben von Konstanze Fliedl. Wien, Zürich: \emph{Europaverlag} 1992, S. 56–57.} \weitereDrucke{2) Hermann Bahr, Arthur Schnitzler: \emph{Briefwechsel, Aufzeichnungen, Dokumente (1891–1931)}. Herausgegeben von Kurt Ifkovits und Martin Anton Müller. Göttingen: \emph{Wallstein} 2018.} }\toendnotes[C]{\smallbreak}\pstart{}{\pb}Herrn \textsc{Dr. Rich.
                     Beer-Hofmann}\pend{}\pstart{}\textsc{Ischl\oindex{Bad Ischl@\textbf{Bad Ischl}|pw}}\pend{}\pstart{}\textsc{Egelmoos 22.\oindex{Eglmoosgasse@\textbf{Eglmoosgasse}, \emph{Bezirk}|pw}}\pend{}{\bigskip}\vspace{1em}
\pstart{}{\pb}Lieber Richard,\pend\vspace{0.5em}
\pstart
           das \textsc{Cachenez} hoffentlich nach Wunſch besorgt. \textsc{Stoll}\orgindex{Stoll und Uhlig@Stoll {\kaufmannsund}  Uhlig|pw}{ }ſchickt’s noch heute, ni{\geminationm}t es auf Verlangen auch wieder zurück; ich finde es{ }ſehr{ }ſchön, was keine Suggeſtion{ }ſein{ }ſoll. –\pend
           
\pstart
           {\pb}Gratulation{ }ſchicken Sie in die Frankgaſſe\oindex{Wien@\textbf{Wien}!IX., Alsergrund@\textbf{IX., Alsergrund}!Frankgasse 1@\textbf{Frankgasse 1}, \emph{Wohngebäude}|pw}, und, \uline{wenn Sie die Braut\pwindex{Schnitzler, Helene 16.\,7.\,1871 Budapest – September 1941 Atlantischer Ozean@\textsc{Schnitzler, Helene} (16.\,7.\,1871 Budapest – September 1941 Atlantischer Ozean)|pwv} kennen}, auch
               auf den Lobkowitzplatz\oindex{Wien@\textbf{Wien}!I., Innere Stadt@\textbf{I., Innere Stadt}!Lobkowitzplatz@\textbf{Lobkowitzplatz}, \emph{Platz}|pw}. –\pend
           
\pstart
           Ich dürfte 13., 14., 15. nach Iſchl\oindex{Bad Ischl@\textbf{Bad Ischl}|pw} ko{\geminationm}en, bleibe
               bis 20. und denke da{\geminationn} mit Ihnen u \textsc{Bahr}\pwindex{Bahr, Hermann 19.\,7.\,1863 Linz – 15.\,1.\,1934 München@\textsc{Bahr, Hermann} (19.\,7.\,1863 Linz – 15.\,1.\,1934 München), \emph{Schriftsteller, Kritiker}|pw}, der uns abholt, nach \textsc{Salzburg\oindex{Salzburg@\textbf{Salzburg}, \emph{Verwaltungsgebiet}|pw}} zu fahren, {\pb}wohin auch Hugo\pwindex{Hofmannsthal, Hugo von 1.\,2.\,1874 Wien – 15.\,7.\,1929 Rodaun@\textsc{Hofmannsthal, Hugo von} (1.\,2.\,1874 Wien – 15.\,7.\,1929 Rodaun), \emph{Schriftsteller}|pw} von der \textsc{Fusch\oindex{Bad Fusch@\textbf{Bad Fusch}|pw}} aus ko{\geminationm}en wird. Ich denke,{ }ſo iſt’s gut? –\pend
           
\pstart
           Hugo\pwindex{Hofmannsthal, Hugo von 1.\,2.\,1874 Wien – 15.\,7.\,1929 Rodaun@\textsc{Hofmannsthal, Hugo von} (1.\,2.\,1874 Wien – 15.\,7.\,1929 Rodaun), \emph{Schriftsteller}|pw} war Freitag früh auf der Durchreiſe von
               der Saleſianergaſſe\oindex{Wien@\textbf{Wien}!III., Landstraße@\textbf{III., Landstraße}!Salesianergasse 12@\textbf{Salesianergasse 12}, \emph{Wohngebäude}|pw} nach Döbling\oindex{XIX., Döbling@\textbf{XIX., Döbling}, \emph{Verwaltungsgebiet}|pw} bei mir. –\pend
           
\pstart
           Was macht der Götterliebling\pwindex{Beer-Hofmann, Richard 11.\,7.\,1866 Wien – 26.\,9.\,1945 New York City@\textsc{Beer-Hofmann, Richard} (11.\,7.\,1866 Wien – 26.\,9.\,1945 New York City), \emph{Schriftsteller}!Tod Georgs@\strich\emph{Der Tod Georgs}|pw}? – Ich bin nicht
                  un{\pb}fleißig. Paul
                  Schulz\pwindex{Schulz, Paul 1.\,7.\,1860 Wien – 31.\,1.\,1919 Kreuzlingen@\textsc{Schulz, Paul} (1.\,7.\,1860 Wien – 31.\,1.\,1919 Kreuzlingen), \emph{Ministerialbeamter, Beamter}|pw} und die Kapper’s\pwindex{Kapper, Friedrich 21.\,4.\,1861 Wien – 22.\,7.\,1939 ebd.@\textsc{Kapper, Friedrich} (21.\,4.\,1861 Wien – 22.\,7.\,1939 ebd.), \emph{Mediziner}|pw}\pwindex{Kapper, Adele 25.\,1.\,1870 Wien – 1941 Vernichtungslager Maly Trostinez@\textsc{Kapper, Adele} (25.\,1.\,1870 Wien – 1941 Vernichtungslager Maly Trostinez)|pw}
               laſſen Sie nur alle wie{ }ſie{ }ſind – wenn wir alle Menſchen ändern könnten wie wir
               wollen,{ }ſo würden{ }ſie uns –{ }ſchrecklich zuwider werden. (Denken Sie nicht drüber
               nach; es iſt ausſichtslos. Der obige Satz ist nemlich {\pb}in mannigfacher Weiſe zu beenden.)\pend
           
\pstart
           Neulich waren \textsc{Fels}\pwindex{Fels, Friedrich Michael *~1864 Bad Dürkheim@\textsc{Fels, Friedrich Michael} (*~1864 Bad Dürkheim), \emph{Journalist}|pw} und \textsc{Korff}\pwindex{Korff, Heinrich von 5.\,6.\,1868 Wien – 18.\,8.\,1938 Berlin@\textsc{Korff, Heinrich von} (5.\,6.\,1868 Wien – 18.\,8.\,1938 Berlin), \emph{Journalist}|pw} auf einmal bei mir. –\pend
           
\pstart
           Ich zerbreche mir den Kopf, warum Sie mir geſchrieben haben; ob wegen Kapper\pwindex{Kapper, Friedrich 21.\,4.\,1861 Wien – 22.\,7.\,1939 ebd.@\textsc{Kapper, Friedrich} (21.\,4.\,1861 Wien – 22.\,7.\,1939 ebd.), \emph{Mediziner}|pw} oder wegen Schulz\pwindex{Schulz, Paul 1.\,7.\,1860 Wien – 31.\,1.\,1919 Kreuzlingen@\textsc{Schulz, Paul} (1.\,7.\,1860 Wien – 31.\,1.\,1919 Kreuzlingen), \emph{Ministerialbeamter, Beamter}|pw} oder wegen meines Bruders\pwindex{Schnitzler, Julius 13.\,7.\,1865 Wien – 29.\,6.\,1939 ebd.@\textsc{Schnitzler, Julius} (13.\,7.\,1865 Wien – 29.\,6.\,1939 ebd.), \emph{Chirurg}|pwv}? – Einen Augenblick hatte ich nemlich den{ }ſchändlichen Ver{\pb}dacht, dß – das{ }ſchwarze,{ }ſchwere,
               weiche, matte Cachenez – Ihres Briefes »erste Schuld und Urſach« wäre. (Ko{\geminationm}t nirgends vor. Wenn man{ }ſich{ }ſchämt, macht man
               Anführungszeichen.)\pend
           
\pstart
           Leben Sie wohl. Ich freue {\pb}mich nicht aufs Siegeln,
               obwohl ich mehr Grund dazu habe wie Sie. –\pend
           
\pstart
           Schreiben Sie mir bald wieder. Herzlichen Gruß{\\[\baselineskip]}Ihr{\\[\baselineskip]}\spacefill\mbox{Arthur}\pend
           \leftskip=0em{}
\pstart
           2. Juli 94. \textsc{Wien\oindex{Wien@\textbf{Wien}, \emph{Verwaltungsgebiet}|pw}}\pend
           \selectlanguage{ngerman}\endnumbering\briefempfaengerindex{Beer-Hofmann, Richard@\textsc{Beer-Hofmann, Richard}!zzzSchnitzler, Arthur@\emph{von Arthur Schnitzler}!1894-07-021@{2. 7. 1894}|)be}\mylabel{L00343h}  \newcommand{\dateiname}{L00343}\newcommand{\titel}{Arthur Schnitzler an Richard Beer-Hofmann, 2. 7. 1894}\newcommand{\editorInnen}{Herausgegeben von Martin Anton Müller}%% latex-leseansicht-abspann.tex
%% Abspann für die Leseansicht.
%% Der Schalter \ifkorrekturansicht ist bereits durch den Vorspann gesetzt.

%% latex-abspann.tex
%% Gemeinsamer Abspann für Korrekturansicht und Leseansicht.
%% Setzt den Schalter \ifkorrekturansicht voraus (gesetzt in den
%% einbindenden Dateien latex-korrekturansicht-abspann.tex bzw.
%% latex-leseansicht-abspann.tex).
%% ---------------------------------------------------------------

\normalsize

% Das esempio-Environment wird nur in der Leseansicht benötigt
\ifkorrekturansicht\else
\newenvironment{esempio}[3]%
{
    \vspace{1.5ex}
    \rlap{\underline{#1}}
    \par
    \setlength{\parindent}{0cm}
    \nopagebreak
    \leftskip=#2cm
    \rightskip=#3cm
}
{
    \par
}
\fi

\doendnotes{C}
\bigskip
\vfill

\clearpage

\footnotesize

\ifkorrekturansicht
  \lohead{\textsc{register}}
\fi

% theindex-Environment neu definieren ohne reledmac
\makeatletter
\renewenvironment{theindex}{%
  \ifkorrekturansicht
    \section*{\indexname}%
  \else
    \subsubsection*{Index der erwähnten Entitäten}%
  \fi
  \setlength{\parindent}{0pt}%
  \setlength{\parskip}{0pt plus 0.3pt}%
  \let\item\@idxitem
}{%
  \ifkorrekturansicht\clearpage\fi
}
\makeatother

\IfFileExists{\jobname-pw.ind}{\input{\jobname-pw.ind}}{}

% Quellenangabe nur in der Leseansicht
\ifkorrekturansicht\else
% Fallback-Definitionen, falls die .tex-Datei \titel etc. nicht gesetzt hat
\providecommand{\titel}{}
\providecommand{\editorInnen}{}
\providecommand{\dateiname}{\jobname}

\vspace{3cm}

\vfill

\footnotesize
\textsc{Quelle}: \titel. Herausgegeben von {\editorInnen}. In: \emph{Arthur Schnitzler: Briefwechsel mit Autorinnen und Autoren}.
 Digitale Edition, https://schnitzler-briefe.acdh.oeaw.ac.at/{\dateiname}.html (Stand \today)
\fi

\end{document}


