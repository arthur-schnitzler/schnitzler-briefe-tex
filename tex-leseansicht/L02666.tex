%% latex-leseansicht-vorspann.tex
%% Vorspann für die Leseansicht.
%% Lädt die gemeinsame Datei latex-vorspann.tex mit nicht gesetztem Schalter.

\newif\ifkorrekturansicht
\korrekturansichtfalse

\input{../tex-inputs/latex-vorspann}


\section[Paul Goldmann an Arthur Schnitzler, 5. 7. 1891]{L02666 Paul Goldmann an Arthur Schnitzler, 5. 7. 1891}
\nopagebreak\mylabel{L02666v}
\rehead{ }\normalsize\beginnumbering\briefempfaengerindex{Schnitzler, Arthur@\textsc{Schnitzler, Arthur}!zzzGoldmann, Paul@\emph{von Paul Goldmann}!1891-07-051@{5. 7. 1891}|(be}
\toendnotes[C]{\smallbreak\pagebreak[2]}
\correspDesc{Versand  durch Paul Goldmann am 5. 7. 1891 in Den Haag
\newline{}Erhalt  durch Arthur Schnitzler am 7. 7. 1891 in Wien}\toendnotes[C]{\smallbreak}
\Standort{DLA, A:Schnitzler, HS.NZ85.1.3162.}
\physDesc{Postkarte, 638 Zeichen
\newline{}Handschrift: schwarze Tinte, deutsche Kurrent
\newline{}Versand: 1) Stempel: »\nobreak{}\oindex{Brüssel@\textbf{Brüssel}, \emph{Hauptstadt}|pwk}\begin{otherlanguage}{dutch}’Sgravenhage\end{otherlanguage}, \begin{otherlanguage}{dutch}5 Jul 91\end{otherlanguage}, 7–8N\nobreak{}«.   2) Stempel: »\nobreak{}\oindex{Wien@\textbf{Wien}, \emph{Verwaltungsgebiet}|pwk}Wien 1/1, 7{[}.{]} 7. 91, 4\textsuperscript{1}/\textsubscript{2}
                                       - 6N, Bestellt\nobreak{}«. 
\newline{}Schnitzler: mit Bleistift das Empfangsdatum »7/ 7 91« und das Jahr »91« vermerkt }\toendnotes[C]{\smallbreak}\pstart{}\textsc{{\pb}Autriche\oindex{Österreich@\textbf{Österreich}|pw}!}\pend{}\pstart{}\textsc{Herrn}\pend{}\pstart{}\textsc{Dr. Arthur Schnitzler}\pend{}\pstart{}\textsc{Wien\oindex{Wien@\textbf{Wien}, \emph{Verwaltungsgebiet}|pw}}\pend{}\pstart{}\textsc{I. Giselastraſse 11\oindex{Wien@\textbf{Wien}!I., Innere Stadt@\textbf{I., Innere Stadt}!Ordination Arthur Schnitzler [Bösendorferstraße 11]@\textbf{Ordination Arthur Schnitzler [Bösendorferstraße 11]}, \emph{Ordination}|pw}.}\pend{}{\bigskip}\vspace{1em}
\pstart
           \noindent{}{\pb}Haag\oindex{Den Haag@\textbf{Den Haag}, \emph{Hauptstadt}|pw}, 6. Juli.\hspace*{1.5em}Mein lieber Arthur! Einen herzlichen Gruß von
               unterwegs. Ich bin zur \label{K_L02666-1v}\edtext{Puppenausſtellung\orgindex{Puppenaustellung@Puppenaustellung|pw}}{\lemma{\textnormal{\emph{Puppenausstellung}}}\Cendnote{\textnormal{Die \emph{Puppenausstellung}\orgindex{Puppenaustellung@Puppenaustellung|pwk} in Scheveningen\oindex{Scheveningen@\textbf{Scheveningen}|pwk}
                  fand vom 4. 7. 1891 bis 4. 8. 1891 statt. Goldmann\pwindex{Goldmann, Paul 31.\,1.\,1865 Breslau – 25.\,9.\,1935 Wien@\textsc{Goldmann, Paul} (31.\,1.\,1865 Breslau – 25.\,9.\,1935 Wien), \emph{Schriftsteller, Journalist}|pwk} schrieb darüber im zweiten
                  Feuilleton über die Reise: Goldmann\pwindex{Goldmann, Paul 31.\,1.\,1865 Breslau – 25.\,9.\,1935 Wien@\textsc{Goldmann, Paul} (31.\,1.\,1865 Breslau – 25.\,9.\,1935 Wien), \emph{Schriftsteller, Journalist}|pwk}: \emph{Holländisches Intermezzo. II.}\pwindex{Goldmann, Paul 31.\,1.\,1865 Breslau – 25.\,9.\,1935 Wien@\textsc{Goldmann, Paul} (31.\,1.\,1865 Breslau – 25.\,9.\,1935 Wien), \emph{Schriftsteller, Journalist}!Holländisches Intermezzo. II.@\strich\emph{Holländisches Intermezzo. II.}|pwk}, Jg. 35, Nr. 193,
                        12. 7. 1891, Erstes Morgenblatt, S. 1–3. (Das erste
                  war zwei Tage vorher erschienen, Nr. 191, 10. 7. 1891, Erstes
                     Morgenblatt, S. 1–3.)}}}\label{K_L02666-1} nach \textsc{Scheveningen\oindex{Scheveningen@\textbf{Scheveningen}|pw}} geſchickt worden u. habe bei dieſer Gelegenheit ein Stück Holland\oindex{Niederlande@\textbf{Niederlande}|pw} mit angeſehen. Unvergeßliche u. unvergleichliche
               Eindrücke in Rotterdam\oindex{Rotterdam@\textbf{Rotterdam}|pw}, Haag\oindex{Den Haag@\textbf{Den Haag}, \emph{Hauptstadt}|pw} und am Meer! Eine neue Welt, in der Alles{ }ſympathiſch
               iſt, \strikeout{ohne} ohne{ }ſchön zu{ }ſein, und wo doch vieles{ }ſchön iſt, vieles neu ohne Gleichen u.{ }ſympathiſch iſt. Näheres aus Brüſſel\oindex{Brüssel@\textbf{Brüssel}, \emph{Hauptstadt}|pw}. – Gekreuzt? Wann haben{ }ſich 2 Briefe von uns gekreuzt? \substVorne{}\textsuperscript{\textcolor{gray}{Seit}}\substDazwischen{}Vor\substHinten{} Deinem \label{T_L02666-1v}\edtext{letzten habe ich Monate
               lang nichts}{\lemma{\textnormal{\emph{letzten … nichts}}}\Cendnote{\textnormal{seitlich am rechten
                  Rand}}}\label{T_L02666-1}{ }\label{T_L02666-2v}\edtext{von Dir erhalten?! – Dein treuer
                  \spacefill\mbox{Paul Goldmann.}}{\lemma{\textnormal{\emph{von … Goldmann.}}}\Cendnote{\textnormal{kopfüber am oberen Rand}}}\label{T_L02666-2}\pend
           \selectlanguage{ngerman}\endnumbering\briefempfaengerindex{Schnitzler, Arthur@\textsc{Schnitzler, Arthur}!zzzGoldmann, Paul@\emph{von Paul Goldmann}!1891-07-051@{5. 7. 1891}|)be}\mylabel{L02666h}  \newcommand{\dateiname}{L02666}\newcommand{\titel}{Paul Goldmann an Arthur Schnitzler, 5. 7. 1891}\newcommand{\editorInnen}{Martin Anton Müller und Laura Untner}%% latex-leseansicht-abspann.tex
%% Abspann für die Leseansicht.
%% Der Schalter \ifkorrekturansicht ist bereits durch den Vorspann gesetzt.

%% latex-abspann.tex
%% Gemeinsamer Abspann für Korrekturansicht und Leseansicht.
%% Setzt den Schalter \ifkorrekturansicht voraus (gesetzt in den
%% einbindenden Dateien latex-korrekturansicht-abspann.tex bzw.
%% latex-leseansicht-abspann.tex).
%% ---------------------------------------------------------------

\normalsize

% Das esempio-Environment wird nur in der Leseansicht benötigt
\ifkorrekturansicht\else
\newenvironment{esempio}[3]%
{
    \vspace{1.5ex}
    \rlap{\underline{#1}}
    \par
    \setlength{\parindent}{0cm}
    \nopagebreak
    \leftskip=#2cm
    \rightskip=#3cm
}
{
    \par
}
\fi

\doendnotes{C}
\bigskip
\vfill

\clearpage

\footnotesize

\ifkorrekturansicht
  \lohead{\textsc{register}}
\fi

% theindex-Environment neu definieren ohne reledmac
\makeatletter
\renewenvironment{theindex}{%
  \ifkorrekturansicht
    \section*{\indexname}%
  \else
    \subsubsection*{Index der erwähnten Entitäten}%
  \fi
  \setlength{\parindent}{0pt}%
  \setlength{\parskip}{0pt plus 0.3pt}%
  \let\item\@idxitem
}{%
  \ifkorrekturansicht\clearpage\fi
}
\makeatother

\IfFileExists{\jobname-pw.ind}{\input{\jobname-pw.ind}}{}

% Quellenangabe nur in der Leseansicht
\ifkorrekturansicht\else
% Fallback-Definitionen, falls die .tex-Datei \titel etc. nicht gesetzt hat
\providecommand{\titel}{}
\providecommand{\editorInnen}{}
\providecommand{\dateiname}{\jobname}

\vspace{3cm}

\vfill

\footnotesize
\textsc{Quelle}: \titel. Herausgegeben von {\editorInnen}. In: \emph{Arthur Schnitzler: Briefwechsel mit Autorinnen und Autoren}.
 Digitale Edition, https://schnitzler-briefe.acdh.oeaw.ac.at/{\dateiname}.html (Stand \today)
\fi

\end{document}


