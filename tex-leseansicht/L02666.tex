%% latex-korrekturansicht-vorspann.tex
%% Vorspann für die Korrekturansicht.
%% Lädt die gemeinsame Datei latex-vorspann.tex mit gesetztem Schalter.

\newif\ifkorrekturansicht
\korrekturansichttrue

\input{../tex-inputs/latex-vorspann}


\section[Paul Goldmann an Arthur Schnitzler, 5. 7. 1891]{L02666 Paul Goldmann an Arthur Schnitzler, 5. 7. 1891}
\nopagebreak\mylabel{L02666v}
\rehead{ }\normalsize\beginnumbering\briefempfaengerindex{Schnitzler, Arthur@\textsc{Schnitzler, Arthur}!zzzGoldmann, Paul@\emph{von Paul Goldmann}!1891-07-051@{5. 7. 1891}|(be}
\toendnotes[C]{\smallbreak\pagebreak[2]}\Standort{DLA, A:Schnitzler, HS.NZ85.1.3162.}
\physDesc{Postkarte, 638 Zeichen
\newline{}Handschrift: 1) schwarze Tinte, deutsche Kurrent\hspace{1em}2) schwarze Tinte, lateinische Kurrent (\noindent{}Adresse)\hspace{1em}
\newline{}Versand: 1) Stempel: »\nobreak{}\oindex{Bruessel@\textbf{Brüssel}, \emph{P.PPLC}|pwk}\begin{otherlanguage}{dutch}’Sgravenhage\end{otherlanguage}, \begin{otherlanguage}{dutch}5 Jul 91\end{otherlanguage}, 7–8N\nobreak{}«.   2) Stempel: »\nobreak{}Wien 1/1, 7{[}.{]} 7. 91, 4\textsuperscript{1}/\textsubscript{2}
                                       - 6N, Bestellt\nobreak{}«. 
\newline{}Schnitzler: mit Bleistift das Empfangsdatum »7/ 7 91« und das Jahr »91« vermerkt }\toendnotes[C]{\smallbreak}\pstart{}{\pb}Autriche\oindex{Oesterreich@\textbf{Österreich}, \emph{A.PCLI}|pw}! \pend{}\pstart{}Herrn\pend{}\pstart{}Dr. Arthur Schnitzler\pend{}\pstart{}Wien\oindex{Wien@\textbf{Wien}, \emph{A.ADM2}|pw}\pend{}\pstart{}I. Giselastraſse 11\oindex{Ordination Arthur Schnitzler [Boesendorferstrasse 11]@\textbf{Ordination Arthur Schnitzler [Bösendorferstraße 11]}, \emph{Ordination}|pw}.\pend{}{\bigskip}\vspace{1em}
\pstart
           \noindent{}{\pb}Haag\oindex{Den Haag@\textbf{Den Haag}, \emph{P.PPLG}|pw}, 6. Juli.\hspace*{1.5em}Mein lieber Arthur! Einen herzlichen Gruß von
               unterwegs. Ich bin zur \label{K_L02666-1v}\edtext{Puppenausſtellung\orgindex{Puppenaustellung@Puppenaustellung|pw}}{\lemma{\textnormal{\emph{Puppenausſtellung}}}\Cendnote{\textnormal{Die \emph{Puppenausstellung}\orgindex{Puppenaustellung@Puppenaustellung|pwk} in Scheveningen\oindex{Scheveningen@\textbf{Scheveningen}, \emph{P.PPL}|pwk}
                  fand vom 4. 7. 1891 bis 4. 8. 1891 statt. Goldmann\pwindex{Goldmann, Paul 31.01.1865 – 25.09.1935@\textsc{Goldmann, Paul} (31.01.1865 – 25.09.1935), \emph{Schriftsteller/Schriftstellerin, Journalist/Journalistin}|pwk} schrieb darüber im zweiten
                  Feuilleton über die Reise: Goldmann\pwindex{Goldmann, Paul 31.01.1865 – 25.09.1935@\textsc{Goldmann, Paul} (31.01.1865 – 25.09.1935), \emph{Schriftsteller/Schriftstellerin, Journalist/Journalistin}|pwk}: \emph{Holländisches Intermezzo. II.}\pwindex{Hollaendisches Intermezzo. II.@\emph{Holländisches Intermezzo. II.}|pwk}, Jg. 35, Nr. 193,
                        12. 7. 1891, Erstes Morgenblatt, S. 1–3. (Das erste
                  war zwei Tage vorher erschienen, Nr. 191, 10. 7. 1891, Erstes
                     Morgenblatt, S. 1–3.)}}}\label{K_L02666-1} nach \textsc{Scheveningen\oindex{Scheveningen@\textbf{Scheveningen}, \emph{P.PPL}|pw}} geſchickt worden u. habe bei dieſer Gelegenheit ein Stück Holland\oindex{Niederlande@\textbf{Niederlande}, \emph{A.PCLI}|pw} mit angeſehen. Unvergeßliche u. unvergleichliche
               Eindrücke in Rotterdam\oindex{Rotterdam@\textbf{Rotterdam}, \emph{P.PPL}|pw}, Haag\oindex{Den Haag@\textbf{Den Haag}, \emph{P.PPLG}|pw} und am Meer! Eine neue Welt, in der Alles ſympathiſch
               iſt, \strikeout{ohne} ohne ſchön zu ſein, und wo doch vieles
               ſchön iſt, vieles neu ohne Gleichen u. ſympathiſch iſt. Näheres aus Brüſſel\oindex{Bruessel@\textbf{Brüssel}, \emph{P.PPLC}|pw}. – Gekreuzt? Wann haben ſich 2 Briefe von uns gekreuzt? \substVorne{}\textsuperscript{\textcolor{gray}{Seit}}\substDazwischen{}Vor\substHinten{} Deinem \label{T_L02666-1v}\edtext{letzten habe ich Monate
               lang nichts}{\lemma{\textnormal{\emph{letzten … nichts}}}\Cendnote{\textnormal{seitlich am rechten
                  Rand}}}\label{T_L02666-1}{ }\label{T_L02666-2v}\edtext{von Dir erhalten?! – Dein treuer
                  \spacefill\mbox{Paul Goldmann.}}{\lemma{\textnormal{\emph{von … Goldmann.}}}\Cendnote{\textnormal{kopfüber am oberen Rand}}}\label{T_L02666-2}\pend
           \selectlanguage{ngerman}\endnumbering\briefempfaengerindex{Schnitzler, Arthur@\textsc{Schnitzler, Arthur}!zzzGoldmann, Paul@\emph{von Paul Goldmann}!1891-07-051@{5. 7. 1891}|)be}\mylabel{L02666h}  \normalsize

\doendnotes{C}
\bigskip
\vfill

\clearpage

\footnotesize

\lohead{\textsc{register}}

% Definiere theindex-Environment komplett neu ohne reledmac
\makeatletter
\renewenvironment{theindex}{%
  \section*{\indexname}%
  \setlength{\parindent}{0pt}%
  \setlength{\parskip}{0pt plus 0.3pt}%
  \let\item\@idxitem
}{%
  \clearpage
}
\makeatother

\IfFileExists{\jobname-pw.ind}{\input{\jobname-pw.ind}}{}

\end{document}

      