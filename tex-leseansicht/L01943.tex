%% latex-korrekturansicht-vorspann.tex
%% Vorspann für die Korrekturansicht.
%% Lädt die gemeinsame Datei latex-vorspann.tex mit gesetztem Schalter.

\newif\ifkorrekturansicht
\korrekturansichttrue

\input{../tex-inputs/latex-vorspann}


\section[Franz Blei an Arthur Schnitzler, 10. 7. 1910]{L01943 Franz Blei an Arthur Schnitzler, 10. 7. 1910}
\nopagebreak\mylabel{L01943v}
\rehead{ }\normalsize\beginnumbering\briefempfaengerindex{Schnitzler, Arthur@\textsc{Schnitzler, Arthur}!zzzBlei, Franz@\emph{von Franz Blei}!1910-07-101@{10. 7. 1910}|(be}
\toendnotes[C]{\smallbreak\pagebreak[2]}\Standort{CUL, Schnitzler, B 14.}
\physDesc{Brief, 1 Blatt, 1 Seite, 379 Zeichen
\newline{}Handschrift: schwarze Tinte, lateinische Kurrent
\newline{}Schnitzler: 1) mit Bleistift beschriftet: »\textsc{Blei}«  2) mit rotem Buntstift zwei Unterstreichungen
\newline{}Ordnung: 1) mit Bleistift von unbekannter Hand nummeriert: »\strikeout{5}«  2) mit Bleistift von unbekannter Hand nummeriert:
                                 »6«}\toendnotes[C]{\smallbreak}
\pstart
           {\pb}Forte dei Marmi, Versilia, Ital.\oindex{Forte dei Marmi@\textbf{Forte dei Marmi}, \emph{P.PPLA3}|pw}{\\}Casa Vignolo\oindex{Casa Vignolo@\textbf{Casa Vignolo}, \emph{Hotel (K.HTL)}|pw}.\pend
           
\pstart{}Wertester Herr Schnitzler,\pend\vspace{0.5em}
\pstart
           der Verleger Georg Müller\pwindex{Mueller, Georg 29.12.1877 – 29.12.1917@\textsc{Müller, Georg} (29.12.1877 – 29.12.1917), \emph{Verleger/Verlegerin}|pw}, München, Josefsplatz 7\oindex{Josephsplatz@\textbf{Josephsplatz}, \emph{Platz (K.PLT)}|pw} möchte gerne in einem schönen Druck von
               600 Exemplaren den »\label{K_L01943-1v}\edtext{Reigen\pwindex{Reigen. Zehn Dialoge@\emph{Reigen. Zehn Dialoge}|pw}« herausgeben}{\lemma{\textnormal{\emph{Reigen« herausgeben}}}\Cendnote{\textnormal{Nicht verwirklicht, Briefe Georg Müllers\pwindex{Mueller, Georg 29.12.1877 – 29.12.1917@\textsc{Müller, Georg} (29.12.1877 – 29.12.1917), \emph{Verleger/Verlegerin}|pwk} finden sich nicht in Schnitzlers Nachlass. Das Vorhaben war Teil einer größeren Buchreihe, die
                  auch, neben anderen, Bahrs\pwindex{Bahr, Hermann 19.07.1863 – 15.01.1934@\textsc{Bahr, Hermann} (19.07.1863 – 15.01.1934), \emph{Schriftsteller/Schriftstellerin, Kritiker/Kritikerin}|pwk}{ }\emph{Die Mutter}\pwindex{Mutter. Drama in drei Akten@\emph{Die Mutter. Drama in drei Akten}|pwk} und \emph{Der Garten der Erkenntnis}\pwindex{Garten der Erkenntnis@\emph{Der Garten der Erkenntnis}|pwk} von Leopold
                     von Andrian-Werburg\pwindex{Andrian-Werburg, Leopold von 09.05.1875 – 19.11.1951@\textsc{Andrian-Werburg, Leopold von} (09.05.1875 – 19.11.1951), \emph{Schriftsteller/Schriftstellerin, Diplomat/Diplomatin}|pwk} hätte enthalten sollen (vgl. Hartmut Walravens,
                     Angela Reinthal: \emph{Franz Blei als Berater des Verlages Georg
                        Müller. Franz Bleis Briefe an Georg Müller}. Wien: \emph{Verlag
                        der Österreichischen Akademie der Wissenschaften}{ }2015, S. 77–79, S. 118–120, S. 129).}}}\label{K_L01943-1}, und ich möchte
               das empfehlend unterstützen. Wenn Sie prinzipiell damit einverstanden sind, bitte ich
               Sie, sich mit G. Müller\pwindex{Mueller, Georg 29.12.1877 – 29.12.1917@\textsc{Müller, Georg} (29.12.1877 – 29.12.1917), \emph{Verleger/Verlegerin}|pw} zu verständigen.\pend
           
\pstart
           Mit bestem Grusse{\\[\baselineskip]}\textcolor{gray}{Ihr}{\\[\baselineskip]}\spacefill\mbox{Frz Blei}\pend
           \leftskip=0em{}
\pstart
           10. Juli 1910\pend
           \selectlanguage{ngerman}\endnumbering\briefempfaengerindex{Schnitzler, Arthur@\textsc{Schnitzler, Arthur}!zzzBlei, Franz@\emph{von Franz Blei}!1910-07-101@{10. 7. 1910}|)be}\mylabel{L01943h}  \normalsize

\doendnotes{C}
\bigskip
\vfill

\clearpage

\footnotesize

\lohead{\textsc{register}}

% Definiere theindex-Environment komplett neu ohne reledmac
\makeatletter
\renewenvironment{theindex}{%
  \section*{\indexname}%
  \setlength{\parindent}{0pt}%
  \setlength{\parskip}{0pt plus 0.3pt}%
  \let\item\@idxitem
}{%
  \clearpage
}
\makeatother

\IfFileExists{\jobname-pw.ind}{\input{\jobname-pw.ind}}{}

\end{document}

      