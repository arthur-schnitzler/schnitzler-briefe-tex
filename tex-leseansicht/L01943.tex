%% latex-leseansicht-vorspann.tex
%% Vorspann für die Leseansicht.
%% Lädt die gemeinsame Datei latex-vorspann.tex mit nicht gesetztem Schalter.

\newif\ifkorrekturansicht
\korrekturansichtfalse

\input{../tex-inputs/latex-vorspann}


\section[Franz Blei an Arthur Schnitzler, 10. 7. 1910]{L01943 Franz Blei an Arthur Schnitzler, 10. 7. 1910}
\nopagebreak\mylabel{L01943v}
\rehead{ }\normalsize\beginnumbering\briefempfaengerindex{Schnitzler, Arthur@\textsc{Schnitzler, Arthur}!zzzBlei, Franz@\emph{von Franz Blei}!1910-07-101@{10. 7. 1910}|(be}
\toendnotes[C]{\smallbreak\pagebreak[2]}
\correspDesc{Versand  durch Franz Blei am 10. 7. 1910 \textbf{Ort fehlend} 
\newline{}Erhalt  durch Arthur Schnitzler im Zeitraum [10. 7. 1910
                  – 14. 7. 1910?] in Wien}\toendnotes[C]{\smallbreak}
\Standort{CUL, Schnitzler, B 14.}
\physDesc{Brief, 1 Blatt, 1 Seite, 379 Zeichen
\newline{}Handschrift: schwarze Tinte, lateinische Kurrent
\newline{}Schnitzler: 1) mit Bleistift beschriftet: »\textsc{Blei}«  2) mit rotem Buntstift zwei Unterstreichungen
\newline{}Ordnung: 1) mit Bleistift von unbekannter Hand nummeriert: »\strikeout{5}«  2) mit Bleistift von unbekannter Hand nummeriert:
                                 »6«}\toendnotes[C]{\smallbreak}
\pstart
           {\pb}Forte dei Marmi, Versilia, Ital.\oindex{Forte dei Marmi@\textbf{Forte dei Marmi}, \emph{Hauptstadt}|pw}{\\}Casa Vignolo\oindex{Casa Vignolo@\textbf{Casa Vignolo}, \emph{Hotel}|pw}.\pend
           
\pstart{}Wertester Herr Schnitzler,\pend\vspace{0.5em}
\pstart
           der Verleger Georg Müller\pwindex{Müller, Georg 29.\,12.\,1877 Mainz – 29.\,12.\,1917 München@\textsc{Müller, Georg} (29.\,12.\,1877 Mainz – 29.\,12.\,1917 München), \emph{Verleger}|pw}, München, Josefsplatz 7\oindex{Josephsplatz@\textbf{Josephsplatz}, \emph{Platz}|pw} möchte gerne in einem schönen Druck von
               600 Exemplaren den »\label{K_L01943-1v}\edtext{Reigen\pwindex{Schnitzler, Arthur 15.\,5.\,1862 Wien – 21.\,10.\,1931 ebd.@\textsc{Schnitzler, Arthur} (15.\,5.\,1862 Wien – 21.\,10.\,1931 ebd.), \emph{Schriftsteller, Mediziner}!Reigen. Zehn Dialoge@\strich\emph{Reigen. Zehn Dialoge}|pw}« herausgeben}{\lemma{\textnormal{\emph{Reigen« herausgeben}}}\Cendnote{\textnormal{Nicht verwirklicht, Briefe Georg Müllers\pwindex{Müller, Georg 29.\,12.\,1877 Mainz – 29.\,12.\,1917 München@\textsc{Müller, Georg} (29.\,12.\,1877 Mainz – 29.\,12.\,1917 München), \emph{Verleger}|pwk} finden sich nicht in Schnitzlers Nachlass. Das Vorhaben war Teil einer größeren Buchreihe, die
                  auch, neben anderen, Bahrs\pwindex{Bahr, Hermann 19.\,7.\,1863 Linz – 15.\,1.\,1934 München@\textsc{Bahr, Hermann} (19.\,7.\,1863 Linz – 15.\,1.\,1934 München), \emph{Schriftsteller, Kritiker}|pwk}{ }\emph{Die Mutter}\pwindex{Bahr, Hermann 19.\,7.\,1863 Linz – 15.\,1.\,1934 München@\textsc{Bahr, Hermann} (19.\,7.\,1863 Linz – 15.\,1.\,1934 München), \emph{Schriftsteller, Kritiker}!Mutter. Drama in drei Akten@\strich\emph{Die Mutter. Drama in drei Akten}|pwk} und \emph{Der Garten der Erkenntnis}\pwindex{Andrian-Werburg, Leopold von 9.\,5.\,1875 Berlin – 19.\,11.\,1951 Fribourg@\textsc{Andrian-Werburg, Leopold von} (9.\,5.\,1875 Berlin – 19.\,11.\,1951 Fribourg), \emph{Schriftsteller, Diplomat}!Garten der Erkenntnis@\strich\emph{Der Garten der Erkenntnis}|pwk} von Leopold
                     von Andrian-Werburg\pwindex{Andrian-Werburg, Leopold von 9.\,5.\,1875 Berlin – 19.\,11.\,1951 Fribourg@\textsc{Andrian-Werburg, Leopold von} (9.\,5.\,1875 Berlin – 19.\,11.\,1951 Fribourg), \emph{Schriftsteller, Diplomat}|pwk} hätte enthalten sollen (vgl. Hartmut Walravens,
                     Angela Reinthal: \emph{Franz Blei als Berater des Verlages Georg
                        Müller. Franz Bleis Briefe an Georg Müller}. Wien: \emph{Verlag
                        der Österreichischen Akademie der Wissenschaften}{ }2015, S. 77–79, S. 118–120, S. 129).}}}\label{K_L01943-1}, und ich möchte
               das empfehlend unterstützen. Wenn Sie prinzipiell damit einverstanden sind, bitte ich
               Sie, sich mit G. Müller\pwindex{Müller, Georg 29.\,12.\,1877 Mainz – 29.\,12.\,1917 München@\textsc{Müller, Georg} (29.\,12.\,1877 Mainz – 29.\,12.\,1917 München), \emph{Verleger}|pw} zu verständigen.\pend
           
\pstart
           Mit bestem Grusse{\\[\baselineskip]}\textcolor{gray}{Ihr}{\\[\baselineskip]}\spacefill\mbox{Frz Blei}\pend
           \leftskip=0em{}
\pstart
           10. Juli 1910\pend
           \selectlanguage{ngerman}\endnumbering\briefempfaengerindex{Schnitzler, Arthur@\textsc{Schnitzler, Arthur}!zzzBlei, Franz@\emph{von Franz Blei}!1910-07-101@{10. 7. 1910}|)be}\mylabel{L01943h}  \newcommand{\dateiname}{L01943}\newcommand{\titel}{Franz Blei an Arthur Schnitzler, 10. 7. 1910}\newcommand{\editorInnen}{Martin Anton Müller und Gerd-Hermann Susen}%% latex-leseansicht-abspann.tex
%% Abspann für die Leseansicht.
%% Der Schalter \ifkorrekturansicht ist bereits durch den Vorspann gesetzt.

%% latex-abspann.tex
%% Gemeinsamer Abspann für Korrekturansicht und Leseansicht.
%% Setzt den Schalter \ifkorrekturansicht voraus (gesetzt in den
%% einbindenden Dateien latex-korrekturansicht-abspann.tex bzw.
%% latex-leseansicht-abspann.tex).
%% ---------------------------------------------------------------

\normalsize

% Das esempio-Environment wird nur in der Leseansicht benötigt
\ifkorrekturansicht\else
\newenvironment{esempio}[3]%
{
    \vspace{1.5ex}
    \rlap{\underline{#1}}
    \par
    \setlength{\parindent}{0cm}
    \nopagebreak
    \leftskip=#2cm
    \rightskip=#3cm
}
{
    \par
}
\fi

\doendnotes{C}
\bigskip
\vfill

\clearpage

\footnotesize

\ifkorrekturansicht
  \lohead{\textsc{register}}
\fi

% theindex-Environment neu definieren ohne reledmac
\makeatletter
\renewenvironment{theindex}{%
  \ifkorrekturansicht
    \section*{\indexname}%
  \else
    \subsubsection*{Index der erwähnten Entitäten}%
  \fi
  \setlength{\parindent}{0pt}%
  \setlength{\parskip}{0pt plus 0.3pt}%
  \let\item\@idxitem
}{%
  \ifkorrekturansicht\clearpage\fi
}
\makeatother

\IfFileExists{\jobname-pw.ind}{\input{\jobname-pw.ind}}{}

% Quellenangabe nur in der Leseansicht
\ifkorrekturansicht\else
% Fallback-Definitionen, falls die .tex-Datei \titel etc. nicht gesetzt hat
\providecommand{\titel}{}
\providecommand{\editorInnen}{}
\providecommand{\dateiname}{\jobname}

\vspace{3cm}

\vfill

\footnotesize
\textsc{Quelle}: \titel. Herausgegeben von {\editorInnen}. In: \emph{Arthur Schnitzler: Briefwechsel mit Autorinnen und Autoren}.
 Digitale Edition, https://schnitzler-briefe.acdh.oeaw.ac.at/{\dateiname}.html (Stand \today)
\fi

\end{document}


