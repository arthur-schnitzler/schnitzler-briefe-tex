%% latex-korrekturansicht-vorspann.tex
%% Vorspann für die Korrekturansicht.
%% Lädt die gemeinsame Datei latex-vorspann.tex mit gesetztem Schalter.

\newif\ifkorrekturansicht
\korrekturansichttrue

\input{../tex-inputs/latex-vorspann}


\section[Albert Ehrenstein an Arthur Schnitzler, 24. 8. 1909]{L01867 Albert Ehrenstein an Arthur Schnitzler, 24. 8. 1909}
\nopagebreak\mylabel{L01867v}
\rehead{ }\normalsize\beginnumbering\briefempfaengerindex{Schnitzler, Arthur@\textsc{Schnitzler, Arthur}!zzzEhrenstein, Albert@\emph{von Albert Ehrenstein}!1909-08-241@{24. 8. 1909}|(be}
\toendnotes[C]{\smallbreak\pagebreak[2]}\Standort{CUL, Schnitzler, B 30.}
\physDesc{Brief, 1 Blatt, 4 Seiten, 1803 Zeichen
\newline{}Handschrift: schwarze Tinte, deutsche Kurrent
\newline{}Schnitzler: mit Bleistift beschriftet: »\textsc{Ehrenstein}« }\toendnotes[C]{\smallbreak}
\pstart
           
\pstart
           {\pb}\textsc{Mixnitz bei Frohnleiten\oindex{Mixnitz@\textbf{Mixnitz}, \emph{P.PPL}|pw}}\pend
           
\pstart
           \raggedleft{}\textsc{24. August 09.}\pend
           \pend
           
\pstart{}\textsc{Sehr geehrter Herr Doktor,}\pend\vspace{0.5em}
\pstart
           anfangs hatte ich die Abſicht, Ihnen meinen Dank für Ihre gütige Intervention bei
               Herrn D\textsuperscript{r}\textsc{Auernheimer}\pwindex{Auernheimer, Raoul 15.04.1876 – 06.01.1948@\textsc{Auernheimer, Raoul} (15.04.1876 – 06.01.1948), \emph{Schriftsteller/Schriftstellerin, Journalist/Journalistin, Kritiker/Kritikerin}|pw} perſönlich abzuſtatten. Da nun ſowieſo alles eins war und meine Studien eine
               ärgerliche Verlängerung erfahren mußten, trat ich eine kleine Reiſe an. Aber als ich
               am 19. dieſes durch Edlach\oindex{Edlach@\textbf{Edlach}, \emph{P.PPL}|pw} kam,
               brachte ich es als ein rechter Traumichnicht nicht {\pb}über mich, Sie, ſehr geehrter Herr Doktor,
               aus Ihrer ländlichen Abgeſchiedenheit aufzuſtören. Mittlerweile iſt faſt jeder reale
               Grund entfallen, um deſſentwillen ich Sie, ſehr geehrter Herr Doktor, bat, für mich
               bei Auernheimer\pwindex{Auernheimer, Raoul 15.04.1876 – 06.01.1948@\textsc{Auernheimer, Raoul} (15.04.1876 – 06.01.1948), \emph{Schriftsteller/Schriftstellerin, Journalist/Journalistin, Kritiker/Kritikerin}|pw} zu ſprechen. Jener Kollege\pwindex{?? [Studienkollege von Albert Ehrenstein] *~1909@\textsc{?? [Studienkollege von Albert Ehrenstein]} (*~1909)|pwv} unterließ es mir
               ſeine Diſſertation\pwindex{?? [Dissertation]@\emph{?? [Dissertation]}|pwv}
               einzuſenden, und ſo fühle ich mich nicht autoriſiert, über ſie ein Wort zu ſprechen.
               Ein Fräulein, das ſehr ſchöne Gedichte und ſehr ſchlechte Novellen ſchreibt, erſuchte
               mich, ihr ein paar Sachen von mir {\pb}zu
               ſchicken, ich ſandte ihr unter anderem »Tubutſch\pwindex{Tubutsch@\emph{Tubutsch}|pw}«, »Baber\pwindex{Tod des Zehir eddin Muhammed Baber@\emph{Tod des Zehir eddin Muhammed Baber}|pw}« und »Apaturien\pwindex{Apaturien@\emph{Apaturien}|pw}« anfangs Auguſt nach Venedig\oindex{Venedig@\textbf{Venedig}, \emph{P.PPLA}|pw} – ſie hat die Sachen bis nun nicht
               erhalten und ich beſitze keine Abſchrift. Ich könnte jetzt nicht einmal beweiſen, daß
               ich einmal literariſch wertbare Dinge geformt habe, und es wird mir kaum etwas
               anderes übrig bleiben, als – wiewohl die Herren M.
                  Duilius\pwindex{Duilius, Gaius 3. Jh. v. u. Z.@\textsc{Duilius, Gaius} (3. Jh. v. u. Z.), \emph{Politiker/Politikerin}|pw}, Theoderich\pwindex{Theoderich der Grosse 451/456 – 30.08.0526@\textsc{Theoderich der Große} (451/456 – 30.08.0526), \emph{König/Königin, Regent/Regentin}|pw} und Guſtav Adolf\pwindex{Gustav II. Adolf von Schweden 19.12.1594 – 16.11.1632@\textsc{Gustav II. Adolf von Schweden} (19.12.1594 – 16.11.1632), \emph{König/Königin, Regent/Regentin}|pw} mir auch bisher gefolgt ſind und es
               ſehr {\pb}preſſant haben – \textsc{nolens volens} allerhand fragwürdige Geſchichten aus dem
               Ärmel zu ſchütteln, und ſie im Herbſt, wenn Sie, ſehr geehrter Herr Doktor, nicht
               allzuviel zu tun haben ſollten, Ihnen vorzulegen, wenn ich Ihnen meine Aufwartung
               machen darf, um eines Urteils über meine wahrſcheinlich verlorenen Handſchriften und
               vielleicht einiger wertvoller Winke für eine etwaige Rekonſtruktion teilhaftig zu
               werden. Hochachtungsvoll Ihr ergebenſter\pend
           
\pstart
           \spacefill\mbox{Albert Ehrenstein,}{\\[\baselineskip]}Pechvogel \textsc{non plus ultra.}\pend
           \leftskip=0em{}\selectlanguage{ngerman}\endnumbering\briefempfaengerindex{Schnitzler, Arthur@\textsc{Schnitzler, Arthur}!zzzEhrenstein, Albert@\emph{von Albert Ehrenstein}!1909-08-241@{24. 8. 1909}|)be}\mylabel{L01867h}  \normalsize

\doendnotes{C}
\bigskip
\vfill

\clearpage

\footnotesize

\lohead{\textsc{register}}

% Definiere theindex-Environment komplett neu ohne reledmac
\makeatletter
\renewenvironment{theindex}{%
  \section*{\indexname}%
  \setlength{\parindent}{0pt}%
  \setlength{\parskip}{0pt plus 0.3pt}%
  \let\item\@idxitem
}{%
  \clearpage
}
\makeatother

\IfFileExists{\jobname-pw.ind}{\input{\jobname-pw.ind}}{}

\end{document}

      