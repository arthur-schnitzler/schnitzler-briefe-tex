\input{../tex-inputs/latex-pdf-vorspann}
\begin{center}
            \textcolor{red}{ENTWURF. ENTZIFFERUNG NOCH NICHT KORREKTURGELESEN}
                      \end{center}
            
               \section[Albert Ehrenstein an Arthur Schnitzler, 24. 8. 1909]{ Albert Ehrenstein an Arthur Schnitzler, 24. 8. 1909}\nopagebreak\mylabel{v}\rehead{ }\begin{ledgroupsized}[t]{13cm}\normalsize\beginnumbering\briefempfaengerindex{Schnitzler, Arthur@\textsc{Schnitzler, Arthur}!zzzEhrenstein, Albert@\emph{von Albert Ehrenstein}!1909-08-241@{24. 8. 1909}|(be} \toendnotes[C]{\smallbreak\pagebreak[2]} \Standort{CUL, Schnitzler, B 30.}
\physDesc{Brief, 1 Blatt, 4 Seiten
\newline{}Handschrift: schwarze Tinte, deutsche Kurrent
\newline{}Schnitzler: mit Bleistift beschriftet: »\textsc{Ehrenstein}« }\toendnotes[C]{\smallbreak}\pstart
           {\pb}\textsc{Mixnitz bei Frohnleiten\oindex{Mixnitz@\textbf{Mixnitz}|pw}}\hfill \textsc{24. August 09.}\pend
           \pstart{}\textsc{Sehr geehrter Herr Doktor,}\pend\pstart
           anfangs hatte ich die Abſicht, Ihnen meinen Dank für Ihre gütige Intervention bei
                    Herrn D\textsuperscript{r} \textsc{Auernheimer}\pwindex{Auernheimer, Raoul 15.04.1876 – 06.01.1948@\textsc{Auernheimer, Raoul} (15.04.1876 – 06.01.1948), \emph{Schriftsteller, Journalist, Kritiker}|pw} perſönlich abzuſtatten. Da nun ſowieſo alles eins war und meine Studien
                    eine ärgerliche Verlängerung erfahren mußten, trat ich eine kleine Reiſe an.
                    Aber als ich am 19. dieſes durch Edlach\oindex{Edlach@\textbf{Edlach}|pw} kam, brachte ich es als ein rechter Traumichnicht nicht {\pb}über mich, Sie, ſehr geehrter Herr Doktor, aus Ihrer
                    ländlichen Abgeſchiedenheit aufzuſtören. Mittlerweile iſt faſt jeder reale Grund
                    entfallen, um deſſentwillen ich Sie, ſehr geehrter Herr Doktor, bat, für mich
                    bei Auernheimer\pwindex{Auernheimer, Raoul 15.04.1876 – 06.01.1948@\textsc{Auernheimer, Raoul} (15.04.1876 – 06.01.1948), \emph{Schriftsteller, Journalist, Kritiker}|pw} zu ſprechen. Jener Kollege\pwindex{?? [Studienkollege von Albert Ehrenstein] *~1909@\textsc{?? [Studienkollege von Albert Ehrenstein]} (*~1909)|pwv} unterließ es mir
                    ſeine Diſſertation\pwindex{?? [Studienkollege von Albert Ehrenstein] *~1909@\textsc{?? [Studienkollege von Albert Ehrenstein]} (*~1909)!?? [Dissertation]1909@\strich\emph{?? [Dissertation]} {[}1909{]}|pwv}
                    einzuſenden, und ſo fühle ich mich nicht autoriſiert, über ſie ein Wort zu
                    ſprechen. Ein Fräulein,
                    das ſehr ſchöne Gedichte und ſehr ſchlechte Novellen ſchreibt, erſuchte mich,
                    ihr ein paar Sachen von mir {\pb}zu ſchicken, ich ſandte
                    ihr unter anderem »Tubutſch\pwindex{Ehrenstein, Albert 23.12.1886 – 08.04.1950@\textsc{Ehrenstein, Albert} (23.12.1886 – 08.04.1950), \emph{Schriftsteller}!Tubutsch1911@\strich\emph{Tubutsch} {[}1911{]}|pw}«, »Baber\pwindex{Ehrenstein, Albert 23.12.1886 – 08.04.1950@\textsc{Ehrenstein, Albert} (23.12.1886 – 08.04.1950), \emph{Schriftsteller}!Tod des Zehir eddin Muhammed Baber1912@\strich\emph{Tod des Zehir eddin Muhammed Baber} {[}1912{]}|pw}« und »Apaturien\pwindex{Ehrenstein, Albert 23.12.1886 – 08.04.1950@\textsc{Ehrenstein, Albert} (23.12.1886 – 08.04.1950), \emph{Schriftsteller}!Apaturien1912@\strich\emph{Apaturien} {[}1912{]}|pw}« anfangs Auguſt nach Venedig\oindex{Venedig@\textbf{Venedig}|pw} – ſie hat die Sachen bis nun nicht erhalten und
                    ich beſitze keine Abſchrift. Ich könnte jetzt nicht einmal beweiſen, daß ich
                    einmal literariſch wertbare Dinge geformt habe, und es wird mir kaum etwas
                    anderes übrig bleiben, als – wiewohl die Herren M.
                        Duilius\pwindex{Duilius, Gaius 3. Jh. v. u. Z.@\textsc{Duilius, Gaius} (3. Jh. v. u. Z.), \emph{Politiker}|pw}, Theoderich\pwindex{Theoderich der Grosse 451/456 – 30.08.0526@\textsc{Theoderich der Große} (451/456 – 30.08.0526), \emph{König, Regent}|pw} und Guſtav Adolf\pwindex{Gustav II. Adolf von Schweden 19.12.1594 – 16.11.1632@\textsc{Gustav II. Adolf von Schweden} (19.12.1594 – 16.11.1632), \emph{Regent/Regentin}|pw} mir auch bisher gefolgt ſind und
                    es ſehr {\pb}preſſant haben – \textsc{nolens
                        volens} allerhand fragwürdige Geſchichten aus dem Ärmel zu ſchütteln,
                    und ſie im Herbſt, wenn Sie, ſehr geehrter Herr Doktor, nicht allzuviel zu tun
                    haben ſollten, Ihnen vorzulegen, wenn ich Ihnen meine Aufwartung machen darf, um
                    eines Urteils über meine wahrſcheinlich verlorenen Handſchriften und vielleicht
                    einiger wertvoller Winke für eine etwaige Rekonſtruktion teilhaftig zu werden.
                    Hochachtungsvoll Ihr ergebenſter\pend
           \pstart
           \spacefill\mbox{Albert Ehrenstein,}{\\[\baselineskip]}Pechvogel \textsc{non plus ultra.}\pend
           \leftskip=0em{}\endnumbering\briefempfaengerindex{Schnitzler, Arthur@\textsc{Schnitzler, Arthur}!zzzEhrenstein, Albert@\emph{von Albert Ehrenstein}!1909-08-241@{24. 8. 1909}|)be}\mylabel{h}\end{ledgroupsized}  \newcommand{\dateiname}{L01867}\newcommand{\titel}{Albert Ehrenstein an Arthur Schnitzler, 24. 8. 1909}\newcommand{\editorInnen}{Martin Anton Müller und Gerd-Hermann Susen}\input{../tex-inputs/latex-pdf-abspann}
      