%% latex-leseansicht-vorspann.tex
%% Vorspann für die Leseansicht.
%% Lädt die gemeinsame Datei latex-vorspann.tex mit nicht gesetztem Schalter.

\newif\ifkorrekturansicht
\korrekturansichtfalse

\input{../tex-inputs/latex-vorspann}


\section[Albert Ehrenstein an Arthur Schnitzler, 24. 8. 1909]{L01867 Albert Ehrenstein an Arthur Schnitzler, 24. 8. 1909}
\nopagebreak\mylabel{L01867v}
\rehead{ }\normalsize\beginnumbering\briefempfaengerindex{Schnitzler, Arthur@\textsc{Schnitzler, Arthur}!zzzEhrenstein, Albert@\emph{von Albert Ehrenstein}!1909-08-241@{24. 8. 1909}|(be}
\toendnotes[C]{\smallbreak\pagebreak[2]}
\correspDesc{Versand  durch Albert Ehrenstein am 24. 8. 1909 in Mixnitz
\newline{}Zustellung  am [25. 8. 1909?] in Wien
\newline{}Erhalt  durch Arthur Schnitzler am [2. 9. 1909?] in Wien}\toendnotes[C]{\smallbreak}
\Standort{CUL, Schnitzler, B 30.}
\physDesc{Brief, 1 Blatt, 4 Seiten, 1803 Zeichen
\newline{}Handschrift: schwarze Tinte, deutsche Kurrent
\newline{}Schnitzler: mit Bleistift beschriftet: »\textsc{Ehrenstein}« }\toendnotes[C]{\smallbreak}
\pstart
           
\pstart
           {\pb}\textsc{Mixnitz bei Frohnleiten\oindex{Mixnitz@\textbf{Mixnitz}|pw}}\pend
           
\pstart
           \raggedleft{}\textsc{24. August 09.}\pend
           \pend
           
\pstart{}\textsc{Sehr geehrter Herr Doktor,}\pend\vspace{0.5em}
\pstart
           anfangs hatte ich die Abſicht, Ihnen meinen Dank für Ihre gütige Intervention bei
               Herrn D\textsuperscript{r}\textsc{Auernheimer}\pwindex{Auernheimer, Raoul 15.\,4.\,1876 Wien – 6.\,1.\,1948 Oakland@\textsc{Auernheimer, Raoul} (15.\,4.\,1876 Wien – 6.\,1.\,1948 Oakland), \emph{Schriftsteller, Journalist, Kritiker}|pw} perſönlich abzuſtatten. Da nun{ }ſowieſo alles eins war und meine Studien eine
               ärgerliche Verlängerung erfahren mußten, trat ich eine kleine Reiſe an. Aber als ich
               am 19. dieſes durch Edlach\oindex{Edlach@\textbf{Edlach}|pw} kam,
               brachte ich es als ein rechter Traumichnicht nicht {\pb}über mich, Sie,{ }ſehr geehrter Herr Doktor,
               aus Ihrer ländlichen Abgeſchiedenheit aufzuſtören. Mittlerweile iſt faſt jeder reale
               Grund entfallen, um deſſentwillen ich Sie,{ }ſehr geehrter Herr Doktor, bat, für mich
               bei Auernheimer\pwindex{Auernheimer, Raoul 15.\,4.\,1876 Wien – 6.\,1.\,1948 Oakland@\textsc{Auernheimer, Raoul} (15.\,4.\,1876 Wien – 6.\,1.\,1948 Oakland), \emph{Schriftsteller, Journalist, Kritiker}|pw} zu{ }ſprechen. Jener Kollege\pwindex{?? [Studienkollege von Albert Ehrenstein] *~1909@\textsc{?? [Studienkollege von Albert Ehrenstein]} (*~1909)|pwv} unterließ es mir{ }ſeine Diſſertation\pwindex{?? [Studienkollege von Albert Ehrenstein] *~1909@\textsc{?? [Studienkollege von Albert Ehrenstein]} (*~1909)!?? [Dissertation]@\strich\emph{?? [Dissertation]}|pwv}
               einzuſenden, und{ }ſo fühle ich mich nicht autoriſiert, über{ }ſie ein Wort zu{ }ſprechen.
               Ein Fräulein, das{ }ſehr{ }ſchöne Gedichte und{ }ſehr{ }ſchlechte Novellen{ }ſchreibt, erſuchte
               mich, ihr ein paar Sachen von mir {\pb}zu{ }ſchicken, ich{ }ſandte ihr unter anderem »Tubutſch\pwindex{Ehrenstein, Albert 23.\,12.\,1886 Wien – 8.\,4.\,1950 New York City@\textsc{Ehrenstein, Albert} (23.\,12.\,1886 Wien – 8.\,4.\,1950 New York City), \emph{Schriftsteller}!Tubutsch@\strich\emph{Tubutsch}|pw}«, »Baber\pwindex{Ehrenstein, Albert 23.\,12.\,1886 Wien – 8.\,4.\,1950 New York City@\textsc{Ehrenstein, Albert} (23.\,12.\,1886 Wien – 8.\,4.\,1950 New York City), \emph{Schriftsteller}!Tod des Zehir eddin Muhammed Baber@\strich\emph{Tod des Zehir eddin Muhammed Baber}|pw}« und »Apaturien\pwindex{Ehrenstein, Albert 23.\,12.\,1886 Wien – 8.\,4.\,1950 New York City@\textsc{Ehrenstein, Albert} (23.\,12.\,1886 Wien – 8.\,4.\,1950 New York City), \emph{Schriftsteller}!Apaturien@\strich\emph{Apaturien}|pw}« anfangs Auguſt nach Venedig\oindex{Venedig@\textbf{Venedig}|pw} –{ }ſie hat die Sachen bis nun nicht
               erhalten und ich beſitze keine Abſchrift. Ich könnte jetzt nicht einmal beweiſen, daß
               ich einmal literariſch wertbare Dinge geformt habe, und es wird mir kaum etwas
               anderes übrig bleiben, als – wiewohl die Herren M.
                  Duilius\pwindex{Duilius, Gaius 3. Jh. v.\,u.\,Z.@\textsc{Duilius, Gaius} (3. Jh. v.\,u.\,Z.), \emph{Politiker}|pw}, Theoderich\pwindex{Theoderich der Große 451/456 Ungarn – 30.\,8.\,526 Ravenna@\textsc{Theoderich der Große} (451/456 Ungarn – 30.\,8.\,526 Ravenna), \emph{König, Regent}|pw} und Guſtav Adolf\pwindex{Gustav II. Adolf von Schweden 19.\,12.\,1594 Stockholm – 16.\,11.\,1632 Lützen@\textsc{Gustav II. Adolf von Schweden} (19.\,12.\,1594 Stockholm – 16.\,11.\,1632 Lützen), \emph{König, Regent}|pw} mir auch bisher gefolgt{ }ſind und es{ }ſehr {\pb}preſſant haben – \textsc{nolens volens} allerhand fragwürdige Geſchichten aus dem
               Ärmel zu{ }ſchütteln, und{ }ſie im Herbſt, wenn Sie,{ }ſehr geehrter Herr Doktor, nicht
               allzuviel zu tun haben{ }ſollten, Ihnen vorzulegen, wenn ich Ihnen meine Aufwartung
               machen darf, um eines Urteils über meine wahrſcheinlich verlorenen Handſchriften und
               vielleicht einiger wertvoller Winke für eine etwaige Rekonſtruktion teilhaftig zu
               werden. Hochachtungsvoll Ihr ergebenſter\pend
           
\pstart
           \spacefill\mbox{Albert Ehrenstein,}{\\[\baselineskip]}Pechvogel \textsc{non plus ultra.}\pend
           \leftskip=0em{}\selectlanguage{ngerman}\endnumbering\briefempfaengerindex{Schnitzler, Arthur@\textsc{Schnitzler, Arthur}!zzzEhrenstein, Albert@\emph{von Albert Ehrenstein}!1909-08-241@{24. 8. 1909}|)be}\mylabel{L01867h}  \newcommand{\dateiname}{L01867}\newcommand{\titel}{Albert Ehrenstein an Arthur Schnitzler, 24. 8. 1909}\newcommand{\editorInnen}{Martin Anton Müller und Gerd-Hermann Susen}%% latex-leseansicht-abspann.tex
%% Abspann für die Leseansicht.
%% Der Schalter \ifkorrekturansicht ist bereits durch den Vorspann gesetzt.

%% latex-abspann.tex
%% Gemeinsamer Abspann für Korrekturansicht und Leseansicht.
%% Setzt den Schalter \ifkorrekturansicht voraus (gesetzt in den
%% einbindenden Dateien latex-korrekturansicht-abspann.tex bzw.
%% latex-leseansicht-abspann.tex).
%% ---------------------------------------------------------------

\normalsize

% Das esempio-Environment wird nur in der Leseansicht benötigt
\ifkorrekturansicht\else
\newenvironment{esempio}[3]%
{
    \vspace{1.5ex}
    \rlap{\underline{#1}}
    \par
    \setlength{\parindent}{0cm}
    \nopagebreak
    \leftskip=#2cm
    \rightskip=#3cm
}
{
    \par
}
\fi

\doendnotes{C}
\bigskip
\vfill

\clearpage

\footnotesize

\ifkorrekturansicht
  \lohead{\textsc{register}}
\fi

% theindex-Environment neu definieren ohne reledmac
\makeatletter
\renewenvironment{theindex}{%
  \ifkorrekturansicht
    \section*{\indexname}%
  \else
    \subsubsection*{Index der erwähnten Entitäten}%
  \fi
  \setlength{\parindent}{0pt}%
  \setlength{\parskip}{0pt plus 0.3pt}%
  \let\item\@idxitem
}{%
  \ifkorrekturansicht\clearpage\fi
}
\makeatother

\IfFileExists{\jobname-pw.ind}{\input{\jobname-pw.ind}}{}

% Quellenangabe nur in der Leseansicht
\ifkorrekturansicht\else
% Fallback-Definitionen, falls die .tex-Datei \titel etc. nicht gesetzt hat
\providecommand{\titel}{}
\providecommand{\editorInnen}{}
\providecommand{\dateiname}{\jobname}

\vspace{3cm}

\vfill

\footnotesize
\textsc{Quelle}: \titel. Herausgegeben von {\editorInnen}. In: \emph{Arthur Schnitzler: Briefwechsel mit Autorinnen und Autoren}.
 Digitale Edition, https://schnitzler-briefe.acdh.oeaw.ac.at/{\dateiname}.html (Stand \today)
\fi

\end{document}


