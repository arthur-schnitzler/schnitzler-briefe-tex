%% latex-leseansicht-vorspann.tex
%% Vorspann für die Leseansicht.
%% Lädt die gemeinsame Datei latex-vorspann.tex mit nicht gesetztem Schalter.

\newif\ifkorrekturansicht
\korrekturansichtfalse

\input{../tex-inputs/latex-vorspann}


\section[ Paul Goldmann an Arthur Schnitzler, 16. 6. [1900]]{L02920 Paul Goldmann an Arthur Schnitzler,  16. 6. [1900]}
\nopagebreak\mylabel{L02920v}
\rehead{ }\normalsize\beginnumbering\briefempfaengerindex{Schnitzler, Arthur@\textsc{Schnitzler, Arthur}!zzzGoldmann, Paul@\emph{von Paul Goldmann}!1900-06-161@{16. 6. [1900]}|(be}
\toendnotes[C]{\smallbreak\pagebreak[2]}
\correspDesc{Versand  durch Paul Goldmann am 16. 6. [1900] in Berlin
\newline{}Erhalt  durch Arthur Schnitzler im Zeitraum [17. 6. 1900 –
                  19. 6. 1900?] in Wien}\toendnotes[C]{\smallbreak}
\Standort{DLA, A:Schnitzler, HS.NZ85.1.3170.}
\physDesc{Brief, 1 Blatt, 4 Seiten, 1692 Zeichen
\newline{}Handschrift: blaue Tinte, deutsche Kurrent
\newline{}Schnitzler: 1) mit Bleistift das Jahr »1900« vermerkt  2) mit rotem Buntstift eine Unterstreichung}\toendnotes[C]{\smallbreak}
\pstart
           {\pb}\textcolor{gray}{\textbf{DESSAUERSTRASSE 19}}\oindex{Dessauer Straße@\textbf{Dessauer Straße}, \emph{Straße}|pw}\pend
           
\pstart
           \raggedleft{}Berlin\oindex{Berlin@\textbf{Berlin}, \emph{Hauptstadt}|pw}, 16. Juni.\pend
           
\pstart\center{}Mein lieber Freund,\pend\vspace{0.5em}
\pstart
           Ich danke Dir für die ausführliche Beantwortung meiner Briefe und freue mich{ }ſehr,
               wieder einmal Direktes von Dir gehört zu haben. Nächſte Woche werde ich
               vorausſichtlich viel zu thun haben. Ich antworte Dir daher gleich, und zwar nur wegen
               der Sommerpläne.\pend
           
\pstart
           Du{ }ſcheinſt einen überwiegend \label{K_L02920-1v}\edtext{weiblichen Sommer}{\lemma{\textnormal{\emph{weiblichen Sommer}}}\Cendnote{\textnormal{Schnitzler war einerseits mit seiner
                  zukünftigen Ehefrau Olga Gussmann\pwindex{Schnitzler, Olga 17.\,1.\,1882 Wien – 13.\,1.\,1970 Lugano@\textsc{Schnitzler, Olga} (17.\,1.\,1882 Wien – 13.\,1.\,1970 Lugano), \emph{Schauspielerin, Sängerin}|pwk},
                  andererseits mit Leopoldine Müller\pwindex{Müller, Leopoldine 1.\,11.\,1873 Wien – 18.\,1.\,1946 ebd.@\textsc{Müller, Leopoldine} (1.\,11.\,1873 Wien – 18.\,1.\,1946 ebd.)|pwk} in einer
                  intimen Beziehung.}}}\label{K_L02920-1} verleben zu wollen.\pend
           
\pstart
           Ich{ }ſinne noch über ein Mittel zur Löſung der Finanz-Schwierigkeiten {\pb}nach, die einer Urlaubsreiſe diesmal bei mir \strikeout{\textcolor{gray}{i}} entgegenſtehen. Habe ich es gefunden und bekomme ich Urlaub – zwei noch{ }ſehr
               fragliche Dinge –{ }ſo möchte ich \strikeout{End\textcolor{gray}{e}} Anfang Auguſt eine Fußwanderung in den Alpen\oindex{Alpen@\textbf{Alpen}|pw}, in Tirol\oindex{Tirol@\textbf{Tirol}, \emph{Land}|pw} womöglich, machen. Erſtens weil es{ }ſchön iſt, zweitens aus
               Abmagerungs-Gründen. Denn ich werde dick wie ein Schwein. Ich frage Dich alſo:\pend
           
\pstart
           1.) Möchteſt Du bei{ }ſo etwas \label{K_L02920-2v}\edtext{mitmachen}{\lemma{\textnormal{\emph{mitmachen}}}\Cendnote{\textnormal{Tatsächlich unternahmen
                     Schnitzler und Goldmann\pwindex{Goldmann, Paul 31.\,1.\,1865 Breslau – 25.\,9.\,1935 Wien@\textsc{Goldmann, Paul} (31.\,1.\,1865 Breslau – 25.\,9.\,1935 Wien), \emph{Schriftsteller, Journalist}|pwk} in der zweiten Augusthälfte{ }1900 gemeinsam mit Richard Beer-Hofmann\pwindex{Beer-Hofmann, Richard 11.\,7.\,1866 Wien – 26.\,9.\,1945 New York City@\textsc{Beer-Hofmann, Richard} (11.\,7.\,1866 Wien – 26.\,9.\,1945 New York City), \emph{Schriftsteller}|pwk}, Alfred Kerr\pwindex{Kerr, Alfred 25.\,12.\,1867 Breslau – 12.\,10.\,1948 Hamburg@\textsc{Kerr, Alfred} (25.\,12.\,1867 Breslau – 12.\,10.\,1948 Hamburg), \emph{Schriftsteller, Kritiker}|pwk} und Leo Van-Jung\pwindex{Van-Jung, Leo 15.\,10.\,1866 Odessa – 2.\,7.\,1939 Riga@\textsc{Van-Jung, Leo} (15.\,10.\,1866 Odessa – 2.\,7.\,1939 Riga), \emph{Gesangspädagoge, Mathematiker}|pwk} eine
                     Alpen\oindex{Alpen@\textbf{Alpen}|pwk}wanderung. Am 16. 8. 1900 in Innsbruck\oindex{Innsbruck@\textbf{Innsbruck}, \emph{Verwaltungsgebiet}|pwk} beginnend, kamen sie über Vorarlberg\oindex{Vorarlberg@\textbf{Vorarlberg}|pwk} und die Schweiz\oindex{Schweiz@\textbf{Schweiz}|pwk} am 27. 8. 1900 in Trafoi\oindex{Trafoi@\textbf{Trafoi}|pwk} an. Van-Jung\pwindex{Van-Jung, Leo 15.\,10.\,1866 Odessa – 2.\,7.\,1939 Riga@\textsc{Van-Jung, Leo} (15.\,10.\,1866 Odessa – 2.\,7.\,1939 Riga), \emph{Gesangspädagoge, Mathematiker}|pwk} war nur bis
                     Schruns\oindex{Schruns@\textbf{Schruns}, \emph{Verwaltungsgebiet}|pwk}, Beer-Hofmann\pwindex{Beer-Hofmann, Richard 11.\,7.\,1866 Wien – 26.\,9.\,1945 New York City@\textsc{Beer-Hofmann, Richard} (11.\,7.\,1866 Wien – 26.\,9.\,1945 New York City), \emph{Schriftsteller}|pwk} und Kerr\pwindex{Kerr, Alfred 25.\,12.\,1867 Breslau – 12.\,10.\,1948 Hamburg@\textsc{Kerr, Alfred} (25.\,12.\,1867 Breslau – 12.\,10.\,1948 Hamburg), \emph{Schriftsteller, Kritiker}|pwk} waren bis Pontresina\oindex{Pontresina@\textbf{Pontresina}|pwk} mit
                  dabei. Beer-Hofmann\pwindex{Beer-Hofmann, Richard 11.\,7.\,1866 Wien – 26.\,9.\,1945 New York City@\textsc{Beer-Hofmann, Richard} (11.\,7.\,1866 Wien – 26.\,9.\,1945 New York City), \emph{Schriftsteller}|pwk} dokumentierte die
                  Wanderung in einer Fotoserie (vgl. Heinrich Schnitzler, Christian
                     Brandstätter und Reinhard Urbach (Herausgeber): \emph{Arthur Schnitzler.
                        Sein Leben. Sein Werk. Seine Zeit}. Mit 324 Abbildungen.
                     Frankfurt
                     am Main: \emph{S. Fischer}{ }1981, S. 79).}}}\label{K_L02920-2}?\pend
           
\pstart
           2.) Was könnte man unternehmen? Dolomiten\oindex{Dolomiten@\textbf{Dolomiten}, \emph{Gebirge}|pw}?\pend
           
\pstart
           3.) Würden \textsc{Richard\pwindex{Beer-Hofmann, Richard 11.\,7.\,1866 Wien – 26.\,9.\,1945 New York City@\textsc{Beer-Hofmann, Richard} (11.\,7.\,1866 Wien – 26.\,9.\,1945 New York City), \emph{Schriftsteller}|pw}} und {\pb}\textsc{Leo\pwindex{Van-Jung, Leo 15.\,10.\,1866 Odessa – 2.\,7.\,1939 Riga@\textsc{Van-Jung, Leo} (15.\,10.\,1866 Odessa – 2.\,7.\,1939 Riga), \emph{Gesangspädagoge, Mathematiker}|pw}} mitkommen?\pend
           
\pstart
           4.) Was macht \textsc{Richard\pwindex{Beer-Hofmann, Richard 11.\,7.\,1866 Wien – 26.\,9.\,1945 New York City@\textsc{Beer-Hofmann, Richard} (11.\,7.\,1866 Wien – 26.\,9.\,1945 New York City), \emph{Schriftsteller}|pw}} überhaupt in dieſem Sommer?\pend
           
\pstart
           5.) Wäre es Dir recht, wenn \textsc{Kerr\pwindex{Kerr, Alfred 25.\,12.\,1867 Breslau – 12.\,10.\,1948 Hamburg@\textsc{Kerr, Alfred} (25.\,12.\,1867 Breslau – 12.\,10.\,1948 Hamburg), \emph{Schriftsteller, Kritiker}|pw}} mitkäme? Ich habe ihm von der Idee geſprochen und ihn zum Mitkommen animirt. Er
               thäte es{ }ſehr gern, iſt aber Dir gegenüber etwas{ }ſchüchtern und erwartet, daß Du ihn
               dazu aufforderſt. Bitte, \label{K_L02920-3v}\edtext{ſchreib’
                  ihm}{\lemma{\textnormal{\emph{schreib’
                  ihm}}}\Cendnote{\textnormal{Vgl. Schnitzlers Brief an Alfred Kerr\pwindex{Kerr, Alfred 25.\,12.\,1867 Breslau – 12.\,10.\,1948 Hamburg@\textsc{Kerr, Alfred} (25.\,12.\,1867 Breslau – 12.\,10.\,1948 Hamburg), \emph{Schriftsteller, Kritiker}|pwk} vom 21. 6. 1900, in dem Schnitzler auf die Aufforderung Goldmanns\pwindex{Goldmann, Paul 31.\,1.\,1865 Breslau – 25.\,9.\,1935 Wien@\textsc{Goldmann, Paul} (31.\,1.\,1865 Breslau – 25.\,9.\,1935 Wien), \emph{Schriftsteller, Journalist}|pwk} verwies, er solle Kerr\pwindex{Kerr, Alfred 25.\,12.\,1867 Breslau – 12.\,10.\,1948 Hamburg@\textsc{Kerr, Alfred} (25.\,12.\,1867 Breslau – 12.\,10.\,1948 Hamburg), \emph{Schriftsteller, Kritiker}|pwk} zur Teilnahme einladen. Elgin
                     Helmstaedt (Herausgeberin): \emph{Alfred Kerr, Arthur Schnitzler. »Es ist eine
                        sehr seltsame Gefühlsmischung, die Sie erwecken«. Briefwechsel
                        1896–1925}. In: \emph{Sinn und Form}, Jg. 69, H. 5,
                         September/Oktober 2017, S. 581–618, hier:
                     S. 602.}}}\label{K_L02920-3} jedenfalls, daß ich Dir{ }ſeine eventuelle Bereitwilligkeit
               mitgetheilt habe, und \strikeout{f\textcolor{gray}{ordere}
                     ih\textcolor{gray}{n}}{ }\strikeout{\textcolor{gray}{a}uf}{ }ſage ihm etwas Freundliches {\pb}darüber. Selbſt wenn ich nicht mitkäme, könnteſt Du
               ja mit ihm\pwindex{Kerr, Alfred 25.\,12.\,1867 Breslau – 12.\,10.\,1948 Hamburg@\textsc{Kerr, Alfred} (25.\,12.\,1867 Breslau – 12.\,10.\,1948 Hamburg), \emph{Schriftsteller, Kritiker}|pwv} immer etwas
               verabreden und hätteſt dann einen{ }ſehr angenehmen Reiſebegleiter für die
               nicht-weiblichen Tage.\pend
           
\pstart
           Kann ich die Reiſe aber nicht ermöglichen,{ }ſo werde ich es wenigſtens einzurichten{ }ſuchen, daß ich \label{K_L02920-4v}\edtext{Anfang September auf ein paar Tage nach Wien\oindex{Wien@\textbf{Wien}, \emph{Verwaltungsgebiet}|pw}}{\lemma{\textnormal{\emph{Anfang … Wien}}}\Cendnote{\textnormal{Goldmann\pwindex{Goldmann, Paul 31.\,1.\,1865 Breslau – 25.\,9.\,1935 Wien@\textsc{Goldmann, Paul} (31.\,1.\,1865 Breslau – 25.\,9.\,1935 Wien), \emph{Schriftsteller, Journalist}|pwk} hielt sich jedenfalls vom 6. 9. 1900 bis zum XXXX Auszeichnungsfehler: Dokument L02930 nicht gefunden in Wien\oindex{Wien@\textbf{Wien}, \emph{Verwaltungsgebiet}|pwk} auf.}}}\label{K_L02920-4} komme.\pend
           
\pstart
           Bitte, antworte mir \uline{bald} auf meine Fragen und{ }ſchreibe \uline{bald} an \textsc{Kerr\pwindex{Kerr, Alfred 25.\,12.\,1867 Breslau – 12.\,10.\,1948 Hamburg@\textsc{Kerr, Alfred} (25.\,12.\,1867 Breslau – 12.\,10.\,1948 Hamburg), \emph{Schriftsteller, Kritiker}|pw}}!\pend
           
\pstart
           Viele treue Grüße! {\\[\baselineskip]}Dein {\\[\baselineskip]}\spacefill\mbox{Paul Goldmann.}\pend
           \leftskip=0em{}\selectlanguage{ngerman}\endnumbering\briefempfaengerindex{Schnitzler, Arthur@\textsc{Schnitzler, Arthur}!zzzGoldmann, Paul@\emph{von Paul Goldmann}!1900-06-161@{16. 6. [1900]}|)be}\mylabel{L02920h}  \newcommand{\dateiname}{L02920}\newcommand{\titel}{Paul Goldmann an Arthur Schnitzler, 16. 6. [1900]}\newcommand{\editorInnen}{Martin Anton Müller und Laura Untner}%% latex-leseansicht-abspann.tex
%% Abspann für die Leseansicht.
%% Der Schalter \ifkorrekturansicht ist bereits durch den Vorspann gesetzt.

%% latex-abspann.tex
%% Gemeinsamer Abspann für Korrekturansicht und Leseansicht.
%% Setzt den Schalter \ifkorrekturansicht voraus (gesetzt in den
%% einbindenden Dateien latex-korrekturansicht-abspann.tex bzw.
%% latex-leseansicht-abspann.tex).
%% ---------------------------------------------------------------

\normalsize

% Das esempio-Environment wird nur in der Leseansicht benötigt
\ifkorrekturansicht\else
\newenvironment{esempio}[3]%
{
    \vspace{1.5ex}
    \rlap{\underline{#1}}
    \par
    \setlength{\parindent}{0cm}
    \nopagebreak
    \leftskip=#2cm
    \rightskip=#3cm
}
{
    \par
}
\fi

\doendnotes{C}
\bigskip
\vfill

\clearpage

\footnotesize

\ifkorrekturansicht
  \lohead{\textsc{register}}
\fi

% theindex-Environment neu definieren ohne reledmac
\makeatletter
\renewenvironment{theindex}{%
  \ifkorrekturansicht
    \section*{\indexname}%
  \else
    \subsubsection*{Index der erwähnten Entitäten}%
  \fi
  \setlength{\parindent}{0pt}%
  \setlength{\parskip}{0pt plus 0.3pt}%
  \let\item\@idxitem
}{%
  \ifkorrekturansicht\clearpage\fi
}
\makeatother

\IfFileExists{\jobname-pw.ind}{\input{\jobname-pw.ind}}{}

% Quellenangabe nur in der Leseansicht
\ifkorrekturansicht\else
% Fallback-Definitionen, falls die .tex-Datei \titel etc. nicht gesetzt hat
\providecommand{\titel}{}
\providecommand{\editorInnen}{}
\providecommand{\dateiname}{\jobname}

\vspace{3cm}

\vfill

\footnotesize
\textsc{Quelle}: \titel. Herausgegeben von {\editorInnen}. In: \emph{Arthur Schnitzler: Briefwechsel mit Autorinnen und Autoren}.
 Digitale Edition, https://schnitzler-briefe.acdh.oeaw.ac.at/{\dateiname}.html (Stand \today)
\fi

\end{document}


