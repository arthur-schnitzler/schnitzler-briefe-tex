%% latex-korrekturansicht-vorspann.tex
%% Vorspann für die Korrekturansicht.
%% Lädt die gemeinsame Datei latex-vorspann.tex mit gesetztem Schalter.

\newif\ifkorrekturansicht
\korrekturansichttrue

\input{../tex-inputs/latex-vorspann}


\section[ Paul Goldmann an Arthur Schnitzler, 16. 6. {[}1900{]}]{L02920 Paul Goldmann an Arthur Schnitzler, 16. 6. {[}1900{]}}
\nopagebreak\mylabel{L02920v}
\rehead{ }\normalsize\beginnumbering\briefempfaengerindex{Schnitzler, Arthur@\textsc{Schnitzler, Arthur}!zzzGoldmann, Paul@\emph{von Paul Goldmann}!1900-06-161@{16. 6. {[}1900{]}}|(be}
\toendnotes[C]{\smallbreak\pagebreak[2]}\Standort{DLA, A:Schnitzler, HS.NZ85.1.3170.}
\physDesc{Brief, 1 Blatt, 4 Seiten, 1692 Zeichen
\newline{}Handschrift: blaue Tinte, deutsche Kurrent
\newline{}Schnitzler: 1) mit Bleistift das Jahr »1900« vermerkt  2) mit rotem Buntstift eine Unterstreichung}\toendnotes[C]{\smallbreak}
\pstart
           {\pb}\textcolor{gray}{\textbf{DESSAUERSTRASSE 19}}\oindex{Dessauer Strasse@\textbf{Dessauer Straße}, \emph{Straße (K.STR)}|pw}\pend
           
\pstart
           \raggedleft{}Berlin\oindex{Berlin@\textbf{Berlin}, \emph{P.PPLC}|pw}, 16. Juni.\pend
           
\pstart\center{}Mein lieber Freund,\pend\vspace{0.5em}
\pstart
           Ich danke Dir für die ausführliche Beantwortung meiner Briefe und freue mich ſehr,
               wieder einmal Direktes von Dir gehört zu haben. Nächſte Woche werde ich
               vorausſichtlich viel zu thun haben. Ich antworte Dir daher gleich, und zwar nur wegen
               der Sommerpläne.\pend
           
\pstart
           Du ſcheinſt einen überwiegend \label{K_L02920-1v}\edtext{weiblichen Sommer}{\lemma{\textnormal{\emph{weiblichen Sommer}}}\Cendnote{\textnormal{Schnitzler war einerseits mit seiner
                  zukünftigen Ehefrau Olga Gussmann\pwindex{Schnitzler, Olga 17.01.1882 – 13.01.1970@\textsc{Schnitzler, Olga} (17.01.1882 – 13.01.1970), \emph{Schauspieler/Schauspielerin, Sänger/Sängerin}|pwk},
                  andererseits mit Leopoldine Müller\pwindex{Mueller, Leopoldine 01.11.1873 – 18.01.1946@\textsc{Müller, Leopoldine} (01.11.1873 – 18.01.1946)|pwk} in einer
                  intimen Beziehung.}}}\label{K_L02920-1} verleben zu wollen.\pend
           
\pstart
           Ich ſinne noch über ein Mittel zur Löſung der Finanz-Schwierigkeiten {\pb}nach, die einer Urlaubsreiſe diesmal bei mir \strikeout{\textcolor{gray}{i}} entgegenſtehen. Habe ich es gefunden und bekomme ich Urlaub – zwei noch ſehr
               fragliche Dinge – ſo möchte ich \strikeout{End\textcolor{gray}{e}} Anfang Auguſt eine Fußwanderung in den Alpen\oindex{Alpen@\textbf{Alpen}, \emph{kein passender Code gefunden}|pw}, in Tirol\oindex{Tirol@\textbf{Tirol}, \emph{A.ADM1}|pw} womöglich, machen. Erſtens weil es ſchön iſt, zweitens aus
               Abmagerungs-Gründen. Denn ich werde dick wie ein Schwein. Ich frage Dich alſo:\pend
           
\pstart
           1.) Möchteſt Du bei ſo etwas \label{K_L02920-2v}\edtext{mitmachen}{\lemma{\textnormal{\emph{mitmachen}}}\Cendnote{\textnormal{Tatsächlich unternahmen
                     Schnitzler und Goldmann\pwindex{Goldmann, Paul 31.01.1865 – 25.09.1935@\textsc{Goldmann, Paul} (31.01.1865 – 25.09.1935), \emph{Schriftsteller/Schriftstellerin, Journalist/Journalistin}|pwk} in der zweiten Augusthälfte{ }1900 gemeinsam mit Richard Beer-Hofmann\pwindex{Beer-Hofmann, Richard 1866-07-11 – 1945-09-26@\textsc{Beer-Hofmann, Richard} (1866-07-11 – 1945-09-26), \emph{Schriftsteller/Schriftstellerin}|pwk}, Alfred Kerr\pwindex{Kerr, Alfred 25.12.1867 – 12.10.1948@\textsc{Kerr, Alfred} (25.12.1867 – 12.10.1948), \emph{Schriftsteller/Schriftstellerin, Kritiker/Kritikerin}|pwk} und Leo Van-Jung\pwindex{Van-Jung, Leo 15.10.1866 – 02.07.1939@\textsc{Van-Jung, Leo} (15.10.1866 – 02.07.1939), \emph{Gesangspädagoge/Gesangspädagogin, Mathematiker/Mathematikerin}|pwk} eine
                     Alpen\oindex{Alpen@\textbf{Alpen}, \emph{kein passender Code gefunden}|pwk}wanderung. Am 16. 8. 1900 in Innsbruck\oindex{Innsbruck@\textbf{Innsbruck}, \emph{A.ADM2}|pwk} beginnend, kamen sie über Vorarlberg\oindex{Vorarlberg@\textbf{Vorarlberg}, \emph{Teil eines Landes (A.LNDX)}|pwk} und die Schweiz\oindex{Schweiz@\textbf{Schweiz}, \emph{A.PCLI}|pwk} am 27. 8. 1900 in Trafoi\oindex{Trafoi@\textbf{Trafoi}, \emph{P.PPL}|pwk} an. Van-Jung\pwindex{Van-Jung, Leo 15.10.1866 – 02.07.1939@\textsc{Van-Jung, Leo} (15.10.1866 – 02.07.1939), \emph{Gesangspädagoge/Gesangspädagogin, Mathematiker/Mathematikerin}|pwk} war nur bis
                     Schruns\oindex{Schruns@\textbf{Schruns}, \emph{A.ADM3}|pwk}, Beer-Hofmann\pwindex{Beer-Hofmann, Richard 1866-07-11 – 1945-09-26@\textsc{Beer-Hofmann, Richard} (1866-07-11 – 1945-09-26), \emph{Schriftsteller/Schriftstellerin}|pwk} und Kerr\pwindex{Kerr, Alfred 25.12.1867 – 12.10.1948@\textsc{Kerr, Alfred} (25.12.1867 – 12.10.1948), \emph{Schriftsteller/Schriftstellerin, Kritiker/Kritikerin}|pwk} waren bis Pontresina\oindex{Pontresina@\textbf{Pontresina}, \emph{P.PPL}|pwk} mit
                  dabei. Beer-Hofmann\pwindex{Beer-Hofmann, Richard 1866-07-11 – 1945-09-26@\textsc{Beer-Hofmann, Richard} (1866-07-11 – 1945-09-26), \emph{Schriftsteller/Schriftstellerin}|pwk} dokumentierte die
                  Wanderung in einer Fotoserie (vgl. Heinrich Schnitzler, Christian
                     Brandstätter und Reinhard Urbach (Herausgeber): \emph{Arthur Schnitzler.
                        Sein Leben. Sein Werk. Seine Zeit}. Mit 324 Abbildungen.
                     Frankfurt
                     am Main: \emph{S. Fischer}{ }1981, S. 79).}}}\label{K_L02920-2}?\pend
           
\pstart
           2.) Was könnte man unternehmen? Dolomiten\oindex{Dolomiten@\textbf{Dolomiten}, \emph{Gebirge (N.GBR)}|pw}?\pend
           
\pstart
           3.) Würden \textsc{Richard\pwindex{Beer-Hofmann, Richard 1866-07-11 – 1945-09-26@\textsc{Beer-Hofmann, Richard} (1866-07-11 – 1945-09-26), \emph{Schriftsteller/Schriftstellerin}|pw}} und {\pb}\textsc{Leo\pwindex{Van-Jung, Leo 15.10.1866 – 02.07.1939@\textsc{Van-Jung, Leo} (15.10.1866 – 02.07.1939), \emph{Gesangspädagoge/Gesangspädagogin, Mathematiker/Mathematikerin}|pw}} mitkommen?\pend
           
\pstart
           4.) Was macht \textsc{Richard\pwindex{Beer-Hofmann, Richard 1866-07-11 – 1945-09-26@\textsc{Beer-Hofmann, Richard} (1866-07-11 – 1945-09-26), \emph{Schriftsteller/Schriftstellerin}|pw}} überhaupt in dieſem Sommer?\pend
           
\pstart
           5.) Wäre es Dir recht, wenn \textsc{Kerr\pwindex{Kerr, Alfred 25.12.1867 – 12.10.1948@\textsc{Kerr, Alfred} (25.12.1867 – 12.10.1948), \emph{Schriftsteller/Schriftstellerin, Kritiker/Kritikerin}|pw}} mitkäme? Ich habe ihm von der Idee geſprochen und ihn zum Mitkommen animirt. Er
               thäte es ſehr gern, iſt aber Dir gegenüber etwas ſchüchtern und erwartet, daß Du ihn
               dazu aufforderſt. Bitte, \label{K_L02920-3v}\edtext{ſchreib’
                  ihm}{\lemma{\textnormal{\emph{ſchreib’
                  ihm}}}\Cendnote{\textnormal{Vgl. Schnitzlers Brief an Alfred Kerr\pwindex{Kerr, Alfred 25.12.1867 – 12.10.1948@\textsc{Kerr, Alfred} (25.12.1867 – 12.10.1948), \emph{Schriftsteller/Schriftstellerin, Kritiker/Kritikerin}|pwk} vom 21. 6. 1900, in dem Schnitzler auf die Aufforderung Goldmanns\pwindex{Goldmann, Paul 31.01.1865 – 25.09.1935@\textsc{Goldmann, Paul} (31.01.1865 – 25.09.1935), \emph{Schriftsteller/Schriftstellerin, Journalist/Journalistin}|pwk} verwies, er solle Kerr\pwindex{Kerr, Alfred 25.12.1867 – 12.10.1948@\textsc{Kerr, Alfred} (25.12.1867 – 12.10.1948), \emph{Schriftsteller/Schriftstellerin, Kritiker/Kritikerin}|pwk} zur Teilnahme einladen. Elgin
                     Helmstaedt (Herausgeberin): \emph{Alfred Kerr, Arthur Schnitzler. »Es ist eine
                        sehr seltsame Gefühlsmischung, die Sie erwecken«. Briefwechsel
                        1896–1925}. In: \emph{Sinn und Form}, Jg. 69, H. 5,
                         September/Oktober 2017, S. 581–618, hier:
                     S. 602.}}}\label{K_L02920-3} jedenfalls, daß ich Dir ſeine eventuelle Bereitwilligkeit
               mitgetheilt habe, und \strikeout{f\textcolor{gray}{ordere}
                     ih\textcolor{gray}{n}}{ }\strikeout{\textcolor{gray}{a}uf} ſage ihm etwas Freundliches {\pb}darüber. Selbſt wenn ich nicht mitkäme, könnteſt Du
               ja mit ihm\pwindex{Kerr, Alfred 25.12.1867 – 12.10.1948@\textsc{Kerr, Alfred} (25.12.1867 – 12.10.1948), \emph{Schriftsteller/Schriftstellerin, Kritiker/Kritikerin}|pwv} immer etwas
               verabreden und hätteſt dann einen ſehr angenehmen Reiſebegleiter für die
               nicht-weiblichen Tage.\pend
           
\pstart
           Kann ich die Reiſe aber nicht ermöglichen, ſo werde ich es wenigſtens einzurichten
               ſuchen, daß ich \label{K_L02920-4v}\edtext{Anfang September auf ein paar Tage nach Wien\oindex{Wien@\textbf{Wien}, \emph{A.ADM2}|pw}}{\lemma{\textnormal{\emph{Anfang … Wien}}}\Cendnote{\textnormal{Goldmann\pwindex{Goldmann, Paul 31.01.1865 – 25.09.1935@\textsc{Goldmann, Paul} (31.01.1865 – 25.09.1935), \emph{Schriftsteller/Schriftstellerin, Journalist/Journalistin}|pwk} hielt sich jedenfalls vom 6. 9. 1900 bis zum 16. 9. [1900] in Wien\oindex{Wien@\textbf{Wien}, \emph{A.ADM2}|pwk} auf.}}}\label{K_L02920-4} komme.\pend
           
\pstart
           Bitte, antworte mir \uline{bald} auf meine Fragen und
               ſchreibe \uline{bald} an \textsc{Kerr\pwindex{Kerr, Alfred 25.12.1867 – 12.10.1948@\textsc{Kerr, Alfred} (25.12.1867 – 12.10.1948), \emph{Schriftsteller/Schriftstellerin, Kritiker/Kritikerin}|pw}}!\pend
           
\pstart
           Viele treue Grüße! {\\[\baselineskip]}Dein {\\[\baselineskip]}\spacefill\mbox{Paul Goldmann.}\pend
           \leftskip=0em{}\selectlanguage{ngerman}\endnumbering\briefempfaengerindex{Schnitzler, Arthur@\textsc{Schnitzler, Arthur}!zzzGoldmann, Paul@\emph{von Paul Goldmann}!1900-06-161@{16. 6. {[}1900{]}}|)be}\mylabel{L02920h}  \normalsize

\doendnotes{C}
\bigskip
\vfill

\clearpage

\footnotesize

\lohead{\textsc{register}}

% Definiere theindex-Environment komplett neu ohne reledmac
\makeatletter
\renewenvironment{theindex}{%
  \section*{\indexname}%
  \setlength{\parindent}{0pt}%
  \setlength{\parskip}{0pt plus 0.3pt}%
  \let\item\@idxitem
}{%
  \clearpage
}
\makeatother

\IfFileExists{\jobname-pw.ind}{\input{\jobname-pw.ind}}{}

\end{document}

      