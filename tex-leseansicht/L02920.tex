%% latex-leseansicht-vorspann.tex
%% Vorspann für die Leseansicht.
%% Lädt die gemeinsame Datei latex-vorspann.tex mit nicht gesetztem Schalter.

\newif\ifkorrekturansicht
\korrekturansichtfalse

\input{../tex-inputs/latex-vorspann}

\begin{center}
            \textcolor{red}{ENTWURF, NICHT FERTIG KORRIGIERT}
                      \end{center}
            
         
         \newcommand{\erwaehntePersonen}{Personen: Richard Beer-Hofmann, Alfred Kerr, Leopoldine Müller, Olga Schnitzler, Leo Van-Jung}
         \newcommand{\erwaehnteInstitutionen}{}
         \newcommand{\erwaehnteOrte}{Orte: Alpen, Berlin, Dessauer Straße, Dolomiten, Innsbruck, Pontresina, Schruns, Schweiz, Tirol, Trafoi, Vorarlberg, Wien}
         \newcommand{\erwaehnteWerke}{
               \section[ Paul Goldmann an Arthur Schnitzler, 16. 6. {[}1900{]}]{ Paul Goldmann an Arthur Schnitzler, 16. 6. {[}1900{]}}\nopagebreak\mylabel{v}\rehead{ }\begin{ledgroupsized}[t]{13cm}\normalsize\beginnumbering \toendnotes[C]{\smallbreak\pagebreak[2]} \Standort{DLA, A:Schnitzler, HS.NZ85.1.3170.}
\physDesc{Brief, 1 Blatt, 4 Seiten
\newline{}Handschrift: blaue Tinte, deutsche Kurrent
\newline{}Schnitzler: 1) mit Bleistift das Jahr »1900« vermerkt  2) mit rotem Buntstift eine Unterstreichung}\toendnotes[C]{\smallbreak}\pstart{}{\pb}\textcolor{gray}{\textbf{DESSAUERSTRASSE 19}}\oindex{Dessauer Strasse@\textbf{Dessauer Straße}|pw}\pend{}{\bigskip}\pstart
           \raggedleft{}Berlin\oindex{Berlin@\textbf{Berlin}|pw}, 16. Juni.\pend
           \pstart\center{}Mein lieber Freund,\pend\pstart
           Ich danke Dir für die ausführliche Beantwortung meiner Briefe und freue mich ſehr,
               wieder einmal Direktes von Dir gehört zu haben. Nächſte Woche werde ich
               vorausſichtlich viel zu thun haben. Ich antworte Dir daher gleich, und zwar nur wegen
               der Sommerpläne.\pend
           \pstart
           Du ſcheinſt einen überwiegend \label{K_L02920-1v}\edtext{weiblichen Sommer}{\lemma{\textnormal{\emph{weiblichen Sommer}}}\Cendnote{\textnormal{Bezug auf Olga Gussmann\pwindex{Schnitzler, Olga 17.01.1882 – 13.01.1970@\textsc{Schnitzler, Olga} (17.01.1882 – 13.01.1970), \emph{Schauspielerin, Sängerin}|pwk} (später Schnitzler\pwindex{Schnitzler, Olga 17.01.1882 – 13.01.1970@\textsc{Schnitzler, Olga} (17.01.1882 – 13.01.1970), \emph{Schauspielerin, Sängerin}|pwkv}) und Leopoldine Müller\pwindex{Mueller, Leopoldine 01.11.1873 – 18.01.1946@\textsc{Müller, Leopoldine} (01.11.1873 – 18.01.1946)|pwk}}}}\label{K_L02920-1h} verleben zu wollen.\pend
           \pstart
           Ich ſinne noch über ein Mittel zur Löſung der Finanz-Schwierigkeiten {\pb}nach, die einer Urlaubsreiſe diesmal bei mir \strikeout{\textcolor{gray}{i}} entgegenſtehen. Habe ich es gefunden und bekomme ich Urlaub – zwei noch ſehr
               fragliche Dinge – ſo möchte ich \strikeout{End\textcolor{gray}{e}} Anfang Auguſt eine Fußwanderung in den Alpen\oindex{Alpen@\textbf{Alpen}|pw}, in Tirol\oindex{Tirol@\textbf{Tirol}|pw} womöglich, machen. Erſtens weil es ſchön iſt, zweitens aus
               Abmagerungs-Gründen. Denn ich werde dick wie ein Schwein. Ich frage Dich alſo:\pend
           \pstart
           1.) Möchteſt Du bei ſo etwas \label{K_L02920-2v}\edtext{mitmachen}{\lemma{\textnormal{\emph{mitmachen}}}\Cendnote{\textnormal{Tatsächlich unternahmen
                     Schnitzler\pwindex{Schnitzler, Arthur 15.05.1862 – 21.10.1931@\textsc{Schnitzler, Arthur} (15.05.1862 – 21.10.1931), \emph{Schriftsteller, Mediziner}|pwk} und Goldmann\pwindex{Goldmann, Paul 31.01.1865 – 25.09.1935@\textsc{Goldmann, Paul} (31.01.1865 – 25.09.1935), \emph{Schriftsteller, Journalist}|pwk} in der zweiten Augusthälte 1900 gemeinsam mit Richard Beer-Hofmann\pwindex{Beer-Hofmann, Richard 1866-07-11 – 1945-09-26@\textsc{Beer-Hofmann, Richard} (1866-07-11 – 1945-09-26), \emph{Schriftsteller}|pwk}, Alfred Kerr\pwindex{Kerr, Alfred 25.12.1867 – 12.10.1948@\textsc{Kerr, Alfred} (25.12.1867 – 12.10.1948), \emph{Schriftsteller, Kritiker}|pwk} und Leo Van-Jung\pwindex{Van-Jung, Leo 15.10.1866 – 02.07.1939@\textsc{Van-Jung, Leo} (15.10.1866 – 02.07.1939), \emph{Gesangspädagoge, Mathematiker}|pwk} eine
                     Alpen\oindex{Alpen@\textbf{Alpen}|pwk}wanderung. Am 16. 8. 1900 in Innsbruck\oindex{Innsbruck@\textbf{Innsbruck}|pwk} beginnend, kamen sie über Vorarlberg\oindex{Vorarlberg@\textbf{Vorarlberg}|pwk} und die Schweiz\oindex{Schweiz@\textbf{Schweiz}|pwk} am 27. 8. 1900 in Trafoi\oindex{Trafoi@\textbf{Trafoi}|pwk} an. Van-Jung\pwindex{Van-Jung, Leo 15.10.1866 – 02.07.1939@\textsc{Van-Jung, Leo} (15.10.1866 – 02.07.1939), \emph{Gesangspädagoge, Mathematiker}|pwk} war nur bis
                     Schruns\oindex{Schruns@\textbf{Schruns}|pwk} mit dabei, Beer-Hofmann\pwindex{Beer-Hofmann, Richard 1866-07-11 – 1945-09-26@\textsc{Beer-Hofmann, Richard} (1866-07-11 – 1945-09-26), \emph{Schriftsteller}|pwk} und Kerr\pwindex{Kerr, Alfred 25.12.1867 – 12.10.1948@\textsc{Kerr, Alfred} (25.12.1867 – 12.10.1948), \emph{Schriftsteller, Kritiker}|pwk} bis Pontresina\oindex{Pontresina@\textbf{Pontresina}|pwk}. Beer-Hofmann\pwindex{Beer-Hofmann, Richard 1866-07-11 – 1945-09-26@\textsc{Beer-Hofmann, Richard} (1866-07-11 – 1945-09-26), \emph{Schriftsteller}|pwk} dokumentierte die Wanderung in
                  einer Fotoserie (vgl. Heinrich Schnitzler, Christian Brandstätter und
                     Reinhard Urbach (Hg.): \emph{Arthur Schnitzler. Sein Leben. Sein
                        Werk. Seine Zeit. Mit 324 Abbildungen}. Frankfurt
                     a. M.: \emph{S. Fischer}{ }1981, S. 79).}}}\label{K_L02920-2h}?\pend
           \pstart
           2.) Was könnte man unternehmen? Dolomiten\oindex{Dolomiten@\textbf{Dolomiten}|pw}?\pend
           \pstart
           3.) Würden \textsc{Richard\pwindex{Beer-Hofmann, Richard 1866-07-11 – 1945-09-26@\textsc{Beer-Hofmann, Richard} (1866-07-11 – 1945-09-26), \emph{Schriftsteller}|pw}} und {\pb}\textsc{Leo\pwindex{Van-Jung, Leo 15.10.1866 – 02.07.1939@\textsc{Van-Jung, Leo} (15.10.1866 – 02.07.1939), \emph{Gesangspädagoge, Mathematiker}|pw}} mitkommen?\pend
           \pstart
           4.) Was macht \textsc{Richard\pwindex{Beer-Hofmann, Richard 1866-07-11 – 1945-09-26@\textsc{Beer-Hofmann, Richard} (1866-07-11 – 1945-09-26), \emph{Schriftsteller}|pw}} überhaupt in dieſem Sommer?\pend
           \pstart
           5.) Wäre es Dir recht, wenn \textsc{Kerr\pwindex{Kerr, Alfred 25.12.1867 – 12.10.1948@\textsc{Kerr, Alfred} (25.12.1867 – 12.10.1948), \emph{Schriftsteller, Kritiker}|pw}} mitkäme? Ich habe ihm von der Idee geſprochen und ihn zum Mitkommen animirt. Er
               thäte es ſehr gern, iſt aber Dir gegenüber etwas ſchüchtern und erwartet, daß Du ihn
               dazu aufforderſt. Bitte, \label{K_L02920-3v}\edtext{ſchreib’
                  ihm}{\lemma{\textnormal{\emph{ſchreib’
                  ihm}}}\Cendnote{\textnormal{Vgl. Schnitzler\pwindex{Schnitzler, Arthur 15.05.1862 – 21.10.1931@\textsc{Schnitzler, Arthur} (15.05.1862 – 21.10.1931), \emph{Schriftsteller, Mediziner}|pwk}s Brief an Alfred Kerr\pwindex{Kerr, Alfred 25.12.1867 – 12.10.1948@\textsc{Kerr, Alfred} (25.12.1867 – 12.10.1948), \emph{Schriftsteller, Kritiker}|pwk} vom 21. 6. 1900:
                     Sieglinde Geisel (Hg.): \emph{Alfred Kerr, Arthur Schnitzler.
                        »Es ist eine sehr seltsame Gefühlsmischung, die Sie erwecken«. Briefwechsel
                        1896–1925}. In: \emph{Sinn und Form} 69/5 (2017), S. 581–618, hier: S. 602.}}}\label{K_L02920-3h}
               jedenfalls, daß ich Dir ſeine eventuelle Bereitwilligkeit mitgetheilt habe, und \strikeout{f\textcolor{gray}{ordere} ih\textcolor{gray}{n}}{ }\strikeout{\textcolor{gray}{a}uf} ſage ihm etwas Freundliches {\pb}darüber. Selbſt wenn ich nicht mitkäme, könnteſt Du
               ja mit ihm\pwindex{Kerr, Alfred 25.12.1867 – 12.10.1948@\textsc{Kerr, Alfred} (25.12.1867 – 12.10.1948), \emph{Schriftsteller, Kritiker}|pwv} immer etwas
               verabreden und hätteſt dann einen ſehr angenehmen Reiſebegleiter für die
               nicht-weiblichen Tage.\pend
           \pstart
           Kann ich die Reiſe aber nicht ermöglichen, ſo werde ich es wenigſtens einzurichten
               ſuchen, daß ich \label{K_L02920-4v}\edtext{Anfang September auf ein paar Tage nach Wien\oindex{Wien@\textbf{Wien}|pw}}{\lemma{\textnormal{\emph{Anfang … Wien}}}\Cendnote{\textnormal{Goldmann\pwindex{Goldmann, Paul 31.01.1865 – 25.09.1935@\textsc{Goldmann, Paul} (31.01.1865 – 25.09.1935), \emph{Schriftsteller, Journalist}|pwk} hielt sich jedenfalls von 6. 9. 1900 bis 16. 9. [1900] in Wien\oindex{Wien@\textbf{Wien}|pwk} auf.}}}\label{K_L02920-4h} komme.\pend
           \pstart
           Bitte, antworte mir \uline{bald} auf meine Fragen und
               ſchreibe \uline{bald} an \textsc{Kerr\pwindex{Kerr, Alfred 25.12.1867 – 12.10.1948@\textsc{Kerr, Alfred} (25.12.1867 – 12.10.1948), \emph{Schriftsteller, Kritiker}|pw}}!\pend
           \pstart
           Viele treue Grüße! {\\[\baselineskip]}Dein {\\[\baselineskip]}\spacefill\mbox{Paul Goldmann.}\pend
           \leftskip=0em{}
         
         \endnumbering\mylabel{h}\end{ledgroupsized}\begin{anhang}\end{anhang}\newcommand{\dateiname}{L02920}\newcommand{\titel}{Paul Goldmann an Arthur Schnitzler, 16. 6. [1900]}\newcommand{\editorInnen}{Martin Anton Müller und Laura Untner}%% latex-leseansicht-abspann.tex
%% Abspann für die Leseansicht.
%% Der Schalter \ifkorrekturansicht ist bereits durch den Vorspann gesetzt.

%% latex-abspann.tex
%% Gemeinsamer Abspann für Korrekturansicht und Leseansicht.
%% Setzt den Schalter \ifkorrekturansicht voraus (gesetzt in den
%% einbindenden Dateien latex-korrekturansicht-abspann.tex bzw.
%% latex-leseansicht-abspann.tex).
%% ---------------------------------------------------------------

\normalsize

% Das esempio-Environment wird nur in der Leseansicht benötigt
\ifkorrekturansicht\else
\newenvironment{esempio}[3]%
{
    \vspace{1.5ex}
    \rlap{\underline{#1}}
    \par
    \setlength{\parindent}{0cm}
    \nopagebreak
    \leftskip=#2cm
    \rightskip=#3cm
}
{
    \par
}
\fi

\doendnotes{C}
\bigskip
\vfill

\clearpage

\footnotesize

\ifkorrekturansicht
  \lohead{\textsc{register}}
\fi

% theindex-Environment neu definieren ohne reledmac
\makeatletter
\renewenvironment{theindex}{%
  \ifkorrekturansicht
    \section*{\indexname}%
  \else
    \subsubsection*{Index der erwähnten Entitäten}%
  \fi
  \setlength{\parindent}{0pt}%
  \setlength{\parskip}{0pt plus 0.3pt}%
  \let\item\@idxitem
}{%
  \ifkorrekturansicht\clearpage\fi
}
\makeatother

\IfFileExists{\jobname-pw.ind}{\input{\jobname-pw.ind}}{}

% Quellenangabe nur in der Leseansicht
\ifkorrekturansicht\else
% Fallback-Definitionen, falls die .tex-Datei \titel etc. nicht gesetzt hat
\providecommand{\titel}{}
\providecommand{\editorInnen}{}
\providecommand{\dateiname}{\jobname}

\vspace{3cm}

\vfill

\footnotesize
\textsc{Quelle}: \titel. Herausgegeben von {\editorInnen}. In: \emph{Arthur Schnitzler: Briefwechsel mit Autorinnen und Autoren}.
 Digitale Edition, https://schnitzler-briefe.acdh.oeaw.ac.at/{\dateiname}.html (Stand \today)
\fi

\end{document}


      