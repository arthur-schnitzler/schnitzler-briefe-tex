%% latex-leseansicht-vorspann.tex
%% Vorspann für die Leseansicht.
%% Lädt die gemeinsame Datei latex-vorspann.tex mit nicht gesetztem Schalter.

\newif\ifkorrekturansicht
\korrekturansichtfalse

\input{../tex-inputs/latex-vorspann}


\section[Richard Beer-Hofmann an Arthur Schnitzler, 19. 8. 1892]{L00115 Richard Beer-Hofmann an Arthur Schnitzler, 19. 8. 1892}
\nopagebreak\mylabel{L00115v}
\rehead{ }\normalsize\beginnumbering\briefempfaengerindex{Schnitzler, Arthur@\textsc{Schnitzler, Arthur}!zzzBeer-Hofmann, Richard@\emph{von Richard Beer-Hofmann}!1892-08-191@{19. 8. 1892}|(be}
\toendnotes[C]{\smallbreak\pagebreak[2]}
\correspDesc{Versand  durch Richard Beer-Hofmann am 19. 8. 1892 in Bad Ischl
\newline{}Erhalt  durch Arthur Schnitzler am 22. 8. 1892 in Wien}\toendnotes[C]{\smallbreak}
\Standort{CUL, Schnitzler, B 8.}
\physDesc{Brief, 2 Blätter, 8 Seiten, 1497 Zeichen
\newline{}Handschrift: Bleistift, lateinische Kurrent
\newline{}Schnitzler: mit Bleistift datiert: »19/8 92« und nummeriert: »9.« }
\buchAbdrucke{\weitereDrucke{Arthur Schnitzler, Richard Beer-Hofmann: \emph{Briefwechsel 1891–1931}. Herausgegeben von Konstanze Fliedl. Wien, Zürich: \emph{Europaverlag} 1992, S. 36–37.} }\toendnotes[C]{\smallbreak}
\pstart
           \noindent{}{\pb}Lieber Arthur! Sie wissen ja, wie schreibfaul ich bin, und wie sehr
               ich mir immer Zeit lasse.\pend
           
\pstart
           Also vor Allem: Ich freue mich sehr, \uline{sehr} sie auf ein
               paar Tage hier zu haben; mit Ihnen {\pb}werde ich freilich kaum gehen können; im Allgemeinen habe ich einen verdorbenen
                  So{\geminationm}er, schlechte Laune in xter Potenz, die erst jetzt
               etwas, nachlässt; gearbeitet {\pb}hab
               ich circa \uuline{15} (!) Druckzeilen – also – nichts. Ausser
               ein paar Gedanken, deren Wert äußerst p\substVorne{}\textsuperscript{o}\substDazwischen{}ro\substHinten{}blematisch ist, also ein verlorener So{\geminationm}er. In
               den nächsten {\pb}Tagen werde ich
               voraussichtlich meine Pantomime\pwindex{Beer-Hofmann, Richard 11.\,7.\,1866 Wien – 26.\,9.\,1945 New York City@\textsc{Beer-Hofmann, Richard} (11.\,7.\,1866 Wien – 26.\,9.\,1945 New York City), \emph{Schriftsteller}!Pierrot Hypnotiseur@\strich\emph{Pierrot Hypnotiseur}|pwv} an Sie senden, und Sie bitten \strikeout{Sie},
               dieselbe durch Ihren Abschreiber\pwindex{?? [Schreibkraft für Arthur Schnitzler] @\textsc{?? [Schreibkraft für Arthur Schnitzler]}|pwv} copiren zu lassen, da ich sie möglicherweise in der nächsten
               Zeit an irgend einen \label{K_L00115-1v}\edtext{Verleger}{\lemma{\textnormal{\emph{Verleger}}}\Cendnote{\textnormal{\emph{Pierrot hypnotiseur}\pwindex{Beer-Hofmann, Richard 11.\,7.\,1866 Wien – 26.\,9.\,1945 New York City@\textsc{Beer-Hofmann, Richard} (11.\,7.\,1866 Wien – 26.\,9.\,1945 New York City), \emph{Schriftsteller}!Pierrot Hypnotiseur@\strich\emph{Pierrot Hypnotiseur}|pwk}, Pantomime von Richard Beer-Hofmann\pwindex{Beer-Hofmann, Richard 11.\,7.\,1866 Wien – 26.\,9.\,1945 New York City@\textsc{Beer-Hofmann, Richard} (11.\,7.\,1866 Wien – 26.\,9.\,1945 New York City), \emph{Schriftsteller}|pwk}, blieb zu Lebzeiten
                  ungedruckt.}}}\label{K_L00115-1}{\pb}\strikeout{u} schicken werde.\pend
           
\pstart
           Ihr »Märchen\pwindex{Schnitzler, Arthur 15.\,5.\,1862 Wien – 21.\,10.\,1931 ebd.@\textsc{Schnitzler, Arthur} (15.\,5.\,1862 Wien – 21.\,10.\,1931 ebd.), \emph{Schriftsteller, Mediziner}!Märchen. Schauspiel in drei Aufzügen@\strich\emph{Das Märchen. Schauspiel in drei Aufzügen}|pw}« und Ihre »Episode\pwindex{Schnitzler, Arthur 15.\,5.\,1862 Wien – 21.\,10.\,1931 ebd.@\textsc{Schnitzler, Arthur} (15.\,5.\,1862 Wien – 21.\,10.\,1931 ebd.), \emph{Schriftsteller, Mediziner}!Episode@\strich\emph{Episode}|pw}« habe ich bereits mehrfach verborgt; könnten Sie mir noch
               vor Ihrer Ankunft – denn die sich dafür Interessirenden reisen bald ab –\pend
           
\pstart
           {\pb}»Anatols Hochzeitsmorgen\pwindex{Schnitzler, Arthur 15.\,5.\,1862 Wien – 21.\,10.\,1931 ebd.@\textsc{Schnitzler, Arthur} (15.\,5.\,1862 Wien – 21.\,10.\,1931 ebd.), \emph{Schriftsteller, Mediziner}!Anatols Hochzeitsmorgen@\strich\emph{Anatols Hochzeitsmorgen}|pw}«\pend
           
\pstart
           »Abschiedsouper\pwindex{Schnitzler, Arthur 15.\,5.\,1862 Wien – 21.\,10.\,1931 ebd.@\textsc{Schnitzler, Arthur} (15.\,5.\,1862 Wien – 21.\,10.\,1931 ebd.), \emph{Schriftsteller, Mediziner}!Abschiedssouper@\strich\emph{Abschiedssouper}|pw}«\pend
           
\pstart
           »Frage an das Schicksal\pwindex{Schnitzler, Arthur 15.\,5.\,1862 Wien – 21.\,10.\,1931 ebd.@\textsc{Schnitzler, Arthur} (15.\,5.\,1862 Wien – 21.\,10.\,1931 ebd.), \emph{Schriftsteller, Mediziner}!Frage an das Schicksal@\strich\emph{Die Frage an das Schicksal}|pw}«\pend
           
\pstart
           senden?\pend
           
\pstart
           Frau Flegmann\pwindex{Flegmann, Bertha 27.\,5.\,1852 Dubrovsky, Polen – 24.\,6.\,1933 Bad Ischl@\textsc{Flegmann, Bertha} (27.\,5.\,1852 Dubrovsky, Polen – 24.\,6.\,1933 Bad Ischl), \emph{Salonnière}|pw}, die wie Sie wissen ein klein
               wenig litterarischen Salon treibt interessirt sich dafür; {\pb}ich würde die Sachen fall\substVorne{}\textsuperscript{ls}\substDazwischen{}s\substHinten{} es nur \uline{Abschriften} sind nicht verborgen,
               sondern vorlesen. »\label{K_L00115-2v}\edtext{\uline{Das} Gedicht\pwindex{Schnitzler, Arthur 15.\,5.\,1862 Wien – 21.\,10.\,1931 ebd.@\textsc{Schnitzler, Arthur} (15.\,5.\,1862 Wien – 21.\,10.\,1931 ebd.), \emph{Schriftsteller, Mediziner}!Anfang vom Ende@\strich\emph{Anfang vom Ende}|pwv}}{\lemma{\textnormal{\emph{Das Gedicht}}}\Cendnote{\textnormal{Arthur Schnitzler: \emph{Anfang vom Ende}\pwindex{Schnitzler, Arthur 15.\,5.\,1862 Wien – 21.\,10.\,1931 ebd.@\textsc{Schnitzler, Arthur} (15.\,5.\,1862 Wien – 21.\,10.\,1931 ebd.), \emph{Schriftsteller, Mediziner}!Anfang vom Ende@\strich\emph{Anfang vom Ende}|pwk}. In: \emph{Deutsche Dichtung}\pwindex{Deutsche Dichtung@\emph{Deutsche Dichtung}|pwk}, Bd. 12, Nr. 8, 15. 7. 1892,
                     S. 192.}}}\label{K_L00115-2}« ist wie ich vom Kleinen Kraus\pwindex{Kraus, Karl 28.\,4.\,1874 Jičín – 12.\,6.\,1936 Wien@\textsc{Kraus, Karl} (28.\,4.\,1874 Jičín – 12.\,6.\,1936 Wien), \emph{Schriftsteller, Publizist, Schriftsteller}|pw} (vide Salten\pwindex{Salten, Felix 6.\,9.\,1869 Budapest – 8.\,10.\,1945 Zürich@\textsc{Salten, Felix} (6.\,9.\,1869 Budapest – 8.\,10.\,1945 Zürich), \emph{Schriftsteller, Journalist, Chefredakteur}|pw}) höre
               in der »Deutschen Dichtung\pwindex{Deutsche Dichtung@\emph{Deutsche Dichtung}|pw}« erschienen. Loris\pwindex{Hofmannsthal, Hugo von 1.\,2.\,1874 Wien – 15.\,7.\,1929 Rodaun@\textsc{Hofmannsthal, Hugo von} (1.\,2.\,1874 Wien – 15.\,7.\,1929 Rodaun), \emph{Schriftsteller}|pw}, der {\pb}wie es scheint gesellschaftlich
               zerrissen wird ist öfters hier, bei mir.\pend
           
\pstart
           Bitte schreiben Sie mir wieder ein paar Zeilen, – und vor allem annonciren Sie Ihr
                  Ko{\geminationm}en. Bitte was macht Schwarzkopf\pwindex{Schwarzkopf, Gustav 7.\,11.\,1853 Wien – 13.\,11.\,1939 ebd.@\textsc{Schwarzkopf, Gustav} (7.\,11.\,1853 Wien – 13.\,11.\,1939 ebd.), \emph{Schriftsteller}|pw}, ich hörte traurige Nachrichten? Herzlichst Ihr\pend
           \pstart \spacefill\mbox{Richard}\pend{}
\pstart
           Ischl\oindex{Bad Ischl@\textbf{Bad Ischl}|pw}{ }19 Aug. 92\pend
           \selectlanguage{ngerman}\endnumbering\briefempfaengerindex{Schnitzler, Arthur@\textsc{Schnitzler, Arthur}!zzzBeer-Hofmann, Richard@\emph{von Richard Beer-Hofmann}!1892-08-191@{19. 8. 1892}|)be}\mylabel{L00115h}  \newcommand{\dateiname}{L00115}\newcommand{\titel}{Richard Beer-Hofmann an Arthur Schnitzler, 19. 8. 1892}\newcommand{\editorInnen}{Martin Anton Müller und Gerd-Hermann Susen}%% latex-leseansicht-abspann.tex
%% Abspann für die Leseansicht.
%% Der Schalter \ifkorrekturansicht ist bereits durch den Vorspann gesetzt.

%% latex-abspann.tex
%% Gemeinsamer Abspann für Korrekturansicht und Leseansicht.
%% Setzt den Schalter \ifkorrekturansicht voraus (gesetzt in den
%% einbindenden Dateien latex-korrekturansicht-abspann.tex bzw.
%% latex-leseansicht-abspann.tex).
%% ---------------------------------------------------------------

\normalsize

% Das esempio-Environment wird nur in der Leseansicht benötigt
\ifkorrekturansicht\else
\newenvironment{esempio}[3]%
{
    \vspace{1.5ex}
    \rlap{\underline{#1}}
    \par
    \setlength{\parindent}{0cm}
    \nopagebreak
    \leftskip=#2cm
    \rightskip=#3cm
}
{
    \par
}
\fi

\doendnotes{C}
\bigskip
\vfill

\clearpage

\footnotesize

\ifkorrekturansicht
  \lohead{\textsc{register}}
\fi

% theindex-Environment neu definieren ohne reledmac
\makeatletter
\renewenvironment{theindex}{%
  \ifkorrekturansicht
    \section*{\indexname}%
  \else
    \subsubsection*{Index der erwähnten Entitäten}%
  \fi
  \setlength{\parindent}{0pt}%
  \setlength{\parskip}{0pt plus 0.3pt}%
  \let\item\@idxitem
}{%
  \ifkorrekturansicht\clearpage\fi
}
\makeatother

\IfFileExists{\jobname-pw.ind}{\input{\jobname-pw.ind}}{}

% Quellenangabe nur in der Leseansicht
\ifkorrekturansicht\else
% Fallback-Definitionen, falls die .tex-Datei \titel etc. nicht gesetzt hat
\providecommand{\titel}{}
\providecommand{\editorInnen}{}
\providecommand{\dateiname}{\jobname}

\vspace{3cm}

\vfill

\footnotesize
\textsc{Quelle}: \titel. Herausgegeben von {\editorInnen}. In: \emph{Arthur Schnitzler: Briefwechsel mit Autorinnen und Autoren}.
 Digitale Edition, https://schnitzler-briefe.acdh.oeaw.ac.at/{\dateiname}.html (Stand \today)
\fi

\end{document}


