%% latex-korrekturansicht-vorspann.tex
%% Vorspann für die Korrekturansicht.
%% Lädt die gemeinsame Datei latex-vorspann.tex mit gesetztem Schalter.

\newif\ifkorrekturansicht
\korrekturansichttrue

\input{../tex-inputs/latex-vorspann}


\section[Richard Beer-Hofmann an Arthur Schnitzler, 19. 8. 1892]{L00115 Richard Beer-Hofmann an Arthur Schnitzler, 19. 8. 1892}
\nopagebreak\mylabel{L00115v}
\rehead{ }\normalsize\beginnumbering\briefempfaengerindex{Schnitzler, Arthur@\textsc{Schnitzler, Arthur}!zzzBeer-Hofmann, Richard@\emph{von Richard Beer-Hofmann}!1892-08-191@{19. 8. 1892}|(be}
\toendnotes[C]{\smallbreak\pagebreak[2]}\Standort{CUL, Schnitzler, B 8.}
\physDesc{Brief, 2 Blätter, 8 Seiten, 1497 Zeichen
\newline{}Handschrift: Bleistift, lateinische Kurrent
\newline{}Schnitzler: mit Bleistift datiert: »19/8 92« und nummeriert: »9.« }
\buchAbdrucke{\weitereDrucke{Arthur Schnitzler, Richard Beer-Hofmann: \emph{Briefwechsel 1891–1931}. Wien, Zürich: \emph{Europaverlag} 1992, S. 36–37.} }\toendnotes[C]{\smallbreak}
\pstart
           \noindent{}{\pb}Lieber Arthur! Sie wissen ja, wie schreibfaul ich bin, und wie sehr
               ich mir immer Zeit lasse.\pend
           
\pstart
           Also vor Allem: Ich freue mich sehr, \uline{sehr} sie auf ein
               paar Tage hier zu haben; mit Ihnen {\pb}werde ich freilich kaum gehen können; im Allgemeinen habe ich einen verdorbenen
                  So{\geminationm}er, schlechte Laune in xter Potenz, die erst jetzt
               etwas, nachlässt; gearbeitet {\pb}hab
               ich circa \uuline{15} (!) Druckzeilen – also – nichts. Ausser
               ein paar Gedanken, deren Wert äußerst p\substVorne{}\textsuperscript{o}\substDazwischen{}ro\substHinten{}blematisch ist, also ein verlorener So{\geminationm}er. In
               den nächsten {\pb}Tagen werde ich
               voraussichtlich meine Pantomime\pwindex{Pierrot Hypnotiseur@\emph{Pierrot Hypnotiseur}|pwv} an Sie senden, und Sie bitten \strikeout{Sie},
               dieselbe durch Ihren Abschreiber\pwindex{?? [Schreibkraft fuer Arthur Schnitzler] @\textsc{?? [Schreibkraft für Arthur Schnitzler]}|pwv} copiren zu lassen, da ich sie möglicherweise in der nächsten
               Zeit an irgend einen \label{K_L00115-1v}\edtext{Verleger}{\lemma{\textnormal{\emph{Verleger}}}\Cendnote{\textnormal{\emph{Pierrot hypnotiseur}\pwindex{Pierrot Hypnotiseur@\emph{Pierrot Hypnotiseur}|pwk}, Pantomime von Richard Beer-Hofmann\pwindex{Beer-Hofmann, Richard 1866-07-11 – 1945-09-26@\textsc{Beer-Hofmann, Richard} (1866-07-11 – 1945-09-26), \emph{Schriftsteller/Schriftstellerin}|pwk}, blieb zu Lebzeiten
                  ungedruckt.}}}\label{K_L00115-1}{\pb}\strikeout{u} schicken werde.\pend
           
\pstart
           Ihr »Märchen\pwindex{Maerchen. Schauspiel in drei Aufzuegen@\emph{Das Märchen. Schauspiel in drei Aufzügen}|pw}« und Ihre »Episode\pwindex{Episode@\emph{Episode}|pw}« habe ich bereits mehrfach verborgt; könnten Sie mir noch
               vor Ihrer Ankunft – denn die sich dafür Interessirenden reisen bald ab – \pend
           
\pstart
           {\pb}»Anatols Hochzeitsmorgen\pwindex{Anatols Hochzeitsmorgen@\emph{Anatols Hochzeitsmorgen}|pw}«\pend
           
\pstart
           »Abschiedsouper\pwindex{Abschiedssouper@\emph{Abschiedssouper}|pw}«\pend
           
\pstart
           »Frage an das Schicksal\pwindex{Frage an das Schicksal@\emph{Die Frage an das Schicksal}|pw}«\pend
           
\pstart
           senden?\pend
           
\pstart
           Frau Flegmann\pwindex{Flegmann, Bertha 27.05.1852 – 24.6.1933@\textsc{Flegmann, Bertha} (27.05.1852 – 24.6.1933), \emph{männliche Salonnière/Salonnière}|pw}, die wie Sie wissen ein klein
               wenig litterarischen Salon treibt interessirt sich dafür; {\pb}ich würde die Sachen fall\substVorne{}\textsuperscript{ls}\substDazwischen{}s\substHinten{} es nur \uline{Abschriften} sind nicht verborgen,
               sondern vorlesen. »\label{K_L00115-2v}\edtext{\uline{Das} Gedicht\pwindex{Anfang vom Ende@\emph{Anfang vom Ende}|pwv}}{\lemma{\textnormal{\emph{Das Gedicht}}}\Cendnote{\textnormal{Arthur Schnitzler: \emph{Anfang vom Ende}\pwindex{Anfang vom Ende@\emph{Anfang vom Ende}|pwk}. In: \emph{Deutsche Dichtung}\pwindex{Deutsche Dichtung@\emph{Deutsche Dichtung}|pwk}, Bd. 12, Nr. 8, 15. 7. 1892,
                     S. 192.}}}\label{K_L00115-2}« ist wie ich vom Kleinen Kraus\pwindex{Kraus, Karl 28.04.1874 – 12.06.1936@\textsc{Kraus, Karl} (28.04.1874 – 12.06.1936), \emph{Schriftsteller/Schriftstellerin, Publizist/Publizistin, Schriftsteller/Schriftstellerin}|pw} (vide Salten\pwindex{Salten, Felix 06.09.1869 – 08.10.1945@\textsc{Salten, Felix} (06.09.1869 – 08.10.1945), \emph{Schriftsteller/Schriftstellerin, Journalist/Journalistin, Chefredakteur/Chefredakteurin}|pw}) höre
               in der »Deutschen Dichtung\pwindex{Deutsche Dichtung@\emph{Deutsche Dichtung}|pw}« erschienen. Loris\pwindex{Hofmannsthal, Hugo von 1874-02-01 – 1929-07-15@\textsc{Hofmannsthal, Hugo von} (1874-02-01 – 1929-07-15), \emph{Schriftsteller/Schriftstellerin}|pw}, der {\pb}wie es scheint gesellschaftlich
               zerrissen wird ist öfters hier, bei mir.\pend
           
\pstart
           Bitte schreiben Sie mir wieder ein paar Zeilen, – und vor allem annonciren Sie Ihr
                  Ko{\geminationm}en. Bitte was macht Schwarzkopf\pwindex{Schwarzkopf, Gustav 07.11.1853 – 13.11.1939@\textsc{Schwarzkopf, Gustav} (07.11.1853 – 13.11.1939), \emph{Schriftsteller/Schriftstellerin}|pw}, ich hörte traurige Nachrichten? Herzlichst Ihr\pend
           \pstart \spacefill\mbox{Richard}\pend{}
\pstart
           Ischl\oindex{Bad Ischl@\textbf{Bad Ischl}, \emph{P.PPL}|pw}{ }19 Aug. 92\pend
           \selectlanguage{ngerman}\endnumbering\briefempfaengerindex{Schnitzler, Arthur@\textsc{Schnitzler, Arthur}!zzzBeer-Hofmann, Richard@\emph{von Richard Beer-Hofmann}!1892-08-191@{19. 8. 1892}|)be}\mylabel{L00115h}  \normalsize

\doendnotes{C}
\bigskip
\vfill

\clearpage

\footnotesize

\lohead{\textsc{register}}

% Definiere theindex-Environment komplett neu ohne reledmac
\makeatletter
\renewenvironment{theindex}{%
  \section*{\indexname}%
  \setlength{\parindent}{0pt}%
  \setlength{\parskip}{0pt plus 0.3pt}%
  \let\item\@idxitem
}{%
  \clearpage
}
\makeatother

\IfFileExists{\jobname-pw.ind}{\input{\jobname-pw.ind}}{}

\end{document}

      