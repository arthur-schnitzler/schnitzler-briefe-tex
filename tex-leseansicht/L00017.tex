%% latex-korrekturansicht-vorspann.tex
%% Vorspann für die Korrekturansicht.
%% Lädt die gemeinsame Datei latex-vorspann.tex mit gesetztem Schalter.

\newif\ifkorrekturansicht
\korrekturansichttrue

\input{../tex-inputs/latex-vorspann}


\section[Arthur Schnitzler an Richard Beer-Hofmann, {[}Juni 1891 – 1900?{]}]{L00017 Arthur Schnitzler an Richard Beer-Hofmann, {[}Juni 1891 – 1900?{]}}
\nopagebreak\mylabel{L00017v}
\rehead{ }\normalsize\beginnumbering\briefempfaengerindex{Beer-Hofmann, Richard@\textsc{Beer-Hofmann, Richard}!zzzSchnitzler, Arthur@\emph{von Arthur Schnitzler}!1900-12-311@{{[}Juni 1891 – 1900?{]}}|(be}
\toendnotes[C]{\smallbreak\pagebreak[2]}\Standort{YCGL, MSS 31.}
\physDesc{Briefkarte, 110 Zeichen
\newline{}Handschrift: Bleistift, deutsche Kurrent}\toendnotes[C]{\smallbreak}
\pstart
           \noindent{}{\pb}Lieber Richard, Frl \label{K_L00017-1v}\edtext{\textsc{Russell}}{\lemma{\textnormal{\emph{Russell}}}\Cendnote{\textnormal{nicht ermittelt. Das
                     Korrespondenzstück ist undatiert, muss aber aus der Zeit stammen, als
                     regelmäßige Treffen im Kaffeehaus stattfanden.}}}\label{K_L00017-1}\pwindex{Russell, Lillian 1860-12-04 – 1922-06-05@\textsc{Russell, Lillian} (1860-12-04 – 1922-06-05), \emph{Schauspieler/Schauspielerin, Filmschauspieler/Filmschauspielerin}|pw} ſagt \uline{nein}, wegen Zeitmangel.\pend
           
\pstart
           Seh ich Sie heut im Café? – Ich hoffe. Herzlichſt\pend
           \pstart Ihr \spacefill\mbox{Arthur}\pend{}\selectlanguage{ngerman}\endnumbering\briefempfaengerindex{Beer-Hofmann, Richard@\textsc{Beer-Hofmann, Richard}!zzzSchnitzler, Arthur@\emph{von Arthur Schnitzler}!1891-06-011@{{[}Juni 1891 – 1900?{]}}|)be}\mylabel{L00017h}  \normalsize

\doendnotes{C}
\bigskip
\vfill

\clearpage

\footnotesize

\lohead{\textsc{register}}

% Definiere theindex-Environment komplett neu ohne reledmac
\makeatletter
\renewenvironment{theindex}{%
  \section*{\indexname}%
  \setlength{\parindent}{0pt}%
  \setlength{\parskip}{0pt plus 0.3pt}%
  \let\item\@idxitem
}{%
  \clearpage
}
\makeatother

\IfFileExists{\jobname-pw.ind}{\input{\jobname-pw.ind}}{}

\end{document}

      