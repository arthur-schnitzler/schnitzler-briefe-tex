%% latex-leseansicht-vorspann.tex
%% Vorspann für die Leseansicht.
%% Lädt die gemeinsame Datei latex-vorspann.tex mit nicht gesetztem Schalter.

\newif\ifkorrekturansicht
\korrekturansichtfalse

\input{../tex-inputs/latex-vorspann}

\begin{center}
            \textcolor{red}{ENTWURF. ENTZIFFERUNG NOCH NICHT KORREKTURGELESEN}
                      \end{center}
            
               \section[Gerhard Wiese an Arthur Schnitzler, 21. 3. 1894]{ Gerhard Wiese an Arthur Schnitzler, 21. 3. 1894}\nopagebreak\mylabel{v}\rehead{ }\begin{ledgroupsized}[t]{13cm}\normalsize\beginnumbering\briefempfaengerindex{Schnitzler, Arthur@\textsc{Schnitzler, Arthur}!zzzWiese, Gerhard@\emph{von Gerhard Wiese}!1894-03-212@{21. 3. 1894}|(be} \toendnotes[C]{\smallbreak\pagebreak[2]} \Standort{CUL, Schnitzler, B 15.}
\physDesc{Brief, 1 Blatt, 2 Seiten
\newline{}Handschrift: schwarze Tinte, deutsche Kurrent
\newline{}Schnitzler: 1) mit Bleistift auf der Rückseite beschriftet: »\textsc{(Blumenthal\pwindex{Blumenthal, Oskar 13.03.1852 – 24.04.1917@\textsc{Blumenthal, Oskar} (13.03.1852 – 24.04.1917), \emph{Schriftsteller, Journalist, Theaterleiter}|pw})}« 2) mit rotem Buntstift nummeriert: »6«\newline{}Ordnung: mit Bleistift von unbekannter Hand nummeriert:
                                                »6« }\toendnotes[C]{\smallbreak}\pstart
           \noindent{}\centering{}{\pb}\textcolor{gray}{\textbf{LESSING-THEATER\orgindex{Lessing-Theater@Lessing-Theater|pw}}}\pend
           \pstart
           \noindent{}\centering{}\textcolor{gray}{\textbf{Director:}}{\\}\textcolor{gray}{\textbf{Dr. Oscar Blumenthal.}}\pend
           \pstart
           \noindent{}\raggedleft{}\textcolor{gray}{\textbf{Berlin N.W.\oindex{Berlin@\textbf{Berlin}|pw}, den}}{ }21. März \textcolor{gray}{\textbf{189}}4.{\\}\textcolor{gray}{\textbf{Friedrich-Carl-Ufer\oindex{Kapelle-Ufer@\textbf{Kapelle-Ufer}|pw}}}.\pend
           \pstart\center{}Sehr geehrter Herr!\pend\pstart
           Wie Sie aus beiliegendem Wochenſpielplan erſehen, iſt die Frage, welcher Einakter\pwindex{\textcolor{red}{\textsuperscript{XXXX1 indx}}!Eisenfresser25.3.1894 – 25.3.1894@\strich\emph{Der Eisenfresser} {[}25.3.1894 – 25.3.1894{]}|pwv}\pwindex{\textcolor{red}{\textsuperscript{XXXX1 indx}}!Eisenfresser25.3.1894 – 25.3.1894@\strich\emph{Der Eisenfresser} {[}25.3.1894 – 25.3.1894{]}|pwv} nach »Niobe\pwindex{Blumenthal, Oskar 13.03.1852 – 24.04.1917@\textsc{Blumenthal, Oskar} (13.03.1852 – 24.04.1917), \emph{Schriftsteller, Journalist, Theaterleiter}!Niobe1894@\strich\emph{Niobe} {[}Übersetzung, 1894{]}|pw}\pwindex{\textcolor{red}{\textsuperscript{XXXX1 indx}}!Niobe1894@\strich\emph{Niobe} {[}1894{]}|pw}« gegeben werden ſoll, bereits
                    entſchieden. Herr \textsc{Dr. Oscar
                            Blumenthal\pwindex{Blumenthal, Oskar 13.03.1852 – 24.04.1917@\textsc{Blumenthal, Oskar} (13.03.1852 – 24.04.1917), \emph{Schriftsteller, Journalist, Theaterleiter}|pw}} weilt zur Zeit in \textsc{Moscau\oindex{Moskau@\textbf{Moskau}|pw}} und kehrt vorausſichtlich erſt Ende April nach Berlin\oindex{Berlin@\textbf{Berlin}|pw} zurück. Wir ſtellen Ihnen ergebenſt anheim, alsdann
                    auf den Inhalt Ihres jüngſten Schreibens zurückzukommen.\pend
           \pstart
           Hochachtungsvoll\pend
           \pstart
           \raggedleft{}\textcolor{gray}{\textbf{\textit{Die Direction}}}{\\}\textcolor{gray}{\textbf{\textit{des}}}{\\}\textcolor{gray}{\textbf{\textit{Lessing-Theaters\orgindex{Lessing-Theater@Lessing-Theater|pw}.}}}\pend
           \pstart \spacefill\mbox{Wieſe}\pend{}\endnumbering\briefempfaengerindex{Schnitzler, Arthur@\textsc{Schnitzler, Arthur}!zzzWiese, Gerhard@\emph{von Gerhard Wiese}!1894-03-212@{21. 3. 1894}|)be}\mylabel{h}\end{ledgroupsized}  \newcommand{\dateiname}{L00307}\newcommand{\titel}{Gerhard Wiese an Arthur Schnitzler, 21. 3. 1894}\newcommand{\editorInnen}{Martin Anton Müller und Gerd-Hermann Susen}%% latex-leseansicht-abspann.tex
%% Abspann für die Leseansicht.
%% Der Schalter \ifkorrekturansicht ist bereits durch den Vorspann gesetzt.

%% latex-abspann.tex
%% Gemeinsamer Abspann für Korrekturansicht und Leseansicht.
%% Setzt den Schalter \ifkorrekturansicht voraus (gesetzt in den
%% einbindenden Dateien latex-korrekturansicht-abspann.tex bzw.
%% latex-leseansicht-abspann.tex).
%% ---------------------------------------------------------------

\normalsize

% Das esempio-Environment wird nur in der Leseansicht benötigt
\ifkorrekturansicht\else
\newenvironment{esempio}[3]%
{
    \vspace{1.5ex}
    \rlap{\underline{#1}}
    \par
    \setlength{\parindent}{0cm}
    \nopagebreak
    \leftskip=#2cm
    \rightskip=#3cm
}
{
    \par
}
\fi

\doendnotes{C}
\bigskip
\vfill

\clearpage

\footnotesize

\ifkorrekturansicht
  \lohead{\textsc{register}}
\fi

% theindex-Environment neu definieren ohne reledmac
\makeatletter
\renewenvironment{theindex}{%
  \ifkorrekturansicht
    \section*{\indexname}%
  \else
    \subsubsection*{Index der erwähnten Entitäten}%
  \fi
  \setlength{\parindent}{0pt}%
  \setlength{\parskip}{0pt plus 0.3pt}%
  \let\item\@idxitem
}{%
  \ifkorrekturansicht\clearpage\fi
}
\makeatother

\IfFileExists{\jobname-pw.ind}{\input{\jobname-pw.ind}}{}

% Quellenangabe nur in der Leseansicht
\ifkorrekturansicht\else
% Fallback-Definitionen, falls die .tex-Datei \titel etc. nicht gesetzt hat
\providecommand{\titel}{}
\providecommand{\editorInnen}{}
\providecommand{\dateiname}{\jobname}

\vspace{3cm}

\vfill

\footnotesize
\textsc{Quelle}: \titel. Herausgegeben von {\editorInnen}. In: \emph{Arthur Schnitzler: Briefwechsel mit Autorinnen und Autoren}.
 Digitale Edition, https://schnitzler-briefe.acdh.oeaw.ac.at/{\dateiname}.html (Stand \today)
\fi

\end{document}


      