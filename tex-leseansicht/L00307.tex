%% latex-korrekturansicht-vorspann.tex
%% Vorspann für die Korrekturansicht.
%% Lädt die gemeinsame Datei latex-vorspann.tex mit gesetztem Schalter.

\newif\ifkorrekturansicht
\korrekturansichttrue

\input{../tex-inputs/latex-vorspann}


\section[Gerhard Wiese an Arthur Schnitzler, 21. 3. 1894]{L00307 Gerhard Wiese an Arthur Schnitzler, 21. 3. 1894}
\nopagebreak\mylabel{L00307v}
\rehead{ }\normalsize\beginnumbering\briefempfaengerindex{Schnitzler, Arthur@\textsc{Schnitzler, Arthur}!zzzWiese, Gerhard@\emph{von Gerhard Wiese}!1894-03-212@{21. 3. 1894}|(be}
\toendnotes[C]{\smallbreak\pagebreak[2]}\Standort{CUL, Schnitzler, B 15.}
\physDesc{Brief, 1 Blatt, 2 Seiten, 426 Zeichen
\newline{}Handschrift: schwarze Tinte, deutsche Kurrent
\newline{}Schnitzler: 1) mit Bleistift auf der Rückseite beschriftet: »\textsc{(Blumenthal\pwindex{Blumenthal, Oskar 13.03.1852 – 24.04.1917@\textsc{Blumenthal, Oskar} (13.03.1852 – 24.04.1917), \emph{Schriftsteller/Schriftstellerin, Journalist/Journalistin, Theaterleiter/Theaterleiterin}|pw})}«  2) mit rotem Buntstift nummeriert: »6«
\newline{}Ordnung: mit Bleistift von unbekannter Hand nummeriert:
                                 »6« }\toendnotes[C]{\smallbreak}
\pstart
           \centering{}{\pb}\textcolor{gray}{\textbf{LESSING-THEATER\orgindex{Lessing-Theater@Lessing-Theater|pw}}}\pend
           
\pstart
           \centering{}\textcolor{gray}{\textbf{Director:}}{\\}\textcolor{gray}{\textbf{Dr. Oscar Blumenthal.}}\pend
           
\pstart
           \raggedleft{}\textcolor{gray}{\textbf{Berlin N.W.\oindex{Berlin@\textbf{Berlin}, \emph{P.PPLC}|pw}, den}}{ }21. März \textcolor{gray}{\textbf{189}}4.{\\}\textcolor{gray}{\textbf{Friedrich-Carl-Ufer\oindex{Kapelle-Ufer@\textbf{Kapelle-Ufer}, \emph{Straße (K.STR)}|pw}}}.\pend
           
\pstart\center{}Sehr geehrter Herr!\pend\vspace{0.5em}
\pstart
           Wie Sie aus beiliegendem Wochenſpielplan erſehen, iſt die Frage, welcher Einakter\pwindex{Eisenfresser@\emph{Der Eisenfresser}|pwv} nach »Niobe\pwindex{Niobe@\emph{Niobe}|pw}« gegeben werden ſoll, bereits entſchieden.
               Herr \textsc{Dr. Oscar Blumenthal\pwindex{Blumenthal, Oskar 13.03.1852 – 24.04.1917@\textsc{Blumenthal, Oskar} (13.03.1852 – 24.04.1917), \emph{Schriftsteller/Schriftstellerin, Journalist/Journalistin, Theaterleiter/Theaterleiterin}|pw}} weilt zur Zeit in \textsc{Moscau\oindex{Moskau@\textbf{Moskau}, \emph{A.ADM1}|pw}} und kehrt vorausſichtlich erſt Ende April nach Berlin\oindex{Berlin@\textbf{Berlin}, \emph{P.PPLC}|pw} zurück. Wir ſtellen Ihnen ergebenſt anheim, alsdann auf
               den Inhalt Ihres jüngſten Schreibens zurückzukommen.\pend
           
\pstart
           Hochachtungsvoll\pend
           
\pstart
           \raggedleft{}\textcolor{gray}{\textbf{\textit{Die Direction}}}{\\}\textcolor{gray}{\textbf{\textit{des}}}{\\}\textcolor{gray}{\textbf{\textit{Lessing-Theaters\orgindex{Lessing-Theater@Lessing-Theater|pw}.}}}\pend
           \pstart \spacefill\mbox{Wieſe}\pend{}\selectlanguage{ngerman}\endnumbering\briefempfaengerindex{Schnitzler, Arthur@\textsc{Schnitzler, Arthur}!zzzWiese, Gerhard@\emph{von Gerhard Wiese}!1894-03-212@{21. 3. 1894}|)be}\mylabel{L00307h}  \normalsize

\doendnotes{C}
\bigskip
\vfill

\clearpage

\footnotesize

\lohead{\textsc{register}}

% Definiere theindex-Environment komplett neu ohne reledmac
\makeatletter
\renewenvironment{theindex}{%
  \section*{\indexname}%
  \setlength{\parindent}{0pt}%
  \setlength{\parskip}{0pt plus 0.3pt}%
  \let\item\@idxitem
}{%
  \clearpage
}
\makeatother

\IfFileExists{\jobname-pw.ind}{\input{\jobname-pw.ind}}{}

\end{document}

      