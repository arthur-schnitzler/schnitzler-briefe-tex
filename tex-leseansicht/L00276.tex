%% latex-leseansicht-vorspann.tex
%% Vorspann für die Leseansicht.
%% Lädt die gemeinsame Datei latex-vorspann.tex mit nicht gesetztem Schalter.

\newif\ifkorrekturansicht
\korrekturansichtfalse

\input{../tex-inputs/latex-vorspann}


               \section[Hermann Bahr an Arthur Schnitzler, 25. 10. 1893]{ Hermann Bahr an Arthur Schnitzler, 25. 10. 1893}\nopagebreak\mylabel{v}\rehead{ }\begin{ledgroupsized}[t]{13cm}\normalsize\beginnumbering\briefempfaengerindex{Schnitzler, Arthur@\textsc{Schnitzler, Arthur}!zzzBahr, Hermann@\emph{von Hermann Bahr}!1893-10-251@{25. 10. 1893}|(be} \toendnotes[C]{\smallbreak\pagebreak[2]} \Standort{CUL, Schnitzler, B 5b.}
\physDesc{Brief, 1 Blatt, 2 Seiten
\newline{}Handschrift  : schwarze Tinte, deutsche Kurrent\newline{}Handschrift Hermann Bahr: schwarze Tinte, deutsche Kurrent (\noindent{}Unterschrift)\newline{}Ordnung: 1) mit rotem Buntstift von unbekannter Hand nummeriert:
                                    »15« 2) mit Bleistift von unbekannter Hand nummeriert:
                                    »15«}\buchAbdrucke{\weitereDrucke{Hermann Bahr, Arthur Schnitzler: \emph{Briefwechsel, Aufzeichnungen, Dokumente (1891–1931)}. Hg. Kurt Ifkovits und Martin Anton Müller. Göttingen: \emph{Wallstein} 2018, S. 45.} }\toendnotes[C]{\smallbreak}\pstart
           \noindent{}{\pb}\textcolor{gray}{\textbf{Deutſche Zeitung\orgindex{Deutsche Zeitung@Deutsche Zeitung|pw}}}\hfill \uline{Wien\oindex{Wien@\textbf{Wien}|pw}}, 25. Octbr. 1893\pend
           \pstart
           \textcolor{gray}{\textbf{Wien\oindex{Wien@\textbf{Wien}|pw}}}\hfill III. Saleſianerg. 12\oindex{Salesianergasse@\textbf{Salesianergasse}|pw}\pend
           \pstart
           \textcolor{gray}{\textbf{IX., Pelikangaſſe 4\oindex{Pelikangasse@\textbf{Pelikangasse}|pw}.}}\pend
           \pstart{}Verehrter Freund!\pend\pstart
           Der Mann um den es ſich handelt heißt Johann Lukas \textsc{Schönlein}\pwindex{Schoenlein, Johann Lukas 30.11.1793 – 23.01.1864@\textsc{Schönlein, Johann Lukas} (30.11.1793 – 23.01.1864), \emph{Mediziner}|pw}. Er iſt der Begründer der ſog. naturhyſteriſchen Schule in der Therapie. Am
                  30. November{ }ſind es hundert Jahre, daß er geboren wurde und ich
               brauche alſo für dieſen Tag ein nicht über ſechs Spalten langes, populäres, \label{K_L00276_1v}\edtext{byografiſches Feuilleton}{\lemma{\textnormal{\emph{byografiſches Feuilleton}}}\Cendnote{\textnormal{nicht erschienen}}}\label{K_L00276_1h}. Können Sie mir das
               verſchaffen?\pend
           \pstart
           Dabei wiederhole ich die bereits \label{K_L00276_2v}\edtext{neulich}{\lemma{\textnormal{\emph{neulich}}}\Cendnote{\textnormal{vermutlich beim Besuch Hofmannsthals\pwindex{Hofmannsthal, Hugo von 01.02.1874 – 15.07.1929@\textsc{Hofmannsthal, Hugo von} (01.02.1874 – 15.07.1929), \emph{Schriftsteller}|pwk} am 22. 10. 1893}}}\label{K_L00276_2h} durch \textsc{Loris}\pwindex{Hofmannsthal, Hugo von 01.02.1874 – 15.07.1929@\textsc{Hofmannsthal, Hugo von} (01.02.1874 – 15.07.1929), \emph{Schriftsteller}|pw} vermittelte Bitte um irgend eine Novellette, ſo kurz als möglich, die ich am
                  \label{K_L00276_3v}\edtext{Tage Ihrer Premiere}{\lemma{\textnormal{\emph{Tage Ihrer Premiere}}}\Cendnote{\textnormal{Am 1. 12. 1893 Premiere von \emph{Das Märchen}\pwindex{Schnitzler, Arthur 15.05.1862 – 21.10.1931@\textsc{Schnitzler, Arthur} (15.05.1862 – 21.10.1931), \emph{Schriftsteller, Mediziner}!Maerchen. Schauspiel in drei Aufzuegen1891 – 1891@\strich\emph{Das Märchen. Schauspiel in drei Aufzügen} {[}1891 – 1891{]}|pwk}; an diesem Tag erschien nichts von
                     Schnitzler\pwindex{Schnitzler, Arthur 15.05.1862 – 21.10.1931@\textsc{Schnitzler, Arthur} (15.05.1862 – 21.10.1931), \emph{Schriftsteller, Mediziner}|pwk}.}}}\label{K_L00276_3h} bringen will.\pend
           \pstart
           {\pb}Kann ich bis längſtens Ende der nächſten Woche auf
               den \label{K_L00276_4v}\edtext{erſten}{\lemma{\textnormal{\emph{erſten}}}\Cendnote{\textnormal{Arthur Schnitzler\pwindex{Schnitzler, Arthur 15.05.1862 – 21.10.1931@\textsc{Schnitzler, Arthur} (15.05.1862 – 21.10.1931), \emph{Schriftsteller, Mediziner}|pwk}: \emph{Spaziergang}\pwindex{Schnitzler, Arthur 15.05.1862 – 21.10.1931@\textsc{Schnitzler, Arthur} (15.05.1862 – 21.10.1931), \emph{Schriftsteller, Mediziner}!Spaziergang06. 12. 1893@\strich\emph{Spaziergang} {[}06. 12. 1893{]}|pwk}. In: \emph{Deutsche
                        Zeitung}\pwindex{Deutsche Zeitung1871 – 1907@\emph{Deutsche Zeitung}|pwk}, Jg. 23, Nr. 7883, 6. 12. 1893, Morgen-Ausgabe,
                     S. 1–2 (heute in A. S. \emph{Entworfenes und Verworfenes} 152–156).}}}\label{K_L00276_4h} der verſprochenen \label{K_L00276_5v}\edtext{Beiträge zur Entdeckung von \textsc{Wien}\oindex{Wien@\textbf{Wien}|pw}\pwindex{Schnitzler, Arthur 15.05.1862 – 21.10.1931@\textsc{Schnitzler, Arthur} (15.05.1862 – 21.10.1931), \emph{Schriftsteller, Mediziner}!Spaziergang06. 12. 1893@\strich\emph{Spaziergang} {[}06. 12. 1893{]}|pwv}}{\lemma{\textnormal{\emph{Beiträge … Wien}}}\Cendnote{\textnormal{\emph{Spaziergang}\pwindex{Schnitzler, Arthur 15.05.1862 – 21.10.1931@\textsc{Schnitzler, Arthur} (15.05.1862 – 21.10.1931), \emph{Schriftsteller, Mediziner}!Spaziergang06. 12. 1893@\strich\emph{Spaziergang} {[}06. 12. 1893{]}|pwk} eröffnete die Serie, die unter dem
                  Titel »Wien\oindex{Wien@\textbf{Wien}|pwk}er Spiegel« laufen sollte. Dem ersten
                  Beitrag war eine erklärende Fußnote beigesellt: »Der ›\so{Wien}\oindex{Wien@\textbf{Wien}|pw}\so{er Spiegel}‹ soll in losen Skizzen die Wien\oindex{Wien@\textbf{Wien}|pw}er Welt, oben und unten, Gesellschaft und Volk, Salon
                     und Straße bringen. Das ganze Wien\oindex{Wien@\textbf{Wien}|pw}er Leben will
                     er Stück für Stück allmälig erzählen. Beiträge haben Ferdinand v. \so{Saar}\pwindex{Saar, Ferdinand von 30.09.1833 – 24.07.1906@\textsc{Saar, Ferdinand von} (30.09.1833 – 24.07.1906), \emph{Schriftsteller}|pw}, Emil \so{Marriot}\pwindex{Marriot, Emil 1855 – 1938@\textsc{Marriot, Emil} (1855 – 1938), \emph{Schriftstellerin}|pw}, Ada \so{Christen}\pwindex{Christen, Ada 1839-03-06 – 1901-05-19@\textsc{Christen, Ada} (1839-03-06 – 1901-05-19), \emph{Schriftstellerin, Schauspielerin}|pw}, C. \so{Karlweis}\pwindex{Karlweis, Carl 23.11.1850 – 27.10.1901@\textsc{Karlweis, Carl} (23.11.1850 – 27.10.1901), \emph{Schriftsteller}|pw}, Gustav \so{Schwarzkopf}\pwindex{Schwarzkopf, Gustav 07.11.1853 – 13.11.1939@\textsc{Schwarzkopf, Gustav} (07.11.1853 – 13.11.1939), \emph{Schriftsteller}|pw}, Vincenz \so{Chiavacci}\pwindex{Chiavacci, Vincenz 15.06.1847 – 02.02.1916@\textsc{Chiavacci, Vincenz} (15.06.1847 – 02.02.1916), \emph{Schriftsteller, Journalist, Beamter}|pw}, Karl \so{Rabis}\pwindex{Rabis, Karl 1836/1837 – 1901@\textsc{Rabis, Karl} (1836/1837 – 1901), \emph{Schriftsteller}|pw}, Theodor \so{Taube}\pwindex{Taube, Theodor 23.02.1840 – 03.07.1904@\textsc{Taube, Theodor} (23.02.1840 – 03.07.1904), \emph{Schriftsteller/Schriftstellerin, Journalist/Journalistin}|pw}, Hugo v. \so{Hofmannsthal}\pwindex{Hofmannsthal, Hugo von 01.02.1874 – 15.07.1929@\textsc{Hofmannsthal, Hugo von} (01.02.1874 – 15.07.1929), \emph{Schriftsteller}|pw}, Arthur \so{Schnitzler}\pwindex{Schnitzler, Arthur 15.05.1862 – 21.10.1931@\textsc{Schnitzler, Arthur} (15.05.1862 – 21.10.1931), \emph{Schriftsteller, Mediziner}|pw}, Dr. \so{Beer-Hofmann}\pwindex{Beer-Hofmann, Richard 11.07.1866 – 26.09.1945@\textsc{Beer-Hofmann, Richard} (11.07.1866 – 26.09.1945), \emph{Schriftsteller}|pw}, Hermann \so{Bahr}\pwindex{Bahr, Hermann 19.07.1863 – 15.01.1934@\textsc{Bahr, Hermann} (19.07.1863 – 15.01.1934), \emph{Schriftsteller, Kritiker}|pw} und Andere versprochen. Anmerkung der Redaction.« Nach dem
                  zweiten Teil, \emph{Heunt is Sunntag!}\pwindex{Taube, Theodor 23.02.1840 – 03.07.1904@\textsc{Taube, Theodor} (23.02.1840 – 03.07.1904), \emph{Schriftsteller/Schriftstellerin, Journalist/Journalistin}!Heunt is Sunntag10. 12. 1893@\strich\emph{Heunt is Sunntag{\rufezeichen}} {[}10. 12. 1893{]}|pwk} von Taube\pwindex{Taube, Theodor 23.02.1840 – 03.07.1904@\textsc{Taube, Theodor} (23.02.1840 – 03.07.1904), \emph{Schriftsteller/Schriftstellerin, Journalist/Journalistin}|pwk} (Nr. 7887,
                        10. 12. 1893, Sonntags-Ausgabe, S. 1–2), und Bahrs\pwindex{Bahr, Hermann 19.07.1863 – 15.01.1934@\textsc{Bahr, Hermann} (19.07.1863 – 15.01.1934), \emph{Schriftsteller, Kritiker}|pwk} Ausscheiden aus der \emph{Deutschen Zeitung}\orgindex{Deutsche Zeitung@Deutsche Zeitung|pwk} wurde sie eingestellt.}}}\label{K_L00276_5h} beſtimmt
               rechnen?\pend
           \pstart
           In herzlicher Freundſchaft{\\[\baselineskip]}\spacefill\mbox{{[}hs. Bahr:{]} HermannBahr}\pend
           \leftskip=0em{}          \endnumbering\briefempfaengerindex{Schnitzler, Arthur@\textsc{Schnitzler, Arthur}!zzzBahr, Hermann@\emph{von Hermann Bahr}!1893-10-251@{25. 10. 1893}|)be}\mylabel{h}\end{ledgroupsized}  \newcommand{\dateiname}{L00276}\newcommand{\titel}{Hermann Bahr an Arthur Schnitzler, 25. 10. 1893}\newcommand{\editorInnen}{ Kurt Ifkovits,  Martin Anton Müller}
            \footnotesize
\begin{ledgroupsized}[t]{11.5cm}
\doendnotes{C}
\end{ledgroupsized}
         %% latex-leseansicht-abspann.tex
%% Abspann für die Leseansicht.
%% Der Schalter \ifkorrekturansicht ist bereits durch den Vorspann gesetzt.

%% latex-abspann.tex
%% Gemeinsamer Abspann für Korrekturansicht und Leseansicht.
%% Setzt den Schalter \ifkorrekturansicht voraus (gesetzt in den
%% einbindenden Dateien latex-korrekturansicht-abspann.tex bzw.
%% latex-leseansicht-abspann.tex).
%% ---------------------------------------------------------------

\normalsize

% Das esempio-Environment wird nur in der Leseansicht benötigt
\ifkorrekturansicht\else
\newenvironment{esempio}[3]%
{
    \vspace{1.5ex}
    \rlap{\underline{#1}}
    \par
    \setlength{\parindent}{0cm}
    \nopagebreak
    \leftskip=#2cm
    \rightskip=#3cm
}
{
    \par
}
\fi

\doendnotes{C}
\bigskip
\vfill

\clearpage

\footnotesize

\ifkorrekturansicht
  \lohead{\textsc{register}}
\fi

% theindex-Environment neu definieren ohne reledmac
\makeatletter
\renewenvironment{theindex}{%
  \ifkorrekturansicht
    \section*{\indexname}%
  \else
    \subsubsection*{Index der erwähnten Entitäten}%
  \fi
  \setlength{\parindent}{0pt}%
  \setlength{\parskip}{0pt plus 0.3pt}%
  \let\item\@idxitem
}{%
  \ifkorrekturansicht\clearpage\fi
}
\makeatother

\IfFileExists{\jobname-pw.ind}{\input{\jobname-pw.ind}}{}

% Quellenangabe nur in der Leseansicht
\ifkorrekturansicht\else
% Fallback-Definitionen, falls die .tex-Datei \titel etc. nicht gesetzt hat
\providecommand{\titel}{}
\providecommand{\editorInnen}{}
\providecommand{\dateiname}{\jobname}

\vspace{3cm}

\vfill

\footnotesize
\textsc{Quelle}: \titel. Herausgegeben von {\editorInnen}. In: \emph{Arthur Schnitzler: Briefwechsel mit Autorinnen und Autoren}.
 Digitale Edition, https://schnitzler-briefe.acdh.oeaw.ac.at/{\dateiname}.html (Stand \today)
\fi

\end{document}


      