%% latex-leseansicht-vorspann.tex
%% Vorspann für die Leseansicht.
%% Lädt die gemeinsame Datei latex-vorspann.tex mit nicht gesetztem Schalter.

\newif\ifkorrekturansicht
\korrekturansichtfalse

\input{../tex-inputs/latex-vorspann}


\section[Hermann Bahr an Arthur Schnitzler, 25. 10. 1893]{L00276 Hermann Bahr an Arthur Schnitzler, 25. 10. 1893}
\nopagebreak\mylabel{L00276v}
\rehead{ }\normalsize\beginnumbering\briefempfaengerindex{Schnitzler, Arthur@\textsc{Schnitzler, Arthur}!zzzBahr, Hermann@\emph{von Hermann Bahr}!1893-10-251@{25. 10. 1893}|(be}
\toendnotes[C]{\smallbreak\pagebreak[2]}
\correspDesc{Versand  durch Hermann Bahr am 25. 10. 1893 in Wien
\newline{}Erhalt  durch Arthur Schnitzler im Zeitraum [25. 10. 1893 – 29. 10. 1893?] in Wien}\toendnotes[C]{\smallbreak}
\Standort{CUL, Schnitzler, B 5b.}
\physDesc{Brief, 1 Blatt, 2 Seiten, 707 Zeichen
\newline{}Handschrift Schreibkraft: schwarze Tinte, deutsche Kurrent
\newline{}Handschrift Hermann Bahr: schwarze Tinte, deutsche Kurrent (\noindent{}Unterschrift)
\newline{}Ordnung: 1) mit rotem Buntstift von unbekannter Hand nummeriert:
                                    »15«  2) mit Bleistift von unbekannter Hand nummeriert:
                                    »15«}
\buchAbdrucke{\weitereDrucke{Hermann Bahr, Arthur Schnitzler: \emph{Briefwechsel, Aufzeichnungen, Dokumente (1891–1931)}. Herausgegeben von Kurt Ifkovits und Martin Anton Müller. Göttingen: \emph{Wallstein} 2018, S. 45.} }\toendnotes[C]{\smallbreak}
\pstart
           {\pb}\textcolor{gray}{\textbf{Deutſche Zeitung\orgindex{Deutsche Zeitung@Deutsche Zeitung|pw}}}\hfill \uline{Wien\oindex{Wien@\textbf{Wien}, \emph{Verwaltungsgebiet}|pw}}, 25. Octbr. 1893\pend
           
\pstart
           \textcolor{gray}{\textbf{Wien\oindex{Wien@\textbf{Wien}, \emph{Verwaltungsgebiet}|pw}}}\hfill III. Saleſianerg. 12\oindex{Wien@\textbf{Wien}!III., Landstraße@\textbf{III., Landstraße}!Salesianergasse 12@\textbf{Salesianergasse 12}, \emph{Wohngebäude}|pw}\pend
           
\pstart
           \textcolor{gray}{\textbf{IX., Pelikangaſſe 4\oindex{Wien@\textbf{Wien}!IX., Alsergrund@\textbf{IX., Alsergrund}!Pelikangasse@\textbf{Pelikangasse}, \emph{Straße}|pw}.}}\pend
           
\pstart{}Verehrter Freund!\pend\vspace{0.5em}
\pstart
           Der Mann um den es{ }ſich handelt heißt Johann Lukas
                     \textsc{Schönlein}\pwindex{Schönlein, Johann Lukas 30.\,11.\,1793 Bamberg – 23.\,1.\,1864 ebd.@\textsc{Schönlein, Johann Lukas} (30.\,11.\,1793 Bamberg – 23.\,1.\,1864 ebd.), \emph{Mediziner}|pw}. Er iſt der Begründer der{ }ſog. naturhyſteriſchen Schule in der Therapie. Am
                  30. November{ }ſind es hundert Jahre, daß er geboren wurde und ich
               brauche alſo für dieſen Tag ein nicht über{ }ſechs Spalten langes, populäres, \label{K_L00276-1v}\edtext{byografiſches Feuilleton}{\lemma{\textnormal{\emph{byografisches Feuilleton}}}\Cendnote{\textnormal{nicht erschienen}}}\label{K_L00276-1}. Können Sie mir das
               verſchaffen?\pend
           
\pstart
           Dabei wiederhole ich die bereits \label{K_L00276-2v}\edtext{neulich}{\lemma{\textnormal{\emph{neulich}}}\Cendnote{\textnormal{Vermutlich geschah das beim Besuch Hofmannsthals\pwindex{Hofmannsthal, Hugo von 1.\,2.\,1874 Wien – 15.\,7.\,1929 Rodaun@\textsc{Hofmannsthal, Hugo von} (1.\,2.\,1874 Wien – 15.\,7.\,1929 Rodaun), \emph{Schriftsteller}|pwk} am 22. 10. 1893.
               }}}\label{K_L00276-2} durch \textsc{Loris}\pwindex{Hofmannsthal, Hugo von 1.\,2.\,1874 Wien – 15.\,7.\,1929 Rodaun@\textsc{Hofmannsthal, Hugo von} (1.\,2.\,1874 Wien – 15.\,7.\,1929 Rodaun), \emph{Schriftsteller}|pw} vermittelte Bitte um irgend eine Novellette,{ }ſo kurz als möglich, die ich am
                  \label{K_L00276-3v}\edtext{Tage Ihrer Premiere}{\lemma{\textnormal{\emph{Tage Ihrer Premiere}}}\Cendnote{\textnormal{Am 1. 12. 1893 Premiere von \emph{Das Märchen}\pwindex{Schnitzler, Arthur 15.\,5.\,1862 Wien – 21.\,10.\,1931 ebd.@\textsc{Schnitzler, Arthur} (15.\,5.\,1862 Wien – 21.\,10.\,1931 ebd.), \emph{Schriftsteller, Mediziner}!Märchen. Schauspiel in drei Aufzügen@\strich\emph{Das Märchen. Schauspiel in drei Aufzügen}|pwk}; an diesem Tag erschien nichts von Schnitzler.}}}\label{K_L00276-3} bringen will.\pend
           
\pstart
           {\pb}Kann ich bis längſtens Ende der nächſten Woche auf
               den \label{K_L00276-4v}\edtext{erſten}{\lemma{\textnormal{\emph{ersten}}}\Cendnote{\textnormal{Arthur Schnitzler: \emph{Spaziergang}\pwindex{Schnitzler, Arthur 15.\,5.\,1862 Wien – 21.\,10.\,1931 ebd.@\textsc{Schnitzler, Arthur} (15.\,5.\,1862 Wien – 21.\,10.\,1931 ebd.), \emph{Schriftsteller, Mediziner}!Spaziergang@\strich\emph{Spaziergang}|pwk}. In: \emph{Deutsche
                        Zeitung}\pwindex{Deutsche Zeitung@\emph{Deutsche Zeitung}|pwk}, Jg. 23, Nr. 7883, 6. 12. 1893, Morgen-Ausgabe,
                     S. 1–2 (heute in vgl. Arthur Schnitzler: \emph{Entworfenes und Verworfenes. Aus dem Nachlaß}. Herausgegeben von Reinhard Urbach. Frankfurt/Main: \emph{S. Fischer}{ }1977, S. 152–156).}}}\label{K_L00276-4} der verſprochenen \label{K_L00276-5v}\edtext{Beiträge zur Entdeckung von \textsc{Wien}\oindex{Wien@\textbf{Wien}, \emph{Verwaltungsgebiet}|pw}\pwindex{Schnitzler, Arthur 15.\,5.\,1862 Wien – 21.\,10.\,1931 ebd.@\textsc{Schnitzler, Arthur} (15.\,5.\,1862 Wien – 21.\,10.\,1931 ebd.), \emph{Schriftsteller, Mediziner}!Spaziergang@\strich\emph{Spaziergang}|pwv}}{\lemma{\textnormal{\emph{Beiträge … Wien}}}\Cendnote{\textnormal{\emph{Spaziergang}\pwindex{Schnitzler, Arthur 15.\,5.\,1862 Wien – 21.\,10.\,1931 ebd.@\textsc{Schnitzler, Arthur} (15.\,5.\,1862 Wien – 21.\,10.\,1931 ebd.), \emph{Schriftsteller, Mediziner}!Spaziergang@\strich\emph{Spaziergang}|pwk} eröffnete die Serie, die unter
                  dem Titel »Wien\oindex{Wien@\textbf{Wien}, \emph{Verwaltungsgebiet}|pwk}er Spiegel« laufen sollte. Dem
                  ersten Beitrag war eine erklärende Fußnote beigesellt: »Der ›\so{Wien}\oindex{Wien@\textbf{Wien}, \emph{Verwaltungsgebiet}|pw}\so{er Spiegel}‹ soll in losen Skizzen die Wien\oindex{Wien@\textbf{Wien}, \emph{Verwaltungsgebiet}|pw}er Welt, oben und unten, Gesellschaft und
                     Volk, Salon und Straße bringen. Das ganze Wien\oindex{Wien@\textbf{Wien}, \emph{Verwaltungsgebiet}|pw}er Leben will er Stück für Stück allmälig erzählen. Beiträge haben
                        Ferdinand v. \so{Saar}\pwindex{Saar, Ferdinand von 30.\,9.\,1833 Wien – 24.\,7.\,1906 ebd.@\textsc{Saar, Ferdinand von} (30.\,9.\,1833 Wien – 24.\,7.\,1906 ebd.), \emph{Schriftsteller}|pw}, Emil \so{Marriot}\pwindex{Marriot, Emil 20.\,11.\,1855 Wien – 1938 ebd.@\textsc{Marriot, Emil} (20.\,11.\,1855 Wien – 1938 ebd.), \emph{Schriftstellerin}|pw}, Ada \so{Christen}\pwindex{Christen, Ada 6.\,3.\,1839 Wien – 19.\,5.\,1901 ebd.@\textsc{Christen, Ada} (6.\,3.\,1839 Wien – 19.\,5.\,1901 ebd.), \emph{Schriftstellerin, Schauspielerin}|pw}, C. \so{Karlweis}\pwindex{Karlweis, Carl 23.\,11.\,1850 Wien – 27.\,10.\,1901 ebd.@\textsc{Karlweis, Carl} (23.\,11.\,1850 Wien – 27.\,10.\,1901 ebd.), \emph{Schriftsteller}|pw}, Gustav \so{Schwarzkopf}\pwindex{Schwarzkopf, Gustav 7.\,11.\,1853 Wien – 13.\,11.\,1939 ebd.@\textsc{Schwarzkopf, Gustav} (7.\,11.\,1853 Wien – 13.\,11.\,1939 ebd.), \emph{Schriftsteller}|pw}, Vincenz \so{Chiavacci}\pwindex{Chiavacci, Vincenz 15.\,6.\,1847 Wien – 2.\,2.\,1916 ebd.@\textsc{Chiavacci, Vincenz} (15.\,6.\,1847 Wien – 2.\,2.\,1916 ebd.), \emph{Schriftsteller, Journalist, Beamter}|pw}, Karl \so{Rabis}\pwindex{Rabis, Karl 1836/1837 Kranichfeld – 1901 Mödling@\textsc{Rabis, Karl} (1836/1837 Kranichfeld – 1901 Mödling), \emph{Schriftsteller}|pw}, Theodor \so{Taube}\pwindex{Taube, Theodor 23.\,2.\,1840 Wien – 3.\,7.\,1904 ebd.@\textsc{Taube, Theodor} (23.\,2.\,1840 Wien – 3.\,7.\,1904 ebd.), \emph{Schriftsteller, Journalist, Redakteur}|pw}, Hugo v. \so{Hofmannsthal}\pwindex{Hofmannsthal, Hugo von 1.\,2.\,1874 Wien – 15.\,7.\,1929 Rodaun@\textsc{Hofmannsthal, Hugo von} (1.\,2.\,1874 Wien – 15.\,7.\,1929 Rodaun), \emph{Schriftsteller}|pw}, Arthur \so{Schnitzler}, Dr. \so{Beer-Hofmann}\pwindex{Beer-Hofmann, Richard 11.\,7.\,1866 Wien – 26.\,9.\,1945 New York City@\textsc{Beer-Hofmann, Richard} (11.\,7.\,1866 Wien – 26.\,9.\,1945 New York City), \emph{Schriftsteller}|pw}, Hermann \so{Bahr}\pwindex{Bahr, Hermann 19.\,7.\,1863 Linz – 15.\,1.\,1934 München@\textsc{Bahr, Hermann} (19.\,7.\,1863 Linz – 15.\,1.\,1934 München), \emph{Schriftsteller, Kritiker}|pw} und Andere versprochen. Anmerkung der Redaction.« Nach dem
                  zweiten Teil, \emph{Heunt is Sunntag!}\pwindex{Taube, Theodor 23.\,2.\,1840 Wien – 3.\,7.\,1904 ebd.@\textsc{Taube, Theodor} (23.\,2.\,1840 Wien – 3.\,7.\,1904 ebd.), \emph{Schriftsteller, Journalist, Redakteur}!Heunt is Sunntag@\strich\emph{Heunt is Sunntag{\rufezeichen}}|pwk} von Taube\pwindex{Taube, Theodor 23.\,2.\,1840 Wien – 3.\,7.\,1904 ebd.@\textsc{Taube, Theodor} (23.\,2.\,1840 Wien – 3.\,7.\,1904 ebd.), \emph{Schriftsteller, Journalist, Redakteur}|pwk} (Nr. 7887,
                        10. 12. 1893, Sonntags-Ausgabe, S. 1–2), und Bahrs\pwindex{Bahr, Hermann 19.\,7.\,1863 Linz – 15.\,1.\,1934 München@\textsc{Bahr, Hermann} (19.\,7.\,1863 Linz – 15.\,1.\,1934 München), \emph{Schriftsteller, Kritiker}|pwk} Ausscheiden aus der \emph{Deutschen Zeitung}\orgindex{Deutsche Zeitung@Deutsche Zeitung|pwk} wurde sie eingestellt.}}}\label{K_L00276-5}
               beſtimmt rechnen?\pend
           
\pstart
           In herzlicher Freundſchaft{\\[\baselineskip]}\spacefill\mbox{{[}hs. Bahr:{]} HermannBahr}\pend
           \leftskip=0em{}\selectlanguage{ngerman}\endnumbering\briefempfaengerindex{Schnitzler, Arthur@\textsc{Schnitzler, Arthur}!zzzBahr, Hermann@\emph{von Hermann Bahr}!1893-10-251@{25. 10. 1893}|)be}\mylabel{L00276h}  \newcommand{\dateiname}{L00276}\newcommand{\titel}{Hermann Bahr an Arthur Schnitzler, 25. 10. 1893}\newcommand{\editorInnen}{Herausgegeben von Martin Anton Müller}%% latex-leseansicht-abspann.tex
%% Abspann für die Leseansicht.
%% Der Schalter \ifkorrekturansicht ist bereits durch den Vorspann gesetzt.

%% latex-abspann.tex
%% Gemeinsamer Abspann für Korrekturansicht und Leseansicht.
%% Setzt den Schalter \ifkorrekturansicht voraus (gesetzt in den
%% einbindenden Dateien latex-korrekturansicht-abspann.tex bzw.
%% latex-leseansicht-abspann.tex).
%% ---------------------------------------------------------------

\normalsize

% Das esempio-Environment wird nur in der Leseansicht benötigt
\ifkorrekturansicht\else
\newenvironment{esempio}[3]%
{
    \vspace{1.5ex}
    \rlap{\underline{#1}}
    \par
    \setlength{\parindent}{0cm}
    \nopagebreak
    \leftskip=#2cm
    \rightskip=#3cm
}
{
    \par
}
\fi

\doendnotes{C}
\bigskip
\vfill

\clearpage

\footnotesize

\ifkorrekturansicht
  \lohead{\textsc{register}}
\fi

% theindex-Environment neu definieren ohne reledmac
\makeatletter
\renewenvironment{theindex}{%
  \ifkorrekturansicht
    \section*{\indexname}%
  \else
    \subsubsection*{Index der erwähnten Entitäten}%
  \fi
  \setlength{\parindent}{0pt}%
  \setlength{\parskip}{0pt plus 0.3pt}%
  \let\item\@idxitem
}{%
  \ifkorrekturansicht\clearpage\fi
}
\makeatother

\IfFileExists{\jobname-pw.ind}{\input{\jobname-pw.ind}}{}

% Quellenangabe nur in der Leseansicht
\ifkorrekturansicht\else
% Fallback-Definitionen, falls die .tex-Datei \titel etc. nicht gesetzt hat
\providecommand{\titel}{}
\providecommand{\editorInnen}{}
\providecommand{\dateiname}{\jobname}

\vspace{3cm}

\vfill

\footnotesize
\textsc{Quelle}: \titel. Herausgegeben von {\editorInnen}. In: \emph{Arthur Schnitzler: Briefwechsel mit Autorinnen und Autoren}.
 Digitale Edition, https://schnitzler-briefe.acdh.oeaw.ac.at/{\dateiname}.html (Stand \today)
\fi

\end{document}


