%% latex-korrekturansicht-vorspann.tex
%% Vorspann für die Korrekturansicht.
%% Lädt die gemeinsame Datei latex-vorspann.tex mit gesetztem Schalter.

\newif\ifkorrekturansicht
\korrekturansichttrue

\input{../tex-inputs/latex-vorspann}


\section[ Felix Salten an Arthur Schnitzler, 18. 8. 1893]{L03128 Felix Salten an Arthur Schnitzler, 18. 8. 1893}
\nopagebreak\mylabel{L03128v}
\rehead{ }\normalsize\beginnumbering\briefempfaengerindex{Schnitzler, Arthur@\textsc{Schnitzler, Arthur}!zzzSalten, Felix@\emph{von Felix Salten}!1893-08-183@{18. 8. 1893}|(be}
\toendnotes[C]{\smallbreak\pagebreak[2]}\Standort{CUL, Schnitzler, B 89, A 1.}
\physDesc{Brief, 2 Blätter, 4 Seiten, 4010 Zeichen
\newline{}Handschrift: Bleistift, lateinische Kurrent
\newline{}Ordnung: mit Bleistift von unbekannter Hand nummeriert: »31« }\toendnotes[C]{\smallbreak}
\pstart
           \raggedleft{}{\pb}Iselsberg\oindex{Iselsberg@\textbf{Iselsberg}, \emph{P.PPL}|pw}, 18. VIII. 93.\pend
           \vspace{0.5em}
\pstart
           Lieber Freund!{ }heute sandte ich Ihnen ein \label{K_L03128-1v}\edtext{Telegramm}{\lemma{\textnormal{\emph{Telegramm}}}\Cendnote{\textnormal{nicht
                  erhalten}}}\label{K_L03128-1} und habe Ihnen noch die Leidensgeschichte meines Rades zu
                  erzählen\textcolor{gray}{.} Mein Rad kam schon vom Eisenbahntransporte nicht ganz
               wol an, die Glocke war abgeschraubt, ein Pedal verbogen. Zudem hat es der Schnellzug
               nicht mitgenommen, es wurde mir von Wien\oindex{Wien@\textbf{Wien}, \emph{A.ADM2}|pw} per
               Postzug nachgeschickt, und man hatte überdies vergessen\textcolor{gray}{,} es in Dölsach\oindex{Doelsach@\textbf{Dölsach}, \emph{A.ADM3}|pw} auszuladen, es fuhr bis München\oindex{Muenchen@\textbf{München}, \emph{P.PPLA}|pw}\textcolor{gray}{,} stand da einen Tag, und wer weiß, wer sich dort mit ihm spielte.
               Allein die Tour von Toblach\oindex{Toblach@\textbf{Toblach}, \emph{A.ADM3}|pw} nach Cortina\oindex{Cortina DAmpezzo@\textbf{Cortina d’Ampezzo}, \emph{P.PPLA3}|pw} ging recht gut vor sich, auch zurück. Da ich noch
                  Vormittag wieder aus Cortina\oindex{Cortina DAmpezzo@\textbf{Cortina d’Ampezzo}, \emph{P.PPLA3}|pw} in
                  Toblach\oindex{Toblach@\textbf{Toblach}, \emph{A.ADM3}|pw} ankam, und bis ½ 8 auf
               den Zug nach Dölsach\oindex{Doelsach@\textbf{Dölsach}, \emph{A.ADM3}|pw} hätte warten müssen, und mir
               überdies die Straße von Toblach\oindex{Toblach@\textbf{Toblach}, \emph{A.ADM3}|pw} hinunter nach Lienz\oindex{Lienz@\textbf{Lienz}, \emph{P.PPLA3}|pw} als vortrefflich geschildert wurde,
               entschloß ich mich weiterzufahren. Nun war es hier{[},{]}{ }{\pb}wie überall, mit den
               Schilderungen der Leute schlecht bestellt. Ich fand wol stetes, oft scharfes Bergab,
               aber eine verwahrloste Straße, voll Schotter, theilweise mit Gras bewachsen, und
               überall faßt fußhoher Staub. Doch ging’s die ganze Strecke noch leidlich, nur eine
               auffallend leichte Lenkbarkeit des \label{K_L03128-2v}\edtext{Gouvernals}{\lemma{\textnormal{\emph{Gouvernals}}}\Cendnote{\textnormal{Fahrradlenker}}}\label{K_L03128-2}, die
               ich mir nicht erklären konnte, bis zwischen Mittewald\oindex{Mittewald an der Drau@\textbf{Mittewald an der Drau}, \emph{P.PPL}|pw}{ }{\kaufmannsund}{ }Lienz\oindex{Lienz@\textbf{Lienz}, \emph{P.PPLA3}|pw} mein Rad einfach zu taumeln begann, und die
               Kugellagerung im Gouvernal bei jeder Schwenkung knackte. Bei näherer Besichtigung,
               ergab sich das der \label{K_L03128-3v}\edtext{Conus}{\lemma{\textnormal{\emph{Conus}}}\Cendnote{\textnormal{kegelförmiges Bauteil}}}\label{K_L03128-3} ganz gelockert
               war, offenbar war einer der Stifte\textcolor{gray}{,} die ihn halten gebrochen. In Lienz\oindex{Lienz@\textbf{Lienz}, \emph{P.PPLA3}|pw} fand ich am selben Tag
               keine Hilfe, es war |:Dienstag:| \label{K_L03128-4v}\edtext{Feiertag}{\lemma{\textnormal{\emph{Feiertag}}}\Cendnote{\textnormal{Mariä Himmelfahrt}}}\label{K_L03128-4} und alles geschloßen. Mittwoch ging ich hinein, und erhielt die Auskunft, man
               müße erst untersuchen, und würde mir die Post sagen laßen. Gestern{ }Abend vom Glockner\oindex{Grossglockner@\textbf{Großglockner}, \emph{T.MT}|pw} zurückgekehrt,
               fand ich die Nachricht, dass einige Kugeln, und (wie ich vermuthet hatte) die Stifte
               gebrochen seien, und dass mein Rad nicht, wie ich {\pb}verlangt hatte bis Sonntag, sondern erst Ende der nächsten Woche fertig
               werden könne. Was jetzt zu thun ist, weiss ich nicht. Abgesehen davon, dass ich nun
               die Aussicht habe hier sitzen zu bleiben, und mich unbeschreiblich zu langweilen, ist
               mir die Sache \label{K_L03128-5v}\edtext{mit Rücksicht auf
                  Sie}{\lemma{\textnormal{\emph{mit Rücksicht auf
                  Sie}}}\Cendnote{\textnormal{Sie hatten eine gemeinsame Radtour
                  geplant, vgl. Felix Salten an Arthur Schnitzler, 14. 8. 1893.}}}\label{K_L03128-5} sehr
               unangenehm. Wie ich mich auf diese Tour gefreut habe, kann ich Ihnen nicht sagen, ich
               habe am ganzen Weg nach Ampezzo\oindex{Ampezzo@\textbf{Ampezzo}, \emph{P.PPLA3}|pw} daran gedacht,
               wie schön es sein wird, hier mit Ihnen nochmals hereinzufahren. Die Parthie nach Heiligenblut\oindex{Heiligenblut am Grossglockner@\textbf{Heiligenblut am Großglockner}, \emph{P.PPLA3}|pw} und von da auf die Franz-Josefshöhe\oindex{Kaiser-Franz-Josefs-Hoehe@\textbf{Kaiser-Franz-Josefs-Höhe}, \emph{Landschaftlicher Punkt}|pw} zur Pasterze\oindex{Pasterze Glacier@\textbf{Pasterze Glacier}, \emph{H.GLCR}|pw} war zwar sehr schön, aber sie hat mich furchtbar
               übermüdet, so dass ich heute nicht aus dem Hause gehe.
               Ich habe sie auch nur meinem Bruder\pwindex{Salzmann, Michael Emil 1858-01-19 – 1908-06-26@\textsc{Salzmann, Michael Emil} (1858-01-19 – 1908-06-26), \emph{Versicherungsbeamter/Versicherungsbeamtin}|pwuv} zuliebe gemacht, weil ich von Ampezzo\oindex{Ampezzo@\textbf{Ampezzo}, \emph{P.PPLA3}|pw} noch müde war, u. dann dachte ich mir, vielleicht
               wird das Rad bis Sonntag od. Montag doch fertig, dann kommen Sie, und ich kann nicht mehr nach Heiligenblut\oindex{Heiligenblut am Grossglockner@\textbf{Heiligenblut am Großglockner}, \emph{P.PPLA3}|pw}. Ich bin so von der Sonne
               verbrannt, dass mir das ganze Gesicht weh thut, und sich mir die Haut vom Halse
               schält. –\pend
           
\pstart
           {\pb}Schreiben Sie mir, bitte,
               wozu Sie sich entschließen. Wenn Sie hier herum eine Tour machen, dann könnten wir
               uns Sonntag doch vielleicht in Toblach\oindex{Toblach@\textbf{Toblach}, \emph{A.ADM3}|pw} treffen, um die Tour nach Cortina\oindex{Cortina DAmpezzo@\textbf{Cortina d’Ampezzo}, \emph{P.PPLA3}|pw} wenigstens gemeinschaftlich zu machen.\pend
           
\pstart
           Ampezzo\oindex{Ampezzo@\textbf{Ampezzo}, \emph{P.PPLA3}|pw} sollen Sie sich unter keiner Bedingung
               entgehen laßen. Man findet nirgends so eine schöne Straße, und so eine Gegend.\pend
           
\pstart
           Jedenfalls wird mir bis auf weiteres nichts übrig bleiben, als versuchen zu
                  schreiben um mir »den Tach um die Ohren zu schlagen.«\pend
           
\pstart
           Noch Eins. Wollen Sie nicht \label{K_L03128-6v}\edtext{zu meinem
                  Papa\pwindex{Salzmann, Philipp 1831-12-24 – 1905-04-02@\textsc{Salzmann, Philipp} (1831-12-24 – 1905-04-02), \emph{Bergbauunternehmer/Bergbauunternehmerin, Projektemacher/Projektemacherin}|pwv} gehen}{\lemma{\textnormal{\emph{zu meinem
                  Papa gehen}}}\Cendnote{\textnormal{nicht nachweisbar}}}\label{K_L03128-6}, und ihm sagen, er
               soll mir mehr Geld geben? Er stellt sich vor, man bekommt hier Alles geschenkt. Sie
               könnten ihm ordentlich zureden, er hört auf Sie, und es würde mir jetzt
                  n\textcolor{gray}{ü}tzen.\pend
           
\pstart
           Jedenfalls bitte ich Sie um baldige Nachricht. Mir träumte heute Frl. \label{K_L03128-7v}\edtext{Fifi\pwindex{Weisswasser, Josefine Lydia von *~01.03.1864@\textsc{Weisswasser, Josefine Lydia von} (*~01.03.1864)|pw}}{\lemma{\textnormal{\emph{Fifi}}}\Cendnote{\textnormal{Josefine Lydia von Weisswasser\pwindex{Weisswasser, Josefine Lydia von *~01.03.1864@\textsc{Weisswasser, Josefine Lydia von} (*~01.03.1864)|pwk}, mit der Schnitzler ein Verhältnis hatte}}}\label{K_L03128-7} käme
               zu mir, und sagte mir, sie habe erfahren, Sie betrügen sie mit Frl. G\pwindex{Gluemer, Marie 03.07.1867 – 16.11.1925@\textsc{Glümer, Marie} (03.07.1867 – 16.11.1925), \emph{Schauspieler/Schauspielerin}|pw}. ich solle ihr helfen. Frl. G.\pwindex{Gluemer, Marie 03.07.1867 – 16.11.1925@\textsc{Glümer, Marie} (03.07.1867 – 16.11.1925), \emph{Schauspieler/Schauspielerin}|pw} saß gerade bei mir und ich wollte sie auf ihre Bitten elektrisiren, denn
               sie behauptete, dann würden Sie sie heirathen\textcolor{gray}{.} Mein Bruder\pwindex{Salzmann, Theodor 1867 – 1926-12-12@\textsc{Salzmann, Theodor} (1867 – 1926-12-12)|pwv} schrie zur Thür
               herein, Minnie B.\pwindex{Schaffgotsch, Hermine von 25.11.1871 – 25.11.1928@\textsc{Schaffgotsch, Hermine von} (25.11.1871 – 25.11.1928)|pw} wolle mich erschlagen, wenn
               ich so was thäte, und ich wusste mir nicht zu helfen und verwünschte Sie mit
               Ihren 3 Frauenzimmern\pwindex{Weisswasser, Josefine Lydia von *~01.03.1864@\textsc{Weisswasser, Josefine Lydia von} (*~01.03.1864)|pwv}\pwindex{Gluemer, Marie 03.07.1867 – 16.11.1925@\textsc{Glümer, Marie} (03.07.1867 – 16.11.1925), \emph{Schauspieler/Schauspielerin}|pwv}\pwindex{Schaffgotsch, Hermine von 25.11.1871 – 25.11.1928@\textsc{Schaffgotsch, Hermine von} (25.11.1871 – 25.11.1928)|pwv}.\pend
           
\pstart
           Heute soll Defregger\pwindex{Defregger, Franz 1835-04-30 – 1921-01-02@\textsc{Defregger, Franz} (1835-04-30 – 1921-01-02), \emph{Maler/Malerin, Künstler/Künstlerin}|pw} her kommen.\pend
           
\pstart
           Seien Sie herzlichst gegrüßt von Ihrem {\\[\baselineskip]}\spacefill\mbox{Salten}\pend
           \leftskip=0em{}\selectlanguage{ngerman}\endnumbering\briefempfaengerindex{Schnitzler, Arthur@\textsc{Schnitzler, Arthur}!zzzSalten, Felix@\emph{von Felix Salten}!1893-08-183@{18. 8. 1893}|)be}\mylabel{L03128h}  \normalsize

\doendnotes{C}
\bigskip
\vfill

\clearpage

\footnotesize

\lohead{\textsc{register}}

% Definiere theindex-Environment komplett neu ohne reledmac
\makeatletter
\renewenvironment{theindex}{%
  \section*{\indexname}%
  \setlength{\parindent}{0pt}%
  \setlength{\parskip}{0pt plus 0.3pt}%
  \let\item\@idxitem
}{%
  \clearpage
}
\makeatother

\IfFileExists{\jobname-pw.ind}{\input{\jobname-pw.ind}}{}

\end{document}

      