%% latex-leseansicht-vorspann.tex
%% Vorspann für die Leseansicht.
%% Lädt die gemeinsame Datei latex-vorspann.tex mit nicht gesetztem Schalter.

\newif\ifkorrekturansicht
\korrekturansichtfalse

\input{../tex-inputs/latex-vorspann}

\begin{center}
            \textcolor{red}{ENTWURF, NICHT FERTIG KORRIGIERT}
                      \end{center}
            
         
         \renewcommand{\erwaehntePersonen}{Personen: Franz Defregger, Marie Glümer, Sophie Link, Michael Emil Salzmann, Philipp Salzmann, Theodor Salzmann, Hermine von Schaffgotsch}
         \renewcommand{\erwaehnteOrte}{Orte: Ampezzo, Cortina d'Ampezzo, Dölsach, Großglockner, Heiligenblut am Großglockner, Innichen, Iselsberg, Kaiser-Franz-Josefs-Höhe, Lienz, Mittewald an der Drau, Pasterze Glacier, Toblach, Wien}
         \renewcommand{\erwaehnteWerke}{}
               \section[Felix Salten an Arthur Schnitzler, 18. 8. 1893]{ Felix Salten an Arthur Schnitzler, 18. 8. 1893}\nopagebreak\mylabel{v}\rehead{ }\begin{ledgroupsized}[t]{13cm}\normalsize\beginnumbering \toendnotes[C]{\smallbreak\pagebreak[2]} \Standort{CUL, Schnitzler, B 89, A 1.}
\physDesc{Brief, 2 Blätter, 4 Seiten
\newline{}Handschrift: Bleistift, lateinische Kurrent}\toendnotes[C]{\smallbreak}\pstart
           \raggedleft{}{\pb}Iselsberg\oindex{Iselsberg@\textbf{Iselsberg}|pw}, 18. VIII.
                  93.\pend
           \pstart
           Lieber Freund! heute sandte ich Ihnen ein Telegramm und habe Ihnen
               noch die Leidensgeschichte meines Rades zu erzählen\textcolor{gray}{.} Mein Rad kam
               schon vom Eisenbahntransporte nicht ganz wol an, die Glocke war abgeschraubt, ein
               Pedal verbogen, zudem hat es der Schnellzug nicht mitgenommen, es wurde mir von Wien\oindex{Wien@\textbf{Wien}|pw} per Postzug nachgeschickt, und man hatte
               überdies vergessen\textcolor{gray}{,} es in Dölsach\oindex{Doelsach@\textbf{Dölsach}|pw} auszuladen, es fuhr bis Inn\oindex{Innichen@\textbf{Innichen}|pw}.\textcolor{gray}{,} stand da einen Tag, und wer weiß, wer sich dort mit
               ihm spielte. Allein die Tour von Toblach\oindex{Toblach@\textbf{Toblach}|pw} nach Cortina\oindex{Cortina d'Ampezzo@\textbf{Cortina d'Ampezzo}|pw} ging recht gut vor sich, auch zurück. Da
               ich noch {[}Vor{]}mittag wieder aus Cortina\oindex{Cortina d'Ampezzo@\textbf{Cortina d'Ampezzo}|pw} in Toblach\oindex{Toblach@\textbf{Toblach}|pw}
                  ankam\textcolor{gray}{,}{ }{[}und{]} bis ½ 8 auf den Zug nach Dölsach\oindex{Doelsach@\textbf{Dölsach}|pw} hätte {[}war{]}ten müssen, und mir
               überdies die {[}Strass{]}e von Toblach\oindex{Toblach@\textbf{Toblach}|pw} hinunter nach Lienz\oindex{Lienz@\textbf{Lienz}|pw}{[}als{]} vortrefflich geschildert wurde, entschloß
                  {[}ich{]} mich weiterzufahren. Nun war es hier {\pb}wie überall, mit den
               Schilderungen der Leute schlecht bestellt. Ich fand wol stetes, oft scharfes Bergab,
               aber eine verwahrloste Straße, voll Schotter, theilweise mit Gras bewachsen, und
               überall faßt fußhoher Staub. Doch ging’s die ganze Strecke noch leidlich, nur eine
               auffallend leichte Lenkbarkeit des Gouvernals, die ich mir nicht erklären konnte, bis
               zwischen Mittewald\oindex{Mittewald an der Drau@\textbf{Mittewald an der Drau}|pw}{ }{\kaufmannsund}{ }Lienz\oindex{Lienz@\textbf{Lienz}|pw} mein Rad einfach zu taumeln begann, und die
               Kugellagerung im Gouvernal bei jeder Schwenkung knackte. Bei näherer Besichtigung,
               ergab sich das der Conus ganz gelockert war, offenbar war einer der
                  Stifte\textcolor{gray}{,} die ihn halten gebrochen. In Lienz fand ich am selben
               Tag keine Hilfe, es war (Dienstag) Feiertag und alles geschloßen.
               Mittwoch ging ich hinein, und erhielt die Auskunft, man müße erst
                  untersuc{[}hen, {]} und würde mir die Post sagen laßen. Gestern
               Abend vom Glockner\oindex{Grossglockner@\textbf{Großglockner}|pw} zurückgekehrt, fand ich die
               Nachricht, dass einige Kugeln, und {[}({]}wie ich vermuthet hatte) die
               Stifte gebrochen seien, und dass mein Rad nicht, wie {[}ich{]}{ }{\pb}verlangt
               hatte bis Sonntag, sondern erst Ende der nächsten Woche fertig
               werden könne. Was jetzt zu thun ist, weiss ich nicht. Abgesehen
                  davon, dass ich nun die Aussicht habe hier
                  sitzen zu bleiben, und mich unbeschreiblich zu
                  langweilen, ist mir die Sache mit Rücksicht auf Sie sehr
               unangenehm. Wie ich mich auf diese Tour gefreut habe, kann
                  ich Ihnen nicht sagen, ich habe am
               ganzen Weg nach Ampezzo\oindex{Ampezzo@\textbf{Ampezzo}|pw}
               daran gedacht, wie schön es sein wird, hier mit Ihnen nochmals
               hereinzufahren. Die Parthie nach Heiligenblut\oindex{Heiligenblut am Grossglockner@\textbf{Heiligenblut am Großglockner}|pw}
               und von da auf die Franz
                  Josefshöhe\oindex{Kaiser-Franz-Josefs-Hoehe@\textbf{Kaiser-Franz-Josefs-Höhe}|pw} zur Pasterze\oindex{Pasterze Glacier@\textbf{Pasterze Glacier}|pw} war zwar sehr schön, aber sie hat mich furchtbar übermüdet, so dass
               ich heute nicht aus dem Hause gehe, Ich habe sie auch nur meinem Bruder\pwindex{Salzmann, Michael Emil 1858-01-19 – 1908-06-26@\textsc{Salzmann, Michael Emil} (1858-01-19 – 1908-06-26), \emph{Versicherungsbeamter}|pwuv} zuliebe gemacht, weil ich von Ampezzo\oindex{Ampezzo@\textbf{Ampezzo}|pw} noch müde war, u. dann dachte ich mir,
               vielleicht wird das Rad bis Sonntag od. Montag doch
                  fertig, dann kommen Sie, und ich kann nicht mehr nach Heiligenbluth\oindex{Heiligenblut am Grossglockner@\textbf{Heiligenblut am Großglockner}|pw}. Ich bin so von der Sonne
               verbrannt, dass mir das ganze Gesicht weh thut, und sich mir die Haut vom Halse
               schält. – \pend
           \pstart
           {\pb}Schreiben Sie mir, bitte, wozu
               Sie sich entschließen\textcolor{gray}{.} Wenn Sie hier herum eine Tour machen, dann
               könnten wir uns Sonntag doch vielleicht in Toblach\oindex{Toblach@\textbf{Toblach}|pw} treffen, um die Tour nach Cortina\oindex{Cortina d'Ampezzo@\textbf{Cortina d'Ampezzo}|pw} wenigstens gemeinschaftlich zu machen. \pend
           \pstart
           Ampezzo\oindex{Ampezzo@\textbf{Ampezzo}|pw} w{[}o{]}llen Sie sich
               unter keiner Bedingung entgehen laßen. Man findet nirgends so eine schöne Straße, und
               so eine Gegend. \pend
           \pstart
           Jedenfalls wird mir bis auf weiteres nichts übrig bleiben, als Verschen zu schreiben,
               um mir »den Tach um die Ohren zu schlagen.« \pend
           \pstart
           Noch Eins. Wollen Sie nicht zu meinem Papa\pwindex{Salzmann, Philipp 1831-12-24 – 1905-04-02@\textsc{Salzmann, Philipp} (1831-12-24 – 1905-04-02), \emph{Bergbauunternehmer}|pwv} gehen, und ihm sagen, er soll mir
               mehr Geld geben? Er stellt sich vor, man bekommt hier Alles geschenkt. Sie könnten
               ihm ordentlich zureden{[},{]} er hört auf Sie, und es würde mir jetzt
                  n\textcolor{gray}{ü}tzen. \pend
           \pstart
           Jedenfalls bitte ich Sie um baldige Nachricht. Mir träumte heute Frl. Sofi\pwindex{Link, Sophie 1860 – 01.10.1900@\textsc{Link, Sophie} (1860 – 01.10.1900), \emph{Sängerin}|pwu} käme zu mir, und sagte mir, sie
               habe erfahren, Sie betrügen sie mit Frl. G.\pwindex{Gluemer, Marie 03.07.1867 – 16.11.1925@\textsc{Glümer, Marie} (03.07.1867 – 16.11.1925), \emph{Schauspielerin}|pw}
               ich solle ihr helfen. Frl. G.\pwindex{Gluemer, Marie 03.07.1867 – 16.11.1925@\textsc{Glümer, Marie} (03.07.1867 – 16.11.1925), \emph{Schauspielerin}|pw} saß gerade bei
               mir und ich wollte sie auf ihre Bitten elektrisie{[}ren,{]} denn sie
               behauptete, dann würden Sie sie heiraten\textcolor{gray}{.} Mein Bruder\pwindex{Salzmann, Theodor 1867 – 1926-12-12@\textsc{Salzmann, Theodor} (1867 – 1926-12-12)|pwv} schrie zur Thür herein, Minnie B.\pwindex{Schaffgotsch, Hermine von 25.11.1871 – 25.11.1928@\textsc{Schaffgotsch, Hermine von} (25.11.1871 – 25.11.1928)|pw} wolle mich erschlagen, wenn ich so was thäte, und
               ich wusste mir nicht zu helfen und verwünschte Sie mit Ihren 3 Frauenzimmern. \pend
           \pstart
           Heute soll Defregger\pwindex{Defregger, Franz 1835-04-30 – 1921-01-02@\textsc{Defregger, Franz} (1835-04-30 – 1921-01-02), \emph{Maler, Künstler}|pw} her kommen. \pend
           \pstart
           Seien Sie herzlichst gegrüßt von {\\[\baselineskip]}Ihrem {\\[\baselineskip]}\spacefill\mbox{Salten}\pend
           \leftskip=0em{}
         
         \endnumbering\mylabel{h}\end{ledgroupsized}\begin{anhang}\end{anhang}\newcommand{\dateiname}{L03128}\newcommand{\titel}{Felix Salten an Arthur Schnitzler, 18. 8. 1893}\newcommand{\editorInnen}{Martin Anton Müller und Laura Untner}%% latex-leseansicht-abspann.tex
%% Abspann für die Leseansicht.
%% Der Schalter \ifkorrekturansicht ist bereits durch den Vorspann gesetzt.

%% latex-abspann.tex
%% Gemeinsamer Abspann für Korrekturansicht und Leseansicht.
%% Setzt den Schalter \ifkorrekturansicht voraus (gesetzt in den
%% einbindenden Dateien latex-korrekturansicht-abspann.tex bzw.
%% latex-leseansicht-abspann.tex).
%% ---------------------------------------------------------------

\normalsize

% Das esempio-Environment wird nur in der Leseansicht benötigt
\ifkorrekturansicht\else
\newenvironment{esempio}[3]%
{
    \vspace{1.5ex}
    \rlap{\underline{#1}}
    \par
    \setlength{\parindent}{0cm}
    \nopagebreak
    \leftskip=#2cm
    \rightskip=#3cm
}
{
    \par
}
\fi

\doendnotes{C}
\bigskip
\vfill

\clearpage

\footnotesize

\ifkorrekturansicht
  \lohead{\textsc{register}}
\fi

% theindex-Environment neu definieren ohne reledmac
\makeatletter
\renewenvironment{theindex}{%
  \ifkorrekturansicht
    \section*{\indexname}%
  \else
    \subsubsection*{Index der erwähnten Entitäten}%
  \fi
  \setlength{\parindent}{0pt}%
  \setlength{\parskip}{0pt plus 0.3pt}%
  \let\item\@idxitem
}{%
  \ifkorrekturansicht\clearpage\fi
}
\makeatother

\IfFileExists{\jobname-pw.ind}{\input{\jobname-pw.ind}}{}

% Quellenangabe nur in der Leseansicht
\ifkorrekturansicht\else
% Fallback-Definitionen, falls die .tex-Datei \titel etc. nicht gesetzt hat
\providecommand{\titel}{}
\providecommand{\editorInnen}{}
\providecommand{\dateiname}{\jobname}

\vspace{3cm}

\vfill

\footnotesize
\textsc{Quelle}: \titel. Herausgegeben von {\editorInnen}. In: \emph{Arthur Schnitzler: Briefwechsel mit Autorinnen und Autoren}.
 Digitale Edition, https://schnitzler-briefe.acdh.oeaw.ac.at/{\dateiname}.html (Stand \today)
\fi

\end{document}


      