%% latex-korrekturansicht-vorspann.tex
%% Vorspann für die Korrekturansicht.
%% Lädt die gemeinsame Datei latex-vorspann.tex mit gesetztem Schalter.

\newif\ifkorrekturansicht
\korrekturansichttrue

\input{../tex-inputs/latex-vorspann}


\section[Arthur Schnitzler an Richard Beer-Hofmann, 3. 9. 1904]{L01435 Arthur Schnitzler an Richard Beer-Hofmann, 3. 9. 1904}
\nopagebreak\mylabel{L01435v}
\rehead{ }\normalsize\beginnumbering\briefempfaengerindex{Beer-Hofmann, Richard@\textsc{Beer-Hofmann, Richard}!zzzSchnitzler, Arthur@\emph{von Arthur Schnitzler}!1904-09-032@{3. 9. 1904}|(be}
\toendnotes[C]{\smallbreak\pagebreak[2]}\Standort{YCGL, MSS 31.}
\physDesc{Brief, 1 Blatt, 2 Seiten, Umschlag, 683 Zeichen
\newline{}Handschrift: Bleistift, deutsche Kurrent
\newline{}Versand: 1) Stempel: »\nobreak{}\oindex{Bad Ischl@\textbf{Bad Ischl}, \emph{P.PPL}|pwk}Ischl, \textcolor{gray}{4}. 9. 04, 6–8V\nobreak{}«.   2) Stempel: »\nobreak{}\oindex{Bad Aussee@\textbf{Bad Aussee}, \emph{P.PPLA3}|pwk}{\pb}Aussee in
                                       Steiermark, \textcolor{gray}{5} 9 04\nobreak{}«. }
\buchAbdrucke{\weitereDrucke{Arthur Schnitzler, Richard Beer-Hofmann: \emph{Briefwechsel 1891–1931}. Wien, Zürich: \emph{Europaverlag} 1992, S. 166.} }\pstart{}{\pb}\textcolor{gray}{\textbf{Hôtel Kaiserkrone, Bad Ischl\oindex{Hotel Kaiserkrone@\textbf{Hotel Kaiserkrone}, \emph{Hotel (K.HTL)}|pw}.}}\pend{}\pstart{}\textcolor{gray}{\textbf{J. G. Haager jun.}}\pwindex{Haager, Johann Georg 12.01.1866 – 14.12.1926@\textsc{Haager, Johann Georg} (12.01.1866 – 14.12.1926), \emph{Hotelier/Hotelière}|pw}\pend{}{\bigskip}\pstart{}\textsc{Herrn Dr Richard}\pend{}\pstart{}\textsc{Beer-Hofmann}\pend{}\pstart{}\textsc{Markt Aussee\oindex{Bad Aussee@\textbf{Bad Aussee}, \emph{P.PPLA3}|pw}}\pend{}\pstart{}\textsc{Villa Frühling}\oindex{Villa Fruehling@\textbf{Villa Frühling}, \emph{Gebäude (K.GBD)}|pw}\pend{}{\bigskip}\vspace{1em}
\pstart
           {\pb}\textcolor{gray}{\textbf{\textbf{Hotel Kaiserkrone\oindex{Hotel Kaiserkrone@\textbf{Hotel Kaiserkrone}, \emph{Hotel (K.HTL)}|pw}}}}\pend
           
\pstart
           \centering{}\textcolor{gray}{\textbf{\textbf{Bad Ischl\oindex{Bad Ischl@\textbf{Bad Ischl}, \emph{P.PPL}|pw}}}}\pend
           
\pstart
           \textcolor{gray}{\textbf{\textbf{Centrale Lage}}}\pend
           
\pstart
           \textcolor{gray}{\textbf{mit schattigem Restaurationsgarten an dem Ischlflusse\oindex{Ischler Ache@\textbf{Ischler Ache}, \emph{H.STM}|pw} gegenüber der Kaiserlichen Villa\oindex{Kaiservilla@\textbf{Kaiservilla}, \emph{Gebäude (K.GBD)}|pw}}}. \pend
           
\pstart
           \textcolor{gray}{\textbf{Lese-Salon}}\hfill \textcolor{gray}{\textbf{Bad Ischl\oindex{Bad Ischl@\textbf{Bad Ischl}, \emph{P.PPL}|pw}, am}}{ }3. 9. 904\pend
           
\pstart
           \textbf{\textcolor{gray}{\textbf{Badezimmer}}}\pend
           
\pstart
           \textbf{\textcolor{gray}{\textbf{Telephon}}}\pend
           
\pstart
           \textbf{\textcolor{gray}{\textbf{Omnibus am Bahnhofe.}}}\pend
           
\pstart
           \textcolor{gray}{\textbf{Bier vom Fass aus \textbf{A. Dreher’s Brauerei}\orgindex{Anton Drehers Brauereien@Anton Drehers Brauereien|pw} in Schwechat\oindex{Schwechat@\textbf{Schwechat}, \emph{P.PPLA3}|pw}.}}\pend
           \vspace{0.5em}
\pstart
           lieber Richard, vor allem gratulir ich Ihnen herzlich zum
               vollendeten \textsc{Charolais}\pwindex{Graf von Charolais. Ein Trauerspiel@\emph{Der Graf von Charolais. Ein Trauerspiel}|pw}. Ferner: wir fahren Montag nach \textsc{Lueg}\oindex{Lueg@\textbf{Lueg}, \emph{Teil eines besiedelten Ortes (A.BSOX)}|pw} und bleiben dort bis etwa Donnerſtag Früh. Ich beſprach heute eben mit Hugo\pwindex{Hofmannsthal, Hugo von 1874-02-01 – 1929-07-15@\textsc{Hofmannsthal, Hugo von} (1874-02-01 – 1929-07-15), \emph{Schriftsteller/Schriftstellerin}|pw}, wie hübſch das wäre, we{\geminationn} Sie auch herüber kämen. Thuen Sie’s doch jedenfalls.
                  Hugo’s\pwindex{Hofmannsthal, Hugo von 1874-02-01 – 1929-07-15@\textsc{Hofmannsthal, Hugo von} (1874-02-01 – 1929-07-15), \emph{Schriftsteller/Schriftstellerin}|pw}\pwindex{Hofmannsthal, Gertrude von 16.03.1880 – 09.11.1959@\textsc{Hofmannsthal, Gertrude von} (16.03.1880 – 09.11.1959)|pw} fahren
                  Mittwoch Abend nach Salzburg\oindex{Salzburg@\textbf{Salzburg}, \emph{A.ADM2}|pw}; Olga\pwindex{Schnitzler, Olga 17.01.1882 – 13.01.1970@\textsc{Schnitzler, Olga} (17.01.1882 – 13.01.1970), \emph{Schauspieler/Schauspielerin, Sänger/Sängerin}|pw} u ich würden {\pb}dann \introOben{}von \textsc{Lueg} aus\introOben{} mit Ihnen nach
                  Auſſee\oindex{Bad Aussee@\textbf{Bad Aussee}, \emph{P.PPLA3}|pw} fahren, wo wir etwa 2–3 Tage (\textsc{Hotel Elisabeth}\oindex{Bade-Hotel Elisabeth@\textbf{Bade-Hotel Elisabeth}, \emph{Hotel (K.HTL)}|pw} wie man uns räth) wohnen wollen. (Unſer weiteres Progra{\geminationm} iſt dann einige Tage Iſchl\oindex{Bad Ischl@\textbf{Bad Ischl}, \emph{P.PPL}|pw}, einige Tage Salzburg\oindex{Salzburg@\textbf{Salzburg}, \emph{A.ADM2}|pw})\pend
           
\pstart
           Jedenfalls, we{\geminationn} Sie nicht ſelbſt kommen, bitte um ein
               Wort Gaſthof \textsc{Lueg}\oindex{Hotel und Pension Lueg@\textbf{Hotel und Pension Lueg}, \emph{Hotel (K.HTL)}|pw}, bei \textsc{St. Gilgen}\oindex{St. Gilgen@\textbf{St. Gilgen}, \emph{A.ADM3}|pw}.\pend
           
\pstart
           Aber kommen Sie.\pend
           
\pstart
           Herzlichſt Ihr{\\[\baselineskip]}\spacefill\mbox{A.}\pend
           \leftskip=0em{}\selectlanguage{ngerman}\endnumbering\briefempfaengerindex{Beer-Hofmann, Richard@\textsc{Beer-Hofmann, Richard}!zzzSchnitzler, Arthur@\emph{von Arthur Schnitzler}!1904-09-032@{3. 9. 1904}|)be}\mylabel{L01435h}  \normalsize

\doendnotes{C}
\bigskip
\vfill

\clearpage

\footnotesize

\lohead{\textsc{register}}

% Definiere theindex-Environment komplett neu ohne reledmac
\makeatletter
\renewenvironment{theindex}{%
  \section*{\indexname}%
  \setlength{\parindent}{0pt}%
  \setlength{\parskip}{0pt plus 0.3pt}%
  \let\item\@idxitem
}{%
  \clearpage
}
\makeatother

\IfFileExists{\jobname-pw.ind}{\input{\jobname-pw.ind}}{}

\end{document}

      