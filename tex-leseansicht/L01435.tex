%% latex-leseansicht-vorspann.tex
%% Vorspann für die Leseansicht.
%% Lädt die gemeinsame Datei latex-vorspann.tex mit nicht gesetztem Schalter.

\newif\ifkorrekturansicht
\korrekturansichtfalse

\input{../tex-inputs/latex-vorspann}


               \section[Arthur Schnitzler an Richard Beer-Hofmann, 3. 9. 1904]{ Arthur Schnitzler an Richard Beer-Hofmann, 3. 9. 1904}\nopagebreak\mylabel{v}\rehead{ }\begin{ledgroupsized}[t]{13cm}\normalsize\beginnumbering\briefempfaengerindex{Beer-Hofmann, Richard@\textsc{Beer-Hofmann, Richard}!zzzSchnitzler, Arthur@\emph{von Arthur Schnitzler}!1904-09-032@{3. 9. 1904}|(be} \toendnotes[C]{\smallbreak\pagebreak[2]} \Standort{YCGL, MSS 31.}
\physDesc{Brief, 1 Blatt, 2 Seiten, Umschlag
\newline{}Handschrift: Bleistift, deutsche Kurrent\newline{}Versand: 1) Stempel: »\nobreak{}\oindex{Bad Ischl@\textbf{Bad Ischl}|pwk}Ischl, \textcolor{gray}{4}. 9. 04, 6–8V\nobreak{}«.  2) Stempel: »\nobreak{}\oindex{Bad Aussee@\textbf{Bad Aussee}|pwk}{\pb}Aussee in
                              Steiermark, \textcolor{gray}{5} 9 04\nobreak{}«. }\buchAbdrucke{\weitereDrucke{Arthur Schnitzler, Richard Beer-Hofmann: \emph{Briefwechsel 1891–1931}. Hg. Konstanze Fliedl. Wien, Zürich: \emph{Europaverlag} 1992, S. 166.} }\pstart{}{\pb}\textcolor{gray}{\textbf{Hôtel Kaiserkrone, Bad Ischl\oindex{Hotel Kaiserkrone@\textbf{Hotel Kaiserkrone}|pw}.}}\pend{}\pstart{}\textcolor{gray}{\textbf{J. G. Haager jun.}}\pwindex{Haager, Johann Georg 12.01.1866 – 14.12.1926@\textsc{Haager, Johann Georg} (12.01.1866 – 14.12.1926), \emph{Hotelier/Hotelière}|pw}\pend{}{\bigskip}\pstart{}\textsc{Herrn Dr Richard }\pend{}\pstart{}\textsc{Beer-Hofmann}\pend{}\pstart{}\textsc{Markt Aussee\oindex{Bad Aussee@\textbf{Bad Aussee}|pw}}\pend{}\pstart{}\textsc{Villa Frühling}\oindex{Villa Fruehling@\textbf{Villa Frühling}|pw}\pend{}{\bigskip}\pstart
           \noindent{}{\pb}\textcolor{gray}{\textbf{\textbf{Hotel Kaiserkrone\oindex{Hotel Kaiserkrone@\textbf{Hotel Kaiserkrone}|pw}}}}\pend
           \pstart
           \centering{}\textcolor{gray}{\textbf{\textbf{Bad Ischl\oindex{Bad Ischl@\textbf{Bad Ischl}|pw}}}}\pend
           \pstart
           \noindent{}\textcolor{gray}{\textbf{\textbf{Centrale Lage}}}\pend
           \pstart
           \textcolor{gray}{\textbf{mit schattigem Restaurationsgarten an dem Ischlflusse gegenüber der Kaiserlichen Villa\oindex{Kaiservilla@\textbf{Kaiservilla}|pw}}}. \pend
           \pstart
           \textcolor{gray}{\textbf{Lese-Salon}}\hfill \textcolor{gray}{\textbf{Bad Ischl\oindex{Bad Ischl@\textbf{Bad Ischl}|pw}, am}}{ }3. 9. 904\pend
           \pstart
           \textbf{\textcolor{gray}{\textbf{Badezimmer}}}\pend
           \pstart
           \textbf{\textcolor{gray}{\textbf{Telephon}}}\pend
           \pstart
           \textbf{\textcolor{gray}{\textbf{Omnibus am Bahnhofe.}}}\pend
           \pstart
           \textcolor{gray}{\textbf{Bier vom Fass aus \textbf{A. Dreher’s Brauerei}\orgindex{Anton Drehers Brauereien@Anton Drehers Brauereien|pw} in Schwechat\oindex{Schwechat@\textbf{Schwechat}|pw}.}}\pend
           \pstart
           lieber Richard, vor allem gratulir ich Ihnen herzlich zum
               vollendeten \textsc{Charolais}\pwindex{Beer-Hofmann, Richard 11.07.1866 – 26.09.1945@\textsc{Beer-Hofmann, Richard} (11.07.1866 – 26.09.1945), \emph{Schriftsteller}!Graf von Charolais. Ein Trauerspiel1904-12-23 – 1904-12-23@\strich\emph{Der Graf von Charolais. Ein Trauerspiel} {[}1904-12-23 – 1904-12-23{]}|pw}. Ferner: wir fahren Montag nach \textsc{Lueg}\oindex{Lueg am Wolfgangsee@\textbf{Lueg am Wolfgangsee}|pw} und bleiben dort bis etwa Donnerſtag Früh. Ich beſprach heute eben mit Hugo\pwindex{Hofmannsthal, Hugo von 01.02.1874 – 15.07.1929@\textsc{Hofmannsthal, Hugo von} (01.02.1874 – 15.07.1929), \emph{Schriftsteller}|pw}, wie hübſch das wäre, we{\geminationn} Sie auch herüber kämen. Thuen Sie’s doch jedenfalls.
               Hugo’s\pwindex{Hofmannsthal, Hugo von 01.02.1874 – 15.07.1929@\textsc{Hofmannsthal, Hugo von} (01.02.1874 – 15.07.1929), \emph{Schriftsteller}|pw}\pwindex{Hofmannsthal, Gertrude von 16.03.1880 – 09.11.1959@\textsc{Hofmannsthal, Gertrude von} (16.03.1880 – 09.11.1959)|pw} fahren Mittwoch Abend nach
                  Salzburg\oindex{Salzburg@\textbf{Salzburg}|pw}; Olga\pwindex{Schnitzler, Olga 17.01.1882 – 13.01.1970@\textsc{Schnitzler, Olga} (17.01.1882 – 13.01.1970), \emph{Schauspielerin, Sängerin}|pw}
               u ich würden {\pb}dann \introOben{}von \textsc{Lueg} aus\introOben{} mit Ihnen nach Auſſee\oindex{Bad Aussee@\textbf{Bad Aussee}|pw} fahren, wo wir etwa 2–3 Tage (\textsc{Hotel Elisabeth}\oindex{Bade-Hotel Elisabeth@\textbf{Bade-Hotel Elisabeth}|pw} wie man uns räth) wohnen wollen. (Unſer weiteres Progra{\geminationm} iſt dann einige Tage Iſchl\oindex{Bad Ischl@\textbf{Bad Ischl}|pw}, einige Tage Salzburg\oindex{Salzburg@\textbf{Salzburg}|pw})\pend
           \pstart
           Jedenfalls, we{\geminationn} Sie nicht ſelbſt kommen, bitte um ein
               Wort Gaſthof \textsc{Lueg}\oindex{Hotel und Pension Lueg@\textbf{Hotel und Pension Lueg}|pw}, bei \textsc{St. Gilgen}\oindex{St. Gilgen@\textbf{St. Gilgen}|pw}.\pend
           \pstart
           Aber kommen Sie.\pend
           \pstart
           Herzlichſt Ihr{\\[\baselineskip]}\spacefill\mbox{A.}\pend
           \leftskip=0em{}\endnumbering\briefempfaengerindex{Beer-Hofmann, Richard@\textsc{Beer-Hofmann, Richard}!zzzSchnitzler, Arthur@\emph{von Arthur Schnitzler}!1904-09-032@{3. 9. 1904}|)be}\mylabel{h}\end{ledgroupsized}  \newcommand{\dateiname}{L01435}\newcommand{\titel}{Arthur Schnitzler an Richard Beer-Hofmann, 3. 9. 1904}\newcommand{\editorInnen}{Martin Anton Müller und Gerd-Hermann Susen}%% latex-leseansicht-abspann.tex
%% Abspann für die Leseansicht.
%% Der Schalter \ifkorrekturansicht ist bereits durch den Vorspann gesetzt.

%% latex-abspann.tex
%% Gemeinsamer Abspann für Korrekturansicht und Leseansicht.
%% Setzt den Schalter \ifkorrekturansicht voraus (gesetzt in den
%% einbindenden Dateien latex-korrekturansicht-abspann.tex bzw.
%% latex-leseansicht-abspann.tex).
%% ---------------------------------------------------------------

\normalsize

% Das esempio-Environment wird nur in der Leseansicht benötigt
\ifkorrekturansicht\else
\newenvironment{esempio}[3]%
{
    \vspace{1.5ex}
    \rlap{\underline{#1}}
    \par
    \setlength{\parindent}{0cm}
    \nopagebreak
    \leftskip=#2cm
    \rightskip=#3cm
}
{
    \par
}
\fi

\doendnotes{C}
\bigskip
\vfill

\clearpage

\footnotesize

\ifkorrekturansicht
  \lohead{\textsc{register}}
\fi

% theindex-Environment neu definieren ohne reledmac
\makeatletter
\renewenvironment{theindex}{%
  \ifkorrekturansicht
    \section*{\indexname}%
  \else
    \subsubsection*{Index der erwähnten Entitäten}%
  \fi
  \setlength{\parindent}{0pt}%
  \setlength{\parskip}{0pt plus 0.3pt}%
  \let\item\@idxitem
}{%
  \ifkorrekturansicht\clearpage\fi
}
\makeatother

\IfFileExists{\jobname-pw.ind}{\input{\jobname-pw.ind}}{}

% Quellenangabe nur in der Leseansicht
\ifkorrekturansicht\else
% Fallback-Definitionen, falls die .tex-Datei \titel etc. nicht gesetzt hat
\providecommand{\titel}{}
\providecommand{\editorInnen}{}
\providecommand{\dateiname}{\jobname}

\vspace{3cm}

\vfill

\footnotesize
\textsc{Quelle}: \titel. Herausgegeben von {\editorInnen}. In: \emph{Arthur Schnitzler: Briefwechsel mit Autorinnen und Autoren}.
 Digitale Edition, https://schnitzler-briefe.acdh.oeaw.ac.at/{\dateiname}.html (Stand \today)
\fi

\end{document}


      