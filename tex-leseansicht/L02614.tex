%% latex-leseansicht-vorspann.tex
%% Vorspann für die Leseansicht.
%% Lädt die gemeinsame Datei latex-vorspann.tex mit nicht gesetztem Schalter.

\newif\ifkorrekturansicht
\korrekturansichtfalse

\input{../tex-inputs/latex-vorspann}


\section[Paul Goldmann an Arthur Schnitzler, 21. 9. [1894]]{L02614 Paul Goldmann an Arthur Schnitzler, 21. 9. [1894]}
\nopagebreak\mylabel{L02614v}
\rehead{ }\normalsize\beginnumbering\briefempfaengerindex{Schnitzler, Arthur@\textsc{Schnitzler, Arthur}!zzzGoldmann, Paul@\emph{von Paul Goldmann}!1894-09-211@{21. 9. [1894]}|(be}
\toendnotes[C]{\smallbreak\pagebreak[2]}
\correspDesc{Versand  durch Paul Goldmann am 21. 9. [1894] in Paris
\newline{}Erhalt  durch Arthur Schnitzler im Zeitraum [22. 9. 1894
                  – 26. 9. 1894?] in Wien}\toendnotes[C]{\smallbreak}
\Standort{DLA, A:Schnitzler, HS.NZ85.1.3164.}
\physDesc{Brief, 2 Blätter, 7 Seiten, 2773 Zeichen
\newline{}Handschrift: schwarze Tinte, deutsche Kurrent
\newline{}Schnitzler: 1) mit Bleistift auf dem ersten Blatt die Jahreszahl »94« vermerkt  2) mit rotem Buntstift drei Unterstreichungen}\toendnotes[C]{\smallbreak}
\pstart
           {\pb}\textcolor{gray}{\textbf{\textbf{Frankfurter Zeitung\orgindex{Frankfurter Zeitung@Frankfurter Zeitung|pw}}}}\pend
           
\pstart
           \textcolor{gray}{\textbf{(Gazette de
                        Francfort\orgindex{Frankfurter Zeitung@Frankfurter Zeitung|pw}).}}\hfill \textsc{Paris\oindex{Paris@\textbf{Paris}, \emph{Hauptstadt}|pw}}, 21. September.\pend
           
\pstart
           \textcolor{gray}{\textbf{\textbf{\begin{otherlanguage}{french}Fondateur\end{otherlanguage} M. L. Sonnemann\pwindex{Sonnemann, Leopold 29.\,10.\,1831 Höchberg – 30.\,10.\,1909 Frankfurt am Main@\textsc{Sonnemann, Leopold} (29.\,10.\,1831 Höchberg – 30.\,10.\,1909 Frankfurt am Main), \emph{Journalist, Herausgeber}|pw}.}}}\pend
           
\pstart
           \textcolor{gray}{\textbf{\begin{otherlanguage}{french}Journal politique, financier,\end{otherlanguage}}}\pend
           
\pstart
           \textcolor{gray}{\textbf{\begin{otherlanguage}{french}commercial et littéraire.\end{otherlanguage}}}\pend
           
\pstart
           \textcolor{gray}{\textbf{\begin{otherlanguage}{french}\textbf{Paraissant trois fois par jour}\end{otherlanguage}}}.\pend
           
\pstart
           \textcolor{gray}{\textbf{\begin{otherlanguage}{french}\textbf{Bureaux à Paris\oindex{Paris@\textbf{Paris}, \emph{Hauptstadt}|pw}:}\end{otherlanguage}}}\pend
           
\pstart
           \textcolor{gray}{\textbf{\begin{otherlanguage}{french}\textbf{24. Rue Feydeau}\oindex{rue Feydeau@\textbf{rue Feydeau}, \emph{Straße}|pw}.\end{otherlanguage}}}\pend
           
\pstart\center{}Mein lieber Freund,\pend\vspace{0.5em}
\pstart
           Ich bin dieſer Tage nach \textsc{Paris\oindex{Paris@\textbf{Paris}, \emph{Hauptstadt}|pw}} zurückgekehrt. Die Frankfurt\oindex{Frankfurt am Main@\textbf{Frankfurt am Main}, \emph{Hauptstadt}|pw}er Zeit war
               auch recht{ }ſchön. Die Meinigen haben gewetteifert, mir den Aufenthalt angenehm zu
                  machen\strikeout{,} und \strikeout{mich}
               mir das Heimathsgefühl zu geben. Sie laſſen Dich Alle vielmals grüßen. Mein Onkel\pwindex{Mamroth, Fedor 21.\,2.\,1851 Breslau – 25.\,6.\,1907 Frankfurt am Main@\textsc{Mamroth, Fedor} (21.\,2.\,1851 Breslau – 25.\,6.\,1907 Frankfurt am Main), \emph{Journalist, Kritiker}|pwv} iſt dieſer Tage auf
               Urlaub gegangen. Wenn er zurückkommt, wirſt Du die erſten \label{K_L02614-1v}\edtext{Bücher zur Beſprechung}{\lemma{\textnormal{\emph{Bücher zur Besprechung}}}\Cendnote{\textnormal{Siehe XXXX Auszeichnungsfehler: Dokument L02612 nicht gefunden.
               }}}\label{K_L02614-1} erhalten. Thu’ mir den einzigen Gefallen und{ }ſtell’ Dir die Sache nicht {\pb}ſo{ }ſchwer vor. Was Dich erſchreckt, iſt lediglich
               eine mechaniſche Schwierigkeit. Man trainirt{ }ſich zum Bücherbeſprechen, wie zu jedem
               andern Ding. Es handelt sich nur darum,{ }ſich mit der nöthigen Sicherheit zum
               Schreibtiſch zu{ }ſetzen und anzufangen. Der Stoff erſcheint Anfangs nicht zu
               bewältigen, aber im Schreiben tritt das Weſentliche \substVorne{}\textsuperscript{klar}\substDazwischen{}klar\substHinten{} hervor und das übrige{ }ſällt ab. Du{ }ſollſt ja auch nur \strikeout{d} über die Bücher referiren und nicht ein
               gerichtsordnungsmäßiges Protocoll {\pb}davon geben.
               Deine \label{K_L02614-2v}\edtext{Pſeudonymitäts-Wünſche}{\lemma{\textnormal{\emph{Pseudonymitäts-Wünsche}}}\Cendnote{\textnormal{Obzwar nicht undenkbar, wurden bislang
                  keine Hinweise gefunden, dass Schnitzler auf
                  diese Weise Texte unter Pseudonym veröffentlicht hätte. Vor allem geht auch die
                  Korrespondenz mit Goldmann\pwindex{Goldmann, Paul 31.\,1.\,1865 Breslau – 25.\,9.\,1935 Wien@\textsc{Goldmann, Paul} (31.\,1.\,1865 Breslau – 25.\,9.\,1935 Wien), \emph{Schriftsteller, Journalist}|pwk} nicht auf solche
                  Texte ein.}}}\label{K_L02614-2} wirſt Du meinem Onkel\pwindex{Mamroth, Fedor 21.\,2.\,1851 Breslau – 25.\,6.\,1907 Frankfurt am Main@\textsc{Mamroth, Fedor} (21.\,2.\,1851 Breslau – 25.\,6.\,1907 Frankfurt am Main), \emph{Journalist, Kritiker}|pwv} bei Überſendung des erſten Feuilletons mittheilen.
               Ich habe{ }ſie ihm bisher \strikeout{\textcolor{gray}{m}} verſchwiegen, weil ich nicht wollte, daß er Dich jetzt{ }ſchon zögern{ }ſehe.\pend
           
\pstart
           Die \textsc{20 fl.} haben bei der Einwechſelung \textsc{40 fr. 40 ct} ergeben. Das Abonnement auf das »\textsc{Journal\orgindex{Le Journal@Le Journal|pw}}« hat \textsc{10 fr.} gekoſtet. Du haſt alſo \textsc{30 fr. 40 ct.} bei mir gut, und ich{ }ſehe Deinen Aufträgen
               entgegen. Dein Abonnement läuft vom 1. \textsc{Oct}. Ich habe aber gebeten, daß {\pb}Du das Blatt\pwindex{Le Journal@\emph{Le Journal}|pwv} bereits von Montag ab erhältſt. \strikeout{Theile} Theile mir mit, ob die Zuſendung regelmäßig erfolgt.\pend
           
\pstart
           Gestern ist \textsc{Herzl\pwindex{Herzl, Theodor 2.\,5.\,1860 Budapest – 3.\,7.\,1904 Edlach@\textsc{Herzl, Theodor} (2.\,5.\,1860 Budapest – 3.\,7.\,1904 Edlach), \emph{Schriftsteller, Journalist}|pw}}{ }\label{K_L02614-3v}\edtext{zurückgekommen}{\lemma{\textnormal{\emph{zurückgekommen}}}\Cendnote{\textnormal{Theodor Herzl\pwindex{Herzl, Theodor 2.\,5.\,1860 Budapest – 3.\,7.\,1904 Edlach@\textsc{Herzl, Theodor} (2.\,5.\,1860 Budapest – 3.\,7.\,1904 Edlach), \emph{Schriftsteller, Journalist}|pwk} war auch in Ischl\oindex{Bad Ischl@\textbf{Bad Ischl}|pwk} gewesen, vgl. A. S.: \emph{Tagebuch}, 31. 8. 1894.}}}\label{K_L02614-3}. Er war bei mir und hat mir erzählt, er
               habe{ }ſich insbeſondere mit \textsc{Burckhardt\pwindex{Burckhard, Max Eugen 14.\,7.\,1854 Korneuburg – 16.\,3.\,1912 Wien@\textsc{Burckhard, Max Eugen} (14.\,7.\,1854 Korneuburg – 16.\,3.\,1912 Wien), \emph{Schriftsteller, Rechtswissenschaftler, Theaterleiter}|pw}} angefreundet. Dieſen habe er vor Allem auf Dich aufmerkſam gemacht. \textsc{B.\pwindex{Burckhard, Max Eugen 14.\,7.\,1854 Korneuburg – 16.\,3.\,1912 Wien@\textsc{Burckhard, Max Eugen} (14.\,7.\,1854 Korneuburg – 16.\,3.\,1912 Wien), \emph{Schriftsteller, Rechtswissenschaftler, Theaterleiter}|pw}}{ }ſcheine{ }ſehr geneigt, Dich zu \label{K_L02614-4v}\edtext{ſpielen}{\lemma{\textnormal{\emph{spielen}}}\Cendnote{\textnormal{Diese Aussage ist bedeutsam,
                  da sie besagt, dass Burckhard\pwindex{Burckhard, Max Eugen 14.\,7.\,1854 Korneuburg – 16.\,3.\,1912 Wien@\textsc{Burckhard, Max Eugen} (14.\,7.\,1854 Korneuburg – 16.\,3.\,1912 Wien), \emph{Schriftsteller, Rechtswissenschaftler, Theaterleiter}|pwk} bereits
                  Willens war, Schnitzler aufzuführen, noch
                  bevor er \emph{Liebelei}\pwindex{Schnitzler, Arthur 15.\,5.\,1862 Wien – 21.\,10.\,1931 ebd.@\textsc{Schnitzler, Arthur} (15.\,5.\,1862 Wien – 21.\,10.\,1931 ebd.), \emph{Schriftsteller, Mediziner}!Liebelei. Schauspiel in drei Akten@\strich\emph{Liebelei. Schauspiel in drei Akten}|pwk} kannte.}}}\label{K_L02614-4},{ }ſobald
               Du nur irgend etwas Burgtheater\orgindex{Burgtheater@Burgtheater|pw}mäßiges hätteſt.
               Inzwiſchen habe \textsc{Herzl\pwindex{Herzl, Theodor 2.\,5.\,1860 Budapest – 3.\,7.\,1904 Edlach@\textsc{Herzl, Theodor} (2.\,5.\,1860 Budapest – 3.\,7.\,1904 Edlach), \emph{Schriftsteller, Journalist}|pw}} gerathen, Dir \label{K_L02614-5v}\edtext{Bearbeitungen {\pb}aus dem Franzöſiſchen}{\lemma{\textnormal{\emph{Bearbeitungen … Französischen}}}\Cendnote{\textnormal{Der Plan bestand länger, vgl. A. S.: \emph{Tagebuch}, 8. 9. 1894 und XXXX Auszeichnungsfehler: Dokument L02792 nicht gefunden. Schnitzler hat keine
                  Übersetzungen und Bühnenbearbeitungen fremder Stücke erstellt.}}}\label{K_L02614-5} zu
               übertragen. \textsc{B.\pwindex{Burckhard, Max Eugen 14.\,7.\,1854 Korneuburg – 16.\,3.\,1912 Wien@\textsc{Burckhard, Max Eugen} (14.\,7.\,1854 Korneuburg – 16.\,3.\,1912 Wien), \emph{Schriftsteller, Rechtswissenschaftler, Theaterleiter}|pw}} werde Dich vielleicht den \textsc{Marivaux\pwindex{Marivaux, Pierre Carlet de 4.\,2.\,1688 Paris – 12.\,2.\,1763 ebd.@\textsc{Marivaux, Pierre Carlet de} (4.\,2.\,1688 Paris – 12.\,2.\,1763 ebd.), \emph{Schriftsteller}|pw}} überſetzen laſſen \textsc{etc.}{ }\textsc{Herzl\pwindex{Herzl, Theodor 2.\,5.\,1860 Budapest – 3.\,7.\,1904 Edlach@\textsc{Herzl, Theodor} (2.\,5.\,1860 Budapest – 3.\,7.\,1904 Edlach), \emph{Schriftsteller, Journalist}|pw}}{ }ſelbſt will ein \label{K_L02614-6v}\edtext{dreiaktiges Luſtſpiel\pwindex{Herzl, Theodor 2.\,5.\,1860 Budapest – 3.\,7.\,1904 Edlach@\textsc{Herzl, Theodor} (2.\,5.\,1860 Budapest – 3.\,7.\,1904 Edlach), \emph{Schriftsteller, Journalist}!Unser Käthchen. Lustspiel in 4 Acten@\strich\emph{Unser Käthchen. Lustspiel in 4 Acten}|pwuv}}{\lemma{\textnormal{\emph{dreiaktiges Lustspiel}}}\Cendnote{\textnormal{Das Lustspiel konnte nicht identifiziert werden. Eventuell könnte
                  das 1898 fertiggestellte Lustspiel \emph{Unser Käthchen}\pwindex{Herzl, Theodor 2.\,5.\,1860 Budapest – 3.\,7.\,1904 Edlach@\textsc{Herzl, Theodor} (2.\,5.\,1860 Budapest – 3.\,7.\,1904 Edlach), \emph{Schriftsteller, Journalist}!Unser Käthchen. Lustspiel in 4 Acten@\strich\emph{Unser Käthchen. Lustspiel in 4 Acten}|pwk} gemeint gewesen sein, an dem Herzl\pwindex{Herzl, Theodor 2.\,5.\,1860 Budapest – 3.\,7.\,1904 Edlach@\textsc{Herzl, Theodor} (2.\,5.\,1860 Budapest – 3.\,7.\,1904 Edlach), \emph{Schriftsteller, Journalist}|pwk}{ }1891 zu arbeiten begonnen hatte}}}\label{K_L02614-6}{ }ſchreiben, von dem er bereits
               zwei Akte liegen hat.\pend
           
\pstart
           Und was machſt Du? Geht das Stück\pwindex{Schnitzler, Arthur 15.\,5.\,1862 Wien – 21.\,10.\,1931 ebd.@\textsc{Schnitzler, Arthur} (15.\,5.\,1862 Wien – 21.\,10.\,1931 ebd.), \emph{Schriftsteller, Mediziner}!Liebelei. Schauspiel in drei Akten@\strich\emph{Liebelei. Schauspiel in drei Akten}|pwv} vorwärts? Fühlſt Du Dich wohl in Wien\oindex{Wien@\textbf{Wien}, \emph{Verwaltungsgebiet}|pw}? Iſt \textsc{Richard\pwindex{Beer-Hofmann, Richard 11.\,7.\,1866 Wien – 26.\,9.\,1945 New York City@\textsc{Beer-Hofmann, Richard} (11.\,7.\,1866 Wien – 26.\,9.\,1945 New York City), \emph{Schriftsteller}|pw}} abgereiſt und wohin? Was hört man von der neuen \textsc{Revue\orgindex{Zeit. Wiener Wochenschrift@Die Zeit. Wiener Wochenschrift|pwv}}?\pend
           
\pstart
           {\pb}Ich freue mich darauf, bald einen Brief von Dir zu
               erhalten. Bin{ }ſonſt recht lebensmüde. Ich{ }ſehe, daß ich auf einem falſchen Wege bin,
               daß ich nicht mehr hierher zurückkehren durſte. Die Arbeit iſt mir zuwider. Ich
               möchte gern nachkommen und kann keinen Schritt thun. So{ }ſühle ich mich zurückbeiben.
               Und da mir dies das Herz zerreißt,{ }ſo glaube ich, daß das unmöglich ein normales Ende
               nehmen kann.\pend
           
\pstart
           {\pb}Sei von Herzen gegrüßt, mein lieber Arthur. Es war{ }ſo{ }ſchön bei \label{K_L02614-7v}\edtext{Euch}{\lemma{\textnormal{\emph{Euch}}}\Cendnote{\textnormal{im Urlaub in Bad Ischl\oindex{Bad Ischl@\textbf{Bad Ischl}|pwk}}}}\label{K_L02614-7}, und es iſt gar{ }ſchwer, nach alledem wieder in \textsc{Paris\oindex{Paris@\textbf{Paris}, \emph{Hauptstadt}|pw}} zu leben.\pend
           
\pstart
           In Treue{\\[\baselineskip]} Dein{\\[\baselineskip]}\spacefill\mbox{Paul Goldmann.}\pend
           \leftskip=0em{}
\pstart
           \noindent{}Bitte, empfiehl’ mich dem Fräulein \textsc{Sandrock\pwindex{Sandrock, Adele 19.\,8.\,1863 Rotterdam – 30.\,8.\,1937 Berlin@\textsc{Sandrock, Adele} (19.\,8.\,1863 Rotterdam – 30.\,8.\,1937 Berlin), \emph{Schauspielerin}|pw}}, wenn Du dazu einmal Gelegenheit haſt, und \strikeout{z\textcolor{gray}{war}} zwar recht herzlich.\pend
           \selectlanguage{ngerman}\endnumbering\briefempfaengerindex{Schnitzler, Arthur@\textsc{Schnitzler, Arthur}!zzzGoldmann, Paul@\emph{von Paul Goldmann}!1894-09-211@{21. 9. [1894]}|)be}\mylabel{L02614h}  \newcommand{\dateiname}{L02614}\newcommand{\titel}{Paul Goldmann an Arthur Schnitzler, 21. 9. [1894]}\newcommand{\editorInnen}{Martin Anton Müller und Laura Untner}%% latex-leseansicht-abspann.tex
%% Abspann für die Leseansicht.
%% Der Schalter \ifkorrekturansicht ist bereits durch den Vorspann gesetzt.

%% latex-abspann.tex
%% Gemeinsamer Abspann für Korrekturansicht und Leseansicht.
%% Setzt den Schalter \ifkorrekturansicht voraus (gesetzt in den
%% einbindenden Dateien latex-korrekturansicht-abspann.tex bzw.
%% latex-leseansicht-abspann.tex).
%% ---------------------------------------------------------------

\normalsize

% Das esempio-Environment wird nur in der Leseansicht benötigt
\ifkorrekturansicht\else
\newenvironment{esempio}[3]%
{
    \vspace{1.5ex}
    \rlap{\underline{#1}}
    \par
    \setlength{\parindent}{0cm}
    \nopagebreak
    \leftskip=#2cm
    \rightskip=#3cm
}
{
    \par
}
\fi

\doendnotes{C}
\bigskip
\vfill

\clearpage

\footnotesize

\ifkorrekturansicht
  \lohead{\textsc{register}}
\fi

% theindex-Environment neu definieren ohne reledmac
\makeatletter
\renewenvironment{theindex}{%
  \ifkorrekturansicht
    \section*{\indexname}%
  \else
    \subsubsection*{Index der erwähnten Entitäten}%
  \fi
  \setlength{\parindent}{0pt}%
  \setlength{\parskip}{0pt plus 0.3pt}%
  \let\item\@idxitem
}{%
  \ifkorrekturansicht\clearpage\fi
}
\makeatother

\IfFileExists{\jobname-pw.ind}{\input{\jobname-pw.ind}}{}

% Quellenangabe nur in der Leseansicht
\ifkorrekturansicht\else
% Fallback-Definitionen, falls die .tex-Datei \titel etc. nicht gesetzt hat
\providecommand{\titel}{}
\providecommand{\editorInnen}{}
\providecommand{\dateiname}{\jobname}

\vspace{3cm}

\vfill

\footnotesize
\textsc{Quelle}: \titel. Herausgegeben von {\editorInnen}. In: \emph{Arthur Schnitzler: Briefwechsel mit Autorinnen und Autoren}.
 Digitale Edition, https://schnitzler-briefe.acdh.oeaw.ac.at/{\dateiname}.html (Stand \today)
\fi

\end{document}


