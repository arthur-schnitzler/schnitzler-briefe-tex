%% latex-leseansicht-vorspann.tex
%% Vorspann für die Leseansicht.
%% Lädt die gemeinsame Datei latex-vorspann.tex mit nicht gesetztem Schalter.

\newif\ifkorrekturansicht
\korrekturansichtfalse

\input{../tex-inputs/latex-vorspann}


\section[Arthur Schnitzler an Romain Rolland, 14. 12. 1914]{L03884 Arthur Schnitzler an Romain Rolland, 14. 12. 1914}
\nopagebreak\mylabel{L03884v}
\rehead{ }\normalsize\beginnumbering\briefempfaengerindex{Rolland, Romain@\textsc{Rolland, Romain}!zzzSchnitzler, Arthur@\emph{von Arthur Schnitzler}!1914-12-141@{14. 12. 1914}|(be}
\toendnotes[C]{\smallbreak\pagebreak[2]}
\correspDesc{Versand  durch Arthur Schnitzler am 14. 12. 1914 in Wien
\newline{}Erhalt  durch Romain Rolland am 16. 12. 1914 in Genf}\toendnotes[C]{\smallbreak}
\Standort{Paris, Bibliothèque Nationale de France, Fonds Romain Rolland, Cote NAF 28400.}
\physDesc{Brief, 2 Blätter, 2 Seiten, Kuvert, 2352 Zeichen
\newline{}Schreibmaschine}\Standort{DLA, A:Schnitzler, 85.1.1714.}
\physDesc{BriefDurchschlag, , 2352 Zeichen
\newline{}Schreibmaschine
\newline{}Handschrift: 1) schwarze Tinte, lateinische Kurrent (\noindent{}Ergänzung eines Buchstabens, Unterschrift)\hspace{1em}2) Bleistift (\noindent{}Ergänzung dreier Beistriche, zweier Buchstaben und einer Umstellung)\hspace{1em}
\newline{}Versand: 1) Einschreiben  2) Stempel: »\nobreak{}\oindex{XXXX Ortsangabe fehlt|pwk}9/3 Wien 72, 14. XII. 14\nobreak{}«.  3) Stempel: »\nobreak{}Überprüft\nobreak{}«.  4) Stempel: »\nobreak{}\oindex{Genf@\textbf{Genf}|pwk}Genève, 16. XII. 14, 12\nobreak{}«. 
\newline{}Rolland: mit schwarzer Tinte Datierung: »14/12/1914« und Vermerk: »\uline{ARL}« 
\newline{}Ordnung: 1) mit Bleistift Kuvert nummeriert: »1«  2) mit Bleistift Blätter (einschliesslich des Kuverts) paginiert: »1« – »3«}
\buchAbdrucke{\weitereDrucke{Arthur Schnitzler: \emph{Briefe 1913–1931}. Herausgegeben von Peter Michael Braunwarth, Richard Miklin, Susanne Pertlik und Heinrich Schnitzler. Frankfurt am Main: \emph{S. Fischer} 1984, S. 63–64.} }\toendnotes[C]{\smallbreak}\pstart{}{\pb}\textcolor{gray}{\textbf{Dr. Arthur Schnitzler}}\pend{}\pstart{}\textcolor{gray}{\textbf{Wien XVIII. Sternwartestrasse 71\oindex{Wien@\textbf{Wien}!XVIII., Währing@\textbf{XVIII., Währing}!Sternwartestraße 71@\textbf{Sternwartestraße 71}, \emph{Wohngebäude}|pw}}}\pend{}{\bigskip}\pstart{}{\pb}Herrn Romain Rolland\pend{}\pstart{}Genf.\oindex{Genf@\textbf{Genf}|pw}\pend{}\pstart{}Hotel Beau Séjour\oindex{XXXX Ortsangabe fehlt|pw}.\pend{}\pstart{}Schweiz\oindex{Schweiz@\textbf{Schweiz}|pw}.\pend{}{\bigskip}\vspace{1em}
\pstart
           {\pb}\textcolor{gray}{\textbf{Dr. Arthur Schnitzler}}\hfill 14. 12. 1914.\pend
           
\pstart
           \textcolor{gray}{\textbf{Wien XVIII. Sternwartestrasse 71\oindex{Wien@\textbf{Wien}!XVIII., Währing@\textbf{XVIII., Währing}!Sternwartestraße 71@\textbf{Sternwartestraße 71}, \emph{Wohngebäude}|pw}}}\pend
           
\pstart\center{}Verehrter Herr Rolland.\pend\vspace{0.5em}
\pstart
           Sie wollen also wirklich, wie mir Stefan Zweig\pwindex{Zweig, Stefan 28.\,11.\,1881 Wien – 23.\,2.\,1942 Petrópolis@\textsc{Zweig, Stefan} (28.\,11.\,1881 Wien – 23.\,2.\,1942 Petrópolis), \emph{Schriftsteller}|pw} sagt,
               die grosse Freundlichkeit haben meine Erklärung\pwindex{Schnitzler, Arthur 15. 5. 1862 Wien – 21. 10. 1931 ebd.@\textsc{Schnitzler, Arthur} (15. 5. 1862 Wien – 21. 10. 1931 ebd.), \emph{Schriftsteller, Mediziner}!Brief Artur Schnitzlers@\strich\emph{Ein Brief Artur Schnitzlers}|pwv} ins Französische zu übersetzen und wünschen überdies, zum Zweck
               der Veröffentlichung in einer deutschen Schweizer\oindex{Schweiz@\textbf{Schweiz}|pw}
               Zeitung\introOben{},\introOben{} ein zweites Exemplar\pwindex{Schnitzler, Arthur 15. 5. 1862 Wien – 21. 10. 1931 ebd.@\textsc{Schnitzler, Arthur} (15. 5. 1862 Wien – 21. 10. 1931 ebd.), \emph{Schriftsteller, Mediziner}!Brief Artur Schnitzlers@\strich\emph{Ein Brief Artur Schnitzlers}|pwv}, das ich
               Ihnen hiemit gerne und mit vielem Dank für Ihre besondere Liebenswürdigkeit zusende.
               Auch mir ist bisher nicht bekannt geworden, dass jener russische Artikel\pwindex{?? [Journalist, der fiktives russisches Interview verantwortet] @\textsc{?? [Journalist, der fiktives russisches Interview verantwortet]}!?? [Fiktives Interview aus St. Petersburg, 1914]@\strich\emph{?? [Fiktives Interview aus St. Petersburg, 1914]}|pwv} den Weg nach anderen Ländern
               gefunden hätte. Die Existenz jenes Artikels\pwindex{?? [Journalist, der fiktives russisches Interview verantwortet] @\textsc{?? [Journalist, der fiktives russisches Interview verantwortet]}!?? [Fiktives Interview aus St. Petersburg, 1914]@\strich\emph{?? [Fiktives Interview aus St. Petersburg, 1914]}|pwv} oder erdichteten Interviews\pwindex{?? [Journalist, der fiktives russisches Interview verantwortet] @\textsc{?? [Journalist, der fiktives russisches Interview verantwortet]}!?? [Fiktives Interview aus St. Petersburg, 1914]@\strich\emph{?? [Fiktives Interview aus St. Petersburg, 1914]}|pwv} – ich weiss bis heute nicht, was es war – steht dennoch
               zweifellos fest und die russischen\oindex{Russland@\textbf{Russland}|pw}{ }Freunde\pwindex{Vengerova, Isabella 1.\,3.\,1877 Minsk – 7.\,2.\,1956 New York City@\textsc{Vengerova, Isabella} (1.\,3.\,1877 Minsk – 7.\,2.\,1956 New York City), \emph{Musikpädagogin, Pianistin}|pwv}\pwindex{Moller, Alice 24.\,4.\,1871 Wien – Oktober 1962@\textsc{Moller, Alice} (24.\,4.\,1871 Wien – Oktober 1962), \emph{Kassierin}|pwv}, die mich auf einem
               komplizierten Umweg davon unterrichtet haben, liessen mir überdies mitteilen, dass
               Versuche\introOben{},\introOben{} in ihren Kreisen die vollkommene Unmöglichkeit einer {\pb}Aut\introOben{}h\introOben{}entizität jener mir
               zugeschriebenen Aeusserungen aus meinem bisher unbescholtenen literarischen
               Lebenswandel zu beweisen, an der allgemeinen Verbitterung und Verhetzung gescheitert
               sind. Wie schon in meiner Erklärung\pwindex{Schnitzler, Arthur 15. 5. 1862 Wien – 21. 10. 1931 ebd.@\textsc{Schnitzler, Arthur} (15. 5. 1862 Wien – 21. 10. 1931 ebd.), \emph{Schriftsteller, Mediziner}!Brief Artur Schnitzlers@\strich\emph{Ein Brief Artur Schnitzlers}|pw} steht, ist es mir
               bisher nicht gelungen mir den Wortlaut jener gefälschten Aeusserungen\pwindex{?? [Journalist, der fiktives russisches Interview verantwortet] @\textsc{?? [Journalist, der fiktives russisches Interview verantwortet]}!?? [Fiktives Interview aus St. Petersburg, 1914]@\strich\emph{?? [Fiktives Interview aus St. Petersburg, 1914]}|pwv} zugänglich zu machen, der Sinn meiner
               Auslassungen sollte  \label{T_L03884-1v}\edtext{nach jenem Blatt ungefähr}{\lemma{\textnormal{\emph{nach … ungefähr}}}\Cendnote{\textnormal{Durch Umstellungszeichen geändert aus: »ungefähr nach jendem Blatt«.}}}\label{T_L03884-1} der folgende gewesen sein: dass ich Tolstoi\pwindex{Tolstoi, Lew Nikolajewitsch 9.\,9.\,1828 Yasnaya Polyana – 20.\,11.\,1910 Lev Tolstoy@\textsc{Tolstoi, Lew Nikolajewitsch} (9.\,9.\,1828 Yasnaya Polyana – 20.\,11.\,1910 Lev Tolstoy), \emph{Schriftsteller}|pw} als einen alten Faselhans bezeichne, von Maeterlinck\pwindex{Maeterlinck, Maurice 29.\,8.\,1862 Gent – 6.\,5.\,1949 Nizza@\textsc{Maeterlinck, Maurice} (29.\,8.\,1862 Gent – 6.\,5.\,1949 Nizza), \emph{Schriftsteller}|pw} behaupte, dass er seine Bauern schinde, von
               Anatole France\pwindex{France, Anatole 16.\,4.\,1844 Paris – 12.\,10.\,1924 Saint-Cyr-sur-Loire@\textsc{France, Anatole} (16.\,4.\,1844 Paris – 12.\,10.\,1924 Saint-Cyr-sur-Loire), \emph{Schriftsteller}|pw}, dass er mich irgendwie bestohlen
               habe, und dass ich endlich die Behauptung aufstellte, Hauptmann\pwindex{Hauptmann, Gerhart 15.\,11.\,1862 Szczawno-Zdrój – 6.\,6.\,1946 Jagniątków@\textsc{Hauptmann, Gerhart} (15.\,11.\,1862 Szczawno-Zdrój – 6.\,6.\,1946 Jagniątków), \emph{Schriftsteller}|pw} sei ein viel grösserer Dichter als Shakespeare\pwindex{Shakespeare, William 23.\,4.\,1564? Stratford-upon-Avon – 3.\,5.\,1616 ebd.@\textsc{Shakespeare, William} (23.\,4.\,1564? Stratford-upon-Avon – 3.\,5.\,1616 ebd.), \emph{Schauspieler, Dramatiker}|pw}. Aus Russland\oindex{Russland@\textbf{Russland}|pw} kam auch das
               dringende Ersuchen an mich gegen diese Verleumdungen etwas zu unternehmen.\pend
           
\pstart
           Dass eine so t\introOben{}h\introOben{}örichte Geschichte mir den ersten Anlass geben würde eine persönlich\introOben{}e\introOben{}{ }{\pb}Verbindung mit Ihnen anzuknüpfen hätten wir uns wohl Beide nicht träumen lassen. Aber
               da es sich nun einmal so fügt, will ich diese Gelegenheit gerne benützen, um Ihnen zu
               sagen, wie sehr ich Sie verehre und mit welchem Vergnügen, mit welcher wachsenden
               Freude ich Ihren wunderschönen »Jean Christophe\pwindex{Rolland, Romain 29.\,1.\,1866 Clamecy – 30.\,12.\,1944 Vézelay@\textsc{Rolland, Romain} (29.\,1.\,1866 Clamecy – 30.\,12.\,1944 Vézelay), \emph{Schriftsteller}!Jean-Christophe@\strich\emph{Jean-Christophe}|pw}« gelesen
               habe. Lassen Sie mich hoffen, dass eine Beziehung, die wenigstens von mir zu Ihnen
               innerlich längst bestanden, so seltsam sie auch in ihrem äusseren Umriss anheben mag,
               in jenen besseren Zeiten, die wir alle ersehnen\introOben{},\introOben{} und vielleicht auch noch früher,
               einen glücklichen Fortgang finde. Für heute aber seien Sie nur nochmals vielmals
               bedankt und herzlich gegrüsst von\pend
           
\pstart
           Ihrem sehr ergebenen{\\[\baselineskip]}\spacefill\mbox{{[}hs.:{]} Arthur Schnitzler}\pend
           \leftskip=0em{}\selectlanguage{ngerman}\endnumbering\briefempfaengerindex{Rolland, Romain@\textsc{Rolland, Romain}!zzzSchnitzler, Arthur@\emph{von Arthur Schnitzler}!1914-12-141@{14. 12. 1914}|)be}\mylabel{L03884h}
\begin{anhang}
\end{anhang}\newcommand{\dateiname}{L03884}\newcommand{\titel}{Arthur Schnitzler an Romain Rolland, 14. 12. 1914}\newcommand{\editorInnen}{Selma Jahnke und Martin Anton Müller}%% latex-leseansicht-abspann.tex
%% Abspann für die Leseansicht.
%% Der Schalter \ifkorrekturansicht ist bereits durch den Vorspann gesetzt.

%% latex-abspann.tex
%% Gemeinsamer Abspann für Korrekturansicht und Leseansicht.
%% Setzt den Schalter \ifkorrekturansicht voraus (gesetzt in den
%% einbindenden Dateien latex-korrekturansicht-abspann.tex bzw.
%% latex-leseansicht-abspann.tex).
%% ---------------------------------------------------------------

\normalsize

% Das esempio-Environment wird nur in der Leseansicht benötigt
\ifkorrekturansicht\else
\newenvironment{esempio}[3]%
{
    \vspace{1.5ex}
    \rlap{\underline{#1}}
    \par
    \setlength{\parindent}{0cm}
    \nopagebreak
    \leftskip=#2cm
    \rightskip=#3cm
}
{
    \par
}
\fi

\doendnotes{C}
\bigskip
\vfill

\clearpage

\footnotesize

\ifkorrekturansicht
  \lohead{\textsc{register}}
\fi

% theindex-Environment neu definieren ohne reledmac
\makeatletter
\renewenvironment{theindex}{%
  \ifkorrekturansicht
    \section*{\indexname}%
  \else
    \subsubsection*{Index der erwähnten Entitäten}%
  \fi
  \setlength{\parindent}{0pt}%
  \setlength{\parskip}{0pt plus 0.3pt}%
  \let\item\@idxitem
}{%
  \ifkorrekturansicht\clearpage\fi
}
\makeatother

\IfFileExists{\jobname-pw.ind}{\input{\jobname-pw.ind}}{}

% Quellenangabe nur in der Leseansicht
\ifkorrekturansicht\else
% Fallback-Definitionen, falls die .tex-Datei \titel etc. nicht gesetzt hat
\providecommand{\titel}{}
\providecommand{\editorInnen}{}
\providecommand{\dateiname}{\jobname}

\vspace{3cm}

\vfill

\footnotesize
\textsc{Quelle}: \titel. Herausgegeben von {\editorInnen}. In: \emph{Arthur Schnitzler: Briefwechsel mit Autorinnen und Autoren}.
 Digitale Edition, https://schnitzler-briefe.acdh.oeaw.ac.at/{\dateiname}.html (Stand \today)
\fi

\end{document}


