%% latex-leseansicht-vorspann.tex
%% Vorspann für die Leseansicht.
%% Lädt die gemeinsame Datei latex-vorspann.tex mit nicht gesetztem Schalter.

\newif\ifkorrekturansicht
\korrekturansichtfalse

\input{../tex-inputs/latex-vorspann}


\section[Georg Brandes an Arthur Schnitzler, 6. 10. 1911]{L02033 Georg Brandes an Arthur Schnitzler, 6. 10. 1911}
\nopagebreak\mylabel{L02033v}
\rehead{ }\normalsize\beginnumbering\briefempfaengerindex{Schnitzler, Arthur@\textsc{Schnitzler, Arthur}!zzzBrandes, Georg@\emph{von Georg Brandes}!1911-10-061@{6. 10. 1911}|(be}
\toendnotes[C]{\smallbreak\pagebreak[2]}
\correspDesc{Versand  durch Georg Brandes am 6. 10. 1911 in Dänemark
\newline{}Erhalt  durch Arthur Schnitzler im Zeitraum [6. 10. 1911
                  – 10. 10. 1911?] \textbf{Ort fehlend} }\toendnotes[C]{\smallbreak}
\Standort{CUL, Schnitzler, B 17.}
\physDesc{Brief, 1 Blatt, 3 Seiten, 1707 Zeichen
\newline{}Handschrift: schwarze Tinte, lateinische Kurrent
\newline{}Schnitzler: 1) mit Bleistift beschriftet: »\textsc{Brandes}«  2) mit rotem Buntstift eine Unterstreichung
\newline{}Ordnung: von unbekannter Hand nummeriert: »36« }
\buchAbdrucke{\weitereDrucke{Georg Brandes, Arthur Schnitzler: \emph{Ein Briefwechsel}. Herausgegeben von Kurt Bergel. Bern: \emph{Francke} 1956, S. 101.} }\toendnotes[C]{\smallbreak}
\pstart
           \raggedleft{}{\pb}Kopenhagen\oindex{Kopenhagen@\textbf{Kopenhagen}, \emph{Hauptstadt}|pw} (genügend Adresse){\\}6 October 11\pend
           
\pstart{}Verehrter und lieber Freund\pend\vspace{0.5em}
\pstart
           Graf Prozor\pwindex{Prozor, Moritz 16.\,1.\,1849 Vilnius – 10.\,5.\,1928 Nizza@\textsc{Prozor, Moritz} (16.\,1.\,1849 Vilnius – 10.\,5.\,1928 Nizza), \emph{Übersetzer, Diplomat}|pw}, russischer Diplomat, vieljähriger
               Uebersetzer Ibsens\pwindex{Ibsen, Henrik 20.\,3.\,1828 Skien – 23.\,5.\,1906 Oslo@\textsc{Ibsen, Henrik} (20.\,3.\,1828 Skien – 23.\,5.\,1906 Oslo), \emph{Schriftsteller}|pw} ins Französische\oindex{Frankreich@\textbf{Frankreich}|pw} – er \strikeout{hatte} hat zur
                  Frau\pwindex{Prozor, Märta Margareta 22.\,2.\,1855 Årdala – 12.\,3.\,1931 Nizza@\textsc{Prozor, Märta Margareta} (22.\,2.\,1855 Årdala – 12.\,3.\,1931 Nizza)|pwv} eine schwedische
               Gräfin und kennt unsere Sprachen – hat eine Tochter\pwindex{Prozor, Grete 28.\,12.\,1885 Paris – 14.\,2.\,1978 Genf@\textsc{Prozor, Grete} (28.\,12.\,1885 Paris – 14.\,2.\,1978 Genf), \emph{Schauspielerin}|pwv}, die durch die Wirksamkeit des Vaters Ibsen\pwindex{Ibsen, Henrik 20.\,3.\,1828 Skien – 23.\,5.\,1906 Oslo@\textsc{Ibsen, Henrik} (20.\,3.\,1828 Skien – 23.\,5.\,1906 Oslo), \emph{Schriftsteller}|pw}-Enthusiastin und \uline{Ibsen}\pwindex{Ibsen, Henrik 20.\,3.\,1828 Skien – 23.\,5.\,1906 Oslo@\textsc{Ibsen, Henrik} (20.\,3.\,1828 Skien – 23.\,5.\,1906 Oslo), \emph{Schriftsteller}|pw}\uline{-Darstellerin} geworden ist.\pend
           
\pstart
           Fräulein Prozor\pwindex{Prozor, Grete 28.\,12.\,1885 Paris – 14.\,2.\,1978 Genf@\textsc{Prozor, Grete} (28.\,12.\,1885 Paris – 14.\,2.\,1978 Genf), \emph{Schauspielerin}|pw}{ }\label{K_L02033-1v}\edtext{soll am \uline{12\textsuperscript{ten}} in Wien\oindex{Wien@\textbf{Wien}, \emph{Verwaltungsgebiet}|pw}{ }Hedda\pwindex{Ibsen, Henrik 20.\,3.\,1828 Skien – 23.\,5.\,1906 Oslo@\textsc{Ibsen, Henrik} (20.\,3.\,1828 Skien – 23.\,5.\,1906 Oslo), \emph{Schriftsteller}!Hedda Gabler@\strich\emph{Hedda Gabler}|pwv} spielen}{\lemma{\textnormal{\emph{soll … spielen}}}\Cendnote{\textnormal{Obwohl \emph{Hedda Gabler}\pwindex{Ibsen, Henrik 20.\,3.\,1828 Skien – 23.\,5.\,1906 Oslo@\textsc{Ibsen, Henrik} (20.\,3.\,1828 Skien – 23.\,5.\,1906 Oslo), \emph{Schriftsteller}!Hedda Gabler@\strich\emph{Hedda Gabler}|pwk} in der Presse als Matinée-Veranstaltung im Carl-Theater\oindex{Wien@\textbf{Wien}!II., Leopoldstadt@\textbf{II., Leopoldstadt}!Carl-Theater@\textbf{Carl-Theater}, \emph{Theater}|pwk} im Rahmen des Gastspiels von Suzanne Desprès\pwindex{Desprès, Suzanne 18.\,12.\,1875 Verdun – 29.\,6.\,1951 Paris@\textsc{Desprès, Suzanne} (18.\,12.\,1875 Verdun – 29.\,6.\,1951 Paris), \emph{Schauspielerin}|pwk} für den 12. 10. 1911
                  angekündigt wurde, ließen sich keine Kritiken zu dieser Inszenierung auffinden. Am
                  gleichen Abend spielte Greta Prozor\pwindex{Prozor, Grete 28.\,12.\,1885 Paris – 14.\,2.\,1978 Genf@\textsc{Prozor, Grete} (28.\,12.\,1885 Paris – 14.\,2.\,1978 Genf), \emph{Schauspielerin}|pwk} in \emph{La Vie de Bohême}\pwindex{Barrière, Théodore 16.\,4.\,1821 Paris – 16.\,10.\,1877 ebd.@\textsc{Barrière, Théodore} (16.\,4.\,1821 Paris – 16.\,10.\,1877 ebd.), \emph{Schriftsteller}!Leben der Bohème@\strich\emph{Das Leben der Bohème}|pwk}\pwindex{Murger, Henri 24.\,3.\,1822 Paris – 28.\,1.\,1861 ebd.@\textsc{Murger, Henri} (24.\,3.\,1822 Paris – 28.\,1.\,1861 ebd.), \emph{Schriftsteller}!Leben der Bohème@\strich\emph{Das Leben der Bohème}|pwk} von Théodor Barrière\pwindex{Barrière, Théodore 16.\,4.\,1821 Paris – 16.\,10.\,1877 ebd.@\textsc{Barrière, Théodore} (16.\,4.\,1821 Paris – 16.\,10.\,1877 ebd.), \emph{Schriftsteller}|pwk} und Henri
                     Murger\pwindex{Murger, Henri 24.\,3.\,1822 Paris – 28.\,1.\,1861 ebd.@\textsc{Murger, Henri} (24.\,3.\,1822 Paris – 28.\,1.\,1861 ebd.), \emph{Schriftsteller}|pwk} die Rolle der Madame de Rouvres\pwindex{Barrière, Théodore 16.\,4.\,1821 Paris – 16.\,10.\,1877 ebd.@\textsc{Barrière, Théodore} (16.\,4.\,1821 Paris – 16.\,10.\,1877 ebd.), \emph{Schriftsteller}!Leben der Bohème@\strich\emph{Das Leben der Bohème}|pwkv}\pwindex{Murger, Henri 24.\,3.\,1822 Paris – 28.\,1.\,1861 ebd.@\textsc{Murger, Henri} (24.\,3.\,1822 Paris – 28.\,1.\,1861 ebd.), \emph{Schriftsteller}!Leben der Bohème@\strich\emph{Das Leben der Bohème}|pwkv}. In Ibsens\pwindex{Ibsen, Henrik 20.\,3.\,1828 Skien – 23.\,5.\,1906 Oslo@\textsc{Ibsen, Henrik} (20.\,3.\,1828 Skien – 23.\,5.\,1906 Oslo), \emph{Schriftsteller}|pwk}{ }\emph{Nora}\pwindex{Ibsen, Henrik 20.\,3.\,1828 Skien – 23.\,5.\,1906 Oslo@\textsc{Ibsen, Henrik} (20.\,3.\,1828 Skien – 23.\,5.\,1906 Oslo), \emph{Schriftsteller}!Nora oder ein Puppenheim. Schauspiel in drei Akten@\strich\emph{Nora oder ein Puppenheim. Schauspiel in drei Akten}|pwk} hatte sie am 8. 10. 1911 die
                  Rolle der Frau Linden\pwindex{Ibsen, Henrik 20.\,3.\,1828 Skien – 23.\,5.\,1906 Oslo@\textsc{Ibsen, Henrik} (20.\,3.\,1828 Skien – 23.\,5.\,1906 Oslo), \emph{Schriftsteller}!Nora oder ein Puppenheim. Schauspiel in drei Akten@\strich\emph{Nora oder ein Puppenheim. Schauspiel in drei Akten}|pwkv}
                  gespielt.}}}\label{K_L02033-1}. Der Vater\pwindex{Prozor, Moritz 16.\,1.\,1849 Vilnius – 10.\,5.\,1928 Nizza@\textsc{Prozor, Moritz} (16.\,1.\,1849 Vilnius – 10.\,5.\,1928 Nizza), \emph{Übersetzer, Diplomat}|pwv} hat mich wiederholt gebeten, ihr die Bahn zu ebnen durch einen Artikel
               in der N. fr. Presse\orgindex{Neue Freie Presse@Neue Freie Presse|pw}. Ich antworte ihm 1) dass
               ich in keinerlei Verbindung mit der N. fr. Presse\orgindex{Neue Freie Presse@Neue Freie Presse|pw}
               stehe 2) dass ich seine Tochter\pwindex{Prozor, Grete 28.\,12.\,1885 Paris – 14.\,2.\,1978 Genf@\textsc{Prozor, Grete} (28.\,12.\,1885 Paris – 14.\,2.\,1978 Genf), \emph{Schauspielerin}|pwv} nie gesehen habe.\pend
           
\pstart
           Er giebt nicht nach, fleht immer als alter Freund, ich möge jemand in Wien\oindex{Wien@\textbf{Wien}, \emph{Verwaltungsgebiet}|pw} seinet{\pb}halber plagen.\pend
           
\pstart
           Ich kenne Niemand, der mit Theatersachen irgendwie in Berührung steht, als Sie
               allein.\pend
           
\pstart
           Meine Bitte ist also: fordern Sie, lieber Freund und in Wien\oindex{Wien@\textbf{Wien}, \emph{Verwaltungsgebiet}|pw} gewiss nicht ohnmächtiger Meister, irgend einen Journalisten auf, das
               Frl. Prozor\pwindex{Prozor, Grete 28.\,12.\,1885 Paris – 14.\,2.\,1978 Genf@\textsc{Prozor, Grete} (28.\,12.\,1885 Paris – 14.\,2.\,1978 Genf), \emph{Schauspielerin}|pw} (in der Truppe von \uline{Suzanne Desprès}\pwindex{Desprès, Suzanne 18.\,12.\,1875 Verdun – 29.\,6.\,1951 Paris@\textsc{Desprès, Suzanne} (18.\,12.\,1875 Verdun – 29.\,6.\,1951 Paris), \emph{Schauspielerin}|pw}) zu interviewen und für Sie ein wenig Stimmung zu machen.\pend
           \leftskip=3em{}
\pstart
           \noindent{}Dies \label{K_L02033-2v}\edtext{ma corvée}{\lemma{\textnormal{\emph{ma corvée}}}\Cendnote{\textnormal{französisch: meine lästige
               Pflicht}}}\label{K_L02033-2}.\pend
           \leftskip=0em{}
\pstart
           Aber ich mag nicht dies langweilige Zeug abschicken ohne Ihnen aufs Neue zu sagen,
               wie lieb ich Sie trotz der Entfernung und meines Alters habe, und wie gerne ich Sie
               wiedersähe.\pend
           
\pstart
           Ich habe in Italien\oindex{Italien@\textbf{Italien}|pw}, Frankreich\oindex{Frankreich@\textbf{Frankreich}|pw} und Dänemark\oindex{Dänemark@\textbf{Dänemark}|pw} in
               diesem Frühjahr 3 Monate durch Venenentzündung verloren. Ich war jetzt in {\pb}Schottland\oindex{Schottland@\textbf{Schottland}, \emph{Land}|pw}, weil die Universität St. Andrews\oindex{University of St. Andrews@\textbf{University of St. Andrews}, \emph{Universität}|pw} mich \label{K_L02033-3v}\edtext{à l’occasion}{\lemma{\textnormal{\emph{à l’occasion}}}\Cendnote{\textnormal{französisch: bei Gelegenheit}}}\label{K_L02033-3} seines 500 jährigen Bestehens zum Ehrendoktor
               ernannt hatte. So sah ich allerlei Malerisches in Schottland\oindex{Schottland@\textbf{Schottland}, \emph{Land}|pw}.\pend
           
\pstart
           Ich weiss jedoch, dass mehr Geist in Wien\oindex{Wien@\textbf{Wien}, \emph{Verwaltungsgebiet}|pw} als in
                  Edinburgh\oindex{Edinburgh@\textbf{Edinburgh}|pw} ist, und Sie sind mir der
               eigentliche Vertreter dieses Geistes.\pend
           
\pstart
           Ihr in alter Freundschaft ergebener{\\[\baselineskip]}\spacefill\mbox{Georg Brandes}\pend
           \leftskip=0em{}
\pstart
           \noindent{}Ich habe leider Ihre Adresse vergessen, was den Brief verspäten wird\pend
           \selectlanguage{ngerman}\endnumbering\briefempfaengerindex{Schnitzler, Arthur@\textsc{Schnitzler, Arthur}!zzzBrandes, Georg@\emph{von Georg Brandes}!1911-10-061@{6. 10. 1911}|)be}\mylabel{L02033h}  \newcommand{\dateiname}{L02033}\newcommand{\titel}{Georg Brandes an Arthur Schnitzler, 6. 10. 1911}\newcommand{\editorInnen}{Martin Anton Müller und Gerd-Hermann Susen}%% latex-leseansicht-abspann.tex
%% Abspann für die Leseansicht.
%% Der Schalter \ifkorrekturansicht ist bereits durch den Vorspann gesetzt.

%% latex-abspann.tex
%% Gemeinsamer Abspann für Korrekturansicht und Leseansicht.
%% Setzt den Schalter \ifkorrekturansicht voraus (gesetzt in den
%% einbindenden Dateien latex-korrekturansicht-abspann.tex bzw.
%% latex-leseansicht-abspann.tex).
%% ---------------------------------------------------------------

\normalsize

% Das esempio-Environment wird nur in der Leseansicht benötigt
\ifkorrekturansicht\else
\newenvironment{esempio}[3]%
{
    \vspace{1.5ex}
    \rlap{\underline{#1}}
    \par
    \setlength{\parindent}{0cm}
    \nopagebreak
    \leftskip=#2cm
    \rightskip=#3cm
}
{
    \par
}
\fi

\doendnotes{C}
\bigskip
\vfill

\clearpage

\footnotesize

\ifkorrekturansicht
  \lohead{\textsc{register}}
\fi

% theindex-Environment neu definieren ohne reledmac
\makeatletter
\renewenvironment{theindex}{%
  \ifkorrekturansicht
    \section*{\indexname}%
  \else
    \subsubsection*{Index der erwähnten Entitäten}%
  \fi
  \setlength{\parindent}{0pt}%
  \setlength{\parskip}{0pt plus 0.3pt}%
  \let\item\@idxitem
}{%
  \ifkorrekturansicht\clearpage\fi
}
\makeatother

\IfFileExists{\jobname-pw.ind}{\input{\jobname-pw.ind}}{}

% Quellenangabe nur in der Leseansicht
\ifkorrekturansicht\else
% Fallback-Definitionen, falls die .tex-Datei \titel etc. nicht gesetzt hat
\providecommand{\titel}{}
\providecommand{\editorInnen}{}
\providecommand{\dateiname}{\jobname}

\vspace{3cm}

\vfill

\footnotesize
\textsc{Quelle}: \titel. Herausgegeben von {\editorInnen}. In: \emph{Arthur Schnitzler: Briefwechsel mit Autorinnen und Autoren}.
 Digitale Edition, https://schnitzler-briefe.acdh.oeaw.ac.at/{\dateiname}.html (Stand \today)
\fi

\end{document}


