%% latex-leseansicht-vorspann.tex
%% Vorspann für die Leseansicht.
%% Lädt die gemeinsame Datei latex-vorspann.tex mit nicht gesetztem Schalter.

\newif\ifkorrekturansicht
\korrekturansichtfalse

\input{../tex-inputs/latex-vorspann}


         
         \renewcommand{\erwaehntePersonen}{Personen: Théodore Barrière, Georg Brandes, Suzanne Desprès, Henrik Ibsen, Henri Murger, Moritz Prozor, Märta Margareta Prozor, Grete Prozor}
         \renewcommand{\erwaehnteInstitutionen}{Institutionen: Neue Freie Presse}
         \renewcommand{\erwaehnteOrte}{Orte: Carl-Theater, Dänemark, Edinburgh, Frankreich, Italien, Kopenhagen, Schottland, University of St. Andrews, Wien}
         \renewcommand{\erwaehnteWerke}{Werke: Das Leben der Bohème, Hedda Gabler, Nora oder ein Puppenheim}
               \section[Georg Brandes an Arthur Schnitzler, 6. 10. 1911]{ Georg Brandes an Arthur Schnitzler, 6. 10. 1911}\nopagebreak\mylabel{v}\rehead{ }\begin{ledgroupsized}[t]{13cm}\normalsize\beginnumbering\briefempfaengerindex{Schnitzler, Arthur@\textsc{Schnitzler, Arthur}!zzzBrandes, Georg@\emph{von Georg Brandes}!1911-10-061@{6. 10. 1911}|(be} \toendnotes[C]{\smallbreak\pagebreak[2]} \Standort{CUL, Schnitzler, B 17.}
\physDesc{Brief, 1 Blatt, 3 Seiten, 1707 Zeichen
\newline{}Handschrift: schwarze Tinte, lateinische Kurrent
\newline{}Schnitzler: 1) mit Bleistift beschriftet: »\textsc{Brandes}«  2) mit rotem Buntstift eine Unterstreichung
\newline{}Ordnung: von unbekannter Hand nummeriert: »36« }\buchAbdrucke{\weitereDrucke{Georg Brandes, Arthur Schnitzler: \emph{Ein Briefwechsel}. Hg. Kurt Bergel. Bern: \emph{Francke} 1956, S. 101.} }\toendnotes[C]{\smallbreak}\pstart
           \raggedleft{}{\pb}Kopenhagen\oindex{Kopenhagen@\textbf{Kopenhagen}|pw} (genügend Adresse){\\}6 October 11\pend
           \pstart{}Verehrter und lieber Freund\pend\pstart
           Graf Prozor\pwindex{Prozor, Moritz 16.01.1849 – 10.05.1928@\textsc{Prozor, Moritz} (16.01.1849 – 10.05.1928), \emph{Übersetzer, Diplomat}|pw}, russischer Diplomat, vieljähriger
               Uebersetzer Ibsens\pwindex{Ibsen, Henrik 20.03.1828 – 23.05.1906@\textsc{Ibsen, Henrik} (20.03.1828 – 23.05.1906), \emph{Schriftsteller}|pw} ins Französische\oindex{Frankreich@\textbf{Frankreich}|pw} – er \strikeout{hatte} hat zur
                  Frau\pwindex{Prozor, Maerta Margareta 22.02.1855 – 12.03.1931@\textsc{Prozor, Märta Margareta} (22.02.1855 – 12.03.1931)|pwv} eine schwedische
               Gräfin und kennt unsere Sprachen – hat eine Tochter\pwindex{Prozor, Grete 28.12.1885 – 14.02.1978@\textsc{Prozor, Grete} (28.12.1885 – 14.02.1978), \emph{Schauspielerin}|pwv}, die durch die Wirksamkeit des Vaters Ibsen\pwindex{Ibsen, Henrik 20.03.1828 – 23.05.1906@\textsc{Ibsen, Henrik} (20.03.1828 – 23.05.1906), \emph{Schriftsteller}|pw}-Enthusiastin und \uline{Ibsen}\pwindex{Ibsen, Henrik 20.03.1828 – 23.05.1906@\textsc{Ibsen, Henrik} (20.03.1828 – 23.05.1906), \emph{Schriftsteller}|pw}\uline{-Darstellerin} geworden ist.\pend
           \pstart
           Fräulein Prozor\pwindex{Prozor, Grete 28.12.1885 – 14.02.1978@\textsc{Prozor, Grete} (28.12.1885 – 14.02.1978), \emph{Schauspielerin}|pw}{ }\label{K_L02033-1v}\edtext{soll am \uline{12\textsuperscript{ten}} in Wien\oindex{Wien@\textbf{Wien}|pw}{ }Hedda\pwindex{Ibsen, Henrik 20.03.1828 – 23.05.1906@\textsc{Ibsen, Henrik} (20.03.1828 – 23.05.1906), \emph{Schriftsteller}!Hedda Gabler1891@\strich\emph{Hedda Gabler} {[}1891{]}|pwv} spielen}{\lemma{\textnormal{\emph{soll … spielen}}}\Cendnote{\textnormal{Obwohl \emph{Hedda Gabler}\pwindex{Ibsen, Henrik 20.03.1828 – 23.05.1906@\textsc{Ibsen, Henrik} (20.03.1828 – 23.05.1906), \emph{Schriftsteller}!Hedda Gabler1891@\strich\emph{Hedda Gabler} {[}1891{]}|pwk} in der Presse als Matinée-Veranstaltung im Carl-Theater\oindex{Carl-Theater@\textbf{Carl-Theater}|pwk} im Rahmen des Gastspiels von Suzanne Desprès\pwindex{Despres, Suzanne 18.12.1875 – 29.06.1951@\textsc{Desprès, Suzanne} (18.12.1875 – 29.06.1951), \emph{Schauspielerin}|pwk} für den 12. 10. 1911
                  angekündigt wurde, ließen sich keine Kritiken zu dieser Inszenierung auffinden. Am
                  gleichen Abend spielte Greta Prozor\pwindex{Prozor, Grete 28.12.1885 – 14.02.1978@\textsc{Prozor, Grete} (28.12.1885 – 14.02.1978), \emph{Schauspielerin}|pwk} in \emph{La Vie de Bohême}\pwindex{Barriere, Theodore 16.4.1821 – 16.10.1877@\textsc{Barrière, Théodore} (16.4.1821 – 16.10.1877), \emph{Schriftsteller}!Leben der Boheme1849@\strich\emph{Das Leben der Bohème} {[}1849{]}|pwk}\pwindex{Murger, Henri 24.03.1822 – 28.01.1861@\textsc{Murger, Henri} (24.03.1822 – 28.01.1861), \emph{Schriftsteller}!Leben der Boheme1849@\strich\emph{Das Leben der Bohème} {[}1849{]}|pwk} von Théodor Barrière\pwindex{Barriere, Theodore 16.4.1821 – 16.10.1877@\textsc{Barrière, Théodore} (16.4.1821 – 16.10.1877), \emph{Schriftsteller}|pwk} und Henri
                     Murger\pwindex{Murger, Henri 24.03.1822 – 28.01.1861@\textsc{Murger, Henri} (24.03.1822 – 28.01.1861), \emph{Schriftsteller}|pwk} die Rolle der Madame de Rouvres\pwindex{Barriere, Theodore 16.4.1821 – 16.10.1877@\textsc{Barrière, Théodore} (16.4.1821 – 16.10.1877), \emph{Schriftsteller}!Leben der Boheme1849@\strich\emph{Das Leben der Bohème} {[}1849{]}|pwkv}\pwindex{Murger, Henri 24.03.1822 – 28.01.1861@\textsc{Murger, Henri} (24.03.1822 – 28.01.1861), \emph{Schriftsteller}!Leben der Boheme1849@\strich\emph{Das Leben der Bohème} {[}1849{]}|pwkv}. In Ibsens\pwindex{Ibsen, Henrik 20.03.1828 – 23.05.1906@\textsc{Ibsen, Henrik} (20.03.1828 – 23.05.1906), \emph{Schriftsteller}|pwk}{ }\emph{Nora}\pwindex{Ibsen, Henrik 20.03.1828 – 23.05.1906@\textsc{Ibsen, Henrik} (20.03.1828 – 23.05.1906), \emph{Schriftsteller}!Nora oder ein Puppenheim1879@\strich\emph{Nora oder ein Puppenheim} {[}1879{]}|pwk} hatte sie am 8. 10. 1911 die
                  Rolle der Frau Linden\pwindex{Ibsen, Henrik 20.03.1828 – 23.05.1906@\textsc{Ibsen, Henrik} (20.03.1828 – 23.05.1906), \emph{Schriftsteller}!Nora oder ein Puppenheim1879@\strich\emph{Nora oder ein Puppenheim} {[}1879{]}|pwkv}
                  gespielt.}}}\label{K_L02033-1h}. Der Vater\pwindex{Prozor, Moritz 16.01.1849 – 10.05.1928@\textsc{Prozor, Moritz} (16.01.1849 – 10.05.1928), \emph{Übersetzer, Diplomat}|pwv} hat mich wiederholt gebeten, ihr die Bahn zu ebnen durch einen Artikel
               in der N. fr. Presse\orgindex{Neue Freie Presse@Neue Freie Presse|pw}. Ich antworte ihm 1) dass
               ich in keinerlei Verbindung mit der N. fr. Presse\orgindex{Neue Freie Presse@Neue Freie Presse|pw}
               stehe 2) dass ich seine Tochter\pwindex{Prozor, Grete 28.12.1885 – 14.02.1978@\textsc{Prozor, Grete} (28.12.1885 – 14.02.1978), \emph{Schauspielerin}|pwv} nie gesehen habe.\pend
           \pstart
           Er giebt nicht nach, fleht immer als alter Freund, ich möge jemand in Wien\oindex{Wien@\textbf{Wien}|pw} seinet{\pb}halber plagen.\pend
           \pstart
           Ich kenne Niemand, der mit Theatersachen irgendwie in Berührung steht, als Sie
               allein.\pend
           \pstart
           Meine Bitte ist also: fordern Sie, lieber Freund und in Wien\oindex{Wien@\textbf{Wien}|pw} gewiss nicht ohnmächtiger Meister, irgend einen Journalisten auf, das
               Frl. Prozor\pwindex{Prozor, Grete 28.12.1885 – 14.02.1978@\textsc{Prozor, Grete} (28.12.1885 – 14.02.1978), \emph{Schauspielerin}|pw} (in der Truppe von \uline{Suzanne Desprès}\pwindex{Despres, Suzanne 18.12.1875 – 29.06.1951@\textsc{Desprès, Suzanne} (18.12.1875 – 29.06.1951), \emph{Schauspielerin}|pw}) zu interviewen und für Sie ein wenig Stimmung zu machen.\pend
           \leftskip=3em{}\pstart
           \noindent{}Dies \label{K_L02033-2v}\edtext{ma corvée}{\lemma{\textnormal{\emph{ma corvée}}}\Cendnote{\textnormal{französisch: meine lästige
               Pflicht}}}\label{K_L02033-2h}.\pend
           \leftskip=0em{}\pstart
           \noindent{}Aber ich mag nicht dies langweilige Zeug abschicken ohne Ihnen aufs Neue zu sagen,
               wie lieb ich Sie trotz der Entfernung und meines Alters habe, und wie gerne ich Sie
               wiedersähe.\pend
           \pstart
           Ich habe in Italien\oindex{Italien@\textbf{Italien}|pw}, Frankreich\oindex{Frankreich@\textbf{Frankreich}|pw} und Dänemark\oindex{Daenemark@\textbf{Dänemark}|pw} in
               diesem Frühjahr 3 Monate durch Venenentzündung verloren. Ich war jetzt in {\pb}Schottland\oindex{Schottland@\textbf{Schottland}|pw}, weil die Universität St. Andrews\oindex{University of St. Andrews@\textbf{University of St. Andrews}|pw} mich \label{K_L02033-3v}\edtext{à l’occasion}{\lemma{\textnormal{\emph{à l’occasion}}}\Cendnote{\textnormal{französisch: bei Gelegenheit}}}\label{K_L02033-3h} seines 500 jährigen Bestehens zum Ehrendoktor
               ernannt hatte. So sah ich allerlei Malerisches in Schottland\oindex{Schottland@\textbf{Schottland}|pw}.\pend
           \pstart
           Ich weiss jedoch, dass mehr Geist in Wien\oindex{Wien@\textbf{Wien}|pw} als in
                  Edinburgh\oindex{Edinburgh@\textbf{Edinburgh}|pw} ist, und Sie sind mir der
               eigentliche Vertreter dieses Geistes.\pend
           \pstart
           Ihr in alter Freundschaft ergebener{\\[\baselineskip]}\spacefill\mbox{Georg Brandes}\pend
           \leftskip=0em{}\pstart
           \noindent{}Ich habe leider Ihre Adresse vergessen, was den Brief verspäten wird\pend
           
         
         \endnumbering\mylabel{h}\end{ledgroupsized}  \newcommand{\dateiname}{L02033}\newcommand{\titel}{Georg Brandes an Arthur Schnitzler, 6. 10. 1911}\newcommand{\editorInnen}{Martin Anton Müller und Gerd-Hermann Susen}%% latex-leseansicht-abspann.tex
%% Abspann für die Leseansicht.
%% Der Schalter \ifkorrekturansicht ist bereits durch den Vorspann gesetzt.

%% latex-abspann.tex
%% Gemeinsamer Abspann für Korrekturansicht und Leseansicht.
%% Setzt den Schalter \ifkorrekturansicht voraus (gesetzt in den
%% einbindenden Dateien latex-korrekturansicht-abspann.tex bzw.
%% latex-leseansicht-abspann.tex).
%% ---------------------------------------------------------------

\normalsize

% Das esempio-Environment wird nur in der Leseansicht benötigt
\ifkorrekturansicht\else
\newenvironment{esempio}[3]%
{
    \vspace{1.5ex}
    \rlap{\underline{#1}}
    \par
    \setlength{\parindent}{0cm}
    \nopagebreak
    \leftskip=#2cm
    \rightskip=#3cm
}
{
    \par
}
\fi

\doendnotes{C}
\bigskip
\vfill

\clearpage

\footnotesize

\ifkorrekturansicht
  \lohead{\textsc{register}}
\fi

% theindex-Environment neu definieren ohne reledmac
\makeatletter
\renewenvironment{theindex}{%
  \ifkorrekturansicht
    \section*{\indexname}%
  \else
    \subsubsection*{Index der erwähnten Entitäten}%
  \fi
  \setlength{\parindent}{0pt}%
  \setlength{\parskip}{0pt plus 0.3pt}%
  \let\item\@idxitem
}{%
  \ifkorrekturansicht\clearpage\fi
}
\makeatother

\IfFileExists{\jobname-pw.ind}{\input{\jobname-pw.ind}}{}

% Quellenangabe nur in der Leseansicht
\ifkorrekturansicht\else
% Fallback-Definitionen, falls die .tex-Datei \titel etc. nicht gesetzt hat
\providecommand{\titel}{}
\providecommand{\editorInnen}{}
\providecommand{\dateiname}{\jobname}

\vspace{3cm}

\vfill

\footnotesize
\textsc{Quelle}: \titel. Herausgegeben von {\editorInnen}. In: \emph{Arthur Schnitzler: Briefwechsel mit Autorinnen und Autoren}.
 Digitale Edition, https://schnitzler-briefe.acdh.oeaw.ac.at/{\dateiname}.html (Stand \today)
\fi

\end{document}


      