%% latex-korrekturansicht-vorspann.tex
%% Vorspann für die Korrekturansicht.
%% Lädt die gemeinsame Datei latex-vorspann.tex mit gesetztem Schalter.

\newif\ifkorrekturansicht
\korrekturansichttrue

\input{../tex-inputs/latex-vorspann}


\section[Arthur Schnitzler an Stefan Zweig, 8. 7. 1911]{L03792 Arthur Schnitzler an Stefan Zweig, 8. 7. 1911}
\nopagebreak\mylabel{L03792v}
\rehead{ }\normalsize\beginnumbering\briefempfaengerindex{Zweig, Stefan@\textsc{Zweig, Stefan}!zzzSchnitzler, Arthur@\emph{von Arthur Schnitzler}!1911-07-081@{8. 7. 1911}|(be}
\toendnotes[C]{\smallbreak\pagebreak[2]}\Standort{Jerusalem, National Library of Israel, ARC. Ms. Var. 305 1 58 Stefan Zweig Collection.}
\physDesc{Postkarte, 1 Blatt, 2 Seiten, 183 Zeichen
\newline{}Handschrift: schwarze Tinte, deutsche Kurrent
\newline{}Versand: Stempel: »\nobreak{}\textcolor{gray}{Wien}, 8. \textcolor{gray}{VII}. 1\textcolor{gray}{1}, X\nobreak{}«.  }\toendnotes[C]{\smallbreak}\pstart{}{\pb}\textcolor{gray}{\textbf{Dr. Arthur Schnitzler}}\pend{}\pstart{}\textcolor{gray}{\textbf{Wien XVIII.
                        Sternwartestrasse 71\oindex{Sternwartestrasse 71@\textbf{Sternwartestraße 71}, \emph{Wohngebäude (K.WHS)}|pw}.}}\pend{}{\bigskip}\pstart{}\textsc{Herrn} Doctor\pend{}\pstart{}\textsc{Stefan Zweig}\pend{}\pstart{}\label{K_L03792-11v}\edtext{Wien I\oindex{I., Innere Stadt@\textbf{I., Innere Stadt}, \emph{A.ADM3}|pw}}{\lemma{\textnormal{\emph{Wien I}}}\Cendnote{\textnormal{Schreibirrtum. Die Wohnung
                     Zweigs\pwindex{Zweig, Stefan 28.11.1881 – 23.02.1942@\textsc{Zweig, Stefan} (28.11.1881 – 23.02.1942), \emph{Schriftsteller/Schriftstellerin}|pwk} lag im 8. Wiener Gemeindebezirk\oindex{XXXX Ortsangabe fehlt|pwk}.}}}\label{K_L03792-11}\pend{}\pstart{}\textsc{Kochgasse 8}\oindex{Kochgasse 8@\textbf{Kochgasse 8}, \emph{Wohngebäude (K.WHS)}|pw}\pend{}{\bigskip}\vspace{1em}
\pstart
           \noindent{}{\pb}Herzlichen Dank für die lieben \label{K_L03792-1v}\edtext{Schneeberg\oindex{Schneeberg@\textbf{Schneeberg}, \emph{Berg (N.BRG)}|pw}-Grüße}{\lemma{\textnormal{\emph{Schneeberg-Grüße}}}\Cendnote{\textnormal{Stefan Zweig an Arthur Schnitzler, 6. 7. 1911.}}}\label{K_L03792-1}. Ihr \label{K_L03792-2v}\edtext{Panama-Feuilleton\pwindex{Stunde zwischen zwei Ozeanen. Der Panamakanal@\emph{Die Stunde zwischen zwei Ozeanen. Der Panamakanal}|pwv}}{\lemma{\textnormal{\emph{Panama-Feuilleton}}}\Cendnote{\textnormal{Stefan Zweig\pwindex{Zweig, Stefan 28.11.1881 – 23.02.1942@\textsc{Zweig, Stefan} (28.11.1881 – 23.02.1942), \emph{Schriftsteller/Schriftstellerin}|pwk}:
                              \emph{Die Stunde zwischen zwei Ozeanen. Der Panamakanal}\pwindex{Stunde zwischen zwei Ozeanen. Der Panamakanal@\emph{Die Stunde zwischen zwei Ozeanen. Der Panamakanal}|pwk}. In: \emph{Neue Freie Presse}\pwindex{Neue Freie Presse@\emph{Neue Freie Presse}|pwk},
                              Nr. 16.835, 6. 7. 1911, Morgenblatt, S. 1–4.}}}\label{K_L03792-2} hat mich ſehr bewegt.\pend
           
\pstart
           Guten Sommer.{\\[\baselineskip]}Ihr{\\[\baselineskip]}\spacefill\mbox{ArthSch}\pend
           \leftskip=0em{}
\pstart
           \noindent{}Auch von meiner Frau\pwindex{Schnitzler, Olga 17.01.1882 – 13.01.1970@\textsc{Schnitzler, Olga} (17.01.1882 – 13.01.1970), \emph{Schauspieler/Schauspielerin, Sänger/Sängerin}|pwv} beste
                  Grüße.\pend
           \selectlanguage{ngerman}\endnumbering\briefempfaengerindex{Zweig, Stefan@\textsc{Zweig, Stefan}!zzzSchnitzler, Arthur@\emph{von Arthur Schnitzler}!1911-07-081@{8. 7. 1911}|)be}\mylabel{L03792h}
\begin{anhang}
\end{anhang}\normalsize

\doendnotes{C}
\bigskip
\vfill

\clearpage

\footnotesize

\lohead{\textsc{register}}

% Definiere theindex-Environment komplett neu ohne reledmac
\makeatletter
\renewenvironment{theindex}{%
  \section*{\indexname}%
  \setlength{\parindent}{0pt}%
  \setlength{\parskip}{0pt plus 0.3pt}%
  \let\item\@idxitem
}{%
  \clearpage
}
\makeatother

\IfFileExists{\jobname-pw.ind}{\input{\jobname-pw.ind}}{}

\end{document}

      