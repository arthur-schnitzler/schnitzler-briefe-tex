%% latex-korrekturansicht-vorspann.tex
%% Vorspann für die Korrekturansicht.
%% Lädt die gemeinsame Datei latex-vorspann.tex mit gesetztem Schalter.

\newif\ifkorrekturansicht
\korrekturansichttrue

\input{../tex-inputs/latex-vorspann}


\section[Hugo von Hofmannsthal an Arthur Schnitzler, 12. 8. {[}1901{]}]{L01161 Hugo von Hofmannsthal an Arthur Schnitzler, 12. 8. {[}1901{]}}
\nopagebreak\mylabel{L01161v}
\rehead{ }\normalsize\beginnumbering\briefempfaengerindex{Schnitzler, Arthur@\textsc{Schnitzler, Arthur}!zzzHofmannsthal, Hugo von@\emph{von Hugo von Hofmannsthal}!1901-08-121@{12. 8. 1901}|(be}
\toendnotes[C]{\smallbreak\pagebreak[2]}\Standort{CUL, Schnitzler, B 43.}
\physDesc{Brief, 1 Blatt, 4 Seiten, 1732 Zeichen
\newline{}Handschrift: schwarze Tinte, deutsche Kurrent
\newline{}Schnitzler: mit Bleistift die Jahreszahl ergänzt:
                                 »901« 
\newline{}Ordnung: 1) mit Bleistift von unbekannter Hand nummeriert: »\strikeout{171}«  2) mit Bleistift von unbekannter Hand nummeriert:
                                    »178«}
\buchAbdrucke{\weitereDrucke{Hugo von Hofmannsthal, Arthur Schnitzler: \emph{Briefwechsel}. Frankfurt am Main: \emph{S. Fischer} 1964, S. 151–152.} }\toendnotes[C]{\smallbreak}
\pstart
           \raggedleft{}{\pb}Rodaun\oindex{Rodaun@\textbf{Rodaun}, \emph{A.ADM4}|pw}{ }12 VIII\pend
           
\pstart{}mein lieber Arthur \pend\vspace{0.5em}
\pstart
           ich freu mich ſo herzlich darüber, daſs Sie dieſen Sommer zufrieden hinbringen. Daran
               kann man glaub ich, am deutlichſten ſelbſt ſehen, wie gern man jemanden hat: ob es
               einen ſehr freut, zu hören, daſs er ſich wohlfühlt. Könnte ich das gleiche nur auch
               von Richard\pwindex{Beer-Hofmann, Richard 1866-07-11 – 1945-09-26@\textsc{Beer-Hofmann, Richard} (1866-07-11 – 1945-09-26), \emph{Schriftsteller/Schriftstellerin}|pw} einmal hören. Was für ein
               ſonderbares Verhängnis iſt über dieſem Menſchen bei faſt lauter glücklichen Anlagen
               und Umſtänden.\hspace*{1.5em}Das iſt {\pb}eine beſonders ſchöne
               Überraſchung, daſs ich Sie ſo bald wiederſehen werde. Das hatte ich mir nicht
               gehofft.\pend
           
\pstart
           Da werden wir zuſammen radfahren. Es iſt wirklich ſo was ſchönes das Radfahren. Ich
               fahre immer gegen Abend, mit meiner Frau\pwindex{Hofmannsthal, Gertrude von 16.03.1880 – 09.11.1959@\textsc{Hofmannsthal, Gertrude von} (16.03.1880 – 09.11.1959)|pwv} oder allein. Wie ſchön ſind dieſe niederöſterreichiſchen\oindex{Niederoesterreich@\textbf{Niederösterreich}, \emph{A.ADM1}|pw} Dörfer, die dunklen Laubmaſſen auf den
               Hügeln, der ſtarke grüne kühle Geruch eines ſchattigen Abhanges, die weißen Straßen
               hügelan und -ab, die bäuriſchen kleinen Gärten. Alles riecht ſo eigen, athmet einem
               ſein {\pb}Weſen entgegen, jede Stunde
               hat ihren beſonderen Geruch; wie ſchön iſt es das alles zu fühlen.\pend
           
\pstart
           Ich habe von hier immer ein Stück bergauf, aber dann ſo ſchöne Wege; gegen die Weſtbahn\orgindex{Westbahnstrecke@Westbahnstrecke|pw} hin, Tullnerbach\oindex{Tullnerbach@\textbf{Tullnerbach}, \emph{A.ADM3}|pw}, Preſsbaum\oindex{Pressbaum@\textbf{Pressbaum}, \emph{P.PPLA3}|pw}, oder über die
                  Sulz\oindex{Sulz im Wienerwald@\textbf{Sulz im Wienerwald}, \emph{P.PPL}|pw} nach der Heiligenkreuz\oindex{Heiligenkreuz@\textbf{Heiligenkreuz}, \emph{A.ADM3}|pw}erſeite.\pend
           
\pstart
           \numberlinefalse{}–\numberlinetrue{}\pend
           
\pstart
           Den Vormittag, ohne Ausnahme, arbeite ich an meinem großen Stück\pwindex{Pompilia oder das Leben@\emph{Pompilia oder das Leben}|pwv}, mit ſehr viel Zurückhaltung und
               Überlegung, ganz anders als ſonſt. Es iſt {\pb}ja auch zum erſten Mal in meinem
               Leben eine wirklich dramatiſche Aufgabe. Schwer iſt es, die Maſſe drängt ſo von allen
               Seiten auf einen ein. Ich ſchreibe den erſten Act in Proſa, vorläufig, um mich zur
               äußerſten Deutlichkeit und Reallität in der Expoſition zu zwingen.\pend
           
\pstart
           Vom zweiten Act an geht die Handlung reißend vorwärts, einer der inhärenten Vorzüge
               dieſes Stoffes.\pend
           
\pstart
           Leben Sie wohl. Auf recht bald.\pend
           
\pstart
           Von Herzen Ihr{\\[\baselineskip]}\spacefill\mbox{Hugo}\pend
           \leftskip=0em{}\selectlanguage{ngerman}\endnumbering\briefempfaengerindex{Schnitzler, Arthur@\textsc{Schnitzler, Arthur}!zzzHofmannsthal, Hugo von@\emph{von Hugo von Hofmannsthal}!1901-08-121@{12. 8. 1901}|)be}\mylabel{L01161h}  \normalsize

\doendnotes{C}
\bigskip
\vfill

\clearpage

\footnotesize

\lohead{\textsc{register}}

% Definiere theindex-Environment komplett neu ohne reledmac
\makeatletter
\renewenvironment{theindex}{%
  \section*{\indexname}%
  \setlength{\parindent}{0pt}%
  \setlength{\parskip}{0pt plus 0.3pt}%
  \let\item\@idxitem
}{%
  \clearpage
}
\makeatother

\IfFileExists{\jobname-pw.ind}{\input{\jobname-pw.ind}}{}

\end{document}

      