%% latex-leseansicht-vorspann.tex
%% Vorspann für die Leseansicht.
%% Lädt die gemeinsame Datei latex-vorspann.tex mit nicht gesetztem Schalter.

\newif\ifkorrekturansicht
\korrekturansichtfalse

\input{../tex-inputs/latex-vorspann}


\section[Hugo von Hofmannsthal an Arthur Schnitzler, 12. 8. [1901]]{L01161 Hugo von Hofmannsthal an Arthur Schnitzler, 12. 8. [1901]}
\nopagebreak\mylabel{L01161v}
\rehead{ }\normalsize\beginnumbering\briefempfaengerindex{Schnitzler, Arthur@\textsc{Schnitzler, Arthur}!zzzHofmannsthal, Hugo von@\emph{von Hugo von Hofmannsthal}!1901-08-121@{12. 8. 1901}|(be}
\toendnotes[C]{\smallbreak\pagebreak[2]}
\correspDesc{Versand  durch Hugo von Hofmannsthal am 12. 8. 1901 in Rodaun
\newline{}Erhalt  durch Arthur Schnitzler im Zeitraum [13. 8. 1901
                  – 17. 8. 1901?] in Wien}\toendnotes[C]{\smallbreak}
\Standort{CUL, Schnitzler, B 43.}
\physDesc{Brief, 1 Blatt, 4 Seiten, 1732 Zeichen
\newline{}Handschrift: schwarze Tinte, deutsche Kurrent
\newline{}Schnitzler: mit Bleistift die Jahreszahl ergänzt:
                                 »901« 
\newline{}Ordnung: 1) mit Bleistift von unbekannter Hand nummeriert: »\strikeout{171}«  2) mit Bleistift von unbekannter Hand nummeriert:
                                    »178«}
\buchAbdrucke{\weitereDrucke{Hugo von Hofmannsthal, Arthur Schnitzler: \emph{Briefwechsel}. Herausgegeben von Therese Nickl und Heinrich Schnitzler. Frankfurt am Main: \emph{S. Fischer} 1964, S. 151–152.} }\toendnotes[C]{\smallbreak}
\pstart
           \raggedleft{}{\pb}Rodaun\oindex{Wien@\textbf{Wien}!XXIII., Liesing@\textbf{XXIII., Liesing}!Rodaun@\textbf{Rodaun}, \emph{Region}|pw}{ }12 VIII\pend
           
\pstart{}mein lieber Arthur\pend\vspace{0.5em}
\pstart
           ich freu mich{ }ſo herzlich darüber, daſs Sie dieſen Sommer zufrieden hinbringen. Daran
               kann man glaub ich, am deutlichſten{ }ſelbſt{ }ſehen, wie gern man jemanden hat: ob es
               einen{ }ſehr freut, zu hören, daſs er{ }ſich wohlfühlt. Könnte ich das gleiche nur auch
               von Richard\pwindex{Beer-Hofmann, Richard 11.\,7.\,1866 Wien – 26.\,9.\,1945 New York City@\textsc{Beer-Hofmann, Richard} (11.\,7.\,1866 Wien – 26.\,9.\,1945 New York City), \emph{Schriftsteller}|pw} einmal hören. Was für ein{ }ſonderbares Verhängnis iſt über dieſem Menſchen bei faſt lauter glücklichen Anlagen
               und Umſtänden.\hspace*{1.5em}Das iſt {\pb}eine beſonders{ }ſchöne
               Überraſchung, daſs ich Sie{ }ſo bald wiederſehen werde. Das hatte ich mir nicht
               gehofft.\pend
           
\pstart
           Da werden wir zuſammen radfahren. Es iſt wirklich{ }ſo was{ }ſchönes das Radfahren. Ich
               fahre immer gegen Abend, mit meiner Frau\pwindex{Hofmannsthal, Gertrude von 16.\,3.\,1880 Wien – 9.\,11.\,1959 Paddington@\textsc{Hofmannsthal, Gertrude von} (16.\,3.\,1880 Wien – 9.\,11.\,1959 Paddington)|pwv} oder allein. Wie{ }ſchön{ }ſind dieſe niederöſterreichiſchen\oindex{Niederösterreich@\textbf{Niederösterreich}, \emph{Land}|pw} Dörfer, die dunklen Laubmaſſen auf den
               Hügeln, der{ }ſtarke grüne kühle Geruch eines{ }ſchattigen Abhanges, die weißen Straßen
               hügelan und -ab, die bäuriſchen kleinen Gärten. Alles riecht{ }ſo eigen, athmet einem{ }ſein {\pb}Weſen entgegen, jede Stunde
               hat ihren beſonderen Geruch; wie{ }ſchön iſt es das alles zu fühlen.\pend
           
\pstart
           Ich habe von hier immer ein Stück bergauf, aber dann{ }ſo{ }ſchöne Wege; gegen die Weſtbahn\orgindex{Westbahnstrecke@Westbahnstrecke|pw} hin, Tullnerbach\oindex{Tullnerbach@\textbf{Tullnerbach}, \emph{Verwaltungsgebiet}|pw}, Preſsbaum\oindex{Pressbaum@\textbf{Pressbaum}, \emph{Hauptstadt}|pw}, oder über die
                  Sulz\oindex{Sulz im Wienerwald@\textbf{Sulz im Wienerwald}|pw} nach der Heiligenkreuz\oindex{Heiligenkreuz@\textbf{Heiligenkreuz}, \emph{Verwaltungsgebiet}|pw}erſeite.\pend
           
\pstart
           \numberlinefalse{}–\numberlinetrue{}\pend
           
\pstart
           Den Vormittag, ohne Ausnahme, arbeite ich an meinem großen Stück\pwindex{Hofmannsthal, Hugo von 1.\,2.\,1874 Wien – 15.\,7.\,1929 Rodaun@\textsc{Hofmannsthal, Hugo von} (1.\,2.\,1874 Wien – 15.\,7.\,1929 Rodaun), \emph{Schriftsteller}!Pompilia oder das Leben@\strich\emph{Pompilia oder das Leben}|pwv}, mit{ }ſehr viel Zurückhaltung und
               Überlegung, ganz anders als{ }ſonſt. Es iſt {\pb}ja auch zum erſten Mal in meinem
               Leben eine wirklich dramatiſche Aufgabe. Schwer iſt es, die Maſſe drängt{ }ſo von allen
               Seiten auf einen ein. Ich{ }ſchreibe den erſten Act in Proſa, vorläufig, um mich zur
               äußerſten Deutlichkeit und Reallität in der Expoſition zu zwingen.\pend
           
\pstart
           Vom zweiten Act an geht die Handlung reißend vorwärts, einer der inhärenten Vorzüge
               dieſes Stoffes.\pend
           
\pstart
           Leben Sie wohl. Auf recht bald.\pend
           
\pstart
           Von Herzen Ihr{\\[\baselineskip]}\spacefill\mbox{Hugo}\pend
           \leftskip=0em{}\selectlanguage{ngerman}\endnumbering\briefempfaengerindex{Schnitzler, Arthur@\textsc{Schnitzler, Arthur}!zzzHofmannsthal, Hugo von@\emph{von Hugo von Hofmannsthal}!1901-08-121@{12. 8. 1901}|)be}\mylabel{L01161h}  \newcommand{\dateiname}{L01161}\newcommand{\titel}{Hugo von Hofmannsthal an Arthur Schnitzler, 12. 8. [1901]}\newcommand{\editorInnen}{Martin Anton Müller und Gerd-Hermann Susen}%% latex-leseansicht-abspann.tex
%% Abspann für die Leseansicht.
%% Der Schalter \ifkorrekturansicht ist bereits durch den Vorspann gesetzt.

%% latex-abspann.tex
%% Gemeinsamer Abspann für Korrekturansicht und Leseansicht.
%% Setzt den Schalter \ifkorrekturansicht voraus (gesetzt in den
%% einbindenden Dateien latex-korrekturansicht-abspann.tex bzw.
%% latex-leseansicht-abspann.tex).
%% ---------------------------------------------------------------

\normalsize

% Das esempio-Environment wird nur in der Leseansicht benötigt
\ifkorrekturansicht\else
\newenvironment{esempio}[3]%
{
    \vspace{1.5ex}
    \rlap{\underline{#1}}
    \par
    \setlength{\parindent}{0cm}
    \nopagebreak
    \leftskip=#2cm
    \rightskip=#3cm
}
{
    \par
}
\fi

\doendnotes{C}
\bigskip
\vfill

\clearpage

\footnotesize

\ifkorrekturansicht
  \lohead{\textsc{register}}
\fi

% theindex-Environment neu definieren ohne reledmac
\makeatletter
\renewenvironment{theindex}{%
  \ifkorrekturansicht
    \section*{\indexname}%
  \else
    \subsubsection*{Index der erwähnten Entitäten}%
  \fi
  \setlength{\parindent}{0pt}%
  \setlength{\parskip}{0pt plus 0.3pt}%
  \let\item\@idxitem
}{%
  \ifkorrekturansicht\clearpage\fi
}
\makeatother

\IfFileExists{\jobname-pw.ind}{\input{\jobname-pw.ind}}{}

% Quellenangabe nur in der Leseansicht
\ifkorrekturansicht\else
% Fallback-Definitionen, falls die .tex-Datei \titel etc. nicht gesetzt hat
\providecommand{\titel}{}
\providecommand{\editorInnen}{}
\providecommand{\dateiname}{\jobname}

\vspace{3cm}

\vfill

\footnotesize
\textsc{Quelle}: \titel. Herausgegeben von {\editorInnen}. In: \emph{Arthur Schnitzler: Briefwechsel mit Autorinnen und Autoren}.
 Digitale Edition, https://schnitzler-briefe.acdh.oeaw.ac.at/{\dateiname}.html (Stand \today)
\fi

\end{document}


