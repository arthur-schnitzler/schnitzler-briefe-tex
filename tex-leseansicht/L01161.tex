%% latex-leseansicht-vorspann.tex
%% Vorspann für die Leseansicht.
%% Lädt die gemeinsame Datei latex-vorspann.tex mit nicht gesetztem Schalter.

\newif\ifkorrekturansicht
\korrekturansichtfalse

\input{../tex-inputs/latex-vorspann}


         
         \renewcommand{\erwaehntePersonen}{Personen: Richard Beer-Hofmann, Gertrude von Hofmannsthal}
         \renewcommand{\erwaehnteInstitutionen}{Institutionen: Westbahnstrecke}
         \renewcommand{\erwaehnteOrte}{Orte: Heiligenkreuz, Niederösterreich, Pressbaum, Rodaun, Sulz im Wienerwald, Tullnerbach, Wien}
         \renewcommand{\erwaehnteWerke}{Werke: Pompilia oder das Leben}
               \section[Hugo von Hofmannsthal an Arthur Schnitzler, 12. 8. {[}1901{]}]{ Hugo von Hofmannsthal an Arthur Schnitzler, 12. 8. {[}1901{]}}\nopagebreak\mylabel{v}\rehead{ }\begin{ledgroupsized}[t]{13cm}\normalsize\beginnumbering \toendnotes[C]{\smallbreak\pagebreak[2]} \Standort{CUL, Schnitzler, B 43.}
\physDesc{Brief, 1 Blatt, 4 Seiten
\newline{}Handschrift: schwarze Tinte, deutsche Kurrent
\newline{}Schnitzler: mit Bleistift die Jahreszahl ergänzt: »901« \newline{}Ordnung: 1) mit Bleistift von unbekannter Hand nummeriert: »\strikeout{171}«  2) mit Bleistift von unbekannter Hand nummeriert: »178«}\buchAbdrucke{\weitereDrucke{Hugo von Hofmannsthal, Arthur Schnitzler: \emph{Briefwechsel}. Hg. Therese Nickl und Heinrich Schnitzler. Frankfurt am Main: \emph{S. Fischer} 1964, S. 151–152.} }\toendnotes[C]{\smallbreak}\pstart
           \raggedleft{}{\pb}Rodaun\oindex{Rodaun@\textbf{Rodaun}|pw}{ }12 VIII\pend
           \pstart{}mein lieber Arthur \pend\pstart
           ich freu mich ſo herzlich darüber, daſs Sie dieſen Sommer zufrieden hinbringen. Daran
               kann man glaub ich, am deutlichſten ſelbſt ſehen, wie gern man jemanden hat: ob es
               einen ſehr freut, zu hören, daſs er ſich wohlfühlt. Könnte ich das gleiche nur auch
               von Richard\pwindex{Beer-Hofmann, Richard 1866-07-11 – 1945-09-26@\textsc{Beer-Hofmann, Richard} (1866-07-11 – 1945-09-26), \emph{Schriftsteller}|pw} einmal hören. Was für ein ſonderbares
               Verhängnis iſt über dieſem Menſchen bei faſt lauter glücklichen Anlagen und
                  Umſtänden.\hspace*{1.5em}Das iſt {\pb}eine beſonders ſchöne
               Überraſchung, daſs ich Sie ſo bald wiederſehen werde. Das hatte ich mir nicht
               gehofft.\pend
           \pstart
           Da werden wir zuſammen radfahren. Es iſt wirklich ſo was ſchönes das Radfahren. Ich
               fahre immer gegen Abend, mit meiner Frau\pwindex{Hofmannsthal, Gertrude von 16.03.1880 – 09.11.1959@\textsc{Hofmannsthal, Gertrude von} (16.03.1880 – 09.11.1959)|pwv} oder allein. Wie ſchön ſind dieſe niederöſterreichiſchen\oindex{Niederoesterreich@\textbf{Niederösterreich}|pw} Dörfer, die dunklen Laubmaſſen auf den Hügeln, der
               ſtarke grüne kühle Geruch eines ſchattigen Abhanges, die weißen Straßen hügelan und
               -ab, die bäuriſchen kleinen Gärten. Alles riecht ſo eigen, athmet einem ſein {\pb}Weſen entgegen, jede Stunde hat
               ihren beſonderen Geruch; wie ſchön iſt es das alles zu fühlen.\pend
           \pstart
           Ich habe von hier immer ein Stück bergauf, aber dann ſo ſchöne Wege; gegen die Weſtbahn\orgindex{Westbahnstrecke@Westbahnstrecke|pw} hin, Tullnerbach\oindex{Tullnerbach@\textbf{Tullnerbach}|pw}, Preſsbaum\oindex{Pressbaum@\textbf{Pressbaum}|pw}, oder über die Sulz\oindex{Sulz im Wienerwald@\textbf{Sulz im Wienerwald}|pw} nach der Heiligenkreuz\oindex{Heiligenkreuz@\textbf{Heiligenkreuz}|pw}erſeite.\pend
           \pstart
           \numberlinefalse{}–\numberlinetrue{}\pend
           \pstart
           Den Vormittag, ohne Ausnahme, arbeite ich an meinem großen Stück\pwindex{Hofmannsthal, Hugo von 1874-02-01 – 1929-07-15@\textsc{Hofmannsthal, Hugo von} (1874-02-01 – 1929-07-15), \emph{Schriftsteller}!Pompilia oder das Leben1901@\strich\emph{Pompilia oder das Leben} {[}1901{]}|pwv}, mit ſehr viel Zurückhaltung und
               Überlegung, ganz anders als ſonſt. Es iſt {\pb}ja auch zum erſten Mal in meinem
               Leben eine wirklich dramatiſche Aufgabe. Schwer iſt es, die Maſſe drängt ſo von allen
               Seiten auf einen ein. Ich ſchreibe den erſten Act in Proſa, vorläufig, um mich zur
               äußerſten Deutlichkeit und Reallität in der Expoſition zu zwingen.\pend
           \pstart
           Vom zweiten Act an geht die Handlung reißend vorwärts, einer der inhärenten Vorzüge
               dieſes Stoffes.\pend
           \pstart
           Leben Sie wohl. Auf recht bald.\pend
           \pstart
           Von Herzen Ihr{\\[\baselineskip]}\spacefill\mbox{Hugo}\pend
           \leftskip=0em{}
         
         \endnumbering\mylabel{h}\end{ledgroupsized}  \newcommand{\dateiname}{L01161}\newcommand{\titel}{Hugo von Hofmannsthal an Arthur Schnitzler, 12. 8. [1901]}\newcommand{\editorInnen}{Martin Anton Müller und Gerd-Hermann Susen}%% latex-leseansicht-abspann.tex
%% Abspann für die Leseansicht.
%% Der Schalter \ifkorrekturansicht ist bereits durch den Vorspann gesetzt.

%% latex-abspann.tex
%% Gemeinsamer Abspann für Korrekturansicht und Leseansicht.
%% Setzt den Schalter \ifkorrekturansicht voraus (gesetzt in den
%% einbindenden Dateien latex-korrekturansicht-abspann.tex bzw.
%% latex-leseansicht-abspann.tex).
%% ---------------------------------------------------------------

\normalsize

% Das esempio-Environment wird nur in der Leseansicht benötigt
\ifkorrekturansicht\else
\newenvironment{esempio}[3]%
{
    \vspace{1.5ex}
    \rlap{\underline{#1}}
    \par
    \setlength{\parindent}{0cm}
    \nopagebreak
    \leftskip=#2cm
    \rightskip=#3cm
}
{
    \par
}
\fi

\doendnotes{C}
\bigskip
\vfill

\clearpage

\footnotesize

\ifkorrekturansicht
  \lohead{\textsc{register}}
\fi

% theindex-Environment neu definieren ohne reledmac
\makeatletter
\renewenvironment{theindex}{%
  \ifkorrekturansicht
    \section*{\indexname}%
  \else
    \subsubsection*{Index der erwähnten Entitäten}%
  \fi
  \setlength{\parindent}{0pt}%
  \setlength{\parskip}{0pt plus 0.3pt}%
  \let\item\@idxitem
}{%
  \ifkorrekturansicht\clearpage\fi
}
\makeatother

\IfFileExists{\jobname-pw.ind}{\input{\jobname-pw.ind}}{}

% Quellenangabe nur in der Leseansicht
\ifkorrekturansicht\else
% Fallback-Definitionen, falls die .tex-Datei \titel etc. nicht gesetzt hat
\providecommand{\titel}{}
\providecommand{\editorInnen}{}
\providecommand{\dateiname}{\jobname}

\vspace{3cm}

\vfill

\footnotesize
\textsc{Quelle}: \titel. Herausgegeben von {\editorInnen}. In: \emph{Arthur Schnitzler: Briefwechsel mit Autorinnen und Autoren}.
 Digitale Edition, https://schnitzler-briefe.acdh.oeaw.ac.at/{\dateiname}.html (Stand \today)
\fi

\end{document}


      