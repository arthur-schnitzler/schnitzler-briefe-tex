%% latex-korrekturansicht-vorspann.tex
%% Vorspann für die Korrekturansicht.
%% Lädt die gemeinsame Datei latex-vorspann.tex mit gesetztem Schalter.

\newif\ifkorrekturansicht
\korrekturansichttrue

\input{../tex-inputs/latex-vorspann}


\section[Hugo von Hofmannsthal an Arthur Schnitzler, 13. 4. 1907]{L01667 Hugo von Hofmannsthal an Arthur Schnitzler, 13. 4. 1907}
\nopagebreak\mylabel{L01667v}
\rehead{ }\normalsize\beginnumbering\briefempfaengerindex{Schnitzler, Arthur@\textsc{Schnitzler, Arthur}!zzzHofmannsthal, Hugo von@\emph{von Hugo von Hofmannsthal}!1907-04-131@{13. 4. 1907}|(be}
\toendnotes[C]{\smallbreak\pagebreak[2]}\Standort{CUL, Schnitzler, B 43.}
\physDesc{Postkarte, 274 Zeichen
\newline{}Handschrift: Bleistift, lateinische Kurrent
\newline{}Versand: 1) Rohrpost  2) Stempel: »\nobreak{}\oindex{I., Innere Stadt@\textbf{I., Innere Stadt}, \emph{A.ADM3}|pwk}1/1 Wien 15, 13 IV 07, 1–N\nobreak{}«.  3) Stempel: »\nobreak{}\oindex{XVIII., Waehring@\textbf{XVIII., Währing}, \emph{A.ADM3}|pwk}18/1 Wien 111, 14 IV 07, 1\textsuperscript{50}\nobreak{}«. 
\newline{}Schnitzler: mit Bleistift datiert: »14/4 907« 
\newline{}Ordnung: 1) mit Bleistift von unbekannter Hand nummeriert:
                                    »270«  2) mit Bleistift von unbekannter Hand nummeriert:
                                    »273«}
\buchAbdrucke{\weitereDrucke{Hugo von Hofmannsthal, Arthur Schnitzler: \emph{Briefwechsel}. Frankfurt am Main: \emph{S. Fischer} 1964, S. 227.} }\toendnotes[C]{\smallbreak}\pstart{}{\pb}Herrn D\textsuperscript{r} Arthur Schnitzler\pend{}\pstart{}Wien\oindex{Wien@\textbf{Wien}, \emph{A.ADM2}|pw}\pend{}\pstart{}XVII Spöttelgasse 7\oindex{Edmund-Weiss-Gasse 7@\textbf{Edmund-Weiß-Gasse 7}, \emph{Wohngebäude (K.WHS)}|pw}\pend{}{\bigskip}\vspace{1em}
\pstart
           \raggedleft{}{\pb}Samstag\pend
           \vspace{0.5em}
\pstart
           Können wir \label{K_L01667-1v}\edtext{morgen Sonntag nachmittg}{\lemma{\textnormal{\emph{morgen Sonntag nachmittg}}}\Cendnote{\textnormal{Siehe A. S.: \emph{Tagebuch}, 14. 4. 1907.
               }}}\label{K_L01667-1} mit Christiane\pwindex{Zimmer, Christiane 14.05.1902 – 05.01.1987@\textsc{Zimmer, Christiane} (14.05.1902 – 05.01.1987)|pw} ko{\geminationm}en? (\uline{Nicht} zum Nachtmahl)\pend
           
\pstart
           Bitte pneumatisch umgehend \uline{Elisabethstraße \introOben{}6\introOben{}}\oindex{Elisabethstrasse [Wien]@\textbf{Elisabethstraße [Wien]}, \emph{Straße (K.STR)}|pw}{ }\introOben{}Schlesinger\pwindex{Schlesinger, Franziska 17.08.1851 – 11.08.1932@\textsc{Schlesinger, Franziska} (17.08.1851 – 11.08.1932)|pw}\introOben{} oder \uline{telephonisch} 229 (vielleicht durch Beers\pwindex{Beer-Hofmann, Richard 1866-07-11 – 1945-09-26@\textsc{Beer-Hofmann, Richard} (1866-07-11 – 1945-09-26), \emph{Schriftsteller/Schriftstellerin}|pwu}\pwindex{Beer-Hofmann, Paula 25.02.1879 – 30.10.1939@\textsc{Beer-Hofmann, Paula} (25.02.1879 – 30.10.1939)|pwu}).\pend
           
\pstart
           \label{T_L01667-1v}\edtext{(die Antwort trifft uns dort bis heute
                  abends 8\textsuperscript{h})}{\lemma{\textnormal{\emph{(die … 8\textsuperscript{h})}}}\Cendnote{\textnormal{oberhalb des Textes in der linken
                  Ecke}}}\label{T_L01667-1}\pend
           \selectlanguage{ngerman}\endnumbering\briefempfaengerindex{Schnitzler, Arthur@\textsc{Schnitzler, Arthur}!zzzHofmannsthal, Hugo von@\emph{von Hugo von Hofmannsthal}!1907-04-131@{13. 4. 1907}|)be}\mylabel{L01667h}  \normalsize

\doendnotes{C}
\bigskip
\vfill

\clearpage

\footnotesize

\lohead{\textsc{register}}

% Definiere theindex-Environment komplett neu ohne reledmac
\makeatletter
\renewenvironment{theindex}{%
  \section*{\indexname}%
  \setlength{\parindent}{0pt}%
  \setlength{\parskip}{0pt plus 0.3pt}%
  \let\item\@idxitem
}{%
  \clearpage
}
\makeatother

\IfFileExists{\jobname-pw.ind}{\input{\jobname-pw.ind}}{}

\end{document}

      