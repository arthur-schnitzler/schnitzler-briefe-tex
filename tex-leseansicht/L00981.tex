%% latex-leseansicht-vorspann.tex
%% Vorspann für die Leseansicht.
%% Lädt die gemeinsame Datei latex-vorspann.tex mit nicht gesetztem Schalter.

\newif\ifkorrekturansicht
\korrekturansichtfalse

\input{../tex-inputs/latex-vorspann}


         
         \newcommand{\erwaehntePersonen}{Personen: Paul Goldmann}
         \newcommand{\erwaehnteOrte}{Orte: Bad Ischl, Florenz, Grand Hotel Britannia, Sankt Michael, Vahrn, Venedig, Wien}
         \newcommand{\erwaehnteWerke}{Werke: Das Bergwerk zu Falun}
               \section[Hugo von Hofmannsthal und Richard Beer-Hofmann an Arthur Schnitzler, 27. {[}9. 1899{]}]{ Hugo von Hofmannsthal und Richard Beer-Hofmann an Arthur Schnitzler,
               27. {[}9. 1899{]}}\nopagebreak\mylabel{v}\rehead{ }\begin{ledgroupsized}[t]{13cm}\normalsize\beginnumbering \toendnotes[C]{\smallbreak\pagebreak[2]} \Standort{CUL, Schnitzler, B 43.}
\physDesc{Brief, 1 Blatt, 2 Seiten
\newline{}Handschrift Richard Beer-Hofmann: schwarze Tinte, lateinische Kurrent\newline{}Handschrift Hugo von Hofmannsthal: schwarze Tinte, deutsche Kurrent
\newline{}Schnitzler: mit Bleistift Monat und Jahreszahl ergänzt: »9. 99.« \newline{}Ordnung: 1) mit Bleistift von unbekannter Hand nummeriert: »\strikeout{162}«  2) mit Bleistift von unbekannter Hand nummeriert: »159«}\buchAbdrucke{\weitereDrucke{Hugo von Hofmannsthal, Arthur Schnitzler: \emph{Briefwechsel}. Hg. Therese Nickl und Heinrich Schnitzler. Frankfurt am Main: \emph{S. Fischer} 1964, S. 130–131.} }\toendnotes[C]{\smallbreak}\pstart
           \raggedleft{}{\pb}Vahrn\oindex{Vahrn@\textbf{Vahrn}|pw}, 27.\pend
           \pstart{}mein lieber Arthur\pend\pstart
           wir ſind beide recht fleißig, ſo ähnlich wie wir 2 in Iſchl\oindex{Bad Ischl@\textbf{Bad Ischl}|pw}. Mein Stück\pwindex{Hofmannsthal, Hugo von 1874-02-01 – 1929-07-15@\textsc{Hofmannsthal, Hugo von} (1874-02-01 – 1929-07-15), \emph{Schriftsteller}!Bergwerk zu Falun1900 – 1933@\strich\emph{Das Bergwerk zu Falun} {[}1900 – 1933{]}|pwv} aber
               wird immer ſchwerer oder ich immer dümmer. Morgen geht der
               Richard nach \textsc{St. Michael im Eppan}\oindex{Sankt Michael@\textbf{Sankt Michael}|pw}, und ich nach Venedig\oindex{Venedig@\textbf{Venedig}|pw}, Hotel Britannia\oindex{Grand Hotel Britannia@\textbf{Grand Hotel Britannia}|pw}. Vielleicht werde ich dort geſcheidter. Dieſes
               wünſcht Ihnen ſehr\pend
           \pstart
           Ihr{\\[\baselineskip]}\spacefill\mbox{Hugo}\pend
           \leftskip=0em{}\pstart
           \noindent{}{[}hs. Beer-Hofmann:{]} Hugos Wünschen schließe ich mich an. Paul\pwindex{Goldmann, Paul 31.01.1865 – 25.09.1935@\textsc{Goldmann, Paul} (31.01.1865 – 25.09.1935), \emph{Schriftsteller, Journalist}|pw} scheint nach Florenz\oindex{Florenz@\textbf{Florenz}|pw}
               gereist zu sein – ohne mich aufzusuchen. Was für Folgerungen hätte Paul\pwindex{Goldmann, Paul 31.01.1865 – 25.09.1935@\textsc{Goldmann, Paul} (31.01.1865 – 25.09.1935), \emph{Schriftsteller, Journalist}|pw} gezogen wenn ich das gethan hätte! Ich bin sehr froh daß
               ich nicht nach Florenz\oindex{Florenz@\textbf{Florenz}|pw} gereist bin u. Paul\pwindex{Goldmann, Paul 31.01.1865 – 25.09.1935@\textsc{Goldmann, Paul} (31.01.1865 – 25.09.1935), \emph{Schriftsteller, Journalist}|pw} in Vahrn\oindex{Vahrn@\textbf{Vahrn}|pw}
               ist. Meine Adresse ist St. Michael im Eppan\oindex{Sankt Michael@\textbf{Sankt Michael}|pw} – und
                  »\uline{fartig}«.\pend
           \pstart
           \label{T_L00981_1v}\edtext{Das}{\lemma{\textnormal{\emph{Das}}}\Cendnote{\textnormal{Ein Pfeil weist auf »fartig«.}}}\label{T_L00981_1h} wünscht Ihnen Ihr{\\[\baselineskip]}\spacefill\mbox{Richard}\pend
           \leftskip=0em{}
         
         \endnumbering\mylabel{h}\end{ledgroupsized}  \newcommand{\dateiname}{L00981}\newcommand{\titel}{Hugo von Hofmannsthal und Richard Beer-Hofmann an Arthur Schnitzler, 27. [9. 1899]}\newcommand{\editorInnen}{Martin Anton Müller und Gerd-Hermann Susen}%% latex-leseansicht-abspann.tex
%% Abspann für die Leseansicht.
%% Der Schalter \ifkorrekturansicht ist bereits durch den Vorspann gesetzt.

%% latex-abspann.tex
%% Gemeinsamer Abspann für Korrekturansicht und Leseansicht.
%% Setzt den Schalter \ifkorrekturansicht voraus (gesetzt in den
%% einbindenden Dateien latex-korrekturansicht-abspann.tex bzw.
%% latex-leseansicht-abspann.tex).
%% ---------------------------------------------------------------

\normalsize

% Das esempio-Environment wird nur in der Leseansicht benötigt
\ifkorrekturansicht\else
\newenvironment{esempio}[3]%
{
    \vspace{1.5ex}
    \rlap{\underline{#1}}
    \par
    \setlength{\parindent}{0cm}
    \nopagebreak
    \leftskip=#2cm
    \rightskip=#3cm
}
{
    \par
}
\fi

\doendnotes{C}
\bigskip
\vfill

\clearpage

\footnotesize

\ifkorrekturansicht
  \lohead{\textsc{register}}
\fi

% theindex-Environment neu definieren ohne reledmac
\makeatletter
\renewenvironment{theindex}{%
  \ifkorrekturansicht
    \section*{\indexname}%
  \else
    \subsubsection*{Index der erwähnten Entitäten}%
  \fi
  \setlength{\parindent}{0pt}%
  \setlength{\parskip}{0pt plus 0.3pt}%
  \let\item\@idxitem
}{%
  \ifkorrekturansicht\clearpage\fi
}
\makeatother

\IfFileExists{\jobname-pw.ind}{\input{\jobname-pw.ind}}{}

% Quellenangabe nur in der Leseansicht
\ifkorrekturansicht\else
% Fallback-Definitionen, falls die .tex-Datei \titel etc. nicht gesetzt hat
\providecommand{\titel}{}
\providecommand{\editorInnen}{}
\providecommand{\dateiname}{\jobname}

\vspace{3cm}

\vfill

\footnotesize
\textsc{Quelle}: \titel. Herausgegeben von {\editorInnen}. In: \emph{Arthur Schnitzler: Briefwechsel mit Autorinnen und Autoren}.
 Digitale Edition, https://schnitzler-briefe.acdh.oeaw.ac.at/{\dateiname}.html (Stand \today)
\fi

\end{document}


      