%% latex-korrekturansicht-vorspann.tex
%% Vorspann für die Korrekturansicht.
%% Lädt die gemeinsame Datei latex-vorspann.tex mit gesetztem Schalter.

\newif\ifkorrekturansicht
\korrekturansichttrue

\input{../tex-inputs/latex-vorspann}


\section[Felix Salten und Arthur Schnitzler an Richard Beer-Hofmann, 3. 8. 190{[}6{]}]{L01621 Felix Salten und Arthur Schnitzler an Richard Beer-Hofmann,
               3. 8. 190{[}6{]}}
\nopagebreak\mylabel{L01621v}
\rehead{ }\normalsize\beginnumbering\briefempfaengerindex{Beer-Hofmann, Richard@\textsc{Beer-Hofmann, Richard}!zzzSchnitzler, Arthur@\emph{von Arthur Schnitzler}!1906-08-032@{3. 8. 190{[}6{]}}|(be}\briefempfaengerindex{Beer-Hofmann, Richard@\textsc{Beer-Hofmann, Richard}!zzzSalten, Felix@\emph{von Felix Salten}!1906-08-032@{3. 8. 190{[}6{]}}|(be}
\toendnotes[C]{\smallbreak\pagebreak[2]}\Standort{YCGL, MSS 31.}
\physDesc{Bildpostkarte, 93 Zeichen
\newline{}Handschrift Felix Salten: schwarze Tinte, lateinische Kurrent
\newline{}Handschrift Arthur Schnitzler: schwarze Tinte, deutsche Kurrent
\newline{}Versand: 1) Stempel: »\nobreak{}\oindex{Marienlyst@\textbf{Marienlyst}, \emph{S.EST}|pwk}Marienlyst Kur- og Søbad
                                       Kiosken\nobreak{}«.   2) Stempel: »\nobreak{}\oindex{Helsingør@\textbf{Helsingør}, \emph{P.PPLA2}|pwk}Helsingor, 3. 8. 03, 6–7F\nobreak{}«.  der Absenderstempel zeigt zweifelsfrei eine falsche
                                 Jahreszahl
\newline{}Ordnung: mit Bleistift von unbekannter Hand datiert: »3. 8.« }\toendnotes[C]{\smallbreak}\pstart{}{\pb}Herrn D\textsuperscript{r} Richard Beer
                  Hofmann\pend{}\pstart{}Rodaun \textsuperscript{b}/ Wien\oindex{Rodaun@\textbf{Rodaun}, \emph{A.ADM4}|pw}\pend{}\pstart{}Wien\oindex{Wien@\textbf{Wien}, \emph{A.ADM2}|pw}\pend{}\pstart{}Österreich\oindex{Oesterreich@\textbf{Österreich}, \emph{A.PCLI}|pw}.\pend{}{\bigskip}
\pstart
           \noindent{}\centering{}{\pb}\textcolor{gray}{\textbf{Krenborg Slot\oindex{Schloss Kronborg@\textbf{Schloss Kronborg}, \emph{Gebäude (K.GBD)}|pw}.}}\pend
           \vspace{1em}
\pstart
           \noindent{}{\pb}Herzliche Grüße\pend
           \pstart \label{K_L01621-1v}\edtext{\spacefill\mbox{Salten}}{\lemma{\textnormal{\emph{Salten}}}\Cendnote{\textnormal{Vgl. Felix Salten und Arthur Schnitzler an Hugo von Hofmannsthal,
               3. 8. 1906.
               }}}\label{K_L01621-1}\pend{}\selectlanguage{ngerman}\vspace{1em}
\pstart
           \noindent{}{[}hs. :{]} Herzlichſt\pend
           \pstart \spacefill\mbox{Arthur}\pend{}\selectlanguage{ngerman}\endnumbering\briefempfaengerindex{Beer-Hofmann, Richard@\textsc{Beer-Hofmann, Richard}!zzzSchnitzler, Arthur@\emph{von Arthur Schnitzler}!1906-08-032@{3. 8. 190{[}6{]}}|)be}\briefempfaengerindex{Beer-Hofmann, Richard@\textsc{Beer-Hofmann, Richard}!zzzSalten, Felix@\emph{von Felix Salten}!1906-08-032@{3. 8. 190{[}6{]}}|)be}\mylabel{L01621h}  \normalsize

\doendnotes{C}
\bigskip
\vfill

\clearpage

\footnotesize

\lohead{\textsc{register}}

% Definiere theindex-Environment komplett neu ohne reledmac
\makeatletter
\renewenvironment{theindex}{%
  \section*{\indexname}%
  \setlength{\parindent}{0pt}%
  \setlength{\parskip}{0pt plus 0.3pt}%
  \let\item\@idxitem
}{%
  \clearpage
}
\makeatother

\IfFileExists{\jobname-pw.ind}{\input{\jobname-pw.ind}}{}

\end{document}

      