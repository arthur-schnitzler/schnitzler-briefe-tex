%% latex-leseansicht-vorspann.tex
%% Vorspann für die Leseansicht.
%% Lädt die gemeinsame Datei latex-vorspann.tex mit nicht gesetztem Schalter.

\newif\ifkorrekturansicht
\korrekturansichtfalse

\input{../tex-inputs/latex-vorspann}


\section[Arthur Schnitzler an Theodor Herzl, 18. 2. 189{[}5{]}]{L03924 Arthur Schnitzler an Theodor Herzl, 18. 2. 189[5]}
\nopagebreak\mylabel{L03924v}
\rehead{ }\normalsize\beginnumbering\briefempfaengerindex{Herzl, Theodor@\textsc{Herzl, Theodor}!zzzSchnitzler, Arthur@\emph{von Arthur Schnitzler}!1895-02-181@{18. 2. 189[5]}|(be}
\toendnotes[C]{\smallbreak\pagebreak[2]}
\correspDesc{Versand  durch Arthur Schnitzler am 18. 2. 189[5] in Wien
\newline{}Erhalt  durch Theodor Herzl in Wien}\toendnotes[C]{\smallbreak}
\Standort{Jerusalem, Central Zionist Archives, H1:1925-9.}
\physDesc{,  Blätter,  Seiten
\newline{}Handschrift: , deutsche Kurrent}
\buchAbdrucke{\weitereDrucke{Arthur Schnitzler: \emph{Briefe 1875–1912}. Herausgegeben von Therese Nickl und Heinrich Schnitzler. Frankfurt am Main: \emph{S. Fischer} 1981, S. 250–253.} }\toendnotes[C]{\smallbreak}
\pstart
           {\pb}\textcolor{gray}{\textbf{AS}}\pend
           
\pstart
           18. 2. 9\textcolor{gray}{5}\pend
           \vspace{0.5em}
\pstart
           Lieber Freund! Das telegra{\geminationm} haben Sie ja. Ich bekam von \textsc{Schick\pwindex{Schik, Friedrich *~6.\,9.\,1857 Wien@\textsc{Schik, Friedrich} (*~6.\,9.\,1857 Wien), \emph{Notar, Journalist, Dramaturg}|pw}} einen kurzen Brief: »Geſtern iſt das Stück\pwindex{Herzl, Theodor 2.\,5.\,1860 Budapest – 3.\,7.\,1904 Edlach@\textsc{Herzl, Theodor} (2.\,5.\,1860 Budapest – 3.\,7.\,1904 Edlach), \emph{Schriftsteller, Journalist}!neue Ghetto. Schauspiel in vier Acten@\strich\emph{Das neue Ghetto. Schauspiel in vier Acten}|pwv}{ }\textsc{Dr. Schnabels} per unfrank. Poſt bei mir eingetroffen: wegen
               Ueberhäufung konnte es von \textsc{Blumenthal\pwindex{Blumenthal, Oskar 13.\,3.\,1852 Berlin – 24.\,4.\,1917 ebd.@\textsc{Blumenthal, Oskar} (13.\,3.\,1852 Berlin – 24.\,4.\,1917 ebd.), \emph{Schriftsteller, Journalist, Theaterleiter}|pw}} gar nicht geleſen werden. Was ſoll nun geſchehen?« – Ich habe mir bei \textsc{Schick\pwindex{Schik, Friedrich *~6.\,9.\,1857 Wien@\textsc{Schik, Friedrich} (*~6.\,9.\,1857 Wien), \emph{Notar, Journalist, Dramaturg}|pw}} ſofort den Brief \textsc{Bl.\pwindex{Blumenthal, Oskar 13.\,3.\,1852 Berlin – 24.\,4.\,1917 ebd.@\textsc{Blumenthal, Oskar} (13.\,3.\,1852 Berlin – 24.\,4.\,1917 ebd.), \emph{Schriftsteller, Journalist, Theaterleiter}|pw}}s. erbeten, den Sie dann gleich erhalten, aber zu weitern Entſchließungen
               brauchen wir ihn ja nicht. Keinesfalls haben Sie einen Grund verſti{\geminationm}t zu ſein –
               Höchſtens über einen menſchlichen Irrthum; – es iſt Ihnen ſchon wieder einmal
               paſſirt, einen Theaterdirector für ein literariſches Individuum zu halten. Warum{ }ſag
               ich: literariſche {\pb}verläßliches, nein, – anſtändiges. – Die
               Idee der \textsc{Pseudonymität} war offenbar nicht einmal{ }ſo gut –
               als \uline{ich} geglaubt habe; – und Sie wiſſen,{ }ſehr viel hab
               ich mir davon nicht verſprochen. Ich finde, Sie \uline{über}ſchätzen die \textsc{Neue Presse\orgindex{Neue Freie Presse@Neue Freie Presse|pw}} und die \uline{unter}ſchätzen ſich. Ich begreife es
               wirklich nicht, daſs man aus einem Namen, aus einer Stellung, die man ſich doch durch
               nichts anderes erworben hat als durch den Werth ſeiner Leiſtungen, nicht wenigſtens
               den Vortheil ziehen ſollte, ſich in einer dieſem Namen u. dieſer Stellung
               entſprechenden Weiſe von jederma{\geminationn} empfangen {\pb}zu laſſen.
               Ihnen, der nicht nur eine abſolut erſte Stellung als Meiſter des deutſchen
               Feu{[}i{]}lletons im höchſten Sinn (mit dem Heimath-Feu{[}i{]}lleton\pwindex{Herzl, Theodor 2.\,5.\,1860 Budapest – 3.\,7.\,1904 Edlach@\textsc{Herzl, Theodor} (2.\,5.\,1860 Budapest – 3.\,7.\,1904 Edlach), \emph{Schriftsteller, Journalist}!Pariser Theater. (»Heimat« von Sudermann)@\strich\emph{Pariser Theater. (»Heimat« von Sudermann)}|pwv} war ich übrigens
               nicht ganz einverſtanden) einni{\geminationm}t, ſondern der auch mit einer Anzahl von Stücken
               erfolgreich aufgetreten, der mit zweien\pwindex{Herzl, Theodor 2.\,5.\,1860 Budapest – 3.\,7.\,1904 Edlach@\textsc{Herzl, Theodor} (2.\,5.\,1860 Budapest – 3.\,7.\,1904 Edlach), \emph{Schriftsteller, Journalist}!Tabarin. Schauspiel in einem Act. Frei nach Catulle Mendès@\strich\emph{Tabarin. Schauspiel in einem Act. Frei nach Catulle Mendès}|pwv}\pwindex{Herzl, Theodor 2.\,5.\,1860 Budapest – 3.\,7.\,1904 Edlach@\textsc{Herzl, Theodor} (2.\,5.\,1860 Budapest – 3.\,7.\,1904 Edlach), \emph{Schriftsteller, Journalist}!Wilddiebe. Lustspiel in vier Akten@\strich\emph{Wilddiebe. Lustspiel in vier Akten}|pwv}\pwindex{\textcolor{red}{\textsuperscript{XXXX indx1}}!Wilddiebe. Lustspiel in vier Akten@\strich\emph{Wilddiebe. Lustspiel in vier Akten}|pwv} ſogar im ſtändigen deutſchen Repertoire der erſten Bühne\orgindex{Burgtheater@Burgtheater|pwv} ſteht – Ihnen wird kein
               vernünftiger Menſch nachſagen, daſs Sie eine eventuelle Aufführung eines neuen
               Stückes Ihrer Stellung als Correſpondent der N. Fr. Pr.\orgindex{Neue Freie Presse@Neue Freie Presse|pw}
               verdanken. Es ist ja geradezu zu komiſch. {\pb}Da we{\geminationn} die \textsc{Pseudonymität} nichts andres bedeuten würde als ein Reiz mehr
               für Ihre Exiſtenz, als ein ſpiel oder ein Spaſs – da{\geminationn} wär es ja gut; aber die
               Pseudonymität iſt ein Hinderniſs, das Sie ſich ſelbſt in den Weg{ }ſtellen, viel größer
               als Sie offenbar geahnt haben. An Ihrer Kraft zweifle ich nicht, Sie wiſſen es – aber
               Sie ſehen, es gibt Hinderniſſe, die einfach nicht zu nehmen ſind. Was thut man
               beiſpielsweiſe in einem Fall wie in dem unſern? Herr \textsc{Blumenthal\pwindex{Blumenthal, Oskar 13.\,3.\,1852 Berlin – 24.\,4.\,1917 ebd.@\textsc{Blumenthal, Oskar} (13.\,3.\,1852 Berlin – 24.\,4.\,1917 ebd.), \emph{Schriftsteller, Journalist, Theaterleiter}|pw}} ſagt: Ich habe keine Zeit, Ihr Stück\pwindex{Herzl, Theodor 2.\,5.\,1860 Budapest – 3.\,7.\,1904 Edlach@\textsc{Herzl, Theodor} (2.\,5.\,1860 Budapest – 3.\,7.\,1904 Edlach), \emph{Schriftsteller, Journalist}!neue Ghetto. Schauspiel in vier Acten@\strich\emph{Das neue Ghetto. Schauspiel in vier Acten}|pwv} zu leſen, – da ſtehen Sie mir! Seien Sie über{\pb}zeugt, daſs Herr Blumenthal\pwindex{Blumenthal, Oskar 13.\,3.\,1852 Berlin – 24.\,4.\,1917 ebd.@\textsc{Blumenthal, Oskar} (13.\,3.\,1852 Berlin – 24.\,4.\,1917 ebd.), \emph{Schriftsteller, Journalist, Theaterleiter}|pw} Zeit gehabt hätte, das
               Stück\pwindex{Herzl, Theodor 2.\,5.\,1860 Budapest – 3.\,7.\,1904 Edlach@\textsc{Herzl, Theodor} (2.\,5.\,1860 Budapest – 3.\,7.\,1904 Edlach), \emph{Schriftsteller, Journalist}!neue Ghetto. Schauspiel in vier Acten@\strich\emph{Das neue Ghetto. Schauspiel in vier Acten}|pwv} des \textsc{Theodor Herzl} zu leſen. – Wollen Sie trotz aller bisherigen Erfahrungen die
                  \textsc{Pseud.}comödie weiter agiren – daſs ich Ihnen ſtets zur
               Verfügung{ }ſtehe, braucht keiner weiteren Verſicherung. Aber meine Anſicht ke{\geminationn}en Sie.
               Die Ideen, die Sie fürs Raimundth.\orgindex{Raimund-Theater@Raimund-Theater|pw} haben, deuten mir
               allerdings darauf hin, daſs Sie ſich dem Director ſelbſt gegenüber zu neuen geneigt
               wären. Dagegen iſt nun natürlich gar nichts einzuwenden. Warum aber wollen Sie nicht
               beim Dtſch. Volkstheater\orgindex{Volkstheater@Volkstheater|pw} zuerſt einen Verſuch machen? So
               viel man gegen diese Bühne – und mit {\pb}wievielem Rechte man
               es vorbringen mag – ich ließe mich noch i{\geminationm}er lieber im Volkstheater\orgindex{Volkstheater@Volkstheater|pw} als im Raimundtheater\orgindex{Raimund-Theater@Raimund-Theater|pw} aufführen.
               Mein perſönliches Verhältnis zu Herrn \textsc{Müller Gutenbrunn\pwindex{Müller-Guttenbrunn, Adam 22.\,10.\,1852 Zăbrani – 5.\,1.\,1923 Wien@\textsc{Müller-Guttenbrunn, Adam} (22.\,10.\,1852 Zăbrani – 5.\,1.\,1923 Wien), \emph{Schriftsteller, Theaterleiter, Beamter}|pw}} iſt das: daſs ich (das weiſs ich beſtimmt) einen tiefen Ekel vor ihm empfinde;
               daſs er mich (das ahne ich) nicht ausſtehen kann – und daſs wir uns höflich grüßen,
               wenn wir uns irgendwo ſehen. Mit dem Volkstheater\orgindex{Volkstheater@Volkstheater|pw}{ }ſteh ich
               jetzt gar nicht; man hat ſich recht{ }ſchäbig gegen mich beno{\geminationm}en und ich halte {\pb}Herrn Bukovics\pwindex{Bukovics, Emerich von 28.\,2.\,1844 Wien – 4.\,7.\,1905 ebd.@\textsc{Bukovics, Emerich von} (28.\,2.\,1844 Wien – 4.\,7.\,1905 ebd.), \emph{Journalist, Theaterleiter}|pw} für einen
               Cretin, Herr Müller\pwindex{Müller-Guttenbrunn, Adam 22.\,10.\,1852 Zăbrani – 5.\,1.\,1923 Wien@\textsc{Müller-Guttenbrunn, Adam} (22.\,10.\,1852 Zăbrani – 5.\,1.\,1923 Wien), \emph{Schriftsteller, Theaterleiter, Beamter}|pw} für einen Gauner und Herrn Geiringer\pwindex{Geiringer, Sigmund 1849 Stupava – 5.\,10.\,1927 Wien@\textsc{Geiringer, Sigmund} (1849 Stupava – 5.\,10.\,1927 Wien), \emph{Unternehmer, Bankier}|pw} für einen Börſianer. Auf die zwei erſteren
               kann ich die Hostie nehmen.– Ich ſtehe alſo weder mit der einen noch mit der andere
               Direction{ }ſo, daſs ich mit der Ausſicht auf irgend welchen Erfolg die Vertretung
               eines pſeudonymen Autors übernehmen kö{\geminationn}te. Doch iſt es ſelbstverſtändlich daſs ich,
                  \textsc{Arthur Schnitzler} jederzeit für meinen in Paris\oindex{Paris@\textbf{Paris}, \emph{Hauptstadt}|pw} weilenden Freund Dr \textsc{Theodor {\pb}Herzl} interveniren kann. Wollen Sie alſo
               meinen kurzen und klaren Rath? Laſſen Sie Ihr Stück\pwindex{Herzl, Theodor 2.\,5.\,1860 Budapest – 3.\,7.\,1904 Edlach@\textsc{Herzl, Theodor} (2.\,5.\,1860 Budapest – 3.\,7.\,1904 Edlach), \emph{Schriftsteller, Journalist}!neue Ghetto. Schauspiel in vier Acten@\strich\emph{Das neue Ghetto. Schauspiel in vier Acten}|pwv} ohne weitern Aufſchub unter Ihrem wahren Namen (etwa
               durch \textsc{Schick\pwindex{Schik, Friedrich *~6.\,9.\,1857 Wien@\textsc{Schik, Friedrich} (*~6.\,9.\,1857 Wien), \emph{Notar, Journalist, Dramaturg}|pw}}, der das \textsc{Mscpt\pwindex{Herzl, Theodor 2.\,5.\,1860 Budapest – 3.\,7.\,1904 Edlach@\textsc{Herzl, Theodor} (2.\,5.\,1860 Budapest – 3.\,7.\,1904 Edlach), \emph{Schriftsteller, Journalist}!neue Ghetto. Schauspiel in vier Acten@\strich\emph{Das neue Ghetto. Schauspiel in vier Acten}|pwv}} jetzt in Händen hat) an das Dtſch. Volkstheater\orgindex{Volkstheater@Volkstheater|pw}{ }ſenden. Haben Sie aber eine Vorliebe fürs Raimundtheater\orgindex{Raimund-Theater@Raimund-Theater|pw}, ſo ſenden Sie es dorthin. \uline{Perſönlich} kann ich leichter mit \textsc{Müller-Gutenbrunn\pwindex{Müller-Guttenbrunn, Adam 22.\,10.\,1852 Zăbrani – 5.\,1.\,1923 Wien@\textsc{Müller-Guttenbrunn, Adam} (22.\,10.\,1852 Zăbrani – 5.\,1.\,1923 Wien), \emph{Schriftsteller, Theaterleiter, Beamter}|pw}} in Angelegenheit Ihres Stücks\pwindex{Herzl, Theodor 2.\,5.\,1860 Budapest – 3.\,7.\,1904 Edlach@\textsc{Herzl, Theodor} (2.\,5.\,1860 Budapest – 3.\,7.\,1904 Edlach), \emph{Schriftsteller, Journalist}!neue Ghetto. Schauspiel in vier Acten@\strich\emph{Das neue Ghetto. Schauspiel in vier Acten}|pwv}
               verkehren als mit \textsc{Bukovics\pwindex{Bukovics, Emerich von 28.\,2.\,1844 Wien – 4.\,7.\,1905 ebd.@\textsc{Bukovics, Emerich von} (28.\,2.\,1844 Wien – 4.\,7.\,1905 ebd.), \emph{Journalist, Theaterleiter}|pw}}, der aller {\pb}Wahrſcheinlichkeit nach (wegen der Burg\orgindex{Burgtheater@Burgtheater|pw}) ſich verpflichtet fühlen wird, mir nicht
               angenehm ſein zu wollen. – Sehr gut kenn ich auch den Regiſſeur des Rmdthts\orgindex{Raimund-Theater@Raimund-Theater|pw}, Herrn \textsc{Langkammer\pwindex{Langkammer, Karl 4.\,8.\,1854 Wien – 18.\,5.\,1936 ebd.@\textsc{Langkammer, Karl} (4.\,8.\,1854 Wien – 18.\,5.\,1936 ebd.), \emph{Theaterleiter, Regisseur, Schauspieler}|pw}}, der, wie mir vorkommt auch mancherlei dreinzureden hat und nebſtbei ein ſehr
               geſcheidter Theatermenſch iſt. Verfügen Sie über mich, mein lieber Freund, ganz nach
               Belieben;– und gerathen Sie um Hi{\geminationm}elswillen nicht in eine kleinmütige Stimmung –
               weil {\pb}Sie wieder einmal die Erfahrung gemacht haben, daſs in
               den Theaterkanzleien ebenſo ſelten große Geiſter als liebenswürdige Menſchen ſitzen –
               wenigſtens gegen »Unbekannte«. Aber Sie ſind wirklich wie ein Menſch, der durch
               eigene Kraft ein Vermögen erworben und plötzlich die Marotte hat, von den paar
               Kreuzern zu leben, mit denen er begann. Sie haben ein Recht dazu, auch einmal Coupons
               abzuſchneiden! –\pend
           
\pstart
           {\pb}– Mein Stück\pwindex{Schnitzler, Arthur 15.\,5.\,1862 Wien – 21.\,10.\,1931 ebd.@\textsc{Schnitzler, Arthur} (15.\,5.\,1862 Wien – 21.\,10.\,1931 ebd.), \emph{Schriftsteller, Mediziner}!Liebelei. Schauspiel in drei Akten@\strich\emph{Liebelei. Schauspiel in drei Akten}|pwv} iſt jetzt auch am deutſchen Theater\orgindex{Deutsches Theater Berlin@Deutsches Theater Berlin|pw} in Berlin\oindex{Berlin@\textbf{Berlin}, \emph{Hauptstadt}|pw} angeno{\geminationm}en:
               ich habe mich nicht geſcheut, Herrn Brahm\pwindex{Brahm, Otto 5.\,2.\,1856 Hamburg – 28.\,11.\,1912 Berlin@\textsc{Brahm, Otto} (5.\,2.\,1856 Hamburg – 28.\,11.\,1912 Berlin), \emph{Theaterleiter, Regisseur}|pw} die
               Mittheilung zu machen, daſs es an der Burg\orgindex{Burgtheater@Burgtheater|pw}
               aufgeführt wird: es iſt wohl nicht unwahrſcheinlich, daſs dieſer Umſtand die Annahme
               beſchleunigt hat. Und doch halte ich mich nicht für einen Streber und doch habe ich
               die Empfindung, daſs in letzter Linie ein event. Erfolg doch nur dem
               Umſtand zu danken ſein wird, {\pb}daſs ich ein nicht mislungenes
                  Stück\pwindex{Schnitzler, Arthur 15.\,5.\,1862 Wien – 21.\,10.\,1931 ebd.@\textsc{Schnitzler, Arthur} (15.\,5.\,1862 Wien – 21.\,10.\,1931 ebd.), \emph{Schriftsteller, Mediziner}!Liebelei. Schauspiel in drei Akten@\strich\emph{Liebelei. Schauspiel in drei Akten}|pwv} geſchrieben habe.
               Seien Sie nicht weniger eitel als ich – oder, ſollt ich nicht ſagen –{ }ſeien Sie nicht
               eitler–?– Ich ſchreibe Ihnen gleich wieder, wie ich von \textsc{Schick\pwindex{Schik, Friedrich *~6.\,9.\,1857 Wien@\textsc{Schik, Friedrich} (*~6.\,9.\,1857 Wien), \emph{Notar, Journalist, Dramaturg}|pw}} den Brief habe. Seien Sie vielmals herzlich gegrüßt u. überzeugt daſs Sie von
               der »guten Freundſchaft« nach der Sie ſich ſehnen – bei mir finden ſollen, was ich zu
               geben vermag.\pend
           
\pstart
           Ihr ergebner{\\[\baselineskip]}\spacefill\mbox{Arth}\pend
           \leftskip=0em{}\selectlanguage{ngerman}\endnumbering\briefempfaengerindex{Herzl, Theodor@\textsc{Herzl, Theodor}!zzzSchnitzler, Arthur@\emph{von Arthur Schnitzler}!1895-02-181@{18. 2. 189[5]}|)be}\mylabel{L03924h}
\begin{anhang}
\end{anhang}\newcommand{\dateiname}{L03924}\newcommand{\titel}{Arthur Schnitzler an Theodor Herzl, 18. 2. 189[5]}\newcommand{\editorInnen}{Herausgegeben von Jahnke, SelmaMüller, Martin Anton}%% latex-leseansicht-abspann.tex
%% Abspann für die Leseansicht.
%% Der Schalter \ifkorrekturansicht ist bereits durch den Vorspann gesetzt.

%% latex-abspann.tex
%% Gemeinsamer Abspann für Korrekturansicht und Leseansicht.
%% Setzt den Schalter \ifkorrekturansicht voraus (gesetzt in den
%% einbindenden Dateien latex-korrekturansicht-abspann.tex bzw.
%% latex-leseansicht-abspann.tex).
%% ---------------------------------------------------------------

\normalsize

% Das esempio-Environment wird nur in der Leseansicht benötigt
\ifkorrekturansicht\else
\newenvironment{esempio}[3]%
{
    \vspace{1.5ex}
    \rlap{\underline{#1}}
    \par
    \setlength{\parindent}{0cm}
    \nopagebreak
    \leftskip=#2cm
    \rightskip=#3cm
}
{
    \par
}
\fi

\doendnotes{C}
\bigskip
\vfill

\clearpage

\footnotesize

\ifkorrekturansicht
  \lohead{\textsc{register}}
\fi

% theindex-Environment neu definieren ohne reledmac
\makeatletter
\renewenvironment{theindex}{%
  \ifkorrekturansicht
    \section*{\indexname}%
  \else
    \subsubsection*{Index der erwähnten Entitäten}%
  \fi
  \setlength{\parindent}{0pt}%
  \setlength{\parskip}{0pt plus 0.3pt}%
  \let\item\@idxitem
}{%
  \ifkorrekturansicht\clearpage\fi
}
\makeatother

\IfFileExists{\jobname-pw.ind}{\input{\jobname-pw.ind}}{}

% Quellenangabe nur in der Leseansicht
\ifkorrekturansicht\else
% Fallback-Definitionen, falls die .tex-Datei \titel etc. nicht gesetzt hat
\providecommand{\titel}{}
\providecommand{\editorInnen}{}
\providecommand{\dateiname}{\jobname}

\vspace{3cm}

\vfill

\footnotesize
\textsc{Quelle}: \titel. Herausgegeben von {\editorInnen}. In: \emph{Arthur Schnitzler: Briefwechsel mit Autorinnen und Autoren}.
 Digitale Edition, https://schnitzler-briefe.acdh.oeaw.ac.at/{\dateiname}.html (Stand \today)
\fi

\end{document}


