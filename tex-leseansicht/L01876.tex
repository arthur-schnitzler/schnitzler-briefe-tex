%% latex-leseansicht-vorspann.tex
%% Vorspann für die Leseansicht.
%% Lädt die gemeinsame Datei latex-vorspann.tex mit nicht gesetztem Schalter.

\newif\ifkorrekturansicht
\korrekturansichtfalse

\input{../tex-inputs/latex-vorspann}


         
         \renewcommand{\erwaehntePersonen}{Personen: Olga Schnitzler, Lili Schnitzler}
         \renewcommand{\erwaehnteOrte}{Orte: Bad Aussee, Wien}
         \renewcommand{\erwaehnteWerke}{}
               \section[Gerty von Hofmannsthal an Olga Schnitzler, 15. 9. 1909]{ Gerty von Hofmannsthal an Olga Schnitzler, 15. 9. 1909}\nopagebreak\mylabel{v}\rehead{ }\begin{ledgroupsized}[t]{13cm}\normalsize\beginnumbering \toendnotes[C]{\smallbreak\pagebreak[2]} \Standort{CUL, Schnitzler, B 43.}
\physDesc{Brief, 1 Blatt, 3 Seiten, 729 Zeichen
\newline{}Handschrift: schwarze Tinte, lateinische Kurrent
\newline{}Schnitzler: mit Bleistift beschriftet: »\textsc{Hofma{\geminationn}sthal}« und datiert: »Sept 909« 
\newline{}Ordnung: mit Bleistift von unbekannter Hand nummeriert:
                                    »308« }\toendnotes[C]{\smallbreak}\pstart
           \raggedleft{}{\pb}\label{K_L01876-1v}\edtext{Mittwoch}{\lemma{\textnormal{\emph{Mittwoch}}}\Cendnote{\textnormal{Die Datierung erfolgt auf den
                     ersten Mittwoch nach der Geburt von Lili
                        Schnitzler\pwindex{Schnitzler, Lili 13.09.1909 – 26.07.1928@\textsc{Schnitzler, Lili} (13.09.1909 – 26.07.1928)|pwk}.}}}\label{K_L01876-1h}\pend
           \pstart
           Liebe Olga, wir freuen uns ja riesig! Wie schön, dass es\pwindex{Schnitzler, Lili 13.09.1909 – 26.07.1928@\textsc{Schnitzler, Lili} (13.09.1909 – 26.07.1928)|pwv} auch ein Mäderl ist; ist’s
               ein schwarzes oder blondes? Ich darf gar nicht anfangen zu fragen sonst werd ich gar
               nicht fertig. Ich lebe in Gedanken alle Stunden und Tag mit und kann mir vorstellen
               wie {\pb}glücklich und zufrieden Sie sein
               werden, wenn alles gut vorübergegangen ist. Da kommen dann so ruhige Tage, in denen
               man sich nur dafür interessiert ob das Kind trinkt, ob es schläft etc. nicht
               wahr?\pend
           \pstart
           Und dem Arthur\pwindex{Schnitzler, Arthur 15.05.1862 – 21.10.1931@\textsc{Schnitzler, Arthur} (15.05.1862 – 21.10.1931), \emph{Schriftsteller, Mediziner}|pw} sagen Sie auch alle Liebe von
               uns {\pb}und wie sehr wir uns über seine
               kleine Tochter\pwindex{Schnitzler, Lili 13.09.1909 – 26.07.1928@\textsc{Schnitzler, Lili} (13.09.1909 – 26.07.1928)|pwv} freuen!\pend
           \pstart
           Wie leid thut’s mir, dass ich nicht so nah von Ihnen bin um Sie zu sehen und Ihnen
               hie und da ein bissl Gesellschaft leisten kann!\pend
           \pstart
           Also Adieu liebe Olga{\\[\baselineskip]}Von Herzen Ihre \spacefill\mbox{Gerty}\pend
           \leftskip=0em{}
         
         \endnumbering\mylabel{h}\end{ledgroupsized}  \newcommand{\dateiname}{L01876}\newcommand{\titel}{Gerty von Hofmannsthal an Olga Schnitzler, 15. 9. 1909}\newcommand{\editorInnen}{Martin Anton Müller und Gerd-Hermann Susen}%% latex-leseansicht-abspann.tex
%% Abspann für die Leseansicht.
%% Der Schalter \ifkorrekturansicht ist bereits durch den Vorspann gesetzt.

%% latex-abspann.tex
%% Gemeinsamer Abspann für Korrekturansicht und Leseansicht.
%% Setzt den Schalter \ifkorrekturansicht voraus (gesetzt in den
%% einbindenden Dateien latex-korrekturansicht-abspann.tex bzw.
%% latex-leseansicht-abspann.tex).
%% ---------------------------------------------------------------

\normalsize

% Das esempio-Environment wird nur in der Leseansicht benötigt
\ifkorrekturansicht\else
\newenvironment{esempio}[3]%
{
    \vspace{1.5ex}
    \rlap{\underline{#1}}
    \par
    \setlength{\parindent}{0cm}
    \nopagebreak
    \leftskip=#2cm
    \rightskip=#3cm
}
{
    \par
}
\fi

\doendnotes{C}
\bigskip
\vfill

\clearpage

\footnotesize

\ifkorrekturansicht
  \lohead{\textsc{register}}
\fi

% theindex-Environment neu definieren ohne reledmac
\makeatletter
\renewenvironment{theindex}{%
  \ifkorrekturansicht
    \section*{\indexname}%
  \else
    \subsubsection*{Index der erwähnten Entitäten}%
  \fi
  \setlength{\parindent}{0pt}%
  \setlength{\parskip}{0pt plus 0.3pt}%
  \let\item\@idxitem
}{%
  \ifkorrekturansicht\clearpage\fi
}
\makeatother

\IfFileExists{\jobname-pw.ind}{\input{\jobname-pw.ind}}{}

% Quellenangabe nur in der Leseansicht
\ifkorrekturansicht\else
% Fallback-Definitionen, falls die .tex-Datei \titel etc. nicht gesetzt hat
\providecommand{\titel}{}
\providecommand{\editorInnen}{}
\providecommand{\dateiname}{\jobname}

\vspace{3cm}

\vfill

\footnotesize
\textsc{Quelle}: \titel. Herausgegeben von {\editorInnen}. In: \emph{Arthur Schnitzler: Briefwechsel mit Autorinnen und Autoren}.
 Digitale Edition, https://schnitzler-briefe.acdh.oeaw.ac.at/{\dateiname}.html (Stand \today)
\fi

\end{document}


      