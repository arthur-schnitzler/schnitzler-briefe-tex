%% latex-korrekturansicht-vorspann.tex
%% Vorspann für die Korrekturansicht.
%% Lädt die gemeinsame Datei latex-vorspann.tex mit gesetztem Schalter.

\newif\ifkorrekturansicht
\korrekturansichttrue

\input{../tex-inputs/latex-vorspann}


\section[Arthur Schnitzler an Richard Beer-Hofmann, 1{[}6.?{]} 11. 1895]{L00513 Arthur Schnitzler an Richard Beer-Hofmann, 1{[}6.?{]} 11. 1895}
\nopagebreak\mylabel{L00513v}
\rehead{ }\normalsize\beginnumbering\briefempfaengerindex{Beer-Hofmann, Richard@\textsc{Beer-Hofmann, Richard}!zzzSchnitzler, Arthur@\emph{von Arthur Schnitzler}!1895-11-161@{1{[}6.?{]} 11. 1895}|(be}
\toendnotes[C]{\smallbreak\pagebreak[2]}\Standort{YCGL, MSS 31.}
\physDesc{Postkarte, 112 Zeichen
\newline{}Handschrift: Bleistift, deutsche Kurrent
\newline{}Versand: Stempel: »\nobreak{}\oindex{I., Innere Stadt@\textbf{I., Innere Stadt}, \emph{A.ADM3}|pwk}Wien 1/\textcolor{gray}{1}, 16. 11. 95, 8–9 N\nobreak{}«.  }\toendnotes[C]{\smallbreak}\pstart{}{\pb}\textsc{Dr. Richard Beer-Hofmann}\pend{}\pstart{}Wien\oindex{Wien@\textbf{Wien}, \emph{A.ADM2}|pw}. \pend{}\pstart{}\textsc{I Wollzeile 15\oindex{Wollzeile@\textbf{Wollzeile}, \emph{Straße (K.STR)}|pw}}\pend{}{\bigskip}\vspace{1em}
\pstart{}{\pb}Lieber Richard,\pend\vspace{0.5em}
\pstart
           vergeſſen Sie nicht \textsc{Johann Strauß} – \label{K_L00513-1v}\edtext{Jabuka\pwindex{Jabuka@\emph{Jabuka}|pw}}{\lemma{\textnormal{\emph{Jabuka}}}\Cendnote{\textnormal{Schnitzler und Beer-Hofmann\pwindex{Beer-Hofmann, Richard 1866-07-11 – 1945-09-26@\textsc{Beer-Hofmann, Richard} (1866-07-11 – 1945-09-26), \emph{Schriftsteller/Schriftstellerin}|pwk} besuchten am 15. 11. 1895 die Aufführung im Theater an der Wien\oindex{Theater an der Wien@\textbf{Theater an der Wien}, \emph{Theater (K.THE)}|pwk}. Der Poststempel verweist eindeutig auf
                  den Nachmittag des Folgetags. Zwar wäre ein falsch gestellter Stempel vorstellbar,
                  aber auch dann würde die Uhrzeit nicht unbedingt mit den im \emph{Tagebuch}\pwindex{Tagebuch@\emph{Tagebuch}|pwk} beschriebenen Ereignissen zusammenpassen.
                  Eventuell blieb die Karte länger als gedacht im Abholpostkasten? Oder – und damit
                  ließe sich der Stempel erklären – es handelt sich um eine Erinnerung in Folge der
                  gemeinsam besuchten Aufführung.}}}\label{K_L00513-1}\pend
           
\pstart
           Herzlich Ihr{\\[\baselineskip]}\spacefill\mbox{Art}\pend
           \leftskip=0em{}\selectlanguage{ngerman}\endnumbering\briefempfaengerindex{Beer-Hofmann, Richard@\textsc{Beer-Hofmann, Richard}!zzzSchnitzler, Arthur@\emph{von Arthur Schnitzler}!1895-11-161@{1{[}6.?{]} 11. 1895}|)be}\mylabel{L00513h}  \normalsize

\doendnotes{C}
\bigskip
\vfill

\clearpage

\footnotesize

\lohead{\textsc{register}}

% Definiere theindex-Environment komplett neu ohne reledmac
\makeatletter
\renewenvironment{theindex}{%
  \section*{\indexname}%
  \setlength{\parindent}{0pt}%
  \setlength{\parskip}{0pt plus 0.3pt}%
  \let\item\@idxitem
}{%
  \clearpage
}
\makeatother

\IfFileExists{\jobname-pw.ind}{\input{\jobname-pw.ind}}{}

\end{document}

      