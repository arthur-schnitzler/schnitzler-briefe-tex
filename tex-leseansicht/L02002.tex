%% latex-leseansicht-vorspann.tex
%% Vorspann für die Leseansicht.
%% Lädt die gemeinsame Datei latex-vorspann.tex mit nicht gesetztem Schalter.

\newif\ifkorrekturansicht
\korrekturansichtfalse

\input{../tex-inputs/latex-vorspann}


         
         \renewcommand{\erwaehntePersonen}{Personen: Robert Adam, Julius Bittner}
         \renewcommand{\erwaehnteInstitutionen}{Institutionen: Der Merker, S. Fischer Verlag}
         \renewcommand{\erwaehnteOrte}{Orte: Wien}
         \renewcommand{\erwaehnteWerke}{Werke: Die Geschichte des Alî ibn Bekkâr mit Schams an-Nahâr, Neidhard}
               \section[Robert Adam an Arthur Schnitzler, 3. 2. 1911]{ Robert Adam an Arthur Schnitzler, 3. 2. 1911}\nopagebreak\mylabel{v}\rehead{ }\begin{ledgroupsized}[t]{13cm}\normalsize\beginnumbering \toendnotes[C]{\smallbreak\pagebreak[2]} \Standort{DLA, A:Schnitzler, HS.NZ85.1.4230,3.}
\physDesc{Brief, 1 Blatt, 4 Seiten, 2108 Zeichen
\newline{}Handschrift: schwarze Tinte, deutsche Kurrent
\newline{}Schnitzler: 1) mit Bleistift beschriftet: »\textsc{Adam}«  2) mit rotem Buntstift eine Unterstreichung}\Standort{Wien, Österreichische Nationalbibliothek, Cod.ser. 52.266, 79.}
\physDesc{handschriftliche Abschrift, 1 Blatt, 1 Seite
\newline{}Handschrift: schwarze Tinte, Gabelsberger Kurzschrift}\Standort{Wien, Österreichische Nationalbibliothek, Cod.ser. 52.266, 79.}
\physDesc{maschinenschriftliche Abschrift, 1 Blatt, 1 Seite
\newline{}Schreibmaschine}\toendnotes[C]{\smallbreak}\pstart
           \raggedleft{}{\pb}Wien\oindex{Wien@\textbf{Wien}|pw}, am 3. Febr. 1911\pend
           \pstart{}Hochverehrter Herr Doktor!\pend\pstart
           Ich muß Ihnen leider berichten, daß der Verſuch, an mein Glück zu appellieren,
               fehlgeſchlagen iſt. Der Verlag S. Fiſcher\orgindex{S. Fischer Verlag@S. Fischer Verlag|pw} hat mir
               mitgeteilt, daß er den »Neidhard\pwindex{Adam, Robert 20.04.1877 – 16.10.1961@\textsc{Adam, Robert} (20.04.1877 – 16.10.1961), \emph{Schriftsteller, Richter}!NeidhardNone@\strich\emph{Neidhard} {[}None{]}|pw}« nicht annehmen
               konnte. Die dem Schreiben beigefügte ſehr liebenswürdige und eingehende Begründung
               dieſer Entſcheidung dürfte ſich in einem Punkte mit dem Hauptbedenken berühren, das
               Sie, hochverehrter Herr Doktor, {\pb}bezüglich des
               ſtofflichen Aufbaus der Komödie mir gegenüber äußerten. Manches iſt mir in der
               Begründung der Abweiſung nicht recht verſtändlich. Es will mir ſcheinen, als ob der
               Verlag bei der Fixierung des Grundthemas meiner Komödie fehlgegriffen hätte;
               wenigſtens iſt das, was im Schreiben als Thema des Stückes bezeichnet wird, nur ein
               Teil deſſen, was nach meiner Abſicht Thema ſein ſollte. Iſt dem ſo, ſo muß die
               Komödie unklarer ſein als ich dachte; und dies wäre jedenfalls ein ſehr arger Fehler.
               Ich war redlich bemüht, den Grundgedanken hervortreten zu laſſen, {\pb}wenn ich es auch – anders als in der arabiſchen Komödie\pwindex{Adam, Robert 20.04.1877 – 16.10.1961@\textsc{Adam, Robert} (20.04.1877 – 16.10.1961), \emph{Schriftsteller, Richter}!Geschichte des Alî ibn Bekkâr mit Schams an-Nahâr1909@\strich\emph{Die Geschichte des Alî ibn Bekkâr mit Schams an-Nahâr} {[}1909{]}|pwv} – abſichtlich vermied,
               im Kontexte einfach herauszuſagen, was ich durch die Handlung verſinnbildlichen
               wollte; die Zwiſchenſpiele, als moderniſierter Chor, ſollten das Amt des Räſonneurs
               übernehmen.\pend
           \pstart
           Dies ſcheint nicht geglückt zu ſein; und um zu verbeſſern, was noch ſich beſſern
               läßt, will ich einen Plan, den ich ſchon vordem faßte, nun ausführen; nämlich,
               wenigſtens in einem kritiſchen Nachwort, das in der Form zweier Briefe von Freunden,
               eines zerreißenden und eines erhebenden, {\pb}gehalten
               ſein ſoll, all das klar auseinanderſetzen, was Mangel und gute Abſicht der Komödie
               (nach Anſicht des Autors) iſt.\pend
           \pstart
           Daß mich das Fehlſchlagen dieſer Hoffnung, obwohl ich’s längſt aufgegeben habe, mir
               Glück zu vindizieren, arg deprimiert, werden Sie begreifen, hochverehrter Herr
               Doktor; aber ich will’s übertünchen.\pend
           \pstart
           Dem »Merker\orgindex{Merker@Der Merker|pw}« habe ich die arabiſche Komödie\pwindex{Adam, Robert 20.04.1877 – 16.10.1961@\textsc{Adam, Robert} (20.04.1877 – 16.10.1961), \emph{Schriftsteller, Richter}!Geschichte des Alî ibn Bekkâr mit Schams an-Nahâr1909@\strich\emph{Die Geschichte des Alî ibn Bekkâr mit Schams an-Nahâr} {[}1909{]}|pw} mit einer Empfehlung des D\textsuperscript{r} Bittner\pwindex{Bittner, Julius 09.04.1874 – 09.01.1939@\textsc{Bittner, Julius} (09.04.1874 – 09.01.1939), \emph{Komponist, Richter}|pw}
               eingeſendet; vorläufig ohne Reſultat.\pend
           \pstart
           Nehmen Sie mir die Länge dieſes Briefes nicht übel, hochverehrter Herr Doktor, und
               ſeien Sie herzlich gegrüßt von Ihrem\pend
           \pstart
           dankbar ergebenen{\\[\baselineskip]}\spacefill\mbox{Robert Adam}\pend
           \leftskip=0em{}
         
         \endnumbering\mylabel{h}\end{ledgroupsized}  \newcommand{\dateiname}{L02002}\newcommand{\titel}{Robert Adam an Arthur Schnitzler, 3. 2. 1911}\newcommand{\editorInnen}{Martin Anton Müller und Gerd-Hermann Susen}%% latex-leseansicht-abspann.tex
%% Abspann für die Leseansicht.
%% Der Schalter \ifkorrekturansicht ist bereits durch den Vorspann gesetzt.

%% latex-abspann.tex
%% Gemeinsamer Abspann für Korrekturansicht und Leseansicht.
%% Setzt den Schalter \ifkorrekturansicht voraus (gesetzt in den
%% einbindenden Dateien latex-korrekturansicht-abspann.tex bzw.
%% latex-leseansicht-abspann.tex).
%% ---------------------------------------------------------------

\normalsize

% Das esempio-Environment wird nur in der Leseansicht benötigt
\ifkorrekturansicht\else
\newenvironment{esempio}[3]%
{
    \vspace{1.5ex}
    \rlap{\underline{#1}}
    \par
    \setlength{\parindent}{0cm}
    \nopagebreak
    \leftskip=#2cm
    \rightskip=#3cm
}
{
    \par
}
\fi

\doendnotes{C}
\bigskip
\vfill

\clearpage

\footnotesize

\ifkorrekturansicht
  \lohead{\textsc{register}}
\fi

% theindex-Environment neu definieren ohne reledmac
\makeatletter
\renewenvironment{theindex}{%
  \ifkorrekturansicht
    \section*{\indexname}%
  \else
    \subsubsection*{Index der erwähnten Entitäten}%
  \fi
  \setlength{\parindent}{0pt}%
  \setlength{\parskip}{0pt plus 0.3pt}%
  \let\item\@idxitem
}{%
  \ifkorrekturansicht\clearpage\fi
}
\makeatother

\IfFileExists{\jobname-pw.ind}{\input{\jobname-pw.ind}}{}

% Quellenangabe nur in der Leseansicht
\ifkorrekturansicht\else
% Fallback-Definitionen, falls die .tex-Datei \titel etc. nicht gesetzt hat
\providecommand{\titel}{}
\providecommand{\editorInnen}{}
\providecommand{\dateiname}{\jobname}

\vspace{3cm}

\vfill

\footnotesize
\textsc{Quelle}: \titel. Herausgegeben von {\editorInnen}. In: \emph{Arthur Schnitzler: Briefwechsel mit Autorinnen und Autoren}.
 Digitale Edition, https://schnitzler-briefe.acdh.oeaw.ac.at/{\dateiname}.html (Stand \today)
\fi

\end{document}


      