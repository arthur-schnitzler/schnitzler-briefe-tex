%% latex-leseansicht-vorspann.tex
%% Vorspann für die Leseansicht.
%% Lädt die gemeinsame Datei latex-vorspann.tex mit nicht gesetztem Schalter.

\newif\ifkorrekturansicht
\korrekturansichtfalse

\input{../tex-inputs/latex-vorspann}


         
         \renewcommand{\erwaehntePersonen}{Personen: Paul Goldmann, Eva Marie Goldmann, Franziska Goldmann, Leo P. Reckford, Rena R. Schlein}
         \renewcommand{\erwaehnteOrte}{Orte: Bahnhof, Berlin, Deutschland, San Marco, Stauffenbergstraße, Sternwartestraße, Venedig, Wien, XVIII., Währing}
         \renewcommand{\erwaehnteWerke}{Werke: Ritterlichkeit}
               \section[Arthur Schnitzler an Paul Goldmann, 25. 4. 1927]{ Arthur Schnitzler an Paul Goldmann, 25. 4. 1927}\nopagebreak\mylabel{v}\rehead{ }\begin{ledgroupsized}[t]{13cm}\normalsize\beginnumbering \toendnotes[C]{\smallbreak\pagebreak[2]} \Standort{DLA, A:Schnitzler, HS85.1.5681.}
\physDesc{Bildpostkarte, Fotokopie, 336 Zeichen
\newline{}Handschrift: Bleistift, lateinische Kurrent
\newline{}Versand: Stempel: »\nobreak{}\oindex{Bahnhof@\textbf{Bahnhof}|pwk}Venezia Ferrovia, 25. IV 1927, 22–23\nobreak{}«.  
\newline{}Zusatz: Von den Korrespondenzstücken Schnitzler\pwindex{Schnitzler, Arthur 15.05.1862 – 21.10.1931@\textsc{Schnitzler, Arthur} (15.05.1862 – 21.10.1931), \emph{Schriftsteller, Mediziner}|pw}s an Goldmann\pwindex{Goldmann, Paul 31.01.1865 – 25.09.1935@\textsc{Goldmann, Paul} (31.01.1865 – 25.09.1935), \emph{Schriftsteller, Journalist}|pw} fehlt weitgehend jede Spur. In der Edition von
                                    Ritterlichkeit\pwindex{Schnitzler, Arthur 15.05.1862 – 21.10.1931@\textsc{Schnitzler, Arthur} (15.05.1862 – 21.10.1931), \emph{Schriftsteller, Mediziner}!Ritterlichkeit1977@\strich\emph{Ritterlichkeit} {[}1977{]}|pw}
                                    (1975) schreibt die Herausgeberin Rena R. Schlein\pwindex{Schlein, Rena R. *~1919-06-20@\textsc{Schlein, Rena R.} (*~1919-06-20)|pw}: »Zwei Telegramme
                                    und ein Brief Schnitzler\pwindex{Schnitzler, Arthur 15.05.1862 – 21.10.1931@\textsc{Schnitzler, Arthur} (15.05.1862 – 21.10.1931), \emph{Schriftsteller, Mediziner}|pw}s
                                    an Goldmann\pwindex{Goldmann, Paul 31.01.1865 – 25.09.1935@\textsc{Goldmann, Paul} (31.01.1865 – 25.09.1935), \emph{Schriftsteller, Journalist}|pw} wurden mir
                                    von Dr. Leo P. Reckford\pwindex{Reckford, Leo P. 1903-05-03 – 1988-10-19@\textsc{Reckford, Leo P.} (1903-05-03 – 1988-10-19), \emph{Mediziner}|pw},
                                    der diese Dokumente von der Familie Goldmann\pwindex{Goldmann, Paul 31.01.1865 – 25.09.1935@\textsc{Goldmann, Paul} (31.01.1865 – 25.09.1935), \emph{Schriftsteller, Journalist}|pw}s zum Geschenk bekam, für meine
                                    Arbeit zur Verfügung gestellt« (S. 1). Reckford\pwindex{Reckford, Leo P. 1903-05-03 – 1988-10-19@\textsc{Reckford, Leo P.} (1903-05-03 – 1988-10-19), \emph{Mediziner}|pw} starb 1988, seine
                                 Nachkommen haben keine Kenntnis von diesen (und etwaigen weiteren)
                                 Korrespondenzstücken und sie sind auch nicht auffindbar. Rena R. Schlein\pwindex{Schlein, Rena R. *~1919-06-20@\textsc{Schlein, Rena R.} (*~1919-06-20)|pw} kam
                                    1919 zur Welt. Ein Kontakt konnte nicht hergestellt
                                 werden. Die vorliegende Schwarz-Weiß-Fotokopie wird im Nachlass Schnitzler\pwindex{Schnitzler, Arthur 15.05.1862 – 21.10.1931@\textsc{Schnitzler, Arthur} (15.05.1862 – 21.10.1931), \emph{Schriftsteller, Mediziner}|pw}s zusammen mit
                                 Kopien zwei der drei in Ritterlichkeit\pwindex{Schnitzler, Arthur 15.05.1862 – 21.10.1931@\textsc{Schnitzler, Arthur} (15.05.1862 – 21.10.1931), \emph{Schriftsteller, Mediziner}!Ritterlichkeit1977@\strich\emph{Ritterlichkeit} {[}1977{]}|pw} abgedruckten Korrespondenzstücken
                                 aufbewahrt, was darauf hindeutet, dass auch diese Postkarte zu
                                 einem bestimmten Zeitpunkt im Besitz Reckford\pwindex{Reckford, Leo P. 1903-05-03 – 1988-10-19@\textsc{Reckford, Leo P.} (1903-05-03 – 1988-10-19), \emph{Mediziner}|pw}s gewesen ist. }\toendnotes[C]{\smallbreak}\pstart{}{\pb}\label{T_L02404-1v}\edtext{\textcolor{gray}{\textbf{A. S.}}}{\lemma{\textnormal{\emph{A. S.}}}\Cendnote{\textnormal{ovaler Absenderkleber}}}\label{T_L02404-1h}\pend{}\pstart{}\textcolor{gray}{\textbf{WIEN, XVIII.}}\oindex{XVIII., Waehring@\textbf{XVIII., Währing}|pw}\pend{}\pstart{}\textcolor{gray}{\textbf{STERNWARTESTR. 71}}\oindex{Sternwartestrasse@\textbf{Sternwartestraße}|pw}\pend{}{\bigskip}\pstart{}{\pb}Germania\oindex{Deutschland@\textbf{Deutschland}|pw}\pend{}\pstart{}Hn Dr Paul Goldmann\pend{}\pstart{}Berlin W\oindex{Berlin@\textbf{Berlin}|pw}\pend{}\pstart{}Bendlerstr 36\oindex{Stauffenbergstrasse@\textbf{Stauffenbergstraße}|pw}\pend{}{\bigskip}\pstart
           \noindent{}\centering{}{\pb}\textcolor{gray}{\textbf{VENEZIA\oindex{Venedig@\textbf{Venedig}|pw} – Piazzetta S. Marco\oindex{San Marco@\textbf{San Marco}|pw} dalla
                     Laguna.}}\pend
           \pstart
           \raggedleft{}{\pb}Venedig\oindex{Venedig@\textbf{Venedig}|pw}{ }25/4\pend
           \pstart
           mein lieber Paul, ich bedaure sehr Euern Besuch versäumt zu haben,
               und grüße Dich, die mir verehrte Gattin\pwindex{Goldmann, Eva Marie 27.10.1877 – 02.11.1937@\textsc{Goldmann, Eva Marie} (27.10.1877 – 02.11.1937)|pwv} und die liebe Tochter\pwindex{Goldmann, Franziska 29.05.1911 – 19.8.1963@\textsc{Goldmann, Franziska} (29.05.1911 – 19.8.1963)|pwv} aufs herzlichste.\pend
           \pstart
           Auf ein gutes Wiedersehen, sei’s in Berlin\oindex{Berlin@\textbf{Berlin}|pw}, in
                  Wien\oindex{Wien@\textbf{Wien}|pw} oder vielleicht einmal im \label{K_L02687-1v}\edtext{So\damage{\textcolor{gray}{mmer}}}{\lemma{\textnormal{\emph{Sommer}}}\Cendnote{\textnormal{Goldmann\pwindex{Goldmann, Paul 31.01.1865 – 25.09.1935@\textsc{Goldmann, Paul} (31.01.1865 – 25.09.1935), \emph{Schriftsteller, Journalist}|pwk} und Schnitzler\pwindex{Schnitzler, Arthur 15.05.1862 – 21.10.1931@\textsc{Schnitzler, Arthur} (15.05.1862 – 21.10.1931), \emph{Schriftsteller, Mediziner}|pwk} sahen sich erst am 7. 10. 1927
                  wieder.}}}\label{K_L02687-1h}?\pend
           \pstart
           Ich dürfte bis Anfang Au\damage{gus}t zu Hause bleiben.\pend
           \pstart Dein \spacefill\mbox{Arthur}\pend{}
         
         \endnumbering\mylabel{h}\end{ledgroupsized}  \newcommand{\dateiname}{L02687}\newcommand{\titel}{Arthur Schnitzler an Paul Goldmann, 25. 4. 1927}\newcommand{\editorInnen}{Martin Anton Müller und Laura Untner}%% latex-leseansicht-abspann.tex
%% Abspann für die Leseansicht.
%% Der Schalter \ifkorrekturansicht ist bereits durch den Vorspann gesetzt.

%% latex-abspann.tex
%% Gemeinsamer Abspann für Korrekturansicht und Leseansicht.
%% Setzt den Schalter \ifkorrekturansicht voraus (gesetzt in den
%% einbindenden Dateien latex-korrekturansicht-abspann.tex bzw.
%% latex-leseansicht-abspann.tex).
%% ---------------------------------------------------------------

\normalsize

% Das esempio-Environment wird nur in der Leseansicht benötigt
\ifkorrekturansicht\else
\newenvironment{esempio}[3]%
{
    \vspace{1.5ex}
    \rlap{\underline{#1}}
    \par
    \setlength{\parindent}{0cm}
    \nopagebreak
    \leftskip=#2cm
    \rightskip=#3cm
}
{
    \par
}
\fi

\doendnotes{C}
\bigskip
\vfill

\clearpage

\footnotesize

\ifkorrekturansicht
  \lohead{\textsc{register}}
\fi

% theindex-Environment neu definieren ohne reledmac
\makeatletter
\renewenvironment{theindex}{%
  \ifkorrekturansicht
    \section*{\indexname}%
  \else
    \subsubsection*{Index der erwähnten Entitäten}%
  \fi
  \setlength{\parindent}{0pt}%
  \setlength{\parskip}{0pt plus 0.3pt}%
  \let\item\@idxitem
}{%
  \ifkorrekturansicht\clearpage\fi
}
\makeatother

\IfFileExists{\jobname-pw.ind}{\input{\jobname-pw.ind}}{}

% Quellenangabe nur in der Leseansicht
\ifkorrekturansicht\else
% Fallback-Definitionen, falls die .tex-Datei \titel etc. nicht gesetzt hat
\providecommand{\titel}{}
\providecommand{\editorInnen}{}
\providecommand{\dateiname}{\jobname}

\vspace{3cm}

\vfill

\footnotesize
\textsc{Quelle}: \titel. Herausgegeben von {\editorInnen}. In: \emph{Arthur Schnitzler: Briefwechsel mit Autorinnen und Autoren}.
 Digitale Edition, https://schnitzler-briefe.acdh.oeaw.ac.at/{\dateiname}.html (Stand \today)
\fi

\end{document}


      