%% latex-leseansicht-vorspann.tex
%% Vorspann für die Leseansicht.
%% Lädt die gemeinsame Datei latex-vorspann.tex mit nicht gesetztem Schalter.

\newif\ifkorrekturansicht
\korrekturansichtfalse

\input{../tex-inputs/latex-vorspann}


\section[Arthur Schnitzler an Paul Goldmann, 25. 4. 1927]{L02687 Arthur Schnitzler an Paul Goldmann, 25. 4. 1927}
\nopagebreak\mylabel{L02687v}
\rehead{ }\normalsize\beginnumbering\briefempfaengerindex{Goldmann, Paul@\textsc{Goldmann, Paul}!zzzSchnitzler, Arthur@\emph{von Arthur Schnitzler}!1927-04-251@{25. 4. 1927}|(be}
\toendnotes[C]{\smallbreak\pagebreak[2]}
\correspDesc{Versand  durch Arthur Schnitzler am 25. 4. 1927 in Venedig
\newline{}Erhalt  durch Paul Goldmann im Zeitraum [26. 4. 1927
                  – 30. 4. 1927?] in Berlin}\toendnotes[C]{\smallbreak}
\Standort{DLA, A:Schnitzler, HS85.1.5681.}
\physDesc{Bildpostkarte, Fotokopie, 337 Zeichen
\newline{}Handschrift: Bleistift, lateinische Kurrent
\newline{}Versand: Stempel: »\nobreak{}\oindex{Stazione di Venezia Santa Lucia@\textbf{Stazione di Venezia Santa Lucia}, \emph{Bahnhofsgebäude}|pwk}Venezia Ferrovia, 25. IV 1927, 22–23\nobreak{}«.  
\newline{}Zusatz: Von den Korrespondenzstücken Schnitzlers an Goldmann\pwindex{Goldmann, Paul 31.\,1.\,1865 Breslau – 25.\,9.\,1935 Wien@\textsc{Goldmann, Paul} (31.\,1.\,1865 Breslau – 25.\,9.\,1935 Wien), \emph{Schriftsteller, Journalist}|pw} fehlt weitgehend jede Spur. In der Edition von
                                    Ritterlichkeit\pwindex{Schnitzler, Arthur 15. 5. 1862 Wien – 21. 10. 1931 ebd.@\textsc{Schnitzler, Arthur} (15. 5. 1862 Wien – 21. 10. 1931 ebd.), \emph{Schriftsteller, Mediziner}!Ritterlichkeit@\strich\emph{Ritterlichkeit}|pw}
                                    (1975) schreibt die Herausgeberin Rena R. Schlein\pwindex{Schlein, Rena R. *~20.\,6.\,1919 Wien@\textsc{Schlein, Rena R.} (*~20.\,6.\,1919 Wien)|pw}: »Zwei Telegramme
                                    und ein Brief Schnitzlers an Goldmann\pwindex{Goldmann, Paul 31.\,1.\,1865 Breslau – 25.\,9.\,1935 Wien@\textsc{Goldmann, Paul} (31.\,1.\,1865 Breslau – 25.\,9.\,1935 Wien), \emph{Schriftsteller, Journalist}|pw} wurden mir
                                    von Dr. Leo P. Reckford\pwindex{Reckford, Leo P. 3.\,5.\,1903 Wien – 19.\,10.\,1988 Manhattan@\textsc{Reckford, Leo P.} (3.\,5.\,1903 Wien – 19.\,10.\,1988 Manhattan), \emph{Laryngologe}|pw},
                                    der diese Dokumente von der Familie Goldmanns\pwindex{Goldmann, Paul 31.\,1.\,1865 Breslau – 25.\,9.\,1935 Wien@\textsc{Goldmann, Paul} (31.\,1.\,1865 Breslau – 25.\,9.\,1935 Wien), \emph{Schriftsteller, Journalist}|pw} zum Geschenk bekam, für meine
                                    Arbeit zur Verfügung gestellt« (S. 1). Reckford\pwindex{Reckford, Leo P. 3.\,5.\,1903 Wien – 19.\,10.\,1988 Manhattan@\textsc{Reckford, Leo P.} (3.\,5.\,1903 Wien – 19.\,10.\,1988 Manhattan), \emph{Laryngologe}|pw} starb 1988, seine
                                 Nachkommen haben keine Kenntnis von diesen (und etwaigen weiteren)
                                 Korrespondenzstücken und sie sind auch nicht auffindbar. Rena R. Schlein\pwindex{Schlein, Rena R. *~20.\,6.\,1919 Wien@\textsc{Schlein, Rena R.} (*~20.\,6.\,1919 Wien)|pw} kam
                                    1919 zur Welt. Ein Kontakt konnte nicht hergestellt
                                 werden. Die vorliegende Schwarz-Weiß-Fotokopie wird im Nachlass Schnitzlers zusammen mit
                                 Kopien von zwei der drei in Ritterlichkeit\pwindex{Schnitzler, Arthur 15. 5. 1862 Wien – 21. 10. 1931 ebd.@\textsc{Schnitzler, Arthur} (15. 5. 1862 Wien – 21. 10. 1931 ebd.), \emph{Schriftsteller, Mediziner}!Ritterlichkeit@\strich\emph{Ritterlichkeit}|pw} abgedruckten Korrespondenzstücke
                                 aufbewahrt. Das deutet darauf hin, dass auch diese Postkarte zu
                                 einem bestimmten Zeitpunkt im Besitz Reckfords\pwindex{Reckford, Leo P. 3.\,5.\,1903 Wien – 19.\,10.\,1988 Manhattan@\textsc{Reckford, Leo P.} (3.\,5.\,1903 Wien – 19.\,10.\,1988 Manhattan), \emph{Laryngologe}|pw} gewesen ist. }\toendnotes[C]{\smallbreak}\pstart{}{\pb}\label{T_L02687-1v}\edtext{\textcolor{gray}{\textbf{A. S.}}}{\lemma{\textnormal{\emph{A. S.}}}\Cendnote{\textnormal{ovaler Absenderkleber}}}\label{T_L02687-1}\pend{}\pstart{}\textcolor{gray}{\textbf{WIEN, XVIII.}}\oindex{XVIII., Währing@\textbf{XVIII., Währing}, \emph{Verwaltungsgebiet}|pw}\pend{}\pstart{}\textcolor{gray}{\textbf{STERNWARTESTR. 71}}\oindex{Wien@\textbf{Wien}!XVIII., Währing@\textbf{XVIII., Währing}!Sternwartestraße 71@\textbf{Sternwartestraße 71}, \emph{Wohngebäude}|pw}\pend{}{\bigskip}\pstart{}{\pb}Germania\oindex{Deutschland@\textbf{Deutschland}|pw}\pend{}\pstart{}Hrn Dr Paul Goldmann\pend{}\pstart{}Berlin W\oindex{Berlin@\textbf{Berlin}, \emph{Hauptstadt}|pw}\pend{}\pstart{}Bendlerstr 36\oindex{Stauffenbergstraße@\textbf{Stauffenbergstraße}, \emph{Straße}|pw}\pend{}{\bigskip}
\pstart
           \noindent{}\centering{}{\pb}\textcolor{gray}{\textbf{VENEZIA\oindex{Venedig@\textbf{Venedig}|pw} – Piazzetta S. Marco\oindex{Piazza San Marco@\textbf{Piazza San Marco}, \emph{Platz}|pw} dalla
                  Laguna.}}\pend
           \vspace{1em}
\pstart
           \raggedleft{}{\pb}Venedig\oindex{Venedig@\textbf{Venedig}|pw}{ }25/4\pend
           \vspace{0.5em}
\pstart
           mein lieber Paul, ich bedaure sehr Euern Besuch versäumt zu haben,
               und grüße Dich, die mir verehrte Gattin\pwindex{Goldmann, Eva Marie 27.\,10.\,1877 Wien – 2.\,11.\,1937 ebd.@\textsc{Goldmann, Eva Marie} (27.\,10.\,1877 Wien – 2.\,11.\,1937 ebd.)|pwv} und die liebe Tochter\pwindex{Goldmann, Franziska 29.\,5.\,1911 Berlin – 19.\,8.\,1963 Rio de Janeiro@\textsc{Goldmann, Franziska} (29.\,5.\,1911 Berlin – 19.\,8.\,1963 Rio de Janeiro), \emph{Schauspielerin}|pwv} aufs herzlichste.\pend
           
\pstart
           Auf ein gutes Wiedersehen, sei’s in Berlin\oindex{Berlin@\textbf{Berlin}, \emph{Hauptstadt}|pw}, in
                  Wien\oindex{Wien@\textbf{Wien}, \emph{Verwaltungsgebiet}|pw} oder vielleicht einmal im \label{K_L02687-1v}\edtext{So\damage{\textcolor{gray}{mmer}}}{\lemma{\textnormal{\emph{Sommer}}}\Cendnote{\textnormal{Goldmann\pwindex{Goldmann, Paul 31.\,1.\,1865 Breslau – 25.\,9.\,1935 Wien@\textsc{Goldmann, Paul} (31.\,1.\,1865 Breslau – 25.\,9.\,1935 Wien), \emph{Schriftsteller, Journalist}|pwk} und Schnitzler sahen sich erst am 7. 10. 1927
                  wieder.}}}\label{K_L02687-1}?\pend
           
\pstart
           Ich dürfte bis Anfang Au\damage{gus}t zu Hause bleiben.\pend
           \pstart Dein \spacefill\mbox{Arthur}\pend{}\selectlanguage{ngerman}\endnumbering\briefempfaengerindex{Goldmann, Paul@\textsc{Goldmann, Paul}!zzzSchnitzler, Arthur@\emph{von Arthur Schnitzler}!1927-04-251@{25. 4. 1927}|)be}\mylabel{L02687h}  \newcommand{\dateiname}{L02687}\newcommand{\titel}{Arthur Schnitzler an Paul Goldmann, 25. 4. 1927}\newcommand{\editorInnen}{Martin Anton Müller und Laura Untner}%% latex-leseansicht-abspann.tex
%% Abspann für die Leseansicht.
%% Der Schalter \ifkorrekturansicht ist bereits durch den Vorspann gesetzt.

%% latex-abspann.tex
%% Gemeinsamer Abspann für Korrekturansicht und Leseansicht.
%% Setzt den Schalter \ifkorrekturansicht voraus (gesetzt in den
%% einbindenden Dateien latex-korrekturansicht-abspann.tex bzw.
%% latex-leseansicht-abspann.tex).
%% ---------------------------------------------------------------

\normalsize

% Das esempio-Environment wird nur in der Leseansicht benötigt
\ifkorrekturansicht\else
\newenvironment{esempio}[3]%
{
    \vspace{1.5ex}
    \rlap{\underline{#1}}
    \par
    \setlength{\parindent}{0cm}
    \nopagebreak
    \leftskip=#2cm
    \rightskip=#3cm
}
{
    \par
}
\fi

\doendnotes{C}
\bigskip
\vfill

\clearpage

\footnotesize

\ifkorrekturansicht
  \lohead{\textsc{register}}
\fi

% theindex-Environment neu definieren ohne reledmac
\makeatletter
\renewenvironment{theindex}{%
  \ifkorrekturansicht
    \section*{\indexname}%
  \else
    \subsubsection*{Index der erwähnten Entitäten}%
  \fi
  \setlength{\parindent}{0pt}%
  \setlength{\parskip}{0pt plus 0.3pt}%
  \let\item\@idxitem
}{%
  \ifkorrekturansicht\clearpage\fi
}
\makeatother

\IfFileExists{\jobname-pw.ind}{\input{\jobname-pw.ind}}{}

% Quellenangabe nur in der Leseansicht
\ifkorrekturansicht\else
% Fallback-Definitionen, falls die .tex-Datei \titel etc. nicht gesetzt hat
\providecommand{\titel}{}
\providecommand{\editorInnen}{}
\providecommand{\dateiname}{\jobname}

\vspace{3cm}

\vfill

\footnotesize
\textsc{Quelle}: \titel. Herausgegeben von {\editorInnen}. In: \emph{Arthur Schnitzler: Briefwechsel mit Autorinnen und Autoren}.
 Digitale Edition, https://schnitzler-briefe.acdh.oeaw.ac.at/{\dateiname}.html (Stand \today)
\fi

\end{document}


