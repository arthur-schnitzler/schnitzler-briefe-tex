%% latex-korrekturansicht-vorspann.tex
%% Vorspann für die Korrekturansicht.
%% Lädt die gemeinsame Datei latex-vorspann.tex mit gesetztem Schalter.

\newif\ifkorrekturansicht
\korrekturansichttrue

\input{../tex-inputs/latex-vorspann}


\section[Arthur Schnitzler an Paul Goldmann, 25. 4. 1927]{L02687 Arthur Schnitzler an Paul Goldmann, 25. 4. 1927}
\nopagebreak\mylabel{L02687v}
\rehead{ }\normalsize\beginnumbering\briefempfaengerindex{Goldmann, Paul@\textsc{Goldmann, Paul}!zzzSchnitzler, Arthur@\emph{von Arthur Schnitzler}!1927-04-251@{25. 4. 1927}|(be}
\toendnotes[C]{\smallbreak\pagebreak[2]}\Standort{DLA, A:Schnitzler, HS85.1.5681.}
\physDesc{Bildpostkarte, Fotokopie337 Zeichen
\newline{}Handschrift: Bleistift, lateinische Kurrent
\newline{}Versand: Stempel: »\nobreak{}\oindex{Stazione di Venezia Santa Lucia@\textbf{Stazione di Venezia Santa Lucia}, \emph{Bahnhofsgebäude (K.BHF)}|pwk}Venezia Ferrovia, 25. IV 1927, 22–23\nobreak{}«.  
\newline{}Zusatz: Von den Korrespondenzstücken Schnitzlers an Goldmann\pwindex{Goldmann, Paul 31.01.1865 – 25.09.1935@\textsc{Goldmann, Paul} (31.01.1865 – 25.09.1935), \emph{Schriftsteller/Schriftstellerin, Journalist/Journalistin}|pw} fehlt weitgehend jede Spur. In der Edition von
                                    Ritterlichkeit\pwindex{Ritterlichkeit@\emph{Ritterlichkeit}|pw}
                                    (1975) schreibt die Herausgeberin Rena R. Schlein\pwindex{Schlein, Rena R. *~1919-06-20@\textsc{Schlein, Rena R.} (*~1919-06-20)|pw}: »Zwei Telegramme
                                    und ein Brief Schnitzlers an Goldmann\pwindex{Goldmann, Paul 31.01.1865 – 25.09.1935@\textsc{Goldmann, Paul} (31.01.1865 – 25.09.1935), \emph{Schriftsteller/Schriftstellerin, Journalist/Journalistin}|pw} wurden mir
                                    von Dr. Leo P. Reckford\pwindex{Reckford, Leo P. 1903-05-03 – 1988-10-19@\textsc{Reckford, Leo P.} (1903-05-03 – 1988-10-19), \emph{Laryngologe/Laryngologin}|pw},
                                    der diese Dokumente von der Familie Goldmanns\pwindex{Goldmann, Paul 31.01.1865 – 25.09.1935@\textsc{Goldmann, Paul} (31.01.1865 – 25.09.1935), \emph{Schriftsteller/Schriftstellerin, Journalist/Journalistin}|pw} zum Geschenk bekam, für meine
                                    Arbeit zur Verfügung gestellt« (S. 1). Reckford\pwindex{Reckford, Leo P. 1903-05-03 – 1988-10-19@\textsc{Reckford, Leo P.} (1903-05-03 – 1988-10-19), \emph{Laryngologe/Laryngologin}|pw} starb 1988, seine
                                 Nachkommen haben keine Kenntnis von diesen (und etwaigen weiteren)
                                 Korrespondenzstücken und sie sind auch nicht auffindbar. Rena R. Schlein\pwindex{Schlein, Rena R. *~1919-06-20@\textsc{Schlein, Rena R.} (*~1919-06-20)|pw} kam
                                    1919 zur Welt. Ein Kontakt konnte nicht hergestellt
                                 werden. Die vorliegende Schwarz-Weiß-Fotokopie wird im Nachlass Schnitzlers zusammen mit
                                 Kopien von zwei der drei in Ritterlichkeit\pwindex{Ritterlichkeit@\emph{Ritterlichkeit}|pw} abgedruckten Korrespondenzstücke
                                 aufbewahrt. Das deutet darauf hin, dass auch diese Postkarte zu
                                 einem bestimmten Zeitpunkt im Besitz Reckfords\pwindex{Reckford, Leo P. 1903-05-03 – 1988-10-19@\textsc{Reckford, Leo P.} (1903-05-03 – 1988-10-19), \emph{Laryngologe/Laryngologin}|pw} gewesen ist. }\toendnotes[C]{\smallbreak}\pstart{}{\pb}\label{T_L02687-1v}\edtext{\textcolor{gray}{\textbf{A. S.}}}{\lemma{\textnormal{\emph{A. S.}}}\Cendnote{\textnormal{ovaler Absenderkleber}}}\label{T_L02687-1}\pend{}\pstart{}\textcolor{gray}{\textbf{WIEN, XVIII.}}\oindex{XVIII., Waehring@\textbf{XVIII., Währing}, \emph{A.ADM3}|pw}\pend{}\pstart{}\textcolor{gray}{\textbf{STERNWARTESTR. 71}}\oindex{Sternwartestrasse 71@\textbf{Sternwartestraße 71}, \emph{Wohngebäude (K.WHS)}|pw}\pend{}{\bigskip}\pstart{}{\pb}Germania\oindex{Deutschland@\textbf{Deutschland}, \emph{A.PCLI}|pw}\pend{}\pstart{}Hrn Dr Paul Goldmann\pend{}\pstart{}Berlin W\oindex{Berlin@\textbf{Berlin}, \emph{P.PPLC}|pw}\pend{}\pstart{}Bendlerstr 36\oindex{Stauffenbergstrasse@\textbf{Stauffenbergstraße}, \emph{Straße (K.STR)}|pw}\pend{}{\bigskip}
\pstart
           \noindent{}\centering{}{\pb}\textcolor{gray}{\textbf{VENEZIA\oindex{Venedig@\textbf{Venedig}, \emph{P.PPLA}|pw} – Piazzetta S. Marco\oindex{Piazza San Marco@\textbf{Piazza San Marco}, \emph{S.SQR}|pw} dalla
                  Laguna.}}\pend
           \vspace{1em}
\pstart
           \raggedleft{}{\pb}Venedig\oindex{Venedig@\textbf{Venedig}, \emph{P.PPLA}|pw}{ }25/4\pend
           \vspace{0.5em}
\pstart
           mein lieber Paul, ich bedaure sehr Euern Besuch versäumt zu haben,
               und grüße Dich, die mir verehrte Gattin\pwindex{Goldmann, Eva Marie 27.10.1877 – 02.11.1937@\textsc{Goldmann, Eva Marie} (27.10.1877 – 02.11.1937)|pwv} und die liebe Tochter\pwindex{Goldmann, Franziska 1911-05-29 – 1963-08-19@\textsc{Goldmann, Franziska} (1911-05-29 – 1963-08-19), \emph{Schauspieler/Schauspielerin}|pwv} aufs herzlichste.\pend
           
\pstart
           Auf ein gutes Wiedersehen, sei’s in Berlin\oindex{Berlin@\textbf{Berlin}, \emph{P.PPLC}|pw}, in
                  Wien\oindex{Wien@\textbf{Wien}, \emph{A.ADM2}|pw} oder vielleicht einmal im \label{K_L02687-1v}\edtext{So\damage{\textcolor{gray}{mmer}}}{\lemma{\textnormal{\emph{Sommer}}}\Cendnote{\textnormal{Goldmann\pwindex{Goldmann, Paul 31.01.1865 – 25.09.1935@\textsc{Goldmann, Paul} (31.01.1865 – 25.09.1935), \emph{Schriftsteller/Schriftstellerin, Journalist/Journalistin}|pwk} und Schnitzler sahen sich erst am 7. 10. 1927
                  wieder.}}}\label{K_L02687-1}?\pend
           
\pstart
           Ich dürfte bis Anfang Au\damage{gus}t zu Hause bleiben.\pend
           \pstart Dein \spacefill\mbox{Arthur}\pend{}\selectlanguage{ngerman}\endnumbering\briefempfaengerindex{Goldmann, Paul@\textsc{Goldmann, Paul}!zzzSchnitzler, Arthur@\emph{von Arthur Schnitzler}!1927-04-251@{25. 4. 1927}|)be}\mylabel{L02687h}  \normalsize

\doendnotes{C}
\bigskip
\vfill

\clearpage

\footnotesize

\lohead{\textsc{register}}

% Definiere theindex-Environment komplett neu ohne reledmac
\makeatletter
\renewenvironment{theindex}{%
  \section*{\indexname}%
  \setlength{\parindent}{0pt}%
  \setlength{\parskip}{0pt plus 0.3pt}%
  \let\item\@idxitem
}{%
  \clearpage
}
\makeatother

\IfFileExists{\jobname-pw.ind}{\input{\jobname-pw.ind}}{}

\end{document}

      