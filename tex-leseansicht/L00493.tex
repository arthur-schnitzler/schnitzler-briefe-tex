%% latex-leseansicht-vorspann.tex
%% Vorspann für die Leseansicht.
%% Lädt die gemeinsame Datei latex-vorspann.tex mit nicht gesetztem Schalter.

\newif\ifkorrekturansicht
\korrekturansichtfalse

\input{../tex-inputs/latex-vorspann}


         
         \renewcommand{\erwaehntePersonen}{Personen: Hermann Bahr, Edmund Boesel, Arnold Böcklin,  Franz von Assisi, Paul Horn,  Poseidippos,  Rufinus, Adele Sandrock, Heinrich Schnitzler, Olga Schnitzler}
         \renewcommand{\erwaehnteInstitutionen}{Institutionen: Philipp Reclam jun.}
         \renewcommand{\erwaehnteOrte}{Orte: Borromäische Inseln, Garda, Italien, Manerba del Garda, Riva del Garda, Salò, Wien}
         \renewcommand{\erwaehnteWerke}{Werke: Anthologie lyrischer und epigrammatischer Dichtungen der alten Griechen, Das Kind, Das Märchen. Schauspiel in drei Aufzügen, Die drei Elixire, Die kleine Komödie, Liebelei. Schauspiel in drei Akten, Spiegelbild der Freundschaft, Vorzug der Magd vor der vornehmen Frau, [Wähne, Philänis…]}
               \section[Richard Beer-Hofmann an Arthur Schnitzler, 24. 9. 1895]{ Richard Beer-Hofmann an Arthur Schnitzler, 24. 9. 1895}\nopagebreak\mylabel{v}\rehead{ }\begin{ledgroupsized}[t]{13cm}\normalsize\beginnumbering \toendnotes[C]{\smallbreak\pagebreak[2]} \Standort{DLA, A:Schnitzler, HS.NZ85.1.5713, S. 43–48.}
\physDesc{Brief, maschinenschriftliche Abschrift, 5 Blätter, 5 Seiten, 4797 Zeichen
\newline{}Schreibmaschine
\newline{}Handschrift: Bleistift, deutsche Kurrent (\noindent{}geringfügige Korrekturen von unbekannter Hand)
\newline{}Zusatz: Original nicht nachweisbar, vgl. die handschriftliche Angabe von
                                    Heinrich Schnitzler\pwindex{Schnitzler, Heinrich 09.08.1902 – 12.07.1982@\textsc{Schnitzler, Heinrich} (09.08.1902 – 12.07.1982), \emph{Regisseur, Schauspieler}|pw} auf
                                 der Mappe B 8 mit den restlichen Originalen der Briefe:
                                    »1 Brief (vom 24. 9. 1895) für Mutter\pwindex{Schnitzler, Olga 17.01.1882 – 13.01.1970@\textsc{Schnitzler, Olga} (17.01.1882 – 13.01.1970), \emph{Schauspielerin, Sängerin}|pwv}
                                    entnommen. H. S.\pwindex{Schnitzler, Heinrich 09.08.1902 – 12.07.1982@\textsc{Schnitzler, Heinrich} (09.08.1902 – 12.07.1982), \emph{Regisseur, Schauspieler}|pw}15. 8. 36.« 
\newline{}Editorischer Hinweis: Die Korrekturen wurden eingearbeitet. }\buchAbdrucke{\weitereDrucke{1) \emph{Literatur und Kunst.} In: \emph{Neue Zürcher Zeitung}, 2. 10. 1955, S. 4.} \weitereDrucke{2) \pwindex{Schnitzler, Olga 17.01.1882 – 13.01.1970@\textsc{Schnitzler, Olga} (17.01.1882 – 13.01.1970), \emph{Schauspielerin, Sängerin}!Spiegelbild der Freundschaft1962@\strich\emph{Spiegelbild der Freundschaft} {[}1962{]}|pwk}Olga Schnitzler: \emph{Spiegelbild der Freundschaft}. Salzburg: \emph{Residenz-Verlag} 1962, S. 141–142.} \weitereDrucke{3) Arthur Schnitzler, Richard Beer-Hofmann: \emph{Briefwechsel 1891–1931}. Hg. Konstanze Fliedl. Wien, Zürich: \emph{Europaverlag} 1992, S. 86–88.} }\toendnotes[C]{\smallbreak}\pstart
           \raggedleft{}{\pb}\label{K_L00493-1v}\edtext{24. 9. 95}{\lemma{\textnormal{\emph{24. 9. 95}}}\Cendnote{\textnormal{Am 26. 9. 1895 antwortet Schnitzler\pwindex{Schnitzler, Arthur 15.05.1862 – 21.10.1931@\textsc{Schnitzler, Arthur} (15.05.1862 – 21.10.1931), \emph{Schriftsteller, Mediziner}|pwk} auf den ersten Brief vom 24. 9. 1895, nicht aber auf diesen. Da er
                     nicht im Original erhalten ist, ist die Möglichkeit gegeben, dass er zu einem
                     anderen Zeitpunkt verfasst ist.}}}\label{K_L00493-1h}.\pend
           \pstart
           Lieber Arthur! Dies schreib ich Ihnen, im Boote liegend, während man
               mich zu einer Insel rudert, auf der ein Jupitertempel stand, aus dem der heilige Franciscus von {\pb}Assisi\pwindex{Franz von Assisi 1182-06-24 – 1226@\textsc{Franz von Assisi} (1182-06-24 – 1226), \emph{Heiliger, Mönch}|pw} – mein Franciscus\pwindex{Franz von Assisi 1182-06-24 – 1226@\textsc{Franz von Assisi} (1182-06-24 – 1226), \emph{Heiliger, Mönch}|pw} – ein
               Kloster gemacht hat. Zugleich lese ich in einem \label{K_L00493-2v}\edtext{Buch\pwindex{Anthologie lyrischer und epigrammatischer Dichtungen der alten Griechen1884@\emph{Anthologie lyrischer und epigrammatischer Dichtungen der alten Griechen} {[}1884{]}|pwv}}{\lemma{\textnormal{\emph{Buch}}}\Cendnote{\textnormal{\emph{Anthologie lyrischer und epigrammatischer
                        Dichtungen der alten Griechen}\pwindex{Anthologie lyrischer und epigrammatischer Dichtungen der alten Griechen1884@\emph{Anthologie lyrischer und epigrammatischer Dichtungen der alten Griechen} {[}1884{]}|pwk}. Hg. Edmund Boesel\pwindex{Boesel, Edmund @\textsc{Boesel, Edmund}|pwk}. Stuttgart: \emph{Philipp
                        Reclam jun.}\orgindex{Philipp Reclam jun.@Philipp Reclam jun.|pwk}{ }{[}1884{]}.}}}\label{K_L00493-2h} wunderschöne Sachen – wie das Buch\pwindex{Anthologie lyrischer und epigrammatischer Dichtungen der alten Griechen1884@\emph{Anthologie lyrischer und epigrammatischer Dichtungen der alten Griechen} {[}1884{]}|pwv} aber heisst schreibe ich hier nicht, denn der Name
               könnte Ihnen entgleiten, und der B\pwindex{Bahr, Hermann 19.07.1863 – 15.01.1934@\textsc{Bahr, Hermann} (19.07.1863 – 15.01.1934), \emph{Schriftsteller, Kritiker}|pw}{\dots} mittelst 3–4 Ausschrotartikeln es einem ruinieren und
               verekeln, aber es ist sehr schön. Im dritten Jahrhundert vor Christi Geburt schreibt
               ein Herr Posidippus\pwindex{Poseidippos um 310 v. u. Z. – um 240 v. u. Z.@\textsc{Poseidippos} (um 310 v. u. Z. – um 240 v. u. Z.), \emph{Epigrammatiker}|pw} – ohne »Märchen\pwindex{Schnitzler, Arthur 15.05.1862 – 21.10.1931@\textsc{Schnitzler, Arthur} (15.05.1862 – 21.10.1931), \emph{Schriftsteller, Mediziner}!Maerchen. Schauspiel in drei Aufzuegen1893-12-01@\strich\emph{Das Märchen. Schauspiel in drei Aufzügen} {[}1893-12-01{]}|pw}« und »Elixire\pwindex{Schnitzler, Arthur 15.05.1862 – 21.10.1931@\textsc{Schnitzler, Arthur} (15.05.1862 – 21.10.1931), \emph{Schriftsteller, Mediziner}!drei Elixire1893@\strich\emph{Die drei Elixire} {[}1893{]}|pw}«-Schmerzen – heiter \label{K_L00493-3v}\edtext{konstatirend}{\lemma{\textnormal{\emph{konstatirend}}}\Cendnote{\textnormal{Das Gedicht findet sich
                  in Boesel\pwindex{Boesel, Edmund @\textsc{Boesel, Edmund}|pwk}s \emph{Anthologie}\pwindex{Anthologie lyrischer und epigrammatischer Dichtungen der alten Griechen1884@\emph{Anthologie lyrischer und epigrammatischer Dichtungen der alten Griechen} {[}1884{]}|pwk} auf den S. 298–299.}}}\label{K_L00493-3h}:\pend
           \stanza{}»Wähne, \label{T_L00493-1v}\edtext{Philänis}{\lemma{\textnormal{\emph{Philänis}}}\Cendnote{\textnormal{Die Abschrift hat »Philanis«, nach der
                        gedruckten Zitatvorlage korrigiert.}}}\label{T_L00493-1h}, nicht mich durch lockende
                     Thränen zu täuschen!\pwindex{Poseidippos um 310 v. u. Z. – um 240 v. u. Z.@\textsc{Poseidippos} (um 310 v. u. Z. – um 240 v. u. Z.), \emph{Epigrammatiker}!Waehne, Philaenis…]None@\strich\emph{[Wähne, Philänis…]} {[}None{]}|pwv}\newverse{}»Freilich, ich weiss ja, du
                     liebst inniger keinen als mich,\pwindex{Poseidippos um 310 v. u. Z. – um 240 v. u. Z.@\textsc{Poseidippos} (um 310 v. u. Z. – um 240 v. u. Z.), \emph{Epigrammatiker}!Waehne, Philaenis…]None@\strich\emph{[Wähne, Philänis…]} {[}None{]}|pwv}\newverse{}»Keinen, – so lange du neben mir
                     liegst. Doch hat dich ein andrer,\pwindex{Poseidippos um 310 v. u. Z. – um 240 v. u. Z.@\textsc{Poseidippos} (um 310 v. u. Z. – um 240 v. u. Z.), \emph{Epigrammatiker}!Waehne, Philaenis…]None@\strich\emph{[Wähne, Philänis…]} {[}None{]}|pwv}\newverse{}»Nun, so liebest du den inniger
                     wieder als mich.\pwindex{Poseidippos um 310 v. u. Z. – um 240 v. u. Z.@\textsc{Poseidippos} (um 310 v. u. Z. – um 240 v. u. Z.), \emph{Epigrammatiker}!Waehne, Philaenis…]None@\strich\emph{[Wähne, Philänis…]} {[}None{]}|pwv}«\stanzaend{}\pstart
           Sollten Ihnen Paul Hörne\pwindex{Horn, Paul 13.02.1867 – 18.01.1936@\textsc{Horn, Paul} (13.02.1867 – 18.01.1936), \emph{Fabrikant}|pw} die »kleine Comödie\pwindex{Schnitzler, Arthur 15.05.1862 – 21.10.1931@\textsc{Schnitzler, Arthur} (15.05.1862 – 21.10.1931), \emph{Schriftsteller, Mediziner}!kleine Komoedie1895-08-01@\strich\emph{Die kleine Komödie} {[}1895-08-01{]}|pw}«, verheirathete Frauen mit dem
               Schmerz anständig zu sein, das »kleine Mädel« der »Liebelei\pwindex{Schnitzler, Arthur 15.05.1862 – 21.10.1931@\textsc{Schnitzler, Arthur} (15.05.1862 – 21.10.1931), \emph{Schriftsteller, Mediziner}!Liebelei. Schauspiel in drei Akten1895-10-09@\strich\emph{Liebelei. Schauspiel in drei Akten} {[}1895-10-09{]}|pw}« (um Gotteswillen, wie ist die Sandrock\pwindex{Sandrock, Adele 1863-08-19 – 1937-08-30@\textsc{Sandrock, Adele} (1863-08-19 – 1937-08-30), \emph{Schauspielerin}|pw} im ersten Akt?) und mir das Dienstmädchen im »Kind\pwindex{Beer-Hofmann, Richard 1866-07-11 – 1945-09-26@\textsc{Beer-Hofmann, Richard} (1866-07-11 – 1945-09-26), \emph{Schriftsteller}!Kind1893@\strich\emph{Das Kind} {[}1893{]}|pw}« (mit Unrecht, denn die schildere ich selbst ja nicht als
               hervorragend begehrenswert) vorwerfen, dann wer{\pb}den
               wir mit Ihnen sagen »lasst uns lächeln« und folgende schöne \label{K_L00493-4v}\edtext{Verse}{\lemma{\textnormal{\emph{Verse}}}\Cendnote{\textnormal{Das Gedicht findet sich in Boesel\pwindex{Boesel, Edmund @\textsc{Boesel, Edmund}|pwk}s \emph{Anthologie}\pwindex{Anthologie lyrischer und epigrammatischer Dichtungen der alten Griechen1884@\emph{Anthologie lyrischer und epigrammatischer Dichtungen der alten Griechen} {[}1884{]}|pwk} auf den
                     S. 299–300.}}}\label{K_L00493-4h} zitieren:\pend
           \stanza{}Statt hoffärtiger Frauen
                     erwählen wir lieber die Magd uns,\pwindex{Rufinus @\textsc{Rufinus}, \emph{Schriftsteller}!Vorzug der Magd vor der vornehmen FrauNone@\strich\emph{Vorzug der Magd vor der vornehmen Frau} {[}None{]}|pwv}\newverse{}Welche den täuschenden Schein
                     üppigen Tandes verschmäht.\pwindex{Rufinus @\textsc{Rufinus}, \emph{Schriftsteller}!Vorzug der Magd vor der vornehmen FrauNone@\strich\emph{Vorzug der Magd vor der vornehmen Frau} {[}None{]}|pwv}\newverse{}Jene, die Haut umduftet von
                     Salböl, schreitet mit Hochmuth\pwindex{Rufinus @\textsc{Rufinus}, \emph{Schriftsteller}!Vorzug der Magd vor der vornehmen FrauNone@\strich\emph{Vorzug der Magd vor der vornehmen Frau} {[}None{]}|pwv}\newverse{}Prunkend einher; und Gefahr
                     bringt es, ihr liebend zu nahn.\pwindex{Rufinus @\textsc{Rufinus}, \emph{Schriftsteller}!Vorzug der Magd vor der vornehmen FrauNone@\strich\emph{Vorzug der Magd vor der vornehmen Frau} {[}None{]}|pwv} (Liebelei\pwindex{Schnitzler, Arthur 15.05.1862 – 21.10.1931@\textsc{Schnitzler, Arthur} (15.05.1862 – 21.10.1931), \emph{Schriftsteller, Mediziner}!Liebelei. Schauspiel in drei Akten1895-10-09@\strich\emph{Liebelei. Schauspiel in drei Akten} {[}1895-10-09{]}|pw}) \newverse{}Diese, geschmückt mit
                     natürlichem Reiz und Farbe, versagt dir\pwindex{Rufinus @\textsc{Rufinus}, \emph{Schriftsteller}!Vorzug der Magd vor der vornehmen FrauNone@\strich\emph{Vorzug der Magd vor der vornehmen Frau} {[}None{]}|pwv}\newverse{}Nimmer das Lager und heischt
                     nimmer ein köstlich Geschenk.\pwindex{Rufinus @\textsc{Rufinus}, \emph{Schriftsteller}!Vorzug der Magd vor der vornehmen FrauNone@\strich\emph{Vorzug der Magd vor der vornehmen Frau} {[}None{]}|pwv}\newverse{}Pyrrhus, ich ahme dir nach, du
                     edler Sohn des Achilleus,\pwindex{Rufinus @\textsc{Rufinus}, \emph{Schriftsteller}!Vorzug der Magd vor der vornehmen FrauNone@\strich\emph{Vorzug der Magd vor der vornehmen Frau} {[}None{]}|pwv}\newverse{}Der du Andromache nahmst an der
                     Hermione Statt.\pwindex{Rufinus @\textsc{Rufinus}, \emph{Schriftsteller}!Vorzug der Magd vor der vornehmen FrauNone@\strich\emph{Vorzug der Magd vor der vornehmen Frau} {[}None{]}|pwv}«\stanzaend{}\pstart
           Das ist von Rufinus\pwindex{Rufinus @\textsc{Rufinus}, \emph{Schriftsteller}|pw}. »\label{K_L00493-5v}\edtext{Zur Bestimmung der Lebenszeit des
                     Rufinus\pwindex{Rufinus @\textsc{Rufinus}, \emph{Schriftsteller}|pw} fehlt uns jeder Anhalt.\pwindex{Poseidippos um 310 v. u. Z. – um 240 v. u. Z.@\textsc{Poseidippos} (um 310 v. u. Z. – um 240 v. u. Z.), \emph{Epigrammatiker}!Waehne, Philaenis…]None@\strich\emph{[Wähne, Philänis…]} {[}None{]}|pwv}}{\lemma{\textnormal{\emph{Zur … Anhalt.}}}\Cendnote{\textnormal{Zitat von S. 247}}}\label{K_L00493-5h}« –\pend
           \pstart
           Ich war auf der Insel und wir fahren im Abendwind (man hat sechs geläutet) zurück.
               Die Insel ist herrlich. Seitdem ich Italien\oindex{Italien@\textbf{Italien}|pw} und
               solche Inseln wie die Borromäischen\oindex{Borromaeische Inseln@\textbf{Borromäische Inseln}|pw} und die
               kenne, bewundere ich Boeklin\pwindex{Boecklin, Arnold 1827-10-16 – 1901-01-16@\textsc{Böcklin, Arnold} (1827-10-16 – 1901-01-16), \emph{Maler}|pw} weniger. Wie dumm
               waren nur die Anderen, dass sie mit solchen Augen solche Schönheiten nicht sahen. Ich
               will recht oft hieher, und in den Süden, man wird ein besserer Mensch hier, alles
               liegt so weit weg, als wenn wir es von grosser Höhe klein, und uns selbst fremd unter
               uns sehen würden. {\pb}Wie widerlich ist das Gesindel, das
               mit ungezieferhafter Unruhe uns zu Hause, in Wien\oindex{Wien@\textbf{Wien}|pw}
               wieder umwimmeln wird. Aber dies Jahr sollen die Recht behalten, die mich »arrogant«
               nennen. Ich will ihnen eine Arroganz »hinlegen« (so sagen doch die Herren, die Ihnen
               die Ehre erweisen Ihr Stück\pwindex{Schnitzler, Arthur 15.05.1862 – 21.10.1931@\textsc{Schnitzler, Arthur} (15.05.1862 – 21.10.1931), \emph{Schriftsteller, Mediziner}!Liebelei. Schauspiel in drei Akten1895-10-09@\strich\emph{Liebelei. Schauspiel in drei Akten} {[}1895-10-09{]}|pwv} zu
               spielen), dass sie starr sein werden. Und meine Arroganz wird nur die sein allein zu
               sein »höflich und allein«. Auch ein Wahlspruch für den Verkehr mit Jenen. Ich denke
               mit vieler Freude auch an unser Beisammensein im Winter, und wenn wir dabei immer den
               Daumen in der hohlen Hand verbergen, »Tütü« machen, und »unberufen« sagen, und uns
               noch ängstigen tut uns vielleicht auch der Neid der Götter nichts. Heute macht die
               Tatsache, dass wir einander haben nur unser Leben schöner und wärmer, aber ich
               glaube, wenn wir einmal alt sein werden und sehr Vieles, an das wir jetzt glauben,
               weit weg von uns sein wird, werden wir einander noch viel mehr bedeuten. Aber das
               möcht ich gar nicht, dass es so kommt, {\pb}dass wir, wenn
               wir alt sind, nichts mehr haben als uns; wir sollen Greise sein mit wunderschönen
               hellen jungen Augen und seidenweichem weissen Haar, und \uuline{sehr} berühmt. So berühmt, dass sich Frauen rühmen, wenn ihre Mütter einmal
               unsere Geliebten waren, und junge Mädchen sich mühen sollen, um reizend zu erscheinen
               – und ich meine »reizend« wörtlich. Und weil wir Blumen lieb haben, und bis dahin
               auch den Wein lieben gelernt haben, kommen aus dem Süden täglich Körbe mit Obst und
               Wein und Blumen. Denn wer hinunterreist in den Süden wird an uns denken müssen, die
               wir, in einer Zeit, wo hässlich geschäftige Menschen lebten, die Reichtum und
               Anerkennung wollten und widerliche Literatur machten, die einzigen waren, die
               wussten, dass es Schönheit und Sonne und Liebe gibt, die nur genossen, und erkannt
               sein will, – nicht mehr. – Jetzt wird es aber ganz dunkel; gegen Riva\oindex{Riva del Garda@\textbf{Riva del Garda}|pw} zu liegt der See im Nebel, gegen Salò\oindex{Salo@\textbf{Salò}|pw} ist der Himmel noch rötlich, und gegen Cap Manerba\oindex{Manerba del Garda@\textbf{Manerba del Garda}|pw} steht im grünlichen Abendhimmel eine zarte silberne
                  {\pb}Sichel. Der Ruderer setzt stark ein, weil die
               Nacht kommt und mit jedem Ruderschlag sprüht mirs feucht ins Gesicht. Unendlich schön
               ists, und es wäre mir sehr leid, wenn ich jetzt ertrinken müsste. – Adieu lieber
               Arthur und grüssen Sie mir auch die, die Sie lieb haben, und die ich nicht kenne. Und
               sie hat Sie wohl jetzt noch mehr lieb als sonst, wo Sie vielleicht am Thor des
               Berühmtseins stehen, und sie wird sehr viel Herzklopfen haben, wenn das Orchester die
               Schlusstakte spielen wird. Nicht wahr! – Herzlichst Ihr\pend
           \pstart \spacefill\mbox{R.}\pend{}\pstart
           \noindent{}Es ist finster.\pend
           
         
         \endnumbering\mylabel{h}\end{ledgroupsized}  \newcommand{\dateiname}{L00493}\newcommand{\titel}{Richard Beer-Hofmann an Arthur Schnitzler, 24. 9. 1895}\newcommand{\editorInnen}{Martin Anton Müller und Gerd-Hermann Susen}%% latex-leseansicht-abspann.tex
%% Abspann für die Leseansicht.
%% Der Schalter \ifkorrekturansicht ist bereits durch den Vorspann gesetzt.

%% latex-abspann.tex
%% Gemeinsamer Abspann für Korrekturansicht und Leseansicht.
%% Setzt den Schalter \ifkorrekturansicht voraus (gesetzt in den
%% einbindenden Dateien latex-korrekturansicht-abspann.tex bzw.
%% latex-leseansicht-abspann.tex).
%% ---------------------------------------------------------------

\normalsize

% Das esempio-Environment wird nur in der Leseansicht benötigt
\ifkorrekturansicht\else
\newenvironment{esempio}[3]%
{
    \vspace{1.5ex}
    \rlap{\underline{#1}}
    \par
    \setlength{\parindent}{0cm}
    \nopagebreak
    \leftskip=#2cm
    \rightskip=#3cm
}
{
    \par
}
\fi

\doendnotes{C}
\bigskip
\vfill

\clearpage

\footnotesize

\ifkorrekturansicht
  \lohead{\textsc{register}}
\fi

% theindex-Environment neu definieren ohne reledmac
\makeatletter
\renewenvironment{theindex}{%
  \ifkorrekturansicht
    \section*{\indexname}%
  \else
    \subsubsection*{Index der erwähnten Entitäten}%
  \fi
  \setlength{\parindent}{0pt}%
  \setlength{\parskip}{0pt plus 0.3pt}%
  \let\item\@idxitem
}{%
  \ifkorrekturansicht\clearpage\fi
}
\makeatother

\IfFileExists{\jobname-pw.ind}{\input{\jobname-pw.ind}}{}

% Quellenangabe nur in der Leseansicht
\ifkorrekturansicht\else
% Fallback-Definitionen, falls die .tex-Datei \titel etc. nicht gesetzt hat
\providecommand{\titel}{}
\providecommand{\editorInnen}{}
\providecommand{\dateiname}{\jobname}

\vspace{3cm}

\vfill

\footnotesize
\textsc{Quelle}: \titel. Herausgegeben von {\editorInnen}. In: \emph{Arthur Schnitzler: Briefwechsel mit Autorinnen und Autoren}.
 Digitale Edition, https://schnitzler-briefe.acdh.oeaw.ac.at/{\dateiname}.html (Stand \today)
\fi

\end{document}


      