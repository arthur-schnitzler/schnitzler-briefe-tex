%% latex-leseansicht-vorspann.tex
%% Vorspann für die Leseansicht.
%% Lädt die gemeinsame Datei latex-vorspann.tex mit nicht gesetztem Schalter.

\newif\ifkorrekturansicht
\korrekturansichtfalse

\input{../tex-inputs/latex-vorspann}


\section[Arthur Schnitzler an Peter Altenberg, 22. 4. 1913]{L02128 Arthur Schnitzler an Peter Altenberg, 22. 4. 1913}
\nopagebreak\mylabel{L02128v}
\rehead{ }\normalsize\beginnumbering\briefempfaengerindex{Altenberg, Peter@\textsc{Altenberg, Peter}!zzzSchnitzler, Arthur@\emph{von Arthur Schnitzler}!1913-04-221@{22. 4. 1913}|(be}
\toendnotes[C]{\smallbreak\pagebreak[2]}
\correspDesc{Versand  durch Arthur Schnitzler am 22. 4. 1913 in Wien
\newline{}Erhalt  durch Peter Altenberg im Zeitraum [22. 4. 1913
                  – 26. 4. 1913?] in Wien}\toendnotes[C]{\smallbreak}
\Standort{DLA, A:Schnitzler, 85.1.237.}
\physDesc{Brief, Durchschlag, 2 Blätter, 2 Seiten, 1577 Zeichen
\newline{}Schreibmaschine
\newline{}HandschriftX2  : roter Buntstift, deutsche Kurrent (\noindent{}Beschriftung: »\textsc{Altenbg}«, »K{[}opie{]}«; Unterstreichungen)}
\buchAbdrucke{\weitereDrucke{1) Kurt Bergel: \emph{Arthur Schnitzlers unveröffentlichte Tragikomödie Das Wort.} In: \emph{Studies in Arthur Schnitzler. Centennial Commemorative
                        Volume}. Herausgegeben von Herbert W. Reichert und Herman Salinger. Chapel Hill: \emph{University of North Carolina Press} 1963, S. 22–23 (UNC Studies in the Germanic Languages and Literatures, 42).} \weitereDrucke{2) Arthur Schnitzler: \emph{Das Wort. Tragikomödie in fünf Akten. Fragment}. Aus dem Nachlaß herausgegeben und eingeleitet von Kurt Bergel. Frankfurt am Main: \emph{S. Fischer} 1966, S. 11.} \weitereDrucke{3) Arthur Schnitzler: \emph{Briefe 1913–1931}. Herausgegeben von Peter Michael Braunwarth, Richard Miklin, Susanne Pertlik und Heinrich Schnitzler. Frankfurt am Main: \emph{S. Fischer} 1984, S. 19–20.} }\toendnotes[C]{\smallbreak}
\pstart
           \raggedleft{}{\pb}22. 4. 1913.\pend
           
\pstart\center{}Lieber Peter Altenberg.\pend\vspace{0.5em}
\pstart
           Gestern habe ich also Ihren Bruder\pwindex{Engländer, Georg 3.\,4.\,1862 Wien – 10.\,4.\,1927 ebd.@\textsc{Engländer, Georg} (3.\,4.\,1862 Wien – 10.\,4.\,1927 ebd.), \emph{Privatbeamter}|pwv} gesprochen und ihm erklärt, dass Sie meiner Ueberzeugung nach die
               Anstalt gerade so gut noch in dieser Woche als später verlassen könnten, da ja die
               Möglichkeit, dass Sie sich in vollkommener Freiheit dem Alkohol wieder allzu sehr
               ergeben, in drei oder vier Wochen keine wesentlich geringere sein dürfte als heute
               oder morgen. Er scheint nun auch durchaus geneigt Sie schon in wenigen Tagen aus dem
                  Sanatorium\oindex{Wien@\textbf{Wien}!XIV., Penzing@\textbf{XIV., Penzing}!Otto-Wagner-Spital@\textbf{Otto-Wagner-Spital}, \emph{Krankenhaus}|pwv} zu nehmen, möchte
               aber gern, was auch ich sehr vernünftig finde, dass Sie wenigstens die erste Zeit auf
               dem Semmering\oindex{Semmering@\textbf{Semmering}, \emph{Verwaltungsgebiet}|pw} noch nicht in einem Hotel, sondern
               eventuell im Kurhaus\oindex{Kurhaus Semmering@\textbf{Kurhaus Semmering}, \emph{Hotel}|pw} bei Dr. Hansy\pwindex{Hansy, Franz 23.\,7.\,1865 Baden bei Wien – 25.\,5.\,1944 Wien@\textsc{Hansy, Franz} (23.\,7.\,1865 Baden bei Wien – 25.\,5.\,1944 Wien), \emph{Mediziner}|pw} zubrächten. Sollte das aber nicht durchführbar sein, so
               wäre er wohl auch mit dem Vorschlag einverstanden, den Sie mir selbst gemacht haben:
               für die ersten Tage den Ihnen sympathischen Wärter\pwindex{?? [Wärter von Peter Altenberg] @\textsc{?? [Wärter von Peter Altenberg]}|pwv} auf den Semmering\oindex{Semmering@\textbf{Semmering}, \emph{Verwaltungsgebiet}|pw} mitzunehmen {\pb}so dass doch ein gewisser
               Uebergang, der auch Ihren jetzigen Aerzten wünschenswert erscheinen dürfte, von der
               Anstaltsbehandlung zum Leben in vollkommener Freiheit geschaffen würde. Ihr Bruder\pwindex{Engländer, Georg 3.\,4.\,1862 Wien – 10.\,4.\,1927 ebd.@\textsc{Engländer, Georg} (3.\,4.\,1862 Wien – 10.\,4.\,1927 ebd.), \emph{Privatbeamter}|pwv} ist es nun einmal, der
               die volle Verantwortung für Sie übernehmen muss. In seinem Interesse liegt es gewiss
               nicht, dass Sie noch länger in der Anstalt verbleiben; wenn nun gewisse eher formelle
               Forderungen noch erfüllt werden müssen, so verlieren Sie doch darüber nicht die
               Geduld; es handelt sich ganz bestimmt nur mehr um wenige Tage. Brauchen Sie noch
               weiterhin meine Intervention eventuell bei Herrn Primarius Richter\pwindex{Richter, Karl 9.\,3.\,1862 Bruntál – 25.\,6.\,1937 Wien@\textsc{Richter, Karl} (9.\,3.\,1862 Bruntál – 25.\,6.\,1937 Wien), \emph{Mediziner, Sanatoriumsleiter}|pw}, so stehe ich Ihnen ganz zur Verfügung.\pend
           
\pstart
           Mit herzlichem Gruss{\\[\baselineskip]}Ihr\pend
           \leftskip=0em{}
\pstart
           \noindent{}Herrn Peter Altenberg, Wien\oindex{Wien@\textbf{Wien}, \emph{Verwaltungsgebiet}|pw}.\pend
           \selectlanguage{ngerman}\endnumbering\briefempfaengerindex{Altenberg, Peter@\textsc{Altenberg, Peter}!zzzSchnitzler, Arthur@\emph{von Arthur Schnitzler}!1913-04-221@{22. 4. 1913}|)be}\mylabel{L02128h}  \newcommand{\dateiname}{L02128}\newcommand{\titel}{Arthur Schnitzler an Peter Altenberg, 22. 4. 1913}\newcommand{\editorInnen}{Martin Anton Müller und Gerd-Hermann Susen}%% latex-leseansicht-abspann.tex
%% Abspann für die Leseansicht.
%% Der Schalter \ifkorrekturansicht ist bereits durch den Vorspann gesetzt.

%% latex-abspann.tex
%% Gemeinsamer Abspann für Korrekturansicht und Leseansicht.
%% Setzt den Schalter \ifkorrekturansicht voraus (gesetzt in den
%% einbindenden Dateien latex-korrekturansicht-abspann.tex bzw.
%% latex-leseansicht-abspann.tex).
%% ---------------------------------------------------------------

\normalsize

% Das esempio-Environment wird nur in der Leseansicht benötigt
\ifkorrekturansicht\else
\newenvironment{esempio}[3]%
{
    \vspace{1.5ex}
    \rlap{\underline{#1}}
    \par
    \setlength{\parindent}{0cm}
    \nopagebreak
    \leftskip=#2cm
    \rightskip=#3cm
}
{
    \par
}
\fi

\doendnotes{C}
\bigskip
\vfill

\clearpage

\footnotesize

\ifkorrekturansicht
  \lohead{\textsc{register}}
\fi

% theindex-Environment neu definieren ohne reledmac
\makeatletter
\renewenvironment{theindex}{%
  \ifkorrekturansicht
    \section*{\indexname}%
  \else
    \subsubsection*{Index der erwähnten Entitäten}%
  \fi
  \setlength{\parindent}{0pt}%
  \setlength{\parskip}{0pt plus 0.3pt}%
  \let\item\@idxitem
}{%
  \ifkorrekturansicht\clearpage\fi
}
\makeatother

\IfFileExists{\jobname-pw.ind}{\input{\jobname-pw.ind}}{}

% Quellenangabe nur in der Leseansicht
\ifkorrekturansicht\else
% Fallback-Definitionen, falls die .tex-Datei \titel etc. nicht gesetzt hat
\providecommand{\titel}{}
\providecommand{\editorInnen}{}
\providecommand{\dateiname}{\jobname}

\vspace{3cm}

\vfill

\footnotesize
\textsc{Quelle}: \titel. Herausgegeben von {\editorInnen}. In: \emph{Arthur Schnitzler: Briefwechsel mit Autorinnen und Autoren}.
 Digitale Edition, https://schnitzler-briefe.acdh.oeaw.ac.at/{\dateiname}.html (Stand \today)
\fi

\end{document}


