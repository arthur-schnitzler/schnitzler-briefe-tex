%% latex-leseansicht-vorspann.tex
%% Vorspann für die Leseansicht.
%% Lädt die gemeinsame Datei latex-vorspann.tex mit nicht gesetztem Schalter.

\newif\ifkorrekturansicht
\korrekturansichtfalse

\input{../tex-inputs/latex-vorspann}


\section[Friedrich M. Fels an Arthur Schnitzler, 19. 9. 1895]{L00488 Friedrich M. Fels an Arthur Schnitzler, 19. 9. 1895}
\nopagebreak\mylabel{L00488v}
\rehead{ }\normalsize\beginnumbering\briefempfaengerindex{Schnitzler, Arthur@\textsc{Schnitzler, Arthur}!zzzFels, Friedrich Michael@\emph{von Friedrich Michael Fels}!1895-09-192@{19. 9. 1895}|(be}
\toendnotes[C]{\smallbreak\pagebreak[2]}
\correspDesc{Versand  durch Friedrich M. Fels am 19. 9. 1895 in Zürich
\newline{}Erhalt  durch Arthur Schnitzler am 21. 9. 1895 in Wien}\toendnotes[C]{\smallbreak}
\Standort{DLA, A:Schnitzler, HS.NZ85.1.2956.}
\physDesc{Brief, 1 Blatt, 4 Seiten, 3452 Zeichen
\newline{}Handschrift: schwarze Tinte, lateinische Kurrent
\newline{}Schnitzler: 1) mit Bleistift nummeriert: »24«  2) mit rotem Buntstift eine Unterstreichung}\toendnotes[C]{\smallbreak}
\pstart
           \raggedleft{}{\pb}Zürich\oindex{Zürich@\textbf{Zürich}|pw}, am 19. September 1895\pend
           
\pstart\center{}Lieber Doktor Schnitzler!\pend\vspace{0.5em}
\pstart
           Verzeihen Sie, daſs ich Ihnen auf Ihren Ischl\oindex{Bad Ischl@\textbf{Bad Ischl}|pw}er
               Brief erst heute antworte. Ich hätte Ihnen gern Gutes von mir berichtet, doch es ist
               mir unmöglich. Es will scheinen, als ob ich gar nie zur Ruhe ko{\geminationm}en kö{\geminationn}e. Die hiesigen
               Zeitungsverhältniſse sind traurig, sehr traurig, und es ist unglaublich, wie viel
               Mühe es kostet, etwas unterzubringen. Fast so viel oder vielleicht mehr als in \substVorne{}\textsuperscript{Zürich\oindex{Zürich@\textbf{Zürich}|pw}}\substDazwischen{}Wien\oindex{Wien@\textbf{Wien}, \emph{Verwaltungsgebiet}|pw}\substHinten{}. Die Neue Zürcher Zeitung\pwindex{Neue Zürcher Zeitung@\emph{Neue Zürcher Zeitung}|pw} hat ein Doppelfeuilleton\pwindex{Fels, Friedrich Michael *~1864 Bad Dürkheim@\textsc{Fels, Friedrich Michael} (*~1864 Bad Dürkheim), \emph{Journalist}!Bulgarische Volksdichtungen@\strich\emph{Bulgarische Volksdichtungen}|pwv} von mir
               gedruckt und mir auf einen zweiten Artikel einen Vorschuſs von 50 francs gewährt;
               jetzt allerdings hat sie eine größere Bestellung bei mir gemacht, eine Reihe von
               Aufsätzen, jeder 500–600 Druckzeilen, in denen ich die Entwickelung der modernen
               deutschen Literatur darlegen soll. Das Honorar freilich ist schlecht genug: pro
               Druckzeile 8 cent 4 Kr. Andere Blätter zahlen bloß 5 cent. So habe ich einen ganzen
               Monat Theaterreferate geschrieben und am Ende 10 francs eingeheimst – hübsch,
               na?!\pend
           
\pstart
           {\pb}Gegenwärtig bin ich von einer neuen Kalamität
               heimgesucht worden. Ich bin nämlich zur Abwechslung von meiner Schweiz\oindex{Schweiz@\textbf{Schweiz}|pw}er Wirtin\pwindex{?? [Vermieterin von F. M. Fels in Zürich] @\textsc{?? [Vermieterin von F. M. Fels in Zürich]}|pwv} (– weil ich ihr die Miete 5 Tage, nachdem sie fällig war, noch nicht
               entrichten ko{\geminationn}te –) unter Zurückbehaltung meiner Sachen
               auf die Straſse gesetzt worden, und hause nun wieder so bei Beka{\geminationn}ten. Ich bin Ihnen, so dreckig mir’s auch ging, in
               diesen letzten 3 Monaten gewiſs nicht mit Bitten zur Last gefallen; ich habe gedacht,
               überhaupt nicht mehr in eine solche Lage ko{\geminationm}en zu kö{\geminationn}en. Nun ist es doch eingetreten, und ich muſs wieder an
               Ihre Güte und Freundschaft appellieren. Wären Sie imstande, zusa{\geminationm}en mit andern mir noch einmal 25 fl zu senden; seien
               Sie überzeugt, ich würde mich nicht an Sie wenden, we{\geminationn}
               ich irgend einen Ausweg wüſste. Die Beka{\geminationn}ten, die ich
               hier habe, sind alle entweder selbst vollständig auf dem Hund, oder sie sind z.Zt. in
               Ferien. We{\geminationn} es in Ihrer Macht steht, meine Bitte zu
               erfüllen, wollen Sie freundlichst einen reko{\geminationm}andierten
               Brief senden an\pend
           
\pstart
           \centering{}\uline{Dr. Friedr. M. Fels}\pend
           
\pstart
           \centering{}\uline{per Adreſse Herrn Hugo
                     Bettauer\pwindex{Bettauer, Hugo 18.\,8.\,1872 Baden bei Wien – 26.\,3.\,1925 Wien@\textsc{Bettauer, Hugo} (18.\,8.\,1872 Baden bei Wien – 26.\,3.\,1925 Wien), \emph{Schriftsteller, Journalist}|pw}}\pend
           
\pstart
           \raggedleft{}\uline{Zürich I, Rämistraſse 2\oindex{Rämistrasse@\textbf{Rämistrasse}, \emph{Straße}|pw}}\pend
           
\pstart
           {\pb}Sie haben wohl J. H. Mackay\pwindex{Mackay, John Henry 6.\,2.\,1864 Greenock – 16.\,5.\,1933 Charlottenburg@\textsc{Mackay, John Henry} (6.\,2.\,1864 Greenock – 16.\,5.\,1933 Charlottenburg), \emph{Schriftsteller}|pw}{ }ſchon gesprochen. Er ist vor ein paar Tagen nach
                  Wien\oindex{Wien@\textbf{Wien}, \emph{Verwaltungsgebiet}|pw} abgereist, um dort eine Woche zu
               verweilen, und ich habe ihm viele, viele Grüſse an Sie aufgetragen. Pollandt\pwindex{Pollandt, Max 26.\,10.\,1861 Wien – 18.\,7.\,1905 Pernitz@\textsc{Pollandt, Max} (26.\,10.\,1861 Wien – 18.\,7.\,1905 Pernitz), \emph{Schauspieler}|pw} wird diesen Winter ans hiesige Stadttheater\oindex{Stadttheater [Zürich]@\textbf{Stadttheater [Zürich]}, \emph{Theater}|pw} ko{\geminationm}en,
               dürfte wohl auch schon hier sein; doch hab ich ihn noch nicht gesehen. Am Volkstheater\oindex{Volkstheater [Zürich]@\textbf{Volkstheater [Zürich]}, \emph{Theater}|pw}{ }ſind auch Wien\oindex{Wien@\textbf{Wien}, \emph{Verwaltungsgebiet}|pw}er: die Je{\geminationn}y
                  Neuhut\pwindex{Neuhut, Jenny *~26.\,5.\,1871 Wien@\textsc{Neuhut, Jenny} (*~26.\,5.\,1871 Wien), \emph{Schauspielerin}|pw}, die Sie wohl noch aus dem Griensteidl\oindex{Wien@\textbf{Wien}!I., Innere Stadt@\textbf{I., Innere Stadt}!Café Griensteidl@\textbf{Café Griensteidl}, \emph{Kaffeehaus}|pw} ke{\geminationn}en (Salten\pwindex{Salten, Felix 6.\,9.\,1869 Budapest – 8.\,10.\,1945 Zürich@\textsc{Salten, Felix} (6.\,9.\,1869 Budapest – 8.\,10.\,1945 Zürich), \emph{Schriftsteller, Journalist, Chefredakteur}|pw} ke{\geminationn}t sie jedenfalls) und
               ein Frl. Josephine Sorger\pwindex{Sorger, Josefine *~5.\,11.\,1878 Wien@\textsc{Sorger, Josefine} (*~5.\,11.\,1878 Wien), \emph{Schauspielerin}|pw}, ein ganz
               allerliebster Käfer.\pend
           
\pstart
           Haben Sie in Wien\oindex{Wien@\textbf{Wien}, \emph{Verwaltungsgebiet}|pw} auch so abscheuliches Wetter
               gehabt? Hier hatten wir 5 Wochen keinen Regen und im Schatten 37°, in der So{\geminationn}e 47° Celsius. Es war zum aus der Haut fahren. Gottlob,
               es ists etwas kühler.\pend
           
\pstart
           Was Sie vielleicht intereſsieren wird, ich werde jetzt anfangen, Stunden zu geben:
               Literaturgeschischte u. dgl. In ein paar Tagen werde ich meine ersten Schüleri{\geminationn}en erhalten: 2 Amerikaneri{\geminationn}en\pwindex{?? [Amerikanische Studentin in Zürich 1] 1895 – 1895@\textsc{?? [Amerikanische Studentin in Zürich 1]} (1895 – 1895)|pwv}\pwindex{?? [Amerikanische Studentin in Zürich 2] 1895 – 1895@\textsc{?? [Amerikanische Studentin in Zürich 2]} (1895 – 1895)|pwv}, denen ich Deutsch beibringen soll, damit
               sie den Vorlesungen beſser folgen kö{\geminationn}en.\pend
           
\pstart
           {\pb}Ihre Novelle in Briefen\pwindex{Schnitzler, Arthur 15.\,5.\,1862 Wien – 21.\,10.\,1931 ebd.@\textsc{Schnitzler, Arthur} (15.\,5.\,1862 Wien – 21.\,10.\,1931 ebd.), \emph{Schriftsteller, Mediziner}!kleine Komödie@\strich\emph{Die kleine Komödie}|pwv} in der N. D. R.\pwindex{Neue Deutsche Rundschau@\emph{Neue Deutsche Rundschau}|pw} habe ich gelesen. Sie ist sehr hübsch, aber – Sie verzeihen mir –
               meines Erachtens auch nicht mehr. Illustrationen kö{\geminationn}en
               ihr nicht schaden.\pend
           
\pstart
           Also leben Sie wohl! verzeihen Sie meine Bitte und erfüllen Sie sie, falls Sie kö{\geminationn}en! und auf jedenfall laſsen Sie wieder einmal etwas
               von Sich hören! Beer-Hofma{\geminationn}\pwindex{Beer-Hofmann, Richard 11.\,7.\,1866 Wien – 26.\,9.\,1945 New York City@\textsc{Beer-Hofmann, Richard} (11.\,7.\,1866 Wien – 26.\,9.\,1945 New York City), \emph{Schriftsteller}|pw}, Hofma{\geminationn}sthal\pwindex{Hofmannsthal, Hugo von 1.\,2.\,1874 Wien – 15.\,7.\,1929 Rodaun@\textsc{Hofmannsthal, Hugo von} (1.\,2.\,1874 Wien – 15.\,7.\,1929 Rodaun), \emph{Schriftsteller}|pw}, Salten\pwindex{Salten, Felix 6.\,9.\,1869 Budapest – 8.\,10.\,1945 Zürich@\textsc{Salten, Felix} (6.\,9.\,1869 Budapest – 8.\,10.\,1945 Zürich), \emph{Schriftsteller, Journalist, Chefredakteur}|pw} etc. bitte ich zu
               grüſsen; vor allen aber seien Sie gegrüſst\pend
           
\pstart
           von{\\[\baselineskip]}Ihrem{\\[\baselineskip]}dankbar ergebenen{\\[\baselineskip]}\spacefill\mbox{Fels}\pend
           \leftskip=0em{}\selectlanguage{ngerman}\endnumbering\briefempfaengerindex{Schnitzler, Arthur@\textsc{Schnitzler, Arthur}!zzzFels, Friedrich Michael@\emph{von Friedrich Michael Fels}!1895-09-192@{19. 9. 1895}|)be}\mylabel{L00488h}  \newcommand{\dateiname}{L00488}\newcommand{\titel}{Friedrich M. Fels an Arthur Schnitzler, 19. 9. 1895}\newcommand{\editorInnen}{Martin Anton Müller und Gerd-Hermann Susen}%% latex-leseansicht-abspann.tex
%% Abspann für die Leseansicht.
%% Der Schalter \ifkorrekturansicht ist bereits durch den Vorspann gesetzt.

%% latex-abspann.tex
%% Gemeinsamer Abspann für Korrekturansicht und Leseansicht.
%% Setzt den Schalter \ifkorrekturansicht voraus (gesetzt in den
%% einbindenden Dateien latex-korrekturansicht-abspann.tex bzw.
%% latex-leseansicht-abspann.tex).
%% ---------------------------------------------------------------

\normalsize

% Das esempio-Environment wird nur in der Leseansicht benötigt
\ifkorrekturansicht\else
\newenvironment{esempio}[3]%
{
    \vspace{1.5ex}
    \rlap{\underline{#1}}
    \par
    \setlength{\parindent}{0cm}
    \nopagebreak
    \leftskip=#2cm
    \rightskip=#3cm
}
{
    \par
}
\fi

\doendnotes{C}
\bigskip
\vfill

\clearpage

\footnotesize

\ifkorrekturansicht
  \lohead{\textsc{register}}
\fi

% theindex-Environment neu definieren ohne reledmac
\makeatletter
\renewenvironment{theindex}{%
  \ifkorrekturansicht
    \section*{\indexname}%
  \else
    \subsubsection*{Index der erwähnten Entitäten}%
  \fi
  \setlength{\parindent}{0pt}%
  \setlength{\parskip}{0pt plus 0.3pt}%
  \let\item\@idxitem
}{%
  \ifkorrekturansicht\clearpage\fi
}
\makeatother

\IfFileExists{\jobname-pw.ind}{\input{\jobname-pw.ind}}{}

% Quellenangabe nur in der Leseansicht
\ifkorrekturansicht\else
% Fallback-Definitionen, falls die .tex-Datei \titel etc. nicht gesetzt hat
\providecommand{\titel}{}
\providecommand{\editorInnen}{}
\providecommand{\dateiname}{\jobname}

\vspace{3cm}

\vfill

\footnotesize
\textsc{Quelle}: \titel. Herausgegeben von {\editorInnen}. In: \emph{Arthur Schnitzler: Briefwechsel mit Autorinnen und Autoren}.
 Digitale Edition, https://schnitzler-briefe.acdh.oeaw.ac.at/{\dateiname}.html (Stand \today)
\fi

\end{document}


