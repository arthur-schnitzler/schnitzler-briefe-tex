%% latex-leseansicht-vorspann.tex
%% Vorspann für die Leseansicht.
%% Lädt die gemeinsame Datei latex-vorspann.tex mit nicht gesetztem Schalter.

\newif\ifkorrekturansicht
\korrekturansichtfalse

\input{../tex-inputs/latex-vorspann}


\section[Arthur Schnitzler an Hugo von Hofmannsthal, 7. 8. 1905]{L01540 Arthur Schnitzler an Hugo von Hofmannsthal, 7. 8. 1905}
\nopagebreak\mylabel{L01540v}
\rehead{ }\normalsize\beginnumbering\briefempfaengerindex{Hofmannsthal, Hugo von@\textsc{Hofmannsthal, Hugo von}!zzzSchnitzler, Arthur@\emph{von Arthur Schnitzler}!1905-08-072@{7. 8. 1905}|(be}
\toendnotes[C]{\smallbreak\pagebreak[2]}
\correspDesc{Versand  durch Arthur Schnitzler am 7. 8. 1905 in Wien
\newline{}Erhalt  durch Hugo von Hofmannsthal im Zeitraum [7. 8. 1905
                  – 11. 8. 1905?] in Wien}\toendnotes[C]{\smallbreak}
\Standort{FDH, Hs-30885,121.}
\physDesc{Brief, 1 Blatt, 4 Seiten, 1192 Zeichen
\newline{}Handschrift: Bleistift, deutsche Kurrent}
\buchAbdrucke{\weitereDrucke{Hugo von Hofmannsthal, Arthur Schnitzler: \emph{Briefwechsel}. Herausgegeben von Therese Nickl und Heinrich Schnitzler. Frankfurt am Main: \emph{S. Fischer} 1964, S. 212.} }\toendnotes[C]{\smallbreak}
\pstart
           \raggedleft{}{\pb}7. 8. 90\substVorne{}\textsuperscript{1}\substDazwischen{}5\substHinten{}\pend
           \vspace{0.5em}
\pstart
           lieber Hugo, erſtens hatte ich begreiflicherweiſe keine Ahnung, daſs
               Sie So{\geminationn}tag{ }ſchon \strikeout{fort} wieder
               fortfahren. Wieſo ich unſer Wiederſehen bis Freitag hinausſchob, werden Sie{ }ſofort
               hören. Heute Montag müſſen wir, wie{ }ſchon ein paar Tage vorherbeſti{\geminationm}t war, weil Hr Steinrück\pwindex{Steinrück, Albert 20.\,5.\,1872 Wetterburg – 11.\,2.\,1929 Berlin@\textsc{Steinrück, Albert} (20.\,5.\,1872 Wetterburg – 11.\,2.\,1929 Berlin), \emph{Schauspieler}|pw} gaſtiert, nach Mödling\oindex{Mödling@\textbf{Mödling}, \emph{Hauptstadt}|pw} –
                  Mittwoch wollten {\pb}wir, zu Heini\pwindex{Schnitzler, Heinrich 9.\,8.\,1902 Hinterbrühl – 12.\,7.\,1982 Wien@\textsc{Schnitzler, Heinrich} (9.\,8.\,1902 Hinterbrühl – 12.\,7.\,1982 Wien), \emph{Regisseur, Schauspieler}|pw}’s 3. Geburtstag in den Prater\oindex{Wien@\textbf{Wien}!II., Leopoldstadt@\textbf{II., Leopoldstadt}!Prater@\textbf{Prater}, \emph{Park}|pw}. Um aber nicht allzuſehr aus dem Arbeiten heraus zu ko{\geminationm}en (we{\geminationn} man eben daran iſt
               was abzuſchließen, \textsc{enervirt} einen das{ }ſehr wie Sie ja
               wiſſen) wollte ich zwiſchen den Reiſetagen immer einen Heimtag, und{ }ſo fiel
               naturgemäſs der Freitag erſt auf Sie. {\pb}Nun
               haben Sie indeſs wohl meine Karte erhalten, die Sie für Mittwoch nach
                  Schönbrunn\oindex{Wien@\textbf{Wien}!XIII., Hietzing@\textbf{XIII., Hietzing}!Schloss Schönbrunn@\textbf{Schloss Schönbrunn}, \emph{Schloss}|pw} bittet (da{ }ſich Heini\pwindex{Schnitzler, Heinrich 9.\,8.\,1902 Hinterbrühl – 12.\,7.\,1982 Wien@\textsc{Schnitzler, Heinrich} (9.\,8.\,1902 Hinterbrühl – 12.\,7.\,1982 Wien), \emph{Regisseur, Schauspieler}|pw} vor die Wahl zwiſchen \label{K_L01540-1v}\edtext{Wurſtl}{\lemma{\textnormal{\emph{Wurstl}}}\Cendnote{\textnormal{Puppentheater mit dem Hanswurst im Wurstelprater\oindex{Wien@\textbf{Wien}!II., Leopoldstadt@\textbf{II., Leopoldstadt}!Wurstelprater@\textbf{Wurstelprater}, \emph{Vergnügungspark}|pwk}}}}\label{K_L01540-1} u \textsc{Menagerie} geſtellt für letztere entschied – u
               kaum hatte Heini\pwindex{Schnitzler, Heinrich 9.\,8.\,1902 Hinterbrühl – 12.\,7.\,1982 Wien@\textsc{Schnitzler, Heinrich} (9.\,8.\,1902 Hinterbrühl – 12.\,7.\,1982 Wien), \emph{Regisseur, Schauspieler}|pw} das ausgeſprochen,{ }ſo war
               mein erſter Gedanke »Hugo«) – und ich hoffe, auch ohne dieſe Karte {\pb}wiſſen Sie, daſs ich mich mindeſtens ebenſo{ }ſehr freue \substVorne{}\textsuperscript{\textcolor{gray}{we{\geminationn}}}\substDazwischen{}Sie\substHinten{} wiederzuſehen als umgekehrt. Ich brauche Sie{ }ſogar, abgeſehen von der
               Sehnsucht, Ende der Woche dringend, insbeſondere wegen des einen Stücks\pwindex{Schnitzler, Arthur 15.\,5.\,1862 Wien – 21.\,10.\,1931 ebd.@\textsc{Schnitzler, Arthur} (15.\,5.\,1862 Wien – 21.\,10.\,1931 ebd.), \emph{Schriftsteller, Mediziner}!Zwischenspiel. Komödie in drei Akten@\strich\emph{Zwischenspiel. Komödie in drei Akten}|pwv}. Ich habe Ihnen zwei\pwindex{Schnitzler, Arthur 15.\,5.\,1862 Wien – 21.\,10.\,1931 ebd.@\textsc{Schnitzler, Arthur} (15.\,5.\,1862 Wien – 21.\,10.\,1931 ebd.), \emph{Schriftsteller, Mediziner}!Ruf des Lebens [Filmentwurf]@\strich\emph{Der Ruf des Lebens [Filmentwurf]}|pwv} vorzuleſen.\pend
           
\pstart
           Nun, wir{ }ſprechen hoffentlich{ }ſchon Mittwoch über das Wie, Wo Wann.\pend
           
\pstart
           Herzlichst Ihr{\\[\baselineskip]}\spacefill\mbox{A.}\pend
           \leftskip=0em{}\selectlanguage{ngerman}\endnumbering\briefempfaengerindex{Hofmannsthal, Hugo von@\textsc{Hofmannsthal, Hugo von}!zzzSchnitzler, Arthur@\emph{von Arthur Schnitzler}!1905-08-072@{7. 8. 1905}|)be}\mylabel{L01540h}  \newcommand{\dateiname}{L01540}\newcommand{\titel}{Arthur Schnitzler an Hugo von Hofmannsthal, 7. 8. 1905}\newcommand{\editorInnen}{Martin Anton Müller und Gerd-Hermann Susen}%% latex-leseansicht-abspann.tex
%% Abspann für die Leseansicht.
%% Der Schalter \ifkorrekturansicht ist bereits durch den Vorspann gesetzt.

%% latex-abspann.tex
%% Gemeinsamer Abspann für Korrekturansicht und Leseansicht.
%% Setzt den Schalter \ifkorrekturansicht voraus (gesetzt in den
%% einbindenden Dateien latex-korrekturansicht-abspann.tex bzw.
%% latex-leseansicht-abspann.tex).
%% ---------------------------------------------------------------

\normalsize

% Das esempio-Environment wird nur in der Leseansicht benötigt
\ifkorrekturansicht\else
\newenvironment{esempio}[3]%
{
    \vspace{1.5ex}
    \rlap{\underline{#1}}
    \par
    \setlength{\parindent}{0cm}
    \nopagebreak
    \leftskip=#2cm
    \rightskip=#3cm
}
{
    \par
}
\fi

\doendnotes{C}
\bigskip
\vfill

\clearpage

\footnotesize

\ifkorrekturansicht
  \lohead{\textsc{register}}
\fi

% theindex-Environment neu definieren ohne reledmac
\makeatletter
\renewenvironment{theindex}{%
  \ifkorrekturansicht
    \section*{\indexname}%
  \else
    \subsubsection*{Index der erwähnten Entitäten}%
  \fi
  \setlength{\parindent}{0pt}%
  \setlength{\parskip}{0pt plus 0.3pt}%
  \let\item\@idxitem
}{%
  \ifkorrekturansicht\clearpage\fi
}
\makeatother

\IfFileExists{\jobname-pw.ind}{\input{\jobname-pw.ind}}{}

% Quellenangabe nur in der Leseansicht
\ifkorrekturansicht\else
% Fallback-Definitionen, falls die .tex-Datei \titel etc. nicht gesetzt hat
\providecommand{\titel}{}
\providecommand{\editorInnen}{}
\providecommand{\dateiname}{\jobname}

\vspace{3cm}

\vfill

\footnotesize
\textsc{Quelle}: \titel. Herausgegeben von {\editorInnen}. In: \emph{Arthur Schnitzler: Briefwechsel mit Autorinnen und Autoren}.
 Digitale Edition, https://schnitzler-briefe.acdh.oeaw.ac.at/{\dateiname}.html (Stand \today)
\fi

\end{document}


