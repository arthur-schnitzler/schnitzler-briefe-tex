%% latex-korrekturansicht-vorspann.tex
%% Vorspann für die Korrekturansicht.
%% Lädt die gemeinsame Datei latex-vorspann.tex mit gesetztem Schalter.

\newif\ifkorrekturansicht
\korrekturansichttrue

\input{../tex-inputs/latex-vorspann}


\section[Arthur Schnitzler an Hugo von Hofmannsthal, 7. 8. 1905]{L01540 Arthur Schnitzler an Hugo von Hofmannsthal, 7. 8. 1905}
\nopagebreak\mylabel{L01540v}
\rehead{ }\normalsize\beginnumbering\briefempfaengerindex{Hofmannsthal, Hugo von@\textsc{Hofmannsthal, Hugo von}!zzzSchnitzler, Arthur@\emph{von Arthur Schnitzler}!1905-08-072@{7. 8. 1905}|(be}
\toendnotes[C]{\smallbreak\pagebreak[2]}\Standort{FDH, Hs-30885,121.}
\physDesc{Brief, 1 Blatt, 4 Seiten, 1192 Zeichen
\newline{}Handschrift: Bleistift, deutsche Kurrent}
\buchAbdrucke{\weitereDrucke{Hugo von Hofmannsthal, Arthur Schnitzler: \emph{Briefwechsel}. Frankfurt am Main: \emph{S. Fischer} 1964, S. 212.} }\toendnotes[C]{\smallbreak}
\pstart
           \raggedleft{}{\pb}7. 8. 90\substVorne{}\textsuperscript{1}\substDazwischen{}5\substHinten{}\pend
           \vspace{0.5em}
\pstart
           lieber Hugo, erſtens hatte ich begreiflicherweiſe keine Ahnung, daſs
               Sie So{\geminationn}tag{ }ſchon \strikeout{fort} wieder
               fortfahren. Wieſo ich unſer Wiederſehen bis Freitag hinausſchob, werden Sie ſofort
               hören. Heute Montag müſſen wir, wie ſchon ein paar Tage vorherbeſti{\geminationm}t war, weil Hr Steinrück\pwindex{Steinrueck, Albert 20.05.1872 – 11.02.1929@\textsc{Steinrück, Albert} (20.05.1872 – 11.02.1929), \emph{Schauspieler/Schauspielerin}|pw} gaſtiert, nach Mödling\oindex{Moedling@\textbf{Mödling}, \emph{P.PPLA3}|pw} –
                  Mittwoch wollten {\pb}wir, zu Heini\pwindex{Schnitzler, Heinrich 09.08.1902 – 12.07.1982@\textsc{Schnitzler, Heinrich} (09.08.1902 – 12.07.1982), \emph{Regisseur/Regisseurin, Schauspieler/Schauspielerin}|pw}’s 3. Geburtstag in den Prater\oindex{Prater@\textbf{Prater}, \emph{Park (K.PRK)}|pw}. Um aber nicht allzuſehr aus dem Arbeiten heraus zu ko{\geminationm}en (we{\geminationn} man eben daran iſt
               was abzuſchließen, \textsc{enervirt} einen das ſehr wie Sie ja
               wiſſen) wollte ich zwiſchen den Reiſetagen immer einen Heimtag, und ſo fiel
               naturgemäſs der Freitag erſt auf Sie. {\pb}Nun
               haben Sie indeſs wohl meine Karte erhalten, die Sie für Mittwoch nach
                  Schönbrunn\oindex{Schloss Schoenbrunn@\textbf{Schloss Schönbrunn}, \emph{Schloss (K.SLS)}|pw} bittet (da ſich Heini\pwindex{Schnitzler, Heinrich 09.08.1902 – 12.07.1982@\textsc{Schnitzler, Heinrich} (09.08.1902 – 12.07.1982), \emph{Regisseur/Regisseurin, Schauspieler/Schauspielerin}|pw} vor die Wahl zwiſchen \label{K_L01540-1v}\edtext{Wurſtl}{\lemma{\textnormal{\emph{Wurſtl}}}\Cendnote{\textnormal{Puppentheater mit dem Hanswurst im Wurstelprater\oindex{Wurstelprater@\textbf{Wurstelprater}, \emph{Vergnügungspark (K.VGN)}|pwk}}}}\label{K_L01540-1} u \textsc{Menagerie} geſtellt für letztere entschied – u
               kaum hatte Heini\pwindex{Schnitzler, Heinrich 09.08.1902 – 12.07.1982@\textsc{Schnitzler, Heinrich} (09.08.1902 – 12.07.1982), \emph{Regisseur/Regisseurin, Schauspieler/Schauspielerin}|pw} das ausgeſprochen, ſo war
               mein erſter Gedanke »Hugo«) – und ich hoffe, auch ohne dieſe Karte {\pb}wiſſen Sie, daſs ich mich mindeſtens ebenſo ſehr freue \substVorne{}\textsuperscript{\textcolor{gray}{we{\geminationn}}}\substDazwischen{}Sie\substHinten{} wiederzuſehen als umgekehrt. Ich brauche Sie ſogar, abgeſehen von der
               Sehnsucht, Ende der Woche dringend, insbeſondere wegen des einen Stücks\pwindex{Zwischenspiel. Komoedie in drei Akten@\emph{Zwischenspiel. Komödie in drei Akten}|pwv}. Ich habe Ihnen zwei\pwindex{Ruf des Lebens [Filmentwurf]@\emph{Der Ruf des Lebens [Filmentwurf]}|pwv} vorzuleſen.\pend
           
\pstart
           Nun, wir ſprechen hoffentlich ſchon Mittwoch über das Wie, Wo Wann.\pend
           
\pstart
           Herzlichst Ihr{\\[\baselineskip]}\spacefill\mbox{A.}\pend
           \leftskip=0em{}\selectlanguage{ngerman}\endnumbering\briefempfaengerindex{Hofmannsthal, Hugo von@\textsc{Hofmannsthal, Hugo von}!zzzSchnitzler, Arthur@\emph{von Arthur Schnitzler}!1905-08-072@{7. 8. 1905}|)be}\mylabel{L01540h}  \normalsize

\doendnotes{C}
\bigskip
\vfill

\clearpage

\footnotesize

\lohead{\textsc{register}}

% Definiere theindex-Environment komplett neu ohne reledmac
\makeatletter
\renewenvironment{theindex}{%
  \section*{\indexname}%
  \setlength{\parindent}{0pt}%
  \setlength{\parskip}{0pt plus 0.3pt}%
  \let\item\@idxitem
}{%
  \clearpage
}
\makeatother

\IfFileExists{\jobname-pw.ind}{\input{\jobname-pw.ind}}{}

\end{document}

      