\input{../tex-inputs/latex-pdf-vorspann}
\begin{center}
            \textcolor{red}{ENTWURF. ENTZIFFERUNG NOCH NICHT KORREKTURGELESEN}
                      \end{center}
            
               \section[Arthur Schnitzler an Hugo von Hofmannsthal, 7. 8. 1905]{ Arthur Schnitzler an Hugo von Hofmannsthal, 7. 8. 1905}\nopagebreak\mylabel{v}\rehead{ }\begin{ledgroupsized}[t]{13cm}\normalsize\beginnumbering\briefempfaengerindex{Hofmannsthal, Hugo von@\textsc{Hofmannsthal, Hugo von}!zzzSchnitzler, Arthur@\emph{von Arthur Schnitzler}!1905-08-072@{7. 8. 1905}|(be} \toendnotes[C]{\smallbreak\pagebreak[2]} \Standort{FDH, Hs-30885,121.}
\physDesc{Brief, 1 Blatt, 4 Seiten
\newline{}Handschrift: Bleistift, deutsche Kurrent}\buchAbdrucke{\weitereDrucke{Hugo von Hofmannsthal, Arthur Schnitzler: \emph{Briefwechsel}. Hg. Therese Nickl und Heinrich Schnitzler. Frankfurt am Main: \emph{S. Fischer} 1964, S. 212.} }\toendnotes[C]{\smallbreak}\pstart
           \raggedleft{}{\pb}7. 8. 90\substVorne{}\textsuperscript{1}\substDazwischen{}5\substHinten{}\pend
           \pstart
           lieber Hugo, erſtens hatte ich begreiflicherweiſe keine Ahnung, daſs
               Sie So{\geminationn}tag{ }ſchon \strikeout{fort} wieder
               fortfahren. Wieſo ich unſer Wiederſehen bis Freitag hinausſchob, werden Sie ſofort
               hören. Heute Montag müſſen wir, wie ſchon ein paar Tage vorherbeſti{\geminationm}t war, weil Hr Steinrück\pwindex{Steinrueck, Albert 20.05.1872 – 11.02.1929@\textsc{Steinrück, Albert} (20.05.1872 – 11.02.1929), \emph{Schauspieler}|pw} gaſtiert, nach Mödling\oindex{Moedling@\textbf{Mödling}|pw} –
                  Mittwoch wollten {\pb}wir, zu Heini\pwindex{Schnitzler, Heinrich 09.08.1902 – 12.07.1982@\textsc{Schnitzler, Heinrich} (09.08.1902 – 12.07.1982), \emph{Regisseur, Schauspieler}|pw}’s 3. Geburtstag in den Prater\oindex{Prater@\textbf{Prater}|pw}. Um aber nicht allzuſehr aus dem Arbeiten heraus zu ko{\geminationm}en (we{\geminationn} man eben daran iſt
               was abzuſchließen, \textsc{enervirt} einen das ſehr wie Sie ja
               wiſſen) wollte ich zwiſchen den Reiſetagen immer einen Heimtag, und ſo fiel
               naturgemäſs der Freitag erſt auf Sie. {\pb}Nun
               haben Sie indeſs wohl meine Karte erhalten, die Sie für Mittwoch nach
                  Schönbrunn\oindex{Schloss Schoenbrunn@\textbf{Schloß Schönbrunn}|pw} bittet (da ſich Heini\pwindex{Schnitzler, Heinrich 09.08.1902 – 12.07.1982@\textsc{Schnitzler, Heinrich} (09.08.1902 – 12.07.1982), \emph{Regisseur, Schauspieler}|pw} vor die Wahl zwiſchen \label{K_L01540_1v}\edtext{Wurſtl}{\lemma{\textnormal{\emph{Wurſtl}}}\Cendnote{\textnormal{Puppentheater mit dem Hanswurst im Wurstelprater\oindex{Wurstelprater@\textbf{Wurstelprater}|pwk}}}}\label{K_L01540_1h} u \textsc{Menagerie} geſtellt für
               letztere entschied – u kaum hatte Heini\pwindex{Schnitzler, Heinrich 09.08.1902 – 12.07.1982@\textsc{Schnitzler, Heinrich} (09.08.1902 – 12.07.1982), \emph{Regisseur, Schauspieler}|pw} das
               ausgeſprochen, ſo war mein erſter Gedanke »Hugo«) – und ich hoffe, auch ohne dieſe
               Karte {\pb}wiſſen Sie, daſs ich mich mindeſtens ebenſo ſehr
               freue \substVorne{}\textsuperscript{\textcolor{gray}{we{\geminationn}}}\substDazwischen{}Sie\substHinten{} wiederzuſehen als umgekehrt. Ich brauche Sie ſogar, abgeſehen von der
               Sehnsucht, Ende der Woche dringend, insbeſondere wegen des einen Stücks\pwindex{Schnitzler, Arthur 15.05.1862 – 21.10.1931@\textsc{Schnitzler, Arthur} (15.05.1862 – 21.10.1931), \emph{Schriftsteller, Mediziner}!Zwischenspiel. Komoedie in drei Akten1905-10-12 – 1905-10-12@\strich\emph{Zwischenspiel. Komödie in drei Akten} {[}1905-10-12 – 1905-10-12{]}|pwv}. Ich habe Ihnen zwei\pwindex{Schnitzler, Arthur 15.05.1862 – 21.10.1931@\textsc{Schnitzler, Arthur} (15.05.1862 – 21.10.1931), \emph{Schriftsteller, Mediziner}!Ruf des Lebens [Filmentwurf]2015@\strich\emph{Der Ruf des Lebens [Filmentwurf]} {[}2015{]}|pwv} vorzuleſen.\pend
           \pstart
           Nun, wir ſprechen hoffentlich ſchon Mittwoch über das Wie, Wo Wann.\pend
           \pstart
           Herzlichst Ihr{\\[\baselineskip]}\spacefill\mbox{A.}\pend
           \leftskip=0em{}\endnumbering\briefempfaengerindex{Hofmannsthal, Hugo von@\textsc{Hofmannsthal, Hugo von}!zzzSchnitzler, Arthur@\emph{von Arthur Schnitzler}!1905-08-072@{7. 8. 1905}|)be}\mylabel{h}\end{ledgroupsized}  \newcommand{\dateiname}{L01540}\newcommand{\titel}{Arthur Schnitzler an Hugo von Hofmannsthal, 7. 8. 1905}\newcommand{\editorInnen}{Martin Anton Müller und Gerd-Hermann Susen}\input{../tex-inputs/latex-pdf-abspann}
      