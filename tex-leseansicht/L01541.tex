%% latex-korrekturansicht-vorspann.tex
%% Vorspann für die Korrekturansicht.
%% Lädt die gemeinsame Datei latex-vorspann.tex mit gesetztem Schalter.

\newif\ifkorrekturansicht
\korrekturansichttrue

\input{../tex-inputs/latex-vorspann}


\section[Hugo von Hofmannsthal an Arthur Schnitzler, 11. {[}8.?{]} 1905]{L01541 Hugo von Hofmannsthal an Arthur Schnitzler, 11. {[}8.?{]} 1905}
\nopagebreak\mylabel{L01541v}
\rehead{ }\normalsize\beginnumbering\briefempfaengerindex{Schnitzler, Arthur@\textsc{Schnitzler, Arthur}!zzzHofmannsthal, Hugo von@\emph{von Hugo von Hofmannsthal}!1905-08-111@{11. {[}8.?{]} 1905}|(be}
\toendnotes[C]{\smallbreak\pagebreak[2]}\Standort{CUL, Schnitzler, B 43.}
\physDesc{Postkarte, 193 Zeichen
\newline{}Handschrift: 1) schwarze Tinte, deutsche Kurrent\hspace{1em}2) schwarze Tinte, lateinische Kurrent (\noindent{}Adresse)\hspace{1em}
\newline{}Versand: 1) Rohrpost  2) Stempel: »\nobreak{}\oindex{I., Innere Stadt@\textbf{I., Innere Stadt}, \emph{A.ADM3}|pwk}Wien 1/1, 11 V\textcolor{gray}{III} 05, 11 10V\nobreak{}«.  3) Stempel: »\nobreak{}\oindex{XVIII., Waehring@\textbf{XVIII., Währing}, \emph{A.ADM3}|pwk}18/1 Wien 111, 11 V\textcolor{gray}{III} 05, XII\textsuperscript{50}\nobreak{}«. 
\newline{}Ordnung: 1) mit Bleistift von unbekannter Hand nummeriert: »\strikeout{209}«  2) mit Bleistift von unbekannter Hand nummeriert: »\strikeout{258}« 3) mit Bleistift von unbekannter Hand nummeriert:
                                    »257«}
\buchAbdrucke{\weitereDrucke{Hugo von Hofmannsthal, Arthur Schnitzler: \emph{Briefwechsel}. Frankfurt am Main: \emph{S. Fischer} 1964, S. 213.} }\toendnotes[C]{\smallbreak}\pstart{}{\pb}Herrn D\textsuperscript{r} Arthur Schnitzler\pend{}\pstart{}Wien\oindex{Wien@\textbf{Wien}, \emph{A.ADM2}|pw}\pend{}\pstart{}XVIII Spöttelgasse 7\oindex{Edmund-Weiss-Gasse 7@\textbf{Edmund-Weiß-Gasse 7}, \emph{Wohngebäude (K.WHS)}|pw}\pend{}\pstart{}pneumatisch\pend{}{\bigskip}\vspace{1em}
\pstart
           \noindent{}{\pb}lieb\textcolor{gray}{en}
                  Leuten, bitte ko{\geminationm}t nicht ſpäter als mit dem
               Zug der ½ 6{ }Hietzing\oindex{XIII., Hietzing@\textbf{XIII., Hietzing}, \emph{A.ADM3}|pw} weggeht, weil wir einen \label{K_L01541-1v}\edtext{Wagen beſtellt}{\lemma{\textnormal{\emph{Wagen beſtellt}}}\Cendnote{\textnormal{Die Poststempel sind bei der Monatsangabe undeutlich. Im
                     Juli 1905 war Hofmannsthal\pwindex{Hofmannsthal, Hugo von 1874-02-01 – 1929-07-15@\textsc{Hofmannsthal, Hugo von} (1874-02-01 – 1929-07-15), \emph{Schriftsteller/Schriftstellerin}|pwk}
                  auf Waffenübung, sodass dies als möglicher Zeitraum ausscheidet. Am 11. 8. 1905 kam es
                  zum Treffen in Rodaun\oindex{Rodaun@\textbf{Rodaun}, \emph{A.ADM4}|pwk}.}}}\label{K_L01541-1} zum ein bisl
               Ausfahren.\pend
           \pstart \spacefill\mbox{Hugo.}\pend{}\selectlanguage{ngerman}\endnumbering\briefempfaengerindex{Schnitzler, Arthur@\textsc{Schnitzler, Arthur}!zzzHofmannsthal, Hugo von@\emph{von Hugo von Hofmannsthal}!1905-08-111@{11. {[}8.?{]} 1905}|)be}\mylabel{L01541h}  \normalsize

\doendnotes{C}
\bigskip
\vfill

\clearpage

\footnotesize

\lohead{\textsc{register}}

% Definiere theindex-Environment komplett neu ohne reledmac
\makeatletter
\renewenvironment{theindex}{%
  \section*{\indexname}%
  \setlength{\parindent}{0pt}%
  \setlength{\parskip}{0pt plus 0.3pt}%
  \let\item\@idxitem
}{%
  \clearpage
}
\makeatother

\IfFileExists{\jobname-pw.ind}{\input{\jobname-pw.ind}}{}

\end{document}

      