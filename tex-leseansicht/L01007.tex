%% latex-korrekturansicht-vorspann.tex
%% Vorspann für die Korrekturansicht.
%% Lädt die gemeinsame Datei latex-vorspann.tex mit gesetztem Schalter.

\newif\ifkorrekturansicht
\korrekturansichttrue

\input{../tex-inputs/latex-vorspann}


\section[Arthur Schnitzler an Richard Beer-Hofmann, 24. 12. 1899]{L01007 Arthur Schnitzler an Richard Beer-Hofmann, 24. 12. 1899}
\nopagebreak\mylabel{L01007v}
\rehead{ }\normalsize\beginnumbering\briefempfaengerindex{Beer-Hofmann, Richard@\textsc{Beer-Hofmann, Richard}!zzzSchnitzler, Arthur@\emph{von Arthur Schnitzler}!1899-12-241@{24. 12. 1899}|(be}
\toendnotes[C]{\smallbreak\pagebreak[2]}\Standort{YCGL, MSS 31.}
\physDesc{Brief, 1 Blatt, 3 Seiten, Umschlag, 613 Zeichen
\newline{}Handschrift: Bleistift, deutsche Kurrent
\newline{}Versand: Stempel: »\nobreak{}\oindex{IX., Alsergrund@\textbf{IX., Alsergrund}, \emph{A.ADM3}|pwk}Wien 9/1, 2{[}4. 12. 1899{]}, 5–6V\nobreak{}«.  }
\buchAbdrucke{\weitereDrucke{Arthur Schnitzler, Richard Beer-Hofmann: \emph{Briefwechsel 1891–1931}. Wien, Zürich: \emph{Europaverlag} 1992, S. 140.} }\toendnotes[C]{\smallbreak}\pstart{}{\pb}Herrn \textsc{Dr. Richard
                     Beer-Hofmann}\pend{}\pstart{}Wien\oindex{Wien@\textbf{Wien}, \emph{A.ADM2}|pw}\pend{}\pstart{}\textsc{I. Wollzeile 15\oindex{Wollzeile@\textbf{Wollzeile}, \emph{Straße (K.STR)}|pw}}.\pend{}{\bigskip}\vspace{1em}
\pstart
           \raggedleft{}{\pb}24. 12. 99\pend
           
\pstart{}mein lieber Richard,\pend\vspace{0.5em}
\pstart
           ich ka{\geminationn} nur ſagen, es iſt geradezu feinſinnig, was
               diesmal keine Beleidigung bedeuten ſoll, und ich bin (wiſſen Sie kein andres Wort?)
               beſchämt, befangen – und verſuche mich mit einem Witz aus der Affaire zu {\pb}ziehen – z. B. daſs ich immer auf einen der 3 Einakter\pwindex{gruene Kakadu – Paracelsus – Die Gefaehrtin. Drei Einakter@\emph{Der grüne Kakadu – Paracelsus – Die Gefährtin. Drei Einakter}|pw} verzichten muſs – bei Ihrem Geſchenk
               auf die Gefährtin\pwindex{Gefaehrtin. Schauspiel in einem Akt@\emph{Die Gefährtin. Schauspiel in einem Akt}|pw} – aber ich will (was gleich
               ein zweiter Witz iſt) die Schachtel ſelbſt als Gefährtin anſehen da sie (dritter
               Witz) keine alte iſt.\pend
           
\pstart
           {\pb}Also herzlichen Dank und Gruſs; auf Wiederſehen
                  \label{K_L01007-1v}\edtext{morgen\pwindex{Glaeubiger. Schauspiel in einem Act@\emph{Gläubiger. Schauspiel in einem Act}|pwv}}{\lemma{\textnormal{\emph{morgen}}}\Cendnote{\textnormal{Am \emph{Theater in der Josefstadt}\orgindex{Theater in der Josefstadt@Theater in der Josefstadt|pwk} wurde am 25. 12. 1899{ }\emph{Gläubiger}\pwindex{Glaeubiger. Schauspiel in einem Act@\emph{Gläubiger. Schauspiel in einem Act}|pwk} von August Strindberg\pwindex{Strindberg, August 22.01.1849 – 14.05.1912@\textsc{Strindberg, August} (22.01.1849 – 14.05.1912), \emph{Schriftsteller/Schriftstellerin}|pwk} und \emph{Die
                     Mondscheinsonate}\pwindex{Mondscheinsonate. Komoedie in einem Akt@\emph{Die Mondscheinsonate. Komödie in einem Akt}|pwk} von Ludwig Wolff\pwindex{Wolff, Ludwig 07.03.1876 – nach 1958@\textsc{Wolff, Ludwig} (07.03.1876 – nach 1958), \emph{Schriftsteller/Schriftstellerin, Dramaturg/Dramaturgin}|pwk}
                  gegeben.}}}\label{K_L01007-1}, wohl ſchon in der Joſefſtadt\oindex{Theater in der Josefstadt@\textbf{Theater in der Josefstadt}, \emph{Theater (K.THE)}|pw}.\pend
           \pstart Ihr \spacefill\mbox{Arthur}\pend{}\selectlanguage{ngerman}\endnumbering\briefempfaengerindex{Beer-Hofmann, Richard@\textsc{Beer-Hofmann, Richard}!zzzSchnitzler, Arthur@\emph{von Arthur Schnitzler}!1899-12-241@{24. 12. 1899}|)be}\mylabel{L01007h}  \normalsize

\doendnotes{C}
\bigskip
\vfill

\clearpage

\footnotesize

\lohead{\textsc{register}}

% Definiere theindex-Environment komplett neu ohne reledmac
\makeatletter
\renewenvironment{theindex}{%
  \section*{\indexname}%
  \setlength{\parindent}{0pt}%
  \setlength{\parskip}{0pt plus 0.3pt}%
  \let\item\@idxitem
}{%
  \clearpage
}
\makeatother

\IfFileExists{\jobname-pw.ind}{\input{\jobname-pw.ind}}{}

\end{document}

      