%% latex-leseansicht-vorspann.tex
%% Vorspann für die Leseansicht.
%% Lädt die gemeinsame Datei latex-vorspann.tex mit nicht gesetztem Schalter.

\newif\ifkorrekturansicht
\korrekturansichtfalse

\input{../tex-inputs/latex-vorspann}


               \section[Arthur Schnitzler an Richard Beer-Hofmann, 24. 12. 1899]{ Arthur Schnitzler an Richard Beer-Hofmann, 24. 12. 1899}\nopagebreak\mylabel{v}\rehead{ }\begin{ledgroupsized}[t]{13cm}\normalsize\beginnumbering\briefempfaengerindex{Beer-Hofmann, Richard@\textsc{Beer-Hofmann, Richard}!zzzSchnitzler, Arthur@\emph{von Arthur Schnitzler}!1899-12-241@{24. 12. 1899}|(be} \toendnotes[C]{\smallbreak\pagebreak[2]} \Standort{YCGL, MSS 31.}
\physDesc{Brief, 1 Blatt, 3 Seiten, Umschlag
\newline{}Handschrift: Bleistift, deutsche Kurrent\newline{}Versand: Stempel: »\nobreak{}\oindex{IX., Alsergrund@\textbf{IX., Alsergrund}|pwk}Wien 9/1, 2{[}4. 12. 1899{]}, 5–6V\nobreak{}«.  }\buchAbdrucke{\weitereDrucke{Arthur Schnitzler, Richard Beer-Hofmann: \emph{Briefwechsel 1891–1931}. Hg. Konstanze Fliedl. Wien, Zürich: \emph{Europaverlag} 1992, S. 140.} }\toendnotes[C]{\smallbreak}\pstart{}{\pb}Herrn \textsc{Dr. Richard
                     Beer-Hofmann}\pend{}\pstart{}Wien\oindex{Wien@\textbf{Wien}|pw}\pend{}\pstart{}\textsc{I. Wollzeile 15\oindex{Wollzeile@\textbf{Wollzeile}|pw}}.\pend{}{\bigskip}\pstart
           \raggedleft{}{\pb}24. 12. 99\pend
           \pstart{}mein lieber Richard,\pend\pstart
           ich ka{\geminationn} nur ſagen, es iſt geradezu feinſinnig, was
               diesmal keine Beleidigung bedeuten ſoll, und ich bin (wiſſen Sie kein andres Wort?)
               beſchämt, befangen – und verſuche mich mit einem Witz aus der Affaire zu {\pb}ziehen – z. B. daſs ich immer auf einen der 3 Einakter\pwindex{Schnitzler, Arthur 15.05.1862 – 21.10.1931@\textsc{Schnitzler, Arthur} (15.05.1862 – 21.10.1931), \emph{Schriftsteller, Mediziner}!gruene Kakadu – Paracelsus – Die Gefaehrtin. Drei Einakter1.3.1899 – 1.3.1899@\strich\emph{Der grüne Kakadu – Paracelsus – Die Gefährtin. Drei Einakter} {[}1.3.1899 – 1.3.1899{]}|pw} verzichten muſs – bei Ihrem Geſchenk auf
               die Gefährtin\pwindex{Schnitzler, Arthur 15.05.1862 – 21.10.1931@\textsc{Schnitzler, Arthur} (15.05.1862 – 21.10.1931), \emph{Schriftsteller, Mediziner}!Gefaehrtin. Schauspiel in einem Akt1899-03-01@\strich\emph{Die Gefährtin. Schauspiel in einem Akt} {[}1899-03-01{]}|pw} – aber ich will (was gleich ein
               zweiter Witz iſt) die Schachtel ſelbſt als Gefährtin anſehen da sie (dritter Witz)
               keine alte iſt.\pend
           \pstart
           {\pb}Also herzlichen Dank und Gruſs; auf Wiederſehen
                  \label{K_L01007_1v}\edtext{morgen\pwindex{Strindberg, August 22.01.1849 – 14.05.1912@\textsc{Strindberg, August} (22.01.1849 – 14.05.1912), \emph{Schriftsteller}!Glaeubiger1892@\strich\emph{Gläubiger} {[}1892{]}|pwv}}{\lemma{\textnormal{\emph{morgen}}}\Cendnote{\textnormal{Am \emph{Theater in der Josefstadt}\orgindex{Theater in der Josefstadt@Theater in der Josefstadt|pwk} wurde am 25. 12. 1899{ }\emph{Gläubiger}\pwindex{Strindberg, August 22.01.1849 – 14.05.1912@\textsc{Strindberg, August} (22.01.1849 – 14.05.1912), \emph{Schriftsteller}!Glaeubiger1892@\strich\emph{Gläubiger} {[}1892{]}|pwk} von August Strindberg\pwindex{Strindberg, August 22.01.1849 – 14.05.1912@\textsc{Strindberg, August} (22.01.1849 – 14.05.1912), \emph{Schriftsteller}|pwk} und \emph{Die
                     Mondscheinsonate}\pwindex{Wolff, Ludwig 07.03.1876 – nach 1958@\textsc{Wolff, Ludwig} (07.03.1876 – nach 1958), \emph{Schriftsteller, Dramaturg}!Mondscheinsonate. Komoedie in einem Akt1899 – 1899@\strich\emph{Die Mondscheinsonate. Komödie in einem Akt} {[}1899 – 1899{]}|pwk} von Ludwig Wolff\pwindex{Wolff, Ludwig 07.03.1876 – nach 1958@\textsc{Wolff, Ludwig} (07.03.1876 – nach 1958), \emph{Schriftsteller, Dramaturg}|pwk}
                  gegeben.}}}\label{K_L01007_1h}, wohl ſchon in der Joſefſtadt\oindex{Theater in der Josefstadt@\textbf{Theater in der Josefstadt}|pw}.\pend
           \pstart Ihr \spacefill\mbox{Arthur}\pend{}\endnumbering\briefempfaengerindex{Beer-Hofmann, Richard@\textsc{Beer-Hofmann, Richard}!zzzSchnitzler, Arthur@\emph{von Arthur Schnitzler}!1899-12-241@{24. 12. 1899}|)be}\mylabel{h}\end{ledgroupsized}  \newcommand{\dateiname}{L01007}\newcommand{\titel}{Arthur Schnitzler an Richard Beer-Hofmann, 24. 12. 1899}\newcommand{\editorInnen}{Martin Anton Müller und Gerd-Hermann Susen}
            \footnotesize
\begin{ledgroupsized}[t]{11.5cm}
\doendnotes{C}
\end{ledgroupsized}
         %% latex-leseansicht-abspann.tex
%% Abspann für die Leseansicht.
%% Der Schalter \ifkorrekturansicht ist bereits durch den Vorspann gesetzt.

%% latex-abspann.tex
%% Gemeinsamer Abspann für Korrekturansicht und Leseansicht.
%% Setzt den Schalter \ifkorrekturansicht voraus (gesetzt in den
%% einbindenden Dateien latex-korrekturansicht-abspann.tex bzw.
%% latex-leseansicht-abspann.tex).
%% ---------------------------------------------------------------

\normalsize

% Das esempio-Environment wird nur in der Leseansicht benötigt
\ifkorrekturansicht\else
\newenvironment{esempio}[3]%
{
    \vspace{1.5ex}
    \rlap{\underline{#1}}
    \par
    \setlength{\parindent}{0cm}
    \nopagebreak
    \leftskip=#2cm
    \rightskip=#3cm
}
{
    \par
}
\fi

\doendnotes{C}
\bigskip
\vfill

\clearpage

\footnotesize

\ifkorrekturansicht
  \lohead{\textsc{register}}
\fi

% theindex-Environment neu definieren ohne reledmac
\makeatletter
\renewenvironment{theindex}{%
  \ifkorrekturansicht
    \section*{\indexname}%
  \else
    \subsubsection*{Index der erwähnten Entitäten}%
  \fi
  \setlength{\parindent}{0pt}%
  \setlength{\parskip}{0pt plus 0.3pt}%
  \let\item\@idxitem
}{%
  \ifkorrekturansicht\clearpage\fi
}
\makeatother

\IfFileExists{\jobname-pw.ind}{\input{\jobname-pw.ind}}{}

% Quellenangabe nur in der Leseansicht
\ifkorrekturansicht\else
% Fallback-Definitionen, falls die .tex-Datei \titel etc. nicht gesetzt hat
\providecommand{\titel}{}
\providecommand{\editorInnen}{}
\providecommand{\dateiname}{\jobname}

\vspace{3cm}

\vfill

\footnotesize
\textsc{Quelle}: \titel. Herausgegeben von {\editorInnen}. In: \emph{Arthur Schnitzler: Briefwechsel mit Autorinnen und Autoren}.
 Digitale Edition, https://schnitzler-briefe.acdh.oeaw.ac.at/{\dateiname}.html (Stand \today)
\fi

\end{document}


      