%% latex-korrekturansicht-vorspann.tex
%% Vorspann für die Korrekturansicht.
%% Lädt die gemeinsame Datei latex-vorspann.tex mit gesetztem Schalter.

\newif\ifkorrekturansicht
\korrekturansichttrue

\input{../tex-inputs/latex-vorspann}


\section[Arthur Schnitzler an Richard Beer-Hofmann, 17. 7. 1927]{L02489 Arthur Schnitzler an Richard Beer-Hofmann, 17. 7. 1927}
\nopagebreak\mylabel{L02489v}
\rehead{ }\normalsize\beginnumbering\briefempfaengerindex{Beer-Hofmann, Richard@\textsc{Beer-Hofmann, Richard}!zzzSchnitzler, Arthur@\emph{von Arthur Schnitzler}!1927-07-171@{17. 7. 1927}|(be}
\toendnotes[C]{\smallbreak\pagebreak[2]}\Standort{YCGL, MSS 31.}
\physDesc{Brief, 1 Blatt, 2 Seiten, Umschlag, 394 Zeichen
\newline{}Handschrift: Bleistift, lateinische Kurrent
\newline{}Versand: ohne postalischen Übermittlungsvermerk }
\buchAbdrucke{\weitereDrucke{Arthur Schnitzler, Richard Beer-Hofmann: \emph{Briefwechsel 1891–1931}. Wien, Zürich: \emph{Europaverlag} 1992, S. 230.} }\toendnotes[C]{\smallbreak}\pstart{}{\pb}\label{T_L02489-1v}\edtext{\textcolor{gray}{\textbf{A. S.}}}{\lemma{\textnormal{\emph{A. S.}}}\Cendnote{\textnormal{ovaler Absenderkleber}}}\label{T_L02489-1}\pend{}\pstart{}\textcolor{gray}{\textbf{WIEN, XVIII.}}\oindex{XVIII., Waehring@\textbf{XVIII., Währing}, \emph{A.ADM3}|pw}\pend{}\pstart{}\textcolor{gray}{\textbf{STERNWARTESTR. 71}}\oindex{Sternwartestrasse 71@\textbf{Sternwartestraße 71}, \emph{Wohngebäude (K.WHS)}|pw}\pend{}{\bigskip}\pstart{}{\pb}Hrn Dr Richard Beer Hofma{\geminationn}\pend{}\pstart{}Wien XVIII\oindex{Wien@\textbf{Wien}, \emph{A.ADM2}|pw}\pend{}\pstart{}Hasenauerstr 59\oindex{Hasenauerstrasse 59@\textbf{Hasenauerstraße 59}, \emph{Wohngebäude (K.WHS)}|pw}\pend{}{\bigskip}\vspace{1em}
\pstart
           \raggedleft{}{\pb}Wien\oindex{Wien@\textbf{Wien}, \emph{A.ADM2}|pw}, 17. 7  927\pend
           \vspace{0.5em}
\pstart
           lieber Richard, wollen Sie heute mit Paula\pwindex{Beer-Hofmann, Paula 25.02.1879 – 30.10.1939@\textsc{Beer-Hofmann, Paula} (25.02.1879 – 30.10.1939)|pw} bei mir nachtmahlen? \introOben{}(8 Uhr)\introOben{} Ich würde mich sehr freuen. »\uline{Nach dem} Nachtmahl« wird nicht acceptirt, da ich in diesem Falle selbst
               außer Haus ginge. Sagen Sie ja, so würde ich auch Frau Dr Menczel\pwindex{Menczel, Rosa 20.07.1874 – Juni 1962@\textsc{Menczel, Rosa} (20.07.1874 – Juni 1962)|pw} bitten lassen. (Er\pwindex{Menczel, Philipp 09.01.1872 – 26.10.1941@\textsc{Menczel, Philipp} (09.01.1872 – 26.10.1941), \emph{Journalist/Journalistin, Rechtsanwalt/Rechtsanwältin}|pwv} ist schon fort, nicht wahr\substVorne{}\textsuperscript{{\dotstwo}}\substDazwischen{}?\substHinten{})\pend
           
\pstart
           {\pb}Auf Wiedersehen hoffentlich,\pend
           
\pstart
           herzlichst\hspace*{1.5em}Ihr{\\[\baselineskip]}\spacefill\mbox{Arthur}\pend
           \leftskip=0em{}\selectlanguage{ngerman}\endnumbering\briefempfaengerindex{Beer-Hofmann, Richard@\textsc{Beer-Hofmann, Richard}!zzzSchnitzler, Arthur@\emph{von Arthur Schnitzler}!1927-07-171@{17. 7. 1927}|)be}\mylabel{L02489h}  \normalsize

\doendnotes{C}
\bigskip
\vfill

\clearpage

\footnotesize

\lohead{\textsc{register}}

% Definiere theindex-Environment komplett neu ohne reledmac
\makeatletter
\renewenvironment{theindex}{%
  \section*{\indexname}%
  \setlength{\parindent}{0pt}%
  \setlength{\parskip}{0pt plus 0.3pt}%
  \let\item\@idxitem
}{%
  \clearpage
}
\makeatother

\IfFileExists{\jobname-pw.ind}{\input{\jobname-pw.ind}}{}

\end{document}

      