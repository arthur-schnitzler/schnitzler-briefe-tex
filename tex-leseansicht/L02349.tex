%% latex-korrekturansicht-vorspann.tex
%% Vorspann für die Korrekturansicht.
%% Lädt die gemeinsame Datei latex-vorspann.tex mit gesetztem Schalter.

\newif\ifkorrekturansicht
\korrekturansichttrue

\input{../tex-inputs/latex-vorspann}


\section[Arthur Schnitzler an Robert Adam, 14. 7. 1920]{L02349 Arthur Schnitzler an Robert Adam, 14. 7. 1920}
\nopagebreak\mylabel{L02349v}
\rehead{ }\normalsize\beginnumbering\briefempfaengerindex{Adam, Robert@\textsc{Adam, Robert}!zzzSchnitzler, Arthur@\emph{von Arthur Schnitzler}!1920-07-141@{14. 7. 1920}|(be}
\toendnotes[C]{\smallbreak\pagebreak[2]}\Standort{DLA, 96.34.2/22.}
\physDesc{Brief, 1 Blatt, 1 Seite, 561 Zeichen
\newline{}Schreibmaschine
\newline{}Handschrift: schwarze Tinte (\noindent{}Unterschrift)}\toendnotes[C]{\smallbreak}
\pstart
           {\pb}\textcolor{gray}{\textbf{Dr. Arthur Schnitzler}}\hfill 14. 7. 1920.\pend
           
\pstart
           \textcolor{gray}{\textbf{Wien XVIII. Sternwartestrasse 71\oindex{Sternwartestrasse 71@\textbf{Sternwartestraße 71}, \emph{Wohngebäude (K.WHS)}|pw}}}\pend
           
\pstart{}Sehr verehrter Herr Doktor.\pend\vspace{0.5em}
\pstart
           So viel ich weiss sind Sie zu allem andern auch ein Kenner der arabischen Sprache.
               Ich lege Ihnen hier \label{K_L02349-1v}\edtext{zwei Zettel}{\lemma{\textnormal{\emph{zwei Zettel}}}\Cendnote{\textnormal{Die Beilage ist nicht erhalten.}}}\label{K_L02349-1} bei, die
               einen arabischen Spruch enthalten sollen, und würde Sie recht sehr um
                  Uebersetzun{[}g{]} resp. Aufklärung bitten. Ich sende dies für alle
               Fälle an Ihre Wien\oindex{Wien@\textbf{Wien}, \emph{A.ADM2}|pw}er Adresse, nehme aber an, dass
               Sie sich schon in Gutenstein\oindex{Gutenstein@\textbf{Gutenstein}, \emph{A.ADM3}|pw} befinden, wo Sie
               sich völlig erholen werden.\pend
           
\pstart
           Entschuldigen Sie die Bemühung und seien Sie herzlichst gegrüsst von{\\[\baselineskip]}Ihrem
               sehr ergebenen{\\[\baselineskip]}\spacefill\mbox{{[}hs.:{]} Arthur Schnitzler}\pend
           \leftskip=0em{}
\pstart
           \noindent{}Herrn Oberlandesgerichtsrat Dr. Adam Pollak.\pend
           \selectlanguage{ngerman}\endnumbering\briefempfaengerindex{Adam, Robert@\textsc{Adam, Robert}!zzzSchnitzler, Arthur@\emph{von Arthur Schnitzler}!1920-07-141@{14. 7. 1920}|)be}\mylabel{L02349h}  \normalsize

\doendnotes{C}
\bigskip
\vfill

\clearpage

\footnotesize

\lohead{\textsc{register}}

% Definiere theindex-Environment komplett neu ohne reledmac
\makeatletter
\renewenvironment{theindex}{%
  \section*{\indexname}%
  \setlength{\parindent}{0pt}%
  \setlength{\parskip}{0pt plus 0.3pt}%
  \let\item\@idxitem
}{%
  \clearpage
}
\makeatother

\IfFileExists{\jobname-pw.ind}{\input{\jobname-pw.ind}}{}

\end{document}

      