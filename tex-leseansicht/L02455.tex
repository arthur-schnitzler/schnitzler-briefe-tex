%% latex-korrekturansicht-vorspann.tex
%% Vorspann für die Korrekturansicht.
%% Lädt die gemeinsame Datei latex-vorspann.tex mit gesetztem Schalter.

\newif\ifkorrekturansicht
\korrekturansichttrue

\input{../tex-inputs/latex-vorspann}


\section[Arthur Schnitzler an Hugo Hofmannsthal, 16. 11. 1925]{L02455 Arthur Schnitzler an Hugo Hofmannsthal, 16. 11. 1925}
\nopagebreak\mylabel{L02455v}
\rehead{ }\normalsize\beginnumbering\briefempfaengerindex{Hofmannsthal, Hugo von@\textsc{Hofmannsthal, Hugo von}!zzzSchnitzler, Arthur@\emph{von Arthur Schnitzler}!1925-11-161@{16. 11. 1925}|(be}
\toendnotes[C]{\smallbreak\pagebreak[2]}\Standort{FDH, Hs-30885,154.}
\physDesc{Brief, 1 Blatt, 2 Seiten, 1683 Zeichen
\newline{}Handschrift: Bleistift, lateinische Kurrent}
\buchAbdrucke{\weitereDrucke{Hugo von Hofmannsthal, Arthur Schnitzler: \emph{Briefwechsel}. Frankfurt am Main: \emph{S. Fischer} 1964, S. 303.} }\toendnotes[C]{\smallbreak}
\pstart
           \raggedleft{}{\pb}Wien\oindex{Wien@\textbf{Wien}, \emph{A.ADM2}|pw}{ }16. 11. 925\pend
           \vspace{0.5em}
\pstart
           mein lieber Hugo, Ihr schönes Stück\pwindex{Turm. Ein Trauerspiel@\emph{Der Turm. Ein Trauerspiel}|pw} hab ich noch in Berlin\oindex{Berlin@\textbf{Berlin}, \emph{P.PPLC}|pw} erhalten
               und es ist recht unhöflich, daſs ich Ihnen nicht gleich gedankt habe. Mit ein Grund
               ist gewesen, daſs ich erst in den letzten Tagen \introOben{}dazu kam\introOben{} den
                  Calderon\pwindex{Calderón de la Barca, Pedro 17.01.1600 – 25.05.1681@\textsc{Calderón de la Barca, Pedro} (17.01.1600 – 25.05.1681), \emph{Schriftsteller/Schriftstellerin}|pw}\pwindex{vida es sueño@\emph{La vida es sueño}|pwv}, der Ihnen dazu eine Anregung gab, zu lesen, und es war mir im höchsten Grad
               interessant, wie völlig neu und selbständig {[}Sie{]} Ihr Drama\pwindex{Turm. Ein Trauerspiel@\emph{Der Turm. Ein Trauerspiel}|pwv} geschrieben haben. Nur
               einige äußere Momente sind erhalten; – nicht nur die Gestalten sind neu geschaffen; –
               auch das Problem, das innere Licht ist etwas ganz neues geworden, und völlig Ihr
               Eigentum. An manchen Stellen wünscht ich mir geringere Weitläufigkeit, und der Humor
               des Dieners ist nicht durchaus nach meinem Sinn, we{\geminationn} ich
               auch fühle, sehr im Stil des ganzen.\pend
           
\pstart
           Ich freue mich, dſs Sie in der Arbeit sind; auch ich bringe allerlei weiter. Eine
               neue Novelle (»Traumnovelle\pwindex{Traumnovelle@\emph{Traumnovelle}|pw}«) erscheint bald;
               mein Versstück »Der Gang zum Weiher\pwindex{Gang zum Weiher. Dramatische Dichtung@\emph{Der Gang zum Weiher. Dramatische Dichtung}|pw}« ist fertig;
               nun dictir ich eine weitere {\pb}Novelle\pwindex{Spiel im Morgengrauen. Novelle@\emph{Spiel im Morgengrauen. Novelle}|pwv}, deren Schluſs noch
               unsicher ist; arbeite an einem Roman\pwindex{Therese. Chronik eines Frauenlebens@\emph{Therese. Chronik eines Frauenlebens}|pwv} (der richtiger eine Chronik zu nennen sein wird); und bringe
               verschiedentliches aphoristische\pwindex{Geist im Wort und der Geist in der Tat@\emph{Der Geist im Wort und der Geist in der Tat}|pwv}\pwindex{Buch der Sprueche und Bedenken@\emph{Buch der Sprüche und Bedenken}|pwv} und fragmentarisches\pwindex{Geist im Wort und der Geist in der Tat@\emph{Der Geist im Wort und der Geist in der Tat}|pwv} in Ordnung so gut es geht, ja einzelnes gewissermaßen in
                  Systeme\pwindex{Buch der Sprueche und Bedenken@\emph{Buch der Sprüche und Bedenken}|pwv}. Theatralisch liegt
               allerlei angefangnes vor, – was ich zuerst fertig machen werde, weiß ich noch
               nicht.\pend
           
\pstart
           Um Ihre Aussee\oindex{Bad Aussee@\textbf{Bad Aussee}, \emph{P.PPLA3}|pw}r Abgeschiedenheit beneid ich Sie
               manchmal – weiß aber nicht, ob ich \strikeout{von} trotz
               zeitweiliger Sehnsucht nach etwas der Art lange aushalten würde. Es ist mancherlei
               Unruhe in meinem Leben; im ganzen fühl ich mich wohl, bei gelegentlichen, am
               häufigsten durch das Gehörleiden verursachten und geförderten Depressionen.\pend
           
\pstart
           Ich hoffe Sie bald wiederzusehen.\pend
           
\pstart
           Seien Sie von Herzen gegrüßt und bedankt!{\\[\baselineskip]}Ihr{\\[\baselineskip]}\spacefill\mbox{Arthur.}\pend
           \leftskip=0em{}\selectlanguage{ngerman}\endnumbering\briefempfaengerindex{Hofmannsthal, Hugo von@\textsc{Hofmannsthal, Hugo von}!zzzSchnitzler, Arthur@\emph{von Arthur Schnitzler}!1925-11-161@{16. 11. 1925}|)be}\mylabel{L02455h}  \normalsize

\doendnotes{C}
\bigskip
\vfill

\clearpage

\footnotesize

\lohead{\textsc{register}}

% Definiere theindex-Environment komplett neu ohne reledmac
\makeatletter
\renewenvironment{theindex}{%
  \section*{\indexname}%
  \setlength{\parindent}{0pt}%
  \setlength{\parskip}{0pt plus 0.3pt}%
  \let\item\@idxitem
}{%
  \clearpage
}
\makeatother

\IfFileExists{\jobname-pw.ind}{\input{\jobname-pw.ind}}{}

\end{document}

      