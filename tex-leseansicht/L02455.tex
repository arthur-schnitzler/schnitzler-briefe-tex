%% latex-leseansicht-vorspann.tex
%% Vorspann für die Leseansicht.
%% Lädt die gemeinsame Datei latex-vorspann.tex mit nicht gesetztem Schalter.

\newif\ifkorrekturansicht
\korrekturansichtfalse

\input{../tex-inputs/latex-vorspann}


         
         \renewcommand{\erwaehntePersonen}{Personen: Pedro Calderón de la Barca, Hugo von Hofmannsthal}
         \renewcommand{\erwaehnteOrte}{Orte: Bad Aussee, Berlin, Wien}
         \renewcommand{\erwaehnteWerke}{Werke: Buch der Sprüche und Bedenken, Das Leben ein Traum, Der Gang zum Weiher. Dramatische Dichtung, Der Geist im Wort und der Geist in der Tat, Der Turm. Ein Trauerspiel, Spiel im Morgengrauen. Novelle, Therese. Chronik eines Frauenlebens, Traumnovelle}
               \section[Arthur Schnitzler an Hugo Hofmannsthal, 16. 11. 1925]{ Arthur Schnitzler an Hugo Hofmannsthal, 16. 11. 1925}\nopagebreak\mylabel{v}\rehead{ }\begin{ledgroupsized}[t]{13cm}\normalsize\beginnumbering \toendnotes[C]{\smallbreak\pagebreak[2]} \Standort{FDH, Hs-30885,154.}
\physDesc{Brief, 1 Blatt, 2 Seiten
\newline{}Handschrift: Bleistift, lateinische Kurrent}\buchAbdrucke{\weitereDrucke{Hugo von Hofmannsthal, Arthur Schnitzler: \emph{Briefwechsel}. Hg. Therese Nickl und Heinrich Schnitzler. Frankfurt am Main: \emph{S. Fischer} 1964, S. 303.} }\toendnotes[C]{\smallbreak}\pstart
           \raggedleft{}{\pb}Wien\oindex{Wien@\textbf{Wien}|pw}{ }16. 11. 925\pend
           \pstart
           mein lieber Hugo, Ihr schönes Stück\pwindex{Hofmannsthal, Hugo von 1874-02-01 – 1929-07-15@\textsc{Hofmannsthal, Hugo von} (1874-02-01 – 1929-07-15), \emph{Schriftsteller}!Turm. Ein Trauerspiel1925@\strich\emph{Der Turm. Ein Trauerspiel} {[}1925{]}|pw} hab ich noch in Berlin\oindex{Berlin@\textbf{Berlin}|pw} erhalten
                    und es ist recht unhöflich, daſs ich Ihnen nicht gleich gedankt habe. Mit ein
                    Grund ist gewesen, daſs ich erst in den letzten Tagen \introOben{}dazu
                        kam\introOben{} den Calderon\pwindex{Calderón de la Barca, Pedro 17.01.1600 – 25.05.1681@\textsc{Calderón de la Barca, Pedro} (17.01.1600 – 25.05.1681), \emph{Schriftsteller}|pw}\pwindex{Calderón de la Barca, Pedro 17.01.1600 – 25.05.1681@\textsc{Calderón de la Barca, Pedro} (17.01.1600 – 25.05.1681), \emph{Schriftsteller}!Leben ein Traum1635@\strich\emph{Das Leben ein Traum} {[}1635{]}|pwv}, der Ihnen dazu eine Anregung gab, zu lesen, und es war mir im höchsten
                    Grad interessant, wie völlig neu und selbständig {[}Sie{]} Ihr
                        Drama\pwindex{Hofmannsthal, Hugo von 1874-02-01 – 1929-07-15@\textsc{Hofmannsthal, Hugo von} (1874-02-01 – 1929-07-15), \emph{Schriftsteller}!Turm. Ein Trauerspiel1925@\strich\emph{Der Turm. Ein Trauerspiel} {[}1925{]}|pwv} geschrieben haben.
                    Nur einige äußere Momente sind erhalten; – nicht nur die Gestalten sind neu
                    geschaffen; – auch das Problem, das innere Licht ist etwas ganz neues geworden,
                    und völlig Ihr Eigentum. An manchen Stellen wünscht ich mir geringere
                    Weitläufigkeit, und der Humor des Dieners ist nicht durchaus nach meinem Sinn,
                        we{\geminationn} ich auch fühle, sehr im Stil des
                    ganzen.\pend
           \pstart
           Ich freue mich, dſs Sie in der Arbeit sind; auch ich bringe allerlei weiter. Eine
                    neue Novelle (»Traumnovelle\pwindex{Schnitzler, Arthur 15.05.1862 – 21.10.1931@\textsc{Schnitzler, Arthur} (15.05.1862 – 21.10.1931), \emph{Schriftsteller, Mediziner}!Traumnovelle1.12.1925 – 1.3.1926@\strich\emph{Traumnovelle} {[}1.12.1925 – 1.3.1926{]}|pw}«) erscheint bald;
                    mein Versstück »Der Gang zum Weiher\pwindex{Schnitzler, Arthur 15.05.1862 – 21.10.1931@\textsc{Schnitzler, Arthur} (15.05.1862 – 21.10.1931), \emph{Schriftsteller, Mediziner}!Gang zum Weiher. Dramatische Dichtung1926@\strich\emph{Der Gang zum Weiher. Dramatische Dichtung} {[}1926{]}|pw}« ist
                    fertig; nun dictir ich eine weitere {\pb}Novelle\pwindex{Schnitzler, Arthur 15.05.1862 – 21.10.1931@\textsc{Schnitzler, Arthur} (15.05.1862 – 21.10.1931), \emph{Schriftsteller, Mediziner}!Spiel im Morgengrauen. Novelle5.12.1926 – 9.1.1927@\strich\emph{Spiel im Morgengrauen. Novelle} {[}5.12.1926 – 9.1.1927{]}|pwv}, deren Schluſs noch
                    unsicher ist; arbeite an einem Roman\pwindex{Schnitzler, Arthur 15.05.1862 – 21.10.1931@\textsc{Schnitzler, Arthur} (15.05.1862 – 21.10.1931), \emph{Schriftsteller, Mediziner}!Therese. Chronik eines Frauenlebens1928@\strich\emph{Therese. Chronik eines Frauenlebens} {[}1928{]}|pwv} (der richtiger eine Chronik zu nennen sein wird); und bringe
                    verschiedentliches aphoristische\pwindex{Schnitzler, Arthur 15.05.1862 – 21.10.1931@\textsc{Schnitzler, Arthur} (15.05.1862 – 21.10.1931), \emph{Schriftsteller, Mediziner}!Geist im Wort und der Geist in der Tat1927@\strich\emph{Der Geist im Wort und der Geist in der Tat} {[}1927{]}|pwv}\pwindex{Schnitzler, Arthur 15.05.1862 – 21.10.1931@\textsc{Schnitzler, Arthur} (15.05.1862 – 21.10.1931), \emph{Schriftsteller, Mediziner}!Buch der Sprueche und Bedenken1927@\strich\emph{Buch der Sprüche und Bedenken} {[}1927{]}|pwv} und fragmentarisches\pwindex{Schnitzler, Arthur 15.05.1862 – 21.10.1931@\textsc{Schnitzler, Arthur} (15.05.1862 – 21.10.1931), \emph{Schriftsteller, Mediziner}!Geist im Wort und der Geist in der Tat1927@\strich\emph{Der Geist im Wort und der Geist in der Tat} {[}1927{]}|pwv} in Ordnung so gut es geht, ja einzelnes gewissermaßen
                    in Systeme\pwindex{Schnitzler, Arthur 15.05.1862 – 21.10.1931@\textsc{Schnitzler, Arthur} (15.05.1862 – 21.10.1931), \emph{Schriftsteller, Mediziner}!Buch der Sprueche und Bedenken1927@\strich\emph{Buch der Sprüche und Bedenken} {[}1927{]}|pwv}. Theatralisch
                    liegt allerlei angefangnes vor, – was ich zuerst fertig machen werde, weiß ich
                    noch nicht.\pend
           \pstart
           Um Ihre Aussee\oindex{Bad Aussee@\textbf{Bad Aussee}|pw}r Abgeschiedenheit beneid ich Sie
                    manchmal – weiß aber nicht, ob ich \strikeout{von} trotz
                    zeitweiliger Sehnsucht nach etwas der Art lange aushalten würde. Es ist
                    mancherlei Unruhe in meinem Leben; im ganzen fühl ich mich wohl, bei
                    gelegentlichen, am häufigsten durch das Gehörleiden verursachten und geförderten
                    Depressionen.\pend
           \pstart
           Ich hoffe Sie bald wiederzusehen.\pend
           \pstart
           Seien Sie von Herzen gegrüßt und bedankt!{\\[\baselineskip]}Ihr{\\[\baselineskip]}\spacefill\mbox{Arthur.}\pend
           \leftskip=0em{}
         
         \endnumbering\mylabel{h}\end{ledgroupsized}  \newcommand{\dateiname}{L02455}\newcommand{\titel}{Arthur Schnitzler an Hugo Hofmannsthal, 16. 11. 1925}\newcommand{\editorInnen}{Martin Anton Müller und Gerd-Hermann Susen}%% latex-leseansicht-abspann.tex
%% Abspann für die Leseansicht.
%% Der Schalter \ifkorrekturansicht ist bereits durch den Vorspann gesetzt.

%% latex-abspann.tex
%% Gemeinsamer Abspann für Korrekturansicht und Leseansicht.
%% Setzt den Schalter \ifkorrekturansicht voraus (gesetzt in den
%% einbindenden Dateien latex-korrekturansicht-abspann.tex bzw.
%% latex-leseansicht-abspann.tex).
%% ---------------------------------------------------------------

\normalsize

% Das esempio-Environment wird nur in der Leseansicht benötigt
\ifkorrekturansicht\else
\newenvironment{esempio}[3]%
{
    \vspace{1.5ex}
    \rlap{\underline{#1}}
    \par
    \setlength{\parindent}{0cm}
    \nopagebreak
    \leftskip=#2cm
    \rightskip=#3cm
}
{
    \par
}
\fi

\doendnotes{C}
\bigskip
\vfill

\clearpage

\footnotesize

\ifkorrekturansicht
  \lohead{\textsc{register}}
\fi

% theindex-Environment neu definieren ohne reledmac
\makeatletter
\renewenvironment{theindex}{%
  \ifkorrekturansicht
    \section*{\indexname}%
  \else
    \subsubsection*{Index der erwähnten Entitäten}%
  \fi
  \setlength{\parindent}{0pt}%
  \setlength{\parskip}{0pt plus 0.3pt}%
  \let\item\@idxitem
}{%
  \ifkorrekturansicht\clearpage\fi
}
\makeatother

\IfFileExists{\jobname-pw.ind}{\input{\jobname-pw.ind}}{}

% Quellenangabe nur in der Leseansicht
\ifkorrekturansicht\else
% Fallback-Definitionen, falls die .tex-Datei \titel etc. nicht gesetzt hat
\providecommand{\titel}{}
\providecommand{\editorInnen}{}
\providecommand{\dateiname}{\jobname}

\vspace{3cm}

\vfill

\footnotesize
\textsc{Quelle}: \titel. Herausgegeben von {\editorInnen}. In: \emph{Arthur Schnitzler: Briefwechsel mit Autorinnen und Autoren}.
 Digitale Edition, https://schnitzler-briefe.acdh.oeaw.ac.at/{\dateiname}.html (Stand \today)
\fi

\end{document}


      