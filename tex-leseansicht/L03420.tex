%% latex-korrekturansicht-vorspann.tex
%% Vorspann für die Korrekturansicht.
%% Lädt die gemeinsame Datei latex-vorspann.tex mit gesetztem Schalter.

\newif\ifkorrekturansicht
\korrekturansichttrue

\input{../tex-inputs/latex-vorspann}


\section[ Felix Salten an Arthur Schnitzler, 21. 4. {[}1906{]}]{L03420 Felix Salten an Arthur Schnitzler, 21. 4. {[}1906{]}}
\nopagebreak\mylabel{L03420v}
\rehead{ }\normalsize\beginnumbering\briefempfaengerindex{Schnitzler, Arthur@\textsc{Schnitzler, Arthur}!zzzSalten, Felix@\emph{von Felix Salten}!1906-04-211@{21. 4. {[}1906{]}}|(be}
\toendnotes[C]{\smallbreak\pagebreak[2]}\Standort{CUL, Schnitzler, B 89, B 1.}
\physDesc{Telegramm, 2 Blätter, 2 Seiten, 405 Zeichen
\newline{}maschinell
\newline{}Versand: 1) mit Bleistift abgeschnittener Vermerk des Namens des für die Transkription
                                 verantwortlichen Postbeamten bzw. der Postbeamtin: »\textsc{\textcolor{gray}{M}atter}\pwindex{Matter @\textsc{Matter}, \emph{Telegrafenbeamter/Telegrafenbeamtin}|pw}«  2) Stempel: »\nobreak{}\oindex{I., Innere Stadt@\textbf{I., Innere Stadt}, \emph{A.ADM3}|pwk}{\pb}{[}Wi{]}\textcolor{gray}{e}n 1/1\nobreak{}«. Stempel: »\nobreak{}21 Apr {[}1906{]}, 5\textsubscript{41}, Ausgefertigt\nobreak{}«. 
\newline{}Ordnung: mit Bleistift von unbekannter Hand nummeriert: »210a« }\toendnotes[C]{\smallbreak}
\pstart
           \centering{}{\pb},+ de charlottenburg\oindex{Charlottenburg@\textbf{Charlottenburg}, \emph{P.PPLX}|pw} 2454 61/60 21 4/25– s .=\pend
           \vspace{0.5em}
\pstart
           reicher\pwindex{Reicher, Emanuel 18.06.1849 – 15.05.1924@\textsc{Reicher, Emanuel} (18.06.1849 – 15.05.1924), \emph{Schauspieler/Schauspielerin}|pw}{ }\label{K_L03420-1v}\edtext{julian\pwindex{einsame Weg. Schauspiel in fuenf Akten@\emph{Der einsame Weg. Schauspiel in fünf Akten}|pwv} so vollstaendig
                  vergriffen}{\lemma{\textnormal{\emph{julian … vergriffen}}}\Cendnote{\textnormal{Zur Wiederaufnahme von \emph{Der einsame Weg}\pwindex{Theater. Der einsame Weg. – Othello. – Die Mitschuldigen. Der Tartueffe. – Der Fall Reinhardt@\emph{Theater. Der einsame Weg. – Othello. – Die Mitschuldigen. Der Tartüffe. – Der Fall Reinhardt}|pwk}{ }siehe Felix Salten u. a. an Arthur Schnitzler, 19. 4. 1906. Während Brahm\pwindex{Brahm, Otto 05.02.1856 – 28.11.1912@\textsc{Brahm, Otto} (05.02.1856 – 28.11.1912), \emph{Theaterleiter/Theaterleiterin, Regisseur/Regisseurin}|pwk} auf dieser Karte von einer »miserabeln
                     Aufführung« schrieb, dürfte auf dieses Telegramm von Salten\pwindex{Salten, Felix 06.09.1869 – 08.10.1945@\textsc{Salten, Felix} (06.09.1869 – 08.10.1945), \emph{Schriftsteller/Schriftstellerin, Journalist/Journalistin, Chefredakteur/Chefredakteurin}|pwk}{ }ein Brief von Schnitzler 
                     an Brahm\pwindex{Brahm, Otto 05.02.1856 – 28.11.1912@\textsc{Brahm, Otto} (05.02.1856 – 28.11.1912), \emph{Theaterleiter/Theaterleiterin, Regisseur/Regisseurin}|pwk} abgesandt worden sein. Am
                     22. 4. 1906 antwortete Brahm\pwindex{Brahm, Otto 05.02.1856 – 28.11.1912@\textsc{Brahm, Otto} (05.02.1856 – 28.11.1912), \emph{Theaterleiter/Theaterleiterin, Regisseur/Regisseurin}|pwk} jedenfalls auf (nicht erhaltene) Kritik an der Besetzung von Julian Fichtner\pwindex{einsame Weg. Schauspiel in fuenf Akten@\emph{Der einsame Weg. Schauspiel in fünf Akten}|pwkv} mit Emanuel Reicher\pwindex{Reicher, Emanuel 18.06.1849 – 15.05.1924@\textsc{Reicher, Emanuel} (18.06.1849 – 15.05.1924), \emph{Schauspieler/Schauspielerin}|pwk} (vgl. \emph{Der Briefwechsel Arthur Schnitzler – Otto Brahm}.
                           Vollständige Ausgabe. Herausgegeben, eingeleitet und erläutert von Oskar
                           Seidlin. Tübingen: \emph{Niemeyer}{ }1975, S. 225–226). In dieser Antwort geht Brahm\pwindex{Brahm, Otto 05.02.1856 – 28.11.1912@\textsc{Brahm, Otto} (05.02.1856 – 28.11.1912), \emph{Theaterleiter/Theaterleiterin, Regisseur/Regisseurin}|pwk} auch explizit auf Salten\pwindex{Salten, Felix 06.09.1869 – 08.10.1945@\textsc{Salten, Felix} (06.09.1869 – 08.10.1945), \emph{Schriftsteller/Schriftstellerin, Journalist/Journalistin, Chefredakteur/Chefredakteurin}|pwk} ein: »So scheint mir das Raisonnabelste, den
                        Julian\pwindex{einsame Weg. Schauspiel in fuenf Akten@\emph{Der einsame Weg. Schauspiel in fünf Akten}|pwv} des Reicher\pwindex{Reicher, Emanuel 18.06.1849 – 15.05.1924@\textsc{Reicher, Emanuel} (18.06.1849 – 15.05.1924), \emph{Schauspieler/Schauspielerin}|pw}, der übrigens auch bessere und
                     feinere Momente hat und den nur ein in Extravaganzen und Salten\pwindex{Salten, Felix 06.09.1869 – 08.10.1945@\textsc{Salten, Felix} (06.09.1869 – 08.10.1945), \emph{Schriftsteller/Schriftstellerin, Journalist/Journalistin, Chefredakteur/Chefredakteurin}|pw}-Mortale geübter Kompetenter unerträglich und
                     höchst gefahrvoll finden wird – es scheint mir, daß wir versuchen müssen, den
                        Reicher\pwindex{Reicher, Emanuel 18.06.1849 – 15.05.1924@\textsc{Reicher, Emanuel} (18.06.1849 – 15.05.1924), \emph{Schauspieler/Schauspielerin}|pw} besser zu machen.«
                     (S. 226.)}}}\label{K_L03420-1} und falsch ausserdem im text\pwindex{einsame Weg. Schauspiel in fuenf Akten@\emph{Der einsame Weg. Schauspiel in fünf Akten}|pwv} so unsicher dass ich es vorzog \label{K_L03420-2v}\edtext{ueberhaupt nichts ueber reprise}{\lemma{\textnormal{\emph{ueberhaupt … reprise}}}\Cendnote{\textnormal{Felix Salten\pwindex{Salten, Felix 06.09.1869 – 08.10.1945@\textsc{Salten, Felix} (06.09.1869 – 08.10.1945), \emph{Schriftsteller/Schriftstellerin, Journalist/Journalistin, Chefredakteur/Chefredakteurin}|pwk}: \emph{Theater. Der einsame Weg. – Othello. – Die Mitschuldigen.
                        Der Tartüffe. – Der Fall Reinhardt}\pwindex{Theater. Der einsame Weg. – Othello. – Die Mitschuldigen. Der Tartueffe. – Der Fall Reinhardt@\emph{Theater. Der einsame Weg. – Othello. – Die Mitschuldigen. Der Tartüffe. – Der Fall Reinhardt}|pwk}. In: \emph{B. Z. am Mittag}\pwindex{B.Z. am Mittag@\emph{B.Z. am Mittag}|pwk}, Jg. 30, Nr. 99, 28. 4. 1906, S. 2 u. 7. Salten\pwindex{Salten, Felix 06.09.1869 – 08.10.1945@\textsc{Salten, Felix} (06.09.1869 – 08.10.1945), \emph{Schriftsteller/Schriftstellerin, Journalist/Journalistin, Chefredakteur/Chefredakteurin}|pwk} behandelte vor allem die Bedeutung, die Berlin\oindex{Berlin@\textbf{Berlin}, \emph{P.PPLC}|pwk}er Inszenierungen mittlerweile für die Wien\oindex{Wien@\textbf{Wien}, \emph{A.ADM2}|pwk}erinnen und Wien\oindex{Wien@\textbf{Wien}, \emph{A.ADM2}|pwk}er hatten, um Bekanntschaft mit Wien\oindex{Wien@\textbf{Wien}, \emph{A.ADM2}|pwk}er Autoren auf der Bühne zu bekommen.}}}\label{K_L03420-2} zu schreiben\pwindex{Theater. Der einsame Weg. – Othello. – Die Mitschuldigen. Der Tartueffe. – Der Fall Reinhardt@\emph{Theater. Der einsame Weg. – Othello. – Die Mitschuldigen. Der Tartüffe. – Der Fall Reinhardt}|pwv}. halte einen anderen,
               vielleicht minder namhaften aber frischen schauspieler fuer {\pb}wien\oindex{Wien@\textbf{Wien}, \emph{A.ADM2}|pw} noch geeigneter als reicher\pwindex{Reicher, Emanuel 18.06.1849 – 15.05.1924@\textsc{Reicher, Emanuel} (18.06.1849 – 15.05.1924), \emph{Schauspieler/Schauspielerin}|pw} der die figur\pwindex{einsame Weg. Schauspiel in fuenf Akten@\emph{Der einsame Weg. Schauspiel in fünf Akten}|pwv} vom grund aus faelscht und viele schoenheiten der
               dichtung in wuesten umwandelt. herzlichst \spacefill\mbox{salten ,+}\pend
           \selectlanguage{ngerman}\endnumbering\briefempfaengerindex{Schnitzler, Arthur@\textsc{Schnitzler, Arthur}!zzzSalten, Felix@\emph{von Felix Salten}!1906-04-211@{21. 4. {[}1906{]}}|)be}\mylabel{L03420h}  \normalsize

\doendnotes{C}
\bigskip
\vfill

\clearpage

\footnotesize

\lohead{\textsc{register}}

% Definiere theindex-Environment komplett neu ohne reledmac
\makeatletter
\renewenvironment{theindex}{%
  \section*{\indexname}%
  \setlength{\parindent}{0pt}%
  \setlength{\parskip}{0pt plus 0.3pt}%
  \let\item\@idxitem
}{%
  \clearpage
}
\makeatother

\IfFileExists{\jobname-pw.ind}{\input{\jobname-pw.ind}}{}

\end{document}

      