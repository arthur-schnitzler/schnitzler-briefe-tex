%% latex-leseansicht-vorspann.tex
%% Vorspann für die Leseansicht.
%% Lädt die gemeinsame Datei latex-vorspann.tex mit nicht gesetztem Schalter.

\newif\ifkorrekturansicht
\korrekturansichtfalse

\input{../tex-inputs/latex-vorspann}

\begin{center}
            \textcolor{red}{ENTWURF, NICHT FERTIG KORRIGIERT}
                      \end{center}
            
         
         \renewcommand{\erwaehntePersonen}{Personen:  Matter, Emanuel Reicher, Rudolf Rittner}
         \renewcommand{\erwaehnteInstitutionen}{Institutionen: Lessing-Theater}
         \renewcommand{\erwaehnteOrte}{Orte: Berlin, Charlottenburg, I., Innere Stadt, Wien}
         \renewcommand{\erwaehnteWerke}{Werke: B.Z. am Mittag, Der einsame Weg. Schauspiel in fünf Akten, Theater. Der einsame Weg. – Othello. – Die Mitschuldigen. Der Tartüffe. – Der Fall Reinhardt}
               \section[Felix Salten an Arthur Schnitzler, 21. 4. {[}1906{]}]{ Felix Salten an Arthur Schnitzler, 21. 4. {[}1906{]}}\nopagebreak\mylabel{v}\rehead{ }\begin{ledgroupsized}[t]{13cm}\normalsize\beginnumbering \toendnotes[C]{\smallbreak\pagebreak[2]} \Standort{CUL, Schnitzler, B 89, B 1.}
\physDesc{Telegramm, 2 Blätter, 2 Seiten
\newline{}maschinell\newline{}Versand: 1) mit Bleistift abgeschnittener Vermerk des Namens des für die
                                 Transkription verantwortlichen Postbeamten, der Postbeamtin:
                                       »\textsc{Matter}\pwindex{Matter @\textsc{Matter}, \emph{Telegrafenbeamter/Telegrafenbeamtin}|pw}«  2) Stempel: »\nobreak{}\oindex{I., Innere Stadt@\textbf{I., Innere Stadt}|pwk}{\pb}{[}Wi{]}\textcolor{gray}{e}n
                                       1/1\nobreak{}«.  3) Stempel: »\nobreak{}21 Apr, 5 27, Ausgefertigt\nobreak{}«. \newline{}Ordnung: mit Bleistift von unbekannter Hand nummeriert:
                                    »210a« }\toendnotes[C]{\smallbreak}\pstart
           \centering{}{\pb} ,+ de charlottenburg\oindex{Charlottenburg@\textbf{Charlottenburg}|pw} 2454 61/60
                  21 4/25– s .=\pend
           \pstart
           reicher\pwindex{Reicher, Emanuel 18.06.1849 – 15.05.1924@\textsc{Reicher, Emanuel} (18.06.1849 – 15.05.1924), \emph{Schauspieler}|pw}{ }julian\pwindex{Schnitzler, Arthur 15.05.1862 – 21.10.1931@\textsc{Schnitzler, Arthur} (15.05.1862 – 21.10.1931), \emph{Schriftsteller, Mediziner}!einsame Weg. Schauspiel in fuenf Akten1904@\strich\emph{Der einsame Weg. Schauspiel in fünf Akten} {[}1904{]}|pwv} so vollstaendig
               vergriffen und falsch ausserdem im text so unsicher dass ich es \label{K_L03420-1v}\edtext{vorzog ueberhaupt nichts ueber reprise zu
                  schreiben\pwindex{Salten, Felix 06.09.1869 – 08.10.1945@\textsc{Salten, Felix} (06.09.1869 – 08.10.1945), \emph{Schriftsteller, Journalist}!Theater. Der einsame Weg. – Othello. – Die Mitschuldigen. Der Tartueffe. – Der Fall Reinhardt20?.4.1906@\strich\emph{Theater. Der einsame Weg. – Othello. – Die Mitschuldigen. Der Tartüffe. – Der Fall Reinhardt} {[}20?.4.1906{]}|pwv}}{\lemma{\textnormal{\emph{vorzog … schreiben}}}\Cendnote{\textnormal{Am 19. 4. 1906 wurde
                     \emph{Der einsame Weg}\pwindex{Schnitzler, Arthur 15.05.1862 – 21.10.1931@\textsc{Schnitzler, Arthur} (15.05.1862 – 21.10.1931), \emph{Schriftsteller, Mediziner}!einsame Weg. Schauspiel in fuenf Akten1904@\strich\emph{Der einsame Weg. Schauspiel in fünf Akten} {[}1904{]}|pwk} vom \emph{Lessing-Theater}\orgindex{Lessing-Theater@Lessing-Theater|pwk} in Berlin\oindex{Berlin@\textbf{Berlin}|pwk} als Neuaufnahme gegeben. Hintergrund bildete das bevorstehende
                  Gastspiel in Wien\oindex{Wien@\textbf{Wien}|pwk}, für das das \emph{Stück}\pwindex{Schnitzler, Arthur 15.05.1862 – 21.10.1931@\textsc{Schnitzler, Arthur} (15.05.1862 – 21.10.1931), \emph{Schriftsteller, Mediziner}!einsame Weg. Schauspiel in fuenf Akten1904@\strich\emph{Der einsame Weg. Schauspiel in fünf Akten} {[}1904{]}|pwk} geplant war. Im Zuge der Neuaufnahme war die
                  Besetzung der Hauptfigur von Rudolf Rittner\pwindex{Rittner, Rudolf 30.06.1869 – 04.02.1943@\textsc{Rittner, Rudolf} (30.06.1869 – 04.02.1943), \emph{Theaterleiter, Schauspieler}|pwk}
                  auf Emanuel Reicher\pwindex{Reicher, Emanuel 18.06.1849 – 15.05.1924@\textsc{Reicher, Emanuel} (18.06.1849 – 15.05.1924), \emph{Schauspieler}|pwk} übergegangen. Salten\pwindex{Salten, Felix 06.09.1869 – 08.10.1945@\textsc{Salten, Felix} (06.09.1869 – 08.10.1945), \emph{Schriftsteller, Journalist}|pwk} hatte eine Sammelrezension geschrieben
                        (Felix Salten\pwindex{Salten, Felix 06.09.1869 – 08.10.1945@\textsc{Salten, Felix} (06.09.1869 – 08.10.1945), \emph{Schriftsteller, Journalist}|pwk}: \emph{Theater. Der einsame Weg. – Othello. – Die Mitschuldigen.
                        Der Tartüffe. – Der Fall Reinhardt}\pwindex{Salten, Felix 06.09.1869 – 08.10.1945@\textsc{Salten, Felix} (06.09.1869 – 08.10.1945), \emph{Schriftsteller, Journalist}!Theater. Der einsame Weg. – Othello. – Die Mitschuldigen. Der Tartueffe. – Der Fall Reinhardt20?.4.1906@\strich\emph{Theater. Der einsame Weg. – Othello. – Die Mitschuldigen. Der Tartüffe. – Der Fall Reinhardt} {[}20?.4.1906{]}|pwk}. In: \emph{B. Z. am Mittag}\pwindex{?? Werk@Nicht ermittelte Verfasserinnen und Verfasser!B.Z. am Mittag1904-10-22 – 1943@\emph{B.Z. am Mittag} {[}1904-10-22 – 1943{]}|pwk}, Jg. XXXX, Nr. YY,
                        20. 4. 1906, S. YY–YY). Darin behandelt
                  er vor allem die Bedeutung, die Berlin\oindex{Berlin@\textbf{Berlin}|pwk}er
                  Inszenierungen mittlerweile für die Wien\oindex{Wien@\textbf{Wien}|pwk}erinnen
                  und Wien\oindex{Wien@\textbf{Wien}|pwk}er haben, um Bekanntschaft mit Wien\oindex{Wien@\textbf{Wien}|pwk}er Autoren auf er Bühne zu bekommen.}}}\label{K_L03420-1h}.
               halte einen anderen, vielleicht minder namhaften aber frischen schauspieler fuer {\pb}wien\oindex{Wien@\textbf{Wien}|pw} noch geeigneter als reicher\pwindex{Reicher, Emanuel 18.06.1849 – 15.05.1924@\textsc{Reicher, Emanuel} (18.06.1849 – 15.05.1924), \emph{Schauspieler}|pw} der die figur vom grund aus faelscht und viele
               schoenheiten der dichtung in wuesten umwandelt.\pend
           \pstart herzlichst \spacefill\mbox{salten ,+}\pend{}
         
         \endnumbering\mylabel{h}\end{ledgroupsized}\begin{anhang}\end{anhang}\newcommand{\dateiname}{L03420}\newcommand{\titel}{Felix Salten an Arthur Schnitzler, 21. 4. [1906]}\newcommand{\editorInnen}{Martin Anton Müller und Laura Untner}%% latex-leseansicht-abspann.tex
%% Abspann für die Leseansicht.
%% Der Schalter \ifkorrekturansicht ist bereits durch den Vorspann gesetzt.

%% latex-abspann.tex
%% Gemeinsamer Abspann für Korrekturansicht und Leseansicht.
%% Setzt den Schalter \ifkorrekturansicht voraus (gesetzt in den
%% einbindenden Dateien latex-korrekturansicht-abspann.tex bzw.
%% latex-leseansicht-abspann.tex).
%% ---------------------------------------------------------------

\normalsize

% Das esempio-Environment wird nur in der Leseansicht benötigt
\ifkorrekturansicht\else
\newenvironment{esempio}[3]%
{
    \vspace{1.5ex}
    \rlap{\underline{#1}}
    \par
    \setlength{\parindent}{0cm}
    \nopagebreak
    \leftskip=#2cm
    \rightskip=#3cm
}
{
    \par
}
\fi

\doendnotes{C}
\bigskip
\vfill

\clearpage

\footnotesize

\ifkorrekturansicht
  \lohead{\textsc{register}}
\fi

% theindex-Environment neu definieren ohne reledmac
\makeatletter
\renewenvironment{theindex}{%
  \ifkorrekturansicht
    \section*{\indexname}%
  \else
    \subsubsection*{Index der erwähnten Entitäten}%
  \fi
  \setlength{\parindent}{0pt}%
  \setlength{\parskip}{0pt plus 0.3pt}%
  \let\item\@idxitem
}{%
  \ifkorrekturansicht\clearpage\fi
}
\makeatother

\IfFileExists{\jobname-pw.ind}{\input{\jobname-pw.ind}}{}

% Quellenangabe nur in der Leseansicht
\ifkorrekturansicht\else
% Fallback-Definitionen, falls die .tex-Datei \titel etc. nicht gesetzt hat
\providecommand{\titel}{}
\providecommand{\editorInnen}{}
\providecommand{\dateiname}{\jobname}

\vspace{3cm}

\vfill

\footnotesize
\textsc{Quelle}: \titel. Herausgegeben von {\editorInnen}. In: \emph{Arthur Schnitzler: Briefwechsel mit Autorinnen und Autoren}.
 Digitale Edition, https://schnitzler-briefe.acdh.oeaw.ac.at/{\dateiname}.html (Stand \today)
\fi

\end{document}


      