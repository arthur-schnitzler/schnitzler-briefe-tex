%% latex-leseansicht-vorspann.tex
%% Vorspann für die Leseansicht.
%% Lädt die gemeinsame Datei latex-vorspann.tex mit nicht gesetztem Schalter.

\newif\ifkorrekturansicht
\korrekturansichtfalse

\input{../tex-inputs/latex-vorspann}


         
         \renewcommand{\erwaehntePersonen}{Personen: Otto Brahm,  Matter, Emanuel Reicher, Felix Salten}
         \renewcommand{\erwaehnteOrte}{Orte: Berlin, Charlottenburg, I., Innere Stadt, Wien}
         \renewcommand{\erwaehnteWerke}{Werke: B.Z. am Mittag, Der einsame Weg. Schauspiel in fünf Akten, Theater. Der einsame Weg. – Othello. – Die Mitschuldigen. Der Tartüffe. – Der Fall Reinhardt}
               \section[ Felix Salten an Arthur Schnitzler, 21. 4. {[}1906{]}]{ Felix Salten an Arthur Schnitzler, 21. 4. {[}1906{]}}\nopagebreak\mylabel{v}\rehead{ }\begin{ledgroupsized}[t]{13cm}\normalsize\beginnumbering\briefempfaengerindex{Schnitzler, Arthur@\textsc{Schnitzler, Arthur}!zzzSalten, Felix@\emph{von Felix Salten}!1906-04-211@{21. 4. {[}1906{]}}|(be} \toendnotes[C]{\smallbreak\pagebreak[2]} \Standort{CUL, Schnitzler, B 89, B 1.}
\physDesc{Telegramm, 2 Blätter, 2 Seiten, 405 Zeichen
\newline{}maschinell
\newline{}Versand: 1) mit Bleistift abgeschnittener Vermerk des Namens des für die Transkription
                                 verantwortlichen Postbeamten bzw. der Postbeamtin: »\textsc{\textcolor{gray}{M}atter}\pwindex{Matter @\textsc{Matter}, \emph{Telegrafenbeamter/Telegrafenbeamtin}|pw}«  2) Stempel: »\nobreak{}\oindex{I., Innere Stadt@\textbf{I., Innere Stadt}|pwk}{\pb}{[}Wi{]}\textcolor{gray}{e}n 1/1\nobreak{}«. Stempel: »\nobreak{}21 Apr {[}1906{]}, 5\textsubscript{41}, Ausgefertigt\nobreak{}«. 
\newline{}Ordnung: mit Bleistift von unbekannter Hand nummeriert: »210a« }\toendnotes[C]{\smallbreak}\pstart
           \centering{}{\pb},+ de charlottenburg\oindex{Charlottenburg@\textbf{Charlottenburg}|pw} 2454 61/60 21 4/25– s .=\pend
           \pstart
           reicher\pwindex{Reicher, Emanuel 18.06.1849 – 15.05.1924@\textsc{Reicher, Emanuel} (18.06.1849 – 15.05.1924), \emph{Schauspieler}|pw}{ }\label{K_L03420-1v}\edtext{julian\pwindex{Schnitzler, Arthur 15.05.1862 – 21.10.1931@\textsc{Schnitzler, Arthur} (15.05.1862 – 21.10.1931), \emph{Schriftsteller, Mediziner}!einsame Weg. Schauspiel in fuenf Akten1904@\strich\emph{Der einsame Weg. Schauspiel in fünf Akten} {[}1904{]}|pwv} so vollstaendig
                  vergriffen}{\lemma{\textnormal{\emph{julian … vergriffen}}}\Cendnote{\textnormal{Zur Wiederaufnahme von \emph{Der einsame Weg}\pwindex{Salten, Felix 06.09.1869 – 08.10.1945@\textsc{Salten, Felix} (06.09.1869 – 08.10.1945), \emph{Schriftsteller, Journalist, Chefredakteur}!Theater. Der einsame Weg. – Othello. – Die Mitschuldigen. Der Tartueffe. – Der Fall Reinhardt1906-04-28@\strich\emph{Theater. Der einsame Weg. – Othello. – Die Mitschuldigen. Der Tartüffe. – Der Fall Reinhardt} {[}1906-04-28{]}|pwk}{ }siehe Felix Salten u. a. an Arthur Schnitzler, 19. 4. 1906. Während Brahm\pwindex{Brahm, Otto 05.02.1856 – 28.11.1912@\textsc{Brahm, Otto} (05.02.1856 – 28.11.1912), \emph{Theaterleiter, Regisseur}|pwk} auf dieser Karte von einer »miserabeln
                     Aufführung« schrieb, dürfte auf dieses Telegramm von Salten\pwindex{Salten, Felix 06.09.1869 – 08.10.1945@\textsc{Salten, Felix} (06.09.1869 – 08.10.1945), \emph{Schriftsteller, Journalist, Chefredakteur}|pwk}{ }ein Brief von Schnitzler\pwindex{Schnitzler, Arthur 15.05.1862 – 21.10.1931@\textsc{Schnitzler, Arthur} (15.05.1862 – 21.10.1931), \emph{Schriftsteller, Mediziner}|pwk} 
                     an Brahm\pwindex{Brahm, Otto 05.02.1856 – 28.11.1912@\textsc{Brahm, Otto} (05.02.1856 – 28.11.1912), \emph{Theaterleiter, Regisseur}|pwk} abgesandt worden sein. Am
                     22. 4. 1906 antwortete Brahm\pwindex{Brahm, Otto 05.02.1856 – 28.11.1912@\textsc{Brahm, Otto} (05.02.1856 – 28.11.1912), \emph{Theaterleiter, Regisseur}|pwk} jedenfalls auf (nicht erhaltene) Kritik an der Besetzung von Julian Fichtner\pwindex{Schnitzler, Arthur 15.05.1862 – 21.10.1931@\textsc{Schnitzler, Arthur} (15.05.1862 – 21.10.1931), \emph{Schriftsteller, Mediziner}!einsame Weg. Schauspiel in fuenf Akten1904@\strich\emph{Der einsame Weg. Schauspiel in fünf Akten} {[}1904{]}|pwkv} mit Emanuel Reicher\pwindex{Reicher, Emanuel 18.06.1849 – 15.05.1924@\textsc{Reicher, Emanuel} (18.06.1849 – 15.05.1924), \emph{Schauspieler}|pwk} (vgl. \emph{Der Briefwechsel Arthur Schnitzler – Otto Brahm}.
                           Vollständige Ausgabe. Herausgegeben, eingeleitet und erläutert von Oskar
                           Seidlin. Tübingen: \emph{Niemeyer}{ }1975, S. 225–226). In dieser Antwort geht Brahm\pwindex{Brahm, Otto 05.02.1856 – 28.11.1912@\textsc{Brahm, Otto} (05.02.1856 – 28.11.1912), \emph{Theaterleiter, Regisseur}|pwk} auch explizit auf Salten\pwindex{Salten, Felix 06.09.1869 – 08.10.1945@\textsc{Salten, Felix} (06.09.1869 – 08.10.1945), \emph{Schriftsteller, Journalist, Chefredakteur}|pwk} ein: »So scheint mir das Raisonnabelste, den
                        Julian\pwindex{Schnitzler, Arthur 15.05.1862 – 21.10.1931@\textsc{Schnitzler, Arthur} (15.05.1862 – 21.10.1931), \emph{Schriftsteller, Mediziner}!einsame Weg. Schauspiel in fuenf Akten1904@\strich\emph{Der einsame Weg. Schauspiel in fünf Akten} {[}1904{]}|pwv} des Reicher\pwindex{Reicher, Emanuel 18.06.1849 – 15.05.1924@\textsc{Reicher, Emanuel} (18.06.1849 – 15.05.1924), \emph{Schauspieler}|pw}, der übrigens auch bessere und
                     feinere Momente hat und den nur ein in Extravaganzen und Salten\pwindex{Salten, Felix 06.09.1869 – 08.10.1945@\textsc{Salten, Felix} (06.09.1869 – 08.10.1945), \emph{Schriftsteller, Journalist, Chefredakteur}|pw}-Mortale geübter Kompetenter unerträglich und
                     höchst gefahrvoll finden wird – es scheint mir, daß wir versuchen müssen, den
                        Reicher\pwindex{Reicher, Emanuel 18.06.1849 – 15.05.1924@\textsc{Reicher, Emanuel} (18.06.1849 – 15.05.1924), \emph{Schauspieler}|pw} besser zu machen.«
                     (S. 226.)}}}\label{K_L03420-1h} und falsch ausserdem im text\pwindex{Schnitzler, Arthur 15.05.1862 – 21.10.1931@\textsc{Schnitzler, Arthur} (15.05.1862 – 21.10.1931), \emph{Schriftsteller, Mediziner}!einsame Weg. Schauspiel in fuenf Akten1904@\strich\emph{Der einsame Weg. Schauspiel in fünf Akten} {[}1904{]}|pwv} so unsicher dass ich es vorzog \label{K_L03420-2v}\edtext{ueberhaupt nichts ueber reprise}{\lemma{\textnormal{\emph{ueberhaupt … reprise}}}\Cendnote{\textnormal{Felix Salten\pwindex{Salten, Felix 06.09.1869 – 08.10.1945@\textsc{Salten, Felix} (06.09.1869 – 08.10.1945), \emph{Schriftsteller, Journalist, Chefredakteur}|pwk}: \emph{Theater. Der einsame Weg. – Othello. – Die Mitschuldigen.
                        Der Tartüffe. – Der Fall Reinhardt}\pwindex{Salten, Felix 06.09.1869 – 08.10.1945@\textsc{Salten, Felix} (06.09.1869 – 08.10.1945), \emph{Schriftsteller, Journalist, Chefredakteur}!Theater. Der einsame Weg. – Othello. – Die Mitschuldigen. Der Tartueffe. – Der Fall Reinhardt1906-04-28@\strich\emph{Theater. Der einsame Weg. – Othello. – Die Mitschuldigen. Der Tartüffe. – Der Fall Reinhardt} {[}1906-04-28{]}|pwk}. In: \emph{B. Z. am Mittag}\pwindex{?? Werk@Nicht ermittelte Verfasserinnen und Verfasser!B.Z. am Mittag1904-10-22 – 1943@\emph{B.Z. am Mittag} {[}1904-10-22 – 1943{]}|pwk}, Jg. 30, Nr. 99, 28. 4. 1906, S. 2 u. 7. Salten\pwindex{Salten, Felix 06.09.1869 – 08.10.1945@\textsc{Salten, Felix} (06.09.1869 – 08.10.1945), \emph{Schriftsteller, Journalist, Chefredakteur}|pwk} behandelte vor allem die Bedeutung, die Berlin\oindex{Berlin@\textbf{Berlin}|pwk}er Inszenierungen mittlerweile für die Wien\oindex{Wien@\textbf{Wien}|pwk}erinnen und Wien\oindex{Wien@\textbf{Wien}|pwk}er hatten, um Bekanntschaft mit Wien\oindex{Wien@\textbf{Wien}|pwk}er Autoren auf der Bühne zu bekommen.}}}\label{K_L03420-2h} zu schreiben\pwindex{Salten, Felix 06.09.1869 – 08.10.1945@\textsc{Salten, Felix} (06.09.1869 – 08.10.1945), \emph{Schriftsteller, Journalist, Chefredakteur}!Theater. Der einsame Weg. – Othello. – Die Mitschuldigen. Der Tartueffe. – Der Fall Reinhardt1906-04-28@\strich\emph{Theater. Der einsame Weg. – Othello. – Die Mitschuldigen. Der Tartüffe. – Der Fall Reinhardt} {[}1906-04-28{]}|pwv}. halte einen anderen,
               vielleicht minder namhaften aber frischen schauspieler fuer {\pb}wien\oindex{Wien@\textbf{Wien}|pw} noch geeigneter als reicher\pwindex{Reicher, Emanuel 18.06.1849 – 15.05.1924@\textsc{Reicher, Emanuel} (18.06.1849 – 15.05.1924), \emph{Schauspieler}|pw} der die figur\pwindex{Schnitzler, Arthur 15.05.1862 – 21.10.1931@\textsc{Schnitzler, Arthur} (15.05.1862 – 21.10.1931), \emph{Schriftsteller, Mediziner}!einsame Weg. Schauspiel in fuenf Akten1904@\strich\emph{Der einsame Weg. Schauspiel in fünf Akten} {[}1904{]}|pwv} vom grund aus faelscht und viele schoenheiten der
               dichtung in wuesten umwandelt. herzlichst \spacefill\mbox{salten ,+}\pend
           
         
         \endnumbering\mylabel{h}\end{ledgroupsized}  \newcommand{\dateiname}{L03420}\newcommand{\titel}{Felix Salten an Arthur Schnitzler, 21. 4. [1906]}\newcommand{\editorInnen}{Martin Anton Müller und Laura Untner}%% latex-leseansicht-abspann.tex
%% Abspann für die Leseansicht.
%% Der Schalter \ifkorrekturansicht ist bereits durch den Vorspann gesetzt.

%% latex-abspann.tex
%% Gemeinsamer Abspann für Korrekturansicht und Leseansicht.
%% Setzt den Schalter \ifkorrekturansicht voraus (gesetzt in den
%% einbindenden Dateien latex-korrekturansicht-abspann.tex bzw.
%% latex-leseansicht-abspann.tex).
%% ---------------------------------------------------------------

\normalsize

% Das esempio-Environment wird nur in der Leseansicht benötigt
\ifkorrekturansicht\else
\newenvironment{esempio}[3]%
{
    \vspace{1.5ex}
    \rlap{\underline{#1}}
    \par
    \setlength{\parindent}{0cm}
    \nopagebreak
    \leftskip=#2cm
    \rightskip=#3cm
}
{
    \par
}
\fi

\doendnotes{C}
\bigskip
\vfill

\clearpage

\footnotesize

\ifkorrekturansicht
  \lohead{\textsc{register}}
\fi

% theindex-Environment neu definieren ohne reledmac
\makeatletter
\renewenvironment{theindex}{%
  \ifkorrekturansicht
    \section*{\indexname}%
  \else
    \subsubsection*{Index der erwähnten Entitäten}%
  \fi
  \setlength{\parindent}{0pt}%
  \setlength{\parskip}{0pt plus 0.3pt}%
  \let\item\@idxitem
}{%
  \ifkorrekturansicht\clearpage\fi
}
\makeatother

\IfFileExists{\jobname-pw.ind}{\input{\jobname-pw.ind}}{}

% Quellenangabe nur in der Leseansicht
\ifkorrekturansicht\else
% Fallback-Definitionen, falls die .tex-Datei \titel etc. nicht gesetzt hat
\providecommand{\titel}{}
\providecommand{\editorInnen}{}
\providecommand{\dateiname}{\jobname}

\vspace{3cm}

\vfill

\footnotesize
\textsc{Quelle}: \titel. Herausgegeben von {\editorInnen}. In: \emph{Arthur Schnitzler: Briefwechsel mit Autorinnen und Autoren}.
 Digitale Edition, https://schnitzler-briefe.acdh.oeaw.ac.at/{\dateiname}.html (Stand \today)
\fi

\end{document}


      