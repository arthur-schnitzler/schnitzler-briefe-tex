%% latex-leseansicht-vorspann.tex
%% Vorspann für die Leseansicht.
%% Lädt die gemeinsame Datei latex-vorspann.tex mit nicht gesetztem Schalter.

\newif\ifkorrekturansicht
\korrekturansichtfalse

\input{../tex-inputs/latex-vorspann}


         
         \renewcommand{\erwaehntePersonen}{Personen: Richard Beer-Hofmann, Olga Schnitzler}
         \renewcommand{\erwaehnteOrte}{Orte: Bad Aussee, Salzburg, St. Gilgen, Villa Frühling}
         \renewcommand{\erwaehnteWerke}{Werke: Der Graf von Charolais. Ein Trauerspiel}
               \section[Arthur Schnitzler an Richard Beer-Hofmann, 10. 9. 1904]{ Arthur Schnitzler an Richard Beer-Hofmann, 10. 9. 1904}\nopagebreak\mylabel{v}\rehead{ }\begin{ledgroupsized}[t]{13cm}\normalsize\beginnumbering \toendnotes[C]{\smallbreak\pagebreak[2]} \Standort{YCGL, MSS 31.}
\physDesc{Brief, 1 Blatt, 3 Seiten, Umschlag, 555 Zeichen
\newline{}Handschrift: Bleistift, deutsche Kurrent
\newline{}Versand: 1) Stempel: »\nobreak{}\oindex{St. Gilgen@\textbf{St. Gilgen}|pwk}St. Gilge\textcolor{gray}{n}, 10{[}. 9. 1904{]}\nobreak{}«.   2) Stempel: »\nobreak{}\oindex{Bad Aussee@\textbf{Bad Aussee}|pwk}{\pb}Aussee in
                                       Steiermark, 11/9 04\nobreak{}«. }\toendnotes[C]{\smallbreak}\pstart{}{\pb}\textsc{Herrn Dr Richard Beer-Hofmann}\pend{}\pstart{}\textsc{Markt Aussee\oindex{Bad Aussee@\textbf{Bad Aussee}|pw}}\pend{}\pstart{}\textsc{Villa Frühling}.\oindex{Villa Fruehling@\textbf{Villa Frühling}|pw}\pend{}{\bigskip}\pstart
           \raggedleft{}{\pb}Samſtag 10. 9. 904\pend
           \pstart
           lieber Richard, ich ſchlage Ihnen vor: Verlaſſen Sie etwa
                  Mittwoch{ }Auſſee\oindex{Bad Aussee@\textbf{Bad Aussee}|pw}, kommen Sie hieher, bleiben Sie 3–4 Tage,
               leſen Sie mir Ihr Stück\pwindex{Beer-Hofmann, Richard 1866-07-11 – 1945-09-26@\textsc{Beer-Hofmann, Richard} (1866-07-11 – 1945-09-26), \emph{Schriftsteller}!Graf von Charolais. Ein Trauerspiel1904-12-23@\strich\emph{Der Graf von Charolais. Ein Trauerspiel} {[}1904-12-23{]}|pwv} vor;
               wir fahren da{\geminationn} mit Ihnen z. E. Montag nach
                  Salzburg\oindex{Salzburg@\textbf{Salzburg}|pw}, woſelbſt wir einige Tage ver{\pb}bringen. Auſſee\oindex{Bad Aussee@\textbf{Bad Aussee}|pw}
               würde mich ja ſehr reizen, we{\geminationn} ich nicht ein ziemlich
               ausgeſprochenes Ruhebedürfnis und einige Scheu vor Hin u Herfahrerei, Ein-
               Auspackerei hätte. Olga\pwindex{Schnitzler, Olga 17.01.1882 – 13.01.1970@\textsc{Schnitzler, Olga} (17.01.1882 – 13.01.1970), \emph{Schauspielerin, Sängerin}|pw} geht es ungefähr
               ebenſo.\pend
           \pstart
           Theilen Sie mir bitte Ihren {\pb}Entſchluſs ev.
               telegraphiſch mit.\pend
           \pstart
           Herzlichſt{\\[\baselineskip]}Ihr{\\[\baselineskip]}\spacefill\mbox{A.}\pend
           \leftskip=0em{}
         
         \endnumbering\mylabel{h}\end{ledgroupsized}  \newcommand{\dateiname}{L01442}\newcommand{\titel}{Arthur Schnitzler an Richard Beer-Hofmann, 10. 9. 1904}\newcommand{\editorInnen}{Martin Anton Müller und Gerd-Hermann Susen}%% latex-leseansicht-abspann.tex
%% Abspann für die Leseansicht.
%% Der Schalter \ifkorrekturansicht ist bereits durch den Vorspann gesetzt.

%% latex-abspann.tex
%% Gemeinsamer Abspann für Korrekturansicht und Leseansicht.
%% Setzt den Schalter \ifkorrekturansicht voraus (gesetzt in den
%% einbindenden Dateien latex-korrekturansicht-abspann.tex bzw.
%% latex-leseansicht-abspann.tex).
%% ---------------------------------------------------------------

\normalsize

% Das esempio-Environment wird nur in der Leseansicht benötigt
\ifkorrekturansicht\else
\newenvironment{esempio}[3]%
{
    \vspace{1.5ex}
    \rlap{\underline{#1}}
    \par
    \setlength{\parindent}{0cm}
    \nopagebreak
    \leftskip=#2cm
    \rightskip=#3cm
}
{
    \par
}
\fi

\doendnotes{C}
\bigskip
\vfill

\clearpage

\footnotesize

\ifkorrekturansicht
  \lohead{\textsc{register}}
\fi

% theindex-Environment neu definieren ohne reledmac
\makeatletter
\renewenvironment{theindex}{%
  \ifkorrekturansicht
    \section*{\indexname}%
  \else
    \subsubsection*{Index der erwähnten Entitäten}%
  \fi
  \setlength{\parindent}{0pt}%
  \setlength{\parskip}{0pt plus 0.3pt}%
  \let\item\@idxitem
}{%
  \ifkorrekturansicht\clearpage\fi
}
\makeatother

\IfFileExists{\jobname-pw.ind}{\input{\jobname-pw.ind}}{}

% Quellenangabe nur in der Leseansicht
\ifkorrekturansicht\else
% Fallback-Definitionen, falls die .tex-Datei \titel etc. nicht gesetzt hat
\providecommand{\titel}{}
\providecommand{\editorInnen}{}
\providecommand{\dateiname}{\jobname}

\vspace{3cm}

\vfill

\footnotesize
\textsc{Quelle}: \titel. Herausgegeben von {\editorInnen}. In: \emph{Arthur Schnitzler: Briefwechsel mit Autorinnen und Autoren}.
 Digitale Edition, https://schnitzler-briefe.acdh.oeaw.ac.at/{\dateiname}.html (Stand \today)
\fi

\end{document}


      