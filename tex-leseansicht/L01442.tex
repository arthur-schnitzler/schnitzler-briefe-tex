%% latex-korrekturansicht-vorspann.tex
%% Vorspann für die Korrekturansicht.
%% Lädt die gemeinsame Datei latex-vorspann.tex mit gesetztem Schalter.

\newif\ifkorrekturansicht
\korrekturansichttrue

\input{../tex-inputs/latex-vorspann}


\section[Arthur Schnitzler an Richard Beer-Hofmann, 10. 9. 1904]{L01442 Arthur Schnitzler an Richard Beer-Hofmann, 10. 9. 1904}
\nopagebreak\mylabel{L01442v}
\rehead{ }\normalsize\beginnumbering\briefempfaengerindex{Beer-Hofmann, Richard@\textsc{Beer-Hofmann, Richard}!zzzSchnitzler, Arthur@\emph{von Arthur Schnitzler}!1904-09-101@{10. 9. 1904}|(be}
\toendnotes[C]{\smallbreak\pagebreak[2]}\Standort{YCGL, MSS 31.}
\physDesc{Brief, 1 Blatt, 3 Seiten, Umschlag, 555 Zeichen
\newline{}Handschrift: Bleistift, deutsche Kurrent
\newline{}Versand: 1) Stempel: »\nobreak{}\oindex{St. Gilgen@\textbf{St. Gilgen}, \emph{A.ADM3}|pwk}St. Gilge\textcolor{gray}{n}, 10{[}. 9. 1904{]}\nobreak{}«.   2) Stempel: »\nobreak{}\oindex{Bad Aussee@\textbf{Bad Aussee}, \emph{P.PPLA3}|pwk}{\pb}Aussee in
                                       Steiermark, 11/9 04\nobreak{}«. }\toendnotes[C]{\smallbreak}\pstart{}{\pb}\textsc{Herrn Dr Richard Beer-Hofmann}\pend{}\pstart{}\textsc{Markt Aussee\oindex{Bad Aussee@\textbf{Bad Aussee}, \emph{P.PPLA3}|pw}}\pend{}\pstart{}\textsc{Villa Frühling}.\oindex{Villa Fruehling@\textbf{Villa Frühling}, \emph{Gebäude (K.GBD)}|pw}\pend{}{\bigskip}\vspace{1em}
\pstart
           \raggedleft{}{\pb}Samſtag 10. 9. 904\pend
           \vspace{0.5em}
\pstart
           lieber Richard, ich ſchlage Ihnen vor: Verlaſſen Sie etwa
                  Mittwoch{ }Auſſee\oindex{Bad Aussee@\textbf{Bad Aussee}, \emph{P.PPLA3}|pw}, kommen Sie hieher, bleiben Sie 3–4 Tage,
               leſen Sie mir Ihr Stück\pwindex{Graf von Charolais. Ein Trauerspiel@\emph{Der Graf von Charolais. Ein Trauerspiel}|pwv} vor;
               wir fahren da{\geminationn} mit Ihnen z. E. Montag nach
                  Salzburg\oindex{Salzburg@\textbf{Salzburg}, \emph{A.ADM2}|pw}, woſelbſt wir einige Tage ver{\pb}bringen. Auſſee\oindex{Bad Aussee@\textbf{Bad Aussee}, \emph{P.PPLA3}|pw}
               würde mich ja ſehr reizen, we{\geminationn} ich nicht ein ziemlich
               ausgeſprochenes Ruhebedürfnis und einige Scheu vor Hin u Herfahrerei, Ein-
               Auspackerei hätte. Olga\pwindex{Schnitzler, Olga 17.01.1882 – 13.01.1970@\textsc{Schnitzler, Olga} (17.01.1882 – 13.01.1970), \emph{Schauspieler/Schauspielerin, Sänger/Sängerin}|pw} geht es ungefähr
               ebenſo.\pend
           
\pstart
           Theilen Sie mir bitte Ihren {\pb}Entſchluſs ev.
               telegraphiſch mit.\pend
           
\pstart
           Herzlichſt{\\[\baselineskip]}Ihr{\\[\baselineskip]}\spacefill\mbox{A.}\pend
           \leftskip=0em{}\selectlanguage{ngerman}\endnumbering\briefempfaengerindex{Beer-Hofmann, Richard@\textsc{Beer-Hofmann, Richard}!zzzSchnitzler, Arthur@\emph{von Arthur Schnitzler}!1904-09-101@{10. 9. 1904}|)be}\mylabel{L01442h}  \normalsize

\doendnotes{C}
\bigskip
\vfill

\clearpage

\footnotesize

\lohead{\textsc{register}}

% Definiere theindex-Environment komplett neu ohne reledmac
\makeatletter
\renewenvironment{theindex}{%
  \section*{\indexname}%
  \setlength{\parindent}{0pt}%
  \setlength{\parskip}{0pt plus 0.3pt}%
  \let\item\@idxitem
}{%
  \clearpage
}
\makeatother

\IfFileExists{\jobname-pw.ind}{\input{\jobname-pw.ind}}{}

\end{document}

      