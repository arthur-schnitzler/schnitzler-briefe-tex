\input{../tex-inputs/latex-pdf-vorspann}
\begin{center}
            \textcolor{red}{ENTWURF. ENTZIFFERUNG NOCH NICHT KORREKTURGELESEN}
                      \end{center}
            
               \section[Arthur Schnitzler an Richard Beer-Hofmann, 10. 9. 1904]{ Arthur Schnitzler an Richard Beer-Hofmann, 10. 9. 1904}\nopagebreak\mylabel{v}\rehead{ }\begin{ledgroupsized}[t]{13cm}\normalsize\beginnumbering\briefempfaengerindex{Beer-Hofmann, Richard@\textsc{Beer-Hofmann, Richard}!zzzSchnitzler, Arthur@\emph{von Arthur Schnitzler}!1904-09-101@{10. 9. 1904}|(be} \toendnotes[C]{\smallbreak\pagebreak[2]} \Standort{YCGL, MSS 31.}
\physDesc{Brief, 1 Blatt, 3 Seiten, Umschlag
\newline{}Handschrift: Bleistift, deutsche Kurrent\newline{}Versand: 1) Stempel: »\nobreak{}\oindex{St. Gilgen@\textbf{St. Gilgen}|pwk}St. Gilge\textcolor{gray}{n}, 10{[}. 9. 1904{]}\nobreak{}«.  2) Stempel: »\nobreak{}\oindex{Bad Aussee@\textbf{Bad Aussee}|pwk}{\pb}Aussee in
                                       Steiermark, 11/9 04\nobreak{}«. }\toendnotes[C]{\smallbreak}\pstart{}{\pb}\textsc{Herrn Dr Richard Beer-Hofmann}\pend{}\pstart{}\textsc{Markt Aussee\oindex{Bad Aussee@\textbf{Bad Aussee}|pw}}\pend{}\pstart{}\textsc{Villa Frühling}.\oindex{Villa Fruehling@\textbf{Villa Frühling}|pw}\pend{}{\bigskip}\pstart
           \raggedleft{}{\pb}Samſtag 10. 9. 904\pend
           \pstart
           lieber Richard, ich ſchlage Ihnen vor: Verlaſſen Sie etwa
                  Mittwoch{ }Auſſee\oindex{Bad Aussee@\textbf{Bad Aussee}|pw}, kommen Sie hieher, bleiben Sie 3–4 Tage,
               leſen Sie mir Ihr Stück\pwindex{Beer-Hofmann, Richard 11.07.1866 – 26.09.1945@\textsc{Beer-Hofmann, Richard} (11.07.1866 – 26.09.1945), \emph{Schriftsteller}!Graf von Charolais. Ein Trauerspiel1904-12-23 – 1904-12-23@\strich\emph{Der Graf von Charolais. Ein Trauerspiel} {[}1904-12-23 – 1904-12-23{]}|pwv} vor; wir
               fahren da{\geminationn} mit Ihnen z. E. Montag nach Salzburg\oindex{Salzburg@\textbf{Salzburg}|pw}, woſelbſt wir einige Tage ver{\pb}bringen. Auſſee\oindex{Bad Aussee@\textbf{Bad Aussee}|pw} würde
               mich ja ſehr reizen, we{\geminationn} ich nicht ein ziemlich
               ausgeſprochenes Ruhebedürfnis und einige Scheu vor Hin u Herfahrerei, Ein-
               Auspackerei hätte. Olga\pwindex{Schnitzler, Olga 17.01.1882 – 13.01.1970@\textsc{Schnitzler, Olga} (17.01.1882 – 13.01.1970), \emph{Schauspielerin, Sängerin}|pw} geht es ungefähr
               ebenſo.\pend
           \pstart
           Theilen Sie mir bitte Ihren {\pb}Entſchluſs ev.
               telegraphiſch mit.\pend
           \pstart
           Herzlichſt{\\[\baselineskip]}Ihr{\\[\baselineskip]}\spacefill\mbox{A.}\pend
           \leftskip=0em{}\endnumbering\briefempfaengerindex{Beer-Hofmann, Richard@\textsc{Beer-Hofmann, Richard}!zzzSchnitzler, Arthur@\emph{von Arthur Schnitzler}!1904-09-101@{10. 9. 1904}|)be}\mylabel{h}\end{ledgroupsized}  \newcommand{\dateiname}{L01442}\newcommand{\titel}{Arthur Schnitzler an Richard Beer-Hofmann, 10. 9. 1904}\newcommand{\editorInnen}{Martin Anton Müller und Gerd-Hermann Susen}\input{../tex-inputs/latex-pdf-abspann}
      