%% latex-korrekturansicht-vorspann.tex
%% Vorspann für die Korrekturansicht.
%% Lädt die gemeinsame Datei latex-vorspann.tex mit gesetztem Schalter.

\newif\ifkorrekturansicht
\korrekturansichttrue

\input{../tex-inputs/latex-vorspann}


\section[Hugo von Hofmannsthal an Arthur Schnitzler, 11. 9. 1911]{L02027 Hugo von Hofmannsthal an Arthur Schnitzler, 11. 9. 1911}
\nopagebreak\mylabel{L02027v}
\rehead{ }\normalsize\beginnumbering\briefempfaengerindex{Schnitzler, Arthur@\textsc{Schnitzler, Arthur}!zzzHofmannsthal, Hugo von@\emph{von Hugo von Hofmannsthal}!1911-09-111@{11. 9. 1911}|(be}
\toendnotes[C]{\smallbreak\pagebreak[2]}\Standort{CUL, Schnitzler, B 43.}
\physDesc{Brief, 1 Blatt, 4 Seiten, 1807 Zeichen
\newline{}Handschrift: schwarze Tinte, deutsche Kurrent
\newline{}Schnitzler: mit Bleistift die Jahreszahl ergänzt: »911« und beschriftet: »\textsc{Hugo}« 
\newline{}Ordnung: 1) mit Bleistift von unbekannter Hand nummeriert: »\strikeout{323}«  2) mit Bleistift von unbekannter Hand nummeriert:
                                    »332«}
\buchAbdrucke{\weitereDrucke{Hugo von Hofmannsthal, Arthur Schnitzler: \emph{Briefwechsel}. Frankfurt am Main: \emph{S. Fischer} 1964, S. 262.} }\toendnotes[C]{\smallbreak}
\pstart
           \raggedleft{}{\pb}\textsc{Aussee}\oindex{Bad Aussee@\textbf{Bad Aussee}, \emph{P.PPLA3}|pw}, 11. IX.\pend
           
\pstart{}mein lieber Arthur \pend\vspace{0.5em}
\pstart
           die traurige Nachricht fand ich, nach einigen trüben Andeutungen durch Freunde, heute
               morgens in der Zeitung – ſo war es unmöglich, zurechtzukommen, um dem \label{K_L02027-1v}\edtext{Begräbnis}{\lemma{\textnormal{\emph{Begräbnis}}}\Cendnote{\textnormal{Dieses fand an eben diesem Tag, dem 11. 9. 1911, statt.}}}\label{K_L02027-1} Ihrer guten Mutter\pwindex{Schnitzler, Louise 1840-07-08 – 1911-09-09@\textsc{Schnitzler, Louise} (1840-07-08 – 1911-09-09)|pwv} beizuwohnen.\hspace*{1.5em}Daſs jemand nicht mehr iſt, iſt auch für den
               Fernerſtehenden unfaſsbar, ja es iſt, als antwortete das menſchliche Innere {\pb}auf die Zumutung, dies
               hinzunehmen, mit einer verdoppelten Lebhaftigkeit der Vorſtellung. So lebt Ihre Mutter\pwindex{Schnitzler, Louise 1840-07-08 – 1911-09-09@\textsc{Schnitzler, Louise} (1840-07-08 – 1911-09-09)|pwv} für mich in dieſen
               Stunden – und immer wieder, nach 10 nach 15, nach 20 Jahren kommt für mich ein
               einſamer Spaziergang, eine ſtockende Arbeitsſtunde, in der ein Todter ſo völlig
               auflebt, dies iſt eines der Geheimniſſe unseres Innern.\pend
           
\pstart
           Es iſt mir ein lieber Gedanke, daſs Sie nach der Qual dieſer Tage daran {\pb}gehen, ein dichteriſches Gebilde\pwindex{weite Land. Tragikomoedie in fuenf Akten@\emph{Das weite Land. Tragikomödie in fünf Akten}|pwv}, in dem ſo viel Ihres
               ſtärkſten wahrſten inneren Lebens zuſammengedrängt iſt, auf die Bühne
                  {[}zu{]} bringen. Daſs man auf dieſe Weiſe, ebenſo wie in den Kindern\pwindex{Zimmer, Christiane 14.05.1902 – 05.01.1987@\textsc{Zimmer, Christiane} (14.05.1902 – 05.01.1987)|pwv}\pwindex{Hofmannsthal, Raimund von 26.5.1906 – 20.03.1974@\textsc{Hofmannsthal, Raimund von} (26.5.1906 – 20.03.1974)|pwv}\pwindex{Hofmannsthal, Franz von 20.10.1903 – 13.07.1929@\textsc{Hofmannsthal, Franz von} (20.10.1903 – 13.07.1929)|pwv},
               irgend etwas von ſich weitergibt, gleichſam ans Unendliche weitergibt, iſt für mich
               eine von den Compenſationen. Es gibt noch geheimnisvollere, wenn man in das Myſterium
               des Lebens eindringt, wie es manchmal geſtattet, aber {\pb}nicht mitteilbar iſt. In den
               Tiefen der Arbeit liegen ſie und auch in den Tiefen des \substVorne{}\textsuperscript{A}\substDazwischen{}a\substHinten{}ufnehmenden Lebens, und ſind Ihnen bekannt wie mir. – Es ſcheint mir in
               manchen Momenten als das einzig Natürliche, jetzt zu Ihnen zu fahren und Tage bei
               Ihnen zu ſein. Ich thäte es augenblicklich, wären Sie auf dem Lande, wo ich wirklich
               andauernd bei Ihnen wäre.\pend
           
\pstart
           Auch hält mich noch etwas zurück. Mein Vater\pwindex{Hofmannsthal, Hugo August von 21.12.1841 – 08.12.1915@\textsc{Hofmannsthal, Hugo August von} (21.12.1841 – 08.12.1915), \emph{Bankdirektor/Bankdirektorin}|pwv} war dieſen ganzen ſchweren So{\geminationm}er in Wien\oindex{Wien@\textbf{Wien}, \emph{A.ADM2}|pw}, iſt jetzt bei uns und freut ſich auf eine kleine aufheiternde Reiſe nach
                  Hamburg\oindex{Hamburg@\textbf{Hamburg}, \emph{P.PPLA}|pw} u. Kopenhagen\oindex{Kopenhagen@\textbf{Kopenhagen}, \emph{P.PPLC}|pw}, \label{T_L02027-1v}\edtext{der ich auch meine
               Herbſtarbeitswochen zunächſt opfere. Wir treten ſie am 16\textsuperscript{ten}}{\lemma{\textnormal{\emph{der … 16\textsuperscript{ten}}}}\Cendnote{\textnormal{quer am linken Rand der letzten
                  Seite}}}\label{T_L02027-1}{ }\label{T_L02027-2v}\edtext{von München\oindex{Muenchen@\textbf{München}, \emph{P.PPLA}|pw} aus an}{\lemma{\textnormal{\emph{von München aus an}}}\Cendnote{\textnormal{weiter quer am
                  rechten Rand der letzten Seite}}}\label{T_L02027-2}.\pend
           \pstart Von Herzen Ihr\spacefill\mbox{Hugo.}\pend{}\selectlanguage{ngerman}\endnumbering\briefempfaengerindex{Schnitzler, Arthur@\textsc{Schnitzler, Arthur}!zzzHofmannsthal, Hugo von@\emph{von Hugo von Hofmannsthal}!1911-09-111@{11. 9. 1911}|)be}\mylabel{L02027h}  \normalsize

\doendnotes{C}
\bigskip
\vfill

\clearpage

\footnotesize

\lohead{\textsc{register}}

% Definiere theindex-Environment komplett neu ohne reledmac
\makeatletter
\renewenvironment{theindex}{%
  \section*{\indexname}%
  \setlength{\parindent}{0pt}%
  \setlength{\parskip}{0pt plus 0.3pt}%
  \let\item\@idxitem
}{%
  \clearpage
}
\makeatother

\IfFileExists{\jobname-pw.ind}{\input{\jobname-pw.ind}}{}

\end{document}

      