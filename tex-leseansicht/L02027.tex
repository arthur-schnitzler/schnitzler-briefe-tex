%% latex-leseansicht-vorspann.tex
%% Vorspann für die Leseansicht.
%% Lädt die gemeinsame Datei latex-vorspann.tex mit nicht gesetztem Schalter.

\newif\ifkorrekturansicht
\korrekturansichtfalse

\input{../tex-inputs/latex-vorspann}


               \section[Hugo von Hofmannsthal an Arthur Schnitzler, 11. 9. 1911]{ Hugo von Hofmannsthal an Arthur Schnitzler, 11. 9. 1911}\nopagebreak\mylabel{v}\rehead{ }\begin{ledgroupsized}[t]{13cm}\normalsize\beginnumbering\briefempfaengerindex{Schnitzler, Arthur@\textsc{Schnitzler, Arthur}!zzzHofmannsthal, Hugo von@\emph{von Hugo von Hofmannsthal}!1911-09-111@{11. 9. 1911}|(be} \toendnotes[C]{\smallbreak\pagebreak[2]} \Standort{CUL, Schnitzler, B 43.}
\physDesc{Brief, 1 Blatt, 4 Seiten
\newline{}Handschrift: schwarze Tinte, deutsche Kurrent
\newline{}Schnitzler: mit Bleistift die Jahreszahl ergänzt: »911« und beschriftet: »\textsc{Hugo}« \newline{}Ordnung: 1) mit Bleistift von unbekannter Hand nummeriert: »\strikeout{323}« 2) mit Bleistift von unbekannter Hand nummeriert: »332«}\buchAbdrucke{\weitereDrucke{Hugo von Hofmannsthal, Arthur Schnitzler: \emph{Briefwechsel}. Hg. Therese Nickl und Heinrich Schnitzler. Frankfurt am Main: \emph{S. Fischer} 1964, S. 262.} }\toendnotes[C]{\smallbreak}\pstart
           \raggedleft{}{\pb}\textsc{Aussee}\oindex{Bad Aussee@\textbf{Bad Aussee}|pw}, 11. IX.\pend
           \pstart{}mein lieber Arthur \pend\pstart
           die traurige Nachricht fand ich, nach einigen trüben Andeutungen durch Freunde, heute
               morgens in der Zeitung – ſo war es unmöglich, zurechtzukommen, um dem \label{K_L02027_1v}\edtext{Begräbnis}{\lemma{\textnormal{\emph{Begräbnis}}}\Cendnote{\textnormal{Dieses fand an eben
                  diesem Tag, dem 11. 9. 1911 statt.}}}\label{K_L02027_1h} Ihrer guten Mutter\pwindex{Schnitzler, Louise 08.07.1840 – 09.09.1911@\textsc{Schnitzler, Louise} (08.07.1840 – 09.09.1911)|pwv} beizuwohnen.\hspace*{1.5em}Daſs jemand nicht mehr iſt, iſt auch für den Fernerſtehenden
               unfaſsbar, ja es iſt, als antwortete das menſchliche Innere {\pb}auf die Zumutung, dies
               hinzunehmen, mit einer verdoppelten Lebhaftigkeit der Vorſtellung. So lebt Ihre Mutter\pwindex{Schnitzler, Louise 08.07.1840 – 09.09.1911@\textsc{Schnitzler, Louise} (08.07.1840 – 09.09.1911)|pwv} für mich in dieſen
               Stunden – und immer wieder, nach 10 nach 15, nach 20 Jahren kommt für mich ein
               einſamer Spaziergang, eine ſtockende Arbeitsſtunde, in der ein Todter ſo völlig
               auflebt, dies iſt eines der Geheimniſſe unseres Innern.\pend
           \pstart
           Es iſt mir ein lieber Gedanke, daſs Sie nach der Qual dieſer Tage daran {\pb}gehen, ein dichteriſches Gebilde\pwindex{Schnitzler, Arthur 15.05.1862 – 21.10.1931@\textsc{Schnitzler, Arthur} (15.05.1862 – 21.10.1931), \emph{Schriftsteller, Mediziner}!weite Land. Tragikomoedie in fuenf Akten1910-10-20@\strich\emph{Das weite Land. Tragikomödie in fünf Akten} {[}1910-10-20{]}|pwv}, in dem ſo viel Ihres ſtärkſten
               wahrſten inneren Lebens zuſammengedrängt iſt, auf die Bühne {[}zu{]}
               bringen. Daſs man auf dieſe Weiſe, ebenſo wie in den Kindern\pwindex{Hofmannsthal, Christiane von 14.05.1902 – 05.01.1987@\textsc{Hofmannsthal, Christiane von} (14.05.1902 – 05.01.1987)|pwv}\pwindex{Hofmannsthal, Raimund von 26.5.1906 – 20.03.1974@\textsc{Hofmannsthal, Raimund von} (26.5.1906 – 20.03.1974)|pwv}\pwindex{Hofmannsthal, Franz von 20.10.1903 – 13.07.1929@\textsc{Hofmannsthal, Franz von} (20.10.1903 – 13.07.1929)|pwv}, irgend etwas von ſich
               weitergibt, gleichſam ans Unendliche weitergibt, iſt für mich eine von den
               Compenſationen. Es gibt noch geheimnisvollere, wenn man in das Myſterium des Lebens
               eindringt, wie es manchmal geſtattet, aber {\pb}nicht mitteilbar iſt. In den
               Tiefen der Arbeit liegen ſie und auch in den Tiefen des \substVorne{}\textsuperscript{A}\substDazwischen{}a\substHinten{}ufnehmenden Lebens, und ſind Ihnen bekannt wie mir. – Es ſcheint mir in
               manchen Momenten als das einzig Natürliche, jetzt zu Ihnen zu fahren und Tage bei
               Ihnen zu ſein. Ich thäte es augenblicklich, wären Sie auf dem Lande, wo ich wirklich
               andauernd bei Ihnen wäre.\pend
           \pstart
           Auch hält mich noch etwas zurück. Mein Vater\pwindex{Hofmannsthal, Hugo August von 21.12.1841 – 08.12.1915@\textsc{Hofmannsthal, Hugo August von} (21.12.1841 – 08.12.1915), \emph{Bankdirektor}|pwv} war dieſen ganzen ſchweren So{\geminationm}er in Wien\oindex{Wien@\textbf{Wien}|pw},
               iſt jetzt bei uns und freut ſich auf eine kleine aufheiternde Reiſe nach Hamburg\oindex{Hamburg@\textbf{Hamburg}|pw} u. Kopenhagen\oindex{Kopenhagen@\textbf{Kopenhagen}|pw}, \label{T_L02027_1v}\edtext{der ich auch meine Herbſtarbeitswochen
               zunächſt opfere. Wir treten ſie am 16\textsuperscript{ten}}{\lemma{\textnormal{\emph{der … 16ten}}}\Cendnote{\textnormal{quer am linken
                  Rand der letzten Seite}}}\label{T_L02027_1h}{ }\label{T_L02027_2v}\edtext{von München\oindex{Muenchen@\textbf{München}|pw} aus an}{\lemma{\textnormal{\emph{von München aus an}}}\Cendnote{\textnormal{weiter quer am rechten Rand der
                  letzten Seite}}}\label{T_L02027_2h}.\pend
           \pstart Von Herzen Ihr\spacefill\mbox{Hugo.}\pend{}\endnumbering\briefempfaengerindex{Schnitzler, Arthur@\textsc{Schnitzler, Arthur}!zzzHofmannsthal, Hugo von@\emph{von Hugo von Hofmannsthal}!1911-09-111@{11. 9. 1911}|)be}\mylabel{h}\end{ledgroupsized}  \newcommand{\dateiname}{L02027}\newcommand{\titel}{Hugo von Hofmannsthal an Arthur Schnitzler, 11. 9. 1911}\newcommand{\editorInnen}{Martin Anton Müller und Gerd-Hermann Susen}%% latex-leseansicht-abspann.tex
%% Abspann für die Leseansicht.
%% Der Schalter \ifkorrekturansicht ist bereits durch den Vorspann gesetzt.

%% latex-abspann.tex
%% Gemeinsamer Abspann für Korrekturansicht und Leseansicht.
%% Setzt den Schalter \ifkorrekturansicht voraus (gesetzt in den
%% einbindenden Dateien latex-korrekturansicht-abspann.tex bzw.
%% latex-leseansicht-abspann.tex).
%% ---------------------------------------------------------------

\normalsize

% Das esempio-Environment wird nur in der Leseansicht benötigt
\ifkorrekturansicht\else
\newenvironment{esempio}[3]%
{
    \vspace{1.5ex}
    \rlap{\underline{#1}}
    \par
    \setlength{\parindent}{0cm}
    \nopagebreak
    \leftskip=#2cm
    \rightskip=#3cm
}
{
    \par
}
\fi

\doendnotes{C}
\bigskip
\vfill

\clearpage

\footnotesize

\ifkorrekturansicht
  \lohead{\textsc{register}}
\fi

% theindex-Environment neu definieren ohne reledmac
\makeatletter
\renewenvironment{theindex}{%
  \ifkorrekturansicht
    \section*{\indexname}%
  \else
    \subsubsection*{Index der erwähnten Entitäten}%
  \fi
  \setlength{\parindent}{0pt}%
  \setlength{\parskip}{0pt plus 0.3pt}%
  \let\item\@idxitem
}{%
  \ifkorrekturansicht\clearpage\fi
}
\makeatother

\IfFileExists{\jobname-pw.ind}{\input{\jobname-pw.ind}}{}

% Quellenangabe nur in der Leseansicht
\ifkorrekturansicht\else
% Fallback-Definitionen, falls die .tex-Datei \titel etc. nicht gesetzt hat
\providecommand{\titel}{}
\providecommand{\editorInnen}{}
\providecommand{\dateiname}{\jobname}

\vspace{3cm}

\vfill

\footnotesize
\textsc{Quelle}: \titel. Herausgegeben von {\editorInnen}. In: \emph{Arthur Schnitzler: Briefwechsel mit Autorinnen und Autoren}.
 Digitale Edition, https://schnitzler-briefe.acdh.oeaw.ac.at/{\dateiname}.html (Stand \today)
\fi

\end{document}


      