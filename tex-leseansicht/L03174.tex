%% latex-korrekturansicht-vorspann.tex
%% Vorspann für die Korrekturansicht.
%% Lädt die gemeinsame Datei latex-vorspann.tex mit gesetztem Schalter.

\newif\ifkorrekturansicht
\korrekturansichttrue

\input{../tex-inputs/latex-vorspann}


\section[ Felix Salten an Arthur Schnitzler, 14. 7. {[}1896{]}]{L03174 Felix Salten an Arthur Schnitzler, 14. 7. {[}1896{]}}
\nopagebreak\mylabel{L03174v}
\rehead{ }\normalsize\beginnumbering\briefempfaengerindex{Schnitzler, Arthur@\textsc{Schnitzler, Arthur}!zzzSalten, Felix@\emph{von Felix Salten}!1896-07-142@{14. 7. {[}1896{]}}|(be}
\toendnotes[C]{\smallbreak\pagebreak[2]}\Standort{CUL, Schnitzler, B 89, A 1.}
\physDesc{Brief, 1 Blatt, 2 Seiten, 977 Zeichen
\newline{}Handschrift: schwarze Tinte, lateinische Kurrent
\newline{}Schnitzler: mit Bleistift die Jahreszahl »96« ergänzt 
\newline{}Ordnung: mit Bleistift von unbekannter Hand nummeriert: »73« }\toendnotes[C]{\smallbreak}
\pstart
           \raggedleft{}{\pb}14. Juli\pend
           \vspace{0.5em}
\pstart
           lieber Arthur, ich habe eigentlich garnichts zu sagen. Ich bin alle
               Tage von ½ 2 Uhr an zu
                  Hause\oindex{Hoerlgasse 16@\textbf{Hörlgasse 16}, \emph{Wohngebäude (K.WHS)}|pwv}, lese und arbeite und lege mich um ½ 11 schlafen. Durch
               das schöne \label{K_L03174-1v}\edtext{Buch\pwindex{Ueber  Goethes Hermann und Dorothea@\emph{Über Goethes Hermann und Dorothea}|pwv} von Victor Hehn\pwindex{Hehn, Victor 1813-10-08 – 1890-03-21@\textsc{Hehn, Victor} (1813-10-08 – 1890-03-21), \emph{Schriftsteller/Schriftstellerin, Historiker/Historikerin, Bibliothekar/Bibliothekarin}|pw}}{\lemma{\textnormal{\emph{Buch von Victor Hehn}}}\Cendnote{\textnormal{Viktor Hehn\pwindex{Hehn, Victor 1813-10-08 – 1890-03-21@\textsc{Hehn, Victor} (1813-10-08 – 1890-03-21), \emph{Schriftsteller/Schriftstellerin, Historiker/Historikerin, Bibliothekar/Bibliothekarin}|pwk}: \emph{Über Goethes Hermann und Dorothea}\pwindex{Ueber  Goethes Hermann und Dorothea@\emph{Über Goethes Hermann und Dorothea}|pwk}.
                     Aus dessen Nachlaß herausgegeben von Albert
                        Leitzmann\pwindex{Leitzmann, Albert 03.08.1867 – 16.04.1950@\textsc{Leitzmann, Albert} (03.08.1867 – 16.04.1950)|pwk} und Theodor
                     Schiemann\pwindex{Schiemann, Theodor 1847-07-17 – 1921-01-26@\textsc{Schiemann, Theodor} (1847-07-17 – 1921-01-26), \emph{Historiker/Historikerin}|pwk}. Stuttgart: \emph{Verlag der J. G. Cotta'schen Buchhandlung Nachfolger}\orgindex{J.G. Cotta sche Buchhandlung Nachfolger@J.G. Cotta’sche Buchhandlung Nachfolger|pwk}{ }1893.}}}\label{K_L03174-1} wurde ich darauf
               gebracht, die »Wahlverwandtschaften\pwindex{Wahlverwandtschaften@\emph{Die Wahlverwandtschaften}|pw}« zu lesen,
               die ich nicht kannte. (Ich weiss schon, aber ich hab sie vor acht Jahren nicht lesen
               können) Das war jetzt sehr viel für mich und hat mir beim Arbeiten merkwürdig
               geholfen. Wenn ich nicht so ganz allein wäre, ohne einen einzigen Menschen, mit dem
               ich sprechen könnte, würde es mir recht gut gehen. Jedenfalls erhalten Sie, bis Sie
               wieder da sind Einiges zu hören, und da ich im August mit
               Frl. M.\pwindex{Salten, Ottilie 07.03.1868 – 22.06.1942@\textsc{Salten, Ottilie} (07.03.1868 – 22.06.1942), \emph{Schauspieler/Schauspielerin}|pw} manches Entscheidende zu erleben
               hoffe, wird auch genug zu erzählen sein. Hören Sie was von Beer-Hofmann\pwindex{Beer-Hofmann, Richard 1866-07-11 – 1945-09-26@\textsc{Beer-Hofmann, Richard} (1866-07-11 – 1945-09-26), \emph{Schriftsteller/Schriftstellerin}|pw}? ich möchte gerne wissen, wie es ihm geht.
               Schreiben {\pb}Sie mir bald, mir
               sind diese Postkarten sehr angenehm; und wenn Sie nach \label{K_L03174-2v}\edtext{Kopenhagen\oindex{Kopenhagen@\textbf{Kopenhagen}, \emph{P.PPLC}|pw}}{\lemma{\textnormal{\emph{Kopenhagen}}}\Cendnote{\textnormal{Schnitzler hatte bereits seine Skandinavien\oindex{Skandinavien@\textbf{Skandinavien}, \emph{Region}|pwk}reise mit einer Schiffsreise an das Nordkap\oindex{Nordkap@\textbf{Nordkap}, \emph{Kap (N.KAP)}|pwk} begonnen. Danach sollte er mit
                     Beer-Hofmann\pwindex{Beer-Hofmann, Richard 1866-07-11 – 1945-09-26@\textsc{Beer-Hofmann, Richard} (1866-07-11 – 1945-09-26), \emph{Schriftsteller/Schriftstellerin}|pwk} und Goldmann\pwindex{Goldmann, Paul 31.01.1865 – 25.09.1935@\textsc{Goldmann, Paul} (31.01.1865 – 25.09.1935), \emph{Schriftsteller/Schriftstellerin, Journalist/Journalistin}|pwk} in Kopenhagen\oindex{Kopenhagen@\textbf{Kopenhagen}, \emph{P.PPLC}|pwk} für einen gemeinsamen Badeurlaub zusammentreffen. }}}\label{K_L03174-2}
               kommen und dort still sitzen, schwingen Sie sich wol zu einem Brief auf.\pend
           
\pstart
           Viele herzliche Grüße {\\[\baselineskip]}\spacefill\mbox{Salten}\pend
           \leftskip=0em{}\selectlanguage{ngerman}\endnumbering\briefempfaengerindex{Schnitzler, Arthur@\textsc{Schnitzler, Arthur}!zzzSalten, Felix@\emph{von Felix Salten}!1896-07-142@{14. 7. {[}1896{]}}|)be}\mylabel{L03174h}  \normalsize

\doendnotes{C}
\bigskip
\vfill

\clearpage

\footnotesize

\lohead{\textsc{register}}

% Definiere theindex-Environment komplett neu ohne reledmac
\makeatletter
\renewenvironment{theindex}{%
  \section*{\indexname}%
  \setlength{\parindent}{0pt}%
  \setlength{\parskip}{0pt plus 0.3pt}%
  \let\item\@idxitem
}{%
  \clearpage
}
\makeatother

\IfFileExists{\jobname-pw.ind}{\input{\jobname-pw.ind}}{}

\end{document}

      