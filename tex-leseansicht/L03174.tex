%% latex-leseansicht-vorspann.tex
%% Vorspann für die Leseansicht.
%% Lädt die gemeinsame Datei latex-vorspann.tex mit nicht gesetztem Schalter.

\newif\ifkorrekturansicht
\korrekturansichtfalse

\input{../tex-inputs/latex-vorspann}


\section[ Felix Salten an Arthur Schnitzler, 14. 7. [1896]]{L03174 Felix Salten an Arthur Schnitzler,  14. 7. [1896]}
\nopagebreak\mylabel{L03174v}
\rehead{ }\normalsize\beginnumbering\briefempfaengerindex{Schnitzler, Arthur@\textsc{Schnitzler, Arthur}!zzzSalten, Felix@\emph{von Felix Salten}!1896-07-142@{14. 7. [1896]}|(be}
\toendnotes[C]{\smallbreak\pagebreak[2]}
\correspDesc{Versand  durch Felix Salten am 14. 7. [1896] in Wien
\newline{}Erhalt  durch Arthur Schnitzler am [23. 7. 1896?] in Trondheim}\toendnotes[C]{\smallbreak}
\Standort{CUL, Schnitzler, B 89, A 1.}
\physDesc{Brief, 1 Blatt, 2 Seiten, 977 Zeichen
\newline{}Handschrift: schwarze Tinte, lateinische Kurrent
\newline{}Schnitzler: mit Bleistift die Jahreszahl »96« ergänzt 
\newline{}Ordnung: mit Bleistift von unbekannter Hand nummeriert: »73« }\toendnotes[C]{\smallbreak}
\pstart
           \raggedleft{}{\pb}14. Juli\pend
           \vspace{0.5em}
\pstart
           lieber Arthur, ich habe eigentlich garnichts zu sagen. Ich bin alle
               Tage von ½ 2 Uhr an zu
                  Hause\oindex{Wien@\textbf{Wien}!IX., Alsergrund@\textbf{IX., Alsergrund}!Hörlgasse 16@\textbf{Hörlgasse 16}, \emph{Wohngebäude}|pwv}, lese und arbeite und lege mich um ½ 11 schlafen. Durch
               das schöne \label{K_L03174-1v}\edtext{Buch\pwindex{Hehn, Victor 8.\,10.\,1813 Tartu – 21.\,3.\,1890 Berlin@\textsc{Hehn, Victor} (8.\,10.\,1813 Tartu – 21.\,3.\,1890 Berlin), \emph{Schriftsteller, Historiker, Bibliothekar}!Über  Goethes Hermann und Dorothea@\strich\emph{Über Goethes Hermann und Dorothea}|pwv} von Victor Hehn\pwindex{Hehn, Victor 8.\,10.\,1813 Tartu – 21.\,3.\,1890 Berlin@\textsc{Hehn, Victor} (8.\,10.\,1813 Tartu – 21.\,3.\,1890 Berlin), \emph{Schriftsteller, Historiker, Bibliothekar}|pw}}{\lemma{\textnormal{\emph{Buch von Victor Hehn}}}\Cendnote{\textnormal{Viktor Hehn\pwindex{Hehn, Victor 8.\,10.\,1813 Tartu – 21.\,3.\,1890 Berlin@\textsc{Hehn, Victor} (8.\,10.\,1813 Tartu – 21.\,3.\,1890 Berlin), \emph{Schriftsteller, Historiker, Bibliothekar}|pwk}: \emph{Über Goethes Hermann und Dorothea}\pwindex{Hehn, Victor 8.\,10.\,1813 Tartu – 21.\,3.\,1890 Berlin@\textsc{Hehn, Victor} (8.\,10.\,1813 Tartu – 21.\,3.\,1890 Berlin), \emph{Schriftsteller, Historiker, Bibliothekar}!Über  Goethes Hermann und Dorothea@\strich\emph{Über Goethes Hermann und Dorothea}|pwk}.
                     Aus dessen Nachlaß herausgegeben von Albert
                        Leitzmann\pwindex{Leitzmann, Albert 3.\,8.\,1867 Magdeburg – 16.\,4.\,1950 Jena@\textsc{Leitzmann, Albert} (3.\,8.\,1867 Magdeburg – 16.\,4.\,1950 Jena)|pwk} und Theodor
                     Schiemann\pwindex{Schiemann, Theodor 17.\,7.\,1847 Grobiņa – 26.\,1.\,1921 Berlin@\textsc{Schiemann, Theodor} (17.\,7.\,1847 Grobiņa – 26.\,1.\,1921 Berlin), \emph{Historiker}|pwk}. Stuttgart: \emph{Verlag der J. G. Cotta’schen Buchhandlung Nachfolger}\orgindex{J.G. Cotta’sche Buchhandlung Nachfolger@J.G. Cotta’sche Buchhandlung Nachfolger|pwk}{ }1893.}}}\label{K_L03174-1} wurde ich darauf
               gebracht, die »Wahlverwandtschaften\pwindex{\textcolor{red}{\textsuperscript{XXXX indx1}}!Wahlverwandtschaften@\strich\emph{Die Wahlverwandtschaften}|pw}« zu lesen,
               die ich nicht kannte. (Ich weiss schon, aber ich hab sie vor acht Jahren nicht lesen
               können) Das war jetzt sehr viel für mich und hat mir beim Arbeiten merkwürdig
               geholfen. Wenn ich nicht so ganz allein wäre, ohne einen einzigen Menschen, mit dem
               ich sprechen könnte, würde es mir recht gut gehen. Jedenfalls erhalten Sie, bis Sie
               wieder da sind Einiges zu hören, und da ich im August mit
               Frl. M.\pwindex{Salten, Ottilie 7.\,3.\,1868 Prag – 22.\,6.\,1942 Zürich@\textsc{Salten, Ottilie} (7.\,3.\,1868 Prag – 22.\,6.\,1942 Zürich), \emph{Schauspielerin}|pw} manches Entscheidende zu erleben
               hoffe, wird auch genug zu erzählen sein. Hören Sie was von Beer-Hofmann\pwindex{Beer-Hofmann, Richard 11.\,7.\,1866 Wien – 26.\,9.\,1945 New York City@\textsc{Beer-Hofmann, Richard} (11.\,7.\,1866 Wien – 26.\,9.\,1945 New York City), \emph{Schriftsteller}|pw}? ich möchte gerne wissen, wie es ihm geht.
               Schreiben {\pb}Sie mir bald, mir
               sind diese Postkarten sehr angenehm; und wenn Sie nach \label{K_L03174-2v}\edtext{Kopenhagen\oindex{Kopenhagen@\textbf{Kopenhagen}, \emph{Hauptstadt}|pw}}{\lemma{\textnormal{\emph{Kopenhagen}}}\Cendnote{\textnormal{Schnitzler hatte bereits seine Skandinavien\oindex{Skandinavien@\textbf{Skandinavien}|pwk}reise mit einer Schiffsreise an das Nordkap\oindex{Nordkap@\textbf{Nordkap}, \emph{Kap}|pwk} begonnen. Danach sollte er mit
                     Beer-Hofmann\pwindex{Beer-Hofmann, Richard 11.\,7.\,1866 Wien – 26.\,9.\,1945 New York City@\textsc{Beer-Hofmann, Richard} (11.\,7.\,1866 Wien – 26.\,9.\,1945 New York City), \emph{Schriftsteller}|pwk} und Goldmann\pwindex{Goldmann, Paul 31.\,1.\,1865 Breslau – 25.\,9.\,1935 Wien@\textsc{Goldmann, Paul} (31.\,1.\,1865 Breslau – 25.\,9.\,1935 Wien), \emph{Schriftsteller, Journalist}|pwk} in Kopenhagen\oindex{Kopenhagen@\textbf{Kopenhagen}, \emph{Hauptstadt}|pwk} für einen gemeinsamen Badeurlaub zusammentreffen. }}}\label{K_L03174-2}
               kommen und dort still sitzen, schwingen Sie sich wol zu einem Brief auf.\pend
           
\pstart
           Viele herzliche Grüße {\\[\baselineskip]}\spacefill\mbox{Salten}\pend
           \leftskip=0em{}\selectlanguage{ngerman}\endnumbering\briefempfaengerindex{Schnitzler, Arthur@\textsc{Schnitzler, Arthur}!zzzSalten, Felix@\emph{von Felix Salten}!1896-07-142@{14. 7. [1896]}|)be}\mylabel{L03174h}  \newcommand{\dateiname}{L03174}\newcommand{\titel}{Felix Salten an Arthur Schnitzler, 14. 7. [1896]}\newcommand{\editorInnen}{Martin Anton Müller und Laura Untner}%% latex-leseansicht-abspann.tex
%% Abspann für die Leseansicht.
%% Der Schalter \ifkorrekturansicht ist bereits durch den Vorspann gesetzt.

%% latex-abspann.tex
%% Gemeinsamer Abspann für Korrekturansicht und Leseansicht.
%% Setzt den Schalter \ifkorrekturansicht voraus (gesetzt in den
%% einbindenden Dateien latex-korrekturansicht-abspann.tex bzw.
%% latex-leseansicht-abspann.tex).
%% ---------------------------------------------------------------

\normalsize

% Das esempio-Environment wird nur in der Leseansicht benötigt
\ifkorrekturansicht\else
\newenvironment{esempio}[3]%
{
    \vspace{1.5ex}
    \rlap{\underline{#1}}
    \par
    \setlength{\parindent}{0cm}
    \nopagebreak
    \leftskip=#2cm
    \rightskip=#3cm
}
{
    \par
}
\fi

\doendnotes{C}
\bigskip
\vfill

\clearpage

\footnotesize

\ifkorrekturansicht
  \lohead{\textsc{register}}
\fi

% theindex-Environment neu definieren ohne reledmac
\makeatletter
\renewenvironment{theindex}{%
  \ifkorrekturansicht
    \section*{\indexname}%
  \else
    \subsubsection*{Index der erwähnten Entitäten}%
  \fi
  \setlength{\parindent}{0pt}%
  \setlength{\parskip}{0pt plus 0.3pt}%
  \let\item\@idxitem
}{%
  \ifkorrekturansicht\clearpage\fi
}
\makeatother

\IfFileExists{\jobname-pw.ind}{\input{\jobname-pw.ind}}{}

% Quellenangabe nur in der Leseansicht
\ifkorrekturansicht\else
% Fallback-Definitionen, falls die .tex-Datei \titel etc. nicht gesetzt hat
\providecommand{\titel}{}
\providecommand{\editorInnen}{}
\providecommand{\dateiname}{\jobname}

\vspace{3cm}

\vfill

\footnotesize
\textsc{Quelle}: \titel. Herausgegeben von {\editorInnen}. In: \emph{Arthur Schnitzler: Briefwechsel mit Autorinnen und Autoren}.
 Digitale Edition, https://schnitzler-briefe.acdh.oeaw.ac.at/{\dateiname}.html (Stand \today)
\fi

\end{document}


