%% latex-korrekturansicht-vorspann.tex
%% Vorspann für die Korrekturansicht.
%% Lädt die gemeinsame Datei latex-vorspann.tex mit gesetztem Schalter.

\newif\ifkorrekturansicht
\korrekturansichttrue

\input{../tex-inputs/latex-vorspann}


\section[Arthur Schnitzler an Hugo von Hofmannsthal, 14. 7. 1899]{L00941 Arthur Schnitzler an Hugo von Hofmannsthal, 14. 7. 1899}
\nopagebreak\mylabel{L00941v}
\rehead{ }\normalsize\beginnumbering\briefempfaengerindex{Hofmannsthal, Hugo von@\textsc{Hofmannsthal, Hugo von}!zzzSchnitzler, Arthur@\emph{von Arthur Schnitzler}!1899-07-141@{14. 7. 1899}|(be}
\toendnotes[C]{\smallbreak\pagebreak[2]}\Standort{FDH, Hs-30885,83.}
\physDesc{Briefkarte, 674 Zeichen
\newline{}Handschrift: Bleistift, deutsche Kurrent}
\buchAbdrucke{\weitereDrucke{Hugo von Hofmannsthal, Arthur Schnitzler: \emph{Briefwechsel}. Frankfurt am Main: \emph{S. Fischer} 1964, S. 125.} }\toendnotes[C]{\smallbreak}
\pstart
           \raggedleft{}{\pb}14/7 99\pend
           \vspace{0.5em}
\pstart
           mein lieber Hugo.{ }Montag reiſe ich wahrſcheinlich ab. Adresse: \textsc{Velden\oindex{Velden am Woerthersee@\textbf{Velden am Wörthersee}, \emph{P.PPL}|pw}, Pension Pundschu\oindex{Pension Pundschu@\textbf{Pension Pundschu}, \emph{Hotel (K.HTL)}|pw}}. Bin dort mit Mama\pwindex{Schnitzler, Louise 1840-07-08 – 1911-09-09@\textsc{Schnitzler, Louise} (1840-07-08 – 1911-09-09)|pwv} u
                  Schweſter\pwindex{Hajek, Gisela 20.12.1867 – 03.02.1953@\textsc{Hajek, Gisela} (20.12.1867 – 03.02.1953)|pwv}. Waſſermann\pwindex{Wassermann, Jakob 10.03.1873 – 01.01.1934@\textsc{Wassermann, Jakob} (10.03.1873 – 01.01.1934), \emph{Schriftsteller/Schriftstellerin}|pw} geht vielleicht mit. Von Richard\pwindex{Beer-Hofmann, Richard 1866-07-11 – 1945-09-26@\textsc{Beer-Hofmann, Richard} (1866-07-11 – 1945-09-26), \emph{Schriftsteller/Schriftstellerin}|pw} hör ich wenig; eben eine Karte; ich hab
               nicht den Eindruck, dſs er in guter Sti{\geminationm}ung ist. – Wie
               lang ich in V.\oindex{Velden am Woerthersee@\textbf{Velden am Wörthersee}, \emph{P.PPL}|pw} bleibe? – 8–14 Tage. Möchte gern
               dann höher. Es bleibt hoffentlich bei Mitte Auguſt für {\pb}uns 2; bitte ſchieben Sie’s nicht viel weiter hinaus, we{\geminationn} es geht. – Was für eine Art 5actiges Stück\pwindex{Bergwerk zu Falun@\emph{Das Bergwerk zu Falun}|pwv} iſt das, was Sie ſchreiben? \pend
           
\pstart
           – Über alles, was ich innerlich durchmache, iſt ſchwer zu ſchreiben. Es iſt wie wenn
               die Wolken i{\geminationm}er tiefer und ſchwerer herabſänken, je
               aufrechter man geht.\pend
           \pstart Herzlich der Ihre \spacefill\mbox{Arth}\pend{}
\pstart
           \noindent{}Grüßen Sie Minnie\pwindex{Schaffgotsch, Hermine von 25.11.1871 – 25.11.1928@\textsc{Schaffgotsch, Hermine von} (25.11.1871 – 25.11.1928)|pw}.\pend
           \selectlanguage{ngerman}\endnumbering\briefempfaengerindex{Hofmannsthal, Hugo von@\textsc{Hofmannsthal, Hugo von}!zzzSchnitzler, Arthur@\emph{von Arthur Schnitzler}!1899-07-141@{14. 7. 1899}|)be}\mylabel{L00941h}  \normalsize

\doendnotes{C}
\bigskip
\vfill

\clearpage

\footnotesize

\lohead{\textsc{register}}

% Definiere theindex-Environment komplett neu ohne reledmac
\makeatletter
\renewenvironment{theindex}{%
  \section*{\indexname}%
  \setlength{\parindent}{0pt}%
  \setlength{\parskip}{0pt plus 0.3pt}%
  \let\item\@idxitem
}{%
  \clearpage
}
\makeatother

\IfFileExists{\jobname-pw.ind}{\input{\jobname-pw.ind}}{}

\end{document}

      