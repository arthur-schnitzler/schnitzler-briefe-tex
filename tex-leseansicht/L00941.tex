%% latex-leseansicht-vorspann.tex
%% Vorspann für die Leseansicht.
%% Lädt die gemeinsame Datei latex-vorspann.tex mit nicht gesetztem Schalter.

\newif\ifkorrekturansicht
\korrekturansichtfalse

\input{../tex-inputs/latex-vorspann}


         
         \renewcommand{\erwaehntePersonen}{Personen: Richard Beer-Hofmann, Gisela Hajek, Hugo von Hofmannsthal, Hermine von Schaffgotsch, Louise Schnitzler, Jakob Wassermann}
         \renewcommand{\erwaehnteOrte}{Orte: Marienbad, Pension Pundschu, Velden am Wörthersee, Wien}
         \renewcommand{\erwaehnteWerke}{Werke: Das Bergwerk zu Falun}
               \section[Arthur Schnitzler an Hugo von Hofmannsthal, 14. 7. 1899]{ Arthur Schnitzler an Hugo von Hofmannsthal, 14. 7. 1899}\nopagebreak\mylabel{v}\rehead{ }\begin{ledgroupsized}[t]{13cm}\normalsize\beginnumbering \toendnotes[C]{\smallbreak\pagebreak[2]} \Standort{FDH, Hs-30885,83.}
\physDesc{Briefkarte, 673 Zeichen
\newline{}Handschrift: Bleistift, deutsche Kurrent}\buchAbdrucke{\weitereDrucke{Hugo von Hofmannsthal, Arthur Schnitzler: \emph{Briefwechsel}. Hg. Therese Nickl und Heinrich Schnitzler. Frankfurt am Main: \emph{S. Fischer} 1964, S. 125.} }\toendnotes[C]{\smallbreak}\pstart
           \raggedleft{}{\pb}14/7 99\pend
           \pstart
           mein lieber Hugo. Montag reiſe ich wahrſcheinlich ab. Adresse: \textsc{Velden\oindex{Velden am Woerthersee@\textbf{Velden am Wörthersee}|pw}, Pension Pundschu\oindex{Pension Pundschu@\textbf{Pension Pundschu}|pw}}. Bin dort mit Mama\pwindex{Schnitzler, Louise 1840-07-08 – 1911-09-09@\textsc{Schnitzler, Louise} (1840-07-08 – 1911-09-09)|pwv} u
                  Schweſter\pwindex{Hajek, Gisela 20.12.1867 – 03.02.1953@\textsc{Hajek, Gisela} (20.12.1867 – 03.02.1953)|pwv}. Waſſermann\pwindex{Wassermann, Jakob 10.03.1873 – 01.01.1934@\textsc{Wassermann, Jakob} (10.03.1873 – 01.01.1934), \emph{Schriftsteller}|pw} geht vielleicht mit. Von Richard\pwindex{Beer-Hofmann, Richard 1866-07-11 – 1945-09-26@\textsc{Beer-Hofmann, Richard} (1866-07-11 – 1945-09-26), \emph{Schriftsteller}|pw} hör ich wenig; eben eine Karte; ich hab
               nicht den Eindruck, dſs er in guter Sti{\geminationm}ung ist. – Wie
               lang ich in V.\oindex{Velden am Woerthersee@\textbf{Velden am Wörthersee}|pw} bleibe? – 8–14 Tage. Möchte gern
               dann höher. Es bleibt hoffentlich bei Mitte Auguſt für {\pb}uns 2; bitte ſchieben Sie’s nicht viel weiter hinaus, we{\geminationn} es geht. – Was für eine Art 5actiges Stück\pwindex{Hofmannsthal, Hugo von 1874-02-01 – 1929-07-15@\textsc{Hofmannsthal, Hugo von} (1874-02-01 – 1929-07-15), \emph{Schriftsteller}!Bergwerk zu Falun1900 – 1933@\strich\emph{Das Bergwerk zu Falun} {[}1900 – 1933{]}|pwv} iſt das, was Sie ſchreiben? \pend
           \pstart
           – Über alles, was ich innerlich durchmache, iſt ſchwer zu ſchreiben. Es iſt wie wenn
               die Wolken i{\geminationm}er tiefer und ſchwerer herabſänken, je
               aufrechter man geht.\pend
           \pstart Herzlich der Ihre \spacefill\mbox{Arth}\pend{}\pstart
           \noindent{}Grüßen Sie Minnie\pwindex{Schaffgotsch, Hermine von 25.11.1871 – 25.11.1928@\textsc{Schaffgotsch, Hermine von} (25.11.1871 – 25.11.1928)|pw}.\pend
           
         
         \endnumbering\mylabel{h}\end{ledgroupsized}  \newcommand{\dateiname}{L00941}\newcommand{\titel}{Arthur Schnitzler an Hugo von Hofmannsthal, 14. 7. 1899}\newcommand{\editorInnen}{Martin Anton Müller und Gerd-Hermann Susen}%% latex-leseansicht-abspann.tex
%% Abspann für die Leseansicht.
%% Der Schalter \ifkorrekturansicht ist bereits durch den Vorspann gesetzt.

%% latex-abspann.tex
%% Gemeinsamer Abspann für Korrekturansicht und Leseansicht.
%% Setzt den Schalter \ifkorrekturansicht voraus (gesetzt in den
%% einbindenden Dateien latex-korrekturansicht-abspann.tex bzw.
%% latex-leseansicht-abspann.tex).
%% ---------------------------------------------------------------

\normalsize

% Das esempio-Environment wird nur in der Leseansicht benötigt
\ifkorrekturansicht\else
\newenvironment{esempio}[3]%
{
    \vspace{1.5ex}
    \rlap{\underline{#1}}
    \par
    \setlength{\parindent}{0cm}
    \nopagebreak
    \leftskip=#2cm
    \rightskip=#3cm
}
{
    \par
}
\fi

\doendnotes{C}
\bigskip
\vfill

\clearpage

\footnotesize

\ifkorrekturansicht
  \lohead{\textsc{register}}
\fi

% theindex-Environment neu definieren ohne reledmac
\makeatletter
\renewenvironment{theindex}{%
  \ifkorrekturansicht
    \section*{\indexname}%
  \else
    \subsubsection*{Index der erwähnten Entitäten}%
  \fi
  \setlength{\parindent}{0pt}%
  \setlength{\parskip}{0pt plus 0.3pt}%
  \let\item\@idxitem
}{%
  \ifkorrekturansicht\clearpage\fi
}
\makeatother

\IfFileExists{\jobname-pw.ind}{\input{\jobname-pw.ind}}{}

% Quellenangabe nur in der Leseansicht
\ifkorrekturansicht\else
% Fallback-Definitionen, falls die .tex-Datei \titel etc. nicht gesetzt hat
\providecommand{\titel}{}
\providecommand{\editorInnen}{}
\providecommand{\dateiname}{\jobname}

\vspace{3cm}

\vfill

\footnotesize
\textsc{Quelle}: \titel. Herausgegeben von {\editorInnen}. In: \emph{Arthur Schnitzler: Briefwechsel mit Autorinnen und Autoren}.
 Digitale Edition, https://schnitzler-briefe.acdh.oeaw.ac.at/{\dateiname}.html (Stand \today)
\fi

\end{document}


      