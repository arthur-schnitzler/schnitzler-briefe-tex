%% latex-leseansicht-vorspann.tex
%% Vorspann für die Leseansicht.
%% Lädt die gemeinsame Datei latex-vorspann.tex mit nicht gesetztem Schalter.

\newif\ifkorrekturansicht
\korrekturansichtfalse

\input{../tex-inputs/latex-vorspann}


\section[ Paul Goldmann an Arthur Schnitzler, 22. 4. 1897]{L02809 Paul Goldmann an Arthur Schnitzler,  22. 4. 1897}
\nopagebreak\mylabel{L02809v}
\rehead{ }\normalsize\beginnumbering\briefempfaengerindex{Schnitzler, Arthur@\textsc{Schnitzler, Arthur}!zzzGoldmann, Paul@\emph{von Paul Goldmann}!1897-04-222@{22. 4. 1897}|(be}
\toendnotes[C]{\smallbreak\pagebreak[2]}
\correspDesc{Versand  durch Paul Goldmann am 22. 4. 1897 in Frankfurt am Main
\newline{}Erhalt  durch Arthur Schnitzler im Zeitraum [23. 4. 1897
                  – 27. 4. 1897?] in Paris}\toendnotes[C]{\smallbreak}
\Standort{DLA, A:Schnitzler, HS.NZ85.1.3167.}
\physDesc{Brief, 1 Blatt, 3 Seiten, 1456 Zeichen
\newline{}Handschrift: schwarze Tinte, deutsche Kurrent
\newline{}Schnitzler: mit rotem Buntstift zwei Unterstreichungen }\toendnotes[C]{\smallbreak}
\pstart
           \centering{}{\pb}\textcolor{gray}{\textbf{Hôtel Deutscher Kaiser\oindex{Hotel Deutscher Kaiser@\textbf{Hotel Deutscher Kaiser}, \emph{Hotel}|pw}}}\pend
           
\pstart
           \centering{}\textcolor{gray}{\textbf{(W. Gömöri\pwindex{Gömöri, Wilhelm 10.\,9.\,1884 – 20.\,10.\,1917@\textsc{Gömöri, Wilhelm} (10.\,9.\,1884 – 20.\,10.\,1917), \emph{Hotelbesitzer}|pw})}}\pend
           
\pstart
           \centering{}\textcolor{gray}{\textbf{Frankfurt a. M.\oindex{Frankfurt am Main@\textbf{Frankfurt am Main}, \emph{Hauptstadt}|pw}}}\pend
           
\pstart
           \textcolor{gray}{\textbf{37 Wiesenhüttenplatz 37\oindex{Hotel Deutscher Kaiser@\textbf{Hotel Deutscher Kaiser}, \emph{Hotel}|pw}.}}\hfill \textcolor{gray}{\textbf{Nahe dem Centralbahnhof\oindex{Frankfurt (Main) Hauptbahnhof@\textbf{Frankfurt (Main) Hauptbahnhof}, \emph{Bahnhofsgebäude}|pw}.}}\pend
           
\pstart
           \raggedleft{}\textcolor{gray}{\textbf{Frankfurt a. M.\oindex{Frankfurt am Main@\textbf{Frankfurt am Main}, \emph{Hauptstadt}|pw}, den}}{ }22. April \textcolor{gray}{\textbf{18}}97.\pend
           
\pstart{}Mein lieber Freund,\pend\vspace{0.5em}
\pstart
           Vielen Dank für Deinen lieben Brief!\pend
           
\pstart
           Ich bin{ }ſeit Sonntag hier (nachdem ich Samſtag den Anſchluß verfehlt und in \textsc{Köln\oindex{Köln@\textbf{Köln}, \emph{Hauptstadt}|pw}} hatte übernachten müſſen). Ich bin noch ganz krank hier angekommen und kann
               mich diesmal gar nicht erholen{[}.{]} Meine Familie iſt{ }ſehr gut mit
               mir. Aber wir{ }ſitzen zuſammen und denken über die ausſichtsloſe Zukunft nach, und das
               iſt nicht heiter. Auf der Redaction\orgindex{Frankfurter Zeitung@Frankfurter Zeitung|pwv} machen{ }ſie{ }ſchiefe Geſichter, daß ich während des \label{K_L02809-1v}\edtext{Krieges}{\lemma{\textnormal{\emph{Krieges}}}\Cendnote{\textnormal{Türk\oindex{Türkei@\textbf{Türkei}|pwkv}isch-Griech\oindex{Griechenland@\textbf{Griechenland}|pwkv}ischer Krieg um Kreta\oindex{Kreta@\textbf{Kreta}, \emph{Insel}|pwk}}}}\label{K_L02809-1} nicht auf meinem Poſten bin. Ich werde alſo wohl bald zurück {\pb}müſſen. Aber jetzt im Ruhen{ }ſehe ich erſt, wie
               abgehetzt und müde gearbeitet ich bin.\pend
           
\pstart
           Alle die Meinigen grüßen Dich herzlichſt.\pend
           
\pstart
           Wenn Du Zeit haſt,{ }ſchreib’ mir noch ein Wort hierher, wie es Dir geht. (Meine
               Adreſſe iſt oben auf den Brief gedruckt).\pend
           
\pstart
           Ich vergaß Dir zu{ }ſagen, daß Du einen Abend (mit ihr\pwindex{Reinhard, Marie 13.\,3.\,1871 Wien – 18.\,3.\,1899 ebd.@\textsc{Reinhard, Marie} (13.\,3.\,1871 Wien – 18.\,3.\,1899 ebd.), \emph{Gesangspädagogin}|pwv}) in die »\label{K_L02809-2v}\edtext{\textsc{Scala\oindex{La Scala@\textbf{La Scala}, \emph{Veranstaltungsgebäude}|pw}}}{\lemma{\textnormal{\emph{Scala}}}\Cendnote{\textnormal{Konzertsaal\oindex{La Scala@\textbf{La Scala}, \emph{Veranstaltungsgebäude}|pwkv}}}}\label{K_L02809-2}« gehen{ }ſollſt.\pend
           
\pstart
           Geſtern{ }ſah ich \textsc{John Gabriel Borkmann\pwindex{Ibsen, Henrik 20.\,3.\,1828 Skien – 23.\,5.\,1906 Oslo@\textsc{Ibsen, Henrik} (20.\,3.\,1828 Skien – 23.\,5.\,1906 Oslo), \emph{Schriftsteller}!John Gabriel Borkman.  Skuespil i fire akter@\strich\emph{John Gabriel Borkman. Skuespil i fire akter}|pw}}. \strikeout{E} Das \strikeout{D\textcolor{gray}{a}}{ }Drama\pwindex{Ibsen, Henrik 20.\,3.\,1828 Skien – 23.\,5.\,1906 Oslo@\textsc{Ibsen, Henrik} (20.\,3.\,1828 Skien – 23.\,5.\,1906 Oslo), \emph{Schriftsteller}!John Gabriel Borkman.  Skuespil i fire akter@\strich\emph{John Gabriel Borkman. Skuespil i fire akter}|pwv} hat zu Zeiten einen
               großartigen tragiſchen Schwung. Ich zähle es zum Beſten, was \strikeout{\textcolor{gray}{×}\-\textcolor{gray}{×}\-\textcolor{gray}{×}}{ }\textsc{Ibsen\pwindex{Ibsen, Henrik 20.\,3.\,1828 Skien – 23.\,5.\,1906 Oslo@\textsc{Ibsen, Henrik} (20.\,3.\,1828 Skien – 23.\,5.\,1906 Oslo), \emph{Schriftsteller}|pw}} gemacht hat.\pend
           
\pstart
           Mein Onkel\pwindex{Mamroth, Fedor 21.\,2.\,1851 Breslau – 25.\,6.\,1907 Frankfurt am Main@\textsc{Mamroth, Fedor} (21.\,2.\,1851 Breslau – 25.\,6.\,1907 Frankfurt am Main), \emph{Journalist, Kritiker}|pwv} iſt voll des Lobes
               über \textsc{Bahrs\pwindex{Bahr, Hermann 19.\,7.\,1863 Linz – 15.\,1.\,1934 München@\textsc{Bahr, Hermann} (19.\,7.\,1863 Linz – 15.\,1.\,1934 München), \emph{Schriftsteller, Kritiker}|pw}}{ }Roman\pwindex{Bahr, Hermann 19.\,7.\,1863 Linz – 15.\,1.\,1934 München@\textsc{Bahr, Hermann} (19.\,7.\,1863 Linz – 15.\,1.\,1934 München), \emph{Schriftsteller, Kritiker}!Theater. Ein Wiener Roman@\strich\emph{Theater. Ein Wiener Roman}|pwv}{ }{\pb}»Theater\pwindex{Bahr, Hermann 19.\,7.\,1863 Linz – 15.\,1.\,1934 München@\textsc{Bahr, Hermann} (19.\,7.\,1863 Linz – 15.\,1.\,1934 München), \emph{Schriftsteller, Kritiker}!Theater. Ein Wiener Roman@\strich\emph{Theater. Ein Wiener Roman}|pw}«.
                  \label{K_L02809-3v}\edtext{Kennſt}{\lemma{\textnormal{\emph{Kennst}}}\Cendnote{\textnormal{Schnitzler erhielt von Bahr\pwindex{Bahr, Hermann 19.\,7.\,1863 Linz – 15.\,1.\,1934 München@\textsc{Bahr, Hermann} (19.\,7.\,1863 Linz – 15.\,1.\,1934 München), \emph{Schriftsteller, Kritiker}|pwk} ein Widmungsexemplar\pwindex{Bahr, Hermann 19.\,7.\,1863 Linz – 15.\,1.\,1934 München@\textsc{Bahr, Hermann} (19.\,7.\,1863 Linz – 15.\,1.\,1934 München), \emph{Schriftsteller, Kritiker}!Theater. Ein Wiener Roman@\strich\emph{Theater. Ein Wiener Roman}|pwkv} (vgl. XXXX Auszeichnungsfehler: Dokument L00655 nicht gefunden). Die Lektüre ist nur über die Leseliste gesichert (vgl. A. S.: \emph{Lektüren}, deutschsprachige Literatur).}}}\label{K_L02809-3} Du das Ding\pwindex{Bahr, Hermann 19.\,7.\,1863 Linz – 15.\,1.\,1934 München@\textsc{Bahr, Hermann} (19.\,7.\,1863 Linz – 15.\,1.\,1934 München), \emph{Schriftsteller, Kritiker}!Theater. Ein Wiener Roman@\strich\emph{Theater. Ein Wiener Roman}|pwv}? Es
               wäre{ }ſchrecklich, wenn \strikeout{\textcolor{gray}{man}} dem Kerl\pwindex{Bahr, Hermann 19.\,7.\,1863 Linz – 15.\,1.\,1934 München@\textsc{Bahr, Hermann} (19.\,7.\,1863 Linz – 15.\,1.\,1934 München), \emph{Schriftsteller, Kritiker}|pwv} wirklich
                  \strikeout{\textcolor{gray}{ei}} einmal etwas Gutes gelungen wäre.\pend
           
\pstart
           Es freut mich, daß Du mir wegen \label{K_L02809-4v}\edtext{Freitag{ }Abend}{\lemma{\textnormal{\emph{Freitag Abend}}}\Cendnote{\textnormal{Siehe XXXX Auszeichnungsfehler: Dokument L02808 nicht gefunden.
               }}}\label{K_L02809-4} nicht böſe biſt. »\label{K_L02809-5v}\edtext{Sie\pwindex{?? [Partnerin? von Paul Goldmann, 1897] @\textsc{?? [Partnerin? von Paul Goldmann, 1897]}|pwv}}{\lemma{\textnormal{\emph{Sie}}}\Cendnote{\textnormal{Nicht identifiziert; womöglich handelte
                  es sich um die am 12. 4. 1897 im \emph{Tagebuch}\pwindex{Schnitzler, Arthur 15.\,5.\,1862 Wien – 21.\,10.\,1931 ebd.@\textsc{Schnitzler, Arthur} (15.\,5.\,1862 Wien – 21.\,10.\,1931 ebd.), \emph{Schriftsteller, Mediziner}!Tagebuch@\strich\emph{Tagebuch}|pwk} erwähnte
                        »›Fanny\pwindex{Fanny @\textsc{Fanny}|pw}‹« oder die am
                     13. 5. 1897
                  erwähnte »Madeleine\pwindex{Madeleine @\textsc{Madeleine}|pw}«.}}}\label{K_L02809-5}« hat mich nicht zurückgehalten, ganz im Gegentheil. Auch da gibts
               allerlei \begin{otherlanguage}{french}\textsc{Malheur}\end{otherlanguage}.\pend
           
\pstart
           Kaufe Dir die{ }ſoeben erſchienene \label{K_L02809-6v}\edtext{\label{K_L02809-7v}\edtext{\textsc{Beaumarchais\pwindex{Beaumarchais, Pierre Augustin Caron de 24.\,1.\,1732 Paris – 18.\,5.\,1799 ebd.@\textsc{Beaumarchais, Pierre Augustin Caron de} (24.\,1.\,1732 Paris – 18.\,5.\,1799 ebd.), \emph{Schriftsteller}|pw}}-Biographie\pwindex{Hallays, André 16.\,3.\,1859 Paris – 30.\,3.\,1930 ebd.@\textsc{Hallays, André} (16.\,3.\,1859 Paris – 30.\,3.\,1930 ebd.), \emph{Journalist, Kunstkritiker, Jurist}!Beaumarchais@\strich\emph{Beaumarchais}|pwv}}{\lemma{\textnormal{\emph{Beaumarchais-Biographie}}}\Cendnote{\textnormal{André Hallays\pwindex{Hallays, André 16.\,3.\,1859 Paris – 30.\,3.\,1930 ebd.@\textsc{Hallays, André} (16.\,3.\,1859 Paris – 30.\,3.\,1930 ebd.), \emph{Journalist, Kunstkritiker, Jurist}|pwk}: \emph{Beaumarchais}\pwindex{Hallays, André 16.\,3.\,1859 Paris – 30.\,3.\,1930 ebd.@\textsc{Hallays, André} (16.\,3.\,1859 Paris – 30.\,3.\,1930 ebd.), \emph{Journalist, Kunstkritiker, Jurist}!Beaumarchais@\strich\emph{Beaumarchais}|pwk}. Paris: \emph{Librarie Hachette}\orgindex{Librairie Hachette@Librairie Hachette|pwk}{ }1897 (\emph{Les Grands Écrivains
                        Français}).
               }}}\label{K_L02809-7} von \textsc{André Hallays\pwindex{Hallays, André 16.\,3.\,1859 Paris – 30.\,3.\,1930 ebd.@\textsc{Hallays, André} (16.\,3.\,1859 Paris – 30.\,3.\,1930 ebd.), \emph{Journalist, Kunstkritiker, Jurist}|pw}}}{\lemma{\textnormal{\emph{Beaumarchais-Biographie … Hallays}}}\Cendnote{\textnormal{Lektüre\pwindex{Hallays, André 16.\,3.\,1859 Paris – 30.\,3.\,1930 ebd.@\textsc{Hallays, André} (16.\,3.\,1859 Paris – 30.\,3.\,1930 ebd.), \emph{Journalist, Kunstkritiker, Jurist}!Beaumarchais@\strich\emph{Beaumarchais}|pwkv} belegbar, vgl. A. S.: \emph{Lektüren}, Frankreich.}}}\label{K_L02809-6}. Ein reizendes Buch\pwindex{Hallays, André 16.\,3.\,1859 Paris – 30.\,3.\,1930 ebd.@\textsc{Hallays, André} (16.\,3.\,1859 Paris – 30.\,3.\,1930 ebd.), \emph{Journalist, Kunstkritiker, Jurist}!Beaumarchais@\strich\emph{Beaumarchais}|pwv}.\pend
           
\pstart
           Grüße mir Deine Freundin\pwindex{Reinhard, Marie 13.\,3.\,1871 Wien – 18.\,3.\,1899 ebd.@\textsc{Reinhard, Marie} (13.\,3.\,1871 Wien – 18.\,3.\,1899 ebd.), \emph{Gesangspädagogin}|pwv} und{ }ſei{ }ſelbſt von Herzen gegrüßt\pend
           
\pstart
           Dein treuer {\\[\baselineskip]}\spacefill\mbox{Paul Goldmn}\pend
           \leftskip=0em{}\selectlanguage{ngerman}\endnumbering\briefempfaengerindex{Schnitzler, Arthur@\textsc{Schnitzler, Arthur}!zzzGoldmann, Paul@\emph{von Paul Goldmann}!1897-04-222@{22. 4. 1897}|)be}\mylabel{L02809h}  \newcommand{\dateiname}{L02809}\newcommand{\titel}{Paul Goldmann an Arthur Schnitzler, 22. 4. 1897}\newcommand{\editorInnen}{Martin Anton Müller und Laura Untner}%% latex-leseansicht-abspann.tex
%% Abspann für die Leseansicht.
%% Der Schalter \ifkorrekturansicht ist bereits durch den Vorspann gesetzt.

%% latex-abspann.tex
%% Gemeinsamer Abspann für Korrekturansicht und Leseansicht.
%% Setzt den Schalter \ifkorrekturansicht voraus (gesetzt in den
%% einbindenden Dateien latex-korrekturansicht-abspann.tex bzw.
%% latex-leseansicht-abspann.tex).
%% ---------------------------------------------------------------

\normalsize

% Das esempio-Environment wird nur in der Leseansicht benötigt
\ifkorrekturansicht\else
\newenvironment{esempio}[3]%
{
    \vspace{1.5ex}
    \rlap{\underline{#1}}
    \par
    \setlength{\parindent}{0cm}
    \nopagebreak
    \leftskip=#2cm
    \rightskip=#3cm
}
{
    \par
}
\fi

\doendnotes{C}
\bigskip
\vfill

\clearpage

\footnotesize

\ifkorrekturansicht
  \lohead{\textsc{register}}
\fi

% theindex-Environment neu definieren ohne reledmac
\makeatletter
\renewenvironment{theindex}{%
  \ifkorrekturansicht
    \section*{\indexname}%
  \else
    \subsubsection*{Index der erwähnten Entitäten}%
  \fi
  \setlength{\parindent}{0pt}%
  \setlength{\parskip}{0pt plus 0.3pt}%
  \let\item\@idxitem
}{%
  \ifkorrekturansicht\clearpage\fi
}
\makeatother

\IfFileExists{\jobname-pw.ind}{\input{\jobname-pw.ind}}{}

% Quellenangabe nur in der Leseansicht
\ifkorrekturansicht\else
% Fallback-Definitionen, falls die .tex-Datei \titel etc. nicht gesetzt hat
\providecommand{\titel}{}
\providecommand{\editorInnen}{}
\providecommand{\dateiname}{\jobname}

\vspace{3cm}

\vfill

\footnotesize
\textsc{Quelle}: \titel. Herausgegeben von {\editorInnen}. In: \emph{Arthur Schnitzler: Briefwechsel mit Autorinnen und Autoren}.
 Digitale Edition, https://schnitzler-briefe.acdh.oeaw.ac.at/{\dateiname}.html (Stand \today)
\fi

\end{document}


