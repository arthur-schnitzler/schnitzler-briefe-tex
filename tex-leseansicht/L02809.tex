%% latex-korrekturansicht-vorspann.tex
%% Vorspann für die Korrekturansicht.
%% Lädt die gemeinsame Datei latex-vorspann.tex mit gesetztem Schalter.

\newif\ifkorrekturansicht
\korrekturansichttrue

\input{../tex-inputs/latex-vorspann}


\section[ Paul Goldmann an Arthur Schnitzler, 22. 4. 1897]{L02809 Paul Goldmann an Arthur Schnitzler, 22. 4. 1897}
\nopagebreak\mylabel{L02809v}
\rehead{ }\normalsize\beginnumbering\briefempfaengerindex{Schnitzler, Arthur@\textsc{Schnitzler, Arthur}!zzzGoldmann, Paul@\emph{von Paul Goldmann}!1897-04-222@{22. 4. 1897}|(be}
\toendnotes[C]{\smallbreak\pagebreak[2]}\Standort{DLA, A:Schnitzler, HS.NZ85.1.3167.}
\physDesc{Brief, 1 Blatt, 3 Seiten, 1456 Zeichen
\newline{}Handschrift: schwarze Tinte, deutsche Kurrent
\newline{}Schnitzler: mit rotem Buntstift zwei Unterstreichungen }\toendnotes[C]{\smallbreak}
\pstart
           \centering{}{\pb}\textcolor{gray}{\textbf{Hôtel Deutscher Kaiser\oindex{Hotel Deutscher Kaiser@\textbf{Hotel Deutscher Kaiser}, \emph{Hotel (K.HTL)}|pw}}}\pend
           
\pstart
           \centering{}\textcolor{gray}{\textbf{(W. Gömöri\pwindex{Goemoeri, Wilhelm 1884-09-10 – 1917-10-20@\textsc{Gömöri, Wilhelm} (1884-09-10 – 1917-10-20), \emph{Hotelbesitzer/Hotelbesitzerin}|pw})}}\pend
           
\pstart
           \centering{}\textcolor{gray}{\textbf{Frankfurt a. M.\oindex{Frankfurt am Main@\textbf{Frankfurt am Main}, \emph{P.PPLA3}|pw}}}\pend
           
\pstart
           \textcolor{gray}{\textbf{37 Wiesenhüttenplatz 37\oindex{Hotel Deutscher Kaiser@\textbf{Hotel Deutscher Kaiser}, \emph{Hotel (K.HTL)}|pw}.}}\hfill \textcolor{gray}{\textbf{Nahe dem Centralbahnhof\oindex{Frankfurt (Main) Hauptbahnhof@\textbf{Frankfurt (Main) Hauptbahnhof}, \emph{Bahnhofsgebäude (K.BHF)}|pw}.}}\pend
           
\pstart
           \raggedleft{}\textcolor{gray}{\textbf{Frankfurt a. M.\oindex{Frankfurt am Main@\textbf{Frankfurt am Main}, \emph{P.PPLA3}|pw}, den}}{ }22. April \textcolor{gray}{\textbf{18}}97.\pend
           
\pstart{}Mein lieber Freund,\pend\vspace{0.5em}
\pstart
           Vielen Dank für Deinen lieben Brief!\pend
           
\pstart
           Ich bin ſeit Sonntag hier (nachdem ich Samſtag den Anſchluß verfehlt und in \textsc{Köln\oindex{Koeln@\textbf{Köln}, \emph{P.PPLA2}|pw}} hatte übernachten müſſen). Ich bin noch ganz krank hier angekommen und kann
               mich diesmal gar nicht erholen{[}.{]} Meine Familie iſt ſehr gut mit
               mir. Aber wir ſitzen zuſammen und denken über die ausſichtsloſe Zukunft nach, und das
               iſt nicht heiter. Auf der Redaction\orgindex{Frankfurter Zeitung@Frankfurter Zeitung|pwv} machen ſie ſchiefe Geſichter, daß ich während des \label{K_L02809-1v}\edtext{Krieges}{\lemma{\textnormal{\emph{Krieges}}}\Cendnote{\textnormal{Türk\oindex{Tuerkei@\textbf{Türkei}, \emph{A.PCLI}|pwkv}isch-Griech\oindex{Griechenland@\textbf{Griechenland}, \emph{A.PCLI}|pwkv}ischer Krieg um Kreta\oindex{Kreta@\textbf{Kreta}, \emph{T.ISL}|pwk}}}}\label{K_L02809-1} nicht auf meinem Poſten bin. Ich werde alſo wohl bald zurück {\pb}müſſen. Aber jetzt im Ruhen ſehe ich erſt, wie
               abgehetzt und müde gearbeitet ich bin.\pend
           
\pstart
           Alle die Meinigen grüßen Dich herzlichſt.\pend
           
\pstart
           Wenn Du Zeit haſt, ſchreib’ mir noch ein Wort hierher, wie es Dir geht. (Meine
               Adreſſe iſt oben auf den Brief gedruckt).\pend
           
\pstart
           Ich vergaß Dir zu ſagen, daß Du einen Abend (mit ihr\pwindex{Reinhard, Marie 1871-03-13 – 1899-03-18@\textsc{Reinhard, Marie} (1871-03-13 – 1899-03-18), \emph{Gesangspädagoge/Gesangspädagogin}|pwv}) in die »\label{K_L02809-2v}\edtext{\textsc{Scala\oindex{La Scala@\textbf{La Scala}, \emph{Veranstaltungsgebäude (K.VSB)}|pw}}}{\lemma{\textnormal{\emph{Scala}}}\Cendnote{\textnormal{Konzertsaal\oindex{La Scala@\textbf{La Scala}, \emph{Veranstaltungsgebäude (K.VSB)}|pwkv}}}}\label{K_L02809-2}« gehen ſollſt.\pend
           
\pstart
           Geſtern ſah ich \textsc{John Gabriel Borkmann\pwindex{John Gabriel Borkman.  Skuespil i fire akter@\emph{John Gabriel Borkman. Skuespil i fire akter}|pw}}. \strikeout{E} Das \strikeout{D\textcolor{gray}{a}}{ }Drama\pwindex{John Gabriel Borkman.  Skuespil i fire akter@\emph{John Gabriel Borkman. Skuespil i fire akter}|pwv} hat zu Zeiten einen
               großartigen tragiſchen Schwung. Ich zähle es zum Beſten, was \strikeout{\textcolor{gray}{×}\-\textcolor{gray}{×}\-\textcolor{gray}{×}}{ }\textsc{Ibsen\pwindex{Ibsen, Henrik 20.03.1828 – 23.05.1906@\textsc{Ibsen, Henrik} (20.03.1828 – 23.05.1906), \emph{Schriftsteller/Schriftstellerin}|pw}} gemacht hat.\pend
           
\pstart
           Mein Onkel\pwindex{Mamroth, Fedor 21.02.1851 – 25.06.1907@\textsc{Mamroth, Fedor} (21.02.1851 – 25.06.1907), \emph{Journalist/Journalistin, Kritiker/Kritikerin}|pwv} iſt voll des Lobes
               über \textsc{Bahrs\pwindex{Bahr, Hermann 19.07.1863 – 15.01.1934@\textsc{Bahr, Hermann} (19.07.1863 – 15.01.1934), \emph{Schriftsteller/Schriftstellerin, Kritiker/Kritikerin}|pw}}{ }Roman\pwindex{Theater. Ein Wiener Roman@\emph{Theater. Ein Wiener Roman}|pwv}{ }{\pb}»Theater\pwindex{Theater. Ein Wiener Roman@\emph{Theater. Ein Wiener Roman}|pw}«.
                  \label{K_L02809-3v}\edtext{Kennſt}{\lemma{\textnormal{\emph{Kennſt}}}\Cendnote{\textnormal{Schnitzler erhielt von Bahr\pwindex{Bahr, Hermann 19.07.1863 – 15.01.1934@\textsc{Bahr, Hermann} (19.07.1863 – 15.01.1934), \emph{Schriftsteller/Schriftstellerin, Kritiker/Kritikerin}|pwk} ein Widmungsexemplar\pwindex{Theater. Ein Wiener Roman@\emph{Theater. Ein Wiener Roman}|pwkv} (vgl. Hermann Bahr: Widmungsexemplar Theater. Roman für Arthur Schnitzler,
               [nach dem 20. 3. 1897]). Die Lektüre ist nur über die Leseliste gesichert (vgl. A. S.: \emph{Lektüren}, deutschsprachige Literatur).}}}\label{K_L02809-3} Du das Ding\pwindex{Theater. Ein Wiener Roman@\emph{Theater. Ein Wiener Roman}|pwv}? Es
               wäre ſchrecklich, wenn \strikeout{\textcolor{gray}{man}} dem Kerl\pwindex{Bahr, Hermann 19.07.1863 – 15.01.1934@\textsc{Bahr, Hermann} (19.07.1863 – 15.01.1934), \emph{Schriftsteller/Schriftstellerin, Kritiker/Kritikerin}|pwv} wirklich
                  \strikeout{\textcolor{gray}{ei}} einmal etwas Gutes gelungen wäre.\pend
           
\pstart
           Es freut mich, daß Du mir wegen \label{K_L02809-4v}\edtext{Freitag{ }Abend}{\lemma{\textnormal{\emph{Freitag Abend}}}\Cendnote{\textnormal{Siehe Paul Goldmann an Arthur Schnitzler, 17. 4. [1897].
               }}}\label{K_L02809-4} nicht böſe biſt. »\label{K_L02809-5v}\edtext{Sie\pwindex{?? [Partnerin? von Paul Goldmann, 1897] @\textsc{?? [Partnerin? von Paul Goldmann, 1897]}|pwv}}{\lemma{\textnormal{\emph{Sie}}}\Cendnote{\textnormal{Nicht identifiziert; womöglich handelte
                  es sich um die am 12. 4. 1897 im \emph{Tagebuch}\pwindex{Tagebuch@\emph{Tagebuch}|pwk} erwähnte
                        »›Fanny\pwindex{Fanny @\textsc{Fanny}|pw}‹« oder die am
                     13. 5. 1897
                  erwähnte »Madeleine\pwindex{Madeleine @\textsc{Madeleine}|pw}«.}}}\label{K_L02809-5}« hat mich nicht zurückgehalten, ganz im Gegentheil. Auch da gibts
               allerlei \begin{otherlanguage}{french}\textsc{Malheur}\end{otherlanguage}.\pend
           
\pstart
           Kaufe Dir die ſoeben erſchienene \label{K_L02809-6v}\edtext{\label{K_L02809-7v}\edtext{\textsc{Beaumarchais\pwindex{Beaumarchais, Pierre Augustin Caron de 24.01.1732 – 18.05.1799@\textsc{Beaumarchais, Pierre Augustin Caron de} (24.01.1732 – 18.05.1799), \emph{Schriftsteller/Schriftstellerin}|pw}}-Biographie\pwindex{Beaumarchais@\emph{Beaumarchais}|pwv}}{\lemma{\textnormal{\emph{Beaumarchais-Biographie}}}\Cendnote{\textnormal{André Hallays\pwindex{Hallays, Andre 1859-03-16 – 1930-03-30@\textsc{Hallays, André} (1859-03-16 – 1930-03-30), \emph{Journalist/Journalistin, Kunstkritiker/Kunstkritikerin, Jurist/Juristin}|pwk}: \emph{Beaumarchais}\pwindex{Beaumarchais@\emph{Beaumarchais}|pwk}. Paris: \emph{Librarie Hachette}\orgindex{Librairie Hachette@Librairie Hachette|pwk}{ }1897 (\emph{Les Grands Écrivains
                        Français}).
               }}}\label{K_L02809-7} von \textsc{André Hallays\pwindex{Hallays, Andre 1859-03-16 – 1930-03-30@\textsc{Hallays, André} (1859-03-16 – 1930-03-30), \emph{Journalist/Journalistin, Kunstkritiker/Kunstkritikerin, Jurist/Juristin}|pw}}}{\lemma{\textnormal{\emph{Beaumarchais-Biographie … Hallays}}}\Cendnote{\textnormal{Lektüre\pwindex{Beaumarchais@\emph{Beaumarchais}|pwkv} belegbar, vgl. A. S.: \emph{Lektüren}, Frankreich.}}}\label{K_L02809-6}. Ein reizendes Buch\pwindex{Beaumarchais@\emph{Beaumarchais}|pwv}.\pend
           
\pstart
           Grüße mir Deine Freundin\pwindex{Reinhard, Marie 1871-03-13 – 1899-03-18@\textsc{Reinhard, Marie} (1871-03-13 – 1899-03-18), \emph{Gesangspädagoge/Gesangspädagogin}|pwv} und
               ſei ſelbſt von Herzen gegrüßt\pend
           
\pstart
           Dein treuer {\\[\baselineskip]}\spacefill\mbox{Paul Goldmn}\pend
           \leftskip=0em{}\selectlanguage{ngerman}\endnumbering\briefempfaengerindex{Schnitzler, Arthur@\textsc{Schnitzler, Arthur}!zzzGoldmann, Paul@\emph{von Paul Goldmann}!1897-04-222@{22. 4. 1897}|)be}\mylabel{L02809h}  \normalsize

\doendnotes{C}
\bigskip
\vfill

\clearpage

\footnotesize

\lohead{\textsc{register}}

% Definiere theindex-Environment komplett neu ohne reledmac
\makeatletter
\renewenvironment{theindex}{%
  \section*{\indexname}%
  \setlength{\parindent}{0pt}%
  \setlength{\parskip}{0pt plus 0.3pt}%
  \let\item\@idxitem
}{%
  \clearpage
}
\makeatother

\IfFileExists{\jobname-pw.ind}{\input{\jobname-pw.ind}}{}

\end{document}

      