%% latex-leseansicht-vorspann.tex
%% Vorspann für die Leseansicht.
%% Lädt die gemeinsame Datei latex-vorspann.tex mit nicht gesetztem Schalter.

\newif\ifkorrekturansicht
\korrekturansichtfalse

\input{../tex-inputs/latex-vorspann}


         
         \renewcommand{\erwaehntePersonen}{Personen: Fanny , Madeleine ,  ?? [Partnerin? von Paul Goldmann, 1897], Hermann Bahr, Pierre Augustin Caron de Beaumarchais, Wilhelm Gömöri, André Hallays, Henrik Ibsen, Fedor Mamroth, Marie Reinhard}
         \renewcommand{\erwaehnteInstitutionen}{Institutionen: Frankfurter Zeitung, Librairie Hachette}
         \renewcommand{\erwaehnteOrte}{Orte: Frankfurt (Main) Hauptbahnhof, Frankfurt am Main, Griechenland, Hotel Deutscher Kaiser, Kreta, Köln, La Scala, Paris, Türkei}
         \renewcommand{\erwaehnteWerke}{Werke: Beaumarchais, John Gabriel Borkman, Tagebuch, Theater. Ein Wiener Roman}
               \section[ Paul Goldmann an Arthur Schnitzler, 22. 4. 1897]{ Paul Goldmann an Arthur Schnitzler, 22. 4. 1897}\nopagebreak\mylabel{v}\rehead{ }\begin{ledgroupsized}[t]{13cm}\normalsize\beginnumbering \toendnotes[C]{\smallbreak\pagebreak[2]} \Standort{DLA, A:Schnitzler, HS.NZ85.1.3167.}
\physDesc{Brief, 1 Blatt, 3 Seiten, 1456 Zeichen
\newline{}Handschrift: schwarze Tinte, deutsche Kurrent
\newline{}Schnitzler: mit rotem Buntstift zwei Unterstreichungen }\toendnotes[C]{\smallbreak}\pstart
           \noindent{}\centering{}{\pb}\textcolor{gray}{\textbf{Hôtel Deutscher Kaiser\oindex{Hotel Deutscher Kaiser@\textbf{Hotel Deutscher Kaiser}|pw}}}\pend
           \pstart
           \noindent{}\centering{}\textcolor{gray}{\textbf{(W. Gömöri\pwindex{Goemoeri, Wilhelm 1884-09-10 – 1917-10-20@\textsc{Gömöri, Wilhelm} (1884-09-10 – 1917-10-20), \emph{Hotelbesitzer}|pw})}}\pend
           \pstart
           \noindent{}\centering{}\textcolor{gray}{\textbf{Frankfurt a. M.\oindex{Frankfurt am Main@\textbf{Frankfurt am Main}|pw}}}\pend
           \pstart
           \noindent{}\textcolor{gray}{\textbf{37 Wiesenhüttenplatz 37\oindex{Hotel Deutscher Kaiser@\textbf{Hotel Deutscher Kaiser}|pw}.}}\hfill \textcolor{gray}{\textbf{Nahe dem Centralbahnhof\oindex{Frankfurt (Main) Hauptbahnhof@\textbf{Frankfurt (Main) Hauptbahnhof}|pw}.}}\pend
           \pstart
           \raggedleft{}\textcolor{gray}{\textbf{Frankfurt a. M.\oindex{Frankfurt am Main@\textbf{Frankfurt am Main}|pw}, den}}{ }22. April \textcolor{gray}{\textbf{18}}97.\pend
           \pstart{}Mein lieber Freund,\pend\pstart
           Vielen Dank für Deinen lieben Brief!\pend
           \pstart
           Ich bin ſeit Sonntag hier (nachdem ich Samſtag den Anſchluß verfehlt und in \textsc{Köln\oindex{Koeln@\textbf{Köln}|pw}} hatte übernachten müſſen). Ich bin noch ganz krank hier angekommen und kann
               mich diesmal gar nicht erholen{[}.{]} Meine Familie iſt ſehr gut mit
               mir. Aber wir ſitzen zuſammen und denken über die ausſichtsloſe Zukunft nach, und das
               iſt nicht heiter. Auf der Redaction\orgindex{Frankfurter Zeitung@Frankfurter Zeitung|pwv} machen ſie ſchiefe Geſichter, daß ich während des \label{K_L02809-1v}\edtext{Krieges}{\lemma{\textnormal{\emph{Krieges}}}\Cendnote{\textnormal{Türk\oindex{Tuerkei@\textbf{Türkei}|pwkv}isch-Griech\oindex{Griechenland@\textbf{Griechenland}|pwkv}ischer Krieg um Kreta\oindex{Kreta@\textbf{Kreta}|pwk}}}}\label{K_L02809-1h} nicht auf meinem Poſten bin. Ich werde alſo wohl bald zurück {\pb}müſſen. Aber jetzt im Ruhen ſehe ich erſt, wie
               abgehetzt und müde gearbeitet ich bin.\pend
           \pstart
           Alle die Meinigen grüßen Dich herzlichſt.\pend
           \pstart
           Wenn Du Zeit haſt, ſchreib’ mir noch ein Wort hierher, wie es Dir geht. (Meine
               Adreſſe iſt oben auf den Brief gedruckt).\pend
           \pstart
           Ich vergaß Dir zu ſagen, daß Du einen Abend (mit ihr\pwindex{Reinhard, Marie 1871-03-13 – 1899-03-18@\textsc{Reinhard, Marie} (1871-03-13 – 1899-03-18), \emph{Gesangspädagogin}|pwv}) in die »\label{K_L02809-2v}\edtext{\textsc{Scala\oindex{La Scala@\textbf{La Scala}|pw}}}{\lemma{\textnormal{\emph{Scala}}}\Cendnote{\textnormal{Konzertsaal\oindex{La Scala@\textbf{La Scala}|pwkv}}}}\label{K_L02809-2h}« gehen ſollſt.\pend
           \pstart
           Geſtern ſah ich \textsc{John Gabriel Borkmann\pwindex{Ibsen, Henrik 20.03.1828 – 23.05.1906@\textsc{Ibsen, Henrik} (20.03.1828 – 23.05.1906), \emph{Schriftsteller}!John Gabriel Borkman1896@\strich\emph{John Gabriel Borkman} {[}1896{]}|pw}}. \strikeout{E} Das \strikeout{D\textcolor{gray}{a}}{ }Drama\pwindex{Ibsen, Henrik 20.03.1828 – 23.05.1906@\textsc{Ibsen, Henrik} (20.03.1828 – 23.05.1906), \emph{Schriftsteller}!John Gabriel Borkman1896@\strich\emph{John Gabriel Borkman} {[}1896{]}|pwv} hat zu Zeiten einen
               großartigen tragiſchen Schwung. Ich zähle es zum Beſten, was \strikeout{\textcolor{gray}{×}\-\textcolor{gray}{×}\-\textcolor{gray}{×}}{ }\textsc{Ibsen\pwindex{Ibsen, Henrik 20.03.1828 – 23.05.1906@\textsc{Ibsen, Henrik} (20.03.1828 – 23.05.1906), \emph{Schriftsteller}|pw}} gemacht hat.\pend
           \pstart
           Mein Onkel\pwindex{Mamroth, Fedor 21.02.1851 – 25.06.1907@\textsc{Mamroth, Fedor} (21.02.1851 – 25.06.1907), \emph{Journalist, Kritiker}|pwv} iſt voll des Lobes
               über \textsc{Bahr\pwindex{Bahr, Hermann 19.07.1863 – 15.01.1934@\textsc{Bahr, Hermann} (19.07.1863 – 15.01.1934), \emph{Schriftsteller, Kritiker}|pw}s}{ }Roman\pwindex{Bahr, Hermann 19.07.1863 – 15.01.1934@\textsc{Bahr, Hermann} (19.07.1863 – 15.01.1934), \emph{Schriftsteller, Kritiker}!Theater. Ein Wiener Roman1897-03-20@\strich\emph{Theater. Ein Wiener Roman} {[}1897-03-20{]}|pwv}{ }{\pb}»Theater\pwindex{Bahr, Hermann 19.07.1863 – 15.01.1934@\textsc{Bahr, Hermann} (19.07.1863 – 15.01.1934), \emph{Schriftsteller, Kritiker}!Theater. Ein Wiener Roman1897-03-20@\strich\emph{Theater. Ein Wiener Roman} {[}1897-03-20{]}|pw}«.
                  \label{K_L02809-3v}\edtext{Kennſt}{\lemma{\textnormal{\emph{Kennſt}}}\Cendnote{\textnormal{Schnitzler\pwindex{Schnitzler, Arthur 15.05.1862 – 21.10.1931@\textsc{Schnitzler, Arthur} (15.05.1862 – 21.10.1931), \emph{Schriftsteller, Mediziner}|pwk} erhielt von Bahr\pwindex{Bahr, Hermann 19.07.1863 – 15.01.1934@\textsc{Bahr, Hermann} (19.07.1863 – 15.01.1934), \emph{Schriftsteller, Kritiker}|pwk} ein Widmungsexemplar\pwindex{Bahr, Hermann 19.07.1863 – 15.01.1934@\textsc{Bahr, Hermann} (19.07.1863 – 15.01.1934), \emph{Schriftsteller, Kritiker}!Theater. Ein Wiener Roman1897-03-20@\strich\emph{Theater. Ein Wiener Roman} {[}1897-03-20{]}|pwkv} (vgl. Hermann Bahr: Widmungsexemplar Theater. Roman für Arthur Schnitzler,
               [nach dem 20. 3. 1897]). Die Lektüre ist nur über die Leseliste gesichert (vgl. A. S.: \emph{Lektüren}, Deutschsprachige-Literatur).}}}\label{K_L02809-3h} Du das Ding\pwindex{Bahr, Hermann 19.07.1863 – 15.01.1934@\textsc{Bahr, Hermann} (19.07.1863 – 15.01.1934), \emph{Schriftsteller, Kritiker}!Theater. Ein Wiener Roman1897-03-20@\strich\emph{Theater. Ein Wiener Roman} {[}1897-03-20{]}|pwv}? Es
               wäre ſchrecklich, wenn \strikeout{\textcolor{gray}{man}} dem Kerl\pwindex{Bahr, Hermann 19.07.1863 – 15.01.1934@\textsc{Bahr, Hermann} (19.07.1863 – 15.01.1934), \emph{Schriftsteller, Kritiker}|pwv} wirklich
                  \strikeout{\textcolor{gray}{ei}} einmal etwas Gutes gelungen wäre.\pend
           \pstart
           Es freut mich, daß Du mir wegen \label{K_L02809-4v}\edtext{Freitag{ }Abend}{\lemma{\textnormal{\emph{Freitag Abend}}}\Cendnote{\textnormal{siehe Paul Goldmann an Arthur Schnitzler, 17. 4. [1897]}}}\label{K_L02809-4h} nicht böſe biſt. »\label{K_L02809-5v}\edtext{Sie\pwindex{?? [Partnerin? von Paul Goldmann, 1897] @\textsc{?? [Partnerin? von Paul Goldmann, 1897]}|pwv}}{\lemma{\textnormal{\emph{Sie}}}\Cendnote{\textnormal{nicht identifiziert; womöglich handelte
                  es sich um die am 12. 4. 1897 im \emph{Tagebuch}\pwindex{Schnitzler, Arthur 15.05.1862 – 21.10.1931@\textsc{Schnitzler, Arthur} (15.05.1862 – 21.10.1931), \emph{Schriftsteller, Mediziner}!Tagebuch1981 – 2000@\strich\emph{Tagebuch} {[}1981 – 2000{]}|pwk} erwähnte
                        »›Fanny\pwindex{Fanny @\textsc{Fanny}|pw}‹« oder die am
                     13. 5. 1897
                  erwähnte »Madeleine\pwindex{Madeleine @\textsc{Madeleine}|pw}«}}}\label{K_L02809-5h}« hat mich nicht zurückgehalten, ganz im Gegentheil. Auch da gibts
               allerlei \begin{otherlanguage}{french}\textsc{Malheur}\end{otherlanguage}.\pend
           \pstart
           Kaufe Dir die ſoeben erſchienene \label{K_L02809-6v}\edtext{\label{K_L02809-7v}\edtext{\textsc{Beaumarchais\pwindex{Beaumarchais, Pierre Augustin Caron de 24.01.1732 – 18.05.1799@\textsc{Beaumarchais, Pierre Augustin Caron de} (24.01.1732 – 18.05.1799), \emph{Schriftsteller}|pw}}-Biographie\pwindex{Hallays, Andre 1859-03-16 – 1930-03-30@\textsc{Hallays, André} (1859-03-16 – 1930-03-30), \emph{Journalist, Kritiker, Jurist}!Beaumarchais1897@\strich\emph{Beaumarchais} {[}1897{]}|pwv}}{\lemma{\textnormal{\emph{Beaumarchais-Biographie}}}\Cendnote{\textnormal{André Hallays\pwindex{Hallays, Andre 1859-03-16 – 1930-03-30@\textsc{Hallays, André} (1859-03-16 – 1930-03-30), \emph{Journalist, Kritiker, Jurist}|pwk}: \emph{Beaumarchais}\pwindex{Hallays, Andre 1859-03-16 – 1930-03-30@\textsc{Hallays, André} (1859-03-16 – 1930-03-30), \emph{Journalist, Kritiker, Jurist}!Beaumarchais1897@\strich\emph{Beaumarchais} {[}1897{]}|pwk}. Paris: \emph{Librarie Hachette}\orgindex{Librairie Hachette@Librairie Hachette|pwk}{ }1897. (\emph{Les Grands Écrivains
                        Français})}}}\label{K_L02809-7h} von \textsc{André Hallays\pwindex{Hallays, Andre 1859-03-16 – 1930-03-30@\textsc{Hallays, André} (1859-03-16 – 1930-03-30), \emph{Journalist, Kritiker, Jurist}|pw}}}{\lemma{\textnormal{\emph{Beaumarchais-Biographie … Hallays}}}\Cendnote{\textnormal{Lektüre\pwindex{Hallays, Andre 1859-03-16 – 1930-03-30@\textsc{Hallays, André} (1859-03-16 – 1930-03-30), \emph{Journalist, Kritiker, Jurist}!Beaumarchais1897@\strich\emph{Beaumarchais} {[}1897{]}|pwkv} belegbar, vgl. A. S.: \emph{Lektüren}, Frankreich}}}\label{K_L02809-6h}. Ein reizendes Buch\pwindex{Hallays, Andre 1859-03-16 – 1930-03-30@\textsc{Hallays, André} (1859-03-16 – 1930-03-30), \emph{Journalist, Kritiker, Jurist}!Beaumarchais1897@\strich\emph{Beaumarchais} {[}1897{]}|pwv}.\pend
           \pstart
           Grüße mir Deine Freundin\pwindex{Reinhard, Marie 1871-03-13 – 1899-03-18@\textsc{Reinhard, Marie} (1871-03-13 – 1899-03-18), \emph{Gesangspädagogin}|pwv} und
               ſei ſelbſt von Herzen gegrüßt\pend
           \pstart
           Dein treuer {\\[\baselineskip]}\spacefill\mbox{Paul Goldmn}\pend
           \leftskip=0em{}
         
         \endnumbering\mylabel{h}\end{ledgroupsized}  \newcommand{\dateiname}{L02809}\newcommand{\titel}{Paul Goldmann an Arthur Schnitzler, 22. 4. 1897}\newcommand{\editorInnen}{Martin Anton Müller und Laura Untner}%% latex-leseansicht-abspann.tex
%% Abspann für die Leseansicht.
%% Der Schalter \ifkorrekturansicht ist bereits durch den Vorspann gesetzt.

%% latex-abspann.tex
%% Gemeinsamer Abspann für Korrekturansicht und Leseansicht.
%% Setzt den Schalter \ifkorrekturansicht voraus (gesetzt in den
%% einbindenden Dateien latex-korrekturansicht-abspann.tex bzw.
%% latex-leseansicht-abspann.tex).
%% ---------------------------------------------------------------

\normalsize

% Das esempio-Environment wird nur in der Leseansicht benötigt
\ifkorrekturansicht\else
\newenvironment{esempio}[3]%
{
    \vspace{1.5ex}
    \rlap{\underline{#1}}
    \par
    \setlength{\parindent}{0cm}
    \nopagebreak
    \leftskip=#2cm
    \rightskip=#3cm
}
{
    \par
}
\fi

\doendnotes{C}
\bigskip
\vfill

\clearpage

\footnotesize

\ifkorrekturansicht
  \lohead{\textsc{register}}
\fi

% theindex-Environment neu definieren ohne reledmac
\makeatletter
\renewenvironment{theindex}{%
  \ifkorrekturansicht
    \section*{\indexname}%
  \else
    \subsubsection*{Index der erwähnten Entitäten}%
  \fi
  \setlength{\parindent}{0pt}%
  \setlength{\parskip}{0pt plus 0.3pt}%
  \let\item\@idxitem
}{%
  \ifkorrekturansicht\clearpage\fi
}
\makeatother

\IfFileExists{\jobname-pw.ind}{\input{\jobname-pw.ind}}{}

% Quellenangabe nur in der Leseansicht
\ifkorrekturansicht\else
% Fallback-Definitionen, falls die .tex-Datei \titel etc. nicht gesetzt hat
\providecommand{\titel}{}
\providecommand{\editorInnen}{}
\providecommand{\dateiname}{\jobname}

\vspace{3cm}

\vfill

\footnotesize
\textsc{Quelle}: \titel. Herausgegeben von {\editorInnen}. In: \emph{Arthur Schnitzler: Briefwechsel mit Autorinnen und Autoren}.
 Digitale Edition, https://schnitzler-briefe.acdh.oeaw.ac.at/{\dateiname}.html (Stand \today)
\fi

\end{document}


      