%% latex-korrekturansicht-vorspann.tex
%% Vorspann für die Korrekturansicht.
%% Lädt die gemeinsame Datei latex-vorspann.tex mit gesetztem Schalter.

\newif\ifkorrekturansicht
\korrekturansichttrue

\input{../tex-inputs/latex-vorspann}


\section[Hermann Bahr an Arthur Schnitzler, 18. 12. 1909]{L01906 Hermann Bahr an Arthur Schnitzler, 18. 12. 1909}
\nopagebreak\mylabel{L01906v}
\rehead{ }\normalsize\beginnumbering\briefempfaengerindex{Schnitzler, Arthur@\textsc{Schnitzler, Arthur}!zzzBahr, Hermann@\emph{von Hermann Bahr}!1909-12-181@{18. 12. 1909}|(be}
\toendnotes[C]{\smallbreak\pagebreak[2]}\Standort{CUL, Schnitzler, B 5b.}
\physDesc{Brief, 1 Blatt, 1 Seite, 300 Zeichen
\newline{}Handschrift: schwarze Tinte, deutsche Kurrent
\newline{}Schnitzler: mit Bleistift ergänzt »Bahr« 
\newline{}Ordnung: mit Bleistift von unbekannter Hand nummeriert:
                                    »164« }
\buchAbdrucke{\weitereDrucke{Hermann Bahr, Arthur Schnitzler: \emph{Briefwechsel, Aufzeichnungen, Dokumente (1891–1931)}. Göttingen: \emph{Wallstein} 2018, S. 430.} }
\pstart
           \raggedleft{}{\pb}Semmering \textsc{Villa Mautner}\oindex{Villa Mauthner-Markhof@\textbf{Villa Mauthner-Markhof}, \emph{Wohngebäude (K.WHS)}|pw}{\\}18. 12. 09\pend
           
\pstart\center{}Lieber Artur!\pend\vspace{0.5em}
\pstart
           Schönſten Dank für Deinen Brief. Wenn wir uns zwiſchen dem 25., wo ich wieder in Wien\oindex{Wien@\textbf{Wien}, \emph{A.ADM2}|pw} bin, und dem 31., wo ich wieder abreiſe,
               einmal ſehen könnten, wärs ſehr ſchön. Ich ſehne mich ſchon lange danach.\pend
           
\pstart
           Mit den ſchönſten Grüßen{\\[\baselineskip]}von Haus zu Haus{\\[\baselineskip]}Dein alter{\\[\baselineskip]}\spacefill\mbox{Hermann}\pend
           \leftskip=0em{}\selectlanguage{ngerman}\endnumbering\briefempfaengerindex{Schnitzler, Arthur@\textsc{Schnitzler, Arthur}!zzzBahr, Hermann@\emph{von Hermann Bahr}!1909-12-181@{18. 12. 1909}|)be}\mylabel{L01906h}  \normalsize

\doendnotes{C}
\bigskip
\vfill

\clearpage

\footnotesize

\lohead{\textsc{register}}

% Definiere theindex-Environment komplett neu ohne reledmac
\makeatletter
\renewenvironment{theindex}{%
  \section*{\indexname}%
  \setlength{\parindent}{0pt}%
  \setlength{\parskip}{0pt plus 0.3pt}%
  \let\item\@idxitem
}{%
  \clearpage
}
\makeatother

\IfFileExists{\jobname-pw.ind}{\input{\jobname-pw.ind}}{}

\end{document}

      