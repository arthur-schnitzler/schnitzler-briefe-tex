%% latex-leseansicht-vorspann.tex
%% Vorspann für die Leseansicht.
%% Lädt die gemeinsame Datei latex-vorspann.tex mit nicht gesetztem Schalter.

\newif\ifkorrekturansicht
\korrekturansichtfalse

\input{../tex-inputs/latex-vorspann}


               \section[Arthur Schnitzler an Richard Beer-Hofmann, 31. 8. 1897]{ Arthur Schnitzler an Richard Beer-Hofmann, 31. 8. 1897}\nopagebreak\mylabel{v}\rehead{ }\begin{ledgroupsized}[t]{13cm}\normalsize\beginnumbering\briefempfaengerindex{Beer-Hofmann, Richard@\textsc{Beer-Hofmann, Richard}!zzzSchnitzler, Arthur@\emph{von Arthur Schnitzler}!1897-08-313@{31. 8. 1897}|(be} \toendnotes[C]{\smallbreak\pagebreak[2]} \Standort{YCGL, MSS 31.}
\physDesc{Briefkarte, Umschlag
\newline{}Handschrift: Bleistift, deutsche Kurrent\newline{}Versand: 1) Rohrpost 2) Stempel: »\nobreak{}\oindex{VIII., Josefstadt@\textbf{VIII., Josefstadt}|pwk}Wien 8/1, \textcolor{gray}{1} IX 97, 9 10V\nobreak{}«. 3) Stempel: »\nobreak{}\oindex{I., Innere Stadt@\textbf{I., Innere Stadt}|pwk}Wien 1/1, 1 XI 97, 9 30V\nobreak{}«. }\toendnotes[C]{\smallbreak}\pstart{}{\pb}\damage{Herrn Dr.}{ }\textsc{Richard Beer Hofmann}\pend{}\pstart{}Wien\oindex{Wien@\textbf{Wien}|pw}\pend{}\pstart{}\textsc{I. Wollzeile 15}\oindex{Wollzeile@\textbf{Wollzeile}|pw}.\pend{}{\bigskip}\pstart
           \noindent{}{\pb}Lieber Richard, Ihren Brief erhielt ich
               um \label{K_L00721_1v}\edtext{¾ 10}{\lemma{\textnormal{\emph{¾ 10}}}\Cendnote{\textnormal{21 Uhr 45}}}\label{K_L00721_1h} im Arkaden\oindex{Cafe Arkaden@\textbf{Café Arkaden}|pw}. War zu müd Sie zu erwarten. Morgen
                  (Mittwoch) hab ich keine Sekunde für mich; denkbar wäre ſehr ſpät \textsc{Arkadencafé}\oindex{Cafe Arkaden@\textbf{Café Arkaden}|pw}. Do{\geminationn}erſtag{ }ſchreib ich Ihnen. Ich bin ſehr, ſehr \label{K_L00721_2v}\edtext{nervös}{\lemma{\textnormal{\emph{nervös}}}\Cendnote{\textnormal{womöglich wegen der bevorstehenden Entbindung seiner
                  Lebensgefährtin Marie Reinhard\pwindex{Reinhard, Marie 13.03.1871 – 18.03.1899@\textsc{Reinhard, Marie} (13.03.1871 – 18.03.1899), \emph{Gesangspädagogin}|pwk}. Am
                     24. 9. 1897 kam ein Kind\pwindex{?? [Totgeborenes Kind von Arthur Schnitzler und Marie Reinhard] 24.9.1897 – 24.9.1897@\textsc{?? [Totgeborenes Kind von Arthur Schnitzler und Marie Reinhard]} (24.9.1897 – 24.9.1897)|pwkv} tot auf die Welt.}}}\label{K_L00721_2h}.\pend
           \pstart
           {\pb}Bei Ihnen geht doch \label{K_L00721_3v}\edtext{alles gut}{\lemma{\textnormal{\emph{alles gut}}}\Cendnote{\textnormal{Am
                     4. 9. 1897 kam die Tochter Mirjam
                     Beer-Hofmann\pwindex{Beer-Hofmann, Mirjam 04.09.1897 – 24.12.1984@\textsc{Beer-Hofmann, Mirjam} (04.09.1897 – 24.12.1984)|pwk} zur Welt.}}}\label{K_L00721_3h}?\pend
           \pstart Herzlich Ihr \spacefill\mbox{Arthur}\pend{}\endnumbering\briefempfaengerindex{Beer-Hofmann, Richard@\textsc{Beer-Hofmann, Richard}!zzzSchnitzler, Arthur@\emph{von Arthur Schnitzler}!1897-08-313@{31. 8. 1897}|)be}\mylabel{h}\end{ledgroupsized}  \newcommand{\dateiname}{L00721}\newcommand{\titel}{Arthur Schnitzler an Richard Beer-Hofmann, 31. 8. 1897}\newcommand{\editorInnen}{ Martin Anton Müller und Gerd-Hermann Susen}
            \footnotesize
\begin{ledgroupsized}[t]{11.5cm}
\doendnotes{C}
\end{ledgroupsized}
         %% latex-leseansicht-abspann.tex
%% Abspann für die Leseansicht.
%% Der Schalter \ifkorrekturansicht ist bereits durch den Vorspann gesetzt.

%% latex-abspann.tex
%% Gemeinsamer Abspann für Korrekturansicht und Leseansicht.
%% Setzt den Schalter \ifkorrekturansicht voraus (gesetzt in den
%% einbindenden Dateien latex-korrekturansicht-abspann.tex bzw.
%% latex-leseansicht-abspann.tex).
%% ---------------------------------------------------------------

\normalsize

% Das esempio-Environment wird nur in der Leseansicht benötigt
\ifkorrekturansicht\else
\newenvironment{esempio}[3]%
{
    \vspace{1.5ex}
    \rlap{\underline{#1}}
    \par
    \setlength{\parindent}{0cm}
    \nopagebreak
    \leftskip=#2cm
    \rightskip=#3cm
}
{
    \par
}
\fi

\doendnotes{C}
\bigskip
\vfill

\clearpage

\footnotesize

\ifkorrekturansicht
  \lohead{\textsc{register}}
\fi

% theindex-Environment neu definieren ohne reledmac
\makeatletter
\renewenvironment{theindex}{%
  \ifkorrekturansicht
    \section*{\indexname}%
  \else
    \subsubsection*{Index der erwähnten Entitäten}%
  \fi
  \setlength{\parindent}{0pt}%
  \setlength{\parskip}{0pt plus 0.3pt}%
  \let\item\@idxitem
}{%
  \ifkorrekturansicht\clearpage\fi
}
\makeatother

\IfFileExists{\jobname-pw.ind}{\input{\jobname-pw.ind}}{}

% Quellenangabe nur in der Leseansicht
\ifkorrekturansicht\else
% Fallback-Definitionen, falls die .tex-Datei \titel etc. nicht gesetzt hat
\providecommand{\titel}{}
\providecommand{\editorInnen}{}
\providecommand{\dateiname}{\jobname}

\vspace{3cm}

\vfill

\footnotesize
\textsc{Quelle}: \titel. Herausgegeben von {\editorInnen}. In: \emph{Arthur Schnitzler: Briefwechsel mit Autorinnen und Autoren}.
 Digitale Edition, https://schnitzler-briefe.acdh.oeaw.ac.at/{\dateiname}.html (Stand \today)
\fi

\end{document}


      