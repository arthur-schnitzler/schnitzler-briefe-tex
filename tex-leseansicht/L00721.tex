%% latex-korrekturansicht-vorspann.tex
%% Vorspann für die Korrekturansicht.
%% Lädt die gemeinsame Datei latex-vorspann.tex mit gesetztem Schalter.

\newif\ifkorrekturansicht
\korrekturansichttrue

\input{../tex-inputs/latex-vorspann}


\section[Arthur Schnitzler an Richard Beer-Hofmann, 31. 8. 1897]{L00721 Arthur Schnitzler an Richard Beer-Hofmann, 31. 8. 1897}
\nopagebreak\mylabel{L00721v}
\rehead{ }\normalsize\beginnumbering\briefempfaengerindex{Beer-Hofmann, Richard@\textsc{Beer-Hofmann, Richard}!zzzSchnitzler, Arthur@\emph{von Arthur Schnitzler}!1897-08-313@{31. 8. 1897}|(be}
\toendnotes[C]{\smallbreak\pagebreak[2]}\Standort{YCGL, MSS 31.}
\physDesc{Briefkarte, , Umschlag, 319 Zeichen
\newline{}Handschrift: Bleistift, deutsche Kurrent
\newline{}Versand: 1) Rohrpost  2) Stempel: »\nobreak{}\oindex{VIII., Josefstadt@\textbf{VIII., Josefstadt}, \emph{A.ADM3}|pwk}Wien 8/1, \textcolor{gray}{1} IX 97, 9 10V\nobreak{}«.  3) Stempel: »\nobreak{}\oindex{I., Innere Stadt@\textbf{I., Innere Stadt}, \emph{A.ADM3}|pwk}Wien 1/1, 1 XI 97, 9 30V\nobreak{}«. }\toendnotes[C]{\smallbreak}\pstart{}{\pb}\damage{Herrn Dr.}{ }\textsc{Richard Beer Hofmann}\pend{}\pstart{}Wien\oindex{Wien@\textbf{Wien}, \emph{A.ADM2}|pw}\pend{}\pstart{}\textsc{I. Wollzeile 15}\oindex{Wollzeile@\textbf{Wollzeile}, \emph{Straße (K.STR)}|pw}.\pend{}{\bigskip}\vspace{1em}
\pstart
           \noindent{}{\pb}Lieber Richard, Ihren Brief erhielt ich
               um \label{K_L00721-1v}\edtext{¾ 10}{\lemma{\textnormal{\emph{¾ 10}}}\Cendnote{\textnormal{21 Uhr 45}}}\label{K_L00721-1} im Arkaden\oindex{Cafe Arkaden@\textbf{Café Arkaden}, \emph{Kaffeehaus (K.KAF)}|pw}. War zu müd Sie zu erwarten.
               Morgen (Mittwoch) hab ich keine Sekunde für mich; denkbar wäre ſehr ſpät
                  \textsc{Arkadencafé}\oindex{Cafe Arkaden@\textbf{Café Arkaden}, \emph{Kaffeehaus (K.KAF)}|pw}. Do{\geminationn}erſtag{ }ſchreib ich Ihnen. Ich bin ſehr, ſehr \label{K_L00721-2v}\edtext{nervös}{\lemma{\textnormal{\emph{nervös}}}\Cendnote{\textnormal{Bei seiner
                     Lebensgefährtin Marie Reinhard\pwindex{Reinhard, Marie 1871-03-13 – 1899-03-18@\textsc{Reinhard, Marie} (1871-03-13 – 1899-03-18), \emph{Gesangspädagoge/Gesangspädagogin}|pwk} stand die Entbindung kurz bevor. Das Kind\pwindex{?? [Totgeborener Sohn von Arthur Schnitzler und Marie Reinhard] 1897-09-24 – 1897-09-24@\textsc{?? [Totgeborener Sohn von Arthur Schnitzler und Marie Reinhard]} (1897-09-24 – 1897-09-24)|pwkv} kam am 24. 9. 1897 tot auf die Welt.}}}\label{K_L00721-2}.\pend
           
\pstart
           {\pb}Bei Ihnen geht doch \label{K_L00721-3v}\edtext{alles gut}{\lemma{\textnormal{\emph{alles gut}}}\Cendnote{\textnormal{Am
                     4. 9. 1897 kam die Tochter Mirjam
                     Beer-Hofmann\pwindex{Beer-Hofmann, Mirjam 04.09.1897 – 24.12.1984@\textsc{Beer-Hofmann, Mirjam} (04.09.1897 – 24.12.1984)|pwk} zur Welt.}}}\label{K_L00721-3}?\pend
           \pstart Herzlich Ihr \spacefill\mbox{Arthur}\pend{}\selectlanguage{ngerman}\endnumbering\briefempfaengerindex{Beer-Hofmann, Richard@\textsc{Beer-Hofmann, Richard}!zzzSchnitzler, Arthur@\emph{von Arthur Schnitzler}!1897-08-313@{31. 8. 1897}|)be}\mylabel{L00721h}  \normalsize

\doendnotes{C}
\bigskip
\vfill

\clearpage

\footnotesize

\lohead{\textsc{register}}

% Definiere theindex-Environment komplett neu ohne reledmac
\makeatletter
\renewenvironment{theindex}{%
  \section*{\indexname}%
  \setlength{\parindent}{0pt}%
  \setlength{\parskip}{0pt plus 0.3pt}%
  \let\item\@idxitem
}{%
  \clearpage
}
\makeatother

\IfFileExists{\jobname-pw.ind}{\input{\jobname-pw.ind}}{}

\end{document}

      