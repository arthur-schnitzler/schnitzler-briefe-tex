%% latex-leseansicht-vorspann.tex
%% Vorspann für die Leseansicht.
%% Lädt die gemeinsame Datei latex-vorspann.tex mit nicht gesetztem Schalter.

\newif\ifkorrekturansicht
\korrekturansichtfalse

\input{../tex-inputs/latex-vorspann}

\begin{center}
            \textcolor{red}{ENTWURF, NICHT FERTIG KORRIGIERT}
                      \end{center}
            
         
         \renewcommand{\erwaehntePersonen}{Personen: Julius von Gans-Ludassy, Gustav Harpner}
         \renewcommand{\erwaehnteInstitutionen}{Institutionen: Ullstein Verlag}
         \renewcommand{\erwaehnteOrte}{Orte: Berlin, Palasthotel Berlin, Wien}
         \renewcommand{\erwaehnteWerke}{Werke: ?? [Artikel über Salten, gegen den dieser klagt]}
               \section[Felix Salten an Arthur Schnitzler, 17. 5. 1906]{ Felix Salten an Arthur Schnitzler, 17. 5. 1906}\nopagebreak\mylabel{v}\rehead{ }\begin{ledgroupsized}[t]{13cm}\normalsize\beginnumbering \toendnotes[C]{\smallbreak\pagebreak[2]} \Standort{CUL, Schnitzler, B 89, B 1.}
\physDesc{Brief, 1 Blatt, 2 Seiten, 1450 Zeichen
\newline{}Handschrift: schwarze Tinte, lateinische Kurrent
\newline{}Ordnung: mit Bleistift von unbekannter Hand nummeriert:
                                    »216« }\toendnotes[C]{\smallbreak}\pstart
           \raggedleft{}{\pb}Berlin\oindex{Berlin@\textbf{Berlin}|pw}, 17. V. 06. \pend
           \pstart
           Lieber, da Sie rasche Auskunft wünschen (warum?) in aller Kürze: ich
               höre von meinem \label{K_L03425-1v}\edtext{Anwalt\pwindex{Harpner, Gustav 25.03.1864 – 10.07.1924@\textsc{Harpner, Gustav} (25.03.1864 – 10.07.1924), \emph{Rechtsanwalt}|pwv}}{\lemma{\textnormal{\emph{Anwalt}}}\Cendnote{\textnormal{Es dürfte sich um Gustav Harpner\pwindex{Harpner, Gustav 25.03.1864 – 10.07.1924@\textsc{Harpner, Gustav} (25.03.1864 – 10.07.1924), \emph{Rechtsanwalt}|pwk} handeln, vgl. Felix Salten an Arthur Schnitzler, [20.? 10. 1906].}}}\label{K_L03425-1h}, dass Herr D\textsuperscript{r}{ }v. Ludaßy\pwindex{Gans-Ludassy, Julius von 13.04.1858 – 30.09.1922@\textsc{Gans-Ludassy, Julius von} (13.04.1858 – 30.09.1922), \emph{Schriftsteller, Journalist, Herausgeber}|pw} sich jetzt hinter die subjective
               Verjährung verkriechen will; d. h. er macht geltend: der bewußte Angriff\pwindex{Gans-Ludassy, Julius von 13.04.1858 – 30.09.1922@\textsc{Gans-Ludassy, Julius von} (13.04.1858 – 30.09.1922), \emph{Schriftsteller, Journalist, Herausgeber}!?? [Artikel ueber Salten, gegen den dieser klagt]Ende 1905/Anfang 1906{[}25.12.1905{]}@\strich\emph{?? [Artikel über Salten, gegen den dieser klagt]} {[}Ende 1905/Anfang 1906{[}25.12.1905{]}{]}|pw} sei wol innerhalb der gesetzlichen Frist nach seinem
                  \uline{Erscheinen} geklagt worden, sei aber sechs Monate
                  \uline{vor} seinem Erscheinen \uline{geschrieben} worden. Er verlangt, dass man die Zeit so misst, dass man von
               dem Tag an rechnet, an welchem die Tat \uline{begangen}
               wurde! Da käme ihm dann der Schutz der Verjährung zu gute, und er hätte mich straflos
               der Bestechlichkeit beschuldigt, weil ich ihn erst verklagte, als ich seinen Artikel\pwindex{Gans-Ludassy, Julius von 13.04.1858 – 30.09.1922@\textsc{Gans-Ludassy, Julius von} (13.04.1858 – 30.09.1922), \emph{Schriftsteller, Journalist, Herausgeber}!?? [Artikel ueber Salten, gegen den dieser klagt]Ende 1905/Anfang 1906{[}25.12.1905{]}@\strich\emph{?? [Artikel über Salten, gegen den dieser klagt]} {[}Ende 1905/Anfang 1906{[}25.12.1905{]}{]}|pwv} gedruckt las, und
               nicht schon, als er ihn aufgeschrieben hatte. »Es wär’ not« – man müßt’ alle 14 Tag
               zu Ludaßy\pwindex{Gans-Ludassy, Julius von 13.04.1858 – 30.09.1922@\textsc{Gans-Ludassy, Julius von} (13.04.1858 – 30.09.1922), \emph{Schriftsteller, Journalist, Herausgeber}|pw} Fragen schicken: »Haben Sie nicht
               eine Gemeinheit gegen mich begangen?« Ob er mit dieser Bemühung durchdringt, weiß ich
               nicht. \pend
           \pstart
           Hier hat Herr D\textsuperscript{r} v.
                  Ludaßy\pwindex{Gans-Ludassy, Julius von 13.04.1858 – 30.09.1922@\textsc{Gans-Ludassy, Julius von} (13.04.1858 – 30.09.1922), \emph{Schriftsteller, Journalist, Herausgeber}|pw} an Ullsteins\orgindex{Ullstein Verlag@Ullstein Verlag|pw} telegrafirt: »Habe
               Ihnen Verlagsproject vorzuschlagen. Bitte mir unter \uline{Vermeidung Saltens} mitzuteilen, wann ich Sie sprechen kann. Wohne Palasthotel\oindex{Palasthotel Berlin@\textbf{Palasthotel Berlin}|pw}. L.\pwindex{Gans-Ludassy, Julius von 13.04.1858 – 30.09.1922@\textsc{Gans-Ludassy, Julius von} (13.04.1858 – 30.09.1922), \emph{Schriftsteller, Journalist, Herausgeber}|pw}« Ullsteins\orgindex{Ullstein Verlag@Ullstein Verlag|pw} haben mir die Depesche
               sofort gezeigt. \pend
           \pstart
           Zu diesen Dingen kann ich mich wol jeder Bemerkung enthalten. \pend
           \pstart
           Nun aber genug. Ich will auch nichts von anderen Dingen {\pb}schreiben, die mir wie Ihnen
               näher u. lieber sind. Es widerstrebt mir aufrichtig, sie in einem Zug mit Ludaßy\pwindex{Gans-Ludassy, Julius von 13.04.1858 – 30.09.1922@\textsc{Gans-Ludassy, Julius von} (13.04.1858 – 30.09.1922), \emph{Schriftsteller, Journalist, Herausgeber}|pw} zu erörtern. Ohnehin störts mich genug,
               dass dieses Schwein sich immer durch unsere Briefe wälzt.\pend
           \pstart
           Herzlichst {\\[\baselineskip]}Ihr \spacefill\mbox{Salten}\pend
           \leftskip=0em{}
         
         \endnumbering\mylabel{h}\end{ledgroupsized}\begin{anhang}\end{anhang}\newcommand{\dateiname}{L03425}\newcommand{\titel}{Felix Salten an Arthur Schnitzler, 17. 5. 1906}\newcommand{\editorInnen}{Martin Anton Müller und Laura Untner}%% latex-leseansicht-abspann.tex
%% Abspann für die Leseansicht.
%% Der Schalter \ifkorrekturansicht ist bereits durch den Vorspann gesetzt.

%% latex-abspann.tex
%% Gemeinsamer Abspann für Korrekturansicht und Leseansicht.
%% Setzt den Schalter \ifkorrekturansicht voraus (gesetzt in den
%% einbindenden Dateien latex-korrekturansicht-abspann.tex bzw.
%% latex-leseansicht-abspann.tex).
%% ---------------------------------------------------------------

\normalsize

% Das esempio-Environment wird nur in der Leseansicht benötigt
\ifkorrekturansicht\else
\newenvironment{esempio}[3]%
{
    \vspace{1.5ex}
    \rlap{\underline{#1}}
    \par
    \setlength{\parindent}{0cm}
    \nopagebreak
    \leftskip=#2cm
    \rightskip=#3cm
}
{
    \par
}
\fi

\doendnotes{C}
\bigskip
\vfill

\clearpage

\footnotesize

\ifkorrekturansicht
  \lohead{\textsc{register}}
\fi

% theindex-Environment neu definieren ohne reledmac
\makeatletter
\renewenvironment{theindex}{%
  \ifkorrekturansicht
    \section*{\indexname}%
  \else
    \subsubsection*{Index der erwähnten Entitäten}%
  \fi
  \setlength{\parindent}{0pt}%
  \setlength{\parskip}{0pt plus 0.3pt}%
  \let\item\@idxitem
}{%
  \ifkorrekturansicht\clearpage\fi
}
\makeatother

\IfFileExists{\jobname-pw.ind}{\input{\jobname-pw.ind}}{}

% Quellenangabe nur in der Leseansicht
\ifkorrekturansicht\else
% Fallback-Definitionen, falls die .tex-Datei \titel etc. nicht gesetzt hat
\providecommand{\titel}{}
\providecommand{\editorInnen}{}
\providecommand{\dateiname}{\jobname}

\vspace{3cm}

\vfill

\footnotesize
\textsc{Quelle}: \titel. Herausgegeben von {\editorInnen}. In: \emph{Arthur Schnitzler: Briefwechsel mit Autorinnen und Autoren}.
 Digitale Edition, https://schnitzler-briefe.acdh.oeaw.ac.at/{\dateiname}.html (Stand \today)
\fi

\end{document}


      