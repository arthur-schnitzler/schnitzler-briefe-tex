\input{../tex-inputs/latex-pdf-vorspann}
\begin{center}
            \textcolor{red}{ENTWURF. ENTZIFFERUNG NOCH NICHT KORREKTURGELESEN}
                      \end{center}
            
               \section[Arthur Schnitzler an Hermann Bahr, 13. 11. 1898]{ Arthur Schnitzler an Hermann Bahr, 13. 11. 1898}\nopagebreak\mylabel{v}\rehead{ }\begin{ledgroupsized}[t]{13cm}\normalsize\beginnumbering\briefempfaengerindex{Bahr, Hermann@\textsc{Bahr, Hermann}!zzzSchnitzler, Arthur@\emph{von Arthur Schnitzler}!1898-11-131@{13. 11. 1898}|(be} \toendnotes[C]{\smallbreak\pagebreak[2]} \Standort{TMW, HS AM 23333 Ba.}
\physDesc{Brief, 1 Blatt, 1 Seite
\newline{}Handschrift: schwarze Tinte, deutsche Kurrent\newline{}Ordnung: Lochung }\buchAbdrucke{\weitereDrucke{1) \emph{14. 11. 1898.} In: Arthur Schnitzler: \emph{The Letters of Arthur Schnitzler to Hermann Bahr}. Edited, annotated, and with an introduction, by Donald G.
                        Daviau. Chapel Hill: \emph{The University of North Carolina Press} 1978, S. 64 (University of North Carolina studies in the Germanic languages
                        and literatures, 89).} \weitereDrucke{2) Hermann Bahr, Arthur Schnitzler: \emph{Briefwechsel, Aufzeichnungen, Dokumente (1891–1931)}. Hg. Kurt Ifkovits und Martin Anton Müller. Göttingen: \emph{Wallstein} 2018, S. 164.} }\toendnotes[C]{\smallbreak}\pstart{}{\pb}Lieber
                  Freund,\pend\pstart
           ich beglückwünſche dich von Herzen zu deinem großen \label{K_L00855_1v}\edtext{Erfolg}{\lemma{\textnormal{\emph{Erfolg}}}\Cendnote{\textnormal{Uraufführung von
                     \emph{Der Star}\pwindex{Bahr, Hermann 19.07.1863 – 15.01.1934@\textsc{Bahr, Hermann} (19.07.1863 – 15.01.1934), \emph{Schriftsteller, Kritiker}!Star1898@\strich\emph{Der Star} {[}1898{]}|pwk} am 12. 11. 1898 im \emph{Lessingtheater}\orgindex{Lessing-Theater@Lessing-Theater|pwk} in Berlin\oindex{Berlin@\textbf{Berlin}|pwk}.}}}\label{K_L00855_1h} in Berlin\oindex{Berlin@\textbf{Berlin}|pw},
                  \textcolor{gray}{und} grüße dich vielmals \pend
           \pstart dein \spacefill\mbox{Arthur Schnitzler}\pend{}\pstart
           Wien\oindex{Wien@\textbf{Wien}|pw}{ }13. 11. 98.\pend
           \endnumbering\briefempfaengerindex{Bahr, Hermann@\textsc{Bahr, Hermann}!zzzSchnitzler, Arthur@\emph{von Arthur Schnitzler}!1898-11-131@{13. 11. 1898}|)be}\mylabel{h}\end{ledgroupsized}  \newcommand{\dateiname}{L00855}\newcommand{\titel}{Arthur Schnitzler an Hermann Bahr, 13. 11. 1898}\newcommand{\editorInnen}{ Kurt Ifkovits,  Martin Anton Müller}\input{../tex-inputs/latex-pdf-abspann}
      