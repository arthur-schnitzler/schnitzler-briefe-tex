%% latex-leseansicht-vorspann.tex
%% Vorspann für die Leseansicht.
%% Lädt die gemeinsame Datei latex-vorspann.tex mit nicht gesetztem Schalter.

\newif\ifkorrekturansicht
\korrekturansichtfalse

\input{../tex-inputs/latex-vorspann}


\section[Arthur Schnitzler an Gustav Schwarzkopf, 7. 7. 1913]{L04222 Arthur Schnitzler an Gustav Schwarzkopf, 7. 7. 1913}
\nopagebreak\mylabel{L04222v}
\rehead{ }\normalsize\beginnumbering\briefempfaengerindex{Schwarzkopf, Gustav@\textsc{Schwarzkopf, Gustav}!zzzSchnitzler, Arthur@\emph{von Arthur Schnitzler}!1913-07-071@{7. 7. 1913}|(be}
\toendnotes[C]{\smallbreak\pagebreak[2]}
\correspDesc{Versand  durch Arthur Schnitzler am 7. 7. 1913 in Wien
\newline{}Erhalt  durch Gustav Schwarzkopf im Zeitraum [8. 7. 1913 – 12. 7. 1913?] in Opatija}\toendnotes[C]{\smallbreak}
\Standort{DLA, A:Schnitzler, HS.1985.1.1897.}
\physDesc{Bildpostkarte, 429 Zeichen
\newline{}Handschrift: schwarze Tinte, deutsche Kurrent
\newline{}Versand: Stempel: »\nobreak{}\oindex{Wien@\textbf{Wien}, \emph{Verwaltungsgebiet}|pwk}Wien 11, 07 VII 13, 1\nobreak{}«.  }\toendnotes[C]{\smallbreak}\pstart{}{\pb}Herrn \textsc{Gustav Schwarzkopf}\pend{}\pstart{}\textsc{Abbazia\oindex{Opatija@\textbf{Opatija}, \emph{Hauptstadt}|pw}}\pend{}\pstart{}\textsc{Wiener Heim\oindex{Pension Wiener Heim@\textbf{Pension Wiener Heim}, \emph{Beherbergungsgebäude}|pw}}\pend{}{\bigskip}
\pstart
           \noindent{}{\pb}{[}Sternwartestrasse 71\oindex{Wien@\textbf{Wien}!XVIII., Währing@\textbf{XVIII., Währing}!Sternwartestraße 71@\textbf{Sternwartestraße 71}, \emph{Wohngebäude}|pw}{]}\pend
           \vspace{1em}
\pstart
           \noindent{}{\pb}lieber Guſtav! Dank für die
               freundlichen \label{K_L04222-1v}\edtext{Nachrichten}{\lemma{\textnormal{\emph{Nachrichten}}}\Cendnote{\textnormal{{XXXX ref} XXXX 4. 7. 1913 }}}\label{K_L04222-1}. Bei
               uns wenig neues. Heini\pwindex{Schnitzler, Heinrich 9.\,8.\,1902 Hinterbrühl – 12.\,7.\,1982 Wien@\textsc{Schnitzler, Heinrich} (9.\,8.\,1902 Hinterbrühl – 12.\,7.\,1982 Wien), \emph{Regisseur, Schauspieler}|pw}{ }ſehr wohl; wir hoffen alſo
                  \label{K_L04222-2v}\edtext{vor dem 20.
                  abfahren}{\lemma{\textnormal{\emph{vor dem 20.
                  abfahren}}}\Cendnote{\textnormal{}}}\label{K_L04222-2} zu können, \label{K_L04222-3v}\edtext{zuerſt wohl Salzburg\oindex{Salzburg@\textbf{Salzburg}, \emph{Verwaltungsgebiet}|pw}}{\lemma{\textnormal{\emph{zuerst wohl Salzburg}}}\Cendnote{\textnormal{1913 hielt sich Schnitzler nicht
                        in Salzburg\oindex{Salzburg@\textbf{Salzburg}, \emph{Verwaltungsgebiet}|pwk} auf.}}}\label{K_L04222-3}, da{\geminationn}{ }\label{K_L04222-4v}\edtext{Brioni\oindex{Brijuni@\textbf{Brijuni}|pw}}{\lemma{\textnormal{\emph{Brioni}}}\Cendnote{\textnormal{Die Abreise verzögerte
                     sich auf den 24. 7. 1913, vgl. XXXX Auszeichnungsfehler: Dokument L04223 nicht gefunden.}}}\label{K_L04222-4}. Wir
               \label{K_L04222-5v}\edtext{planen nun für nachher eine Schiffsreiſe}{\lemma{\textnormal{\emph{planen … Schiffsreise}}}\Cendnote{\textnormal{Diese Reise fand erst
               im Folgejahr statt.}}}\label{K_L04222-5} mit dem Nordd.
                  Lloyd\orgindex{Norddeutscher Lloyd@Norddeutscher Lloyd|pw}{ }Genua\oindex{Genua@\textbf{Genua}|pw}–Hamburg\oindex{Hamburg@\textbf{Hamburg}|pw}. Aber das iſt ein weites Feld und ein weites
               Meer.\pend
           
\pstart
           Wir grüßen Sie herzlichſt{\\[\baselineskip]} Ihr \spacefill\mbox{Arthur}\pend
           \leftskip=0em{}
\pstart
           \noindent{}{\pb}Auch ein Wien\oindex{Wien@\textbf{Wien}, \emph{Verwaltungsgebiet}|pw}er
                  Heim. \textsc{XVIII.
                        Sternwartestr 71}\oindex{Wien@\textbf{Wien}!XVIII., Währing@\textbf{XVIII., Währing}!Sternwartestraße 71@\textbf{Sternwartestraße 71}, \emph{Wohngebäude}|pw}\pend
           \selectlanguage{ngerman}\endnumbering\briefempfaengerindex{Schwarzkopf, Gustav@\textsc{Schwarzkopf, Gustav}!zzzSchnitzler, Arthur@\emph{von Arthur Schnitzler}!1913-07-071@{7. 7. 1913}|)be}\mylabel{L04222h}
\begin{anhang}
\end{anhang}\newcommand{\dateiname}{L04222}\newcommand{\titel}{Arthur Schnitzler an Gustav Schwarzkopf, 7. 7. 1913}\newcommand{\editorInnen}{Herausgegeben von Jahnke, SelmaMüller, Martin Anton}%% latex-leseansicht-abspann.tex
%% Abspann für die Leseansicht.
%% Der Schalter \ifkorrekturansicht ist bereits durch den Vorspann gesetzt.

%% latex-abspann.tex
%% Gemeinsamer Abspann für Korrekturansicht und Leseansicht.
%% Setzt den Schalter \ifkorrekturansicht voraus (gesetzt in den
%% einbindenden Dateien latex-korrekturansicht-abspann.tex bzw.
%% latex-leseansicht-abspann.tex).
%% ---------------------------------------------------------------

\normalsize

% Das esempio-Environment wird nur in der Leseansicht benötigt
\ifkorrekturansicht\else
\newenvironment{esempio}[3]%
{
    \vspace{1.5ex}
    \rlap{\underline{#1}}
    \par
    \setlength{\parindent}{0cm}
    \nopagebreak
    \leftskip=#2cm
    \rightskip=#3cm
}
{
    \par
}
\fi

\doendnotes{C}
\bigskip
\vfill

\clearpage

\footnotesize

\ifkorrekturansicht
  \lohead{\textsc{register}}
\fi

% theindex-Environment neu definieren ohne reledmac
\makeatletter
\renewenvironment{theindex}{%
  \ifkorrekturansicht
    \section*{\indexname}%
  \else
    \subsubsection*{Index der erwähnten Entitäten}%
  \fi
  \setlength{\parindent}{0pt}%
  \setlength{\parskip}{0pt plus 0.3pt}%
  \let\item\@idxitem
}{%
  \ifkorrekturansicht\clearpage\fi
}
\makeatother

\IfFileExists{\jobname-pw.ind}{\input{\jobname-pw.ind}}{}

% Quellenangabe nur in der Leseansicht
\ifkorrekturansicht\else
% Fallback-Definitionen, falls die .tex-Datei \titel etc. nicht gesetzt hat
\providecommand{\titel}{}
\providecommand{\editorInnen}{}
\providecommand{\dateiname}{\jobname}

\vspace{3cm}

\vfill

\footnotesize
\textsc{Quelle}: \titel. Herausgegeben von {\editorInnen}. In: \emph{Arthur Schnitzler: Briefwechsel mit Autorinnen und Autoren}.
 Digitale Edition, https://schnitzler-briefe.acdh.oeaw.ac.at/{\dateiname}.html (Stand \today)
\fi

\end{document}


