%% latex-leseansicht-vorspann.tex
%% Vorspann für die Leseansicht.
%% Lädt die gemeinsame Datei latex-vorspann.tex mit nicht gesetztem Schalter.

\newif\ifkorrekturansicht
\korrekturansichtfalse

\input{../tex-inputs/latex-vorspann}


\section[ Felix Salten an Arthur Schnitzler, 28. 4. 1899]{L03288 Felix Salten an Arthur Schnitzler,  28. 4. 1899}
\nopagebreak\mylabel{L03288v}
\rehead{ }\normalsize\beginnumbering\briefempfaengerindex{Schnitzler, Arthur@\textsc{Schnitzler, Arthur}!zzzSalten, Felix@\emph{von Felix Salten}!1899-04-283@{28. 4. 1899}|(be}
\toendnotes[C]{\smallbreak\pagebreak[2]}
\correspDesc{Versand  durch Felix Salten am 28. 4. 1899 in Wien
\newline{}Erhalt  durch Arthur Schnitzler im Zeitraum [29. 4. 1899
                  – 3. 5. 1899?] in Berlin?}\toendnotes[C]{\smallbreak}
\Standort{CUL, Schnitzler, B 89, A 2.}
\physDesc{Brief, 1 Blatt, 1 Seite, 697 Zeichen
\newline{}Handschrift: schwarze Tinte, lateinische Kurrent
\newline{}Ordnung: mit Bleistift von unbekannter Hand nummeriert: »112« }\toendnotes[C]{\smallbreak}
\pstart
           \raggedleft{}{\pb}Hietzing\oindex{XIII., Hietzing@\textbf{XIII., Hietzing}, \emph{Verwaltungsgebiet}|pw}, 28. April 99\pend
           \vspace{0.5em}
\pstart
           Lieber Freund, leider war ich in den letzten Tagen wieder durch
               vielerlei ernste Angelegenheiten so gehetzt, dass ich nicht zu Ihnen konnte. Auch
               meine Berlin\oindex{Berlin@\textbf{Berlin}, \emph{Hauptstadt}|pw}er Reise, die ich so gerne gemacht
               hätte, musste unterbleiben, weil die \label{K_L03288-1v}\edtext{Geschichte mit Otti\pwindex{Salten, Ottilie 7.\,3.\,1868 Prag – 22.\,6.\,1942 Zürich@\textsc{Salten, Ottilie} (7.\,3.\,1868 Prag – 22.\,6.\,1942 Zürich), \emph{Schauspielerin}|pw}}{\lemma{\textnormal{\emph{Geschichte mit Otti}}}\Cendnote{\textnormal{Paul
                     Schlenther\pwindex{Schlenther, Paul 20.\,8.\,1854 Chernyakhovsk – 30.\,4.\,1916 Berlin@\textsc{Schlenther, Paul} (20.\,8.\,1854 Chernyakhovsk – 30.\,4.\,1916 Berlin), \emph{Schriftsteller, Kritiker, Theaterleiter}|pwk} hatte ihr am 25. 2. 1899 mündlich
                  mitgeteilt, dass der bestehende Vertrag mit dem \emph{Burgtheater}\orgindex{Burgtheater@Burgtheater|pwk} mit Ende der Theatersaison ablaufe und nicht weiter
                  verlängert werden würde. Trotz verschiedener Proteste – Schnitzler schrieb dem Direktor am 15. 6. 1899 einen
                  Brief – blieb es dabei. }}}\label{K_L03288-1} noch immer zu keinem Abschluß gekommen ist. Sie
               leidet entsetzlich unter der großen wie unter den vielen kleinen Gemeinheiten, welche
               ihr angethan werden. Hirschfeld\pwindex{Hirschfeld, Georg 11.\,2.\,1873 Berlin – 17.\,1.\,1942 München@\textsc{Hirschfeld, Georg} (11.\,2.\,1873 Berlin – 17.\,1.\,1942 München), \emph{Schriftsteller}|pw} ist, wie Sie
               wissen werden, in Hietzing\oindex{XIII., Hietzing@\textbf{XIII., Hietzing}, \emph{Verwaltungsgebiet}|pw} und wohnt gleich neben
               mir. Sonst sehe ich Niemanden. Bitte, vielleicht schreiben Sie mir: wie es Ihnen
               geht, und wie Ihre \label{K_L03288-2v}\edtext{Prémière\pwindex{Schnitzler, Arthur 15.\,5.\,1862 Wien – 21.\,10.\,1931 ebd.@\textsc{Schnitzler, Arthur} (15.\,5.\,1862 Wien – 21.\,10.\,1931 ebd.), \emph{Schriftsteller, Mediziner}!grüne Kakadu – Paracelsus – Die Gefährtin. Drei Einakter@\strich\emph{Der grüne Kakadu – Paracelsus – Die Gefährtin. Drei Einakter}|pwv}}{\lemma{\textnormal{\emph{Prémière}}}\Cendnote{\textnormal{Schnitzler weilte in Berlin\oindex{Berlin@\textbf{Berlin}, \emph{Hauptstadt}|pwk}, um bei den Proben für die Premiere seines
                  Einakterzyklus’ \emph{Der grüne Kakadu – Paracelsus –
                     Die Gefährtin}\pwindex{Schnitzler, Arthur 15.\,5.\,1862 Wien – 21.\,10.\,1931 ebd.@\textsc{Schnitzler, Arthur} (15.\,5.\,1862 Wien – 21.\,10.\,1931 ebd.), \emph{Schriftsteller, Mediziner}!grüne Kakadu – Paracelsus – Die Gefährtin. Drei Einakter@\strich\emph{Der grüne Kakadu – Paracelsus – Die Gefährtin. Drei Einakter}|pwk} am 29. 4. 1899 am \emph{Deutschen Theater}\orgindex{Deutsches Theater Berlin@Deutsches Theater Berlin|pwk}
                  teilzunehmen. Er kehrte am 3. 5. 1899 nach Wien\oindex{Wien@\textbf{Wien}, \emph{Verwaltungsgebiet}|pwk} zurück und
                  sah Salten\pwindex{Salten, Felix 6.\,9.\,1869 Budapest – 8.\,10.\,1945 Zürich@\textsc{Salten, Felix} (6.\,9.\,1869 Budapest – 8.\,10.\,1945 Zürich), \emph{Schriftsteller, Journalist, Chefredakteur}|pwk} nachweislich am 11. 5. 1899
                  wieder.}}}\label{K_L03288-2} ausgefallen ist, wann Sie wiederkommen, und wann wir uns sehen.\pend
           
\pstart
           Sehr herzlich Ihr treuer {\\[\baselineskip]}\spacefill\mbox{Felix Salten}\pend
           \leftskip=0em{}\selectlanguage{ngerman}\endnumbering\briefempfaengerindex{Schnitzler, Arthur@\textsc{Schnitzler, Arthur}!zzzSalten, Felix@\emph{von Felix Salten}!1899-04-283@{28. 4. 1899}|)be}\mylabel{L03288h}  \newcommand{\dateiname}{L03288}\newcommand{\titel}{Felix Salten an Arthur Schnitzler, 28. 4. 1899}\newcommand{\editorInnen}{Martin Anton Müller und Laura Untner}%% latex-leseansicht-abspann.tex
%% Abspann für die Leseansicht.
%% Der Schalter \ifkorrekturansicht ist bereits durch den Vorspann gesetzt.

%% latex-abspann.tex
%% Gemeinsamer Abspann für Korrekturansicht und Leseansicht.
%% Setzt den Schalter \ifkorrekturansicht voraus (gesetzt in den
%% einbindenden Dateien latex-korrekturansicht-abspann.tex bzw.
%% latex-leseansicht-abspann.tex).
%% ---------------------------------------------------------------

\normalsize

% Das esempio-Environment wird nur in der Leseansicht benötigt
\ifkorrekturansicht\else
\newenvironment{esempio}[3]%
{
    \vspace{1.5ex}
    \rlap{\underline{#1}}
    \par
    \setlength{\parindent}{0cm}
    \nopagebreak
    \leftskip=#2cm
    \rightskip=#3cm
}
{
    \par
}
\fi

\doendnotes{C}
\bigskip
\vfill

\clearpage

\footnotesize

\ifkorrekturansicht
  \lohead{\textsc{register}}
\fi

% theindex-Environment neu definieren ohne reledmac
\makeatletter
\renewenvironment{theindex}{%
  \ifkorrekturansicht
    \section*{\indexname}%
  \else
    \subsubsection*{Index der erwähnten Entitäten}%
  \fi
  \setlength{\parindent}{0pt}%
  \setlength{\parskip}{0pt plus 0.3pt}%
  \let\item\@idxitem
}{%
  \ifkorrekturansicht\clearpage\fi
}
\makeatother

\IfFileExists{\jobname-pw.ind}{\input{\jobname-pw.ind}}{}

% Quellenangabe nur in der Leseansicht
\ifkorrekturansicht\else
% Fallback-Definitionen, falls die .tex-Datei \titel etc. nicht gesetzt hat
\providecommand{\titel}{}
\providecommand{\editorInnen}{}
\providecommand{\dateiname}{\jobname}

\vspace{3cm}

\vfill

\footnotesize
\textsc{Quelle}: \titel. Herausgegeben von {\editorInnen}. In: \emph{Arthur Schnitzler: Briefwechsel mit Autorinnen und Autoren}.
 Digitale Edition, https://schnitzler-briefe.acdh.oeaw.ac.at/{\dateiname}.html (Stand \today)
\fi

\end{document}


