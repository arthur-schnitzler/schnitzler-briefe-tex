%% latex-korrekturansicht-vorspann.tex
%% Vorspann für die Korrekturansicht.
%% Lädt die gemeinsame Datei latex-vorspann.tex mit gesetztem Schalter.

\newif\ifkorrekturansicht
\korrekturansichttrue

\input{../tex-inputs/latex-vorspann}


\section[ Felix Salten an Arthur Schnitzler, 28. 4. 1899]{L03288 Felix Salten an Arthur Schnitzler, 28. 4. 1899}
\nopagebreak\mylabel{L03288v}
\rehead{ }\normalsize\beginnumbering\briefempfaengerindex{Schnitzler, Arthur@\textsc{Schnitzler, Arthur}!zzzSalten, Felix@\emph{von Felix Salten}!1899-04-282@{28. 4. 1899}|(be}
\toendnotes[C]{\smallbreak\pagebreak[2]}\Standort{CUL, Schnitzler, B 89, A 2.}
\physDesc{Brief, 1 Blatt, 1 Seite, 697 Zeichen
\newline{}Handschrift: schwarze Tinte, lateinische Kurrent
\newline{}Ordnung: mit Bleistift von unbekannter Hand nummeriert: »112« }\toendnotes[C]{\smallbreak}
\pstart
           \raggedleft{}{\pb}Hietzing\oindex{XIII., Hietzing@\textbf{XIII., Hietzing}, \emph{A.ADM3}|pw}, 28. April 99\pend
           \vspace{0.5em}
\pstart
           Lieber Freund, leider war ich in den letzten Tagen wieder durch
               vielerlei ernste Angelegenheiten so gehetzt, dass ich nicht zu Ihnen konnte. Auch
               meine Berlin\oindex{Berlin@\textbf{Berlin}, \emph{P.PPLC}|pw}er Reise, die ich so gerne gemacht
               hätte, musste unterbleiben, weil die \label{K_L03288-1v}\edtext{Geschichte mit Otti\pwindex{Salten, Ottilie 07.03.1868 – 22.06.1942@\textsc{Salten, Ottilie} (07.03.1868 – 22.06.1942), \emph{Schauspieler/Schauspielerin}|pw}}{\lemma{\textnormal{\emph{Geschichte mit Otti}}}\Cendnote{\textnormal{Paul
                     Schlenther\pwindex{Schlenther, Paul 20.08.1854 – 30.04.1916@\textsc{Schlenther, Paul} (20.08.1854 – 30.04.1916), \emph{Schriftsteller/Schriftstellerin, Kritiker/Kritikerin, Theaterleiter/Theaterleiterin}|pwk} hatte ihr am 25. 2. 1899 mündlich
                  mitgeteilt, dass der bestehende Vertrag mit dem \emph{Burgtheater}\orgindex{Burgtheater@Burgtheater|pwk} mit Ende der Theatersaison ablaufe und nicht weiter
                  verlängert werden würde. Trotz verschiedener Proteste – Schnitzler schrieb dem Direktor am 15. 6. 1899 einen
                  Brief – blieb es dabei. }}}\label{K_L03288-1} noch immer zu keinem Abschluß gekommen ist. Sie
               leidet entsetzlich unter der großen wie unter den vielen kleinen Gemeinheiten, welche
               ihr angethan werden. Hirschfeld\pwindex{Hirschfeld, Georg 11.02.1873 – 17.01.1942@\textsc{Hirschfeld, Georg} (11.02.1873 – 17.01.1942), \emph{Schriftsteller/Schriftstellerin}|pw} ist, wie Sie
               wissen werden, in Hietzing\oindex{XIII., Hietzing@\textbf{XIII., Hietzing}, \emph{A.ADM3}|pw} und wohnt gleich neben
               mir. Sonst sehe ich Niemanden. Bitte, vielleicht schreiben Sie mir: wie es Ihnen
               geht, und wie Ihre \label{K_L03288-2v}\edtext{Prémière\pwindex{gruene Kakadu – Paracelsus – Die Gefaehrtin. Drei Einakter@\emph{Der grüne Kakadu – Paracelsus – Die Gefährtin. Drei Einakter}|pwv}}{\lemma{\textnormal{\emph{Prémière}}}\Cendnote{\textnormal{Schnitzler weilte in Berlin\oindex{Berlin@\textbf{Berlin}, \emph{P.PPLC}|pwk}, um bei den Proben für die Premiere seines
                  Einakterzyklus’ \emph{Der grüne Kakadu – Paracelsus –
                     Die Gefährtin}\pwindex{gruene Kakadu – Paracelsus – Die Gefaehrtin. Drei Einakter@\emph{Der grüne Kakadu – Paracelsus – Die Gefährtin. Drei Einakter}|pwk} am 29. 4. 1899 am \emph{Deutschen Theater}\orgindex{Deutsches Theater Berlin@Deutsches Theater Berlin|pwk}
                  teilzunehmen. Er kehrte am 3. 5. 1899 nach Wien\oindex{Wien@\textbf{Wien}, \emph{A.ADM2}|pwk} zurück und
                  sah Salten\pwindex{Salten, Felix 06.09.1869 – 08.10.1945@\textsc{Salten, Felix} (06.09.1869 – 08.10.1945), \emph{Schriftsteller/Schriftstellerin, Journalist/Journalistin, Chefredakteur/Chefredakteurin}|pwk} nachweislich am 11. 5. 1899
                  wieder.}}}\label{K_L03288-2} ausgefallen ist, wann Sie wiederkommen, und wann wir uns sehen.\pend
           
\pstart
           Sehr herzlich Ihr treuer {\\[\baselineskip]}\spacefill\mbox{Felix Salten}\pend
           \leftskip=0em{}\selectlanguage{ngerman}\endnumbering\briefempfaengerindex{Schnitzler, Arthur@\textsc{Schnitzler, Arthur}!zzzSalten, Felix@\emph{von Felix Salten}!1899-04-282@{28. 4. 1899}|)be}\mylabel{L03288h}  \normalsize

\doendnotes{C}
\bigskip
\vfill

\clearpage

\footnotesize

\lohead{\textsc{register}}

% Definiere theindex-Environment komplett neu ohne reledmac
\makeatletter
\renewenvironment{theindex}{%
  \section*{\indexname}%
  \setlength{\parindent}{0pt}%
  \setlength{\parskip}{0pt plus 0.3pt}%
  \let\item\@idxitem
}{%
  \clearpage
}
\makeatother

\IfFileExists{\jobname-pw.ind}{\input{\jobname-pw.ind}}{}

\end{document}

      