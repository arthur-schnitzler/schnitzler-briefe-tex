\input{../tex-inputs/latex-pdf-vorspann}
\begin{center}
            \textcolor{red}{ENTWURF. ENTZIFFERUNG NOCH NICHT KORREKTURGELESEN}
                      \end{center}
            
               \section[Arthur und Olga Schnitzler an Hermann Bahr, 6. 5. 1908]{ Arthur und Olga Schnitzler an Hermann Bahr, 6. 5. 1908}\nopagebreak\mylabel{v}\rehead{ }\begin{ledgroupsized}[t]{13cm}\normalsize\beginnumbering\briefempfaengerindex{Bahr, Hermann@\textsc{Bahr, Hermann}!zzzSchnitzler, Olga@\emph{von Olga Schnitzler}!1908-05-061@{6. 5. 1908}|(be}\briefempfaengerindex{Bahr, Hermann@\textsc{Bahr, Hermann}!zzzSchnitzler, Arthur@\emph{von Arthur Schnitzler}!1908-05-061@{6. 5. 1908}|(be} \toendnotes[C]{\smallbreak\pagebreak[2]} \Standort{TMW, HS AM 60147 Ba.}
\physDesc{Bildpostkarte
\newline{}Handschrift Arthur Schnitzler: Bleistift, deutsche Kurrent\newline{}Handschrift Olga Schnitzler: Bleistift\newline{}Versand: Stempel: »\nobreak{}\oindex{Muenchen@\textbf{München}|pwk}München, 6. Mai 08, 5–6 N\nobreak{}«.  \newline{}Ordnung: Lochung \newline{}Zusatz: Postkartenmotiv von Heinrich
                                    Kley\pwindex{Kley, Heinrich 1863-04-15 – 1945-02-08@\textsc{Kley, Heinrich} (1863-04-15 – 1945-02-08), \emph{Maler, Grafiker}|pw} }\buchAbdrucke{\weitereDrucke{1) \emph{6. 5. 1908, Abschrift.} In: Arthur Schnitzler: \emph{The Letters of Arthur Schnitzler to Hermann Bahr}. Edited, annotated, and with an introduction, by Donald G.
                        Daviau. Chapel Hill: \emph{The University of North Carolina Press} 1978, S. 102 (University of North Carolina studies in the Germanic languages
                        and literatures, 89).} \weitereDrucke{2) Hermann Bahr, Arthur Schnitzler: \emph{Briefwechsel, Aufzeichnungen, Dokumente (1891–1931)}. Hg. Kurt Ifkovits und Martin Anton Müller. Göttingen: \emph{Wallstein} 2018, S. 403.} }\toendnotes[C]{\smallbreak}\pstart{}{\pb}\textsc{Herrn Hermann Bahr}\pend{}\pstart{}\textsc{Wien Ob St Veit}\oindex{Ober Sankt Veit@\textbf{Ober Sankt Veit}|pw}\pend{}\pstart{}\textsc{Veitlissengasse\oindex{Veitlissengasse@\textbf{Veitlissengasse}|pw}}\pend{}{\bigskip}\pstart
           \noindent{}\centering{}\textcolor{gray}{\textbf{{\pb}Im Hofgarten München\oindex{Hofgarten@\textbf{Hofgarten}|pw}}}\pend
           \pstart
           6/5 908\pend
           \pstart
           herzliche Grüße, in \label{K_L01770_1v}\edtext{Erinnerung an
                  1894}{\lemma{\textnormal{\emph{Erinnerung an
                  1894}}}\Cendnote{\textnormal{an den gemeinsamen Aufenthalt im
                     Juni 1894}}}\label{K_L01770_1h}\pend
           \pstart
           Dein{\\[\baselineskip]}\spacefill\mbox{Arthur}\pend
           \leftskip=0em{}\pstart \spacefill\mbox{{[}hs. O. Schnitzler:{]} OlgaSchnitzler}\pend{}\endnumbering\briefempfaengerindex{Bahr, Hermann@\textsc{Bahr, Hermann}!zzzSchnitzler, Olga@\emph{von Olga Schnitzler}!1908-05-061@{6. 5. 1908}|)be}\briefempfaengerindex{Bahr, Hermann@\textsc{Bahr, Hermann}!zzzSchnitzler, Arthur@\emph{von Arthur Schnitzler}!1908-05-061@{6. 5. 1908}|)be}\mylabel{h}\end{ledgroupsized}  \newcommand{\dateiname}{L01770}\newcommand{\titel}{Arthur und Olga Schnitzler an Hermann Bahr, 6. 5. 1908}\newcommand{\editorInnen}{ Kurt Ifkovits,  Martin Anton Müller}\input{../tex-inputs/latex-pdf-abspann}
      