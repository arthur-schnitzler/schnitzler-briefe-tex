%% latex-korrekturansicht-vorspann.tex
%% Vorspann für die Korrekturansicht.
%% Lädt die gemeinsame Datei latex-vorspann.tex mit gesetztem Schalter.

\newif\ifkorrekturansicht
\korrekturansichttrue

\input{../tex-inputs/latex-vorspann}


\section[Arthur und Olga Schnitzler an Hermann Bahr, 6. 5. 1908]{L01770 Arthur und Olga Schnitzler an Hermann Bahr, 6. 5. 1908}
\nopagebreak\mylabel{L01770v}
\rehead{ }\normalsize\beginnumbering\briefempfaengerindex{Bahr, Hermann@\textsc{Bahr, Hermann}!zzzSchnitzler, Olga@\emph{von Olga Schnitzler}!1908-05-061@{6. 5. 1908}|(be}\briefempfaengerindex{Bahr, Hermann@\textsc{Bahr, Hermann}!zzzSchnitzler, Arthur@\emph{von Arthur Schnitzler}!1908-05-061@{6. 5. 1908}|(be}
\toendnotes[C]{\smallbreak\pagebreak[2]}\Standort{TMW, HS AM 60147 Ba.}
\physDesc{Bildpostkarte, 115 Zeichen
\newline{}Handschrift Arthur Schnitzler: 1) Bleistift, deutsche Kurrent\hspace{1em}2) Bleistift, lateinische Kurrent (\noindent{}Adresse)\hspace{1em}
\newline{}Handschrift Olga Schnitzler: Bleistift
\newline{}Versand: Stempel: »\nobreak{}\oindex{Muenchen@\textbf{München}, \emph{P.PPLA}|pwk}München, 6. Mai 08, 5–6 N\nobreak{}«.  
\newline{}Ordnung: Lochung 
\newline{}Zusatz: Postkartenmotiv von Heinrich
                                    Kley\pwindex{Kley, Heinrich 1863-04-15 – 1945-02-08@\textsc{Kley, Heinrich} (1863-04-15 – 1945-02-08), \emph{Maler/Malerin, Grafiker/Grafikerin}|pw} }
\buchAbdrucke{\weitereDrucke{1) Arthur Schnitzler: \emph{The Letters of Arthur Schnitzler to Hermann Bahr}. Chapel Hill: \emph{The University of North Carolina Press} 1978, S. 102.} \weitereDrucke{2) Hermann Bahr, Arthur Schnitzler: \emph{Briefwechsel, Aufzeichnungen, Dokumente (1891–1931)}. Göttingen: \emph{Wallstein} 2018, S. 403.} }\toendnotes[C]{\smallbreak}\pstart{}{\pb}Herrn Hermann
                  Bahr\pend{}\pstart{}Wien Ob St Veit\oindex{Ober Sankt Veit@\textbf{Ober Sankt Veit}, \emph{P.PPLX}|pw}\pend{}\pstart{}Veitlissengasse\oindex{Veitlissengasse@\textbf{Veitlissengasse}, \emph{Straße (K.STR)}|pw}\pend{}{\bigskip}
\pstart
           \noindent{}\centering{}{\pb}\textcolor{gray}{\textbf{Im Hofgarten München\oindex{Hofgarten [Muenchen]@\textbf{Hofgarten [München]}, \emph{Park (K.PRK)}|pw}}}\pend
           \vspace{1em}
\pstart
           {\pb}6/5 908\pend
           \vspace{0.5em}
\pstart
           herzliche Grüße, in \label{K_L01770-1v}\edtext{Erinnerung an
                  1894}{\lemma{\textnormal{\emph{Erinnerung an
                  1894}}}\Cendnote{\textnormal{an den gemeinsamen Aufenthalt im
                     Juni 1894}}}\label{K_L01770-1}\pend
           
\pstart
           Dein{\\[\baselineskip]}\spacefill\mbox{Arthur}\pend
           \leftskip=0em{}\selectlanguage{ngerman}\vspace{1em}\pstart \spacefill\mbox{{[}hs. :{]} OlgaSchnitzler}\pend{}\selectlanguage{ngerman}\endnumbering\briefempfaengerindex{Bahr, Hermann@\textsc{Bahr, Hermann}!zzzSchnitzler, Olga@\emph{von Olga Schnitzler}!1908-05-061@{6. 5. 1908}|)be}\briefempfaengerindex{Bahr, Hermann@\textsc{Bahr, Hermann}!zzzSchnitzler, Arthur@\emph{von Arthur Schnitzler}!1908-05-061@{6. 5. 1908}|)be}\mylabel{L01770h}  \normalsize

\doendnotes{C}
\bigskip
\vfill

\clearpage

\footnotesize

\lohead{\textsc{register}}

% Definiere theindex-Environment komplett neu ohne reledmac
\makeatletter
\renewenvironment{theindex}{%
  \section*{\indexname}%
  \setlength{\parindent}{0pt}%
  \setlength{\parskip}{0pt plus 0.3pt}%
  \let\item\@idxitem
}{%
  \clearpage
}
\makeatother

\IfFileExists{\jobname-pw.ind}{\input{\jobname-pw.ind}}{}

\end{document}

      