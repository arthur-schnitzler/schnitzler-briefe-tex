%% latex-leseansicht-vorspann.tex
%% Vorspann für die Leseansicht.
%% Lädt die gemeinsame Datei latex-vorspann.tex mit nicht gesetztem Schalter.

\newif\ifkorrekturansicht
\korrekturansichtfalse

\input{../tex-inputs/latex-vorspann}


\section[Lou Andreas-Salomé an Arthur Schnitzler, {{[}}2. 5.? 1895{{]}}]{L00436 Lou Andreas-Salomé an Arthur Schnitzler, {[}2. 5.? 1895{]}}
\nopagebreak\mylabel{L00436v}
\rehead{ }\normalsize\beginnumbering\briefempfaengerindex{Schnitzler, Arthur@\textsc{Schnitzler, Arthur}!zzzAndreas-Salomé, Lou@\emph{von Lou Andreas-Salomé}!1895-05-021@{{[}2. 5.? 1895{]}}|(be}
\toendnotes[C]{\smallbreak\pagebreak[2]}
\correspDesc{Versand  durch Lou Andreas-Salomé am [2. 5.? 1895] in Wien
\newline{}Erhalt  durch Arthur Schnitzler im Zeitraum [2. 5. 1895
                  – 6. 5. 1895?] in Wien}\toendnotes[C]{\smallbreak}
\Standort{CUL, Schnitzler, B 3.}
\physDesc{Brief, 1 Blatt, 1 Seite, 390 Zeichen
\newline{}Handschrift: schwarze Tinte, deutsche Kurrent
\newline{}Schnitzler: mit Bleistift datiert: »April 95« 
\newline{}Ordnung: mit rotem Buntstift von unbekannter Hand doppelt nummeriert: »\strikeout{2}3« }\toendnotes[C]{\smallbreak}
\pstart
           {\pb}\textsc{Hotel Royal}\oindex{Wien@\textbf{Wien}!I., Innere Stadt@\textbf{I., Innere Stadt}!Hotel Royal@\textbf{Hotel Royal}, \emph{Hotel}|pw}{\\}\label{K_L00436-1v}\edtext{Donnerſtag}{\lemma{\textnormal{\emph{Donnerstag}}}\Cendnote{\textnormal{Der Brief vom 28. 4. 1895
                     schließt es aus, dass die Datierung Schnitzlers zutrifft und dieser Brief an einem Donnerstag im
                        April 1895 verfasst wurde. Datiert man ihn auf den ersten
                     Donnerstag im Mai, fügen sich die Objekte gut zusammen, vertauscht sich aber
                     die Reihenfolge der nummerierten Objekte »4« und »3«.}}}\label{K_L00436-1}.\pend
           
\pstart{}Sehr geehrter Herr \textsc{D\textsuperscript{r}},\pend\vspace{0.5em}
\pstart
           morgen bin ich um 6 Uhr noch nicht zu Hauſe, aber dafür in Ihrer
               nächſten Nähe, nämlich in der Univerſität\oindex{Wien@\textbf{Wien}!I., Innere Stadt@\textbf{I., Innere Stadt}!Universität Wien@\textbf{Universität Wien}, \emph{Universität}|pw}
               (Hörſaal N\textsuperscript{o} 35) Wäre es nicht am einfachſten, Sie holten
               mich dort ab und wir{ }ſuchten uns von dort einen Plauderwinkel? Für den Fall, daß
               Ihnen das nicht paßt, treffen Sie mich gegen 7 Uhr in meiner
               Hôtelwohnung.\pend
           
\pstart
           Mit beſtem Gruß{\\[\baselineskip]}\spacefill\mbox{Lou Andreas-Salomé.}\pend
           \leftskip=0em{}\selectlanguage{ngerman}\endnumbering\briefempfaengerindex{Schnitzler, Arthur@\textsc{Schnitzler, Arthur}!zzzAndreas-Salomé, Lou@\emph{von Lou Andreas-Salomé}!1895-05-021@{{[}2. 5.? 1895{]}}|)be}\mylabel{L00436h}  \newcommand{\dateiname}{L00436}\newcommand{\titel}{Lou Andreas-Salomé an Arthur Schnitzler, [2. 5.? 1895]}\newcommand{\editorInnen}{Martin Anton Müller und Gerd-Hermann Susen}%% latex-leseansicht-abspann.tex
%% Abspann für die Leseansicht.
%% Der Schalter \ifkorrekturansicht ist bereits durch den Vorspann gesetzt.

%% latex-abspann.tex
%% Gemeinsamer Abspann für Korrekturansicht und Leseansicht.
%% Setzt den Schalter \ifkorrekturansicht voraus (gesetzt in den
%% einbindenden Dateien latex-korrekturansicht-abspann.tex bzw.
%% latex-leseansicht-abspann.tex).
%% ---------------------------------------------------------------

\normalsize

% Das esempio-Environment wird nur in der Leseansicht benötigt
\ifkorrekturansicht\else
\newenvironment{esempio}[3]%
{
    \vspace{1.5ex}
    \rlap{\underline{#1}}
    \par
    \setlength{\parindent}{0cm}
    \nopagebreak
    \leftskip=#2cm
    \rightskip=#3cm
}
{
    \par
}
\fi

\doendnotes{C}
\bigskip
\vfill

\clearpage

\footnotesize

\ifkorrekturansicht
  \lohead{\textsc{register}}
\fi

% theindex-Environment neu definieren ohne reledmac
\makeatletter
\renewenvironment{theindex}{%
  \ifkorrekturansicht
    \section*{\indexname}%
  \else
    \subsubsection*{Index der erwähnten Entitäten}%
  \fi
  \setlength{\parindent}{0pt}%
  \setlength{\parskip}{0pt plus 0.3pt}%
  \let\item\@idxitem
}{%
  \ifkorrekturansicht\clearpage\fi
}
\makeatother

\IfFileExists{\jobname-pw.ind}{\input{\jobname-pw.ind}}{}

% Quellenangabe nur in der Leseansicht
\ifkorrekturansicht\else
% Fallback-Definitionen, falls die .tex-Datei \titel etc. nicht gesetzt hat
\providecommand{\titel}{}
\providecommand{\editorInnen}{}
\providecommand{\dateiname}{\jobname}

\vspace{3cm}

\vfill

\footnotesize
\textsc{Quelle}: \titel. Herausgegeben von {\editorInnen}. In: \emph{Arthur Schnitzler: Briefwechsel mit Autorinnen und Autoren}.
 Digitale Edition, https://schnitzler-briefe.acdh.oeaw.ac.at/{\dateiname}.html (Stand \today)
\fi

\end{document}


