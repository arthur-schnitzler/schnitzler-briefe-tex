%% latex-leseansicht-vorspann.tex
%% Vorspann für die Leseansicht.
%% Lädt die gemeinsame Datei latex-vorspann.tex mit nicht gesetztem Schalter.

\newif\ifkorrekturansicht
\korrekturansichtfalse

\input{../tex-inputs/latex-vorspann}


\section[Stefan Großmann an Arthur Schnitzler, 2. 1. 192{[}6?{]}]{L02464 Stefan Großmann an Arthur Schnitzler, 2. 1. 192[6?]}
\nopagebreak\mylabel{L02464v}
\rehead{ }\normalsize\beginnumbering\briefempfaengerindex{Schnitzler, Arthur@\textsc{Schnitzler, Arthur}!zzzGroßmann, Stefan@\emph{von Stefan Großmann}!1926-01-022@{2. 1. 192[6?]}|(be}
\toendnotes[C]{\smallbreak\pagebreak[2]}
\correspDesc{Versand  durch Stefan Großmann am 2. 1. 192[6?] in Berlin
\newline{}Erhalt  durch Arthur Schnitzler im Zeitraum [3. 1. 1926
                  – 7. 1. 1926?] in Wien}\toendnotes[C]{\smallbreak}
\Standort{DLA, A:Schnitzler, HS.NZ85.1.3232.}
\physDesc{Brief, 1 Blatt, 1 Seite, 472 Zeichen
\newline{}Schreibmaschine
\newline{}Handschrift: blaue Tinte (\noindent{}Unterschrift)
\newline{}Schnitzler: mit rotem Buntstift beschriftet: »\textsc{Großma{\geminationn}}« }\toendnotes[C]{\smallbreak}
\pstart
           \centering{}{\pb}\textcolor{gray}{\textbf{Das Tage-Buch\orgindex{Tage-Buch@Das Tage-Buch|pw}}}\pend
           
\pstart
           \centering{}\textcolor{gray}{\textbf{\emph{Herausgeber: Stefan Großmann und Leopold Schwarzschild\pwindex{Schwarzschild, Leopold 8.\,12.\,1891 Frankfurt am Main – 2.\,10.\,1950 Santa Margherita Ligure@\textsc{Schwarzschild, Leopold} (8.\,12.\,1891 Frankfurt am Main – 2.\,10.\,1950 Santa Margherita Ligure), \emph{Publizist}|pw}}}}\pend
           
\pstart
           \centering{}\textcolor{gray}{\textbf{Tagebuchverlag m. b. H., Berlin SW 19\oindex{Berlin@\textbf{Berlin}, \emph{Hauptstadt}|pw}}}\pend
           
\pstart
           \centering{}\textcolor{gray}{\textbf{BEUTHSTRASSE 19\oindex{Beuthstrasse@\textbf{Beuthstrasse}, \emph{Straße}|pw}}}\pend
           
\pstart
           \centering{}\textcolor{gray}{\textbf{\emph{Telegramm-Adresse: Tagebuch Berlin\oindex{Berlin@\textbf{Berlin}, \emph{Hauptstadt}|pw} ⋅ Fernsprecher: Merkur 8790–8792}}}\pend
           
\pstart
           \centering{}\textcolor{gray}{\textbf{\emph{\so{Sprechstunde der Redaktion: 12–1 Uhr}}}}\pend
           
\pstart
           \centering{}\textcolor{gray}{\textbf{*}}\pend
           
\pstart
           Tgb./Gr./Schl.\hfill Berlin\oindex{Berlin@\textbf{Berlin}, \emph{Hauptstadt}|pw}, den \label{K_L02464-1v}\edtext{2. Januar 1925}{\lemma{\textnormal{\emph{2. Januar 1925}}}\Cendnote{\textnormal{Es dürfte sich bei der Jahreszahl
                        um einen Irrtum handeln, da andernfalls mehrere Korrespondenzstücke als
                        verlustig zu gelten hätten.}}}\label{K_L02464-1}.\pend
           {\vspace{1\baselineskip}}
\pstart
           \raggedleft{}Herrn\pend
           
\pstart
           \raggedleft{}Dr. Arthur \so{Schnitzler}\pend
           
\pstart
           \raggedleft{}\so{Wien XVIII}\oindex{XVIII., Währing@\textbf{XVIII., Währing}, \emph{Verwaltungsgebiet}|pw}\pend
           
\pstart
           \raggedleft{}Sternwartestr. 71\oindex{Wien@\textbf{Wien}!XVIII., Währing@\textbf{XVIII., Währing}!Sternwartestraße 71@\textbf{Sternwartestraße 71}, \emph{Wohngebäude}|pw}.\pend
           
\pstart\center{}Verehrter Herr Doktor!\pend\vspace{0.5em}
\pstart
           Ich muss Ihrem liebenswürdigen Brief vom 24. v. Mts. widersprechen, wenn
               ich ihn etwa als eine Zurücknahme Ihrer freundlichen Zusage, einen Beitrag fürs TAGE-BUCH\orgindex{Tage-Buch@Das Tage-Buch|pw} zu schicken, ansehen soll.\pend
           
\pstart
           Sie würden mich zu grossem Dank verpflichten, wenn Sie recht bald Ihre Zusage
               erfüllen wollten.\pend
           
\pstart
           Ich begrüsse Sie mit ausgezeichneter Hochschätzung{\\[\baselineskip]}Ihr sehr ergebener{\\[\baselineskip]}\spacefill\mbox{{[}hs.:{]} Großmann}\pend
           \leftskip=0em{}\selectlanguage{ngerman}\endnumbering\briefempfaengerindex{Schnitzler, Arthur@\textsc{Schnitzler, Arthur}!zzzGroßmann, Stefan@\emph{von Stefan Großmann}!1926-01-022@{2. 1. 192[6?]}|)be}\mylabel{L02464h}  \newcommand{\dateiname}{L02464}\newcommand{\titel}{Stefan Großmann an Arthur Schnitzler, 2. 1. 192[6?]}\newcommand{\editorInnen}{Herausgegeben von Martin Anton Müller}%% latex-leseansicht-abspann.tex
%% Abspann für die Leseansicht.
%% Der Schalter \ifkorrekturansicht ist bereits durch den Vorspann gesetzt.

%% latex-abspann.tex
%% Gemeinsamer Abspann für Korrekturansicht und Leseansicht.
%% Setzt den Schalter \ifkorrekturansicht voraus (gesetzt in den
%% einbindenden Dateien latex-korrekturansicht-abspann.tex bzw.
%% latex-leseansicht-abspann.tex).
%% ---------------------------------------------------------------

\normalsize

% Das esempio-Environment wird nur in der Leseansicht benötigt
\ifkorrekturansicht\else
\newenvironment{esempio}[3]%
{
    \vspace{1.5ex}
    \rlap{\underline{#1}}
    \par
    \setlength{\parindent}{0cm}
    \nopagebreak
    \leftskip=#2cm
    \rightskip=#3cm
}
{
    \par
}
\fi

\doendnotes{C}
\bigskip
\vfill

\clearpage

\footnotesize

\ifkorrekturansicht
  \lohead{\textsc{register}}
\fi

% theindex-Environment neu definieren ohne reledmac
\makeatletter
\renewenvironment{theindex}{%
  \ifkorrekturansicht
    \section*{\indexname}%
  \else
    \subsubsection*{Index der erwähnten Entitäten}%
  \fi
  \setlength{\parindent}{0pt}%
  \setlength{\parskip}{0pt plus 0.3pt}%
  \let\item\@idxitem
}{%
  \ifkorrekturansicht\clearpage\fi
}
\makeatother

\IfFileExists{\jobname-pw.ind}{\input{\jobname-pw.ind}}{}

% Quellenangabe nur in der Leseansicht
\ifkorrekturansicht\else
% Fallback-Definitionen, falls die .tex-Datei \titel etc. nicht gesetzt hat
\providecommand{\titel}{}
\providecommand{\editorInnen}{}
\providecommand{\dateiname}{\jobname}

\vspace{3cm}

\vfill

\footnotesize
\textsc{Quelle}: \titel. Herausgegeben von {\editorInnen}. In: \emph{Arthur Schnitzler: Briefwechsel mit Autorinnen und Autoren}.
 Digitale Edition, https://schnitzler-briefe.acdh.oeaw.ac.at/{\dateiname}.html (Stand \today)
\fi

\end{document}


