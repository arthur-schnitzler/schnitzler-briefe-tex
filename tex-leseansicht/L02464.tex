%% latex-korrekturansicht-vorspann.tex
%% Vorspann für die Korrekturansicht.
%% Lädt die gemeinsame Datei latex-vorspann.tex mit gesetztem Schalter.

\newif\ifkorrekturansicht
\korrekturansichttrue

\input{../tex-inputs/latex-vorspann}


\section[Stefan Großmann an Arthur Schnitzler, 2. 1. 192{[}6?{]}]{L02464 Stefan Großmann an Arthur Schnitzler, 2. 1. 192{[}6?{]}}
\nopagebreak\mylabel{L02464v}
\rehead{ }\normalsize\beginnumbering\briefempfaengerindex{Schnitzler, Arthur@\textsc{Schnitzler, Arthur}!zzzGrossmann, Stefan@\emph{von Stefan Großmann}!1926-01-021@{2. 1. 192{[}6?{]}}|(be}
\toendnotes[C]{\smallbreak\pagebreak[2]}\Standort{DLA, A:Schnitzler, HS.NZ85.1.3232.}
\physDesc{Brief, 1 Blatt, 1 Seite, 472 Zeichen
\newline{}Schreibmaschine
\newline{}Handschrift: blaue Tinte (\noindent{}Unterschrift)
\newline{}Schnitzler: mit rotem Buntstift beschriftet: »\textsc{Großma{\geminationn}}« }\toendnotes[C]{\smallbreak}
\pstart
           \centering{}{\pb}\textcolor{gray}{\textbf{Das Tage-Buch\orgindex{Tage-Buch@Das Tage-Buch|pw}}}\pend
           
\pstart
           \centering{}\textcolor{gray}{\textbf{\emph{Herausgeber: Stefan Großmann und Leopold Schwarzschild\pwindex{Schwarzschild, Leopold 1891-12-08 – 1950-10-02@\textsc{Schwarzschild, Leopold} (1891-12-08 – 1950-10-02), \emph{Publizist/Publizistin}|pw}}}}\pend
           
\pstart
           \centering{}\textcolor{gray}{\textbf{Tagebuchverlag m. b. H., Berlin SW 19\oindex{Berlin@\textbf{Berlin}, \emph{P.PPLC}|pw}}}\pend
           
\pstart
           \centering{}\textcolor{gray}{\textbf{BEUTHSTRASSE 19\oindex{Beuthstrasse@\textbf{Beuthstrasse}, \emph{Straße (K.STR)}|pw}}}\pend
           
\pstart
           \centering{}\textcolor{gray}{\textbf{\emph{Telegramm-Adresse: Tagebuch Berlin\oindex{Berlin@\textbf{Berlin}, \emph{P.PPLC}|pw} ⋅ Fernsprecher: Merkur 8790–8792}}}\pend
           
\pstart
           \centering{}\textcolor{gray}{\textbf{\emph{\so{Sprechstunde der Redaktion: 12–1 Uhr}}}}\pend
           
\pstart
           \centering{}\textcolor{gray}{\textbf{*}}\pend
           
\pstart
           Tgb./Gr./Schl.\hfill Berlin\oindex{Berlin@\textbf{Berlin}, \emph{P.PPLC}|pw}, den \label{K_L02464-1v}\edtext{2. Januar 1925}{\lemma{\textnormal{\emph{2. Januar 1925}}}\Cendnote{\textnormal{Es dürfte sich bei der Jahreszahl
                        um einen Irrtum handeln, da andernfalls mehrere Korrespondenzstücke als
                        verlustig zu gelten hätten.}}}\label{K_L02464-1}.\pend
           {\vspace{1\baselineskip}}
\pstart
           \raggedleft{}Herrn\pend
           
\pstart
           \raggedleft{}Dr. Arthur \so{Schnitzler}\pend
           
\pstart
           \raggedleft{}\so{Wien XVIII}\oindex{XVIII., Waehring@\textbf{XVIII., Währing}, \emph{A.ADM3}|pw}\pend
           
\pstart
           \raggedleft{}Sternwartestr. 71\oindex{Sternwartestrasse 71@\textbf{Sternwartestraße 71}, \emph{Wohngebäude (K.WHS)}|pw}.\pend
           
\pstart\center{}Verehrter Herr Doktor!\pend\vspace{0.5em}
\pstart
           Ich muss Ihrem liebenswürdigen Brief vom 24. v. Mts. widersprechen, wenn
               ich ihn etwa als eine Zurücknahme Ihrer freundlichen Zusage, einen Beitrag fürs TAGE-BUCH\orgindex{Tage-Buch@Das Tage-Buch|pw} zu schicken, ansehen soll.\pend
           
\pstart
           Sie würden mich zu grossem Dank verpflichten, wenn Sie recht bald Ihre Zusage
               erfüllen wollten.\pend
           
\pstart
           Ich begrüsse Sie mit ausgezeichneter Hochschätzung{\\[\baselineskip]}Ihr sehr ergebener{\\[\baselineskip]}\spacefill\mbox{{[}hs.:{]} Großmann}\pend
           \leftskip=0em{}\selectlanguage{ngerman}\endnumbering\briefempfaengerindex{Schnitzler, Arthur@\textsc{Schnitzler, Arthur}!zzzGrossmann, Stefan@\emph{von Stefan Großmann}!1926-01-021@{2. 1. 192{[}6?{]}}|)be}\mylabel{L02464h}  \normalsize

\doendnotes{C}
\bigskip
\vfill

\clearpage

\footnotesize

\lohead{\textsc{register}}

% Definiere theindex-Environment komplett neu ohne reledmac
\makeatletter
\renewenvironment{theindex}{%
  \section*{\indexname}%
  \setlength{\parindent}{0pt}%
  \setlength{\parskip}{0pt plus 0.3pt}%
  \let\item\@idxitem
}{%
  \clearpage
}
\makeatother

\IfFileExists{\jobname-pw.ind}{\input{\jobname-pw.ind}}{}

\end{document}

      