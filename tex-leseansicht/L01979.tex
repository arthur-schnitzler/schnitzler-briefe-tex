%% latex-korrekturansicht-vorspann.tex
%% Vorspann für die Korrekturansicht.
%% Lädt die gemeinsame Datei latex-vorspann.tex mit gesetztem Schalter.

\newif\ifkorrekturansicht
\korrekturansichttrue

\input{../tex-inputs/latex-vorspann}


\section[Arthur Schnitzler an Hugo von Hofmannsthal, 10. 11. 1910]{L01979 Arthur Schnitzler an Hugo von Hofmannsthal, 10. 11. 1910}
\nopagebreak\mylabel{L01979v}
\rehead{ }\normalsize\beginnumbering\briefempfaengerindex{Hofmannsthal, Hugo von@\textsc{Hofmannsthal, Hugo von}!zzzSchnitzler, Arthur@\emph{von Arthur Schnitzler}!1910-11-101@{10. 11. 1910}|(be}
\toendnotes[C]{\smallbreak\pagebreak[2]}\Standort{FDH, Hs-30885,141.}
\physDesc{Brief, 1 Blatt, 2 Seiten, 522 Zeichen
\newline{}Handschrift: schwarze Tinte, deutsche Kurrent}
\buchAbdrucke{\weitereDrucke{Hugo von Hofmannsthal, Arthur Schnitzler: \emph{Briefwechsel}. Frankfurt am Main: \emph{S. Fischer} 1964, S. 259.} }\toendnotes[C]{\smallbreak}
\pstart
           {\pb}\textcolor{gray}{\textbf{Dr. Arthur Schnitzler}}\hfill 10. 11. 1910\pend
           
\pstart
           \textcolor{gray}{\textbf{Wien XVIII. Sternwartestrasse 71\oindex{Sternwartestrasse 71@\textbf{Sternwartestraße 71}, \emph{Wohngebäude (K.WHS)}|pw}}}\pend
           
\pstart{}mein lieber Hugo,\pend\vspace{0.5em}
\pstart
           Ihre guten Worte hätten auch ſchlimmeres wieder gut machen können! Nun aber hätt ich
               ein rechtes Bedürfnis Ihnen wieder die Hand zu drücken und mit Ihnen zu reden.
               Wolltet Ihr nicht einmal ganz gemütlich – vorläufig {\pb}ohne
                  Vorleſung\pwindex{Rosenkavalier@\emph{Der Rosenkavalier}|pwv} – nur wir vier\pwindex{Schnitzler, Olga 17.01.1882 – 13.01.1970@\textsc{Schnitzler, Olga} (17.01.1882 – 13.01.1970), \emph{Schauspieler/Schauspielerin, Sänger/Sängerin}|pwv}\pwindex{Hofmannsthal, Gertrude von 16.03.1880 – 09.11.1959@\textsc{Hofmannsthal, Gertrude von} (16.03.1880 – 09.11.1959)|pwv} – noch vor dem
                  \textsc{Medardus}\pwindex{junge Medardus. Dramatische Historie in einem Vorspiel und fuenf Aufzuegen@\emph{Der junge Medardus. Dramatische Historie in einem Vorspiel und fünf Aufzügen}|pw} bei uns nachtmahlen? Wählen Sie einen \label{K_L01979-1v}\edtext{Abend}{\lemma{\textnormal{\emph{Abend}}}\Cendnote{\textnormal{Siehe A. S.: \emph{Tagebuch}, 16. 11. 1910.
               }}}\label{K_L01979-1} (der um ½ 7 anfangen ka{\geminationn}) – anfangs
               nächſter Woche von Dinſtag ab; aber ſchreiben Sie rechtzeitig. (Die \textsc{Med.\pwindex{junge Medardus. Dramatische Historie in einem Vorspiel und fuenf Aufzuegen@\emph{Der junge Medardus. Dramatische Historie in einem Vorspiel und fünf Aufzügen}|pw}-Première} iſt gewiſs noch nicht am
                  19. Vielleicht 23. od 24. November.)\pend
           
\pstart
           Herzlichst{\\[\baselineskip]}Ihr{\\[\baselineskip]}\spacefill\mbox{Arthur.}\pend
           \leftskip=0em{}\selectlanguage{ngerman}\endnumbering\briefempfaengerindex{Hofmannsthal, Hugo von@\textsc{Hofmannsthal, Hugo von}!zzzSchnitzler, Arthur@\emph{von Arthur Schnitzler}!1910-11-101@{10. 11. 1910}|)be}\mylabel{L01979h}  \normalsize

\doendnotes{C}
\bigskip
\vfill

\clearpage

\footnotesize

\lohead{\textsc{register}}

% Definiere theindex-Environment komplett neu ohne reledmac
\makeatletter
\renewenvironment{theindex}{%
  \section*{\indexname}%
  \setlength{\parindent}{0pt}%
  \setlength{\parskip}{0pt plus 0.3pt}%
  \let\item\@idxitem
}{%
  \clearpage
}
\makeatother

\IfFileExists{\jobname-pw.ind}{\input{\jobname-pw.ind}}{}

\end{document}

      