\input{../tex-inputs/latex-pdf-vorspann}
\begin{center}
            \textcolor{red}{ENTWURF. ENTZIFFERUNG NOCH NICHT KORREKTURGELESEN}
                      \end{center}
            
               \section[Hugo von Hofmannsthal an Arthur Schnitzler, 17. 10. {[}1895{]}]{ Hugo von Hofmannsthal an Arthur Schnitzler, 17. 10. {[}1895{]}}\nopagebreak\mylabel{v}\rehead{ }\begin{ledgroupsized}[t]{13cm}\normalsize\beginnumbering\briefempfaengerindex{Schnitzler, Arthur@\textsc{Schnitzler, Arthur}!zzzHofmannsthal, Hugo von@\emph{von Hugo von Hofmannsthal}!1895-10-171@{17. 10. {[}1895{]}}|(be} \toendnotes[C]{\smallbreak\pagebreak[2]} \Standort{CUL, Schnitzler, B 43.}
\physDesc{Brief, 1 Blatt, 3 Seiten
\newline{}Handschrift: schwarze Tinte, deutsche Kurrent
\newline{}Schnitzler: mit Bleistift die Jahreszahl ergänzt: »95« und nummeriert: »76« }\buchAbdrucke{\weitereDrucke{Hugo von Hofmannsthal, Arthur Schnitzler: \emph{Briefwechsel}. Hg. Therese Nickl und Heinrich Schnitzler. Frankfurt am Main: \emph{S. Fischer} 1964, S. 63.} }\toendnotes[C]{\smallbreak}\pstart
           \raggedleft{}{\pb}Venedig\oindex{Venedig@\textbf{Venedig}|pw}{ }17. October\pend
           \pstart
           am Sonntag{ }Früh hab ich Sie beſucht, aber nur 3 Frauen mit Beſen gefunden. Ich
                    wollte Ihnen ſagen, daſs ich nach den Zeitungen und dem Reden der Leute wirklich
                    glaube, daſs Sie jetzt dieſes unberechenbare und ſchwer zu definierende erworben
                    haben, womit man Aufmerkſamkeit und Bewunderung erzwingen kann. Ich glaube, Sie
                    dürfen ſich jetzt erlauben, für die Darſtellung {\pb}tiefer und kühner Dinge auf
                    mehreren Beifall zu rechnen als bloß auf den von 3 oder 4 Freunden.\pend
           \pstart
           Richard\pwindex{Beer-Hofmann, Richard 11.07.1866 – 26.09.1945@\textsc{Beer-Hofmann, Richard} (11.07.1866 – 26.09.1945), \emph{Schriftsteller}|pw} hat mir die geſcheidte Kritik\pwindex{Burgtheater [Rechte der Seele/Liebelei]14.10.1895 – 14.10.1895@\emph{Burgtheater [Rechte der Seele/Liebelei]} {[}14.10.1895 – 14.10.1895{]}|pwv} von Berger\pwindex{Berger, Alfred von 30.04.1853 – 24.08.1912@\textsc{Berger, Alfred von} (30.04.1853 – 24.08.1912), \emph{Schriftsteller, Journalist, Theaterleiter}|pw} geſchickt und die \label{K_L00508_1v}\edtext{Verſpottung\pwindex{Jung-Wiener Dichter. (Zur Burgtheater-Premiere.)14.10.1895 – 14.10.1895@\emph{Jung-Wiener Dichter. (Zur Burgtheater-Première.)} {[}14.10.1895 – 14.10.1895{]}|pwv}}{\lemma{\textnormal{\emph{Verſpottung}}}\Cendnote{\textnormal{Der Text geht nicht nur auf die
                            \emph{Liebelei}\pwindex{Schnitzler, Arthur 15.05.1862 – 21.10.1931@\textsc{Schnitzler, Arthur} (15.05.1862 – 21.10.1931), \emph{Schriftsteller, Mediziner}!Liebelei. Schauspiel in drei Akten9. 10. 1895@\strich\emph{Liebelei. Schauspiel in drei Akten} {[}9. 10. 1895{]}|pwk} ein, sondern auch auf
                            Hofmannsthal\pwindex{Hofmannsthal, Hugo von 01.02.1874 – 15.07.1929@\textsc{Hofmannsthal, Hugo von} (01.02.1874 – 15.07.1929), \emph{Schriftsteller}|pwk} und Beer-Hofmann\pwindex{Beer-Hofmann, Richard 11.07.1866 – 26.09.1945@\textsc{Beer-Hofmann, Richard} (11.07.1866 – 26.09.1945), \emph{Schriftsteller}|pwk}.}}}\label{K_L00508_1h} von dem Anonymen\pwindex{Der Reporter @\textsc{Der Reporter}, \emph{Journalist/Journalistin}|pw}. Iſt es der kleine Kraus\pwindex{Kraus, Karl 28.04.1874 – 12.06.1936@\textsc{Kraus, Karl} (28.04.1874 – 12.06.1936), \emph{Schriftsteller, Publizist}|pw}? Es hat mich unterhalten, ich wäre froh,
                    wenn ſolche Sachen viel öfter geſchrieben würden und auch Caricaturen von uns
                    gezeichnet. {\pb}Das wird ſich
                    auch immer ſteigern je mutiger und beſſer wir werden; ich denke, von der
                    Generation von Philologen und Dilettanten, die vor uns war, wirds nicht viel
                    Verhöhnungen geben.\pend
           \pstart
           Hier arbeit ich nicht, aber werds wohl nachher.\pend
           \pstart
           Adieu. Herzlich Ihr{\\[\baselineskip]}\spacefill\mbox{Hugo.}\pend
           \leftskip=0em{}\endnumbering\briefempfaengerindex{Schnitzler, Arthur@\textsc{Schnitzler, Arthur}!zzzHofmannsthal, Hugo von@\emph{von Hugo von Hofmannsthal}!1895-10-171@{17. 10. {[}1895{]}}|)be}\mylabel{h}\end{ledgroupsized}  \newcommand{\dateiname}{L00508}\newcommand{\titel}{Hugo von Hofmannsthal an Arthur Schnitzler, 17. 10. [1895]}\newcommand{\editorInnen}{Martin Anton Müller und Gerd-Hermann Susen}\input{../tex-inputs/latex-pdf-abspann}
      