%% latex-leseansicht-vorspann.tex
%% Vorspann für die Leseansicht.
%% Lädt die gemeinsame Datei latex-vorspann.tex mit nicht gesetztem Schalter.

\newif\ifkorrekturansicht
\korrekturansichtfalse

\input{../tex-inputs/latex-vorspann}


         
         \renewcommand{\erwaehntePersonen}{Personen: Richard Beer-Hofmann, Alfred von Berger,  Der Reporter, Hugo von Hofmannsthal, Karl Kraus}
         \renewcommand{\erwaehnteOrte}{Orte: Venedig, Wien}
         \renewcommand{\erwaehnteWerke}{Werke: Burgtheater [Rechte der Seele, Liebelei], Extrapost. Unparteiische Montags-Zeitung, Jung-Wiener Dichter. (Zur Burgtheater-Première.), Liebelei. Schauspiel in drei Akten}
               \section[Hugo von Hofmannsthal an Arthur Schnitzler, 17. 10. {[}1895{]}]{ Hugo von Hofmannsthal an Arthur Schnitzler, 17. 10. {[}1895{]}}\nopagebreak\mylabel{v}\rehead{ }\begin{ledgroupsized}[t]{13cm}\normalsize\beginnumbering \toendnotes[C]{\smallbreak\pagebreak[2]} \Standort{CUL, Schnitzler, B 43.}
\physDesc{Brief, 1 Blatt, 3 Seiten, 976 Zeichen
\newline{}Handschrift: schwarze Tinte, deutsche Kurrent
\newline{}Schnitzler: mit Bleistift die Jahreszahl ergänzt: »95« und nummeriert: »76« }\buchAbdrucke{\weitereDrucke{Hugo von Hofmannsthal, Arthur Schnitzler: \emph{Briefwechsel}. Hg. Therese Nickl und Heinrich Schnitzler. Frankfurt am Main: \emph{S. Fischer} 1964, S. 63.} }\toendnotes[C]{\smallbreak}\pstart
           \raggedleft{}{\pb}Venedig\oindex{Venedig@\textbf{Venedig}|pw}{ }17. October\pend
           \pstart
           am Sonntag{ }Früh hab ich Sie beſucht, aber nur 3 Frauen mit Beſen gefunden. Ich
               wollte Ihnen ſagen, daſs ich nach den Zeitungen und dem Reden der Leute wirklich
               glaube, daſs Sie jetzt dieſes unberechenbare und ſchwer zu definierende erworben
               haben, womit man Aufmerkſamkeit und Bewunderung erzwingen kann. Ich glaube, Sie
               dürfen ſich jetzt erlauben, für die Darſtellung {\pb}tiefer und kühner Dinge auf
               mehreren Beifall zu rechnen als bloß auf den von 3 oder 4 Freunden.\pend
           \pstart
           Richard\pwindex{Beer-Hofmann, Richard 1866-07-11 – 1945-09-26@\textsc{Beer-Hofmann, Richard} (1866-07-11 – 1945-09-26), \emph{Schriftsteller}|pw} hat mir die geſcheidte Kritik\pwindex{Berger, Alfred von 30.04.1853 – 24.08.1912@\textsc{Berger, Alfred von} (30.04.1853 – 24.08.1912), \emph{Schriftsteller, Journalist, Theaterleiter}!Burgtheater [Rechte der Seele, Liebelei]1895-10-14@\strich\emph{Burgtheater [Rechte der Seele, Liebelei]} {[}1895-10-14{]}|pwv} von Berger\pwindex{Berger, Alfred von 30.04.1853 – 24.08.1912@\textsc{Berger, Alfred von} (30.04.1853 – 24.08.1912), \emph{Schriftsteller, Journalist, Theaterleiter}|pw} geſchickt und die \label{K_L00508-1v}\edtext{Verſpottung\pwindex{Jung-Wiener Dichter. (Zur Burgtheater-Premiere.)14. 10. 1895@\emph{Jung-Wiener Dichter. (Zur Burgtheater-Première.)} {[}14. 10. 1895{]}|pwv}}{\lemma{\textnormal{\emph{Verſpottung}}}\Cendnote{\textnormal{Der Reporter\pwindex{Der Reporter @\textsc{Der Reporter}, \emph{Journalist}|pwk}: \emph{Jung-Wiener Dichter. (Zur Burgtheater-Première.)}\pwindex{Jung-Wiener Dichter. (Zur Burgtheater-Premiere.)14. 10. 1895@\emph{Jung-Wiener Dichter. (Zur Burgtheater-Première.)} {[}14. 10. 1895{]}|pwk} In:
                        \emph{Extrapost}\pwindex{?? Werk@Nicht ermittelte Verfasserinnen und Verfasser!Extrapost. Unparteiische Montags-Zeitung1882 – 1905@\emph{Extrapost. Unparteiische Montags-Zeitung} {[}1882 – 1905{]}|pwk}, Jg. 14, Nr. 717,
                        14. 10. 1895, S. 1–2. Der Text geht nicht nur auf die
                     \emph{Liebelei}\pwindex{Schnitzler, Arthur 15.05.1862 – 21.10.1931@\textsc{Schnitzler, Arthur} (15.05.1862 – 21.10.1931), \emph{Schriftsteller, Mediziner}!Liebelei. Schauspiel in drei Akten1895-10-09@\strich\emph{Liebelei. Schauspiel in drei Akten} {[}1895-10-09{]}|pwk} ein, sondern auch auf Hofmannsthal\pwindex{Hofmannsthal, Hugo von 1874-02-01 – 1929-07-15@\textsc{Hofmannsthal, Hugo von} (1874-02-01 – 1929-07-15), \emph{Schriftsteller}|pwk} und Beer-Hofmann\pwindex{Beer-Hofmann, Richard 1866-07-11 – 1945-09-26@\textsc{Beer-Hofmann, Richard} (1866-07-11 – 1945-09-26), \emph{Schriftsteller}|pwk}.}}}\label{K_L00508-1h} von dem Anonymen\pwindex{Der Reporter @\textsc{Der Reporter}, \emph{Journalist}|pw}. Iſt es der kleine Kraus\pwindex{Kraus, Karl 28.04.1874 – 12.06.1936@\textsc{Kraus, Karl} (28.04.1874 – 12.06.1936), \emph{Schriftsteller, Publizist}|pw}? Es hat mich unterhalten, ich wäre froh, wenn ſolche Sachen viel öfter
               geſchrieben würden und auch Caricaturen von uns gezeichnet. {\pb}Das wird ſich auch immer ſteigern
               je mutiger und beſſer wir werden; ich denke, von der Generation von Philologen und
               Dilettanten, die vor uns war, wirds nicht viel Verhöhnungen geben.\pend
           \pstart
           Hier arbeit ich nicht, aber werds wohl nachher.\pend
           \pstart
           Adieu. Herzlich Ihr{\\[\baselineskip]}\spacefill\mbox{Hugo.}\pend
           \leftskip=0em{}
         
         \endnumbering\mylabel{h}\end{ledgroupsized}  \newcommand{\dateiname}{L00508}\newcommand{\titel}{Hugo von Hofmannsthal an Arthur Schnitzler, 17. 10. [1895]}\newcommand{\editorInnen}{Martin Anton Müller und Gerd-Hermann Susen}%% latex-leseansicht-abspann.tex
%% Abspann für die Leseansicht.
%% Der Schalter \ifkorrekturansicht ist bereits durch den Vorspann gesetzt.

%% latex-abspann.tex
%% Gemeinsamer Abspann für Korrekturansicht und Leseansicht.
%% Setzt den Schalter \ifkorrekturansicht voraus (gesetzt in den
%% einbindenden Dateien latex-korrekturansicht-abspann.tex bzw.
%% latex-leseansicht-abspann.tex).
%% ---------------------------------------------------------------

\normalsize

% Das esempio-Environment wird nur in der Leseansicht benötigt
\ifkorrekturansicht\else
\newenvironment{esempio}[3]%
{
    \vspace{1.5ex}
    \rlap{\underline{#1}}
    \par
    \setlength{\parindent}{0cm}
    \nopagebreak
    \leftskip=#2cm
    \rightskip=#3cm
}
{
    \par
}
\fi

\doendnotes{C}
\bigskip
\vfill

\clearpage

\footnotesize

\ifkorrekturansicht
  \lohead{\textsc{register}}
\fi

% theindex-Environment neu definieren ohne reledmac
\makeatletter
\renewenvironment{theindex}{%
  \ifkorrekturansicht
    \section*{\indexname}%
  \else
    \subsubsection*{Index der erwähnten Entitäten}%
  \fi
  \setlength{\parindent}{0pt}%
  \setlength{\parskip}{0pt plus 0.3pt}%
  \let\item\@idxitem
}{%
  \ifkorrekturansicht\clearpage\fi
}
\makeatother

\IfFileExists{\jobname-pw.ind}{\input{\jobname-pw.ind}}{}

% Quellenangabe nur in der Leseansicht
\ifkorrekturansicht\else
% Fallback-Definitionen, falls die .tex-Datei \titel etc. nicht gesetzt hat
\providecommand{\titel}{}
\providecommand{\editorInnen}{}
\providecommand{\dateiname}{\jobname}

\vspace{3cm}

\vfill

\footnotesize
\textsc{Quelle}: \titel. Herausgegeben von {\editorInnen}. In: \emph{Arthur Schnitzler: Briefwechsel mit Autorinnen und Autoren}.
 Digitale Edition, https://schnitzler-briefe.acdh.oeaw.ac.at/{\dateiname}.html (Stand \today)
\fi

\end{document}


      