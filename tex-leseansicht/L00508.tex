%% latex-leseansicht-vorspann.tex
%% Vorspann für die Leseansicht.
%% Lädt die gemeinsame Datei latex-vorspann.tex mit nicht gesetztem Schalter.

\newif\ifkorrekturansicht
\korrekturansichtfalse

\input{../tex-inputs/latex-vorspann}


\section[Hugo von Hofmannsthal an Arthur Schnitzler, 17. 10. {[}1895{]}]{L00508 Hugo von Hofmannsthal an Arthur Schnitzler, 17. 10. [1895]}
\nopagebreak\mylabel{L00508v}
\rehead{ }\normalsize\beginnumbering\briefempfaengerindex{Schnitzler, Arthur@\textsc{Schnitzler, Arthur}!zzzHofmannsthal, Hugo von@\emph{von Hugo von Hofmannsthal}!1895-10-172@{17. 10. [1895]}|(be}
\toendnotes[C]{\smallbreak\pagebreak[2]}
\correspDesc{Versand  durch Hugo von Hofmannsthal am 17. 10. [1895] in Venedig
\newline{}Erhalt  durch Arthur Schnitzler im Zeitraum [18. 10. 1895 – 22. 10. 1895?] in Wien}\toendnotes[C]{\smallbreak}
\Standort{CUL, Schnitzler, B 43.}
\physDesc{Brief, 1 Blatt, 3 Seiten, 976 Zeichen
\newline{}Handschrift: schwarze Tinte, deutsche Kurrent
\newline{}Schnitzler: mit Bleistift die Jahreszahl ergänzt: »95« und nummeriert: »76« }
\buchAbdrucke{\weitereDrucke{Hugo von Hofmannsthal, Arthur Schnitzler: \emph{Briefwechsel}. Herausgegeben von Therese Nickl und Heinrich Schnitzler. Frankfurt am Main: \emph{S. Fischer} 1964, S. 63.} }\toendnotes[C]{\smallbreak}
\pstart
           \raggedleft{}{\pb}Venedig\oindex{Venedig@\textbf{Venedig}|pw}{ }17. October\pend
           \vspace{0.5em}
\pstart
           am Sonntag{ }Früh hab ich Sie beſucht, aber nur 3 Frauen mit Beſen gefunden. Ich
               wollte Ihnen{ }ſagen, daſs ich nach den Zeitungen und dem Reden der Leute wirklich
               glaube, daſs Sie jetzt dieſes unberechenbare und{ }ſchwer zu definierende erworben
               haben, womit man Aufmerkſamkeit und Bewunderung erzwingen kann. Ich glaube, Sie
               dürfen{ }ſich jetzt erlauben, für die Darſtellung {\pb}tiefer und kühner Dinge auf
               mehreren Beifall zu rechnen als bloß auf den von 3 oder 4 Freunden.\pend
           
\pstart
           Richard\pwindex{Beer-Hofmann, Richard 11.\,7.\,1866 Wien – 26.\,9.\,1945 New York City@\textsc{Beer-Hofmann, Richard} (11.\,7.\,1866 Wien – 26.\,9.\,1945 New York City), \emph{Schriftsteller}|pw} hat mir die geſcheidte Kritik\pwindex{Berger, Alfred von 30.\,4.\,1853 Wien – 24.\,8.\,1912 ebd.@\textsc{Berger, Alfred von} (30.\,4.\,1853 Wien – 24.\,8.\,1912 ebd.), \emph{Schriftsteller, Journalist, Theaterleiter}!Burgtheater [Rechte der Seele, Liebelei]@\strich\emph{Burgtheater [Rechte der Seele, Liebelei]}|pwv} von Berger\pwindex{Berger, Alfred von 30.\,4.\,1853 Wien – 24.\,8.\,1912 ebd.@\textsc{Berger, Alfred von} (30.\,4.\,1853 Wien – 24.\,8.\,1912 ebd.), \emph{Schriftsteller, Journalist, Theaterleiter}|pw} geſchickt und die \label{K_L00508-1v}\edtext{Verſpottung\pwindex{Der Reporter @\textsc{Der Reporter}, \emph{Journalist}!Jung-Wiener Dichter. (Zur Burgtheater-Première.)@\strich\emph{Jung-Wiener Dichter. (Zur Burgtheater-Première.)}|pwv}}{\lemma{\textnormal{\emph{Verspottung}}}\Cendnote{\textnormal{Der Reporter\pwindex{Der Reporter @\textsc{Der Reporter}, \emph{Journalist}|pwk}: \emph{Jung-Wiener Dichter. (Zur Burgtheater-Première.)}\pwindex{Der Reporter @\textsc{Der Reporter}, \emph{Journalist}!Jung-Wiener Dichter. (Zur Burgtheater-Première.)@\strich\emph{Jung-Wiener Dichter. (Zur Burgtheater-Première.)}|pwk} In:
                        \emph{Extrapost}\pwindex{Extrapost. Unparteiische Montags-Zeitung@\emph{Extrapost. Unparteiische Montags-Zeitung}|pwk}, Jg. 14, Nr. 717,
                        14. 10. 1895, S. 1–2. Der Text geht nicht nur auf 
                     \emph{Liebelei}\pwindex{Schnitzler, Arthur 15.\,5.\,1862 Wien – 21.\,10.\,1931 ebd.@\textsc{Schnitzler, Arthur} (15.\,5.\,1862 Wien – 21.\,10.\,1931 ebd.), \emph{Schriftsteller, Mediziner}!Liebelei. Schauspiel in drei Akten@\strich\emph{Liebelei. Schauspiel in drei Akten}|pwk} ein, sondern auch auf Hofmannsthal\pwindex{Hofmannsthal, Hugo von 1.\,2.\,1874 Wien – 15.\,7.\,1929 Rodaun@\textsc{Hofmannsthal, Hugo von} (1.\,2.\,1874 Wien – 15.\,7.\,1929 Rodaun), \emph{Schriftsteller}|pwk} und Beer-Hofmann\pwindex{Beer-Hofmann, Richard 11.\,7.\,1866 Wien – 26.\,9.\,1945 New York City@\textsc{Beer-Hofmann, Richard} (11.\,7.\,1866 Wien – 26.\,9.\,1945 New York City), \emph{Schriftsteller}|pwk}.}}}\label{K_L00508-1} von dem Anonymen\pwindex{Der Reporter @\textsc{Der Reporter}, \emph{Journalist}|pw}. Iſt es der kleine Kraus\pwindex{Kraus, Karl 28.\,4.\,1874 Jičín – 12.\,6.\,1936 Wien@\textsc{Kraus, Karl} (28.\,4.\,1874 Jičín – 12.\,6.\,1936 Wien), \emph{Schriftsteller, Publizist, Schriftsteller}|pw}? Es hat mich unterhalten, ich wäre froh, wenn{ }ſolche Sachen viel öfter
               geſchrieben würden und auch Caricaturen von uns gezeichnet. {\pb}Das wird{ }ſich auch immer{ }ſteigern
               je mutiger und beſſer wir werden; ich denke, von der Generation von Philologen und
               Dilettanten, die vor uns war, wirds nicht viel Verhöhnungen geben.\pend
           
\pstart
           Hier arbeit ich nicht, aber werds wohl nachher.\pend
           
\pstart
           Adieu. Herzlich Ihr{\\[\baselineskip]}\spacefill\mbox{Hugo.}\pend
           \leftskip=0em{}\selectlanguage{ngerman}\endnumbering\briefempfaengerindex{Schnitzler, Arthur@\textsc{Schnitzler, Arthur}!zzzHofmannsthal, Hugo von@\emph{von Hugo von Hofmannsthal}!1895-10-172@{17. 10. [1895]}|)be}\mylabel{L00508h}  \newcommand{\dateiname}{L00508}\newcommand{\titel}{Hugo von Hofmannsthal an Arthur Schnitzler, 17. 10. [1895]}\newcommand{\editorInnen}{Martin Anton Müller und Gerd-Hermann Susen}%% latex-leseansicht-abspann.tex
%% Abspann für die Leseansicht.
%% Der Schalter \ifkorrekturansicht ist bereits durch den Vorspann gesetzt.

%% latex-abspann.tex
%% Gemeinsamer Abspann für Korrekturansicht und Leseansicht.
%% Setzt den Schalter \ifkorrekturansicht voraus (gesetzt in den
%% einbindenden Dateien latex-korrekturansicht-abspann.tex bzw.
%% latex-leseansicht-abspann.tex).
%% ---------------------------------------------------------------

\normalsize

% Das esempio-Environment wird nur in der Leseansicht benötigt
\ifkorrekturansicht\else
\newenvironment{esempio}[3]%
{
    \vspace{1.5ex}
    \rlap{\underline{#1}}
    \par
    \setlength{\parindent}{0cm}
    \nopagebreak
    \leftskip=#2cm
    \rightskip=#3cm
}
{
    \par
}
\fi

\doendnotes{C}
\bigskip
\vfill

\clearpage

\footnotesize

\ifkorrekturansicht
  \lohead{\textsc{register}}
\fi

% theindex-Environment neu definieren ohne reledmac
\makeatletter
\renewenvironment{theindex}{%
  \ifkorrekturansicht
    \section*{\indexname}%
  \else
    \subsubsection*{Index der erwähnten Entitäten}%
  \fi
  \setlength{\parindent}{0pt}%
  \setlength{\parskip}{0pt plus 0.3pt}%
  \let\item\@idxitem
}{%
  \ifkorrekturansicht\clearpage\fi
}
\makeatother

\IfFileExists{\jobname-pw.ind}{\input{\jobname-pw.ind}}{}

% Quellenangabe nur in der Leseansicht
\ifkorrekturansicht\else
% Fallback-Definitionen, falls die .tex-Datei \titel etc. nicht gesetzt hat
\providecommand{\titel}{}
\providecommand{\editorInnen}{}
\providecommand{\dateiname}{\jobname}

\vspace{3cm}

\vfill

\footnotesize
\textsc{Quelle}: \titel. Herausgegeben von {\editorInnen}. In: \emph{Arthur Schnitzler: Briefwechsel mit Autorinnen und Autoren}.
 Digitale Edition, https://schnitzler-briefe.acdh.oeaw.ac.at/{\dateiname}.html (Stand \today)
\fi

\end{document}


