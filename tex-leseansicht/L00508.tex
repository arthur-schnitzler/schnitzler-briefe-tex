%% latex-korrekturansicht-vorspann.tex
%% Vorspann für die Korrekturansicht.
%% Lädt die gemeinsame Datei latex-vorspann.tex mit gesetztem Schalter.

\newif\ifkorrekturansicht
\korrekturansichttrue

\input{../tex-inputs/latex-vorspann}


\section[Hugo von Hofmannsthal an Arthur Schnitzler, 17. 10. {[}1895{]}]{L00508 Hugo von Hofmannsthal an Arthur Schnitzler, 17. 10. {[}1895{]}}
\nopagebreak\mylabel{L00508v}
\rehead{ }\normalsize\beginnumbering\briefempfaengerindex{Schnitzler, Arthur@\textsc{Schnitzler, Arthur}!zzzHofmannsthal, Hugo von@\emph{von Hugo von Hofmannsthal}!1895-10-172@{17. 10. {[}1895{]}}|(be}
\toendnotes[C]{\smallbreak\pagebreak[2]}\Standort{CUL, Schnitzler, B 43.}
\physDesc{Brief, 1 Blatt, 3 Seiten, 976 Zeichen
\newline{}Handschrift: schwarze Tinte, deutsche Kurrent
\newline{}Schnitzler: mit Bleistift die Jahreszahl ergänzt: »95« und nummeriert: »76« }
\buchAbdrucke{\weitereDrucke{Hugo von Hofmannsthal, Arthur Schnitzler: \emph{Briefwechsel}. Frankfurt am Main: \emph{S. Fischer} 1964, S. 63.} }\toendnotes[C]{\smallbreak}
\pstart
           \raggedleft{}{\pb}Venedig\oindex{Venedig@\textbf{Venedig}, \emph{P.PPLA}|pw}{ }17. October\pend
           \vspace{0.5em}
\pstart
           am Sonntag{ }Früh hab ich Sie beſucht, aber nur 3 Frauen mit Beſen gefunden. Ich
               wollte Ihnen ſagen, daſs ich nach den Zeitungen und dem Reden der Leute wirklich
               glaube, daſs Sie jetzt dieſes unberechenbare und ſchwer zu definierende erworben
               haben, womit man Aufmerkſamkeit und Bewunderung erzwingen kann. Ich glaube, Sie
               dürfen ſich jetzt erlauben, für die Darſtellung {\pb}tiefer und kühner Dinge auf
               mehreren Beifall zu rechnen als bloß auf den von 3 oder 4 Freunden.\pend
           
\pstart
           Richard\pwindex{Beer-Hofmann, Richard 1866-07-11 – 1945-09-26@\textsc{Beer-Hofmann, Richard} (1866-07-11 – 1945-09-26), \emph{Schriftsteller/Schriftstellerin}|pw} hat mir die geſcheidte Kritik\pwindex{Burgtheater [Rechte der Seele, Liebelei]@\emph{Burgtheater [Rechte der Seele, Liebelei]}|pwv} von Berger\pwindex{Berger, Alfred von 30.04.1853 – 24.08.1912@\textsc{Berger, Alfred von} (30.04.1853 – 24.08.1912), \emph{Schriftsteller/Schriftstellerin, Journalist/Journalistin, Theaterleiter/Theaterleiterin}|pw} geſchickt und die \label{K_L00508-1v}\edtext{Verſpottung\pwindex{Jung-Wiener Dichter. (Zur Burgtheater-Premiere.)@\emph{Jung-Wiener Dichter. (Zur Burgtheater-Première.)}|pwv}}{\lemma{\textnormal{\emph{Verſpottung}}}\Cendnote{\textnormal{Der Reporter\pwindex{Der Reporter @\textsc{Der Reporter}, \emph{Journalist/Journalistin}|pwk}: \emph{Jung-Wiener Dichter. (Zur Burgtheater-Première.)}\pwindex{Jung-Wiener Dichter. (Zur Burgtheater-Premiere.)@\emph{Jung-Wiener Dichter. (Zur Burgtheater-Première.)}|pwk} In:
                        \emph{Extrapost}\pwindex{Extrapost. Unparteiische Montags-Zeitung@\emph{Extrapost. Unparteiische Montags-Zeitung}|pwk}, Jg. 14, Nr. 717,
                        14. 10. 1895, S. 1–2. Der Text geht nicht nur auf 
                     \emph{Liebelei}\pwindex{Liebelei. Schauspiel in drei Akten@\emph{Liebelei. Schauspiel in drei Akten}|pwk} ein, sondern auch auf Hofmannsthal\pwindex{Hofmannsthal, Hugo von 1874-02-01 – 1929-07-15@\textsc{Hofmannsthal, Hugo von} (1874-02-01 – 1929-07-15), \emph{Schriftsteller/Schriftstellerin}|pwk} und Beer-Hofmann\pwindex{Beer-Hofmann, Richard 1866-07-11 – 1945-09-26@\textsc{Beer-Hofmann, Richard} (1866-07-11 – 1945-09-26), \emph{Schriftsteller/Schriftstellerin}|pwk}.}}}\label{K_L00508-1} von dem Anonymen\pwindex{Der Reporter @\textsc{Der Reporter}, \emph{Journalist/Journalistin}|pw}. Iſt es der kleine Kraus\pwindex{Kraus, Karl 28.04.1874 – 12.06.1936@\textsc{Kraus, Karl} (28.04.1874 – 12.06.1936), \emph{Schriftsteller/Schriftstellerin, Publizist/Publizistin, Schriftsteller/Schriftstellerin}|pw}? Es hat mich unterhalten, ich wäre froh, wenn ſolche Sachen viel öfter
               geſchrieben würden und auch Caricaturen von uns gezeichnet. {\pb}Das wird ſich auch immer ſteigern
               je mutiger und beſſer wir werden; ich denke, von der Generation von Philologen und
               Dilettanten, die vor uns war, wirds nicht viel Verhöhnungen geben.\pend
           
\pstart
           Hier arbeit ich nicht, aber werds wohl nachher.\pend
           
\pstart
           Adieu. Herzlich Ihr{\\[\baselineskip]}\spacefill\mbox{Hugo.}\pend
           \leftskip=0em{}\selectlanguage{ngerman}\endnumbering\briefempfaengerindex{Schnitzler, Arthur@\textsc{Schnitzler, Arthur}!zzzHofmannsthal, Hugo von@\emph{von Hugo von Hofmannsthal}!1895-10-172@{17. 10. {[}1895{]}}|)be}\mylabel{L00508h}  \normalsize

\doendnotes{C}
\bigskip
\vfill

\clearpage

\footnotesize

\lohead{\textsc{register}}

% Definiere theindex-Environment komplett neu ohne reledmac
\makeatletter
\renewenvironment{theindex}{%
  \section*{\indexname}%
  \setlength{\parindent}{0pt}%
  \setlength{\parskip}{0pt plus 0.3pt}%
  \let\item\@idxitem
}{%
  \clearpage
}
\makeatother

\IfFileExists{\jobname-pw.ind}{\input{\jobname-pw.ind}}{}

\end{document}

      