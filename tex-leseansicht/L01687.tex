%% latex-leseansicht-vorspann.tex
%% Vorspann für die Leseansicht.
%% Lädt die gemeinsame Datei latex-vorspann.tex mit nicht gesetztem Schalter.

\newif\ifkorrekturansicht
\korrekturansichtfalse

\input{../tex-inputs/latex-vorspann}


\section[Arthur und Olga Schnitzler an Richard Beer-Hofmann, 28. 6. 1907]{L01687 Arthur und Olga Schnitzler an Richard Beer-Hofmann, 28. 6. 1907}
\nopagebreak\mylabel{L01687v}
\rehead{ }\normalsize\beginnumbering\briefempfaengerindex{Beer-Hofmann, Richard@\textsc{Beer-Hofmann, Richard}!zzzSchnitzler, Olga@\emph{von Olga Schnitzler}!1907-06-281@{28. 6. 1907}|(be}\briefempfaengerindex{Beer-Hofmann, Richard@\textsc{Beer-Hofmann, Richard}!zzzSchnitzler, Arthur@\emph{von Arthur Schnitzler}!1907-06-281@{28. 6. 1907}|(be}
\toendnotes[C]{\smallbreak\pagebreak[2]}
\correspDesc{Versand  durch Arthur Schnitzler, Olga Schnitzler am 28. 6. 1907 in Veldes
\newline{}Übermittlung  am 29. 6. 1907 in Veldes
\newline{}Erhalt  durch Richard Beer-Hofmann im Zeitraum [29. 6. 1907
                  – 3. 7. 1907?] in Wien}\toendnotes[C]{\smallbreak}
\Standort{YCGL, MSS 31.}
\physDesc{Bildpostkarte, 115 Zeichen
\newline{}Handschrift Arthur Schnitzler: Bleistift, deutsche Kurrent
\newline{}Handschrift Olga Schnitzler: Bleistift
\newline{}Versand: Stempel: »\nobreak{}\oindex{St. Johann am Wocheiner See@\textbf{St. Johann am Wocheiner See}, \emph{Kirche}|pwk}{[}St. Jo{]}hann am Woche{[}iner See –
                                          Sv. J{]}anez pri Jezeru, 2\textcolor{gray}{9} 6 0{[}7{]}\nobreak{}«.  
\newline{}Ordnung: mit Bleistift von unbekannter Hand datiert: »28. 6.« }\pstart{}\textsc{{\pb}Dr Richard Beer-Hofmann}\pend{}\pstart{}\textsc{Wien XVIII\oindex{XVIII., Währing@\textbf{XVIII., Währing}, \emph{Verwaltungsgebiet}|pw}}\pend{}\pstart{}\textsc{Hasenauerstr. 59\oindex{Wien@\textbf{Wien}!XVIII., Währing@\textbf{XVIII., Währing}!Hasenauerstraße 59@\textbf{Hasenauerstraße 59}, \emph{Wohngebäude}|pw}.}\pend{}{\bigskip}
\pstart
           \noindent{}\centering{}{\pb}\textcolor{gray}{\textbf{Der Woch. See\oindex{Wocheiner See@\textbf{Wocheiner See}, \emph{See}|pw}}}\pend
           \vspace{1em}
\pstart
           \noindent{}{\pb}Herzliche Grüße!\pend
           \pstart Ihr \spacefill\mbox{Arthur}\pend{}
\pstart
           28. 6. 907\pend
           
\pstart
           \textsc{St. Johann, Wocheinersee}\oindex{St. Johann am Wocheiner See@\textbf{St. Johann am Wocheiner See}, \emph{Kirche}|pw}\pend
           \selectlanguage{ngerman}\vspace{1em}\pstart \spacefill\mbox{{[}hs. Schnitzler:{]} OlgaS.}\pend{}\selectlanguage{ngerman}\endnumbering\briefempfaengerindex{Beer-Hofmann, Richard@\textsc{Beer-Hofmann, Richard}!zzzSchnitzler, Olga@\emph{von Olga Schnitzler}!1907-06-281@{28. 6. 1907}|)be}\briefempfaengerindex{Beer-Hofmann, Richard@\textsc{Beer-Hofmann, Richard}!zzzSchnitzler, Arthur@\emph{von Arthur Schnitzler}!1907-06-281@{28. 6. 1907}|)be}\mylabel{L01687h}  \newcommand{\dateiname}{L01687}\newcommand{\titel}{Arthur und Olga Schnitzler an Richard Beer-Hofmann, 28. 6. 1907}\newcommand{\editorInnen}{Martin Anton Müller und Gerd-Hermann Susen}%% latex-leseansicht-abspann.tex
%% Abspann für die Leseansicht.
%% Der Schalter \ifkorrekturansicht ist bereits durch den Vorspann gesetzt.

%% latex-abspann.tex
%% Gemeinsamer Abspann für Korrekturansicht und Leseansicht.
%% Setzt den Schalter \ifkorrekturansicht voraus (gesetzt in den
%% einbindenden Dateien latex-korrekturansicht-abspann.tex bzw.
%% latex-leseansicht-abspann.tex).
%% ---------------------------------------------------------------

\normalsize

% Das esempio-Environment wird nur in der Leseansicht benötigt
\ifkorrekturansicht\else
\newenvironment{esempio}[3]%
{
    \vspace{1.5ex}
    \rlap{\underline{#1}}
    \par
    \setlength{\parindent}{0cm}
    \nopagebreak
    \leftskip=#2cm
    \rightskip=#3cm
}
{
    \par
}
\fi

\doendnotes{C}
\bigskip
\vfill

\clearpage

\footnotesize

\ifkorrekturansicht
  \lohead{\textsc{register}}
\fi

% theindex-Environment neu definieren ohne reledmac
\makeatletter
\renewenvironment{theindex}{%
  \ifkorrekturansicht
    \section*{\indexname}%
  \else
    \subsubsection*{Index der erwähnten Entitäten}%
  \fi
  \setlength{\parindent}{0pt}%
  \setlength{\parskip}{0pt plus 0.3pt}%
  \let\item\@idxitem
}{%
  \ifkorrekturansicht\clearpage\fi
}
\makeatother

\IfFileExists{\jobname-pw.ind}{\input{\jobname-pw.ind}}{}

% Quellenangabe nur in der Leseansicht
\ifkorrekturansicht\else
% Fallback-Definitionen, falls die .tex-Datei \titel etc. nicht gesetzt hat
\providecommand{\titel}{}
\providecommand{\editorInnen}{}
\providecommand{\dateiname}{\jobname}

\vspace{3cm}

\vfill

\footnotesize
\textsc{Quelle}: \titel. Herausgegeben von {\editorInnen}. In: \emph{Arthur Schnitzler: Briefwechsel mit Autorinnen und Autoren}.
 Digitale Edition, https://schnitzler-briefe.acdh.oeaw.ac.at/{\dateiname}.html (Stand \today)
\fi

\end{document}


