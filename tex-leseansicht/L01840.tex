%% latex-leseansicht-vorspann.tex
%% Vorspann für die Leseansicht.
%% Lädt die gemeinsame Datei latex-vorspann.tex mit nicht gesetztem Schalter.

\newif\ifkorrekturansicht
\korrekturansichtfalse

\input{../tex-inputs/latex-vorspann}


         
         \renewcommand{\erwaehntePersonen}{Personen: Maurice Donnay}
         \renewcommand{\erwaehnteInstitutionen}{Institutionen: Wiener Freie Volksbühne}
         \renewcommand{\erwaehnteOrte}{Orte: Königseggasse, Lustspieltheater (Wien), Ottakringerstraße, VI., Mariahilf, Verbandsheim, Wien}
         \renewcommand{\erwaehnteWerke}{Werke: Lysistrata}
               \section[Albert Ehrenstein an Arthur Schnitzler, 6. 5. 1909]{ Albert Ehrenstein an Arthur Schnitzler, 6. 5. 1909}\nopagebreak\mylabel{v}\rehead{ }\begin{ledgroupsized}[t]{13cm}\normalsize\beginnumbering \toendnotes[C]{\smallbreak\pagebreak[2]} \Standort{CUL, Schnitzler, B 30.}
\physDesc{Brief, 1 Blatt, 4 Seiten, 2716 Zeichen
\newline{}Handschrift: schwarze Tinte, deutsche Kurrent
\newline{}Schnitzler: mit Bleistift beschriftet: »\textsc{Ehrenstein}« }\buchAbdrucke{\weitereDrucke{Albert Ehrenstein: \emph{Briefe}. Hg. Hanni Mittelmann. München: \emph{Boer} 1989, S. 29–30 (Werke, 1).} }\toendnotes[C]{\smallbreak}\pstart
           {\pb}\textsc{Wien, XVI. Ottakringerstr. 114\oindex{Ottakringerstrasse@\textbf{Ottakringerstraße}|pw}}.\hfill \textsc{6. Mai. 09}.\pend
           \pstart{}\textsc{Sehr geehrter Herr Doktor!}\pend\pstart
           Wenn ich keinen Zwicker trage (und aus Eitelkeit trage ich meiſtens keinen), ſo bin
               ich recht kurzſichtig; überdies und auch dann iſt mein Perſonengedächtnis ein
               ziemlich mangelhaftes und geſtörtes, warum? Darüber möchte ich gerne etwas näheres
               erfahren. Jedenfalls haben ſich meine Augen ſchon manchen Ulk mit mir erlaubt, die
               ärgerlichſten und gröbſten Verwechslungen ſind mir zugestoßen. Die anfänglich
               vorhanden geweſene Geneigtheit, jede Agnoszierung ohne weiteres für wichtig
               anzuſehen, iſt infolgedeſſen einem ſo zweifelſüchtigen Mißtrauen gegen alle
               Wahrnehmung gewichen, daß es mir nur ſehr ſelten gelingt, einen Begegnenden richtig
               zu identifizieren oder gar ſtets davon überzeigt zu ſein. Wie ich glaube, iſt mir ein
               derartiges Malheur ſchon einmal Ihnen gegenüber, ſehr geehrter \introOben{}Herr\introOben{} Doktor, paſſiert, in einer Tramway nach der \label{K_L01840_1v}\edtext{Premiere}{\lemma{\textnormal{\emph{Premiere}}}\Cendnote{\textnormal{Am 29. 12. 1906 im Lustspieltheater\oindex{Lustspieltheater (Wien)@\textbf{Lustspieltheater (Wien)}|pwk} in Wien\oindex{Wien@\textbf{Wien}|pwk}, Schnitzler\pwindex{Schnitzler, Arthur 15.05.1862 – 21.10.1931@\textsc{Schnitzler, Arthur} (15.05.1862 – 21.10.1931), \emph{Schriftsteller, Mediziner}|pwk} war nicht bei der Premiere.}}}\label{K_L01840_1h}
               der Donnay\pwindex{Donnay, Maurice 12.10.1859 – 31.03.1945@\textsc{Donnay, Maurice} (12.10.1859 – 31.03.1945), \emph{Schriftsteller}|pw}’ſchen {\pb}Lyſiſtrata\pwindex{Donnay, Maurice 12.10.1859 – 31.03.1945@\textsc{Donnay, Maurice} (12.10.1859 – 31.03.1945), \emph{Schriftsteller}!Lysistrata1897@\strich\emph{Lysistrata} {[}1897{]}|pw}. Ein anderesmal nach einer \label{K_L01840_2v}\edtext{Vorleſung im Mariahilf\oindex{VI., Mariahilf@\textbf{VI., Mariahilf}|pw}er Arbeiterheim}{\lemma{\textnormal{\emph{Vorleſung … Arbeiterheim}}}\Cendnote{\textnormal{Gemeint
                  ist die Vorlesung am 16. 10. 1907 für die \emph{Wiener Freie Volksbühne}\orgindex{Wiener Freie Volksbuehne@Wiener Freie Volksbühne|pwk} im sozialdemokratischen Verbandsheim\oindex{Verbandsheim@\textbf{Verbandsheim}|pwk} in der Königseggasse 10\oindex{Koenigseggasse@\textbf{Königseggasse}|pwk}.}}}\label{K_L01840_2h} verſchlug mir die Befangenheit jeden
               Gruß. Ein gewiſſer kindlicher und doch dämoniſcher Trotz und Eigenſinn verbietet es,
               wenn man ſich von der ersten Lähmung des Willens erholt hat, baldmöglichſt den Fehler
               gutzumachen. Nach dem Geſetz der Trägheit geht man den einmal genommenen Weg
               verdroſſen oder ratlos weiter, und bevor man ſich von der Überrumplung durch die
               ſelbſtverſchuldeten Ereigniſſe freigemacht hat, ſagt man ſicher »Jetzt iſt ſchon
               alles gleichgültig.« Ich würde derartige Erlebniſſe trotz ihrer Wiederkehr gewiß
               nicht ſo tragiſch nehmen, wenn ich nicht wüßte, wie ſehr derartige
               Unterlaſſungsſünden dem Selbſtvernichtungstriebe entſprechen, krankhaftes Benehmen
               und davon Betroffenen nicht gerade das Leben erleichtert. Das ſchlechter werdende
               Gehör trägt auch nicht dazu bei, {\pb}die Lage
               angenehmer zu machen, verſäumte Grüße ſummierten ſich mit oft wider Willen
               emporgefahrenen biſſigen Antworten auf falsch verſtandenen Bemerkungen, und entrißen
               mir die wenigen Freunde. Es iſt eben ſelbſt der Teilnahmsvollſte nicht immer in der
               Stimmung, kurzſichtigen Unverſtand von Hochmut, Eigentümlichkeit und Schrullen von
               Überhebung zu ſondern. Sollte Mittwoch, den 5. Mai um 9\textsuperscript{h} früh meinerſeits Ihnen gegenüber eine Kette
               neuerlicher Verſtöße oder Sinnestäuſchungen vorgefallen ſein, ſo wäre es mir ſehr
               lieb, wenn ich von allerhand quälenden Betrachtungen befreit würde. Faſt ſcheint es
               ſo, als ſtellte ich die unmöglichſten Dinge bloß zu dem Zwecke an, auch nachträglich
               entſchuldigen zu können. Nie tat ich das Plauſible, ſeit jeher ſchon war ich mir
               ziemlich wehrlos ausgeſetzt, und wenn es irgend anginge, zöge ich {\pb}wahrhaftig mit größtem Vergnügen aus mir
               aus.\pend
           \pstart
           Hochachtungsvoll{\\[\baselineskip]}Ihr ergebenſter{\\[\baselineskip]}\spacefill\mbox{Albert Ehrenstein.}\pend
           \leftskip=0em{}
         
         \endnumbering\mylabel{h}\end{ledgroupsized}  \newcommand{\dateiname}{L01840}\newcommand{\titel}{Albert Ehrenstein an Arthur Schnitzler, 6. 5. 1909}\newcommand{\editorInnen}{Martin Anton Müller und Gerd-Hermann Susen}%% latex-leseansicht-abspann.tex
%% Abspann für die Leseansicht.
%% Der Schalter \ifkorrekturansicht ist bereits durch den Vorspann gesetzt.

%% latex-abspann.tex
%% Gemeinsamer Abspann für Korrekturansicht und Leseansicht.
%% Setzt den Schalter \ifkorrekturansicht voraus (gesetzt in den
%% einbindenden Dateien latex-korrekturansicht-abspann.tex bzw.
%% latex-leseansicht-abspann.tex).
%% ---------------------------------------------------------------

\normalsize

% Das esempio-Environment wird nur in der Leseansicht benötigt
\ifkorrekturansicht\else
\newenvironment{esempio}[3]%
{
    \vspace{1.5ex}
    \rlap{\underline{#1}}
    \par
    \setlength{\parindent}{0cm}
    \nopagebreak
    \leftskip=#2cm
    \rightskip=#3cm
}
{
    \par
}
\fi

\doendnotes{C}
\bigskip
\vfill

\clearpage

\footnotesize

\ifkorrekturansicht
  \lohead{\textsc{register}}
\fi

% theindex-Environment neu definieren ohne reledmac
\makeatletter
\renewenvironment{theindex}{%
  \ifkorrekturansicht
    \section*{\indexname}%
  \else
    \subsubsection*{Index der erwähnten Entitäten}%
  \fi
  \setlength{\parindent}{0pt}%
  \setlength{\parskip}{0pt plus 0.3pt}%
  \let\item\@idxitem
}{%
  \ifkorrekturansicht\clearpage\fi
}
\makeatother

\IfFileExists{\jobname-pw.ind}{\input{\jobname-pw.ind}}{}

% Quellenangabe nur in der Leseansicht
\ifkorrekturansicht\else
% Fallback-Definitionen, falls die .tex-Datei \titel etc. nicht gesetzt hat
\providecommand{\titel}{}
\providecommand{\editorInnen}{}
\providecommand{\dateiname}{\jobname}

\vspace{3cm}

\vfill

\footnotesize
\textsc{Quelle}: \titel. Herausgegeben von {\editorInnen}. In: \emph{Arthur Schnitzler: Briefwechsel mit Autorinnen und Autoren}.
 Digitale Edition, https://schnitzler-briefe.acdh.oeaw.ac.at/{\dateiname}.html (Stand \today)
\fi

\end{document}


      