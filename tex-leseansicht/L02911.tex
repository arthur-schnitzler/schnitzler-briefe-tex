%% latex-korrekturansicht-vorspann.tex
%% Vorspann für die Korrekturansicht.
%% Lädt die gemeinsame Datei latex-vorspann.tex mit gesetztem Schalter.

\newif\ifkorrekturansicht
\korrekturansichttrue

\input{../tex-inputs/latex-vorspann}


\section[ Paul Goldmann an Arthur Schnitzler, 18. 4. {[}1900{]}]{L02911 Paul Goldmann an Arthur Schnitzler, 18. 4. {[}1900{]}}
\nopagebreak\mylabel{L02911v}
\rehead{ }\normalsize\beginnumbering\briefempfaengerindex{Schnitzler, Arthur@\textsc{Schnitzler, Arthur}!zzzGoldmann, Paul@\emph{von Paul Goldmann}!1900-04-181@{18. 4. {[}1900{]}}|(be}
\toendnotes[C]{\smallbreak\pagebreak[2]}\Standort{DLA, A:Schnitzler, HS.NZ85.1.3170.}
\physDesc{Brief, 2 Blätter, 8 Seiten, 4263 Zeichen
\newline{}Handschrift: blaue Tinte, deutsche Kurrent
\newline{}Schnitzler: 1) mit Bleistift das Jahr »900« vermerkt  2) mit rotem Buntstift fünf Unterstreichungen}\toendnotes[C]{\smallbreak}
\pstart
           {\pb}\textcolor{gray}{\textbf{DESSAUERSTRASSE 19}}\oindex{Dessauer Strasse@\textbf{Dessauer Straße}, \emph{Straße (K.STR)}|pw}\pend
           
\pstart
           \raggedleft{}Berlin\oindex{Berlin@\textbf{Berlin}, \emph{P.PPLC}|pw}, 18. April.\pend
           
\pstart{}Mein lieber Freund,\pend\vspace{0.5em}
\pstart
           Ich habe mich ſehr mit Deinem lieben Briefe gefreut. Lange habe ich ihn erwartet und
               wußte mir gar nicht zu erklären, warum ich ſo ganz ohne Nachricht blieb. Ich war \strikeout{\textcolor{gray}{auc}h} zum \label{K_L02911-1v}\edtext{\textsc{Speidel\pwindex{Speidel, Ludwig 1830-04-11 – 1906-02-03@\textsc{Speidel, Ludwig} (1830-04-11 – 1906-02-03), \emph{Journalist/Journalistin, Kritiker/Kritikerin}|pw}}-Banket}{\lemma{\textnormal{\emph{Speidel-Banket}}}\Cendnote{\textnormal{Am Nachmittag des 15. 4. 1900 fand ein
                  großes Bankett anlässlich des 70. Geburtstags von Ludwig Speidel\pwindex{Speidel, Ludwig 1830-04-11 – 1906-02-03@\textsc{Speidel, Ludwig} (1830-04-11 – 1906-02-03), \emph{Journalist/Journalistin, Kritiker/Kritikerin}|pwk} statt. Schnitzler war einer der über 50 Teilnehmenden aus dem Kulturbetrieb.
                     »Widerwärtig«, notierte er sich dazu im \emph{Tagebuch}\pwindex{Tagebuch@\emph{Tagebuch}|pwk}.}}}\label{K_L02911-1} geladen und hätte darum ſehr gut nach Wien\oindex{Wien@\textbf{Wien}, \emph{A.ADM2}|pw} kommen können, und die N. Fr. Pr.\orgindex{Neue Freie Presse@Neue Freie Presse|pw} hätte mir überdies die Reiſe bezahlen müſſen. Aber
               wenn ich nach Wien\oindex{Wien@\textbf{Wien}, \emph{A.ADM2}|pw} komme, ſo komme ich
               Deinetwegen. Und da ich ſo gar nichts von Dir hörte, ........ Aber laſſen wir das! Mir hat meine Hypochondrie wieder einmal \strikeout{\textcolor{gray}{×}} einen Streich geſpielt, und es thut mir nun doppelt leid, um die ſchönen
               Oſtertage gekommen zu ſein, {\pb}die ich mit Dir hätte
               verleben können.\pend
           
\pstart
           Was Deine \label{K_L02911-2v}\edtext{Furcht vor dem
                  Altwerden}{\lemma{\textnormal{\emph{Furcht vor dem
                  Altwerden}}}\Cendnote{\textnormal{In wenigen Tagen, am 15. 5. 1900, sollte
                     Schnitzler seinen 38. Geburtstag
                  begehen.}}}\label{K_L02911-2} anlangt, – nein, wirklich, mit 38 Jahren iſt man noch nicht alt.
               Und wenn Du Dir das früher einmal als das Ende aller Dinge vorgeſtellt haſt, ſo haſt
               Du eben früher das Leben nicht gekannt, wie man ja ſo Manches ſich unrichtig
               vorſtellt, wenn man gar zu jung iſt. Früher haben Dich die Frauen geliebt, weil
               Du 20 Jahre alt warſt; jetzt haben ſie viel mehr Gründe, Dich zu lieben, und dabei
               biſt Du immer noch jung genug, daß es ihnen Vergnügen macht. Die Geliebten, die Dich
               ſeinerzeit durch \introOben{}den\introOben{} Hinweis \strikeout{auf
                  ihre} beruhigt haben, daß ihre anderen Anbeter Ende der Dreißig ſeien, haben
               dieſen Anderen wahrſcheinlich mit Hinweis auf Dich geſagt: »Das iſt {\pb}ein unreifer Junge. Lieben aber kann man nur einen
               wirklichen Mann.« Wie alt, glaubſt Du, war \textsc{Don Juan}?
               Jedenfalls nicht zwanzig Jahre. Meiner Anſicht nach hatte er zwiſchen 35 und 40, wenn
               nicht darüber{\dotsseven}\pend
           
\pstart
           Auf Deine \label{K_L02911-3v}\edtext{Novelle\pwindex{Frau Bertha Garlan. Roman@\emph{Frau Bertha Garlan. Roman}|pwv}}{\lemma{\textnormal{\emph{Novelle}}}\Cendnote{\textnormal{Schnitzler hatte \emph{Frau Bertha Garlan}\pwindex{Frau Bertha Garlan. Roman@\emph{Frau Bertha Garlan. Roman}|pwk} am 1. 1. 1900 begonnen und am 16. 4. 1900
                  fertiggestellt.}}}\label{K_L02911-3} freue ich mich ſehr. Was wird eigentlich aus der \textsc{Beatrice\pwindex{Schleier der Beatrice. Schauspiel in fuenf Akten@\emph{Der Schleier der Beatrice. Schauspiel in fünf Akten}|pw}}? Wann beginnen die \label{K_L02911-4v}\edtext{Proben}{\lemma{\textnormal{\emph{Proben}}}\Cendnote{\textnormal{Schnitzler glaubte zu diesem Zeitpunkt noch,
                  dass das Stück\pwindex{Schleier der Beatrice. Schauspiel in fuenf Akten@\emph{Der Schleier der Beatrice. Schauspiel in fünf Akten}|pwkv} am \emph{Burgtheater}\orgindex{Burgtheater@Burgtheater|pwk} aufgeführt werden sollte. Siehe Paul Goldmann an Arthur Schnitzler, 12. 11. [1899].}}}\label{K_L02911-4}?\pend
           
\pstart
           Wie beneide ich Dich um Dein Arbeiten! Ich ſelbſt bringe es nicht zu Stande. Ich habe
               jetzt, nach Wochen angeſpannteſter Arbeit, auch wieder Wochen faſt vollkommener Ruhe.
               Das wäre die Zeit, etwas zu ſchaffen. Ich zermartere mir den Kopf, will heut ein
               Drama ſchreiben, morgen eine Novelle. Aber Alles {\pb}zerrinnt wieder im Nebel. Und ich vergeude meine Zeit mit Beſuchen, mit
               überflüſſiger Reporter-Arbeit und Anderem, wie ja überhaupt der Journalismus eine
               große Zeitvertrödelung iſt. Dabei habe ich das Gefühl, es ſteckt doch noch etwas mehr
               in mir. Aber ich weiß nicht, was ich will. Ich würde Denjengen\strikeout{,} wie einen Erlöſer begrüßen, der mir einen Rath geben,
               mich auf eine größere Arbeit hinweiſen würde, die \strikeout{\textcolor{gray}{me}i\textcolor{gray}{n}} meinen Fähigkeiten entſpräche. Aber, ich weiß, dieſen Rath kann man ſich nur
               ſelbſt geben. Und bei mir finde ich keinen. Ich habe mich ſelten innerlich ſo elend
               gefühlt, mich ſelten ſo verachtet. Große Prätentionen, und innerlich {\pb}Alles leer, le{[}e{]}r! Meine einzige
               Leiſtung iſt, daß ich täglich fetter werde{\dotsfour}\pend
           
\pstart
           Im Sommer werde ich wohl meinen Urlaub bekommen. Aber ich werde ihn in Berlin\oindex{Berlin@\textbf{Berlin}, \emph{P.PPLC}|pw} verbringen müſſen, weil ich diesmal keine
               fünf Mark übrig haben werde, um zu reiſen. Der Hausſtand, den ich hier mit meiner Mutter\pwindex{Goldmann, Clementine 1842-05-15 – 1924-02-24@\textsc{Goldmann, Clementine} (1842-05-15 – 1924-02-24)|pwv} führe, \strikeout{ver} nimmt faſt mein ganzes Gehalt in Anſpruch. Der Reſt
               geht für Schulden-Abzahlungen aller Art drauf; und Nebenverdienſt iſt ausgeſchloſſen.
               Nach \label{K_L02911-5v}\edtext{\textsc{Paris\oindex{Paris@\textbf{Paris}, \emph{P.PPLC}|pw}}}{\lemma{\textnormal{\emph{Paris}}}\Cendnote{\textnormal{Schnitzler dürfte sich erkundigt haben, ob
                     Goldmann\pwindex{Goldmann, Paul 31.01.1865 – 25.09.1935@\textsc{Goldmann, Paul} (31.01.1865 – 25.09.1935), \emph{Schriftsteller/Schriftstellerin, Journalist/Journalistin}|pwk} zur Weltausstellung nach Paris\oindex{Paris@\textbf{Paris}, \emph{P.PPLC}|pwk} (15. 4. 1900 – 12. 11. 1900) zu fahren
                  gedachte.}}}\label{K_L02911-5} fahre ich unter dieſen Umſtänden natürlich nicht.\pend
           
\pstart
           {\pb}Kennſt Du \label{K_L02911-6v}\edtext{\textsc{Flauberts\pwindex{Flaubert, Gustave 12.12.1821 – 08.05.1880@\textsc{Flaubert, Gustave} (12.12.1821 – 08.05.1880), \emph{Schriftsteller/Schriftstellerin}|pw}}{ }Briefe\pwindex{Correspondance. 4 Bde.@\emph{Correspondance. 4 Bde.}|pwv}}{\lemma{\textnormal{\emph{Flauberts Briefe}}}\Cendnote{\textnormal{Gustave Flaubert\pwindex{Flaubert, Gustave 12.12.1821 – 08.05.1880@\textsc{Flaubert, Gustave} (12.12.1821 – 08.05.1880), \emph{Schriftsteller/Schriftstellerin}|pwk}: \emph{Correspondance}\pwindex{Correspondance. 4 Bde.@\emph{Correspondance. 4 Bde.}|pwk}. 4 Bde. Paris\oindex{Paris@\textbf{Paris}, \emph{P.PPLC}|pwk}: \emph{Charpentier {\kaufmannsund} Cie}\orgindex{Charpentier@Charpentier|pwk}{ }1887–1893. Schnitzler kannte zumindest eine
                  spätere Ausgabe\pwindex{Correspondance@\emph{Correspondance}|pwkv} (vgl. A. S.: \emph{Lektüren}, Frankreich).}}}\label{K_L02911-6}? Wenn
               nicht, ſo mußt Du ſie gleich leſen, und zwar gleich den dritten und vierten Band\pwindex{Correspondance. 4 Bde.@\emph{Correspondance. 4 Bde.}|pwv}; die Jugendbriefe\pwindex{Correspondance. 4 Bde.@\emph{Correspondance. 4 Bde.}|pwv} in den erſten beiden ſind
               nicht intereſſant. Ich habe ſie jetzt wieder vorgeholt. Jeder Menſch, der ſchreibt,
                  \strikeout{muß} findet darin Troſt, Befreiung und Belehrung.
               Auf dem ſpeciell ſchriftſtelleriſchen Gebiete geben ſie Einem faſt ſo viel, wie
                  \label{K_L02911-7v}\edtext{Goethes\pwindex{Goethe, Johann Wolfgang von 1749-08-28 – 1832-03-22@\textsc{Goethe, Johann Wolfgang von} (1749-08-28 – 1832-03-22), \emph{Schriftsteller/Schriftstellerin}|pw}s Geſpräche\pwindex{Goethes Unterhaltungen mit dem Kanzler Friedrich von Mueller@\emph{Goethes Unterhaltungen mit dem Kanzler Friedrich von Müller}|pwv}}{\lemma{\textnormal{\emph{Goethess Geſpräche}}}\Cendnote{\textnormal{Siehe Paul Goldmann an Arthur Schnitzler, 25. 9. [1899].
               }}}\label{K_L02911-7}; nur ſind ſie nicht ſo univerſell menſchlich, wie dieſe. \textsc{Flaubert\pwindex{Flaubert, Gustave 12.12.1821 – 08.05.1880@\textsc{Flaubert, Gustave} (12.12.1821 – 08.05.1880), \emph{Schriftsteller/Schriftstellerin}|pw}} iſt eben doch kein Menſch, ſondern \strikeout{nur} nur ein
                  Fran\oindex{Frankreich@\textbf{Frankreich}, \emph{A.PCLI}|pwv}zoſe{\dotsfour}\pend
           
\pstart
           Von \textsc{Gusti\pwindex{Gluemer, Auguste 1862-03-16 – 1956@\textsc{Glümer, Auguste} (1862-03-16 – 1956), \emph{Lehrer/Lehrerin}|pw}} weiß\strikeout{’} ich Dir nichts {\pb}zu berichten. Das eigentliche Leben der beiden Mädels\pwindex{Gluemer, Auguste 1862-03-16 – 1956@\textsc{Glümer, Auguste} (1862-03-16 – 1956), \emph{Lehrer/Lehrerin}|pwv}\pwindex{Gluemer, Marie 03.07.1867 – 16.11.1925@\textsc{Glümer, Marie} (03.07.1867 – 16.11.1925), \emph{Schauspieler/Schauspielerin}|pwv} bleibt mir
               verſchloſſen. Trotz aller Herzlichkeit der Beziehungen beſteht zwiſchen uns doch
               keine rechte Sympathie, und innerlich ſtehen wir uns fremd gegenüber.\pend
           
\pstart
           Was macht \label{K_L02911-8v}\edtext{\textsc{Richard\pwindex{Beer-Hofmann, Richard 1866-07-11 – 1945-09-26@\textsc{Beer-Hofmann, Richard} (1866-07-11 – 1945-09-26), \emph{Schriftsteller/Schriftstellerin}|pw}}}{\lemma{\textnormal{\emph{Richard}}}\Cendnote{\textnormal{Goldmann\pwindex{Goldmann, Paul 31.01.1865 – 25.09.1935@\textsc{Goldmann, Paul} (31.01.1865 – 25.09.1935), \emph{Schriftsteller/Schriftstellerin, Journalist/Journalistin}|pwk} bezog sich auf Beer-Hofmanns\pwindex{Beer-Hofmann, Richard 1866-07-11 – 1945-09-26@\textsc{Beer-Hofmann, Richard} (1866-07-11 – 1945-09-26), \emph{Schriftsteller/Schriftstellerin}|pwk} Trauerspiel \emph{Der Graf von Charolais}\pwindex{Graf von Charolais. Ein Trauerspiel@\emph{Der Graf von Charolais. Ein Trauerspiel}|pwk}, an dem dieser bereits seit 1899 arbeitete. Zu Beer-Hofmanns\pwindex{Beer-Hofmann, Richard 1866-07-11 – 1945-09-26@\textsc{Beer-Hofmann, Richard} (1866-07-11 – 1945-09-26), \emph{Schriftsteller/Schriftstellerin}|pwk}
                  Reisen im Sommer 1900 siehe Eugene Weber: \emph{Richard Beer-Hofmann: Daten mitgeteilt von Eugene Weber}.
                     In: \emph{Modern Austrian Literature} 17/2 (1984), S. 13–42, hier: S. 23.}}}\label{K_L02911-8}? Arbeitet er an ſeinem
                  Drama\pwindex{Graf von Charolais. Ein Trauerspiel@\emph{Der Graf von Charolais. Ein Trauerspiel}|pwv}? Und
               was wird er im Sommer machen? Wirſt Du mit ihm zuſammen ſein?\pend
           
\pstart
           Geſtern ſprach ich wieder einmal \textsc{Kerr\pwindex{Kerr, Alfred 25.12.1867 – 12.10.1948@\textsc{Kerr, Alfred} (25.12.1867 – 12.10.1948), \emph{Schriftsteller/Schriftstellerin, Kritiker/Kritikerin}|pw}} nach langer Pauſe. Er ſcheint eine \label{K_L02911-9v}\edtext{große Liebe\pwindex{Wendt, Anna @\textsc{Wendt, Anna}|pwv}}{\lemma{\textnormal{\emph{große Liebe}}}\Cendnote{\textnormal{Bezug auf Anna Wendt\pwindex{Wendt, Anna @\textsc{Wendt, Anna}|pwk}, die Alfred
                     Kerr\pwindex{Kerr, Alfred 25.12.1867 – 12.10.1948@\textsc{Kerr, Alfred} (25.12.1867 – 12.10.1948), \emph{Schriftsteller/Schriftstellerin, Kritiker/Kritikerin}|pwk} im April 1900 kennengelernt hatte (vgl.
                     Deborah Vietor-Engländer: \emph{Alfred Kerr. Die
                        Biographie}. Reinbek bei Hamburg:
                        \emph{Rowohlt}{ }2016, S. 229 [E-Book-Ausgabe]).}}}\label{K_L02911-9} zu
               haben. Ich mag ihn ſehr gern trotz mancher Geſchmack-Defekte; aber er ſchließt ſich
               mir nicht auf. {\pb}Und wir bleiben fremd.\pend
           
\pstart
           Wann ſehe ich Dich wieder? Wann kommſt Du nach \label{K_L02911-10v}\edtext{Berlin\oindex{Berlin@\textbf{Berlin}, \emph{P.PPLC}|pw}}{\lemma{\textnormal{\emph{Berlin}}}\Cendnote{\textnormal{Siehe Paul Goldmann an Arthur Schnitzler, 13. 4. [1900].
               }}}\label{K_L02911-10}?\pend
           
\pstart
           Viele treue Grüße! {\\[\baselineskip]}Dein {\\[\baselineskip]}\spacefill\mbox{Paul Goldmann}\pend
           \leftskip=0em{}
\pstart
           \noindent{}Meine Mutter\pwindex{Goldmann, Clementine 1842-05-15 – 1924-02-24@\textsc{Goldmann, Clementine} (1842-05-15 – 1924-02-24)|pwv} dankt für
                  Deine Grüße und erwidert ſie herzlichſt.\pend
           \selectlanguage{ngerman}\endnumbering\briefempfaengerindex{Schnitzler, Arthur@\textsc{Schnitzler, Arthur}!zzzGoldmann, Paul@\emph{von Paul Goldmann}!1900-04-181@{18. 4. {[}1900{]}}|)be}\mylabel{L02911h}  \normalsize

\doendnotes{C}
\bigskip
\vfill

\clearpage

\footnotesize

\lohead{\textsc{register}}

% Definiere theindex-Environment komplett neu ohne reledmac
\makeatletter
\renewenvironment{theindex}{%
  \section*{\indexname}%
  \setlength{\parindent}{0pt}%
  \setlength{\parskip}{0pt plus 0.3pt}%
  \let\item\@idxitem
}{%
  \clearpage
}
\makeatother

\IfFileExists{\jobname-pw.ind}{\input{\jobname-pw.ind}}{}

\end{document}

      