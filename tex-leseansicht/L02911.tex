%% latex-leseansicht-vorspann.tex
%% Vorspann für die Leseansicht.
%% Lädt die gemeinsame Datei latex-vorspann.tex mit nicht gesetztem Schalter.

\newif\ifkorrekturansicht
\korrekturansichtfalse

\input{../tex-inputs/latex-vorspann}

\begin{center}
            \textcolor{red}{ENTWURF, NICHT FERTIG KORRIGIERT}
                      \end{center}
            
         
         \newcommand{\erwaehntePersonen}{Personen: Richard Beer-Hofmann, Auguste Chlum, Gustave Flaubert, Marie Glümer, Johann Wolfgang von Goethe, Clementine Goldmann, Alfred Kerr, Ludwig Speidel, Anna Wendt}
         \newcommand{\erwaehnteInstitutionen}{Institutionen: Charpentier, Neue Freie Presse}
         \newcommand{\erwaehnteOrte}{Orte: Berlin, Dessauer Straße, Frankreich, Paris, Wien}
         \newcommand{\erwaehnteWerke}{Werke: Correspondance, Correspondance. 4 Bde., Der Graf von Charolais. Ein Trauerspiel, Der Schleier der Beatrice. Schauspiel in fünf Akten, Frau Bertha Garlan. Roman, Goethes Unterhaltungen mit dem Kanzler Friedrich von Müller, Tagebuch}
               \section[ Paul Goldmann an Arthur Schnitzler, 18. 4. {[}1900{]}]{ Paul Goldmann an Arthur Schnitzler, 18. 4. {[}1900{]}}\nopagebreak\mylabel{v}\rehead{ }\begin{ledgroupsized}[t]{13cm}\normalsize\beginnumbering \toendnotes[C]{\smallbreak\pagebreak[2]} \Standort{DLA, A:Schnitzler, HS.NZ85.1.3170.}
\physDesc{Brief, 2 Blätter, 8 Seiten
\newline{}Handschrift: blaue Tinte, deutsche Kurrent
\newline{}Schnitzler: 1) mit Bleistift das Jahr »{[}1{]}900« vermerkt  2) mit rotem Buntstift fünf Unterstreichungen}\toendnotes[C]{\smallbreak}\pstart{}{\pb}\textcolor{gray}{\textbf{DESSAUERSTRASSE 19}}\oindex{Dessauer Strasse@\textbf{Dessauer Straße}|pw}\pend{}{\bigskip}\pstart
           \raggedleft{}Berlin\oindex{Berlin@\textbf{Berlin}|pw}, 18. April.\pend
           \pstart{}Mein lieber Freund,\pend\pstart
           Ich habe mich ſehr mit Deinem lieben Briefe gefreut. Lange habe ich ihn erwartet und
               wußte mir gar nicht zu erklären, warum ich ſo ganz ohne Nachricht blieb. Ich war \strikeout{\textcolor{gray}{nac}h} zum \label{K_L02911-1v}\edtext{\textsc{Speidel\pwindex{Speidel, Ludwig 1830-04-11 – 1906-02-03@\textsc{Speidel, Ludwig} (1830-04-11 – 1906-02-03), \emph{Journalist, Kritiker}|pw}}-Banket}{\lemma{\textnormal{\emph{Speidel-Banket}}}\Cendnote{\textnormal{Schnitzler\pwindex{Schnitzler, Arthur 15.05.1862 – 21.10.1931@\textsc{Schnitzler, Arthur} (15.05.1862 – 21.10.1931), \emph{Schriftsteller, Mediziner}|pwk} nahm an dem Bankett von Ludwig Speidel\pwindex{Speidel, Ludwig 1830-04-11 – 1906-02-03@\textsc{Speidel, Ludwig} (1830-04-11 – 1906-02-03), \emph{Journalist, Kritiker}|pwk} am 15. 4. 1900 teil.
                     »Widerwärtig«, notierte er sich dazu im \emph{Tagebuch}\pwindex{Schnitzler, Arthur 15.05.1862 – 21.10.1931@\textsc{Schnitzler, Arthur} (15.05.1862 – 21.10.1931), \emph{Schriftsteller, Mediziner}!Tagebuch1981 – 2000@\strich\emph{Tagebuch} {[}1981 – 2000{]}|pwk}.}}}\label{K_L02911-1h} geladen und hätte darum ſehr gut nach Wien\oindex{Wien@\textbf{Wien}|pw} kommen können und die N. Fr. Pr.\orgindex{Neue Freie Presse@Neue Freie Presse|pw} hätte mir überdies die Reiſe bezahlen müſſen. Aber
               wenn ich nach Wien\oindex{Wien@\textbf{Wien}|pw} komme, ſo komme ich
               Deinetwegen. Und da ich ſo gar nichts von Dir hörte, ........ Aber laſſen wir das! Mir hat meine Hypochondrie wieder einmal \strikeout{\textcolor{gray}{×}\-\textcolor{gray}{×}} einen Streich geſpielt, und es thut mir nun doppelt leid, um die ſchönen
               Oſtertage gekommen zu ſein, {\pb}die ich mit Dir hätte
               verleben können.\pend
           \pstart
           Was Deine \label{K_L02911-2v}\edtext{Furcht vor dem
                  Altwerden}{\lemma{\textnormal{\emph{Furcht vor dem
                  Altwerden}}}\Cendnote{\textnormal{vermutlich Bezug auf Schnitzler\pwindex{Schnitzler, Arthur 15.05.1862 – 21.10.1931@\textsc{Schnitzler, Arthur} (15.05.1862 – 21.10.1931), \emph{Schriftsteller, Mediziner}|pwk}s bevorstehenden 38. Geburtstag am
                     15. 5. 1900}}}\label{K_L02911-2h} anlangt, – nein, wirklich, mit 38 Jahren iſt man noch nicht alt. Und wenn Du
               Dir das früher einmal als das Ende aller Dinge vorgeſtellt haſt, ſo haſt Du eben
               früher das Leben nicht gekannt, wie man ja ſo Manches ſich unrichtig vorſtellt, wenn
               man gar zu jung iſt. Früher haben Dich die Frauen geliebt, weil Du 20 Jahre alt
               warſt; jetzt haben ſie viel mehr Gründe, Dich zu lieben, und dabei biſt Du immer noch
               jung genug, daß es ihnen Vergnügen macht. Die Geliebten, die Dich ſeinerzeit durch
                  \introOben{}den\introOben{} Hinweis \strikeout{auf ihre}
               beruhigt haben, daß ihre anderen Anbeter Ende der Dreißig ſeien, haben dieſen Anderen
               wahrſcheinlich mit Hinweis auf Dich geſagt: »Das iſt {\pb}ein unreifer Junge. Lieben aber kann man nur einen wirklichen Mann.« Wie alt,
               glaubſt Du, war \textsc{Don Juan}? Jedenfalls nicht zwanzig Jahre.
               Meiner Anſicht nach hatte er zwiſchen 35 und 40, wenn nicht darüber{\dotsseven}\pend
           \pstart
           Auf Deine \label{K_L02911-3v}\edtext{Novelle\pwindex{Schnitzler, Arthur 15.05.1862 – 21.10.1931@\textsc{Schnitzler, Arthur} (15.05.1862 – 21.10.1931), \emph{Schriftsteller, Mediziner}!Frau Bertha Garlan. Roman15.1.1901 – 15.3.1901@\strich\emph{Frau Bertha Garlan. Roman} {[}15.1.1901 – 15.3.1901{]}|pwv}}{\lemma{\textnormal{\emph{Novelle}}}\Cendnote{\textnormal{Schnitzler\pwindex{Schnitzler, Arthur 15.05.1862 – 21.10.1931@\textsc{Schnitzler, Arthur} (15.05.1862 – 21.10.1931), \emph{Schriftsteller, Mediziner}|pwk} hatte \emph{Frau Bertha Garlan}\pwindex{Schnitzler, Arthur 15.05.1862 – 21.10.1931@\textsc{Schnitzler, Arthur} (15.05.1862 – 21.10.1931), \emph{Schriftsteller, Mediziner}!Frau Bertha Garlan. Roman15.1.1901 – 15.3.1901@\strich\emph{Frau Bertha Garlan. Roman} {[}15.1.1901 – 15.3.1901{]}|pwk} am 1. 1. 1900 begonnen und am 16. 4. 1900
                  fertiggestellt.}}}\label{K_L02911-3h} freue ich mich ſehr. Was wird eigentlich aus der \textsc{Beatrice\pwindex{Schnitzler, Arthur 15.05.1862 – 21.10.1931@\textsc{Schnitzler, Arthur} (15.05.1862 – 21.10.1931), \emph{Schriftsteller, Mediziner}!Schleier der Beatrice. Schauspiel in fuenf Akten1900-12-01@\strich\emph{Der Schleier der Beatrice. Schauspiel in fünf Akten} {[}1900-12-01{]}|pw}}? Wann beginnen die \label{K_L02911-4v}\edtext{Proben}{\lemma{\textnormal{\emph{Proben}}}\Cendnote{\textnormal{Schnitzler\pwindex{Schnitzler, Arthur 15.05.1862 – 21.10.1931@\textsc{Schnitzler, Arthur} (15.05.1862 – 21.10.1931), \emph{Schriftsteller, Mediziner}|pwk} war das erste Mal am 23. 11. 1900 bei Proben
                  für die Uraufführung von \emph{Der Schleier der
                     Beatrice}\pwindex{Schnitzler, Arthur 15.05.1862 – 21.10.1931@\textsc{Schnitzler, Arthur} (15.05.1862 – 21.10.1931), \emph{Schriftsteller, Mediziner}!Schleier der Beatrice. Schauspiel in fuenf Akten1900-12-01@\strich\emph{Der Schleier der Beatrice. Schauspiel in fünf Akten} {[}1900-12-01{]}|pwk} anwesend.}}}\label{K_L02911-4h}?\pend
           \pstart
           Wie beneide ich Dich um Dein Arbeiten! Ich ſelbſt bringe es nicht zu Stande. Ich habe
               jetzt, nach Wochen angeſpannteſter Arbeit, auch wieder Wochen faſt vollkommener Ruhe.
               Das wäre die Zeit, etwas zu ſchaffen. Ich zermartere mir den Kopf, will heut ein
               Drama ſchreiben, morgen eine Novelle. Aber Alles {\pb}zerrinnt wieder im Nebel. Und ich vergeude meine Zeit mit Beſuchen, mit
               überflüſſiger Reporter-Arbeit und Anderem, wie ja überhaupt der Journalismus eine
               große Zeitvertrödelung iſt. Dabei habe ich das Gefühl, es ſteckt doch noch etwas mehr
               in mir. Aber ich weiß nicht, was ich will. Ich würde Denjengen\strikeout{,} wie einen Erlöſer begrüßen, der mir einen Rath geben,
               mich auf eine größere Arbeit hinweiſen würde, die \strikeout{\textcolor{gray}{me}i\textcolor{gray}{n}} meinen Fähigkeiten entſpräche. Aber, ich weiß, dieſen Rath kann man ſich nur
               ſelbſt geben. Und bei mir finde ich keinen. Ich habe mich ſelten innerlich ſo elend
               gefühlt, mich ſelten ſo verachtet. Große Prätentionen, und innerlich {\pb}Alles leer, le{[}e{]}r! Meine einzige
               Leiſtung iſt, daß ich täglich fetter werde{\dots}\pend
           \pstart
           Im Sommer werde ich wohl meinen Urlaub bekommen. Aber ich werde ihn in Berlin\oindex{Berlin@\textbf{Berlin}|pw} verbringen müſſen, weil ich diesmal keine
               fünf Mark übrig haben werde, um zu reiſen. Der Hausſtand, den ich hier mit meiner Mutter\pwindex{Goldmann, Clementine 1842-05-15 – 1924-02-24@\textsc{Goldmann, Clementine} (1842-05-15 – 1924-02-24)|pwv} führe, \strikeout{ver} nimmt faſt mein ganzes Gehalt in Anſpruch. Der Reſt
               geht für Schulden-Abzahlungen aller Art drauf; und Nebenverdienſt iſt ausgeſchloſſen.
               Nach \label{K_L02911-5v}\edtext{\textsc{Paris\oindex{Paris@\textbf{Paris}|pw}}}{\lemma{\textnormal{\emph{Paris}}}\Cendnote{\textnormal{Womöglich erkundigte sich Schnitzler\pwindex{Schnitzler, Arthur 15.05.1862 – 21.10.1931@\textsc{Schnitzler, Arthur} (15.05.1862 – 21.10.1931), \emph{Schriftsteller, Mediziner}|pwk}, ob Goldmann\pwindex{Goldmann, Paul 31.01.1865 – 25.09.1935@\textsc{Goldmann, Paul} (31.01.1865 – 25.09.1935), \emph{Schriftsteller, Journalist}|pwk} zur Weltausstellung nach Paris\oindex{Paris@\textbf{Paris}|pwk} (15. 4. 1900–12. 11. 1900) fahre.}}}\label{K_L02911-5h} fahre ich unter dieſen
               Umſtänden natürlich nicht.\pend
           \pstart
           {\pb}Kenſt Du \label{K_L02911-42v}\edtext{\textsc{Flaubert\pwindex{Flaubert, Gustave 12.12.1821 – 08.05.1880@\textsc{Flaubert, Gustave} (12.12.1821 – 08.05.1880), \emph{Schriftsteller}|pw}s}{ }Briefe\pwindex{Flaubert, Gustave 12.12.1821 – 08.05.1880@\textsc{Flaubert, Gustave} (12.12.1821 – 08.05.1880), \emph{Schriftsteller}!Correspondance. 4 Bde.1887 – 1893@\strich\emph{Correspondance. 4 Bde.} {[}1887 – 1893{]}|pwv}}{\lemma{\textnormal{\emph{Flauberts Briefe}}}\Cendnote{\textnormal{Gustave Flaubert\pwindex{Flaubert, Gustave 12.12.1821 – 08.05.1880@\textsc{Flaubert, Gustave} (12.12.1821 – 08.05.1880), \emph{Schriftsteller}|pwk}: \emph{Correspondance}\pwindex{Flaubert, Gustave 12.12.1821 – 08.05.1880@\textsc{Flaubert, Gustave} (12.12.1821 – 08.05.1880), \emph{Schriftsteller}!Correspondance. 4 Bde.1887 – 1893@\strich\emph{Correspondance. 4 Bde.} {[}1887 – 1893{]}|pwk}. 4 Bde. Paris\oindex{Paris@\textbf{Paris}|pwk}: \emph{Charpentier & Cie}\orgindex{Charpentier@Charpentier|pwk}{ }1887–1893. Schnitzler\pwindex{Schnitzler, Arthur 15.05.1862 – 21.10.1931@\textsc{Schnitzler, Arthur} (15.05.1862 – 21.10.1931), \emph{Schriftsteller, Mediziner}|pwk} kannte eine spätere Ausgabe\pwindex{Flaubert, Gustave 12.12.1821 – 08.05.1880@\textsc{Flaubert, Gustave} (12.12.1821 – 08.05.1880), \emph{Schriftsteller}!Correspondance1926 – 1933@\strich\emph{Correspondance} {[}1926 – 1933{]}|pwkv} (vgl. A. S.: \emph{Lektüren}, Frankreich).}}}\label{K_L02911-42h}? Wenn
               nicht, ſo mußt Du ſie gleich leſen, und zwar gleich den dritten und vierten Band\pwindex{Flaubert, Gustave 12.12.1821 – 08.05.1880@\textsc{Flaubert, Gustave} (12.12.1821 – 08.05.1880), \emph{Schriftsteller}!Correspondance. 4 Bde.1887 – 1893@\strich\emph{Correspondance. 4 Bde.} {[}1887 – 1893{]}|pwv}; die Jugendbriefe\pwindex{Flaubert, Gustave 12.12.1821 – 08.05.1880@\textsc{Flaubert, Gustave} (12.12.1821 – 08.05.1880), \emph{Schriftsteller}!Correspondance. 4 Bde.1887 – 1893@\strich\emph{Correspondance. 4 Bde.} {[}1887 – 1893{]}|pwv} in den erſten beiden ſind
               nicht intereſſant. Ich habe ſie jetzt wieder vorgeholt. Jeder Menſch, der ſchreibt,
                  \strikeout{muß} findet darin Troſt, Befreiung und Belehrung.
               Auf dem ſpeciell ſchriftſtelleriſchen Gebiete geben ſie Einem faſt ſo viel, wie
                  \label{K_L02911-7v}\edtext{Goethe\pwindex{Goethe, Johann Wolfgang von 1749-08-28 – 1832-03-22@\textsc{Goethe, Johann Wolfgang von} (1749-08-28 – 1832-03-22), \emph{Schriftsteller}|pw}s Geſpräche\pwindex{\textcolor{red}{\textsuperscript{XXXX1 indx}}!Goethes Unterhaltungen mit dem Kanzler Friedrich von Mueller1870@\strich\emph{Goethes Unterhaltungen mit dem Kanzler Friedrich von Müller} {[}1870{]}|pwv}}{\lemma{\textnormal{\emph{Goethes Geſpräche}}}\Cendnote{\textnormal{siehe Paul Goldmann an Arthur Schnitzler, 25. 9. [1899]}}}\label{K_L02911-7h}; nur ſind ſie nicht ſo univerſell menſchlich, wie dieſe. \textsc{Flaubert\pwindex{Flaubert, Gustave 12.12.1821 – 08.05.1880@\textsc{Flaubert, Gustave} (12.12.1821 – 08.05.1880), \emph{Schriftsteller}|pw}} iſt eben doch kein Menſch, ſondern \strikeout{nur} nur ein
                  Fran\oindex{Frankreich@\textbf{Frankreich}|pwv}zoſe{\dotsfour}\pend
           \pstart
           Von \textsc{Gusti\pwindex{Chlum, Auguste 16.03.1862 – 1956@\textsc{Chlum, Auguste} (16.03.1862 – 1956)|pw}} weiß\strikeout{’} ich Dir nichts {\pb}zu berichten. Das eigentliche Leben der beiden Mädels\pwindex{Chlum, Auguste 16.03.1862 – 1956@\textsc{Chlum, Auguste} (16.03.1862 – 1956)|pwv}\pwindex{Gluemer, Marie 03.07.1867 – 16.11.1925@\textsc{Glümer, Marie} (03.07.1867 – 16.11.1925), \emph{Schauspielerin}|pwv} bleibt mir
               verſchloſſen. Trotz aller Herzlichkeit der Beziehungen beſteht zwiſchen uns doch
               keine rechte Sympathie, und innerlich ſtehen wir uns fremd gegenüber.\pend
           \pstart
           Was macht \label{K_L02911-8v}\edtext{\textsc{Richard\pwindex{Beer-Hofmann, Richard 1866-07-11 – 1945-09-26@\textsc{Beer-Hofmann, Richard} (1866-07-11 – 1945-09-26), \emph{Schriftsteller}|pw}}}{\lemma{\textnormal{\emph{Richard}}}\Cendnote{\textnormal{Goldmann\pwindex{Goldmann, Paul 31.01.1865 – 25.09.1935@\textsc{Goldmann, Paul} (31.01.1865 – 25.09.1935), \emph{Schriftsteller, Journalist}|pwk} bezog sich vermutlich auf Beer-Hofmann\pwindex{Beer-Hofmann, Richard 1866-07-11 – 1945-09-26@\textsc{Beer-Hofmann, Richard} (1866-07-11 – 1945-09-26), \emph{Schriftsteller}|pwk}s Trauerspiel \emph{Der Graf von Charolais}\pwindex{Beer-Hofmann, Richard 1866-07-11 – 1945-09-26@\textsc{Beer-Hofmann, Richard} (1866-07-11 – 1945-09-26), \emph{Schriftsteller}!Graf von Charolais. Ein Trauerspiel1904-12-23@\strich\emph{Der Graf von Charolais. Ein Trauerspiel} {[}1904-12-23{]}|pwk}, an dem er bereits seit 1899 arbeitete. Zu Beer-Hofmann\pwindex{Beer-Hofmann, Richard 1866-07-11 – 1945-09-26@\textsc{Beer-Hofmann, Richard} (1866-07-11 – 1945-09-26), \emph{Schriftsteller}|pwk}s Reisen im Sommer 1900 vgl.
                     Eugene Weber: \emph{Richard Beer-Hofmann: Daten mitgeteilt von
                        Eugene Weber}. In: \emph{Modern Austrian
                        Literature} 17/2 (1984), S. 13–42, hier:
                     S. 23.}}}\label{K_L02911-8h}? Arbeitet er an ſeinem Drama\pwindex{Beer-Hofmann, Richard 1866-07-11 – 1945-09-26@\textsc{Beer-Hofmann, Richard} (1866-07-11 – 1945-09-26), \emph{Schriftsteller}!Graf von Charolais. Ein Trauerspiel1904-12-23@\strich\emph{Der Graf von Charolais. Ein Trauerspiel} {[}1904-12-23{]}|pwv}? Und was wird er im Sommer machen? Wirſt
               Du mit ihm zuſammen ſein?\pend
           \pstart
           Geſtern ſprach ich wieder einmal \textsc{Kerr\pwindex{Kerr, Alfred 25.12.1867 – 12.10.1948@\textsc{Kerr, Alfred} (25.12.1867 – 12.10.1948), \emph{Schriftsteller, Kritiker}|pw}} nach langer Pauſe. Er ſcheint \textcolor{gray}{nun}{ }\label{K_L02911-11v}\edtext{große Liebe\pwindex{Wendt, Anna @\textsc{Wendt, Anna}|pwv}}{\lemma{\textnormal{\emph{große Liebe}}}\Cendnote{\textnormal{Bezug auf Anna Wendt\pwindex{Wendt, Anna @\textsc{Wendt, Anna}|pwk}, die Alfred
                     Kerr\pwindex{Kerr, Alfred 25.12.1867 – 12.10.1948@\textsc{Kerr, Alfred} (25.12.1867 – 12.10.1948), \emph{Schriftsteller, Kritiker}|pwk} im April 1900 kennengelernt hatte (vgl.
                     Deborah Vietor-Engländer: \emph{Alfred Kerr. Die
                        Biographie}. Reinbek bei Hamburg:
                        \emph{Rowohlt}{ }2016, S. 229 [E-Book-Ausgabe])}}}\label{K_L02911-11h} zu haben. Ich mag ihn ſehr gern trotz mancher Geſchmack-Defekte; aber er
               ſchließt ſich mir nicht auf. {\pb}Und wir bleiben
               fremd.\pend
           \pstart
           Wann ſehe ich Dich wieder? Wann kommſt Du nach \label{K_L02911-12v}\edtext{Berlin\oindex{Berlin@\textbf{Berlin}|pw}}{\lemma{\textnormal{\emph{Berlin}}}\Cendnote{\textnormal{siehe Paul Goldmann an Arthur Schnitzler, 13. 4. [1900]}}}\label{K_L02911-12h}?\pend
           \pstart
           Viele treue Grüße! {\\[\baselineskip]}Dein {\\[\baselineskip]}\spacefill\mbox{Paul Goldmann}\pend
           \leftskip=0em{}\pstart
           \noindent{}Meine Mutter\pwindex{Goldmann, Clementine 1842-05-15 – 1924-02-24@\textsc{Goldmann, Clementine} (1842-05-15 – 1924-02-24)|pwv} dankt für
                  Deine Grüße und erwidert ſie herzlichſt.\pend
           
         
         \endnumbering\mylabel{h}\end{ledgroupsized}\begin{anhang}\end{anhang}\newcommand{\dateiname}{L02911}\newcommand{\titel}{Paul Goldmann an Arthur Schnitzler, 18. 4. [1900]}\newcommand{\editorInnen}{Martin Anton Müller und Laura Untner}%% latex-leseansicht-abspann.tex
%% Abspann für die Leseansicht.
%% Der Schalter \ifkorrekturansicht ist bereits durch den Vorspann gesetzt.

%% latex-abspann.tex
%% Gemeinsamer Abspann für Korrekturansicht und Leseansicht.
%% Setzt den Schalter \ifkorrekturansicht voraus (gesetzt in den
%% einbindenden Dateien latex-korrekturansicht-abspann.tex bzw.
%% latex-leseansicht-abspann.tex).
%% ---------------------------------------------------------------

\normalsize

% Das esempio-Environment wird nur in der Leseansicht benötigt
\ifkorrekturansicht\else
\newenvironment{esempio}[3]%
{
    \vspace{1.5ex}
    \rlap{\underline{#1}}
    \par
    \setlength{\parindent}{0cm}
    \nopagebreak
    \leftskip=#2cm
    \rightskip=#3cm
}
{
    \par
}
\fi

\doendnotes{C}
\bigskip
\vfill

\clearpage

\footnotesize

\ifkorrekturansicht
  \lohead{\textsc{register}}
\fi

% theindex-Environment neu definieren ohne reledmac
\makeatletter
\renewenvironment{theindex}{%
  \ifkorrekturansicht
    \section*{\indexname}%
  \else
    \subsubsection*{Index der erwähnten Entitäten}%
  \fi
  \setlength{\parindent}{0pt}%
  \setlength{\parskip}{0pt plus 0.3pt}%
  \let\item\@idxitem
}{%
  \ifkorrekturansicht\clearpage\fi
}
\makeatother

\IfFileExists{\jobname-pw.ind}{\input{\jobname-pw.ind}}{}

% Quellenangabe nur in der Leseansicht
\ifkorrekturansicht\else
% Fallback-Definitionen, falls die .tex-Datei \titel etc. nicht gesetzt hat
\providecommand{\titel}{}
\providecommand{\editorInnen}{}
\providecommand{\dateiname}{\jobname}

\vspace{3cm}

\vfill

\footnotesize
\textsc{Quelle}: \titel. Herausgegeben von {\editorInnen}. In: \emph{Arthur Schnitzler: Briefwechsel mit Autorinnen und Autoren}.
 Digitale Edition, https://schnitzler-briefe.acdh.oeaw.ac.at/{\dateiname}.html (Stand \today)
\fi

\end{document}


      