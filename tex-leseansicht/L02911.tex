%% latex-leseansicht-vorspann.tex
%% Vorspann für die Leseansicht.
%% Lädt die gemeinsame Datei latex-vorspann.tex mit nicht gesetztem Schalter.

\newif\ifkorrekturansicht
\korrekturansichtfalse

\input{../tex-inputs/latex-vorspann}


\section[ Paul Goldmann an Arthur Schnitzler, 18. 4. [1900]]{L02911 Paul Goldmann an Arthur Schnitzler,  18. 4. [1900]}
\nopagebreak\mylabel{L02911v}
\rehead{ }\normalsize\beginnumbering\briefempfaengerindex{Schnitzler, Arthur@\textsc{Schnitzler, Arthur}!zzzGoldmann, Paul@\emph{von Paul Goldmann}!1900-04-181@{18. 4. [1900]}|(be}
\toendnotes[C]{\smallbreak\pagebreak[2]}
\correspDesc{Versand  durch Paul Goldmann am 18. 4. [1900] in Berlin
\newline{}Erhalt  durch Arthur Schnitzler im Zeitraum [19. 4. 1900
                  – 23. 4. 1900?] in Wien}\toendnotes[C]{\smallbreak}
\Standort{DLA, A:Schnitzler, HS.NZ85.1.3170.}
\physDesc{Brief, 2 Blätter, 8 Seiten, 4263 Zeichen
\newline{}Handschrift: blaue Tinte, deutsche Kurrent
\newline{}Schnitzler: 1) mit Bleistift das Jahr »900« vermerkt  2) mit rotem Buntstift fünf Unterstreichungen}\toendnotes[C]{\smallbreak}
\pstart
           {\pb}\textcolor{gray}{\textbf{DESSAUERSTRASSE 19}}\oindex{Dessauer Straße@\textbf{Dessauer Straße}, \emph{Straße}|pw}\pend
           
\pstart
           \raggedleft{}Berlin\oindex{Berlin@\textbf{Berlin}, \emph{Hauptstadt}|pw}, 18. April.\pend
           
\pstart{}Mein lieber Freund,\pend\vspace{0.5em}
\pstart
           Ich habe mich{ }ſehr mit Deinem lieben Briefe gefreut. Lange habe ich ihn erwartet und
               wußte mir gar nicht zu erklären, warum ich{ }ſo ganz ohne Nachricht blieb. Ich war \strikeout{\textcolor{gray}{auc}h} zum \label{K_L02911-1v}\edtext{\textsc{Speidel\pwindex{Speidel, Ludwig 11.\,4.\,1830 Ulm – 3.\,2.\,1906 Wien@\textsc{Speidel, Ludwig} (11.\,4.\,1830 Ulm – 3.\,2.\,1906 Wien), \emph{Journalist, Kritiker}|pw}}-Banket}{\lemma{\textnormal{\emph{Speidel-Banket}}}\Cendnote{\textnormal{Am Nachmittag des 15. 4. 1900 fand ein
                  großes Bankett anlässlich des 70. Geburtstags von Ludwig Speidel\pwindex{Speidel, Ludwig 11.\,4.\,1830 Ulm – 3.\,2.\,1906 Wien@\textsc{Speidel, Ludwig} (11.\,4.\,1830 Ulm – 3.\,2.\,1906 Wien), \emph{Journalist, Kritiker}|pwk} statt. Schnitzler war einer der über 50 Teilnehmenden aus dem Kulturbetrieb.
                     »Widerwärtig«, notierte er sich dazu im \emph{Tagebuch}\pwindex{Schnitzler, Arthur 15.\,5.\,1862 Wien – 21.\,10.\,1931 ebd.@\textsc{Schnitzler, Arthur} (15.\,5.\,1862 Wien – 21.\,10.\,1931 ebd.), \emph{Schriftsteller, Mediziner}!Tagebuch@\strich\emph{Tagebuch}|pwk}.}}}\label{K_L02911-1} geladen und hätte darum{ }ſehr gut nach Wien\oindex{Wien@\textbf{Wien}, \emph{Verwaltungsgebiet}|pw} kommen können, und die N. Fr. Pr.\orgindex{Neue Freie Presse@Neue Freie Presse|pw} hätte mir überdies die Reiſe bezahlen müſſen. Aber
               wenn ich nach Wien\oindex{Wien@\textbf{Wien}, \emph{Verwaltungsgebiet}|pw} komme,{ }ſo komme ich
               Deinetwegen. Und da ich{ }ſo gar nichts von Dir hörte, ........ Aber laſſen wir das! Mir hat meine Hypochondrie wieder einmal \strikeout{\textcolor{gray}{×}} einen Streich geſpielt, und es thut mir nun doppelt leid, um die{ }ſchönen
               Oſtertage gekommen zu{ }ſein, {\pb}die ich mit Dir hätte
               verleben können.\pend
           
\pstart
           Was Deine \label{K_L02911-2v}\edtext{Furcht vor dem
                  Altwerden}{\lemma{\textnormal{\emph{Furcht vor dem
                  Altwerden}}}\Cendnote{\textnormal{In wenigen Tagen, am 15. 5. 1900, sollte
                     Schnitzler seinen 38. Geburtstag
                  begehen.}}}\label{K_L02911-2} anlangt, – nein, wirklich, mit 38 Jahren iſt man noch nicht alt.
               Und wenn Du Dir das früher einmal als das Ende aller Dinge vorgeſtellt haſt,{ }ſo haſt
               Du eben früher das Leben nicht gekannt, wie man ja{ }ſo Manches{ }ſich unrichtig
               vorſtellt, wenn man gar zu jung iſt. Früher haben Dich die Frauen geliebt, weil
               Du 20 Jahre alt warſt; jetzt haben{ }ſie viel mehr Gründe, Dich zu lieben, und dabei
               biſt Du immer noch jung genug, daß es ihnen Vergnügen macht. Die Geliebten, die Dich{ }ſeinerzeit durch \introOben{}den\introOben{} Hinweis \strikeout{auf
                  ihre} beruhigt haben, daß ihre anderen Anbeter Ende der Dreißig{ }ſeien, haben
               dieſen Anderen wahrſcheinlich mit Hinweis auf Dich geſagt: »Das iſt {\pb}ein unreifer Junge. Lieben aber kann man nur einen
               wirklichen Mann.« Wie alt, glaubſt Du, war \textsc{Don Juan}?
               Jedenfalls nicht zwanzig Jahre. Meiner Anſicht nach hatte er zwiſchen 35 und 40, wenn
               nicht darüber{\dotsseven}\pend
           
\pstart
           Auf Deine \label{K_L02911-3v}\edtext{Novelle\pwindex{Schnitzler, Arthur 15.\,5.\,1862 Wien – 21.\,10.\,1931 ebd.@\textsc{Schnitzler, Arthur} (15.\,5.\,1862 Wien – 21.\,10.\,1931 ebd.), \emph{Schriftsteller, Mediziner}!Frau Bertha Garlan. Roman@\strich\emph{Frau Bertha Garlan. Roman}|pwv}}{\lemma{\textnormal{\emph{Novelle}}}\Cendnote{\textnormal{Schnitzler hatte \emph{Frau Bertha Garlan}\pwindex{Schnitzler, Arthur 15.\,5.\,1862 Wien – 21.\,10.\,1931 ebd.@\textsc{Schnitzler, Arthur} (15.\,5.\,1862 Wien – 21.\,10.\,1931 ebd.), \emph{Schriftsteller, Mediziner}!Frau Bertha Garlan. Roman@\strich\emph{Frau Bertha Garlan. Roman}|pwk} am 1. 1. 1900 begonnen und am 16. 4. 1900
                  fertiggestellt.}}}\label{K_L02911-3} freue ich mich{ }ſehr. Was wird eigentlich aus der \textsc{Beatrice\pwindex{Schnitzler, Arthur 15.\,5.\,1862 Wien – 21.\,10.\,1931 ebd.@\textsc{Schnitzler, Arthur} (15.\,5.\,1862 Wien – 21.\,10.\,1931 ebd.), \emph{Schriftsteller, Mediziner}!Schleier der Beatrice. Schauspiel in fünf Akten@\strich\emph{Der Schleier der Beatrice. Schauspiel in fünf Akten}|pw}}? Wann beginnen die \label{K_L02911-4v}\edtext{Proben}{\lemma{\textnormal{\emph{Proben}}}\Cendnote{\textnormal{Schnitzler glaubte zu diesem Zeitpunkt noch,
                  dass das Stück\pwindex{Schnitzler, Arthur 15.\,5.\,1862 Wien – 21.\,10.\,1931 ebd.@\textsc{Schnitzler, Arthur} (15.\,5.\,1862 Wien – 21.\,10.\,1931 ebd.), \emph{Schriftsteller, Mediziner}!Schleier der Beatrice. Schauspiel in fünf Akten@\strich\emph{Der Schleier der Beatrice. Schauspiel in fünf Akten}|pwkv} am \emph{Burgtheater}\orgindex{Burgtheater@Burgtheater|pwk} aufgeführt werden sollte. Siehe XXXX Auszeichnungsfehler: Dokument L02893 nicht gefunden.}}}\label{K_L02911-4}?\pend
           
\pstart
           Wie beneide ich Dich um Dein Arbeiten! Ich{ }ſelbſt bringe es nicht zu Stande. Ich habe
               jetzt, nach Wochen angeſpannteſter Arbeit, auch wieder Wochen faſt vollkommener Ruhe.
               Das wäre die Zeit, etwas zu{ }ſchaffen. Ich zermartere mir den Kopf, will heut ein
               Drama{ }ſchreiben, morgen eine Novelle. Aber Alles {\pb}zerrinnt wieder im Nebel. Und ich vergeude meine Zeit mit Beſuchen, mit
               überflüſſiger Reporter-Arbeit und Anderem, wie ja überhaupt der Journalismus eine
               große Zeitvertrödelung iſt. Dabei habe ich das Gefühl, es{ }ſteckt doch noch etwas mehr
               in mir. Aber ich weiß nicht, was ich will. Ich würde Denjengen\strikeout{,} wie einen Erlöſer begrüßen, der mir einen Rath geben,
               mich auf eine größere Arbeit hinweiſen würde, die \strikeout{\textcolor{gray}{me}i\textcolor{gray}{n}} meinen Fähigkeiten entſpräche. Aber, ich weiß, dieſen Rath kann man{ }ſich nur{ }ſelbſt geben. Und bei mir finde ich keinen. Ich habe mich{ }ſelten innerlich{ }ſo elend
               gefühlt, mich{ }ſelten{ }ſo verachtet. Große Prätentionen, und innerlich {\pb}Alles leer, le{[}e{]}r! Meine einzige
               Leiſtung iſt, daß ich täglich fetter werde{\dotsfour}\pend
           
\pstart
           Im Sommer werde ich wohl meinen Urlaub bekommen. Aber ich werde ihn in Berlin\oindex{Berlin@\textbf{Berlin}, \emph{Hauptstadt}|pw} verbringen müſſen, weil ich diesmal keine
               fünf Mark übrig haben werde, um zu reiſen. Der Hausſtand, den ich hier mit meiner Mutter\pwindex{Goldmann, Clementine 15.\,5.\,1842 Breslau – 24.\,2.\,1924 Frankfurt am Main@\textsc{Goldmann, Clementine} (15.\,5.\,1842 Breslau – 24.\,2.\,1924 Frankfurt am Main)|pwv} führe, \strikeout{ver} nimmt faſt mein ganzes Gehalt in Anſpruch. Der Reſt
               geht für Schulden-Abzahlungen aller Art drauf; und Nebenverdienſt iſt ausgeſchloſſen.
               Nach \label{K_L02911-5v}\edtext{\textsc{Paris\oindex{Paris@\textbf{Paris}, \emph{Hauptstadt}|pw}}}{\lemma{\textnormal{\emph{Paris}}}\Cendnote{\textnormal{Schnitzler dürfte sich erkundigt haben, ob
                     Goldmann\pwindex{Goldmann, Paul 31.\,1.\,1865 Breslau – 25.\,9.\,1935 Wien@\textsc{Goldmann, Paul} (31.\,1.\,1865 Breslau – 25.\,9.\,1935 Wien), \emph{Schriftsteller, Journalist}|pwk} zur Weltausstellung nach Paris\oindex{Paris@\textbf{Paris}, \emph{Hauptstadt}|pwk} (15. 4. 1900 – 12. 11. 1900) zu fahren
                  gedachte.}}}\label{K_L02911-5} fahre ich unter dieſen Umſtänden natürlich nicht.\pend
           
\pstart
           {\pb}Kennſt Du \label{K_L02911-6v}\edtext{\textsc{Flauberts\pwindex{Flaubert, Gustave 12.\,12.\,1821 Rouen – 8.\,5.\,1880 Canteleu@\textsc{Flaubert, Gustave} (12.\,12.\,1821 Rouen – 8.\,5.\,1880 Canteleu), \emph{Schriftsteller}|pw}}{ }Briefe\pwindex{Flaubert, Gustave 12.\,12.\,1821 Rouen – 8.\,5.\,1880 Canteleu@\textsc{Flaubert, Gustave} (12.\,12.\,1821 Rouen – 8.\,5.\,1880 Canteleu), \emph{Schriftsteller}!Correspondance. 4 Bde.@\strich\emph{Correspondance. 4 Bde.}|pwv}}{\lemma{\textnormal{\emph{Flauberts Briefe}}}\Cendnote{\textnormal{Gustave Flaubert\pwindex{Flaubert, Gustave 12.\,12.\,1821 Rouen – 8.\,5.\,1880 Canteleu@\textsc{Flaubert, Gustave} (12.\,12.\,1821 Rouen – 8.\,5.\,1880 Canteleu), \emph{Schriftsteller}|pwk}: \emph{Correspondance}\pwindex{Flaubert, Gustave 12.\,12.\,1821 Rouen – 8.\,5.\,1880 Canteleu@\textsc{Flaubert, Gustave} (12.\,12.\,1821 Rouen – 8.\,5.\,1880 Canteleu), \emph{Schriftsteller}!Correspondance. 4 Bde.@\strich\emph{Correspondance. 4 Bde.}|pwk}. 4 Bde. Paris\oindex{Paris@\textbf{Paris}, \emph{Hauptstadt}|pwk}: \emph{Charpentier {\kaufmannsund} Cie}\orgindex{Charpentier@Charpentier|pwk}{ }1887–1893. Schnitzler kannte zumindest eine
                  spätere Ausgabe\pwindex{Flaubert, Gustave 12.\,12.\,1821 Rouen – 8.\,5.\,1880 Canteleu@\textsc{Flaubert, Gustave} (12.\,12.\,1821 Rouen – 8.\,5.\,1880 Canteleu), \emph{Schriftsteller}!Correspondance@\strich\emph{Correspondance}|pwkv} (vgl. A. S.: \emph{Lektüren}, Frankreich).}}}\label{K_L02911-6}? Wenn
               nicht,{ }ſo mußt Du{ }ſie gleich leſen, und zwar gleich den dritten und vierten Band\pwindex{Flaubert, Gustave 12.\,12.\,1821 Rouen – 8.\,5.\,1880 Canteleu@\textsc{Flaubert, Gustave} (12.\,12.\,1821 Rouen – 8.\,5.\,1880 Canteleu), \emph{Schriftsteller}!Correspondance. 4 Bde.@\strich\emph{Correspondance. 4 Bde.}|pwv}; die Jugendbriefe\pwindex{Flaubert, Gustave 12.\,12.\,1821 Rouen – 8.\,5.\,1880 Canteleu@\textsc{Flaubert, Gustave} (12.\,12.\,1821 Rouen – 8.\,5.\,1880 Canteleu), \emph{Schriftsteller}!Correspondance. 4 Bde.@\strich\emph{Correspondance. 4 Bde.}|pwv} in den erſten beiden{ }ſind
               nicht intereſſant. Ich habe{ }ſie jetzt wieder vorgeholt. Jeder Menſch, der{ }ſchreibt,
                  \strikeout{muß} findet darin Troſt, Befreiung und Belehrung.
               Auf dem{ }ſpeciell{ }ſchriftſtelleriſchen Gebiete geben{ }ſie Einem faſt{ }ſo viel, wie
                  \label{K_L02911-7v}\edtext{Goethes\pwindex{Goethe, Johann Wolfgang von 28.\,8.\,1749 Frankfurt am Main – 22.\,3.\,1832 Weimar@\textsc{Goethe, Johann Wolfgang von} (28.\,8.\,1749 Frankfurt am Main – 22.\,3.\,1832 Weimar), \emph{Schriftsteller}|pw}s Geſpräche\pwindex{\textcolor{red}{\textsuperscript{XXXX indx1}}!Goethes Unterhaltungen mit dem Kanzler Friedrich von Müller@\strich\emph{Goethes Unterhaltungen mit dem Kanzler Friedrich von Müller}|pwv}}{\lemma{\textnormal{\emph{Goethess Gespräche}}}\Cendnote{\textnormal{Siehe XXXX Auszeichnungsfehler: Dokument L02887 nicht gefunden.
               }}}\label{K_L02911-7}; nur{ }ſind{ }ſie nicht{ }ſo univerſell menſchlich, wie dieſe. \textsc{Flaubert\pwindex{Flaubert, Gustave 12.\,12.\,1821 Rouen – 8.\,5.\,1880 Canteleu@\textsc{Flaubert, Gustave} (12.\,12.\,1821 Rouen – 8.\,5.\,1880 Canteleu), \emph{Schriftsteller}|pw}} iſt eben doch kein Menſch,{ }ſondern \strikeout{nur} nur ein
                  Fran\oindex{Frankreich@\textbf{Frankreich}|pwv}zoſe{\dotsfour}\pend
           
\pstart
           Von \textsc{Gusti\pwindex{Glümer, Auguste 16.\,3.\,1862 Wien – 1956@\textsc{Glümer, Auguste} (16.\,3.\,1862 Wien – 1956), \emph{Lehrerin}|pw}} weiß\strikeout{’} ich Dir nichts {\pb}zu berichten. Das eigentliche Leben der beiden Mädels\pwindex{Glümer, Auguste 16.\,3.\,1862 Wien – 1956@\textsc{Glümer, Auguste} (16.\,3.\,1862 Wien – 1956), \emph{Lehrerin}|pwv}\pwindex{Glümer, Marie 3.\,7.\,1867 Wien – 16.\,11.\,1925 München@\textsc{Glümer, Marie} (3.\,7.\,1867 Wien – 16.\,11.\,1925 München), \emph{Schauspielerin}|pwv} bleibt mir
               verſchloſſen. Trotz aller Herzlichkeit der Beziehungen beſteht zwiſchen uns doch
               keine rechte Sympathie, und innerlich{ }ſtehen wir uns fremd gegenüber.\pend
           
\pstart
           Was macht \label{K_L02911-8v}\edtext{\textsc{Richard\pwindex{Beer-Hofmann, Richard 11.\,7.\,1866 Wien – 26.\,9.\,1945 New York City@\textsc{Beer-Hofmann, Richard} (11.\,7.\,1866 Wien – 26.\,9.\,1945 New York City), \emph{Schriftsteller}|pw}}}{\lemma{\textnormal{\emph{Richard}}}\Cendnote{\textnormal{Goldmann\pwindex{Goldmann, Paul 31.\,1.\,1865 Breslau – 25.\,9.\,1935 Wien@\textsc{Goldmann, Paul} (31.\,1.\,1865 Breslau – 25.\,9.\,1935 Wien), \emph{Schriftsteller, Journalist}|pwk} bezog sich auf Beer-Hofmanns\pwindex{Beer-Hofmann, Richard 11.\,7.\,1866 Wien – 26.\,9.\,1945 New York City@\textsc{Beer-Hofmann, Richard} (11.\,7.\,1866 Wien – 26.\,9.\,1945 New York City), \emph{Schriftsteller}|pwk} Trauerspiel \emph{Der Graf von Charolais}\pwindex{Beer-Hofmann, Richard 11.\,7.\,1866 Wien – 26.\,9.\,1945 New York City@\textsc{Beer-Hofmann, Richard} (11.\,7.\,1866 Wien – 26.\,9.\,1945 New York City), \emph{Schriftsteller}!Graf von Charolais. Ein Trauerspiel@\strich\emph{Der Graf von Charolais. Ein Trauerspiel}|pwk}, an dem dieser bereits seit 1899 arbeitete. Zu Beer-Hofmanns\pwindex{Beer-Hofmann, Richard 11.\,7.\,1866 Wien – 26.\,9.\,1945 New York City@\textsc{Beer-Hofmann, Richard} (11.\,7.\,1866 Wien – 26.\,9.\,1945 New York City), \emph{Schriftsteller}|pwk}
                  Reisen im Sommer 1900 siehe Eugene Weber: \emph{Richard Beer-Hofmann: Daten mitgeteilt von Eugene Weber}.
                     In: \emph{Modern Austrian Literature} 17/2 (1984), S. 13–42, hier: S. 23.}}}\label{K_L02911-8}? Arbeitet er an{ }ſeinem
                  Drama\pwindex{Beer-Hofmann, Richard 11.\,7.\,1866 Wien – 26.\,9.\,1945 New York City@\textsc{Beer-Hofmann, Richard} (11.\,7.\,1866 Wien – 26.\,9.\,1945 New York City), \emph{Schriftsteller}!Graf von Charolais. Ein Trauerspiel@\strich\emph{Der Graf von Charolais. Ein Trauerspiel}|pwv}? Und
               was wird er im Sommer machen? Wirſt Du mit ihm zuſammen{ }ſein?\pend
           
\pstart
           Geſtern{ }ſprach ich wieder einmal \textsc{Kerr\pwindex{Kerr, Alfred 25.\,12.\,1867 Breslau – 12.\,10.\,1948 Hamburg@\textsc{Kerr, Alfred} (25.\,12.\,1867 Breslau – 12.\,10.\,1948 Hamburg), \emph{Schriftsteller, Kritiker}|pw}} nach langer Pauſe. Er{ }ſcheint eine \label{K_L02911-9v}\edtext{große Liebe\pwindex{Wendt, Anna @\textsc{Wendt, Anna}|pwv}}{\lemma{\textnormal{\emph{große Liebe}}}\Cendnote{\textnormal{Bezug auf Anna Wendt\pwindex{Wendt, Anna @\textsc{Wendt, Anna}|pwk}, die Alfred
                     Kerr\pwindex{Kerr, Alfred 25.\,12.\,1867 Breslau – 12.\,10.\,1948 Hamburg@\textsc{Kerr, Alfred} (25.\,12.\,1867 Breslau – 12.\,10.\,1948 Hamburg), \emph{Schriftsteller, Kritiker}|pwk} im April 1900 kennengelernt hatte (vgl.
                     Deborah Vietor-Engländer: \emph{Alfred Kerr. Die
                        Biographie}. Reinbek bei Hamburg:
                        \emph{Rowohlt}{ }2016, S. 229 [E-Book-Ausgabe]).}}}\label{K_L02911-9} zu
               haben. Ich mag ihn{ }ſehr gern trotz mancher Geſchmack-Defekte; aber er{ }ſchließt{ }ſich
               mir nicht auf. {\pb}Und wir bleiben fremd.\pend
           
\pstart
           Wann{ }ſehe ich Dich wieder? Wann kommſt Du nach \label{K_L02911-10v}\edtext{Berlin\oindex{Berlin@\textbf{Berlin}, \emph{Hauptstadt}|pw}}{\lemma{\textnormal{\emph{Berlin}}}\Cendnote{\textnormal{Siehe XXXX Auszeichnungsfehler: Dokument L02910 nicht gefunden.
               }}}\label{K_L02911-10}?\pend
           
\pstart
           Viele treue Grüße! {\\[\baselineskip]}Dein {\\[\baselineskip]}\spacefill\mbox{Paul Goldmann}\pend
           \leftskip=0em{}
\pstart
           \noindent{}Meine Mutter\pwindex{Goldmann, Clementine 15.\,5.\,1842 Breslau – 24.\,2.\,1924 Frankfurt am Main@\textsc{Goldmann, Clementine} (15.\,5.\,1842 Breslau – 24.\,2.\,1924 Frankfurt am Main)|pwv} dankt für
                  Deine Grüße und erwidert{ }ſie herzlichſt.\pend
           \selectlanguage{ngerman}\endnumbering\briefempfaengerindex{Schnitzler, Arthur@\textsc{Schnitzler, Arthur}!zzzGoldmann, Paul@\emph{von Paul Goldmann}!1900-04-181@{18. 4. [1900]}|)be}\mylabel{L02911h}  \newcommand{\dateiname}{L02911}\newcommand{\titel}{Paul Goldmann an Arthur Schnitzler, 18. 4. [1900]}\newcommand{\editorInnen}{Martin Anton Müller und Laura Untner}%% latex-leseansicht-abspann.tex
%% Abspann für die Leseansicht.
%% Der Schalter \ifkorrekturansicht ist bereits durch den Vorspann gesetzt.

%% latex-abspann.tex
%% Gemeinsamer Abspann für Korrekturansicht und Leseansicht.
%% Setzt den Schalter \ifkorrekturansicht voraus (gesetzt in den
%% einbindenden Dateien latex-korrekturansicht-abspann.tex bzw.
%% latex-leseansicht-abspann.tex).
%% ---------------------------------------------------------------

\normalsize

% Das esempio-Environment wird nur in der Leseansicht benötigt
\ifkorrekturansicht\else
\newenvironment{esempio}[3]%
{
    \vspace{1.5ex}
    \rlap{\underline{#1}}
    \par
    \setlength{\parindent}{0cm}
    \nopagebreak
    \leftskip=#2cm
    \rightskip=#3cm
}
{
    \par
}
\fi

\doendnotes{C}
\bigskip
\vfill

\clearpage

\footnotesize

\ifkorrekturansicht
  \lohead{\textsc{register}}
\fi

% theindex-Environment neu definieren ohne reledmac
\makeatletter
\renewenvironment{theindex}{%
  \ifkorrekturansicht
    \section*{\indexname}%
  \else
    \subsubsection*{Index der erwähnten Entitäten}%
  \fi
  \setlength{\parindent}{0pt}%
  \setlength{\parskip}{0pt plus 0.3pt}%
  \let\item\@idxitem
}{%
  \ifkorrekturansicht\clearpage\fi
}
\makeatother

\IfFileExists{\jobname-pw.ind}{\input{\jobname-pw.ind}}{}

% Quellenangabe nur in der Leseansicht
\ifkorrekturansicht\else
% Fallback-Definitionen, falls die .tex-Datei \titel etc. nicht gesetzt hat
\providecommand{\titel}{}
\providecommand{\editorInnen}{}
\providecommand{\dateiname}{\jobname}

\vspace{3cm}

\vfill

\footnotesize
\textsc{Quelle}: \titel. Herausgegeben von {\editorInnen}. In: \emph{Arthur Schnitzler: Briefwechsel mit Autorinnen und Autoren}.
 Digitale Edition, https://schnitzler-briefe.acdh.oeaw.ac.at/{\dateiname}.html (Stand \today)
\fi

\end{document}


