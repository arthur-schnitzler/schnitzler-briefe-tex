%% latex-leseansicht-vorspann.tex
%% Vorspann für die Leseansicht.
%% Lädt die gemeinsame Datei latex-vorspann.tex mit nicht gesetztem Schalter.

\newif\ifkorrekturansicht
\korrekturansichtfalse

\input{../tex-inputs/latex-vorspann}


\section[Arthur Schnitzler an Richard Beer-Hofmann, 12. 1. 1892]{L00060 Arthur Schnitzler an Richard Beer-Hofmann, 12. 1. 1892}
\nopagebreak\mylabel{L00060v}
\rehead{ }\normalsize\beginnumbering\briefempfaengerindex{Beer-Hofmann, Richard@\textsc{Beer-Hofmann, Richard}!zzzSchnitzler, Arthur@\emph{von Arthur Schnitzler}!1892-01-121@{12. 1. 1892}|(be}
\toendnotes[C]{\smallbreak\pagebreak[2]}
\correspDesc{Versand  durch Arthur Schnitzler am 12. 1. 1892 in Wien
\newline{}Erhalt  durch Richard Beer-Hofmann am 12. 1. 1892 in Wien}\toendnotes[C]{\smallbreak}
\Standort{YCGL, MSS 31.}
\physDesc{Postkarte, 223 Zeichen
\newline{}Handschrift: Bleistift, deutsche Kurrent
\newline{}Versand: 1) Rohrpost  2) Stempel: »\nobreak{}\oindex{Wien@\textbf{Wien}, \emph{Verwaltungsgebiet}|pwk}Wien
                                                  Fleischmarkt, 1\textcolor{gray}{2} 1 92, 3–4 N\nobreak{}«.  3) Stempel: »\nobreak{}\oindex{III., Landstraße@\textbf{III., Landstraße}, \emph{Verwaltungsgebiet}|pwk}Wien 3/1, 12. 11. 92, 4–5 N\nobreak{}«. }\pstart{}\textsc{{\pb}Hrn Dr. Richard Beer-Hofmann}\pend{}\pstart{}\textsc{Wien\oindex{Wien@\textbf{Wien}, \emph{Verwaltungsgebiet}|pw}}\pend{}\pstart{}\textsc{III. Seidlgasse 30\oindex{Wien@\textbf{Wien}!III., Landstraße@\textbf{III., Landstraße}!Seidlgasse@\textbf{Seidlgasse}, \emph{Straße}|pw}.}\pend{}{\bigskip}\vspace{1em}
\pstart
           \noindent{}{\pb}Lieber Richard,\pend
           \settowidth{\longeste}{P. G.:}\settowidth{\longestz}{Paris, 51, rue Vivienne,}\settowidth{\longestd}{}\settowidth{\longestv}{}\settowidth{\longestf}{}\addtolength\longeste{1em}
        \addtolength\longestz{1em}
      \pstart\noindent\makebox[\the\longeste][l]{P. G.\pwindex{Goldmann, Paul 31.\,1.\,1865 Breslau – 25.\,9.\,1935 Wien@\textsc{Goldmann, Paul} (31.\,1.\,1865 Breslau – 25.\,9.\,1935 Wien), \emph{Schriftsteller, Journalist}|pw}:}\makebox[\the\longestz][l]{\textsc{Paris, 51, rue Vivienne\oindex{rue Vivienne@\textbf{rue Vivienne}, \emph{Straße}|pw},}}
                  \pend\pstart\noindent\makebox[\the\longeste][l]{}\makebox[\the\longestz][l]{\textsc{gazette de Francfort\orgindex{Frankfurter Zeitung@Frankfurter Zeitung|pw}.}}
                  \pend
\pstart
           Ob ich heute im Caffé weiſs ich nicht; hoffentlich aber{ }ſehen wir uns \uline{ſehr} bald.\pend
           \pstart Herzlich Ihr getreuer\spacefill\mbox{Arthur.}\pend{}\selectlanguage{ngerman}\endnumbering\briefempfaengerindex{Beer-Hofmann, Richard@\textsc{Beer-Hofmann, Richard}!zzzSchnitzler, Arthur@\emph{von Arthur Schnitzler}!1892-01-121@{12. 1. 1892}|)be}\mylabel{L00060h}  \newcommand{\dateiname}{L00060}\newcommand{\titel}{Arthur Schnitzler an Richard Beer-Hofmann, 12. 1. 1892}\newcommand{\editorInnen}{Martin Anton Müller und Gerd-Hermann Susen}%% latex-leseansicht-abspann.tex
%% Abspann für die Leseansicht.
%% Der Schalter \ifkorrekturansicht ist bereits durch den Vorspann gesetzt.

%% latex-abspann.tex
%% Gemeinsamer Abspann für Korrekturansicht und Leseansicht.
%% Setzt den Schalter \ifkorrekturansicht voraus (gesetzt in den
%% einbindenden Dateien latex-korrekturansicht-abspann.tex bzw.
%% latex-leseansicht-abspann.tex).
%% ---------------------------------------------------------------

\normalsize

% Das esempio-Environment wird nur in der Leseansicht benötigt
\ifkorrekturansicht\else
\newenvironment{esempio}[3]%
{
    \vspace{1.5ex}
    \rlap{\underline{#1}}
    \par
    \setlength{\parindent}{0cm}
    \nopagebreak
    \leftskip=#2cm
    \rightskip=#3cm
}
{
    \par
}
\fi

\doendnotes{C}
\bigskip
\vfill

\clearpage

\footnotesize

\ifkorrekturansicht
  \lohead{\textsc{register}}
\fi

% theindex-Environment neu definieren ohne reledmac
\makeatletter
\renewenvironment{theindex}{%
  \ifkorrekturansicht
    \section*{\indexname}%
  \else
    \subsubsection*{Index der erwähnten Entitäten}%
  \fi
  \setlength{\parindent}{0pt}%
  \setlength{\parskip}{0pt plus 0.3pt}%
  \let\item\@idxitem
}{%
  \ifkorrekturansicht\clearpage\fi
}
\makeatother

\IfFileExists{\jobname-pw.ind}{\input{\jobname-pw.ind}}{}

% Quellenangabe nur in der Leseansicht
\ifkorrekturansicht\else
% Fallback-Definitionen, falls die .tex-Datei \titel etc. nicht gesetzt hat
\providecommand{\titel}{}
\providecommand{\editorInnen}{}
\providecommand{\dateiname}{\jobname}

\vspace{3cm}

\vfill

\footnotesize
\textsc{Quelle}: \titel. Herausgegeben von {\editorInnen}. In: \emph{Arthur Schnitzler: Briefwechsel mit Autorinnen und Autoren}.
 Digitale Edition, https://schnitzler-briefe.acdh.oeaw.ac.at/{\dateiname}.html (Stand \today)
\fi

\end{document}


