%% latex-leseansicht-vorspann.tex
%% Vorspann für die Leseansicht.
%% Lädt die gemeinsame Datei latex-vorspann.tex mit nicht gesetztem Schalter.

\newif\ifkorrekturansicht
\korrekturansichtfalse

\input{../tex-inputs/latex-vorspann}


         
         \newcommand{\erwaehntePersonen}{Personen: Felix Braun}
         \newcommand{\erwaehnteInstitutionen}{}
         \newcommand{\erwaehnteOrte}{Orte: Karlsruhe, Sieveringer Straße, Sternwartestraße, Stuttgart, Wien, XIX., Döbling, XVIII., Währing}
         \newcommand{\erwaehnteWerke}{Werke: Tantalos}
               \section[Arthur Schnitzler an Felix Braun, {[}nach dem 15. 3. 1926?{]}]{ Arthur Schnitzler an Felix Braun, {[}nach dem 15. 3. 1926?{]}}\nopagebreak\mylabel{v}\rehead{ }\begin{ledgroupsized}[t]{13cm}\normalsize\beginnumbering \toendnotes[C]{\smallbreak\pagebreak[2]} \Standort{Wienbibliothek im Rathaus, H.I.N.-198.048.}
\physDesc{Bildpostkarte
\newline{}Handschrift: schwarze Tinte, deutsche Kurrent}\toendnotes[C]{\smallbreak}\pstart{}{\pb}\label{T_L02469-1v}\edtext{\textcolor{gray}{\textbf{A. S.}}}{\lemma{\textnormal{\emph{A. S.}}}\Cendnote{\textnormal{ovaler Absenderkleber}}}\label{T_L02469-1h}\pend{}\pstart{}\textcolor{gray}{\textbf{WIEN, XVIII.}}\oindex{XVIII., Waehring@\textbf{XVIII., Währing}|pw}\pend{}\pstart{}\textcolor{gray}{\textbf{STERNWARTESTR. 71}}\oindex{Sternwartestrasse@\textbf{Sternwartestraße}|pw}\pend{}{\bigskip}\pstart{}Hrn \textsc{Felix Braun}\pend{}\pstart{}\textsc{Wien} XIX\oindex{XIX., Doebling@\textbf{XIX., Döbling}|pw}\pend{}\pstart{}\textsc{Sieveringer Straße} 93\oindex{Sieveringer Strasse@\textbf{Sieveringer Straße}|pw}\pend{}{\bigskip}\pstart
           \noindent{}\centering{}{\pb}{[}Sternwartestraße 71\oindex{Sternwartestrasse@\textbf{Sternwartestraße}|pw}{]}\pend
           \pstart
           {\pb}Schönen Dank für ihren lieben Brief,
                    u Glückwunſch zum \label{K_L02469_1v}\edtext{Stuttgart\oindex{Stuttgart@\textbf{Stuttgart}|pw}er Erfolg}{\lemma{\textnormal{\emph{Stuttgarter Erfolg}}}\Cendnote{\textnormal{Offensichtlich eine Verwechslung mit Karlsruhe\oindex{Karlsruhe@\textbf{Karlsruhe}|pwk}, wovon Braun\pwindex{Braun, Felix 04.11.1885 – 29.11.1973@\textsc{Braun, Felix} (04.11.1885 – 29.11.1973), \emph{Schriftsteller}|pwk} sprach und wo am 27. 3. 1926 die Uraufführung
                        von \emph{Tantalos}\pwindex{Braun, Felix 04.11.1885 – 29.11.1973@\textsc{Braun, Felix} (04.11.1885 – 29.11.1973), \emph{Schriftsteller}!Tantalos1917@\strich\emph{Tantalos} {[}1917{]}|pwk} stattfindet. Die Karte ist
                        undatiert, der Stempel nicht verlässlich lesbar. Der Zeitraum
                            1925–1926 lässt sich durch das
                        Briefmarkenporto »8 Groschen« eingrenzen. Obzwar der Verlust
                        von Korrespondenzstücken nicht ausgeschlossen werden kann, spricht Braun\pwindex{Braun, Felix 04.11.1885 – 29.11.1973@\textsc{Braun, Felix} (04.11.1885 – 29.11.1973), \emph{Schriftsteller}|pwk} nur in Ausnahmefällen in Briefen
                        von sich selbst.}}}\label{K_L02469_1h}. Ihr ergebner\pend
           \pstart \spacefill\mbox{Arthur Schnitzler}\pend{}
         
         \endnumbering\mylabel{h}\end{ledgroupsized}  \newcommand{\dateiname}{L02469}\newcommand{\titel}{Arthur Schnitzler an Felix Braun, [nach dem 15. 3. 1926?]}\newcommand{\editorInnen}{Martin Anton Müller und Gerd-Hermann Susen}%% latex-leseansicht-abspann.tex
%% Abspann für die Leseansicht.
%% Der Schalter \ifkorrekturansicht ist bereits durch den Vorspann gesetzt.

%% latex-abspann.tex
%% Gemeinsamer Abspann für Korrekturansicht und Leseansicht.
%% Setzt den Schalter \ifkorrekturansicht voraus (gesetzt in den
%% einbindenden Dateien latex-korrekturansicht-abspann.tex bzw.
%% latex-leseansicht-abspann.tex).
%% ---------------------------------------------------------------

\normalsize

% Das esempio-Environment wird nur in der Leseansicht benötigt
\ifkorrekturansicht\else
\newenvironment{esempio}[3]%
{
    \vspace{1.5ex}
    \rlap{\underline{#1}}
    \par
    \setlength{\parindent}{0cm}
    \nopagebreak
    \leftskip=#2cm
    \rightskip=#3cm
}
{
    \par
}
\fi

\doendnotes{C}
\bigskip
\vfill

\clearpage

\footnotesize

\ifkorrekturansicht
  \lohead{\textsc{register}}
\fi

% theindex-Environment neu definieren ohne reledmac
\makeatletter
\renewenvironment{theindex}{%
  \ifkorrekturansicht
    \section*{\indexname}%
  \else
    \subsubsection*{Index der erwähnten Entitäten}%
  \fi
  \setlength{\parindent}{0pt}%
  \setlength{\parskip}{0pt plus 0.3pt}%
  \let\item\@idxitem
}{%
  \ifkorrekturansicht\clearpage\fi
}
\makeatother

\IfFileExists{\jobname-pw.ind}{\input{\jobname-pw.ind}}{}

% Quellenangabe nur in der Leseansicht
\ifkorrekturansicht\else
% Fallback-Definitionen, falls die .tex-Datei \titel etc. nicht gesetzt hat
\providecommand{\titel}{}
\providecommand{\editorInnen}{}
\providecommand{\dateiname}{\jobname}

\vspace{3cm}

\vfill

\footnotesize
\textsc{Quelle}: \titel. Herausgegeben von {\editorInnen}. In: \emph{Arthur Schnitzler: Briefwechsel mit Autorinnen und Autoren}.
 Digitale Edition, https://schnitzler-briefe.acdh.oeaw.ac.at/{\dateiname}.html (Stand \today)
\fi

\end{document}


      