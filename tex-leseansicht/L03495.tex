%% latex-leseansicht-vorspann.tex
%% Vorspann für die Leseansicht.
%% Lädt die gemeinsame Datei latex-vorspann.tex mit nicht gesetztem Schalter.

\newif\ifkorrekturansicht
\korrekturansichtfalse

\input{../tex-inputs/latex-vorspann}


         
         \renewcommand{\erwaehntePersonen}{Personen:  Raffaello Sanzio da Urbino, Felix Salten}
         \renewcommand{\erwaehnteOrte}{Orte: Bologna, Edmund-Weiß-Gasse 7, Florenz, Pinacoteca Nazionale di Bologna, Ravenna, Rimini, Wien, Österreich}
         \renewcommand{\erwaehnteWerke}{Werke: Der Schleier der Beatrice. Schauspiel in fünf Akten, Die Verzückung der Heiligen Cäcilia, Die Zeit, Unsichere Reise}
               \section[ Felix Salten an Arthur Schnitzler, 25. 4. 1908]{ Felix Salten an Arthur Schnitzler, 25. 4. 1908}\nopagebreak\mylabel{v}\rehead{ }\begin{ledgroupsized}[t]{13cm}\normalsize\beginnumbering\briefempfaengerindex{Schnitzler, Arthur@\textsc{Schnitzler, Arthur}!zzzSalten, Felix@\emph{von Felix Salten}!1908-04-251@{25. 4. 1908}|(be} \toendnotes[C]{\smallbreak\pagebreak[2]} \Standort{CUL, Schnitzler, B 89, B 1.}
\physDesc{Bildpostkarte, 188 Zeichen
\newline{}Handschrift: schwarze Tinte, lateinische Kurrent
\newline{}Versand: Stempel: »\nobreak{}\oindex{Bologna@\textbf{Bologna}|pwk}\textcolor{gray}{Bologna}, 25{[}. 4. 1908{]}\nobreak{}«.  
\newline{}Ordnung: mit Bleistift von unbekannter Hand nummeriert: »244« }\toendnotes[C]{\smallbreak}\pstart{}{\pb}Vienna\oindex{Wien@\textbf{Wien}|pw}{ }Austria\oindex{Oesterreich@\textbf{Österreich}|pw}\pend{}\pstart{}Herrn D\textsuperscript{r} Arthur Schnitzler\pend{}\pstart{}Wien\oindex{Wien@\textbf{Wien}|pw}\pend{}\pstart{}XVIII. Spöttelgaße 7\oindex{Edmund-Weiss-Gasse 7@\textbf{Edmund-Weiß-Gasse 7}|pw}\pend{}{\bigskip}\pstart
           \noindent{}\centering{}{\pb}\textcolor{gray}{\textbf{BOLOGNA – R. Pinacoteca\oindex{Pinacoteca Nazionale di Bologna@\textbf{Pinacoteca Nazionale di Bologna}|pw}. S. Cecilia\pwindex{Raffaello Sanzio da Urbino 28. 3. oder 6. 4. 1483 – 6.04.1520@\textsc{Raffaello Sanzio da Urbino} (28. 3. oder 6. 4. 1483 – 6.04.1520), \emph{Maler}!Verzueckung der Heiligen Caecilia1514@\strich\emph{Die Verzückung der Heiligen Cäcilia} {[}1514{]}|pw} (Raffaello
                        Sanzio\pwindex{Raffaello Sanzio da Urbino 28. 3. oder 6. 4. 1483 – 6.04.1520@\textsc{Raffaello Sanzio da Urbino} (28. 3. oder 6. 4. 1483 – 6.04.1520), \emph{Maler}|pw})}}\pend
           \pstart
           {\pb}»Das Leben ist die Fülle, nicht die Zeit {\dotstwo}\pwindex{Schnitzler, Arthur 15.05.1862 – 21.10.1931@\textsc{Schnitzler, Arthur} (15.05.1862 – 21.10.1931), \emph{Schriftsteller, Mediziner}!Schleier der Beatrice. Schauspiel in fuenf Akten1900-12-01@\strich\emph{Der Schleier der Beatrice. Schauspiel in fünf Akten} {[}1900-12-01{]}|pwv}« \pend
           \pstart
           Aus einem \label{K_L03495-1v}\edtext{Drama\pwindex{Schnitzler, Arthur 15.05.1862 – 21.10.1931@\textsc{Schnitzler, Arthur} (15.05.1862 – 21.10.1931), \emph{Schriftsteller, Mediziner}!Schleier der Beatrice. Schauspiel in fuenf Akten1900-12-01@\strich\emph{Der Schleier der Beatrice. Schauspiel in fünf Akten} {[}1900-12-01{]}|pwv}}{\lemma{\textnormal{\emph{Drama}}}\Cendnote{\textnormal{\emph{Der Schleier der Beatrice}\pwindex{Schnitzler, Arthur 15.05.1862 – 21.10.1931@\textsc{Schnitzler, Arthur} (15.05.1862 – 21.10.1931), \emph{Schriftsteller, Mediziner}!Schleier der Beatrice. Schauspiel in fuenf Akten1900-12-01@\strich\emph{Der Schleier der Beatrice. Schauspiel in fünf Akten} {[}1900-12-01{]}|pwk}; das Zitat sind die
                  Schlussworte des Herzog\pwindex{Schnitzler, Arthur 15.05.1862 – 21.10.1931@\textsc{Schnitzler, Arthur} (15.05.1862 – 21.10.1931), \emph{Schriftsteller, Mediziner}!Schleier der Beatrice. Schauspiel in fuenf Akten1900-12-01@\strich\emph{Der Schleier der Beatrice. Schauspiel in fünf Akten} {[}1900-12-01{]}|pwkv}s}}}\label{K_L03495-1h}, das hier in \label{K_L03495-2v}\edtext{Bologna\oindex{Bologna@\textbf{Bologna}|pw}}{\lemma{\textnormal{\emph{Bologna}}}\Cendnote{\textnormal{Am Ende seines
                     Feuilletons \emph{Unsichere Reise}\pwindex{Salten, Felix 06.09.1869 – 08.10.1945@\textsc{Salten, Felix} (06.09.1869 – 08.10.1945), \emph{Schriftsteller, Journalist}!Unsichere Reise1908-04-26@\strich\emph{Unsichere Reise} {[}1908-04-26{]}|pwk} (Felix Salten\pwindex{Salten, Felix 06.09.1869 – 08.10.1945@\textsc{Salten, Felix} (06.09.1869 – 08.10.1945), \emph{Schriftsteller, Journalist}|pwk}: \emph{Unsichere Reise}\pwindex{Salten, Felix 06.09.1869 – 08.10.1945@\textsc{Salten, Felix} (06.09.1869 – 08.10.1945), \emph{Schriftsteller, Journalist}!Unsichere Reise1908-04-26@\strich\emph{Unsichere Reise} {[}1908-04-26{]}|pwk}. In: \emph{Die Zeit}\pwindex{Zeit1902-09-27 – 1919@\emph{Die Zeit} {[}1902-09-27 – 1919{]}|pwk}, 
                        Jg. 7, Nr. 2.008, 26. 4. 1908, Morgenblatt, S. 1–3, hier 3.) überlegt der Erzähler/Salten\pwindex{Salten, Felix 06.09.1869 – 08.10.1945@\textsc{Salten, Felix} (06.09.1869 – 08.10.1945), \emph{Schriftsteller, Journalist}|pwk} noch,
                     ob er tatsächlich weiter nach Bologna\oindex{Bologna@\textbf{Bologna}|pwk} und Florenz\oindex{Florenz@\textbf{Florenz}|pwk} soll und nicht eine andere Route wählen und
                     nach Ravenna\oindex{Ravenna@\textbf{Ravenna}|pwk} und Rimini\oindex{Rimini@\textbf{Rimini}|pwk} zu gehen,
                    wo er noch
                  nie war.}}}\label{K_L03495-2h}
               spielt, mit herzlichen Grüßen {\\}Ihr {\\}\spacefill\mbox{Salten}\pend
           \pstart
           25./4. 08\pend
           
         
         \endnumbering\mylabel{h}\end{ledgroupsized}  \newcommand{\dateiname}{L03495}\newcommand{\titel}{Felix Salten an Arthur Schnitzler, 25. 4. 1908}\newcommand{\editorInnen}{Martin Anton Müller und Laura Untner}%% latex-leseansicht-abspann.tex
%% Abspann für die Leseansicht.
%% Der Schalter \ifkorrekturansicht ist bereits durch den Vorspann gesetzt.

%% latex-abspann.tex
%% Gemeinsamer Abspann für Korrekturansicht und Leseansicht.
%% Setzt den Schalter \ifkorrekturansicht voraus (gesetzt in den
%% einbindenden Dateien latex-korrekturansicht-abspann.tex bzw.
%% latex-leseansicht-abspann.tex).
%% ---------------------------------------------------------------

\normalsize

% Das esempio-Environment wird nur in der Leseansicht benötigt
\ifkorrekturansicht\else
\newenvironment{esempio}[3]%
{
    \vspace{1.5ex}
    \rlap{\underline{#1}}
    \par
    \setlength{\parindent}{0cm}
    \nopagebreak
    \leftskip=#2cm
    \rightskip=#3cm
}
{
    \par
}
\fi

\doendnotes{C}
\bigskip
\vfill

\clearpage

\footnotesize

\ifkorrekturansicht
  \lohead{\textsc{register}}
\fi

% theindex-Environment neu definieren ohne reledmac
\makeatletter
\renewenvironment{theindex}{%
  \ifkorrekturansicht
    \section*{\indexname}%
  \else
    \subsubsection*{Index der erwähnten Entitäten}%
  \fi
  \setlength{\parindent}{0pt}%
  \setlength{\parskip}{0pt plus 0.3pt}%
  \let\item\@idxitem
}{%
  \ifkorrekturansicht\clearpage\fi
}
\makeatother

\IfFileExists{\jobname-pw.ind}{\input{\jobname-pw.ind}}{}

% Quellenangabe nur in der Leseansicht
\ifkorrekturansicht\else
% Fallback-Definitionen, falls die .tex-Datei \titel etc. nicht gesetzt hat
\providecommand{\titel}{}
\providecommand{\editorInnen}{}
\providecommand{\dateiname}{\jobname}

\vspace{3cm}

\vfill

\footnotesize
\textsc{Quelle}: \titel. Herausgegeben von {\editorInnen}. In: \emph{Arthur Schnitzler: Briefwechsel mit Autorinnen und Autoren}.
 Digitale Edition, https://schnitzler-briefe.acdh.oeaw.ac.at/{\dateiname}.html (Stand \today)
\fi

\end{document}


      