%% latex-leseansicht-vorspann.tex
%% Vorspann für die Leseansicht.
%% Lädt die gemeinsame Datei latex-vorspann.tex mit nicht gesetztem Schalter.

\newif\ifkorrekturansicht
\korrekturansichtfalse

\input{../tex-inputs/latex-vorspann}


\section[ Felix Salten an Arthur Schnitzler, 25. 4. 1908]{L03495 Felix Salten an Arthur Schnitzler,  25. 4. 1908}
\nopagebreak\mylabel{L03495v}
\rehead{ }\normalsize\beginnumbering\briefempfaengerindex{Schnitzler, Arthur@\textsc{Schnitzler, Arthur}!zzzSalten, Felix@\emph{von Felix Salten}!1908-04-251@{25. 4. 1908}|(be}
\toendnotes[C]{\smallbreak\pagebreak[2]}
\correspDesc{Versand  durch Felix Salten am 25. 4. 1908 in Bologna
\newline{}Erhalt  durch Arthur Schnitzler am [14. 5. 1908?] in Wien}\toendnotes[C]{\smallbreak}
\Standort{CUL, Schnitzler, B 89, B 1.}
\physDesc{Bildpostkarte, 188 Zeichen
\newline{}Handschrift: schwarze Tinte, lateinische Kurrent
\newline{}Versand: Stempel: »\nobreak{}\oindex{Bologna@\textbf{Bologna}|pwk}\textcolor{gray}{Bologna}, 25{[}. 4. 1908{]}\nobreak{}«.  
\newline{}Ordnung: mit Bleistift von unbekannter Hand nummeriert: »244« }\toendnotes[C]{\smallbreak}\pstart{}{\pb}Vienna\oindex{Wien@\textbf{Wien}, \emph{Verwaltungsgebiet}|pw}{ }Austria\oindex{Österreich@\textbf{Österreich}|pw}\pend{}\pstart{}Herrn D\textsuperscript{r} Arthur Schnitzler\pend{}\pstart{}Wien\oindex{Wien@\textbf{Wien}, \emph{Verwaltungsgebiet}|pw}\pend{}\pstart{}XVIII. Spöttelgaße 7\oindex{Wien@\textbf{Wien}!XVIII., Währing@\textbf{XVIII., Währing}!Edmund-Weiß-Gasse 7@\textbf{Edmund-Weiß-Gasse 7}, \emph{Wohngebäude}|pw}\pend{}{\bigskip}
\pstart
           \noindent{}\centering{}{\pb}\textcolor{gray}{\textbf{BOLOGNA – R. Pinacoteca\oindex{Pinacoteca Nazionale di Bologna@\textbf{Pinacoteca Nazionale di Bologna}, \emph{Museum}|pw}. S. Cecilia\pwindex{Raffaello Sanzio da Urbino 28. 3. oder 6. 4. 1483 Urbino – 6.\,4.\,1520 Rom@\textsc{Raffaello Sanzio da Urbino} (28. 3. oder 6. 4. 1483 Urbino – 6.\,4.\,1520 Rom), \emph{Maler}!Verzückung der Heiligen Cäcilia@\strich\emph{Die Verzückung der Heiligen Cäcilia}|pw} (Raffaello
                        Sanzio\pwindex{Raffaello Sanzio da Urbino 28. 3. oder 6. 4. 1483 Urbino – 6.\,4.\,1520 Rom@\textsc{Raffaello Sanzio da Urbino} (28. 3. oder 6. 4. 1483 Urbino – 6.\,4.\,1520 Rom), \emph{Maler}|pw})}}\pend
           \vspace{1em}
\pstart
           \noindent{}{\pb}»Das Leben ist die Fülle, nicht die Zeit {\dotstwo}\pwindex{Schnitzler, Arthur 15.\,5.\,1862 Wien – 21.\,10.\,1931 ebd.@\textsc{Schnitzler, Arthur} (15.\,5.\,1862 Wien – 21.\,10.\,1931 ebd.), \emph{Schriftsteller, Mediziner}!Schleier der Beatrice. Schauspiel in fünf Akten@\strich\emph{Der Schleier der Beatrice. Schauspiel in fünf Akten}|pwv}«\pend
           
\pstart
           Aus einem \label{K_L03495-1v}\edtext{Drama\pwindex{Schnitzler, Arthur 15.\,5.\,1862 Wien – 21.\,10.\,1931 ebd.@\textsc{Schnitzler, Arthur} (15.\,5.\,1862 Wien – 21.\,10.\,1931 ebd.), \emph{Schriftsteller, Mediziner}!Schleier der Beatrice. Schauspiel in fünf Akten@\strich\emph{Der Schleier der Beatrice. Schauspiel in fünf Akten}|pwv}}{\lemma{\textnormal{\emph{Drama}}}\Cendnote{\textnormal{Bei dem Zitat handelt es sich um die Schlussworte
                  von 
                  \emph{Der Schleier der Beatrice}\pwindex{Schnitzler, Arthur 15.\,5.\,1862 Wien – 21.\,10.\,1931 ebd.@\textsc{Schnitzler, Arthur} (15.\,5.\,1862 Wien – 21.\,10.\,1931 ebd.), \emph{Schriftsteller, Mediziner}!Schleier der Beatrice. Schauspiel in fünf Akten@\strich\emph{Der Schleier der Beatrice. Schauspiel in fünf Akten}|pwk}.}}}\label{K_L03495-1}, das hier in \label{K_L03495-2v}\edtext{Bologna\oindex{Bologna@\textbf{Bologna}|pw}}{\lemma{\textnormal{\emph{Bologna}}}\Cendnote{\textnormal{Am Ende seines
                     Feuilletons \emph{Unsichere Reise}\pwindex{Salten, Felix 6.\,9.\,1869 Budapest – 8.\,10.\,1945 Zürich@\textsc{Salten, Felix} (6.\,9.\,1869 Budapest – 8.\,10.\,1945 Zürich), \emph{Schriftsteller, Journalist, Chefredakteur}!Unsichere Reise@\strich\emph{Unsichere Reise}|pwk} (Felix Salten\pwindex{Salten, Felix 6.\,9.\,1869 Budapest – 8.\,10.\,1945 Zürich@\textsc{Salten, Felix} (6.\,9.\,1869 Budapest – 8.\,10.\,1945 Zürich), \emph{Schriftsteller, Journalist, Chefredakteur}|pwk}: \emph{Unsichere Reise}\pwindex{Salten, Felix 6.\,9.\,1869 Budapest – 8.\,10.\,1945 Zürich@\textsc{Salten, Felix} (6.\,9.\,1869 Budapest – 8.\,10.\,1945 Zürich), \emph{Schriftsteller, Journalist, Chefredakteur}!Unsichere Reise@\strich\emph{Unsichere Reise}|pwk}. In: \emph{Die Zeit}\pwindex{Zeit@\emph{Die Zeit}|pwk}, 
                        Jg. 7, Nr. 2008, 26. 4. 1908, Morgenblatt, S. 1–3, hier 3.) überlegt der Erzähler/Salten\pwindex{Salten, Felix 6.\,9.\,1869 Budapest – 8.\,10.\,1945 Zürich@\textsc{Salten, Felix} (6.\,9.\,1869 Budapest – 8.\,10.\,1945 Zürich), \emph{Schriftsteller, Journalist, Chefredakteur}|pwk} noch,
                     ob er tatsächlich weiter nach Bologna\oindex{Bologna@\textbf{Bologna}|pwk} und Florenz\oindex{Florenz@\textbf{Florenz}|pwk} fahren solle. Stattdessen spielt er
                  mit dem Plan einer anderen Route, die ihn
                     nach Ravenna\oindex{Ravenna@\textbf{Ravenna}, \emph{Hauptstadt}|pwk} und Rimini\oindex{Rimini@\textbf{Rimini}, \emph{Hauptstadt}|pwk} führen würde, 
                    wo er noch
                  nie war.}}}\label{K_L03495-2}
               spielt, mit herzlichen Grüßen {\\}Ihr {\\}\spacefill\mbox{Salten}\pend
           
\pstart
           25./4. 08\pend
           \selectlanguage{ngerman}\endnumbering\briefempfaengerindex{Schnitzler, Arthur@\textsc{Schnitzler, Arthur}!zzzSalten, Felix@\emph{von Felix Salten}!1908-04-251@{25. 4. 1908}|)be}\mylabel{L03495h}  \newcommand{\dateiname}{L03495}\newcommand{\titel}{Felix Salten an Arthur Schnitzler, 25. 4. 1908}\newcommand{\editorInnen}{Martin Anton Müller und Laura Untner}%% latex-leseansicht-abspann.tex
%% Abspann für die Leseansicht.
%% Der Schalter \ifkorrekturansicht ist bereits durch den Vorspann gesetzt.

%% latex-abspann.tex
%% Gemeinsamer Abspann für Korrekturansicht und Leseansicht.
%% Setzt den Schalter \ifkorrekturansicht voraus (gesetzt in den
%% einbindenden Dateien latex-korrekturansicht-abspann.tex bzw.
%% latex-leseansicht-abspann.tex).
%% ---------------------------------------------------------------

\normalsize

% Das esempio-Environment wird nur in der Leseansicht benötigt
\ifkorrekturansicht\else
\newenvironment{esempio}[3]%
{
    \vspace{1.5ex}
    \rlap{\underline{#1}}
    \par
    \setlength{\parindent}{0cm}
    \nopagebreak
    \leftskip=#2cm
    \rightskip=#3cm
}
{
    \par
}
\fi

\doendnotes{C}
\bigskip
\vfill

\clearpage

\footnotesize

\ifkorrekturansicht
  \lohead{\textsc{register}}
\fi

% theindex-Environment neu definieren ohne reledmac
\makeatletter
\renewenvironment{theindex}{%
  \ifkorrekturansicht
    \section*{\indexname}%
  \else
    \subsubsection*{Index der erwähnten Entitäten}%
  \fi
  \setlength{\parindent}{0pt}%
  \setlength{\parskip}{0pt plus 0.3pt}%
  \let\item\@idxitem
}{%
  \ifkorrekturansicht\clearpage\fi
}
\makeatother

\IfFileExists{\jobname-pw.ind}{\input{\jobname-pw.ind}}{}

% Quellenangabe nur in der Leseansicht
\ifkorrekturansicht\else
% Fallback-Definitionen, falls die .tex-Datei \titel etc. nicht gesetzt hat
\providecommand{\titel}{}
\providecommand{\editorInnen}{}
\providecommand{\dateiname}{\jobname}

\vspace{3cm}

\vfill

\footnotesize
\textsc{Quelle}: \titel. Herausgegeben von {\editorInnen}. In: \emph{Arthur Schnitzler: Briefwechsel mit Autorinnen und Autoren}.
 Digitale Edition, https://schnitzler-briefe.acdh.oeaw.ac.at/{\dateiname}.html (Stand \today)
\fi

\end{document}


