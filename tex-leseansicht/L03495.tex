%% latex-korrekturansicht-vorspann.tex
%% Vorspann für die Korrekturansicht.
%% Lädt die gemeinsame Datei latex-vorspann.tex mit gesetztem Schalter.

\newif\ifkorrekturansicht
\korrekturansichttrue

\input{../tex-inputs/latex-vorspann}


\section[ Felix Salten an Arthur Schnitzler, 25. 4. 1908]{L03495 Felix Salten an Arthur Schnitzler, 25. 4. 1908}
\nopagebreak\mylabel{L03495v}
\rehead{ }\normalsize\beginnumbering\briefempfaengerindex{Schnitzler, Arthur@\textsc{Schnitzler, Arthur}!zzzSalten, Felix@\emph{von Felix Salten}!1908-04-251@{25. 4. 1908}|(be}
\toendnotes[C]{\smallbreak\pagebreak[2]}\Standort{CUL, Schnitzler, B 89, B 1.}
\physDesc{Bildpostkarte, 188 Zeichen
\newline{}Handschrift: schwarze Tinte, lateinische Kurrent
\newline{}Versand: Stempel: »\nobreak{}\oindex{Bologna@\textbf{Bologna}, \emph{P.PPLA}|pwk}\textcolor{gray}{Bologna}, 25{[}. 4. 1908{]}\nobreak{}«.  
\newline{}Ordnung: mit Bleistift von unbekannter Hand nummeriert: »244« }\toendnotes[C]{\smallbreak}\pstart{}{\pb}Vienna\oindex{Wien@\textbf{Wien}, \emph{A.ADM2}|pw}{ }Austria\oindex{Oesterreich@\textbf{Österreich}, \emph{A.PCLI}|pw}\pend{}\pstart{}Herrn D\textsuperscript{r} Arthur Schnitzler\pend{}\pstart{}Wien\oindex{Wien@\textbf{Wien}, \emph{A.ADM2}|pw}\pend{}\pstart{}XVIII. Spöttelgaße 7\oindex{Edmund-Weiss-Gasse 7@\textbf{Edmund-Weiß-Gasse 7}, \emph{Wohngebäude (K.WHS)}|pw}\pend{}{\bigskip}
\pstart
           \noindent{}\centering{}{\pb}\textcolor{gray}{\textbf{BOLOGNA – R. Pinacoteca\oindex{Pinacoteca Nazionale di Bologna@\textbf{Pinacoteca Nazionale di Bologna}, \emph{Museum (K.MUS)}|pw}. S. Cecilia\pwindex{Verzueckung der Heiligen Caecilia@\emph{Die Verzückung der Heiligen Cäcilia}|pw} (Raffaello
                        Sanzio\pwindex{Raffaello Sanzio da Urbino 28. 3. oder 6. 4. 1483 – 6.04.1520@\textsc{Raffaello Sanzio da Urbino} (28. 3. oder 6. 4. 1483 – 6.04.1520), \emph{Maler/Malerin}|pw})}}\pend
           \vspace{1em}
\pstart
           \noindent{}{\pb}»Das Leben ist die Fülle, nicht die Zeit {\dotstwo}\pwindex{Schleier der Beatrice. Schauspiel in fuenf Akten@\emph{Der Schleier der Beatrice. Schauspiel in fünf Akten}|pwv}« \pend
           
\pstart
           Aus einem \label{K_L03495-1v}\edtext{Drama\pwindex{Schleier der Beatrice. Schauspiel in fuenf Akten@\emph{Der Schleier der Beatrice. Schauspiel in fünf Akten}|pwv}}{\lemma{\textnormal{\emph{Drama}}}\Cendnote{\textnormal{Bei dem Zitat handelt es sich um die Schlussworte
                  von 
                  \emph{Der Schleier der Beatrice}\pwindex{Schleier der Beatrice. Schauspiel in fuenf Akten@\emph{Der Schleier der Beatrice. Schauspiel in fünf Akten}|pwk}.}}}\label{K_L03495-1}, das hier in \label{K_L03495-2v}\edtext{Bologna\oindex{Bologna@\textbf{Bologna}, \emph{P.PPLA}|pw}}{\lemma{\textnormal{\emph{Bologna}}}\Cendnote{\textnormal{Am Ende seines
                     Feuilletons \emph{Unsichere Reise}\pwindex{Unsichere Reise@\emph{Unsichere Reise}|pwk} (Felix Salten\pwindex{Salten, Felix 06.09.1869 – 08.10.1945@\textsc{Salten, Felix} (06.09.1869 – 08.10.1945), \emph{Schriftsteller/Schriftstellerin, Journalist/Journalistin, Chefredakteur/Chefredakteurin}|pwk}: \emph{Unsichere Reise}\pwindex{Unsichere Reise@\emph{Unsichere Reise}|pwk}. In: \emph{Die Zeit}\pwindex{Zeit@\emph{Die Zeit}|pwk}, 
                        Jg. 7, Nr. 2008, 26. 4. 1908, Morgenblatt, S. 1–3, hier 3.) überlegt der Erzähler/Salten\pwindex{Salten, Felix 06.09.1869 – 08.10.1945@\textsc{Salten, Felix} (06.09.1869 – 08.10.1945), \emph{Schriftsteller/Schriftstellerin, Journalist/Journalistin, Chefredakteur/Chefredakteurin}|pwk} noch,
                     ob er tatsächlich weiter nach Bologna\oindex{Bologna@\textbf{Bologna}, \emph{P.PPLA}|pwk} und Florenz\oindex{Florenz@\textbf{Florenz}, \emph{P.PPLA}|pwk} fahren solle. Stattdessen spielt er
                  mit dem Plan einer anderen Route, die ihn
                     nach Ravenna\oindex{Ravenna@\textbf{Ravenna}, \emph{P.PPLA2}|pwk} und Rimini\oindex{Rimini@\textbf{Rimini}, \emph{P.PPLA2}|pwk} führen würde, 
                    wo er noch
                  nie war.}}}\label{K_L03495-2}
               spielt, mit herzlichen Grüßen {\\}Ihr {\\}\spacefill\mbox{Salten}\pend
           
\pstart
           25./4. 08\pend
           \selectlanguage{ngerman}\endnumbering\briefempfaengerindex{Schnitzler, Arthur@\textsc{Schnitzler, Arthur}!zzzSalten, Felix@\emph{von Felix Salten}!1908-04-251@{25. 4. 1908}|)be}\mylabel{L03495h}  \normalsize

\doendnotes{C}
\bigskip
\vfill

\clearpage

\footnotesize

\lohead{\textsc{register}}

% Definiere theindex-Environment komplett neu ohne reledmac
\makeatletter
\renewenvironment{theindex}{%
  \section*{\indexname}%
  \setlength{\parindent}{0pt}%
  \setlength{\parskip}{0pt plus 0.3pt}%
  \let\item\@idxitem
}{%
  \clearpage
}
\makeatother

\IfFileExists{\jobname-pw.ind}{\input{\jobname-pw.ind}}{}

\end{document}

      