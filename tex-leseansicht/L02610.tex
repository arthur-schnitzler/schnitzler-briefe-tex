%% latex-leseansicht-vorspann.tex
%% Vorspann für die Leseansicht.
%% Lädt die gemeinsame Datei latex-vorspann.tex mit nicht gesetztem Schalter.

\newif\ifkorrekturansicht
\korrekturansichtfalse

\input{../tex-inputs/latex-vorspann}


         
         \renewcommand{\erwaehntePersonen}{Personen: Richard Beer-Hofmann, Paul Goldmann, Hugo von Hofmannsthal, Leopold Sonnemann}
         \renewcommand{\erwaehnteInstitutionen}{Institutionen: Frankfurter Zeitung, Theater Orange}
         \renewcommand{\erwaehnteOrte}{Orte: Bad Ischl, Florenz, Frankfurt am Main, Frankreich, Genua, Italien, Orange, Paris, Pisa, Salzburg, Südtirol, Tirol, Venedig, rue Feydeau}
         \renewcommand{\erwaehnteWerke}{}
               \section[ Paul Goldmann an Arthur Schnitzler, 9. 8. {[}1894{]}]{ Paul Goldmann an Arthur Schnitzler, 9. 8. {[}1894{]}}\nopagebreak\mylabel{v}\rehead{ }\begin{ledgroupsized}[t]{13cm}\normalsize\beginnumbering \toendnotes[C]{\smallbreak\pagebreak[2]} \Standort{DLA, A:Schnitzler, HS.NZ85.1.3164.}
\physDesc{Brief, 1 Blatt, 4 Seiten, 1734 Zeichen
\newline{}Handschrift: schwarze Tinte, deutsche Kurrent
\newline{}Schnitzler: mit Bleistift auf dem ersten Blatt die Jahreszahl »94« vermerkt }\toendnotes[C]{\smallbreak}\pstart
           \noindent{}{\pb}\textcolor{gray}{\textbf{Frankfurter Zeitung\orgindex{Frankfurter Zeitung@Frankfurter Zeitung|pw}}}\hfill \textsc{Paris\oindex{Paris@\textbf{Paris}|pw}}, 9. Auguſt.\pend
           \pstart
           \textcolor{gray}{\textbf{(Gazette de
                     Francfort\orgindex{Frankfurter Zeitung@Frankfurter Zeitung|pw}).}}\pend
           \pstart
           \textcolor{gray}{\textbf{\begin{otherlanguage}{french}Fondateur\end{otherlanguage}{ }\textbf{M. L. Sonnemann\pwindex{Sonnemann, Leopold 1831-10-29 – 1909-10-30@\textsc{Sonnemann, Leopold} (1831-10-29 – 1909-10-30), \emph{Journalist, Herausgeber}|pw}}.}}\pend
           \pstart
           \textcolor{gray}{\textbf{\begin{otherlanguage}{french}Journal politique, financier,\end{otherlanguage}}}\pend
           \pstart
           \textcolor{gray}{\textbf{\begin{otherlanguage}{french}commercial et littéraire.\end{otherlanguage}}}\pend
           \pstart
           \textcolor{gray}{\textbf{\begin{otherlanguage}{french}\textbf{Paraissant trois fois par jour}\end{otherlanguage}}}.\pend
           \pstart
           \textcolor{gray}{\textbf{\begin{otherlanguage}{french}\textbf{Bureaux à Paris\oindex{Paris@\textbf{Paris}|pw}:}\end{otherlanguage}}}\pend
           \pstart
           \textcolor{gray}{\textbf{\begin{otherlanguage}{french}\textbf{24. Rue Feydeau}\oindex{rue Feydeau@\textbf{rue Feydeau}|pw}.\end{otherlanguage}}}\pend
           \pstart\center{}Mein lieber Freund,\pend\pstart
           Alles kracht plötzlich zuſammen. Geſtern erhielt ich
                  \label{K_L02610-1v}\edtext{\textsc{\begin{otherlanguage}{french}Ordre\end{otherlanguage}}}{\lemma{\textnormal{\emph{Ordre}}}\Cendnote{\textnormal{französisch: Order, Befehl}}}\label{K_L02610-1h} von
               meinem Journal\orgindex{Frankfurter Zeitung@Frankfurter Zeitung|pwv}, ſofort meinen
               Urlaub anzutreten und nach \textsc{Orange\oindex{Orange@\textbf{Orange}|pw}} zu fahren, um über die Aufführungen im antiken
                  Theater\orgindex{Theater Orange@Theater Orange|pw} zu berichten. Es iſt ekelhaft und gemein, aber da gibt es keine
                  Weigerung\textcolor{gray}{.} Demgemäß ändern ſich ſämmtliche Dispoſitionen. Mein
               Urlaub geht auf dieſe Weiſe bereits am 12. September
               zu Ende. {\pb}Und da ich die letzten acht Tage in \uline{Frankfurt\oindex{Frankfurt am Main@\textbf{Frankfurt am Main}|pw} verbringen muß}, ſo könnte ich
               nur zwiſchen dem 20. Auguſt und 3. September mit Euch\pwindex{Hofmannsthal, Hugo von 1874-02-01 – 1929-07-15@\textsc{Hofmannsthal, Hugo von} (1874-02-01 – 1929-07-15), \emph{Schriftsteller}|pwv}\pwindex{Beer-Hofmann, Richard 1866-07-11 – 1945-09-26@\textsc{Beer-Hofmann, Richard} (1866-07-11 – 1945-09-26), \emph{Schriftsteller}|pwv}{ }\label{K_L02610-2v}\edtext{zuſammen}{\lemma{\textnormal{\emph{zuſammen}}}\Cendnote{\textnormal{er schreibt »zuſammen zu«}}}\label{K_L02610-2h} ſein. Ich
               würde Alles thun, um dieſes Zuſammenſein zu ermöglichen, keine Reiſe ſcheuen \textsc{etc}. Ich habe ein ſolches Bedürfniß danach, mir Eure lieben
               Geſichter aufzufriſchen, mit Euch\pwindex{Hofmannsthal, Hugo von 1874-02-01 – 1929-07-15@\textsc{Hofmannsthal, Hugo von} (1874-02-01 – 1929-07-15), \emph{Schriftsteller}|pwv}\pwindex{Beer-Hofmann, Richard 1866-07-11 – 1945-09-26@\textsc{Beer-Hofmann, Richard} (1866-07-11 – 1945-09-26), \emph{Schriftsteller}|pwv} zu plaudern und mich bei Euch\pwindex{Hofmannsthal, Hugo von 1874-02-01 – 1929-07-15@\textsc{Hofmannsthal, Hugo von} (1874-02-01 – 1929-07-15), \emph{Schriftsteller}|pwv}\pwindex{Beer-Hofmann, Richard 1866-07-11 – 1945-09-26@\textsc{Beer-Hofmann, Richard} (1866-07-11 – 1945-09-26), \emph{Schriftsteller}|pwv} moraliſch und
               geiſtig zu kräftigen. Ich wäre tief traurig, wenn dieſes Zuſammenſein unmöglich wäre.
               Kann {\pb}ich nicht Alle\pwindex{Hofmannsthal, Hugo von 1874-02-01 – 1929-07-15@\textsc{Hofmannsthal, Hugo von} (1874-02-01 – 1929-07-15), \emph{Schriftsteller}|pwv}\pwindex{Beer-Hofmann, Richard 1866-07-11 – 1945-09-26@\textsc{Beer-Hofmann, Richard} (1866-07-11 – 1945-09-26), \emph{Schriftsteller}|pwv} ſehen, ſo möchte ich wenigſtens
               mit Einem zuſammenſein, am Liebſten natürlich mit Dir.\pend
           \pstart
           Kurzum: Könntet Ihr die Reiſe in Tirol\oindex{Tirol@\textbf{Tirol}|pw}\oindex{Suedtirol@\textbf{Südtirol}|pw} um
               acht Tage \label{K_L02610-3v}\edtext{früher beginnen}{\lemma{\textnormal{\emph{früher beginnen}}}\Cendnote{\textnormal{Am 23. 8. 1894 kam Goldmann\pwindex{Goldmann, Paul 31.01.1865 – 25.09.1935@\textsc{Goldmann, Paul} (31.01.1865 – 25.09.1935), \emph{Schriftsteller, Journalist}|pwk} direkt nach Ischl\oindex{Bad Ischl@\textbf{Bad Ischl}|pwk}.}}}\label{K_L02610-3h}, ſo käme ich direct aus Südfrankreich\oindex{Frankreich@\textbf{Frankreich}|pw} nach Tirol\oindex{Tirol@\textbf{Tirol}|pw}\oindex{Suedtirol@\textbf{Südtirol}|pw}. Am
               Liebſten wäre es mir freilich, wenn wir uns in Italien\oindex{Italien@\textbf{Italien}|pw} treffen könnten. \textsc{Pisa\oindex{Pisa@\textbf{Pisa}|pw}}{ }\textsc{Genua\oindex{Genua@\textbf{Genua}|pw}}, \textsc{Florenz\oindex{Florenz@\textbf{Florenz}|pw}}, \textsc{Venedig\oindex{Venedig@\textbf{Venedig}|pw}}. Wie herrlich wäre es z. B., wenn wir acht Tage in Venedig\oindex{Venedig@\textbf{Venedig}|pw}{ }\strikeout{b\textcolor{gray}{u}} bummeln könnten! Sollteſt Du das nicht zu {\pb}machen vermögen? Aber ich mache dir keine weitern Vorſchläge und überlaſſe Alles
               deiner Güte und Freundſchaft.\pend
           \pstart
           Schreibe mir ſofort nach dem Empfang dieſes Briefes an meine Pariſ\oindex{Paris@\textbf{Paris}|pw}er Adresse, oder telegraphire mir dorthin (\textsc{Goldmann}, \textsc{Paris, 24. Feydeau\oindex{rue Feydeau@\textbf{rue Feydeau}|pw}}). Ich habe \begin{otherlanguage}{french}Ordre\end{otherlanguage} gegeben, daß mir Briefe
               nachgeſchickt und Telegramme nachtelegraphirt werden. Gib mir auch an, wohin ich dir
               brieflich oder telegraphiſch antworten kann? Von Herzen\pend
           \pstart
           Dein{\\[\baselineskip]}\spacefill\mbox{Paul Goldmann.}\pend
           \leftskip=0em{}\pstart
           \noindent{}{\pb}\label{T_L02610-1v}\edtext{Tauſend Dank für den lieben \label{K_L02610-4v}\edtext{Brief aus \textsc{Salzburg\oindex{Salzburg@\textbf{Salzburg}|pw}}}{\lemma{\textnormal{\emph{Brief aus Salzburg}}}\Cendnote{\textnormal{Schnitzler\pwindex{Schnitzler, Arthur 15.05.1862 – 21.10.1931@\textsc{Schnitzler, Arthur} (15.05.1862 – 21.10.1931), \emph{Schriftsteller, Mediziner}|pwk} war von 1. 8. 1894 weg
                     vier Tage in Salzburg\oindex{Salzburg@\textbf{Salzburg}|pwk}, bevor er am 5. 8. 1894 nach
                        Ischl\oindex{Bad Ischl@\textbf{Bad Ischl}|pwk} weiterreiste.}}}\label{K_L02610-4h}}{\lemma{\textnormal{\emph{Tauſend … Salzburg}}}\Cendnote{\textnormal{auf der ersten Seite, unterhalb des
                     Textes}}}\label{T_L02610-1h}\pend
           
         
         \endnumbering\mylabel{h}\end{ledgroupsized}  \newcommand{\dateiname}{L02610}\newcommand{\titel}{Paul Goldmann an Arthur Schnitzler, 9. 8. [1894]}\newcommand{\editorInnen}{Martin Anton Müller und Laura Untner}%% latex-leseansicht-abspann.tex
%% Abspann für die Leseansicht.
%% Der Schalter \ifkorrekturansicht ist bereits durch den Vorspann gesetzt.

%% latex-abspann.tex
%% Gemeinsamer Abspann für Korrekturansicht und Leseansicht.
%% Setzt den Schalter \ifkorrekturansicht voraus (gesetzt in den
%% einbindenden Dateien latex-korrekturansicht-abspann.tex bzw.
%% latex-leseansicht-abspann.tex).
%% ---------------------------------------------------------------

\normalsize

% Das esempio-Environment wird nur in der Leseansicht benötigt
\ifkorrekturansicht\else
\newenvironment{esempio}[3]%
{
    \vspace{1.5ex}
    \rlap{\underline{#1}}
    \par
    \setlength{\parindent}{0cm}
    \nopagebreak
    \leftskip=#2cm
    \rightskip=#3cm
}
{
    \par
}
\fi

\doendnotes{C}
\bigskip
\vfill

\clearpage

\footnotesize

\ifkorrekturansicht
  \lohead{\textsc{register}}
\fi

% theindex-Environment neu definieren ohne reledmac
\makeatletter
\renewenvironment{theindex}{%
  \ifkorrekturansicht
    \section*{\indexname}%
  \else
    \subsubsection*{Index der erwähnten Entitäten}%
  \fi
  \setlength{\parindent}{0pt}%
  \setlength{\parskip}{0pt plus 0.3pt}%
  \let\item\@idxitem
}{%
  \ifkorrekturansicht\clearpage\fi
}
\makeatother

\IfFileExists{\jobname-pw.ind}{\input{\jobname-pw.ind}}{}

% Quellenangabe nur in der Leseansicht
\ifkorrekturansicht\else
% Fallback-Definitionen, falls die .tex-Datei \titel etc. nicht gesetzt hat
\providecommand{\titel}{}
\providecommand{\editorInnen}{}
\providecommand{\dateiname}{\jobname}

\vspace{3cm}

\vfill

\footnotesize
\textsc{Quelle}: \titel. Herausgegeben von {\editorInnen}. In: \emph{Arthur Schnitzler: Briefwechsel mit Autorinnen und Autoren}.
 Digitale Edition, https://schnitzler-briefe.acdh.oeaw.ac.at/{\dateiname}.html (Stand \today)
\fi

\end{document}


      