%% latex-leseansicht-vorspann.tex
%% Vorspann für die Leseansicht.
%% Lädt die gemeinsame Datei latex-vorspann.tex mit nicht gesetztem Schalter.

\newif\ifkorrekturansicht
\korrekturansichtfalse

\input{../tex-inputs/latex-vorspann}


\section[ Paul Goldmann an Arthur Schnitzler, 9. 8. [1894]]{L02610 Paul Goldmann an Arthur Schnitzler,  9. 8. [1894]}
\nopagebreak\mylabel{L02610v}
\rehead{ }\normalsize\beginnumbering\briefempfaengerindex{Schnitzler, Arthur@\textsc{Schnitzler, Arthur}!zzzGoldmann, Paul@\emph{von Paul Goldmann}!1894-08-091@{9. 8. [1894]}|(be}
\toendnotes[C]{\smallbreak\pagebreak[2]}
\correspDesc{Versand  durch Paul Goldmann am 9. 8. [1894] in Paris
\newline{}Erhalt  durch Arthur Schnitzler im Zeitraum [10. 8. 1894
                  – 14. 8. 1894?] in Bad Ischl}\toendnotes[C]{\smallbreak}
\Standort{DLA, A:Schnitzler, HS.NZ85.1.3164.}
\physDesc{Brief, 1 Blatt, 4 Seiten, 1734 Zeichen
\newline{}Handschrift: schwarze Tinte, deutsche Kurrent
\newline{}Schnitzler: mit Bleistift auf dem ersten Blatt die Jahreszahl »94« vermerkt }\toendnotes[C]{\smallbreak}
\pstart
           {\pb}\textcolor{gray}{\textbf{Frankfurter Zeitung\orgindex{Frankfurter Zeitung@Frankfurter Zeitung|pw}}}\hfill \textsc{Paris\oindex{Paris@\textbf{Paris}, \emph{Hauptstadt}|pw}}, 9. Auguſt.\pend
           
\pstart
           \textcolor{gray}{\textbf{(Gazette de
                     Francfort\orgindex{Frankfurter Zeitung@Frankfurter Zeitung|pw}).}}\pend
           
\pstart
           \textcolor{gray}{\textbf{\begin{otherlanguage}{french}Fondateur\end{otherlanguage}{ }\textbf{M. L. Sonnemann\pwindex{Sonnemann, Leopold 29.\,10.\,1831 Höchberg – 30.\,10.\,1909 Frankfurt am Main@\textsc{Sonnemann, Leopold} (29.\,10.\,1831 Höchberg – 30.\,10.\,1909 Frankfurt am Main), \emph{Journalist, Herausgeber}|pw}}.}}\pend
           
\pstart
           \textcolor{gray}{\textbf{\begin{otherlanguage}{french}Journal politique, financier,\end{otherlanguage}}}\pend
           
\pstart
           \textcolor{gray}{\textbf{\begin{otherlanguage}{french}commercial et littéraire.\end{otherlanguage}}}\pend
           
\pstart
           \textcolor{gray}{\textbf{\begin{otherlanguage}{french}\textbf{Paraissant trois fois par jour}\end{otherlanguage}}}.\pend
           
\pstart
           \textcolor{gray}{\textbf{\begin{otherlanguage}{french}\textbf{Bureaux à Paris\oindex{Paris@\textbf{Paris}, \emph{Hauptstadt}|pw}:}\end{otherlanguage}}}\pend
           
\pstart
           \textcolor{gray}{\textbf{\begin{otherlanguage}{french}\textbf{24. Rue Feydeau}\oindex{rue Feydeau@\textbf{rue Feydeau}, \emph{Straße}|pw}.\end{otherlanguage}}}\pend
           
\pstart\center{}Mein lieber Freund,\pend\vspace{0.5em}
\pstart
           Alles kracht plötzlich zuſammen. Geſtern erhielt ich
                  \label{K_L02610-1v}\edtext{\textsc{\begin{otherlanguage}{french}Ordre\end{otherlanguage}}}{\lemma{\textnormal{\emph{Ordre}}}\Cendnote{\textnormal{französisch: Order, Befehl}}}\label{K_L02610-1} von
               meinem Journal\orgindex{Frankfurter Zeitung@Frankfurter Zeitung|pwv},{ }ſofort meinen
               Urlaub anzutreten und nach \textsc{Orange\oindex{Orange@\textbf{Orange}, \emph{Region}|pw}} zu fahren, um über die Aufführungen im antiken
                  Theater\orgindex{Theater Orange@Theater Orange|pw} zu berichten. Es iſt ekelhaft und gemein, aber da gibt es keine
                  Weigerung\textcolor{gray}{.} Demgemäß ändern{ }ſich{ }ſämmtliche Dispoſitionen. Mein
               Urlaub geht auf dieſe Weiſe bereits am 12. September
               zu Ende. {\pb}Und da ich die letzten acht Tage in \uline{Frankfurt\oindex{Frankfurt am Main@\textbf{Frankfurt am Main}, \emph{Hauptstadt}|pw} verbringen muß},{ }ſo könnte ich
               nur zwiſchen dem 20. Auguſt und 3. September mit Euch\pwindex{Hofmannsthal, Hugo von 1.\,2.\,1874 Wien – 15.\,7.\,1929 Rodaun@\textsc{Hofmannsthal, Hugo von} (1.\,2.\,1874 Wien – 15.\,7.\,1929 Rodaun), \emph{Schriftsteller}|pwv}\pwindex{Beer-Hofmann, Richard 11.\,7.\,1866 Wien – 26.\,9.\,1945 New York City@\textsc{Beer-Hofmann, Richard} (11.\,7.\,1866 Wien – 26.\,9.\,1945 New York City), \emph{Schriftsteller}|pwv}{ }\label{K_L02610-2v}\edtext{zuſammen}{\lemma{\textnormal{\emph{zusammen}}}\Cendnote{\textnormal{Er schreibt »zuſammen zu«.}}}\label{K_L02610-2}{ }ſein. Ich
               würde Alles thun, um dieſes Zuſammenſein zu ermöglichen, keine Reiſe{ }ſcheuen \textsc{etc}. Ich habe ein{ }ſolches Bedürfniß danach, mir Eure lieben
               Geſichter aufzufriſchen, mit Euch\pwindex{Hofmannsthal, Hugo von 1.\,2.\,1874 Wien – 15.\,7.\,1929 Rodaun@\textsc{Hofmannsthal, Hugo von} (1.\,2.\,1874 Wien – 15.\,7.\,1929 Rodaun), \emph{Schriftsteller}|pwv}\pwindex{Beer-Hofmann, Richard 11.\,7.\,1866 Wien – 26.\,9.\,1945 New York City@\textsc{Beer-Hofmann, Richard} (11.\,7.\,1866 Wien – 26.\,9.\,1945 New York City), \emph{Schriftsteller}|pwv} zu plaudern und mich bei Euch\pwindex{Hofmannsthal, Hugo von 1.\,2.\,1874 Wien – 15.\,7.\,1929 Rodaun@\textsc{Hofmannsthal, Hugo von} (1.\,2.\,1874 Wien – 15.\,7.\,1929 Rodaun), \emph{Schriftsteller}|pwv}\pwindex{Beer-Hofmann, Richard 11.\,7.\,1866 Wien – 26.\,9.\,1945 New York City@\textsc{Beer-Hofmann, Richard} (11.\,7.\,1866 Wien – 26.\,9.\,1945 New York City), \emph{Schriftsteller}|pwv} moraliſch und
               geiſtig zu kräftigen. Ich wäre tief traurig, wenn dieſes Zuſammenſein unmöglich wäre.
               Kann {\pb}ich nicht Alle\pwindex{Hofmannsthal, Hugo von 1.\,2.\,1874 Wien – 15.\,7.\,1929 Rodaun@\textsc{Hofmannsthal, Hugo von} (1.\,2.\,1874 Wien – 15.\,7.\,1929 Rodaun), \emph{Schriftsteller}|pwv}\pwindex{Beer-Hofmann, Richard 11.\,7.\,1866 Wien – 26.\,9.\,1945 New York City@\textsc{Beer-Hofmann, Richard} (11.\,7.\,1866 Wien – 26.\,9.\,1945 New York City), \emph{Schriftsteller}|pwv}{ }ſehen,{ }ſo möchte ich wenigſtens
               mit Einem zuſammenſein, am Liebſten natürlich mit Dir.\pend
           
\pstart
           Kurzum: Könntet Ihr die Reiſe in Tirol\oindex{Tirol@\textbf{Tirol}, \emph{Land}|pw}\oindex{Südtirol@\textbf{Südtirol}, \emph{Verwaltungsgebiet}|pw} um
               acht Tage \label{K_L02610-3v}\edtext{früher beginnen}{\lemma{\textnormal{\emph{früher beginnen}}}\Cendnote{\textnormal{Am 23. 8. 1894 kam Goldmann\pwindex{Goldmann, Paul 31.\,1.\,1865 Breslau – 25.\,9.\,1935 Wien@\textsc{Goldmann, Paul} (31.\,1.\,1865 Breslau – 25.\,9.\,1935 Wien), \emph{Schriftsteller, Journalist}|pwk} direkt nach Ischl\oindex{Bad Ischl@\textbf{Bad Ischl}|pwk}.}}}\label{K_L02610-3},{ }ſo käme ich direct aus Südfrankreich\oindex{Frankreich@\textbf{Frankreich}|pw} nach Tirol\oindex{Tirol@\textbf{Tirol}, \emph{Land}|pw}\oindex{Südtirol@\textbf{Südtirol}, \emph{Verwaltungsgebiet}|pw}. Am
               Liebſten wäre es mir freilich, wenn wir uns in Italien\oindex{Italien@\textbf{Italien}|pw} treffen könnten. \textsc{Pisa\oindex{Pisa@\textbf{Pisa}, \emph{Hauptstadt}|pw}}{ }\textsc{Genua\oindex{Genua@\textbf{Genua}|pw}}, \textsc{Florenz\oindex{Florenz@\textbf{Florenz}|pw}}, \textsc{Venedig\oindex{Venedig@\textbf{Venedig}|pw}}. Wie herrlich wäre es z. B., wenn wir acht Tage in Venedig\oindex{Venedig@\textbf{Venedig}|pw}{ }\strikeout{b\textcolor{gray}{u}} bummeln könnten! Sollteſt Du das nicht zu {\pb}machen vermögen? Aber ich mache dir keine weitern Vorſchläge und überlaſſe Alles
               deiner Güte und Freundſchaft.\pend
           
\pstart
           Schreibe mir{ }ſofort nach dem Empfang dieſes Briefes an meine Pariſ\oindex{Paris@\textbf{Paris}, \emph{Hauptstadt}|pw}er Adresse, oder telegraphire mir dorthin (\textsc{Goldmann}, \textsc{Paris, 24. Feydeau\oindex{rue Feydeau@\textbf{rue Feydeau}, \emph{Straße}|pw}}). Ich habe \begin{otherlanguage}{french}Ordre\end{otherlanguage} gegeben, daß mir Briefe
               nachgeſchickt und Telegramme nachtelegraphirt werden. Gib mir auch an, wohin ich dir
               brieflich oder telegraphiſch antworten kann? Von Herzen\pend
           
\pstart
           Dein{\\[\baselineskip]}\spacefill\mbox{Paul Goldmann.}\pend
           \leftskip=0em{}
\pstart
           \noindent{}{\pb}\label{T_L02610-1v}\edtext{Tauſend Dank für den lieben \label{K_L02610-4v}\edtext{Brief aus \textsc{Salzburg\oindex{Salzburg@\textbf{Salzburg}, \emph{Verwaltungsgebiet}|pw}}}{\lemma{\textnormal{\emph{Brief aus Salzburg}}}\Cendnote{\textnormal{Schnitzler war ab dem 1. 8. 1894
                     vier Tage in Salzburg\oindex{Salzburg@\textbf{Salzburg}, \emph{Verwaltungsgebiet}|pwk}, bevor er am 5. 8. 1894 nach
                        Ischl\oindex{Bad Ischl@\textbf{Bad Ischl}|pwk} weiterreiste.}}}\label{K_L02610-4}}{\lemma{\textnormal{\emph{Tausend … Salzburg}}}\Cendnote{\textnormal{auf der ersten Seite, unterhalb des
                     Textes}}}\label{T_L02610-1}\pend
           \selectlanguage{ngerman}\endnumbering\briefempfaengerindex{Schnitzler, Arthur@\textsc{Schnitzler, Arthur}!zzzGoldmann, Paul@\emph{von Paul Goldmann}!1894-08-091@{9. 8. [1894]}|)be}\mylabel{L02610h}  \newcommand{\dateiname}{L02610}\newcommand{\titel}{Paul Goldmann an Arthur Schnitzler, 9. 8. [1894]}\newcommand{\editorInnen}{Martin Anton Müller und Laura Untner}%% latex-leseansicht-abspann.tex
%% Abspann für die Leseansicht.
%% Der Schalter \ifkorrekturansicht ist bereits durch den Vorspann gesetzt.

%% latex-abspann.tex
%% Gemeinsamer Abspann für Korrekturansicht und Leseansicht.
%% Setzt den Schalter \ifkorrekturansicht voraus (gesetzt in den
%% einbindenden Dateien latex-korrekturansicht-abspann.tex bzw.
%% latex-leseansicht-abspann.tex).
%% ---------------------------------------------------------------

\normalsize

% Das esempio-Environment wird nur in der Leseansicht benötigt
\ifkorrekturansicht\else
\newenvironment{esempio}[3]%
{
    \vspace{1.5ex}
    \rlap{\underline{#1}}
    \par
    \setlength{\parindent}{0cm}
    \nopagebreak
    \leftskip=#2cm
    \rightskip=#3cm
}
{
    \par
}
\fi

\doendnotes{C}
\bigskip
\vfill

\clearpage

\footnotesize

\ifkorrekturansicht
  \lohead{\textsc{register}}
\fi

% theindex-Environment neu definieren ohne reledmac
\makeatletter
\renewenvironment{theindex}{%
  \ifkorrekturansicht
    \section*{\indexname}%
  \else
    \subsubsection*{Index der erwähnten Entitäten}%
  \fi
  \setlength{\parindent}{0pt}%
  \setlength{\parskip}{0pt plus 0.3pt}%
  \let\item\@idxitem
}{%
  \ifkorrekturansicht\clearpage\fi
}
\makeatother

\IfFileExists{\jobname-pw.ind}{\input{\jobname-pw.ind}}{}

% Quellenangabe nur in der Leseansicht
\ifkorrekturansicht\else
% Fallback-Definitionen, falls die .tex-Datei \titel etc. nicht gesetzt hat
\providecommand{\titel}{}
\providecommand{\editorInnen}{}
\providecommand{\dateiname}{\jobname}

\vspace{3cm}

\vfill

\footnotesize
\textsc{Quelle}: \titel. Herausgegeben von {\editorInnen}. In: \emph{Arthur Schnitzler: Briefwechsel mit Autorinnen und Autoren}.
 Digitale Edition, https://schnitzler-briefe.acdh.oeaw.ac.at/{\dateiname}.html (Stand \today)
\fi

\end{document}


