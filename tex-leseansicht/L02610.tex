\input{../tex-inputs/latex-pdf-vorspann}
\begin{center}
            \textcolor{red}{ENTWURF. ENTZIFFERUNG NOCH NICHT KORREKTURGELESEN}
                      \end{center}
            
               \section[Paul Goldmann an Arthur Schnitzler, 9. 8. {[}1894{]}]{ Paul Goldmann an Arthur Schnitzler, 9. 8. {[}1894{]}}\nopagebreak\mylabel{v}\rehead{ }\begin{ledgroupsized}[t]{13cm}\normalsize\beginnumbering\briefempfaengerindex{Schnitzler, Arthur@\textsc{Schnitzler, Arthur}!zzzGoldmann, Paul@\emph{von Paul Goldmann}!1894-08-091@{9. 8. {[}1894{]}}|(be} \toendnotes[C]{\smallbreak\pagebreak[2]} \Standort{DLA, A:Schnitzler, HS.NZ85.1.3164.}
\physDesc{Brief, 1 Blatt, 4 Seiten
\newline{}Handschrift: schwarze Tinte, deutsche Kurrent
\newline{}Schnitzler: mit Bleistift auf dem ersten Blatt die Jahreszahl
                                       »94« vermerkt }\toendnotes[C]{\smallbreak}\pstart
           \noindent{}{\pb}\textcolor{gray}{\textbf{Frankfurter Zeitung\orgindex{Frankfurter Zeitung@Frankfurter Zeitung|pw}.}}\hfill \textsc{Paris\oindex{Paris@\textbf{Paris}|pw}}, 9. Auguſt.\pend
           \pstart
           \textcolor{gray}{\textbf{(Gazette de
                  Francfort\orgindex{Frankfurter Zeitung@Frankfurter Zeitung|pw}.)}}\pend
           \pstart
           \textcolor{gray}{\textbf{\begin{otherlanguage}{french}Fondateur\end{otherlanguage}{ }\textbf{M. L.
                  Sonnemann\pwindex{Sonnemann, Leopold 1831-10-29 – 1909-10-30@\textsc{Sonnemann, Leopold} (1831-10-29 – 1909-10-30), \emph{Journalist, Herausgeber}|pw}}.}}\pend
           \pstart
           \textcolor{gray}{\textbf{\begin{otherlanguage}{french}Journal politique,
                        financier,\end{otherlanguage}}}\pend
           \pstart
           \textcolor{gray}{\textbf{\begin{otherlanguage}{french}commercial et
                     littéraire.\end{otherlanguage}}}\pend
           \pstart
           \textcolor{gray}{\textbf{\begin{otherlanguage}{french}\textbf{Paraissant trois fois
                           par jour}\end{otherlanguage}}}.\pend
           \pstart
           \textcolor{gray}{\textbf{–}}\pend
           \pstart
           \textcolor{gray}{\textbf{\begin{otherlanguage}{french}\textbf{Bureaux à Paris\oindex{Paris@\textbf{Paris}|pw}:}\end{otherlanguage}}}\pend
           \pstart
           \textcolor{gray}{\textbf{\begin{otherlanguage}{french}\textbf{24. Rue Feydeau}\oindex{rue Feydeau@\textbf{rue Feydeau}|pw}.\end{otherlanguage}}}\pend
           \pstart\center{}Mein lieber Freund,\pend\pstart
           Alles kracht plötzlich zuſammen. Geſtern erhielt ich \textsc{\begin{otherlanguage}{french}Ordre\end{otherlanguage}} von meinem Journal\orgindex{Frankfurter Zeitung@Frankfurter Zeitung|pwv}, ſofort meinen Urlaub anzutreten und
               nach \textsc{Orange\oindex{Orange@\textbf{Orange}|pw}} zu fahren, um
               über die Aufführungen im antiken Theater\orgindex{Theater Orange@Theater Orange|pw} zu
               berichten. Es iſt ekelhaft und gemein, aber da gibt es keine Weigerung. Demgemäß
               ändern ſich ſämmtliche Dispoſitionen. Mein Urlaub geht auf dieſe Weiſe bereits am
                  12. September zu Ende. {\pb}Und da ich
               die letzten acht Tage in \uline{Frankfurt\oindex{Frankfurt am Main@\textbf{Frankfurt am Main}|pw} verbringen muß}, ſo könnte ich nur zwiſchen dem
                  20. Auguſt und 3. September mit Euch\pwindex{Hofmannsthal, Hugo von 01.02.1874 – 15.07.1929@\textsc{Hofmannsthal, Hugo von} (01.02.1874 – 15.07.1929), \emph{Schriftsteller}|pwv}\pwindex{Beer-Hofmann, Richard 11.07.1866 – 26.09.1945@\textsc{Beer-Hofmann, Richard} (11.07.1866 – 26.09.1945), \emph{Schriftsteller}|pwv}\label{K_L02610-2v}\edtext{zuſammen}{\lemma{\textnormal{\emph{zuſammen}}}\Cendnote{\textnormal{er schreibt »zuſammen zu«}}}\label{K_L02610-2h} ſein. Ich
               würde Alles thun, um dieſes Zuſammenſein zu ermöglichen, keine Reiſe ſcheuen \textsc{etc}. Ich habe ein ſolches Bedürfniß danach, mir Eure lieben
               Geſichter aufzufriſchen, mit Euch\pwindex{Hofmannsthal, Hugo von 01.02.1874 – 15.07.1929@\textsc{Hofmannsthal, Hugo von} (01.02.1874 – 15.07.1929), \emph{Schriftsteller}|pwv}\pwindex{Beer-Hofmann, Richard 11.07.1866 – 26.09.1945@\textsc{Beer-Hofmann, Richard} (11.07.1866 – 26.09.1945), \emph{Schriftsteller}|pwv} zu plaudern und mich bei Euch\pwindex{Hofmannsthal, Hugo von 01.02.1874 – 15.07.1929@\textsc{Hofmannsthal, Hugo von} (01.02.1874 – 15.07.1929), \emph{Schriftsteller}|pwv}\pwindex{Beer-Hofmann, Richard 11.07.1866 – 26.09.1945@\textsc{Beer-Hofmann, Richard} (11.07.1866 – 26.09.1945), \emph{Schriftsteller}|pwv} moraliſch und
               geiſtig zu kräftigen. Ich wäre tief traurig, wenn dieſes Zuſammenſein unmöglich wäre.
               Kann {\pb} ich nicht Alle\pwindex{Hofmannsthal, Hugo von 01.02.1874 – 15.07.1929@\textsc{Hofmannsthal, Hugo von} (01.02.1874 – 15.07.1929), \emph{Schriftsteller}|pwv}\pwindex{Beer-Hofmann, Richard 11.07.1866 – 26.09.1945@\textsc{Beer-Hofmann, Richard} (11.07.1866 – 26.09.1945), \emph{Schriftsteller}|pwv} ſehen, ſo möchte ich wenigſtens mit
               Einem Zuſammenſein, am Liebſten natürlich mit Dir.\pend
           \pstart
           Kurzum: Könntet Ihr die Reiſe in Tirol\oindex{Tirol@\textbf{Tirol}|pw}\oindex{Suedtirol@\textbf{Südtirol}|pw} um acht
               Tage \label{K_L02610-1v}\edtext{früher beginnen}{\lemma{\textnormal{\emph{früher beginnen}}}\Cendnote{\textnormal{Am 23. 8. 1894 kam Goldmann in Bad Ischl\oindex{Bad Ischl@\textbf{Bad Ischl}|pwk} an, er reiste also nicht nach Tirol\oindex{Tirol@\textbf{Tirol}|pwk}.}}}\label{K_L02610-1h}, ſo käme ich direct aus Südfrankreich\oindex{Frankreich@\textbf{Frankreich}|pw} nach Tirol\oindex{Tirol@\textbf{Tirol}|pw}\oindex{Suedtirol@\textbf{Südtirol}|pw}.
               Am Liebſten wäre es mir freilich, wenn wir uns in Italien\oindex{Italien@\textbf{Italien}|pw} treffen könnten. \textsc{Pisa\oindex{Pisa@\textbf{Pisa}|pw}}{ }\textsc{Genua\oindex{Genua@\textbf{Genua}|pw}}, \textsc{Florenz\oindex{Florenz@\textbf{Florenz}|pw}}, \textsc{Venedig\oindex{Venedig@\textbf{Venedig}|pw}}. Wie herrlich wäre
               es z. B., wenn wir acht Tage in Venedig\oindex{Venedig@\textbf{Venedig}|pw}{ }\strikeout{be} bummeln könnten! Sollteſt Du das nicht zu {\pb}machen vermögen? Aber ich mache dir keine weitern
               Vorſchläge und überlaſſe Alles deiner Güte und Freundſchaft.\pend
           \pstart
           Schreibe mir ſofort nach dem Empfang dieſes Briefes an meine Pariſ\oindex{Paris@\textbf{Paris}|pw}er Adresse, oder telegraphiere mir dorthin (\textsc{Goldmann}, \textsc{Paris, 24. Feydeau\oindex{rue Feydeau@\textbf{rue Feydeau}|pw}}). Ich habe \begin{otherlanguage}{french}Ordre\end{otherlanguage} gegeben, daß mir Briefe nachgeſchickt und Telegramme
               nachtelegraphirt werden. Gib mir auch an, wohin ich dir brieflich oder telegraphiſch
               antworten kann?\pend
           \pstart
           Von Herzen{\\[\baselineskip]} Dein{\\[\baselineskip]}\spacefill\mbox{Paul Goldmann}\pend
           \leftskip=0em{}\pstart
           \noindent{}{\pb}\label{T_L02610-1v}\edtext{Tauſend Dank für den lieben Brief aus \textsc{Salzburg\oindex{Salzburg@\textbf{Salzburg}|pw}}}{\lemma{\textnormal{\emph{Tauſend … Salzburg}}}\Cendnote{\textnormal{auf der ersten Seite, unterhalb des Textes}}}\label{T_L02610-1h}\pend
           \endnumbering\briefempfaengerindex{Schnitzler, Arthur@\textsc{Schnitzler, Arthur}!zzzGoldmann, Paul@\emph{von Paul Goldmann}!1894-08-091@{9. 8. {[}1894{]}}|)be}\mylabel{h}\end{ledgroupsized}  \newcommand{\dateiname}{L02610}\newcommand{\titel}{Paul Goldmann an Arthur Schnitzler, 9. 8. [1894]}\newcommand{\editorInnen}{Martin Anton Müller und Laura Untner}\input{../tex-inputs/latex-pdf-abspann}
      