%% latex-leseansicht-vorspann.tex
%% Vorspann für die Leseansicht.
%% Lädt die gemeinsame Datei latex-vorspann.tex mit nicht gesetztem Schalter.

\newif\ifkorrekturansicht
\korrekturansichtfalse

\input{../tex-inputs/latex-vorspann}


\section[Arthur Schnitzler an Georg Brandes, 29. 4. 1911]{L02018 Arthur Schnitzler an Georg Brandes, 29. 4. 1911}
\nopagebreak\mylabel{L02018v}
\rehead{ }\normalsize\beginnumbering\briefempfaengerindex{Brandes, Georg@\textsc{Brandes, Georg}!zzzSchnitzler, Arthur@\emph{von Arthur Schnitzler}!1911-04-291@{29. 4. 1911}|(be}
\toendnotes[C]{\smallbreak\pagebreak[2]}
\correspDesc{Versand  durch Arthur Schnitzler am 29. 4. 1911 in Garmisch-Partenkirchen
\newline{}Erhalt  durch Georg Brandes im Zeitraum [30. 4. 1911
                  – 4. 5. 1911?] in Paris}\toendnotes[C]{\smallbreak}
\Standort{Kopenhagen, Det Kongelige Bibliotek, Georg Brandes Arkiv, box 125.}
\physDesc{Bildpostkarte, 259 Zeichen
\newline{}Handschrift: Bleistift, deutsche Kurrent
\newline{}Versand: 1) Stempel: »\nobreak{}\oindex{Garmisch-Partenkirchen@\textbf{Garmisch-Partenkirchen}, \emph{Hauptstadt}|pwk}P{[}arten{]}kirchen, {[}29{]}. Apr. {[}11{]}, 6–7Nm\nobreak{}«.   2) mit blauem Buntstift »138/31« über dem Adressfeld
                                 notiert
\newline{}Ordnung: 1) die linke Ecke abgerissen, eine Briefmarke entfernt  2) mit Bleistift von unbekannter Hand nummeriert:
                                    »31«}
\buchAbdrucke{\weitereDrucke{Georg Brandes, Arthur Schnitzler: \emph{Ein Briefwechsel}. Herausgegeben von Kurt Bergel. Bern: \emph{Francke} 1956, S. 101.} }\pstart{}{\pb}Herrn Prof. \textsc{Georg
                     Brandes}\pend{}\pstart{}\textsc{Paris}\oindex{Paris@\textbf{Paris}, \emph{Hauptstadt}|pw}\pend{}\pstart{}\textsc{Hotel Lutetia}\oindex{Hôtel Lutetia@\textbf{Hôtel Lutetia}, \emph{Hotel}|pw}\pend{}\pstart{}\textsc{Boulevrd Pasqual}\oindex{Boulevard Raspail@\textbf{Boulevard Raspail}, \emph{Straße}|pw}\pend{}{\bigskip}
\pstart
           \noindent{}\centering{}{\pb}\textcolor{gray}{\textbf{Garmisch\oindex{Garmisch-Partenkirchen@\textbf{Garmisch-Partenkirchen}, \emph{Hauptstadt}|pw} geg. Wetterstein\oindex{Wettersteingebirge@\textbf{Wettersteingebirge}, \emph{Gebirge}|pw}}}\pend
           \vspace{1em}
\pstart
           {\pb}\textsc{Partenkirchen}\oindex{Garmisch-Partenkirchen@\textbf{Garmisch-Partenkirchen}, \emph{Hauptstadt}|pw}, 29. 4. 11\pend
           \vspace{0.5em}
\pstart
           Ihre Karte, verehrter Herr Br\damage{andes,} und die Druckſchriften{ }ſind uns nach Mentone\oindex{Menton@\textbf{Menton}|pw} nachgewandert, u. am Ende meiner Reiſe, dank ich und grüß ich
               herzlichſt in alter Treue.\pend
           
\pstart
           Ihr{\\[\baselineskip]}\spacefill\mbox{Arth Schni}\pend
           \leftskip=0em{}\selectlanguage{ngerman}\endnumbering\briefempfaengerindex{Brandes, Georg@\textsc{Brandes, Georg}!zzzSchnitzler, Arthur@\emph{von Arthur Schnitzler}!1911-04-291@{29. 4. 1911}|)be}\mylabel{L02018h}  \newcommand{\dateiname}{L02018}\newcommand{\titel}{Arthur Schnitzler an Georg Brandes, 29. 4. 1911}\newcommand{\editorInnen}{Martin Anton Müller und Gerd-Hermann Susen}%% latex-leseansicht-abspann.tex
%% Abspann für die Leseansicht.
%% Der Schalter \ifkorrekturansicht ist bereits durch den Vorspann gesetzt.

%% latex-abspann.tex
%% Gemeinsamer Abspann für Korrekturansicht und Leseansicht.
%% Setzt den Schalter \ifkorrekturansicht voraus (gesetzt in den
%% einbindenden Dateien latex-korrekturansicht-abspann.tex bzw.
%% latex-leseansicht-abspann.tex).
%% ---------------------------------------------------------------

\normalsize

% Das esempio-Environment wird nur in der Leseansicht benötigt
\ifkorrekturansicht\else
\newenvironment{esempio}[3]%
{
    \vspace{1.5ex}
    \rlap{\underline{#1}}
    \par
    \setlength{\parindent}{0cm}
    \nopagebreak
    \leftskip=#2cm
    \rightskip=#3cm
}
{
    \par
}
\fi

\doendnotes{C}
\bigskip
\vfill

\clearpage

\footnotesize

\ifkorrekturansicht
  \lohead{\textsc{register}}
\fi

% theindex-Environment neu definieren ohne reledmac
\makeatletter
\renewenvironment{theindex}{%
  \ifkorrekturansicht
    \section*{\indexname}%
  \else
    \subsubsection*{Index der erwähnten Entitäten}%
  \fi
  \setlength{\parindent}{0pt}%
  \setlength{\parskip}{0pt plus 0.3pt}%
  \let\item\@idxitem
}{%
  \ifkorrekturansicht\clearpage\fi
}
\makeatother

\IfFileExists{\jobname-pw.ind}{\input{\jobname-pw.ind}}{}

% Quellenangabe nur in der Leseansicht
\ifkorrekturansicht\else
% Fallback-Definitionen, falls die .tex-Datei \titel etc. nicht gesetzt hat
\providecommand{\titel}{}
\providecommand{\editorInnen}{}
\providecommand{\dateiname}{\jobname}

\vspace{3cm}

\vfill

\footnotesize
\textsc{Quelle}: \titel. Herausgegeben von {\editorInnen}. In: \emph{Arthur Schnitzler: Briefwechsel mit Autorinnen und Autoren}.
 Digitale Edition, https://schnitzler-briefe.acdh.oeaw.ac.at/{\dateiname}.html (Stand \today)
\fi

\end{document}


