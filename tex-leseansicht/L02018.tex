%% latex-korrekturansicht-vorspann.tex
%% Vorspann für die Korrekturansicht.
%% Lädt die gemeinsame Datei latex-vorspann.tex mit gesetztem Schalter.

\newif\ifkorrekturansicht
\korrekturansichttrue

\input{../tex-inputs/latex-vorspann}


\section[Arthur Schnitzler an Georg Brandes, 29. 4. 1911]{L02018 Arthur Schnitzler an Georg Brandes, 29. 4. 1911}
\nopagebreak\mylabel{L02018v}
\rehead{ }\normalsize\beginnumbering\briefempfaengerindex{Brandes, Georg@\textsc{Brandes, Georg}!zzzSchnitzler, Arthur@\emph{von Arthur Schnitzler}!1911-04-291@{29. 4. 1911}|(be}
\toendnotes[C]{\smallbreak\pagebreak[2]}\Standort{Kopenhagen, Det Kongelige Bibliotek, Georg Brandes Arkiv, box 125.}
\physDesc{Bildpostkarte, 259 Zeichen
\newline{}Handschrift: Bleistift, deutsche Kurrent
\newline{}Versand: 1) Stempel: »\nobreak{}\oindex{Garmisch-Partenkirchen@\textbf{Garmisch-Partenkirchen}, \emph{P.PPLA3}|pwk}P{[}arten{]}kirchen, {[}29{]}. Apr. {[}11{]}, 6–7Nm\nobreak{}«.   2) mit blauem Buntstift »138/31« über dem Adressfeld
                                 notiert
\newline{}Ordnung: 1) die linke Ecke abgerissen, eine Briefmarke entfernt  2) mit Bleistift von unbekannter Hand nummeriert:
                                    »31«}
\buchAbdrucke{\weitereDrucke{Georg Brandes, Arthur Schnitzler: \emph{Ein Briefwechsel}. Bern: \emph{Francke} 1956, S. 101.} }\pstart{}{\pb}Herrn Prof. \textsc{Georg
                     Brandes}\pend{}\pstart{}\textsc{Paris}\oindex{Paris@\textbf{Paris}, \emph{P.PPLC}|pw}\pend{}\pstart{}\textsc{Hotel Lutetia}\oindex{Hôtel Lutetia@\textbf{Hôtel Lutetia}, \emph{Hotel (K.HTL)}|pw}\pend{}\pstart{}\textsc{Boulevrd Pasqual}\oindex{Boulevard Raspail@\textbf{Boulevard Raspail}, \emph{Straße (K.STR)}|pw}\pend{}{\bigskip}
\pstart
           \noindent{}\centering{}{\pb}\textcolor{gray}{\textbf{Garmisch\oindex{Garmisch-Partenkirchen@\textbf{Garmisch-Partenkirchen}, \emph{P.PPLA3}|pw} geg. Wetterstein\oindex{Wettersteingebirge@\textbf{Wettersteingebirge}, \emph{T.MTS}|pw}}}\pend
           \vspace{1em}
\pstart
           {\pb}\textsc{Partenkirchen}\oindex{Garmisch-Partenkirchen@\textbf{Garmisch-Partenkirchen}, \emph{P.PPLA3}|pw}, 29. 4. 11\pend
           \vspace{0.5em}
\pstart
           Ihre Karte, verehrter Herr Br\damage{andes,} und die Druckſchriften ſind uns nach Mentone\oindex{Menton@\textbf{Menton}, \emph{P.PPL}|pw} nachgewandert, u. am Ende meiner Reiſe, dank ich und grüß ich
               herzlichſt in alter Treue.\pend
           
\pstart
           Ihr{\\[\baselineskip]}\spacefill\mbox{Arth Schni}\pend
           \leftskip=0em{}\selectlanguage{ngerman}\endnumbering\briefempfaengerindex{Brandes, Georg@\textsc{Brandes, Georg}!zzzSchnitzler, Arthur@\emph{von Arthur Schnitzler}!1911-04-291@{29. 4. 1911}|)be}\mylabel{L02018h}  \normalsize

\doendnotes{C}
\bigskip
\vfill

\clearpage

\footnotesize

\lohead{\textsc{register}}

% Definiere theindex-Environment komplett neu ohne reledmac
\makeatletter
\renewenvironment{theindex}{%
  \section*{\indexname}%
  \setlength{\parindent}{0pt}%
  \setlength{\parskip}{0pt plus 0.3pt}%
  \let\item\@idxitem
}{%
  \clearpage
}
\makeatother

\IfFileExists{\jobname-pw.ind}{\input{\jobname-pw.ind}}{}

\end{document}

      