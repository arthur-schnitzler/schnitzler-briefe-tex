%% latex-leseansicht-vorspann.tex
%% Vorspann für die Leseansicht.
%% Lädt die gemeinsame Datei latex-vorspann.tex mit nicht gesetztem Schalter.

\newif\ifkorrekturansicht
\korrekturansichtfalse

\input{../tex-inputs/latex-vorspann}


         \renewcommand{\erwaehnteOrte}{Orte: Café Central, Wien}
         \renewcommand{\erwaehnteWerke}{}
               \section[Friedrich M. Fels an Arthur Schnitzler, {[}20. 1. 1893{]}]{ Friedrich M. Fels an Arthur Schnitzler, {[}20. 1. 1893{]}}\nopagebreak\mylabel{v}\rehead{ }\begin{ledgroupsized}[t]{13cm}\normalsize\beginnumbering \toendnotes[C]{\smallbreak\pagebreak[2]} \Standort{DLA, A:Schnitzler, HS.NZ85.1.2956.}
\physDesc{Brief, 1 Blatt, 1 Seite, 850 Zeichen
\newline{}Handschrift: schwarze Tinte, lateinische Kurrent
\newline{}Schnitzler: mit Bleistift datiert: »20/1 93« und nummeriert: »2« }\toendnotes[C]{\smallbreak}\pstart
           \noindent{}{\pb}Lieber Dr Schnitzler! Heute früh beschloß, die Apathie fahren zu
               laßen und selbst energisch mich zum Fleischfreßer auszubilden. Wolan! Progra{\geminationm}: Bureau, Eßen, Café. Allerdings die Kälte hat mich
               scheußlich niedergesti{\geminationm}t; das ist ja abscheulich. Im
               Bureau habe ich mir vom Diener aus dem Ihnen beka{\geminationn}ten
               Lokal genau unsere Speisekarte von neulich wi{[}e{]}derholen laßen und
               habe \uline{das Ganze aufgefreßen}, was genügt. Nun werde
               wahrscheinlich Central\oindex{Cafe Central@\textbf{Café Central}|pw} gehen und mit Rücksicht
               auf Zeitung, Beka{\geminationn}ten u. v. a. Abort.\pend
           \pstart
           Ob Sie mit meinem heutigen Tag zufrieden sind, weiß \label{T_L00160-1v}\edtext{ich}{\lemma{\textnormal{\emph{ich}}}\Cendnote{\textnormal{Er schreibt:
                     »ich ich«.}}}\label{T_L00160-1h} nicht, obwol es eigentlich \introOben{}gut\introOben{} angebracht ist, aber, ich glaube, mit der Instruktion, die Sie mir
               gegeben, sti{\geminationm}t es wenig.\pend
           \pstart
           Jedenfalls, damit ich nicht ganz in dieser Selbstverständlichkeit bleibe, ersuche ich
               Sie, mich morgen in meinen Bureaustunden zu besuchen, zu strafen, zu kasteien,\pend
           \pstart \spacefill\mbox{Fels}\pend{}\pstart
           \noindent{}Herzl. Gruß!\pend
           
         
         \endnumbering\mylabel{h}\end{ledgroupsized}  \newcommand{\dateiname}{L00160}\newcommand{\titel}{Friedrich M. Fels an Arthur Schnitzler, [20. 1. 1893]}\newcommand{\editorInnen}{Martin Anton Müller und Gerd-Hermann Susen}%% latex-leseansicht-abspann.tex
%% Abspann für die Leseansicht.
%% Der Schalter \ifkorrekturansicht ist bereits durch den Vorspann gesetzt.

%% latex-abspann.tex
%% Gemeinsamer Abspann für Korrekturansicht und Leseansicht.
%% Setzt den Schalter \ifkorrekturansicht voraus (gesetzt in den
%% einbindenden Dateien latex-korrekturansicht-abspann.tex bzw.
%% latex-leseansicht-abspann.tex).
%% ---------------------------------------------------------------

\normalsize

% Das esempio-Environment wird nur in der Leseansicht benötigt
\ifkorrekturansicht\else
\newenvironment{esempio}[3]%
{
    \vspace{1.5ex}
    \rlap{\underline{#1}}
    \par
    \setlength{\parindent}{0cm}
    \nopagebreak
    \leftskip=#2cm
    \rightskip=#3cm
}
{
    \par
}
\fi

\doendnotes{C}
\bigskip
\vfill

\clearpage

\footnotesize

\ifkorrekturansicht
  \lohead{\textsc{register}}
\fi

% theindex-Environment neu definieren ohne reledmac
\makeatletter
\renewenvironment{theindex}{%
  \ifkorrekturansicht
    \section*{\indexname}%
  \else
    \subsubsection*{Index der erwähnten Entitäten}%
  \fi
  \setlength{\parindent}{0pt}%
  \setlength{\parskip}{0pt plus 0.3pt}%
  \let\item\@idxitem
}{%
  \ifkorrekturansicht\clearpage\fi
}
\makeatother

\IfFileExists{\jobname-pw.ind}{\input{\jobname-pw.ind}}{}

% Quellenangabe nur in der Leseansicht
\ifkorrekturansicht\else
% Fallback-Definitionen, falls die .tex-Datei \titel etc. nicht gesetzt hat
\providecommand{\titel}{}
\providecommand{\editorInnen}{}
\providecommand{\dateiname}{\jobname}

\vspace{3cm}

\vfill

\footnotesize
\textsc{Quelle}: \titel. Herausgegeben von {\editorInnen}. In: \emph{Arthur Schnitzler: Briefwechsel mit Autorinnen und Autoren}.
 Digitale Edition, https://schnitzler-briefe.acdh.oeaw.ac.at/{\dateiname}.html (Stand \today)
\fi

\end{document}


      