%% latex-leseansicht-vorspann.tex
%% Vorspann für die Leseansicht.
%% Lädt die gemeinsame Datei latex-vorspann.tex mit nicht gesetztem Schalter.

\newif\ifkorrekturansicht
\korrekturansichtfalse

\input{../tex-inputs/latex-vorspann}


         
         \newcommand{\erwaehntePersonen}{Personen: Frieda Pollak}
         \newcommand{\erwaehnteOrte}{Orte: Deutschland, Rodaun, Wien}
         \newcommand{\erwaehnteWerke}{Werke: Die Schwestern oder Casanova in Spa. Lustspiel in Versen}
               \section[Hugo Hofmannsthal an Arthur Schnitzler, 31. 3. 1920]{ Hugo Hofmannsthal an Arthur Schnitzler, 31. 3. 1920}\nopagebreak\mylabel{v}\rehead{ }\begin{ledgroupsized}[t]{13cm}\normalsize\beginnumbering \toendnotes[C]{\smallbreak\pagebreak[2]} \Standort{CUL, Schnitzler, B 43.}
\physDesc{Brief, 1 Blatt, 4 Seiten
\newline{}Handschrift: schwarze Tinte, deutsche Kurrent\newline{}Ordnung: 1) mit Bleistift von Frieda
                                    Pollak\pwindex{Pollak, Frieda 08.12.1881 – 13.07.1937@\textsc{Pollak, Frieda} (08.12.1881 – 13.07.1937), \emph{Sekretärin}|pw} (?) mit dem Buchstaben »A«
                                 (Abgeschrieben/Abschrift) gekennzeichnet  2) mit Bleistift von unbekannter Hand nummeriert: »\textcolor{gray}{2}65« 3) mit Bleistift von unbekannter Hand nummeriert:
                                    »365«}\buchAbdrucke{\weitereDrucke{Hugo von Hofmannsthal, Arthur Schnitzler: \emph{Briefwechsel}. Hg. Therese Nickl und Heinrich Schnitzler. Frankfurt am Main: \emph{S. Fischer} 1964, S. 292.} }\toendnotes[C]{\smallbreak}\pstart
           \raggedleft{}{\pb}Wien\oindex{Wien@\textbf{Wien}|pw}{\\}31 III 20\pend
           \pstart{}mein lieber Arthur\pend\pstart
           ich fühle nach den Berichten u. allem was man ſo hört daſs der \label{K_L02339_1v}\edtext{Luſtſpielabend\pwindex{Schnitzler, Arthur 15.05.1862 – 21.10.1931@\textsc{Schnitzler, Arthur} (15.05.1862 – 21.10.1931), \emph{Schriftsteller, Mediziner}!Schwestern oder Casanova in Spa. Lustspiel in Versen01. 10. 1919@\strich\emph{Die Schwestern oder Casanova in Spa. Lustspiel in Versen} {[}01. 10. 1919{]}|pwv}}{\lemma{\textnormal{\emph{Luſtſpielabend}}}\Cendnote{\textnormal{Uraufführung von \emph{Die Schwestern}\pwindex{Schnitzler, Arthur 15.05.1862 – 21.10.1931@\textsc{Schnitzler, Arthur} (15.05.1862 – 21.10.1931), \emph{Schriftsteller, Mediziner}!Schwestern oder Casanova in Spa. Lustspiel in Versen01. 10. 1919@\strich\emph{Die Schwestern oder Casanova in Spa. Lustspiel in Versen} {[}01. 10. 1919{]}|pwk} am 26. 3. 1920.}}}\label{K_L02339_1h}{ }\uline{ſehr} gut gegangen iſt, trotz mittelmäßiger
               Schauſpielerei, und daſs auch andere, Repriſen-abende \uline{ſehr} gut gegangen ſind und daſs überhaupt, wenigſtens in dieſem Betracht,
               eine gute Zeit für Sie iſt, und ich freue mich darüber ſo herzlich als ich nur kann.
               Sie ſind faſt der einzige höhere Schriftſteller, der ſich wirklich ein Publicum, was
               ja ganz etwas anderes {\pb}iſt, als
               eine Gemeinde, zuſa{\geminationm}engebracht hat, und dies ſowohl hier
               als in Deutſchland\oindex{Deutschland@\textbf{Deutschland}|pw} – und hier insbeſondere ſcheinen
               mir manchmal Ihre Arbeiten, wenn ich darüber nachdenke, wirklich die einzigen zu
               ſein, durch deren Aufführung überhaupt ein höheres Theaterleben mit dem Character der
               Gegenwärtigkeit noch beſteht.\pend
           \pstart
           Warum, nebſt allem übrigen Unheil, auch die Schauſpielkunſt in Wien\oindex{Wien@\textbf{Wien}|pw}{ }ſo herabko{\geminationm}en muſste,
               daſs ein Menſch wie ich kaum zweimal {\pb}im Jahr ſich überwinden kann in
               eines dieſer Theater hineinzugehen – das bleibt unerfindlich. Mit »ein Menſch wie
               ich« meine ich einen Menſchen, der \uline{gern} ins Theater
               geht, den ein guter Characterſpieler intereſſiert, ein wirklicher Volkskomiker
               entzückt, ein leidliches Zuſa{\geminationm}enſpiel feſſelt, alles was
               nicht ganz platt u. plump u. übel provinciell iſt, noch anzieht! Und wohin iſt
               überhaupt das Wien\oindex{Wien@\textbf{Wien}|pw}eriſche an dieſen Wien\oindex{Wien@\textbf{Wien}|pw}er Bühnen geko{\geminationm}en? Und wo iſt
               irgend ein beſti{\geminationm}ter Geſchmack, {\pb}irgend eine Intention, irgend eine
               Richtung? Was iſt das für eine grauenvolle Confuſion, für ein Sa{\geminationm}elſurium anſtatt eines Repertoire! Dies alles iſt
               freilich nur ein Detail in einer finſtern Epoche – aber wie könnte man ſich freuen,
               wenn man über dieſer Scheinwelt nur einigermaßen mit Luſt die wirkliche vergeſſen
               könnte.\pend
           \pstart
           In den »\textsc{Casanova}\pwindex{Schnitzler, Arthur 15.05.1862 – 21.10.1931@\textsc{Schnitzler, Arthur} (15.05.1862 – 21.10.1931), \emph{Schriftsteller, Mediziner}!Schwestern oder Casanova in Spa. Lustspiel in Versen01. 10. 1919@\strich\emph{Die Schwestern oder Casanova in Spa. Lustspiel in Versen} {[}01. 10. 1919{]}|pw}« gehe ich natürlich ſobald meine rheumatiſchen Füße mich ſo weit tragen. Ich
               habe böſe 9 Wochen hinter mir, dies iſt das letzte \label{K_L02339_2v}\edtext{\textsc{residuum}}{\lemma{\textnormal{\emph{residuum}}}\Cendnote{\textnormal{lateinisch: Rest; hier im medizinischen
                  Sinne von: Restsymptome einer abheilenden Erkrankung}}}\label{K_L02339_2h}.\pend
           \pstart
           Von Herzen Ihr{\\[\baselineskip]}\spacefill\mbox{Hugo}\pend
           \leftskip=0em{}\pstart
           \textsc{PS}. Über Oſtern{ }ſind wir in R.\oindex{Rodaun@\textbf{Rodaun}|pw}\pend
           
         
         \endnumbering\mylabel{h}\end{ledgroupsized}  \newcommand{\dateiname}{L02339}\newcommand{\titel}{Hugo Hofmannsthal an Arthur Schnitzler, 31. 3. 1920}\newcommand{\editorInnen}{Martin Anton Müller und Gerd-Hermann Susen}%% latex-leseansicht-abspann.tex
%% Abspann für die Leseansicht.
%% Der Schalter \ifkorrekturansicht ist bereits durch den Vorspann gesetzt.

%% latex-abspann.tex
%% Gemeinsamer Abspann für Korrekturansicht und Leseansicht.
%% Setzt den Schalter \ifkorrekturansicht voraus (gesetzt in den
%% einbindenden Dateien latex-korrekturansicht-abspann.tex bzw.
%% latex-leseansicht-abspann.tex).
%% ---------------------------------------------------------------

\normalsize

% Das esempio-Environment wird nur in der Leseansicht benötigt
\ifkorrekturansicht\else
\newenvironment{esempio}[3]%
{
    \vspace{1.5ex}
    \rlap{\underline{#1}}
    \par
    \setlength{\parindent}{0cm}
    \nopagebreak
    \leftskip=#2cm
    \rightskip=#3cm
}
{
    \par
}
\fi

\doendnotes{C}
\bigskip
\vfill

\clearpage

\footnotesize

\ifkorrekturansicht
  \lohead{\textsc{register}}
\fi

% theindex-Environment neu definieren ohne reledmac
\makeatletter
\renewenvironment{theindex}{%
  \ifkorrekturansicht
    \section*{\indexname}%
  \else
    \subsubsection*{Index der erwähnten Entitäten}%
  \fi
  \setlength{\parindent}{0pt}%
  \setlength{\parskip}{0pt plus 0.3pt}%
  \let\item\@idxitem
}{%
  \ifkorrekturansicht\clearpage\fi
}
\makeatother

\IfFileExists{\jobname-pw.ind}{\input{\jobname-pw.ind}}{}

% Quellenangabe nur in der Leseansicht
\ifkorrekturansicht\else
% Fallback-Definitionen, falls die .tex-Datei \titel etc. nicht gesetzt hat
\providecommand{\titel}{}
\providecommand{\editorInnen}{}
\providecommand{\dateiname}{\jobname}

\vspace{3cm}

\vfill

\footnotesize
\textsc{Quelle}: \titel. Herausgegeben von {\editorInnen}. In: \emph{Arthur Schnitzler: Briefwechsel mit Autorinnen und Autoren}.
 Digitale Edition, https://schnitzler-briefe.acdh.oeaw.ac.at/{\dateiname}.html (Stand \today)
\fi

\end{document}


      