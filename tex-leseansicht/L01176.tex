%% latex-leseansicht-vorspann.tex
%% Vorspann für die Leseansicht.
%% Lädt die gemeinsame Datei latex-vorspann.tex mit nicht gesetztem Schalter.

\newif\ifkorrekturansicht
\korrekturansichtfalse

\input{../tex-inputs/latex-vorspann}


         
         \renewcommand{\erwaehntePersonen}{Personen: Richard Beer-Hofmann}
         \renewcommand{\erwaehnteOrte}{Orte: I., Innere Stadt, Rodaun, Wien, Wollzeile}
         \renewcommand{\erwaehnteWerke}{Werke: Hanna Jagert. Komödie}
               \section[Arthur Schnitzler an Richard Beer-Hofmann, 30. 9. 1901]{ Arthur Schnitzler an Richard Beer-Hofmann, 30. 9. 1901}\nopagebreak\mylabel{v}\rehead{ }\begin{ledgroupsized}[t]{13cm}\normalsize\beginnumbering\briefempfaengerindex{Beer-Hofmann, Richard@\textsc{Beer-Hofmann, Richard}!zzzSchnitzler, Arthur@\emph{von Arthur Schnitzler}!1901-09-301@{30. 9. 1901}|(be} \toendnotes[C]{\smallbreak\pagebreak[2]} \Standort{YCGL, MSS 31.}
\physDesc{Brief, 1 Blatt, 1 Seite, Umschlag, 311 Zeichen (Briefpapier mit Trauerrand)
\newline{}Handschrift: Bleistift, deutsche Kurrent
\newline{}Versand: 1) Rohrpost  2) Stempel: »\nobreak{}Wien, 30 IX 01, 11 10\nobreak{}«.  3) Stempel: »\nobreak{}\oindex{I., Innere Stadt@\textbf{I., Innere Stadt}|pwk}{\pb}Wien 1/1, 30 IX 01, 11 30\nobreak{}«. 
\newline{}Ordnung: mit Bleistift von unbekannter Hand datiert: »30. 9.« }\buchAbdrucke{\weitereDrucke{Arthur Schnitzler, Richard Beer-Hofmann: \emph{Briefwechsel 1891–1931}. Hg. Konstanze Fliedl. Wien, Zürich: \emph{Europaverlag} 1992, S. 156.} }\pstart{}{\pb}Herrn \textsc{Dr Rich
                     Beer-Hofmann}\pend{}\pstart{}Wien\oindex{Wien@\textbf{Wien}|pw}\pend{}\pstart{}\textsc{I. Wollzeile 15\oindex{Wollzeile@\textbf{Wollzeile}|pw}}.\pend{}{\bigskip}\pstart
           \noindent{}{\pb}lieber Richard, wieſo hört u ſieht man nichts von Ihnen? Neulich war
               ich bei Ihnen, Sie waren in Rodaun\oindex{Rodaun@\textbf{Rodaun}|pw}. Wollen Sie
               heut mit mir zu \uline{Hannah Jagert\pwindex{\textcolor{red}{\textsuperscript{XXXX1 indx}}!Hanna Jagert. Komoedie2. 4. 1893@\strich\emph{Hanna Jagert. Komödie} {[}2. 4. 1893{]}|pw}? Ich habe 2 Sitze}.
               Telephoniren Sie mir etwa 2, oder pneumatisiren Sie. \introOben{}(Rendezvous wäre im Foyer unten.)\introOben{}\pend
           \pstart Herzlichſt Ihr\spacefill\mbox{Arthur}\pend{}
         
         \endnumbering\mylabel{h}\end{ledgroupsized}  \newcommand{\dateiname}{L01176}\newcommand{\titel}{Arthur Schnitzler an Richard Beer-Hofmann, 30. 9. 1901}\newcommand{\editorInnen}{Martin Anton Müller und Gerd-Hermann Susen}%% latex-leseansicht-abspann.tex
%% Abspann für die Leseansicht.
%% Der Schalter \ifkorrekturansicht ist bereits durch den Vorspann gesetzt.

%% latex-abspann.tex
%% Gemeinsamer Abspann für Korrekturansicht und Leseansicht.
%% Setzt den Schalter \ifkorrekturansicht voraus (gesetzt in den
%% einbindenden Dateien latex-korrekturansicht-abspann.tex bzw.
%% latex-leseansicht-abspann.tex).
%% ---------------------------------------------------------------

\normalsize

% Das esempio-Environment wird nur in der Leseansicht benötigt
\ifkorrekturansicht\else
\newenvironment{esempio}[3]%
{
    \vspace{1.5ex}
    \rlap{\underline{#1}}
    \par
    \setlength{\parindent}{0cm}
    \nopagebreak
    \leftskip=#2cm
    \rightskip=#3cm
}
{
    \par
}
\fi

\doendnotes{C}
\bigskip
\vfill

\clearpage

\footnotesize

\ifkorrekturansicht
  \lohead{\textsc{register}}
\fi

% theindex-Environment neu definieren ohne reledmac
\makeatletter
\renewenvironment{theindex}{%
  \ifkorrekturansicht
    \section*{\indexname}%
  \else
    \subsubsection*{Index der erwähnten Entitäten}%
  \fi
  \setlength{\parindent}{0pt}%
  \setlength{\parskip}{0pt plus 0.3pt}%
  \let\item\@idxitem
}{%
  \ifkorrekturansicht\clearpage\fi
}
\makeatother

\IfFileExists{\jobname-pw.ind}{\input{\jobname-pw.ind}}{}

% Quellenangabe nur in der Leseansicht
\ifkorrekturansicht\else
% Fallback-Definitionen, falls die .tex-Datei \titel etc. nicht gesetzt hat
\providecommand{\titel}{}
\providecommand{\editorInnen}{}
\providecommand{\dateiname}{\jobname}

\vspace{3cm}

\vfill

\footnotesize
\textsc{Quelle}: \titel. Herausgegeben von {\editorInnen}. In: \emph{Arthur Schnitzler: Briefwechsel mit Autorinnen und Autoren}.
 Digitale Edition, https://schnitzler-briefe.acdh.oeaw.ac.at/{\dateiname}.html (Stand \today)
\fi

\end{document}


      