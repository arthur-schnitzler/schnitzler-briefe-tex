%% latex-leseansicht-vorspann.tex
%% Vorspann für die Leseansicht.
%% Lädt die gemeinsame Datei latex-vorspann.tex mit nicht gesetztem Schalter.

\newif\ifkorrekturansicht
\korrekturansichtfalse

\input{../tex-inputs/latex-vorspann}


\section[Richard Beer-Hofmann an Arthur Schnitzler, 15. 6. 1898]{L00805 Richard Beer-Hofmann an Arthur Schnitzler, 15. 6. 1898}
\nopagebreak\mylabel{L00805v}
\rehead{ }\normalsize\beginnumbering\briefempfaengerindex{Schnitzler, Arthur@\textsc{Schnitzler, Arthur}!zzzBeer-Hofmann, Richard@\emph{von Richard Beer-Hofmann}!1898-06-151@{15. 6. 1898}|(be}
\toendnotes[C]{\smallbreak\pagebreak[2]}
\correspDesc{Versand  durch Richard Beer-Hofmann am 15. 6. 1898 in Steindorf am Ossiacher See
\newline{}Erhalt  durch Arthur Schnitzler im Zeitraum [16. 6. 1898
                  – 20. 6. 1898?] in Wien}\toendnotes[C]{\smallbreak}
\Standort{CUL, Schnitzler, B 8.}
\physDesc{Brief, 1 Blatt, 3 Seiten, 736 Zeichen
\newline{}Handschrift: Bleistift, lateinische Kurrent
\newline{}Ordnung: mit Bleistift von unbekannter Hand nummeriert:
                                    »115« }
\buchAbdrucke{\weitereDrucke{Arthur Schnitzler, Richard Beer-Hofmann: \emph{Briefwechsel 1891–1931}. Herausgegeben von Konstanze Fliedl. Wien, Zürich: \emph{Europaverlag} 1992, S. 118.} }\toendnotes[C]{\smallbreak}
\pstart
           \raggedleft{}{\pb}Steindorf\oindex{Steindorf am Ossiacher See@\textbf{Steindorf am Ossiacher See}, \emph{Verwaltungsgebiet}|pw}{ }15/VI 1898\pend
           \vspace{0.5em}
\pstart
           Lieber Arthur! Ich sende Ihnen beiliegend den Korrecturbogen \uline{des}{ }Gedichtes\pwindex{Beer-Hofmann, Richard 11.\,7.\,1866 Wien – 26.\,9.\,1945 New York City@\textsc{Beer-Hofmann, Richard} (11.\,7.\,1866 Wien – 26.\,9.\,1945 New York City), \emph{Schriftsteller}!Schlaflied für Mirjam@\strich\emph{Schlaflied für Mirjam}|pwv}. Bitte senden Sie ihn mir baldmöglichst zurück und sagen Sie mir
               ob Sie irgend etwas in der Interpunction stört oder nicht. Außerdem: IV. Zeile, II.
               Strophe:\pend
           
\pstart
           »Dir, und auch mir, und uns\textcolor{gray}{«} etc. –\pend
           \leftskip=3em{}
\pstart
           \noindent{}oder\pend
           \leftskip=0em{}
\pstart
           Dir, und mir, und uns etc. –?\pend
           
\pstart
           {\pb}Wenn Sie Hugo\pwindex{Hofmannsthal, Hugo von 1.\,2.\,1874 Wien – 15.\,7.\,1929 Rodaun@\textsc{Hofmannsthal, Hugo von} (1.\,2.\,1874 Wien – 15.\,7.\,1929 Rodaun), \emph{Schriftsteller}|pw} sehen fragen Sie ihn darum; aber bestellen Sie ihn wegen
               dieser wichtigen Angelegenheit nicht ins Caffée da er ja jetzt zu arbeiten hat. –\pend
           
\pstart
           Hier gießt es »noch so sehr, und wie geht es Ihnen«?\pend
           
\pstart
           Bleibt Schlenther\pwindex{Schlenther, Paul 20.\,8.\,1854 Chernyakhovsk – 30.\,4.\,1916 Berlin@\textsc{Schlenther, Paul} (20.\,8.\,1854 Chernyakhovsk – 30.\,4.\,1916 Berlin), \emph{Schriftsteller, Kritiker, Theaterleiter}|pw}?\pend
           
\pstart
           Brief, Carton, haben Sie ja wol erhalten? Bitte reco{\geminationm}andiren Sie den Brief mit dem Korrecturbo{\pb}gen. Hugo\pwindex{Hofmannsthal, Hugo von 1.\,2.\,1874 Wien – 15.\,7.\,1929 Rodaun@\textsc{Hofmannsthal, Hugo von} (1.\,2.\,1874 Wien – 15.\,7.\,1929 Rodaun), \emph{Schriftsteller}|pw}, Schwarzkopf\pwindex{Schwarzkopf, Gustav 7.\,11.\,1853 Wien – 13.\,11.\,1939 ebd.@\textsc{Schwarzkopf, Gustav} (7.\,11.\,1853 Wien – 13.\,11.\,1939 ebd.), \emph{Schriftsteller}|pw}, Leo\pwindex{Van-Jung, Leo 15.\,10.\,1866 Odessa – 2.\,7.\,1939 Riga@\textsc{Van-Jung, Leo} (15.\,10.\,1866 Odessa – 2.\,7.\,1939 Riga), \emph{Gesangspädagoge, Mathematiker}|pw} grüßen Sie herzlich von mir.\pend
           
\pstart
           Herzlichst Ihr{\\[\baselineskip]}\spacefill\mbox{Richard}\pend
           \leftskip=0em{}\selectlanguage{ngerman}\vspace{1em}
\pstart
           \noindent{}Soeben erhalte ich Brief von Hugo\pwindex{Hofmannsthal, Hugo von 1.\,2.\,1874 Wien – 15.\,7.\,1929 Rodaun@\textsc{Hofmannsthal, Hugo von} (1.\,2.\,1874 Wien – 15.\,7.\,1929 Rodaun), \emph{Schriftsteller}|pw} –\pend
           \selectlanguage{ngerman}\endnumbering\briefempfaengerindex{Schnitzler, Arthur@\textsc{Schnitzler, Arthur}!zzzBeer-Hofmann, Richard@\emph{von Richard Beer-Hofmann}!1898-06-151@{15. 6. 1898}|)be}\mylabel{L00805h}  \newcommand{\dateiname}{L00805}\newcommand{\titel}{Richard Beer-Hofmann an Arthur Schnitzler, 15. 6. 1898}\newcommand{\editorInnen}{Martin Anton Müller und Gerd-Hermann Susen}%% latex-leseansicht-abspann.tex
%% Abspann für die Leseansicht.
%% Der Schalter \ifkorrekturansicht ist bereits durch den Vorspann gesetzt.

%% latex-abspann.tex
%% Gemeinsamer Abspann für Korrekturansicht und Leseansicht.
%% Setzt den Schalter \ifkorrekturansicht voraus (gesetzt in den
%% einbindenden Dateien latex-korrekturansicht-abspann.tex bzw.
%% latex-leseansicht-abspann.tex).
%% ---------------------------------------------------------------

\normalsize

% Das esempio-Environment wird nur in der Leseansicht benötigt
\ifkorrekturansicht\else
\newenvironment{esempio}[3]%
{
    \vspace{1.5ex}
    \rlap{\underline{#1}}
    \par
    \setlength{\parindent}{0cm}
    \nopagebreak
    \leftskip=#2cm
    \rightskip=#3cm
}
{
    \par
}
\fi

\doendnotes{C}
\bigskip
\vfill

\clearpage

\footnotesize

\ifkorrekturansicht
  \lohead{\textsc{register}}
\fi

% theindex-Environment neu definieren ohne reledmac
\makeatletter
\renewenvironment{theindex}{%
  \ifkorrekturansicht
    \section*{\indexname}%
  \else
    \subsubsection*{Index der erwähnten Entitäten}%
  \fi
  \setlength{\parindent}{0pt}%
  \setlength{\parskip}{0pt plus 0.3pt}%
  \let\item\@idxitem
}{%
  \ifkorrekturansicht\clearpage\fi
}
\makeatother

\IfFileExists{\jobname-pw.ind}{\input{\jobname-pw.ind}}{}

% Quellenangabe nur in der Leseansicht
\ifkorrekturansicht\else
% Fallback-Definitionen, falls die .tex-Datei \titel etc. nicht gesetzt hat
\providecommand{\titel}{}
\providecommand{\editorInnen}{}
\providecommand{\dateiname}{\jobname}

\vspace{3cm}

\vfill

\footnotesize
\textsc{Quelle}: \titel. Herausgegeben von {\editorInnen}. In: \emph{Arthur Schnitzler: Briefwechsel mit Autorinnen und Autoren}.
 Digitale Edition, https://schnitzler-briefe.acdh.oeaw.ac.at/{\dateiname}.html (Stand \today)
\fi

\end{document}


