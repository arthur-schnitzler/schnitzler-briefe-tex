%% latex-korrekturansicht-vorspann.tex
%% Vorspann für die Korrekturansicht.
%% Lädt die gemeinsame Datei latex-vorspann.tex mit gesetztem Schalter.

\newif\ifkorrekturansicht
\korrekturansichttrue

\input{../tex-inputs/latex-vorspann}


\section[Therese Rie-Andro an Arthur Schnitzler, 25. 12. 1927]{L02568 Therese Rie-Andro an Arthur Schnitzler, 25. 12. 1927}
\nopagebreak\mylabel{L02568v}
\rehead{ }\normalsize\beginnumbering\briefempfaengerindex{Schnitzler, Arthur@\textsc{Schnitzler, Arthur}!zzzRie, Therese@\emph{von Therese Rie}!1927-12-253@{25. 12. 1927}|(be}
\toendnotes[C]{\smallbreak\pagebreak[2]}\Standort{CUL, Schnitzler, B 82.}
\physDesc{Brief, 1 Blatt, 1 Seite, 1063 Zeichen
\newline{}Handschrift: schwarze Tinte, lateinische Kurrent
\newline{}Schnitzler: mit rotem Buntstift beschriftet: »\textsc{Aph\pwindex{Buch der Sprueche und Bedenken@\emph{Buch der Sprüche und Bedenken}|pw}}« und fünf Unterstreichungen }\toendnotes[C]{\smallbreak}
\pstart
           \centering{}{\pb}Wien\oindex{Wien@\textbf{Wien}, \emph{A.ADM2}|pw}, Weihnachten 1927.\pend
           
\pstart
           \centering{}IV, Schönburgstr. 48\oindex{Schoenburgstrasse@\textbf{Schönburgstraße}, \emph{Straße (K.STR)}|pw}.\pend
           
\pstart{}Verehrter Herr Doktor,\pend\vspace{0.5em}
\pstart
           Bitte, nehmen Sie einen Brief wie eine leise und bescheidene Sti{\geminationm}e, die bis zu Ihnen will. Ich möchte Ihnen nichts
               andres sagen, als daſs mich Ihre Sprüche und
                  Bedenken\pwindex{Buch der Sprueche und Bedenken@\emph{Buch der Sprüche und Bedenken}|pw} so tief ergriffen haben, wie es mir nur noch einmal im Leben
               geschehen ist: als ich in meiner Jugend Nietzsches\pwindex{Nietzsche, Friedrich 15.10.1844 – 25.08.1900@\textsc{Nietzsche, Friedrich} (15.10.1844 – 25.08.1900), \emph{Schriftsteller/Schriftstellerin, Philosoph/Philosophin}|pw}{ }Morgenröthe\pwindex{Morgenroete. Gedanken ueber die moralischen Vorurteile@\emph{Morgenröte. Gedanken über die moralischen Vorurteile}|pw} zum erstenmale
               in die Hand bekam. Damals sprangen Tränen auf – wie gestern, als ich Ihr Buch\pwindex{Buch der Sprueche und Bedenken@\emph{Buch der Sprüche und Bedenken}|pwv} las. Wie schön, dass einem
               dergleichen noch passieren kann! Da iſt jedes Wort erlebt und erfühlt und erblutet{\dots} Ich drücke das sehr schlecht aus, aber Sie haben ja
               selbst von dem Riesen \label{K_L02568-1v}\edtext{gesprochen}{\lemma{\textnormal{\emph{gesprochen}}}\Cendnote{\textnormal{»Vom steilen Weg ist Lipp’ und Herz verdorrt,{ / }Doch endlich lohnt ein köstliches Gelingen:{ / }Der Wahrheit Tempel ragt an heil’gem Ort. –{ / }Da dröhnt es aus dem Dunkel: Weiche fort!{ / }Hier wird kein Sterblicher sich Einlaß zwingen,{ / }Ein Riese hält am Tore Wacht: das Wort.« (S. 15.)}}}\label{K_L02568-1}, der an einer Tür der Wahrheit Wache
                  \textcolor{gray}{hält}, dem Wort. Ich kann mit ihm nicht ringen, bei einem
               Boxkampf zwischen ihm und mir käme nicht viel heraus. Ich möchte Ihnen nur ganz
               subjektiv danken für dieses – vielleicht schönſte Ihrer Bücher. Und ich habe keine
               andre Berechtigung dazu, es zu tun, als daſs ich von Jugend auf mit Ihren Gestalten
               gelebt habe und daſs sie mich bis zum heutigen Tage begleiten.\pend
           
\pstart
           Ihre{\\[\baselineskip]}\spacefill\mbox{ThereseRie-Andro.}\pend
           \leftskip=0em{}\selectlanguage{ngerman}\endnumbering\briefempfaengerindex{Schnitzler, Arthur@\textsc{Schnitzler, Arthur}!zzzRie, Therese@\emph{von Therese Rie}!1927-12-253@{25. 12. 1927}|)be}\mylabel{L02568h}  \normalsize

\doendnotes{C}
\bigskip
\vfill

\clearpage

\footnotesize

\lohead{\textsc{register}}

% Definiere theindex-Environment komplett neu ohne reledmac
\makeatletter
\renewenvironment{theindex}{%
  \section*{\indexname}%
  \setlength{\parindent}{0pt}%
  \setlength{\parskip}{0pt plus 0.3pt}%
  \let\item\@idxitem
}{%
  \clearpage
}
\makeatother

\IfFileExists{\jobname-pw.ind}{\input{\jobname-pw.ind}}{}

\end{document}

      