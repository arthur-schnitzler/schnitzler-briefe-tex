%% latex-leseansicht-vorspann.tex
%% Vorspann für die Leseansicht.
%% Lädt die gemeinsame Datei latex-vorspann.tex mit nicht gesetztem Schalter.

\newif\ifkorrekturansicht
\korrekturansichtfalse

\input{../tex-inputs/latex-vorspann}


               \section[Therese Rie-Andro an Arthur Schnitzler, 25. 12. 1927]{ Therese Rie-Andro an Arthur Schnitzler, 25. 12. 1927}\nopagebreak\mylabel{v}\rehead{ }\begin{ledgroupsized}[t]{13cm}\normalsize\beginnumbering\briefempfaengerindex{Schnitzler, Arthur@\textsc{Schnitzler, Arthur}!zzzRie, Therese@\emph{von Therese Rie}!1927-12-253@{25. 12. 1927}|(be} \toendnotes[C]{\smallbreak\pagebreak[2]} \Standort{CUL, Schnitzler, B 82.}
\physDesc{Brief, 1 Blatt, 1 Seite
\newline{}Handschrift: schwarze Tinte, lateinische Kurrent
\newline{}Schnitzler: mit rotem Buntstift beschriftet: »\textsc{Aph\pwindex{Schnitzler, Arthur 15.05.1862 – 21.10.1931@\textsc{Schnitzler, Arthur} (15.05.1862 – 21.10.1931), \emph{Schriftsteller, Mediziner}!Buch der Sprueche und Bedenken1927@\strich\emph{Buch der Sprüche und Bedenken} {[}1927{]}|pw}}« und fünf Unterstreichungen }\toendnotes[C]{\smallbreak}\pstart
           \centering{}{\pb}Wien\oindex{Wien@\textbf{Wien}|pw}, Weihnachten
                        1927.\pend
           \pstart
           \centering{}IV, Schönburgstr. 48\oindex{Schoenburgstrasse@\textbf{Schönburgstraße}|pw}.\pend
           \pstart{}Verehrter Herr Doktor,\pend\pstart
           Bitte, nehmen Sie einen Brief wie eine leise und bescheidene Sti{\geminationm}e, die bis zu Ihnen will. Ich möchte Ihnen nichts
                    andres sagen, als daſs mich Ihre Sprüche und
                        Bedenken\pwindex{Schnitzler, Arthur 15.05.1862 – 21.10.1931@\textsc{Schnitzler, Arthur} (15.05.1862 – 21.10.1931), \emph{Schriftsteller, Mediziner}!Buch der Sprueche und Bedenken1927@\strich\emph{Buch der Sprüche und Bedenken} {[}1927{]}|pw} so tief ergriffen haben, wie es mir nur noch einmal im Leben
                    geschehen ist: als ich in meiner Jugend Nietzsche\pwindex{Nietzsche, Friedrich 15.10.1844 – 25.08.1900@\textsc{Nietzsche, Friedrich} (15.10.1844 – 25.08.1900), \emph{Schriftsteller, Philosoph}|pw}s Morgenröthe\pwindex{Nietzsche, Friedrich 15.10.1844 – 25.08.1900@\textsc{Nietzsche, Friedrich} (15.10.1844 – 25.08.1900), \emph{Schriftsteller, Philosoph}!Morgenroete. Gedanken ueber die moralischen Vorurteile1881-07-01 – 1881-07-01@\strich\emph{Morgenröte. Gedanken über die moralischen Vorurteile} {[}1881-07-01 – 1881-07-01{]}|pw} zum
                    erstenmale in die Hand bekam. Damals sprangen Tränen auf – wie gestern, als ich
                    Ihr Buch\pwindex{Schnitzler, Arthur 15.05.1862 – 21.10.1931@\textsc{Schnitzler, Arthur} (15.05.1862 – 21.10.1931), \emph{Schriftsteller, Mediziner}!Buch der Sprueche und Bedenken1927@\strich\emph{Buch der Sprüche und Bedenken} {[}1927{]}|pwv} las. Wie schön,
                    dass einem dergleichen noch passieren kann! Da iſt jedes Wort erlebt und erfühlt
                    und erblutet{\dots} Ich drücke das sehr schlecht aus, aber
                    Sie haben ja selbst von dem Riesen \label{K_L02568-1v}\edtext{gesprochen}{\lemma{\textnormal{\emph{gesprochen}}}\Cendnote{\textnormal{»Vom steilen Weg ist Lipp’ und Herz verdorrt,{ / }Doch endlich lohnt ein köstliches Gelingen:{ / }Der Wahrheit Tempel ragt an heil’gem Ort. –{ / }Da dröhnt es aus dem Dunkel: Weiche fort!{ / }Hier wird kein Sterblicher sich Einlaß zwingen,{ / }Ein Riese hält am Tore Wacht: das Wort.« (S. 15)}}}\label{K_L02568-1h}, der an einer Tür der Wahrheit Wache
                        \textcolor{gray}{hält}, dem Wort. Ich kann mit ihm nicht ringen, bei einem
                    Boxkampf zwischen ihm und mir käme nicht viel heraus. Ich möchte Ihnen nur ganz
                    subjektiv danken für dieses – vielleicht schönſte Ihrer Bücher. Und ich habe
                    keine andre Berechtigung dazu, es zu tun, als daſs ich von Jugend auf mit Ihren
                    Gestalten gelebt habe und daſs sie mich bis zum heutigen Tage begleiten.\pend
           \pstart
           Ihre{\\[\baselineskip]}\spacefill\mbox{ThereseRie-Andro.}\pend
           \leftskip=0em{}\endnumbering\briefempfaengerindex{Schnitzler, Arthur@\textsc{Schnitzler, Arthur}!zzzRie, Therese@\emph{von Therese Rie}!1927-12-253@{25. 12. 1927}|)be}\mylabel{h}\end{ledgroupsized}  \newcommand{\dateiname}{L02568}\newcommand{\titel}{Therese Rie-Andro an Arthur Schnitzler, 25. 12. 1927}\newcommand{\editorInnen}{Martin Anton Müller und Gerd-Hermann Susen}
            \footnotesize
\begin{ledgroupsized}[t]{11.5cm}
\doendnotes{C}
\end{ledgroupsized}
         %% latex-leseansicht-abspann.tex
%% Abspann für die Leseansicht.
%% Der Schalter \ifkorrekturansicht ist bereits durch den Vorspann gesetzt.

%% latex-abspann.tex
%% Gemeinsamer Abspann für Korrekturansicht und Leseansicht.
%% Setzt den Schalter \ifkorrekturansicht voraus (gesetzt in den
%% einbindenden Dateien latex-korrekturansicht-abspann.tex bzw.
%% latex-leseansicht-abspann.tex).
%% ---------------------------------------------------------------

\normalsize

% Das esempio-Environment wird nur in der Leseansicht benötigt
\ifkorrekturansicht\else
\newenvironment{esempio}[3]%
{
    \vspace{1.5ex}
    \rlap{\underline{#1}}
    \par
    \setlength{\parindent}{0cm}
    \nopagebreak
    \leftskip=#2cm
    \rightskip=#3cm
}
{
    \par
}
\fi

\doendnotes{C}
\bigskip
\vfill

\clearpage

\footnotesize

\ifkorrekturansicht
  \lohead{\textsc{register}}
\fi

% theindex-Environment neu definieren ohne reledmac
\makeatletter
\renewenvironment{theindex}{%
  \ifkorrekturansicht
    \section*{\indexname}%
  \else
    \subsubsection*{Index der erwähnten Entitäten}%
  \fi
  \setlength{\parindent}{0pt}%
  \setlength{\parskip}{0pt plus 0.3pt}%
  \let\item\@idxitem
}{%
  \ifkorrekturansicht\clearpage\fi
}
\makeatother

\IfFileExists{\jobname-pw.ind}{\input{\jobname-pw.ind}}{}

% Quellenangabe nur in der Leseansicht
\ifkorrekturansicht\else
% Fallback-Definitionen, falls die .tex-Datei \titel etc. nicht gesetzt hat
\providecommand{\titel}{}
\providecommand{\editorInnen}{}
\providecommand{\dateiname}{\jobname}

\vspace{3cm}

\vfill

\footnotesize
\textsc{Quelle}: \titel. Herausgegeben von {\editorInnen}. In: \emph{Arthur Schnitzler: Briefwechsel mit Autorinnen und Autoren}.
 Digitale Edition, https://schnitzler-briefe.acdh.oeaw.ac.at/{\dateiname}.html (Stand \today)
\fi

\end{document}


      