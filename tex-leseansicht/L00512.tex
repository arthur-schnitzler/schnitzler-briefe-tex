%% latex-leseansicht-vorspann.tex
%% Vorspann für die Leseansicht.
%% Lädt die gemeinsame Datei latex-vorspann.tex mit nicht gesetztem Schalter.

\newif\ifkorrekturansicht
\korrekturansichtfalse

\input{../tex-inputs/latex-vorspann}


         
         \renewcommand{\erwaehntePersonen}{Personen: Richard Beer-Hofmann, Georg Zelzer}
         \renewcommand{\erwaehnteInstitutionen}{Institutionen: Volkstheater}
         \renewcommand{\erwaehnteOrte}{Orte: Burgtheater, I., Innere Stadt, Wien, Wollzeile}
         \renewcommand{\erwaehnteWerke}{Werke: Das Glück im Winkel, Goldene Herzen. Wiener Volksstück in 4 Akten}
               \section[Arthur Schnitzler an Richard Beer-Hofmann, 7. 11. 1895]{ Arthur Schnitzler an Richard Beer-Hofmann, 7. 11. 1895}\nopagebreak\mylabel{v}\rehead{ }\begin{ledgroupsized}[t]{13cm}\normalsize\beginnumbering \toendnotes[C]{\smallbreak\pagebreak[2]} \Standort{YCGL, MSS 31.}
\physDesc{Postkarte
\newline{}Handschrift: Bleistift, deutsche Kurrent\newline{}Versand: Stempel: »\nobreak{}\oindex{I., Innere Stadt@\textbf{I., Innere Stadt}|pwk}Wien 1/1, 7. 11. 95, 8–9 N\nobreak{}«.  }\toendnotes[C]{\smallbreak}\pstart{}{\pb}Herrn \textsc{Dr. Richard
                     Beer-Hofmann}\pend{}\pstart{}Wien\oindex{Wien@\textbf{Wien}|pw}. \pend{}\pstart{}\textsc{I Wollzeile 15\oindex{Wollzeile@\textbf{Wollzeile}|pw}}\pend{}{\bigskip}\pstart
           \noindent{}{\pb}Lieber Richard, für den Fall,
               daß ich Sie früher nicht ſehe: den Sitz zu \label{K_L00512_1v}\edtext{Goldene Herzen\pwindex{\textcolor{red}{\textsuperscript{XXXX1 indx}}!Goldene Herzen. Wiener Volksstueck in 4 Akten9. 11. 1895@\strich\emph{Goldene Herzen. Wiener Volksstück in 4 Akten} {[}9. 11. 1895{]}|pw}}{\lemma{\textnormal{\emph{Goldene Herzen}}}\Cendnote{\textnormal{Uraufführung am
                  9. 11. 1895 am \emph{Deutschen Volkstheater}\orgindex{Volkstheater@Volkstheater|pwk}; Schnitzler\pwindex{Schnitzler, Arthur 15.05.1862 – 21.10.1931@\textsc{Schnitzler, Arthur} (15.05.1862 – 21.10.1931), \emph{Schriftsteller, Mediziner}|pwk} besuchte gleichfalls die Vorstellung.}}}\label{K_L00512_1h} erhalten Sie
               Samſtag zugeſandt. Auch \textsc{Zelzer}\pwindex{Zelzer, Georg 1850? – 3.1.1903@\textsc{Zelzer, Georg} (1850? – 3.1.1903), \emph{Kassier}|pw} hab ich ſchon wegen des \label{K_L00512_2v}\edtext{Winkelglücks\pwindex{\textcolor{red}{\textsuperscript{XXXX1 indx}}!Glueck im Winkel1895@\strich\emph{Das Glück im Winkel} {[}1895{]}|pw}}{\lemma{\textnormal{\emph{Winkelglücks}}}\Cendnote{\textnormal{Karten für die Uraufführung am
                  11. 11. 1895 am Burgtheater\oindex{Burgtheater@\textbf{Burgtheater}|pwk}; Schnitzler\pwindex{Schnitzler, Arthur 15.05.1862 – 21.10.1931@\textsc{Schnitzler, Arthur} (15.05.1862 – 21.10.1931), \emph{Schriftsteller, Mediziner}|pwk} besuchte gleichfalls die Vorstellung.
               }}}\label{K_L00512_2h} geſprochen.\pend
           \pstart
           Herzlich Ihr{\\[\baselineskip]}\spacefill\mbox{Arthur}\pend
           \leftskip=0em{}
         
         \endnumbering\mylabel{h}\end{ledgroupsized}  \newcommand{\dateiname}{L00512}\newcommand{\titel}{Arthur Schnitzler an Richard Beer-Hofmann, 7. 11. 1895}\newcommand{\editorInnen}{Martin Anton Müller und Gerd-Hermann Susen}%% latex-leseansicht-abspann.tex
%% Abspann für die Leseansicht.
%% Der Schalter \ifkorrekturansicht ist bereits durch den Vorspann gesetzt.

%% latex-abspann.tex
%% Gemeinsamer Abspann für Korrekturansicht und Leseansicht.
%% Setzt den Schalter \ifkorrekturansicht voraus (gesetzt in den
%% einbindenden Dateien latex-korrekturansicht-abspann.tex bzw.
%% latex-leseansicht-abspann.tex).
%% ---------------------------------------------------------------

\normalsize

% Das esempio-Environment wird nur in der Leseansicht benötigt
\ifkorrekturansicht\else
\newenvironment{esempio}[3]%
{
    \vspace{1.5ex}
    \rlap{\underline{#1}}
    \par
    \setlength{\parindent}{0cm}
    \nopagebreak
    \leftskip=#2cm
    \rightskip=#3cm
}
{
    \par
}
\fi

\doendnotes{C}
\bigskip
\vfill

\clearpage

\footnotesize

\ifkorrekturansicht
  \lohead{\textsc{register}}
\fi

% theindex-Environment neu definieren ohne reledmac
\makeatletter
\renewenvironment{theindex}{%
  \ifkorrekturansicht
    \section*{\indexname}%
  \else
    \subsubsection*{Index der erwähnten Entitäten}%
  \fi
  \setlength{\parindent}{0pt}%
  \setlength{\parskip}{0pt plus 0.3pt}%
  \let\item\@idxitem
}{%
  \ifkorrekturansicht\clearpage\fi
}
\makeatother

\IfFileExists{\jobname-pw.ind}{\input{\jobname-pw.ind}}{}

% Quellenangabe nur in der Leseansicht
\ifkorrekturansicht\else
% Fallback-Definitionen, falls die .tex-Datei \titel etc. nicht gesetzt hat
\providecommand{\titel}{}
\providecommand{\editorInnen}{}
\providecommand{\dateiname}{\jobname}

\vspace{3cm}

\vfill

\footnotesize
\textsc{Quelle}: \titel. Herausgegeben von {\editorInnen}. In: \emph{Arthur Schnitzler: Briefwechsel mit Autorinnen und Autoren}.
 Digitale Edition, https://schnitzler-briefe.acdh.oeaw.ac.at/{\dateiname}.html (Stand \today)
\fi

\end{document}


      