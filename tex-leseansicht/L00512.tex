\input{../tex-inputs/latex-pdf-vorspann}
\begin{center}
            \textcolor{red}{ENTWURF. ENTZIFFERUNG NOCH NICHT KORREKTURGELESEN}
                      \end{center}
            
               \section[Arthur Schnitzler an Richard Beer-Hofmann, 7. 11. 1895]{ Arthur Schnitzler an Richard Beer-Hofmann, 7. 11. 1895}\nopagebreak\mylabel{v}\rehead{ }\begin{ledgroupsized}[t]{13cm}\normalsize\beginnumbering\briefempfaengerindex{Beer-Hofmann, Richard@\textsc{Beer-Hofmann, Richard}!zzzSchnitzler, Arthur@\emph{von Arthur Schnitzler}!1895-11-071@{7. 11. 1895}|(be} \toendnotes[C]{\smallbreak\pagebreak[2]} \Standort{YCGL, MSS 31.}
\physDesc{Postkarte
\newline{}Handschrift: Bleistift, deutsche Kurrent\newline{}Versand: Stempel: »\nobreak{}\oindex{I., Innere Stadt@\textbf{I., Innere Stadt}|pwk}Wien 1/1, 7. 11. 95, 8–9 N\nobreak{}«.  }\toendnotes[C]{\smallbreak}\pstart{}{\pb}Herrn \textsc{Dr. Richard
                     Beer-Hofmann}\pend{}\pstart{}Wien\oindex{Wien@\textbf{Wien}|pw}. \pend{}\pstart{}\textsc{I Wollzeile 15\oindex{Wollzeile@\textbf{Wollzeile}|pw}}\pend{}{\bigskip}\pstart
           \noindent{}{\pb}Lieber Richard, für den Fall,
               daß ich Sie früher nicht ſehe: den Sitz zu \label{K_L00512_1v}\edtext{Goldene Herzen\pwindex{\textcolor{red}{\textsuperscript{XXXX1 indx}}!Goldene Herzen. Wiener Volksstueck in 4 Akten9. 11. 1895@\strich\emph{Goldene Herzen. Wiener Volksstück in 4 Akten} {[}9. 11. 1895{]}|pw}}{\lemma{\textnormal{\emph{Goldene Herzen}}}\Cendnote{\textnormal{Uraufführung am
                     9. 11. 1895, auch Schnitzler\pwindex{Schnitzler, Arthur 15.05.1862 – 21.10.1931@\textsc{Schnitzler, Arthur} (15.05.1862 – 21.10.1931), \emph{Schriftsteller, Mediziner}|pwk} besuchte die Vorstellung.}}}\label{K_L00512_1h} erhalten Sie
               Samſtag zugeſandt. Auch \textsc{Zelzer}\pwindex{Zelzer, Georg 1850? – 3.1.1903@\textsc{Zelzer, Georg} (1850? – 3.1.1903), \emph{Kassier}|pw} hab ich ſchon wegen des \label{K_L00512_2v}\edtext{Winkelglücks\pwindex{\textcolor{red}{\textsuperscript{XXXX1 indx}}!Glueck im Winkel1895@\strich\emph{Das Glück im Winkel} {[}1895{]}|pw}}{\lemma{\textnormal{\emph{Winkelglücks}}}\Cendnote{\textnormal{Karten für die Uraufführung am
                     9. 11. 1895 am Burgtheater\oindex{Burgtheater@\textbf{Burgtheater}|pwk}}}}\label{K_L00512_2h} geſprochen.\pend
           \pstart
           Herzlich Ihr{\\[\baselineskip]}\spacefill\mbox{Arthur}\pend
           \leftskip=0em{}\endnumbering\briefempfaengerindex{Beer-Hofmann, Richard@\textsc{Beer-Hofmann, Richard}!zzzSchnitzler, Arthur@\emph{von Arthur Schnitzler}!1895-11-071@{7. 11. 1895}|)be}\mylabel{h}\end{ledgroupsized}  \newcommand{\dateiname}{L00512}\newcommand{\titel}{Arthur Schnitzler an Richard Beer-Hofmann, 7. 11. 1895}\newcommand{\editorInnen}{Martin Anton Müller und Gerd-Hermann Susen}\input{../tex-inputs/latex-pdf-abspann}
      