%% latex-leseansicht-vorspann.tex
%% Vorspann für die Leseansicht.
%% Lädt die gemeinsame Datei latex-vorspann.tex mit nicht gesetztem Schalter.

\newif\ifkorrekturansicht
\korrekturansichtfalse

\input{../tex-inputs/latex-vorspann}


\section[ Paul Goldmann an Arthur Schnitzler, 16. 10. {[}1898{]}]{L02861 Paul Goldmann an Arthur Schnitzler,  16. 10. [1898]}
\nopagebreak\mylabel{L02861v}
\rehead{ }\normalsize\beginnumbering\briefempfaengerindex{Schnitzler, Arthur@\textsc{Schnitzler, Arthur}!zzzGoldmann, Paul@\emph{von Paul Goldmann}!1898-10-161@{16. 10. [1898]}|(be}
\toendnotes[C]{\smallbreak\pagebreak[2]}
\correspDesc{Versand  durch Paul Goldmann am 16. 10. [1898] in Peking
\newline{}Erhalt  durch Arthur Schnitzler im Zeitraum [18. 11. 1898 – 18. 12. 1898?] in Wien}\toendnotes[C]{\smallbreak}
\Standort{DLA, A:Schnitzler, HS.NZ85.1.3168.}
\physDesc{Brief, 2 Blätter, 8 Seiten, 4429 Zeichen
\newline{}Handschrift: blaue Tinte, deutsche Kurrent
\newline{}Schnitzler: 1) mit Bleistift das Jahr »98« vermerkt  2) mit rotem Buntstift eine seitliche Markierung}\toendnotes[C]{\smallbreak}
\pstart
           \raggedleft{}{\pb}\textsc{Peking\oindex{Peking@\textbf{Peking}, \emph{Hauptstadt}|pw}}, 16. Oktober.\pend
           
\pstart\center{}Mein lieber Freund,\pend\vspace{0.5em}
\pstart
           Alle Deine Karten von unterwegs,{ }ſowie Deinen lieben Brief aus \label{K_L02861-1v}\edtext{\textsc{Luzern\oindex{Luzern@\textbf{Luzern}|pw}}}{\lemma{\textnormal{\emph{Luzern}}}\Cendnote{\textnormal{Schnitzler hielt sich zwischen 21. 8. 1898 und 28. 8. 1898 in Luzern\oindex{Luzern@\textbf{Luzern}|pwk} auf. Siehe zu Schnitzlers Sommerreise im Jahr 1898 auch XXXX Auszeichnungsfehler: Dokument L02845 nicht gefunden.}}}\label{K_L02861-1} habe
               ich erhalten.\pend
           
\pstart
           Ich freue mich, zu erfahren, daß der Sommer{ }ſo angenehm für Dich verlaufen iſt.
               Hoffentlich bleibt von dem guten Reſultat etwas für den Winter zurück. \strikeout{\textcolor{gray}{×}} Ich kann Dich immer nur wieder darauf hinweiſen: Wenn Alles, was Dich quält,{ }ſich auf Reiſen{ }ſo ganz verliert, kann es doch unmöglich materielle Geſtalt haben. Im
               Übrigen hoffe ich, daß die viele Arbeit, die Du vorhaſt, ein gutes Heilmittel gegen
               die Hypochondrie{ }ſein wird. Schon aus dem Grunde {\pb}bin ich{ }ſehr froh über alle Deine neuen Pläne, von denen Du{ }ſchreibſt. Aber auch{ }ſonſt (unberufen!) iſt es prächtig, wie{ }ſich{ }ſo \strikeout{\textcolor{gray}{V}} Vieles in Dir regt und wie es aus Dir{ }ſo reich herausblüht!\pend
           
\pstart
           Was Du über die Disciplin beim Schaffen{ }ſagſt, iſt{ }ſehr{ }ſchön, aber ich meine, es{ }ſtimmt nicht. Man{ }ſoll{ }ſich nicht{ }ſo fataliſtiſch hinſetzen\strikeout{,} und einfach das aus{ }ſich herausfließen laſſen, was in Einem liegt. Was in
               Dir liegt, iſt zu einem Zehntel vielleicht Natur, zu \strikeout{neun} neun Zehnteln aber das, was Du in Dich hineingelegt haſt. Der
               Schriftſteller iſt doch ein Product aus Natur und aus{ }ſich{ }ſelbſt. Er iſt in
               fortwährender Entwickelung begriffen; und {\pb}während
               er an einem Werke arbeitet, \strikeout{\textcolor{gray}{×}{ }\textcolor{gray}{arb}} arbeitet er zugleich ebenſo an{ }ſich{ }ſelbſt. Gewiß{ }ſoll Jeder nur{ }ſchaffen, was
               er vermag. Aber Jeder{ }ſoll auch beſtrebt{ }ſein, \strikeout{im}
               immer mehr zu vermögen. Gewiß darf Keiner aus{ }ſeiner Art herauswollen. Doch in{ }ſeiner
               Art kann Jeder Alles anſtreben, und auf allen Arten kann man zum Höchſten kommen, wie
                  \strikeout{ja} ja alle Wege zum{ }ſelben Bergesgipfel führen.
               Blaſe Du nur ruhig Deine Flöte, die{ }ſo liebe Klänge gibt. Ich meine \strikeout{nicht} nicht, daß Du auf einmal anfangen{ }ſollſt, die
               Geige zu{ }ſtreichen. Aber ich möchte, daß Du auf Deiner Flöte auch \strikeout{ein} einmal ein \label{K_L02861-2v}\edtext{\uline{anderes}{ }{\pb}Lied}{\lemma{\textnormal{\emph{anderes Lied}}}\Cendnote{\textnormal{Siehe XXXX Auszeichnungsfehler: Dokument L02848 nicht gefunden.
               }}}\label{K_L02861-2}{ }ſpielſt. Die Gleichniſſe{ }ſind alle falſch. Laſſen wir alſo die Gleichniſſe!
               Ich meine: Aus Deinen Novellen\pwindex{Schnitzler, Arthur 15.\,5.\,1862 Wien – 21.\,10.\,1931 ebd.@\textsc{Schnitzler, Arthur} (15.\,5.\,1862 Wien – 21.\,10.\,1931 ebd.), \emph{Schriftsteller, Mediziner}!Frau des Weisen. Novelletten@\strich\emph{Die Frau des Weisen. Novelletten}|pwv}{ }ſehe ich wieder, \strikeout{wie \textcolor{gray}{×}\-\textcolor{gray}{×}\-\textcolor{gray}{×}\-\textcolor{gray}{×}\-\textcolor{gray}{×}\-\textcolor{gray}{×}{ }\textcolor{gray}{×}\-\textcolor{gray}{×}\-\textcolor{gray}{×}\-\textcolor{gray}{×}\-\textcolor{gray}{×}\-\textcolor{gray}{×}{ }\textcolor{gray}{Du}} wie Du große menſchliche Töne zu finden vermagſt. Nur{ }ſteckt das immer in
               einer Liebesgeſchichte gleichſam als Epiſode drin. Warum nicht die Liebesgeſchichte
               einmal weglaſſen und das große Menſchliche \strikeout{\textcolor{gray}{al}}{ }\strikeout{ſch} allein{ }ſchreiben, ohne alle Liebe? Oder meinſt Du
               wirklich, daß Du ein »\textsc{Erotiker\pwindex{Lothar, Rudolf 23.\,2.\,1865 Budapest – 2.\,10.\,1943 ebd.@\textsc{Lothar, Rudolf} (23.\,2.\,1865 Budapest – 2.\,10.\,1943 ebd.), \emph{Schriftsteller, Journalist, Theaterdirektor}!Briefe an eine Dame@\strich\emph{Briefe an eine Dame}|pwv}}« biſt, wie dieſes Rindvieh \textsc{Lothar\pwindex{Lothar, Rudolf 23.\,2.\,1865 Budapest – 2.\,10.\,1943 ebd.@\textsc{Lothar, Rudolf} (23.\,2.\,1865 Budapest – 2.\,10.\,1943 ebd.), \emph{Schriftsteller, Journalist, Theaterdirektor}|pw}}{ }\label{K_L02861-3v}\edtext{geſchrieben\pwindex{Lothar, Rudolf 23.\,2.\,1865 Budapest – 2.\,10.\,1943 ebd.@\textsc{Lothar, Rudolf} (23.\,2.\,1865 Budapest – 2.\,10.\,1943 ebd.), \emph{Schriftsteller, Journalist, Theaterdirektor}!Briefe an eine Dame@\strich\emph{Briefe an eine Dame}|pwv}}{\lemma{\textnormal{\emph{geschrieben}}}\Cendnote{\textnormal{Rudolf Lothar\pwindex{Lothar, Rudolf 23.\,2.\,1865 Budapest – 2.\,10.\,1943 ebd.@\textsc{Lothar, Rudolf} (23.\,2.\,1865 Budapest – 2.\,10.\,1943 ebd.), \emph{Schriftsteller, Journalist, Theaterdirektor}|pwk}: \emph{Briefe an eine Dame}\pwindex{Lothar, Rudolf 23.\,2.\,1865 Budapest – 2.\,10.\,1943 ebd.@\textsc{Lothar, Rudolf} (23.\,2.\,1865 Budapest – 2.\,10.\,1943 ebd.), \emph{Schriftsteller, Journalist, Theaterdirektor}!Briefe an eine Dame@\strich\emph{Briefe an eine Dame}|pwk}. In: \emph{Die Wage. Eine Wiener Wochenschrift}\pwindex{Wage. Eine Wiener Wochenschrift@\emph{Die Wage. Eine Wiener Wochenschrift}|pwk}, Jg. 1, Nr. 26,
                        25. 6. 1898, S. 439–440, hier:
                  S. 439.}}}\label{K_L02861-3} hat?\pend
           
\pstart
           Ich bin{ }ſchon ungemein geſpannt auf Dein neues \label{K_L02861-4v}\edtext{Stück\pwindex{Schnitzler, Arthur 15.\,5.\,1862 Wien – 21.\,10.\,1931 ebd.@\textsc{Schnitzler, Arthur} (15.\,5.\,1862 Wien – 21.\,10.\,1931 ebd.), \emph{Schriftsteller, Mediziner}!Vermächtnis. Schauspiel in drei Akten@\strich\emph{Das Vermächtnis. Schauspiel in drei Akten}|pwv}}{\lemma{\textnormal{\emph{Stück}}}\Cendnote{\textnormal{\emph{Das Vermächtnis}\pwindex{Schnitzler, Arthur 15.\,5.\,1862 Wien – 21.\,10.\,1931 ebd.@\textsc{Schnitzler, Arthur} (15.\,5.\,1862 Wien – 21.\,10.\,1931 ebd.), \emph{Schriftsteller, Mediziner}!Vermächtnis. Schauspiel in drei Akten@\strich\emph{Das Vermächtnis. Schauspiel in drei Akten}|pwk} wurde am 8. 10. 1898 am Deutschen Theater\oindex{Deutsches Theater Berlin@\textbf{Deutsches Theater Berlin}, \emph{Theater}|pwk} in Berlin\oindex{Berlin@\textbf{Berlin}, \emph{Hauptstadt}|pwk} uraufgeführt.}}}\label{K_L02861-4} – mehr auf das Stück\pwindex{Schnitzler, Arthur 15.\,5.\,1862 Wien – 21.\,10.\,1931 ebd.@\textsc{Schnitzler, Arthur} (15.\,5.\,1862 Wien – 21.\,10.\,1931 ebd.), \emph{Schriftsteller, Mediziner}!Vermächtnis. Schauspiel in drei Akten@\strich\emph{Das Vermächtnis. Schauspiel in drei Akten}|pwv}{ }{\pb}ſelbſt, als auf das, was das Publicum dazu{ }ſagt.
               Die Idee iſt vortrefflich, und ich{ }ſtelle mir ein{ }ſehr zu Herzen gehendes Drama vor{\dotsfive}\pend
           
\pstart
           Ich bin nun{ }ſchon faſt drei Wochen in \textsc{Peking\oindex{Peking@\textbf{Peking}, \emph{Hauptstadt}|pw}}, dem grauenhafteſten Schmutzneſt der Welt, habe aber manches Intereſſante
               miterlebt, bin auch einmal beinahe dem chin\oindex{China@\textbf{China}|pwv}eſiſchen Pöbel in die Hände gerathen, was{ }ſehr{ }ſchlecht
               hätte ablaufen können. Aber auch die Gefahr hat ihren Reiz – beſonders \strikeout{\textcolor{gray}{×}\-\textcolor{gray}{×}\-\textcolor{gray}{×}} nachher. Zugleich iſt{ }ſie eine gute Lection: Man lernt, ruhig und entſchloſſen{ }ſich zu benehmen. Morgen fahre {\pb}ich wieder nach \textsc{Tientsin\oindex{Tianjin@\textbf{Tianjin}|pw}}, von da nach \textsc{Shanghai\oindex{Shanghai@\textbf{Shanghai}|pw}} zurück. Was dann werden wird, iſt unklar; und dunkel iſt auch, was nach meiner
               Rückkehr geſchehen{ }ſoll. In Wien\oindex{Wien@\textbf{Wien}, \emph{Verwaltungsgebiet}|pw} bleiben? Was{ }ſoll
               ich in einem Lande\oindex{Österreich@\textbf{Österreich}|pwv} machen, wo
               man die Leute einſperrt, wenn{ }ſie vor dem Sakrament nicht den Hut abnehmen? Ich
               glaube, in vier Wochen wäre ich ausgewieſen oder im Gefängniß. Und wem fehle ich in
                  Wien\oindex{Wien@\textbf{Wien}, \emph{Verwaltungsgebiet}|pw}? Dir? Es iſt{ }ſehr lieb, daß Du das{ }ſagſt.
               Aber ich \strikeout{\textcolor{gray}{×}} weiß nicht, ob es gut wäre, wenn wir wieder in einer Stadt {\pb}zuſammenlebten. Wir kennen eigentlich nur unſere
               guten Eigenſchaften und haben unſere{ }ſchlechten vergeſſen. Wer weiß, \strikeout{ob}{ } ob dieſe uns nicht jetzt, wo wir nicht mehr die
               Anpaſſungs-Fähigkeit von ehedem haben,{ }ſehr{ }ſtören \strikeout{und} würden. Wer weiß, \strikeout{was bei} wieviel
               Trennendes{ }ſich bei einem dauernden Zuſammenleben zwiſchen uns plötzlich aufrichtig
               würde! Und wem fehle ich{ }ſonſt in Wien\oindex{Wien@\textbf{Wien}, \emph{Verwaltungsgebiet}|pw}? Keinem
               Menſchen, nicht einmal dem \textsc{Richard\pwindex{Beer-Hofmann, Richard 11.\,7.\,1866 Wien – 26.\,9.\,1945 New York City@\textsc{Beer-Hofmann, Richard} (11.\,7.\,1866 Wien – 26.\,9.\,1945 New York City), \emph{Schriftsteller}|pw}}. Wo{ }ſoll überhaupt in dieſer Stadt\oindex{Wien@\textbf{Wien}, \emph{Verwaltungsgebiet}|pwv} für mich ein Platz{ }ſein? Ich kann ihn nirgends entdecken{\dotsfive}\pend
           
\pstart
           Ich \strikeout{b\textcolor{gray}{at}} bat Dich{ }ſchon, Deine {\pb}lieben Briefe fortan
               an meine Mutter\pwindex{Goldmann, Clementine 15.\,5.\,1842 Breslau – 24.\,2.\,1924 Frankfurt am Main@\textsc{Goldmann, Clementine} (15.\,5.\,1842 Breslau – 24.\,2.\,1924 Frankfurt am Main)|pwv} zu{ }ſenden,
               welche telegraphiſch meine neue Adreſſe erfahren wird. Ich{ }ſelbſt kann Dir
               einſtweilen keine angeben.\pend
           
\pstart
           Empfiehl’ mich Deiner Freundin\pwindex{Reinhard, Marie 13.\,3.\,1871 Wien – 18.\,3.\,1899 ebd.@\textsc{Reinhard, Marie} (13.\,3.\,1871 Wien – 18.\,3.\,1899 ebd.), \emph{Gesangspädagogin}|pwv} und{ }ſei Du{ }ſelbſt von Herzen begrüßt!\pend
           
\pstart
           Dein treuer {\\[\baselineskip]}\spacefill\mbox{Paul Goldmann.}\pend
           \leftskip=0em{}
\pstart
           \noindent{}Bitte,{ }ſage dem \label{K_L02861-5v}\edtext{Herrn\pwindex{Friedmann, Louis Philipp 29.\,6.\,1861 Paris – 1.\,4.\,1939 Wien@\textsc{Friedmann, Louis Philipp} (29.\,6.\,1861 Paris – 1.\,4.\,1939 Wien), \emph{Industrieller, Bergsteiger}|pwuv}}{\lemma{\textnormal{\emph{Herrn}}}\Cendnote{\textnormal{Louis Friedmann\pwindex{Friedmann, Louis Philipp 29.\,6.\,1861 Paris – 1.\,4.\,1939 Wien@\textsc{Friedmann, Louis Philipp} (29.\,6.\,1861 Paris – 1.\,4.\,1939 Wien), \emph{Industrieller, Bergsteiger}|pwk}, vgl. XXXX Auszeichnungsfehler: Dokument L02845 nicht gefunden.}}}\label{K_L02861-5}, der mir die Empfehlung an den \textsc{Dr. von Rosthorn\pwindex{Rosthorn, Arthur 14.\,4.\,1862 Wien – 17.\,12.\,1945 Oed@\textsc{Rosthorn, Arthur} (14.\,4.\,1862 Wien – 17.\,12.\,1945 Oed), \emph{Schriftsteller, Diplomat, Sinologe}|pw}} überſandt hat, daß ich keine Zeit hatte,{ }ſie abzugeben. Es liegt mir daran,
                  daß Du ihm das{ }ſagſt. Ich erkläre es Dir{ }ſpäter einmal.\pend
           
\pstart
           In einem fran\oindex{Frankreich@\textbf{Frankreich}|pwv}zöſiſchen
                  Blatte las ich Berichte über den \label{K_L02861-6v}\edtext{Zioniſten-Congreß}{\lemma{\textnormal{\emph{Zionisten-Congreß}}}\Cendnote{\textnormal{Der zweite
                     Zionistenkongress hatte zwischen 28. 8. 1898 und
                        31. 8. 1898 unter dem Vorsitz Theodor Herzls\pwindex{Herzl, Theodor 2.\,5.\,1860 Budapest – 3.\,7.\,1904 Edlach@\textsc{Herzl, Theodor} (2.\,5.\,1860 Budapest – 3.\,7.\,1904 Edlach), \emph{Schriftsteller, Journalist}|pwk} in Basel\oindex{Basel@\textbf{Basel}|pwk} stattgefunden. Goldmann\pwindex{Goldmann, Paul 31.\,1.\,1865 Breslau – 25.\,9.\,1935 Wien@\textsc{Goldmann, Paul} (31.\,1.\,1865 Breslau – 25.\,9.\,1935 Wien), \emph{Schriftsteller, Journalist}|pwk} hatte bereits in vorangegangenen Briefen an Schnitzler Kritik an Herzl\pwindex{Herzl, Theodor 2.\,5.\,1860 Budapest – 3.\,7.\,1904 Edlach@\textsc{Herzl, Theodor} (2.\,5.\,1860 Budapest – 3.\,7.\,1904 Edlach), \emph{Schriftsteller, Journalist}|pwk} und dem Zionismus geäußert. Siehe etwa die Briefe vom XXXX Auszeichnungsfehler: Dokument L02742 nicht gefunden und XXXX Auszeichnungsfehler: Dokument L02841 nicht gefunden.}}}\label{K_L02861-6}. Das
                  wird doch ein recht widerlicher Unfug!\pend
           \selectlanguage{ngerman}\endnumbering\briefempfaengerindex{Schnitzler, Arthur@\textsc{Schnitzler, Arthur}!zzzGoldmann, Paul@\emph{von Paul Goldmann}!1898-10-161@{16. 10. [1898]}|)be}\mylabel{L02861h}  \newcommand{\dateiname}{L02861}\newcommand{\titel}{Paul Goldmann an Arthur Schnitzler, 16. 10. [1898]}\newcommand{\editorInnen}{Martin Anton Müller und Laura Untner}%% latex-leseansicht-abspann.tex
%% Abspann für die Leseansicht.
%% Der Schalter \ifkorrekturansicht ist bereits durch den Vorspann gesetzt.

%% latex-abspann.tex
%% Gemeinsamer Abspann für Korrekturansicht und Leseansicht.
%% Setzt den Schalter \ifkorrekturansicht voraus (gesetzt in den
%% einbindenden Dateien latex-korrekturansicht-abspann.tex bzw.
%% latex-leseansicht-abspann.tex).
%% ---------------------------------------------------------------

\normalsize

% Das esempio-Environment wird nur in der Leseansicht benötigt
\ifkorrekturansicht\else
\newenvironment{esempio}[3]%
{
    \vspace{1.5ex}
    \rlap{\underline{#1}}
    \par
    \setlength{\parindent}{0cm}
    \nopagebreak
    \leftskip=#2cm
    \rightskip=#3cm
}
{
    \par
}
\fi

\doendnotes{C}
\bigskip
\vfill

\clearpage

\footnotesize

\ifkorrekturansicht
  \lohead{\textsc{register}}
\fi

% theindex-Environment neu definieren ohne reledmac
\makeatletter
\renewenvironment{theindex}{%
  \ifkorrekturansicht
    \section*{\indexname}%
  \else
    \subsubsection*{Index der erwähnten Entitäten}%
  \fi
  \setlength{\parindent}{0pt}%
  \setlength{\parskip}{0pt plus 0.3pt}%
  \let\item\@idxitem
}{%
  \ifkorrekturansicht\clearpage\fi
}
\makeatother

\IfFileExists{\jobname-pw.ind}{\input{\jobname-pw.ind}}{}

% Quellenangabe nur in der Leseansicht
\ifkorrekturansicht\else
% Fallback-Definitionen, falls die .tex-Datei \titel etc. nicht gesetzt hat
\providecommand{\titel}{}
\providecommand{\editorInnen}{}
\providecommand{\dateiname}{\jobname}

\vspace{3cm}

\vfill

\footnotesize
\textsc{Quelle}: \titel. Herausgegeben von {\editorInnen}. In: \emph{Arthur Schnitzler: Briefwechsel mit Autorinnen und Autoren}.
 Digitale Edition, https://schnitzler-briefe.acdh.oeaw.ac.at/{\dateiname}.html (Stand \today)
\fi

\end{document}


