%% latex-leseansicht-vorspann.tex
%% Vorspann für die Leseansicht.
%% Lädt die gemeinsame Datei latex-vorspann.tex mit nicht gesetztem Schalter.

\newif\ifkorrekturansicht
\korrekturansichtfalse

\input{../tex-inputs/latex-vorspann}


         
         \renewcommand{\erwaehntePersonen}{Personen: Richard Beer-Hofmann, Paul Goldmann, Alfred Kerr, Felix Salten, Ottilie Salten, Leo Van-Jung}
         \renewcommand{\erwaehnteOrte}{Orte: Alpen, Bad Ischl, Karlsbad, Meran, Pressbaum, Schruns, Vorarlberg}
         \renewcommand{\erwaehnteWerke}{}
               \section[ Felix Salten an Arthur Schnitzler, 5. 8. 1900]{ Felix Salten an Arthur Schnitzler, 5. 8. 1900}\nopagebreak\mylabel{v}\rehead{ }\begin{ledgroupsized}[t]{13cm}\normalsize\beginnumbering\briefempfaengerindex{Schnitzler, Arthur@\textsc{Schnitzler, Arthur}!zzzSalten, Felix@\emph{von Felix Salten}!1900-08-052@{5. 8. 1900}|(be} \toendnotes[C]{\smallbreak\pagebreak[2]} \Standort{CUL, Schnitzler, B 89, A 2.}
\physDesc{Brief, 1 Blatt, 2 Seiten, 594 Zeichen
\newline{}Handschrift: Bleistift, lateinische Kurrent
\newline{}Ordnung: mit Bleistift von unbekannter Hand nummeriert: »131« }\toendnotes[C]{\smallbreak}\pstart
           \raggedleft{}{\pb}Pressbaum\oindex{Pressbaum@\textbf{Pressbaum}|pw}, 5./VIII. 00\pend
           \pstart
           Lieber Freund, wahrscheinlich komme ich noch vor dem 10. nach \label{K_L03307-1v}\edtext{Ischl\oindex{Bad Ischl@\textbf{Bad Ischl}|pw}}{\lemma{\textnormal{\emph{Ischl}}}\Cendnote{\textnormal{Am 17. 8. 1900 startete Schnitzler\pwindex{Schnitzler, Arthur 15.05.1862 – 21.10.1931@\textsc{Schnitzler, Arthur} (15.05.1862 – 21.10.1931), \emph{Schriftsteller, Mediziner}|pwk} gemeinsam mit Richard Beer-Hofmann\pwindex{Beer-Hofmann, Richard 1866-07-11 – 1945-09-26@\textsc{Beer-Hofmann, Richard} (1866-07-11 – 1945-09-26), \emph{Schriftsteller}|pwk}, Paul Goldmann\pwindex{Goldmann, Paul 31.01.1865 – 25.09.1935@\textsc{Goldmann, Paul} (31.01.1865 – 25.09.1935), \emph{Schriftsteller, Journalist}|pwk}, Alfred Kerr\pwindex{Kerr, Alfred 25.12.1867 – 12.10.1948@\textsc{Kerr, Alfred} (25.12.1867 – 12.10.1948), \emph{Schriftsteller, Kritiker}|pwk} und Leo
                           Van-Jung\pwindex{Van-Jung, Leo 15.10.1866 – 02.07.1939@\textsc{Van-Jung, Leo} (15.10.1866 – 02.07.1939), \emph{Gesangspädagoge, Mathematiker}|pwk} eine Alpen\oindex{Alpen@\textbf{Alpen}|pwk}wanderung in Schruns\oindex{Schruns@\textbf{Schruns}|pwk} (Vorarlberg\oindex{Vorarlberg@\textbf{Vorarlberg}|pwk}). Am 28. 8. 1900 reiste Schnitzler\pwindex{Schnitzler, Arthur 15.05.1862 – 21.10.1931@\textsc{Schnitzler, Arthur} (15.05.1862 – 21.10.1931), \emph{Schriftsteller, Mediziner}|pwk} alleine weiter nach Meran\oindex{Meran@\textbf{Meran}|pwk}, wo er schließlich auf Salten\pwindex{Salten, Felix 06.09.1869 – 08.10.1945@\textsc{Salten, Felix} (06.09.1869 – 08.10.1945), \emph{Schriftsteller, Journalist, Chefredakteur}|pwk} traf.}}}\label{K_L03307-1h}. Ungefähr Dienstag{ }Abend oder Mittwoch{ }früh. Aber ich werde eher den Schluß der Parthie mitmachen, als den
               Anfang. Ich kann am 12. noch nicht von Ischl\oindex{Bad Ischl@\textbf{Bad Ischl}|pw} fort, weil Otti\pwindex{Salten, Ottilie 07.03.1868 – 22.06.1942@\textsc{Salten, Ottilie} (07.03.1868 – 22.06.1942), \emph{Schauspielerin}|pw} die Vorarlberg\oindex{Vorarlberg@\textbf{Vorarlberg}|pw}er Sache nicht
               mitmacht, sondern mich allein fahren läßt. So will ich doch bis 16. od. 17. bei ihr
               bleiben und dann direct nach {\pb}Schruns\oindex{Schruns@\textbf{Schruns}|pw} fahren. Ich dachte nicht, dass die
               Parthie schon so bald losgeht. Übrigens machen wir wol mündlich noch alles nähere
               aus.\pend
           \pstart
           Auf \label{K_L03307-2v}\edtext{Wiedersehen, vorraussichtlich in Ischl\oindex{Bad Ischl@\textbf{Bad Ischl}|pw}}{\lemma{\textnormal{\emph{Wiedersehen, … Ischl}}}\Cendnote{\textnormal{In Ischl\oindex{Bad Ischl@\textbf{Bad Ischl}|pwk} trafen sie sich nicht, weil sich Saltens\pwindex{Salten, Felix 06.09.1869 – 08.10.1945@\textsc{Salten, Felix} (06.09.1869 – 08.10.1945), \emph{Schriftsteller, Journalist, Chefredakteur}|pwk} Ankunft auf den 
                  14. 8. 1900 verzögerte und 
                  Schnitzler\pwindex{Schnitzler, Arthur 15.05.1862 – 21.10.1931@\textsc{Schnitzler, Arthur} (15.05.1862 – 21.10.1931), \emph{Schriftsteller, Mediziner}|pwk} den Ort am 10. 8. 1900 verließ.}}}\label{K_L03307-2h}.\pend
           \pstart
           Herzlichst Ihr {\\[\baselineskip]}\spacefill\mbox{Salten.}\pend
           \leftskip=0em{}\pstart
           \noindent{}Otti\pwindex{Salten, Ottilie 07.03.1868 – 22.06.1942@\textsc{Salten, Ottilie} (07.03.1868 – 22.06.1942), \emph{Schauspielerin}|pw} ist jetzt in Karlsbad\oindex{Karlsbad@\textbf{Karlsbad}|pw}.\pend
           
         
         \endnumbering\mylabel{h}\end{ledgroupsized}  \newcommand{\dateiname}{L03307}\newcommand{\titel}{Felix Salten an Arthur Schnitzler, 5. 8. 1900}\newcommand{\editorInnen}{Martin Anton Müller und Laura Untner}%% latex-leseansicht-abspann.tex
%% Abspann für die Leseansicht.
%% Der Schalter \ifkorrekturansicht ist bereits durch den Vorspann gesetzt.

%% latex-abspann.tex
%% Gemeinsamer Abspann für Korrekturansicht und Leseansicht.
%% Setzt den Schalter \ifkorrekturansicht voraus (gesetzt in den
%% einbindenden Dateien latex-korrekturansicht-abspann.tex bzw.
%% latex-leseansicht-abspann.tex).
%% ---------------------------------------------------------------

\normalsize

% Das esempio-Environment wird nur in der Leseansicht benötigt
\ifkorrekturansicht\else
\newenvironment{esempio}[3]%
{
    \vspace{1.5ex}
    \rlap{\underline{#1}}
    \par
    \setlength{\parindent}{0cm}
    \nopagebreak
    \leftskip=#2cm
    \rightskip=#3cm
}
{
    \par
}
\fi

\doendnotes{C}
\bigskip
\vfill

\clearpage

\footnotesize

\ifkorrekturansicht
  \lohead{\textsc{register}}
\fi

% theindex-Environment neu definieren ohne reledmac
\makeatletter
\renewenvironment{theindex}{%
  \ifkorrekturansicht
    \section*{\indexname}%
  \else
    \subsubsection*{Index der erwähnten Entitäten}%
  \fi
  \setlength{\parindent}{0pt}%
  \setlength{\parskip}{0pt plus 0.3pt}%
  \let\item\@idxitem
}{%
  \ifkorrekturansicht\clearpage\fi
}
\makeatother

\IfFileExists{\jobname-pw.ind}{\input{\jobname-pw.ind}}{}

% Quellenangabe nur in der Leseansicht
\ifkorrekturansicht\else
% Fallback-Definitionen, falls die .tex-Datei \titel etc. nicht gesetzt hat
\providecommand{\titel}{}
\providecommand{\editorInnen}{}
\providecommand{\dateiname}{\jobname}

\vspace{3cm}

\vfill

\footnotesize
\textsc{Quelle}: \titel. Herausgegeben von {\editorInnen}. In: \emph{Arthur Schnitzler: Briefwechsel mit Autorinnen und Autoren}.
 Digitale Edition, https://schnitzler-briefe.acdh.oeaw.ac.at/{\dateiname}.html (Stand \today)
\fi

\end{document}


      