%% latex-korrekturansicht-vorspann.tex
%% Vorspann für die Korrekturansicht.
%% Lädt die gemeinsame Datei latex-vorspann.tex mit gesetztem Schalter.

\newif\ifkorrekturansicht
\korrekturansichttrue

\input{../tex-inputs/latex-vorspann}


\section[ Felix Salten an Arthur Schnitzler, 5. 8. 1900]{L03307 Felix Salten an Arthur Schnitzler, 5. 8. 1900}
\nopagebreak\mylabel{L03307v}
\rehead{ }\normalsize\beginnumbering\briefempfaengerindex{Schnitzler, Arthur@\textsc{Schnitzler, Arthur}!zzzSalten, Felix@\emph{von Felix Salten}!1900-08-052@{5. 8. 1900}|(be}
\toendnotes[C]{\smallbreak\pagebreak[2]}\Standort{CUL, Schnitzler, B 89, A 2.}
\physDesc{Brief, 1 Blatt, 2 Seiten, 594 Zeichen
\newline{}Handschrift: Bleistift, lateinische Kurrent
\newline{}Ordnung: mit Bleistift von unbekannter Hand nummeriert: »131« }\toendnotes[C]{\smallbreak}
\pstart
           \raggedleft{}{\pb}Pressbaum\oindex{Pressbaum@\textbf{Pressbaum}, \emph{P.PPLA3}|pw}, 5./VIII. 00\pend
           \vspace{0.5em}
\pstart
           Lieber Freund, wahrscheinlich komme ich noch vor dem 10. nach \label{K_L03307-1v}\edtext{Ischl\oindex{Bad Ischl@\textbf{Bad Ischl}, \emph{P.PPL}|pw}}{\lemma{\textnormal{\emph{Ischl}}}\Cendnote{\textnormal{Am 17. 8. 1900 startete Schnitzler gemeinsam mit Richard Beer-Hofmann\pwindex{Beer-Hofmann, Richard 1866-07-11 – 1945-09-26@\textsc{Beer-Hofmann, Richard} (1866-07-11 – 1945-09-26), \emph{Schriftsteller/Schriftstellerin}|pwk}, Paul Goldmann\pwindex{Goldmann, Paul 31.01.1865 – 25.09.1935@\textsc{Goldmann, Paul} (31.01.1865 – 25.09.1935), \emph{Schriftsteller/Schriftstellerin, Journalist/Journalistin}|pwk}, Alfred Kerr\pwindex{Kerr, Alfred 25.12.1867 – 12.10.1948@\textsc{Kerr, Alfred} (25.12.1867 – 12.10.1948), \emph{Schriftsteller/Schriftstellerin, Kritiker/Kritikerin}|pwk} und Leo
                           Van-Jung\pwindex{Van-Jung, Leo 15.10.1866 – 02.07.1939@\textsc{Van-Jung, Leo} (15.10.1866 – 02.07.1939), \emph{Gesangspädagoge/Gesangspädagogin, Mathematiker/Mathematikerin}|pwk} eine Alpen\oindex{Alpen@\textbf{Alpen}, \emph{kein passender Code gefunden}|pwk}wanderung in Schruns\oindex{Schruns@\textbf{Schruns}, \emph{A.ADM3}|pwk} (Vorarlberg\oindex{Vorarlberg@\textbf{Vorarlberg}, \emph{Teil eines Landes (A.LNDX)}|pwk}). Am 28. 8. 1900 reiste Schnitzler alleine weiter nach Meran\oindex{Meran@\textbf{Meran}, \emph{P.PPLA3}|pwk}, wo er schließlich auf Salten\pwindex{Salten, Felix 06.09.1869 – 08.10.1945@\textsc{Salten, Felix} (06.09.1869 – 08.10.1945), \emph{Schriftsteller/Schriftstellerin, Journalist/Journalistin, Chefredakteur/Chefredakteurin}|pwk} traf.}}}\label{K_L03307-1}. Ungefähr Dienstag{ }Abend oder Mittwoch{ }früh. Aber ich werde eher den Schluß der Parthie mitmachen, als den
               Anfang. Ich kann am 12. noch nicht von Ischl\oindex{Bad Ischl@\textbf{Bad Ischl}, \emph{P.PPL}|pw} fort, weil Otti\pwindex{Salten, Ottilie 07.03.1868 – 22.06.1942@\textsc{Salten, Ottilie} (07.03.1868 – 22.06.1942), \emph{Schauspieler/Schauspielerin}|pw} die Vorarlberg\oindex{Vorarlberg@\textbf{Vorarlberg}, \emph{Teil eines Landes (A.LNDX)}|pw}er Sache nicht
               mitmacht, sondern mich allein fahren läßt. So will ich doch bis 16. od. 17. bei ihr
               bleiben und dann direct nach {\pb}Schruns\oindex{Schruns@\textbf{Schruns}, \emph{A.ADM3}|pw} fahren. Ich dachte nicht, dass die
               Parthie schon so bald losgeht. Übrigens machen wir wol mündlich noch alles nähere
               aus.\pend
           
\pstart
           Auf \label{K_L03307-2v}\edtext{Wiedersehen, vorraussichtlich in Ischl\oindex{Bad Ischl@\textbf{Bad Ischl}, \emph{P.PPL}|pw}}{\lemma{\textnormal{\emph{Wiedersehen, … Ischl}}}\Cendnote{\textnormal{In Ischl\oindex{Bad Ischl@\textbf{Bad Ischl}, \emph{P.PPL}|pwk} trafen sie sich nicht, weil sich Saltens\pwindex{Salten, Felix 06.09.1869 – 08.10.1945@\textsc{Salten, Felix} (06.09.1869 – 08.10.1945), \emph{Schriftsteller/Schriftstellerin, Journalist/Journalistin, Chefredakteur/Chefredakteurin}|pwk} Ankunft auf den 
                  14. 8. 1900 verzögerte und 
                  Schnitzler den Ort am 10. 8. 1900 verließ.}}}\label{K_L03307-2}.\pend
           
\pstart
           Herzlichst Ihr {\\[\baselineskip]}\spacefill\mbox{Salten.}\pend
           \leftskip=0em{}
\pstart
           \noindent{}Otti\pwindex{Salten, Ottilie 07.03.1868 – 22.06.1942@\textsc{Salten, Ottilie} (07.03.1868 – 22.06.1942), \emph{Schauspieler/Schauspielerin}|pw} ist jetzt in Karlsbad\oindex{Karlsbad@\textbf{Karlsbad}, \emph{P.PPLA}|pw}.\pend
           \selectlanguage{ngerman}\endnumbering\briefempfaengerindex{Schnitzler, Arthur@\textsc{Schnitzler, Arthur}!zzzSalten, Felix@\emph{von Felix Salten}!1900-08-052@{5. 8. 1900}|)be}\mylabel{L03307h}  \normalsize

\doendnotes{C}
\bigskip
\vfill

\clearpage

\footnotesize

\lohead{\textsc{register}}

% Definiere theindex-Environment komplett neu ohne reledmac
\makeatletter
\renewenvironment{theindex}{%
  \section*{\indexname}%
  \setlength{\parindent}{0pt}%
  \setlength{\parskip}{0pt plus 0.3pt}%
  \let\item\@idxitem
}{%
  \clearpage
}
\makeatother

\IfFileExists{\jobname-pw.ind}{\input{\jobname-pw.ind}}{}

\end{document}

      