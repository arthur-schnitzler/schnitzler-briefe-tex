%% latex-korrekturansicht-vorspann.tex
%% Vorspann für die Korrekturansicht.
%% Lädt die gemeinsame Datei latex-vorspann.tex mit gesetztem Schalter.

\newif\ifkorrekturansicht
\korrekturansichttrue

\input{../tex-inputs/latex-vorspann}


\section[Arthur Schnitzler an Stefan Zweig, 17. 10. 1904]{L03804 Arthur Schnitzler an Stefan Zweig, 17. 10. 1904}
\nopagebreak\mylabel{L03804v}
\rehead{ }\normalsize\beginnumbering\briefempfaengerindex{Zweig, Stefan@\textsc{Zweig, Stefan}!zzzSchnitzler, Arthur@\emph{von Arthur Schnitzler}!1904-10-171@{17. 10. 1904}|(be}
\toendnotes[C]{\smallbreak\pagebreak[2]}\Standort{Jerusalem, National Library of Israel, ARC. Ms. Var. 305 1 58 Stefan Zweig Collection.}
\physDesc{Kartenbrief, 1 Blatt, 2 Seiten, 259 Zeichen
\newline{}Handschrift: schwarze Tinte, deutsche Kurrent
\newline{}Versand: 1) Stempel: »\nobreak{}\oindex{XVIII., Waehring@\textbf{XVIII., Währing}, \emph{A.ADM3}|pwk}\textcolor{gray}{18/1 Wien} 110, 18. X. 04, XII\nobreak{}«.   2) Stempel: »\nobreak{}18. 10. 0\textcolor{gray}{4}, 3–4 ½, Bestellt\nobreak{}«. }\toendnotes[C]{\smallbreak}\pstart{}{\pb}Herrn \textsc{Dr Phil. Stefan Zweig}\pend{}\pstart{}Wien I\oindex{I., Innere Stadt@\textbf{I., Innere Stadt}, \emph{A.ADM3}|pw}\pend{}\pstart{}\textsc{Rathhaustraße 17}\oindex{Rathausstrasse 17@\textbf{Rathausstraße 17}, \emph{Wohngebäude (K.WHS)}|pw}.\pend{}{\bigskip}\vspace{1em}
\pstart
           \raggedleft{}{\pb}\textsc{XVIII Spoettelgasse 7\oindex{Edmund-Weiss-Gasse@\textbf{Edmund-Weiß-Gasse}, \emph{R.ST}|pw}}\pend
           
\pstart
           \raggedleft{}Wien\oindex{Wien@\textbf{Wien}, \emph{A.ADM2}|pw}, 17. 10 90\textcolor{gray}{4}\pend
           
\pstart{}verehrteſter Herr Doktor,\pend\vspace{0.5em}
\pstart
           ſchönſten Dank für die liebenswürdg Überſendg Ihres Novellenbuches\pwindex{Liebe der Erika Ewald. Novellen@\emph{Die Liebe der Erika Ewald. Novellen}|pwv}, auf deſſen \label{K_L03804-1v}\edtext{Lecture}{\lemma{\textnormal{\emph{Lecture}}}\Cendnote{\textnormal{Schnitzler
                  vermerkt \emph{Die Liebe der Erika Ewald}\pwindex{Liebe der Erika Ewald. Novellen@\emph{Die Liebe der Erika Ewald. Novellen}|pwk} in seiner Lektüreliste, 
                  siehe A. S.: \emph{Lektüren}, deutschsprachige Literatur.}}}\label{K_L03804-1} ich mich aufricht\textcolor{gray}{g} freue.\pend
           
\pstart
           Ihr Sie hochſchätzender{\\[\baselineskip]}\spacefill\mbox{Arthur Schnitzler.}\pend
           \leftskip=0em{}\selectlanguage{ngerman}\endnumbering\briefempfaengerindex{Zweig, Stefan@\textsc{Zweig, Stefan}!zzzSchnitzler, Arthur@\emph{von Arthur Schnitzler}!1904-10-171@{17. 10. 1904}|)be}\mylabel{L03804h}  \normalsize

\doendnotes{C}
\bigskip
\vfill

\clearpage

\footnotesize

\lohead{\textsc{register}}

% Definiere theindex-Environment komplett neu ohne reledmac
\makeatletter
\renewenvironment{theindex}{%
  \section*{\indexname}%
  \setlength{\parindent}{0pt}%
  \setlength{\parskip}{0pt plus 0.3pt}%
  \let\item\@idxitem
}{%
  \clearpage
}
\makeatother

\IfFileExists{\jobname-pw.ind}{\input{\jobname-pw.ind}}{}

\end{document}

      