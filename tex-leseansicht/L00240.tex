%% latex-leseansicht-vorspann.tex
%% Vorspann für die Leseansicht.
%% Lädt die gemeinsame Datei latex-vorspann.tex mit nicht gesetztem Schalter.

\newif\ifkorrekturansicht
\korrekturansichtfalse

\input{../tex-inputs/latex-vorspann}


         
         \newcommand{\erwaehntePersonen}{Personen:  ?? [Schreibkraft für Arthur Schnitzler], Ludwig Bauer, Richard Beer-Hofmann, Markus Benedict, Marianne Benedict, Bertha Flegmann, Carl Freund, Karoline Gribl, Markus Hajek, Josef Jarno, Henri Murger, Heinrich Osten, Felix Salten, Josefine Skura, Grethe Wreden}
         \newcommand{\erwaehnteInstitutionen}{Institutionen: Berliner Börsen-Courier}
         \newcommand{\erwaehnteOrte}{Orte: Bad Ischl, Burgring, Grillparzerstraße, Kahlenberg, Klosterneuburg, Salzburg, Schulgasse, Tulln an der Donau, Wien}
         \newcommand{\erwaehnteWerke}{Werke: Abschiedssouper, Aus Ischl, Das Kind, Der Tod Georgs, Die Presse, Illustrirtes Wiener Extrablatt, Ischler Brief, Ischler Wochenblatt, Wiener Allgemeine Zeitung, [Abschiedsouper in Ischl]}
               \section[Arthur Schnitzler an Richard Beer-Hofmann, 22. 7. 1893]{ Arthur Schnitzler an Richard Beer-Hofmann, 22. 7. 1893}\nopagebreak\mylabel{v}\rehead{ }\begin{ledgroupsized}[t]{13cm}\normalsize\beginnumbering \toendnotes[C]{\smallbreak\pagebreak[2]} \Standort{YCGL, MSS 31.}
\physDesc{Brief, 1 Blatt (Briefpapier mit Trauerrand), 4 Seiten, Umschlag mit Trauerrand
\newline{}Handschrift: schwarze Tinte, deutsche Kurrent\newline{}Versand: 1) Stempel: »\nobreak{}Wien 9/3, 22. 7. 93, 2–3 M\nobreak{}«.   2) Stempel: »\nobreak{}\oindex{Salzburg@\textbf{Salzburg}|pwk}Salzburg Stadt, 23 7 93, 2 N\nobreak{}«.  3) mit schwarzer Tinte von unbekannter Hand die beiden Adresszeilen
                                 gestrichen und ersetzt durch: »\noindent{}\textsc{Post Restante}{ / }\textsc{Salzburg\oindex{Salzburg@\textbf{Salzburg}|pw}}«}\buchAbdrucke{\weitereDrucke{Arthur Schnitzler, Richard Beer-Hofmann: \emph{Briefwechsel 1891–1931}. Hg. Konstanze Fliedl. Wien, Zürich: \emph{Europaverlag} 1992, S. 47.} }\toendnotes[C]{\smallbreak}\pstart{}{\pb}Herrn \textsc{Dr. Richard
                     Beer-Hofmann}\pend{}\pstart{}\textsc{Ischl\oindex{Bad Ischl@\textbf{Bad Ischl}|pw}}\pend{}\pstart{}\textsc{Schulgasse 8\oindex{Schulgasse@\textbf{Schulgasse}|pw}}.\pend{}{\bigskip}\pstart
           \raggedleft{}{\pb}Wien\oindex{Wien@\textbf{Wien}|pw}{ }22. 7. 93\pend
           \pstart{}Lieber Richard,\pend\pstart
           die Abſchrift Ihrer Novelle\pwindex{Beer-Hofmann, Richard 1866-07-11 – 1945-09-26@\textsc{Beer-Hofmann, Richard} (1866-07-11 – 1945-09-26), \emph{Schriftsteller}!Kind1893@\strich\emph{Das Kind} {[}1893{]}|pwv}
               dürfte Montag oder Dinſtag beendet \strikeout{wurde} werden, obwohl ſie erſt heute begonnen wird. Mein
               deſignirter Abſchreiber war ausgezogen – und ſchreibt nicht mehr; ein zweiter, den er
               mir empfahl, refuſirte gleichfalls und empfahl mir einen dritten\pwindex{?? [Schreibkraft fuer Arthur Schnitzler] @\textsc{?? [Schreibkraft für Arthur Schnitzler]}|pwv}, welcher heute bei mir war, einen
                  {\pb}guten Eindruck auf mich machte, u dem ich endlich
                  Das Kind\pwindex{Beer-Hofmann, Richard 1866-07-11 – 1945-09-26@\textsc{Beer-Hofmann, Richard} (1866-07-11 – 1945-09-26), \emph{Schriftsteller}!Kind1893@\strich\emph{Das Kind} {[}1893{]}|pw} übergab. –\pend
           \pstart
           War was\pwindex{Aus Ischl21. 7. 1893@\emph{Aus Ischl} {[}21. 7. 1893{]}|pwv} in der alten Preſſe\pwindex{?? Werk@Nicht ermittelte Verfasserinnen und Verfasser!Presse1848-07-03@\emph{Die Presse} {[}1848-07-03{]}|pw} über Abſch.\textsc{s}.\pwindex{Schnitzler, Arthur 15.05.1862 – 21.10.1931@\textsc{Schnitzler, Arthur} (15.05.1862 – 21.10.1931), \emph{Schriftsteller, Mediziner}!Abschiedssouper1892@\strich\emph{Abschiedssouper} {[}1892{]}|pw}? – Was ſagen Sie zu der Allgem. Zeitung\pwindex{?? Werk@Nicht ermittelte Verfasserinnen und Verfasser!Wiener Allgemeine Zeitung1.3.1880 – 11.2.1934@\emph{Wiener Allgemeine Zeitung} {[}1.3.1880 – 11.2.1934{]}|pw}\pwindex{Ischler Brief18. 07. 1893@\emph{Ischler Brief} {[}18. 07. 1893{]}|pwv}? Champagner – alſo \textsc{Murger}\pwindex{Murger, Henri 24.03.1822 – 28.01.1861@\textsc{Murger, Henri} (24.03.1822 – 28.01.1861), \emph{Schriftsteller}|pw} – weil ſie beim \textsc{Murger}\pwindex{Murger, Henri 24.03.1822 – 28.01.1861@\textsc{Murger, Henri} (24.03.1822 – 28.01.1861), \emph{Schriftsteller}|pw} verhungern. Soll ich mich bei \textsc{Osten}\pwindex{Osten, Heinrich 16.08.1855 – 01.08.1931@\textsc{Osten, Heinrich} (16.08.1855 – 01.08.1931), \emph{Schriftsteller, Journalist}|pw} bedanken? – War im \textsc{Börsencourier}\orgindex{Berliner Boersen-Courier@Berliner Börsen-Courier|pw} was? Den krieg’ ich auch nie zu Geſichte. –\pend
           \pstart
           Neulich machte ich mit \textsc{Salten}\pwindex{Salten, Felix 06.09.1869 – 08.10.1945@\textsc{Salten, Felix} (06.09.1869 – 08.10.1945), \emph{Schriftsteller, Journalist}|pw} eine wunderſchöne \textsc{Bicycletour} von \textsc{Klosterneubg}\oindex{Klosterneuburg@\textbf{Klosterneuburg}|pw} nach \textsc{Tulln}\oindex{Tulln an der Donau@\textbf{Tulln an der Donau}|pw}{ }{\pb}am Donauufer. Ihr
               müſſt unbedingt fahren lernen –\pend
           \pstart
           – Meine Sti{\geminationm}ung iſt recht ſchlecht; die Luft iſt
               drückend und unausſtehlich, und manche \textsc{Hypochondrien} quälen
               mich. Geſchrieben – noch nichts, die Zeit iſt ſo zerſplittert; ein ewiges Hin
                  un\textcolor{gray}{d} Her von der Klinik auf die Druckerei – in die Grillparzerſtr.\oindex{Grillparzerstrasse@\textbf{Grillparzerstraße}|pw} – auf den Burgring\oindex{Burgring@\textbf{Burgring}|pw} – zu meinem Schwager\pwindex{Hajek, Markus 25.11.1861 – 04.04.1941@\textsc{Hajek, Markus} (25.11.1861 – 04.04.1941), \emph{Mediziner, Laryngologe}|pwv} – auf den Kahlenberg\oindex{Kahlenberg@\textbf{Kahlenberg}|pw} u. ſ. w. –\pend
           \pstart
           Was gibts \substVorne{}\textsuperscript{aus}\substDazwischen{}in\substHinten{}{ }\textsc{Ischl}\oindex{Bad Ischl@\textbf{Bad Ischl}|pw}? – Sprachen {\pb}Sie Benedikt\pwindex{Benedict, Markus 17.09.1834 – 26.2.1909@\textsc{Benedict, Markus} (17.09.1834 – 26.2.1909), \emph{Industrieller}|pw}\pwindex{Benedict, Marianne 01.01.1848 – 12.05.1930@\textsc{Benedict, Marianne} (01.01.1848 – 12.05.1930)|pw}’s häufig? – Was macht der Götterliebling\pwindex{Beer-Hofmann, Richard 1866-07-11 – 1945-09-26@\textsc{Beer-Hofmann, Richard} (1866-07-11 – 1945-09-26), \emph{Schriftsteller}!Tod Georgs1900@\strich\emph{Der Tod Georgs} {[}1900{]}|pw}? – Hat Freund\pwindex{Freund, Carl @\textsc{Freund, Carl}, \emph{Verleger}|pw}{ }ſchon der \textsc{Fl.}\pwindex{Flegmann, Bertha 27.05.1852 – 24.6.1933@\textsc{Flegmann, Bertha} (27.05.1852 – 24.6.1933), \emph{Salonnière}|pw} geantwortet? – Wird noch viel über das Stück\pwindex{Schnitzler, Arthur 15.05.1862 – 21.10.1931@\textsc{Schnitzler, Arthur} (15.05.1862 – 21.10.1931), \emph{Schriftsteller, Mediziner}!Abschiedssouper1892@\strich\emph{Abschiedssouper} {[}1892{]}|pwv} geſchimpft? – Wirds noch einmal aufgeführt? –
               Sprechen Sie \textsc{Jarno}\pwindex{Jarno, Josef 24.08.1865 – 11.01.1932@\textsc{Jarno, Josef} (24.08.1865 – 11.01.1932), \emph{Theaterleiter, Schauspieler}|pw}? – Wie gehts der kleinen \textsc{Wreden}\pwindex{Wreden, Grethe @\textsc{Wreden, Grethe}, \emph{Schauspielerin}|pw}? – Sie werden allerdings keine Luſt haben, es zu erforſchen. – Iſt die \textsc{Griebl}\pwindex{Gribl, Karoline *~08.10.1867@\textsc{Gribl, Karoline} (*~08.10.1867), \emph{Schauspielerin}|pw} und die alte \textsc{Friese}\pwindex{Skura, Josefine 1841 – 1913@\textsc{Skura, Josefine} (1841 – 1913), \emph{Schauspielerin}|pw}{ }ſchon ins Kloſter gegangen?\pend
           \pstart
           Schreiben Sie bald, we{\geminationn} auch wenig\pend
           \pstart Herzlich Ihr \spacefill\mbox{ArthurSch}\pend{}\pstart
           \noindent{}\label{T_L00240_1v}\edtext{Senden Sie mir das Iſchler Wochenblatt\pwindex{?? Werk@Nicht ermittelte Verfasserinnen und Verfasser!Ischler Wochenblatt1876 – 1915@\emph{Ischler Wochenblatt} {[}1876 – 1915{]}|pw} mit der \label{K_L00240-44v}\edtext{Kritik}{\lemma{\textnormal{\emph{Kritik}}}\Cendnote{\textnormal{Im \emph{Ischler Wochenblatt}\pwindex{?? Werk@Nicht ermittelte Verfasserinnen und Verfasser!Ischler Wochenblatt1876 – 1915@\emph{Ischler Wochenblatt} {[}1876 – 1915{]}|pwk} erschien keine Kritik. Möglicherweise verwechselte Schnitzler\pwindex{Schnitzler, Arthur 15.05.1862 – 21.10.1931@\textsc{Schnitzler, Arthur} (15.05.1862 – 21.10.1931), \emph{Schriftsteller, Mediziner}|pwk}
                        es mit der Notiz\pwindex{Abschiedsouper in Ischl]18. 07. 1893@\emph{[Abschiedsouper in Ischl]} {[}18. 07. 1893{]}|pwkv} von Julius Bauer\pwindex{Bauer, Ludwig 05.09.1876 – 01.02.1935@\textsc{Bauer, Ludwig} (05.09.1876 – 01.02.1935), \emph{Schriftsteller, Journalist}|pwk}, von der Beer-Hofmann\pwindex{Beer-Hofmann, Richard 1866-07-11 – 1945-09-26@\textsc{Beer-Hofmann, Richard} (1866-07-11 – 1945-09-26), \emph{Schriftsteller}|pwk}
                        in seinem Brief vom 18. 7. 1893 sprach? (\emph{Illustriertes Wiener Extrablatt}\pwindex{Illustrirtes Wiener Extrablatt1872 – 1928@\emph{Illustrirtes Wiener Extrablatt} {[}1872 – 1928{]}|pwk}, Jg. 22, Nr. 196, 18. 7. 1893, S. 5.)}}}\label{K_L00240-44h}}{\lemma{\textnormal{\emph{Senden … Kritik}}}\Cendnote{\textnormal{auf der ersten Seite neben dem Datum
                     auf dem Kopf.}}}\label{T_L00240_1h}\pend
           
         
         \endnumbering\mylabel{h}\end{ledgroupsized}  \newcommand{\dateiname}{L00240}\newcommand{\titel}{Arthur Schnitzler an Richard Beer-Hofmann, 22. 7. 1893}\newcommand{\editorInnen}{Martin Anton Müller und Gerd-Hermann Susen}%% latex-leseansicht-abspann.tex
%% Abspann für die Leseansicht.
%% Der Schalter \ifkorrekturansicht ist bereits durch den Vorspann gesetzt.

%% latex-abspann.tex
%% Gemeinsamer Abspann für Korrekturansicht und Leseansicht.
%% Setzt den Schalter \ifkorrekturansicht voraus (gesetzt in den
%% einbindenden Dateien latex-korrekturansicht-abspann.tex bzw.
%% latex-leseansicht-abspann.tex).
%% ---------------------------------------------------------------

\normalsize

% Das esempio-Environment wird nur in der Leseansicht benötigt
\ifkorrekturansicht\else
\newenvironment{esempio}[3]%
{
    \vspace{1.5ex}
    \rlap{\underline{#1}}
    \par
    \setlength{\parindent}{0cm}
    \nopagebreak
    \leftskip=#2cm
    \rightskip=#3cm
}
{
    \par
}
\fi

\doendnotes{C}
\bigskip
\vfill

\clearpage

\footnotesize

\ifkorrekturansicht
  \lohead{\textsc{register}}
\fi

% theindex-Environment neu definieren ohne reledmac
\makeatletter
\renewenvironment{theindex}{%
  \ifkorrekturansicht
    \section*{\indexname}%
  \else
    \subsubsection*{Index der erwähnten Entitäten}%
  \fi
  \setlength{\parindent}{0pt}%
  \setlength{\parskip}{0pt plus 0.3pt}%
  \let\item\@idxitem
}{%
  \ifkorrekturansicht\clearpage\fi
}
\makeatother

\IfFileExists{\jobname-pw.ind}{\input{\jobname-pw.ind}}{}

% Quellenangabe nur in der Leseansicht
\ifkorrekturansicht\else
% Fallback-Definitionen, falls die .tex-Datei \titel etc. nicht gesetzt hat
\providecommand{\titel}{}
\providecommand{\editorInnen}{}
\providecommand{\dateiname}{\jobname}

\vspace{3cm}

\vfill

\footnotesize
\textsc{Quelle}: \titel. Herausgegeben von {\editorInnen}. In: \emph{Arthur Schnitzler: Briefwechsel mit Autorinnen und Autoren}.
 Digitale Edition, https://schnitzler-briefe.acdh.oeaw.ac.at/{\dateiname}.html (Stand \today)
\fi

\end{document}


      