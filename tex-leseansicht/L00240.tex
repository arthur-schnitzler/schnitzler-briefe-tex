%% latex-korrekturansicht-vorspann.tex
%% Vorspann für die Korrekturansicht.
%% Lädt die gemeinsame Datei latex-vorspann.tex mit gesetztem Schalter.

\newif\ifkorrekturansicht
\korrekturansichttrue

\input{../tex-inputs/latex-vorspann}


\section[Arthur Schnitzler an Richard Beer-Hofmann, 22. 7. 1893]{L00240 Arthur Schnitzler an Richard Beer-Hofmann, 22. 7. 1893}
\nopagebreak\mylabel{L00240v}
\rehead{ }\normalsize\beginnumbering\briefempfaengerindex{Beer-Hofmann, Richard@\textsc{Beer-Hofmann, Richard}!zzzSchnitzler, Arthur@\emph{von Arthur Schnitzler}!1893-07-221@{22. 7. 1893}|(be}
\toendnotes[C]{\smallbreak\pagebreak[2]}\Standort{YCGL, MSS 31.}
\physDesc{Brief, 1 Blatt, 4 Seiten, Umschlag, 1582 Zeichen (Umschlag und Briefpapier mit Trauerrand )
\newline{}Handschrift: schwarze Tinte, deutsche Kurrent
\newline{}Versand: 1) Stempel: »\nobreak{}\oindex{IX., Alsergrund@\textbf{IX., Alsergrund}, \emph{A.ADM3}|pwk}Wien 9/3, 22. 7. 93, 2–3 M\nobreak{}«.   2) Stempel: »\nobreak{}\oindex{Salzburg@\textbf{Salzburg}, \emph{A.ADM2}|pwk}Salzburg Stadt, 23 7 93, 2 N\nobreak{}«.  3) mit schwarzer Tinte von unbekannter Hand die beiden Adresszeilen gestrichen und
                                 ersetzt durch: »\noindent{}\textsc{Post Restante}{ / }\textsc{Salzburg\oindex{Salzburg@\textbf{Salzburg}, \emph{A.ADM2}|pw}}«}
\buchAbdrucke{\weitereDrucke{Arthur Schnitzler, Richard Beer-Hofmann: \emph{Briefwechsel 1891–1931}. Wien, Zürich: \emph{Europaverlag} 1992, S. 47.} }\toendnotes[C]{\smallbreak}\pstart{}{\pb}Herrn \textsc{Dr. Richard
                     Beer-Hofmann}\pend{}\pstart{}\textsc{Ischl\oindex{Bad Ischl@\textbf{Bad Ischl}, \emph{P.PPL}|pw}}\pend{}\pstart{}\textsc{Schulgasse 8\oindex{Schulgasse@\textbf{Schulgasse}, \emph{Straße (K.STR)}|pw}}.\pend{}{\bigskip}\vspace{1em}
\pstart
           \raggedleft{}{\pb}Wien\oindex{Wien@\textbf{Wien}, \emph{A.ADM2}|pw}{ }22. 7. 93\pend
           
\pstart{}Lieber Richard,\pend\vspace{0.5em}
\pstart
           die Abſchrift Ihrer Novelle\pwindex{Kind@\emph{Das Kind}|pwv}
               dürfte Montag oder Dinſtag beendet \strikeout{wurde} werden, obwohl ſie erſt heute begonnen wird. Mein
               deſignirter Abſchreiber war ausgezogen – und ſchreibt nicht mehr; ein zweiter, den er
               mir empfahl, refuſirte gleichfalls und empfahl mir einen dritten\pwindex{?? [Schreibkraft fuer Arthur Schnitzler] @\textsc{?? [Schreibkraft für Arthur Schnitzler]}|pwv}, welcher heute bei mir war, einen
                  {\pb}guten Eindruck auf mich machte, u dem ich endlich
                  Das Kind\pwindex{Kind@\emph{Das Kind}|pw} übergab. –\pend
           
\pstart
           War was\pwindex{Aus Ischl@\emph{Aus Ischl}|pwv} in der alten Preſſe\pwindex{Presse@\emph{Die Presse}|pw} über Abſch.\textsc{s}.\pwindex{Abschiedssouper@\emph{Abschiedssouper}|pw}? – Was ſagen Sie zu der Allgem. Zeitung\pwindex{Wiener Allgemeine Zeitung@\emph{Wiener Allgemeine Zeitung}|pw}\pwindex{Ischler Brief@\emph{Ischler Brief}|pwv}? Champagner – alſo \textsc{Murger}\pwindex{Murger, Henri 24.03.1822 – 28.01.1861@\textsc{Murger, Henri} (24.03.1822 – 28.01.1861), \emph{Schriftsteller/Schriftstellerin}|pw} – weil ſie beim \textsc{Murger}\pwindex{Murger, Henri 24.03.1822 – 28.01.1861@\textsc{Murger, Henri} (24.03.1822 – 28.01.1861), \emph{Schriftsteller/Schriftstellerin}|pw} verhungern. Soll ich mich bei \textsc{Osten}\pwindex{Osten, Heinrich 16.08.1855 – 01.08.1931@\textsc{Osten, Heinrich} (16.08.1855 – 01.08.1931), \emph{Schriftsteller/Schriftstellerin, Journalist/Journalistin}|pw} bedanken? – War im \textsc{Börsencourier}\orgindex{Berliner Boersen-Courier@Berliner Börsen-Courier|pw} was? Den krieg’ ich auch nie zu Geſichte. –\pend
           
\pstart
           Neulich machte ich mit \textsc{Salten}\pwindex{Salten, Felix 06.09.1869 – 08.10.1945@\textsc{Salten, Felix} (06.09.1869 – 08.10.1945), \emph{Schriftsteller/Schriftstellerin, Journalist/Journalistin, Chefredakteur/Chefredakteurin}|pw} eine wunderſchöne \textsc{Bicycletour} von \textsc{Klosterneubg}\oindex{Klosterneuburg@\textbf{Klosterneuburg}, \emph{P.PPLA3}|pw} nach \textsc{Tulln}\oindex{Tulln an der Donau@\textbf{Tulln an der Donau}, \emph{A.ADM3}|pw}{ }{\pb}am Donauufer\oindex{Donau@\textbf{Donau}, \emph{Fluss (N.FLS)}|pw}.
               Ihr müſſt unbedingt fahren lernen –\pend
           
\pstart
           – Meine Sti{\geminationm}ung iſt recht ſchlecht; die Luft iſt
               drückend und unausſtehlich, und manche \textsc{Hypochondrien} quälen
               mich. Geſchrieben – noch nichts, die Zeit iſt ſo zerſplittert; ein ewiges Hin
                  un\textcolor{gray}{d} Her von der Klinik auf die Druckerei – in die Grillparzerſtr.\oindex{Grillparzerstrasse@\textbf{Grillparzerstraße}, \emph{R.ST}|pw} – auf den \label{K_L00240-1v}\edtext{Burgring\oindex{Wohnung und Ordination Johann Schnitzler Burgring 1@\textbf{Wohnung und Ordination Johann Schnitzler Burgring 1}, \emph{Ordination}|pwv}}{\lemma{\textnormal{\emph{Burgring}}}\Cendnote{\textnormal{Schnitzler dürfte nach dem Tod seines Vaters\pwindex{Schnitzler, Johann 10.04.1835 – 02.05.1893@\textsc{Schnitzler, Johann} (10.04.1835 – 02.05.1893), \emph{Laryngologe/Laryngologin}|pwkv} dessen Ordination\oindex{Wohnung und Ordination Johann Schnitzler Burgring 1@\textbf{Wohnung und Ordination Johann Schnitzler Burgring 1}, \emph{Ordination}|pwkv} weiter betreut
                  haben.}}}\label{K_L00240-1} – zu meinem Schwager\pwindex{Hajek, Markus 25.11.1861 – 04.04.1941@\textsc{Hajek, Markus} (25.11.1861 – 04.04.1941), \emph{Mediziner/Medizinerin, Laryngologe/Laryngologin}|pwv} – auf den Kahlenberg\oindex{Kahlenberg@\textbf{Kahlenberg}, \emph{T.MT}|pw}
               u. ſ. w. –\pend
           
\pstart
           Was gibts \substVorne{}\textsuperscript{aus}\substDazwischen{}in\substHinten{}{ }\textsc{Ischl}\oindex{Bad Ischl@\textbf{Bad Ischl}, \emph{P.PPL}|pw}? – Sprachen {\pb}Sie Benedikt\pwindex{Benedict, Markus 17.09.1834 – 26.2.1909@\textsc{Benedict, Markus} (17.09.1834 – 26.2.1909), \emph{Industrieller/Industrielle}|pw}\pwindex{Benedict, Marianne 01.01.1848 – 12.05.1930@\textsc{Benedict, Marianne} (01.01.1848 – 12.05.1930)|pw}’s häufig? – Was macht der Götterliebling\pwindex{Tod Georgs@\emph{Der Tod Georgs}|pw}? – Hat Freund\pwindex{Freund, Carl @\textsc{Freund, Carl}, \emph{Verleger/Verlegerin}|pw}{ }ſchon der \textsc{Fl.}\pwindex{Flegmann, Bertha 27.05.1852 – 24.6.1933@\textsc{Flegmann, Bertha} (27.05.1852 – 24.6.1933), \emph{männliche Salonnière/Salonnière}|pw} geantwortet? – Wird noch viel über das Stück\pwindex{Abschiedssouper@\emph{Abschiedssouper}|pwv} geſchimpft? – Wirds noch einmal aufgeführt? –
               Sprechen Sie \textsc{Jarno}\pwindex{Jarno, Josef 24.08.1865 – 11.01.1932@\textsc{Jarno, Josef} (24.08.1865 – 11.01.1932), \emph{Theaterleiter/Theaterleiterin, Schauspieler/Schauspielerin}|pw}? – Wie gehts der kleinen \textsc{Wreden}\pwindex{Wreden, Grethe @\textsc{Wreden, Grethe}, \emph{Schauspieler/Schauspielerin}|pw}? – Sie werden allerdings keine Luſt haben, es zu erforſchen. – Iſt die \textsc{Griebl}\pwindex{Gribl, Karoline *~08.10.1867@\textsc{Gribl, Karoline} (*~08.10.1867), \emph{Schauspieler/Schauspielerin}|pw} und die alte \textsc{Friese}\pwindex{Skura, Josefine 1841 – 1913@\textsc{Skura, Josefine} (1841 – 1913), \emph{Schauspieler/Schauspielerin}|pw}{ }ſchon ins Kloſter gegangen?\pend
           
\pstart
           Schreiben Sie bald, we{\geminationn} auch wenig\pend
           \pstart Herzlich Ihr \spacefill\mbox{ArthurSch}\pend{}
\pstart
           \noindent{}\label{T_L00240-1v}\edtext{Senden Sie mir das Iſchler Wochenblatt\pwindex{Ischler Wochenblatt@\emph{Ischler Wochenblatt}|pw} mit der \label{K_L00240-2v}\edtext{Kritik}{\lemma{\textnormal{\emph{Kritik}}}\Cendnote{\textnormal{Im \emph{Ischler
                        Wochenblatt}\pwindex{Ischler Wochenblatt@\emph{Ischler Wochenblatt}|pwk} erschien keine Kritik. Möglicherweise verwechselte Schnitzler es mit der Notiz\pwindex{Abschiedsouper in Ischl]@\emph{[Abschiedsouper in Ischl]}|pwkv} von Julius Bauer\pwindex{Bauer, Ludwig 05.09.1876 – 01.02.1935@\textsc{Bauer, Ludwig} (05.09.1876 – 01.02.1935), \emph{Schriftsteller/Schriftstellerin, Journalist/Journalistin}|pwk}, von der Beer-Hofmann\pwindex{Beer-Hofmann, Richard 1866-07-11 – 1945-09-26@\textsc{Beer-Hofmann, Richard} (1866-07-11 – 1945-09-26), \emph{Schriftsteller/Schriftstellerin}|pwk} in seinem Brief vom 18. 7. 1893 sprach. (\emph{Illustrirtes Wiener Extrablatt}\pwindex{Illustrirtes Wiener Extrablatt@\emph{Illustrirtes Wiener Extrablatt}|pwk}, Jg. 22,
                        Nr. 196, 18. 7. 1893, S. 5.)}}}\label{K_L00240-2}}{\lemma{\textnormal{\emph{Senden … Kritik}}}\Cendnote{\textnormal{Auf der ersten Seite neben dem Datum
                     auf dem Kopf geschrieben.}}}\label{T_L00240-1}\pend
           \selectlanguage{ngerman}\endnumbering\briefempfaengerindex{Beer-Hofmann, Richard@\textsc{Beer-Hofmann, Richard}!zzzSchnitzler, Arthur@\emph{von Arthur Schnitzler}!1893-07-221@{22. 7. 1893}|)be}\mylabel{L00240h}  \normalsize

\doendnotes{C}
\bigskip
\vfill

\clearpage

\footnotesize

\lohead{\textsc{register}}

% Definiere theindex-Environment komplett neu ohne reledmac
\makeatletter
\renewenvironment{theindex}{%
  \section*{\indexname}%
  \setlength{\parindent}{0pt}%
  \setlength{\parskip}{0pt plus 0.3pt}%
  \let\item\@idxitem
}{%
  \clearpage
}
\makeatother

\IfFileExists{\jobname-pw.ind}{\input{\jobname-pw.ind}}{}

\end{document}

      