%% latex-leseansicht-vorspann.tex
%% Vorspann für die Leseansicht.
%% Lädt die gemeinsame Datei latex-vorspann.tex mit nicht gesetztem Schalter.

\newif\ifkorrekturansicht
\korrekturansichtfalse

\input{../tex-inputs/latex-vorspann}


\section[Arthur Schnitzler an Richard Beer-Hofmann, 22. 7. 1893]{L00240 Arthur Schnitzler an Richard Beer-Hofmann, 22. 7. 1893}
\nopagebreak\mylabel{L00240v}
\rehead{ }\normalsize\beginnumbering\briefempfaengerindex{Beer-Hofmann, Richard@\textsc{Beer-Hofmann, Richard}!zzzSchnitzler, Arthur@\emph{von Arthur Schnitzler}!1893-07-221@{22. 7. 1893}|(be}
\toendnotes[C]{\smallbreak\pagebreak[2]}
\correspDesc{Versand  durch Arthur Schnitzler am 22. 7. 1893 in Wien
\newline{}Umleitung  in Bad Ischl
\newline{}Erhalt  durch Richard Beer-Hofmann am 23. 7. 1893 in Salzburg}\toendnotes[C]{\smallbreak}
\Standort{YCGL, MSS 31.}
\physDesc{Brief, 1 Blatt, 4 Seiten, Kuvert, 1582 Zeichen (Umschlag und Briefpapier mit Trauerrand )
\newline{}Handschrift: schwarze Tinte, deutsche Kurrent
\newline{}Versand: 1) Stempel: »\nobreak{}\oindex{IX., Alsergrund@\textbf{IX., Alsergrund}, \emph{Verwaltungsgebiet}|pwk}Wien 9/3, 22. 7. 93, 2–3 M\nobreak{}«.   2) Stempel: »\nobreak{}\oindex{Salzburg@\textbf{Salzburg}, \emph{Verwaltungsgebiet}|pwk}Salzburg Stadt, 23 7 93, 2 N\nobreak{}«.  3) mit schwarzer Tinte von unbekannter Hand die beiden Adresszeilen gestrichen und
                                 ersetzt durch: »\noindent{}\textsc{Post Restante}{ / }\textsc{Salzburg\oindex{Salzburg@\textbf{Salzburg}, \emph{Verwaltungsgebiet}|pw}}«}
\buchAbdrucke{\weitereDrucke{Arthur Schnitzler, Richard Beer-Hofmann: \emph{Briefwechsel 1891–1931}. Herausgegeben von Konstanze Fliedl. Wien, Zürich: \emph{Europaverlag} 1992, S. 47.} }\toendnotes[C]{\smallbreak}\pstart{}{\pb}Herrn \textsc{Dr. Richard
                     Beer-Hofmann}\pend{}\pstart{}\textsc{Ischl\oindex{Bad Ischl@\textbf{Bad Ischl}|pw}}\pend{}\pstart{}\textsc{Schulgasse 8\oindex{Schulgasse@\textbf{Schulgasse}, \emph{Straße}|pw}}.\pend{}{\bigskip}\vspace{1em}
\pstart
           \raggedleft{}{\pb}Wien\oindex{Wien@\textbf{Wien}, \emph{Verwaltungsgebiet}|pw}{ }22. 7. 93\pend
           
\pstart{}Lieber Richard,\pend\vspace{0.5em}
\pstart
           die Abſchrift Ihrer Novelle\pwindex{Beer-Hofmann, Richard 11.\,7.\,1866 Wien – 26.\,9.\,1945 New York City@\textsc{Beer-Hofmann, Richard} (11.\,7.\,1866 Wien – 26.\,9.\,1945 New York City), \emph{Schriftsteller}!Kind@\strich\emph{Das Kind}|pwv}
               dürfte Montag oder Dinſtag beendet \strikeout{wurde} werden, obwohl{ }ſie erſt heute begonnen wird. Mein
               deſignirter Abſchreiber war ausgezogen – und{ }ſchreibt nicht mehr; ein zweiter, den er
               mir empfahl, refuſirte gleichfalls und empfahl mir einen dritten\pwindex{?? [Schreibkraft für Arthur Schnitzler] @\textsc{?? [Schreibkraft für Arthur Schnitzler]}|pwv}, welcher heute bei mir war, einen
                  {\pb}guten Eindruck auf mich machte, u dem ich endlich
                  Das Kind\pwindex{Beer-Hofmann, Richard 11.\,7.\,1866 Wien – 26.\,9.\,1945 New York City@\textsc{Beer-Hofmann, Richard} (11.\,7.\,1866 Wien – 26.\,9.\,1945 New York City), \emph{Schriftsteller}!Kind@\strich\emph{Das Kind}|pw} übergab. –\pend
           
\pstart
           War was\pwindex{\textcolor{red}{\textsuperscript{XXXX indx1}}!Aus Ischl@\strich\emph{Aus Ischl}|pwv} in der alten Preſſe\pwindex{Presse@\emph{Die Presse}|pw} über Abſch.\textsc{s}.\pwindex{Schnitzler, Arthur 15.\,5.\,1862 Wien – 21.\,10.\,1931 ebd.@\textsc{Schnitzler, Arthur} (15.\,5.\,1862 Wien – 21.\,10.\,1931 ebd.), \emph{Schriftsteller, Mediziner}!Abschiedssouper@\strich\emph{Abschiedssouper}|pw}? – Was{ }ſagen Sie zu der Allgem. Zeitung\pwindex{Wiener Allgemeine Zeitung@\emph{Wiener Allgemeine Zeitung}|pw}\pwindex{\textcolor{red}{\textsuperscript{XXXX indx1}}!Ischler Brief@\strich\emph{Ischler Brief}|pwv}? Champagner – alſo \textsc{Murger}\pwindex{Murger, Henri 24.\,3.\,1822 Paris – 28.\,1.\,1861 ebd.@\textsc{Murger, Henri} (24.\,3.\,1822 Paris – 28.\,1.\,1861 ebd.), \emph{Schriftsteller}|pw} – weil{ }ſie beim \textsc{Murger}\pwindex{Murger, Henri 24.\,3.\,1822 Paris – 28.\,1.\,1861 ebd.@\textsc{Murger, Henri} (24.\,3.\,1822 Paris – 28.\,1.\,1861 ebd.), \emph{Schriftsteller}|pw} verhungern. Soll ich mich bei \textsc{Osten}\pwindex{Osten, Heinrich 16.\,8.\,1855 Brody [Ukraine] – 1.\,8.\,1931 Wien@\textsc{Osten, Heinrich} (16.\,8.\,1855 Brody [Ukraine] – 1.\,8.\,1931 Wien), \emph{Schriftsteller, Journalist}|pw} bedanken? – War im \textsc{Börsencourier}\orgindex{Berliner Börsen-Courier@Berliner Börsen-Courier|pw} was? Den krieg’ ich auch nie zu Geſichte. –\pend
           
\pstart
           Neulich machte ich mit \textsc{Salten}\pwindex{Salten, Felix 6.\,9.\,1869 Budapest – 8.\,10.\,1945 Zürich@\textsc{Salten, Felix} (6.\,9.\,1869 Budapest – 8.\,10.\,1945 Zürich), \emph{Schriftsteller, Journalist, Chefredakteur}|pw} eine wunderſchöne \textsc{Bicycletour} von \textsc{Klosterneubg}\oindex{Klosterneuburg@\textbf{Klosterneuburg}, \emph{Hauptstadt}|pw} nach \textsc{Tulln}\oindex{Tulln an der Donau@\textbf{Tulln an der Donau}, \emph{Verwaltungsgebiet}|pw}{ }{\pb}am Donauufer\oindex{Donau@\textbf{Donau}, \emph{Fluss}|pw}.
               Ihr müſſt unbedingt fahren lernen –\pend
           
\pstart
           – Meine Sti{\geminationm}ung iſt recht{ }ſchlecht; die Luft iſt
               drückend und unausſtehlich, und manche \textsc{Hypochondrien} quälen
               mich. Geſchrieben – noch nichts, die Zeit iſt{ }ſo zerſplittert; ein ewiges Hin
                  un\textcolor{gray}{d} Her von der Klinik auf die Druckerei – in die Grillparzerſtr.\oindex{Wien@\textbf{Wien}!I., Innere Stadt@\textbf{I., Innere Stadt}!Grillparzerstraße@\textbf{Grillparzerstraße}, \emph{Straße}|pw} – auf den \label{K_L00240-1v}\edtext{Burgring\oindex{Wien@\textbf{Wien}!I., Innere Stadt@\textbf{I., Innere Stadt}!Wohnung und Ordination Johann Schnitzler Burgring 1@\textbf{Wohnung und Ordination Johann Schnitzler Burgring 1}, \emph{Ordination}|pwv}}{\lemma{\textnormal{\emph{Burgring}}}\Cendnote{\textnormal{Schnitzler dürfte nach dem Tod seines Vaters\pwindex{Schnitzler, Johann 10.\,4.\,1835 Nagykanizsa – 2.\,5.\,1893 Wien@\textsc{Schnitzler, Johann} (10.\,4.\,1835 Nagykanizsa – 2.\,5.\,1893 Wien), \emph{Laryngologe}|pwkv} dessen Ordination\oindex{Wien@\textbf{Wien}!I., Innere Stadt@\textbf{I., Innere Stadt}!Wohnung und Ordination Johann Schnitzler Burgring 1@\textbf{Wohnung und Ordination Johann Schnitzler Burgring 1}, \emph{Ordination}|pwkv} weiter betreut
                  haben.}}}\label{K_L00240-1} – zu meinem Schwager\pwindex{Hajek, Markus 25.\,11.\,1861 Vršac – 4.\,4.\,1941 London@\textsc{Hajek, Markus} (25.\,11.\,1861 Vršac – 4.\,4.\,1941 London), \emph{Mediziner, Laryngologe}|pwv} – auf den Kahlenberg\oindex{Wien@\textbf{Wien}!XIX., Döbling@\textbf{XIX., Döbling}!Kahlenberg@\textbf{Kahlenberg}, \emph{Berg}|pw}
               u. ſ. w. –\pend
           
\pstart
           Was gibts \substVorne{}\textsuperscript{aus}\substDazwischen{}in\substHinten{}{ }\textsc{Ischl}\oindex{Bad Ischl@\textbf{Bad Ischl}|pw}? – Sprachen {\pb}Sie Benedikt\pwindex{Benedict, Markus 17.\,9.\,1834 Mikulov – 26.\,2.\,1909 Kärntnerring 13@\textsc{Benedict, Markus} (17.\,9.\,1834 Mikulov – 26.\,2.\,1909 Kärntnerring 13), \emph{Industrieller}|pw}\pwindex{Benedict, Marianne 1.\,1.\,1848 Bratislava – 12.\,5.\,1930 Wien@\textsc{Benedict, Marianne} (1.\,1.\,1848 Bratislava – 12.\,5.\,1930 Wien)|pw}’s häufig? – Was macht der Götterliebling\pwindex{Beer-Hofmann, Richard 11.\,7.\,1866 Wien – 26.\,9.\,1945 New York City@\textsc{Beer-Hofmann, Richard} (11.\,7.\,1866 Wien – 26.\,9.\,1945 New York City), \emph{Schriftsteller}!Tod Georgs@\strich\emph{Der Tod Georgs}|pw}? – Hat Freund\pwindex{Freund, Carl @\textsc{Freund, Carl}, \emph{Verleger}|pw}{ }ſchon der \textsc{Fl.}\pwindex{Flegmann, Bertha 27.\,5.\,1852 Dubrovsky, Polen – 24.\,6.\,1933 Bad Ischl@\textsc{Flegmann, Bertha} (27.\,5.\,1852 Dubrovsky, Polen – 24.\,6.\,1933 Bad Ischl), \emph{Salonnière}|pw} geantwortet? – Wird noch viel über das Stück\pwindex{Schnitzler, Arthur 15.\,5.\,1862 Wien – 21.\,10.\,1931 ebd.@\textsc{Schnitzler, Arthur} (15.\,5.\,1862 Wien – 21.\,10.\,1931 ebd.), \emph{Schriftsteller, Mediziner}!Abschiedssouper@\strich\emph{Abschiedssouper}|pwv} geſchimpft? – Wirds noch einmal aufgeführt? –
               Sprechen Sie \textsc{Jarno}\pwindex{Jarno, Josef 24.\,8.\,1865 Budapest – 11.\,1.\,1932 Wien@\textsc{Jarno, Josef} (24.\,8.\,1865 Budapest – 11.\,1.\,1932 Wien), \emph{Theaterleiter, Schauspieler}|pw}? – Wie gehts der kleinen \textsc{Wreden}\pwindex{Wreden, Grethe @\textsc{Wreden, Grethe}, \emph{Schauspielerin}|pw}? – Sie werden allerdings keine Luſt haben, es zu erforſchen. – Iſt die \textsc{Griebl}\pwindex{Gribl, Karoline *~8.\,10.\,1867 Baden bei Wien@\textsc{Gribl, Karoline} (*~8.\,10.\,1867 Baden bei Wien), \emph{Schauspielerin}|pw} und die alte \textsc{Friese}\pwindex{Skura, Josefine 1841 – 1913@\textsc{Skura, Josefine} (1841 – 1913), \emph{Schauspielerin}|pw}{ }ſchon ins Kloſter gegangen?\pend
           
\pstart
           Schreiben Sie bald, we{\geminationn} auch wenig\pend
           \pstart Herzlich Ihr \spacefill\mbox{ArthurSch}\pend{}
\pstart
           \noindent{}\label{T_L00240-1v}\edtext{Senden Sie mir das Iſchler Wochenblatt\pwindex{Ischler Wochenblatt@\emph{Ischler Wochenblatt}|pw} mit der \label{K_L00240-2v}\edtext{Kritik}{\lemma{\textnormal{\emph{Kritik}}}\Cendnote{\textnormal{Im \emph{Ischler
                        Wochenblatt}\pwindex{Ischler Wochenblatt@\emph{Ischler Wochenblatt}|pwk} erschien keine Kritik. Möglicherweise verwechselte Schnitzler es mit der Notiz\pwindex{Abschiedsouper in Ischl]@\emph{[Abschiedsouper in Ischl]}|pwkv} von Julius Bauer\pwindex{Bauer, Ludwig 5.\,9.\,1876 Wien – 1.\,2.\,1935 Lugano@\textsc{Bauer, Ludwig} (5.\,9.\,1876 Wien – 1.\,2.\,1935 Lugano), \emph{Schriftsteller, Journalist}|pwk}, von der Beer-Hofmann\pwindex{Beer-Hofmann, Richard 11.\,7.\,1866 Wien – 26.\,9.\,1945 New York City@\textsc{Beer-Hofmann, Richard} (11.\,7.\,1866 Wien – 26.\,9.\,1945 New York City), \emph{Schriftsteller}|pwk} in seinem Brief vom XXXX Auszeichnungsfehler: Dokument L00237 nicht gefunden sprach. (\emph{Illustrirtes Wiener Extrablatt}\pwindex{Illustrirtes Wiener Extrablatt@\emph{Illustrirtes Wiener Extrablatt}|pwk}, Jg. 22,
                        Nr. 196, 18. 7. 1893, S. 5.)}}}\label{K_L00240-2}}{\lemma{\textnormal{\emph{Senden … Kritik}}}\Cendnote{\textnormal{Auf der ersten Seite neben dem Datum
                     auf dem Kopf geschrieben.}}}\label{T_L00240-1}\pend
           \selectlanguage{ngerman}\endnumbering\briefempfaengerindex{Beer-Hofmann, Richard@\textsc{Beer-Hofmann, Richard}!zzzSchnitzler, Arthur@\emph{von Arthur Schnitzler}!1893-07-221@{22. 7. 1893}|)be}\mylabel{L00240h}  \newcommand{\dateiname}{L00240}\newcommand{\titel}{Arthur Schnitzler an Richard Beer-Hofmann, 22. 7. 1893}\newcommand{\editorInnen}{Martin Anton Müller und Gerd-Hermann Susen}%% latex-leseansicht-abspann.tex
%% Abspann für die Leseansicht.
%% Der Schalter \ifkorrekturansicht ist bereits durch den Vorspann gesetzt.

%% latex-abspann.tex
%% Gemeinsamer Abspann für Korrekturansicht und Leseansicht.
%% Setzt den Schalter \ifkorrekturansicht voraus (gesetzt in den
%% einbindenden Dateien latex-korrekturansicht-abspann.tex bzw.
%% latex-leseansicht-abspann.tex).
%% ---------------------------------------------------------------

\normalsize

% Das esempio-Environment wird nur in der Leseansicht benötigt
\ifkorrekturansicht\else
\newenvironment{esempio}[3]%
{
    \vspace{1.5ex}
    \rlap{\underline{#1}}
    \par
    \setlength{\parindent}{0cm}
    \nopagebreak
    \leftskip=#2cm
    \rightskip=#3cm
}
{
    \par
}
\fi

\doendnotes{C}
\bigskip
\vfill

\clearpage

\footnotesize

\ifkorrekturansicht
  \lohead{\textsc{register}}
\fi

% theindex-Environment neu definieren ohne reledmac
\makeatletter
\renewenvironment{theindex}{%
  \ifkorrekturansicht
    \section*{\indexname}%
  \else
    \subsubsection*{Index der erwähnten Entitäten}%
  \fi
  \setlength{\parindent}{0pt}%
  \setlength{\parskip}{0pt plus 0.3pt}%
  \let\item\@idxitem
}{%
  \ifkorrekturansicht\clearpage\fi
}
\makeatother

\IfFileExists{\jobname-pw.ind}{\input{\jobname-pw.ind}}{}

% Quellenangabe nur in der Leseansicht
\ifkorrekturansicht\else
% Fallback-Definitionen, falls die .tex-Datei \titel etc. nicht gesetzt hat
\providecommand{\titel}{}
\providecommand{\editorInnen}{}
\providecommand{\dateiname}{\jobname}

\vspace{3cm}

\vfill

\footnotesize
\textsc{Quelle}: \titel. Herausgegeben von {\editorInnen}. In: \emph{Arthur Schnitzler: Briefwechsel mit Autorinnen und Autoren}.
 Digitale Edition, https://schnitzler-briefe.acdh.oeaw.ac.at/{\dateiname}.html (Stand \today)
\fi

\end{document}


