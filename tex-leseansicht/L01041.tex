%% latex-leseansicht-vorspann.tex
%% Vorspann für die Leseansicht.
%% Lädt die gemeinsame Datei latex-vorspann.tex mit nicht gesetztem Schalter.

\newif\ifkorrekturansicht
\korrekturansichtfalse

\input{../tex-inputs/latex-vorspann}


\section[Felix und Ottilie Salten und Arthur Schnitzler an Richard Beer-Hofmann, 26. 5. 1900]{L01041 Felix und Ottilie Salten und Arthur Schnitzler an Richard
               Beer-Hofmann, 26. 5. 1900}
\nopagebreak\mylabel{L01041v}
\rehead{ }\normalsize\beginnumbering\briefempfaengerindex{Beer-Hofmann, Richard@\textsc{Beer-Hofmann, Richard}!zzzSalten, Ottilie@\emph{von Ottilie Salten}!1900-05-261@{26. 5. 1900}|(be}\briefempfaengerindex{Beer-Hofmann, Richard@\textsc{Beer-Hofmann, Richard}!zzzSalten, Felix@\emph{von Felix Salten}!1900-05-261@{26. 5. 1900}|(be}\briefempfaengerindex{Beer-Hofmann, Richard@\textsc{Beer-Hofmann, Richard}!zzzSchnitzler, Arthur@\emph{von Arthur Schnitzler}!1900-05-261@{26. 5. 1900}|(be}
\toendnotes[C]{\smallbreak\pagebreak[2]}
\correspDesc{Versand  durch Arthur Schnitzler, Felix Salten, Ottilie Salten am 26. 5. 1900 in Puchberg am Schneeberg
\newline{}Erhalt  durch Richard Beer-Hofmann im Zeitraum [27. 5. 1900
                  – 31. 5. 1900?] in Wien}\toendnotes[C]{\smallbreak}
\Standort{YCGL, MSS 31.}
\physDesc{Bildpostkarte, 118 Zeichen
\newline{}Handschrift Arthur Schnitzler: Bleistift, deutsche Kurrent
\newline{}Handschrift Felix Salten: Bleistift, lateinische Kurrent (\noindent{}einschließlich der Anschrift)
\newline{}Handschrift Ottilie Salten: Bleistift
\newline{}Versand: Stempel: »\nobreak{}\oindex{Puchberg am Schneeberg@\textbf{Puchberg am Schneeberg}, \emph{Hauptstadt}|pwk}Puchberg am Schneeberge, 26. 5. 00\nobreak{}«.  }\pstart{}{\pb}Herrn \textsc{D\textsuperscript{r} Richard Beer-Hofmann}\pend{}\pstart{}Wien\oindex{Wien@\textbf{Wien}, \emph{Verwaltungsgebiet}|pw}\pend{}\pstart{}\textsc{I. Wollzeile 15\oindex{Wien@\textbf{Wien}!I., Innere Stadt@\textbf{I., Innere Stadt}!Wollzeile 15 (»Berthahof«)@\textbf{Wollzeile 15 (»Berthahof«)}, \emph{Wohngebäude}|pw}}\pend{}{\bigskip}
\pstart
           \noindent{}\centering{}{\pb}\textcolor{gray}{\textbf{Gruss vom Schneeberg\oindex{Schneeberg@\textbf{Schneeberg}, \emph{Berg}|pw}.}}\pend
           \vspace{1em}
\pstart
           {\pb}\spacefill\mbox{Salten}{\\[\baselineskip]}\spacefill\mbox{{[}hs. Salten:{]} Otti}\pend
           \leftskip=0em{}\selectlanguage{ngerman}\vspace{1em}
\pstart
           \noindent{}{[}hs. Schnitzler:{]} Puchberg\oindex{Puchberg am Schneeberg@\textbf{Puchberg am Schneeberg}, \emph{Hauptstadt}|pw}{ }ſehr hübſch, das Hotel angenehm. Viele
               Grüße\pend
           
\pstart
           Ihr \spacefill\mbox{Arthur.}{\\[\baselineskip]}\pend
           \leftskip=0em{}\selectlanguage{ngerman}\endnumbering\briefempfaengerindex{Beer-Hofmann, Richard@\textsc{Beer-Hofmann, Richard}!zzzSalten, Ottilie@\emph{von Ottilie Salten}!1900-05-261@{26. 5. 1900}|)be}\briefempfaengerindex{Beer-Hofmann, Richard@\textsc{Beer-Hofmann, Richard}!zzzSalten, Felix@\emph{von Felix Salten}!1900-05-261@{26. 5. 1900}|)be}\briefempfaengerindex{Beer-Hofmann, Richard@\textsc{Beer-Hofmann, Richard}!zzzSchnitzler, Arthur@\emph{von Arthur Schnitzler}!1900-05-261@{26. 5. 1900}|)be}\mylabel{L01041h}  \newcommand{\dateiname}{L01041}\newcommand{\titel}{Felix und Ottilie Salten und Arthur Schnitzler an Richard Beer-Hofmann, 26. 5. 1900}\newcommand{\editorInnen}{Martin Anton Müller und Gerd-Hermann Susen}%% latex-leseansicht-abspann.tex
%% Abspann für die Leseansicht.
%% Der Schalter \ifkorrekturansicht ist bereits durch den Vorspann gesetzt.

%% latex-abspann.tex
%% Gemeinsamer Abspann für Korrekturansicht und Leseansicht.
%% Setzt den Schalter \ifkorrekturansicht voraus (gesetzt in den
%% einbindenden Dateien latex-korrekturansicht-abspann.tex bzw.
%% latex-leseansicht-abspann.tex).
%% ---------------------------------------------------------------

\normalsize

% Das esempio-Environment wird nur in der Leseansicht benötigt
\ifkorrekturansicht\else
\newenvironment{esempio}[3]%
{
    \vspace{1.5ex}
    \rlap{\underline{#1}}
    \par
    \setlength{\parindent}{0cm}
    \nopagebreak
    \leftskip=#2cm
    \rightskip=#3cm
}
{
    \par
}
\fi

\doendnotes{C}
\bigskip
\vfill

\clearpage

\footnotesize

\ifkorrekturansicht
  \lohead{\textsc{register}}
\fi

% theindex-Environment neu definieren ohne reledmac
\makeatletter
\renewenvironment{theindex}{%
  \ifkorrekturansicht
    \section*{\indexname}%
  \else
    \subsubsection*{Index der erwähnten Entitäten}%
  \fi
  \setlength{\parindent}{0pt}%
  \setlength{\parskip}{0pt plus 0.3pt}%
  \let\item\@idxitem
}{%
  \ifkorrekturansicht\clearpage\fi
}
\makeatother

\IfFileExists{\jobname-pw.ind}{\input{\jobname-pw.ind}}{}

% Quellenangabe nur in der Leseansicht
\ifkorrekturansicht\else
% Fallback-Definitionen, falls die .tex-Datei \titel etc. nicht gesetzt hat
\providecommand{\titel}{}
\providecommand{\editorInnen}{}
\providecommand{\dateiname}{\jobname}

\vspace{3cm}

\vfill

\footnotesize
\textsc{Quelle}: \titel. Herausgegeben von {\editorInnen}. In: \emph{Arthur Schnitzler: Briefwechsel mit Autorinnen und Autoren}.
 Digitale Edition, https://schnitzler-briefe.acdh.oeaw.ac.at/{\dateiname}.html (Stand \today)
\fi

\end{document}


