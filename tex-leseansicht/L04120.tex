%% latex-leseansicht-vorspann.tex
%% Vorspann für die Leseansicht.
%% Lädt die gemeinsame Datei latex-vorspann.tex mit nicht gesetztem Schalter.

\newif\ifkorrekturansicht
\korrekturansichtfalse

\input{../tex-inputs/latex-vorspann}


\section[Arthur Schnitzler an Gustav Schwarzkopf, 23. 5. 1897]{L04120 Arthur Schnitzler an Gustav Schwarzkopf, 23. 5. 1897}
\nopagebreak\mylabel{L04120v}
\rehead{ }\normalsize\beginnumbering\briefempfaengerindex{Schwarzkopf, Gustav@\textsc{Schwarzkopf, Gustav}!zzzSchnitzler, Arthur@\emph{von Arthur Schnitzler}!1897-05-231@{23. 5. 1897}|(be}
\toendnotes[C]{\smallbreak\pagebreak[2]}
\correspDesc{Versand  durch Arthur Schnitzler am 23. 5. 1897 in Paris
\newline{}Erhalt  durch Gustav Schwarzkopf im Zeitraum [24. 5. 1897 – 28. 5. 1897?] in Wien}\toendnotes[C]{\smallbreak}
\Standort{CUL, Schnitzler, B 96.}
\physDesc{Brief, 1 Blatt, 4 Seiten, 684 Zeichen
\newline{}Handschrift: schwarze Tinte, deutsche Kurrent}
\buchAbdrucke{\weitereDrucke{Arthur Schnitzler: \emph{Briefe 1875–1912}. Herausgegeben von Therese Nickl und Heinrich Schnitzler. Frankfurt am Main: \emph{S. Fischer} 1981, S. 325–326.} }\toendnotes[C]{\smallbreak}
\pstart
           \raggedleft{}{\pb}23. 5. 97\pend
           
\pstart
           \raggedleft{}\textsc{Paris\oindex{Paris@\textbf{Paris}, \emph{Hauptstadt}|pw}}\pend
           \vspace{0.5em}
\pstart
           Lieber Guſtav, ich benütze den letzten Tag meines Pariſer\oindex{Paris@\textbf{Paris}, \emph{Hauptstadt}|pw} Aufenthalts dazu, um Ihnen für Ihre freundlichen
               \label{K_L04120-11v}\edtext{Geburtstagswünſche}{\lemma{\textnormal{\emph{Geburtstagswünsche}}}\Cendnote{\textnormal{{XXXX ref}. XXXX 14.5.1897}}}\label{K_L04120-11} zu danken – Sie ſind und bleiben eben, trotz \textsc{Karlweis\pwindex{Karlweis, Carl 23.\,11.\,1850 Wien – 27.\,10.\,1901 ebd.@\textsc{Karlweis, Carl} (23.\,11.\,1850 Wien – 27.\,10.\,1901 ebd.), \emph{Schriftsteller}|pw}}, der blutigſte Satyriker von Wien\oindex{Wien@\textbf{Wien}, \emph{Verwaltungsgebiet}|pw}! – Ich
               reiſe morgen {\pb}nach London\oindex{London@\textbf{London}, \emph{Hauptstadt}|pw}, will Ihnen aber nicht durch Angabe einer Adreſſe
               Verpflichtungen aufbürden und hoffe Sie ſehr bald, wohl in den erſten Junitagen,
               »heiter« und wohl in Wien\oindex{Wien@\textbf{Wien}, \emph{Verwaltungsgebiet}|pw} wiederzuſehen. Bis
               dahin ſeien Sie mir aufs herzlichſte {\pb}gegrüßt!\pend
           \pstart der Ihre \spacefill\mbox{ArthurSchn}\pend{}
\pstart
           \noindent{}Um Misdeutungen vorzubeugen: der oben bemerkte »letzte Pariſer\oindex{Paris@\textbf{Paris}, \emph{Hauptstadt}|pw} Tag« wurde auch noch anderweitg benutzt. Eben \label{K_L04120-1v}\edtext{ko{\geminationm}t Goldma{\geminationn}\pwindex{Goldmann, Paul 31.\,1.\,1865 Breslau – 25.\,9.\,1935 Wien@\textsc{Goldmann, Paul} (31.\,1.\,1865 Breslau – 25.\,9.\,1935 Wien), \emph{Schriftsteller, Journalist}|pw}}{\lemma{\textnormal{\emph{kommt Goldmann}}}\Cendnote{\textnormal{Vgl. A. S.: \emph{Tagebuch}, 23. 5. 1897.}}}\label{K_L04120-1}, der mich zu einem Ausflug abholt u mich
                  bittet {\pb}Sie beſtens von ihm zu
                  grüßen.\pend
           \selectlanguage{ngerman}\endnumbering\briefempfaengerindex{Schwarzkopf, Gustav@\textsc{Schwarzkopf, Gustav}!zzzSchnitzler, Arthur@\emph{von Arthur Schnitzler}!1897-05-231@{23. 5. 1897}|)be}\mylabel{L04120h}
\begin{anhang}
\end{anhang}\newcommand{\dateiname}{L04120}\newcommand{\titel}{Arthur Schnitzler an Gustav Schwarzkopf, 23. 5. 1897}\newcommand{\editorInnen}{Herausgegeben von Jahnke, SelmaMüller, Martin Anton}%% latex-leseansicht-abspann.tex
%% Abspann für die Leseansicht.
%% Der Schalter \ifkorrekturansicht ist bereits durch den Vorspann gesetzt.

%% latex-abspann.tex
%% Gemeinsamer Abspann für Korrekturansicht und Leseansicht.
%% Setzt den Schalter \ifkorrekturansicht voraus (gesetzt in den
%% einbindenden Dateien latex-korrekturansicht-abspann.tex bzw.
%% latex-leseansicht-abspann.tex).
%% ---------------------------------------------------------------

\normalsize

% Das esempio-Environment wird nur in der Leseansicht benötigt
\ifkorrekturansicht\else
\newenvironment{esempio}[3]%
{
    \vspace{1.5ex}
    \rlap{\underline{#1}}
    \par
    \setlength{\parindent}{0cm}
    \nopagebreak
    \leftskip=#2cm
    \rightskip=#3cm
}
{
    \par
}
\fi

\doendnotes{C}
\bigskip
\vfill

\clearpage

\footnotesize

\ifkorrekturansicht
  \lohead{\textsc{register}}
\fi

% theindex-Environment neu definieren ohne reledmac
\makeatletter
\renewenvironment{theindex}{%
  \ifkorrekturansicht
    \section*{\indexname}%
  \else
    \subsubsection*{Index der erwähnten Entitäten}%
  \fi
  \setlength{\parindent}{0pt}%
  \setlength{\parskip}{0pt plus 0.3pt}%
  \let\item\@idxitem
}{%
  \ifkorrekturansicht\clearpage\fi
}
\makeatother

\IfFileExists{\jobname-pw.ind}{\input{\jobname-pw.ind}}{}

% Quellenangabe nur in der Leseansicht
\ifkorrekturansicht\else
% Fallback-Definitionen, falls die .tex-Datei \titel etc. nicht gesetzt hat
\providecommand{\titel}{}
\providecommand{\editorInnen}{}
\providecommand{\dateiname}{\jobname}

\vspace{3cm}

\vfill

\footnotesize
\textsc{Quelle}: \titel. Herausgegeben von {\editorInnen}. In: \emph{Arthur Schnitzler: Briefwechsel mit Autorinnen und Autoren}.
 Digitale Edition, https://schnitzler-briefe.acdh.oeaw.ac.at/{\dateiname}.html (Stand \today)
\fi

\end{document}


