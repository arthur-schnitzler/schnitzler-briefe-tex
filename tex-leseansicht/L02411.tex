%% latex-leseansicht-vorspann.tex
%% Vorspann für die Leseansicht.
%% Lädt die gemeinsame Datei latex-vorspann.tex mit nicht gesetztem Schalter.

\newif\ifkorrekturansicht
\korrekturansichtfalse

\input{../tex-inputs/latex-vorspann}


         
         \newcommand{\erwaehntePersonen}{Personen: Paul Géraldy, Hedwig Heller}
         \newcommand{\erwaehnteOrte}{Orte: Sieveringer Straße, Wien}
         \newcommand{\erwaehnteWerke}{
               \section[Felix Braun an Arthur Schnitzler, 27. 3. 1924]{ Felix Braun an Arthur Schnitzler, 27. 3. 1924}\nopagebreak\mylabel{v}\rehead{ }\begin{ledgroupsized}[t]{13cm}\normalsize\beginnumbering \toendnotes[C]{\smallbreak\pagebreak[2]} \Standort{DLA, A:Schnitzler, HS.NZ85.1.2604,2.}
\physDesc{Briefkarte
\newline{}Handschrift: schwarze Tinte, deutsche Kurrent
\newline{}Schnitzler: 1) mit Bleistift beschriftet: »\textsc{Siestr. 191}\oindex{Sieveringer Strasse@\textbf{Sieveringer Straße}|pw}«  2) mit rotem Buntstift eine Unterstreichung}\pstart
           \centering{}{\pb}Wien\oindex{Wien@\textbf{Wien}|pw}, den 27. III. 1924\pend
           \pstart{}Verehrter Herr Doktor!\pend\pstart
           Erlauben Sie, daß ich Ihnen ein Dankwort ſchreibe für die große
                    Liebenswürdigkeit, mit der Sie, wie mir Frau Heller\pwindex{Heller, Hedwig 19.04.1881 – 16.05.1947@\textsc{Heller, Hedwig} (19.04.1881 – 16.05.1947)|pw} heute zu meiner Freude erzählte, als es ſich um die Zuweiſung
                    des \textsc{Paul Géraldy}\pwindex{Geraldy, Paul 06.03.1885 – 09.03.1983@\textsc{Géraldy, Paul} (06.03.1885 – 09.03.1983), \emph{Schriftsteller}|pw} beſtimmten Honorars an einen Wien\oindex{Wien@\textbf{Wien}|pw}er
                    Schriftſteller handelte, für mich eingetreten ſind. Es hat mich tief gerührt,
                    daß Sie es waren, der mir dieſe Ehrung zuerkannt hat. Seien Sie, verehrter Herr
                    Doktor, dafür von Herzen bedankt!\pend
           \pstart
           Mit beſter Empfehlung, in beſonderer Verehrung, Ihr{\\[\baselineskip]}\spacefill\mbox{Felix Braun.}\pend
           \leftskip=0em{}
         
         \endnumbering\mylabel{h}\end{ledgroupsized}  \newcommand{\dateiname}{L02411}\newcommand{\titel}{Felix Braun an Arthur Schnitzler, 27. 3. 1924}\newcommand{\editorInnen}{Martin Anton Müller und Gerd-Hermann Susen}%% latex-leseansicht-abspann.tex
%% Abspann für die Leseansicht.
%% Der Schalter \ifkorrekturansicht ist bereits durch den Vorspann gesetzt.

%% latex-abspann.tex
%% Gemeinsamer Abspann für Korrekturansicht und Leseansicht.
%% Setzt den Schalter \ifkorrekturansicht voraus (gesetzt in den
%% einbindenden Dateien latex-korrekturansicht-abspann.tex bzw.
%% latex-leseansicht-abspann.tex).
%% ---------------------------------------------------------------

\normalsize

% Das esempio-Environment wird nur in der Leseansicht benötigt
\ifkorrekturansicht\else
\newenvironment{esempio}[3]%
{
    \vspace{1.5ex}
    \rlap{\underline{#1}}
    \par
    \setlength{\parindent}{0cm}
    \nopagebreak
    \leftskip=#2cm
    \rightskip=#3cm
}
{
    \par
}
\fi

\doendnotes{C}
\bigskip
\vfill

\clearpage

\footnotesize

\ifkorrekturansicht
  \lohead{\textsc{register}}
\fi

% theindex-Environment neu definieren ohne reledmac
\makeatletter
\renewenvironment{theindex}{%
  \ifkorrekturansicht
    \section*{\indexname}%
  \else
    \subsubsection*{Index der erwähnten Entitäten}%
  \fi
  \setlength{\parindent}{0pt}%
  \setlength{\parskip}{0pt plus 0.3pt}%
  \let\item\@idxitem
}{%
  \ifkorrekturansicht\clearpage\fi
}
\makeatother

\IfFileExists{\jobname-pw.ind}{\input{\jobname-pw.ind}}{}

% Quellenangabe nur in der Leseansicht
\ifkorrekturansicht\else
% Fallback-Definitionen, falls die .tex-Datei \titel etc. nicht gesetzt hat
\providecommand{\titel}{}
\providecommand{\editorInnen}{}
\providecommand{\dateiname}{\jobname}

\vspace{3cm}

\vfill

\footnotesize
\textsc{Quelle}: \titel. Herausgegeben von {\editorInnen}. In: \emph{Arthur Schnitzler: Briefwechsel mit Autorinnen und Autoren}.
 Digitale Edition, https://schnitzler-briefe.acdh.oeaw.ac.at/{\dateiname}.html (Stand \today)
\fi

\end{document}


      