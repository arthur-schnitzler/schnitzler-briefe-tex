%% latex-korrekturansicht-vorspann.tex
%% Vorspann für die Korrekturansicht.
%% Lädt die gemeinsame Datei latex-vorspann.tex mit gesetztem Schalter.

\newif\ifkorrekturansicht
\korrekturansichttrue

\input{../tex-inputs/latex-vorspann}


\section[Lou Andreas-Salomé an Arthur Schnitzler, 22. 11. 1895]{L00514 Lou Andreas-Salomé an Arthur Schnitzler, 22. 11. 1895}
\nopagebreak\mylabel{L00514v}
\rehead{ }\normalsize\beginnumbering\briefempfaengerindex{Schnitzler, Arthur@\textsc{Schnitzler, Arthur}!zzzAndreas-Salome, Lou@\emph{von Lou Andreas-Salomé}!1895-11-221@{22. 11. 1895}|(be}
\toendnotes[C]{\smallbreak\pagebreak[2]}\Standort{CUL, Schnitzler, B 3.}
\physDesc{Postkarte, 400 Zeichen
\newline{}Handschrift: schwarze Tinte, deutsche Kurrent
\newline{}Versand: 1) Stempel: »\nobreak{}\oindex{I., Innere Stadt@\textbf{I., Innere Stadt}, \emph{A.ADM3}|pwk}Wien 1/1, 22. 11. 95, 11–12 N\nobreak{}«.   2) Stempel: »\nobreak{}\oindex{IX., Alsergrund@\textbf{IX., Alsergrund}, \emph{A.ADM3}|pwk}Wien 9/3, 23. 11. 95, 8.V, Bestellt\nobreak{}«. 
\newline{}Schnitzler: mit rotem Buntstift eine Unterstreichung 
\newline{}Ordnung: mit rotem Buntstift von unbekannter Hand nummeriert: »7« }\pstart{}{\pb}Herrn \textsc{D\textsuperscript{r}}\pend{}\pstart{}\textsc{Arthur Schnitzler}\pend{}\pstart{}\textsc{Wien\oindex{Wien@\textbf{Wien}, \emph{A.ADM2}|pw}}\pend{}\pstart{}Frankgasse 1\oindex{Frankgasse 1@\textbf{Frankgasse 1}, \emph{Wohngebäude (K.WHS)}|pw}. \pend{}{\bigskip}\vspace{1em}
\pstart
           \noindent{}{\pb}Lieber Herr \textsc{D\textsuperscript{r}}, wahrſcheinlich gehe ich morgen (Sonnabend) in Ihrer Sprechſtunde bei Ihnen
               vor. Ich habe den Auftrag bekommen, ſo ſchnell als möglich ein Exemplar
               der »Liebelei\pwindex{Liebelei. Schauspiel in drei Akten@\emph{Liebelei. Schauspiel in drei Akten}|pw}« behufs einer däniſchen\oindex{Daenemark@\textbf{Dänemark}, \emph{A.PCLI}|pw} Uebersetzung nach Kopenhagen\oindex{Kopenhagen@\textbf{Kopenhagen}, \emph{P.PPLC}|pw} zu ſenden und möchte Sie deswegen ſprechen. Seit geſtern bin ich,
               zusammen mit Frieda von Bülow\pwindex{Buelow, Frieda von 12.10.1857 – 12.03.1909@\textsc{Bülow, Frieda von} (12.10.1857 – 12.03.1909), \emph{Schriftsteller/Schriftstellerin}|pw}, im Hôtel \textsc{Royal}\oindex{Hotel Royal@\textbf{Hotel Royal}, \emph{Hotel (K.HTL)}|pw}.\pend
           
\pstart
           Mit herzlichem Gruß{\\[\baselineskip]}\spacefill\mbox{LouAS.}\pend
           \leftskip=0em{}\selectlanguage{ngerman}\endnumbering\briefempfaengerindex{Schnitzler, Arthur@\textsc{Schnitzler, Arthur}!zzzAndreas-Salome, Lou@\emph{von Lou Andreas-Salomé}!1895-11-221@{22. 11. 1895}|)be}\mylabel{L00514h}  \normalsize

\doendnotes{C}
\bigskip
\vfill

\clearpage

\footnotesize

\lohead{\textsc{register}}

% Definiere theindex-Environment komplett neu ohne reledmac
\makeatletter
\renewenvironment{theindex}{%
  \section*{\indexname}%
  \setlength{\parindent}{0pt}%
  \setlength{\parskip}{0pt plus 0.3pt}%
  \let\item\@idxitem
}{%
  \clearpage
}
\makeatother

\IfFileExists{\jobname-pw.ind}{\input{\jobname-pw.ind}}{}

\end{document}

      