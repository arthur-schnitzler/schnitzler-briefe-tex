\input{../tex-inputs/latex-pdf-vorspann}
\begin{center}
            \textcolor{red}{ENTWURF. ENTZIFFERUNG NOCH NICHT KORREKTURGELESEN}
                      \end{center}
            
               \section[Lou Andreas-Salomé an Arthur Schnitzler, 22. 11. 1895]{ Lou Andreas-Salomé an Arthur Schnitzler, 22. 11. 1895}\nopagebreak\mylabel{v}\rehead{ }\begin{ledgroupsized}[t]{13cm}\normalsize\beginnumbering\briefempfaengerindex{Schnitzler, Arthur@\textsc{Schnitzler, Arthur}!zzzAndreas-Salome, Lou@\emph{von Lou Andreas-Salomé}!1895-11-221@{22. 11. 1895}|(be} \toendnotes[C]{\smallbreak\pagebreak[2]} \Standort{CUL, Schnitzler, B 3.}
\physDesc{Postkarte
\newline{}Handschrift: schwarze Tinte, deutsche Kurrent\newline{}Versand: 1) Stempel: »\nobreak{}\oindex{I., Innere Stadt@\textbf{I., Innere Stadt}|pwk}Wien 1/1, 22. 11. 95, 11–12 N\nobreak{}«.  2) Stempel: »\nobreak{}\oindex{IX., Alsergrund@\textbf{IX., Alsergrund}|pwk}Wien 9/3, 23. 11. 95, 8.V, Bestellt\nobreak{}«. 
\newline{}Schnitzler: mit rotem Buntstift eine Unterstreichung \newline{}Ordnung: mit rotem Buntstift von unbekannter Hand
                                    nummeriert: »7« }\pstart{}{\pb}Herrn \textsc{D\textsuperscript{r}}\pend{}\pstart{}\textsc{Arthur Schnitzler}\pend{}\pstart{}\textsc{Wien\oindex{Wien@\textbf{Wien}|pw}}\pend{}\pstart{}Frankgasse 1\oindex{Frankgasse@\textbf{Frankgasse}|pw}.
                    \pend{}{\bigskip}\pstart
           \noindent{}{\pb}Lieber Herr \textsc{D\textsuperscript{r}}, wahrſcheinlich gehe ich morgen (Sonnabend) in Ihrer Sprechſtunde bei
                    Ihnen vor. Ich habe den Auftrag bekommen, ſo ſchnell als möglich
                    ein Exemplar der »Liebelei\pwindex{Schnitzler, Arthur 15.05.1862 – 21.10.1931@\textsc{Schnitzler, Arthur} (15.05.1862 – 21.10.1931), \emph{Schriftsteller, Mediziner}!Liebelei. Schauspiel in drei Akten9. 10. 1895@\strich\emph{Liebelei. Schauspiel in drei Akten} {[}9. 10. 1895{]}|pw}« behufs einer
                        däniſchen\oindex{Daenemark@\textbf{Dänemark}|pw} Uebersetzung nach Kopenhagen\oindex{Kopenhagen@\textbf{Kopenhagen}|pw}
                    zu ſenden und möchte Sie deswegen ſprechen. Seit geſtern bin ich, zusammen mit
                        Frieda von Bülow\pwindex{Buelow, Frieda von 12.10.1857 – 12.03.1909@\textsc{Bülow, Frieda von} (12.10.1857 – 12.03.1909), \emph{Schriftstellerin}|pw}, im Hôtel \textsc{Royal}\oindex{Hotel Royal@\textbf{Hotel Royal}|pw}.\pend
           \pstart
           Mit herzlichem Gruß{\\[\baselineskip]}\spacefill\mbox{LouAS.}\pend
           \leftskip=0em{}\endnumbering\briefempfaengerindex{Schnitzler, Arthur@\textsc{Schnitzler, Arthur}!zzzAndreas-Salome, Lou@\emph{von Lou Andreas-Salomé}!1895-11-221@{22. 11. 1895}|)be}\mylabel{h}\end{ledgroupsized}  \newcommand{\dateiname}{L00514}\newcommand{\titel}{Lou Andreas-Salomé an Arthur Schnitzler, 22. 11. 1895}\newcommand{\editorInnen}{Martin Anton Müller und Gerd-Hermann Susen}\input{../tex-inputs/latex-pdf-abspann}
      