%% latex-korrekturansicht-vorspann.tex
%% Vorspann für die Korrekturansicht.
%% Lädt die gemeinsame Datei latex-vorspann.tex mit gesetztem Schalter.

\newif\ifkorrekturansicht
\korrekturansichttrue

\input{../tex-inputs/latex-vorspann}


\section[ Paul Goldmann an Arthur Schnitzler, 7. 10. {[}1901{]}]{L03088 Paul Goldmann an Arthur Schnitzler, 7. 10. {[}1901{]}}
\nopagebreak\mylabel{L03088v}
\rehead{ }\normalsize\beginnumbering\briefempfaengerindex{Schnitzler, Arthur@\textsc{Schnitzler, Arthur}!zzzGoldmann, Paul@\emph{von Paul Goldmann}!1901-10-072@{7. 10. {[}1901{]}}|(be}
\toendnotes[C]{\smallbreak\pagebreak[2]}\Standort{DLA, A:Schnitzler, HS.NZ85.1.3171.}
\physDesc{Brief, 1 Blatt, 4 Seiten, 2010 Zeichen
\newline{}Handschrift: blaue Tinte, deutsche Kurrent
\newline{}Schnitzler: 1) mit Bleistift das Jahr »901« vermerkt  2) mit rotem Buntstift acht Unterstreichungen}\toendnotes[C]{\smallbreak}
\pstart
           \raggedleft{}{\pb}\textcolor{gray}{\textbf{DESSAUERSTRASSE 19}}\oindex{Dessauer Strasse@\textbf{Dessauer Straße}, \emph{Straße (K.STR)}|pw}\pend
           
\pstart
           Berlin\oindex{Berlin@\textbf{Berlin}, \emph{P.PPLC}|pw}, 7. Oktober.\pend
           
\pstart\center{}Mein lieber Freund,\pend\vspace{0.5em}
\pstart
           Dein Brief iſt im Ganzen recht erfreulich, – mit Ausnahme von Kopfſchmerzen und
                  \label{K_L03088-1v}\edtext{Ohrenklingen}{\lemma{\textnormal{\emph{Ohrenklingen}}}\Cendnote{\textnormal{Schnitzler litt seit
                     Herbst 1896 an Otosklerose – einer Verknöcherung des Innenohrs mit
                  zunehmender Schwerhörigkeit.}}}\label{K_L03088-1}, gegen die ich Dir leider nicht helfen kann.
               Das ſpielt in Deinem Leben offenbar dieſelbe Rolle, wie \label{K_L03088-2v}\edtext{\textsc{Benedikt\pwindex{Benedikt, Moriz 27.05.1849 – 18.03.1920@\textsc{Benedikt, Moriz} (27.05.1849 – 18.03.1920), \emph{Journalist/Journalistin, Herausgeber/Herausgeberin}|pw}}}{\lemma{\textnormal{\emph{Benedikt}}}\Cendnote{\textnormal{Moriz Benedikt\pwindex{Benedikt, Moriz 27.05.1849 – 18.03.1920@\textsc{Benedikt, Moriz} (27.05.1849 – 18.03.1920), \emph{Journalist/Journalistin, Herausgeber/Herausgeberin}|pwk} war als Herausgeber der \emph{Neuen Freien Presse}\pwindex{Neue Freie Presse@\emph{Neue Freie Presse}|pwk}{ }Goldmanns\pwindex{Goldmann, Paul 31.01.1865 – 25.09.1935@\textsc{Goldmann, Paul} (31.01.1865 – 25.09.1935), \emph{Schriftsteller/Schriftstellerin, Journalist/Journalistin}|pwk} Vorgesetzter.}}}\label{K_L03088-2} in dem
               meinen. Es ſcheint, daß zu jedem Leben ein wenig \textsc{Benedikt\pwindex{Benedikt, Moriz 27.05.1849 – 18.03.1920@\textsc{Benedikt, Moriz} (27.05.1849 – 18.03.1920), \emph{Journalist/Journalistin, Herausgeber/Herausgeberin}|pw}} gehört.\pend
           
\pstart
           Gegen ein \label{K_L03088-3v}\edtext{Auftreten \textsc{Olgas\pwindex{Schnitzler, Olga 17.01.1882 – 13.01.1970@\textsc{Schnitzler, Olga} (17.01.1882 – 13.01.1970), \emph{Schauspieler/Schauspielerin, Sänger/Sängerin}|pw}} bei \textsc{Salten\pwindex{Salten, Felix 06.09.1869 – 08.10.1945@\textsc{Salten, Felix} (06.09.1869 – 08.10.1945), \emph{Schriftsteller/Schriftstellerin, Journalist/Journalistin, Chefredakteur/Chefredakteurin}|pw}\orgindex{Jung-Wiener Theater zum Lieben Augustin@Jung-Wiener Theater zum Lieben Augustin|pwv}}}{\lemma{\textnormal{\emph{Auftreten … Salten}}}\Cendnote{\textnormal{Olga\pwindex{Schnitzler, Olga 17.01.1882 – 13.01.1970@\textsc{Schnitzler, Olga} (17.01.1882 – 13.01.1970), \emph{Schauspieler/Schauspielerin, Sänger/Sängerin}|pwk} trat nicht im Jung-Wiener Theater zum lieben Augustin\orgindex{Jung-Wiener Theater zum Lieben Augustin@Jung-Wiener Theater zum Lieben Augustin|pwkv}
                  auf.}}}\label{K_L03088-3} wäre ich entſchieden. Soll ihr für alle Zeiten die \strikeout{\textcolor{gray}{×}}{ }\label{K_L03088-4v}\edtext{Überbrettl\orgindex{Ueberbrettl@Überbrettl|pwv}}{\lemma{\textnormal{\emph{Überbrettl}}}\Cendnote{\textnormal{Vorbild für das Kabarett \emph{Jung-Wiener Theater zum Lieben Augustin}\orgindex{Jung-Wiener Theater zum Lieben Augustin@Jung-Wiener Theater zum Lieben Augustin|pwk} war das \emph{Überbrettl}\orgindex{Ueberbrettl@Überbrettl|pwk}, siehe Paul Goldmann an Arthur Schnitzler, 18. 2. [1901]. }}}\label{K_L03088-4}-Marke aufgeprägt werden? Das Programm
               der \textsc{Saltenschen\pwindex{Salten, Felix 06.09.1869 – 08.10.1945@\textsc{Salten, Felix} (06.09.1869 – 08.10.1945), \emph{Schriftsteller/Schriftstellerin, Journalist/Journalistin, Chefredakteur/Chefredakteurin}|pw}}{ }Unternehmung\orgindex{Jung-Wiener Theater zum Lieben Augustin@Jung-Wiener Theater zum Lieben Augustin|pwv}, das ich heut in der N. Fr.
                  Pr.\pwindex{Neue Freie Presse@\emph{Neue Freie Presse}|pw}{ }\label{K_L03088-5v}\edtext{leſe\pwindex{Theater- und Kunstnachrichten [zur Eroeffnung des Jung-Wiener Theaters zum lieben Augustin]@\emph{Theater- und Kunstnachrichten [zur Eröffnung des Jung-Wiener Theaters zum lieben Augustin]}|pwv}}{\lemma{\textnormal{\emph{leſe}}}\Cendnote{\textnormal{[O. V.]: \emph{Theater- und Kunstnachrichten.
                        [Zur Eröffnung des Jung-Wiener Theaters zum lieben Augustin]}\pwindex{Theater- und Kunstnachrichten [zur Eroeffnung des Jung-Wiener Theaters zum lieben Augustin]@\emph{Theater- und Kunstnachrichten [zur Eröffnung des Jung-Wiener Theaters zum lieben Augustin]}|pwk}. In: \emph{Neue Freie Presse}\pwindex{Neue Freie Presse@\emph{Neue Freie Presse}|pwk}, Nr. 13.332, 6. 10. 1901, S. 8.}}}\label{K_L03088-5}, iſt ein großer
               Kuddelmuddel. Der Mann\pwindex{Salten, Felix 06.09.1869 – 08.10.1945@\textsc{Salten, Felix} (06.09.1869 – 08.10.1945), \emph{Schriftsteller/Schriftstellerin, Journalist/Journalistin, Chefredakteur/Chefredakteurin}|pwv}
               ſcheint \strikeout{ab\textcolor{gray}{ſ}o} abſolut nicht zu
               wiſſen, was er will.\pend
           
\pstart
           »Lebendige Stunden\pwindex{Lebendige Stunden. Vier Einakter@\emph{Lebendige Stunden. Vier Einakter}|pw}« iſt ein hübſcher Titel. {\pb}Aber er ſagt mir nichts. \label{K_L03088-6v}\edtext{Warum »lebendig«? Warum »Stunden«?}{\lemma{\textnormal{\emph{Warum … »Stunden«?}}}\Cendnote{\textnormal{Schnitzler verwendete den Titel des
                  Einakters \emph{Lebendige Stunden}\pwindex{Lebendige Stunden@\emph{Lebendige Stunden}|pwk} auch als
                  gemeinsamen Übertitel einer Einaktersammlung. Damit rekurrierte er mit dem Titel
                     \emph{Lebendige Stunden}\pwindex{Lebendige Stunden. Vier Einakter@\emph{Lebendige Stunden. Vier Einakter}|pwk} wohl auf das verbindende
                  thematische Element des Zyklus\pwindex{Lebendige Stunden. Vier Einakter@\emph{Lebendige Stunden. Vier Einakter}|pwkv}: das Verhältnis von Kunst und Leben, das immer wieder vom Tod
                  durchkreuzt wird. In seinem späteren Feuilleton\pwindex{Berliner Theater. (»Lebendige Stunden« von Arthur Schnitzler.)@\emph{Berliner Theater. (»Lebendige Stunden« von Arthur Schnitzler.)}|pwkv} kritisierte Goldmann\pwindex{Goldmann, Paul 31.01.1865 – 25.09.1935@\textsc{Goldmann, Paul} (31.01.1865 – 25.09.1935), \emph{Schriftsteller/Schriftstellerin, Journalist/Journalistin}|pwk} den Titel noch einmal, vgl. Paul Goldmann\pwindex{Goldmann, Paul 31.01.1865 – 25.09.1935@\textsc{Goldmann, Paul} (31.01.1865 – 25.09.1935), \emph{Schriftsteller/Schriftstellerin, Journalist/Journalistin}|pwk}: \emph{Berliner Theater. (»Lebendige Stunden« von Arthur Schnitzler)}\pwindex{Berliner Theater. (»Lebendige Stunden« von Arthur Schnitzler.)@\emph{Berliner Theater. (»Lebendige Stunden« von Arthur Schnitzler.)}|pwk}. In:
                        \emph{Neue Freie Presse}\pwindex{Neue Freie Presse@\emph{Neue Freie Presse}|pwk}, Nr. 13.438, 22. 1. 1902, Morgenblatt, S. 1–4.}}}\label{K_L03088-6} Und
               Worte ohne \strikeout{Sin\textcolor{gray}{n}} Sinn zu gebrauchen, blos weil ſie ſchön klingen, iſt doch gar zu \label{K_L03088-7v}\edtext{\textsc{Hoffmannsthal\pwindex{Hofmannsthal, Hugo von 1874-02-01 – 1929-07-15@\textsc{Hofmannsthal, Hugo von} (1874-02-01 – 1929-07-15), \emph{Schriftsteller/Schriftstellerin}|pw}isch}}{\lemma{\textnormal{\emph{Hoffmannsthalisch}}}\Cendnote{\textnormal{Siehe Paul Goldmann an Arthur Schnitzler, 19. 6. [1894].
               }}}\label{K_L03088-7}.\pend
           
\pstart
           Ich ſah neulich »Einſame Menſchen\pwindex{Einsame Menschen. Drama@\emph{Einsame Menschen. Drama}|pw}« und war ſtarr
               über die Talentloſigkeit. Ich begreife Euch nicht, daß Ihr dieſen \label{K_L03088-8v}\edtext{Menſchen\pwindex{Hauptmann, Gerhart 15.11.1862 – 06.06.1946@\textsc{Hauptmann, Gerhart} (15.11.1862 – 06.06.1946), \emph{Schriftsteller/Schriftstellerin}|pwv}}{\lemma{\textnormal{\emph{Menſchen}}}\Cendnote{\textnormal{Zu Goldmanns\pwindex{Goldmann, Paul 31.01.1865 – 25.09.1935@\textsc{Goldmann, Paul} (31.01.1865 – 25.09.1935), \emph{Schriftsteller/Schriftstellerin, Journalist/Journalistin}|pwk} Kritik an Gerhart
                     Hauptmann\pwindex{Hauptmann, Gerhart 15.11.1862 – 06.06.1946@\textsc{Hauptmann, Gerhart} (15.11.1862 – 06.06.1946), \emph{Schriftsteller/Schriftstellerin}|pwk} siehe etwa Paul Goldmann an Arthur Schnitzler, 31. 12. [1900].
                  Siehe auch Paul Goldmann\pwindex{Goldmann, Paul 31.01.1865 – 25.09.1935@\textsc{Goldmann, Paul} (31.01.1865 – 25.09.1935), \emph{Schriftsteller/Schriftstellerin, Journalist/Journalistin}|pwk}: \emph{Berliner Theater. »Einsame Menschen« im Deutschen Theater}\pwindex{Berliner Theater. »Einsame Menschen« im Deutschen Theater@\emph{Berliner Theater. »Einsame Menschen« im Deutschen Theater}|pwk}.
                     In: \emph{Neue Freie Presse}\pwindex{Neue Freie Presse@\emph{Neue Freie Presse}|pwk}, Nr. 13.345, 19. 10. 1901, Morgenblatt, S. 1–3.}}}\label{K_L03088-8}
               auch nur einen Augenblick ernſt nehmen könnt.\pend
           
\pstart
           Ein ſehr \strikeout{ſchö\textcolor{gray}{ne}} ſchönes Stück iſt \label{K_L03088-9v}\edtext{»Die Hoffnung\pwindex{Hoffnung auf Segen. Eine Fischertragoedie in vier Acten@\emph{Die Hoffnung auf Segen. Eine Fischertragödie in vier Acten}|pw}«}{\lemma{\textnormal{\emph{»Die Hoffnung«}}}\Cendnote{\textnormal{niederl. \begin{otherlanguage}{dutch}\emph{Op hoop van zegen. Spel van de zee in vier
                        bedrijven}\pwindex{Op hoop van zegen. Spel van de zee in vier bedrijven@\emph{Op hoop van zegen. Spel van de zee in vier bedrijven}|pwk}\end{otherlanguage}, Uraufführung am 24. 12. 1900 in Amsterdam\oindex{Amsterdam@\textbf{Amsterdam}, \emph{P.PPLC}|pwk}}}}\label{K_L03088-9} von \textsc{Heyermans\pwindex{Heijermans, Herman 03.12.1864 – 22.11.1924@\textsc{Heijermans, Herman} (03.12.1864 – 22.11.1924), \emph{Schriftsteller/Schriftstellerin, Journalist/Journalistin}|pw}}. Der Verfaſſer\pwindex{Heijermans, Herman 03.12.1864 – 22.11.1924@\textsc{Heijermans, Herman} (03.12.1864 – 22.11.1924), \emph{Schriftsteller/Schriftstellerin, Journalist/Journalistin}|pwv} ein
               Jude, – \label{K_L03088-10v}\edtext{reichen Rheders\pwindex{Heijermans, Herman (Sr.) 1824 – 1910@\textsc{Heijermans, Herman (Sr.)} (1824 – 1910), \emph{Zeitungsredakteur/Zeitungsredakteurin}|pwv}{ }Sohn\pwindex{Heijermans, Herman 03.12.1864 – 22.11.1924@\textsc{Heijermans, Herman} (03.12.1864 – 22.11.1924), \emph{Schriftsteller/Schriftstellerin, Journalist/Journalistin}|pwv}}{\lemma{\textnormal{\emph{reichen Rheders Sohn}}}\Cendnote{\textnormal{Das dürfte auf einer Verwechslung
                  beruhen, der Vater Herman Heijermans
                     (senior)\pwindex{Heijermans, Herman (Sr.) 1824 – 1910@\textsc{Heijermans, Herman (Sr.)} (1824 – 1910), \emph{Zeitungsredakteur/Zeitungsredakteurin}|pwk} war Redakteur.}}}\label{K_L03088-10}\textcolor{gray}{.} Die Berlin\oindex{Berlin@\textbf{Berlin}, \emph{P.PPLC}|pw}er Kritik hat das
                  Stück\pwindex{Hoffnung auf Segen. Eine Fischertragoedie in vier Acten@\emph{Die Hoffnung auf Segen. Eine Fischertragödie in vier Acten}|pw} verriſſen, – Allen voran \label{K_L03088-11v}\edtext{\textsc{Kerr\pwindex{Kerr, Alfred 25.12.1867 – 12.10.1948@\textsc{Kerr, Alfred} (25.12.1867 – 12.10.1948), \emph{Schriftsteller/Schriftstellerin, Kritiker/Kritikerin}|pw}\pwindex{Hoffnung«. »Ein Seestueck« von Heyermans. Erst-Auffuehrung im Deutschen Theater@\emph{»Die Hoffnung«. »Ein Seestück« von Heyermans. Erst-Aufführung im Deutschen Theater}|pwv}}}{\lemma{\textnormal{\emph{Kerr}}}\Cendnote{\textnormal{Alfred Kerr\pwindex{Kerr, Alfred 25.12.1867 – 12.10.1948@\textsc{Kerr, Alfred} (25.12.1867 – 12.10.1948), \emph{Schriftsteller/Schriftstellerin, Kritiker/Kritikerin}|pwk}: \emph{»Die Hoffnung«. »Ein Seestück« von Heyermans.
                        Erst-Aufführung im Deutschen Theater}\pwindex{Hoffnung«. »Ein Seestueck« von Heyermans. Erst-Auffuehrung im Deutschen Theater@\emph{»Die Hoffnung«. »Ein Seestück« von Heyermans. Erst-Aufführung im Deutschen Theater}|pwk}. In: \emph{Der Tag}\pwindex{Tag@\emph{Der Tag}|pwk}, Nr. 431, 1. 10. 1901, S. [1]–2.}}}\label{K_L03088-11}, der doch zu Zeiten enervirend
               verſtändnißlos iſt.\pend
           
\pstart
           Was \label{K_L03088-12v}\edtext{\textsc{Glümers\pwindex{Gluemer, Marie 03.07.1867 – 16.11.1925@\textsc{Glümer, Marie} (03.07.1867 – 16.11.1925), \emph{Schauspieler/Schauspielerin}|pw}\pwindex{Gluemer, Auguste 1862-03-16 – 1956@\textsc{Glümer, Auguste} (1862-03-16 – 1956), \emph{Lehrer/Lehrerin}|pw}}}{\lemma{\textnormal{\emph{Glümers}}}\Cendnote{\textnormal{Siehe Paul Goldmann an Arthur Schnitzler, 28. 9. [1901].
               }}}\label{K_L03088-12} anlangt, ſo bin ich nicht beleidigt, ſondern erbittert. \strikeout{I\textcolor{gray}{hre}}{ }{\pb}Ich verzeihe Alles, nur keine Ungezogenheiten.
               Gratulirt habe ich nicht, und ich werde auch nicht gratuliren.\pend
           
\pstart
           Die \textsc{Triesch\pwindex{Triesch, Irene 13.04.1877 – 24.11.1964@\textsc{Triesch, Irene} (13.04.1877 – 24.11.1964), \emph{Schauspieler/Schauspielerin}|pw}} iſt unglücklich, wird \label{K_L03088-13v}\edtext{falſch
                  beſchäftigt}{\lemma{\textnormal{\emph{falſch
                  beſchäftigt}}}\Cendnote{\textnormal{Irene Triesch\pwindex{Triesch, Irene 13.04.1877 – 24.11.1964@\textsc{Triesch, Irene} (13.04.1877 – 24.11.1964), \emph{Schauspieler/Schauspielerin}|pwk} hatte ihren letzten Auftritt
                  am \emph{Frankfurter Stadttheater}\orgindex{Frankfurter Stadttheater@Frankfurter Stadttheater|pwk} am 24. 8. 1901. Danach ging sie an das \emph{Deutsche Theater Berlin}\orgindex{Deutsches Theater Berlin@Deutsches Theater Berlin|pwk}. Dort trat sie Anfang Oktober 1901 in Gerhart
                     Hauptmanns\pwindex{Hauptmann, Gerhart 15.11.1862 – 06.06.1946@\textsc{Hauptmann, Gerhart} (15.11.1862 – 06.06.1946), \emph{Schriftsteller/Schriftstellerin}|pwk}{ }\emph{Einsame Menschen}\pwindex{Einsame Menschen. Drama@\emph{Einsame Menschen. Drama}|pwk} als Anna Mahr\pwindex{Einsame Menschen. Drama@\emph{Einsame Menschen. Drama}|pwkv} auf.}}}\label{K_L03088-13} und
               ſehnt ſich nach Deinen Stücken\pwindex{Lebendige Stunden. Vier Einakter@\emph{Lebendige Stunden. Vier Einakter}|pwv}. Iſt mir im Übrigen \strikeout{ſehr} zuwider, weil
               ſie gerade die zwei Typen repräſentirt, die ich nicht vertragen kann: den der Jüdin
               und den der Komödiantin.\pend
           
\pstart
           Sage dem \textsc{Richard\pwindex{Beer-Hofmann, Richard 1866-07-11 – 1945-09-26@\textsc{Beer-Hofmann, Richard} (1866-07-11 – 1945-09-26), \emph{Schriftsteller/Schriftstellerin}|pw}}, daß die Frau Profeſſor \textsc{Döpler\pwindex{Doepler, Berta 1822? – 1902-02-10@\textsc{Doepler, Berta} (1822? – 1902-02-10)|pw}} ſich mit \strikeout{Moph} Morphium \label{K_L03088-14v}\edtext{vergiftet}{\lemma{\textnormal{\emph{vergiftet}}}\Cendnote{\textnormal{Berta Doepler\pwindex{Doepler, Berta 1822? – 1902-02-10@\textsc{Doepler, Berta} (1822? – 1902-02-10)|pwk}, eine Cousine von Else Lasker-Schüler\pwindex{Lasker-Schueler, Else 11.02.1869 – 22.01.1945@\textsc{Lasker-Schüler, Else} (11.02.1869 – 22.01.1945), \emph{Dichter/Dichterin}|pwk}, verstarb wenige Wochen
                  später, am 10. 2. 1902, als Folge eines Sprungs aus dem Fenster. Zu Beer-Hofmanns\pwindex{Beer-Hofmann, Richard 1866-07-11 – 1945-09-26@\textsc{Beer-Hofmann, Richard} (1866-07-11 – 1945-09-26), \emph{Schriftsteller/Schriftstellerin}|pwk} Bekanntschaft mit
                  ihr siehe Richard Beer-Hofmann an Arthur Schnitzler, 22. 2. 1900.}}}\label{K_L03088-14} hat, um
               den unerträglichen Schmerzen zu entgehen, die ihre unheilbare Krankheit ihr bereitet
               hat.\pend
           
\pstart
           Wollen wir dem \label{K_L03088-15v}\edtext{\textsc{Peter Dorner\pwindex{Dorner, Peter 17.02.1857 – 01.04.1931@\textsc{Dorner, Peter} (17.02.1857 – 01.04.1931), \emph{Schmied/Schmiedin, Kunsthandwerker/Kunsthandwerkerin, Kunstschmied/Kunstschmiedin}|pw}}}{\lemma{\textnormal{\emph{Peter Dorner}}}\Cendnote{\textnormal{Siehe Paul Goldmann an Arthur Schnitzler, 23. 9. [1901].
               }}}\label{K_L03088-15} nicht zuſammen das Werk\pwindex{Schmiedekunst seit dem Ende der Renaissance@\emph{Die Schmiedekunst seit dem Ende der Renaissance}|pwuv} über die »Deutſche Schmiedekunſt\pwindex{Schmiedekunst seit dem Ende der Renaissance@\emph{Die Schmiedekunst seit dem Ende der Renaissance}|pwuv}« ſchenken? Du 22 \textsc{MK} und ich 22 \textsc{MK}.\pend
           
\pstart
           {\pb}Lies’ in der letzten »Zukunft\pwindex{Zukunft@\emph{Die Zukunft}|pw}« den geiſtvollen Aufſatz \label{K_L03088-16v}\edtext{»Phyſiologie des
                  Kunſtempfindens\pwindex{Physiologie des Kunstempfindens. Der Grundsatz@\emph{Physiologie des Kunstempfindens. Der Grundsatz}|pw}«}{\lemma{\textnormal{\emph{»Phyſiologie des Kunſtempfindens«}}}\Cendnote{\textnormal{[Walter Rathenau\pwindex{Rathenau, Walther 29.09.1867 – 24.06.1922@\textsc{Rathenau, Walther} (29.09.1867 – 24.06.1922), \emph{Politiker/Politikerin, Industrieller/Industrielle}|pwk}]: \emph{Physiologie des Kunstempfindens. Der Grundsatz}\pwindex{Physiologie des Kunstempfindens. Der Grundsatz@\emph{Physiologie des Kunstempfindens. Der Grundsatz}|pwk}. In:
                        \emph{Die Zukunft}\pwindex{Zukunft@\emph{Die Zukunft}|pwk}, Bd. 37, 5. 10. 1901, S. 34–48.}}}\label{K_L03088-16}.\pend
           
\pstart
           Viele herzliche Grüße an die Mädels\pwindex{Schnitzler, Olga 17.01.1882 – 13.01.1970@\textsc{Schnitzler, Olga} (17.01.1882 – 13.01.1970), \emph{Schauspieler/Schauspielerin, Sänger/Sängerin}|pwv}\pwindex{Steinrueck, Elisabeth 19.11.1885 – 07.04.1920@\textsc{Steinrück, Elisabeth} (19.11.1885 – 07.04.1920)|pwv} und an Dich. {\\[\baselineskip]}Dein {\\[\baselineskip]}\spacefill\mbox{Paul Goldmann.}\pend
           \leftskip=0em{}\selectlanguage{ngerman}\endnumbering\briefempfaengerindex{Schnitzler, Arthur@\textsc{Schnitzler, Arthur}!zzzGoldmann, Paul@\emph{von Paul Goldmann}!1901-10-072@{7. 10. {[}1901{]}}|)be}\mylabel{L03088h}  \normalsize

\doendnotes{C}
\bigskip
\vfill

\clearpage

\footnotesize

\lohead{\textsc{register}}

% Definiere theindex-Environment komplett neu ohne reledmac
\makeatletter
\renewenvironment{theindex}{%
  \section*{\indexname}%
  \setlength{\parindent}{0pt}%
  \setlength{\parskip}{0pt plus 0.3pt}%
  \let\item\@idxitem
}{%
  \clearpage
}
\makeatother

\IfFileExists{\jobname-pw.ind}{\input{\jobname-pw.ind}}{}

\end{document}

      