%% latex-leseansicht-vorspann.tex
%% Vorspann für die Leseansicht.
%% Lädt die gemeinsame Datei latex-vorspann.tex mit nicht gesetztem Schalter.

\newif\ifkorrekturansicht
\korrekturansichtfalse

\input{../tex-inputs/latex-vorspann}


\section[ Paul Goldmann an Arthur Schnitzler, 7. 10. {[}1901{]}]{L03088 Paul Goldmann an Arthur Schnitzler,  7. 10. [1901]}
\nopagebreak\mylabel{L03088v}
\rehead{ }\normalsize\beginnumbering\briefempfaengerindex{Schnitzler, Arthur@\textsc{Schnitzler, Arthur}!zzzGoldmann, Paul@\emph{von Paul Goldmann}!1901-10-072@{7. 10. [1901]}|(be}
\toendnotes[C]{\smallbreak\pagebreak[2]}
\correspDesc{Versand  durch Paul Goldmann am 7. 10. [1901] in Berlin
\newline{}Erhalt  durch Arthur Schnitzler im Zeitraum [8. 10. 1901
                  – 12. 10. 1901?] in Wien}\toendnotes[C]{\smallbreak}
\Standort{DLA, A:Schnitzler, HS.NZ85.1.3171.}
\physDesc{Brief, 1 Blatt, 4 Seiten, 2010 Zeichen
\newline{}Handschrift: blaue Tinte, deutsche Kurrent
\newline{}Schnitzler: 1) mit Bleistift das Jahr »901« vermerkt  2) mit rotem Buntstift acht Unterstreichungen}\toendnotes[C]{\smallbreak}
\pstart
           \raggedleft{}{\pb}\textcolor{gray}{\textbf{DESSAUERSTRASSE 19}}\oindex{Dessauer Straße@\textbf{Dessauer Straße}, \emph{Straße}|pw}\pend
           
\pstart
           Berlin\oindex{Berlin@\textbf{Berlin}, \emph{Hauptstadt}|pw}, 7. Oktober.\pend
           
\pstart\center{}Mein lieber Freund,\pend\vspace{0.5em}
\pstart
           Dein Brief iſt im Ganzen recht erfreulich, – mit Ausnahme von Kopfſchmerzen und
                  \label{K_L03088-1v}\edtext{Ohrenklingen}{\lemma{\textnormal{\emph{Ohrenklingen}}}\Cendnote{\textnormal{Schnitzler litt seit
                     Herbst 1896 an Otosklerose – einer Verknöcherung des Innenohrs mit
                  zunehmender Schwerhörigkeit.}}}\label{K_L03088-1}, gegen die ich Dir leider nicht helfen kann.
               Das{ }ſpielt in Deinem Leben offenbar dieſelbe Rolle, wie \label{K_L03088-2v}\edtext{\textsc{Benedikt\pwindex{Benedikt, Moriz 27.\,5.\,1849 Kvačice – 18.\,3.\,1920 Wien@\textsc{Benedikt, Moriz} (27.\,5.\,1849 Kvačice – 18.\,3.\,1920 Wien), \emph{Journalist, Herausgeber}|pw}}}{\lemma{\textnormal{\emph{Benedikt}}}\Cendnote{\textnormal{Moriz Benedikt\pwindex{Benedikt, Moriz 27.\,5.\,1849 Kvačice – 18.\,3.\,1920 Wien@\textsc{Benedikt, Moriz} (27.\,5.\,1849 Kvačice – 18.\,3.\,1920 Wien), \emph{Journalist, Herausgeber}|pwk} war als Herausgeber der \emph{Neuen Freien Presse}\pwindex{Neue Freie Presse@\emph{Neue Freie Presse}|pwk}{ }Goldmanns\pwindex{Goldmann, Paul 31.\,1.\,1865 Breslau – 25.\,9.\,1935 Wien@\textsc{Goldmann, Paul} (31.\,1.\,1865 Breslau – 25.\,9.\,1935 Wien), \emph{Schriftsteller, Journalist}|pwk} Vorgesetzter.}}}\label{K_L03088-2} in dem
               meinen. Es{ }ſcheint, daß zu jedem Leben ein wenig \textsc{Benedikt\pwindex{Benedikt, Moriz 27.\,5.\,1849 Kvačice – 18.\,3.\,1920 Wien@\textsc{Benedikt, Moriz} (27.\,5.\,1849 Kvačice – 18.\,3.\,1920 Wien), \emph{Journalist, Herausgeber}|pw}} gehört.\pend
           
\pstart
           Gegen ein \label{K_L03088-3v}\edtext{Auftreten \textsc{Olgas\pwindex{Schnitzler, Olga 17.\,1.\,1882 Wien – 13.\,1.\,1970 Lugano@\textsc{Schnitzler, Olga} (17.\,1.\,1882 Wien – 13.\,1.\,1970 Lugano), \emph{Schauspielerin, Sängerin}|pw}} bei \textsc{Salten\pwindex{Salten, Felix 6.\,9.\,1869 Budapest – 8.\,10.\,1945 Zürich@\textsc{Salten, Felix} (6.\,9.\,1869 Budapest – 8.\,10.\,1945 Zürich), \emph{Schriftsteller, Journalist, Chefredakteur}|pw}\orgindex{Jung-Wiener Theater zum Lieben Augustin@Jung-Wiener Theater zum Lieben Augustin|pwv}}}{\lemma{\textnormal{\emph{Auftreten … Salten}}}\Cendnote{\textnormal{Olga\pwindex{Schnitzler, Olga 17.\,1.\,1882 Wien – 13.\,1.\,1970 Lugano@\textsc{Schnitzler, Olga} (17.\,1.\,1882 Wien – 13.\,1.\,1970 Lugano), \emph{Schauspielerin, Sängerin}|pwk} trat nicht im Jung-Wiener Theater zum lieben Augustin\orgindex{Jung-Wiener Theater zum Lieben Augustin@Jung-Wiener Theater zum Lieben Augustin|pwkv}
                  auf.}}}\label{K_L03088-3} wäre ich entſchieden. Soll ihr für alle Zeiten die \strikeout{\textcolor{gray}{×}}{ }\label{K_L03088-4v}\edtext{Überbrettl\orgindex{Überbrettl@Überbrettl|pwv}}{\lemma{\textnormal{\emph{Überbrettl}}}\Cendnote{\textnormal{Vorbild für das Kabarett \emph{Jung-Wiener Theater zum Lieben Augustin}\orgindex{Jung-Wiener Theater zum Lieben Augustin@Jung-Wiener Theater zum Lieben Augustin|pwk} war das \emph{Überbrettl}\orgindex{Überbrettl@Überbrettl|pwk}, siehe XXXX Auszeichnungsfehler: Dokument L03059 nicht gefunden. }}}\label{K_L03088-4}-Marke aufgeprägt werden? Das Programm
               der \textsc{Saltenschen\pwindex{Salten, Felix 6.\,9.\,1869 Budapest – 8.\,10.\,1945 Zürich@\textsc{Salten, Felix} (6.\,9.\,1869 Budapest – 8.\,10.\,1945 Zürich), \emph{Schriftsteller, Journalist, Chefredakteur}|pw}}{ }Unternehmung\orgindex{Jung-Wiener Theater zum Lieben Augustin@Jung-Wiener Theater zum Lieben Augustin|pwv}, das ich heut in der N. Fr.
                  Pr.\pwindex{Neue Freie Presse@\emph{Neue Freie Presse}|pw}{ }\label{K_L03088-5v}\edtext{leſe\pwindex{Theater- und Kunstnachrichten [zur Eröffnung des Jung-Wiener Theaters zum lieben Augustin]@\emph{Theater- und Kunstnachrichten [zur Eröffnung des Jung-Wiener Theaters zum lieben Augustin]}|pwv}}{\lemma{\textnormal{\emph{lese}}}\Cendnote{\textnormal{[O. V.]: \emph{Theater- und Kunstnachrichten.
                        [Zur Eröffnung des Jung-Wiener Theaters zum lieben Augustin]}\pwindex{Theater- und Kunstnachrichten [zur Eröffnung des Jung-Wiener Theaters zum lieben Augustin]@\emph{Theater- und Kunstnachrichten [zur Eröffnung des Jung-Wiener Theaters zum lieben Augustin]}|pwk}. In: \emph{Neue Freie Presse}\pwindex{Neue Freie Presse@\emph{Neue Freie Presse}|pwk}, Nr. 13.332, 6. 10. 1901, S. 8.}}}\label{K_L03088-5}, iſt ein großer
               Kuddelmuddel. Der Mann\pwindex{Salten, Felix 6.\,9.\,1869 Budapest – 8.\,10.\,1945 Zürich@\textsc{Salten, Felix} (6.\,9.\,1869 Budapest – 8.\,10.\,1945 Zürich), \emph{Schriftsteller, Journalist, Chefredakteur}|pwv}{ }ſcheint \strikeout{ab\textcolor{gray}{ſ}o} abſolut nicht zu
               wiſſen, was er will.\pend
           
\pstart
           »Lebendige Stunden\pwindex{Schnitzler, Arthur 15.\,5.\,1862 Wien – 21.\,10.\,1931 ebd.@\textsc{Schnitzler, Arthur} (15.\,5.\,1862 Wien – 21.\,10.\,1931 ebd.), \emph{Schriftsteller, Mediziner}!Lebendige Stunden. Vier Einakter@\strich\emph{Lebendige Stunden. Vier Einakter}|pw}« iſt ein hübſcher Titel. {\pb}Aber er{ }ſagt mir nichts. \label{K_L03088-6v}\edtext{Warum »lebendig«? Warum »Stunden«?}{\lemma{\textnormal{\emph{Warum … »Stunden«?}}}\Cendnote{\textnormal{Schnitzler verwendete den Titel des
                  Einakters \emph{Lebendige Stunden}\pwindex{Schnitzler, Arthur 15.\,5.\,1862 Wien – 21.\,10.\,1931 ebd.@\textsc{Schnitzler, Arthur} (15.\,5.\,1862 Wien – 21.\,10.\,1931 ebd.), \emph{Schriftsteller, Mediziner}!Lebendige Stunden@\strich\emph{Lebendige Stunden}|pwk} auch als
                  gemeinsamen Übertitel einer Einaktersammlung. Damit rekurrierte er mit dem Titel
                     \emph{Lebendige Stunden}\pwindex{Schnitzler, Arthur 15.\,5.\,1862 Wien – 21.\,10.\,1931 ebd.@\textsc{Schnitzler, Arthur} (15.\,5.\,1862 Wien – 21.\,10.\,1931 ebd.), \emph{Schriftsteller, Mediziner}!Lebendige Stunden. Vier Einakter@\strich\emph{Lebendige Stunden. Vier Einakter}|pwk} wohl auf das verbindende
                  thematische Element des Zyklus\pwindex{Schnitzler, Arthur 15.\,5.\,1862 Wien – 21.\,10.\,1931 ebd.@\textsc{Schnitzler, Arthur} (15.\,5.\,1862 Wien – 21.\,10.\,1931 ebd.), \emph{Schriftsteller, Mediziner}!Lebendige Stunden. Vier Einakter@\strich\emph{Lebendige Stunden. Vier Einakter}|pwkv}: das Verhältnis von Kunst und Leben, das immer wieder vom Tod
                  durchkreuzt wird. In seinem späteren Feuilleton\pwindex{Goldmann, Paul 31.\,1.\,1865 Breslau – 25.\,9.\,1935 Wien@\textsc{Goldmann, Paul} (31.\,1.\,1865 Breslau – 25.\,9.\,1935 Wien), \emph{Schriftsteller, Journalist}!Berliner Theater. (»Lebendige Stunden« von Arthur Schnitzler.)@\strich\emph{Berliner Theater. (»Lebendige Stunden« von Arthur Schnitzler.)}|pwkv} kritisierte Goldmann\pwindex{Goldmann, Paul 31.\,1.\,1865 Breslau – 25.\,9.\,1935 Wien@\textsc{Goldmann, Paul} (31.\,1.\,1865 Breslau – 25.\,9.\,1935 Wien), \emph{Schriftsteller, Journalist}|pwk} den Titel noch einmal, vgl. Paul Goldmann\pwindex{Goldmann, Paul 31.\,1.\,1865 Breslau – 25.\,9.\,1935 Wien@\textsc{Goldmann, Paul} (31.\,1.\,1865 Breslau – 25.\,9.\,1935 Wien), \emph{Schriftsteller, Journalist}|pwk}: \emph{Berliner Theater. (»Lebendige Stunden« von Arthur Schnitzler)}\pwindex{Goldmann, Paul 31.\,1.\,1865 Breslau – 25.\,9.\,1935 Wien@\textsc{Goldmann, Paul} (31.\,1.\,1865 Breslau – 25.\,9.\,1935 Wien), \emph{Schriftsteller, Journalist}!Berliner Theater. (»Lebendige Stunden« von Arthur Schnitzler.)@\strich\emph{Berliner Theater. (»Lebendige Stunden« von Arthur Schnitzler.)}|pwk}. In:
                        \emph{Neue Freie Presse}\pwindex{Neue Freie Presse@\emph{Neue Freie Presse}|pwk}, Nr. 13.438, 22. 1. 1902, Morgenblatt, S. 1–4.}}}\label{K_L03088-6} Und
               Worte ohne \strikeout{Sin\textcolor{gray}{n}} Sinn zu gebrauchen, blos weil{ }ſie{ }ſchön klingen, iſt doch gar zu \label{K_L03088-7v}\edtext{\textsc{Hoffmannsthal\pwindex{Hofmannsthal, Hugo von 1.\,2.\,1874 Wien – 15.\,7.\,1929 Rodaun@\textsc{Hofmannsthal, Hugo von} (1.\,2.\,1874 Wien – 15.\,7.\,1929 Rodaun), \emph{Schriftsteller}|pw}isch}}{\lemma{\textnormal{\emph{Hoffmannsthalisch}}}\Cendnote{\textnormal{Siehe XXXX Auszeichnungsfehler: Dokument L02627 nicht gefunden.
               }}}\label{K_L03088-7}.\pend
           
\pstart
           Ich{ }ſah neulich »Einſame Menſchen\pwindex{Hauptmann, Gerhart 15.\,11.\,1862 Szczawno-Zdrój – 6.\,6.\,1946 Jagniątków@\textsc{Hauptmann, Gerhart} (15.\,11.\,1862 Szczawno-Zdrój – 6.\,6.\,1946 Jagniątków), \emph{Schriftsteller}!Einsame Menschen. Drama@\strich\emph{Einsame Menschen. Drama}|pw}« und war{ }ſtarr
               über die Talentloſigkeit. Ich begreife Euch nicht, daß Ihr dieſen \label{K_L03088-8v}\edtext{Menſchen\pwindex{Hauptmann, Gerhart 15.\,11.\,1862 Szczawno-Zdrój – 6.\,6.\,1946 Jagniątków@\textsc{Hauptmann, Gerhart} (15.\,11.\,1862 Szczawno-Zdrój – 6.\,6.\,1946 Jagniątków), \emph{Schriftsteller}|pwv}}{\lemma{\textnormal{\emph{Menschen}}}\Cendnote{\textnormal{Zu Goldmanns\pwindex{Goldmann, Paul 31.\,1.\,1865 Breslau – 25.\,9.\,1935 Wien@\textsc{Goldmann, Paul} (31.\,1.\,1865 Breslau – 25.\,9.\,1935 Wien), \emph{Schriftsteller, Journalist}|pwk} Kritik an Gerhart
                     Hauptmann\pwindex{Hauptmann, Gerhart 15.\,11.\,1862 Szczawno-Zdrój – 6.\,6.\,1946 Jagniątków@\textsc{Hauptmann, Gerhart} (15.\,11.\,1862 Szczawno-Zdrój – 6.\,6.\,1946 Jagniątków), \emph{Schriftsteller}|pwk} siehe etwa XXXX Auszeichnungsfehler: Dokument L02947 nicht gefunden.
                  Siehe auch Paul Goldmann\pwindex{Goldmann, Paul 31.\,1.\,1865 Breslau – 25.\,9.\,1935 Wien@\textsc{Goldmann, Paul} (31.\,1.\,1865 Breslau – 25.\,9.\,1935 Wien), \emph{Schriftsteller, Journalist}|pwk}: \emph{Berliner Theater. »Einsame Menschen« im Deutschen Theater}\pwindex{Goldmann, Paul 31.\,1.\,1865 Breslau – 25.\,9.\,1935 Wien@\textsc{Goldmann, Paul} (31.\,1.\,1865 Breslau – 25.\,9.\,1935 Wien), \emph{Schriftsteller, Journalist}!Berliner Theater. »Einsame Menschen« im Deutschen Theater@\strich\emph{Berliner Theater. »Einsame Menschen« im Deutschen Theater}|pwk}.
                     In: \emph{Neue Freie Presse}\pwindex{Neue Freie Presse@\emph{Neue Freie Presse}|pwk}, Nr. 13.345, 19. 10. 1901, Morgenblatt, S. 1–3.}}}\label{K_L03088-8}
               auch nur einen Augenblick ernſt nehmen könnt.\pend
           
\pstart
           Ein{ }ſehr \strikeout{ſchö\textcolor{gray}{ne}}{ }ſchönes Stück iſt \label{K_L03088-9v}\edtext{»Die Hoffnung\pwindex{Heijermans, Herman 3.\,12.\,1864 Rotterdam – 22.\,11.\,1924 Zandvoort@\textsc{Heijermans, Herman} (3.\,12.\,1864 Rotterdam – 22.\,11.\,1924 Zandvoort), \emph{Schriftsteller, Journalist}!Hoffnung auf Segen. Eine Fischertragödie in vier Acten@\strich\emph{Die Hoffnung auf Segen. Eine Fischertragödie in vier Acten}|pw}«}{\lemma{\textnormal{\emph{»Die Hoffnung«}}}\Cendnote{\textnormal{niederl. \begin{otherlanguage}{dutch}\emph{Op hoop van zegen. Spel van de zee in vier
                        bedrijven}\pwindex{Heijermans, Herman 3.\,12.\,1864 Rotterdam – 22.\,11.\,1924 Zandvoort@\textsc{Heijermans, Herman} (3.\,12.\,1864 Rotterdam – 22.\,11.\,1924 Zandvoort), \emph{Schriftsteller, Journalist}!Op hoop van zegen. Spel van de zee in vier bedrijven@\strich\emph{Op hoop van zegen. Spel van de zee in vier bedrijven}|pwk}\end{otherlanguage}, Uraufführung\eventindex{Amsterdam@\textbf{Amsterdam}!Uraufführung von Op hoop van zegen. Spel van de zee in vier bedrijven, 24.12.1900@Uraufführung von Op hoop van zegen. Spel van de zee in vier bedrijven, 24.12.1900|pwkv} am 24. 12. 1900 in Amsterdam\oindex{Amsterdam@\textbf{Amsterdam}, \emph{Hauptstadt}|pwk}}}}\label{K_L03088-9} von \textsc{Heyermans\pwindex{Heijermans, Herman 3.\,12.\,1864 Rotterdam – 22.\,11.\,1924 Zandvoort@\textsc{Heijermans, Herman} (3.\,12.\,1864 Rotterdam – 22.\,11.\,1924 Zandvoort), \emph{Schriftsteller, Journalist}|pw}}. Der Verfaſſer\pwindex{Heijermans, Herman 3.\,12.\,1864 Rotterdam – 22.\,11.\,1924 Zandvoort@\textsc{Heijermans, Herman} (3.\,12.\,1864 Rotterdam – 22.\,11.\,1924 Zandvoort), \emph{Schriftsteller, Journalist}|pwv} ein
               Jude, – \label{K_L03088-10v}\edtext{reichen Rheders\pwindex{Heijermans, Herman (Sr.) 1824 Rotterdam – 1910 ebd.@\textsc{Heijermans, Herman (Sr.)} (1824 Rotterdam – 1910 ebd.), \emph{Zeitungsredakteur}|pwv}{ }Sohn\pwindex{Heijermans, Herman 3.\,12.\,1864 Rotterdam – 22.\,11.\,1924 Zandvoort@\textsc{Heijermans, Herman} (3.\,12.\,1864 Rotterdam – 22.\,11.\,1924 Zandvoort), \emph{Schriftsteller, Journalist}|pwv}}{\lemma{\textnormal{\emph{reichen Rheders Sohn}}}\Cendnote{\textnormal{Das dürfte auf einer Verwechslung
                  beruhen, der Vater Herman Heijermans
                     (senior)\pwindex{Heijermans, Herman (Sr.) 1824 Rotterdam – 1910 ebd.@\textsc{Heijermans, Herman (Sr.)} (1824 Rotterdam – 1910 ebd.), \emph{Zeitungsredakteur}|pwk} war Redakteur.}}}\label{K_L03088-10}\textcolor{gray}{.} Die Berlin\oindex{Berlin@\textbf{Berlin}, \emph{Hauptstadt}|pw}er Kritik hat das
                  Stück\pwindex{Heijermans, Herman 3.\,12.\,1864 Rotterdam – 22.\,11.\,1924 Zandvoort@\textsc{Heijermans, Herman} (3.\,12.\,1864 Rotterdam – 22.\,11.\,1924 Zandvoort), \emph{Schriftsteller, Journalist}!Hoffnung auf Segen. Eine Fischertragödie in vier Acten@\strich\emph{Die Hoffnung auf Segen. Eine Fischertragödie in vier Acten}|pw} verriſſen, – Allen voran \label{K_L03088-11v}\edtext{\textsc{Kerr\pwindex{Kerr, Alfred 25.\,12.\,1867 Breslau – 12.\,10.\,1948 Hamburg@\textsc{Kerr, Alfred} (25.\,12.\,1867 Breslau – 12.\,10.\,1948 Hamburg), \emph{Schriftsteller, Kritiker}|pw}\pwindex{Kerr, Alfred 25.\,12.\,1867 Breslau – 12.\,10.\,1948 Hamburg@\textsc{Kerr, Alfred} (25.\,12.\,1867 Breslau – 12.\,10.\,1948 Hamburg), \emph{Schriftsteller, Kritiker}!Hoffnung«. »Ein Seestück« von Heyermans. Erst-Aufführung im Deutschen Theater@\strich\emph{»Die Hoffnung«. »Ein Seestück« von Heyermans. Erst-Aufführung im Deutschen Theater}|pwv}}}{\lemma{\textnormal{\emph{Kerr}}}\Cendnote{\textnormal{Alfred Kerr\pwindex{Kerr, Alfred 25.\,12.\,1867 Breslau – 12.\,10.\,1948 Hamburg@\textsc{Kerr, Alfred} (25.\,12.\,1867 Breslau – 12.\,10.\,1948 Hamburg), \emph{Schriftsteller, Kritiker}|pwk}: \emph{»Die Hoffnung«. »Ein Seestück« von Heyermans.
                        Erst-Aufführung im Deutschen Theater}\pwindex{Kerr, Alfred 25.\,12.\,1867 Breslau – 12.\,10.\,1948 Hamburg@\textsc{Kerr, Alfred} (25.\,12.\,1867 Breslau – 12.\,10.\,1948 Hamburg), \emph{Schriftsteller, Kritiker}!Hoffnung«. »Ein Seestück« von Heyermans. Erst-Aufführung im Deutschen Theater@\strich\emph{»Die Hoffnung«. »Ein Seestück« von Heyermans. Erst-Aufführung im Deutschen Theater}|pwk}. In: \emph{Der Tag}\pwindex{Tag@\emph{Der Tag}|pwk}, Nr. 431, 1. 10. 1901, S. [1]–2.}}}\label{K_L03088-11}, der doch zu Zeiten enervirend
               verſtändnißlos iſt.\pend
           
\pstart
           Was \label{K_L03088-12v}\edtext{\textsc{Glümers\pwindex{Glümer, Marie 3.\,7.\,1867 Wien – 16.\,11.\,1925 München@\textsc{Glümer, Marie} (3.\,7.\,1867 Wien – 16.\,11.\,1925 München), \emph{Schauspielerin}|pw}\pwindex{Glümer, Auguste 16.\,3.\,1862 Wien – 1956@\textsc{Glümer, Auguste} (16.\,3.\,1862 Wien – 1956), \emph{Lehrerin}|pw}}}{\lemma{\textnormal{\emph{Glümers}}}\Cendnote{\textnormal{Siehe XXXX Auszeichnungsfehler: Dokument L03087 nicht gefunden.
               }}}\label{K_L03088-12} anlangt,{ }ſo bin ich nicht beleidigt,{ }ſondern erbittert. \strikeout{I\textcolor{gray}{hre}}{ }{\pb}Ich verzeihe Alles, nur keine Ungezogenheiten.
               Gratulirt habe ich nicht, und ich werde auch nicht gratuliren.\pend
           
\pstart
           Die \textsc{Triesch\pwindex{Triesch, Irene 13.\,4.\,1877 Wien – 24.\,11.\,1964 Basel@\textsc{Triesch, Irene} (13.\,4.\,1877 Wien – 24.\,11.\,1964 Basel), \emph{Schauspielerin}|pw}} iſt unglücklich, wird \label{K_L03088-13v}\edtext{falſch
                  beſchäftigt}{\lemma{\textnormal{\emph{falsch
                  beschäftigt}}}\Cendnote{\textnormal{Irene Triesch\pwindex{Triesch, Irene 13.\,4.\,1877 Wien – 24.\,11.\,1964 Basel@\textsc{Triesch, Irene} (13.\,4.\,1877 Wien – 24.\,11.\,1964 Basel), \emph{Schauspielerin}|pwk} hatte ihren letzten Auftritt
                  am \emph{Frankfurter Stadttheater}\orgindex{Frankfurter Stadttheater@Frankfurter Stadttheater|pwk} am 24. 8. 1901. Danach ging sie an das \emph{Deutsche Theater Berlin}\orgindex{Deutsches Theater Berlin@Deutsches Theater Berlin|pwk}. Dort trat sie Anfang Oktober 1901 in Gerhart
                     Hauptmanns\pwindex{Hauptmann, Gerhart 15.\,11.\,1862 Szczawno-Zdrój – 6.\,6.\,1946 Jagniątków@\textsc{Hauptmann, Gerhart} (15.\,11.\,1862 Szczawno-Zdrój – 6.\,6.\,1946 Jagniątków), \emph{Schriftsteller}|pwk}{ }\emph{Einsame Menschen}\pwindex{Hauptmann, Gerhart 15.\,11.\,1862 Szczawno-Zdrój – 6.\,6.\,1946 Jagniątków@\textsc{Hauptmann, Gerhart} (15.\,11.\,1862 Szczawno-Zdrój – 6.\,6.\,1946 Jagniątków), \emph{Schriftsteller}!Einsame Menschen. Drama@\strich\emph{Einsame Menschen. Drama}|pwk} als Anna Mahr\pwindex{Hauptmann, Gerhart 15.\,11.\,1862 Szczawno-Zdrój – 6.\,6.\,1946 Jagniątków@\textsc{Hauptmann, Gerhart} (15.\,11.\,1862 Szczawno-Zdrój – 6.\,6.\,1946 Jagniątków), \emph{Schriftsteller}!Einsame Menschen. Drama@\strich\emph{Einsame Menschen. Drama}|pwkv} auf.}}}\label{K_L03088-13} und{ }ſehnt{ }ſich nach Deinen Stücken\pwindex{Schnitzler, Arthur 15.\,5.\,1862 Wien – 21.\,10.\,1931 ebd.@\textsc{Schnitzler, Arthur} (15.\,5.\,1862 Wien – 21.\,10.\,1931 ebd.), \emph{Schriftsteller, Mediziner}!Lebendige Stunden. Vier Einakter@\strich\emph{Lebendige Stunden. Vier Einakter}|pwv}. Iſt mir im Übrigen \strikeout{ſehr} zuwider, weil{ }ſie gerade die zwei Typen repräſentirt, die ich nicht vertragen kann: den der Jüdin
               und den der Komödiantin.\pend
           
\pstart
           Sage dem \textsc{Richard\pwindex{Beer-Hofmann, Richard 11.\,7.\,1866 Wien – 26.\,9.\,1945 New York City@\textsc{Beer-Hofmann, Richard} (11.\,7.\,1866 Wien – 26.\,9.\,1945 New York City), \emph{Schriftsteller}|pw}}, daß die Frau Profeſſor \textsc{Döpler\pwindex{Doepler, Berta 1822? – 10.\,2.\,1902@\textsc{Doepler, Berta} (1822? – 10.\,2.\,1902)|pw}}{ }ſich mit \strikeout{Moph} Morphium \label{K_L03088-14v}\edtext{vergiftet}{\lemma{\textnormal{\emph{vergiftet}}}\Cendnote{\textnormal{Berta Doepler\pwindex{Doepler, Berta 1822? – 10.\,2.\,1902@\textsc{Doepler, Berta} (1822? – 10.\,2.\,1902)|pwk}, eine Cousine von Else Lasker-Schüler\pwindex{Lasker-Schüler, Else 11.\,2.\,1869 Elberfeld – 22.\,1.\,1945 Jerusalem@\textsc{Lasker-Schüler, Else} (11.\,2.\,1869 Elberfeld – 22.\,1.\,1945 Jerusalem), \emph{Dichterin}|pwk}, verstarb wenige Wochen
                  später, am 10. 2. 1902, als Folge eines Sprungs aus dem Fenster. Zu Beer-Hofmanns\pwindex{Beer-Hofmann, Richard 11.\,7.\,1866 Wien – 26.\,9.\,1945 New York City@\textsc{Beer-Hofmann, Richard} (11.\,7.\,1866 Wien – 26.\,9.\,1945 New York City), \emph{Schriftsteller}|pwk} Bekanntschaft mit
                  ihr siehe XXXX Auszeichnungsfehler: Dokument L01016 nicht gefunden.}}}\label{K_L03088-14} hat, um
               den unerträglichen Schmerzen zu entgehen, die ihre unheilbare Krankheit ihr bereitet
               hat.\pend
           
\pstart
           Wollen wir dem \label{K_L03088-15v}\edtext{\textsc{Peter Dorner\pwindex{Dorner, Peter 17.\,2.\,1857 Welsberg-Taisten – 1.\,4.\,1931 ebd.@\textsc{Dorner, Peter} (17.\,2.\,1857 Welsberg-Taisten – 1.\,4.\,1931 ebd.), \emph{Schmied, Kunsthandwerker, Kunstschmied}|pw}}}{\lemma{\textnormal{\emph{Peter Dorner}}}\Cendnote{\textnormal{Siehe XXXX Auszeichnungsfehler: Dokument L03085 nicht gefunden.
               }}}\label{K_L03088-15} nicht zuſammen das Werk\pwindex{\textcolor{red}{\textsuperscript{XXXX indx1}}!Schmiedekunst seit dem Ende der Renaissance@\strich\emph{Die Schmiedekunst seit dem Ende der Renaissance}|pwuv} über die »Deutſche Schmiedekunſt\pwindex{\textcolor{red}{\textsuperscript{XXXX indx1}}!Schmiedekunst seit dem Ende der Renaissance@\strich\emph{Die Schmiedekunst seit dem Ende der Renaissance}|pwuv}«{ }ſchenken? Du 22 \textsc{MK} und ich 22 \textsc{MK}.\pend
           
\pstart
           {\pb}Lies’ in der letzten »Zukunft\pwindex{Zukunft@\emph{Die Zukunft}|pw}« den geiſtvollen Aufſatz \label{K_L03088-16v}\edtext{»Phyſiologie des
                  Kunſtempfindens\pwindex{Physiologie des Kunstempfindens. Der Grundsatz@\emph{Physiologie des Kunstempfindens. Der Grundsatz}|pw}«}{\lemma{\textnormal{\emph{»Physiologie des Kunstempfindens«}}}\Cendnote{\textnormal{[Walter Rathenau\pwindex{Rathenau, Walther 29.\,9.\,1867 Berlin – 24.\,6.\,1922 ebd.@\textsc{Rathenau, Walther} (29.\,9.\,1867 Berlin – 24.\,6.\,1922 ebd.), \emph{Politiker, Industrieller}|pwk}]: \emph{Physiologie des Kunstempfindens. Der Grundsatz}\pwindex{Physiologie des Kunstempfindens. Der Grundsatz@\emph{Physiologie des Kunstempfindens. Der Grundsatz}|pwk}. In:
                        \emph{Die Zukunft}\pwindex{Zukunft@\emph{Die Zukunft}|pwk}, Bd. 37, 5. 10. 1901, S. 34–48.}}}\label{K_L03088-16}.\pend
           
\pstart
           Viele herzliche Grüße an die Mädels\pwindex{Schnitzler, Olga 17.\,1.\,1882 Wien – 13.\,1.\,1970 Lugano@\textsc{Schnitzler, Olga} (17.\,1.\,1882 Wien – 13.\,1.\,1970 Lugano), \emph{Schauspielerin, Sängerin}|pwv}\pwindex{Steinrück, Elisabeth 19.\,11.\,1885 – 7.\,4.\,1920 Partenkirchen@\textsc{Steinrück, Elisabeth} (19.\,11.\,1885 – 7.\,4.\,1920 Partenkirchen)|pwv} und an Dich. {\\[\baselineskip]}Dein {\\[\baselineskip]}\spacefill\mbox{Paul Goldmann.}\pend
           \leftskip=0em{}\selectlanguage{ngerman}\endnumbering\briefempfaengerindex{Schnitzler, Arthur@\textsc{Schnitzler, Arthur}!zzzGoldmann, Paul@\emph{von Paul Goldmann}!1901-10-072@{7. 10. [1901]}|)be}\mylabel{L03088h}  \newcommand{\dateiname}{L03088}\newcommand{\titel}{Paul Goldmann an Arthur Schnitzler, 7. 10. [1901]}\newcommand{\editorInnen}{Martin Anton Müller und Laura Untner}%% latex-leseansicht-abspann.tex
%% Abspann für die Leseansicht.
%% Der Schalter \ifkorrekturansicht ist bereits durch den Vorspann gesetzt.

%% latex-abspann.tex
%% Gemeinsamer Abspann für Korrekturansicht und Leseansicht.
%% Setzt den Schalter \ifkorrekturansicht voraus (gesetzt in den
%% einbindenden Dateien latex-korrekturansicht-abspann.tex bzw.
%% latex-leseansicht-abspann.tex).
%% ---------------------------------------------------------------

\normalsize

% Das esempio-Environment wird nur in der Leseansicht benötigt
\ifkorrekturansicht\else
\newenvironment{esempio}[3]%
{
    \vspace{1.5ex}
    \rlap{\underline{#1}}
    \par
    \setlength{\parindent}{0cm}
    \nopagebreak
    \leftskip=#2cm
    \rightskip=#3cm
}
{
    \par
}
\fi

\doendnotes{C}
\bigskip
\vfill

\clearpage

\footnotesize

\ifkorrekturansicht
  \lohead{\textsc{register}}
\fi

% theindex-Environment neu definieren ohne reledmac
\makeatletter
\renewenvironment{theindex}{%
  \ifkorrekturansicht
    \section*{\indexname}%
  \else
    \subsubsection*{Index der erwähnten Entitäten}%
  \fi
  \setlength{\parindent}{0pt}%
  \setlength{\parskip}{0pt plus 0.3pt}%
  \let\item\@idxitem
}{%
  \ifkorrekturansicht\clearpage\fi
}
\makeatother

\IfFileExists{\jobname-pw.ind}{\input{\jobname-pw.ind}}{}

% Quellenangabe nur in der Leseansicht
\ifkorrekturansicht\else
% Fallback-Definitionen, falls die .tex-Datei \titel etc. nicht gesetzt hat
\providecommand{\titel}{}
\providecommand{\editorInnen}{}
\providecommand{\dateiname}{\jobname}

\vspace{3cm}

\vfill

\footnotesize
\textsc{Quelle}: \titel. Herausgegeben von {\editorInnen}. In: \emph{Arthur Schnitzler: Briefwechsel mit Autorinnen und Autoren}.
 Digitale Edition, https://schnitzler-briefe.acdh.oeaw.ac.at/{\dateiname}.html (Stand \today)
\fi

\end{document}


