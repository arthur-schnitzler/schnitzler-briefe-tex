%% latex-leseansicht-vorspann.tex
%% Vorspann für die Leseansicht.
%% Lädt die gemeinsame Datei latex-vorspann.tex mit nicht gesetztem Schalter.

\newif\ifkorrekturansicht
\korrekturansichtfalse

\input{../tex-inputs/latex-vorspann}


\section[Arthur Schnitzler an Richard Beer-Hofmann, 2. 8. 1914]{L02189 Arthur Schnitzler an Richard Beer-Hofmann, 2. 8. 1914}
\nopagebreak\mylabel{L02189v}
\rehead{ }\normalsize\beginnumbering\briefempfaengerindex{Beer-Hofmann, Richard@\textsc{Beer-Hofmann, Richard}!zzzSchnitzler, Arthur@\emph{von Arthur Schnitzler}!1914-08-021@{2. 8. 1914}|(be}
\toendnotes[C]{\smallbreak\pagebreak[2]}
\correspDesc{Versand  durch Arthur Schnitzler am 2. 8. 1914 in Celerina
\newline{}Erhalt  durch Richard Beer-Hofmann im Zeitraum [3. 8. 1914
                  – 7. 8. 1914?] in Weißenbach am Attersee}\toendnotes[C]{\smallbreak}
\Standort{YCGL, MSS 31.}
\physDesc{Bildpostkarte, 353 Zeichen
\newline{}Handschrift: Bleistift, deutsche Kurrent
\newline{}Versand: Stempel: »\nobreak{}\oindex{Celerina@\textbf{Celerina}|pwk}\textcolor{gray}{C}eler\textcolor{gray}{ina}\textcolor{gray}{(Graubünden)}, 2. VIII. 14, 9\nobreak{}«.  
\newline{}Beer-Hofmann: mit blauem Buntstift den Erhalt und die Beantwortung
                                    markiert: »E.B.« }
\buchAbdrucke{\weitereDrucke{Arthur Schnitzler, Richard Beer-Hofmann: \emph{Briefwechsel 1891–1931}. Herausgegeben von Konstanze Fliedl. Wien, Zürich: \emph{Europaverlag} 1992, S. 219.} }\toendnotes[C]{\smallbreak}\pstart{}{\pb}Hrn \textsc{Dr Richard}\pend{}\pstart{}\textsc{Beer Hofmann}\pend{}\pstart{}\textsc{Weissenbach\oindex{Weißenbach am Attersee@\textbf{Weißenbach am Attersee}, \emph{Verwaltungsgebiet}|pw}.}\pend{}\pstart{}\textsc{Am Attersee\oindex{Attersee@\textbf{Attersee}, \emph{See}|pw}}\pend{}\pstart{}\textsc{Ober Oesterreich}\oindex{Oberösterreich@\textbf{Oberösterreich}, \emph{Land}|pw}.\pend{}{\bigskip}
\pstart
           \noindent{}\centering{}{\pb}\textcolor{gray}{\textbf{\label{T_L02189-1v}\edtext{Cresta Palace, Celerina\oindex{Cresta Palace@\textbf{Cresta Palace}, \emph{Hotel}|pw}}{\lemma{\textnormal{\emph{Cresta Palace, Celerina}}}\Cendnote{\textnormal{von Schnitzler unterstrichen}}}\label{T_L02189-1}}}\pend
           \vspace{1em}
\pstart
           \noindent{}{\pb}2. 8. 914. Wir{ }ſind – u bleiben vorläufig hier –
               (Bahnverkehr in der Schweiz\oindex{Schweiz@\textbf{Schweiz}|pw} nur für Militär
               zuläſſig; –) laſſen Sie uns wiſſen wie es Ihnen geht; – hier wär es wundervoll – we{\geminationn} man ruhiger {\pb}ſein
               könnte. Das Engadin\oindex{Engadin@\textbf{Engadin}, \emph{Tal}|pw} wie ausgeſtorben. Hier im
                  Hotel\oindex{Cresta Palace@\textbf{Cresta Palace}, \emph{Hotel}|pw} kaum noch 20 Gäſte.\pend
           
\pstart
           Wir grüßen herzlichſt.{\\[\baselineskip]}Ihr \spacefill\mbox{Arthur}\pend
           \leftskip=0em{}\selectlanguage{ngerman}\endnumbering\briefempfaengerindex{Beer-Hofmann, Richard@\textsc{Beer-Hofmann, Richard}!zzzSchnitzler, Arthur@\emph{von Arthur Schnitzler}!1914-08-021@{2. 8. 1914}|)be}\mylabel{L02189h}  \newcommand{\dateiname}{L02189}\newcommand{\titel}{Arthur Schnitzler an Richard Beer-Hofmann, 2. 8. 1914}\newcommand{\editorInnen}{Martin Anton Müller und Gerd-Hermann Susen}%% latex-leseansicht-abspann.tex
%% Abspann für die Leseansicht.
%% Der Schalter \ifkorrekturansicht ist bereits durch den Vorspann gesetzt.

%% latex-abspann.tex
%% Gemeinsamer Abspann für Korrekturansicht und Leseansicht.
%% Setzt den Schalter \ifkorrekturansicht voraus (gesetzt in den
%% einbindenden Dateien latex-korrekturansicht-abspann.tex bzw.
%% latex-leseansicht-abspann.tex).
%% ---------------------------------------------------------------

\normalsize

% Das esempio-Environment wird nur in der Leseansicht benötigt
\ifkorrekturansicht\else
\newenvironment{esempio}[3]%
{
    \vspace{1.5ex}
    \rlap{\underline{#1}}
    \par
    \setlength{\parindent}{0cm}
    \nopagebreak
    \leftskip=#2cm
    \rightskip=#3cm
}
{
    \par
}
\fi

\doendnotes{C}
\bigskip
\vfill

\clearpage

\footnotesize

\ifkorrekturansicht
  \lohead{\textsc{register}}
\fi

% theindex-Environment neu definieren ohne reledmac
\makeatletter
\renewenvironment{theindex}{%
  \ifkorrekturansicht
    \section*{\indexname}%
  \else
    \subsubsection*{Index der erwähnten Entitäten}%
  \fi
  \setlength{\parindent}{0pt}%
  \setlength{\parskip}{0pt plus 0.3pt}%
  \let\item\@idxitem
}{%
  \ifkorrekturansicht\clearpage\fi
}
\makeatother

\IfFileExists{\jobname-pw.ind}{\input{\jobname-pw.ind}}{}

% Quellenangabe nur in der Leseansicht
\ifkorrekturansicht\else
% Fallback-Definitionen, falls die .tex-Datei \titel etc. nicht gesetzt hat
\providecommand{\titel}{}
\providecommand{\editorInnen}{}
\providecommand{\dateiname}{\jobname}

\vspace{3cm}

\vfill

\footnotesize
\textsc{Quelle}: \titel. Herausgegeben von {\editorInnen}. In: \emph{Arthur Schnitzler: Briefwechsel mit Autorinnen und Autoren}.
 Digitale Edition, https://schnitzler-briefe.acdh.oeaw.ac.at/{\dateiname}.html (Stand \today)
\fi

\end{document}


