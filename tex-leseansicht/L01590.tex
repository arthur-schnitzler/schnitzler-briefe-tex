\input{../tex-inputs/latex-pdf-vorspann}
\begin{center}
            \textcolor{red}{ENTWURF. ENTZIFFERUNG NOCH NICHT KORREKTURGELESEN}
                      \end{center}
            
               \section[Arthur Schnitzler an Georg Brandes, 13. 3. 1906]{ Arthur Schnitzler an Georg Brandes, 13. 3. 1906}\nopagebreak\mylabel{v}\rehead{ }\begin{ledgroupsized}[t]{13cm}\normalsize\beginnumbering\briefempfaengerindex{Brandes, Georg@\textsc{Brandes, Georg}!zzzSchnitzler, Arthur@\emph{von Arthur Schnitzler}!1906-03-131@{13. 3. 1906}|(be} \toendnotes[C]{\smallbreak\pagebreak[2]} \Standort{Kopenhagen, Det Kongelige Bibliotek, Georg Brandes Arkiv, box 125.}
\physDesc{Brief, 2 Blätter, 8 Seiten
\newline{}Handschrift: schwarze Tinte, deutsche Kurrent\newline{}Ordnung: mit Bleistift von unbekannter Hand nummeriert:
                                    »25.«, teilweise mit Unterstreichungen
                                 möglicherweise schwierig zu lesender Stellen in blauem
                                 Buntstift }\buchAbdrucke{\weitereDrucke{1) Georg Brandes, Arthur Schnitzler: \emph{Ein Briefwechsel}. Hg. Kurt Bergel. Bern: \emph{Francke} 1956, S. 92–93.} \weitereDrucke{2) Arthur Schnitzler: \emph{Briefe 1875–1912}. Hg. Therese Nickl und Heinrich Schnitzler. Frankfurt am Main: \emph{S. Fischer} 1981, S. 527–528.} }\toendnotes[C]{\smallbreak}\pstart
           \noindent{}{\pb}\textcolor{gray}{\textbf{Dr. Arthur Schnitzler}}\hfill 13. 3. 906\pend
           \pstart
           \textcolor{gray}{\textbf{Wien, XVIII. Spoettelgasse 7\oindex{Edmund-Weiss-Gasse@\textbf{Edmund-Weiß-Gasse}|pw}.}}\pend
           \pstart{}lieber und verehrter Herr Brandes,\pend\pstart
           Ihr Brief hat mir diesmal beſonders wohlgethan. Auch mir iſt der »\textsc{Ruf des Lebens}\pwindex{Schnitzler, Arthur 15.05.1862 – 21.10.1931@\textsc{Schnitzler, Arthur} (15.05.1862 – 21.10.1931), \emph{Schriftsteller, Mediziner}!Ruf des Lebens. Schauspiel in drei Akten1906-02-20@\strich\emph{Der Ruf des Lebens. Schauspiel in drei Akten} {[}1906-02-20{]}|pw}« werth, zum mindeſten in ſeinen erſten zwei Akten; mit dem dritten habe ich
               viel Mühe gehabt, und er iſt doch lange nicht das geworden, was ich wollte. Die Macht
               des »erſten Einfalls« iſt zu groſs; ich ſehe ein, daſs ich {\pb}mich in einem gewiſſen Augenblick von dieſem
               erſten Einfall hätte befreien \introOben{}müſſen\introOben{} und die Sache ſo
               dramatiſch weiterführen, als ich ſie begonnen. Es kam am Ende doch nicht darauf an zu
               ſagen, daſs man auch aus den furchtbarſten Schickſalen emportauchen ka{\geminationn}, daſs wir nur den Widerhall von Worten bringen
               u.ſ.w. –; – aber in Dramen erledigt ein alberner Dolchſtich {\pb}oder ein Fenſterſprung im Wahnſinn alle Dinge viel
               entſcheidender als die tiefſte und glatteſte Weisheit. (Ich ſage: tief und glatt;
               eben die tiefſte bleibt ja glatt, we{\geminationn} wir nicht unſern
               eignen Weg hin gegangen ſind.) Aber was red ich da. Ich bin entfernt davon, Sie von
               Ihrer Sympathie für mein Stück abbringen zu wollen. Ich kann ſie beſſer brauchen als
               je. Was Sie im Tag\orgindex{Tag@Der Tag|pw}{ }geleſen\pwindex{\textcolor{red}{\textsuperscript{XXXX1 indx}}!Ruf des Lebens.« Schauspiel von Artur Schnitzler. Erste Auffuehrung im Lessingtheater27.2.1906 – 27.21906@\strich\emph{»Der Ruf des Lebens.« Schauspiel von Artur Schnitzler. Erste Aufführung im Lessingtheater} {[}27.2.1906 – 27.21906{]}|pwv}, war {\pb}gewiſs nicht das unverſtändigſte – und noch
               gewiſſer nicht das böſeſte, was man mir diesmal nachgeſagt. Da es im 2. Akt knallt
               und da im 1. Akt vergiftet wird, hat man mich \label{K_L01590_1v}\edtext{als Spekulanten bezeichnet}{\lemma{\textnormal{\emph{als … bezeichnet}}}\Cendnote{\textnormal{nicht ermittelt}}}\label{K_L01590_1h}, einen Kerl, der auf dieſe ordinär
               theatraliſche Art durch Tantiemen ein reicher Mann werden möchte. (Eine Spekulation,
               umſo verächtlicher, als ſie nicht geglückt iſt, ſtand irgendwo zu {\pb}leſen.) Knallt es nicht – ſo heißen mich dieſelben
               Leute einen »Novelliſten« u.ſ.w. In Rußland\oindex{Russland@\textbf{Russland}|pw}{ }ſcheint das Stück ſehr gefallen zu haben. – Mir iſt
               im phantaſtiſchen zuweilen ſehr wohl, insbeſondere we{\geminationn}
               ich aus der dü{\geminationn}eren Atmoſphäre des ausſchließlich
               pſychologiſchen hinabgeſtiegen komme.\pend
           \pstart
           Ich hoffe ſehr, Sie heuer noch zu ſehn. Wenn alles gut geht, möcht ich nemlich im
               Sommer mit Frau\pwindex{Schnitzler, Olga 17.01.1882 – 13.01.1970@\textsc{Schnitzler, Olga} (17.01.1882 – 13.01.1970), \emph{Schauspielerin, Sängerin}|pwv} und Kind\pwindex{Schnitzler, Heinrich 09.08.1902 – 12.07.1982@\textsc{Schnitzler, Heinrich} (09.08.1902 – 12.07.1982), \emph{Regisseur, Schauspieler}|pwv} an die däniſche\oindex{Daenemark@\textbf{Dänemark}|pw}{ }{\pb}Küſte. Dieſer \label{K_L01590_2v}\edtext{Sommer 96}{\lemma{\textnormal{\emph{Sommer 96}}}\Cendnote{\textnormal{Schnitzlers erste Reise nach Dänemark\oindex{Daenemark@\textbf{Dänemark}|pwk} und zum
                  Nordkap}}}\label{K_L01590_2h} bleibt für mich eine der mildeſten,
               beruhigendſten Erinnerungen. So wohl wie in jenen Buchenwäldern war mir ſelten zu
               Muthe. Nun hat ſich ja vieles in meiner Exiſtenz gut und ſchön geſtaltet, aber was
               iſt alles in dieſen zehn Jahren geſchehn! Sie ſagen, daſs meine Arbeiten eine ſo
               große Spannweite haben, weil ein Theil dem Tod, der andere der Liebe gewidmet {\pb}ſei. Kein Wunder. In dieſer Spannweite hat nicht
               mehr und nicht weniger Platz als das Leben. Freilich iſt mir ſehr wohl bewußt, daſs
               in dem, was ich bisher geſchrieben, mehr von der Sehnſucht nach dem Leben, von einer
               ſehr tiefen Ahnung und wohl auch von einem Begreifen des Lebens zu ſpüren iſt, als
               vom Leben ſelbſt. »Des Lebens Ruf {\dots} ach, ſeine Fülle
               nicht!« (Suchen Sie nicht etwa, wo der Vers {\pb}ſteht, es iſt ein geſchwindeltes Citat.)\pend
           \pstart
           Leben Sie wohl und ſeien Sie herzlichſt bedankt und gegrüßt{\\[\baselineskip]}von Ihrem{\\[\baselineskip]}\spacefill\mbox{ArthSchnitzler}\pend
           \leftskip=0em{}\endnumbering\briefempfaengerindex{Brandes, Georg@\textsc{Brandes, Georg}!zzzSchnitzler, Arthur@\emph{von Arthur Schnitzler}!1906-03-131@{13. 3. 1906}|)be}\mylabel{h}\end{ledgroupsized}  \newcommand{\dateiname}{L01590}\newcommand{\titel}{Arthur Schnitzler an Georg Brandes, 13. 3. 1906}\newcommand{\editorInnen}{Martin Anton Müller und Gerd-Hermann Susen}\input{../tex-inputs/latex-pdf-abspann}
      