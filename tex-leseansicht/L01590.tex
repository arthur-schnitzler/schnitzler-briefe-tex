%% latex-leseansicht-vorspann.tex
%% Vorspann für die Leseansicht.
%% Lädt die gemeinsame Datei latex-vorspann.tex mit nicht gesetztem Schalter.

\newif\ifkorrekturansicht
\korrekturansichtfalse

\input{../tex-inputs/latex-vorspann}


\section[Arthur Schnitzler an Georg Brandes, 13. 3. 1906]{L01590 Arthur Schnitzler an Georg Brandes, 13. 3. 1906}
\nopagebreak\mylabel{L01590v}
\rehead{ }\normalsize\beginnumbering\briefempfaengerindex{Brandes, Georg@\textsc{Brandes, Georg}!zzzSchnitzler, Arthur@\emph{von Arthur Schnitzler}!1906-03-131@{13. 3. 1906}|(be}
\toendnotes[C]{\smallbreak\pagebreak[2]}
\correspDesc{Versand  durch Arthur Schnitzler am 13. 3. 1906 in Wien
\newline{}Erhalt  durch Georg Brandes im Zeitraum [13. 3. 1906
                  – 17. 3. 1906?] \textbf{Ort fehlend} }\toendnotes[C]{\smallbreak}
\Standort{Kopenhagen, Det Kongelige Bibliotek, Georg Brandes Arkiv, box 125.}
\physDesc{Brief, 2 Blätter, 8 Seiten, 2781 Zeichen
\newline{}Handschrift: schwarze Tinte, deutsche Kurrent
\newline{}Ordnung: mit Bleistift von unbekannter Hand nummeriert:
                                    »25.«, teilweise mit Unterstreichungen
                                 möglicherweise schwierig zu lesender Stellen in blauem
                                 Buntstift }
\buchAbdrucke{\weitereDrucke{1) Georg Brandes, Arthur Schnitzler: \emph{Ein Briefwechsel}. Herausgegeben von Kurt Bergel. Bern: \emph{Francke} 1956, S. 92–93.} \weitereDrucke{2) Arthur Schnitzler: \emph{Briefe 1875–1912}. Herausgegeben von Therese Nickl und Heinrich Schnitzler. Frankfurt am Main: \emph{S. Fischer} 1981, S. 527–528.} }\toendnotes[C]{\smallbreak}
\pstart
           {\pb}\textcolor{gray}{\textbf{Dr. Arthur Schnitzler}}\hfill 13. 3. 906\pend
           
\pstart
           \textcolor{gray}{\textbf{Wien, XVIII. Spoettelgasse 7\oindex{Wien@\textbf{Wien}!XVIII., Währing@\textbf{XVIII., Währing}!Edmund-Weiß-Gasse 7@\textbf{Edmund-Weiß-Gasse 7}, \emph{Wohngebäude}|pw}.}}\pend
           
\pstart{}lieber und verehrter Herr Brandes,\pend\vspace{0.5em}
\pstart
           Ihr Brief hat mir diesmal beſonders wohlgethan. Auch mir iſt der »\textsc{Ruf des Lebens}\pwindex{Schnitzler, Arthur 15.\,5.\,1862 Wien – 21.\,10.\,1931 ebd.@\textsc{Schnitzler, Arthur} (15.\,5.\,1862 Wien – 21.\,10.\,1931 ebd.), \emph{Schriftsteller, Mediziner}!Ruf des Lebens. Schauspiel in drei Akten@\strich\emph{Der Ruf des Lebens. Schauspiel in drei Akten}|pw}« werth, zum mindeſten in{ }ſeinen erſten zwei Akten; mit dem dritten habe ich
               viel Mühe gehabt, und er iſt doch lange nicht das geworden, was ich wollte. Die Macht
               des »erſten Einfalls« iſt zu groſs; ich{ }ſehe ein, daſs ich {\pb}mich in einem gewiſſen Augenblick von dieſem
               erſten Einfall hätte befreien \introOben{}müſſen\introOben{} und die Sache{ }ſo
               dramatiſch weiterführen, als ich{ }ſie begonnen. Es kam am Ende doch nicht darauf an zu{ }ſagen, daſs man auch aus den furchtbarſten Schickſalen emportauchen ka{\geminationn}, daſs wir nur den Widerhall von Worten bringen
               u.ſ.w. –; – aber in Dramen erledigt ein alberner Dolchſtich {\pb}oder ein Fenſterſprung im Wahnſinn alle Dinge viel
               entſcheidender als die tiefſte und glatteſte Weisheit. (Ich{ }ſage: tief und glatt;
               eben die tiefſte bleibt ja glatt, we{\geminationn} wir nicht unſern
               eignen Weg hin gegangen{ }ſind.) Aber was red ich da. Ich bin entfernt davon, Sie von
               Ihrer Sympathie für mein Stück abbringen zu wollen. Ich kann{ }ſie beſſer brauchen als
               je. Was Sie im Tag\orgindex{Tag@Der Tag|pw}{ }geleſen\pwindex{\textcolor{red}{\textsuperscript{XXXX indx1}}!Ruf des Lebens.« Schauspiel von Artur Schnitzler. Erste Aufführung im Lessingtheater@\strich\emph{»Der Ruf des Lebens.« Schauspiel von Artur Schnitzler. Erste Aufführung im Lessingtheater}|pwv}, war {\pb}gewiſs nicht das unverſtändigſte – und noch
               gewiſſer nicht das böſeſte, was man mir diesmal nachgeſagt. Da es im 2. Akt knallt
               und da im 1. Akt vergiftet wird, hat man mich \label{K_L01590-1v}\edtext{als Spekulanten bezeichnet}{\lemma{\textnormal{\emph{als … bezeichnet}}}\Cendnote{\textnormal{nicht ermittelt}}}\label{K_L01590-1}, einen Kerl, der auf dieſe ordinär
               theatraliſche Art durch Tantiemen ein reicher Mann werden möchte. (Eine Spekulation,
               umſo verächtlicher, als{ }ſie nicht geglückt iſt,{ }ſtand irgendwo zu {\pb}leſen.) Knallt es nicht –{ }ſo heißen mich dieſelben
               Leute einen »Novelliſten« u.ſ.w. In Rußland\oindex{Russland@\textbf{Russland}|pw}{ }ſcheint das Stück{ }ſehr gefallen zu haben. – Mir iſt
               im phantaſtiſchen zuweilen{ }ſehr wohl, insbeſondere we{\geminationn}
               ich aus der dü{\geminationn}eren Atmoſphäre des ausſchließlich
               pſychologiſchen hinabgeſtiegen komme.\pend
           
\pstart
           Ich hoffe{ }ſehr, Sie heuer noch zu{ }ſehn. Wenn alles gut geht, möcht ich nemlich im
               Sommer mit Frau\pwindex{Schnitzler, Olga 17.\,1.\,1882 Wien – 13.\,1.\,1970 Lugano@\textsc{Schnitzler, Olga} (17.\,1.\,1882 Wien – 13.\,1.\,1970 Lugano), \emph{Schauspielerin, Sängerin}|pwv} und Kind\pwindex{Schnitzler, Heinrich 9.\,8.\,1902 Hinterbrühl – 12.\,7.\,1982 Wien@\textsc{Schnitzler, Heinrich} (9.\,8.\,1902 Hinterbrühl – 12.\,7.\,1982 Wien), \emph{Regisseur, Schauspieler}|pwv} an die däniſche\oindex{Dänemark@\textbf{Dänemark}|pw}{ }{\pb}Küſte. Dieſer \label{K_L01590-2v}\edtext{Sommer 96}{\lemma{\textnormal{\emph{Sommer 96}}}\Cendnote{\textnormal{Schnitzlers erste Reise nach Dänemark\oindex{Dänemark@\textbf{Dänemark}|pwk} und zum Nordkap\oindex{Nordkap@\textbf{Nordkap}, \emph{Kap}|pwk}}}}\label{K_L01590-2} bleibt für mich eine der mildeſten, beruhigendſten Erinnerungen. So wohl wie
               in jenen Buchenwäldern war mir{ }ſelten zu Muthe. Nun hat{ }ſich ja vieles in meiner
               Exiſtenz gut und{ }ſchön geſtaltet, aber was iſt alles in dieſen zehn Jahren geſchehn!
               Sie{ }ſagen, daſs meine Arbeiten eine{ }ſo große Spannweite haben, weil ein Theil dem
               Tod, der andere der Liebe gewidmet {\pb}ſei. Kein
               Wunder. In dieſer Spannweite hat nicht mehr und nicht weniger Platz als das Leben.
               Freilich iſt mir{ }ſehr wohl bewußt, daſs in dem, was ich bisher geſchrieben, mehr von
               der Sehnſucht nach dem Leben, von einer{ }ſehr tiefen Ahnung und wohl auch von einem
               Begreifen des Lebens zu{ }ſpüren iſt, als vom Leben{ }ſelbſt. »Des Lebens Ruf {\dots} ach,{ }ſeine Fülle nicht!« (Suchen Sie nicht etwa, wo der
               Vers {\pb}ſteht, es iſt ein geſchwindeltes Citat.)\pend
           
\pstart
           Leben Sie wohl und{ }ſeien Sie herzlichſt bedankt und gegrüßt{\\[\baselineskip]}von Ihrem{\\[\baselineskip]}\spacefill\mbox{ArthSchnitzler}\pend
           \leftskip=0em{}\selectlanguage{ngerman}\endnumbering\briefempfaengerindex{Brandes, Georg@\textsc{Brandes, Georg}!zzzSchnitzler, Arthur@\emph{von Arthur Schnitzler}!1906-03-131@{13. 3. 1906}|)be}\mylabel{L01590h}  \newcommand{\dateiname}{L01590}\newcommand{\titel}{Arthur Schnitzler an Georg Brandes, 13. 3. 1906}\newcommand{\editorInnen}{Martin Anton Müller und Gerd-Hermann Susen}%% latex-leseansicht-abspann.tex
%% Abspann für die Leseansicht.
%% Der Schalter \ifkorrekturansicht ist bereits durch den Vorspann gesetzt.

%% latex-abspann.tex
%% Gemeinsamer Abspann für Korrekturansicht und Leseansicht.
%% Setzt den Schalter \ifkorrekturansicht voraus (gesetzt in den
%% einbindenden Dateien latex-korrekturansicht-abspann.tex bzw.
%% latex-leseansicht-abspann.tex).
%% ---------------------------------------------------------------

\normalsize

% Das esempio-Environment wird nur in der Leseansicht benötigt
\ifkorrekturansicht\else
\newenvironment{esempio}[3]%
{
    \vspace{1.5ex}
    \rlap{\underline{#1}}
    \par
    \setlength{\parindent}{0cm}
    \nopagebreak
    \leftskip=#2cm
    \rightskip=#3cm
}
{
    \par
}
\fi

\doendnotes{C}
\bigskip
\vfill

\clearpage

\footnotesize

\ifkorrekturansicht
  \lohead{\textsc{register}}
\fi

% theindex-Environment neu definieren ohne reledmac
\makeatletter
\renewenvironment{theindex}{%
  \ifkorrekturansicht
    \section*{\indexname}%
  \else
    \subsubsection*{Index der erwähnten Entitäten}%
  \fi
  \setlength{\parindent}{0pt}%
  \setlength{\parskip}{0pt plus 0.3pt}%
  \let\item\@idxitem
}{%
  \ifkorrekturansicht\clearpage\fi
}
\makeatother

\IfFileExists{\jobname-pw.ind}{\input{\jobname-pw.ind}}{}

% Quellenangabe nur in der Leseansicht
\ifkorrekturansicht\else
% Fallback-Definitionen, falls die .tex-Datei \titel etc. nicht gesetzt hat
\providecommand{\titel}{}
\providecommand{\editorInnen}{}
\providecommand{\dateiname}{\jobname}

\vspace{3cm}

\vfill

\footnotesize
\textsc{Quelle}: \titel. Herausgegeben von {\editorInnen}. In: \emph{Arthur Schnitzler: Briefwechsel mit Autorinnen und Autoren}.
 Digitale Edition, https://schnitzler-briefe.acdh.oeaw.ac.at/{\dateiname}.html (Stand \today)
\fi

\end{document}


