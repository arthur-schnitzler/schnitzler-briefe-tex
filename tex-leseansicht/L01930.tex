%% latex-korrekturansicht-vorspann.tex
%% Vorspann für die Korrekturansicht.
%% Lädt die gemeinsame Datei latex-vorspann.tex mit gesetztem Schalter.

\newif\ifkorrekturansicht
\korrekturansichttrue

\input{../tex-inputs/latex-vorspann}


\section[Arthur Schnitzler an Hugo von Hofmannsthal, 15. 5. 1910]{L01930 Arthur Schnitzler an Hugo von Hofmannsthal, 15. 5. 1910}
\nopagebreak\mylabel{L01930v}
\rehead{ }\normalsize\beginnumbering\briefempfaengerindex{Hofmannsthal, Hugo von@\textsc{Hofmannsthal, Hugo von}!zzzSchnitzler, Arthur@\emph{von Arthur Schnitzler}!1910-05-151@{5. 5. 1910}|(be}
\toendnotes[C]{\smallbreak\pagebreak[2]}\Standort{FDH, Hs-30885,137.}
\physDesc{Kartenbrief, 590 Zeichen
\newline{}Handschrift: schwarze Tinte, deutsche Kurrent
\newline{}Versand: 1) Stempel: »\nobreak{}\oindex{IX., Alsergrund@\textbf{IX., Alsergrund}, \emph{A.ADM3}|pwk}9/4 Wien, 15. V. 10, 6\nobreak{}«.   2) Stempel: »\nobreak{}\oindex{Rodaun@\textbf{Rodaun}, \emph{A.ADM4}|pwk}Rodaun, 16. V. 10, 6\nobreak{}«. }
\buchAbdrucke{\weitereDrucke{Hugo von Hofmannsthal, Arthur Schnitzler: \emph{Briefwechsel}. Frankfurt am Main: \emph{S. Fischer} 1964, S. 250.} }\toendnotes[C]{\smallbreak}\pstart{}{\pb}Herrn \textsc{Dr Hugo von
                     Hofmannsthal}\pend{}\pstart{}Rodaun\oindex{Rodaun@\textbf{Rodaun}, \emph{A.ADM4}|pw}\pend{}\pstart{}Badgaſſe 5\oindex{Badgasse@\textbf{Badgasse}, \emph{Straße (K.STR)}|pw}.\pend{}{\bigskip}\vspace{1em}
\pstart
           \raggedleft{}{\pb}15/5 910\pend
           
\pstart{}lieber Hugo, \pend\vspace{0.5em}
\pstart
           ich gratulire herzlich; es war ein ſchöner \label{K_L01930-1v}\edtext{Abend}{\lemma{\textnormal{\emph{Abend}}}\Cendnote{\textnormal{Vgl. A. S.: \emph{Tagebuch}, 13. 5. 1910.
               }}}\label{K_L01930-1}. Die Umarbeitung\pwindex{Cristinas Heimreise. Komoedie@\emph{Cristinas Heimreise. Komödie}|pwv} find
               ich in der Anlage famos, aber an einzelnen Stellen noch nicht vollko{\geminationm}en fertig. Vielleicht iſt es nur ein halbes Dutzend
               Worte der \textsc{Cristina}\pwindex{Cristinas Heimreise. Komoedie@\emph{Cristinas Heimreise. Komödie}|pwv}, die mir fehlen – und vielleicht fehlen ſie mir nur, weil ich von dieſer
               anmutvollen Geſtalt noch irgend etwas vernehmen möchte, eh ſie aus der ſchönen Welt
               dieſer Komödie\pwindex{Cristinas Heimreise. Komoedie@\emph{Cristinas Heimreise. Komödie}|pwv}{ }ſcheidet.\pend
           
\pstart
           Wir reiſen Dinſtag in die Schweiz\oindex{Schweiz@\textbf{Schweiz}, \emph{A.PCLI}|pw}
               auf circa 3 Wochen. Und ſehen \substVorne{}\textsuperscript{uns}\substDazwischen{}Sie\substHinten{} hoffentlich bald nach unſrer Rückkehr.\pend
           \pstart Viele Grüße von Haus zu Haus Ihr \spacefill\mbox{A.}\pend{}\selectlanguage{ngerman}\endnumbering\briefempfaengerindex{Hofmannsthal, Hugo von@\textsc{Hofmannsthal, Hugo von}!zzzSchnitzler, Arthur@\emph{von Arthur Schnitzler}!1910-05-151@{5. 5. 1910}|)be}\mylabel{L01930h}  \normalsize

\doendnotes{C}
\bigskip
\vfill

\clearpage

\footnotesize

\lohead{\textsc{register}}

% Definiere theindex-Environment komplett neu ohne reledmac
\makeatletter
\renewenvironment{theindex}{%
  \section*{\indexname}%
  \setlength{\parindent}{0pt}%
  \setlength{\parskip}{0pt plus 0.3pt}%
  \let\item\@idxitem
}{%
  \clearpage
}
\makeatother

\IfFileExists{\jobname-pw.ind}{\input{\jobname-pw.ind}}{}

\end{document}

      