%% latex-leseansicht-vorspann.tex
%% Vorspann für die Leseansicht.
%% Lädt die gemeinsame Datei latex-vorspann.tex mit nicht gesetztem Schalter.

\newif\ifkorrekturansicht
\korrekturansichtfalse

\input{../tex-inputs/latex-vorspann}


         
         \renewcommand{\erwaehntePersonen}{Personen: Hugo von Hofmannsthal}
         \renewcommand{\erwaehnteOrte}{Orte: Badgasse, IX., Alsergrund, Rodaun, Schweiz, Wien}
         \renewcommand{\erwaehnteWerke}{Werke: Cristinas Heimreise. Komödie}
               \section[Arthur Schnitzler an Hugo von Hofmannsthal, 15. 5. 1910]{ Arthur Schnitzler an Hugo von Hofmannsthal, 15. 5. 1910}\nopagebreak\mylabel{v}\rehead{ }\begin{ledgroupsized}[t]{13cm}\normalsize\beginnumbering \toendnotes[C]{\smallbreak\pagebreak[2]} \Standort{FDH, Hs-30885,137.}
\physDesc{Kartenbrief
\newline{}Handschrift: schwarze Tinte, deutsche Kurrent\newline{}Versand: 1) Stempel: »\nobreak{}\oindex{IX., Alsergrund@\textbf{IX., Alsergrund}|pwk}9/4 Wien, 15. V. 10, 6\nobreak{}«.   2) Stempel: »\nobreak{}\oindex{Rodaun@\textbf{Rodaun}|pwk}Rodaun, 16. V. 10, 6\nobreak{}«. }\buchAbdrucke{\weitereDrucke{Hugo von Hofmannsthal, Arthur Schnitzler: \emph{Briefwechsel}. Hg. Therese Nickl und Heinrich Schnitzler. Frankfurt am Main: \emph{S. Fischer} 1964, S. 250.} }\toendnotes[C]{\smallbreak}\pstart{}{\pb}Herrn \textsc{Dr Hugo von
                     Hofmannsthal}\pend{}\pstart{}Rodaun\oindex{Rodaun@\textbf{Rodaun}|pw}\pend{}\pstart{}Badgaſſe 5\oindex{Badgasse@\textbf{Badgasse}|pw}.\pend{}{\bigskip}\pstart
           \raggedleft{}{\pb}15/5 910\pend
           \pstart{}lieber Hugo, \pend\pstart
           ich gratulire herzlich; es war ein ſchöner \label{K_L01930_1v}\edtext{Abend}{\lemma{\textnormal{\emph{Abend}}}\Cendnote{\textnormal{vgl. A. S.: \emph{Tagebuch}, 13. 5. 1910}}}\label{K_L01930_1h}. Die Umarbeitung\pwindex{Hofmannsthal, Hugo von 1874-02-01 – 1929-07-15@\textsc{Hofmannsthal, Hugo von} (1874-02-01 – 1929-07-15), \emph{Schriftsteller}!Cristinas Heimreise. Komoedie11. 2. 1910@\strich\emph{Cristinas Heimreise. Komödie} {[}11. 2. 1910{]}|pwv} find ich in der
               Anlage famos, aber an einzelnen Stellen noch nicht vollko{\geminationm}en fertig. Vielleicht iſt es nur ein halbes Dutzend Worte der \textsc{Cristina}\pwindex{Hofmannsthal, Hugo von 1874-02-01 – 1929-07-15@\textsc{Hofmannsthal, Hugo von} (1874-02-01 – 1929-07-15), \emph{Schriftsteller}!Cristinas Heimreise. Komoedie11. 2. 1910@\strich\emph{Cristinas Heimreise. Komödie} {[}11. 2. 1910{]}|pwv}, die mir
               fehlen – und vielleicht fehlen ſie mir nur, weil ich von dieſer anmutvollen Geſtalt
               noch irgend etwas vernehmen möchte, eh ſie aus der ſchönen Welt dieſer Komödie\pwindex{Hofmannsthal, Hugo von 1874-02-01 – 1929-07-15@\textsc{Hofmannsthal, Hugo von} (1874-02-01 – 1929-07-15), \emph{Schriftsteller}!Cristinas Heimreise. Komoedie11. 2. 1910@\strich\emph{Cristinas Heimreise. Komödie} {[}11. 2. 1910{]}|pwv}{ }ſcheidet.\pend
           \pstart
           Wir reiſen Dinſtag in die Schweiz\oindex{Schweiz@\textbf{Schweiz}|pw} auf
               circa 3 Wochen. Und ſehen \substVorne{}\textsuperscript{uns}\substDazwischen{}Sie\substHinten{} hoffentlich bald nach unſrer Rückkehr.\pend
           \pstart Viele Grüße von Haus zu Haus Ihr \spacefill\mbox{A.}\pend{}
         
         \endnumbering\mylabel{h}\end{ledgroupsized}  \newcommand{\dateiname}{L01930}\newcommand{\titel}{Arthur Schnitzler an Hugo von Hofmannsthal, 15. 5. 1910}\newcommand{\editorInnen}{Martin Anton Müller und Gerd-Hermann Susen}%% latex-leseansicht-abspann.tex
%% Abspann für die Leseansicht.
%% Der Schalter \ifkorrekturansicht ist bereits durch den Vorspann gesetzt.

%% latex-abspann.tex
%% Gemeinsamer Abspann für Korrekturansicht und Leseansicht.
%% Setzt den Schalter \ifkorrekturansicht voraus (gesetzt in den
%% einbindenden Dateien latex-korrekturansicht-abspann.tex bzw.
%% latex-leseansicht-abspann.tex).
%% ---------------------------------------------------------------

\normalsize

% Das esempio-Environment wird nur in der Leseansicht benötigt
\ifkorrekturansicht\else
\newenvironment{esempio}[3]%
{
    \vspace{1.5ex}
    \rlap{\underline{#1}}
    \par
    \setlength{\parindent}{0cm}
    \nopagebreak
    \leftskip=#2cm
    \rightskip=#3cm
}
{
    \par
}
\fi

\doendnotes{C}
\bigskip
\vfill

\clearpage

\footnotesize

\ifkorrekturansicht
  \lohead{\textsc{register}}
\fi

% theindex-Environment neu definieren ohne reledmac
\makeatletter
\renewenvironment{theindex}{%
  \ifkorrekturansicht
    \section*{\indexname}%
  \else
    \subsubsection*{Index der erwähnten Entitäten}%
  \fi
  \setlength{\parindent}{0pt}%
  \setlength{\parskip}{0pt plus 0.3pt}%
  \let\item\@idxitem
}{%
  \ifkorrekturansicht\clearpage\fi
}
\makeatother

\IfFileExists{\jobname-pw.ind}{\input{\jobname-pw.ind}}{}

% Quellenangabe nur in der Leseansicht
\ifkorrekturansicht\else
% Fallback-Definitionen, falls die .tex-Datei \titel etc. nicht gesetzt hat
\providecommand{\titel}{}
\providecommand{\editorInnen}{}
\providecommand{\dateiname}{\jobname}

\vspace{3cm}

\vfill

\footnotesize
\textsc{Quelle}: \titel. Herausgegeben von {\editorInnen}. In: \emph{Arthur Schnitzler: Briefwechsel mit Autorinnen und Autoren}.
 Digitale Edition, https://schnitzler-briefe.acdh.oeaw.ac.at/{\dateiname}.html (Stand \today)
\fi

\end{document}


      