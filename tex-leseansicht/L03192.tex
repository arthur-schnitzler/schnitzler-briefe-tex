%% latex-leseansicht-vorspann.tex
%% Vorspann für die Leseansicht.
%% Lädt die gemeinsame Datei latex-vorspann.tex mit nicht gesetztem Schalter.

\newif\ifkorrekturansicht
\korrekturansichtfalse

\input{../tex-inputs/latex-vorspann}


\section[ Paul Goldmann an Arthur Schnitzler, 14. 1. {[}1902{]}]{L03192 Paul Goldmann an Arthur Schnitzler,  14. 1. [1902]}
\nopagebreak\mylabel{L03192v}
\rehead{ }\normalsize\beginnumbering\briefempfaengerindex{Schnitzler, Arthur@\textsc{Schnitzler, Arthur}!zzzGoldmann, Paul@\emph{von Paul Goldmann}!1902-01-141@{14. 1. [1902]}|(be}
\toendnotes[C]{\smallbreak\pagebreak[2]}
\correspDesc{Versand  durch Paul Goldmann am 14. 1. [1902] in Berlin
\newline{}Erhalt  durch Arthur Schnitzler am [15. 1. 1902?] in Wien}\toendnotes[C]{\smallbreak}
\Standort{DLA, A:Schnitzler, HS.NZ85.1.3172.}
\physDesc{Brief, 1 Blatt, 3 Seiten, 1305 Zeichen
\newline{}Handschrift: blaue Tinte, deutsche Kurrent
\newline{}Schnitzler: 1) mit Bleistift das Jahr »902« vermerkt  2) mit rotem Buntstift vier Unterstreichungen}\toendnotes[C]{\smallbreak}
\pstart
           \raggedleft{}{\pb}\textcolor{gray}{\textbf{DESSAUERSTRASSE 19}}\oindex{Dessauer Straße@\textbf{Dessauer Straße}, \emph{Straße}|pw}\pend
           
\pstart
           Berlin\oindex{Berlin@\textbf{Berlin}, \emph{Hauptstadt}|pw}, 14. Januar.\pend
           
\pstart\center{}Mein lieber Freund,\pend\vspace{0.5em}
\pstart
           In Eile – denn ich habe unbeſchreiblich viel zu thun – Dank für Deine lieben Briefe!
               Es freut mich, daß es \textsc{Olga\pwindex{Schnitzler, Olga 17.\,1.\,1882 Wien – 13.\,1.\,1970 Lugano@\textsc{Schnitzler, Olga} (17.\,1.\,1882 Wien – 13.\,1.\,1970 Lugano), \emph{Schauspielerin, Sängerin}|pw}} gut geht und daß Ihr demnächſt \label{K_L03192-1v}\edtext{aufs Land ziehen}{\lemma{\textnormal{\emph{aufs Land ziehen}}}\Cendnote{\textnormal{Olga Gussmann\pwindex{Schnitzler, Olga 17.\,1.\,1882 Wien – 13.\,1.\,1970 Lugano@\textsc{Schnitzler, Olga} (17.\,1.\,1882 Wien – 13.\,1.\,1970 Lugano), \emph{Schauspielerin, Sängerin}|pwk} war erneut schwanger. Auch sie
                  sollte, wie bereits Marie Reinhard\pwindex{Reinhard, Marie 13.\,3.\,1871 Wien – 18.\,3.\,1899 ebd.@\textsc{Reinhard, Marie} (13.\,3.\,1871 Wien – 18.\,3.\,1899 ebd.), \emph{Gesangspädagogin}|pwk} im Jahre
                     1897, außerhalb Wiens\oindex{Wien@\textbf{Wien}, \emph{Verwaltungsgebiet}|pwk} gebären. Dafür suchte Schnitzler eine geeignete Unterkunft. Am 3. 2. 1902 zogen Olga\pwindex{Schnitzler, Olga 17.\,1.\,1882 Wien – 13.\,1.\,1970 Lugano@\textsc{Schnitzler, Olga} (17.\,1.\,1882 Wien – 13.\,1.\,1970 Lugano), \emph{Schauspielerin, Sängerin}|pwk} und ihre Schwester Elisabeth\pwindex{Steinrück, Elisabeth 19.\,11.\,1885 – 7.\,4.\,1920 Partenkirchen@\textsc{Steinrück, Elisabeth} (19.\,11.\,1885 – 7.\,4.\,1920 Partenkirchen)|pwk} vorübergehend in ein Mödling\oindex{Mödling@\textbf{Mödling}, \emph{Hauptstadt}|pwk}er Kurhaus\oindex{Hotel Kursalon Mödling@\textbf{Hotel Kursalon Mödling}, \emph{Hotel}|pwk}, dann zu Christine Schönberger\pwindex{Schönberger, Christine 17.\,11.\,1875 – 3.\,2.\,1971 Wien@\textsc{Schönberger, Christine} (17.\,11.\,1875 – 3.\,2.\,1971 Wien), \emph{Gastwirtin}|pwk} in das Wirtshaus Zum goldenen Stern\oindex{Zum goldenen Stern@\textbf{Zum goldenen Stern}, \emph{Lokal}|pwk} (vgl. A. S.: \emph{Tagebuch}, 1. 3. 1902 und XXXX Auszeichnungsfehler: Dokument L01206 nicht gefunden). Im März fand Schnitzler schließlich
                  eine Villa\oindex{Hauptstraße 56@\textbf{Hauptstraße 56}, \emph{Wohngebäude}|pwk} in der Hinterbrühl\oindex{Hinterbrühl@\textbf{Hinterbrühl}, \emph{Hauptstadt}|pwk} (vgl. A. S.: \emph{Tagebuch}, 21. 3. 1902), wo Olga\pwindex{Schnitzler, Olga 17.\,1.\,1882 Wien – 13.\,1.\,1970 Lugano@\textsc{Schnitzler, Olga} (17.\,1.\,1882 Wien – 13.\,1.\,1970 Lugano), \emph{Schauspielerin, Sängerin}|pwk} am 9. 8. 1902{ }Heinrich Schnitzler\pwindex{Schnitzler, Heinrich 9.\,8.\,1902 Hinterbrühl – 12.\,7.\,1982 Wien@\textsc{Schnitzler, Heinrich} (9.\,8.\,1902 Hinterbrühl – 12.\,7.\,1982 Wien), \emph{Regisseur, Schauspieler}|pwk} zur Welt
                  brachte.}}}\label{K_L03192-1} wollt. Wird Euch Beiden wohlthun. Mit \textsc{Liesl\pwindex{Steinrück, Elisabeth 19.\,11.\,1885 – 7.\,4.\,1920 Partenkirchen@\textsc{Steinrück, Elisabeth} (19.\,11.\,1885 – 7.\,4.\,1920 Partenkirchen)|pw}} iſt es ein Kreuz. Wäre{ }ſie nur{ }ſchon \label{K_L03192-2v}\edtext{fertig}{\lemma{\textnormal{\emph{fertig}}}\Cendnote{\textnormal{Elisabeth Gussmann\pwindex{Steinrück, Elisabeth 19.\,11.\,1885 – 7.\,4.\,1920 Partenkirchen@\textsc{Steinrück, Elisabeth} (19.\,11.\,1885 – 7.\,4.\,1920 Partenkirchen)|pwk} wurde finanziell von Schnitzler erhalten; ihre Ausbildung war noch
                  nicht fertig. Siehe XXXX Auszeichnungsfehler: Dokument L03059 nicht gefunden und
                     A. S.: \emph{Tagebuch}, 11. 1. 1902. }}}\label{K_L03192-2}! Setzt \strikeout{Ihr do} ihr doch einmal ordentlich den Kopf zurecht!\pend
           
\pstart
           Daß \label{K_L03192-3v}\edtext{\textsc{Brahm\pwindex{Brahm, Otto 5.\,2.\,1856 Hamburg – 28.\,11.\,1912 Berlin@\textsc{Brahm, Otto} (5.\,2.\,1856 Hamburg – 28.\,11.\,1912 Berlin), \emph{Theaterleiter, Regisseur}|pw}} nach Wien\oindex{Wien@\textbf{Wien}, \emph{Verwaltungsgebiet}|pw} kommt, \strikeout{will ich} um Deine Stücke\pwindex{Schnitzler, Arthur 15.\,5.\,1862 Wien – 21.\,10.\,1931 ebd.@\textsc{Schnitzler, Arthur} (15.\,5.\,1862 Wien – 21.\,10.\,1931 ebd.), \emph{Schriftsteller, Mediziner}!Lebendige Stunden. Vier Einakter@\strich\emph{Lebendige Stunden. Vier Einakter}|pwv} aufzuführen}{\lemma{\textnormal{\emph{Brahm … aufzuführen}}}\Cendnote{\textnormal{Das \emph{Deutsche Theater Berlin}\orgindex{Deutsches Theater Berlin@Deutsches Theater Berlin|pwk} gastierte 1902 am Wien\oindex{Wien@\textbf{Wien}, \emph{Verwaltungsgebiet}|pwk}er Carl-Theater\oindex{Wien@\textbf{Wien}!II., Leopoldstadt@\textbf{II., Leopoldstadt}!Carl-Theater@\textbf{Carl-Theater}, \emph{Theater}|pwk}. Die Premiere von \emph{Lebendige Stunden}\pwindex{Schnitzler, Arthur 15.\,5.\,1862 Wien – 21.\,10.\,1931 ebd.@\textsc{Schnitzler, Arthur} (15.\,5.\,1862 Wien – 21.\,10.\,1931 ebd.), \emph{Schriftsteller, Mediziner}!Lebendige Stunden. Vier Einakter@\strich\emph{Lebendige Stunden. Vier Einakter}|pwk} fand dort am 6. 5. 1902
                  statt.}}}\label{K_L03192-3}, {\pb}will ich nur \label{K_L03192-4v}\edtext{melden\pwindex{Kleine Chronik. [Das Wiener Gastspiel des Berliner Deutschen Theaters.]@\emph{Kleine Chronik. [Das Wiener Gastspiel des Berliner Deutschen Theaters.]}|pwv}}{\lemma{\textnormal{\emph{melden}}}\Cendnote{\textnormal{[Paul Goldmann\pwindex{Goldmann, Paul 31.\,1.\,1865 Breslau – 25.\,9.\,1935 Wien@\textsc{Goldmann, Paul} (31.\,1.\,1865 Breslau – 25.\,9.\,1935 Wien), \emph{Schriftsteller, Journalist}|pwk}]: \emph{Kleine Chronik. [Das Wiener Gastspiel des Berliner
                        Deutschen Theaters]}\pwindex{Kleine Chronik. [Das Wiener Gastspiel des Berliner Deutschen Theaters.]@\emph{Kleine Chronik. [Das Wiener Gastspiel des Berliner Deutschen Theaters.]}|pwk}. In: \emph{Neue Freie
                        Presse}\pwindex{Neue Freie Presse@\emph{Neue Freie Presse}|pwk}, Nr. 13.433, 17. 1. 1902,
                     Abendblatt, S. 1.}}}\label{K_L03192-4}, wenn Du meinſt, es könnte für Dich irgendwie
               von Nutzen{ }ſein. Eine »Nachricht« will ich von Dir nicht haben; Du haſt mich \strikeout{\textcolor{gray}{falß}} mißverſtanden. Wenn ich alſo bis Donnerſtag von
               Dir nichts höre, werde ich \strikeout{nach Wien\oindex{Wien@\textbf{Wien}, \emph{Verwaltungsgebiet}|pw}} annehmen, daß es Dir angemeſſen erſcheint, wenn ich die Meldung\pwindex{Kleine Chronik. [Das Wiener Gastspiel des Berliner Deutschen Theaters.]@\emph{Kleine Chronik. [Das Wiener Gastspiel des Berliner Deutschen Theaters.]}|pwv} nach Wien\oindex{Wien@\textbf{Wien}, \emph{Verwaltungsgebiet}|pw}{ }ſende, und werde{ }ſie abtelegraphiren.\pend
           
\pstart
           Ich hab\textcolor{gray}{e} bereits angefangen, das \label{K_L03192-5v}\edtext{Feuilleton\pwindex{Goldmann, Paul 31.\,1.\,1865 Breslau – 25.\,9.\,1935 Wien@\textsc{Goldmann, Paul} (31.\,1.\,1865 Breslau – 25.\,9.\,1935 Wien), \emph{Schriftsteller, Journalist}!Berliner Theater. (»Lebendige Stunden« von Arthur Schnitzler.)@\strich\emph{Berliner Theater. (»Lebendige Stunden« von Arthur Schnitzler.)}|pwv}}{\lemma{\textnormal{\emph{Feuilleton}}}\Cendnote{\textnormal{Paul Goldmann\pwindex{Goldmann, Paul 31.\,1.\,1865 Breslau – 25.\,9.\,1935 Wien@\textsc{Goldmann, Paul} (31.\,1.\,1865 Breslau – 25.\,9.\,1935 Wien), \emph{Schriftsteller, Journalist}|pwk}: \emph{Berliner Theater. (»Lebendige Stunden« von Arthur
                        Schnitzler)}\pwindex{Goldmann, Paul 31.\,1.\,1865 Breslau – 25.\,9.\,1935 Wien@\textsc{Goldmann, Paul} (31.\,1.\,1865 Breslau – 25.\,9.\,1935 Wien), \emph{Schriftsteller, Journalist}!Berliner Theater. (»Lebendige Stunden« von Arthur Schnitzler.)@\strich\emph{Berliner Theater. (»Lebendige Stunden« von Arthur Schnitzler.)}|pwk}. In: \emph{Neue Freie
                        Presse}\pwindex{Neue Freie Presse@\emph{Neue Freie Presse}|pwk}, Nr. 13.438, 22. 1. 1902,
                     Morgenblatt, S. 1–4. Schnitzler
                  ärgerte sich über das kritische Feuilleton\pwindex{Goldmann, Paul 31.\,1.\,1865 Breslau – 25.\,9.\,1935 Wien@\textsc{Goldmann, Paul} (31.\,1.\,1865 Breslau – 25.\,9.\,1935 Wien), \emph{Schriftsteller, Journalist}!Berliner Theater. (»Lebendige Stunden« von Arthur Schnitzler.)@\strich\emph{Berliner Theater. (»Lebendige Stunden« von Arthur Schnitzler.)}|pwkv} (vgl. A. S.: \emph{Tagebuch}, 22. 1. 1902 und 28. 1. 1902), das die Beziehung der beiden über Jahre hinweg – noch
                  bis zum großen Streit Ende 1910/Anfang 1911 – belasten sollte.}}}\label{K_L03192-5} über Deine Stücke\pwindex{Schnitzler, Arthur 15.\,5.\,1862 Wien – 21.\,10.\,1931 ebd.@\textsc{Schnitzler, Arthur} (15.\,5.\,1862 Wien – 21.\,10.\,1931 ebd.), \emph{Schriftsteller, Mediziner}!Lebendige Stunden. Vier Einakter@\strich\emph{Lebendige Stunden. Vier Einakter}|pwv} zu{ }ſchreiben, bin aber nicht über
               die erſten Zeilen herausgekommen. Unabläſſig wird mir die Feder {\pb}aus der Hand geriſſen. Die Arbeit{ }ſelbſt iſt die{ }ſchwerſte, die ich je gemacht. Ich muß mich zwingen (und das iſt ein harter Zwang),
               mit eiſiger Kälte zu erwägen, und mich auszudrücken und muß mir einreden, daß ich
               über die Stücke\pwindex{Schnitzler, Arthur 15.\,5.\,1862 Wien – 21.\,10.\,1931 ebd.@\textsc{Schnitzler, Arthur} (15.\,5.\,1862 Wien – 21.\,10.\,1931 ebd.), \emph{Schriftsteller, Mediziner}!Lebendige Stunden. Vier Einakter@\strich\emph{Lebendige Stunden. Vier Einakter}|pwv} eines mir
               unbekannten Herrn \textsc{Arthur Schnitzler}{ }ſchreibe. Wenn die
               Parlamentsſeſſion{ }ſo weiter geht, – dann weiß Gott, wann ich fertig werde.\pend
           
\pstart
           Grüße mir \textsc{Olga\pwindex{Schnitzler, Olga 17.\,1.\,1882 Wien – 13.\,1.\,1970 Lugano@\textsc{Schnitzler, Olga} (17.\,1.\,1882 Wien – 13.\,1.\,1970 Lugano), \emph{Schauspielerin, Sängerin}|pw}} und{ }ſei{ }ſelbſt von Herzen gegrüßt! {\\[\baselineskip]}Dein \spacefill\mbox{Paul Goldmnn}\pend
           \leftskip=0em{}\selectlanguage{ngerman}\endnumbering\briefempfaengerindex{Schnitzler, Arthur@\textsc{Schnitzler, Arthur}!zzzGoldmann, Paul@\emph{von Paul Goldmann}!1902-01-141@{14. 1. [1902]}|)be}\mylabel{L03192h}  \newcommand{\dateiname}{L03192}\newcommand{\titel}{Paul Goldmann an Arthur Schnitzler, 14. 1. [1902]}\newcommand{\editorInnen}{Martin Anton Müller und Laura Untner}%% latex-leseansicht-abspann.tex
%% Abspann für die Leseansicht.
%% Der Schalter \ifkorrekturansicht ist bereits durch den Vorspann gesetzt.

%% latex-abspann.tex
%% Gemeinsamer Abspann für Korrekturansicht und Leseansicht.
%% Setzt den Schalter \ifkorrekturansicht voraus (gesetzt in den
%% einbindenden Dateien latex-korrekturansicht-abspann.tex bzw.
%% latex-leseansicht-abspann.tex).
%% ---------------------------------------------------------------

\normalsize

% Das esempio-Environment wird nur in der Leseansicht benötigt
\ifkorrekturansicht\else
\newenvironment{esempio}[3]%
{
    \vspace{1.5ex}
    \rlap{\underline{#1}}
    \par
    \setlength{\parindent}{0cm}
    \nopagebreak
    \leftskip=#2cm
    \rightskip=#3cm
}
{
    \par
}
\fi

\doendnotes{C}
\bigskip
\vfill

\clearpage

\footnotesize

\ifkorrekturansicht
  \lohead{\textsc{register}}
\fi

% theindex-Environment neu definieren ohne reledmac
\makeatletter
\renewenvironment{theindex}{%
  \ifkorrekturansicht
    \section*{\indexname}%
  \else
    \subsubsection*{Index der erwähnten Entitäten}%
  \fi
  \setlength{\parindent}{0pt}%
  \setlength{\parskip}{0pt plus 0.3pt}%
  \let\item\@idxitem
}{%
  \ifkorrekturansicht\clearpage\fi
}
\makeatother

\IfFileExists{\jobname-pw.ind}{\input{\jobname-pw.ind}}{}

% Quellenangabe nur in der Leseansicht
\ifkorrekturansicht\else
% Fallback-Definitionen, falls die .tex-Datei \titel etc. nicht gesetzt hat
\providecommand{\titel}{}
\providecommand{\editorInnen}{}
\providecommand{\dateiname}{\jobname}

\vspace{3cm}

\vfill

\footnotesize
\textsc{Quelle}: \titel. Herausgegeben von {\editorInnen}. In: \emph{Arthur Schnitzler: Briefwechsel mit Autorinnen und Autoren}.
 Digitale Edition, https://schnitzler-briefe.acdh.oeaw.ac.at/{\dateiname}.html (Stand \today)
\fi

\end{document}


