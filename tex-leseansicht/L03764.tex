%% latex-korrekturansicht-vorspann.tex
%% Vorspann für die Korrekturansicht.
%% Lädt die gemeinsame Datei latex-vorspann.tex mit gesetztem Schalter.

\newif\ifkorrekturansicht
\korrekturansichttrue

\input{../tex-inputs/latex-vorspann}


\section[Arthur Schnitzler an Stefan Zweig, 24. 7. 1910]{L03764 Arthur Schnitzler an Stefan Zweig, 24. 7. 1910}
\nopagebreak\mylabel{L03764v}
\rehead{ }\normalsize\beginnumbering\briefempfaengerindex{Zweig, Stefan@\textsc{Zweig, Stefan}!zzzSchnitzler, Arthur@\emph{von Arthur Schnitzler}!1910-07-241@{24. 7. 1910}|(be}
\toendnotes[C]{\smallbreak\pagebreak[2]}\Standort{Jerusalem, National Library of Israel, ARC. Ms. Var. 305 1 58 Stefan Zweig Collection.}
\physDesc{Brief, 1 Blatt, 2 Seiten, 528 Zeichen
\newline{}Handschrift: schwarze Tinte, deutsche Kurrent}\toendnotes[C]{\smallbreak}
\pstart
           {\pb}\textcolor{gray}{\textbf{Dr. Arthur Schnitzler}}\hfill \uline{\textsc{XVIII. Sternwartestr 71\oindex{Sternwartestrasse 71@\textbf{Sternwartestraße 71}, \emph{Wohngebäude (K.WHS)}|pw}},}\pend
           
\pstart
           \textcolor{gray}{\textbf{\strikeout{Wien XVIII. Spoettelgasse 7\oindex{Edmund-Weiss-Gasse@\textbf{Edmund-Weiß-Gasse}, \emph{R.ST}|pw}.}}}\hfill 24. 7. 1910\pend
           \vspace{0.5em}
\pstart
           lieber Herr Doctor, in \label{K_L03764-1v}\edtext{Überſiedlgsfreud – u. leid}{\lemma{\textnormal{\emph{Überſiedlgsfreud – u. leid}}}\Cendnote{\textnormal{Wie auch aus den handschriftlichen
                  Änderungen am gedruckten Briefkopf erkenntlich ist, war Schnitzler mit seiner Familie\pwindex{Schnitzler, Olga 17.01.1882 – 13.01.1970@\textsc{Schnitzler, Olga} (17.01.1882 – 13.01.1970), \emph{Schauspieler/Schauspielerin, Sänger/Sängerin}|pwkv}\pwindex{Cappellini, Lili 13.09.1909 – 26.07.1928@\textsc{Cappellini, Lili} (13.09.1909 – 26.07.1928)|pwkv}\pwindex{Schnitzler, Heinrich 09.08.1902 – 12.07.1982@\textsc{Schnitzler, Heinrich} (09.08.1902 – 12.07.1982), \emph{Regisseur/Regisseurin, Schauspieler/Schauspielerin}|pwkv}
                  eben umgezogen. Am 13. 7. 1910 siedelten sie in ein eigenes Haus in 
                  der Sternwartestraße 71\oindex{Sternwartestrasse 71@\textbf{Sternwartestraße 71}, \emph{Wohngebäude (K.WHS)}|pwk}, um die Ecke der alten Wohnung.}}}\label{K_L03764-1} bin ich nicht dazu
               geko{\geminationm}en, Ihnen für die freundliche Überſendg Ihrer \label{K_L03764-2v}\edtext{\textsc{Verhaeren}\pwindex{Verhaeren, Emile 21.05.1855 – 27.11.1916@\textsc{Verhaeren, Émile} (21.05.1855 – 27.11.1916), \emph{Schriftsteller/Schriftstellerin, Schriftsteller/Schriftstellerin, Krimiautor/Krimiautorin}|pw}-Nachdichtungen\pwindex{Ausgewaehlte Gedichte. In Nachdichtung@\emph{Ausgewählte Gedichte. In Nachdichtung}|pwv}}{\lemma{\textnormal{\emph{Verhaeren-Nachdichtungen}}}\Cendnote{\textnormal{Zweig\pwindex{Zweig, Stefan 28.11.1881 – 23.02.1942@\textsc{Zweig, Stefan} (28.11.1881 – 23.02.1942), \emph{Schriftsteller/Schriftstellerin}|pwk} hatte 1904 seine Auswahl
                  von Gedichtübersetzungen von Émile
                     Verhaeren\pwindex{Verhaeren, Emile 21.05.1855 – 27.11.1916@\textsc{Verhaeren, Émile} (21.05.1855 – 27.11.1916), \emph{Schriftsteller/Schriftstellerin, Schriftsteller/Schriftstellerin, Krimiautor/Krimiautorin}|pwk} (»Ausgewählte Gedichte. In Nachdichtung von Stefan Zweig\pwindex{Zweig, Stefan 28.11.1881 – 23.02.1942@\textsc{Zweig, Stefan} (28.11.1881 – 23.02.1942), \emph{Schriftsteller/Schriftstellerin}|pw}«) im Verlag \emph{Schuster {\kaufmannsund} und Loeffler}\orgindex{Schuster und Loeffler@Schuster {\kaufmannsund}  Loeffler|pwk} herausgebracht.
                     1910 erschien eine Neuausgabe, diesmal im \emph{Insel-Verlag}\orgindex{Insel Verlag@Insel Verlag|pwk}. Am 13. 6. 1910 erhielt der
                  Verleger Anton Kippenberg\pwindex{Kippenberg, Anton 22.05.1874 – 22.09.1950@\textsc{Kippenberg, Anton} (22.05.1874 – 22.09.1950), \emph{Kritiker/Kritikerin, Übersetzer/Übersetzerin, Verleger/Verlegerin}|pwk} von Zweig\pwindex{Zweig, Stefan 28.11.1881 – 23.02.1942@\textsc{Zweig, Stefan} (28.11.1881 – 23.02.1942), \emph{Schriftsteller/Schriftstellerin}|pwk} die Nachricht, die Liste für die zu
                  verschickenden Bücher »nächster Tage« zu senden. (Anton Kippenberg\pwindex{Kippenberg, Anton 22.05.1874 – 22.09.1950@\textsc{Kippenberg, Anton} (22.05.1874 – 22.09.1950), \emph{Kritiker/Kritikerin, Übersetzer/Übersetzerin, Verleger/Verlegerin}|pwk}, Stefan Zweig\pwindex{Zweig, Stefan 28.11.1881 – 23.02.1942@\textsc{Zweig, Stefan} (28.11.1881 – 23.02.1942), \emph{Schriftsteller/Schriftstellerin}|pwk}: \emph{Briefwechsel
                        1905–1937}. Ausgewählt von Oliver Matuschek und Klemens Renoldner.
                     Hg. und kommentiert von Oliver Matuschek unter Mitwirkung von Klemens
                     Renoldner. Berlin: \emph{Insel Verlag}{ }2022, S. 107.) Entsprechend dürfte Schnitzler das Buch in der zweiten
                     Hälfte des Juni erhalten haben.}}}\label{K_L03764-2} zu danken – ich thu es nun, mit
               verſpäteter Herzlichkeit, und freu mich ſehr darauf das Buch\pwindex{Ausgewaehlte Gedichte. In Nachdichtung@\emph{Ausgewählte Gedichte. In Nachdichtung}|pwv} (es ſieht wunderſchön aus) in einigen
               der nächſten ruhigen Stunden, {\pb}wahrſcheinlich auf einer
               kleinen Reiſe, zu leſen.\pend
           
\pstart
           Ich hoffe, Sie haben einen ſchönen So{\geminationm}er vor ſich und arbeiten nicht nur zu andrer ſondern
               auch zu eigenem Ruhm.\pend
           
\pstart
           Herzlichen Gruſs{\\[\baselineskip]}Ihr{\\[\baselineskip]}\spacefill\mbox{Arth Schnitzler}\pend
           \leftskip=0em{}\selectlanguage{ngerman}\endnumbering\briefempfaengerindex{Zweig, Stefan@\textsc{Zweig, Stefan}!zzzSchnitzler, Arthur@\emph{von Arthur Schnitzler}!1910-07-241@{24. 7. 1910}|)be}\mylabel{L03764h}  \normalsize

\doendnotes{C}
\bigskip
\vfill

\clearpage

\footnotesize

\lohead{\textsc{register}}

% Definiere theindex-Environment komplett neu ohne reledmac
\makeatletter
\renewenvironment{theindex}{%
  \section*{\indexname}%
  \setlength{\parindent}{0pt}%
  \setlength{\parskip}{0pt plus 0.3pt}%
  \let\item\@idxitem
}{%
  \clearpage
}
\makeatother

\IfFileExists{\jobname-pw.ind}{\input{\jobname-pw.ind}}{}

\end{document}

      