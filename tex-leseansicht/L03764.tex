%% latex-leseansicht-vorspann.tex
%% Vorspann für die Leseansicht.
%% Lädt die gemeinsame Datei latex-vorspann.tex mit nicht gesetztem Schalter.

\newif\ifkorrekturansicht
\korrekturansichtfalse

\input{../tex-inputs/latex-vorspann}


\section[Arthur Schnitzler an Stefan Zweig, 24. 7. 1910]{L03764 Arthur Schnitzler an Stefan Zweig, 24. 7. 1910}
\nopagebreak\mylabel{L03764v}
\rehead{ }\normalsize\beginnumbering\briefempfaengerindex{Zweig, Stefan@\textsc{Zweig, Stefan}!zzzSchnitzler, Arthur@\emph{von Arthur Schnitzler}!1910-07-241@{24. 7. 1910}|(be}
\toendnotes[C]{\smallbreak\pagebreak[2]}
\correspDesc{Versand  durch Arthur Schnitzler am 24. 7. 1910 in Wien
\newline{}Erhalt  durch Stefan Zweig im Zeitraum [24. 7. 1910
                  – 27. 7. 1910?] in Wien}\toendnotes[C]{\smallbreak}
\Standort{Jerusalem, National Library of Israel, ARC. Ms. Var. 305 1 58 Stefan Zweig Collection.}
\physDesc{Brief, 1 Blatt, 2 Seiten, 528 Zeichen
\newline{}Handschrift: schwarze Tinte, deutsche Kurrent}\toendnotes[C]{\smallbreak}
\pstart
           {\pb}\textcolor{gray}{\textbf{Dr. Arthur Schnitzler}}\hfill \uline{\textsc{XVIII. Sternwartestr 71\oindex{Wien@\textbf{Wien}!XVIII., Währing@\textbf{XVIII., Währing}!Sternwartestraße 71@\textbf{Sternwartestraße 71}, \emph{Wohngebäude}|pw}},}\pend
           
\pstart
           \textcolor{gray}{\textbf{\strikeout{Wien XVIII. Spoettelgasse 7\oindex{Wien@\textbf{Wien}!XVIII., Währing@\textbf{XVIII., Währing}!Edmund-Weiß-Gasse@\textbf{Edmund-Weiß-Gasse}, \emph{Straße}|pw}.}}}\hfill 24. 7. 1910\pend
           \vspace{0.5em}
\pstart
           lieber Herr Doctor, in \label{K_L03764-1v}\edtext{Überſiedlgsfreud – u. leid}{\lemma{\textnormal{\emph{Übersiedlgsfreud – u. leid}}}\Cendnote{\textnormal{Wie auch aus den handschriftlichen
                  Änderungen am gedruckten Briefkopf erkenntlich ist, war Schnitzler mit seiner Familie\pwindex{Schnitzler, Olga 17.\,1.\,1882 Wien – 13.\,1.\,1970 Lugano@\textsc{Schnitzler, Olga} (17.\,1.\,1882 Wien – 13.\,1.\,1970 Lugano), \emph{Schauspielerin, Sängerin}|pwkv}\pwindex{Cappellini, Lili 13.\,9.\,1909 Wien – 26.\,7.\,1928 Venedig@\textsc{Cappellini, Lili} (13.\,9.\,1909 Wien – 26.\,7.\,1928 Venedig)|pwkv}\pwindex{Schnitzler, Heinrich 9.\,8.\,1902 Hinterbrühl – 12.\,7.\,1982 Wien@\textsc{Schnitzler, Heinrich} (9.\,8.\,1902 Hinterbrühl – 12.\,7.\,1982 Wien), \emph{Regisseur, Schauspieler}|pwkv}
                  eben umgezogen. Am 13. 7. 1910 siedelten sie in ein eigenes Haus in 
                  der Sternwartestraße 71\oindex{Wien@\textbf{Wien}!XVIII., Währing@\textbf{XVIII., Währing}!Sternwartestraße 71@\textbf{Sternwartestraße 71}, \emph{Wohngebäude}|pwk}, um die Ecke der alten Wohnung.}}}\label{K_L03764-1} bin ich nicht dazu
               geko{\geminationm}en, Ihnen für die freundliche Überſendg Ihrer \label{K_L03764-2v}\edtext{\textsc{Verhaeren}\pwindex{Verhaeren, Émile 21.\,5.\,1855 Sint-Amands – 27.\,11.\,1916 Rouen@\textsc{Verhaeren, Émile} (21.\,5.\,1855 Sint-Amands – 27.\,11.\,1916 Rouen), \emph{Schriftsteller, Schriftsteller, Krimiautor}|pw}-Nachdichtungen\pwindex{Verhaeren, Émile 21.\,5.\,1855 Sint-Amands – 27.\,11.\,1916 Rouen@\textsc{Verhaeren, Émile} (21.\,5.\,1855 Sint-Amands – 27.\,11.\,1916 Rouen), \emph{Schriftsteller, Schriftsteller, Krimiautor}!Ausgewählte Gedichte. In Nachdichtung@\strich\emph{Ausgewählte Gedichte. In Nachdichtung}|pwv}}{\lemma{\textnormal{\emph{Verhaeren-Nachdichtungen}}}\Cendnote{\textnormal{Zweig\pwindex{Zweig, Stefan 28.\,11.\,1881 Wien – 23.\,2.\,1942 Petrópolis@\textsc{Zweig, Stefan} (28.\,11.\,1881 Wien – 23.\,2.\,1942 Petrópolis), \emph{Schriftsteller}|pwk} hatte 1904 seine Auswahl
                  von Gedichtübersetzungen von Émile
                     Verhaeren\pwindex{Verhaeren, Émile 21.\,5.\,1855 Sint-Amands – 27.\,11.\,1916 Rouen@\textsc{Verhaeren, Émile} (21.\,5.\,1855 Sint-Amands – 27.\,11.\,1916 Rouen), \emph{Schriftsteller, Schriftsteller, Krimiautor}|pwk} (»Ausgewählte Gedichte. In Nachdichtung von Stefan Zweig\pwindex{Zweig, Stefan 28.\,11.\,1881 Wien – 23.\,2.\,1942 Petrópolis@\textsc{Zweig, Stefan} (28.\,11.\,1881 Wien – 23.\,2.\,1942 Petrópolis), \emph{Schriftsteller}|pw}«) im Verlag \emph{Schuster {\kaufmannsund} und Loeffler}\orgindex{Schuster und Loeffler@Schuster {\kaufmannsund}  Loeffler|pwk} herausgebracht.
                     1910 erschien eine Neuausgabe, diesmal im \emph{Insel-Verlag}\orgindex{Insel Verlag@Insel Verlag|pwk}. Am 13. 6. 1910 erhielt der
                  Verleger Anton Kippenberg\pwindex{Kippenberg, Anton 22.\,5.\,1874 Bremen – 22.\,9.\,1950 Luzern@\textsc{Kippenberg, Anton} (22.\,5.\,1874 Bremen – 22.\,9.\,1950 Luzern), \emph{Kritiker, Übersetzer, Verlagsleiter}|pwk} von Zweig\pwindex{Zweig, Stefan 28.\,11.\,1881 Wien – 23.\,2.\,1942 Petrópolis@\textsc{Zweig, Stefan} (28.\,11.\,1881 Wien – 23.\,2.\,1942 Petrópolis), \emph{Schriftsteller}|pwk} die Nachricht, die Liste für die zu
                  verschickenden Bücher »nächster Tage« zu senden. (Anton Kippenberg\pwindex{Kippenberg, Anton 22.\,5.\,1874 Bremen – 22.\,9.\,1950 Luzern@\textsc{Kippenberg, Anton} (22.\,5.\,1874 Bremen – 22.\,9.\,1950 Luzern), \emph{Kritiker, Übersetzer, Verlagsleiter}|pwk}, Stefan Zweig\pwindex{Zweig, Stefan 28.\,11.\,1881 Wien – 23.\,2.\,1942 Petrópolis@\textsc{Zweig, Stefan} (28.\,11.\,1881 Wien – 23.\,2.\,1942 Petrópolis), \emph{Schriftsteller}|pwk}: \emph{Briefwechsel
                        1905–1937}. Ausgewählt von Oliver Matuschek und Klemens Renoldner.
                     Hg. und kommentiert von Oliver Matuschek unter Mitwirkung von Klemens
                     Renoldner. Berlin: \emph{Insel Verlag}{ }2022, S. 107.) Entsprechend dürfte Schnitzler das Buch\pwindex{Verhaeren, Émile 21.\,5.\,1855 Sint-Amands – 27.\,11.\,1916 Rouen@\textsc{Verhaeren, Émile} (21.\,5.\,1855 Sint-Amands – 27.\,11.\,1916 Rouen), \emph{Schriftsteller, Schriftsteller, Krimiautor}!Ausgewählte Gedichte. In Nachdichtung@\strich\emph{Ausgewählte Gedichte. In Nachdichtung}|pwkv} in der zweiten Hälfte des Juni erhalten haben.}}}\label{K_L03764-2} zu danken – ich thu es nun, mit
               verſpäteter Herzlichkeit, und freu mich{ }ſehr darauf das Buch\pwindex{Verhaeren, Émile 21.\,5.\,1855 Sint-Amands – 27.\,11.\,1916 Rouen@\textsc{Verhaeren, Émile} (21.\,5.\,1855 Sint-Amands – 27.\,11.\,1916 Rouen), \emph{Schriftsteller, Schriftsteller, Krimiautor}!Ausgewählte Gedichte. In Nachdichtung@\strich\emph{Ausgewählte Gedichte. In Nachdichtung}|pwv} (es{ }ſieht wunderſchön aus) in einigen
               der nächſten ruhigen Stunden, {\pb}wahrſcheinlich auf einer
               kleinen Reiſe, zu leſen.\pend
           
\pstart
           Ich hoffe, Sie haben einen{ }ſchönen So{\geminationm}er vor{ }ſich und arbeiten nicht nur zu andrer{ }ſondern
               auch zu eigenem Ruhm.\pend
           
\pstart
           Herzlichen Gruſs{\\[\baselineskip]}Ihr{\\[\baselineskip]}\spacefill\mbox{Arth Schnitzler}\pend
           \leftskip=0em{}\selectlanguage{ngerman}\endnumbering\briefempfaengerindex{Zweig, Stefan@\textsc{Zweig, Stefan}!zzzSchnitzler, Arthur@\emph{von Arthur Schnitzler}!1910-07-241@{24. 7. 1910}|)be}\mylabel{L03764h}  \newcommand{\dateiname}{L03764}\newcommand{\titel}{Arthur Schnitzler an Stefan Zweig, 24. 7. 1910}\newcommand{\editorInnen}{Selma Jahnke und Martin Anton Müller}%% latex-leseansicht-abspann.tex
%% Abspann für die Leseansicht.
%% Der Schalter \ifkorrekturansicht ist bereits durch den Vorspann gesetzt.

%% latex-abspann.tex
%% Gemeinsamer Abspann für Korrekturansicht und Leseansicht.
%% Setzt den Schalter \ifkorrekturansicht voraus (gesetzt in den
%% einbindenden Dateien latex-korrekturansicht-abspann.tex bzw.
%% latex-leseansicht-abspann.tex).
%% ---------------------------------------------------------------

\normalsize

% Das esempio-Environment wird nur in der Leseansicht benötigt
\ifkorrekturansicht\else
\newenvironment{esempio}[3]%
{
    \vspace{1.5ex}
    \rlap{\underline{#1}}
    \par
    \setlength{\parindent}{0cm}
    \nopagebreak
    \leftskip=#2cm
    \rightskip=#3cm
}
{
    \par
}
\fi

\doendnotes{C}
\bigskip
\vfill

\clearpage

\footnotesize

\ifkorrekturansicht
  \lohead{\textsc{register}}
\fi

% theindex-Environment neu definieren ohne reledmac
\makeatletter
\renewenvironment{theindex}{%
  \ifkorrekturansicht
    \section*{\indexname}%
  \else
    \subsubsection*{Index der erwähnten Entitäten}%
  \fi
  \setlength{\parindent}{0pt}%
  \setlength{\parskip}{0pt plus 0.3pt}%
  \let\item\@idxitem
}{%
  \ifkorrekturansicht\clearpage\fi
}
\makeatother

\IfFileExists{\jobname-pw.ind}{\input{\jobname-pw.ind}}{}

% Quellenangabe nur in der Leseansicht
\ifkorrekturansicht\else
% Fallback-Definitionen, falls die .tex-Datei \titel etc. nicht gesetzt hat
\providecommand{\titel}{}
\providecommand{\editorInnen}{}
\providecommand{\dateiname}{\jobname}

\vspace{3cm}

\vfill

\footnotesize
\textsc{Quelle}: \titel. Herausgegeben von {\editorInnen}. In: \emph{Arthur Schnitzler: Briefwechsel mit Autorinnen und Autoren}.
 Digitale Edition, https://schnitzler-briefe.acdh.oeaw.ac.at/{\dateiname}.html (Stand \today)
\fi

\end{document}


