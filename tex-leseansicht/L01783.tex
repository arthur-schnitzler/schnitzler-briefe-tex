%% latex-leseansicht-vorspann.tex
%% Vorspann für die Leseansicht.
%% Lädt die gemeinsame Datei latex-vorspann.tex mit nicht gesetztem Schalter.

\newif\ifkorrekturansicht
\korrekturansichtfalse

\input{../tex-inputs/latex-vorspann}


               \section[Max Burckhard an Arthur Schnitzler, 14. 7. 1908]{ Max Burckhard an Arthur Schnitzler, 14. 7. 1908}\nopagebreak\mylabel{v}\rehead{ }\begin{ledgroupsized}[t]{13cm}\normalsize\beginnumbering\briefempfaengerindex{Schnitzler, Arthur@\textsc{Schnitzler, Arthur}!zzzBurckhard, Max Eugen@\emph{von Max Eugen Burckhard}!1908-07-141@{14. 7. 1908}|(be} \toendnotes[C]{\smallbreak\pagebreak[2]} \Standort{CUL, Schnitzler, B 20.}
\physDesc{Brief, 1 Blatt, 3 Seiten
\newline{}Handschrift: schwarze Tinte, deutsche Kurrent\newline{}Ordnung: mit Bleistift von unbekannter Hand nummeriert: »23« }\toendnotes[C]{\smallbreak}\pstart
           \noindent{}{\pb}\textcolor{gray}{\textbf{D\textsuperscript{r.} Max Burckhard}}\hfill \textcolor{gray}{\textbf{\strikeout{Wien, IX. Porzellangasse 48\oindex{Porzellangasse@\textbf{Porzellangasse}|pw}}{ }..........}}\pend
           \pstart
           \raggedleft{}\textcolor{gray}{\textbf{St. Gilgen\oindex{St. Gilgen@\textbf{St. Gilgen}|pw}}}{ }14. 7. 08\pend
           \pstart{}Sehr verehrter lieber Herr Doctor!\pend\pstart
           Ich beglückwünſche Sie ſehr \strikeout{für} zu Ihrem
                    Aufenthalt, den mir Ihre liebe Karte meldet. Ich war einmal wenige Tage auf der
                        Seiſeralm\oindex{Seiser Alm@\textbf{Seiser Alm}|pw} – allerdings zur Schnittzeit. Es
                    war dort nicht nur wunderſchön, ſondern auch anſonſt außerordentlich erheiternd;
                    es war damals das einzigemal, daſs ich Gelegenheit hatte, das ſüdtiroliſche\oindex{Suedtirol@\textbf{Südtirol}|pw} Volksleben (von ſeiner angenehmſten Seite)
                    kennen zu lernen. Freilich hatte ich mich mit großen Vorräthen an feſtem und
                    flüßigem Proviant eingeführt und hatte ſchon vorher die Bekanntſchaft einiger
                    Theilnehmerinnen auf demSchlern gemacht.\pend
           \pstart
           {\pb}Nun, und ſind Sie uns St. Gilgnern\oindex{St. Gilgen@\textbf{St. Gilgen}|pw} ganz untreu geworden? Da es anfängt, Momente
                    zu geben, in denen ich mir einbilden kann, daſs ich mich noch einmal
                    zuſammenklaube, bilde ich mir ein, daſs ich davon etwas davon haben würde, wenn
                    Sie mit Ihrer verehrten Frau Gemahlin\pwindex{Schnitzler, Olga 17.01.1882 – 13.01.1970@\textsc{Schnitzler, Olga} (17.01.1882 – 13.01.1970), \emph{Schauspielerin, Sängerin}|pwv} hier wieder einmal in die heimiſchen Berge zukehren. Wie
                    herrliche Spaziergänge es hier gibt, das habe ich eigentlich erſt entdeckt, ſeit
                    die Facultät sich ablehnend gegen größere Spaziergänge ausgeſprochen hat.\pend
           \pstart
           {\pb}In herzlicher Verehrung mit Handkuſs
                    an Ihre liebe Frau\pwindex{Schnitzler, Olga 17.01.1882 – 13.01.1970@\textsc{Schnitzler, Olga} (17.01.1882 – 13.01.1970), \emph{Schauspielerin, Sängerin}|pwv} und
                    herzlichſtem Gruß\pend
           \pstart
           Ihr getreu ergebener{\\[\baselineskip]}\spacefill\mbox{D\textsuperscript{r}Burckhard}\pend
           \leftskip=0em{}          \endnumbering\briefempfaengerindex{Schnitzler, Arthur@\textsc{Schnitzler, Arthur}!zzzBurckhard, Max Eugen@\emph{von Max Eugen Burckhard}!1908-07-141@{14. 7. 1908}|)be}\mylabel{h}\end{ledgroupsized}  \newcommand{\dateiname}{L01783}\newcommand{\titel}{Max Burckhard an Arthur Schnitzler, 14. 7. 1908}\newcommand{\editorInnen}{Martin Anton Müller und Gerd-Hermann Susen}%% latex-leseansicht-abspann.tex
%% Abspann für die Leseansicht.
%% Der Schalter \ifkorrekturansicht ist bereits durch den Vorspann gesetzt.

%% latex-abspann.tex
%% Gemeinsamer Abspann für Korrekturansicht und Leseansicht.
%% Setzt den Schalter \ifkorrekturansicht voraus (gesetzt in den
%% einbindenden Dateien latex-korrekturansicht-abspann.tex bzw.
%% latex-leseansicht-abspann.tex).
%% ---------------------------------------------------------------

\normalsize

% Das esempio-Environment wird nur in der Leseansicht benötigt
\ifkorrekturansicht\else
\newenvironment{esempio}[3]%
{
    \vspace{1.5ex}
    \rlap{\underline{#1}}
    \par
    \setlength{\parindent}{0cm}
    \nopagebreak
    \leftskip=#2cm
    \rightskip=#3cm
}
{
    \par
}
\fi

\doendnotes{C}
\bigskip
\vfill

\clearpage

\footnotesize

\ifkorrekturansicht
  \lohead{\textsc{register}}
\fi

% theindex-Environment neu definieren ohne reledmac
\makeatletter
\renewenvironment{theindex}{%
  \ifkorrekturansicht
    \section*{\indexname}%
  \else
    \subsubsection*{Index der erwähnten Entitäten}%
  \fi
  \setlength{\parindent}{0pt}%
  \setlength{\parskip}{0pt plus 0.3pt}%
  \let\item\@idxitem
}{%
  \ifkorrekturansicht\clearpage\fi
}
\makeatother

\IfFileExists{\jobname-pw.ind}{\input{\jobname-pw.ind}}{}

% Quellenangabe nur in der Leseansicht
\ifkorrekturansicht\else
% Fallback-Definitionen, falls die .tex-Datei \titel etc. nicht gesetzt hat
\providecommand{\titel}{}
\providecommand{\editorInnen}{}
\providecommand{\dateiname}{\jobname}

\vspace{3cm}

\vfill

\footnotesize
\textsc{Quelle}: \titel. Herausgegeben von {\editorInnen}. In: \emph{Arthur Schnitzler: Briefwechsel mit Autorinnen und Autoren}.
 Digitale Edition, https://schnitzler-briefe.acdh.oeaw.ac.at/{\dateiname}.html (Stand \today)
\fi

\end{document}


      