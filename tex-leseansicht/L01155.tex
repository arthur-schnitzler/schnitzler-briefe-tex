%% latex-leseansicht-vorspann.tex
%% Vorspann für die Leseansicht.
%% Lädt die gemeinsame Datei latex-vorspann.tex mit nicht gesetztem Schalter.

\newif\ifkorrekturansicht
\korrekturansichtfalse

\input{../tex-inputs/latex-vorspann}


\section[Edith Brandes an Arthur Schnitzler, 31. 7. 1901]{L01155 Edith Brandes an Arthur Schnitzler, 31. 7. 1901}
\nopagebreak\mylabel{L01155v}
\rehead{ }\normalsize\beginnumbering\briefempfaengerindex{Schnitzler, Arthur@\textsc{Schnitzler, Arthur}!zzzPhilipp, Edith@\emph{von Edith Philipp}!1901-07-311@{31. 7. 1901}|(be}
\toendnotes[C]{\smallbreak\pagebreak[2]}
\correspDesc{Versand  durch Edith Brandes am 31. 7. 1901 \textbf{Ort fehlend} 
\newline{}Weiterleitung  in Wien
\newline{}Erhalt  durch Arthur Schnitzler im Zeitraum [31. 7. 1901
                  – 4. 8. 1901?] in Vahrn}\toendnotes[C]{\smallbreak}
\Standort{CUL, Schnitzler, B 17.}
\physDesc{Brief, 1 Blatt, 2 Seiten, 444 Zeichen
\newline{}Handschrift: schwarze Tinte, lateinische Kurrent
\newline{}Ordnung: mit Bleistift von unbekannter Hand nummeriert:
                                    »28« 
\newline{}Zusatz: florales Jugendstil-Briefpapier mit aufgedruckten Tauben }\Standort{DLA, A:Schnitzler, HS.NZ85.1.2595.}
\physDesc{maschinenschriftliche Abschrift, 1 Blatt, 1 Seite, 444 Zeichen
\newline{}Schreibmaschine}
\buchAbdrucke{\weitereDrucke{Georg Brandes, Arthur Schnitzler: \emph{Ein Briefwechsel}. Herausgegeben von Kurt Bergel. Bern: \emph{Francke} 1956, S. 91.} }\toendnotes[C]{\smallbreak}
\pstart
           \raggedleft{}{\pb}\uline{Mittwoch 31-7-1901}\pend
           
\pstart{}Verehrter Herr Schnitzler!\pend\vspace{0.5em}
\pstart
           Seien Sie aufs herzlichste bedankt für das hübsche Gedicht, worüber ich mich sehr
               gefreut habe. Es gehört in Zukunft zu den Zierden meines Albums. An Papa\pwindex{Brandes, Georg 4.\,2.\,1842 Kopenhagen – 19.\,2.\,1927 ebd.@\textsc{Brandes, Georg} (4.\,2.\,1842 Kopenhagen – 19.\,2.\,1927 ebd.)|pwv} habe ich Ihre Grüsse schriftlich
               bestellt, da er sich augenblicklich in Karlsbad\oindex{Karlsbad@\textbf{Karlsbad}|pw}
               befindet. –\pend
           
\pstart
           {\pb}Ich hoffe sehr Sie einmal
               persönlich kennen zu lernen, wird Ihr Weg Sie nicht mal wieder hierher führen?\pend
           
\pstart
           Mit besten Grüssen und nochmals dankend{\\[\baselineskip]}\spacefill\mbox{Edith Brandes.}\pend
           \leftskip=0em{}\selectlanguage{ngerman}\endnumbering\briefempfaengerindex{Schnitzler, Arthur@\textsc{Schnitzler, Arthur}!zzzPhilipp, Edith@\emph{von Edith Philipp}!1901-07-311@{31. 7. 1901}|)be}\mylabel{L01155h}  \newcommand{\dateiname}{L01155}\newcommand{\titel}{Edith Brandes an Arthur Schnitzler, 31. 7. 1901}\newcommand{\editorInnen}{Martin Anton Müller und Gerd-Hermann Susen}%% latex-leseansicht-abspann.tex
%% Abspann für die Leseansicht.
%% Der Schalter \ifkorrekturansicht ist bereits durch den Vorspann gesetzt.

%% latex-abspann.tex
%% Gemeinsamer Abspann für Korrekturansicht und Leseansicht.
%% Setzt den Schalter \ifkorrekturansicht voraus (gesetzt in den
%% einbindenden Dateien latex-korrekturansicht-abspann.tex bzw.
%% latex-leseansicht-abspann.tex).
%% ---------------------------------------------------------------

\normalsize

% Das esempio-Environment wird nur in der Leseansicht benötigt
\ifkorrekturansicht\else
\newenvironment{esempio}[3]%
{
    \vspace{1.5ex}
    \rlap{\underline{#1}}
    \par
    \setlength{\parindent}{0cm}
    \nopagebreak
    \leftskip=#2cm
    \rightskip=#3cm
}
{
    \par
}
\fi

\doendnotes{C}
\bigskip
\vfill

\clearpage

\footnotesize

\ifkorrekturansicht
  \lohead{\textsc{register}}
\fi

% theindex-Environment neu definieren ohne reledmac
\makeatletter
\renewenvironment{theindex}{%
  \ifkorrekturansicht
    \section*{\indexname}%
  \else
    \subsubsection*{Index der erwähnten Entitäten}%
  \fi
  \setlength{\parindent}{0pt}%
  \setlength{\parskip}{0pt plus 0.3pt}%
  \let\item\@idxitem
}{%
  \ifkorrekturansicht\clearpage\fi
}
\makeatother

\IfFileExists{\jobname-pw.ind}{\input{\jobname-pw.ind}}{}

% Quellenangabe nur in der Leseansicht
\ifkorrekturansicht\else
% Fallback-Definitionen, falls die .tex-Datei \titel etc. nicht gesetzt hat
\providecommand{\titel}{}
\providecommand{\editorInnen}{}
\providecommand{\dateiname}{\jobname}

\vspace{3cm}

\vfill

\footnotesize
\textsc{Quelle}: \titel. Herausgegeben von {\editorInnen}. In: \emph{Arthur Schnitzler: Briefwechsel mit Autorinnen und Autoren}.
 Digitale Edition, https://schnitzler-briefe.acdh.oeaw.ac.at/{\dateiname}.html (Stand \today)
\fi

\end{document}


