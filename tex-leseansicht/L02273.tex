%% latex-korrekturansicht-vorspann.tex
%% Vorspann für die Korrekturansicht.
%% Lädt die gemeinsame Datei latex-vorspann.tex mit gesetztem Schalter.

\newif\ifkorrekturansicht
\korrekturansichttrue

\input{../tex-inputs/latex-vorspann}


\section[Richard Beer-Hofmann an Arthur Schnitzler, 5. 9. 1917]{L02273 Richard Beer-Hofmann an Arthur Schnitzler, 5. 9. 1917}
\nopagebreak\mylabel{L02273v}
\rehead{ }\normalsize\beginnumbering\briefempfaengerindex{Schnitzler, Arthur@\textsc{Schnitzler, Arthur}!zzzBeer-Hofmann, Richard@\emph{von Richard Beer-Hofmann}!1917-09-051@{5. 9. 1917}|(be}
\toendnotes[C]{\smallbreak\pagebreak[2]}\Standort{CUL, Schnitzler, B 8.}
\physDesc{Postkarte, 521 Zeichen
\newline{}Handschrift: Bleistift, lateinische Kurrent
\newline{}Versand: Stempel: »\nobreak{}\oindex{Bad Ischl@\textbf{Bad Ischl}, \emph{P.PPL}|pwk}Bad Is{[}chl{]}, 5. IX. 17, VII\nobreak{}«.  
\newline{}Ordnung: mit Bleistift von unbekannter Hand nummeriert:
                                    »265« }\toendnotes[C]{\smallbreak}\pstart{}{\pb}Beer-Hofmann, Bad Ischl\oindex{Bad Ischl@\textbf{Bad Ischl}, \emph{P.PPL}|pw}\pend{}{\bigskip}\pstart{}Herrn\pend{}\pstart{}D\textsuperscript{r} Arthur Schnitzler\pend{}\pstart{}Partenkirchen\oindex{Partenkirchen@\textbf{Partenkirchen}, \emph{Teil eines besiedelten Ortes (A.BSOX)}|pw}\pend{}\pstart{}Haus Tannenberg\oindex{Haus Tannenberg@\textbf{Haus Tannenberg}, \emph{Beherbergungsgebäude (K.BHB)}|pw}\pend{}{\bigskip}\vspace{1em}
\pstart
           \noindent{}{\pb}Lieber Artur! Von Salzburg\oindex{Salzburg@\textbf{Salzburg}, \emph{A.ADM2}|pw}
               zurück, finde ich Ihre Karte vor. Ich glaube dass ich kaum vor
                  28–30ten Sept in Berlin\oindex{Berlin@\textbf{Berlin}, \emph{P.PPLC}|pw} sein dürfte. Jedenfalls verständig ich Sie (– falls Sie nicht in Wien\oindex{Wien@\textbf{Wien}, \emph{A.ADM2}|pw} sein sollten Ihre jeweilige Adresse!) wann ich
               reise.\pend
           
\pstart
           Der \label{K_L02273-1v}\edtext{zweite Brief}{\lemma{\textnormal{\emph{zweite Brief}}}\Cendnote{\textnormal{Siehe Arthur Schnitzler an Richard Beer-Hofmann, 23. 7. 1917.
               }}}\label{K_L02273-1} ist angelangt, – ich habe ein böses Gewissen so säumig {\pb}gewesen zu sein – aber Versti{\geminationm}tsein behält man für sich – oder nicht »man« – aber
               ich.\pend
           
\pstart
           Herzliche Grüsse Ihnen, Ihrer Frau\pwindex{Schnitzler, Olga 17.01.1882 – 13.01.1970@\textsc{Schnitzler, Olga} (17.01.1882 – 13.01.1970), \emph{Schauspieler/Schauspielerin, Sänger/Sängerin}|pwv} und Ihrer Schwägerin\pwindex{Steinrueck, Elisabeth 19.11.1885 – 07.04.1920@\textsc{Steinrück, Elisabeth} (19.11.1885 – 07.04.1920)|pwv}! Ihr{\\[\baselineskip]}\spacefill\mbox{Richard}\pend
           \leftskip=0em{}\selectlanguage{ngerman}\endnumbering\briefempfaengerindex{Schnitzler, Arthur@\textsc{Schnitzler, Arthur}!zzzBeer-Hofmann, Richard@\emph{von Richard Beer-Hofmann}!1917-09-051@{5. 9. 1917}|)be}\mylabel{L02273h}  \normalsize

\doendnotes{C}
\bigskip
\vfill

\clearpage

\footnotesize

\lohead{\textsc{register}}

% Definiere theindex-Environment komplett neu ohne reledmac
\makeatletter
\renewenvironment{theindex}{%
  \section*{\indexname}%
  \setlength{\parindent}{0pt}%
  \setlength{\parskip}{0pt plus 0.3pt}%
  \let\item\@idxitem
}{%
  \clearpage
}
\makeatother

\IfFileExists{\jobname-pw.ind}{\input{\jobname-pw.ind}}{}

\end{document}

      