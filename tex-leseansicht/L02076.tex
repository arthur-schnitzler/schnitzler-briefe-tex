%% latex-leseansicht-vorspann.tex
%% Vorspann für die Leseansicht.
%% Lädt die gemeinsame Datei latex-vorspann.tex mit nicht gesetztem Schalter.

\newif\ifkorrekturansicht
\korrekturansichtfalse

\input{../tex-inputs/latex-vorspann}


\section[Hugo von Hofmannsthal an Olga Schnitzler, 5. 7. {[}1912{]}]{L02076 Hugo von Hofmannsthal an Olga Schnitzler, 5. 7. [1912]}
\nopagebreak\mylabel{L02076v}
\rehead{ }\normalsize\beginnumbering\briefempfaengerindex{Schnitzler, Olga@\textsc{Schnitzler, Olga}!zzzHofmannsthal, Hugo von@\emph{von Hugo von Hofmannsthal}!1912-07-051@{5. 7. [1912]}|(be}
\toendnotes[C]{\smallbreak\pagebreak[2]}
\correspDesc{Versand  durch Hugo von Hofmannsthal am 5. 7. [1912] in Rodaun
\newline{}Erhalt  durch Olga Schnitzler im Zeitraum [6. 7. 1912
                  – 10. 7. 1912?] in Wien}\toendnotes[C]{\smallbreak}
\Standort{CUL, Schnitzler, B 43.}
\physDesc{Briefkarte, 640 Zeichen
\newline{}Handschrift: schwarze Tinte, deutsche Kurrent
\newline{}Ordnung: 1) von Schnitzler mit Bleistift die Jahreszahl ergänzt: »912« und beschriftet: »\textsc{Hugo}«  2) mit Bleistift von unbekannter Hand nummeriert: »\strikeout{328}« 3) mit Bleistift von unbekannter Hand nummeriert:
                                    »338«}
\buchAbdrucke{\weitereDrucke{Hugo von Hofmannsthal, Arthur Schnitzler: \emph{Briefwechsel}. Herausgegeben von Therese Nickl und Heinrich Schnitzler. Frankfurt am Main: \emph{S. Fischer} 1964, S. 385.} }\toendnotes[C]{\smallbreak}
\pstart
           \raggedleft{}{\pb}Rodaun\oindex{Wien@\textbf{Wien}!XXIII., Liesing@\textbf{XXIII., Liesing}!Rodaun@\textbf{Rodaun}, \emph{Region}|pw}{ }\substVorne{}\textsuperscript{6}\substDazwischen{}5\substHinten{}. VII.\pend
           
\pstart{}liebe Olga,\pend\vspace{0.5em}
\pstart
           gerade geſtern Abend fand ich einen{ }ſehr netten Brief von Steinrück\pwindex{Steinrück, Albert 20.\,5.\,1872 Wetterburg – 11.\,2.\,1929 Berlin@\textsc{Steinrück, Albert} (20.\,5.\,1872 Wetterburg – 11.\,2.\,1929 Berlin), \emph{Schauspieler}|pw} aus Tutzing\oindex{Tutzing@\textbf{Tutzing}, \emph{Hauptstadt}|pw},
               alſo liegt kein Grund vor, ihn zu erziehen.\hspace*{1.5em}Ich{ }ſchicke Ihnen demnächst Ariadne\pwindex{Hofmannsthal, Hugo von 1.\,2.\,1874 Wien – 15.\,7.\,1929 Rodaun@\textsc{Hofmannsthal, Hugo von} (1.\,2.\,1874 Wien – 15.\,7.\,1929 Rodaun), \emph{Schriftsteller}!Ariadne auf Naxos. Oper in einem Aufzug@\strich\emph{Ariadne auf Naxos. Oper in einem Aufzug}|pw} und den Sa{\geminationm}elband meiner
                  jugendlichen Arbeiten\pwindex{Hofmannsthal, Hugo von 1.\,2.\,1874 Wien – 15.\,7.\,1929 Rodaun@\textsc{Hofmannsthal, Hugo von} (1.\,2.\,1874 Wien – 15.\,7.\,1929 Rodaun), \emph{Schriftsteller}!Gedichte und kleinen Dramen@\strich\emph{Die Gedichte und kleinen Dramen}|pw} und würde mich{ }ſehr freuen wenn Sie beides in den So{\geminationm}er mitnähmen.\pend
           
\pstart
           {\pb}Man{ }ſieht{ }ſich gar{ }ſo{ }ſelten! Das
               Leben iſt{ }ſo kurz, auf einmal wird man todt{ }ſein und es dann{ }ſehr bedauern. Ko{\geminationm}t Ihr beide oder ko{\geminationm}t Arthur doch noch nächſte Woche für 1–1½ Tage
               nach Vöslau\oindex{Bad Vöslau@\textbf{Bad Vöslau}, \emph{Hauptstadt}|pw}{ }ſo würde ich{ }ſehr gern von der Hinterbrühl\oindex{Hinterbrühl@\textbf{Hinterbrühl}, \emph{Hauptstadt}|pw} hinüberfahren für eine Stunde \label{K_L02076-1v}\edtext{Zuſa{\geminationm}enſein}{\lemma{\textnormal{\emph{Zusammensein}}}\Cendnote{\textnormal{Siehe A. S.: \emph{Tagebuch}, 10. 7. 1912.
               }}}\label{K_L02076-1}.\pend
           
\pstart
           Erbitte alſo eventuell Depeſche \textsc{Villa Louis
                        Friedmann\pwindex{Friedmann, Louis Philipp 29.\,6.\,1861 Paris – 1.\,4.\,1939 Wien@\textsc{Friedmann, Louis Philipp} (29.\,6.\,1861 Paris – 1.\,4.\,1939 Wien), \emph{Industrieller, Bergsteiger}|pw}\oindex{Villa Friedmann@\textbf{Villa Friedmann}, \emph{Gebäude}|pw}}.\pend
           \pstart \label{T_L02076-1v}\edtext{Freundschaftlich Ihr}{\lemma{\textnormal{\emph{Freundschaftlich Ihr}}}\Cendnote{\textnormal{quer am linken Rand}}}\label{T_L02076-1}\spacefill\mbox{Hugo}\pend{}\selectlanguage{ngerman}\endnumbering\briefempfaengerindex{Schnitzler, Olga@\textsc{Schnitzler, Olga}!zzzHofmannsthal, Hugo von@\emph{von Hugo von Hofmannsthal}!1912-07-051@{5. 7. [1912]}|)be}\mylabel{L02076h}  \newcommand{\dateiname}{L02076}\newcommand{\titel}{Hugo von Hofmannsthal an Olga Schnitzler, 5. 7. [1912]}\newcommand{\editorInnen}{Martin Anton Müller und Gerd-Hermann Susen}%% latex-leseansicht-abspann.tex
%% Abspann für die Leseansicht.
%% Der Schalter \ifkorrekturansicht ist bereits durch den Vorspann gesetzt.

%% latex-abspann.tex
%% Gemeinsamer Abspann für Korrekturansicht und Leseansicht.
%% Setzt den Schalter \ifkorrekturansicht voraus (gesetzt in den
%% einbindenden Dateien latex-korrekturansicht-abspann.tex bzw.
%% latex-leseansicht-abspann.tex).
%% ---------------------------------------------------------------

\normalsize

% Das esempio-Environment wird nur in der Leseansicht benötigt
\ifkorrekturansicht\else
\newenvironment{esempio}[3]%
{
    \vspace{1.5ex}
    \rlap{\underline{#1}}
    \par
    \setlength{\parindent}{0cm}
    \nopagebreak
    \leftskip=#2cm
    \rightskip=#3cm
}
{
    \par
}
\fi

\doendnotes{C}
\bigskip
\vfill

\clearpage

\footnotesize

\ifkorrekturansicht
  \lohead{\textsc{register}}
\fi

% theindex-Environment neu definieren ohne reledmac
\makeatletter
\renewenvironment{theindex}{%
  \ifkorrekturansicht
    \section*{\indexname}%
  \else
    \subsubsection*{Index der erwähnten Entitäten}%
  \fi
  \setlength{\parindent}{0pt}%
  \setlength{\parskip}{0pt plus 0.3pt}%
  \let\item\@idxitem
}{%
  \ifkorrekturansicht\clearpage\fi
}
\makeatother

\IfFileExists{\jobname-pw.ind}{\input{\jobname-pw.ind}}{}

% Quellenangabe nur in der Leseansicht
\ifkorrekturansicht\else
% Fallback-Definitionen, falls die .tex-Datei \titel etc. nicht gesetzt hat
\providecommand{\titel}{}
\providecommand{\editorInnen}{}
\providecommand{\dateiname}{\jobname}

\vspace{3cm}

\vfill

\footnotesize
\textsc{Quelle}: \titel. Herausgegeben von {\editorInnen}. In: \emph{Arthur Schnitzler: Briefwechsel mit Autorinnen und Autoren}.
 Digitale Edition, https://schnitzler-briefe.acdh.oeaw.ac.at/{\dateiname}.html (Stand \today)
\fi

\end{document}


