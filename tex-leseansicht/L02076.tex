%% latex-korrekturansicht-vorspann.tex
%% Vorspann für die Korrekturansicht.
%% Lädt die gemeinsame Datei latex-vorspann.tex mit gesetztem Schalter.

\newif\ifkorrekturansicht
\korrekturansichttrue

\input{../tex-inputs/latex-vorspann}


\section[Hugo von Hofmannsthal an Olga Schnitzler, 5. 7. {[}1912{]}]{L02076 Hugo von Hofmannsthal an Olga Schnitzler, 5. 7. {[}1912{]}}
\nopagebreak\mylabel{L02076v}
\rehead{ }\normalsize\beginnumbering\briefempfaengerindex{Schnitzler, Olga@\textsc{Schnitzler, Olga}!zzzHofmannsthal, Hugo von@\emph{von Hugo von Hofmannsthal}!1912-07-051@{5. 7. {[}1912{]}}|(be}
\toendnotes[C]{\smallbreak\pagebreak[2]}\Standort{CUL, Schnitzler, B 43.}
\physDesc{Briefkarte, 640 Zeichen
\newline{}Handschrift: schwarze Tinte, deutsche Kurrent
\newline{}Ordnung: 1) von Schnitzler mit Bleistift die Jahreszahl ergänzt: »912« und beschriftet: »\textsc{Hugo}«  2) mit Bleistift von unbekannter Hand nummeriert: »\strikeout{328}« 3) mit Bleistift von unbekannter Hand nummeriert:
                                    »338«}
\buchAbdrucke{\weitereDrucke{Hugo von Hofmannsthal, Arthur Schnitzler: \emph{Briefwechsel}. Frankfurt am Main: \emph{S. Fischer} 1964, S. 385.} }\toendnotes[C]{\smallbreak}
\pstart
           \raggedleft{}{\pb}Rodaun\oindex{Rodaun@\textbf{Rodaun}, \emph{A.ADM4}|pw}{ }\substVorne{}\textsuperscript{6}\substDazwischen{}5\substHinten{}. VII.\pend
           
\pstart{}liebe Olga,\pend\vspace{0.5em}
\pstart
           gerade geſtern Abend fand ich einen ſehr netten Brief von Steinrück\pwindex{Steinrueck, Albert 20.05.1872 – 11.02.1929@\textsc{Steinrück, Albert} (20.05.1872 – 11.02.1929), \emph{Schauspieler/Schauspielerin}|pw} aus Tutzing\oindex{Tutzing@\textbf{Tutzing}, \emph{P.PPLA4}|pw},
               alſo liegt kein Grund vor, ihn zu erziehen.\hspace*{1.5em}Ich
               ſchicke Ihnen demnächst Ariadne\pwindex{Ariadne auf Naxos. Oper in einem Aufzug@\emph{Ariadne auf Naxos. Oper in einem Aufzug}|pw} und den Sa{\geminationm}elband meiner
                  jugendlichen Arbeiten\pwindex{Gedichte und kleinen Dramen@\emph{Die Gedichte und kleinen Dramen}|pw} und würde mich ſehr freuen wenn Sie beides in den So{\geminationm}er mitnähmen.\pend
           
\pstart
           {\pb}Man ſieht ſich gar ſo ſelten! Das
               Leben iſt ſo kurz, auf einmal wird man todt ſein und es dann ſehr bedauern. Ko{\geminationm}t Ihr beide oder ko{\geminationm}t Arthur doch noch nächſte Woche für 1–1½ Tage
               nach Vöslau\oindex{Bad Voeslau@\textbf{Bad Vöslau}, \emph{P.PPLA3}|pw}{ }ſo würde ich ſehr gern von der Hinterbrühl\oindex{Hinterbruehl@\textbf{Hinterbrühl}, \emph{P.PPLA3}|pw} hinüberfahren für eine Stunde \label{K_L02076-1v}\edtext{Zuſa{\geminationm}enſein}{\lemma{\textnormal{\emph{Zuſammenſein}}}\Cendnote{\textnormal{Siehe A. S.: \emph{Tagebuch}, 10. 7. 1912.
               }}}\label{K_L02076-1}.\pend
           
\pstart
           Erbitte alſo eventuell Depeſche \textsc{Villa Louis
                        Friedmann\pwindex{Friedmann, Louis Philipp 29.06.1861 – 01.04.1939@\textsc{Friedmann, Louis Philipp} (29.06.1861 – 01.04.1939), \emph{Industrieller/Industrielle, Bergsteiger/Bergsteigerin}|pw}\oindex{Villa Friedmann@\textbf{Villa Friedmann}, \emph{Gebäude (K.GBD)}|pw}}.\pend
           \pstart \label{T_L02076-1v}\edtext{Freundschaftlich Ihr}{\lemma{\textnormal{\emph{Freundschaftlich Ihr}}}\Cendnote{\textnormal{quer am linken Rand}}}\label{T_L02076-1}\spacefill\mbox{Hugo}\pend{}\selectlanguage{ngerman}\endnumbering\briefempfaengerindex{Schnitzler, Olga@\textsc{Schnitzler, Olga}!zzzHofmannsthal, Hugo von@\emph{von Hugo von Hofmannsthal}!1912-07-051@{5. 7. {[}1912{]}}|)be}\mylabel{L02076h}  \normalsize

\doendnotes{C}
\bigskip
\vfill

\clearpage

\footnotesize

\lohead{\textsc{register}}

% Definiere theindex-Environment komplett neu ohne reledmac
\makeatletter
\renewenvironment{theindex}{%
  \section*{\indexname}%
  \setlength{\parindent}{0pt}%
  \setlength{\parskip}{0pt plus 0.3pt}%
  \let\item\@idxitem
}{%
  \clearpage
}
\makeatother

\IfFileExists{\jobname-pw.ind}{\input{\jobname-pw.ind}}{}

\end{document}

      