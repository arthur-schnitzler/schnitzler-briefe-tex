%% latex-leseansicht-vorspann.tex
%% Vorspann für die Leseansicht.
%% Lädt die gemeinsame Datei latex-vorspann.tex mit nicht gesetztem Schalter.

\newif\ifkorrekturansicht
\korrekturansichtfalse

\input{../tex-inputs/latex-vorspann}


         
         \renewcommand{\erwaehntePersonen}{Personen: Louis Philipp Friedmann, Hugo von Hofmannsthal, Olga Schnitzler, Albert Steinrück}
         \renewcommand{\erwaehnteOrte}{Orte: Bad Vöslau, Hinterbrühl, Rodaun, Tutzing, Villa Friedmann, Wien}
         \renewcommand{\erwaehnteWerke}{Werke: Ariadne auf Naxos. Oper in einem Aufzug, Die Gedichte und kleinen Dramen}
               \section[Hugo von Hofmannsthal an Olga Schnitzler, 5. 7. {[}1912{]}]{ Hugo von Hofmannsthal an Olga Schnitzler, 5. 7. {[}1912{]}}\nopagebreak\mylabel{v}\rehead{ }\begin{ledgroupsized}[t]{13cm}\normalsize\beginnumbering\briefempfaengerindex{Schnitzler, Olga@\textsc{Schnitzler, Olga}!zzzHofmannsthal, Hugo von@\emph{von Hugo von Hofmannsthal}!1912-07-051@{5. 7. {[}1912{]}}|(be} \toendnotes[C]{\smallbreak\pagebreak[2]} \Standort{CUL, Schnitzler, B 43.}
\physDesc{Briefkarte, 640 Zeichen
\newline{}Handschrift: schwarze Tinte, deutsche Kurrent
\newline{}Ordnung: 1) von Schnitzler mit Bleistift die Jahreszahl ergänzt: »912« und beschriftet: »\textsc{Hugo}«  2) mit Bleistift von unbekannter Hand nummeriert: »\strikeout{328}« 3) mit Bleistift von unbekannter Hand nummeriert:
                                    »338«}\buchAbdrucke{\weitereDrucke{Hugo von Hofmannsthal, Arthur Schnitzler: \emph{Briefwechsel}. Hg. Therese Nickl und Heinrich Schnitzler. Frankfurt am Main: \emph{S. Fischer} 1964, S. 385.} }\toendnotes[C]{\smallbreak}\pstart
           \raggedleft{}{\pb}Rodaun\oindex{Rodaun@\textbf{Rodaun}|pw}{ }\substVorne{}\textsuperscript{6}\substDazwischen{}5\substHinten{}. VII.\pend
           \pstart{}liebe Olga,\pend\pstart
           gerade geſtern Abend fand ich einen ſehr netten Brief von Steinrück\pwindex{Steinrueck, Albert 20.05.1872 – 11.02.1929@\textsc{Steinrück, Albert} (20.05.1872 – 11.02.1929), \emph{Schauspieler}|pw} aus Tutzing\oindex{Tutzing@\textbf{Tutzing}|pw},
               alſo liegt kein Grund vor, ihn zu erziehen.\hspace*{1.5em}Ich
               ſchicke Ihnen demnächst Ariadne\pwindex{Hofmannsthal, Hugo von 1874-02-01 – 1929-07-15@\textsc{Hofmannsthal, Hugo von} (1874-02-01 – 1929-07-15), \emph{Schriftsteller}!Ariadne auf Naxos. Oper in einem Aufzug1912@\strich\emph{Ariadne auf Naxos. Oper in einem Aufzug} {[}1912{]}|pw} und den Sa{\geminationm}elband meiner
                  jugendlichen Arbeiten\pwindex{Hofmannsthal, Hugo von 1874-02-01 – 1929-07-15@\textsc{Hofmannsthal, Hugo von} (1874-02-01 – 1929-07-15), \emph{Schriftsteller}!Gedichte und kleinen Dramen1912@\strich\emph{Die Gedichte und kleinen Dramen} {[}1912{]}|pw} und würde mich ſehr freuen wenn Sie beides in den So{\geminationm}er mitnähmen.\pend
           \pstart
           {\pb}Man ſieht ſich gar ſo ſelten! Das
               Leben iſt ſo kurz, auf einmal wird man todt ſein und es dann ſehr bedauern. Ko{\geminationm}t Ihr beide oder ko{\geminationm}t Arthur\pwindex{Schnitzler, Arthur 15.05.1862 – 21.10.1931@\textsc{Schnitzler, Arthur} (15.05.1862 – 21.10.1931), \emph{Schriftsteller, Mediziner}|pw} doch noch nächſte Woche für 1–1½ Tage
               nach Vöslau\oindex{Bad Voeslau@\textbf{Bad Vöslau}|pw}{ }ſo würde ich ſehr gern von der Hinterbrühl\oindex{Hinterbruehl@\textbf{Hinterbrühl}|pw} hinüberfahren für eine Stunde \label{K_L02076-1v}\edtext{Zuſa{\geminationm}enſein}{\lemma{\textnormal{\emph{Zuſammenſein}}}\Cendnote{\textnormal{siehe A. S.: \emph{Tagebuch}, 10. 7. 1912}}}\label{K_L02076-1h}.\pend
           \pstart
           Erbitte alſo eventuell Depeſche \textsc{Villa Louis
                        Friedmann\pwindex{Friedmann, Louis Philipp 29.06.1861 – 01.04.1939@\textsc{Friedmann, Louis Philipp} (29.06.1861 – 01.04.1939), \emph{Industrieller, Bergsteiger}|pw}\oindex{Villa Friedmann@\textbf{Villa Friedmann}|pw}}.\pend
           \pstart \label{T_L02076-1v}\edtext{Freundschaftlich Ihr}{\lemma{\textnormal{\emph{Freundschaftlich Ihr}}}\Cendnote{\textnormal{quer am linken Rand}}}\label{T_L02076-1h}\spacefill\mbox{Hugo}\pend{}
         
         \endnumbering\mylabel{h}\end{ledgroupsized}  \newcommand{\dateiname}{L02076}\newcommand{\titel}{Hugo von Hofmannsthal an Olga Schnitzler, 5. 7. [1912]}\newcommand{\editorInnen}{Martin Anton Müller und Gerd-Hermann Susen}%% latex-leseansicht-abspann.tex
%% Abspann für die Leseansicht.
%% Der Schalter \ifkorrekturansicht ist bereits durch den Vorspann gesetzt.

%% latex-abspann.tex
%% Gemeinsamer Abspann für Korrekturansicht und Leseansicht.
%% Setzt den Schalter \ifkorrekturansicht voraus (gesetzt in den
%% einbindenden Dateien latex-korrekturansicht-abspann.tex bzw.
%% latex-leseansicht-abspann.tex).
%% ---------------------------------------------------------------

\normalsize

% Das esempio-Environment wird nur in der Leseansicht benötigt
\ifkorrekturansicht\else
\newenvironment{esempio}[3]%
{
    \vspace{1.5ex}
    \rlap{\underline{#1}}
    \par
    \setlength{\parindent}{0cm}
    \nopagebreak
    \leftskip=#2cm
    \rightskip=#3cm
}
{
    \par
}
\fi

\doendnotes{C}
\bigskip
\vfill

\clearpage

\footnotesize

\ifkorrekturansicht
  \lohead{\textsc{register}}
\fi

% theindex-Environment neu definieren ohne reledmac
\makeatletter
\renewenvironment{theindex}{%
  \ifkorrekturansicht
    \section*{\indexname}%
  \else
    \subsubsection*{Index der erwähnten Entitäten}%
  \fi
  \setlength{\parindent}{0pt}%
  \setlength{\parskip}{0pt plus 0.3pt}%
  \let\item\@idxitem
}{%
  \ifkorrekturansicht\clearpage\fi
}
\makeatother

\IfFileExists{\jobname-pw.ind}{\input{\jobname-pw.ind}}{}

% Quellenangabe nur in der Leseansicht
\ifkorrekturansicht\else
% Fallback-Definitionen, falls die .tex-Datei \titel etc. nicht gesetzt hat
\providecommand{\titel}{}
\providecommand{\editorInnen}{}
\providecommand{\dateiname}{\jobname}

\vspace{3cm}

\vfill

\footnotesize
\textsc{Quelle}: \titel. Herausgegeben von {\editorInnen}. In: \emph{Arthur Schnitzler: Briefwechsel mit Autorinnen und Autoren}.
 Digitale Edition, https://schnitzler-briefe.acdh.oeaw.ac.at/{\dateiname}.html (Stand \today)
\fi

\end{document}


      