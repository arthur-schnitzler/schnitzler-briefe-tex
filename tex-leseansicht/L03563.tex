%% latex-korrekturansicht-vorspann.tex
%% Vorspann für die Korrekturansicht.
%% Lädt die gemeinsame Datei latex-vorspann.tex mit gesetztem Schalter.

\newif\ifkorrekturansicht
\korrekturansichttrue

\input{../tex-inputs/latex-vorspann}


\section[ Felix und Ottilie Salten an Arthur und Olga Schnitzler, 14. 5. 1914]{L03563 Felix und Ottilie Salten an Arthur und Olga
               Schnitzler, 14. 5. 1914}
\nopagebreak\mylabel{L03563v}
\rehead{ }\normalsize\beginnumbering\briefempfaengerindex{Schnitzler, Olga@\textsc{Schnitzler, Olga}!zzzSalten, Ottilie@\emph{von Ottilie Salten}!1914-05-142@{14. 5. 1914}|(be}\briefempfaengerindex{Schnitzler, Olga@\textsc{Schnitzler, Olga}!zzzSalten, Felix@\emph{von Felix Salten}!1914-05-142@{14. 5. 1914}|(be}\briefempfaengerindex{Schnitzler, Arthur@\textsc{Schnitzler, Arthur}!zzzSalten, Ottilie@\emph{von Ottilie Salten}!1914-05-142@{14. 5. 1914}|(be}\briefempfaengerindex{Schnitzler, Arthur@\textsc{Schnitzler, Arthur}!zzzSalten, Felix@\emph{von Felix Salten}!1914-05-142@{14. 5. 1914}|(be}
\toendnotes[C]{\smallbreak\pagebreak[2]}\Standort{CUL, Schnitzler, B 89, B 2.}
\physDesc{Bildpostkarte, 300 Zeichen
\newline{}Handschrift Felix Salten: schwarze Tinte, lateinische Kurrent
\newline{}Handschrift Ottilie Salten: schwarze Tinte, deutsche Kurrent
\newline{}Versand: Stempel: »\nobreak{}\oindex{Cuxhaven@\textbf{Cuxhaven}, \emph{P.PPLA3}|pwk}C{[}uxh{]}aven 1, 14. 5. 14, 3–4 N.\nobreak{}«.  
\newline{}Ordnung: mit Bleistift von unbekannter Hand nummeriert: »276« }\toendnotes[C]{\smallbreak}\pstart{}{\pb}Herrn u. Frau D\textsuperscript{r} Arthur Schnitzler\pend{}\pstart{}Wien\oindex{Wien@\textbf{Wien}, \emph{A.ADM2}|pw}\pend{}\pstart{}XVIII. Sternwartestraße 71\oindex{Sternwartestrasse 71@\textbf{Sternwartestraße 71}, \emph{Wohngebäude (K.WHS)}|pw}\pend{}{\bigskip}
\pstart
           \noindent{}\centering{}{\pb}\textcolor{gray}{\textbf{HAMBURG-AMERIKA-LINIE\orgindex{Hamburg-Amerika-Linie@Hamburg-Amerika-Linie|pw}}}\pend
           
\pstart
           \centering{}\textcolor{gray}{\textbf{An Bord des Vierschrauben-Turbinen-Schnellpostdampfers}}\pend
           
\pstart
           \centering{}\textcolor{gray}{\textbf{»VATERLAND\orgindex{Schnellpostdampfer Vaterland@Schnellpostdampfer Vaterland|pw}«}}\pend
           
\pstart
           \textcolor{gray}{\textbf{Speisesaal I. Klasse}}\pend
           \vspace{1em}
\pstart
           {\pb}\textcolor{gray}{\textbf{den}}{ }14. V. 14.\pend
           \vspace{0.5em}
\pstart
           Nun sind wir doch \label{K_L03563-1v}\edtext{auch zur See\oindex{Nordsee@\textbf{Nordsee}, \emph{H.SEA}|pwv}}{\lemma{\textnormal{\emph{auch zur See}}}\Cendnote{\textnormal{Arthur und Olga Schnitzler\pwindex{Schnitzler, Olga 17.01.1882 – 13.01.1970@\textsc{Schnitzler, Olga} (17.01.1882 – 13.01.1970), \emph{Schauspieler/Schauspielerin, Sänger/Sängerin}|pwk} befanden sich auf einer
                  Schiffsreise, die am 13. 5. 1914 in Genua\oindex{Genua@\textbf{Genua}, \emph{P.PPLA}|pwk} begann und
                  am 22. 5. 1914 in
                     IJmuiden\oindex{IJmuiden@\textbf{IJmuiden}, \emph{P.PPL}|pwk} endete. Am 20. 5. 1914 machten
                  sie – wenige Tage nach Saltens\pwindex{Salten, Felix 06.09.1869 – 08.10.1945@\textsc{Salten, Felix} (06.09.1869 – 08.10.1945), \emph{Schriftsteller/Schriftstellerin, Journalist/Journalistin, Chefredakteur/Chefredakteurin}|pwk}\pwindex{Salten, Ottilie 07.03.1868 – 22.06.1942@\textsc{Salten, Ottilie} (07.03.1868 – 22.06.1942), \emph{Schauspieler/Schauspielerin}|pwk} –
                  in Southampton\oindex{Southampton@\textbf{Southampton}, \emph{P.PPLA2}|pwk} Station. }}}\label{K_L03563-1}. Wir fahren
               in einer Stunde. Steigen in \begin{otherlanguage}{english}Southampton\end{otherlanguage}\oindex{Southampton@\textbf{Southampton}, \emph{P.PPLA2}|pw} aus, fahren von dort nach London\oindex{London@\textbf{London}, \emph{P.PPLC}|pw} u. Paris\oindex{Paris@\textbf{Paris}, \emph{P.PPLC}|pw}. Viele herzliche Grüße und Reisewünsche gehen von uns zu Ihnen. Ihr {\\}\spacefill\mbox{Felix Salten}\pend
           \selectlanguage{ngerman}\vspace{1em}
\pstart
           \noindent{}{[}hs. :{]} Herzliche Grüße\pend
           \pstart \spacefill\mbox{Ottilie Salten}\pend{}\selectlanguage{ngerman}\endnumbering\briefempfaengerindex{Schnitzler, Olga@\textsc{Schnitzler, Olga}!zzzSalten, Ottilie@\emph{von Ottilie Salten}!1914-05-142@{14. 5. 1914}|)be}\briefempfaengerindex{Schnitzler, Olga@\textsc{Schnitzler, Olga}!zzzSalten, Felix@\emph{von Felix Salten}!1914-05-142@{14. 5. 1914}|)be}\briefempfaengerindex{Schnitzler, Arthur@\textsc{Schnitzler, Arthur}!zzzSalten, Ottilie@\emph{von Ottilie Salten}!1914-05-142@{14. 5. 1914}|)be}\briefempfaengerindex{Schnitzler, Arthur@\textsc{Schnitzler, Arthur}!zzzSalten, Felix@\emph{von Felix Salten}!1914-05-142@{14. 5. 1914}|)be}\mylabel{L03563h}  \normalsize

\doendnotes{C}
\bigskip
\vfill

\clearpage

\footnotesize

\lohead{\textsc{register}}

% Definiere theindex-Environment komplett neu ohne reledmac
\makeatletter
\renewenvironment{theindex}{%
  \section*{\indexname}%
  \setlength{\parindent}{0pt}%
  \setlength{\parskip}{0pt plus 0.3pt}%
  \let\item\@idxitem
}{%
  \clearpage
}
\makeatother

\IfFileExists{\jobname-pw.ind}{\input{\jobname-pw.ind}}{}

\end{document}

      