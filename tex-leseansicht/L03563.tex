%% latex-leseansicht-vorspann.tex
%% Vorspann für die Leseansicht.
%% Lädt die gemeinsame Datei latex-vorspann.tex mit nicht gesetztem Schalter.

\newif\ifkorrekturansicht
\korrekturansichtfalse

\input{../tex-inputs/latex-vorspann}


\section[ Felix und Ottilie Salten an Arthur und Olga Schnitzler, 14. 5. 1914]{L03563 Felix und Ottilie Salten an Arthur und Olga
               Schnitzler,  14. 5. 1914}
\nopagebreak\mylabel{L03563v}
\rehead{ }\normalsize\beginnumbering\briefempfaengerindex{Schnitzler, Olga@\textsc{Schnitzler, Olga}!zzzSalten, Ottilie@\emph{von Ottilie Salten}!1914-05-142@{14. 5. 1914}|(be}\briefempfaengerindex{Schnitzler, Olga@\textsc{Schnitzler, Olga}!zzzSalten, Felix@\emph{von Felix Salten}!1914-05-142@{14. 5. 1914}|(be}\briefempfaengerindex{Schnitzler, Arthur@\textsc{Schnitzler, Arthur}!zzzSalten, Ottilie@\emph{von Ottilie Salten}!1914-05-142@{14. 5. 1914}|(be}\briefempfaengerindex{Schnitzler, Arthur@\textsc{Schnitzler, Arthur}!zzzSalten, Felix@\emph{von Felix Salten}!1914-05-142@{14. 5. 1914}|(be}
\toendnotes[C]{\smallbreak\pagebreak[2]}
\correspDesc{Versand  durch Felix Salten, Ottilie Salten am 14. 5. 1914 in Cuxhaven
\newline{}Erhalt  durch Arthur Schnitzler, Olga Schnitzler im Zeitraum [15. 5. 1914
                  – 19. 5. 1914?] in Wien}\toendnotes[C]{\smallbreak}
\Standort{CUL, Schnitzler, B 89, B 2.}
\physDesc{Bildpostkarte, 300 Zeichen
\newline{}Handschrift Felix Salten: schwarze Tinte, lateinische Kurrent
\newline{}Handschrift Ottilie Salten: schwarze Tinte, deutsche Kurrent
\newline{}Versand: Stempel: »\nobreak{}\oindex{Cuxhaven@\textbf{Cuxhaven}, \emph{Hauptstadt}|pwk}C{[}uxh{]}aven 1, 14. 5. 14, 3–4 N.\nobreak{}«.  
\newline{}Ordnung: mit Bleistift von unbekannter Hand nummeriert: »276« }\toendnotes[C]{\smallbreak}\pstart{}{\pb}Herrn u. Frau D\textsuperscript{r} Arthur Schnitzler\pend{}\pstart{}Wien\oindex{Wien@\textbf{Wien}, \emph{Verwaltungsgebiet}|pw}\pend{}\pstart{}XVIII. Sternwartestraße 71\oindex{Wien@\textbf{Wien}!XVIII., Währing@\textbf{XVIII., Währing}!Sternwartestraße 71@\textbf{Sternwartestraße 71}, \emph{Wohngebäude}|pw}\pend{}{\bigskip}
\pstart
           \noindent{}\centering{}{\pb}\textcolor{gray}{\textbf{HAMBURG-AMERIKA-LINIE\orgindex{Hamburg-Amerika-Linie@Hamburg-Amerika-Linie|pw}}}\pend
           
\pstart
           \centering{}\textcolor{gray}{\textbf{An Bord des Vierschrauben-Turbinen-Schnellpostdampfers}}\pend
           
\pstart
           \centering{}\textcolor{gray}{\textbf{»VATERLAND\orgindex{Schnellpostdampfer Vaterland@Schnellpostdampfer Vaterland|pw}«}}\pend
           
\pstart
           \textcolor{gray}{\textbf{Speisesaal I. Klasse}}\pend
           \vspace{1em}
\pstart
           {\pb}\textcolor{gray}{\textbf{den}}{ }14. V. 14.\pend
           \vspace{0.5em}
\pstart
           Nun sind wir doch \label{K_L03563-1v}\edtext{auch zur See\oindex{Nordsee@\textbf{Nordsee}, \emph{Meer}|pwv}}{\lemma{\textnormal{\emph{auch zur See}}}\Cendnote{\textnormal{Arthur und Olga Schnitzler\pwindex{Schnitzler, Olga 17.\,1.\,1882 Wien – 13.\,1.\,1970 Lugano@\textsc{Schnitzler, Olga} (17.\,1.\,1882 Wien – 13.\,1.\,1970 Lugano), \emph{Schauspielerin, Sängerin}|pwk} befanden sich auf einer
                  Schiffsreise, die am 13. 5. 1914 in Genua\oindex{Genua@\textbf{Genua}|pwk} begann und
                  am 22. 5. 1914 in
                     IJmuiden\oindex{IJmuiden@\textbf{IJmuiden}|pwk} endete. Am 20. 5. 1914 machten
                  sie – wenige Tage nach Saltens\pwindex{Salten, Felix 6.\,9.\,1869 Budapest – 8.\,10.\,1945 Zürich@\textsc{Salten, Felix} (6.\,9.\,1869 Budapest – 8.\,10.\,1945 Zürich), \emph{Schriftsteller, Journalist, Chefredakteur}|pwk}\pwindex{Salten, Ottilie 7.\,3.\,1868 Prag – 22.\,6.\,1942 Zürich@\textsc{Salten, Ottilie} (7.\,3.\,1868 Prag – 22.\,6.\,1942 Zürich), \emph{Schauspielerin}|pwk} –
                  in Southampton\oindex{Southampton@\textbf{Southampton}, \emph{Hauptstadt}|pwk} Station. }}}\label{K_L03563-1}. Wir fahren
               in einer Stunde. Steigen in \begin{otherlanguage}{english}Southampton\end{otherlanguage}\oindex{Southampton@\textbf{Southampton}, \emph{Hauptstadt}|pw} aus, fahren von dort nach London\oindex{London@\textbf{London}, \emph{Hauptstadt}|pw} u. Paris\oindex{Paris@\textbf{Paris}, \emph{Hauptstadt}|pw}. Viele herzliche Grüße und Reisewünsche gehen von uns zu Ihnen. Ihr {\\}\spacefill\mbox{Felix Salten}\pend
           \selectlanguage{ngerman}\vspace{1em}
\pstart
           \noindent{}{[}hs. Salten:{]} Herzliche Grüße\pend
           \pstart \spacefill\mbox{Ottilie Salten}\pend{}\selectlanguage{ngerman}\endnumbering\briefempfaengerindex{Schnitzler, Olga@\textsc{Schnitzler, Olga}!zzzSalten, Ottilie@\emph{von Ottilie Salten}!1914-05-142@{14. 5. 1914}|)be}\briefempfaengerindex{Schnitzler, Olga@\textsc{Schnitzler, Olga}!zzzSalten, Felix@\emph{von Felix Salten}!1914-05-142@{14. 5. 1914}|)be}\briefempfaengerindex{Schnitzler, Arthur@\textsc{Schnitzler, Arthur}!zzzSalten, Ottilie@\emph{von Ottilie Salten}!1914-05-142@{14. 5. 1914}|)be}\briefempfaengerindex{Schnitzler, Arthur@\textsc{Schnitzler, Arthur}!zzzSalten, Felix@\emph{von Felix Salten}!1914-05-142@{14. 5. 1914}|)be}\mylabel{L03563h}  \newcommand{\dateiname}{L03563}\newcommand{\titel}{Felix und Ottilie Salten an Arthur und Olga Schnitzler, 14. 5. 1914}\newcommand{\editorInnen}{Martin Anton Müller und Laura Untner}%% latex-leseansicht-abspann.tex
%% Abspann für die Leseansicht.
%% Der Schalter \ifkorrekturansicht ist bereits durch den Vorspann gesetzt.

%% latex-abspann.tex
%% Gemeinsamer Abspann für Korrekturansicht und Leseansicht.
%% Setzt den Schalter \ifkorrekturansicht voraus (gesetzt in den
%% einbindenden Dateien latex-korrekturansicht-abspann.tex bzw.
%% latex-leseansicht-abspann.tex).
%% ---------------------------------------------------------------

\normalsize

% Das esempio-Environment wird nur in der Leseansicht benötigt
\ifkorrekturansicht\else
\newenvironment{esempio}[3]%
{
    \vspace{1.5ex}
    \rlap{\underline{#1}}
    \par
    \setlength{\parindent}{0cm}
    \nopagebreak
    \leftskip=#2cm
    \rightskip=#3cm
}
{
    \par
}
\fi

\doendnotes{C}
\bigskip
\vfill

\clearpage

\footnotesize

\ifkorrekturansicht
  \lohead{\textsc{register}}
\fi

% theindex-Environment neu definieren ohne reledmac
\makeatletter
\renewenvironment{theindex}{%
  \ifkorrekturansicht
    \section*{\indexname}%
  \else
    \subsubsection*{Index der erwähnten Entitäten}%
  \fi
  \setlength{\parindent}{0pt}%
  \setlength{\parskip}{0pt plus 0.3pt}%
  \let\item\@idxitem
}{%
  \ifkorrekturansicht\clearpage\fi
}
\makeatother

\IfFileExists{\jobname-pw.ind}{\input{\jobname-pw.ind}}{}

% Quellenangabe nur in der Leseansicht
\ifkorrekturansicht\else
% Fallback-Definitionen, falls die .tex-Datei \titel etc. nicht gesetzt hat
\providecommand{\titel}{}
\providecommand{\editorInnen}{}
\providecommand{\dateiname}{\jobname}

\vspace{3cm}

\vfill

\footnotesize
\textsc{Quelle}: \titel. Herausgegeben von {\editorInnen}. In: \emph{Arthur Schnitzler: Briefwechsel mit Autorinnen und Autoren}.
 Digitale Edition, https://schnitzler-briefe.acdh.oeaw.ac.at/{\dateiname}.html (Stand \today)
\fi

\end{document}


