%% latex-leseansicht-vorspann.tex
%% Vorspann für die Leseansicht.
%% Lädt die gemeinsame Datei latex-vorspann.tex mit nicht gesetztem Schalter.

\newif\ifkorrekturansicht
\korrekturansichtfalse

\input{../tex-inputs/latex-vorspann}


               \section[Arthur Schnitzler an Hugo Hofmannsthal, 26. 12. 1925]{ Arthur Schnitzler an Hugo Hofmannsthal, 26. 12. 1925}\nopagebreak\mylabel{v}\rehead{ }\begin{ledgroupsized}[t]{13cm}\normalsize\beginnumbering\briefempfaengerindex{Hofmannsthal, Hugo von@\textsc{Hofmannsthal, Hugo von}!zzzSchnitzler, Arthur@\emph{von Arthur Schnitzler}!1925-12-262@{26. 12. 1925}|(be} \toendnotes[C]{\smallbreak\pagebreak[2]} \Standort{FDH, Hs-30885,155.}
\physDesc{Brief, 1 Blatt, 2 Seiten
\newline{}Handschrift: Bleistift, lateinische Kurrent}\buchAbdrucke{\weitereDrucke{Hugo von Hofmannsthal, Arthur Schnitzler: \emph{Briefwechsel}. Hg. Therese Nickl und Heinrich Schnitzler. Frankfurt am Main: \emph{S. Fischer} 1964, S. 304.} }\toendnotes[C]{\smallbreak}\pstart
           \raggedleft{}{\pb}Wien\oindex{Wien@\textbf{Wien}|pw}, 26/12 925\pend
           \pstart
           mein lieber Hugo, viel Dank für den \label{K_L02461_1v}\edtext{Briefwechsel\pwindex{Strauss, Richard 11.06.1864 – 08.09.1949@\textsc{Strauss, Richard} (11.06.1864 – 08.09.1949), \emph{Theaterleiter, Komponist, Dirigent}!Briefwechsel mit Hugo von Hofmannsthal1926@\strich\emph{Briefwechsel mit Hugo von Hofmannsthal} {[}1926{]}|pwv}}{\lemma{\textnormal{\emph{Briefwechsel}}}\Cendnote{\textnormal{Richard Strauss\pwindex{Strauss, Richard 11.06.1864 – 08.09.1949@\textsc{Strauss, Richard} (11.06.1864 – 08.09.1949), \emph{Theaterleiter, Komponist, Dirigent}|pwk}: \emph{Briefwechsel mit Hugo von Hofmannsthal}\pwindex{Strauss, Richard 11.06.1864 – 08.09.1949@\textsc{Strauss, Richard} (11.06.1864 – 08.09.1949), \emph{Theaterleiter, Komponist, Dirigent}!Briefwechsel mit Hugo von Hofmannsthal1926@\strich\emph{Briefwechsel mit Hugo von Hofmannsthal} {[}1926{]}|pwk}. Berlin,
                            Wien, Leipzig: \emph{Paul Zsolnay}\orgindex{Paul Zsolnay Verlag@Paul Zsolnay Verlag|pwk}{ }1926.}}}\label{K_L02461_1h}. Ich find ihn ganz besonders interessant, aufschließend,
                    anregend und – nebstbei, unglaublich amüsant. Ein wahres Feiertagsvergnügen{\dots}\pend
           \pstart
           Ihr \label{K_L02461_2v}\edtext{Schiller\pwindex{Schiller, Friedrich von 10.11.1759 – 09.05.1805@\textsc{Schiller, Friedrich von} (10.11.1759 – 09.05.1805), \emph{Schriftsteller, Historiker, Philosoph}|pw}-Artikel\pwindex{Hofmannsthal, Hugo von 01.02.1874 – 15.07.1929@\textsc{Hofmannsthal, Hugo von} (01.02.1874 – 15.07.1929), \emph{Schriftsteller}!Schiller25. 12. 1925@\strich\emph{Schiller} {[}25. 12. 1925{]}|pw}}{\lemma{\textnormal{\emph{Schiller-Artikel}}}\Cendnote{\textnormal{Hugo v. Hofmannstal\pwindex{Hofmannsthal, Hugo von 01.02.1874 – 15.07.1929@\textsc{Hofmannsthal, Hugo von} (01.02.1874 – 15.07.1929), \emph{Schriftsteller}|pwk}: \emph{Schiller}\pwindex{Hofmannsthal, Hugo von 01.02.1874 – 15.07.1929@\textsc{Hofmannsthal, Hugo von} (01.02.1874 – 15.07.1929), \emph{Schriftsteller}!Schiller25. 12. 1925@\strich\emph{Schiller} {[}25. 12. 1925{]}|pwk}. In: \emph{Neue
                                Freie Presse}\pwindex{Neue Freie Presse1864 – 1939@\emph{Neue Freie Presse}|pwk}, Nr. 22013,
                                25. 12. 1925, Weihnachtsbeilage,
                            S. 29–33.}}}\label{K_L02461_2h} in d N. Fr. Pr\pwindex{Neue Freie Presse1864 – 1939@\emph{Neue Freie Presse}|pw} war ganz außerordentlich. Ich glaube nicht, daſs es heute in
                    Deutschland neben Ihnen einen Schriftsteller gibt, der im »Essayistischen« (im
                    höchsten Sinn) an dieses Niveau heranreicht. In jedem Absatz, jedem Satz – spürt
                    man den Dichter, – oder vielmehr beide, Schiller\pwindex{Schiller, Friedrich von 10.11.1759 – 09.05.1805@\textsc{Schiller, Friedrich von} (10.11.1759 – 09.05.1805), \emph{Schriftsteller, Historiker, Philosoph}|pw} un\textcolor{gray}{d} Sie; – (ohne dſs Sie je
                    »poetisch« werden, was übrigens den Feuilletonisten eher passirt); – es {\pb}ist mir ein rechtes Bedürfnis, Ihnen bei dieser
                    Gelegenheit wieder einmal – ach man unterläßt es so oft –! meine liebende
                    Bewunderung auszudrücken.\pend
           \pstart
           Alles beste zum neuen Jahr{\\[\baselineskip]}Von Herzen Ihr{\\[\baselineskip]}\spacefill\mbox{Arthur}\pend
           \leftskip=0em{}\endnumbering\briefempfaengerindex{Hofmannsthal, Hugo von@\textsc{Hofmannsthal, Hugo von}!zzzSchnitzler, Arthur@\emph{von Arthur Schnitzler}!1925-12-262@{26. 12. 1925}|)be}\mylabel{h}\end{ledgroupsized}  \newcommand{\dateiname}{L02461}\newcommand{\titel}{Arthur Schnitzler an Hugo Hofmannsthal, 26. 12. 1925}\newcommand{\editorInnen}{Martin Anton Müller und Gerd-Hermann Susen}%% latex-leseansicht-abspann.tex
%% Abspann für die Leseansicht.
%% Der Schalter \ifkorrekturansicht ist bereits durch den Vorspann gesetzt.

%% latex-abspann.tex
%% Gemeinsamer Abspann für Korrekturansicht und Leseansicht.
%% Setzt den Schalter \ifkorrekturansicht voraus (gesetzt in den
%% einbindenden Dateien latex-korrekturansicht-abspann.tex bzw.
%% latex-leseansicht-abspann.tex).
%% ---------------------------------------------------------------

\normalsize

% Das esempio-Environment wird nur in der Leseansicht benötigt
\ifkorrekturansicht\else
\newenvironment{esempio}[3]%
{
    \vspace{1.5ex}
    \rlap{\underline{#1}}
    \par
    \setlength{\parindent}{0cm}
    \nopagebreak
    \leftskip=#2cm
    \rightskip=#3cm
}
{
    \par
}
\fi

\doendnotes{C}
\bigskip
\vfill

\clearpage

\footnotesize

\ifkorrekturansicht
  \lohead{\textsc{register}}
\fi

% theindex-Environment neu definieren ohne reledmac
\makeatletter
\renewenvironment{theindex}{%
  \ifkorrekturansicht
    \section*{\indexname}%
  \else
    \subsubsection*{Index der erwähnten Entitäten}%
  \fi
  \setlength{\parindent}{0pt}%
  \setlength{\parskip}{0pt plus 0.3pt}%
  \let\item\@idxitem
}{%
  \ifkorrekturansicht\clearpage\fi
}
\makeatother

\IfFileExists{\jobname-pw.ind}{\input{\jobname-pw.ind}}{}

% Quellenangabe nur in der Leseansicht
\ifkorrekturansicht\else
% Fallback-Definitionen, falls die .tex-Datei \titel etc. nicht gesetzt hat
\providecommand{\titel}{}
\providecommand{\editorInnen}{}
\providecommand{\dateiname}{\jobname}

\vspace{3cm}

\vfill

\footnotesize
\textsc{Quelle}: \titel. Herausgegeben von {\editorInnen}. In: \emph{Arthur Schnitzler: Briefwechsel mit Autorinnen und Autoren}.
 Digitale Edition, https://schnitzler-briefe.acdh.oeaw.ac.at/{\dateiname}.html (Stand \today)
\fi

\end{document}


      