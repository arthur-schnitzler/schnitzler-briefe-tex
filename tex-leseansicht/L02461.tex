%% latex-korrekturansicht-vorspann.tex
%% Vorspann für die Korrekturansicht.
%% Lädt die gemeinsame Datei latex-vorspann.tex mit gesetztem Schalter.

\newif\ifkorrekturansicht
\korrekturansichttrue

\input{../tex-inputs/latex-vorspann}


\section[Arthur Schnitzler an Hugo Hofmannsthal, 26. 12. 1925]{L02461 Arthur Schnitzler an Hugo Hofmannsthal, 26. 12. 1925}
\nopagebreak\mylabel{L02461v}
\rehead{ }\normalsize\beginnumbering\briefempfaengerindex{Hofmannsthal, Hugo von@\textsc{Hofmannsthal, Hugo von}!zzzSchnitzler, Arthur@\emph{von Arthur Schnitzler}!1925-12-262@{26. 12. 1925}|(be}
\toendnotes[C]{\smallbreak\pagebreak[2]}\Standort{FDH, Hs-30885,155.}
\physDesc{Brief, 1 Blatt, 2 Seiten, 787 Zeichen
\newline{}Handschrift: Bleistift, lateinische Kurrent}
\buchAbdrucke{\weitereDrucke{Hugo von Hofmannsthal, Arthur Schnitzler: \emph{Briefwechsel}. Frankfurt am Main: \emph{S. Fischer} 1964, S. 304.} }\toendnotes[C]{\smallbreak}
\pstart
           \raggedleft{}{\pb}Wien\oindex{Wien@\textbf{Wien}, \emph{A.ADM2}|pw}, 26/12 925\pend
           \vspace{0.5em}
\pstart
           mein lieber Hugo, viel Dank für den \label{K_L02461-1v}\edtext{Briefwechsel\pwindex{Briefwechsel mit Hugo von Hofmannsthal@\emph{Briefwechsel mit Hugo von Hofmannsthal}|pwv}}{\lemma{\textnormal{\emph{Briefwechsel}}}\Cendnote{\textnormal{Richard Strauss\pwindex{Strauss, Richard 11.06.1864 – 08.09.1949@\textsc{Strauss, Richard} (11.06.1864 – 08.09.1949), \emph{Theaterleiter/Theaterleiterin, Komponist/Komponistin, Dirigent/Dirigentin}|pwk}: \emph{Briefwechsel mit Hugo von Hofmannsthal}\pwindex{Briefwechsel mit Hugo von Hofmannsthal@\emph{Briefwechsel mit Hugo von Hofmannsthal}|pwk}. Berlin, Wien,
                     Leipzig: \emph{Paul Zsolnay}\orgindex{Paul Zsolnay Verlag@Paul Zsolnay Verlag|pwk}{ }1926.}}}\label{K_L02461-1}. Ich find ihn ganz besonders interessant, aufschließend, anregend
               und – nebstbei, unglaublich amüsant. Ein wahres Feiertagsvergnügen{\dots}\pend
           
\pstart
           Ihr \label{K_L02461-2v}\edtext{Schiller\pwindex{Schiller, Friedrich von 10.11.1759 – 09.05.1805@\textsc{Schiller, Friedrich von} (10.11.1759 – 09.05.1805), \emph{Schriftsteller/Schriftstellerin, Historiker/Historikerin, Philosoph/Philosophin}|pw}-Artikel\pwindex{Schiller@\emph{Schiller}|pw}}{\lemma{\textnormal{\emph{Schiller-Artikel}}}\Cendnote{\textnormal{Hugo v. Hofmannstal\pwindex{Hofmannsthal, Hugo von 1874-02-01 – 1929-07-15@\textsc{Hofmannsthal, Hugo von} (1874-02-01 – 1929-07-15), \emph{Schriftsteller/Schriftstellerin}|pwk}: \emph{Schiller}\pwindex{Schiller@\emph{Schiller}|pwk}. In: \emph{Neue
                        Freie Presse}\pwindex{Neue Freie Presse@\emph{Neue Freie Presse}|pwk}, Nr. 22.013, 25. 12. 1925, Weihnachtsbeilage,
                     S. 29–33.}}}\label{K_L02461-2} in d N. Fr. Pr\pwindex{Neue Freie Presse@\emph{Neue Freie Presse}|pw} war
               ganz außerordentlich. Ich glaube nicht, daſs es heute in Deutschland neben Ihnen
               einen Schriftsteller gibt, der im »Essayistischen« (im höchsten Sinn) an dieses
               Niveau heranreicht. In jedem Absatz, jedem Satz – spürt man den Dichter, – oder
               vielmehr beide, Schiller\pwindex{Schiller, Friedrich von 10.11.1759 – 09.05.1805@\textsc{Schiller, Friedrich von} (10.11.1759 – 09.05.1805), \emph{Schriftsteller/Schriftstellerin, Historiker/Historikerin, Philosoph/Philosophin}|pw}
                  un\textcolor{gray}{d} Sie; – (ohne dſs Sie je »poetisch« werden, was übrigens den
               Feuilletonisten eher passirt); – es {\pb}ist mir ein rechtes
               Bedürfnis, Ihnen bei dieser Gelegenheit wieder einmal – ach man unterläßt es so
               oft –! meine liebende Bewunderung auszudrücken.\pend
           
\pstart
           Alles beste zum neuen Jahr{\\[\baselineskip]}Von Herzen Ihr{\\[\baselineskip]}\spacefill\mbox{Arthur}\pend
           \leftskip=0em{}\selectlanguage{ngerman}\endnumbering\briefempfaengerindex{Hofmannsthal, Hugo von@\textsc{Hofmannsthal, Hugo von}!zzzSchnitzler, Arthur@\emph{von Arthur Schnitzler}!1925-12-262@{26. 12. 1925}|)be}\mylabel{L02461h}  \normalsize

\doendnotes{C}
\bigskip
\vfill

\clearpage

\footnotesize

\lohead{\textsc{register}}

% Definiere theindex-Environment komplett neu ohne reledmac
\makeatletter
\renewenvironment{theindex}{%
  \section*{\indexname}%
  \setlength{\parindent}{0pt}%
  \setlength{\parskip}{0pt plus 0.3pt}%
  \let\item\@idxitem
}{%
  \clearpage
}
\makeatother

\IfFileExists{\jobname-pw.ind}{\input{\jobname-pw.ind}}{}

\end{document}

      