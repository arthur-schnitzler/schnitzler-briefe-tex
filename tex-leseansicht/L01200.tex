%% latex-korrekturansicht-vorspann.tex
%% Vorspann für die Korrekturansicht.
%% Lädt die gemeinsame Datei latex-vorspann.tex mit gesetztem Schalter.

\newif\ifkorrekturansicht
\korrekturansichttrue

\input{../tex-inputs/latex-vorspann}


\section[Hugo von Hofmannsthal an Arthur Schnitzler, {[}5. 2. 1902?{]}]{L01200 Hugo von Hofmannsthal an Arthur Schnitzler, {[}5. 2. 1902?{]}}
\nopagebreak\mylabel{L01200v}
\rehead{ }\normalsize\beginnumbering\briefempfaengerindex{Schnitzler, Arthur@\textsc{Schnitzler, Arthur}!zzzHofmannsthal, Hugo von@\emph{von Hugo von Hofmannsthal}!1902-02-051@{{[}5. 2. 1902?{]}}|(be}
\toendnotes[C]{\smallbreak\pagebreak[2]}\Standort{CUL, Schnitzler, B 43.}
\physDesc{Brief, 1 Blatt, 2 Seiten, 518 Zeichen
\newline{}Handschrift: schwarze Tinte, deutsche Kurrent
\newline{}Schnitzler: mit Bleistift datiert: »\textsc{Anf Feber 902}« 
\newline{}Ordnung: 1) mit Bleistift von unbekannter Hand nummeriert: »\strikeout{191}«  2) mit Bleistift von unbekannter Hand nummeriert:
                                    »184«}
\buchAbdrucke{\weitereDrucke{Hugo von Hofmannsthal, Arthur Schnitzler: \emph{Briefwechsel}. Frankfurt am Main: \emph{S. Fischer} 1964, S. 153.} }\toendnotes[C]{\smallbreak}
\pstart
           \raggedleft{}{\pb}Mittwoch{ }abends.\pend
           
\pstart{}lieber Arthur\pend\vspace{0.5em}
\pstart
           es wäre ſchön wenn man zusa{\geminationm}en ſpazieren gehen könnte!
               Wir waren heute über Liechtenſtein\oindex{Burg Liechtenstein@\textbf{Burg Liechtenstein}, \emph{Schloss (K.SLS)}|pw} bei Ihnen,
               leider vergeblich.\pend
           
\pstart
           Es würde mir eine große Freude machen, wenn Sie Sonntag gegen
                  ½ 7 zu mir kommen und zum Nachtmahl bleiben würden. Es kommt \textsc{Zemlinsky}\pwindex{Zemlinsky, Alexander von 14.10.1871 – 16.03.1942@\textsc{Zemlinsky, Alexander von} (14.10.1871 – 16.03.1942), \emph{Komponist/Komponistin, Dirigent/Dirigentin, Musiker/Musikerin}|pw}, {\pb}der einiges aus dem \textsc{Ballet}\pwindex{Triumph der Zeit@\emph{Der Triumph der Zeit}|pwv}{ }ſpielen will, Herr \textsc{J. Wolff}\pwindex{Wolff, Erich J. 03.12.1874 – 20.03.1913@\textsc{Wolff, Erich J.} (03.12.1874 – 20.03.1913), \emph{Komponist/Komponistin, Pianist/Pianistin}|pw}, der die \textsc{Pantomime}\pwindex{Schueler. Pantomime in einem Aufzug@\emph{Der Schüler. Pantomime in einem Aufzug}|pwv} auffallend hübſch componiert hat, eine Frau\pwindex{?? [Saengerin] 9.2.1902 – 9.2.1902@\textsc{?? [Sängerin]} (9.2.1902 – 9.2.1902)|pwv}, welche ſingt, ſonſt niemand.\pend
           \pstart Adieu. Von Herzen \spacefill\mbox{Hugo}\pend{}
\pstart
           Samſtag bin ich nicht heraußen.\pend
           
\pstart
           \numberlinefalse{}–\numberlinetrue{}\pend
           
\pstart
           Sie haben Sonntag zur Rückfahrt Dampftramway um 9\textsuperscript{h}40.\pend
           \selectlanguage{ngerman}\endnumbering\briefempfaengerindex{Schnitzler, Arthur@\textsc{Schnitzler, Arthur}!zzzHofmannsthal, Hugo von@\emph{von Hugo von Hofmannsthal}!1902-02-051@{{[}5. 2. 1902?{]}}|)be}\mylabel{L01200h}  \normalsize

\doendnotes{C}
\bigskip
\vfill

\clearpage

\footnotesize

\lohead{\textsc{register}}

% Definiere theindex-Environment komplett neu ohne reledmac
\makeatletter
\renewenvironment{theindex}{%
  \section*{\indexname}%
  \setlength{\parindent}{0pt}%
  \setlength{\parskip}{0pt plus 0.3pt}%
  \let\item\@idxitem
}{%
  \clearpage
}
\makeatother

\IfFileExists{\jobname-pw.ind}{\input{\jobname-pw.ind}}{}

\end{document}

      