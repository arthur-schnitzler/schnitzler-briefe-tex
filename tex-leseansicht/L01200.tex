\input{../tex-inputs/latex-pdf-vorspann}
\begin{center}
            \textcolor{red}{ENTWURF. ENTZIFFERUNG NOCH NICHT KORREKTURGELESEN}
                      \end{center}
            
               \section[Hugo von Hofmannsthal an Arthur Schnitzler, {[}5. 2. 1902?{]}]{ Hugo von Hofmannsthal an Arthur Schnitzler, {[}5. 2. 1902?{]}}\nopagebreak\mylabel{v}\rehead{ }\begin{ledgroupsized}[t]{13cm}\normalsize\beginnumbering\briefempfaengerindex{Schnitzler, Arthur@\textsc{Schnitzler, Arthur}!zzzHofmannsthal, Hugo von@\emph{von Hugo von Hofmannsthal}!1902-02-051@{{[}5. 2. 1902?{]}}|(be} \toendnotes[C]{\smallbreak\pagebreak[2]} \Standort{CUL, Schnitzler, B 43.}
\physDesc{Brief, 1 Blatt, 2 Seiten
\newline{}Handschrift: schwarze Tinte, deutsche Kurrent
\newline{}Schnitzler: mit Bleistift datiert: »\textsc{Anf Feber 902}« \newline{}Ordnung: 1) mit Bleistift von unbekannter Hand nummeriert:
                              »\strikeout{191}« 2) mit Bleistift von unbekannter Hand nummeriert: »184«}\buchAbdrucke{\weitereDrucke{Hugo von Hofmannsthal, Arthur Schnitzler: \emph{Briefwechsel}. Hg. Therese Nickl und Heinrich Schnitzler. Frankfurt am Main: \emph{S. Fischer} 1964, S. 153.} }\toendnotes[C]{\smallbreak}\pstart
           \raggedleft{}{\pb}Mittwoch{ }abends.\pend
           \pstart{}lieber Arthur\pend\pstart
           es wäre ſchön wenn man zusa{\geminationm}en ſpazieren gehen könnte!
               Wir waren heute über Liechtenſtein\oindex{Burg Liechtenstein@\textbf{Burg Liechtenstein}|pw} bei Ihnen,
               leider vergeblich.\pend
           \pstart
           Es würde mir eine große Freude machen, wenn Sie Sonntag gegen
                  ½ 7 zu mir kommen und zum Nachtmahl bleiben würden. Es kommt \textsc{Zemlinsky}\pwindex{Zemlinsky, Alexander von 14.10.1871 – 16.03.1942@\textsc{Zemlinsky, Alexander von} (14.10.1871 – 16.03.1942), \emph{Komponist}|pw}, {\pb}der einiges aus dem \textsc{Ballet}\pwindex{Hofmannsthal, Hugo von 01.02.1874 – 15.07.1929@\textsc{Hofmannsthal, Hugo von} (01.02.1874 – 15.07.1929), \emph{Schriftsteller}!Triumph der Zeit1.9.1901 – 1.9.1901@\strich\emph{Der Triumph der Zeit} {[}1.9.1901 – 1.9.1901{]}|pwv}{ }ſpielen will, Herr \textsc{J. Wolff}\pwindex{Wolff, Erich J. 03.12.1874 – 20.03.1913@\textsc{Wolff, Erich J.} (03.12.1874 – 20.03.1913), \emph{Komponist, Pianist}|pw}, der die \textsc{Pantomime}\pwindex{Hofmannsthal, Hugo von 01.02.1874 – 15.07.1929@\textsc{Hofmannsthal, Hugo von} (01.02.1874 – 15.07.1929), \emph{Schriftsteller}!Schueler. Pantomime in einem Aufzug1.11.1901 – 1.11.1901@\strich\emph{Der Schüler. Pantomime in einem Aufzug} {[}1.11.1901 – 1.11.1901{]}|pwv} auffallend hübſch componiert hat, eine Frau\pwindex{?? [Saengerin] 9.2.1902 – 9.2.1902@\textsc{?? [Sängerin]} (9.2.1902 – 9.2.1902)|pwv}, welche ſingt, ſonſt niemand.\pend
           \pstart Adieu. Von Herzen \spacefill\mbox{Hugo}\pend{}\pstart
           Samſtag bin ich nicht heraußen.\pend
           \pstart
           \numberlinefalse{}–\numberlinetrue{}\pend
           \pstart
           Sie haben Sonntag zur Rückfahrt Dampftramway um 9\textsuperscript{h}40.\pend
           \endnumbering\briefempfaengerindex{Schnitzler, Arthur@\textsc{Schnitzler, Arthur}!zzzHofmannsthal, Hugo von@\emph{von Hugo von Hofmannsthal}!1902-02-051@{{[}5. 2. 1902?{]}}|)be}\mylabel{h}\end{ledgroupsized}  \newcommand{\dateiname}{L01200}\newcommand{\titel}{Hugo von Hofmannsthal an Arthur Schnitzler, [5. 2. 1902?]}\newcommand{\editorInnen}{Martin Anton Müller und Gerd-Hermann Susen}\input{../tex-inputs/latex-pdf-abspann}
      