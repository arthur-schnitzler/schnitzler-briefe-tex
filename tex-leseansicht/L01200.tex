%% latex-leseansicht-vorspann.tex
%% Vorspann für die Leseansicht.
%% Lädt die gemeinsame Datei latex-vorspann.tex mit nicht gesetztem Schalter.

\newif\ifkorrekturansicht
\korrekturansichtfalse

\input{../tex-inputs/latex-vorspann}


\section[Hugo von Hofmannsthal an Arthur Schnitzler, {{[}}5. 2. 1902?{{]}}]{L01200 Hugo von Hofmannsthal an Arthur Schnitzler, {[}5. 2. 1902?{]}}
\nopagebreak\mylabel{L01200v}
\rehead{ }\normalsize\beginnumbering\briefempfaengerindex{Schnitzler, Arthur@\textsc{Schnitzler, Arthur}!zzzHofmannsthal, Hugo von@\emph{von Hugo von Hofmannsthal}!1902-02-051@{{[}5. 2. 1902?{]}}|(be}
\toendnotes[C]{\smallbreak\pagebreak[2]}
\correspDesc{Versand  durch Hugo von Hofmannsthal am [5. 2. 1902?] in Wien
\newline{}Erhalt  durch Arthur Schnitzler im Zeitraum [5. 2. 1902
                  – 9. 2. 1902?] in Wien}\toendnotes[C]{\smallbreak}
\Standort{CUL, Schnitzler, B 43.}
\physDesc{Brief, 1 Blatt, 2 Seiten, 518 Zeichen
\newline{}Handschrift: schwarze Tinte, deutsche Kurrent
\newline{}Schnitzler: mit Bleistift datiert: »\textsc{Anf Feber 902}« 
\newline{}Ordnung: 1) mit Bleistift von unbekannter Hand nummeriert: »\strikeout{191}«  2) mit Bleistift von unbekannter Hand nummeriert:
                                    »184«}
\buchAbdrucke{\weitereDrucke{Hugo von Hofmannsthal, Arthur Schnitzler: \emph{Briefwechsel}. Herausgegeben von Therese Nickl und Heinrich Schnitzler. Frankfurt am Main: \emph{S. Fischer} 1964, S. 153.} }\toendnotes[C]{\smallbreak}
\pstart
           \raggedleft{}{\pb}Mittwoch{ }abends.\pend
           
\pstart{}lieber Arthur\pend\vspace{0.5em}
\pstart
           es wäre{ }ſchön wenn man zusa{\geminationm}en{ }ſpazieren gehen könnte!
               Wir waren heute über Liechtenſtein\oindex{Burg Liechtenstein@\textbf{Burg Liechtenstein}, \emph{Schloss}|pw} bei Ihnen,
               leider vergeblich.\pend
           
\pstart
           Es würde mir eine große Freude machen, wenn Sie Sonntag gegen
                  ½ 7 zu mir kommen und zum Nachtmahl bleiben würden. Es kommt \textsc{Zemlinsky}\pwindex{Zemlinsky, Alexander von 14.\,10.\,1871 Wien – 16.\,3.\,1942 Larchmont@\textsc{Zemlinsky, Alexander von} (14.\,10.\,1871 Wien – 16.\,3.\,1942 Larchmont), \emph{Komponist, Dirigent, Musiker}|pw}, {\pb}der einiges aus dem \textsc{Ballet}\pwindex{Hofmannsthal, Hugo von 1.\,2.\,1874 Wien – 15.\,7.\,1929 Rodaun@\textsc{Hofmannsthal, Hugo von} (1.\,2.\,1874 Wien – 15.\,7.\,1929 Rodaun), \emph{Schriftsteller}!Triumph der Zeit@\strich\emph{Der Triumph der Zeit}|pwv}{ }ſpielen will, Herr \textsc{J. Wolff}\pwindex{Wolff, Erich J. 3.\,12.\,1874 Wien – 20.\,3.\,1913 New York City@\textsc{Wolff, Erich J.} (3.\,12.\,1874 Wien – 20.\,3.\,1913 New York City), \emph{Komponist, Pianist}|pw}, der die \textsc{Pantomime}\pwindex{Hofmannsthal, Hugo von 1.\,2.\,1874 Wien – 15.\,7.\,1929 Rodaun@\textsc{Hofmannsthal, Hugo von} (1.\,2.\,1874 Wien – 15.\,7.\,1929 Rodaun), \emph{Schriftsteller}!Schüler. Pantomime in einem Aufzug@\strich\emph{Der Schüler. Pantomime in einem Aufzug}|pwv} auffallend hübſch componiert hat, eine Frau\pwindex{?? [Sängerin] 9.\,2.\,1902 – 9.\,2.\,1902@\textsc{?? [Sängerin]} (9.\,2.\,1902 – 9.\,2.\,1902)|pwv}, welche{ }ſingt,{ }ſonſt niemand.\pend
           \pstart Adieu. Von Herzen \spacefill\mbox{Hugo}\pend{}
\pstart
           Samſtag bin ich nicht heraußen.\pend
           
\pstart
           \numberlinefalse{}–\numberlinetrue{}\pend
           
\pstart
           Sie haben Sonntag zur Rückfahrt Dampftramway um 9\textsuperscript{h}40.\pend
           \selectlanguage{ngerman}\endnumbering\briefempfaengerindex{Schnitzler, Arthur@\textsc{Schnitzler, Arthur}!zzzHofmannsthal, Hugo von@\emph{von Hugo von Hofmannsthal}!1902-02-051@{{[}5. 2. 1902?{]}}|)be}\mylabel{L01200h}  \newcommand{\dateiname}{L01200}\newcommand{\titel}{Hugo von Hofmannsthal an Arthur Schnitzler, [5. 2. 1902?]}\newcommand{\editorInnen}{Martin Anton Müller und Gerd-Hermann Susen}%% latex-leseansicht-abspann.tex
%% Abspann für die Leseansicht.
%% Der Schalter \ifkorrekturansicht ist bereits durch den Vorspann gesetzt.

%% latex-abspann.tex
%% Gemeinsamer Abspann für Korrekturansicht und Leseansicht.
%% Setzt den Schalter \ifkorrekturansicht voraus (gesetzt in den
%% einbindenden Dateien latex-korrekturansicht-abspann.tex bzw.
%% latex-leseansicht-abspann.tex).
%% ---------------------------------------------------------------

\normalsize

% Das esempio-Environment wird nur in der Leseansicht benötigt
\ifkorrekturansicht\else
\newenvironment{esempio}[3]%
{
    \vspace{1.5ex}
    \rlap{\underline{#1}}
    \par
    \setlength{\parindent}{0cm}
    \nopagebreak
    \leftskip=#2cm
    \rightskip=#3cm
}
{
    \par
}
\fi

\doendnotes{C}
\bigskip
\vfill

\clearpage

\footnotesize

\ifkorrekturansicht
  \lohead{\textsc{register}}
\fi

% theindex-Environment neu definieren ohne reledmac
\makeatletter
\renewenvironment{theindex}{%
  \ifkorrekturansicht
    \section*{\indexname}%
  \else
    \subsubsection*{Index der erwähnten Entitäten}%
  \fi
  \setlength{\parindent}{0pt}%
  \setlength{\parskip}{0pt plus 0.3pt}%
  \let\item\@idxitem
}{%
  \ifkorrekturansicht\clearpage\fi
}
\makeatother

\IfFileExists{\jobname-pw.ind}{\input{\jobname-pw.ind}}{}

% Quellenangabe nur in der Leseansicht
\ifkorrekturansicht\else
% Fallback-Definitionen, falls die .tex-Datei \titel etc. nicht gesetzt hat
\providecommand{\titel}{}
\providecommand{\editorInnen}{}
\providecommand{\dateiname}{\jobname}

\vspace{3cm}

\vfill

\footnotesize
\textsc{Quelle}: \titel. Herausgegeben von {\editorInnen}. In: \emph{Arthur Schnitzler: Briefwechsel mit Autorinnen und Autoren}.
 Digitale Edition, https://schnitzler-briefe.acdh.oeaw.ac.at/{\dateiname}.html (Stand \today)
\fi

\end{document}


