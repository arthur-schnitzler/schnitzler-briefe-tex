%% latex-leseansicht-vorspann.tex
%% Vorspann für die Leseansicht.
%% Lädt die gemeinsame Datei latex-vorspann.tex mit nicht gesetztem Schalter.

\newif\ifkorrekturansicht
\korrekturansichtfalse

\input{../tex-inputs/latex-vorspann}

\begin{center}
            \textcolor{red}{ENTWURF, NICHT FERTIG KORRIGIERT}
                      \end{center}
            
         
         \renewcommand{\erwaehntePersonen}{Personen: Julius Bauer, Eduard Michael Kafka, Bertha Karlsburg, Max L.}
         \renewcommand{\erwaehnteOrte}{Orte: Berggasse, Café Kremser, Ordination Dr. Arthur Schnitzler, Wien}
         \renewcommand{\erwaehnteWerke}{Werke: Tagebuch}
               \section[Felix Salten an Arthur Schnitzler, {[}22. 1.? 1892{]}]{ Felix Salten an Arthur Schnitzler, {[}22. 1.? 1892{]}}\nopagebreak\mylabel{v}\rehead{ }\begin{ledgroupsized}[t]{13cm}\normalsize\beginnumbering \toendnotes[C]{\smallbreak\pagebreak[2]} \Standort{CUL, Schnitzler, B 89, A 1.}
\physDesc{Karte
\newline{}Handschrift: Bleistift, lateinische Kurrent
\newline{}Schnitzler: mit Bleistift datiert: »Anfang 92« }\toendnotes[C]{\smallbreak}\pstart
           \noindent{}{\pb}Lieber Freund! Es wäre mir gerade gestern \uline{sehr} lieb gewesen, wenn Sie in's Kremser\oindex{Cafe Kremser@\textbf{Café Kremser}|pw}
                  geko{\geminationm}en wären. Ich hatte eine \label{K_L03107-1v}\edtext{Begegnung mit B.\pwindex{Karlsburg, Bertha @\textsc{Karlsburg, Bertha}, \emph{Schauspielerin}|pw}}{\lemma{\textnormal{\emph{Begegnung mit B.}}}\Cendnote{\textnormal{Es dürfte sich
                  bei B.\pwindex{Karlsburg, Bertha @\textsc{Karlsburg, Bertha}, \emph{Schauspielerin}|pwk} um jene Person handeln, von der Schnitzler\pwindex{Schnitzler, Arthur 15.05.1862 – 21.10.1931@\textsc{Schnitzler, Arthur} (15.05.1862 – 21.10.1931), \emph{Schriftsteller, Mediziner}|pwk} am 24. 1. 1892 in sein \emph{Tagebuch}\pwindex{Schnitzler, Arthur 15.05.1862 – 21.10.1931@\textsc{Schnitzler, Arthur} (15.05.1862 – 21.10.1931), \emph{Schriftsteller, Mediziner}!Tagebuch1981 – 2000@\strich\emph{Tagebuch} {[}1981 – 2000{]}|pwk} schreibt: »Salten\pwindex{Salten, Felix 06.09.1869 – 08.10.1945@\textsc{Salten, Felix} (06.09.1869 – 08.10.1945), \emph{Schriftsteller, Journalist}|pw} hat von Kafka\pwindex{Kafka, Eduard Michael 11.03.1869 – 06.08.1893@\textsc{Kafka, Eduard Michael} (11.03.1869 – 06.08.1893), \emph{Redakteur}|pw} erfahren,
                           daß seine Gel.\pwindex{Karlsburg, Bertha @\textsc{Karlsburg, Bertha}, \emph{Schauspielerin}|pwv} seit Sommer ein Verh. mit Max L.\pwindex{L., Max @\textsc{L., Max}|pw} habe. Trotzdem verführt sie ihn
                     weiter.« – Da der Eintrag aber von einem Sonntag stammt, Schnitzler\pwindex{Schnitzler, Arthur 15.05.1862 – 21.10.1931@\textsc{Schnitzler, Arthur} (15.05.1862 – 21.10.1931), \emph{Schriftsteller, Mediziner}|pwk}s
                  Ordination also nicht besetzt war, ist anzunehmen, dass das undatierte Korrespondenzstück kurz
                  vorher gelaufen ist, etwa am Freitag, 22. 1. 1892.}}}\label{K_L03107-1h} hatte Gefühlsergüße anzuhören und bin infolgedessen
               ganz hin.\pend
           \pstart
           Ich muss jetzt zu Kafka\pwindex{Kafka, Eduard Michael 11.03.1869 – 06.08.1893@\textsc{Kafka, Eduard Michael} (11.03.1869 – 06.08.1893), \emph{Redakteur}|pw}, u. dann rasch zu Bauer\pwindex{Bauer, Julius 15.10.1853 – 11.06.1941@\textsc{Bauer, Julius} (15.10.1853 – 11.06.1941), \emph{Schriftsteller, Journalist, Kritiker}|pw}, sonst wäre ich in Ihre Ordination\oindex{Ordination Dr. Arthur Schnitzler@\textbf{Ordination Dr. Arthur Schnitzler}|pwv} gekommen. Es ist möglich, dass B.\pwindex{Karlsburg, Bertha @\textsc{Karlsburg, Bertha}, \emph{Schauspielerin}|pw} mich noch aufpaßt, ich habe heute schon
               wenigstens von ihr einen überschweng{\pb}lichen Brief bekommen. \pend
           \pstart
           Bitte, seien Sie im Kremser\oindex{Cafe Kremser@\textbf{Café Kremser}|pw} heute abend.\pend
           \pstart Herzlich Ihr\pend{}\pstart
           \centering{}\textcolor{gray}{\textbf{FELIX SALTEN}}\pend
           \pstart
           \noindent{}\raggedleft{}\textcolor{gray}{\textbf{IX., BERGGASSE 13\oindex{Berggasse@\textbf{Berggasse}|pw}. }}\pend
           
         
         \endnumbering\mylabel{h}\end{ledgroupsized}\begin{anhang}\end{anhang}\newcommand{\dateiname}{L03107}\newcommand{\titel}{Felix Salten an Arthur Schnitzler, [22. 1.? 1892]}\newcommand{\editorInnen}{Martin Anton Müller und Laura Untner}%% latex-leseansicht-abspann.tex
%% Abspann für die Leseansicht.
%% Der Schalter \ifkorrekturansicht ist bereits durch den Vorspann gesetzt.

%% latex-abspann.tex
%% Gemeinsamer Abspann für Korrekturansicht und Leseansicht.
%% Setzt den Schalter \ifkorrekturansicht voraus (gesetzt in den
%% einbindenden Dateien latex-korrekturansicht-abspann.tex bzw.
%% latex-leseansicht-abspann.tex).
%% ---------------------------------------------------------------

\normalsize

% Das esempio-Environment wird nur in der Leseansicht benötigt
\ifkorrekturansicht\else
\newenvironment{esempio}[3]%
{
    \vspace{1.5ex}
    \rlap{\underline{#1}}
    \par
    \setlength{\parindent}{0cm}
    \nopagebreak
    \leftskip=#2cm
    \rightskip=#3cm
}
{
    \par
}
\fi

\doendnotes{C}
\bigskip
\vfill

\clearpage

\footnotesize

\ifkorrekturansicht
  \lohead{\textsc{register}}
\fi

% theindex-Environment neu definieren ohne reledmac
\makeatletter
\renewenvironment{theindex}{%
  \ifkorrekturansicht
    \section*{\indexname}%
  \else
    \subsubsection*{Index der erwähnten Entitäten}%
  \fi
  \setlength{\parindent}{0pt}%
  \setlength{\parskip}{0pt plus 0.3pt}%
  \let\item\@idxitem
}{%
  \ifkorrekturansicht\clearpage\fi
}
\makeatother

\IfFileExists{\jobname-pw.ind}{\input{\jobname-pw.ind}}{}

% Quellenangabe nur in der Leseansicht
\ifkorrekturansicht\else
% Fallback-Definitionen, falls die .tex-Datei \titel etc. nicht gesetzt hat
\providecommand{\titel}{}
\providecommand{\editorInnen}{}
\providecommand{\dateiname}{\jobname}

\vspace{3cm}

\vfill

\footnotesize
\textsc{Quelle}: \titel. Herausgegeben von {\editorInnen}. In: \emph{Arthur Schnitzler: Briefwechsel mit Autorinnen und Autoren}.
 Digitale Edition, https://schnitzler-briefe.acdh.oeaw.ac.at/{\dateiname}.html (Stand \today)
\fi

\end{document}


      