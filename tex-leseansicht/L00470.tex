%% latex-leseansicht-vorspann.tex
%% Vorspann für die Leseansicht.
%% Lädt die gemeinsame Datei latex-vorspann.tex mit nicht gesetztem Schalter.

\newif\ifkorrekturansicht
\korrekturansichtfalse

\input{../tex-inputs/latex-vorspann}


\section[Lou Andreas-Salomé an Arthur Schnitzler, 9. 8. 1895]{L00470 Lou Andreas-Salomé an Arthur Schnitzler, 9. 8. 1895}
\nopagebreak\mylabel{L00470v}
\rehead{ }\normalsize\beginnumbering\briefempfaengerindex{Schnitzler, Arthur@\textsc{Schnitzler, Arthur}!zzzAndreas-Salomé, Lou@\emph{von Lou Andreas-Salomé}!1895-08-091@{9. 8. 1895}|(be}
\toendnotes[C]{\smallbreak\pagebreak[2]}
\correspDesc{Versand  durch Lou Andreas-Salomé am 9. 8. 1895 in Berlin
\newline{}Zustellung  am 10. 8. 1895 in Wien
\newline{}Erhalt  durch Arthur Schnitzler am 11. 8. 1895 in Wien}\toendnotes[C]{\smallbreak}
\Standort{CUL, Schnitzler, B 3.}
\physDesc{Postkarte, 384 Zeichen
\newline{}Handschrift: schwarze Tinte, deutsche Kurrent
\newline{}Versand: 1) Stempel: »\nobreak{}\oindex{Schmargendorf@\textbf{Schmargendorf}, \emph{Ehemaliger Ort}|pwk}Schmargendorf, 9/8 95, 3–4 N\nobreak{}«.   2) Stempel: »\nobreak{}\oindex{IX., Alsergrund@\textbf{IX., Alsergrund}, \emph{Verwaltungsgebiet}|pwk}Wien 9/3, 10.8 95, 7.N, Bestellt\nobreak{}«. 
\newline{}Ordnung: mit rotem Buntstift von unbekannter Hand nummeriert:
                                    »5« }\pstart{}{\pb}Herrn \textsc{D\textsuperscript{r}}\pend{}\pstart{}\textsc{Arthur Schnitzler}\pend{}\pstart{}\textsc{Wien\oindex{Wien@\textbf{Wien}, \emph{Verwaltungsgebiet}|pw}}\pend{}\pstart{}Frankgasse \textsc{N\textsuperscript{o}} 1\oindex{Wien@\textbf{Wien}!IX., Alsergrund@\textbf{IX., Alsergrund}!Frankgasse 1@\textbf{Frankgasse 1}, \emph{Wohngebäude}|pw}. \pend{}{\bigskip}\vspace{1em}
\pstart
           \noindent{}{\pb}Lieber Herr \textsc{D\textsuperscript{r}}! ich{ }ſchreibe Ihnen auf gut Glück in Ihre Wien\oindex{Wien@\textbf{Wien}, \emph{Verwaltungsgebiet}|pw}er Wohnung, um Ihnen zu erzählen, daß ich alſo wirklich nach
                  München\oindex{München@\textbf{München}|pw} reiſe und daß ich am \uline{Montag den 19\textsuperscript{ten}
                     Auguſt}{ }\uline{nach Salzburg\oindex{Salzburg@\textbf{Salzburg}, \emph{Verwaltungsgebiet}|pw}} komme um dort bis gegen den 25\textsuperscript{ten} zu bleiben. Nachrichten erreichen mich \uline{hier}
               bis zum Mittwoch Morgen, 14. Auguſt.\pend
           
\pstart
           Auf Wiederſehen!\pend
           \pstart Mit herzlichem Gruß \spacefill\mbox{Lou Andreas-Salomé}\pend{}\selectlanguage{ngerman}\endnumbering\briefempfaengerindex{Schnitzler, Arthur@\textsc{Schnitzler, Arthur}!zzzAndreas-Salomé, Lou@\emph{von Lou Andreas-Salomé}!1895-08-091@{9. 8. 1895}|)be}\mylabel{L00470h}  \newcommand{\dateiname}{L00470}\newcommand{\titel}{Lou Andreas-Salomé an Arthur Schnitzler, 9. 8. 1895}\newcommand{\editorInnen}{Martin Anton Müller und Gerd-Hermann Susen}%% latex-leseansicht-abspann.tex
%% Abspann für die Leseansicht.
%% Der Schalter \ifkorrekturansicht ist bereits durch den Vorspann gesetzt.

%% latex-abspann.tex
%% Gemeinsamer Abspann für Korrekturansicht und Leseansicht.
%% Setzt den Schalter \ifkorrekturansicht voraus (gesetzt in den
%% einbindenden Dateien latex-korrekturansicht-abspann.tex bzw.
%% latex-leseansicht-abspann.tex).
%% ---------------------------------------------------------------

\normalsize

% Das esempio-Environment wird nur in der Leseansicht benötigt
\ifkorrekturansicht\else
\newenvironment{esempio}[3]%
{
    \vspace{1.5ex}
    \rlap{\underline{#1}}
    \par
    \setlength{\parindent}{0cm}
    \nopagebreak
    \leftskip=#2cm
    \rightskip=#3cm
}
{
    \par
}
\fi

\doendnotes{C}
\bigskip
\vfill

\clearpage

\footnotesize

\ifkorrekturansicht
  \lohead{\textsc{register}}
\fi

% theindex-Environment neu definieren ohne reledmac
\makeatletter
\renewenvironment{theindex}{%
  \ifkorrekturansicht
    \section*{\indexname}%
  \else
    \subsubsection*{Index der erwähnten Entitäten}%
  \fi
  \setlength{\parindent}{0pt}%
  \setlength{\parskip}{0pt plus 0.3pt}%
  \let\item\@idxitem
}{%
  \ifkorrekturansicht\clearpage\fi
}
\makeatother

\IfFileExists{\jobname-pw.ind}{\input{\jobname-pw.ind}}{}

% Quellenangabe nur in der Leseansicht
\ifkorrekturansicht\else
% Fallback-Definitionen, falls die .tex-Datei \titel etc. nicht gesetzt hat
\providecommand{\titel}{}
\providecommand{\editorInnen}{}
\providecommand{\dateiname}{\jobname}

\vspace{3cm}

\vfill

\footnotesize
\textsc{Quelle}: \titel. Herausgegeben von {\editorInnen}. In: \emph{Arthur Schnitzler: Briefwechsel mit Autorinnen und Autoren}.
 Digitale Edition, https://schnitzler-briefe.acdh.oeaw.ac.at/{\dateiname}.html (Stand \today)
\fi

\end{document}


