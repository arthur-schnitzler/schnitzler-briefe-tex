%% latex-leseansicht-vorspann.tex
%% Vorspann für die Leseansicht.
%% Lädt die gemeinsame Datei latex-vorspann.tex mit nicht gesetztem Schalter.

\newif\ifkorrekturansicht
\korrekturansichtfalse

\input{../tex-inputs/latex-vorspann}


               \section[Hugo von Hofmannsthal an Arthur Schnitzler, 13. 10. 1893]{ Hugo von Hofmannsthal an Arthur Schnitzler, 13. 10. 1893}\nopagebreak\mylabel{v}\rehead{ }\begin{ledgroupsized}[t]{13cm}\normalsize\beginnumbering\briefempfaengerindex{Schnitzler, Arthur@\textsc{Schnitzler, Arthur}!zzzHofmannsthal, Hugo von@\emph{von Hugo von Hofmannsthal}!1893-10-132@{13. 10. 1893}|(be} \toendnotes[C]{\smallbreak\pagebreak[2]} \Standort{CUL, Schnitzler, B 43.}
\physDesc{Postkarte
\newline{}Handschrift: Bleistift, deutsche Kurrent\newline{}Versand: 1) Rohrpost 2) Stempel: »\nobreak{}Wien 3/1, 13 X 93, 6 20N\nobreak{}«. 3) Stempel: »\nobreak{}Wien 1/1, 13 X 93, 6 50N\nobreak{}«. 
\newline{}Schnitzler: mit Bleistift nummeriert: »53« und datiert: »13/X 93« }\buchAbdrucke{\weitereDrucke{Hugo von Hofmannsthal, Arthur Schnitzler: \emph{Briefwechsel}. Hg. Therese Nickl und Heinrich Schnitzler. Frankfurt am Main: \emph{S. Fischer} 1964, S. 47.} }\toendnotes[C]{\smallbreak}\pstart{}{\pb}\textsc{Herrn D\textsuperscript{r} Arthur
                            Schnitzler}\pend{}\pstart{}\textsc{I. Grillparzerstrasse 7\oindex{Grillparzerstrasse@\textbf{Grillparzerstraße}|pw}}\pend{}\pstart{}\textsc{Wien\oindex{Wien@\textbf{Wien}|pw}}\pend{}{\bigskip}\pstart{}{\pb}lieber Arthur!\pend\pstart
           Der arme \label{K_L00271_1v}\edtext{Rudolf Schwarzk.\pwindex{Schwarzkopf, Rudolf 25.05.1861 – 13.10.1893@\textsc{Schwarzkopf, Rudolf} (25.05.1861 – 13.10.1893), \emph{Schriftsteller}|pw}}{\lemma{\textnormal{\emph{Rudolf Schwarzk.}}}\Cendnote{\textnormal{Die Todesanzeige (\emph{Neue Freie Presse}\orgindex{Neue Freie Presse@Neue Freie Presse|pwk}, Nr. 10470,
                                15. 10. 1893, S. 14) belegt die
                        Datierung.}}}\label{K_L00271_1h} iſt heute früh in Meran\oindex{Meran@\textbf{Meran}|pw} geſtorben. Guſtav\pwindex{Schwarzkopf, Gustav 07.11.1853 – 13.11.1939@\textsc{Schwarzkopf, Gustav} (07.11.1853 – 13.11.1939), \emph{Schriftsteller}|pw}
                    begräbt ihn dort. Max\pwindex{Schwarzkopf, Max 12.06.1857 – 14.04.1928@\textsc{Schwarzkopf, Max} (12.06.1857 – 14.04.1928), \emph{Rechtsanwalt}|pw} iſt hier. Es iſt ſehr
                    traurig.\pend
           \pstart \spacefill\mbox{Hugo.}\pend{}\endnumbering\briefempfaengerindex{Schnitzler, Arthur@\textsc{Schnitzler, Arthur}!zzzHofmannsthal, Hugo von@\emph{von Hugo von Hofmannsthal}!1893-10-132@{13. 10. 1893}|)be}\mylabel{h}\end{ledgroupsized}  \newcommand{\dateiname}{L00271}\newcommand{\titel}{Hugo von Hofmannsthal an Arthur Schnitzler, 13. 10. 1893}\newcommand{\editorInnen}{Martin Anton Müller und Gerd-Hermann Susen}
            \footnotesize
\begin{ledgroupsized}[t]{11.5cm}
\doendnotes{C}
\end{ledgroupsized}
         %% latex-leseansicht-abspann.tex
%% Abspann für die Leseansicht.
%% Der Schalter \ifkorrekturansicht ist bereits durch den Vorspann gesetzt.

%% latex-abspann.tex
%% Gemeinsamer Abspann für Korrekturansicht und Leseansicht.
%% Setzt den Schalter \ifkorrekturansicht voraus (gesetzt in den
%% einbindenden Dateien latex-korrekturansicht-abspann.tex bzw.
%% latex-leseansicht-abspann.tex).
%% ---------------------------------------------------------------

\normalsize

% Das esempio-Environment wird nur in der Leseansicht benötigt
\ifkorrekturansicht\else
\newenvironment{esempio}[3]%
{
    \vspace{1.5ex}
    \rlap{\underline{#1}}
    \par
    \setlength{\parindent}{0cm}
    \nopagebreak
    \leftskip=#2cm
    \rightskip=#3cm
}
{
    \par
}
\fi

\doendnotes{C}
\bigskip
\vfill

\clearpage

\footnotesize

\ifkorrekturansicht
  \lohead{\textsc{register}}
\fi

% theindex-Environment neu definieren ohne reledmac
\makeatletter
\renewenvironment{theindex}{%
  \ifkorrekturansicht
    \section*{\indexname}%
  \else
    \subsubsection*{Index der erwähnten Entitäten}%
  \fi
  \setlength{\parindent}{0pt}%
  \setlength{\parskip}{0pt plus 0.3pt}%
  \let\item\@idxitem
}{%
  \ifkorrekturansicht\clearpage\fi
}
\makeatother

\IfFileExists{\jobname-pw.ind}{\input{\jobname-pw.ind}}{}

% Quellenangabe nur in der Leseansicht
\ifkorrekturansicht\else
% Fallback-Definitionen, falls die .tex-Datei \titel etc. nicht gesetzt hat
\providecommand{\titel}{}
\providecommand{\editorInnen}{}
\providecommand{\dateiname}{\jobname}

\vspace{3cm}

\vfill

\footnotesize
\textsc{Quelle}: \titel. Herausgegeben von {\editorInnen}. In: \emph{Arthur Schnitzler: Briefwechsel mit Autorinnen und Autoren}.
 Digitale Edition, https://schnitzler-briefe.acdh.oeaw.ac.at/{\dateiname}.html (Stand \today)
\fi

\end{document}


      