%% latex-leseansicht-vorspann.tex
%% Vorspann für die Leseansicht.
%% Lädt die gemeinsame Datei latex-vorspann.tex mit nicht gesetztem Schalter.

\newif\ifkorrekturansicht
\korrekturansichtfalse

\input{../tex-inputs/latex-vorspann}


\section[Max Burckhard an Arthur Schnitzler, {{[}}19. 11.? 1897{{]}}]{L00744 Max Burckhard an Arthur Schnitzler, {[}19. 11.? 1897{]}}
\nopagebreak\mylabel{L00744v}
\rehead{ }\normalsize\beginnumbering\briefempfaengerindex{Schnitzler, Arthur@\textsc{Schnitzler, Arthur}!zzzBurckhard, Max Eugen@\emph{von Max Eugen Burckhard}!1897-11-191@{{[}19. 11.? 1897{]}}|(be}
\toendnotes[C]{\smallbreak\pagebreak[2]}
\correspDesc{Versand  durch Max Burckhard am [19. 11.? 1897] in Wien
\newline{}Erhalt  durch Arthur Schnitzler im Zeitraum [19. 11. 1897 – 23. 11. 1897?] in Wien}\toendnotes[C]{\smallbreak}
\Standort{CUL, Schnitzler, B 20.}
\physDesc{Visitenkarte, 294 Zeichen
\newline{}Handschrift: schwarze Tinte, deutsche Kurrent
\newline{}Schnitzler: mit Bleistift ergänzte Jahreszahl: »97« 
\newline{}Ordnung: mit Bleistift von unbekannter Hand nummeriert:
                                    »30« }\Standort{DLA, A:Schnitzler, HS.NZ85.1.2665, S.  [12].}
\physDesc{maschinenschriftliche Abschrift, 1 Blatt, 1 Seite, 294 Zeichen
\newline{}Schreibmaschine}\toendnotes[C]{\smallbreak}
\pstart
           {\pb}\textcolor{gray}{\textbf{\textsc{D\textsuperscript{r.} Burckhard}}}\pend
           
\pstart
           \textcolor{gray}{\textbf{\textsc{IX. Frankgasse 1\oindex{Wien@\textbf{Wien}!IX., Alsergrund@\textbf{IX., Alsergrund}!Frankgasse 1@\textbf{Frankgasse 1}, \emph{Wohngebäude}|pw}.}}}\pend
           
\pstart{}{\pb}Lieber verehrter Herr Doctor!\pend\vspace{0.5em}
\pstart
           Ich war Ihrer \substVorne{}\textsuperscript{\textcolor{gray}{×}\-\textcolor{gray}{×}\-\textcolor{gray}{×}\-\textcolor{gray}{×}}\substDazwischen{}freund\substHinten{}ſchaftlichen Geſinnung vertrauend bereits heute Vormittag{ }ſo frei Ihnen eine
                  \label{K_L00744-1v}\edtext{Gaſtkarte}{\lemma{\textnormal{\emph{Gastkarte}}}\Cendnote{\textnormal{Das Korrespondenzstück ist undatiert. Im Herbst
                     1897 wurden zwei Theaterstücke Burckhards\pwindex{Burckhard, Max Eugen 14.\,7.\,1854 Korneuburg – 16.\,3.\,1912 Wien@\textsc{Burckhard, Max Eugen} (14.\,7.\,1854 Korneuburg – 16.\,3.\,1912 Wien), \emph{Schriftsteller, Rechtswissenschaftler, Theaterleiter}|pwk} uraufgeführt. Bei der Uraufführung\eventindex{Raimund-Theater@\textbf{Raimund-Theater}!Uraufführung von ’s Katherl, 25.11.1897@Uraufführung von ’s Katherl, 25.11.1897|pwkv} von \emph{’s Katherl}\pwindex{Burckhard, Max Eugen 14.\,7.\,1854 Korneuburg – 16.\,3.\,1912 Wien@\textsc{Burckhard, Max Eugen} (14.\,7.\,1854 Korneuburg – 16.\,3.\,1912 Wien), \emph{Schriftsteller, Rechtswissenschaftler, Theaterleiter}!’s Katherl. Volksstück in fünf Aufzügen@\strich\emph{’s Katherl. Volksstück in fünf Aufzügen}|pwk} am 25. 11. 1897 war Schnitzler verreist. Von \emph{Die Bürgermeisterwahl}\pwindex{Burckhard, Max Eugen 14.\,7.\,1854 Korneuburg – 16.\,3.\,1912 Wien@\textsc{Burckhard, Max Eugen} (14.\,7.\,1854 Korneuburg – 16.\,3.\,1912 Wien), \emph{Schriftsteller, Rechtswissenschaftler, Theaterleiter}!Bürgermeisterwahl. Ländliche Comödie in vier Acten@\strich\emph{Die Bürgermeisterwahl. Ländliche Comödie in vier Acten}|pwk} besuchte er die erste Vorstellung\eventindex{Volkstheater@\textbf{Volkstheater}!Uraufführung von Die Bürgermeisterwahl, 20.11.1897@Uraufführung von Die Bürgermeisterwahl, 20.11.1897|pwkv} am
                     20. 11. 1897 im
                     Deutschen Volkstheater\oindex{Wien@\textbf{Wien}!VII., Neubau@\textbf{VII., Neubau}!Volkstheater@\textbf{Volkstheater}, \emph{Theater}|pwk}, sodass dieses
                     Korrespondenzstück am Vorabend der Premiere\eventindex{Volkstheater@\textbf{Volkstheater}!Uraufführung von Die Bürgermeisterwahl, 20.11.1897@Uraufführung von Die Bürgermeisterwahl, 20.11.1897|pwkv} gelaufen sein könnte.}}}\label{K_L00744-1} für
               morgen zu{ }ſenden, die jedenfalls im Lauf des Nachmittags in Ihren Händen{ }ſein wird.
               Ich danke Ihnen herzlich für Ihre liebenswürdigen Zeilen.\pend
           
\pstart
           Herzlichst{\\[\baselineskip]}\spacefill\mbox{DrBurc}\pend
           \leftskip=0em{}\selectlanguage{ngerman}\endnumbering\briefempfaengerindex{Schnitzler, Arthur@\textsc{Schnitzler, Arthur}!zzzBurckhard, Max Eugen@\emph{von Max Eugen Burckhard}!1897-11-191@{{[}19. 11.? 1897{]}}|)be}\mylabel{L00744h}  \newcommand{\dateiname}{L00744}\newcommand{\titel}{Max Burckhard an Arthur Schnitzler, [19. 11.? 1897]}\newcommand{\editorInnen}{Martin Anton Müller und Gerd-Hermann Susen}%% latex-leseansicht-abspann.tex
%% Abspann für die Leseansicht.
%% Der Schalter \ifkorrekturansicht ist bereits durch den Vorspann gesetzt.

%% latex-abspann.tex
%% Gemeinsamer Abspann für Korrekturansicht und Leseansicht.
%% Setzt den Schalter \ifkorrekturansicht voraus (gesetzt in den
%% einbindenden Dateien latex-korrekturansicht-abspann.tex bzw.
%% latex-leseansicht-abspann.tex).
%% ---------------------------------------------------------------

\normalsize

% Das esempio-Environment wird nur in der Leseansicht benötigt
\ifkorrekturansicht\else
\newenvironment{esempio}[3]%
{
    \vspace{1.5ex}
    \rlap{\underline{#1}}
    \par
    \setlength{\parindent}{0cm}
    \nopagebreak
    \leftskip=#2cm
    \rightskip=#3cm
}
{
    \par
}
\fi

\doendnotes{C}
\bigskip
\vfill

\clearpage

\footnotesize

\ifkorrekturansicht
  \lohead{\textsc{register}}
\fi

% theindex-Environment neu definieren ohne reledmac
\makeatletter
\renewenvironment{theindex}{%
  \ifkorrekturansicht
    \section*{\indexname}%
  \else
    \subsubsection*{Index der erwähnten Entitäten}%
  \fi
  \setlength{\parindent}{0pt}%
  \setlength{\parskip}{0pt plus 0.3pt}%
  \let\item\@idxitem
}{%
  \ifkorrekturansicht\clearpage\fi
}
\makeatother

\IfFileExists{\jobname-pw.ind}{\input{\jobname-pw.ind}}{}

% Quellenangabe nur in der Leseansicht
\ifkorrekturansicht\else
% Fallback-Definitionen, falls die .tex-Datei \titel etc. nicht gesetzt hat
\providecommand{\titel}{}
\providecommand{\editorInnen}{}
\providecommand{\dateiname}{\jobname}

\vspace{3cm}

\vfill

\footnotesize
\textsc{Quelle}: \titel. Herausgegeben von {\editorInnen}. In: \emph{Arthur Schnitzler: Briefwechsel mit Autorinnen und Autoren}.
 Digitale Edition, https://schnitzler-briefe.acdh.oeaw.ac.at/{\dateiname}.html (Stand \today)
\fi

\end{document}


