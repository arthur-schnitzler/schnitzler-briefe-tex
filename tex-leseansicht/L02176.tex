%% latex-korrekturansicht-vorspann.tex
%% Vorspann für die Korrekturansicht.
%% Lädt die gemeinsame Datei latex-vorspann.tex mit gesetztem Schalter.

\newif\ifkorrekturansicht
\korrekturansichttrue

\input{../tex-inputs/latex-vorspann}


\section[Hugo von Hofmannsthal an Arthur Schnitzler, 16. 4. {[}1914{]}]{L02176 Hugo von Hofmannsthal an Arthur Schnitzler, 16. 4. {[}1914{]}}
\nopagebreak\mylabel{L02176v}
\rehead{ }\normalsize\beginnumbering\briefempfaengerindex{Schnitzler, Arthur@\textsc{Schnitzler, Arthur}!zzzHofmannsthal, Hugo von@\emph{von Hugo von Hofmannsthal}!1914-04-162@{16. 4. {[}1914{]}}|(be}
\toendnotes[C]{\smallbreak\pagebreak[2]}\Standort{CUL, Schnitzler, B 43.}
\physDesc{Briefkarte, 936 Zeichen
\newline{}Handschrift: schwarze Tinte, deutsche Kurrent
\newline{}Schnitzler: mit Bleistift die Jahreszahl ergänzt: »914« und beschriftet: »Hofm« 
\newline{}Ordnung: 1) mit Bleistift von unbekannter Hand nummeriert: »\strikeout{336}«  2) mit Bleistift von unbekannter Hand nummeriert:
                                    »349«}
\buchAbdrucke{\weitereDrucke{Hugo von Hofmannsthal, Arthur Schnitzler: \emph{Briefwechsel}. Frankfurt am Main: \emph{S. Fischer} 1964, S. 274–275.} }\toendnotes[C]{\smallbreak}
\pstart
           \raggedleft{}{\pb}Rodaun\oindex{Rodaun@\textbf{Rodaun}, \emph{A.ADM4}|pw}{ }16 IV.\pend
           
\pstart{}mein lieber Arthur \pend\vspace{0.5em}
\pstart
           auch mir iſt das Notwendige, das Conſtante in allem Menſchlichen mit reifenden Jahren
               immer ſtärker vor Augen und in der Seele – und es war nichts anderes als was Sie
               bezeichnen: »leiſe Wehmut« – was mich hatte dieſe Zeilen vom Semmering\oindex{Semmering@\textbf{Semmering}, \emph{A.ADM3}|pw}{ }ſchreiben laſſen.\hspace*{1.5em}Inzwiſchen war ich ein wenig in Nieder-\oindex{Niederoesterreich@\textbf{Niederösterreich}, \emph{A.ADM1}|pw} und
                  Oberoeſterreich\oindex{Oberoesterreich@\textbf{Oberösterreich}, \emph{A.ADM1}|pw}, {\pb}per Auto, ganz im Flug: Amſtetten\oindex{Amstetten@\textbf{Amstetten}, \emph{A.ADM3}|pw} – Iſchl\oindex{Bad Ischl@\textbf{Bad Ischl}, \emph{P.PPL}|pw} – Salzburg\oindex{Salzburg@\textbf{Salzburg}, \emph{A.ADM2}|pw} – dann zurück nach Wels\oindex{Wels@\textbf{Wels}, \emph{P.PPLA2}|pw} – Enns\oindex{Enns@\textbf{Enns}, \emph{P.PPL}|pw}, bei \textsc{Wallsee}\oindex{Wallsee@\textbf{Wallsee}, \emph{P.PPL}|pw} über die \textsc{Donau}\oindex{Donau@\textbf{Donau}, \emph{Fluss (N.FLS)}|pw}, am nördlichen Ufer weiter, eine Nacht in \textsc{Dürnstein}\oindex{Duernstein@\textbf{Dürnstein}, \emph{Besiedelter Ort (A.BSO)}|pw}: dies alles, nächſte Landſchaft, wird mir immer ergreifender, immer
               abgrundtiefer – auch mein eigenes Verhältnis dazu, durch Blut und Nicht-Blut,
               Verbundenheit und Sehnſucht, Nah-ſein und Fern-ſein. Wenn dies ſo fortgeht, ſo muſs
               ja das Alter eine wehrhafte zitternde, leicht fiebernde Jugend ſein. – Wir erwarten
               in dieſen Tagen \textsc{Schroeder}\pwindex{Schroeder, Rudolf Alexander 26.01.1878 – 22.08.1962@\textsc{Schröder, Rudolf Alexander} (26.01.1878 – 22.08.1962), \emph{Schriftsteller/Schriftstellerin}|pw}; ko{\geminationm}t er nicht, was auch leicht möglich, ſo ſind
               wir in allernächſter Zeit \label{T_L02176-1v}\edtext{bei Euch. Von
               Herzen Ihr}{\lemma{\textnormal{\emph{bei Euch. Von
               Herzen Ihr}}}\Cendnote{\textnormal{weiter quer am linken
                  Rand}}}\label{T_L02176-1}\pend
           \pstart \spacefill\mbox{Hugo.}\pend{}\selectlanguage{ngerman}\endnumbering\briefempfaengerindex{Schnitzler, Arthur@\textsc{Schnitzler, Arthur}!zzzHofmannsthal, Hugo von@\emph{von Hugo von Hofmannsthal}!1914-04-162@{16. 4. {[}1914{]}}|)be}\mylabel{L02176h}  \normalsize

\doendnotes{C}
\bigskip
\vfill

\clearpage

\footnotesize

\lohead{\textsc{register}}

% Definiere theindex-Environment komplett neu ohne reledmac
\makeatletter
\renewenvironment{theindex}{%
  \section*{\indexname}%
  \setlength{\parindent}{0pt}%
  \setlength{\parskip}{0pt plus 0.3pt}%
  \let\item\@idxitem
}{%
  \clearpage
}
\makeatother

\IfFileExists{\jobname-pw.ind}{\input{\jobname-pw.ind}}{}

\end{document}

      