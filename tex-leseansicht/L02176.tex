%% latex-leseansicht-vorspann.tex
%% Vorspann für die Leseansicht.
%% Lädt die gemeinsame Datei latex-vorspann.tex mit nicht gesetztem Schalter.

\newif\ifkorrekturansicht
\korrekturansichtfalse

\input{../tex-inputs/latex-vorspann}


\section[Hugo von Hofmannsthal an Arthur Schnitzler, 16. 4. {[}1914{]}]{L02176 Hugo von Hofmannsthal an Arthur Schnitzler, 16. 4. [1914]}
\nopagebreak\mylabel{L02176v}
\rehead{ }\normalsize\beginnumbering\briefempfaengerindex{Schnitzler, Arthur@\textsc{Schnitzler, Arthur}!zzzHofmannsthal, Hugo von@\emph{von Hugo von Hofmannsthal}!1914-04-162@{16. 4. [1914]}|(be}
\toendnotes[C]{\smallbreak\pagebreak[2]}
\correspDesc{Versand  durch Hugo von Hofmannsthal am 16. 4. [1914] in Rodaun
\newline{}Erhalt  durch Arthur Schnitzler im Zeitraum [17. 4. 1914
                  – 21. 4. 1914?] in Wien}\toendnotes[C]{\smallbreak}
\Standort{CUL, Schnitzler, B 43.}
\physDesc{Briefkarte, 936 Zeichen
\newline{}Handschrift: schwarze Tinte, deutsche Kurrent
\newline{}Schnitzler: mit Bleistift die Jahreszahl ergänzt: »914« und beschriftet: »Hofm« 
\newline{}Ordnung: 1) mit Bleistift von unbekannter Hand nummeriert: »\strikeout{336}«  2) mit Bleistift von unbekannter Hand nummeriert:
                                    »349«}
\buchAbdrucke{\weitereDrucke{Hugo von Hofmannsthal, Arthur Schnitzler: \emph{Briefwechsel}. Herausgegeben von Therese Nickl und Heinrich Schnitzler. Frankfurt am Main: \emph{S. Fischer} 1964, S. 274–275.} }\toendnotes[C]{\smallbreak}
\pstart
           \raggedleft{}{\pb}Rodaun\oindex{Wien@\textbf{Wien}!XXIII., Liesing@\textbf{XXIII., Liesing}!Rodaun@\textbf{Rodaun}, \emph{Region}|pw}{ }16 IV.\pend
           
\pstart{}mein lieber Arthur\pend\vspace{0.5em}
\pstart
           auch mir iſt das Notwendige, das Conſtante in allem Menſchlichen mit reifenden Jahren
               immer{ }ſtärker vor Augen und in der Seele – und es war nichts anderes als was Sie
               bezeichnen: »leiſe Wehmut« – was mich hatte dieſe Zeilen vom Semmering\oindex{Semmering@\textbf{Semmering}, \emph{Verwaltungsgebiet}|pw}{ }ſchreiben laſſen.\hspace*{1.5em}Inzwiſchen war ich ein wenig in Nieder-\oindex{Niederösterreich@\textbf{Niederösterreich}, \emph{Land}|pw} und
                  Oberoeſterreich\oindex{Oberösterreich@\textbf{Oberösterreich}, \emph{Land}|pw}, {\pb}per Auto, ganz im Flug: Amſtetten\oindex{Amstetten@\textbf{Amstetten}, \emph{Verwaltungsgebiet}|pw} – Iſchl\oindex{Bad Ischl@\textbf{Bad Ischl}|pw} – Salzburg\oindex{Salzburg@\textbf{Salzburg}, \emph{Verwaltungsgebiet}|pw} – dann zurück nach Wels\oindex{Wels@\textbf{Wels}, \emph{Hauptstadt}|pw} – Enns\oindex{Enns@\textbf{Enns}|pw}, bei \textsc{Wallsee}\oindex{Wallsee@\textbf{Wallsee}|pw} über die \textsc{Donau}\oindex{Donau@\textbf{Donau}, \emph{Fluss}|pw}, am nördlichen Ufer weiter, eine Nacht in \textsc{Dürnstein}\oindex{Dürnstein@\textbf{Dürnstein}|pw}: dies alles, nächſte Landſchaft, wird mir immer ergreifender, immer
               abgrundtiefer – auch mein eigenes Verhältnis dazu, durch Blut und Nicht-Blut,
               Verbundenheit und Sehnſucht, Nah-ſein und Fern-ſein. Wenn dies{ }ſo fortgeht,{ }ſo muſs
               ja das Alter eine wehrhafte zitternde, leicht fiebernde Jugend{ }ſein. – Wir erwarten
               in dieſen Tagen \textsc{Schroeder}\pwindex{Schröder, Rudolf Alexander 26.\,1.\,1878 Bremen – 22.\,8.\,1962 Bad Wiessee@\textsc{Schröder, Rudolf Alexander} (26.\,1.\,1878 Bremen – 22.\,8.\,1962 Bad Wiessee), \emph{Schriftsteller}|pw}; ko{\geminationm}t er nicht, was auch leicht möglich,{ }ſo{ }ſind
               wir in allernächſter Zeit \label{T_L02176-1v}\edtext{bei Euch. Von
               Herzen Ihr}{\lemma{\textnormal{\emph{bei Euch. Von
               Herzen Ihr}}}\Cendnote{\textnormal{weiter quer am linken
                  Rand}}}\label{T_L02176-1}\pend
           \pstart \spacefill\mbox{Hugo.}\pend{}\selectlanguage{ngerman}\endnumbering\briefempfaengerindex{Schnitzler, Arthur@\textsc{Schnitzler, Arthur}!zzzHofmannsthal, Hugo von@\emph{von Hugo von Hofmannsthal}!1914-04-162@{16. 4. [1914]}|)be}\mylabel{L02176h}  \newcommand{\dateiname}{L02176}\newcommand{\titel}{Hugo von Hofmannsthal an Arthur Schnitzler, 16. 4. [1914]}\newcommand{\editorInnen}{Martin Anton Müller und Gerd-Hermann Susen}%% latex-leseansicht-abspann.tex
%% Abspann für die Leseansicht.
%% Der Schalter \ifkorrekturansicht ist bereits durch den Vorspann gesetzt.

%% latex-abspann.tex
%% Gemeinsamer Abspann für Korrekturansicht und Leseansicht.
%% Setzt den Schalter \ifkorrekturansicht voraus (gesetzt in den
%% einbindenden Dateien latex-korrekturansicht-abspann.tex bzw.
%% latex-leseansicht-abspann.tex).
%% ---------------------------------------------------------------

\normalsize

% Das esempio-Environment wird nur in der Leseansicht benötigt
\ifkorrekturansicht\else
\newenvironment{esempio}[3]%
{
    \vspace{1.5ex}
    \rlap{\underline{#1}}
    \par
    \setlength{\parindent}{0cm}
    \nopagebreak
    \leftskip=#2cm
    \rightskip=#3cm
}
{
    \par
}
\fi

\doendnotes{C}
\bigskip
\vfill

\clearpage

\footnotesize

\ifkorrekturansicht
  \lohead{\textsc{register}}
\fi

% theindex-Environment neu definieren ohne reledmac
\makeatletter
\renewenvironment{theindex}{%
  \ifkorrekturansicht
    \section*{\indexname}%
  \else
    \subsubsection*{Index der erwähnten Entitäten}%
  \fi
  \setlength{\parindent}{0pt}%
  \setlength{\parskip}{0pt plus 0.3pt}%
  \let\item\@idxitem
}{%
  \ifkorrekturansicht\clearpage\fi
}
\makeatother

\IfFileExists{\jobname-pw.ind}{\input{\jobname-pw.ind}}{}

% Quellenangabe nur in der Leseansicht
\ifkorrekturansicht\else
% Fallback-Definitionen, falls die .tex-Datei \titel etc. nicht gesetzt hat
\providecommand{\titel}{}
\providecommand{\editorInnen}{}
\providecommand{\dateiname}{\jobname}

\vspace{3cm}

\vfill

\footnotesize
\textsc{Quelle}: \titel. Herausgegeben von {\editorInnen}. In: \emph{Arthur Schnitzler: Briefwechsel mit Autorinnen und Autoren}.
 Digitale Edition, https://schnitzler-briefe.acdh.oeaw.ac.at/{\dateiname}.html (Stand \today)
\fi

\end{document}


