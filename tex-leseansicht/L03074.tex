%% latex-korrekturansicht-vorspann.tex
%% Vorspann für die Korrekturansicht.
%% Lädt die gemeinsame Datei latex-vorspann.tex mit gesetztem Schalter.

\newif\ifkorrekturansicht
\korrekturansichttrue

\input{../tex-inputs/latex-vorspann}


\section[ Paul Goldmann an Arthur Schnitzler, 21. 7. 1901]{L03074 Paul Goldmann an Arthur Schnitzler, 21. 7. 1901}
\nopagebreak\mylabel{L03074v}
\rehead{ }\normalsize\beginnumbering\briefempfaengerindex{Schnitzler, Arthur@\textsc{Schnitzler, Arthur}!zzzGoldmann, Paul@\emph{von Paul Goldmann}!1901-07-211@{21. 7. 1901}|(be}
\toendnotes[C]{\smallbreak\pagebreak[2]}\Standort{DLA, A:Schnitzler, HS.NZ85.1.3171.}
\physDesc{Postkarte, 390 Zeichen
\newline{}Handschrift: 1) blaue Tinte, deutsche Kurrent\hspace{1em}2) blaue Tinte, lateinische Kurrent (\noindent{}Adresse)\hspace{1em}
\newline{}Versand: 1) Stempel: »\nobreak{}\oindex{Berlin@\textbf{Berlin}, \emph{P.PPLC}|pwk}Berlin,
                                       W\textcolor{gray}{.} 9, 21. 7. 01, 2–3 N\nobreak{}«.   2) Stempel: »\nobreak{}\oindex{IX., Alsergrund@\textbf{IX., Alsergrund}, \emph{A.ADM3}|pwk}Wien 9/3 72, 22. 7. 01, 9. {[}V{]}, Bes{[}tell{]}t\nobreak{}«. 
\newline{}Schnitzler: mit Bleistift das Datum »21/7 901« vermerkt }\toendnotes[C]{\smallbreak}\pstart{}{\pb}Herrn\pend{}\pstart{}Dr. Arthur Schnitzler\pend{}\pstart{}Wien\oindex{Wien@\textbf{Wien}, \emph{A.ADM2}|pw}\pend{}\pstart{}IX. Frankgaſse 1\oindex{Frankgasse 1@\textbf{Frankgasse 1}, \emph{Wohngebäude (K.WHS)}|pw}.\pend{}{\bigskip}\vspace{1em}
\pstart
           {\pb}Berlin\oindex{Berlin@\textbf{Berlin}, \emph{P.PPLC}|pw}, 21. Juli.\hfill Mein lieber Freund,\pend
           \vspace{0.5em}
\pstart
           Haſt Du meinen nach \label{K_L03074-1v}\edtext{\textsc{Vahrn\oindex{Vahrn@\textbf{Vahrn}, \emph{P.PPLA3}|pw}} geſandten Brief}{\lemma{\textnormal{\emph{Vahrn geſandten Brief}}}\Cendnote{\textnormal{Paul Goldmann an Arthur Schnitzler, 19. 7. [1901].
               }}}\label{K_L03074-1} erhalten? Du hatteſt keine nähere Adreſſe angegeben. Ich ſchreibe alſo der
               Sicherheit halber noch einmal an Deine Wien\oindex{Wien@\textbf{Wien}, \emph{A.ADM2}|pw}er
               Adreſſe, daß ich Montag reiſe u. Ende der Woche in \textsc{Seekirn\oindex{Sekirn@\textbf{Sekirn}, \emph{P.PPL}|pw}} bei \textsc{Hirschfeld\pwindex{Hirschfeld, Robert 17.09.1857 – 02.04.1914@\textsc{Hirschfeld, Robert} (17.09.1857 – 02.04.1914), \emph{Journalist/Journalistin, Musikkritiker/Musikkritikerin}|pw}} ſein dürfte, wo ich Deine lieben Nachrichten zu finden hoffe. \pend
           \pstart Herzlichſt Dein \spacefill\mbox{P. G.}\pend{}\selectlanguage{ngerman}\endnumbering\briefempfaengerindex{Schnitzler, Arthur@\textsc{Schnitzler, Arthur}!zzzGoldmann, Paul@\emph{von Paul Goldmann}!1901-07-211@{21. 7. 1901}|)be}\mylabel{L03074h}  \normalsize

\doendnotes{C}
\bigskip
\vfill

\clearpage

\footnotesize

\lohead{\textsc{register}}

% Definiere theindex-Environment komplett neu ohne reledmac
\makeatletter
\renewenvironment{theindex}{%
  \section*{\indexname}%
  \setlength{\parindent}{0pt}%
  \setlength{\parskip}{0pt plus 0.3pt}%
  \let\item\@idxitem
}{%
  \clearpage
}
\makeatother

\IfFileExists{\jobname-pw.ind}{\input{\jobname-pw.ind}}{}

\end{document}

      