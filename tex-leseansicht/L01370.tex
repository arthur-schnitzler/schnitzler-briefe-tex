%% latex-leseansicht-vorspann.tex
%% Vorspann für die Leseansicht.
%% Lädt die gemeinsame Datei latex-vorspann.tex mit nicht gesetztem Schalter.

\newif\ifkorrekturansicht
\korrekturansichtfalse

\input{../tex-inputs/latex-vorspann}


               \section[Hermann Bahr an Arthur Schnitzler, 5. 2. 1904]{ Hermann Bahr an Arthur Schnitzler, 5. 2. 1904}\nopagebreak\mylabel{v}\rehead{ }\begin{ledgroupsized}[t]{13cm}\normalsize\beginnumbering\briefempfaengerindex{Schnitzler, Arthur@\textsc{Schnitzler, Arthur}!zzzBahr, Hermann@\emph{von Hermann Bahr}!1904-02-051@{5. 2. 1904}|(be} \toendnotes[C]{\smallbreak\pagebreak[2]} \Standort{CUL, Schnitzler, B 5b.}
\physDesc{Brief, 1 Blatt, 2 Seiten
\newline{}Handschrift: schwarze Tinte, deutsche Kurrent\newline{}Ordnung: mit Bleistift von unbekannter Hand nummeriert: »109« }\buchAbdrucke{\weitereDrucke{Hermann Bahr, Arthur Schnitzler: \emph{Briefwechsel, Aufzeichnungen, Dokumente (1891–1931)}. Hg. Kurt Ifkovits und Martin Anton Müller. Göttingen: \emph{Wallstein} 2018, S. 295–296.} }\toendnotes[C]{\smallbreak}\pstart
           \raggedleft{}{\pb}5. 2. 04\pend
           \pstart{}Lieber Arthur!\pend\pstart
           Mich berührt natürlich der Fichtner\pwindex{Schnitzler, Arthur 15.05.1862 – 21.10.1931@\textsc{Schnitzler, Arthur} (15.05.1862 – 21.10.1931), \emph{Schriftsteller, Mediziner}!einsame Weg. Schauspiel in fuenf Akten1904@\strich\emph{Der einsame Weg. Schauspiel in fünf Akten} {[}1904{]}|pwv} am meiſten, in welchem
               ich unheimlich viel von mir finde (meine Sachen ließen sich kritiſch gar nicht beſſer
               bezeichnen als damit daß ich mich leider auch in ihnen ſozusagen nur vorübergehend
                  aufhielt). Ich verſtehe auch das \damage{Ver}hältnis Julian –
               Wegrath\pwindex{Schnitzler, Arthur 15.05.1862 – 21.10.1931@\textsc{Schnitzler, Arthur} (15.05.1862 – 21.10.1931), \emph{Schriftsteller, Mediziner}!einsame Weg. Schauspiel in fuenf Akten1904@\strich\emph{Der einsame Weg. Schauspiel in fünf Akten} {[}1904{]}|pwv}, ebenſo das Julian –
                  Felix\pwindex{Schnitzler, Arthur 15.05.1862 – 21.10.1931@\textsc{Schnitzler, Arthur} (15.05.1862 – 21.10.1931), \emph{Schriftsteller, Mediziner}!einsame Weg. Schauspiel in fuenf Akten1904@\strich\emph{Der einsame Weg. Schauspiel in fünf Akten} {[}1904{]}|pwv}{ }ſo gut, während ich mir das Sala – Johanna\pwindex{Schnitzler, Arthur 15.05.1862 – 21.10.1931@\textsc{Schnitzler, Arthur} (15.05.1862 – 21.10.1931), \emph{Schriftsteller, Mediziner}!einsame Weg. Schauspiel in fuenf Akten1904@\strich\emph{Der einsame Weg. Schauspiel in fünf Akten} {[}1904{]}|pwv} nicht ganz erklären und mich
               darin nicht zurechtfinden kann. Außerdem miſcht ſich jetzt bei mir Perſönliches in
               alles, ſo die Neugierde, die mich plagt, ob Sala\pwindex{Schnitzler, Arthur 15.05.1862 – 21.10.1931@\textsc{Schnitzler, Arthur} (15.05.1862 – 21.10.1931), \emph{Schriftsteller, Mediziner}!einsame Weg. Schauspiel in fuenf Akten1904@\strich\emph{Der einsame Weg. Schauspiel in fünf Akten} {[}1904{]}|pwv} nicht vollkommen meinen Herzzuſtand hat und wie der
               Arzt dann \textcolor{gray}{denn} doch ſeinen Tod faſt auf den Tag zu wiſſen glauben
               kann – was ſehr albern von mir iſt.\pend
           \pstart
           {\pb}Kritiſch möcht ich ſagen: Daß in dem Stück\pwindex{Schnitzler, Arthur 15.05.1862 – 21.10.1931@\textsc{Schnitzler, Arthur} (15.05.1862 – 21.10.1931), \emph{Schriftsteller, Mediziner}!einsame Weg. Schauspiel in fuenf Akten1904@\strich\emph{Der einsame Weg. Schauspiel in fünf Akten} {[}1904{]}|pwv} viel mehr angeſchlagen und
               aufgeregt als zuletzt ausgelöſt wird, was ich weniger problematiſch als muſikaliſch
               meine. Für mein Gefühl iſt das Stück\pwindex{Schnitzler, Arthur 15.05.1862 – 21.10.1931@\textsc{Schnitzler, Arthur} (15.05.1862 – 21.10.1931), \emph{Schriftsteller, Mediziner}!einsame Weg. Schauspiel in fuenf Akten1904@\strich\emph{Der einsame Weg. Schauspiel in fünf Akten} {[}1904{]}|pwv} aus, bevor es ſeine Stimmungsmotive naturgemäß hat aus- und ablaufen
               laſſen.\pend
           \pstart
           \label{K_L01370_1v}\edtext{Prachtvoll find ich den \textsc{Cassian}\pwindex{Schnitzler, Arthur 15.05.1862 – 21.10.1931@\textsc{Schnitzler, Arthur} (15.05.1862 – 21.10.1931), \emph{Schriftsteller, Mediziner}!einsame Weg. Schauspiel in fuenf Akten1904@\strich\emph{Der einsame Weg. Schauspiel in fünf Akten} {[}1904{]}|pw}}{\lemma{\textnormal{\emph{Prachtvoll … Cassian}}}\Cendnote{\textnormal{Erstdruck: \emph{Der tapfere Cassian. Burleske in einem Akt}\pwindex{Schnitzler, Arthur 15.05.1862 – 21.10.1931@\textsc{Schnitzler, Arthur} (15.05.1862 – 21.10.1931), \emph{Schriftsteller, Mediziner}!tapfere Cassian. Puppenspiel in einem Akt01. 02. 1904@\strich\emph{Der tapfere Cassian. Puppenspiel in einem Akt} {[}01. 02. 1904{]}|pwk}.
                     In: \emph{Neue Rundschau}\pwindex{neue Rundschau1904@\emph{Die neue Rundschau}|pwk}, Jg. 15, H. 2,
                        1. 2. 1904, S. 227–247.}}}\label{K_L01370_1h} und bedaure nur, daß die
               blöden Deutſchen\oindex{Deutschland@\textbf{Deutschland}|pw} für ſolchen argloſen und rein
               ſinnlichen und darum künſtleriſch reinen Humor nun einmal keine Organe
                  habe{[}n{]}.\pend
           \pstart
           Da ich mich ſehr ſchlecht fühle, iſt es möglich, daß ich ſchon ſehr bald hier
               weggehe, vielleicht nach Abbazia\oindex{Opatija@\textbf{Opatija}|pw}. Jedenfalls lockt
               mich der Gedanke, Dich im April in Taormina\oindex{Taormina@\textbf{Taormina}|pw} zu
               finden, ſehr. Hoffentlich.\pend
           \pstart
           Grüß Deine Frau\pwindex{Schnitzler, Olga 17.01.1882 – 13.01.1970@\textsc{Schnitzler, Olga} (17.01.1882 – 13.01.1970), \emph{Schauspielerin, Sängerin}|pwv}, Brahm\pwindex{Brahm, Otto 05.02.1856 – 28.11.1912@\textsc{Brahm, Otto} (05.02.1856 – 28.11.1912), \emph{Theaterleiter, Regisseur}|pw}, den ſtark von Reinhardt\pwindex{Reinhard, Marie 13.03.1871 – 18.03.1899@\textsc{Reinhard, Marie} (13.03.1871 – 18.03.1899), \emph{Gesangspädagogin}|pw} bekümmerten Trebitſch\pwindex{Trebitsch, Siegfried 22.12.1868 – 03.06.1956@\textsc{Trebitsch, Siegfried} (22.12.1868 – 03.06.1956), \emph{Schriftsteller, Übersetzer}|pw} und –
               ich wär ſehr froh, wenn der »einſame Weg\pwindex{Schnitzler, Arthur 15.05.1862 – 21.10.1931@\textsc{Schnitzler, Arthur} (15.05.1862 – 21.10.1931), \emph{Schriftsteller, Mediziner}!einsame Weg. Schauspiel in fuenf Akten1904@\strich\emph{Der einsame Weg. Schauspiel in fünf Akten} {[}1904{]}|pw}« ein
               großer Erfolg würde!\pend
           \pstart
           Herzlichſt{\\[\baselineskip]}\spacefill\mbox{Hermann}\pend
           \leftskip=0em{}\endnumbering\briefempfaengerindex{Schnitzler, Arthur@\textsc{Schnitzler, Arthur}!zzzBahr, Hermann@\emph{von Hermann Bahr}!1904-02-051@{5. 2. 1904}|)be}\mylabel{h}\end{ledgroupsized}  \newcommand{\dateiname}{L01370}\newcommand{\titel}{Hermann Bahr an Arthur Schnitzler, 5. 2. 1904}\newcommand{\editorInnen}{ Kurt Ifkovits,  Martin Anton Müller}%% latex-leseansicht-abspann.tex
%% Abspann für die Leseansicht.
%% Der Schalter \ifkorrekturansicht ist bereits durch den Vorspann gesetzt.

%% latex-abspann.tex
%% Gemeinsamer Abspann für Korrekturansicht und Leseansicht.
%% Setzt den Schalter \ifkorrekturansicht voraus (gesetzt in den
%% einbindenden Dateien latex-korrekturansicht-abspann.tex bzw.
%% latex-leseansicht-abspann.tex).
%% ---------------------------------------------------------------

\normalsize

% Das esempio-Environment wird nur in der Leseansicht benötigt
\ifkorrekturansicht\else
\newenvironment{esempio}[3]%
{
    \vspace{1.5ex}
    \rlap{\underline{#1}}
    \par
    \setlength{\parindent}{0cm}
    \nopagebreak
    \leftskip=#2cm
    \rightskip=#3cm
}
{
    \par
}
\fi

\doendnotes{C}
\bigskip
\vfill

\clearpage

\footnotesize

\ifkorrekturansicht
  \lohead{\textsc{register}}
\fi

% theindex-Environment neu definieren ohne reledmac
\makeatletter
\renewenvironment{theindex}{%
  \ifkorrekturansicht
    \section*{\indexname}%
  \else
    \subsubsection*{Index der erwähnten Entitäten}%
  \fi
  \setlength{\parindent}{0pt}%
  \setlength{\parskip}{0pt plus 0.3pt}%
  \let\item\@idxitem
}{%
  \ifkorrekturansicht\clearpage\fi
}
\makeatother

\IfFileExists{\jobname-pw.ind}{\input{\jobname-pw.ind}}{}

% Quellenangabe nur in der Leseansicht
\ifkorrekturansicht\else
% Fallback-Definitionen, falls die .tex-Datei \titel etc. nicht gesetzt hat
\providecommand{\titel}{}
\providecommand{\editorInnen}{}
\providecommand{\dateiname}{\jobname}

\vspace{3cm}

\vfill

\footnotesize
\textsc{Quelle}: \titel. Herausgegeben von {\editorInnen}. In: \emph{Arthur Schnitzler: Briefwechsel mit Autorinnen und Autoren}.
 Digitale Edition, https://schnitzler-briefe.acdh.oeaw.ac.at/{\dateiname}.html (Stand \today)
\fi

\end{document}


      