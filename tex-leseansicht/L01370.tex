%% latex-korrekturansicht-vorspann.tex
%% Vorspann für die Korrekturansicht.
%% Lädt die gemeinsame Datei latex-vorspann.tex mit gesetztem Schalter.

\newif\ifkorrekturansicht
\korrekturansichttrue

\input{../tex-inputs/latex-vorspann}


\section[Hermann Bahr an Arthur Schnitzler, 5. 2. 1904]{L01370 Hermann Bahr an Arthur Schnitzler, 5. 2. 1904}
\nopagebreak\mylabel{L01370v}
\rehead{ }\normalsize\beginnumbering\briefempfaengerindex{Schnitzler, Arthur@\textsc{Schnitzler, Arthur}!zzzBahr, Hermann@\emph{von Hermann Bahr}!1904-02-051@{5. 2. 1904}|(be}
\toendnotes[C]{\smallbreak\pagebreak[2]}\Standort{CUL, Schnitzler, B 5b.}
\physDesc{Brief, 1 Blatt, 2 Seiten, 1476 Zeichen
\newline{}Handschrift: schwarze Tinte, deutsche Kurrent
\newline{}Ordnung: mit Bleistift von unbekannter Hand nummeriert:
                                    »109« }
\buchAbdrucke{\weitereDrucke{Hermann Bahr, Arthur Schnitzler: \emph{Briefwechsel, Aufzeichnungen, Dokumente (1891–1931)}. Göttingen: \emph{Wallstein} 2018, S. 295–296.} }\toendnotes[C]{\smallbreak}
\pstart
           \raggedleft{}{\pb}5. 2. 04\pend
           
\pstart{}Lieber Arthur!\pend\vspace{0.5em}
\pstart
           Mich berührt natürlich der Fichtner\pwindex{einsame Weg. Schauspiel in fuenf Akten@\emph{Der einsame Weg. Schauspiel in fünf Akten}|pwv} am meiſten, in welchem ich unheimlich viel von mir finde (meine
               Sachen ließen sich kritiſch gar nicht beſſer bezeichnen als damit daß ich mich leider
               auch in ihnen ſozusagen nur vorübergehend aufhielt). Ich verſtehe auch das \damage{Ver}hältnis Julian –
                  Wegrath\pwindex{einsame Weg. Schauspiel in fuenf Akten@\emph{Der einsame Weg. Schauspiel in fünf Akten}|pwv}, ebenſo das Julian
                  – Felix\pwindex{einsame Weg. Schauspiel in fuenf Akten@\emph{Der einsame Weg. Schauspiel in fünf Akten}|pwv}{ }ſo gut, während ich mir das Sala – Johanna\pwindex{einsame Weg. Schauspiel in fuenf Akten@\emph{Der einsame Weg. Schauspiel in fünf Akten}|pwv} nicht ganz erklären und mich
               darin nicht zurechtfinden kann. Außerdem miſcht ſich jetzt bei mir Perſönliches in
               alles, ſo die Neugierde, die mich plagt, ob Sala\pwindex{einsame Weg. Schauspiel in fuenf Akten@\emph{Der einsame Weg. Schauspiel in fünf Akten}|pwv} nicht vollkommen meinen Herzzuſtand hat und wie der
               Arzt dann \textcolor{gray}{denn} doch ſeinen Tod faſt auf den Tag zu wiſſen glauben
               kann – was ſehr albern von mir iſt.\pend
           
\pstart
           {\pb}Kritiſch möcht ich ſagen: Daß in dem Stück\pwindex{einsame Weg. Schauspiel in fuenf Akten@\emph{Der einsame Weg. Schauspiel in fünf Akten}|pwv} viel mehr angeſchlagen
               und aufgeregt als zuletzt ausgelöſt wird, was ich weniger problematiſch als
               muſikaliſch meine. Für mein Gefühl iſt das Stück\pwindex{einsame Weg. Schauspiel in fuenf Akten@\emph{Der einsame Weg. Schauspiel in fünf Akten}|pwv} aus, bevor es ſeine Stimmungsmotive naturgemäß hat
               aus- und ablaufen laſſen.\pend
           
\pstart
           \label{K_L01370-1v}\edtext{Prachtvoll find ich den \textsc{Cassian}\pwindex{einsame Weg. Schauspiel in fuenf Akten@\emph{Der einsame Weg. Schauspiel in fünf Akten}|pw}}{\lemma{\textnormal{\emph{Prachtvoll … Cassian}}}\Cendnote{\textnormal{Erstdruck: \emph{Der tapfere Cassian. Burleske in einem
                     Akt}\pwindex{tapfere Cassian. Puppenspiel in einem Akt@\emph{Der tapfere Cassian. Puppenspiel in einem Akt}|pwk}. In: \emph{Die neue Rundschau}\pwindex{neue Rundschau@\emph{Die neue Rundschau}|pwk}, Jg. 15,
                     H. 2, 1. 2. 1904, S. 227–247.}}}\label{K_L01370-1} und bedaure nur, daß
               die blöden Deutſchen\oindex{Deutschland@\textbf{Deutschland}, \emph{A.PCLI}|pw} für ſolchen argloſen und
               rein ſinnlichen und darum künſtleriſch reinen Humor nun einmal keine Organe
                  habe{[}n{]}.\pend
           
\pstart
           Da ich mich ſehr ſchlecht fühle, iſt es möglich, daß ich ſchon ſehr bald hier
               weggehe, vielleicht nach Abbazia\oindex{Opatija@\textbf{Opatija}, \emph{P.PPLA2}|pw}. Jedenfalls
               lockt mich der Gedanke, Dich im April in Taormina\oindex{Taormina@\textbf{Taormina}, \emph{P.PPLA3}|pw}
               zu finden, ſehr. Hoffentlich.\pend
           
\pstart
           Grüß Deine Frau\pwindex{Schnitzler, Olga 17.01.1882 – 13.01.1970@\textsc{Schnitzler, Olga} (17.01.1882 – 13.01.1970), \emph{Schauspieler/Schauspielerin, Sänger/Sängerin}|pwv}, Brahm\pwindex{Brahm, Otto 05.02.1856 – 28.11.1912@\textsc{Brahm, Otto} (05.02.1856 – 28.11.1912), \emph{Theaterleiter/Theaterleiterin, Regisseur/Regisseurin}|pw}, den ſtark von Reinhardt\pwindex{Reinhard, Marie 1871-03-13 – 1899-03-18@\textsc{Reinhard, Marie} (1871-03-13 – 1899-03-18), \emph{Gesangspädagoge/Gesangspädagogin}|pw} bekümmerten Trebitſch\pwindex{Trebitsch, Siegfried 22.12.1868 – 03.06.1956@\textsc{Trebitsch, Siegfried} (22.12.1868 – 03.06.1956), \emph{Schriftsteller/Schriftstellerin, Übersetzer/Übersetzerin}|pw} und – ich wär ſehr froh, wenn der »einſame Weg\pwindex{einsame Weg. Schauspiel in fuenf Akten@\emph{Der einsame Weg. Schauspiel in fünf Akten}|pw}« ein großer Erfolg würde!\pend
           
\pstart
           Herzlichſt{\\[\baselineskip]}\spacefill\mbox{Hermann}\pend
           \leftskip=0em{}\selectlanguage{ngerman}\endnumbering\briefempfaengerindex{Schnitzler, Arthur@\textsc{Schnitzler, Arthur}!zzzBahr, Hermann@\emph{von Hermann Bahr}!1904-02-051@{5. 2. 1904}|)be}\mylabel{L01370h}  \normalsize

\doendnotes{C}
\bigskip
\vfill

\clearpage

\footnotesize

\lohead{\textsc{register}}

% Definiere theindex-Environment komplett neu ohne reledmac
\makeatletter
\renewenvironment{theindex}{%
  \section*{\indexname}%
  \setlength{\parindent}{0pt}%
  \setlength{\parskip}{0pt plus 0.3pt}%
  \let\item\@idxitem
}{%
  \clearpage
}
\makeatother

\IfFileExists{\jobname-pw.ind}{\input{\jobname-pw.ind}}{}

\end{document}

      