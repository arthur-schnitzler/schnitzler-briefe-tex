%% latex-leseansicht-vorspann.tex
%% Vorspann für die Leseansicht.
%% Lädt die gemeinsame Datei latex-vorspann.tex mit nicht gesetztem Schalter.

\newif\ifkorrekturansicht
\korrekturansichtfalse

\input{../tex-inputs/latex-vorspann}


\section[Hermann Bahr an Arthur Schnitzler, 5. 2. 1904]{L01370 Hermann Bahr an Arthur Schnitzler, 5. 2. 1904}
\nopagebreak\mylabel{L01370v}
\rehead{ }\normalsize\beginnumbering\briefempfaengerindex{Schnitzler, Arthur@\textsc{Schnitzler, Arthur}!zzzBahr, Hermann@\emph{von Hermann Bahr}!1904-02-051@{5. 2. 1904}|(be}
\toendnotes[C]{\smallbreak\pagebreak[2]}
\correspDesc{Versand  durch Hermann Bahr am 5. 2. 1904 in Marbach am Bodensee
\newline{}Erhalt  durch Arthur Schnitzler im Zeitraum [6. 2. 1904
                  – 10. 2. 1904?] in Berlin}\toendnotes[C]{\smallbreak}
\Standort{CUL, Schnitzler, B 5b.}
\physDesc{Brief, 1 Blatt, 2 Seiten, 1476 Zeichen
\newline{}Handschrift: schwarze Tinte, deutsche Kurrent
\newline{}Ordnung: mit Bleistift von unbekannter Hand nummeriert:
                                    »109« }
\buchAbdrucke{\weitereDrucke{Hermann Bahr, Arthur Schnitzler: \emph{Briefwechsel, Aufzeichnungen, Dokumente (1891–1931)}. Herausgegeben von Kurt Ifkovits und Martin Anton Müller. Göttingen: \emph{Wallstein} 2018, S. 295–296.} }\toendnotes[C]{\smallbreak}
\pstart
           \raggedleft{}{\pb}5. 2. 04\pend
           
\pstart{}Lieber Arthur!\pend\vspace{0.5em}
\pstart
           Mich berührt natürlich der Fichtner\pwindex{Schnitzler, Arthur 15.\,5.\,1862 Wien – 21.\,10.\,1931 ebd.@\textsc{Schnitzler, Arthur} (15.\,5.\,1862 Wien – 21.\,10.\,1931 ebd.), \emph{Schriftsteller, Mediziner}!einsame Weg. Schauspiel in fünf Akten@\strich\emph{Der einsame Weg. Schauspiel in fünf Akten}|pwv} am meiſten, in welchem ich unheimlich viel von mir finde (meine
               Sachen ließen sich kritiſch gar nicht beſſer bezeichnen als damit daß ich mich leider
               auch in ihnen{ }ſozusagen nur vorübergehend aufhielt). Ich verſtehe auch das \damage{Ver}hältnis Julian –
                  Wegrath\pwindex{Schnitzler, Arthur 15.\,5.\,1862 Wien – 21.\,10.\,1931 ebd.@\textsc{Schnitzler, Arthur} (15.\,5.\,1862 Wien – 21.\,10.\,1931 ebd.), \emph{Schriftsteller, Mediziner}!einsame Weg. Schauspiel in fünf Akten@\strich\emph{Der einsame Weg. Schauspiel in fünf Akten}|pwv}, ebenſo das Julian
                  – Felix\pwindex{Schnitzler, Arthur 15.\,5.\,1862 Wien – 21.\,10.\,1931 ebd.@\textsc{Schnitzler, Arthur} (15.\,5.\,1862 Wien – 21.\,10.\,1931 ebd.), \emph{Schriftsteller, Mediziner}!einsame Weg. Schauspiel in fünf Akten@\strich\emph{Der einsame Weg. Schauspiel in fünf Akten}|pwv}{ }ſo gut, während ich mir das Sala – Johanna\pwindex{Schnitzler, Arthur 15.\,5.\,1862 Wien – 21.\,10.\,1931 ebd.@\textsc{Schnitzler, Arthur} (15.\,5.\,1862 Wien – 21.\,10.\,1931 ebd.), \emph{Schriftsteller, Mediziner}!einsame Weg. Schauspiel in fünf Akten@\strich\emph{Der einsame Weg. Schauspiel in fünf Akten}|pwv} nicht ganz erklären und mich
               darin nicht zurechtfinden kann. Außerdem miſcht{ }ſich jetzt bei mir Perſönliches in
               alles,{ }ſo die Neugierde, die mich plagt, ob Sala\pwindex{Schnitzler, Arthur 15.\,5.\,1862 Wien – 21.\,10.\,1931 ebd.@\textsc{Schnitzler, Arthur} (15.\,5.\,1862 Wien – 21.\,10.\,1931 ebd.), \emph{Schriftsteller, Mediziner}!einsame Weg. Schauspiel in fünf Akten@\strich\emph{Der einsame Weg. Schauspiel in fünf Akten}|pwv} nicht vollkommen meinen Herzzuſtand hat und wie der
               Arzt dann \textcolor{gray}{denn} doch{ }ſeinen Tod faſt auf den Tag zu wiſſen glauben
               kann – was{ }ſehr albern von mir iſt.\pend
           
\pstart
           {\pb}Kritiſch möcht ich{ }ſagen: Daß in dem Stück\pwindex{Schnitzler, Arthur 15.\,5.\,1862 Wien – 21.\,10.\,1931 ebd.@\textsc{Schnitzler, Arthur} (15.\,5.\,1862 Wien – 21.\,10.\,1931 ebd.), \emph{Schriftsteller, Mediziner}!einsame Weg. Schauspiel in fünf Akten@\strich\emph{Der einsame Weg. Schauspiel in fünf Akten}|pwv} viel mehr angeſchlagen
               und aufgeregt als zuletzt ausgelöſt wird, was ich weniger problematiſch als
               muſikaliſch meine. Für mein Gefühl iſt das Stück\pwindex{Schnitzler, Arthur 15.\,5.\,1862 Wien – 21.\,10.\,1931 ebd.@\textsc{Schnitzler, Arthur} (15.\,5.\,1862 Wien – 21.\,10.\,1931 ebd.), \emph{Schriftsteller, Mediziner}!einsame Weg. Schauspiel in fünf Akten@\strich\emph{Der einsame Weg. Schauspiel in fünf Akten}|pwv} aus, bevor es{ }ſeine Stimmungsmotive naturgemäß hat
               aus- und ablaufen laſſen.\pend
           
\pstart
           \label{K_L01370-1v}\edtext{Prachtvoll find ich den \textsc{Cassian}\pwindex{Schnitzler, Arthur 15.\,5.\,1862 Wien – 21.\,10.\,1931 ebd.@\textsc{Schnitzler, Arthur} (15.\,5.\,1862 Wien – 21.\,10.\,1931 ebd.), \emph{Schriftsteller, Mediziner}!einsame Weg. Schauspiel in fünf Akten@\strich\emph{Der einsame Weg. Schauspiel in fünf Akten}|pw}}{\lemma{\textnormal{\emph{Prachtvoll … Cassian}}}\Cendnote{\textnormal{Erstdruck: \emph{Der tapfere Cassian. Burleske in einem
                     Akt}\pwindex{Schnitzler, Arthur 15.\,5.\,1862 Wien – 21.\,10.\,1931 ebd.@\textsc{Schnitzler, Arthur} (15.\,5.\,1862 Wien – 21.\,10.\,1931 ebd.), \emph{Schriftsteller, Mediziner}!tapfere Cassian. Puppenspiel in einem Akt@\strich\emph{Der tapfere Cassian. Puppenspiel in einem Akt}|pwk}. In: \emph{Die neue Rundschau}\pwindex{neue Rundschau@\emph{Die neue Rundschau}|pwk}, Jg. 15,
                     H. 2, 1. 2. 1904, S. 227–247.}}}\label{K_L01370-1} und bedaure nur, daß
               die blöden Deutſchen\oindex{Deutschland@\textbf{Deutschland}|pw} für{ }ſolchen argloſen und
               rein{ }ſinnlichen und darum künſtleriſch reinen Humor nun einmal keine Organe
                  habe{[}n{]}.\pend
           
\pstart
           Da ich mich{ }ſehr{ }ſchlecht fühle, iſt es möglich, daß ich{ }ſchon{ }ſehr bald hier
               weggehe, vielleicht nach Abbazia\oindex{Opatija@\textbf{Opatija}, \emph{Hauptstadt}|pw}. Jedenfalls
               lockt mich der Gedanke, Dich im April in Taormina\oindex{Taormina@\textbf{Taormina}, \emph{Hauptstadt}|pw}
               zu finden,{ }ſehr. Hoffentlich.\pend
           
\pstart
           Grüß Deine Frau\pwindex{Schnitzler, Olga 17.\,1.\,1882 Wien – 13.\,1.\,1970 Lugano@\textsc{Schnitzler, Olga} (17.\,1.\,1882 Wien – 13.\,1.\,1970 Lugano), \emph{Schauspielerin, Sängerin}|pwv}, Brahm\pwindex{Brahm, Otto 5.\,2.\,1856 Hamburg – 28.\,11.\,1912 Berlin@\textsc{Brahm, Otto} (5.\,2.\,1856 Hamburg – 28.\,11.\,1912 Berlin), \emph{Theaterleiter, Regisseur}|pw}, den{ }ſtark von Reinhardt\pwindex{Reinhard, Marie 13.\,3.\,1871 Wien – 18.\,3.\,1899 ebd.@\textsc{Reinhard, Marie} (13.\,3.\,1871 Wien – 18.\,3.\,1899 ebd.), \emph{Gesangspädagogin}|pw} bekümmerten Trebitſch\pwindex{Trebitsch, Siegfried 22.\,12.\,1868 Wien – 3.\,6.\,1956 Zürich@\textsc{Trebitsch, Siegfried} (22.\,12.\,1868 Wien – 3.\,6.\,1956 Zürich), \emph{Schriftsteller, Übersetzer}|pw} und – ich wär{ }ſehr froh, wenn der »einſame Weg\pwindex{Schnitzler, Arthur 15.\,5.\,1862 Wien – 21.\,10.\,1931 ebd.@\textsc{Schnitzler, Arthur} (15.\,5.\,1862 Wien – 21.\,10.\,1931 ebd.), \emph{Schriftsteller, Mediziner}!einsame Weg. Schauspiel in fünf Akten@\strich\emph{Der einsame Weg. Schauspiel in fünf Akten}|pw}« ein großer Erfolg würde!\pend
           
\pstart
           Herzlichſt{\\[\baselineskip]}\spacefill\mbox{Hermann}\pend
           \leftskip=0em{}\selectlanguage{ngerman}\endnumbering\briefempfaengerindex{Schnitzler, Arthur@\textsc{Schnitzler, Arthur}!zzzBahr, Hermann@\emph{von Hermann Bahr}!1904-02-051@{5. 2. 1904}|)be}\mylabel{L01370h}  \newcommand{\dateiname}{L01370}\newcommand{\titel}{Hermann Bahr an Arthur Schnitzler, 5. 2. 1904}\newcommand{\editorInnen}{Herausgegeben von Martin Anton Müller}%% latex-leseansicht-abspann.tex
%% Abspann für die Leseansicht.
%% Der Schalter \ifkorrekturansicht ist bereits durch den Vorspann gesetzt.

%% latex-abspann.tex
%% Gemeinsamer Abspann für Korrekturansicht und Leseansicht.
%% Setzt den Schalter \ifkorrekturansicht voraus (gesetzt in den
%% einbindenden Dateien latex-korrekturansicht-abspann.tex bzw.
%% latex-leseansicht-abspann.tex).
%% ---------------------------------------------------------------

\normalsize

% Das esempio-Environment wird nur in der Leseansicht benötigt
\ifkorrekturansicht\else
\newenvironment{esempio}[3]%
{
    \vspace{1.5ex}
    \rlap{\underline{#1}}
    \par
    \setlength{\parindent}{0cm}
    \nopagebreak
    \leftskip=#2cm
    \rightskip=#3cm
}
{
    \par
}
\fi

\doendnotes{C}
\bigskip
\vfill

\clearpage

\footnotesize

\ifkorrekturansicht
  \lohead{\textsc{register}}
\fi

% theindex-Environment neu definieren ohne reledmac
\makeatletter
\renewenvironment{theindex}{%
  \ifkorrekturansicht
    \section*{\indexname}%
  \else
    \subsubsection*{Index der erwähnten Entitäten}%
  \fi
  \setlength{\parindent}{0pt}%
  \setlength{\parskip}{0pt plus 0.3pt}%
  \let\item\@idxitem
}{%
  \ifkorrekturansicht\clearpage\fi
}
\makeatother

\IfFileExists{\jobname-pw.ind}{\input{\jobname-pw.ind}}{}

% Quellenangabe nur in der Leseansicht
\ifkorrekturansicht\else
% Fallback-Definitionen, falls die .tex-Datei \titel etc. nicht gesetzt hat
\providecommand{\titel}{}
\providecommand{\editorInnen}{}
\providecommand{\dateiname}{\jobname}

\vspace{3cm}

\vfill

\footnotesize
\textsc{Quelle}: \titel. Herausgegeben von {\editorInnen}. In: \emph{Arthur Schnitzler: Briefwechsel mit Autorinnen und Autoren}.
 Digitale Edition, https://schnitzler-briefe.acdh.oeaw.ac.at/{\dateiname}.html (Stand \today)
\fi

\end{document}


