%% latex-leseansicht-vorspann.tex
%% Vorspann für die Leseansicht.
%% Lädt die gemeinsame Datei latex-vorspann.tex mit nicht gesetztem Schalter.

\newif\ifkorrekturansicht
\korrekturansichtfalse

\input{../tex-inputs/latex-vorspann}


\section[Arthur Schnitzler an Richard Beer-Hofmann, 10. 11. 1894]{L00400 Arthur Schnitzler an Richard Beer-Hofmann, 10. 11. 1894}
\nopagebreak\mylabel{L00400v}
\rehead{ }\normalsize\beginnumbering\briefempfaengerindex{Beer-Hofmann, Richard@\textsc{Beer-Hofmann, Richard}!zzzSchnitzler, Arthur@\emph{von Arthur Schnitzler}!1894-11-101@{10. 11. 1894}|(be}
\toendnotes[C]{\smallbreak\pagebreak[2]}
\correspDesc{Versand  durch Arthur Schnitzler am 10. 11. 1894 in Wien
\newline{}Erhalt  durch Richard Beer-Hofmann im Zeitraum [10. 11. 1894 – 14. 11. 1894?] in Wien}\toendnotes[C]{\smallbreak}
\Standort{YCGL, MSS 31.}
\physDesc{Postkarte, 223 Zeichen
\newline{}Handschrift: Bleistift, deutsche Kurrent
\newline{}Versand: Stempel: »\nobreak{}\oindex{I., Innere Stadt@\textbf{I., Innere Stadt}, \emph{Verwaltungsgebiet}|pwk}Wien 1/1, 10. 11. 94, 7–8 N\nobreak{}«.  }\toendnotes[C]{\smallbreak}\pstart{}{\pb}Herrn \textsc{Dr. Rich
                     Beer-Hofmann}\pend{}\pstart{}Wien\oindex{Wien@\textbf{Wien}, \emph{Verwaltungsgebiet}|pw}\pend{}\pstart{}\textsc{I Wollzeile 15\oindex{Wien@\textbf{Wien}!I., Innere Stadt@\textbf{I., Innere Stadt}!Wollzeile 15 (»Berthahof«)@\textbf{Wollzeile 15 (»Berthahof«)}, \emph{Wohngebäude}|pw}}\pend{}{\bigskip}\vspace{1em}
\pstart
           \raggedleft{}{\pb}Samſtag\pend
           \vspace{0.5em}
\pstart
           Lieber Richard. Ich bin heut beim Doppelſelbſtmord\pwindex{\textcolor{red}{\textsuperscript{XXXX indx1}}!Doppelselbstmord@\strich\emph{Doppelselbstmord}|pw}, da{\geminationn} im \textsc{Griensteidl\oindex{Wien@\textbf{Wien}!I., Innere Stadt@\textbf{I., Innere Stadt}!Café Griensteidl@\textbf{Café Griensteidl}, \emph{Kaffeehaus}|pw}}. Was morgen mit der \label{K_L00400-1v}\edtext{\textsc{Josefstadt}\oindex{Wien@\textbf{Wien}!VIII., Josefstadt@\textbf{VIII., Josefstadt}!Theater in der Josefstadt@\textbf{Theater in der Josefstadt}, \emph{Theater}|pw}}{\lemma{\textnormal{\emph{Josefstadt}}}\Cendnote{\textnormal{Am Nachmittag fand die Premiere des
                  Mimodramas \emph{Der Buckelhans}\pwindex{Blanchard de la Bretesche, Pierre *~19.\,8.\,1840 Paris@\textsc{Blanchard de la Bretesche, Pierre} (*~19.\,8.\,1840 Paris), \emph{Schriftsteller, Journalist}!Buckelhans@\strich\emph{Der Buckelhans}|pwk} von Pierre Blanchard de la Bretesche\pwindex{Blanchard de la Bretesche, Pierre *~19.\,8.\,1840 Paris@\textsc{Blanchard de la Bretesche, Pierre} (*~19.\,8.\,1840 Paris), \emph{Schriftsteller, Journalist}|pwk}
                  statt.}}}\label{K_L00400-1} los, weiſs ich noch nicht; Hugo\pwindex{Hofmannsthal, Hugo von 1.\,2.\,1874 Wien – 15.\,7.\,1929 Rodaun@\textsc{Hofmannsthal, Hugo von} (1.\,2.\,1874 Wien – 15.\,7.\,1929 Rodaun), \emph{Schriftsteller}|pw} hat mir geſagt, dſs er ko{\geminationm}t.\pend
           \pstart Herzlich Ihr \spacefill\mbox{Arthur}\pend{}\selectlanguage{ngerman}\endnumbering\briefempfaengerindex{Beer-Hofmann, Richard@\textsc{Beer-Hofmann, Richard}!zzzSchnitzler, Arthur@\emph{von Arthur Schnitzler}!1894-11-101@{10. 11. 1894}|)be}\mylabel{L00400h}  \newcommand{\dateiname}{L00400}\newcommand{\titel}{Arthur Schnitzler an Richard Beer-Hofmann, 10. 11. 1894}\newcommand{\editorInnen}{Herausgegeben von Martin Anton Müller}%% latex-leseansicht-abspann.tex
%% Abspann für die Leseansicht.
%% Der Schalter \ifkorrekturansicht ist bereits durch den Vorspann gesetzt.

%% latex-abspann.tex
%% Gemeinsamer Abspann für Korrekturansicht und Leseansicht.
%% Setzt den Schalter \ifkorrekturansicht voraus (gesetzt in den
%% einbindenden Dateien latex-korrekturansicht-abspann.tex bzw.
%% latex-leseansicht-abspann.tex).
%% ---------------------------------------------------------------

\normalsize

% Das esempio-Environment wird nur in der Leseansicht benötigt
\ifkorrekturansicht\else
\newenvironment{esempio}[3]%
{
    \vspace{1.5ex}
    \rlap{\underline{#1}}
    \par
    \setlength{\parindent}{0cm}
    \nopagebreak
    \leftskip=#2cm
    \rightskip=#3cm
}
{
    \par
}
\fi

\doendnotes{C}
\bigskip
\vfill

\clearpage

\footnotesize

\ifkorrekturansicht
  \lohead{\textsc{register}}
\fi

% theindex-Environment neu definieren ohne reledmac
\makeatletter
\renewenvironment{theindex}{%
  \ifkorrekturansicht
    \section*{\indexname}%
  \else
    \subsubsection*{Index der erwähnten Entitäten}%
  \fi
  \setlength{\parindent}{0pt}%
  \setlength{\parskip}{0pt plus 0.3pt}%
  \let\item\@idxitem
}{%
  \ifkorrekturansicht\clearpage\fi
}
\makeatother

\IfFileExists{\jobname-pw.ind}{\input{\jobname-pw.ind}}{}

% Quellenangabe nur in der Leseansicht
\ifkorrekturansicht\else
% Fallback-Definitionen, falls die .tex-Datei \titel etc. nicht gesetzt hat
\providecommand{\titel}{}
\providecommand{\editorInnen}{}
\providecommand{\dateiname}{\jobname}

\vspace{3cm}

\vfill

\footnotesize
\textsc{Quelle}: \titel. Herausgegeben von {\editorInnen}. In: \emph{Arthur Schnitzler: Briefwechsel mit Autorinnen und Autoren}.
 Digitale Edition, https://schnitzler-briefe.acdh.oeaw.ac.at/{\dateiname}.html (Stand \today)
\fi

\end{document}


