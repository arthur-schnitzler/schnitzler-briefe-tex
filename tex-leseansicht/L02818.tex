%% latex-leseansicht-vorspann.tex
%% Vorspann für die Leseansicht.
%% Lädt die gemeinsame Datei latex-vorspann.tex mit nicht gesetztem Schalter.

\newif\ifkorrekturansicht
\korrekturansichtfalse

\input{../tex-inputs/latex-vorspann}


         
         \renewcommand{\erwaehntePersonen}{Personen: Hermann Bahr, Richard Beer-Hofmann, Paul Goldmann, Max Graf, Leopold Sonnemann}
         \renewcommand{\erwaehnteInstitutionen}{Institutionen: Frankfurter Zeitung, Houghton Library}
         \renewcommand{\erwaehnteOrte}{Orte: Bad Ischl, Bayreuth, München, Paris, Schliersee, rue de la Bourse}
         \renewcommand{\erwaehnteWerke}{}
               \section[ Paul Goldmann an Arthur Schnitzler, 18. 7. {[}1897{]}]{ Paul Goldmann an Arthur Schnitzler, 18. 7. {[}1897{]}}\nopagebreak\mylabel{v}\rehead{ }\begin{ledgroupsized}[t]{13cm}\normalsize\beginnumbering\briefempfaengerindex{Schnitzler, Arthur@\textsc{Schnitzler, Arthur}!zzzGoldmann, Paul@\emph{von Paul Goldmann}!1897-07-182@{18. 7. {[}1897{]}}|(be} \toendnotes[C]{\smallbreak\pagebreak[2]} \Standort{DLA, A:Schnitzler, HS.NZ85.1.3167.}
\physDesc{Brief, 1 Blatt, 1 Seite, 546 Zeichen
\newline{}Handschrift: blaue Tinte, deutsche Kurrent
\newline{}Schnitzler: 1) mit Bleistift das Jahr »97« vermerkt  2) mit rotem Buntstift eine Unterstreichung}\toendnotes[C]{\smallbreak}\pstart
           \noindent{}{\pb}\textcolor{gray}{\textbf{\textbf{Frankfurter Zeitung\orgindex{Frankfurter Zeitung@Frankfurter Zeitung|pw}}}}\pend
           \pstart
           \textcolor{gray}{\textbf{(\begin{otherlanguage}{french}Gazette de Francfort\end{otherlanguage}\orgindex{Frankfurter Zeitung@Frankfurter Zeitung|pw}).}}\pend
           \pstart
           \textcolor{gray}{\textbf{\textbf{\begin{otherlanguage}{french}Fondateur M.\end{otherlanguage}{ }L. Sonnemann\pwindex{Sonnemann, Leopold 1831-10-29 – 1909-10-30@\textsc{Sonnemann, Leopold} (1831-10-29 – 1909-10-30), \emph{Journalist, Herausgeber}|pw}.}}}\pend
           \pstart
           \begin{otherlanguage}{french}\textcolor{gray}{\textbf{Journal politique, financier,}}\end{otherlanguage}\pend
           \pstart
           \begin{otherlanguage}{french}\textcolor{gray}{\textbf{commercial et littéraire.}}\end{otherlanguage}\pend
           \pstart
           \begin{otherlanguage}{french}\textcolor{gray}{\textbf{\textbf{Paraissant trois fois par jour.}}}\end{otherlanguage}\pend
           \pstart
           \begin{otherlanguage}{french}\textcolor{gray}{\textbf{\textbf{Bureau à Paris\oindex{Paris@\textbf{Paris}|pw}}}}\end{otherlanguage}\hfill \textsc{Paris\oindex{Paris@\textbf{Paris}|pw}}, 18. Juli.\pend
           \pstart
           \begin{otherlanguage}{french}\textcolor{gray}{\textbf{\textbf{10 \so{Rue de la Bourse}\oindex{rue de la Bourse@\textbf{rue de la Bourse}|pw}.}}}\end{otherlanguage}\pend
           \pstart\center{}Mein lieber Freund,\pend\pstart
           Setzen wir alſo die Sache ſo feſt: Am 11. Auguſt muß
               ich in \textsc{Bayreuth\oindex{Bayreuth@\textbf{Bayreuth}|pw}} ſein. Von da fahre ich nach \textsc{Muenchen\oindex{Muenchen@\textbf{München}|pw}} und komme ſo \label{K_L02818-1v}\edtext{zwiſchen 15. u. 20. Auguſt}{\lemma{\textnormal{\emph{zwiſchen … 20. Auguſt}}}\Cendnote{\textnormal{Goldmann\pwindex{Goldmann, Paul 31.01.1865 – 25.09.1935@\textsc{Goldmann, Paul} (31.01.1865 – 25.09.1935), \emph{Schriftsteller, Journalist}|pwk} kam am 19. 8. 1897 in Bad Ischl\oindex{Bad Ischl@\textbf{Bad Ischl}|pwk} an und blieb bis 30. 8. 1897.}}}\label{K_L02818-1h}
               nach \textsc{Ischl\oindex{Bad Ischl@\textbf{Bad Ischl}|pw}}. Dort bleibe ich mit Euch\pwindex{Beer-Hofmann, Richard 1866-07-11 – 1945-09-26@\textsc{Beer-Hofmann, Richard} (1866-07-11 – 1945-09-26), \emph{Schriftsteller}|pwv} zuſammen, ſolange es geht und ſahre dann über \textsc{Muenchen\oindex{Muenchen@\textbf{München}|pw}} nach \textsc{Paris\oindex{Paris@\textbf{Paris}|pw}} zurück. Bitte, laß’ mich umgehend wiſſen, ob Du mit dieſem Programm
               einverſtanden biſt?\pend
           \pstart
           Viele treue Grüße an Dich und \textsc{Richard\pwindex{Beer-Hofmann, Richard 1866-07-11 – 1945-09-26@\textsc{Beer-Hofmann, Richard} (1866-07-11 – 1945-09-26), \emph{Schriftsteller}|pw}}!\pend
           \pstart
           Dein {\\[\baselineskip]}\spacefill\mbox{Paul Goldmnn}\pend
           \leftskip=0em{}\pstart
           \noindent{}\textsc{Richard\pwindex{Beer-Hofmann, Richard 1866-07-11 – 1945-09-26@\textsc{Beer-Hofmann, Richard} (1866-07-11 – 1945-09-26), \emph{Schriftsteller}|pw}} ſoll auch am 11. Auguſt nach \label{K_L02818-2v}\edtext{\textsc{Bayreuth\oindex{Bayreuth@\textbf{Bayreuth}|pw}}}{\lemma{\textnormal{\emph{Bayreuth}}}\Cendnote{\textnormal{Siehe Paul Goldmann an Arthur Schnitzler, 13. 7. [1897].
                  }}}\label{K_L02818-2h} kommen u. dann mit mir über \textsc{Muenchen\oindex{Muenchen@\textbf{München}|pw}} nach \textsc{Ischl\oindex{Bad Ischl@\textbf{Bad Ischl}|pw}} zurückfahren.\pend
           \pstart
           \label{T_L02818-1v}\edtext{Muß ich fürchten, den \label{K_L02818-3v}\edtext{\textsc{Bahr\pwindex{Bahr, Hermann 19.07.1863 – 15.01.1934@\textsc{Bahr, Hermann} (19.07.1863 – 15.01.1934), \emph{Schriftsteller, Kritiker}|pw}} in \textsc{Ischl\oindex{Bad Ischl@\textbf{Bad Ischl}|pw}}}{\lemma{\textnormal{\emph{Bahr in Ischl}}}\Cendnote{\textnormal{Hermann Bahr\pwindex{Bahr, Hermann 19.07.1863 – 15.01.1934@\textsc{Bahr, Hermann} (19.07.1863 – 15.01.1934), \emph{Schriftsteller, Kritiker}|pwk} verbrachte seine
                     Sommerfrische 1897 am Schliersee\oindex{Schliersee@\textbf{Schliersee}|pwk}, war aber durch das Hochwasser (siehe Paul Goldmann an Arthur Schnitzler, 1. 8. [1897]) gezwungen, auf der
                     Reise dorthin Anfang August ein paar Tage in Ischl\oindex{Bad Ischl@\textbf{Bad Ischl}|pwk} Aufenthalt zu nehmen. Wie wichtig es Goldmann\pwindex{Goldmann, Paul 31.01.1865 – 25.09.1935@\textsc{Goldmann, Paul} (31.01.1865 – 25.09.1935), \emph{Schriftsteller, Journalist}|pwk} warm 
                     nicht mit Bahr\pwindex{Bahr, Hermann 19.07.1863 – 15.01.1934@\textsc{Bahr, Hermann} (19.07.1863 – 15.01.1934), \emph{Schriftsteller, Kritiker}|pwk} zusammenzutreffen,
                     geht aus der Nachschrift eines Briefs an Beer-Hofmann\pwindex{Beer-Hofmann, Richard 1866-07-11 – 1945-09-26@\textsc{Beer-Hofmann, Richard} (1866-07-11 – 1945-09-26), \emph{Schriftsteller}|pwk} vom 24. 7. {[}1897{]} hervor:
                        »Sorg’ mir nur dafür, daß ich in \textsc{Ischl}\oindex{Bad Ischl@\textbf{Bad Ischl}|pw} keinen \textsc{Bahr\pwindex{Bahr, Hermann 19.07.1863 – 15.01.1934@\textsc{Bahr, Hermann} (19.07.1863 – 15.01.1934), \emph{Schriftsteller, Kritiker}|pw}} und keinen \textsc{Graf\pwindex{Graf, Max 01.10.1873 – 24.06.1958@\textsc{Graf, Max} (01.10.1873 – 24.06.1958), \emph{Kritiker}|pw}} treffe. Ich will mir nicht meine Ferien durch Beſtialität verderben
                        laſſen.« (\emph{Houghton Library},
                        Harvard, Signatur 825.978.)}}}\label{K_L02818-3h} zu
                     treffen.}{\lemma{\textnormal{\emph{Muß … treffen.}}}\Cendnote{\textnormal{seitlich entlang des
                     Mittelfalzes}}}\label{T_L02818-1h}\pend
           
         
         \endnumbering\mylabel{h}\end{ledgroupsized}  \newcommand{\dateiname}{L02818}\newcommand{\titel}{Paul Goldmann an Arthur Schnitzler, 18. 7. [1897]}\newcommand{\editorInnen}{Martin Anton Müller und Laura Untner}%% latex-leseansicht-abspann.tex
%% Abspann für die Leseansicht.
%% Der Schalter \ifkorrekturansicht ist bereits durch den Vorspann gesetzt.

%% latex-abspann.tex
%% Gemeinsamer Abspann für Korrekturansicht und Leseansicht.
%% Setzt den Schalter \ifkorrekturansicht voraus (gesetzt in den
%% einbindenden Dateien latex-korrekturansicht-abspann.tex bzw.
%% latex-leseansicht-abspann.tex).
%% ---------------------------------------------------------------

\normalsize

% Das esempio-Environment wird nur in der Leseansicht benötigt
\ifkorrekturansicht\else
\newenvironment{esempio}[3]%
{
    \vspace{1.5ex}
    \rlap{\underline{#1}}
    \par
    \setlength{\parindent}{0cm}
    \nopagebreak
    \leftskip=#2cm
    \rightskip=#3cm
}
{
    \par
}
\fi

\doendnotes{C}
\bigskip
\vfill

\clearpage

\footnotesize

\ifkorrekturansicht
  \lohead{\textsc{register}}
\fi

% theindex-Environment neu definieren ohne reledmac
\makeatletter
\renewenvironment{theindex}{%
  \ifkorrekturansicht
    \section*{\indexname}%
  \else
    \subsubsection*{Index der erwähnten Entitäten}%
  \fi
  \setlength{\parindent}{0pt}%
  \setlength{\parskip}{0pt plus 0.3pt}%
  \let\item\@idxitem
}{%
  \ifkorrekturansicht\clearpage\fi
}
\makeatother

\IfFileExists{\jobname-pw.ind}{\input{\jobname-pw.ind}}{}

% Quellenangabe nur in der Leseansicht
\ifkorrekturansicht\else
% Fallback-Definitionen, falls die .tex-Datei \titel etc. nicht gesetzt hat
\providecommand{\titel}{}
\providecommand{\editorInnen}{}
\providecommand{\dateiname}{\jobname}

\vspace{3cm}

\vfill

\footnotesize
\textsc{Quelle}: \titel. Herausgegeben von {\editorInnen}. In: \emph{Arthur Schnitzler: Briefwechsel mit Autorinnen und Autoren}.
 Digitale Edition, https://schnitzler-briefe.acdh.oeaw.ac.at/{\dateiname}.html (Stand \today)
\fi

\end{document}


      