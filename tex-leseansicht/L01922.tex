%% latex-korrekturansicht-vorspann.tex
%% Vorspann für die Korrekturansicht.
%% Lädt die gemeinsame Datei latex-vorspann.tex mit gesetztem Schalter.

\newif\ifkorrekturansicht
\korrekturansichttrue

\input{../tex-inputs/latex-vorspann}


\section[Max Burckhard an Arthur Schnitzler, 8. 4. 1910]{L01922 Max Burckhard an Arthur Schnitzler, 8. 4. 1910}
\nopagebreak\mylabel{L01922v}
\rehead{ }\normalsize\beginnumbering\briefempfaengerindex{Schnitzler, Arthur@\textsc{Schnitzler, Arthur}!zzzBurckhard, Max Eugen@\emph{von Max Eugen Burckhard}!1910-04-081@{8. 4. 1910}|(be}
\toendnotes[C]{\smallbreak\pagebreak[2]}\Standort{TMW, HS Schn 1/73/1.}
\physDesc{Brief, 1 Blatt, 1 Seite, 754 Zeichen
\newline{}Handschrift: schwarze Tinte, deutsche Kurrent}
\pstart
           {\pb}\textcolor{gray}{\textbf{Dr. Max Burckhard}}\hfill \textcolor{gray}{\textbf{Wien, IX. Porzellangasse 48\oindex{Porzellangasse@\textbf{Porzellangasse}, \emph{Straße (K.STR)}|pw}}}{ }..........\pend
           
\pstart
           \raggedleft{}\textcolor{gray}{\textbf{St. Gilgen\oindex{St. Gilgen@\textbf{St. Gilgen}, \emph{A.ADM3}|pw}}}{ }8. 4. 10\pend
           
\pstart{}Lieber verehrter Herr Doctor!\pend\vspace{0.5em}
\pstart
           Ich habe Zweifel, ob ein Brief, den ich geſtern an Sie ſchrieb, aufgegeben wurde, und
               ſage daher vorſichtsweiſe heute nochmal Dank für Ihren lieben Brief, den ich bei der
               Rückkehr aus Portofino\oindex{Portofino@\textbf{Portofino}, \emph{P.PPLA3}|pw} vorfand. Mich hat es
               außerordentlich gefreut, daſs Trinacria\pwindex{Trinacria@\emph{Trinacria}|pw} Sie
               intereſſiert hat, da ich bei perſönlichen Reminiscenzen i{\geminationm}er ganz beſonders unſicher bin über die Wirkung auf andere. Ich habe Sicilien\oindex{Sizilien@\textbf{Sizilien}, \emph{A.ADM1}|pw}{ }ſo gerne gewonnen, daſs ich fünfmal unten war und
               bei ſolchen Gelegenheiten nicht nur ſehr viel herumgeradelt u -gekraxelt bin, ſondern
               auch bis in die Tiefe archäologiſcher Localſtudien geſunken bin.\pend
           
\pstart
           Auf ſehr baldiges Wiederſehen in Wien\oindex{Wien@\textbf{Wien}, \emph{A.ADM2}|pw}, und
               hoffentlich wieder in St. Gilgen\oindex{St. Gilgen@\textbf{St. Gilgen}, \emph{A.ADM3}|pw}. Mit Handkuß u
               herzl Grüßen{\\[\baselineskip]}Ihr{\\[\baselineskip]}\spacefill\mbox{D\textsuperscript{r}Burckhard}\pend
           \leftskip=0em{}\selectlanguage{ngerman}\endnumbering\briefempfaengerindex{Schnitzler, Arthur@\textsc{Schnitzler, Arthur}!zzzBurckhard, Max Eugen@\emph{von Max Eugen Burckhard}!1910-04-081@{8. 4. 1910}|)be}\mylabel{L01922h}  \normalsize

\doendnotes{C}
\bigskip
\vfill

\clearpage

\footnotesize

\lohead{\textsc{register}}

% Definiere theindex-Environment komplett neu ohne reledmac
\makeatletter
\renewenvironment{theindex}{%
  \section*{\indexname}%
  \setlength{\parindent}{0pt}%
  \setlength{\parskip}{0pt plus 0.3pt}%
  \let\item\@idxitem
}{%
  \clearpage
}
\makeatother

\IfFileExists{\jobname-pw.ind}{\input{\jobname-pw.ind}}{}

\end{document}

      