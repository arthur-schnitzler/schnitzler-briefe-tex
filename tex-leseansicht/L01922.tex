%% latex-leseansicht-vorspann.tex
%% Vorspann für die Leseansicht.
%% Lädt die gemeinsame Datei latex-vorspann.tex mit nicht gesetztem Schalter.

\newif\ifkorrekturansicht
\korrekturansichtfalse

\input{../tex-inputs/latex-vorspann}


\section[Max Burckhard an Arthur Schnitzler, 8. 4. 1910]{L01922 Max Burckhard an Arthur Schnitzler, 8. 4. 1910}
\nopagebreak\mylabel{L01922v}
\rehead{ }\normalsize\beginnumbering\briefempfaengerindex{Schnitzler, Arthur@\textsc{Schnitzler, Arthur}!zzzBurckhard, Max Eugen@\emph{von Max Eugen Burckhard}!1910-04-081@{8. 4. 1910}|(be}
\toendnotes[C]{\smallbreak\pagebreak[2]}
\correspDesc{Versand  durch Max Burckhard am 8. 4. 1910 in St. Gilgen
\newline{}Erhalt  durch Arthur Schnitzler im Zeitraum [9. 4. 1910
                  – 13. 4. 1910?] in Wien}\toendnotes[C]{\smallbreak}
\Standort{TMW, HS Schn 1/73/1.}
\physDesc{Brief, 1 Blatt, 1 Seite, 754 Zeichen
\newline{}Handschrift: schwarze Tinte, deutsche Kurrent}
\pstart
           {\pb}\textcolor{gray}{\textbf{Dr. Max Burckhard}}\hfill \textcolor{gray}{\textbf{Wien, IX. Porzellangasse 48\oindex{Wien@\textbf{Wien}!IX., Alsergrund@\textbf{IX., Alsergrund}!Porzellangasse@\textbf{Porzellangasse}, \emph{Straße}|pw}}}{ }..........\pend
           
\pstart
           \raggedleft{}\textcolor{gray}{\textbf{St. Gilgen\oindex{St. Gilgen@\textbf{St. Gilgen}, \emph{Verwaltungsgebiet}|pw}}}{ }8. 4. 10\pend
           
\pstart{}Lieber verehrter Herr Doctor!\pend\vspace{0.5em}
\pstart
           Ich habe Zweifel, ob ein Brief, den ich geſtern an Sie{ }ſchrieb, aufgegeben wurde, und{ }ſage daher vorſichtsweiſe heute nochmal Dank für Ihren lieben Brief, den ich bei der
               Rückkehr aus Portofino\oindex{Portofino@\textbf{Portofino}, \emph{Hauptstadt}|pw} vorfand. Mich hat es
               außerordentlich gefreut, daſs Trinacria\pwindex{Burckhard, Max Eugen 14.\,7.\,1854 Korneuburg – 16.\,3.\,1912 Wien@\textsc{Burckhard, Max Eugen} (14.\,7.\,1854 Korneuburg – 16.\,3.\,1912 Wien), \emph{Schriftsteller, Rechtswissenschaftler, Theaterleiter}!Trinacria@\strich\emph{Trinacria}|pw} Sie
               intereſſiert hat, da ich bei perſönlichen Reminiscenzen i{\geminationm}er ganz beſonders unſicher bin über die Wirkung auf andere. Ich habe Sicilien\oindex{Sizilien@\textbf{Sizilien}, \emph{Land}|pw}{ }ſo gerne gewonnen, daſs ich fünfmal unten war und
               bei{ }ſolchen Gelegenheiten nicht nur{ }ſehr viel herumgeradelt u -gekraxelt bin,{ }ſondern
               auch bis in die Tiefe archäologiſcher Localſtudien geſunken bin.\pend
           
\pstart
           Auf{ }ſehr baldiges Wiederſehen in Wien\oindex{Wien@\textbf{Wien}, \emph{Verwaltungsgebiet}|pw}, und
               hoffentlich wieder in St. Gilgen\oindex{St. Gilgen@\textbf{St. Gilgen}, \emph{Verwaltungsgebiet}|pw}. Mit Handkuß u
               herzl Grüßen{\\[\baselineskip]}Ihr{\\[\baselineskip]}\spacefill\mbox{D\textsuperscript{r}Burckhard}\pend
           \leftskip=0em{}\selectlanguage{ngerman}\endnumbering\briefempfaengerindex{Schnitzler, Arthur@\textsc{Schnitzler, Arthur}!zzzBurckhard, Max Eugen@\emph{von Max Eugen Burckhard}!1910-04-081@{8. 4. 1910}|)be}\mylabel{L01922h}  \newcommand{\dateiname}{L01922}\newcommand{\titel}{Max Burckhard an Arthur Schnitzler, 8. 4. 1910}\newcommand{\editorInnen}{Martin Anton Müller und Gerd-Hermann Susen}%% latex-leseansicht-abspann.tex
%% Abspann für die Leseansicht.
%% Der Schalter \ifkorrekturansicht ist bereits durch den Vorspann gesetzt.

%% latex-abspann.tex
%% Gemeinsamer Abspann für Korrekturansicht und Leseansicht.
%% Setzt den Schalter \ifkorrekturansicht voraus (gesetzt in den
%% einbindenden Dateien latex-korrekturansicht-abspann.tex bzw.
%% latex-leseansicht-abspann.tex).
%% ---------------------------------------------------------------

\normalsize

% Das esempio-Environment wird nur in der Leseansicht benötigt
\ifkorrekturansicht\else
\newenvironment{esempio}[3]%
{
    \vspace{1.5ex}
    \rlap{\underline{#1}}
    \par
    \setlength{\parindent}{0cm}
    \nopagebreak
    \leftskip=#2cm
    \rightskip=#3cm
}
{
    \par
}
\fi

\doendnotes{C}
\bigskip
\vfill

\clearpage

\footnotesize

\ifkorrekturansicht
  \lohead{\textsc{register}}
\fi

% theindex-Environment neu definieren ohne reledmac
\makeatletter
\renewenvironment{theindex}{%
  \ifkorrekturansicht
    \section*{\indexname}%
  \else
    \subsubsection*{Index der erwähnten Entitäten}%
  \fi
  \setlength{\parindent}{0pt}%
  \setlength{\parskip}{0pt plus 0.3pt}%
  \let\item\@idxitem
}{%
  \ifkorrekturansicht\clearpage\fi
}
\makeatother

\IfFileExists{\jobname-pw.ind}{\input{\jobname-pw.ind}}{}

% Quellenangabe nur in der Leseansicht
\ifkorrekturansicht\else
% Fallback-Definitionen, falls die .tex-Datei \titel etc. nicht gesetzt hat
\providecommand{\titel}{}
\providecommand{\editorInnen}{}
\providecommand{\dateiname}{\jobname}

\vspace{3cm}

\vfill

\footnotesize
\textsc{Quelle}: \titel. Herausgegeben von {\editorInnen}. In: \emph{Arthur Schnitzler: Briefwechsel mit Autorinnen und Autoren}.
 Digitale Edition, https://schnitzler-briefe.acdh.oeaw.ac.at/{\dateiname}.html (Stand \today)
\fi

\end{document}


