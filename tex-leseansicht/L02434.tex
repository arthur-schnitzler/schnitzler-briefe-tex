%% latex-leseansicht-vorspann.tex
%% Vorspann für die Leseansicht.
%% Lädt die gemeinsame Datei latex-vorspann.tex mit nicht gesetztem Schalter.

\newif\ifkorrekturansicht
\korrekturansichtfalse

\input{../tex-inputs/latex-vorspann}


         
         \renewcommand{\erwaehntePersonen}{Personen: Georg Brandes, Gertrud Rung}
         \renewcommand{\erwaehnteOrte}{Orte: Berlin, Kopenhagen, Sternwartestraße 71, Wien, XVIII., Währing}
         \renewcommand{\erwaehnteWerke}{}
               \section[Arthur Schnitzler an Georg Brandes, 22. 2. 1925]{ Arthur Schnitzler an Georg Brandes, 22. 2. 1925}\nopagebreak\mylabel{v}\rehead{ }\begin{ledgroupsized}[t]{13cm}\normalsize\beginnumbering\briefempfaengerindex{Brandes, Georg@\textsc{Brandes, Georg}!zzzSchnitzler, Arthur@\emph{von Arthur Schnitzler}!1925-02-221@{22. 2. 1925}|(be} \toendnotes[C]{\smallbreak\pagebreak[2]} \Standort{Kopenhagen, Det Kongelige Bibliotek, Georg Brandes Arkiv, box 125.}
\physDesc{Postkarte, 479 Zeichen
\newline{}Handschrift: schwarze Tinte, lateinische Kurrent
\newline{}Versand: 1) Stempel: »\nobreak{}\oindex{XVIII., Waehring@\textbf{XVIII., Währing}|pwk}18 Wien 110, 23. II. 2\textcolor{gray}{5}, 9\nobreak{}«.   2) Stempel: »\nobreak{}\oindex{Kopenhagen@\textbf{Kopenhagen}|pwk}Kjøbenhavn, 25. 2. 25, 10–11\nobreak{}«. 
\newline{}Ordnung: mit Bleistift von unbekannter Hand beschriftet
                                    »Schnit« und nummeriert:
                                 »51.« }\buchAbdrucke{\weitereDrucke{Georg Brandes, Arthur Schnitzler: \emph{Ein Briefwechsel}. Hg. Kurt Bergel. Bern: \emph{Francke} 1956, S. 144–145.} }\toendnotes[C]{\smallbreak}\pstart{}{\pb}\label{T_L02434-1v}\edtext{\textcolor{gray}{\textbf{A. S.}}}{\lemma{\textnormal{\emph{A. S.}}}\Cendnote{\textnormal{ovaler Absenderkleber}}}\label{T_L02434-1h}\pend{}\pstart{}\textcolor{gray}{\textbf{WIEN, XVIII.}}\oindex{XVIII., Waehring@\textbf{XVIII., Währing}|pw}\pend{}\pstart{}\textcolor{gray}{\textbf{STERNWARTESTR. 71}}\oindex{Sternwartestrasse 71@\textbf{Sternwartestraße 71}|pw}\pend{}{\bigskip}\pstart{}Herrn Prof. Georg Brandes\pend{}\pstart{}Kopenhagen\oindex{Kopenhagen@\textbf{Kopenhagen}|pw}. \pend{}{\bigskip}\pstart
           \raggedleft{}{\pb}Wien\oindex{Wien@\textbf{Wien}|pw}, 22. 2. 25\pend
           \pstart
           Verehrter lieber Freund, hoffentlich ka{\geminationn} ichs so einrichten, dſs ich zur Zeit Ihrer Ankunft in Berlin\oindex{Berlin@\textbf{Berlin}|pw} noch dort bin – meine dortige Anwesenheit war, aus
               verschiedenen Gründen für früher projectirt. Vielleicht ist Frau Rung\pwindex{Rung, Gertrud 26.03.1882 – 25.04.1959@\textsc{Rung, Gertrud} (26.03.1882 – 25.04.1959), \emph{Übersetzerin, Sekretärin}|pw}, der ich mich bestens empfehle auch noch so gütig, mir
               mitzutheilen, \uline{an welchem Tage} Sie schon in Berlin\oindex{Berlin@\textbf{Berlin}|pw}{ }\uline{eintreffen}. Sie sind vielleicht schon vor dem 25\substVorne{}\textsuperscript{.}\substDazwischen{}/3\substHinten{} dort?\pend
           \pstart
           In treuer und herzlicher Verehrung{\\[\baselineskip]}Ihr \spacefill\mbox{Arthur Schnitzler}\pend
           \leftskip=0em{}
         
         \endnumbering\mylabel{h}\end{ledgroupsized}  \newcommand{\dateiname}{L02434}\newcommand{\titel}{Arthur Schnitzler an Georg Brandes, 22. 2. 1925}\newcommand{\editorInnen}{Martin Anton Müller und Gerd-Hermann Susen}%% latex-leseansicht-abspann.tex
%% Abspann für die Leseansicht.
%% Der Schalter \ifkorrekturansicht ist bereits durch den Vorspann gesetzt.

%% latex-abspann.tex
%% Gemeinsamer Abspann für Korrekturansicht und Leseansicht.
%% Setzt den Schalter \ifkorrekturansicht voraus (gesetzt in den
%% einbindenden Dateien latex-korrekturansicht-abspann.tex bzw.
%% latex-leseansicht-abspann.tex).
%% ---------------------------------------------------------------

\normalsize

% Das esempio-Environment wird nur in der Leseansicht benötigt
\ifkorrekturansicht\else
\newenvironment{esempio}[3]%
{
    \vspace{1.5ex}
    \rlap{\underline{#1}}
    \par
    \setlength{\parindent}{0cm}
    \nopagebreak
    \leftskip=#2cm
    \rightskip=#3cm
}
{
    \par
}
\fi

\doendnotes{C}
\bigskip
\vfill

\clearpage

\footnotesize

\ifkorrekturansicht
  \lohead{\textsc{register}}
\fi

% theindex-Environment neu definieren ohne reledmac
\makeatletter
\renewenvironment{theindex}{%
  \ifkorrekturansicht
    \section*{\indexname}%
  \else
    \subsubsection*{Index der erwähnten Entitäten}%
  \fi
  \setlength{\parindent}{0pt}%
  \setlength{\parskip}{0pt plus 0.3pt}%
  \let\item\@idxitem
}{%
  \ifkorrekturansicht\clearpage\fi
}
\makeatother

\IfFileExists{\jobname-pw.ind}{\input{\jobname-pw.ind}}{}

% Quellenangabe nur in der Leseansicht
\ifkorrekturansicht\else
% Fallback-Definitionen, falls die .tex-Datei \titel etc. nicht gesetzt hat
\providecommand{\titel}{}
\providecommand{\editorInnen}{}
\providecommand{\dateiname}{\jobname}

\vspace{3cm}

\vfill

\footnotesize
\textsc{Quelle}: \titel. Herausgegeben von {\editorInnen}. In: \emph{Arthur Schnitzler: Briefwechsel mit Autorinnen und Autoren}.
 Digitale Edition, https://schnitzler-briefe.acdh.oeaw.ac.at/{\dateiname}.html (Stand \today)
\fi

\end{document}


      