%% latex-korrekturansicht-vorspann.tex
%% Vorspann für die Korrekturansicht.
%% Lädt die gemeinsame Datei latex-vorspann.tex mit gesetztem Schalter.

\newif\ifkorrekturansicht
\korrekturansichttrue

\input{../tex-inputs/latex-vorspann}


\section[Arthur Schnitzler an Georg Brandes, 22. 2. 1925]{L02434 Arthur Schnitzler an Georg Brandes, 22. 2. 1925}
\nopagebreak\mylabel{L02434v}
\rehead{ }\normalsize\beginnumbering\briefempfaengerindex{Brandes, Georg@\textsc{Brandes, Georg}!zzzSchnitzler, Arthur@\emph{von Arthur Schnitzler}!1925-02-221@{22. 2. 1925}|(be}
\toendnotes[C]{\smallbreak\pagebreak[2]}\Standort{Kopenhagen, Det Kongelige Bibliotek, Georg Brandes Arkiv, box 125.}
\physDesc{Postkarte, 479 Zeichen
\newline{}Handschrift: schwarze Tinte, lateinische Kurrent
\newline{}Versand: 1) Stempel: »\nobreak{}\oindex{XVIII., Waehring@\textbf{XVIII., Währing}, \emph{A.ADM3}|pwk}18 Wien 110, 23. II. 2\textcolor{gray}{5}, 9\nobreak{}«.   2) Stempel: »\nobreak{}\oindex{Kopenhagen@\textbf{Kopenhagen}, \emph{P.PPLC}|pwk}Kjøbenhavn, 25. 2. 25, 10–11\nobreak{}«. 
\newline{}Ordnung: mit Bleistift von unbekannter Hand beschriftet
                                    »Schnit« und nummeriert:
                                 »51.« }
\buchAbdrucke{\weitereDrucke{Georg Brandes, Arthur Schnitzler: \emph{Ein Briefwechsel}. Bern: \emph{Francke} 1956, S. 144–145.} }\toendnotes[C]{\smallbreak}\pstart{}{\pb}\label{T_L02434-1v}\edtext{\textcolor{gray}{\textbf{A. S.}}}{\lemma{\textnormal{\emph{A. S.}}}\Cendnote{\textnormal{ovaler Absenderkleber}}}\label{T_L02434-1}\pend{}\pstart{}\textcolor{gray}{\textbf{WIEN, XVIII.}}\oindex{XVIII., Waehring@\textbf{XVIII., Währing}, \emph{A.ADM3}|pw}\pend{}\pstart{}\textcolor{gray}{\textbf{STERNWARTESTR. 71}}\oindex{Sternwartestrasse 71@\textbf{Sternwartestraße 71}, \emph{Wohngebäude (K.WHS)}|pw}\pend{}{\bigskip}\pstart{}Herrn Prof. Georg Brandes\pend{}\pstart{}Kopenhagen\oindex{Kopenhagen@\textbf{Kopenhagen}, \emph{P.PPLC}|pw}. \pend{}{\bigskip}\vspace{1em}
\pstart
           \raggedleft{}{\pb}Wien\oindex{Wien@\textbf{Wien}, \emph{A.ADM2}|pw}, 22. 2. 25\pend
           \vspace{0.5em}
\pstart
           Verehrter lieber Freund, hoffentlich ka{\geminationn} ichs so einrichten, dſs ich zur Zeit Ihrer Ankunft in Berlin\oindex{Berlin@\textbf{Berlin}, \emph{P.PPLC}|pw} noch dort bin – meine dortige Anwesenheit war, aus
               verschiedenen Gründen für früher projectirt. Vielleicht ist Frau Rung\pwindex{Rung, Gertrud 26.03.1882 – 25.04.1959@\textsc{Rung, Gertrud} (26.03.1882 – 25.04.1959), \emph{Übersetzer/Übersetzerin, Sekretär/Sekretärin}|pw}, der ich mich bestens empfehle auch noch so gütig, mir
               mitzutheilen, \uline{an welchem Tage} Sie schon in Berlin\oindex{Berlin@\textbf{Berlin}, \emph{P.PPLC}|pw}{ }\uline{eintreffen}. Sie sind vielleicht schon vor dem 25\substVorne{}\textsuperscript{.}\substDazwischen{}/3\substHinten{} dort?\pend
           
\pstart
           In treuer und herzlicher Verehrung{\\[\baselineskip]}Ihr \spacefill\mbox{Arthur Schnitzler}\pend
           \leftskip=0em{}\selectlanguage{ngerman}\endnumbering\briefempfaengerindex{Brandes, Georg@\textsc{Brandes, Georg}!zzzSchnitzler, Arthur@\emph{von Arthur Schnitzler}!1925-02-221@{22. 2. 1925}|)be}\mylabel{L02434h}  \normalsize

\doendnotes{C}
\bigskip
\vfill

\clearpage

\footnotesize

\lohead{\textsc{register}}

% Definiere theindex-Environment komplett neu ohne reledmac
\makeatletter
\renewenvironment{theindex}{%
  \section*{\indexname}%
  \setlength{\parindent}{0pt}%
  \setlength{\parskip}{0pt plus 0.3pt}%
  \let\item\@idxitem
}{%
  \clearpage
}
\makeatother

\IfFileExists{\jobname-pw.ind}{\input{\jobname-pw.ind}}{}

\end{document}

      