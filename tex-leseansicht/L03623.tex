%% latex-leseansicht-vorspann.tex
%% Vorspann für die Leseansicht.
%% Lädt die gemeinsame Datei latex-vorspann.tex mit nicht gesetztem Schalter.

\newif\ifkorrekturansicht
\korrekturansichtfalse

\input{../tex-inputs/latex-vorspann}


\section[Stefan Zweig an Arthur Schnitzler, {[}22. 10. 1908?{]}]{L03623 Stefan Zweig an Arthur Schnitzler, {[}22. 10. 1908?{]}}
\nopagebreak\mylabel{L03623v}
\rehead{ }\normalsize\beginnumbering\briefempfaengerindex{Schnitzler, Arthur@\textsc{Schnitzler, Arthur}!zzzZweig, Stefan@\emph{von Stefan Zweig}!1908-10-222@{{[}22. 10. 1908?{]}}|(be}
\toendnotes[C]{\smallbreak\pagebreak[2]}
\correspDesc{Versand  durch Stefan Zweig am [22. 10. 1908?] in Wien
\newline{}Erhalt  durch Arthur Schnitzler im Zeitraum [22. 10. 1908 – 25. 10. 1908?] in Wien}\toendnotes[C]{\smallbreak}
\Standort{CUL, Schnitzler, B 118.}
\physDesc{Brief, 1 Blatt, 1 Seite, 517 Zeichen
\newline{}Handschrift: lila Tinte, lateinische Kurrent
\newline{}Schnitzler: mit Bleistift »\textsc{Zweig}« und datiert: »\textsc{908 Sept?}« }
\buchAbdrucke{\weitereDrucke{Stefan Zweig: \emph{Briefwechsel mit Hermann Bahr, Sigmund Freud, Rainer Maria
                        Rilke und Arthur Schnitzler}. Herausgegeben von Jeffrey B. Berlin, Hans-Ulrich Lindken und Donald A. Prater. Frankfurt am Main: \emph{S. Fischer} 1987, S. 356.} }\toendnotes[C]{\smallbreak}
\pstart
           {\pb}\textcolor{gray}{\textbf{SZ}}\hfill \textcolor{gray}{\textbf{VIII. KOCHGASSE 8\oindex{Wien@\textbf{Wien}!VIII., Josefstadt@\textbf{VIII., Josefstadt}!Kochgasse 8@\textbf{Kochgasse 8}, \emph{Wohngebäude}|pw}}}\pend
           
\pstart
           \raggedleft{}\textcolor{gray}{\textbf{WIEN\oindex{Wien@\textbf{Wien}, \emph{Verwaltungsgebiet}|pw},}}{ }Donnerstag\pend
           \vspace{0.5em}{\vspace{1\baselineskip}}
\pstart
           Sehr verehrter Herr Doktor, als ich Ihren \label{K_L03623-1v}\edtext{Brief}{\lemma{\textnormal{\emph{Brief}}}\Cendnote{\textnormal{XXXX Auszeichnungsfehler: Dokument L03801 nicht gefunden.}}}\label{K_L03623-1} in die Hand nahm dachte ich
               mir – ehe ich ihn öffnete – noch in meine erste Freude hinein: Wenn Sie nur nicht
                  Samstag sagen! Nun sagten Sie es wirklich und Samstag ist der einzige
               für mich unaufschiebbar besetzte Abend (eine \label{K_L03623-2v}\edtext{Vorlesung im »Volksheim\oindex{Wien@\textbf{Wien}!XVI., Ottakring@\textbf{XVI., Ottakring}!Volkshochschule Ottakring@\textbf{Volkshochschule Ottakring}, \emph{Gebäude}|pw}\orgindex{Verein Volksheim@Verein Volksheim|pw}«}{\lemma{\textnormal{\emph{Vorlesung im »Volksheim«}}}\Cendnote{\textnormal{Am
                     24. 10. 1908 sprach Stefan Zweig\pwindex{Zweig, Stefan 28.\,11.\,1881 Wien – 23.\,2.\,1942 Petrópolis@\textsc{Zweig, Stefan} (28.\,11.\,1881 Wien – 23.\,2.\,1942 Petrópolis), \emph{Schriftsteller}|pwk} im Volksheim\oindex{Wien@\textbf{Wien}!XVI., Ottakring@\textbf{XVI., Ottakring}!Volkshochschule Ottakring@\textbf{Volkshochschule Ottakring}, \emph{Gebäude}|pwk} am Koflerpark\oindex{Wien@\textbf{Wien}!XVI., Ottakring@\textbf{XVI., Ottakring}!Ludo-Hartmann-Platz@\textbf{Ludo-Hartmann-Platz}, \emph{Platz}|pwk} (heute Ludo-Hartmann-Platz\oindex{Wien@\textbf{Wien}!XVI., Ottakring@\textbf{XVI., Ottakring}!Ludo-Hartmann-Platz@\textbf{Ludo-Hartmann-Platz}, \emph{Platz}|pwk}) über Honoré de Balzac\pwindex{Balzac, Honoré de 20.\,5.\,1799 Tours – 18.\,8.\,1850 Paris@\textsc{Balzac, Honoré de} (20.\,5.\,1799 Tours – 18.\,8.\,1850 Paris), \emph{Schriftsteller}|pwk}. }}}\label{K_L03623-2}, die ich Herrn Professor Reich\pwindex{Reich, Emil 29.\,10.\,1864 Korycany – 13.\,12.\,1940 Wien@\textsc{Reich, Emil} (29.\,10.\,1864 Korycany – 13.\,12.\,1940 Wien)|pw} zusagte.)\pend
           
\pstart
           Wollen Sie, verehrter Herr Doktor, es noch einmal versuchen? Jeder Tag, jede Stunde
               ist mir recht, die Sie mir für \label{K_L03623-3v}\edtext{nächste Woche}{\lemma{\textnormal{\emph{nächste Woche}}}\Cendnote{\textnormal{Das geplante Treffen
                  verschob sich nochmals um eine Woche, auf den 2. 11. 1908.}}}\label{K_L03623-3} bestimmen wollten.\pend
           
\pstart
           In Verehrung getreu{\\[\baselineskip]}\spacefill\mbox{StefanZweig}\pend
           \leftskip=0em{}\selectlanguage{ngerman}\endnumbering\briefempfaengerindex{Schnitzler, Arthur@\textsc{Schnitzler, Arthur}!zzzZweig, Stefan@\emph{von Stefan Zweig}!1908-10-222@{{[}22. 10. 1908?{]}}|)be}\mylabel{L03623h}  \newcommand{\dateiname}{L03623}\newcommand{\titel}{Stefan Zweig an Arthur Schnitzler, [22. 10. 1908?]}\newcommand{\editorInnen}{Selma Jahnke und Martin Anton Müller}%% latex-leseansicht-abspann.tex
%% Abspann für die Leseansicht.
%% Der Schalter \ifkorrekturansicht ist bereits durch den Vorspann gesetzt.

%% latex-abspann.tex
%% Gemeinsamer Abspann für Korrekturansicht und Leseansicht.
%% Setzt den Schalter \ifkorrekturansicht voraus (gesetzt in den
%% einbindenden Dateien latex-korrekturansicht-abspann.tex bzw.
%% latex-leseansicht-abspann.tex).
%% ---------------------------------------------------------------

\normalsize

% Das esempio-Environment wird nur in der Leseansicht benötigt
\ifkorrekturansicht\else
\newenvironment{esempio}[3]%
{
    \vspace{1.5ex}
    \rlap{\underline{#1}}
    \par
    \setlength{\parindent}{0cm}
    \nopagebreak
    \leftskip=#2cm
    \rightskip=#3cm
}
{
    \par
}
\fi

\doendnotes{C}
\bigskip
\vfill

\clearpage

\footnotesize

\ifkorrekturansicht
  \lohead{\textsc{register}}
\fi

% theindex-Environment neu definieren ohne reledmac
\makeatletter
\renewenvironment{theindex}{%
  \ifkorrekturansicht
    \section*{\indexname}%
  \else
    \subsubsection*{Index der erwähnten Entitäten}%
  \fi
  \setlength{\parindent}{0pt}%
  \setlength{\parskip}{0pt plus 0.3pt}%
  \let\item\@idxitem
}{%
  \ifkorrekturansicht\clearpage\fi
}
\makeatother

\IfFileExists{\jobname-pw.ind}{\input{\jobname-pw.ind}}{}

% Quellenangabe nur in der Leseansicht
\ifkorrekturansicht\else
% Fallback-Definitionen, falls die .tex-Datei \titel etc. nicht gesetzt hat
\providecommand{\titel}{}
\providecommand{\editorInnen}{}
\providecommand{\dateiname}{\jobname}

\vspace{3cm}

\vfill

\footnotesize
\textsc{Quelle}: \titel. Herausgegeben von {\editorInnen}. In: \emph{Arthur Schnitzler: Briefwechsel mit Autorinnen und Autoren}.
 Digitale Edition, https://schnitzler-briefe.acdh.oeaw.ac.at/{\dateiname}.html (Stand \today)
\fi

\end{document}


