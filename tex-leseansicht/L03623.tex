%% latex-korrekturansicht-vorspann.tex
%% Vorspann für die Korrekturansicht.
%% Lädt die gemeinsame Datei latex-vorspann.tex mit gesetztem Schalter.

\newif\ifkorrekturansicht
\korrekturansichttrue

\input{../tex-inputs/latex-vorspann}


\section[Stefan Zweig an Arthur Schnitzler, {[}22. 10. 1908?{]}]{L03623 Stefan Zweig an Arthur Schnitzler, {[}22. 10. 1908?{]}}
\nopagebreak\mylabel{L03623v}
\rehead{ }\normalsize\beginnumbering\briefempfaengerindex{Schnitzler, Arthur@\textsc{Schnitzler, Arthur}!zzzZweig, Stefan@\emph{von Stefan Zweig}!1908-10-222@{{[}22. 10. 1908?{]}}|(be}
\toendnotes[C]{\smallbreak\pagebreak[2]}\Standort{CUL, Schnitzler, B 118.}
\physDesc{Brief, 1 Blatt, 1 Seite, 518 Zeichen
\newline{}Handschrift: lila Tinte, lateinische Kurrent
\newline{}Schnitzler: mit Bleistift »\textsc{Zweig}« und datiert: »\textsc{908 Sept?}« }
\buchAbdrucke{\weitereDrucke{Stefan Zweig: \emph{Briefwechsel mit Hermann Bahr, Sigmund Freud, Rainer Maria
                        Rilke und Arthur Schnitzler}. Frankfurt am Main: \emph{S. Fischer} 1987, S. 356.} }\toendnotes[C]{\smallbreak}
\pstart
           {\pb}\textcolor{gray}{\textbf{SZ}}\hfill \textcolor{gray}{\textbf{VIII. KOCHGASSE 8\oindex{Kochgasse 8@\textbf{Kochgasse 8}, \emph{Wohngebäude (K.WHS)}|pw}}}\pend
           
\pstart
           \raggedleft{}\textcolor{gray}{\textbf{WIEN\oindex{Wien@\textbf{Wien}, \emph{A.ADM2}|pw}, }}{ }Donnerstag\pend
           \vspace{0.5em}{\vspace{1\baselineskip}}
\pstart
           Sehr verehrter Herr Doktor, als ich Ihren \label{K_L03623-1v}\edtext{Brief}{\lemma{\textnormal{\emph{Brief}}}\Cendnote{\textnormal{Arthur Schnitzler an Stefan Zweig, 22. 10. 1908.}}}\label{K_L03623-1} in die Hand nahm dachte ich
               mir – ehe ich ihn öffnete – noch in meine erste Freude hinein: Wenn Sie nur nicht
                  Samstag sagen! Nun sagten Sie es wirklich und Samstag ist der einzige
               für mich unaufschiebbar besetzte Abend (eine \label{K_L03623-2v}\edtext{Vorlesung im »Volksheim\oindex{Volkshochschule Ottakring@\textbf{Volkshochschule Ottakring}, \emph{Gebäude (K.GBD)}|pw}\orgindex{Verein Volksheim@Verein Volksheim|pw}«}{\lemma{\textnormal{\emph{Vorlesung im »Volksheim«}}}\Cendnote{\textnormal{Am
                     24. 10. 1908 sprach Stefan Zweig\pwindex{Zweig, Stefan 28.11.1881 – 23.02.1942@\textsc{Zweig, Stefan} (28.11.1881 – 23.02.1942), \emph{Schriftsteller/Schriftstellerin}|pwk} im Volksheim\oindex{Volkshochschule Ottakring@\textbf{Volkshochschule Ottakring}, \emph{Gebäude (K.GBD)}|pwk} am Koflerpark\oindex{Ludo-Hartmann-Platz@\textbf{Ludo-Hartmann-Platz}, \emph{Platz (K.PLT)}|pwk} (heute Ludo-Hartmann-Platz\oindex{Ludo-Hartmann-Platz@\textbf{Ludo-Hartmann-Platz}, \emph{Platz (K.PLT)}|pwk}) über Honoré de Balzac\pwindex{Balzac, Honore de 20.05.1799 – 18.08.1850@\textsc{Balzac, Honoré de} (20.05.1799 – 18.08.1850), \emph{Schriftsteller/Schriftstellerin}|pwk}. }}}\label{K_L03623-2}, die ich Herrn Professor Reich\pwindex{Reich, Emil 29.10.1864 – 13.12.1940@\textsc{Reich, Emil} (29.10.1864 – 13.12.1940)|pw} zusagte.) \pend
           
\pstart
           Wollen Sie, verehrter Herr Doktor, es noch einmal versuchen? Jeder Tag, jede Stunde
               ist mir recht, die Sie mir für \label{K_L03623-3v}\edtext{nächste Woche}{\lemma{\textnormal{\emph{nächste Woche}}}\Cendnote{\textnormal{Das geplante Treffen
                  verschob sich nochmals um eine Woche, auf den 2. 11. 1908.}}}\label{K_L03623-3} bestimmen wollten.\pend
           
\pstart
           In Verehrung getreu{\\[\baselineskip]}\spacefill\mbox{StefanZweig}\pend
           \leftskip=0em{}\selectlanguage{ngerman}\endnumbering\briefempfaengerindex{Schnitzler, Arthur@\textsc{Schnitzler, Arthur}!zzzZweig, Stefan@\emph{von Stefan Zweig}!1908-10-222@{{[}22. 10. 1908?{]}}|)be}\mylabel{L03623h}  \normalsize

\doendnotes{C}
\bigskip
\vfill

\clearpage

\footnotesize

\lohead{\textsc{register}}

% Definiere theindex-Environment komplett neu ohne reledmac
\makeatletter
\renewenvironment{theindex}{%
  \section*{\indexname}%
  \setlength{\parindent}{0pt}%
  \setlength{\parskip}{0pt plus 0.3pt}%
  \let\item\@idxitem
}{%
  \clearpage
}
\makeatother

\IfFileExists{\jobname-pw.ind}{\input{\jobname-pw.ind}}{}

\end{document}

      