%% latex-korrekturansicht-vorspann.tex
%% Vorspann für die Korrekturansicht.
%% Lädt die gemeinsame Datei latex-vorspann.tex mit gesetztem Schalter.

\newif\ifkorrekturansicht
\korrekturansichttrue

\input{../tex-inputs/latex-vorspann}


\section[Richard Beer-Hofmann an Arthur Schnitzler, 12. 5. 1897]{L00675 Richard Beer-Hofmann an Arthur Schnitzler, 12. 5. 1897}
\nopagebreak\mylabel{L00675v}
\rehead{ }\normalsize\beginnumbering\briefempfaengerindex{Schnitzler, Arthur@\textsc{Schnitzler, Arthur}!zzzBeer-Hofmann, Richard@\emph{von Richard Beer-Hofmann}!1897-05-121@{12. 5. 1897}|(be}
\toendnotes[C]{\smallbreak\pagebreak[2]}\Standort{CUL, Schnitzler, B 8.}
\physDesc{Brief, 2 Blätter, 7 Seiten, 1632 Zeichen
\newline{}Handschrift: Bleistift, lateinische Kurrent
\newline{}Ordnung: mit Bleistift von unbekannter Hand nummeriert:
                                    »95« }
\buchAbdrucke{\weitereDrucke{Arthur Schnitzler, Richard Beer-Hofmann: \emph{Briefwechsel 1891–1931}. Wien, Zürich: \emph{Europaverlag} 1992, S. 103–104.} }\toendnotes[C]{\smallbreak}
\pstart
           \centering{}{\pb}Ischl\oindex{Bad Ischl@\textbf{Bad Ischl}, \emph{P.PPL}|pw}. 12/V 97\pend
           \vspace{0.5em}
\pstart
           Lieber Arthur! Ich habe einen recht starken Luftröhrenkatarrh gehabt
               (war auch bei Ihrem Schwager\pwindex{Hajek, Markus 25.11.1861 – 04.04.1941@\textsc{Hajek, Markus} (25.11.1861 – 04.04.1941), \emph{Mediziner/Medizinerin, Laryngologe/Laryngologin}|pw}) und bin deshalb,
               (Luftveränderung) und auch um für P.\pwindex{Beer-Hofmann, Paula 25.02.1879 – 30.10.1939@\textsc{Beer-Hofmann, Paula} (25.02.1879 – 30.10.1939)|pw} Wohnung
               zu suchen am 7/V hieher gereist; übermorgen fahre ich wieder nach Wien\oindex{Wien@\textbf{Wien}, \emph{A.ADM2}|pw} zurück. Anfangs Juni ko{\geminationm}e {\pb}ich dann wieder mit Papa\pwindex{Hofmann, Alois 30.3.1830 – 11.7.1907@\textsc{Hofmann, Alois} (30.3.1830 – 11.7.1907), \emph{Industrieller/Industrielle}|pwv}
               hieher – in unsere alte Wohnung im Egelmoos\oindex{Eglmoosgasse@\textbf{Eglmoosgasse}, \emph{Bezirk (A.BZK)}|pw}. P.\pwindex{Beer-Hofmann, Paula 25.02.1879 – 30.10.1939@\textsc{Beer-Hofmann, Paula} (25.02.1879 – 30.10.1939)|pw} wohnt schon hier in einem kleinen Zi{\geminationm}er, in einem kleinen Haus und ist recht lieb und gut.
               – (Sie werden jetzt lächeln und dieselbe Zärtlichkeit bei sich suchen und finden –
               außer Sie sind ein gottverlassenes {\pb}Scheusaal)\noindent{}die 2 a im letzten Worte sind ein orthographischer Irrthum – keine Feinheit Über Ihr und Goldmanns\pwindex{Goldmann, Paul 31.01.1865 – 25.09.1935@\textsc{Goldmann, Paul} (31.01.1865 – 25.09.1935), \emph{Schriftsteller/Schriftstellerin, Journalist/Journalistin}|pw} Schicksa\strikeout{a}l \strikeout{B} bei dem
               Brandunglück hab ich mir keine Sorgen gemacht. Von Goldmann\pwindex{Goldmann, Paul 31.01.1865 – 25.09.1935@\textsc{Goldmann, Paul} (31.01.1865 – 25.09.1935), \emph{Schriftsteller/Schriftstellerin, Journalist/Journalistin}|pw} wußte ich daß er noch nicht in Paris\oindex{Paris@\textbf{Paris}, \emph{P.PPLC}|pw} war, – ich sprach am selben {\pb}Tag telefonisch mit Ihrer Mama\pwindex{Schnitzler, Louise 1840-07-08 – 1911-09-09@\textsc{Schnitzler, Louise} (1840-07-08 – 1911-09-09)|pwv}, und daß Sie nicht zu
               dergleichen Dingen gehen war mir bekannt.\pend
           
\pstart
           – Wahrscheinlich sind Ihnen aber bei diesem Anlasse alte (»Ihrige«) oder auch neue
               Novellenstoffe von Hinterbliebenen eingefallen; auch {\pb}die Notwendigkeit des Testaments
               machen wird sehr deutlich. –\pend
           
\pstart
           Paul Goldmann\pwindex{Goldmann, Paul 31.01.1865 – 25.09.1935@\textsc{Goldmann, Paul} (31.01.1865 – 25.09.1935), \emph{Schriftsteller/Schriftstellerin, Journalist/Journalistin}|pw} wird – da er ja immer aus allen
               Ereignissen wie die Biene den Honig saugt – aus der Tatsache daß ich \uline{Ihnen} schreibe, irgendwelche Schlüße auf mein Verhältniß zu
               ihm ziehen, und erklären {\pb}»Siehst
               Du, \uline{Dir} schreibt er«! Dann folgt Ihr
               Beruhigungsversuch; dann sagt Paul\pwindex{Goldmann, Paul 31.01.1865 – 25.09.1935@\textsc{Goldmann, Paul} (31.01.1865 – 25.09.1935), \emph{Schriftsteller/Schriftstellerin, Journalist/Journalistin}|pw} sehr
               großartig resignirt: »Laß das Kinderl – ich weiß ja– –! Ja – ja!« Sollte er aber die
               Gemeinheit der Gesinnung soweit treiben, daß er sich vor Aufregung {\pb}auf den eigenen Fuß tritt, –
               »Pardon« ruft und ein Erdbeben markirt, – dann schimpfen Sie ihn gehörig in meinem
               Namen zusa{\geminationm}en. –\pend
           
\pstart
           Wann kommen Sie? –\pend
           
\pstart
           Was macht Paul\pwindex{Goldmann, Paul 31.01.1865 – 25.09.1935@\textsc{Goldmann, Paul} (31.01.1865 – 25.09.1935), \emph{Schriftsteller/Schriftstellerin, Journalist/Journalistin}|pw} im So{\geminationm}er?\pend
           
\pstart
           Herzlichst{\\[\baselineskip]}\spacefill\mbox{Richard}\pend
           \leftskip=0em{}
\pstart
           »\uline{Deutlicher schreiben!}«\pend
           \selectlanguage{ngerman}\endnumbering\briefempfaengerindex{Schnitzler, Arthur@\textsc{Schnitzler, Arthur}!zzzBeer-Hofmann, Richard@\emph{von Richard Beer-Hofmann}!1897-05-121@{12. 5. 1897}|)be}\mylabel{L00675h}  \normalsize

\doendnotes{C}
\bigskip
\vfill

\clearpage

\footnotesize

\lohead{\textsc{register}}

% Definiere theindex-Environment komplett neu ohne reledmac
\makeatletter
\renewenvironment{theindex}{%
  \section*{\indexname}%
  \setlength{\parindent}{0pt}%
  \setlength{\parskip}{0pt plus 0.3pt}%
  \let\item\@idxitem
}{%
  \clearpage
}
\makeatother

\IfFileExists{\jobname-pw.ind}{\input{\jobname-pw.ind}}{}

\end{document}

      