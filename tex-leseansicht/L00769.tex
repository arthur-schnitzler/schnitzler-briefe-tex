%% latex-korrekturansicht-vorspann.tex
%% Vorspann für die Korrekturansicht.
%% Lädt die gemeinsame Datei latex-vorspann.tex mit gesetztem Schalter.

\newif\ifkorrekturansicht
\korrekturansichttrue

\input{../tex-inputs/latex-vorspann}


\section[Hugo von Hofmannsthal an Arthur Schnitzler, 25. 1. 1898]{L00769 Hugo von Hofmannsthal an Arthur Schnitzler, 25. 1. 1898}
\nopagebreak\mylabel{L00769v}
\rehead{ }\normalsize\beginnumbering\briefempfaengerindex{Schnitzler, Arthur@\textsc{Schnitzler, Arthur}!zzzHofmannsthal, Hugo von@\emph{von Hugo von Hofmannsthal}!1898-01-254@{25. 1. 1898}|(be}
\toendnotes[C]{\smallbreak\pagebreak[2]}\Standort{CUL, Schnitzler, B 43b/1.}
\physDesc{Postkarte, 439 Zeichen
\newline{}Handschrift: Bleistift, deutsche Kurrent
\newline{}Versand: 1) Rohrpost  2) Stempel: »\nobreak{}Wien 3/3, 25 1 98, 10 20V\nobreak{}«.  3) Stempel: »\nobreak{}\oindex{IX., Alsergrund@\textbf{IX., Alsergrund}, \emph{A.ADM3}|pwk}Wien 9/2, 25 1 98, 11 10V\nobreak{}«. 
\newline{}Schnitzler: mit Bleistift datiert: »25/1 98« 
\newline{}Ordnung: 1) mit Bleistift von unbekannter Hand nummeriert: »\strikeout{109}«  2) mit Bleistift von unbekannter Hand nummeriert:
                                    »106«}
\buchAbdrucke{\weitereDrucke{Hugo von Hofmannsthal, Arthur Schnitzler: \emph{Briefwechsel}. Frankfurt am Main: \emph{S. Fischer} 1964, S. 98–99.} }\toendnotes[C]{\smallbreak}\pstart{}{\pb}\textsc{Herrn D\textsuperscript{r} Arthur Schnitzler}\pend{}\pstart{}\textsc{Franckgasse 1}\hspace*{2.5em}IX\oindex{Frankgasse 1@\textbf{Frankgasse 1}, \emph{Wohngebäude (K.WHS)}|pw}\pend{}{\bigskip}\vspace{1em}
\pstart
           \raggedleft{}{\pb}10\textsuperscript{h} früh\pend
           \vspace{0.5em}
\pstart
           \textsc{Poldy\pwindex{Andrian-Werburg, Leopold von 09.05.1875 – 19.11.1951@\textsc{Andrian-Werburg, Leopold von} (09.05.1875 – 19.11.1951), \emph{Schriftsteller/Schriftstellerin, Diplomat/Diplomatin}|pw}} iſt wegen »\label{K_L00769-1v}\edtext{mangelnder \textsc{Patellarreflexe}}{\lemma{\textnormal{\emph{mangelnder Patellarreflexe}}}\Cendnote{\textnormal{Durch leichten Schlag auf die unterhalb
                  der Kniescheibe befindliche Sehne wird ein Reflex ausgelöst. Das Unterbleiben
                  einer Reaktion kann auf eine Erkrankung des Nervensystems verweisen.}}}\label{K_L00769-1}« außer
               ſich und will durchaus ich ſoll Ihnen um die »Wahrheit« telefonieren, ihm dann
               ſchreiben. Ich halte das für Zeitverluſt, ſchreibe ihm beruhigend pneumatiſch, \uline{als ob ich sie gefragt hätte}. Sollte er zu Ihnen
               kommen, ſo thuen Sie als ob ich gefragt hätte. Sollte etwas zu ſagen ſein, was ich
               nicht glaube, bitte ſchreiben Sie \uline{mir ſogleich}.\pend
           \pstart Ihr \spacefill\mbox{Hugo}\pend{}\selectlanguage{ngerman}\endnumbering\briefempfaengerindex{Schnitzler, Arthur@\textsc{Schnitzler, Arthur}!zzzHofmannsthal, Hugo von@\emph{von Hugo von Hofmannsthal}!1898-01-254@{25. 1. 1898}|)be}\mylabel{L00769h}  \normalsize

\doendnotes{C}
\bigskip
\vfill

\clearpage

\footnotesize

\lohead{\textsc{register}}

% Definiere theindex-Environment komplett neu ohne reledmac
\makeatletter
\renewenvironment{theindex}{%
  \section*{\indexname}%
  \setlength{\parindent}{0pt}%
  \setlength{\parskip}{0pt plus 0.3pt}%
  \let\item\@idxitem
}{%
  \clearpage
}
\makeatother

\IfFileExists{\jobname-pw.ind}{\input{\jobname-pw.ind}}{}

\end{document}

      