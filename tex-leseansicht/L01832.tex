%% latex-korrekturansicht-vorspann.tex
%% Vorspann für die Korrekturansicht.
%% Lädt die gemeinsame Datei latex-vorspann.tex mit gesetztem Schalter.

\newif\ifkorrekturansicht
\korrekturansichttrue

\input{../tex-inputs/latex-vorspann}


\section[Arthur Schnitzler an Robert Adam, 20. 3. 1909]{L01832 Arthur Schnitzler an Robert Adam, 20. 3. 1909}
\nopagebreak\mylabel{L01832v}
\rehead{ }\normalsize\beginnumbering\briefempfaengerindex{Adam, Robert@\textsc{Adam, Robert}!zzzSchnitzler, Arthur@\emph{von Arthur Schnitzler}!1909-03-201@{20. 3. 1909}|(be}
\toendnotes[C]{\smallbreak\pagebreak[2]}\Standort{DLA, 96.34.1/1.}
\physDesc{Brief, 1 Blatt, 1 Seite, Umschlag, 1028 Zeichen
\newline{}Schreibmaschine
\newline{}Handschrift: schwarze Tinte, lateinische Kurrent (\noindent{}Grußformel, Unterschrift sowie zwei Korrekturen)
\newline{}Versand: Stempel: »\nobreak{}\oindex{XVIII., Waehring@\textbf{XVIII., Währing}, \emph{A.ADM3}|pwk}18/1 Wien 110, 20. III. 09, 4\nobreak{}«.  }\Standort{DLA, A:Schnitzler, 85.1.1621.}
\physDesc{Brief, Durchschlag1 Blatt, 1 Seite, Umschlag, 1028 Zeichen
\newline{}Schreibmaschine
\newline{}Handschrift Frieda Pollak: Bleistift, lateinische Kurrent (\noindent{}am oberen Rand beschrieben mit »Adam« und am unteren
                                 Rand mit »Robert Adam«)
\newline{}Handschrift Arthur Schnitzler: roter Buntstift, deutsche Kurrent (\noindent{}Streichung von »Robert Adam«, »Adam«
                                 überschrieben: »\textsc{Adam}« und beschriftet mit: »\textsc{Po}{[}llak{]}«)}\toendnotes[C]{\smallbreak}\pstart{}{\pb}\textcolor{gray}{\textbf{Dr. Arthur Schnitzler}}\pend{}\pstart{}\textcolor{gray}{\textbf{Wien, XVIII. Spoettelgasse 7}}.\oindex{Edmund-Weiss-Gasse 7@\textbf{Edmund-Weiß-Gasse 7}, \emph{Wohngebäude (K.WHS)}|pw}\pend{}{\bigskip}\pstart{}{\pb}Herrn\pend{}\pstart{}Robert Adam.\pend{}\pstart{}\so{Wien XII}\oindex{XII., Meidling@\textbf{XII., Meidling}, \emph{A.ADM3}|pw}\pend{}\pstart{}Meidlinger Hauptstraße 56\oindex{Meidlinger Hauptstrasse@\textbf{Meidlinger Hauptstraße}, \emph{Straße (K.STR)}|pw}\pend{}{\bigskip}\vspace{1em}
\pstart
           {\pb}\textcolor{gray}{\textbf{Dr. Arthur Schnitzler}}\hfill 20. März 09.\pend
           
\pstart
           \textcolor{gray}{\textbf{Wien XVIII. Spoettelgasse 7\oindex{Edmund-Weiss-Gasse 7@\textbf{Edmund-Weiß-Gasse 7}, \emph{Wohngebäude (K.WHS)}|pw}.}}\pend
           
\pstart{}Sehr geehrter Herr,\pend\vspace{0.5em}
\pstart
           Ihre anmutige \label{K_L01832-1v}\edtext{Harun ar Raschid\pwindex{Harun ar-Raschid um 763 – 809@\textsc{Harun ar-Raschid} (um 763 – 809), \emph{Kalif/Kalifin}|pw} Komödie\pwindex{Geschichte des Alî ibn Bekkâr mit Schams an-Nahâr@\emph{Die Geschichte des Alî ibn Bekkâr mit Schams an-Nahâr}|pwv}}{\lemma{\textnormal{\emph{Harun ar Raschid Komödie}}}\Cendnote{\textnormal{Hârûn ar-Raschid\pwindex{Harun ar-Raschid um 763 – 809@\textsc{Harun ar-Raschid} (um 763 – 809), \emph{Kalif/Kalifin}|pwk} ist eine von sechs Personen
                  der Komödie \emph{Die Geschichte des Alî ibn Bekkâr mit
                     Schams an-Nahâr}\pwindex{Geschichte des Alî ibn Bekkâr mit Schams an-Nahâr@\emph{Die Geschichte des Alî ibn Bekkâr mit Schams an-Nahâr}|pwk}.}}}\label{K_L01832-1} habe ich mit wirklichem Vergnügen gelesen. Man
               wünschte \introOben{}wohl\introOben{} sie auf einer Bühne zu sehn, wenn auch schwer
               zu sagen ist auf welcher. Mir persönlich würde ja eine Aufführung kaum etwas Neues
               bieten, aber da die Fähigkeit Stücke zu lesen eine selbst bei sonst klugen Menschen
               wenig ausgebildete ist, würden sich manche und nicht die geringsten Reize Ihres Stückes\pwindex{Geschichte des Alî ibn Bekkâr mit Schams an-Nahâr@\emph{Die Geschichte des Alî ibn Bekkâr mit Schams an-Nahâr}|pwv} doch erst auf der Scene
               enthüllen. Anderseits ist zu bedenken, dass gerade hier eine nicht ganz vorzügliche
               Darstellung vieles Feine vergröbern\substVorne{}\textsuperscript{und}\substDazwischen{},\substHinten{} das dramatisch dünne Ihrer Komödie\pwindex{Geschichte des Alî ibn Bekkâr mit Schams an-Nahâr@\emph{Die Geschichte des Alî ibn Bekkâr mit Schams an-Nahâr}|pwv} aufdecken und die eigentümliche Melodie des Verses kaum zur Geltung
               bringen würde\substVorne{}\textsuperscript{.}\substDazwischen{};\substHinten{} womit sich also der Zirkel in einer für junge Autoren keineswegs
               erfreulichen Weise geschlossen zu haben scheint. Jedenfalls danke ich persönlich
               bestens für die liebenswürdige Uebersendung und wünsche der zarten Komödie Glück,
               woher es auch kommen möge.\pend
           
\pstart
           {[}hs.:{]} Ihr sehr ergebener{\\[\baselineskip]}\spacefill\mbox{Arthur Schnitzler}\pend
           \leftskip=0em{}\selectlanguage{ngerman}\endnumbering\briefempfaengerindex{Adam, Robert@\textsc{Adam, Robert}!zzzSchnitzler, Arthur@\emph{von Arthur Schnitzler}!1909-03-201@{20. 3. 1909}|)be}\mylabel{L01832h}  \normalsize

\doendnotes{C}
\bigskip
\vfill

\clearpage

\footnotesize

\lohead{\textsc{register}}

% Definiere theindex-Environment komplett neu ohne reledmac
\makeatletter
\renewenvironment{theindex}{%
  \section*{\indexname}%
  \setlength{\parindent}{0pt}%
  \setlength{\parskip}{0pt plus 0.3pt}%
  \let\item\@idxitem
}{%
  \clearpage
}
\makeatother

\IfFileExists{\jobname-pw.ind}{\input{\jobname-pw.ind}}{}

\end{document}

      