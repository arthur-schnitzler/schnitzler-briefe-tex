%% latex-leseansicht-vorspann.tex
%% Vorspann für die Leseansicht.
%% Lädt die gemeinsame Datei latex-vorspann.tex mit nicht gesetztem Schalter.

\newif\ifkorrekturansicht
\korrekturansichtfalse

\input{../tex-inputs/latex-vorspann}


               \section[Arthur Schnitzler an Robert Adam, 20. 3. 1909]{ Arthur Schnitzler an Robert Adam, 20. 3. 1909}\nopagebreak\mylabel{v}\rehead{ }\begin{ledgroupsized}[t]{13cm}\normalsize\beginnumbering\briefempfaengerindex{Adam, Robert@\textsc{Adam, Robert}!zzzSchnitzler, Arthur@\emph{von Arthur Schnitzler}!1909-03-201@{20. 3. 1909}|(be} \toendnotes[C]{\smallbreak\pagebreak[2]} \Standort{DLA, 96.34.1/1.}
\physDesc{Brief, 1 Blatt, 1 Seite, Umschlag
\newline{}Schreibmaschine
\newline{}Handschrift: schwarze Tinte, lateinische Kurrent (\noindent{}Grußformel, Unterschrift sowie zwei Korrekturen)\newline{}Versand: Stempel: »\nobreak{}\oindex{XVIII., Waehring@\textbf{XVIII., Währing}|pwk}18/1 Wien 110, 20. III. 09, 4\nobreak{}«.  }\Standort{DLA, A:Schnitzler, 85.1.1621.}
\physDesc{Brief, 1 Blatt, 1 Seite, Umschlag, maschineller Durchschlag
\newline{}Schreibmaschine
\newline{}Handschrift Frieda Pollak: Bleistift, lateinische Kurrent (\noindent{}am oberen Rand beschrieben mit »Adam« und am unteren
                                 Rand mit »Robert Adam«)\newline{}Handschrift Arthur Schnitzler: roter Buntstift, deutsche Kurrent (\noindent{}Streichung von »Robert Adam«, »Adam«
                                 überschrieben: »\textsc{Adam}« und beschriftet mit: »\textsc{Po}{[}llak{]}«)}\toendnotes[C]{\smallbreak}\pstart{}{\pb}\textcolor{gray}{\textbf{Dr. Arthur Schnitzler}}\pend{}\pstart{}\textcolor{gray}{\textbf{Wien, XVIII. Spoettelgasse 7}}.\oindex{Edmund-Weiss-Gasse@\textbf{Edmund-Weiß-Gasse}|pw}\pend{}{\bigskip}\pstart{}{\pb}Herrn\pend{}\pstart{}Robert Adam.\pend{}\pstart{}\so{Wien XII}\oindex{XII., Meidling@\textbf{XII., Meidling}|pw}\pend{}\pstart{}Meidlinger Hauptstraße 56\oindex{Meidlinger Hauptstrasse@\textbf{Meidlinger Hauptstraße}|pw}\pend{}{\bigskip}\pstart
           \noindent{}{\pb}\textcolor{gray}{\textbf{Dr. Arthur Schnitzler}}\hfill 20. März 09.\pend
           \pstart
           \textcolor{gray}{\textbf{Wien XVIII. Spoettelgasse 7\oindex{Edmund-Weiss-Gasse@\textbf{Edmund-Weiß-Gasse}|pw}.}}\pend
           \pstart{}Sehr geehrter Herr,\pend\pstart
           Ihre anmutige \label{K_L01832_1v}\edtext{Harun ar Raschid\pwindex{Harun ar-Raschid um 763 – 809@\textsc{Harun ar-Raschid} (um 763 – 809), \emph{Kalif}|pw} Komödie\pwindex{Adam, Robert 20.04.1877 – 16.10.1961@\textsc{Adam, Robert} (20.04.1877 – 16.10.1961), \emph{Schriftsteller, Richter}!Geschichte des Alî ibn Bekkâr mit Schams an-Nahâr1909@\strich\emph{Die Geschichte des Alî ibn Bekkâr mit Schams an-Nahâr} {[}1909{]}|pwv}}{\lemma{\textnormal{\emph{Harun ar Raschid Komödie}}}\Cendnote{\textnormal{Hârûn ar-Raschid\pwindex{Harun ar-Raschid um 763 – 809@\textsc{Harun ar-Raschid} (um 763 – 809), \emph{Kalif}|pwk} ist eine von sechs Personen
                  der Komödie \emph{Die Geschichte des Alî ibn Bekkâr mit
                     Schams an-Nahâr}\pwindex{Adam, Robert 20.04.1877 – 16.10.1961@\textsc{Adam, Robert} (20.04.1877 – 16.10.1961), \emph{Schriftsteller, Richter}!Geschichte des Alî ibn Bekkâr mit Schams an-Nahâr1909@\strich\emph{Die Geschichte des Alî ibn Bekkâr mit Schams an-Nahâr} {[}1909{]}|pwk}.}}}\label{K_L01832_1h} habe ich mit wirklichem Vergnügen gelesen. Man
               wünschte \introOben{}wohl\introOben{} sie auf einer Bühne zu sehn, wenn auch schwer
               zu sagen ist auf welcher. Mir persönlich würde ja eine Aufführung kaum etwas Neues
               bieten, aber da die Fähigkeit Stücke zu lesen eine selbst bei sonst klugen Menschen
               wenig ausgebildete ist, würden sich manche und nicht die geringsten Reize Ihres Stückes\pwindex{Adam, Robert 20.04.1877 – 16.10.1961@\textsc{Adam, Robert} (20.04.1877 – 16.10.1961), \emph{Schriftsteller, Richter}!Geschichte des Alî ibn Bekkâr mit Schams an-Nahâr1909@\strich\emph{Die Geschichte des Alî ibn Bekkâr mit Schams an-Nahâr} {[}1909{]}|pwv} doch erst auf der Scene
               enthüllen. Anderseits ist zu bedenken, dass gerade hier eine nicht ganz vorzügliche
               Darstellung vieles Feine vergröbern\substVorne{}\textsuperscript{und}\substDazwischen{},\substHinten{} das dramatisch dünne Ihrer Komödie\pwindex{Adam, Robert 20.04.1877 – 16.10.1961@\textsc{Adam, Robert} (20.04.1877 – 16.10.1961), \emph{Schriftsteller, Richter}!Geschichte des Alî ibn Bekkâr mit Schams an-Nahâr1909@\strich\emph{Die Geschichte des Alî ibn Bekkâr mit Schams an-Nahâr} {[}1909{]}|pwv} aufdecken und die eigentümliche Melodie des Verses kaum zur Geltung
               bringen würde\substVorne{}\textsuperscript{.}\substDazwischen{};\substHinten{} womit sich also der Zirkel in einer für junge Autoren keineswegs
               erfreulichen Weise geschlossen zu haben scheint. Jedenfalls danke ich persönlich
               bestens für die liebenswürdige Uebersendung und wünsche der zarten Komödie Glück,
               woher es auch kommen möge.\pend
           \pstart
           {[}hs.:{]} Ihr sehr ergebener{\\[\baselineskip]}Arthur Schnitzler\pend
           \leftskip=0em{}          \endnumbering\briefempfaengerindex{Adam, Robert@\textsc{Adam, Robert}!zzzSchnitzler, Arthur@\emph{von Arthur Schnitzler}!1909-03-201@{20. 3. 1909}|)be}\mylabel{h}\end{ledgroupsized}  \newcommand{\dateiname}{L01832}\newcommand{\titel}{Arthur Schnitzler an Robert Adam, 20. 3. 1909}\newcommand{\editorInnen}{Martin Anton Müller und Gerd-Hermann Susen}
            \footnotesize
\begin{ledgroupsized}[t]{11.5cm}
\doendnotes{C}
\end{ledgroupsized}
         %% latex-leseansicht-abspann.tex
%% Abspann für die Leseansicht.
%% Der Schalter \ifkorrekturansicht ist bereits durch den Vorspann gesetzt.

%% latex-abspann.tex
%% Gemeinsamer Abspann für Korrekturansicht und Leseansicht.
%% Setzt den Schalter \ifkorrekturansicht voraus (gesetzt in den
%% einbindenden Dateien latex-korrekturansicht-abspann.tex bzw.
%% latex-leseansicht-abspann.tex).
%% ---------------------------------------------------------------

\normalsize

% Das esempio-Environment wird nur in der Leseansicht benötigt
\ifkorrekturansicht\else
\newenvironment{esempio}[3]%
{
    \vspace{1.5ex}
    \rlap{\underline{#1}}
    \par
    \setlength{\parindent}{0cm}
    \nopagebreak
    \leftskip=#2cm
    \rightskip=#3cm
}
{
    \par
}
\fi

\doendnotes{C}
\bigskip
\vfill

\clearpage

\footnotesize

\ifkorrekturansicht
  \lohead{\textsc{register}}
\fi

% theindex-Environment neu definieren ohne reledmac
\makeatletter
\renewenvironment{theindex}{%
  \ifkorrekturansicht
    \section*{\indexname}%
  \else
    \subsubsection*{Index der erwähnten Entitäten}%
  \fi
  \setlength{\parindent}{0pt}%
  \setlength{\parskip}{0pt plus 0.3pt}%
  \let\item\@idxitem
}{%
  \ifkorrekturansicht\clearpage\fi
}
\makeatother

\IfFileExists{\jobname-pw.ind}{\input{\jobname-pw.ind}}{}

% Quellenangabe nur in der Leseansicht
\ifkorrekturansicht\else
% Fallback-Definitionen, falls die .tex-Datei \titel etc. nicht gesetzt hat
\providecommand{\titel}{}
\providecommand{\editorInnen}{}
\providecommand{\dateiname}{\jobname}

\vspace{3cm}

\vfill

\footnotesize
\textsc{Quelle}: \titel. Herausgegeben von {\editorInnen}. In: \emph{Arthur Schnitzler: Briefwechsel mit Autorinnen und Autoren}.
 Digitale Edition, https://schnitzler-briefe.acdh.oeaw.ac.at/{\dateiname}.html (Stand \today)
\fi

\end{document}


      