%% latex-leseansicht-vorspann.tex
%% Vorspann für die Leseansicht.
%% Lädt die gemeinsame Datei latex-vorspann.tex mit nicht gesetztem Schalter.

\newif\ifkorrekturansicht
\korrekturansichtfalse

\input{../tex-inputs/latex-vorspann}


\section[Arthur Schnitzler an Richard Beer-Hofmann, 18. 3. 1897]{L00654 Arthur Schnitzler an Richard Beer-Hofmann, 18. 3. 1897}
\nopagebreak\mylabel{L00654v}
\rehead{ }\normalsize\beginnumbering\briefempfaengerindex{Beer-Hofmann, Richard@\textsc{Beer-Hofmann, Richard}!zzzSchnitzler, Arthur@\emph{von Arthur Schnitzler}!1897-03-181@{18. 3. 1897}|(be}
\toendnotes[C]{\smallbreak\pagebreak[2]}
\correspDesc{Versand  durch Arthur Schnitzler am 18. 3. 1897 in Wien
\newline{}Erhalt  durch Richard Beer-Hofmann am 18. 3. 1897 in Wien}\toendnotes[C]{\smallbreak}
\Standort{YCGL, MSS 31.}
\physDesc{Postkarte, 128 Zeichen
\newline{}Handschrift: Bleistift, deutsche Kurrent
\newline{}Versand: 1) Rohrpost  2) Stempel: »\nobreak{}\oindex{I., Innere Stadt@\textbf{I., Innere Stadt}, \emph{Verwaltungsgebiet}|pwk}Wien 1/1, 18 {[}3. 1897{]}, 7 30 V\nobreak{}«.  3) Stempel: »\nobreak{}\oindex{I., Innere Stadt@\textbf{I., Innere Stadt}, \emph{Verwaltungsgebiet}|pwk}Wien 1/1, 18 III 97, 7 40 V\nobreak{}«. }\toendnotes[C]{\smallbreak}\pstart{}{\pb}Herrn Dr \textsc{Richard
                     Beer-Hofmann}\pend{}\pstart{}Wien\oindex{Wien@\textbf{Wien}, \emph{Verwaltungsgebiet}|pw}\pend{}\pstart{}\textsc{I. Wollzeile 15\oindex{Wien@\textbf{Wien}!I., Innere Stadt@\textbf{I., Innere Stadt}!Wollzeile 15 (»Berthahof«)@\textbf{Wollzeile 15 (»Berthahof«)}, \emph{Wohngebäude}|pw}.}\pend{}{\bigskip}\vspace{1em}
\pstart
           \noindent{}{\pb}Raimundtheater\oindex{Wien@\textbf{Wien}!VI., Mariahilf@\textbf{VI., Mariahilf}!Raimund-Theater@\textbf{Raimund-Theater}, \emph{Theater}|pw}!\pend
           
\pstart
           Vergeſſen Sie nicht!{\\}2 \label{K_L00654-1v}\edtext{Sitze\pwindex{\textcolor{red}{\textsuperscript{XXXX indx1}}!Sklavin. Schauspiel in vier Aufzügen@\strich\emph{Die Sklavin. Schauspiel in vier Aufzügen}|pwv}}{\lemma{\textnormal{\emph{Sitze}}}\Cendnote{\textnormal{Schnitzler besuchte die Premiere von \emph{Die Sklavin}\pwindex{\textcolor{red}{\textsuperscript{XXXX indx1}}!Sklavin. Schauspiel in vier Aufzügen@\strich\emph{Die Sklavin. Schauspiel in vier Aufzügen}|pwk} (\emph{Cambridge University Library}, A 179a).}}}\label{K_L00654-1}! Mir{ }ſchicken!\pend
           
\pstart
           \label{T_L00654-1v}\edtext{\label{K_L00654-2v}\edtext{Von mir keine Grüße}{\lemma{\textnormal{\emph{Von mir keine Grüße}}}\Cendnote{\textnormal{In der Handschrift von Beer-Hofmann\pwindex{Beer-Hofmann, Richard 11.\,7.\,1866 Wien – 26.\,9.\,1945 New York City@\textsc{Beer-Hofmann, Richard} (11.\,7.\,1866 Wien – 26.\,9.\,1945 New York City), \emph{Schriftsteller}|pwk} steht mit Bleistift in lateinischer
                  Kurrentschrift auf der Karte geschrieben: »Herzliche Grüße von
                     Richard«. Die Reaktion Schnitzlers bezieht sich darauf, wobei zwei Abläufe denkbar sind: Der Gruß
                  befand sich auf der Karte, als Schnitzler
                  beschloss, sie wiederzuverwenden. Oder Beer-Hofmann\pwindex{Beer-Hofmann, Richard 11.\,7.\,1866 Wien – 26.\,9.\,1945 New York City@\textsc{Beer-Hofmann, Richard} (11.\,7.\,1866 Wien – 26.\,9.\,1945 New York City), \emph{Schriftsteller}|pwk} ergänzte den Gruß, als er die gewünschten Theaterkarten
                  zusammen mit dieser Karte retournierte, woraufhin Schnitzler seine Reaktion notierte und erneut zurücksandte.}}}\label{K_L00654-2}}{\lemma{\textnormal{\emph{Von mir keine Grüße}}}\Cendnote{\textnormal{am oberen Rand auf dem Kopf}}}\label{T_L00654-1}\spacefill\mbox{Arth}\pend
           \selectlanguage{ngerman}\endnumbering\briefempfaengerindex{Beer-Hofmann, Richard@\textsc{Beer-Hofmann, Richard}!zzzSchnitzler, Arthur@\emph{von Arthur Schnitzler}!1897-03-181@{18. 3. 1897}|)be}\mylabel{L00654h}  \newcommand{\dateiname}{L00654}\newcommand{\titel}{Arthur Schnitzler an Richard Beer-Hofmann, 18. 3. 1897}\newcommand{\editorInnen}{Herausgegeben von Martin Anton Müller}%% latex-leseansicht-abspann.tex
%% Abspann für die Leseansicht.
%% Der Schalter \ifkorrekturansicht ist bereits durch den Vorspann gesetzt.

%% latex-abspann.tex
%% Gemeinsamer Abspann für Korrekturansicht und Leseansicht.
%% Setzt den Schalter \ifkorrekturansicht voraus (gesetzt in den
%% einbindenden Dateien latex-korrekturansicht-abspann.tex bzw.
%% latex-leseansicht-abspann.tex).
%% ---------------------------------------------------------------

\normalsize

% Das esempio-Environment wird nur in der Leseansicht benötigt
\ifkorrekturansicht\else
\newenvironment{esempio}[3]%
{
    \vspace{1.5ex}
    \rlap{\underline{#1}}
    \par
    \setlength{\parindent}{0cm}
    \nopagebreak
    \leftskip=#2cm
    \rightskip=#3cm
}
{
    \par
}
\fi

\doendnotes{C}
\bigskip
\vfill

\clearpage

\footnotesize

\ifkorrekturansicht
  \lohead{\textsc{register}}
\fi

% theindex-Environment neu definieren ohne reledmac
\makeatletter
\renewenvironment{theindex}{%
  \ifkorrekturansicht
    \section*{\indexname}%
  \else
    \subsubsection*{Index der erwähnten Entitäten}%
  \fi
  \setlength{\parindent}{0pt}%
  \setlength{\parskip}{0pt plus 0.3pt}%
  \let\item\@idxitem
}{%
  \ifkorrekturansicht\clearpage\fi
}
\makeatother

\IfFileExists{\jobname-pw.ind}{\input{\jobname-pw.ind}}{}

% Quellenangabe nur in der Leseansicht
\ifkorrekturansicht\else
% Fallback-Definitionen, falls die .tex-Datei \titel etc. nicht gesetzt hat
\providecommand{\titel}{}
\providecommand{\editorInnen}{}
\providecommand{\dateiname}{\jobname}

\vspace{3cm}

\vfill

\footnotesize
\textsc{Quelle}: \titel. Herausgegeben von {\editorInnen}. In: \emph{Arthur Schnitzler: Briefwechsel mit Autorinnen und Autoren}.
 Digitale Edition, https://schnitzler-briefe.acdh.oeaw.ac.at/{\dateiname}.html (Stand \today)
\fi

\end{document}


