%% latex-leseansicht-vorspann.tex
%% Vorspann für die Leseansicht.
%% Lädt die gemeinsame Datei latex-vorspann.tex mit nicht gesetztem Schalter.

\newif\ifkorrekturansicht
\korrekturansichtfalse

\input{../tex-inputs/latex-vorspann}


         
         \renewcommand{\erwaehntePersonen}{Personen: Richard Beer-Hofmann}
         \renewcommand{\erwaehnteOrte}{Orte: I., Innere Stadt, Raimund-Theater, Wien, Wollzeile}
         \renewcommand{\erwaehnteWerke}{Werke: Die Sklavin. Schauspiel in vier Aufzügen}
               \section[Arthur Schnitzler an Richard Beer-Hofmann, 18. 3. 1897]{ Arthur Schnitzler an Richard Beer-Hofmann, 18. 3. 1897}\nopagebreak\mylabel{v}\rehead{ }\begin{ledgroupsized}[t]{13cm}\normalsize\beginnumbering \toendnotes[C]{\smallbreak\pagebreak[2]} \Standort{YCGL, MSS 31.}
\physDesc{Postkarte, 128 Zeichen
\newline{}Handschrift: Bleistift, deutsche Kurrent
\newline{}Versand: 1) Rohrpost  2) Stempel: »\nobreak{}\oindex{I., Innere Stadt@\textbf{I., Innere Stadt}|pwk}Wien 1/1, 18 {[}3. 1897{]}, 7 30 V\nobreak{}«.  3) Stempel: »\nobreak{}\oindex{I., Innere Stadt@\textbf{I., Innere Stadt}|pwk}Wien 1/1, 18 III 97, 7 40 V\nobreak{}«. }\toendnotes[C]{\smallbreak}\pstart{}{\pb}Herrn Dr \textsc{Richard
                     Beer-Hofmann}\pend{}\pstart{}Wien\oindex{Wien@\textbf{Wien}|pw}\pend{}\pstart{}\textsc{I. Wollzeile 15\oindex{Wollzeile@\textbf{Wollzeile}|pw}.}\pend{}{\bigskip}\pstart
           \noindent{}{\pb}Raimundtheater\oindex{Raimund-Theater@\textbf{Raimund-Theater}|pw}!\pend
           \pstart
           Vergeſſen Sie nicht!{\\}2 \label{K_L00654_1v}\edtext{Sitze\pwindex{\textcolor{red}{\textsuperscript{XXXX1 indx}}!Sklavin. Schauspiel in vier Aufzuegen1891@\strich\emph{Die Sklavin. Schauspiel in vier Aufzügen} {[}1891{]}|pwv}}{\lemma{\textnormal{\emph{Sitze}}}\Cendnote{\textnormal{Schnitzler\pwindex{Schnitzler, Arthur 15.05.1862 – 21.10.1931@\textsc{Schnitzler, Arthur} (15.05.1862 – 21.10.1931), \emph{Schriftsteller, Mediziner}|pwk} besuchte die Premiere von \emph{Die Sklavin}\pwindex{\textcolor{red}{\textsuperscript{XXXX1 indx}}!Sklavin. Schauspiel in vier Aufzuegen1891@\strich\emph{Die Sklavin. Schauspiel in vier Aufzügen} {[}1891{]}|pwk}. (\emph{Cambridge University Library}, A 179a)}}}\label{K_L00654_1h}! Mir ſchicken!\pend
           \pstart
           \label{T_L00654_1v}\edtext{\label{K_L00654_2v}\edtext{Von mir keine Grüße}{\lemma{\textnormal{\emph{Von mir keine Grüße}}}\Cendnote{\textnormal{In der Handschrift von Beer-Hofmann\pwindex{Beer-Hofmann, Richard 1866-07-11 – 1945-09-26@\textsc{Beer-Hofmann, Richard} (1866-07-11 – 1945-09-26), \emph{Schriftsteller}|pwk} steht mit Bleistift in lateinischer
                  Kurrentschrift auf der Karte geschrieben: »Herzliche Grüße von
                     Richard«. Die Reaktion Schnitzler\pwindex{Schnitzler, Arthur 15.05.1862 – 21.10.1931@\textsc{Schnitzler, Arthur} (15.05.1862 – 21.10.1931), \emph{Schriftsteller, Mediziner}|pwk}s bezieht sich darauf, wobei zwei Abläufe denkbar sind: Der Gruß
                  befand sich auf der Karte, als Schnitzler\pwindex{Schnitzler, Arthur 15.05.1862 – 21.10.1931@\textsc{Schnitzler, Arthur} (15.05.1862 – 21.10.1931), \emph{Schriftsteller, Mediziner}|pwk}
                  beschloss, sie wiederzuverwenden. Oder Beer-Hofmann\pwindex{Beer-Hofmann, Richard 1866-07-11 – 1945-09-26@\textsc{Beer-Hofmann, Richard} (1866-07-11 – 1945-09-26), \emph{Schriftsteller}|pwk} ergänzte den Gruß, als er die gewünschten Theaterkarten
                  zusammen mit dieser Karte retournierte, woraufhin Schnitzler\pwindex{Schnitzler, Arthur 15.05.1862 – 21.10.1931@\textsc{Schnitzler, Arthur} (15.05.1862 – 21.10.1931), \emph{Schriftsteller, Mediziner}|pwk} seine Reaktion notierte und erneut zurücksandte.}}}\label{K_L00654_2h}}{\lemma{\textnormal{\emph{Von mir keine Grüße}}}\Cendnote{\textnormal{am oberen Rand auf dem Kopf}}}\label{T_L00654_1h}\spacefill\mbox{Arth}\pend
           
         
         \endnumbering\mylabel{h}\end{ledgroupsized}  \newcommand{\dateiname}{L00654}\newcommand{\titel}{Arthur Schnitzler an Richard Beer-Hofmann, 18. 3. 1897}\newcommand{\editorInnen}{ Martin Anton Müller und Gerd-Hermann Susen}%% latex-leseansicht-abspann.tex
%% Abspann für die Leseansicht.
%% Der Schalter \ifkorrekturansicht ist bereits durch den Vorspann gesetzt.

%% latex-abspann.tex
%% Gemeinsamer Abspann für Korrekturansicht und Leseansicht.
%% Setzt den Schalter \ifkorrekturansicht voraus (gesetzt in den
%% einbindenden Dateien latex-korrekturansicht-abspann.tex bzw.
%% latex-leseansicht-abspann.tex).
%% ---------------------------------------------------------------

\normalsize

% Das esempio-Environment wird nur in der Leseansicht benötigt
\ifkorrekturansicht\else
\newenvironment{esempio}[3]%
{
    \vspace{1.5ex}
    \rlap{\underline{#1}}
    \par
    \setlength{\parindent}{0cm}
    \nopagebreak
    \leftskip=#2cm
    \rightskip=#3cm
}
{
    \par
}
\fi

\doendnotes{C}
\bigskip
\vfill

\clearpage

\footnotesize

\ifkorrekturansicht
  \lohead{\textsc{register}}
\fi

% theindex-Environment neu definieren ohne reledmac
\makeatletter
\renewenvironment{theindex}{%
  \ifkorrekturansicht
    \section*{\indexname}%
  \else
    \subsubsection*{Index der erwähnten Entitäten}%
  \fi
  \setlength{\parindent}{0pt}%
  \setlength{\parskip}{0pt plus 0.3pt}%
  \let\item\@idxitem
}{%
  \ifkorrekturansicht\clearpage\fi
}
\makeatother

\IfFileExists{\jobname-pw.ind}{\input{\jobname-pw.ind}}{}

% Quellenangabe nur in der Leseansicht
\ifkorrekturansicht\else
% Fallback-Definitionen, falls die .tex-Datei \titel etc. nicht gesetzt hat
\providecommand{\titel}{}
\providecommand{\editorInnen}{}
\providecommand{\dateiname}{\jobname}

\vspace{3cm}

\vfill

\footnotesize
\textsc{Quelle}: \titel. Herausgegeben von {\editorInnen}. In: \emph{Arthur Schnitzler: Briefwechsel mit Autorinnen und Autoren}.
 Digitale Edition, https://schnitzler-briefe.acdh.oeaw.ac.at/{\dateiname}.html (Stand \today)
\fi

\end{document}


      