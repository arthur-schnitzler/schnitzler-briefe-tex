%% latex-korrekturansicht-vorspann.tex
%% Vorspann für die Korrekturansicht.
%% Lädt die gemeinsame Datei latex-vorspann.tex mit gesetztem Schalter.

\newif\ifkorrekturansicht
\korrekturansichttrue

\input{../tex-inputs/latex-vorspann}


\section[Arthur Schnitzler an Richard Beer-Hofmann, 18. 3. 1897]{L00654 Arthur Schnitzler an Richard Beer-Hofmann, 18. 3. 1897}
\nopagebreak\mylabel{L00654v}
\rehead{ }\normalsize\beginnumbering\briefempfaengerindex{Beer-Hofmann, Richard@\textsc{Beer-Hofmann, Richard}!zzzSchnitzler, Arthur@\emph{von Arthur Schnitzler}!1897-03-181@{18. 3. 1897}|(be}
\toendnotes[C]{\smallbreak\pagebreak[2]}\Standort{YCGL, MSS 31.}
\physDesc{Postkarte, 128 Zeichen
\newline{}Handschrift: Bleistift, deutsche Kurrent
\newline{}Versand: 1) Rohrpost  2) Stempel: »\nobreak{}\oindex{I., Innere Stadt@\textbf{I., Innere Stadt}, \emph{A.ADM3}|pwk}Wien 1/1, 18 {[}3. 1897{]}, 7 30 V\nobreak{}«.  3) Stempel: »\nobreak{}\oindex{I., Innere Stadt@\textbf{I., Innere Stadt}, \emph{A.ADM3}|pwk}Wien 1/1, 18 III 97, 7 40 V\nobreak{}«. }\toendnotes[C]{\smallbreak}\pstart{}{\pb}Herrn Dr \textsc{Richard
                     Beer-Hofmann}\pend{}\pstart{}Wien\oindex{Wien@\textbf{Wien}, \emph{A.ADM2}|pw}\pend{}\pstart{}\textsc{I. Wollzeile 15\oindex{Wollzeile@\textbf{Wollzeile}, \emph{Straße (K.STR)}|pw}.}\pend{}{\bigskip}\vspace{1em}
\pstart
           \noindent{}{\pb}Raimundtheater\oindex{Raimund-Theater@\textbf{Raimund-Theater}, \emph{Theater (K.THE)}|pw}!\pend
           
\pstart
           Vergeſſen Sie nicht!{\\}2 \label{K_L00654-1v}\edtext{Sitze\pwindex{Sklavin. Schauspiel in vier Aufzuegen@\emph{Die Sklavin. Schauspiel in vier Aufzügen}|pwv}}{\lemma{\textnormal{\emph{Sitze}}}\Cendnote{\textnormal{Schnitzler besuchte die Premiere von \emph{Die Sklavin}\pwindex{Sklavin. Schauspiel in vier Aufzuegen@\emph{Die Sklavin. Schauspiel in vier Aufzügen}|pwk} (\emph{Cambridge University Library}, A 179a).}}}\label{K_L00654-1}! Mir ſchicken!\pend
           
\pstart
           \label{T_L00654-1v}\edtext{\label{K_L00654-2v}\edtext{Von mir keine Grüße}{\lemma{\textnormal{\emph{Von mir keine Grüße}}}\Cendnote{\textnormal{In der Handschrift von Beer-Hofmann\pwindex{Beer-Hofmann, Richard 1866-07-11 – 1945-09-26@\textsc{Beer-Hofmann, Richard} (1866-07-11 – 1945-09-26), \emph{Schriftsteller/Schriftstellerin}|pwk} steht mit Bleistift in lateinischer
                  Kurrentschrift auf der Karte geschrieben: »Herzliche Grüße von
                     Richard«. Die Reaktion Schnitzlers bezieht sich darauf, wobei zwei Abläufe denkbar sind: Der Gruß
                  befand sich auf der Karte, als Schnitzler
                  beschloss, sie wiederzuverwenden. Oder Beer-Hofmann\pwindex{Beer-Hofmann, Richard 1866-07-11 – 1945-09-26@\textsc{Beer-Hofmann, Richard} (1866-07-11 – 1945-09-26), \emph{Schriftsteller/Schriftstellerin}|pwk} ergänzte den Gruß, als er die gewünschten Theaterkarten
                  zusammen mit dieser Karte retournierte, woraufhin Schnitzler seine Reaktion notierte und erneut zurücksandte.}}}\label{K_L00654-2}}{\lemma{\textnormal{\emph{Von mir keine Grüße}}}\Cendnote{\textnormal{am oberen Rand auf dem Kopf}}}\label{T_L00654-1}\spacefill\mbox{Arth}\pend
           \selectlanguage{ngerman}\endnumbering\briefempfaengerindex{Beer-Hofmann, Richard@\textsc{Beer-Hofmann, Richard}!zzzSchnitzler, Arthur@\emph{von Arthur Schnitzler}!1897-03-181@{18. 3. 1897}|)be}\mylabel{L00654h}  \normalsize

\doendnotes{C}
\bigskip
\vfill

\clearpage

\footnotesize

\lohead{\textsc{register}}

% Definiere theindex-Environment komplett neu ohne reledmac
\makeatletter
\renewenvironment{theindex}{%
  \section*{\indexname}%
  \setlength{\parindent}{0pt}%
  \setlength{\parskip}{0pt plus 0.3pt}%
  \let\item\@idxitem
}{%
  \clearpage
}
\makeatother

\IfFileExists{\jobname-pw.ind}{\input{\jobname-pw.ind}}{}

\end{document}

      