%% latex-korrekturansicht-vorspann.tex
%% Vorspann für die Korrekturansicht.
%% Lädt die gemeinsame Datei latex-vorspann.tex mit gesetztem Schalter.

\newif\ifkorrekturansicht
\korrekturansichttrue

\input{../tex-inputs/latex-vorspann}


\section[ Paul Goldmann an Arthur Schnitzler, 9. 12. {[}1900{]}]{L02944 Paul Goldmann an Arthur Schnitzler, 9. 12. {[}1900{]}}
\nopagebreak\mylabel{L02944v}
\rehead{ }\normalsize\beginnumbering\briefempfaengerindex{Schnitzler, Arthur@\textsc{Schnitzler, Arthur}!zzzGoldmann, Paul@\emph{von Paul Goldmann}!1900-12-091@{9. 12. {[}1900{]}}|(be}
\toendnotes[C]{\smallbreak\pagebreak[2]}\Standort{DLA, A:Schnitzler, HS.NZ85.1.3170.}
\physDesc{Brief, 1 Blatt, 4 Seiten, 2409 Zeichen
\newline{}Handschrift: blaue Tinte, deutsche Kurrent
\newline{}Beilage: handschriftlicher Brief, 2 Blätter, 3 Seiten, schwarze Tinte,
                                 deutsche Kurrent 
\newline{}Schnitzler: 1) mit Bleistift das Jahr »900« vermerkt  2) mit rotem Buntstift drei Unterstreichungen und zwei »X«}\toendnotes[C]{\smallbreak}
\pstart
           \raggedleft{}{\pb}\textcolor{gray}{\textbf{DESSAUERSTRASSE 19}}\oindex{Dessauer Strasse@\textbf{Dessauer Straße}, \emph{Straße (K.STR)}|pw}\pend
           
\pstart
           Berlin\oindex{Berlin@\textbf{Berlin}, \emph{P.PPLC}|pw}, 9. December.\pend
           
\pstart\center{}Mein lieber Freund,\pend\vspace{0.5em}
\pstart
           Endlich geſtern konnte ich Frl. \textsc{\label{K_L02944-1v}\edtext{Glümer\pwindex{Gluemer, Marie 03.07.1867 – 16.11.1925@\textsc{Glümer, Marie} (03.07.1867 – 16.11.1925), \emph{Schauspieler/Schauspielerin}|pw}}{\lemma{\textnormal{\emph{Glümer}}}\Cendnote{\textnormal{Marie Glümer\pwindex{Gluemer, Marie 03.07.1867 – 16.11.1925@\textsc{Glümer, Marie} (03.07.1867 – 16.11.1925), \emph{Schauspieler/Schauspielerin}|pwk} war für die Uraufführung
                     von \emph{Der Schleier der Beatrice}\pwindex{Schleier der Beatrice. Schauspiel in fuenf Akten@\emph{Der Schleier der Beatrice. Schauspiel in fünf Akten}|pwk} nach Breslau\oindex{Breslau@\textbf{Breslau}, \emph{P.PPLA}|pwk} gereist.}}}\label{K_L02944-1}} ſprechen. Das ſcheint ja eine hübſche Schweinerei geweſen zu ſein, dieſe Breslau\oindex{Breslau@\textbf{Breslau}, \emph{P.PPLA}|pw}er Aufführung\pwindex{Schleier der Beatrice. Schauspiel in fuenf Akten@\emph{Der Schleier der Beatrice. Schauspiel in fünf Akten}|pwv}. Ja, Breslau\oindex{Breslau@\textbf{Breslau}, \emph{P.PPLA}|pw}!
               Man muß \label{K_L02944-2v}\edtext{in dieſer Stadt\oindex{Breslau@\textbf{Breslau}, \emph{P.PPLA}|pwv} geboren}{\lemma{\textnormal{\emph{in dieſer Stadt geboren}}}\Cendnote{\textnormal{Goldmann\pwindex{Goldmann, Paul 31.01.1865 – 25.09.1935@\textsc{Goldmann, Paul} (31.01.1865 – 25.09.1935), \emph{Schriftsteller/Schriftstellerin, Journalist/Journalistin}|pwk} meinte sich selbst.}}}\label{K_L02944-2} ſein, um
               ſie ganz würdigen zu können.\pend
           
\pstart
           Heut ſprach ich den \label{K_L02944-3v}\edtext{Direktor {\pb}\textsc{Martin\pwindex{Marton, Paul Martin @\textsc{Marton, Paul Martin}, \emph{Schriftsteller/Schriftstellerin, Theaterleiter/Theaterleiterin}|pw}}}{\lemma{\textnormal{\emph{Direktor Martin}}}\Cendnote{\textnormal{Paul Martin Marton\pwindex{Marton, Paul Martin @\textsc{Marton, Paul Martin}, \emph{Schriftsteller/Schriftstellerin, Theaterleiter/Theaterleiterin}|pwk}, Direktor der Berlin\oindex{Berlin@\textbf{Berlin}, \emph{P.PPLC}|pwk}er \emph{Secessionsbühne}\orgindex{Secessionsbuehne@Secessionsbühne|pwk} und späterer Ehemann von Marie Glümer\pwindex{Gluemer, Marie 03.07.1867 – 16.11.1925@\textsc{Glümer, Marie} (03.07.1867 – 16.11.1925), \emph{Schauspieler/Schauspielerin}|pwk}}}}\label{K_L02944-3} und habe ihm rieſig zugeredet, die \textsc{Triesch\pwindex{Triesch, Irene 13.04.1877 – 24.11.1964@\textsc{Triesch, Irene} (13.04.1877 – 24.11.1964), \emph{Schauspieler/Schauspielerin}|pw}}, die er haben kann, zu engagiren. Dann wird er die \label{K_L02944-4v}\edtext{»\textsc{Beatrice\pwindex{Schleier der Beatrice. Schauspiel in fuenf Akten@\emph{Der Schleier der Beatrice. Schauspiel in fünf Akten}|pw}}« ſpielen}{\lemma{\textnormal{\emph{»Beatrice« ſpielen}}}\Cendnote{\textnormal{Dazu kam es nicht.}}}\label{K_L02944-4},
               und es wird gut werden.\pend
           
\pstart
           Dem \label{K_L02944-5v}\edtext{Volkstheater\orgindex{Volkstheater@Volkstheater|pw}}{\lemma{\textnormal{\emph{Volkstheater}}}\Cendnote{\textnormal{Das entspricht einer Kehrtwende, vgl. Paul Goldmann an Arthur Schnitzler, 21. 6. [1900].}}}\label{K_L02944-5} ſollteſt Du das
                  Stück\pwindex{Schleier der Beatrice. Schauspiel in fuenf Akten@\emph{Der Schleier der Beatrice. Schauspiel in fünf Akten}|pwv} ruhig geben. So
               ſchlimm wie in Breslau\oindex{Breslau@\textbf{Breslau}, \emph{P.PPLA}|pw} kann es keinesfalls
               werden.\pend
           
\pstart
           Die N. Fr. Pr.\orgindex{Neue Freie Presse@Neue Freie Presse|pw} hat wieder einmal, wie Du {\pb}beifolgendem Briefe des \textsc{Dr.}{ }\textsc{Freund\pwindex{Freund, Erich 1866-08-13 – 1940@\textsc{Freund, Erich} (1866-08-13 – 1940), \emph{Kritiker/Kritikerin, Musikjournalist/Musikjournalistin}|pw}} erſehen wirſt, in ihrem Glanze gezeigt.\pend
           
\pstart
           Iſt die \strikeout{\textcolor{gray}{»}}Oreſtie\pwindex{Orestie@\emph{Orestie}|pw} im Burgtheater\orgindex{Burgtheater@Burgtheater|pw} wirklich ſo großartig, wie \textsc{Wittmann\pwindex{Wittmann, Hugo 16.10.1839 – 06.02.1923@\textsc{Wittmann, Hugo} (16.10.1839 – 06.02.1923), \emph{Schriftsteller/Schriftstellerin, Journalist/Journalistin}|pw}}{ }\label{K_L02944-6v}\edtext{behauptet\pwindex{Burgtheater. (Zum erstenmale: Die Orestie. Tragoedie in drei Stuecken. Aus dem Griechischen des Aischylos. Nach der Uebersetzung des Freiherrn Ulrich v. Wilamowitz-Moellendorff fuer die moderne Buehne bearbeitet.)@\emph{Burgtheater. (Zum erstenmale: Die Orestie. Tragödie in drei Stücken. Aus dem Griechischen des Aischylos. Nach der Übersetzung des Freiherrn Ulrich v. Wilamowitz-Moellendorff für die moderne Bühne bearbeitet.)}|pwv}}{\lemma{\textnormal{\emph{behauptet}}}\Cendnote{\textnormal{[Hugo Wittmann\pwindex{Wittmann, Hugo 16.10.1839 – 06.02.1923@\textsc{Wittmann, Hugo} (16.10.1839 – 06.02.1923), \emph{Schriftsteller/Schriftstellerin, Journalist/Journalistin}|pwk}]: \emph{Burgtheater. (Zum erstenmale: Die Orestie. Tragödie in drei
                        Stücken. Aus dem Griechischen des Aischylos. Nach der Übersetzung des
                        Freiherrn Ulrich v. Wilamowitz-Moellendorff für die moderne Bühne
                        bearbeitet.)}\pwindex{Burgtheater. (Zum erstenmale: Die Orestie. Tragoedie in drei Stuecken. Aus dem Griechischen des Aischylos. Nach der Uebersetzung des Freiherrn Ulrich v. Wilamowitz-Moellendorff fuer die moderne Buehne bearbeitet.)@\emph{Burgtheater. (Zum erstenmale: Die Orestie. Tragödie in drei Stücken. Aus dem Griechischen des Aischylos. Nach der Übersetzung des Freiherrn Ulrich v. Wilamowitz-Moellendorff für die moderne Bühne bearbeitet.)}|pwk} In: \emph{Neue Freie
                     Presse}\pwindex{Neue Freie Presse@\emph{Neue Freie Presse}|pwk}, Nr. 13.037, 8. 12. 1900,
                     Morgenblatt, S. 1–3.}}}\label{K_L02944-6}? Ich habe Mißtrauen. \strikeout{Er \textcolor{gray}{w}}{ }\textsc{Wittmann\pwindex{Wittmann, Hugo 16.10.1839 – 06.02.1923@\textsc{Wittmann, Hugo} (16.10.1839 – 06.02.1923), \emph{Schriftsteller/Schriftstellerin, Journalist/Journalistin}|pw}} iſt auch kein Kritiker, ſondern ein Mann, dem es nur darum zu thun iſt, hübſch
               über eine Sache zu ſchreiben, {\pb}wobei die Sache
               ſelbſt \introOben{}ihm\introOben{} ſehr gleichgiltig iſt.\pend
           
\pstart
           Viele treue Grüße! {\\[\baselineskip]}Dein {\\[\baselineskip]}\spacefill\mbox{Paul Goldmann.}\pend
           \leftskip=0em{}
\pstart
           \noindent{}Leſen: \label{K_L02944-7v}\edtext{Geſpräche Friedrichs des Gr.\pwindex{Friedrich II. von Preussen 24.01.1712 – 17.08.1786@\textsc{Friedrich II. von Preußen} (24.01.1712 – 17.08.1786), \emph{König/Königin}|pw} mit \textsc{Henri de Catt\pwindex{Catt, Henri de 1725-06-25 – 1795-11-23@\textsc{Catt, Henri de} (1725-06-25 – 1795-11-23), \emph{Gelehrter/Gelehrte, Sekretär/Sekretärin}|pw}}\pwindex{Gespraeche Friedrichs des Grossen mit Henri de Catt@\emph{Gespräche Friedrichs des Großen mit Henri de Catt}|pw} (Grenzboten-Sammlung\pwindex{Grenzboten-Sammlung@\emph{Grenzboten-Sammlung}|pw})}{\lemma{\textnormal{\emph{Geſpräche … (Grenzboten-Sammlung)}}}\Cendnote{\textnormal{\emph{Gespräche Friedrichs des Großen mit Henri
                           de Catt}\pwindex{Gespraeche Friedrichs des Grossen mit Henri de Catt@\emph{Gespräche Friedrichs des Großen mit Henri de Catt}|pwk}. Leipzig\oindex{Leipzig@\textbf{Leipzig}, \emph{P.PPLA3}|pwk}: \emph{Fr. Wilh. Grunow}\orgindex{Fr. Wilh. Grunow@Fr. Wilh. Grunow|pwk}{ }1885 (\emph{Grenzboten-Sammlung}\pwindex{Grenzboten-Sammlung@\emph{Grenzboten-Sammlung}|pwk} II, 8).
                  }}}\label{K_L02944-7}.\pend
           
\pstart
           \label{K_L02944-8v}\edtext{\textsc{Dr. Wilhelm Bode\pwindex{Bode, Wilhelm 30.03.1862 – 24.10.1922@\textsc{Bode, Wilhelm} (30.03.1862 – 24.10.1922)|pw}}: Goethes Lebenskunſt\pwindex{Goethes Lebenskunst@\emph{Goethes Lebenskunst}|pw}}{\lemma{\textnormal{\emph{Dr. … Lebenskunſt}}}\Cendnote{\textnormal{Wilhelm Bode\pwindex{Bode, Wilhelm 30.03.1862 – 24.10.1922@\textsc{Bode, Wilhelm} (30.03.1862 – 24.10.1922)|pwk}: \emph{Goethes Lebenskunst}\pwindex{Goethes Lebenskunst@\emph{Goethes Lebenskunst}|pwk}. Berlin\oindex{Berlin@\textbf{Berlin}, \emph{P.PPLC}|pwk}: \emph{Ernst Siegfried Mittler {\kaufmannsund} Sohn}\orgindex{Ernst Siegfried Mittler und Sohn@Ernst Siegfried Mittler {\kaufmannsund}  Sohn|pwk}{ }1901.}}}\label{K_L02944-8}.\pend
           \selectlanguage{ngerman}\vspace{1em}{\vspace{1\baselineskip}}
\pstart
           {\pb}\textcolor{gray}{\textbf{\textsc{Dr. Erich
                        Freund\pwindex{Freund, Erich 1866-08-13 – 1940@\textsc{Freund, Erich} (1866-08-13 – 1940), \emph{Kritiker/Kritikerin, Musikjournalist/Musikjournalistin}|pw}.}}}\pend
           
\pstart
           \raggedleft{}\textcolor{gray}{\textbf{Breslau V\oindex{Breslau@\textbf{Breslau}, \emph{P.PPLA}|pw},}}{ }{[}hs. :{]} 5. 12. \textcolor{gray}{\textbf{190}}0\pend
           
\pstart
           \raggedleft{}\textcolor{gray}{\textbf{Tauentzienplatz 1\textsuperscript{a.}\oindex{Tadeusz-Kościuszko-Platz@\textbf{Tadeusz-Kościuszko-Platz}, \emph{Platz (K.PLT)}|pw}}}\pend
           
\pstart{}Liebes \textsc{Paulchen}!\pend\vspace{0.5em}
\pstart
           Ganz wie ich fürchtete, iſt meiner Telegraphirerei\pwindex{Man telegraphirt uns aus Breslau…]@\emph{[Man telegraphirt uns aus Breslau…]}|pwv} für die N. fr. Pr.\orgindex{Neue Freie Presse@Neue Freie Presse|pw} für
               mich nichts als Arbeit und Ärger herausgekommen. Die Première\pwindex{Schleier der Beatrice. Schauspiel in fuenf Akten@\emph{Der Schleier der Beatrice. Schauspiel in fünf Akten}|pwv} dauerte bis 11, ich
               raſte per Wagen nach dem Amt, hielt in Eile die von Dir beſtellten ca 180 Mark hin,
               mußte drängeln, daß ich mit dem einzigen dienſtführenden Beamten, der \substVorne{}\textsuperscript{ſolche}\substDazwischen{}lange\substHinten{} Depeſchen nicht gewohnt iſt, zu Rande kam, war erſt nach 12 Uhr
               für die \textsc{Morgen Ztg\pwindex{Neue Freie Presse@\emph{Neue Freie Presse}|pwv}} frei, ſo daß dieſe am meiſten {\pb}zu kurz, ich
               aber erſt um 1 Uhr zum Nachtmahlen kam, und das Reſultat der ganzen
               Schererei war, daß ich am nächſten Tage nur ein
                  \label{K_L02944-9v}\edtext{Drittel meines Telegra{\geminationm}s}{\lemma{\textnormal{\emph{Drittel … Telegramms}}}\Cendnote{\textnormal{
                     »– Man telegraphirt uns aus \so{Breslau}\oindex{Breslau@\textbf{Breslau}, \emph{P.PPLA}|pw}: Arthur \so{Schnitzler}’s neuestes Drama ›\so{Der Schleier der Beatrice}\pwindex{Schleier der Beatrice. Schauspiel in fuenf Akten@\emph{Der Schleier der Beatrice. Schauspiel in fünf Akten}|pw}‹ kam heute nach mehrfachen Verzögerungen auf die Bühne des \so{Lobe-Theaters}\orgindex{Lobe-Theater@Lobe-Theater|pw}. Das in Vers und Prosa geschriebene Werk\pwindex{Schleier der Beatrice. Schauspiel in fuenf Akten@\emph{Der Schleier der Beatrice. Schauspiel in fünf Akten}|pwv} ist ein farbenprangendes Renaissance-Gemälde von
                     bizarrer Kühnheit. Seine Schönheiten breiten sich wie ein schimmernder Mantel
                     über das Gerüst der Handlung. Das Publicum nahm die drei ersten Acte\pwindex{Schleier der Beatrice. Schauspiel in fuenf Akten@\emph{Der Schleier der Beatrice. Schauspiel in fünf Akten}|pwv} mit Enthusiasmus, die beiden
                     letzten aber mit immer stärkerem Widerspruch auf.« [Erich Freund\pwindex{Freund, Erich 1866-08-13 – 1940@\textsc{Freund, Erich} (1866-08-13 – 1940), \emph{Kritiker/Kritikerin, Musikjournalist/Musikjournalistin}|pwk}]: \emph{[Man telegraphirt uns aus Breslau]}\pwindex{Man telegraphirt uns aus Breslau…]@\emph{[Man telegraphirt uns aus Breslau…]}|pwk}. In: \emph{Neue Freie Presse}\pwindex{Neue Freie Presse@\emph{Neue Freie Presse}|pwk}, Nr. 13.031,
                        2. 12. 1900, Morgenblatt, S. 10. }}}\label{K_L02944-9}, vor allem kein
               Wort über die erbärmliche, ſaumäßige, empörende Aufführung in der N. fr. Pr.\pwindex{Neue Freie Presse@\emph{Neue Freie Presse}|pw}{ }finde\pwindex{Man telegraphirt uns aus Breslau…]@\emph{[Man telegraphirt uns aus Breslau…]}|pwv}. Wahrſcheinlich iſt die
                  \label{K_L02944-10v}\edtext{Freundſchaft für Herrn \textsc{Dr Löwe\pwindex{Loewe, Theodor 1855-01-01 – 1935@\textsc{Loewe, Theodor} (1855-01-01 – 1935), \emph{Theaterleiter/Theaterleiterin}|pw}}}{\lemma{\textnormal{\emph{Freundſchaft … Löwe}}}\Cendnote{\textnormal{Siehe Paul Goldmann an Arthur Schnitzler, 14. 10. [1900].
               }}}\label{K_L02944-10} dort ſo ſtark, daß ſie alle anderen Rückſichten tödtet, ſelbſt die auf
               Schnitzler, der am ſchwerſten durch dieſe lächerliche Vorſtellung geſchädigt wurde
                  \introOben{}und mich darum bat, darauf beſonders hinzuweiſen\introOben{}. Ich
               habe ſoeben an die {\pb}dortige Redaktion\orgindex{Neue Freie Presse@Neue Freie Presse|pwv} geſchrieben und um Erklärung
               erſucht. Auf ein Honorar verzichte ich gern. Bemerken will ich doch, daß ich nach
               Deiner Anweiſung rechtzeitig um Beihalten des Platzes in der So{\geminationn}tags-Nu{\geminationm}er\pwindex{Neue Freie Presse@\emph{Neue Freie Presse}|pwv} erſucht hatte. Sollten Dich die \introOben{}hieſigen\introOben{} Kritiken über Stück\pwindex{Schleier der Beatrice. Schauspiel in fuenf Akten@\emph{Der Schleier der Beatrice. Schauspiel in fünf Akten}|pwv} od Aufführung intereſſiren, ſo ſende ich ſie Dir. Am
                  Dienſtag brachte der \textsc{B. Börsen Cour.\pwindex{Berliner Boersen-Courier@\emph{Berliner Börsen-Courier}|pw}} eine längere \label{K_L02944-11v}\edtext{Beſprechung\pwindex{Vor den Coulissen [Schleier der Beatrice]@\emph{Vor den Coulissen [Schleier der Beatrice]}|pwv}}{\lemma{\textnormal{\emph{Beſprechung}}}\Cendnote{\textnormal{F.\pwindex{Freund, Erich 1866-08-13 – 1940@\textsc{Freund, Erich} (1866-08-13 – 1940), \emph{Kritiker/Kritikerin, Musikjournalist/Musikjournalistin}|pwk} [ = Erich Freund\pwindex{Freund, Erich 1866-08-13 – 1940@\textsc{Freund, Erich} (1866-08-13 – 1940), \emph{Kritiker/Kritikerin, Musikjournalist/Musikjournalistin}|pwk}]: \emph{Vor den
                     Coulissen}\pwindex{Vor den Coulissen [Schleier der Beatrice]@\emph{Vor den Coulissen [Schleier der Beatrice]}|pwk}. In: \emph{Berliner
                        Börsen-Courier}\pwindex{Berliner Boersen-Courier@\emph{Berliner Börsen-Courier}|pwk}, Jg. 33, Nr. 566, 4. 12. 1900, Morgen-Ausgabe, 1. Beilage, S. 4.}}}\label{K_L02944-11} von
               mir.\pend
           
\pstart
           Es grüßt Dich herzlichſt {\\[\baselineskip]}Dein getreuer {\\[\baselineskip]}\spacefill\mbox{Freund\pwindex{Freund, Erich 1866-08-13 – 1940@\textsc{Freund, Erich} (1866-08-13 – 1940), \emph{Kritiker/Kritikerin, Musikjournalist/Musikjournalistin}|pw}}\pend
           \leftskip=0em{}\selectlanguage{ngerman}\endnumbering\briefempfaengerindex{Schnitzler, Arthur@\textsc{Schnitzler, Arthur}!zzzGoldmann, Paul@\emph{von Paul Goldmann}!1900-12-091@{9. 12. {[}1900{]}}|)be}\mylabel{L02944h}  \normalsize

\doendnotes{C}
\bigskip
\vfill

\clearpage

\footnotesize

\lohead{\textsc{register}}

% Definiere theindex-Environment komplett neu ohne reledmac
\makeatletter
\renewenvironment{theindex}{%
  \section*{\indexname}%
  \setlength{\parindent}{0pt}%
  \setlength{\parskip}{0pt plus 0.3pt}%
  \let\item\@idxitem
}{%
  \clearpage
}
\makeatother

\IfFileExists{\jobname-pw.ind}{\input{\jobname-pw.ind}}{}

\end{document}

      