%% latex-leseansicht-vorspann.tex
%% Vorspann für die Leseansicht.
%% Lädt die gemeinsame Datei latex-vorspann.tex mit nicht gesetztem Schalter.

\newif\ifkorrekturansicht
\korrekturansichtfalse

\input{../tex-inputs/latex-vorspann}


\section[ Paul Goldmann an Arthur Schnitzler, 9. 12. {[}1900{]}]{L02944 Paul Goldmann an Arthur Schnitzler,  9. 12. [1900]}
\nopagebreak\mylabel{L02944v}
\rehead{ }\normalsize\beginnumbering\briefempfaengerindex{Schnitzler, Arthur@\textsc{Schnitzler, Arthur}!zzzGoldmann, Paul@\emph{von Paul Goldmann}!1900-12-091@{9. 12. [1900]}|(be}
\toendnotes[C]{\smallbreak\pagebreak[2]}
\correspDesc{Versand  durch Paul Goldmann am 9. 12. [1900] in Berlin
\newline{}Erhalt  durch Arthur Schnitzler im Zeitraum [10. 12. 1900 – 14. 12. 1900?] in Wien}\toendnotes[C]{\smallbreak}
\Standort{DLA, A:Schnitzler, HS.NZ85.1.3170.}
\physDesc{Brief, 1 Blatt, 4 Seiten, 2410 Zeichen
\newline{}Handschrift: blaue Tinte, deutsche Kurrent
\newline{}Beilage: handschriftlicher Brief, 2 Blätter, 3 Seiten, schwarze Tinte,
                                 deutsche Kurrent 
\newline{}Schnitzler: 1) mit Bleistift das Jahr »900« vermerkt  2) mit rotem Buntstift drei Unterstreichungen und zwei »X«}\toendnotes[C]{\smallbreak}
\pstart
           \raggedleft{}{\pb}\textcolor{gray}{\textbf{DESSAUERSTRASSE 19}}\oindex{Dessauer Straße@\textbf{Dessauer Straße}, \emph{Straße}|pw}\pend
           
\pstart
           Berlin\oindex{Berlin@\textbf{Berlin}, \emph{Hauptstadt}|pw}, 9. December.\pend
           
\pstart\center{}Mein lieber Freund,\pend\vspace{0.5em}
\pstart
           Endlich geſtern konnte ich Frl. \textsc{\label{K_L02944-1v}\edtext{Glümer\pwindex{Glümer, Marie 3.\,7.\,1867 Wien – 16.\,11.\,1925 München@\textsc{Glümer, Marie} (3.\,7.\,1867 Wien – 16.\,11.\,1925 München), \emph{Schauspielerin}|pw}}{\lemma{\textnormal{\emph{Glümer}}}\Cendnote{\textnormal{Marie Glümer\pwindex{Glümer, Marie 3.\,7.\,1867 Wien – 16.\,11.\,1925 München@\textsc{Glümer, Marie} (3.\,7.\,1867 Wien – 16.\,11.\,1925 München), \emph{Schauspielerin}|pwk} war für die Uraufführung
                        von \emph{Der Schleier der Beatrice}\pwindex{Schnitzler, Arthur 15.\,5.\,1862 Wien – 21.\,10.\,1931 ebd.@\textsc{Schnitzler, Arthur} (15.\,5.\,1862 Wien – 21.\,10.\,1931 ebd.), \emph{Schriftsteller, Mediziner}!Schleier der Beatrice. Schauspiel in fünf Akten@\strich\emph{Der Schleier der Beatrice. Schauspiel in fünf Akten}|pwk}\eventindex{Lobe-Theater@\textbf{Lobe-Theater}!Uraufführung von Der Schleier der Beatrice, 1.12.1900@Uraufführung von Der Schleier der Beatrice, 1.12.1900|pwk} nach Breslau\oindex{Breslau@\textbf{Breslau}|pwk} gereist.}}}\label{K_L02944-1}}{ }ſprechen. Das{ }ſcheint ja eine hübſche Schweinerei geweſen zu{ }ſein, dieſe Breslau\oindex{Breslau@\textbf{Breslau}|pw}er Aufführung\eventindex{Lobe-Theater@\textbf{Lobe-Theater}!Uraufführung von Der Schleier der Beatrice, 1.12.1900@Uraufführung von Der Schleier der Beatrice, 1.12.1900|pwv}\pwindex{Schnitzler, Arthur 15.\,5.\,1862 Wien – 21.\,10.\,1931 ebd.@\textsc{Schnitzler, Arthur} (15.\,5.\,1862 Wien – 21.\,10.\,1931 ebd.), \emph{Schriftsteller, Mediziner}!Schleier der Beatrice. Schauspiel in fünf Akten@\strich\emph{Der Schleier der Beatrice. Schauspiel in fünf Akten}|pwv}. Ja, Breslau\oindex{Breslau@\textbf{Breslau}|pw}!
               Man muß \label{K_L02944-2v}\edtext{in dieſer Stadt\oindex{Breslau@\textbf{Breslau}|pwv} geboren}{\lemma{\textnormal{\emph{in dieser Stadt geboren}}}\Cendnote{\textnormal{Goldmann\pwindex{Goldmann, Paul 31.\,1.\,1865 Breslau – 25.\,9.\,1935 Wien@\textsc{Goldmann, Paul} (31.\,1.\,1865 Breslau – 25.\,9.\,1935 Wien), \emph{Schriftsteller, Journalist}|pwk} meinte sich selbst.}}}\label{K_L02944-2}{ }ſein, um{ }ſie ganz würdigen zu können.\pend
           
\pstart
           Heut{ }ſprach ich den \label{K_L02944-3v}\edtext{Direktor {\pb}\textsc{Martin\pwindex{Marton, Paul Martin @\textsc{Marton, Paul Martin}, \emph{Schriftsteller, Theaterleiter}|pw}}}{\lemma{\textnormal{\emph{Direktor Martin}}}\Cendnote{\textnormal{Paul Martin Marton\pwindex{Marton, Paul Martin @\textsc{Marton, Paul Martin}, \emph{Schriftsteller, Theaterleiter}|pwk}, Direktor der Berlin\oindex{Berlin@\textbf{Berlin}, \emph{Hauptstadt}|pwk}er \emph{Secessionsbühne}\orgindex{Secessionsbühne@Secessionsbühne|pwk} und späterer Ehemann von Marie Glümer\pwindex{Glümer, Marie 3.\,7.\,1867 Wien – 16.\,11.\,1925 München@\textsc{Glümer, Marie} (3.\,7.\,1867 Wien – 16.\,11.\,1925 München), \emph{Schauspielerin}|pwk}}}}\label{K_L02944-3} und habe ihm rieſig zugeredet, die \textsc{Triesch\pwindex{Triesch, Irene 13.\,4.\,1877 Wien – 24.\,11.\,1964 Basel@\textsc{Triesch, Irene} (13.\,4.\,1877 Wien – 24.\,11.\,1964 Basel), \emph{Schauspielerin}|pw}}, die er haben kann, zu engagiren. Dann wird er die \label{K_L02944-4v}\edtext{»\textsc{Beatrice\pwindex{Schnitzler, Arthur 15.\,5.\,1862 Wien – 21.\,10.\,1931 ebd.@\textsc{Schnitzler, Arthur} (15.\,5.\,1862 Wien – 21.\,10.\,1931 ebd.), \emph{Schriftsteller, Mediziner}!Schleier der Beatrice. Schauspiel in fünf Akten@\strich\emph{Der Schleier der Beatrice. Schauspiel in fünf Akten}|pw}}«{ }ſpielen}{\lemma{\textnormal{\emph{»Beatrice« spielen}}}\Cendnote{\textnormal{Dazu kam es nicht.}}}\label{K_L02944-4},
               und es wird gut werden.\pend
           
\pstart
           Dem \label{K_L02944-5v}\edtext{Volkstheater\orgindex{Volkstheater@Volkstheater|pw}}{\lemma{\textnormal{\emph{Volkstheater}}}\Cendnote{\textnormal{Das entspricht einer Kehrtwende, vgl. XXXX Auszeichnungsfehler: Dokument L02921 nicht gefunden.}}}\label{K_L02944-5}{ }ſollteſt Du das
                  Stück\pwindex{Schnitzler, Arthur 15.\,5.\,1862 Wien – 21.\,10.\,1931 ebd.@\textsc{Schnitzler, Arthur} (15.\,5.\,1862 Wien – 21.\,10.\,1931 ebd.), \emph{Schriftsteller, Mediziner}!Schleier der Beatrice. Schauspiel in fünf Akten@\strich\emph{Der Schleier der Beatrice. Schauspiel in fünf Akten}|pwv} ruhig geben. So{ }ſchlimm wie in Breslau\oindex{Breslau@\textbf{Breslau}|pw} kann es keinesfalls
               werden.\pend
           
\pstart
           Die N. Fr. Pr.\orgindex{Neue Freie Presse@Neue Freie Presse|pw} hat wieder einmal, wie Du {\pb}beifolgendem Briefe des \textsc{Dr.}{ }\textsc{Freund\pwindex{Freund, Erich 13.\,8.\,1866 Breslau – 1940 Berlin@\textsc{Freund, Erich} (13.\,8.\,1866 Breslau – 1940 Berlin), \emph{Kritiker, Musikjournalist}|pw}} erſehen wirſt, in ihrem Glanze gezeigt.\pend
           
\pstart
           Iſt die \strikeout{\textcolor{gray}{»}}{ }Oreſtie\eventindex{Premiere von Orestie, 06.12.1900@Premiere von Orestie, 06.12.1900|pwv}\pwindex{\textcolor{red}{\textsuperscript{XXXX indx1}}!Orestie@\strich\emph{Orestie}|pw} im Burgtheater\orgindex{Burgtheater@Burgtheater|pw} wirklich{ }ſo großartig, wie \textsc{Wittmann\pwindex{Wittmann, Hugo 16.\,10.\,1839 Ulm – 6.\,2.\,1923 Wien@\textsc{Wittmann, Hugo} (16.\,10.\,1839 Ulm – 6.\,2.\,1923 Wien), \emph{Schriftsteller, Journalist}|pw}}{ }\label{K_L02944-6v}\edtext{behauptet\pwindex{Burgtheater. (Zum erstenmale: Die Orestie. Tragödie in drei Stücken. Aus dem Griechischen des Aischylos. Nach der Übersetzung des Freiherrn Ulrich v. Wilamowitz-Moellendorff für die moderne Bühne bearbeitet.)@\emph{Burgtheater. (Zum erstenmale: Die Orestie. Tragödie in drei Stücken. Aus dem Griechischen des Aischylos. Nach der Übersetzung des Freiherrn Ulrich v. Wilamowitz-Moellendorff für die moderne Bühne bearbeitet.)}|pwv}}{\lemma{\textnormal{\emph{behauptet}}}\Cendnote{\textnormal{[Hugo Wittmann\pwindex{Wittmann, Hugo 16.\,10.\,1839 Ulm – 6.\,2.\,1923 Wien@\textsc{Wittmann, Hugo} (16.\,10.\,1839 Ulm – 6.\,2.\,1923 Wien), \emph{Schriftsteller, Journalist}|pwk}]: \emph{Burgtheater. (Zum erstenmale: Die Orestie. Tragödie in drei
                        Stücken. Aus dem Griechischen des Aischylos. Nach der Übersetzung des
                        Freiherrn Ulrich v. Wilamowitz-Moellendorff für die moderne Bühne
                        bearbeitet.)}\pwindex{Burgtheater. (Zum erstenmale: Die Orestie. Tragödie in drei Stücken. Aus dem Griechischen des Aischylos. Nach der Übersetzung des Freiherrn Ulrich v. Wilamowitz-Moellendorff für die moderne Bühne bearbeitet.)@\emph{Burgtheater. (Zum erstenmale: Die Orestie. Tragödie in drei Stücken. Aus dem Griechischen des Aischylos. Nach der Übersetzung des Freiherrn Ulrich v. Wilamowitz-Moellendorff für die moderne Bühne bearbeitet.)}|pwk} In: \emph{Neue Freie
                     Presse}\pwindex{Neue Freie Presse@\emph{Neue Freie Presse}|pwk}, Nr. 13.037, 8. 12. 1900,
                     Morgenblatt, S. 1–3.}}}\label{K_L02944-6}? Ich habe Mißtrauen. \strikeout{Er \textcolor{gray}{w}}{ }\textsc{Wittmann\pwindex{Wittmann, Hugo 16.\,10.\,1839 Ulm – 6.\,2.\,1923 Wien@\textsc{Wittmann, Hugo} (16.\,10.\,1839 Ulm – 6.\,2.\,1923 Wien), \emph{Schriftsteller, Journalist}|pw}} iſt auch kein Kritiker,{ }ſondern ein Mann, dem es nur darum zu thun iſt, hübſch
               über eine Sache zu{ }ſchreiben, {\pb}wobei die Sache{ }ſelbſt \introOben{}ihm\introOben{}{ }ſehr gleichgiltig iſt.\pend
           
\pstart
           Viele treue Grüße! {\\[\baselineskip]}Dein {\\[\baselineskip]}\spacefill\mbox{Paul Goldmann.}\pend
           \leftskip=0em{}
\pstart
           \noindent{}Leſen: \label{K_L02944-7v}\edtext{Geſpräche Friedrichs des Gr.\pwindex{Friedrich II. von Preußen 24.\,1.\,1712 Berlin – 17.\,8.\,1786 Potsdam@\textsc{Friedrich II. von Preußen} (24.\,1.\,1712 Berlin – 17.\,8.\,1786 Potsdam), \emph{König}|pw} mit \textsc{Henri de Catt\pwindex{Catt, Henri de 25.\,6.\,1725 Morges – 23.\,11.\,1795 Potsdam@\textsc{Catt, Henri de} (25.\,6.\,1725 Morges – 23.\,11.\,1795 Potsdam), \emph{Gelehrter, Sekretär}|pw}}\pwindex{Gespräche Friedrichs des Großen mit Henri de Catt@\emph{Gespräche Friedrichs des Großen mit Henri de Catt}|pw} (Grenzboten-Sammlung\pwindex{Grenzboten-Sammlung@\emph{Grenzboten-Sammlung}|pw})}{\lemma{\textnormal{\emph{Gespräche … (Grenzboten-Sammlung)}}}\Cendnote{\textnormal{\emph{Gespräche Friedrichs des Großen mit Henri
                           de Catt}\pwindex{Gespräche Friedrichs des Großen mit Henri de Catt@\emph{Gespräche Friedrichs des Großen mit Henri de Catt}|pwk}. Leipzig\oindex{Leipzig@\textbf{Leipzig}, \emph{Hauptstadt}|pwk}: \emph{Fr. Wilh. Grunow}\orgindex{Fr. Wilh. Grunow@Fr. Wilh. Grunow|pwk}{ }1885 (\emph{Grenzboten-Sammlung}\pwindex{Grenzboten-Sammlung@\emph{Grenzboten-Sammlung}|pwk} II, 8).
                  }}}\label{K_L02944-7}.\pend
           
\pstart
           \label{K_L02944-8v}\edtext{\textsc{Dr. Wilhelm Bode\pwindex{Bode, Wilhelm 30.\,3.\,1862 Hornhausen – 24.\,10.\,1922 Weimar@\textsc{Bode, Wilhelm} (30.\,3.\,1862 Hornhausen – 24.\,10.\,1922 Weimar)|pw}}: Goethes Lebenskunſt\pwindex{Bode, Wilhelm 30.\,3.\,1862 Hornhausen – 24.\,10.\,1922 Weimar@\textsc{Bode, Wilhelm} (30.\,3.\,1862 Hornhausen – 24.\,10.\,1922 Weimar)!Goethes Lebenskunst@\strich\emph{Goethes Lebenskunst}|pw}}{\lemma{\textnormal{\emph{Dr. … Lebenskunst}}}\Cendnote{\textnormal{Wilhelm Bode\pwindex{Bode, Wilhelm 30.\,3.\,1862 Hornhausen – 24.\,10.\,1922 Weimar@\textsc{Bode, Wilhelm} (30.\,3.\,1862 Hornhausen – 24.\,10.\,1922 Weimar)|pwk}: \emph{Goethes Lebenskunst}\pwindex{Bode, Wilhelm 30.\,3.\,1862 Hornhausen – 24.\,10.\,1922 Weimar@\textsc{Bode, Wilhelm} (30.\,3.\,1862 Hornhausen – 24.\,10.\,1922 Weimar)!Goethes Lebenskunst@\strich\emph{Goethes Lebenskunst}|pwk}. Berlin\oindex{Berlin@\textbf{Berlin}, \emph{Hauptstadt}|pwk}: \emph{Ernst Siegfried Mittler {\kaufmannsund} Sohn}\orgindex{Ernst Siegfried Mittler und Sohn@Ernst Siegfried Mittler {\kaufmannsund}  Sohn|pwk}{ }1901.}}}\label{K_L02944-8}.\pend
           \selectlanguage{ngerman}\vspace{1em}{\vspace{1\baselineskip}}
\pstart
           {\pb}\textcolor{gray}{\textbf{\textsc{Dr. Erich
                        Freund\pwindex{Freund, Erich 13.\,8.\,1866 Breslau – 1940 Berlin@\textsc{Freund, Erich} (13.\,8.\,1866 Breslau – 1940 Berlin), \emph{Kritiker, Musikjournalist}|pw}.}}}\pend
           
\pstart
           \raggedleft{}\textcolor{gray}{\textbf{Breslau V\oindex{Breslau@\textbf{Breslau}|pw},}}{ }{[}hs. Freund:{]} 5. 12. \textcolor{gray}{\textbf{190}}0\pend
           
\pstart
           \raggedleft{}\textcolor{gray}{\textbf{Tauentzienplatz 1\textsuperscript{a.}\oindex{Tadeusz-Kościuszko-Platz@\textbf{Tadeusz-Kościuszko-Platz}, \emph{Platz}|pw}}}\pend
           
\pstart{}Liebes \textsc{Paulchen}!\pend\vspace{0.5em}
\pstart
           Ganz wie ich fürchtete, iſt meiner Telegraphirerei\pwindex{Man telegraphirt uns aus Breslau…]@\emph{[Man telegraphirt uns aus Breslau…]}|pwv} für die N. fr. Pr.\orgindex{Neue Freie Presse@Neue Freie Presse|pw} für
               mich nichts als Arbeit und Ärger herausgekommen. Die Première\eventindex{Lobe-Theater@\textbf{Lobe-Theater}!Uraufführung von Der Schleier der Beatrice, 1.12.1900@Uraufführung von Der Schleier der Beatrice, 1.12.1900|pwv}\pwindex{Schnitzler, Arthur 15.\,5.\,1862 Wien – 21.\,10.\,1931 ebd.@\textsc{Schnitzler, Arthur} (15.\,5.\,1862 Wien – 21.\,10.\,1931 ebd.), \emph{Schriftsteller, Mediziner}!Schleier der Beatrice. Schauspiel in fünf Akten@\strich\emph{Der Schleier der Beatrice. Schauspiel in fünf Akten}|pwv} dauerte bis 11, ich
               raſte per Wagen nach dem Amt, hielt in Eile die von Dir beſtellten ca 180 Mark hin,
               mußte drängeln, daß ich mit dem einzigen dienſtführenden Beamten, der \substVorne{}\textsuperscript{ſolche}\substDazwischen{}lange\substHinten{} Depeſchen nicht gewohnt iſt, zu Rande kam, war erſt nach 12 Uhr
               für die \textsc{Morgen Ztg\pwindex{Neue Freie Presse@\emph{Neue Freie Presse}|pwv}} frei,{ }ſo daß dieſe am meiſten {\pb}zu kurz, ich
               aber erſt um 1 Uhr zum Nachtmahlen kam, und das Reſultat der ganzen
               Schererei war, daß ich am nächſten Tage nur ein
                  \label{K_L02944-9v}\edtext{Drittel meines Telegra{\geminationm}s}{\lemma{\textnormal{\emph{Drittel … Telegramms}}}\Cendnote{\textnormal{
                     »– Man telegraphirt uns aus \so{Breslau}\oindex{Breslau@\textbf{Breslau}|pw}: Arthur \so{Schnitzler}’s neuestes Drama ›\so{Der Schleier der Beatrice}\pwindex{Schnitzler, Arthur 15.\,5.\,1862 Wien – 21.\,10.\,1931 ebd.@\textsc{Schnitzler, Arthur} (15.\,5.\,1862 Wien – 21.\,10.\,1931 ebd.), \emph{Schriftsteller, Mediziner}!Schleier der Beatrice. Schauspiel in fünf Akten@\strich\emph{Der Schleier der Beatrice. Schauspiel in fünf Akten}|pw}‹ kam heute nach mehrfachen Verzögerungen auf die Bühne des \so{Lobe-Theaters}\orgindex{Lobe-Theater@Lobe-Theater|pw}. Das in Vers und Prosa geschriebene Werk\pwindex{Schnitzler, Arthur 15.\,5.\,1862 Wien – 21.\,10.\,1931 ebd.@\textsc{Schnitzler, Arthur} (15.\,5.\,1862 Wien – 21.\,10.\,1931 ebd.), \emph{Schriftsteller, Mediziner}!Schleier der Beatrice. Schauspiel in fünf Akten@\strich\emph{Der Schleier der Beatrice. Schauspiel in fünf Akten}|pwv} ist ein farbenprangendes Renaissance-Gemälde von
                     bizarrer Kühnheit. Seine Schönheiten breiten sich wie ein schimmernder Mantel
                     über das Gerüst der Handlung. Das Publicum nahm die drei ersten Acte\pwindex{Schnitzler, Arthur 15.\,5.\,1862 Wien – 21.\,10.\,1931 ebd.@\textsc{Schnitzler, Arthur} (15.\,5.\,1862 Wien – 21.\,10.\,1931 ebd.), \emph{Schriftsteller, Mediziner}!Schleier der Beatrice. Schauspiel in fünf Akten@\strich\emph{Der Schleier der Beatrice. Schauspiel in fünf Akten}|pwv} mit Enthusiasmus, die beiden
                     letzten aber mit immer stärkerem Widerspruch auf.« [Erich Freund\pwindex{Freund, Erich 13.\,8.\,1866 Breslau – 1940 Berlin@\textsc{Freund, Erich} (13.\,8.\,1866 Breslau – 1940 Berlin), \emph{Kritiker, Musikjournalist}|pwk}]: \emph{[Man telegraphirt uns aus Breslau]}\pwindex{Man telegraphirt uns aus Breslau…]@\emph{[Man telegraphirt uns aus Breslau…]}|pwk}. In: \emph{Neue Freie Presse}\pwindex{Neue Freie Presse@\emph{Neue Freie Presse}|pwk}, Nr. 13.031,
                        2. 12. 1900, Morgenblatt, S. 10. }}}\label{K_L02944-9}, vor allem kein
               Wort über die erbärmliche,{ }ſaumäßige, empörende Aufführung\eventindex{Lobe-Theater@\textbf{Lobe-Theater}!Uraufführung von Der Schleier der Beatrice, 1.12.1900@Uraufführung von Der Schleier der Beatrice, 1.12.1900|pwv} in der N. fr. Pr.\pwindex{Neue Freie Presse@\emph{Neue Freie Presse}|pw}{ }finde\pwindex{Man telegraphirt uns aus Breslau…]@\emph{[Man telegraphirt uns aus Breslau…]}|pwv}. Wahrſcheinlich iſt die
                  \label{K_L02944-10v}\edtext{Freundſchaft für Herrn \textsc{Dr Löwe\pwindex{Loewe, Theodor 1.\,1.\,1855 Wien – 1935 Breslau@\textsc{Loewe, Theodor} (1.\,1.\,1855 Wien – 1935 Breslau), \emph{Theaterleiter}|pw}}}{\lemma{\textnormal{\emph{Freundschaft … Löwe}}}\Cendnote{\textnormal{Siehe XXXX Auszeichnungsfehler: Dokument L02936 nicht gefunden.
               }}}\label{K_L02944-10} dort{ }ſo{ }ſtark, daß{ }ſie alle anderen Rückſichten tödtet,{ }ſelbſt die auf
               Schnitzler, der am{ }ſchwerſten durch dieſe lächerliche Vorſtellung geſchädigt wurde
                  \introOben{}und mich darum bat, darauf beſonders hinzuweiſen\introOben{}. Ich
               habe{ }ſoeben an die {\pb}dortige Redaktion\orgindex{Neue Freie Presse@Neue Freie Presse|pwv} geſchrieben und um Erklärung
               erſucht. Auf ein Honorar verzichte ich gern. Bemerken will ich doch, daß ich nach
               Deiner Anweiſung rechtzeitig um Beihalten des Platzes in der So{\geminationn}tags-Nu{\geminationm}er\pwindex{Neue Freie Presse@\emph{Neue Freie Presse}|pwv} erſucht hatte. Sollten Dich die \introOben{}hieſigen\introOben{} Kritiken über Stück\pwindex{Schnitzler, Arthur 15.\,5.\,1862 Wien – 21.\,10.\,1931 ebd.@\textsc{Schnitzler, Arthur} (15.\,5.\,1862 Wien – 21.\,10.\,1931 ebd.), \emph{Schriftsteller, Mediziner}!Schleier der Beatrice. Schauspiel in fünf Akten@\strich\emph{Der Schleier der Beatrice. Schauspiel in fünf Akten}|pwv} od Aufführung\eventindex{Lobe-Theater@\textbf{Lobe-Theater}!Uraufführung von Der Schleier der Beatrice, 1.12.1900@Uraufführung von Der Schleier der Beatrice, 1.12.1900|pwv} intereſſiren,{ }ſo{ }ſende ich{ }ſie Dir. Am
                  Dienſtag brachte der \textsc{B. Börsen Cour.\pwindex{Berliner Börsen-Courier@\emph{Berliner Börsen-Courier}|pw}} eine längere \label{K_L02944-11v}\edtext{Beſprechung\pwindex{Freund, Erich 13.\,8.\,1866 Breslau – 1940 Berlin@\textsc{Freund, Erich} (13.\,8.\,1866 Breslau – 1940 Berlin), \emph{Kritiker, Musikjournalist}!Vor den Coulissen [Schleier der Beatrice]@\strich\emph{Vor den Coulissen [Schleier der Beatrice]}|pwv}}{\lemma{\textnormal{\emph{Besprechung}}}\Cendnote{\textnormal{F.\pwindex{Freund, Erich 13.\,8.\,1866 Breslau – 1940 Berlin@\textsc{Freund, Erich} (13.\,8.\,1866 Breslau – 1940 Berlin), \emph{Kritiker, Musikjournalist}|pwk} [ = Erich Freund\pwindex{Freund, Erich 13.\,8.\,1866 Breslau – 1940 Berlin@\textsc{Freund, Erich} (13.\,8.\,1866 Breslau – 1940 Berlin), \emph{Kritiker, Musikjournalist}|pwk}]: \emph{Vor den
                     Coulissen}\pwindex{Freund, Erich 13.\,8.\,1866 Breslau – 1940 Berlin@\textsc{Freund, Erich} (13.\,8.\,1866 Breslau – 1940 Berlin), \emph{Kritiker, Musikjournalist}!Vor den Coulissen [Schleier der Beatrice]@\strich\emph{Vor den Coulissen [Schleier der Beatrice]}|pwk}. In: \emph{Berliner
                        Börsen-Courier}\pwindex{Berliner Börsen-Courier@\emph{Berliner Börsen-Courier}|pwk}, Jg. 33, Nr. 566, 4. 12. 1900, Morgen-Ausgabe, 1. Beilage, S. 4.}}}\label{K_L02944-11} von
               mir.\pend
           
\pstart
           Es grüßt Dich herzlichſt {\\[\baselineskip]}Dein getreuer {\\[\baselineskip]}\spacefill\mbox{Freund\pwindex{Freund, Erich 13.\,8.\,1866 Breslau – 1940 Berlin@\textsc{Freund, Erich} (13.\,8.\,1866 Breslau – 1940 Berlin), \emph{Kritiker, Musikjournalist}|pw}}\pend
           \leftskip=0em{}\selectlanguage{ngerman}\endnumbering\briefempfaengerindex{Schnitzler, Arthur@\textsc{Schnitzler, Arthur}!zzzGoldmann, Paul@\emph{von Paul Goldmann}!1900-12-091@{9. 12. [1900]}|)be}\mylabel{L02944h}  \newcommand{\dateiname}{L02944}\newcommand{\titel}{Paul Goldmann an Arthur Schnitzler, 9. 12. [1900]}\newcommand{\editorInnen}{Martin Anton Müller und Laura Untner}%% latex-leseansicht-abspann.tex
%% Abspann für die Leseansicht.
%% Der Schalter \ifkorrekturansicht ist bereits durch den Vorspann gesetzt.

%% latex-abspann.tex
%% Gemeinsamer Abspann für Korrekturansicht und Leseansicht.
%% Setzt den Schalter \ifkorrekturansicht voraus (gesetzt in den
%% einbindenden Dateien latex-korrekturansicht-abspann.tex bzw.
%% latex-leseansicht-abspann.tex).
%% ---------------------------------------------------------------

\normalsize

% Das esempio-Environment wird nur in der Leseansicht benötigt
\ifkorrekturansicht\else
\newenvironment{esempio}[3]%
{
    \vspace{1.5ex}
    \rlap{\underline{#1}}
    \par
    \setlength{\parindent}{0cm}
    \nopagebreak
    \leftskip=#2cm
    \rightskip=#3cm
}
{
    \par
}
\fi

\doendnotes{C}
\bigskip
\vfill

\clearpage

\footnotesize

\ifkorrekturansicht
  \lohead{\textsc{register}}
\fi

% theindex-Environment neu definieren ohne reledmac
\makeatletter
\renewenvironment{theindex}{%
  \ifkorrekturansicht
    \section*{\indexname}%
  \else
    \subsubsection*{Index der erwähnten Entitäten}%
  \fi
  \setlength{\parindent}{0pt}%
  \setlength{\parskip}{0pt plus 0.3pt}%
  \let\item\@idxitem
}{%
  \ifkorrekturansicht\clearpage\fi
}
\makeatother

\IfFileExists{\jobname-pw.ind}{\input{\jobname-pw.ind}}{}

% Quellenangabe nur in der Leseansicht
\ifkorrekturansicht\else
% Fallback-Definitionen, falls die .tex-Datei \titel etc. nicht gesetzt hat
\providecommand{\titel}{}
\providecommand{\editorInnen}{}
\providecommand{\dateiname}{\jobname}

\vspace{3cm}

\vfill

\footnotesize
\textsc{Quelle}: \titel. Herausgegeben von {\editorInnen}. In: \emph{Arthur Schnitzler: Briefwechsel mit Autorinnen und Autoren}.
 Digitale Edition, https://schnitzler-briefe.acdh.oeaw.ac.at/{\dateiname}.html (Stand \today)
\fi

\end{document}


