%% latex-korrekturansicht-vorspann.tex
%% Vorspann für die Korrekturansicht.
%% Lädt die gemeinsame Datei latex-vorspann.tex mit gesetztem Schalter.

\newif\ifkorrekturansicht
\korrekturansichttrue

\input{../tex-inputs/latex-vorspann}


\section[Felix Salten an Arthur Schnitzler, {[}27. 4. 1894?{]}]{L03120 Felix Salten an Arthur Schnitzler, {[}27. 4. 1894?{]}}
\nopagebreak\mylabel{L03120v}
\rehead{ }\normalsize\beginnumbering\briefempfaengerindex{Schnitzler, Arthur@\textsc{Schnitzler, Arthur}!zzzSalten, Felix@\emph{von Felix Salten}!1894-04-271@{{[}27. 4. 1894?{]}}|(be}
\toendnotes[C]{\smallbreak\pagebreak[2]}\Standort{CUL, Schnitzler, B 89, A 1.}
\physDesc{Brief, 1 Blatt, 2 Seiten, 404 Zeichen
\newline{}Handschrift: Bleistift, lateinische Kurrent
\newline{}Schnitzler: mit Bleistift datiert: »92« 
\newline{}Ordnung: mit Bleistift von unbekannter Hand nummeriert: »23« }\toendnotes[C]{\smallbreak}
\pstart
           \noindent{}{\pb}Lieber Freund! Ihr Brief von gestern hat
               mich leider nicht zu Hause getroffen, ich kam den Abend überhaupt nicht
               nach Hause, weil ich bei \label{K_L03120-1v}\edtext{Pagliacci\pwindex{I Pagliacci. Opera in 2 atti@\emph{I Pagliacci. Opera in 2 atti}|pw}}{\lemma{\textnormal{\emph{Pagliacci}}}\Cendnote{\textnormal{Das Korrespondenzstück wurde von Salten\pwindex{Salten, Felix 06.09.1869 – 08.10.1945@\textsc{Salten, Felix} (06.09.1869 – 08.10.1945), \emph{Schriftsteller/Schriftstellerin, Journalist/Journalistin, Chefredakteur/Chefredakteurin}|pwk} nicht datiert. Die Datierung Schnitzlers auf »92« ist
                  falsch, da er erst am 13. 6. 1893 Fahrradfahren lernte und erst ab dem 19. 7. 1893 gemeinsame
                  Ausfahrten mit Salten\pwindex{Salten, Felix 06.09.1869 – 08.10.1945@\textsc{Salten, Felix} (06.09.1869 – 08.10.1945), \emph{Schriftsteller/Schriftstellerin, Journalist/Journalistin, Chefredakteur/Chefredakteurin}|pwk} unternahm. Obzwar \emph{I Pagliacci}\pwindex{I Pagliacci. Opera in 2 atti@\emph{I Pagliacci. Opera in 2 atti}|pwk} erstmals am 17. 9. 1892 bei der \emph{Wiener Musik- und Theaterausstellung}\orgindex{Internationale Ausstellung fuer Musik und Theaterwesen@Internationale Ausstellung für Musik und Theaterwesen|pwk} in Wien\oindex{Wien@\textbf{Wien}, \emph{A.ADM2}|pwk} gegeben wurde und danach einige Aufführungen folgten (Schnitzler selbst sah das Stück\pwindex{I Pagliacci. Opera in 2 atti@\emph{I Pagliacci. Opera in 2 atti}|pwkv} am 25. 9. 1892), war es nach einer Pause erst wieder
                  ab dem 19. 11. 1893 auf dem Spielplan, diesmal in der Wien\oindex{Wien@\textbf{Wien}, \emph{A.ADM2}|pwk}er \emph{Hofoper}\orgindex{K.K. Hof-Oper@K.K. Hof-Oper|pwk}. Das
                  deutet auf den Frühling 1894 für dieses Korrespondenzstück. Dies ist
                  zugleich der einzige Zeitraum im \emph{Tagebuch}\pwindex{Tagebuch@\emph{Tagebuch}|pwk}, für
                  den mehrere Radausflüge nach Mödling\oindex{Moedling@\textbf{Mödling}, \emph{P.PPLA3}|pwk} belegt
                  sind. Sucht man nach Radausflügen an Tagen nach einer Aufführung von \emph{I Pagliacci}\pwindex{I Pagliacci. Opera in 2 atti@\emph{I Pagliacci. Opera in 2 atti}|pwk}, kommt nur die Aufführung vom
                     26. 4. 1894 zur Datierung des Korrespondenzstücks in
                  Betracht.}}}\label{K_L03120-1} war, und dann in der Stadt\oindex{Wien@\textbf{Wien}, \emph{A.ADM2}|pwv} soupirte. Schade, dass ich nicht wusste, Sie sind im Café.
               Nach Mödling\oindex{Moedling@\textbf{Mödling}, \emph{P.PPLA3}|pw} kann ich heute auch
               nicht {\pb}fahren, weil das Bicycle
               gebrochen ist. Zeigen Sie mir an, wann Sie wieder ins Aubö\textcolor{gray}{ck}\oindex{Cafe Reichsrath (Inh. Karl Auboeck)@\textbf{Café Reichsrath (Inh. Karl Auböck)}, \emph{Kaffeehaus (K.KAF)}|pw} kommen, ich sehne mich schon wirklich danach\pend
           
\pstart
           Herzlich {\\[\baselineskip]}Ihr {\\[\baselineskip]}\spacefill\mbox{Salten}\pend
           \leftskip=0em{}\selectlanguage{ngerman}\endnumbering\briefempfaengerindex{Schnitzler, Arthur@\textsc{Schnitzler, Arthur}!zzzSalten, Felix@\emph{von Felix Salten}!1894-04-271@{{[}27. 4. 1894?{]}}|)be}\mylabel{L03120h}  \normalsize

\doendnotes{C}
\bigskip
\vfill

\clearpage

\footnotesize

\lohead{\textsc{register}}

% Definiere theindex-Environment komplett neu ohne reledmac
\makeatletter
\renewenvironment{theindex}{%
  \section*{\indexname}%
  \setlength{\parindent}{0pt}%
  \setlength{\parskip}{0pt plus 0.3pt}%
  \let\item\@idxitem
}{%
  \clearpage
}
\makeatother

\IfFileExists{\jobname-pw.ind}{\input{\jobname-pw.ind}}{}

\end{document}

      