%% latex-leseansicht-vorspann.tex
%% Vorspann für die Leseansicht.
%% Lädt die gemeinsame Datei latex-vorspann.tex mit nicht gesetztem Schalter.

\newif\ifkorrekturansicht
\korrekturansichtfalse

\input{../tex-inputs/latex-vorspann}

\begin{center}
            \textcolor{red}{ENTWURF, NICHT FERTIG KORRIGIERT}
                      \end{center}
            
         \renewcommand{\erwaehnteInstitutionen}{Institutionen: Wiener Musik- und Theaterausstellung 1892}
         \renewcommand{\erwaehnteOrte}{Orte: Café Reichsrath (Inh. Karl Auböck), Mödling, Theater an der Wien, Wien}
         \renewcommand{\erwaehnteWerke}{Werke: Pagliacci, Tagebuch}
               \section[Felix Salten an Arthur Schnitzler, {[}Juni 1893?{]}]{ Felix Salten an Arthur Schnitzler, {[}Juni 1893?{]}}\nopagebreak\mylabel{v}\rehead{ }\begin{ledgroupsized}[t]{13cm}\normalsize\beginnumbering \toendnotes[C]{\smallbreak\pagebreak[2]} \Standort{CUL, Schnitzler, B 89, A 1.}
\physDesc{Brief, 1 Blatt, 2 Seiten
\newline{}Handschrift: Bleistift, lateinische Kurrent}\toendnotes[C]{\smallbreak}\pstart
           \noindent{}{\pb}Lieber Freund! Ihr Brief von gestern hat mich leider nicht zu Hause
               getroffen, ich kam den Abend überhaupt nicht nach Hause, weil ich bei \label{K_L03120-1v}\edtext{Pagliacci\pwindex{\textcolor{red}{\textsuperscript{XXXX1 indx}}!Pagliacci1892-05-21@\strich\emph{Pagliacci} {[}1892-05-21{]}|pw}}{\lemma{\textnormal{\emph{Pagliacci}}}\Cendnote{\textnormal{Das Stück hatte am
                     17. 9. 1892 bei der \emph{Wiener Musik- und Theaterausstellung}\orgindex{Wiener Musik- und Theaterausstellung 1892@Wiener Musik- und Theaterausstellung 1892|pwk} seine Wien\oindex{Wien@\textbf{Wien}|pwk}premiere und wurde in Folge mehrfach gegeben. Schnitzler\pwindex{Schnitzler, Arthur 15.05.1862 – 21.10.1931@\textsc{Schnitzler, Arthur} (15.05.1862 – 21.10.1931), \emph{Schriftsteller, Mediziner}|pwk} sah das Stück am 25. 9. 1892, davor war er nicht in der Stadt, so
                  dass dies als frühester Zeitpunkt für das undatierte Korrespondenzstück angesetzt
                  werden kann. In Folge wird von einem Radausflug nach Mödling\oindex{Moedling@\textbf{Mödling}|pwk} gesprochen, was mit Oktober überraschend spät im
                  Jahr wäre und Schnitzler\pwindex{Schnitzler, Arthur 15.05.1862 – 21.10.1931@\textsc{Schnitzler, Arthur} (15.05.1862 – 21.10.1931), \emph{Schriftsteller, Mediziner}|pwk} im Oktober die
                  Stadt auch nicht verlassen haben dürfte. Das spricht dafür, dass es sich bei Schnitzler\pwindex{Schnitzler, Arthur 15.05.1862 – 21.10.1931@\textsc{Schnitzler, Arthur} (15.05.1862 – 21.10.1931), \emph{Schriftsteller, Mediziner}|pwk}s Datierung des Korrespondenzstücks
                  auf das Jahr »92« um einen Irrtum handelt. Ein Besuch
                  im Café Auböck\oindex{Cafe Reichsrath (Inh. Karl Auboeck)@\textbf{Café Reichsrath (Inh. Karl Auböck)}|pwk} ist im \emph{Tagebuch}\pwindex{Schnitzler, Arthur 15.05.1862 – 21.10.1931@\textsc{Schnitzler, Arthur} (15.05.1862 – 21.10.1931), \emph{Schriftsteller, Mediziner}!Tagebuch1981 – 2000@\strich\emph{Tagebuch} {[}1981 – 2000{]}|pwk} überhaupt nur für den 29. 5. 1893 und den
                     7. 9. 1893
                  belegt. Ab dem 3. 6. 1892 wurden im Zuge eines Gastspiels
                  mehrmals \emph{Pagliacci}\pwindex{\textcolor{red}{\textsuperscript{XXXX1 indx}}!Pagliacci1892-05-21@\strich\emph{Pagliacci} {[}1892-05-21{]}|pwk}-Aufführungen am Theater an der Wien\oindex{Theater an der Wien@\textbf{Theater an der Wien}|pwk} gegeben, danach setzte an den
                  Theatern die Sommerpause ein, weswegen eine Datierung auf Juni 1893
                  vornehmbar scheint.}}}\label{K_L03120-1h} war, und dann in der Stadt soupirte. Schade, dass ich
               nicht wusste, Sie sind im Café. Nach Mödling\oindex{Moedling@\textbf{Mödling}|pw} kann
               ich heute auch nicht {\pb}fahren, weil
               das Bicycle gebrochen ist. Zeigen Sie mir an, wann Sie wieder ins Aubö\textcolor{gray}{ck}\oindex{Cafe Reichsrath (Inh. Karl Auboeck)@\textbf{Café Reichsrath (Inh. Karl Auböck)}|pw} kommen, ich sehne mich schon wirklich danach.\pend
           \pstart
           Herzlich {\\[\baselineskip]}Ihr {\\[\baselineskip]}\spacefill\mbox{Salten}\pend
           \leftskip=0em{}
         
         \endnumbering\mylabel{h}\end{ledgroupsized}\begin{anhang}\end{anhang}\newcommand{\dateiname}{L03120}\newcommand{\titel}{Felix Salten an Arthur Schnitzler, [Juni 1893?]}\newcommand{\editorInnen}{Martin Anton Müller und Laura Untner}%% latex-leseansicht-abspann.tex
%% Abspann für die Leseansicht.
%% Der Schalter \ifkorrekturansicht ist bereits durch den Vorspann gesetzt.

%% latex-abspann.tex
%% Gemeinsamer Abspann für Korrekturansicht und Leseansicht.
%% Setzt den Schalter \ifkorrekturansicht voraus (gesetzt in den
%% einbindenden Dateien latex-korrekturansicht-abspann.tex bzw.
%% latex-leseansicht-abspann.tex).
%% ---------------------------------------------------------------

\normalsize

% Das esempio-Environment wird nur in der Leseansicht benötigt
\ifkorrekturansicht\else
\newenvironment{esempio}[3]%
{
    \vspace{1.5ex}
    \rlap{\underline{#1}}
    \par
    \setlength{\parindent}{0cm}
    \nopagebreak
    \leftskip=#2cm
    \rightskip=#3cm
}
{
    \par
}
\fi

\doendnotes{C}
\bigskip
\vfill

\clearpage

\footnotesize

\ifkorrekturansicht
  \lohead{\textsc{register}}
\fi

% theindex-Environment neu definieren ohne reledmac
\makeatletter
\renewenvironment{theindex}{%
  \ifkorrekturansicht
    \section*{\indexname}%
  \else
    \subsubsection*{Index der erwähnten Entitäten}%
  \fi
  \setlength{\parindent}{0pt}%
  \setlength{\parskip}{0pt plus 0.3pt}%
  \let\item\@idxitem
}{%
  \ifkorrekturansicht\clearpage\fi
}
\makeatother

\IfFileExists{\jobname-pw.ind}{\input{\jobname-pw.ind}}{}

% Quellenangabe nur in der Leseansicht
\ifkorrekturansicht\else
% Fallback-Definitionen, falls die .tex-Datei \titel etc. nicht gesetzt hat
\providecommand{\titel}{}
\providecommand{\editorInnen}{}
\providecommand{\dateiname}{\jobname}

\vspace{3cm}

\vfill

\footnotesize
\textsc{Quelle}: \titel. Herausgegeben von {\editorInnen}. In: \emph{Arthur Schnitzler: Briefwechsel mit Autorinnen und Autoren}.
 Digitale Edition, https://schnitzler-briefe.acdh.oeaw.ac.at/{\dateiname}.html (Stand \today)
\fi

\end{document}


      