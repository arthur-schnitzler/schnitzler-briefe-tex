%% latex-leseansicht-vorspann.tex
%% Vorspann für die Leseansicht.
%% Lädt die gemeinsame Datei latex-vorspann.tex mit nicht gesetztem Schalter.

\newif\ifkorrekturansicht
\korrekturansichtfalse

\input{../tex-inputs/latex-vorspann}

\begin{center}
            \textcolor{red}{ENTWURF, NICHT FERTIG KORRIGIERT}
                      \end{center}
            
         
         \renewcommand{\erwaehntePersonen}{Personen: Felix Salten}
         \renewcommand{\erwaehnteInstitutionen}{Institutionen: Internationale Ausstellung für Musik und Theaterwesen, K.K. Hof-Oper}
         \renewcommand{\erwaehnteOrte}{Orte: Café Reichsrath (Inh. Karl Auböck), Mödling, Wien}
         \renewcommand{\erwaehnteWerke}{Werke: Pagliacci, Tagebuch}
               \section[Felix Salten an Arthur Schnitzler, {[}27. 4. 1894?{]}]{ Felix Salten an Arthur Schnitzler, {[}27. 4. 1894?{]}}\nopagebreak\mylabel{v}\rehead{ }\begin{ledgroupsized}[t]{13cm}\normalsize\beginnumbering \toendnotes[C]{\smallbreak\pagebreak[2]} \Standort{CUL, Schnitzler, B 89, A 1.}
\physDesc{Brief, 1 Blatt, 2 Seiten, 405 Zeichen
\newline{}Handschrift: Bleistift, lateinische Kurrent
\newline{}Schnitzler: mit Bleistift datiert: »92« 
\newline{}Ordnung: mit Bleistift von unbekannter Hand nummeriert: »23« }\toendnotes[C]{\smallbreak}\pstart
           \noindent{}{\pb}Lieber Freund! Ihr Brief von gestern\textcolor{gray}{,} hat mich leider nicht zu Hause getroffen, ich kam den
                  Abend überhaupt nicht nach Hause, weil ich bei \label{K_L03120-1v}\edtext{Pagliacci\pwindex{\textcolor{red}{\textsuperscript{XXXX1 indx}}!Pagliacci1892-05-21@\strich\emph{Pagliacci} {[}1892-05-21{]}|pw}}{\lemma{\textnormal{\emph{Pagliacci}}}\Cendnote{\textnormal{Das Korrespondenzstück wurde von Salten\pwindex{Salten, Felix 06.09.1869 – 08.10.1945@\textsc{Salten, Felix} (06.09.1869 – 08.10.1945), \emph{Schriftsteller, Journalist}|pwk} nicht datiert, die Datierung Schnitzler\pwindex{Schnitzler, Arthur 15.05.1862 – 21.10.1931@\textsc{Schnitzler, Arthur} (15.05.1862 – 21.10.1931), \emph{Schriftsteller, Mediziner}|pwk}s auf »92« ist falsch, da er erst am 13. 6. 1893 Fahrradfahren lernte und erst
                  ab dem 
                  19. 7. 1893 gemeinsame Ausfahrten mit 
                   Salten\pwindex{Salten, Felix 06.09.1869 – 08.10.1945@\textsc{Salten, Felix} (06.09.1869 – 08.10.1945), \emph{Schriftsteller, Journalist}|pwk} unternahm. Obzwar 
                  \emph{I Pagliacci}\pwindex{\textcolor{red}{\textsuperscript{XXXX1 indx}}!Pagliacci1892-05-21@\strich\emph{Pagliacci} {[}1892-05-21{]}|pwk} erstmals am 17. 9. 1892 bei der \emph{Wiener Musik- und
                        Theaterausstellung}\orgindex{Internationale Ausstellung fuer Musik und Theaterwesen@Internationale Ausstellung für Musik und Theaterwesen|pwk} in Wien\oindex{Wien@\textbf{Wien}|pwk}
                  gegeben wurde und danach einige Aufführungen folgten (Schnitzler\pwindex{Schnitzler, Arthur 15.05.1862 – 21.10.1931@\textsc{Schnitzler, Arthur} (15.05.1862 – 21.10.1931), \emph{Schriftsteller, Mediziner}|pwk} selbst sah das Stück\pwindex{\textcolor{red}{\textsuperscript{XXXX1 indx}}!Pagliacci1892-05-21@\strich\emph{Pagliacci} {[}1892-05-21{]}|pwkv} am 25. 9. 1892), war es zwischenzeitig erst wieder ab dem 19. 11. 1893
                  am Spielplan, diesmal in der Wien\oindex{Wien@\textbf{Wien}|pwk}er 
                  \emph{Hofoper}\orgindex{K.K. Hof-Oper@K.K. Hof-Oper|pwk}. Das deutet auf den Frühling 1894 für dieses Korrespondenzstück.
                  Dies ist zugleich der einzige Zeitraum im \emph{Tagebuch}\pwindex{\textcolor{red}{\textsuperscript{XXXX1 indx}}!Tagebuch1981 – 2000@\strich\emph{Tagebuch} {[}Hrsg., 1981 – 2000{]}|pwk}, 
                  in dem mehrere Radausflüge nach Mödling\oindex{Moedling@\textbf{Mödling}|pwk} belegt sind.
                  Sucht man nach Radausflügen an Tagen nach einer Aufführung von \emph{I Pagliacci}\pwindex{\textcolor{red}{\textsuperscript{XXXX1 indx}}!Pagliacci1892-05-21@\strich\emph{Pagliacci} {[}1892-05-21{]}|pwk}, kommt 
                  nur die Aufführung vom 26. 4. 1894 zur Datierung des
                  Korrespondenzstücks in Betracht.}}}\label{K_L03120-1h} war, und dann in der Stadt\oindex{Wien@\textbf{Wien}|pwv} soupirte. Schade, dass ich nicht wusste, Sie sind im Café.
               Nach Mödling\oindex{Moedling@\textbf{Mödling}|pw} kann ich heute auch nicht {\pb}fahren, weil das Bicycle gebrochen ist. Zeigen Sie mir an, wann Sie wieder ins Aubö\textcolor{gray}{ck}\oindex{Cafe Reichsrath (Inh. Karl Auboeck)@\textbf{Café Reichsrath (Inh. Karl Auböck)}|pw} kommen, ich sehne mich schon wirklich danach\pend
           \pstart
           Herzlich {\\[\baselineskip]}Ihr {\\[\baselineskip]}\spacefill\mbox{Salten}\pend
           \leftskip=0em{}
         
         \endnumbering\mylabel{h}\end{ledgroupsized}  \newcommand{\dateiname}{L03120}\newcommand{\titel}{Felix Salten an Arthur Schnitzler, [27. 4. 1894?]}\newcommand{\editorInnen}{Martin Anton Müller und Laura Untner}%% latex-leseansicht-abspann.tex
%% Abspann für die Leseansicht.
%% Der Schalter \ifkorrekturansicht ist bereits durch den Vorspann gesetzt.

%% latex-abspann.tex
%% Gemeinsamer Abspann für Korrekturansicht und Leseansicht.
%% Setzt den Schalter \ifkorrekturansicht voraus (gesetzt in den
%% einbindenden Dateien latex-korrekturansicht-abspann.tex bzw.
%% latex-leseansicht-abspann.tex).
%% ---------------------------------------------------------------

\normalsize

% Das esempio-Environment wird nur in der Leseansicht benötigt
\ifkorrekturansicht\else
\newenvironment{esempio}[3]%
{
    \vspace{1.5ex}
    \rlap{\underline{#1}}
    \par
    \setlength{\parindent}{0cm}
    \nopagebreak
    \leftskip=#2cm
    \rightskip=#3cm
}
{
    \par
}
\fi

\doendnotes{C}
\bigskip
\vfill

\clearpage

\footnotesize

\ifkorrekturansicht
  \lohead{\textsc{register}}
\fi

% theindex-Environment neu definieren ohne reledmac
\makeatletter
\renewenvironment{theindex}{%
  \ifkorrekturansicht
    \section*{\indexname}%
  \else
    \subsubsection*{Index der erwähnten Entitäten}%
  \fi
  \setlength{\parindent}{0pt}%
  \setlength{\parskip}{0pt plus 0.3pt}%
  \let\item\@idxitem
}{%
  \ifkorrekturansicht\clearpage\fi
}
\makeatother

\IfFileExists{\jobname-pw.ind}{\input{\jobname-pw.ind}}{}

% Quellenangabe nur in der Leseansicht
\ifkorrekturansicht\else
% Fallback-Definitionen, falls die .tex-Datei \titel etc. nicht gesetzt hat
\providecommand{\titel}{}
\providecommand{\editorInnen}{}
\providecommand{\dateiname}{\jobname}

\vspace{3cm}

\vfill

\footnotesize
\textsc{Quelle}: \titel. Herausgegeben von {\editorInnen}. In: \emph{Arthur Schnitzler: Briefwechsel mit Autorinnen und Autoren}.
 Digitale Edition, https://schnitzler-briefe.acdh.oeaw.ac.at/{\dateiname}.html (Stand \today)
\fi

\end{document}


      