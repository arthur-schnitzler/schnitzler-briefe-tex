%% latex-leseansicht-vorspann.tex
%% Vorspann für die Leseansicht.
%% Lädt die gemeinsame Datei latex-vorspann.tex mit nicht gesetztem Schalter.

\newif\ifkorrekturansicht
\korrekturansichtfalse

\input{../tex-inputs/latex-vorspann}


\section[ Paul Goldmann und Theodore Rottenberg an Arthur Schnitzler, 18. 8. {[}1903{]}]{L03384 Paul Goldmann und Theodore Rottenberg an Arthur
               Schnitzler,  18. 8. [1903]}
\nopagebreak\mylabel{L03384v}
\rehead{ }\normalsize\beginnumbering\briefempfaengerindex{Schnitzler, Arthur@\textsc{Schnitzler, Arthur}!zzzRottenberg, Theodore@\emph{von Theodore Rottenberg}!1903-08-182@{18. 8. [1903]}|(be}\briefempfaengerindex{Schnitzler, Arthur@\textsc{Schnitzler, Arthur}!zzzGoldmann, Paul@\emph{von Paul Goldmann}!1903-08-182@{18. 8. [1903]}|(be}
\toendnotes[C]{\smallbreak\pagebreak[2]}
\correspDesc{Versand  durch Paul Goldmann, Theodore Rottenberg am 18. 8. [1903] in Riva del Garda
\newline{}Erhalt  durch Arthur Schnitzler am 18. 8. [1903] in Riva del Garda}\toendnotes[C]{\smallbreak}
\Standort{DLA, A:Schnitzler, HS.NZ85.1.3173.}
\physDesc{Brief, 1 Blatt, 2 Seiten, 576 Zeichen
\newline{}Handschrift Paul Goldmann: schwarze Tinte, deutsche Kurrent
\newline{}Handschrift Theodore Rottenberg: schwarze Tinte, deutsche Kurrent}\toendnotes[C]{\smallbreak}
\pstart
           {\pb}\textcolor{gray}{\textbf{\textbf{PALAST HOTEL LIDO\oindex{Palast Hotel Lido@\textbf{Palast Hotel Lido}, \emph{Hotel}|pw}}}}\pend
           
\pstart
           \textcolor{gray}{\textbf{\textbf{RIVA\oindex{Riva del Garda@\textbf{Riva del Garda}, \emph{Hauptstadt}|pw}}}}{\\}\textcolor{gray}{\textbf{AM GARDASEE\oindex{Lago di Garda@\textbf{Lago di Garda}, \emph{See}|pw}}}\pend
           
\pstart
           \textcolor{gray}{\textbf{Gleiche Direction:}}\pend
           
\pstart
           \textcolor{gray}{\textbf{\textbf{Grand Hotel Lavarone\oindex{Grand Hotel Lavarone@\textbf{Grand Hotel Lavarone}, \emph{Hotel}|pw} – Lavarone\oindex{Lavarone@\textbf{Lavarone}, \emph{Verwaltungsgebiet}|pw}}}}\pend
           
\pstart
           \textcolor{gray}{\textbf{(1200 m. ü. M., Saison Juni bis October)}}\hfill \textcolor{gray}{\textbf{\emph{Riva\oindex{Riva del Garda@\textbf{Riva del Garda}, \emph{Hauptstadt}|pw},}}}{ }18. Auguſt \textcolor{gray}{\textbf{\emph{1903}}}\pend
           
\pstart{}Mein lieber Freund,\pend\vspace{0.5em}
\pstart
           Hoffentlich haſt Du meine \label{K_L03384-1v}\edtext{heutige Depeſche}{\lemma{\textnormal{\emph{heutige Depesche}}}\Cendnote{\textnormal{Gemeint sein dürfte das Telegramm vom Vortag: XXXX Auszeichnungsfehler: Dokument L02631 nicht gefunden. Goldmann\pwindex{Goldmann, Paul 31.\,1.\,1865 Breslau – 25.\,9.\,1935 Wien@\textsc{Goldmann, Paul} (31.\,1.\,1865 Breslau – 25.\,9.\,1935 Wien), \emph{Schriftsteller, Journalist}|pwk} dürfte davon ausgegangen sein, dass Schnitzler erst am 18. 8. 1903 nach Madonna di Campiglio\oindex{Madonna di Campiglio@\textbf{Madonna di Campiglio}|pwk} kommen würde, womit sich das
                     »postlagernd« auf dem Telegramm erklärt. Schnitzler war aber zum Zeitpunkt des vorliegenden
                  Schreibens bereits auf dem Weg nach Riva\oindex{Riva del Garda@\textbf{Riva del Garda}, \emph{Hauptstadt}|pwk}, wo
                  sie sich am selben Tag noch trafen und miteinander soupierten. Die Abreise von Goldmann\pwindex{Goldmann, Paul 31.\,1.\,1865 Breslau – 25.\,9.\,1935 Wien@\textsc{Goldmann, Paul} (31.\,1.\,1865 Breslau – 25.\,9.\,1935 Wien), \emph{Schriftsteller, Journalist}|pwk} und Rottenberg\pwindex{Rottenberg, Theodore 7.\,9.\,1875 – 5.\,4.\,1945 Limburg an der Lahn@\textsc{Rottenberg, Theodore} (7.\,9.\,1875 – 5.\,4.\,1945 Limburg an der Lahn)|pwk} dürfte also erst am Abend stattgefunden haben –
                  oder gemeinsam mit Schnitzler am Folgetag.
                  Vermutlich war dieses Schreiben bereits in Riva\oindex{Riva del Garda@\textbf{Riva del Garda}, \emph{Hauptstadt}|pwk} in Schnitzlers Unterkunft
                  hinterlegt, als sie aufeinandertrafen.}}}\label{K_L03384-1} noch erhalten. Wir { }\uline{müſſen}{ }heut von hier\oindex{Riva del Garda@\textbf{Riva del Garda}, \emph{Hauptstadt}|pwv} fort. \strikeout{\textcolor{gray}{S}} Die \label{K_L03384-2v}\edtext{Gründe}{\lemma{\textnormal{\emph{Gründe}}}\Cendnote{\textnormal{Rottenberg\pwindex{Rottenberg, Theodore 7.\,9.\,1875 – 5.\,4.\,1945 Limburg an der Lahn@\textsc{Rottenberg, Theodore} (7.\,9.\,1875 – 5.\,4.\,1945 Limburg an der Lahn)|pwk} war verheiratet. Es dürfte eine
                  ihnen bekannte Person ebenfalls in der Stadt\oindex{Riva del Garda@\textbf{Riva del Garda}, \emph{Hauptstadt}|pwk} gewesen sein und sie 
                  fürchteten Tratsch, vgl. XXXX Auszeichnungsfehler: Dokument L03385 nicht gefunden.}}}\label{K_L03384-2} werden wir Dir
               mündlich{ }ſagen, und Du wirſt{ }ſie begreifen. Wir hätten uns Beide unendlich gefreut,
               Dich hier zu{ }ſehen, hoffen aber, das Verſäumte in \textsc{Trient\oindex{Trient@\textbf{Trient}|pw}}{ }\strikeout{und} oder \textsc{Lavarone\oindex{Lavarone@\textbf{Lavarone}, \emph{Verwaltungsgebiet}|pw}}{ }nachzuholen. In \label{K_L03384-3v}\edtext{\textsc{Trient\oindex{Trient@\textbf{Trient}|pw}} bleiben \strikeout{b} wir bis morgen{ }Mittag}{\lemma{\textnormal{\emph{Trient … Mittag}}}\Cendnote{\textnormal{Das verschob sich um einen Tag: Schnitzler fuhr entweder mit den beiden am
                     19. 8. 1903
                  gemeinsam nach Trient\oindex{Trient@\textbf{Trient}|pwk} oder holte sie dort ein.
                  Die gemeinsame Weiterreise fand dann am 20. 8. 1903 statt.}}}\label{K_L03384-3} (\textsc{Hotel Carloni\oindex{Hotel Carloni@\textbf{Hotel Carloni}, \emph{Hotel}|pw}}), dann \textsc{Lavarone\oindex{Lavarone@\textbf{Lavarone}, \emph{Verwaltungsgebiet}|pw}} (\textsc{Grand Hôtel Central\oindex{Grand Hotel Lavarone@\textbf{Grand Hotel Lavarone}, \emph{Hotel}|pw}})\pend
           
\pstart
           {\pb}Sei uns nicht böſe! Und komme uns \uline{ſo bald als möglich} nach!\pend
           
\pstart
           Tauſend Grüße! {\\[\baselineskip]}Dein {\\[\baselineskip]}\spacefill\mbox{Paul Goldmann}\pend
           \leftskip=0em{}
\pstart
           \noindent{}{[}hs. Rottenberg:{]} Viele herzliche Grüße, hoffentlich können Sie{ }ſich
                  die Gründe denken u ko{\geminationm}en uns gleich nach! –\pend
           \selectlanguage{ngerman}\endnumbering\briefempfaengerindex{Schnitzler, Arthur@\textsc{Schnitzler, Arthur}!zzzRottenberg, Theodore@\emph{von Theodore Rottenberg}!1903-08-182@{18. 8. [1903]}|)be}\briefempfaengerindex{Schnitzler, Arthur@\textsc{Schnitzler, Arthur}!zzzGoldmann, Paul@\emph{von Paul Goldmann}!1903-08-182@{18. 8. [1903]}|)be}\mylabel{L03384h}  \newcommand{\dateiname}{L03384}\newcommand{\titel}{Paul Goldmann und Theodore Rottenberg an Arthur Schnitzler, 18. 8. [1903]}\newcommand{\editorInnen}{Martin Anton Müller und Laura Untner}%% latex-leseansicht-abspann.tex
%% Abspann für die Leseansicht.
%% Der Schalter \ifkorrekturansicht ist bereits durch den Vorspann gesetzt.

%% latex-abspann.tex
%% Gemeinsamer Abspann für Korrekturansicht und Leseansicht.
%% Setzt den Schalter \ifkorrekturansicht voraus (gesetzt in den
%% einbindenden Dateien latex-korrekturansicht-abspann.tex bzw.
%% latex-leseansicht-abspann.tex).
%% ---------------------------------------------------------------

\normalsize

% Das esempio-Environment wird nur in der Leseansicht benötigt
\ifkorrekturansicht\else
\newenvironment{esempio}[3]%
{
    \vspace{1.5ex}
    \rlap{\underline{#1}}
    \par
    \setlength{\parindent}{0cm}
    \nopagebreak
    \leftskip=#2cm
    \rightskip=#3cm
}
{
    \par
}
\fi

\doendnotes{C}
\bigskip
\vfill

\clearpage

\footnotesize

\ifkorrekturansicht
  \lohead{\textsc{register}}
\fi

% theindex-Environment neu definieren ohne reledmac
\makeatletter
\renewenvironment{theindex}{%
  \ifkorrekturansicht
    \section*{\indexname}%
  \else
    \subsubsection*{Index der erwähnten Entitäten}%
  \fi
  \setlength{\parindent}{0pt}%
  \setlength{\parskip}{0pt plus 0.3pt}%
  \let\item\@idxitem
}{%
  \ifkorrekturansicht\clearpage\fi
}
\makeatother

\IfFileExists{\jobname-pw.ind}{\input{\jobname-pw.ind}}{}

% Quellenangabe nur in der Leseansicht
\ifkorrekturansicht\else
% Fallback-Definitionen, falls die .tex-Datei \titel etc. nicht gesetzt hat
\providecommand{\titel}{}
\providecommand{\editorInnen}{}
\providecommand{\dateiname}{\jobname}

\vspace{3cm}

\vfill

\footnotesize
\textsc{Quelle}: \titel. Herausgegeben von {\editorInnen}. In: \emph{Arthur Schnitzler: Briefwechsel mit Autorinnen und Autoren}.
 Digitale Edition, https://schnitzler-briefe.acdh.oeaw.ac.at/{\dateiname}.html (Stand \today)
\fi

\end{document}


