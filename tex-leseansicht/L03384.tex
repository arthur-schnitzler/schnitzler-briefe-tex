%% latex-korrekturansicht-vorspann.tex
%% Vorspann für die Korrekturansicht.
%% Lädt die gemeinsame Datei latex-vorspann.tex mit gesetztem Schalter.

\newif\ifkorrekturansicht
\korrekturansichttrue

\input{../tex-inputs/latex-vorspann}


\section[ Paul Goldmann und Theodore Rottenberg an Arthur Schnitzler, 18. 8. {[}1903{]}]{L03384 Paul Goldmann und Theodore Rottenberg an Arthur
               Schnitzler, 18. 8. {[}1903{]}}
\nopagebreak\mylabel{L03384v}
\rehead{ }\normalsize\beginnumbering\briefempfaengerindex{Schnitzler, Arthur@\textsc{Schnitzler, Arthur}!zzzRottenberg, Theodore@\emph{von Theodore Rottenberg}!1903-08-182@{18. 8. {[}1903{]}}|(be}\briefempfaengerindex{Schnitzler, Arthur@\textsc{Schnitzler, Arthur}!zzzGoldmann, Paul@\emph{von Paul Goldmann}!1903-08-182@{18. 8. {[}1903{]}}|(be}
\toendnotes[C]{\smallbreak\pagebreak[2]}\Standort{DLA, A:Schnitzler, HS.NZ85.1.3173.}
\physDesc{Brief, 1 Blatt, 2 Seiten, 576 Zeichen
\newline{}Handschrift Paul Goldmann: schwarze Tinte, deutsche Kurrent
\newline{}Handschrift Theodore Rottenberg: schwarze Tinte, deutsche Kurrent}\toendnotes[C]{\smallbreak}
\pstart
           {\pb}\textcolor{gray}{\textbf{\textbf{PALAST HOTEL LIDO\oindex{Palast Hotel Lido@\textbf{Palast Hotel Lido}, \emph{Hotel (K.HTL)}|pw}}}}\pend
           
\pstart
           \textcolor{gray}{\textbf{\textbf{RIVA\oindex{Riva del Garda@\textbf{Riva del Garda}, \emph{P.PPLA3}|pw}}}}{\\}\textcolor{gray}{\textbf{AM GARDASEE\oindex{Lago di Garda@\textbf{Lago di Garda}, \emph{See (N.SEE)}|pw}}}\pend
           
\pstart
           \textcolor{gray}{\textbf{Gleiche Direction:}}\pend
           
\pstart
           \textcolor{gray}{\textbf{\textbf{Grand Hotel Lavarone\oindex{Grand Hotel Lavarone@\textbf{Grand Hotel Lavarone}, \emph{Hotel (K.HTL)}|pw} – Lavarone\oindex{Lavarone@\textbf{Lavarone}, \emph{A.ADM3}|pw}}}}\pend
           
\pstart
           \textcolor{gray}{\textbf{(1200 m. ü. M., Saison Juni bis October)}}\hfill \textcolor{gray}{\textbf{\emph{Riva\oindex{Riva del Garda@\textbf{Riva del Garda}, \emph{P.PPLA3}|pw},}}}{ }18. Auguſt \textcolor{gray}{\textbf{\emph{1903}}}\pend
           
\pstart{}Mein lieber Freund,\pend\vspace{0.5em}
\pstart
           Hoffentlich haſt Du meine \label{K_L03384-1v}\edtext{heutige Depeſche}{\lemma{\textnormal{\emph{heutige Depeſche}}}\Cendnote{\textnormal{Gemeint sein dürfte das Telegramm vom Vortag: Paul Goldmann an Arthur Schnitzler, 17. 8. 1903. Goldmann\pwindex{Goldmann, Paul 31.01.1865 – 25.09.1935@\textsc{Goldmann, Paul} (31.01.1865 – 25.09.1935), \emph{Schriftsteller/Schriftstellerin, Journalist/Journalistin}|pwk} dürfte davon ausgegangen sein, dass Schnitzler erst am 18. 8. 1903 nach Madonna di Campiglio\oindex{Madonna di Campiglio@\textbf{Madonna di Campiglio}, \emph{P.PPL}|pwk} kommen würde, womit sich das
                     »postlagernd« auf dem Telegramm erklärt. Schnitzler war aber zum Zeitpunkt des vorliegenden
                  Schreibens bereits auf dem Weg nach Riva\oindex{Riva del Garda@\textbf{Riva del Garda}, \emph{P.PPLA3}|pwk}, wo
                  sie sich am selben Tag noch trafen und miteinander soupierten. Die Abreise von Goldmann\pwindex{Goldmann, Paul 31.01.1865 – 25.09.1935@\textsc{Goldmann, Paul} (31.01.1865 – 25.09.1935), \emph{Schriftsteller/Schriftstellerin, Journalist/Journalistin}|pwk} und Rottenberg\pwindex{Rottenberg, Theodore 1875-09-07 – 1945-04-05@\textsc{Rottenberg, Theodore} (1875-09-07 – 1945-04-05)|pwk} dürfte also erst am Abend stattgefunden haben –
                  oder gemeinsam mit Schnitzler am Folgetag.
                  Vermutlich war dieses Schreiben bereits in Riva\oindex{Riva del Garda@\textbf{Riva del Garda}, \emph{P.PPLA3}|pwk} in Schnitzlers Unterkunft
                  hinterlegt, als sie aufeinandertrafen.}}}\label{K_L03384-1} noch erhalten. Wir { }\uline{müſſen}{ }heut von hier\oindex{Riva del Garda@\textbf{Riva del Garda}, \emph{P.PPLA3}|pwv} fort. \strikeout{\textcolor{gray}{S}} Die \label{K_L03384-2v}\edtext{Gründe}{\lemma{\textnormal{\emph{Gründe}}}\Cendnote{\textnormal{Rottenberg\pwindex{Rottenberg, Theodore 1875-09-07 – 1945-04-05@\textsc{Rottenberg, Theodore} (1875-09-07 – 1945-04-05)|pwk} war verheiratet. Es dürfte eine
                  ihnen bekannte Person ebenfalls in der Stadt\oindex{Riva del Garda@\textbf{Riva del Garda}, \emph{P.PPLA3}|pwk} gewesen sein und sie 
                  fürchteten Tratsch, vgl. Paul Goldmann und Theodore Rottenberg an Arthur
               Schnitzler, 29. 8. 1903.}}}\label{K_L03384-2} werden wir Dir
               mündlich ſagen, und Du wirſt ſie begreifen. Wir hätten uns Beide unendlich gefreut,
               Dich hier zu ſehen, hoffen aber, das Verſäumte in \textsc{Trient\oindex{Trient@\textbf{Trient}, \emph{P.PPLA}|pw}}{ }\strikeout{und} oder \textsc{Lavarone\oindex{Lavarone@\textbf{Lavarone}, \emph{A.ADM3}|pw}}{ }nachzuholen. In \label{K_L03384-3v}\edtext{\textsc{Trient\oindex{Trient@\textbf{Trient}, \emph{P.PPLA}|pw}} bleiben \strikeout{b} wir bis morgen{ }Mittag}{\lemma{\textnormal{\emph{Trient … Mittag}}}\Cendnote{\textnormal{Das verschob sich um einen Tag: Schnitzler fuhr entweder mit den beiden am
                     19. 8. 1903
                  gemeinsam nach Trient\oindex{Trient@\textbf{Trient}, \emph{P.PPLA}|pwk} oder holte sie dort ein.
                  Die gemeinsame Weiterreise fand dann am 20. 8. 1903 statt.}}}\label{K_L03384-3} (\textsc{Hotel Carloni\oindex{Hotel Carloni@\textbf{Hotel Carloni}, \emph{Hotel (K.HTL)}|pw}}), dann \textsc{Lavarone\oindex{Lavarone@\textbf{Lavarone}, \emph{A.ADM3}|pw}} (\textsc{Grand Hôtel Central\oindex{Grand Hotel Lavarone@\textbf{Grand Hotel Lavarone}, \emph{Hotel (K.HTL)}|pw}})\pend
           
\pstart
           {\pb}Sei uns nicht böſe! Und komme uns \uline{ſo bald als möglich} nach!\pend
           
\pstart
           Tauſend Grüße! {\\[\baselineskip]}Dein {\\[\baselineskip]}\spacefill\mbox{Paul Goldmann}\pend
           \leftskip=0em{}
\pstart
           \noindent{}{[}hs. :{]} Viele herzliche Grüße, hoffentlich können Sie ſich
                  die Gründe denken u ko{\geminationm}en uns gleich nach! –\pend
           \selectlanguage{ngerman}\endnumbering\briefempfaengerindex{Schnitzler, Arthur@\textsc{Schnitzler, Arthur}!zzzRottenberg, Theodore@\emph{von Theodore Rottenberg}!1903-08-182@{18. 8. {[}1903{]}}|)be}\briefempfaengerindex{Schnitzler, Arthur@\textsc{Schnitzler, Arthur}!zzzGoldmann, Paul@\emph{von Paul Goldmann}!1903-08-182@{18. 8. {[}1903{]}}|)be}\mylabel{L03384h}  \normalsize

\doendnotes{C}
\bigskip
\vfill

\clearpage

\footnotesize

\lohead{\textsc{register}}

% Definiere theindex-Environment komplett neu ohne reledmac
\makeatletter
\renewenvironment{theindex}{%
  \section*{\indexname}%
  \setlength{\parindent}{0pt}%
  \setlength{\parskip}{0pt plus 0.3pt}%
  \let\item\@idxitem
}{%
  \clearpage
}
\makeatother

\IfFileExists{\jobname-pw.ind}{\input{\jobname-pw.ind}}{}

\end{document}

      