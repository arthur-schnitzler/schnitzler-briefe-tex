%% latex-leseansicht-vorspann.tex
%% Vorspann für die Leseansicht.
%% Lädt die gemeinsame Datei latex-vorspann.tex mit nicht gesetztem Schalter.

\newif\ifkorrekturansicht
\korrekturansichtfalse

\input{../tex-inputs/latex-vorspann}

\begin{center}
            \textcolor{red}{ENTWURF, NICHT FERTIG KORRIGIERT}
                      \end{center}
            
         
         \renewcommand{\erwaehntePersonen}{Personen: Richard Beer-Hofmann, Paul Goldmann, Hugo von Hofmannsthal, Caroline Kotter, Maria Charlotte Lamberg, Alexander Wilhelm Neuman, Charlotte Pohl-Glas, Moritz Rosenthal, Hans Schnitzler}
         \renewcommand{\erwaehnteOrte}{Orte: Bad Ischl, Hotel und Pension Rudolfshöhe (Leopold Petter), Kopenhagen, Rügen, Sylt, Szczecin, Wien}
         \renewcommand{\erwaehnteWerke}{Werke: Der Tod Georgs, Die Münchener Kunstausstellungen. I. Im königl. Glaspalast, Die Münchener Kunstausstellungen. II. Im königl. Glaspalast, Münchener Brief. (Orig.-Corr. der »Wiener Allg. Ztg.«), Quer durch den Wurstelprater}
               \section[Felix Salten an Arthur Schnitzler, 22. 7. 1895]{ Felix Salten an Arthur Schnitzler, 22. 7. 1895}\nopagebreak\mylabel{v}\rehead{ }\begin{ledgroupsized}[t]{13cm}\normalsize\beginnumbering \toendnotes[C]{\smallbreak\pagebreak[2]} \Standort{CUL, Schnitzler, B 89, A 1.}
\physDesc{Brief, 1 Blatt, 4 Seiten, 3998 Zeichen
\newline{}Handschrift: Bleistift, lateinische Kurrent
\newline{}Ordnung: mit Bleistift von unbekannter Hand nummeriert:
                                    »58« }\toendnotes[C]{\smallbreak}\pstart
           {\pb}22. Juli 95. \pend
           \pstart
           Lieber Freund! Das war sehr lieb von Ihnen, dass Sie mir
               mittheilten, ich werde oft gelobt, es hat mich sehr gefreut, denn ich begreife immer
               mehr, dass der Hugo\pwindex{Hofmannsthal, Hugo von 1874-02-01 – 1929-07-15@\textsc{Hofmannsthal, Hugo von} (1874-02-01 – 1929-07-15), \emph{Schriftsteller}|pw} recht hat, wenn er sagt:
               »Ich möcht mehr g’lobt werd’n.« Sie können sich vorstellen, welches Gewicht ich auf
               das Urtheil von Neumann\pwindex{Neuman, Alexander Wilhelm 03.05.1860 – 06.05.1936@\textsc{Neuman, Alexander Wilhelm} (03.05.1860 – 06.05.1936), \emph{Redakteur, Kaufmann}|pw} lege. Jetzt erst
               glaube ich, dass ich doch etwas kann. Ich habe mir jetzt meine Feuilletons\pwindex{Salten, Felix 06.09.1869 – 08.10.1945@\textsc{Salten, Felix} (06.09.1869 – 08.10.1945), \emph{Schriftsteller, Journalist}!Muenchener Kunstausstellungen. I. Im koenigl. Glaspalast1895-07-24@\strich\emph{Die Münchener Kunstausstellungen. I. Im königl. Glaspalast} {[}1895-07-24{]}|pwv}\pwindex{Salten, Felix 06.09.1869 – 08.10.1945@\textsc{Salten, Felix} (06.09.1869 – 08.10.1945), \emph{Schriftsteller, Journalist}!Muenchener Kunstausstellungen. II. Im koenigl. Glaspalast1895-07-25@\strich\emph{Die Münchener Kunstausstellungen. II. Im königl. Glaspalast} {[}1895-07-25{]}|pwv}\pwindex{Muenchener Brief. (Orig.-Corr. der »Wiener Allg. Ztg.«)1895-07-06@\emph{Münchener Brief. (Orig.-Corr. der »Wiener Allg. Ztg.«)} {[}1895-07-06{]}|pwv}
               zusammenstellen laßen, und schicke sie Ihnen morgen. Wählen Sie davon welche aus, und
               senden Sie das am Goldmann\pwindex{Goldmann, Paul 31.01.1865 – 25.09.1935@\textsc{Goldmann, Paul} (31.01.1865 – 25.09.1935), \emph{Schriftsteller, Journalist}|pw} weiter, ja? Dass
               ich Beer-Hofmann\pwindex{Beer-Hofmann, Richard 1866-07-11 – 1945-09-26@\textsc{Beer-Hofmann, Richard} (1866-07-11 – 1945-09-26), \emph{Schriftsteller}|pw} nichts geschrieben habe, soll
               nicht missdeutet werden. Zu einem Brief lag rein äußerlich nichts vor, und ihm auf
               den Wurstelprater\pwindex{Salten, Felix 06.09.1869 – 08.10.1945@\textsc{Salten, Felix} (06.09.1869 – 08.10.1945), \emph{Schriftsteller, Journalist}!Quer durch den Wurstelprater1895-06-02 – 1895-06-09@\strich\emph{Quer durch den Wurstelprater} {[}1895-06-02 – 1895-06-09{]}|pw} eine Widmung schreiben, mochte
               ich nicht, weil ich ja nicht wusste, wie ihm der Wurstelprater\pwindex{Salten, Felix 06.09.1869 – 08.10.1945@\textsc{Salten, Felix} (06.09.1869 – 08.10.1945), \emph{Schriftsteller, Journalist}!Quer durch den Wurstelprater1895-06-02 – 1895-06-09@\strich\emph{Quer durch den Wurstelprater} {[}1895-06-02 – 1895-06-09{]}|pw} gefallen werde, und weil, – nun Sie wissen ja dass ich da
               vielleicht ein bisschen zu sehr empfindlich bin. Ich weiss ja auch heute nicht, ob er
               was davon hält, und so konnte ich ihm bis heute nichts schreiben. Übrigens vermuthe
               ich, dass er ihm nicht gefallen hat, weil Sie mir das sonst sicher geschrieben
               hätten. Dabei kann ich aber nicht begreifen, {\pb}seit wann wir uns das \uline{nicht }mittheilen. Das Sie einen kleinen Neffen\pwindex{Schnitzler, Hans 11.07.1895 – 26.03.1967@\textsc{Schnitzler, Hans} (11.07.1895 – 26.03.1967), \emph{Mediziner}|pwv} haben, wusste ich,
               aber das kann mir doch nicht imponiren, da ich doch zwei Töchter\pwindex{Lamberg, Maria Charlotte 1895-03-24 – 1895-07-27@\textsc{Lamberg, Maria Charlotte} (1895-03-24 – 1895-07-27)|pwv}\pwindex{Kotter, Caroline 1893-07-07 – 1964-07-01@\textsc{Kotter, Caroline} (1893-07-07 – 1964-07-01)|pwv} habe! Übrigens habe ich
               jetzt wieder acht Schreckenstage mitgemacht. Ich bin nämlich einmal doch erlegen, und
               so kamen dann die acht langen Tage. Endlich erschien die Gefahr doch beseitigt und
               ich atmete auf. Es wäre wirklich zu schrecklich gewesen. Übrigens verbringe ich nach
               dieser Seite hin arge Tage. Scenen, Scenen, Scenen. Wie einem \uline{da} zu Muthe wird, können nicht einmal Sie recht wissen. Es gibt
               gegenwärtig, besonders aber heute, keine Frau, die mir unausstehlicher wäre als meine
                  Geliebte\pwindex{Pohl-Glas, Charlotte 1873-01-01 – 1944-02-15@\textsc{Pohl-Glas, Charlotte} (1873-01-01 – 1944-02-15), \emph{Schriftstellerin, Politikerin, Sozialistin}|pwuv}.
               Sie hat übrigens gestern, als wir eine Stunde lang wortlos und wüthend nebeneinander
               saßen plötzlich gesagt: »Uns sollte man Knütteln auseinander jagen.« O, wie recht!
               Wir sind übrigens in ein Stadium getreten, in welchem jeder Streit sofort ausartet
               und nicht wieder gut zu machende gegenseitige Beschimpfungen hervorruft, ich tue
               nichts, um das zu mildern {\pb}und könnte es auch nicht. Intensiv denke ich ans Fortreisen, wo ich dann durch Ruhe
               und lieberen Umgang mich zu erholen, und ihr\pwindex{Pohl-Glas, Charlotte 1873-01-01 – 1944-02-15@\textsc{Pohl-Glas, Charlotte} (1873-01-01 – 1944-02-15), \emph{Schriftstellerin, Politikerin, Sozialistin}|pwuv} durch Briefe unsere
               Nichtzusammengehörigkeit eindringlich vorzustellen beabsichtige. Dass B.-H.\pwindex{Beer-Hofmann, Richard 1866-07-11 – 1945-09-26@\textsc{Beer-Hofmann, Richard} (1866-07-11 – 1945-09-26), \emph{Schriftsteller}|pw} erst Anfangs September
               fahren will ist fatal, aber da er den »Götterliebling\pwindex{Beer-Hofmann, Richard 1866-07-11 – 1945-09-26@\textsc{Beer-Hofmann, Richard} (1866-07-11 – 1945-09-26), \emph{Schriftsteller}!Tod Georgs1900@\strich\emph{Der Tod Georgs} {[}1900{]}|pw}« fertig macht, läßt sich nichts thun. Das ist jedenfalls
               wichtiger, und wenn er im Herbste erscheinen will soll er doch dazu schauen, noch
               diesen Monat (August) fertig zu werden. Mit mir steht die Sache so: Ich kann den
                  13. od. 14. August fort, \uline{muss} aber
               jedenfalls den 1. September zurück sein. Wenn wir zusammen reisen, dann müssten Sie
               sich \uline{längstens} bis 1. Aug. entschloßen haben, damit
               ich mich danach einrichten kann. Für diesen Fall käme ich \uline{nicht} nach Ischl\oindex{Bad Ischl@\textbf{Bad Ischl}|pw}, sondern wir träfen uns
               entweder in Wien\oindex{Wien@\textbf{Wien}|pw}, oder am 16. Aug. in Stettin\oindex{Szczecin@\textbf{Szczecin}|pw}, da ich auf 1 Tag nach der Insel Rügen\oindex{Ruegen@\textbf{Rügen}|pw} muss. \pend
           \pstart
           Nun aber folgendes: Moriz Rosenthal\pwindex{Rosenthal, Moritz 17.12.1862 – 03.09.1946@\textsc{Rosenthal, Moritz} (17.12.1862 – 03.09.1946), \emph{Komponist, Pianist}|pw}, den ich
               heute sprach, sagte mir, er könne nicht \uline{dringend
                  genug} vor Kopenhagen\oindex{Kopenhagen@\textbf{Kopenhagen}|pw} warnen. Es sei
               weder schön noch gut dort, ferner theuer, schlechte Bäder etc. Er rät Rügen\oindex{Ruegen@\textbf{Rügen}|pw} an, oder Sylt\oindex{Sylt@\textbf{Sylt}|pw}, \uline{gewiss nicht}Kopenhagen\oindex{Kopenhagen@\textbf{Kopenhagen}|pw}. Geht es noch, dass daran gerüttelt
               wird? \pend
           \pstart
           Ferner: wenn Sie nicht sehr gerne von {\pb}Ischl\oindex{Bad Ischl@\textbf{Bad Ischl}|pw} früher weggingen, als bis BH.\pwindex{Beer-Hofmann, Richard 1866-07-11 – 1945-09-26@\textsc{Beer-Hofmann, Richard} (1866-07-11 – 1945-09-26), \emph{Schriftsteller}|pw} fährt, oder auch die anderen\pwindex{Goldmann, Paul 31.01.1865 – 25.09.1935@\textsc{Goldmann, Paul} (31.01.1865 – 25.09.1935), \emph{Schriftsteller, Journalist}|pwv} in Kphg.\oindex{Kopenhagen@\textbf{Kopenhagen}|pw}
               eintreffen, bin ich auch bereit auf die Reise zu verzichten. Für diesen Fall käme ich
               dann am 13. oder 14. Aug. einfach nach Ischl\oindex{Bad Ischl@\textbf{Bad Ischl}|pw},
               ginge zum Leopold\oindex{Hotel und Pension Rudolfshoehe (Leopold Petter)@\textbf{Hotel und Pension Rudolfshöhe (Leopold Petter)}|pw}, nähme mein Bicycle mit, und
               bliebe ruhig bis 1. September dort. Wie es Ihnen angenehmer ist, mögen
               Sie nun entscheiden. Ich muß gestehen, dass es mir im Grunde gleich ist, wie u. wo
               ich die 14 Tage verbringe, ich möchte nur gerne \uline{rechtzeitig} wissen, (also bis 1. Aug.) was geschieht. Mir kommt es in meiner
               momentanen Verfassung lediglich darauf an überhaupt nur dort zu fahren und ein
               bisschen Ruhe zu haben. \pend
           \pstart
           Schreiben Sie bald und leben Sie recht wohl. Ich grüße Beer Hofmann\pwindex{Beer-Hofmann, Richard 1866-07-11 – 1945-09-26@\textsc{Beer-Hofmann, Richard} (1866-07-11 – 1945-09-26), \emph{Schriftsteller}|pw} und Sie \pend
           \pstart
           Herzlichst Ihr {\\[\baselineskip]}\spacefill\mbox{Salten}\pend
           \leftskip=0em{}
         
         \endnumbering\mylabel{h}\end{ledgroupsized}\begin{anhang}\end{anhang}\newcommand{\dateiname}{L03159}\newcommand{\titel}{Felix Salten an Arthur Schnitzler, 22. 7. 1895}\newcommand{\editorInnen}{Martin Anton Müller und Laura Untner}%% latex-leseansicht-abspann.tex
%% Abspann für die Leseansicht.
%% Der Schalter \ifkorrekturansicht ist bereits durch den Vorspann gesetzt.

%% latex-abspann.tex
%% Gemeinsamer Abspann für Korrekturansicht und Leseansicht.
%% Setzt den Schalter \ifkorrekturansicht voraus (gesetzt in den
%% einbindenden Dateien latex-korrekturansicht-abspann.tex bzw.
%% latex-leseansicht-abspann.tex).
%% ---------------------------------------------------------------

\normalsize

% Das esempio-Environment wird nur in der Leseansicht benötigt
\ifkorrekturansicht\else
\newenvironment{esempio}[3]%
{
    \vspace{1.5ex}
    \rlap{\underline{#1}}
    \par
    \setlength{\parindent}{0cm}
    \nopagebreak
    \leftskip=#2cm
    \rightskip=#3cm
}
{
    \par
}
\fi

\doendnotes{C}
\bigskip
\vfill

\clearpage

\footnotesize

\ifkorrekturansicht
  \lohead{\textsc{register}}
\fi

% theindex-Environment neu definieren ohne reledmac
\makeatletter
\renewenvironment{theindex}{%
  \ifkorrekturansicht
    \section*{\indexname}%
  \else
    \subsubsection*{Index der erwähnten Entitäten}%
  \fi
  \setlength{\parindent}{0pt}%
  \setlength{\parskip}{0pt plus 0.3pt}%
  \let\item\@idxitem
}{%
  \ifkorrekturansicht\clearpage\fi
}
\makeatother

\IfFileExists{\jobname-pw.ind}{\input{\jobname-pw.ind}}{}

% Quellenangabe nur in der Leseansicht
\ifkorrekturansicht\else
% Fallback-Definitionen, falls die .tex-Datei \titel etc. nicht gesetzt hat
\providecommand{\titel}{}
\providecommand{\editorInnen}{}
\providecommand{\dateiname}{\jobname}

\vspace{3cm}

\vfill

\footnotesize
\textsc{Quelle}: \titel. Herausgegeben von {\editorInnen}. In: \emph{Arthur Schnitzler: Briefwechsel mit Autorinnen und Autoren}.
 Digitale Edition, https://schnitzler-briefe.acdh.oeaw.ac.at/{\dateiname}.html (Stand \today)
\fi

\end{document}


      