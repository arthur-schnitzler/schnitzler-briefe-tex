%% latex-leseansicht-vorspann.tex
%% Vorspann für die Leseansicht.
%% Lädt die gemeinsame Datei latex-vorspann.tex mit nicht gesetztem Schalter.

\newif\ifkorrekturansicht
\korrekturansichtfalse

\input{../tex-inputs/latex-vorspann}


         
         \renewcommand{\erwaehntePersonen}{Personen: Richard Beer-Hofmann, Paul Goldmann, Hugo von Hofmannsthal, Caroline Kotter, Maria Charlotte Lamberg, Alexander Wilhelm Neuman, Charlotte Pohl-Glas, Moritz Rosenthal, Felix Salten, Hans Schnitzler, Julius Schnitzler, Helene Schnitzler}
         \renewcommand{\erwaehnteInstitutionen}{Institutionen: Frankfurter Zeitung}
         \renewcommand{\erwaehnteOrte}{Orte: Bad Ischl, Glaspalast, Hotel und Pension Rudolfshöhe (Leopold Petter), Kopenhagen, München, Paris, Rügen, Skandinavien, Sylt, Szczecin, Wien}
         \renewcommand{\erwaehnteWerke}{Werke: Der Tod Georgs, Die Münchener Kunstausstellungen. I. Im königl. Glaspalast, Die Münchener Kunstausstellungen. II. Im königl. Glaspalast, Münchener Brief. (Orig.-Corr. der »Wiener Allg. Ztg.«), Quer durch den Wurstelprater, Wiener Allgemeine Zeitung}
               \section[ Felix Salten an Arthur Schnitzler, 22. 7. 1895]{ Felix Salten an Arthur Schnitzler, 22. 7. 1895}\nopagebreak\mylabel{v}\rehead{ }\begin{ledgroupsized}[t]{13cm}\normalsize\beginnumbering \toendnotes[C]{\smallbreak\pagebreak[2]} \Standort{CUL, Schnitzler, B 89, A 1.}
\physDesc{Brief, 1 Blatt, 4 Seiten, 3989 Zeichen
\newline{}Handschrift: Bleistift, lateinische Kurrent
\newline{}Ordnung: mit Bleistift von unbekannter Hand nummeriert: »58« }\toendnotes[C]{\smallbreak}\pstart
           \raggedleft{}{\pb}22. Juli 95.\pend
           \pstart
           Lieber Freund! Das war sehr lieb von Ihnen, dass Sie
               mir mittheilten, ich werde oft gelobt, es hat mich sehr gefreut, denn ich begreife
               immer mehr, dass der Hugo\pwindex{Hofmannsthal, Hugo von 1874-02-01 – 1929-07-15@\textsc{Hofmannsthal, Hugo von} (1874-02-01 – 1929-07-15), \emph{Schriftsteller}|pw} recht hat, wenn er
               sagt: »Ich möcht mehr g’lobt werd’n.« Sie können sich vorstellen, welches Gewicht ich
               auf das Urtheil von Neumann\pwindex{Neuman, Alexander Wilhelm 03.05.1860 – 06.05.1936@\textsc{Neuman, Alexander Wilhelm} (03.05.1860 – 06.05.1936), \emph{Redakteur, Kaufmann}|pw} lege. Jetzt erst
               glaube ich, dass ich doch etwas kann. Ich habe mir jetzt meine \label{K_L03159-1v}\edtext{Feuilletons\pwindex{Salten, Felix 06.09.1869 – 08.10.1945@\textsc{Salten, Felix} (06.09.1869 – 08.10.1945), \emph{Schriftsteller, Journalist}!Muenchener Kunstausstellungen. I. Im koenigl. Glaspalast1895-07-24@\strich\emph{Die Münchener Kunstausstellungen. I. Im königl. Glaspalast} {[}1895-07-24{]}|pwv}\pwindex{Salten, Felix 06.09.1869 – 08.10.1945@\textsc{Salten, Felix} (06.09.1869 – 08.10.1945), \emph{Schriftsteller, Journalist}!Muenchener Kunstausstellungen. II. Im koenigl. Glaspalast1895-07-25@\strich\emph{Die Münchener Kunstausstellungen. II. Im königl. Glaspalast} {[}1895-07-25{]}|pwv}\pwindex{Muenchener Brief. (Orig.-Corr. der »Wiener Allg. Ztg.«)1895-07-06@\emph{Münchener Brief. (Orig.-Corr. der »Wiener Allg. Ztg.«)} {[}1895-07-06{]}|pwv}}{\lemma{\textnormal{\emph{Feuilletons}}}\Cendnote{\textnormal{Er dürfte sich im Besonderen auf die
                  aktuellen Texte über München\oindex{Muenchen@\textbf{München}|pwk}er
                  Kunstausstellungen im Glaspalast\oindex{Glaspalast@\textbf{Glaspalast}|pwk} beziehen: f. s.\pwindex{Salten, Felix 06.09.1869 – 08.10.1945@\textsc{Salten, Felix} (06.09.1869 – 08.10.1945), \emph{Schriftsteller, Journalist}|pwk}: \emph{Münchener Brief. (Orig.-Corr. der »Wiener Allg. Ztg.«)}\pwindex{Muenchener Brief. (Orig.-Corr. der »Wiener Allg. Ztg.«)1895-07-06@\emph{Münchener Brief. (Orig.-Corr. der »Wiener Allg. Ztg.«)} {[}1895-07-06{]}|pwk}. In: \emph{Wiener Allgemeinen Zeitung}\pwindex{?? Werk@Nicht ermittelte Verfasserinnen und Verfasser!Wiener Allgemeine Zeitung1.3.1880 – 11.2.1934@\emph{Wiener Allgemeine Zeitung} {[}1.3.1880 – 11.2.1934{]}|pwk}, Nr. 5.200,
                        6. 7. 1895, S. 8. Felix Salten\pwindex{Salten, Felix 06.09.1869 – 08.10.1945@\textsc{Salten, Felix} (06.09.1869 – 08.10.1945), \emph{Schriftsteller, Journalist}|pwk}: \emph{Die Münchener Kunstausstellungen. I. Im königl.
                        Glaspalast}\pwindex{Salten, Felix 06.09.1869 – 08.10.1945@\textsc{Salten, Felix} (06.09.1869 – 08.10.1945), \emph{Schriftsteller, Journalist}!Muenchener Kunstausstellungen. I. Im koenigl. Glaspalast1895-07-24@\strich\emph{Die Münchener Kunstausstellungen. I. Im königl. Glaspalast} {[}1895-07-24{]}|pwk}. In: \emph{Wiener Allgemeinen
                        Zeitung}\pwindex{?? Werk@Nicht ermittelte Verfasserinnen und Verfasser!Wiener Allgemeine Zeitung1.3.1880 – 11.2.1934@\emph{Wiener Allgemeine Zeitung} {[}1.3.1880 – 11.2.1934{]}|pwk}, Nr. 5.215, 24. 7. 1895,
                     S. 2. Felix Salten\pwindex{Salten, Felix 06.09.1869 – 08.10.1945@\textsc{Salten, Felix} (06.09.1869 – 08.10.1945), \emph{Schriftsteller, Journalist}|pwk}: Die Münchener Kunstausstellungen. II. Im
                        königl. Glaspalast\pwindex{Salten, Felix 06.09.1869 – 08.10.1945@\textsc{Salten, Felix} (06.09.1869 – 08.10.1945), \emph{Schriftsteller, Journalist}!Muenchener Kunstausstellungen. II. Im koenigl. Glaspalast1895-07-25@\strich\emph{Die Münchener Kunstausstellungen. II. Im königl. Glaspalast} {[}1895-07-25{]}|pwkv}. In: \emph{Wiener
                        Allgemeinen Zeitung}\pwindex{?? Werk@Nicht ermittelte Verfasserinnen und Verfasser!Wiener Allgemeine Zeitung1.3.1880 – 11.2.1934@\emph{Wiener Allgemeine Zeitung} {[}1.3.1880 – 11.2.1934{]}|pwk}, Nr. 5.216, 25. 7. 1895, S. 2–3. Die Zusammenstellung für Goldmann\pwindex{Goldmann, Paul 31.01.1865 – 25.09.1935@\textsc{Goldmann, Paul} (31.01.1865 – 25.09.1935), \emph{Schriftsteller, Journalist}|pwk} dürfte erfolgen, weil dieser als
                  Korrespondent für die \emph{Frankfurter Zeitung}\orgindex{Frankfurter Zeitung@Frankfurter Zeitung|pwk} in
                     Paris\oindex{Paris@\textbf{Paris}|pwk} tätig war.}}}\label{K_L03159-1h} zusammenstellen
               laßen, und schicke sie Ihnen morgen. Wählen Sie davon
               welche aus, und senden Sie das an Goldmann\pwindex{Goldmann, Paul 31.01.1865 – 25.09.1935@\textsc{Goldmann, Paul} (31.01.1865 – 25.09.1935), \emph{Schriftsteller, Journalist}|pw}
               weiter, ja? Dass ich \label{K_L03159-2v}\edtext{Beer Hofmann\pwindex{Beer-Hofmann, Richard 1866-07-11 – 1945-09-26@\textsc{Beer-Hofmann, Richard} (1866-07-11 – 1945-09-26), \emph{Schriftsteller}|pw} nichts geschrieben}{\lemma{\textnormal{\emph{Beer … geschrieben}}}\Cendnote{\textnormal{Der Versand des Zeitungsabdrucks \emph{Quer durch den Wurstelprater}\pwindex{Salten, Felix 06.09.1869 – 08.10.1945@\textsc{Salten, Felix} (06.09.1869 – 08.10.1945), \emph{Schriftsteller, Journalist}!Quer durch den Wurstelprater1895-06-02 – 1895-06-09@\strich\emph{Quer durch den Wurstelprater} {[}1895-06-02 – 1895-06-09{]}|pwk} (siehe Felix Salten an Arthur Schnitzler, 16. 7. [1895]) war also ohne
                  Begleitschreiben erfolgt.}}}\label{K_L03159-2h} habe, soll nicht missdeutet werden. Zu einem
               Brief lag rein äußerlich nichts vor, und ihm auf den Wurstelprater\pwindex{Salten, Felix 06.09.1869 – 08.10.1945@\textsc{Salten, Felix} (06.09.1869 – 08.10.1945), \emph{Schriftsteller, Journalist}!Quer durch den Wurstelprater1895-06-02 – 1895-06-09@\strich\emph{Quer durch den Wurstelprater} {[}1895-06-02 – 1895-06-09{]}|pw} eine Widmung schreiben, mochte ich nicht, weil ich ja nicht
               wusste, wie ihm der Wurstelprater\pwindex{Salten, Felix 06.09.1869 – 08.10.1945@\textsc{Salten, Felix} (06.09.1869 – 08.10.1945), \emph{Schriftsteller, Journalist}!Quer durch den Wurstelprater1895-06-02 – 1895-06-09@\strich\emph{Quer durch den Wurstelprater} {[}1895-06-02 – 1895-06-09{]}|pw} gefallen
               werde, und weil, – nun Sie wissen ja dass ich da vielleicht ein bisschen zu sehr
               empfindlich bin. Ich weiss ja auch heute nicht, ob er
                  \textcolor{gray}{w}as davon hält, und so konnte ich ihm bis heute nichts schreiben. Übrigens vermuthe ich, dass er
               ihm nicht gefallen hat, weil Sie mir das sonst sicher geschrieben hätten. Dabei kann
               ich aber nicht begreifen, {\pb}seit wann wir uns das \uline{nicht} mittheilen. Das Sie
               einen kleinen \label{K_L03159-3v}\edtext{Neffe\pwindex{Schnitzler, Hans 11.07.1895 – 26.03.1967@\textsc{Schnitzler, Hans} (11.07.1895 – 26.03.1967), \emph{Chirurg}|pwv}n }{\lemma{\textnormal{\emph{Neffen }}}\Cendnote{\textnormal{Hans Schnitzler\pwindex{Schnitzler, Hans 11.07.1895 – 26.03.1967@\textsc{Schnitzler, Hans} (11.07.1895 – 26.03.1967), \emph{Chirurg}|pwk}, der gemeinsame Sohn von
                     Julius\pwindex{Schnitzler, Julius 13.07.1865 – 29.06.1939@\textsc{Schnitzler, Julius} (13.07.1865 – 29.06.1939), \emph{Chirurg}|pwk} und Helene Schnitzler\pwindex{Schnitzler, Helene 16.07.1871 – September 1941@\textsc{Schnitzler, Helene} (16.07.1871 – September 1941)|pwk}, war am 11. 7. 1895 geboren worden.}}}\label{K_L03159-3h} haben, wusste ich, aber das kann mir
               doch nicht imponiren, da ich doch zwei Töchter\pwindex{Lamberg, Maria Charlotte 1895-03-24 – 1895-07-27@\textsc{Lamberg, Maria Charlotte} (1895-03-24 – 1895-07-27)|pwv}\pwindex{Kotter, Caroline 1893-07-07 – 1964-07-01@\textsc{Kotter, Caroline} (1893-07-07 – 1964-07-01)|pwv} habe! Übrigens habe ich jetzt wieder acht
               Schreckenstage mitgemacht. Ich bin nämlich einmal doch erlegen, und so kamen dann die
               acht langen Tage. Endlich erschien die Gefahr doch beseitigt und ich atmete auf. Es
               wäre wirklich zu schrecklich gewesen. Übrigens verbringe ich nach dieser Seite hin
               arge Tage. Scenen, Scenen, Scenen. Wie einem \uline{da} zu
               Muthe wird, können nicht einmal Sie recht wissen. Es gibt gegenwärtig, besonders aber
                  heute, keine Frau, die mir unausstehlicher wäre als
                  \label{K_L03159-4v}\edtext{meine Geliebte\pwindex{Pohl-Glas, Charlotte 1873-01-01 – 1944-02-15@\textsc{Pohl-Glas, Charlotte} (1873-01-01 – 1944-02-15), \emph{Schriftstellerin, Politikerin, Sozialistin}|pwuv}}{\lemma{\textnormal{\emph{meine Geliebte}}}\Cendnote{\textnormal{wohl Charlotte Glas\pwindex{Pohl-Glas, Charlotte 1873-01-01 – 1944-02-15@\textsc{Pohl-Glas, Charlotte} (1873-01-01 – 1944-02-15), \emph{Schriftstellerin, Politikerin, Sozialistin}|pwk}}}}\label{K_L03159-4h}. Sie hat übrigens gestern, als wir eine Stunde
               lang wortlos und wüthend nebeneinander saßen, plötzlich gesagt: »Uns sollte man mit
               Knütteln auseinander jagen.« O, wie recht! Wir sind übrigens in ein Stadium getreten,
               in welchem jeder Streit sofort ausartet und nicht wieder gutzumachende gegenseitige
               Beschimpfungen hervorruft, ich thue nichts, um das zu mildern\textcolor{gray}{,}{ }{\pb}und könnte es auch nicht.
               Intensiv denke ich ans Fortreisen, wo ich denn durch Ruhe und lieberen Umgang mich zu
               erholen, und ihr\pwindex{Pohl-Glas, Charlotte 1873-01-01 – 1944-02-15@\textsc{Pohl-Glas, Charlotte} (1873-01-01 – 1944-02-15), \emph{Schriftstellerin, Politikerin, Sozialistin}|pwuv} durch Briefe unsere Nichtzusammengehörigkeit eindringlich vorzustellen
               beabsichtige. Dass B.-H.\pwindex{Beer-Hofmann, Richard 1866-07-11 – 1945-09-26@\textsc{Beer-Hofmann, Richard} (1866-07-11 – 1945-09-26), \emph{Schriftsteller}|pw} erst \label{K_L03159-5v}\edtext{Anfangs September fahren}{\lemma{\textnormal{\emph{Anfangs September fahren}}}\Cendnote{\textnormal{Zu
                     Schnitzler\pwindex{Schnitzler, Arthur 15.05.1862 – 21.10.1931@\textsc{Schnitzler, Arthur} (15.05.1862 – 21.10.1931), \emph{Schriftsteller, Mediziner}|pwk}s erster Skandinavien\oindex{Skandinavien@\textbf{Skandinavien}|pwk}reise kam es erst ein Jahr später, im August 1896, aber ohne Salten\pwindex{Salten, Felix 06.09.1869 – 08.10.1945@\textsc{Salten, Felix} (06.09.1869 – 08.10.1945), \emph{Schriftsteller, Journalist}|pwk}, dafür mit Paul Goldmann\pwindex{Goldmann, Paul 31.01.1865 – 25.09.1935@\textsc{Goldmann, Paul} (31.01.1865 – 25.09.1935), \emph{Schriftsteller, Journalist}|pwk}
                  und Richard Beer-Hofmann\pwindex{Beer-Hofmann, Richard 1866-07-11 – 1945-09-26@\textsc{Beer-Hofmann, Richard} (1866-07-11 – 1945-09-26), \emph{Schriftsteller}|pwk}. Vgl. Felix Salten an Arthur Schnitzler, 16. 7. [1895]. }}}\label{K_L03159-5h} will ist
               fatal, aber da er den »Götterliebling\pwindex{Beer-Hofmann, Richard 1866-07-11 – 1945-09-26@\textsc{Beer-Hofmann, Richard} (1866-07-11 – 1945-09-26), \emph{Schriftsteller}!Tod Georgs1900@\strich\emph{Der Tod Georgs} {[}1900{]}|pw}« fertig
               macht, läßt sich nichts thun, das ist jedenfalls wichtiger, und wenn er im Herbste
               erscheinen will soll er doch dazu schauen, noch diesen Monat (August) fertig zu werden. Mit mir steht die Sache so: Ich kann den 13. od. 14. August fort;
                  \uline{muss} aber jedesfalls den 1. September zurück sein. Wenn wir zusammen reisen, dann müssten Sie sich
                  \uline{längstens} bis 1. Aug. entschloßen haben, damit ich mich danach einrichten kann. Für
               diesen Fall käme ich \uline{nicht} nach Ischl\oindex{Bad Ischl@\textbf{Bad Ischl}|pw}, sondern wir träfen uns entweder in Wien\oindex{Wien@\textbf{Wien}|pw}, oder \substVorne{}\textsuperscript{i\textcolor{gray}{n}}\substDazwischen{}am\substHinten{}{ }16. Aug. in Stettin\oindex{Szczecin@\textbf{Szczecin}|pw}, da ich auf 1 Tag nach der Insel Rügen\oindex{Ruegen@\textbf{Rügen}|pw} muss. Nun aber folgendes: Moriz
                  Rosenthal\pwindex{Rosenthal, Moritz 17.12.1862 – 03.09.1946@\textsc{Rosenthal, Moritz} (17.12.1862 – 03.09.1946), \emph{Komponist, Pianist}|pw}, den ich heute sprach, sagte mir, er
               könne nicht \uline{dringend genug} vor Kopenhagen\oindex{Kopenhagen@\textbf{Kopenhagen}|pw} warnen. Es sei weder schön noch gut dort, ferner
               theuer, schlechte Bäder ec. Er räth Rügen\oindex{Ruegen@\textbf{Rügen}|pw} an,
               oder Sylt\oindex{Sylt@\textbf{Sylt}|pw}, \uline{gewiss
                  nicht}{ }Kopenhagen\oindex{Kopenhagen@\textbf{Kopenhagen}|pw}. Geht es noch, dass daran gerüttelt
               wird? Ferner: Wenn Sie nicht sehr gerne von {\pb}\label{K_L03159-6v}\edtext{Ischl\oindex{Bad Ischl@\textbf{Bad Ischl}|pw}}{\lemma{\textnormal{\emph{Ischl}}}\Cendnote{\textnormal{Schnitzler\pwindex{Schnitzler, Arthur 15.05.1862 – 21.10.1931@\textsc{Schnitzler, Arthur} (15.05.1862 – 21.10.1931), \emph{Schriftsteller, Mediziner}|pwk} war, abgesehen von einer kurzen
                  Unterbrechung, zwischen 15. 7. 1895 und 19. 8. 1895 in Ischl\oindex{Bad Ischl@\textbf{Bad Ischl}|pwk}. Danach
                  machte er mit Salten\pwindex{Salten, Felix 06.09.1869 – 08.10.1945@\textsc{Salten, Felix} (06.09.1869 – 08.10.1945), \emph{Schriftsteller, Journalist}|pwk} eine Radtour nach München\oindex{Muenchen@\textbf{München}|pwk}, wo er bis 6. 8. 1895
                  blieb.}}}\label{K_L03159-6h} früher weggingen, als bis BH.\pwindex{Beer-Hofmann, Richard 1866-07-11 – 1945-09-26@\textsc{Beer-Hofmann, Richard} (1866-07-11 – 1945-09-26), \emph{Schriftsteller}|pw}
               fährt, oder auch \label{K_L03159-7v}\edtext{die Anderen\pwindex{Goldmann, Paul 31.01.1865 – 25.09.1935@\textsc{Goldmann, Paul} (31.01.1865 – 25.09.1935), \emph{Schriftsteller, Journalist}|pwv}}{\lemma{\textnormal{\emph{die Anderen}}}\Cendnote{\textnormal{jedenfalls Bezug auf Paul Goldmann\pwindex{Goldmann, Paul 31.01.1865 – 25.09.1935@\textsc{Goldmann, Paul} (31.01.1865 – 25.09.1935), \emph{Schriftsteller, Journalist}|pwk}}}}\label{K_L03159-7h} in Kphg.\oindex{Kopenhagen@\textbf{Kopenhagen}|pw} eintreffen, bin ich auch
               bereit auf die Reise zu verzichten. Für diesen Fall \substVorne{}\textsuperscript{\textcolor{gray}{könnte}}{\allowbreak}\substDazwischen{}käme\substHinten{} ich dann am 13. oder 14. Aug. einfach nach Ischl\oindex{Bad Ischl@\textbf{Bad Ischl}|pw}, ginge zum
                  Leopold\oindex{Hotel und Pension Rudolfshoehe (Leopold Petter)@\textbf{Hotel und Pension Rudolfshöhe (Leopold Petter)}|pw}, nähme mein Bicycle mit, und bliebe
               ruhig bis 1. September dort. Wie es Ihnen angenehmer
               ist, mögen Sie nun entscheiden. Ich muß gestehen, dass es mir im Grunde gleich ist,
               wie u. wo ich die 14 Tage verbringe, ich möchte nur gerne \uline{rechtzeitig} wissen, (also bis 1. Aug.) was
               geschieht. Mir kommt es in meiner momentanen Verfassung lediglich darauf an überhaupt
               nur fort zu \substVorne{}\textsuperscript{\textcolor{gray}{ko}}\substDazwischen{}fa\substHinten{}hren, und ein bischen Ruhe zu haben.\pend
           \pstart
           Schreiben \substVorne{}\textsuperscript{sich g}{\allowbreak}\substDazwischen{}Sie b\substHinten{}ald und \uline{leben} Sie recht wol. Ich grüße Beer Hofmann\pwindex{Beer-Hofmann, Richard 1866-07-11 – 1945-09-26@\textsc{Beer-Hofmann, Richard} (1866-07-11 – 1945-09-26), \emph{Schriftsteller}|pw} und Sie {\\[\baselineskip]}herzlichst Ihr {\\[\baselineskip]}\spacefill\mbox{Salten}\pend
           \leftskip=0em{}
         
         \endnumbering\mylabel{h}\end{ledgroupsized}  \newcommand{\dateiname}{L03159}\newcommand{\titel}{Felix Salten an Arthur Schnitzler, 22. 7. 1895}\newcommand{\editorInnen}{Martin Anton Müller und Laura Untner}%% latex-leseansicht-abspann.tex
%% Abspann für die Leseansicht.
%% Der Schalter \ifkorrekturansicht ist bereits durch den Vorspann gesetzt.

%% latex-abspann.tex
%% Gemeinsamer Abspann für Korrekturansicht und Leseansicht.
%% Setzt den Schalter \ifkorrekturansicht voraus (gesetzt in den
%% einbindenden Dateien latex-korrekturansicht-abspann.tex bzw.
%% latex-leseansicht-abspann.tex).
%% ---------------------------------------------------------------

\normalsize

% Das esempio-Environment wird nur in der Leseansicht benötigt
\ifkorrekturansicht\else
\newenvironment{esempio}[3]%
{
    \vspace{1.5ex}
    \rlap{\underline{#1}}
    \par
    \setlength{\parindent}{0cm}
    \nopagebreak
    \leftskip=#2cm
    \rightskip=#3cm
}
{
    \par
}
\fi

\doendnotes{C}
\bigskip
\vfill

\clearpage

\footnotesize

\ifkorrekturansicht
  \lohead{\textsc{register}}
\fi

% theindex-Environment neu definieren ohne reledmac
\makeatletter
\renewenvironment{theindex}{%
  \ifkorrekturansicht
    \section*{\indexname}%
  \else
    \subsubsection*{Index der erwähnten Entitäten}%
  \fi
  \setlength{\parindent}{0pt}%
  \setlength{\parskip}{0pt plus 0.3pt}%
  \let\item\@idxitem
}{%
  \ifkorrekturansicht\clearpage\fi
}
\makeatother

\IfFileExists{\jobname-pw.ind}{\input{\jobname-pw.ind}}{}

% Quellenangabe nur in der Leseansicht
\ifkorrekturansicht\else
% Fallback-Definitionen, falls die .tex-Datei \titel etc. nicht gesetzt hat
\providecommand{\titel}{}
\providecommand{\editorInnen}{}
\providecommand{\dateiname}{\jobname}

\vspace{3cm}

\vfill

\footnotesize
\textsc{Quelle}: \titel. Herausgegeben von {\editorInnen}. In: \emph{Arthur Schnitzler: Briefwechsel mit Autorinnen und Autoren}.
 Digitale Edition, https://schnitzler-briefe.acdh.oeaw.ac.at/{\dateiname}.html (Stand \today)
\fi

\end{document}


      