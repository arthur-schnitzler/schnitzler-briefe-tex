%% latex-leseansicht-vorspann.tex
%% Vorspann für die Leseansicht.
%% Lädt die gemeinsame Datei latex-vorspann.tex mit nicht gesetztem Schalter.

\newif\ifkorrekturansicht
\korrekturansichtfalse

\input{../tex-inputs/latex-vorspann}


         
         \renewcommand{\erwaehntePersonen}{Personen: Richard Beer-Hofmann, Georges Courteline, Clemens von Franckenstein, Paul Naschauer, Felix Salten, Siegfried Trebitsch}
         \renewcommand{\erwaehnteInstitutionen}{Institutionen: Wiener Schachclub}
         \renewcommand{\erwaehnteOrte}{Orte: Raimund-Theater, Wien}
         \renewcommand{\erwaehnteWerke}{Werke: Boubouroche. Lebensbild in 2 Acten, Boubouroche. Pièce en deux actes, Tagebuch}
               \section[ Felix Salten an Arthur Schnitzler, {[}18. 11. 1899{]}]{ Felix Salten an Arthur Schnitzler, {[}18. 11. 1899{]}}\nopagebreak\mylabel{v}\rehead{ }\begin{ledgroupsized}[t]{13cm}\normalsize\beginnumbering \toendnotes[C]{\smallbreak\pagebreak[2]} \Standort{CUL, Schnitzler, B 89, A 2.}
\physDesc{Brief, 1 Blatt, 2 Seiten, 428 Zeichen
\newline{}Handschrift: Bleistift, lateinische Kurrent
\newline{}Schnitzler: mit Bleistift datiert: »18/1\textcolor{gray}{1} 99.« 
\newline{}Ordnung: mit Bleistift von unbekannter Hand nummeriert: »126« }\toendnotes[C]{\smallbreak}\pstart{}{\pb}Lieber Freund,\pend\pstart
           \label{K_L03302-1v}\edtext{Morgen, längstens Dienstag}{\lemma{\textnormal{\emph{Morgen, … Dienstag}}}\Cendnote{\textnormal{Die zweite Ziffer der Monatsangabe von
                     Schnitzler\pwindex{Schnitzler, Arthur 15.05.1862 – 21.10.1931@\textsc{Schnitzler, Arthur} (15.05.1862 – 21.10.1931), \emph{Schriftsteller, Mediziner}|pwk}s Datierung könnte auch als »2«
                  gelesen werden. Das lässt sich aber durch den Inhalt ausschließen, da der
                     19. 12. 1899 ein Dienstag war und die Unterscheidung zwischen
                  »morgen«/»Dienstag« keinen Sinn ergäbe.}}}\label{K_L03302-1h} bringe ich Ihnen \label{K_L03302-2v}\edtext{»Boubouroch\pwindex{Courteline, Georges 25.06.1858 – 25.06.1929@\textsc{Courteline, Georges} (25.06.1858 – 25.06.1929), \emph{Schriftsteller}!Boubouroche. Lebensbild in 2 Acten1900-01-31@\strich\emph{Boubouroche. Lebensbild in 2 Acten} {[}1900-01-31{]}|pw}.«}{\lemma{\textnormal{\emph{»Boubouroch.«}}}\Cendnote{\textnormal{Es dürfte sich um
                  eine französischsprachige Ausgabe von Georges
                     Courteline\pwindex{Courteline, Georges 25.06.1858 – 25.06.1929@\textsc{Courteline, Georges} (25.06.1858 – 25.06.1929), \emph{Schriftsteller}|pwk}s \emph{Boubouroche}\pwindex{Courteline, Georges 25.06.1858 – 25.06.1929@\textsc{Courteline, Georges} (25.06.1858 – 25.06.1929), \emph{Schriftsteller}!Boubouroche. Piece en deux actes1893-04-27@\strich\emph{Boubouroche. Pièce en deux actes} {[}1893-04-27{]}|pwk} gehandelt
                  haben. Durch die zeitliche Nähe zur deutschsprachigen Premiere am 31. 1. 1900 im Raimundtheater\oindex{Raimund-Theater@\textbf{Raimund-Theater}|pwk} wäre auch ein Bühnenmanuskript
                  der Bearbeitung\pwindex{Courteline, Georges 25.06.1858 – 25.06.1929@\textsc{Courteline, Georges} (25.06.1858 – 25.06.1929), \emph{Schriftsteller}!Boubouroche. Lebensbild in 2 Acten1900-01-31@\strich\emph{Boubouroche. Lebensbild in 2 Acten} {[}1900-01-31{]}|pwkv} von Siegfried Trebitsch\pwindex{Trebitsch, Siegfried 22.12.1868 – 03.06.1956@\textsc{Trebitsch, Siegfried} (22.12.1868 – 03.06.1956), \emph{Schriftsteller, Übersetzer}|pwk} denkbar. Schnitzler\pwindex{Schnitzler, Arthur 15.05.1862 – 21.10.1931@\textsc{Schnitzler, Arthur} (15.05.1862 – 21.10.1931), \emph{Schriftsteller, Mediziner}|pwk} besuchte wenige Wochen später die
                  Premiere.}}}\label{K_L03302-2h} Ich glaube, die practischen Zwecke zu kennen, u. wenn ich mich
               nicht irre, sind sie sehr gut. In den \label{K_L03302-3v}\edtext{Club\orgindex{Wiener Schachclub@Wiener Schachclub|pwv}}{\lemma{\textnormal{\emph{Club}}}\Cendnote{\textnormal{Gemeint ist der \emph{Wiener Schachclub}\orgindex{Wiener Schachclub@Wiener Schachclub|pwk}, dem in den kommenden Wochen neben Salten\pwindex{Salten, Felix 06.09.1869 – 08.10.1945@\textsc{Salten, Felix} (06.09.1869 – 08.10.1945), \emph{Schriftsteller, Journalist}|pwk} und Schnitzler\pwindex{Schnitzler, Arthur 15.05.1862 – 21.10.1931@\textsc{Schnitzler, Arthur} (15.05.1862 – 21.10.1931), \emph{Schriftsteller, Mediziner}|pwk} auch Beer-Hofmann\pwindex{Beer-Hofmann, Richard 1866-07-11 – 1945-09-26@\textsc{Beer-Hofmann, Richard} (1866-07-11 – 1945-09-26), \emph{Schriftsteller}|pwk} und
                     Hofmannsthal\pwindex{\textcolor{red}{\textsuperscript{XXXX1 indx}}|pwk} beitraten. Am
                     1. 1. 1900 brachte die \emph{Wiener Schachzeitung}\textcolor{red}{\textsuperscript{XXXX indx}} ihre Namen als bei der letzten Sitzung neu
                  aufgenommene Mitglieder. Paul Naschauer\pwindex{Naschauer, Paul 06.09.1866 – 20.05.1900@\textsc{Naschauer, Paul} (06.09.1866 – 20.05.1900)|pwk} und
                     Leo van Jung\pwindex{\textcolor{red}{\textsuperscript{XXXX1 indx}}|pwk} wurden im gleichen Blatt am
                     12. 12. 1899 als neue Mitglieder kenntlich gemacht,
                  dürften also die Verbindungspersonen zum Club\orgindex{Wiener Schachclub@Wiener Schachclub|pwkv} gewesen sein.}}}\label{K_L03302-3h} trete ich ein, wenn die
               Anmeldung collectiv erfolgt, und wenn die \label{K_L03302-4v}\edtext{60 fl. honoris causa}{\lemma{\textnormal{\emph{60 fl. honoris causa}}}\Cendnote{\textnormal{Die Beitrittsgebühr entspricht im Jahr 2024 einem Wert von
                  1.000 €.}}}\label{K_L03302-4h} nachgelaßen {\pb}werden. In diesem Fall, bitte, melden Sie mich an, da ich ja doch Naschauer\pwindex{Naschauer, Paul 06.09.1866 – 20.05.1900@\textsc{Naschauer, Paul} (06.09.1866 – 20.05.1900)|pw} nicht so bald spreche.\pend
           \pstart
           Auf \label{K_L03302-5v}\edtext{Wiedersehen also morgen}{\lemma{\textnormal{\emph{Wiedersehen also morgen}}}\Cendnote{\textnormal{Im \emph{Tagebuch}\pwindex{\textcolor{red}{\textsuperscript{XXXX1 indx}}!Tagebuch1981 – 2000@\strich\emph{Tagebuch} {[}Hrsg., 1981 – 2000{]}|pwk} wird für den 19. 11. 1899 kein Treffen vermerkt. Eventuell aber wollte man sich bei
                     Beer-Hofmann\pwindex{Beer-Hofmann, Richard 1866-07-11 – 1945-09-26@\textsc{Beer-Hofmann, Richard} (1866-07-11 – 1945-09-26), \emph{Schriftsteller}|pwk} treffen, vgl. Hugo von Hofmannsthal an Arthur Schnitzler, [19. 11. 1899?].}}}\label{K_L03302-5h}, \label{K_L03302-6v}\edtext{längstens Dienstag}{\lemma{\textnormal{\emph{längstens Dienstag}}}\Cendnote{\textnormal{Ebenfalls nicht im \emph{Tagebuch}\pwindex{\textcolor{red}{\textsuperscript{XXXX1 indx}}!Tagebuch1981 – 2000@\strich\emph{Tagebuch} {[}Hrsg., 1981 – 2000{]}|pwk} erwähnt. Eventuell sahen sie sich am Dienstag,
                  dem 21. 11. 1899, im
                  Konzert von Clemens von Franckenstein\pwindex{Franckenstein, Clemens von 14.07.1875 – 19.08.1942@\textsc{Franckenstein, Clemens von} (14.07.1875 – 19.08.1942), \emph{Theaterleiter, Komponist, Dirigent}|pwk}, das
                     Schnitzler\pwindex{Schnitzler, Arthur 15.05.1862 – 21.10.1931@\textsc{Schnitzler, Arthur} (15.05.1862 – 21.10.1931), \emph{Schriftsteller, Mediziner}|pwk} besuchte.}}}\label{K_L03302-6h},
               {\\[\baselineskip]}herzlichst {\\[\baselineskip]}Ihr {\\[\baselineskip]}\spacefill\mbox{Salten}\pend
           \leftskip=0em{}
         
         \endnumbering\mylabel{h}\end{ledgroupsized}  \newcommand{\dateiname}{L03302}\newcommand{\titel}{Felix Salten an Arthur Schnitzler, [18. 11. 1899]}\newcommand{\editorInnen}{Martin Anton Müller und Laura Untner}%% latex-leseansicht-abspann.tex
%% Abspann für die Leseansicht.
%% Der Schalter \ifkorrekturansicht ist bereits durch den Vorspann gesetzt.

%% latex-abspann.tex
%% Gemeinsamer Abspann für Korrekturansicht und Leseansicht.
%% Setzt den Schalter \ifkorrekturansicht voraus (gesetzt in den
%% einbindenden Dateien latex-korrekturansicht-abspann.tex bzw.
%% latex-leseansicht-abspann.tex).
%% ---------------------------------------------------------------

\normalsize

% Das esempio-Environment wird nur in der Leseansicht benötigt
\ifkorrekturansicht\else
\newenvironment{esempio}[3]%
{
    \vspace{1.5ex}
    \rlap{\underline{#1}}
    \par
    \setlength{\parindent}{0cm}
    \nopagebreak
    \leftskip=#2cm
    \rightskip=#3cm
}
{
    \par
}
\fi

\doendnotes{C}
\bigskip
\vfill

\clearpage

\footnotesize

\ifkorrekturansicht
  \lohead{\textsc{register}}
\fi

% theindex-Environment neu definieren ohne reledmac
\makeatletter
\renewenvironment{theindex}{%
  \ifkorrekturansicht
    \section*{\indexname}%
  \else
    \subsubsection*{Index der erwähnten Entitäten}%
  \fi
  \setlength{\parindent}{0pt}%
  \setlength{\parskip}{0pt plus 0.3pt}%
  \let\item\@idxitem
}{%
  \ifkorrekturansicht\clearpage\fi
}
\makeatother

\IfFileExists{\jobname-pw.ind}{\input{\jobname-pw.ind}}{}

% Quellenangabe nur in der Leseansicht
\ifkorrekturansicht\else
% Fallback-Definitionen, falls die .tex-Datei \titel etc. nicht gesetzt hat
\providecommand{\titel}{}
\providecommand{\editorInnen}{}
\providecommand{\dateiname}{\jobname}

\vspace{3cm}

\vfill

\footnotesize
\textsc{Quelle}: \titel. Herausgegeben von {\editorInnen}. In: \emph{Arthur Schnitzler: Briefwechsel mit Autorinnen und Autoren}.
 Digitale Edition, https://schnitzler-briefe.acdh.oeaw.ac.at/{\dateiname}.html (Stand \today)
\fi

\end{document}


      