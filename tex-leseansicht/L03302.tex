%% latex-korrekturansicht-vorspann.tex
%% Vorspann für die Korrekturansicht.
%% Lädt die gemeinsame Datei latex-vorspann.tex mit gesetztem Schalter.

\newif\ifkorrekturansicht
\korrekturansichttrue

\input{../tex-inputs/latex-vorspann}


\section[ Felix Salten an Arthur Schnitzler, {[}18. 11. 1899{]}]{L03302 Felix Salten an Arthur Schnitzler, {[}18. 11. 1899{]}}
\nopagebreak\mylabel{L03302v}
\rehead{ }\normalsize\beginnumbering\briefempfaengerindex{Schnitzler, Arthur@\textsc{Schnitzler, Arthur}!zzzSalten, Felix@\emph{von Felix Salten}!1899-11-182@{{[}18. 11. 1899{]}}|(be}
\toendnotes[C]{\smallbreak\pagebreak[2]}\Standort{CUL, Schnitzler, B 89, A 2.}
\physDesc{Brief, 1 Blatt, 2 Seiten, 428 Zeichen
\newline{}Handschrift: Bleistift, lateinische Kurrent
\newline{}Schnitzler: mit Bleistift datiert: »18/1\textcolor{gray}{1} 99.« 
\newline{}Ordnung: mit Bleistift von unbekannter Hand nummeriert: »126« }\toendnotes[C]{\smallbreak}
\pstart{}{\pb}Lieber Freund,\pend\vspace{0.5em}
\pstart
           \label{K_L03302-1v}\edtext{Morgen, längstens Dienstag}{\lemma{\textnormal{\emph{Morgen, … Dienstag}}}\Cendnote{\textnormal{Die zweite Ziffer der Monatsangabe von
                     Schnitzlers Datierung könnte auch als »2«
                  gelesen werden. Das lässt sich aber durch den Inhalt ausschließen, da der
                     19. 12. 1899 ein Dienstag war und die Unterscheidung zwischen
                  »morgen«/»Dienstag« keinen Sinn ergäbe.}}}\label{K_L03302-1} bringe ich Ihnen \label{K_L03302-2v}\edtext{»Boubouroch\pwindex{Boubouroche. Lebensbild in 2 Acten@\emph{Boubouroche. Lebensbild in 2 Acten}|pw}.«}{\lemma{\textnormal{\emph{»Boubouroch.«}}}\Cendnote{\textnormal{Es dürfte sich um
                  eine französischsprachige Ausgabe von Georges
                     Courtelines\pwindex{Courteline, Georges 25.06.1858 – 25.06.1929@\textsc{Courteline, Georges} (25.06.1858 – 25.06.1929), \emph{Schriftsteller/Schriftstellerin}|pwk}{ }\emph{Boubouroche}\pwindex{Boubouroche. Piece en deux actes@\emph{Boubouroche. Pièce en deux actes}|pwk} gehandelt
                  haben. Durch die zeitliche Nähe zur deutschsprachigen Premiere am 31. 1. 1900 im Raimundtheater\oindex{Raimund-Theater@\textbf{Raimund-Theater}, \emph{Theater (K.THE)}|pwk} wäre auch ein Bühnenmanuskript
                  der Bearbeitung\pwindex{Boubouroche. Lebensbild in 2 Acten@\emph{Boubouroche. Lebensbild in 2 Acten}|pwkv} von Siegfried Trebitsch\pwindex{Trebitsch, Siegfried 22.12.1868 – 03.06.1956@\textsc{Trebitsch, Siegfried} (22.12.1868 – 03.06.1956), \emph{Schriftsteller/Schriftstellerin, Übersetzer/Übersetzerin}|pwk} denkbar. Schnitzler besuchte wenige Wochen später die
                  Premiere.}}}\label{K_L03302-2} Ich glaube, die practischen Zwecke zu kennen, u. wenn ich mich
               nicht irre, sind sie sehr gut. In den \label{K_L03302-3v}\edtext{Club\orgindex{Wiener Schachclub@Wiener Schachclub|pwv}}{\lemma{\textnormal{\emph{Club}}}\Cendnote{\textnormal{Gemeint ist der \emph{Wiener Schachclub}\orgindex{Wiener Schachclub@Wiener Schachclub|pwk}, dem in den kommenden Wochen neben Salten\pwindex{Salten, Felix 06.09.1869 – 08.10.1945@\textsc{Salten, Felix} (06.09.1869 – 08.10.1945), \emph{Schriftsteller/Schriftstellerin, Journalist/Journalistin, Chefredakteur/Chefredakteurin}|pwk} und Schnitzler auch Beer-Hofmann\pwindex{Beer-Hofmann, Richard 1866-07-11 – 1945-09-26@\textsc{Beer-Hofmann, Richard} (1866-07-11 – 1945-09-26), \emph{Schriftsteller/Schriftstellerin}|pwk} und
                     Hofmannsthal\pwindex{Hofmannsthal, Hugo von 1874-02-01 – 1929-07-15@\textsc{Hofmannsthal, Hugo von} (1874-02-01 – 1929-07-15), \emph{Schriftsteller/Schriftstellerin}|pwk} beitraten. Am
                     1. 1. 1900 brachte die \emph{Wiener Schachzeitung}\pwindex{Wiener Schachzeitung@\emph{Wiener Schachzeitung}|pwk} ihre Namen als bei der letzten Sitzung neu
                  aufgenommene Mitglieder. Paul Naschauer\pwindex{Naschauer, Paul 06.09.1866 – 20.05.1900@\textsc{Naschauer, Paul} (06.09.1866 – 20.05.1900)|pwk} und
                     Leo van Jung\pwindex{Van-Jung, Leo 15.10.1866 – 02.07.1939@\textsc{Van-Jung, Leo} (15.10.1866 – 02.07.1939), \emph{Gesangspädagoge/Gesangspädagogin, Mathematiker/Mathematikerin}|pwk} wurden im gleichen Blatt am
                     12. 12. 1899 als neue Mitglieder verlautbart,
                  dürften also die Verbindungspersonen zum Club\orgindex{Wiener Schachclub@Wiener Schachclub|pwkv} gewesen sein.}}}\label{K_L03302-3} trete ich ein, wenn die
               Anmeldung collectiv erfolgt, und wenn die \label{K_L03302-4v}\edtext{60 fl. honoris causa}{\lemma{\textnormal{\emph{60 fl. honoris causa}}}\Cendnote{\textnormal{Die Beitrittsgebühr entspricht im Jahr 2024 einem Wert von
                  1000 €.}}}\label{K_L03302-4} nachgelaßen {\pb}werden. In diesem Fall, bitte, melden Sie mich an, da ich ja doch Naschauer\pwindex{Naschauer, Paul 06.09.1866 – 20.05.1900@\textsc{Naschauer, Paul} (06.09.1866 – 20.05.1900)|pw} nicht so bald spreche.\pend
           
\pstart
           Auf \label{K_L03302-5v}\edtext{Wiedersehen also morgen}{\lemma{\textnormal{\emph{Wiedersehen also morgen}}}\Cendnote{\textnormal{Im \emph{Tagebuch}\pwindex{Tagebuch@\emph{Tagebuch}|pwk} wird für den 19. 11. 1899 kein Treffen vermerkt. Eventuell aber wollte man sich bei
                     Beer-Hofmann\pwindex{Beer-Hofmann, Richard 1866-07-11 – 1945-09-26@\textsc{Beer-Hofmann, Richard} (1866-07-11 – 1945-09-26), \emph{Schriftsteller/Schriftstellerin}|pwk} treffen, vgl. Hugo von Hofmannsthal an Arthur Schnitzler, [19. 11. 1899?].}}}\label{K_L03302-5}, \label{K_L03302-6v}\edtext{längstens Dienstag}{\lemma{\textnormal{\emph{längstens Dienstag}}}\Cendnote{\textnormal{Ebenfalls nicht im \emph{Tagebuch}\pwindex{Tagebuch@\emph{Tagebuch}|pwk} erwähnt. Eventuell sahen sie sich am Dienstag,
                  dem 21. 11. 1899, im
                  Konzert von Clemens von Franckenstein\pwindex{Franckenstein, Clemens von 14.07.1875 – 19.08.1942@\textsc{Franckenstein, Clemens von} (14.07.1875 – 19.08.1942), \emph{Theaterleiter/Theaterleiterin, Komponist/Komponistin, Dirigent/Dirigentin}|pwk}, das
                     Schnitzler besuchte.}}}\label{K_L03302-6},
               {\\[\baselineskip]}herzlichst {\\[\baselineskip]}Ihr {\\[\baselineskip]}\spacefill\mbox{Salten}\pend
           \leftskip=0em{}\selectlanguage{ngerman}\endnumbering\briefempfaengerindex{Schnitzler, Arthur@\textsc{Schnitzler, Arthur}!zzzSalten, Felix@\emph{von Felix Salten}!1899-11-182@{{[}18. 11. 1899{]}}|)be}\mylabel{L03302h}  \normalsize

\doendnotes{C}
\bigskip
\vfill

\clearpage

\footnotesize

\lohead{\textsc{register}}

% Definiere theindex-Environment komplett neu ohne reledmac
\makeatletter
\renewenvironment{theindex}{%
  \section*{\indexname}%
  \setlength{\parindent}{0pt}%
  \setlength{\parskip}{0pt plus 0.3pt}%
  \let\item\@idxitem
}{%
  \clearpage
}
\makeatother

\IfFileExists{\jobname-pw.ind}{\input{\jobname-pw.ind}}{}

\end{document}

      