%% latex-leseansicht-vorspann.tex
%% Vorspann für die Leseansicht.
%% Lädt die gemeinsame Datei latex-vorspann.tex mit nicht gesetztem Schalter.

\newif\ifkorrekturansicht
\korrekturansichtfalse

\input{../tex-inputs/latex-vorspann}

\begin{center}
            \textcolor{red}{ENTWURF, NICHT FERTIG KORRIGIERT}
                      \end{center}
            
         
         \renewcommand{\erwaehntePersonen}{Personen: Paul Naschauer}
         \renewcommand{\erwaehnteInstitutionen}{Institutionen: Wiener Schachclub}
         \renewcommand{\erwaehnteOrte}{Orte: Wien}
         \renewcommand{\erwaehnteWerke}{Werke: Boubouroche. Tragische Posse in zwei Akten}
               \section[Felix Salten an Arthur Schnitzler, {[}18. 11. 1899{]}]{ Felix Salten an Arthur Schnitzler, {[}18. 11. 1899{]}}\nopagebreak\mylabel{v}\rehead{ }\begin{ledgroupsized}[t]{13cm}\normalsize\beginnumbering \toendnotes[C]{\smallbreak\pagebreak[2]} \Standort{CUL, Schnitzler, B 89, A 2.}
\physDesc{Brief, 1 Blatt, 2 Seiten
\newline{}Handschrift: Bleistift, lateinische Kurrent
\newline{}Schnitzler: mit Bleistift datiert:
                                    »18/11 99.« \newline{}Ordnung: mit Bleistift von unbekannter Hand nummeriert:
                                    »126« }\toendnotes[C]{\smallbreak}\pstart{}{\pb}Lieber Freund,
               \pend\pstart
           Morgen, längstens Dienstag bringe ich Ihnen »Boubouroche\pwindex{\textcolor{red}{\textsuperscript{XXXX1 indx}}!Boubouroche. Tragische Posse in zwei Akten1900-01-31@\strich\emph{Boubouroche. Tragische Posse in zwei Akten} {[}1900-01-31{]}|pw}.« Ich glaube, die practischen Zwecke
               zu kennen, u. wenn ich mich nicht irre, sind sie sehr gut. In den Club\orgindex{Wiener Schachclub@Wiener Schachclub|pwv} trete ich ein, wenn die Anmeldung
               collectiv erfolgt, und wenn die 60fl. honoris causa nachgelaßen {\pb}werden. In diesem Fall, bitte,
               melden Sie mich an, da ich ja doch Naschauer\pwindex{Naschauer, Paul 06.09.1866 – 20.05.1900@\textsc{Naschauer, Paul} (06.09.1866 – 20.05.1900)|pw}
               nicht so bald spreche. Auf Wiedersehen also morgen, längstens Dienstag, \pend
           \pstart
           herzlichst {\\[\baselineskip]}Ihr {\\[\baselineskip]}\spacefill\mbox{Salten}\pend
           \leftskip=0em{}
         
         \endnumbering\mylabel{h}\end{ledgroupsized}\begin{anhang}\end{anhang}\newcommand{\dateiname}{L03302}\newcommand{\titel}{Felix Salten an Arthur Schnitzler, [18. 11. 1899]}\newcommand{\editorInnen}{Martin Anton Müller und Laura Untner}%% latex-leseansicht-abspann.tex
%% Abspann für die Leseansicht.
%% Der Schalter \ifkorrekturansicht ist bereits durch den Vorspann gesetzt.

%% latex-abspann.tex
%% Gemeinsamer Abspann für Korrekturansicht und Leseansicht.
%% Setzt den Schalter \ifkorrekturansicht voraus (gesetzt in den
%% einbindenden Dateien latex-korrekturansicht-abspann.tex bzw.
%% latex-leseansicht-abspann.tex).
%% ---------------------------------------------------------------

\normalsize

% Das esempio-Environment wird nur in der Leseansicht benötigt
\ifkorrekturansicht\else
\newenvironment{esempio}[3]%
{
    \vspace{1.5ex}
    \rlap{\underline{#1}}
    \par
    \setlength{\parindent}{0cm}
    \nopagebreak
    \leftskip=#2cm
    \rightskip=#3cm
}
{
    \par
}
\fi

\doendnotes{C}
\bigskip
\vfill

\clearpage

\footnotesize

\ifkorrekturansicht
  \lohead{\textsc{register}}
\fi

% theindex-Environment neu definieren ohne reledmac
\makeatletter
\renewenvironment{theindex}{%
  \ifkorrekturansicht
    \section*{\indexname}%
  \else
    \subsubsection*{Index der erwähnten Entitäten}%
  \fi
  \setlength{\parindent}{0pt}%
  \setlength{\parskip}{0pt plus 0.3pt}%
  \let\item\@idxitem
}{%
  \ifkorrekturansicht\clearpage\fi
}
\makeatother

\IfFileExists{\jobname-pw.ind}{\input{\jobname-pw.ind}}{}

% Quellenangabe nur in der Leseansicht
\ifkorrekturansicht\else
% Fallback-Definitionen, falls die .tex-Datei \titel etc. nicht gesetzt hat
\providecommand{\titel}{}
\providecommand{\editorInnen}{}
\providecommand{\dateiname}{\jobname}

\vspace{3cm}

\vfill

\footnotesize
\textsc{Quelle}: \titel. Herausgegeben von {\editorInnen}. In: \emph{Arthur Schnitzler: Briefwechsel mit Autorinnen und Autoren}.
 Digitale Edition, https://schnitzler-briefe.acdh.oeaw.ac.at/{\dateiname}.html (Stand \today)
\fi

\end{document}


      