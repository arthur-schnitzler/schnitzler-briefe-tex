%% latex-leseansicht-vorspann.tex
%% Vorspann für die Leseansicht.
%% Lädt die gemeinsame Datei latex-vorspann.tex mit nicht gesetztem Schalter.

\newif\ifkorrekturansicht
\korrekturansichtfalse

\input{../tex-inputs/latex-vorspann}


\section[ Felix Salten an Arthur Schnitzler, 31. 7. 1916]{L03573 Felix Salten an Arthur Schnitzler,  31. 7. 1916}
\nopagebreak\mylabel{L03573v}
\rehead{ }\normalsize\beginnumbering\briefempfaengerindex{Schnitzler, Arthur@\textsc{Schnitzler, Arthur}!zzzSalten, Felix@\emph{von Felix Salten}!1916-07-311@{31. 7. 1916}|(be}
\toendnotes[C]{\smallbreak\pagebreak[2]}
\correspDesc{Versand  durch Felix Salten am 31. 7. 1916 in Unterach am Attersee
\newline{}Erhalt  durch Arthur Schnitzler im Zeitraum [1. 8. 1916
                  – 5. 8. 1916?] in Altaussee}\toendnotes[C]{\smallbreak}
\Standort{CUL, Schnitzler, B 89, B 2.}
\physDesc{Bildpostkarte, 585 Zeichen
\newline{}Handschrift: schwarze Tinte, lateinische Kurrent
\newline{}Versand: Stempel: »\nobreak{}\oindex{Unterach am Attersee@\textbf{Unterach am Attersee}|pwk}Unterach am Attersee, 31. VII. 16\nobreak{}«.  
\newline{}Ordnung: 1) mit Bleistift von Frieda Pollak\pwindex{Pollak, Frieda 8.\,12.\,1881 Wien – 13.\,7.\,1937 ebd.@\textsc{Pollak, Frieda} (8.\,12.\,1881 Wien – 13.\,7.\,1937 ebd.), \emph{Sekretärin}|pw} (?) mit
                                 dem Buchstaben »A« (Abgeschrieben/Abschrift)
                                 gekennzeichnet  2) mit Bleistift von unbekannter Hand nummeriert: »286«}\toendnotes[C]{\smallbreak}\pstart{}{\pb}Herrn\pend{}\pstart{}D\textsuperscript{r} Arthur Schnitzler\pend{}\pstart{}Alt-Aussee\oindex{Altaussee@\textbf{Altaussee}, \emph{Verwaltungsgebiet}|pw}\pend{}\pstart{}Fischerndorf\oindex{Fischerndorf@\textbf{Fischerndorf}|pw}\pend{}{\bigskip}
\pstart
           \noindent{}\centering{}{\pb}\textcolor{gray}{\textbf{Salzkammergut\oindex{Salzkammergut@\textbf{Salzkammergut}, \emph{Region}|pw}. Berghof\oindex{Berghof@\textbf{Berghof}, \emph{Wohngebäude}|pw} bei Unterach\oindex{Unterach am Attersee@\textbf{Unterach am Attersee}|pw}.}}\pend
           \vspace{1em}
\pstart
           \noindent{}{\pb}Vielen Dank für Ihre liebe
                  \label{K_L03573-1v}\edtext{Karte}{\lemma{\textnormal{\emph{Karte}}}\Cendnote{\textnormal{nicht erhalten}}}\label{K_L03573-1}, die ich hier vorfand. Ich bin erst vor
               wenigen Tagen gekommen und finde es hier\oindex{Unterach am Attersee@\textbf{Unterach am Attersee}|pwv} wieder einmal herrlich schön. Sie sollten doch (endlich) \label{K_L03573-2v}\edtext{einmal mit Olga\pwindex{Schnitzler, Olga 17.\,1.\,1882 Wien – 13.\,1.\,1970 Lugano@\textsc{Schnitzler, Olga} (17.\,1.\,1882 Wien – 13.\,1.\,1970 Lugano), \emph{Schauspielerin, Sängerin}|pw} herüberkommen}{\lemma{\textnormal{\emph{einmal … herüberkommen}}}\Cendnote{\textnormal{Zu Schnitzlers Verhältnis zum Berghof\oindex{Berghof@\textbf{Berghof}, \emph{Wohngebäude}|pwk}{ }siehe XXXX Auszeichnungsfehler: Dokument L03114 nicht gefunden.}}}\label{K_L03573-2}. Zu arbeiten habe ich hier noch nicht begonnen. Meine 
               Einakter\pwindex{Salten, Felix 6.\,9.\,1869 Budapest – 8.\,10.\,1945 Zürich@\textsc{Salten, Felix} (6.\,9.\,1869 Budapest – 8.\,10.\,1945 Zürich), \emph{Schriftsteller, Journalist, Chefredakteur}!Kinder der Freude. Drei Einakter@\strich\emph{Kinder der Freude. Drei Einakter}|pwv}
                sind in Wien\oindex{Wien@\textbf{Wien}, \emph{Verwaltungsgebiet}|pw} fertig geworden und schon
               verschickt. Auch an Herrn Steinrück\pwindex{Steinrück, Albert 20.\,5.\,1872 Wetterburg – 11.\,2.\,1929 Berlin@\textsc{Steinrück, Albert} (20.\,5.\,1872 Wetterburg – 11.\,2.\,1929 Berlin), \emph{Schauspieler}|pw}, der es
               gewünscht hat. Wir beabsichtigen nächstens einmal \label{K_L03573-3v}\edtext{zur Marie Brüll\pwindex{Brüll, Marie 24.\,5.\,1861 Wien – 27.\,11.\,1932 ebd.@\textsc{Brüll, Marie} (24.\,5.\,1861 Wien – 27.\,11.\,1932 ebd.)|pw}
               hinüberzufahren und rechnen natürlich darauf, dabei Sie und Frau Olga\pwindex{Schnitzler, Olga 17.\,1.\,1882 Wien – 13.\,1.\,1970 Lugano@\textsc{Schnitzler, Olga} (17.\,1.\,1882 Wien – 13.\,1.\,1970 Lugano), \emph{Schauspielerin, Sängerin}|pw} zu sehen}{\lemma{\textnormal{\emph{zur … sehen}}}\Cendnote{\textnormal{nicht nachweisbar}}}\label{K_L03573-3}.\pend
           
\pstart
           Inzwischen viele herzliche Grüße von uns\pwindex{Salten, Ottilie 7.\,3.\,1868 Prag – 22.\,6.\,1942 Zürich@\textsc{Salten, Ottilie} (7.\,3.\,1868 Prag – 22.\,6.\,1942 Zürich), \emph{Schauspielerin}|pwv} zu Ihnen, {\\[\baselineskip]}Ihr {\\[\baselineskip]}\spacefill\mbox{Felix Salten}\pend
           \leftskip=0em{}\selectlanguage{ngerman}\endnumbering\briefempfaengerindex{Schnitzler, Arthur@\textsc{Schnitzler, Arthur}!zzzSalten, Felix@\emph{von Felix Salten}!1916-07-311@{31. 7. 1916}|)be}\mylabel{L03573h}  \newcommand{\dateiname}{L03573}\newcommand{\titel}{Felix Salten an Arthur Schnitzler, 31. 7. 1916}\newcommand{\editorInnen}{Martin Anton Müller und Laura Untner}%% latex-leseansicht-abspann.tex
%% Abspann für die Leseansicht.
%% Der Schalter \ifkorrekturansicht ist bereits durch den Vorspann gesetzt.

%% latex-abspann.tex
%% Gemeinsamer Abspann für Korrekturansicht und Leseansicht.
%% Setzt den Schalter \ifkorrekturansicht voraus (gesetzt in den
%% einbindenden Dateien latex-korrekturansicht-abspann.tex bzw.
%% latex-leseansicht-abspann.tex).
%% ---------------------------------------------------------------

\normalsize

% Das esempio-Environment wird nur in der Leseansicht benötigt
\ifkorrekturansicht\else
\newenvironment{esempio}[3]%
{
    \vspace{1.5ex}
    \rlap{\underline{#1}}
    \par
    \setlength{\parindent}{0cm}
    \nopagebreak
    \leftskip=#2cm
    \rightskip=#3cm
}
{
    \par
}
\fi

\doendnotes{C}
\bigskip
\vfill

\clearpage

\footnotesize

\ifkorrekturansicht
  \lohead{\textsc{register}}
\fi

% theindex-Environment neu definieren ohne reledmac
\makeatletter
\renewenvironment{theindex}{%
  \ifkorrekturansicht
    \section*{\indexname}%
  \else
    \subsubsection*{Index der erwähnten Entitäten}%
  \fi
  \setlength{\parindent}{0pt}%
  \setlength{\parskip}{0pt plus 0.3pt}%
  \let\item\@idxitem
}{%
  \ifkorrekturansicht\clearpage\fi
}
\makeatother

\IfFileExists{\jobname-pw.ind}{\input{\jobname-pw.ind}}{}

% Quellenangabe nur in der Leseansicht
\ifkorrekturansicht\else
% Fallback-Definitionen, falls die .tex-Datei \titel etc. nicht gesetzt hat
\providecommand{\titel}{}
\providecommand{\editorInnen}{}
\providecommand{\dateiname}{\jobname}

\vspace{3cm}

\vfill

\footnotesize
\textsc{Quelle}: \titel. Herausgegeben von {\editorInnen}. In: \emph{Arthur Schnitzler: Briefwechsel mit Autorinnen und Autoren}.
 Digitale Edition, https://schnitzler-briefe.acdh.oeaw.ac.at/{\dateiname}.html (Stand \today)
\fi

\end{document}


