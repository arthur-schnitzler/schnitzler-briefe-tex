%% latex-leseansicht-vorspann.tex
%% Vorspann für die Leseansicht.
%% Lädt die gemeinsame Datei latex-vorspann.tex mit nicht gesetztem Schalter.

\newif\ifkorrekturansicht
\korrekturansichtfalse

\input{../tex-inputs/latex-vorspann}

\begin{center}
            \textcolor{red}{ENTWURF, NICHT FERTIG KORRIGIERT}
                      \end{center}
            
         
         \renewcommand{\erwaehntePersonen}{Personen: Marie Brüll, Frieda Pollak, Ottilie Salten, Olga Schnitzler, Albert Steinrück}
         \renewcommand{\erwaehnteOrte}{Orte: Altaussee, Berghof, Fischerndorf, Salzkammergut, Unterach am Attersee, Wien}
         \renewcommand{\erwaehnteWerke}{Werke: Kinder der Freude. Drei Einakter, Tagebuch}
               \section[ Felix Salten an Arthur Schnitzler, 31. 7. 1916]{ Felix Salten an Arthur Schnitzler, 31. 7. 1916}\nopagebreak\mylabel{v}\rehead{ }\begin{ledgroupsized}[t]{13cm}\normalsize\beginnumbering \toendnotes[C]{\smallbreak\pagebreak[2]} \Standort{CUL, Schnitzler, B 89, B 2.}
\physDesc{Bildpostkarte, 586 Zeichen
\newline{}Handschrift: schwarze Tinte, lateinische Kurrent
\newline{}Versand: Stempel: »\nobreak{}\oindex{Unterach am Attersee@\textbf{Unterach am Attersee}|pwk}Unterach am Attersee, 31. VII. 16 a\nobreak{}«.  
\newline{}Ordnung: 1) mit Bleistift von Frieda Pollak\pwindex{Pollak, Frieda 08.12.1881 – 13.07.1937@\textsc{Pollak, Frieda} (08.12.1881 – 13.07.1937), \emph{Sekretärin}|pw} (?) mit
                                 dem Buchstaben »A« (Abgeschrieben/Abschrift)
                                 gekennzeichnet  2) mit Bleistift von unbekannter Hand nummeriert: »286«}\toendnotes[C]{\smallbreak}\pstart{}{\pb}Herrn\pend{}\pstart{}D\textsuperscript{r} Arthur Schnitzler\pend{}\pstart{}Alt-Aussee\oindex{Altaussee@\textbf{Altaussee}|pw}\pend{}\pstart{}Fischerndorf\oindex{Fischerndorf@\textbf{Fischerndorf}|pw}\pend{}{\bigskip}\pstart
           \noindent{}\centering{}{\pb}\textcolor{gray}{\textbf{Salzkammergut\oindex{Salzkammergut@\textbf{Salzkammergut}|pw}. Berghof\oindex{Berghof@\textbf{Berghof}|pw} bei Unterach\oindex{Unterach am Attersee@\textbf{Unterach am Attersee}|pw}.}}\pend
           \pstart
           {\pb}Vielen Dank für Ihre liebe
                  \label{K_L03573-1v}\edtext{Karte}{\lemma{\textnormal{\emph{Karte}}}\Cendnote{\textnormal{nicht erhalten}}}\label{K_L03573-1h}, die ich hier vorfand. Ich bin erst vor
               wenigen Tagen gekommen und finde es hier\oindex{Unterach am Attersee@\textbf{Unterach am Attersee}|pwv} wieder einmal herrlich schön. Sie sollten doch (endlich) \label{K_L03573-2v}\edtext{einmal mit Olga\pwindex{Schnitzler, Olga 17.01.1882 – 13.01.1970@\textsc{Schnitzler, Olga} (17.01.1882 – 13.01.1970), \emph{Schauspielerin, Sängerin}|pw} herüberkommen}{\lemma{\textnormal{\emph{einmal … herüberkommen}}}\Cendnote{\textnormal{Aus Schnitzler\pwindex{Schnitzler, Arthur 15.05.1862 – 21.10.1931@\textsc{Schnitzler, Arthur} (15.05.1862 – 21.10.1931), \emph{Schriftsteller, Mediziner}|pwk}s \emph{Tagebuch}\pwindex{\textcolor{red}{\textsuperscript{XXXX1 indx}}!Tagebuch1981 – 2000@\strich\emph{Tagebuch} {[}Hrsg., 1981 – 2000{]}|pwk} geht nicht hervor, dass sie der Einladung Folge
                  leisteten.}}}\label{K_L03573-2h}. Zu arbeiten habe ich hier noch nicht begonnen. Meine \label{K_L03573-3v}\edtext{Einakter\pwindex{Salten, Felix 06.09.1869 – 08.10.1945@\textsc{Salten, Felix} (06.09.1869 – 08.10.1945), \emph{Schriftsteller, Journalist}!Kinder der Freude. Drei Einakter1916-11-05@\strich\emph{Kinder der Freude. Drei Einakter} {[}1916-11-05{]}|pwv}}{\lemma{\textnormal{\emph{Einakter}}}\Cendnote{\textnormal{der Einakterzyklus \emph{Kinder der Freude}\pwindex{Salten, Felix 06.09.1869 – 08.10.1945@\textsc{Salten, Felix} (06.09.1869 – 08.10.1945), \emph{Schriftsteller, Journalist}!Kinder der Freude. Drei Einakter1916-11-05@\strich\emph{Kinder der Freude. Drei Einakter} {[}1916-11-05{]}|pwk}}}}\label{K_L03573-3h} sind in Wien\oindex{Wien@\textbf{Wien}|pw} fertig geworden und schon
               verschickt. Auch an Herrn Steinrück\pwindex{Steinrueck, Albert 20.05.1872 – 11.02.1929@\textsc{Steinrück, Albert} (20.05.1872 – 11.02.1929), \emph{Schauspieler}|pw}, der es
               gewünscht hat. Wir beabsichtigen nächstens einmal \label{K_L03573-4v}\edtext{zur Marie Brüll\pwindex{Bruell, Marie 24.05.1861 – 27.11.1932@\textsc{Brüll, Marie} (24.05.1861 – 27.11.1932)|pw}
               hinüberzufahren und rechnen natürlich darauf, dabei Sie und Frau Olga\pwindex{Schnitzler, Olga 17.01.1882 – 13.01.1970@\textsc{Schnitzler, Olga} (17.01.1882 – 13.01.1970), \emph{Schauspielerin, Sängerin}|pw} zu sehen}{\lemma{\textnormal{\emph{zur … sehen}}}\Cendnote{\textnormal{nicht nachweisbar}}}\label{K_L03573-4h}. Inzwischen viele herzliche Grüße von uns\pwindex{Salten, Ottilie 07.03.1868 – 22.06.1942@\textsc{Salten, Ottilie} (07.03.1868 – 22.06.1942), \emph{Schauspielerin}|pwv} zu Ihnen, {\\}Ihr {\\}\spacefill\mbox{Felix Salten}\pend
           
         
         \endnumbering\mylabel{h}\end{ledgroupsized}  \newcommand{\dateiname}{L03573}\newcommand{\titel}{Felix Salten an Arthur Schnitzler, 31. 7. 1916}\newcommand{\editorInnen}{Martin Anton Müller und Laura Untner}%% latex-leseansicht-abspann.tex
%% Abspann für die Leseansicht.
%% Der Schalter \ifkorrekturansicht ist bereits durch den Vorspann gesetzt.

%% latex-abspann.tex
%% Gemeinsamer Abspann für Korrekturansicht und Leseansicht.
%% Setzt den Schalter \ifkorrekturansicht voraus (gesetzt in den
%% einbindenden Dateien latex-korrekturansicht-abspann.tex bzw.
%% latex-leseansicht-abspann.tex).
%% ---------------------------------------------------------------

\normalsize

% Das esempio-Environment wird nur in der Leseansicht benötigt
\ifkorrekturansicht\else
\newenvironment{esempio}[3]%
{
    \vspace{1.5ex}
    \rlap{\underline{#1}}
    \par
    \setlength{\parindent}{0cm}
    \nopagebreak
    \leftskip=#2cm
    \rightskip=#3cm
}
{
    \par
}
\fi

\doendnotes{C}
\bigskip
\vfill

\clearpage

\footnotesize

\ifkorrekturansicht
  \lohead{\textsc{register}}
\fi

% theindex-Environment neu definieren ohne reledmac
\makeatletter
\renewenvironment{theindex}{%
  \ifkorrekturansicht
    \section*{\indexname}%
  \else
    \subsubsection*{Index der erwähnten Entitäten}%
  \fi
  \setlength{\parindent}{0pt}%
  \setlength{\parskip}{0pt plus 0.3pt}%
  \let\item\@idxitem
}{%
  \ifkorrekturansicht\clearpage\fi
}
\makeatother

\IfFileExists{\jobname-pw.ind}{\input{\jobname-pw.ind}}{}

% Quellenangabe nur in der Leseansicht
\ifkorrekturansicht\else
% Fallback-Definitionen, falls die .tex-Datei \titel etc. nicht gesetzt hat
\providecommand{\titel}{}
\providecommand{\editorInnen}{}
\providecommand{\dateiname}{\jobname}

\vspace{3cm}

\vfill

\footnotesize
\textsc{Quelle}: \titel. Herausgegeben von {\editorInnen}. In: \emph{Arthur Schnitzler: Briefwechsel mit Autorinnen und Autoren}.
 Digitale Edition, https://schnitzler-briefe.acdh.oeaw.ac.at/{\dateiname}.html (Stand \today)
\fi

\end{document}


      