%% latex-korrekturansicht-vorspann.tex
%% Vorspann für die Korrekturansicht.
%% Lädt die gemeinsame Datei latex-vorspann.tex mit gesetztem Schalter.

\newif\ifkorrekturansicht
\korrekturansichttrue

\input{../tex-inputs/latex-vorspann}


\section[Arthur Schnitzler an Stefan Zweig, 9. 4. 1915]{L03768 Arthur Schnitzler an Stefan Zweig, 9. 4. 1915}
\nopagebreak\mylabel{L03768v}
\rehead{ }\normalsize\beginnumbering\briefempfaengerindex{Zweig, Stefan@\textsc{Zweig, Stefan}!zzzSchnitzler, Arthur@\emph{von Arthur Schnitzler}!1915-04-091@{9. 4. 1915}|(be}
\toendnotes[C]{\smallbreak\pagebreak[2]}\Standort{Jerusalem, National Library of Israel, ARC. Ms. Var. 305 1 58 Stefan Zweig Collection.}
\physDesc{Brief, 1 Blatt, 2 Seiten, 1180 Zeichen
\newline{}Schreibmaschine
\newline{}Handschrift: schwarze Tinte, lateinische Kurrent (\noindent{}Korrekturen, Ergänzungen, Unterschrift)}\toendnotes[C]{\smallbreak}
\pstart
           {\pb}\textcolor{gray}{\textbf{Dr. Arthur Schnitzler}}\hfill 9. 4. 1915. \pend
           
\pstart
           \textcolor{gray}{\textbf{Wien XVIII. Sternwartestrasse 71\oindex{Sternwartestrasse 71@\textbf{Sternwartestraße 71}, \emph{Wohngebäude (K.WHS)}|pw}}}\pend
           
\pstart{}Lieber Herr Doktor!\pend\vspace{0.5em}
\pstart
           Ich bin nicht weniger empört als Sie und sehe gleich Ihnen in dieser \introOben{}vorläufig erst\introOben{} beabsichtigten Entfernung eines – um hier nur
               das ganz Unzweifelhafte, auch von Uebelwollenden nicht zu Bezweifelnde auszusprechen
               – höchst verdienstvollen Mannes\pwindex{Rosenbaum, Richard 04.11.1867 – 25.06.1942@\textsc{Rosenbaum, Richard} (04.11.1867 – 25.06.1942), \emph{Dramaturg/Dramaturgin, Verleger/Verlegerin}|pwv} von einem so verantwortungsvollen Posten nach ehrenvollster
               siebzehnjähriger Dienstzeit\introOben{},\introOben{} unter Gründen, die \introOben{}–\introOben{} wenigstens so
               weit sie mir bekannt geworden sind, nur als Scheingründe \substVorne{}\textsuperscript{bezeichnet werden}\substDazwischen{}gelten\substHinten{} können, – auch ich sehe darin ein Symptom, – nicht das erste und keinesfalls
               das letzte – eines Geistes, ja vielleicht einer Weltanschauung, als deren tiefsten
               und letzten Ausdruck man wohl jenen Ihnen kaum unbekannt gebliebenen Ausspruch eines
               hohen Herrn\pwindex{Karl I. von Oesterreich-Ungarn 17.08.1887 – 01.04.1922@\textsc{Karl I. von Österreich-Ungarn} (17.08.1887 – 01.04.1922), \emph{Kaiser/Kaiserin, Offizier/Offizierin, Kaiserlicher Rat/Kaiserliche Rätin}|pwv} bezeichnen kann
               und der lautete: »\label{K_L03768-1v}\edtext{Wie kann man Rosenbaum\pwindex{Rosenbaum, Richard 04.11.1867 – 25.06.1942@\textsc{Rosenbaum, Richard} (04.11.1867 – 25.06.1942), \emph{Dramaturg/Dramaturgin, Verleger/Verlegerin}|pw} heissen?}{\lemma{\textnormal{\emph{Wie … heissen?}}}\Cendnote{\textnormal{Vgl. A. S.: \emph{Tagebuch}, 25. 9. 1912.}}}\label{K_L03768-1}« Über
               die ganze Angelegenheit, was in {\pb}Hinsicht auf sie
               geschehen sollte und könnte, und noch über mancherlei anderes mit Ihnen zu sprechen
               wird mir höchst erwünscht sein; vielleicht nachtmahlen Sie Anfang der nächsten Woche
               einmal bei uns und lassen mir nächstens in der Früh zwischen
               10–11 telephonieren, zu welcher
               Stunde ich Sie am sichersten anrufen kann.\pend
           
\pstart
           Herzlichst{\\[\baselineskip]}Ihr{\\[\baselineskip]}\spacefill\mbox{{[}hs.:{]} Arthur Schnitzler}\pend
           \leftskip=0em{}\selectlanguage{ngerman}\endnumbering\briefempfaengerindex{Zweig, Stefan@\textsc{Zweig, Stefan}!zzzSchnitzler, Arthur@\emph{von Arthur Schnitzler}!1915-04-091@{9. 4. 1915}|)be}\mylabel{L03768h}
\begin{anhang}
\end{anhang}\normalsize

\doendnotes{C}
\bigskip
\vfill

\clearpage

\footnotesize

\lohead{\textsc{register}}

% Definiere theindex-Environment komplett neu ohne reledmac
\makeatletter
\renewenvironment{theindex}{%
  \section*{\indexname}%
  \setlength{\parindent}{0pt}%
  \setlength{\parskip}{0pt plus 0.3pt}%
  \let\item\@idxitem
}{%
  \clearpage
}
\makeatother

\IfFileExists{\jobname-pw.ind}{\input{\jobname-pw.ind}}{}

\end{document}

      