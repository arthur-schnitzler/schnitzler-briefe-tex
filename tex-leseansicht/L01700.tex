%% latex-korrekturansicht-vorspann.tex
%% Vorspann für die Korrekturansicht.
%% Lädt die gemeinsame Datei latex-vorspann.tex mit gesetztem Schalter.

\newif\ifkorrekturansicht
\korrekturansichttrue

\input{../tex-inputs/latex-vorspann}


\section[Arthur Schnitzler an Richard Beer-Hofmann, 19. 8. 1907]{L01700 Arthur Schnitzler an Richard Beer-Hofmann, 19. 8. 1907}
\nopagebreak\mylabel{L01700v}
\rehead{ }\normalsize\beginnumbering\briefempfaengerindex{Beer-Hofmann, Richard@\textsc{Beer-Hofmann, Richard}!zzzSchnitzler, Arthur@\emph{von Arthur Schnitzler}!1907-08-191@{19. 8. 1907}|(be}
\toendnotes[C]{\smallbreak\pagebreak[2]}\Standort{YCGL, MSS 31.}
\physDesc{Bildpostkarte, 426 Zeichen
\newline{}Handschrift: Bleistift, deutsche Kurrent
\newline{}Versand: 1) Stempel: »\nobreak{}\oindex{Wildbad Waldbrunn@\textbf{Wildbad Waldbrunn}, \emph{S.SPA}|pwk}Wildbad Waldbrunn, 19. Aug. 1907\nobreak{}«.   2) Stempel: »\nobreak{}\oindex{Welsberg-Taisten@\textbf{Welsberg-Taisten}, \emph{A.ADM3}|pwk}Wel{[}sberg{]}, 19. \textcolor{gray}{8}. 07\nobreak{}«.  3) Stempel: »\nobreak{}\oindex{XVIII., Waehring@\textbf{XVIII., Währing}, \emph{A.ADM3}|pwk}18/1 Wien 110, 22. VIII. 07, IX\nobreak{}«.  4) mit schwarzer Tinte von unbekannter Hand nachgesandt nach: »Kärnten\oindex{Kaernten@\textbf{Kärnten}, \emph{A.ADM1}|pw}{ }Villach\oindex{Villach@\textbf{Villach}, \emph{A.ADM3}|pw}{ }Hotel Moser\oindex{Hotel Mosser@\textbf{Hotel Mosser}, \emph{Hotel (K.HTL)}|pw}«
\newline{}Beer-Hofmann: mit Bleistift das Datum der Beantwortung vermerkt: »B
                                       23/VIII 07« }
\buchAbdrucke{\weitereDrucke{Arthur Schnitzler, Richard Beer-Hofmann: \emph{Briefwechsel 1891–1931}. Wien, Zürich: \emph{Europaverlag} 1992, S. 182–183.} }\toendnotes[C]{\smallbreak}\pstart{}{\pb}\textsc{Dr. Richard}\pend{}\pstart{}\textsc{Beer-Hofmann}\pend{}\pstart{}Wien XVIII\oindex{XVIII., Waehring@\textbf{XVIII., Währing}, \emph{A.ADM3}|pw}\pend{}\pstart{}\textsc{Hasenauerstr} 59\oindex{Hasenauerstrasse 59@\textbf{Hasenauerstraße 59}, \emph{Wohngebäude (K.WHS)}|pw}. \pend{}{\bigskip}
\pstart
           \noindent{}\centering{}{\pb}\textcolor{gray}{\textbf{Wildbad Waldbrunn\oindex{Wildbad Waldbrunn@\textbf{Wildbad Waldbrunn}, \emph{S.SPA}|pw} bei Welsberg\oindex{Welsberg-Taisten@\textbf{Welsberg-Taisten}, \emph{A.ADM3}|pw} im Pustertale\oindex{Pustertal@\textbf{Pustertal}, \emph{T.VAL}|pw}.}}\pend
           \vspace{1em}
\pstart
           \raggedleft{}{\pb}19. 8. 907\pend
           \vspace{0.5em}
\pstart
           lieber Richard; wir bleiben hier bis \label{K_L01700-1v}\edtext{26.
               (27.. 28}{\lemma{\textnormal{\emph{26.
               (27.. 28}}}\Cendnote{\textnormal{Schnitzler hielt sich noch bis 26. 8. 1907 im Wildbad Waldbrunn\oindex{Wildbad Waldbrunn@\textbf{Wildbad Waldbrunn}, \emph{S.SPA}|pwk} auf.}}}\label{K_L01700-1})?; (Donnerſtag ko{\geminationm}t vielleicht \label{K_L01700-2v}\edtext{Goldmann\pwindex{Goldmann, Paul 31.01.1865 – 25.09.1935@\textsc{Goldmann, Paul} (31.01.1865 – 25.09.1935), \emph{Schriftsteller/Schriftstellerin, Journalist/Journalistin}|pw}}{\lemma{\textnormal{\emph{Goldmann}}}\Cendnote{\textnormal{Zu einem Zusammentreffen
                  mit Goldmann\pwindex{Goldmann, Paul 31.01.1865 – 25.09.1935@\textsc{Goldmann, Paul} (31.01.1865 – 25.09.1935), \emph{Schriftsteller/Schriftstellerin, Journalist/Journalistin}|pwk} kam es nicht, vgl. Arthur Schnitzler an Richard Beer-Hofmann, 25. 8. 1907.}}}\label{K_L01700-2}) da{\geminationn}{ }Dolomitenſtraße\oindex{Grosse Dolomitenstrasse@\textbf{Große Dolomitenstraße}, \emph{Straße (K.STR)}|pw} (wahrſcheinlich Grödner Thal\oindex{Val Badia@\textbf{Val Badia}, \emph{T.VAL}|pw} – Grödner Joch\oindex{Groedner Joch@\textbf{Grödner Joch}, \emph{Berg (N.BRG)}|pw}{ }{[}–{]}{ }Pordoj\oindex{Pordoijoch@\textbf{Pordoijoch}, \emph{Berg (N.BRG)}|pw} – Vigo\oindex{Vigo di Fassa@\textbf{Vigo di Fassa}, \emph{P.PPLA3}|pw} – Karerſee\oindex{Karersee@\textbf{Karersee}, \emph{See (N.SEE)}|pw}, – Bozen\oindex{Bozen@\textbf{Bozen}, \emph{P.PPLA2}|pw}.) da{\geminationn}{ }Meran\oindex{Meran@\textbf{Meran}, \emph{P.PPLA3}|pw}, eventuell Gardaſee\oindex{Lago di Garda@\textbf{Lago di Garda}, \emph{See (N.SEE)}|pw}, zurück zwiſchen 5 u. 10, mit
               event. Aufenthalt (I{\geminationn}sbruck\oindex{Innsbruck@\textbf{Innsbruck}, \emph{A.ADM2}|pw}, Salzburg\oindex{Salzburg@\textbf{Salzburg}, \emph{A.ADM2}|pw}.) Auf Wiederſehen
               hoffentlich.\pend
           
\pstart
           Herzlichſt Ihr{\\[\baselineskip]}\spacefill\mbox{A.}\pend
           \leftskip=0em{}
\pstart
           \noindent{}Schreiben Sie ein Wort! Viele Grüße über die Lande.\pend
           \selectlanguage{ngerman}\endnumbering\briefempfaengerindex{Beer-Hofmann, Richard@\textsc{Beer-Hofmann, Richard}!zzzSchnitzler, Arthur@\emph{von Arthur Schnitzler}!1907-08-191@{19. 8. 1907}|)be}\mylabel{L01700h}  \normalsize

\doendnotes{C}
\bigskip
\vfill

\clearpage

\footnotesize

\lohead{\textsc{register}}

% Definiere theindex-Environment komplett neu ohne reledmac
\makeatletter
\renewenvironment{theindex}{%
  \section*{\indexname}%
  \setlength{\parindent}{0pt}%
  \setlength{\parskip}{0pt plus 0.3pt}%
  \let\item\@idxitem
}{%
  \clearpage
}
\makeatother

\IfFileExists{\jobname-pw.ind}{\input{\jobname-pw.ind}}{}

\end{document}

      