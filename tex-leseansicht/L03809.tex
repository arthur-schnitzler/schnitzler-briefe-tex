%% latex-korrekturansicht-vorspann.tex
%% Vorspann für die Korrekturansicht.
%% Lädt die gemeinsame Datei latex-vorspann.tex mit gesetztem Schalter.

\newif\ifkorrekturansicht
\korrekturansichttrue

\input{../tex-inputs/latex-vorspann}


\section[Arthur Schnitzler an Stefan Zweig, 23. 3. 1908]{L03809 Arthur Schnitzler an Stefan Zweig, 23. 3. 1908}
\nopagebreak\mylabel{L03809v}
\rehead{ }\normalsize\beginnumbering\briefempfaengerindex{Zweig, Stefan@\textsc{Zweig, Stefan}!zzzSchnitzler, Arthur@\emph{von Arthur Schnitzler}!1908-03-231@{23. 3. 1908}|(be}
\toendnotes[C]{\smallbreak\pagebreak[2]}\Standort{Jerusalem, National Library of Israel, ARC. Ms. Var. 305 1 58 Stefan Zweig Collection.}
\physDesc{Visitenkarte, 1 Blatt, 2 Seiten, 98 Zeichen
\newline{}Handschrift: schwarze Tinte, deutsche Kurrent}
\pstart
           \raggedleft{}{\pb}23. 3. 908\pend
           \vspace{0.5em}
\pstart
           \centering{}\textcolor{gray}{\textbf{D\textsuperscript{r} Arthur Schnitzler}}\pend
           
\pstart
           dankt Ihnen, verehrter Herr Doktor, herzlichſt für die liebenswürdg
               Zuſendg d\textcolor{gray}{es}{ }\textsc{Thersites}\pwindex{Tersites. Ein Trauerspiel in drei Aufzuegen@\emph{Tersites. Ein Trauerspiel in drei Aufzügen}|pw}. \pend
           
\pstart
           \raggedleft{}\textcolor{gray}{\textbf{Wien\oindex{Wien@\textbf{Wien}, \emph{A.ADM2}|pw}.}}\pend
           \selectlanguage{ngerman}\endnumbering\briefempfaengerindex{Zweig, Stefan@\textsc{Zweig, Stefan}!zzzSchnitzler, Arthur@\emph{von Arthur Schnitzler}!1908-03-231@{23. 3. 1908}|)be}\mylabel{L03809h}  \normalsize

\doendnotes{C}
\bigskip
\vfill

\clearpage

\footnotesize

\lohead{\textsc{register}}

% Definiere theindex-Environment komplett neu ohne reledmac
\makeatletter
\renewenvironment{theindex}{%
  \section*{\indexname}%
  \setlength{\parindent}{0pt}%
  \setlength{\parskip}{0pt plus 0.3pt}%
  \let\item\@idxitem
}{%
  \clearpage
}
\makeatother

\IfFileExists{\jobname-pw.ind}{\input{\jobname-pw.ind}}{}

\end{document}

      