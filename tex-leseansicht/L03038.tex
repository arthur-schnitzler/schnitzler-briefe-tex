%% latex-korrekturansicht-vorspann.tex
%% Vorspann für die Korrekturansicht.
%% Lädt die gemeinsame Datei latex-vorspann.tex mit gesetztem Schalter.

\newif\ifkorrekturansicht
\korrekturansichttrue

\input{../tex-inputs/latex-vorspann}


\section[ Arthur Schnitzler an Felix Salten, {[}10. 6. 1901?{]}]{L03038 Arthur Schnitzler an Felix Salten, {[}10. 6. 1901?{]}}
\nopagebreak\mylabel{L03038v}
\rehead{ }\normalsize\beginnumbering\briefempfaengerindex{Salten, Felix@\textsc{Salten, Felix}!zzzSchnitzler, Arthur@\emph{von Arthur Schnitzler}!1901-06-101@{{[}10. 6. 1901?{]}}|(be}
\toendnotes[C]{\smallbreak\pagebreak[2]}\Standort{Wienbibliothek im Rathaus, ZPH 1681, 2.1.516.}
\physDesc{Brief, 1 Blatt, 3 Seiten, 454 Zeichen
\newline{}Handschrift: Bleistift, deutsche Kurrent
\newline{}Ordnung: mit Bleistift von unbekannter Hand Nummerierung der Doppelseiten des
                                 Konvoluts: »20«–»21« }\toendnotes[C]{\smallbreak}
\pstart
           \raggedleft{}{\pb}\uline{Montag}\pend
           \vspace{0.5em}
\pstart
           lieber Freund, ich erfuhr, dſs Sie nicht in Karlsbad\oindex{Karlsbad@\textbf{Karlsbad}, \emph{P.PPLA}|pw} ſondern hier\oindex{Wien@\textbf{Wien}, \emph{A.ADM2}|pwv} ſind, ſuchte Sie Vormittg in Ihrer Wohnung\oindex{Kochgasse@\textbf{Kochgasse}, \emph{Straße (K.STR)}|pwv} und der \textsc{Redaction\orgindex{Wiener Allgemeine Zeitung@Wiener Allgemeine Zeitung|pwv}}, um Ihnen Adieu zu ſagen\pend
           
\pstart
           {\pb}Ich \introOben{}(\textsc{resp}. wir\pwindex{Schnitzler, Olga 17.01.1882 – 13.01.1970@\textsc{Schnitzler, Olga} (17.01.1882 – 13.01.1970), \emph{Schauspieler/Schauspielerin, Sänger/Sängerin}|pwv})\introOben{}{ }\label{K_L03038-1v}\edtext{fahre morgen}{\lemma{\textnormal{\emph{fahre morgen}}}\Cendnote{\textnormal{Die Datierung des Korrespondenzstücks
                  kann dadurch mit Hilfe des \emph{Tagebuchs}\pwindex{Tagebuch@\emph{Tagebuch}|pwk} und den
                  impliziten Hinweisen auf die bevorstehenden literarischen Arbeiten\pwindex{einsame Weg. Schauspiel in fuenf Akten@\emph{Der einsame Weg. Schauspiel in fünf Akten}|pwkv}\pwindex{Lebendige Stunden@\emph{Lebendige Stunden}|pwkv}\pwindex{Frau mit dem Dolche@\emph{Die Frau mit dem Dolche}|pwkv}
                  erfolgen.}}}\label{K_L03038-1} vorläufg nach Salzburg\oindex{Salzburg@\textbf{Salzburg}, \emph{A.ADM2}|pw}
               (wahrſcheinlich) alles \label{K_L03038-2v}\edtext{weitere}{\lemma{\textnormal{\emph{weitere}}}\Cendnote{\textnormal{Schnitzlers Sommeraufenthalt dauerte
                  bis zum 29. 8. 1901, an welchem Tag er nach Wien\oindex{Wien@\textbf{Wien}, \emph{A.ADM2}|pwk} zurückkehrte.
               }}}\label{K_L03038-2} iſt noch unbeſtimmt. Sagen Sie mir ein Wort von Ihren Plänen. Briefe werden
               mir nachgeſchickt.\pend
           
\pstart
           {\pb}Ein ſchönes \label{K_L03038-3v}\edtext{3aktiges modernes Stück\pwindex{einsame Weg. Schauspiel in fuenf Akten@\emph{Der einsame Weg. Schauspiel in fünf Akten}|pwv}}{\lemma{\textnormal{\emph{3aktiges modernes Stück}}}\Cendnote{\textnormal{\emph{Der einsame Weg}\pwindex{einsame Weg. Schauspiel in fuenf Akten@\emph{Der einsame Weg. Schauspiel in fünf Akten}|pwk}, den Schnitzler am 21. 7. 1901 vorläufig abschloss.}}}\label{K_L03038-3}, innerlich
               ganz fertig, hoff ich ſehr im Sommer zu vollenden, überdies \label{K_L03038-4v}\edtext{2 Einakter\pwindex{Lebendige Stunden@\emph{Lebendige Stunden}|pwv}\pwindex{Frau mit dem Dolche@\emph{Die Frau mit dem Dolche}|pwv}}{\lemma{\textnormal{\emph{2 Einakter}}}\Cendnote{\textnormal{Den Einakter 
                  \emph{Lebendige Stunden}\pwindex{Lebendige Stunden@\emph{Lebendige Stunden}|pwk} beendete er am 28. 7. 1901. Die Arbeit am Einakter \emph{Die Frau mit dem Dolche}\pwindex{Frau mit dem Dolche@\emph{Die Frau mit dem Dolche}|pwk} wurde am 3. 8. 1901 abgeschlossen.}}}\label{K_L03038-4}.\pend
           
\pstart
           Herzlichſt Ihr {\\[\baselineskip]}\spacefill\mbox{ArthurSch}\pend
           \leftskip=0em{}\selectlanguage{ngerman}\endnumbering\briefempfaengerindex{Salten, Felix@\textsc{Salten, Felix}!zzzSchnitzler, Arthur@\emph{von Arthur Schnitzler}!1901-06-101@{{[}10. 6. 1901?{]}}|)be}\mylabel{L03038h}  \normalsize

\doendnotes{C}
\bigskip
\vfill

\clearpage

\footnotesize

\lohead{\textsc{register}}

% Definiere theindex-Environment komplett neu ohne reledmac
\makeatletter
\renewenvironment{theindex}{%
  \section*{\indexname}%
  \setlength{\parindent}{0pt}%
  \setlength{\parskip}{0pt plus 0.3pt}%
  \let\item\@idxitem
}{%
  \clearpage
}
\makeatother

\IfFileExists{\jobname-pw.ind}{\input{\jobname-pw.ind}}{}

\end{document}

      