\input{../tex-inputs/latex-pdf-vorspann}
\begin{center}
            \textcolor{red}{ENTWURF. ENTZIFFERUNG NOCH NICHT KORREKTURGELESEN}
                      \end{center}
            
               \section[Arthur Schnitzler an Richard Beer-Hofmann, 1. 12. 1892]{ Arthur Schnitzler an Richard Beer-Hofmann, 1. 12. 1892}\nopagebreak\mylabel{v}\rehead{ }\begin{ledgroupsized}[t]{13cm}\normalsize\beginnumbering\briefempfaengerindex{Beer-Hofmann, Richard@\textsc{Beer-Hofmann, Richard}!zzzSchnitzler, Arthur@\emph{von Arthur Schnitzler}!1892-12-011@{1. 12. 1892}|(be} \toendnotes[C]{\smallbreak\pagebreak[2]} \Standort{YCGL, MSS 31.}
\physDesc{Brief, 1 Blatt, 2 Seiten, Umschlag
\newline{}Handschrift: Bleistift, deutsche Kurrent\newline{}Versand: ohne postalischen Übermittlungsvermerk }\buchAbdrucke{\weitereDrucke{Arthur Schnitzler, Richard Beer-Hofmann: \emph{Briefwechsel 1891–1931}. Hg. Konstanze Fliedl. Wien, Zürich: \emph{Europaverlag} 1992, S. 40.} }\pstart{}{\pb}\textsc{Herrn Dr Rich Beer Hofmann}\pend{}\pstart{}\textsc{Wien\oindex{Wien@\textbf{Wien}|pw}.}\pend{}\pstart{}\textsc{I Wollzeile 15\oindex{Wollzeile@\textbf{Wollzeile}|pw}}.\pend{}{\bigskip}\pstart{}{\pb}Lieber Richard, \pend\pstart
           hier ſchickt mir \textsc{Beraton}\pwindex{Beraton, Ferry 06.12.1859 – 11.02.1900@\textsc{Bératon, Ferry} (06.12.1859 – 11.02.1900), \emph{Schriftsteller/Schriftstellerin, Journalist/Journalistin, Bildender Künstler/Bildende Künstlerin >> Maler/Malerin}|pw} den Sitz für Sie. Ich
               denke, wir treffen uns zwiſchen 6 und ½ 7 im \textsc{Grstdl}\oindex{Cafe Griensteidl@\textbf{Café Griensteidl}|pw} und fahren zuſa{\geminationm}en hinaus. Ich zweifle
               nicht, daſs uns da ein ſehr billiger \introOben{}u praktischer\introOben{} Modus
               einfallen wird; z. B. mit dem Fiaker bis zur \textsc{Elisabeth}brücke\oindex{Elisabethbruecke@\textbf{Elisabethbrücke}|pw} und da{\geminationn}{ }{\pb}mit der Tram. –\pend
           \pstart
           Herzlich Ihr{\\[\baselineskip]}\spacefill\mbox{Arthur.}\pend
           \leftskip=0em{}\pstart
           1/12 92.
               \pend
           \endnumbering\briefempfaengerindex{Beer-Hofmann, Richard@\textsc{Beer-Hofmann, Richard}!zzzSchnitzler, Arthur@\emph{von Arthur Schnitzler}!1892-12-011@{1. 12. 1892}|)be}\mylabel{h}\end{ledgroupsized}  \newcommand{\dateiname}{L00140}\newcommand{\titel}{Arthur Schnitzler an Richard Beer-Hofmann, 1. 12. 1892}\newcommand{\editorInnen}{Martin Anton Müller und Gerd-Hermann Susen}\input{../tex-inputs/latex-pdf-abspann}
      