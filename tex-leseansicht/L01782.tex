\input{../tex-inputs/latex-pdf-vorspann}
\begin{center}
            \textcolor{red}{ENTWURF. ENTZIFFERUNG NOCH NICHT KORREKTURGELESEN}
                      \end{center}
            
               \section[Hugo von Hofmannsthal an Arthur Schnitzler, 11. 7. 1908]{ Hugo von Hofmannsthal an Arthur Schnitzler, 11. 7. 1908}\nopagebreak\mylabel{v}\rehead{ }\begin{ledgroupsized}[t]{13cm}\normalsize\beginnumbering\briefempfaengerindex{Schnitzler, Arthur@\textsc{Schnitzler, Arthur}!zzzHofmannsthal, Hugo von@\emph{von Hugo von Hofmannsthal}!1908-07-111@{11. 7. 1908}|(be} \toendnotes[C]{\smallbreak\pagebreak[2]} \Standort{CUL, Schnitzler, B 43.}
\physDesc{Bildpostkarte
\newline{}Handschrift: Bleistift, deutsche Kurrent\newline{}Versand: Stempel: »\nobreak{}\oindex{Salzburg@\textbf{Salzburg}|pwk}Salzburg 2, 11. VII. 08, 6\nobreak{}«.  
\newline{}Schnitzler: mit Bleistift beschriftet: »\textsc{Hofma}« und »Hugo« \newline{}Ordnung: 1) mit Bleistift von unbekannter Hand nummeriert:
                              »328« 2) mit Bleistift von unbekannter Hand nummeriert: »298«}\buchAbdrucke{\weitereDrucke{Hugo von Hofmannsthal, Arthur Schnitzler: \emph{Briefwechsel}. Hg. Therese Nickl und Heinrich Schnitzler. Frankfurt am Main: \emph{S. Fischer} 1964, S. 237.} }\pstart{}{\pb}\textsc{Herrn}\pend{}\pstart{}\textsc{D\textsuperscript{r} Arthur Schnitzler}\pend{}\pstart{}\textsc{Seis am Schlern}\oindex{Seis am Schlern@\textbf{Seis am Schlern}|pw}\pend{}\pstart{}\textsc{Villa Heufler}\oindex{Villa Heufler@\textbf{Villa Heufler}|pw}\pend{}\pstart{}\textsc{Süd Tirol}\oindex{Suedtirol@\textbf{Südtirol}|pw}\pend{}{\bigskip}\pstart
           \noindent{}\centering{}\textcolor{gray}{\textbf{{\pb}Salzburg\oindex{Salzburg@\textbf{Salzburg}|pw} – Kiosk
                        Tomaselli\oindex{Cafe Tomaselli@\textbf{Café Tomaselli}|pw}.}}\pend
           \pstart
           \raggedleft{}11. VII.\pend
           \pstart
           Wie lange waren wir ſchon nicht hier zuſa{\geminationm}en!\pend
           \pstart
           Ich gehe (weiterarbeitend) heute in die Fuſch\oindex{Bad Fusch@\textbf{Bad Fusch}|pw}, von
               wo ich Ihnen ſchreibe.\pend
           \pstart  Herzlichſt \spacefill\mbox{Hugo.}\pend{}\endnumbering\briefempfaengerindex{Schnitzler, Arthur@\textsc{Schnitzler, Arthur}!zzzHofmannsthal, Hugo von@\emph{von Hugo von Hofmannsthal}!1908-07-111@{11. 7. 1908}|)be}\mylabel{h}\end{ledgroupsized}  \newcommand{\dateiname}{L01782}\newcommand{\titel}{Hugo von Hofmannsthal an Arthur Schnitzler, 11. 7. 1908}\newcommand{\editorInnen}{Martin Anton Müller und Gerd-Hermann Susen}\input{../tex-inputs/latex-pdf-abspann}
      