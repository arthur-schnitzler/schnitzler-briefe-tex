%% latex-korrekturansicht-vorspann.tex
%% Vorspann für die Korrekturansicht.
%% Lädt die gemeinsame Datei latex-vorspann.tex mit gesetztem Schalter.

\newif\ifkorrekturansicht
\korrekturansichttrue

\input{../tex-inputs/latex-vorspann}


\section[Hugo von Hofmannsthal an Arthur Schnitzler, 11. 7. 1908]{L01782 Hugo von Hofmannsthal an Arthur Schnitzler, 11. 7. 1908}
\nopagebreak\mylabel{L01782v}
\rehead{ }\normalsize\beginnumbering\briefempfaengerindex{Schnitzler, Arthur@\textsc{Schnitzler, Arthur}!zzzHofmannsthal, Hugo von@\emph{von Hugo von Hofmannsthal}!1908-07-111@{11. 7. 1908}|(be}
\toendnotes[C]{\smallbreak\pagebreak[2]}\Standort{CUL, Schnitzler, B 43.}
\physDesc{Bildpostkarte, 201 Zeichen
\newline{}Handschrift: 1) Bleistift, deutsche Kurrent\hspace{1em}2) Bleistift, lateinische Kurrent (\noindent{}Adresse)\hspace{1em}
\newline{}Versand: Stempel: »\nobreak{}\oindex{Salzburg@\textbf{Salzburg}, \emph{A.ADM2}|pwk}Salzburg 2, 11. VII. 08, 6\nobreak{}«.  
\newline{}Schnitzler: mit Bleistift beschriftet: »\textsc{Hofma}« und »Hugo« 
\newline{}Ordnung: 1) mit Bleistift von unbekannter Hand nummeriert:
                                    »328«  2) mit Bleistift von unbekannter Hand nummeriert:
                                    »298«}
\buchAbdrucke{\weitereDrucke{Hugo von Hofmannsthal, Arthur Schnitzler: \emph{Briefwechsel}. Frankfurt am Main: \emph{S. Fischer} 1964, S. 237.} }\pstart{}{\pb}Herrn\pend{}\pstart{}D\textsuperscript{r} Arthur Schnitzler\pend{}\pstart{}Seis am Schlern\oindex{Seis am Schlern@\textbf{Seis am Schlern}, \emph{P.PPL}|pw}\pend{}\pstart{}Villa Heufler\oindex{Villa Heufler@\textbf{Villa Heufler}, \emph{Beherbergungsgebäude (K.BHB)}|pw}\pend{}\pstart{}Süd Tirol\oindex{Suedtirol@\textbf{Südtirol}, \emph{A.ADM2}|pw}\pend{}{\bigskip}
\pstart
           \noindent{}\centering{}{\pb}\textcolor{gray}{\textbf{Salzburg\oindex{Salzburg@\textbf{Salzburg}, \emph{A.ADM2}|pw} – Kiosk Tomaselli\oindex{Cafe Tomaselli@\textbf{Café Tomaselli}, \emph{Kaffeehaus (K.KAF)}|pw}.}}\pend
           \vspace{1em}
\pstart
           \raggedleft{}{\pb}11. VII.\pend
           \vspace{0.5em}
\pstart
           Wie lange waren wir ſchon nicht hier zuſa{\geminationm}en!\pend
           
\pstart
           Ich gehe (weiterarbeitend) heute in die Fuſch\oindex{Bad Fusch@\textbf{Bad Fusch}, \emph{A.ADM3}|pw},
               von wo ich Ihnen ſchreibe.\pend
           \pstart  Herzlichſt \spacefill\mbox{Hugo.}\pend{}\selectlanguage{ngerman}\endnumbering\briefempfaengerindex{Schnitzler, Arthur@\textsc{Schnitzler, Arthur}!zzzHofmannsthal, Hugo von@\emph{von Hugo von Hofmannsthal}!1908-07-111@{11. 7. 1908}|)be}\mylabel{L01782h}  \normalsize

\doendnotes{C}
\bigskip
\vfill

\clearpage

\footnotesize

\lohead{\textsc{register}}

% Definiere theindex-Environment komplett neu ohne reledmac
\makeatletter
\renewenvironment{theindex}{%
  \section*{\indexname}%
  \setlength{\parindent}{0pt}%
  \setlength{\parskip}{0pt plus 0.3pt}%
  \let\item\@idxitem
}{%
  \clearpage
}
\makeatother

\IfFileExists{\jobname-pw.ind}{\input{\jobname-pw.ind}}{}

\end{document}

      