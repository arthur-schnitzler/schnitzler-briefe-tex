%% latex-leseansicht-vorspann.tex
%% Vorspann für die Leseansicht.
%% Lädt die gemeinsame Datei latex-vorspann.tex mit nicht gesetztem Schalter.

\newif\ifkorrekturansicht
\korrekturansichtfalse

\input{../tex-inputs/latex-vorspann}


\section[Berta Zuckerkandl an Arthur Schnitzler, {{[}}25. 11. 1911?{{]}}]{L04003 Berta Zuckerkandl an Arthur Schnitzler, {[}25. 11. 1911?{]}}
\nopagebreak\mylabel{L04003v}
\rehead{ }\normalsize\beginnumbering\briefempfaengerindex{Schnitzler, Arthur@\textsc{Schnitzler, Arthur}!zzzZuckerkandl, Berta@\emph{von Berta Zuckerkandl}!1911-11-251@{{[}25. 11. 1911?{]}}|(be}
\toendnotes[C]{\smallbreak\pagebreak[2]}
\correspDesc{Versand  durch Berta Zuckerkandl am [25. 11. 1911?] in Wien
\newline{}Erhalt  durch Arthur Schnitzler im Zeitraum [25. 11. 1911 – 28. 11. 1911?] in Wien}\toendnotes[C]{\smallbreak}
\Standort{CUL, Schnitzler, B 200.}
\physDesc{Brief, 1 Blatt, 4 Seiten, 1213 Zeichen
\newline{}Handschrift: schwarze Tinte, lateinische Kurrent
\newline{}Schnitzler: 1) mit Bleistift beschriftet: »Zuckerkandl« und datiert: »25/11 911«  2) mit rotem Buntstift eine Unterstreichung}\toendnotes[C]{\smallbreak}
\pstart
           \noindent{}{\pb}Verehrter Herr Doktor! Bitte
               lachen Sie mich nicht aus wenn ich Etwas sagen werde – was Ihnen vielleicht als
               Unsinn erscheinen wird. Sie sind zwar ein Genie – aber leider auch ein Mann, was
               wieder ein wenig einschränkend wirkt. Ihre Frau\pwindex{Schnitzler, Olga 17.\,1.\,1882 Wien – 13.\,1.\,1970 Lugano@\textsc{Schnitzler, Olga} (17.\,1.\,1882 Wien – 13.\,1.\,1970 Lugano), \emph{Schauspielerin, Sängerin}|pwv}{ }{\pb}wird mich gewiss verstehen, denn sie hat
               die klarsten klügsten schönen Frauen-Augen – die ich gesehn.\pend
           
\pstart
           Also — Monsieur \label{K_L04003-1v}\edtext{Paul Poiret\pwindex{Poiret, Paul 20.\,4.\,1879 Paris – 1944@\textsc{Poiret, Paul} (20.\,4.\,1879 Paris – 1944), \emph{Modeschöpfer}|pw} ist in Wien\oindex{Wien@\textbf{Wien}, \emph{Verwaltungsgebiet}|pw}}{\lemma{\textnormal{\emph{Paul Poiret ist in Wien}}}\Cendnote{\textnormal{Das Korrespondenzstück ist von der Absenderin\pwindex{Zuckerkandl, Berta 13.\,4.\,1864 Wien – 16.\,10.\,1945 Paris@\textsc{Zuckerkandl, Berta} (13.\,4.\,1864 Wien – 16.\,10.\,1945 Paris), \emph{Schriftstellerin, Journalistin, Übersetzerin}|pwkv} nicht
                  datiert. Die Datierung auf den 25. 11. 1911, die Schnitzler darauf vermerkt hat, erscheint plausibel. Als
                  der Modeschöpfer Paul Poiret\pwindex{Poiret, Paul 20.\,4.\,1879 Paris – 1944@\textsc{Poiret, Paul} (20.\,4.\,1879 Paris – 1944), \emph{Modeschöpfer}|pwk} am
                     24. 11. 1911 in Wien\oindex{Wien@\textbf{Wien}, \emph{Verwaltungsgebiet}|pwk} ankam,
                  traf ihn Berta Zuckerkandl\pwindex{Zuckerkandl, Berta 13.\,4.\,1864 Wien – 16.\,10.\,1945 Paris@\textsc{Zuckerkandl, Berta} (13.\,4.\,1864 Wien – 16.\,10.\,1945 Paris), \emph{Schriftstellerin, Journalistin, Übersetzerin}|pwk} noch am selben
                  Abend und berichtete darüber in einem exklusiven Zeitungsartikel: \emph{Neuestes. Bei Paul Poiret}\pwindex{Zuckerkandl, Berta 13.\,4.\,1864 Wien – 16.\,10.\,1945 Paris@\textsc{Zuckerkandl, Berta} (13.\,4.\,1864 Wien – 16.\,10.\,1945 Paris), \emph{Schriftstellerin, Journalistin, Übersetzerin}!Neuestes. Bei Paul Poiret@\strich\emph{Neuestes. Bei Paul Poiret}|pwk}. In: \emph{Wiener Allgemeine Zeitung}\pwindex{Wiener Allgemeine Zeitung@\emph{Wiener Allgemeine Zeitung}|pwk}, Nr. 10.094,
                        25.\,11.\,1911, 6-Uhr-Blatt, S. 5–6. Wie sie später
                  darlegte, war der Kontakt zu Poiret\pwindex{Poiret, Paul 20.\,4.\,1879 Paris – 1944@\textsc{Poiret, Paul} (20.\,4.\,1879 Paris – 1944), \emph{Modeschöpfer}|pwk} von
                  Zuckerkandls\pwindex{Zuckerkandl, Berta 13.\,4.\,1864 Wien – 16.\,10.\,1945 Paris@\textsc{Zuckerkandl, Berta} (13.\,4.\,1864 Wien – 16.\,10.\,1945 Paris), \emph{Schriftstellerin, Journalistin, Übersetzerin}|pwk} in Paris\oindex{Paris@\textbf{Paris}, \emph{Hauptstadt}|pwk} lebenden Schwester Sophie Clemenceau\pwindex{Clemenceau, Sophie 25.\,5.\,1862 – 24.\,9.\,1937@\textsc{Clemenceau, Sophie} (25.\,5.\,1862 – 24.\,9.\,1937)|pwk} hergestellt worden. Zuckerkandl\pwindex{Zuckerkandl, Berta 13.\,4.\,1864 Wien – 16.\,10.\,1945 Paris@\textsc{Zuckerkandl, Berta} (13.\,4.\,1864 Wien – 16.\,10.\,1945 Paris), \emph{Schriftstellerin, Journalistin, Übersetzerin}|pwk} nahm sich seiner in Wien\oindex{Wien@\textbf{Wien}, \emph{Verwaltungsgebiet}|pwk} an und machte ihn mit Wiener\oindex{Wien@\textbf{Wien}, \emph{Verwaltungsgebiet}|pwk} Künstlern bekannt, vgl. \emph{Die Wahrheit über Paul Poiret. Anläßlich des
                        Zusammenbruchs des Salons, der seinen Namen führt}\pwindex{Zuckerkandl, Berta 13.\,4.\,1864 Wien – 16.\,10.\,1945 Paris@\textsc{Zuckerkandl, Berta} (13.\,4.\,1864 Wien – 16.\,10.\,1945 Paris), \emph{Schriftstellerin, Journalistin, Übersetzerin}!Wahrheit über Paul Poiret. Anläßlich des Zusammenbruchs des Salons, der seinen Namen führt@\strich\emph{Die Wahrheit über Paul Poiret. Anläßlich des Zusammenbruchs des Salons, der seinen Namen führt}|pwk}. In: \emph{Neues Wiener Journal}\pwindex{Neues Wiener Journal@\emph{Neues Wiener Journal}|pwk}, Jg. 39, Nr. 13.756,
                        8. 9. 1931, S. 5–6.}}}\label{K_L04003-1}. Der phantastische Künstler
                  {\kaufmannsund} Mode-Schöpfer der Paris\oindex{Paris@\textbf{Paris}, \emph{Hauptstadt}|pw} revolutionirt hat. Er ist mit allen {\pb}Theater-Direktoren und allen regierenden
               Künstlerinnen viel intimer als die gefürchtetsten Kritiker. Es ist ein Fanatiker
               aller Intellektualitäten. Er wird wenn wir wollen viel Stimmung machen können. Für
               Ihr Werk im Allgemeinen – für das »Weite Land\pwindex{Schnitzler, Arthur 15.\,5.\,1862 Wien – 21.\,10.\,1931 ebd.@\textsc{Schnitzler, Arthur} (15.\,5.\,1862 Wien – 21.\,10.\,1931 ebd.), \emph{Schriftsteller, Mediziner}!weite Land. Tragikomödie in fünf Akten@\strich\emph{Das weite Land. Tragikomödie in fünf Akten}|pw}«
               im Besonderen. Nun speist er mit {\pb}Madame
                  Poiret\pwindex{Boulet, Denise 13.\,3.\,1886 Elbeuf – 9.\,6.\,1982 Paris@\textsc{Boulet, Denise} (13.\,3.\,1886 Elbeuf – 9.\,6.\,1982 Paris), \emph{Model}|pw}{ }Dienstag{ }ein Uhr Mittag bei mir. \uline{Kleinster}{ }\uline{Kreis}, da ich zum ersten Mal wieder ein paar Menschen
               bei mir sehe. Möchten Sie nicht mit Ihrer verehrten Frau\pwindex{Schnitzler, Olga 17.\,1.\,1882 Wien – 13.\,1.\,1970 Lugano@\textsc{Schnitzler, Olga} (17.\,1.\,1882 Wien – 13.\,1.\,1970 Lugano), \emph{Schauspielerin, Sängerin}|pwv} ko{\geminationm}en? Das Sie schlecht
               oder wenig – französisch\oindex{Frankreich@\textbf{Frankreich}|pw} sprechen macht gar
               nichts – da beinahe alle Anwesenden in derselben Lage sein werden, {\kaufmannsund} ich den Kontakt herstelle. Aber ich halte es für sehr
               gut – wenn eine \label{K_L04003-2v}\edtext{persönliche
                  Bekanntschaft}{\lemma{\textnormal{\emph{persönliche Bekanntschaft}}}\Cendnote{\textnormal{Dazu kam es nicht. Schnitzler erwähnt im \emph{Tagebuch}\pwindex{Schnitzler, Arthur 15.\,5.\,1862 Wien – 21.\,10.\,1931 ebd.@\textsc{Schnitzler, Arthur} (15.\,5.\,1862 Wien – 21.\,10.\,1931 ebd.), \emph{Schriftsteller, Mediziner}!Tagebuch@\strich\emph{Tagebuch}|pwk} weder an dem zur Debatte stehenden 28. 11. 1911 noch in
                  den Tagen darauf eine Begegnung mit Poiret\pwindex{Poiret, Paul 20.\,4.\,1879 Paris – 1944@\textsc{Poiret, Paul} (20.\,4.\,1879 Paris – 1944), \emph{Modeschöpfer}|pwk}
                  oder einen Besuch bei Zuckerkandl\pwindex{Zuckerkandl, Berta 13.\,4.\,1864 Wien – 16.\,10.\,1945 Paris@\textsc{Zuckerkandl, Berta} (13.\,4.\,1864 Wien – 16.\,10.\,1945 Paris), \emph{Schriftstellerin, Journalistin, Übersetzerin}|pwk}.}}}\label{K_L04003-2} zu
               machen wäre. Bitte {\pb}\label{T_L04003-1v}\edtext{herzlichst telephonische oder \label{K_L04003-3v}\edtext{pneu}{\lemma{\textnormal{\emph{pneu}}}\Cendnote{\textnormal{Pneumatische Post ist ein Synonym für Rohrpost.}}}\label{K_L04003-3}
                  Antwort.}{\lemma{\textnormal{\emph{herzlichst … Antwort.}}}\Cendnote{\textnormal{Der Briefschluss befindet sich
                  auf der ersten Seite am oberen Rand um neunzig Grad gedreht.}}}\label{T_L04003-1}\pend
           
\pstart
           Ihre {\\[\baselineskip]}\spacefill\mbox{B. Zuckerkandl}\pend
           \leftskip=0em{}\selectlanguage{ngerman}\endnumbering\briefempfaengerindex{Schnitzler, Arthur@\textsc{Schnitzler, Arthur}!zzzZuckerkandl, Berta@\emph{von Berta Zuckerkandl}!1911-11-251@{{[}25. 11. 1911?{]}}|)be}\mylabel{L04003h}
\begin{anhang}
\end{anhang}\newcommand{\dateiname}{L04003}\newcommand{\titel}{Berta Zuckerkandl an Arthur Schnitzler, [25. 11. 1911?]}\newcommand{\editorInnen}{Herausgegeben von Jahnke, SelmaMüller, Martin Anton}%% latex-leseansicht-abspann.tex
%% Abspann für die Leseansicht.
%% Der Schalter \ifkorrekturansicht ist bereits durch den Vorspann gesetzt.

%% latex-abspann.tex
%% Gemeinsamer Abspann für Korrekturansicht und Leseansicht.
%% Setzt den Schalter \ifkorrekturansicht voraus (gesetzt in den
%% einbindenden Dateien latex-korrekturansicht-abspann.tex bzw.
%% latex-leseansicht-abspann.tex).
%% ---------------------------------------------------------------

\normalsize

% Das esempio-Environment wird nur in der Leseansicht benötigt
\ifkorrekturansicht\else
\newenvironment{esempio}[3]%
{
    \vspace{1.5ex}
    \rlap{\underline{#1}}
    \par
    \setlength{\parindent}{0cm}
    \nopagebreak
    \leftskip=#2cm
    \rightskip=#3cm
}
{
    \par
}
\fi

\doendnotes{C}
\bigskip
\vfill

\clearpage

\footnotesize

\ifkorrekturansicht
  \lohead{\textsc{register}}
\fi

% theindex-Environment neu definieren ohne reledmac
\makeatletter
\renewenvironment{theindex}{%
  \ifkorrekturansicht
    \section*{\indexname}%
  \else
    \subsubsection*{Index der erwähnten Entitäten}%
  \fi
  \setlength{\parindent}{0pt}%
  \setlength{\parskip}{0pt plus 0.3pt}%
  \let\item\@idxitem
}{%
  \ifkorrekturansicht\clearpage\fi
}
\makeatother

\IfFileExists{\jobname-pw.ind}{\input{\jobname-pw.ind}}{}

% Quellenangabe nur in der Leseansicht
\ifkorrekturansicht\else
% Fallback-Definitionen, falls die .tex-Datei \titel etc. nicht gesetzt hat
\providecommand{\titel}{}
\providecommand{\editorInnen}{}
\providecommand{\dateiname}{\jobname}

\vspace{3cm}

\vfill

\footnotesize
\textsc{Quelle}: \titel. Herausgegeben von {\editorInnen}. In: \emph{Arthur Schnitzler: Briefwechsel mit Autorinnen und Autoren}.
 Digitale Edition, https://schnitzler-briefe.acdh.oeaw.ac.at/{\dateiname}.html (Stand \today)
\fi

\end{document}


