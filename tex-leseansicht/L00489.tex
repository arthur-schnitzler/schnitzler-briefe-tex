%% latex-leseansicht-vorspann.tex
%% Vorspann für die Leseansicht.
%% Lädt die gemeinsame Datei latex-vorspann.tex mit nicht gesetztem Schalter.

\newif\ifkorrekturansicht
\korrekturansichtfalse

\input{../tex-inputs/latex-vorspann}


\section[Arthur Schnitzler an Richard Beer-Hofmann, 21. 9. 1895]{L00489 Arthur Schnitzler an Richard Beer-Hofmann, 21. 9. 1895}
\nopagebreak\mylabel{L00489v}
\rehead{ }\normalsize\beginnumbering\briefempfaengerindex{Beer-Hofmann, Richard@\textsc{Beer-Hofmann, Richard}!zzzSchnitzler, Arthur@\emph{von Arthur Schnitzler}!1895-09-211@{21. 9. 1895}|(be}
\toendnotes[C]{\smallbreak\pagebreak[2]}
\correspDesc{Versand  durch Arthur Schnitzler am 21. 9. 1895 in Wien
\newline{}Weiterleitung  am 22. 9. 1895 in Riva del Garda
\newline{}Erhalt  durch Richard Beer-Hofmann am 24. 9. 1895 in Gardone Riviera}\toendnotes[C]{\smallbreak}
\Standort{YCGL, MSS 31.}
\physDesc{Brief, 2 Blätter, 8 Seiten, Kuvert, 1899 Zeichen
\newline{}Handschrift: 1) Bleistift, deutsche Kurrent\hspace{1em}2) schwarze Tinte, deutsche Kurrent (\noindent{}Umschlag)\hspace{1em}
\newline{}Versand: 1) mit schwarzer Tinte von unbekannter Hand nachgesandt nach »\textsc{Gardone\oindex{Gardone Riviera@\textbf{Gardone Riviera}, \emph{Verwaltungsgebiet}|pw}
                                          p{[}ost{]}. r{[}estante{]}}.«  2) Stempel: »\nobreak{}\oindex{Wien@\textbf{Wien}, \emph{Verwaltungsgebiet}|pwk}Wien, 21. 9. \textcolor{gray}{9}5\nobreak{}«.  3) Stempel: »\nobreak{}\oindex{Riva del Garda@\textbf{Riva del Garda}, \emph{Hauptstadt}|pwk}{\pb}Riva, 22. 9. 95\nobreak{}«.  4) Stempel: »\nobreak{}\oindex{Gardone Riviera@\textbf{Gardone Riviera}, \emph{Verwaltungsgebiet}|pwk}Gardone Riviera, 24 9 95\nobreak{}«. }
\buchAbdrucke{\weitereDrucke{Arthur Schnitzler, Richard Beer-Hofmann: \emph{Briefwechsel 1891–1931}. Herausgegeben von Konstanze Fliedl. Wien, Zürich: \emph{Europaverlag} 1992, S. 83.} }\toendnotes[C]{\smallbreak}\pstart{}{\pb}Herrn \textsc{Dr. Richard
                     Beer-Hofmann}\pend{}\pstart{}\textsc{Riva am G}\damage{\textcolor{gray}{ardasee}}\oindex{Riva del Garda@\textbf{Riva del Garda}, \emph{Hauptstadt}|pw}\pend{}\pstart{}\textsc{post restante}\pend{}{\bigskip}\vspace{1em}
\pstart
           \raggedleft{}{\pb}21. 9. 95\pend
           \vspace{0.5em}
\pstart
           Lieber Richard, meine Karte haben Sie wohl. In \textsc{Riva}\oindex{Riva del Garda@\textbf{Riva del Garda}, \emph{Hauptstadt}|pw} iſt es \uline{mir} nemlich vor 3 Jahren paſſirt, daſs
               der Poſtbeamte mir die Briefe an mich nicht gab – ich verlangte damals die Einläufe
               durchzuſehen, da entdeckte ich meine Briefe. Und ich hatte nicht gepfiffen! –\pend
           
\pstart
           {\pb}Die \label{K_L00489-1v}\edtext{Leſeprobe\pwindex{Schnitzler, Arthur 15.\,5.\,1862 Wien – 21.\,10.\,1931 ebd.@\textsc{Schnitzler, Arthur} (15.\,5.\,1862 Wien – 21.\,10.\,1931 ebd.), \emph{Schriftsteller, Mediziner}!Liebelei. Schauspiel in drei Akten@\strich\emph{Liebelei. Schauspiel in drei Akten}|pwv}}{\lemma{\textnormal{\emph{Leseprobe}}}\Cendnote{\textnormal{Vgl. A. S.: \emph{Tagebuch}, 18. 9. 1895.
               }}}\label{K_L00489-1} fiel gut aus. Frl. S.\pwindex{Sandrock, Adele 19.\,8.\,1863 Rotterdam – 30.\,8.\,1937 Berlin@\textsc{Sandrock, Adele} (19.\,8.\,1863 Rotterdam – 30.\,8.\,1937 Berlin), \emph{Schauspielerin}|pw} ignorirte mich,
               aber that{ }ſehr ergriffen von dem Stück\pwindex{Schnitzler, Arthur 15.\,5.\,1862 Wien – 21.\,10.\,1931 ebd.@\textsc{Schnitzler, Arthur} (15.\,5.\,1862 Wien – 21.\,10.\,1931 ebd.), \emph{Schriftsteller, Mediziner}!Liebelei. Schauspiel in drei Akten@\strich\emph{Liebelei. Schauspiel in drei Akten}|pwv}, Nachmittag telephonirte{ }ſie \label{K_L00489-2v}\edtext{\textsc{en bon camerade}}{\lemma{\textnormal{\emph{en bon camerade}}}\Cendnote{\textnormal{französisch: kameradschaftlich}}}\label{K_L00489-2}.
                  So{\geminationn}enthal\pwindex{Sonnenthal, Adolf von 21.\,12.\,1834 Budapest – 4.\,4.\,1909 Prag@\textsc{Sonnenthal, Adolf von} (21.\,12.\,1834 Budapest – 4.\,4.\,1909 Prag), \emph{Schauspieler}|pw} hat
               »gute Hoffnung«. Beim 1. Akt\pwindex{Schnitzler, Arthur 15.\,5.\,1862 Wien – 21.\,10.\,1931 ebd.@\textsc{Schnitzler, Arthur} (15.\,5.\,1862 Wien – 21.\,10.\,1931 ebd.), \emph{Schriftsteller, Mediziner}!Liebelei. Schauspiel in drei Akten@\strich\emph{Liebelei. Schauspiel in drei Akten}|pwv}
               wurde viel gelacht. Vom 3.\pwindex{Schnitzler, Arthur 15.\,5.\,1862 Wien – 21.\,10.\,1931 ebd.@\textsc{Schnitzler, Arthur} (15.\,5.\,1862 Wien – 21.\,10.\,1931 ebd.), \emph{Schriftsteller, Mediziner}!Liebelei. Schauspiel in drei Akten@\strich\emph{Liebelei. Schauspiel in drei Akten}|pwv}
               verſpricht man{ }ſich{ }ſichre Wirkung. Dem 2.\pwindex{Schnitzler, Arthur 15.\,5.\,1862 Wien – 21.\,10.\,1931 ebd.@\textsc{Schnitzler, Arthur} (15.\,5.\,1862 Wien – 21.\,10.\,1931 ebd.), \emph{Schriftsteller, Mediziner}!Liebelei. Schauspiel in drei Akten@\strich\emph{Liebelei. Schauspiel in drei Akten}|pwv}{ }ſcheint man am wenigſtens zu vertrauen.
                  {\pb}\textsc{Mitterwurzer}\pwindex{Mitterwurzer, Friedrich 16.\,10.\,1844 Dresden – 13.\,2.\,1897 Wien@\textsc{Mitterwurzer, Friedrich} (16.\,10.\,1844 Dresden – 13.\,2.\,1897 Wien), \emph{Schauspieler}|pw} war nicht anweſend; er{ }ſpielt aber{ }ſicher, ließ{ }ſich officiell entſchuldigen.
               Die \textsc{Kallina}\pwindex{Kallina, Anna 31.\,3.\,1874 Wien – 4.\,1.\,1948 ebd.@\textsc{Kallina, Anna} (31.\,3.\,1874 Wien – 4.\,1.\,1948 ebd.), \emph{Schauspielerin}|pw} wird überraſchen. Dazu will \textsc{Burckhard}\pwindex{Burckhard, Max Eugen 14.\,7.\,1854 Korneuburg – 16.\,3.\,1912 Wien@\textsc{Burckhard, Max Eugen} (14.\,7.\,1854 Korneuburg – 16.\,3.\,1912 Wien), \emph{Schriftsteller, Rechtswissenschaftler, Theaterleiter}|pw} einen Einakter von \textsc{Giacosa}\pwindex{Giacosa, Giuseppe 21.\,10.\,1847 Colleretto Giacosa – 2.\,9.\,1906 ebd.@\textsc{Giacosa, Giuseppe} (21.\,10.\,1847 Colleretto Giacosa – 2.\,9.\,1906 ebd.), \emph{Schriftsteller}|pw}{ }Rechte der Seele\pwindex{Giacosa, Giuseppe 21.\,10.\,1847 Colleretto Giacosa – 2.\,9.\,1906 ebd.@\textsc{Giacosa, Giuseppe} (21.\,10.\,1847 Colleretto Giacosa – 2.\,9.\,1906 ebd.), \emph{Schriftsteller}!Rechte der Seele. Schauspiel in einem Act@\strich\emph{Rechte der Seele. Schauspiel in einem Act}|pw} geben; während der Leſeprobe
               half er den \label{K_L00489-3v}\edtext{\textsc{Laube}\pwindex{Laube, Heinrich 18.\,9.\,1806 Sprottau – 1.\,8.\,1884 Wien@\textsc{Laube, Heinrich} (18.\,9.\,1806 Sprottau – 1.\,8.\,1884 Wien), \emph{Schriftsteller, Theaterleiter}|pw} in Sprottau\oindex{Sprottau@\textbf{Sprottau}|pw}}{\lemma{\textnormal{\emph{Laube in Sprottau}}}\Cendnote{\textnormal{Die Enthüllung des Denkmals für Heinrich Laube\pwindex{Laube, Heinrich 18.\,9.\,1806 Sprottau – 1.\,8.\,1884 Wien@\textsc{Laube, Heinrich} (18.\,9.\,1806 Sprottau – 1.\,8.\,1884 Wien), \emph{Schriftsteller, Theaterleiter}|pwk} in dessen Geburtsstadt fand
                  ebenfalls am 18. 9. 1895 statt.}}}\label{K_L00489-3} ent{\pb}hüllen. Ich wünſchte ihm angenehme Enthüllung. Er{ }ſagte, die Enthüllung des Fräulein \label{K_L00489-4v}\edtext{\textsc{Dandler}\pwindex{Dandler, Anna 14.\,3.\,1862 Stuttgart – 17.\,9.\,1930 Wiesbaden@\textsc{Dandler, Anna} (14.\,3.\,1862 Stuttgart – 17.\,9.\,1930 Wiesbaden), \emph{Schauspielerin}|pw}}{\lemma{\textnormal{\emph{Dandler}}}\Cendnote{\textnormal{Anna Dandler\pwindex{Dandler, Anna 14.\,3.\,1862 Stuttgart – 17.\,9.\,1930 Wiesbaden@\textsc{Dandler, Anna} (14.\,3.\,1862 Stuttgart – 17.\,9.\,1930 Wiesbaden), \emph{Schauspielerin}|pwk} war zeitlebens für das \emph{Münchner Hoftheater}\orgindex{Nationaltheater München@Nationaltheater München|pwk} tätig. Ob hier eine
                  sexuelle Zote (anzunehmen) oder der Wunsch ausgedrückt wird, sie ans \emph{Burgtheater}\orgindex{Burgtheater@Burgtheater|pwk} zu holen (weniger wahrscheinlich),
                  kann nicht geklärt werden.}}}\label{K_L00489-4} zöge er vor. –\pend
           
\pstart
           \textsc{Fels}\pwindex{Fels, Friedrich Michael *~1864 Bad Dürkheim@\textsc{Fels, Friedrich Michael} (*~1864 Bad Dürkheim), \emph{Journalist}|pw}{ }ſchreibt mir \label{K_L00489-5v}\edtext{heute}{\lemma{\textnormal{\emph{heute}}}\Cendnote{\textnormal{XXXX Auszeichnungsfehler: Dokument L00488 nicht gefunden.}}}\label{K_L00489-5}. Sie können{ }ſich denken.
               Er appellirt an uns zuſa{\geminationm}en, die Summe iſt 25 fl. Ich
               hab ihm gleich 10 fl {\pb}geſchickt. Darf ich ihm auch für
               Sie was{ }ſchicken? Auch an Hugo\pwindex{Hofmannsthal, Hugo von 1.\,2.\,1874 Wien – 15.\,7.\,1929 Rodaun@\textsc{Hofmannsthal, Hugo von} (1.\,2.\,1874 Wien – 15.\,7.\,1929 Rodaun), \emph{Schriftsteller}|pw} wandt ich mich
               bereits. –\pend
           
\pstart
           Geſtern war ich beim »\label{K_L00489-6v}\edtext{Pelikan\pwindex{\textcolor{red}{\textsuperscript{XXXX indx1}}!Pelikan. Schauspiel in fünf Aufzügen@\strich\emph{Ein Pelikan. Schauspiel in fünf Aufzügen}|pw}}{\lemma{\textnormal{\emph{Pelikan}}}\Cendnote{\textnormal{im Burgtheater\oindex{Wien@\textbf{Wien}!I., Innere Stadt@\textbf{I., Innere Stadt}!Burgtheater@\textbf{Burgtheater}, \emph{Theater}|pwk}}}}\label{K_L00489-6}«. Dieſes Blaßwerden guter Stücke iſt{ }ſeltſam. – \label{K_L00489-7v}\edtext{Heute}{\lemma{\textnormal{\emph{Heute}}}\Cendnote{\textnormal{Gegeben wurde
                  zum ersten Mal \emph{Die Doppelhochzeit}\pwindex{Léon, Victor 4.\,1.\,1858 Senica – 23.\,2.\,1940 Wien@\textsc{Léon, Victor} (4.\,1.\,1858 Senica – 23.\,2.\,1940 Wien), \emph{Schriftsteller, Dramaturg}!Doppelhochzeit@\strich\emph{Die Doppelhochzeit}|pwk}\pwindex{Waldberg, Heinrich von 2.\,3.\,1860 Iași – 20.\,10.\,1942 Konzentrationslager Theresienstadt@\textsc{Waldberg, Heinrich von} (2.\,3.\,1860 Iași – 20.\,10.\,1942 Konzentrationslager Theresienstadt), \emph{Schriftsteller}!Doppelhochzeit@\strich\emph{Die Doppelhochzeit}|pwk} von Victor Léon\pwindex{Léon, Victor 4.\,1.\,1858 Senica – 23.\,2.\,1940 Wien@\textsc{Léon, Victor} (4.\,1.\,1858 Senica – 23.\,2.\,1940 Wien), \emph{Schriftsteller, Dramaturg}|pwk} und Heinrich von Waldberg\pwindex{Waldberg, Heinrich von 2.\,3.\,1860 Iași – 20.\,10.\,1942 Konzentrationslager Theresienstadt@\textsc{Waldberg, Heinrich von} (2.\,3.\,1860 Iași – 20.\,10.\,1942 Konzentrationslager Theresienstadt), \emph{Schriftsteller}|pwk}, Musik von Josef Hellmesberger\pwindex{Hellmesberger, Josef 9.\,4.\,1855 Wien – 26.\,4.\,1907 ebd.@\textsc{Hellmesberger, Josef} (9.\,4.\,1855 Wien – 26.\,4.\,1907 ebd.), \emph{Komponist}|pwk}.}}}\label{K_L00489-7} geh ich zur Eröffnung der \textsc{Josefstadt}\oindex{Wien@\textbf{Wien}!VIII., Josefstadt@\textbf{VIII., Josefstadt}!Theater in der Josefstadt@\textbf{Theater in der Josefstadt}, \emph{Theater}|pw}. – Gearbeitet hab ich noch i{\geminationm}er gar nichts; heute
                  {\pb}Nacht will ich anfangen. Glauben Sie? –\pend
           
\pstart
           Das Datum der L.\pwindex{Schnitzler, Arthur 15.\,5.\,1862 Wien – 21.\,10.\,1931 ebd.@\textsc{Schnitzler, Arthur} (15.\,5.\,1862 Wien – 21.\,10.\,1931 ebd.), \emph{Schriftsteller, Mediziner}!Liebelei. Schauspiel in drei Akten@\strich\emph{Liebelei. Schauspiel in drei Akten}|pw} iſt noch nicht
               feſtgeſtellt. –\pend
           
\pstart
           Den Hugo\pwindex{Hofmannsthal, Hugo von 1.\,2.\,1874 Wien – 15.\,7.\,1929 Rodaun@\textsc{Hofmannsthal, Hugo von} (1.\,2.\,1874 Wien – 15.\,7.\,1929 Rodaun), \emph{Schriftsteller}|pw} hab ich geſtern begegnet, vorgeſtern
               iſt er zurückgeko{\geminationm}en. Er{ }ſieht gut aus,
               »wettergebräunt«. Nach und nach wird man zu allen Worten Anführungszeichen {\pb}machen müſſen – das wird dann das Ende der Literatur
               sein.\pend
           
\pstart
           Wie geht’s Ihnen? Nächſtens{ }ſchreiben Sie mir einen Brief{ }ſtatt einer Depeſche; da
               werde ich weniger erſchrecken und mich beſſer unterhalten. Ich wünſche Ihnen weiter
               gute Laune, {\pb}gutes Wetter, gute Sti{\geminationm}ung und lebhafte Empfindung Ihrer Freiheit und Ihres
               Lebens.\pend
           
\pstart
           Herzliche Grüße Ihr{\\[\baselineskip]}\spacefill\mbox{Arthur}\pend
           \leftskip=0em{}\selectlanguage{ngerman}\endnumbering\briefempfaengerindex{Beer-Hofmann, Richard@\textsc{Beer-Hofmann, Richard}!zzzSchnitzler, Arthur@\emph{von Arthur Schnitzler}!1895-09-211@{21. 9. 1895}|)be}\mylabel{L00489h}  \newcommand{\dateiname}{L00489}\newcommand{\titel}{Arthur Schnitzler an Richard Beer-Hofmann, 21. 9. 1895}\newcommand{\editorInnen}{Martin Anton Müller und Gerd-Hermann Susen}%% latex-leseansicht-abspann.tex
%% Abspann für die Leseansicht.
%% Der Schalter \ifkorrekturansicht ist bereits durch den Vorspann gesetzt.

%% latex-abspann.tex
%% Gemeinsamer Abspann für Korrekturansicht und Leseansicht.
%% Setzt den Schalter \ifkorrekturansicht voraus (gesetzt in den
%% einbindenden Dateien latex-korrekturansicht-abspann.tex bzw.
%% latex-leseansicht-abspann.tex).
%% ---------------------------------------------------------------

\normalsize

% Das esempio-Environment wird nur in der Leseansicht benötigt
\ifkorrekturansicht\else
\newenvironment{esempio}[3]%
{
    \vspace{1.5ex}
    \rlap{\underline{#1}}
    \par
    \setlength{\parindent}{0cm}
    \nopagebreak
    \leftskip=#2cm
    \rightskip=#3cm
}
{
    \par
}
\fi

\doendnotes{C}
\bigskip
\vfill

\clearpage

\footnotesize

\ifkorrekturansicht
  \lohead{\textsc{register}}
\fi

% theindex-Environment neu definieren ohne reledmac
\makeatletter
\renewenvironment{theindex}{%
  \ifkorrekturansicht
    \section*{\indexname}%
  \else
    \subsubsection*{Index der erwähnten Entitäten}%
  \fi
  \setlength{\parindent}{0pt}%
  \setlength{\parskip}{0pt plus 0.3pt}%
  \let\item\@idxitem
}{%
  \ifkorrekturansicht\clearpage\fi
}
\makeatother

\IfFileExists{\jobname-pw.ind}{\input{\jobname-pw.ind}}{}

% Quellenangabe nur in der Leseansicht
\ifkorrekturansicht\else
% Fallback-Definitionen, falls die .tex-Datei \titel etc. nicht gesetzt hat
\providecommand{\titel}{}
\providecommand{\editorInnen}{}
\providecommand{\dateiname}{\jobname}

\vspace{3cm}

\vfill

\footnotesize
\textsc{Quelle}: \titel. Herausgegeben von {\editorInnen}. In: \emph{Arthur Schnitzler: Briefwechsel mit Autorinnen und Autoren}.
 Digitale Edition, https://schnitzler-briefe.acdh.oeaw.ac.at/{\dateiname}.html (Stand \today)
\fi

\end{document}


