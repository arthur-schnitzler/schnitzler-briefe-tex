%% latex-leseansicht-vorspann.tex
%% Vorspann für die Leseansicht.
%% Lädt die gemeinsame Datei latex-vorspann.tex mit nicht gesetztem Schalter.

\newif\ifkorrekturansicht
\korrekturansichtfalse

\input{../tex-inputs/latex-vorspann}


         
         \renewcommand{\erwaehntePersonen}{Personen: Richard Beer-Hofmann, Max Eugen Burckhard, Anna Dandler, Friedrich Michael Fels, Giuseppe Giacosa, Josef Hellmesberger, Hugo von Hofmannsthal, Anna Kallina, Heinrich Laube, Victor Léon, Friedrich Mitterwurzer, Adele Sandrock, Adolf von Sonnenthal, Heinrich von Waldberg}
         \renewcommand{\erwaehnteInstitutionen}{Institutionen: Burgtheater, Nationaltheater München}
         \renewcommand{\erwaehnteOrte}{Orte: Burgtheater, Gardone Riviera, Riva del Garda, Sprottau, Theater in der Josefstadt, Wien}
         \renewcommand{\erwaehnteWerke}{Werke: Die Doppelhochzeit, Ein Pelikan. Schauspiel in fünf Aufzügen, Liebelei. Schauspiel in drei Akten, Rechte der Seele. Schauspiel in einem Act}
               \section[Arthur Schnitzler an Richard Beer-Hofmann, 21. 9. 1895]{ Arthur Schnitzler an Richard Beer-Hofmann, 21. 9. 1895}\nopagebreak\mylabel{v}\rehead{ }\begin{ledgroupsized}[t]{13cm}\normalsize\beginnumbering\briefempfaengerindex{Beer-Hofmann, Richard@\textsc{Beer-Hofmann, Richard}!zzzSchnitzler, Arthur@\emph{von Arthur Schnitzler}!1895-09-211@{21. 9. 1895}|(be} \toendnotes[C]{\smallbreak\pagebreak[2]} \Standort{YCGL, MSS 31.}
\physDesc{Brief, 2 Blätter, 8 Seiten, Umschlag, 1899 Zeichen
\newline{}Handschrift: 1) Bleistift, deutsche Kurrent\hspace{1em}2) schwarze Tinte, deutsche Kurrent (\noindent{}Umschlag)\hspace{1em}
\newline{}Versand: 1) mit schwarzer Tinte von unbekannter Hand nachgesandt nach »\textsc{Gardone\oindex{Gardone Riviera@\textbf{Gardone Riviera}|pw}
                                          p{[}ost{]}. r{[}estante{]}}.«  2) Stempel: »\nobreak{}Wien, 21. 9. \textcolor{gray}{9}5\nobreak{}«.  3) Stempel: »\nobreak{}\oindex{Riva del Garda@\textbf{Riva del Garda}|pwk}{\pb}Riva, 22. 9. 95\nobreak{}«.  4) Stempel: »\nobreak{}\oindex{Gardone Riviera@\textbf{Gardone Riviera}|pwk}Gardone Riviera, 24 9 95\nobreak{}«. }\buchAbdrucke{\weitereDrucke{Arthur Schnitzler, Richard Beer-Hofmann: \emph{Briefwechsel 1891–1931}. Hg. Konstanze Fliedl. Wien, Zürich: \emph{Europaverlag} 1992, S. 83.} }\toendnotes[C]{\smallbreak}\pstart{}{\pb}Herrn \textsc{Dr. Richard
                     Beer-Hofmann}\pend{}\pstart{}\textsc{Riva am G}\damage{\textcolor{gray}{ardasee}}\oindex{Riva del Garda@\textbf{Riva del Garda}|pw}\pend{}\pstart{}\textsc{post restante}\pend{}{\bigskip}\pstart
           \raggedleft{}{\pb}21. 9. 95\pend
           \pstart
           Lieber Richard, meine Karte haben Sie wohl. In \textsc{Riva}\oindex{Riva del Garda@\textbf{Riva del Garda}|pw} iſt es \uline{mir} nemlich vor 3 Jahren paſſirt, daſs
               der Poſtbeamte mir die Briefe an mich nicht gab – ich verlangte damals die Einläufe
               durchzuſehen, da entdeckte ich meine Briefe. Und ich hatte nicht gepfiffen! –\pend
           \pstart
           {\pb}Die \label{K_L00489-1v}\edtext{Leſeprobe\pwindex{Schnitzler, Arthur 15.05.1862 – 21.10.1931@\textsc{Schnitzler, Arthur} (15.05.1862 – 21.10.1931), \emph{Schriftsteller, Mediziner}!Liebelei. Schauspiel in drei Akten1895-10-09@\strich\emph{Liebelei. Schauspiel in drei Akten} {[}1895-10-09{]}|pwv}}{\lemma{\textnormal{\emph{Leſeprobe}}}\Cendnote{\textnormal{Vgl. A. S.: \emph{Tagebuch}, 18. 9. 1895.
               }}}\label{K_L00489-1h} fiel gut aus. Frl. S.\pwindex{Sandrock, Adele 1863-08-19 – 1937-08-30@\textsc{Sandrock, Adele} (1863-08-19 – 1937-08-30), \emph{Schauspielerin}|pw} ignorirte mich,
               aber that ſehr ergriffen von dem Stück\pwindex{Schnitzler, Arthur 15.05.1862 – 21.10.1931@\textsc{Schnitzler, Arthur} (15.05.1862 – 21.10.1931), \emph{Schriftsteller, Mediziner}!Liebelei. Schauspiel in drei Akten1895-10-09@\strich\emph{Liebelei. Schauspiel in drei Akten} {[}1895-10-09{]}|pwv}, Nachmittag telephonirte ſie \label{K_L00489-2v}\edtext{\textsc{en bon camerade}}{\lemma{\textnormal{\emph{en bon camerade}}}\Cendnote{\textnormal{französisch: kameradschaftlich}}}\label{K_L00489-2h}.
                  So{\geminationn}enthal\pwindex{Sonnenthal, Adolf von 1834-12-21 – 1909-04-04@\textsc{Sonnenthal, Adolf von} (1834-12-21 – 1909-04-04), \emph{Schauspieler}|pw} hat
               »gute Hoffnung«. Beim 1. Akt\pwindex{Schnitzler, Arthur 15.05.1862 – 21.10.1931@\textsc{Schnitzler, Arthur} (15.05.1862 – 21.10.1931), \emph{Schriftsteller, Mediziner}!Liebelei. Schauspiel in drei Akten1895-10-09@\strich\emph{Liebelei. Schauspiel in drei Akten} {[}1895-10-09{]}|pwv}
               wurde viel gelacht. Vom 3.\pwindex{Schnitzler, Arthur 15.05.1862 – 21.10.1931@\textsc{Schnitzler, Arthur} (15.05.1862 – 21.10.1931), \emph{Schriftsteller, Mediziner}!Liebelei. Schauspiel in drei Akten1895-10-09@\strich\emph{Liebelei. Schauspiel in drei Akten} {[}1895-10-09{]}|pwv}
               verſpricht man ſich ſichre Wirkung. Dem 2.\pwindex{Schnitzler, Arthur 15.05.1862 – 21.10.1931@\textsc{Schnitzler, Arthur} (15.05.1862 – 21.10.1931), \emph{Schriftsteller, Mediziner}!Liebelei. Schauspiel in drei Akten1895-10-09@\strich\emph{Liebelei. Schauspiel in drei Akten} {[}1895-10-09{]}|pwv}{ }ſcheint man am wenigſtens zu vertrauen.
                  {\pb}\textsc{Mitterwurzer}\pwindex{Mitterwurzer, Friedrich 16.10.1844 – 13.02.1897@\textsc{Mitterwurzer, Friedrich} (16.10.1844 – 13.02.1897), \emph{Schauspieler}|pw} war nicht anweſend; er ſpielt aber ſicher, ließ ſich officiell entſchuldigen.
               Die \textsc{Kallina}\pwindex{Kallina, Anna 31.03.1874 – 04.01.1948@\textsc{Kallina, Anna} (31.03.1874 – 04.01.1948), \emph{Schauspielerin}|pw} wird überraſchen. Dazu will \textsc{Burckhard}\pwindex{Burckhard, Max Eugen 14.07.1854 – 16.03.1912@\textsc{Burckhard, Max Eugen} (14.07.1854 – 16.03.1912), \emph{Schriftsteller, Rechtswissenschaftler, Theaterleiter}|pw} einen Einakter von \textsc{Giacosa}\pwindex{Giacosa, Giuseppe 21.10.1847 – 02.09.1906@\textsc{Giacosa, Giuseppe} (21.10.1847 – 02.09.1906), \emph{Schriftsteller}|pw}{ }Rechte der Seele\pwindex{Giacosa, Giuseppe 21.10.1847 – 02.09.1906@\textsc{Giacosa, Giuseppe} (21.10.1847 – 02.09.1906), \emph{Schriftsteller}!Rechte der Seele. Schauspiel in einem Act1895-10-09@\strich\emph{Rechte der Seele. Schauspiel in einem Act} {[}1895-10-09{]}|pw} geben; während der Leſeprobe
               half er den \label{K_L00489-3v}\edtext{\textsc{Laube}\pwindex{Laube, Heinrich 1806-09-18 – 1884-08-01@\textsc{Laube, Heinrich} (1806-09-18 – 1884-08-01), \emph{Schriftsteller, Theaterleiter}|pw} in Sprottau\oindex{Sprottau@\textbf{Sprottau}|pw}}{\lemma{\textnormal{\emph{Laube in Sprottau}}}\Cendnote{\textnormal{Die Enthüllung des Denkmals für Heinrich Laube\pwindex{Laube, Heinrich 1806-09-18 – 1884-08-01@\textsc{Laube, Heinrich} (1806-09-18 – 1884-08-01), \emph{Schriftsteller, Theaterleiter}|pwk} in dessen Geburtsstadt fand
                  ebenfalls am 18. 9. 1895 statt.}}}\label{K_L00489-3h} ent{\pb}hüllen. Ich wünſchte ihm angenehme Enthüllung. Er
               ſagte, die Enthüllung des Fräulein \label{K_L00489-4v}\edtext{\textsc{Dandler}\pwindex{Dandler, Anna 1862-03-14 – 1930-09-17@\textsc{Dandler, Anna} (1862-03-14 – 1930-09-17), \emph{Schauspielerin}|pw}}{\lemma{\textnormal{\emph{Dandler}}}\Cendnote{\textnormal{Anna Dandler\pwindex{Dandler, Anna 1862-03-14 – 1930-09-17@\textsc{Dandler, Anna} (1862-03-14 – 1930-09-17), \emph{Schauspielerin}|pwk} war zeitlebens für das \emph{Münchner Hoftheater}\orgindex{Nationaltheater Muenchen@Nationaltheater München|pwk} tätig. Ob hier eine
                  sexuelle Zote (anzunehmen) oder der Wunsch ausgedrückt wird, sie ans \emph{Burgtheater}\orgindex{Burgtheater@Burgtheater|pwk} zu holen (weniger wahrscheinlich),
                  kann nicht geklärt werden.}}}\label{K_L00489-4h} zöge er vor. –\pend
           \pstart
           \textsc{Fels}\pwindex{Fels, Friedrich Michael *~1864@\textsc{Fels, Friedrich Michael} (*~1864), \emph{Journalist}|pw}{ }ſchreibt mir \label{K_L00489-5v}\edtext{heute}{\lemma{\textnormal{\emph{heute}}}\Cendnote{\textnormal{Friedrich M. Fels an Arthur Schnitzler, 19. 9. 1895.}}}\label{K_L00489-5h}. Sie können ſich denken.
               Er appellirt an uns zuſa{\geminationm}en, die Summe iſt 25 fl. Ich
               hab ihm gleich 10 fl {\pb}geſchickt. Darf ich ihm auch für
               Sie was ſchicken? Auch an Hugo\pwindex{Hofmannsthal, Hugo von 1874-02-01 – 1929-07-15@\textsc{Hofmannsthal, Hugo von} (1874-02-01 – 1929-07-15), \emph{Schriftsteller}|pw} wandt ich mich
               bereits. –\pend
           \pstart
           Geſtern war ich beim »\label{K_L00489-6v}\edtext{Pelikan\pwindex{\textcolor{red}{\textsuperscript{XXXX1 indx}}!Pelikan. Schauspiel in fuenf Aufzuegen1. 12. 1862@\strich\emph{Ein Pelikan. Schauspiel in fünf Aufzügen} {[}1. 12. 1862{]}|pw}}{\lemma{\textnormal{\emph{Pelikan}}}\Cendnote{\textnormal{im Burgtheater\oindex{Burgtheater@\textbf{Burgtheater}|pwk}}}}\label{K_L00489-6h}«. Dieſes Blaßwerden guter Stücke iſt ſeltſam. – \label{K_L00489-7v}\edtext{Heute}{\lemma{\textnormal{\emph{Heute}}}\Cendnote{\textnormal{Gegeben wurde
                  zum ersten Mal \emph{Die Doppelhochzeit}\pwindex{Leon, Victor 4.1.1858 – 23.2.1940@\textsc{Léon, Victor} (4.1.1858 – 23.2.1940), \emph{Schriftsteller, Dramaturg}!Doppelhochzeit21. 9. 1895@\strich\emph{Die Doppelhochzeit} {[}21. 9. 1895{]}|pwk}\pwindex{Waldberg, Heinrich von 02.03.1860 – 20.10.1942@\textsc{Waldberg, Heinrich von} (02.03.1860 – 20.10.1942), \emph{Schriftsteller}!Doppelhochzeit21. 9. 1895@\strich\emph{Die Doppelhochzeit} {[}21. 9. 1895{]}|pwk} von Victor Léon\pwindex{Leon, Victor 4.1.1858 – 23.2.1940@\textsc{Léon, Victor} (4.1.1858 – 23.2.1940), \emph{Schriftsteller, Dramaturg}|pwk} und Heinrich von Waldberg\pwindex{Waldberg, Heinrich von 02.03.1860 – 20.10.1942@\textsc{Waldberg, Heinrich von} (02.03.1860 – 20.10.1942), \emph{Schriftsteller}|pwk}, Musik von Josef Hellmesberger\pwindex{Hellmesberger, Josef 09.04.1855 – 26.04.1907@\textsc{Hellmesberger, Josef} (09.04.1855 – 26.04.1907), \emph{Komponist}|pwk}.}}}\label{K_L00489-7h} geh ich zur Eröffnung der \textsc{Josefstadt}\oindex{Theater in der Josefstadt@\textbf{Theater in der Josefstadt}|pw}. – Gearbeitet hab ich noch i{\geminationm}er gar nichts; heute
                  {\pb}Nacht will ich anfangen. Glauben Sie? –\pend
           \pstart
           Das Datum der L.\pwindex{Schnitzler, Arthur 15.05.1862 – 21.10.1931@\textsc{Schnitzler, Arthur} (15.05.1862 – 21.10.1931), \emph{Schriftsteller, Mediziner}!Liebelei. Schauspiel in drei Akten1895-10-09@\strich\emph{Liebelei. Schauspiel in drei Akten} {[}1895-10-09{]}|pw} iſt noch nicht
               feſtgeſtellt. –\pend
           \pstart
           Den Hugo\pwindex{Hofmannsthal, Hugo von 1874-02-01 – 1929-07-15@\textsc{Hofmannsthal, Hugo von} (1874-02-01 – 1929-07-15), \emph{Schriftsteller}|pw} hab ich geſtern begegnet, vorgeſtern
               iſt er zurückgeko{\geminationm}en. Er ſieht gut aus,
               »wettergebräunt«. Nach und nach wird man zu allen Worten Anführungszeichen {\pb}machen müſſen – das wird dann das Ende der Literatur
               sein.\pend
           \pstart
           Wie geht’s Ihnen? Nächſtens ſchreiben Sie mir einen Brief ſtatt einer Depeſche; da
               werde ich weniger erſchrecken und mich beſſer unterhalten. Ich wünſche Ihnen weiter
               gute Laune, {\pb}gutes Wetter, gute Sti{\geminationm}ung und lebhafte Empfindung Ihrer Freiheit und Ihres
               Lebens.\pend
           \pstart
           Herzliche Grüße Ihr{\\[\baselineskip]}\spacefill\mbox{Arthur}\pend
           \leftskip=0em{}
         
         \endnumbering\mylabel{h}\end{ledgroupsized}  \newcommand{\dateiname}{L00489}\newcommand{\titel}{Arthur Schnitzler an Richard Beer-Hofmann, 21. 9. 1895}\newcommand{\editorInnen}{Martin Anton Müller und Gerd-Hermann Susen}%% latex-leseansicht-abspann.tex
%% Abspann für die Leseansicht.
%% Der Schalter \ifkorrekturansicht ist bereits durch den Vorspann gesetzt.

%% latex-abspann.tex
%% Gemeinsamer Abspann für Korrekturansicht und Leseansicht.
%% Setzt den Schalter \ifkorrekturansicht voraus (gesetzt in den
%% einbindenden Dateien latex-korrekturansicht-abspann.tex bzw.
%% latex-leseansicht-abspann.tex).
%% ---------------------------------------------------------------

\normalsize

% Das esempio-Environment wird nur in der Leseansicht benötigt
\ifkorrekturansicht\else
\newenvironment{esempio}[3]%
{
    \vspace{1.5ex}
    \rlap{\underline{#1}}
    \par
    \setlength{\parindent}{0cm}
    \nopagebreak
    \leftskip=#2cm
    \rightskip=#3cm
}
{
    \par
}
\fi

\doendnotes{C}
\bigskip
\vfill

\clearpage

\footnotesize

\ifkorrekturansicht
  \lohead{\textsc{register}}
\fi

% theindex-Environment neu definieren ohne reledmac
\makeatletter
\renewenvironment{theindex}{%
  \ifkorrekturansicht
    \section*{\indexname}%
  \else
    \subsubsection*{Index der erwähnten Entitäten}%
  \fi
  \setlength{\parindent}{0pt}%
  \setlength{\parskip}{0pt plus 0.3pt}%
  \let\item\@idxitem
}{%
  \ifkorrekturansicht\clearpage\fi
}
\makeatother

\IfFileExists{\jobname-pw.ind}{\input{\jobname-pw.ind}}{}

% Quellenangabe nur in der Leseansicht
\ifkorrekturansicht\else
% Fallback-Definitionen, falls die .tex-Datei \titel etc. nicht gesetzt hat
\providecommand{\titel}{}
\providecommand{\editorInnen}{}
\providecommand{\dateiname}{\jobname}

\vspace{3cm}

\vfill

\footnotesize
\textsc{Quelle}: \titel. Herausgegeben von {\editorInnen}. In: \emph{Arthur Schnitzler: Briefwechsel mit Autorinnen und Autoren}.
 Digitale Edition, https://schnitzler-briefe.acdh.oeaw.ac.at/{\dateiname}.html (Stand \today)
\fi

\end{document}


      