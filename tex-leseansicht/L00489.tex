%% latex-korrekturansicht-vorspann.tex
%% Vorspann für die Korrekturansicht.
%% Lädt die gemeinsame Datei latex-vorspann.tex mit gesetztem Schalter.

\newif\ifkorrekturansicht
\korrekturansichttrue

\input{../tex-inputs/latex-vorspann}


\section[Arthur Schnitzler an Richard Beer-Hofmann, 21. 9. 1895]{L00489 Arthur Schnitzler an Richard Beer-Hofmann, 21. 9. 1895}
\nopagebreak\mylabel{L00489v}
\rehead{ }\normalsize\beginnumbering\briefempfaengerindex{Beer-Hofmann, Richard@\textsc{Beer-Hofmann, Richard}!zzzSchnitzler, Arthur@\emph{von Arthur Schnitzler}!1895-09-211@{21. 9. 1895}|(be}
\toendnotes[C]{\smallbreak\pagebreak[2]}\Standort{YCGL, MSS 31.}
\physDesc{Brief, 2 Blätter, 8 Seiten, Umschlag, 1899 Zeichen
\newline{}Handschrift: 1) Bleistift, deutsche Kurrent\hspace{1em}2) schwarze Tinte, deutsche Kurrent (\noindent{}Umschlag)\hspace{1em}
\newline{}Versand: 1) mit schwarzer Tinte von unbekannter Hand nachgesandt nach »\textsc{Gardone\oindex{Gardone Riviera@\textbf{Gardone Riviera}, \emph{A.ADM3}|pw}
                                          p{[}ost{]}. r{[}estante{]}}.«  2) Stempel: »\nobreak{}Wien, 21. 9. \textcolor{gray}{9}5\nobreak{}«.  3) Stempel: »\nobreak{}\oindex{Riva del Garda@\textbf{Riva del Garda}, \emph{P.PPLA3}|pwk}{\pb}Riva, 22. 9. 95\nobreak{}«.  4) Stempel: »\nobreak{}\oindex{Gardone Riviera@\textbf{Gardone Riviera}, \emph{A.ADM3}|pwk}Gardone Riviera, 24 9 95\nobreak{}«. }
\buchAbdrucke{\weitereDrucke{Arthur Schnitzler, Richard Beer-Hofmann: \emph{Briefwechsel 1891–1931}. Wien, Zürich: \emph{Europaverlag} 1992, S. 83.} }\toendnotes[C]{\smallbreak}\pstart{}{\pb}Herrn \textsc{Dr. Richard
                     Beer-Hofmann}\pend{}\pstart{}\textsc{Riva am G}\damage{\textcolor{gray}{ardasee}}\oindex{Riva del Garda@\textbf{Riva del Garda}, \emph{P.PPLA3}|pw}\pend{}\pstart{}\textsc{post restante}\pend{}{\bigskip}\vspace{1em}
\pstart
           \raggedleft{}{\pb}21. 9. 95\pend
           \vspace{0.5em}
\pstart
           Lieber Richard, meine Karte haben Sie wohl. In \textsc{Riva}\oindex{Riva del Garda@\textbf{Riva del Garda}, \emph{P.PPLA3}|pw} iſt es \uline{mir} nemlich vor 3 Jahren paſſirt, daſs
               der Poſtbeamte mir die Briefe an mich nicht gab – ich verlangte damals die Einläufe
               durchzuſehen, da entdeckte ich meine Briefe. Und ich hatte nicht gepfiffen! –\pend
           
\pstart
           {\pb}Die \label{K_L00489-1v}\edtext{Leſeprobe\pwindex{Liebelei. Schauspiel in drei Akten@\emph{Liebelei. Schauspiel in drei Akten}|pwv}}{\lemma{\textnormal{\emph{Leſeprobe}}}\Cendnote{\textnormal{Vgl. A. S.: \emph{Tagebuch}, 18. 9. 1895.
               }}}\label{K_L00489-1} fiel gut aus. Frl. S.\pwindex{Sandrock, Adele 1863-08-19 – 1937-08-30@\textsc{Sandrock, Adele} (1863-08-19 – 1937-08-30), \emph{Schauspieler/Schauspielerin}|pw} ignorirte mich,
               aber that ſehr ergriffen von dem Stück\pwindex{Liebelei. Schauspiel in drei Akten@\emph{Liebelei. Schauspiel in drei Akten}|pwv}, Nachmittag telephonirte ſie \label{K_L00489-2v}\edtext{\textsc{en bon camerade}}{\lemma{\textnormal{\emph{en bon camerade}}}\Cendnote{\textnormal{französisch: kameradschaftlich}}}\label{K_L00489-2}.
                  So{\geminationn}enthal\pwindex{Sonnenthal, Adolf von 1834-12-21 – 1909-04-04@\textsc{Sonnenthal, Adolf von} (1834-12-21 – 1909-04-04), \emph{Schauspieler/Schauspielerin}|pw} hat
               »gute Hoffnung«. Beim 1. Akt\pwindex{Liebelei. Schauspiel in drei Akten@\emph{Liebelei. Schauspiel in drei Akten}|pwv}
               wurde viel gelacht. Vom 3.\pwindex{Liebelei. Schauspiel in drei Akten@\emph{Liebelei. Schauspiel in drei Akten}|pwv}
               verſpricht man ſich ſichre Wirkung. Dem 2.\pwindex{Liebelei. Schauspiel in drei Akten@\emph{Liebelei. Schauspiel in drei Akten}|pwv}{ }ſcheint man am wenigſtens zu vertrauen.
                  {\pb}\textsc{Mitterwurzer}\pwindex{Mitterwurzer, Friedrich 16.10.1844 – 13.02.1897@\textsc{Mitterwurzer, Friedrich} (16.10.1844 – 13.02.1897), \emph{Schauspieler/Schauspielerin}|pw} war nicht anweſend; er ſpielt aber ſicher, ließ ſich officiell entſchuldigen.
               Die \textsc{Kallina}\pwindex{Kallina, Anna 31.03.1874 – 04.01.1948@\textsc{Kallina, Anna} (31.03.1874 – 04.01.1948), \emph{Schauspieler/Schauspielerin}|pw} wird überraſchen. Dazu will \textsc{Burckhard}\pwindex{Burckhard, Max Eugen 14.07.1854 – 16.03.1912@\textsc{Burckhard, Max Eugen} (14.07.1854 – 16.03.1912), \emph{Schriftsteller/Schriftstellerin, Rechtswissenschaftler/Rechtswissenschaftlerin, Theaterleiter/Theaterleiterin}|pw} einen Einakter von \textsc{Giacosa}\pwindex{Giacosa, Giuseppe 21.10.1847 – 02.09.1906@\textsc{Giacosa, Giuseppe} (21.10.1847 – 02.09.1906), \emph{Schriftsteller/Schriftstellerin}|pw}{ }Rechte der Seele\pwindex{Rechte der Seele. Schauspiel in einem Act@\emph{Rechte der Seele. Schauspiel in einem Act}|pw} geben; während der Leſeprobe
               half er den \label{K_L00489-3v}\edtext{\textsc{Laube}\pwindex{Laube, Heinrich 1806-09-18 – 1884-08-01@\textsc{Laube, Heinrich} (1806-09-18 – 1884-08-01), \emph{Schriftsteller/Schriftstellerin, Theaterleiter/Theaterleiterin}|pw} in Sprottau\oindex{Sprottau@\textbf{Sprottau}, \emph{P.PPL}|pw}}{\lemma{\textnormal{\emph{Laube in Sprottau}}}\Cendnote{\textnormal{Die Enthüllung des Denkmals für Heinrich Laube\pwindex{Laube, Heinrich 1806-09-18 – 1884-08-01@\textsc{Laube, Heinrich} (1806-09-18 – 1884-08-01), \emph{Schriftsteller/Schriftstellerin, Theaterleiter/Theaterleiterin}|pwk} in dessen Geburtsstadt fand
                  ebenfalls am 18. 9. 1895 statt.}}}\label{K_L00489-3} ent{\pb}hüllen. Ich wünſchte ihm angenehme Enthüllung. Er
               ſagte, die Enthüllung des Fräulein \label{K_L00489-4v}\edtext{\textsc{Dandler}\pwindex{Dandler, Anna 1862-03-14 – 1930-09-17@\textsc{Dandler, Anna} (1862-03-14 – 1930-09-17), \emph{Schauspieler/Schauspielerin}|pw}}{\lemma{\textnormal{\emph{Dandler}}}\Cendnote{\textnormal{Anna Dandler\pwindex{Dandler, Anna 1862-03-14 – 1930-09-17@\textsc{Dandler, Anna} (1862-03-14 – 1930-09-17), \emph{Schauspieler/Schauspielerin}|pwk} war zeitlebens für das \emph{Münchner Hoftheater}\orgindex{Nationaltheater Muenchen@Nationaltheater München|pwk} tätig. Ob hier eine
                  sexuelle Zote (anzunehmen) oder der Wunsch ausgedrückt wird, sie ans \emph{Burgtheater}\orgindex{Burgtheater@Burgtheater|pwk} zu holen (weniger wahrscheinlich),
                  kann nicht geklärt werden.}}}\label{K_L00489-4} zöge er vor. –\pend
           
\pstart
           \textsc{Fels}\pwindex{Fels, Friedrich Michael *~1864@\textsc{Fels, Friedrich Michael} (*~1864), \emph{Journalist/Journalistin}|pw}{ }ſchreibt mir \label{K_L00489-5v}\edtext{heute}{\lemma{\textnormal{\emph{heute}}}\Cendnote{\textnormal{Friedrich M. Fels an Arthur Schnitzler, 19. 9. 1895.}}}\label{K_L00489-5}. Sie können ſich denken.
               Er appellirt an uns zuſa{\geminationm}en, die Summe iſt 25 fl. Ich
               hab ihm gleich 10 fl {\pb}geſchickt. Darf ich ihm auch für
               Sie was ſchicken? Auch an Hugo\pwindex{Hofmannsthal, Hugo von 1874-02-01 – 1929-07-15@\textsc{Hofmannsthal, Hugo von} (1874-02-01 – 1929-07-15), \emph{Schriftsteller/Schriftstellerin}|pw} wandt ich mich
               bereits. –\pend
           
\pstart
           Geſtern war ich beim »\label{K_L00489-6v}\edtext{Pelikan\pwindex{Pelikan. Schauspiel in fuenf Aufzuegen@\emph{Ein Pelikan. Schauspiel in fünf Aufzügen}|pw}}{\lemma{\textnormal{\emph{Pelikan}}}\Cendnote{\textnormal{im Burgtheater\oindex{Burgtheater@\textbf{Burgtheater}, \emph{S.THTR}|pwk}}}}\label{K_L00489-6}«. Dieſes Blaßwerden guter Stücke iſt ſeltſam. – \label{K_L00489-7v}\edtext{Heute}{\lemma{\textnormal{\emph{Heute}}}\Cendnote{\textnormal{Gegeben wurde
                  zum ersten Mal \emph{Die Doppelhochzeit}\pwindex{Doppelhochzeit@\emph{Die Doppelhochzeit}|pwk} von Victor Léon\pwindex{Leon, Victor 4.1.1858 – 23.2.1940@\textsc{Léon, Victor} (4.1.1858 – 23.2.1940), \emph{Schriftsteller/Schriftstellerin, Dramaturg/Dramaturgin}|pwk} und Heinrich von Waldberg\pwindex{Waldberg, Heinrich von 02.03.1860 – 20.10.1942@\textsc{Waldberg, Heinrich von} (02.03.1860 – 20.10.1942), \emph{Schriftsteller/Schriftstellerin}|pwk}, Musik von Josef Hellmesberger\pwindex{Hellmesberger, Josef 09.04.1855 – 26.04.1907@\textsc{Hellmesberger, Josef} (09.04.1855 – 26.04.1907), \emph{Komponist/Komponistin}|pwk}.}}}\label{K_L00489-7} geh ich zur Eröffnung der \textsc{Josefstadt}\oindex{Theater in der Josefstadt@\textbf{Theater in der Josefstadt}, \emph{Theater (K.THE)}|pw}. – Gearbeitet hab ich noch i{\geminationm}er gar nichts; heute
                  {\pb}Nacht will ich anfangen. Glauben Sie? –\pend
           
\pstart
           Das Datum der L.\pwindex{Liebelei. Schauspiel in drei Akten@\emph{Liebelei. Schauspiel in drei Akten}|pw} iſt noch nicht
               feſtgeſtellt. –\pend
           
\pstart
           Den Hugo\pwindex{Hofmannsthal, Hugo von 1874-02-01 – 1929-07-15@\textsc{Hofmannsthal, Hugo von} (1874-02-01 – 1929-07-15), \emph{Schriftsteller/Schriftstellerin}|pw} hab ich geſtern begegnet, vorgeſtern
               iſt er zurückgeko{\geminationm}en. Er ſieht gut aus,
               »wettergebräunt«. Nach und nach wird man zu allen Worten Anführungszeichen {\pb}machen müſſen – das wird dann das Ende der Literatur
               sein.\pend
           
\pstart
           Wie geht’s Ihnen? Nächſtens ſchreiben Sie mir einen Brief ſtatt einer Depeſche; da
               werde ich weniger erſchrecken und mich beſſer unterhalten. Ich wünſche Ihnen weiter
               gute Laune, {\pb}gutes Wetter, gute Sti{\geminationm}ung und lebhafte Empfindung Ihrer Freiheit und Ihres
               Lebens.\pend
           
\pstart
           Herzliche Grüße Ihr{\\[\baselineskip]}\spacefill\mbox{Arthur}\pend
           \leftskip=0em{}\selectlanguage{ngerman}\endnumbering\briefempfaengerindex{Beer-Hofmann, Richard@\textsc{Beer-Hofmann, Richard}!zzzSchnitzler, Arthur@\emph{von Arthur Schnitzler}!1895-09-211@{21. 9. 1895}|)be}\mylabel{L00489h}  \normalsize

\doendnotes{C}
\bigskip
\vfill

\clearpage

\footnotesize

\lohead{\textsc{register}}

% Definiere theindex-Environment komplett neu ohne reledmac
\makeatletter
\renewenvironment{theindex}{%
  \section*{\indexname}%
  \setlength{\parindent}{0pt}%
  \setlength{\parskip}{0pt plus 0.3pt}%
  \let\item\@idxitem
}{%
  \clearpage
}
\makeatother

\IfFileExists{\jobname-pw.ind}{\input{\jobname-pw.ind}}{}

\end{document}

      