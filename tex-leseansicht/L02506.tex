%% latex-korrekturansicht-vorspann.tex
%% Vorspann für die Korrekturansicht.
%% Lädt die gemeinsame Datei latex-vorspann.tex mit gesetztem Schalter.

\newif\ifkorrekturansicht
\korrekturansichttrue

\input{../tex-inputs/latex-vorspann}


\section[Arthur Schnitzler an Gerty Hofmannsthal, 4. 11. 1928]{L02506 Arthur Schnitzler an Gerty Hofmannsthal, 4. 11. 1928}
\nopagebreak\mylabel{L02506v}
\rehead{ }\normalsize\beginnumbering\briefempfaengerindex{Hofmannsthal, Gertrude von@\textsc{Hofmannsthal, Gertrude von}!zzzSchnitzler, Arthur@\emph{von Arthur Schnitzler}!1928-11-041@{4. 11. 1928}|(be}
\toendnotes[C]{\smallbreak\pagebreak[2]}\Standort{FDH, Hs-30997,159.}
\physDesc{Briefkarte, 240 Zeichen (Briefkarte mit Trauerrand )
\newline{}Handschrift: schwarze Tinte, deutsche Kurrent}
\buchAbdrucke{\weitereDrucke{Hugo von Hofmannsthal, Arthur Schnitzler: \emph{Briefwechsel}. Frankfurt am Main: \emph{S. Fischer} 1964, S. 397.} }\toendnotes[C]{\smallbreak}
\pstart
           {\pb}Wien\oindex{Wien@\textbf{Wien}, \emph{A.ADM2}|pw}, 4. 11. 928\pend
           \vspace{0.5em}
\pstart
           ich \label{K_L02506-1v}\edtext{danke Ihnen ſehr}{\lemma{\textnormal{\emph{danke Ihnen ſehr}}}\Cendnote{\textnormal{Eventuell eine Reaktion auf ein nicht
                  erhaltenes Beileidsschreiben anlässlich des Todes von Schnitzlers Nachbarn Ferdinand Schmutzer\pwindex{Schmutzer, Ferdinand 21.05.1870 – 26.10.1928@\textsc{Schmutzer, Ferdinand} (21.05.1870 – 26.10.1928), \emph{Maler/Malerin, Radierer/Radiererin, Fotograf/Fotografin}|pwk} am 26. 10. 1928.}}}\label{K_L02506-1}, liebe Gerty.
                  We{\geminationn}{ }Hugo\pwindex{Hofmannsthal, Hugo von 1874-02-01 – 1929-07-15@\textsc{Hofmannsthal, Hugo von} (1874-02-01 – 1929-07-15), \emph{Schriftsteller/Schriftstellerin}|pw} wieder nach \label{K_L02506-2v}\edtext{Wien\oindex{Wien@\textbf{Wien}, \emph{A.ADM2}|pw} ko{\geminationm}t}{\lemma{\textnormal{\emph{Wien kommt}}}\Cendnote{\textnormal{Hugo Hofmannsthal\pwindex{Hofmannsthal, Hugo von 1874-02-01 – 1929-07-15@\textsc{Hofmannsthal, Hugo von} (1874-02-01 – 1929-07-15), \emph{Schriftsteller/Schriftstellerin}|pwk} hielt sich in Bad Aussee\oindex{Bad Aussee@\textbf{Bad Aussee}, \emph{P.PPLA3}|pwk} auf.}}}\label{K_L02506-2}, ſo hoff ich ſehr, Sie
               beide zu ſehn. Für heute ka{\geminationn} ich noch nicht viel mehr
               ſagen. Ihre Freundſchaft tief empfindend mit vielen Grüßen an Hugo\pwindex{Hofmannsthal, Hugo von 1874-02-01 – 1929-07-15@\textsc{Hofmannsthal, Hugo von} (1874-02-01 – 1929-07-15), \emph{Schriftsteller/Schriftstellerin}|pw} und {\pb}Sie Gerty.\pend
           
\pstart
           Ihr{\\[\baselineskip]}\spacefill\mbox{Arthur}\pend
           \leftskip=0em{}\selectlanguage{ngerman}\endnumbering\briefempfaengerindex{Hofmannsthal, Gertrude von@\textsc{Hofmannsthal, Gertrude von}!zzzSchnitzler, Arthur@\emph{von Arthur Schnitzler}!1928-11-041@{4. 11. 1928}|)be}\mylabel{L02506h}  \normalsize

\doendnotes{C}
\bigskip
\vfill

\clearpage

\footnotesize

\lohead{\textsc{register}}

% Definiere theindex-Environment komplett neu ohne reledmac
\makeatletter
\renewenvironment{theindex}{%
  \section*{\indexname}%
  \setlength{\parindent}{0pt}%
  \setlength{\parskip}{0pt plus 0.3pt}%
  \let\item\@idxitem
}{%
  \clearpage
}
\makeatother

\IfFileExists{\jobname-pw.ind}{\input{\jobname-pw.ind}}{}

\end{document}

      