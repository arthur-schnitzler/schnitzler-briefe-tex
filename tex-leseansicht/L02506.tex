%% latex-leseansicht-vorspann.tex
%% Vorspann für die Leseansicht.
%% Lädt die gemeinsame Datei latex-vorspann.tex mit nicht gesetztem Schalter.

\newif\ifkorrekturansicht
\korrekturansichtfalse

\input{../tex-inputs/latex-vorspann}


         
         \renewcommand{\erwaehntePersonen}{Personen: Hugo von Hofmannsthal, Gertrude von Hofmannsthal, Ferdinand Schmutzer}
         \renewcommand{\erwaehnteOrte}{Orte: Bad Aussee, Wien}
         \renewcommand{\erwaehnteWerke}{}
               \section[Arthur Schnitzler an Gerty Hofmannsthal, 4. 11. 1928]{ Arthur Schnitzler an Gerty Hofmannsthal, 4. 11. 1928}\nopagebreak\mylabel{v}\rehead{ }\begin{ledgroupsized}[t]{13cm}\normalsize\beginnumbering\briefempfaengerindex{Hofmannsthal, Gertrude von@\textsc{Hofmannsthal, Gertrude von}!zzzSchnitzler, Arthur@\emph{von Arthur Schnitzler}!1928-11-041@{4. 11. 1928}|(be} \toendnotes[C]{\smallbreak\pagebreak[2]} \Standort{FDH, Hs-30997,159.}
\physDesc{Briefkarte, 240 Zeichen (Trauerrand )
\newline{}Handschrift: schwarze Tinte, deutsche Kurrent}\buchAbdrucke{\weitereDrucke{Hugo von Hofmannsthal, Arthur Schnitzler: \emph{Briefwechsel}. Hg. Therese Nickl und Heinrich Schnitzler. Frankfurt am Main: \emph{S. Fischer} 1964, S. 397.} }\toendnotes[C]{\smallbreak}\pstart
           {\pb}Wien\oindex{Wien@\textbf{Wien}|pw}, 4. 11. 928\pend
           \pstart
           ich \label{K_L02506-1v}\edtext{danke Ihnen ſehr}{\lemma{\textnormal{\emph{danke Ihnen ſehr}}}\Cendnote{\textnormal{Eventuell eine Reaktion auf ein nicht
                  erhaltenes Beileidsschreiben anlässlich des Todes von Schnitzlers\pwindex{Schnitzler, Arthur 15.05.1862 – 21.10.1931@\textsc{Schnitzler, Arthur} (15.05.1862 – 21.10.1931), \emph{Schriftsteller, Mediziner}|pwk} Nachbarn Ferdinand Schmutzer\pwindex{Schmutzer, Ferdinand 21.05.1870 – 26.10.1928@\textsc{Schmutzer, Ferdinand} (21.05.1870 – 26.10.1928), \emph{Maler, Radierer, Fotograf}|pwk} am 26. 10. 1928.}}}\label{K_L02506-1h}, liebe Gerty.
                  We{\geminationn}{ }Hugo\pwindex{Hofmannsthal, Hugo von 1874-02-01 – 1929-07-15@\textsc{Hofmannsthal, Hugo von} (1874-02-01 – 1929-07-15), \emph{Schriftsteller}|pw} wieder nach \label{K_L02506-2v}\edtext{Wien\oindex{Wien@\textbf{Wien}|pw} ko{\geminationm}t}{\lemma{\textnormal{\emph{Wien kommt}}}\Cendnote{\textnormal{Hugo Hofmannsthal\pwindex{Hofmannsthal, Hugo von 1874-02-01 – 1929-07-15@\textsc{Hofmannsthal, Hugo von} (1874-02-01 – 1929-07-15), \emph{Schriftsteller}|pwk} hielt sich in Bad Aussee\oindex{Bad Aussee@\textbf{Bad Aussee}|pwk} auf.}}}\label{K_L02506-2h}, ſo hoff ich ſehr, Sie
               beide zu ſehn. Für heute ka{\geminationn} ich noch nicht viel mehr
               ſagen. Ihre Freundſchaft tief empfindend mit vielen Grüßen an Hugo\pwindex{Hofmannsthal, Hugo von 1874-02-01 – 1929-07-15@\textsc{Hofmannsthal, Hugo von} (1874-02-01 – 1929-07-15), \emph{Schriftsteller}|pw} und {\pb}Sie Gerty.\pend
           \pstart
           Ihr{\\[\baselineskip]}\spacefill\mbox{Arthur}\pend
           \leftskip=0em{}
         
         \endnumbering\mylabel{h}\end{ledgroupsized}  \newcommand{\dateiname}{L02506}\newcommand{\titel}{Arthur Schnitzler an Gerty Hofmannsthal, 4. 11. 1928}\newcommand{\editorInnen}{Martin Anton Müller und Gerd-Hermann Susen}%% latex-leseansicht-abspann.tex
%% Abspann für die Leseansicht.
%% Der Schalter \ifkorrekturansicht ist bereits durch den Vorspann gesetzt.

%% latex-abspann.tex
%% Gemeinsamer Abspann für Korrekturansicht und Leseansicht.
%% Setzt den Schalter \ifkorrekturansicht voraus (gesetzt in den
%% einbindenden Dateien latex-korrekturansicht-abspann.tex bzw.
%% latex-leseansicht-abspann.tex).
%% ---------------------------------------------------------------

\normalsize

% Das esempio-Environment wird nur in der Leseansicht benötigt
\ifkorrekturansicht\else
\newenvironment{esempio}[3]%
{
    \vspace{1.5ex}
    \rlap{\underline{#1}}
    \par
    \setlength{\parindent}{0cm}
    \nopagebreak
    \leftskip=#2cm
    \rightskip=#3cm
}
{
    \par
}
\fi

\doendnotes{C}
\bigskip
\vfill

\clearpage

\footnotesize

\ifkorrekturansicht
  \lohead{\textsc{register}}
\fi

% theindex-Environment neu definieren ohne reledmac
\makeatletter
\renewenvironment{theindex}{%
  \ifkorrekturansicht
    \section*{\indexname}%
  \else
    \subsubsection*{Index der erwähnten Entitäten}%
  \fi
  \setlength{\parindent}{0pt}%
  \setlength{\parskip}{0pt plus 0.3pt}%
  \let\item\@idxitem
}{%
  \ifkorrekturansicht\clearpage\fi
}
\makeatother

\IfFileExists{\jobname-pw.ind}{\input{\jobname-pw.ind}}{}

% Quellenangabe nur in der Leseansicht
\ifkorrekturansicht\else
% Fallback-Definitionen, falls die .tex-Datei \titel etc. nicht gesetzt hat
\providecommand{\titel}{}
\providecommand{\editorInnen}{}
\providecommand{\dateiname}{\jobname}

\vspace{3cm}

\vfill

\footnotesize
\textsc{Quelle}: \titel. Herausgegeben von {\editorInnen}. In: \emph{Arthur Schnitzler: Briefwechsel mit Autorinnen und Autoren}.
 Digitale Edition, https://schnitzler-briefe.acdh.oeaw.ac.at/{\dateiname}.html (Stand \today)
\fi

\end{document}


      