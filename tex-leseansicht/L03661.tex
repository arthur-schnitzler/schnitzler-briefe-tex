%% latex-korrekturansicht-vorspann.tex
%% Vorspann für die Korrekturansicht.
%% Lädt die gemeinsame Datei latex-vorspann.tex mit gesetztem Schalter.

\newif\ifkorrekturansicht
\korrekturansichttrue

\input{../tex-inputs/latex-vorspann}


\section[Stefan Zweig an Arthur Schnitzler, 15. 8. 1917]{L03661 Stefan Zweig an Arthur Schnitzler, 15. 8. 1917}
\nopagebreak\mylabel{L03661v}
\rehead{ }\normalsize\beginnumbering\briefempfaengerindex{Schnitzler, Arthur@\textsc{Schnitzler, Arthur}!zzzZweig, Stefan@\emph{von Stefan Zweig}!1917-08-151@{15. 8. 1917}|(be}
\toendnotes[C]{\smallbreak\pagebreak[2]}\Standort{CUL, Schnitzler, B 118.}
\physDesc{Brief, 2 Blätter, 6 Seiten, 4717 Zeichen
\newline{}Handschrift: schwarze Tinte, lateinische Kurrent
\newline{}Schnitzler: 1) mit Bleistift »\uline{\textsc{Zweig}}«  2) mit rotem Buntstift vier Unterstreichungen}
\buchAbdrucke{\weitereDrucke{Stefan Zweig: \emph{Briefwechsel mit Hermann Bahr, Sigmund Freud, Rainer Maria
                        Rilke und Arthur Schnitzler}. Frankfurt am Main: \emph{S. Fischer} 1987, S. 403–405.} }\toendnotes[C]{\smallbreak}
\pstart
           {\pb}Kalksburg bei Wien\oindex{Haselbrunnerstrasse 12@\textbf{Haselbrunnerstraße 12}, \emph{Wohngebäude (K.WHS)}|pw}, 1\substVorne{}\textsuperscript{4}\substDazwischen{}5\substHinten{}. August 1917\pend
           
\pstart{}Lieber verehrter Herr Doktor,\pend\vspace{0.5em}
\pstart
           ein neues Buch\pwindex{Doktor Graesler, Badearzt@\emph{Doktor Gräsler, Badearzt}|pwv} von Ihnen ist
               so gute und erwünschte Gabe, dass, kaum es mir gestern zu Händen kam, ich es sofort
               an dem heutigen dienstfreien Tage gelesen habe und meinen Dank dafür mit ein paar
               (gewiss unzulänglichen) Worten in Ihre Wiener Adresse\oindex{Sternwartestrasse 71@\textbf{Sternwartestraße 71}, \emph{Wohngebäude (K.WHS)}|pwv} geben will, unkund ob es Sie hier oder in
               irgend einer Landschaft erreicht. Ich danke Ihnen aufrichtig: denn von Ihren
               Prosabüchern habe ich immer einen künstlerischen Gewinn über die rein menschliche
               Anteilnahme hinaus. Nur wer selbst vom Handwerk ist, kann diese ganz ins Unscheinbare
               verborgene Contrapunktik der Novelle\pwindex{Doktor Graesler, Badearzt@\emph{Doktor Gräsler, Badearzt}|pwv} würdigen, wie Sie (besonders in der Figur der Schwester\pwindex{Doktor Graesler, Badearzt@\emph{Doktor Gräsler, Badearzt}|pwv}) scheinbar Wichtigstes
               vorausnimmt, {\pb}und zu erledigen scheint,
               innerlich aber doch tätig sein lässt, um in überraschender Verwandlung den
               Schwerpunkt dann immer wieder neu und neu zu verschieben, sodass wie ein Kreisel die
                  Erzählung\pwindex{Doktor Graesler, Badearzt@\emph{Doktor Gräsler, Badearzt}|pwv} nie fällt,
               sondern in ständiger anreizender Schwebe bleibt. Diese Überraschungen, die aus allen
               Characteren hier vorbrechen und im tiefern Sinne doch wieder nicht überraschen, weil
               sie logisch sind, bilden für mich die Meisterhaftigkeit der Novelle\pwindex{Doktor Graesler, Badearzt@\emph{Doktor Gräsler, Badearzt}|pwv}: immer geht sie den Weg, den man
               nicht vermutet und immer in ein Ziel hinein. So wird auch der innerlich trockene und
               mir eigentlich wenig wichtige Mensch, als den ich Gräsler\pwindex{Doktor Graesler, Badearzt@\emph{Doktor Gräsler, Badearzt}|pwv} empfinde, ohne dass er eigentlich problematisch
               wäre, ungemein interessant, weil er, gleichsam aus sich selbst erwachend, sich immer
               an anderer Stelle findet, als er eigentlich wollte. Vielleicht war gerade dies Ihre
               innere (und dann unendlich sublim geführte) Absicht, hier einer persönlichen
               Primitivität, deren {\pb}Pedanterie Sie doch
               so nachdrücklich betonen, das Unerwartete und Ungemässe als Conflict und Contrast zu
               geben. Wirklich es ist ein Weg von Überraschung in Überraschung, dieses Buch\pwindex{Doktor Graesler, Badearzt@\emph{Doktor Gräsler, Badearzt}|pwv}!\pend
           
\pstart
           Freilich, wie es zu Ende ist, halte auch ich inne! Der Kreisel fällt kraftlos zu
               Boden, wie ihm die Schnelle des Wirbels fehlt und den Doktor Gräsler\pwindex{Doktor Graesler, Badearzt@\emph{Doktor Gräsler, Badearzt}|pwv} fasse und fühle ich nicht mehr ganz auf den
               letzten Blättern. Sein Entschluss, ist es Resignation, Schwäche, Unsicherheit – sein
               Leben ist es zuende, oder vielmehr, beginnt nicht hier das eigentlich Tragische
               seiner Existenz? Ich bin aufrichtig genug gegen Sie – oder vielleicht gegen mich
               (denn gegen den nicht genug Gestaltenden oder gegen den nicht geung Verstehenden
               wendet sich dieser Einspruch) um zu sagen, dass ich den Abschluss nicht als
               Abschluss, nicht als restlose Auflösung empfinde. Die Gestalten des Buches sind mit
               seltener Meisterschaft, eine nach der andern, in ihrer irdischen und seelischen Form
               abgeschlossen, {\pb}er selbst der Tragende,
               der Mittelpunkt, ist mir noch in der Schwebe des Schicksals. Vielleicht fehlt nur
               hier eine Einsicht, aber da ich Ihnen in aller Verehrung doch als Aufrichtiger
               gegenüber stehe, muss ich bekennen, dass mein angereizter Hunger des Miterlebens sich
               nicht gesättigt empfindet und ich habe mir über den Rand des Buches\pwindex{Doktor Graesler, Badearzt@\emph{Doktor Gräsler, Badearzt}|pwv} nachträumend in verschiedensten
               Formen diese Existenz weitergedichtet. Aber vielleicht ist dies ja das Beste an einem
               Buche, wenn es nicht nur das passive Geniessen befriedigt, sondern noch eine
               geheimnisvolle Gährung des Gefühls zurücklässt, die selber noch einmal die Gestalten
               umwühlt und verwandelt.\pend
           
\pstart
           Nochmals, aus ganzem Herzen meinen Dank! In den nächsten Tagen sage ich ihn auch
               durch das gestaltete Wort, durch mein neues Buch\pwindex{Jeremias. Ein dramatische Dichtung in neun Bildern@\emph{Jeremias. Ein dramatische Dichtung in neun Bildern}|pwv}. In diesen drei Jahren erniedrigenden, urlaublosen,
               täglichen Dienstes habe ich mit Anspannung aller Kräfte endlich dieses {\pb}Werk\pwindex{Jeremias. Ein dramatische Dichtung in neun Bildern@\emph{Jeremias. Ein dramatische Dichtung in neun Bildern}|pwv} vollendet, das mein enziger Trost, meine innere
               Sicherheit gegen den Widersinn der Zeit war. Bewusst habe ich die Gesetze des realen
               Theaters missachtet und wie Sie es im Medardus\pwindex{junge Medardus. Dramatische Historie in einem Vorspiel und fuenf Aufzuegen@\emph{Der junge Medardus. Dramatische Historie in einem Vorspiel und fünf Aufzügen}|pw}
               taten, die Grenze von Raum und Zeit weit überschritten. Selbstverständlich kann es,
               schon aus Censurgründen, kein Theater während der Kriegszeit spielen, aber ich habe
               schon Annahmen und Zusicherungen für später und das Bewusstsein, nicht ganz vergebens
               dreier Jahre gepresste freie Stunden unter Aufgabe aller Geselligkeit, aller
               Freundschaft, aller Freude an dieses Werk gewandt zu haben. Jetzt freilich schlägt
               mir die Müdigkeit schwer in den Nacken: ich frage mich warum es mir als Einzigen
               versagt ist (nehme ich Werfel\pwindex{Werfel, Franz 10.09.1890 – 26.08.1945@\textsc{Werfel, Franz} (10.09.1890 – 26.08.1945), \emph{Schriftsteller/Schriftstellerin}|pw} aus) einmal
               einen Monat frei und sich selbst gehörig leben zu dürfen und nicht täglich, nun fast
               1000 Tage schon, in ein so stumpfsinniges Joch gezwängt zu sein. Aber ich klage nicht
               mehr: das Stück\pwindex{Jeremias. Ein dramatische Dichtung in neun Bildern@\emph{Jeremias. Ein dramatische Dichtung in neun Bildern}|pwv} selbst ist ja
               meine verwandelte und erhobe{\pb}ne Klage und
               Anklage wider die Zeit.\pend
           
\pstart
           In \label{K_L03661-1v}\edtext{wenigen Tagen}{\lemma{\textnormal{\emph{wenigen Tagen}}}\Cendnote{\textnormal{Das verzögerte sich noch, vgl. Stefan Zweig an Arthur Schnitzler, 25. 8. 1917.}}}\label{K_L03661-1} ist es in Ihren
               Händen und wenn ein oder das andere daraus ihrem Herzen \strikeout{\textcolor{gray}{×}\-\textcolor{gray}{×}\-\textcolor{gray}{×}\-\textcolor{gray}{×}} nah wird, \strikeout{ich} empfinde ich viel als verklärt und entschuldigt.
               Ich grüsse Sie und Ihre verehrte Frau Gemahlin\pwindex{Schnitzler, Olga 17.01.1882 – 13.01.1970@\textsc{Schnitzler, Olga} (17.01.1882 – 13.01.1970), \emph{Schauspieler/Schauspielerin, Sänger/Sängerin}|pwv} in alter Treue und Verehrung! Ihr\pend
           \pstart \spacefill\mbox{Stefan Zweig}\pend{}\selectlanguage{ngerman}\endnumbering\briefempfaengerindex{Schnitzler, Arthur@\textsc{Schnitzler, Arthur}!zzzZweig, Stefan@\emph{von Stefan Zweig}!1917-08-151@{15. 8. 1917}|)be}\mylabel{L03661h}
\begin{anhang}
\end{anhang}\normalsize

\doendnotes{C}
\bigskip
\vfill

\clearpage

\footnotesize

\lohead{\textsc{register}}

% Definiere theindex-Environment komplett neu ohne reledmac
\makeatletter
\renewenvironment{theindex}{%
  \section*{\indexname}%
  \setlength{\parindent}{0pt}%
  \setlength{\parskip}{0pt plus 0.3pt}%
  \let\item\@idxitem
}{%
  \clearpage
}
\makeatother

\IfFileExists{\jobname-pw.ind}{\input{\jobname-pw.ind}}{}

\end{document}

      