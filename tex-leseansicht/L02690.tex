%% latex-leseansicht-vorspann.tex
%% Vorspann für die Leseansicht.
%% Lädt die gemeinsame Datei latex-vorspann.tex mit nicht gesetztem Schalter.

\newif\ifkorrekturansicht
\korrekturansichtfalse

\input{../tex-inputs/latex-vorspann}


\section[Paul Goldmann an Arthur Schnitzler, {{[}}30.? 1. 1896{{]}}]{L02690 Paul Goldmann an Arthur Schnitzler, {[}30.? 1. 1896{]}}
\nopagebreak\mylabel{L02690v}
\rehead{ }\normalsize\beginnumbering\briefempfaengerindex{Schnitzler, Arthur@\textsc{Schnitzler, Arthur}!zzzGoldmann, Paul@\emph{von Paul Goldmann}!1896-01-301@{{[}30.? 1. 1896{]}}|(be}
\toendnotes[C]{\smallbreak\pagebreak[2]}
\correspDesc{Versand  durch Paul Goldmann am [30.? 1. 1896] in Paris
\newline{}Erhalt  durch Arthur Schnitzler am [30.? 1. 1896] in Berlin}\toendnotes[C]{\smallbreak}
\Standort{DLA, A:Schnitzler, HS.NZ85.1.3166.}
\physDesc{Telegramm, 221 Zeichen
\newline{}maschinell
\newline{}Schnitzler: mit Bleistift datiert: »Jan\textcolor{gray}{u} 96« 
\newline{}Ordnung: beschnitten }\toendnotes[C]{\smallbreak}
\pstart
           \centering{}{\pb}w\oindex{Wien@\textbf{Wien}, \emph{Verwaltungsgebiet}|pw} de { }paris\oindex{Paris@\textbf{Paris}, \emph{Hauptstadt}|pw} 18798. \label{K_L02690-1v}\edtext{30.}{\lemma{\textnormal{\emph{30.}}}\Cendnote{\textnormal{Vermutlich der Kalendertag, an dem
                     das Telegramm versandt wurde.}}}\label{K_L02690-1}{ }12\textcolor{gray}{.} =\pend
           \vspace{0.5em}
\pstart
           vielen dank fuer liebes \label{K_L02690-2v}\edtext{anerbieten}{\lemma{\textnormal{\emph{anerbieten}}}\Cendnote{\textnormal{Unter der Voraussetzung,
                  dass die Datierung stimmt, könnte es sich um eine Einladung nach Berlin\oindex{Berlin@\textbf{Berlin}, \emph{Hauptstadt}|pwk} gehandelt haben, wo am 4. 2. 1896 die Premiere von \emph{Liebelei}\pwindex{Schnitzler, Arthur 15.\,5.\,1862 Wien – 21.\,10.\,1931 ebd.@\textsc{Schnitzler, Arthur} (15.\,5.\,1862 Wien – 21.\,10.\,1931 ebd.), \emph{Schriftsteller, Mediziner}!Liebelei. Schauspiel in drei Akten@\strich\emph{Liebelei. Schauspiel in drei Akten}|pwk} am \emph{Deutschen Theater}\orgindex{Deutsches Theater Berlin@Deutsches Theater Berlin|pwk} bevorstand. Da Schnitzler an diesem Tag bereits in Berlin\oindex{Berlin@\textbf{Berlin}, \emph{Hauptstadt}|pwk} ankam, bleibt unklar, ob das Telegramm dahin gesandt wurde, von
                     Wien\oindex{Wien@\textbf{Wien}, \emph{Verwaltungsgebiet}|pwk} nachgesandt wurde oder (am
                  unwahrscheinlichsten) bis zur Rückkehr nach Wien\oindex{Wien@\textbf{Wien}, \emph{Verwaltungsgebiet}|pwk} am 11. 2. 1896 liegen blieb.}}}\label{K_L02690-2} aber leider unmoeglich aus
               zahlreichen gruenden hauptsaechlich geldmangel und schwierigkeit inmitten saison ohne
               zwingendsten grund urlaub zu bekommen\pend
           \pstart gruss = \spacefill\mbox{goldmann}\pend{}\selectlanguage{ngerman}\endnumbering\briefempfaengerindex{Schnitzler, Arthur@\textsc{Schnitzler, Arthur}!zzzGoldmann, Paul@\emph{von Paul Goldmann}!1896-01-301@{{[}30.? 1. 1896{]}}|)be}\mylabel{L02690h}  \newcommand{\dateiname}{L02690}\newcommand{\titel}{Paul Goldmann an Arthur Schnitzler, [30.? 1. 1896]}\newcommand{\editorInnen}{Martin Anton Müller und Laura Untner}%% latex-leseansicht-abspann.tex
%% Abspann für die Leseansicht.
%% Der Schalter \ifkorrekturansicht ist bereits durch den Vorspann gesetzt.

%% latex-abspann.tex
%% Gemeinsamer Abspann für Korrekturansicht und Leseansicht.
%% Setzt den Schalter \ifkorrekturansicht voraus (gesetzt in den
%% einbindenden Dateien latex-korrekturansicht-abspann.tex bzw.
%% latex-leseansicht-abspann.tex).
%% ---------------------------------------------------------------

\normalsize

% Das esempio-Environment wird nur in der Leseansicht benötigt
\ifkorrekturansicht\else
\newenvironment{esempio}[3]%
{
    \vspace{1.5ex}
    \rlap{\underline{#1}}
    \par
    \setlength{\parindent}{0cm}
    \nopagebreak
    \leftskip=#2cm
    \rightskip=#3cm
}
{
    \par
}
\fi

\doendnotes{C}
\bigskip
\vfill

\clearpage

\footnotesize

\ifkorrekturansicht
  \lohead{\textsc{register}}
\fi

% theindex-Environment neu definieren ohne reledmac
\makeatletter
\renewenvironment{theindex}{%
  \ifkorrekturansicht
    \section*{\indexname}%
  \else
    \subsubsection*{Index der erwähnten Entitäten}%
  \fi
  \setlength{\parindent}{0pt}%
  \setlength{\parskip}{0pt plus 0.3pt}%
  \let\item\@idxitem
}{%
  \ifkorrekturansicht\clearpage\fi
}
\makeatother

\IfFileExists{\jobname-pw.ind}{\input{\jobname-pw.ind}}{}

% Quellenangabe nur in der Leseansicht
\ifkorrekturansicht\else
% Fallback-Definitionen, falls die .tex-Datei \titel etc. nicht gesetzt hat
\providecommand{\titel}{}
\providecommand{\editorInnen}{}
\providecommand{\dateiname}{\jobname}

\vspace{3cm}

\vfill

\footnotesize
\textsc{Quelle}: \titel. Herausgegeben von {\editorInnen}. In: \emph{Arthur Schnitzler: Briefwechsel mit Autorinnen und Autoren}.
 Digitale Edition, https://schnitzler-briefe.acdh.oeaw.ac.at/{\dateiname}.html (Stand \today)
\fi

\end{document}


