%% latex-korrekturansicht-vorspann.tex
%% Vorspann für die Korrekturansicht.
%% Lädt die gemeinsame Datei latex-vorspann.tex mit gesetztem Schalter.

\newif\ifkorrekturansicht
\korrekturansichttrue

\input{../tex-inputs/latex-vorspann}


\section[Paul Goldmann an Arthur Schnitzler, {[}30.? 1. 1896{]}]{L02690 Paul Goldmann an Arthur Schnitzler, {[}30.? 1. 1896{]}}
\nopagebreak\mylabel{L02690v}
\rehead{ }\normalsize\beginnumbering\briefempfaengerindex{Schnitzler, Arthur@\textsc{Schnitzler, Arthur}!zzzGoldmann, Paul@\emph{von Paul Goldmann}!1896-01-301@{{[}30.? 1. 1896{]}}|(be}
\toendnotes[C]{\smallbreak\pagebreak[2]}\Standort{DLA, A:Schnitzler, HS.NZ85.1.3166.}
\physDesc{Telegramm, 221 Zeichen
\newline{}maschinell
\newline{}Schnitzler: mit Bleistift datiert: »Jan\textcolor{gray}{u} 96« 
\newline{}Ordnung: beschnitten }\toendnotes[C]{\smallbreak}
\pstart
           \centering{}{\pb}w\oindex{Wien@\textbf{Wien}, \emph{A.ADM2}|pw} de { }paris\oindex{Paris@\textbf{Paris}, \emph{P.PPLC}|pw} 18798. \label{K_L02690-1v}\edtext{30.}{\lemma{\textnormal{\emph{30.}}}\Cendnote{\textnormal{Vermutlich der Kalendertag, an dem
                     das Telegramm versandt wurde.}}}\label{K_L02690-1}{ }12\textcolor{gray}{.} =\pend
           \vspace{0.5em}
\pstart
           vielen dank fuer liebes \label{K_L02690-2v}\edtext{anerbieten}{\lemma{\textnormal{\emph{anerbieten}}}\Cendnote{\textnormal{Unter der Voraussetzung,
                  dass die Datierung stimmt, könnte es sich um eine Einladung nach Berlin\oindex{Berlin@\textbf{Berlin}, \emph{P.PPLC}|pwk} gehandelt haben, wo am 4. 2. 1896 die Premiere von \emph{Liebelei}\pwindex{Liebelei. Schauspiel in drei Akten@\emph{Liebelei. Schauspiel in drei Akten}|pwk} am \emph{Deutschen Theater}\orgindex{Deutsches Theater Berlin@Deutsches Theater Berlin|pwk} bevorstand. Da Schnitzler an diesem Tag bereits in Berlin\oindex{Berlin@\textbf{Berlin}, \emph{P.PPLC}|pwk} ankam, bleibt unklar, ob das Telegramm dahin gesandt wurde, von
                     Wien\oindex{Wien@\textbf{Wien}, \emph{A.ADM2}|pwk} nachgesandt wurde oder (am
                  unwahrscheinlichsten) bis zur Rückkehr nach Wien\oindex{Wien@\textbf{Wien}, \emph{A.ADM2}|pwk} am 11. 2. 1896 liegen blieb.}}}\label{K_L02690-2} aber leider unmoeglich aus
               zahlreichen gruenden hauptsaechlich geldmangel und schwierigkeit inmitten saison ohne
               zwingendsten grund urlaub zu bekommen\pend
           \pstart gruss = \spacefill\mbox{goldmann}\pend{}\selectlanguage{ngerman}\endnumbering\briefempfaengerindex{Schnitzler, Arthur@\textsc{Schnitzler, Arthur}!zzzGoldmann, Paul@\emph{von Paul Goldmann}!1896-01-301@{{[}30.? 1. 1896{]}}|)be}\mylabel{L02690h}  \normalsize

\doendnotes{C}
\bigskip
\vfill

\clearpage

\footnotesize

\lohead{\textsc{register}}

% Definiere theindex-Environment komplett neu ohne reledmac
\makeatletter
\renewenvironment{theindex}{%
  \section*{\indexname}%
  \setlength{\parindent}{0pt}%
  \setlength{\parskip}{0pt plus 0.3pt}%
  \let\item\@idxitem
}{%
  \clearpage
}
\makeatother

\IfFileExists{\jobname-pw.ind}{\input{\jobname-pw.ind}}{}

\end{document}

      