%% latex-leseansicht-vorspann.tex
%% Vorspann für die Leseansicht.
%% Lädt die gemeinsame Datei latex-vorspann.tex mit nicht gesetztem Schalter.

\newif\ifkorrekturansicht
\korrekturansichtfalse

\input{../tex-inputs/latex-vorspann}


\section[Arthur Schnitzler an Hugo Hofmannsthal, 1. 10. 1919]{L02327 Arthur Schnitzler an Hugo Hofmannsthal, 1. 10. 1919}
\nopagebreak\mylabel{L02327v}
\rehead{ }\normalsize\beginnumbering\briefempfaengerindex{Hofmannsthal, Hugo von@\textsc{Hofmannsthal, Hugo von}!zzzSchnitzler, Arthur@\emph{von Arthur Schnitzler}!1919-10-011@{1. 10. 1919}|(be}
\toendnotes[C]{\smallbreak\pagebreak[2]}
\correspDesc{Versand  durch Arthur Schnitzler am 1. 10. 1919 in Wien
\newline{}Erhalt  durch Hugo von Hofmannsthal im Zeitraum [2. 10. 1919
                  – 6. 10. 1919?] in Bad Aussee}\toendnotes[C]{\smallbreak}
\Standort{FDH, Hs-30885,149.}
\physDesc{Brief, 2 Blätter, 3 Seiten, 3118 Zeichen
\newline{}Handschrift: Bleistift, lateinische Kurrent}
\buchAbdrucke{\weitereDrucke{1) Hugo von Hofmannsthal, Arthur Schnitzler: \emph{Briefwechsel}. Herausgegeben von Therese Nickl und Heinrich Schnitzler. Frankfurt am Main: \emph{S. Fischer} 1964, S. 285–286.} \weitereDrucke{2) Arthur Schnitzler: \emph{Briefe 1913–1931}. Herausgegeben von Peter Michael Braunwarth, Richard Miklin, Susanne Pertlik und Heinrich Schnitzler. Frankfurt am Main: \emph{S. Fischer} 1984, S. 195–197.} }\toendnotes[C]{\smallbreak}
\pstart
           \raggedleft{}{\pb}1. 10. 19{\\}Wien\oindex{Wien@\textbf{Wien}, \emph{Verwaltungsgebiet}|pw}\pend
           \vspace{0.5em}
\pstart
           mein lieber Hugo, vor ein paar Wochen schon hat mir die Hofrätin\pwindex{Zuckerkandl, Berta 13.\,4.\,1864 Wien – 16.\,10.\,1945 Paris@\textsc{Zuckerkandl, Berta} (13.\,4.\,1864 Wien – 16.\,10.\,1945 Paris), \emph{Schriftstellerin, Journalistin, Übersetzerin}|pw} gesagt, Sie seien auf einen Brief an
               mich ohne Antwort geblieben; ich will Ihnen nur mittheilen, dſs Ihr Brief vom
                  19. 9. der erste ist, den ich seit vielen Monaten von Ihnen erhielt –
               der letzte berichtete von Ihrem leidenden Zustand und ich schrieb Ihnen darauf, dſs
               ich gern einmal zu Ihnen nach Rodaun\oindex{Wien@\textbf{Wien}!XXIII., Liesing@\textbf{XXIII., Liesing}!Rodaun@\textbf{Rodaun}, \emph{Region}|pw} käme, aber
               darauf hatt ich von Ihnen nichts weiter gehört. Nun freuts mich sehr dſs die neueste
               Kunde so arbeitsfroh und hoffnungsvoll klingt und es wäre wahrhaftig schön, we{\geminationn} man wieder einmal einer jener feiertäglichen
               Vorlesestunden entgegensehen dürfte – die nur im Lauf der Jahre um so viel seltener
               geworden sind als selbst die seltensten Feiertage. Und was für eine Reihe von
               festlich ergreifenden Abenden – von jenem ersten an, an dem Sie mir, an einem warmen
                  \label{K_L02327-1v}\edtext{Juniabend}{\lemma{\textnormal{\emph{Juniabend}}}\Cendnote{\textnormal{Siehe A. S.: \emph{Tagebuch}, 7. 10. 1891.
               }}}\label{K_L02327-1} war es, in der Giselastraße\oindex{Wien@\textbf{Wien}!I., Innere Stadt@\textbf{I., Innere Stadt}!Ordination Arthur Schnitzler [Bösendorferstraße 11]@\textbf{Ordination Arthur Schnitzler [Bösendorferstraße 11]}, \emph{Ordination}|pw}, »Gestern\pwindex{Hofmannsthal, Hugo von 1.\,2.\,1874 Wien – 15.\,7.\,1929 Rodaun@\textsc{Hofmannsthal, Hugo von} (1.\,2.\,1874 Wien – 15.\,7.\,1929 Rodaun), \emph{Schriftsteller}!Gestern. Dramatische Studie in einem Akt in Versen@\strich\emph{Gestern. Dramatische Studie in einem Akt in Versen}|pw}« vorlasen – oder war ich es, der mit dem
                  »\label{K_L02327-2v}\edtext{Märchen\pwindex{Schnitzler, Arthur 15.\,5.\,1862 Wien – 21.\,10.\,1931 ebd.@\textsc{Schnitzler, Arthur} (15.\,5.\,1862 Wien – 21.\,10.\,1931 ebd.), \emph{Schriftsteller, Mediziner}!Märchen. Schauspiel in drei Aufzügen@\strich\emph{Das Märchen. Schauspiel in drei Aufzügen}|pw}}{\lemma{\textnormal{\emph{Märchen}}}\Cendnote{\textnormal{Diese Lesung fand am 25. 6. 1891 in der Seidlgasse\oindex{Wien@\textbf{Wien}!III., Landstraße@\textbf{III., Landstraße}!Seidlgasse@\textbf{Seidlgasse}, \emph{Straße}|pwk} statt. Aber bereits früher lassen
                  sich solche Lesungen im privaten Kreis nachweisen.}}}\label{K_L02327-2}« anfing, in der Seidlgasse\oindex{Wien@\textbf{Wien}!III., Landstraße@\textbf{III., Landstraße}!Seidlgasse@\textbf{Seidlgasse}, \emph{Straße}|pw}, bei Richard\pwindex{Beer-Hofmann, Richard 11.\,7.\,1866 Wien – 26.\,9.\,1945 New York City@\textsc{Beer-Hofmann, Richard} (11.\,7.\,1866 Wien – 26.\,9.\,1945 New York City), \emph{Schriftsteller}|pw} – ich weiß nicht mehr? Es kam wirklich wenig darauf an, ob das Werk
               als solches mehr oder weniger vollendet war – der Beifall geringer oder größer – im
               Rückblick bleiben es durchaus Stunden der kräftigsten, belebtesten Atmosphäre –
               bessere, reinere: als wenn man dasselbe Werk zum ersten Mal der Oeffentlich{\pb}keit zu praesentiren hatte. Ich bin höchst gespannt was
               Sie aus Altaussee\oindex{Altaussee@\textbf{Altaussee}, \emph{Verwaltungsgebiet}|pw} mitbringen werden. Mit meiner
               Arbeit (Stück\pwindex{Schnitzler, Arthur 15.\,5.\,1862 Wien – 21.\,10.\,1931 ebd.@\textsc{Schnitzler, Arthur} (15.\,5.\,1862 Wien – 21.\,10.\,1931 ebd.), \emph{Schriftsteller, Mediziner}!Gang zum Weiher. Dramatische Dichtung@\strich\emph{Der Gang zum Weiher. Dramatische Dichtung}|pwv}) geht es so
               langsam vorwärts, dſs ich fast von einem Stillstand sprechen kann – obzwar ich die
               Continuität zum mindesten durch beharrliches Anstarren unbeschriebener Papierblätter
               oder Ausstreichen des Geschriebenen festzuhalten versuche. Das letzte, was ich fertig
               gemacht \introOben{}habe\introOben{}, sind die »Schwestern\pwindex{Schnitzler, Arthur 15.\,5.\,1862 Wien – 21.\,10.\,1931 ebd.@\textsc{Schnitzler, Arthur} (15.\,5.\,1862 Wien – 21.\,10.\,1931 ebd.), \emph{Schriftsteller, Mediziner}!Schwestern oder Casanova in Spa. Lustspiel in Versen@\strich\emph{Die Schwestern oder Casanova in Spa. Lustspiel in Versen}|pw}«, die bei Reinhardt\pwindex{Reinhardt, Max 9.\,9.\,1873 Baden bei Wien – 30.\,10.\,1943 New York City@\textsc{Reinhardt, Max} (9.\,9.\,1873 Baden bei Wien – 30.\,10.\,1943 New York City), \emph{Theaterleiter, Regisseur, Schauspieler}|pw} kommen
               sollen; – mir selbst ist selten was von mir so lieb gewesen. Ich hab allerlei vor,
               manches aus den letzten Jahren ist sogar recht weit gediehen; aber meine Arbeitskraft
               ist – wohl unter dem Einfluss dieses grauenhaften Weltzustandes – so tief herunter
               wie noch nie. Zu einer größern Reise hab ich mich nicht entschließen können, nun lädt
               mich meine Schwägerin\pwindex{Steinrück, Elisabeth 19.\,11.\,1885 – 7.\,4.\,1920 Partenkirchen@\textsc{Steinrück, Elisabeth} (19.\,11.\,1885 – 7.\,4.\,1920 Partenkirchen)|pwv}{ }sehr dringend nach Partenkirchen\oindex{Partenkirchen@\textbf{Partenkirchen}, \emph{Teil eines besiedelten Ortes}|pw} (wohin auch Olga\pwindex{Schnitzler, Olga 17.\,1.\,1882 Wien – 13.\,1.\,1970 Lugano@\textsc{Schnitzler, Olga} (17.\,1.\,1882 Wien – 13.\,1.\,1970 Lugano), \emph{Schauspielerin, Sängerin}|pw} im
               Anschluss an ein Münchn\oindex{München@\textbf{München}|pw}er Concert\strikeout{)} gehen wird); aber mich graut vor Wartesälen,
               Bahncoupés, Zollvisitationen, Gepäckaufgeben; und so wird auch daraus kaum was
               werden. Ich bin in diesem Sommer {\pb}nur in Reichenau\oindex{Reichenau an der Rax@\textbf{Reichenau an der Rax}, \emph{Verwaltungsgebiet}|pw} gewesen, \label{K_L02327-3v}\edtext{einmal zehn Tage}{\lemma{\textnormal{\emph{einmal zehn Tage}}}\Cendnote{\textnormal{vom 7. 8. 1919 bis
                  zum 20. 8. 1919}}}\label{K_L02327-3} (mit all den Meinen) einmal \label{K_L02327-4v}\edtext{drei Tage}{\lemma{\textnormal{\emph{drei Tage}}}\Cendnote{\textnormal{vom 8. 9. 1919 bis zum 12. 9. 1919}}}\label{K_L02327-4}; – das ist für mich ein Ort so erfüllt von Erinnerungen der mannigfachsten
               Art, dſs ich ihnen, in der schweren Sti{\geminationm}ung dieser So{\geminationm}ertage, kaum gewachsen war. Immerhin wurden mir in
               tausend und mehr Metern Höhe, auf Wiesen, an Waldesrand, ein paar gute Stunden.\pend
           
\pstart
           – We{\geminationn} nicht früher mein lieber Hugo so sehe ich Sie wohl
               bei der \label{K_L02327-5v}\edtext{Generalprobe}{\lemma{\textnormal{\emph{Generalprobe}}}\Cendnote{\textnormal{Vgl. A. S.: \emph{Tagebuch}, 8. 10. 1919.
               }}}\label{K_L02327-5} der sonnigen Frau\pwindex{Strauss, Richard 11.\,6.\,1864 München – 8.\,9.\,1949 Garmisch-Partenkirchen@\textsc{Strauss, Richard} (11.\,6.\,1864 München – 8.\,9.\,1949 Garmisch-Partenkirchen), \emph{Theaterleiter, Komponist, Dirigent}!Frau ohne Schatten. Oper in drei Akten@\strich\emph{Die Frau ohne Schatten. Oper in drei Akten}|pwv}\pwindex{Hofmannsthal, Hugo von 1.\,2.\,1874 Wien – 15.\,7.\,1929 Rodaun@\textsc{Hofmannsthal, Hugo von} (1.\,2.\,1874 Wien – 15.\,7.\,1929 Rodaun), \emph{Schriftsteller}!Frau ohne Schatten. Oper in drei Akten@\strich\emph{Die Frau ohne Schatten. Oper in drei Akten}|pwv} (ich
               habe Strauß\pwindex{Strauss, Richard 11.\,6.\,1864 München – 8.\,9.\,1949 Garmisch-Partenkirchen@\textsc{Strauss, Richard} (11.\,6.\,1864 München – 8.\,9.\,1949 Garmisch-Partenkirchen), \emph{Theaterleiter, Komponist, Dirigent}|pw} um Einlaß gebeten, auch für Olga\pwindex{Schnitzler, Olga 17.\,1.\,1882 Wien – 13.\,1.\,1970 Lugano@\textsc{Schnitzler, Olga} (17.\,1.\,1882 Wien – 13.\,1.\,1970 Lugano), \emph{Schauspielerin, Sängerin}|pw}, hoffentlich gehts) – ich kenne schon
               allerlei daraus vom Clavier her und freu mich ganz besonders. Haben Sie de{\geminationn} nun auch die Märchen-Erzählung\pwindex{Hofmannsthal, Hugo von 1.\,2.\,1874 Wien – 15.\,7.\,1929 Rodaun@\textsc{Hofmannsthal, Hugo von} (1.\,2.\,1874 Wien – 15.\,7.\,1929 Rodaun), \emph{Schriftsteller}!Frau ohne Schatten. Erzählung@\strich\emph{Die Frau ohne Schatten. Erzählung}|pwv}, von der Sie mir öfters sprachen – die
               denselben Stoff behandelt, fertig gemacht?\pend
           
\pstart
           – Ich schicke diese Zeilen noch nach Aussee\oindex{Bad Aussee@\textbf{Bad Aussee}, \emph{Hauptstadt}|pw}.
               Haben Sie weiterhin gute, reiche Tage!\pend
           
\pstart
           Von Herzen Ihr{\\[\baselineskip]}\spacefill\mbox{Arth}\pend
           \leftskip=0em{}\selectlanguage{ngerman}\endnumbering\briefempfaengerindex{Hofmannsthal, Hugo von@\textsc{Hofmannsthal, Hugo von}!zzzSchnitzler, Arthur@\emph{von Arthur Schnitzler}!1919-10-011@{1. 10. 1919}|)be}\mylabel{L02327h}  \newcommand{\dateiname}{L02327}\newcommand{\titel}{Arthur Schnitzler an Hugo Hofmannsthal, 1. 10. 1919}\newcommand{\editorInnen}{Martin Anton Müller und Gerd-Hermann Susen}%% latex-leseansicht-abspann.tex
%% Abspann für die Leseansicht.
%% Der Schalter \ifkorrekturansicht ist bereits durch den Vorspann gesetzt.

%% latex-abspann.tex
%% Gemeinsamer Abspann für Korrekturansicht und Leseansicht.
%% Setzt den Schalter \ifkorrekturansicht voraus (gesetzt in den
%% einbindenden Dateien latex-korrekturansicht-abspann.tex bzw.
%% latex-leseansicht-abspann.tex).
%% ---------------------------------------------------------------

\normalsize

% Das esempio-Environment wird nur in der Leseansicht benötigt
\ifkorrekturansicht\else
\newenvironment{esempio}[3]%
{
    \vspace{1.5ex}
    \rlap{\underline{#1}}
    \par
    \setlength{\parindent}{0cm}
    \nopagebreak
    \leftskip=#2cm
    \rightskip=#3cm
}
{
    \par
}
\fi

\doendnotes{C}
\bigskip
\vfill

\clearpage

\footnotesize

\ifkorrekturansicht
  \lohead{\textsc{register}}
\fi

% theindex-Environment neu definieren ohne reledmac
\makeatletter
\renewenvironment{theindex}{%
  \ifkorrekturansicht
    \section*{\indexname}%
  \else
    \subsubsection*{Index der erwähnten Entitäten}%
  \fi
  \setlength{\parindent}{0pt}%
  \setlength{\parskip}{0pt plus 0.3pt}%
  \let\item\@idxitem
}{%
  \ifkorrekturansicht\clearpage\fi
}
\makeatother

\IfFileExists{\jobname-pw.ind}{\input{\jobname-pw.ind}}{}

% Quellenangabe nur in der Leseansicht
\ifkorrekturansicht\else
% Fallback-Definitionen, falls die .tex-Datei \titel etc. nicht gesetzt hat
\providecommand{\titel}{}
\providecommand{\editorInnen}{}
\providecommand{\dateiname}{\jobname}

\vspace{3cm}

\vfill

\footnotesize
\textsc{Quelle}: \titel. Herausgegeben von {\editorInnen}. In: \emph{Arthur Schnitzler: Briefwechsel mit Autorinnen und Autoren}.
 Digitale Edition, https://schnitzler-briefe.acdh.oeaw.ac.at/{\dateiname}.html (Stand \today)
\fi

\end{document}


