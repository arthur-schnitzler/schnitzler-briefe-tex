%% latex-korrekturansicht-vorspann.tex
%% Vorspann für die Korrekturansicht.
%% Lädt die gemeinsame Datei latex-vorspann.tex mit gesetztem Schalter.

\newif\ifkorrekturansicht
\korrekturansichttrue

\input{../tex-inputs/latex-vorspann}


\section[Arthur Schnitzler an Hugo Hofmannsthal, 1. 10. 1919]{L02327 Arthur Schnitzler an Hugo Hofmannsthal, 1. 10. 1919}
\nopagebreak\mylabel{L02327v}
\rehead{ }\normalsize\beginnumbering\briefempfaengerindex{Hofmannsthal, Hugo von@\textsc{Hofmannsthal, Hugo von}!zzzSchnitzler, Arthur@\emph{von Arthur Schnitzler}!1919-10-011@{1. 10. 1919}|(be}
\toendnotes[C]{\smallbreak\pagebreak[2]}\Standort{FDH, Hs-30885,149.}
\physDesc{Brief, 2 Blätter, 3 Seiten, 3118 Zeichen
\newline{}Handschrift: Bleistift, lateinische Kurrent}
\buchAbdrucke{\weitereDrucke{1) Hugo von Hofmannsthal, Arthur Schnitzler: \emph{Briefwechsel}. Frankfurt am Main: \emph{S. Fischer} 1964, S. 285–286.} \weitereDrucke{2) Arthur Schnitzler: \emph{Briefe 1913–1931}. Frankfurt am Main: \emph{S. Fischer} 1984, S. 195–197.} }\toendnotes[C]{\smallbreak}
\pstart
           \raggedleft{}{\pb}1. 10. 19{\\}Wien\oindex{Wien@\textbf{Wien}, \emph{A.ADM2}|pw}\pend
           \vspace{0.5em}
\pstart
           mein lieber Hugo, vor ein paar Wochen schon hat mir die Hofrätin\pwindex{Zuckerkandl, Berta 13.04.1864 – 16.10.1945@\textsc{Zuckerkandl, Berta} (13.04.1864 – 16.10.1945), \emph{Journalist/Journalistin, Übersetzer/Übersetzerin}|pw} gesagt, Sie seien auf einen Brief an
               mich ohne Antwort geblieben; ich will Ihnen nur mittheilen, dſs Ihr Brief vom
                  19. 9. der erste ist, den ich seit vielen Monaten von Ihnen erhielt –
               der letzte berichtete von Ihrem leidenden Zustand und ich schrieb Ihnen darauf, dſs
               ich gern einmal zu Ihnen nach Rodaun\oindex{Rodaun@\textbf{Rodaun}, \emph{A.ADM4}|pw} käme, aber
               darauf hatt ich von Ihnen nichts weiter gehört. Nun freuts mich sehr dſs die neueste
               Kunde so arbeitsfroh und hoffnungsvoll klingt und es wäre wahrhaftig schön, we{\geminationn} man wieder einmal einer jener feiertäglichen
               Vorlesestunden entgegensehen dürfte – die nur im Lauf der Jahre um so viel seltener
               geworden sind als selbst die seltensten Feiertage. Und was für eine Reihe von
               festlich ergreifenden Abenden – von jenem ersten an, an dem Sie mir, an einem warmen
                  \label{K_L02327-1v}\edtext{Juniabend}{\lemma{\textnormal{\emph{Juniabend}}}\Cendnote{\textnormal{Siehe A. S.: \emph{Tagebuch}, 7. 10. 1891.
               }}}\label{K_L02327-1} war es, in der Giselastraße\oindex{Ordination Arthur Schnitzler [Boesendorferstrasse 11]@\textbf{Ordination Arthur Schnitzler [Bösendorferstraße 11]}, \emph{Ordination}|pw}, »Gestern\pwindex{Gestern. Dramatische Studie in einem Akt in Versen@\emph{Gestern. Dramatische Studie in einem Akt in Versen}|pw}« vorlasen – oder war ich es, der mit dem
                  »\label{K_L02327-2v}\edtext{Märchen\pwindex{Maerchen. Schauspiel in drei Aufzuegen@\emph{Das Märchen. Schauspiel in drei Aufzügen}|pw}}{\lemma{\textnormal{\emph{Märchen}}}\Cendnote{\textnormal{Diese Lesung fand am 25. 6. 1891 in der Seidlgasse\oindex{Seidlgasse@\textbf{Seidlgasse}, \emph{Straße (K.STR)}|pwk} statt. Aber bereits früher lassen
                  sich solche Lesungen im privaten Kreis nachweisen.}}}\label{K_L02327-2}« anfing, in der Seidlgasse\oindex{Seidlgasse@\textbf{Seidlgasse}, \emph{Straße (K.STR)}|pw}, bei Richard\pwindex{Beer-Hofmann, Richard 1866-07-11 – 1945-09-26@\textsc{Beer-Hofmann, Richard} (1866-07-11 – 1945-09-26), \emph{Schriftsteller/Schriftstellerin}|pw} – ich weiß nicht mehr? Es kam wirklich wenig darauf an, ob das Werk
               als solches mehr oder weniger vollendet war – der Beifall geringer oder größer – im
               Rückblick bleiben es durchaus Stunden der kräftigsten, belebtesten Atmosphäre –
               bessere, reinere: als wenn man dasselbe Werk zum ersten Mal der Oeffentlich{\pb}keit zu praesentiren hatte. Ich bin höchst gespannt was
               Sie aus Altaussee\oindex{Altaussee@\textbf{Altaussee}, \emph{A.ADM3}|pw} mitbringen werden. Mit meiner
               Arbeit (Stück\pwindex{Gang zum Weiher. Dramatische Dichtung@\emph{Der Gang zum Weiher. Dramatische Dichtung}|pwv}) geht es so
               langsam vorwärts, dſs ich fast von einem Stillstand sprechen kann – obzwar ich die
               Continuität zum mindesten durch beharrliches Anstarren unbeschriebener Papierblätter
               oder Ausstreichen des Geschriebenen festzuhalten versuche. Das letzte, was ich fertig
               gemacht \introOben{}habe\introOben{}, sind die »Schwestern\pwindex{Schwestern oder Casanova in Spa. Lustspiel in Versen@\emph{Die Schwestern oder Casanova in Spa. Lustspiel in Versen}|pw}«, die bei Reinhardt\pwindex{Reinhardt, Max 09.09.1873 – 30.10.1943@\textsc{Reinhardt, Max} (09.09.1873 – 30.10.1943), \emph{Theaterleiter/Theaterleiterin, Regisseur/Regisseurin, Schauspieler/Schauspielerin}|pw} kommen
               sollen; – mir selbst ist selten was von mir so lieb gewesen. Ich hab allerlei vor,
               manches aus den letzten Jahren ist sogar recht weit gediehen; aber meine Arbeitskraft
               ist – wohl unter dem Einfluss dieses grauenhaften Weltzustandes – so tief herunter
               wie noch nie. Zu einer größern Reise hab ich mich nicht entschließen können, nun lädt
               mich meine Schwägerin\pwindex{Steinrueck, Elisabeth 19.11.1885 – 07.04.1920@\textsc{Steinrück, Elisabeth} (19.11.1885 – 07.04.1920)|pwv}{ }sehr dringend nach Partenkirchen\oindex{Partenkirchen@\textbf{Partenkirchen}, \emph{Teil eines besiedelten Ortes (A.BSOX)}|pw} (wohin auch Olga\pwindex{Schnitzler, Olga 17.01.1882 – 13.01.1970@\textsc{Schnitzler, Olga} (17.01.1882 – 13.01.1970), \emph{Schauspieler/Schauspielerin, Sänger/Sängerin}|pw} im
               Anschluss an ein Münchn\oindex{Muenchen@\textbf{München}, \emph{P.PPLA}|pw}er Concert\strikeout{)} gehen wird); aber mich graut vor Wartesälen,
               Bahncoupés, Zollvisitationen, Gepäckaufgeben; und so wird auch daraus kaum was
               werden. Ich bin in diesem Sommer {\pb}nur in Reichenau\oindex{Reichenau an der Rax@\textbf{Reichenau an der Rax}, \emph{A.ADM3}|pw} gewesen, \label{K_L02327-3v}\edtext{einmal zehn Tage}{\lemma{\textnormal{\emph{einmal zehn Tage}}}\Cendnote{\textnormal{vom 7. 8. 1919 bis
                  zum 20. 8. 1919}}}\label{K_L02327-3} (mit all den Meinen) einmal \label{K_L02327-4v}\edtext{drei Tage}{\lemma{\textnormal{\emph{drei Tage}}}\Cendnote{\textnormal{vom 8. 9. 1919 bis zum 12. 9. 1919}}}\label{K_L02327-4}; – das ist für mich ein Ort so erfüllt von Erinnerungen der mannigfachsten
               Art, dſs ich ihnen, in der schweren Sti{\geminationm}ung dieser So{\geminationm}ertage, kaum gewachsen war. Immerhin wurden mir in
               tausend und mehr Metern Höhe, auf Wiesen, an Waldesrand, ein paar gute Stunden.\pend
           
\pstart
           – We{\geminationn} nicht früher mein lieber Hugo so sehe ich Sie wohl
               bei der \label{K_L02327-5v}\edtext{Generalprobe}{\lemma{\textnormal{\emph{Generalprobe}}}\Cendnote{\textnormal{Vgl. A. S.: \emph{Tagebuch}, 8. 10. 1919.
               }}}\label{K_L02327-5} der sonnigen Frau\pwindex{Frau ohne Schatten. Oper in drei Akten@\emph{Die Frau ohne Schatten. Oper in drei Akten}|pwv} (ich
               habe Strauß\pwindex{Strauss, Richard 11.06.1864 – 08.09.1949@\textsc{Strauss, Richard} (11.06.1864 – 08.09.1949), \emph{Theaterleiter/Theaterleiterin, Komponist/Komponistin, Dirigent/Dirigentin}|pw} um Einlaß gebeten, auch für Olga\pwindex{Schnitzler, Olga 17.01.1882 – 13.01.1970@\textsc{Schnitzler, Olga} (17.01.1882 – 13.01.1970), \emph{Schauspieler/Schauspielerin, Sänger/Sängerin}|pw}, hoffentlich gehts) – ich kenne schon
               allerlei daraus vom Clavier her und freu mich ganz besonders. Haben Sie de{\geminationn} nun auch die Märchen-Erzählung\pwindex{Frau ohne Schatten. Erzaehlung@\emph{Die Frau ohne Schatten. Erzählung}|pwv}, von der Sie mir öfters sprachen – die
               denselben Stoff behandelt, fertig gemacht? \pend
           
\pstart
           – Ich schicke diese Zeilen noch nach Aussee\oindex{Bad Aussee@\textbf{Bad Aussee}, \emph{P.PPLA3}|pw}.
               Haben Sie weiterhin gute, reiche Tage! \pend
           
\pstart
           Von Herzen Ihr{\\[\baselineskip]}\spacefill\mbox{Arth}\pend
           \leftskip=0em{}\selectlanguage{ngerman}\endnumbering\briefempfaengerindex{Hofmannsthal, Hugo von@\textsc{Hofmannsthal, Hugo von}!zzzSchnitzler, Arthur@\emph{von Arthur Schnitzler}!1919-10-011@{1. 10. 1919}|)be}\mylabel{L02327h}  \normalsize

\doendnotes{C}
\bigskip
\vfill

\clearpage

\footnotesize

\lohead{\textsc{register}}

% Definiere theindex-Environment komplett neu ohne reledmac
\makeatletter
\renewenvironment{theindex}{%
  \section*{\indexname}%
  \setlength{\parindent}{0pt}%
  \setlength{\parskip}{0pt plus 0.3pt}%
  \let\item\@idxitem
}{%
  \clearpage
}
\makeatother

\IfFileExists{\jobname-pw.ind}{\input{\jobname-pw.ind}}{}

\end{document}

      