%% latex-leseansicht-vorspann.tex
%% Vorspann für die Leseansicht.
%% Lädt die gemeinsame Datei latex-vorspann.tex mit nicht gesetztem Schalter.

\newif\ifkorrekturansicht
\korrekturansichtfalse

\input{../tex-inputs/latex-vorspann}


         
         \renewcommand{\erwaehntePersonen}{Personen: Richard Beer-Hofmann, Hugo von Hofmannsthal, Max Reinhardt, Olga Schnitzler, Elisabeth Steinrück, Richard Strauss, Berta Zuckerkandl}
         \renewcommand{\erwaehnteOrte}{Orte: Altaussee, Bad Aussee, Bösendorferstraße, München, Partenkirchen, Reichenau an der Rax, Rodaun, Seidlgasse, Wien}
         \renewcommand{\erwaehnteWerke}{Werke: Das Märchen. Schauspiel in drei Aufzügen, Der Gang zum Weiher. Dramatische Dichtung, Die Frau ohne Schatten. Erzählung, Die Frau ohne Schatten. Oper in drei Akten, Die Schwestern oder Casanova in Spa. Lustspiel in Versen, Gestern. Dramatische Studie in einem Akt in Versen}
               \section[Arthur Schnitzler an Hugo Hofmannsthal, 1. 10. 1919]{ Arthur Schnitzler an Hugo Hofmannsthal, 1. 10. 1919}\nopagebreak\mylabel{v}\rehead{ }\begin{ledgroupsized}[t]{13cm}\normalsize\beginnumbering \toendnotes[C]{\smallbreak\pagebreak[2]} \Standort{FDH, Hs-30885,149.}
\physDesc{Brief, 2 Blätter, 3 Seiten, 3118 Zeichen
\newline{}Handschrift: Bleistift, lateinische Kurrent}\buchAbdrucke{\weitereDrucke{1) Hugo von Hofmannsthal, Arthur Schnitzler: \emph{Briefwechsel}. Hg. Therese Nickl und Heinrich Schnitzler. Frankfurt am Main: \emph{S. Fischer} 1964, S. 285–286.} \weitereDrucke{2) Arthur Schnitzler: \emph{Briefe 1913–1931}. Hg. Peter Michael Braunwarth, Richard Miklin, Susanne Pertlik und Heinrich Schnitzler. Frankfurt am Main: \emph{S. Fischer} 1984, S. 195–197.} }\toendnotes[C]{\smallbreak}\pstart
           \raggedleft{}{\pb}1. 10. 19{\\}Wien\oindex{Wien@\textbf{Wien}|pw}\pend
           \pstart
           mein lieber Hugo, vor ein paar Wochen schon hat mir die Hofrätin\pwindex{Zuckerkandl, Berta 13.04.1864 – 16.10.1945@\textsc{Zuckerkandl, Berta} (13.04.1864 – 16.10.1945), \emph{Journalistin, Übersetzerin}|pw} gesagt, Sie seien auf einen Brief an
               mich ohne Antwort geblieben; ich will Ihnen nur mittheilen, dſs Ihr Brief vom
                  19. 9. der erste ist, den ich seit vielen Monaten von Ihnen erhielt –
               der letzte berichtete von Ihrem leidenden Zustand und ich schrieb Ihnen darauf, dſs
               ich gern einmal zu Ihnen nach Rodaun\oindex{Rodaun@\textbf{Rodaun}|pw} käme, aber
               darauf hatt ich von Ihnen nichts weiter gehört. Nun freuts mich sehr dſs die neueste
               Kunde so arbeitsfroh und hoffnungsvoll klingt und es wäre wahrhaftig schön, we{\geminationn} man wieder einmal einer jener feiertäglichen
               Vorlesestunden entgegensehen dürfte – die nur im Lauf der Jahre um so viel seltener
               geworden sind als selbst die seltensten Feiertage. Und was für eine Reihe von
               festlich ergreifenden Abenden – von jenem ersten an, an dem Sie mir, an einem warmen
                  \label{K_L02327-1v}\edtext{Juniabend}{\lemma{\textnormal{\emph{Juniabend}}}\Cendnote{\textnormal{siehe A. S.: \emph{Tagebuch}, 7. 10. 1891}}}\label{K_L02327-1h} war es, in der Giselastraße\oindex{XXXX Ortsangabe fehlt|pw}, »Gestern\pwindex{Hofmannsthal, Hugo von 1874-02-01 – 1929-07-15@\textsc{Hofmannsthal, Hugo von} (1874-02-01 – 1929-07-15), \emph{Schriftsteller}!Gestern. Dramatische Studie in einem Akt in Versen15. 10. 1891@\strich\emph{Gestern. Dramatische Studie in einem Akt in Versen} {[}15. 10. 1891{]}|pw}« vorlasen – oder war ich es, der mit dem
                  »\label{K_L02327-2v}\edtext{Märchen\pwindex{Schnitzler, Arthur 15.05.1862 – 21.10.1931@\textsc{Schnitzler, Arthur} (15.05.1862 – 21.10.1931), \emph{Schriftsteller, Mediziner}!Maerchen. Schauspiel in drei Aufzuegen1893-12-01@\strich\emph{Das Märchen. Schauspiel in drei Aufzügen} {[}1893-12-01{]}|pw}}{\lemma{\textnormal{\emph{Märchen}}}\Cendnote{\textnormal{Diese Lesung fand am 25. 6. 1891 in der Seidlgasse\oindex{Seidlgasse@\textbf{Seidlgasse}|pwk} statt. Aber bereits früher lassen
                  sich solche Lesungen im privaten Kreis nachweisen.}}}\label{K_L02327-2h}« anfing, in der Seidlgasse\oindex{Seidlgasse@\textbf{Seidlgasse}|pw}, bei Richard\pwindex{Beer-Hofmann, Richard 1866-07-11 – 1945-09-26@\textsc{Beer-Hofmann, Richard} (1866-07-11 – 1945-09-26), \emph{Schriftsteller}|pw} – ich weiß nicht mehr? Es kam wirklich wenig darauf an, ob das Werk
               als solches mehr oder weniger vollendet war – der Beifall geringer oder größer – im
               Rückblick bleiben es durchaus Stunden der kräftigsten, belebtesten Atmosphäre –
               bessere, reinere: als wenn man dasselbe Werk zum ersten Mal der Oeffentlich{\pb}keit zu praesentiren hatte. Ich bin höchst gespannt was
               Sie aus Altaussee\oindex{Altaussee@\textbf{Altaussee}|pw} mitbringen werden. Mit meiner
               Arbeit (Stück\pwindex{Schnitzler, Arthur 15.05.1862 – 21.10.1931@\textsc{Schnitzler, Arthur} (15.05.1862 – 21.10.1931), \emph{Schriftsteller, Mediziner}!Gang zum Weiher. Dramatische Dichtung1926@\strich\emph{Der Gang zum Weiher. Dramatische Dichtung} {[}1926{]}|pwv}) geht es so
               langsam vorwärts, dſs ich fast von einem Stillstand sprechen kann – obzwar ich die
               Continuität zum mindesten durch beharrliches Anstarren unbeschriebener Papierblätter
               oder Ausstreichen des Geschriebenen festzuhalten versuche. Das letzte, was ich fertig
               gemacht \introOben{}habe\introOben{}, sind die »Schwestern\pwindex{Schnitzler, Arthur 15.05.1862 – 21.10.1931@\textsc{Schnitzler, Arthur} (15.05.1862 – 21.10.1931), \emph{Schriftsteller, Mediziner}!Schwestern oder Casanova in Spa. Lustspiel in Versen01. 10. 1919@\strich\emph{Die Schwestern oder Casanova in Spa. Lustspiel in Versen} {[}01. 10. 1919{]}|pw}«, die bei Reinhardt\pwindex{Reinhardt, Max 09.09.1873 – 30.10.1943@\textsc{Reinhardt, Max} (09.09.1873 – 30.10.1943), \emph{Theaterleiter, Regisseur, Schauspieler}|pw} kommen
               sollen; – mir selbst ist selten was von mir so lieb gewesen. Ich hab allerlei vor,
               manches aus den letzten Jahren ist sogar recht weit gediehen; aber meine Arbeitskraft
               ist – wohl unter dem Einfluss dieses grauenhaften Weltzustandes – so tief herunter
               wie noch nie. Zu einer größern Reise hab ich mich nicht entschließen können, nun lädt
               mich meine Schwägerin\pwindex{Steinrueck, Elisabeth 19.11.1885 – 07.04.1920@\textsc{Steinrück, Elisabeth} (19.11.1885 – 07.04.1920)|pwv}{ }sehr dringend nach Partenkirchen\oindex{Partenkirchen@\textbf{Partenkirchen}|pw} (wohin auch Olga\pwindex{Schnitzler, Olga 17.01.1882 – 13.01.1970@\textsc{Schnitzler, Olga} (17.01.1882 – 13.01.1970), \emph{Schauspielerin, Sängerin}|pw} im
               Anschluss an ein Münchn\oindex{Muenchen@\textbf{München}|pw}er Concert\strikeout{)} gehen wird); aber mich graut vor Wartesälen,
               Bahncoupés, Zollvisitationen, Gepäckaufgeben; und so wird auch daraus kaum was
               werden. Ich bin in diesem Sommer {\pb}nur in Reichenau\oindex{Reichenau an der Rax@\textbf{Reichenau an der Rax}|pw} gewesen, \label{K_L02327-3v}\edtext{einmal zehn Tage}{\lemma{\textnormal{\emph{einmal zehn Tage}}}\Cendnote{\textnormal{vom 7. 8. 1919 bis
                  zum 20. 8. 1919}}}\label{K_L02327-3h} (mit all den Meinen) einmal \label{K_L02327-4v}\edtext{drei Tage}{\lemma{\textnormal{\emph{drei Tage}}}\Cendnote{\textnormal{vom 8. 9. 1919 bis zum 12. 9. 1919}}}\label{K_L02327-4h}; – das ist für mich ein Ort so erfüllt von Erinnerungen der mannigfachsten
               Art, dſs ich ihnen, in der schweren Sti{\geminationm}ung dieser So{\geminationm}ertage, kaum gewachsen war. Immerhin wurden mir in
               tausend und mehr Metern Höhe, auf Wiesen, an Waldesrand, ein paar gute Stunden.\pend
           \pstart
           – We{\geminationn} nicht früher mein lieber Hugo so sehe ich Sie wohl
               bei der \label{K_L02327-5v}\edtext{Generalprobe}{\lemma{\textnormal{\emph{Generalprobe}}}\Cendnote{\textnormal{vgl. A. S.: \emph{Tagebuch}, 8. 10. 1919}}}\label{K_L02327-5h} der sonnigen Frau\pwindex{Strauss, Richard 11.06.1864 – 08.09.1949@\textsc{Strauss, Richard} (11.06.1864 – 08.09.1949), \emph{Theaterleiter, Komponist, Dirigent}!Frau ohne Schatten. Oper in drei Akten10. 10. 1919@\strich\emph{Die Frau ohne Schatten. Oper in drei Akten} {[}10. 10. 1919{]}|pwv}\pwindex{Hofmannsthal, Hugo von 1874-02-01 – 1929-07-15@\textsc{Hofmannsthal, Hugo von} (1874-02-01 – 1929-07-15), \emph{Schriftsteller}!Frau ohne Schatten. Oper in drei Akten10. 10. 1919@\strich\emph{Die Frau ohne Schatten. Oper in drei Akten} {[}10. 10. 1919{]}|pwv} (ich
               habe Strauß\pwindex{Strauss, Richard 11.06.1864 – 08.09.1949@\textsc{Strauss, Richard} (11.06.1864 – 08.09.1949), \emph{Theaterleiter, Komponist, Dirigent}|pw} um Einlaß gebeten, auch für Olga\pwindex{Schnitzler, Olga 17.01.1882 – 13.01.1970@\textsc{Schnitzler, Olga} (17.01.1882 – 13.01.1970), \emph{Schauspielerin, Sängerin}|pw}, hoffentlich gehts) – ich kenne schon
               allerlei daraus vom Clavier her und freu mich ganz besonders. Haben Sie de{\geminationn} nun auch die Märchen-Erzählung\pwindex{Hofmannsthal, Hugo von 1874-02-01 – 1929-07-15@\textsc{Hofmannsthal, Hugo von} (1874-02-01 – 1929-07-15), \emph{Schriftsteller}!Frau ohne Schatten. Erzaehlung1919@\strich\emph{Die Frau ohne Schatten. Erzählung} {[}1919{]}|pwv}, von der Sie mir öfters sprachen – die
               denselben Stoff behandelt, fertig gemacht? \pend
           \pstart
           – Ich schicke diese Zeilen noch nach Aussee\oindex{Bad Aussee@\textbf{Bad Aussee}|pw}.
               Haben Sie weiterhin gute, reiche Tage! \pend
           \pstart
           Von Herzen Ihr{\\[\baselineskip]}\spacefill\mbox{Arth}\pend
           \leftskip=0em{}
         
         \endnumbering\mylabel{h}\end{ledgroupsized}  \newcommand{\dateiname}{L02327}\newcommand{\titel}{Arthur Schnitzler an Hugo Hofmannsthal, 1. 10. 1919}\newcommand{\editorInnen}{Martin Anton Müller und Gerd-Hermann Susen}%% latex-leseansicht-abspann.tex
%% Abspann für die Leseansicht.
%% Der Schalter \ifkorrekturansicht ist bereits durch den Vorspann gesetzt.

%% latex-abspann.tex
%% Gemeinsamer Abspann für Korrekturansicht und Leseansicht.
%% Setzt den Schalter \ifkorrekturansicht voraus (gesetzt in den
%% einbindenden Dateien latex-korrekturansicht-abspann.tex bzw.
%% latex-leseansicht-abspann.tex).
%% ---------------------------------------------------------------

\normalsize

% Das esempio-Environment wird nur in der Leseansicht benötigt
\ifkorrekturansicht\else
\newenvironment{esempio}[3]%
{
    \vspace{1.5ex}
    \rlap{\underline{#1}}
    \par
    \setlength{\parindent}{0cm}
    \nopagebreak
    \leftskip=#2cm
    \rightskip=#3cm
}
{
    \par
}
\fi

\doendnotes{C}
\bigskip
\vfill

\clearpage

\footnotesize

\ifkorrekturansicht
  \lohead{\textsc{register}}
\fi

% theindex-Environment neu definieren ohne reledmac
\makeatletter
\renewenvironment{theindex}{%
  \ifkorrekturansicht
    \section*{\indexname}%
  \else
    \subsubsection*{Index der erwähnten Entitäten}%
  \fi
  \setlength{\parindent}{0pt}%
  \setlength{\parskip}{0pt plus 0.3pt}%
  \let\item\@idxitem
}{%
  \ifkorrekturansicht\clearpage\fi
}
\makeatother

\IfFileExists{\jobname-pw.ind}{\input{\jobname-pw.ind}}{}

% Quellenangabe nur in der Leseansicht
\ifkorrekturansicht\else
% Fallback-Definitionen, falls die .tex-Datei \titel etc. nicht gesetzt hat
\providecommand{\titel}{}
\providecommand{\editorInnen}{}
\providecommand{\dateiname}{\jobname}

\vspace{3cm}

\vfill

\footnotesize
\textsc{Quelle}: \titel. Herausgegeben von {\editorInnen}. In: \emph{Arthur Schnitzler: Briefwechsel mit Autorinnen und Autoren}.
 Digitale Edition, https://schnitzler-briefe.acdh.oeaw.ac.at/{\dateiname}.html (Stand \today)
\fi

\end{document}


      