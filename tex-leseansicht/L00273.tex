%% latex-korrekturansicht-vorspann.tex
%% Vorspann für die Korrekturansicht.
%% Lädt die gemeinsame Datei latex-vorspann.tex mit gesetztem Schalter.

\newif\ifkorrekturansicht
\korrekturansichttrue

\input{../tex-inputs/latex-vorspann}


\section[Ferdinand von Saar an Arthur Schnitzler, 15. 10. 1893]{L00273 Ferdinand von Saar an Arthur Schnitzler, 15. 10. 1893}
\nopagebreak\mylabel{L00273v}
\rehead{ }\normalsize\beginnumbering\briefempfaengerindex{Schnitzler, Arthur@\textsc{Schnitzler, Arthur}!zzzSaar, Ferdinand von@\emph{von Ferdinand von Saar}!1893-10-151@{15. 10. 1893}|(be}
\toendnotes[C]{\smallbreak\pagebreak[2]}\Standort{CUL, Schnitzler, B 88.}
\physDesc{Brief, 1 Blatt, 1 Seite, 385 Zeichen
\newline{}Handschrift: schwarze Tinte, deutsche Kurrent
\newline{}Schnitzler: mit Bleistift nummeriert: »1« }\toendnotes[C]{\smallbreak}
\pstart
           \raggedleft{}{\pb}\textsc{Oberdöbling} Hauptſtraße 98\oindex{Doeblinger Hauptstrasse@\textbf{Döblinger Hauptstraße}, \emph{Straße (K.STR)}|pw}.{\\}15\textsuperscript{t\textcolor{gray}{en}} October 1893.\pend
           
\pstart{}Sehr geehrter Herr und junger College!\pend\vspace{0.5em}
\pstart
           Ich kann Ihnen heute nur mit wenigen Worten danken für die freundliche Überſendung
               der drei Werke\pwindex{Anatol@\emph{Anatol}|pwv}\pwindex{Maerchen. Schauspiel in drei Aufzuegen@\emph{Das Märchen. Schauspiel in drei Aufzügen}|pwv}\pwindex{Alkandi s Lied@\emph{Alkandi’s Lied}|pwv}. Nicht einmal eines davon konnte ich bis jetzt vornehmen, ſo viel und
               ſo vieles liegt noch auf mir. Laſſen Sie alſo noch einige Geduld angedeihen\pend
           
\pstart
           Ihrem, Ihnen{\\[\baselineskip]}in wahrer Hochachtung{\\[\baselineskip]}ergebenen{\\[\baselineskip]}\spacefill\mbox{Ferdinand von Saar.}\pend
           \leftskip=0em{}\selectlanguage{ngerman}\endnumbering\briefempfaengerindex{Schnitzler, Arthur@\textsc{Schnitzler, Arthur}!zzzSaar, Ferdinand von@\emph{von Ferdinand von Saar}!1893-10-151@{15. 10. 1893}|)be}\mylabel{L00273h}  \normalsize

\doendnotes{C}
\bigskip
\vfill

\clearpage

\footnotesize

\lohead{\textsc{register}}

% Definiere theindex-Environment komplett neu ohne reledmac
\makeatletter
\renewenvironment{theindex}{%
  \section*{\indexname}%
  \setlength{\parindent}{0pt}%
  \setlength{\parskip}{0pt plus 0.3pt}%
  \let\item\@idxitem
}{%
  \clearpage
}
\makeatother

\IfFileExists{\jobname-pw.ind}{\input{\jobname-pw.ind}}{}

\end{document}

      