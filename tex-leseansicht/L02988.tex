%% latex-korrekturansicht-vorspann.tex
%% Vorspann für die Korrekturansicht.
%% Lädt die gemeinsame Datei latex-vorspann.tex mit gesetztem Schalter.

\newif\ifkorrekturansicht
\korrekturansichttrue

\input{../tex-inputs/latex-vorspann}


\section[ Arthur Schnitzler an Felix Salten, 7. 11. 1903]{L02988 Arthur Schnitzler an Felix Salten, 7. 11. 1903}
\nopagebreak\mylabel{L02988v}
\rehead{ }\normalsize\beginnumbering\briefempfaengerindex{Salten, Felix@\textsc{Salten, Felix}!zzzSchnitzler, Arthur@\emph{von Arthur Schnitzler}!1903-11-071@{7. 11. 1903}|(be}
\toendnotes[C]{\smallbreak\pagebreak[2]}\Standort{Wienbibliothek im Rathaus, ZPH 1681, 2.1.516.}
\physDesc{Brief, 6 Blätter, 21 Seiten, 6947 Zeichen
\newline{}Handschrift: Bleistift, deutsche Kurrent
\newline{}Ordnung: mit Bleistift von unbekannter Hand Nummerierung der Doppelseiten des
                                 Konvoluts: »40«–»50« }
\buchAbdrucke{\weitereDrucke{Arthur Schnitzler: \emph{Briefe 1875–1912}. Frankfurt am Main: \emph{S. Fischer} 1981, S. 468–471.} }\toendnotes[C]{\smallbreak}
\pstart
           \raggedleft{}{\pb}\textsc{Semmering\oindex{Semmering@\textbf{Semmering}, \emph{A.ADM3}|pw}}{ }7. 11. 903.{\\}6 Uhr Abd\pend
           \vspace{0.5em}
\pstart
           lieber, wir ko{\geminationm}en eben von einem
                  \label{K_L02988-1v}\edtext{Ausflug}{\lemma{\textnormal{\emph{Ausflug}}}\Cendnote{\textnormal{Siehe A. S.: \emph{Tagebuch}, 7. 11. 1903.
               }}}\label{K_L02988-1} zurück und ich finde in der Zeit\pwindex{Zeit@\emph{Die Zeit}|pw} Ihr
                  \label{K_L02988-2v}\edtext{Reigenfeu{[}i{]}lleton\pwindex{Arthur Schnitzler und sein »Reigen«@\emph{Arthur Schnitzler und sein »Reigen«}|pwv}}{\lemma{\textnormal{\emph{Reigenfeuilleton}}}\Cendnote{\textnormal{Felix Salten\pwindex{Salten, Felix 06.09.1869 – 08.10.1945@\textsc{Salten, Felix} (06.09.1869 – 08.10.1945), \emph{Schriftsteller/Schriftstellerin, Journalist/Journalistin, Chefredakteur/Chefredakteurin}|pwk}: \emph{Arthur Schnitzler und sein »Reigen«}\pwindex{Arthur Schnitzler und sein »Reigen«@\emph{Arthur Schnitzler und sein »Reigen«}|pwk}. In: \emph{Die Zeit}\pwindex{Zeit@\emph{Die Zeit}|pwk}, Jg. 2, Nr. 398, 7. 11. 1903, Morgenblatt, S. 1–2.}}}\label{K_L02988-2}.
               Über ſeinen künſtleriſchen Werth iſt weiter nichts zu ſagen; es iſt vorzüglich. Und
                  we{\geminationn} es den Titel trüge »Anatol\pwindex{Anatol@\emph{Anatol}|pwv} u der Reigen\pwindex{Reigen. Zehn Dialoge@\emph{Reigen. Zehn Dialoge}|pw}«, ſo wäre es einfach meiſterhaft zu
                  nenne\textcolor{gray}{n.} Da es aber heißt: Arth.
                  Schn. u ſein Reigen\pwindex{Arthur Schnitzler und sein »Reigen«@\emph{Arthur Schnitzler und sein »Reigen«}|pw}, ſo habe ich \strikeout{drauf}
               einiges zu bemerken, und da Sie es geſchrieben, ſo müſſen Sie {\pb}meinen Bemerkungen verzeihen, \substVorne{}\textsuperscript{daſs }\substDazwischen{}we{\geminationn}\substHinten{} ſie etwa einen Ton des Erſtaunens verrathen ſollten, auf den Sie
               wahrſcheinlich nicht vorbereitet ſind. Aber ich möchte nicht, daſs sich durch \label{K_L02988-3v}\edtext{Unaufrichtigkeit oder Zurückhaltung
               meinerſeits unſre Beziehungen ganz überflüſſigerweiſe verdunkeln}{\lemma{\textnormal{\emph{Unaufrichtigkeit … verdunkeln}}}\Cendnote{\textnormal{Zu jüngeren Schwierigkeiten in der
                  Beziehung vgl. Felix Salten an Arthur Schnitzler, [12. 10. 1903] und Arthur Schnitzler an Felix Salten, 12. 10. [1903].}}}\label{K_L02988-3}{ }\strikeout{oder nur}{ }\strikeout{\textcolor{gray}{×}\-\textcolor{gray}{×}\-\textcolor{gray}{×}\-\textcolor{gray}{×}\-\textcolor{gray}{×}{ }\textcolor{gray}{×}\-\textcolor{gray}{×}\-\textcolor{gray}{×}} ſoll\substVorne{}\textsuperscript{en}\substDazwischen{}ten\substHinten{}, ſondern ziehe es vor, Ihnen gleich, vielleicht allzuſehr in der erſten
               Erregung, aber völlig ehrlich {\pb}zu ſagen, was
               ich gegen Ihr Feu{[}i{]}lleton\pwindex{Arthur Schnitzler und sein »Reigen«@\emph{Arthur Schnitzler und sein »Reigen«}|pwv} auf dem Herzen habe\textcolor{gray}{.}{ }\label{T_L02988-1v}\edtext{Es kam mir vor allem überraſchender als
               ich ſagen ka{\geminationn}, meine bisherige Production von Ihnen als
               Goldſchmiedearbeit u Kleinkunſt abgethan zu leſen.}{\lemma{\textnormal{\emph{Es … leſen.}}}\Cendnote{\textnormal{mit einem doppelten seitlichen Strich entlang des
                  Mittelfalzes markiert}}}\label{T_L02988-1} Aus der Art u Weiſe wie Sie ſich bisher im
               perſönlichen Verkehr und in kritiſch-öffentlicher Erörterung vernehmen ließen, hab
               ich nicht vermuthet, daſs Sie Liebelei\pwindex{Liebelei. Schauspiel in drei Akten@\emph{Liebelei. Schauspiel in drei Akten}|pw} oder Kakadu\pwindex{gruene Kakadu. Groteske in einem Akt@\emph{Der grüne Kakadu. Groteske in einem Akt}|pw} oder {\pb}Lebendige Stunden\pwindex{Lebendige Stunden. Vier Einakter@\emph{Lebendige Stunden. Vier Einakter}|pw} oder Bertha Garlan\pwindex{Frau Bertha Garlan. Roman@\emph{Frau Bertha Garlan. Roman}|pw} zur Kleinkunſt rechnen. Vielleicht haben Sie
               Recht (ich glaube es nicht) – und ich muſs mich nur fragen, wie ich Sie bis zum
               heutigen Tage in allen Ihren Äußerungen über meine Sachen ſo ſehr habe misverſtehen
                  kö{\geminationn}en. \textcolor{gray}{U.} Wie oft haben wir
               gemeinſchaftlich unſern Aerger, unſern Zorn über die Kritiken ausgeſprochen, {\pb}die, aus den verſchiedenſten Gründen, in
               jeder weiblichen Figur, die ohne den Trauring am Finger auftr\substVorne{}\textsuperscript{at}\substDazwischen{}itt\substHinten{}, mit ſataniſchem Behagen, das »süße Mädel« wiederzuerke{\geminationn}en vorgaben {\dotsfour} für die Chriſtine\pwindex{Liebelei. Schauspiel in drei Akten@\emph{Liebelei. Schauspiel in drei Akten}|pwv} und Mizi\pwindex{Liebelei. Schauspiel in drei Akten@\emph{Liebelei. Schauspiel in drei Akten}|pwv} und Franziska\pwindex{Vermaechtnis. Schauspiel in drei Akten@\emph{Das Vermächtnis. Schauspiel in drei Akten}|pwv} und Toni\pwindex{Vermaechtnis. Schauspiel in drei Akten@\emph{Das Vermächtnis. Schauspiel in drei Akten}|pwv} und Margarethe\pwindex{Literatur@\emph{Literatur}|pwv} und Léocadie\pwindex{gruene Kakadu. Groteske in einem Akt@\emph{Der grüne Kakadu. Groteske in einem Akt}|pwv} und womöglich auch \introOben{}die verwittwete\introOben{}{ }Bertha Garlan\pwindex{Frau Bertha Garlan. Roman@\emph{Frau Bertha Garlan. Roman}|pwv} und die
               ehebrecheriſche Pauline\pwindex{Frau mit dem Dolche@\emph{Die Frau mit dem Dolche}|pwv} nichts
               waren als die gleiche Geſtalt unter verſchiedenen Na{\pb}men – und nun muſs ich es bei Ihnen \strikeout{\textcolor{gray}{wied}} leſen, daſs \substVorne{}\textsuperscript{die niedliche, la}\substDazwischen{}es\substHinten{} immer die gleiche »niedliche«, »langwierige« »Gefährtin« war, die mich
               begleitet hat und daſs \strikeout{es} mir erſt \substVorne{}\textsuperscript{mit}\substDazwischen{}in\substHinten{} der \label{T_L02988-2v}\edtext{\textsc{Beatrice\pwindex{Schleier der Beatrice. Schauspiel in fuenf Akten@\emph{Der Schleier der Beatrice. Schauspiel in fünf Akten}|pwv}} eine einigermaßen \uline{neue} Verkleidung der
               altbekannten Figur gelungen}{\lemma{\textnormal{\emph{Beatrice … gelungen}}}\Cendnote{\textnormal{mit einem
                  doppelten seitlichen Strich entlang des Mittelfalzes markiert}}}\label{T_L02988-2} iſt{\dotstwo} Wie oft haben wir darüber geklagt, wie Leichtfertigkeit
               und unguter Wille jederzeit daran ſind, den producirend\textcolor{gray}{en}{ }{\pb}Künſtler in ein Kaſtl zu ſperren, wie oft
               waren wir ergri{\geminationm}t\strikeout{,} über
               die Leute – verzeihen Sie dſs ich mich ſelbſt citire – \substVorne{}\textsuperscript{die}\substDazwischen{}fü\substHinten{}r die der Ma{\geminationn}, der ein oder zwei Mal \substVorne{}\textsuperscript{in}\substDazwischen{}ſe\substHinten{}ine grüne Cravate getragen – immer u \label{T_L02988-3v}\edtext{immer der Herr mit der grüne Cravate bleibt}{\lemma{\textnormal{\emph{immer … bleibt}}}\Cendnote{\textnormal{mit einem doppelten seitlichen Strich
                  entlang des rechten Randes markiert}}}\label{T_L02988-3} – und möge er ſich ein oder
                  \textcolor{gray}{zwei} Mal mit anderfarbigen Crataven gezeigt haben – und nun
               ſind Sie es, den {\pb}ich rufen höre: »Er aber
               darf nicht weiterko{\geminationm}en {\dotstwo} So
               nicht –« »Nun muſs ein andrer Rauſch den Künſtler umfangen –« als hätte mich wirklich
               mein Lebtag nichts andres intereſſirt, als – wie \label{K_L02988-55v}\edtext{Herzl\pwindex{Herzl, Theodor 1860-05-02 – 1904-07-03@\textsc{Herzl, Theodor} (1860-05-02 – 1904-07-03), \emph{Schriftsteller/Schriftstellerin, Journalist/Journalistin}|pw} einmal ſchrieb
               »ob die Poldi den Franzl kriegt, oder ob der Rudi der Tini untreu wird}{\lemma{\textnormal{\emph{Herzl … wird}}}\Cendnote{\textnormal{»Daß es noch größere Fragen
                        gebe, als ob die Mitzi mit dem Rudi vom Ferdl plötzlich verlassen worden
                        sei, scheint er in seinen Werken nicht zu wissen.\pwindex{Feuilleton. Carl-Theater. (»Freiwild«, Schauspiel von Arthur Schnitzler.)@\emph{Feuilleton. Carl-Theater. (»Freiwild«, Schauspiel von Arthur Schnitzler.)}|pwv}« H.\pwindex{Herzl, Theodor 1860-05-02 – 1904-07-03@\textsc{Herzl, Theodor} (1860-05-02 – 1904-07-03), \emph{Schriftsteller/Schriftstellerin, Journalist/Journalistin}|pwk} [ = Theodor Herzl\pwindex{Herzl, Theodor 1860-05-02 – 1904-07-03@\textsc{Herzl, Theodor} (1860-05-02 – 1904-07-03), \emph{Schriftsteller/Schriftstellerin, Journalist/Journalistin}|pwk}]: \emph{Feuilleton.
                        Carl-Theater. (»Freiwild«, Schauspiel von Arthur Schnitzler)}\pwindex{Feuilleton. Carl-Theater. (»Freiwild«, Schauspiel von Arthur Schnitzler.)@\emph{Feuilleton. Carl-Theater. (»Freiwild«, Schauspiel von Arthur Schnitzler.)}|pwk}. In: \emph{Neue Freie Presse}\pwindex{Neue Freie Presse@\emph{Neue Freie Presse}|pwk}, Nr. 12.024, 13. 2. 1898, S. 1–2. Schnitzler hatte sich auch über dieses Feuilleton\pwindex{Feuilleton. Carl-Theater. (»Freiwild«, Schauspiel von Arthur Schnitzler.)@\emph{Feuilleton. Carl-Theater. (»Freiwild«, Schauspiel von Arthur Schnitzler.)}|pwkv} geärgert, vgl. A. S.: \emph{Tagebuch}, 13. 2. 1898 und Paul Goldmann an Arthur Schnitzler, 7. 3. [1898].}}}\label{K_L02988-55}«{\dots}
               als hätt ich immer nur die gleichen Menſchen geſtaltet\textcolor{gray}{,}{ }{\pb}ewig die gleichen Situationen dargeſtellt –
               ewig u immer nur die grüne Cravate getragen! Und wieder frag ich mich: Ja hat er am
               Ende Recht? {\dotstwo} Iſt es nicht ſehr wahrſcheinlich, daſs er
               Recht hat, gerad er, der dich ſeit deinen erſten Anfängen \substVorne{}\textsuperscript{ſchä}\substDazwischen{}ke\substHinten{}nnt und ſchätzt – und befindeſt du dich am Ende wirklich in der lächerlichen
               Selbſttäuſchung mancher {\pb}Künſtler, die ihr
               kunſtgewerbliches Bemühn für echtes Kunſtbeſtreben, und ihren Winkel für eine Welt
               halten? Und mußt Du wirklich jedesmal we{\geminationn} du ein
               weibliches Weſen neu zu geſtalten glaubteſt \label{T_L02988-4v}\edtext{auf den Hohnruf gefaſſt ſein}{\lemma{\textnormal{\emph{auf … ſein}}}\Cendnote{\textnormal{mit einem doppelten seitlichen Strich entlang des linken
                  Randes markiert}}}\label{T_L02988-4}{ }{\dots} das ſüße Mädel {\dotstwo} Und jedesmal
                  we{\geminationn} du \substVorne{}\textsuperscript{die}\substDazwischen{}eine neue\substHinten{} Beziehung zwiſchen zwei Menſchen verſchiedenen Geſchlechtes dar\substVorne{}\textsuperscript{ſtellen}\substDazwischen{}zustellen\substHinten{}{ }{\pb}denkſt – vor dem Echo »Liebelei\pwindex{Liebelei. Schauspiel in drei Akten@\emph{Liebelei. Schauspiel in drei Akten}|pw}« zittern – und immer immer wieder, we{\geminationn} du in eingebildeter Freiheit mit den Gebilden deiner
               Phantaſie zu ſchalten meinſt – immer wieder erfahren, daſs du in dem alten Kaſtl
               ſteckſt, daſs du nie verlaſſen haſt? – Ich will es Ihnen nicht verhehlen {\dots} niemals noch hatt ich ſoſehr das
                  Gefühl\textcolor{gray}{:}{\pb}Es iſt alles vergeblich – du biſt etikettirt
               auf Lebenszeit, als während der Lecture Ihres Feuilletons\pwindex{Arthur Schnitzler und sein »Reigen«@\emph{Arthur Schnitzler und sein »Reigen«}|pwv} – ſo viel Lob und Anerke{\geminationn}ung Sie im übrigen über meine Kleinkunſt aus\substVorne{}\textsuperscript{\textcolor{gray}{ſch}}\substDazwischen{}gi\substHinten{}eßen – und ſoſehr ich überzeugt bin, daſs Sie von allen Seiten den Vorwurf
               hören werden, mich in einen unverdienten Himmel gehoben zu haben. Der Reigen\pwindex{Reigen. Zehn Dialoge@\emph{Reigen. Zehn Dialoge}|pw} iſt 1896/\textcolor{gray}{9}7 geſchrieben. Es iſt {\pb}Ihnen bekannt, daſs ich ſeither einiges
               andres gedichtet habe, gelungnes u minder gelungens. Die \textsc{Beatrice\pwindex{Schleier der Beatrice. Schauspiel in fuenf Akten@\emph{Der Schleier der Beatrice. Schauspiel in fünf Akten}|pw}} ziehen Sie allerdings noch in den Kreis Ihrer Betrachtungen – als höchſte
               Etappe auf meine\textcolor{gray}{m} Süßen Mädl Weg. Auch der Lieutenant Guſtl\pwindex{Lieutenant Gustl. Novelle@\emph{Lieutenant Gustl. Novelle}|pw} wird flüchtig erwähnt. Meiner Anſicht nach
               wäre beides überflüſſig geweſen, we{\geminationn}{ }{\pb}Ihr Feu{[}i{]}lleton\pwindex{Arthur Schnitzler und sein »Reigen«@\emph{Arthur Schnitzler und sein »Reigen«}|pwv} den Titel
                  trüge\textcolor{gray}{:}{ }Anatol\pwindex{Anatol@\emph{Anatol}|pw} und der Reigen\pwindex{Reigen. Zehn Dialoge@\emph{Reigen. Zehn Dialoge}|pw}. Aber es heißt Arthur Schnitzler u
                  ſein Reigen\pwindex{Arthur Schnitzler und sein »Reigen«@\emph{Arthur Schnitzler und sein »Reigen«}|pw}. Und \uline{Sie} haben es geſchrieben.
               Nicht einmal; hundertmal haben wir über meine Production,
                  einhunde\textcolor{gray}{rt} Mal über meine Intention geſprochen{\dotstwo} Nicht einmal unter dieſen hundert iſt mir eine Ahnung
               aufgedämmert, daſs Sie auch heute noch den Reigen\pwindex{Reigen. Zehn Dialoge@\emph{Reigen. Zehn Dialoge}|pw}{ }{\pb}als das Endglied meines bisherigen Wirkens
               auffaſſen konnten, daſs Sie glaubten ich ſt\substVorne{}\textsuperscript{eh}\substDazwischen{}ünd\substHinten{}e heute noch dort, wo ich \substVorne{}\textsuperscript{am}\substDazwischen{}bei\substHinten{} Abſchluſs des Reigens\pwindex{Reigen. Zehn Dialoge@\emph{Reigen. Zehn Dialoge}|pw} ſtand – aber \introOben{}daſs ich\introOben{} ſelbſt innerhalb der Epoche, die vom Anatol\pwindex{Anatol@\emph{Anatol}|pw} bis zum Reigen\pwindex{Reigen. Zehn Dialoge@\emph{Reigen. Zehn Dialoge}|pw}
               geht, von Ihnen als Goldſchmiedarbeiter u Kleinkünſtler angeſehen w\substVorne{}\textsuperscript{erden }\substDazwischen{}ürde\substHinten{} – hab ich bis {\pb}zum heutigen Tag
               nicht geahnt, und, darauf ko{\geminationm}t es an, \label{T_L02988-5v}\edtext{keines Ihrer Worte konnte mich \introOben{}bis heute\introOben{} vermuthen laſſen}{\lemma{\textnormal{\emph{keines … laſſen}}}\Cendnote{\textnormal{mit einem doppelten seitlichen Strich entlang des linken
                  Randes markiert}}}\label{T_L02988-5}, daſs Sie mich ſo und nicht anders werthen. Gegenüber dem
               Befremden, daſs ich in dieſer Hinſicht empfinde, ko{\geminationm}t
               heute, ſeien Sie mir nicht böſe, die Freude noch nicht {\pb}auf, daſs Sie vieles von mir mit ſo hohen
               Worten preiſen und daſs Sie noch beſſers von mir zu erwarten ſcheinen. Aber gerade
               unſer Verhältnis \introOben{}über\introOben{} das ſo oft \strikeout{\textcolor{gray}{unter}} Wolken von Misverſtändniſſen und Verſti{\geminationm}ungen
               hinziehen, verlangt nach Gewitter\textcolor{gray}{n} und reinem Himmel. Es iſt
               möglich, daſs Sie mich in dieſem Augenblick für {\pb}anmaßend halten und mich zu der traurigen
               Sorte rechnen, »die aber wirklich auch den leiſesten Tadel nicht vertragen«. So iſt
               es nicht lieber Freund. Ich weiſs, beſſer als irgend ein andrer, was mir und meinen
               Arbeiten vorzuwerfen iſt. Auch meine Grenzen ke{\geminationn} ich.
               Weiſs auch, daſs mein Beſtreben, ſie aus{\pb}zudehnen, nicht immer von Erfolg begleitet war. Aber darüber glaubt ich bis heute
               mit Ihnen einig zu ſein – daſs die mir Unrecht thaten, die auch in dem Dichter der
                  Liebelei\pwindex{Liebelei. Schauspiel in drei Akten@\emph{Liebelei. Schauspiel in drei Akten}|pw} und des Kakadu\pwindex{gruene Kakadu. Groteske in einem Akt@\emph{Der grüne Kakadu. Groteske in einem Akt}|pw} nur den »Kleinkünſtler« erkennen
                  wollte{[}n{]} – und die – für die ich im Kakadu\pwindex{gruene Kakadu. Groteske in einem Akt@\emph{Der grüne Kakadu. Groteske in einem Akt}|pw}{ }{\dotstwo} in der \textsc{Beatrice\pwindex{Schleier der Beatrice. Schauspiel in fuenf Akten@\emph{Der Schleier der Beatrice. Schauspiel in fünf Akten}|pw}}{ }{\dotstwo} in der \textsc{Ber{\pb}tha
                     Garlan\pwindex{Frau Bertha Garlan. Roman@\emph{Frau Bertha Garlan. Roman}|pw}} – von dem gleichen Rauſch umfangen war {\dotstwo} als im Anatol\pwindex{Anatol@\emph{Anatol}|pw}{ }{\dots} – Und daſs gerade dieſe Töne, die mich an anderm Ort und
               von andern Muſikern ſo oft verletzt haben – ſo deutlich unter der ſonſt ſo ſchönen
               Melodie Ihres Feu{[}i{]}lletons\pwindex{Arthur Schnitzler und sein »Reigen«@\emph{Arthur Schnitzler und sein »Reigen«}|pwv} von heute
               mitklingen, dieſem Feu{[}i{]}lleton\pwindex{Arthur Schnitzler und sein »Reigen«@\emph{Arthur Schnitzler und sein »Reigen«}|pwv}, mit dem Sie mich gewiſs durchaus {\pb}zu erfreuen glaubten – da\substVorne{}\textsuperscript{ſ}\substDazwischen{}s\substHinten{} hat mir, – Sie werden es vielleicht verſtehen, eine bittre Stund verurſacht,
               und ich h\substVorne{}\textsuperscript{alte}\substDazwischen{}ielt\substHinten{} es für angemeſſen, Ihnen das nicht zu verſchweige\textcolor{gray}{n.}\pend
           
\pstart
           Ihr {\\[\baselineskip]}\spacefill\mbox{A. S.}\pend
           \leftskip=0em{}\selectlanguage{ngerman}\endnumbering\briefempfaengerindex{Salten, Felix@\textsc{Salten, Felix}!zzzSchnitzler, Arthur@\emph{von Arthur Schnitzler}!1903-11-071@{7. 11. 1903}|)be}\mylabel{L02988h}  \normalsize

\doendnotes{C}
\bigskip
\vfill

\clearpage

\footnotesize

\lohead{\textsc{register}}

% Definiere theindex-Environment komplett neu ohne reledmac
\makeatletter
\renewenvironment{theindex}{%
  \section*{\indexname}%
  \setlength{\parindent}{0pt}%
  \setlength{\parskip}{0pt plus 0.3pt}%
  \let\item\@idxitem
}{%
  \clearpage
}
\makeatother

\IfFileExists{\jobname-pw.ind}{\input{\jobname-pw.ind}}{}

\end{document}

      