%% latex-leseansicht-vorspann.tex
%% Vorspann für die Leseansicht.
%% Lädt die gemeinsame Datei latex-vorspann.tex mit nicht gesetztem Schalter.

\newif\ifkorrekturansicht
\korrekturansichtfalse

\input{../tex-inputs/latex-vorspann}


\section[Paul Goldmann an Arthur Schnitzler, 19. 5. [1895]]{L02735 Paul Goldmann an Arthur Schnitzler, 19. 5. [1895]}
\nopagebreak\mylabel{L02735v}
\rehead{ }\normalsize\beginnumbering\briefempfaengerindex{Schnitzler, Arthur@\textsc{Schnitzler, Arthur}!zzzGoldmann, Paul@\emph{von Paul Goldmann}!1895-05-191@{19. 5. [1895]}|(be}
\toendnotes[C]{\smallbreak\pagebreak[2]}
\correspDesc{Versand  durch Paul Goldmann am 19. 5. [1895] in Paris
\newline{}Erhalt  durch Arthur Schnitzler im Zeitraum [20. 5. 1895
                  – 24. 5. 1895?] in Wien}\toendnotes[C]{\smallbreak}
\Standort{DLA, A:Schnitzler, HS.NZ85.1.3165.}
\physDesc{Brief, 3 Blätter, 12 Seiten, 4017 Zeichen
\newline{}Handschrift: schwarze Tinte, deutsche Kurrent
\newline{}Schnitzler: 1) mit Bleistift das Jahr »95« vermerkt  2) mit rotem Buntstift eine Unterstreichung}\toendnotes[C]{\smallbreak}
\pstart
           {\pb}\textcolor{gray}{\textbf{\textbf{Frankfurter Zeitung\orgindex{Frankfurter Zeitung@Frankfurter Zeitung|pw}}}}\pend
           
\pstart
           \textcolor{gray}{\textbf{(\begin{otherlanguage}{french}Gazette de Francfort\end{otherlanguage}\orgindex{Frankfurter Zeitung@Frankfurter Zeitung|pw}).}}\pend
           
\pstart
           \textcolor{gray}{\textbf{\textbf{\begin{otherlanguage}{french}Fondateur M. L.
                              Sonnemann\pwindex{Sonnemann, Leopold 29.\,10.\,1831 Höchberg – 30.\,10.\,1909 Frankfurt am Main@\textsc{Sonnemann, Leopold} (29.\,10.\,1831 Höchberg – 30.\,10.\,1909 Frankfurt am Main), \emph{Journalist, Herausgeber}|pw}\end{otherlanguage}.}}}\pend
           
\pstart
           \begin{otherlanguage}{french}\textcolor{gray}{\textbf{Journal politique, financier,}}\end{otherlanguage}\hfill \textsc{Paris\oindex{Paris@\textbf{Paris}, \emph{Hauptstadt}|pw}}, 19. Mai.\pend
           
\pstart
           \begin{otherlanguage}{french}\textcolor{gray}{\textbf{commercial et littéraire.}}\end{otherlanguage}\pend
           
\pstart
           \begin{otherlanguage}{french}\textcolor{gray}{\textbf{\textbf{Paraissant trois fois par jour.}}}\end{otherlanguage}\pend
           
\pstart
           \begin{otherlanguage}{french}\textcolor{gray}{\textbf{\textbf{Bureau à Paris\oindex{Paris@\textbf{Paris}, \emph{Hauptstadt}|pw}}}}\end{otherlanguage}\pend
           
\pstart
           \begin{otherlanguage}{french}\textcolor{gray}{\textbf{\textbf{24. Rue Feydeau\oindex{rue Feydeau@\textbf{rue Feydeau}, \emph{Straße}|pw}.}}}\end{otherlanguage}\pend
           
\pstart\center{}Mein lieber Freund,\pend\vspace{0.5em}
\pstart
           Gewiß, gewiß –{ }ſeit ich von Frankfurt\oindex{Frankfurt am Main@\textbf{Frankfurt am Main}, \emph{Hauptstadt}|pw} zurück bin,
               liegt es mir{ }ſchwer auf der Seele. Täglich will ich Dir{ }ſchreiben. Aber ich habe
               unmenſchlich zu thun. \strikeout{Lie} Lieſt Du die »Frankfurter Zeitung\pwindex{Frankfurter Zeitung@\emph{Frankfurter Zeitung}|pw}« noch? Jeden Tag kannſt Du es{ }ſehen: \label{K_L02735-1v}\edtext{\textsc{Salon}\pwindex{Goldmann, Paul 31.\,1.\,1865 Breslau – 25.\,9.\,1935 Wien@\textsc{Goldmann, Paul} (31.\,1.\,1865 Breslau – 25.\,9.\,1935 Wien), \emph{Schriftsteller, Journalist}!Pariser Malerei. (Der Salon der Champs Elysées.) [I].@\strich\emph{Pariser Malerei. (Der Salon der Champs Elysées.) [I].}|pwv}}{\lemma{\textnormal{\emph{Salon}}}\Cendnote{\textnormal{Paul Goldmann\pwindex{Goldmann, Paul 31.\,1.\,1865 Breslau – 25.\,9.\,1935 Wien@\textsc{Goldmann, Paul} (31.\,1.\,1865 Breslau – 25.\,9.\,1935 Wien), \emph{Schriftsteller, Journalist}|pwk}: \emph{Pariser Malerei. (Der Salon der Champs Elysées)}\pwindex{Goldmann, Paul 31.\,1.\,1865 Breslau – 25.\,9.\,1935 Wien@\textsc{Goldmann, Paul} (31.\,1.\,1865 Breslau – 25.\,9.\,1935 Wien), \emph{Schriftsteller, Journalist}!Pariser Malerei. (Der Salon der Champs Elysées.) [I].@\strich\emph{Pariser Malerei. (Der Salon der Champs Elysées.) [I].}|pwk}. In:
                        \emph{Frankfurter Zeitung}\pwindex{Frankfurter Zeitung@\emph{Frankfurter Zeitung}|pwk}, Jg. 39, Nr. 135,
                        16. 5. 1895, Erstes Morgenblatt, S. 1–2; Nr. 136,
                        17. 5. 1895, Erstes Morgenblatt, S. 1–2. Bereits am
                  Monatsanfang hatte er zur Ausstellung geschrieben: G.\pwindex{Goldmann, Paul 31.\,1.\,1865 Breslau – 25.\,9.\,1935 Wien@\textsc{Goldmann, Paul} (31.\,1.\,1865 Breslau – 25.\,9.\,1935 Wien), \emph{Schriftsteller, Journalist}|pwk} [ = Paul Goldmann\pwindex{Goldmann, Paul 31.\,1.\,1865 Breslau – 25.\,9.\,1935 Wien@\textsc{Goldmann, Paul} (31.\,1.\,1865 Breslau – 25.\,9.\,1935 Wien), \emph{Schriftsteller, Journalist}|pwk}]: \emph{Firnißtag im Salon de
                        Champs Elysées}\pwindex{Goldmann, Paul 31.\,1.\,1865 Breslau – 25.\,9.\,1935 Wien@\textsc{Goldmann, Paul} (31.\,1.\,1865 Breslau – 25.\,9.\,1935 Wien), \emph{Schriftsteller, Journalist}!Firnißtag im Salon de Champs Elysées@\strich\emph{Firnißtag im Salon de Champs Elysées}|pwk}. In: \emph{Frankfurter
                        Zeitung}\pwindex{Frankfurter Zeitung@\emph{Frankfurter Zeitung}|pwk}, Jg. 39, Nr. 121, 2. 5. 1895, Zweites Morgenblatt,
                     S. 1.}}}\label{K_L02735-1}, \label{K_L02735-2v}\edtext{Kammer\pwindex{Goldmann, Paul 31.\,1.\,1865 Breslau – 25.\,9.\,1935 Wien@\textsc{Goldmann, Paul} (31.\,1.\,1865 Breslau – 25.\,9.\,1935 Wien), \emph{Schriftsteller, Journalist}!Kammer@\strich\emph{Die Kammer}|pw}}{\lemma{\textnormal{\emph{Kammer}}}\Cendnote{\textnormal{G.\pwindex{Goldmann, Paul 31.\,1.\,1865 Breslau – 25.\,9.\,1935 Wien@\textsc{Goldmann, Paul} (31.\,1.\,1865 Breslau – 25.\,9.\,1935 Wien), \emph{Schriftsteller, Journalist}|pwk} [ = Paul Goldmann\pwindex{Goldmann, Paul 31.\,1.\,1865 Breslau – 25.\,9.\,1935 Wien@\textsc{Goldmann, Paul} (31.\,1.\,1865 Breslau – 25.\,9.\,1935 Wien), \emph{Schriftsteller, Journalist}|pwk}]: \emph{Die Kammer}\pwindex{Goldmann, Paul 31.\,1.\,1865 Breslau – 25.\,9.\,1935 Wien@\textsc{Goldmann, Paul} (31.\,1.\,1865 Breslau – 25.\,9.\,1935 Wien), \emph{Schriftsteller, Journalist}!Kammer@\strich\emph{Die Kammer}|pwk}. In:
                        \emph{Frankfurter Zeitung}\pwindex{Frankfurter Zeitung@\emph{Frankfurter Zeitung}|pwk}, Jg. 39, Nr. 135,
                        16. 5. 1895, Drittes Morgenblatt, S. 1.}}}\label{K_L02735-2}, \label{K_L02735-3v}\edtext{Tannhäuſer\pwindex{\textcolor{red}{\textsuperscript{XXXX indx1}}!Tannhäuser und der Sängerkrieg auf Wartburg@\strich\emph{Tannhäuser und der Sängerkrieg auf Wartburg}|pw}\pwindex{\textcolor{red}{\textsuperscript{XXXX indx1}}!Tannhäuser und der Sängerkrieg auf Wartburg@\strich\emph{Tannhäuser und der Sängerkrieg auf Wartburg}|pw}}{\lemma{\textnormal{\emph{Tannhäuser}}}\Cendnote{\textnormal{G.\pwindex{Goldmann, Paul 31.\,1.\,1865 Breslau – 25.\,9.\,1935 Wien@\textsc{Goldmann, Paul} (31.\,1.\,1865 Breslau – 25.\,9.\,1935 Wien), \emph{Schriftsteller, Journalist}|pwk} [ = Paul Goldmann\pwindex{Goldmann, Paul 31.\,1.\,1865 Breslau – 25.\,9.\,1935 Wien@\textsc{Goldmann, Paul} (31.\,1.\,1865 Breslau – 25.\,9.\,1935 Wien), \emph{Schriftsteller, Journalist}|pwk}]: \emph{»Tannhäuser« in
                        Paris}\pwindex{Goldmann, Paul 31.\,1.\,1865 Breslau – 25.\,9.\,1935 Wien@\textsc{Goldmann, Paul} (31.\,1.\,1865 Breslau – 25.\,9.\,1935 Wien), \emph{Schriftsteller, Journalist}!Tannhäuser« in Paris@\strich\emph{»Tannhäuser« in Paris}|pwk}. In: \emph{Frankfurter Zeitung}\pwindex{Frankfurter Zeitung@\emph{Frankfurter Zeitung}|pwk},
                     Jg. 39, Nr. 131, 12. 5. 1895, Erstes Morgenblatt,
                  S. 1–2.}}}\label{K_L02735-3}, \label{K_L02735-4v}\edtext{Japan\oindex{Japan@\textbf{Japan}|pw}}{\lemma{\textnormal{\emph{Japan}}}\Cendnote{\textnormal{Worauf sich Goldmann\pwindex{Goldmann, Paul 31.\,1.\,1865 Breslau – 25.\,9.\,1935 Wien@\textsc{Goldmann, Paul} (31.\,1.\,1865 Breslau – 25.\,9.\,1935 Wien), \emph{Schriftsteller, Journalist}|pwk} hier bezog, ist unklar. Mögliche Erklärungen: Es
                  handelt sich um ein Feuilleton, das länger zurücklag, beispielsweise: A. B.\pwindex{A. B. @\textsc{A. B.}|pwk}: \emph{Eine japanische Kaiserstadt}\pwindex{A. B. @\textsc{A. B.}!Eine japanische Kaiserstadt@\strich\emph{Eine japanische Kaiserstadt}|pwk}. In: \emph{Frankfurter Zeitung}\pwindex{Frankfurter Zeitung@\emph{Frankfurter Zeitung}|pwk}, Jg. 39, Nr. 111, 22. 4. 1895,
                     Morgenblatt, S. 1–2. (Dagegen spricht das Namenskürzel, für das es bei
                     Goldmann\pwindex{Goldmann, Paul 31.\,1.\,1865 Breslau – 25.\,9.\,1935 Wien@\textsc{Goldmann, Paul} (31.\,1.\,1865 Breslau – 25.\,9.\,1935 Wien), \emph{Schriftsteller, Journalist}|pwk} keinen Beleg gibt.) Oder es
                  könnte sich um die kleine, nicht namentlich gekennzeichnete Meldung\pwindex{Japan@\emph{Japan}|pwkv} aus Japan\oindex{Japan@\textbf{Japan}|pwk} handeln, die am 18. 5. 1895 erschienen ist und
                  die möglicherweise ohne Quellenangabe aus einer französischen Zeitung entnommen
                  wurde (Nr. 137, Erstes Morgenblatt, S. 1). Ferner wäre denkbar, dass
                  ein Text nur in einem Teil der Ausgabe\pwindex{Frankfurter Zeitung@\emph{Frankfurter Zeitung}|pwkv} enthalten war.}}}\label{K_L02735-4}{ }\textsc{etc. etc.} Und dann{ }ſchreibe ich Dir nicht, weil ich endlich
               das Bedürfniß {\pb}fühle, Dir den \uline{großen} Brief zu{ }ſchreiben und Dir gar{ }ſoviel zu{ }ſagen haben:
               Innerliches, nichts äußerlich Neues. Nun muß ich aber doch \strikeout{\textcolor{gray}{m}it} noch einmal den kurzen Brief abſenden. Heut Sonntag{ }Nachmittag wollte ich Dir ausführlich{ }ſchreiben. Ich blieb eigens
               deshalb zu Hauſe. Da kam wieder dieſe verfluchte Tagesarbeit dazwiſchen. Nun iſt es
                  ſieben Uhr, und es bleibt mir nur Zeit zu einem {\pb}raſchen Gruß.\pend
           
\pstart
           Gruß und Dank! Für{ }ſoviel Treues und Liebes habe ich Dir zu danken. Eure\pwindex{Andreas-Salomé, Lou 12.\,2.\,1861 Sankt Petersburg – 5.\,2.\,1937 Göttingen@\textsc{Andreas-Salomé, Lou} (12.\,2.\,1861 Sankt Petersburg – 5.\,2.\,1937 Göttingen), \emph{Schriftstellerin}|pwv}\pwindex{Beer-Hofmann, Richard 11.\,7.\,1866 Wien – 26.\,9.\,1945 New York City@\textsc{Beer-Hofmann, Richard} (11.\,7.\,1866 Wien – 26.\,9.\,1945 New York City), \emph{Schriftsteller}|pwv} Karte vom \label{K_L02735-5v}\edtext{\textsc{Kahlenberge\oindex{Wien@\textbf{Wien}!XIX., Döbling@\textbf{XIX., Döbling}!Kahlenberg@\textbf{Kahlenberg}, \emph{Berg}|pw}}}{\lemma{\textnormal{\emph{Kahlenberge}}}\Cendnote{\textnormal{Am 8. 5. 1895 waren Richard Beer-Hofmann\pwindex{Beer-Hofmann, Richard 11.\,7.\,1866 Wien – 26.\,9.\,1945 New York City@\textsc{Beer-Hofmann, Richard} (11.\,7.\,1866 Wien – 26.\,9.\,1945 New York City), \emph{Schriftsteller}|pwk}, Lou Andreas-Salomé\pwindex{Andreas-Salomé, Lou 12.\,2.\,1861 Sankt Petersburg – 5.\,2.\,1937 Göttingen@\textsc{Andreas-Salomé, Lou} (12.\,2.\,1861 Sankt Petersburg – 5.\,2.\,1937 Göttingen), \emph{Schriftstellerin}|pwk} und Schnitzler
                  am Kahlenberg\oindex{Wien@\textbf{Wien}!XIX., Döbling@\textbf{XIX., Döbling}!Kahlenberg@\textbf{Kahlenberg}, \emph{Berg}|pwk} und dürften eine Postkarte an
                     Goldmann\pwindex{Goldmann, Paul 31.\,1.\,1865 Breslau – 25.\,9.\,1935 Wien@\textsc{Goldmann, Paul} (31.\,1.\,1865 Breslau – 25.\,9.\,1935 Wien), \emph{Schriftsteller, Journalist}|pwk} geschickt haben.}}}\label{K_L02735-5}, die
               Photographie, Deine lieben Briefe haben mich{ }ſo innig erfreut! Es thut mir{ }ſo wohl,
               daß Ihr und Du beſonders an mich denkſt, daß ich mich ein wenig bei Euch weiß. Dieſe
               kleinen Gaben bewegen mich{ }ſehr –{ }ſie rühren mich (wenn das nicht {\pb}ſo ein dummes Wort wäre). Dank, tauſend Dank!\pend
           
\pstart
           Daß Ihr\pwindex{Beer-Hofmann, Richard 11.\,7.\,1866 Wien – 26.\,9.\,1945 New York City@\textsc{Beer-Hofmann, Richard} (11.\,7.\,1866 Wien – 26.\,9.\,1945 New York City), \emph{Schriftsteller}|pwv} mit Frau \textsc{Andreas\pwindex{Andreas-Salomé, Lou 12.\,2.\,1861 Sankt Petersburg – 5.\,2.\,1937 Göttingen@\textsc{Andreas-Salomé, Lou} (12.\,2.\,1861 Sankt Petersburg – 5.\,2.\,1937 Göttingen), \emph{Schriftstellerin}|pw}} Freund geworden{ }ſeid, iſt{ }ſo gekommen, wie ich es erwartet. Sie gehört zu uns.
               Denn{ }ſie iſt ein lieber, feiner und ehrlicher Menſch\pwindex{Andreas-Salomé, Lou 12.\,2.\,1861 Sankt Petersburg – 5.\,2.\,1937 Göttingen@\textsc{Andreas-Salomé, Lou} (12.\,2.\,1861 Sankt Petersburg – 5.\,2.\,1937 Göttingen), \emph{Schriftstellerin}|pwv}. Und ich weiß aus Erfahrung, wie wohl der Umgang mit
               dieſer Frau\pwindex{Andreas-Salomé, Lou 12.\,2.\,1861 Sankt Petersburg – 5.\,2.\,1937 Göttingen@\textsc{Andreas-Salomé, Lou} (12.\,2.\,1861 Sankt Petersburg – 5.\,2.\,1937 Göttingen), \emph{Schriftstellerin}|pwv} thut!
               Klimatiſche Wirkung – das{ }ſagſt Du{ }ſehr gut. Aber nun iſt Eines zu beachten: {\pb}Dieſe Frau\pwindex{Andreas-Salomé, Lou 12.\,2.\,1861 Sankt Petersburg – 5.\,2.\,1937 Göttingen@\textsc{Andreas-Salomé, Lou} (12.\,2.\,1861 Sankt Petersburg – 5.\,2.\,1937 Göttingen), \emph{Schriftstellerin}|pwv}, die{ }ſo ganz unperſönlich wirkt – manchmal{ }ſo wie
               abſoluter Verſtand und abſolute Wahrheit – hat eine heiße Sehnſucht, aus dieſer
               Verſtandes-Sphäre herauszukommen. Sie will \uline{Weib}{ }ſein,
               will lieben und geliebt werden. Und wenn{ }ſie aus dem Abſoluten ins Menſchliche
               niederſteigen wollte – in den Tag hinein, wie \strikeout{das} die
               erſte beſte kleine {\pb}\label{K_L02735-6v}\edtext{Nähterin}{\lemma{\textnormal{\emph{Nähterin}}}\Cendnote{\textnormal{veraltet: Näherin}}}\label{K_L02735-6} – wenn ich Weibliche\substVorne{}\textsuperscript{\textcolor{gray}{r}}\substDazwischen{}s\substHinten{} an ihr merkte – \label{K_L02735-7v}\edtext{\begin{otherlanguage}{french}\textsc{des douceurs, des chatteries}\end{otherlanguage}}{\lemma{\textnormal{\emph{des … chatteries}}}\Cendnote{\textnormal{französisch: Schmeicheleien,
                  Zärtlichkeiten}}}\label{K_L02735-7} – Weibliches, das{ }ſo gar nicht zu ihr gehört (obwohl{ }ſie
               auch nicht unangenehm männlich ist) – dann war{ }ſie \strikeout{im}
               mir immer verhaßt. Jawohl, ein nervöſer Haß! Gegen dieſe Frau\pwindex{Andreas-Salomé, Lou 12.\,2.\,1861 Sankt Petersburg – 5.\,2.\,1937 Göttingen@\textsc{Andreas-Salomé, Lou} (12.\,2.\,1861 Sankt Petersburg – 5.\,2.\,1937 Göttingen), \emph{Schriftstellerin}|pwv}, die mir{ }ſo viel Gutes gethan, wie
               Wenige auf \strikeout{a} der Welt! Die an mich geglaubt! Die{ }ſich
               die Mühe genommen hat, an {\pb}mich zu glauben! Es iſt
               abſcheulich! Aber zu Zeiten haßte ich{ }ſie, ich muß es Dir{ }ſagen. In einer gewiſſen
               Entfernung \strikeout{war{ }ſ} hatte ich eine große Verehrung für{ }ſie. Je näher{ }ſie mir kam, umſo weniger{ }ſympathiſch wurde{ }ſie mir.\pend
           
\pstart
           Nun wohl, die Frau\pwindex{Andreas-Salomé, Lou 12.\,2.\,1861 Sankt Petersburg – 5.\,2.\,1937 Göttingen@\textsc{Andreas-Salomé, Lou} (12.\,2.\,1861 Sankt Petersburg – 5.\,2.\,1937 Göttingen), \emph{Schriftstellerin}|pwv} weiß mit
               ihrem unfehlbaren Verſtande{ }ſehr wohl, daß{ }ſie dieſe unperſönliche Wirkung ausübt.
               »Klimatischer {\pb}Einfluß«, man kann es nicht beſſer{ }ſagen. Sie\pwindex{Andreas-Salomé, Lou 12.\,2.\,1861 Sankt Petersburg – 5.\,2.\,1937 Göttingen@\textsc{Andreas-Salomé, Lou} (12.\,2.\,1861 Sankt Petersburg – 5.\,2.\,1937 Göttingen), \emph{Schriftstellerin}|pwv} will aber
               perſönlich wirken – als Weib wirken. Und das iſt nun die Tragödie ihres Lebens.\pend
           
\pstart
           Daß{ }ſie{ }ſich zu Euch\pwindex{Beer-Hofmann, Richard 11.\,7.\,1866 Wien – 26.\,9.\,1945 New York City@\textsc{Beer-Hofmann, Richard} (11.\,7.\,1866 Wien – 26.\,9.\,1945 New York City), \emph{Schriftsteller}|pwv}
               hingezogen fühlt, verſtehe ich{ }ſehr gut. Sie hat{ }ſich für mich intereſſirt, weil ich
               ein Typus war, den{ }ſie noch nicht kannte: warm, melancholiſch, weich und \strikeout{wien\oindex{Wien@\textbf{Wien}, \emph{Verwaltungsgebiet}|pw}\textcolor{gray}{e}} überhaupt wien\oindex{Wien@\textbf{Wien}, \emph{Verwaltungsgebiet}|pw}eriſch. Und nun findet{ }ſie
               bei Euch\pwindex{Beer-Hofmann, Richard 11.\,7.\,1866 Wien – 26.\,9.\,1945 New York City@\textsc{Beer-Hofmann, Richard} (11.\,7.\,1866 Wien – 26.\,9.\,1945 New York City), \emph{Schriftsteller}|pwv} dieſen {\pb}\strikeout{Tys} Typus in{ }ſeiner Vervollkommung, während ich doch
               nur Anſätze dazu habe. Und gerade das iſt es, wonach ſie\pwindex{Andreas-Salomé, Lou 12.\,2.\,1861 Sankt Petersburg – 5.\,2.\,1937 Göttingen@\textsc{Andreas-Salomé, Lou} (12.\,2.\,1861 Sankt Petersburg – 5.\,2.\,1937 Göttingen), \emph{Schriftstellerin}|pwv}{ }ſich{ }ſehnt: dieſer Gemüthston, in dem{ }ſoviel warmes Leben iſt........\pend
           
\pstart
           Nach \label{K_L02735-8v}\edtext{\textsc{Kopenhagen\oindex{Kopenhagen@\textbf{Kopenhagen}, \emph{Hauptstadt}|pw}}}{\lemma{\textnormal{\emph{Kopenhagen}}}\Cendnote{\textnormal{Die Reise fand erst ein Jahr später als
                  geplant, im August 1896, statt. Goldmann\pwindex{Goldmann, Paul 31.\,1.\,1865 Breslau – 25.\,9.\,1935 Wien@\textsc{Goldmann, Paul} (31.\,1.\,1865 Breslau – 25.\,9.\,1935 Wien), \emph{Schriftsteller, Journalist}|pwk} kam ebenfalls mit.}}}\label{K_L02735-8} kann ich nicht kommen.
               Ich muß im Auguſt nach \textsc{Tölz\oindex{Bad Tölz@\textbf{Bad Tölz}, \emph{Hauptstadt}|pw}}, zur Kur. Werde ich Dich{ }ſehen? Du wirſt {\pb}Dich
               natürlich in Deinen Plänen durch mich nicht{ }ſtören laſſen. \strikeout{\textcolor{gray}{×}\-\textcolor{gray}{×}\-\textcolor{gray}{×}\-\textcolor{gray}{×}}{ }\textsc{Kopenhagen\oindex{Kopenhagen@\textbf{Kopenhagen}, \emph{Hauptstadt}|pw}} mußt und{ }ſollſt Du{ }ſehen. Aber vielleicht ließe{ }ſich doch eine Vereinbarung
               treffen für die Rückreiſe.\pend
           
\pstart
           Ich{ }ſende Dir anbei wieder einige Artikel. Beſonders in der »\textsc{\begin{otherlanguage}{french}Revue Blanche\pwindex{Revue blanche@\emph{La Revue blanche}|pw}\end{otherlanguage}}« mache ich Dich aufmerkſam auf die \label{K_L02735-9v}\edtext{Vertheidigung\pwindex{Adam, Paul 6.\,12.\,1862 Paris – 2.\,1.\,1920 ebd.@\textsc{Adam, Paul} (6.\,12.\,1862 Paris – 2.\,1.\,1920 ebd.), \emph{Schriftsteller, Kunstkritiker}!Assaut malicieux«@\strich\emph{»L’Assaut malicieux«}|pwv}}{\lemma{\textnormal{\emph{Vertheidigung}}}\Cendnote{\textnormal{Paul Adam\pwindex{Adam, Paul 6.\,12.\,1862 Paris – 2.\,1.\,1920 ebd.@\textsc{Adam, Paul} (6.\,12.\,1862 Paris – 2.\,1.\,1920 ebd.), \emph{Schriftsteller, Kunstkritiker}|pwk}: \emph{»L’Assaut malicieux«}\pwindex{Adam, Paul 6.\,12.\,1862 Paris – 2.\,1.\,1920 ebd.@\textsc{Adam, Paul} (6.\,12.\,1862 Paris – 2.\,1.\,1920 ebd.), \emph{Schriftsteller, Kunstkritiker}!Assaut malicieux«@\strich\emph{»L’Assaut malicieux«}|pwk}. In: \emph{La Revue blanche}\pwindex{Revue blanche@\emph{La Revue blanche}|pwk}, Jg. 8, Nr. 47, 15. 5. 1895, S. 458–462.}}}\label{K_L02735-9} des \textsc{Oscar Wilde\pwindex{Wilde, Oscar 16.\,10.\,1854 Dublin – 30.\,11.\,1900 Paris@\textsc{Wilde, Oscar} (16.\,10.\,1854 Dublin – 30.\,11.\,1900 Paris), \emph{Schriftsteller}|pw}} durch \textsc{Paul Adam\pwindex{Adam, Paul 6.\,12.\,1862 Paris – 2.\,1.\,1920 ebd.@\textsc{Adam, Paul} (6.\,12.\,1862 Paris – 2.\,1.\,1920 ebd.), \emph{Schriftsteller, Kunstkritiker}|pw}}. Ferner{ }ſende ich Dir ein {\pb}dummes Stück\pwindex{\textcolor{red}{\textsuperscript{XXXX indx1}}!amour s’amuse. Saynète@\strich\emph{L’amour s’amuse. Saynète}|pwv} »\begin{otherlanguage}{french}\textsc{L’amour s’amuse\pwindex{\textcolor{red}{\textsuperscript{XXXX indx1}}!amour s’amuse. Saynète@\strich\emph{L’amour s’amuse. Saynète}|pw}}\end{otherlanguage}«, das \uline{nicht} zu leſen iſt. Aber es iſt von
                  \textsc{\uline{Ibels\pwindex{Ibels, Henri-Gabriel 30.\,11.\,1867 Paris – 31.\,1.\,1936 ebd.@\textsc{Ibels, Henri-Gabriel} (30.\,11.\,1867 Paris – 31.\,1.\,1936 ebd.), \emph{Schriftsteller, Maler, Grafiker}|pw}}} illustrirt, einem neuen Künſtler\pwindex{Ibels, Henri-Gabriel 30.\,11.\,1867 Paris – 31.\,1.\,1936 ebd.@\textsc{Ibels, Henri-Gabriel} (30.\,11.\,1867 Paris – 31.\,1.\,1936 ebd.), \emph{Schriftsteller, Maler, Grafiker}|pwv}, deſſen{ }ſeltſame Art Dich intereſſiren wird. Den »\begin{otherlanguage}{french}\textsc{Courrier Francais\pwindex{Le Courrier français@\emph{Le Courrier français}|pw}}\end{otherlanguage}«{ }ſende ich Dir nur wegen der \label{K_L02735-10v}\edtext{Zeichnung\pwindex{Willette, Adolphe Léon 30.\,7.\,1857 Châlons-sur-Marne – 4.\,2.\,1926 Paris@\textsc{Willette, Adolphe Léon} (30.\,7.\,1857 Châlons-sur-Marne – 4.\,2.\,1926 Paris), \emph{Maler, Karikaturist, Illustrator}!Funérailles@\strich\emph{Les Funérailles}|pwv} von \textsc{Willette\pwindex{Willette, Adolphe Léon 30.\,7.\,1857 Châlons-sur-Marne – 4.\,2.\,1926 Paris@\textsc{Willette, Adolphe Léon} (30.\,7.\,1857 Châlons-sur-Marne – 4.\,2.\,1926 Paris), \emph{Maler, Karikaturist, Illustrator}|pw}} in der Mitte des Heft\pwindex{Le Courrier français@\emph{Le Courrier français}|pwv}es}{\lemma{\textnormal{\emph{Zeichnung … Heftes}}}\Cendnote{\textnormal{Vermutlich
                  handelt es sich um \emph{Les Funérailles}\pwindex{Willette, Adolphe Léon 30.\,7.\,1857 Châlons-sur-Marne – 4.\,2.\,1926 Paris@\textsc{Willette, Adolphe Léon} (30.\,7.\,1857 Châlons-sur-Marne – 4.\,2.\,1926 Paris), \emph{Maler, Karikaturist, Illustrator}!Funérailles@\strich\emph{Les Funérailles}|pwk}, das im Heft\pwindex{Le Courrier français@\emph{Le Courrier français}|pwkv}es vom 12. 5. 1895 auf
                  einer Doppelseite in der Mitte abgebildet ist.}}}\label{K_L02735-10}. Endlich mein \textsc{Salon}-Feuilleton\pwindex{Goldmann, Paul 31.\,1.\,1865 Breslau – 25.\,9.\,1935 Wien@\textsc{Goldmann, Paul} (31.\,1.\,1865 Breslau – 25.\,9.\,1935 Wien), \emph{Schriftsteller, Journalist}!Pariser Malerei. (Der Salon der Champs Elysées.) [I].@\strich\emph{Pariser Malerei. (Der Salon der Champs Elysées.) [I].}|pwv}. Ich habe es hauptſächlich für
               Dich geſchrieben und,{ }ſowenig es mir gefällt, möchte {\pb}ich doch daß Du es lieſt.\pend
           
\pstart
           Grüß’ Dich Gott, mein lieber Freund! Grüße \textsc{Richard\pwindex{Beer-Hofmann, Richard 11.\,7.\,1866 Wien – 26.\,9.\,1945 New York City@\textsc{Beer-Hofmann, Richard} (11.\,7.\,1866 Wien – 26.\,9.\,1945 New York City), \emph{Schriftsteller}|pw}} und die Frau \textsc{Andreas\pwindex{Andreas-Salomé, Lou 12.\,2.\,1861 Sankt Petersburg – 5.\,2.\,1937 Göttingen@\textsc{Andreas-Salomé, Lou} (12.\,2.\,1861 Sankt Petersburg – 5.\,2.\,1937 Göttingen), \emph{Schriftstellerin}|pw}}.\pend
           
\pstart
           Schreib’ mir bald!\pend
           
\pstart
           Und nächſtens bekommſt Du den großen Brief!\pend
           
\pstart
           Ich umarme {\\[\baselineskip]}Dich von Herzen {\\[\baselineskip]}Dein {\\[\baselineskip]}\spacefill\mbox{Paul Goldmann.}\pend
           \leftskip=0em{}\selectlanguage{ngerman}\endnumbering\briefempfaengerindex{Schnitzler, Arthur@\textsc{Schnitzler, Arthur}!zzzGoldmann, Paul@\emph{von Paul Goldmann}!1895-05-191@{19. 5. [1895]}|)be}\mylabel{L02735h}  \newcommand{\dateiname}{L02735}\newcommand{\titel}{Paul Goldmann an Arthur Schnitzler, 19. 5. [1895]}\newcommand{\editorInnen}{Martin Anton Müller und Laura Untner}%% latex-leseansicht-abspann.tex
%% Abspann für die Leseansicht.
%% Der Schalter \ifkorrekturansicht ist bereits durch den Vorspann gesetzt.

%% latex-abspann.tex
%% Gemeinsamer Abspann für Korrekturansicht und Leseansicht.
%% Setzt den Schalter \ifkorrekturansicht voraus (gesetzt in den
%% einbindenden Dateien latex-korrekturansicht-abspann.tex bzw.
%% latex-leseansicht-abspann.tex).
%% ---------------------------------------------------------------

\normalsize

% Das esempio-Environment wird nur in der Leseansicht benötigt
\ifkorrekturansicht\else
\newenvironment{esempio}[3]%
{
    \vspace{1.5ex}
    \rlap{\underline{#1}}
    \par
    \setlength{\parindent}{0cm}
    \nopagebreak
    \leftskip=#2cm
    \rightskip=#3cm
}
{
    \par
}
\fi

\doendnotes{C}
\bigskip
\vfill

\clearpage

\footnotesize

\ifkorrekturansicht
  \lohead{\textsc{register}}
\fi

% theindex-Environment neu definieren ohne reledmac
\makeatletter
\renewenvironment{theindex}{%
  \ifkorrekturansicht
    \section*{\indexname}%
  \else
    \subsubsection*{Index der erwähnten Entitäten}%
  \fi
  \setlength{\parindent}{0pt}%
  \setlength{\parskip}{0pt plus 0.3pt}%
  \let\item\@idxitem
}{%
  \ifkorrekturansicht\clearpage\fi
}
\makeatother

\IfFileExists{\jobname-pw.ind}{\input{\jobname-pw.ind}}{}

% Quellenangabe nur in der Leseansicht
\ifkorrekturansicht\else
% Fallback-Definitionen, falls die .tex-Datei \titel etc. nicht gesetzt hat
\providecommand{\titel}{}
\providecommand{\editorInnen}{}
\providecommand{\dateiname}{\jobname}

\vspace{3cm}

\vfill

\footnotesize
\textsc{Quelle}: \titel. Herausgegeben von {\editorInnen}. In: \emph{Arthur Schnitzler: Briefwechsel mit Autorinnen und Autoren}.
 Digitale Edition, https://schnitzler-briefe.acdh.oeaw.ac.at/{\dateiname}.html (Stand \today)
\fi

\end{document}


