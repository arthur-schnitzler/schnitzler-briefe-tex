%% latex-leseansicht-vorspann.tex
%% Vorspann für die Leseansicht.
%% Lädt die gemeinsame Datei latex-vorspann.tex mit nicht gesetztem Schalter.

\newif\ifkorrekturansicht
\korrekturansichtfalse

\input{../tex-inputs/latex-vorspann}

\begin{center}
            \textcolor{red}{ENTWURF. ENTZIFFERUNG NOCH NICHT KORREKTURGELESEN}
                      \end{center}
            
               \section[Hermann Bahr an Arthur Schnitzler, 7. 3. 1914]{ Hermann Bahr an Arthur Schnitzler, 7. 3. 1914}\nopagebreak\mylabel{v}\rehead{ }\begin{ledgroupsized}[t]{13cm}\normalsize\beginnumbering\briefempfaengerindex{Schnitzler, Arthur@\textsc{Schnitzler, Arthur}!zzzBahr, Hermann@\emph{von Hermann Bahr}!1914-03-071@{7. 3. 1914}|(be} \toendnotes[C]{\smallbreak\pagebreak[2]} \Standort{CUL, Schnitzler, B 5b.}
\physDesc{Brief, 1 Blatt, 2 Seiten
\newline{}Handschrift: schwarze Tinte, deutsche Kurrent
\newline{}Schnitzler: 1) mit Bleistift ergänzt »Bahr« 2) mit rotem Buntstift vereinzelte Unterstreichungen\newline{}Ordnung: mit Bleistift von unbekannter Hand nummeriert:
                                    »179« }\buchAbdrucke{\weitereDrucke{Hermann Bahr, Arthur Schnitzler: \emph{Briefwechsel, Aufzeichnungen, Dokumente (1891–1931)}. Hg. Kurt Ifkovits und Martin Anton Müller. Göttingen: \emph{Wallstein} 2018, S. 492–493.} }\toendnotes[C]{\smallbreak}\pstart
           \raggedleft{}{\pb}Salzburg\oindex{Salzburg@\textbf{Salzburg}|pw}{ }7. 3. 14\pend
           \pstart\center{}Lieber Arthur!\pend\pstart
           Ich bin {[}in{]} der letzten Zeit ſo viel herumgegaukelt (\label{K_L02166_1v}\edtext{Czernowitz\oindex{Czernowitz@\textbf{Czernowitz}|pw}}{\lemma{\textnormal{\emph{Czernowitz}}}\Cendnote{\textnormal{am 13. 1. 1914}}}\label{K_L02166_1h}, \label{K_L02166_2v}\edtext{Lemberg\oindex{Lviv@\textbf{Lviv}|pw}}{\lemma{\textnormal{\emph{Lemberg}}}\Cendnote{\textnormal{bereits zuvor, am 12. 1. 1914}}}\label{K_L02166_2h}, \label{K_L02166_3v}\edtext{Brünn\oindex{Bruenn@\textbf{Brünn}|pw}}{\lemma{\textnormal{\emph{Brünn}}}\Cendnote{\textnormal{am 14. 1. 1914}}}\label{K_L02166_3h}, dann \label{K_L02166_4v}\edtext{Berchtesgaden\oindex{Berchtesgaden@\textbf{Berchtesgaden}|pw}}{\lemma{\textnormal{\emph{Berchtesgaden}}}\Cendnote{\textnormal{vom 29. 1. bis zum 4. 2. 1914}}}\label{K_L02166_4h}{ }ſkiend, dann Münchener\oindex{Muenchen@\textbf{München}|pw}{ }\label{K_L02166_5v}\edtext{Suffragetten}{\lemma{\textnormal{\emph{Suffragetten}}}\Cendnote{\textnormal{am 19. 2. 1914 Vortrag über das
                  »Frauenstimmrecht« in München\oindex{Muenchen@\textbf{München}|pwk}}}}\label{K_L02166_5h}, dann
                  \label{K_L02166_6v}\edtext{Darmſtadt\oindex{Darmstadt@\textbf{Darmstadt}|pw}}{\lemma{\textnormal{\emph{Darmſtadt}}}\Cendnote{\textnormal{vom 27. 2. bis zum 1. 3. 1914}}}\label{K_L02166_6h} bei Hofe – die Welt iſt ſehr rund), daß ich jetzt erſt dazu komme, Dir zu
               ſagen, wie furchtbar leid mir tat, Euren\pwindex{Schnitzler, Olga 17.01.1882 – 13.01.1970@\textsc{Schnitzler, Olga} (17.01.1882 – 13.01.1970), \emph{Schauspielerin, Sängerin}|pwv} lieben Beſuch verſäumt zu haben. So gern möcht ich Euch Beide\pwindex{Schnitzler, Olga 17.01.1882 – 13.01.1970@\textsc{Schnitzler, Olga} (17.01.1882 – 13.01.1970), \emph{Schauspielerin, Sängerin}|pwv} wieder einmal ſehen, ſo gern Euch\pwindex{Schnitzler, Olga 17.01.1882 – 13.01.1970@\textsc{Schnitzler, Olga} (17.01.1882 – 13.01.1970), \emph{Schauspielerin, Sängerin}|pwv} unſere Behauſung und den
               Park zeigen, ſo viel hätt ich Dich zu fragen, Dir zu ſagen! Hoffentlich {\pb}trifft ſichs das nächſte Mal beſſer. Aber wann wird
               dies nächſte Mal ſein? Wir gehen ja heuer schon zu Pfingſten nach
                  \label{K_L02166_7v}\edtext{Venedig\oindex{Venedig@\textbf{Venedig}|pw}}{\lemma{\textnormal{\emph{Venedig}}}\Cendnote{\textnormal{vom 6. bis zum
                     25. 6. 1914v}}}\label{K_L02166_7h}, da wir Ende Juni{ }ſchon nach \label{K_L02166_8v}\edtext{Bayreuth\oindex{Bayreuth@\textbf{Bayreuth}|pw}}{\lemma{\textnormal{\emph{Bayreuth}}}\Cendnote{\textnormal{vom 27. 7. bis zum
                     14. 8. 1914}}}\label{K_L02166_8h} müſſen, bis Ende Auguſt dort bleiben und uns alſo eigentlich
               jetzt ſchon auf den Herbſt hier freuen, bevor noch der Frühling da iſt.\pend
           \pstart
           Laſſt es Euch immer gut gehen, grüß auch die Kinder\pwindex{Schnitzler, Heinrich 09.08.1902 – 12.07.1982@\textsc{Schnitzler, Heinrich} (09.08.1902 – 12.07.1982), \emph{Regisseur, Schauspieler}|pwv}\pwindex{Schnitzler, Lili 13.09.1909 – 26.07.1928@\textsc{Schnitzler, Lili} (13.09.1909 – 26.07.1928)|pwv}, wenn ſie gleich nichts von mir wiſſen, herzlich von
               mir und bleibt mir gut, wie ich Euch\pwindex{Schnitzler, Olga 17.01.1882 – 13.01.1970@\textsc{Schnitzler, Olga} (17.01.1882 – 13.01.1970), \emph{Schauspielerin, Sängerin}|pwv} immer derſelbe bleiben will, eben dieſer Euer alter\pend
           \pstart \spacefill\mbox{Hermann}\pend{}\endnumbering\briefempfaengerindex{Schnitzler, Arthur@\textsc{Schnitzler, Arthur}!zzzBahr, Hermann@\emph{von Hermann Bahr}!1914-03-071@{7. 3. 1914}|)be}\mylabel{h}\end{ledgroupsized}  \newcommand{\dateiname}{L02166}\newcommand{\titel}{Hermann Bahr an Arthur Schnitzler, 7. 3. 1914}\newcommand{\editorInnen}{ Kurt Ifkovits,  Martin Anton Müller}%% latex-leseansicht-abspann.tex
%% Abspann für die Leseansicht.
%% Der Schalter \ifkorrekturansicht ist bereits durch den Vorspann gesetzt.

%% latex-abspann.tex
%% Gemeinsamer Abspann für Korrekturansicht und Leseansicht.
%% Setzt den Schalter \ifkorrekturansicht voraus (gesetzt in den
%% einbindenden Dateien latex-korrekturansicht-abspann.tex bzw.
%% latex-leseansicht-abspann.tex).
%% ---------------------------------------------------------------

\normalsize

% Das esempio-Environment wird nur in der Leseansicht benötigt
\ifkorrekturansicht\else
\newenvironment{esempio}[3]%
{
    \vspace{1.5ex}
    \rlap{\underline{#1}}
    \par
    \setlength{\parindent}{0cm}
    \nopagebreak
    \leftskip=#2cm
    \rightskip=#3cm
}
{
    \par
}
\fi

\doendnotes{C}
\bigskip
\vfill

\clearpage

\footnotesize

\ifkorrekturansicht
  \lohead{\textsc{register}}
\fi

% theindex-Environment neu definieren ohne reledmac
\makeatletter
\renewenvironment{theindex}{%
  \ifkorrekturansicht
    \section*{\indexname}%
  \else
    \subsubsection*{Index der erwähnten Entitäten}%
  \fi
  \setlength{\parindent}{0pt}%
  \setlength{\parskip}{0pt plus 0.3pt}%
  \let\item\@idxitem
}{%
  \ifkorrekturansicht\clearpage\fi
}
\makeatother

\IfFileExists{\jobname-pw.ind}{\input{\jobname-pw.ind}}{}

% Quellenangabe nur in der Leseansicht
\ifkorrekturansicht\else
% Fallback-Definitionen, falls die .tex-Datei \titel etc. nicht gesetzt hat
\providecommand{\titel}{}
\providecommand{\editorInnen}{}
\providecommand{\dateiname}{\jobname}

\vspace{3cm}

\vfill

\footnotesize
\textsc{Quelle}: \titel. Herausgegeben von {\editorInnen}. In: \emph{Arthur Schnitzler: Briefwechsel mit Autorinnen und Autoren}.
 Digitale Edition, https://schnitzler-briefe.acdh.oeaw.ac.at/{\dateiname}.html (Stand \today)
\fi

\end{document}


      