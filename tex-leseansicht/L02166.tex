%% latex-korrekturansicht-vorspann.tex
%% Vorspann für die Korrekturansicht.
%% Lädt die gemeinsame Datei latex-vorspann.tex mit gesetztem Schalter.

\newif\ifkorrekturansicht
\korrekturansichttrue

\input{../tex-inputs/latex-vorspann}


\section[Hermann Bahr an Arthur Schnitzler, 7. 3. 1914]{L02166 Hermann Bahr an Arthur Schnitzler, 7. 3. 1914}
\nopagebreak\mylabel{L02166v}
\rehead{ }\normalsize\beginnumbering\briefempfaengerindex{Schnitzler, Arthur@\textsc{Schnitzler, Arthur}!zzzBahr, Hermann@\emph{von Hermann Bahr}!1914-03-071@{7. 3. 1914}|(be}
\toendnotes[C]{\smallbreak\pagebreak[2]}\Standort{CUL, Schnitzler, B 5b.}
\physDesc{Brief, 1 Blatt, 2 Seiten, 960 Zeichen
\newline{}Handschrift: schwarze Tinte, deutsche Kurrent
\newline{}Schnitzler: 1) mit Bleistift ergänzt »Bahr«  2) mit rotem Buntstift vereinzelte Unterstreichungen
\newline{}Ordnung: mit Bleistift von unbekannter Hand nummeriert:
                                    »179« }
\buchAbdrucke{\weitereDrucke{Hermann Bahr, Arthur Schnitzler: \emph{Briefwechsel, Aufzeichnungen, Dokumente (1891–1931)}. Göttingen: \emph{Wallstein} 2018, S. 492–493.} }\toendnotes[C]{\smallbreak}
\pstart
           \raggedleft{}{\pb}Salzburg\oindex{Salzburg@\textbf{Salzburg}, \emph{A.ADM2}|pw}{ }7. 3. 14\pend
           
\pstart\center{}Lieber Arthur!\pend\vspace{0.5em}
\pstart
           Ich bin {[}in{]} der letzten Zeit ſo viel herumgegaukelt (\label{K_L02166-1v}\edtext{Czernowitz\oindex{Czernowitz@\textbf{Czernowitz}, \emph{P.PPLA}|pw}}{\lemma{\textnormal{\emph{Czernowitz}}}\Cendnote{\textnormal{am 13. 1. 1914}}}\label{K_L02166-1}, \label{K_L02166-2v}\edtext{Lemberg\oindex{Lviv@\textbf{Lviv}, \emph{P.PPLA}|pw}}{\lemma{\textnormal{\emph{Lemberg}}}\Cendnote{\textnormal{bereits zuvor, am 12. 1. 1914}}}\label{K_L02166-2}, \label{K_L02166-3v}\edtext{Brünn\oindex{Bruenn@\textbf{Brünn}, \emph{P.PPLA}|pw}}{\lemma{\textnormal{\emph{Brünn}}}\Cendnote{\textnormal{am 14. 1. 1914}}}\label{K_L02166-3}, dann \label{K_L02166-4v}\edtext{Berchtesgaden\oindex{Berchtesgaden@\textbf{Berchtesgaden}, \emph{P.PPL}|pw}}{\lemma{\textnormal{\emph{Berchtesgaden}}}\Cendnote{\textnormal{vom 29. 1. 1914 bis zum 4. 2. 1914}}}\label{K_L02166-4}{ }ſkiend, dann Münchener\oindex{Muenchen@\textbf{München}, \emph{P.PPLA}|pw}{ }\label{K_L02166-5v}\edtext{Suffragetten}{\lemma{\textnormal{\emph{Suffragetten}}}\Cendnote{\textnormal{Am 19. 2. 1914 hielt Bahr\pwindex{Bahr, Hermann 19.07.1863 – 15.01.1934@\textsc{Bahr, Hermann} (19.07.1863 – 15.01.1934), \emph{Schriftsteller/Schriftstellerin, Kritiker/Kritikerin}|pwk} in München\oindex{Muenchen@\textbf{München}, \emph{P.PPLA}|pwk} einen Vortrag über das
                  »Frauenstimmrecht«.
               }}}\label{K_L02166-5}, dann \label{K_L02166-6v}\edtext{Darmſtadt\oindex{Darmstadt@\textbf{Darmstadt}, \emph{P.PPLA2}|pw}}{\lemma{\textnormal{\emph{Darmſtadt}}}\Cendnote{\textnormal{vom 27. 2. 1914 bis zum 1. 3. 1914}}}\label{K_L02166-6} bei Hofe – die Welt iſt ſehr rund), daß ich jetzt erſt dazu komme, Dir zu
               ſagen, wie furchtbar leid mir tat, Euren\pwindex{Schnitzler, Olga 17.01.1882 – 13.01.1970@\textsc{Schnitzler, Olga} (17.01.1882 – 13.01.1970), \emph{Schauspieler/Schauspielerin, Sänger/Sängerin}|pwv} lieben Beſuch verſäumt zu haben. So gern möcht ich Euch Beide\pwindex{Schnitzler, Olga 17.01.1882 – 13.01.1970@\textsc{Schnitzler, Olga} (17.01.1882 – 13.01.1970), \emph{Schauspieler/Schauspielerin, Sänger/Sängerin}|pwv} wieder einmal ſehen, ſo gern Euch\pwindex{Schnitzler, Olga 17.01.1882 – 13.01.1970@\textsc{Schnitzler, Olga} (17.01.1882 – 13.01.1970), \emph{Schauspieler/Schauspielerin, Sänger/Sängerin}|pwv} unſere Behauſung und den
               Park zeigen, ſo viel hätt ich Dich zu fragen, Dir zu ſagen! Hoffentlich {\pb}trifft ſichs das nächſte Mal beſſer. Aber wann wird
               dies nächſte Mal ſein? Wir gehen ja heuer schon zu Pfingſten nach
                  \label{K_L02166-7v}\edtext{Venedig\oindex{Venedig@\textbf{Venedig}, \emph{P.PPLA}|pw}}{\lemma{\textnormal{\emph{Venedig}}}\Cendnote{\textnormal{vom 6. 6. 1914 bis zum
                     25. 6. 1914}}}\label{K_L02166-7}, da wir Ende Juni{ }ſchon nach \label{K_L02166-8v}\edtext{Bayreuth\oindex{Bayreuth@\textbf{Bayreuth}, \emph{P.PPLA2}|pw}}{\lemma{\textnormal{\emph{Bayreuth}}}\Cendnote{\textnormal{vom 27. 7. 1914 bis zum
                     14. 8. 1914}}}\label{K_L02166-8} müſſen, bis Ende Auguſt dort bleiben und uns alſo eigentlich
               jetzt ſchon auf den Herbſt hier freuen, bevor noch der Frühling da iſt.\pend
           
\pstart
           Laſſt es Euch immer gut gehen, grüß auch die Kinder\pwindex{Schnitzler, Heinrich 09.08.1902 – 12.07.1982@\textsc{Schnitzler, Heinrich} (09.08.1902 – 12.07.1982), \emph{Regisseur/Regisseurin, Schauspieler/Schauspielerin}|pwv}\pwindex{Cappellini, Lili 13.09.1909 – 26.07.1928@\textsc{Cappellini, Lili} (13.09.1909 – 26.07.1928)|pwv}, wenn ſie gleich nichts von
               mir wiſſen, herzlich von mir und bleibt mir gut, wie ich Euch\pwindex{Schnitzler, Olga 17.01.1882 – 13.01.1970@\textsc{Schnitzler, Olga} (17.01.1882 – 13.01.1970), \emph{Schauspieler/Schauspielerin, Sänger/Sängerin}|pwv} immer derſelbe bleiben will, eben
               dieſer Euer alter\pend
           \pstart \spacefill\mbox{Hermann}\pend{}\selectlanguage{ngerman}\endnumbering\briefempfaengerindex{Schnitzler, Arthur@\textsc{Schnitzler, Arthur}!zzzBahr, Hermann@\emph{von Hermann Bahr}!1914-03-071@{7. 3. 1914}|)be}\mylabel{L02166h}  \normalsize

\doendnotes{C}
\bigskip
\vfill

\clearpage

\footnotesize

\lohead{\textsc{register}}

% Definiere theindex-Environment komplett neu ohne reledmac
\makeatletter
\renewenvironment{theindex}{%
  \section*{\indexname}%
  \setlength{\parindent}{0pt}%
  \setlength{\parskip}{0pt plus 0.3pt}%
  \let\item\@idxitem
}{%
  \clearpage
}
\makeatother

\IfFileExists{\jobname-pw.ind}{\input{\jobname-pw.ind}}{}

\end{document}

      