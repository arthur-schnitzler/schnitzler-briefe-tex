%% latex-leseansicht-vorspann.tex
%% Vorspann für die Leseansicht.
%% Lädt die gemeinsame Datei latex-vorspann.tex mit nicht gesetztem Schalter.

\newif\ifkorrekturansicht
\korrekturansichtfalse

\input{../tex-inputs/latex-vorspann}


\section[Hermann Bahr an Arthur Schnitzler, 7. 3. 1914]{L02166 Hermann Bahr an Arthur Schnitzler, 7. 3. 1914}
\nopagebreak\mylabel{L02166v}
\rehead{ }\normalsize\beginnumbering\briefempfaengerindex{Schnitzler, Arthur@\textsc{Schnitzler, Arthur}!zzzBahr, Hermann@\emph{von Hermann Bahr}!1914-03-071@{7. 3. 1914}|(be}
\toendnotes[C]{\smallbreak\pagebreak[2]}
\correspDesc{Versand  durch Hermann Bahr am 7. 3. 1914 in Salzburg
\newline{}Erhalt  durch Arthur Schnitzler im Zeitraum [8. 3. 1914
                  – 12. 3. 1914?] in Wien}\toendnotes[C]{\smallbreak}
\Standort{CUL, Schnitzler, B 5b.}
\physDesc{Brief, 1 Blatt, 2 Seiten, 960 Zeichen
\newline{}Handschrift: schwarze Tinte, deutsche Kurrent
\newline{}Schnitzler: 1) mit Bleistift ergänzt »Bahr«  2) mit rotem Buntstift vereinzelte Unterstreichungen
\newline{}Ordnung: mit Bleistift von unbekannter Hand nummeriert:
                                    »179« }
\buchAbdrucke{\weitereDrucke{Hermann Bahr, Arthur Schnitzler: \emph{Briefwechsel, Aufzeichnungen, Dokumente (1891–1931)}. Herausgegeben von Kurt Ifkovits und Martin Anton Müller. Göttingen: \emph{Wallstein} 2018, S. 492–493.} }\toendnotes[C]{\smallbreak}
\pstart
           \raggedleft{}{\pb}Salzburg\oindex{Salzburg@\textbf{Salzburg}, \emph{Verwaltungsgebiet}|pw}{ }7. 3. 14\pend
           
\pstart\center{}Lieber Arthur!\pend\vspace{0.5em}
\pstart
           Ich bin {[}in{]} der letzten Zeit{ }ſo viel herumgegaukelt (\label{K_L02166-1v}\edtext{Czernowitz\oindex{Czernowitz@\textbf{Czernowitz}|pw}}{\lemma{\textnormal{\emph{Czernowitz}}}\Cendnote{\textnormal{am 13. 1. 1914}}}\label{K_L02166-1}, \label{K_L02166-2v}\edtext{Lemberg\oindex{Lviv@\textbf{Lviv}|pw}}{\lemma{\textnormal{\emph{Lemberg}}}\Cendnote{\textnormal{bereits zuvor, am 12. 1. 1914}}}\label{K_L02166-2}, \label{K_L02166-3v}\edtext{Brünn\oindex{Brünn@\textbf{Brünn}|pw}}{\lemma{\textnormal{\emph{Brünn}}}\Cendnote{\textnormal{am 14. 1. 1914}}}\label{K_L02166-3}, dann \label{K_L02166-4v}\edtext{Berchtesgaden\oindex{Berchtesgaden@\textbf{Berchtesgaden}|pw}}{\lemma{\textnormal{\emph{Berchtesgaden}}}\Cendnote{\textnormal{vom 29. 1. 1914 bis zum 4. 2. 1914}}}\label{K_L02166-4}{ }ſkiend, dann Münchener\oindex{München@\textbf{München}|pw}{ }\label{K_L02166-5v}\edtext{Suffragetten}{\lemma{\textnormal{\emph{Suffragetten}}}\Cendnote{\textnormal{Am 19. 2. 1914 hielt Bahr\pwindex{Bahr, Hermann 19.\,7.\,1863 Linz – 15.\,1.\,1934 München@\textsc{Bahr, Hermann} (19.\,7.\,1863 Linz – 15.\,1.\,1934 München), \emph{Schriftsteller, Kritiker}|pwk} in München\oindex{München@\textbf{München}|pwk} einen Vortrag über das
                  »Frauenstimmrecht«.
               }}}\label{K_L02166-5}, dann \label{K_L02166-6v}\edtext{Darmſtadt\oindex{Darmstadt@\textbf{Darmstadt}, \emph{Hauptstadt}|pw}}{\lemma{\textnormal{\emph{Darmstadt}}}\Cendnote{\textnormal{vom 27. 2. 1914 bis zum 1. 3. 1914}}}\label{K_L02166-6} bei Hofe – die Welt iſt{ }ſehr rund), daß ich jetzt erſt dazu komme, Dir zu{ }ſagen, wie furchtbar leid mir tat, Euren\pwindex{Schnitzler, Olga 17.\,1.\,1882 Wien – 13.\,1.\,1970 Lugano@\textsc{Schnitzler, Olga} (17.\,1.\,1882 Wien – 13.\,1.\,1970 Lugano), \emph{Schauspielerin, Sängerin}|pwv} lieben Beſuch verſäumt zu haben. So gern möcht ich Euch Beide\pwindex{Schnitzler, Olga 17.\,1.\,1882 Wien – 13.\,1.\,1970 Lugano@\textsc{Schnitzler, Olga} (17.\,1.\,1882 Wien – 13.\,1.\,1970 Lugano), \emph{Schauspielerin, Sängerin}|pwv} wieder einmal{ }ſehen,{ }ſo gern Euch\pwindex{Schnitzler, Olga 17.\,1.\,1882 Wien – 13.\,1.\,1970 Lugano@\textsc{Schnitzler, Olga} (17.\,1.\,1882 Wien – 13.\,1.\,1970 Lugano), \emph{Schauspielerin, Sängerin}|pwv} unſere Behauſung und den
               Park zeigen,{ }ſo viel hätt ich Dich zu fragen, Dir zu{ }ſagen! Hoffentlich {\pb}trifft{ }ſichs das nächſte Mal beſſer. Aber wann wird
               dies nächſte Mal{ }ſein? Wir gehen ja heuer schon zu Pfingſten nach
                  \label{K_L02166-7v}\edtext{Venedig\oindex{Venedig@\textbf{Venedig}|pw}}{\lemma{\textnormal{\emph{Venedig}}}\Cendnote{\textnormal{vom 6. 6. 1914 bis zum
                     25. 6. 1914}}}\label{K_L02166-7}, da wir Ende Juni{ }ſchon nach \label{K_L02166-8v}\edtext{Bayreuth\oindex{Bayreuth@\textbf{Bayreuth}, \emph{Hauptstadt}|pw}}{\lemma{\textnormal{\emph{Bayreuth}}}\Cendnote{\textnormal{vom 27. 7. 1914 bis zum
                     14. 8. 1914}}}\label{K_L02166-8} müſſen, bis Ende Auguſt dort bleiben und uns alſo eigentlich
               jetzt{ }ſchon auf den Herbſt hier freuen, bevor noch der Frühling da iſt.\pend
           
\pstart
           Laſſt es Euch immer gut gehen, grüß auch die Kinder\pwindex{Schnitzler, Heinrich 9.\,8.\,1902 Hinterbrühl – 12.\,7.\,1982 Wien@\textsc{Schnitzler, Heinrich} (9.\,8.\,1902 Hinterbrühl – 12.\,7.\,1982 Wien), \emph{Regisseur, Schauspieler}|pwv}\pwindex{Cappellini, Lili 13.\,9.\,1909 Wien – 26.\,7.\,1928 Venedig@\textsc{Cappellini, Lili} (13.\,9.\,1909 Wien – 26.\,7.\,1928 Venedig)|pwv}, wenn{ }ſie gleich nichts von
               mir wiſſen, herzlich von mir und bleibt mir gut, wie ich Euch\pwindex{Schnitzler, Olga 17.\,1.\,1882 Wien – 13.\,1.\,1970 Lugano@\textsc{Schnitzler, Olga} (17.\,1.\,1882 Wien – 13.\,1.\,1970 Lugano), \emph{Schauspielerin, Sängerin}|pwv} immer derſelbe bleiben will, eben
               dieſer Euer alter\pend
           \pstart \spacefill\mbox{Hermann}\pend{}\selectlanguage{ngerman}\endnumbering\briefempfaengerindex{Schnitzler, Arthur@\textsc{Schnitzler, Arthur}!zzzBahr, Hermann@\emph{von Hermann Bahr}!1914-03-071@{7. 3. 1914}|)be}\mylabel{L02166h}  \newcommand{\dateiname}{L02166}\newcommand{\titel}{Hermann Bahr an Arthur Schnitzler, 7. 3. 1914}\newcommand{\editorInnen}{Herausgegeben von Martin Anton Müller}%% latex-leseansicht-abspann.tex
%% Abspann für die Leseansicht.
%% Der Schalter \ifkorrekturansicht ist bereits durch den Vorspann gesetzt.

%% latex-abspann.tex
%% Gemeinsamer Abspann für Korrekturansicht und Leseansicht.
%% Setzt den Schalter \ifkorrekturansicht voraus (gesetzt in den
%% einbindenden Dateien latex-korrekturansicht-abspann.tex bzw.
%% latex-leseansicht-abspann.tex).
%% ---------------------------------------------------------------

\normalsize

% Das esempio-Environment wird nur in der Leseansicht benötigt
\ifkorrekturansicht\else
\newenvironment{esempio}[3]%
{
    \vspace{1.5ex}
    \rlap{\underline{#1}}
    \par
    \setlength{\parindent}{0cm}
    \nopagebreak
    \leftskip=#2cm
    \rightskip=#3cm
}
{
    \par
}
\fi

\doendnotes{C}
\bigskip
\vfill

\clearpage

\footnotesize

\ifkorrekturansicht
  \lohead{\textsc{register}}
\fi

% theindex-Environment neu definieren ohne reledmac
\makeatletter
\renewenvironment{theindex}{%
  \ifkorrekturansicht
    \section*{\indexname}%
  \else
    \subsubsection*{Index der erwähnten Entitäten}%
  \fi
  \setlength{\parindent}{0pt}%
  \setlength{\parskip}{0pt plus 0.3pt}%
  \let\item\@idxitem
}{%
  \ifkorrekturansicht\clearpage\fi
}
\makeatother

\IfFileExists{\jobname-pw.ind}{\input{\jobname-pw.ind}}{}

% Quellenangabe nur in der Leseansicht
\ifkorrekturansicht\else
% Fallback-Definitionen, falls die .tex-Datei \titel etc. nicht gesetzt hat
\providecommand{\titel}{}
\providecommand{\editorInnen}{}
\providecommand{\dateiname}{\jobname}

\vspace{3cm}

\vfill

\footnotesize
\textsc{Quelle}: \titel. Herausgegeben von {\editorInnen}. In: \emph{Arthur Schnitzler: Briefwechsel mit Autorinnen und Autoren}.
 Digitale Edition, https://schnitzler-briefe.acdh.oeaw.ac.at/{\dateiname}.html (Stand \today)
\fi

\end{document}


