%% latex-korrekturansicht-vorspann.tex
%% Vorspann für die Korrekturansicht.
%% Lädt die gemeinsame Datei latex-vorspann.tex mit gesetztem Schalter.

\newif\ifkorrekturansicht
\korrekturansichttrue

\input{../tex-inputs/latex-vorspann}


\section[Elsa Plessner an Arthur Schnitzler, 10. 1. 1900]{L03724 Elsa Plessner an Arthur Schnitzler, 10. 1. 1900}
\nopagebreak\mylabel{L03724v}
\rehead{ }\normalsize\beginnumbering\briefempfaengerindex{Schnitzler, Arthur@\textsc{Schnitzler, Arthur}!zzzPlessner, Elsa@\emph{von Elsa Plessner}!1900-01-101@{10. 1. 1900}|(be}
\toendnotes[C]{\smallbreak\pagebreak[2]}\Standort{DLA, A:Schnitzler, HS.1985.1.419.}
\physDesc{Brief,  Blätter, 4 Seiten, 2015 Zeichen
\newline{}Handschrift: , lateinische Kurrent}\toendnotes[C]{\smallbreak}
\pstart
           {\pb}Wien I. Kärnthnerstraße 10\oindex{Kaerntner Strasse 10@\textbf{Kärntner Straße 10}, \emph{Wohngebäude (K.WHS)}|pw}\pend
           
\pstart
           \raggedleft{}den 10. Januar 1900\pend
           
\pstart{}Verehrter Herr Doctor!\pend\vspace{0.5em}
\pstart
           So schnell!! Dafür danke ich
      Ihnen doppelt!
      \pend
           
\pstart
           Ihr heutiger{ }\label{K_L03724-1v}\edtext{Brief}{\lemma{\textnormal{\emph{Brief}}}\Cendnote{\textnormal{nicht überliefert}}}\label{K_L03724-1} hat mir
      viel Freude gemacht. Sie haben
      nicht über »Schlamperei« und
      »Leichtsinn« geschimpft, wie sonst
      immer – das ist für mich der
      größte Erfolg! – – Sehr überrascht war ich, dass Sie die
      Theaterwirksamkeit \strikeout{»}des »ersten C.\pwindex{erste Kapitel. Schauspiel in drei Akten@\emph{Das erste Kapitel. Schauspiel in drei Akten}|pw}« in Abrede stellen. Zugegeben
      dass der Stoff eigentlich für
      eine Novelle gepasst hätte –
      ich selbst habe ihn darauf{\pb}hin ernstlich studiert, – bot er
      mir andrerseits durch die zahlreichen, auch in der Novelle nothwendigen Scenen – d. h. Dialoge, durch
      die Steigerung der Handlung und
      deren geringe Zeitdauer (1 ½ Tage)
      unleugbare dramatische, ja
      sogar Bühnenmöglichkeiten.
      Sie haben ja ganz recht – der
      Stoff ist sehr dünn und ich
      habe das nicht übersehen – aber
      er hat mich trotzdem gereizt –
      und ich will doch die Probe auf
      die Bühnentragfähigkeit machen. – Als Erstlingsstück
      ist es rettungslos dem Durchfallen geweiht – das weiß ich. –
      Aber als zweites – auf einen {\pb}
      gewissen literarischen Credit
      hin, will ich den Versuch
      einer Aufführung wagen. –\pend
           
\pstart
           D. h. ein auswärtiges großes
      Theater wird gegen Ende März
         ein anderes Stück\pwindex{Ehrlosen. Schauspiel in drei Acten@\emph{Die Ehrlosen. Schauspiel in drei Acten}|pwuv} von mir
         \label{K_L03724-2v}\edtext{aufführen}{\lemma{\textnormal{\emph{aufführen}}}\Cendnote{\textnormal{Vermutlich ist von dem Schauspiel \emph{Die Ehrlosen}\pwindex{Ehrlosen. Schauspiel in drei Acten@\emph{Die Ehrlosen. Schauspiel in drei Acten}|pwk} die Rede, das allerdings erst im Jahr darauf am 16. 3. 1901 am Volkstheater\oindex{Volkstheater@\textbf{Volkstheater}, \emph{Theater (K.THE)}|pwk} in Wien\oindex{Wien@\textbf{Wien}, \emph{A.ADM2}|pwk} uraufgeführt wurde.}}}\label{K_L03724-2} – und das weitere
      wird sich finden. Doch
      das ist Zukunftsmusik – .\pend
           
\pstart
           Für heute will ich Ihnen
      nur nochmals herzlich danken
      und schließlich noch \strikeout{B} bemerken, daß Sie ganz recht hatten
      bezüglich der Widmung! Ich
      hatte sie \introOben{}mit Bleistift\introOben{} auf das Titelblatt
         meines Conceptes\pwindex{erste Kapitel. Schauspiel in drei Akten@\emph{Das erste Kapitel. Schauspiel in drei Akten}|pwv} ge\substVorne{}\textsuperscript{schrieben}\substDazwischen{}setzt\substHinten{} und
      mich, so oft ich mich zur {\pb}Arbeit setzte – daran »gestimmt«.
      Als es die Abschreiberin erhielt,
      vergaß ich ganz auf diese \introOben{}nur\introOben{} zu
      meinem persönlichen Gebrauch
      dienenden Zeilen. So sind sie
      auf die zwei Abschriften übergegangen – die natürlich nicht
      für die Öffentlichkeit bestimmt
      sind – geschweige erst für Herrn
      »Fery Derffler\pwindex{Derfler, Ferdinand 1836/1837 – 1902-10-02@\textsc{Derfler, Ferdinand} (1836/1837 – 1902-10-02), \emph{Fotograf/Fotografin, Versicherungsbeamter/Versicherungsbeamtin, Sachbuchautor/Sachbuchautorin}|pwu}«. – Auch ich
      liebe keine Intimitäten mit
      dem Publikum. Ich bitte
      Sie also, mich einer \uline{solchen}
      Geschmacklosigkeit doch nicht
      für fähig zu halten – so
      viele andere ich auch auf dem
      Gewissen haben möge.\pend
           
\pstart
           Mit alter Verehrung{\\[\baselineskip]}\spacefill\mbox{Elsa Plessner}.
                  \pend
           \leftskip=0em{}\selectlanguage{ngerman}\endnumbering\briefempfaengerindex{Schnitzler, Arthur@\textsc{Schnitzler, Arthur}!zzzPlessner, Elsa@\emph{von Elsa Plessner}!1900-01-101@{10. 1. 1900}|)be}\mylabel{L03724h}
\begin{anhang}
\end{anhang}\normalsize

\doendnotes{C}
\bigskip
\vfill

\clearpage

\footnotesize

\lohead{\textsc{register}}

% Definiere theindex-Environment komplett neu ohne reledmac
\makeatletter
\renewenvironment{theindex}{%
  \section*{\indexname}%
  \setlength{\parindent}{0pt}%
  \setlength{\parskip}{0pt plus 0.3pt}%
  \let\item\@idxitem
}{%
  \clearpage
}
\makeatother

\IfFileExists{\jobname-pw.ind}{\input{\jobname-pw.ind}}{}

\end{document}

      