%% latex-leseansicht-vorspann.tex
%% Vorspann für die Leseansicht.
%% Lädt die gemeinsame Datei latex-vorspann.tex mit nicht gesetztem Schalter.

\newif\ifkorrekturansicht
\korrekturansichtfalse

\input{../tex-inputs/latex-vorspann}


\section[Albert Ehrenstein an Arthur Schnitzler, 10. 2. 1910]{L01912 Albert Ehrenstein an Arthur Schnitzler, 10. 2. 1910}
\nopagebreak\mylabel{L01912v}
\rehead{ }\normalsize\beginnumbering\briefempfaengerindex{Schnitzler, Arthur@\textsc{Schnitzler, Arthur}!zzzEhrenstein, Albert@\emph{von Albert Ehrenstein}!1910-02-101@{10. 2. 1910}|(be}
\toendnotes[C]{\smallbreak\pagebreak[2]}
\correspDesc{Versand  durch Albert Ehrenstein am 10. 2. 1910 in Wien
\newline{}Erhalt  durch Arthur Schnitzler im Zeitraum [10. 2. 1910
                  – 14. 2. 1910?] in Wien}\toendnotes[C]{\smallbreak}
\Standort{CUL, Schnitzler, B 30.}
\physDesc{Brief, 1 Blatt, 2 Seiten, 1371 Zeichen
\newline{}Handschrift: schwarze Tinte, deutsche Kurrent
\newline{}Schnitzler: mit Bleistift beschriftet: »\textsc{Ehrenstein}« }
\buchAbdrucke{\weitereDrucke{Albert Ehrenstein: \emph{Briefe}. Herausgegeben von Hanni Mittelmann. München: \emph{Boer} 1989, S. 37 (Werke, 1).} }\toendnotes[C]{\smallbreak}
\pstart
           
\pstart
           {\pb}XVI. \textsc{Ottakringerstr.}
                        114\oindex{Wien@\textbf{Wien}!XVI., Ottakring@\textbf{XVI., Ottakring}!Ottakringer Straße@\textbf{Ottakringer Straße}, \emph{Straße}|pw}\oindex{Wien@\textbf{Wien}!XVII., Hernals@\textbf{XVII., Hernals}!Ottakringer Straße@\textbf{Ottakringer Straße}, \emph{Straße}|pw}.\pend
           
\pstart
           \raggedleft{}10{\\}II{\\}1910\pend
           \pend
           
\pstart{}Sehr geehrter Herr Doktor,\pend\vspace{0.5em}
\pstart
           geſtern endlich erhielt ich Antwort von Herrn Bie\pwindex{Bie, Oskar 9.\,2.\,1864 Breslau – 21.\,4.\,1938 Berlin@\textsc{Bie, Oskar} (9.\,2.\,1864 Breslau – 21.\,4.\,1938 Berlin), \emph{Schriftsteller, Journalist, Redakteur}|pw}, die ich beilege, da ich mich in deren Interpretation nicht{ }ſicher
               fühle. Ich weiß vor allem nicht, ob ich dem Schreiben entnehmen darf, »Tubutſch\pwindex{Ehrenstein, Albert 23.\,12.\,1886 Wien – 8.\,4.\,1950 New York City@\textsc{Ehrenstein, Albert} (23.\,12.\,1886 Wien – 8.\,4.\,1950 New York City), \emph{Schriftsteller}!Tubutsch@\strich\emph{Tubutsch}|pw}« werde – was mir den Fang eines
               Verlegers erleichtern würde – nach einer Umarbeitung rundſchau\pwindex{neue Rundschau@\emph{Die neue Rundschau}|pw}möglich{ }ſein. Das wäre mir am liebſten, Denn eſſayiſtiſch habe ich
               mich noch nicht recht verſucht, das Wien\oindex{Wien@\textbf{Wien}, \emph{Verwaltungsgebiet}|pw}er Leben
               iſt mir unbekannt und was Herr Bie\pwindex{Bie, Oskar 9.\,2.\,1864 Breslau – 21.\,4.\,1938 Berlin@\textsc{Bie, Oskar} (9.\,2.\,1864 Breslau – 21.\,4.\,1938 Berlin), \emph{Schriftsteller, Journalist, Redakteur}|pw} unter einem
               netten Thema verſteht (er meint wohl{ }ſo etwas wie die \label{K_L01912-1v}\edtext{Hofrichter\pwindex{Hofrichter, Adolf 20.\,1.\,1880 Rychnov nad Kněžnou – 29.\,12.\,1945 Wien@\textsc{Hofrichter, Adolf} (20.\,1.\,1880 Rychnov nad Kněžnou – 29.\,12.\,1945 Wien), \emph{Militär}|pw}}{\lemma{\textnormal{\emph{Hofrichter}}}\Cendnote{\textnormal{Adolf Hofrichter\pwindex{Hofrichter, Adolf 20.\,1.\,1880 Rychnov nad Kněžnou – 29.\,12.\,1945 Wien@\textsc{Hofrichter, Adolf} (20.\,1.\,1880 Rychnov nad Kněžnou – 29.\,12.\,1945 Wien), \emph{Militär}|pwk} wurde im Frühjahr der
                  Prozess gemacht. Ihm wurde vorgeworfen, als Aphrodisiakum getarnte Zyankalikapseln
                  an höherrangige Militärs geschickt zu haben, um für seine Beförderung Platz zu
                  machen. Da es bis zum Geständnis ein Indizienverfahren war, fand der Prozess unter
                  reger Anteilnahme der Öffentlichkeit statt.}}}\label{K_L01912-1}- oder \label{K_L01912-2v}\edtext{Borowska\pwindex{Borowska, Janina @\textsc{Borowska, Janina}, \emph{Medizinstudentin}|pw}affaire}{\lemma{\textnormal{\emph{Borowskaaffaire}}}\Cendnote{\textnormal{Janina Borowska\pwindex{Borowska, Janina @\textsc{Borowska, Janina}, \emph{Medizinstudentin}|pwk} wurde 1909 von
                  dem Vorwurf freigesprochen, eine Spionin zu sein. Während des Prozesses begannen
                  sie und ihr Anwalt eine Affäre, die dieser nach einiger Zeit lösen wollte. Am
                     5. 6. 1909 wurde er tot in seinem Bett gefunden, neben ihm Borowska\pwindex{Borowska, Janina @\textsc{Borowska, Janina}, \emph{Medizinstudentin}|pwk}. Im folgenden Prozess gelang es
                  nicht, den von ihr behaupteten Suizid zu wiederlegen und sie wurde am
                     10. 10. 1910 in Krakau\oindex{Krakau@\textbf{Krakau}, \emph{Verwaltungsgebiet}|pwk}
                  freigesprochen.}}}\label{K_L01912-2}) hat auf mich bei meiner Gefühlsſtumpfheit kaum je einen zu
               druckfähiger Meinungsäußerung {\pb}drängenden
               Eindruck gemacht. Gern aber würde ich mich z. B. Schroeder\pwindex{Schröder, Rudolf Alexander 26.\,1.\,1878 Bremen – 22.\,8.\,1962 Bad Wiessee@\textsc{Schröder, Rudolf Alexander} (26.\,1.\,1878 Bremen – 22.\,8.\,1962 Bad Wiessee), \emph{Schriftsteller}|pw}’s \label{K_L01912-3v}\edtext{Homer\pwindex{Homer @\textsc{Homer}, \emph{Schriftsteller}|pw}überſetzung\pwindex{Homer @\textsc{Homer}, \emph{Schriftsteller}!Odyssee@\strich\emph{Odyssee}|pwv}}{\lemma{\textnormal{\emph{Homerübersetzung}}}\Cendnote{\textnormal{\emph{Die Odyssee}\pwindex{Homer @\textsc{Homer}, \emph{Schriftsteller}!Odyssee@\strich\emph{Odyssee}|pwk}. Neu ins Deutsche übertragen
                     von Rudolf Alexander Schröder\pwindex{Schröder, Rudolf Alexander 26.\,1.\,1878 Bremen – 22.\,8.\,1962 Bad Wiessee@\textsc{Schröder, Rudolf Alexander} (26.\,1.\,1878 Bremen – 22.\,8.\,1962 Bad Wiessee), \emph{Schriftsteller}|pwk}. Gedruckt
                     in 425 Exemplaren. Leipzig: \emph{Insel}\orgindex{Insel Verlag@Insel Verlag|pwk}{ }1910.}}}\label{K_L01912-3} befaſſen, wenn mir das Buch dieſes exkluſiven Autors zugänglich
               wäre. Vielleicht können Sie, hochverehrter Herr Doktor, mir raten und zugleich mir
               eine zweite Frage beantworten, die mich{ }ſehr intereſſiert. Wann nämlich der junge Herr Medardus\pwindex{Schnitzler, Arthur 15.\,5.\,1862 Wien – 21.\,10.\,1931 ebd.@\textsc{Schnitzler, Arthur} (15.\,5.\,1862 Wien – 21.\,10.\,1931 ebd.), \emph{Schriftsteller, Mediziner}!junge Medardus. Dramatische Historie in einem Vorspiel und fünf Aufzügen@\strich\emph{Der junge Medardus. Dramatische Historie in einem Vorspiel und fünf Aufzügen}|pw} urſprünglich im Buchhandel
               hätte erſcheinen{ }ſollen, wenn er nicht (um die Zeit Ihrer \label{K_L01912-4v}\edtext{Volkstheater\orgindex{Volkstheater@Volkstheater|pw}premiere\eventindex{Burgtheater@\textbf{Burgtheater}!Uraufführung von Der junge Medardus, 24.11.1910@Uraufführung von Der junge Medardus, 24.11.1910|pw}}{\lemma{\textnormal{\emph{Volkstheaterpremiere}}}\Cendnote{\textnormal{Es handelt sich um eine Verwechslung Ehrensteins\pwindex{Ehrenstein, Albert 23.\,12.\,1886 Wien – 8.\,4.\,1950 New York City@\textsc{Ehrenstein, Albert} (23.\,12.\,1886 Wien – 8.\,4.\,1950 New York City), \emph{Schriftsteller}|pwk}. Die Uraufführung\eventindex{Burgtheater@\textbf{Burgtheater}!Uraufführung von Der junge Medardus, 24.11.1910@Uraufführung von Der junge Medardus, 24.11.1910|pwkv} von \emph{Der
                     junge Herr Medardus}\pwindex{Schnitzler, Arthur 15.\,5.\,1862 Wien – 21.\,10.\,1931 ebd.@\textsc{Schnitzler, Arthur} (15.\,5.\,1862 Wien – 21.\,10.\,1931 ebd.), \emph{Schriftsteller, Mediziner}!junge Medardus. Dramatische Historie in einem Vorspiel und fünf Aufzügen@\strich\emph{Der junge Medardus. Dramatische Historie in einem Vorspiel und fünf Aufzügen}|pwk} war immer für das \emph{Burgtheater}\orgindex{Burgtheater@Burgtheater|pwk} geplant und fand an diesem Theater am 24. 11. 1910
                  statt.}}}\label{K_L01912-4}?) zurückgezogen worden wäre?\pend
           
\pstart
           Indem ich herzlichſt für Ihre Empfehlung danke, die,{ }ſcheint es, diesmal doch zu
               einem für das deutſche Schrifttum erfreulichen Resultate\strikeout{n} führen dürfte, bin ich mit den beſten Grüßen\pend
           
\pstart
           Hochachtungsvoll{\\[\baselineskip]}Ihr ergebenſter{\\[\baselineskip]}\spacefill\mbox{Albert Ehrenstein.}\pend
           \leftskip=0em{}\selectlanguage{ngerman}\endnumbering\briefempfaengerindex{Schnitzler, Arthur@\textsc{Schnitzler, Arthur}!zzzEhrenstein, Albert@\emph{von Albert Ehrenstein}!1910-02-101@{10. 2. 1910}|)be}\mylabel{L01912h}  \newcommand{\dateiname}{L01912}\newcommand{\titel}{Albert Ehrenstein an Arthur Schnitzler, 10. 2. 1910}\newcommand{\editorInnen}{Martin Anton Müller und Gerd-Hermann Susen}%% latex-leseansicht-abspann.tex
%% Abspann für die Leseansicht.
%% Der Schalter \ifkorrekturansicht ist bereits durch den Vorspann gesetzt.

%% latex-abspann.tex
%% Gemeinsamer Abspann für Korrekturansicht und Leseansicht.
%% Setzt den Schalter \ifkorrekturansicht voraus (gesetzt in den
%% einbindenden Dateien latex-korrekturansicht-abspann.tex bzw.
%% latex-leseansicht-abspann.tex).
%% ---------------------------------------------------------------

\normalsize

% Das esempio-Environment wird nur in der Leseansicht benötigt
\ifkorrekturansicht\else
\newenvironment{esempio}[3]%
{
    \vspace{1.5ex}
    \rlap{\underline{#1}}
    \par
    \setlength{\parindent}{0cm}
    \nopagebreak
    \leftskip=#2cm
    \rightskip=#3cm
}
{
    \par
}
\fi

\doendnotes{C}
\bigskip
\vfill

\clearpage

\footnotesize

\ifkorrekturansicht
  \lohead{\textsc{register}}
\fi

% theindex-Environment neu definieren ohne reledmac
\makeatletter
\renewenvironment{theindex}{%
  \ifkorrekturansicht
    \section*{\indexname}%
  \else
    \subsubsection*{Index der erwähnten Entitäten}%
  \fi
  \setlength{\parindent}{0pt}%
  \setlength{\parskip}{0pt plus 0.3pt}%
  \let\item\@idxitem
}{%
  \ifkorrekturansicht\clearpage\fi
}
\makeatother

\IfFileExists{\jobname-pw.ind}{\input{\jobname-pw.ind}}{}

% Quellenangabe nur in der Leseansicht
\ifkorrekturansicht\else
% Fallback-Definitionen, falls die .tex-Datei \titel etc. nicht gesetzt hat
\providecommand{\titel}{}
\providecommand{\editorInnen}{}
\providecommand{\dateiname}{\jobname}

\vspace{3cm}

\vfill

\footnotesize
\textsc{Quelle}: \titel. Herausgegeben von {\editorInnen}. In: \emph{Arthur Schnitzler: Briefwechsel mit Autorinnen und Autoren}.
 Digitale Edition, https://schnitzler-briefe.acdh.oeaw.ac.at/{\dateiname}.html (Stand \today)
\fi

\end{document}


