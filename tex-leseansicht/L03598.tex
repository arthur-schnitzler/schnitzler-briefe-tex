%% latex-korrekturansicht-vorspann.tex
%% Vorspann für die Korrekturansicht.
%% Lädt die gemeinsame Datei latex-vorspann.tex mit gesetztem Schalter.

\newif\ifkorrekturansicht
\korrekturansichttrue

\input{../tex-inputs/latex-vorspann}


\section[ Arthur Schnitzler: Widmungsexemplar Sterben für Felix Salten, 8. 12. 1897]{L03598 Arthur Schnitzler: Widmungsexemplar Sterben für Felix
               Salten, 8. 12. 1897}
\nopagebreak\mylabel{L03598v}
\rehead{ }\normalsize\beginnumbering\briefempfaengerindex{Salten, Felix@\textsc{Salten, Felix}!zzzSchnitzler, Arthur@\emph{von Arthur Schnitzler}!1897-12-082@{8. 12. 1897}|(be}
\toendnotes[C]{\smallbreak\pagebreak[2]}\Standort{Wienbibliothek im Rathaus, A-32897/2.Ex., DS-2018-9494.}
\physDesc{Widmung am Schmutztitel, 61 Zeichen
\newline{}Handschrift: schwarze Tinte, deutsche Kurrent
\newline{}Ordnung: 1) mit schwarzer Tinte am Titelblatt gestrichene Regalerfassung: »\noindent{}IN\textsuperscript{o} 2198 WN\textsuperscript{o} 1288{ / }XI b«  2) mit schwarzer Tinte ausgefüllter Stempel: »\noindent{}\textcolor{gray}{\textbf{\textit{Felix Salten}}}{ / }\textcolor{gray}{\textbf{\textit{Inv. Nr.}}}{ }4467{ / }\textcolor{gray}{\textbf{\textit{Werk Nr.}}}{ }2195{ / }\textcolor{gray}{\textbf{\textit{Schrank}}}{ }XIV A. Z. \textcolor{gray}{\textbf{\textit{Fach}}} b« 
\newline{}Zusatz: mit Bleistift von unbekannter Hand oberhalb des Vorsatzblatts nicht verlässlich zu
                                 entziffernder, beschnittener Vermerk: »12 \textcolor{gray}{×}\-\textcolor{gray}{×}\-\textcolor{gray}{×}\-\textcolor{gray}{×}« }
\pstart
           \noindent{}{\pb}Meinem lieben Felix Salten\pend
           
\pstart
           herzlichſt{\\[\baselineskip]}\spacefill\mbox{ArthSch}\pend
           \leftskip=0em{}
\pstart
           Wien\oindex{Wien@\textbf{Wien}, \emph{A.ADM2}|pw}, 8. 12. 97.\pend
           {\vspace{1\baselineskip}}
\pstart
           \centering{}\textcolor{gray}{\textbf{Sterben\pwindex{Sterben. Novelle@\emph{Sterben. Novelle}|pw}.
               }}\pend
           \selectlanguage{ngerman}\vspace{1em}{\vspace{1\baselineskip}}
\pstart
           \centering{}{\pb}\textcolor{gray}{\textbf{\textbf{ARTHUR SCHNITZLER}}}\pend
           
\pstart
           \centering{}\textcolor{gray}{\textbf{\textbf{Sterben}\pwindex{Sterben. Novelle@\emph{Sterben. Novelle}|pw}}}\pend
           
\pstart
           \centering{}\textcolor{gray}{\textbf{\textbf{Novelle}}}\pend
           
\pstart
           \centering{}\textcolor{gray}{\textbf{\textbf{Zweite Auflage}}}\pend
           {\vspace{1\baselineskip}}
\pstart
           \centering{}\textcolor{gray}{\textbf{\textbf{Berlin}\oindex{Berlin@\textbf{Berlin}, \emph{P.PPLC}|pw}}}\pend
           
\pstart
           \centering{}\textcolor{gray}{\textbf{\textbf{\so{S. Fischer, Verlag}}\orgindex{S. Fischer Verlag@S. Fischer Verlag|pw}}}\pend
           
\pstart
           \centering{}\textcolor{gray}{\textbf{1898.}}\pend
           \selectlanguage{ngerman}\endnumbering\briefempfaengerindex{Salten, Felix@\textsc{Salten, Felix}!zzzSchnitzler, Arthur@\emph{von Arthur Schnitzler}!1897-12-082@{8. 12. 1897}|)be}\mylabel{L03598h}  \normalsize

\doendnotes{C}
\bigskip
\vfill

\clearpage

\footnotesize

\lohead{\textsc{register}}

% Definiere theindex-Environment komplett neu ohne reledmac
\makeatletter
\renewenvironment{theindex}{%
  \section*{\indexname}%
  \setlength{\parindent}{0pt}%
  \setlength{\parskip}{0pt plus 0.3pt}%
  \let\item\@idxitem
}{%
  \clearpage
}
\makeatother

\IfFileExists{\jobname-pw.ind}{\input{\jobname-pw.ind}}{}

\end{document}

      