%% latex-leseansicht-vorspann.tex
%% Vorspann für die Leseansicht.
%% Lädt die gemeinsame Datei latex-vorspann.tex mit nicht gesetztem Schalter.

\newif\ifkorrekturansicht
\korrekturansichtfalse

\input{../tex-inputs/latex-vorspann}


\section[Arthur Schnitzler an Berta Zuckerkandl, 26. 1. 1924]{L03952 Arthur Schnitzler an Berta Zuckerkandl, 26. 1. 1924}
\nopagebreak\mylabel{L03952v}
\rehead{ }\normalsize\beginnumbering\briefempfaengerindex{Zuckerkandl, Berta@\textsc{Zuckerkandl, Berta}!zzzSchnitzler, Arthur@\emph{von Arthur Schnitzler}!1924-01-261@{26. 1. 1924}|(be}
\toendnotes[C]{\smallbreak\pagebreak[2]}
\correspDesc{Versand  durch Arthur Schnitzler am 26. 1. 1924 in Wien
\newline{}Erhalt  durch Berta Zuckerkandl im Zeitraum [17. 1. 1924 – 21. 1. 1924?] in Paris}\toendnotes[C]{\smallbreak}
\Standort{DLA, HS.1985.1.2282.}
\physDesc{Brief, Durchschlag, 1 Blatt, 1 Seite, 1246 Zeichen
\newline{}Schreibmaschine
\newline{}Handschrift: roter Buntstift, lateinische Kurrent (\noindent{}beschriftet: »\uline{Zuckerkandl}«, sechs Unterstreichungen)}\toendnotes[C]{\smallbreak}
\pstart
           \raggedleft{}{\pb}26. 1. 1924.\pend
           
\pstart{}Liebe und verehrte Frau Hofrätin.\pend\vspace{0.5em}
\pstart
           Ich hatte an Stock\orgindex{Éditions Stock@Éditions Stock|pw} nur deshalb
               \label{K_L03952-1v}\edtext{nochmals geschrieben}{\lemma{\textnormal{\emph{nochmals geschrieben}}}\Cendnote{\textnormal{Arthur Schnitzler an Jacques Boutelleau\pwindex{Chardonne, Jacques 2.\,1.\,1884 Barbezieux-Saint-Hilaire – 29.\,5.\,1968 La Frette-sur-Seine@\textsc{Chardonne, Jacques} (2.\,1.\,1884 Barbezieux-Saint-Hilaire – 29.\,5.\,1968 La Frette-sur-Seine), \emph{Schriftsteller, Verleger}|pwk}, 17. 1. 1924, \emph{Deutsches Literaturarchiv Marbach}, HS.1985.1.1297. Schnitzler bat um Auskunft, welche seiner Texte der Verlag\orgindex{Éditions Stock@Éditions Stock|pwkv} publizieren werde, um deren Übersetzungsrechte nicht anderweitig zu vergeben.}}}\label{K_L03952-1}, weil ein neuer, allerdings etwas vager \label{K_L03952-2v}\edtext{Antrag}{\lemma{\textnormal{\emph{Antrag}}}\Cendnote{\textnormal{nicht überliefert}}}\label{K_L03952-2} von einem Herrn
               Nicolaus Nathan\pwindex{Nathan, Nicolas @\textsc{Nathan, Nicolas}, \emph{Übersetzer}|pw} (Paris, Hotel Ronceray\oindex{Paris@\textbf{Paris}, \emph{Hauptstadt}|pw}) an
      mich gelangt war, den ich natürlich ganz
               \label{K_L03952-3v}\edtext{dilatorisch}{\lemma{\textnormal{\emph{dilatorisch}}}\Cendnote{\textnormal{verzögernd, hinhaltend}}}\label{K_L03952-3}{ }\label{K_L03952-4v}\edtext{behandelte}{\lemma{\textnormal{\emph{behandelte}}}\Cendnote{\textnormal{Arthur Schnitzler an Nicolaus Nathan\pwindex{Nathan, Nicolas @\textsc{Nathan, Nicolas}, \emph{Übersetzer}|pwk}, 17. 1. 1924, \emph{Deutsches Literaturarchiv Marbach}, HS.1985.1.1485. Schnitzler schrieb: »doch möchte ich jedenfalls Stocks\orgindex{Éditions Stock@Éditions Stock|pw} Antwort abwarten, ehe ich auf Ihr freundliches Angebot eingehe«.}}}\label{K_L03952-4}; ich wollte sie damit nicht während Ihres Pariser\oindex{Paris@\textbf{Paris}, \emph{Hauptstadt}|pw} Aufenthalts
               bemühen. Was Sie mir über Herrn \label{K_L03952-5v}\edtext{L. P.}{\lemma{\textnormal{\emph{L. P.}}}\Cendnote{\textnormal{Möglicherweise ist der Regisseur Aurélien Lugné-Poe\pwindex{Lugné-Poe, Aurélien-Marie 27.\,12.\,1869 Paris – 19.\,6.\,1940 Villeneuve-les-Avignon@\textsc{Lugné-Poe, Aurélien-Marie} (27.\,12.\,1869 Paris – 19.\,6.\,1940 Villeneuve-les-Avignon), \emph{Theaterleiter, Regisseur, Schauspieler}|pwk} gemeint.}}}\label{K_L03952-5} einerseits und Herrn Boutelleau\pwindex{Chardonne, Jacques 2.\,1.\,1884 Barbezieux-Saint-Hilaire – 29.\,5.\,1968 La Frette-sur-Seine@\textsc{Chardonne, Jacques} (2.\,1.\,1884 Barbezieux-Saint-Hilaire – 29.\,5.\,1968 La Frette-sur-Seine), \emph{Schriftsteller, Verleger}|pw} anderseits
               \label{K_L03952-6v}\edtext{schreiben}{\lemma{\textnormal{\emph{schreiben}}}\Cendnote{\textnormal{Zuckerkandls\pwindex{Zuckerkandl, Berta 13.\,4.\,1864 Wien – 16.\,10.\,1945 Paris@\textsc{Zuckerkandl, Berta} (13.\,4.\,1864 Wien – 16.\,10.\,1945 Paris), \emph{Schriftstellerin, Journalistin, Übersetzerin}|pwk} Brief ist nicht überliefert.}}}\label{K_L03952-6}, ist zwar nicht sehr erfreulich, aber
      Sie werden nicht erwarten, dass ich mich sehr
      überrascht zeige. \label{K_L03952-7v}\edtext{\begin{otherlanguage}{french}Nous avons vu d’autres!\end{otherlanguage}}{\lemma{\textnormal{\emph{Nous avons vu d’autres!}}}\Cendnote{\textnormal{französisch: Wir haben schon andere gesehen!}}}\label{K_L03952-7}\pend
           
\pstart
           Dass
               Paul Geraldy\pwindex{Géraldy, Paul 6.\,3.\,1885 Paris – 9.\,3.\,1983 Neuilly-sur-Seine@\textsc{Géraldy, Paul} (6.\,3.\,1885 Paris – 9.\,3.\,1983 Neuilly-sur-Seine), \emph{Schriftsteller}|pw} sich meiner Angelegenheit
               und insbesondere des »Casanova\pwindex{Schnitzler, Arthur 15. 5. 1862 Wien – 21. 10. 1931 ebd.@\textsc{Schnitzler, Arthur} (15. 5. 1862 Wien – 21. 10. 1931 ebd.), \emph{Schriftsteller, Mediziner}!Casanovas Heimfahrt@\strich\emph{Casanovas Heimfahrt}|pwv}« so warm annimmt, danke ich ihm herzlichst. Ueber die
      Uebersetzungsfrage habe ich ja hoffentlich
      bald \label{K_L03952-8v}\edtext{Gelegenheit}{\lemma{\textnormal{\emph{Gelegenheit}}}\Cendnote{\textnormal{Vgl. A. S.: \emph{Tagebuch}, 15. 2. 1924.}}}\label{K_L03952-8} persönlich mit Ihnen, liebe und verehrte Freundin, zu reden.\pend
           
\pstart
           Indess habe ich hier das charmante Vater und Sohn-Stück\pwindex{Guitry, Sacha 21.\,1.\,1885 Sankt Petersburg – 24.\,7.\,1957 Paris@\textsc{Guitry, Sacha} (21.\,1.\,1885 Sankt Petersburg – 24.\,7.\,1957 Paris), \emph{Schriftsteller, Regisseur, Schauspieler}!Mein Vater hat recht gehabt. Komödie in drei Akten@\strich\emph{Mein Vater hat recht gehabt. Komödie in drei Akten}|pwv} von Sacha Guitry\pwindex{Guitry, Sacha 21.\,1.\,1885 Sankt Petersburg – 24.\,7.\,1957 Paris@\textsc{Guitry, Sacha} (21.\,1.\,1885 Sankt Petersburg – 24.\,7.\,1957 Paris), \emph{Schriftsteller, Regisseur, Schauspieler}|pw} mit
               den beiden Thiemigs\pwindex{Thimig, Hugo 16.\,6.\,1854 Dresden – 24.\,9.\,1944 Wien@\textsc{Thimig, Hugo} (16.\,6.\,1854 Dresden – 24.\,9.\,1944 Wien), \emph{Theaterleiter, Schauspieler}|pw}\pwindex{Thimig, Hermann 3.\,10.\,1890 Wien – 7.\,7.\,1982 ebd.@\textsc{Thimig, Hermann} (3.\,10.\,1890 Wien – 7.\,7.\,1982 ebd.), \emph{Schauspieler}|pw}{ }\label{K_L03952-9v}\edtext{gesehen\eventindex{Kammerspiele Wien@\textbf{Kammerspiele Wien}!Aufführung von Mein Vater hat recht gehabt, 16.1.1924@Aufführung von Mein Vater hat recht gehabt, 16.1.1924|pwv}}{\lemma{\textnormal{\emph{gesehen}}}\Cendnote{\textnormal{Vgl. A. S.: \emph{Kulturveranstaltungen}, 16. 1. 1924. Zuckerkandl\pwindex{Zuckerkandl, Berta 13.\,4.\,1864 Wien – 16.\,10.\,1945 Paris@\textsc{Zuckerkandl, Berta} (13.\,4.\,1864 Wien – 16.\,10.\,1945 Paris), \emph{Schriftstellerin, Journalistin, Übersetzerin}|pwk} war die Übersetzerin der Komödie \emph{Mein Vater hat recht gehabt}\pwindex{Guitry, Sacha 21.\,1.\,1885 Sankt Petersburg – 24.\,7.\,1957 Paris@\textsc{Guitry, Sacha} (21.\,1.\,1885 Sankt Petersburg – 24.\,7.\,1957 Paris), \emph{Schriftsteller, Regisseur, Schauspieler}!Mein Vater hat recht gehabt. Komödie in drei Akten@\strich\emph{Mein Vater hat recht gehabt. Komödie in drei Akten}|pwk}.}}}\label{K_L03952-9}, das, wie sie jedenfalls wissen, noch immer sehr volle
               Häuser macht. Die beiden Thimigs\pwindex{Thimig, Hugo 16.\,6.\,1854 Dresden – 24.\,9.\,1944 Wien@\textsc{Thimig, Hugo} (16.\,6.\,1854 Dresden – 24.\,9.\,1944 Wien), \emph{Theaterleiter, Schauspieler}|pw}\pwindex{Thimig, Hermann 3.\,10.\,1890 Wien – 7.\,7.\,1982 ebd.@\textsc{Thimig, Hermann} (3.\,10.\,1890 Wien – 7.\,7.\,1982 ebd.), \emph{Schauspieler}|pw} sind wirklich entzückend, das weibliche Element ist
      mässiger vertreten. Sie bringen wieder hoffentlich manches Schöne aus Paris\oindex{Paris@\textbf{Paris}, \emph{Hauptstadt}|pw} mit.\pend
           
\pstart
           Ueber alles, was mich betrifft,
      Literarisches und Menschliches erzähle ich
      Ihnen mündlich.\pend
           
\pstart
           Alles Herzliche und auf ein{\\[\baselineskip]} gutes Wiedersehen.{\\[\baselineskip]} Ihr dankbar ergebener\pend
           \leftskip=0em{}\selectlanguage{ngerman}\endnumbering\briefempfaengerindex{Zuckerkandl, Berta@\textsc{Zuckerkandl, Berta}!zzzSchnitzler, Arthur@\emph{von Arthur Schnitzler}!1924-01-261@{26. 1. 1924}|)be}\mylabel{L03952h}
\begin{anhang}
\end{anhang}\newcommand{\dateiname}{L03952}\newcommand{\titel}{Arthur Schnitzler an Berta Zuckerkandl, 26. 1. 1924}\newcommand{\editorInnen}{Herausgegeben von Jahnke, SelmaMüller, Martin Anton}%% latex-leseansicht-abspann.tex
%% Abspann für die Leseansicht.
%% Der Schalter \ifkorrekturansicht ist bereits durch den Vorspann gesetzt.

%% latex-abspann.tex
%% Gemeinsamer Abspann für Korrekturansicht und Leseansicht.
%% Setzt den Schalter \ifkorrekturansicht voraus (gesetzt in den
%% einbindenden Dateien latex-korrekturansicht-abspann.tex bzw.
%% latex-leseansicht-abspann.tex).
%% ---------------------------------------------------------------

\normalsize

% Das esempio-Environment wird nur in der Leseansicht benötigt
\ifkorrekturansicht\else
\newenvironment{esempio}[3]%
{
    \vspace{1.5ex}
    \rlap{\underline{#1}}
    \par
    \setlength{\parindent}{0cm}
    \nopagebreak
    \leftskip=#2cm
    \rightskip=#3cm
}
{
    \par
}
\fi

\doendnotes{C}
\bigskip
\vfill

\clearpage

\footnotesize

\ifkorrekturansicht
  \lohead{\textsc{register}}
\fi

% theindex-Environment neu definieren ohne reledmac
\makeatletter
\renewenvironment{theindex}{%
  \ifkorrekturansicht
    \section*{\indexname}%
  \else
    \subsubsection*{Index der erwähnten Entitäten}%
  \fi
  \setlength{\parindent}{0pt}%
  \setlength{\parskip}{0pt plus 0.3pt}%
  \let\item\@idxitem
}{%
  \ifkorrekturansicht\clearpage\fi
}
\makeatother

\IfFileExists{\jobname-pw.ind}{\input{\jobname-pw.ind}}{}

% Quellenangabe nur in der Leseansicht
\ifkorrekturansicht\else
% Fallback-Definitionen, falls die .tex-Datei \titel etc. nicht gesetzt hat
\providecommand{\titel}{}
\providecommand{\editorInnen}{}
\providecommand{\dateiname}{\jobname}

\vspace{3cm}

\vfill

\footnotesize
\textsc{Quelle}: \titel. Herausgegeben von {\editorInnen}. In: \emph{Arthur Schnitzler: Briefwechsel mit Autorinnen und Autoren}.
 Digitale Edition, https://schnitzler-briefe.acdh.oeaw.ac.at/{\dateiname}.html (Stand \today)
\fi

\end{document}


