%% latex-leseansicht-vorspann.tex
%% Vorspann für die Leseansicht.
%% Lädt die gemeinsame Datei latex-vorspann.tex mit nicht gesetztem Schalter.

\newif\ifkorrekturansicht
\korrekturansichtfalse

\input{../tex-inputs/latex-vorspann}


\section[Arthur Schnitzler an Hugo von Hofmannsthal, 22. 8. 1916]{L02238 Arthur Schnitzler an Hugo von Hofmannsthal, 22. 8. 1916}
\nopagebreak\mylabel{L02238v}
\rehead{ }\normalsize\beginnumbering\briefempfaengerindex{Hofmannsthal, Hugo von@\textsc{Hofmannsthal, Hugo von}!zzzSchnitzler, Arthur@\emph{von Arthur Schnitzler}!1916-08-221@{22. 8. 1916}|(be}
\toendnotes[C]{\smallbreak\pagebreak[2]}
\correspDesc{Versand  durch Arthur Schnitzler am 22. 8. 1916 in Altaussee
\newline{}Erhalt  durch Hugo von Hofmannsthal im Zeitraum [23. 8. 1916
                  – 27. 8. 1916?] in Wien}\toendnotes[C]{\smallbreak}
\Standort{FDH, Hs-30885,4.}
\physDesc{Brief, 1 Blatt, 2 Seiten, 1782 Zeichen
\newline{}Schreibmaschine
\newline{}Handschrift: schwarze Tinte, lateinische Kurrent (\noindent{}Einfügung, Unterschrift)
\newline{}Zusatz: Eine Fotokopie findet sich unter der Signatur
                                 »Hs-30885,148a«. }
\buchAbdrucke{\weitereDrucke{Hugo von Hofmannsthal, Arthur Schnitzler: \emph{Briefwechsel}. Herausgegeben von Therese Nickl und Heinrich Schnitzler. Frankfurt am Main: \emph{S. Fischer} 1964, S. 279.} }\toendnotes[C]{\smallbreak}
\pstart
           {\pb}\textcolor{gray}{\textbf{Dr. Arthur Schnitzler}}\hfill Alt-Aussee\oindex{Altaussee@\textbf{Altaussee}, \emph{Verwaltungsgebiet}|pw},
                     22. 8. 1916.\pend
           
\pstart
           \textcolor{gray}{\textbf{Wien XVIII. Sternwartestrasse 71\oindex{Wien@\textbf{Wien}!XVIII., Währing@\textbf{XVIII., Währing}!Sternwartestraße 71@\textbf{Sternwartestraße 71}, \emph{Wohngebäude}|pw}}}\pend
           
\pstart\center{}Lieber Hugo.\pend\vspace{0.5em}
\pstart
           »Der letzte Tanz\pwindex{Ehrhart-Ehrhartstein, Robert 12.\,9.\,1870 Innsbruck – 11.\,11.\,1956 Baden bei Wien@\textsc{Ehrhart-Ehrhartstein, Robert} (12.\,9.\,1870 Innsbruck – 11.\,11.\,1956 Baden bei Wien), \emph{Schriftsteller, Ministerialbeamter}!letzte Tanz@\strich\emph{Der letzte Tanz}|pw}« ist eine sehr anmutig,
               vielleicht manchmal zu ausführlich erzählte Geschichte, in der ein zartes Seelchen
               von Amadeus Hofmann\pwindex{Hoffmann, Ernst Theodor Amadeus 24.\,1.\,1776 Kaliningrad – 25.\,6.\,1822 Berlin@\textsc{Hoffmann, Ernst Theodor Amadeus} (24.\,1.\,1776 Kaliningrad – 25.\,6.\,1822 Berlin), \emph{Schriftsteller, Komponist, Zeichner}|pw} steckt und um die eine
               reinliche Atmosphäre von Saar\pwindex{Saar, Ferdinand von 30.\,9.\,1833 Wien – 24.\,7.\,1906 ebd.@\textsc{Saar, Ferdinand von} (30.\,9.\,1833 Wien – 24.\,7.\,1906 ebd.), \emph{Schriftsteller}|pw} und Stifter\pwindex{Stifter, Adalbert 23.\,10.\,1805 Horní Planá – 28.\,1.\,1868 Linz@\textsc{Stifter, Adalbert} (23.\,10.\,1805 Horní Planá – 28.\,1.\,1868 Linz), \emph{Schriftsteller}|pw} schwebt. Sie schienen gewisse Bedenken
               hinsichtlich dessen, zu hegen, was Ihnen wie ein Rahmen erscheint. Aber Rahmen und
               Bild sind ja hier durchaus eins, ja, der Rahmen ohne das Bild wäre so gut wie nichts
               und das Bild ohne den Rahmen nicht viel mehr. Dass die Minna\pwindex{Ehrhart-Ehrhartstein, Robert 12.\,9.\,1870 Innsbruck – 11.\,11.\,1956 Baden bei Wien@\textsc{Ehrhart-Ehrhartstein, Robert} (12.\,9.\,1870 Innsbruck – 11.\,11.\,1956 Baden bei Wien), \emph{Schriftsteller, Ministerialbeamter}!letzte Tanz@\strich\emph{Der letzte Tanz}|pwv} eigentlich das Aquarell und der alte
               Herr eigentlich die kleine Holzfigur vorstellt, macht ja den Reiz der Geschichte aus,
               der von Anfang bis zum Ende gleichmässig bescheiden fortwirkt, sich am stärksten in
               den sonderbaren Anweisungen des süssen und gelegentlich etwas süsslichen Mädel und in
               den Kunststücken des alten {\pb}Herrn erweist, (unter denen ich
               das mit dem abgehauenen falschen Kopf als in jedem Sinne aus dem Stil fallend lieber
               missen möchte) und der nur am Ende ein wenig nachlässt, weil man doch, ich will nicht
               sagen eine Pointe oder gar eine Lösung, – aber doch irgend einen Schlusseinfall
               erwartet hätte, der das Ganze in einer höheren Sphäre abschliessen sollte als dies
               die Erklärung des rationalistischen Willibald\pwindex{Ehrhart-Ehrhartstein, Robert 12.\,9.\,1870 Innsbruck – 11.\,11.\,1956 Baden bei Wien@\textsc{Ehrhart-Ehrhartstein, Robert} (12.\,9.\,1870 Innsbruck – 11.\,11.\,1956 Baden bei Wien), \emph{Schriftsteller, Ministerialbeamter}!letzte Tanz@\strich\emph{Der letzte Tanz}|pwv} vermag. Weiteren Arbeiten des Autors\pwindex{Ehrhart-Ehrhartstein, Robert 12.\,9.\,1870 Innsbruck – 11.\,11.\,1956 Baden bei Wien@\textsc{Ehrhart-Ehrhartstein, Robert} (12.\,9.\,1870 Innsbruck – 11.\,11.\,1956 Baden bei Wien), \emph{Schriftsteller, Ministerialbeamter}|pwv}, in dem ich vorläufig mehr Geschmack als Eigenart,
               mehr Kultur als Inspiration, mehr wohltuende Zärtlichkeit für Wien\oindex{Wien@\textbf{Wien}, \emph{Verwaltungsgebiet}|pw} als unmittelbar poetische Empfindung zu entdecken glaube,
               sehe ich mit umso günstigerem Vorurteil entgegen, als die Biedermeierei seines
               Vorwurfs \introOben{}sich\introOben{} nirgends in Affektation und die freundlichste
               Phantastik seines Stoffes kaum je sich ins Absurde verliert; – Versuchungen, denen
               vielleicht mancher künstlerisch stärkere Erzähler in solchem Fall unterlegen wäre. –
               Herzlichen Dank und Gruss\pend
           
\pstart
           Ihr{\\[\baselineskip]}\spacefill\mbox{{[}hs.:{]} Arthur}\pend
           \leftskip=0em{}\selectlanguage{ngerman}\endnumbering\briefempfaengerindex{Hofmannsthal, Hugo von@\textsc{Hofmannsthal, Hugo von}!zzzSchnitzler, Arthur@\emph{von Arthur Schnitzler}!1916-08-221@{22. 8. 1916}|)be}\mylabel{L02238h}  \newcommand{\dateiname}{L02238}\newcommand{\titel}{Arthur Schnitzler an Hugo von Hofmannsthal, 22. 8. 1916}\newcommand{\editorInnen}{Martin Anton Müller und Gerd-Hermann Susen}%% latex-leseansicht-abspann.tex
%% Abspann für die Leseansicht.
%% Der Schalter \ifkorrekturansicht ist bereits durch den Vorspann gesetzt.

%% latex-abspann.tex
%% Gemeinsamer Abspann für Korrekturansicht und Leseansicht.
%% Setzt den Schalter \ifkorrekturansicht voraus (gesetzt in den
%% einbindenden Dateien latex-korrekturansicht-abspann.tex bzw.
%% latex-leseansicht-abspann.tex).
%% ---------------------------------------------------------------

\normalsize

% Das esempio-Environment wird nur in der Leseansicht benötigt
\ifkorrekturansicht\else
\newenvironment{esempio}[3]%
{
    \vspace{1.5ex}
    \rlap{\underline{#1}}
    \par
    \setlength{\parindent}{0cm}
    \nopagebreak
    \leftskip=#2cm
    \rightskip=#3cm
}
{
    \par
}
\fi

\doendnotes{C}
\bigskip
\vfill

\clearpage

\footnotesize

\ifkorrekturansicht
  \lohead{\textsc{register}}
\fi

% theindex-Environment neu definieren ohne reledmac
\makeatletter
\renewenvironment{theindex}{%
  \ifkorrekturansicht
    \section*{\indexname}%
  \else
    \subsubsection*{Index der erwähnten Entitäten}%
  \fi
  \setlength{\parindent}{0pt}%
  \setlength{\parskip}{0pt plus 0.3pt}%
  \let\item\@idxitem
}{%
  \ifkorrekturansicht\clearpage\fi
}
\makeatother

\IfFileExists{\jobname-pw.ind}{\input{\jobname-pw.ind}}{}

% Quellenangabe nur in der Leseansicht
\ifkorrekturansicht\else
% Fallback-Definitionen, falls die .tex-Datei \titel etc. nicht gesetzt hat
\providecommand{\titel}{}
\providecommand{\editorInnen}{}
\providecommand{\dateiname}{\jobname}

\vspace{3cm}

\vfill

\footnotesize
\textsc{Quelle}: \titel. Herausgegeben von {\editorInnen}. In: \emph{Arthur Schnitzler: Briefwechsel mit Autorinnen und Autoren}.
 Digitale Edition, https://schnitzler-briefe.acdh.oeaw.ac.at/{\dateiname}.html (Stand \today)
\fi

\end{document}


