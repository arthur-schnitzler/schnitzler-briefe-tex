\input{../tex-inputs/latex-pdf-vorspann}
\begin{center}
            \textcolor{red}{ENTWURF. ENTZIFFERUNG NOCH NICHT KORREKTURGELESEN}
                      \end{center}
            
               \section[Arthur Schnitzler an Hugo von Hofmannsthal, 22. 8. 1916]{ Arthur Schnitzler an Hugo von Hofmannsthal, 22. 8. 1916}\nopagebreak\mylabel{v}\rehead{ }\begin{ledgroupsized}[t]{13cm}\normalsize\beginnumbering\briefempfaengerindex{Hofmannsthal, Hugo von@\textsc{Hofmannsthal, Hugo von}!zzzSchnitzler, Arthur@\emph{von Arthur Schnitzler}!1916-08-221@{22. 8. 1916}|(be} \toendnotes[C]{\smallbreak\pagebreak[2]} \Standort{FDH, Hs-30885,4.}
\physDesc{Brief, 1 Blatt, 2 Seiten
\newline{}Schreibmaschine
\newline{}Handschrift: schwarze Tinte, lateinische Kurrent (\noindent{}Einfügung,
                                        Unterschrift)\newline{}Zusatz: Eine Fotokopie findet sich unter der Signatur
                                            »Hs-30885,148a«. }\buchAbdrucke{\weitereDrucke{Hugo von Hofmannsthal, Arthur Schnitzler: \emph{Briefwechsel}. Hg. Therese Nickl und Heinrich Schnitzler. Frankfurt am Main: \emph{S. Fischer} 1964, S. 279.} }\toendnotes[C]{\smallbreak}\pstart
           \noindent{}{\pb}\textcolor{gray}{\textbf{Dr. Arthur Schnitzler}}\hfill Alt-Aussee\oindex{Altaussee@\textbf{Altaussee}|pw},
                            22. 8. 1916.\pend
           \pstart
           \textcolor{gray}{\textbf{Wien XVIII. Sternwartestrasse 71\oindex{Sternwartestrasse@\textbf{Sternwartestraße}|pw}}}\pend
           \pstart\center{}Lieber Hugo.\pend\pstart
           »Der letzte Tanz\pwindex{Ehrhart von Ehrhartstein, Robert 12.09.1870 – 11.11.1956@\textsc{Ehrhart von Ehrhartstein, Robert} (12.09.1870 – 11.11.1956), \emph{Schriftsteller, Ministerialbeamter}!letzte Tanz1919@\strich\emph{Der letzte Tanz} {[}1919{]}|pw}« ist eine sehr anmutig,
                    vielleicht manchmal zu ausführlich erzählte Geschichte, in der ein zartes
                    Seelchen von Amadeus Hofmann\pwindex{Hoffmann, Ernst Theodor Amadeus 1776-01-24 – 1822-06-25@\textsc{Hoffmann, Ernst Theodor Amadeus} (1776-01-24 – 1822-06-25), \emph{Schriftsteller, Komponist, Zeichner}|pw} steckt und um
                    die eine reinliche Atmosphäre von Saar\pwindex{Saar, Ferdinand von 30.09.1833 – 24.07.1906@\textsc{Saar, Ferdinand von} (30.09.1833 – 24.07.1906), \emph{Schriftsteller}|pw} und
                        Stifter\pwindex{Stifter, Adalbert 23.10.1805 – 28.01.1868@\textsc{Stifter, Adalbert} (23.10.1805 – 28.01.1868), \emph{Schriftsteller}|pw} schwebt. Sie schienen gewisse
                    Bedenken hinsichtlich dessen, zu hegen, was Ihnen wie ein Rahmen erscheint. Aber
                    Rahmen und Bild sind ja hier durchaus eins, ja, der Rahmen ohne das Bild wäre so
                    gut wie nichts und das Bild ohne den Rahmen nicht viel mehr. Dass die Minna\pwindex{Ehrhart von Ehrhartstein, Robert 12.09.1870 – 11.11.1956@\textsc{Ehrhart von Ehrhartstein, Robert} (12.09.1870 – 11.11.1956), \emph{Schriftsteller, Ministerialbeamter}!letzte Tanz1919@\strich\emph{Der letzte Tanz} {[}1919{]}|pwv} eigentlich das
                    Aquarell und der alte Herr eigentlich die kleine Holzfigur vorstellt, macht ja
                    den Reiz der Geschichte aus, der von Anfang bis zum Ende gleichmässig bescheiden
                    fortwirkt, sich am stärksten in den sonderbaren Anweisungen des süssen und
                    gelegentlich etwas süsslichen Mädel und in den Kunststücken des alten {\pb}Herrn erweist, (unter denen ich das mit dem
                    abgehauenen falschen Kopf als in jedem Sinne aus dem Stil fallend lieber missen
                    möchte) und der nur am Ende ein wenig nachlässt, weil man doch, ich will nicht
                    sagen eine Pointe oder gar eine Lösung, – aber doch irgend einen Schlusseinfall
                    erwartet hätte, der das Ganze in einer höheren Sphäre abschliessen sollte als
                    dies die Erklärung des rationalistischen Willibald\pwindex{Ehrhart von Ehrhartstein, Robert 12.09.1870 – 11.11.1956@\textsc{Ehrhart von Ehrhartstein, Robert} (12.09.1870 – 11.11.1956), \emph{Schriftsteller, Ministerialbeamter}!letzte Tanz1919@\strich\emph{Der letzte Tanz} {[}1919{]}|pwv} vermag. Weiteren Arbeiten des Autors\pwindex{Ehrhart von Ehrhartstein, Robert 12.09.1870 – 11.11.1956@\textsc{Ehrhart von Ehrhartstein, Robert} (12.09.1870 – 11.11.1956), \emph{Schriftsteller, Ministerialbeamter}|pwv}, in dem ich
                    vorläufig mehr Geschmack als Eigenart, mehr Kultur als Inspiration, mehr
                    wohltuende Zärtlichkeit für Wien\oindex{Wien@\textbf{Wien}|pw} als unmittelbar
                    poetische Empfindung zu entdecken glaube, sehe ich mit umso günstigerem
                    Vorurteil entgegen, als die Biedermeierei seines Vorwurfs \introOben{}sich\introOben{} nirgends in Affektation und die freundlichste Phantastik seines
                    Stoffes kaum je sich ins Absurde verliert; – Versuchungen, denen vielleicht
                    mancher künstlerisch stärkere Erzähler in solchem Fall unterlegen wäre. –
                    Herzlichen Dank und Gruss\pend
           \pstart
           Ihr{\\[\baselineskip]}\spacefill\mbox{{[}hs.:{]} Arthur}\pend
           \leftskip=0em{}\endnumbering\briefempfaengerindex{Hofmannsthal, Hugo von@\textsc{Hofmannsthal, Hugo von}!zzzSchnitzler, Arthur@\emph{von Arthur Schnitzler}!1916-08-221@{22. 8. 1916}|)be}\mylabel{h}\end{ledgroupsized}  \newcommand{\dateiname}{L02238}\newcommand{\titel}{Arthur Schnitzler an Hugo von Hofmannsthal, 22. 8. 1916}\newcommand{\editorInnen}{Martin Anton Müller und Gerd-Hermann Susen}\input{../tex-inputs/latex-pdf-abspann}
      