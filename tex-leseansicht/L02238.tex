%% latex-korrekturansicht-vorspann.tex
%% Vorspann für die Korrekturansicht.
%% Lädt die gemeinsame Datei latex-vorspann.tex mit gesetztem Schalter.

\newif\ifkorrekturansicht
\korrekturansichttrue

\input{../tex-inputs/latex-vorspann}


\section[Arthur Schnitzler an Hugo von Hofmannsthal, 22. 8. 1916]{L02238 Arthur Schnitzler an Hugo von Hofmannsthal, 22. 8. 1916}
\nopagebreak\mylabel{L02238v}
\rehead{ }\normalsize\beginnumbering\briefempfaengerindex{Hofmannsthal, Hugo von@\textsc{Hofmannsthal, Hugo von}!zzzSchnitzler, Arthur@\emph{von Arthur Schnitzler}!1916-08-221@{22. 8. 1916}|(be}
\toendnotes[C]{\smallbreak\pagebreak[2]}\Standort{FDH, Hs-30885,4.}
\physDesc{Brief, 1 Blatt, 2 Seiten, 1782 Zeichen
\newline{}Schreibmaschine
\newline{}Handschrift: schwarze Tinte, lateinische Kurrent (\noindent{}Einfügung, Unterschrift)
\newline{}Zusatz: Eine Fotokopie findet sich unter der Signatur
                                 »Hs-30885,148a«. }
\buchAbdrucke{\weitereDrucke{Hugo von Hofmannsthal, Arthur Schnitzler: \emph{Briefwechsel}. Frankfurt am Main: \emph{S. Fischer} 1964, S. 279.} }\toendnotes[C]{\smallbreak}
\pstart
           {\pb}\textcolor{gray}{\textbf{Dr. Arthur Schnitzler}}\hfill Alt-Aussee\oindex{Altaussee@\textbf{Altaussee}, \emph{A.ADM3}|pw},
                     22. 8. 1916.\pend
           
\pstart
           \textcolor{gray}{\textbf{Wien XVIII. Sternwartestrasse 71\oindex{Sternwartestrasse 71@\textbf{Sternwartestraße 71}, \emph{Wohngebäude (K.WHS)}|pw}}}\pend
           
\pstart\center{}Lieber Hugo.\pend\vspace{0.5em}
\pstart
           »Der letzte Tanz\pwindex{letzte Tanz@\emph{Der letzte Tanz}|pw}« ist eine sehr anmutig,
               vielleicht manchmal zu ausführlich erzählte Geschichte, in der ein zartes Seelchen
               von Amadeus Hofmann\pwindex{Hoffmann, Ernst Theodor Amadeus 1776-01-24 – 1822-06-25@\textsc{Hoffmann, Ernst Theodor Amadeus} (1776-01-24 – 1822-06-25), \emph{Schriftsteller/Schriftstellerin, Komponist/Komponistin, Zeichner/Zeichnerin}|pw} steckt und um die eine
               reinliche Atmosphäre von Saar\pwindex{Saar, Ferdinand von 30.09.1833 – 24.07.1906@\textsc{Saar, Ferdinand von} (30.09.1833 – 24.07.1906), \emph{Schriftsteller/Schriftstellerin}|pw} und Stifter\pwindex{Stifter, Adalbert 23.10.1805 – 28.01.1868@\textsc{Stifter, Adalbert} (23.10.1805 – 28.01.1868), \emph{Schriftsteller/Schriftstellerin}|pw} schwebt. Sie schienen gewisse Bedenken
               hinsichtlich dessen, zu hegen, was Ihnen wie ein Rahmen erscheint. Aber Rahmen und
               Bild sind ja hier durchaus eins, ja, der Rahmen ohne das Bild wäre so gut wie nichts
               und das Bild ohne den Rahmen nicht viel mehr. Dass die Minna\pwindex{letzte Tanz@\emph{Der letzte Tanz}|pwv} eigentlich das Aquarell und der alte
               Herr eigentlich die kleine Holzfigur vorstellt, macht ja den Reiz der Geschichte aus,
               der von Anfang bis zum Ende gleichmässig bescheiden fortwirkt, sich am stärksten in
               den sonderbaren Anweisungen des süssen und gelegentlich etwas süsslichen Mädel und in
               den Kunststücken des alten {\pb}Herrn erweist, (unter denen ich
               das mit dem abgehauenen falschen Kopf als in jedem Sinne aus dem Stil fallend lieber
               missen möchte) und der nur am Ende ein wenig nachlässt, weil man doch, ich will nicht
               sagen eine Pointe oder gar eine Lösung, – aber doch irgend einen Schlusseinfall
               erwartet hätte, der das Ganze in einer höheren Sphäre abschliessen sollte als dies
               die Erklärung des rationalistischen Willibald\pwindex{letzte Tanz@\emph{Der letzte Tanz}|pwv} vermag. Weiteren Arbeiten des Autors\pwindex{Ehrhart-Ehrhartstein, Robert 12.09.1870 – 11.11.1956@\textsc{Ehrhart-Ehrhartstein, Robert} (12.09.1870 – 11.11.1956), \emph{Schriftsteller/Schriftstellerin, Ministerialbeamter/Ministerialbeamte}|pwv}, in dem ich vorläufig mehr Geschmack als Eigenart,
               mehr Kultur als Inspiration, mehr wohltuende Zärtlichkeit für Wien\oindex{Wien@\textbf{Wien}, \emph{A.ADM2}|pw} als unmittelbar poetische Empfindung zu entdecken glaube,
               sehe ich mit umso günstigerem Vorurteil entgegen, als die Biedermeierei seines
               Vorwurfs \introOben{}sich\introOben{} nirgends in Affektation und die freundlichste
               Phantastik seines Stoffes kaum je sich ins Absurde verliert; – Versuchungen, denen
               vielleicht mancher künstlerisch stärkere Erzähler in solchem Fall unterlegen wäre. –
               Herzlichen Dank und Gruss\pend
           
\pstart
           Ihr{\\[\baselineskip]}\spacefill\mbox{{[}hs.:{]} Arthur}\pend
           \leftskip=0em{}\selectlanguage{ngerman}\endnumbering\briefempfaengerindex{Hofmannsthal, Hugo von@\textsc{Hofmannsthal, Hugo von}!zzzSchnitzler, Arthur@\emph{von Arthur Schnitzler}!1916-08-221@{22. 8. 1916}|)be}\mylabel{L02238h}  \normalsize

\doendnotes{C}
\bigskip
\vfill

\clearpage

\footnotesize

\lohead{\textsc{register}}

% Definiere theindex-Environment komplett neu ohne reledmac
\makeatletter
\renewenvironment{theindex}{%
  \section*{\indexname}%
  \setlength{\parindent}{0pt}%
  \setlength{\parskip}{0pt plus 0.3pt}%
  \let\item\@idxitem
}{%
  \clearpage
}
\makeatother

\IfFileExists{\jobname-pw.ind}{\input{\jobname-pw.ind}}{}

\end{document}

      