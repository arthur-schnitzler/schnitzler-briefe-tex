%% latex-leseansicht-vorspann.tex
%% Vorspann für die Leseansicht.
%% Lädt die gemeinsame Datei latex-vorspann.tex mit nicht gesetztem Schalter.

\newif\ifkorrekturansicht
\korrekturansichtfalse

\input{../tex-inputs/latex-vorspann}


\section[Arthur Schnitzler an Wilhelm Bölsche, 19. 12. 1891]{L00055 Arthur Schnitzler an Wilhelm Bölsche, 19. 12. 1891}
\nopagebreak\mylabel{L00055v}
\rehead{ }\normalsize\beginnumbering\briefempfaengerindex{Bölsche, Wilhelm@\textsc{Bölsche, Wilhelm}!zzzSchnitzler, Arthur@\emph{von Arthur Schnitzler}!1891-12-191@{19. 12. 1891}|(be}
\toendnotes[C]{\smallbreak\pagebreak[2]}
\correspDesc{Versand  durch Arthur Schnitzler am 19. 12. 1891 in Wien
\newline{}Erhalt  durch Wilhelm Bölsche im Zeitraum [20. 12. 1891 – 24. 12. 1891?] in Berlin}\toendnotes[C]{\smallbreak}
\Standort{Wrocław, Biblioteka Uniwersytecka, Böl.Pis 1761.}
\physDesc{Brief, 1 Blatt, 2 Seiten, 657 Zeichen
\newline{}Handschrift: schwarze Tinte, deutsche Kurrent}
\buchAbdrucke{\weitereDrucke{1) Alois Woldan: \emph{Arthur Schnitzler – Briefe an Wilhelm Bölsche.} In: \emph{Germanica Wratislaviensia} (1987) Nr. 77, S. 459.} \weitereDrucke{2) Wilhelm Bölsche: \emph{Briefwechsel. Mit Autoren der Freien Bühne}. Herausgegeben von Gerd-Hermann Susen. Berlin: \emph{Weidler} 2010, S. 674 (Werke und Briefe. Wissenschaftliche Ausgabe, Briefe I).} }
\pstart
           
\pstart
           {\pb}\textsc{Wien, I. Gisel\damage{as}traße 11}\oindex{Wien@\textbf{Wien}!I., Innere Stadt@\textbf{I., Innere Stadt}!Kärntnerring 12/Bösendorferstraße 11@\textbf{Kärntnerring 12/Bösendorferstraße 11}, \emph{Wohngebäude}|pw}.\pend
           
\pstart
           \raggedleft{}Am 19. Dez 91.\pend
           \pend
           
\pstart\center{}Sehr geehrter Herr,\pend\vspace{0.5em}
\pstart
           beſten Dank für Ihre liebenswürdige Aufforderung, der ich mit beſonderm Vergnügen
                  nachko{\geminationm}en werde.\pend
           
\pstart
           Erlauben Sie mir zugleich, Ihnen das beiliegende Schauſpiel\pwindex{Schnitzler, Arthur 15.\,5.\,1862 Wien – 21.\,10.\,1931 ebd.@\textsc{Schnitzler, Arthur} (15.\,5.\,1862 Wien – 21.\,10.\,1931 ebd.), \emph{Schriftsteller, Mediziner}!Märchen. Schauspiel in drei Aufzügen@\strich\emph{Das Märchen. Schauspiel in drei Aufzügen}|pw} als Zeichen meines aufrichtigen Vertrauens zu überſenden – ich
               überreiche es \uline{nicht} dem Redacteur der Freien Bühne\pwindex{Freie Bühne für modernes Leben@\emph{Freie Bühne für modernes Leben}|pw}, da ich es vor einer eventuellen
               Aufführung nicht veröffentlichen will,{ }ſondern dem von mir hochgeſchätzten
               Schriftſteller, dem es vielleicht {\pb}einiges Intereſſe
               gewähren wird.\pend
           
\pstart
           Es iſt im übrigen, was ich als \uline{ganz private
                  Mittheilung} aufzufaſſen bitte, am \textsc{Lessing}theater\orgindex{Lessing-Theater@Lessing-Theater|pw} angeno{\geminationm}en.\pend
           
\pstart
           Mit ausgezeichneter Hochachtung{\\[\baselineskip]}Ihr ergebner{\\[\baselineskip]}\spacefill\mbox{DrArthurSchnitzler}\pend
           \leftskip=0em{}\selectlanguage{ngerman}\endnumbering\briefempfaengerindex{Bölsche, Wilhelm@\textsc{Bölsche, Wilhelm}!zzzSchnitzler, Arthur@\emph{von Arthur Schnitzler}!1891-12-191@{19. 12. 1891}|)be}\mylabel{L00055h}  \newcommand{\dateiname}{L00055}\newcommand{\titel}{Arthur Schnitzler an Wilhelm Bölsche, 19. 12. 1891}\newcommand{\editorInnen}{Martin Anton Müller und Gerd-Hermann Susen}%% latex-leseansicht-abspann.tex
%% Abspann für die Leseansicht.
%% Der Schalter \ifkorrekturansicht ist bereits durch den Vorspann gesetzt.

%% latex-abspann.tex
%% Gemeinsamer Abspann für Korrekturansicht und Leseansicht.
%% Setzt den Schalter \ifkorrekturansicht voraus (gesetzt in den
%% einbindenden Dateien latex-korrekturansicht-abspann.tex bzw.
%% latex-leseansicht-abspann.tex).
%% ---------------------------------------------------------------

\normalsize

% Das esempio-Environment wird nur in der Leseansicht benötigt
\ifkorrekturansicht\else
\newenvironment{esempio}[3]%
{
    \vspace{1.5ex}
    \rlap{\underline{#1}}
    \par
    \setlength{\parindent}{0cm}
    \nopagebreak
    \leftskip=#2cm
    \rightskip=#3cm
}
{
    \par
}
\fi

\doendnotes{C}
\bigskip
\vfill

\clearpage

\footnotesize

\ifkorrekturansicht
  \lohead{\textsc{register}}
\fi

% theindex-Environment neu definieren ohne reledmac
\makeatletter
\renewenvironment{theindex}{%
  \ifkorrekturansicht
    \section*{\indexname}%
  \else
    \subsubsection*{Index der erwähnten Entitäten}%
  \fi
  \setlength{\parindent}{0pt}%
  \setlength{\parskip}{0pt plus 0.3pt}%
  \let\item\@idxitem
}{%
  \ifkorrekturansicht\clearpage\fi
}
\makeatother

\IfFileExists{\jobname-pw.ind}{\input{\jobname-pw.ind}}{}

% Quellenangabe nur in der Leseansicht
\ifkorrekturansicht\else
% Fallback-Definitionen, falls die .tex-Datei \titel etc. nicht gesetzt hat
\providecommand{\titel}{}
\providecommand{\editorInnen}{}
\providecommand{\dateiname}{\jobname}

\vspace{3cm}

\vfill

\footnotesize
\textsc{Quelle}: \titel. Herausgegeben von {\editorInnen}. In: \emph{Arthur Schnitzler: Briefwechsel mit Autorinnen und Autoren}.
 Digitale Edition, https://schnitzler-briefe.acdh.oeaw.ac.at/{\dateiname}.html (Stand \today)
\fi

\end{document}


