%% latex-korrekturansicht-vorspann.tex
%% Vorspann für die Korrekturansicht.
%% Lädt die gemeinsame Datei latex-vorspann.tex mit gesetztem Schalter.

\newif\ifkorrekturansicht
\korrekturansichttrue

\input{../tex-inputs/latex-vorspann}


\section[Arthur Schnitzler an Wilhelm Bölsche, 19. 12. 1891]{L00055 Arthur Schnitzler an Wilhelm Bölsche, 19. 12. 1891}
\nopagebreak\mylabel{L00055v}
\rehead{ }\normalsize\beginnumbering\briefempfaengerindex{Boelsche, Wilhelm@\textsc{Bölsche, Wilhelm}!zzzSchnitzler, Arthur@\emph{von Arthur Schnitzler}!1891-12-191@{19. 12. 1891}|(be}
\toendnotes[C]{\smallbreak\pagebreak[2]}\Standort{Wrocław, Biblioteka Uniwersytecka, Böl.Pis 1761.}
\physDesc{Brief, 1 Blatt, 2 Seiten, 657 Zeichen
\newline{}Handschrift: schwarze Tinte, deutsche Kurrent}
\buchAbdrucke{\weitereDrucke{1) \emph{Germanica Wratislaviensia} (1987) Nr. 77, S. 459.} \weitereDrucke{2) Wilhelm Bölsche: \emph{Briefwechsel. Mit Autoren der Freien Bühne}. Berlin: \emph{Weidler} 2010, S. 674.} }
\pstart
           
\pstart
           {\pb}\textsc{Wien, I. Gisel\damage{as}traße 11}\oindex{Kaerntnerring 12/Boesendorferstrasse 11@\textbf{Kärntnerring 12/Bösendorferstraße 11}, \emph{Wohngebäude (K.WHS)}|pw}.\pend
           
\pstart
           \raggedleft{}Am 19. Dez 91.\pend
           \pend
           
\pstart\center{}Sehr geehrter Herr,\pend\vspace{0.5em}
\pstart
           beſten Dank für Ihre liebenswürdige Aufforderung, der ich mit beſonderm Vergnügen
                  nachko{\geminationm}en werde.\pend
           
\pstart
           Erlauben Sie mir zugleich, Ihnen das beiliegende Schauſpiel\pwindex{Maerchen. Schauspiel in drei Aufzuegen@\emph{Das Märchen. Schauspiel in drei Aufzügen}|pw} als Zeichen meines aufrichtigen Vertrauens zu überſenden – ich
               überreiche es \uline{nicht} dem Redacteur der Freien Bühne\pwindex{Freie Buehne fuer modernes Leben@\emph{Freie Bühne für modernes Leben}|pw}, da ich es vor einer eventuellen
               Aufführung nicht veröffentlichen will, ſondern dem von mir hochgeſchätzten
               Schriftſteller, dem es vielleicht {\pb}einiges Intereſſe
               gewähren wird.\pend
           
\pstart
           Es iſt im übrigen, was ich als \uline{ganz private
                  Mittheilung} aufzufaſſen bitte, am \textsc{Lessing}theater\orgindex{Lessing-Theater@Lessing-Theater|pw} angeno{\geminationm}en.\pend
           
\pstart
           Mit ausgezeichneter Hochachtung{\\[\baselineskip]}Ihr ergebner{\\[\baselineskip]}\spacefill\mbox{DrArthurSchnitzler}\pend
           \leftskip=0em{}\selectlanguage{ngerman}\endnumbering\briefempfaengerindex{Boelsche, Wilhelm@\textsc{Bölsche, Wilhelm}!zzzSchnitzler, Arthur@\emph{von Arthur Schnitzler}!1891-12-191@{19. 12. 1891}|)be}\mylabel{L00055h}  \normalsize

\doendnotes{C}
\bigskip
\vfill

\clearpage

\footnotesize

\lohead{\textsc{register}}

% Definiere theindex-Environment komplett neu ohne reledmac
\makeatletter
\renewenvironment{theindex}{%
  \section*{\indexname}%
  \setlength{\parindent}{0pt}%
  \setlength{\parskip}{0pt plus 0.3pt}%
  \let\item\@idxitem
}{%
  \clearpage
}
\makeatother

\IfFileExists{\jobname-pw.ind}{\input{\jobname-pw.ind}}{}

\end{document}

      