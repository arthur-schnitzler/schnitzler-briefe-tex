%% latex-leseansicht-vorspann.tex
%% Vorspann für die Leseansicht.
%% Lädt die gemeinsame Datei latex-vorspann.tex mit nicht gesetztem Schalter.

\newif\ifkorrekturansicht
\korrekturansichtfalse

\input{../tex-inputs/latex-vorspann}


         
         \renewcommand{\erwaehntePersonen}{Personen: Richard Beer-Hofmann, Naëmah Beer-Hofmann, Gabriel Beer-Hofmann, Jakob Minor}
         \renewcommand{\erwaehnteInstitutionen}{Institutionen: Franz-Grillparzer-Preis, Neue Freie Presse}
         \renewcommand{\erwaehnteOrte}{Orte: Wien}
         \renewcommand{\erwaehnteWerke}{Werke: Die Verleihung des Grillparzer-Preises, Neue Freie Presse, Zwischenspiel. Komödie in drei Akten}
               \section[Richard Beer-Hofmann an Arthur Schnitzler, {[}17. 1. 1908{]}]{ Richard Beer-Hofmann an Arthur Schnitzler, {[}17. 1. 1908{]}}\nopagebreak\mylabel{v}\rehead{ }\begin{ledgroupsized}[t]{13cm}\normalsize\beginnumbering\briefempfaengerindex{Schnitzler, Arthur@\textsc{Schnitzler, Arthur}!zzzBeer-Hofmann, Richard@\emph{von Richard Beer-Hofmann}!1908-01-171@{{[}17. 1. 1908{]}}|(be} \toendnotes[C]{\smallbreak\pagebreak[2]} \Standort{CUL, Schnitzler, B 8.}
\physDesc{Brief, 1 Blatt, 3 Seiten, 455 Zeichen (Briefpapier mit Trauerrand)
\newline{}Handschrift: blauer Buntstift, lateinische Kurrent
\newline{}Schnitzler: mit Bleistift datiert: »17/1 908« 
\newline{}Ordnung: mit Bleistift von unbekannter Hand nummeriert:
                                    »218« }\buchAbdrucke{\weitereDrucke{Arthur Schnitzler, Richard Beer-Hofmann: \emph{Briefwechsel 1891–1931}. Hg. Konstanze Fliedl. Wien, Zürich: \emph{Europaverlag} 1992, S. 188.} }\toendnotes[C]{\smallbreak}\pstart
           \raggedleft{}{\pb}Freitag\pend
           \pstart
           Lieber Arthur! Ich freue mich\orgindex{Franz-Grillparzer-Preis@Franz-Grillparzer-Preis|pwv} von ganzem Herzen – besonders nach dem letzten Gespräch das wir
               hierüber hatten. Ob \label{K_L01752-1v}\edtext{Minor\pwindex{Minor, Jakob 15.04.1855 – 07.10.1912@\textsc{Minor, Jakob} (15.04.1855 – 07.10.1912)|pw}s Motivirung}{\lemma{\textnormal{\emph{Minors Motivirung}}}\Cendnote{\textnormal{Der Einigung auf Schnitzler\pwindex{Schnitzler, Arthur 15.05.1862 – 21.10.1931@\textsc{Schnitzler, Arthur} (15.05.1862 – 21.10.1931), \emph{Schriftsteller, Mediziner}|pwk} lag ein Kompromiss innerhalb der Jury zugrunde. Jakob Minor\pwindex{Minor, Jakob 15.04.1855 – 07.10.1912@\textsc{Minor, Jakob} (15.04.1855 – 07.10.1912)|pwk}, der Vorsitzende der Kommission,
                  begründete die Entscheidung so: »Für das Votum des Preisrichterkollegiums
                     kam, wie Hofrat Minor\pwindex{Minor, Jakob 15.04.1855 – 07.10.1912@\textsc{Minor, Jakob} (15.04.1855 – 07.10.1912)|pw} in seinem Referat
                     ausführte, in erster Linie das Stück\pwindex{Schnitzler, Arthur 15.05.1862 – 21.10.1931@\textsc{Schnitzler, Arthur} (15.05.1862 – 21.10.1931), \emph{Schriftsteller, Mediziner}!Zwischenspiel. Komoedie in drei Akten1905-10-12@\strich\emph{Zwischenspiel. Komödie in drei Akten} {[}1905-10-12{]}|pwv}, das den Preis erhielt, in Betracht und erst in
                     zweiter Linie der Dichter.« ([O. V.;] \emph{Die Verleihung des Grillparzer-Preises}\pwindex{?? Werk@Nicht ermittelte Verfasserinnen und Verfasser!Verleihung des Grillparzer-Preises1908-01-16@\emph{Die Verleihung des Grillparzer-Preises} {[}1908-01-16{]}|pwk}. In: \emph{Neue Freie Presse}\pwindex{Neue Freie Presse1864 – 1939@\emph{Neue Freie Presse} {[}1864 – 1939{]}|pwk}, Nr. 15590,
                        16. 1. 1908, Morgenblatt, S. 8).}}}\label{K_L01752-1h} eine Perfidie,
               oder ein ungeschicktes Compliment war wird sich kaum feststellen lassen.\pend
           \pstart
           Auch die \label{K_L01752-11v}\edtext{N. Fr. Pr.\orgindex{Neue Freie Presse@Neue Freie Presse|pw} war wieder einmal recht
               herzig}{\lemma{\textnormal{\emph{N. Fr. Pr. … herzig}}}\Cendnote{\textnormal{vgl. A. S.: \emph{Tagebuch}, 18. 1. 1908}}}\label{K_L01752-11h}.\pend
           \pstart
           {\pb}Bitte lassen Sie doch von sich
               hören – hören, wörtlich geno{\geminationm}en – ich kann nichts
               dazutun. Naëmah\pwindex{Beer-Hofmann, Naemah 20.12.1898 – 10.11.1971@\textsc{Beer-Hofmann, Naëmah} (20.12.1898 – 10.11.1971)|pw}, Bubi\pwindex{Beer-Hofmann, Gabriel 09.01.1901 – 24.03.1971@\textsc{Beer-Hofmann, Gabriel} (09.01.1901 – 24.03.1971), \emph{Schriftsteller, Filmagent}|pw} haben {\pb}Influenza gehabt, sind noch zu Bett, wir Andern noch nicht. Herzlichst\pend
           \pstart
           Ihr{\\[\baselineskip]}\spacefill\mbox{Richard}\pend
           \leftskip=0em{}
         
         \endnumbering\mylabel{h}\end{ledgroupsized}  \newcommand{\dateiname}{L01752}\newcommand{\titel}{Richard Beer-Hofmann an Arthur Schnitzler, [17. 1. 1908]}\newcommand{\editorInnen}{Martin Anton Müller und Gerd-Hermann Susen}%% latex-leseansicht-abspann.tex
%% Abspann für die Leseansicht.
%% Der Schalter \ifkorrekturansicht ist bereits durch den Vorspann gesetzt.

%% latex-abspann.tex
%% Gemeinsamer Abspann für Korrekturansicht und Leseansicht.
%% Setzt den Schalter \ifkorrekturansicht voraus (gesetzt in den
%% einbindenden Dateien latex-korrekturansicht-abspann.tex bzw.
%% latex-leseansicht-abspann.tex).
%% ---------------------------------------------------------------

\normalsize

% Das esempio-Environment wird nur in der Leseansicht benötigt
\ifkorrekturansicht\else
\newenvironment{esempio}[3]%
{
    \vspace{1.5ex}
    \rlap{\underline{#1}}
    \par
    \setlength{\parindent}{0cm}
    \nopagebreak
    \leftskip=#2cm
    \rightskip=#3cm
}
{
    \par
}
\fi

\doendnotes{C}
\bigskip
\vfill

\clearpage

\footnotesize

\ifkorrekturansicht
  \lohead{\textsc{register}}
\fi

% theindex-Environment neu definieren ohne reledmac
\makeatletter
\renewenvironment{theindex}{%
  \ifkorrekturansicht
    \section*{\indexname}%
  \else
    \subsubsection*{Index der erwähnten Entitäten}%
  \fi
  \setlength{\parindent}{0pt}%
  \setlength{\parskip}{0pt plus 0.3pt}%
  \let\item\@idxitem
}{%
  \ifkorrekturansicht\clearpage\fi
}
\makeatother

\IfFileExists{\jobname-pw.ind}{\input{\jobname-pw.ind}}{}

% Quellenangabe nur in der Leseansicht
\ifkorrekturansicht\else
% Fallback-Definitionen, falls die .tex-Datei \titel etc. nicht gesetzt hat
\providecommand{\titel}{}
\providecommand{\editorInnen}{}
\providecommand{\dateiname}{\jobname}

\vspace{3cm}

\vfill

\footnotesize
\textsc{Quelle}: \titel. Herausgegeben von {\editorInnen}. In: \emph{Arthur Schnitzler: Briefwechsel mit Autorinnen und Autoren}.
 Digitale Edition, https://schnitzler-briefe.acdh.oeaw.ac.at/{\dateiname}.html (Stand \today)
\fi

\end{document}


      