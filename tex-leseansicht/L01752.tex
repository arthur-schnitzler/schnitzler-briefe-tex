%% latex-korrekturansicht-vorspann.tex
%% Vorspann für die Korrekturansicht.
%% Lädt die gemeinsame Datei latex-vorspann.tex mit gesetztem Schalter.

\newif\ifkorrekturansicht
\korrekturansichttrue

\input{../tex-inputs/latex-vorspann}


\section[Richard Beer-Hofmann an Arthur Schnitzler, {[}17. 1. 1908{]}]{L01752 Richard Beer-Hofmann an Arthur Schnitzler, {[}17. 1. 1908{]}}
\nopagebreak\mylabel{L01752v}
\rehead{ }\normalsize\beginnumbering\briefempfaengerindex{Schnitzler, Arthur@\textsc{Schnitzler, Arthur}!zzzBeer-Hofmann, Richard@\emph{von Richard Beer-Hofmann}!1908-01-171@{{[}17. 1. 1908{]}}|(be}
\toendnotes[C]{\smallbreak\pagebreak[2]}\Standort{CUL, Schnitzler, B 8.}
\physDesc{Brief, 1 Blatt, 3 Seiten, 455 Zeichen (Briefpapier mit Trauerrand)
\newline{}Handschrift: blauer Buntstift, lateinische Kurrent
\newline{}Schnitzler: mit Bleistift datiert: »17/1 908« 
\newline{}Ordnung: mit Bleistift von unbekannter Hand nummeriert:
                                    »218« }
\buchAbdrucke{\weitereDrucke{Arthur Schnitzler, Richard Beer-Hofmann: \emph{Briefwechsel 1891–1931}. Wien, Zürich: \emph{Europaverlag} 1992, S. 188.} }\toendnotes[C]{\smallbreak}
\pstart
           \raggedleft{}{\pb}Freitag\pend
           \vspace{0.5em}
\pstart
           Lieber Arthur! Ich freue mich\orgindex{Franz-Grillparzer-Preis@Franz-Grillparzer-Preis|pwv} von ganzem Herzen – besonders nach dem letzten Gespräch das wir
               hierüber hatten. Ob \label{K_L01752-1v}\edtext{Minors\pwindex{Minor, Jakob 15.04.1855 – 07.10.1912@\textsc{Minor, Jakob} (15.04.1855 – 07.10.1912)|pw} Motivirung}{\lemma{\textnormal{\emph{Minors Motivirung}}}\Cendnote{\textnormal{Der Einigung auf Schnitzler lag ein Kompromiss innerhalb der Jury zugrunde. Offiziell wurde
                  die Entscheidung durch 
                  Jakob Minor\pwindex{Minor, Jakob 15.04.1855 – 07.10.1912@\textsc{Minor, Jakob} (15.04.1855 – 07.10.1912)|pwk}, den Vorsitzenden der Kommission,
                  so motiviert: »Für das Votum des Preisrichterkollegiums
                     kam, wie Hofrat Minor\pwindex{Minor, Jakob 15.04.1855 – 07.10.1912@\textsc{Minor, Jakob} (15.04.1855 – 07.10.1912)|pw} in seinem Referat
                     ausführte, in erster Linie das Stück\pwindex{Zwischenspiel. Komoedie in drei Akten@\emph{Zwischenspiel. Komödie in drei Akten}|pwv}, das den Preis erhielt, in Betracht und erst in
                     zweiter Linie der Dichter.« [O. V.]: \emph{Die Verleihung des Grillparzer-Preises}\pwindex{Verleihung des Grillparzer-Preises@\emph{Die Verleihung des Grillparzer-Preises}|pwk}. In: \emph{Neue Freie Presse}\pwindex{Neue Freie Presse@\emph{Neue Freie Presse}|pwk}, Nr. 15.590,
                        16. 1. 1908, Morgenblatt, S. 8.}}}\label{K_L01752-1} eine Perfidie,
               oder ein ungeschicktes Compliment war wird sich kaum feststellen lassen.\pend
           
\pstart
           Auch die \label{K_L01752-2v}\edtext{N. Fr. Pr.\orgindex{Neue Freie Presse@Neue Freie Presse|pw} war wieder einmal recht
               herzig}{\lemma{\textnormal{\emph{N. Fr. Pr. … herzig}}}\Cendnote{\textnormal{Vgl. A. S.: \emph{Tagebuch}, 18. 1. 1908.
               }}}\label{K_L01752-2}.\pend
           
\pstart
           {\pb}Bitte lassen Sie doch von sich
               hören – hören, wörtlich geno{\geminationm}en – ich kann nichts
               dazutun. Naëmah\pwindex{Beer-Hofmann, Naemah 20.12.1898 – 10.11.1971@\textsc{Beer-Hofmann, Naëmah} (20.12.1898 – 10.11.1971)|pw}, Bubi\pwindex{Beer-Hofmann, Gabriel 09.01.1901 – 24.03.1971@\textsc{Beer-Hofmann, Gabriel} (09.01.1901 – 24.03.1971), \emph{Schriftsteller/Schriftstellerin, Filmagent/Filmagentin}|pw} haben {\pb}Influenza gehabt, sind noch zu Bett, wir Andern noch nicht. Herzlichst\pend
           
\pstart
           Ihr{\\[\baselineskip]}\spacefill\mbox{Richard}\pend
           \leftskip=0em{}\selectlanguage{ngerman}\endnumbering\briefempfaengerindex{Schnitzler, Arthur@\textsc{Schnitzler, Arthur}!zzzBeer-Hofmann, Richard@\emph{von Richard Beer-Hofmann}!1908-01-171@{{[}17. 1. 1908{]}}|)be}\mylabel{L01752h}  \normalsize

\doendnotes{C}
\bigskip
\vfill

\clearpage

\footnotesize

\lohead{\textsc{register}}

% Definiere theindex-Environment komplett neu ohne reledmac
\makeatletter
\renewenvironment{theindex}{%
  \section*{\indexname}%
  \setlength{\parindent}{0pt}%
  \setlength{\parskip}{0pt plus 0.3pt}%
  \let\item\@idxitem
}{%
  \clearpage
}
\makeatother

\IfFileExists{\jobname-pw.ind}{\input{\jobname-pw.ind}}{}

\end{document}

      