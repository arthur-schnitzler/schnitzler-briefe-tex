%% latex-leseansicht-vorspann.tex
%% Vorspann für die Leseansicht.
%% Lädt die gemeinsame Datei latex-vorspann.tex mit nicht gesetztem Schalter.

\newif\ifkorrekturansicht
\korrekturansichtfalse

\input{../tex-inputs/latex-vorspann}


\section[Paul Goldmann an Arthur Schnitzler, {[}10.? 10. 1895{]}]{L02693 Paul Goldmann an Arthur Schnitzler, {[}10.? 10. 1895{]}}
\nopagebreak\mylabel{L02693v}
\rehead{ }\normalsize\beginnumbering\briefempfaengerindex{Schnitzler, Arthur@\textsc{Schnitzler, Arthur}!zzzGoldmann, Paul@\emph{von Paul Goldmann}!1895-10-103@{{[}10.? 10. 1895{]}}|(be}
\toendnotes[C]{\smallbreak\pagebreak[2]}
\correspDesc{Versand  durch Paul Goldmann am [10.? 10. 1895] in Paris
\newline{}Erhalt  durch Arthur Schnitzler am [10. 10. 1895?] in Wien}\toendnotes[C]{\smallbreak}
\Standort{DLA, A:Schnitzler, HS.NZ85.1.3165.}
\physDesc{Telegramm, 310 Zeichen
\newline{}maschinell
\newline{}Schnitzler: mit Bleistift datiert: »Oct 9\textcolor{gray}{5}« 
\newline{}Ordnung: beschnitten }\toendnotes[C]{\smallbreak}
\pstart
           \centering{}{\pb}\damage{\textcolor{gray}{paris\oindex{Paris@\textbf{Paris}, \emph{Hauptstadt}|pw}}} 45789 \damage{\textcolor{gray}{58 10 10 113}}\pend
           \vspace{0.5em}
\pstart
           ob der \label{K_L02693-1v}\edtext{erfolg}{\lemma{\textnormal{\emph{erfolg}}}\Cendnote{\textnormal{Am Vortag, dem 9. 10. 1895, hatte die Uraufführung\eventindex{Burgtheater@\textbf{Burgtheater}!Uraufführung von Liebelei, Premiere von Rechte der Seele, 9.10.1895@Uraufführung von Liebelei, Premiere von Rechte der Seele, 9.10.1895|pwkv} von \emph{Liebelei}\pwindex{Schnitzler, Arthur 15.\,5.\,1862 Wien – 21.\,10.\,1931 ebd.@\textsc{Schnitzler, Arthur} (15.\,5.\,1862 Wien – 21.\,10.\,1931 ebd.), \emph{Schriftsteller, Mediziner}!Liebelei. Schauspiel in drei Akten@\strich\emph{Liebelei. Schauspiel in drei Akten}|pwk} am \emph{Burgtheater}\orgindex{Burgtheater@Burgtheater|pwk}
               stattgefunden.}}}\label{K_L02693-1} nachhaelt ist einstweilen \label{T_L02693-1v}\edtext{gleichgiltig}{\lemma{\textnormal{\emph{gleichgiltig}}}\Cendnote{\textnormal{In der Vorlage steht: »gleichgittig«.}}}\label{T_L02693-1} wichtig war nur der gestrige{ }abend er ist gut verlaufen folglich ist das werk\pwindex{Schnitzler, Arthur 15.\,5.\,1862 Wien – 21.\,10.\,1931 ebd.@\textsc{Schnitzler, Arthur} (15.\,5.\,1862 Wien – 21.\,10.\,1931 ebd.), \emph{Schriftsteller, Mediziner}!Liebelei. Schauspiel in drei Akten@\strich\emph{Liebelei. Schauspiel in drei Akten}|pwv} gelungen\pend
           
\pstart
           ich danke dir fuer die frohe nachricht und \label{T_L02693-2v}\edtext{beglueckwuensche}{\lemma{\textnormal{\emph{beglueckwuensche}}}\Cendnote{\textnormal{In der Vorlage steht: »begluekwensche«.}}}\label{T_L02693-2} dich von ganzem herzen
               es musste so kommen aber es ist doch schoen dass es so kam\pend
           \pstart gruesse = \spacefill\mbox{goldmann}\pend{}\selectlanguage{ngerman}\endnumbering\briefempfaengerindex{Schnitzler, Arthur@\textsc{Schnitzler, Arthur}!zzzGoldmann, Paul@\emph{von Paul Goldmann}!1895-10-103@{{[}10.? 10. 1895{]}}|)be}\mylabel{L02693h}  \newcommand{\dateiname}{L02693}\newcommand{\titel}{Paul Goldmann an Arthur Schnitzler, [10.? 10. 1895]}\newcommand{\editorInnen}{Martin Anton Müller und Laura Untner}%% latex-leseansicht-abspann.tex
%% Abspann für die Leseansicht.
%% Der Schalter \ifkorrekturansicht ist bereits durch den Vorspann gesetzt.

%% latex-abspann.tex
%% Gemeinsamer Abspann für Korrekturansicht und Leseansicht.
%% Setzt den Schalter \ifkorrekturansicht voraus (gesetzt in den
%% einbindenden Dateien latex-korrekturansicht-abspann.tex bzw.
%% latex-leseansicht-abspann.tex).
%% ---------------------------------------------------------------

\normalsize

% Das esempio-Environment wird nur in der Leseansicht benötigt
\ifkorrekturansicht\else
\newenvironment{esempio}[3]%
{
    \vspace{1.5ex}
    \rlap{\underline{#1}}
    \par
    \setlength{\parindent}{0cm}
    \nopagebreak
    \leftskip=#2cm
    \rightskip=#3cm
}
{
    \par
}
\fi

\doendnotes{C}
\bigskip
\vfill

\clearpage

\footnotesize

\ifkorrekturansicht
  \lohead{\textsc{register}}
\fi

% theindex-Environment neu definieren ohne reledmac
\makeatletter
\renewenvironment{theindex}{%
  \ifkorrekturansicht
    \section*{\indexname}%
  \else
    \subsubsection*{Index der erwähnten Entitäten}%
  \fi
  \setlength{\parindent}{0pt}%
  \setlength{\parskip}{0pt plus 0.3pt}%
  \let\item\@idxitem
}{%
  \ifkorrekturansicht\clearpage\fi
}
\makeatother

\IfFileExists{\jobname-pw.ind}{\input{\jobname-pw.ind}}{}

% Quellenangabe nur in der Leseansicht
\ifkorrekturansicht\else
% Fallback-Definitionen, falls die .tex-Datei \titel etc. nicht gesetzt hat
\providecommand{\titel}{}
\providecommand{\editorInnen}{}
\providecommand{\dateiname}{\jobname}

\vspace{3cm}

\vfill

\footnotesize
\textsc{Quelle}: \titel. Herausgegeben von {\editorInnen}. In: \emph{Arthur Schnitzler: Briefwechsel mit Autorinnen und Autoren}.
 Digitale Edition, https://schnitzler-briefe.acdh.oeaw.ac.at/{\dateiname}.html (Stand \today)
\fi

\end{document}


