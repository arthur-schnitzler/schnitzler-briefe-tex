%% latex-korrekturansicht-vorspann.tex
%% Vorspann für die Korrekturansicht.
%% Lädt die gemeinsame Datei latex-vorspann.tex mit gesetztem Schalter.

\newif\ifkorrekturansicht
\korrekturansichttrue

\input{../tex-inputs/latex-vorspann}


\section[Arthur Schnitzler an Hugo Hofmannsthal, 21. 7. 1928]{L02504 Arthur Schnitzler an Hugo Hofmannsthal, 21. 7. 1928}
\nopagebreak\mylabel{L02504v}
\rehead{ }\normalsize\beginnumbering\briefempfaengerindex{Hofmannsthal, Hugo von@\textsc{Hofmannsthal, Hugo von}!zzzSchnitzler, Arthur@\emph{von Arthur Schnitzler}!1928-07-211@{21. 7. 1928}|(be}
\toendnotes[C]{\smallbreak\pagebreak[2]}\Standort{FDH, Hs-30885,158.}
\physDesc{Brief, 2 Blätter, 3 Seiten, 2869 Zeichen
\newline{}Handschrift: schwarze Tinte, lateinische Kurrent
\newline{}Ordnung: 1) mit Bleistift von unbekannter Hand beschriftet: »5 Tage
                                    vor Lili\pwindex{Cappellini, Lili 13.09.1909 – 26.07.1928@\textsc{Cappellini, Lili} (13.09.1909 – 26.07.1928)|pw}’s
                                 Tod«  2) mit Bleistift von unbekannter Hand das zweite Blatt nummeriert
                                 mit »2.«}
\buchAbdrucke{\weitereDrucke{1) Hugo von Hofmannsthal, Arthur Schnitzler: \emph{Briefwechsel}. Frankfurt am Main: \emph{S. Fischer} 1964, S. 310–311.} \weitereDrucke{2) Arthur Schnitzler: \emph{Briefe 1913–1931}. Frankfurt am Main: \emph{S. Fischer} 1984, S. 557–559.} }\toendnotes[C]{\smallbreak}
\pstart
           \raggedleft{}{\pb}Wien\oindex{Wien@\textbf{Wien}, \emph{A.ADM2}|pw}, 21. 7. 928\pend
           \vspace{0.5em}
\pstart
           mein lieber Hugo, Sie werden schon von unserer Freundin-Hofrätin\pwindex{Zuckerkandl, Berta 13.04.1864 – 16.10.1945@\textsc{Zuckerkandl, Berta} (13.04.1864 – 16.10.1945), \emph{Journalist/Journalistin, Übersetzer/Übersetzerin}|pwv} gehört haben, wie sehr Ihr
               Brief über die Therese\pwindex{Therese. Chronik eines Frauenlebens@\emph{Therese. Chronik eines Frauenlebens}|pw} mich gefreut hat; – das
               Buch hat, sowohl beim Publikum, als bei den paar Menschen, auf die es \introOben{}mir\introOben{} besonders anko{\geminationm}t, mehr Erfolg
               als ich je hätte vermuthen dürfen. Die Entstehungsgeschichte ist einigermaßen
               merkwürdig, ich erzähle Ihnen einmal mehr davon.\pend
           
\pstart
           – Christiane\pwindex{Zimmer, Christiane 14.05.1902 – 05.01.1987@\textsc{Zimmer, Christiane} (14.05.1902 – 05.01.1987)|pw} war mir immer außerordentlich
               sympathisch – ich glaube das klare, gerade, kluge wahrhaft verläßliche ihres Wesens
               seit jeher gespürt zu haben u bin froh, daſs der rechte Mann\pwindex{Zimmer, Heinrich 06.12.1890 – 20.03.1943@\textsc{Zimmer, Heinrich} (06.12.1890 – 20.03.1943), \emph{Indologe/Indologin}|pwv} die rechte Wahl getroffen hat. Mögen
               Sie ihr bald das Heidelberg\oindex{Heidelberg@\textbf{Heidelberg}, \emph{P.PPLA3}|pw}er Häuschen bauen
               können. Meine Kinder\pwindex{Cappellini, Lili 13.09.1909 – 26.07.1928@\textsc{Cappellini, Lili} (13.09.1909 – 26.07.1928)|pwv}\pwindex{Cappellini, Arnoldo 10.08.1889 – 08.05.1954@\textsc{Cappellini, Arnoldo} (10.08.1889 – 08.05.1954)|pwv} in Venedig\oindex{Venedig@\textbf{Venedig}, \emph{P.PPLA}|pw} haben jetzt etliche
               Wohnungsschwierigkeiten durch einen kläglichen wahrhaft Goldoni\pwindex{Goldoni, Carlo 25.02.1707 – 06.02.1793@\textsc{Goldoni, Carlo} (25.02.1707 – 06.02.1793), \emph{Schriftsteller/Schriftstellerin}|pw}schen Hausherrn\pwindex{Levi @\textsc{Levi}, \emph{Vermieter/Vermieterin, Commandante/Commandantin}|pwv} – (»nur halt daſs er leider lebt«.) – Im übrigen
               sind sie glücklich, und ich hab ihn\pwindex{Cappellini, Arnoldo 10.08.1889 – 08.05.1954@\textsc{Cappellini, Arnoldo} (10.08.1889 – 08.05.1954)|pwv} (von Lili\pwindex{Cappellini, Lili 13.09.1909 – 26.07.1928@\textsc{Cappellini, Lili} (13.09.1909 – 26.07.1928)|pw} gar nicht zu reden) sehr
               gern. Sie wissen, daſs wir drei im Frühjahr eine schöne Reise gemacht haben. Corfu\oindex{Korfu@\textbf{Korfu}, \emph{Insel (N.INS)}|pw}, Athen\oindex{Athen@\textbf{Athen}, \emph{P.PPLC}|pw}, Kon{\pb}stantinopel\oindex{Istanbul@\textbf{Istanbul}, \emph{A.ADM1}|pw}, Rhodus\oindex{Rhodos@\textbf{Rhodos}, \emph{P.PPLA2}|pw}. Jetzt war Heini\pwindex{Schnitzler, Heinrich 09.08.1902 – 12.07.1982@\textsc{Schnitzler, Heinrich} (09.08.1902 – 12.07.1982), \emph{Regisseur/Regisseurin, Schauspieler/Schauspielerin}|pw} 10 Tage bei mir, und ich habe viel Freude
               von ihm gehabt.\pend
           
\pstart
           Die So{\geminationm}ermonate wer\textcolor{gray}{d} ich wohl hier
               verbringen; ich sehe recht viel Menschen, insbesondere Amerika\oindex{Amerika@\textbf{Amerika}, \emph{kein passender Code gefunden}|pw} findet sich in zahlreichen, oft verständnisvollen Exemplaren ein.
               Mit dem Arbeiten geht es ganz leidlich, aber Dilettant, der ich bin und bleibe, spiel
               ich mich mit Figuren und Stoffen mehr herum, – und eigentlich lieber, als daſs ich
               die Dictatur meines sogenannten Talentes oder wie wir es nennen wollen über sie
               ausübe. Immerhin wird gelegentlich schon wieder was herausko{\geminationm}en, und ans Geldverdienen muſs man ja leider immer
               ernstlicher und continuirlicher denken.\pend
           
\pstart
           Die aegyptische\pwindex{aegyptische Helena@\emph{Die ägyptische Helena}|pwv} hab ich
               natürlich schon gekannt; in der \label{K_L02504-1v}\edtext{Oper\oindex{Oper@\textbf{Oper}, \emph{Oper (K.OPR)}|pw}}{\lemma{\textnormal{\emph{Oper}}}\Cendnote{\textnormal{Die Wien\oindex{Wien@\textbf{Wien}, \emph{A.ADM2}|pwk}er Erstaufführung von \emph{Die ägyptische
                     Helena}\pwindex{aegyptische Helena@\emph{Die ägyptische Helena}|pwk} fand am 11. 6. 1928 statt, Schnitzler war aber erst am 23. 6. 1928 in der Vorstellung.}}}\label{K_L02504-1} hab
               ich einen schönen Eindruck gehabt, und es war mir über alle Maßen interessant, Ihre
               Dichtung so für mich hin zu lesen – und daſs Musik mir immer mitklang, spricht für
               Dichter wie für Musiker. Es ist unglaublich, wie Ihre Sprache Möglichkeiten u
               Einfälle des Componisten oft vorauszuahnen scheint; es ist wahrhaftig Dichtung für
               Musik und aus Musik zugleich. Die beiden Akte sind mir {\pb}jeder für sich, einleuchtender, als in ihrem innern Zusa{\geminationm}enhang; das ganze Problem hat mich sehr bewegt, und ich
               denke, Sie hätten es noch tiefer erschöpft, we{\geminationn} Sie sich
               – ohne jeden Gedanken \substVorne{}\textsuperscript{an die}\substDazwischen{}an\substHinten{} un\textcolor{gray}{d} ohne jede Rücksicht auf Melodisirung \substVorne{}\textsuperscript{,}\substDazwischen{}un\textcolor{gray}{d}\substHinten{} auf Operisierung Ihrem dramatischen Ingenium hätten hingeben dürfen (wie ich
               derartiges in Ihren einleitenden \label{K_L02504-2v}\edtext{Worten\pwindex{aegyptische Helena«@\emph{»Die ägyptische Helena«}|pwv}}{\lemma{\textnormal{\emph{Worten}}}\Cendnote{\textnormal{Hugo von Hofmannsthal\pwindex{Hofmannsthal, Hugo von 1874-02-01 – 1929-07-15@\textsc{Hofmannsthal, Hugo von} (1874-02-01 – 1929-07-15), \emph{Schriftsteller/Schriftstellerin}|pwk}: \emph{»Die ägyptische Helena«}\pwindex{aegyptische Helena«@\emph{»Die ägyptische Helena«}|pwk}. In: \emph{Neue Freie Presse}\pwindex{Neue Freie Presse@\emph{Neue Freie Presse}|pwk}, Nr. 22.832,
                        8. 4. 1928, S. 31–33.}}}\label{K_L02504-2}, schon in d N. Fr Pr.\orgindex{Neue Freie Presse@Neue Freie Presse|pw} wunderbar angedeutet fand.). Nur mit den
               Liebestränken, besonders den Dosirungsmöglichkeiten konnt ich mich nicht befreunden;
               irgendwo in mir steckt doch ein Pedant und Rationalist und der Teufel soll mich
               holen, am Ende gar ein Recensent.\pend
           
\pstart
           Nun mein lieber Hugo lassen Sie sich nochmals danken – und nach allen Richtungen
               bestes und gutes wünschen. Und wer weiſs vielleicht sieht man sich sogar wieder
               einmal.{\\[\baselineskip]}Ihr getreuer{\\[\baselineskip]}\spacefill\mbox{Arth}\pend
           \leftskip=0em{}\selectlanguage{ngerman}\endnumbering\briefempfaengerindex{Hofmannsthal, Hugo von@\textsc{Hofmannsthal, Hugo von}!zzzSchnitzler, Arthur@\emph{von Arthur Schnitzler}!1928-07-211@{21. 7. 1928}|)be}\mylabel{L02504h}  \normalsize

\doendnotes{C}
\bigskip
\vfill

\clearpage

\footnotesize

\lohead{\textsc{register}}

% Definiere theindex-Environment komplett neu ohne reledmac
\makeatletter
\renewenvironment{theindex}{%
  \section*{\indexname}%
  \setlength{\parindent}{0pt}%
  \setlength{\parskip}{0pt plus 0.3pt}%
  \let\item\@idxitem
}{%
  \clearpage
}
\makeatother

\IfFileExists{\jobname-pw.ind}{\input{\jobname-pw.ind}}{}

\end{document}

      