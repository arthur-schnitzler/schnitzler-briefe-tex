%% latex-leseansicht-vorspann.tex
%% Vorspann für die Leseansicht.
%% Lädt die gemeinsame Datei latex-vorspann.tex mit nicht gesetztem Schalter.

\newif\ifkorrekturansicht
\korrekturansichtfalse

\input{../tex-inputs/latex-vorspann}


\section[Arthur Schnitzler an Robert Adam, 7. 2. 1911]{L02004 Arthur Schnitzler an Robert Adam, 7. 2. 1911}
\nopagebreak\mylabel{L02004v}
\rehead{ }\normalsize\beginnumbering\briefempfaengerindex{Adam, Robert@\textsc{Adam, Robert}!zzzSchnitzler, Arthur@\emph{von Arthur Schnitzler}!1911-02-071@{7. 2. 1911}|(be}
\toendnotes[C]{\smallbreak\pagebreak[2]}
\correspDesc{Versand  durch Arthur Schnitzler am 7. 2. 1911 in Wien
\newline{}Erhalt  durch Robert Adam im Zeitraum [7. 2. 1911
                  – 11. 2. 1911?] in Wien}\toendnotes[C]{\smallbreak}
\Standort{DLA, 96.34.1/4.}
\physDesc{Bildpostkarte, 214 Zeichen
\newline{}Handschrift: schwarze Tinte, deutsche Kurrent
\newline{}Versand: Stempel: »\nobreak{}\oindex{IX., Alsergrund@\textbf{IX., Alsergrund}, \emph{Verwaltungsgebiet}|pwk}9/1 Wien, 7. {[}2. 1911{]}, 5\nobreak{}«.  }\pstart{}{\pb}Herrn \textsc{Robert Adam},\pend{}\pstart{}Wien XII\oindex{XII., Meidling@\textbf{XII., Meidling}, \emph{Verwaltungsgebiet}|pw}.\pend{}\pstart{}\textsc{Meidlinger Hptstr 56}\oindex{Wien@\textbf{Wien}!XII., Meidling@\textbf{XII., Meidling}!Meidlinger Hauptstraße@\textbf{Meidlinger Hauptstraße}, \emph{Straße}|pw}.\pend{}{\bigskip}
\pstart
           \noindent{}\centering{}{\pb}\textcolor{gray}{\textbf{Türkenschanz-Park\oindex{Wien@\textbf{Wien}!XVIII., Währing@\textbf{XVIII., Währing}!Türkenschanzpark@\textbf{Türkenschanzpark}, \emph{Park}|pw}{ }Aussichtsturm. Wien XVIII/1\oindex{XVIII., Währing@\textbf{XVIII., Währing}, \emph{Verwaltungsgebiet}|pw}}}\pend
           \vspace{1em}
\pstart
           \noindent{}{\pb}Beſten Dank für die freundlichen Mittheilungen. Aller
               Anfang {\dotsfour} Aber ich will nicht gar zu originell{ }ſein.\pend
           
\pstart
           Verbindlichſt grüßt{\\[\baselineskip]}Ihr{ }ſehr ergebner{\\[\baselineskip]}\spacefill\mbox{Arthur Schnitzler}\pend
           \leftskip=0em{}
\pstart
           \raggedleft{}7/2. 911.\pend
           \selectlanguage{ngerman}\endnumbering\briefempfaengerindex{Adam, Robert@\textsc{Adam, Robert}!zzzSchnitzler, Arthur@\emph{von Arthur Schnitzler}!1911-02-071@{7. 2. 1911}|)be}\mylabel{L02004h}  \newcommand{\dateiname}{L02004}\newcommand{\titel}{Arthur Schnitzler an Robert Adam, 7. 2. 1911}\newcommand{\editorInnen}{Martin Anton Müller und Gerd-Hermann Susen}%% latex-leseansicht-abspann.tex
%% Abspann für die Leseansicht.
%% Der Schalter \ifkorrekturansicht ist bereits durch den Vorspann gesetzt.

%% latex-abspann.tex
%% Gemeinsamer Abspann für Korrekturansicht und Leseansicht.
%% Setzt den Schalter \ifkorrekturansicht voraus (gesetzt in den
%% einbindenden Dateien latex-korrekturansicht-abspann.tex bzw.
%% latex-leseansicht-abspann.tex).
%% ---------------------------------------------------------------

\normalsize

% Das esempio-Environment wird nur in der Leseansicht benötigt
\ifkorrekturansicht\else
\newenvironment{esempio}[3]%
{
    \vspace{1.5ex}
    \rlap{\underline{#1}}
    \par
    \setlength{\parindent}{0cm}
    \nopagebreak
    \leftskip=#2cm
    \rightskip=#3cm
}
{
    \par
}
\fi

\doendnotes{C}
\bigskip
\vfill

\clearpage

\footnotesize

\ifkorrekturansicht
  \lohead{\textsc{register}}
\fi

% theindex-Environment neu definieren ohne reledmac
\makeatletter
\renewenvironment{theindex}{%
  \ifkorrekturansicht
    \section*{\indexname}%
  \else
    \subsubsection*{Index der erwähnten Entitäten}%
  \fi
  \setlength{\parindent}{0pt}%
  \setlength{\parskip}{0pt plus 0.3pt}%
  \let\item\@idxitem
}{%
  \ifkorrekturansicht\clearpage\fi
}
\makeatother

\IfFileExists{\jobname-pw.ind}{\input{\jobname-pw.ind}}{}

% Quellenangabe nur in der Leseansicht
\ifkorrekturansicht\else
% Fallback-Definitionen, falls die .tex-Datei \titel etc. nicht gesetzt hat
\providecommand{\titel}{}
\providecommand{\editorInnen}{}
\providecommand{\dateiname}{\jobname}

\vspace{3cm}

\vfill

\footnotesize
\textsc{Quelle}: \titel. Herausgegeben von {\editorInnen}. In: \emph{Arthur Schnitzler: Briefwechsel mit Autorinnen und Autoren}.
 Digitale Edition, https://schnitzler-briefe.acdh.oeaw.ac.at/{\dateiname}.html (Stand \today)
\fi

\end{document}


