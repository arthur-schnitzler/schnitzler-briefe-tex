%% latex-leseansicht-vorspann.tex
%% Vorspann für die Leseansicht.
%% Lädt die gemeinsame Datei latex-vorspann.tex mit nicht gesetztem Schalter.

\newif\ifkorrekturansicht
\korrekturansichtfalse

\input{../tex-inputs/latex-vorspann}


\section[Arthur Schnitzler an Thomas Mann, 18. 11. 1923]{L02404 Arthur Schnitzler an Thomas Mann, 18. 11. 1923}
\nopagebreak\mylabel{L02404v}
\rehead{ }\normalsize\beginnumbering\briefempfaengerindex{Mann, Thomas@\textsc{Mann, Thomas}!zzzSchnitzler, Arthur@\emph{von Arthur Schnitzler}!1923-11-181@{18. 11. 1923}|(be}
\toendnotes[C]{\smallbreak\pagebreak[2]}
\correspDesc{Versand  durch Arthur Schnitzler am 18. 11. 1923 in Wien
\newline{}Erhalt  durch Thomas Mann im Zeitraum [19. 11. 1923 – 23. 11. 1923?] in München}\toendnotes[C]{\smallbreak}
\Standort{Zürich, Thomas-Mann-Archiv, B-II-SCHNM-2.}
\physDesc{Brief, 1 Blatt, 2 Seiten, Kuvert, 1354 Zeichen
\newline{}Handschrift: schwarze Tinte, lateinische Kurrent}\toendnotes[C]{\smallbreak}\pstart{}{\pb}\label{T_L02404-1v}\edtext{\textcolor{gray}{\textbf{A. S.}}}{\lemma{\textnormal{\emph{A. S.}}}\Cendnote{\textnormal{ovaler Absenderkleber}}}\label{T_L02404-1}\pend{}\pstart{}\textcolor{gray}{\textbf{WIEN, XVIII.}}\oindex{XVIII., Währing@\textbf{XVIII., Währing}, \emph{Verwaltungsgebiet}|pw}\pend{}\pstart{}\textcolor{gray}{\textbf{STERNWARTESTR. 71}}\oindex{Wien@\textbf{Wien}!XVIII., Währing@\textbf{XVIII., Währing}!Sternwartestraße 71@\textbf{Sternwartestraße 71}, \emph{Wohngebäude}|pw}\pend{}{\bigskip}\pstart{}{\pb}Herrn Thomas Mann\pend{}\pstart{}München\oindex{München@\textbf{München}|pw}\pend{}\pstart{}\strikeout{Puch}{ }{[}ms.:{]} Puschingerstr. 1\oindex{Poschingerstraße@\textbf{Poschingerstraße}, \emph{Straße}|pw}. \pend{}{\bigskip}\vspace{1em}
\pstart
           \raggedleft{}{\pb}Wien\oindex{Wien@\textbf{Wien}, \emph{Verwaltungsgebiet}|pw}, 18. 11. 923\pend
           
\pstart{}lieber und verehrter Herr Thomas Mann,\pend\vspace{0.5em}
\pstart
           dürft ich mir im geringsten das Recht und die Kraft zugestehen, Sie zu Fortführung u
               Beendigung des Felix Krull\pwindex{Mann, Thomas 6.\,6.\,1875 Lübeck – 12.\,8.\,1955 Zürich@\textsc{Mann, Thomas} (6.\,6.\,1875 Lübeck – 12.\,8.\,1955 Zürich), \emph{Schriftsteller}!Bekenntnisse des Hochstaplers Felix Krull@\strich\emph{Bekenntnisse des Hochstaplers Felix Krull}|pw} anzuspornen, ich thät
               es, we{\geminationn} man so sagen darf, aus vollen Stiefeln. Das Fragment\pwindex{Mann, Thomas 6.\,6.\,1875 Lübeck – 12.\,8.\,1955 Zürich@\textsc{Mann, Thomas} (6.\,6.\,1875 Lübeck – 12.\,8.\,1955 Zürich), \emph{Schriftsteller}!Bekenntnisse des Hochstaplers Felix Krull@\strich\emph{Bekenntnisse des Hochstaplers Felix Krull}|pwv}, das vorliegt, find
               ich köstlich und kostbar. Ich weiß nicht, ob Sie selbst (verzeihen Sie die Anmaßung)
               die völlige Einzigartigkeit Ihrer Stimme so zu spüren im Stande sind, wie der Leser –
               aber ich wünschte, daß Sie das »Buch der Kindheit« einmal nur als Kenner und
               Genießer, nicht nebstbei als der Verfasser sich zu Gemüthe führten, – Sie hätten die
               reinste Freude und empfänden die Pflicht und den Drang zu
                  »\textcolor{gray}{erinnern}«, – wie ich sie empfand.\pend
           
\pstart
           Ich wünschte zum Beschluss dieser Zeilen {\pb}nicht von der Stadt\oindex{München@\textbf{München}|pwv} reden,
               in der Sie leben, von der Welt, in der wir alle leben – nur die Hoffnung aussprechen,
               daß Sie mit den Ihren sich so wohl befinden, als es überhaupt möglich. Man erzählt
               sich, dß Sie bald nach Wien\oindex{Wien@\textbf{Wien}, \emph{Verwaltungsgebiet}|pw} kommen wollen. Wir
               sehen einander hoffentlich gewiss wieder.\pend
           
\pstart
           Seien Sie, mit Ihrer verehrten Gattin\pwindex{Mann, Katia 24.\,7.\,1883 Feldafing – 25.\,4.\,1980 Kilchberg@\textsc{Mann, Katia} (24.\,7.\,1883 Feldafing – 25.\,4.\,1980 Kilchberg)|pwv}{\\[\baselineskip]}sehr herzlich gegrüßt von Ihrem{\\[\baselineskip]}freundschaftlich ergebenen{\\[\baselineskip]}\spacefill\mbox{Arthur Schnitzler}\pend
           \leftskip=0em{}
\pstart
           \noindent{}{[}({]}Darf ich vielleicht auch noch erwähnen, daß mein 21jähriger
                     Sohn\pwindex{Schnitzler, Heinrich 9.\,8.\,1902 Hinterbrühl – 12.\,7.\,1982 Wien@\textsc{Schnitzler, Heinrich} (9.\,8.\,1902 Hinterbrühl – 12.\,7.\,1982 Wien), \emph{Regisseur, Schauspieler}|pwv}, wie meine
                  14jährige Tochter\pwindex{Cappellini, Lili 13.\,9.\,1909 Wien – 26.\,7.\,1928 Venedig@\textsc{Cappellini, Lili} (13.\,9.\,1909 Wien – 26.\,7.\,1928 Venedig)|pwv} (die ein
                  bischen über ihre Jahre hinaus ist) von Ihrem Fragment\pwindex{Mann, Thomas 6.\,6.\,1875 Lübeck – 12.\,8.\,1955 Zürich@\textsc{Mann, Thomas} (6.\,6.\,1875 Lübeck – 12.\,8.\,1955 Zürich), \emph{Schriftsteller}!Bekenntnisse des Hochstaplers Felix Krull@\strich\emph{Bekenntnisse des Hochstaplers Felix Krull}|pwv} in gleicher Weise entzückt waren?)\pend
           \selectlanguage{ngerman}\endnumbering\briefempfaengerindex{Mann, Thomas@\textsc{Mann, Thomas}!zzzSchnitzler, Arthur@\emph{von Arthur Schnitzler}!1923-11-181@{18. 11. 1923}|)be}\mylabel{L02404h}  \newcommand{\dateiname}{L02404}\newcommand{\titel}{Arthur Schnitzler an Thomas Mann, 18. 11. 1923}\newcommand{\editorInnen}{Martin Anton Müller und Gerd-Hermann Susen}%% latex-leseansicht-abspann.tex
%% Abspann für die Leseansicht.
%% Der Schalter \ifkorrekturansicht ist bereits durch den Vorspann gesetzt.

%% latex-abspann.tex
%% Gemeinsamer Abspann für Korrekturansicht und Leseansicht.
%% Setzt den Schalter \ifkorrekturansicht voraus (gesetzt in den
%% einbindenden Dateien latex-korrekturansicht-abspann.tex bzw.
%% latex-leseansicht-abspann.tex).
%% ---------------------------------------------------------------

\normalsize

% Das esempio-Environment wird nur in der Leseansicht benötigt
\ifkorrekturansicht\else
\newenvironment{esempio}[3]%
{
    \vspace{1.5ex}
    \rlap{\underline{#1}}
    \par
    \setlength{\parindent}{0cm}
    \nopagebreak
    \leftskip=#2cm
    \rightskip=#3cm
}
{
    \par
}
\fi

\doendnotes{C}
\bigskip
\vfill

\clearpage

\footnotesize

\ifkorrekturansicht
  \lohead{\textsc{register}}
\fi

% theindex-Environment neu definieren ohne reledmac
\makeatletter
\renewenvironment{theindex}{%
  \ifkorrekturansicht
    \section*{\indexname}%
  \else
    \subsubsection*{Index der erwähnten Entitäten}%
  \fi
  \setlength{\parindent}{0pt}%
  \setlength{\parskip}{0pt plus 0.3pt}%
  \let\item\@idxitem
}{%
  \ifkorrekturansicht\clearpage\fi
}
\makeatother

\IfFileExists{\jobname-pw.ind}{\input{\jobname-pw.ind}}{}

% Quellenangabe nur in der Leseansicht
\ifkorrekturansicht\else
% Fallback-Definitionen, falls die .tex-Datei \titel etc. nicht gesetzt hat
\providecommand{\titel}{}
\providecommand{\editorInnen}{}
\providecommand{\dateiname}{\jobname}

\vspace{3cm}

\vfill

\footnotesize
\textsc{Quelle}: \titel. Herausgegeben von {\editorInnen}. In: \emph{Arthur Schnitzler: Briefwechsel mit Autorinnen und Autoren}.
 Digitale Edition, https://schnitzler-briefe.acdh.oeaw.ac.at/{\dateiname}.html (Stand \today)
\fi

\end{document}


