%% latex-leseansicht-vorspann.tex
%% Vorspann für die Leseansicht.
%% Lädt die gemeinsame Datei latex-vorspann.tex mit nicht gesetztem Schalter.

\newif\ifkorrekturansicht
\korrekturansichtfalse

\input{../tex-inputs/latex-vorspann}


               \section[Arthur Schnitzler an Thomas Mann, 18. 11. 1923]{ Arthur Schnitzler an Thomas Mann, 18. 11. 1923}\nopagebreak\mylabel{v}\rehead{ }\begin{ledgroupsized}[t]{13cm}\normalsize\beginnumbering\briefempfaengerindex{Mann, Thomas@\textsc{Mann, Thomas}!zzzSchnitzler, Arthur@\emph{von Arthur Schnitzler}!1923-11-181@{18. 11. 1923}|(be} \toendnotes[C]{\smallbreak\pagebreak[2]} \Standort{Zürich, Thomas-Mann-Archiv, B-II-SCHNM-2.}
\physDesc{Brief, 1 Blatt, 2 Seiten, Umschlag
\newline{}Handschrift: schwarze Tinte, lateinische Kurrent}\toendnotes[C]{\smallbreak}\pstart{}{\pb}\label{T_L02404-1v}\edtext{\textcolor{gray}{\textbf{A. S.}}}{\lemma{\textnormal{\emph{A. S.}}}\Cendnote{\textnormal{ovaler Absenderkleber}}}\label{T_L02404-1h}\pend{}\pstart{}\textcolor{gray}{\textbf{WIEN, XVIII.}}\oindex{XVIII., Waehring@\textbf{XVIII., Währing}|pw}\pend{}\pstart{}\textcolor{gray}{\textbf{STERNWARTESTR. 71}}\oindex{Sternwartestrasse@\textbf{Sternwartestraße}|pw}\pend{}{\bigskip}\pstart{}{\pb}Herrn Thomas Mann\pend{}\pstart{}München\oindex{Muenchen@\textbf{München}|pw}\pend{}\pstart{}\strikeout{Puch}{ }{[}ms.:{]} Puschingerstr. 1\oindex{Poschingerstrasse@\textbf{Poschingerstraße}|pw}. \pend{}{\bigskip}\pstart
           \raggedleft{}{\pb}Wien\oindex{Wien@\textbf{Wien}|pw}, 18. 11. 923\pend
           \pstart{}lieber und verehrter Herr Thomas Mann,\pend\pstart
           dürft ich mir im geringsten das Recht und die Kraft zugestehen, Sie zu
                    Fortführung u Beendigung des Felix Krull\pwindex{Mann, Thomas 06.06.1875 – 12.08.1955@\textsc{Mann, Thomas} (06.06.1875 – 12.08.1955), \emph{Schriftsteller}!Bekenntnisse des Hochstaplers Felix Krull1922@\strich\emph{Bekenntnisse des Hochstaplers Felix Krull} {[}1922{]}|pw}
                    anzuspornen, ich thät es, we{\geminationn} man so sagen darf,
                    aus vollen Stiefeln. Das Fragment\pwindex{Mann, Thomas 06.06.1875 – 12.08.1955@\textsc{Mann, Thomas} (06.06.1875 – 12.08.1955), \emph{Schriftsteller}!Bekenntnisse des Hochstaplers Felix Krull1922@\strich\emph{Bekenntnisse des Hochstaplers Felix Krull} {[}1922{]}|pwv}, das vorliegt, find ich köstlich und kostbar. Ich weiß nicht,
                    ob Sie selbst (verzeihen Sie die Anmaßung) die völlige Einzigartigkeit Ihrer
                    Stimme so zu spüren im Stande sind, wie der Leser – aber ich wünschte, daß Sie
                    das »Buch der Kindheit« einmal nur als Kenner und Genießer, nicht nebstbei als
                    der Verfasser sich zu Gemüthe führten, – Sie hätten die reinste Freude und
                    empfänden die Pflicht und den Drang zu »\textcolor{gray}{erinnern}«, – wie ich
                    sie empfand.\pend
           \pstart
           Ich wünschte zum Beschluss dieser Zeilen {\pb}nicht von der Stadt\oindex{Muenchen@\textbf{München}|pwv}
                    reden, in der Sie leben, von der Welt, in der wir alle leben – nur die Hoffnung
                    aussprechen, daß Sie mit den Ihren sich so wohl befinden, als es überhaupt
                    möglich. Man erzählt sich, dß Sie bald nach Wien\oindex{Wien@\textbf{Wien}|pw}
                    kommen wollen. Wir sehen einander hoffentlich gewiss wieder.\pend
           \pstart
           Seien Sie, mit Ihrer verehrten Gattin\pwindex{Mann, Katia 24.07.1883 – 25.04.1980@\textsc{Mann, Katia} (24.07.1883 – 25.04.1980)|pwv}{\\[\baselineskip]}sehr herzlich gegrüßt von Ihrem{\\[\baselineskip]}freundschaftlich ergebenen{\\[\baselineskip]}\spacefill\mbox{Arthur Schnitzler}\pend
           \leftskip=0em{}\pstart
           \noindent{}{[}({]}Darf ich vielleicht auch noch erwähnen, daß mein
                        21jähriger Sohn\pwindex{Schnitzler, Heinrich 09.08.1902 – 12.07.1982@\textsc{Schnitzler, Heinrich} (09.08.1902 – 12.07.1982), \emph{Regisseur, Schauspieler}|pwv}, wie
                        meine 14jährige Tochter\pwindex{Schnitzler, Lili 13.09.1909 – 26.07.1928@\textsc{Schnitzler, Lili} (13.09.1909 – 26.07.1928)|pwv}
                        (die ein bischen über ihre Jahre hinaus ist) von Ihrem Fragment\pwindex{Mann, Thomas 06.06.1875 – 12.08.1955@\textsc{Mann, Thomas} (06.06.1875 – 12.08.1955), \emph{Schriftsteller}!Bekenntnisse des Hochstaplers Felix Krull1922@\strich\emph{Bekenntnisse des Hochstaplers Felix Krull} {[}1922{]}|pwv} in gleicher Weise entzückt
                        waren?) \pend
                     \endnumbering\briefempfaengerindex{Mann, Thomas@\textsc{Mann, Thomas}!zzzSchnitzler, Arthur@\emph{von Arthur Schnitzler}!1923-11-181@{18. 11. 1923}|)be}\mylabel{h}\end{ledgroupsized}  \newcommand{\dateiname}{L02404}\newcommand{\titel}{Arthur Schnitzler an Thomas Mann, 18. 11. 1923}\newcommand{\editorInnen}{Martin Anton Müller und Gerd-Hermann Susen}
            \footnotesize
\begin{ledgroupsized}[t]{11.5cm}
\doendnotes{C}
\end{ledgroupsized}
         %% latex-leseansicht-abspann.tex
%% Abspann für die Leseansicht.
%% Der Schalter \ifkorrekturansicht ist bereits durch den Vorspann gesetzt.

%% latex-abspann.tex
%% Gemeinsamer Abspann für Korrekturansicht und Leseansicht.
%% Setzt den Schalter \ifkorrekturansicht voraus (gesetzt in den
%% einbindenden Dateien latex-korrekturansicht-abspann.tex bzw.
%% latex-leseansicht-abspann.tex).
%% ---------------------------------------------------------------

\normalsize

% Das esempio-Environment wird nur in der Leseansicht benötigt
\ifkorrekturansicht\else
\newenvironment{esempio}[3]%
{
    \vspace{1.5ex}
    \rlap{\underline{#1}}
    \par
    \setlength{\parindent}{0cm}
    \nopagebreak
    \leftskip=#2cm
    \rightskip=#3cm
}
{
    \par
}
\fi

\doendnotes{C}
\bigskip
\vfill

\clearpage

\footnotesize

\ifkorrekturansicht
  \lohead{\textsc{register}}
\fi

% theindex-Environment neu definieren ohne reledmac
\makeatletter
\renewenvironment{theindex}{%
  \ifkorrekturansicht
    \section*{\indexname}%
  \else
    \subsubsection*{Index der erwähnten Entitäten}%
  \fi
  \setlength{\parindent}{0pt}%
  \setlength{\parskip}{0pt plus 0.3pt}%
  \let\item\@idxitem
}{%
  \ifkorrekturansicht\clearpage\fi
}
\makeatother

\IfFileExists{\jobname-pw.ind}{\input{\jobname-pw.ind}}{}

% Quellenangabe nur in der Leseansicht
\ifkorrekturansicht\else
% Fallback-Definitionen, falls die .tex-Datei \titel etc. nicht gesetzt hat
\providecommand{\titel}{}
\providecommand{\editorInnen}{}
\providecommand{\dateiname}{\jobname}

\vspace{3cm}

\vfill

\footnotesize
\textsc{Quelle}: \titel. Herausgegeben von {\editorInnen}. In: \emph{Arthur Schnitzler: Briefwechsel mit Autorinnen und Autoren}.
 Digitale Edition, https://schnitzler-briefe.acdh.oeaw.ac.at/{\dateiname}.html (Stand \today)
\fi

\end{document}


      