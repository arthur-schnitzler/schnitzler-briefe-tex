%% latex-leseansicht-vorspann.tex
%% Vorspann für die Leseansicht.
%% Lädt die gemeinsame Datei latex-vorspann.tex mit nicht gesetztem Schalter.

\newif\ifkorrekturansicht
\korrekturansichtfalse

\input{../tex-inputs/latex-vorspann}


\section[Stefan Zweig an Arthur Schnitzler, 25. 11. 1915]{L03652 Stefan Zweig an Arthur Schnitzler, 25. 11. 1915}
\nopagebreak\mylabel{L03652v}
\rehead{ }\normalsize\beginnumbering\briefempfaengerindex{Schnitzler, Arthur@\textsc{Schnitzler, Arthur}!zzzZweig, Stefan@\emph{von Stefan Zweig}!1915-11-251@{25. 11. 1915}|(be}
\toendnotes[C]{\smallbreak\pagebreak[2]}
\correspDesc{Versand  durch Stefan Zweig am 25. 11. 1915 in Wien
\newline{}Übermittlung  am 26. 11. 1915 in Wien
\newline{}Erhalt  durch Arthur Schnitzler im Zeitraum [26. 11. 1915 – 30. 11. 1915?] in Wien}\toendnotes[C]{\smallbreak}
\Standort{CUL, Schnitzler, B 118.}
\physDesc{Bildpostkarte, 478 Zeichen
\newline{}Handschrift: schwarze Tinte, lateinische Kurrent
\newline{}Versand: Stempel: »\nobreak{}\oindex{VII., Neubau@\textbf{VII., Neubau}, \emph{Verwaltungsgebiet}|pwk}7 \textcolor{gray}{Wien}, 26. 11. 15, 4\nobreak{}«.  }
\buchAbdrucke{\weitereDrucke{Stefan Zweig: \emph{Briefwechsel mit Hermann Bahr, Sigmund Freud, Rainer Maria
                        Rilke und Arthur Schnitzler}. Herausgegeben von Jeffrey B. Berlin, Hans-Ulrich Lindken und Donald A. Prater. Frankfurt am Main: \emph{S. Fischer} 1987, S. 398.} }\toendnotes[C]{\smallbreak}\pstart{}{\pb}D\textsuperscript{r}
                  Arthur Schnitzler\pend{}\pstart{}Wien – Cottage\oindex{Wien@\textbf{Wien}!XVIII., Währing@\textbf{XVIII., Währing}!Währinger Cottage@\textbf{Währinger Cottage}, \emph{Teil eines besiedelten Ortes}|pw}\pend{}\pstart{}\label{K_L03652-1v}\edtext{Sternwartestrasse 72}{\lemma{\textnormal{\emph{Sternwartestrasse 72}}}\Cendnote{\textnormal{Zweig\pwindex{Zweig, Stefan 28.\,11.\,1881 Wien – 23.\,2.\,1942 Petrópolis@\textsc{Zweig, Stefan} (28.\,11.\,1881 Wien – 23.\,2.\,1942 Petrópolis), \emph{Schriftsteller}|pwk} wechselt
                        bei der Adressierung seiner Schreiben an Schnitzler immer wieder zwischen der falschen Hausnummer
                           »72« und der richtigen
                  »71«.}}}\label{K_L03652-1}\oindex{Wien@\textbf{Wien}!XVIII., Währing@\textbf{XVIII., Währing}!Sternwartestraße 71@\textbf{Sternwartestraße 71}, \emph{Wohngebäude}|pw}\pend{}{\bigskip}
\pstart
           \noindent{}\centering{}{\pb}\textcolor{gray}{\textbf{Wien – Schönbrunn, röm. Ruine\oindex{Wien@\textbf{Wien}!XIII., Hietzing@\textbf{XIII., Hietzing}!Römische Ruine [Schlosspark Schönbrunn]@\textbf{Römische Ruine [Schlosspark Schönbrunn]}, \emph{Monument}|pw}}}\pend
           \vspace{1em}
\pstart
           \noindent{}{\pb}Lieber verehrter Herr
                  Doktor, am 29. Januar ist Romain Rollands\pwindex{Rolland, Romain 29.\,1.\,1866 Clamecy – 30.\,12.\,1944 Vézelay@\textsc{Rolland, Romain} (29.\,1.\,1866 Clamecy – 30.\,12.\,1944 Vézelay), \emph{Schriftsteller}|pw} fünfzigster Geburtstag. Seine Freunde und
               alle, die ihm für seine menschliche Haltung in dieser Zeit dankbar sind, wollen ihm
               zu diesem Tage ein Wort \label{K_L03652-2v}\edtext{telegrafieren}{\lemma{\textnormal{\emph{telegrafieren}}}\Cendnote{\textnormal{XXXX Auszeichnungsfehler: Dokument L04214 nicht gefunden.}}}\label{K_L03652-2}. Ist es auch Ihre Absicht, so sage ich Ihnen
               auf jeden Fall seine Adresse \uline{Genf–Champel}, Hotel Beau Sejour\oindex{Hôtel Beau-Séjour@\textbf{Hôtel Beau-Séjour}, \emph{Hotel}|pw}. \label{K_L03652-3v}\edtext{Gestern\eventindex{Wiener Konzerthaus@\textbf{Wiener Konzerthaus}!1. Sonatenabend von Arnold Rosé und Bruno Walter, 24.11.1915@1. Sonatenabend von Arnold Rosé und Bruno Walter, 24.11.1915|pwv} sah ich Sie von ferne bei Rosé\pwindex{Rosé, Arnold 24.\,10.\,1863 Iași – 25.\,8.\,1946 London@\textsc{Rosé, Arnold} (24.\,10.\,1863 Iași – 25.\,8.\,1946 London), \emph{Violinist}|pw}}{\lemma{\textnormal{\emph{Gestern … Rosé}}}\Cendnote{\textnormal{Am 24. 11. 1915 besuchte Schnitzler
                  einen Sonatenabend von Arnold Rosé\pwindex{Rosé, Arnold 24.\,10.\,1863 Iași – 25.\,8.\,1946 London@\textsc{Rosé, Arnold} (24.\,10.\,1863 Iași – 25.\,8.\,1946 London), \emph{Violinist}|pwk} und Bruno Walter\pwindex{Walter, Bruno 15.\,9.\,1876 Berlin – 17.\,2.\,1962 Beverly Hills@\textsc{Walter, Bruno} (15.\,9.\,1876 Berlin – 17.\,2.\,1962 Beverly Hills), \emph{Theaterleiter, Komponist, Dirigent}|pwk}\eventindex{Wiener Konzerthaus@\textbf{Wiener Konzerthaus}!1. Sonatenabend von Arnold Rosé und Bruno Walter, 24.11.1915@1. Sonatenabend von Arnold Rosé und Bruno Walter, 24.11.1915|pwkv} im Wiener Konzerthaus\oindex{Wien@\textbf{Wien}!III., Landstraße@\textbf{III., Landstraße}!Wiener Konzerthaus@\textbf{Wiener Konzerthaus}, \emph{Konzertsaal}|pwk}. }}}\label{K_L03652-3}. Es war herrlich über alle
               Maassen.\pend
           
\pstart
           Treulichst Ihr{\\[\baselineskip]}\spacefill\mbox{Stefan Zweig}\pend
           \leftskip=0em{}\selectlanguage{ngerman}\endnumbering\briefempfaengerindex{Schnitzler, Arthur@\textsc{Schnitzler, Arthur}!zzzZweig, Stefan@\emph{von Stefan Zweig}!1915-11-251@{25. 11. 1915}|)be}\mylabel{L03652h}  \newcommand{\dateiname}{L03652}\newcommand{\titel}{Stefan Zweig an Arthur Schnitzler, 25. 11. 1915}\newcommand{\editorInnen}{Selma Jahnke und Martin Anton Müller}%% latex-leseansicht-abspann.tex
%% Abspann für die Leseansicht.
%% Der Schalter \ifkorrekturansicht ist bereits durch den Vorspann gesetzt.

%% latex-abspann.tex
%% Gemeinsamer Abspann für Korrekturansicht und Leseansicht.
%% Setzt den Schalter \ifkorrekturansicht voraus (gesetzt in den
%% einbindenden Dateien latex-korrekturansicht-abspann.tex bzw.
%% latex-leseansicht-abspann.tex).
%% ---------------------------------------------------------------

\normalsize

% Das esempio-Environment wird nur in der Leseansicht benötigt
\ifkorrekturansicht\else
\newenvironment{esempio}[3]%
{
    \vspace{1.5ex}
    \rlap{\underline{#1}}
    \par
    \setlength{\parindent}{0cm}
    \nopagebreak
    \leftskip=#2cm
    \rightskip=#3cm
}
{
    \par
}
\fi

\doendnotes{C}
\bigskip
\vfill

\clearpage

\footnotesize

\ifkorrekturansicht
  \lohead{\textsc{register}}
\fi

% theindex-Environment neu definieren ohne reledmac
\makeatletter
\renewenvironment{theindex}{%
  \ifkorrekturansicht
    \section*{\indexname}%
  \else
    \subsubsection*{Index der erwähnten Entitäten}%
  \fi
  \setlength{\parindent}{0pt}%
  \setlength{\parskip}{0pt plus 0.3pt}%
  \let\item\@idxitem
}{%
  \ifkorrekturansicht\clearpage\fi
}
\makeatother

\IfFileExists{\jobname-pw.ind}{\input{\jobname-pw.ind}}{}

% Quellenangabe nur in der Leseansicht
\ifkorrekturansicht\else
% Fallback-Definitionen, falls die .tex-Datei \titel etc. nicht gesetzt hat
\providecommand{\titel}{}
\providecommand{\editorInnen}{}
\providecommand{\dateiname}{\jobname}

\vspace{3cm}

\vfill

\footnotesize
\textsc{Quelle}: \titel. Herausgegeben von {\editorInnen}. In: \emph{Arthur Schnitzler: Briefwechsel mit Autorinnen und Autoren}.
 Digitale Edition, https://schnitzler-briefe.acdh.oeaw.ac.at/{\dateiname}.html (Stand \today)
\fi

\end{document}


