%% latex-korrekturansicht-vorspann.tex
%% Vorspann für die Korrekturansicht.
%% Lädt die gemeinsame Datei latex-vorspann.tex mit gesetztem Schalter.

\newif\ifkorrekturansicht
\korrekturansichttrue

\input{../tex-inputs/latex-vorspann}


\section[Stefan Zweig an Arthur Schnitzler, 25. 11. 1915]{L03652 Stefan Zweig an Arthur Schnitzler, 25. 11. 1915}
\nopagebreak\mylabel{L03652v}
\rehead{ }\normalsize\beginnumbering\briefempfaengerindex{Schnitzler, Arthur@\textsc{Schnitzler, Arthur}!zzzZweig, Stefan@\emph{von Stefan Zweig}!1915-11-251@{25. 11. 1915}|(be}
\toendnotes[C]{\smallbreak\pagebreak[2]}\Standort{CUL, Schnitzler, B 118.}
\physDesc{Bildpostkarte, 480 Zeichen
\newline{}Handschrift: schwarze Tinte, lateinische Kurrent
\newline{}Versand: Stempel: »\nobreak{}\oindex{VII., Neubau@\textbf{VII., Neubau}, \emph{A.ADM3}|pwk}7 \textcolor{gray}{Wien}, 25. 11. 15\nobreak{}«.  }
\buchAbdrucke{\weitereDrucke{Stefan Zweig: \emph{Briefwechsel mit Hermann Bahr, Sigmund Freud, Rainer Maria
                        Rilke und Arthur Schnitzler}. Frankfurt am Main: \emph{S. Fischer} 1987, S. 398.} }\toendnotes[C]{\smallbreak}\pstart{}{\pb}D\textsuperscript{r}
                  Arthur Schnitzler\pend{}\pstart{}Wien – Cottage\oindex{Waehringer Cottage@\textbf{Währinger Cottage}, \emph{Teil eines besiedelten Ortes (A.BSOX)}|pw}\pend{}\pstart{}\label{K_L03652-1v}\edtext{Sternwartestrasse 72}{\lemma{\textnormal{\emph{Sternwartestrasse 72}}}\Cendnote{\textnormal{Zweig\pwindex{Zweig, Stefan 28.11.1881 – 23.02.1942@\textsc{Zweig, Stefan} (28.11.1881 – 23.02.1942), \emph{Schriftsteller/Schriftstellerin}|pwk} wechselt
                        bei der Adressierung seiner Schreiben an Schnitzler immer wieder zwischen der falschen Hausnummer
                           »72« und der richtigen
                  »71«.}}}\label{K_L03652-1}\oindex{Sternwartestrasse 71@\textbf{Sternwartestraße 71}, \emph{Wohngebäude (K.WHS)}|pw}\pend{}{\bigskip}
\pstart
           \noindent{}{\pb}\textcolor{gray}{\textbf{Wien – Schönbrunn, röm. Ruine\oindex{Roemische Ruine [Schlosspark Schoenbrunn]@\textbf{Römische Ruine [Schlosspark Schönbrunn]}, \emph{Monument (K.MON)}|pw}}}\pend
           \vspace{1em}
\pstart
           \noindent{}{\pb}Lieber verehrter Herr
                  Doktor, am 29. Januar ist Romain Rollands\pwindex{Rolland, Romain 29.01.1866 – 30.12.1944@\textsc{Rolland, Romain} (29.01.1866 – 30.12.1944), \emph{Schriftsteller/Schriftstellerin}|pw} fünfzigster Geburtstag. Seine Freunde und
               alle, die ihm für seine menschliche Haltung in dieser Zeit dankbar sind, wollen ihm
               zu diesem Tage ein Wort telegrafieren. Ist es auch Ihre Absicht, so sage ich Ihnen
               auf jeden Fall seine Adresse \uline{Genf–Champel}, Hotel Beau Sejour\oindex{Hôtel Beau-Sejour@\textbf{Hôtel Beau-Séjour}, \emph{Hotel (K.HTL)}|pw}. \label{K_L03652-2v}\edtext{Gestern sah ich Sie von ferne bei Rosé\pwindex{Rose, Arnold 24.10.1863 – 25.08.1946@\textsc{Rosé, Arnold} (24.10.1863 – 25.08.1946), \emph{Violinist/Violinistin}|pw}}{\lemma{\textnormal{\emph{Gestern … Rosé}}}\Cendnote{\textnormal{Am 24. 11. 1915 besuchte Schnitzler
                  einen Sonatenabend von Arnold Rosé\pwindex{Rose, Arnold 24.10.1863 – 25.08.1946@\textsc{Rosé, Arnold} (24.10.1863 – 25.08.1946), \emph{Violinist/Violinistin}|pwk} und Bruno Walter\pwindex{Walter, Bruno 15.09.1876 – 17.02.1962@\textsc{Walter, Bruno} (15.09.1876 – 17.02.1962), \emph{Theaterleiter/Theaterleiterin, Komponist/Komponistin, Dirigent/Dirigentin}|pwk} im Wiener Konzerthaus\oindex{Wiener Konzerthaus@\textbf{Wiener Konzerthaus}, \emph{Konzertsaal (K.KNZ)}|pwk}. }}}\label{K_L03652-2}. Es war herrlich über alle
               Maassen.\pend
           
\pstart
           Treulichst Ihr{\\[\baselineskip]}\spacefill\mbox{Stefan Zweig}\pend
           \leftskip=0em{}\selectlanguage{ngerman}\endnumbering\briefempfaengerindex{Schnitzler, Arthur@\textsc{Schnitzler, Arthur}!zzzZweig, Stefan@\emph{von Stefan Zweig}!1915-11-251@{25. 11. 1915}|)be}\mylabel{L03652h}
\begin{anhang}
\end{anhang}\normalsize

\doendnotes{C}
\bigskip
\vfill

\clearpage

\footnotesize

\lohead{\textsc{register}}

% Definiere theindex-Environment komplett neu ohne reledmac
\makeatletter
\renewenvironment{theindex}{%
  \section*{\indexname}%
  \setlength{\parindent}{0pt}%
  \setlength{\parskip}{0pt plus 0.3pt}%
  \let\item\@idxitem
}{%
  \clearpage
}
\makeatother

\IfFileExists{\jobname-pw.ind}{\input{\jobname-pw.ind}}{}

\end{document}

      