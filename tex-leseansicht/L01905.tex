%% latex-leseansicht-vorspann.tex
%% Vorspann für die Leseansicht.
%% Lädt die gemeinsame Datei latex-vorspann.tex mit nicht gesetztem Schalter.

\newif\ifkorrekturansicht
\korrekturansichtfalse

\input{../tex-inputs/latex-vorspann}


\section[Franz Blei an Arthur Schnitzler, {{[}}17. 12. 1909{{]}}]{L01905 Franz Blei an Arthur Schnitzler, {[}17. 12. 1909{]}}
\nopagebreak\mylabel{L01905v}
\rehead{ }\normalsize\beginnumbering\briefempfaengerindex{Schnitzler, Arthur@\textsc{Schnitzler, Arthur}!zzzBlei, Franz@\emph{von Franz Blei}!1909-12-171@{{[}17. 12. 1909{]}}|(be}
\toendnotes[C]{\smallbreak\pagebreak[2]}
\correspDesc{Versand  durch Franz Blei am [17. 12. 1909] in München
\newline{}Erhalt  durch Arthur Schnitzler im Zeitraum [18. 12. 1909 – 22. 12. 1909?] in Wien}\toendnotes[C]{\smallbreak}
\Standort{CUL, Schnitzler, B 14.}
\physDesc{Brief, 1 Blatt, 1 Seite, 590 Zeichen
\newline{}Handschrift: schwarze Tinte, lateinische Kurrent
\newline{}Schnitzler: 1) mit Bleistift beschriftet: »\textsc{Blei}« und datiert: »17/12 09«  2) mit rotem Buntstift zwei Unterstreichungen
\newline{}Ordnung: mit Bleistift von unbekannter Hand nummeriert:
                                 »4« }\toendnotes[C]{\smallbreak}
\pstart{}{\pb}Sehr verehrter Herr
                  Schnitzler,\pend\vspace{0.5em}
\pstart
           ich danke Ihnen sehr, dass Sie an den Hyper.\orgindex{Hyperion@Hyperion|pw}
               gedacht haben und würde – wie Sie sich denken können – mit Freuden das Vorspiel\pwindex{Schnitzler, Arthur 15.\,5.\,1862 Wien – 21.\,10.\,1931 ebd.@\textsc{Schnitzler, Arthur} (15.\,5.\,1862 Wien – 21.\,10.\,1931 ebd.), \emph{Schriftsteller, Mediziner}!junge Medardus. Dramatische Historie in einem Vorspiel und fünf Aufzügen@\strich\emph{Der junge Medardus. Dramatische Historie in einem Vorspiel und fünf Aufzügen}|pwv} drucken, wenn der Verleger\pwindex{Weber, Hans von 22.\,4.\,1872 Dresden – 22.\,4.\,1924 München@\textsc{Weber, Hans von} (22.\,4.\,1872 Dresden – 22.\,4.\,1924 München), \emph{Verleger}|pwv} nicht anderer Meinung
               wäre damit, dass er mir die Unmöglichkeit beweist, dass die Zeitschrift das verlangte
               Honorar zahlen kann, auch nicht zahlen könnte, wenn sie mehr als 540 Abonnenten hätte
               und die ganze Auflage von 1000 Ex. abonniert wäre. Der Hyper\orgindex{Hyperion@Hyperion|pw} ist für keinen der Betheiligten irgendwann einmal ein Geschäft. – So
               kann ich also nur traurig für Ihre Freundlichkeit danken.\pend
           
\pstart
           Ich bin Ihr immer ergebner{\\[\baselineskip]}\spacefill\mbox{Frz Blei}\pend
           \leftskip=0em{}\selectlanguage{ngerman}\endnumbering\briefempfaengerindex{Schnitzler, Arthur@\textsc{Schnitzler, Arthur}!zzzBlei, Franz@\emph{von Franz Blei}!1909-12-171@{{[}17. 12. 1909{]}}|)be}\mylabel{L01905h}  \newcommand{\dateiname}{L01905}\newcommand{\titel}{Franz Blei an Arthur Schnitzler, [17. 12. 1909]}\newcommand{\editorInnen}{Martin Anton Müller und Gerd-Hermann Susen}%% latex-leseansicht-abspann.tex
%% Abspann für die Leseansicht.
%% Der Schalter \ifkorrekturansicht ist bereits durch den Vorspann gesetzt.

%% latex-abspann.tex
%% Gemeinsamer Abspann für Korrekturansicht und Leseansicht.
%% Setzt den Schalter \ifkorrekturansicht voraus (gesetzt in den
%% einbindenden Dateien latex-korrekturansicht-abspann.tex bzw.
%% latex-leseansicht-abspann.tex).
%% ---------------------------------------------------------------

\normalsize

% Das esempio-Environment wird nur in der Leseansicht benötigt
\ifkorrekturansicht\else
\newenvironment{esempio}[3]%
{
    \vspace{1.5ex}
    \rlap{\underline{#1}}
    \par
    \setlength{\parindent}{0cm}
    \nopagebreak
    \leftskip=#2cm
    \rightskip=#3cm
}
{
    \par
}
\fi

\doendnotes{C}
\bigskip
\vfill

\clearpage

\footnotesize

\ifkorrekturansicht
  \lohead{\textsc{register}}
\fi

% theindex-Environment neu definieren ohne reledmac
\makeatletter
\renewenvironment{theindex}{%
  \ifkorrekturansicht
    \section*{\indexname}%
  \else
    \subsubsection*{Index der erwähnten Entitäten}%
  \fi
  \setlength{\parindent}{0pt}%
  \setlength{\parskip}{0pt plus 0.3pt}%
  \let\item\@idxitem
}{%
  \ifkorrekturansicht\clearpage\fi
}
\makeatother

\IfFileExists{\jobname-pw.ind}{\input{\jobname-pw.ind}}{}

% Quellenangabe nur in der Leseansicht
\ifkorrekturansicht\else
% Fallback-Definitionen, falls die .tex-Datei \titel etc. nicht gesetzt hat
\providecommand{\titel}{}
\providecommand{\editorInnen}{}
\providecommand{\dateiname}{\jobname}

\vspace{3cm}

\vfill

\footnotesize
\textsc{Quelle}: \titel. Herausgegeben von {\editorInnen}. In: \emph{Arthur Schnitzler: Briefwechsel mit Autorinnen und Autoren}.
 Digitale Edition, https://schnitzler-briefe.acdh.oeaw.ac.at/{\dateiname}.html (Stand \today)
\fi

\end{document}


