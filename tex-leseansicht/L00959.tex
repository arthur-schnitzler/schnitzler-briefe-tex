\input{../tex-inputs/latex-pdf-vorspann}
\begin{center}
            \textcolor{red}{ENTWURF. ENTZIFFERUNG NOCH NICHT KORREKTURGELESEN}
                      \end{center}
            
               \section[Arthur Schnitzler an Richard Beer-Hofmann mit Beilage Alfred Gold an Schnitzler, 17. 8. 1899]{ Arthur Schnitzler an Richard Beer-Hofmann mit Beilage Alfred Gold an
               Schnitzler, 17. 8. 1899}\nopagebreak\mylabel{v}\rehead{ }\begin{ledgroupsized}[t]{13cm}\normalsize\beginnumbering\briefempfaengerindex{Beer-Hofmann, Richard@\textsc{Beer-Hofmann, Richard}!zzzSchnitzler, Arthur@\emph{von Arthur Schnitzler}!1899-08-171@{17. 8. 1899}|(be} \toendnotes[C]{\smallbreak\pagebreak[2]} \Standort{YCGL, MSS 31.}
\physDesc{Briefkarte, Umschlag, Fragment
\newline{}Handschrift: schwarze Tinte, deutsche Kurrent\newline{}Beilage: Alfred Gold\pwindex{Gold, Alfred 28.06.1874 – 24.10.1958@\textsc{Gold, Alfred} (28.06.1874 – 24.10.1958), \emph{Schriftsteller, Journalist, Kunsthändler}|pw}: Brief, 1 Blatt, 1 Seite, schwarze Tinte, Kurrentschrift. Diese wird
                                 in Beer-Hofmanns Nachlass unter den Briefen Schnitzlers aufbewahrt.
                                 Die Zuordnung als Beilage basiert darauf, dass das Brieffragment
                                 zeitlich mit der Übermittlung des Gold\pwindex{Gold, Alfred 28.06.1874 – 24.10.1958@\textsc{Gold, Alfred} (28.06.1874 – 24.10.1958), \emph{Schriftsteller, Journalist, Kunsthändler}|pw}-Briefes zusammenfällt \newline{}Versand: 1) Stempel: »\nobreak{}\oindex{Bad Ischl@\textbf{Bad Ischl}|pwk}Isch\textcolor{gray}{l}, 17. 8. \textcolor{gray}{99}, 12–1 N\nobreak{}«.  2) Stempel: »\nobreak{}\oindex{Seeboden@\textbf{Seeboden}|pwk}Seeboden, 17. 8. 99\nobreak{}«. \newline{}Ordnung: mit Bleistift von unbekannter Hand: »Anfang
                                    fehlt?« und datiert »17. 8. 1899« }\toendnotes[C]{\smallbreak}\pstart{}{\pb}\textsc{Kärnthen}\oindex{Kaernten@\textbf{Kärnten}|pw}.\pend{}\pstart{}Herrn \textsc{Dr. Richard Beer-Hofmann}\pend{}\pstart{}\textsc{Seeboden am Millstätter}see\oindex{Seeboden@\textbf{Seeboden}|pw}\pend{}\pstart{}\textsc{Villa Platzer}\oindex{Villa Platzer@\textbf{Villa Platzer}|pw}\pend{}{\bigskip}\pstart
           \noindent{}{\pb}hatte es ſchon auf dem Bahnhof für Sie mit – vergaſs
               natürlich es Ihnen zu geben.\pend
           \pstart Herzlichen Gruß! Ihr \spacefill\mbox{Arthur}\pend{}\pstart
           17/8\pend
           {\bigskip}\pstart
           \noindent{}{\pb}{[}hs. Gold:{]} \textcolor{gray}{\textbf{»Die Zeit\orgindex{Zeit. Wiener Wochenschrift@Die Zeit. Wiener Wochenschrift|pw}«}}\hfill \textcolor{gray}{\textbf{\textbf{Wien\oindex{Wien@\textbf{Wien}|pw}}, den}}{ }14. 8. \textcolor{gray}{\textbf{189}}9\pend
           \pstart
           \textcolor{gray}{\textbf{Wiener Wochenſchrift}}\hfill IX/3, Günthergasse 1\oindex{Guenthergasse@\textbf{Günthergasse}|pw}.\pend
           \pstart
           \textcolor{gray}{\textbf{\textbf{Herausgeber}:}}{\\}\textcolor{gray}{\textbf{Profeſſor Dr. I. Singer\pwindex{Singer, Isidor 16.01.1857 – 08.12.1927@\textsc{Singer, Isidor} (16.01.1857 – 08.12.1927), \emph{Journalist, Herausgeber, Soziologe}|pw},
                        Hermann Bahr\pwindex{Bahr, Hermann 19.07.1863 – 15.01.1934@\textsc{Bahr, Hermann} (19.07.1863 – 15.01.1934), \emph{Schriftsteller, Kritiker}|pw}, Dr. Heinrich Kanner\pwindex{Kanner, Heinrich 09.11.1864 – 15.02.1930@\textsc{Kanner, Heinrich} (09.11.1864 – 15.02.1930), \emph{Publizist}|pw}.}}\pend
           \pstart
           \textcolor{gray}{\textbf{Telephon Nr. 6415.}}\pend
           \pstart{}Verehrter D\textsuperscript{r} Schnitzler,\pend\pstart
           Es iſt ſo gut wie ſicher, daſs ich mit der Novelle\pwindex{Beer-Hofmann, Richard 11.07.1866 – 26.09.1945@\textsc{Beer-Hofmann, Richard} (11.07.1866 – 26.09.1945), \emph{Schriftsteller}!Tod Georgs. Fragment4.11.1899 – 25.11.1899@\strich\emph{Der Tod Georgs. Fragment} {[}4.11.1899 – 25.11.1899{]}|pwv}{ }ſchon im October beginnen kann (in der
               Nr. vom 7.) Bitte mir aber, wenn irgend möglich, das Mſcr.\pwindex{Beer-Hofmann, Richard 11.07.1866 – 26.09.1945@\textsc{Beer-Hofmann, Richard} (11.07.1866 – 26.09.1945), \emph{Schriftsteller}!Tod Georgs. Fragment4.11.1899 – 25.11.1899@\strich\emph{Der Tod Georgs. Fragment} {[}4.11.1899 – 25.11.1899{]}|pwv}{ }\uline{noch im Auguſt} – u. zw. mit den Abtheilungen des
               Verf.– zu ſchicken. Besten Dank für frdl. Vermittlung.\pend
           \pstart
           In Eile Ihr herzlich ergebener{\\[\baselineskip]}\spacefill\mbox{AlfGold\pwindex{Gold, Alfred 28.06.1874 – 24.10.1958@\textsc{Gold, Alfred} (28.06.1874 – 24.10.1958), \emph{Schriftsteller, Journalist, Kunsthändler}|pw}}\pend
           \leftskip=0em{}\pstart
           \noindent{}Grüße an B.-H. u. Waſſermann\pwindex{Wassermann, Jakob 10.03.1873 – 01.01.1934@\textsc{Wassermann, Jakob} (10.03.1873 – 01.01.1934), \emph{Schriftsteller}|pw}.\pend
           \pstart
           Herrn D\textsuperscript{r} Alfred Schnitzler\pend
           \pstart
           \textsc{Ischl}\oindex{Bad Ischl@\textbf{Bad Ischl}|pw}\pend
           \pstart
           \textcolor{gray}{\textbf{\label{T_L00959_1v}\edtext{Alle für »Die Zeit\orgindex{Zeit. Wiener Wochenschrift@Die Zeit. Wiener Wochenschrift|pw}« beſtimmten Zuſchriften und Sendungen ſind an die
                  Redaction der »Zeit\orgindex{Zeit. Wiener Wochenschrift@Die Zeit. Wiener Wochenschrift|pw}« und \textbf{nicht} an die Perſon eines der Herausgeber zu richten.}{\lemma{\textnormal{\emph{Alle … richten.}}}\Cendnote{\textnormal{am unteren Rand der Seite}}}\label{T_L00959_1h}}}\pend
           \endnumbering\briefempfaengerindex{Beer-Hofmann, Richard@\textsc{Beer-Hofmann, Richard}!zzzSchnitzler, Arthur@\emph{von Arthur Schnitzler}!1899-08-171@{17. 8. 1899}|)be}\mylabel{h}\end{ledgroupsized}  \newcommand{\dateiname}{L00959}\newcommand{\titel}{Arthur Schnitzler an Richard Beer-Hofmann mit Beilage Alfred Gold an Schnitzler, 17. 8. 1899}\newcommand{\editorInnen}{Martin Anton Müller und Gerd-Hermann Susen}\input{../tex-inputs/latex-pdf-abspann}
      