%% latex-leseansicht-vorspann.tex
%% Vorspann für die Leseansicht.
%% Lädt die gemeinsame Datei latex-vorspann.tex mit nicht gesetztem Schalter.

\newif\ifkorrekturansicht
\korrekturansichtfalse

\input{../tex-inputs/latex-vorspann}


\section[Arthur Schnitzler an Richard Beer-Hofmann mit Beilage Alfred Gold an Schnitzler, 17. 8. 1899]{L00959 Arthur Schnitzler an Richard Beer-Hofmann mit Beilage Alfred Gold an
               Schnitzler, 17. 8. 1899}
\nopagebreak\mylabel{L00959v}
\rehead{ }\normalsize\beginnumbering\briefempfaengerindex{Beer-Hofmann, Richard@\textsc{Beer-Hofmann, Richard}!zzzSchnitzler, Arthur@\emph{von Arthur Schnitzler}!1899-08-171@{17. 8. 1899}|(be}
\toendnotes[C]{\smallbreak\pagebreak[2]}
\correspDesc{Versand  durch Arthur Schnitzler am 17. 8. 1899 in Bad Ischl
\newline{}Erhalt  durch Richard Beer-Hofmann am 19. 8. 1899 in Seeboden}\toendnotes[C]{\smallbreak}
\Standort{YCGL, MSS 31.}
\physDesc{Briefkarte, , Kuvert, 583 Zeichen, Fragment
\newline{}Handschrift: schwarze Tinte, deutsche Kurrent
\newline{}Beilage: Alfred Gold\pwindex{Gold, Alfred 28.\,6.\,1874 Wien – 24.\,10.\,1958 New York City@\textsc{Gold, Alfred} (28.\,6.\,1874 Wien – 24.\,10.\,1958 New York City), \emph{Schriftsteller, Journalist, Kunsthändler}|pw}: Brief, 1 Blatt,
                                 1 Seite, schwarze Tinte, Kurrentschrift. Diese wird in
                                 Beer-Hofmanns Nachlass unter den Briefen Schnitzlers aufbewahrt.
                                 Die Zuordnung als Beilage basiert darauf, dass das Brieffragment
                                 zeitlich mit der Übermittlung des Gold\pwindex{Gold, Alfred 28.\,6.\,1874 Wien – 24.\,10.\,1958 New York City@\textsc{Gold, Alfred} (28.\,6.\,1874 Wien – 24.\,10.\,1958 New York City), \emph{Schriftsteller, Journalist, Kunsthändler}|pw}-Briefes zusammenfällt 
\newline{}Versand: 1) Stempel: »\nobreak{}\oindex{Bad Ischl@\textbf{Bad Ischl}|pwk}Isch\textcolor{gray}{l}, 17. 8. \textcolor{gray}{99}, 12–1 N\nobreak{}«.   2) Stempel: »\nobreak{}\oindex{Seeboden am Millstättersee@\textbf{Seeboden am Millstättersee}|pwk}Seeboden, 17. 8. 99\nobreak{}«. 
\newline{}Ordnung: mit Bleistift von unbekannter Hand: »Anfang
                                    fehlt?« und datiert »17. 8. 1899« }\toendnotes[C]{\smallbreak}\pstart{}{\pb}\textsc{Kärnthen}\oindex{Kärnten@\textbf{Kärnten}, \emph{Land}|pw}.\pend{}\pstart{}Herrn \textsc{Dr. Richard Beer-Hofmann}\pend{}\pstart{}\textsc{Seeboden am Millstätter}see\oindex{Seeboden am Millstättersee@\textbf{Seeboden am Millstättersee}|pw}\pend{}\pstart{}\textsc{Villa Platzer}\oindex{Villa Platzer@\textbf{Villa Platzer}, \emph{Gebäude}|pw}\pend{}{\bigskip}\vspace{1em}
\pstart
           \noindent{}{\pb}hatte es{ }ſchon auf dem \label{K_L00959-1v}\edtext{Bahnhof\oindex{Bahnhof Bad Ischl@\textbf{Bahnhof Bad Ischl}, \emph{Bahnhofsgebäude}|pwuv}\oindex{Bahnhof Bad Aussee@\textbf{Bahnhof Bad Aussee}, \emph{Bahnhofsgebäude}|pwuv}}{\lemma{\textnormal{\emph{Bahnhof}}}\Cendnote{\textnormal{Das verweist auf ein Treffen, das nur
                     rekonstruiert werden kann, für das es aber keine unmittelbaren Zeugnisse gibt. Am XXXX Auszeichnungsfehler: Dokument L00958 nicht gefunden schrieb 
                     Hofmannsthal\pwindex{Hofmannsthal, Hugo von 1.\,2.\,1874 Wien – 15.\,7.\,1929 Rodaun@\textsc{Hofmannsthal, Hugo von} (1.\,2.\,1874 Wien – 15.\,7.\,1929 Rodaun), \emph{Schriftsteller}|pwk}, dass Beer-Hofmann\pwindex{Beer-Hofmann, Richard 11.\,7.\,1866 Wien – 26.\,9.\,1945 New York City@\textsc{Beer-Hofmann, Richard} (11.\,7.\,1866 Wien – 26.\,9.\,1945 New York City), \emph{Schriftsteller}|pwk} am 17. 8. 1899
                     nach Aussee\oindex{Bad Aussee@\textbf{Bad Aussee}, \emph{Hauptstadt}|pwk} kommen wolle. Schnitzler dürfte ebenfalls für einen Tag hingereist sein, 
                     wofür die Begegnung am Bahnhof spricht. Am XXXX Auszeichnungsfehler: Dokument L00960 nicht gefunden, dem Folgetag, schrieb
                     Beer-Hofmann\pwindex{Beer-Hofmann, Richard 11.\,7.\,1866 Wien – 26.\,9.\,1945 New York City@\textsc{Beer-Hofmann, Richard} (11.\,7.\,1866 Wien – 26.\,9.\,1945 New York City), \emph{Schriftsteller}|pwk} von der Rückreise.}}}\label{K_L00959-1} für Sie mit – vergaſs
               natürlich es Ihnen zu geben.\pend
           \pstart Herzlichen Gruß! Ihr \spacefill\mbox{Arthur}\pend{}
\pstart
           17/8\pend
           \selectlanguage{ngerman}\vspace{1em}{\vspace{1\baselineskip}}
\pstart
           {\pb}{[}hs. Gold:{]} \textcolor{gray}{\textbf{»Die Zeit\orgindex{Zeit. Wiener Wochenschrift@Die Zeit. Wiener Wochenschrift|pw}«}}\hfill \textcolor{gray}{\textbf{\textbf{Wien\oindex{Wien@\textbf{Wien}, \emph{Verwaltungsgebiet}|pw}}, den}}{ }14. 8. \textcolor{gray}{\textbf{189}}9\pend
           
\pstart
           \textcolor{gray}{\textbf{Wiener Wochenſchrift}}\hfill IX/3, Günthergasse 1\oindex{Wien@\textbf{Wien}!IX., Alsergrund@\textbf{IX., Alsergrund}!Günthergasse@\textbf{Günthergasse}, \emph{Straße}|pw}.\pend
           
\pstart
           \textcolor{gray}{\textbf{\textbf{Herausgeber}:}}{\\}\textcolor{gray}{\textbf{Profeſſor Dr. I. Singer\pwindex{Singer, Isidor 16.\,1.\,1857 Budapest – 8.\,12.\,1927 Wien@\textsc{Singer, Isidor} (16.\,1.\,1857 Budapest – 8.\,12.\,1927 Wien), \emph{Journalist, Herausgeber, Soziologe}|pw}, Hermann Bahr\pwindex{Bahr, Hermann 19.\,7.\,1863 Linz – 15.\,1.\,1934 München@\textsc{Bahr, Hermann} (19.\,7.\,1863 Linz – 15.\,1.\,1934 München), \emph{Schriftsteller, Kritiker}|pw},
                        Dr. Heinrich Kanner\pwindex{Kanner, Heinrich 9.\,11.\,1864 Galați – 15.\,2.\,1930 Wien@\textsc{Kanner, Heinrich} (9.\,11.\,1864 Galați – 15.\,2.\,1930 Wien), \emph{Herausgeber, Publizist}|pw}.}}\pend
           
\pstart
           \textcolor{gray}{\textbf{Telephon Nr. 6415.}}\pend
           
\pstart{}Verehrter D\textsuperscript{r} Schnitzler,\pend\vspace{0.5em}
\pstart
           Es iſt{ }ſo gut wie{ }ſicher, daſs ich mit der Novelle\pwindex{Beer-Hofmann, Richard 11.\,7.\,1866 Wien – 26.\,9.\,1945 New York City@\textsc{Beer-Hofmann, Richard} (11.\,7.\,1866 Wien – 26.\,9.\,1945 New York City), \emph{Schriftsteller}!Tod Georgs. Fragment@\strich\emph{Der Tod Georgs. Fragment}|pwv}{ }ſchon im October beginnen kann (in der
               Nr. vom 7.) Bitte mir aber, wenn irgend möglich, das Mſcr.\pwindex{Beer-Hofmann, Richard 11.\,7.\,1866 Wien – 26.\,9.\,1945 New York City@\textsc{Beer-Hofmann, Richard} (11.\,7.\,1866 Wien – 26.\,9.\,1945 New York City), \emph{Schriftsteller}!Tod Georgs. Fragment@\strich\emph{Der Tod Georgs. Fragment}|pwv}{ }\uline{noch im Auguſt} – u. zw. mit den Abtheilungen des
               Verf.– zu{ }ſchicken. Besten Dank für frdl. Vermittlung.\pend
           
\pstart
           In Eile Ihr herzlich ergebener{\\[\baselineskip]}\spacefill\mbox{AlfGold\pwindex{Gold, Alfred 28.\,6.\,1874 Wien – 24.\,10.\,1958 New York City@\textsc{Gold, Alfred} (28.\,6.\,1874 Wien – 24.\,10.\,1958 New York City), \emph{Schriftsteller, Journalist, Kunsthändler}|pw}}\pend
           \leftskip=0em{}
\pstart
           \noindent{}Grüße an B.-H. u. Waſſermann\pwindex{Wassermann, Jakob 10.\,3.\,1873 Fürth – 1.\,1.\,1934 Altaussee@\textsc{Wassermann, Jakob} (10.\,3.\,1873 Fürth – 1.\,1.\,1934 Altaussee), \emph{Schriftsteller}|pw}.\pend
           
\pstart
           Herrn D\textsuperscript{r} Alfred Schnitzler\pend
           
\pstart
           \textsc{Ischl}\oindex{Bad Ischl@\textbf{Bad Ischl}|pw}\pend
           
\pstart
           \textcolor{gray}{\textbf{\label{T_L00959-1v}\edtext{Alle für »Die Zeit\orgindex{Zeit. Wiener Wochenschrift@Die Zeit. Wiener Wochenschrift|pw}« beſtimmten Zuſchriften und Sendungen{ }ſind an die
                  Redaction der »Zeit\orgindex{Zeit. Wiener Wochenschrift@Die Zeit. Wiener Wochenschrift|pw}« und \textbf{nicht} an die Perſon eines der Herausgeber zu richten.}{\lemma{\textnormal{\emph{Alle … richten.}}}\Cendnote{\textnormal{am unteren Rand der Seite}}}\label{T_L00959-1}}}\pend
           \selectlanguage{ngerman}\endnumbering\briefempfaengerindex{Beer-Hofmann, Richard@\textsc{Beer-Hofmann, Richard}!zzzSchnitzler, Arthur@\emph{von Arthur Schnitzler}!1899-08-171@{17. 8. 1899}|)be}\mylabel{L00959h}  \newcommand{\dateiname}{L00959}\newcommand{\titel}{Arthur Schnitzler an Richard Beer-Hofmann mit Beilage Alfred Gold an Schnitzler, 17. 8. 1899}\newcommand{\editorInnen}{Martin Anton Müller und Gerd-Hermann Susen}%% latex-leseansicht-abspann.tex
%% Abspann für die Leseansicht.
%% Der Schalter \ifkorrekturansicht ist bereits durch den Vorspann gesetzt.

%% latex-abspann.tex
%% Gemeinsamer Abspann für Korrekturansicht und Leseansicht.
%% Setzt den Schalter \ifkorrekturansicht voraus (gesetzt in den
%% einbindenden Dateien latex-korrekturansicht-abspann.tex bzw.
%% latex-leseansicht-abspann.tex).
%% ---------------------------------------------------------------

\normalsize

% Das esempio-Environment wird nur in der Leseansicht benötigt
\ifkorrekturansicht\else
\newenvironment{esempio}[3]%
{
    \vspace{1.5ex}
    \rlap{\underline{#1}}
    \par
    \setlength{\parindent}{0cm}
    \nopagebreak
    \leftskip=#2cm
    \rightskip=#3cm
}
{
    \par
}
\fi

\doendnotes{C}
\bigskip
\vfill

\clearpage

\footnotesize

\ifkorrekturansicht
  \lohead{\textsc{register}}
\fi

% theindex-Environment neu definieren ohne reledmac
\makeatletter
\renewenvironment{theindex}{%
  \ifkorrekturansicht
    \section*{\indexname}%
  \else
    \subsubsection*{Index der erwähnten Entitäten}%
  \fi
  \setlength{\parindent}{0pt}%
  \setlength{\parskip}{0pt plus 0.3pt}%
  \let\item\@idxitem
}{%
  \ifkorrekturansicht\clearpage\fi
}
\makeatother

\IfFileExists{\jobname-pw.ind}{\input{\jobname-pw.ind}}{}

% Quellenangabe nur in der Leseansicht
\ifkorrekturansicht\else
% Fallback-Definitionen, falls die .tex-Datei \titel etc. nicht gesetzt hat
\providecommand{\titel}{}
\providecommand{\editorInnen}{}
\providecommand{\dateiname}{\jobname}

\vspace{3cm}

\vfill

\footnotesize
\textsc{Quelle}: \titel. Herausgegeben von {\editorInnen}. In: \emph{Arthur Schnitzler: Briefwechsel mit Autorinnen und Autoren}.
 Digitale Edition, https://schnitzler-briefe.acdh.oeaw.ac.at/{\dateiname}.html (Stand \today)
\fi

\end{document}


