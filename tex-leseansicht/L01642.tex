%% latex-leseansicht-vorspann.tex
%% Vorspann für die Leseansicht.
%% Lädt die gemeinsame Datei latex-vorspann.tex mit nicht gesetztem Schalter.

\newif\ifkorrekturansicht
\korrekturansichtfalse

\input{../tex-inputs/latex-vorspann}


         
         \renewcommand{\erwaehntePersonen}{Personen: Richard Beer-Hofmann, Paula Beer-Hofmann, Olga Schnitzler}
         \renewcommand{\erwaehnteInstitutionen}{Institutionen: Rosé-Quartett}
         \renewcommand{\erwaehnteOrte}{Orte: Edmund-Weiß-Gasse 7, Wien}
         \renewcommand{\erwaehnteWerke}{}
               \section[Arthur Schnitzler an Richard Beer-Hofmann, 17. 12. 1906]{ Arthur Schnitzler an Richard Beer-Hofmann, 17. 12. 1906}\nopagebreak\mylabel{v}\rehead{ }\begin{ledgroupsized}[t]{13cm}\normalsize\beginnumbering\briefempfaengerindex{Beer-Hofmann, Richard@\textsc{Beer-Hofmann, Richard}!zzzSchnitzler, Arthur@\emph{von Arthur Schnitzler}!1906-12-171@{17. 12. 1906}|(be} \toendnotes[C]{\smallbreak\pagebreak[2]} \Standort{YCGL, MSS 31.}
\physDesc{Briefkarte, , Umschlag, 310 Zeichen
\newline{}Handschrift: Bleistift, deutsche Kurrent
\newline{}Versand: ohne postalischen Übermittlungsvermerk }\toendnotes[C]{\smallbreak}\pstart{}{\pb}\textcolor{gray}{\textbf{Dr. Arthur Schnitzler}}\pend{}\pstart{}\textcolor{gray}{\textbf{Wien, XVIII. Spoettelgasse 7\oindex{Edmund-Weiss-Gasse 7@\textbf{Edmund-Weiß-Gasse 7}|pw}.}}\pend{}{\bigskip}\pstart{}{\pb}\textsc{Herrn Dr Richard Beer-Hofmann}\pend{}\pstart{}Wien\oindex{Wien@\textbf{Wien}|pw}\pend{}{\bigskip}\pstart
           \noindent{}{\pb}\textcolor{gray}{\textbf{Dr. Arthur Schnitzler}}\hfill 17/12 906.\pend
           \pstart
           \textcolor{gray}{\textbf{Wien, XVIII. Spoettelgasse 7\oindex{Edmund-Weiss-Gasse 7@\textbf{Edmund-Weiß-Gasse 7}|pw}.}}\pend
           \pstart
           lieber Richard,{ }\label{K_L01642-1v}\edtext{morgen Dinſtag{ }\textsc{Rosé Quartett\orgindex{Rose-Quartett@Rosé-Quartett|pw}}}{\lemma{\textnormal{\emph{morgen … Quartett}}}\Cendnote{\textnormal{vgl. A. S.: \emph{Tagebuch}, 18. 12. 1906}}}\label{K_L01642-1h}; wäre beifolgende Karte frei (neben mir) und ich würde mich freuen, we{\geminationn} Sie mitkämen. \uline{Höchſt
                  wahrſcheinlich}, da meine Frau\pwindex{Schnitzler, Olga 17.01.1882 – 13.01.1970@\textsc{Schnitzler, Olga} (17.01.1882 – 13.01.1970), \emph{Schauspielerin, Sängerin}|pw} etwas
               verſchnupft folgt morgen Vormttg eine zweite Karte zu Ihner, \textsc{resp}{ }Paula\pwindex{Beer-Hofmann, Paula 25.02.1879 – 30.10.1939@\textsc{Beer-Hofmann, Paula} (25.02.1879 – 30.10.1939)|pw}’s Verfügung.\pend
           \pstart Herzlichſt Ihr \spacefill\mbox{A.}\pend{}
         
         \endnumbering\mylabel{h}\end{ledgroupsized}  \newcommand{\dateiname}{L01642}\newcommand{\titel}{Arthur Schnitzler an Richard Beer-Hofmann, 17. 12. 1906}\newcommand{\editorInnen}{Martin Anton Müller und Gerd-Hermann Susen}%% latex-leseansicht-abspann.tex
%% Abspann für die Leseansicht.
%% Der Schalter \ifkorrekturansicht ist bereits durch den Vorspann gesetzt.

%% latex-abspann.tex
%% Gemeinsamer Abspann für Korrekturansicht und Leseansicht.
%% Setzt den Schalter \ifkorrekturansicht voraus (gesetzt in den
%% einbindenden Dateien latex-korrekturansicht-abspann.tex bzw.
%% latex-leseansicht-abspann.tex).
%% ---------------------------------------------------------------

\normalsize

% Das esempio-Environment wird nur in der Leseansicht benötigt
\ifkorrekturansicht\else
\newenvironment{esempio}[3]%
{
    \vspace{1.5ex}
    \rlap{\underline{#1}}
    \par
    \setlength{\parindent}{0cm}
    \nopagebreak
    \leftskip=#2cm
    \rightskip=#3cm
}
{
    \par
}
\fi

\doendnotes{C}
\bigskip
\vfill

\clearpage

\footnotesize

\ifkorrekturansicht
  \lohead{\textsc{register}}
\fi

% theindex-Environment neu definieren ohne reledmac
\makeatletter
\renewenvironment{theindex}{%
  \ifkorrekturansicht
    \section*{\indexname}%
  \else
    \subsubsection*{Index der erwähnten Entitäten}%
  \fi
  \setlength{\parindent}{0pt}%
  \setlength{\parskip}{0pt plus 0.3pt}%
  \let\item\@idxitem
}{%
  \ifkorrekturansicht\clearpage\fi
}
\makeatother

\IfFileExists{\jobname-pw.ind}{\input{\jobname-pw.ind}}{}

% Quellenangabe nur in der Leseansicht
\ifkorrekturansicht\else
% Fallback-Definitionen, falls die .tex-Datei \titel etc. nicht gesetzt hat
\providecommand{\titel}{}
\providecommand{\editorInnen}{}
\providecommand{\dateiname}{\jobname}

\vspace{3cm}

\vfill

\footnotesize
\textsc{Quelle}: \titel. Herausgegeben von {\editorInnen}. In: \emph{Arthur Schnitzler: Briefwechsel mit Autorinnen und Autoren}.
 Digitale Edition, https://schnitzler-briefe.acdh.oeaw.ac.at/{\dateiname}.html (Stand \today)
\fi

\end{document}


      