%% latex-korrekturansicht-vorspann.tex
%% Vorspann für die Korrekturansicht.
%% Lädt die gemeinsame Datei latex-vorspann.tex mit gesetztem Schalter.

\newif\ifkorrekturansicht
\korrekturansichttrue

\input{../tex-inputs/latex-vorspann}


\section[Arthur Schnitzler an Richard Beer-Hofmann, 17. 12. 1906]{L01642 Arthur Schnitzler an Richard Beer-Hofmann, 17. 12. 1906}
\nopagebreak\mylabel{L01642v}
\rehead{ }\normalsize\beginnumbering\briefempfaengerindex{Beer-Hofmann, Richard@\textsc{Beer-Hofmann, Richard}!zzzSchnitzler, Arthur@\emph{von Arthur Schnitzler}!1906-12-171@{17. 12. 1906}|(be}
\toendnotes[C]{\smallbreak\pagebreak[2]}\Standort{YCGL, MSS 31.}
\physDesc{Briefkarte, , Umschlag, 311 Zeichen
\newline{}Handschrift: Bleistift, deutsche Kurrent
\newline{}Versand: ohne postalischen Übermittlungsvermerk }\toendnotes[C]{\smallbreak}\pstart{}{\pb}\textcolor{gray}{\textbf{Dr. Arthur Schnitzler}}\pend{}\pstart{}\textcolor{gray}{\textbf{Wien, XVIII. Spoettelgasse 7\oindex{Edmund-Weiss-Gasse 7@\textbf{Edmund-Weiß-Gasse 7}, \emph{Wohngebäude (K.WHS)}|pw}.}}\pend{}{\bigskip}\pstart{}{\pb}\textsc{Herrn Dr Richard Beer-Hofmann}\pend{}\pstart{}Wien\oindex{Wien@\textbf{Wien}, \emph{A.ADM2}|pw}\pend{}{\bigskip}\vspace{1em}
\pstart
           {\pb}\textcolor{gray}{\textbf{Dr. Arthur Schnitzler}}\hfill 17/12 906.\pend
           
\pstart
           \textcolor{gray}{\textbf{Wien, XVIII. Spoettelgasse 7\oindex{Edmund-Weiss-Gasse 7@\textbf{Edmund-Weiß-Gasse 7}, \emph{Wohngebäude (K.WHS)}|pw}.}}\pend
           \vspace{0.5em}
\pstart
           lieber Richard,{ }\label{K_L01642-1v}\edtext{morgen Dinſtag{ }\textsc{Rosé Quartett\orgindex{Rose-Quartett@Rosé-Quartett|pw}}}{\lemma{\textnormal{\emph{morgen … Quartett}}}\Cendnote{\textnormal{Vgl. A. S.: \emph{Tagebuch}, 18. 12. 1906.
               }}}\label{K_L01642-1}; wäre beifolgende Karte frei (neben mir) und ich würde mich freuen, we{\geminationn} Sie mitkämen. \uline{Höchſt
                  wahrſcheinlich}, da meine Frau\pwindex{Schnitzler, Olga 17.01.1882 – 13.01.1970@\textsc{Schnitzler, Olga} (17.01.1882 – 13.01.1970), \emph{Schauspieler/Schauspielerin, Sänger/Sängerin}|pw} etwas
               verſchnupft folgt morgen Vormttg eine zweite Karte zu Ihner, \textsc{resp}{ }Paula\pwindex{Beer-Hofmann, Paula 25.02.1879 – 30.10.1939@\textsc{Beer-Hofmann, Paula} (25.02.1879 – 30.10.1939)|pw}’s Verfügung.\pend
           \pstart Herzlichſt Ihr \spacefill\mbox{A.}\pend{}\selectlanguage{ngerman}\endnumbering\briefempfaengerindex{Beer-Hofmann, Richard@\textsc{Beer-Hofmann, Richard}!zzzSchnitzler, Arthur@\emph{von Arthur Schnitzler}!1906-12-171@{17. 12. 1906}|)be}\mylabel{L01642h}  \normalsize

\doendnotes{C}
\bigskip
\vfill

\clearpage

\footnotesize

\lohead{\textsc{register}}

% Definiere theindex-Environment komplett neu ohne reledmac
\makeatletter
\renewenvironment{theindex}{%
  \section*{\indexname}%
  \setlength{\parindent}{0pt}%
  \setlength{\parskip}{0pt plus 0.3pt}%
  \let\item\@idxitem
}{%
  \clearpage
}
\makeatother

\IfFileExists{\jobname-pw.ind}{\input{\jobname-pw.ind}}{}

\end{document}

      