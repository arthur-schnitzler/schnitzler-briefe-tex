%% latex-korrekturansicht-vorspann.tex
%% Vorspann für die Korrekturansicht.
%% Lädt die gemeinsame Datei latex-vorspann.tex mit gesetztem Schalter.

\newif\ifkorrekturansicht
\korrekturansichttrue

\input{../tex-inputs/latex-vorspann}


\section[Arthur Schnitzler an Georg Brandes, 13. 7. 1906]{L01610 Arthur Schnitzler an Georg Brandes, 13. 7. 1906}
\nopagebreak\mylabel{L01610v}
\rehead{ }\normalsize\beginnumbering\briefempfaengerindex{Brandes, Georg@\textsc{Brandes, Georg}!zzzSchnitzler, Arthur@\emph{von Arthur Schnitzler}!1906-07-131@{13. 7. 1906}|(be}
\toendnotes[C]{\smallbreak\pagebreak[2]}\Standort{Kopenhagen, Det Kongelige Bibliotek, Georg Brandes Arkiv, box 125.}
\physDesc{Briefkarte, 973 Zeichen
\newline{}Handschrift: schwarze Tinte, lateinische Kurrent
\newline{}Ordnung: mit schwarzer Tinte von unbekannter Hand in der linken oberen
                                 Ecke: »\textsc{S}« vermerkt, womöglich zur archivalischen Einordnung }
\buchAbdrucke{\weitereDrucke{Georg Brandes, Arthur Schnitzler: \emph{Ein Briefwechsel}. Bern: \emph{Francke} 1956, S. 94.} }
\pstart
           {\pb}\textcolor{gray}{\textbf{Dr. Arthur Schnitzler}}\hfill 13. Juli 906\pend
           
\pstart
           \textcolor{gray}{\textbf{Wien, XVIII. Spoettelgasse 7\oindex{Edmund-Weiss-Gasse 7@\textbf{Edmund-Weiß-Gasse 7}, \emph{Wohngebäude (K.WHS)}|pw}.}}\pend
           
\pstart{}verehrtester Herr Brandes,\pend\vspace{0.5em}
\pstart
           entschuldigen Sie, dass ich neulich gar so miserabel schrieb. Der Sie grüßen liess,
               ist Brahm\pwindex{Brahm, Otto 05.02.1856 – 28.11.1912@\textsc{Brahm, Otto} (05.02.1856 – 28.11.1912), \emph{Theaterleiter/Theaterleiterin, Regisseur/Regisseurin}|pw} (der übrigens möglicherweise in
               diese Gegend ko{\geminationm}t.) Dass Sie schon aus Bett und Spital
               heraus sind, freut mich sehr. Aber glauben Sie um Gotteswillen nicht, dass ich auf
               »Gegenbesuche« od. dergl. Anspruch mache. Freilich möchte ich Sie sehr gerne noch
               einmal sehen, ehe ich Daenemark\oindex{Daenemark@\textbf{Dänemark}, \emph{A.PCLI}|pw} verlasse (was
               kaum vor 3–4 Wochen der Fall sein wird), aber wenn Ihnen Marienlyst\oindex{Marienlyst@\textbf{Marienlyst}, \emph{S.EST}|pw}{ }{\pb}die geringste Unbequemlichkeit macht, so erlauben
               Sie mir vielleicht wieder einmal, Sie in Kopenhagen\oindex{Kopenhagen@\textbf{Kopenhagen}, \emph{P.PPLC}|pw} heimzusuchen. Jedenfalls werd ich mich melden, we{\geminationn} ich auf der Rückreise ein paar Tage Aufenthalt mache.
               Aber wenn Sie hieher ko{\geminationm}en (es ist wirklich wunderschön
               da), haben Sie die Güte, mich vorher wissen zu lassen. Ich möchte doch nicht gern in
                  Schweden\oindex{Schweden@\textbf{Schweden}, \emph{A.PCLI}|pw} drüben, in Skodsborg\oindex{Skodsborg@\textbf{Skodsborg}, \emph{P.PPL}|pw} oder – in Kopenhagen\oindex{Kopenhagen@\textbf{Kopenhagen}, \emph{P.PPLC}|pw} sein, wenn Sie in Marienlyst\oindex{Marienlyst@\textbf{Marienlyst}, \emph{S.EST}|pw}
               erscheinen.\pend
           
\pstart
           Herzlichen Gruß. Ihr sehr ergebener{\\[\baselineskip]}\spacefill\mbox{Arthur Schnitzler}\pend
           \leftskip=0em{}\selectlanguage{ngerman}\endnumbering\briefempfaengerindex{Brandes, Georg@\textsc{Brandes, Georg}!zzzSchnitzler, Arthur@\emph{von Arthur Schnitzler}!1906-07-131@{13. 7. 1906}|)be}\mylabel{L01610h}  \normalsize

\doendnotes{C}
\bigskip
\vfill

\clearpage

\footnotesize

\lohead{\textsc{register}}

% Definiere theindex-Environment komplett neu ohne reledmac
\makeatletter
\renewenvironment{theindex}{%
  \section*{\indexname}%
  \setlength{\parindent}{0pt}%
  \setlength{\parskip}{0pt plus 0.3pt}%
  \let\item\@idxitem
}{%
  \clearpage
}
\makeatother

\IfFileExists{\jobname-pw.ind}{\input{\jobname-pw.ind}}{}

\end{document}

      