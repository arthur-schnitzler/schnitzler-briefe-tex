%% latex-leseansicht-vorspann.tex
%% Vorspann für die Leseansicht.
%% Lädt die gemeinsame Datei latex-vorspann.tex mit nicht gesetztem Schalter.

\newif\ifkorrekturansicht
\korrekturansichtfalse

\input{../tex-inputs/latex-vorspann}


\section[Arthur Schnitzler an Hermann Bahr, 17. 11. 1910]{L01981 Arthur Schnitzler an Hermann Bahr, 17. 11. 1910}
\nopagebreak\mylabel{L01981v}
\rehead{ }\normalsize\beginnumbering\briefempfaengerindex{Bahr, Hermann@\textsc{Bahr, Hermann}!zzzSchnitzler, Arthur@\emph{von Arthur Schnitzler}!1910-11-171@{17. 11. 1910}|(be}
\toendnotes[C]{\smallbreak\pagebreak[2]}
\correspDesc{Versand  durch Arthur Schnitzler am 17. 11. 1910 in Wien
\newline{}Erhalt  durch Hermann Bahr im Zeitraum [17. 11. 1910 – 21. 11. 1910?] \textbf{Ort fehlend} }\toendnotes[C]{\smallbreak}
\Standort{TMW, HS AM 23392 Ba.}
\physDesc{Brief, 2 Blätter, 4 Seiten, 4135 Zeichen
\newline{}Schreibmaschine (\noindent{}Text und Nummerierung des zweiten Blattes:
                                       »3«)
\newline{}Handschrift: schwarze Tinte, deutsche Kurrent (\noindent{}Korrekturen, Schlussformel, Unterschschrift)}\Standort{DLA, A:Schnitzler, 85.1.294/3.}
\physDesc{Brief, Durchschlag, 2 Blätter, 4 Seiten, 4135 Zeichen
\newline{}Schreibmaschine
\newline{}Handschrift: Bleistift, deutsche Kurrent (\noindent{}Korrekturen im letzten Absatz und Schlussformel: »Mit v. fr.
                                    Grüßen / Dein / A«)}
\buchAbdrucke{\weitereDrucke{1) Arthur Schnitzler: \emph{Briefe 1875–1912}. Herausgegeben von Therese Nickl und Heinrich Schnitzler. Frankfurt am Main: \emph{S. Fischer} 1981, S. 633–635.} \weitereDrucke{2) \emph{17. 11. 1910.} In: Arthur Schnitzler: \emph{The Letters of Arthur Schnitzler to Hermann Bahr}. Edited, annotated, and with an introduction, by Donald G. Daviau. Chapel Hill: \emph{The University of North Carolina Press} 1978, S. 106–108 (University of North Carolina studies in the Germanic languages
                        and literatures, 89).} \weitereDrucke{3) Hermann Bahr, Arthur Schnitzler: \emph{Briefwechsel, Aufzeichnungen, Dokumente (1891–1931)}. Herausgegeben von Kurt Ifkovits und Martin Anton Müller. Göttingen: \emph{Wallstein} 2018, S. 443–445.} }\toendnotes[C]{\smallbreak}
\pstart
           {\pb}\textcolor{gray}{\textbf{Dr. Arthur Schnitzler}}\hfill 17. 11. 1910.\pend
           
\pstart
           \textcolor{gray}{\textbf{Wien XVIII. Sternwartestrasse 71\oindex{Wien@\textbf{Wien}!XVIII., Währing@\textbf{XVIII., Währing}!Sternwartestraße 71@\textbf{Sternwartestraße 71}, \emph{Wohngebäude}|pw}}}\pend
           
\pstart{}Lieber Hermann.\pend\vspace{0.5em}
\pstart
           Schönsten Dank für Deinen lieben Brief. Jedenfalls tut es mir leid, dass Du nicht
               über mein Stück\pwindex{Schnitzler, Arthur 15.\,5.\,1862 Wien – 21.\,10.\,1931 ebd.@\textsc{Schnitzler, Arthur} (15.\,5.\,1862 Wien – 21.\,10.\,1931 ebd.), \emph{Schriftsteller, Mediziner}!junge Medardus. Dramatische Historie in einem Vorspiel und fünf Aufzügen@\strich\emph{Der junge Medardus. Dramatische Historie in einem Vorspiel und fünf Aufzügen}|pwv} schreiben
               wirst, denn was immer Du unter den Unannehmlichkeiten verstehst, die daraus für Dich,
               für mich, für alle Beteiligten folgen könnten, für mich wären sie jedenfalls durch
               das Vergnügen reichlich aufgewogen eine ausführliche Darlegung Deiner mir immer
               wertvollen Meinungen zu lesen. Ueberdies erscheint das Stück\pwindex{Schnitzler, Arthur 15.\,5.\,1862 Wien – 21.\,10.\,1931 ebd.@\textsc{Schnitzler, Arthur} (15.\,5.\,1862 Wien – 21.\,10.\,1931 ebd.), \emph{Schriftsteller, Mediziner}!junge Medardus. Dramatische Historie in einem Vorspiel und fünf Aufzügen@\strich\emph{Der junge Medardus. Dramatische Historie in einem Vorspiel und fünf Aufzügen}|pwv} etwa \label{LL120-1v}acht Tage vor der Premiere im Buchhandel\label{LL120-1h}, so dass eine
               Aeusserung über das Werk als solches ohne Rücksicht auf die Darstellung nicht als
               unstatthaft aufgefasst werden könnte.\pend
           
\pstart
           Das Missverständnis, das Du befürchtest, ich hätte in dem Medardus\pwindex{Schnitzler, Arthur 15.\,5.\,1862 Wien – 21.\,10.\,1931 ebd.@\textsc{Schnitzler, Arthur} (15.\,5.\,1862 Wien – 21.\,10.\,1931 ebd.), \emph{Schriftsteller, Mediziner}!junge Medardus. Dramatische Historie in einem Vorspiel und fünf Aufzügen@\strich\emph{Der junge Medardus. Dramatische Historie in einem Vorspiel und fünf Aufzügen}|pwv} einen tragischen Helden zeichnen
               wollen, kann meines Erachtens als solches überhaupt nicht auftreten. Dass Viele sich
               so stellen werden, als glaubten sie, ich selber hielte den Medardus\pwindex{Schnitzler, Arthur 15.\,5.\,1862 Wien – 21.\,10.\,1931 ebd.@\textsc{Schnitzler, Arthur} (15.\,5.\,1862 Wien – 21.\,10.\,1931 ebd.), \emph{Schriftsteller, Mediziner}!junge Medardus. Dramatische Historie in einem Vorspiel und fünf Aufzügen@\strich\emph{Der junge Medardus. Dramatische Historie in einem Vorspiel und fünf Aufzügen}|pwv} für einen tragischen Helden, ist
               hingegen selbstverständlich. In {\pb}dieser Voraussicht war
               ich nahe daran der Buchausgabe ein kurzes Geleitwort mitzugeben ungefähr des
               folgenden Inhalts: \introOben{}»\introOben{}Es ist mir bekannt, dass dieses Stück
               sehr lang und dass der Medardus ein ausnehmend inkonsequentes Subjekt ist.\introOben{}« (\introOben{}Darum \label{T_L01981-1v}\edtext{passieren}{\lemma{\textnormal{\emph{passieren}}}\Cendnote{\textnormal{korrigiert aus:
                     »passierem«}}}\label{T_L01981-1} ihm ja so sonderbare Dinge.\introOben{})\introOben{} Aber am Ende sind in dem Drama selbst so klare Ansichten
               über das Wesen des Medardus\pwindex{Schnitzler, Arthur 15.\,5.\,1862 Wien – 21.\,10.\,1931 ebd.@\textsc{Schnitzler, Arthur} (15.\,5.\,1862 Wien – 21.\,10.\,1931 ebd.), \emph{Schriftsteller, Mediziner}!junge Medardus. Dramatische Historie in einem Vorspiel und fünf Aufzügen@\strich\emph{Der junge Medardus. Dramatische Historie in einem Vorspiel und fünf Aufzügen}|pwv}
               ausgesprochen, hauptsächlich durch Eschenbacher\pwindex{Schnitzler, Arthur 15.\,5.\,1862 Wien – 21.\,10.\,1931 ebd.@\textsc{Schnitzler, Arthur} (15.\,5.\,1862 Wien – 21.\,10.\,1931 ebd.), \emph{Schriftsteller, Mediziner}!junge Medardus. Dramatische Historie in einem Vorspiel und fünf Aufzügen@\strich\emph{Der junge Medardus. Dramatische Historie in einem Vorspiel und fünf Aufzügen}|pwv}, durch Etzelt\pwindex{Schnitzler, Arthur 15.\,5.\,1862 Wien – 21.\,10.\,1931 ebd.@\textsc{Schnitzler, Arthur} (15.\,5.\,1862 Wien – 21.\,10.\,1931 ebd.), \emph{Schriftsteller, Mediziner}!junge Medardus. Dramatische Historie in einem Vorspiel und fünf Aufzügen@\strich\emph{Der junge Medardus. Dramatische Historie in einem Vorspiel und fünf Aufzügen}|pwv} und auch durch die Frau Klähr\pwindex{Schnitzler, Arthur 15.\,5.\,1862 Wien – 21.\,10.\,1931 ebd.@\textsc{Schnitzler, Arthur} (15.\,5.\,1862 Wien – 21.\,10.\,1931 ebd.), \emph{Schriftsteller, Mediziner}!junge Medardus. Dramatische Historie in einem Vorspiel und fünf Aufzügen@\strich\emph{Der junge Medardus. Dramatische Historie in einem Vorspiel und fünf Aufzügen}|pwv}, dass der Unverstand, der sich durch die dramatische Historie
               selbst nicht belehren liesse, auch mit einem solchen Vorwort nichts anzufangen
               wüsste. Auch glaube ich mich mit Dir eines Sinnes, wenn ich behaupte, dass kein
               dramatischer Autor verpflichtet ist in den Mittelpunkt seiner Stücke gerade einen
                  sogenannt\introOben{}en\introOben{} tragischen Helden hineinzustellen. Der Hamlet\pwindex{\textcolor{red}{\textsuperscript{XXXX indx1}}!Hamlet@\strich\emph{Hamlet}|pwv} ist es im dogmatischen
               Sinne so wenig als der \label{K_L01981-1v}\edtext{Oswald\pwindex{Ibsen, Henrik 20.\,3.\,1828 Skien – 23.\,5.\,1906 Oslo@\textsc{Ibsen, Henrik} (20.\,3.\,1828 Skien – 23.\,5.\,1906 Oslo), \emph{Schriftsteller}!Gespenster@\strich\emph{Gespenster}|pwv}}{\lemma{\textnormal{\emph{Oswald}}}\Cendnote{\textnormal{Figur aus \emph{Gespenster}\pwindex{Ibsen, Henrik 20.\,3.\,1828 Skien – 23.\,5.\,1906 Oslo@\textsc{Ibsen, Henrik} (20.\,3.\,1828 Skien – 23.\,5.\,1906 Oslo), \emph{Schriftsteller}!Gespenster@\strich\emph{Gespenster}|pwk} von Ibsen\pwindex{Ibsen, Henrik 20.\,3.\,1828 Skien – 23.\,5.\,1906 Oslo@\textsc{Ibsen, Henrik} (20.\,3.\,1828 Skien – 23.\,5.\,1906 Oslo), \emph{Schriftsteller}|pwk}}}}\label{K_L01981-1}, der \label{K_L01981-2v}\edtext{Prinz von Homburg\pwindex{Kleist, Heinrich von 18.\,10.\,1777 Frankfurt (Oder) – 21.\,11.\,1811 Kleiner Wannsee@\textsc{Kleist, Heinrich von} (18.\,10.\,1777 Frankfurt (Oder) – 21.\,11.\,1811 Kleiner Wannsee), \emph{Schriftsteller}!Prinz Friedrich von Homburg oder die Schlacht bei Fehrbellin@\strich\emph{Prinz Friedrich von Homburg oder die Schlacht bei Fehrbellin}|pwv}}{\lemma{\textnormal{\emph{Prinz von Homburg}}}\Cendnote{\textnormal{die Titelrolle in \emph{Prinz Friedrich von Homburg}\pwindex{Kleist, Heinrich von 18.\,10.\,1777 Frankfurt (Oder) – 21.\,11.\,1811 Kleiner Wannsee@\textsc{Kleist, Heinrich von} (18.\,10.\,1777 Frankfurt (Oder) – 21.\,11.\,1811 Kleiner Wannsee), \emph{Schriftsteller}!Prinz Friedrich von Homburg oder die Schlacht bei Fehrbellin@\strich\emph{Prinz Friedrich von Homburg oder die Schlacht bei Fehrbellin}|pwk} von Kleist\pwindex{Kleist, Heinrich von 18.\,10.\,1777 Frankfurt (Oder) – 21.\,11.\,1811 Kleiner Wannsee@\textsc{Kleist, Heinrich von} (18.\,10.\,1777 Frankfurt (Oder) – 21.\,11.\,1811 Kleiner Wannsee), \emph{Schriftsteller}|pwk}}}}\label{K_L01981-2} so wenig als der \label{K_L01981-3v}\edtext{Tasso\pwindex{Goethe, Johann Wolfgang von 28.\,8.\,1749 Frankfurt am Main – 22.\,3.\,1832 Weimar@\textsc{Goethe, Johann Wolfgang von} (28.\,8.\,1749 Frankfurt am Main – 22.\,3.\,1832 Weimar), \emph{Schriftsteller}!Torquato Tasso. Ein Schauspiel@\strich\emph{Torquato Tasso. Ein Schauspiel}|pwv}}{\lemma{\textnormal{\emph{Tasso}}}\Cendnote{\textnormal{die Titelrolle in \emph{Torquato Tasso}\pwindex{Goethe, Johann Wolfgang von 28.\,8.\,1749 Frankfurt am Main – 22.\,3.\,1832 Weimar@\textsc{Goethe, Johann Wolfgang von} (28.\,8.\,1749 Frankfurt am Main – 22.\,3.\,1832 Weimar), \emph{Schriftsteller}!Torquato Tasso. Ein Schauspiel@\strich\emph{Torquato Tasso. Ein Schauspiel}|pwk} von Goethe\pwindex{Goethe, Johann Wolfgang von 28.\,8.\,1749 Frankfurt am Main – 22.\,3.\,1832 Weimar@\textsc{Goethe, Johann Wolfgang von} (28.\,8.\,1749 Frankfurt am Main – 22.\,3.\,1832 Weimar), \emph{Schriftsteller}|pwk}}}}\label{K_L01981-3}. Dies sind natürlich Beispiele nicht etwa Vergleiche. Kein Zweifel übrigens,
               dass sich der Autor nach dieser Richtung umso mehr erlauben darf je verstorbener er
               ist. – Was Deine weitere Befürchtung anbe{\pb}langt, dass das
               Publikum ein anderes Stück zu sehen bekommen wird als ich geschrieben habe, so ist
               sie zum Teil vielleicht gerechtfertigt, aber nicht durchaus als Befürchtung. Ich habe
               für die Zwecke der Bühne nicht nur sehr viel gestrichen, sondern auch gewisse
               Umstellungen vorgenommen; Kompromisse ohne die auch manche andere\introOben{},\introOben{} und grössere\introOben{},\introOben{} Werke sich auf der Bühne nicht
               hätten halten, ja nicht einmal auf sie hätten gelangen können. Leider muss ich auch
               zugestehen, dass der Medardus\pwindex{Schnitzler, Arthur 15.\,5.\,1862 Wien – 21.\,10.\,1931 ebd.@\textsc{Schnitzler, Arthur} (15.\,5.\,1862 Wien – 21.\,10.\,1931 ebd.), \emph{Schriftsteller, Mediziner}!junge Medardus. Dramatische Historie in einem Vorspiel und fünf Aufzügen@\strich\emph{Der junge Medardus. Dramatische Historie in einem Vorspiel und fünf Aufzügen}|pwv}
               selbst heute in dem Burgtheater\oindex{Wien@\textbf{Wien}!I., Innere Stadt@\textbf{I., Innere Stadt}!Burgtheater@\textbf{Burgtheater}, \emph{Theater}|pw} nicht zu besetzen
               ist (\uline{\label{T_L01981-2v}\edtext{dies ganz unter uns}{\lemma{\textnormal{\emph{dies ganz unter uns}}}\Cendnote{\textnormal{Unterstreichung mit Tinte von der
                     Schreiberin, vgl. Karte vom 19. 11. 1910.}}}\label{T_L01981-2}}). Der Einzige, der ihn heute überhaupt spielen könnte, ist Moissi\pwindex{Moissi, Alexander 2.\,4.\,1879 Triest – 22.\,3.\,1935 Wien@\textsc{Moissi, Alexander} (2.\,4.\,1879 Triest – 22.\,3.\,1935 Wien), \emph{Schauspieler}|pw}. \label{K_L01981-4v}\edtext{Reinhardt\pwindex{Reinhardt, Max 9.\,9.\,1873 Baden bei Wien – 30.\,10.\,1943 New York City@\textsc{Reinhardt, Max} (9.\,9.\,1873 Baden bei Wien – 30.\,10.\,1943 New York City), \emph{Theaterleiter, Regisseur, Schauspieler}|pw}, als ich ihm das Stück vorlas}{\lemma{\textnormal{\emph{Reinhardt, … vorlas}}}\Cendnote{\textnormal{am 26. 8. 1909 in München\oindex{München@\textbf{München}|pwk}}}}\label{K_L01981-4}, war auch ganz entschlossen ihm diese Rolle zuzuteilen, erst später erfuhr
               ich, dass er das Stück nur dann geben wollte, wenn ich ihm noch ein zweites
               überliesse, worauf ich aus prinzipiellen Gründen nicht einging. Bei Reinhardt\pwindex{Reinhardt, Max 9.\,9.\,1873 Baden bei Wien – 30.\,10.\,1943 New York City@\textsc{Reinhardt, Max} (9.\,9.\,1873 Baden bei Wien – 30.\,10.\,1943 New York City), \emph{Theaterleiter, Regisseur, Schauspieler}|pw} wären zweifellos auch die
               Massenszenen besser herausgekommen als es bei uns der Fall sein wird. Aber die übrige
               Besetzung hier ist zum grösseren und wichtigeren Teile von der Art, dass keine
               deutsche Bühne sie heute besser bieten könnte. Die Bleibtreu\pwindex{Bleibtreu, Hedwig 23.\,12.\,1868 Linz – 24.\,1.\,1958 Wien@\textsc{Bleibtreu, Hedwig} (23.\,12.\,1868 Linz – 24.\,1.\,1958 Wien), \emph{Schauspielerin}|pw} als Frau Klähr, Balaithy\pwindex{Balajthy, Robert von 30.\,10.\,1855 Wien – 30.\,8.\,1924 ebd.@\textsc{Balajthy, Robert von} (30.\,10.\,1855 Wien – 30.\,8.\,1924 ebd.), \emph{Schauspieler}|pw}{ }{\pb}als Eschenbacher, Tressler\pwindex{Tressler, Otto 13.\,4.\,1871 Stuttgart – 27.\,4.\,1965 Wien@\textsc{Tressler, Otto} (13.\,4.\,1871 Stuttgart – 27.\,4.\,1965 Wien), \emph{Schauspieler, Bildhauer}|pw} als Etzelt, Korff\pwindex{Korff, Arnold 2.\,8.\,1868 Wien – 2.\,6.\,1944 New York City@\textsc{Korff, Arnold} (2.\,8.\,1868 Wien – 2.\,6.\,1944 New York City), \emph{Schauspieler}|pw} als Wachshuber, Hartmann\pwindex{Hartmann, Ernst 8.\,1.\,1844 Hamburg – 10.\,10.\,1911 Wien@\textsc{Hartmann, Ernst} (8.\,1.\,1844 Hamburg – 10.\,10.\,1911 Wien), \emph{Schauspieler}|pw} als Herzog, Heine\pwindex{Heine, Albert 16.\,11.\,1867 Braunschweig – 13.\,4.\,1949 Westerland@\textsc{Heine, Albert} (16.\,11.\,1867 Braunschweig – 13.\,4.\,1949 Westerland), \emph{Theaterleiter, Schauspieler}|pw} als
               Assalagny, von der Medelsky\pwindex{Medelsky, Lotte 18.\,5.\,1880 Wien – 4.\,12.\,1960 Nussdorf am Attersee@\textsc{Medelsky, Lotte} (18.\,5.\,1880 Wien – 4.\,12.\,1960 Nussdorf am Attersee), \emph{Schauspielerin}|pw}, der Wolgemut\pwindex{Wohlgemuth, Else 1.\,1.\,1881 Berlin – 30.\,5.\,1972 Wien@\textsc{Wohlgemuth, Else} (1.\,1.\,1881 Berlin – 30.\,5.\,1972 Wien), \emph{Schauspielerin}|pw}, von Reimers\pwindex{Reimers, Georg 4.\,4.\,1860 Altona – 15.\,4.\,1936 Wien@\textsc{Reimers, Georg} (4.\,4.\,1860 Altona – 15.\,4.\,1936 Wien), \emph{Schauspieler}|pw} und Strassny\pwindex{Strassni, Fritz 14.\,12.\,1868 Wien – 14.\,9.\,1942 Konzentrationslager Theresienstadt@\textsc{Strassni, Fritz} (14.\,12.\,1868 Wien – 14.\,9.\,1942 Konzentrationslager Theresienstadt), \emph{Schauspieler}|pw} und Heller\pwindex{Heller, Eduard 1854 – 28.\,2.\,1935 Wien@\textsc{Heller, Eduard} (1854 – 28.\,2.\,1935 Wien), \emph{Schauspieler}|pw} und Andern ganz zu geschweigen, das sind
               Leistungen im Einzelnen, meist auch im Zusammenspiel, dass Du, lieber Hermann, wenn
               Du die Vorstellung zu sehen bekämest gewiss nicht von herumdilettierenden
               Herrschaften sprächest, sondern das denen überliessest (es wird ja nicht an ihnen
               fehlen) denen vorgefasste Meinungen den teuersten und ach so bequemen Besitz
               bedeuten.\pend
           
\pstart
           Nun will ich Dir noch von Herzen glückliche Vortragsreise wünschen und \strikeout{die} diesmal \substVorne{}\textsuperscript{hoffentlich}\substDazwischen{}die Hoffnung\substHinten{} nicht vergeblich\strikeout{e Hoffnung} aussprechen Dich
               und Deine verehrte Frau Gemahlin\pwindex{Bahr-Mildenburg, Anna 29.\,11.\,1872 Wien – 27.\,1.\,1947 ebd.@\textsc{Bahr-Mildenburg, Anna} (29.\,11.\,1872 Wien – 27.\,1.\,1947 ebd.), \emph{Sängerin}|pwv} recht bald nach Deiner Rückkehr bei uns zu sehen. Ich selbst fahre
               etwa am 7. Dezember nach München\oindex{München@\textbf{München}|pw} (\label{K_L01981-5v}\edtext{Vorlesung}{\lemma{\textnormal{\emph{Vorlesung}}}\Cendnote{\textnormal{am
                     9. 12. 1909}}}\label{K_L01981-5}) und auch nach Partenkirchen\oindex{Garmisch-Partenkirchen@\textbf{Garmisch-Partenkirchen}, \emph{Hauptstadt}|pw} zu meiner
                  Schwägerin\pwindex{Steinrück, Elisabeth 19.\,11.\,1885 – 7.\,4.\,1920 Partenkirchen@\textsc{Steinrück, Elisabeth} (19.\,11.\,1885 – 7.\,4.\,1920 Partenkirchen)|pwv}. Um den 15. herum denke ich wieder daheim zu sein.\pend
           
\pstart
           {[}hs.:{]} Mit vielen treuen Grüßen{\\[\baselineskip]}Dein{\\[\baselineskip]}\spacefill\mbox{Arthur.}\pend
           \leftskip=0em{}\selectlanguage{ngerman}\endnumbering\briefempfaengerindex{Bahr, Hermann@\textsc{Bahr, Hermann}!zzzSchnitzler, Arthur@\emph{von Arthur Schnitzler}!1910-11-171@{17. 11. 1910}|)be}\mylabel{L01981h}  \newcommand{\dateiname}{L01981}\newcommand{\titel}{Arthur Schnitzler an Hermann Bahr, 17. 11. 1910}\newcommand{\editorInnen}{Herausgegeben von Martin Anton Müller}%% latex-leseansicht-abspann.tex
%% Abspann für die Leseansicht.
%% Der Schalter \ifkorrekturansicht ist bereits durch den Vorspann gesetzt.

%% latex-abspann.tex
%% Gemeinsamer Abspann für Korrekturansicht und Leseansicht.
%% Setzt den Schalter \ifkorrekturansicht voraus (gesetzt in den
%% einbindenden Dateien latex-korrekturansicht-abspann.tex bzw.
%% latex-leseansicht-abspann.tex).
%% ---------------------------------------------------------------

\normalsize

% Das esempio-Environment wird nur in der Leseansicht benötigt
\ifkorrekturansicht\else
\newenvironment{esempio}[3]%
{
    \vspace{1.5ex}
    \rlap{\underline{#1}}
    \par
    \setlength{\parindent}{0cm}
    \nopagebreak
    \leftskip=#2cm
    \rightskip=#3cm
}
{
    \par
}
\fi

\doendnotes{C}
\bigskip
\vfill

\clearpage

\footnotesize

\ifkorrekturansicht
  \lohead{\textsc{register}}
\fi

% theindex-Environment neu definieren ohne reledmac
\makeatletter
\renewenvironment{theindex}{%
  \ifkorrekturansicht
    \section*{\indexname}%
  \else
    \subsubsection*{Index der erwähnten Entitäten}%
  \fi
  \setlength{\parindent}{0pt}%
  \setlength{\parskip}{0pt plus 0.3pt}%
  \let\item\@idxitem
}{%
  \ifkorrekturansicht\clearpage\fi
}
\makeatother

\IfFileExists{\jobname-pw.ind}{\input{\jobname-pw.ind}}{}

% Quellenangabe nur in der Leseansicht
\ifkorrekturansicht\else
% Fallback-Definitionen, falls die .tex-Datei \titel etc. nicht gesetzt hat
\providecommand{\titel}{}
\providecommand{\editorInnen}{}
\providecommand{\dateiname}{\jobname}

\vspace{3cm}

\vfill

\footnotesize
\textsc{Quelle}: \titel. Herausgegeben von {\editorInnen}. In: \emph{Arthur Schnitzler: Briefwechsel mit Autorinnen und Autoren}.
 Digitale Edition, https://schnitzler-briefe.acdh.oeaw.ac.at/{\dateiname}.html (Stand \today)
\fi

\end{document}


