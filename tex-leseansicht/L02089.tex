%% latex-korrekturansicht-vorspann.tex
%% Vorspann für die Korrekturansicht.
%% Lädt die gemeinsame Datei latex-vorspann.tex mit gesetztem Schalter.

\newif\ifkorrekturansicht
\korrekturansichttrue

\input{../tex-inputs/latex-vorspann}


\section[Hugo von Hofmannsthal an Arthur Schnitzler, 21. 9. 1912]{L02089 Hugo von Hofmannsthal an Arthur Schnitzler, 21. 9. 1912}
\nopagebreak\mylabel{L02089v}
\rehead{ }\normalsize\beginnumbering\briefempfaengerindex{Schnitzler, Arthur@\textsc{Schnitzler, Arthur}!zzzHofmannsthal, Hugo von@\emph{von Hugo von Hofmannsthal}!1912-09-211@{21. 9. 1912}|(be}
\toendnotes[C]{\smallbreak\pagebreak[2]}\Standort{CUL, Schnitzler, B 43.}
\physDesc{Bildpostkarte, 340 Zeichen
\newline{}Handschrift: schwarze Tinte, deutsche Kurrent
\newline{}Versand: Stempel: »\nobreak{}\oindex{Sankt Michael@\textbf{Sankt Michael}, \emph{Bezirk (A.BZK)}|pwk}St. Michael in Eppan, 22. IX. 12\nobreak{}«.  
\newline{}Schnitzler: mit Bleistift die Jahreszahl ergänzt: »912« 
\newline{}Ordnung: 1) mit Bleistift von unbekannter Hand nummeriert: »\strikeout{330}«  2) mit Bleistift von unbekannter Hand nummeriert:
                                    »340«}
\buchAbdrucke{\weitereDrucke{Hugo von Hofmannsthal, Arthur Schnitzler: \emph{Briefwechsel}. Frankfurt am Main: \emph{S. Fischer} 1964, S. 269.} }\toendnotes[C]{\smallbreak}\pstart{}{\pb}\textsc{Herrn D\textsuperscript{r} Arthur Schnitzler}\pend{}\pstart{}\textsc{Wien}\oindex{Wien@\textbf{Wien}, \emph{A.ADM2}|pw}\pend{}\pstart{}\textsc{XVIII Sternwartestrasse 71\oindex{Sternwartestrasse 71@\textbf{Sternwartestraße 71}, \emph{Wohngebäude (K.WHS)}|pw}}.\pend{}{\bigskip}
\pstart
           \noindent{}\centering{}{\pb}\textcolor{gray}{\textbf{Schloss Gandegg\oindex{Schloss Gandegg@\textbf{Schloss Gandegg}, \emph{Schloss (K.SLS)}|pw} in Eppan (Überetsch)\oindex{Eppan an der Weinstrasse@\textbf{Eppan an der Weinstraße}, \emph{A.ADM3}|pw}. Tirol\oindex{Suedtirol@\textbf{Südtirol}, \emph{A.ADM2}|pw}.}}\pend
           \vspace{1em}
\pstart
           {\pb}Gandegg\oindex{Schloss Gandegg@\textbf{Schloss Gandegg}, \emph{Schloss (K.SLS)}|pw}{ }21. IX.\pend
           \vspace{0.5em}
\pstart
           Dies Schloſs\oindex{Schloss Gandegg@\textbf{Schloss Gandegg}, \emph{Schloss (K.SLS)}|pwv}{ }ſteht leer, wir habens gemiethet und genießen ein
               letztes oder erſtes Stück So{\geminationm}er. Ich verſuche – was Sie
               beim letzten Mal als Wunſch ausgeſprochen haben, mein lieber Arthur: – zu erzählen.
               Der Stoff\pwindex{Andreas oder Die Vereinigten@\emph{Andreas oder Die Vereinigten}|pwv} iſt ſchön, ich will
               mir viel Mühe geben. Von Herzen\pend
           \pstart Ihr \spacefill\mbox{Hugo.}\pend{}\selectlanguage{ngerman}\endnumbering\briefempfaengerindex{Schnitzler, Arthur@\textsc{Schnitzler, Arthur}!zzzHofmannsthal, Hugo von@\emph{von Hugo von Hofmannsthal}!1912-09-211@{21. 9. 1912}|)be}\mylabel{L02089h}  \normalsize

\doendnotes{C}
\bigskip
\vfill

\clearpage

\footnotesize

\lohead{\textsc{register}}

% Definiere theindex-Environment komplett neu ohne reledmac
\makeatletter
\renewenvironment{theindex}{%
  \section*{\indexname}%
  \setlength{\parindent}{0pt}%
  \setlength{\parskip}{0pt plus 0.3pt}%
  \let\item\@idxitem
}{%
  \clearpage
}
\makeatother

\IfFileExists{\jobname-pw.ind}{\input{\jobname-pw.ind}}{}

\end{document}

      