\input{../tex-inputs/latex-pdf-vorspann}
\begin{center}
            \textcolor{red}{ENTWURF. ENTZIFFERUNG NOCH NICHT KORREKTURGELESEN}
                      \end{center}
            
               \section[Hugo von Hofmannsthal an Arthur Schnitzler, 21. 9. 1912]{ Hugo von Hofmannsthal an Arthur Schnitzler, 21. 9. 1912}\nopagebreak\mylabel{v}\rehead{ }\begin{ledgroupsized}[t]{13cm}\normalsize\beginnumbering\briefempfaengerindex{Schnitzler, Arthur@\textsc{Schnitzler, Arthur}!zzzHofmannsthal, Hugo von@\emph{von Hugo von Hofmannsthal}!1912-09-211@{21. 9. 1912}|(be} \toendnotes[C]{\smallbreak\pagebreak[2]} \Standort{CUL, Schnitzler, B 43.}
\physDesc{Bildpostkarte
\newline{}Handschrift: schwarze Tinte, deutsche Kurrent\newline{}Versand: Stempel: »\nobreak{}\oindex{Sankt Michael@\textbf{Sankt Michael}|pwk}St. Michael in Eppan, 22. IX. 12\nobreak{}«.  
\newline{}Schnitzler: mit Bleistift die Jahreszahl ergänzt: »912« \newline{}Ordnung: 1) mit Bleistift von unbekannter Hand nummeriert: »\strikeout{330}« 2) mit Bleistift von unbekannter Hand nummeriert: »340«}\buchAbdrucke{\weitereDrucke{Hugo von Hofmannsthal, Arthur Schnitzler: \emph{Briefwechsel}. Hg. Therese Nickl und Heinrich Schnitzler. Frankfurt am Main: \emph{S. Fischer} 1964, S. 269.} }\toendnotes[C]{\smallbreak}\pstart{}{\pb}\textsc{Herrn
                        D\textsuperscript{r} Arthur Schnitzler}\pend{}\pstart{}\textsc{Wien}\oindex{Wien@\textbf{Wien}|pw}\pend{}\pstart{}\textsc{XVIII Sternwartestrasse 71\oindex{Sternwartestrasse@\textbf{Sternwartestraße}|pw}}.\pend{}{\bigskip}\pstart
           \noindent{}\centering{}\textcolor{gray}{\textbf{{\pb}Schloss Gandegg\oindex{Schloss Gandegg@\textbf{Schloss Gandegg}|pw} in Eppan (Überetsch)\oindex{Eppan an der Weinstrasse@\textbf{Eppan an der Weinstraße}|pw}. Tirol\oindex{Suedtirol@\textbf{Südtirol}|pw}.}}\pend
           \pstart
           {\pb}Gandegg\oindex{Schloss Gandegg@\textbf{Schloss Gandegg}|pw}{ }21. IX.\pend
           \pstart
           Dies Schloſs\oindex{Schloss Gandegg@\textbf{Schloss Gandegg}|pwv}{ }ſteht leer, wir habens gemiethet und genießen ein
               letztes oder erſtes Stück So{\geminationm}er. Ich verſuche – was Sie
               beim letzten Mal als Wunſch ausgeſprochen haben, mein lieber Arthur: – zu erzählen.
               Der Stoff\pwindex{Hofmannsthal, Hugo von 01.02.1874 – 15.07.1929@\textsc{Hofmannsthal, Hugo von} (01.02.1874 – 15.07.1929), \emph{Schriftsteller}!Andreas oder Die Vereinigten1930@\strich\emph{Andreas oder Die Vereinigten} {[}1930{]}|pwv} iſt ſchön, ich will mir
               viel Mühe geben. Von Herzen\pend
           \pstart Ihr \spacefill\mbox{Hugo.}\pend{}\endnumbering\briefempfaengerindex{Schnitzler, Arthur@\textsc{Schnitzler, Arthur}!zzzHofmannsthal, Hugo von@\emph{von Hugo von Hofmannsthal}!1912-09-211@{21. 9. 1912}|)be}\mylabel{h}\end{ledgroupsized}  \newcommand{\dateiname}{L02089}\newcommand{\titel}{Hugo von Hofmannsthal an Arthur Schnitzler, 21. 9. 1912}\newcommand{\editorInnen}{Martin Anton Müller und Gerd-Hermann Susen}\input{../tex-inputs/latex-pdf-abspann}
      