%% latex-leseansicht-vorspann.tex
%% Vorspann für die Leseansicht.
%% Lädt die gemeinsame Datei latex-vorspann.tex mit nicht gesetztem Schalter.

\newif\ifkorrekturansicht
\korrekturansichtfalse

\input{../tex-inputs/latex-vorspann}


\section[Theodor Herzl an Arthur Schnitzler, 29. 7. 1892]{L03823 Theodor Herzl an Arthur Schnitzler, 29. 7. 1892}
\nopagebreak\mylabel{L03823v}
\rehead{ }\normalsize\beginnumbering\briefempfaengerindex{Schnitzler, Arthur@\textsc{Schnitzler, Arthur}!zzzHerzl, Theodor@\emph{von Theodor Herzl}!1892-07-292@{29. 7. 1892}|(be}
\toendnotes[C]{\smallbreak\pagebreak[2]}
\correspDesc{Versand  durch Theodor Herzl am 29. 7. 1892 in Houlgate
\newline{}Erhalt  durch Arthur Schnitzler im Zeitraum [30. 7. 1892
                  – 3. 8. 1892?] in Wien}\toendnotes[C]{\smallbreak}
\Standort{CUL, Schnitzler, B 39.}
\physDesc{Brief, 3 Blätter, 11 Seiten, 8164 Zeichen
\newline{}Handschrift: schwarze Tinte, lateinische Kurrent (\noindent{}Nummerierung des dritten Bogens: »3«)
\newline{}Ordnung: 1) mit Bleistift von unbekannter Hand nummeriert: »4«  2) mit blauem Buntstift von Leon Kellner\pwindex{Kellner, Leon 17.\,4.\,1859 Tarnów – 5.\,12.\,1928 Wien@\textsc{Kellner, Leon} (17.\,4.\,1859 Tarnów – 5.\,12.\,1928 Wien), \emph{Zionist, Literaturhistoriker, Anglist}|pw} Markierung von Stellen für
                                 die Publikation 3) mit Bleistift mutmaßlich von Leon Kellner\pwindex{Kellner, Leon 17.\,4.\,1859 Tarnów – 5.\,12.\,1928 Wien@\textsc{Kellner, Leon} (17.\,4.\,1859 Tarnów – 5.\,12.\,1928 Wien), \emph{Zionist, Literaturhistoriker, Anglist}|pw} Markierung interessanter
                                 Stellen}
\buchAbdrucke{\weitereDrucke{1) \pwindex{Kellner, Leon 17.\,4.\,1859 Tarnów – 5.\,12.\,1928 Wien@\textsc{Kellner, Leon} (17.\,4.\,1859 Tarnów – 5.\,12.\,1928 Wien), \emph{Zionist, Literaturhistoriker, Anglist}!Theodor Herzls Lehrjahre (1860–1895). Nach den handschriftlichen Quellen@\strich\emph{Theodor Herzls Lehrjahre (1860–1895). Nach den handschriftlichen Quellen}|pwk}\emph{[Auszug].} In: Leon Kellner: \emph{Theodor Herzls Lehrjahre (1860–1895). Nach den handschriftlichen Quellen}. Wien, Berlin: \emph{R. Löwit-Verlag} 1920, S. 108.} \weitereDrucke{2) H. M. [=Hermann Menkes]: \emph{Briefwechsel zwischen Theodor Herzl und Artur Schnitzler. Die Lehrjahre des berühmten Zionisten.} In: \emph{Neues Wiener Journal}, Jg. 28, Nr. 9540, 29. 5. 1920, S. 3–4.} \weitereDrucke{3) \emph{Herzl-Briefe}. Herausgegeben und eingeleitet Manfred Georg. Berlin: \emph{Brandusche Verlagsbuchhandlung} [1935], S. 28–30.} \weitereDrucke{4) \pwindex{Schnitzler, Olga 17.\,1.\,1882 Wien – 13.\,1.\,1970 Lugano@\textsc{Schnitzler, Olga} (17.\,1.\,1882 Wien – 13.\,1.\,1970 Lugano), \emph{Schauspielerin, Sängerin}!Spiegelbild der Freundschaft@\strich\emph{Spiegelbild der Freundschaft}|pwk}Olga Schnitzler: \emph{Spiegelbild der Freundschaft}. Salzburg: \emph{Residenz-Verlag} 1962, S. 82–84.} \weitereDrucke{5) Theodor Herzl: \emph{Briefe und
                        autobiographische Notizen 1866–1895}. Bearbeitet von Johannes Wachten in Zusammenarbeit mit Chaya Harel, Daisy Tycho und Manfred Winkler. Berlin, Frankfurt am Main, Wien: \emph{Propyläen} 1983, S. 498–502 (Briefe und Tagebücher. Herausgegeben von Alex Bein, Hermann Greive, Moshe Schaerf, Julius H. Schoeps und Johannes Wachten, 1).} }\toendnotes[C]{\smallbreak}
\pstart
           \raggedleft{}{\pb}Beuzeval-Houlgate\oindex{Houlgate@\textbf{Houlgate}|pw}{\\}(Calvados) France\oindex{Calvados@\textbf{Calvados}, \emph{Verwaltungsgebiet}|pw}{\\}Villa des oeillets\oindex{Villa des Oeillets@\textbf{Villa des Oeillets}, \emph{Hotel}|pw}{\\}29/7 92\pend
           
\pstart{}Lieber D\textsuperscript{r} Schnitzler!\pend\vspace{0.5em}
\pstart
           D\textsuperscript{r}{ }Goldmann\pwindex{Goldmann, Paul 31.\,1.\,1865 Breslau – 25.\,9.\,1935 Wien@\textsc{Goldmann, Paul} (31.\,1.\,1865 Breslau – 25.\,9.\,1935 Wien), \emph{Schriftsteller, Journalist}|pw} hat mir eine grosse Freude bereitet,
               als er mir zur Bekanntschaft mit Ihrem »Märchen\pwindex{Schnitzler, Arthur 15.\,5.\,1862 Wien – 21.\,10.\,1931 ebd.@\textsc{Schnitzler, Arthur} (15.\,5.\,1862 Wien – 21.\,10.\,1931 ebd.), \emph{Schriftsteller, Mediziner}!Märchen. Schauspiel in drei Aufzügen@\strich\emph{Das Märchen. Schauspiel in drei Aufzügen}|pw}«
               verhalf. Wir sassen damals auf der Journalistentribüne des Palais Bourbon,\oindex{Palais Bourbon@\textbf{Palais Bourbon}, \emph{Regierungsgebäude}|pw} wohin mich ja der wunderliche Lauf meines
               Lebens gebracht hat, u. sprachen von Wien\oindex{Wien@\textbf{Wien}, \emph{Verwaltungsgebiet}|pw}, das mir
               mit den Leiden u. Beschwerlichkeiten an denen es für mich so voll war zu verblassen
               beginnt. Goldmann\pwindex{Goldmann, Paul 31.\,1.\,1865 Breslau – 25.\,9.\,1935 Wien@\textsc{Goldmann, Paul} (31.\,1.\,1865 Breslau – 25.\,9.\,1935 Wien), \emph{Schriftsteller, Journalist}|pw} war sehr entzückt von diesem
                  Wien\oindex{Wien@\textbf{Wien}, \emph{Verwaltungsgebiet}|pw}. Stadt seiner Freunde! Er nannte Ihren
               Namen. Ich war ziemlich erstaunt, Sie so rühmen zu hören. Gestatten Sie mir das zu
               sagen. Ich hatte, obwol mir {\pb}schon
               einige Ihrer Dialogsachen die so viel Geist sprühen bekannt waren, doch nicht seine
               Meinung von Ihrem Talent. Persönlich waren Sie mir aber geradezu unsympathisch. Ich
               hatte Sie in der letzten Zeit in Gesellschaft einiger \label{K_L03823-1v}\edtext{»Jungen«}{\lemma{\textnormal{\emph{»Jungen«}}}\Cendnote{\textnormal{Mitglieder der literarischen Vereinigung \emph{Jung
                     Wien}\orgindex{Jung Wien@Jung Wien|pwk}}}}\label{K_L03823-1} von Profession gesehen, und die früheren Begegnungen hatten mich in Ihnen
               einen dünkelhaften Menschen sehen lassen, der auf allerlei Albernheiten der
               Gesellschaft herumritt.\pend
           
\pstart
           So thöricht und gewissenlos ist unser Urtheil. Vielleicht ist
               Ihnen auch schon Ähnliches passirt – vielleicht sogar mit mir. Ich bin jedenfalls
               voll von Reue über meine Leichtfertigkeit, u. ich bat Ihnen Alles ab, als ich Ihr Märchen\pwindex{Schnitzler, Arthur 15.\,5.\,1862 Wien – 21.\,10.\,1931 ebd.@\textsc{Schnitzler, Arthur} (15.\,5.\,1862 Wien – 21.\,10.\,1931 ebd.), \emph{Schriftsteller, Mediziner}!Märchen. Schauspiel in drei Aufzügen@\strich\emph{Das Märchen. Schauspiel in drei Aufzügen}|pw} gelesen hatte. Denn wir, die selber
               wissen, welche Schmerzen sich uns zum Gedicht verklären, sehen ja auch den {\pb}Menschen deutlich, der uns das
               Menschliche erzählt. Der Rückschluss auf den Schreiber – wenigstens auf die Zeit
               seiner Entwicklung, in der es entstand – aus dem Geschriebenen ist ganz untrüglich.
               Und wie ich dieses feine Werk\pwindex{Schnitzler, Arthur 15.\,5.\,1862 Wien – 21.\,10.\,1931 ebd.@\textsc{Schnitzler, Arthur} (15.\,5.\,1862 Wien – 21.\,10.\,1931 ebd.), \emph{Schriftsteller, Mediziner}!Märchen. Schauspiel in drei Aufzügen@\strich\emph{Das Märchen. Schauspiel in drei Aufzügen}|pwv}
               bewegt genoss, hatte ich ungefähr den Gedanken \strikeout{, der}
               jenes Sentimentalen, der eines versäumten Weibes gedenkt: \label{K_L03823-2v}\edtext{oh, \begin{otherlanguage}{french}toi que j’ eusse aimé – oh toi qui
                  le savais\end{otherlanguage}}{\lemma{\textnormal{\emph{oh, … savais}}}\Cendnote{\textnormal{französisch, Gedichtzitat nach
                  Baudelaire: »O Du, Dich hätte ich geliebt, o Du, Du hättest es
                     gewusst!«, im Original an weiblichen, bei Herzl\pwindex{Herzl, Theodor 2.\,5.\,1860 Budapest – 3.\,7.\,1904 Edlach@\textsc{Herzl, Theodor} (2.\,5.\,1860 Budapest – 3.\,7.\,1904 Edlach), \emph{Schriftsteller, Journalist}|pwk} an männlichen Adressaten. Charles Baudelaire\pwindex{Baudelaire, Charles 9.\,4.\,1821 Paris – 31.\,8.\,1867 ebd.@\textsc{Baudelaire, Charles} (9.\,4.\,1821 Paris – 31.\,8.\,1867 ebd.), \emph{Schriftsteller}|pwk}: À une passante. In:
                        \emph{Les fleurs du mal}\pwindex{Baudelaire, Charles 9.\,4.\,1821 Paris – 31.\,8.\,1867 ebd.@\textsc{Baudelaire, Charles} (9.\,4.\,1821 Paris – 31.\,8.\,1867 ebd.), \emph{Schriftsteller}!fleurs du mal@\strich\emph{Les fleurs du mal}|pwk}, Paris\oindex{Paris@\textbf{Paris}, \emph{Hauptstadt}|pwk}: Poulet-Malassis et de Broise 1861, N°
                     XCIII, S. 216–217.}}}\label{K_L03823-2}. Ja, wahrhaftig, Sie hätten sich längst denken
               können, dass ich Sie lieben würde, wenn ich das von Ihnen wüsste, was ich jetzt
               weiss.\pend
           
\pstart
           Dass Sie mich, der ich doch \label{K_L03823-3v}\edtext{in
               Ihrem Bezirk\oindex{I., Innere Stadt@\textbf{I., Innere Stadt}, \emph{Verwaltungsgebiet}|pwv}}{\lemma{\textnormal{\emph{in
               Ihrem Bezirk}}}\Cendnote{\textnormal{Vor seinem Aufbruch nach Paris\oindex{Paris@\textbf{Paris}, \emph{Hauptstadt}|pwk} hatte Herzl\pwindex{Herzl, Theodor 2.\,5.\,1860 Budapest – 3.\,7.\,1904 Edlach@\textsc{Herzl, Theodor} (2.\,5.\,1860 Budapest – 3.\,7.\,1904 Edlach), \emph{Schriftsteller, Journalist}|pwk} in der Marc-Aurel-Str. 7\oindex{Wien@\textbf{Wien}!I., Innere Stadt@\textbf{I., Innere Stadt}!Marc-Aurel-Straße 7@\textbf{Marc-Aurel-Straße 7}, \emph{Wohngebäude}|pwk}
                  gewohnt, Schnitzler lebte zu diesem
                  Zeitpunkt in der Giselastr. 11\oindex{Wien@\textbf{Wien}!I., Innere Stadt@\textbf{I., Innere Stadt}!Kärntnerring 12/Bösendorferstraße 11@\textbf{Kärntnerring 12/Bösendorferstraße 11}, \emph{Wohngebäude}|pwk}.}}}\label{K_L03823-3}
               wohnte, nicht aufsuchten, ist jetzt für mich eine Beschämung wie wenn man nicht zu
               einem Fest mit Anderen geladen wird, u. es würde mich zur Demuth mahnen wenn ich
               solcher Winke bedürfte. Denn offenbar haben Sie {\pb}in meinen Schmierereien, die an
               öffentlicheren Orten erschienen, nie einen solchen Ton gefunden, der Ihnen zum Herzen
               ging. Sie sind zu fein, mein lieber Poet, um in dieser Bemerkung nicht genau das zu
               finden, was sie enthalten soll.\pend
           
\pstart
           Gerade zu solchen Leuten, wie Sie sind, hätte ich
               immer gern gesprochen. Scheint mir nicht gelungen zu sein.\pend
           
\pstart
           Ich, mein lieber
               Schnitzler bin übrigens bereits mit mir in Ordnung. Auf dem Theater, mit dem ich
               abgeschlossen habe, ist es mir schlecht u. närrisch ergangen. \label{K_L03823-4v}\edtext{Stücke\pwindex{Herzl, Theodor 2.\,5.\,1860 Budapest – 3.\,7.\,1904 Edlach@\textsc{Herzl, Theodor} (2.\,5.\,1860 Budapest – 3.\,7.\,1904 Edlach), \emph{Schriftsteller, Journalist}!Muttersöhnchen. Lustspiel in vier Akten@\strich\emph{Muttersöhnchen. Lustspiel in vier Akten}|pwv}\pwindex{Herzl, Theodor 2.\,5.\,1860 Budapest – 3.\,7.\,1904 Edlach@\textsc{Herzl, Theodor} (2.\,5.\,1860 Budapest – 3.\,7.\,1904 Edlach), \emph{Schriftsteller, Journalist}!Enttäuschten. Komödie in vier Acten@\strich\emph{Die Enttäuschten. Komödie in vier Acten}|pwv}, an die ich
                  glaubte}{\lemma{\textnormal{\emph{Stücke, … glaubte}}}\Cendnote{\textnormal{zum Beispiel \emph{Die Enttäuschten}\pwindex{Herzl, Theodor 2.\,5.\,1860 Budapest – 3.\,7.\,1904 Edlach@\textsc{Herzl, Theodor} (2.\,5.\,1860 Budapest – 3.\,7.\,1904 Edlach), \emph{Schriftsteller, Journalist}!Enttäuschten. Komödie in vier Acten@\strich\emph{Die Enttäuschten. Komödie in vier Acten}|pwk} und \emph{Muttersöhnchen}\pwindex{Herzl, Theodor 2.\,5.\,1860 Budapest – 3.\,7.\,1904 Edlach@\textsc{Herzl, Theodor} (2.\,5.\,1860 Budapest – 3.\,7.\,1904 Edlach), \emph{Schriftsteller, Journalist}!Muttersöhnchen. Lustspiel in vier Akten@\strich\emph{Muttersöhnchen. Lustspiel in vier Akten}|pwk}}}}\label{K_L03823-4}, in denen ich künstlerisch strebte, kamen gar nicht zum Vorschein. Wenn ich
               in einer gewissen gierigen Verzweiflung zum Handwerk hinabstieg, wurde ich aufgeführt
               – und \strikeout{von \textcolor{gray}{×}\-\textcolor{gray}{×}\-\textcolor{gray}{×}\-\textcolor{gray}{×}} verhöhnt. \strikeout{\textcolor{gray}{×}\-\textcolor{gray}{×}\-\textcolor{gray}{×}{ }\textcolor{gray}{×}\-\textcolor{gray}{×}\-\textcolor{gray}{×}\-\textcolor{gray}{×}\-\textcolor{gray}{×}\-\textcolor{gray}{×}\-\textcolor{gray}{×}\-\textcolor{gray}{×}{ }\textcolor{gray}{×}\-\textcolor{gray}{×}\-\textcolor{gray}{×}\-\textcolor{gray}{×}{ }\textcolor{gray}{×}\-\textcolor{gray}{×}\-\textcolor{gray}{×}\-\textcolor{gray}{×}\-\textcolor{gray}{×}\-\textcolor{gray}{×}{ }\textcolor{gray}{×}\-\textcolor{gray}{×}\-\textcolor{gray}{×}\-\textcolor{gray}{×}{ }\textcolor{gray}{×}\-\textcolor{gray}{×}\-\textcolor{gray}{×}{ }\textcolor{gray}{×}\-\textcolor{gray}{×}\-\textcolor{gray}{×}\-\textcolor{gray}{×}\-\textcolor{gray}{×}} wenn ich – was äusserst selten geschieht – an meinen Platz in der deutschen
               Literaturwelt denke, muss ich ergötzt lachen. Ich stehe weit {\pb}hinter Triesch\pwindex{Triesch, Friedrich Gustav 16.\,6.\,1845 Wien – 24.\,5.\,1907 ebd.@\textsc{Triesch, Friedrich Gustav} (16.\,6.\,1845 Wien – 24.\,5.\,1907 ebd.), \emph{Schriftsteller}|pw}. Muss Ihnen aber sagen, dass ich dadurch nicht bitter geworden bin.
               Alle Schmerzen erziehen uns. Und mit einer heiteren Philosophie, die ich früher nicht
               kannte, sehe ich dem Treiben auf dem Markt zu. »Junge«, »Alte«, Realisten, u. s. w.
                  \label{K_L03823-5v}\edtext{\begin{otherlanguage}{french}je m’en fous\end{otherlanguage}}{\lemma{\textnormal{\emph{je m’en fous}}}\Cendnote{\textnormal{französisch: das schert mich
                  nicht}}}\label{K_L03823-5}.\pend
           
\pstart
           Wenn ich aber so ein Talent wie Ihres aufblühen sehe, freue ich
               mich, wie wenn ich nie ein Literat, das heisst ein engherziger unduldsamer neidischer
               boshafter Tropf gewesen wäre, freue mich wie über die Nelken da unten im Garten, die
               erwachen. \strikeout{von \textcolor{gray}{×}\-\textcolor{gray}{×}\-\textcolor{gray}{×}\-\textcolor{gray}{×}{ }\textcolor{gray}{×}\-\textcolor{gray}{×}\-\textcolor{gray}{×}{ }\textcolor{gray}{×}\-\textcolor{gray}{×}\-\textcolor{gray}{×}\-\textcolor{gray}{×}\-\textcolor{gray}{×}{ }\textcolor{gray}{×}\-\textcolor{gray}{×}\-\textcolor{gray}{×}}\pend
           
\pstart
           Nehmen Sie es mir nicht übel, wenn ich so gesetzt spreche, es soll keine
               verletzendere Ueberlegenheit sein, als die eines etwas älteren Bruders. Denn Ihre Art
               u schreiben muthet mich ganz verwandtschaftlich an. So ungefähr mein Lieber, hätte
               ich wol auch {\pb}schreiben mögen.\pend
           
\pstart
           Wenn ich nicht sehr irre, sind Sie auf dem rechten Weg. Ich hoffe bestimmt, dass Sie
               die reizendsten Lustspiele schreiben werden, die wir seit Sardou\pwindex{Sardou, Victorien 7.\,9.\,1831 Paris – 8.\,11.\,1908 ebd.@\textsc{Sardou, Victorien} (7.\,9.\,1831 Paris – 8.\,11.\,1908 ebd.), \emph{Schriftsteller}|pw} hatten.\pend
           
\pstart
           Das ist kein Lapsus. Ich meine nicht den Sardou\pwindex{Sardou, Victorien 7.\,9.\,1831 Paris – 8.\,11.\,1908 ebd.@\textsc{Sardou, Victorien} (7.\,9.\,1831 Paris – 8.\,11.\,1908 ebd.), \emph{Schriftsteller}|pw} der Atrappen, Bühnenmätzchen, Sarah-Bernhardt\pwindex{Bernhardt, Sarah 22.\,10.\,1844 Paris – 26.\,3.\,1923 ebd.@\textsc{Bernhardt, Sarah} (22.\,10.\,1844 Paris – 26.\,3.\,1923 ebd.), \emph{Schauspielerin}|pw}rollen u. s. w. sondern den, dessen Ton an einigen Stellen
               der pattes de mouche\pwindex{Sardou, Victorien 7.\,9.\,1831 Paris – 8.\,11.\,1908 ebd.@\textsc{Sardou, Victorien} (7.\,9.\,1831 Paris – 8.\,11.\,1908 ebd.), \emph{Schriftsteller}!letzte Brief. Lustspiel in 3 Akten@\strich\emph{Der letzte Brief. Lustspiel in 3 Akten}|pw} klingt. Werden Sie
               standhaft sein, sich vom Pöbel der Theater u. von den Bengeln der Kritik nicht
               beirren lassen, werden Sie – was ich leider nicht that – sich selber treu bleiben, so
               werden Sie der Alkandi\pwindex{Al-Kindi, Yakub 800 Kufa – 873 Baghdad@\textsc{Al-Kindi, Yakub} (800 Kufa – 873 Baghdad), \emph{Philosoph, Gelehrter, Mathematiker}|pw} werden, dessen Lied
               Alle singen müssen, ob sie wollen oder nicht. Sie haben ein sehr süsses Lied in der
               Kehle, mein lieber Schnitzler, vergröbern {\pb}Sie es um Gotteswillen nicht. Bleiben
               Sie nur sich treu.\pend
           
\pstart
           Ich will damit nicht sagen, dass Sie sich in die Anatol\pwindex{Schnitzler, Arthur 15.\,5.\,1862 Wien – 21.\,10.\,1931 ebd.@\textsc{Schnitzler, Arthur} (15.\,5.\,1862 Wien – 21.\,10.\,1931 ebd.), \emph{Schriftsteller, Mediziner}!Anatol@\strich\emph{Anatol}|pw}-Manier verbeissen sollen. Ich kenne einige dieser reizenden Sachen
               (ich rechnne darauf, dass Sie mir das Buch\pwindex{Schnitzler, Arthur 15.\,5.\,1862 Wien – 21.\,10.\,1931 ebd.@\textsc{Schnitzler, Arthur} (15.\,5.\,1862 Wien – 21.\,10.\,1931 ebd.), \emph{Schriftsteller, Mediziner}!Anatol@\strich\emph{Anatol}|pwv} schicken) u. hatte schon eine kleine Selbstgefälligkeit
               in diesem charmanten Schwerenötherton wahrgenommen. Ich zweifle keinen Augenblick
               daran, dass Sie sich davon freimachen u. höher, hoch schwingen werden.\pend
           
\pstart
           Ich weiss nicht, wie es dem Märchen\pwindex{Schnitzler, Arthur 15.\,5.\,1862 Wien – 21.\,10.\,1931 ebd.@\textsc{Schnitzler, Arthur} (15.\,5.\,1862 Wien – 21.\,10.\,1931 ebd.), \emph{Schriftsteller, Mediziner}!Märchen. Schauspiel in drei Aufzügen@\strich\emph{Das Märchen. Schauspiel in drei Aufzügen}|pw} bei der
               Aufführung gehen wird. Vielleicht schlecht. Lassen Sie sich dadurch nicht eine
               Sekunde verwirren. Es ist ein gutes Stück, dem die Lümmelhaftigkeit von
               Premièrebesuchern nichts anhaben kann. Man wird finden, dass es ein nicht neues
               Problem nicht {\pb}bühnenmässig energisch
               genug behandelt. Speciell die Franzosen\oindex{Frankreich@\textbf{Frankreich}|pw} haben
               derlei Conflicte mit der erforderlichen Rücksichtslosigkeit erschöpft. Ich gestehe
               Ihnen sogar, dass ich auch von dem Schluss nicht ganz »befriedigt« war – so wahr ist
               es, dass wir uns der Rohheit u. Trivialität üblicher Ansichten nie ganz entziehen
               können. Ich glaube, ich erwartete etwas Sterben zum Schlusse. Dann aber erinnerte ich
               mich der vielen anmuthigen Sachen, die vom Anfang bis zu diesem matten Ende vorkamen,
               u. ich dachte \strikeout{mir} dankbar: Bist doch mein lieber
               Poet!\pend
           
\pstart
           Ja, später gab ich Ihnen sogar Recht. So wahrheitsliebend die kleinen
               Äusserlichkeiten bis hinunter zum Salonwienerisch\oindex{Wien@\textbf{Wien}, \emph{Verwaltungsgebiet}|pw}
               (das ich vor 7 Jahren in einem schlechten Stück\pwindex{Herzl, Theodor 2.\,5.\,1860 Budapest – 3.\,7.\,1904 Edlach@\textsc{Herzl, Theodor} (2.\,5.\,1860 Budapest – 3.\,7.\,1904 Edlach), \emph{Schriftsteller, Journalist}!Muttersöhnchen. Lustspiel in vier Akten@\strich\emph{Muttersöhnchen. Lustspiel in vier Akten}|pwv} auch versuchte) sind – das Lustspiel oder wenn Sie
               wollen {\pb}Schauspiel \label{K_L03823-6v}\edtext{foutirt sich}{\lemma{\textnormal{\emph{foutirt sich}}}\Cendnote{\textnormal{kümmert sich nicht}}}\label{K_L03823-6} im Grunde der elenden Wirklichkeit.
               Es spielt zwischen Traum u. Erwachen. Und das ist für mich sein holder Reiz. Und da
               in diesem Gebiet der Stimmung, halber Farben und des Dämmerns wäre eine gewaltsam
               klärende Lösung thöricht u. störend. Ich glaube, Ihr Gefühl hat Ihnen das Rechte
               eingegeben. Das Raisonnement hat Unrecht.\pend
           
\pstart
           Man wird auch sagen: Wo existirt diese Gesellschaft, diese feuilletonredenden
               Künstler, die mit \label{K_L03823-7v}\edtext{Gigerln}{\lemma{\textnormal{\emph{Gigerln}}}\Cendnote{\textnormal{wienerisch: Gecken, Stutzer}}}\label{K_L03823-7}, zwei ungleichen Schwestern und einem ernstnüchternen Mann
               im Hause dieser blinden Mutter verkehren? Ich habe den Eindruck, dass da halbwahre
               Beobachtungen vorliegen, getrübt durch literarische Erinnerungen, aber farbig
               verklärt durch das liebende Auge eines wirklichen Dichters. Ich vermuthe, dass Sie
               allmälig auf die halben Wahrheiten verzichten werden, die nur stören {\pb}u. dass Sie sich wenn Sie
               Empfindungsdramen u. Stimmungslustspiele schreiben, \strikeout{werden} von den kleinen Allotriis des Wiener\oindex{Wien@\textbf{Wien}, \emph{Verwaltungsgebiet}|pw} Dialects u. dgl. zurückziehen werden. Das ist \label{K_L03823-8v}\edtext{\begin{otherlanguage}{french}charge d’atelier\end{otherlanguage}}{\lemma{\textnormal{\emph{charge d’atelier}}}\Cendnote{\textnormal{französisch: Werkstattstück}}}\label{K_L03823-8}, die wahrscheinlich diesmal bei der Aufführung
               gut wirken wird, Sie aber nicht verführen soll, hier Ihre Wirkungen zu suchen.\pend
           
\pstart
           Mit einem Wort, lieber Schnitzler: ich glaube an Sie. Und wenn Sie immer nur zu Ihrem
               eigenen Gaudium schreiben werden, so werden Sie uns Andere sehr erfreuen. Ein
               abschreckendes Beispiel, wohin die Concessionen führen, sehen Sie an mir. Ich könnte
               freilich manchen Umstand meines Lebens als mildernd für meine Irrthümer anführen,
               aber wer kümmert sich um Anderes als das vorliegende Werk.\pend
           
\pstart
           Noch möchte ich wünschen, dass Sie nicht in die Kloaken einer \label{K_L03823-9v}\edtext{\begin{otherlanguage}{french}Coterie\end{otherlanguage}}{\lemma{\textnormal{\emph{Coterie}}}\Cendnote{\textnormal{französisch: Klüngel}}}\label{K_L03823-9}{ }gerathen,
               sich von kleinlichem {\pb}Lob nicht
               verzärteln, von engherzigem Tadel nicht verdrossen machen lassen sollen. Als einem
               feinen Menschen wird Ihnen der Schlamm des Theaterlebens manchmal die Galle heben.
               Man spuckt aus u. geht weiter. Vorwärts.\pend
           
\pstart
           Ich werde Ihre Fortschritte mit Freude und Theilnahme verfolgen. Zu weit ins
               Geschmackvolle glaube ich nicht verirrt zu sein, und wenn ich, der ich ja auch wie
               der erstbeste dumme Junge und wie andere durchgefallene Dramatiker Kritiken schreibe,
               wenn ich Ihre Sachen mit Wohlgefallen lese, wage ich zu vermuthen, dass Sie auch
               einem Publicum gefallen werden.\pend
           
\pstart
           Ich grüsse Sie herzlich als Ihr aufrichtig ergebener{\\[\baselineskip]}\spacefill\mbox{Th. Herzl}\pend
           \leftskip=0em{}\selectlanguage{ngerman}\endnumbering\briefempfaengerindex{Schnitzler, Arthur@\textsc{Schnitzler, Arthur}!zzzHerzl, Theodor@\emph{von Theodor Herzl}!1892-07-292@{29. 7. 1892}|)be}\mylabel{L03823h}
\begin{anhang}
\end{anhang}\newcommand{\dateiname}{L03823}\newcommand{\titel}{Theodor Herzl an Arthur Schnitzler, 29. 7. 1892}\newcommand{\editorInnen}{Selma Jahnke und Martin Anton Müller}%% latex-leseansicht-abspann.tex
%% Abspann für die Leseansicht.
%% Der Schalter \ifkorrekturansicht ist bereits durch den Vorspann gesetzt.

%% latex-abspann.tex
%% Gemeinsamer Abspann für Korrekturansicht und Leseansicht.
%% Setzt den Schalter \ifkorrekturansicht voraus (gesetzt in den
%% einbindenden Dateien latex-korrekturansicht-abspann.tex bzw.
%% latex-leseansicht-abspann.tex).
%% ---------------------------------------------------------------

\normalsize

% Das esempio-Environment wird nur in der Leseansicht benötigt
\ifkorrekturansicht\else
\newenvironment{esempio}[3]%
{
    \vspace{1.5ex}
    \rlap{\underline{#1}}
    \par
    \setlength{\parindent}{0cm}
    \nopagebreak
    \leftskip=#2cm
    \rightskip=#3cm
}
{
    \par
}
\fi

\doendnotes{C}
\bigskip
\vfill

\clearpage

\footnotesize

\ifkorrekturansicht
  \lohead{\textsc{register}}
\fi

% theindex-Environment neu definieren ohne reledmac
\makeatletter
\renewenvironment{theindex}{%
  \ifkorrekturansicht
    \section*{\indexname}%
  \else
    \subsubsection*{Index der erwähnten Entitäten}%
  \fi
  \setlength{\parindent}{0pt}%
  \setlength{\parskip}{0pt plus 0.3pt}%
  \let\item\@idxitem
}{%
  \ifkorrekturansicht\clearpage\fi
}
\makeatother

\IfFileExists{\jobname-pw.ind}{\input{\jobname-pw.ind}}{}

% Quellenangabe nur in der Leseansicht
\ifkorrekturansicht\else
% Fallback-Definitionen, falls die .tex-Datei \titel etc. nicht gesetzt hat
\providecommand{\titel}{}
\providecommand{\editorInnen}{}
\providecommand{\dateiname}{\jobname}

\vspace{3cm}

\vfill

\footnotesize
\textsc{Quelle}: \titel. Herausgegeben von {\editorInnen}. In: \emph{Arthur Schnitzler: Briefwechsel mit Autorinnen und Autoren}.
 Digitale Edition, https://schnitzler-briefe.acdh.oeaw.ac.at/{\dateiname}.html (Stand \today)
\fi

\end{document}


