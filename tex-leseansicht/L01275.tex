%% latex-leseansicht-vorspann.tex
%% Vorspann für die Leseansicht.
%% Lädt die gemeinsame Datei latex-vorspann.tex mit nicht gesetztem Schalter.

\newif\ifkorrekturansicht
\korrekturansichtfalse

\input{../tex-inputs/latex-vorspann}


\section[Arthur Schnitzler an Gerhart Hauptmann, 8. 3. 1903]{L01275 Arthur Schnitzler an Gerhart Hauptmann, 8. 3. 1903}
\nopagebreak\mylabel{L01275v}
\rehead{ }\normalsize\beginnumbering\briefempfaengerindex{Hauptmann, Gerhart@\textsc{Hauptmann, Gerhart}!zzzSchnitzler, Arthur@\emph{von Arthur Schnitzler}!1903-03-081@{8. 3. 1903}|(be}
\toendnotes[C]{\smallbreak\pagebreak[2]}
\correspDesc{Versand  durch Arthur Schnitzler am 8. 3. 1903 in Berlin
\newline{}Erhalt  durch Gerhart Hauptmann am 8. 3. 1903 in Agnetendorf}\toendnotes[C]{\smallbreak}
\Standort{Staatsbibliothek Berlin – Preußischer Kulturbesitz, GHBrBl A:Schnitzler (8).}
\physDesc{Telegramm, 217 Zeichen
\newline{}HandschriftX2  : blauer Buntstift, deutsche Kurrent
\newline{}Versand: »\noindent{}\textcolor{gray}{\textbf{\textbf{Aufgenommen} von}}{ }\textcolor{gray}{Il}{ }\textcolor{gray}{\textbf{den}}{ }8\textcolor{gray}{\textbf{/}}3{ }\textcolor{gray}{\textbf{um}}{ }9 \textcolor{gray}{\textbf{Uhr}} 35\textcolor{gray}{\textbf{M}}{ }\textcolor{gray}{17}{ }\textcolor{gray}{\textbf{durch}}{ }Weichert\pwindex{Weichert *~8.\,3.\,1903@\textsc{Weichert} (*~8.\,3.\,1903), \emph{Telegrafenbeamter/Telegrafenbeamtin}|pw}.« 
\newline{}Ordnung: Lochung }\toendnotes[C]{\smallbreak}\pstart{}{\pb}Gerhart Hauptmann\pend{}\pstart{}Agnetendorf\oindex{Jagniątków@\textbf{Jagniątków}|pw}\pend{}{\bigskip}\vspace{1em}
\pstart
           
\pstart
           {\pb}\textcolor{gray}{\textbf{Telegramm aus}}{ }Berlin 9\oindex{Berlin@\textbf{Berlin}, \emph{Hauptstadt}|pw}\pend
           
\pstart
           \raggedleft{}24 \textcolor{gray}{\textbf{W.}}{ }\textcolor{gray}{\textbf{190}}3 \textcolor{gray}{\textbf{den}} 8\textcolor{gray}{\textbf{\textsuperscript{ten}}} 3{ }\textcolor{gray}{\textbf{um}}{ }9 \textcolor{gray}{\textbf{Uhr}} 3 \textcolor{gray}{\textbf{Min.}} m\pend
           \pend
           \vspace{0.5em}
\pstart
           Hätten Ihre lieben Wünſche{ }ſo viel Kraft gehabt als Sie mich erfreuten es\pwindex{Schnitzler, Arthur 15.\,5.\,1862 Wien – 21.\,10.\,1931 ebd.@\textsc{Schnitzler, Arthur} (15.\,5.\,1862 Wien – 21.\,10.\,1931 ebd.), \emph{Schriftsteller, Mediziner}!Schleier der Beatrice. Schauspiel in fünf Akten@\strich\emph{Der Schleier der Beatrice. Schauspiel in fünf Akten}|pwv} wäre ein großer Erfolg
               geworden. Ich grüße Sie in herzlicher Freundſchaft\pend
           \pstart \spacefill\mbox{Arthur Schnitzler}\pend{}\selectlanguage{ngerman}\endnumbering\briefempfaengerindex{Hauptmann, Gerhart@\textsc{Hauptmann, Gerhart}!zzzSchnitzler, Arthur@\emph{von Arthur Schnitzler}!1903-03-081@{8. 3. 1903}|)be}\mylabel{L01275h}  \newcommand{\dateiname}{L01275}\newcommand{\titel}{Arthur Schnitzler an Gerhart Hauptmann, 8. 3. 1903}\newcommand{\editorInnen}{Herausgegeben von Martin Anton Müller}%% latex-leseansicht-abspann.tex
%% Abspann für die Leseansicht.
%% Der Schalter \ifkorrekturansicht ist bereits durch den Vorspann gesetzt.

%% latex-abspann.tex
%% Gemeinsamer Abspann für Korrekturansicht und Leseansicht.
%% Setzt den Schalter \ifkorrekturansicht voraus (gesetzt in den
%% einbindenden Dateien latex-korrekturansicht-abspann.tex bzw.
%% latex-leseansicht-abspann.tex).
%% ---------------------------------------------------------------

\normalsize

% Das esempio-Environment wird nur in der Leseansicht benötigt
\ifkorrekturansicht\else
\newenvironment{esempio}[3]%
{
    \vspace{1.5ex}
    \rlap{\underline{#1}}
    \par
    \setlength{\parindent}{0cm}
    \nopagebreak
    \leftskip=#2cm
    \rightskip=#3cm
}
{
    \par
}
\fi

\doendnotes{C}
\bigskip
\vfill

\clearpage

\footnotesize

\ifkorrekturansicht
  \lohead{\textsc{register}}
\fi

% theindex-Environment neu definieren ohne reledmac
\makeatletter
\renewenvironment{theindex}{%
  \ifkorrekturansicht
    \section*{\indexname}%
  \else
    \subsubsection*{Index der erwähnten Entitäten}%
  \fi
  \setlength{\parindent}{0pt}%
  \setlength{\parskip}{0pt plus 0.3pt}%
  \let\item\@idxitem
}{%
  \ifkorrekturansicht\clearpage\fi
}
\makeatother

\IfFileExists{\jobname-pw.ind}{\input{\jobname-pw.ind}}{}

% Quellenangabe nur in der Leseansicht
\ifkorrekturansicht\else
% Fallback-Definitionen, falls die .tex-Datei \titel etc. nicht gesetzt hat
\providecommand{\titel}{}
\providecommand{\editorInnen}{}
\providecommand{\dateiname}{\jobname}

\vspace{3cm}

\vfill

\footnotesize
\textsc{Quelle}: \titel. Herausgegeben von {\editorInnen}. In: \emph{Arthur Schnitzler: Briefwechsel mit Autorinnen und Autoren}.
 Digitale Edition, https://schnitzler-briefe.acdh.oeaw.ac.at/{\dateiname}.html (Stand \today)
\fi

\end{document}


