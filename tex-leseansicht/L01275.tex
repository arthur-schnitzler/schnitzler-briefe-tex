%% latex-leseansicht-vorspann.tex
%% Vorspann für die Leseansicht.
%% Lädt die gemeinsame Datei latex-vorspann.tex mit nicht gesetztem Schalter.

\newif\ifkorrekturansicht
\korrekturansichtfalse

\input{../tex-inputs/latex-vorspann}


         
         \renewcommand{\erwaehntePersonen}{Personen: Gerhart Hauptmann,  Weichert}
         \renewcommand{\erwaehnteOrte}{Orte: Agnetendorf, Berlin}
         \renewcommand{\erwaehnteWerke}{Werke: Der Schleier der Beatrice. Schauspiel in fünf Akten}
               \section[Arthur Schnitzler an Gerhart Hauptmann, 8. 3. 1903]{ Arthur Schnitzler an Gerhart Hauptmann, 8. 3. 1903}\nopagebreak\mylabel{v}\rehead{ }\begin{ledgroupsized}[t]{13cm}\normalsize\beginnumbering \toendnotes[C]{\smallbreak\pagebreak[2]} \Standort{Staatsbibliothek Berlin – Preußischer Kulturbesitz, GHBrBl A:Schnitzler (8).}
\physDesc{Telegramm
\newline{}Handschrift  Weichert: blauer Buntstift, deutsche Kurrent\newline{}Versand: »\noindent{}\textcolor{gray}{\textbf{\textbf{Aufgenommen} von}}{ }\textcolor{gray}{Il}{ }\textcolor{gray}{\textbf{den}}{ }8\textcolor{gray}{\textbf{/}}3{ }\textcolor{gray}{\textbf{um}}{ }9 \textcolor{gray}{\textbf{Uhr}} 35\textcolor{gray}{\textbf{M}}{ }\textcolor{gray}{17}{ }\textcolor{gray}{\textbf{durch}}{ }Weichert\pwindex{Weichert *~8.3.1903@\textsc{Weichert} (*~8.3.1903), \emph{Telegrafenbeamter/Telegrafenbeamtin}|pw}.« \newline{}Ordnung: Lochung }\toendnotes[C]{\smallbreak}\pstart{}{\pb}Gerhart Hauptmann\pend{}\pstart{}Agnetendorf\oindex{Agnetendorf@\textbf{Agnetendorf}|pw}\pend{}{\bigskip}\pstart
           {\pb}\textcolor{gray}{\textbf{Telegramm aus}}{ }Berlin 9\oindex{Berlin@\textbf{Berlin}|pw}\hfill 24 \textcolor{gray}{\textbf{W.}}{ }\textcolor{gray}{\textbf{190}}3 \textcolor{gray}{\textbf{den}} 8\textcolor{gray}{\textbf{\textsuperscript{ten}}} 3{ }\textcolor{gray}{\textbf{um}} 9 \textcolor{gray}{\textbf{Uhr}} 3 \textcolor{gray}{\textbf{Min.}} m\pend
           \pstart
           Hätten Ihre lieben Wünſche ſo viel Kraft gehabt als Sie mich erfreuten es\pwindex{Schnitzler, Arthur 15.05.1862 – 21.10.1931@\textsc{Schnitzler, Arthur} (15.05.1862 – 21.10.1931), \emph{Schriftsteller, Mediziner}!Schleier der Beatrice. Schauspiel in fuenf Akten1900-12-01@\strich\emph{Der Schleier der Beatrice. Schauspiel in fünf Akten} {[}1900-12-01{]}|pwv} wäre ein großer Erfolg
               geworden. Ich grüße Sie in herzlicher Freundſchaft\pend
           \pstart \spacefill\mbox{Arthur Schnitzler}\pend{}
         
         \endnumbering\mylabel{h}\end{ledgroupsized}  \newcommand{\dateiname}{L01275}\newcommand{\titel}{Arthur Schnitzler an Gerhart Hauptmann, 8. 3. 1903}\newcommand{\editorInnen}{ Martin Anton Müller und Gerd-Hermann Susen}%% latex-leseansicht-abspann.tex
%% Abspann für die Leseansicht.
%% Der Schalter \ifkorrekturansicht ist bereits durch den Vorspann gesetzt.

%% latex-abspann.tex
%% Gemeinsamer Abspann für Korrekturansicht und Leseansicht.
%% Setzt den Schalter \ifkorrekturansicht voraus (gesetzt in den
%% einbindenden Dateien latex-korrekturansicht-abspann.tex bzw.
%% latex-leseansicht-abspann.tex).
%% ---------------------------------------------------------------

\normalsize

% Das esempio-Environment wird nur in der Leseansicht benötigt
\ifkorrekturansicht\else
\newenvironment{esempio}[3]%
{
    \vspace{1.5ex}
    \rlap{\underline{#1}}
    \par
    \setlength{\parindent}{0cm}
    \nopagebreak
    \leftskip=#2cm
    \rightskip=#3cm
}
{
    \par
}
\fi

\doendnotes{C}
\bigskip
\vfill

\clearpage

\footnotesize

\ifkorrekturansicht
  \lohead{\textsc{register}}
\fi

% theindex-Environment neu definieren ohne reledmac
\makeatletter
\renewenvironment{theindex}{%
  \ifkorrekturansicht
    \section*{\indexname}%
  \else
    \subsubsection*{Index der erwähnten Entitäten}%
  \fi
  \setlength{\parindent}{0pt}%
  \setlength{\parskip}{0pt plus 0.3pt}%
  \let\item\@idxitem
}{%
  \ifkorrekturansicht\clearpage\fi
}
\makeatother

\IfFileExists{\jobname-pw.ind}{\input{\jobname-pw.ind}}{}

% Quellenangabe nur in der Leseansicht
\ifkorrekturansicht\else
% Fallback-Definitionen, falls die .tex-Datei \titel etc. nicht gesetzt hat
\providecommand{\titel}{}
\providecommand{\editorInnen}{}
\providecommand{\dateiname}{\jobname}

\vspace{3cm}

\vfill

\footnotesize
\textsc{Quelle}: \titel. Herausgegeben von {\editorInnen}. In: \emph{Arthur Schnitzler: Briefwechsel mit Autorinnen und Autoren}.
 Digitale Edition, https://schnitzler-briefe.acdh.oeaw.ac.at/{\dateiname}.html (Stand \today)
\fi

\end{document}


      