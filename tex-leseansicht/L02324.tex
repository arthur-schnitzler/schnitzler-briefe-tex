%% latex-korrekturansicht-vorspann.tex
%% Vorspann für die Korrekturansicht.
%% Lädt die gemeinsame Datei latex-vorspann.tex mit gesetztem Schalter.

\newif\ifkorrekturansicht
\korrekturansichttrue

\input{../tex-inputs/latex-vorspann}


\section[Arthur Schnitzler an Robert Adam, 5. 8. 1919]{L02324 Arthur Schnitzler an Robert Adam, 5. 8. 1919}
\nopagebreak\mylabel{L02324v}
\rehead{ }\normalsize\beginnumbering\briefempfaengerindex{Adam, Robert@\textsc{Adam, Robert}!zzzSchnitzler, Arthur@\emph{von Arthur Schnitzler}!1919-08-051@{5. 8. 1919}|(be}
\toendnotes[C]{\smallbreak\pagebreak[2]}\Standort{DLA, 96.34.2/18.}
\physDesc{Postkarte, 619 Zeichen
\newline{}Handschrift: schwarze Tinte, deutsche Kurrent
\newline{}Versand: 1) zuerst nachgesandt nach Karlsbad\oindex{Karlsbad@\textbf{Karlsbad}, \emph{P.PPLA}|pw}, Beamtenkurhaus\oindex{Beamtenkurhaus zum Goldenden Kreuz@\textbf{Beamtenkurhaus zum Goldenden Kreuz}, \emph{Hotel (K.HTL)}|pw}, dann zurück nach Wien\oindex{Wien@\textbf{Wien}, \emph{A.ADM2}|pw} in die Meidlinger Hauptstraße 58\oindex{Meidlinger Hauptstrasse@\textbf{Meidlinger Hauptstraße}, \emph{Straße (K.STR)}|pw}  2) Stempel: »\nobreak{}\oindex{XVIII., Waehring@\textbf{XVIII., Währing}, \emph{A.ADM3}|pwk}18\textsubscript{1} Wien
                                       110, 5. VIII. 19, 7\nobreak{}«. }\toendnotes[C]{\smallbreak}\pstart{}{\pb}A. S. Wien XVIII, \textsc{Sternwartestr} 71\oindex{Sternwartestrasse 71@\textbf{Sternwartestraße 71}, \emph{Wohngebäude (K.WHS)}|pw}\pend{}{\bigskip}\pstart{}Herrn \textsc{Dr. Robert Adam}\pend{}\pstart{}\textsc{Pollak}\pend{}\pstart{}Landes\textcolor{gray}{gerichtsrat}h\pend{}\pstart{}\textsc{Wien} XII\oindex{XII., Meidling@\textbf{XII., Meidling}, \emph{A.ADM3}|pw}.\pend{}\pstart{}\textsc{Meidlinger Hptstr} 52\oindex{Meidlinger Hauptstrasse@\textbf{Meidlinger Hauptstraße}, \emph{Straße (K.STR)}|pw}. \pend{}{\bigskip}\vspace{1em}
\pstart
           \raggedleft{}{\pb}5. 8. 1919\pend
           \vspace{0.5em}
\pstart
           Verehrter Herr Doktor, vielen Dank für Ihre liebe Karte aus Karlsbad\oindex{Karlsbad@\textbf{Karlsbad}, \emph{P.PPLA}|pw}. Wie lange hab ich ſchon nichts von Ihnen
               gehört! Morgen fahr ich auf ein paar Tage oder Wochen (je nachdem ob ich mich dort
               wohl fühle) nach Reichenau\oindex{Reichenau an der Rax@\textbf{Reichenau an der Rax}, \emph{A.ADM3}|pw}, wo ſich Frau\pwindex{Schnitzler, Olga 17.01.1882 – 13.01.1970@\textsc{Schnitzler, Olga} (17.01.1882 – 13.01.1970), \emph{Schauspieler/Schauspielerin, Sänger/Sängerin}|pwv} u Tochter\pwindex{Cappellini, Lili 13.09.1909 – 26.07.1928@\textsc{Cappellini, Lili} (13.09.1909 – 26.07.1928)|pwv} ſeit 14 Tagen befinden. Mein Sohn\pwindex{Schnitzler, Heinrich 09.08.1902 – 12.07.1982@\textsc{Schnitzler, Heinrich} (09.08.1902 – 12.07.1982), \emph{Regisseur/Regisseurin, Schauspieler/Schauspielerin}|pwv} begleitet mich. Bitte
               laſſen Sie michs wiſſen, ſobald Sie {\pb}wieder in Wien\oindex{Wien@\textbf{Wien}, \emph{A.ADM2}|pw} ſind. Haben Sie aus dem Volkstheater\oindex{Volkstheater@\textbf{Volkstheater}, \emph{Theater (K.THE)}|pw} was neues erfahren? Intereſſe iſt vorhanden,
               beſonders bei Roſenthal\pwindex{Rosenthal, Friedrich 20.07.1885 – 31.08.1942@\textsc{Rosenthal, Friedrich} (20.07.1885 – 31.08.1942), \emph{Regisseur/Regisseurin, Dramaturg/Dramaturgin}|pw}. Auf recht bald
               alſo.\pend
           
\pstart
           Herzlichſt grüßt Sie Ihr{\\[\baselineskip]}\spacefill\mbox{Arthur Schnitzler}\pend
           \leftskip=0em{}\selectlanguage{ngerman}\endnumbering\briefempfaengerindex{Adam, Robert@\textsc{Adam, Robert}!zzzSchnitzler, Arthur@\emph{von Arthur Schnitzler}!1919-08-051@{5. 8. 1919}|)be}\mylabel{L02324h}  \normalsize

\doendnotes{C}
\bigskip
\vfill

\clearpage

\footnotesize

\lohead{\textsc{register}}

% Definiere theindex-Environment komplett neu ohne reledmac
\makeatletter
\renewenvironment{theindex}{%
  \section*{\indexname}%
  \setlength{\parindent}{0pt}%
  \setlength{\parskip}{0pt plus 0.3pt}%
  \let\item\@idxitem
}{%
  \clearpage
}
\makeatother

\IfFileExists{\jobname-pw.ind}{\input{\jobname-pw.ind}}{}

\end{document}

      