%% latex-leseansicht-vorspann.tex
%% Vorspann für die Leseansicht.
%% Lädt die gemeinsame Datei latex-vorspann.tex mit nicht gesetztem Schalter.

\newif\ifkorrekturansicht
\korrekturansichtfalse

\input{../tex-inputs/latex-vorspann}


\section[Arthur Schnitzler an Robert Adam, 5. 8. 1919]{L02324 Arthur Schnitzler an Robert Adam, 5. 8. 1919}
\nopagebreak\mylabel{L02324v}
\rehead{ }\normalsize\beginnumbering\briefempfaengerindex{Adam, Robert@\textsc{Adam, Robert}!zzzSchnitzler, Arthur@\emph{von Arthur Schnitzler}!1919-08-051@{5. 8. 1919}|(be}
\toendnotes[C]{\smallbreak\pagebreak[2]}
\correspDesc{Versand  durch Arthur Schnitzler am 5. 8. 1919 in Wien
\newline{}Weiterleitung  am 5. 8. 1919 in Karlsbad
\newline{}Erhalt  durch Robert Adam im Zeitraum [6. 8. 1919
                  – 8. 8. 1919?] in Wien}\toendnotes[C]{\smallbreak}
\Standort{DLA, 96.34.2/18.}
\physDesc{Postkarte, 619 Zeichen
\newline{}Handschrift: schwarze Tinte, deutsche Kurrent
\newline{}Versand: 1) zuerst nachgesandt nach Karlsbad\oindex{Karlsbad@\textbf{Karlsbad}|pw}, Beamtenkurhaus\oindex{Beamtenkurhaus zum Goldenden Kreuz@\textbf{Beamtenkurhaus zum Goldenden Kreuz}, \emph{Hotel}|pw}, dann zurück nach Wien\oindex{Wien@\textbf{Wien}, \emph{Verwaltungsgebiet}|pw} in die Meidlinger Hauptstraße 58\oindex{Wien@\textbf{Wien}!XII., Meidling@\textbf{XII., Meidling}!Meidlinger Hauptstraße@\textbf{Meidlinger Hauptstraße}, \emph{Straße}|pw}  2) Stempel: »\nobreak{}\oindex{XVIII., Währing@\textbf{XVIII., Währing}, \emph{Verwaltungsgebiet}|pwk}18\textsubscript{1} Wien
                                       110, 5. VIII. 19, 7\nobreak{}«. }\toendnotes[C]{\smallbreak}\pstart{}{\pb}A. S. Wien XVIII, \textsc{Sternwartestr} 71\oindex{Wien@\textbf{Wien}!XVIII., Währing@\textbf{XVIII., Währing}!Sternwartestraße 71@\textbf{Sternwartestraße 71}, \emph{Wohngebäude}|pw}\pend{}{\bigskip}\pstart{}Herrn \textsc{Dr. Robert Adam}\pend{}\pstart{}\textsc{Pollak}\pend{}\pstart{}Landes\textcolor{gray}{gerichtsrat}h\pend{}\pstart{}\textsc{Wien} XII\oindex{XII., Meidling@\textbf{XII., Meidling}, \emph{Verwaltungsgebiet}|pw}.\pend{}\pstart{}\textsc{Meidlinger Hptstr} 52\oindex{Wien@\textbf{Wien}!XII., Meidling@\textbf{XII., Meidling}!Meidlinger Hauptstraße@\textbf{Meidlinger Hauptstraße}, \emph{Straße}|pw}. \pend{}{\bigskip}\vspace{1em}
\pstart
           \raggedleft{}{\pb}5. 8. 1919\pend
           \vspace{0.5em}
\pstart
           Verehrter Herr Doktor, vielen Dank für Ihre liebe Karte aus Karlsbad\oindex{Karlsbad@\textbf{Karlsbad}|pw}. Wie lange hab ich{ }ſchon nichts von Ihnen
               gehört! Morgen fahr ich auf ein paar Tage oder Wochen (je nachdem ob ich mich dort
               wohl fühle) nach Reichenau\oindex{Reichenau an der Rax@\textbf{Reichenau an der Rax}, \emph{Verwaltungsgebiet}|pw}, wo{ }ſich Frau\pwindex{Schnitzler, Olga 17.\,1.\,1882 Wien – 13.\,1.\,1970 Lugano@\textsc{Schnitzler, Olga} (17.\,1.\,1882 Wien – 13.\,1.\,1970 Lugano), \emph{Schauspielerin, Sängerin}|pwv} u Tochter\pwindex{Cappellini, Lili 13.\,9.\,1909 Wien – 26.\,7.\,1928 Venedig@\textsc{Cappellini, Lili} (13.\,9.\,1909 Wien – 26.\,7.\,1928 Venedig)|pwv}{ }ſeit 14 Tagen befinden. Mein Sohn\pwindex{Schnitzler, Heinrich 9.\,8.\,1902 Hinterbrühl – 12.\,7.\,1982 Wien@\textsc{Schnitzler, Heinrich} (9.\,8.\,1902 Hinterbrühl – 12.\,7.\,1982 Wien), \emph{Regisseur, Schauspieler}|pwv} begleitet mich. Bitte
               laſſen Sie michs wiſſen,{ }ſobald Sie {\pb}wieder in Wien\oindex{Wien@\textbf{Wien}, \emph{Verwaltungsgebiet}|pw}{ }ſind. Haben Sie aus dem Volkstheater\oindex{Wien@\textbf{Wien}!VII., Neubau@\textbf{VII., Neubau}!Volkstheater@\textbf{Volkstheater}, \emph{Theater}|pw} was neues erfahren? Intereſſe iſt vorhanden,
               beſonders bei Roſenthal\pwindex{Rosenthal, Friedrich 20.\,7.\,1885 Wien – 31.\,8.\,1942 Konzentrationslager Auschwitz-Birkenau@\textsc{Rosenthal, Friedrich} (20.\,7.\,1885 Wien – 31.\,8.\,1942 Konzentrationslager Auschwitz-Birkenau), \emph{Regisseur, Dramaturg}|pw}. Auf recht bald
               alſo.\pend
           
\pstart
           Herzlichſt grüßt Sie Ihr{\\[\baselineskip]}\spacefill\mbox{Arthur Schnitzler}\pend
           \leftskip=0em{}\selectlanguage{ngerman}\endnumbering\briefempfaengerindex{Adam, Robert@\textsc{Adam, Robert}!zzzSchnitzler, Arthur@\emph{von Arthur Schnitzler}!1919-08-051@{5. 8. 1919}|)be}\mylabel{L02324h}  \newcommand{\dateiname}{L02324}\newcommand{\titel}{Arthur Schnitzler an Robert Adam, 5. 8. 1919}\newcommand{\editorInnen}{Martin Anton Müller und Gerd-Hermann Susen}%% latex-leseansicht-abspann.tex
%% Abspann für die Leseansicht.
%% Der Schalter \ifkorrekturansicht ist bereits durch den Vorspann gesetzt.

%% latex-abspann.tex
%% Gemeinsamer Abspann für Korrekturansicht und Leseansicht.
%% Setzt den Schalter \ifkorrekturansicht voraus (gesetzt in den
%% einbindenden Dateien latex-korrekturansicht-abspann.tex bzw.
%% latex-leseansicht-abspann.tex).
%% ---------------------------------------------------------------

\normalsize

% Das esempio-Environment wird nur in der Leseansicht benötigt
\ifkorrekturansicht\else
\newenvironment{esempio}[3]%
{
    \vspace{1.5ex}
    \rlap{\underline{#1}}
    \par
    \setlength{\parindent}{0cm}
    \nopagebreak
    \leftskip=#2cm
    \rightskip=#3cm
}
{
    \par
}
\fi

\doendnotes{C}
\bigskip
\vfill

\clearpage

\footnotesize

\ifkorrekturansicht
  \lohead{\textsc{register}}
\fi

% theindex-Environment neu definieren ohne reledmac
\makeatletter
\renewenvironment{theindex}{%
  \ifkorrekturansicht
    \section*{\indexname}%
  \else
    \subsubsection*{Index der erwähnten Entitäten}%
  \fi
  \setlength{\parindent}{0pt}%
  \setlength{\parskip}{0pt plus 0.3pt}%
  \let\item\@idxitem
}{%
  \ifkorrekturansicht\clearpage\fi
}
\makeatother

\IfFileExists{\jobname-pw.ind}{\input{\jobname-pw.ind}}{}

% Quellenangabe nur in der Leseansicht
\ifkorrekturansicht\else
% Fallback-Definitionen, falls die .tex-Datei \titel etc. nicht gesetzt hat
\providecommand{\titel}{}
\providecommand{\editorInnen}{}
\providecommand{\dateiname}{\jobname}

\vspace{3cm}

\vfill

\footnotesize
\textsc{Quelle}: \titel. Herausgegeben von {\editorInnen}. In: \emph{Arthur Schnitzler: Briefwechsel mit Autorinnen und Autoren}.
 Digitale Edition, https://schnitzler-briefe.acdh.oeaw.ac.at/{\dateiname}.html (Stand \today)
\fi

\end{document}


