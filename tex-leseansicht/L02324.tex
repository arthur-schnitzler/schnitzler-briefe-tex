%% latex-leseansicht-vorspann.tex
%% Vorspann für die Leseansicht.
%% Lädt die gemeinsame Datei latex-vorspann.tex mit nicht gesetztem Schalter.

\newif\ifkorrekturansicht
\korrekturansichtfalse

\input{../tex-inputs/latex-vorspann}


         
         \newcommand{\erwaehntePersonen}{Personen: Robert Adam, Friedrich Rosenthal, Olga Schnitzler, Lili Schnitzler, Heinrich Schnitzler}
         \newcommand{\erwaehnteInstitutionen}{}
         \newcommand{\erwaehnteOrte}{Orte: Beamtenkurhaus zum Goldenden Kreuz, Karlsbad, Meidlinger Hauptstraße, Reichenau an der Rax, Sternwartestraße, Volkstheater, Wien, XII., Meidling, XVIII., Währing}
         \newcommand{\erwaehnteWerke}{
               \section[Arthur Schnitzler an Robert Adam, 5. 8. 1919]{ Arthur Schnitzler an Robert Adam, 5. 8. 1919}\nopagebreak\mylabel{v}\rehead{ }\begin{ledgroupsized}[t]{13cm}\normalsize\beginnumbering \toendnotes[C]{\smallbreak\pagebreak[2]} \Standort{DLA, 96.34.2/18.}
\physDesc{Postkarte
\newline{}Handschrift: schwarze Tinte, deutsche Kurrent\newline{}Versand: 1) zuerst nachgesandt nach Karlsbad\oindex{Karlsbad@\textbf{Karlsbad}|pw}, Beamtenkurhaus\oindex{Beamtenkurhaus zum Goldenden Kreuz@\textbf{Beamtenkurhaus zum Goldenden Kreuz}|pw}, dann zurück nach Wien\oindex{Wien@\textbf{Wien}|pw} in die Meidlinger Hauptstraße 58\oindex{Meidlinger Hauptstrasse@\textbf{Meidlinger Hauptstraße}|pw}  2) Stempel: »\nobreak{}\oindex{XVIII., Waehring@\textbf{XVIII., Währing}|pwk}18\textsubscript{1} Wien
                                        110, 5. VIII. 19, 7\nobreak{}«. }\toendnotes[C]{\smallbreak}\pstart{}{\pb}A. S. Wien XVIII, \textsc{Sternwartestr} 71\oindex{Sternwartestrasse@\textbf{Sternwartestraße}|pw}\pend{}{\bigskip}\pstart{}Herrn \textsc{Dr. Robert Adam}\pend{}\pstart{}\textsc{Pollak}\pend{}\pstart{}Landes\textcolor{gray}{gerichtsrat}h\pend{}\pstart{}\textsc{Wien} XII\oindex{XII., Meidling@\textbf{XII., Meidling}|pw}.\pend{}\pstart{}\textsc{Meidlinger Hptstr} 52\oindex{Meidlinger Hauptstrasse@\textbf{Meidlinger Hauptstraße}|pw}. \pend{}{\bigskip}\pstart
           \raggedleft{}{\pb}5. 8. 1919\pend
           \pstart
           Verehrter Herr Doktor, vielen Dank für Ihre liebe Karte aus Karlsbad\oindex{Karlsbad@\textbf{Karlsbad}|pw}. Wie lange hab ich ſchon nichts von
                    Ihnen gehört! Morgen fahr ich auf ein paar Tage oder Wochen (je nachdem ob ich
                    mich dort wohl fühle) nach Reichenau\oindex{Reichenau an der Rax@\textbf{Reichenau an der Rax}|pw}, wo ſich
                        Frau\pwindex{Schnitzler, Olga 17.01.1882 – 13.01.1970@\textsc{Schnitzler, Olga} (17.01.1882 – 13.01.1970), \emph{Schauspielerin, Sängerin}|pwv} u Tochter\pwindex{Schnitzler, Lili 13.09.1909 – 26.07.1928@\textsc{Schnitzler, Lili} (13.09.1909 – 26.07.1928)|pwv} ſeit 14 Tagen
                    befinden. Mein Sohn\pwindex{Schnitzler, Heinrich 09.08.1902 – 12.07.1982@\textsc{Schnitzler, Heinrich} (09.08.1902 – 12.07.1982), \emph{Regisseur, Schauspieler}|pwv}
                    begleitet mich. Bitte laſſen Sie michs wiſſen, ſobald Sie {\pb}wieder in
                        Wien\oindex{Wien@\textbf{Wien}|pw} ſind. Haben Sie aus dem Volkstheater\oindex{Volkstheater@\textbf{Volkstheater}|pw} was neues erfahren? Intereſſe iſt vorhanden,
                    beſonders bei Roſenthal\pwindex{Rosenthal, Friedrich 20.07.1885 – 31.08.1942@\textsc{Rosenthal, Friedrich} (20.07.1885 – 31.08.1942), \emph{Regisseur, Dramaturg}|pw}. Auf recht bald
                    alſo.\pend
           \pstart
           Herzlichſt grüßt Sie Ihr{\\[\baselineskip]}\spacefill\mbox{Arthur Schnitzler}\pend
           \leftskip=0em{}
         
         \endnumbering\mylabel{h}\end{ledgroupsized}  \newcommand{\dateiname}{L02324}\newcommand{\titel}{Arthur Schnitzler an Robert Adam, 5. 8. 1919}\newcommand{\editorInnen}{Martin Anton Müller und Gerd-Hermann Susen}%% latex-leseansicht-abspann.tex
%% Abspann für die Leseansicht.
%% Der Schalter \ifkorrekturansicht ist bereits durch den Vorspann gesetzt.

%% latex-abspann.tex
%% Gemeinsamer Abspann für Korrekturansicht und Leseansicht.
%% Setzt den Schalter \ifkorrekturansicht voraus (gesetzt in den
%% einbindenden Dateien latex-korrekturansicht-abspann.tex bzw.
%% latex-leseansicht-abspann.tex).
%% ---------------------------------------------------------------

\normalsize

% Das esempio-Environment wird nur in der Leseansicht benötigt
\ifkorrekturansicht\else
\newenvironment{esempio}[3]%
{
    \vspace{1.5ex}
    \rlap{\underline{#1}}
    \par
    \setlength{\parindent}{0cm}
    \nopagebreak
    \leftskip=#2cm
    \rightskip=#3cm
}
{
    \par
}
\fi

\doendnotes{C}
\bigskip
\vfill

\clearpage

\footnotesize

\ifkorrekturansicht
  \lohead{\textsc{register}}
\fi

% theindex-Environment neu definieren ohne reledmac
\makeatletter
\renewenvironment{theindex}{%
  \ifkorrekturansicht
    \section*{\indexname}%
  \else
    \subsubsection*{Index der erwähnten Entitäten}%
  \fi
  \setlength{\parindent}{0pt}%
  \setlength{\parskip}{0pt plus 0.3pt}%
  \let\item\@idxitem
}{%
  \ifkorrekturansicht\clearpage\fi
}
\makeatother

\IfFileExists{\jobname-pw.ind}{\input{\jobname-pw.ind}}{}

% Quellenangabe nur in der Leseansicht
\ifkorrekturansicht\else
% Fallback-Definitionen, falls die .tex-Datei \titel etc. nicht gesetzt hat
\providecommand{\titel}{}
\providecommand{\editorInnen}{}
\providecommand{\dateiname}{\jobname}

\vspace{3cm}

\vfill

\footnotesize
\textsc{Quelle}: \titel. Herausgegeben von {\editorInnen}. In: \emph{Arthur Schnitzler: Briefwechsel mit Autorinnen und Autoren}.
 Digitale Edition, https://schnitzler-briefe.acdh.oeaw.ac.at/{\dateiname}.html (Stand \today)
\fi

\end{document}


      