%% latex-korrekturansicht-vorspann.tex
%% Vorspann für die Korrekturansicht.
%% Lädt die gemeinsame Datei latex-vorspann.tex mit gesetztem Schalter.

\newif\ifkorrekturansicht
\korrekturansichttrue

\input{../tex-inputs/latex-vorspann}


\section[ Felix Salten an Arthur Schnitzler, 26. 8. 1909]{L03507 Felix Salten an Arthur Schnitzler, 26. 8. 1909}
\nopagebreak\mylabel{L03507v}
\rehead{ }\normalsize\beginnumbering\briefempfaengerindex{Schnitzler, Arthur@\textsc{Schnitzler, Arthur}!zzzSalten, Felix@\emph{von Felix Salten}!1909-08-261@{26. 8. 1909}|(be}
\toendnotes[C]{\smallbreak\pagebreak[2]}\Standort{CUL, Schnitzler, B 89, B 1.}
\physDesc{Bildpostkarte, 250 Zeichen
\newline{}Handschrift: schwarze Tinte, lateinische Kurrent
\newline{}Versand: Stempel: »\nobreak{}\oindex{Innsbruck@\textbf{Innsbruck}, \emph{A.ADM2}|pwk}Inns\textcolor{gray}{b}{[}ruck{]} 3, 26. VIII. 09, 10\nobreak{}«.  
\newline{}Schnitzler: mit Bleistift Vermerk: »\textsc{Salten}« 
\newline{}Ordnung: mit Bleistift von unbekannter Hand nummeriert: »257« }\toendnotes[C]{\smallbreak}\pstart{}{\pb}Herrn D\textsuperscript{r} Arthur Schnitzler\pend{}\pstart{}München\oindex{Muenchen@\textbf{München}, \emph{P.PPLA}|pw}\pend{}\pstart{}Hotel vier Jahreszeiten\oindex{Hotel Vier Jahreszeiten@\textbf{Hotel Vier Jahreszeiten}, \emph{Hotel (K.HTL)}|pw}\pend{}{\bigskip}
\pstart
           \noindent{}\centering{}{\pb}\textcolor{gray}{\textbf{Innsbruck. Maria Theresienstr.\oindex{Maria-Theresien-Strasse [Innsbruck]@\textbf{Maria-Theresien-Straße [Innsbruck]}, \emph{Straße (K.STR)}|pw}}}\pend
           \vspace{1em}
\pstart
           \noindent{}{\pb}Ich bin Montag, Dienstag in Bregenz\oindex{Bregenz@\textbf{Bregenz}, \emph{P.PPLA}|pw}, Hotel Europe\oindex{Hotel de l Europe [Bregenz]@\textbf{Hotel de l’Europe [Bregenz]}, \emph{Hotel (K.HTL)}|pw}.
               Habe gehört, dass Sie beabsichtigen, \label{K_L03507-1v}\edtext{auch hinzukommen}{\lemma{\textnormal{\emph{auch hinzukommen}}}\Cendnote{\textnormal{Die vorangehende Karte mit vergleichbarem Inhalt dürfte 
                  Schnitzler  nicht mehr vor seiner Abreise erreicht haben, vgl. Felix Salten an Arthur Schnitzler, 23. 8. 1909. Sie trafen sich weder in  Bregenz\oindex{Bregenz@\textbf{Bregenz}, \emph{P.PPLA}|pwk} noch in Salzburg\oindex{Salzburg@\textbf{Salzburg}, \emph{A.ADM2}|pwk}.
               }}}\label{K_L03507-1}. Jedenfalls bitte ich um Nachricht, wann Sie in Salzburg\oindex{Salzburg@\textbf{Salzburg}, \emph{A.ADM2}|pw} sind.\pend
           
\pstart
           herzlichst {\\[\baselineskip]}Ihr {\\[\baselineskip]}\spacefill\mbox{Salten}\pend
           \leftskip=0em{}
\pstart
           Innsbruck\oindex{Innsbruck@\textbf{Innsbruck}, \emph{A.ADM2}|pw}, 26/8 09\pend
           \selectlanguage{ngerman}\endnumbering\briefempfaengerindex{Schnitzler, Arthur@\textsc{Schnitzler, Arthur}!zzzSalten, Felix@\emph{von Felix Salten}!1909-08-261@{26. 8. 1909}|)be}\mylabel{L03507h}  \normalsize

\doendnotes{C}
\bigskip
\vfill

\clearpage

\footnotesize

\lohead{\textsc{register}}

% Definiere theindex-Environment komplett neu ohne reledmac
\makeatletter
\renewenvironment{theindex}{%
  \section*{\indexname}%
  \setlength{\parindent}{0pt}%
  \setlength{\parskip}{0pt plus 0.3pt}%
  \let\item\@idxitem
}{%
  \clearpage
}
\makeatother

\IfFileExists{\jobname-pw.ind}{\input{\jobname-pw.ind}}{}

\end{document}

      