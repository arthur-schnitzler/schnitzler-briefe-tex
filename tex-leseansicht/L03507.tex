%% latex-leseansicht-vorspann.tex
%% Vorspann für die Leseansicht.
%% Lädt die gemeinsame Datei latex-vorspann.tex mit nicht gesetztem Schalter.

\newif\ifkorrekturansicht
\korrekturansichtfalse

\input{../tex-inputs/latex-vorspann}


         
         \renewcommand{\erwaehntePersonen}{Personen: Felix Salten}
         \renewcommand{\erwaehnteOrte}{Orte: Bregenz, Hotel Vier Jahreszeiten, Hotel de l’Europe, Innsbruck, Maria-Theresien-Straße, München, Salzburg}
         \renewcommand{\erwaehnteWerke}{}
               \section[ Felix Salten an Arthur Schnitzler, 26. 8. 1909]{ Felix Salten an Arthur Schnitzler, 26. 8. 1909}\nopagebreak\mylabel{v}\rehead{ }\begin{ledgroupsized}[t]{13cm}\normalsize\beginnumbering\briefempfaengerindex{Schnitzler, Arthur@\textsc{Schnitzler, Arthur}!zzzSalten, Felix@\emph{von Felix Salten}!1909-08-261@{26. 8. 1909}|(be} \toendnotes[C]{\smallbreak\pagebreak[2]} \Standort{CUL, Schnitzler, B 89, B 1.}
\physDesc{Bildpostkarte, 250 Zeichen
\newline{}Handschrift: schwarze Tinte, lateinische Kurrent
\newline{}Versand: Stempel: »\nobreak{}\oindex{Innsbruck@\textbf{Innsbruck}|pwk}Inns\textcolor{gray}{b}{[}ruck{]} 3, 26. VIII. 09, 10\nobreak{}«.  
\newline{}Schnitzler: mit Bleistift Vermerk: »\textsc{Salten}« 
\newline{}Ordnung: mit Bleistift von unbekannter Hand nummeriert: »257« }\toendnotes[C]{\smallbreak}\pstart{}{\pb}Herrn D\textsuperscript{r} Arthur Schnitzler\pend{}\pstart{}München\oindex{Muenchen@\textbf{München}|pw}\pend{}\pstart{}Hotel vier Jahreszeiten\oindex{Hotel Vier Jahreszeiten@\textbf{Hotel Vier Jahreszeiten}|pw}\pend{}{\bigskip}\pstart
           \noindent{}\centering{}{\pb}\textcolor{gray}{\textbf{Innsbruck\oindex{Innsbruck@\textbf{Innsbruck}|pw}. Maria Theresienstr.\oindex{Maria-Theresien-Strasse@\textbf{Maria-Theresien-Straße}|pw}}}\pend
           \pstart
           {\pb}Ich bin Montag, Dienstag in Bregenz\oindex{Bregenz@\textbf{Bregenz}|pw}, Hotel Europe\oindex{Hotel de l Europe@\textbf{Hotel de l’Europe}|pw}.
               Habe gehört, dass Sie beabsichtigen, \label{K_L03507-1v}\edtext{auch hinzukommen}{\lemma{\textnormal{\emph{auch hinzukommen}}}\Cendnote{\textnormal{Die vorangehende Karte mit vergleichbarem Inhalt dürfte 
                  Schnitzler\pwindex{Schnitzler, Arthur 15.05.1862 – 21.10.1931@\textsc{Schnitzler, Arthur} (15.05.1862 – 21.10.1931), \emph{Schriftsteller, Mediziner}|pwk}  nicht mehr vor seiner Abreise erreicht haben, vgl. Felix Salten an Arthur Schnitzler, 23. 8. 1909. Sie trafen sich weder in  Bregenz\oindex{Bregenz@\textbf{Bregenz}|pwk} noch in Salzburg\oindex{Salzburg@\textbf{Salzburg}|pwk}.
               }}}\label{K_L03507-1h}. Jedenfalls bitte ich um Nachricht, wann Sie in Salzburg\oindex{Salzburg@\textbf{Salzburg}|pw} sind.\pend
           \pstart
           herzlichst {\\[\baselineskip]}Ihr {\\[\baselineskip]}\spacefill\mbox{Salten}\pend
           \leftskip=0em{}\pstart
           Innsbruck\oindex{Innsbruck@\textbf{Innsbruck}|pw}, 26/8 09\pend
           
         
         \endnumbering\mylabel{h}\end{ledgroupsized}  \newcommand{\dateiname}{L03507}\newcommand{\titel}{Felix Salten an Arthur Schnitzler, 26. 8. 1909}\newcommand{\editorInnen}{Martin Anton Müller und Laura Untner}%% latex-leseansicht-abspann.tex
%% Abspann für die Leseansicht.
%% Der Schalter \ifkorrekturansicht ist bereits durch den Vorspann gesetzt.

%% latex-abspann.tex
%% Gemeinsamer Abspann für Korrekturansicht und Leseansicht.
%% Setzt den Schalter \ifkorrekturansicht voraus (gesetzt in den
%% einbindenden Dateien latex-korrekturansicht-abspann.tex bzw.
%% latex-leseansicht-abspann.tex).
%% ---------------------------------------------------------------

\normalsize

% Das esempio-Environment wird nur in der Leseansicht benötigt
\ifkorrekturansicht\else
\newenvironment{esempio}[3]%
{
    \vspace{1.5ex}
    \rlap{\underline{#1}}
    \par
    \setlength{\parindent}{0cm}
    \nopagebreak
    \leftskip=#2cm
    \rightskip=#3cm
}
{
    \par
}
\fi

\doendnotes{C}
\bigskip
\vfill

\clearpage

\footnotesize

\ifkorrekturansicht
  \lohead{\textsc{register}}
\fi

% theindex-Environment neu definieren ohne reledmac
\makeatletter
\renewenvironment{theindex}{%
  \ifkorrekturansicht
    \section*{\indexname}%
  \else
    \subsubsection*{Index der erwähnten Entitäten}%
  \fi
  \setlength{\parindent}{0pt}%
  \setlength{\parskip}{0pt plus 0.3pt}%
  \let\item\@idxitem
}{%
  \ifkorrekturansicht\clearpage\fi
}
\makeatother

\IfFileExists{\jobname-pw.ind}{\input{\jobname-pw.ind}}{}

% Quellenangabe nur in der Leseansicht
\ifkorrekturansicht\else
% Fallback-Definitionen, falls die .tex-Datei \titel etc. nicht gesetzt hat
\providecommand{\titel}{}
\providecommand{\editorInnen}{}
\providecommand{\dateiname}{\jobname}

\vspace{3cm}

\vfill

\footnotesize
\textsc{Quelle}: \titel. Herausgegeben von {\editorInnen}. In: \emph{Arthur Schnitzler: Briefwechsel mit Autorinnen und Autoren}.
 Digitale Edition, https://schnitzler-briefe.acdh.oeaw.ac.at/{\dateiname}.html (Stand \today)
\fi

\end{document}


      