\input{../tex-inputs/latex-pdf-vorspann}
\begin{center}
            \textcolor{red}{ENTWURF. ENTZIFFERUNG NOCH NICHT KORREKTURGELESEN}
                      \end{center}
            
               \section[Arthur Schnitzler an Richard Beer-Hofmann, 19. 10. 1896]{ Arthur Schnitzler an Richard Beer-Hofmann, 19. 10. 1896}\nopagebreak\mylabel{v}\rehead{ }\begin{ledgroupsized}[t]{13cm}\normalsize\beginnumbering\briefempfaengerindex{Beer-Hofmann, Richard@\textsc{Beer-Hofmann, Richard}!zzzSchnitzler, Arthur@\emph{von Arthur Schnitzler}!1896-10-191@{19. 10. 1896}|(be} \toendnotes[C]{\smallbreak\pagebreak[2]} \Standort{YCGL, MSS 31.}
\physDesc{Postkarte
\newline{}Handschrift: Bleistift, deutsche Kurrent\newline{}Versand: 1) Stempel: »\nobreak{}\oindex{IX., Alsergrund@\textbf{IX., Alsergrund}|pwk}Wien 9/3, 19. 10. 96, 3–4 N\nobreak{}«.  2) Stempel: »\nobreak{}\oindex{I., Innere Stadt@\textbf{I., Innere Stadt}|pwk}Wien 1/\textcolor{gray}{1}, 19. 10. 96, 5–6½ N, Bestellt\nobreak{}«. }\pstart{}{\pb}Herrn \textsc{Dr. Rich.
                     Beer-Hofmann}\pend{}\pstart{}Wien\oindex{Wien@\textbf{Wien}|pw}\pend{}\pstart{}I. \textsc{Wollzeile 15}\oindex{Wollzeile@\textbf{Wollzeile}|pw}.\pend{}{\bigskip}\pstart
           \noindent{}Lieber Richard, Dinſtg bin ich natürlich wieder vor
                  ½ 10 zu Hauſe – ich ſchreibe auch den andern u hoffe dſs alle
                  ko{\geminationm}en\pend
           \pstart
           Herzlichſt{\\[\baselineskip]}\spacefill\mbox{A}\pend
           \leftskip=0em{}\endnumbering\briefempfaengerindex{Beer-Hofmann, Richard@\textsc{Beer-Hofmann, Richard}!zzzSchnitzler, Arthur@\emph{von Arthur Schnitzler}!1896-10-191@{19. 10. 1896}|)be}\mylabel{h}\end{ledgroupsized}  \newcommand{\dateiname}{L00607}\newcommand{\titel}{Arthur Schnitzler an Richard Beer-Hofmann, 19. 10. 1896}\newcommand{\editorInnen}{Martin Anton Müller und Gerd-Hermann Susen}\input{../tex-inputs/latex-pdf-abspann}
      