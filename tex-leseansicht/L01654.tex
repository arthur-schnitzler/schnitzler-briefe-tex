%% latex-leseansicht-vorspann.tex
%% Vorspann für die Leseansicht.
%% Lädt die gemeinsame Datei latex-vorspann.tex mit nicht gesetztem Schalter.

\newif\ifkorrekturansicht
\korrekturansichtfalse

\input{../tex-inputs/latex-vorspann}


\section[Max Burckhard an Arthur Schnitzler, 26. 1. 1907]{L01654 Max Burckhard an Arthur Schnitzler, 26. 1. 1907}
\nopagebreak\mylabel{L01654v}
\rehead{ }\normalsize\beginnumbering\briefempfaengerindex{Schnitzler, Arthur@\textsc{Schnitzler, Arthur}!zzzBurckhard, Max Eugen@\emph{von Max Eugen Burckhard}!1907-01-261@{26. 1. 1907}|(be}
\toendnotes[C]{\smallbreak\pagebreak[2]}
\correspDesc{Versand  durch Max Burckhard am 26. 1. 1907 in Wien
\newline{}Erhalt  durch Arthur Schnitzler im Zeitraum [26. 1. 1907
                  – 30. 1. 1907?] in Wien}\toendnotes[C]{\smallbreak}
\Standort{CUL, Schnitzler, B 20.}
\physDesc{Brief, 1 Blatt, 1 Seite, 502 Zeichen
\newline{}Handschrift: schwarze Tinte, deutsche Kurrent
\newline{}Ordnung: mit Bleistift von unbekannter Hand nummeriert:
                                    »17« }\toendnotes[C]{\smallbreak}
\pstart
           {\pb}\textcolor{gray}{\textbf{D\textsuperscript{r.} Max Burckhard}}\hfill \textcolor{gray}{\textbf{Wien, IX. Porzellangasse 48\oindex{Wien@\textbf{Wien}!IX., Alsergrund@\textbf{IX., Alsergrund}!Porzellangasse@\textbf{Porzellangasse}, \emph{Straße}|pw}}}{ }26. 1. 07.\pend
           
\pstart
           \raggedleft{}\textcolor{gray}{\textbf{\strikeout{St. Gilgen}}}\hspace*{3.5em}\pend
           
\pstart{}Sehr verehrter lieber Herr Doctor!\pend\vspace{0.5em}
\pstart
           Ich danke Ihnen und der hochverehrten lieben gnädigen Frau\pwindex{Schnitzler, Olga 17.\,1.\,1882 Wien – 13.\,1.\,1970 Lugano@\textsc{Schnitzler, Olga} (17.\,1.\,1882 Wien – 13.\,1.\,1970 Lugano), \emph{Schauspielerin, Sängerin}|pwv} herzlichſt für die Einladung. Ich muß
               leider{ }ſchon heute Abend abfahren, da mir das Wetter hier nicht taugt, u. ich war
               darum in{ }ſolcher Hetze, daſs es mir nicht einmal möglich war, obwohl ich es{ }ſo gerne
               gethan hätte, unter Tags einmal in der Spöttelgaſſe\oindex{Wien@\textbf{Wien}!XVIII., Währing@\textbf{XVIII., Währing}!Edmund-Weiß-Gasse 7@\textbf{Edmund-Weiß-Gasse 7}, \emph{Wohngebäude}|pw} vorzuſprechen. So{ }ſage ich denn auf Wiederſehen im Frühjahr!
               Herzlichſt und mit Handkuß an die verehrte gnädige Frau\pwindex{Schnitzler, Olga 17.\,1.\,1882 Wien – 13.\,1.\,1970 Lugano@\textsc{Schnitzler, Olga} (17.\,1.\,1882 Wien – 13.\,1.\,1970 Lugano), \emph{Schauspielerin, Sängerin}|pwv} Ihr treu ergebener\pend
           \pstart \spacefill\mbox{D\textsuperscript{r}Burckhard}\pend{}\selectlanguage{ngerman}\endnumbering\briefempfaengerindex{Schnitzler, Arthur@\textsc{Schnitzler, Arthur}!zzzBurckhard, Max Eugen@\emph{von Max Eugen Burckhard}!1907-01-261@{26. 1. 1907}|)be}\mylabel{L01654h}  \newcommand{\dateiname}{L01654}\newcommand{\titel}{Max Burckhard an Arthur Schnitzler, 26. 1. 1907}\newcommand{\editorInnen}{Martin Anton Müller und Gerd-Hermann Susen}%% latex-leseansicht-abspann.tex
%% Abspann für die Leseansicht.
%% Der Schalter \ifkorrekturansicht ist bereits durch den Vorspann gesetzt.

%% latex-abspann.tex
%% Gemeinsamer Abspann für Korrekturansicht und Leseansicht.
%% Setzt den Schalter \ifkorrekturansicht voraus (gesetzt in den
%% einbindenden Dateien latex-korrekturansicht-abspann.tex bzw.
%% latex-leseansicht-abspann.tex).
%% ---------------------------------------------------------------

\normalsize

% Das esempio-Environment wird nur in der Leseansicht benötigt
\ifkorrekturansicht\else
\newenvironment{esempio}[3]%
{
    \vspace{1.5ex}
    \rlap{\underline{#1}}
    \par
    \setlength{\parindent}{0cm}
    \nopagebreak
    \leftskip=#2cm
    \rightskip=#3cm
}
{
    \par
}
\fi

\doendnotes{C}
\bigskip
\vfill

\clearpage

\footnotesize

\ifkorrekturansicht
  \lohead{\textsc{register}}
\fi

% theindex-Environment neu definieren ohne reledmac
\makeatletter
\renewenvironment{theindex}{%
  \ifkorrekturansicht
    \section*{\indexname}%
  \else
    \subsubsection*{Index der erwähnten Entitäten}%
  \fi
  \setlength{\parindent}{0pt}%
  \setlength{\parskip}{0pt plus 0.3pt}%
  \let\item\@idxitem
}{%
  \ifkorrekturansicht\clearpage\fi
}
\makeatother

\IfFileExists{\jobname-pw.ind}{\input{\jobname-pw.ind}}{}

% Quellenangabe nur in der Leseansicht
\ifkorrekturansicht\else
% Fallback-Definitionen, falls die .tex-Datei \titel etc. nicht gesetzt hat
\providecommand{\titel}{}
\providecommand{\editorInnen}{}
\providecommand{\dateiname}{\jobname}

\vspace{3cm}

\vfill

\footnotesize
\textsc{Quelle}: \titel. Herausgegeben von {\editorInnen}. In: \emph{Arthur Schnitzler: Briefwechsel mit Autorinnen und Autoren}.
 Digitale Edition, https://schnitzler-briefe.acdh.oeaw.ac.at/{\dateiname}.html (Stand \today)
\fi

\end{document}


