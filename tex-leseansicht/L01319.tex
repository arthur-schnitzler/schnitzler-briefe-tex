%% latex-korrekturansicht-vorspann.tex
%% Vorspann für die Korrekturansicht.
%% Lädt die gemeinsame Datei latex-vorspann.tex mit gesetztem Schalter.

\newif\ifkorrekturansicht
\korrekturansichttrue

\input{../tex-inputs/latex-vorspann}


\section[Hugo von Hofmannsthal an Arthur Schnitzler, 27. 9. {[}1903{]}]{L01319 Hugo von Hofmannsthal an Arthur Schnitzler, 27. 9. {[}1903{]}}
\nopagebreak\mylabel{L01319v}
\rehead{ }\normalsize\beginnumbering\briefempfaengerindex{Schnitzler, Arthur@\textsc{Schnitzler, Arthur}!zzzHofmannsthal, Hugo von@\emph{von Hugo von Hofmannsthal}!1903-09-271@{27. 9. {[}1903{]}}|(be}
\toendnotes[C]{\smallbreak\pagebreak[2]}\Standort{CUL, Schnitzler, B 43b/1.}
\physDesc{Brief, 1 Blatt, 3 Seiten, 811 Zeichen
\newline{}Handschrift: schwarze Tinte, deutsche Kurrent
\newline{}Schnitzler: mit Bleistift die Jahreszahl »903« ergänzt 
\newline{}Ordnung: 1) mit Bleistift von unbekannter Hand nummeriert: »\strikeout{215}«  2) mit Bleistift von unbekannter Hand nummeriert:
                                    »201«}
\buchAbdrucke{\weitereDrucke{1) Hugo von Hofmannsthal, Arthur Schnitzler: \emph{Briefwechsel}. Frankfurt am Main: \emph{S. Fischer} 1964, S. 174.} \weitereDrucke{2) Hermann Bahr, Arthur Schnitzler: \emph{Briefwechsel, Aufzeichnungen, Dokumente (1891–1931)}. Göttingen: \emph{Wallstein} 2018, S. 271.} }\toendnotes[C]{\smallbreak}
\pstart
           \raggedleft{}{\pb}27. IX.\pend
           \vspace{0.5em}
\pstart
           \label{OL444-1v}\label{OL444-1h}lieber, ich vergeſſe nun ſchon 6mal, \strikeout{daſs} Sie zu erinnern, daß Sie mir die \label{K_L01319-1v}\edtext{Photographie\pwindex{Arthur Schnitzler [Halbprofil 1903]@\emph{Arthur Schnitzler [Halbprofil 1903]}|pwv}}{\lemma{\textnormal{\emph{Photographie}}}\Cendnote{\textnormal{Siehe Arthur Schnitzler an Hermann Bahr, 19. 7. 1903.
               }}}\label{K_L01319-1} (die gleiche wie der Bahr\pwindex{Bahr, Hermann 19.07.1863 – 15.01.1934@\textsc{Bahr, Hermann} (19.07.1863 – 15.01.1934), \emph{Schriftsteller/Schriftstellerin, Kritiker/Kritikerin}|pw}{ }\label{T_L01319-1v}\edtext{hat)}{\lemma{\textnormal{\emph{hat)}}}\Cendnote{\textnormal{An dieser Stelle ein senkrechter Strich, mutmaßlich als
                  Erinnerung von Schnitzler gemacht, um sich an diese
                  Bitte zu erinnern.}}}\label{T_L01319-1} verſprochen haben.\hspace*{1.5em}Sehr ſchön war es \label{K_L01319-2v}\edtext{gestern}{\lemma{\textnormal{\emph{gestern}}}\Cendnote{\textnormal{Vgl. A. S.: \emph{Tagebuch}, 26. 9. 1903.
               }}}\label{K_L01319-2} abend, in den lieben Zimmern, das Kind\pwindex{Schnitzler, Heinrich 09.08.1902 – 12.07.1982@\textsc{Schnitzler, Heinrich} (09.08.1902 – 12.07.1982), \emph{Regisseur/Regisseurin, Schauspieler/Schauspielerin}|pwv}, das ſchöne Singen, und alles zuſammen. Sie können
               ſich vielleicht kaum vorſtellen, wie ſehr einem ein paar Lieder von einer ſchönen
               jungen Sti{\geminationm}e freuen, wenn {\pb}man immerfort das Gefühl hat, zu
               wenig Muſik zu hören, wie wir.\hspace*{1.5em}Aber ſpielen darf ſie
               nicht dabei, es geniert einen in der Erinnerung faſt noch mehr wie im Augenblick
               ſelbſt: und ſo wunderbar es iſt, die Reflexe eines Liedes auf der Stirn und in den
               Augen eines Singenden mehr noch zu fühlen als zu ſehen, ſo ſehr {\pb}verletzt es wirklich die
               Beſcheidenheit der Natur und der Kunſt zugleich, wenn man beim Singen agiert.\pend
           
\pstart
           Auf Wiederſehen \label{K_L01319-3v}\edtext{Samstag}{\lemma{\textnormal{\emph{Samstag}}}\Cendnote{\textnormal{Siehe A. S.: \emph{Tagebuch}, 3. 10. 1903.
               }}}\label{K_L01319-3}.\pend
           
\pstart
           Von Herzen{\\[\baselineskip]}\spacefill\mbox{Hugo.}\pend
           \leftskip=0em{}\selectlanguage{ngerman}\endnumbering\briefempfaengerindex{Schnitzler, Arthur@\textsc{Schnitzler, Arthur}!zzzHofmannsthal, Hugo von@\emph{von Hugo von Hofmannsthal}!1903-09-271@{27. 9. {[}1903{]}}|)be}\mylabel{L01319h}  \normalsize

\doendnotes{C}
\bigskip
\vfill

\clearpage

\footnotesize

\lohead{\textsc{register}}

% Definiere theindex-Environment komplett neu ohne reledmac
\makeatletter
\renewenvironment{theindex}{%
  \section*{\indexname}%
  \setlength{\parindent}{0pt}%
  \setlength{\parskip}{0pt plus 0.3pt}%
  \let\item\@idxitem
}{%
  \clearpage
}
\makeatother

\IfFileExists{\jobname-pw.ind}{\input{\jobname-pw.ind}}{}

\end{document}

      