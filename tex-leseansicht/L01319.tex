%% latex-leseansicht-vorspann.tex
%% Vorspann für die Leseansicht.
%% Lädt die gemeinsame Datei latex-vorspann.tex mit nicht gesetztem Schalter.

\newif\ifkorrekturansicht
\korrekturansichtfalse

\input{../tex-inputs/latex-vorspann}


               \section[Hugo von Hofmannsthal an Arthur Schnitzler, 27. 9. {[}1903{]}]{ Hugo von Hofmannsthal an Arthur Schnitzler, 27. 9. {[}1903{]}}\nopagebreak\mylabel{v}\rehead{ }\begin{ledgroupsized}[t]{13cm}\normalsize\beginnumbering\briefempfaengerindex{Schnitzler, Arthur@\textsc{Schnitzler, Arthur}!zzzHofmannsthal, Hugo von@\emph{von Hugo von Hofmannsthal}!1903-09-271@{27. 9. {[}1903{]}}|(be} \toendnotes[C]{\smallbreak\pagebreak[2]} \Standort{CUL, Schnitzler, B 43b/1.}
\physDesc{Brief, 1 Blatt, 3 Seiten
\newline{}Handschrift: schwarze Tinte, deutsche Kurrent
\newline{}Schnitzler: mit Bleistift die Jahreszahl »903« ergänzt \newline{}Ordnung: 1) mit Bleistift von unbekannter Hand nummeriert:
                                    »\strikeout{215}« 2) mit Bleistift von unbekannter Hand nummeriert:
                                    »201«}\buchAbdrucke{\weitereDrucke{1) Hugo von Hofmannsthal, Arthur Schnitzler: \emph{Briefwechsel}. Hg. Therese Nickl und Heinrich Schnitzler. Frankfurt am Main: \emph{S. Fischer} 1964, S. 174.} \weitereDrucke{2) Hermann Bahr, Arthur Schnitzler: \emph{Briefwechsel, Aufzeichnungen, Dokumente
                                (1891–1931)}. Hg. Kurt Ifkovits und Martin Anton Müller. Göttingen: \emph{Wallstein} 2018, S. 271.} }\toendnotes[C]{\smallbreak}\pstart
           \raggedleft{}{\pb}27. IX.\pend
           \pstart
           \label{OL444-1v}\label{OL444-1h}lieber, ich vergeſſe nun ſchon 6mal, \strikeout{daſs} Sie zu erinnern, daß Sie mir die Photographie\pwindex{\textcolor{red}{\textsuperscript{XXXX1 indx}}!Arthur Schnitzler1903@\strich\emph{Arthur Schnitzler} {[}1903{]}|pwv} (die gleiche wie der Bahr\pwindex{Bahr, Hermann 19.07.1863 – 15.01.1934@\textsc{Bahr, Hermann} (19.07.1863 – 15.01.1934), \emph{Schriftsteller, Kritiker}|pw}{ }\label{T_L01319_1v}\edtext{hat)}{\lemma{\textnormal{\emph{hat)}}}\Cendnote{\textnormal{An dieser
                        Stelle ein senkrechter Strich, mutmaßlich als Erinnerung von Schnitzler\pwindex{Schnitzler, Arthur 15.05.1862 – 21.10.1931@\textsc{Schnitzler, Arthur} (15.05.1862 – 21.10.1931), \emph{Schriftsteller, Mediziner}|pwk} gemacht, um das nicht zu
                        vergessen.}}}\label{T_L01319_1h} verſprochen haben.\hspace*{1.5em}Sehr
                    ſchön war es \label{K_L01319_1v}\edtext{gestern}{\lemma{\textnormal{\emph{gestern}}}\Cendnote{\textnormal{vgl. A. S.: \emph{Tagebuch}, 26. 9. 1903}}}\label{K_L01319_1h} abend, in den lieben
                    Zimmern, das Kind\pwindex{Schnitzler, Heinrich 09.08.1902 – 12.07.1982@\textsc{Schnitzler, Heinrich} (09.08.1902 – 12.07.1982), \emph{Regisseur, Schauspieler}|pwv}, das
                    ſchöne Singen, und alles zuſammen. Sie können ſich vielleicht kaum vorſtellen,
                    wie ſehr einem ein paar Lieder von einer ſchönen jungen Sti{\geminationm}e freuen, wenn {\pb}man immerfort das Gefühl
                    hat, zu wenig Muſik zu hören, wie wir.\hspace*{1.5em}Aber
                    ſpielen darf ſie nicht dabei, es geniert einen in der Erinnerung faſt noch mehr
                    wie im Augenblick ſelbſt: und ſo wunderbar es iſt, die Reflexe eines Liedes auf
                    der Stirn und in den Augen eines Singenden mehr noch zu fühlen als zu ſehen, ſo
                    ſehr {\pb}verletzt es wirklich
                    die Beſcheidenheit der Natur und der Kunſt zugleich, wenn man beim Singen
                    agiert.\pend
           \pstart
           Auf Wiederſehen \label{K_L01319_2v}\edtext{Samstag}{\lemma{\textnormal{\emph{Samstag}}}\Cendnote{\textnormal{siehe A. S.: \emph{Tagebuch}, 3. 10. 1903}}}\label{K_L01319_2h}.\pend
           \pstart
           Von Herzen{\\[\baselineskip]}\spacefill\mbox{Hugo.}\pend
           \leftskip=0em{}          \endnumbering\briefempfaengerindex{Schnitzler, Arthur@\textsc{Schnitzler, Arthur}!zzzHofmannsthal, Hugo von@\emph{von Hugo von Hofmannsthal}!1903-09-271@{27. 9. {[}1903{]}}|)be}\mylabel{h}\end{ledgroupsized}  \newcommand{\dateiname}{L01319}\newcommand{\titel}{Hugo von Hofmannsthal an Arthur Schnitzler, 27. 9. [1903]}\newcommand{\editorInnen}{ Martin Anton Müller und Gerd-Hermann Susen}
            \footnotesize
\begin{ledgroupsized}[t]{11.5cm}
\doendnotes{C}
\end{ledgroupsized}
         %% latex-leseansicht-abspann.tex
%% Abspann für die Leseansicht.
%% Der Schalter \ifkorrekturansicht ist bereits durch den Vorspann gesetzt.

%% latex-abspann.tex
%% Gemeinsamer Abspann für Korrekturansicht und Leseansicht.
%% Setzt den Schalter \ifkorrekturansicht voraus (gesetzt in den
%% einbindenden Dateien latex-korrekturansicht-abspann.tex bzw.
%% latex-leseansicht-abspann.tex).
%% ---------------------------------------------------------------

\normalsize

% Das esempio-Environment wird nur in der Leseansicht benötigt
\ifkorrekturansicht\else
\newenvironment{esempio}[3]%
{
    \vspace{1.5ex}
    \rlap{\underline{#1}}
    \par
    \setlength{\parindent}{0cm}
    \nopagebreak
    \leftskip=#2cm
    \rightskip=#3cm
}
{
    \par
}
\fi

\doendnotes{C}
\bigskip
\vfill

\clearpage

\footnotesize

\ifkorrekturansicht
  \lohead{\textsc{register}}
\fi

% theindex-Environment neu definieren ohne reledmac
\makeatletter
\renewenvironment{theindex}{%
  \ifkorrekturansicht
    \section*{\indexname}%
  \else
    \subsubsection*{Index der erwähnten Entitäten}%
  \fi
  \setlength{\parindent}{0pt}%
  \setlength{\parskip}{0pt plus 0.3pt}%
  \let\item\@idxitem
}{%
  \ifkorrekturansicht\clearpage\fi
}
\makeatother

\IfFileExists{\jobname-pw.ind}{\input{\jobname-pw.ind}}{}

% Quellenangabe nur in der Leseansicht
\ifkorrekturansicht\else
% Fallback-Definitionen, falls die .tex-Datei \titel etc. nicht gesetzt hat
\providecommand{\titel}{}
\providecommand{\editorInnen}{}
\providecommand{\dateiname}{\jobname}

\vspace{3cm}

\vfill

\footnotesize
\textsc{Quelle}: \titel. Herausgegeben von {\editorInnen}. In: \emph{Arthur Schnitzler: Briefwechsel mit Autorinnen und Autoren}.
 Digitale Edition, https://schnitzler-briefe.acdh.oeaw.ac.at/{\dateiname}.html (Stand \today)
\fi

\end{document}


      