%% latex-leseansicht-vorspann.tex
%% Vorspann für die Leseansicht.
%% Lädt die gemeinsame Datei latex-vorspann.tex mit nicht gesetztem Schalter.

\newif\ifkorrekturansicht
\korrekturansichtfalse

\input{../tex-inputs/latex-vorspann}


         
         \newcommand{\erwaehntePersonen}{Personen: Hermann Bahr, Max Eugen Burckhard, Karl Lamprecht, Felix Salten, Olga Schnitzler}
         \newcommand{\erwaehnteInstitutionen}{Institutionen: Berliner Tageblatt, Rudolf Gaertners Verlagsbuchhandlung, Vossische Zeitung}
         \newcommand{\erwaehnteOrte}{Orte: Bösendorfer-Saal, Ober Sankt Veit, Semmering, Wien, Österreich}
         \newcommand{\erwaehnteWerke}{Werke: Anatol, Arthur Schnitzler und sein »Reigen«, Deutsche Geschichte. Erster Ergänzungsband. Zur jüngsten deutschen Vergangenheit, Die Zeit, Die verbotene »Reigen«-Vorlesung, Excentric, Liebelei. Schauspiel in drei Akten, Reigen. Zehn Dialoge}
               \section[Arthur Schnitzler an Hermann Bahr, 10. 11. 1903]{ Arthur Schnitzler an Hermann Bahr, 10. 11. 1903}\nopagebreak\mylabel{v}\rehead{ }\begin{ledgroupsized}[t]{13cm}\normalsize\beginnumbering \toendnotes[C]{\smallbreak\pagebreak[2]} \Standort{TMW, HS AM 23359 Ba.}
\physDesc{Brief, 2 Blätter, 6 Seiten
\newline{}Handschrift: schwarze Tinte, deutsche Kurrent\newline{}Ordnung: 1) Lochung  2) mit Bleistift von unbekannter Hand das zweite Blatt datiert »10. 11. 03« und mit »II« versehen}\buchAbdrucke{\weitereDrucke{1) Arthur Schnitzler: \emph{Briefe 1875–1912}. Hg. Therese Nickl und Heinrich Schnitzler. Frankfurt am Main: \emph{S. Fischer} 1981, S. 473–474.} \weitereDrucke{2) \emph{10. 11. 1903.} In: Arthur Schnitzler: \emph{The Letters of Arthur Schnitzler to Hermann Bahr}. Edited, annotated, and with an introduction, by Donald G.
                        Daviau. Chapel Hill: \emph{The University of North Carolina Press} 1978, S. 80–81 (University of North Carolina studies in the Germanic languages
                        and literatures, 89).} \weitereDrucke{3) Hermann Bahr, Arthur Schnitzler: \emph{Briefwechsel, Aufzeichnungen, Dokumente (1891–1931)}. Hg. Kurt Ifkovits und Martin Anton Müller. Göttingen: \emph{Wallstein} 2018, S. 278–279.} }\toendnotes[C]{\smallbreak}\pstart
           \raggedleft{}{\pb}Wien\oindex{Wien@\textbf{Wien}|pw}{ }10. 11. 903.\pend
           \pstart{}mein lieber Hermann,\pend\pstart
           ich danke dir herzlich, dſs du die \textsc{Exc.\pwindex{Schnitzler, Arthur 15.05.1862 – 21.10.1931@\textsc{Schnitzler, Arthur} (15.05.1862 – 21.10.1931), \emph{Schriftsteller, Mediziner}!Excentric16. 07. 1902@\strich\emph{Excentric} {[}16. 07. 1902{]}|pw}} zu \damage{e}inem ſo ſchönen Erfolg gebracht ha{[}ſ{]}t u gratulire dir zu
               dem ganzen Abend. Ich war mit Olga\pwindex{Schnitzler, Olga 17.01.1882 – 13.01.1970@\textsc{Schnitzler, Olga} (17.01.1882 – 13.01.1970), \emph{Schauspielerin, Sängerin}|pw} auf d Semmering\oindex{Semmering@\textbf{Semmering}|pw}; darum haben wir dich nicht um Karten
               gebeten. Ich ſelbſt wäre übrigens keineswegs \substVorne{}\textsuperscript{dort}\substDazwischen{}im Bös-Saal\oindex{Boesendorfer-Saal@\textbf{Bösendorfer-Saal}|pw}\substHinten{} geweſen – denn, du verſtehſt es gewiſs, ich kann mir eigene Sachen vor
               großem Publikum nicht vorleſen laſſen. –\pend
           \pstart
           Der Recurs iſt prachtvoll. Und ich würde ihn mit Freuden vor die nächſte {\pb}Auflage des Reigen\pwindex{Schnitzler, Arthur 15.05.1862 – 21.10.1931@\textsc{Schnitzler, Arthur} (15.05.1862 – 21.10.1931), \emph{Schriftsteller, Mediziner}!Reigen. Zehn Dialoge1900@\strich\emph{Reigen. Zehn Dialoge} {[}1900{]}|pw} drucken laſſen – we{\geminationn} er nicht ſo viel Lob über mich enthielte. Man läßt
               ſich gerne an fremden Höfen mit ſchmetternden Trompetenſtößen empfangen – aber \substVorne{}\textsuperscript{ich}\substDazwischen{}man\substHinten{} ka{\geminationn}\substVorne{}\textsuperscript{m}\substDazwischen{}ſ\substHinten{}ich doch nicht im eigenen Hauſe feiern laſſen{\dotstwo}
               Doch wäre es zu ſchade, wenn dieſes Meiſterſtück der Oeffentlichkeit vorenthalten
               würde. Daſs ſich in Wien\oindex{Wien@\textbf{Wien}|pw} nichts würde anfangen
               laſſen, war vorauszuſetzen. Die Kerle ſind ja nicht mehr feig, weil ihnen even{\pb}tuell was geſchehen
               könnte – ſondern aus Liebe zur Sache. Wie wärs denn mit dem Ausland? Berliner Tageblatt\orgindex{Berliner Tageblatt@Berliner Tageblatt|pw} (oder Voſſiſche\orgindex{Vossische Zeitung@Vossische Zeitung|pw}?) wären vielleicht zu gewinnen? Wenn kein Tagesblatt, eine Wochen
               oder Monatsſchrift? – Wie immer – ich danke dir und \textsc{Burckhard\pwindex{Burckhard, Max Eugen 14.07.1854 – 16.03.1912@\textsc{Burckhard, Max Eugen} (14.07.1854 – 16.03.1912), \emph{Schriftsteller, Rechtswissenschaftler, Theaterleiter}|pw}\strikeout{t}} vielmals und wärmſtens. Was iſt das übrigens für eine \label{K_L01338_1v}\edtext{Stelle im \textsc{Lamprecht}\pwindex{?? Werk@Nicht ermittelte Verfasserinnen und Verfasser!verbotene »Reigen«-Vorlesung03. 11. 1903@\emph{Die verbotene »Reigen«-Vorlesung} {[}03. 11. 1903{]}|pwv}}{\lemma{\textnormal{\emph{Stelle im Lamprecht}}}\Cendnote{\textnormal{Vgl. [O. V.:] \emph{Die verbotene
                        »Reigen«-Vorlesung}\pwindex{?? Werk@Nicht ermittelte Verfasserinnen und Verfasser!verbotene »Reigen«-Vorlesung03. 11. 1903@\emph{Die verbotene »Reigen«-Vorlesung} {[}03. 11. 1903{]}|pwk}. In: \emph{Die Zeit}\pwindex{Zeit1902 – 1919@\emph{Die Zeit} {[}1902 – 1919{]}|pwk},
                     Jg. 2, Nr. 396, 5. 11. 1903, S. 3: »In den weiteren
                     Darlegungen des Rekurses bespricht Bahr\pwindex{Bahr, Hermann 19.07.1863 – 15.01.1934@\textsc{Bahr, Hermann} (19.07.1863 – 15.01.1934), \emph{Schriftsteller, Kritiker}|pw} die
                     literarische Persönlichkeit Artur
                        Schnitzlers\pwindex{Schnitzler, Arthur 15.05.1862 – 21.10.1931@\textsc{Schnitzler, Arthur} (15.05.1862 – 21.10.1931), \emph{Schriftsteller, Mediziner}|pw}. Er führt an, daß Schnitzler\pwindex{Schnitzler, Arthur 15.05.1862 – 21.10.1931@\textsc{Schnitzler, Arthur} (15.05.1862 – 21.10.1931), \emph{Schriftsteller, Mediziner}|pw} als österreichischer\oindex{Oesterreich@\textbf{Österreich}|pw}
                     Dichter auch im Ausland stets an erster Stelle genannt werde, daß Schnitzler\pwindex{Schnitzler, Arthur 15.05.1862 – 21.10.1931@\textsc{Schnitzler, Arthur} (15.05.1862 – 21.10.1931), \emph{Schriftsteller, Mediziner}|pw}’s Wirken vielfache Auszeichnungen
                     erhielt, daß der Historiker Lamprecht\pwindex{Lamprecht, Karl 1856-02-25 – 1915-05-10@\textsc{Lamprecht, Karl} (1856-02-25 – 1915-05-10), \emph{Historiker}|pw} über
                     den Wien\oindex{Wien@\textbf{Wien}|pw}er in anerkennender Weise sich
                     ausgesprochen habe, [{\dots}]«. Das dürfte
                  wiederum auf die allgemeinen Ausführungen über Schnitzler\pwindex{Schnitzler, Arthur 15.05.1862 – 21.10.1931@\textsc{Schnitzler, Arthur} (15.05.1862 – 21.10.1931), \emph{Schriftsteller, Mediziner}|pwk} in Karl Lamprecht\pwindex{Lamprecht, Karl 1856-02-25 – 1915-05-10@\textsc{Lamprecht, Karl} (1856-02-25 – 1915-05-10), \emph{Historiker}|pwk}s \emph{Deutsche Geschichte. Erster Ergänzungsband}\pwindex{Lamprecht, Karl 1856-02-25 – 1915-05-10@\textsc{Lamprecht, Karl} (1856-02-25 – 1915-05-10), \emph{Historiker}!Deutsche Geschichte. Erster Ergaenzungsband. Zur juengsten deutschen Vergangenheit1902@\strich\emph{Deutsche Geschichte. Erster Ergänzungsband. Zur jüngsten deutschen Vergangenheit} {[}1902{]}|pwk}
                     (Berlin: \emph{R. Gaertners
                        Verlagsbuchhandlung}\orgindex{Rudolf Gaertners Verlagsbuchhandlung@Rudolf Gaertners Verlagsbuchhandlung|pwk}{ }1902, S. 362) Bezug nehmen.}}}\label{K_L01338_1h}, die durch die Blätter
               ging? Ich habe nichts geleſen.\pend
           \pstart
           Salten\pwindex{Salten, Felix 06.09.1869 – 08.10.1945@\textsc{Salten, Felix} (06.09.1869 – 08.10.1945), \emph{Schriftsteller, Journalist}|pw} thu ich gewiſs nicht Unrecht. {\pb}Lies nur – we{\geminationn} es ſo viel Intereſſe für dich hat, – \substVorne{}\textsuperscript{den}\substDazwischen{}meinen\substHinten{} ganzen \label{K_L01338_2v}\edtext{Brief an Salten\pwindex{Salten, Felix 06.09.1869 – 08.10.1945@\textsc{Salten, Felix} (06.09.1869 – 08.10.1945), \emph{Schriftsteller, Journalist}|pw}}{\lemma{\textnormal{\emph{Brief an Salten}}}\Cendnote{\textnormal{vom 7. 11. 1903, abgedruckt
                  in A. S.\emph{Briefe} I,468–470.}}}\label{K_L01338_2h}. Nicht um
               Lob und Tadel handelt es ſich. Das weſentliche für mich bleibt, daſs in dem \label{K_L01338_3v}\edtext{Feuilleton\pwindex{Salten, Felix 06.09.1869 – 08.10.1945@\textsc{Salten, Felix} (06.09.1869 – 08.10.1945), \emph{Schriftsteller, Journalist}!Arthur Schnitzler und sein »Reigen«07. 11. 1903@\strich\emph{Arthur Schnitzler und sein »Reigen«} {[}07. 11. 1903{]}|pwv}}{\lemma{\textnormal{\emph{Feuilleton}}}\Cendnote{\textnormal{Felix Salten\pwindex{Salten, Felix 06.09.1869 – 08.10.1945@\textsc{Salten, Felix} (06.09.1869 – 08.10.1945), \emph{Schriftsteller, Journalist}|pwk}: \emph{Arthur Schnitzler und sein Reigen}\pwindex{Salten, Felix 06.09.1869 – 08.10.1945@\textsc{Salten, Felix} (06.09.1869 – 08.10.1945), \emph{Schriftsteller, Journalist}!Arthur Schnitzler und sein »Reigen«07. 11. 1903@\strich\emph{Arthur Schnitzler und sein »Reigen«} {[}07. 11. 1903{]}|pwk}. In: \emph{Die Zeit}\pwindex{Zeit1902 – 1919@\emph{Die Zeit} {[}1902 – 1919{]}|pwk}, Jg. 2, Nr. 398, 7. 11. 1903,
                  S. 1–2.}}}\label{K_L01338_3h} genau \uline{die}{ }Sachen \introOben{}zu meinen Ungunſten\introOben{}
               drinſtehen – über deren mangelnde Berechtigung ſich ſein Verfaſſer Dutzendemale mir
               gegenüber ausgeſprochen. Lies den Brief. – Und das ärgerliche – worüber wir auch ſo
               oft geſprochen haben – der Verſuch, einem Dichter Gebiete abzuſtecken – oder zu
               verwehren. Ich, als einziger Menſch auf der bewohnten Erde, ſoll nicht mehr {\pb}das Recht haben,
               erotiſche Beziehungen zu ſchildern, oder unverehelichte junge Damen darzuſtellen? –
               Es werden nach mir noch etwa hunderttauſend Bücher von Liebe und Liebelei\pwindex{Schnitzler, Arthur 15.05.1862 – 21.10.1931@\textsc{Schnitzler, Arthur} (15.05.1862 – 21.10.1931), \emph{Schriftsteller, Mediziner}!Liebelei. Schauspiel in drei Akten1895-10-09@\strich\emph{Liebelei. Schauspiel in drei Akten} {[}1895-10-09{]}|pwv}, ſüßen und ſauren Mädeln, und Anatolen und Mäxen\pwindex{Schnitzler, Arthur 15.05.1862 – 21.10.1931@\textsc{Schnitzler, Arthur} (15.05.1862 – 21.10.1931), \emph{Schriftsteller, Mediziner}!Anatol1892-10-29@\strich\emph{Anatol} {[}1892-10-29{]}|pwv} geſchrieben
               werden – wie ſie vor mir geſchrieben worden ſind. Und gerade ich beko{\geminationm} immer ſozuſagen einen Krach in den Schädel, wenn auch
               nur \substVorne{}\textsuperscript{ein}\substDazwischen{}aus\substHinten{} der Ferne ein Hauch von Erotik über meine Geſtalten weht? {\pb}Und der letzte Krach
               geht gerade von Salten\pwindex{Salten, Felix 06.09.1869 – 08.10.1945@\textsc{Salten, Felix} (06.09.1869 – 08.10.1945), \emph{Schriftsteller, Journalist}|pw} aus, mit dem
               gemeinſchaftlich ich mich über diese Kräche \introOben{}ſo oft\introOben{} beluſtigt
               und geärgert habe? – Aber laſſen wir das auf eventuelle mündliche Unterhaltung. – Ich
               darf dich wohl dieſer Tage wieder in St Veit\oindex{Ober Sankt Veit@\textbf{Ober Sankt Veit}|pw}
               aufſuchen?\pend
           \pstart
           Herzlichſt dein getreuer{\\[\baselineskip]}\spacefill\mbox{Arthur.}\pend
           \leftskip=0em{}
         
         \endnumbering\mylabel{h}\end{ledgroupsized}  \newcommand{\dateiname}{L01338}\newcommand{\titel}{Arthur Schnitzler an Hermann Bahr, 10. 11. 1903}\newcommand{\editorInnen}{ Kurt Ifkovits,  Martin Anton Müller}%% latex-leseansicht-abspann.tex
%% Abspann für die Leseansicht.
%% Der Schalter \ifkorrekturansicht ist bereits durch den Vorspann gesetzt.

%% latex-abspann.tex
%% Gemeinsamer Abspann für Korrekturansicht und Leseansicht.
%% Setzt den Schalter \ifkorrekturansicht voraus (gesetzt in den
%% einbindenden Dateien latex-korrekturansicht-abspann.tex bzw.
%% latex-leseansicht-abspann.tex).
%% ---------------------------------------------------------------

\normalsize

% Das esempio-Environment wird nur in der Leseansicht benötigt
\ifkorrekturansicht\else
\newenvironment{esempio}[3]%
{
    \vspace{1.5ex}
    \rlap{\underline{#1}}
    \par
    \setlength{\parindent}{0cm}
    \nopagebreak
    \leftskip=#2cm
    \rightskip=#3cm
}
{
    \par
}
\fi

\doendnotes{C}
\bigskip
\vfill

\clearpage

\footnotesize

\ifkorrekturansicht
  \lohead{\textsc{register}}
\fi

% theindex-Environment neu definieren ohne reledmac
\makeatletter
\renewenvironment{theindex}{%
  \ifkorrekturansicht
    \section*{\indexname}%
  \else
    \subsubsection*{Index der erwähnten Entitäten}%
  \fi
  \setlength{\parindent}{0pt}%
  \setlength{\parskip}{0pt plus 0.3pt}%
  \let\item\@idxitem
}{%
  \ifkorrekturansicht\clearpage\fi
}
\makeatother

\IfFileExists{\jobname-pw.ind}{\input{\jobname-pw.ind}}{}

% Quellenangabe nur in der Leseansicht
\ifkorrekturansicht\else
% Fallback-Definitionen, falls die .tex-Datei \titel etc. nicht gesetzt hat
\providecommand{\titel}{}
\providecommand{\editorInnen}{}
\providecommand{\dateiname}{\jobname}

\vspace{3cm}

\vfill

\footnotesize
\textsc{Quelle}: \titel. Herausgegeben von {\editorInnen}. In: \emph{Arthur Schnitzler: Briefwechsel mit Autorinnen und Autoren}.
 Digitale Edition, https://schnitzler-briefe.acdh.oeaw.ac.at/{\dateiname}.html (Stand \today)
\fi

\end{document}


      