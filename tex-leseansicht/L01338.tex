%% latex-leseansicht-vorspann.tex
%% Vorspann für die Leseansicht.
%% Lädt die gemeinsame Datei latex-vorspann.tex mit nicht gesetztem Schalter.

\newif\ifkorrekturansicht
\korrekturansichtfalse

\input{../tex-inputs/latex-vorspann}


\section[Arthur Schnitzler an Hermann Bahr, 10. 11. 1903]{L01338 Arthur Schnitzler an Hermann Bahr, 10. 11. 1903}
\nopagebreak\mylabel{L01338v}
\rehead{ }\normalsize\beginnumbering\briefempfaengerindex{Bahr, Hermann@\textsc{Bahr, Hermann}!zzzSchnitzler, Arthur@\emph{von Arthur Schnitzler}!1903-11-101@{10. 11. 1903}|(be}
\toendnotes[C]{\smallbreak\pagebreak[2]}
\correspDesc{Versand  durch Arthur Schnitzler am 10. 11. 1903 in Wien
\newline{}Erhalt  durch Hermann Bahr im Zeitraum [10. 11. 1903 – 14. 11. 1903?] in Wien}\toendnotes[C]{\smallbreak}
\Standort{TMW, HS AM 23359 Ba.}
\physDesc{Brief, 2 Blätter, 6 Seiten, 2537 Zeichen
\newline{}Handschrift: schwarze Tinte, deutsche Kurrent
\newline{}Ordnung: 1) Lochung  2) mit Bleistift von unbekannter Hand das zweite Blatt datiert »10. 11. 03« und mit »II« versehen}
\buchAbdrucke{\weitereDrucke{1) Arthur Schnitzler: \emph{Briefe 1875–1912}. Herausgegeben von Therese Nickl und Heinrich Schnitzler. Frankfurt am Main: \emph{S. Fischer} 1981, S. 473–474.} \weitereDrucke{2) \emph{10. 11. 1903.} In: Arthur Schnitzler: \emph{The Letters of Arthur Schnitzler to Hermann Bahr}. Edited, annotated, and with an introduction, by Donald G. Daviau. Chapel Hill: \emph{The University of North Carolina Press} 1978, S. 80–81 (University of North Carolina studies in the Germanic languages
                        and literatures, 89).} \weitereDrucke{3) Hermann Bahr, Arthur Schnitzler: \emph{Briefwechsel, Aufzeichnungen, Dokumente (1891–1931)}. Herausgegeben von Kurt Ifkovits und Martin Anton Müller. Göttingen: \emph{Wallstein} 2018, S. 278–279.} }\toendnotes[C]{\smallbreak}
\pstart
           \raggedleft{}{\pb}Wien\oindex{Wien@\textbf{Wien}, \emph{Verwaltungsgebiet}|pw}{ }10. 11. 903.\pend
           
\pstart{}mein lieber Hermann,\pend\vspace{0.5em}
\pstart
           ich danke dir herzlich, dſs du die \textsc{Exc.\pwindex{Schnitzler, Arthur 15.\,5.\,1862 Wien – 21.\,10.\,1931 ebd.@\textsc{Schnitzler, Arthur} (15.\,5.\,1862 Wien – 21.\,10.\,1931 ebd.), \emph{Schriftsteller, Mediziner}!Excentric@\strich\emph{Excentric}|pw}} zu \damage{e}inem{ }ſo{ }ſchönen Erfolg gebracht ha{[}ſ{]}t u gratulire dir zu
               dem ganzen Abend. Ich war mit Olga\pwindex{Schnitzler, Olga 17.\,1.\,1882 Wien – 13.\,1.\,1970 Lugano@\textsc{Schnitzler, Olga} (17.\,1.\,1882 Wien – 13.\,1.\,1970 Lugano), \emph{Schauspielerin, Sängerin}|pw} auf d Semmering\oindex{Semmering@\textbf{Semmering}, \emph{Verwaltungsgebiet}|pw}; darum haben wir dich nicht um Karten
               gebeten. Ich{ }ſelbſt wäre übrigens keineswegs \substVorne{}\textsuperscript{dort}\substDazwischen{}im Bös-Saal\oindex{Wien@\textbf{Wien}!I., Innere Stadt@\textbf{I., Innere Stadt}!Bösendorfer-Saal@\textbf{Bösendorfer-Saal}, \emph{Veranstaltungsgebäude}|pw}\substHinten{} geweſen – denn, du verſtehſt es gewiſs, ich kann mir eigene Sachen vor
               großem Publikum nicht vorleſen laſſen. –\pend
           
\pstart
           Der Recurs iſt prachtvoll. Und ich würde ihn mit Freuden vor die nächſte {\pb}Auflage des Reigen\pwindex{Schnitzler, Arthur 15.\,5.\,1862 Wien – 21.\,10.\,1931 ebd.@\textsc{Schnitzler, Arthur} (15.\,5.\,1862 Wien – 21.\,10.\,1931 ebd.), \emph{Schriftsteller, Mediziner}!Reigen. Zehn Dialoge@\strich\emph{Reigen. Zehn Dialoge}|pw} drucken laſſen – we{\geminationn} er nicht{ }ſo viel Lob über mich enthielte. Man läßt{ }ſich gerne an fremden Höfen mit{ }ſchmetternden Trompetenſtößen empfangen – aber \substVorne{}\textsuperscript{ich}\substDazwischen{}man\substHinten{} ka{\geminationn}\substVorne{}\textsuperscript{m}\substDazwischen{}ſ\substHinten{}ich doch nicht im eigenen Hauſe feiern laſſen{\dotstwo}
               Doch wäre es zu{ }ſchade, wenn dieſes Meiſterſtück der Oeffentlichkeit vorenthalten
               würde. Daſs{ }ſich in Wien\oindex{Wien@\textbf{Wien}, \emph{Verwaltungsgebiet}|pw} nichts würde anfangen
               laſſen, war vorauszuſetzen. Die Kerle{ }ſind ja nicht mehr feig, weil ihnen even{\pb}tuell was geſchehen
               könnte –{ }ſondern aus Liebe zur Sache. Wie wärs denn mit dem Ausland? Berliner Tageblatt\orgindex{Berliner Tageblatt@Berliner Tageblatt|pw} (oder Voſſiſche\orgindex{Vossische Zeitung@Vossische Zeitung|pw}?) wären vielleicht zu gewinnen? Wenn kein Tagesblatt,
               eine Wochen oder Monatsſchrift? – Wie immer – ich danke dir und \textsc{Burckhard\pwindex{Burckhard, Max Eugen 14.\,7.\,1854 Korneuburg – 16.\,3.\,1912 Wien@\textsc{Burckhard, Max Eugen} (14.\,7.\,1854 Korneuburg – 16.\,3.\,1912 Wien), \emph{Schriftsteller, Rechtswissenschaftler, Theaterleiter}|pw}\strikeout{t}} vielmals und wärmſtens. Was iſt das übrigens für eine \label{K_L01338-1v}\edtext{Stelle im \textsc{Lamprecht}\pwindex{verbotene »Reigen«-Vorlesung@\emph{Die verbotene »Reigen«-Vorlesung}|pwv}}{\lemma{\textnormal{\emph{Stelle im Lamprecht}}}\Cendnote{\textnormal{Vgl. [O. V.]: \emph{Die verbotene
                        »Reigen«-Vorlesung}\pwindex{verbotene »Reigen«-Vorlesung@\emph{Die verbotene »Reigen«-Vorlesung}|pwk}. In: \emph{Die Zeit}\pwindex{Zeit@\emph{Die Zeit}|pwk},
                     Jg. 2, Nr. 396, 5. 11. 1903, S. 3: »In den weiteren
                     Darlegungen des Rekurses bespricht Bahr\pwindex{Bahr, Hermann 19.\,7.\,1863 Linz – 15.\,1.\,1934 München@\textsc{Bahr, Hermann} (19.\,7.\,1863 Linz – 15.\,1.\,1934 München), \emph{Schriftsteller, Kritiker}|pw}
                     die literarische Persönlichkeit Artur
                        Schnitzlers. Er führt an, daß Schnitzler als österreichischer\oindex{Österreich@\textbf{Österreich}|pw}
                     Dichter auch im Ausland stets an erster Stelle genannt werde, daß Schnitzler’s Wirken vielfache
                     Auszeichnungen erhielt, daß der Historiker Lamprecht\pwindex{Lamprecht, Karl 25.\,2.\,1856 Jessen – 10.\,5.\,1915 Leipzig@\textsc{Lamprecht, Karl} (25.\,2.\,1856 Jessen – 10.\,5.\,1915 Leipzig), \emph{Historiker}|pw} über den Wien\oindex{Wien@\textbf{Wien}, \emph{Verwaltungsgebiet}|pw}er in
                     anerkennender Weise sich ausgesprochen habe, [{\dots}]«. Das dürfte wiederum auf die allgemeinen Ausführungen über Schnitzler in Karl Lamprechts\pwindex{Lamprecht, Karl 25.\,2.\,1856 Jessen – 10.\,5.\,1915 Leipzig@\textsc{Lamprecht, Karl} (25.\,2.\,1856 Jessen – 10.\,5.\,1915 Leipzig), \emph{Historiker}|pwk}{ }\emph{Deutsche
                     Geschichte. Erster Ergänzungsband}\pwindex{Lamprecht, Karl 25.\,2.\,1856 Jessen – 10.\,5.\,1915 Leipzig@\textsc{Lamprecht, Karl} (25.\,2.\,1856 Jessen – 10.\,5.\,1915 Leipzig), \emph{Historiker}!Deutsche Geschichte. Erster Ergänzungsband. Zur jüngsten deutschen Vergangenheit@\strich\emph{Deutsche Geschichte. Erster Ergänzungsband. Zur jüngsten deutschen Vergangenheit}|pwk} (Berlin: \emph{R. Gaertners Verlagsbuchhandlung}\orgindex{Rudolf Gaertners Verlagsbuchhandlung@Rudolf Gaertners Verlagsbuchhandlung|pwk}{ }1902, S. 362) Bezug nehmen.}}}\label{K_L01338-1}, die durch die Blätter
               ging? Ich habe nichts geleſen.\pend
           
\pstart
           Salten\pwindex{Salten, Felix 6.\,9.\,1869 Budapest – 8.\,10.\,1945 Zürich@\textsc{Salten, Felix} (6.\,9.\,1869 Budapest – 8.\,10.\,1945 Zürich), \emph{Schriftsteller, Journalist, Chefredakteur}|pw} thu ich gewiſs nicht Unrecht. {\pb}Lies nur – we{\geminationn} es{ }ſo viel Intereſſe für dich hat, – \substVorne{}\textsuperscript{den}\substDazwischen{}meinen\substHinten{} ganzen \label{K_L01338-2v}\edtext{Brief an Salten\pwindex{Salten, Felix 6.\,9.\,1869 Budapest – 8.\,10.\,1945 Zürich@\textsc{Salten, Felix} (6.\,9.\,1869 Budapest – 8.\,10.\,1945 Zürich), \emph{Schriftsteller, Journalist, Chefredakteur}|pw}}{\lemma{\textnormal{\emph{Brief an Salten}}}\Cendnote{\textnormal{XXXX Auszeichnungsfehler: Dokument L02988 nicht gefunden.
               }}}\label{K_L01338-2}. Nicht um Lob und Tadel handelt es{ }ſich. Das weſentliche für mich bleibt,
               daſs in dem \label{K_L01338-3v}\edtext{Feuilleton\pwindex{Salten, Felix 6.\,9.\,1869 Budapest – 8.\,10.\,1945 Zürich@\textsc{Salten, Felix} (6.\,9.\,1869 Budapest – 8.\,10.\,1945 Zürich), \emph{Schriftsteller, Journalist, Chefredakteur}!Arthur Schnitzler und sein »Reigen«@\strich\emph{Arthur Schnitzler und sein »Reigen«}|pwv}}{\lemma{\textnormal{\emph{Feuilleton}}}\Cendnote{\textnormal{Felix Salten\pwindex{Salten, Felix 6.\,9.\,1869 Budapest – 8.\,10.\,1945 Zürich@\textsc{Salten, Felix} (6.\,9.\,1869 Budapest – 8.\,10.\,1945 Zürich), \emph{Schriftsteller, Journalist, Chefredakteur}|pwk}: \emph{Arthur Schnitzler und sein Reigen}\pwindex{Salten, Felix 6.\,9.\,1869 Budapest – 8.\,10.\,1945 Zürich@\textsc{Salten, Felix} (6.\,9.\,1869 Budapest – 8.\,10.\,1945 Zürich), \emph{Schriftsteller, Journalist, Chefredakteur}!Arthur Schnitzler und sein »Reigen«@\strich\emph{Arthur Schnitzler und sein »Reigen«}|pwk}. In: \emph{Die Zeit}\pwindex{Zeit@\emph{Die Zeit}|pwk}, Jg. 2, Nr. 398, 7. 11. 1903,
                     S. 1–2.}}}\label{K_L01338-3} genau \uline{die}{ }Sachen \introOben{}zu meinen Ungunſten\introOben{}
               drinſtehen – über deren mangelnde Berechtigung{ }ſich{ }ſein Verfaſſer Dutzendemale mir
               gegenüber ausgeſprochen. Lies den Brief. – Und das ärgerliche – worüber wir auch{ }ſo
               oft geſprochen haben – der Verſuch, einem Dichter Gebiete abzuſtecken – oder zu
               verwehren. Ich, als einziger Menſch auf der bewohnten Erde,{ }ſoll nicht mehr {\pb}das Recht haben,
               erotiſche Beziehungen zu{ }ſchildern, oder unverehelichte junge Damen darzuſtellen? –
               Es werden nach mir noch etwa hunderttauſend Bücher von Liebe und Liebelei\pwindex{Schnitzler, Arthur 15.\,5.\,1862 Wien – 21.\,10.\,1931 ebd.@\textsc{Schnitzler, Arthur} (15.\,5.\,1862 Wien – 21.\,10.\,1931 ebd.), \emph{Schriftsteller, Mediziner}!Liebelei. Schauspiel in drei Akten@\strich\emph{Liebelei. Schauspiel in drei Akten}|pwv},{ }ſüßen und{ }ſauren Mädeln, und Anatolen und Mäxen\pwindex{Schnitzler, Arthur 15.\,5.\,1862 Wien – 21.\,10.\,1931 ebd.@\textsc{Schnitzler, Arthur} (15.\,5.\,1862 Wien – 21.\,10.\,1931 ebd.), \emph{Schriftsteller, Mediziner}!Anatol@\strich\emph{Anatol}|pwv} geſchrieben
               werden – wie{ }ſie vor mir geſchrieben worden{ }ſind. Und gerade ich beko{\geminationm} immer{ }ſozuſagen einen Krach in den Schädel, wenn auch
               nur \substVorne{}\textsuperscript{ein}\substDazwischen{}aus\substHinten{} der Ferne ein Hauch von Erotik über meine Geſtalten weht? {\pb}Und der letzte Krach
               geht gerade von Salten\pwindex{Salten, Felix 6.\,9.\,1869 Budapest – 8.\,10.\,1945 Zürich@\textsc{Salten, Felix} (6.\,9.\,1869 Budapest – 8.\,10.\,1945 Zürich), \emph{Schriftsteller, Journalist, Chefredakteur}|pw} aus, mit dem
               gemeinſchaftlich ich mich über diese Kräche \introOben{}ſo oft\introOben{} beluſtigt
               und geärgert habe? – Aber laſſen wir das auf eventuelle mündliche Unterhaltung. – Ich
               darf dich wohl dieſer Tage wieder in St Veit\oindex{Wien@\textbf{Wien}!XIII., Hietzing@\textbf{XIII., Hietzing}!Ober Sankt Veit@\textbf{Ober Sankt Veit}, \emph{Ehemaliger Ort}|pw}
               aufſuchen?\pend
           
\pstart
           Herzlichſt dein getreuer{\\[\baselineskip]}\spacefill\mbox{Arthur.}\pend
           \leftskip=0em{}\selectlanguage{ngerman}\endnumbering\briefempfaengerindex{Bahr, Hermann@\textsc{Bahr, Hermann}!zzzSchnitzler, Arthur@\emph{von Arthur Schnitzler}!1903-11-101@{10. 11. 1903}|)be}\mylabel{L01338h}  \newcommand{\dateiname}{L01338}\newcommand{\titel}{Arthur Schnitzler an Hermann Bahr, 10. 11. 1903}\newcommand{\editorInnen}{Herausgegeben von Martin Anton Müller}%% latex-leseansicht-abspann.tex
%% Abspann für die Leseansicht.
%% Der Schalter \ifkorrekturansicht ist bereits durch den Vorspann gesetzt.

%% latex-abspann.tex
%% Gemeinsamer Abspann für Korrekturansicht und Leseansicht.
%% Setzt den Schalter \ifkorrekturansicht voraus (gesetzt in den
%% einbindenden Dateien latex-korrekturansicht-abspann.tex bzw.
%% latex-leseansicht-abspann.tex).
%% ---------------------------------------------------------------

\normalsize

% Das esempio-Environment wird nur in der Leseansicht benötigt
\ifkorrekturansicht\else
\newenvironment{esempio}[3]%
{
    \vspace{1.5ex}
    \rlap{\underline{#1}}
    \par
    \setlength{\parindent}{0cm}
    \nopagebreak
    \leftskip=#2cm
    \rightskip=#3cm
}
{
    \par
}
\fi

\doendnotes{C}
\bigskip
\vfill

\clearpage

\footnotesize

\ifkorrekturansicht
  \lohead{\textsc{register}}
\fi

% theindex-Environment neu definieren ohne reledmac
\makeatletter
\renewenvironment{theindex}{%
  \ifkorrekturansicht
    \section*{\indexname}%
  \else
    \subsubsection*{Index der erwähnten Entitäten}%
  \fi
  \setlength{\parindent}{0pt}%
  \setlength{\parskip}{0pt plus 0.3pt}%
  \let\item\@idxitem
}{%
  \ifkorrekturansicht\clearpage\fi
}
\makeatother

\IfFileExists{\jobname-pw.ind}{\input{\jobname-pw.ind}}{}

% Quellenangabe nur in der Leseansicht
\ifkorrekturansicht\else
% Fallback-Definitionen, falls die .tex-Datei \titel etc. nicht gesetzt hat
\providecommand{\titel}{}
\providecommand{\editorInnen}{}
\providecommand{\dateiname}{\jobname}

\vspace{3cm}

\vfill

\footnotesize
\textsc{Quelle}: \titel. Herausgegeben von {\editorInnen}. In: \emph{Arthur Schnitzler: Briefwechsel mit Autorinnen und Autoren}.
 Digitale Edition, https://schnitzler-briefe.acdh.oeaw.ac.at/{\dateiname}.html (Stand \today)
\fi

\end{document}


