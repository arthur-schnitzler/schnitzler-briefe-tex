%% latex-korrekturansicht-vorspann.tex
%% Vorspann für die Korrekturansicht.
%% Lädt die gemeinsame Datei latex-vorspann.tex mit gesetztem Schalter.

\newif\ifkorrekturansicht
\korrekturansichttrue

\input{../tex-inputs/latex-vorspann}


\section[Arthur Schnitzler an Hermann Bahr, 10. 11. 1903]{L01338 Arthur Schnitzler an Hermann Bahr, 10. 11. 1903}
\nopagebreak\mylabel{L01338v}
\rehead{ }\normalsize\beginnumbering\briefempfaengerindex{Bahr, Hermann@\textsc{Bahr, Hermann}!zzzSchnitzler, Arthur@\emph{von Arthur Schnitzler}!1903-11-101@{10. 11. 1903}|(be}
\toendnotes[C]{\smallbreak\pagebreak[2]}\Standort{TMW, HS AM 23359 Ba.}
\physDesc{Brief, 2 Blätter, 6 Seiten, 2537 Zeichen
\newline{}Handschrift: schwarze Tinte, deutsche Kurrent
\newline{}Ordnung: 1) Lochung  2) mit Bleistift von unbekannter Hand das zweite Blatt datiert »10. 11. 03« und mit »II« versehen}
\buchAbdrucke{\weitereDrucke{1) Arthur Schnitzler: \emph{Briefe 1875–1912}. Frankfurt am Main: \emph{S. Fischer} 1981, S. 473–474.} \weitereDrucke{2) Arthur Schnitzler: \emph{The Letters of Arthur Schnitzler to Hermann Bahr}. Chapel Hill: \emph{The University of North Carolina Press} 1978, S. 80–81.} \weitereDrucke{3) Hermann Bahr, Arthur Schnitzler: \emph{Briefwechsel, Aufzeichnungen, Dokumente (1891–1931)}. Göttingen: \emph{Wallstein} 2018, S. 278–279.} }\toendnotes[C]{\smallbreak}
\pstart
           \raggedleft{}{\pb}Wien\oindex{Wien@\textbf{Wien}, \emph{A.ADM2}|pw}{ }10. 11. 903.\pend
           
\pstart{}mein lieber Hermann,\pend\vspace{0.5em}
\pstart
           ich danke dir herzlich, dſs du die \textsc{Exc.\pwindex{Excentric@\emph{Excentric}|pw}} zu \damage{e}inem ſo ſchönen Erfolg gebracht ha{[}ſ{]}t u gratulire dir zu
               dem ganzen Abend. Ich war mit Olga\pwindex{Schnitzler, Olga 17.01.1882 – 13.01.1970@\textsc{Schnitzler, Olga} (17.01.1882 – 13.01.1970), \emph{Schauspieler/Schauspielerin, Sänger/Sängerin}|pw} auf d Semmering\oindex{Semmering@\textbf{Semmering}, \emph{A.ADM3}|pw}; darum haben wir dich nicht um Karten
               gebeten. Ich ſelbſt wäre übrigens keineswegs \substVorne{}\textsuperscript{dort}\substDazwischen{}im Bös-Saal\oindex{Boesendorfer-Saal@\textbf{Bösendorfer-Saal}, \emph{Veranstaltungsgebäude (K.VSB)}|pw}\substHinten{} geweſen – denn, du verſtehſt es gewiſs, ich kann mir eigene Sachen vor
               großem Publikum nicht vorleſen laſſen. –\pend
           
\pstart
           Der Recurs iſt prachtvoll. Und ich würde ihn mit Freuden vor die nächſte {\pb}Auflage des Reigen\pwindex{Reigen. Zehn Dialoge@\emph{Reigen. Zehn Dialoge}|pw} drucken laſſen – we{\geminationn} er nicht ſo viel Lob über mich enthielte. Man läßt
               ſich gerne an fremden Höfen mit ſchmetternden Trompetenſtößen empfangen – aber \substVorne{}\textsuperscript{ich}\substDazwischen{}man\substHinten{} ka{\geminationn}\substVorne{}\textsuperscript{m}\substDazwischen{}ſ\substHinten{}ich doch nicht im eigenen Hauſe feiern laſſen{\dotstwo}
               Doch wäre es zu ſchade, wenn dieſes Meiſterſtück der Oeffentlichkeit vorenthalten
               würde. Daſs ſich in Wien\oindex{Wien@\textbf{Wien}, \emph{A.ADM2}|pw} nichts würde anfangen
               laſſen, war vorauszuſetzen. Die Kerle ſind ja nicht mehr feig, weil ihnen even{\pb}tuell was geſchehen
               könnte – ſondern aus Liebe zur Sache. Wie wärs denn mit dem Ausland? Berliner Tageblatt\orgindex{Berliner Tageblatt@Berliner Tageblatt|pw} (oder Voſſiſche\orgindex{Vossische Zeitung@Vossische Zeitung|pw}?) wären vielleicht zu gewinnen? Wenn kein Tagesblatt,
               eine Wochen oder Monatsſchrift? – Wie immer – ich danke dir und \textsc{Burckhard\pwindex{Burckhard, Max Eugen 14.07.1854 – 16.03.1912@\textsc{Burckhard, Max Eugen} (14.07.1854 – 16.03.1912), \emph{Schriftsteller/Schriftstellerin, Rechtswissenschaftler/Rechtswissenschaftlerin, Theaterleiter/Theaterleiterin}|pw}\strikeout{t}} vielmals und wärmſtens. Was iſt das übrigens für eine \label{K_L01338-1v}\edtext{Stelle im \textsc{Lamprecht}\pwindex{verbotene »Reigen«-Vorlesung@\emph{Die verbotene »Reigen«-Vorlesung}|pwv}}{\lemma{\textnormal{\emph{Stelle im Lamprecht}}}\Cendnote{\textnormal{Vgl. [O. V.]: \emph{Die verbotene
                        »Reigen«-Vorlesung}\pwindex{verbotene »Reigen«-Vorlesung@\emph{Die verbotene »Reigen«-Vorlesung}|pwk}. In: \emph{Die Zeit}\pwindex{Zeit@\emph{Die Zeit}|pwk},
                     Jg. 2, Nr. 396, 5. 11. 1903, S. 3: »In den weiteren
                     Darlegungen des Rekurses bespricht Bahr\pwindex{Bahr, Hermann 19.07.1863 – 15.01.1934@\textsc{Bahr, Hermann} (19.07.1863 – 15.01.1934), \emph{Schriftsteller/Schriftstellerin, Kritiker/Kritikerin}|pw}
                     die literarische Persönlichkeit Artur
                        Schnitzlers. Er führt an, daß Schnitzler als österreichischer\oindex{Oesterreich@\textbf{Österreich}, \emph{A.PCLI}|pw}
                     Dichter auch im Ausland stets an erster Stelle genannt werde, daß Schnitzler’s Wirken vielfache
                     Auszeichnungen erhielt, daß der Historiker Lamprecht\pwindex{Lamprecht, Karl 1856-02-25 – 1915-05-10@\textsc{Lamprecht, Karl} (1856-02-25 – 1915-05-10), \emph{Historiker/Historikerin}|pw} über den Wien\oindex{Wien@\textbf{Wien}, \emph{A.ADM2}|pw}er in
                     anerkennender Weise sich ausgesprochen habe, [{\dots}]«. Das dürfte wiederum auf die allgemeinen Ausführungen über Schnitzler in Karl Lamprechts\pwindex{Lamprecht, Karl 1856-02-25 – 1915-05-10@\textsc{Lamprecht, Karl} (1856-02-25 – 1915-05-10), \emph{Historiker/Historikerin}|pwk}{ }\emph{Deutsche
                     Geschichte. Erster Ergänzungsband}\pwindex{Deutsche Geschichte. Erster Ergaenzungsband. Zur juengsten deutschen Vergangenheit@\emph{Deutsche Geschichte. Erster Ergänzungsband. Zur jüngsten deutschen Vergangenheit}|pwk} (Berlin: \emph{R. Gaertners Verlagsbuchhandlung}\orgindex{Rudolf Gaertners Verlagsbuchhandlung@Rudolf Gaertners Verlagsbuchhandlung|pwk}{ }1902, S. 362) Bezug nehmen.}}}\label{K_L01338-1}, die durch die Blätter
               ging? Ich habe nichts geleſen.\pend
           
\pstart
           Salten\pwindex{Salten, Felix 06.09.1869 – 08.10.1945@\textsc{Salten, Felix} (06.09.1869 – 08.10.1945), \emph{Schriftsteller/Schriftstellerin, Journalist/Journalistin, Chefredakteur/Chefredakteurin}|pw} thu ich gewiſs nicht Unrecht. {\pb}Lies nur – we{\geminationn} es ſo viel Intereſſe für dich hat, – \substVorne{}\textsuperscript{den}\substDazwischen{}meinen\substHinten{} ganzen \label{K_L01338-2v}\edtext{Brief an Salten\pwindex{Salten, Felix 06.09.1869 – 08.10.1945@\textsc{Salten, Felix} (06.09.1869 – 08.10.1945), \emph{Schriftsteller/Schriftstellerin, Journalist/Journalistin, Chefredakteur/Chefredakteurin}|pw}}{\lemma{\textnormal{\emph{Brief an Salten}}}\Cendnote{\textnormal{Arthur Schnitzler an Felix Salten, 7. 11. 1903.
               }}}\label{K_L01338-2}. Nicht um Lob und Tadel handelt es ſich. Das weſentliche für mich bleibt,
               daſs in dem \label{K_L01338-3v}\edtext{Feuilleton\pwindex{Arthur Schnitzler und sein »Reigen«@\emph{Arthur Schnitzler und sein »Reigen«}|pwv}}{\lemma{\textnormal{\emph{Feuilleton}}}\Cendnote{\textnormal{Felix Salten\pwindex{Salten, Felix 06.09.1869 – 08.10.1945@\textsc{Salten, Felix} (06.09.1869 – 08.10.1945), \emph{Schriftsteller/Schriftstellerin, Journalist/Journalistin, Chefredakteur/Chefredakteurin}|pwk}: \emph{Arthur Schnitzler und sein Reigen}\pwindex{Arthur Schnitzler und sein »Reigen«@\emph{Arthur Schnitzler und sein »Reigen«}|pwk}. In: \emph{Die Zeit}\pwindex{Zeit@\emph{Die Zeit}|pwk}, Jg. 2, Nr. 398, 7. 11. 1903,
                     S. 1–2.}}}\label{K_L01338-3} genau \uline{die}{ }Sachen \introOben{}zu meinen Ungunſten\introOben{}
               drinſtehen – über deren mangelnde Berechtigung ſich ſein Verfaſſer Dutzendemale mir
               gegenüber ausgeſprochen. Lies den Brief. – Und das ärgerliche – worüber wir auch ſo
               oft geſprochen haben – der Verſuch, einem Dichter Gebiete abzuſtecken – oder zu
               verwehren. Ich, als einziger Menſch auf der bewohnten Erde, ſoll nicht mehr {\pb}das Recht haben,
               erotiſche Beziehungen zu ſchildern, oder unverehelichte junge Damen darzuſtellen? –
               Es werden nach mir noch etwa hunderttauſend Bücher von Liebe und Liebelei\pwindex{Liebelei. Schauspiel in drei Akten@\emph{Liebelei. Schauspiel in drei Akten}|pwv}, ſüßen und ſauren Mädeln, und Anatolen und Mäxen\pwindex{Anatol@\emph{Anatol}|pwv} geſchrieben
               werden – wie ſie vor mir geſchrieben worden ſind. Und gerade ich beko{\geminationm} immer ſozuſagen einen Krach in den Schädel, wenn auch
               nur \substVorne{}\textsuperscript{ein}\substDazwischen{}aus\substHinten{} der Ferne ein Hauch von Erotik über meine Geſtalten weht? {\pb}Und der letzte Krach
               geht gerade von Salten\pwindex{Salten, Felix 06.09.1869 – 08.10.1945@\textsc{Salten, Felix} (06.09.1869 – 08.10.1945), \emph{Schriftsteller/Schriftstellerin, Journalist/Journalistin, Chefredakteur/Chefredakteurin}|pw} aus, mit dem
               gemeinſchaftlich ich mich über diese Kräche \introOben{}ſo oft\introOben{} beluſtigt
               und geärgert habe? – Aber laſſen wir das auf eventuelle mündliche Unterhaltung. – Ich
               darf dich wohl dieſer Tage wieder in St Veit\oindex{Ober Sankt Veit@\textbf{Ober Sankt Veit}, \emph{P.PPLX}|pw}
               aufſuchen?\pend
           
\pstart
           Herzlichſt dein getreuer{\\[\baselineskip]}\spacefill\mbox{Arthur.}\pend
           \leftskip=0em{}\selectlanguage{ngerman}\endnumbering\briefempfaengerindex{Bahr, Hermann@\textsc{Bahr, Hermann}!zzzSchnitzler, Arthur@\emph{von Arthur Schnitzler}!1903-11-101@{10. 11. 1903}|)be}\mylabel{L01338h}  \normalsize

\doendnotes{C}
\bigskip
\vfill

\clearpage

\footnotesize

\lohead{\textsc{register}}

% Definiere theindex-Environment komplett neu ohne reledmac
\makeatletter
\renewenvironment{theindex}{%
  \section*{\indexname}%
  \setlength{\parindent}{0pt}%
  \setlength{\parskip}{0pt plus 0.3pt}%
  \let\item\@idxitem
}{%
  \clearpage
}
\makeatother

\IfFileExists{\jobname-pw.ind}{\input{\jobname-pw.ind}}{}

\end{document}

      