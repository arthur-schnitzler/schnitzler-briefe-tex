%% latex-korrekturansicht-vorspann.tex
%% Vorspann für die Korrekturansicht.
%% Lädt die gemeinsame Datei latex-vorspann.tex mit gesetztem Schalter.

\newif\ifkorrekturansicht
\korrekturansichttrue

\input{../tex-inputs/latex-vorspann}


\section[Arthur Schnitzler an Richard Beer-Hofmann, {[}13. 9. 1896{]}]{L00588 Arthur Schnitzler an Richard Beer-Hofmann, {[}13. 9. 1896{]}}
\nopagebreak\mylabel{L00588v}
\rehead{ }\normalsize\beginnumbering\briefempfaengerindex{Beer-Hofmann, Richard@\textsc{Beer-Hofmann, Richard}!zzzSchnitzler, Arthur@\emph{von Arthur Schnitzler}!1896-09-131@{{[}13. 9. 1896{]}}|(be}
\toendnotes[C]{\smallbreak\pagebreak[2]}\Standort{YCGL, MSS 31.}
\physDesc{Briefkarte, , Umschlag, 320 Zeichen
\newline{}Handschrift: 1) Bleistift, deutsche Kurrent\hspace{1em}2) Bleistift, lateinische Kurrent (\noindent{}Adresse)\hspace{1em}
\newline{}Versand: ohne postalischen Übermittlungsvermerk }
\buchAbdrucke{\weitereDrucke{Arthur Schnitzler, Richard Beer-Hofmann: \emph{Briefwechsel 1891–1931}. Wien, Zürich: \emph{Europaverlag} 1992, S. 96.} }\toendnotes[C]{\smallbreak}\pstart{}{\pb}An Dr. Rich. Beer Hofmann\pend{}{\bigskip}\vspace{1em}
\pstart
           \raggedleft{}{\pb}\label{T_L00588-1v}\edtext{12/IX 96}{\lemma{\textnormal{\emph{12/IX 96}}}\Cendnote{\textnormal{auf der Rückseite des Umschlags}}}\label{T_L00588-1}\pend
           
\pstart
           \raggedleft{}{\pb}\label{K_L00588-1v}\edtext{So{\geminationn}tag}{\lemma{\textnormal{\emph{Sonntag}}}\Cendnote{\textnormal{Der 13. 9. 1896 war ein
                     Sonntag, Schnitzler irrt sich mit der
                     Beschriftung »12/IX 96«.}}}\label{K_L00588-1}. – ½ 6. \textsc{N. M.}\pend
           \vspace{0.5em}
\pstart
           Lieber Richard, wie ka{\geminationn} man nicht
               einmal eine Poſt zu Haus laſſen wo man zu finden wäre! Ich ko{\geminationm}e per Rad von Mödling\oindex{Moedling@\textbf{Mödling}, \emph{P.PPLA3}|pw} – {\pb}Tini\pwindex{Schoenberger, Christine 1875-11-17 – 1971-02-03@\textsc{Schönberger, Christine} (1875-11-17 – 1971-02-03), \emph{Gastwirt/Gastwirtin}|pw} – Alland\oindex{Alland@\textbf{Alland}, \emph{A.ADM3}|pw} – Neuhaus\oindex{Neuhaus@\textbf{Neuhaus}, \emph{P.PPL}|pw} – Pottenſtein\oindex{Pottenstein@\textbf{Pottenstein}, \emph{P.PPLA3}|pw} – Antonsgaſſe 4\oindex{Antonsgasse@\textbf{Antonsgasse}, \emph{Straße (K.STR)}|pw} – Franzensgaſſe 54\oindex{Kaiser-Franz-Ring@\textbf{Kaiser-Franz-Ring}, \emph{Straße (K.STR)}|pw} –\pend
           
\pstart
           Der Doctor Schwarzkopf\pwindex{Schwarzkopf, Gustav 07.11.1853 – 13.11.1939@\textsc{Schwarzkopf, Gustav} (07.11.1853 – 13.11.1939), \emph{Schriftsteller/Schriftstellerin}|pw} iſt auch da, der grüßt
               Sie, aber nicht ſo herzlich wie ich.\pend
           \pstart Ihr \spacefill\mbox{ArthSch}\pend{}\selectlanguage{ngerman}\endnumbering\briefempfaengerindex{Beer-Hofmann, Richard@\textsc{Beer-Hofmann, Richard}!zzzSchnitzler, Arthur@\emph{von Arthur Schnitzler}!1896-09-131@{{[}13. 9. 1896{]}}|)be}\mylabel{L00588h}  \normalsize

\doendnotes{C}
\bigskip
\vfill

\clearpage

\footnotesize

\lohead{\textsc{register}}

% Definiere theindex-Environment komplett neu ohne reledmac
\makeatletter
\renewenvironment{theindex}{%
  \section*{\indexname}%
  \setlength{\parindent}{0pt}%
  \setlength{\parskip}{0pt plus 0.3pt}%
  \let\item\@idxitem
}{%
  \clearpage
}
\makeatother

\IfFileExists{\jobname-pw.ind}{\input{\jobname-pw.ind}}{}

\end{document}

      