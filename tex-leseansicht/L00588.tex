%% latex-leseansicht-vorspann.tex
%% Vorspann für die Leseansicht.
%% Lädt die gemeinsame Datei latex-vorspann.tex mit nicht gesetztem Schalter.

\newif\ifkorrekturansicht
\korrekturansichtfalse

\input{../tex-inputs/latex-vorspann}


\section[Arthur Schnitzler an Richard Beer-Hofmann, {[}13. 9. 1896{]}]{L00588 Arthur Schnitzler an Richard Beer-Hofmann, {[}13. 9. 1896{]}}
\nopagebreak\mylabel{L00588v}
\rehead{ }\normalsize\beginnumbering\briefempfaengerindex{Beer-Hofmann, Richard@\textsc{Beer-Hofmann, Richard}!zzzSchnitzler, Arthur@\emph{von Arthur Schnitzler}!1896-09-131@{{[}13. 9. 1896{]}}|(be}
\toendnotes[C]{\smallbreak\pagebreak[2]}
\correspDesc{Versand  durch Arthur Schnitzler am [13. 9. 1896] in Wien
\newline{}Erhalt  durch Richard Beer-Hofmann im Zeitraum [14. 9. 1896
                  – 18. 9. 1896?] in Baden bei Wien}\toendnotes[C]{\smallbreak}
\Standort{YCGL, MSS 31.}
\physDesc{Briefkarte, , Kuvert, 320 Zeichen
\newline{}Handschrift: Bleistift, deutsche Kurrent
\newline{}Versand: ohne postalischen Übermittlungsvermerk }
\buchAbdrucke{\weitereDrucke{Arthur Schnitzler, Richard Beer-Hofmann: \emph{Briefwechsel 1891–1931}. Herausgegeben von Konstanze Fliedl. Wien, Zürich: \emph{Europaverlag} 1992, S. 96.} }\toendnotes[C]{\smallbreak}\pstart{}\textsc{{\pb}An Dr. Rich. Beer Hofmann}\pend{}{\bigskip}\vspace{1em}
\pstart
           \raggedleft{}{\pb}\label{T_L00588-1v}\edtext{12/IX 96}{\lemma{\textnormal{\emph{12/IX 96}}}\Cendnote{\textnormal{auf der Rückseite des Umschlags}}}\label{T_L00588-1}\pend
           
\pstart
           \raggedleft{}{\pb}\label{K_L00588-1v}\edtext{So{\geminationn}tag}{\lemma{\textnormal{\emph{Sonntag}}}\Cendnote{\textnormal{Der 13. 9. 1896 war ein
                     Sonntag, Schnitzler irrt sich mit der
                     Beschriftung »12/IX 96«.}}}\label{K_L00588-1}. – ½ 6. \textsc{N. M.}\pend
           \vspace{0.5em}
\pstart
           Lieber Richard, wie ka{\geminationn} man nicht
               einmal eine Poſt zu Haus laſſen wo man zu finden wäre! Ich ko{\geminationm}e per Rad von Mödling\oindex{Mödling@\textbf{Mödling}, \emph{Hauptstadt}|pw} – {\pb}Tini\pwindex{Kepert, Christine 17.\,11.\,1875 – 3.\,2.\,1971 Wien@\textsc{Kepert, Christine} (17.\,11.\,1875 – 3.\,2.\,1971 Wien), \emph{Gastwirtin}|pw} – Alland\oindex{Alland@\textbf{Alland}, \emph{Verwaltungsgebiet}|pw} – Neuhaus\oindex{Neuhaus@\textbf{Neuhaus}|pw} – Pottenſtein\oindex{Pottenstein@\textbf{Pottenstein}, \emph{Hauptstadt}|pw} – Antonsgaſſe 4\oindex{Antonsgasse@\textbf{Antonsgasse}, \emph{Straße}|pw} – Franzensgaſſe 54\oindex{Kaiser-Franz-Ring@\textbf{Kaiser-Franz-Ring}, \emph{Straße}|pw} –\pend
           
\pstart
           Der Doctor Schwarzkopf\pwindex{Schwarzkopf, Gustav 7.\,11.\,1853 Wien – 13.\,11.\,1939 ebd.@\textsc{Schwarzkopf, Gustav} (7.\,11.\,1853 Wien – 13.\,11.\,1939 ebd.), \emph{Schriftsteller}|pw} iſt auch da, der grüßt
               Sie, aber nicht{ }ſo herzlich wie ich.\pend
           \pstart Ihr \spacefill\mbox{ArthSch}\pend{}\selectlanguage{ngerman}\endnumbering\briefempfaengerindex{Beer-Hofmann, Richard@\textsc{Beer-Hofmann, Richard}!zzzSchnitzler, Arthur@\emph{von Arthur Schnitzler}!1896-09-131@{{[}13. 9. 1896{]}}|)be}\mylabel{L00588h}  \newcommand{\dateiname}{L00588}\newcommand{\titel}{Arthur Schnitzler an Richard Beer-Hofmann, [13. 9. 1896]}\newcommand{\editorInnen}{Martin Anton Müller und Gerd-Hermann Susen}%% latex-leseansicht-abspann.tex
%% Abspann für die Leseansicht.
%% Der Schalter \ifkorrekturansicht ist bereits durch den Vorspann gesetzt.

%% latex-abspann.tex
%% Gemeinsamer Abspann für Korrekturansicht und Leseansicht.
%% Setzt den Schalter \ifkorrekturansicht voraus (gesetzt in den
%% einbindenden Dateien latex-korrekturansicht-abspann.tex bzw.
%% latex-leseansicht-abspann.tex).
%% ---------------------------------------------------------------

\normalsize

% Das esempio-Environment wird nur in der Leseansicht benötigt
\ifkorrekturansicht\else
\newenvironment{esempio}[3]%
{
    \vspace{1.5ex}
    \rlap{\underline{#1}}
    \par
    \setlength{\parindent}{0cm}
    \nopagebreak
    \leftskip=#2cm
    \rightskip=#3cm
}
{
    \par
}
\fi

\doendnotes{C}
\bigskip
\vfill

\clearpage

\footnotesize

\ifkorrekturansicht
  \lohead{\textsc{register}}
\fi

% theindex-Environment neu definieren ohne reledmac
\makeatletter
\renewenvironment{theindex}{%
  \ifkorrekturansicht
    \section*{\indexname}%
  \else
    \subsubsection*{Index der erwähnten Entitäten}%
  \fi
  \setlength{\parindent}{0pt}%
  \setlength{\parskip}{0pt plus 0.3pt}%
  \let\item\@idxitem
}{%
  \ifkorrekturansicht\clearpage\fi
}
\makeatother

\IfFileExists{\jobname-pw.ind}{\input{\jobname-pw.ind}}{}

% Quellenangabe nur in der Leseansicht
\ifkorrekturansicht\else
% Fallback-Definitionen, falls die .tex-Datei \titel etc. nicht gesetzt hat
\providecommand{\titel}{}
\providecommand{\editorInnen}{}
\providecommand{\dateiname}{\jobname}

\vspace{3cm}

\vfill

\footnotesize
\textsc{Quelle}: \titel. Herausgegeben von {\editorInnen}. In: \emph{Arthur Schnitzler: Briefwechsel mit Autorinnen und Autoren}.
 Digitale Edition, https://schnitzler-briefe.acdh.oeaw.ac.at/{\dateiname}.html (Stand \today)
\fi

\end{document}


