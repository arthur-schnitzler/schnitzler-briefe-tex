%% latex-leseansicht-vorspann.tex
%% Vorspann für die Leseansicht.
%% Lädt die gemeinsame Datei latex-vorspann.tex mit nicht gesetztem Schalter.

\newif\ifkorrekturansicht
\korrekturansichtfalse

\input{../tex-inputs/latex-vorspann}


         
         \renewcommand{\erwaehntePersonen}{Personen: Richard Beer-Hofmann, Ferdinand von Saar, Felix Salten}
         \renewcommand{\erwaehnteOrte}{Orte: Musikverein, Wien}
         \renewcommand{\erwaehnteWerke}{Werke: Die Räuber}
               \section[Hugo von Hofmannsthal an Arthur Schnitzler, {[}13. 1. 1893{]}]{ Hugo von Hofmannsthal an Arthur Schnitzler, {[}13. 1. 1893{]}}\nopagebreak\mylabel{v}\rehead{ }\begin{ledgroupsized}[t]{13cm}\normalsize\beginnumbering \toendnotes[C]{\smallbreak\pagebreak[2]} \Standort{CUL, Schnitzler, B 43.}
\physDesc{Briefkarte mit aufgeprägtem Wappen
\newline{}Handschrift: schwarze Tinte, deutsche Kurrent
\newline{}Schnitzler: mit Bleistift datiert: »13/1 93« \newline{}Ordnung: mit Bleistift von unbekannter Hand nummeriert:
                                        »39« }\buchAbdrucke{\weitereDrucke{Hugo von Hofmannsthal, Arthur Schnitzler: \emph{Briefwechsel}. Hg. Therese Nickl und Heinrich Schnitzler. Frankfurt am Main: \emph{S. Fischer} 1964, S. 35.} }\toendnotes[C]{\smallbreak}\pstart
           \raggedleft{}{\pb}Freitag.\pend
           \pstart{}mein lieber Arthur.\pend\pstart
           Ich habe den Sitz\pwindex{\textcolor{red}{\textsuperscript{XXXX1 indx}}!Raeuber1781@\strich\emph{Die Räuber} {[}1781{]}|pwv} für \textsc{Samstag} natürlich
                    genommen, kann aber leider nicht gehen, weil am ſelben Abend eine \label{K_L00159_1v}\edtext{Vorleſung F. v. \textsc{Saars}\pwindex{Saar, Ferdinand von 30.09.1833 – 24.07.1906@\textsc{Saar, Ferdinand von} (30.09.1833 – 24.07.1906), \emph{Schriftsteller}|pw}}{\lemma{\textnormal{\emph{Vorleſung F. v. Saars}}}\Cendnote{\textnormal{Die Lesung fand am
                            14. 1. 1893 im Kleinen
                            Musikvereinssaal\oindex{Musikverein@\textbf{Musikverein}|pwk} statt.}}}\label{K_L00159_1h}{ }ſtattfindet, zu der zu kommen ich ſeit langer
                    Zeit verſprochen habe. Ich hoffe aber beſtimmt, wenn mir nicht abgeſchrieben
                    wird, Richard\pwindex{Beer-Hofmann, Richard 1866-07-11 – 1945-09-26@\textsc{Beer-Hofmann, Richard} (1866-07-11 – 1945-09-26), \emph{Schriftsteller}|pw} u. Salten\pwindex{Salten, Felix 06.09.1869 – 08.10.1945@\textsc{Salten, Felix} (06.09.1869 – 08.10.1945), \emph{Schriftsteller, Journalist}|pw} am Sonntag bei Ihnen zu treffen und
                    wünſche Euch für \textsc{Samstag} beſte Unterhaltung.\pend
           \pstart
           Herzlichſt Ihr{\\[\baselineskip]}\spacefill\mbox{Hugo}\pend
           \leftskip=0em{}\pstart
           \noindent{}ehemals Schriftſteller.\pend
           
         
         \endnumbering\mylabel{h}\end{ledgroupsized}  \newcommand{\dateiname}{L00159}\newcommand{\titel}{Hugo von Hofmannsthal an Arthur Schnitzler, [13. 1. 1893]}\newcommand{\editorInnen}{Martin Anton Müller und Gerd-Hermann Susen}%% latex-leseansicht-abspann.tex
%% Abspann für die Leseansicht.
%% Der Schalter \ifkorrekturansicht ist bereits durch den Vorspann gesetzt.

%% latex-abspann.tex
%% Gemeinsamer Abspann für Korrekturansicht und Leseansicht.
%% Setzt den Schalter \ifkorrekturansicht voraus (gesetzt in den
%% einbindenden Dateien latex-korrekturansicht-abspann.tex bzw.
%% latex-leseansicht-abspann.tex).
%% ---------------------------------------------------------------

\normalsize

% Das esempio-Environment wird nur in der Leseansicht benötigt
\ifkorrekturansicht\else
\newenvironment{esempio}[3]%
{
    \vspace{1.5ex}
    \rlap{\underline{#1}}
    \par
    \setlength{\parindent}{0cm}
    \nopagebreak
    \leftskip=#2cm
    \rightskip=#3cm
}
{
    \par
}
\fi

\doendnotes{C}
\bigskip
\vfill

\clearpage

\footnotesize

\ifkorrekturansicht
  \lohead{\textsc{register}}
\fi

% theindex-Environment neu definieren ohne reledmac
\makeatletter
\renewenvironment{theindex}{%
  \ifkorrekturansicht
    \section*{\indexname}%
  \else
    \subsubsection*{Index der erwähnten Entitäten}%
  \fi
  \setlength{\parindent}{0pt}%
  \setlength{\parskip}{0pt plus 0.3pt}%
  \let\item\@idxitem
}{%
  \ifkorrekturansicht\clearpage\fi
}
\makeatother

\IfFileExists{\jobname-pw.ind}{\input{\jobname-pw.ind}}{}

% Quellenangabe nur in der Leseansicht
\ifkorrekturansicht\else
% Fallback-Definitionen, falls die .tex-Datei \titel etc. nicht gesetzt hat
\providecommand{\titel}{}
\providecommand{\editorInnen}{}
\providecommand{\dateiname}{\jobname}

\vspace{3cm}

\vfill

\footnotesize
\textsc{Quelle}: \titel. Herausgegeben von {\editorInnen}. In: \emph{Arthur Schnitzler: Briefwechsel mit Autorinnen und Autoren}.
 Digitale Edition, https://schnitzler-briefe.acdh.oeaw.ac.at/{\dateiname}.html (Stand \today)
\fi

\end{document}


      