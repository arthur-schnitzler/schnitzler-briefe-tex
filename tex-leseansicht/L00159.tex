%% latex-korrekturansicht-vorspann.tex
%% Vorspann für die Korrekturansicht.
%% Lädt die gemeinsame Datei latex-vorspann.tex mit gesetztem Schalter.

\newif\ifkorrekturansicht
\korrekturansichttrue

\input{../tex-inputs/latex-vorspann}


\section[Hugo von Hofmannsthal an Arthur Schnitzler, {[}13. 1. 1893{]}]{L00159 Hugo von Hofmannsthal an Arthur Schnitzler, {[}13. 1. 1893{]}}
\nopagebreak\mylabel{L00159v}
\rehead{ }\normalsize\beginnumbering\briefempfaengerindex{Schnitzler, Arthur@\textsc{Schnitzler, Arthur}!zzzHofmannsthal, Hugo von@\emph{von Hugo von Hofmannsthal}!1893-01-131@{{[}13. 1. 1893{]}}|(be}
\toendnotes[C]{\smallbreak\pagebreak[2]}\Standort{CUL, Schnitzler, B 43.}
\physDesc{Briefkarte, 409 Zeichen (aufgeprägtes Wappen )
\newline{}Handschrift: schwarze Tinte, deutsche Kurrent
\newline{}Schnitzler: mit Bleistift datiert: »13/1 93« 
\newline{}Ordnung: mit Bleistift von unbekannter Hand nummeriert:
                                    »39« }
\buchAbdrucke{\weitereDrucke{Hugo von Hofmannsthal, Arthur Schnitzler: \emph{Briefwechsel}. Frankfurt am Main: \emph{S. Fischer} 1964, S. 35.} }\toendnotes[C]{\smallbreak}
\pstart
           \raggedleft{}{\pb}Freitag.\pend
           
\pstart{}mein lieber Arthur.\pend\vspace{0.5em}
\pstart
           Ich habe den Sitz\pwindex{Raeuber. Ein Schauspiel@\emph{Die Räuber. Ein Schauspiel}|pwv} für \textsc{Samstag} natürlich genommen, kann aber leider nicht gehen, weil am ſelben Abend eine
                  \label{K_L00159-1v}\edtext{Vorleſung F. v. \textsc{Saars}\pwindex{Saar, Ferdinand von 30.09.1833 – 24.07.1906@\textsc{Saar, Ferdinand von} (30.09.1833 – 24.07.1906), \emph{Schriftsteller/Schriftstellerin}|pw}}{\lemma{\textnormal{\emph{Vorleſung F. v. Saars}}}\Cendnote{\textnormal{Die Lesung fand am
                     14. 1. 1893 im Kleinen
                     Musikvereinssaal\oindex{Musikverein@\textbf{Musikverein}, \emph{Konzertsaal (K.KNZ)}|pwk} statt.}}}\label{K_L00159-1}{ }ſtattfindet, zu der zu kommen ich ſeit langer Zeit
               verſprochen habe. Ich hoffe aber beſtimmt, wenn mir nicht abgeſchrieben wird, Richard\pwindex{Beer-Hofmann, Richard 1866-07-11 – 1945-09-26@\textsc{Beer-Hofmann, Richard} (1866-07-11 – 1945-09-26), \emph{Schriftsteller/Schriftstellerin}|pw} u. Salten\pwindex{Salten, Felix 06.09.1869 – 08.10.1945@\textsc{Salten, Felix} (06.09.1869 – 08.10.1945), \emph{Schriftsteller/Schriftstellerin, Journalist/Journalistin, Chefredakteur/Chefredakteurin}|pw} am Sonntag bei Ihnen zu treffen und wünſche Euch für \textsc{Samstag} beſte Unterhaltung.\pend
           
\pstart
           Herzlichſt Ihr{\\[\baselineskip]}\spacefill\mbox{Hugo}\pend
           \leftskip=0em{}
\pstart
           \noindent{}ehemals Schriftſteller.\pend
           \selectlanguage{ngerman}\endnumbering\briefempfaengerindex{Schnitzler, Arthur@\textsc{Schnitzler, Arthur}!zzzHofmannsthal, Hugo von@\emph{von Hugo von Hofmannsthal}!1893-01-131@{{[}13. 1. 1893{]}}|)be}\mylabel{L00159h}  \normalsize

\doendnotes{C}
\bigskip
\vfill

\clearpage

\footnotesize

\lohead{\textsc{register}}

% Definiere theindex-Environment komplett neu ohne reledmac
\makeatletter
\renewenvironment{theindex}{%
  \section*{\indexname}%
  \setlength{\parindent}{0pt}%
  \setlength{\parskip}{0pt plus 0.3pt}%
  \let\item\@idxitem
}{%
  \clearpage
}
\makeatother

\IfFileExists{\jobname-pw.ind}{\input{\jobname-pw.ind}}{}

\end{document}

      