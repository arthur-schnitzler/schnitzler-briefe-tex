%% latex-korrekturansicht-vorspann.tex
%% Vorspann für die Korrekturansicht.
%% Lädt die gemeinsame Datei latex-vorspann.tex mit gesetztem Schalter.

\newif\ifkorrekturansicht
\korrekturansichttrue

\input{../tex-inputs/latex-vorspann}


\section[ Arthur Schnitzler: Widmungsexemplar Der einsame Weg für Felix Salten, {[}10.?{]} 2. 1904]{L03604 Arthur Schnitzler: Widmungsexemplar Der einsame Weg für Felix
               Salten, {[}10.?{]} 2. 1904}
\nopagebreak\mylabel{L03604v}
\rehead{ }\normalsize\beginnumbering\briefempfaengerindex{Salten, Felix@\textsc{Salten, Felix}!zzzSchnitzler, Arthur@\emph{von Arthur Schnitzler}!1904-02-102@{{[}10.?{]} 2. 1904}|(be}
\toendnotes[C]{\smallbreak\pagebreak[2]}\Standort{Wienbibliothek im Rathaus, A-355811, DS-2019-56.}
\physDesc{Widmung am Schmutztitel, 46 Zeichen
\newline{}Handschrift: schwarze Tinte, deutsche Kurrent}\toendnotes[C]{\smallbreak}
\pstart
           \noindent{}{\pb}Meinem lieben Felix Salten\pend
           \pstart \spacefill\mbox{ArthSch}\pend{}
\pstart
           \label{K_L03604-1v}\edtext{Feber 904}{\lemma{\textnormal{\emph{Feber 904}}}\Cendnote{\textnormal{Siehe A. S.: \emph{Tagebuch}, 10. 2. 1904.
                  }}}\label{K_L03604-1}.\pend
           \selectlanguage{ngerman}\vspace{1em}{\vspace{1\baselineskip}}
\pstart
           \centering{}{\pb}\textcolor{gray}{\textbf{\textbf{Der einſame Weg\pwindex{einsame Weg. Schauspiel in fuenf Akten@\emph{Der einsame Weg. Schauspiel in fünf Akten}|pw}}}}\pend
           
\pstart
           \centering{}\textcolor{gray}{\textbf{Schauſpiel in fünf Akten}}\pend
           
\pstart
           \centering{}\textcolor{gray}{\textbf{von}}\pend
           
\pstart
           \centering{}\textcolor{gray}{\textbf{\textbf{Arthur Schnitzler}}}\pend
           {\vspace{1\baselineskip}}
\pstart
           \centering{}\textcolor{gray}{\textbf{Zweite Auflage}}\pend
           {\vspace{1\baselineskip}}
\pstart
           \centering{}\textcolor{gray}{\textbf{Berlin\oindex{Berlin@\textbf{Berlin}, \emph{P.PPLC}|pw}{ }1904}}\pend
           
\pstart
           \centering{}\textcolor{gray}{\textbf{S. Fiſcher, Verlag\orgindex{S. Fischer Verlag@S. Fischer Verlag|pw}}}\pend
           \selectlanguage{ngerman}\endnumbering\briefempfaengerindex{Salten, Felix@\textsc{Salten, Felix}!zzzSchnitzler, Arthur@\emph{von Arthur Schnitzler}!1904-02-102@{{[}10.?{]} 2. 1904}|)be}\mylabel{L03604h}  \normalsize

\doendnotes{C}
\bigskip
\vfill

\clearpage

\footnotesize

\lohead{\textsc{register}}

% Definiere theindex-Environment komplett neu ohne reledmac
\makeatletter
\renewenvironment{theindex}{%
  \section*{\indexname}%
  \setlength{\parindent}{0pt}%
  \setlength{\parskip}{0pt plus 0.3pt}%
  \let\item\@idxitem
}{%
  \clearpage
}
\makeatother

\IfFileExists{\jobname-pw.ind}{\input{\jobname-pw.ind}}{}

\end{document}

      