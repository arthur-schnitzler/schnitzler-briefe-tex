%% latex-korrekturansicht-vorspann.tex
%% Vorspann für die Korrekturansicht.
%% Lädt die gemeinsame Datei latex-vorspann.tex mit gesetztem Schalter.

\newif\ifkorrekturansicht
\korrekturansichttrue

\input{../tex-inputs/latex-vorspann}


\section[Stefan Zweig an Arthur Schnitzler, {[}zwischen 18. und 24. 10.? 1910{]}]{L03625 Stefan Zweig an Arthur Schnitzler, {[}zwischen 18. und
               24. 10.? 1910{]}}
\nopagebreak\mylabel{L03625v}
\rehead{ }\normalsize\beginnumbering\briefempfaengerindex{Schnitzler, Arthur@\textsc{Schnitzler, Arthur}!zzzZweig, Stefan@\emph{von Stefan Zweig}!1910-10-243@{{[}zwischen 18. und
                  24. 10.? 1910{]}}|(be}
\toendnotes[C]{\smallbreak\pagebreak[2]}\Standort{CUL, Schnitzler, B 118.}
\physDesc{Brief, 1 Blatt, 2 Seiten, 788 Zeichen
\newline{}Handschrift: blaue Tinte, lateinische Kurrent
\newline{}Schnitzler: mit Bleistift »\textsc{Zweig}« }
\buchAbdrucke{\weitereDrucke{Stefan Zweig: \emph{Briefwechsel mit Hermann Bahr, Sigmund Freud, Rainer Maria
                        Rilke und Arthur Schnitzler}. Frankfurt am Main: \emph{S. Fischer} 1987, S. 361.} }\toendnotes[C]{\smallbreak}
\pstart
           {\pb}Wien VIII. Kochgasse 8\oindex{Kochgasse 8@\textbf{Kochgasse 8}, \emph{Wohngebäude (K.WHS)}|pw}\pend
           
\pstart{}Sehr verehrter Herr Doktor,\pend\vspace{0.5em}
\pstart
           ich weiss nicht, ob Sie schon \label{K_L03625-1v}\edtext{in Wien\oindex{Wien@\textbf{Wien}, \emph{A.ADM2}|pw}{ }}{\lemma{\textnormal{\emph{in Wien }}}\Cendnote{\textnormal{Schnitzler und seine Frau\pwindex{Schnitzler, Olga 17.01.1882 – 13.01.1970@\textsc{Schnitzler, Olga} (17.01.1882 – 13.01.1970), \emph{Schauspieler/Schauspielerin, Sänger/Sängerin}|pwkv} verbrachten die Tage vom 16. 10. 1910 bis zum
                     19. 10. 1910 am
                     Semmering\oindex{Semmering@\textbf{Semmering}, \emph{A.ADM3}|pwk}. Stefan Zweig\pwindex{Zweig, Stefan 28.11.1881 – 23.02.1942@\textsc{Zweig, Stefan} (28.11.1881 – 23.02.1942), \emph{Schriftsteller/Schriftstellerin}|pwk}, der Schnitzler am 11. 10. 1910 bei einer Lesung Jakob
                     Wassermanns\pwindex{Wassermann, Jakob 10.03.1873 – 01.01.1934@\textsc{Wassermann, Jakob} (10.03.1873 – 01.01.1934), \emph{Schriftsteller/Schriftstellerin}|pwk} getroffen hatte, wusste vermutlich von diesem Ausflug. Nach
                  hinten zeitlich begrenzt ist der Brief durch die Erwähnung der \emph{Uraufführung von \emph{Der junge
                        Medardus}\pwindex{junge Medardus. Dramatische Historie in einem Vorspiel und fuenf
                  Aufzuegen@\emph{Der junge Medardus. Dramatische Historie in einem Vorspiel und fünf Aufzügen}|pwk} am 24. 11. 1910}\eventindex{Burgtheater@\textbf{Burgtheater}!Urauffuehrung von Der junge Medardus, 24.11.1910@Uraufführung von Der junge Medardus, 24.11.1910|pwk}. Schnitzler unternahm in dieser Zeit
                  keine weiteren Reisen, es muss also von dieser die Rede sein. Entsprechend dürfte
                  dieses Korrespondenzstück in den Zeitraum fallen, der zwei Tage nach Beginn der
                  Reise und fünf Tage nach ihrem Ende anzusetzen ist.}}}\label{K_L03625-1} sind und ob ich meine
               Bitte an Sie richten darf. Sie will nicht viel: ich möchte Sie gerne wieder einmal
               besuchen dürfen, das ist Alles. Man erzählt mir viel von Ihrem neuen Stück\pwindex{weite Land. Tragikomoedie in fuenf Akten@\emph{Das weite Land. Tragikomödie in fünf Akten}|pwv} und so viel Gutes, dass ich ganz
               ungeduldig werde und das Frühjahr kaum erwarten kann: freilich ist zuvor noch die
               Freude der \label{K_L03625-2v}\edtext{»Medardus\pwindex{junge Medardus. Dramatische Historie in einem Vorspiel und fuenf
                  Aufzuegen@\emph{Der junge Medardus. Dramatische Historie in einem Vorspiel und fünf Aufzügen}|pw}«-Première\eventindex{Burgtheater@\textbf{Burgtheater}!Urauffuehrung von Der junge Medardus, 24.11.1910@Uraufführung von Der junge Medardus, 24.11.1910|pw}}{\lemma{\textnormal{\emph{»Medardus«-Première}}}\Cendnote{\textnormal{Schnitzlers Schauspiel Der junge Medardus\pwindex{junge Medardus. Dramatische Historie in einem Vorspiel und fuenf
                  Aufzuegen@\emph{Der junge Medardus. Dramatische Historie in einem Vorspiel und fünf Aufzügen}|pwkv} wurde am 24. 11. 1910 am Burgtheater\oindex{Burgtheater@\textbf{Burgtheater}, \emph{S.THTR}|pwk}{ }uraufgeführt\eventindex{Burgtheater@\textbf{Burgtheater}!Urauffuehrung von Der junge Medardus, 24.11.1910@Uraufführung von Der junge Medardus, 24.11.1910|pwkv}.}}}\label{K_L03625-2}! Wie
               viel Ihnen doch in den letzten Jahren gelungen ist, wir, die wir Ihr Werk lieben,
               sind immer \strikeout{noch} ungeduldig und wollen noch immer {\pb}mehr – das müssen Sie uns verzeihen,
               dass wir bei aller Liebe am Gegebenen noch nicht genug haben und uns doppelt auf das
               Werdende freuen.\pend
           
\pstart
           Mit den besten Empfehlungen an Ihre Frau Gemahlin\pwindex{Schnitzler, Olga 17.01.1882 – 13.01.1970@\textsc{Schnitzler, Olga} (17.01.1882 – 13.01.1970), \emph{Schauspieler/Schauspielerin, Sänger/Sängerin}|pwv} und den ergebensten Grüssen{\\[\baselineskip]}in Verehrung
               getreu{\\[\baselineskip]} Ihr{\\[\baselineskip]}\spacefill\mbox{Stefan Zweig}\pend
           \leftskip=0em{}\selectlanguage{ngerman}\endnumbering\briefempfaengerindex{Schnitzler, Arthur@\textsc{Schnitzler, Arthur}!zzzZweig, Stefan@\emph{von Stefan Zweig}!1910-10-183@{{[}zwischen 18. und
                  24. 10.? 1910{]}}|)be}\mylabel{L03625h}  \normalsize

\doendnotes{C}
\bigskip
\vfill

\clearpage

\footnotesize

\lohead{\textsc{register}}

% Definiere theindex-Environment komplett neu ohne reledmac
\makeatletter
\renewenvironment{theindex}{%
  \section*{\indexname}%
  \setlength{\parindent}{0pt}%
  \setlength{\parskip}{0pt plus 0.3pt}%
  \let\item\@idxitem
}{%
  \clearpage
}
\makeatother

\IfFileExists{\jobname-pw.ind}{\input{\jobname-pw.ind}}{}

\end{document}

      