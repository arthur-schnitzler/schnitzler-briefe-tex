%% latex-leseansicht-vorspann.tex
%% Vorspann für die Leseansicht.
%% Lädt die gemeinsame Datei latex-vorspann.tex mit nicht gesetztem Schalter.

\newif\ifkorrekturansicht
\korrekturansichtfalse

\input{../tex-inputs/latex-vorspann}


\section[Eduard Michael Kafka an Arthur Schnitzler, 7. 3. 1893]{L00187 Eduard Michael Kafka an Arthur Schnitzler, 7. 3. 1893}
\nopagebreak\mylabel{L00187v}
\rehead{ }\normalsize\beginnumbering\briefempfaengerindex{Schnitzler, Arthur@\textsc{Schnitzler, Arthur}!zzzKafka, Eduard Michael@\emph{von Eduard Michael Kafka}!1893-03-071@{7. 3. 1893}|(be}
\toendnotes[C]{\smallbreak\pagebreak[2]}
\correspDesc{Versand  durch Eduard Michael Kafka am 7. 3. 1893 in Kreiensen
\newline{}Erhalt  durch Arthur Schnitzler im Zeitraum [8. 3. 1893 –
            12. 3. 1893?] in Wien}\toendnotes[C]{\smallbreak}
\Standort{DLA, A:Schnitzler, HS.NZ85.1.3604.}
\physDesc{Brief, 1 Blatt, 1 Seite, 763 Zeichen
\newline{}Handschrift: schwarze Tinte, deutsche Kurrent
\newline{}Schnitzler: mit rotem Buntstift eine Unterstreichung }
\pstart
           \centering{}{\pb}\textcolor{gray}{\textbf{Wilh. Sundermeyer\pwindex{Sundermeyer, Wilhelm @\textsc{Sundermeyer, Wilhelm}|pw}}}\pend
           
\pstart
           \centering{}\textcolor{gray}{\textbf{Bahnhof Kreiensen\oindex{Bahnhof@\textbf{Bahnhof}, \emph{Bahnhofsgebäude}|pw}.}}\pend
           
\pstart
           \raggedleft{}\textcolor{gray}{\textbf{Kreiensen\oindex{Kreiensen@\textbf{Kreiensen}|pw}, den}}{ }7/III \textcolor{gray}{\textbf{189}}3.\pend
           
\pstart{}Lieber Schnitzler,\pend\vspace{0.5em}
\pstart
           bitte, wollen Sie die Güte haben, mir ein Ex. »Anatol\pwindex{Schnitzler, Arthur 15.\,5.\,1862 Wien – 21.\,10.\,1931 ebd.@\textsc{Schnitzler, Arthur} (15.\,5.\,1862 Wien – 21.\,10.\,1931 ebd.), \emph{Schriftsteller, Mediziner}!Anatol@\strich\emph{Anatol}|pw}« möglichſt umgehend nach München\oindex{München@\textbf{München}|pw} ,
          oder beſſer nach \uline{Mannheim\oindex{Mannheim@\textbf{Mannheim}, \emph{Hauptstadt}|pw}} (Pfälzer Hof\oindex{Pfälzer Hof@\textbf{Pfälzer Hof}, \emph{Hotel}|pw}){ }ſenden. –\pend
           
\pstart
           Es that mir{ }ſehr leid, Sie vor einigen Tagen, als ich über Brünn\oindex{Brünn@\textbf{Brünn}|pw} u. Prag\oindex{Prag@\textbf{Prag}, \emph{Land}|pw} , ein paar Stunden
          in Wien\oindex{Wien@\textbf{Wien}, \emph{Verwaltungsgebiet}|pw} weilte, nicht getroffen zu haben.\pend
           
\pstart
           Man erzählte mir Trauriges von Fels\pwindex{Fels, Friedrich Michael *~1864 Bad Dürkheim@\textsc{Fels, Friedrich Michael} (*~1864 Bad Dürkheim), \emph{Journalist}|pw} ; es war mir
          eine warme Freude, zu hören, daſs Sie{ }ſich{ }ſeiner nach Kräften annehmen. Bitte,{ }ſchreiben
          Sie mir doch gütigſt ein paar Zeilen, wie es ihm geht, – oder, lieber,{ }ſenden Sie mir
          seine Adreſſe; ich will, da ich ihm nun doch wol kaum mehr werde beſuchen können – vor
          meiner ſchwediſch\oindex{Schweden@\textbf{Schweden}|pw} - norwegiſchen\oindex{Norwegen@\textbf{Norwegen}|pw} Reiſe – gerne ein paar Zeilen an ihn richten.\pend
           
\pstart
           Leben Sie recht wohl, lieber Freund, u.{ }ſeien Sie herzlichſt gegrüßt\pend
           
\pstart
           von Ihrem getreuen {\\[\baselineskip]}\spacefill\mbox{EMKafka}\pend
           \leftskip=0em{}\selectlanguage{ngerman}\endnumbering\briefempfaengerindex{Schnitzler, Arthur@\textsc{Schnitzler, Arthur}!zzzKafka, Eduard Michael@\emph{von Eduard Michael Kafka}!1893-03-071@{7. 3. 1893}|)be}\mylabel{L00187h}  \newcommand{\dateiname}{L00187}\newcommand{\titel}{Eduard Michael Kafka an Arthur Schnitzler, 7. 3. 1893}\newcommand{\editorInnen}{Martin Anton Müller und Gerd-Hermann Susen}%% latex-leseansicht-abspann.tex
%% Abspann für die Leseansicht.
%% Der Schalter \ifkorrekturansicht ist bereits durch den Vorspann gesetzt.

%% latex-abspann.tex
%% Gemeinsamer Abspann für Korrekturansicht und Leseansicht.
%% Setzt den Schalter \ifkorrekturansicht voraus (gesetzt in den
%% einbindenden Dateien latex-korrekturansicht-abspann.tex bzw.
%% latex-leseansicht-abspann.tex).
%% ---------------------------------------------------------------

\normalsize

% Das esempio-Environment wird nur in der Leseansicht benötigt
\ifkorrekturansicht\else
\newenvironment{esempio}[3]%
{
    \vspace{1.5ex}
    \rlap{\underline{#1}}
    \par
    \setlength{\parindent}{0cm}
    \nopagebreak
    \leftskip=#2cm
    \rightskip=#3cm
}
{
    \par
}
\fi

\doendnotes{C}
\bigskip
\vfill

\clearpage

\footnotesize

\ifkorrekturansicht
  \lohead{\textsc{register}}
\fi

% theindex-Environment neu definieren ohne reledmac
\makeatletter
\renewenvironment{theindex}{%
  \ifkorrekturansicht
    \section*{\indexname}%
  \else
    \subsubsection*{Index der erwähnten Entitäten}%
  \fi
  \setlength{\parindent}{0pt}%
  \setlength{\parskip}{0pt plus 0.3pt}%
  \let\item\@idxitem
}{%
  \ifkorrekturansicht\clearpage\fi
}
\makeatother

\IfFileExists{\jobname-pw.ind}{\input{\jobname-pw.ind}}{}

% Quellenangabe nur in der Leseansicht
\ifkorrekturansicht\else
% Fallback-Definitionen, falls die .tex-Datei \titel etc. nicht gesetzt hat
\providecommand{\titel}{}
\providecommand{\editorInnen}{}
\providecommand{\dateiname}{\jobname}

\vspace{3cm}

\vfill

\footnotesize
\textsc{Quelle}: \titel. Herausgegeben von {\editorInnen}. In: \emph{Arthur Schnitzler: Briefwechsel mit Autorinnen und Autoren}.
 Digitale Edition, https://schnitzler-briefe.acdh.oeaw.ac.at/{\dateiname}.html (Stand \today)
\fi

\end{document}


