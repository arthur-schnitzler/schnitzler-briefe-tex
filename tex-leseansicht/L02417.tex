%% latex-korrekturansicht-vorspann.tex
%% Vorspann für die Korrekturansicht.
%% Lädt die gemeinsame Datei latex-vorspann.tex mit gesetztem Schalter.

\newif\ifkorrekturansicht
\korrekturansichttrue

\input{../tex-inputs/latex-vorspann}


\section[Thomas Mann an Arthur Schnitzler, 22. 10. 1924]{L02417 Thomas Mann an Arthur Schnitzler, 22. 10. 1924}
\nopagebreak\mylabel{L02417v}
\rehead{ }\normalsize\beginnumbering\briefempfaengerindex{Schnitzler, Arthur@\textsc{Schnitzler, Arthur}!zzzMann, Thomas@\emph{von Thomas Mann}!1924-10-221@{22. 10. 1924}|(be}
\toendnotes[C]{\smallbreak\pagebreak[2]}\Standort{CUL, Schnitzler, B 67.}
\physDesc{Brief, 1 Blatt, 2 Seiten, 1048 Zeichen
\newline{}Handschrift: schwarze Tinte, deutsche Kurrent
\newline{}Schnitzler: 1) mit Bleistift beschriftet: »\textsc{Thomas Ma{\geminationn}}«  2) mit Bleistift unterhalb des Brieftextes Antwortskizze:
                                    »Der Zumuthg den Zauberberg\pwindex{Zauberberg. Roman@\emph{Der Zauberberg. Roman}|pw} zu leſen{\dotstwo}{ }ſeh« 3) mit rotem Buntstift mehrere Unterstreichungen}
\buchAbdrucke{\weitereDrucke{1) \emph{Modern Austrian Literature}, Jg. 7 (1974) Nr. 1/2, S. 22.} \weitereDrucke{2) Hans-Ulrich Lindken: \emph{Arthur Schnitzler. Aspekte und Akzente. Materialien zu Leben
                        und Werk}. Frankfurt am Main, Bern, Göttingen: \emph{Peter Lang} 1984, S. 197.} }\toendnotes[C]{\smallbreak}
\pstart
           \raggedleft{}{\pb}\textsc{Sestri-Lev.\oindex{Sestri Levante@\textbf{Sestri Levante}, \emph{P.PPLA3}|pw}} den 22. X. 24.\pend
           
\pstart{}Verehrter Herr Dr. Schnitzler,\pend\vspace{0.5em}
\pstart
           es iſt mir ein Bedürfnis, Ihnen für die ſchönen Stunden zu danken, die ich hier mit
               der Lektüre Ihrer neuen Komödie\pwindex{Komoedie der Verfuehrung. In drei Akten@\emph{Komödie der Verführung. In drei Akten}|pwv} verbrachte, dieſes glänzenden, leidenſchaftlichen
               Geſellchaftsſtückes, das die Maße und Grenzen dieſer Gattung auf ſo feſtliche Weiſe
               weitert oder ſoll man ſagen: zerbricht. Ich kann es kaum erwarten, das Werk auf dem
               Theater zu ſehen, und doch bangt mir auch wieder davor. Werden unſere Schauſpieler
               eine »Konverſation« beherrſchen, die ſich jeden Augenblick zur Sprache des großen
               Dramas erhebt? Jedenfalls hoffe ich, daß das Münchener
                  Reſidenztheater\oindex{Residenztheater Muenchen@\textbf{Residenztheater München}, \emph{Theater (K.THE)}|pw} recht bald die Gelegenheit ergreift, zu {\pb}zeigen, was es kann.\pend
           
\pstart
           Nächſten Monat verſendet Fiſcher\orgindex{S. Fischer Verlag@S. Fischer Verlag|pw} meinen Roman
                  »Der Zauberberg\pwindex{Zauberberg. Roman@\emph{Der Zauberberg. Roman}|pw}«. Natürlich werde ich ihn
               bitten, Ihnen ein Exemplar zu ſchicken, aber Sie bitte ich, erblicken Sie keinerlei
               Zumutung darin! Ich denke ſehr zögernd über die Menſchenmöglichkeit des unförmigen
                  Opus\pwindex{Zauberberg. Roman@\emph{Der Zauberberg. Roman}|pwv} und entbinde jeden,
               dem ich es zugehen laſſe, feierlich von jeder Aeußerung darüber.\pend
           
\pstart
           Ihr ergebenſter{\\[\baselineskip]}\spacefill\mbox{Thomas Mann.}\pend
           \leftskip=0em{}\selectlanguage{ngerman}\endnumbering\briefempfaengerindex{Schnitzler, Arthur@\textsc{Schnitzler, Arthur}!zzzMann, Thomas@\emph{von Thomas Mann}!1924-10-221@{22. 10. 1924}|)be}\mylabel{L02417h}  \normalsize

\doendnotes{C}
\bigskip
\vfill

\clearpage

\footnotesize

\lohead{\textsc{register}}

% Definiere theindex-Environment komplett neu ohne reledmac
\makeatletter
\renewenvironment{theindex}{%
  \section*{\indexname}%
  \setlength{\parindent}{0pt}%
  \setlength{\parskip}{0pt plus 0.3pt}%
  \let\item\@idxitem
}{%
  \clearpage
}
\makeatother

\IfFileExists{\jobname-pw.ind}{\input{\jobname-pw.ind}}{}

\end{document}

      