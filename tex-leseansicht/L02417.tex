%% latex-leseansicht-vorspann.tex
%% Vorspann für die Leseansicht.
%% Lädt die gemeinsame Datei latex-vorspann.tex mit nicht gesetztem Schalter.

\newif\ifkorrekturansicht
\korrekturansichtfalse

\input{../tex-inputs/latex-vorspann}


         
         \renewcommand{\erwaehntePersonen}{Personen: Thomas Mann}
         \renewcommand{\erwaehnteInstitutionen}{Institutionen: S. Fischer Verlag}
         \renewcommand{\erwaehnteOrte}{Orte: Residenztheater München, Sestri Levante, Wien}
         \renewcommand{\erwaehnteWerke}{Werke: Der Zauberberg. Roman, Komödie der Verführung. In drei Akten}
               \section[Thomas Mann an Arthur Schnitzler, 22. 10. 1924]{ Thomas Mann an Arthur Schnitzler, 22. 10. 1924}\nopagebreak\mylabel{v}\rehead{ }\begin{ledgroupsized}[t]{13cm}\normalsize\beginnumbering\briefempfaengerindex{Schnitzler, Arthur@\textsc{Schnitzler, Arthur}!zzzMann, Thomas@\emph{von Thomas Mann}!1924-10-221@{22. 10. 1924}|(be} \toendnotes[C]{\smallbreak\pagebreak[2]} \Standort{CUL, Schnitzler, B 67.}
\physDesc{Brief, 1 Blatt, 2 Seiten, 1048 Zeichen
\newline{}Handschrift: schwarze Tinte, deutsche Kurrent
\newline{}Schnitzler: 1) mit Bleistift beschriftet: »\textsc{Thomas Ma{\geminationn}}«  2) mit Bleistift unterhalb des Brieftextes Antwortskizze:
                                    »Der Zumuthg den Zauberberg\pwindex{Mann, Thomas 06.06.1875 – 12.08.1955@\textsc{Mann, Thomas} (06.06.1875 – 12.08.1955), \emph{Schriftsteller}!Zauberberg. Roman1924@\strich\emph{Der Zauberberg. Roman} {[}1924{]}|pw} zu leſen{\dotstwo}{ }ſeh« 3) mit rotem Buntstift mehrere Unterstreichungen}\buchAbdrucke{\weitereDrucke{1) Hertha Krotkoff: \emph{Arthur Schnitzler – Thomas Mann: Briefe.} In: \emph{Modern Austrian Literature}, Jg. 7 (1974) Nr. 1/2, S. 22.} \weitereDrucke{2) Hans-Ulrich Lindken: \emph{Arthur Schnitzler. Aspekte und Akzente. Materialien zu Leben
                        und Werk}. Frankfurt am Main, Bern, Göttingen: \emph{Peter Lang} 1984, S. 197 (Europäische Hochschulschriften, Reihe 1, Deutsche Sprache und
                        Literatur, 754).} }\toendnotes[C]{\smallbreak}\pstart
           \raggedleft{}{\pb}\textsc{Sestri-Lev.\oindex{Sestri Levante@\textbf{Sestri Levante}|pw}} den 22. X. 24.\pend
           \pstart{}Verehrter Herr Dr. Schnitzler,\pend\pstart
           es iſt mir ein Bedürfnis, Ihnen für die ſchönen Stunden zu danken, die ich hier mit
               der Lektüre Ihrer neuen Komödie\pwindex{Schnitzler, Arthur 15.05.1862 – 21.10.1931@\textsc{Schnitzler, Arthur} (15.05.1862 – 21.10.1931), \emph{Schriftsteller, Mediziner}!Komoedie der Verfuehrung. In drei Akten1924@\strich\emph{Komödie der Verführung. In drei Akten} {[}1924{]}|pwv} verbrachte, dieſes glänzenden, leidenſchaftlichen
               Geſellchaftsſtückes, das die Maße und Grenzen dieſer Gattung auf ſo feſtliche Weiſe
               weitert oder ſoll man ſagen: zerbricht. Ich kann es kaum erwarten, das Werk auf dem
               Theater zu ſehen, und doch bangt mir auch wieder davor. Werden unſere Schauſpieler
               eine »Konverſation« beherrſchen, die ſich jeden Augenblick zur Sprache des großen
               Dramas erhebt? Jedenfalls hoffe ich, daß das Münchener
                  Reſidenztheater\oindex{Residenztheater Muenchen@\textbf{Residenztheater München}|pw} recht bald die Gelegenheit ergreift, zu {\pb}zeigen, was es kann.\pend
           \pstart
           Nächſten Monat verſendet Fiſcher\orgindex{S. Fischer Verlag@S. Fischer Verlag|pw} meinen Roman
                  »Der Zauberberg\pwindex{Mann, Thomas 06.06.1875 – 12.08.1955@\textsc{Mann, Thomas} (06.06.1875 – 12.08.1955), \emph{Schriftsteller}!Zauberberg. Roman1924@\strich\emph{Der Zauberberg. Roman} {[}1924{]}|pw}«. Natürlich werde ich ihn
               bitten, Ihnen ein Exemplar zu ſchicken, aber Sie bitte ich, erblicken Sie keinerlei
               Zumutung darin! Ich denke ſehr zögernd über die Menſchenmöglichkeit des unförmigen
                  Opus\pwindex{Mann, Thomas 06.06.1875 – 12.08.1955@\textsc{Mann, Thomas} (06.06.1875 – 12.08.1955), \emph{Schriftsteller}!Zauberberg. Roman1924@\strich\emph{Der Zauberberg. Roman} {[}1924{]}|pwv} und entbinde jeden,
               dem ich es zugehen laſſe, feierlich von jeder Aeußerung darüber.\pend
           \pstart
           Ihr ergebenſter{\\[\baselineskip]}\spacefill\mbox{Thomas Mann.}\pend
           \leftskip=0em{}
         
         \endnumbering\mylabel{h}\end{ledgroupsized}  \newcommand{\dateiname}{L02417}\newcommand{\titel}{Thomas Mann an Arthur Schnitzler, 22. 10. 1924}\newcommand{\editorInnen}{Martin Anton Müller und Gerd-Hermann Susen}%% latex-leseansicht-abspann.tex
%% Abspann für die Leseansicht.
%% Der Schalter \ifkorrekturansicht ist bereits durch den Vorspann gesetzt.

%% latex-abspann.tex
%% Gemeinsamer Abspann für Korrekturansicht und Leseansicht.
%% Setzt den Schalter \ifkorrekturansicht voraus (gesetzt in den
%% einbindenden Dateien latex-korrekturansicht-abspann.tex bzw.
%% latex-leseansicht-abspann.tex).
%% ---------------------------------------------------------------

\normalsize

% Das esempio-Environment wird nur in der Leseansicht benötigt
\ifkorrekturansicht\else
\newenvironment{esempio}[3]%
{
    \vspace{1.5ex}
    \rlap{\underline{#1}}
    \par
    \setlength{\parindent}{0cm}
    \nopagebreak
    \leftskip=#2cm
    \rightskip=#3cm
}
{
    \par
}
\fi

\doendnotes{C}
\bigskip
\vfill

\clearpage

\footnotesize

\ifkorrekturansicht
  \lohead{\textsc{register}}
\fi

% theindex-Environment neu definieren ohne reledmac
\makeatletter
\renewenvironment{theindex}{%
  \ifkorrekturansicht
    \section*{\indexname}%
  \else
    \subsubsection*{Index der erwähnten Entitäten}%
  \fi
  \setlength{\parindent}{0pt}%
  \setlength{\parskip}{0pt plus 0.3pt}%
  \let\item\@idxitem
}{%
  \ifkorrekturansicht\clearpage\fi
}
\makeatother

\IfFileExists{\jobname-pw.ind}{\input{\jobname-pw.ind}}{}

% Quellenangabe nur in der Leseansicht
\ifkorrekturansicht\else
% Fallback-Definitionen, falls die .tex-Datei \titel etc. nicht gesetzt hat
\providecommand{\titel}{}
\providecommand{\editorInnen}{}
\providecommand{\dateiname}{\jobname}

\vspace{3cm}

\vfill

\footnotesize
\textsc{Quelle}: \titel. Herausgegeben von {\editorInnen}. In: \emph{Arthur Schnitzler: Briefwechsel mit Autorinnen und Autoren}.
 Digitale Edition, https://schnitzler-briefe.acdh.oeaw.ac.at/{\dateiname}.html (Stand \today)
\fi

\end{document}


      