%% latex-leseansicht-vorspann.tex
%% Vorspann für die Leseansicht.
%% Lädt die gemeinsame Datei latex-vorspann.tex mit nicht gesetztem Schalter.

\newif\ifkorrekturansicht
\korrekturansichtfalse

\input{../tex-inputs/latex-vorspann}


\section[Sigmund Freud an Arthur Schnitzler, 24. 5. 1926]{L03890 Sigmund Freud an Arthur Schnitzler, 24. 5. 1926}
\nopagebreak\mylabel{L03890v}
\rehead{ }\normalsize\beginnumbering\briefempfaengerindex{Schnitzler, Arthur@\textsc{Schnitzler, Arthur}!zzzFreud, Sigmund@\emph{von Sigmund Freud}!1926-05-241@{24. 5. 1926}|(be}
\toendnotes[C]{\smallbreak\pagebreak[2]}
\correspDesc{Versand  durch Sigmund Freud am 24. 5. 1926 in Wien
\newline{}Erhalt  durch Arthur Schnitzler im Zeitraum [24. 5. 1926
                  – 27. 5. 1926?] in Wien}\toendnotes[C]{\smallbreak}
\Standort{Washington, DC, Library of Congress, Freud Archives, C41F8.}
\physDesc{Brief, Fotokopie, 1 Blatt, 2 Seiten, 1077 Zeichen
\newline{}Handschrift: schwarze Tinte, deutsche Kurrent
\newline{}Zusatz: Der Verbleib des Originals ist ungeklärt. Zum Zeitpunkt der
                                 ersten Edition 1955 befand es sich im Besitz von Heinrich Schnitzler\pwindex{Schnitzler, Heinrich 9.\,8.\,1902 Hinterbrühl – 12.\,7.\,1982 Wien@\textsc{Schnitzler, Heinrich} (9.\,8.\,1902 Hinterbrühl – 12.\,7.\,1982 Wien), \emph{Regisseur, Schauspieler}|pw}. }
\buchAbdrucke{\weitereDrucke{1) Sigmund Freud: \emph{Briefe an Arthur Schnitzler.}Herausgegeben von Henry Schnitzler In: \emph{Neue deutsche Rundschau}, Jg. 66 (Januar 1955) Nr. 1, S. 99–100.} \weitereDrucke{2) Sigmund Freud: \emph{Sigmund Freud Edition. Digitale historisch-kritische
                        Gesamtausgabe}. Herausgegeben von Christine Diercks, Arkadi Blatow und Elisabeth Skale. (2014–2025) \url{https://www.freudedition.net/briefe/freud-sigmund/schnitzler-arthur/1926/05/24}.} }\toendnotes[C]{\smallbreak}
\pstart
           \raggedleft{}{\pb}24. 5. 26\pend
           
\pstart
           \textcolor{gray}{\textbf{PROF. D\textsuperscript{R.} FREUD}}\hfill \textcolor{gray}{\textbf{WIEN IX., BERGGASSE 19\oindex{Wien@\textbf{Wien}!IX., Alsergrund@\textbf{IX., Alsergrund}!Berggasse 19@\textbf{Berggasse 19}, \emph{Wohngebäude}|pw}. }}\pend
           
\pstart\center{}Verehrteſter!\pend\vspace{0.5em}
\pstart
           Ich weiß nicht, ob Sie{ }ſchon zurück{ }ſind. Wenn nicht, werden dieſe Zeilen des Dankes
               für Ihren \label{K_L03890-1v}\edtext{Gruß von der See\oindex{Mittelmeer@\textbf{Mittelmeer}|pwv}\oindex{Atlantischer Ozean@\textbf{Atlantischer Ozean}|pwv}}{\lemma{\textnormal{\emph{Gruß von der See}}}\Cendnote{\textnormal{Die Postkarte ist nicht erhalten. Schnitzler unternahm gemeinsam mit seiner
                  Tochter Lili\pwindex{Cappellini, Lili 13.\,9.\,1909 Wien – 26.\,7.\,1928 Venedig@\textsc{Cappellini, Lili} (13.\,9.\,1909 Wien – 26.\,7.\,1928 Venedig)|pwk} eine Schiffsreise durch das westliche Mittelmeer\oindex{Mittelmeer@\textbf{Mittelmeer}|pwk} nach Las Palmas\oindex{Las Palmas de Gran Canaria@\textbf{Las Palmas de Gran Canaria}|pwk} bis nach Hamburg\oindex{Hamburg@\textbf{Hamburg}|pwk}. Sie begann mit dem Nachtzug am 15. 4. 1926 und ging am 19. 5. 1926 zu Ende, als
                  er in Berlin\oindex{Berlin@\textbf{Berlin}, \emph{Hauptstadt}|pwk} den Nachtzug nach Wien\oindex{Wien@\textbf{Wien}, \emph{Verwaltungsgebiet}|pwk} bestieg.}}}\label{K_L03890-1} Ihre Heimkehr erwarten.\pend
           
\pstart
           Das Ereignis gieng beſſer vorüber, als ich erwartet. Viel Herzlichkeit kein Mis{[}s{]}ton,
               dank vor allem der aufrichtigen Enthaltung der offiziellen Kreiſe. (Zu denen ja die
               sozialiſtiſche Wien\oindex{Wien@\textbf{Wien}, \emph{Verwaltungsgebiet}|pw}er Kommune nicht zält). Die
               Juden haben{ }ſich von allen Seiten und aller Orten mit Begeiſterung meiner Person
               bemächtigt, als ob ich ein gottesfürchtiger großer Rabbi wäre. Ich habe nichts
               dagegen, nachdem ich meine Stellung zum {\pb}Glauben
               unzweideutig \label{K_L03890-2v}\edtext{klargelegt\pwindex{Freud, Sigmund 6.\,5.\,1856 Pribor – 23.\,9.\,1939 London@\textsc{Freud, Sigmund} (6.\,5.\,1856 Pribor – 23.\,9.\,1939 London), \emph{Psychoanalytiker}!Zwangshandlungen und Religionsübungen@\strich\emph{Zwangshandlungen und Religionsübungen}|pwv}}{\lemma{\textnormal{\emph{klargelegt}}}\Cendnote{\textnormal{Mehrere Schriften kommen in Frage – vor allem verfasste Freud\pwindex{Freud, Sigmund 6.\,5.\,1856 Pribor – 23.\,9.\,1939 London@\textsc{Freud, Sigmund} (6.\,5.\,1856 Pribor – 23.\,9.\,1939 London), \emph{Psychoanalytiker}|pwk} noch einige weitere, in denen er über
                  die Rolle der Religion reflektierte. Er könnte sich hier auf seinen Aufsatz \emph{Zwangshandlungen und Religionsübungen}\pwindex{Freud, Sigmund 6.\,5.\,1856 Pribor – 23.\,9.\,1939 London@\textsc{Freud, Sigmund} (6.\,5.\,1856 Pribor – 23.\,9.\,1939 London), \emph{Psychoanalytiker}!Zwangshandlungen und Religionsübungen@\strich\emph{Zwangshandlungen und Religionsübungen}|pwk}
                  beziehen, der im Mai 1907 die erste Nummer der \emph{Zeitschrift für Religionspsychologie}\pwindex{Zeitschrift für Religionspsychologie@\emph{Zeitschrift für Religionspsychologie}|pwk} eröffnete
                     (Bd. 1, H. 1, S. 4–12). Insofern er auf Kenntnis
                  durch Schnitzler setzt, dürfte er sich
                  vielleicht auf \emph{Totem und Tabu}\pwindex{Freud, Sigmund 6.\,5.\,1856 Pribor – 23.\,9.\,1939 London@\textsc{Freud, Sigmund} (6.\,5.\,1856 Pribor – 23.\,9.\,1939 London), \emph{Psychoanalytiker}!Totem und Tabu@\strich\emph{Totem und Tabu}|pwk} beziehen.
               }}}\label{K_L03890-2} habe. Das Judentum bedeutet mir noch{ }ſehr viel affektiv.\pend
           
\pstart
           Mit dem 70ſten Geburtstag iſt doch ein Gefü{[}h{]}l großer Befreiung
               verbunden geweſen: Endlich hat man das Recht zu jenem Ausruf des Steinklopferhanns\pwindex{Anzengruber, Ludwig 29.\,11.\,1839 Wien – 10.\,12.\,1889 ebd.@\textsc{Anzengruber, Ludwig} (29.\,11.\,1839 Wien – 10.\,12.\,1889 ebd.), \emph{Schriftsteller}!Kreuzelschreiber. Bauerncomödie mit Gesang in drei Acten@\strich\emph{Die Kreuzelschreiber. Bauerncomödie mit Gesang in drei Acten}|pwv}: \label{K_L03890-3v}\edtext{Es kann
                  der nix g’schehen\pwindex{Anzengruber, Ludwig 29.\,11.\,1839 Wien – 10.\,12.\,1889 ebd.@\textsc{Anzengruber, Ludwig} (29.\,11.\,1839 Wien – 10.\,12.\,1889 ebd.), \emph{Schriftsteller}!Kreuzelschreiber. Bauerncomödie mit Gesang in drei Acten@\strich\emph{Die Kreuzelschreiber. Bauerncomödie mit Gesang in drei Acten}|pw}}{\lemma{\textnormal{\emph{Es … g’schehen}}}\Cendnote{\textnormal{dialektal,
                  eigentlich: Es kann dir nix gschehn. Mehrfach wiederholter Ausspruch der Figur des
                     Steinklopferhans\pwindex{Anzengruber, Ludwig 29.\,11.\,1839 Wien – 10.\,12.\,1889 ebd.@\textsc{Anzengruber, Ludwig} (29.\,11.\,1839 Wien – 10.\,12.\,1889 ebd.), \emph{Schriftsteller}!Kreuzelschreiber. Bauerncomödie mit Gesang in drei Acten@\strich\emph{Die Kreuzelschreiber. Bauerncomödie mit Gesang in drei Acten}|pwkv} in der
                  Bauernkomödie \emph{Die Kreuzelschreiber}\pwindex{Anzengruber, Ludwig 29.\,11.\,1839 Wien – 10.\,12.\,1889 ebd.@\textsc{Anzengruber, Ludwig} (29.\,11.\,1839 Wien – 10.\,12.\,1889 ebd.), \emph{Schriftsteller}!Kreuzelschreiber. Bauerncomödie mit Gesang in drei Acten@\strich\emph{Die Kreuzelschreiber. Bauerncomödie mit Gesang in drei Acten}|pwk}
                     (1872) von Ludwig
                     Anzengruber\pwindex{Anzengruber, Ludwig 29.\,11.\,1839 Wien – 10.\,12.\,1889 ebd.@\textsc{Anzengruber, Ludwig} (29.\,11.\,1839 Wien – 10.\,12.\,1889 ebd.), \emph{Schriftsteller}|pwk}, der zu einer verbreiteten Redewendung geworden war. Der hier
                  von Freud\pwindex{Freud, Sigmund 6.\,5.\,1856 Pribor – 23.\,9.\,1939 London@\textsc{Freud, Sigmund} (6.\,5.\,1856 Pribor – 23.\,9.\,1939 London), \emph{Psychoanalytiker}|pwk} hergestellte Bezug zur
                  Sterblichkeit entspricht der ursprünglichen Verwendung im Stück.}}}\label{K_L03890-3}!
               Sonderbar, denn die Za{[}h{]}l iſt doch nur eine Konvention.\pend
           
\pstart
           Am 15 Juni gehen wir\pwindex{Freud, Martha 26.\,7.\,1861 Hamburg – 2.\,11.\,1951 London@\textsc{Freud, Martha} (26.\,7.\,1861 Hamburg – 2.\,11.\,1951 London)|pwv} auf den Semmering\oindex{Semmering@\textbf{Semmering}, \emph{Verwaltungsgebiet}|pw}. Es{ }ſoll doch
               nicht ein Vorrecht des Kranken bleiben, Sie öfter zu{ }ſehen.\pend
           
\pstart
           In herzl Ergebenheit{\\[\baselineskip]} Ihr \spacefill\mbox{Freud}\pend
           \leftskip=0em{}
\pstart
           \noindent{}P. S. Über Ihre Traumnovelle\pwindex{Schnitzler, Arthur 15.\,5.\,1862 Wien – 21.\,10.\,1931 ebd.@\textsc{Schnitzler, Arthur} (15.\,5.\,1862 Wien – 21.\,10.\,1931 ebd.), \emph{Schriftsteller, Mediziner}!Traumnovelle@\strich\emph{Traumnovelle}|pw} habe ich mir
                  einige Gedanken gemacht.\pend
           \selectlanguage{ngerman}\endnumbering\briefempfaengerindex{Schnitzler, Arthur@\textsc{Schnitzler, Arthur}!zzzFreud, Sigmund@\emph{von Sigmund Freud}!1926-05-241@{24. 5. 1926}|)be}\mylabel{L03890h}
\begin{anhang}
\end{anhang}\newcommand{\dateiname}{L03890}\newcommand{\titel}{Sigmund Freud an Arthur Schnitzler, 24. 5. 1926}\newcommand{\editorInnen}{Selma Jahnke und Martin Anton Müller}%% latex-leseansicht-abspann.tex
%% Abspann für die Leseansicht.
%% Der Schalter \ifkorrekturansicht ist bereits durch den Vorspann gesetzt.

%% latex-abspann.tex
%% Gemeinsamer Abspann für Korrekturansicht und Leseansicht.
%% Setzt den Schalter \ifkorrekturansicht voraus (gesetzt in den
%% einbindenden Dateien latex-korrekturansicht-abspann.tex bzw.
%% latex-leseansicht-abspann.tex).
%% ---------------------------------------------------------------

\normalsize

% Das esempio-Environment wird nur in der Leseansicht benötigt
\ifkorrekturansicht\else
\newenvironment{esempio}[3]%
{
    \vspace{1.5ex}
    \rlap{\underline{#1}}
    \par
    \setlength{\parindent}{0cm}
    \nopagebreak
    \leftskip=#2cm
    \rightskip=#3cm
}
{
    \par
}
\fi

\doendnotes{C}
\bigskip
\vfill

\clearpage

\footnotesize

\ifkorrekturansicht
  \lohead{\textsc{register}}
\fi

% theindex-Environment neu definieren ohne reledmac
\makeatletter
\renewenvironment{theindex}{%
  \ifkorrekturansicht
    \section*{\indexname}%
  \else
    \subsubsection*{Index der erwähnten Entitäten}%
  \fi
  \setlength{\parindent}{0pt}%
  \setlength{\parskip}{0pt plus 0.3pt}%
  \let\item\@idxitem
}{%
  \ifkorrekturansicht\clearpage\fi
}
\makeatother

\IfFileExists{\jobname-pw.ind}{\input{\jobname-pw.ind}}{}

% Quellenangabe nur in der Leseansicht
\ifkorrekturansicht\else
% Fallback-Definitionen, falls die .tex-Datei \titel etc. nicht gesetzt hat
\providecommand{\titel}{}
\providecommand{\editorInnen}{}
\providecommand{\dateiname}{\jobname}

\vspace{3cm}

\vfill

\footnotesize
\textsc{Quelle}: \titel. Herausgegeben von {\editorInnen}. In: \emph{Arthur Schnitzler: Briefwechsel mit Autorinnen und Autoren}.
 Digitale Edition, https://schnitzler-briefe.acdh.oeaw.ac.at/{\dateiname}.html (Stand \today)
\fi

\end{document}


