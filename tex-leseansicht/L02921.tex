%% latex-leseansicht-vorspann.tex
%% Vorspann für die Leseansicht.
%% Lädt die gemeinsame Datei latex-vorspann.tex mit nicht gesetztem Schalter.

\newif\ifkorrekturansicht
\korrekturansichtfalse

\input{../tex-inputs/latex-vorspann}

\begin{center}
            \textcolor{red}{ENTWURF, NICHT FERTIG KORRIGIERT}
                      \end{center}
            
         
         \newcommand{\erwaehntePersonen}{Personen: Albert Bassermann, Alfred von Berger, Auguste Chlum, Marie Glümer, Georg Hirschfeld, Stella Hohenfels, Paul Lindau, Paul Schlenther, Irene Triesch}
         \newcommand{\erwaehnteInstitutionen}{Institutionen: Berliner Theater, Burgtheater, Deutsches Schauspielhaus in Hamburg, Frankfurter Stadttheater, Volkstheater}
         \newcommand{\erwaehnteOrte}{Orte: Alpen, Berlin, Dessauer Straße, Dolomiten, Frankfurt am Main, Hamburg, Schlesien, Vorarlberg, Wien}
         \newcommand{\erwaehnteWerke}{Werke: Der Schleier der Beatrice. Schauspiel in fünf Akten}
               \section[ Paul Goldmann an Arthur Schnitzler, 21. 6. {[}1900{]}]{ Paul Goldmann an Arthur Schnitzler, 21. 6. {[}1900{]}}\nopagebreak\mylabel{v}\rehead{ }\begin{ledgroupsized}[t]{13cm}\normalsize\beginnumbering \toendnotes[C]{\smallbreak\pagebreak[2]} \Standort{DLA, A:Schnitzler, HS.NZ85.1.3170.}
\physDesc{Brief, 1 Blatt, 4 Seiten
\newline{}Handschrift: blaue Tinte, deutsche Kurrent
\newline{}Schnitzler: 1) mit Bleistift das Jahr »{[}1{]}900« vermerkt  2) mit rotem Buntstift sechs Unterstreichungen}\toendnotes[C]{\smallbreak}\pstart
           \noindent{}{\pb}\textcolor{gray}{\textbf{DESSAUERSTRASSE 19}}\oindex{Dessauer Strasse@\textbf{Dessauer Straße}|pw}\hfill Berlin\oindex{Berlin@\textbf{Berlin}|pw}, 21. Juni.\pend
           \pstart
           \centering{}Mein lieber Freund,\pend
           \pstart
           \noindent{}Das iſt ein großes \label{K_L02921-1v}\edtext{Ärgerniß}{\lemma{\textnormal{\emph{Ärgerniß}}}\Cendnote{\textnormal{Bezug auf die Absage Paul Schlenther\pwindex{Schlenther, Paul 20.08.1854 – 30.04.1916@\textsc{Schlenther, Paul} (20.08.1854 – 30.04.1916), \emph{Schriftsteller, Kritiker, Theaterleiter}|pwk}s, \emph{Der
                     Schleier der Beatrice}\pwindex{Schnitzler, Arthur 15.05.1862 – 21.10.1931@\textsc{Schnitzler, Arthur} (15.05.1862 – 21.10.1931), \emph{Schriftsteller, Mediziner}!Schleier der Beatrice. Schauspiel in fuenf Akten1900-12-01@\strich\emph{Der Schleier der Beatrice. Schauspiel in fünf Akten} {[}1900-12-01{]}|pwk} am \emph{Burgtheater}\orgindex{Burgtheater@Burgtheater|pwk} doch nicht
                  aufzuführen (siehe Paul Goldmann an Arthur Schnitzler, 12. 11. [1899])}}}\label{K_L02921-1h}, und es thut mir unendlich leid, daß es Dir nicht erſpart geblieben iſt.
               Von Herrn \textsc{Schlenther\pwindex{Schlenther, Paul 20.08.1854 – 30.04.1916@\textsc{Schlenther, Paul} (20.08.1854 – 30.04.1916), \emph{Schriftsteller, Kritiker, Theaterleiter}|pw}} freilich überraſcht es mich nicht, und es iſt eigentlich viel natürlicher, daß
               er Dein Stück\pwindex{Schnitzler, Arthur 15.05.1862 – 21.10.1931@\textsc{Schnitzler, Arthur} (15.05.1862 – 21.10.1931), \emph{Schriftsteller, Mediziner}!Schleier der Beatrice. Schauspiel in fuenf Akten1900-12-01@\strich\emph{Der Schleier der Beatrice. Schauspiel in fünf Akten} {[}1900-12-01{]}|pwv} nicht aufführt,
               als daß er es aufführt. Dieſem nüchternen Berlin\oindex{Berlin@\textbf{Berlin}|pw}er\pwindex{Schlenther, Paul 20.08.1854 – 30.04.1916@\textsc{Schlenther, Paul} (20.08.1854 – 30.04.1916), \emph{Schriftsteller, Kritiker, Theaterleiter}|pwv} liegt Dein Werk\pwindex{Schnitzler, Arthur 15.05.1862 – 21.10.1931@\textsc{Schnitzler, Arthur} (15.05.1862 – 21.10.1931), \emph{Schriftsteller, Mediziner}!Schleier der Beatrice. Schauspiel in fuenf Akten1900-12-01@\strich\emph{Der Schleier der Beatrice. Schauspiel in fünf Akten} {[}1900-12-01{]}|pwv} mit all’ ſeinen poetiſchen Schönheiten
                  \strikeout{\textcolor{gray}{ja}} ſo fern! Ja, wenn es ſchleſi\oindex{Schlesien@\textbf{Schlesien}|pwv}ſche Bauern wären oder eine Berlin\oindex{Berlin@\textbf{Berlin}|pw}er jüdiſche Familie, wie in den »Milieuſtücken« von \textsc{Hirschfeld\pwindex{Hirschfeld, Georg 11.02.1873 – 17.01.1942@\textsc{Hirschfeld, Georg} (11.02.1873 – 17.01.1942), \emph{Schriftsteller}|pw}}! Wie ſoll ein \textsc{Schlenther\pwindex{Schlenther, Paul 20.08.1854 – 30.04.1916@\textsc{Schlenther, Paul} (20.08.1854 – 30.04.1916), \emph{Schriftsteller, Kritiker, Theaterleiter}|pw}} Deine »\textsc{Beatrice\pwindex{Schnitzler, Arthur 15.05.1862 – 21.10.1931@\textsc{Schnitzler, Arthur} (15.05.1862 – 21.10.1931), \emph{Schriftsteller, Mediziner}!Schleier der Beatrice. Schauspiel in fuenf Akten1900-12-01@\strich\emph{Der Schleier der Beatrice. Schauspiel in fünf Akten} {[}1900-12-01{]}|pw}}« verſtehen? Wenn Du ruhig nachdenkſt, wirſt Du {\pb}ſelbſt einſehen, daß es nicht möglich iſt. Dabei glaube ich noch nicht einmal, daß
               der Refüs ſich in erſter Linie gegen Dein Werk\pwindex{Schnitzler, Arthur 15.05.1862 – 21.10.1931@\textsc{Schnitzler, Arthur} (15.05.1862 – 21.10.1931), \emph{Schriftsteller, Mediziner}!Schleier der Beatrice. Schauspiel in fuenf Akten1900-12-01@\strich\emph{Der Schleier der Beatrice. Schauspiel in fünf Akten} {[}1900-12-01{]}|pwv} richtet. Es mag Mancherlei dabei mitgeſpielt haben: Der
               Herr Direktor\pwindex{Schlenther, Paul 20.08.1854 – 30.04.1916@\textsc{Schlenther, Paul} (20.08.1854 – 30.04.1916), \emph{Schriftsteller, Kritiker, Theaterleiter}|pwv} war zu faul,
               dieſes große Drama\pwindex{Schnitzler, Arthur 15.05.1862 – 21.10.1931@\textsc{Schnitzler, Arthur} (15.05.1862 – 21.10.1931), \emph{Schriftsteller, Mediziner}!Schleier der Beatrice. Schauspiel in fuenf Akten1900-12-01@\strich\emph{Der Schleier der Beatrice. Schauspiel in fünf Akten} {[}1900-12-01{]}|pwv}
               einzuſtudiren, was keine leichte Aufgabe iſt. Dann hat er ſich wohl auch vor den
               Koſten der Ausſtattung gefürchtet. Das darf er dem durch ſeine Wirthſchaft ohnehin
               ſchon ſo ſehr aus dem Gleichgewicht gebrachten Büdget des Burgtheater\orgindex{Burgtheater@Burgtheater|pw}s nicht mehr zumuthen. Und ſo weiter.\pend
           \pstart
           Du wirſt an Herrn \textsc{Schlenther\pwindex{Schlenther, Paul 20.08.1854 – 30.04.1916@\textsc{Schlenther, Paul} (20.08.1854 – 30.04.1916), \emph{Schriftsteller, Kritiker, Theaterleiter}|pw}} ſchon alle wünſchenswerthe Genugthuung {\pb}erleben. In dieſer Hinſicht bin ich ohne Sorge. Jetzt handelt es ſich nur darum,
               daß Dein Drama\pwindex{Schnitzler, Arthur 15.05.1862 – 21.10.1931@\textsc{Schnitzler, Arthur} (15.05.1862 – 21.10.1931), \emph{Schriftsteller, Mediziner}!Schleier der Beatrice. Schauspiel in fuenf Akten1900-12-01@\strich\emph{Der Schleier der Beatrice. Schauspiel in fünf Akten} {[}1900-12-01{]}|pwv} unter allen
               Umſtänden \label{K_L02921-2v}\edtext{aufgeführt}{\lemma{\textnormal{\emph{aufgeführt}}}\Cendnote{\textnormal{siehe Paul Goldmann an Arthur Schnitzler, 12. 11. [1899]}}}\label{K_L02921-2h} wird. Vom \label{K_L02921-76v}\edtext{Wien\oindex{Wien@\textbf{Wien}|pw}er Volkstheater\orgindex{Volkstheater@Volkstheater|pw}}{\lemma{\textnormal{\emph{Wiener Volkstheater}}}\Cendnote{\textnormal{\emph{Der Schleier der Beatrice}\pwindex{Schnitzler, Arthur 15.05.1862 – 21.10.1931@\textsc{Schnitzler, Arthur} (15.05.1862 – 21.10.1931), \emph{Schriftsteller, Mediziner}!Schleier der Beatrice. Schauspiel in fuenf Akten1900-12-01@\strich\emph{Der Schleier der Beatrice. Schauspiel in fünf Akten} {[}1900-12-01{]}|pwk} wurde, trotz
                  mehrmaliger Anläufe in den Jahren 1908 (vgl. A. S.: \emph{Tagebuch}, 25. 2. 1908 und 6. 3. 1900) und 1924 (vgl. A. S.: \emph{Tagebuch}, 29. 6. 1900, nicht am \emph{Volkstheater}\orgindex{Volkstheater@Volkstheater|pwk}
                  aufgeführt.}}}\label{K_L02921-76h} möchte ich dringend abrathen. Dort haben ſie zu plumpe Hände
               für das Stück\pwindex{Schnitzler, Arthur 15.05.1862 – 21.10.1931@\textsc{Schnitzler, Arthur} (15.05.1862 – 21.10.1931), \emph{Schriftsteller, Mediziner}!Schleier der Beatrice. Schauspiel in fuenf Akten1900-12-01@\strich\emph{Der Schleier der Beatrice. Schauspiel in fünf Akten} {[}1900-12-01{]}|pwv}. Aber, \strikeout{da} ich möchte Dir dringend das »Berliner Theater\orgindex{Berliner Theater@Berliner Theater|pw}« empfehlen. \textsc{Lindau\pwindex{Lindau, Paul 03.06.1839 – 31.01.1919@\textsc{Lindau, Paul} (03.06.1839 – 31.01.1919), \emph{Schriftsteller, Kritiker, Theaterleiter}|pw}} wird das Werk\pwindex{Schnitzler, Arthur 15.05.1862 – 21.10.1931@\textsc{Schnitzler, Arthur} (15.05.1862 – 21.10.1931), \emph{Schriftsteller, Mediziner}!Schleier der Beatrice. Schauspiel in fuenf Akten1900-12-01@\strich\emph{Der Schleier der Beatrice. Schauspiel in fünf Akten} {[}1900-12-01{]}|pwv} mit Liebe
               einſtudiren. Die Ausſtattung wird zwar dürftig ſein; aber \label{K_L02921-78v}\edtext{\textsc{Bassermann\pwindex{Bassermann, Albert 07.09.1867 – 15.05.1952@\textsc{Bassermann, Albert} (07.09.1867 – 15.05.1952), \emph{Schauspieler}|pw}}}{\lemma{\textnormal{\emph{Bassermann}}}\Cendnote{\textnormal{siehe Paul Goldmann an Arthur Schnitzler, [6.?] 2. 1903}}}\label{K_L02921-78h} wäre ein glänzender Vertreter für den Herzog\pwindex{Schnitzler, Arthur 15.05.1862 – 21.10.1931@\textsc{Schnitzler, Arthur} (15.05.1862 – 21.10.1931), \emph{Schriftsteller, Mediziner}!Schleier der Beatrice. Schauspiel in fuenf Akten1900-12-01@\strich\emph{Der Schleier der Beatrice. Schauspiel in fünf Akten} {[}1900-12-01{]}|pwv}. Auch \label{K_L02921-56v}\edtext{\textsc{Berger\pwindex{Berger, Alfred von 30.04.1853 – 24.08.1912@\textsc{Berger, Alfred von} (30.04.1853 – 24.08.1912), \emph{Schriftsteller, Journalist, Theaterleiter}|pw}}}{\lemma{\textnormal{\emph{Berger}}}\Cendnote{\textnormal{Alfred von Berger\pwindex{Berger, Alfred von 30.04.1853 – 24.08.1912@\textsc{Berger, Alfred von} (30.04.1853 – 24.08.1912), \emph{Schriftsteller, Journalist, Theaterleiter}|pwk} hatte \emph{Der Schleier der Beatrice}\pwindex{Schnitzler, Arthur 15.05.1862 – 21.10.1931@\textsc{Schnitzler, Arthur} (15.05.1862 – 21.10.1931), \emph{Schriftsteller, Mediziner}!Schleier der Beatrice. Schauspiel in fuenf Akten1900-12-01@\strich\emph{Der Schleier der Beatrice. Schauspiel in fünf Akten} {[}1900-12-01{]}|pwk} für das \emph{Deutsche Schauspielhaus in Hamburg}\orgindex{Deutsches Schauspielhaus in Hamburg@Deutsches Schauspielhaus in Hamburg|pwk} bereits abgelehnt (vgl. A. S.: \emph{Tagebuch}, 17. 2. 1900).}}}\label{K_L02921-56h} würde
               das Stück\pwindex{Schnitzler, Arthur 15.05.1862 – 21.10.1931@\textsc{Schnitzler, Arthur} (15.05.1862 – 21.10.1931), \emph{Schriftsteller, Mediziner}!Schleier der Beatrice. Schauspiel in fuenf Akten1900-12-01@\strich\emph{Der Schleier der Beatrice. Schauspiel in fünf Akten} {[}1900-12-01{]}|pwv} gewiß gern in ſeinem
               neuen Hamburg\oindex{Hamburg@\textbf{Hamburg}|pw}er TheaterXXXX ORGangabe fehlt aufführen, und die \textsc{Hohenfels\pwindex{Hohenfels, Stella 16.04.1854 – 21.02.1920@\textsc{Hohenfels, Stella} (16.04.1854 – 21.02.1920), \emph{Schauspielerin}|pw}} ſpielt vielleicht die \textsc{Beatrice\pwindex{Schnitzler, Arthur 15.05.1862 – 21.10.1931@\textsc{Schnitzler, Arthur} (15.05.1862 – 21.10.1931), \emph{Schriftsteller, Mediziner}!Schleier der Beatrice. Schauspiel in fuenf Akten1900-12-01@\strich\emph{Der Schleier der Beatrice. Schauspiel in fünf Akten} {[}1900-12-01{]}|pwv}}. Wirklich ſpielen kann dieſe Rolle {\pb}allerdings
               nur eine: die \label{K_L02921-31v}\edtext{\textsc{Triesch\pwindex{Triesch, Irene 13.04.1877 – 24.11.1964@\textsc{Triesch, Irene} (13.04.1877 – 24.11.1964), \emph{Schauspielerin}|pw}}}{\lemma{\textnormal{\emph{Triesch}}}\Cendnote{\textnormal{siehe Paul Goldmann an Arthur Schnitzler, 20. 2. 1900}}}\label{K_L02921-31h} in Frankfurt\oindex{Frankfurt am Main@\textbf{Frankfurt am Main}|pw}, und darum wäre es
               vielleicht auch nicht ſchlecht, das Stück\pwindex{Schnitzler, Arthur 15.05.1862 – 21.10.1931@\textsc{Schnitzler, Arthur} (15.05.1862 – 21.10.1931), \emph{Schriftsteller, Mediziner}!Schleier der Beatrice. Schauspiel in fuenf Akten1900-12-01@\strich\emph{Der Schleier der Beatrice. Schauspiel in fünf Akten} {[}1900-12-01{]}|pwv} zur Erſtaufführung nach Frankfurt\oindex{Frankfurt am Main@\textbf{Frankfurt am Main}|pw}\orgindex{Frankfurter Stadttheater@Frankfurter Stadttheater|pwv} zu geben.\pend
           \pstart
           Wenn Du willſt, gehe ich hier perſönlich zu \label{K_L02921-45v}\edtext{\textsc{Lindau\pwindex{Lindau, Paul 03.06.1839 – 31.01.1919@\textsc{Lindau, Paul} (03.06.1839 – 31.01.1919), \emph{Schriftsteller, Kritiker, Theaterleiter}|pw}}}{\lemma{\textnormal{\emph{Lindau}}}\Cendnote{\textnormal{Es sind keine Bemühungen um eine
                  Aufführung von \emph{Der Schleier der Beatrice}\pwindex{Schnitzler, Arthur 15.05.1862 – 21.10.1931@\textsc{Schnitzler, Arthur} (15.05.1862 – 21.10.1931), \emph{Schriftsteller, Mediziner}!Schleier der Beatrice. Schauspiel in fuenf Akten1900-12-01@\strich\emph{Der Schleier der Beatrice. Schauspiel in fünf Akten} {[}1900-12-01{]}|pwk} in
                     Paul Lindau\pwindex{Lindau, Paul 03.06.1839 – 31.01.1919@\textsc{Lindau, Paul} (03.06.1839 – 31.01.1919), \emph{Schriftsteller, Kritiker, Theaterleiter}|pwk}s \emph{Berliner Theater}\orgindex{Berliner Theater@Berliner Theater|pwk} bekannt.}}}\label{K_L02921-45h} hin.\pend
           \pstart
           Laß’ mich bald wiſſen, was Du beſchloſſen haſt, und ſchreib’ mir auch, wie \strikeout{es \textcolor{gray}{m}} es mit der \label{K_L02921-4v}\edtext{Alpen\oindex{Alpen@\textbf{Alpen}|pw}wanderung im Auguſt}{\lemma{\textnormal{\emph{Alpenwanderung im Auguſt}}}\Cendnote{\textnormal{siehe Paul Goldmann an Arthur Schnitzler, 16. 6. [1900]}}}\label{K_L02921-4h} ſteht. Die Dolomiten\oindex{Dolomiten@\textbf{Dolomiten}|pw} wären mir
               allerdings lieber als Vorarlberg\oindex{Vorarlberg@\textbf{Vorarlberg}|pw}.\pend
           \pstart
           Viele treue Grüße! {\\[\baselineskip]}Dein {\\[\baselineskip]}\spacefill\mbox{Paul Goldmnn}\pend
           \leftskip=0em{}\pstart
           \noindent{}Wenn Du die \label{K_L02921-5v}\edtext{Fräuleins \textsc{Glümer}\pwindex{Gluemer, Marie 03.07.1867 – 16.11.1925@\textsc{Glümer, Marie} (03.07.1867 – 16.11.1925), \emph{Schauspielerin}|pwv}\pwindex{Chlum, Auguste 16.03.1862 – 1956@\textsc{Chlum, Auguste} (16.03.1862 – 1956)|pwv} ſiehſt}{\lemma{\textnormal{\emph{Fräuleins Glümer ſiehſt}}}\Cendnote{\textnormal{Marie Glümer\pwindex{Gluemer, Marie 03.07.1867 – 16.11.1925@\textsc{Glümer, Marie} (03.07.1867 – 16.11.1925), \emph{Schauspielerin}|pwkv} traf Schnitzler\pwindex{Schnitzler, Arthur 15.05.1862 – 21.10.1931@\textsc{Schnitzler, Arthur} (15.05.1862 – 21.10.1931), \emph{Schriftsteller, Mediziner}|pwk} am 27. 6. 1900
                     wieder.}}}\label{K_L02921-5h}, ſo ſag’ ihnen, daß ich ihnen herzlichſt für ihre lieben Briefe
                  und Karten danke. Ich weiß leider ihre Adreſſe nicht.\pend
           
         
         \endnumbering\mylabel{h}\end{ledgroupsized}\begin{anhang}\end{anhang}\newcommand{\dateiname}{L02921}\newcommand{\titel}{Paul Goldmann an Arthur Schnitzler, 21. 6. [1900]}\newcommand{\editorInnen}{Martin Anton Müller und Laura Untner}%% latex-leseansicht-abspann.tex
%% Abspann für die Leseansicht.
%% Der Schalter \ifkorrekturansicht ist bereits durch den Vorspann gesetzt.

%% latex-abspann.tex
%% Gemeinsamer Abspann für Korrekturansicht und Leseansicht.
%% Setzt den Schalter \ifkorrekturansicht voraus (gesetzt in den
%% einbindenden Dateien latex-korrekturansicht-abspann.tex bzw.
%% latex-leseansicht-abspann.tex).
%% ---------------------------------------------------------------

\normalsize

% Das esempio-Environment wird nur in der Leseansicht benötigt
\ifkorrekturansicht\else
\newenvironment{esempio}[3]%
{
    \vspace{1.5ex}
    \rlap{\underline{#1}}
    \par
    \setlength{\parindent}{0cm}
    \nopagebreak
    \leftskip=#2cm
    \rightskip=#3cm
}
{
    \par
}
\fi

\doendnotes{C}
\bigskip
\vfill

\clearpage

\footnotesize

\ifkorrekturansicht
  \lohead{\textsc{register}}
\fi

% theindex-Environment neu definieren ohne reledmac
\makeatletter
\renewenvironment{theindex}{%
  \ifkorrekturansicht
    \section*{\indexname}%
  \else
    \subsubsection*{Index der erwähnten Entitäten}%
  \fi
  \setlength{\parindent}{0pt}%
  \setlength{\parskip}{0pt plus 0.3pt}%
  \let\item\@idxitem
}{%
  \ifkorrekturansicht\clearpage\fi
}
\makeatother

\IfFileExists{\jobname-pw.ind}{\input{\jobname-pw.ind}}{}

% Quellenangabe nur in der Leseansicht
\ifkorrekturansicht\else
% Fallback-Definitionen, falls die .tex-Datei \titel etc. nicht gesetzt hat
\providecommand{\titel}{}
\providecommand{\editorInnen}{}
\providecommand{\dateiname}{\jobname}

\vspace{3cm}

\vfill

\footnotesize
\textsc{Quelle}: \titel. Herausgegeben von {\editorInnen}. In: \emph{Arthur Schnitzler: Briefwechsel mit Autorinnen und Autoren}.
 Digitale Edition, https://schnitzler-briefe.acdh.oeaw.ac.at/{\dateiname}.html (Stand \today)
\fi

\end{document}


      