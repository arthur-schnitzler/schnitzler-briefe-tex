%% latex-korrekturansicht-vorspann.tex
%% Vorspann für die Korrekturansicht.
%% Lädt die gemeinsame Datei latex-vorspann.tex mit gesetztem Schalter.

\newif\ifkorrekturansicht
\korrekturansichttrue

\input{../tex-inputs/latex-vorspann}


\section[ Paul Goldmann an Arthur Schnitzler, 21. 6. {[}1900{]}]{L02921 Paul Goldmann an Arthur Schnitzler, 21. 6. {[}1900{]}}
\nopagebreak\mylabel{L02921v}
\rehead{ }\normalsize\beginnumbering\briefempfaengerindex{Schnitzler, Arthur@\textsc{Schnitzler, Arthur}!zzzGoldmann, Paul@\emph{von Paul Goldmann}!1900-06-213@{21. 6. {[}1900{]}}|(be}
\toendnotes[C]{\smallbreak\pagebreak[2]}\Standort{DLA, A:Schnitzler, HS.NZ85.1.3170.}
\physDesc{Brief, 1 Blatt, 4 Seiten, 2271 Zeichen
\newline{}Handschrift: blaue Tinte, deutsche Kurrent
\newline{}Schnitzler: 1) mit Bleistift das Jahr »900« vermerkt  2) mit rotem Buntstift sechs Unterstreichungen}\toendnotes[C]{\smallbreak}
\pstart
           \noindent{}
\pstart
           {\pb}\textcolor{gray}{\textbf{DESSAUERSTRASSE 19}}\oindex{Dessauer Strasse@\textbf{Dessauer Straße}, \emph{Straße (K.STR)}|pw}\pend
           
\pstart
           \raggedleft{}Berlin\oindex{Berlin@\textbf{Berlin}, \emph{P.PPLC}|pw}, 21. Juni.\pend
           \pend
           
\pstart
           \centering{}Mein lieber Freund,\pend
           
\pstart
           Das iſt ein großes \label{K_L02921-1v}\edtext{Ärgerniß}{\lemma{\textnormal{\emph{Ärgerniß}}}\Cendnote{\textnormal{Schnitzler war wegen der Absage Paul Schlenthers\pwindex{Schlenther, Paul 20.08.1854 – 30.04.1916@\textsc{Schlenther, Paul} (20.08.1854 – 30.04.1916), \emph{Schriftsteller/Schriftstellerin, Kritiker/Kritikerin, Theaterleiter/Theaterleiterin}|pwk}, \emph{Der Schleier der Beatrice}\pwindex{Schleier der Beatrice. Schauspiel in fuenf Akten@\emph{Der Schleier der Beatrice. Schauspiel in fünf Akten}|pwk} am \emph{Burgtheater}\orgindex{Burgtheater@Burgtheater|pwk} aufzuführen, verärgert. Vgl. Arthur Schnitzler an Richard Beer-Hofmann, 19. 6. 1900.}}}\label{K_L02921-1}, und es thut
               mir unendlich leid, daß es Dir nicht erſpart geblieben iſt. Von Herrn \textsc{Schlenther\pwindex{Schlenther, Paul 20.08.1854 – 30.04.1916@\textsc{Schlenther, Paul} (20.08.1854 – 30.04.1916), \emph{Schriftsteller/Schriftstellerin, Kritiker/Kritikerin, Theaterleiter/Theaterleiterin}|pw}} freilich überraſcht es mich nicht, und es iſt eigentlich viel natürlicher, daß
               er Dein Stück\pwindex{Schleier der Beatrice. Schauspiel in fuenf Akten@\emph{Der Schleier der Beatrice. Schauspiel in fünf Akten}|pwv} nicht aufführt,
               als daß er es aufführt. Dieſem nüchternen Berlin\oindex{Berlin@\textbf{Berlin}, \emph{P.PPLC}|pw}er\pwindex{Schlenther, Paul 20.08.1854 – 30.04.1916@\textsc{Schlenther, Paul} (20.08.1854 – 30.04.1916), \emph{Schriftsteller/Schriftstellerin, Kritiker/Kritikerin, Theaterleiter/Theaterleiterin}|pwv} liegt Dein Werk\pwindex{Schleier der Beatrice. Schauspiel in fuenf Akten@\emph{Der Schleier der Beatrice. Schauspiel in fünf Akten}|pwv} mit all’ ſeinen poetiſchen Schönheiten
                  \strikeout{\textcolor{gray}{ja}} ſo fern! Ja, wenn es ſchleſi\oindex{Schlesien@\textbf{Schlesien}, \emph{L.RGN}|pwv}ſche Bauern wären oder eine Berlin\oindex{Berlin@\textbf{Berlin}, \emph{P.PPLC}|pw}er jüdiſche Familie, wie in den »Milieuſtücken« von \textsc{Hirschfeld\pwindex{Hirschfeld, Georg 11.02.1873 – 17.01.1942@\textsc{Hirschfeld, Georg} (11.02.1873 – 17.01.1942), \emph{Schriftsteller/Schriftstellerin}|pw}}! Wie ſoll ein \textsc{Schlenther\pwindex{Schlenther, Paul 20.08.1854 – 30.04.1916@\textsc{Schlenther, Paul} (20.08.1854 – 30.04.1916), \emph{Schriftsteller/Schriftstellerin, Kritiker/Kritikerin, Theaterleiter/Theaterleiterin}|pw}} Deine »\textsc{Beatrice\pwindex{Schleier der Beatrice. Schauspiel in fuenf Akten@\emph{Der Schleier der Beatrice. Schauspiel in fünf Akten}|pw}}« verſtehen? Wenn Du ruhig nachdenkſt, wirſt Du {\pb}ſelbſt einſehen, daß es nicht möglich iſt. Dabei glaube ich noch nicht einmal, daß
               der Refüs ſich in erſter Linie gegen Dein Werk\pwindex{Schleier der Beatrice. Schauspiel in fuenf Akten@\emph{Der Schleier der Beatrice. Schauspiel in fünf Akten}|pwv} richtet. Es mag Mancherlei dabei mitgeſpielt haben: Der
               Herr Direktor\pwindex{Schlenther, Paul 20.08.1854 – 30.04.1916@\textsc{Schlenther, Paul} (20.08.1854 – 30.04.1916), \emph{Schriftsteller/Schriftstellerin, Kritiker/Kritikerin, Theaterleiter/Theaterleiterin}|pwv} war zu faul,
               dieſes große Drama\pwindex{Schleier der Beatrice. Schauspiel in fuenf Akten@\emph{Der Schleier der Beatrice. Schauspiel in fünf Akten}|pwv}
               einzuſtudiren, was keine leichte Aufgabe iſt. Dann hat er ſich wohl auch vor den
               Koſten der Ausſtattung gefürchtet. Das darf er dem durch ſeine Wirthſchaft ohnehin
               ſchon ſo ſehr aus dem Gleichgewicht gebrachten Büdget des Burgtheaters\orgindex{Burgtheater@Burgtheater|pw} nicht mehr zumuthen. Und ſo weiter.\pend
           
\pstart
           Du wirſt an Herrn \textsc{Schlenther\pwindex{Schlenther, Paul 20.08.1854 – 30.04.1916@\textsc{Schlenther, Paul} (20.08.1854 – 30.04.1916), \emph{Schriftsteller/Schriftstellerin, Kritiker/Kritikerin, Theaterleiter/Theaterleiterin}|pw}} ſchon alle wünſchenswerthe Genugthuung {\pb}erleben. In dieſer Hinſicht bin ich ohne Sorge. Jetzt handelt es ſich nur darum,
               daß Dein Drama\pwindex{Schleier der Beatrice. Schauspiel in fuenf Akten@\emph{Der Schleier der Beatrice. Schauspiel in fünf Akten}|pwv} unter allen
               Umſtänden \label{K_L02921-2v}\edtext{aufgeführt}{\lemma{\textnormal{\emph{aufgeführt}}}\Cendnote{\textnormal{Siehe Paul Goldmann an Arthur Schnitzler, 12. 11. [1899].
               }}}\label{K_L02921-2} wird. Vom \label{K_L02921-3v}\edtext{Wien\oindex{Wien@\textbf{Wien}, \emph{A.ADM2}|pw}er Volkstheater\orgindex{Volkstheater@Volkstheater|pw}}{\lemma{\textnormal{\emph{Wiener Volkstheater}}}\Cendnote{\textnormal{1908 gab es einen Anlauf, \emph{Der Schleier der Beatrice}\pwindex{Schleier der Beatrice. Schauspiel in fuenf Akten@\emph{Der Schleier der Beatrice. Schauspiel in fünf Akten}|pwk} am \emph{Volkstheater}\orgindex{Volkstheater@Volkstheater|pwk} aufzuführen. Dazu kam es jedoch nicht. Vgl. A. S.: \emph{Tagebuch}, 25. 2. 1908 und 6. 3. 1908.}}}\label{K_L02921-3}
               möchte ich dringend abrathen. Dort haben ſie zu plumpe Hände für das Stück\pwindex{Schleier der Beatrice. Schauspiel in fuenf Akten@\emph{Der Schleier der Beatrice. Schauspiel in fünf Akten}|pwv}. Aber\strikeout{,
                  da} ich möchte Dir dringend das »Berliner
                  Theater\orgindex{Berliner Theater@Berliner Theater|pw}« empfehlen. \textsc{Lindau\pwindex{Lindau, Paul 03.06.1839 – 31.01.1919@\textsc{Lindau, Paul} (03.06.1839 – 31.01.1919), \emph{Schriftsteller/Schriftstellerin, Kritiker/Kritikerin, Theaterleiter/Theaterleiterin}|pw}} wird das Werk\pwindex{Schleier der Beatrice. Schauspiel in fuenf Akten@\emph{Der Schleier der Beatrice. Schauspiel in fünf Akten}|pwv} mit Liebe
               einſtudiren. Die Ausſtattung wird zwar dürftig ſein; aber \label{K_L02921-4v}\edtext{\textsc{Bassermann\pwindex{Bassermann, Albert 07.09.1867 – 15.05.1952@\textsc{Bassermann, Albert} (07.09.1867 – 15.05.1952), \emph{Schauspieler/Schauspielerin}|pw}}}{\lemma{\textnormal{\emph{Bassermann}}}\Cendnote{\textnormal{Zur Wunschbesetzung Goldmanns\pwindex{Goldmann, Paul 31.01.1865 – 25.09.1935@\textsc{Goldmann, Paul} (31.01.1865 – 25.09.1935), \emph{Schriftsteller/Schriftstellerin, Journalist/Journalistin}|pwk} mit Albert
                     Bassermann\pwindex{Bassermann, Albert 07.09.1867 – 15.05.1952@\textsc{Bassermann, Albert} (07.09.1867 – 15.05.1952), \emph{Schauspieler/Schauspielerin}|pwk} und Irene Triesch\pwindex{Triesch, Irene 13.04.1877 – 24.11.1964@\textsc{Triesch, Irene} (13.04.1877 – 24.11.1964), \emph{Schauspieler/Schauspielerin}|pwk} kam es
                  erst 1904, siehe Paul Goldmann an Arthur Schnitzler, [6.?] 2. 1903.}}}\label{K_L02921-4} wäre ein glänzender Vertreter für den Herzog\pwindex{Schleier der Beatrice. Schauspiel in fuenf Akten@\emph{Der Schleier der Beatrice. Schauspiel in fünf Akten}|pwv}. Auch \label{K_L02921-5v}\edtext{\textsc{Berger\pwindex{Berger, Alfred von 30.04.1853 – 24.08.1912@\textsc{Berger, Alfred von} (30.04.1853 – 24.08.1912), \emph{Schriftsteller/Schriftstellerin, Journalist/Journalistin, Theaterleiter/Theaterleiterin}|pw}}}{\lemma{\textnormal{\emph{Berger}}}\Cendnote{\textnormal{Alfred von Berger\pwindex{Berger, Alfred von 30.04.1853 – 24.08.1912@\textsc{Berger, Alfred von} (30.04.1853 – 24.08.1912), \emph{Schriftsteller/Schriftstellerin, Journalist/Journalistin, Theaterleiter/Theaterleiterin}|pwk} hatte \emph{Der Schleier der Beatrice}\pwindex{Schleier der Beatrice. Schauspiel in fuenf Akten@\emph{Der Schleier der Beatrice. Schauspiel in fünf Akten}|pwk} für das \emph{Deutsche Schauspielhaus in Hamburg}\orgindex{Deutsches Schauspielhaus in Hamburg@Deutsches Schauspielhaus in Hamburg|pwk} bereits abgelehnt (vgl. A. S.: \emph{Tagebuch}, 17. 2. 1900).}}}\label{K_L02921-5} würde
               das Stück\pwindex{Schleier der Beatrice. Schauspiel in fuenf Akten@\emph{Der Schleier der Beatrice. Schauspiel in fünf Akten}|pwv} gewiß gern in ſeinem
               neuen Hamburg\oindex{Hamburg@\textbf{Hamburg}, \emph{P.PPLA}|pw}er Theater\orgindex{Deutsches Schauspielhaus in Hamburg@Deutsches Schauspielhaus in Hamburg|pwv} aufführen, und die \textsc{Hohenfels\pwindex{Hohenfels, Stella 16.04.1857 – 21.02.1920@\textsc{Hohenfels, Stella} (16.04.1857 – 21.02.1920), \emph{Schauspieler/Schauspielerin}|pw}} ſpielt vielleicht die \textsc{Beatrice\pwindex{Schleier der Beatrice. Schauspiel in fuenf Akten@\emph{Der Schleier der Beatrice. Schauspiel in fünf Akten}|pwv}}. Wirklich ſpielen kann dieſe Rolle {\pb}allerdings
               nur eine: die \label{K_L02921-6v}\edtext{\textsc{Triesch\pwindex{Triesch, Irene 13.04.1877 – 24.11.1964@\textsc{Triesch, Irene} (13.04.1877 – 24.11.1964), \emph{Schauspieler/Schauspielerin}|pw}}}{\lemma{\textnormal{\emph{Triesch}}}\Cendnote{\textnormal{Vgl. Paul Goldmann an Arthur Schnitzler, 20. 2. 1900.
               }}}\label{K_L02921-6} in Frankfurt\oindex{Frankfurt am Main@\textbf{Frankfurt am Main}, \emph{P.PPLA3}|pw}, und darum wäre es
               vielleicht auch nicht ſchlecht, das Stück\pwindex{Schleier der Beatrice. Schauspiel in fuenf Akten@\emph{Der Schleier der Beatrice. Schauspiel in fünf Akten}|pwv} zur Erſtaufführung nach Frankfurt\oindex{Frankfurt am Main@\textbf{Frankfurt am Main}, \emph{P.PPLA3}|pw}\orgindex{Frankfurter Stadttheater@Frankfurter Stadttheater|pwv} zu geben.\pend
           
\pstart
           Wenn Du willſt, gehe ich hier perſönlich zu \label{K_L02921-7v}\edtext{\textsc{Lindau\pwindex{Lindau, Paul 03.06.1839 – 31.01.1919@\textsc{Lindau, Paul} (03.06.1839 – 31.01.1919), \emph{Schriftsteller/Schriftstellerin, Kritiker/Kritikerin, Theaterleiter/Theaterleiterin}|pw}}}{\lemma{\textnormal{\emph{Lindau}}}\Cendnote{\textnormal{Es sind keine Bemühungen um eine
                  Aufführung von \emph{Der Schleier der Beatrice}\pwindex{Schleier der Beatrice. Schauspiel in fuenf Akten@\emph{Der Schleier der Beatrice. Schauspiel in fünf Akten}|pwk} in
                     Paul Lindaus\pwindex{Lindau, Paul 03.06.1839 – 31.01.1919@\textsc{Lindau, Paul} (03.06.1839 – 31.01.1919), \emph{Schriftsteller/Schriftstellerin, Kritiker/Kritikerin, Theaterleiter/Theaterleiterin}|pwk}{ }\emph{Berliner Theater}\orgindex{Berliner Theater@Berliner Theater|pwk} bekannt.}}}\label{K_L02921-7} hin.\pend
           
\pstart
           Laß’ mich bald wiſſen, was Du beſchloſſen haſt, und ſchreib’ mir auch, wie \strikeout{es \textcolor{gray}{m}} es mit der \label{K_L02921-8v}\edtext{Alpen\oindex{Alpen@\textbf{Alpen}, \emph{kein passender Code gefunden}|pw}wanderung im Auguſt}{\lemma{\textnormal{\emph{Alpenwanderung im Auguſt}}}\Cendnote{\textnormal{Siehe Paul Goldmann an Arthur Schnitzler, 16. 6. [1900].
               }}}\label{K_L02921-8} ſteht. Die \textsc{Dolomiten}\oindex{Dolomiten@\textbf{Dolomiten}, \emph{Gebirge (N.GBR)}|pw} wären mir allerdings lieber als Vorarlberg\oindex{Vorarlberg@\textbf{Vorarlberg}, \emph{Teil eines Landes (A.LNDX)}|pw}.\pend
           
\pstart
           Viele treue Grüße! {\\[\baselineskip]}Dein {\\[\baselineskip]}\spacefill\mbox{Paul Goldmnn}\pend
           \leftskip=0em{}
\pstart
           \noindent{}Wenn Du die \label{K_L02921-9v}\edtext{Fräuleins \textsc{Glümer}\pwindex{Gluemer, Marie 03.07.1867 – 16.11.1925@\textsc{Glümer, Marie} (03.07.1867 – 16.11.1925), \emph{Schauspieler/Schauspielerin}|pwv}\pwindex{Gluemer, Auguste 1862-03-16 – 1956@\textsc{Glümer, Auguste} (1862-03-16 – 1956), \emph{Lehrer/Lehrerin}|pwv} ſiehſt}{\lemma{\textnormal{\emph{Fräuleins Glümer ſiehſt}}}\Cendnote{\textnormal{Marie Glümer\pwindex{Gluemer, Marie 03.07.1867 – 16.11.1925@\textsc{Glümer, Marie} (03.07.1867 – 16.11.1925), \emph{Schauspieler/Schauspielerin}|pwk} und Auguste Chlum\pwindex{Gluemer, Auguste 1862-03-16 – 1956@\textsc{Glümer, Auguste} (1862-03-16 – 1956), \emph{Lehrer/Lehrerin}|pwk} waren nach der Entlassung Glümers\pwindex{Gluemer, Marie 03.07.1867 – 16.11.1925@\textsc{Glümer, Marie} (03.07.1867 – 16.11.1925), \emph{Schauspieler/Schauspielerin}|pwk} aus Berlin\oindex{Berlin@\textbf{Berlin}, \emph{P.PPLC}|pwk} (siehe Paul Goldmann an Arthur Schnitzler, 31. 5. [1900]) nach Wien\oindex{Wien@\textbf{Wien}, \emph{A.ADM2}|pwk} gekommen, wo
                        Schnitzler sie regelmäßig
                  traf.}}}\label{K_L02921-9}, ſo ſag’ ihnen, daß ich ihnen herzlichſt für ihre lieben Briefe und
                  Karten danke. Ich weiß leider ihre Adreſſe nicht.\pend
           \selectlanguage{ngerman}\endnumbering\briefempfaengerindex{Schnitzler, Arthur@\textsc{Schnitzler, Arthur}!zzzGoldmann, Paul@\emph{von Paul Goldmann}!1900-06-213@{21. 6. {[}1900{]}}|)be}\mylabel{L02921h}  \normalsize

\doendnotes{C}
\bigskip
\vfill

\clearpage

\footnotesize

\lohead{\textsc{register}}

% Definiere theindex-Environment komplett neu ohne reledmac
\makeatletter
\renewenvironment{theindex}{%
  \section*{\indexname}%
  \setlength{\parindent}{0pt}%
  \setlength{\parskip}{0pt plus 0.3pt}%
  \let\item\@idxitem
}{%
  \clearpage
}
\makeatother

\IfFileExists{\jobname-pw.ind}{\input{\jobname-pw.ind}}{}

\end{document}

      