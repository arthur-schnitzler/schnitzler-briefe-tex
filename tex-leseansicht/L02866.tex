%% latex-korrekturansicht-vorspann.tex
%% Vorspann für die Korrekturansicht.
%% Lädt die gemeinsame Datei latex-vorspann.tex mit gesetztem Schalter.

\newif\ifkorrekturansicht
\korrekturansichttrue

\input{../tex-inputs/latex-vorspann}


\section[Otto Brahm, Gerhart Hauptmann und Margarete Marschalk an Arthur Schnitzler, 21. 6. 1903]{L02866 Otto Brahm, Gerhart Hauptmann und Margarete Marschalk an Arthur
               Schnitzler, 21. 6. 1903}
\nopagebreak\mylabel{L02866v}
\rehead{ }\normalsize\beginnumbering\briefempfaengerindex{Schnitzler, Arthur@\textsc{Schnitzler, Arthur}!zzzHauptmann, Margarete@\emph{von Margarete Hauptmann}!1903-06-211@{21. 6. 1903}|(be}\briefempfaengerindex{Schnitzler, Arthur@\textsc{Schnitzler, Arthur}!zzzHauptmann, Gerhart@\emph{von Gerhart Hauptmann}!1903-06-211@{21. 6. 1903}|(be}\briefempfaengerindex{Schnitzler, Arthur@\textsc{Schnitzler, Arthur}!zzzBrahm, Otto@\emph{von Otto Brahm}!1903-06-211@{21. 6. 1903}|(be}
\toendnotes[C]{\smallbreak\pagebreak[2]}\Standort{CUL, Schnitzler, B 16.}
\physDesc{Bildpostkarte, 282 Zeichen
\newline{}Handschrift Otto Brahm: Bleistift, lateinische Kurrent
\newline{}Handschrift Gerhart Hauptmann: Bleistift, lateinische Kurrent
\newline{}Handschrift Margarete Hauptmann: Bleistift, lateinische Kurrent
\newline{}Versand: 1) Stempel: »\nobreak{}\oindex{Snĕžne jámy@\textbf{Snĕžne jámy}, \emph{T.MT}|pwk}Schneegrubenbaude Richard
                                       Gerlich, 21. 6. 1903\nobreak{}«.   2) Stempel: »\nobreak{}\oindex{Szklarska Poręba@\textbf{Szklarska Poręba}, \emph{P.PPL}|pwk}Schreiberhau, 21. 6. 03, 6–7 N.\nobreak{}«.  3) Stempel: »\nobreak{}\oindex{IX., Alsergrund@\textbf{IX., Alsergrund}, \emph{A.ADM3}|pwk}9/3 Wien 72, 22. 6. 03, 7. N, Bestellt\nobreak{}«. 
\newline{}Schnitzler: mit Bleistift datiert: »22/6 90\textcolor{gray}{3}« 
\newline{}Ordnung: mit Bleistift nummeriert: »132« }
\buchAbdrucke{\weitereDrucke{\emph{Der Briefwechsel Arthur Schnitzler – Otto Brahm. Vollständige
                        Ausgabe}. Tübingen: \emph{Niemeyer} 1975, S. 141–142.} }\toendnotes[C]{\smallbreak}\pstart{}{\pb}Herrn Dr Arthur
                  Schnitzler\pend{}\pstart{}Wien IX.\oindex{IX., Alsergrund@\textbf{IX., Alsergrund}, \emph{A.ADM3}|pw}\pend{}\pstart{}Frankgasse 1\oindex{Frankgasse 1@\textbf{Frankgasse 1}, \emph{Wohngebäude (K.WHS)}|pw}.\pend{}{\bigskip}
\pstart
           \noindent{}\centering{}{\pb}\textcolor{gray}{\textbf{Schneegrube mit Baude\oindex{Snĕžne jámy@\textbf{Snĕžne jámy}, \emph{T.MT}|pw}.}}\pend
           \vspace{1em}
\pstart
           \noindent{}{\pb}Grüsse an Sie und die
               liebenswerthe \label{K_L02866-1v}\edtext{Comödie}{\lemma{\textnormal{\emph{Comödie}}}\Cendnote{\textnormal{unklar; möglicherweise die Komödie, über
                  die Schnitzler{ }Hofmannsthal\pwindex{Hofmannsthal, Hugo von 1874-02-01 – 1929-07-15@\textsc{Hofmannsthal, Hugo von} (1874-02-01 – 1929-07-15), \emph{Schriftsteller/Schriftstellerin}|pwk} am 26. 6. 1903 schrieb.}}}\label{K_L02866-1}\pend
           \pstart \spacefill\mbox{OBrahm}\pend{}\selectlanguage{ngerman}\vspace{1em}
\pstart
           \noindent{}{[}hs. :{]} Was freut Sie nur, lieber Herr Schnitzler? Eine Frage,
               die ich von Ihnen mal beantwortet haben möchte. Beim nächsten Wiedersehen!\pend
           \pstart Ihr \spacefill\mbox{Gerhart Hauptmann}\pend{}\selectlanguage{ngerman}\vspace{1em}
\pstart
           \noindent{}{[}hs. :{]} Freundlicher Gruss\textcolor{gray}{!}\pend
           \pstart \spacefill\mbox{Margarete Marschalk}\pend{}\selectlanguage{ngerman}\endnumbering\briefempfaengerindex{Schnitzler, Arthur@\textsc{Schnitzler, Arthur}!zzzHauptmann, Margarete@\emph{von Margarete Hauptmann}!1903-06-211@{21. 6. 1903}|)be}\briefempfaengerindex{Schnitzler, Arthur@\textsc{Schnitzler, Arthur}!zzzHauptmann, Gerhart@\emph{von Gerhart Hauptmann}!1903-06-211@{21. 6. 1903}|)be}\briefempfaengerindex{Schnitzler, Arthur@\textsc{Schnitzler, Arthur}!zzzBrahm, Otto@\emph{von Otto Brahm}!1903-06-211@{21. 6. 1903}|)be}\mylabel{L02866h}  \normalsize

\doendnotes{C}
\bigskip
\vfill

\clearpage

\footnotesize

\lohead{\textsc{register}}

% Definiere theindex-Environment komplett neu ohne reledmac
\makeatletter
\renewenvironment{theindex}{%
  \section*{\indexname}%
  \setlength{\parindent}{0pt}%
  \setlength{\parskip}{0pt plus 0.3pt}%
  \let\item\@idxitem
}{%
  \clearpage
}
\makeatother

\IfFileExists{\jobname-pw.ind}{\input{\jobname-pw.ind}}{}

\end{document}

      