%% latex-leseansicht-vorspann.tex
%% Vorspann für die Leseansicht.
%% Lädt die gemeinsame Datei latex-vorspann.tex mit nicht gesetztem Schalter.

\newif\ifkorrekturansicht
\korrekturansichtfalse

\input{../tex-inputs/latex-vorspann}


         
         \renewcommand{\erwaehntePersonen}{Personen: Otto Brahm, Gerhart Hauptmann, Margarete Hauptmann, Hugo von Hofmannsthal}
         \renewcommand{\erwaehnteOrte}{Orte: Frankgasse 1, IX., Alsergrund, Snĕžne jámy, Szklarska Poręba, Wien}
         \renewcommand{\erwaehnteWerke}{}
               \section[Otto Brahm, Gerhart Hauptmann und Margarete Marschalk an Arthur Schnitzler, 21. 6. 1903]{ Otto Brahm, Gerhart Hauptmann und Margarete Marschalk an Arthur
               Schnitzler, 21. 6. 1903}\nopagebreak\mylabel{v}\rehead{ }\begin{ledgroupsized}[t]{13cm}\normalsize\beginnumbering\briefempfaengerindex{Schnitzler, Arthur@\textsc{Schnitzler, Arthur}!zzzHauptmann, Margarete@\emph{von Margarete Hauptmann}!1903-06-211@{21. 6. 1903}|(be}\briefempfaengerindex{Schnitzler, Arthur@\textsc{Schnitzler, Arthur}!zzzHauptmann, Gerhart@\emph{von Gerhart Hauptmann}!1903-06-211@{21. 6. 1903}|(be}\briefempfaengerindex{Schnitzler, Arthur@\textsc{Schnitzler, Arthur}!zzzBrahm, Otto@\emph{von Otto Brahm}!1903-06-211@{21. 6. 1903}|(be} \toendnotes[C]{\smallbreak\pagebreak[2]} \Standort{CUL, Schnitzler, B 16.}
\physDesc{Bildpostkarte, 282 Zeichen
\newline{}Handschrift Otto Brahm: Bleistift, lateinische Kurrent\newline{}Handschrift Gerhart Hauptmann: Bleistift, lateinische Kurrent\newline{}Handschrift Margarete Hauptmann: Bleistift, lateinische Kurrent
\newline{}Versand: 1) Stempel: »\nobreak{}\oindex{Snĕžne jámy@\textbf{Snĕžne jámy}|pwk}Schneegrubenbaude Richard
                                       Gerlich, 21. 6. 1903\nobreak{}«.   2) Stempel: »\nobreak{}\oindex{Szklarska Poręba@\textbf{Szklarska Poręba}|pwk}Schreiberhau, 21. 6. 03, 6–7 N.\nobreak{}«.  3) Stempel: »\nobreak{}\oindex{IX., Alsergrund@\textbf{IX., Alsergrund}|pwk}9/3 Wien 72, 22. 6. 03, 7. N, Bestellt\nobreak{}«. 
\newline{}Schnitzler: mit Bleistift datiert: »22/6 90\textcolor{gray}{3}« 
\newline{}Ordnung: mit Bleistift nummeriert: »132« }\buchAbdrucke{\weitereDrucke{\emph{Der Briefwechsel Arthur Schnitzler – Otto Brahm. Vollständige
                        Ausgabe}. Herausgegeben, eingeleitet und erläutert von Oskar Seidlin. Tübingen: \emph{Niemeyer} 1975, S. 141–142.} }\toendnotes[C]{\smallbreak}\pstart{}{\pb}Herrn Dr Arthur Schnitzler\pend{}\pstart{}Wien IX.\oindex{IX., Alsergrund@\textbf{IX., Alsergrund}|pw}\pend{}\pstart{}Frankgasse 1\oindex{Frankgasse 1@\textbf{Frankgasse 1}|pw}.\pend{}{\bigskip}\pstart
           \noindent{}\centering{}{\pb}\textcolor{gray}{\textbf{Schneegrube mit Baude\oindex{Snĕžne jámy@\textbf{Snĕžne jámy}|pw}.}}\pend
           \pstart
           Grüsse an Sie und die liebenswerthe \label{K_L02866-1v}\edtext{Comödie}{\lemma{\textnormal{\emph{Comödie}}}\Cendnote{\textnormal{unklar; möglicherweise die Komödie, über die Schnitzler\pwindex{Schnitzler, Arthur 15.05.1862 – 21.10.1931@\textsc{Schnitzler, Arthur} (15.05.1862 – 21.10.1931), \emph{Schriftsteller, Mediziner}|pwk}{ }Hofmannsthal\pwindex{Hofmannsthal, Hugo von 1874-02-01 – 1929-07-15@\textsc{Hofmannsthal, Hugo von} (1874-02-01 – 1929-07-15), \emph{Schriftsteller}|pwk} am 26. 6. 1903
                     schrieb.}}}\label{K_L02866-1h}\pend
           \pstart \spacefill\mbox{OBrahm}\pend{}\pstart
           \noindent{}{[}hs. Gerhart Hauptmann:{]} Was freut Sie nur, lieber Herr Schnitzler? Eine Frage, die ich von Ihnen mal
               beantwortet haben möchte. Beim nächsten Wiedersehen!\pend
           \pstart Ihr \spacefill\mbox{Gerhart Hauptmann}\pend{}\pstart
           \noindent{}{[}hs. Margarete Hauptmann:{]} Freundlicher Gruss\textcolor{gray}{!}\pend
           \pstart \spacefill\mbox{Margarete Marschalk}\pend{}
         
         \endnumbering\mylabel{h}\end{ledgroupsized}  \newcommand{\dateiname}{L02866}\newcommand{\titel}{Otto Brahm, Gerhart Hauptmann und Margarete Marschalk an Arthur Schnitzler, 21. 6. 1903}\newcommand{\editorInnen}{ Martin Anton Müller und Gerd-Hermann Susen}%% latex-leseansicht-abspann.tex
%% Abspann für die Leseansicht.
%% Der Schalter \ifkorrekturansicht ist bereits durch den Vorspann gesetzt.

%% latex-abspann.tex
%% Gemeinsamer Abspann für Korrekturansicht und Leseansicht.
%% Setzt den Schalter \ifkorrekturansicht voraus (gesetzt in den
%% einbindenden Dateien latex-korrekturansicht-abspann.tex bzw.
%% latex-leseansicht-abspann.tex).
%% ---------------------------------------------------------------

\normalsize

% Das esempio-Environment wird nur in der Leseansicht benötigt
\ifkorrekturansicht\else
\newenvironment{esempio}[3]%
{
    \vspace{1.5ex}
    \rlap{\underline{#1}}
    \par
    \setlength{\parindent}{0cm}
    \nopagebreak
    \leftskip=#2cm
    \rightskip=#3cm
}
{
    \par
}
\fi

\doendnotes{C}
\bigskip
\vfill

\clearpage

\footnotesize

\ifkorrekturansicht
  \lohead{\textsc{register}}
\fi

% theindex-Environment neu definieren ohne reledmac
\makeatletter
\renewenvironment{theindex}{%
  \ifkorrekturansicht
    \section*{\indexname}%
  \else
    \subsubsection*{Index der erwähnten Entitäten}%
  \fi
  \setlength{\parindent}{0pt}%
  \setlength{\parskip}{0pt plus 0.3pt}%
  \let\item\@idxitem
}{%
  \ifkorrekturansicht\clearpage\fi
}
\makeatother

\IfFileExists{\jobname-pw.ind}{\input{\jobname-pw.ind}}{}

% Quellenangabe nur in der Leseansicht
\ifkorrekturansicht\else
% Fallback-Definitionen, falls die .tex-Datei \titel etc. nicht gesetzt hat
\providecommand{\titel}{}
\providecommand{\editorInnen}{}
\providecommand{\dateiname}{\jobname}

\vspace{3cm}

\vfill

\footnotesize
\textsc{Quelle}: \titel. Herausgegeben von {\editorInnen}. In: \emph{Arthur Schnitzler: Briefwechsel mit Autorinnen und Autoren}.
 Digitale Edition, https://schnitzler-briefe.acdh.oeaw.ac.at/{\dateiname}.html (Stand \today)
\fi

\end{document}


      