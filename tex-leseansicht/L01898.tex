%% latex-leseansicht-vorspann.tex
%% Vorspann für die Leseansicht.
%% Lädt die gemeinsame Datei latex-vorspann.tex mit nicht gesetztem Schalter.

\newif\ifkorrekturansicht
\korrekturansichtfalse

\input{../tex-inputs/latex-vorspann}


\section[Olga Schnitzler an Richard und Paula Beer-Hofmann, {{[}}12. 12. 1909{{]}}]{L01898 Olga Schnitzler an Richard und Paula Beer-Hofmann, {[}12. 12. 1909{]}}
\nopagebreak\mylabel{L01898v}
\rehead{ }\normalsize\beginnumbering\briefempfaengerindex{Beer-Hofmann, Paula@\textsc{Beer-Hofmann, Paula}!zzzSchnitzler, Olga@\emph{von Olga Schnitzler}!1909-12-121@{{[}12. 12. 1909{]}}|(be}\briefempfaengerindex{Beer-Hofmann, Richard@\textsc{Beer-Hofmann, Richard}!zzzSchnitzler, Olga@\emph{von Olga Schnitzler}!1909-12-121@{{[}12. 12. 1909{]}}|(be}
\toendnotes[C]{\smallbreak\pagebreak[2]}
\correspDesc{Versand  durch Olga Schnitzler am [12. 12. 1909] in Wien
\newline{}Erhalt  durch Richard Beer-Hofmann, Paula Beer-Hofmann am 12. 12. 1909 in Wien}\toendnotes[C]{\smallbreak}
\Standort{YCGL, MSS 31.}
\physDesc{Briefkarte, , Kuvert, 287 Zeichen
\newline{}Handschrift: schwarze Tinte, lateinische Kurrent
\newline{}Versand: ohne postalischen Übermittlungsvermerk 
\newline{}Zusatz: mit Bleistift Umschlag oben rechts beschriftet:
                                    »S.« }
\buchAbdrucke{\weitereDrucke{Arthur Schnitzler, Richard Beer-Hofmann: \emph{Briefwechsel 1891–1931}. Herausgegeben von Konstanze Fliedl. Wien, Zürich: \emph{Europaverlag} 1992, S. 206.} }\toendnotes[C]{\smallbreak}\pstart{}{\pb}Herrn u. Frau\pend{}\pstart{}D\textsuperscript{r} Richard Beer-Hofmann\pend{}\pstart{}Hasenauerstrasse 59\oindex{Wien@\textbf{Wien}!XVIII., Währing@\textbf{XVIII., Währing}!Hasenauerstraße 59@\textbf{Hasenauerstraße 59}, \emph{Wohngebäude}|pw}. \pend{}{\bigskip}\vspace{1em}
\pstart
           \noindent{}{\pb}Liebe Beer-Hofmanns,{ }\label{K_L01898-1v}\edtext{morgen}{\lemma{\textnormal{\emph{morgen}}}\Cendnote{\textnormal{Siehe A. S.: \emph{Tagebuch}, 13. 12. 1909.
               }}}\label{K_L01898-1}{ }Abend kommen Kaufmann\pwindex{Kaufmann, Arthur 4.\,4.\,1872 Iași – 25.\,7.\,1938 Wien@\textsc{Kaufmann, Arthur} (4.\,4.\,1872 Iași – 25.\,7.\,1938 Wien), \emph{Rechtswissenschaftler, Privatgelehrte, Privatier}|pw}, Leo\pwindex{Van-Jung, Leo 15.\,10.\,1866 Odessa – 2.\,7.\,1939 Riga@\textsc{Van-Jung, Leo} (15.\,10.\,1866 Odessa – 2.\,7.\,1939 Riga), \emph{Gesangspädagoge, Mathematiker}|pw} und wahrscheinlich auch Saltens\pwindex{Salten, Felix 6.\,9.\,1869 Budapest – 8.\,10.\,1945 Zürich@\textsc{Salten, Felix} (6.\,9.\,1869 Budapest – 8.\,10.\,1945 Zürich), \emph{Schriftsteller, Journalist, Chefredakteur}|pw}\pwindex{Salten, Ottilie 7.\,3.\,1868 Prag – 22.\,6.\,1942 Zürich@\textsc{Salten, Ottilie} (7.\,3.\,1868 Prag – 22.\,6.\,1942 Zürich), \emph{Schauspielerin}|pw} zu uns, wir würden uns sehr freuen und das
               Gefühl von 2 aufgesetzten Kronen haben, wenn Ihr auch kämet.\pend
           
\pstart
           Herzliche Grüsse!{\\[\baselineskip]}\spacefill\mbox{Olga.}\pend
           \leftskip=0em{}
\pstart
           \noindent{}\label{K_L01898-2v}\edtext{Sonntag}{\lemma{\textnormal{\emph{Sonntag}}}\Cendnote{\textnormal{Die Datierung folgt dem
                     Tagebucheintrag vom 13. 12. 1909}}}\label{K_L01898-2}.\hspace*{1.5em}Bitte früh kommen!\pend
           \selectlanguage{ngerman}\endnumbering\briefempfaengerindex{Beer-Hofmann, Paula@\textsc{Beer-Hofmann, Paula}!zzzSchnitzler, Olga@\emph{von Olga Schnitzler}!1909-12-121@{{[}12. 12. 1909{]}}|)be}\briefempfaengerindex{Beer-Hofmann, Richard@\textsc{Beer-Hofmann, Richard}!zzzSchnitzler, Olga@\emph{von Olga Schnitzler}!1909-12-121@{{[}12. 12. 1909{]}}|)be}\mylabel{L01898h}  \newcommand{\dateiname}{L01898}\newcommand{\titel}{Olga Schnitzler an Richard und Paula Beer-Hofmann, [12. 12. 1909]}\newcommand{\editorInnen}{Martin Anton Müller und Gerd-Hermann Susen}%% latex-leseansicht-abspann.tex
%% Abspann für die Leseansicht.
%% Der Schalter \ifkorrekturansicht ist bereits durch den Vorspann gesetzt.

%% latex-abspann.tex
%% Gemeinsamer Abspann für Korrekturansicht und Leseansicht.
%% Setzt den Schalter \ifkorrekturansicht voraus (gesetzt in den
%% einbindenden Dateien latex-korrekturansicht-abspann.tex bzw.
%% latex-leseansicht-abspann.tex).
%% ---------------------------------------------------------------

\normalsize

% Das esempio-Environment wird nur in der Leseansicht benötigt
\ifkorrekturansicht\else
\newenvironment{esempio}[3]%
{
    \vspace{1.5ex}
    \rlap{\underline{#1}}
    \par
    \setlength{\parindent}{0cm}
    \nopagebreak
    \leftskip=#2cm
    \rightskip=#3cm
}
{
    \par
}
\fi

\doendnotes{C}
\bigskip
\vfill

\clearpage

\footnotesize

\ifkorrekturansicht
  \lohead{\textsc{register}}
\fi

% theindex-Environment neu definieren ohne reledmac
\makeatletter
\renewenvironment{theindex}{%
  \ifkorrekturansicht
    \section*{\indexname}%
  \else
    \subsubsection*{Index der erwähnten Entitäten}%
  \fi
  \setlength{\parindent}{0pt}%
  \setlength{\parskip}{0pt plus 0.3pt}%
  \let\item\@idxitem
}{%
  \ifkorrekturansicht\clearpage\fi
}
\makeatother

\IfFileExists{\jobname-pw.ind}{\input{\jobname-pw.ind}}{}

% Quellenangabe nur in der Leseansicht
\ifkorrekturansicht\else
% Fallback-Definitionen, falls die .tex-Datei \titel etc. nicht gesetzt hat
\providecommand{\titel}{}
\providecommand{\editorInnen}{}
\providecommand{\dateiname}{\jobname}

\vspace{3cm}

\vfill

\footnotesize
\textsc{Quelle}: \titel. Herausgegeben von {\editorInnen}. In: \emph{Arthur Schnitzler: Briefwechsel mit Autorinnen und Autoren}.
 Digitale Edition, https://schnitzler-briefe.acdh.oeaw.ac.at/{\dateiname}.html (Stand \today)
\fi

\end{document}


