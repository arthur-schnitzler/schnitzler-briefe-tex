%% latex-leseansicht-vorspann.tex
%% Vorspann für die Leseansicht.
%% Lädt die gemeinsame Datei latex-vorspann.tex mit nicht gesetztem Schalter.

\newif\ifkorrekturansicht
\korrekturansichtfalse

\input{../tex-inputs/latex-vorspann}


         
         \renewcommand{\erwaehntePersonen}{Personen: Hermann Bahr, Hermann Beer, Richard Beer-Hofmann, Otto Brahm, Paula Dehmel, Hugo von Hofmannsthal, Raphael Löwenfeld, Adele Sandrock, Hans Bernhard Schlesinger, Olga Schnitzler, Heinrich Schnitzler}
         \renewcommand{\erwaehnteInstitutionen}{Institutionen: Volkstheater}
         \renewcommand{\erwaehnteOrte}{Orte: Berlin, Biebrich, Burg Liechtenstein, Rom, Wien, Wiesbaden}
         \renewcommand{\erwaehnteWerke}{Werke: Der Schleier der Beatrice. Schauspiel in fünf Akten, Der einsame Weg. Schauspiel in fünf Akten, Lebendige Stunden. Vier Einakter}
               \section[Arthur Schnitzler an Hugo von Hofmannsthal, 7. 10. 1902]{ Arthur Schnitzler an Hugo von Hofmannsthal, 7. 10. 1902}\nopagebreak\mylabel{v}\rehead{ }\begin{ledgroupsized}[t]{13cm}\normalsize\beginnumbering\briefempfaengerindex{Hofmannsthal, Hugo von@\textsc{Hofmannsthal, Hugo von}!zzzSchnitzler, Arthur@\emph{von Arthur Schnitzler}!1902-10-071@{7. 10. 1902}|(be} \toendnotes[C]{\smallbreak\pagebreak[2]} \Standort{FDH, Hs-30885,98.}
\physDesc{Brief, 2 Blätter, 7 Seiten, 2587 Zeichen
\newline{}Handschrift: schwarze Tinte, deutsche Kurrent
\newline{}Ordnung: mit Bleistift von Schnitzler (?) mutmaßlich bei der Durchsicht
                                 der Korrespondenz 1929 das zweite Blatt datiert: »7/10 902« }\buchAbdrucke{\weitereDrucke{1) Hugo von Hofmannsthal, Arthur Schnitzler: \emph{Briefwechsel}. Hg. Therese Nickl und Heinrich Schnitzler. Frankfurt am Main: \emph{S. Fischer} 1964, S. 160–161.} \weitereDrucke{2) Hermann Bahr, Arthur Schnitzler: \emph{Briefwechsel, Aufzeichnungen, Dokumente (1891–1931)}. Hg. Kurt Ifkovits und Martin Anton Müller. Göttingen: \emph{Wallstein} 2018, S. 244.} }\toendnotes[C]{\smallbreak}\pstart
           \raggedleft{}{\pb}Wien\oindex{Wien@\textbf{Wien}|pw}, 7. X. 902\pend
           \pstart
           mein lieber Hugo,  Ihren Brief hab ich mit meiner Antwort zugleich
               an Bahr\pwindex{Bahr, Hermann 19.07.1863 – 15.01.1934@\textsc{Bahr, Hermann} (19.07.1863 – 15.01.1934), \emph{Schriftsteller, Kritiker}|pw} geſchickt; habe mich gleichfalls gegen
               monatliche Verpflichtung verwehrt, mich aber zu gelegentlichen die Monatsrate
               überſteigenden Beiträgen bereit erklärt. Ich fand den Brief der Frau D.\pwindex{Dehmel, Paula 1862-12-31 – 1918-07-08@\textsc{Dehmel, Paula} (1862-12-31 – 1918-07-08), \emph{Schriftstellerin}|pw} von einer bemerkenswerten Taktloſigkeit.\pend
           \pstart
           Leider bin ich nicht mehr dazu geko{\geminationm}en, Sie vor Ihrer
               Abreiſe zu ſehn; die Um{\pb}zugspräparationen hatten
               begonnen; nun ſind die Meinen\pwindex{Schnitzler, Olga 17.01.1882 – 13.01.1970@\textsc{Schnitzler, Olga} (17.01.1882 – 13.01.1970), \emph{Schauspielerin, Sängerin}|pwv}\pwindex{Schnitzler, Heinrich 09.08.1902 – 12.07.1982@\textsc{Schnitzler, Heinrich} (09.08.1902 – 12.07.1982), \emph{Regisseur, Schauspieler}|pwv} natürlich ſchon geraume Zeit herin; nur fehlen leider vorläufig die
               meiſten Möbel, wie das im Wien\oindex{Wien@\textbf{Wien}|pw}er Lieferantenweſen
               nun einmal nicht anders ſein kann. Aber es genirt nicht beſonders, u ich bin recht
               froh, daſs wir ſo nah von einander sind.\pend
           \pstart
           Mit dem Stück\pwindex{Schnitzler, Arthur 15.05.1862 – 21.10.1931@\textsc{Schnitzler, Arthur} (15.05.1862 – 21.10.1931), \emph{Schriftsteller, Mediziner}!einsame Weg. Schauspiel in fuenf Akten1904@\strich\emph{Der einsame Weg. Schauspiel in fünf Akten} {[}1904{]}|pwv} bin ich etliche
               Male ſtecken geblieben; heut iſt die Arbeit ſeit längerer Zeit das erſte Mal wieder
               beſſer gegangen, und ich werde wohl zu {\pb}Ende kommen –
               wenn auch nicht in dieſem Moment. Ich ſchreibe das Stück nun bis zum Schluſs und
               halte es ſelbſt \introOben{}nur\introOben{} für eine ſehr ausführliche Skizze. Wenn
               dann einige Auftritte fertiger ſind als ich geahnt, ſoll es mich angenehm
               überraſchen. Keinesfalls ſetz ich mir einen Termin. – Hans\pwindex{Schlesinger, Hans Bernhard 20.07.1875 – 13.3.1932@\textsc{Schlesinger, Hans Bernhard} (20.07.1875 – 13.3.1932), \emph{Maler}|pw} hab ich anläßlich des Leichenbegängniſſes von Richard\pwindex{Beer-Hofmann, Richard 1866-07-11 – 1945-09-26@\textsc{Beer-Hofmann, Richard} (1866-07-11 – 1945-09-26), \emph{Schriftsteller}|pw}’s Vater\pwindex{Beer, Hermann 10.8.1835 – 03.10.1902@\textsc{Beer, Hermann} (10.8.1835 – 03.10.1902), \emph{Rechtsanwalt}|pwv} geſehen, und habe viel Sympathie für ihn. –\pend
           \pstart
           Anfang nächſter Woche denke ich nach Berlin\oindex{Berlin@\textbf{Berlin}|pw} zu
               fahren; für acht Tage etwa. {\pb}Brahm\pwindex{Brahm, Otto 05.02.1856 – 28.11.1912@\textsc{Brahm, Otto} (05.02.1856 – 28.11.1912), \emph{Theaterleiter, Regisseur}|pw}{ }ſcheint plötzlich von Stücken ſo überſchwemmt zu
               werden, daſs die liebe \textsc{Beatrice}\pwindex{Schnitzler, Arthur 15.05.1862 – 21.10.1931@\textsc{Schnitzler, Arthur} (15.05.1862 – 21.10.1931), \emph{Schriftsteller, Mediziner}!Schleier der Beatrice. Schauspiel in fuenf Akten1900-12-01@\strich\emph{Der Schleier der Beatrice. Schauspiel in fünf Akten} {[}1900-12-01{]}|pw} wieder unter den Tiſch fallen wird. Aber ich denke, unterm Tiſch wird der \textsc{Loewenfeld}\pwindex{Loewenfeld, Raphael 11.02.1854 – 28.12.1910@\textsc{Löwenfeld, Raphael} (11.02.1854 – 28.12.1910), \emph{Theaterleiter}|pw}{ }ſitzen. –\pend
           \pstart
           – Die Leb. St.\pwindex{Schnitzler, Arthur 15.05.1862 – 21.10.1931@\textsc{Schnitzler, Arthur} (15.05.1862 – 21.10.1931), \emph{Schriftsteller, Mediziner}!Lebendige Stunden. Vier Einakter1901-12-23@\strich\emph{Lebendige Stunden. Vier Einakter} {[}1901-12-23{]}|pw} kommen im März mit
               der Sandrock\pwindex{Sandrock, Adele 1863-08-19 – 1937-08-30@\textsc{Sandrock, Adele} (1863-08-19 – 1937-08-30), \emph{Schauspielerin}|pw} am Volksth.\orgindex{Volkstheater@Volkstheater|pw} zur Aufführung. –\pend
           \pstart
           Ich bin ſchon ſehr geſpannt von Ihnen zu hören. Ich verſpreche mir für Sie von dem
                  römiſchen\oindex{Rom@\textbf{Rom}|pw} Aufenthalt unendlich viel. Laſſen
               Sie ſich nur nicht verſtimmen, wenn {\pb}Arbeitsluſt u kraft
               nicht gleich wieder da ſind. Denken Sie nur was »Production« für ein unfaßbares,
               unmeßbares und unbegreifliches Ding iſt – wie wir zuweilen ſchaffen, ohne es zu
               bemerken u ein andres Mal (mir geht es öfters ſo!) in aller Geſchäftigkeit ſo gut wie
               nichts geleiſtet haben. – Daſs das »Aufgeſchriebene« das einzige ist, was von den
               Fernerſtehenden controlirt werden ka{\geminationn}, ſollte uns nie
               verwirren. Für die {\pb}andern werd ich gewiſs nie ein
               Dichter ſein wie ich es vor 3 Jahren einmal auf einem einſamen Spaziergang von \textsc{Wiesbaden\oindex{Wiesbaden@\textbf{Wiesbaden}|pw}} nach \textsc{Biberich}\oindex{Biebrich@\textbf{Biebrich}|pw} und heuer im Sommer zehn oder gar zwanzig Minuten auf dem Lichtenstein\oindex{Burg Liechtenstein@\textbf{Burg Liechtenstein}|pw} war – Und das »übrig bleiben« ka{\geminationn} doch wohl kein \textsc{Criterium}{ }ſein. In hundert – oder zehntauſend oder
               ſiebzigtauſend Jahren iſt gar nichts {\pb}übrig.\pend
           \pstart
           Aber das führt ins allgemeine, und da weht einem die Luft zu kalt um die Ohren.\pend
           \pstart
           Schreiben Sie mir bald. Ich grüße Sie herzlich\hspace*{1.5em}Ihr{\\[\baselineskip]}\spacefill\mbox{A.}\pend
           \leftskip=0em{}
         
         \endnumbering\mylabel{h}\end{ledgroupsized}  \newcommand{\dateiname}{L01238}\newcommand{\titel}{Arthur Schnitzler an Hugo von Hofmannsthal, 7. 10. 1902}\newcommand{\editorInnen}{ Martin Anton Müller und Gerd-Hermann Susen}%% latex-leseansicht-abspann.tex
%% Abspann für die Leseansicht.
%% Der Schalter \ifkorrekturansicht ist bereits durch den Vorspann gesetzt.

%% latex-abspann.tex
%% Gemeinsamer Abspann für Korrekturansicht und Leseansicht.
%% Setzt den Schalter \ifkorrekturansicht voraus (gesetzt in den
%% einbindenden Dateien latex-korrekturansicht-abspann.tex bzw.
%% latex-leseansicht-abspann.tex).
%% ---------------------------------------------------------------

\normalsize

% Das esempio-Environment wird nur in der Leseansicht benötigt
\ifkorrekturansicht\else
\newenvironment{esempio}[3]%
{
    \vspace{1.5ex}
    \rlap{\underline{#1}}
    \par
    \setlength{\parindent}{0cm}
    \nopagebreak
    \leftskip=#2cm
    \rightskip=#3cm
}
{
    \par
}
\fi

\doendnotes{C}
\bigskip
\vfill

\clearpage

\footnotesize

\ifkorrekturansicht
  \lohead{\textsc{register}}
\fi

% theindex-Environment neu definieren ohne reledmac
\makeatletter
\renewenvironment{theindex}{%
  \ifkorrekturansicht
    \section*{\indexname}%
  \else
    \subsubsection*{Index der erwähnten Entitäten}%
  \fi
  \setlength{\parindent}{0pt}%
  \setlength{\parskip}{0pt plus 0.3pt}%
  \let\item\@idxitem
}{%
  \ifkorrekturansicht\clearpage\fi
}
\makeatother

\IfFileExists{\jobname-pw.ind}{\input{\jobname-pw.ind}}{}

% Quellenangabe nur in der Leseansicht
\ifkorrekturansicht\else
% Fallback-Definitionen, falls die .tex-Datei \titel etc. nicht gesetzt hat
\providecommand{\titel}{}
\providecommand{\editorInnen}{}
\providecommand{\dateiname}{\jobname}

\vspace{3cm}

\vfill

\footnotesize
\textsc{Quelle}: \titel. Herausgegeben von {\editorInnen}. In: \emph{Arthur Schnitzler: Briefwechsel mit Autorinnen und Autoren}.
 Digitale Edition, https://schnitzler-briefe.acdh.oeaw.ac.at/{\dateiname}.html (Stand \today)
\fi

\end{document}


      