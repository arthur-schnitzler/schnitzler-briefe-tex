%% latex-leseansicht-vorspann.tex
%% Vorspann für die Leseansicht.
%% Lädt die gemeinsame Datei latex-vorspann.tex mit nicht gesetztem Schalter.

\newif\ifkorrekturansicht
\korrekturansichtfalse

\input{../tex-inputs/latex-vorspann}


\section[Arthur Schnitzler an Hugo von Hofmannsthal, 7. 10. 1902]{L01238 Arthur Schnitzler an Hugo von Hofmannsthal, 7. 10. 1902}
\nopagebreak\mylabel{L01238v}
\rehead{ }\normalsize\beginnumbering\briefempfaengerindex{Hofmannsthal, Hugo von@\textsc{Hofmannsthal, Hugo von}!zzzSchnitzler, Arthur@\emph{von Arthur Schnitzler}!1902-10-071@{7. 10. 1902}|(be}
\toendnotes[C]{\smallbreak\pagebreak[2]}
\correspDesc{Versand  durch Arthur Schnitzler am 7. 10. 1902 in Wien
\newline{}Erhalt  durch Hugo von Hofmannsthal im Zeitraum [7. 10. 1902
                  – 11. 10. 1902?] \textbf{Ort fehlend} }\toendnotes[C]{\smallbreak}
\Standort{FDH, Hs-30885,98.}
\physDesc{Brief, 2 Blätter, 7 Seiten, 2587 Zeichen
\newline{}Handschrift: schwarze Tinte, deutsche Kurrent
\newline{}Ordnung: mit Bleistift von Schnitzler (?) mutmaßlich bei der Durchsicht
                                 der Korrespondenz 1929 das zweite Blatt datiert: »7/10 902« }
\buchAbdrucke{\weitereDrucke{1) Hugo von Hofmannsthal, Arthur Schnitzler: \emph{Briefwechsel}. Herausgegeben von Therese Nickl und Heinrich Schnitzler. Frankfurt am Main: \emph{S. Fischer} 1964, S. 160–161.} \weitereDrucke{2) Hermann Bahr, Arthur Schnitzler: \emph{Briefwechsel, Aufzeichnungen, Dokumente (1891–1931)}. Herausgegeben von Kurt Ifkovits und Martin Anton Müller. Göttingen: \emph{Wallstein} 2018, S. 244.} }\toendnotes[C]{\smallbreak}
\pstart
           \raggedleft{}{\pb}Wien\oindex{Wien@\textbf{Wien}, \emph{Verwaltungsgebiet}|pw}, 7. X. 902\pend
           \vspace{0.5em}
\pstart
           mein lieber Hugo,  Ihren Brief hab ich mit meiner Antwort zugleich
               an Bahr\pwindex{Bahr, Hermann 19.\,7.\,1863 Linz – 15.\,1.\,1934 München@\textsc{Bahr, Hermann} (19.\,7.\,1863 Linz – 15.\,1.\,1934 München), \emph{Schriftsteller, Kritiker}|pw} geſchickt; habe mich gleichfalls gegen
               monatliche Verpflichtung verwehrt, mich aber zu gelegentlichen die Monatsrate
               überſteigenden Beiträgen bereit erklärt. Ich fand den Brief der Frau D.\pwindex{Dehmel, Paula 31.\,12.\,1862 Berlin – 8.\,7.\,1918 ebd.@\textsc{Dehmel, Paula} (31.\,12.\,1862 Berlin – 8.\,7.\,1918 ebd.), \emph{Schriftstellerin}|pw} von einer bemerkenswerten Taktloſigkeit.\pend
           
\pstart
           Leider bin ich nicht mehr dazu geko{\geminationm}en, Sie vor Ihrer
               Abreiſe zu{ }ſehn; die Um{\pb}zugspräparationen hatten
               begonnen; nun{ }ſind die Meinen\pwindex{Schnitzler, Olga 17.\,1.\,1882 Wien – 13.\,1.\,1970 Lugano@\textsc{Schnitzler, Olga} (17.\,1.\,1882 Wien – 13.\,1.\,1970 Lugano), \emph{Schauspielerin, Sängerin}|pwv}\pwindex{Schnitzler, Heinrich 9.\,8.\,1902 Hinterbrühl – 12.\,7.\,1982 Wien@\textsc{Schnitzler, Heinrich} (9.\,8.\,1902 Hinterbrühl – 12.\,7.\,1982 Wien), \emph{Regisseur, Schauspieler}|pwv} natürlich{ }ſchon geraume Zeit herin; nur fehlen leider vorläufig die
               meiſten Möbel, wie das im Wien\oindex{Wien@\textbf{Wien}, \emph{Verwaltungsgebiet}|pw}er Lieferantenweſen
               nun einmal nicht anders{ }ſein kann. Aber es genirt nicht beſonders, u ich bin recht
               froh, daſs wir{ }ſo nah von einander sind.\pend
           
\pstart
           Mit dem Stück\pwindex{Schnitzler, Arthur 15.\,5.\,1862 Wien – 21.\,10.\,1931 ebd.@\textsc{Schnitzler, Arthur} (15.\,5.\,1862 Wien – 21.\,10.\,1931 ebd.), \emph{Schriftsteller, Mediziner}!einsame Weg. Schauspiel in fünf Akten@\strich\emph{Der einsame Weg. Schauspiel in fünf Akten}|pwv} bin ich etliche
               Male{ }ſtecken geblieben; heut iſt die Arbeit{ }ſeit längerer Zeit das erſte Mal wieder
               beſſer gegangen, und ich werde wohl zu {\pb}Ende kommen –
               wenn auch nicht in dieſem Moment. Ich{ }ſchreibe das Stück nun bis zum Schluſs und
               halte es{ }ſelbſt \introOben{}nur\introOben{} für eine{ }ſehr ausführliche Skizze. Wenn
               dann einige Auftritte fertiger{ }ſind als ich geahnt,{ }ſoll es mich angenehm
               überraſchen. Keinesfalls{ }ſetz ich mir einen Termin. – Hans\pwindex{Schlesinger, Hans Bernhard 20.\,7.\,1875 Wien – 13.\,3.\,1932 Salzburg@\textsc{Schlesinger, Hans Bernhard} (20.\,7.\,1875 Wien – 13.\,3.\,1932 Salzburg), \emph{Maler}|pw} hab ich anläßlich des Leichenbegängniſſes von Richard\pwindex{Beer-Hofmann, Richard 11.\,7.\,1866 Wien – 26.\,9.\,1945 New York City@\textsc{Beer-Hofmann, Richard} (11.\,7.\,1866 Wien – 26.\,9.\,1945 New York City), \emph{Schriftsteller}|pw}’s Vater\pwindex{Beer, Hermann 10.\,8.\,1835 Radiměř – 3.\,10.\,1902 Wien@\textsc{Beer, Hermann} (10.\,8.\,1835 Radiměř – 3.\,10.\,1902 Wien), \emph{Rechtsanwalt}|pwv} geſehen, und habe viel Sympathie für ihn. –\pend
           
\pstart
           Anfang nächſter Woche denke ich nach Berlin\oindex{Berlin@\textbf{Berlin}, \emph{Hauptstadt}|pw} zu
               fahren; für acht Tage etwa. {\pb}Brahm\pwindex{Brahm, Otto 5.\,2.\,1856 Hamburg – 28.\,11.\,1912 Berlin@\textsc{Brahm, Otto} (5.\,2.\,1856 Hamburg – 28.\,11.\,1912 Berlin), \emph{Theaterleiter, Regisseur}|pw}{ }ſcheint plötzlich von Stücken{ }ſo überſchwemmt zu
               werden, daſs die liebe \textsc{Beatrice}\pwindex{Schnitzler, Arthur 15.\,5.\,1862 Wien – 21.\,10.\,1931 ebd.@\textsc{Schnitzler, Arthur} (15.\,5.\,1862 Wien – 21.\,10.\,1931 ebd.), \emph{Schriftsteller, Mediziner}!Schleier der Beatrice. Schauspiel in fünf Akten@\strich\emph{Der Schleier der Beatrice. Schauspiel in fünf Akten}|pw} wieder unter den Tiſch fallen wird. Aber ich denke, unterm Tiſch wird der \textsc{Loewenfeld}\pwindex{Löwenfeld, Raphael 11.\,2.\,1854 Poznan – 28.\,12.\,1910 Berlin@\textsc{Löwenfeld, Raphael} (11.\,2.\,1854 Poznan – 28.\,12.\,1910 Berlin), \emph{Theaterleiter}|pw}{ }ſitzen. –\pend
           
\pstart
           – Die Leb. St.\pwindex{Schnitzler, Arthur 15.\,5.\,1862 Wien – 21.\,10.\,1931 ebd.@\textsc{Schnitzler, Arthur} (15.\,5.\,1862 Wien – 21.\,10.\,1931 ebd.), \emph{Schriftsteller, Mediziner}!Lebendige Stunden. Vier Einakter@\strich\emph{Lebendige Stunden. Vier Einakter}|pw} kommen im März mit
               der Sandrock\pwindex{Sandrock, Adele 19.\,8.\,1863 Rotterdam – 30.\,8.\,1937 Berlin@\textsc{Sandrock, Adele} (19.\,8.\,1863 Rotterdam – 30.\,8.\,1937 Berlin), \emph{Schauspielerin}|pw} am Volksth.\orgindex{Volkstheater@Volkstheater|pw} zur Aufführung. –\pend
           
\pstart
           Ich bin{ }ſchon{ }ſehr geſpannt von Ihnen zu hören. Ich verſpreche mir für Sie von dem
                  römiſchen\oindex{Rom@\textbf{Rom}, \emph{Hauptstadt}|pw} Aufenthalt unendlich viel. Laſſen
               Sie{ }ſich nur nicht verſtimmen, wenn {\pb}Arbeitsluſt u kraft
               nicht gleich wieder da{ }ſind. Denken Sie nur was »Production« für ein unfaßbares,
               unmeßbares und unbegreifliches Ding iſt – wie wir zuweilen{ }ſchaffen, ohne es zu
               bemerken u ein andres Mal (mir geht es öfters{ }ſo!) in aller Geſchäftigkeit{ }ſo gut wie
               nichts geleiſtet haben. – Daſs das »Aufgeſchriebene« das einzige ist, was von den
               Fernerſtehenden controlirt werden ka{\geminationn},{ }ſollte uns nie
               verwirren. Für die {\pb}andern werd ich gewiſs nie ein
               Dichter{ }ſein wie ich es vor 3 Jahren einmal auf einem einſamen Spaziergang von \textsc{Wiesbaden\oindex{Wiesbaden@\textbf{Wiesbaden}|pw}} nach \textsc{Biberich}\oindex{Biebrich@\textbf{Biebrich}, \emph{Teil eines besiedelten Ortes}|pw} und heuer im Sommer zehn oder gar zwanzig Minuten auf dem Lichtenstein\oindex{Burg Liechtenstein@\textbf{Burg Liechtenstein}, \emph{Schloss}|pw} war – Und das »übrig bleiben« ka{\geminationn} doch wohl kein \textsc{Criterium}{ }ſein. In hundert – oder zehntauſend oder{ }ſiebzigtauſend Jahren iſt gar nichts {\pb}übrig.\pend
           
\pstart
           Aber das führt ins allgemeine, und da weht einem die Luft zu kalt um die Ohren.\pend
           
\pstart
           Schreiben Sie mir bald. Ich grüße Sie herzlich\hspace*{1.5em}Ihr{\\[\baselineskip]}\spacefill\mbox{A.}\pend
           \leftskip=0em{}\selectlanguage{ngerman}\endnumbering\briefempfaengerindex{Hofmannsthal, Hugo von@\textsc{Hofmannsthal, Hugo von}!zzzSchnitzler, Arthur@\emph{von Arthur Schnitzler}!1902-10-071@{7. 10. 1902}|)be}\mylabel{L01238h}  \newcommand{\dateiname}{L01238}\newcommand{\titel}{Arthur Schnitzler an Hugo von Hofmannsthal, 7. 10. 1902}\newcommand{\editorInnen}{Herausgegeben von Martin Anton Müller}%% latex-leseansicht-abspann.tex
%% Abspann für die Leseansicht.
%% Der Schalter \ifkorrekturansicht ist bereits durch den Vorspann gesetzt.

%% latex-abspann.tex
%% Gemeinsamer Abspann für Korrekturansicht und Leseansicht.
%% Setzt den Schalter \ifkorrekturansicht voraus (gesetzt in den
%% einbindenden Dateien latex-korrekturansicht-abspann.tex bzw.
%% latex-leseansicht-abspann.tex).
%% ---------------------------------------------------------------

\normalsize

% Das esempio-Environment wird nur in der Leseansicht benötigt
\ifkorrekturansicht\else
\newenvironment{esempio}[3]%
{
    \vspace{1.5ex}
    \rlap{\underline{#1}}
    \par
    \setlength{\parindent}{0cm}
    \nopagebreak
    \leftskip=#2cm
    \rightskip=#3cm
}
{
    \par
}
\fi

\doendnotes{C}
\bigskip
\vfill

\clearpage

\footnotesize

\ifkorrekturansicht
  \lohead{\textsc{register}}
\fi

% theindex-Environment neu definieren ohne reledmac
\makeatletter
\renewenvironment{theindex}{%
  \ifkorrekturansicht
    \section*{\indexname}%
  \else
    \subsubsection*{Index der erwähnten Entitäten}%
  \fi
  \setlength{\parindent}{0pt}%
  \setlength{\parskip}{0pt plus 0.3pt}%
  \let\item\@idxitem
}{%
  \ifkorrekturansicht\clearpage\fi
}
\makeatother

\IfFileExists{\jobname-pw.ind}{\input{\jobname-pw.ind}}{}

% Quellenangabe nur in der Leseansicht
\ifkorrekturansicht\else
% Fallback-Definitionen, falls die .tex-Datei \titel etc. nicht gesetzt hat
\providecommand{\titel}{}
\providecommand{\editorInnen}{}
\providecommand{\dateiname}{\jobname}

\vspace{3cm}

\vfill

\footnotesize
\textsc{Quelle}: \titel. Herausgegeben von {\editorInnen}. In: \emph{Arthur Schnitzler: Briefwechsel mit Autorinnen und Autoren}.
 Digitale Edition, https://schnitzler-briefe.acdh.oeaw.ac.at/{\dateiname}.html (Stand \today)
\fi

\end{document}


