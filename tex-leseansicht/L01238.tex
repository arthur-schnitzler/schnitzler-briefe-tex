%% latex-korrekturansicht-vorspann.tex
%% Vorspann für die Korrekturansicht.
%% Lädt die gemeinsame Datei latex-vorspann.tex mit gesetztem Schalter.

\newif\ifkorrekturansicht
\korrekturansichttrue

\input{../tex-inputs/latex-vorspann}


\section[Arthur Schnitzler an Hugo von Hofmannsthal, 7. 10. 1902]{L01238 Arthur Schnitzler an Hugo von Hofmannsthal, 7. 10. 1902}
\nopagebreak\mylabel{L01238v}
\rehead{ }\normalsize\beginnumbering\briefempfaengerindex{Hofmannsthal, Hugo von@\textsc{Hofmannsthal, Hugo von}!zzzSchnitzler, Arthur@\emph{von Arthur Schnitzler}!1902-10-071@{7. 10. 1902}|(be}
\toendnotes[C]{\smallbreak\pagebreak[2]}\Standort{FDH, Hs-30885,98.}
\physDesc{Brief, 2 Blätter, 7 Seiten, 2587 Zeichen
\newline{}Handschrift: schwarze Tinte, deutsche Kurrent
\newline{}Ordnung: mit Bleistift von Schnitzler (?) mutmaßlich bei der Durchsicht
                                 der Korrespondenz 1929 das zweite Blatt datiert: »7/10 902« }
\buchAbdrucke{\weitereDrucke{1) Hugo von Hofmannsthal, Arthur Schnitzler: \emph{Briefwechsel}. Frankfurt am Main: \emph{S. Fischer} 1964, S. 160–161.} \weitereDrucke{2) Hermann Bahr, Arthur Schnitzler: \emph{Briefwechsel, Aufzeichnungen, Dokumente (1891–1931)}. Göttingen: \emph{Wallstein} 2018, S. 244.} }\toendnotes[C]{\smallbreak}
\pstart
           \raggedleft{}{\pb}Wien\oindex{Wien@\textbf{Wien}, \emph{A.ADM2}|pw}, 7. X. 902\pend
           \vspace{0.5em}
\pstart
           mein lieber Hugo,  Ihren Brief hab ich mit meiner Antwort zugleich
               an Bahr\pwindex{Bahr, Hermann 19.07.1863 – 15.01.1934@\textsc{Bahr, Hermann} (19.07.1863 – 15.01.1934), \emph{Schriftsteller/Schriftstellerin, Kritiker/Kritikerin}|pw} geſchickt; habe mich gleichfalls gegen
               monatliche Verpflichtung verwehrt, mich aber zu gelegentlichen die Monatsrate
               überſteigenden Beiträgen bereit erklärt. Ich fand den Brief der Frau D.\pwindex{Dehmel, Paula 1862-12-31 – 1918-07-08@\textsc{Dehmel, Paula} (1862-12-31 – 1918-07-08), \emph{Schriftsteller/Schriftstellerin}|pw} von einer bemerkenswerten Taktloſigkeit.\pend
           
\pstart
           Leider bin ich nicht mehr dazu geko{\geminationm}en, Sie vor Ihrer
               Abreiſe zu ſehn; die Um{\pb}zugspräparationen hatten
               begonnen; nun ſind die Meinen\pwindex{Schnitzler, Olga 17.01.1882 – 13.01.1970@\textsc{Schnitzler, Olga} (17.01.1882 – 13.01.1970), \emph{Schauspieler/Schauspielerin, Sänger/Sängerin}|pwv}\pwindex{Schnitzler, Heinrich 09.08.1902 – 12.07.1982@\textsc{Schnitzler, Heinrich} (09.08.1902 – 12.07.1982), \emph{Regisseur/Regisseurin, Schauspieler/Schauspielerin}|pwv} natürlich ſchon geraume Zeit herin; nur fehlen leider vorläufig die
               meiſten Möbel, wie das im Wien\oindex{Wien@\textbf{Wien}, \emph{A.ADM2}|pw}er Lieferantenweſen
               nun einmal nicht anders ſein kann. Aber es genirt nicht beſonders, u ich bin recht
               froh, daſs wir ſo nah von einander sind.\pend
           
\pstart
           Mit dem Stück\pwindex{einsame Weg. Schauspiel in fuenf Akten@\emph{Der einsame Weg. Schauspiel in fünf Akten}|pwv} bin ich etliche
               Male ſtecken geblieben; heut iſt die Arbeit ſeit längerer Zeit das erſte Mal wieder
               beſſer gegangen, und ich werde wohl zu {\pb}Ende kommen –
               wenn auch nicht in dieſem Moment. Ich ſchreibe das Stück nun bis zum Schluſs und
               halte es ſelbſt \introOben{}nur\introOben{} für eine ſehr ausführliche Skizze. Wenn
               dann einige Auftritte fertiger ſind als ich geahnt, ſoll es mich angenehm
               überraſchen. Keinesfalls ſetz ich mir einen Termin. – Hans\pwindex{Schlesinger, Hans Bernhard 20.07.1875 – 13.3.1932@\textsc{Schlesinger, Hans Bernhard} (20.07.1875 – 13.3.1932), \emph{Maler/Malerin}|pw} hab ich anläßlich des Leichenbegängniſſes von Richard\pwindex{Beer-Hofmann, Richard 1866-07-11 – 1945-09-26@\textsc{Beer-Hofmann, Richard} (1866-07-11 – 1945-09-26), \emph{Schriftsteller/Schriftstellerin}|pw}’s Vater\pwindex{Beer, Hermann 10.8.1835 – 03.10.1902@\textsc{Beer, Hermann} (10.8.1835 – 03.10.1902), \emph{Rechtsanwalt/Rechtsanwältin}|pwv} geſehen, und habe viel Sympathie für ihn. –\pend
           
\pstart
           Anfang nächſter Woche denke ich nach Berlin\oindex{Berlin@\textbf{Berlin}, \emph{P.PPLC}|pw} zu
               fahren; für acht Tage etwa. {\pb}Brahm\pwindex{Brahm, Otto 05.02.1856 – 28.11.1912@\textsc{Brahm, Otto} (05.02.1856 – 28.11.1912), \emph{Theaterleiter/Theaterleiterin, Regisseur/Regisseurin}|pw}{ }ſcheint plötzlich von Stücken ſo überſchwemmt zu
               werden, daſs die liebe \textsc{Beatrice}\pwindex{Schleier der Beatrice. Schauspiel in fuenf Akten@\emph{Der Schleier der Beatrice. Schauspiel in fünf Akten}|pw} wieder unter den Tiſch fallen wird. Aber ich denke, unterm Tiſch wird der \textsc{Loewenfeld}\pwindex{Loewenfeld, Raphael 11.02.1854 – 28.12.1910@\textsc{Löwenfeld, Raphael} (11.02.1854 – 28.12.1910), \emph{Theaterleiter/Theaterleiterin}|pw}{ }ſitzen. –\pend
           
\pstart
           – Die Leb. St.\pwindex{Lebendige Stunden. Vier Einakter@\emph{Lebendige Stunden. Vier Einakter}|pw} kommen im März mit
               der Sandrock\pwindex{Sandrock, Adele 1863-08-19 – 1937-08-30@\textsc{Sandrock, Adele} (1863-08-19 – 1937-08-30), \emph{Schauspieler/Schauspielerin}|pw} am Volksth.\orgindex{Volkstheater@Volkstheater|pw} zur Aufführung. –\pend
           
\pstart
           Ich bin ſchon ſehr geſpannt von Ihnen zu hören. Ich verſpreche mir für Sie von dem
                  römiſchen\oindex{Rom@\textbf{Rom}, \emph{P.PPLC}|pw} Aufenthalt unendlich viel. Laſſen
               Sie ſich nur nicht verſtimmen, wenn {\pb}Arbeitsluſt u kraft
               nicht gleich wieder da ſind. Denken Sie nur was »Production« für ein unfaßbares,
               unmeßbares und unbegreifliches Ding iſt – wie wir zuweilen ſchaffen, ohne es zu
               bemerken u ein andres Mal (mir geht es öfters ſo!) in aller Geſchäftigkeit ſo gut wie
               nichts geleiſtet haben. – Daſs das »Aufgeſchriebene« das einzige ist, was von den
               Fernerſtehenden controlirt werden ka{\geminationn}, ſollte uns nie
               verwirren. Für die {\pb}andern werd ich gewiſs nie ein
               Dichter ſein wie ich es vor 3 Jahren einmal auf einem einſamen Spaziergang von \textsc{Wiesbaden\oindex{Wiesbaden@\textbf{Wiesbaden}, \emph{P.PPLA}|pw}} nach \textsc{Biberich}\oindex{Biebrich@\textbf{Biebrich}, \emph{Teil eines besiedelten Ortes (A.BSOX)}|pw} und heuer im Sommer zehn oder gar zwanzig Minuten auf dem Lichtenstein\oindex{Burg Liechtenstein@\textbf{Burg Liechtenstein}, \emph{Schloss (K.SLS)}|pw} war – Und das »übrig bleiben« ka{\geminationn} doch wohl kein \textsc{Criterium}{ }ſein. In hundert – oder zehntauſend oder
               ſiebzigtauſend Jahren iſt gar nichts {\pb}übrig.\pend
           
\pstart
           Aber das führt ins allgemeine, und da weht einem die Luft zu kalt um die Ohren.\pend
           
\pstart
           Schreiben Sie mir bald. Ich grüße Sie herzlich\hspace*{1.5em}Ihr{\\[\baselineskip]}\spacefill\mbox{A.}\pend
           \leftskip=0em{}\selectlanguage{ngerman}\endnumbering\briefempfaengerindex{Hofmannsthal, Hugo von@\textsc{Hofmannsthal, Hugo von}!zzzSchnitzler, Arthur@\emph{von Arthur Schnitzler}!1902-10-071@{7. 10. 1902}|)be}\mylabel{L01238h}  \normalsize

\doendnotes{C}
\bigskip
\vfill

\clearpage

\footnotesize

\lohead{\textsc{register}}

% Definiere theindex-Environment komplett neu ohne reledmac
\makeatletter
\renewenvironment{theindex}{%
  \section*{\indexname}%
  \setlength{\parindent}{0pt}%
  \setlength{\parskip}{0pt plus 0.3pt}%
  \let\item\@idxitem
}{%
  \clearpage
}
\makeatother

\IfFileExists{\jobname-pw.ind}{\input{\jobname-pw.ind}}{}

\end{document}

      