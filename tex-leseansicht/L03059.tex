%% latex-leseansicht-vorspann.tex
%% Vorspann für die Leseansicht.
%% Lädt die gemeinsame Datei latex-vorspann.tex mit nicht gesetztem Schalter.

\newif\ifkorrekturansicht
\korrekturansichtfalse

\input{../tex-inputs/latex-vorspann}


         
         \renewcommand{\erwaehntePersonen}{Personen: Richard Beer-Hofmann, Jean-Gaspard Deburau, Marie Glümer, Paul Goldmann, Yvette Guilbert, Paul Lindau, Rudolf Renvers, Camille Saint-Saëns, Max Schiller, Olga Schnitzler, Elisabeth Steinrück, Ernst von Wolzogen}
         \renewcommand{\erwaehnteInstitutionen}{Institutionen: Berliner Theater, Überbrettl}
         \renewcommand{\erwaehnteOrte}{Orte: ?? [Sanatorium], Berlin, Dessauer Straße, Frankreich, Wien}
         \renewcommand{\erwaehnteWerke}{Werke: Der blinde Geronimo und sein Bruder, Die Zeit. Wiener Wochenschrift, Frau Bertha Garlan. Roman, Jugend, Neue Deutsche Rundschau, Sylvesternacht. Ein Dialog}
               \section[ Paul Goldmann an Arthur Schnitzler, 18. 2. {[}1901{]}]{ Paul Goldmann an Arthur Schnitzler, 18. 2. {[}1901{]}}\nopagebreak\mylabel{v}\rehead{ }\begin{ledgroupsized}[t]{13cm}\normalsize\beginnumbering\briefempfaengerindex{Schnitzler, Arthur@\textsc{Schnitzler, Arthur}!zzzGoldmann, Paul@\emph{von Paul Goldmann}!1901-02-181@{18. 2. {[}1901{]}}|(be} \toendnotes[C]{\smallbreak\pagebreak[2]} \Standort{DLA, A:Schnitzler, HS.NZ85.1.3171.}
\physDesc{Brief, 1 Blatt, 3 Seiten, 1835 Zeichen
\newline{}Handschrift: blaue Tinte, deutsche Kurrent
\newline{}Schnitzler: 1) mit Bleistift das Jahr »901« vermerkt  2) mit rotem Buntstift sechs Unterstreichungen}\toendnotes[C]{\smallbreak}\pstart
           \noindent{}\raggedleft{}{\pb}\textcolor{gray}{\textbf{DESSAUERSTRASSE 19\oindex{Dessauer Strasse@\textbf{Dessauer Straße}|pw}}}\pend
           \pstart
           Berlin\oindex{Berlin@\textbf{Berlin}|pw}, 18. Februar.\pend
           \pstart\center{}Mein lieber Freund,\pend\pstart
           Ich war Freitag bei \textsc{Mizzi Gl.\pwindex{Gluemer, Marie 03.07.1867 – 16.11.1925@\textsc{Glümer, Marie} (03.07.1867 – 16.11.1925), \emph{Schauspielerin}|pw}}, ehe ſie ins \label{K_L03059-1v}\edtext{Sanatorium\oindex{?? [Sanatorium]@\textbf{?? [Sanatorium]}|pwv}}{\lemma{\textnormal{\emph{Sanatorium}}}\Cendnote{\textnormal{nicht ermittelt}}}\label{K_L03059-1h} ging. Seither
               keine Nachricht. Auch ich verſtehe abſolut nicht, was ſie hat, bin aber feſt
               überzeugt, daß es nicht \label{K_L03059-2v}\edtext{\textsc{Neuralgie}}{\lemma{\textnormal{\emph{Neuralgie}}}\Cendnote{\textnormal{siehe Paul Goldmann an Arthur Schnitzler, 12. 2. [1901]}}}\label{K_L03059-2h} ſein kann. Das arme Mädel\pwindex{Gluemer, Marie 03.07.1867 – 16.11.1925@\textsc{Glümer, Marie} (03.07.1867 – 16.11.1925), \emph{Schauspielerin}|pwv} iſt ſehr heruntergekommen. Ich habe immer eine Blutkrankheit
               vermuthet, und aus den vagen Andeutungen, die \textsc{Renvers\pwindex{Renvers, Rudolf 1854-02-18 – 1909-03-22@\textsc{Renvers, Rudolf} (1854-02-18 – 1909-03-22), \emph{Mediziner, Universitätslehrer}|pw}} gemacht zu haben ſcheint, höre ich etwas wie eine Beſtätigung heraus
               (Blutzerſetzung?). Ich kann zu \textsc{Renvers\pwindex{Renvers, Rudolf 1854-02-18 – 1909-03-22@\textsc{Renvers, Rudolf} (1854-02-18 – 1909-03-22), \emph{Mediziner, Universitätslehrer}|pw}} nicht gehen. \label{K_L03059-3v}\edtext{\begin{otherlanguage}{french}\textsc{À quel titre}\end{otherlanguage}}{\lemma{\textnormal{\emph{À quel titre}}}\Cendnote{\textnormal{französisch: auf welcher Grundlage, mit
                  welchem Recht}}}\label{K_L03059-3h}? Aber ich hoffe doch noch einen Weg zu finden, um mich an
               mediziniſcher Quelle zu informiren.\pend
           \pstart
           Daß Du den \label{K_L03059-4v}\edtext{Plan haſt herzukommen}{\lemma{\textnormal{\emph{Plan haſt herzukommen}}}\Cendnote{\textnormal{Schnitzler\pwindex{Schnitzler, Arthur 15.05.1862 – 21.10.1931@\textsc{Schnitzler, Arthur} (15.05.1862 – 21.10.1931), \emph{Schriftsteller, Mediziner}|pwk} war zwischen 3. 3. 1901 und 10. 3. 1901 in Berlin\oindex{Berlin@\textbf{Berlin}|pwk}.}}}\label{K_L03059-4h}, iſt ſehr ſchön. Ich hoffe, Du
               führſt ihn aus.\pend
           \pstart
           {\pb}Es iſt nicht unmöglich, daß ich für \textsc{Olga\pwindex{Schnitzler, Olga 17.01.1882 – 13.01.1970@\textsc{Schnitzler, Olga} (17.01.1882 – 13.01.1970), \emph{Schauspielerin, Sängerin}|pw}} etwas bei \label{K_L03059-5v}\edtext{\textsc{Lindau\pwindex{Lindau, Paul 03.06.1839 – 31.01.1919@\textsc{Lindau, Paul} (03.06.1839 – 31.01.1919), \emph{Schriftsteller, Kritiker, Theaterleiter}|pw}}}{\lemma{\textnormal{\emph{Lindau}}}\Cendnote{\textnormal{Paul Lindau\pwindex{Lindau, Paul 03.06.1839 – 31.01.1919@\textsc{Lindau, Paul} (03.06.1839 – 31.01.1919), \emph{Schriftsteller, Kritiker, Theaterleiter}|pwk} leitete das \emph{Berliner Theater}\orgindex{Berliner Theater@Berliner Theater|pwk}. Schnitzler\pwindex{Schnitzler, Arthur 15.05.1862 – 21.10.1931@\textsc{Schnitzler, Arthur} (15.05.1862 – 21.10.1931), \emph{Schriftsteller, Mediziner}|pwk} versuchte für seine Partnerin (und zukünftige Ehefrau) Olga\pwindex{Schnitzler, Olga 17.01.1882 – 13.01.1970@\textsc{Schnitzler, Olga} (17.01.1882 – 13.01.1970), \emph{Schauspielerin, Sängerin}|pwk} ein Engagement zu finden.}}}\label{K_L03059-5h} thun
               könnte. Aber Du müßteſt auch eingreifen, Dein Wort würde mehr ins Gewicht fallen als
               meines. \label{K_L03059-6v}\edtext{\textsc{Wolzogen\pwindex{Wolzogen, Ernst von 23.04.1855 – 30.07.1934@\textsc{Wolzogen, Ernst von} (23.04.1855 – 30.07.1934), \emph{Schriftsteller}|pw}}}{\lemma{\textnormal{\emph{Wolzogen}}}\Cendnote{\textnormal{Ernst von Wolzogen\pwindex{Wolzogen, Ernst von 23.04.1855 – 30.07.1934@\textsc{Wolzogen, Ernst von} (23.04.1855 – 30.07.1934), \emph{Schriftsteller}|pwk}, der 1901 das literarische Kabarett \emph{Überbrettl}\orgindex{Ueberbrettl@Überbrettl|pwk} (auch bekannt als Wolzogen-Theater\orgindex{Ueberbrettl@Überbrettl|pwkv} und Buntes Theater\orgindex{Ueberbrettl@Überbrettl|pwkv}) in Berlin\oindex{Berlin@\textbf{Berlin}|pwk} gegründet hatte}}}\label{K_L03059-6h} kenne ich perſönlich. Auch bei ihm könnteſt
               Du viel ausrichten, ich könnte nur mithelfen. Aber wäre das Überbrettl\orgindex{Ueberbrettl@Überbrettl|pw} denn eine Exiſtenz? Und \strikeout{\textcolor{gray}{biſt}{ }\textcolor{gray}{×}} iſt die \label{K_L03059-7v}\edtext{Kleine\pwindex{Steinrueck, Elisabeth 19.11.1885 – 07.04.1920@\textsc{Steinrück, Elisabeth} (19.11.1885 – 07.04.1920)|pwv}}{\lemma{\textnormal{\emph{Kleine}}}\Cendnote{\textnormal{Hier dürfte ein Wechsel in der Rede von
                     Olga\pwindex{Schnitzler, Olga 17.01.1882 – 13.01.1970@\textsc{Schnitzler, Olga} (17.01.1882 – 13.01.1970), \emph{Schauspielerin, Sängerin}|pwkv} zu ihrer jüngeren
                  Schwester Elisabeth Gussmann\pwindex{Steinrueck, Elisabeth 19.11.1885 – 07.04.1920@\textsc{Steinrück, Elisabeth} (19.11.1885 – 07.04.1920)|pwk} stattfinden, da
                  sich Goldmann\pwindex{Goldmann, Paul 31.01.1865 – 25.09.1935@\textsc{Goldmann, Paul} (31.01.1865 – 25.09.1935), \emph{Schriftsteller, Journalist}|pwk} für eine Anstellung von Elisabeth Gussmann\pwindex{Steinrueck, Elisabeth 19.11.1885 – 07.04.1920@\textsc{Steinrück, Elisabeth} (19.11.1885 – 07.04.1920)|pwk} einsetzte. Vgl. Paul Goldmann an Arthur Schnitzler, 6. 4. [1901] und die Korrespondenz
                  zwischen Goldmann\pwindex{Goldmann, Paul 31.01.1865 – 25.09.1935@\textsc{Goldmann, Paul} (31.01.1865 – 25.09.1935), \emph{Schriftsteller, Journalist}|pwk} und Elisabeth Gussmann\pwindex{Steinrueck, Elisabeth 19.11.1885 – 07.04.1920@\textsc{Steinrück, Elisabeth} (19.11.1885 – 07.04.1920)|pwk}: \emph{DLA}, HS.1985.1.5246.}}}\label{K_L03059-7h} mit ihren Studien
               ſchon fertig?\pend
           \pstart
           \textsc{Yvette Guilbert\pwindex{Guilbert, Yvette 20.01.1865 – 03.02.1944@\textsc{Guilbert, Yvette} (20.01.1865 – 03.02.1944), \emph{Schauspielerin, Sängerin}|pw}}, deren Mann\pwindex{Schiller, Max 1861-07-17 – 1952-05-21@\textsc{Schiller, Max} (1861-07-17 – 1952-05-21), \emph{Chemiker}|pwv} Dich kennt
               und liebt (Deine Werke nämlich), läßt Dich fragen, ob Du ihr nicht einen \label{K_L03059-8v}\edtext{Einakter ſchreiben}{\lemma{\textnormal{\emph{Einakter ſchreiben}}}\Cendnote{\textnormal{nicht geschehen}}}\label{K_L03059-8h} möchteſt? Eine \label{K_L03059-9v}\edtext{\textsc{Pierrot}}{\lemma{\textnormal{\emph{Pierrot}}}\Cendnote{\textnormal{männlicher Komödienfigurentyp, der
                  insbesondere durch den fran\oindex{Frankreich@\textbf{Frankreich}|pwkv}zösischen Pantomimen Jean-Gaspard
                     Deburau\pwindex{Deburau, Jean-Gaspard 1796-07-31 – 1846-06-17@\textsc{Deburau, Jean-Gaspard} (1796-07-31 – 1846-06-17), \emph{Pantomime}|pwk} berühmt wurde}}}\label{K_L03059-9h}-Komödie, und zwar einen revolutionären \textsc{Pierrot}. Keine \textsc{Pantomime}. Die
               Komödie ſoll von einem großen fran\oindex{Frankreich@\textbf{Frankreich}|pwv}zöſiſchen Componiſten (vielleicht \textsc{Saint-Saëns\pwindex{Saint-Saens, Camille 1835-10-09 – 1921-12-16@\textsc{Saint-Saëns, Camille} (1835-10-09 – 1921-12-16), \emph{Komponist}|pw}}) {\pb}in Muſik geſetzt werden. Bitte, antworte mir
               ſofort, da ich der \textsc{Mad. Yvette\pwindex{Guilbert, Yvette 20.01.1865 – 03.02.1944@\textsc{Guilbert, Yvette} (20.01.1865 – 03.02.1944), \emph{Schauspielerin, Sängerin}|pw}} noch Beſcheid geben möchte, ſolange ſie hier\oindex{Berlin@\textbf{Berlin}|pwv} iſt.\pend
           \pstart
           Den \label{K_L03059-10v}\edtext{Roman\pwindex{Schnitzler, Arthur 15.05.1862 – 21.10.1931@\textsc{Schnitzler, Arthur} (15.05.1862 – 21.10.1931), \emph{Schriftsteller, Mediziner}!Frau Bertha Garlan. Roman15.1.1901 – 15.3.1901@\strich\emph{Frau Bertha Garlan. Roman} {[}15.1.1901 – 15.3.1901{]}|pwv}}{\lemma{\textnormal{\emph{Roman}}}\Cendnote{\textnormal{Arthur Schnitzler\pwindex{Schnitzler, Arthur 15.05.1862 – 21.10.1931@\textsc{Schnitzler, Arthur} (15.05.1862 – 21.10.1931), \emph{Schriftsteller, Mediziner}|pwk}: \emph{Frau Bertha Garlan. Roman}\pwindex{Schnitzler, Arthur 15.05.1862 – 21.10.1931@\textsc{Schnitzler, Arthur} (15.05.1862 – 21.10.1931), \emph{Schriftsteller, Mediziner}!Frau Bertha Garlan. Roman15.1.1901 – 15.3.1901@\strich\emph{Frau Bertha Garlan. Roman} {[}15.1.1901 – 15.3.1901{]}|pwk}. In: \emph{Neue Deutsche Rundschau}\pwindex{Neue Deutsche Rundschau1894-01-01 – 1903-12-31@\emph{Neue Deutsche Rundschau} {[}1894-01-01 – 1903-12-31{]}|pwk}, Jg. 12, H. 1, Januar 1901, S. 41–64; H. 2, Februar 1901, S. 181–206; H. 3, März
                        1901, S. 237–272.}}}\label{K_L03059-10h} in der N. D. Rundſchau\pwindex{Neue Deutsche Rundschau1894-01-01 – 1903-12-31@\emph{Neue Deutsche Rundschau} {[}1894-01-01 – 1903-12-31{]}|pw} leſe ich nicht, weil ich mir das Werk\pwindex{Schnitzler, Arthur 15.05.1862 – 21.10.1931@\textsc{Schnitzler, Arthur} (15.05.1862 – 21.10.1931), \emph{Schriftsteller, Mediziner}!Frau Bertha Garlan. Roman15.1.1901 – 15.3.1901@\strich\emph{Frau Bertha Garlan. Roman} {[}15.1.1901 – 15.3.1901{]}|pwv} nicht will in
               Fortſetzungen zerhacken laſſen. Sehr reizend war der \label{K_L03059-11v}\edtext{Dialog\pwindex{Schnitzler, Arthur 15.05.1862 – 21.10.1931@\textsc{Schnitzler, Arthur} (15.05.1862 – 21.10.1931), \emph{Schriftsteller, Mediziner}!Sylvesternacht. Ein Dialog1901-02-18@\strich\emph{Sylvesternacht. Ein Dialog} {[}1901-02-18{]}|pwv}}{\lemma{\textnormal{\emph{Dialog}}}\Cendnote{\textnormal{Arthur Schnitzler\pwindex{Schnitzler, Arthur 15.05.1862 – 21.10.1931@\textsc{Schnitzler, Arthur} (15.05.1862 – 21.10.1931), \emph{Schriftsteller, Mediziner}|pwk}: \emph{Sylvesternacht. Ein Dialog}\pwindex{Schnitzler, Arthur 15.05.1862 – 21.10.1931@\textsc{Schnitzler, Arthur} (15.05.1862 – 21.10.1931), \emph{Schriftsteller, Mediziner}!Sylvesternacht. Ein Dialog1901-02-18@\strich\emph{Sylvesternacht. Ein Dialog} {[}1901-02-18{]}|pwk}. In: \emph{Jugend}\pwindex{?? Werk@Nicht ermittelte Verfasserinnen und Verfasser!Jugend1896 – 1940@\emph{Jugend} {[}1896 – 1940{]}|pwk}, Jg. 6, Nr. 8, 18. 2. 1901, S. 118–119, 121–122.}}}\label{K_L03059-11h} in der »Jugend\pwindex{?? Werk@Nicht ermittelte Verfasserinnen und Verfasser!Jugend1896 – 1940@\emph{Jugend} {[}1896 – 1940{]}|pw}«. Weniger gefallen hat mir der »\label{K_L03059-12v}\edtext{Blinde \textsc{Hieronymo}\pwindex{Schnitzler, Arthur 15.05.1862 – 21.10.1931@\textsc{Schnitzler, Arthur} (15.05.1862 – 21.10.1931), \emph{Schriftsteller, Mediziner}!blinde Geronimo und sein Bruder22.12.1900 – 12.1.1901@\strich\emph{Der blinde Geronimo und sein Bruder} {[}22.12.1900 – 12.1.1901{]}|pw}}{\lemma{\textnormal{\emph{Blinde Hieronymo}}}\Cendnote{\textnormal{Arthur Schnitzler\pwindex{Schnitzler, Arthur 15.05.1862 – 21.10.1931@\textsc{Schnitzler, Arthur} (15.05.1862 – 21.10.1931), \emph{Schriftsteller, Mediziner}|pwk}: \emph{Der blinde Geronimo und sein Bruder}\pwindex{Schnitzler, Arthur 15.05.1862 – 21.10.1931@\textsc{Schnitzler, Arthur} (15.05.1862 – 21.10.1931), \emph{Schriftsteller, Mediziner}!blinde Geronimo und sein Bruder22.12.1900 – 12.1.1901@\strich\emph{Der blinde Geronimo und sein Bruder} {[}22.12.1900 – 12.1.1901{]}|pwk}. In: \emph{Die Zeit}\pwindex{Zeit. Wiener Wochenschrift1894 – 1904@\emph{Die Zeit. Wiener Wochenschrift} {[}1894 – 1904{]}|pwk}, Bd. 25, Nr. 325, 22. 12. 1900, S. 190–191; Nr. 326, 29. 12. 1900, S. 207–208; , Bd. 26, Nr. 327, 5. 1. 1901, S. 15–16; Nr. 328, 12. 1. 1901, S. 31–32.}}}\label{K_L03059-12h}«. Die
               Geſchichte iſt geiſtvoll ausgedacht, bleibt aber weit zurück hinter der wilden Tragik
               des \label{K_L03059-13v}\edtext{Originals}{\lemma{\textnormal{\emph{Originals}}}\Cendnote{\textnormal{siehe Paul Goldmann an Arthur Schnitzler, 28. 8. [1900]}}}\label{K_L03059-13h}.\pend
           \pstart
           \textsc{Richard\pwindex{Beer-Hofmann, Richard 1866-07-11 – 1945-09-26@\textsc{Beer-Hofmann, Richard} (1866-07-11 – 1945-09-26), \emph{Schriftsteller}|pw}} hat mir nicht geſchrieben. Sag’ ihm auch nichts mehr. Der Teufel ſoll ihn
               holen!\pend
           \pstart
           Viele treue Grüße! {\\[\baselineskip]}Dein \spacefill\mbox{Paul Goldmann.}\pend
           \leftskip=0em{}
         
         \endnumbering\mylabel{h}\end{ledgroupsized}  \newcommand{\dateiname}{L03059}\newcommand{\titel}{Paul Goldmann an Arthur Schnitzler, 18. 2. [1901]}\newcommand{\editorInnen}{Martin Anton Müller und Laura Untner}%% latex-leseansicht-abspann.tex
%% Abspann für die Leseansicht.
%% Der Schalter \ifkorrekturansicht ist bereits durch den Vorspann gesetzt.

%% latex-abspann.tex
%% Gemeinsamer Abspann für Korrekturansicht und Leseansicht.
%% Setzt den Schalter \ifkorrekturansicht voraus (gesetzt in den
%% einbindenden Dateien latex-korrekturansicht-abspann.tex bzw.
%% latex-leseansicht-abspann.tex).
%% ---------------------------------------------------------------

\normalsize

% Das esempio-Environment wird nur in der Leseansicht benötigt
\ifkorrekturansicht\else
\newenvironment{esempio}[3]%
{
    \vspace{1.5ex}
    \rlap{\underline{#1}}
    \par
    \setlength{\parindent}{0cm}
    \nopagebreak
    \leftskip=#2cm
    \rightskip=#3cm
}
{
    \par
}
\fi

\doendnotes{C}
\bigskip
\vfill

\clearpage

\footnotesize

\ifkorrekturansicht
  \lohead{\textsc{register}}
\fi

% theindex-Environment neu definieren ohne reledmac
\makeatletter
\renewenvironment{theindex}{%
  \ifkorrekturansicht
    \section*{\indexname}%
  \else
    \subsubsection*{Index der erwähnten Entitäten}%
  \fi
  \setlength{\parindent}{0pt}%
  \setlength{\parskip}{0pt plus 0.3pt}%
  \let\item\@idxitem
}{%
  \ifkorrekturansicht\clearpage\fi
}
\makeatother

\IfFileExists{\jobname-pw.ind}{\input{\jobname-pw.ind}}{}

% Quellenangabe nur in der Leseansicht
\ifkorrekturansicht\else
% Fallback-Definitionen, falls die .tex-Datei \titel etc. nicht gesetzt hat
\providecommand{\titel}{}
\providecommand{\editorInnen}{}
\providecommand{\dateiname}{\jobname}

\vspace{3cm}

\vfill

\footnotesize
\textsc{Quelle}: \titel. Herausgegeben von {\editorInnen}. In: \emph{Arthur Schnitzler: Briefwechsel mit Autorinnen und Autoren}.
 Digitale Edition, https://schnitzler-briefe.acdh.oeaw.ac.at/{\dateiname}.html (Stand \today)
\fi

\end{document}


      