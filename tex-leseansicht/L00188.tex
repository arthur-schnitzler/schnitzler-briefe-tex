%% latex-leseansicht-vorspann.tex
%% Vorspann für die Leseansicht.
%% Lädt die gemeinsame Datei latex-vorspann.tex mit nicht gesetztem Schalter.

\newif\ifkorrekturansicht
\korrekturansichtfalse

\input{../tex-inputs/latex-vorspann}


\section[Arthur Schnitzler an Richard Beer-Hofmann, 8. 3. 1893]{L00188 Arthur Schnitzler an Richard Beer-Hofmann, 8. 3. 1893}
\nopagebreak\mylabel{L00188v}
\rehead{ }\normalsize\beginnumbering\briefempfaengerindex{Beer-Hofmann, Richard@\textsc{Beer-Hofmann, Richard}!zzzSchnitzler, Arthur@\emph{von Arthur Schnitzler}!1893-03-081@{8. 3. 1893}|(be}
\toendnotes[C]{\smallbreak\pagebreak[2]}
\correspDesc{Versand  durch Arthur Schnitzler am 8. 3. 1893 in Opatija
\newline{}Erhalt  durch Richard Beer-Hofmann im Zeitraum [8. 3. 1893 – 12. 3. 1893?] \textbf{Ort fehlend} }\toendnotes[C]{\smallbreak}
\Standort{YCGL, MSS 31.}
\physDesc{Brief, 1 Blatt, 4 Seiten, Kuvert, 971 Zeichen
\newline{}Handschrift: lila Tinte, deutsche Kurrent
\newline{}Versand: 1) Stempel: »\nobreak{}\oindex{Opatija@\textbf{Opatija}, \emph{Hauptstadt}|pwk}Abbazia, 9 3 93\nobreak{}«.   2) Stempel: »\nobreak{}10/3. 93, 11½V–1N\nobreak{}«. }
\buchAbdrucke{\weitereDrucke{Arthur Schnitzler, Richard Beer-Hofmann: \emph{Briefwechsel 1891–1931}. Herausgegeben von Konstanze Fliedl. Wien, Zürich: \emph{Europaverlag} 1992, S. 43.} }\toendnotes[C]{\smallbreak}\pstart{}{\pb}\textsc{Herrn Doctor Richard Beer Hofmann}\pend{}\pstart{}\textsc{Wien\oindex{Wien@\textbf{Wien}, \emph{Verwaltungsgebiet}|pw}.}\pend{}\pstart{}\textsc{I Wollzeile 15\oindex{Wien@\textbf{Wien}!I., Innere Stadt@\textbf{I., Innere Stadt}!Wollzeile 15 (»Berthahof«)@\textbf{Wollzeile 15 (»Berthahof«)}, \emph{Wohngebäude}|pw}.}\pend{}{\bigskip}\vspace{1em}
\pstart{}{\pb}Lieber Richard,\pend\vspace{0.5em}
\pstart
           ich habe eine Bitte an Sie. Wollen Sie die Liebenswürdigkeit haben, mir für \uuline{So{\geminationn}tag} Abend einen Sitz\pwindex{\textcolor{red}{\textsuperscript{XXXX indx1}}!Aus der Vorstadt. Volksstück mit Gesang in 3 Acten@\strich\emph{Aus der Vorstadt. Volksstück mit Gesang in 3 Acten}|pwv}\pwindex{\textcolor{red}{\textsuperscript{XXXX indx1}}!Aus der Vorstadt. Volksstück mit Gesang in 3 Acten@\strich\emph{Aus der Vorstadt. Volksstück mit Gesang in 3 Acten}|pwv}
                    ins Volkstheater\oindex{Wien@\textbf{Wien}!VII., Neubau@\textbf{VII., Neubau}!Volkstheater@\textbf{Volkstheater}, \emph{Theater}|pw} zu beſorgen? Gern ginge ich
                    mit Ihnen, Sie werden aber wohl \label{K_L00188-1v}\edtext{Samſtag}{\lemma{\textnormal{\emph{Samstag}}}\Cendnote{\textnormal{Die Uraufführung von \emph{Aus der
                                Vorstadt}\pwindex{\textcolor{red}{\textsuperscript{XXXX indx1}}!Aus der Vorstadt. Volksstück mit Gesang in 3 Acten@\strich\emph{Aus der Vorstadt. Volksstück mit Gesang in 3 Acten}|pwk}\pwindex{\textcolor{red}{\textsuperscript{XXXX indx1}}!Aus der Vorstadt. Volksstück mit Gesang in 3 Acten@\strich\emph{Aus der Vorstadt. Volksstück mit Gesang in 3 Acten}|pwk}\eventindex{Volkstheater@\textbf{Volkstheater}!Uraufführung von Aus der Vorstadt, 11.3.1893@Uraufführung von Aus der Vorstadt, 11.3.1893|pwk} fand am 11. 3. 1893 statt. }}}\label{K_L00188-1} gehn? – Vielleicht{ }ſitzt \textsc{Loris}\pwindex{Hofmannsthal, Hugo von 1.\,2.\,1874 Wien – 15.\,7.\,1929 Rodaun@\textsc{Hofmannsthal, Hugo von} (1.\,2.\,1874 Wien – 15.\,7.\,1929 Rodaun), \emph{Schriftsteller}|pw} oder {\pb}\textsc{Salten}\pwindex{Salten, Felix 6.\,9.\,1869 Budapest – 8.\,10.\,1945 Zürich@\textsc{Salten, Felix} (6.\,9.\,1869 Budapest – 8.\,10.\,1945 Zürich), \emph{Schriftsteller, Journalist, Chefredakteur}|pw}{ }\introOben{}oder \textsc{Schwarzkopf}\pwindex{Schwarzkopf, Gustav 7.\,11.\,1853 Wien – 13.\,11.\,1939 ebd.@\textsc{Schwarzkopf, Gustav} (7.\,11.\,1853 Wien – 13.\,11.\,1939 ebd.), \emph{Schriftsteller}|pw}\introOben{} an meiner Seite? –\pend
           
\pstart
           Daſs ich den Sitz am liebsten Mittelgang Ecke, 1, 2, 3, oder 4. Reihe hätte,
                    brauch ich Ihnen nicht zu verſichern. – Finde ich ihn nicht bei mir,{ }ſo{ }ſchmeichle ich mir mit der Hoffnung, daſs Sie ihn mir am So{\geminationn}tag{ }Nachmittag um 5 Uhr perſönlich überbringen wollen; jedenfalls würde
                    ich {\pb}mich{ }ſehr freuen, Sie und die oben
                    genannten, wenn Ihr nichts beſſres vorhabt, auf eine Stunde bei mir zu{ }ſehn.
                        So{\geminationn}tag früh komm ich nämlich an.\pend
           
\pstart
           Herzliche Grüße und entſchuldigen Sie die Mühe gütigſt! – Grüßen Sie mir auch die
                    andern! Ich befinde mich{ }ſehr wohl – {\pb}es iſt kein
                    leerer Wahn, – was kein leerer Wahn, folgt mündlich.\pend
           
\pstart
           Der Ihrige herzlichſt {\\[\baselineskip]}\spacefill\mbox{Arthur}\pend
           \leftskip=0em{}
\pstart
           \textsc{Abbazia}\oindex{Opatija@\textbf{Opatija}, \emph{Hauptstadt}|pw}, 8. 3. 93.\pend
           \selectlanguage{ngerman}\endnumbering\briefempfaengerindex{Beer-Hofmann, Richard@\textsc{Beer-Hofmann, Richard}!zzzSchnitzler, Arthur@\emph{von Arthur Schnitzler}!1893-03-081@{8. 3. 1893}|)be}\mylabel{L00188h}  \newcommand{\dateiname}{L00188}\newcommand{\titel}{Arthur Schnitzler an Richard Beer-Hofmann, 8. 3. 1893}\newcommand{\editorInnen}{Martin Anton Müller und Gerd-Hermann Susen}%% latex-leseansicht-abspann.tex
%% Abspann für die Leseansicht.
%% Der Schalter \ifkorrekturansicht ist bereits durch den Vorspann gesetzt.

%% latex-abspann.tex
%% Gemeinsamer Abspann für Korrekturansicht und Leseansicht.
%% Setzt den Schalter \ifkorrekturansicht voraus (gesetzt in den
%% einbindenden Dateien latex-korrekturansicht-abspann.tex bzw.
%% latex-leseansicht-abspann.tex).
%% ---------------------------------------------------------------

\normalsize

% Das esempio-Environment wird nur in der Leseansicht benötigt
\ifkorrekturansicht\else
\newenvironment{esempio}[3]%
{
    \vspace{1.5ex}
    \rlap{\underline{#1}}
    \par
    \setlength{\parindent}{0cm}
    \nopagebreak
    \leftskip=#2cm
    \rightskip=#3cm
}
{
    \par
}
\fi

\doendnotes{C}
\bigskip
\vfill

\clearpage

\footnotesize

\ifkorrekturansicht
  \lohead{\textsc{register}}
\fi

% theindex-Environment neu definieren ohne reledmac
\makeatletter
\renewenvironment{theindex}{%
  \ifkorrekturansicht
    \section*{\indexname}%
  \else
    \subsubsection*{Index der erwähnten Entitäten}%
  \fi
  \setlength{\parindent}{0pt}%
  \setlength{\parskip}{0pt plus 0.3pt}%
  \let\item\@idxitem
}{%
  \ifkorrekturansicht\clearpage\fi
}
\makeatother

\IfFileExists{\jobname-pw.ind}{\input{\jobname-pw.ind}}{}

% Quellenangabe nur in der Leseansicht
\ifkorrekturansicht\else
% Fallback-Definitionen, falls die .tex-Datei \titel etc. nicht gesetzt hat
\providecommand{\titel}{}
\providecommand{\editorInnen}{}
\providecommand{\dateiname}{\jobname}

\vspace{3cm}

\vfill

\footnotesize
\textsc{Quelle}: \titel. Herausgegeben von {\editorInnen}. In: \emph{Arthur Schnitzler: Briefwechsel mit Autorinnen und Autoren}.
 Digitale Edition, https://schnitzler-briefe.acdh.oeaw.ac.at/{\dateiname}.html (Stand \today)
\fi

\end{document}


