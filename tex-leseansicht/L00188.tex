%% latex-korrekturansicht-vorspann.tex
%% Vorspann für die Korrekturansicht.
%% Lädt die gemeinsame Datei latex-vorspann.tex mit gesetztem Schalter.

\newif\ifkorrekturansicht
\korrekturansichttrue

\input{../tex-inputs/latex-vorspann}


\section[Arthur Schnitzler an Richard Beer-Hofmann, 8. 3. 1893]{L00188 Arthur Schnitzler an Richard Beer-Hofmann, 8. 3. 1893}
\nopagebreak\mylabel{L00188v}
\rehead{ }\normalsize\beginnumbering\briefempfaengerindex{Beer-Hofmann, Richard@\textsc{Beer-Hofmann, Richard}!zzzSchnitzler, Arthur@\emph{von Arthur Schnitzler}!1893-03-081@{8. 3. 1893}|(be}
\toendnotes[C]{\smallbreak\pagebreak[2]}\Standort{YCGL, MSS 31.}
\physDesc{Brief, 1 Blatt, 4 Seiten, Umschlag, 971 Zeichen
\newline{}Handschrift: blaue Tinte, deutsche Kurrent
\newline{}Versand: 1) Stempel: »\nobreak{}\oindex{Opatija@\textbf{Opatija}, \emph{P.PPLA2}|pwk}Abbazia, 9 3 93\nobreak{}«.   2) Stempel: »\nobreak{}10/3. 93, 11½V–1N\nobreak{}«. }
\buchAbdrucke{\weitereDrucke{Arthur Schnitzler, Richard Beer-Hofmann: \emph{Briefwechsel 1891–1931}. Wien, Zürich: \emph{Europaverlag} 1992, S. 43.} }\toendnotes[C]{\smallbreak}\pstart{}{\pb}\textsc{Herrn Doctor Richard Beer Hofmann}\pend{}\pstart{}\textsc{Wien\oindex{Wien@\textbf{Wien}, \emph{A.ADM2}|pw} . }\pend{}\pstart{}\textsc{I Wollzeile 15.\oindex{Wollzeile@\textbf{Wollzeile}, \emph{Straße (K.STR)}|pw}} . \pend{}{\bigskip}\vspace{1em}
\pstart{}{\pb}Lieber Richard, \pend\vspace{0.5em}
\pstart
            ich habe eine Bitte an Sie. Wollen Sie die Liebenswürdigkeit haben, mir für \uuline{ So {\geminationn} tag } Abend einen Sitz\pwindex{Aus der Vorstadt. Volksstueck mit Gesang in 3 Acten@\emph{Aus der Vorstadt. Volksstück mit Gesang in 3 Acten}|pwv}
                    ins Volkstheater\oindex{Volkstheater@\textbf{Volkstheater}, \emph{Theater (K.THE)}|pw} zu beſorgen? Gern ginge ich
                    mit Ihnen, Sie werden aber wohl \label{K_L00188-1v}\edtext{Samſtag}{\lemma{\textnormal{\emph{Samſtag}}}\Cendnote{\textnormal{Die \emph{Uraufführung von \emph{Aus der
                                Vorstadt}\pwindex{Aus der Vorstadt. Volksstueck mit Gesang in 3 Acten@\emph{Aus der Vorstadt. Volksstück mit Gesang in 3 Acten}|pwk}}\eventindex{Volkstheater@\textbf{Volkstheater}!Urauffuehrung von Aus der Vorstadt, 11.3.1893@Uraufführung von Aus der Vorstadt, 11.3.1893|pwk} fand am 11. 3. 1893 statt. }}}\label{K_L00188-1}
                    gehn? – Vielleicht ſitzt \textsc{Loris}\pwindex{Hofmannsthal, Hugo von 1874-02-01 – 1929-07-15@\textsc{Hofmannsthal, Hugo von} (1874-02-01 – 1929-07-15), \emph{Schriftsteller/Schriftstellerin}|pw} oder {\pb}\textsc{Salten}\pwindex{Salten, Felix 06.09.1869 – 08.10.1945@\textsc{Salten, Felix} (06.09.1869 – 08.10.1945), \emph{Schriftsteller/Schriftstellerin, Journalist/Journalistin, Chefredakteur/Chefredakteurin}|pw}{ }\introOben{} oder \textsc{Schwarzkopf}\pwindex{Schwarzkopf, Gustav 07.11.1853 – 13.11.1939@\textsc{Schwarzkopf, Gustav} (07.11.1853 – 13.11.1939), \emph{Schriftsteller/Schriftstellerin}|pw}\introOben{} an meiner Seite? – \pend
           
\pstart
            Daſs ich den Sitz am liebsten Mittelgang Ecke, 1, 2, 3, oder 4. Reihe hätte,
                    brauch ich Ihnen nicht zu verſichern. – Finde ich ihn nicht bei mir, ſo
                    ſchmeichle ich mir mit der Hoffnung, daſs Sie ihn mir am  So {\geminationn} tag { }Nachmittag um 5 Uhr perſönlich überbringen wollen; jedenfalls würde
                    ich {\pb}mich ſehr freuen, Sie und die oben
                    genannten, wenn Ihr nichts beſſres vorhabt, auf eine Stunde bei mir zu ſehn. So
                        {\geminationn} tag früh komm ich nämlich an. \pend
           
\pstart
            Herzliche Grüße und entſchuldigen Sie die Mühe gütigſt! – Grüßen Sie mir auch
                    die andern! Ich befinde mich ſehr wohl – {\pb}es iſt
                    kein leerer Wahn, – was kein leerer Wahn, folgt mündlich. \pend
           
\pstart
            Der Ihrige herzlichſt {\\[\baselineskip]}\spacefill\mbox{Arthur}\pend
           \leftskip=0em{}
\pstart
           \textsc{Abbazia}\oindex{Opatija@\textbf{Opatija}, \emph{P.PPLA2}|pw} , 8. 3. 93.\pend
           \selectlanguage{ngerman}\endnumbering\briefempfaengerindex{Beer-Hofmann, Richard@\textsc{Beer-Hofmann, Richard}!zzzSchnitzler, Arthur@\emph{von Arthur Schnitzler}!1893-03-081@{8. 3. 1893}|)be}\mylabel{L00188h}  \normalsize

\doendnotes{C}
\bigskip
\vfill

\clearpage

\footnotesize

\lohead{\textsc{register}}

% Definiere theindex-Environment komplett neu ohne reledmac
\makeatletter
\renewenvironment{theindex}{%
  \section*{\indexname}%
  \setlength{\parindent}{0pt}%
  \setlength{\parskip}{0pt plus 0.3pt}%
  \let\item\@idxitem
}{%
  \clearpage
}
\makeatother

\IfFileExists{\jobname-pw.ind}{\input{\jobname-pw.ind}}{}

\end{document}

      