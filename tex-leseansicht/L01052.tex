%% latex-korrekturansicht-vorspann.tex
%% Vorspann für die Korrekturansicht.
%% Lädt die gemeinsame Datei latex-vorspann.tex mit gesetztem Schalter.

\newif\ifkorrekturansicht
\korrekturansichttrue

\input{../tex-inputs/latex-vorspann}


\section[Arthur Schnitzler an Richard Beer-Hofmann, 7. 7. 1900]{L01052 Arthur Schnitzler an Richard Beer-Hofmann, 7. 7. 1900}
\nopagebreak\mylabel{L01052v}
\rehead{ }\normalsize\beginnumbering\briefempfaengerindex{Beer-Hofmann, Richard@\textsc{Beer-Hofmann, Richard}!zzzSchnitzler, Arthur@\emph{von Arthur Schnitzler}!1900-07-071@{7. 7. 1900}|(be}
\toendnotes[C]{\smallbreak\pagebreak[2]}\Standort{YCGL, MSS 31.}
\physDesc{Brief, 1 Blatt, 4 Seiten, Umschlag, 931 Zeichen
\newline{}Handschrift: Bleistift, deutsche Kurrent
\newline{}Versand: 1) Stempel: »\nobreak{}\oindex{Reichenau an der Rax@\textbf{Reichenau an der Rax}, \emph{A.ADM3}|pwk}Reichenau N.Ö., 8 \textcolor{gray}{7} 00\nobreak{}«.   2) Stempel: »\nobreak{}\oindex{Altaussee@\textbf{Altaussee}, \emph{A.ADM3}|pwk}{\pb}Alt-Aussee, 8 7 00\nobreak{}«. }
\buchAbdrucke{\weitereDrucke{Arthur Schnitzler, Richard Beer-Hofmann: \emph{Briefwechsel 1891–1931}. Wien, Zürich: \emph{Europaverlag} 1992, S. 147.} }\toendnotes[C]{\smallbreak}\pstart{}{\pb}Herrn \textsc{Dr. Richard}\pend{}\pstart{}\textsc{Beer-Hofmann}\pend{}\pstart{}\textsc{Altaussee\oindex{Altaussee@\textbf{Altaussee}, \emph{A.ADM3}|pw}.}\pend{}\pstart{}\textsc{Steiermark\oindex{Steiermark@\textbf{Steiermark}, \emph{A.ADM1}|pw}}.\pend{}{\bigskip}\vspace{1em}
\pstart{}{\pb}lieber Richard,\pend\vspace{0.5em}
\pstart
           Danke für den nachgeſandten Brief, hier die Revanche. Wie geht es Ihrer Frau\pwindex{Beer-Hofmann, Paula 25.02.1879 – 30.10.1939@\textsc{Beer-Hofmann, Paula} (25.02.1879 – 30.10.1939)|pwv}? Schreiben Sie mir das
               hieher, Reichenau, Curhaus\oindex{Kurhaus Rudolfsbad@\textbf{Kurhaus Rudolfsbad}, \emph{Sanatorium (K.SAN)}|pw}. \label{K_L01052-1v}\edtext{Paul\pwindex{Goldmann, Paul 31.01.1865 – 25.09.1935@\textsc{Goldmann, Paul} (31.01.1865 – 25.09.1935), \emph{Schriftsteller/Schriftstellerin, Journalist/Journalistin}|pw} iſt mit dem 15. Auguſt, Innsbruck\oindex{Innsbruck@\textbf{Innsbruck}, \emph{A.ADM2}|pw} einverſtanden}{\lemma{\textnormal{\emph{Paul … einverſtanden}}}\Cendnote{\textnormal{Siehe Paul Goldmann an Arthur Schnitzler, 5. 7. [1900].
               }}}\label{K_L01052-1}, Kerr\pwindex{Kerr, Alfred 25.12.1867 – 12.10.1948@\textsc{Kerr, Alfred} (25.12.1867 – 12.10.1948), \emph{Schriftsteller/Schriftstellerin, Kritiker/Kritikerin}|pw} wohl auch; wir könnten nun die
               Sache bald {\pb}endgiltig fixiren. Ich ſehe Sie wohl noch
               Anfang Auguſt, entweder in Iſchl\oindex{Bad Ischl@\textbf{Bad Ischl}, \emph{P.PPL}|pw} oder in Auſſee\oindex{Bad Aussee@\textbf{Bad Aussee}, \emph{P.PPLA3}|pw}; oder Salzburg\oindex{Salzburg@\textbf{Salzburg}, \emph{A.ADM2}|pw}. Hier bleibe ich wahrſcheinlich 10–14 Tage. Dann? – Die paar Tage
               zwiſchen Altauſſee\oindex{Altaussee@\textbf{Altaussee}, \emph{A.ADM3}|pw} und Reichenau\oindex{Reichenau an der Rax@\textbf{Reichenau an der Rax}, \emph{A.ADM3}|pw} waren ganz anſprechend. (Wir lieben die Frauen, die uns gleichgiltig
                  ſind\pwindex{Liebelei. Schauspiel in drei Akten@\emph{Liebelei. Schauspiel in drei Akten}|pwv}{ }\textsc{etc}.) – Ich entwerfe {\pb}immerfort an dem Fünfactigen\pwindex{Weg ins Freie. Roman@\emph{Der Weg ins Freie. Roman}|pwv}
               herum. (Die Entrüſteten\pwindex{Weg ins Freie. Roman@\emph{Der Weg ins Freie. Roman}|pwv} wird
               es nicht heißen, da bisher kein Entrüſteter drin vorko{\geminationm}t; der beſte Titel wäre eine Geſte, mit dem Begleitton: Tz, – aber nicht ſo
               jüdiſch, wie das letzte Capitel von Georgs
               Tod\pwindex{Tod Georgs@\emph{Der Tod Georgs}|pw}.) ((An dieſer Stelle wird der Co{\geminationm}entator unſres
               Briefwechſels irrſinnig werden.))\pend
           
\pstart
           {\pb}Leben Sie wohl.\pend
           
\pstart
           Von Herzen Ihr{\\[\baselineskip]}\spacefill\mbox{Arthur}\pend
           \leftskip=0em{}
\pstart
           7. 7. 900.\pend
           \selectlanguage{ngerman}\endnumbering\briefempfaengerindex{Beer-Hofmann, Richard@\textsc{Beer-Hofmann, Richard}!zzzSchnitzler, Arthur@\emph{von Arthur Schnitzler}!1900-07-071@{7. 7. 1900}|)be}\mylabel{L01052h}  \normalsize

\doendnotes{C}
\bigskip
\vfill

\clearpage

\footnotesize

\lohead{\textsc{register}}

% Definiere theindex-Environment komplett neu ohne reledmac
\makeatletter
\renewenvironment{theindex}{%
  \section*{\indexname}%
  \setlength{\parindent}{0pt}%
  \setlength{\parskip}{0pt plus 0.3pt}%
  \let\item\@idxitem
}{%
  \clearpage
}
\makeatother

\IfFileExists{\jobname-pw.ind}{\input{\jobname-pw.ind}}{}

\end{document}

      