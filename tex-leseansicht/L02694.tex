%% latex-korrekturansicht-vorspann.tex
%% Vorspann für die Korrekturansicht.
%% Lädt die gemeinsame Datei latex-vorspann.tex mit gesetztem Schalter.

\newif\ifkorrekturansicht
\korrekturansichttrue

\input{../tex-inputs/latex-vorspann}


\section[Paul Goldmann an Arthur Schnitzler, {[}21. 12. 1895?{]}]{L02694 Paul Goldmann an Arthur Schnitzler, {[}21. 12. 1895?{]}}
\nopagebreak\mylabel{L02694v}
\rehead{ }\normalsize\beginnumbering\briefempfaengerindex{Schnitzler, Arthur@\textsc{Schnitzler, Arthur}!zzzGoldmann, Paul@\emph{von Paul Goldmann}!1895-12-211@{{[}21. 12. 1895?{]}}|(be}
\toendnotes[C]{\smallbreak\pagebreak[2]}\Standort{DLA, A:Schnitzler, HS.NZ85.1.3165.}
\physDesc{Telegramm, 92 Zeichen
\newline{}maschinell
\newline{}Versand: mit Bleistift vier nicht entzifferte Zeichen: »\textcolor{gray}{×}\-\textcolor{gray}{×}{ }\textcolor{gray}{×}\-\textcolor{gray}{×}« 
\newline{}Ordnung: beschnitten }\toendnotes[C]{\smallbreak}
\pstart
           \centering{}{\pb}w\oindex{Wien@\textbf{Wien}, \emph{A.ADM2}|pw}{ }paris\oindex{Paris@\textbf{Paris}, \emph{P.PPLC}|pw} 22598 15 { }1 28 :=\pend
           \vspace{0.5em}
\pstart
           \label{K_L02694-1v}\edtext{geld aus brief gestohlen}{\lemma{\textnormal{\emph{geld aus brief gestohlen}}}\Cendnote{\textnormal{Die Datierung erfolgt mit dem
                  nachgesandten Brief (Paul Goldmann an Arthur Schnitzler, 21. 12. [1895]), in dem
                  einerseits der eben stattgefundene Erhalt von Schnitzlers geöffnetem Brief thematisiert, andererseits auf dieses
                  Telegramm Bezug genommen wird.}}}\label{K_L02694-1} reclamire sofort postdirection\pend
           \pstart gruss \spacefill\mbox{goldmann =}\pend{}\selectlanguage{ngerman}\endnumbering\briefempfaengerindex{Schnitzler, Arthur@\textsc{Schnitzler, Arthur}!zzzGoldmann, Paul@\emph{von Paul Goldmann}!1895-12-211@{{[}21. 12. 1895?{]}}|)be}\mylabel{L02694h}  \normalsize

\doendnotes{C}
\bigskip
\vfill

\clearpage

\footnotesize

\lohead{\textsc{register}}

% Definiere theindex-Environment komplett neu ohne reledmac
\makeatletter
\renewenvironment{theindex}{%
  \section*{\indexname}%
  \setlength{\parindent}{0pt}%
  \setlength{\parskip}{0pt plus 0.3pt}%
  \let\item\@idxitem
}{%
  \clearpage
}
\makeatother

\IfFileExists{\jobname-pw.ind}{\input{\jobname-pw.ind}}{}

\end{document}

      