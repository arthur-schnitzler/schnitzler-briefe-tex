%% latex-korrekturansicht-vorspann.tex
%% Vorspann für die Korrekturansicht.
%% Lädt die gemeinsame Datei latex-vorspann.tex mit gesetztem Schalter.

\newif\ifkorrekturansicht
\korrekturansichttrue

\input{../tex-inputs/latex-vorspann}


\section[ Paul Goldmann an Arthur Schnitzler, 4. 10. {[}1900{]}]{L02934 Paul Goldmann an Arthur Schnitzler, 4. 10. {[}1900{]}}
\nopagebreak\mylabel{L02934v}
\rehead{ }\normalsize\beginnumbering\briefempfaengerindex{Schnitzler, Arthur@\textsc{Schnitzler, Arthur}!zzzGoldmann, Paul@\emph{von Paul Goldmann}!1900-10-042@{4. 10. {[}1900{]}}|(be}
\toendnotes[C]{\smallbreak\pagebreak[2]}\Standort{DLA, A:Schnitzler, HS.NZ85.1.3170.}
\physDesc{Brief, 1 Blatt, 4 Seiten, 1306 Zeichen
\newline{}Handschrift: blaue Tinte, deutsche Kurrent
\newline{}Beilage: drei aufgeklebte, beschnittene Zeitungsausschnitte 
\newline{}Schnitzler: 1) mit Bleistift das Jahr »900« vermerkt  2) mit rotem Buntstift drei Unterstreichungen}\toendnotes[C]{\smallbreak}
\pstart
           {\pb}Berlin\oindex{Berlin@\textbf{Berlin}, \emph{P.PPLC}|pw}, 4. Oktober.\hfill \textcolor{gray}{\textbf{DESSAUERSTRASSE 19}}\pend
           
\pstart\center{}Mein lieber Freund,\pend\vspace{0.5em}
\pstart
           Ich danke Dir von Herzen für Deine lieben Briefe, insbeſondere für den wunderſchönen
               von neulich, den \strikeout{d\textcolor{gray}{×}\-\textcolor{gray}{×}} ich ausführlich beantworten werde, ſobald ich Zeit finde.\pend
           
\pstart
           Die zweite Auflage\pwindex{Sommer in China. Reisebilder. Zweite, durchgesehene und vermehrte Auflage@\emph{Ein Sommer in China. Reisebilder. Zweite, durchgesehene und vermehrte Auflage}|pwv} meines Buch\pwindex{Sommer in China. Reisebilder@\emph{Ein Sommer in China. Reisebilder}|pwv}es erſcheint erſt \label{K_L02934-1v}\edtext{in einigen Wochen}{\lemma{\textnormal{\emph{in einigen Wochen}}}\Cendnote{\textnormal{Die zweite
                     Auflage\pwindex{Sommer in China. Reisebilder. Zweite, durchgesehene und vermehrte Auflage@\emph{Ein Sommer in China. Reisebilder. Zweite, durchgesehene und vermehrte Auflage}|pwkv} von \emph{Ein Sommer in China}\pwindex{Sommer in China. Reisebilder@\emph{Ein Sommer in China. Reisebilder}|pwk}
                  erschien am 22. 11. 1900.}}}\label{K_L02934-1}. Der Idiot von
                  \label{K_L02934-2v}\edtext{Verleger\pwindex{Oswalt, Wilhelm Ernst 1877-03-15 – 1942-06-30@\textsc{Oswalt, Wilhelm Ernst} (1877-03-15 – 1942-06-30), \emph{Verleger/Verlegerin}|pwuv}}{\lemma{\textnormal{\emph{Verleger}}}\Cendnote{\textnormal{vermutlich Wilhelm Ernst Oswalt\pwindex{Oswalt, Wilhelm Ernst 1877-03-15 – 1942-06-30@\textsc{Oswalt, Wilhelm Ernst} (1877-03-15 – 1942-06-30), \emph{Verleger/Verlegerin}|pwk} vom Frankfurt\oindex{Frankfurt am Main@\textbf{Frankfurt am Main}, \emph{P.PPLA3}|pwk}er Verlag \emph{Rütten {\kaufmannsund} Loening}\orgindex{Ruetten und Loening@Rütten {\kaufmannsund}  Loening|pwk}}}}\label{K_L02934-2} kann mit der Drucklegung nicht fertig werden. Selbſtverſtändlich geht ein Exemplar\pwindex{Sommer in China. Reisebilder. Zweite, durchgesehene und vermehrte Auflage@\emph{Ein Sommer in China. Reisebilder. Zweite, durchgesehene und vermehrte Auflage}|pwv} an die angegebene
               Adreſſe.\pend
           
\pstart
           {\pb}Geſtern hatten wir hier \label{K_L02934-3v}\edtext{»Roſenmontag\pwindex{Rosenmontag@\emph{Rosenmontag}|pw}« von \textsc{Hartleben\pwindex{Hartleben, Otto Erich 03.06.1864 – 11.02.1905@\textsc{Hartleben, Otto Erich} (03.06.1864 – 11.02.1905), \emph{Schriftsteller/Schriftstellerin}|pw}}}{\lemma{\textnormal{\emph{»Roſenmontag« von Hartleben}}}\Cendnote{\textnormal{im Deutschen Theater\oindex{Deutsches Theater Berlin@\textbf{Deutsches Theater Berlin}, \emph{Theater (K.THE)}|pwk}}}}\label{K_L02934-3}. \strikeout{\textcolor{gray}{W}}{ }\label{K_L02934-4v}\edtext{»Unſer \textsc{Otto Erich\pwindex{Hartleben, Otto Erich 03.06.1864 – 11.02.1905@\textsc{Hartleben, Otto Erich} (03.06.1864 – 11.02.1905), \emph{Schriftsteller/Schriftstellerin}|pw}}.\pwindex{Kritische Tagebuchblaetter@\emph{Kritische Tagebuchblätter}|pwuv}«}{\lemma{\textnormal{\emph{»Unſer Otto Erich.«}}}\Cendnote{\textnormal{zur stehenden Wendung
                  gewordene Phrase, die womöglich auf eine Rezension von Bahr\pwindex{Bahr, Hermann 19.07.1863 – 15.01.1934@\textsc{Bahr, Hermann} (19.07.1863 – 15.01.1934), \emph{Schriftsteller/Schriftstellerin, Kritiker/Kritikerin}|pwk} zurückgeht. Vgl. Hermann Bahr\pwindex{Bahr, Hermann 19.07.1863 – 15.01.1934@\textsc{Bahr, Hermann} (19.07.1863 – 15.01.1934), \emph{Schriftsteller/Schriftstellerin, Kritiker/Kritikerin}|pwk}: \emph{Die
                        Erziehung zur Ehe. (»Die Lore«, Plauderei in einem Act von Otto Erich
                        Hartleben; »Die Erziehung zur Ehe«, Satire in drei Acten von Otto Erich
                        Hartleben. Zum ersten Mal aufgeführt im Deutschen Volkstheater am 11.
                        September 1897)}\pwindex{Erziehung zur Ehe. (»Die Lore«, Plauderei in einem Act von Otto Erich Hartleben; »Die Erziehung zur Ehe«, Satire in drei Acten von Otto Erich Hartleben. Zum ersten Mal aufgefuehrt im Deutschen Volkstheater am 11. September 1897)@\emph{Die Erziehung zur Ehe. (»Die Lore«, Plauderei in einem Act von Otto Erich Hartleben; »Die Erziehung zur Ehe«, Satire in drei Acten von Otto Erich Hartleben. Zum ersten Mal aufgeführt im Deutschen Volkstheater am 11. September 1897)}|pwk}. In: \emph{Die Zeit}\pwindex{Zeit. Wiener Wochenschrift@\emph{Die Zeit. Wiener Wochenschrift}|pwk},
                     Jg. 12, Nr. 155, 18. 9. 1897, S. 188–189.}}}\label{K_L02934-4} Guter
               erſter Akt\pwindex{Rosenmontag@\emph{Rosenmontag}|pwv}. Sobald das \introOben{}eigentliche\introOben{}{ }Drama\pwindex{Rosenmontag@\emph{Rosenmontag}|pwv} anfängt, eine von \strikeout{A\textcolor{gray}{k}} Akt zu Akt troſtloſer werdende Unfähigkeit und Leere. So ein Burſch\pwindex{Hartleben, Otto Erich 03.06.1864 – 11.02.1905@\textsc{Hartleben, Otto Erich} (03.06.1864 – 11.02.1905), \emph{Schriftsteller/Schriftstellerin}|pwv} ohne \strikeout{Wärme} Wärme und Poeſie, der ſich als Dichter aufſpielt, weil es in der
               deutſchen Literatur zufällig an ſolchen mangelte!\pend
           
\pstart
           \textsc{Bahr\pwindex{Bahr, Hermann 19.07.1863 – 15.01.1934@\textsc{Bahr, Hermann} (19.07.1863 – 15.01.1934), \emph{Schriftsteller/Schriftstellerin, Kritiker/Kritikerin}|pw}} ſcheint auch ein liebes Stück\pwindex{Wienerinnen. Lustspiel in drei Akten@\emph{Wienerinnen. Lustspiel in drei Akten}|pwv} geſchrieben zu haben. Wir haben hier folgende Berichte\pwindex{Theater und Musik [Wienerinnen]@\emph{Theater und Musik [Wienerinnen]}|pwv}\pwindex{Theaterchronik [Wienerinnen]@\emph{Theaterchronik [Wienerinnen]}|pwv}\pwindex{Vor den Coulissen [Bahrs Wienerinnen]@\emph{Vor den Coulissen [Bahrs Wienerinnen]}|pwv}{ }{\pb}erhalten:\pend
           
\pstart
           \substVorne{}\textsuperscript{B\textcolor{gray}{er}}\substDazwischen{}Vo\pwindex{Vossische Zeitung@\emph{Vossische Zeitung}|pwv}\substHinten{}ſſiſche Zeitung\pwindex{Vossische Zeitung@\emph{Vossische Zeitung}|pwv}:\pend
           
\pstart
           \label{K_L02934-5v}\edtext{\textcolor{gray}{\textbf{Im \so{Deutſchen Volkstheater}\oindex{Volkstheater@\textbf{Volkstheater}, \emph{Theater (K.THE)}|pw} hatte heute ein neues Stück \textbf{»Die Wienerinnen\pwindex{Wienerinnen. Lustspiel in drei Akten@\emph{Wienerinnen. Lustspiel in drei Akten}|pw}«} von
                        \so{Hermann Bahr}\pwindex{Bahr, Hermann 19.07.1863 – 15.01.1934@\textsc{Bahr, Hermann} (19.07.1863 – 15.01.1934), \emph{Schriftsteller/Schriftstellerin, Kritiker/Kritikerin}|pw} einen durchſchlagenden Erfolg.}}\pwindex{Theater und Musik [Wienerinnen]@\emph{Theater und Musik [Wienerinnen]}|pwv}}{\lemma{\textnormal{\emph{Im … Erfolg.}}}\Cendnote{\textnormal{Auszug aus [O. V.]: \emph{Theater und Musik}\pwindex{Theater und Musik [Wienerinnen]@\emph{Theater und Musik [Wienerinnen]}|pwk}. In: \emph{Vossische Zeitung}\pwindex{Vossische Zeitung@\emph{Vossische Zeitung}|pwk}, Nr. 464, 4. 10. 1900, Morgen-Ausgabe, S. [16].
               }}}\label{K_L02934-5}\pend
           {\vspace{1\baselineskip}}
\pstart
           Berliner Tageblatt\pwindex{Berliner Tageblatt@\emph{Berliner Tageblatt}|pw}:\pend
           
\pstart
           \label{K_L02934-6v}\edtext{\textcolor{gray}{\textbf{Aus \textbf{Wien}\oindex{Wien@\textbf{Wien}, \emph{A.ADM2}|pw} meldet uns ein Privat-Telegramm: Hermann Bahrs\pwindex{Bahr, Hermann 19.07.1863 – 15.01.1934@\textsc{Bahr, Hermann} (19.07.1863 – 15.01.1934), \emph{Schriftsteller/Schriftstellerin, Kritiker/Kritikerin}|pw}s Luſtſpiel »\so{Wienerinnen}\pwindex{Wienerinnen. Lustspiel in drei Akten@\emph{Wienerinnen. Lustspiel in drei Akten}|pw}« hatte einen kompleten Mißerfolg.}}\pwindex{Theaterchronik [Wienerinnen]@\emph{Theaterchronik [Wienerinnen]}|pwv}}{\lemma{\textnormal{\emph{Aus … Mißerfolg.}}}\Cendnote{\textnormal{Auszug aus [O. V.]: \emph{Theaterchronik}\pwindex{Theaterchronik [Wienerinnen]@\emph{Theaterchronik [Wienerinnen]}|pwk}. In: \emph{Berliner Tageblatt}\pwindex{Berliner Tageblatt@\emph{Berliner Tageblatt}|pwk}, Jg. 29, Nr. 504, 4. 10. 1900, Morgen-Ausgabe,
                        S. {[}3{]}. }}}\label{K_L02934-6}\pend
           {\vspace{1\baselineskip}}
\pstart
           Dieſe \label{K_L02934-7v}\edtext{zwei Kritiker}{\lemma{\textnormal{\emph{zwei Kritiker}}}\Cendnote{\textnormal{nicht ermittelt}}}\label{K_L02934-7} ſcheinen das neue
                  Werk\pwindex{Wienerinnen. Lustspiel in drei Akten@\emph{Wienerinnen. Lustspiel in drei Akten}|pwv} von verſchiedenen
               Geſichtspunkten aus zu betrachten. Im »Börſencourier\pwindex{Berliner Boersen-Courier@\emph{Berliner Börsen-Courier}|pw}« aber ſchmückt \textsc{Siegfried Löwy\pwindex{Loewy, Siegfried 01.11.1857 – 08.05.1931@\textsc{Loewy, Siegfried} (01.11.1857 – 08.05.1931), \emph{Schriftsteller/Schriftstellerin, Journalist/Journalistin}|pw}} ſich folgendermaßen aus:\pend
           
\pstart
           \label{K_L02934-8v}\edtext{{\pb}\textcolor{gray}{\textbf{Das »süße Wien\oindex{Wien@\textbf{Wien}, \emph{A.ADM2}|pw}er Mädel«
                     iſt durch Arthur Schnitzler’s farbenſatte Schilderung mit ihrer ergreifenden
                     Wendung in’s Tragiſche in ſeiner ganzen Echtheit \label{K_L02934-9v}\edtext{in »Liebelei\pwindex{Liebelei. Schauspiel in drei Akten@\emph{Liebelei. Schauspiel in drei Akten}|pw}« zum
                     erſten Male auf die Bühne gebracht}{\lemma{\textnormal{\emph{in … gebracht}}}\Cendnote{\textnormal{Siehe zum Begriff »süßes Mädel« auch Paul Goldmann an Arthur Schnitzler, 4. 10. [1900].
                     }}}\label{K_L02934-9} worden, das Mädel aus dem Volke, die kleine, liebe \label{K_L02934-10v}\edtext{Griſette}{\lemma{\textnormal{\emph{Griſette}}}\Cendnote{\textnormal{unverheiratete junge Frau niederen Standes, die etwa
                        als Modistin, Fabrikarbeiterin, Näherin oder Wäscherin ihren Unterhalt
                        selbst finanziert (bekannt aus der fran\oindex{Frankreich@\textbf{Frankreich}, \emph{A.PCLI}|pwkv}zösischen Literatur des 19. Jahrhunderts)}}}\label{K_L02934-10},
                     die ja ſchließlich nicht blos in Wien\oindex{Wien@\textbf{Wien}, \emph{A.ADM2}|pw} zu
                     finden iſt, der aber die Wien\oindex{Wien@\textbf{Wien}, \emph{A.ADM2}|pw}er Art, der Wien\oindex{Wien@\textbf{Wien}, \emph{A.ADM2}|pw}er Humor ſo ganz beſonders gut zu Geſicht
                     ſteht. Ein gründlicher Kenner der Wien\oindex{Wien@\textbf{Wien}, \emph{A.ADM2}|pw}er
                     Verhältniſſe, ein geiſtreicher Spottvogel, Hermann \so{Bahr}\pwindex{Bahr, Hermann 19.07.1863 – 15.01.1934@\textsc{Bahr, Hermann} (19.07.1863 – 15.01.1934), \emph{Schriftsteller/Schriftstellerin, Kritiker/Kritikerin}|pw}, hat nun in ſeinem ſoeben aufgeführten Luſtſpiel »\so{Wienerinnen}\pwindex{Wienerinnen. Lustspiel in drei Akten@\emph{Wienerinnen. Lustspiel in drei Akten}|pw}« einen anderen Typus der mit dem Waſſer der blauen Donau\oindex{Donau@\textbf{Donau}, \emph{Fluss (N.FLS)}|pw} getauften – manchmal auch nicht getauften
                     weiblichen Jugend von heute gezeichnet.}}\pwindex{Vor den Coulissen [Bahrs Wienerinnen]@\emph{Vor den Coulissen [Bahrs Wienerinnen]}|pwv}}{\lemma{\textnormal{\emph{Das … gezeichnet.}}}\Cendnote{\textnormal{S. L. [ = Siegfried Loewy]\pwindex{Loewy, Siegfried 01.11.1857 – 08.05.1931@\textsc{Loewy, Siegfried} (01.11.1857 – 08.05.1931), \emph{Schriftsteller/Schriftstellerin, Journalist/Journalistin}|pwk}: \emph{Vor den Coulissen}\pwindex{Vor den Coulissen [Bahrs Wienerinnen]@\emph{Vor den Coulissen [Bahrs Wienerinnen]}|pwk}. In: \emph{Berliner Börsen-Courier}\pwindex{Berliner Boersen-Courier@\emph{Berliner Börsen-Courier}|pwk}, Jg. 33, Nr. 464, 4. 10. 1900, Morgen-Ausgabe, 1. Beilage,
                     S. [1–2].}}}\label{K_L02934-8}\pend
           {\vspace{1\baselineskip}}
\pstart
           Bitte, liebſter Freund, wenn Du eine Minute Zeit haſt, ſchreib’ mir in drei Worten
               die Wahrheit!\pend
           
\pstart
           Was haſt Du zu den herrlichen \label{K_L02934-11v}\edtext{\textsc{Nietzsche\pwindex{Nietzsche, Friedrich 15.10.1844 – 25.08.1900@\textsc{Nietzsche, Friedrich} (15.10.1844 – 25.08.1900), \emph{Schriftsteller/Schriftstellerin, Philosoph/Philosophin}|pw}}-Briefe\pwindex{erste Nietzsche@\emph{Der erste Nietzsche}|pwv}n in der N. Fr. Pr.\pwindex{Neue Freie Presse@\emph{Neue Freie Presse}|pw}}{\lemma{\textnormal{\emph{Nietzsche-Briefen … Pr.}}}\Cendnote{\textnormal{Bezug auf die Feuilletonreihe \emph{Der erste Nietzsche}\pwindex{erste Nietzsche@\emph{Der erste Nietzsche}|pwk} von Malwida von Meysenbug\pwindex{Meysenbug, Malwida von 1816-10-28 – 1903-04-26@\textsc{Meysenbug, Malwida von} (1816-10-28 – 1903-04-26), \emph{Schriftsteller/Schriftstellerin, Kulturförderer/Kulturfördererin}|pwk}, die zwischen 18. 9. 1900 (Nr. 12956) und 28. 9. 1900 (Nr. 12966) in der \emph{Neuen
                     Freien Presse}\pwindex{Neue Freie Presse@\emph{Neue Freie Presse}|pwk} erschienen war}}}\label{K_L02934-11} geſagt?\pend
           
\pstart
           Viele treue Grüße! {\\[\baselineskip]}Dein {\\[\baselineskip]}\spacefill\mbox{Paul Goldmann}\pend
           \leftskip=0em{}
\pstart
           \noindent{}\textsc{Brandes\pwindex{Brandes, Georg 04.02.1842 – 19.02.1927@\textsc{Brandes, Georg} (04.02.1842 – 19.02.1927)|pw}} war hier\oindex{Berlin@\textbf{Berlin}, \emph{P.PPLC}|pwv} und iſt zu
                  einem \label{K_L02934-12v}\edtext{weiblichen Rendezvous\pwindex{?? [intime Partnerin von Georg Brandes, Dresden, Oktober 1900] @\textsc{?? [intime Partnerin von Georg Brandes, Dresden, Oktober 1900]}|pwv}}{\lemma{\textnormal{\emph{weiblichen Rendezvous}}}\Cendnote{\textnormal{jedenfalls nicht Maria Stona\pwindex{Stona, Maria 01.12.1861 – 30.03.1944@\textsc{Stona, Maria} (01.12.1861 – 30.03.1944), \emph{Schriftsteller/Schriftstellerin}|pwk}, die enttäuscht war, dass Georg Brandes\pwindex{Brandes, Georg 04.02.1842 – 19.02.1927@\textsc{Brandes, Georg} (04.02.1842 – 19.02.1927)|pwkv} nicht auch zu ihr
                     reiste (vgl. Martin Pelc: \emph{Maria Stona und ihr Salon in
                           Strzebowitz. Kultur am Rande der Monarchie, der Republik und des
                           Kanons}. Opava: \emph{Europäischer
                           Strukturfonds}/\emph{Schlesische Universität}{ }2014, S. 126.)}}}\label{K_L02934-12}, wie er ſelbſt
                  mittheilt, nach Dresden\oindex{Dresden@\textbf{Dresden}, \emph{P.PPLA}|pw} gefahren.\pend
           \selectlanguage{ngerman}\endnumbering\briefempfaengerindex{Schnitzler, Arthur@\textsc{Schnitzler, Arthur}!zzzGoldmann, Paul@\emph{von Paul Goldmann}!1900-10-042@{4. 10. {[}1900{]}}|)be}\mylabel{L02934h}  \normalsize

\doendnotes{C}
\bigskip
\vfill

\clearpage

\footnotesize

\lohead{\textsc{register}}

% Definiere theindex-Environment komplett neu ohne reledmac
\makeatletter
\renewenvironment{theindex}{%
  \section*{\indexname}%
  \setlength{\parindent}{0pt}%
  \setlength{\parskip}{0pt plus 0.3pt}%
  \let\item\@idxitem
}{%
  \clearpage
}
\makeatother

\IfFileExists{\jobname-pw.ind}{\input{\jobname-pw.ind}}{}

\end{document}

      