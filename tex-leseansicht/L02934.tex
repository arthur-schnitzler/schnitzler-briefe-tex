%% latex-leseansicht-vorspann.tex
%% Vorspann für die Leseansicht.
%% Lädt die gemeinsame Datei latex-vorspann.tex mit nicht gesetztem Schalter.

\newif\ifkorrekturansicht
\korrekturansichtfalse

\input{../tex-inputs/latex-vorspann}

\begin{center}
            \textcolor{red}{ENTWURF, NICHT FERTIG KORRIGIERT}
                      \end{center}
            
         
         \newcommand{\erwaehntePersonen}{Personen: Hermann Bahr, Oskar Blumenthal, Georg Brandes, Otto Erich Hartleben, Siegfried Loewy, Malwida von Meysenbug, Friedrich Nietzsche, Wilhelm Ernst Oswalt, Maria Stona}
         \newcommand{\erwaehnteInstitutionen}{Institutionen: Rütten & Loening}
         \newcommand{\erwaehnteOrte}{Orte: Berlin, Deutsches Theater Berlin, Donau, Dresden, Frankfurt am Main, Frankreich, Volkstheater, Wien}
         \newcommand{\erwaehnteWerke}{Werke: Berliner Börsen-Courier, Berliner Tageblatt, Der erste Nietzsche, Ein Sommer in China. Reisebilder, Ein Sommer in China. Reisebilder. Zweite, durchgesehene und vermehrte Auflage, Kritische Tagebuchblätter, Liebelei. Schauspiel in drei Akten, Neue Freie Presse, Rosenmontag, Theater und Musik [Wienerinnen], Theaterchronik [Wienerinnen], Vossische Zeitung, Wienerinnen. Lustspiel in drei Akten, [Rezension von Wienerinnen von Siegfried Loewy]}
               \section[ Paul Goldmann an Arthur Schnitzler, 4. 10. {[}1900{]}]{ Paul Goldmann an Arthur Schnitzler, 4. 10. {[}1900{]}}\nopagebreak\mylabel{v}\rehead{ }\begin{ledgroupsized}[t]{13cm}\normalsize\beginnumbering \toendnotes[C]{\smallbreak\pagebreak[2]} \Standort{DLA, A:Schnitzler, HS.NZ85.1.3170.}
\physDesc{Brief, 1 Blatt, 4 Seiten
\newline{}Handschrift: blaue Tinte, deutsche Kurrent\newline{}Beilage: drei aufgeklebte beschnittene Zeitungsausschnitte 
\newline{}Schnitzler: 1) mit Bleistift das Jahr »{[}1{]}900« vermerkt  2) mit rotem Buntstift drei Unterstreichungen}\toendnotes[C]{\smallbreak}\pstart{}{\pb}\textcolor{gray}{\textbf{DESSAUERSTRASSE 19}}\pend{}{\bigskip}\pstart
           Berlin\oindex{Berlin@\textbf{Berlin}|pw}, 4. Oktober.\pend
           \pstart\center{}Mein lieber Freund,\pend\pstart
           Ich danke Dir von Herzen für Deine lieben Briefe, insbeſondere für den wunderſchönen
               von neulich, den \strikeout{d\textcolor{gray}{×}\-\textcolor{gray}{×}} ich ausführlich beantworten werde, ſobald ich Zeit finde.\pend
           \pstart
           Die zweite Auflage\pwindex{Goldmann, Paul 31.01.1865 – 25.09.1935@\textsc{Goldmann, Paul} (31.01.1865 – 25.09.1935), \emph{Schriftsteller, Journalist}!Sommer in China. Reisebilder. Zweite, durchgesehene und vermehrte Auflage1900-10-27@\strich\emph{Ein Sommer in China. Reisebilder. Zweite, durchgesehene und vermehrte Auflage} {[}1900-10-27{]}|pwv} meines Buch\pwindex{Goldmann, Paul 31.01.1865 – 25.09.1935@\textsc{Goldmann, Paul} (31.01.1865 – 25.09.1935), \emph{Schriftsteller, Journalist}!Sommer in China. Reisebilder1899-05-02@\strich\emph{Ein Sommer in China. Reisebilder} {[}1899-05-02{]}|pwv}es erſcheint erſt \label{K_L02934-1v}\edtext{in einigen Wochen}{\lemma{\textnormal{\emph{in einigen Wochen}}}\Cendnote{\textnormal{Die zweite
                     Auflage\pwindex{Goldmann, Paul 31.01.1865 – 25.09.1935@\textsc{Goldmann, Paul} (31.01.1865 – 25.09.1935), \emph{Schriftsteller, Journalist}!Sommer in China. Reisebilder. Zweite, durchgesehene und vermehrte Auflage1900-10-27@\strich\emph{Ein Sommer in China. Reisebilder. Zweite, durchgesehene und vermehrte Auflage} {[}1900-10-27{]}|pwkv} von \emph{Ein Sommer in China}\pwindex{Goldmann, Paul 31.01.1865 – 25.09.1935@\textsc{Goldmann, Paul} (31.01.1865 – 25.09.1935), \emph{Schriftsteller, Journalist}!Sommer in China. Reisebilder1899-05-02@\strich\emph{Ein Sommer in China. Reisebilder} {[}1899-05-02{]}|pwk}
                  erschien am 22. 11. 1900.}}}\label{K_L02934-1h}. Der \label{K_L02934-2v}\edtext{Idiot von
                     Verleger\pwindex{Oswalt, Wilhelm Ernst 1877-03-15 – 1942-06-30@\textsc{Oswalt, Wilhelm Ernst} (1877-03-15 – 1942-06-30), \emph{Verleger}|pwuv}\orgindex{Ruetten und Loening@Rütten {\kaufmannsund}  Loening|pwv}}{\lemma{\textnormal{\emph{Idiot von
                     Verleger}}}\Cendnote{\textnormal{vermutlich Wilhelm Ernst Oswalt\pwindex{Oswalt, Wilhelm Ernst 1877-03-15 – 1942-06-30@\textsc{Oswalt, Wilhelm Ernst} (1877-03-15 – 1942-06-30), \emph{Verleger}|pwk} vom Frankfurt\oindex{Frankfurt am Main@\textbf{Frankfurt am Main}|pwk}er Verlag \emph{Rütten &
                     Loening}\orgindex{Ruetten und Loening@Rütten {\kaufmannsund}  Loening|pwk}}}}\label{K_L02934-2h} kann mit der Drucklegung nicht fertig werden. Selbſtverſtändlich geht ein Exemplar\pwindex{Goldmann, Paul 31.01.1865 – 25.09.1935@\textsc{Goldmann, Paul} (31.01.1865 – 25.09.1935), \emph{Schriftsteller, Journalist}!Sommer in China. Reisebilder. Zweite, durchgesehene und vermehrte Auflage1900-10-27@\strich\emph{Ein Sommer in China. Reisebilder. Zweite, durchgesehene und vermehrte Auflage} {[}1900-10-27{]}|pwv} an die angegebene
               Adreſſe.\pend
           \pstart
           {\pb}Geſtern hatten wir hier \label{K_L02934-3v}\edtext{»Roſenmontag\pwindex{Hartleben, Otto Erich 03.06.1864 – 11.02.1905@\textsc{Hartleben, Otto Erich} (03.06.1864 – 11.02.1905), \emph{Schriftsteller}!Rosenmontag1900@\strich\emph{Rosenmontag} {[}1900{]}|pw}« von \textsc{Hartleben\pwindex{Hartleben, Otto Erich 03.06.1864 – 11.02.1905@\textsc{Hartleben, Otto Erich} (03.06.1864 – 11.02.1905), \emph{Schriftsteller}|pw}}}{\lemma{\textnormal{\emph{»Roſenmontag« von Hartleben}}}\Cendnote{\textnormal{im Deutschen Theater\oindex{Deutsches Theater Berlin@\textbf{Deutsches Theater Berlin}|pwk}}}}\label{K_L02934-3h}. \strikeout{\textcolor{gray}{W}}{ }\label{K_L02934-7v}\edtext{»Unſer \textsc{Otto Erich\pwindex{Hartleben, Otto Erich 03.06.1864 – 11.02.1905@\textsc{Hartleben, Otto Erich} (03.06.1864 – 11.02.1905), \emph{Schriftsteller}|pw}}.\pwindex{Blumenthal, Oskar 13.03.1852 – 24.04.1917@\textsc{Blumenthal, Oskar} (13.03.1852 – 24.04.1917), \emph{Schriftsteller, Journalist, Theaterleiter}!Kritische Tagebuchblaetter1900-01-08@\strich\emph{Kritische Tagebuchblätter} {[}1900-01-08{]}|pwuv}«}{\lemma{\textnormal{\emph{»Unſer Otto Erich.«}}}\Cendnote{\textnormal{zur stehenden Wendung
                  gewordene Phrase, die womöglich auf Oskar
                     Blumenthal\pwindex{Blumenthal, Oskar 13.03.1852 – 24.04.1917@\textsc{Blumenthal, Oskar} (13.03.1852 – 24.04.1917), \emph{Schriftsteller, Journalist, Theaterleiter}|pwk} zurückgeht (vgl. Oskar Blumenthal\pwindex{Blumenthal, Oskar 13.03.1852 – 24.04.1917@\textsc{Blumenthal, Oskar} (13.03.1852 – 24.04.1917), \emph{Schriftsteller, Journalist, Theaterleiter}|pwk}: \emph{Kritische Tagebuchblätter}\pwindex{Blumenthal, Oskar 13.03.1852 – 24.04.1917@\textsc{Blumenthal, Oskar} (13.03.1852 – 24.04.1917), \emph{Schriftsteller, Journalist, Theaterleiter}!Kritische Tagebuchblaetter1900-01-08@\strich\emph{Kritische Tagebuchblätter} {[}1900-01-08{]}|pwk}. In: \emph{Berliner Tageblatt und Handels-Zeitung}\pwindex{?? Werk@Nicht ermittelte Verfasserinnen und Verfasser!Berliner Tageblatt1872 – 1939@\emph{Berliner Tageblatt} {[}1872 – 1939{]}|pwk}, Jg. 29, Nr. 13,
                        8. 1. 1900, Abend-Ausgabe, S. 1–3, hier:
                     S. 1)}}}\label{K_L02934-7h} Guter erſter Akt\pwindex{Hartleben, Otto Erich 03.06.1864 – 11.02.1905@\textsc{Hartleben, Otto Erich} (03.06.1864 – 11.02.1905), \emph{Schriftsteller}!Rosenmontag1900@\strich\emph{Rosenmontag} {[}1900{]}|pwv}. Sobald das \introOben{}eigentliche\introOben{}{ }Drama\pwindex{Hartleben, Otto Erich 03.06.1864 – 11.02.1905@\textsc{Hartleben, Otto Erich} (03.06.1864 – 11.02.1905), \emph{Schriftsteller}!Rosenmontag1900@\strich\emph{Rosenmontag} {[}1900{]}|pwv} anfängt, eine von \strikeout{A\textcolor{gray}{k}} Akt zu Akt troſtloſer werdende Unfähigkeit und Leere. So ein Burſch\pwindex{Hartleben, Otto Erich 03.06.1864 – 11.02.1905@\textsc{Hartleben, Otto Erich} (03.06.1864 – 11.02.1905), \emph{Schriftsteller}|pwv} ohne \strikeout{Wärme} Wärme und Poeſie, der ſich als Dichter aufſpielt, weil es in der
               deutſchen Literatur zufällig an ſolchen mangelte!\pend
           \pstart
           \textsc{Bahr\pwindex{Bahr, Hermann 19.07.1863 – 15.01.1934@\textsc{Bahr, Hermann} (19.07.1863 – 15.01.1934), \emph{Schriftsteller, Kritiker}|pw}} ſcheint auch ein liebes Stück\pwindex{Bahr, Hermann 19.07.1863 – 15.01.1934@\textsc{Bahr, Hermann} (19.07.1863 – 15.01.1934), \emph{Schriftsteller, Kritiker}!Wienerinnen. Lustspiel in drei Akten1900@\strich\emph{Wienerinnen. Lustspiel in drei Akten} {[}1900{]}|pwv} geſchrieben zu haben. Wir haben hier folgende Berichte {\pb}erhalten:\pend
           \pstart
           \substVorne{}\textsuperscript{B\textcolor{gray}{er}}\substDazwischen{}Vo\pwindex{?? Werk@Nicht ermittelte Verfasserinnen und Verfasser!Vossische Zeitung1617 – 1934@\emph{Vossische Zeitung} {[}1617 – 1934{]}|pwv}\substHinten{}ſſiſche Zeitung\pwindex{?? Werk@Nicht ermittelte Verfasserinnen und Verfasser!Vossische Zeitung1617 – 1934@\emph{Vossische Zeitung} {[}1617 – 1934{]}|pwv}:\pend
           \pstart
           \label{K_L02934-5v}\edtext{\textcolor{gray}{\textbf{Im \so{Deutſchen Volkstheater\oindex{Volkstheater@\textbf{Volkstheater}|pw}} hatte heute ein neues Stück »\textbf{Die Wienerinnen\pwindex{Bahr, Hermann 19.07.1863 – 15.01.1934@\textsc{Bahr, Hermann} (19.07.1863 – 15.01.1934), \emph{Schriftsteller, Kritiker}!Wienerinnen. Lustspiel in drei Akten1900@\strich\emph{Wienerinnen. Lustspiel in drei Akten} {[}1900{]}|pw}}« von \so{Hermann Bahr\pwindex{Bahr, Hermann 19.07.1863 – 15.01.1934@\textsc{Bahr, Hermann} (19.07.1863 – 15.01.1934), \emph{Schriftsteller, Kritiker}|pw}} einen durchſchlagenden Erfolg.}}}{\lemma{\textnormal{\emph{Im … Erfolg.}}}\Cendnote{\textnormal{Auszug aus [O. V.]: \emph{Theater und Musik}\pwindex{?? Werk@Nicht ermittelte Verfasserinnen und Verfasser!Theater und Musik [Wienerinnen]1900-10-04@\emph{Theater und Musik [Wienerinnen]} {[}1900-10-04{]}|pwk}. In: \emph{Vossische Zeitung}\pwindex{?? Werk@Nicht ermittelte Verfasserinnen und Verfasser!Vossische Zeitung1617 – 1934@\emph{Vossische Zeitung} {[}1617 – 1934{]}|pwk}, Nr. 464, 4. 10. 1900, Morgen-Ausgabe, S. [16]}}}\label{K_L02934-5h}\pend
           \pstart
           Berliner Tageblatt\pwindex{?? Werk@Nicht ermittelte Verfasserinnen und Verfasser!Berliner Tageblatt1872 – 1939@\emph{Berliner Tageblatt} {[}1872 – 1939{]}|pw}:\pend
           \pstart
           \label{K_L02934-6v}\edtext{\textcolor{gray}{\textbf{Aus \textbf{Wien}\oindex{Wien@\textbf{Wien}|pw} meldet uns ein Privat-Telegramm: Hermann
                     Bahr\pwindex{Bahr, Hermann 19.07.1863 – 15.01.1934@\textsc{Bahr, Hermann} (19.07.1863 – 15.01.1934), \emph{Schriftsteller, Kritiker}|pw}s Luſtſpiel »\so{Wienerinnen\pwindex{Bahr, Hermann 19.07.1863 – 15.01.1934@\textsc{Bahr, Hermann} (19.07.1863 – 15.01.1934), \emph{Schriftsteller, Kritiker}!Wienerinnen. Lustspiel in drei Akten1900@\strich\emph{Wienerinnen. Lustspiel in drei Akten} {[}1900{]}|pw}}« hatte einen kompleten Mißerfolg.}}}{\lemma{\textnormal{\emph{Aus … Mißerfolg.}}}\Cendnote{\textnormal{Auszug aus [O. V.]: \emph{Theaterchronik}\pwindex{?? Werk@Nicht ermittelte Verfasserinnen und Verfasser!Theaterchronik [Wienerinnen]1900-10-04@\emph{Theaterchronik [Wienerinnen]} {[}1900-10-04{]}|pwk}. In: \emph{Berliner Tageblatt}\pwindex{?? Werk@Nicht ermittelte Verfasserinnen und Verfasser!Berliner Tageblatt1872 – 1939@\emph{Berliner Tageblatt} {[}1872 – 1939{]}|pwk}, Jg. 29, Nr. 504, 4. 10. 1900, Morgen-Ausgabe, S. [3]}}}\label{K_L02934-6h}\pend
           \pstart
           Dieſe zwei \label{K_L02934-21v}\edtext{Kritiker}{\lemma{\textnormal{\emph{Kritiker}}}\Cendnote{\textnormal{nicht ermittelt}}}\label{K_L02934-21h} ſcheinen das neue
                  Werk\pwindex{Bahr, Hermann 19.07.1863 – 15.01.1934@\textsc{Bahr, Hermann} (19.07.1863 – 15.01.1934), \emph{Schriftsteller, Kritiker}!Wienerinnen. Lustspiel in drei Akten1900@\strich\emph{Wienerinnen. Lustspiel in drei Akten} {[}1900{]}|pwv} von verſchiedenen
               Geſichtspunkten aus zu betrachten. Im »Börſencourier\pwindex{?? Werk@Nicht ermittelte Verfasserinnen und Verfasser!Berliner Boersen-Courier1868 – 1933@\emph{Berliner Börsen-Courier} {[}1868 – 1933{]}|pw}« aber ſchmückt \textsc{Siegfried Löwy\pwindex{Loewy, Siegfried 01.11.1857 – 08.05.1931@\textsc{Loewy, Siegfried} (01.11.1857 – 08.05.1931), \emph{Schriftsteller, Journalist}|pw}} ſich folgendermaßen aus:\pend
           \pstart
           \label{K_L02934-9v}\edtext{{\pb}\textcolor{gray}{\textbf{Das »süße Wien\oindex{Wien@\textbf{Wien}|pw}er Mädel« iſt
                  durch Arthur Schnitzler’s farbenſatte Schilderung mit ihrer ergreifenden Wendung
                  in’s Tragiſche in ſeiner ganzen Echtheit \label{K_L02934-17v}\edtext{in »Liebelei\pwindex{Schnitzler, Arthur 15.05.1862 – 21.10.1931@\textsc{Schnitzler, Arthur} (15.05.1862 – 21.10.1931), \emph{Schriftsteller, Mediziner}!Liebelei. Schauspiel in drei Akten1895-10-09@\strich\emph{Liebelei. Schauspiel in drei Akten} {[}1895-10-09{]}|pw}« zum
                  erſten Male auf die Bühne gebracht}{\lemma{\textnormal{\emph{in … gebracht}}}\Cendnote{\textnormal{siehe zum Begriff »süßel Mädel« auch Paul Goldmann an Arthur Schnitzler, 4. 10. [1900]}}}\label{K_L02934-17h} worden, das Mädel aus dem Volke, die kleine, liebe \label{K_L02934-12v}\edtext{Griſette}{\lemma{\textnormal{\emph{Griſette}}}\Cendnote{\textnormal{unverheiratete junge Frau niederen Standes, die etwa als
                     Modistin, Fabrikarbeiterin, Näherin oder Wäscherin ihren Unterhalt selbst
                     finanziert (bekannt aus der fran\oindex{Frankreich@\textbf{Frankreich}|pwkv}zösischen Literatur des 19. Jahrhunderts)}}}\label{K_L02934-12h}, die ja
                  ſchließlich nicht blos in Wien\oindex{Wien@\textbf{Wien}|pw} zu finden iſt,
                  der aber die Wien\oindex{Wien@\textbf{Wien}|pw}er Art, der Wien\oindex{Wien@\textbf{Wien}|pw}er Humor ſo ganz beſonders gut zu Geſicht ſteht. Ein
                  gründlicher Kenner der Wien\oindex{Wien@\textbf{Wien}|pw}er Verhältniſſe, ein
                  geiſtreicher Spottvogel, Hermann \so{Bahr}\pwindex{Bahr, Hermann 19.07.1863 – 15.01.1934@\textsc{Bahr, Hermann} (19.07.1863 – 15.01.1934), \emph{Schriftsteller, Kritiker}|pw}, hat nun in ſeinem ſoeben aufgeführten Luſtſpiel »\so{Wienerinnen\pwindex{Bahr, Hermann 19.07.1863 – 15.01.1934@\textsc{Bahr, Hermann} (19.07.1863 – 15.01.1934), \emph{Schriftsteller, Kritiker}!Wienerinnen. Lustspiel in drei Akten1900@\strich\emph{Wienerinnen. Lustspiel in drei Akten} {[}1900{]}|pw}}« einen anderen Typus der mit dem Waſſer der blauen Donau\oindex{Donau@\textbf{Donau}|pw} getauften – manchmal auch nicht getauften weiblichen
                  Jugend von heute gezeichnet.}}}{\lemma{\textnormal{\emph{Das … gezeichnet.}}}\Cendnote{\textnormal{Auszug aus \emph{XXXX}\pwindex{Loewy, Siegfried 01.11.1857 – 08.05.1931@\textsc{Loewy, Siegfried} (01.11.1857 – 08.05.1931), \emph{Schriftsteller, Journalist}!Rezension von Wienerinnen von Siegfried Loewy]1900-10@\strich\emph{[Rezension von Wienerinnen von Siegfried Loewy]} {[}1900-10{]}|pwk}}}}\label{K_L02934-9h}\pend
           \pstart
           Bitte, liebſter Freund, wenn Du eine Minute Zeit haſt, ſchreib’ mir in drei Worten
               die Wahrheit!\pend
           \pstart
           Was haſt Du zu den herrlichen \label{K_L02934-18v}\edtext{\textsc{Nietzsche\pwindex{Nietzsche, Friedrich 15.10.1844 – 25.08.1900@\textsc{Nietzsche, Friedrich} (15.10.1844 – 25.08.1900), \emph{Schriftsteller, Philosoph}|pw}}-Briefe\pwindex{Meysenbug, Malwida von 1816-10-28 – 1903-04-26@\textsc{Meysenbug, Malwida von} (1816-10-28 – 1903-04-26), \emph{Kulturfördererin, Schriftstellerin}!erste Nietzsche1900-09-18 – 1900-09-28@\strich\emph{Der erste Nietzsche} {[}1900-09-18 – 1900-09-28{]}|pwv}n in der N. Fr. Pr.\pwindex{Neue Freie Presse1864 – 1939@\emph{Neue Freie Presse} {[}1864 – 1939{]}|pw}}{\lemma{\textnormal{\emph{Nietzsche-Briefen … Pr.}}}\Cendnote{\textnormal{Bezug auf die Feuilletonreihe \emph{Der erste Nietzsche}\pwindex{Meysenbug, Malwida von 1816-10-28 – 1903-04-26@\textsc{Meysenbug, Malwida von} (1816-10-28 – 1903-04-26), \emph{Kulturfördererin, Schriftstellerin}!erste Nietzsche1900-09-18 – 1900-09-28@\strich\emph{Der erste Nietzsche} {[}1900-09-18 – 1900-09-28{]}|pwk} von Malwida von Meysenbug\pwindex{Meysenbug, Malwida von 1816-10-28 – 1903-04-26@\textsc{Meysenbug, Malwida von} (1816-10-28 – 1903-04-26), \emph{Kulturfördererin, Schriftstellerin}|pwk}, die zwischen 18. 9. 1900 (Nr. 12956) und 28. 9. 1900 (Nr. 12966) in der \emph{Neuen
                     Freien Presse}\pwindex{Neue Freie Presse1864 – 1939@\emph{Neue Freie Presse} {[}1864 – 1939{]}|pwk} erschienen war}}}\label{K_L02934-18h} geſagt?\pend
           \pstart
           Viele treue Grüße! {\\[\baselineskip]}Dein {\\[\baselineskip]}\spacefill\mbox{Paul Goldmann}\pend
           \leftskip=0em{}\pstart
           \noindent{}\textsc{Brandes\pwindex{Brandes, Georg 04.02.1842 – 19.02.1927@\textsc{Brandes, Georg} (04.02.1842 – 19.02.1927)|pw}} war hie\substVorne{}\textsuperscript{\textcolor{gray}{u}}\substDazwischen{}r\substHinten{}\oindex{Berlin@\textbf{Berlin}|pwv} und iſt zu einem \label{K_L02934-19v}\edtext{weiblichen
                     Rendezvous\pwindex{Stona, Maria 01.12.1861 – 30.03.1944@\textsc{Stona, Maria} (01.12.1861 – 30.03.1944), \emph{Schriftstellerin}|pwuv}}{\lemma{\textnormal{\emph{weiblichen
                     Rendezvous}}}\Cendnote{\textnormal{möglicherweise Maria Stona\pwindex{Stona, Maria 01.12.1861 – 30.03.1944@\textsc{Stona, Maria} (01.12.1861 – 30.03.1944), \emph{Schriftstellerin}|pwk}, vgl. Martin Pelc: \emph{Maria Stona und ihr Salon in Strzebowitz. Kultur am Rande der Monarchie,
                           der Republik und des Kanons}. Opava:
                           \emph{Europäischer Strukturfonds}/\emph{Schlesische
                           Universität}{ }2014, S. 126}}}\label{K_L02934-19h}, wie er ſelbſt mittheilt, nach Dresden\oindex{Dresden@\textbf{Dresden}|pw}
                  gefahren.\pend
           
         
         \endnumbering\mylabel{h}\end{ledgroupsized}\begin{anhang}\end{anhang}\newcommand{\dateiname}{L02934}\newcommand{\titel}{Paul Goldmann an Arthur Schnitzler, 4. 10. [1900]}\newcommand{\editorInnen}{Martin Anton Müller und Laura Untner}%% latex-leseansicht-abspann.tex
%% Abspann für die Leseansicht.
%% Der Schalter \ifkorrekturansicht ist bereits durch den Vorspann gesetzt.

%% latex-abspann.tex
%% Gemeinsamer Abspann für Korrekturansicht und Leseansicht.
%% Setzt den Schalter \ifkorrekturansicht voraus (gesetzt in den
%% einbindenden Dateien latex-korrekturansicht-abspann.tex bzw.
%% latex-leseansicht-abspann.tex).
%% ---------------------------------------------------------------

\normalsize

% Das esempio-Environment wird nur in der Leseansicht benötigt
\ifkorrekturansicht\else
\newenvironment{esempio}[3]%
{
    \vspace{1.5ex}
    \rlap{\underline{#1}}
    \par
    \setlength{\parindent}{0cm}
    \nopagebreak
    \leftskip=#2cm
    \rightskip=#3cm
}
{
    \par
}
\fi

\doendnotes{C}
\bigskip
\vfill

\clearpage

\footnotesize

\ifkorrekturansicht
  \lohead{\textsc{register}}
\fi

% theindex-Environment neu definieren ohne reledmac
\makeatletter
\renewenvironment{theindex}{%
  \ifkorrekturansicht
    \section*{\indexname}%
  \else
    \subsubsection*{Index der erwähnten Entitäten}%
  \fi
  \setlength{\parindent}{0pt}%
  \setlength{\parskip}{0pt plus 0.3pt}%
  \let\item\@idxitem
}{%
  \ifkorrekturansicht\clearpage\fi
}
\makeatother

\IfFileExists{\jobname-pw.ind}{\input{\jobname-pw.ind}}{}

% Quellenangabe nur in der Leseansicht
\ifkorrekturansicht\else
% Fallback-Definitionen, falls die .tex-Datei \titel etc. nicht gesetzt hat
\providecommand{\titel}{}
\providecommand{\editorInnen}{}
\providecommand{\dateiname}{\jobname}

\vspace{3cm}

\vfill

\footnotesize
\textsc{Quelle}: \titel. Herausgegeben von {\editorInnen}. In: \emph{Arthur Schnitzler: Briefwechsel mit Autorinnen und Autoren}.
 Digitale Edition, https://schnitzler-briefe.acdh.oeaw.ac.at/{\dateiname}.html (Stand \today)
\fi

\end{document}


      