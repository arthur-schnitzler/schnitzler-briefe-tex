%% latex-leseansicht-vorspann.tex
%% Vorspann für die Leseansicht.
%% Lädt die gemeinsame Datei latex-vorspann.tex mit nicht gesetztem Schalter.

\newif\ifkorrekturansicht
\korrekturansichtfalse

\input{../tex-inputs/latex-vorspann}


\section[Jakob Julius David an Arthur Schnitzler, {[}27. 3. 1895{]}]{L00427 Jakob Julius David an Arthur Schnitzler, {[}27. 3. 1895{]}}
\nopagebreak\mylabel{L00427v}
\rehead{ }\normalsize\beginnumbering\briefempfaengerindex{Schnitzler, Arthur@\textsc{Schnitzler, Arthur}!zzzDavid, Jakob Julius@\emph{von Jakob Julius David}!1895-03-272@{{[}27. 3. 1895{]}}|(be}
\toendnotes[C]{\smallbreak\pagebreak[2]}
\correspDesc{Versand  durch Jakob Julius David am [27. 3. 1895] in Wien
\newline{}Erhalt  durch Arthur Schnitzler im Zeitraum [27. 3. 1895
                  – 31. 3. 1895?] in Wien}\toendnotes[C]{\smallbreak}
\Standort{CUL, Schnitzler, B 25.}
\physDesc{Brief, 1 Blatt, 1 Seite, 195 Zeichen
\newline{}Handschrift: schwarze Tinte, lateinische Kurrent
\newline{}Schnitzler: mit Bleistift zuerst unleserlich datiert, das dann gestrichen
                                 und neuerlich: »27/3 95.« 
\newline{}Ordnung: mit Bleistift von unbekannter Hand nummeriert:
                                 »2« }\toendnotes[C]{\smallbreak}
\pstart\center{}{\pb}Werther Herr Doctor!\pend\vspace{0.5em}
\pstart
           \label{K_L00427-1v}\edtext{Es\pwindex{Poppenberg, Felix 13.\,10.\,1869 Charlottenburg – 18.\,10.\,1915 ebd.@\textsc{Poppenberg, Felix} (13.\,10.\,1869 Charlottenburg – 18.\,10.\,1915 ebd.), \emph{Schriftsteller, Kritiker}!Buchmacher und Künstler@\strich\emph{Buchmacher und Künstler}|pwv}}{\lemma{\textnormal{\emph{Es}}}\Cendnote{\textnormal{Wohl diese Rezension von \emph{Sterben}\pwindex{Schnitzler, Arthur 15.\,5.\,1862 Wien – 21.\,10.\,1931 ebd.@\textsc{Schnitzler, Arthur} (15.\,5.\,1862 Wien – 21.\,10.\,1931 ebd.), \emph{Schriftsteller, Mediziner}!Sterben. Novelle@\strich\emph{Sterben. Novelle}|pwk}: Felix Poppenberg\pwindex{Poppenberg, Felix 13.\,10.\,1869 Charlottenburg – 18.\,10.\,1915 ebd.@\textsc{Poppenberg, Felix} (13.\,10.\,1869 Charlottenburg – 18.\,10.\,1915 ebd.), \emph{Schriftsteller, Kritiker}|pwk}: \emph{Buchmacher und Künstler}\pwindex{Poppenberg, Felix 13.\,10.\,1869 Charlottenburg – 18.\,10.\,1915 ebd.@\textsc{Poppenberg, Felix} (13.\,10.\,1869 Charlottenburg – 18.\,10.\,1915 ebd.), \emph{Schriftsteller, Kritiker}!Buchmacher und Künstler@\strich\emph{Buchmacher und Künstler}|pwk}. In: \emph{Das Magazin für Litteratur}\pwindex{Magazin für die Literatur des Auslandes@\emph{Magazin für die Literatur des Auslandes}|pwk}, Jg. 64, Nr. 9,
                        2. 3. 1895, Sp. 265–270, hier Sp. 269–270.}}}\label{K_L00427-1} war im
                  Februar oder März. Neuma{\geminationn}-Hofer\pwindex{Neumann-Hofer, Gilbert Otto 4.\,2.\,1857 Bol’shiye Berezhki – 14.\,4.\,1941 Detmold@\textsc{Neumann-Hofer, Gilbert Otto} (4.\,2.\,1857 Bol’shiye Berezhki – 14.\,4.\,1941 Detmold), \emph{Kritiker, Theaterleiter}|pw} stellt Ihnen die Nu{\geminationm}er\pwindex{Magazin für die Literatur des Auslandes@\emph{Magazin für die Literatur des Auslandes}|pwv} sicher zur Verfügung. Ich weiß nicht, wohin ich das Ding kramte. Mir
               liegt an den Sachen so gar nichts.\pend
           
\pstart
           Herzlichst{\\[\baselineskip]}\spacefill\mbox{David}\pend
           \leftskip=0em{}\selectlanguage{ngerman}\endnumbering\briefempfaengerindex{Schnitzler, Arthur@\textsc{Schnitzler, Arthur}!zzzDavid, Jakob Julius@\emph{von Jakob Julius David}!1895-03-272@{{[}27. 3. 1895{]}}|)be}\mylabel{L00427h}  \newcommand{\dateiname}{L00427}\newcommand{\titel}{Jakob Julius David an Arthur Schnitzler, [27. 3. 1895]}\newcommand{\editorInnen}{Martin Anton Müller und Gerd-Hermann Susen}%% latex-leseansicht-abspann.tex
%% Abspann für die Leseansicht.
%% Der Schalter \ifkorrekturansicht ist bereits durch den Vorspann gesetzt.

%% latex-abspann.tex
%% Gemeinsamer Abspann für Korrekturansicht und Leseansicht.
%% Setzt den Schalter \ifkorrekturansicht voraus (gesetzt in den
%% einbindenden Dateien latex-korrekturansicht-abspann.tex bzw.
%% latex-leseansicht-abspann.tex).
%% ---------------------------------------------------------------

\normalsize

% Das esempio-Environment wird nur in der Leseansicht benötigt
\ifkorrekturansicht\else
\newenvironment{esempio}[3]%
{
    \vspace{1.5ex}
    \rlap{\underline{#1}}
    \par
    \setlength{\parindent}{0cm}
    \nopagebreak
    \leftskip=#2cm
    \rightskip=#3cm
}
{
    \par
}
\fi

\doendnotes{C}
\bigskip
\vfill

\clearpage

\footnotesize

\ifkorrekturansicht
  \lohead{\textsc{register}}
\fi

% theindex-Environment neu definieren ohne reledmac
\makeatletter
\renewenvironment{theindex}{%
  \ifkorrekturansicht
    \section*{\indexname}%
  \else
    \subsubsection*{Index der erwähnten Entitäten}%
  \fi
  \setlength{\parindent}{0pt}%
  \setlength{\parskip}{0pt plus 0.3pt}%
  \let\item\@idxitem
}{%
  \ifkorrekturansicht\clearpage\fi
}
\makeatother

\IfFileExists{\jobname-pw.ind}{\input{\jobname-pw.ind}}{}

% Quellenangabe nur in der Leseansicht
\ifkorrekturansicht\else
% Fallback-Definitionen, falls die .tex-Datei \titel etc. nicht gesetzt hat
\providecommand{\titel}{}
\providecommand{\editorInnen}{}
\providecommand{\dateiname}{\jobname}

\vspace{3cm}

\vfill

\footnotesize
\textsc{Quelle}: \titel. Herausgegeben von {\editorInnen}. In: \emph{Arthur Schnitzler: Briefwechsel mit Autorinnen und Autoren}.
 Digitale Edition, https://schnitzler-briefe.acdh.oeaw.ac.at/{\dateiname}.html (Stand \today)
\fi

\end{document}


