%% latex-leseansicht-vorspann.tex
%% Vorspann für die Leseansicht.
%% Lädt die gemeinsame Datei latex-vorspann.tex mit nicht gesetztem Schalter.

\newif\ifkorrekturansicht
\korrekturansichtfalse

\input{../tex-inputs/latex-vorspann}


         
         \renewcommand{\erwaehntePersonen}{Personen: Jakob Julius David, Gilbert Otto Neumann-Hofer, Felix Poppenberg}
         \renewcommand{\erwaehnteOrte}{Orte: Wien}
         \renewcommand{\erwaehnteWerke}{Werke: Buchmacher und Künstler, Magazin für die Literatur des Auslandes, Sterben. Novelle}
               \section[Jakob Julius David an Arthur Schnitzler, {[}27. 3. 1895{]}]{ Jakob Julius David an Arthur Schnitzler, {[}27. 3. 1895{]}}\nopagebreak\mylabel{v}\rehead{ }\begin{ledgroupsized}[t]{13cm}\normalsize\beginnumbering \toendnotes[C]{\smallbreak\pagebreak[2]} \Standort{CUL, Schnitzler, B 25.}
\physDesc{Brief, 1 Blatt, 1 Seite, 195 Zeichen
\newline{}Handschrift: schwarze Tinte, lateinische Kurrent
\newline{}Schnitzler: mit Bleistift zuerst unleserlich datiert, das dann gestrichen
                                 und neuerlich: »27/3 95.« 
\newline{}Ordnung: mit Bleistift von unbekannter Hand nummeriert:
                                 »2« }\toendnotes[C]{\smallbreak}\pstart\center{}{\pb}Werther Herr Doctor!\pend\pstart
           \label{K_L00427-1v}\edtext{Es\pwindex{Poppenberg, Felix 13.10.1869 – 18.10.1915@\textsc{Poppenberg, Felix} (13.10.1869 – 18.10.1915), \emph{Schriftsteller, Kritiker}!Buchmacher und Kuenstler2. 3. 1895@\strich\emph{Buchmacher und Künstler} {[}2. 3. 1895{]}|pwv}}{\lemma{\textnormal{\emph{Es}}}\Cendnote{\textnormal{Wohl diese Rezension von \emph{Sterben}\pwindex{Schnitzler, Arthur 15.05.1862 – 21.10.1931@\textsc{Schnitzler, Arthur} (15.05.1862 – 21.10.1931), \emph{Schriftsteller, Mediziner}!Sterben. Novelle1894-10-01 – 1894-12-01@\strich\emph{Sterben. Novelle} {[}1894-10-01 – 1894-12-01{]}|pwk}: Felix Poppenberg\pwindex{Poppenberg, Felix 13.10.1869 – 18.10.1915@\textsc{Poppenberg, Felix} (13.10.1869 – 18.10.1915), \emph{Schriftsteller, Kritiker}|pwk}: \emph{Buchmacher und Künstler}\pwindex{Poppenberg, Felix 13.10.1869 – 18.10.1915@\textsc{Poppenberg, Felix} (13.10.1869 – 18.10.1915), \emph{Schriftsteller, Kritiker}!Buchmacher und Kuenstler2. 3. 1895@\strich\emph{Buchmacher und Künstler} {[}2. 3. 1895{]}|pwk}. In: \emph{Das Magazin für Litteratur}\pwindex{?? Werk@Nicht ermittelte Verfasserinnen und Verfasser!Magazin fuer die Literatur des Auslandes1832 – 1915@\emph{Magazin für die Literatur des Auslandes} {[}1832 – 1915{]}|pwk}, Jg. 64, Nr. 9,
                        2. 3. 1895, Sp. 265–270, hier Sp. 269–270.}}}\label{K_L00427-1h} war im
                  Februar oder März. Neuma{\geminationn}-Hofer\pwindex{Neumann-Hofer, Gilbert Otto 04.02.1857 – 14.04.1941@\textsc{Neumann-Hofer, Gilbert Otto} (04.02.1857 – 14.04.1941), \emph{Kritiker, Theaterleiter}|pw} stellt Ihnen die Nu{\geminationm}er\pwindex{?? Werk@Nicht ermittelte Verfasserinnen und Verfasser!Magazin fuer die Literatur des Auslandes1832 – 1915@\emph{Magazin für die Literatur des Auslandes} {[}1832 – 1915{]}|pwv} sicher zur Verfügung. Ich weiß nicht, wohin ich das Ding kramte. Mir
               liegt an den Sachen so gar nichts.\pend
           \pstart
           Herzlichst{\\[\baselineskip]}\spacefill\mbox{David}\pend
           \leftskip=0em{}
         
         \endnumbering\mylabel{h}\end{ledgroupsized}  \newcommand{\dateiname}{L00427}\newcommand{\titel}{Jakob Julius David an Arthur Schnitzler, [27. 3. 1895]}\newcommand{\editorInnen}{Martin Anton Müller und Gerd-Hermann Susen}%% latex-leseansicht-abspann.tex
%% Abspann für die Leseansicht.
%% Der Schalter \ifkorrekturansicht ist bereits durch den Vorspann gesetzt.

%% latex-abspann.tex
%% Gemeinsamer Abspann für Korrekturansicht und Leseansicht.
%% Setzt den Schalter \ifkorrekturansicht voraus (gesetzt in den
%% einbindenden Dateien latex-korrekturansicht-abspann.tex bzw.
%% latex-leseansicht-abspann.tex).
%% ---------------------------------------------------------------

\normalsize

% Das esempio-Environment wird nur in der Leseansicht benötigt
\ifkorrekturansicht\else
\newenvironment{esempio}[3]%
{
    \vspace{1.5ex}
    \rlap{\underline{#1}}
    \par
    \setlength{\parindent}{0cm}
    \nopagebreak
    \leftskip=#2cm
    \rightskip=#3cm
}
{
    \par
}
\fi

\doendnotes{C}
\bigskip
\vfill

\clearpage

\footnotesize

\ifkorrekturansicht
  \lohead{\textsc{register}}
\fi

% theindex-Environment neu definieren ohne reledmac
\makeatletter
\renewenvironment{theindex}{%
  \ifkorrekturansicht
    \section*{\indexname}%
  \else
    \subsubsection*{Index der erwähnten Entitäten}%
  \fi
  \setlength{\parindent}{0pt}%
  \setlength{\parskip}{0pt plus 0.3pt}%
  \let\item\@idxitem
}{%
  \ifkorrekturansicht\clearpage\fi
}
\makeatother

\IfFileExists{\jobname-pw.ind}{\input{\jobname-pw.ind}}{}

% Quellenangabe nur in der Leseansicht
\ifkorrekturansicht\else
% Fallback-Definitionen, falls die .tex-Datei \titel etc. nicht gesetzt hat
\providecommand{\titel}{}
\providecommand{\editorInnen}{}
\providecommand{\dateiname}{\jobname}

\vspace{3cm}

\vfill

\footnotesize
\textsc{Quelle}: \titel. Herausgegeben von {\editorInnen}. In: \emph{Arthur Schnitzler: Briefwechsel mit Autorinnen und Autoren}.
 Digitale Edition, https://schnitzler-briefe.acdh.oeaw.ac.at/{\dateiname}.html (Stand \today)
\fi

\end{document}


      