%% latex-leseansicht-vorspann.tex
%% Vorspann für die Leseansicht.
%% Lädt die gemeinsame Datei latex-vorspann.tex mit nicht gesetztem Schalter.

\newif\ifkorrekturansicht
\korrekturansichtfalse

\input{../tex-inputs/latex-vorspann}

\begin{center}
            \textcolor{red}{ENTWURF, NICHT FERTIG KORRIGIERT}
                      \end{center}
            
         
         \renewcommand{\erwaehntePersonen}{Personen: Felix Salten}
         \renewcommand{\erwaehnteInstitutionen}{Institutionen: Concordia}
         \renewcommand{\erwaehnteOrte}{Orte: Kaufmännischer Verein, Wien}
         \renewcommand{\erwaehnteWerke}{Werke: Liebelei. Schauspiel in drei Akten}
               \section[Arthur Schnitzler an Felix Salten, {[}9. 6. 1896?{]}]{ Arthur Schnitzler an Felix Salten, {[}9. 6. 1896?{]}}\nopagebreak\mylabel{v}\rehead{ }\begin{ledgroupsized}[t]{13cm}\normalsize\beginnumbering \toendnotes[C]{\smallbreak\pagebreak[2]} \Standort{Wienbibliothek im Rathaus, ZPH 1681, 2.1.516.}
\physDesc{Brief, 1 Blatt, 3 Seiten, 301 Zeichen
\newline{}Handschrift: Bleistift, deutsche Kurrent
\newline{}Ordnung: mit Bleistift von unbekannter Hand Nummerierung der Blätter des
                                 Konvoluts: »22«–»23« }\toendnotes[C]{\smallbreak}\pstart
           \raggedleft{}{\pb}Dinſtag\pend
           \pstart
           lieber, wollen Sie heut Abend mit mir in eine verborgne Loge jener
                  \label{K_L03039-111v}\edtext{Liebelei\pwindex{Schnitzler, Arthur 15.05.1862 – 21.10.1931@\textsc{Schnitzler, Arthur} (15.05.1862 – 21.10.1931), \emph{Schriftsteller, Mediziner}!Liebelei. Schauspiel in drei Akten1895-10-09@\strich\emph{Liebelei. Schauspiel in drei Akten} {[}1895-10-09{]}|pw}-Aufführg}{\lemma{\textnormal{\emph{Liebelei-Aufführg}}}\Cendnote{\textnormal{Zwei Dienstage, an denen Schnitzler\pwindex{Schnitzler, Arthur 15.05.1862 – 21.10.1931@\textsc{Schnitzler, Arthur} (15.05.1862 – 21.10.1931), \emph{Schriftsteller, Mediziner}|pwk} in \emph{Liebelei}\pwindex{Schnitzler, Arthur 15.05.1862 – 21.10.1931@\textsc{Schnitzler, Arthur} (15.05.1862 – 21.10.1931), \emph{Schriftsteller, Mediziner}!Liebelei. Schauspiel in drei Akten1895-10-09@\strich\emph{Liebelei. Schauspiel in drei Akten} {[}1895-10-09{]}|pwk}-Aufführungen
                  war, bieten sich zur Datierung dieses Korrespondenzstücks an. Bei der am 15. 1. 1901 handelte
                  es sich um eine Inszenierung von Schauspielschülerinnen im Kaufmännischen Verein\oindex{Kaufmaennischer Verein@\textbf{Kaufmännischer Verein}|pwk}, die Existenz einer
                     »geheimen Loge« scheint eher abwegig. In einem Brief, den Salten\pwindex{Salten, Felix 06.09.1869 – 08.10.1945@\textsc{Salten, Felix} (06.09.1869 – 08.10.1945), \emph{Schriftsteller, Journalist}|pwk} mutmaßlich am selben Tag Schnitzler\pwindex{Schnitzler, Arthur 15.05.1862 – 21.10.1931@\textsc{Schnitzler, Arthur} (15.05.1862 – 21.10.1931), \emph{Schriftsteller, Mediziner}|pwk} sendet, deutet er an, am Abend
                  möglicherweise verhindert zu sein, womit sein Fernbleiben erklärt ist (Felix Salten an Arthur Schnitzler, [9. 6. 1896?]).}}}\label{K_L03039-111h} gehen \introOben{}(½ 8)\introOben{}, ſo laſſen Sie michs gütigſt am frühen Nachmittg
               wiſſen. Ich hole Sie da{\geminationn}, we{\geminationn}s Ihnen {\pb}recht iſt, um \label{K_L03039-1v}\edtext{¼{ }8}{\lemma{\textnormal{\emph{¼ 8}}}\Cendnote{\textnormal{19 Uhr 15}}}\label{K_L03039-1h} oder ½ in Ihrer Wohnung
               ab? \pend
           \pstart
           Herzlichſt {\\[\baselineskip]}Ihr {\\[\baselineskip]}\spacefill\mbox{Arth}\pend
           \leftskip=0em{}\pstart
           \noindent{}{\pb}Und noch eins: ich habe geſtern mit Ihnen
                  im \label{K_L03039-1112v}\edtext{Club\orgindex{Concordia@Concordia|pwuv}}{\lemma{\textnormal{\emph{Club}}}\Cendnote{\textnormal{Welcher Klub gemeint ist, lässt
                     sich derzeit nicht bestimmen. Da Schnitzler\pwindex{Schnitzler, Arthur 15.05.1862 – 21.10.1931@\textsc{Schnitzler, Arthur} (15.05.1862 – 21.10.1931), \emph{Schriftsteller, Mediziner}|pwk} seit zumindest 13. 10. 1889 Veranstaltungen und den Club der
                        \emph{Concordia}\orgindex{Concordia@Concordia|pwk} besuchte, ist das vermutlich
                     der gemeinte.}}}\label{K_L03039-1112h} soupirt. \pend
           
         
         \endnumbering\mylabel{h}\end{ledgroupsized}\begin{anhang}\end{anhang}\newcommand{\dateiname}{L03039}\newcommand{\titel}{Arthur Schnitzler an Felix Salten, [9. 6. 1896?]}\newcommand{\editorInnen}{Martin Anton Müller und Laura Untner}%% latex-leseansicht-abspann.tex
%% Abspann für die Leseansicht.
%% Der Schalter \ifkorrekturansicht ist bereits durch den Vorspann gesetzt.

%% latex-abspann.tex
%% Gemeinsamer Abspann für Korrekturansicht und Leseansicht.
%% Setzt den Schalter \ifkorrekturansicht voraus (gesetzt in den
%% einbindenden Dateien latex-korrekturansicht-abspann.tex bzw.
%% latex-leseansicht-abspann.tex).
%% ---------------------------------------------------------------

\normalsize

% Das esempio-Environment wird nur in der Leseansicht benötigt
\ifkorrekturansicht\else
\newenvironment{esempio}[3]%
{
    \vspace{1.5ex}
    \rlap{\underline{#1}}
    \par
    \setlength{\parindent}{0cm}
    \nopagebreak
    \leftskip=#2cm
    \rightskip=#3cm
}
{
    \par
}
\fi

\doendnotes{C}
\bigskip
\vfill

\clearpage

\footnotesize

\ifkorrekturansicht
  \lohead{\textsc{register}}
\fi

% theindex-Environment neu definieren ohne reledmac
\makeatletter
\renewenvironment{theindex}{%
  \ifkorrekturansicht
    \section*{\indexname}%
  \else
    \subsubsection*{Index der erwähnten Entitäten}%
  \fi
  \setlength{\parindent}{0pt}%
  \setlength{\parskip}{0pt plus 0.3pt}%
  \let\item\@idxitem
}{%
  \ifkorrekturansicht\clearpage\fi
}
\makeatother

\IfFileExists{\jobname-pw.ind}{\input{\jobname-pw.ind}}{}

% Quellenangabe nur in der Leseansicht
\ifkorrekturansicht\else
% Fallback-Definitionen, falls die .tex-Datei \titel etc. nicht gesetzt hat
\providecommand{\titel}{}
\providecommand{\editorInnen}{}
\providecommand{\dateiname}{\jobname}

\vspace{3cm}

\vfill

\footnotesize
\textsc{Quelle}: \titel. Herausgegeben von {\editorInnen}. In: \emph{Arthur Schnitzler: Briefwechsel mit Autorinnen und Autoren}.
 Digitale Edition, https://schnitzler-briefe.acdh.oeaw.ac.at/{\dateiname}.html (Stand \today)
\fi

\end{document}


      