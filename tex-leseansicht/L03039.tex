%% latex-leseansicht-vorspann.tex
%% Vorspann für die Leseansicht.
%% Lädt die gemeinsame Datei latex-vorspann.tex mit nicht gesetztem Schalter.

\newif\ifkorrekturansicht
\korrekturansichtfalse

\input{../tex-inputs/latex-vorspann}


\section[ Arthur Schnitzler an Felix Salten, {[}15. 10. 1895?{]}]{L03039 Arthur Schnitzler an Felix Salten,  [15. 10. 1895?]}
\nopagebreak\mylabel{L03039v}
\rehead{ }\normalsize\beginnumbering\briefempfaengerindex{Salten, Felix@\textsc{Salten, Felix}!zzzSchnitzler, Arthur@\emph{von Arthur Schnitzler}!1895-10-151@{{[}15. 10. 1895?{]}}|(be}
\toendnotes[C]{\smallbreak\pagebreak[2]}
\correspDesc{Versand  durch Arthur Schnitzler am [15. 10. 1895?] in Wien
\newline{}Erhalt  durch Felix Salten am [15. 10. 1895?] in Wien}\toendnotes[C]{\smallbreak}
\Standort{Wienbibliothek im Rathaus, ZPH 1681, 2.1.516.}
\physDesc{Brief, 1 Blatt, 3 Seiten, 288 Zeichen
\newline{}Handschrift: Bleistift, deutsche Kurrent
\newline{}Ordnung: mit Bleistift von unbekannter Hand Nummerierung der Doppelseiten des
                                 Konvoluts: »22«–»23« }\toendnotes[C]{\smallbreak}
\pstart
           \raggedleft{}{\pb}Dinſtag\pend
           \vspace{0.5em}
\pstart
           lieber, wollen Sie heut{ }Abend mit mir in eine verborgne Loge\oindex{Wien@\textbf{Wien}!I., Innere Stadt@\textbf{I., Innere Stadt}!Burgtheater@\textbf{Burgtheater}, \emph{Theater}|pwuv} jener \label{K_L03039-1v}\edtext{Liebelei\pwindex{Schnitzler, Arthur 15.\,5.\,1862 Wien – 21.\,10.\,1931 ebd.@\textsc{Schnitzler, Arthur} (15.\,5.\,1862 Wien – 21.\,10.\,1931 ebd.), \emph{Schriftsteller, Mediziner}!Liebelei. Schauspiel in drei Akten@\strich\emph{Liebelei. Schauspiel in drei Akten}|pw}-Auffühg\eventindex{Burgtheater@\textbf{Burgtheater}!Aufführung von Liebelei, Rechte der Seele, 15.10.1895@Aufführung von Liebelei, Rechte der Seele, 15.10.1895|pwuv}}{\lemma{\textnormal{\emph{Liebelei-Auffühg}}}\Cendnote{\textnormal{Drei Dienstage, an denen Schnitzler in \emph{Liebelei}\pwindex{Schnitzler, Arthur 15.\,5.\,1862 Wien – 21.\,10.\,1931 ebd.@\textsc{Schnitzler, Arthur} (15.\,5.\,1862 Wien – 21.\,10.\,1931 ebd.), \emph{Schriftsteller, Mediziner}!Liebelei. Schauspiel in drei Akten@\strich\emph{Liebelei. Schauspiel in drei Akten}|pwk}-Aufführungen
                  war, bieten sich zur Datierung dieses Korrespondenzstücks an. Bei der am 15. 1. 1901 handelte
                  es sich um eine Inszenierung\eventindex{Kaufmännischer Verein@\textbf{Kaufmännischer Verein}!Aufführung von Liebelei, 15.1.1901@Aufführung von Liebelei, 15.1.1901|pwkv} von Schauspielschülerinnen im Kaufmännischen Verein\oindex{Wien@\textbf{Wien}!I., Innere Stadt@\textbf{I., Innere Stadt}!Kaufmännischer Verein@\textbf{Kaufmännischer Verein}|pwk}. Hier scheint die Existenz einer
                     »geheimen Loge« abwegig. Zur Aufführung\eventindex{Burgtheater@\textbf{Burgtheater}!Aufführung von Rechte der Seele, Liebelei, 9.6.1896@Aufführung von Rechte der Seele, Liebelei, 9.6.1896|pwkv}, die Schnitzler am 9. 6. 1896 besuchte, gibt es einen Brief, den Salten\pwindex{Salten, Felix 6.\,9.\,1869 Budapest – 8.\,10.\,1945 Zürich@\textsc{Salten, Felix} (6.\,9.\,1869 Budapest – 8.\,10.\,1945 Zürich), \emph{Schriftsteller, Journalist, Chefredakteur}|pwk} an diesem Tag Schnitzler sandte, XXXX Auszeichnungsfehler: Dokument L03172 nicht gefunden. Darin
                  deutete er an, am Abend möglicherweise verhindert zu sein, doch sind
                  nicht alle Fragen des vorliegenden Schreibens beantwortet, so dass dieser Brief
                  unabhängig vom vorliegenden Schreiben entstanden sein dürfte und sich nicht
                  zwingend eine Datierung daraus ergibt. Am wahrscheinlichsten scheint es, dass Schnitzler unmittelbar nach der Uraufführung\eventindex{Burgtheater@\textbf{Burgtheater}!Uraufführung von Liebelei, Premiere von Rechte der Seele, 9.10.1895@Uraufführung von Liebelei, Premiere von Rechte der Seele, 9.10.1895|pwkv} die 4. Aufführung\eventindex{Burgtheater@\textbf{Burgtheater}!Aufführung von Liebelei, Rechte der Seele, 15.10.1895@Aufführung von Liebelei, Rechte der Seele, 15.10.1895|pwkv} besuchte und
                  sich nicht zuletzt deshalb nicht zeigen wollte, weil er bereits nach dem 2. Akt
                  die Vorstellung verließ.}}}\label{K_L03039-1} gehen \introOben{}(\label{K_L03039-2v}\edtext{½ 8}{\lemma{\textnormal{\emph{½ 8}}}\Cendnote{\textnormal{Die Aufführung\eventindex{Burgtheater@\textbf{Burgtheater}!Aufführung von Liebelei, Rechte der Seele, 15.10.1895@Aufführung von Liebelei, Rechte der Seele, 15.10.1895|pwkv} war zwar für 7 Uhr
                     angesetzt, aber zwischen \emph{Rechte der Seele}\pwindex{\textcolor{red}{\textsuperscript{XXXX indx1}}!Rechte der Seele. Schauspiel in einem Act@\strich\emph{Rechte der Seele. Schauspiel in einem Act}|pwk}
                     und \emph{Liebelei}\pwindex{Schnitzler, Arthur 15.\,5.\,1862 Wien – 21.\,10.\,1931 ebd.@\textsc{Schnitzler, Arthur} (15.\,5.\,1862 Wien – 21.\,10.\,1931 ebd.), \emph{Schriftsteller, Mediziner}!Liebelei. Schauspiel in drei Akten@\strich\emph{Liebelei. Schauspiel in drei Akten}|pwk} fand eine längere Pause
                     statt.}}}\label{K_L03039-2})\introOben{},{ }ſo laſſen Sie michs gütigſt am frühen
                  Nachmittg wiſſen. Ich hole \substVorne{}\textsuperscript{\textcolor{gray}{ſ}}\substDazwischen{}S\substHinten{}ie da{\geminationn}, we{\geminationn}s Ihnen
                  {\pb}recht iſt, um \label{K_L03039-3v}\edtext{¼ 8}{\lemma{\textnormal{\emph{¼ 8}}}\Cendnote{\textnormal{19 Uhr 15}}}\label{K_L03039-3} oder ½ in
               Ihrer Wohnung\oindex{Wien@\textbf{Wien}!IX., Alsergrund@\textbf{IX., Alsergrund}!Hörlgasse@\textbf{Hörlgasse}, \emph{Straße}|pwv} ab?\pend
           
\pstart
           Herzlichſt {\\[\baselineskip]}Ihr {\\[\baselineskip]}\spacefill\mbox{Arth}\pend
           \leftskip=0em{}
\pstart
           \noindent{}{\pb}Und noch eins: ich habe \label{K_L03039-4v}\edtext{geſtern}{\lemma{\textnormal{\emph{gestern}}}\Cendnote{\textnormal{Wieso Schnitzler für den Vorabend ein Alibi benötigte,
                     erschließt sich aus dem \emph{Tagebuch}\pwindex{Schnitzler, Arthur 15.\,5.\,1862 Wien – 21.\,10.\,1931 ebd.@\textsc{Schnitzler, Arthur} (15.\,5.\,1862 Wien – 21.\,10.\,1931 ebd.), \emph{Schriftsteller, Mediziner}!Tagebuch@\strich\emph{Tagebuch}|pwk}
                     nicht.}}}\label{K_L03039-4} mit Ihnen im \label{K_L03039-5v}\edtext{Club\orgindex{Concordia. Journalisten- und Schriftstellerverein@Concordia. Journalisten- und Schriftstellerverein|pwuv}}{\lemma{\textnormal{\emph{Club}}}\Cendnote{\textnormal{Welcher Klub gemeint war, lässt sich
                     nicht mit Sicherheit bestimmen. Da Schnitzler seit zumindest 13. 10. 1889 Veranstaltungen des Clubs der \emph{Concordia}\orgindex{Concordia. Journalisten- und Schriftstellerverein@Concordia. Journalisten- und Schriftstellerverein|pwk} besuchte, könnte dieser gemeint
                     sein. In den \emph{Wiener Schachclub}\orgindex{Wiener Schachclub@Wiener Schachclub|pwk} trat er erst
                        Ende 1899 ein, was sich mit der gegenwärtigen Datierung nicht
                     vereinbaren lässt. Die Unsicherheit, welche Clubs Schnitzler frequentierte, kann auch als Hinweis auf die
                     durchaus beträchtlichen Lücken im verfügbaren Wissen über Schnitzler genommen werden, die trotz des \emph{Tagebuchs}\pwindex{Schnitzler, Arthur 15.\,5.\,1862 Wien – 21.\,10.\,1931 ebd.@\textsc{Schnitzler, Arthur} (15.\,5.\,1862 Wien – 21.\,10.\,1931 ebd.), \emph{Schriftsteller, Mediziner}!Tagebuch@\strich\emph{Tagebuch}|pwk} exisitieren.}}}\label{K_L03039-5} soupirt.\pend
           \selectlanguage{ngerman}\endnumbering\briefempfaengerindex{Salten, Felix@\textsc{Salten, Felix}!zzzSchnitzler, Arthur@\emph{von Arthur Schnitzler}!1895-10-151@{{[}15. 10. 1895?{]}}|)be}\mylabel{L03039h}  \newcommand{\dateiname}{L03039}\newcommand{\titel}{Arthur Schnitzler an Felix Salten, [15. 10. 1895?]}\newcommand{\editorInnen}{Martin Anton Müller und Laura Untner}%% latex-leseansicht-abspann.tex
%% Abspann für die Leseansicht.
%% Der Schalter \ifkorrekturansicht ist bereits durch den Vorspann gesetzt.

%% latex-abspann.tex
%% Gemeinsamer Abspann für Korrekturansicht und Leseansicht.
%% Setzt den Schalter \ifkorrekturansicht voraus (gesetzt in den
%% einbindenden Dateien latex-korrekturansicht-abspann.tex bzw.
%% latex-leseansicht-abspann.tex).
%% ---------------------------------------------------------------

\normalsize

% Das esempio-Environment wird nur in der Leseansicht benötigt
\ifkorrekturansicht\else
\newenvironment{esempio}[3]%
{
    \vspace{1.5ex}
    \rlap{\underline{#1}}
    \par
    \setlength{\parindent}{0cm}
    \nopagebreak
    \leftskip=#2cm
    \rightskip=#3cm
}
{
    \par
}
\fi

\doendnotes{C}
\bigskip
\vfill

\clearpage

\footnotesize

\ifkorrekturansicht
  \lohead{\textsc{register}}
\fi

% theindex-Environment neu definieren ohne reledmac
\makeatletter
\renewenvironment{theindex}{%
  \ifkorrekturansicht
    \section*{\indexname}%
  \else
    \subsubsection*{Index der erwähnten Entitäten}%
  \fi
  \setlength{\parindent}{0pt}%
  \setlength{\parskip}{0pt plus 0.3pt}%
  \let\item\@idxitem
}{%
  \ifkorrekturansicht\clearpage\fi
}
\makeatother

\IfFileExists{\jobname-pw.ind}{\input{\jobname-pw.ind}}{}

% Quellenangabe nur in der Leseansicht
\ifkorrekturansicht\else
% Fallback-Definitionen, falls die .tex-Datei \titel etc. nicht gesetzt hat
\providecommand{\titel}{}
\providecommand{\editorInnen}{}
\providecommand{\dateiname}{\jobname}

\vspace{3cm}

\vfill

\footnotesize
\textsc{Quelle}: \titel. Herausgegeben von {\editorInnen}. In: \emph{Arthur Schnitzler: Briefwechsel mit Autorinnen und Autoren}.
 Digitale Edition, https://schnitzler-briefe.acdh.oeaw.ac.at/{\dateiname}.html (Stand \today)
\fi

\end{document}


