%% latex-leseansicht-vorspann.tex
%% Vorspann für die Leseansicht.
%% Lädt die gemeinsame Datei latex-vorspann.tex mit nicht gesetztem Schalter.

\newif\ifkorrekturansicht
\korrekturansichtfalse

\input{../tex-inputs/latex-vorspann}


\section[ Felix Salten an Arthur Schnitzler, 22. 10. 1911]{L03554 Felix Salten an Arthur Schnitzler,  22. 10. 1911}
\nopagebreak\mylabel{L03554v}
\rehead{ }\normalsize\beginnumbering\briefempfaengerindex{Schnitzler, Arthur@\textsc{Schnitzler, Arthur}!zzzSalten, Felix@\emph{von Felix Salten}!1911-10-222@{22. 10. 1911}|(be}
\toendnotes[C]{\smallbreak\pagebreak[2]}
\correspDesc{Versand  durch Felix Salten am 22. 10. 1911 in Wien
\newline{}Erhalt  durch Arthur Schnitzler im Zeitraum [22. 10. 1911 – 24. 10. 1911?] in Wien}\toendnotes[C]{\smallbreak}
\Standort{CUL, Schnitzler, B 89, B 2.}
\physDesc{Brief, 1 Blatt, 2 Seiten, 5228 Zeichen
\newline{}Handschrift: schwarze Tinte, lateinische Kurrent
\newline{}Ordnung: mit Bleistift von unbekannter Hand nummeriert: »269« }\toendnotes[C]{\smallbreak}
\pstart
           {\pb}\textcolor{gray}{\textbf{\textsc{Felix Salten}}}\hfill Wien\oindex{Wien@\textbf{Wien}, \emph{Verwaltungsgebiet}|pw}, 22. X. 11\pend
           \vspace{0.5em}
\pstart
           Lieber, in einer sehr angenehmen Weise ergibt es sich mir aus Ihrem
               Brief, für den ich Ihnen bestens danke, dass Discussionen dieser Art zwischen uns
               durch keinen anderen Zusatz in ihrer Sachlichkeit entfärbt werden. Aus Ihrem Brief
               glaube ich ein gewisses Vertrauen in mein Verhältnis zu Ihren Arbeiten folgern zu
               dürfen, und das überhebt mich, Ihnen erst noch zu sagen, wie groß mein Respect und
               meine Zuneigung für jede produktive Arbeit im Allgemeinen und für die Ihrige im
               besonderen ist. Ohne weiters gebe ich Ihnen denn auch die Möglichkeit zu, dass Sie in
               allen Teilen recht haben. Doch kann ich mich des Eindrucks nicht erwehren, dass Sie
               meine Einwände anders auslegen, als ich sie gemeint habe, und möchte deshalb noch ein
               Wort darüber sagen. Zunächst, dass jenes Missverständnis in meinem Lloyd-Feuilleton\pwindex{Salten, Felix 6.\,9.\,1869 Budapest – 8.\,10.\,1945 Zürich@\textsc{Salten, Felix} (6.\,9.\,1869 Budapest – 8.\,10.\,1945 Zürich), \emph{Schriftsteller, Journalist, Chefredakteur}!Burgtheater. »Das weite Land.« Tragikomödie von Arthur Schnitzler@\strich\emph{Burgtheater. »Das weite Land.« Tragikomödie von Arthur Schnitzler}|pwv} garnicht besteht. Dort
               schrieb ich ja, dass Hofreiter\pwindex{Schnitzler, Arthur 15.\,5.\,1862 Wien – 21.\,10.\,1931 ebd.@\textsc{Schnitzler, Arthur} (15.\,5.\,1862 Wien – 21.\,10.\,1931 ebd.), \emph{Schriftsteller, Mediziner}!weite Land. Tragikomödie in fünf Akten@\strich\emph{Das weite Land. Tragikomödie in fünf Akten}|pwv}
               nicht bei dem Kinde bleiben, nicht im Vaterschaftsgefühl auslaufen könne. Ich hab das
               natürlich begriffen, und brauchte das »auf nach Amerika!\pwindex{Schnitzler, Arthur 15.\,5.\,1862 Wien – 21.\,10.\,1931 ebd.@\textsc{Schnitzler, Arthur} (15.\,5.\,1862 Wien – 21.\,10.\,1931 ebd.), \emph{Schriftsteller, Mediziner}!weite Land. Tragikomödie in fünf Akten@\strich\emph{Das weite Land. Tragikomödie in fünf Akten}|pwv}« umso weniger, als ja auch diese Reise
               zweifelhaft und im Grunde unwesentlich wäre. Im Lloyd\pwindex{Salten, Felix 6.\,9.\,1869 Budapest – 8.\,10.\,1945 Zürich@\textsc{Salten, Felix} (6.\,9.\,1869 Budapest – 8.\,10.\,1945 Zürich), \emph{Schriftsteller, Journalist, Chefredakteur}!Burgtheater. »Das weite Land.« Tragikomödie von Arthur Schnitzler@\strich\emph{Burgtheater. »Das weite Land.« Tragikomödie von Arthur Schnitzler}|pwv} steht das auch garnicht, wie Sie annehmen, als ein
               Gedanke von mir gegen eine (von mir missverstandene) Absicht von Ihnen. Vielmehr:
               dass dieser Kinder-Ruf ebenso wie die Antwort, die ihm wird, mir in ihrer »Anwendung«
               nicht überzeugend scheine. Wenn das Kind »Vater\pwindex{Schnitzler, Arthur 15.\,5.\,1862 Wien – 21.\,10.\,1931 ebd.@\textsc{Schnitzler, Arthur} (15.\,5.\,1862 Wien – 21.\,10.\,1931 ebd.), \emph{Schriftsteller, Mediziner}!weite Land. Tragikomödie in fünf Akten@\strich\emph{Das weite Land. Tragikomödie in fünf Akten}|pwv}« ruft, der Vater »ich komme\pwindex{Schnitzler, Arthur 15.\,5.\,1862 Wien – 21.\,10.\,1931 ebd.@\textsc{Schnitzler, Arthur} (15.\,5.\,1862 Wien – 21.\,10.\,1931 ebd.), \emph{Schriftsteller, Mediziner}!weite Land. Tragikomödie in fünf Akten@\strich\emph{Das weite Land. Tragikomödie in fünf Akten}|pwv}« antwortet, und diese beiden Akzente den Ausklang
               des Stück\pwindex{Schnitzler, Arthur 15.\,5.\,1862 Wien – 21.\,10.\,1931 ebd.@\textsc{Schnitzler, Arthur} (15.\,5.\,1862 Wien – 21.\,10.\,1931 ebd.), \emph{Schriftsteller, Mediziner}!weite Land. Tragikomödie in fünf Akten@\strich\emph{Das weite Land. Tragikomödie in fünf Akten}|pwv}es geben, mit ihrer
               inneren und in der Sekunde unwidersprechlich wirkenden Bedeutung den Ausklang des Stück\pwindex{Schnitzler, Arthur 15.\,5.\,1862 Wien – 21.\,10.\,1931 ebd.@\textsc{Schnitzler, Arthur} (15.\,5.\,1862 Wien – 21.\,10.\,1931 ebd.), \emph{Schriftsteller, Mediziner}!weite Land. Tragikomödie in fünf Akten@\strich\emph{Das weite Land. Tragikomödie in fünf Akten}|pw}es bestimmen, dann wird ein Anschein
               geweckt, meinte ich, ein Ausblick geöffnet, den doch das Besinnen der folgenden
               Sekunden schon verwirft. Mit Gründen, die ich ja anführte und die ja die Ihren sind.
               Noch genauer: es ist psychologisch sicherlich richig, dass Hofreiter\pwindex{Schnitzler, Arthur 15.\,5.\,1862 Wien – 21.\,10.\,1931 ebd.@\textsc{Schnitzler, Arthur} (15.\,5.\,1862 Wien – 21.\,10.\,1931 ebd.), \emph{Schriftsteller, Mediziner}!weite Land. Tragikomödie in fünf Akten@\strich\emph{Das weite Land. Tragikomödie in fünf Akten}|pwv}, von der Stimme seines Kindes
               getroffen, aufwimmert. In diesem Moment. Es mag auch richtig sein, dass er dem Knaben
               sofort entgegenstürzt, obwol er sich – in diesem Moment – auch nicht eben fähig
               fühlen könnte, ihn zu sehen. Dennoch: er gibt einer Augenblicksregung nach. Einer
               begreiflichen. Aber es ist zugleich auch der letzte Moment des ganzen Stück\pwindex{Schnitzler, Arthur 15.\,5.\,1862 Wien – 21.\,10.\,1931 ebd.@\textsc{Schnitzler, Arthur} (15.\,5.\,1862 Wien – 21.\,10.\,1931 ebd.), \emph{Schriftsteller, Mediziner}!weite Land. Tragikomödie in fünf Akten@\strich\emph{Das weite Land. Tragikomödie in fünf Akten}|pwv}es. Die stärkste Betonung also (glaubt
               man) desjenigen, was übrig bleibt. Mein Einwand gilt also nur der sekundenlang
               falschen Perspektive, die innerhalb der Komödie freilich eine psychologische Stütze
               sein kann, die aber an ihrem Schluß doch eine ganz andere Kunst-ökonomische Bedeutung
               hat. Mein Einwand ist der, dass hier eine absolut psychologische Richtigkeit mit
               einer dramatischen Richtigkeit kollidirt, wodurch beide aufgehoben werden. Was nicht
               geschähe, wenn Hofreiter\pwindex{Schnitzler, Arthur 15.\,5.\,1862 Wien – 21.\,10.\,1931 ebd.@\textsc{Schnitzler, Arthur} (15.\,5.\,1862 Wien – 21.\,10.\,1931 ebd.), \emph{Schriftsteller, Mediziner}!weite Land. Tragikomödie in fünf Akten@\strich\emph{Das weite Land. Tragikomödie in fünf Akten}|pwv}, vom
               Rufen seines Kindes ereilt, zwar aufwimmern würde, aber erstarrt, von allem, was er
               erlebt hat, geschwächt, regungslos stehen bliebe. Die Perspektive wäre dann die: dass
               jenes Kind im Garten draußen vergeblich ruft, und dass dem zerstörten Manne auf der
               Szene nichts mehr übrig ist. Und die letzte, stärkste Betonung des Stück\pwindex{Schnitzler, Arthur 15.\,5.\,1862 Wien – 21.\,10.\,1931 ebd.@\textsc{Schnitzler, Arthur} (15.\,5.\,1862 Wien – 21.\,10.\,1931 ebd.), \emph{Schriftsteller, Mediziner}!weite Land. Tragikomödie in fünf Akten@\strich\emph{Das weite Land. Tragikomödie in fünf Akten}|pwv}es wäre dann so eindringlich, dass sie
               keine Sekunde lang anders gedeutet werden könnte.\pend
           
\pstart
           Nur noch eines: ich habe nicht daran gedacht, ethische Bedenken vorzubringen, kann
               mich auch nicht besinnen, jemals Einwände der Moral gegen die Gestalten eines
               Kunstwerkes erhoben zu haben und wundere mich, dass Sie’s so nennen. Aber den
               menschlichen Inhalt einer Gestalt werden wir doch wol immer wägen. Das ist, abseits
               von Ethik, eine Frage des künstlerischen Materials und seiner Behandlung. So habe ich
               bei Hofreiter\pwindex{Schnitzler, Arthur 15.\,5.\,1862 Wien – 21.\,10.\,1931 ebd.@\textsc{Schnitzler, Arthur} (15.\,5.\,1862 Wien – 21.\,10.\,1931 ebd.), \emph{Schriftsteller, Mediziner}!weite Land. Tragikomödie in fünf Akten@\strich\emph{Das weite Land. Tragikomödie in fünf Akten}|pwv}, wenn ich sein
               Persönlichkeitsgewicht und den tragisch gewendeten Niederschwung des Stück\pwindex{Schnitzler, Arthur 15.\,5.\,1862 Wien – 21.\,10.\,1931 ebd.@\textsc{Schnitzler, Arthur} (15.\,5.\,1862 Wien – 21.\,10.\,1931 ebd.), \emph{Schriftsteller, Mediziner}!weite Land. Tragikomödie in fünf Akten@\strich\emph{Das weite Land. Tragikomödie in fünf Akten}|pwv}es zusammenhalte, seine Konsistenz an
               der Wucht des Ernstes meße, in den er gestellt ist, die {\pb}Empfindung, dass hier zwischen
               dem Material und seiner Behandlung irgendwelche Widersprüche bestehen. Widersprüche,
               die ich mir aus manchen Temperamentsquellen des Dichters gewiß erklären kann, auch
               damit, dass irgend ein tieferes Mitleben in Ihnen dem Hofreiter\pwindex{Schnitzler, Arthur 15.\,5.\,1862 Wien – 21.\,10.\,1931 ebd.@\textsc{Schnitzler, Arthur} (15.\,5.\,1862 Wien – 21.\,10.\,1931 ebd.), \emph{Schriftsteller, Mediziner}!weite Land. Tragikomödie in fünf Akten@\strich\emph{Das weite Land. Tragikomödie in fünf Akten}|pwv} gleichsam mit einer feinen Persönlichkeitsfaser
               noch verbunden blieb, dass ein letztes Loslösen und ganzes Freiwerden des Schöpfers
               vom Geschöpf dadurch nicht stattfand, und damit auch nicht dies freie, die ganze
               Komplexkeit der Figur überschauende Spiel des Schöpfers mit dem Geschöpf. Noch
               genauer: dass dasjenige, das der Ernst des Hofreiters\pwindex{Schnitzler, Arthur 15.\,5.\,1862 Wien – 21.\,10.\,1931 ebd.@\textsc{Schnitzler, Arthur} (15.\,5.\,1862 Wien – 21.\,10.\,1931 ebd.), \emph{Schriftsteller, Mediziner}!weite Land. Tragikomödie in fünf Akten@\strich\emph{Das weite Land. Tragikomödie in fünf Akten}|pwv} ist, sich nicht immer von Ihrem, des Dichters
               Ernst differenzirt, dass beides manchmal zusammenfließt, und sie aus einem Ursprung
               zu kommen scheint. Eine Gestalt scheint es jetzt, die gelegentlich noch von einer
               persönlichen Sentimentalität des Dichters umwittert ist. (Was Sie nicht mißverstehen
               werden.) Man brauchte aber diesen Hach nur wegzublasen und die vollkommenste Figur
               für die vollkommenste und edelste Komödie träte hervor.\pend
           
\pstart
           Es ist natürlich schwer für uns, für Sie wie für mich, über diese Dinge einig zu
               werden. Besonders in Briefen. Aber ich denke, wir haben im Allgemeinen und in Ihrer
               Arbeit so viele Treffpunkte, dass wir uns dieser einen Divergenz getrösten können. –
               Der Dagobert\pwindex{\textcolor{red}{\textsuperscript{XXXX indx1}}!gute König Dagobert. Lustspiel in vier Aufzügen@\strich\emph{Der gute König Dagobert. Lustspiel in vier Aufzügen}|pw} ist am \label{K_L03554-1v}\edtext{14. November. Am 13. (Montag)}{\lemma{\textnormal{\emph{14. November. Am 13. (Montag)}}}\Cendnote{\textnormal{Das verzögerte sich auf
                  den 17. 11. 1911 – für die Generalprobe – und den 18. 11. 1911 für die Premiere.}}}\label{K_L03554-1} ist die
               Generalprobe, und ich werde mich natürlich sehr freuen, wenn Sie Beide\pwindex{Schnitzler, Olga 17.\,1.\,1882 Wien – 13.\,1.\,1970 Lugano@\textsc{Schnitzler, Olga} (17.\,1.\,1882 Wien – 13.\,1.\,1970 Lugano), \emph{Schauspielerin, Sängerin}|pwv} kommen wollen. Inzwischen hoffe ich,
               Sie noch zu sehen. Meine Frau\pwindex{Salten, Ottilie 7.\,3.\,1868 Prag – 22.\,6.\,1942 Zürich@\textsc{Salten, Ottilie} (7.\,3.\,1868 Prag – 22.\,6.\,1942 Zürich), \emph{Schauspielerin}|pwv} ist für wenige Tage in Berlin\oindex{Berlin@\textbf{Berlin}, \emph{Hauptstadt}|pw}.\pend
           
\pstart
           Herzlichste Grüße {\\[\baselineskip]}Ihr {\\[\baselineskip]}\spacefill\mbox{Salten}\pend
           \leftskip=0em{}\selectlanguage{ngerman}\endnumbering\briefempfaengerindex{Schnitzler, Arthur@\textsc{Schnitzler, Arthur}!zzzSalten, Felix@\emph{von Felix Salten}!1911-10-222@{22. 10. 1911}|)be}\mylabel{L03554h}  \newcommand{\dateiname}{L03554}\newcommand{\titel}{Felix Salten an Arthur Schnitzler, 22. 10. 1911}\newcommand{\editorInnen}{Martin Anton Müller und Laura Untner}%% latex-leseansicht-abspann.tex
%% Abspann für die Leseansicht.
%% Der Schalter \ifkorrekturansicht ist bereits durch den Vorspann gesetzt.

%% latex-abspann.tex
%% Gemeinsamer Abspann für Korrekturansicht und Leseansicht.
%% Setzt den Schalter \ifkorrekturansicht voraus (gesetzt in den
%% einbindenden Dateien latex-korrekturansicht-abspann.tex bzw.
%% latex-leseansicht-abspann.tex).
%% ---------------------------------------------------------------

\normalsize

% Das esempio-Environment wird nur in der Leseansicht benötigt
\ifkorrekturansicht\else
\newenvironment{esempio}[3]%
{
    \vspace{1.5ex}
    \rlap{\underline{#1}}
    \par
    \setlength{\parindent}{0cm}
    \nopagebreak
    \leftskip=#2cm
    \rightskip=#3cm
}
{
    \par
}
\fi

\doendnotes{C}
\bigskip
\vfill

\clearpage

\footnotesize

\ifkorrekturansicht
  \lohead{\textsc{register}}
\fi

% theindex-Environment neu definieren ohne reledmac
\makeatletter
\renewenvironment{theindex}{%
  \ifkorrekturansicht
    \section*{\indexname}%
  \else
    \subsubsection*{Index der erwähnten Entitäten}%
  \fi
  \setlength{\parindent}{0pt}%
  \setlength{\parskip}{0pt plus 0.3pt}%
  \let\item\@idxitem
}{%
  \ifkorrekturansicht\clearpage\fi
}
\makeatother

\IfFileExists{\jobname-pw.ind}{\input{\jobname-pw.ind}}{}

% Quellenangabe nur in der Leseansicht
\ifkorrekturansicht\else
% Fallback-Definitionen, falls die .tex-Datei \titel etc. nicht gesetzt hat
\providecommand{\titel}{}
\providecommand{\editorInnen}{}
\providecommand{\dateiname}{\jobname}

\vspace{3cm}

\vfill

\footnotesize
\textsc{Quelle}: \titel. Herausgegeben von {\editorInnen}. In: \emph{Arthur Schnitzler: Briefwechsel mit Autorinnen und Autoren}.
 Digitale Edition, https://schnitzler-briefe.acdh.oeaw.ac.at/{\dateiname}.html (Stand \today)
\fi

\end{document}


