%% latex-korrekturansicht-vorspann.tex
%% Vorspann für die Korrekturansicht.
%% Lädt die gemeinsame Datei latex-vorspann.tex mit gesetztem Schalter.

\newif\ifkorrekturansicht
\korrekturansichttrue

\input{../tex-inputs/latex-vorspann}


\section[Arthur Schnitzler an Wilhelm Bölsche, 1. 6. 1893]{L00215 Arthur Schnitzler an Wilhelm Bölsche, 1. 6. 1893}
\nopagebreak\mylabel{L00215v}
\rehead{ }\normalsize\beginnumbering\briefempfaengerindex{Boelsche, Wilhelm@\textsc{Bölsche, Wilhelm}!zzzSchnitzler, Arthur@\emph{von Arthur Schnitzler}!1893-06-011@{1. 6. 1893}|(be}
\toendnotes[C]{\smallbreak\pagebreak[2]}\Standort{Wrocław, Biblioteka Uniwersytecka, Böl.Pis 1767.}
\physDesc{Brief, 1 Blatt, 4 Seiten, 1041 Zeichen (Briefpapier mit Trauerrand)
\newline{}Handschrift: schwarze Tinte, deutsche Kurrent}
\buchAbdrucke{\weitereDrucke{1) \emph{Germanica Wratislaviensia} (1987) Nr. 77, S. 461–462.} \weitereDrucke{2) Wilhelm Bölsche: \emph{Briefwechsel. Mit Autoren der Freien Bühne}. Berlin: \emph{Weidler} 2010, S. 685.} }\toendnotes[C]{\smallbreak}
\pstart
           \raggedleft{}{\pb}1. Juni 93\pend
           
\pstart{}Sehr geehrter Herr\label{K_L00215-1v}\edtext{Doktor}{\lemma{\textnormal{\emph{Doktor}}}\Cendnote{\textnormal{Bölsche\pwindex{Boelsche, Wilhelm 02.01.1861 – 31.08.1939@\textsc{Bölsche, Wilhelm} (02.01.1861 – 31.08.1939), \emph{Schriftsteller/Schriftstellerin, Publizist/Publizistin}|pwk} hatte zwar studiert, aber keinen
                     Universitätsabschluss.}}}\label{K_L00215-1},\pend\vspace{0.5em}
\pstart
           eine Frage: Wollen Sie mein dreiaktiges Schauſpiel \uline{Das Märchen}\pwindex{Maerchen. Schauspiel in drei Aufzuegen@\emph{Das Märchen. Schauspiel in drei Aufzügen}|pw}, welches nächſte Saiſon am Leſſingtheater\orgindex{Lessing-Theater@Lessing-Theater|pw}
               zur Aufführung kommt, in der \uline{Freien Bühne}\pwindex{Freie Buehne fuer den Entwickelungskampf der Zeit@\emph{Freie Bühne für den Entwickelungskampf der Zeit}|pw} bringen? Falls Sie im Princip einverſtanden ſind, ſo erlaube ich mir die
               weitere Frage, \uline{unter welchen Bedingungen}{ }{\pb}und wann Sie mit der Veröffentlichung begi{\geminationn}en kö{\geminationn}ten. Mir läge daran,
               daſs der erſte Akt ſchon im Juliheft erſchiene – das Stück ſelbſt hab ich \strikeout{vor} Ihnen vor etwa 1 Jahre als Manuscript gedruckt,
               eingeſchickt; ich ſende Ihnen natürlich ein andres Exemplar, ſobald Sie das Drama
               veröffentlichen wollen. –\pend
           
\pstart
           Vor etwa 6 oder 7 Wochen hab {\pb}ich Ihnen eine kleine Skizze
               geſandt »Die Braut\pwindex{Braut@\emph{Die Braut}|pw}« – was iſt’s mit der? –\pend
           
\pstart
           – Jedenfalls will ich noch das höfliche Erſuchen hinzuſetzen, mich nicht zu lang auf
               Antwort warten zu laſſen; es kommt mir auf eine raſche Erledigung meiner Frage an,
               und ich appellire an Ihre Liebenswürdigkeit, mir Ihre Entſcheidung in möglichſt
               kurzer Zeit zu{\pb}ko{\geminationm}en zu
               laſſen.\pend
           
\pstart
           Mit beſondrer Hochachtung{\\[\baselineskip]}\spacefill\mbox{Dr Arthur Schnitzler}\pend
           \leftskip=0em{}
\pstart
           \noindent{}\textsc{Wien I. Grillparzerstraße 7}\oindex{Grillparzerstrasse@\textbf{Grillparzerstraße}, \emph{R.ST}|pw}.\pend
           \selectlanguage{ngerman}\endnumbering\briefempfaengerindex{Boelsche, Wilhelm@\textsc{Bölsche, Wilhelm}!zzzSchnitzler, Arthur@\emph{von Arthur Schnitzler}!1893-06-011@{1. 6. 1893}|)be}\mylabel{L00215h}  \normalsize

\doendnotes{C}
\bigskip
\vfill

\clearpage

\footnotesize

\lohead{\textsc{register}}

% Definiere theindex-Environment komplett neu ohne reledmac
\makeatletter
\renewenvironment{theindex}{%
  \section*{\indexname}%
  \setlength{\parindent}{0pt}%
  \setlength{\parskip}{0pt plus 0.3pt}%
  \let\item\@idxitem
}{%
  \clearpage
}
\makeatother

\IfFileExists{\jobname-pw.ind}{\input{\jobname-pw.ind}}{}

\end{document}

      