%% latex-leseansicht-vorspann.tex
%% Vorspann für die Leseansicht.
%% Lädt die gemeinsame Datei latex-vorspann.tex mit nicht gesetztem Schalter.

\newif\ifkorrekturansicht
\korrekturansichtfalse

\input{../tex-inputs/latex-vorspann}


\section[Arthur Schnitzler an Wilhelm Bölsche, 1. 6. 1893]{L00215 Arthur Schnitzler an Wilhelm Bölsche, 1. 6. 1893}
\nopagebreak\mylabel{L00215v}
\rehead{ }\normalsize\beginnumbering\briefempfaengerindex{Bölsche, Wilhelm@\textsc{Bölsche, Wilhelm}!zzzSchnitzler, Arthur@\emph{von Arthur Schnitzler}!1893-06-011@{1. 6. 1893}|(be}
\toendnotes[C]{\smallbreak\pagebreak[2]}
\correspDesc{Versand  durch Arthur Schnitzler am 1. 6. 1893 in Wien
\newline{}Erhalt  durch Wilhelm Bölsche im Zeitraum [2. 6. 1893
                  – 6. 6. 1893?] in Berlin}\toendnotes[C]{\smallbreak}
\Standort{Wrocław, Biblioteka Uniwersytecka, Böl.Pis 1767.}
\physDesc{Brief, 1 Blatt, 4 Seiten, 1041 Zeichen (Briefpapier mit Trauerrand)
\newline{}Handschrift: schwarze Tinte, deutsche Kurrent}
\buchAbdrucke{\weitereDrucke{1) Alois Woldan: \emph{Arthur Schnitzler – Briefe an Wilhelm Bölsche.} In: \emph{Germanica Wratislaviensia} (1987) Nr. 77, S. 461–462.} \weitereDrucke{2) Wilhelm Bölsche: \emph{Briefwechsel. Mit Autoren der Freien Bühne}. Herausgegeben von Gerd-Hermann Susen. Berlin: \emph{Weidler} 2010, S. 685 (Werke und Briefe. Wissenschaftliche Ausgabe, Briefe I).} }\toendnotes[C]{\smallbreak}
\pstart
           \raggedleft{}{\pb}1. Juni 93\pend
           
\pstart{}Sehr geehrter Herr\label{K_L00215-1v}\edtext{Doktor}{\lemma{\textnormal{\emph{Doktor}}}\Cendnote{\textnormal{Bölsche\pwindex{Bölsche, Wilhelm 2.\,1.\,1861 Köln – 31.\,8.\,1939 Szklarska Poręba@\textsc{Bölsche, Wilhelm} (2.\,1.\,1861 Köln – 31.\,8.\,1939 Szklarska Poręba), \emph{Schriftsteller, Publizist}|pwk} hatte zwar studiert, aber keinen
                     Universitätsabschluss.}}}\label{K_L00215-1},\pend\vspace{0.5em}
\pstart
           eine Frage: Wollen Sie mein dreiaktiges Schauſpiel \uline{Das Märchen}\pwindex{Schnitzler, Arthur 15.\,5.\,1862 Wien – 21.\,10.\,1931 ebd.@\textsc{Schnitzler, Arthur} (15.\,5.\,1862 Wien – 21.\,10.\,1931 ebd.), \emph{Schriftsteller, Mediziner}!Märchen. Schauspiel in drei Aufzügen@\strich\emph{Das Märchen. Schauspiel in drei Aufzügen}|pw}, welches nächſte Saiſon am Leſſingtheater\orgindex{Lessing-Theater@Lessing-Theater|pw}
               zur Aufführung kommt, in der \uline{Freien Bühne}\pwindex{Freie Bühne für den Entwickelungskampf der Zeit@\emph{Freie Bühne für den Entwickelungskampf der Zeit}|pw} bringen? Falls Sie im Princip einverſtanden{ }ſind,{ }ſo erlaube ich mir die
               weitere Frage, \uline{unter welchen Bedingungen}{ }{\pb}und wann Sie mit der Veröffentlichung begi{\geminationn}en kö{\geminationn}ten. Mir läge daran,
               daſs der erſte Akt{ }ſchon im Juliheft erſchiene – das Stück{ }ſelbſt hab ich \strikeout{vor} Ihnen vor etwa 1 Jahre als Manuscript gedruckt,
               eingeſchickt; ich{ }ſende Ihnen natürlich ein andres Exemplar,{ }ſobald Sie das Drama
               veröffentlichen wollen. –\pend
           
\pstart
           Vor etwa 6 oder 7 Wochen hab {\pb}ich Ihnen eine kleine Skizze
               geſandt »Die Braut\pwindex{Schnitzler, Arthur 15.\,5.\,1862 Wien – 21.\,10.\,1931 ebd.@\textsc{Schnitzler, Arthur} (15.\,5.\,1862 Wien – 21.\,10.\,1931 ebd.), \emph{Schriftsteller, Mediziner}!Braut@\strich\emph{Die Braut}|pw}« – was iſt’s mit der? –\pend
           
\pstart
           – Jedenfalls will ich noch das höfliche Erſuchen hinzuſetzen, mich nicht zu lang auf
               Antwort warten zu laſſen; es kommt mir auf eine raſche Erledigung meiner Frage an,
               und ich appellire an Ihre Liebenswürdigkeit, mir Ihre Entſcheidung in möglichſt
               kurzer Zeit zu{\pb}ko{\geminationm}en zu
               laſſen.\pend
           
\pstart
           Mit beſondrer Hochachtung{\\[\baselineskip]}\spacefill\mbox{Dr Arthur Schnitzler}\pend
           \leftskip=0em{}
\pstart
           \noindent{}\textsc{Wien I. Grillparzerstraße 7}\oindex{Wien@\textbf{Wien}!I., Innere Stadt@\textbf{I., Innere Stadt}!Grillparzerstraße@\textbf{Grillparzerstraße}, \emph{Straße}|pw}.\pend
           \selectlanguage{ngerman}\endnumbering\briefempfaengerindex{Bölsche, Wilhelm@\textsc{Bölsche, Wilhelm}!zzzSchnitzler, Arthur@\emph{von Arthur Schnitzler}!1893-06-011@{1. 6. 1893}|)be}\mylabel{L00215h}  \newcommand{\dateiname}{L00215}\newcommand{\titel}{Arthur Schnitzler an Wilhelm Bölsche, 1. 6. 1893}\newcommand{\editorInnen}{Martin Anton Müller und Gerd-Hermann Susen}%% latex-leseansicht-abspann.tex
%% Abspann für die Leseansicht.
%% Der Schalter \ifkorrekturansicht ist bereits durch den Vorspann gesetzt.

%% latex-abspann.tex
%% Gemeinsamer Abspann für Korrekturansicht und Leseansicht.
%% Setzt den Schalter \ifkorrekturansicht voraus (gesetzt in den
%% einbindenden Dateien latex-korrekturansicht-abspann.tex bzw.
%% latex-leseansicht-abspann.tex).
%% ---------------------------------------------------------------

\normalsize

% Das esempio-Environment wird nur in der Leseansicht benötigt
\ifkorrekturansicht\else
\newenvironment{esempio}[3]%
{
    \vspace{1.5ex}
    \rlap{\underline{#1}}
    \par
    \setlength{\parindent}{0cm}
    \nopagebreak
    \leftskip=#2cm
    \rightskip=#3cm
}
{
    \par
}
\fi

\doendnotes{C}
\bigskip
\vfill

\clearpage

\footnotesize

\ifkorrekturansicht
  \lohead{\textsc{register}}
\fi

% theindex-Environment neu definieren ohne reledmac
\makeatletter
\renewenvironment{theindex}{%
  \ifkorrekturansicht
    \section*{\indexname}%
  \else
    \subsubsection*{Index der erwähnten Entitäten}%
  \fi
  \setlength{\parindent}{0pt}%
  \setlength{\parskip}{0pt plus 0.3pt}%
  \let\item\@idxitem
}{%
  \ifkorrekturansicht\clearpage\fi
}
\makeatother

\IfFileExists{\jobname-pw.ind}{\input{\jobname-pw.ind}}{}

% Quellenangabe nur in der Leseansicht
\ifkorrekturansicht\else
% Fallback-Definitionen, falls die .tex-Datei \titel etc. nicht gesetzt hat
\providecommand{\titel}{}
\providecommand{\editorInnen}{}
\providecommand{\dateiname}{\jobname}

\vspace{3cm}

\vfill

\footnotesize
\textsc{Quelle}: \titel. Herausgegeben von {\editorInnen}. In: \emph{Arthur Schnitzler: Briefwechsel mit Autorinnen und Autoren}.
 Digitale Edition, https://schnitzler-briefe.acdh.oeaw.ac.at/{\dateiname}.html (Stand \today)
\fi

\end{document}


