%% latex-leseansicht-vorspann.tex
%% Vorspann für die Leseansicht.
%% Lädt die gemeinsame Datei latex-vorspann.tex mit nicht gesetztem Schalter.

\newif\ifkorrekturansicht
\korrekturansichtfalse

\input{../tex-inputs/latex-vorspann}


\section[Arthur Schnitzler an Richard Beer-Hofmann, 4. 11. 1896]{L00616 Arthur Schnitzler an Richard Beer-Hofmann, 4. 11. 1896}
\nopagebreak\mylabel{L00616v}
\rehead{ }\normalsize\beginnumbering\briefempfaengerindex{Beer-Hofmann, Richard@\textsc{Beer-Hofmann, Richard}!zzzSchnitzler, Arthur@\emph{von Arthur Schnitzler}!1896-11-042@{4. 11. 1896}|(be}
\toendnotes[C]{\smallbreak\pagebreak[2]}
\correspDesc{Versand  durch Arthur Schnitzler am 4. 11. 1896 in Berlin
\newline{}Erhalt  durch Richard Beer-Hofmann am 4. 11. 1896 in Wien}\toendnotes[C]{\smallbreak}
\Standort{YCGL, MSS 31.}
\physDesc{Telegramm, 121 Zeichen
\newline{}maschinell
\newline{}Versand: 1) Stempel: »\nobreak{}\oindex{I., Innere Stadt@\textbf{I., Innere Stadt}, \emph{Verwaltungsgebiet}|pwk}Wien 1/1, 14 XI 96, 8 40 N\nobreak{}«.   2) mit dem Namen des empfangenden Telegrafenbeamten bestempelt: »\noindent{}\textcolor{gray}{\textbf{\textit{11 Bln.\oindex{Berlin@\textbf{Berlin}, \emph{Hauptstadt}|pw}
                                          42}}}{ / }\textcolor{gray}{\textbf{\textit{Krejči\pwindex{Krejči 1896 – 1896@\textsc{Krejči} (1896 – 1896), \emph{Telegrafenbeamter/Telegrafenbeamtin}|pw}}}}.«}\toendnotes[C]{\smallbreak}\pstart{}{\pb}richard beer-hofmann wien\oindex{Wien@\textbf{Wien}, \emph{Verwaltungsgebiet}|pw}\pend{}\pstart{}wollzeile 15\oindex{Wien@\textbf{Wien}!I., Innere Stadt@\textbf{I., Innere Stadt}!Wollzeile 15 (»Berthahof«)@\textbf{Wollzeile 15 (»Berthahof«)}, \emph{Wohngebäude}|pw}\pend{}{\bigskip}\vspace{1em}
\pstart
           \noindent{}{\pb}Wien\oindex{Wien@\textbf{Wien}, \emph{Verwaltungsgebiet}|pw} de berlin\oindex{Berlin@\textbf{Berlin}, \emph{Hauptstadt}|pw}
               1407 15 7 36 =\pend
           
\pstart
           = herzlichen dank ihnen und den andern\pwindex{Schwarzkopf, Gustav 7.\,11.\,1853 Wien – 13.\,11.\,1939 ebd.@\textsc{Schwarzkopf, Gustav} (7.\,11.\,1853 Wien – 13.\,11.\,1939 ebd.), \emph{Schriftsteller}|pwv}\pwindex{Schwarzkopf, Max 12.\,6.\,1857 Wien – 14.\,4.\,1928 ebd.@\textsc{Schwarzkopf, Max} (12.\,6.\,1857 Wien – 14.\,4.\,1928 ebd.), \emph{Rechtsanwalt}|pwv}\pwindex{Schwarzkopf, Emil 17.\,9.\,1851 Wien – 28.\,1.\,1928 ebd.@\textsc{Schwarzkopf, Emil} (17.\,9.\,1851 Wien – 28.\,1.\,1928 ebd.), \emph{Übersetzer, Komponist, Musiklehrer}|pwv}\pwindex{Léon, Victor 4.\,1.\,1858 Senica – 23.\,2.\,1940 Wien@\textsc{Léon, Victor} (4.\,1.\,1858 Senica – 23.\,2.\,1940 Wien), \emph{Schriftsteller, Dramaturg}|pwv}\pwindex{Feld, Leo 14.\,2.\,1869 Augsburg – 5.\,9.\,1924 Florenz@\textsc{Feld, Leo} (14.\,2.\,1869 Augsburg – 5.\,9.\,1924 Florenz), \emph{Schriftsteller, Übersetzer, Dirigent}|pwv}\pwindex{Engel, Alexander 10.\,4.\,1868 Necpaly – 17.\,11.\,1940 Wien@\textsc{Engel, Alexander} (10.\,4.\,1868 Necpaly – 17.\,11.\,1940 Wien), \emph{Schriftsteller, Journalist}|pwv} viele gruesse \spacefill\mbox{arthur. +}\pend
           \selectlanguage{ngerman}\endnumbering\briefempfaengerindex{Beer-Hofmann, Richard@\textsc{Beer-Hofmann, Richard}!zzzSchnitzler, Arthur@\emph{von Arthur Schnitzler}!1896-11-042@{4. 11. 1896}|)be}\mylabel{L00616h}  \newcommand{\dateiname}{L00616}\newcommand{\titel}{Arthur Schnitzler an Richard Beer-Hofmann, 4. 11. 1896}\newcommand{\editorInnen}{Martin Anton Müller und Gerd-Hermann Susen}%% latex-leseansicht-abspann.tex
%% Abspann für die Leseansicht.
%% Der Schalter \ifkorrekturansicht ist bereits durch den Vorspann gesetzt.

%% latex-abspann.tex
%% Gemeinsamer Abspann für Korrekturansicht und Leseansicht.
%% Setzt den Schalter \ifkorrekturansicht voraus (gesetzt in den
%% einbindenden Dateien latex-korrekturansicht-abspann.tex bzw.
%% latex-leseansicht-abspann.tex).
%% ---------------------------------------------------------------

\normalsize

% Das esempio-Environment wird nur in der Leseansicht benötigt
\ifkorrekturansicht\else
\newenvironment{esempio}[3]%
{
    \vspace{1.5ex}
    \rlap{\underline{#1}}
    \par
    \setlength{\parindent}{0cm}
    \nopagebreak
    \leftskip=#2cm
    \rightskip=#3cm
}
{
    \par
}
\fi

\doendnotes{C}
\bigskip
\vfill

\clearpage

\footnotesize

\ifkorrekturansicht
  \lohead{\textsc{register}}
\fi

% theindex-Environment neu definieren ohne reledmac
\makeatletter
\renewenvironment{theindex}{%
  \ifkorrekturansicht
    \section*{\indexname}%
  \else
    \subsubsection*{Index der erwähnten Entitäten}%
  \fi
  \setlength{\parindent}{0pt}%
  \setlength{\parskip}{0pt plus 0.3pt}%
  \let\item\@idxitem
}{%
  \ifkorrekturansicht\clearpage\fi
}
\makeatother

\IfFileExists{\jobname-pw.ind}{\input{\jobname-pw.ind}}{}

% Quellenangabe nur in der Leseansicht
\ifkorrekturansicht\else
% Fallback-Definitionen, falls die .tex-Datei \titel etc. nicht gesetzt hat
\providecommand{\titel}{}
\providecommand{\editorInnen}{}
\providecommand{\dateiname}{\jobname}

\vspace{3cm}

\vfill

\footnotesize
\textsc{Quelle}: \titel. Herausgegeben von {\editorInnen}. In: \emph{Arthur Schnitzler: Briefwechsel mit Autorinnen und Autoren}.
 Digitale Edition, https://schnitzler-briefe.acdh.oeaw.ac.at/{\dateiname}.html (Stand \today)
\fi

\end{document}


