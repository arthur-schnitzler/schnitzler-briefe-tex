%% latex-korrekturansicht-vorspann.tex
%% Vorspann für die Korrekturansicht.
%% Lädt die gemeinsame Datei latex-vorspann.tex mit gesetztem Schalter.

\newif\ifkorrekturansicht
\korrekturansichttrue

\input{../tex-inputs/latex-vorspann}


\section[Hugo von Hofmannsthal an Arthur Schnitzler, {[}16. 1. 1899{]}]{L00881 Hugo von Hofmannsthal an Arthur Schnitzler, {[}16. 1. 1899{]}}
\nopagebreak\mylabel{L00881v}
\rehead{ }\normalsize\beginnumbering\briefempfaengerindex{Schnitzler, Arthur@\textsc{Schnitzler, Arthur}!zzzHofmannsthal, Hugo von@\emph{von Hugo von Hofmannsthal}!1899-01-161@{{[}16. 1. 1899{]}}|(be}
\toendnotes[C]{\smallbreak\pagebreak[2]}\Standort{CUL, Schnitzler, B 43.}
\physDesc{Brief, 1 Blatt, 1 Seite, 378 Zeichen
\newline{}Handschrift: schwarze Tinte, deutsche Kurrent
\newline{}Schnitzler: mit Bleistift datiert: »Jän 99?« 
\newline{}Ordnung: 1) mit Bleistift von unbekannter Hand nummeriert: »\strikeout{132}«  2) mit Bleistift von unbekannter Hand nummeriert:
                                    »129«}
\buchAbdrucke{\weitereDrucke{Hugo von Hofmannsthal, Arthur Schnitzler: \emph{Briefwechsel}. Frankfurt am Main: \emph{S. Fischer} 1964, S. 115.} }\toendnotes[C]{\smallbreak}
\pstart
           \raggedleft{}{\pb}Montag{ }abend\pend
           
\pstart{}lieber Arthur,\pend\vspace{0.5em}
\pstart
           es möchte mir \uline{ſehr} viel dranliegen ſchon morgen
                  \label{K_L00881-1v}\edtext{Dienstag}{\lemma{\textnormal{\emph{Dienstag}}}\Cendnote{\textnormal{Die Datierung Schnitzlers dürfte stimmen, am Dienstag, den
                     17. 1. 1899 las
                     Hofmannsthal\pwindex{Hofmannsthal, Hugo von 1874-02-01 – 1929-07-15@\textsc{Hofmannsthal, Hugo von} (1874-02-01 – 1929-07-15), \emph{Schriftsteller/Schriftstellerin}|pwk} bei ihm \emph{Der Abenteurer und die Sängerin}\pwindex{Abenteurer und die Saengerin oder Die Geschenke des Lebens@\emph{Der Abenteurer und die Sängerin oder Die Geschenke des Lebens}|pwk} vor. Neben anderen war auch
                  der in der Folge angesprochene Richard
                     Beer-Hofmann\pwindex{Beer-Hofmann, Richard 1866-07-11 – 1945-09-26@\textsc{Beer-Hofmann, Richard} (1866-07-11 – 1945-09-26), \emph{Schriftsteller/Schriftstellerin}|pwk} anwesend.}}}\label{K_L00881-1}{ }abend bei Ihnen zu leſen. Wenn es Ihnen paſst ſchreiben Sie bitte gar
               nicht, dann ko{\geminationm}e ich von ſelbſt um ½ 9, und
                  Richard\pwindex{Beer-Hofmann, Richard 1866-07-11 – 1945-09-26@\textsc{Beer-Hofmann, Richard} (1866-07-11 – 1945-09-26), \emph{Schriftsteller/Schriftstellerin}|pw} um ½ 10. Kö{\geminationn}en Sie ſich aber nicht frei machen, dann ſchreiben Sie
               mir und Richard\pwindex{Beer-Hofmann, Richard 1866-07-11 – 1945-09-26@\textsc{Beer-Hofmann, Richard} (1866-07-11 – 1945-09-26), \emph{Schriftsteller/Schriftstellerin}|pw} umgehend, ob wir beide
                  Mittwoch ko{\geminationm}en ſollen. Mir wär aber halt
               morgen viel lieber.\pend
           
\pstart
           Von Herzen Ihr{\\[\baselineskip]}\spacefill\mbox{Hugo.}\pend
           \leftskip=0em{}\selectlanguage{ngerman}\endnumbering\briefempfaengerindex{Schnitzler, Arthur@\textsc{Schnitzler, Arthur}!zzzHofmannsthal, Hugo von@\emph{von Hugo von Hofmannsthal}!1899-01-161@{{[}16. 1. 1899{]}}|)be}\mylabel{L00881h}  \normalsize

\doendnotes{C}
\bigskip
\vfill

\clearpage

\footnotesize

\lohead{\textsc{register}}

% Definiere theindex-Environment komplett neu ohne reledmac
\makeatletter
\renewenvironment{theindex}{%
  \section*{\indexname}%
  \setlength{\parindent}{0pt}%
  \setlength{\parskip}{0pt plus 0.3pt}%
  \let\item\@idxitem
}{%
  \clearpage
}
\makeatother

\IfFileExists{\jobname-pw.ind}{\input{\jobname-pw.ind}}{}

\end{document}

      