%% latex-korrekturansicht-vorspann.tex
%% Vorspann für die Korrekturansicht.
%% Lädt die gemeinsame Datei latex-vorspann.tex mit gesetztem Schalter.

\newif\ifkorrekturansicht
\korrekturansichttrue

\input{../tex-inputs/latex-vorspann}


\section[Arthur Schnitzler an Stefan Zweig, 16. 2. 1916]{L03770 Arthur Schnitzler an Stefan Zweig, 16. 2. 1916}
\nopagebreak\mylabel{L03770v}
\rehead{ }\normalsize\beginnumbering\briefempfaengerindex{Zweig, Stefan@\textsc{Zweig, Stefan}!zzzSchnitzler, Arthur@\emph{von Arthur Schnitzler}!1916-02-161@{16. 2. 1916}|(be}
\toendnotes[C]{\smallbreak\pagebreak[2]}\Standort{Jerusalem, National Library of Israel, ARC. Ms. Var. 305 1 58 Stefan Zweig Collection.}
\physDesc{Postkarte, 1 Blatt, 2 Seiten, 400 Zeichen
\newline{}Schreibmaschine
\newline{}Handschrift: schwarze Tinte (\noindent{}Unterschrift)
\newline{}Versand: Stempel: »\nobreak{}\oindex{I., Innere Stadt@\textbf{I., Innere Stadt}, \emph{A.ADM3}|pwk}1/1 Wien 8, 16. II. 16, 7\nobreak{}«.  }\toendnotes[C]{\smallbreak}\pstart{}{\pb}Wien XVIII. Sternwartestrasse 71\oindex{Sternwartestrasse 71@\textbf{Sternwartestraße 71}, \emph{Wohngebäude (K.WHS)}|pw}.\pend{}\pstart{}Dr. Arthur Schnitzler\pend{}{\bigskip}\pstart{}Herrn\pend{}\pstart{}Dr. Stefan Zweig\pend{}\pstart{}Wien VIII.\oindex{VIII., Josefstadt@\textbf{VIII., Josefstadt}, \emph{A.ADM3}|pw}\pend{}\pstart{}Kochgasse 8\oindex{Kochgasse 8@\textbf{Kochgasse 8}, \emph{Wohngebäude (K.WHS)}|pw}.\pend{}{\bigskip}\vspace{1em}
\pstart
           \raggedleft{}{\pb}16. 2. 1916.\pend
           
\pstart{}Lieber Doktor Zweig.\pend\vspace{0.5em}
\pstart
           Möchten sie so freundlich sein, Herrn Josef
                  Popper\pwindex{Popper-Lynkeus, Josef 21.02.1838 – 22.12.1921@\textsc{Popper-Lynkeus, Josef} (21.02.1838 – 22.12.1921), \emph{Schriftsteller/Schriftstellerin}|pw}, Wien XIII. Wolterg. 2a\oindex{Woltergasse 2a@\textbf{Woltergasse 2a}, \emph{Wohngebäude (K.WHS)}|pw} die
               Adresse des Herrn Romain Rolland\pwindex{Rolland, Romain 29.01.1866 – 30.12.1944@\textsc{Rolland, Romain} (29.01.1866 – 30.12.1944), \emph{Schriftsteller/Schriftstellerin}|pw} mitzuteilen?
               Ihr letzter \label{K_L03770-1v}\edtext{Brief}{\lemma{\textnormal{\emph{Brief}}}\Cendnote{\textnormal{Siehe Stefan Zweig an Arthur Schnitzler, 25. 11. 1915.}}}\label{K_L03770-1}, in dem Sie sie mir mitteilten, lässt sich nicht auffinden und mir ist
               sie entfallen.\pend
           
\pstart
           Herzlichst grüssend und dankend{\\[\baselineskip]}Ihr{\\[\baselineskip]}\spacefill\mbox{{[}hs.:{]} Arth Schnitzler}\pend
           \leftskip=0em{}\selectlanguage{ngerman}\endnumbering\briefempfaengerindex{Zweig, Stefan@\textsc{Zweig, Stefan}!zzzSchnitzler, Arthur@\emph{von Arthur Schnitzler}!1916-02-161@{16. 2. 1916}|)be}\mylabel{L03770h}
\begin{anhang}
\end{anhang}\normalsize

\doendnotes{C}
\bigskip
\vfill

\clearpage

\footnotesize

\lohead{\textsc{register}}

% Definiere theindex-Environment komplett neu ohne reledmac
\makeatletter
\renewenvironment{theindex}{%
  \section*{\indexname}%
  \setlength{\parindent}{0pt}%
  \setlength{\parskip}{0pt plus 0.3pt}%
  \let\item\@idxitem
}{%
  \clearpage
}
\makeatother

\IfFileExists{\jobname-pw.ind}{\input{\jobname-pw.ind}}{}

\end{document}

      