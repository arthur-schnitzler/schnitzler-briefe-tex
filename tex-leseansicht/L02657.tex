%% latex-leseansicht-vorspann.tex
%% Vorspann für die Leseansicht.
%% Lädt die gemeinsame Datei latex-vorspann.tex mit nicht gesetztem Schalter.

\newif\ifkorrekturansicht
\korrekturansichtfalse

\input{../tex-inputs/latex-vorspann}

\begin{center}
            \textcolor{red}{ENTWURF, NICHT FERTIG KORRIGIERT}
                      \end{center}
            
               \section[Paul Goldmann an Arthur Schnitzler, {[}6. 2. 1903?{]}]{ Paul Goldmann an Arthur Schnitzler, {[}6. 2. 1903?{]}}\nopagebreak\mylabel{v}\rehead{ }\begin{ledgroupsized}[t]{13cm}\normalsize\beginnumbering\briefempfaengerindex{Schnitzler, Arthur@\textsc{Schnitzler, Arthur}!zzzGoldmann, Paul@\emph{von Paul Goldmann}!1903-02-061@{{[}6. 2. 1903?{]}}|(be} \toendnotes[C]{\smallbreak\pagebreak[2]} \Standort{DLA, A:Schnitzler, HS.NZ85.1.3173.}
\physDesc{Telegramm
\newline{}maschinell\newline{}Versand: »\noindent{}\textcolor{gray}{\textbf{Aufgenommen durch}}{ }\textcolor{gray}{\textbf{\textit{J. Khom\pwindex{Khom, Jakob @\textsc{Khom, Jakob}, \emph{Telegrafenbeamter}|pw}}}}« \newline{}Ordnung: beschnitten }\toendnotes[C]{\smallbreak}\pstart
           {\pb}de berlin\oindex{Berlin@\textbf{Berlin}|pw}
                  25111 17 69-35,-m=\pend
           \pstart
           tageblatt\pwindex{Berliner Tageblatt1872 – 1939@\emph{Berliner Tageblatt}|pw} bringt \label{K_L02657-1v}\edtext{ankuendigung\pwindex{?? Werk@Nicht ermittelte Verfasserinnen und Verfasser!?? Ankuendigung, dass Schleier der Beatrice diesen Monat Premiere habe]6.2.1903?@\emph{[?? Ankündigung, dass Schleier der Beatrice diesen Monat Premiere habe]} {[}6.2.1903?{]}|pwv}}{\lemma{\textnormal{\emph{ankuendigung}}}\Cendnote{\textnormal{\emph{XXXX}\pwindex{?? Werk@Nicht ermittelte Verfasserinnen und Verfasser!?? Ankuendigung, dass Schleier der Beatrice diesen Monat Premiere habe]6.2.1903?@\emph{[?? Ankündigung, dass Schleier der Beatrice diesen Monat Premiere habe]} {[}6.2.1903?{]}|pwk}. In: \emph{Berliner Tageblatt}\pwindex{Berliner Tageblatt1872 – 1939@\emph{Berliner Tageblatt}|pwk}, YYYY}}}\label{K_L02657-1h} dass beatrice\pwindex{Schnitzler, Arthur 15.05.1862 – 21.10.1931@\textsc{Schnitzler, Arthur} (15.05.1862 – 21.10.1931), \emph{Schriftsteller, Mediziner}!Schleier der Beatrice. Schauspiel in fuenf Akten1900-12-01 – 1900-12-01@\strich\emph{Der Schleier der Beatrice. Schauspiel in fünf Akten} {[}1900-12-01 – 1900-12-01{]}|pw} noch diesen monat aufgefuehrt wird.\pend
           \pstart gruss = \spacefill\mbox{goldmann .+}\pend{}          \endnumbering\briefempfaengerindex{Schnitzler, Arthur@\textsc{Schnitzler, Arthur}!zzzGoldmann, Paul@\emph{von Paul Goldmann}!1903-02-061@{{[}6. 2. 1903?{]}}|)be}\mylabel{h}\end{ledgroupsized}  \newcommand{\dateiname}{L02657}\newcommand{\titel}{Paul Goldmann an Arthur Schnitzler, [6. 2. 1903?]}\newcommand{\editorInnen}{Martin Anton Müller und Laura Untner}
            \footnotesize
\begin{ledgroupsized}[t]{11.5cm}
\doendnotes{C}
\end{ledgroupsized}
         %% latex-leseansicht-abspann.tex
%% Abspann für die Leseansicht.
%% Der Schalter \ifkorrekturansicht ist bereits durch den Vorspann gesetzt.

%% latex-abspann.tex
%% Gemeinsamer Abspann für Korrekturansicht und Leseansicht.
%% Setzt den Schalter \ifkorrekturansicht voraus (gesetzt in den
%% einbindenden Dateien latex-korrekturansicht-abspann.tex bzw.
%% latex-leseansicht-abspann.tex).
%% ---------------------------------------------------------------

\normalsize

% Das esempio-Environment wird nur in der Leseansicht benötigt
\ifkorrekturansicht\else
\newenvironment{esempio}[3]%
{
    \vspace{1.5ex}
    \rlap{\underline{#1}}
    \par
    \setlength{\parindent}{0cm}
    \nopagebreak
    \leftskip=#2cm
    \rightskip=#3cm
}
{
    \par
}
\fi

\doendnotes{C}
\bigskip
\vfill

\clearpage

\footnotesize

\ifkorrekturansicht
  \lohead{\textsc{register}}
\fi

% theindex-Environment neu definieren ohne reledmac
\makeatletter
\renewenvironment{theindex}{%
  \ifkorrekturansicht
    \section*{\indexname}%
  \else
    \subsubsection*{Index der erwähnten Entitäten}%
  \fi
  \setlength{\parindent}{0pt}%
  \setlength{\parskip}{0pt plus 0.3pt}%
  \let\item\@idxitem
}{%
  \ifkorrekturansicht\clearpage\fi
}
\makeatother

\IfFileExists{\jobname-pw.ind}{\input{\jobname-pw.ind}}{}

% Quellenangabe nur in der Leseansicht
\ifkorrekturansicht\else
% Fallback-Definitionen, falls die .tex-Datei \titel etc. nicht gesetzt hat
\providecommand{\titel}{}
\providecommand{\editorInnen}{}
\providecommand{\dateiname}{\jobname}

\vspace{3cm}

\vfill

\footnotesize
\textsc{Quelle}: \titel. Herausgegeben von {\editorInnen}. In: \emph{Arthur Schnitzler: Briefwechsel mit Autorinnen und Autoren}.
 Digitale Edition, https://schnitzler-briefe.acdh.oeaw.ac.at/{\dateiname}.html (Stand \today)
\fi

\end{document}


      