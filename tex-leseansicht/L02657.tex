%% latex-leseansicht-vorspann.tex
%% Vorspann für die Leseansicht.
%% Lädt die gemeinsame Datei latex-vorspann.tex mit nicht gesetztem Schalter.

\newif\ifkorrekturansicht
\korrekturansichtfalse

\input{../tex-inputs/latex-vorspann}


         
         \renewcommand{\erwaehntePersonen}{Personen: Albert Bassermann, Paul Goldmann, Jakob Khom, Irene Triesch}
         \renewcommand{\erwaehnteInstitutionen}{Institutionen: Carl-Theater, Deutsches Theater Berlin}
         \renewcommand{\erwaehnteOrte}{Orte: Berlin, Wien}
         \renewcommand{\erwaehnteWerke}{Werke: Berliner Tageblatt, Der Schleier der Beatrice. Schauspiel in fünf Akten, [Im Deutschen Theater soll »Der Schleier der Beatrice«]}
               \section[Paul Goldmann an Arthur Schnitzler, {[}6.?{]} 2. 1903]{ Paul Goldmann an Arthur Schnitzler, {[}6.?{]} 2. 1903}\nopagebreak\mylabel{v}\rehead{ }\begin{ledgroupsized}[t]{13cm}\normalsize\beginnumbering\briefempfaengerindex{Schnitzler, Arthur@\textsc{Schnitzler, Arthur}!zzzGoldmann, Paul@\emph{von Paul Goldmann}!1903-02-061@{{[}6.?{]} 2. 1903}|(be} \toendnotes[C]{\smallbreak\pagebreak[2]} \Standort{DLA, A:Schnitzler, HS.NZ85.1.3173.}
\physDesc{Telegramm, 118 Zeichen
\newline{}maschinell
\newline{}Versand: »\noindent{}\textcolor{gray}{\textbf{Aufgenommen durch}}{ }\textcolor{gray}{\textbf{\textit{J. Khom\pwindex{Khom, Jakob @\textsc{Khom, Jakob}, \emph{Telegrafenbeamter}|pw}}}}« 
\newline{}Ordnung: beschnitten }\toendnotes[C]{\smallbreak}\pstart
           \centering{}{\pb}de berlin\oindex{Berlin@\textbf{Berlin}|pw}
                  25111 17 6{ }9-35,–m=\pend
           \pstart
           tageblatt\pwindex{?? Werk@Nicht ermittelte Verfasserinnen und Verfasser!Berliner Tageblatt1872 – 1939@\emph{Berliner Tageblatt} {[}1872 – 1939{]}|pw} bringt \label{K_L02657-1v}\edtext{ankuendigung\pwindex{?? Werk@Nicht ermittelte Verfasserinnen und Verfasser!Im Deutschen Theater soll »Der Schleier der Beatrice«]1903-02-05@\emph{[Im Deutschen Theater soll »Der Schleier der Beatrice«]} {[}1903-02-05{]}|pwv}}{\lemma{\textnormal{\emph{ankuendigung}}}\Cendnote{\textnormal{»Im \so{Deutschen Theater}\orgindex{Deutsches Theater Berlin@Deutsches Theater Berlin|pw} soll ›\so{Der Schleier der Beatrice}\pwindex{Schnitzler, Arthur 15.05.1862 – 21.10.1931@\textsc{Schnitzler, Arthur} (15.05.1862 – 21.10.1931), \emph{Schriftsteller, Mediziner}!Schleier der Beatrice. Schauspiel in fuenf Akten1900-12-01@\strich\emph{Der Schleier der Beatrice. Schauspiel in fünf Akten} {[}1900-12-01{]}|pw}‹\pwindex{?? Werk@Nicht ermittelte Verfasserinnen und Verfasser!Im Deutschen Theater soll »Der Schleier der Beatrice«]1903-02-05@\emph{[Im Deutschen Theater soll »Der Schleier der Beatrice«]} {[}1903-02-05{]}|pwv} von \so{Arthur Schnitzler}\pwindex{Schnitzler, Arthur 15.05.1862 – 21.10.1931@\textsc{Schnitzler, Arthur} (15.05.1862 – 21.10.1931), \emph{Schriftsteller, Mediziner}|pw} noch in diesem Monat zur Aufführung kommen. Die Hauptrollen spielen Irene Triesch\pwindex{Triesch, Irene 13.04.1877 – 24.11.1964@\textsc{Triesch, Irene} (13.04.1877 – 24.11.1964), \emph{Schauspielerin}|pw} und Bassermann\pwindex{Bassermann, Albert 07.09.1867 – 15.05.1952@\textsc{Bassermann, Albert} (07.09.1867 – 15.05.1952), \emph{Schauspieler}|pw}. Im \so{Mai} gastiert
                     das Deutsche Theater\orgindex{Deutsches Theater Berlin@Deutsches Theater Berlin|pw} im \so{Wien}\oindex{Wien@\textbf{Wien}|pw}\so{er}{ }Karl-Theater\orgindex{Carl-Theater@Carl-Theater|pw}.« In: \emph{Berliner Tageblatt}\pwindex{?? Werk@Nicht ermittelte Verfasserinnen und Verfasser!Berliner Tageblatt1872 – 1939@\emph{Berliner Tageblatt} {[}1872 – 1939{]}|pwk}, Jg. 32, Nr. 65,
                        5. 2. 1904, Abendblatt, S. 3.}}}\label{K_L02657-1h} dass beatrice\pwindex{Schnitzler, Arthur 15.05.1862 – 21.10.1931@\textsc{Schnitzler, Arthur} (15.05.1862 – 21.10.1931), \emph{Schriftsteller, Mediziner}!Schleier der Beatrice. Schauspiel in fuenf Akten1900-12-01@\strich\emph{Der Schleier der Beatrice. Schauspiel in fünf Akten} {[}1900-12-01{]}|pw} noch diesen monat aufgefuehrt wird.\pend
           \pstart gruss = \spacefill\mbox{goldmann .+}\pend{}
         
         \endnumbering\mylabel{h}\end{ledgroupsized}  \newcommand{\dateiname}{L02657}\newcommand{\titel}{Paul Goldmann an Arthur Schnitzler, [6.?] 2. 1903}\newcommand{\editorInnen}{Martin Anton Müller und Laura Untner}%% latex-leseansicht-abspann.tex
%% Abspann für die Leseansicht.
%% Der Schalter \ifkorrekturansicht ist bereits durch den Vorspann gesetzt.

%% latex-abspann.tex
%% Gemeinsamer Abspann für Korrekturansicht und Leseansicht.
%% Setzt den Schalter \ifkorrekturansicht voraus (gesetzt in den
%% einbindenden Dateien latex-korrekturansicht-abspann.tex bzw.
%% latex-leseansicht-abspann.tex).
%% ---------------------------------------------------------------

\normalsize

% Das esempio-Environment wird nur in der Leseansicht benötigt
\ifkorrekturansicht\else
\newenvironment{esempio}[3]%
{
    \vspace{1.5ex}
    \rlap{\underline{#1}}
    \par
    \setlength{\parindent}{0cm}
    \nopagebreak
    \leftskip=#2cm
    \rightskip=#3cm
}
{
    \par
}
\fi

\doendnotes{C}
\bigskip
\vfill

\clearpage

\footnotesize

\ifkorrekturansicht
  \lohead{\textsc{register}}
\fi

% theindex-Environment neu definieren ohne reledmac
\makeatletter
\renewenvironment{theindex}{%
  \ifkorrekturansicht
    \section*{\indexname}%
  \else
    \subsubsection*{Index der erwähnten Entitäten}%
  \fi
  \setlength{\parindent}{0pt}%
  \setlength{\parskip}{0pt plus 0.3pt}%
  \let\item\@idxitem
}{%
  \ifkorrekturansicht\clearpage\fi
}
\makeatother

\IfFileExists{\jobname-pw.ind}{\input{\jobname-pw.ind}}{}

% Quellenangabe nur in der Leseansicht
\ifkorrekturansicht\else
% Fallback-Definitionen, falls die .tex-Datei \titel etc. nicht gesetzt hat
\providecommand{\titel}{}
\providecommand{\editorInnen}{}
\providecommand{\dateiname}{\jobname}

\vspace{3cm}

\vfill

\footnotesize
\textsc{Quelle}: \titel. Herausgegeben von {\editorInnen}. In: \emph{Arthur Schnitzler: Briefwechsel mit Autorinnen und Autoren}.
 Digitale Edition, https://schnitzler-briefe.acdh.oeaw.ac.at/{\dateiname}.html (Stand \today)
\fi

\end{document}


      