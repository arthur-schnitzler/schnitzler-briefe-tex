%% latex-korrekturansicht-vorspann.tex
%% Vorspann für die Korrekturansicht.
%% Lädt die gemeinsame Datei latex-vorspann.tex mit gesetztem Schalter.

\newif\ifkorrekturansicht
\korrekturansichttrue

\input{../tex-inputs/latex-vorspann}


\section[Paul Goldmann an Arthur Schnitzler, {[}6.?{]} 2. 1903]{L02657 Paul Goldmann an Arthur Schnitzler, {[}6.?{]} 2. 1903}
\nopagebreak\mylabel{L02657v}
\rehead{ }\normalsize\beginnumbering\briefempfaengerindex{Schnitzler, Arthur@\textsc{Schnitzler, Arthur}!zzzGoldmann, Paul@\emph{von Paul Goldmann}!1903-02-061@{{[}6.?{]} 2. 1903}|(be}
\toendnotes[C]{\smallbreak\pagebreak[2]}\Standort{DLA, A:Schnitzler, HS.NZ85.1.3173.}
\physDesc{Telegramm, 118 Zeichen
\newline{}maschinell
\newline{}Versand: »\noindent{}\textcolor{gray}{\textbf{Aufgenommen durch}}{ }\textcolor{gray}{\textbf{\textit{J. Khom\pwindex{Khom, Jakob @\textsc{Khom, Jakob}, \emph{Telegrafenbeamter/Telegrafenbeamtin}|pw}}}}« 
\newline{}Ordnung: beschnitten }\toendnotes[C]{\smallbreak}
\pstart
           \centering{}{\pb}de berlin\oindex{Berlin@\textbf{Berlin}, \emph{P.PPLC}|pw}
                  25111 17 6{ }9-35,–m=\pend
           \vspace{0.5em}
\pstart
           tageblatt\pwindex{Berliner Tageblatt@\emph{Berliner Tageblatt}|pw} bringt \label{K_L02657-1v}\edtext{ankuendigung\pwindex{Im Deutschen Theater soll »Der Schleier der Beatrice«]@\emph{[Im Deutschen Theater soll »Der Schleier der Beatrice«]}|pwv}}{\lemma{\textnormal{\emph{ankuendigung}}}\Cendnote{\textnormal{»Im \so{Deutschen Theater}\orgindex{Deutsches Theater Berlin@Deutsches Theater Berlin|pw} soll ›\so{Der Schleier der Beatrice}\pwindex{Schleier der Beatrice. Schauspiel in fuenf Akten@\emph{Der Schleier der Beatrice. Schauspiel in fünf Akten}|pw}‹\pwindex{Im Deutschen Theater soll »Der Schleier der Beatrice«]@\emph{[Im Deutschen Theater soll »Der Schleier der Beatrice«]}|pwv} von \so{Arthur Schnitzler} noch in diesem Monat zur Aufführung kommen. Die Hauptrollen spielen Irene Triesch\pwindex{Triesch, Irene 13.04.1877 – 24.11.1964@\textsc{Triesch, Irene} (13.04.1877 – 24.11.1964), \emph{Schauspieler/Schauspielerin}|pw} und Bassermann\pwindex{Bassermann, Albert 07.09.1867 – 15.05.1952@\textsc{Bassermann, Albert} (07.09.1867 – 15.05.1952), \emph{Schauspieler/Schauspielerin}|pw}. Im \so{Mai} gastiert
                     das Deutsche Theater\orgindex{Deutsches Theater Berlin@Deutsches Theater Berlin|pw} im \so{Wien}\oindex{Wien@\textbf{Wien}, \emph{A.ADM2}|pw}\so{er}{ }Karl-Theater\orgindex{Carl-Theater@Carl-Theater|pw}.« In: \emph{Berliner Tageblatt}\pwindex{Berliner Tageblatt@\emph{Berliner Tageblatt}|pwk}, Jg. 32, Nr. 65,
                        5. 2. 1904, Abendblatt, S. 3.}}}\label{K_L02657-1} dass beatrice\pwindex{Schleier der Beatrice. Schauspiel in fuenf Akten@\emph{Der Schleier der Beatrice. Schauspiel in fünf Akten}|pw} noch diesen monat aufgefuehrt wird.\pend
           \pstart gruss = \spacefill\mbox{goldmann .+}\pend{}\selectlanguage{ngerman}\endnumbering\briefempfaengerindex{Schnitzler, Arthur@\textsc{Schnitzler, Arthur}!zzzGoldmann, Paul@\emph{von Paul Goldmann}!1903-02-061@{{[}6.?{]} 2. 1903}|)be}\mylabel{L02657h}  \normalsize

\doendnotes{C}
\bigskip
\vfill

\clearpage

\footnotesize

\lohead{\textsc{register}}

% Definiere theindex-Environment komplett neu ohne reledmac
\makeatletter
\renewenvironment{theindex}{%
  \section*{\indexname}%
  \setlength{\parindent}{0pt}%
  \setlength{\parskip}{0pt plus 0.3pt}%
  \let\item\@idxitem
}{%
  \clearpage
}
\makeatother

\IfFileExists{\jobname-pw.ind}{\input{\jobname-pw.ind}}{}

\end{document}

      