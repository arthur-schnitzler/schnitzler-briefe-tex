%% latex-leseansicht-vorspann.tex
%% Vorspann für die Leseansicht.
%% Lädt die gemeinsame Datei latex-vorspann.tex mit nicht gesetztem Schalter.

\newif\ifkorrekturansicht
\korrekturansichtfalse

\input{../tex-inputs/latex-vorspann}


         
         \renewcommand{\erwaehntePersonen}{Personen: Felix Salten, Johann Schnitzler}
         \renewcommand{\erwaehnteOrte}{Orte: Wien}
         \renewcommand{\erwaehnteWerke}{Werke: Die Gesellschaft. Monatsschrift für Litteratur, Kunst und Sozialpolitik, Morgenandacht, Tagebuch}
               \section[Arthur Schnitzler an Felix Salten, {[}23. 1. 1894?{]}]{ Arthur Schnitzler an Felix Salten, {[}23. 1. 1894?{]}}\nopagebreak\mylabel{v}\rehead{ }\begin{ledgroupsized}[t]{13cm}\normalsize\beginnumbering\briefempfaengerindex{Salten, Felix@\textsc{Salten, Felix}!zzzSchnitzler, Arthur@\emph{von Arthur Schnitzler}!1894-01-231@{{[}23. 1. 1894?{]}}|(be} \toendnotes[C]{\smallbreak\pagebreak[2]} \Standort{Wienbibliothek im Rathaus, ZPH 1681, 2.1.516.}
\physDesc{Brief, 1 Blatt, 2 Seiten, 508 Zeichen (Briefpapier mit Trauerrand)
\newline{}Handschrift: Bleistift, deutsche Kurrent
\newline{}Ordnung: mit Bleistift von unbekannter Hand nummeriert: »19« }\toendnotes[C]{\smallbreak}\pstart\center{}{\pb}Lieber!\pend\pstart
           Was ſind das für \label{K_L03037-1v}\edtext{Lächerlichkeiten}{\lemma{\textnormal{\emph{Lächerlichkeiten}}}\Cendnote{\textnormal{Das Korrespondenzstück ist undatiert und
                  nur unzuverlässig datierbar. Die folgende Annäherung erlaubt die Einordnung:
                  Durch die Verwendung von Briefpapier mit Trauerrand lässt es sich in das Jahr nach
                  dem Tod des Vaters\pwindex{Schnitzler, Johann 10.04.1835 – 02.05.1893@\textsc{Schnitzler, Johann} (10.04.1835 – 02.05.1893), \emph{Laryngologe}|pwkv} am
                     2. 5. 1893
                  verorten. Am 25. 10. 1893 hatte Schnitzler\pwindex{Schnitzler, Arthur 15.05.1862 – 21.10.1931@\textsc{Schnitzler, Arthur} (15.05.1862 – 21.10.1931), \emph{Schriftsteller, Mediziner}|pwk}
                  das in Folge zitierte – bereits 1891 erschienene – Gedicht\pwindex{Schnitzler, Arthur 15.05.1862 – 21.10.1931@\textsc{Schnitzler, Arthur} (15.05.1862 – 21.10.1931), \emph{Schriftsteller, Mediziner}!Morgenandacht1. 2. 1891@\strich\emph{Morgenandacht} {[}1. 2. 1891{]}|pwkv} in Gegenwart Saltens\pwindex{Salten, Felix 06.09.1869 – 08.10.1945@\textsc{Salten, Felix} (06.09.1869 – 08.10.1945), \emph{Schriftsteller, Journalist, Chefredakteur}|pwk} vorgetragen. Das kann als Indiz dafür genommen
                  werden, dass das Schreiben danach abgefasst wurde. Für den so ermittelten
                  Zeitraum gibt es im \emph{Tagebuch}\pwindex{\textcolor{red}{\textsuperscript{XXXX1 indx}}!Tagebuch1981 – 2000@\strich\emph{Tagebuch} {[}Hrsg., 1981 – 2000{]}|pwk} keine Aussage,
                  die sich unmittelbar mit der hier geäußerten Verärgerung in Beziehung setzen
                  lässt. Unter den überlieferten Briefen Saltens\pwindex{Salten, Felix 06.09.1869 – 08.10.1945@\textsc{Salten, Felix} (06.09.1869 – 08.10.1945), \emph{Schriftsteller, Journalist, Chefredakteur}|pwk} hingegen könnte jener vom [24. 1. 1894] diesem gefolgt sein. Zumindest fügen sich
                  die Angaben zu einem möglichen Treffen am Folgetag gut zusammen und Schnitzler\pwindex{Schnitzler, Arthur 15.05.1862 – 21.10.1931@\textsc{Schnitzler, Arthur} (15.05.1862 – 21.10.1931), \emph{Schriftsteller, Mediziner}|pwk} könnte auf die Schulden bei ihm
                  angespielt haben.}}}\label{K_L03037-1h}? Bin ich ein grüner Oberſchwan? Bin ich ein verlobter
               Fähnrich, dem der Tiefſinn die Leuchter hinters Fenſter geſetzt hat? Oder hab ich gar
               die Gewohnheit, Sternſchnuppen im Cylinder aufzufangen? Beſſer iſt
               es ſchon, wenn Sie mich morgen zwiſchen {\pb}½ 6 und 6 aufſuchen. –\pend
           \pstart
           Es wäre möglich, daſs ich Sie morgen im Laufe des
                  Nachmittags aufſuche – kanns aber nicht verſprechen.\pend
           \pstart
           Herzliche Grüße. Was Sie mir ſchrieben, »\label{K_L03037-2v}\edtext{das iſt von einem böſen Wahn der
                  trügevolle Schimmer\pwindex{Schnitzler, Arthur 15.05.1862 – 21.10.1931@\textsc{Schnitzler, Arthur} (15.05.1862 – 21.10.1931), \emph{Schriftsteller, Mediziner}!Morgenandacht1. 2. 1891@\strich\emph{Morgenandacht} {[}1. 2. 1891{]}|pwv}}{\lemma{\textnormal{\emph{das … Schimmer}}}\Cendnote{\textnormal{In Schnitzlers\pwindex{Schnitzler, Arthur 15.05.1862 – 21.10.1931@\textsc{Schnitzler, Arthur} (15.05.1862 – 21.10.1931), \emph{Schriftsteller, Mediziner}|pwk} Gedicht \emph{Morgenandacht}\pwindex{Schnitzler, Arthur 15.05.1862 – 21.10.1931@\textsc{Schnitzler, Arthur} (15.05.1862 – 21.10.1931), \emph{Schriftsteller, Mediziner}!Morgenandacht1. 2. 1891@\strich\emph{Morgenandacht} {[}1. 2. 1891{]}|pwk}
                  heißt es in der 8. Strophe: »Das war von einem holden Wahn{ / }Der trügevolle
                     Schimmer« (\emph{Die Gesellschaft}\pwindex{Gesellschaft. Monatsschrift fuer Litteratur, Kunst und Sozialpolitik1885 – 1902@\emph{Die Gesellschaft. Monatsschrift für Litteratur, Kunst und Sozialpolitik} {[}1885 – 1902{]}|pwk}, Jg. 7, Bd. 1, H. 2,
                        Februar 1891, S. 190).}}}\label{K_L03037-2h}.« \pend
           \pstart Ihr \spacefill\mbox{ArthSchn}\pend{}
         
         \endnumbering\mylabel{h}\end{ledgroupsized}  \newcommand{\dateiname}{L03037}\newcommand{\titel}{Arthur Schnitzler an Felix Salten, [23. 1. 1894?]}\newcommand{\editorInnen}{Martin Anton Müller und Laura Untner}%% latex-leseansicht-abspann.tex
%% Abspann für die Leseansicht.
%% Der Schalter \ifkorrekturansicht ist bereits durch den Vorspann gesetzt.

%% latex-abspann.tex
%% Gemeinsamer Abspann für Korrekturansicht und Leseansicht.
%% Setzt den Schalter \ifkorrekturansicht voraus (gesetzt in den
%% einbindenden Dateien latex-korrekturansicht-abspann.tex bzw.
%% latex-leseansicht-abspann.tex).
%% ---------------------------------------------------------------

\normalsize

% Das esempio-Environment wird nur in der Leseansicht benötigt
\ifkorrekturansicht\else
\newenvironment{esempio}[3]%
{
    \vspace{1.5ex}
    \rlap{\underline{#1}}
    \par
    \setlength{\parindent}{0cm}
    \nopagebreak
    \leftskip=#2cm
    \rightskip=#3cm
}
{
    \par
}
\fi

\doendnotes{C}
\bigskip
\vfill

\clearpage

\footnotesize

\ifkorrekturansicht
  \lohead{\textsc{register}}
\fi

% theindex-Environment neu definieren ohne reledmac
\makeatletter
\renewenvironment{theindex}{%
  \ifkorrekturansicht
    \section*{\indexname}%
  \else
    \subsubsection*{Index der erwähnten Entitäten}%
  \fi
  \setlength{\parindent}{0pt}%
  \setlength{\parskip}{0pt plus 0.3pt}%
  \let\item\@idxitem
}{%
  \ifkorrekturansicht\clearpage\fi
}
\makeatother

\IfFileExists{\jobname-pw.ind}{\input{\jobname-pw.ind}}{}

% Quellenangabe nur in der Leseansicht
\ifkorrekturansicht\else
% Fallback-Definitionen, falls die .tex-Datei \titel etc. nicht gesetzt hat
\providecommand{\titel}{}
\providecommand{\editorInnen}{}
\providecommand{\dateiname}{\jobname}

\vspace{3cm}

\vfill

\footnotesize
\textsc{Quelle}: \titel. Herausgegeben von {\editorInnen}. In: \emph{Arthur Schnitzler: Briefwechsel mit Autorinnen und Autoren}.
 Digitale Edition, https://schnitzler-briefe.acdh.oeaw.ac.at/{\dateiname}.html (Stand \today)
\fi

\end{document}


      