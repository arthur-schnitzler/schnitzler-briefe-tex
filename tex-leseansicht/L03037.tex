%% latex-leseansicht-vorspann.tex
%% Vorspann für die Leseansicht.
%% Lädt die gemeinsame Datei latex-vorspann.tex mit nicht gesetztem Schalter.

\newif\ifkorrekturansicht
\korrekturansichtfalse

\input{../tex-inputs/latex-vorspann}

\begin{center}
            \textcolor{red}{ENTWURF, NICHT FERTIG KORRIGIERT}
                      \end{center}
            
         
         \newcommand{\erwaehntePersonen}{Personen: Felix Salten, Johann Schnitzler}
         \newcommand{\erwaehnteWerke}{Werke: Die Gesellschaft. Monatsschrift für Litteratur, Kunst und Sozialpolitik, Morgenandacht}
               \section[Arthur Schnitzler an Felix Salten, {[}26. 10. 1893 – 2. 5. 1894?{]}]{ Arthur Schnitzler an Felix Salten, {[}26. 10. 1893 – 2. 5. 1894?{]}}\nopagebreak\mylabel{v}\rehead{ }\begin{ledgroupsized}[t]{13cm}\normalsize\beginnumbering \toendnotes[C]{\smallbreak\pagebreak[2]} \Standort{Wienbibliothek im Rathaus, ZPH 1681, 2.1.516.}
\physDesc{
\newline{}Handschrift: , deutsche Kurrent}\toendnotes[C]{\smallbreak}\pstart{}{\pb}Lieber!\pend\pstart
           Was ſind das für \label{K_L03037-2v}\edtext{Lächerlichkeiten}{\lemma{\textnormal{\emph{Lächerlichkeiten}}}\Cendnote{\textnormal{Das Korrespondenzstück ist undatiert. Durch die Verwendung
               von Briefpapier lässt es sich in das Trauerjahr nach dem Tod des Vaters\pwindex{Schnitzler, Johann 10.04.1835 – 02.05.1893@\textsc{Schnitzler, Johann} (10.04.1835 – 02.05.1893), \emph{Laryngologe}|pwkv} am 2. 5. 1893
            verorten. Am 25. 10. 1893 trug Schnitzler\pwindex{Schnitzler, Arthur 15.05.1862 – 21.10.1931@\textsc{Schnitzler, Arthur} (15.05.1862 – 21.10.1931), \emph{Schriftsteller, Mediziner}|pwk} das Gedicht in Gegenwart Salten\pwindex{Salten, Felix 06.09.1869 – 08.10.1945@\textsc{Salten, Felix} (06.09.1869 – 08.10.1945), \emph{Schriftsteller, Journalist}|pwk}s vor, was
            zumindest als Indiz genommen werden kann, dass das Schreiben danach abgefasst ist.}}}\label{K_L03037-2h}? Bin ich ein grüner Oberſchwan? Bin ich ein
               verlobter Fähnrich, dem der Tiefſinn die Leuchter hinters Fenſter geſetzt hat? Oder
               hab ich gar die Gewohnheit, Sternſchnuppen in Cylinder aufzufangen? Beſſer iſt es
               ſchon, wenn Sie mich morgen zwiſchen {\pb}½ 6 und
                  6 aufſuchen.– \pend
           \pstart
           Es wäre möglich, daſs ich Sie morgen im Laufe des Nachmittags aufſuche – kanns aber
               nicht verſprechen. \pend
           \pstart
           Herzliche Grüße. Was Sie mir ſchrieben, »\label{K_L03037-1v}\edtext{das iſt von
                  einem böſen Wahn der trügevolle Schimmer\pwindex{Schnitzler, Arthur 15.05.1862 – 21.10.1931@\textsc{Schnitzler, Arthur} (15.05.1862 – 21.10.1931), \emph{Schriftsteller, Mediziner}!Morgenandacht1. 2. 1891@\strich\emph{Morgenandacht} {[}1. 2. 1891{]}|pwv}}{\lemma{\textnormal{\emph{das … Schimmer}}}\Cendnote{\textnormal{In Schnitzler\pwindex{Schnitzler, Arthur 15.05.1862 – 21.10.1931@\textsc{Schnitzler, Arthur} (15.05.1862 – 21.10.1931), \emph{Schriftsteller, Mediziner}|pwk}s Gedicht
                     \emph{Morgenandacht}\pwindex{Schnitzler, Arthur 15.05.1862 – 21.10.1931@\textsc{Schnitzler, Arthur} (15.05.1862 – 21.10.1931), \emph{Schriftsteller, Mediziner}!Morgenandacht1. 2. 1891@\strich\emph{Morgenandacht} {[}1. 2. 1891{]}|pwk} heißt es in der 8.
                     Strophe: »Das war von einem holden Wahn / Der trügevolle
                     Schimmer«. (\emph{Die
                     Gesellschaft}\pwindex{Gesellschaft. Monatsschrift fuer Litteratur, Kunst und Sozialpolitik1885 – 1902@\emph{Die Gesellschaft. Monatsschrift für Litteratur, Kunst und Sozialpolitik} {[}1885 – 1902{]}|pwk}, Jg. 7, Bd. 1, H. 2, Februar
                     1891, S. 190.)}}}\label{K_L03037-1h}.« \pend
           \pstart Ihr \spacefill\mbox{ArthSchn}\pend{}
         
         \endnumbering\mylabel{h}\end{ledgroupsized}\begin{anhang}\end{anhang}\newcommand{\dateiname}{L03037}\newcommand{\titel}{Arthur Schnitzler an Felix Salten, [26. 10. 1893 – 2. 5. 1894?]}\newcommand{\editorInnen}{Martin Anton Müller und Laura Untner}%% latex-leseansicht-abspann.tex
%% Abspann für die Leseansicht.
%% Der Schalter \ifkorrekturansicht ist bereits durch den Vorspann gesetzt.

%% latex-abspann.tex
%% Gemeinsamer Abspann für Korrekturansicht und Leseansicht.
%% Setzt den Schalter \ifkorrekturansicht voraus (gesetzt in den
%% einbindenden Dateien latex-korrekturansicht-abspann.tex bzw.
%% latex-leseansicht-abspann.tex).
%% ---------------------------------------------------------------

\normalsize

% Das esempio-Environment wird nur in der Leseansicht benötigt
\ifkorrekturansicht\else
\newenvironment{esempio}[3]%
{
    \vspace{1.5ex}
    \rlap{\underline{#1}}
    \par
    \setlength{\parindent}{0cm}
    \nopagebreak
    \leftskip=#2cm
    \rightskip=#3cm
}
{
    \par
}
\fi

\doendnotes{C}
\bigskip
\vfill

\clearpage

\footnotesize

\ifkorrekturansicht
  \lohead{\textsc{register}}
\fi

% theindex-Environment neu definieren ohne reledmac
\makeatletter
\renewenvironment{theindex}{%
  \ifkorrekturansicht
    \section*{\indexname}%
  \else
    \subsubsection*{Index der erwähnten Entitäten}%
  \fi
  \setlength{\parindent}{0pt}%
  \setlength{\parskip}{0pt plus 0.3pt}%
  \let\item\@idxitem
}{%
  \ifkorrekturansicht\clearpage\fi
}
\makeatother

\IfFileExists{\jobname-pw.ind}{\input{\jobname-pw.ind}}{}

% Quellenangabe nur in der Leseansicht
\ifkorrekturansicht\else
% Fallback-Definitionen, falls die .tex-Datei \titel etc. nicht gesetzt hat
\providecommand{\titel}{}
\providecommand{\editorInnen}{}
\providecommand{\dateiname}{\jobname}

\vspace{3cm}

\vfill

\footnotesize
\textsc{Quelle}: \titel. Herausgegeben von {\editorInnen}. In: \emph{Arthur Schnitzler: Briefwechsel mit Autorinnen und Autoren}.
 Digitale Edition, https://schnitzler-briefe.acdh.oeaw.ac.at/{\dateiname}.html (Stand \today)
\fi

\end{document}


      