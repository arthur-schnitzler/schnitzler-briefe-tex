%% latex-leseansicht-vorspann.tex
%% Vorspann für die Leseansicht.
%% Lädt die gemeinsame Datei latex-vorspann.tex mit nicht gesetztem Schalter.

\newif\ifkorrekturansicht
\korrekturansichtfalse

\input{../tex-inputs/latex-vorspann}


\section[ Felix Salten an Arthur Schnitzler, [20. 12. 1904]]{L03402 Felix Salten an Arthur Schnitzler,  [20. 12. 1904]}
\nopagebreak\mylabel{L03402v}
\rehead{ }\normalsize\beginnumbering\briefempfaengerindex{Schnitzler, Arthur@\textsc{Schnitzler, Arthur}!zzzSalten, Felix@\emph{von Felix Salten}!1904-12-201@{{[}20. 12. 1904{]}}|(be}
\toendnotes[C]{\smallbreak\pagebreak[2]}
\correspDesc{Versand  durch Felix Salten am [20. 12. 1904] in Wien
\newline{}Erhalt  durch Arthur Schnitzler am [20. 12. 1904] in Wien}\toendnotes[C]{\smallbreak}
\Standort{CUL, Schnitzler, B 89, B 1.}
\physDesc{Brief, 1 Blatt, 1 Seite, 236 Zeichen
\newline{}Handschrift: schwarze Tinte, lateinische Kurrent
\newline{}Schnitzler: mit Bleistift datiert: »20/12 904« 
\newline{}Ordnung: mit Bleistift von unbekannter Hand nummeriert: »195« }\toendnotes[C]{\smallbreak}
\pstart
           \raggedleft{}{\pb}Dienstag\pend
           \vspace{0.5em}
\pstart
           Lieber, für den überraschenden und prächtigen \label{K_L03402-1v}\edtext{Donatello}{\lemma{\textnormal{\emph{Donatello}}}\Cendnote{\textnormal{Vgl. A. S.: \emph{Tagebuch}, 9. 12. 1904.
                     }}}\label{K_L03402-1}\pwindex{Donatello um 1368 Florenz – 13.\,12.\,1466 ebd.@\textsc{Donatello} (um 1368 Florenz – 13.\,12.\,1466 ebd.), \emph{Bildhauer}!?? [Gipsnachbildung einer Statue von Donatello]@\strich\emph{?? [Gipsnachbildung einer Statue von Donatello]}|pwv}\pwindex{Donatello um 1368 Florenz – 13.\,12.\,1466 ebd.@\textsc{Donatello} (um 1368 Florenz – 13.\,12.\,1466 ebd.), \emph{Bildhauer}|pw} bedanken wir\pwindex{Salten, Ottilie 7.\,3.\,1868 Prag – 22.\,6.\,1942 Zürich@\textsc{Salten, Ottilie} (7.\,3.\,1868 Prag – 22.\,6.\,1942 Zürich), \emph{Schauspielerin}|pwv} uns
               herzlich und erfreut.\pend
           
\pstart
           Wir sind auch beim \label{K_L03402-2v}\edtext{Mahler\pwindex{Mahler, Gustav 7.\,7.\,1860 Kaliště – 18.\,5.\,1911 Wien@\textsc{Mahler, Gustav} (7.\,7.\,1860 Kaliště – 18.\,5.\,1911 Wien), \emph{Theaterleiter, Komponist, Dirigent}|pw}-Conzert}{\lemma{\textnormal{\emph{Mahler-Conzert}}}\Cendnote{\textnormal{Am 22. 12. 1904 wurde die \emph{3. Sinfonie in
                     d-Moll}\pwindex{Mahler, Gustav 7.\,7.\,1860 Kaliště – 18.\,5.\,1911 Wien@\textsc{Mahler, Gustav} (7.\,7.\,1860 Kaliště – 18.\,5.\,1911 Wien), \emph{Theaterleiter, Komponist, Dirigent}!3. Sinfonie in d-Moll@\strich\emph{3. Sinfonie in d-Moll}|pwk} im Großen Musikvereinssaal\oindex{Wien@\textbf{Wien}!I., Innere Stadt@\textbf{I., Innere Stadt}!Musikverein@\textbf{Musikverein}, \emph{Konzertsaal}|pwk}
                  gegeben. Wie aus den folgenden Briefen hervorgeht, verpassten sie sich im Riedhof\oindex{Wien@\textbf{Wien}!VIII., Josefstadt@\textbf{VIII., Josefstadt}!Riedhof@\textbf{Riedhof}, \emph{Lokal}|pwk}.}}}\label{K_L03402-2}, und könnten dann ev. zusammen
               in den Riedhof\oindex{Wien@\textbf{Wien}!VIII., Josefstadt@\textbf{VIII., Josefstadt}!Riedhof@\textbf{Riedhof}, \emph{Lokal}|pw}, jedesfalls aber uns dort nachher
               treffen.\pend
           
\pstart
           Herzlichst {\\[\baselineskip]}Ihr {\\[\baselineskip]}\spacefill\mbox{Salten}\pend
           \leftskip=0em{}\selectlanguage{ngerman}\endnumbering\briefempfaengerindex{Schnitzler, Arthur@\textsc{Schnitzler, Arthur}!zzzSalten, Felix@\emph{von Felix Salten}!1904-12-201@{{[}20. 12. 1904{]}}|)be}\mylabel{L03402h}  \newcommand{\dateiname}{L03402}\newcommand{\titel}{Felix Salten an Arthur Schnitzler, [20. 12. 1904]}\newcommand{\editorInnen}{Martin Anton Müller und Laura Untner}%% latex-leseansicht-abspann.tex
%% Abspann für die Leseansicht.
%% Der Schalter \ifkorrekturansicht ist bereits durch den Vorspann gesetzt.

%% latex-abspann.tex
%% Gemeinsamer Abspann für Korrekturansicht und Leseansicht.
%% Setzt den Schalter \ifkorrekturansicht voraus (gesetzt in den
%% einbindenden Dateien latex-korrekturansicht-abspann.tex bzw.
%% latex-leseansicht-abspann.tex).
%% ---------------------------------------------------------------

\normalsize

% Das esempio-Environment wird nur in der Leseansicht benötigt
\ifkorrekturansicht\else
\newenvironment{esempio}[3]%
{
    \vspace{1.5ex}
    \rlap{\underline{#1}}
    \par
    \setlength{\parindent}{0cm}
    \nopagebreak
    \leftskip=#2cm
    \rightskip=#3cm
}
{
    \par
}
\fi

\doendnotes{C}
\bigskip
\vfill

\clearpage

\footnotesize

\ifkorrekturansicht
  \lohead{\textsc{register}}
\fi

% theindex-Environment neu definieren ohne reledmac
\makeatletter
\renewenvironment{theindex}{%
  \ifkorrekturansicht
    \section*{\indexname}%
  \else
    \subsubsection*{Index der erwähnten Entitäten}%
  \fi
  \setlength{\parindent}{0pt}%
  \setlength{\parskip}{0pt plus 0.3pt}%
  \let\item\@idxitem
}{%
  \ifkorrekturansicht\clearpage\fi
}
\makeatother

\IfFileExists{\jobname-pw.ind}{\input{\jobname-pw.ind}}{}

% Quellenangabe nur in der Leseansicht
\ifkorrekturansicht\else
% Fallback-Definitionen, falls die .tex-Datei \titel etc. nicht gesetzt hat
\providecommand{\titel}{}
\providecommand{\editorInnen}{}
\providecommand{\dateiname}{\jobname}

\vspace{3cm}

\vfill

\footnotesize
\textsc{Quelle}: \titel. Herausgegeben von {\editorInnen}. In: \emph{Arthur Schnitzler: Briefwechsel mit Autorinnen und Autoren}.
 Digitale Edition, https://schnitzler-briefe.acdh.oeaw.ac.at/{\dateiname}.html (Stand \today)
\fi

\end{document}


