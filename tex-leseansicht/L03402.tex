%% latex-leseansicht-vorspann.tex
%% Vorspann für die Leseansicht.
%% Lädt die gemeinsame Datei latex-vorspann.tex mit nicht gesetztem Schalter.

\newif\ifkorrekturansicht
\korrekturansichtfalse

\input{../tex-inputs/latex-vorspann}


         
         \renewcommand{\erwaehntePersonen}{Personen:  Donatello, Gustav Mahler, Ottilie Salten}
         \renewcommand{\erwaehnteOrte}{Orte: Musikverein, Riedhof, Wien}
         \renewcommand{\erwaehnteWerke}{Werke: ?? [Gipsnachbildung einer Statue von Donatello], Symphonie Nr. 3 D-Moll}
               \section[ Felix Salten an Arthur Schnitzler, {[}20. 12. 1904{]}]{ Felix Salten an Arthur Schnitzler, {[}20. 12. 1904{]}}\nopagebreak\mylabel{v}\rehead{ }\begin{ledgroupsized}[t]{13cm}\normalsize\beginnumbering \toendnotes[C]{\smallbreak\pagebreak[2]} \Standort{CUL, Schnitzler, B 89, B 1.}
\physDesc{Brief, 1 Blatt, 1 Seite, 236 Zeichen
\newline{}Handschrift: schwarze Tinte, lateinische Kurrent
\newline{}Schnitzler: mit Bleistift datiert: »20/12 904« 
\newline{}Ordnung: mit Bleistift von unbekannter Hand nummeriert: »195« }\toendnotes[C]{\smallbreak}\pstart
           \raggedleft{}{\pb}Dienstag\pend
           \pstart
           Lieber, für den überraschenden und prächtigen \label{K_L03402-1v}\edtext{Donatello}{\lemma{\textnormal{\emph{Donatello}}}\Cendnote{\textnormal{vgl. A. S.: \emph{Tagebuch}, 9. 12. 1904}}}\label{K_L03402-1h}\pwindex{Donatello um 1368 – 13.12.1466@\textsc{Donatello} (um 1368 – 13.12.1466), \emph{Bildhauer}!?? [Gipsnachbildung einer Statue von Donatello]1904@\strich\emph{?? [Gipsnachbildung einer Statue von Donatello]} {[}1904{]}|pwv}\pwindex{Donatello um 1368 – 13.12.1466@\textsc{Donatello} (um 1368 – 13.12.1466), \emph{Bildhauer}|pw} bedanken wir\pwindex{Salten, Ottilie 07.03.1868 – 22.06.1942@\textsc{Salten, Ottilie} (07.03.1868 – 22.06.1942), \emph{Schauspielerin}|pwv} uns
               herzlich und erfreut.\pend
           \pstart
           Wir sind auch beim \label{K_L03402-2v}\edtext{Mahler\pwindex{Mahler, Gustav 07.07.1860 – 18.05.1911@\textsc{Mahler, Gustav} (07.07.1860 – 18.05.1911), \emph{Theaterleiter, Komponist, Dirigent}|pw}-Conzert}{\lemma{\textnormal{\emph{Mahler-Conzert}}}\Cendnote{\textnormal{Am 22. 12. 1904 wurde die \emph{3. Sinfonie in
                     d-Moll}\pwindex{Mahler, Gustav 07.07.1860 – 18.05.1911@\textsc{Mahler, Gustav} (07.07.1860 – 18.05.1911), \emph{Theaterleiter, Komponist, Dirigent}!Symphonie Nr. 3 D-Moll1902@\strich\emph{Symphonie Nr. 3 D-Moll} {[}1902{]}|pwk} im Großen Musikvereinssaal\oindex{Musikverein@\textbf{Musikverein}|pwk}
                  gegeben. Wie aus den folgenden Briefen hervorgeht, verpassten sie sich im Riedhof\oindex{Riedhof@\textbf{Riedhof}|pwk}.}}}\label{K_L03402-2h}, und könnten dann ev. zusammen
               in den Riedhof\oindex{Riedhof@\textbf{Riedhof}|pw}, jedesfalls aber uns dort nachher
               treffen.\pend
           \pstart
           Herzlichst {\\[\baselineskip]}Ihr {\\[\baselineskip]}\spacefill\mbox{Salten}\pend
           \leftskip=0em{}
         
         \endnumbering\mylabel{h}\end{ledgroupsized}  \newcommand{\dateiname}{L03402}\newcommand{\titel}{Felix Salten an Arthur Schnitzler, [20. 12. 1904]}\newcommand{\editorInnen}{Martin Anton Müller und Laura Untner}%% latex-leseansicht-abspann.tex
%% Abspann für die Leseansicht.
%% Der Schalter \ifkorrekturansicht ist bereits durch den Vorspann gesetzt.

%% latex-abspann.tex
%% Gemeinsamer Abspann für Korrekturansicht und Leseansicht.
%% Setzt den Schalter \ifkorrekturansicht voraus (gesetzt in den
%% einbindenden Dateien latex-korrekturansicht-abspann.tex bzw.
%% latex-leseansicht-abspann.tex).
%% ---------------------------------------------------------------

\normalsize

% Das esempio-Environment wird nur in der Leseansicht benötigt
\ifkorrekturansicht\else
\newenvironment{esempio}[3]%
{
    \vspace{1.5ex}
    \rlap{\underline{#1}}
    \par
    \setlength{\parindent}{0cm}
    \nopagebreak
    \leftskip=#2cm
    \rightskip=#3cm
}
{
    \par
}
\fi

\doendnotes{C}
\bigskip
\vfill

\clearpage

\footnotesize

\ifkorrekturansicht
  \lohead{\textsc{register}}
\fi

% theindex-Environment neu definieren ohne reledmac
\makeatletter
\renewenvironment{theindex}{%
  \ifkorrekturansicht
    \section*{\indexname}%
  \else
    \subsubsection*{Index der erwähnten Entitäten}%
  \fi
  \setlength{\parindent}{0pt}%
  \setlength{\parskip}{0pt plus 0.3pt}%
  \let\item\@idxitem
}{%
  \ifkorrekturansicht\clearpage\fi
}
\makeatother

\IfFileExists{\jobname-pw.ind}{\input{\jobname-pw.ind}}{}

% Quellenangabe nur in der Leseansicht
\ifkorrekturansicht\else
% Fallback-Definitionen, falls die .tex-Datei \titel etc. nicht gesetzt hat
\providecommand{\titel}{}
\providecommand{\editorInnen}{}
\providecommand{\dateiname}{\jobname}

\vspace{3cm}

\vfill

\footnotesize
\textsc{Quelle}: \titel. Herausgegeben von {\editorInnen}. In: \emph{Arthur Schnitzler: Briefwechsel mit Autorinnen und Autoren}.
 Digitale Edition, https://schnitzler-briefe.acdh.oeaw.ac.at/{\dateiname}.html (Stand \today)
\fi

\end{document}


      