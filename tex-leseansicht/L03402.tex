%% latex-korrekturansicht-vorspann.tex
%% Vorspann für die Korrekturansicht.
%% Lädt die gemeinsame Datei latex-vorspann.tex mit gesetztem Schalter.

\newif\ifkorrekturansicht
\korrekturansichttrue

\input{../tex-inputs/latex-vorspann}


\section[ Felix Salten an Arthur Schnitzler, {[}20. 12. 1904{]}]{L03402 Felix Salten an Arthur Schnitzler, {[}20. 12. 1904{]}}
\nopagebreak\mylabel{L03402v}
\rehead{ }\normalsize\beginnumbering\briefempfaengerindex{Schnitzler, Arthur@\textsc{Schnitzler, Arthur}!zzzSalten, Felix@\emph{von Felix Salten}!1904-12-201@{{[}20. 12. 1904{]}}|(be}
\toendnotes[C]{\smallbreak\pagebreak[2]}\Standort{CUL, Schnitzler, B 89, B 1.}
\physDesc{Brief, 1 Blatt, 1 Seite, 236 Zeichen
\newline{}Handschrift: schwarze Tinte, lateinische Kurrent
\newline{}Schnitzler: mit Bleistift datiert: »20/12 904« 
\newline{}Ordnung: mit Bleistift von unbekannter Hand nummeriert: »195« }\toendnotes[C]{\smallbreak}
\pstart
           \raggedleft{}{\pb}Dienstag\pend
           \vspace{0.5em}
\pstart
           Lieber, für den überraschenden und prächtigen \label{K_L03402-1v}\edtext{Donatello}{\lemma{\textnormal{\emph{Donatello}}}\Cendnote{\textnormal{Vgl. A. S.: \emph{Tagebuch}, 9. 12. 1904.
                     }}}\label{K_L03402-1}\pwindex{?? [Gipsnachbildung einer Statue von Donatello]@\emph{?? [Gipsnachbildung einer Statue von Donatello]}|pwv}\pwindex{Donatello um 1368 – 13.12.1466@\textsc{Donatello} (um 1368 – 13.12.1466), \emph{Bildhauer/Bildhauerin}|pw} bedanken wir\pwindex{Salten, Ottilie 07.03.1868 – 22.06.1942@\textsc{Salten, Ottilie} (07.03.1868 – 22.06.1942), \emph{Schauspieler/Schauspielerin}|pwv} uns
               herzlich und erfreut.\pend
           
\pstart
           Wir sind auch beim \label{K_L03402-2v}\edtext{Mahler\pwindex{Mahler, Gustav 07.07.1860 – 18.05.1911@\textsc{Mahler, Gustav} (07.07.1860 – 18.05.1911), \emph{Theaterleiter/Theaterleiterin, Komponist/Komponistin, Dirigent/Dirigentin}|pw}-Conzert}{\lemma{\textnormal{\emph{Mahler-Conzert}}}\Cendnote{\textnormal{Am 22. 12. 1904 wurde die \emph{3. Sinfonie in
                     d-Moll}\pwindex{3. Sinfonie in d-Moll@\emph{3. Sinfonie in d-Moll}|pwk} im Großen Musikvereinssaal\oindex{Musikverein@\textbf{Musikverein}, \emph{Konzertsaal (K.KNZ)}|pwk}
                  gegeben. Wie aus den folgenden Briefen hervorgeht, verpassten sie sich im Riedhof\oindex{Riedhof@\textbf{Riedhof}, \emph{Lokal (K.LKL)}|pwk}.}}}\label{K_L03402-2}, und könnten dann ev. zusammen
               in den Riedhof\oindex{Riedhof@\textbf{Riedhof}, \emph{Lokal (K.LKL)}|pw}, jedesfalls aber uns dort nachher
               treffen.\pend
           
\pstart
           Herzlichst {\\[\baselineskip]}Ihr {\\[\baselineskip]}\spacefill\mbox{Salten}\pend
           \leftskip=0em{}\selectlanguage{ngerman}\endnumbering\briefempfaengerindex{Schnitzler, Arthur@\textsc{Schnitzler, Arthur}!zzzSalten, Felix@\emph{von Felix Salten}!1904-12-201@{{[}20. 12. 1904{]}}|)be}\mylabel{L03402h}  \normalsize

\doendnotes{C}
\bigskip
\vfill

\clearpage

\footnotesize

\lohead{\textsc{register}}

% Definiere theindex-Environment komplett neu ohne reledmac
\makeatletter
\renewenvironment{theindex}{%
  \section*{\indexname}%
  \setlength{\parindent}{0pt}%
  \setlength{\parskip}{0pt plus 0.3pt}%
  \let\item\@idxitem
}{%
  \clearpage
}
\makeatother

\IfFileExists{\jobname-pw.ind}{\input{\jobname-pw.ind}}{}

\end{document}

      