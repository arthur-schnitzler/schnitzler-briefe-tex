%% latex-leseansicht-vorspann.tex
%% Vorspann für die Leseansicht.
%% Lädt die gemeinsame Datei latex-vorspann.tex mit nicht gesetztem Schalter.

\newif\ifkorrekturansicht
\korrekturansichtfalse

\input{../tex-inputs/latex-vorspann}


\section[Marie Herzfeld an Arthur Schnitzler, 19.\,4.\,1909]{L02593 Marie Herzfeld an Arthur Schnitzler, 19.\,4.\,1909}
\nopagebreak\mylabel{L02593v}
\rehead{ }\normalsize\beginnumbering\briefempfaengerindex{Schnitzler, Arthur@\textsc{Schnitzler, Arthur}!zzzHerzfeld, Marie@\emph{von Marie Herzfeld}!1909-04-191@{19.\,4.\,1909}|(be}
\toendnotes[C]{\smallbreak\pagebreak[2]}
\correspDesc{Versand  durch Marie Herzfeld am 19. 4. 1909 in Wien
\newline{}Erhalt  durch Arthur Schnitzler im Zeitraum [19. 4. 1909
                  – 23. 4. 1909?] in Wien}\toendnotes[C]{\smallbreak}
\Standort{DLA, A:Schnitzler, HS.1985.1.03436,4.}
\physDesc{Brief, 1 Blatt, 4 Seiten, 1258 Zeichen
\newline{}Handschrift: schwarze Tinte, lateinische Kurrent
\newline{}Schnitzler: 1) mit Bleistift Vermerk »\textsc{Herzfeld}«  2) mit rotem Buntstift Vermerk »\textsc{Tesi\pwindex{Rotenstern-Tesi, Anna *~11.\,1.\,1871 Odessa@\textsc{Rotenstern-Tesi, Anna} (*~11.\,1.\,1871 Odessa), \emph{Übersetzerin}|pw}, Mi\textcolor{gray}{c}h\textcolor{gray}{ae}{[}lis{]}\pwindex{Michaëlis, Sophus 14.\,5.\,1865 Odense – 28.\,1.\,1932 Kopenhagen@\textsc{Michaëlis, Sophus} (14.\,5.\,1865 Odense – 28.\,1.\,1932 Kopenhagen), \emph{Schriftsteller}|pw}}« und drei Unterstreichungen}\toendnotes[C]{\smallbreak}
\pstart
           \raggedleft{}{\pb}Wien II/\textsubscript{2},
                     Lichtenauerg. 5\oindex{Wien@\textbf{Wien}!II., Leopoldstadt@\textbf{II., Leopoldstadt}!Lichtenauergasse@\textbf{Lichtenauergasse}, \emph{Straße}|pw}{\\}den 19. April 1909\pend
           
\pstart{}Sehr geehrter Herr Doktor!\pend\vspace{0.5em}
\pstart
           Es besuchte mich heute Frau Anna Tesi\pwindex{Rotenstern-Tesi, Anna *~11.\,1.\,1871 Odessa@\textsc{Rotenstern-Tesi, Anna} (*~11.\,1.\,1871 Odessa), \emph{Übersetzerin}|pw}, die mir sagt, Sie hätten von Soph. Michaëlis\pwindex{Michaëlis, Sophus 14.\,5.\,1865 Odense – 28.\,1.\,1932 Kopenhagen@\textsc{Michaëlis, Sophus} (14.\,5.\,1865 Odense – 28.\,1.\,1932 Kopenhagen), \emph{Schriftsteller}|pw}{ }\label{K_L02593-1v}\edtext{»Revolutionshochzeit\pwindex{Revolutionsbryllup. Skuespil i tre Akter@\emph{Revolutionsbryllup. Skuespil i tre Akter}|pw}«}{\lemma{\textnormal{\emph{»Revolutionshochzeit«}}}\Cendnote{\textnormal{Sophus Michaelis\pwindex{Michaëlis, Sophus 14.\,5.\,1865 Odense – 28.\,1.\,1932 Kopenhagen@\textsc{Michaëlis, Sophus} (14.\,5.\,1865 Odense – 28.\,1.\,1932 Kopenhagen), \emph{Schriftsteller}|pwk}: \emph{Revolutionshochzeit. Schauspiel in drei Aufzügen}\pwindex{Michaëlis, Sophus 14.\,5.\,1865 Odense – 28.\,1.\,1932 Kopenhagen@\textsc{Michaëlis, Sophus} (14.\,5.\,1865 Odense – 28.\,1.\,1932 Kopenhagen), \emph{Schriftsteller}!Revolutionshochzeit. Schauspiel in drei Aufzügen@\strich\emph{Revolutionshochzeit. Schauspiel in drei Aufzügen}|pwk}. Aus
                     dem Dänischen von Marie Herzfeld\pwindex{Herzfeld, Marie 20.\,3.\,1855 Kőszeg – 22.\,9.\,1940 Mining@\textsc{Herzfeld, Marie} (20.\,3.\,1855 Kőszeg – 22.\,9.\,1940 Mining), \emph{Schriftstellerin, Übersetzerin}|pwk}. Berlin\oindex{Berlin@\textbf{Berlin}, \emph{Hauptstadt}|pwk}: \emph{Erich
                        Reiss-Verlag}\orgindex{Erich-Reiss-Verlag@Erich-Reiss-Verlag|pwk}{ }1909.}}}\label{K_L02593-1} gesprochen und ihr Interesse für das Stück\pwindex{Revolutionsbryllup. Skuespil i tre Akter@\emph{Revolutionsbryllup. Skuespil i tre Akter}|pwv} so lebhaft erweckt, dass sie gern das
               Uebersetzungs- und Vertretungsrecht für Frankreich\oindex{Frankreich@\textbf{Frankreich}|pw} und Russland\oindex{Russland@\textbf{Russland}|pw} erwerben
               möchte. {\pb}Sie sagt, sie sei \label{K_L02593-2v}\edtext{Ihre russ. Uebersetzerin}{\lemma{\textnormal{\emph{Ihre russ. Uebersetzerin}}}\Cendnote{\textnormal{Autorisierter Übersetzer Schnitzlers war der Ehemann von Anna
                     Rotenstern-Tesi\pwindex{Rotenstern-Tesi, Anna *~11.\,1.\,1871 Odessa@\textsc{Rotenstern-Tesi, Anna} (*~11.\,1.\,1871 Odessa), \emph{Übersetzerin}|pwk}, Peter Rotenstern\pwindex{Rotenstern, Peter 10.\,1.\,1868 Odessa – 1944@\textsc{Rotenstern, Peter} (10.\,1.\,1868 Odessa – 1944), \emph{Journalist, Übersetzer}|pwk}.
                  Zu den von Tesi\pwindex{Rotenstern-Tesi, Anna *~11.\,1.\,1871 Odessa@\textsc{Rotenstern-Tesi, Anna} (*~11.\,1.\,1871 Odessa), \emph{Übersetzerin}|pwk} übersetzten Texten siehe XXXX Auszeichnungsfehler: Dokument L02597 nicht gefunden.}}}\label{K_L02593-2}, sei Mitglied
               der Société des Auteurs dramatiques in Paris\oindex{Paris@\textbf{Paris}, \emph{Hauptstadt}|pw}\orgindex{Société des Auteurs et Compositeurs Dramatiques@Société des Auteurs et Compositeurs Dramatiques|pw} u. in Moskau\oindex{Moskau@\textbf{Moskau}, \emph{Land}|pw}\orgindex{Verein dramatischer und musikalischer Autoren@Verein dramatischer und musikalischer Autoren|pwv} (Petersburg\oindex{Sankt Petersburg@\textbf{Sankt Petersburg}|pw}?). Sie will Abmachungen, die
               dahin gehen, dass sie die Hälfte aller Tantièmen u. Honorare mir abliefert. Die
               Controlle, sagt sie, sei in Händen jener Sociétés\orgindex{Société des Auteurs et Compositeurs Dramatiques@Société des Auteurs et Compositeurs Dramatiques|pwv}\orgindex{Verein dramatischer und musikalischer Autoren@Verein dramatischer und musikalischer Autoren|pwv}. Ich glaube ihr alles; sie macht mir
                  persön{\pb}lich einen vertrauenerweckenden Eindruck; aber
               sie hat einen Gesellschafter, ihren Mann\pwindex{Rotenstern, Peter 10.\,1.\,1868 Odessa – 1944@\textsc{Rotenstern, Peter} (10.\,1.\,1868 Odessa – 1944), \emph{Journalist, Übersetzer}|pwv}, den sie mir nicht zeigte {\dots}
               kurz – so sehr ich in der Regel meinem Gefühl folge, so muss ich, als Vertreterin der
               Interessen des dänischen
                  Dichters\pwindex{Michaëlis, Sophus 14.\,5.\,1865 Odense – 28.\,1.\,1932 Kopenhagen@\textsc{Michaëlis, Sophus} (14.\,5.\,1865 Odense – 28.\,1.\,1932 Kopenhagen), \emph{Schriftsteller}|pwv} dennoch etwas vorsichtig sein und wäre Ihnen daher sehr dankbar,
               wenn Sie mir sagten, {\pb}ob \introOben{}a)\introOben{} Sie
               mit dem Ehepaar Tesi\pwindex{Rotenstern-Tesi, Anna *~11.\,1.\,1871 Odessa@\textsc{Rotenstern-Tesi, Anna} (*~11.\,1.\,1871 Odessa), \emph{Übersetzerin}|pw}\pwindex{Rotenstern, Peter 10.\,1.\,1868 Odessa – 1944@\textsc{Rotenstern, Peter} (10.\,1.\,1868 Odessa – 1944), \emph{Journalist, Übersetzer}|pw} gute Erfahrungen
               machten und ob b) die Bedingungen, die man mir bietet, billige sind. Ich habe mich um
               dergleichen nie bekümmert und bin naïv wie ein neugeborenes Kind.\pend
           
\pstart
           Pardon, dass ich Ihnen Mühe mache! Ihnen stets zu Gegendiensten, in großer
               Schätzung\pend
           \pstart \spacefill\mbox{Marie Herzfeld}\pend{}\selectlanguage{ngerman}\endnumbering\briefempfaengerindex{Schnitzler, Arthur@\textsc{Schnitzler, Arthur}!zzzHerzfeld, Marie@\emph{von Marie Herzfeld}!1909-04-191@{19.\,4.\,1909}|)be}\mylabel{L02593h}  \newcommand{\dateiname}{L02593}\newcommand{\titel}{Marie Herzfeld an Arthur Schnitzler, 19. 4. 1909}\newcommand{\editorInnen}{Martin Anton Müller und Laura Untner}%% latex-leseansicht-abspann.tex
%% Abspann für die Leseansicht.
%% Der Schalter \ifkorrekturansicht ist bereits durch den Vorspann gesetzt.

%% latex-abspann.tex
%% Gemeinsamer Abspann für Korrekturansicht und Leseansicht.
%% Setzt den Schalter \ifkorrekturansicht voraus (gesetzt in den
%% einbindenden Dateien latex-korrekturansicht-abspann.tex bzw.
%% latex-leseansicht-abspann.tex).
%% ---------------------------------------------------------------

\normalsize

% Das esempio-Environment wird nur in der Leseansicht benötigt
\ifkorrekturansicht\else
\newenvironment{esempio}[3]%
{
    \vspace{1.5ex}
    \rlap{\underline{#1}}
    \par
    \setlength{\parindent}{0cm}
    \nopagebreak
    \leftskip=#2cm
    \rightskip=#3cm
}
{
    \par
}
\fi

\doendnotes{C}
\bigskip
\vfill

\clearpage

\footnotesize

\ifkorrekturansicht
  \lohead{\textsc{register}}
\fi

% theindex-Environment neu definieren ohne reledmac
\makeatletter
\renewenvironment{theindex}{%
  \ifkorrekturansicht
    \section*{\indexname}%
  \else
    \subsubsection*{Index der erwähnten Entitäten}%
  \fi
  \setlength{\parindent}{0pt}%
  \setlength{\parskip}{0pt plus 0.3pt}%
  \let\item\@idxitem
}{%
  \ifkorrekturansicht\clearpage\fi
}
\makeatother

\IfFileExists{\jobname-pw.ind}{\input{\jobname-pw.ind}}{}

% Quellenangabe nur in der Leseansicht
\ifkorrekturansicht\else
% Fallback-Definitionen, falls die .tex-Datei \titel etc. nicht gesetzt hat
\providecommand{\titel}{}
\providecommand{\editorInnen}{}
\providecommand{\dateiname}{\jobname}

\vspace{3cm}

\vfill

\footnotesize
\textsc{Quelle}: \titel. Herausgegeben von {\editorInnen}. In: \emph{Arthur Schnitzler: Briefwechsel mit Autorinnen und Autoren}.
 Digitale Edition, https://schnitzler-briefe.acdh.oeaw.ac.at/{\dateiname}.html (Stand \today)
\fi

\end{document}


