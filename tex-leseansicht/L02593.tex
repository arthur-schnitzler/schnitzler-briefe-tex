%% latex-korrekturansicht-vorspann.tex
%% Vorspann für die Korrekturansicht.
%% Lädt die gemeinsame Datei latex-vorspann.tex mit gesetztem Schalter.

\newif\ifkorrekturansicht
\korrekturansichttrue

\input{../tex-inputs/latex-vorspann}


\section[Marie Herzfeld an Arthur Schnitzler, 19. 4. 1909]{L02593 Marie Herzfeld an Arthur Schnitzler, 19. 4. 1909}
\nopagebreak\mylabel{L02593v}
\rehead{ }\normalsize\beginnumbering\briefempfaengerindex{Schnitzler, Arthur@\textsc{Schnitzler, Arthur}!zzzHerzfeld, Marie@\emph{von Marie Herzfeld}!1909-04-191@{19. 4. 1909}|(be}
\toendnotes[C]{\smallbreak\pagebreak[2]}\Standort{DLA, A:Schnitzler, HS.1985.1.03436,4.}
\physDesc{Brief, 1 Blatt, 4 Seiten, 1258 Zeichen
\newline{}Handschrift: schwarze Tinte, lateinische Kurrent
\newline{}Schnitzler: 1) mit Bleistift Vermerk »\textsc{Herzfeld}«  2) mit rotem Buntstift Vermerk »\textsc{Tesi\pwindex{Rotenstern-Tesi, Anna *~1871-01-11@\textsc{Rotenstern-Tesi, Anna} (*~1871-01-11), \emph{Übersetzer/Übersetzerin}|pw}, Mi\textcolor{gray}{c}h\textcolor{gray}{ae}{[}lis{]}\pwindex{Michaelis, Sophus 14.05.1865 – 28.01.1932@\textsc{Michaëlis, Sophus} (14.05.1865 – 28.01.1932), \emph{Schriftsteller/Schriftstellerin}|pw}}« und drei Unterstreichungen}\toendnotes[C]{\smallbreak}
\pstart
           \raggedleft{}{\pb}Wien II/\textsubscript{2},
                     Lichtenauerg. 5\oindex{Lichtenauergasse@\textbf{Lichtenauergasse}, \emph{Straße (K.STR)}|pw}{\\}den 19. April 1909\pend
           
\pstart{}Sehr geehrter Herr Doktor!\pend\vspace{0.5em}
\pstart
           Es besuchte mich heute Frau Anna Tesi\pwindex{Rotenstern-Tesi, Anna *~1871-01-11@\textsc{Rotenstern-Tesi, Anna} (*~1871-01-11), \emph{Übersetzer/Übersetzerin}|pw}, die mir sagt, Sie hätten von Soph. Michaëlis\pwindex{Michaelis, Sophus 14.05.1865 – 28.01.1932@\textsc{Michaëlis, Sophus} (14.05.1865 – 28.01.1932), \emph{Schriftsteller/Schriftstellerin}|pw}{ }\label{K_L02593-1v}\edtext{»Revolutionshochzeit\pwindex{Revolutionsbryllup. Skuespil i tre Akter@\emph{Revolutionsbryllup. Skuespil i tre Akter}|pw}«}{\lemma{\textnormal{\emph{»Revolutionshochzeit«}}}\Cendnote{\textnormal{Sophus Michaelis\pwindex{Michaelis, Sophus 14.05.1865 – 28.01.1932@\textsc{Michaëlis, Sophus} (14.05.1865 – 28.01.1932), \emph{Schriftsteller/Schriftstellerin}|pwk}: \emph{Revolutionshochzeit. Schauspiel in drei Aufzügen}\pwindex{Revolutionshochzeit. Schauspiel in drei Aufzuegen@\emph{Revolutionshochzeit. Schauspiel in drei Aufzügen}|pwk}. Aus
                     dem Dänischen von Marie Herzfeld\pwindex{Herzfeld, Marie 20.03.1855 – 22.09.1940@\textsc{Herzfeld, Marie} (20.03.1855 – 22.09.1940), \emph{Schriftsteller/Schriftstellerin, Übersetzer/Übersetzerin}|pwk}. Berlin\oindex{Berlin@\textbf{Berlin}, \emph{P.PPLC}|pwk}: \emph{Erich
                        Reiss-Verlag}\orgindex{Erich-Reiss-Verlag@Erich-Reiss-Verlag|pwk}{ }1909.}}}\label{K_L02593-1} gesprochen und ihr Interesse für das Stück\pwindex{Revolutionsbryllup. Skuespil i tre Akter@\emph{Revolutionsbryllup. Skuespil i tre Akter}|pwv} so lebhaft erweckt, dass sie gern das
               Uebersetzungs- und Vertretungsrecht für Frankreich\oindex{Frankreich@\textbf{Frankreich}, \emph{A.PCLI}|pw} und Russland\oindex{Russland@\textbf{Russland}, \emph{A.PCLI}|pw} erwerben
               möchte. {\pb}Sie sagt, sie sei \label{K_L02593-2v}\edtext{Ihre russ. Uebersetzerin}{\lemma{\textnormal{\emph{Ihre russ. Uebersetzerin}}}\Cendnote{\textnormal{Autorisierter Übersetzer Schnitzlers war der Ehemann von Anna
                     Rotenstern-Tesi\pwindex{Rotenstern-Tesi, Anna *~1871-01-11@\textsc{Rotenstern-Tesi, Anna} (*~1871-01-11), \emph{Übersetzer/Übersetzerin}|pwk}, Peter Rotenstern\pwindex{Rotenstern, Peter 10.01.1868 – 1944@\textsc{Rotenstern, Peter} (10.01.1868 – 1944), \emph{Journalist/Journalistin, Übersetzer/Übersetzerin}|pwk}.
                  Zu den von Tesi\pwindex{Rotenstern-Tesi, Anna *~1871-01-11@\textsc{Rotenstern-Tesi, Anna} (*~1871-01-11), \emph{Übersetzer/Übersetzerin}|pwk} übersetzten Texten siehe Arthur Schnitzler an Marie Herzfeld, 20. 4. 1909.}}}\label{K_L02593-2}, sei Mitglied
               der Société des Auteurs dramatiques in Paris\oindex{Paris@\textbf{Paris}, \emph{P.PPLC}|pw}\orgindex{Societe des Auteurs et Compositeurs Dramatiques@Société des Auteurs et Compositeurs Dramatiques|pw} u. in Moskau\oindex{Moskau@\textbf{Moskau}, \emph{A.ADM1}|pw}\orgindex{Verein dramatischer und musikalischer Autoren@Verein dramatischer und musikalischer Autoren|pwv} (Petersburg\oindex{Sankt Petersburg@\textbf{Sankt Petersburg}, \emph{P.PPLA}|pw}?). Sie will Abmachungen, die
               dahin gehen, dass sie die Hälfte aller Tantièmen u. Honorare mir abliefert. Die
               Controlle, sagt sie, sei in Händen jener Sociétés\orgindex{Societe des Auteurs et Compositeurs Dramatiques@Société des Auteurs et Compositeurs Dramatiques|pwv}\orgindex{Verein dramatischer und musikalischer Autoren@Verein dramatischer und musikalischer Autoren|pwv}. Ich glaube ihr alles; sie macht mir
                  persön{\pb}lich einen vertrauenerweckenden Eindruck; aber
               sie hat einen Gesellschafter, ihren Mann\pwindex{Rotenstern, Peter 10.01.1868 – 1944@\textsc{Rotenstern, Peter} (10.01.1868 – 1944), \emph{Journalist/Journalistin, Übersetzer/Übersetzerin}|pwv}, den sie mir nicht zeigte {\dots}
               kurz – so sehr ich in der Regel meinem Gefühl folge, so muss ich, als Vertreterin der
               Interessen des dänischen
                  Dichters\pwindex{Michaelis, Sophus 14.05.1865 – 28.01.1932@\textsc{Michaëlis, Sophus} (14.05.1865 – 28.01.1932), \emph{Schriftsteller/Schriftstellerin}|pwv} dennoch etwas vorsichtig sein und wäre Ihnen daher sehr dankbar,
               wenn Sie mir sagten, {\pb}ob \introOben{}a)\introOben{} Sie
               mit dem Ehepaar Tesi\pwindex{Rotenstern-Tesi, Anna *~1871-01-11@\textsc{Rotenstern-Tesi, Anna} (*~1871-01-11), \emph{Übersetzer/Übersetzerin}|pw}\pwindex{Rotenstern, Peter 10.01.1868 – 1944@\textsc{Rotenstern, Peter} (10.01.1868 – 1944), \emph{Journalist/Journalistin, Übersetzer/Übersetzerin}|pw} gute Erfahrungen
               machten und ob b) die Bedingungen, die man mir bietet, billige sind. Ich habe mich um
               dergleichen nie bekümmert und bin naïv wie ein neugeborenes Kind.\pend
           
\pstart
           Pardon, dass ich Ihnen Mühe mache! Ihnen stets zu Gegendiensten, in großer
               Schätzung\pend
           \pstart \spacefill\mbox{Marie Herzfeld}\pend{}\selectlanguage{ngerman}\endnumbering\briefempfaengerindex{Schnitzler, Arthur@\textsc{Schnitzler, Arthur}!zzzHerzfeld, Marie@\emph{von Marie Herzfeld}!1909-04-191@{19. 4. 1909}|)be}\mylabel{L02593h}  \normalsize

\doendnotes{C}
\bigskip
\vfill

\clearpage

\footnotesize

\lohead{\textsc{register}}

% Definiere theindex-Environment komplett neu ohne reledmac
\makeatletter
\renewenvironment{theindex}{%
  \section*{\indexname}%
  \setlength{\parindent}{0pt}%
  \setlength{\parskip}{0pt plus 0.3pt}%
  \let\item\@idxitem
}{%
  \clearpage
}
\makeatother

\IfFileExists{\jobname-pw.ind}{\input{\jobname-pw.ind}}{}

\end{document}

      