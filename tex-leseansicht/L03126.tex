%% latex-leseansicht-vorspann.tex
%% Vorspann für die Leseansicht.
%% Lädt die gemeinsame Datei latex-vorspann.tex mit nicht gesetztem Schalter.

\newif\ifkorrekturansicht
\korrekturansichtfalse

\input{../tex-inputs/latex-vorspann}


         
         \renewcommand{\erwaehntePersonen}{Personen: Franz Defregger, Josef Putzenbacher, Felix Salten, Gustav Schwarzkopf, Emil Schwarzkopf, Max Schwarzkopf, Rudolf Schwarzkopf}
         \renewcommand{\erwaehnteOrte}{Orte: Cortina d'Ampezzo, Dölsach, Lienz, Pfarrkirche Dölsach, Polen, Putzenbacher, Toblach, Wien}
         \renewcommand{\erwaehnteWerke}{Werke: Heilige Familie}
               \section[ Felix Salten an Arthur Schnitzler, 12. 8. 1893]{ Felix Salten an Arthur Schnitzler, 12. 8. 1893}\nopagebreak\mylabel{v}\rehead{ }\begin{ledgroupsized}[t]{13cm}\normalsize\beginnumbering\briefempfaengerindex{Schnitzler, Arthur@\textsc{Schnitzler, Arthur}!zzzSalten, Felix@\emph{von Felix Salten}!1893-08-125@{12. 8. 1893}|(be} \toendnotes[C]{\smallbreak\pagebreak[2]} \Standort{CUL, Schnitzler, B 89, A 1.}
\physDesc{Brief, 1 Blatt, 2 Seiten, 602 Zeichen
\newline{}Handschrift: Bleistift, lateinische Kurrent
\newline{}Ordnung: mit Bleistift von unbekannter Hand nummeriert: »29« }\toendnotes[C]{\smallbreak}\pstart
           \noindent{}{\pb}Lieber Freund!{ }Hier\oindex{Doelsach@\textbf{Dölsach}|pwv} ist es einfach herrlich.
                  Gestern mit Rad und Hund in \strikeout{Dölsach}\oindex{Doelsach@\textbf{Dölsach}|pw}{ }\substVorne{}\textsuperscript{g}\substDazwischen{}L\substHinten{}ienz\oindex{Lienz@\textbf{Lienz}|pw} gewesen, und dort eine Einladung zu einem Radfahrfeste erhalten.
               Im Coupé mit einem poln\oindex{Polen@\textbf{Polen}|pw}ischen Juden über’s –
               Bicycle gesprochen. Nächste Woche fahre ich per Bahn nach Toblach\oindex{Toblach@\textbf{Toblach}|pw}, von da nach Cortina\oindex{Cortina d'Ampezzo@\textbf{Cortina d'Ampezzo}|pw}. Dann berichte ich über Alles.\pend
           \pstart
           Hier in der kleinen Dorfkirche\oindex{Pfarrkirche Doelsach@\textbf{Pfarrkirche Dölsach}|pwv} ist das Original von Defregger\pwindex{Defregger, Franz 1835-04-30 – 1921-01-02@\textsc{Defregger, Franz} (1835-04-30 – 1921-01-02), \emph{Maler, Künstler}|pw}’s Madonna\pwindex{Defregger, Franz 1835-04-30 – 1921-01-02@\textsc{Defregger, Franz} (1835-04-30 – 1921-01-02), \emph{Maler, Künstler}!Heilige Familie1872@\strich\emph{Heilige Familie} {[}1872{]}|pw}, und viele Jugendskizzen, wie Portraits von ihm zeigt der \label{K_L03126-1v}\edtext{Wirth\pwindex{Putzenbacher, Josef 1836-05-23 – 1904-07-05@\textsc{Putzenbacher, Josef} (1836-05-23 – 1904-07-05), \emph{Gastwirt}|pwuv} in
               seiner Stube\oindex{Putzenbacher@\textbf{Putzenbacher}|pwuv}}{\lemma{\textnormal{\emph{Wirth in
               seiner Stube}}}\Cendnote{\textnormal{Josef Putzenbacher\oindex{Putzenbacher@\textbf{Putzenbacher}|pwk}\pwindex{Putzenbacher, Josef 1836-05-23 – 1904-07-05@\textsc{Putzenbacher, Josef} (1836-05-23 – 1904-07-05), \emph{Gastwirt}|pwk}?}}}\label{K_L03126-1h}. Wenn Sie schreiben, dann {\pb}bitte Dölsach\oindex{Doelsach@\textbf{Dölsach}|pw}{ }\textsuperscript{b}/Lienz\oindex{Lienz@\textbf{Lienz}|pw}, poste
               restante.\pend
           \pstart
           Grüßen Sie Schwarzkopf’s\pwindex{Schwarzkopf, Gustav 07.11.1853 – 13.11.1939@\textsc{Schwarzkopf, Gustav} (07.11.1853 – 13.11.1939), \emph{Schriftsteller}|pw}\pwindex{Schwarzkopf, Emil 17.09.1851 – 28.01.1928@\textsc{Schwarzkopf, Emil} (17.09.1851 – 28.01.1928), \emph{Übersetzer, Komponist}|pw}\pwindex{Schwarzkopf, Max 12.06.1857 – 14.04.1928@\textsc{Schwarzkopf, Max} (12.06.1857 – 14.04.1928), \emph{Rechtsanwalt}|pw}\pwindex{Schwarzkopf, Rudolf 25.05.1861 – 13.10.1893@\textsc{Schwarzkopf, Rudolf} (25.05.1861 – 13.10.1893), \emph{Schriftsteller}|pw} und seien Sie herzlich gegrüßt\pend
           \pstart
           Ihr treuer {\\[\baselineskip]}\spacefill\mbox{Salten}\pend
           \leftskip=0em{}\pstart
           \noindent{}Dölsach\oindex{Doelsach@\textbf{Dölsach}|pw}, 12 Aug. 93.\pend
           
         
         \endnumbering\mylabel{h}\end{ledgroupsized}  \newcommand{\dateiname}{L03126}\newcommand{\titel}{Felix Salten an Arthur Schnitzler, 12. 8. 1893}\newcommand{\editorInnen}{Martin Anton Müller und Laura Untner}%% latex-leseansicht-abspann.tex
%% Abspann für die Leseansicht.
%% Der Schalter \ifkorrekturansicht ist bereits durch den Vorspann gesetzt.

%% latex-abspann.tex
%% Gemeinsamer Abspann für Korrekturansicht und Leseansicht.
%% Setzt den Schalter \ifkorrekturansicht voraus (gesetzt in den
%% einbindenden Dateien latex-korrekturansicht-abspann.tex bzw.
%% latex-leseansicht-abspann.tex).
%% ---------------------------------------------------------------

\normalsize

% Das esempio-Environment wird nur in der Leseansicht benötigt
\ifkorrekturansicht\else
\newenvironment{esempio}[3]%
{
    \vspace{1.5ex}
    \rlap{\underline{#1}}
    \par
    \setlength{\parindent}{0cm}
    \nopagebreak
    \leftskip=#2cm
    \rightskip=#3cm
}
{
    \par
}
\fi

\doendnotes{C}
\bigskip
\vfill

\clearpage

\footnotesize

\ifkorrekturansicht
  \lohead{\textsc{register}}
\fi

% theindex-Environment neu definieren ohne reledmac
\makeatletter
\renewenvironment{theindex}{%
  \ifkorrekturansicht
    \section*{\indexname}%
  \else
    \subsubsection*{Index der erwähnten Entitäten}%
  \fi
  \setlength{\parindent}{0pt}%
  \setlength{\parskip}{0pt plus 0.3pt}%
  \let\item\@idxitem
}{%
  \ifkorrekturansicht\clearpage\fi
}
\makeatother

\IfFileExists{\jobname-pw.ind}{\input{\jobname-pw.ind}}{}

% Quellenangabe nur in der Leseansicht
\ifkorrekturansicht\else
% Fallback-Definitionen, falls die .tex-Datei \titel etc. nicht gesetzt hat
\providecommand{\titel}{}
\providecommand{\editorInnen}{}
\providecommand{\dateiname}{\jobname}

\vspace{3cm}

\vfill

\footnotesize
\textsc{Quelle}: \titel. Herausgegeben von {\editorInnen}. In: \emph{Arthur Schnitzler: Briefwechsel mit Autorinnen und Autoren}.
 Digitale Edition, https://schnitzler-briefe.acdh.oeaw.ac.at/{\dateiname}.html (Stand \today)
\fi

\end{document}


      