%% latex-leseansicht-vorspann.tex
%% Vorspann für die Leseansicht.
%% Lädt die gemeinsame Datei latex-vorspann.tex mit nicht gesetztem Schalter.

\newif\ifkorrekturansicht
\korrekturansichtfalse

\input{../tex-inputs/latex-vorspann}


\section[ Felix Salten an Arthur Schnitzler, 12. 8. 1893]{L03126 Felix Salten an Arthur Schnitzler,  12. 8. 1893}
\nopagebreak\mylabel{L03126v}
\rehead{ }\normalsize\beginnumbering\briefempfaengerindex{Schnitzler, Arthur@\textsc{Schnitzler, Arthur}!zzzSalten, Felix@\emph{von Felix Salten}!1893-08-126@{12. 8. 1893}|(be}
\toendnotes[C]{\smallbreak\pagebreak[2]}
\correspDesc{Versand  durch Felix Salten am 12. 8. 1893 in Dölsach
\newline{}Erhalt  durch Arthur Schnitzler im Zeitraum [13. 8. 1893
                  – 14. 8. 1893?] in Wien}\toendnotes[C]{\smallbreak}
\Standort{CUL, Schnitzler, B 89, A 1.}
\physDesc{Brief, 1 Blatt, 2 Seiten, 602 Zeichen
\newline{}Handschrift: Bleistift, lateinische Kurrent
\newline{}Ordnung: mit Bleistift von unbekannter Hand nummeriert: »29« }\toendnotes[C]{\smallbreak}
\pstart
           \noindent{}{\pb}Lieber Freund!{ }Hier\oindex{Dölsach@\textbf{Dölsach}, \emph{Verwaltungsgebiet}|pwv} ist es einfach herrlich.
                  Gestern mit Rad und Hund in \strikeout{Dölsach}\oindex{Dölsach@\textbf{Dölsach}, \emph{Verwaltungsgebiet}|pw}{ }\substVorne{}\textsuperscript{g}\substDazwischen{}L\substHinten{}ienz\oindex{Lienz@\textbf{Lienz}, \emph{Hauptstadt}|pw} gewesen, und dort eine Einladung zu einem Radfahrfeste erhalten.
               Im Coupé mit einem poln\oindex{Polen@\textbf{Polen}|pw}ischen Juden über’s –
               Bicycle gesprochen. Nächste Woche fahre ich per Bahn nach Toblach\oindex{Toblach@\textbf{Toblach}, \emph{Verwaltungsgebiet}|pw}, von da nach Cortina\oindex{Cortina d’Ampezzo@\textbf{Cortina d’Ampezzo}, \emph{Hauptstadt}|pw}. Dann berichte ich über Alles.\pend
           
\pstart
           Hier in der kleinen Dorfkirche\oindex{Pfarrkirche Dölsach@\textbf{Pfarrkirche Dölsach}, \emph{Kirche}|pwv} ist das Original von Defregger\pwindex{Defregger, Franz 30.\,4.\,1835 Stronach – 2.\,1.\,1921 München@\textsc{Defregger, Franz} (30.\,4.\,1835 Stronach – 2.\,1.\,1921 München), \emph{Maler, Künstler}|pw}’s Madonna\pwindex{Defregger, Franz 30.\,4.\,1835 Stronach – 2.\,1.\,1921 München@\textsc{Defregger, Franz} (30.\,4.\,1835 Stronach – 2.\,1.\,1921 München), \emph{Maler, Künstler}!Heilige Familie@\strich\emph{Heilige Familie}|pw}, und viele Jugendskizzen, wie Portraits von ihm zeigt der \label{K_L03126-1v}\edtext{Wirth\pwindex{Putzenbacher, Josef 23.\,5.\,1836 – 5.\,7.\,1904@\textsc{Putzenbacher, Josef} (23.\,5.\,1836 – 5.\,7.\,1904), \emph{Gastwirt}|pwuv} in
               seiner Stube\oindex{Putzenbacher@\textbf{Putzenbacher}, \emph{Gastgewerbegebäude}|pwuv}}{\lemma{\textnormal{\emph{Wirth in
               seiner Stube}}}\Cendnote{\textnormal{Josef Putzenbacher\oindex{Putzenbacher@\textbf{Putzenbacher}, \emph{Gastgewerbegebäude}|pwk}\pwindex{Putzenbacher, Josef 23.\,5.\,1836 – 5.\,7.\,1904@\textsc{Putzenbacher, Josef} (23.\,5.\,1836 – 5.\,7.\,1904), \emph{Gastwirt}|pwk}?}}}\label{K_L03126-1}. Wenn Sie schreiben, dann {\pb}bitte Dölsach\oindex{Dölsach@\textbf{Dölsach}, \emph{Verwaltungsgebiet}|pw}{ }\textsuperscript{b}/Lienz\oindex{Lienz@\textbf{Lienz}, \emph{Hauptstadt}|pw}, poste
               restante.\pend
           
\pstart
           Grüßen Sie Schwarzkopf’s\pwindex{Schwarzkopf, Gustav 7.\,11.\,1853 Wien – 13.\,11.\,1939 ebd.@\textsc{Schwarzkopf, Gustav} (7.\,11.\,1853 Wien – 13.\,11.\,1939 ebd.), \emph{Schriftsteller}|pw}\pwindex{Schwarzkopf, Emil 17.\,9.\,1851 Wien – 28.\,1.\,1928 ebd.@\textsc{Schwarzkopf, Emil} (17.\,9.\,1851 Wien – 28.\,1.\,1928 ebd.), \emph{Übersetzer, Komponist, Musiklehrer}|pw}\pwindex{Schwarzkopf, Max 12.\,6.\,1857 Wien – 14.\,4.\,1928 ebd.@\textsc{Schwarzkopf, Max} (12.\,6.\,1857 Wien – 14.\,4.\,1928 ebd.), \emph{Rechtsanwalt}|pw}\pwindex{Schwarzkopf, Rudolf 25.\,5.\,1861 Wien – 13.\,10.\,1893 Meran@\textsc{Schwarzkopf, Rudolf} (25.\,5.\,1861 Wien – 13.\,10.\,1893 Meran), \emph{Schriftsteller}|pw} und seien Sie herzlich gegrüßt\pend
           
\pstart
           Ihr treuer {\\[\baselineskip]}\spacefill\mbox{Salten}\pend
           \leftskip=0em{}
\pstart
           \noindent{}Dölsach\oindex{Dölsach@\textbf{Dölsach}, \emph{Verwaltungsgebiet}|pw}, 12 Aug. 93.\pend
           \selectlanguage{ngerman}\endnumbering\briefempfaengerindex{Schnitzler, Arthur@\textsc{Schnitzler, Arthur}!zzzSalten, Felix@\emph{von Felix Salten}!1893-08-126@{12. 8. 1893}|)be}\mylabel{L03126h}  \newcommand{\dateiname}{L03126}\newcommand{\titel}{Felix Salten an Arthur Schnitzler, 12. 8. 1893}\newcommand{\editorInnen}{Martin Anton Müller und Laura Untner}%% latex-leseansicht-abspann.tex
%% Abspann für die Leseansicht.
%% Der Schalter \ifkorrekturansicht ist bereits durch den Vorspann gesetzt.

%% latex-abspann.tex
%% Gemeinsamer Abspann für Korrekturansicht und Leseansicht.
%% Setzt den Schalter \ifkorrekturansicht voraus (gesetzt in den
%% einbindenden Dateien latex-korrekturansicht-abspann.tex bzw.
%% latex-leseansicht-abspann.tex).
%% ---------------------------------------------------------------

\normalsize

% Das esempio-Environment wird nur in der Leseansicht benötigt
\ifkorrekturansicht\else
\newenvironment{esempio}[3]%
{
    \vspace{1.5ex}
    \rlap{\underline{#1}}
    \par
    \setlength{\parindent}{0cm}
    \nopagebreak
    \leftskip=#2cm
    \rightskip=#3cm
}
{
    \par
}
\fi

\doendnotes{C}
\bigskip
\vfill

\clearpage

\footnotesize

\ifkorrekturansicht
  \lohead{\textsc{register}}
\fi

% theindex-Environment neu definieren ohne reledmac
\makeatletter
\renewenvironment{theindex}{%
  \ifkorrekturansicht
    \section*{\indexname}%
  \else
    \subsubsection*{Index der erwähnten Entitäten}%
  \fi
  \setlength{\parindent}{0pt}%
  \setlength{\parskip}{0pt plus 0.3pt}%
  \let\item\@idxitem
}{%
  \ifkorrekturansicht\clearpage\fi
}
\makeatother

\IfFileExists{\jobname-pw.ind}{\input{\jobname-pw.ind}}{}

% Quellenangabe nur in der Leseansicht
\ifkorrekturansicht\else
% Fallback-Definitionen, falls die .tex-Datei \titel etc. nicht gesetzt hat
\providecommand{\titel}{}
\providecommand{\editorInnen}{}
\providecommand{\dateiname}{\jobname}

\vspace{3cm}

\vfill

\footnotesize
\textsc{Quelle}: \titel. Herausgegeben von {\editorInnen}. In: \emph{Arthur Schnitzler: Briefwechsel mit Autorinnen und Autoren}.
 Digitale Edition, https://schnitzler-briefe.acdh.oeaw.ac.at/{\dateiname}.html (Stand \today)
\fi

\end{document}


