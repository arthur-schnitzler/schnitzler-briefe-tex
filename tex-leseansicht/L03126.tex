%% latex-korrekturansicht-vorspann.tex
%% Vorspann für die Korrekturansicht.
%% Lädt die gemeinsame Datei latex-vorspann.tex mit gesetztem Schalter.

\newif\ifkorrekturansicht
\korrekturansichttrue

\input{../tex-inputs/latex-vorspann}


\section[ Felix Salten an Arthur Schnitzler, 12. 8. 1893]{L03126 Felix Salten an Arthur Schnitzler, 12. 8. 1893}
\nopagebreak\mylabel{L03126v}
\rehead{ }\normalsize\beginnumbering\briefempfaengerindex{Schnitzler, Arthur@\textsc{Schnitzler, Arthur}!zzzSalten, Felix@\emph{von Felix Salten}!1893-08-125@{12. 8. 1893}|(be}
\toendnotes[C]{\smallbreak\pagebreak[2]}\Standort{CUL, Schnitzler, B 89, A 1.}
\physDesc{Brief, 1 Blatt, 2 Seiten, 602 Zeichen
\newline{}Handschrift: Bleistift, lateinische Kurrent
\newline{}Ordnung: mit Bleistift von unbekannter Hand nummeriert: »29« }\toendnotes[C]{\smallbreak}
\pstart
           \noindent{}{\pb}Lieber Freund!{ }Hier\oindex{Doelsach@\textbf{Dölsach}, \emph{A.ADM3}|pwv} ist es einfach herrlich.
                  Gestern mit Rad und Hund in \strikeout{Dölsach}\oindex{Doelsach@\textbf{Dölsach}, \emph{A.ADM3}|pw}{ }\substVorne{}\textsuperscript{g}\substDazwischen{}L\substHinten{}ienz\oindex{Lienz@\textbf{Lienz}, \emph{P.PPLA3}|pw} gewesen, und dort eine Einladung zu einem Radfahrfeste erhalten.
               Im Coupé mit einem poln\oindex{Polen@\textbf{Polen}, \emph{A.PCLI}|pw}ischen Juden über’s –
               Bicycle gesprochen. Nächste Woche fahre ich per Bahn nach Toblach\oindex{Toblach@\textbf{Toblach}, \emph{A.ADM3}|pw}, von da nach Cortina\oindex{Cortina DAmpezzo@\textbf{Cortina d’Ampezzo}, \emph{P.PPLA3}|pw}. Dann berichte ich über Alles.\pend
           
\pstart
           Hier in der kleinen Dorfkirche\oindex{Pfarrkirche Doelsach@\textbf{Pfarrkirche Dölsach}, \emph{Kirche (K.KRC)}|pwv} ist das Original von Defregger\pwindex{Defregger, Franz 1835-04-30 – 1921-01-02@\textsc{Defregger, Franz} (1835-04-30 – 1921-01-02), \emph{Maler/Malerin, Künstler/Künstlerin}|pw}’s Madonna\pwindex{Heilige Familie@\emph{Heilige Familie}|pw}, und viele Jugendskizzen, wie Portraits von ihm zeigt der \label{K_L03126-1v}\edtext{Wirth\pwindex{Putzenbacher, Josef 1836-05-23 – 1904-07-05@\textsc{Putzenbacher, Josef} (1836-05-23 – 1904-07-05), \emph{Gastwirt/Gastwirtin}|pwuv} in
               seiner Stube\oindex{Putzenbacher@\textbf{Putzenbacher}, \emph{Gastgewerbegebäude (K.GGW)}|pwuv}}{\lemma{\textnormal{\emph{Wirth in
               seiner Stube}}}\Cendnote{\textnormal{Josef Putzenbacher\oindex{Putzenbacher@\textbf{Putzenbacher}, \emph{Gastgewerbegebäude (K.GGW)}|pwk}\pwindex{Putzenbacher, Josef 1836-05-23 – 1904-07-05@\textsc{Putzenbacher, Josef} (1836-05-23 – 1904-07-05), \emph{Gastwirt/Gastwirtin}|pwk}?}}}\label{K_L03126-1}. Wenn Sie schreiben, dann {\pb}bitte Dölsach\oindex{Doelsach@\textbf{Dölsach}, \emph{A.ADM3}|pw}{ }\textsuperscript{b}/Lienz\oindex{Lienz@\textbf{Lienz}, \emph{P.PPLA3}|pw}, poste
               restante.\pend
           
\pstart
           Grüßen Sie Schwarzkopf’s\pwindex{Schwarzkopf, Gustav 07.11.1853 – 13.11.1939@\textsc{Schwarzkopf, Gustav} (07.11.1853 – 13.11.1939), \emph{Schriftsteller/Schriftstellerin}|pw}\pwindex{Schwarzkopf, Emil 17.09.1851 – 28.01.1928@\textsc{Schwarzkopf, Emil} (17.09.1851 – 28.01.1928), \emph{Übersetzer/Übersetzerin, Komponist/Komponistin}|pw}\pwindex{Schwarzkopf, Max 12.06.1857 – 14.04.1928@\textsc{Schwarzkopf, Max} (12.06.1857 – 14.04.1928), \emph{Rechtsanwalt/Rechtsanwältin}|pw}\pwindex{Schwarzkopf, Rudolf 25.05.1861 – 13.10.1893@\textsc{Schwarzkopf, Rudolf} (25.05.1861 – 13.10.1893), \emph{Schriftsteller/Schriftstellerin}|pw} und seien Sie herzlich gegrüßt\pend
           
\pstart
           Ihr treuer {\\[\baselineskip]}\spacefill\mbox{Salten}\pend
           \leftskip=0em{}
\pstart
           \noindent{}Dölsach\oindex{Doelsach@\textbf{Dölsach}, \emph{A.ADM3}|pw}, 12 Aug. 93.\pend
           \selectlanguage{ngerman}\endnumbering\briefempfaengerindex{Schnitzler, Arthur@\textsc{Schnitzler, Arthur}!zzzSalten, Felix@\emph{von Felix Salten}!1893-08-125@{12. 8. 1893}|)be}\mylabel{L03126h}  \normalsize

\doendnotes{C}
\bigskip
\vfill

\clearpage

\footnotesize

\lohead{\textsc{register}}

% Definiere theindex-Environment komplett neu ohne reledmac
\makeatletter
\renewenvironment{theindex}{%
  \section*{\indexname}%
  \setlength{\parindent}{0pt}%
  \setlength{\parskip}{0pt plus 0.3pt}%
  \let\item\@idxitem
}{%
  \clearpage
}
\makeatother

\IfFileExists{\jobname-pw.ind}{\input{\jobname-pw.ind}}{}

\end{document}

      