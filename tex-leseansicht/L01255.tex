%% latex-leseansicht-vorspann.tex
%% Vorspann für die Leseansicht.
%% Lädt die gemeinsame Datei latex-vorspann.tex mit nicht gesetztem Schalter.

\newif\ifkorrekturansicht
\korrekturansichtfalse

\input{../tex-inputs/latex-vorspann}


         
         \renewcommand{\erwaehntePersonen}{Personen: Leopold von Andrian-Werburg, Hermann Bahr, Clemens von Franckenstein, Robert Michel, Max von Millenkovich, Max Reinhardt}
         \renewcommand{\erwaehnteInstitutionen}{Institutionen: Burgtheater, Nationaltheater München}
         \renewcommand{\erwaehnteOrte}{Orte: München, Partenkirchen, Sternwartestraße 71, Wien}
         \renewcommand{\erwaehnteWerke}{Werke: Die Schwestern oder Casanova in Spa. Lustspiel in Versen}
               \section[Arthur Schnitzler an Hermann Bahr, 14. 9. 1918]{ Arthur Schnitzler an Hermann Bahr, 14. 9. 1918}\nopagebreak\mylabel{v}\rehead{ }\begin{ledgroupsized}[t]{13cm}\normalsize\beginnumbering\briefempfaengerindex{Bahr, Hermann@\textsc{Bahr, Hermann}!zzzSchnitzler, Arthur@\emph{von Arthur Schnitzler}!1918-09-141@{14. 9. 1918}|(be} \toendnotes[C]{\smallbreak\pagebreak[2]} \Standort{Israel, Oriel Leibzon, Privatbesitz.}
\physDesc{Briefkarte, 770 Zeichen
\newline{}Handschrift: schwarze Tinte, lateinische Kurrent
\newline{}Ordnung: Lochung, professionell repariert 
\newline{}Zusatz: Versteigerung bei Stargardt, April 2022, Lot 124 }\toendnotes[C]{\smallbreak}\pstart
           \textcolor{gray}{\textbf{{\pb}Dr. Arthur Schnitzler}}\hfill Wien\oindex{Wien@\textbf{Wien}|pw}, 14. 9. 1918\pend
           \pstart
           \textcolor{gray}{\textbf{Wien XVIII. Sternwartestrasse 71\oindex{Sternwartestrasse 71@\textbf{Sternwartestraße 71}|pw}}}\pend
           \pstart
           lieber Hermann, beigeschlossen mein neues Stück\pwindex{Schnitzler, Arthur 15.05.1862 – 21.10.1931@\textsc{Schnitzler, Arthur} (15.05.1862 – 21.10.1931), \emph{Schriftsteller, Mediziner}!Schwestern oder Casanova in Spa. Lustspiel in Versen01. 10. 1919@\strich\emph{Die Schwestern oder Casanova in Spa. Lustspiel in Versen} {[}01. 10. 1919{]}|pwv}, das ich hiemit der Direction des Burgtheater\orgindex{Burgtheater@Burgtheater|pw}s zu \uline{über}reichen mir erlaube, ohne es vorläufg \uline{ein}zureichen. Über die Gründe dieser feinen Unterscheidung reden wir, sobald
               du es gelesen hast. Zur äußeren Geschichte: Milenkovich\pwindex{Millenkovich, Max von 1866-03-02 – 1945-02-05@\textsc{Millenkovich, Max von} (1866-03-02 – 1945-02-05), \emph{Schriftsteller, Theaterleiter, Beamter}|pw} hat die Entscheidg so lange hinausgezogen, daß ich mir das Stück\pwindex{Schnitzler, Arthur 15.05.1862 – 21.10.1931@\textsc{Schnitzler, Arthur} (15.05.1862 – 21.10.1931), \emph{Schriftsteller, Mediziner}!Schwestern oder Casanova in Spa. Lustspiel in Versen01. 10. 1919@\strich\emph{Die Schwestern oder Casanova in Spa. Lustspiel in Versen} {[}01. 10. 1919{]}|pw} wieder \label{K_L01255-1v}\edtext{zurückerbat}{\lemma{\textnormal{\emph{zurückerbat}}}\Cendnote{\textnormal{siehe A. S.: \emph{Tagebuch}, 4. 3. 1918}}}\label{K_L01255-1h}. Reinhardt\pwindex{Reinhardt, Max 09.09.1873 – 30.10.1943@\textsc{Reinhardt, Max} (09.09.1873 – 30.10.1943), \emph{Theaterleiter, Regisseur, Schauspieler}|pw} führt es (\label{K_L01255-2v}\edtext{Contract}{\lemma{\textnormal{\emph{Contract}}}\Cendnote{\textnormal{Der Vertragsentwurf vom 20. 12. 1917 ist abgedruckt in: \emph{Der Briefwechsel Arthur Schnitzlers mit Max Reinhardt und
                        dessen Mitarbeitern}. Herausgegeben von Renate Wagner.
                     Salzburg: \emph{Otto Müller Verlag}{ }1971, S. 81. Die Aufführung\pwindex{Schnitzler, Arthur 15.05.1862 – 21.10.1931@\textsc{Schnitzler, Arthur} (15.05.1862 – 21.10.1931), \emph{Schriftsteller, Mediziner}!Schwestern oder Casanova in Spa. Lustspiel in Versen01. 10. 1919@\strich\emph{Die Schwestern oder Casanova in Spa. Lustspiel in Versen} {[}01. 10. 1919{]}|pwkv} von \emph{Die
                     Schwestern oder Casanova in Spa}\pwindex{Schnitzler, Arthur 15.05.1862 – 21.10.1931@\textsc{Schnitzler, Arthur} (15.05.1862 – 21.10.1931), \emph{Schriftsteller, Mediziner}!Schwestern oder Casanova in Spa. Lustspiel in Versen01. 10. 1919@\strich\emph{Die Schwestern oder Casanova in Spa. Lustspiel in Versen} {[}01. 10. 1919{]}|pwk} verzögerte sich bis zum 7. 2. 1921, dann nahm das Theater\orgindex{Burgtheater@Burgtheater|pwkv} von dem Plan einer Inszenierung
                  Abstand.}}}\label{K_L01255-2h}) bis spätestens 28. Feber 1919 auf.
               An \label{K_L01255-3v}\edtext{Franckenstein\pwindex{Franckenstein, Clemens von 14.07.1875 – 19.08.1942@\textsc{Franckenstein, Clemens von} (14.07.1875 – 19.08.1942), \emph{Theaterleiter, Komponist, Dirigent}|pw} hab ich’s {\pb}von Partenkirchen\oindex{Partenkirchen@\textbf{Partenkirchen}|pw}
               aus vor meiner Abreise (am 10. d.) gesandt}{\lemma{\textnormal{\emph{Franckenstein … gesandt}}}\Cendnote{\textnormal{Am 10. 9. 1918 reiste Schnitzler\pwindex{Schnitzler, Arthur 15.05.1862 – 21.10.1931@\textsc{Schnitzler, Arthur} (15.05.1862 – 21.10.1931), \emph{Schriftsteller, Mediziner}|pwk} von Partenkirchen\oindex{Partenkirchen@\textbf{Partenkirchen}|pwk} nach München\oindex{Muenchen@\textbf{München}|pwk}, wo Clemens von Franckenstein\pwindex{Franckenstein, Clemens von 14.07.1875 – 19.08.1942@\textsc{Franckenstein, Clemens von} (14.07.1875 – 19.08.1942), \emph{Theaterleiter, Komponist, Dirigent}|pwk} das \emph{Nationaltheater}\orgindex{Nationaltheater Muenchen@Nationaltheater München|pwk} leitete. Am 22. 9. 1918
                  telefonierte dieser Schnitzler\pwindex{Schnitzler, Arthur 15.05.1862 – 21.10.1931@\textsc{Schnitzler, Arthur} (15.05.1862 – 21.10.1931), \emph{Schriftsteller, Mediziner}|pwk} eine Absage,
                  da das Stück \emph{Die Schwestern}\pwindex{Schnitzler, Arthur 15.05.1862 – 21.10.1931@\textsc{Schnitzler, Arthur} (15.05.1862 – 21.10.1931), \emph{Schriftsteller, Mediziner}!Schwestern oder Casanova in Spa. Lustspiel in Versen01. 10. 1919@\strich\emph{Die Schwestern oder Casanova in Spa. Lustspiel in Versen} {[}01. 10. 1919{]}|pwk} für manche zu
                  anstößig wäre.}}}\label{K_L01255-3h}. Im übrigen hat noch keine Theaterleitung Einsicht \damage{in} das Mscrpt\pwindex{Schnitzler, Arthur 15.05.1862 – 21.10.1931@\textsc{Schnitzler, Arthur} (15.05.1862 – 21.10.1931), \emph{Schriftsteller, Mediziner}!Schwestern oder Casanova in Spa. Lustspiel in Versen01. 10. 1919@\strich\emph{Die Schwestern oder Casanova in Spa. Lustspiel in Versen} {[}01. 10. 1919{]}|pwv} erhalten.
               Dies sind Correcturbogen; das Buch\pwindex{Schnitzler, Arthur 15.05.1862 – 21.10.1931@\textsc{Schnitzler, Arthur} (15.05.1862 – 21.10.1931), \emph{Schriftsteller, Mediziner}!Schwestern oder Casanova in Spa. Lustspiel in Versen01. 10. 1919@\strich\emph{Die Schwestern oder Casanova in Spa. Lustspiel in Versen} {[}01. 10. 1919{]}|pwv} ist noch nicht fertig.\pend
           \pstart
           In jedem Fall freu ich mich dich bald wiederzusehen, sei es bei mir oder in Deinem
               Bureau. Grüße an Andrian\pwindex{Andrian-Werburg, Leopold von 09.05.1875 – 19.11.1951@\textsc{Andrian-Werburg, Leopold von} (09.05.1875 – 19.11.1951), \emph{Schriftsteller, Diplomat}|pw} und Michel\pwindex{Michel, Robert 24.02.1876 – 12.02.1957@\textsc{Michel, Robert} (24.02.1876 – 12.02.1957), \emph{Schriftsteller}|pw}.\pend
           \pstart
           Von Herzen Dein {\\[\baselineskip]}\spacefill\mbox{Arthur}\pend
           \leftskip=0em{}
         
         \endnumbering\mylabel{h}\end{ledgroupsized}  \newcommand{\dateiname}{L01255}\newcommand{\titel}{Arthur Schnitzler an Hermann Bahr, 14. 9. 1918}\newcommand{\editorInnen}{ Martin Anton Müller und  Laura Untner}%% latex-leseansicht-abspann.tex
%% Abspann für die Leseansicht.
%% Der Schalter \ifkorrekturansicht ist bereits durch den Vorspann gesetzt.

%% latex-abspann.tex
%% Gemeinsamer Abspann für Korrekturansicht und Leseansicht.
%% Setzt den Schalter \ifkorrekturansicht voraus (gesetzt in den
%% einbindenden Dateien latex-korrekturansicht-abspann.tex bzw.
%% latex-leseansicht-abspann.tex).
%% ---------------------------------------------------------------

\normalsize

% Das esempio-Environment wird nur in der Leseansicht benötigt
\ifkorrekturansicht\else
\newenvironment{esempio}[3]%
{
    \vspace{1.5ex}
    \rlap{\underline{#1}}
    \par
    \setlength{\parindent}{0cm}
    \nopagebreak
    \leftskip=#2cm
    \rightskip=#3cm
}
{
    \par
}
\fi

\doendnotes{C}
\bigskip
\vfill

\clearpage

\footnotesize

\ifkorrekturansicht
  \lohead{\textsc{register}}
\fi

% theindex-Environment neu definieren ohne reledmac
\makeatletter
\renewenvironment{theindex}{%
  \ifkorrekturansicht
    \section*{\indexname}%
  \else
    \subsubsection*{Index der erwähnten Entitäten}%
  \fi
  \setlength{\parindent}{0pt}%
  \setlength{\parskip}{0pt plus 0.3pt}%
  \let\item\@idxitem
}{%
  \ifkorrekturansicht\clearpage\fi
}
\makeatother

\IfFileExists{\jobname-pw.ind}{\input{\jobname-pw.ind}}{}

% Quellenangabe nur in der Leseansicht
\ifkorrekturansicht\else
% Fallback-Definitionen, falls die .tex-Datei \titel etc. nicht gesetzt hat
\providecommand{\titel}{}
\providecommand{\editorInnen}{}
\providecommand{\dateiname}{\jobname}

\vspace{3cm}

\vfill

\footnotesize
\textsc{Quelle}: \titel. Herausgegeben von {\editorInnen}. In: \emph{Arthur Schnitzler: Briefwechsel mit Autorinnen und Autoren}.
 Digitale Edition, https://schnitzler-briefe.acdh.oeaw.ac.at/{\dateiname}.html (Stand \today)
\fi

\end{document}


      