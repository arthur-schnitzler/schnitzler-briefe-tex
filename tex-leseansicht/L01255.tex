%% latex-korrekturansicht-vorspann.tex
%% Vorspann für die Korrekturansicht.
%% Lädt die gemeinsame Datei latex-vorspann.tex mit gesetztem Schalter.

\newif\ifkorrekturansicht
\korrekturansichttrue

\input{../tex-inputs/latex-vorspann}


\section[Arthur Schnitzler an Hermann Bahr, 14. 9. 1918]{L01255 Arthur Schnitzler an Hermann Bahr, 14. 9. 1918}
\nopagebreak\mylabel{L01255v}
\rehead{ }\normalsize\beginnumbering\briefempfaengerindex{Bahr, Hermann@\textsc{Bahr, Hermann}!zzzSchnitzler, Arthur@\emph{von Arthur Schnitzler}!1918-09-141@{14. 9. 1918}|(be}
\toendnotes[C]{\smallbreak\pagebreak[2]}\Standort{Israel, Oriel Leibzon, Privatbesitz.}
\physDesc{Briefkarte, 770 Zeichen
\newline{}Handschrift: schwarze Tinte, lateinische Kurrent
\newline{}Ordnung: Lochung, professionell repariert 
\newline{}Zusatz: Versteigerung bei Stargardt, April 2022, Lot 124 }\toendnotes[C]{\smallbreak}
\pstart
           
\pstart
           \textcolor{gray}{\textbf{{\pb}Dr. Arthur Schnitzler}}\pend
           
\pstart
           \raggedleft{}Wien\oindex{Wien@\textbf{Wien}, \emph{A.ADM2}|pw}, 14. 9. 1918\pend
           \pend
           
\pstart
           \textcolor{gray}{\textbf{Wien XVIII. Sternwartestrasse 71\oindex{Sternwartestrasse 71@\textbf{Sternwartestraße 71}, \emph{Wohngebäude (K.WHS)}|pw}}}\pend
           \vspace{0.5em}
\pstart
           lieber Hermann, beigeschlossen mein neues Stück\pwindex{Schwestern oder Casanova in Spa. Lustspiel in Versen@\emph{Die Schwestern oder Casanova in Spa. Lustspiel in Versen}|pwv}, das ich hiemit der Direction des Burgtheaters\orgindex{Burgtheater@Burgtheater|pw} zu \uline{über}reichen mir erlaube, ohne es vorläufg \uline{ein}zureichen. Über die Gründe dieser feinen Unterscheidung reden wir, sobald
               du es gelesen hast. Zur äußeren Geschichte: Milenkovich\pwindex{Millenkovich, Max von 1866-03-02 – 1945-02-05@\textsc{Millenkovich, Max von} (1866-03-02 – 1945-02-05), \emph{Schriftsteller/Schriftstellerin, Theaterleiter/Theaterleiterin, Beamter/Beamte}|pw} hat die Entscheidg so lange hinausgezogen, daß ich mir das Stück\pwindex{Schwestern oder Casanova in Spa. Lustspiel in Versen@\emph{Die Schwestern oder Casanova in Spa. Lustspiel in Versen}|pw} wieder \label{K_L01255-1v}\edtext{zurückerbat}{\lemma{\textnormal{\emph{zurückerbat}}}\Cendnote{\textnormal{Siehe A. S.: \emph{Tagebuch}, 4. 3. 1918.
               }}}\label{K_L01255-1}. Reinhardt\pwindex{Reinhardt, Max 09.09.1873 – 30.10.1943@\textsc{Reinhardt, Max} (09.09.1873 – 30.10.1943), \emph{Theaterleiter/Theaterleiterin, Regisseur/Regisseurin, Schauspieler/Schauspielerin}|pw} führt es (\label{K_L01255-2v}\edtext{Contract}{\lemma{\textnormal{\emph{Contract}}}\Cendnote{\textnormal{Der Vertragsentwurf vom 20. 12. 1917 ist abgedruckt in: \emph{Der Briefwechsel Arthur Schnitzlers mit Max Reinhardt und
                        dessen Mitarbeitern}. Herausgegeben von Renate Wagner.
                     Salzburg: \emph{Otto Müller Verlag}{ }1971, S. 81. Die Aufführung\pwindex{Schwestern oder Casanova in Spa. Lustspiel in Versen@\emph{Die Schwestern oder Casanova in Spa. Lustspiel in Versen}|pwkv} von \emph{Die
                     Schwestern oder Casanova in Spa}\pwindex{Schwestern oder Casanova in Spa. Lustspiel in Versen@\emph{Die Schwestern oder Casanova in Spa. Lustspiel in Versen}|pwk} verzögerte sich bis zum 7. 2. 1921, dann nahm das Theater\orgindex{Burgtheater@Burgtheater|pwkv} von dem Plan einer Inszenierung
                  Abstand.}}}\label{K_L01255-2}) bis spätestens 28. Feber 1919 auf.
               An \label{K_L01255-3v}\edtext{Franckenstein\pwindex{Franckenstein, Clemens von 14.07.1875 – 19.08.1942@\textsc{Franckenstein, Clemens von} (14.07.1875 – 19.08.1942), \emph{Theaterleiter/Theaterleiterin, Komponist/Komponistin, Dirigent/Dirigentin}|pw} hab ich’s {\pb}von Partenkirchen\oindex{Partenkirchen@\textbf{Partenkirchen}, \emph{Teil eines besiedelten Ortes (A.BSOX)}|pw}
               aus vor meiner Abreise (am 10. d.) gesandt}{\lemma{\textnormal{\emph{Franckenstein … gesandt}}}\Cendnote{\textnormal{Am 10. 9. 1918 reiste Schnitzler von Partenkirchen\oindex{Partenkirchen@\textbf{Partenkirchen}, \emph{Teil eines besiedelten Ortes (A.BSOX)}|pwk} nach München\oindex{Muenchen@\textbf{München}, \emph{P.PPLA}|pwk}, wo Clemens von Franckenstein\pwindex{Franckenstein, Clemens von 14.07.1875 – 19.08.1942@\textsc{Franckenstein, Clemens von} (14.07.1875 – 19.08.1942), \emph{Theaterleiter/Theaterleiterin, Komponist/Komponistin, Dirigent/Dirigentin}|pwk} das \emph{Nationaltheater}\orgindex{Nationaltheater Muenchen@Nationaltheater München|pwk} leitete. Am 22. 9. 1918
                  telefonierte dieser Schnitzler eine Absage,
                  da das Stück \emph{Die Schwestern}\pwindex{Schwestern oder Casanova in Spa. Lustspiel in Versen@\emph{Die Schwestern oder Casanova in Spa. Lustspiel in Versen}|pwk} für manche zu
                  anstößig wäre.}}}\label{K_L01255-3}. Im übrigen hat noch keine Theaterleitung Einsicht \damage{in} das Mscrpt\pwindex{Schwestern oder Casanova in Spa. Lustspiel in Versen@\emph{Die Schwestern oder Casanova in Spa. Lustspiel in Versen}|pwv} erhalten.
               Dies sind Correcturbogen; das Buch\pwindex{Schwestern oder Casanova in Spa. Lustspiel in Versen@\emph{Die Schwestern oder Casanova in Spa. Lustspiel in Versen}|pwv} ist noch nicht fertig.\pend
           
\pstart
           In jedem Fall freu ich mich dich bald wiederzusehen, sei es bei mir oder in Deinem
               Bureau. Grüße an Andrian\pwindex{Andrian-Werburg, Leopold von 09.05.1875 – 19.11.1951@\textsc{Andrian-Werburg, Leopold von} (09.05.1875 – 19.11.1951), \emph{Schriftsteller/Schriftstellerin, Diplomat/Diplomatin}|pw} und Michel\pwindex{Michel, Robert 24.02.1876 – 12.02.1957@\textsc{Michel, Robert} (24.02.1876 – 12.02.1957), \emph{Schriftsteller/Schriftstellerin, Offizier/Offizierin, Krimiautor/Krimiautorin}|pw}.\pend
           
\pstart
           Von Herzen Dein {\\[\baselineskip]}\spacefill\mbox{Arthur}\pend
           \leftskip=0em{}\selectlanguage{ngerman}\endnumbering\briefempfaengerindex{Bahr, Hermann@\textsc{Bahr, Hermann}!zzzSchnitzler, Arthur@\emph{von Arthur Schnitzler}!1918-09-141@{14. 9. 1918}|)be}\mylabel{L01255h}  \normalsize

\doendnotes{C}
\bigskip
\vfill

\clearpage

\footnotesize

\lohead{\textsc{register}}

% Definiere theindex-Environment komplett neu ohne reledmac
\makeatletter
\renewenvironment{theindex}{%
  \section*{\indexname}%
  \setlength{\parindent}{0pt}%
  \setlength{\parskip}{0pt plus 0.3pt}%
  \let\item\@idxitem
}{%
  \clearpage
}
\makeatother

\IfFileExists{\jobname-pw.ind}{\input{\jobname-pw.ind}}{}

\end{document}

      