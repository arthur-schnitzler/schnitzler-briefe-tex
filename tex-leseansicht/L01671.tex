%% latex-korrekturansicht-vorspann.tex
%% Vorspann für die Korrekturansicht.
%% Lädt die gemeinsame Datei latex-vorspann.tex mit gesetztem Schalter.

\newif\ifkorrekturansicht
\korrekturansichttrue

\input{../tex-inputs/latex-vorspann}


\section[Hermann Bahr an Arthur Schnitzler, 27. 4. 1907]{L01671 Hermann Bahr an Arthur Schnitzler, 27. 4. 1907}
\nopagebreak\mylabel{L01671v}
\rehead{ }\normalsize\beginnumbering\briefempfaengerindex{Schnitzler, Arthur@\textsc{Schnitzler, Arthur}!zzzBahr, Hermann@\emph{von Hermann Bahr}!1907-04-271@{27. 4. 1907}|(be}
\toendnotes[C]{\smallbreak\pagebreak[2]}\Standort{CUL, Schnitzler, B 5b.}
\physDesc{Kartenbrief, 259 Zeichen
\newline{}Handschrift: 1) Bleistift, deutsche Kurrent\hspace{1em}2) Bleistift, lateinische Kurrent (\noindent{}Adresse)\hspace{1em}
\newline{}Versand: 1) Stempel: »\nobreak{}\oindex{X., Favoriten@\textbf{X., Favoriten}, \emph{A.ADM3}|pwk}Wien 10/2, 27. IV. 07, 8\nobreak{}«.   2) Stempel: »\nobreak{}\oindex{XVIII., Waehring@\textbf{XVIII., Währing}, \emph{A.ADM3}|pwk}18/1 Wien, 28. IV. 07\nobreak{}«. 
\newline{}Ordnung: mit Bleistift von unbekannter Hand nummeriert:
                                    »149« }
\buchAbdrucke{\weitereDrucke{Hermann Bahr, Arthur Schnitzler: \emph{Briefwechsel, Aufzeichnungen, Dokumente (1891–1931)}. Göttingen: \emph{Wallstein} 2018, S. 392.} }\toendnotes[C]{\smallbreak}\pstart{}{\pb}D\textsuperscript{r} Artur
                  Schnitzler\pend{}\pstart{}Wien XVIII\oindex{XVIII., Waehring@\textbf{XVIII., Währing}, \emph{A.ADM3}|pw}\pend{}\pstart{}Spöttelgasse 7\oindex{Edmund-Weiss-Gasse 7@\textbf{Edmund-Weiß-Gasse 7}, \emph{Wohngebäude (K.WHS)}|pw}\pend{}{\bigskip}\vspace{1em}
\pstart
           \raggedleft{}{\pb}Südbahnhof\oindex{Suedbahnhof@\textbf{Südbahnhof}, \emph{Bahnhofsgebäude (K.BHF)}|pw}{\\}27. 4\pend
           \vspace{0.5em}
\pstart
           Ich fahre eben auf den Semmering (Südbahnhotel)\oindex{Suedbahnhotel [Semmering]@\textbf{Südbahnhotel [Semmering]}, \emph{Hotel (K.HTL)}|pw},
               um \label{K_L01671-1v}\edtext{Dienſtag}{\lemma{\textnormal{\emph{Dienſtag}}}\Cendnote{\textnormal{den 30. 4. 1907}}}\label{K_L01671-1} Früh von dort mit dem Frühſchnellzug nach Trieſt\oindex{Triest@\textbf{Triest}, \emph{A.ADM3}|pw} zu fahren. \label{K_L01671-2v}\edtext{Komm
                  mit!}{\lemma{\textnormal{\emph{Komm
                  mit!}}}\Cendnote{\textnormal{Einen inhaltlich und sprachlich
                  beinahe identischen Brief schrieb Bahr\pwindex{Bahr, Hermann 19.07.1863 – 15.01.1934@\textsc{Bahr, Hermann} (19.07.1863 – 15.01.1934), \emph{Schriftsteller/Schriftstellerin, Kritiker/Kritikerin}|pwk} am
                  selben Tag an Beer-Hofmann\pwindex{Beer-Hofmann, Richard 1866-07-11 – 1945-09-26@\textsc{Beer-Hofmann, Richard} (1866-07-11 – 1945-09-26), \emph{Schriftsteller/Schriftstellerin}|pwk}.}}}\label{K_L01671-2} In
               dieſem Fall erwartet bis Montag Mittag telegrafiſche Nachricht\pend
           
\pstart
           Dein{\\[\baselineskip]}\spacefill\mbox{Hermann B}\pend
           \leftskip=0em{}\selectlanguage{ngerman}\endnumbering\briefempfaengerindex{Schnitzler, Arthur@\textsc{Schnitzler, Arthur}!zzzBahr, Hermann@\emph{von Hermann Bahr}!1907-04-271@{27. 4. 1907}|)be}\mylabel{L01671h}  \normalsize

\doendnotes{C}
\bigskip
\vfill

\clearpage

\footnotesize

\lohead{\textsc{register}}

% Definiere theindex-Environment komplett neu ohne reledmac
\makeatletter
\renewenvironment{theindex}{%
  \section*{\indexname}%
  \setlength{\parindent}{0pt}%
  \setlength{\parskip}{0pt plus 0.3pt}%
  \let\item\@idxitem
}{%
  \clearpage
}
\makeatother

\IfFileExists{\jobname-pw.ind}{\input{\jobname-pw.ind}}{}

\end{document}

      