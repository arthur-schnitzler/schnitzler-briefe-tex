%% latex-leseansicht-vorspann.tex
%% Vorspann für die Leseansicht.
%% Lädt die gemeinsame Datei latex-vorspann.tex mit nicht gesetztem Schalter.

\newif\ifkorrekturansicht
\korrekturansichtfalse

\input{../tex-inputs/latex-vorspann}


         
         \newcommand{\erwaehntePersonen}{Personen: Hermann Bahr, Clementine Goldmann, Leopold Sonnemann}
         \newcommand{\erwaehnteInstitutionen}{Institutionen: Frankfurter Zeitung}
         \newcommand{\erwaehnteOrte}{Orte: Bad Ischl, Frankfurt am Main, Paris, Rossertstraße, Salzkammergut, Wien, rue de la Bourse, Österreich}
         \newcommand{\erwaehnteWerke}{Werke: Leander. Novelle}
               \section[ Paul Goldmann an Arthur Schnitzler, 1. 8. {[}1897{]}]{ Paul Goldmann an Arthur Schnitzler, 1. 8. {[}1897{]}}\nopagebreak\mylabel{v}\rehead{ }\begin{ledgroupsized}[t]{13cm}\normalsize\beginnumbering \toendnotes[C]{\smallbreak\pagebreak[2]} \Standort{DLA, A:Schnitzler, HS.NZ85.1.3167.}
\physDesc{Brief, 1 Blatt, 2 Seiten
\newline{}Handschrift: blaue Tinte, deutsche Kurrent
\newline{}Schnitzler: mit schwarzer Tinte das Jahr »97.« vermerkt }\toendnotes[C]{\smallbreak}\pstart
           \noindent{}{\pb}\textcolor{gray}{\textbf{\textbf{Frankfurter Zeitung\orgindex{Frankfurter Zeitung@Frankfurter Zeitung|pw}}}}\pend
           \pstart
           \textcolor{gray}{\textbf{(\begin{otherlanguage}{french}Gazette de Francfort\end{otherlanguage}\orgindex{Frankfurter Zeitung@Frankfurter Zeitung|pw}).}}\pend
           \pstart
           \textcolor{gray}{\textbf{\textbf{\begin{otherlanguage}{french}Fondateur M.\end{otherlanguage}{ }L. Sonnemann\pwindex{Sonnemann, Leopold 1831-10-29 – 1909-10-30@\textsc{Sonnemann, Leopold} (1831-10-29 – 1909-10-30), \emph{Journalist, Herausgeber}|pw}.}}}\pend
           \pstart
           \begin{otherlanguage}{french}\textcolor{gray}{\textbf{Journal politique, financier,}}\end{otherlanguage}\pend
           \pstart
           \begin{otherlanguage}{french}\textcolor{gray}{\textbf{commercial et littéraire.}}\end{otherlanguage}\pend
           \pstart
           \begin{otherlanguage}{french}\textcolor{gray}{\textbf{\textbf{Paraissant trois fois par jour.}}}\end{otherlanguage}\hfill \textsc{Paris\oindex{Paris@\textbf{Paris}|pw}}, 1. Auguſt.\pend
           \pstart
           \begin{otherlanguage}{french}\textcolor{gray}{\textbf{\textbf{Bureau à Paris\oindex{Paris@\textbf{Paris}|pw}}}}\end{otherlanguage}\pend
           \pstart
           \begin{otherlanguage}{french}\textcolor{gray}{\textbf{\textbf{10 \so{Rue de la Bourse}\oindex{rue de la Bourse@\textbf{rue de la Bourse}|pw}.}}}\end{otherlanguage}\pend
           \pstart\center{}Mein lieber Freund,\pend\pstart
           Mittwoch oder Donnerſtag fahre ich von hier\oindex{Paris@\textbf{Paris}|pwv} fort nach Frankfurt\oindex{Frankfurt am Main@\textbf{Frankfurt am Main}|pw} (wenn nichts
               dazwiſchen kommt). Ich bitte Dich, mir \uline{ſofort} nach
               Empfang dieſes Briefes an die Adreſſe meiner Mutter\pwindex{Goldmann, Clementine 1842-05-15 – 1924-02-24@\textsc{Goldmann, Clementine} (1842-05-15 – 1924-02-24)|pwv} (\textsc{Frau Clementine Goldmann\pwindex{Goldmann, Clementine 1842-05-15 – 1924-02-24@\textsc{Goldmann, Clementine} (1842-05-15 – 1924-02-24)|pw}}, \textsc{Frankfurt a. Main, Rossertstraße 15\oindex{Rossertstrasse@\textbf{Rossertstraße}|pw}}) zu ſchreiben, ob die \label{K_L02820-1v}\edtext{Überſchwemmungen in \textsc{Ischl\oindex{Bad Ischl@\textbf{Bad Ischl}|pw}}}{\lemma{\textnormal{\emph{Überſchwemmungen in Ischl}}}\Cendnote{\textnormal{Zwischen Ende Juli und Anfang August 1897 kam es im Salzkammergut\oindex{Salzkammergut@\textbf{Salzkammergut}|pwk}, aber auch in ganz Österreich\oindex{Oesterreich@\textbf{Österreich}|pwk} zu schweren Überschwemmungen. Hermann Bahr\pwindex{Bahr, Hermann 19.07.1863 – 15.01.1934@\textsc{Bahr, Hermann} (19.07.1863 – 15.01.1934), \emph{Schriftsteller, Kritiker}|pwk} verfasste eine literarische
                  Bearbeitung einer Reise durch die Hochwassergebiete mit der Novelle \emph{Leander}\pwindex{Bahr, Hermann 19.07.1863 – 15.01.1934@\textsc{Bahr, Hermann} (19.07.1863 – 15.01.1934), \emph{Schriftsteller, Kritiker}!Leander. Novelle1899-04-01 – 1899-05-06@\strich\emph{Leander. Novelle} {[}1899-04-01 – 1899-05-06{]}|pwk}.}}}\label{K_L02820-1h} nichts an unſerem Programm
               ändern oder ob wir uns unter dieſen Umſtänden vielleicht anderswo treffen müſſen? Es
               iſt für mich ſehr wichtig, dies \uline{bald} zu erfahren, da
               ich mir von Frankfurt\oindex{Frankfurt am Main@\textbf{Frankfurt am Main}|pw}{ }{\pb}aus ein Rundreiſe-Billet nehmen muß.\pend
           \pstart
           Ich freue mich unendlich darauf, Dich bald wiederzuſehen.\pend
           \pstart
           Viele treue Grüße\textcolor{gray}{!}\pend
           \pstart
           Dein {\\[\baselineskip]}\spacefill\mbox{Paul Goldm}\pend
           \leftskip=0em{}
         
         \endnumbering\mylabel{h}\end{ledgroupsized}  \newcommand{\dateiname}{L02820}\newcommand{\titel}{Paul Goldmann an Arthur Schnitzler, 1. 8. [1897]}\newcommand{\editorInnen}{Martin Anton Müller und Laura Untner}%% latex-leseansicht-abspann.tex
%% Abspann für die Leseansicht.
%% Der Schalter \ifkorrekturansicht ist bereits durch den Vorspann gesetzt.

%% latex-abspann.tex
%% Gemeinsamer Abspann für Korrekturansicht und Leseansicht.
%% Setzt den Schalter \ifkorrekturansicht voraus (gesetzt in den
%% einbindenden Dateien latex-korrekturansicht-abspann.tex bzw.
%% latex-leseansicht-abspann.tex).
%% ---------------------------------------------------------------

\normalsize

% Das esempio-Environment wird nur in der Leseansicht benötigt
\ifkorrekturansicht\else
\newenvironment{esempio}[3]%
{
    \vspace{1.5ex}
    \rlap{\underline{#1}}
    \par
    \setlength{\parindent}{0cm}
    \nopagebreak
    \leftskip=#2cm
    \rightskip=#3cm
}
{
    \par
}
\fi

\doendnotes{C}
\bigskip
\vfill

\clearpage

\footnotesize

\ifkorrekturansicht
  \lohead{\textsc{register}}
\fi

% theindex-Environment neu definieren ohne reledmac
\makeatletter
\renewenvironment{theindex}{%
  \ifkorrekturansicht
    \section*{\indexname}%
  \else
    \subsubsection*{Index der erwähnten Entitäten}%
  \fi
  \setlength{\parindent}{0pt}%
  \setlength{\parskip}{0pt plus 0.3pt}%
  \let\item\@idxitem
}{%
  \ifkorrekturansicht\clearpage\fi
}
\makeatother

\IfFileExists{\jobname-pw.ind}{\input{\jobname-pw.ind}}{}

% Quellenangabe nur in der Leseansicht
\ifkorrekturansicht\else
% Fallback-Definitionen, falls die .tex-Datei \titel etc. nicht gesetzt hat
\providecommand{\titel}{}
\providecommand{\editorInnen}{}
\providecommand{\dateiname}{\jobname}

\vspace{3cm}

\vfill

\footnotesize
\textsc{Quelle}: \titel. Herausgegeben von {\editorInnen}. In: \emph{Arthur Schnitzler: Briefwechsel mit Autorinnen und Autoren}.
 Digitale Edition, https://schnitzler-briefe.acdh.oeaw.ac.at/{\dateiname}.html (Stand \today)
\fi

\end{document}


      