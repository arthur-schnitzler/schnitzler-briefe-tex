%% latex-leseansicht-vorspann.tex
%% Vorspann für die Leseansicht.
%% Lädt die gemeinsame Datei latex-vorspann.tex mit nicht gesetztem Schalter.

\newif\ifkorrekturansicht
\korrekturansichtfalse

\input{../tex-inputs/latex-vorspann}


\section[Friedrich M. Fels an Arthur Schnitzler, 1{[}6{]}. 2. 1893]{L00176 Friedrich M. Fels an Arthur Schnitzler, 1[6]. 2. 1893}
\nopagebreak\mylabel{L00176v}
\rehead{ }\normalsize\beginnumbering\briefempfaengerindex{Schnitzler, Arthur@\textsc{Schnitzler, Arthur}!zzzFels, Friedrich Michael@\emph{von Friedrich Michael Fels}!1893-02-161@{1[6]. 2. 1893}|(be}
\toendnotes[C]{\smallbreak\pagebreak[2]}
\correspDesc{Versand  durch Friedrich M. Fels am 1[6]. 2. 1893 in Meran
\newline{}Erhalt  durch Arthur Schnitzler am 18. 2. 1893 in Wien}\toendnotes[C]{\smallbreak}
\Standort{DLA, A:Schnitzler, HS.NZ85.1.2956.}
\physDesc{Brief, 1 Blatt, 3 Seiten, 2957 Zeichen
\newline{}Handschrift: schwarze Tinte, lateinische Kurrent
\newline{}Schnitzler: mit Bleistift nummeriert: »8« und unterhalb der
                                 Datumsangabe klein »16« vermerkt }\toendnotes[C]{\smallbreak}
\pstart
           \raggedleft{}{\pb}Meran-Obermais, Hotel Erzherzog Rainer\oindex{Hotel Erzherzog Rainer [Meran]@\textbf{Hotel Erzherzog Rainer [Meran]}, \emph{Hotel}|pw}{\\}\label{K_L00176-1v}\edtext{18. Februar 1893}{\lemma{\textnormal{\emph{18. Februar 1893}}}\Cendnote{\textnormal{Obzwar eindeutig auf den
                        18. datiert, geht aus dem Korrespondenzstück Schnitzlers an Hofmannsthal\pwindex{Hofmannsthal, Hugo von 1.\,2.\,1874 Wien – 15.\,7.\,1929 Rodaun@\textsc{Hofmannsthal, Hugo von} (1.\,2.\,1874 Wien – 15.\,7.\,1929 Rodaun), \emph{Schriftsteller}|pwk} hervor, dass er an diesem Tag bereits in Wien\oindex{Wien@\textbf{Wien}, \emph{Verwaltungsgebiet}|pwk} war.}}}\label{K_L00176-1}.\pend
           
\pstart{}Lieber Dr. Schnitzler!\pend\vspace{0.5em}
\pstart
           Verzeihen Sie, daſs ich Ihnen heute erst schreibe; aber erst gestern hat sich
               entschieden, wo ich wohne, – und ich bin i{\geminationm}er so müde!
               Aber ich will der Reihe nach erzählen.\pend
           
\pstart
           Die Fahrt war furchtbar ermüdend: zum Mittageſsen in Franzensfeste\oindex{Franzensfeste@\textbf{Franzensfeste}, \emph{Verwaltungsgebiet}|pw} 20 Minuten Aufenthalt, in Villach\oindex{Villach@\textbf{Villach}, \emph{Verwaltungsgebiet}|pw} 15 – das war alles. Zum Glück hatte ich verhältnismäſsig angenehme
               Gesellschaft, darunter Dr. Rullma{\geminationn}\pwindex{Rullmann, Wilhelm 10.\,12.\,1841 Bieber – 7.\,10.\,1918 Schlüchtern@\textsc{Rullmann, Wilhelm} (10.\,12.\,1841 Bieber – 7.\,10.\,1918 Schlüchtern), \emph{Schriftsteller, Redakteur}|pw}, den Redakteur des \label{K_L00176-2v}\edtext{Grazer Tagblatts\orgindex{Grazer Tagblatt@Grazer Tagblatt|pw}\orgindex{Tagespost@Tagespost|pwv}}{\lemma{\textnormal{\emph{Grazer Tagblatts}}}\Cendnote{\textnormal{Dies ist falsch, Wilhelm Rullmann\pwindex{Rullmann, Wilhelm 10.\,12.\,1841 Bieber – 7.\,10.\,1918 Schlüchtern@\textsc{Rullmann, Wilhelm} (10.\,12.\,1841 Bieber – 7.\,10.\,1918 Schlüchtern), \emph{Schriftsteller, Redakteur}|pwk} arbeitete für
                  die \emph{Grazer Tagespost}\orgindex{Tagespost@Tagespost|pwk}.}}}\label{K_L00176-2}. Er lebt jetzt
               auch hier, wohnt aber unten in der Stadt\oindex{Meran@\textbf{Meran}, \emph{Hauptstadt}|pwv}.\pend
           
\pstart
           Dr. Schreiber\pwindex{Schreiber, Joseph 17.\,3.\,1835 Česká Lípa – 28.\,9.\,1908 Bad Aussee@\textsc{Schreiber, Joseph} (17.\,3.\,1835 Česká Lípa – 28.\,9.\,1908 Bad Aussee), \emph{Mediziner, Sanatoriumsleiter, Arzt}|pw}{ }ſa{\geminationm}t Gemahlin\pwindex{Schreiber, Clara 27.\,10.\,1848 Wien – 8.\,2.\,1905 Meran@\textsc{Schreiber, Clara} (27.\,10.\,1848 Wien – 8.\,2.\,1905 Meran), \emph{Schriftstellerin}|pwv} haben mich äuſserst freundlich und
               liebenswürdig empfangen; letztere läſst bestens danken. Sehr unangenehm aber waren
               die Eröffnungen, die mir ihr Herr Gemahl\pwindex{Schreiber, Joseph 17.\,3.\,1835 Česká Lípa – 28.\,9.\,1908 Bad Aussee@\textsc{Schreiber, Joseph} (17.\,3.\,1835 Česká Lípa – 28.\,9.\,1908 Bad Aussee), \emph{Mediziner, Sanatoriumsleiter, Arzt}|pwv} machte. Nachdem er konstatiert hatte, daſs ich im höchsten Grad
               anämisch sei, erklärte er mir rund heraus, von einer Heilung bi{\geminationn}en 4 Wochen – ich getraute mich gar nicht mehr, von
               16 Tagen zu sprechen – kö{\geminationn}e überhaupt nicht die Rede
               sein; \uline{vor 15. Mai}{ }{\pb}\uline{d. h. vor 3 Monaten} kö{\geminationn}e
               er mich nicht entlaſsen. Dabei sagte er nicht etwa: We{\geminationn}
               Sie früher fortgehen, werden Sie später die Folgen zu spüren haben – o nein! sondern
               ganz einfach: »Sie werden vor 3 Monaten nicht arbeitsfähig sein!« Das ist doch ein
               Argument, das zieht.\pend
           
\pstart
           Sehen Sie, lieber Dr., ich hatte Recht, als ich meinte, es sei fertig mit mir. Die
               Aussichten auf die deutsche Zeitung\orgindex{Deutsche Zeitung@Deutsche Zeitung|pw}{ }ſind doch entschieden vorbei, und auch die Kunstchronik\orgindex{Allgemeine Kunst-Chronik@Allgemeine Kunst-Chronik|pw} wird bei einer so langen Abwesenheit
               verloren sein. Also stehe ich, we{\geminationn} ich nach Wien\oindex{Wien@\textbf{Wien}, \emph{Verwaltungsgebiet}|pw} ko{\geminationm}e, wieder ohne
               jede Einnahme da, der Mildthätigkeit überlaſsen. – Auf der andern Seite sehe ich
               absolut nicht ein, wie so lange den Aufenthalt in Meran\oindex{Meran@\textbf{Meran}, \emph{Hauptstadt}|pw} bestreiten. Die Pension im Hotel ohne Wein, Licht und Heizung beträgt
               3 fl (ich habe, als Journalist, von den üblichen 4 fl einen abgehandelt. Alle Leute,
               auch Dr. Schreiber\pwindex{Schreiber, Joseph 17.\,3.\,1835 Česká Lípa – 28.\,9.\,1908 Bad Aussee@\textsc{Schreiber, Joseph} (17.\,3.\,1835 Česká Lípa – 28.\,9.\,1908 Bad Aussee), \emph{Mediziner, Sanatoriumsleiter, Arzt}|pw}, haben mir zum Hotel
               geraten, weil ich hier Gesellschaft und mehr Anregung finde als im Privatquartier;
               auch sei’s nicht teuerer); da ich absolut nicht gehen ka{\geminationn} und \uline{darf}, muſs ich mir jeden Tag einen Rollwagen
               nehmen, der fl 1.–1.20 kostet; nehmen Sie dazu Wein, Licht, Heizung, Cigarren etc –
               so kö{\geminationn}en Sie sich ungefähr einen Begriff von den
               Ausgaben machen. Dagegen werde ich noch einnehmen:\pend
           
\pstart
           {\pb}1) die Su{\geminationm}e, die
               Sie so gütig waren, mir zu versprechen\pend
           
\pstart
           2) das Ergebnis zweier Sa{\geminationm}lungen, die Steinbach\pwindex{Steinbach, Josef 3.\,1.\,1850 Pécs – 1927@\textsc{Steinbach, Josef} (3.\,1.\,1850 Pécs – 1927), \emph{Schriftsteller, Mediziner, Übersetzer}|pw} bei der Neuen Freien Preſse\orgindex{Neue Freie Presse@Neue Freie Presse|pw} und Gelber\pwindex{Gelber, Ludwig 9.\,11.\,1865 Podhajce – 21.\,5.\,1931 Wien@\textsc{Gelber, Ludwig} (9.\,11.\,1865 Podhajce – 21.\,5.\,1931 Wien), \emph{Rechtsanwalt}|pw} beim Neuen Tagblatt\orgindex{Neues Wiener Tagblatt@Neues Wiener Tagblatt|pw} veranstalten
               werden (we{\geminationn}{ }ſie es thun!)\pend
           
\pstart
           3) eine Unterstützung von je 50 fl, die ich vielleicht! von der Concordia\orgindex{Concordia. Journalisten- und Schriftstellerverein@Concordia. Journalisten- und Schriftstellerverein|pw} und von der Schillerstiftung\orgindex{Deutsche Schillerstiftung@Deutsche Schillerstiftung|pw} erhalte. – Das ist zwar viel, aber es reicht doch
               nicht. – –\pend
           
\pstart
           Jetzt leben Sie wol – meine Hand ist müde, und Sie wiſsen alles Wichtige – und seien
               Sie nebst Beer-Hofma{\geminationn}\pwindex{Beer-Hofmann, Richard 11.\,7.\,1866 Wien – 26.\,9.\,1945 New York City@\textsc{Beer-Hofmann, Richard} (11.\,7.\,1866 Wien – 26.\,9.\,1945 New York City), \emph{Schriftsteller}|pw}, Loris\pwindex{Hofmannsthal, Hugo von 1.\,2.\,1874 Wien – 15.\,7.\,1929 Rodaun@\textsc{Hofmannsthal, Hugo von} (1.\,2.\,1874 Wien – 15.\,7.\,1929 Rodaun), \emph{Schriftsteller}|pw} und den andern herzlich gegrüſst
               von\pend
           
\pstart
           Ihrem{\\[\baselineskip]}\spacefill\mbox{Fels}\pend
           \leftskip=0em{}
\pstart
           \noindent{}Für wie schwach mich Schreiber\pwindex{Schreiber, Joseph 17.\,3.\,1835 Česká Lípa – 28.\,9.\,1908 Bad Aussee@\textsc{Schreiber, Joseph} (17.\,3.\,1835 Česká Lípa – 28.\,9.\,1908 Bad Aussee), \emph{Mediziner, Sanatoriumsleiter, Arzt}|pw} erklärt,
                     kö{\geminationn}en Sie aus meiner Kurvorschrift ersehen:\pend
           
\pstart
           1) ¼ Ltr Milch mit 1 Kaffeelöffel Cognac 4mal tägl.\pend
           
\pstart
           2) Waschung 27°, Halbbad 26° mit \uline{sanften}
                  Frottierungen und Übergieſsungen. »Man ka{\geminationn} ja mit
                  Ihnen nichts anfangen.«\pend
           \selectlanguage{ngerman}\endnumbering\briefempfaengerindex{Schnitzler, Arthur@\textsc{Schnitzler, Arthur}!zzzFels, Friedrich Michael@\emph{von Friedrich Michael Fels}!1893-02-161@{1[6]. 2. 1893}|)be}\mylabel{L00176h}  \newcommand{\dateiname}{L00176}\newcommand{\titel}{Friedrich M. Fels an Arthur Schnitzler, 1[6]. 2. 1893}\newcommand{\editorInnen}{Martin Anton Müller und Gerd-Hermann Susen}%% latex-leseansicht-abspann.tex
%% Abspann für die Leseansicht.
%% Der Schalter \ifkorrekturansicht ist bereits durch den Vorspann gesetzt.

%% latex-abspann.tex
%% Gemeinsamer Abspann für Korrekturansicht und Leseansicht.
%% Setzt den Schalter \ifkorrekturansicht voraus (gesetzt in den
%% einbindenden Dateien latex-korrekturansicht-abspann.tex bzw.
%% latex-leseansicht-abspann.tex).
%% ---------------------------------------------------------------

\normalsize

% Das esempio-Environment wird nur in der Leseansicht benötigt
\ifkorrekturansicht\else
\newenvironment{esempio}[3]%
{
    \vspace{1.5ex}
    \rlap{\underline{#1}}
    \par
    \setlength{\parindent}{0cm}
    \nopagebreak
    \leftskip=#2cm
    \rightskip=#3cm
}
{
    \par
}
\fi

\doendnotes{C}
\bigskip
\vfill

\clearpage

\footnotesize

\ifkorrekturansicht
  \lohead{\textsc{register}}
\fi

% theindex-Environment neu definieren ohne reledmac
\makeatletter
\renewenvironment{theindex}{%
  \ifkorrekturansicht
    \section*{\indexname}%
  \else
    \subsubsection*{Index der erwähnten Entitäten}%
  \fi
  \setlength{\parindent}{0pt}%
  \setlength{\parskip}{0pt plus 0.3pt}%
  \let\item\@idxitem
}{%
  \ifkorrekturansicht\clearpage\fi
}
\makeatother

\IfFileExists{\jobname-pw.ind}{\input{\jobname-pw.ind}}{}

% Quellenangabe nur in der Leseansicht
\ifkorrekturansicht\else
% Fallback-Definitionen, falls die .tex-Datei \titel etc. nicht gesetzt hat
\providecommand{\titel}{}
\providecommand{\editorInnen}{}
\providecommand{\dateiname}{\jobname}

\vspace{3cm}

\vfill

\footnotesize
\textsc{Quelle}: \titel. Herausgegeben von {\editorInnen}. In: \emph{Arthur Schnitzler: Briefwechsel mit Autorinnen und Autoren}.
 Digitale Edition, https://schnitzler-briefe.acdh.oeaw.ac.at/{\dateiname}.html (Stand \today)
\fi

\end{document}


