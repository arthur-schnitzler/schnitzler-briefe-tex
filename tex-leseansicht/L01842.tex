%% latex-korrekturansicht-vorspann.tex
%% Vorspann für die Korrekturansicht.
%% Lädt die gemeinsame Datei latex-vorspann.tex mit gesetztem Schalter.

\newif\ifkorrekturansicht
\korrekturansichttrue

\input{../tex-inputs/latex-vorspann}


\section[Hermann Bahr an Arthur Schnitzler, 10. 5. 1909]{L01842 Hermann Bahr an Arthur Schnitzler, 10. 5. 1909}
\nopagebreak\mylabel{L01842v}
\rehead{ }\normalsize\beginnumbering\briefempfaengerindex{Schnitzler, Arthur@\textsc{Schnitzler, Arthur}!zzzBahr, Hermann@\emph{von Hermann Bahr}!1909-05-101@{10. 5. 1909}|(be}
\toendnotes[C]{\smallbreak\pagebreak[2]}\Standort{CUL, Schnitzler, B 5b.}
\physDesc{Brief, 1 Blatt, 1 Seite, 216 Zeichen
\newline{}Handschrift: blaue Tinte, deutsche Kurrent
\newline{}Ordnung: mit Bleistift von unbekannter Hand nummeriert:
                                    »156« }
\buchAbdrucke{\weitereDrucke{Hermann Bahr, Arthur Schnitzler: \emph{Briefwechsel, Aufzeichnungen, Dokumente (1891–1931)}. Göttingen: \emph{Wallstein} 2018, S. 416.} }\toendnotes[C]{\smallbreak}
\pstart
           \raggedleft{}{\pb}10. 5. 09\pend
           
\pstart\center{}Lieber Arthur!\pend\vspace{0.5em}
\pstart
           Laß Dir den jungen Dichter\pwindex{Nagy, Gerhard von *~20.10.1886@\textsc{Nagy, Gerhard von} (*~20.10.1886), \emph{Philosoph/Philosophin}|pwv},
               der Dir dieſe Zeilen bringt, herzlichſt empfolen ſein wird von \pend
           
\pstart
           Deinem alten{\\[\baselineskip]}\spacefill\mbox{Hermann Bahr}\pend
           \leftskip=0em{}
\pstart
           \noindent{}Und die schönſten Grüße an Frau Olga\pwindex{Schnitzler, Olga 17.01.1882 – 13.01.1970@\textsc{Schnitzler, Olga} (17.01.1882 – 13.01.1970), \emph{Schauspieler/Schauspielerin, Sänger/Sängerin}|pw}!\pend
           
\pstart
           Herrn \textsc{D\textsuperscript{r} Arthur Schnitzler XVIII Spöttelgasse 7\oindex{Edmund-Weiss-Gasse 7@\textbf{Edmund-Weiß-Gasse 7}, \emph{Wohngebäude (K.WHS)}|pw}}\pend
           \selectlanguage{ngerman}\endnumbering\briefempfaengerindex{Schnitzler, Arthur@\textsc{Schnitzler, Arthur}!zzzBahr, Hermann@\emph{von Hermann Bahr}!1909-05-101@{10. 5. 1909}|)be}\mylabel{L01842h}  \normalsize

\doendnotes{C}
\bigskip
\vfill

\clearpage

\footnotesize

\lohead{\textsc{register}}

% Definiere theindex-Environment komplett neu ohne reledmac
\makeatletter
\renewenvironment{theindex}{%
  \section*{\indexname}%
  \setlength{\parindent}{0pt}%
  \setlength{\parskip}{0pt plus 0.3pt}%
  \let\item\@idxitem
}{%
  \clearpage
}
\makeatother

\IfFileExists{\jobname-pw.ind}{\input{\jobname-pw.ind}}{}

\end{document}

      