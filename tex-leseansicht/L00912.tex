%% latex-korrekturansicht-vorspann.tex
%% Vorspann für die Korrekturansicht.
%% Lädt die gemeinsame Datei latex-vorspann.tex mit gesetztem Schalter.

\newif\ifkorrekturansicht
\korrekturansichttrue

\input{../tex-inputs/latex-vorspann}


\section[Hugo von Hofmannsthal an Arthur Schnitzler, {[}5. 4. 1899{]}]{L00912 Hugo von Hofmannsthal an Arthur Schnitzler, {[}5. 4. 1899{]}}
\nopagebreak\mylabel{L00912v}
\rehead{ }\normalsize\beginnumbering\briefempfaengerindex{Schnitzler, Arthur@\textsc{Schnitzler, Arthur}!zzzHofmannsthal, Hugo von@\emph{von Hugo von Hofmannsthal}!1899-04-051@{{[}5. 4. 1899{]}}|(be}
\toendnotes[C]{\smallbreak\pagebreak[2]}\Standort{CUL, Schnitzler, B 43.}
\physDesc{Brief, 1 Blatt, 1 Seite, 172 Zeichen
\newline{}Handschrift: Bleistift, deutsche Kurrent
\newline{}Schnitzler: mit Bleistift datiert: »5/4 99« 
\newline{}Ordnung: 1) mit Bleistift von unbekannter Hand nummeriert: »\strikeout{145}«  2) mit Bleistift von unbekannter Hand nummeriert:
                                    »142«}
\buchAbdrucke{\weitereDrucke{Hugo von Hofmannsthal, Arthur Schnitzler: \emph{Briefwechsel}. Frankfurt am Main: \emph{S. Fischer} 1964, S. 122.} }\toendnotes[C]{\smallbreak}
\pstart{}{\pb}lieber\pend\vspace{0.5em}
\pstart
           wenn es eine Stunde giebt, wo man Sie untertags trifft und nicht ſtört, ſo ſchreiben
               Sie mir ſie. Ich reiſe kaum vor Montag wegen der armen \label{K_L00912-1v}\edtext{Familie S.\pwindex{Schlesinger, Franziska 17.08.1851 – 11.08.1932@\textsc{Schlesinger, Franziska} (17.08.1851 – 11.08.1932)|pw}\pwindex{Hofmannsthal, Gertrude von 16.03.1880 – 09.11.1959@\textsc{Hofmannsthal, Gertrude von} (16.03.1880 – 09.11.1959)|pw}\pwindex{Schlesinger, Emil 10.05.1844 – 31.05.1899@\textsc{Schlesinger, Emil} (10.05.1844 – 31.05.1899), \emph{Bankdirektor/Bankdirektorin}|pw}}{\lemma{\textnormal{\emph{Familie S.}}}\Cendnote{\textnormal{Emil Schlesinger\pwindex{Neuwirth, Josef 06.05.1839 – 20.05.1895@\textsc{Neuwirth, Josef} (06.05.1839 – 20.05.1895), \emph{Politiker/Politikerin, Journalist/Journalistin}|pwk} starb wenige Wochen später, am
                     31. 5. 1899.}}}\label{K_L00912-1}\pend
           
\pstart
           Von Herzen Ihr{\\[\baselineskip]}\spacefill\mbox{Hugo.}\pend
           \leftskip=0em{}\selectlanguage{ngerman}\endnumbering\briefempfaengerindex{Schnitzler, Arthur@\textsc{Schnitzler, Arthur}!zzzHofmannsthal, Hugo von@\emph{von Hugo von Hofmannsthal}!1899-04-051@{{[}5. 4. 1899{]}}|)be}\mylabel{L00912h}  \normalsize

\doendnotes{C}
\bigskip
\vfill

\clearpage

\footnotesize

\lohead{\textsc{register}}

% Definiere theindex-Environment komplett neu ohne reledmac
\makeatletter
\renewenvironment{theindex}{%
  \section*{\indexname}%
  \setlength{\parindent}{0pt}%
  \setlength{\parskip}{0pt plus 0.3pt}%
  \let\item\@idxitem
}{%
  \clearpage
}
\makeatother

\IfFileExists{\jobname-pw.ind}{\input{\jobname-pw.ind}}{}

\end{document}

      