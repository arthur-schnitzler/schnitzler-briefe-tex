%% latex-korrekturansicht-vorspann.tex
%% Vorspann für die Korrekturansicht.
%% Lädt die gemeinsame Datei latex-vorspann.tex mit gesetztem Schalter.

\newif\ifkorrekturansicht
\korrekturansichttrue

\input{../tex-inputs/latex-vorspann}


\section[Hugo Hofmannsthal an Arthur Schnitzler, 16. 1. 1923]{L02396 Hugo Hofmannsthal an Arthur Schnitzler, 16. 1. 1923}
\nopagebreak\mylabel{L02396v}
\rehead{ }\normalsize\beginnumbering\briefempfaengerindex{Schnitzler, Arthur@\textsc{Schnitzler, Arthur}!zzzHofmannsthal, Hugo von@\emph{von Hugo von Hofmannsthal}!1923-01-161@{16. 1. 1923}|(be}
\toendnotes[C]{\smallbreak\pagebreak[2]}\Standort{CUL, Schnitzler, B 43.}
\physDesc{Brief, 1 Blatt, 2 Seiten, 1509 Zeichen
\newline{}Handschrift: schwarze Tinte, lateinische Kurrent
\newline{}Schnitzler: 1) mit Bleistift beschriftet: »\textsc{Hugo}«  2) mit rotem Buntstift mehrere Unterstreichungen
\newline{}Ordnung: 1) mit Bleistift von Frieda
                                    Pollak\pwindex{Pollak, Frieda 08.12.1881 – 13.07.1937@\textsc{Pollak, Frieda} (08.12.1881 – 13.07.1937), \emph{Sekretär/Sekretärin}|pw} (?) mit dem Buchstaben »A«
                                 (Abgeschrieben/Abschrift) gekennzeichnet  2) mit Bleistift von unbekannter Hand nummeriert: »\strikeout{368}« 3) mit Bleistift von unbekannter Hand nummeriert:
                                    »372«}
\buchAbdrucke{\weitereDrucke{Hugo von Hofmannsthal, Arthur Schnitzler: \emph{Briefwechsel}. Frankfurt am Main: \emph{S. Fischer} 1964, S. 297–298.} }\toendnotes[C]{\smallbreak}
\pstart
           \raggedleft{}{\pb}Rodaun\oindex{Rodaun@\textbf{Rodaun}, \emph{A.ADM4}|pw}{ }16 I 23\pend
           
\pstart{}mein lieber Arthur\pend\vspace{0.5em}
\pstart
           es freut mich so, dass ich wieder einmal von Ihnen einen Brief beko{\geminationm}e. – Zuletzt habe ich Sie im September
               gesehen – aber Sie mich nicht – bei der \label{K_L02396-1v}\edtext{Première der Dame Kobold\pwindex{Dame Kobold@\emph{Dame Kobold}|pw}}{\lemma{\textnormal{\emph{Première der Dame Kobold}}}\Cendnote{\textnormal{Siehe A. S.: \emph{Tagebuch}, 16. 9. 1922.
               }}}\label{K_L02396-1}. Sie standen neben Ihrer kleinen großen Tochter\pwindex{Cappellini, Lili 13.09.1909 – 26.07.1928@\textsc{Cappellini, Lili} (13.09.1909 – 26.07.1928)|pwv}, mir zugekehrt. Ich war auf der Gallerie und ich
               sah Sie mit dem Opernglas an. Wie inhaltsvoll und freundlich war mir Ihr Gesicht! Wie
               wenn ich ein Buch von tausend Seiten, dessen Inhalt ich aber gut kenne – in einem
               Augenblick überblättert hätte.\pend
           
\pstart
           Wie gerne würde ich Sie manchmal sehen, lieber Arthur. In die Stadt ko{\geminationm}e ich fast nie. Ich behalte das kleine Absteigquartier
               so lange man mirs lässt, aber ich beheize die Wohnung nicht mehr, \uline{betreibe} sie nicht mehr, halte dort keine Bedienerin. Ich
               kann das alles nicht mehr. Ich bin durch den Marksturz in eine fast unhaltbare
               materielle Situation geraten. Aber davon will {\pb}ich Sie durchaus nicht
               unterhalten. – Wenn es im März freundlich ist, dann möchte ich einmal
               vormittag zu Ihnen ko{\geminationm}en, mit Ihnen spazierengehen u.
               bei Ihnen essen. Ich weiss ja da\textcolor{gray}{s}s es Sie beschwert, hier herüber
               zu fahren! –\pend
           
\pstart
           Mit Strauss\pwindex{Strauss, Richard 11.06.1864 – 08.09.1949@\textsc{Strauss, Richard} (11.06.1864 – 08.09.1949), \emph{Theaterleiter/Theaterleiterin, Komponist/Komponistin, Dirigent/Dirigentin}|pw} würde ich sehr ungerne über die
                  Opernsache\pwindex{Schleier der Beatrice. Schauspiel in fuenf Akten@\emph{Der Schleier der Beatrice. Schauspiel in fünf Akten}|pwv} reden – aber mit
                  Schalk\pwindex{Schalk, Franz 27.05.1863 – 03.09.1931@\textsc{Schalk, Franz} (27.05.1863 – 03.09.1931), \emph{Theaterleiter/Theaterleiterin, Dirigent/Dirigentin}|pw} gerne wenn Sie wollen (obwohl es
               eben so aussichtslos ist da ich den Standpunkt kenne und die enormen Argumente die
               man für ihn geltend machen kann) – nur möchte ich abwarten, bis Schalk\pwindex{Schalk, Franz 27.05.1863 – 03.09.1931@\textsc{Schalk, Franz} (27.05.1863 – 03.09.1931), \emph{Theaterleiter/Theaterleiterin, Dirigent/Dirigentin}|pw} die schwere Sorge um seine Frau\pwindex{Schalk, Lili 04.11.1872 – 29.09.1966@\textsc{Schalk, Lili} (04.11.1872 – 29.09.1966), \emph{Schriftsteller/Schriftstellerin, Sänger/Sängerin}|pwv} los ist, die seit Wochen höchst elend
               darniederliegt mit einer Gelenksentzündung.\pend
           
\pstart
           Adieu, lieber Arthur.\pend
           
\pstart
           Von Herzen, wie immer,{\\[\baselineskip]}Ihr{\\[\baselineskip]}\spacefill\mbox{Hugo.}\pend
           \leftskip=0em{}\selectlanguage{ngerman}\endnumbering\briefempfaengerindex{Schnitzler, Arthur@\textsc{Schnitzler, Arthur}!zzzHofmannsthal, Hugo von@\emph{von Hugo von Hofmannsthal}!1923-01-161@{16. 1. 1923}|)be}\mylabel{L02396h}  \normalsize

\doendnotes{C}
\bigskip
\vfill

\clearpage

\footnotesize

\lohead{\textsc{register}}

% Definiere theindex-Environment komplett neu ohne reledmac
\makeatletter
\renewenvironment{theindex}{%
  \section*{\indexname}%
  \setlength{\parindent}{0pt}%
  \setlength{\parskip}{0pt plus 0.3pt}%
  \let\item\@idxitem
}{%
  \clearpage
}
\makeatother

\IfFileExists{\jobname-pw.ind}{\input{\jobname-pw.ind}}{}

\end{document}

      