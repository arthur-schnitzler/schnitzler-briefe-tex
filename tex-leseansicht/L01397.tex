%% latex-leseansicht-vorspann.tex
%% Vorspann für die Leseansicht.
%% Lädt die gemeinsame Datei latex-vorspann.tex mit nicht gesetztem Schalter.

\newif\ifkorrekturansicht
\korrekturansichtfalse

\input{../tex-inputs/latex-vorspann}


         \renewcommand{\erwaehnteWerke}{}
               \section[Hermann Bahr an Arthur Schnitzler, 28. 4. 1904]{ Hermann Bahr an Arthur Schnitzler, 28. 4. 1904}\nopagebreak\mylabel{v}\rehead{ }\begin{ledgroupsized}[t]{13cm}\normalsize\beginnumbering \toendnotes[C]{\smallbreak\pagebreak[2]} \Standort{CUL, Schnitzler, B 5b.}
\physDesc{Postkarte
\newline{}Handschrift: schwarze Tinte, deutsche Kurrent\newline{}Versand: 1) Rohrpost  2) Stempel: »\nobreak{}\oindex{XXXX Ortsangabe fehlt|pwk}Wien 13/5, 28. IV. 04\nobreak{}«.  3) Stempel: »\nobreak{}28. IV. 04\nobreak{}«.  4) Stempel: »\nobreak{}\oindex{XXXX Ortsangabe fehlt|pwk}Wien 18, 4.10N\nobreak{}«. \newline{}Ordnung: mit Bleistift von unbekannter Hand nummeriert: »116« }\buchAbdrucke{\weitereDrucke{Hermann Bahr, Arthur Schnitzler: \emph{Briefwechsel, Aufzeichnungen, Dokumente (1891–1931)}. Hg. Kurt Ifkovits und Martin Anton Müller. Göttingen: \emph{Wallstein} 2018, S. 306–307.} }\toendnotes[C]{\smallbreak}\pstart{}{\pb}Pneumatisch\pend{}\pstart{}Herrn \textsc{D\textsuperscript{r} Arthur Schnitzler}\pend{}\pstart{}\textsc{Wien XVIII}\oindex{XXXX Ortsangabe fehlt|pw}\pend{}\pstart{}\textsc{Spöttelgasse 7}\oindex{XXXX Ortsangabe fehlt|pw}\pend{}{\bigskip}\pstart
           \raggedleft{}{\pb}28. \textcolor{gray}{4}\pend
           \pstart{}Lieber Arthur!\pend\pstart
           Dein Brief u Deine Karten kamen um Viertel nach zehn abends an, ich hätte nicht vor
               elf in Hietzing\oindex{XXXX Ortsangabe fehlt|pw}{ }ſein können u \label{K_L01397_1v}\edtext{Euch\pwindex{\textcolor{red}{\textsuperscript{XXXX1 indx}}|pwv}\pwindex{\textcolor{red}{\textsuperscript{XXXX1 indx}}|pwv}\pwindex{\textcolor{red}{\textsuperscript{XXXX1 indx}}|pwv}\pwindex{\textcolor{red}{\textsuperscript{XXXX1 indx}}|pwv}\pwindex{\textcolor{red}{\textsuperscript{XXXX1 indx}}|pwv}}{\lemma{\textnormal{\emph{Euch}}}\Cendnote{\textnormal{Anwesend waren Richard\pwindex{\textcolor{red}{\textsuperscript{XXXX1 indx}}|pwk} und Paula
                     Beer-Hofmann\pwindex{\textcolor{red}{\textsuperscript{XXXX1 indx}}|pwk}, Gerty Hofmannsthal\pwindex{\textcolor{red}{\textsuperscript{XXXX1 indx}}|pwk}, Felix Salten\pwindex{\textcolor{red}{\textsuperscript{XXXX1 indx}}|pwk} und Arthur\pwindex{Schnitzler, Arthur 15.05.1862 – 21.10.1931@\textsc{Schnitzler, Arthur} (15.05.1862 – 21.10.1931), \emph{Schriftsteller, Mediziner}|pwk} und Olga Schnitzler\pwindex{\textcolor{red}{\textsuperscript{XXXX1 indx}}|pwk}.}}}\label{K_L01397_1h}
               dann wol nicht mehr getroffen. Mir war ſehr leid. Könnteſt Du mir \label{K_L01397_2v}\edtext{Samſtag}{\lemma{\textnormal{\emph{Samſtag}}}\Cendnote{\textnormal{Am 30. 4. Zum gewünschten Treffen dürfte es nicht
                  gekommen sein, da Schnitzler\pwindex{Schnitzler, Arthur 15.05.1862 – 21.10.1931@\textsc{Schnitzler, Arthur} (15.05.1862 – 21.10.1931), \emph{Schriftsteller, Mediziner}|pwk} an diesem Tag
                  seine Italien\oindex{XXXX Ortsangabe fehlt|pwk}reise begann.}}}\label{K_L01397_2h} zwiſchen \substVorne{}\textsuperscript{\textcolor{gray}{×}\-\textcolor{gray}{×}}\substDazwischen{}fünf\substHinten{} und ſechs ein Rendezvous in der Stadt geben?\pend
           \pstart
           Herzlichſt{\\[\baselineskip]}mit vielen Grüßen an Deine Fr.\pwindex{\textcolor{red}{\textsuperscript{XXXX1 indx}}|pwv}\hspace*{1.5em}\spacefill\mbox{Herm}\pend
           \leftskip=0em{}
         
         \endnumbering\mylabel{h}\end{ledgroupsized}  \newcommand{\dateiname}{L01397}\newcommand{\titel}{Hermann Bahr an Arthur Schnitzler, 28. 4. 1904}\newcommand{\editorInnen}{ Kurt Ifkovits,  Martin Anton Müller}%% latex-leseansicht-abspann.tex
%% Abspann für die Leseansicht.
%% Der Schalter \ifkorrekturansicht ist bereits durch den Vorspann gesetzt.

%% latex-abspann.tex
%% Gemeinsamer Abspann für Korrekturansicht und Leseansicht.
%% Setzt den Schalter \ifkorrekturansicht voraus (gesetzt in den
%% einbindenden Dateien latex-korrekturansicht-abspann.tex bzw.
%% latex-leseansicht-abspann.tex).
%% ---------------------------------------------------------------

\normalsize

% Das esempio-Environment wird nur in der Leseansicht benötigt
\ifkorrekturansicht\else
\newenvironment{esempio}[3]%
{
    \vspace{1.5ex}
    \rlap{\underline{#1}}
    \par
    \setlength{\parindent}{0cm}
    \nopagebreak
    \leftskip=#2cm
    \rightskip=#3cm
}
{
    \par
}
\fi

\doendnotes{C}
\bigskip
\vfill

\clearpage

\footnotesize

\ifkorrekturansicht
  \lohead{\textsc{register}}
\fi

% theindex-Environment neu definieren ohne reledmac
\makeatletter
\renewenvironment{theindex}{%
  \ifkorrekturansicht
    \section*{\indexname}%
  \else
    \subsubsection*{Index der erwähnten Entitäten}%
  \fi
  \setlength{\parindent}{0pt}%
  \setlength{\parskip}{0pt plus 0.3pt}%
  \let\item\@idxitem
}{%
  \ifkorrekturansicht\clearpage\fi
}
\makeatother

\IfFileExists{\jobname-pw.ind}{\input{\jobname-pw.ind}}{}

% Quellenangabe nur in der Leseansicht
\ifkorrekturansicht\else
% Fallback-Definitionen, falls die .tex-Datei \titel etc. nicht gesetzt hat
\providecommand{\titel}{}
\providecommand{\editorInnen}{}
\providecommand{\dateiname}{\jobname}

\vspace{3cm}

\vfill

\footnotesize
\textsc{Quelle}: \titel. Herausgegeben von {\editorInnen}. In: \emph{Arthur Schnitzler: Briefwechsel mit Autorinnen und Autoren}.
 Digitale Edition, https://schnitzler-briefe.acdh.oeaw.ac.at/{\dateiname}.html (Stand \today)
\fi

\end{document}


      