%% latex-leseansicht-vorspann.tex
%% Vorspann für die Leseansicht.
%% Lädt die gemeinsame Datei latex-vorspann.tex mit nicht gesetztem Schalter.

\newif\ifkorrekturansicht
\korrekturansichtfalse

\input{../tex-inputs/latex-vorspann}


\section[ Felix Salten an Arthur Schnitzler, {[}6. 11.? 1907{]}]{L03486 Felix Salten an Arthur Schnitzler,  [6. 11.? 1907]}
\nopagebreak\mylabel{L03486v}
\rehead{ }\normalsize\beginnumbering\briefempfaengerindex{Schnitzler, Arthur@\textsc{Schnitzler, Arthur}!zzzSalten, Felix@\emph{von Felix Salten}!1907-11-061@{{[}6. 11.? 1907{]}}|(be}
\toendnotes[C]{\smallbreak\pagebreak[2]}
\correspDesc{Versand  durch Felix Salten am [6. 11.? 1907] in Wien
\newline{}Erhalt  durch Arthur Schnitzler im Zeitraum [6. 11. 1907
                  – 7. 11. 1907?] in Wien}\toendnotes[C]{\smallbreak}
\Standort{CUL, Schnitzler, B 89, B 1.}
\physDesc{Brief, 1 Blatt, 1 Seite, 382 Zeichen
\newline{}Handschrift: schwarze Tinte, lateinische Kurrent
\newline{}Schnitzler: mit Bleistift datiert: »März 07\textcolor{gray}{?}« 
\newline{}Ordnung: mit Bleistift von unbekannter Hand nummeriert: »229« }\toendnotes[C]{\smallbreak}
\pstart
           \raggedleft{}{\pb}\label{K_L03486-1v}\edtext{Mittwoch}{\lemma{\textnormal{\emph{Mittwoch}}}\Cendnote{\textnormal{Die Datierung gelingt über den Umweg,
                     dass der an einem Donnerstag geschriebene Brief vom XXXX Auszeichnungsfehler: Dokument L03010 nicht gefunden auf alle die
                     im vorliegenden Brief angeschnittenen Themen antwortet. Entsprechend ist dieses
                     Korrespondenzstück auf den Vortag zu datieren.}}}\label{K_L03486-1}\pend
           
\pstart{}Lieber,\pend\vspace{0.5em}
\pstart
           vielleicht können wir \label{K_L03486-2v}\edtext{Samstag{ }nach dem Theater}{\lemma{\textnormal{\emph{Samstag nach dem Theater}}}\Cendnote{\textnormal{Die Premiere von Saltens\pwindex{Salten, Felix 6.\,9.\,1869 Budapest – 8.\,10.\,1945 Zürich@\textsc{Salten, Felix} (6.\,9.\,1869 Budapest – 8.\,10.\,1945 Zürich), \emph{Schriftsteller, Journalist, Chefredakteur}|pwk} Stück \emph{Vom andern Ufer}\pwindex{Salten, Felix 6.\,9.\,1869 Budapest – 8.\,10.\,1945 Zürich@\textsc{Salten, Felix} (6.\,9.\,1869 Budapest – 8.\,10.\,1945 Zürich), \emph{Schriftsteller, Journalist, Chefredakteur}!Vom andern Ufer. Einakter@\strich\emph{Vom andern Ufer. Einakter}|pwk} fand am 9. 11. 1907 am \emph{Volkstheater}\orgindex{Volkstheater@Volkstheater|pwk} statt. Schnitzler nahm teil, danach saß man im Meissl {\kaufmannsund} Schaden\oindex{Wien@\textbf{Wien}!I., Innere Stadt@\textbf{I., Innere Stadt}!Meissl {\kaufmannsund} Schadn@\textbf{Meissl {\kaufmannsund} Schadn}, \emph{Hotel}|pwk}
                  zusammen.}}}\label{K_L03486-2} beisammen sein? Mir ist es ganz egal wo; ich möchte nur irgendwo
               hin gehen, wo wenig Leute sind. Wenn Sie Richard\pwindex{Beer-Hofmann, Richard 11.\,7.\,1866 Wien – 26.\,9.\,1945 New York City@\textsc{Beer-Hofmann, Richard} (11.\,7.\,1866 Wien – 26.\,9.\,1945 New York City), \emph{Schriftsteller}|pw} sehen, bitte, sagen Sie es ihm auch. Ich höre, dass Herr Kainz\pwindex{Kainz, Josef 2.\,1.\,1858 Mosonmagyaróvár – 20.\,9.\,1910 Wien@\textsc{Kainz, Josef} (2.\,1.\,1858 Mosonmagyaróvár – 20.\,9.\,1910 Wien), \emph{Schauspieler}|pw} ins Theater geht; natürlich wär es mir
               angenehm, wenn er mit käme. Auch Speidels\pwindex{Speidel, Felix 2.\,7.\,1875 Stuttgart – 3.\,10.\,1952 Unterach am Attersee@\textsc{Speidel, Felix} (2.\,7.\,1875 Stuttgart – 3.\,10.\,1952 Unterach am Attersee), \emph{Schriftsteller, Verleger}|pw}\pwindex{Speidel-Haeberle, Else 11.\,7.\,1877 Stuttgart – 21.\,7.\,1937 Augustenfeld@\textsc{Speidel-Haeberle, Else} (11.\,7.\,1877 Stuttgart – 21.\,7.\,1937 Augustenfeld), \emph{Schauspielerin}|pw} werden dann wol mit uns sein. Bitte um eine Zeile.\pend
           
\pstart
           Herzlichst Ihr {\\[\baselineskip]}\spacefill\mbox{Salten}\pend
           \leftskip=0em{}\selectlanguage{ngerman}\endnumbering\briefempfaengerindex{Schnitzler, Arthur@\textsc{Schnitzler, Arthur}!zzzSalten, Felix@\emph{von Felix Salten}!1907-11-061@{{[}6. 11.? 1907{]}}|)be}\mylabel{L03486h}  \newcommand{\dateiname}{L03486}\newcommand{\titel}{Felix Salten an Arthur Schnitzler, [6. 11.? 1907]}\newcommand{\editorInnen}{Martin Anton Müller und Laura Untner}%% latex-leseansicht-abspann.tex
%% Abspann für die Leseansicht.
%% Der Schalter \ifkorrekturansicht ist bereits durch den Vorspann gesetzt.

%% latex-abspann.tex
%% Gemeinsamer Abspann für Korrekturansicht und Leseansicht.
%% Setzt den Schalter \ifkorrekturansicht voraus (gesetzt in den
%% einbindenden Dateien latex-korrekturansicht-abspann.tex bzw.
%% latex-leseansicht-abspann.tex).
%% ---------------------------------------------------------------

\normalsize

% Das esempio-Environment wird nur in der Leseansicht benötigt
\ifkorrekturansicht\else
\newenvironment{esempio}[3]%
{
    \vspace{1.5ex}
    \rlap{\underline{#1}}
    \par
    \setlength{\parindent}{0cm}
    \nopagebreak
    \leftskip=#2cm
    \rightskip=#3cm
}
{
    \par
}
\fi

\doendnotes{C}
\bigskip
\vfill

\clearpage

\footnotesize

\ifkorrekturansicht
  \lohead{\textsc{register}}
\fi

% theindex-Environment neu definieren ohne reledmac
\makeatletter
\renewenvironment{theindex}{%
  \ifkorrekturansicht
    \section*{\indexname}%
  \else
    \subsubsection*{Index der erwähnten Entitäten}%
  \fi
  \setlength{\parindent}{0pt}%
  \setlength{\parskip}{0pt plus 0.3pt}%
  \let\item\@idxitem
}{%
  \ifkorrekturansicht\clearpage\fi
}
\makeatother

\IfFileExists{\jobname-pw.ind}{\input{\jobname-pw.ind}}{}

% Quellenangabe nur in der Leseansicht
\ifkorrekturansicht\else
% Fallback-Definitionen, falls die .tex-Datei \titel etc. nicht gesetzt hat
\providecommand{\titel}{}
\providecommand{\editorInnen}{}
\providecommand{\dateiname}{\jobname}

\vspace{3cm}

\vfill

\footnotesize
\textsc{Quelle}: \titel. Herausgegeben von {\editorInnen}. In: \emph{Arthur Schnitzler: Briefwechsel mit Autorinnen und Autoren}.
 Digitale Edition, https://schnitzler-briefe.acdh.oeaw.ac.at/{\dateiname}.html (Stand \today)
\fi

\end{document}


