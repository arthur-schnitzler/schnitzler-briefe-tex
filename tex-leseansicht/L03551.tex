%% latex-leseansicht-vorspann.tex
%% Vorspann für die Leseansicht.
%% Lädt die gemeinsame Datei latex-vorspann.tex mit nicht gesetztem Schalter.

\newif\ifkorrekturansicht
\korrekturansichtfalse

\input{../tex-inputs/latex-vorspann}

\begin{center}
            \textcolor{red}{ENTWURF, NICHT FERTIG KORRIGIERT}
                      \end{center}
            
         
         \renewcommand{\erwaehntePersonen}{Personen: Alfred von Berger, Samuel Fischer, Hedwig Fischer, Josef Kainz, Gustav Mahler, Felix Salten, Ottilie Salten, Olga Schnitzler, Elisabeth Steinrück}
         \renewcommand{\erwaehnteInstitutionen}{Institutionen: Burgtheater}
         \renewcommand{\erwaehnteOrte}{Orte: Baden bei Wien, Berghof, München, Neue Musik-Festhalle, Südtirol, Unterach am Attersee, Wien}
         \renewcommand{\erwaehnteWerke}{Werke: 8. Sinfonie, Artur Schnitzler im Hofburgtheater, Das weite Land. Tragikomödie in fünf Akten, Der junge Medardus. Dramatische Historie in einem Vorspiel und fünf Aufzügen, Neue Freie Presse, Sappho. Trauerspiel in fünf Aufzügen}
               \section[ Felix Salten an Arthur Schnitzler, 17. 8. 1910]{ Felix Salten an Arthur Schnitzler, 17. 8. 1910}\nopagebreak\mylabel{v}\rehead{ }\begin{ledgroupsized}[t]{13cm}\normalsize\beginnumbering\briefempfaengerindex{Schnitzler, Arthur@\textsc{Schnitzler, Arthur}!zzzSalten, Felix@\emph{von Felix Salten}!1910-08-171@{17. 8. 1910}|(be} \toendnotes[C]{\smallbreak\pagebreak[2]} \Standort{CUL, Schnitzler, B 89, B 2.}
\physDesc{Brief, 1 Blatt, 1 Seite, 563 Zeichen
\newline{}Handschrift: schwarze Tinte, lateinische Kurrent
\newline{}Schnitzler: mit Bleistift Vermerk: »\textsc{Salte{[}n{]}}« 
\newline{}Ordnung: mit Bleistift von unbekannter Hand nummeriert: »266« }\toendnotes[C]{\smallbreak}\pstart
           \noindent{}\raggedleft{}{\pb}Unterach\oindex{Unterach am Attersee@\textbf{Unterach am Attersee}|pw}, Berghof\oindex{Berghof@\textbf{Berghof}|pw}.\pend
           \pstart
           \raggedleft{}17. VIII. 10\pend
           \pstart{}Lieber,\pend\pstart
           wir bleiben, denk’ ich, bis gegen den 10. September{ }hier\oindex{Unterach am Attersee@\textbf{Unterach am Attersee}|pwv}\textcolor{gray}{,} und Fischers\pwindex{Fischer, Samuel 24.12.1859 – 15.10.1934@\textsc{Fischer, Samuel} (24.12.1859 – 15.10.1934), \emph{Verleger}|pw}\pwindex{Fischer, Hedwig 08.09.1871 – 11.04.1952@\textsc{Fischer, Hedwig} (08.09.1871 – 11.04.1952)|pw},
               die zur \label{K_L03551-1v}\edtext{Mahler-Symphonie\pwindex{Mahler, Gustav 07.07.1860 – 18.05.1911@\textsc{Mahler, Gustav} (07.07.1860 – 18.05.1911), \emph{Theaterleiter, Komponist, Dirigent}!8. Sinfonie1910-09-12@\strich\emph{8. Sinfonie} {[}1910-09-12{]}|pwv} nach München\oindex{Muenchen@\textbf{München}|pw}}{\lemma{\textnormal{\emph{Mahler-Symphonie nach München}}}\Cendnote{\textnormal{Am 12. 9. 1910 fand in der Neuen
                     Musik-Halle\oindex{Neue Musik-Festhalle@\textbf{Neue Musik-Festhalle}|pwk} die Uraufführung der \emph{8.
                     Sinfonie}\pwindex{Mahler, Gustav 07.07.1860 – 18.05.1911@\textsc{Mahler, Gustav} (07.07.1860 – 18.05.1911), \emph{Theaterleiter, Komponist, Dirigent}!8. Sinfonie1910-09-12@\strich\emph{8. Sinfonie} {[}1910-09-12{]}|pwk} unter der Leitung Gustav
                     Mahler\pwindex{Mahler, Gustav 07.07.1860 – 18.05.1911@\textsc{Mahler, Gustav} (07.07.1860 – 18.05.1911), \emph{Theaterleiter, Komponist, Dirigent}|pwk}s statt.}}}\label{K_L03551-1h} wollen, werden wol auch so lange da sein. Wenn wir
               Aussicht hätten, Sie Beide\pwindex{Schnitzler, Olga 17.01.1882 – 13.01.1970@\textsc{Schnitzler, Olga} (17.01.1882 – 13.01.1970), \emph{Schauspielerin, Sängerin}|pwv}
               hier \label{K_L03551-2v}\edtext{auf dem Berghof\oindex{Berghof@\textbf{Berghof}|pw} zu begrüßen}{\lemma{\textnormal{\emph{auf … begrüßen}}}\Cendnote{\textnormal{Zu Schnitzler\pwindex{Schnitzler, Arthur 15.05.1862 – 21.10.1931@\textsc{Schnitzler, Arthur} (15.05.1862 – 21.10.1931), \emph{Schriftsteller, Mediziner}|pwk}s
                     Verhältnis zum Berghof\oindex{Berghof@\textbf{Berghof}|pwk}{ }siehe Felix Salten an Arthur Schnitzler, [25.? 8. 1892].}}}\label{K_L03551-2h}, würden wir uns herzlich freuen. Wann glauben Sie, dass
               Sie hierher kommen könnten? In der Zeitung lese ich, dass Sie mit dem \label{K_L03551-3v}\edtext{Burgtheater\orgindex{Burgtheater@Burgtheater|pw} einig}{\lemma{\textnormal{\emph{Burgtheater einig}}}\Cendnote{\textnormal{Am 14. 8. 1910 schrieb die \emph{Neue Freie Presse}\pwindex{Neue Freie Presse1864 – 1939@\emph{Neue Freie Presse} {[}1864 – 1939{]}|pwk}: »[\so{Artur Schnitzler}\pwindex{Schnitzler, Arthur 15.05.1862 – 21.10.1931@\textsc{Schnitzler, Arthur} (15.05.1862 – 21.10.1931), \emph{Schriftsteller, Mediziner}|pw}\so{{ }im{ }}\so{Hofburgtheater}\orgindex{Burgtheater@Burgtheater|pw}] In der kommenden Saiſon des Hofburgtheaters\orgindex{Burgtheater@Burgtheater|pw}, welches am 1.{ }September mit ›Sappho\pwindex{\textcolor{red}{\textsuperscript{XXXX1 indx}}!Sappho. Trauerspiel in fuenf Aufzuegen1818@\strich\emph{Sappho. Trauerspiel in fünf Aufzügen} {[}1818{]}|pw}‹
                     eröffnet wird, werden zwei neue Werke Artur
                        Schnitzler\pwindex{Schnitzler, Arthur 15.05.1862 – 21.10.1931@\textsc{Schnitzler, Arthur} (15.05.1862 – 21.10.1931), \emph{Schriftsteller, Mediziner}|pw} zur Aufführung gelangen. Als zweite Novität des Burgtheaters\orgindex{Burgtheater@Burgtheater|pw} geht ›\textsc{Der junge Herr Medardus}\pwindex{Schnitzler, Arthur 15.05.1862 – 21.10.1931@\textsc{Schnitzler, Arthur} (15.05.1862 – 21.10.1931), \emph{Schriftsteller, Mediziner}!junge Medardus. Dramatische Historie in einem Vorspiel und fuenf
                  Aufzuegen1910-10-26@\strich\emph{Der junge Medardus. Dramatische Historie in einem Vorspiel und fünf Aufzügen} {[}1910-10-26{]}|pw}‹ in Szene. {[}\ldots{]} Außer
                     dieſem Werke hat Direktor Alfred Freiherr v.
                           \so{Berger}\pwindex{Berger, Alfred von 30.04.1853 – 24.08.1912@\textsc{Berger, Alfred von} (30.04.1853 – 24.08.1912), \emph{Schriftsteller, Journalist, Theaterleiter}|pw} auch Schnitzler\pwindex{Schnitzler, Arthur 15.05.1862 – 21.10.1931@\textsc{Schnitzler, Arthur} (15.05.1862 – 21.10.1931), \emph{Schriftsteller, Mediziner}|pw}s Schauſpiel ›\so{Das weite Land}\pwindex{Schnitzler, Arthur 15.05.1862 – 21.10.1931@\textsc{Schnitzler, Arthur} (15.05.1862 – 21.10.1931), \emph{Schriftsteller, Mediziner}!weite Land. Tragikomoedie in fuenf Akten1910-10-20@\strich\emph{Das weite Land. Tragikomödie in fünf Akten} {[}1910-10-20{]}|pw}‹, das zum Teil in Baden bei Wien\oindex{Baden bei Wien@\textbf{Baden bei Wien}|pw}, zum
                     Teil in Tirol\oindex{Suedtirol@\textbf{Südtirol}|pw} ſpielt, zur Aufführung
                     angenommen. Die männliche Hauptrolle wird Herr \so{Kainz}\pwindex{Kainz, Josef 02.01.1858 – 20.09.1910@\textsc{Kainz, Josef} (02.01.1858 – 20.09.1910), \emph{Schauspieler}|pw} ſpielen.« ([O. V.]: \emph{Artur
                        Schnitzler im Hofburgtheater}\pwindex{?? Werk@Nicht ermittelte Verfasserinnen und Verfasser!Artur Schnitzler im Hofburgtheater1910-08-14@\emph{Artur Schnitzler im Hofburgtheater} {[}1910-08-14{]}|pwk}. In: \emph{Neue
                        Freie Presse}\pwindex{Neue Freie Presse1864 – 1939@\emph{Neue Freie Presse} {[}1864 – 1939{]}|pwk}, Nr. 16.515, 14. 8. 1910, Morgenblatt,
                     S. 15.)}}}\label{K_L03551-3h} sind, was mich sehr freut. Was ist »das weite Land\pwindex{Schnitzler, Arthur 15.05.1862 – 21.10.1931@\textsc{Schnitzler, Arthur} (15.05.1862 – 21.10.1931), \emph{Schriftsteller, Mediziner}!weite Land. Tragikomoedie in fuenf Akten1910-10-20@\strich\emph{Das weite Land. Tragikomödie in fünf Akten} {[}1910-10-20{]}|pw}«{\dots}?\pend
           \pstart
           Viele Grüße von uns\pwindex{Salten, Ottilie 07.03.1868 – 22.06.1942@\textsc{Salten, Ottilie} (07.03.1868 – 22.06.1942), \emph{Schauspielerin}|pwv} zu
               Ihnen, und die Bitte, uns \uline{bald} Nachricht zu geben,
                  \label{K_L03551-4v}\edtext{wie es Ihrer Schwägerin\pwindex{Steinrueck, Elisabeth 19.11.1885 – 07.04.1920@\textsc{Steinrück, Elisabeth} (19.11.1885 – 07.04.1920)|pwv} geht}{\lemma{\textnormal{\emph{wie … geht}}}\Cendnote{\textnormal{vgl. Arthur Schnitzler an Felix Salten, 8. 8. 191[0]}}}\label{K_L03551-4h}! Herzlichst {\\[\baselineskip]}Ihr {\\[\baselineskip]}\spacefill\mbox{Felix Salten}\pend
           \leftskip=0em{}
         
         \endnumbering\mylabel{h}\end{ledgroupsized}  \newcommand{\dateiname}{L03551}\newcommand{\titel}{Felix Salten an Arthur Schnitzler, 17. 8. 1910}\newcommand{\editorInnen}{Martin Anton Müller und Laura Untner}%% latex-leseansicht-abspann.tex
%% Abspann für die Leseansicht.
%% Der Schalter \ifkorrekturansicht ist bereits durch den Vorspann gesetzt.

%% latex-abspann.tex
%% Gemeinsamer Abspann für Korrekturansicht und Leseansicht.
%% Setzt den Schalter \ifkorrekturansicht voraus (gesetzt in den
%% einbindenden Dateien latex-korrekturansicht-abspann.tex bzw.
%% latex-leseansicht-abspann.tex).
%% ---------------------------------------------------------------

\normalsize

% Das esempio-Environment wird nur in der Leseansicht benötigt
\ifkorrekturansicht\else
\newenvironment{esempio}[3]%
{
    \vspace{1.5ex}
    \rlap{\underline{#1}}
    \par
    \setlength{\parindent}{0cm}
    \nopagebreak
    \leftskip=#2cm
    \rightskip=#3cm
}
{
    \par
}
\fi

\doendnotes{C}
\bigskip
\vfill

\clearpage

\footnotesize

\ifkorrekturansicht
  \lohead{\textsc{register}}
\fi

% theindex-Environment neu definieren ohne reledmac
\makeatletter
\renewenvironment{theindex}{%
  \ifkorrekturansicht
    \section*{\indexname}%
  \else
    \subsubsection*{Index der erwähnten Entitäten}%
  \fi
  \setlength{\parindent}{0pt}%
  \setlength{\parskip}{0pt plus 0.3pt}%
  \let\item\@idxitem
}{%
  \ifkorrekturansicht\clearpage\fi
}
\makeatother

\IfFileExists{\jobname-pw.ind}{\input{\jobname-pw.ind}}{}

% Quellenangabe nur in der Leseansicht
\ifkorrekturansicht\else
% Fallback-Definitionen, falls die .tex-Datei \titel etc. nicht gesetzt hat
\providecommand{\titel}{}
\providecommand{\editorInnen}{}
\providecommand{\dateiname}{\jobname}

\vspace{3cm}

\vfill

\footnotesize
\textsc{Quelle}: \titel. Herausgegeben von {\editorInnen}. In: \emph{Arthur Schnitzler: Briefwechsel mit Autorinnen und Autoren}.
 Digitale Edition, https://schnitzler-briefe.acdh.oeaw.ac.at/{\dateiname}.html (Stand \today)
\fi

\end{document}


      