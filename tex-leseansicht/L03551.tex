%% latex-leseansicht-vorspann.tex
%% Vorspann für die Leseansicht.
%% Lädt die gemeinsame Datei latex-vorspann.tex mit nicht gesetztem Schalter.

\newif\ifkorrekturansicht
\korrekturansichtfalse

\input{../tex-inputs/latex-vorspann}


\section[ Felix Salten an Arthur Schnitzler, 17. 8. 1910]{L03551 Felix Salten an Arthur Schnitzler,  17. 8. 1910}
\nopagebreak\mylabel{L03551v}
\rehead{ }\normalsize\beginnumbering\briefempfaengerindex{Schnitzler, Arthur@\textsc{Schnitzler, Arthur}!zzzSalten, Felix@\emph{von Felix Salten}!1910-08-171@{17. 8. 1910}|(be}
\toendnotes[C]{\smallbreak\pagebreak[2]}
\correspDesc{Versand  durch Felix Salten am 17. 8. 1910 in Unterach am Attersee
\newline{}Erhalt  durch Arthur Schnitzler im Zeitraum [18. 8. 1910
                  – 22. 8. 1910?] in Wien}\toendnotes[C]{\smallbreak}
\Standort{CUL, Schnitzler, B 89, B 2.}
\physDesc{Brief, 1 Blatt, 1 Seite, 563 Zeichen
\newline{}Handschrift: schwarze Tinte, lateinische Kurrent
\newline{}Schnitzler: mit Bleistift Vermerk: »\textsc{Salte{[}n{]}}« 
\newline{}Ordnung: mit Bleistift von unbekannter Hand nummeriert: »266« }\toendnotes[C]{\smallbreak}
\pstart
           \raggedleft{}{\pb}Unterach\oindex{Unterach am Attersee@\textbf{Unterach am Attersee}|pw}, Berghof\oindex{Berghof@\textbf{Berghof}, \emph{Wohngebäude}|pw}.\pend
           
\pstart
           \raggedleft{}17. VIII. 10\pend
           
\pstart{}Lieber,\pend\vspace{0.5em}
\pstart
           wir bleiben, denk’ ich, bis gegen den 10. September{ }hier\oindex{Unterach am Attersee@\textbf{Unterach am Attersee}|pwv}\textcolor{gray}{,} und Fischers\pwindex{Fischer, Samuel 24.\,12.\,1859 Liptovský Mikuláš – 15.\,10.\,1934 Berlin@\textsc{Fischer, Samuel} (24.\,12.\,1859 Liptovský Mikuláš – 15.\,10.\,1934 Berlin), \emph{Verleger}|pw}\pwindex{Fischer, Hedwig 8.\,9.\,1871 Szczecin – 11.\,4.\,1952 Königstein im Taunus@\textsc{Fischer, Hedwig} (8.\,9.\,1871 Szczecin – 11.\,4.\,1952 Königstein im Taunus)|pw},
               die zur \label{K_L03551-1v}\edtext{Mahler-Symphonie\pwindex{Mahler, Gustav 7.\,7.\,1860 Kaliště – 18.\,5.\,1911 Wien@\textsc{Mahler, Gustav} (7.\,7.\,1860 Kaliště – 18.\,5.\,1911 Wien), \emph{Theaterleiter, Komponist, Dirigent}!8. Sinfonie in Es-Dur@\strich\emph{8. Sinfonie in Es-Dur}|pwv} nach München\oindex{München@\textbf{München}|pw}}{\lemma{\textnormal{\emph{Mahler-Symphonie nach München}}}\Cendnote{\textnormal{Am 12. 9. 1910 fand in der Neuen
                     Musik-Halle\oindex{Neue Musik-Festhalle@\textbf{Neue Musik-Festhalle}, \emph{Gebäude}|pwk} die Uraufführung der \emph{8.
                     Sinfonie}\pwindex{Mahler, Gustav 7.\,7.\,1860 Kaliště – 18.\,5.\,1911 Wien@\textsc{Mahler, Gustav} (7.\,7.\,1860 Kaliště – 18.\,5.\,1911 Wien), \emph{Theaterleiter, Komponist, Dirigent}!8. Sinfonie in Es-Dur@\strich\emph{8. Sinfonie in Es-Dur}|pwk} unter der Leitung Gustav
                     Mahlers\pwindex{Mahler, Gustav 7.\,7.\,1860 Kaliště – 18.\,5.\,1911 Wien@\textsc{Mahler, Gustav} (7.\,7.\,1860 Kaliště – 18.\,5.\,1911 Wien), \emph{Theaterleiter, Komponist, Dirigent}|pwk} statt.}}}\label{K_L03551-1} wollen, werden wol auch so lange da sein. Wenn wir
               Aussicht hätten, Sie Beide\pwindex{Schnitzler, Olga 17.\,1.\,1882 Wien – 13.\,1.\,1970 Lugano@\textsc{Schnitzler, Olga} (17.\,1.\,1882 Wien – 13.\,1.\,1970 Lugano), \emph{Schauspielerin, Sängerin}|pwv}
               hier \label{K_L03551-2v}\edtext{auf dem Berghof\oindex{Berghof@\textbf{Berghof}, \emph{Wohngebäude}|pw} zu begrüßen}{\lemma{\textnormal{\emph{auf … begrüßen}}}\Cendnote{\textnormal{Zu Schnitzlers Verhältnis zum Berghof\oindex{Berghof@\textbf{Berghof}, \emph{Wohngebäude}|pwk}{ }siehe XXXX Auszeichnungsfehler: Dokument L03114 nicht gefunden.}}}\label{K_L03551-2}, würden wir uns herzlich freuen. Wann glauben Sie, dass
               Sie hierher kommen könnten? In der Zeitung lese ich, dass Sie mit dem \label{K_L03551-3v}\edtext{Burgtheater\orgindex{Burgtheater@Burgtheater|pw} einig}{\lemma{\textnormal{\emph{Burgtheater einig}}}\Cendnote{\textnormal{Am 14. 8. 1910 schrieb die \emph{Neue Freie Presse}\pwindex{Neue Freie Presse@\emph{Neue Freie Presse}|pwk}: »[\so{Artur Schnitzler}\hspace*{1em}\so{im}\hspace*{1em}\so{Hofburgtheater}\orgindex{Burgtheater@Burgtheater|pw}] In der kommenden Saiſon des Hofburgtheaters\orgindex{Burgtheater@Burgtheater|pw}, welches am 1.{ }September mit ›Sappho\pwindex{\textcolor{red}{\textsuperscript{XXXX indx1}}!Sappho. Trauerspiel in fünf Aufzügen@\strich\emph{Sappho. Trauerspiel in fünf Aufzügen}|pw}‹
                     eröffnet wird, werden zwei neue Werke Artur
                        Schnitzler zur Aufführung gelangen. Als zweite Novität des Burgtheaters\orgindex{Burgtheater@Burgtheater|pw} geht ›\textsc{Der junge Herr Medardus}\pwindex{Schnitzler, Arthur 15.\,5.\,1862 Wien – 21.\,10.\,1931 ebd.@\textsc{Schnitzler, Arthur} (15.\,5.\,1862 Wien – 21.\,10.\,1931 ebd.), \emph{Schriftsteller, Mediziner}!junge Medardus. Dramatische Historie in einem Vorspiel und fünf Aufzügen@\strich\emph{Der junge Medardus. Dramatische Historie in einem Vorspiel und fünf Aufzügen}|pw}‹ in Szene. {[}\ldots{]} Außer
                     dieſem Werke hat Direktor Alfred Freiherr v.
                           \so{Berger}\pwindex{Berger, Alfred von 30.\,4.\,1853 Wien – 24.\,8.\,1912 ebd.@\textsc{Berger, Alfred von} (30.\,4.\,1853 Wien – 24.\,8.\,1912 ebd.), \emph{Schriftsteller, Journalist, Theaterleiter}|pw} auch Schnitzlers Schauſpiel ›\so{Das weite Land}\pwindex{Schnitzler, Arthur 15.\,5.\,1862 Wien – 21.\,10.\,1931 ebd.@\textsc{Schnitzler, Arthur} (15.\,5.\,1862 Wien – 21.\,10.\,1931 ebd.), \emph{Schriftsteller, Mediziner}!weite Land. Tragikomödie in fünf Akten@\strich\emph{Das weite Land. Tragikomödie in fünf Akten}|pw}‹, das zum Teil in Baden bei Wien\oindex{Baden bei Wien@\textbf{Baden bei Wien}, \emph{Hauptstadt}|pw}, zum
                     Teil in Tirol\oindex{Südtirol@\textbf{Südtirol}, \emph{Verwaltungsgebiet}|pw}{ }ſpielt, zur Aufführung
                     angenommen. Die männliche Hauptrolle wird Herr \so{Kainz}\pwindex{Kainz, Josef 2.\,1.\,1858 Mosonmagyaróvár – 20.\,9.\,1910 Wien@\textsc{Kainz, Josef} (2.\,1.\,1858 Mosonmagyaróvár – 20.\,9.\,1910 Wien), \emph{Schauspieler}|pw}{ }ſpielen.« [O. V.]: \emph{Artur
                        Schnitzler im Hofburgtheater}\pwindex{Artur Schnitzler im Hofburgtheater@\emph{Artur Schnitzler im Hofburgtheater}|pwk}. In: \emph{Neue
                        Freie Presse}\pwindex{Neue Freie Presse@\emph{Neue Freie Presse}|pwk}, Nr. 16.515, 14. 8. 1910, Morgenblatt,
                     S. 15.}}}\label{K_L03551-3} sind, was mich sehr freut. Was ist »das weite Land\pwindex{Schnitzler, Arthur 15.\,5.\,1862 Wien – 21.\,10.\,1931 ebd.@\textsc{Schnitzler, Arthur} (15.\,5.\,1862 Wien – 21.\,10.\,1931 ebd.), \emph{Schriftsteller, Mediziner}!weite Land. Tragikomödie in fünf Akten@\strich\emph{Das weite Land. Tragikomödie in fünf Akten}|pw}«{\dots}?\pend
           
\pstart
           Viele Grüße von uns\pwindex{Salten, Ottilie 7.\,3.\,1868 Prag – 22.\,6.\,1942 Zürich@\textsc{Salten, Ottilie} (7.\,3.\,1868 Prag – 22.\,6.\,1942 Zürich), \emph{Schauspielerin}|pwv} zu
               Ihnen, und die Bitte, uns \uline{bald} Nachricht zu geben,
                  \label{K_L03551-4v}\edtext{wie es Ihrer Schwägerin\pwindex{Steinrück, Elisabeth 19.\,11.\,1885 – 7.\,4.\,1920 Partenkirchen@\textsc{Steinrück, Elisabeth} (19.\,11.\,1885 – 7.\,4.\,1920 Partenkirchen)|pwv} geht}{\lemma{\textnormal{\emph{wie … geht}}}\Cendnote{\textnormal{Vgl. XXXX Auszeichnungsfehler: Dokument L03018 nicht gefunden.
               }}}\label{K_L03551-4}! Herzlichst {\\[\baselineskip]}Ihr {\\[\baselineskip]}\spacefill\mbox{Felix Salten}\pend
           \leftskip=0em{}\selectlanguage{ngerman}\endnumbering\briefempfaengerindex{Schnitzler, Arthur@\textsc{Schnitzler, Arthur}!zzzSalten, Felix@\emph{von Felix Salten}!1910-08-171@{17. 8. 1910}|)be}\mylabel{L03551h}  \newcommand{\dateiname}{L03551}\newcommand{\titel}{Felix Salten an Arthur Schnitzler, 17. 8. 1910}\newcommand{\editorInnen}{Martin Anton Müller und Laura Untner}%% latex-leseansicht-abspann.tex
%% Abspann für die Leseansicht.
%% Der Schalter \ifkorrekturansicht ist bereits durch den Vorspann gesetzt.

%% latex-abspann.tex
%% Gemeinsamer Abspann für Korrekturansicht und Leseansicht.
%% Setzt den Schalter \ifkorrekturansicht voraus (gesetzt in den
%% einbindenden Dateien latex-korrekturansicht-abspann.tex bzw.
%% latex-leseansicht-abspann.tex).
%% ---------------------------------------------------------------

\normalsize

% Das esempio-Environment wird nur in der Leseansicht benötigt
\ifkorrekturansicht\else
\newenvironment{esempio}[3]%
{
    \vspace{1.5ex}
    \rlap{\underline{#1}}
    \par
    \setlength{\parindent}{0cm}
    \nopagebreak
    \leftskip=#2cm
    \rightskip=#3cm
}
{
    \par
}
\fi

\doendnotes{C}
\bigskip
\vfill

\clearpage

\footnotesize

\ifkorrekturansicht
  \lohead{\textsc{register}}
\fi

% theindex-Environment neu definieren ohne reledmac
\makeatletter
\renewenvironment{theindex}{%
  \ifkorrekturansicht
    \section*{\indexname}%
  \else
    \subsubsection*{Index der erwähnten Entitäten}%
  \fi
  \setlength{\parindent}{0pt}%
  \setlength{\parskip}{0pt plus 0.3pt}%
  \let\item\@idxitem
}{%
  \ifkorrekturansicht\clearpage\fi
}
\makeatother

\IfFileExists{\jobname-pw.ind}{\input{\jobname-pw.ind}}{}

% Quellenangabe nur in der Leseansicht
\ifkorrekturansicht\else
% Fallback-Definitionen, falls die .tex-Datei \titel etc. nicht gesetzt hat
\providecommand{\titel}{}
\providecommand{\editorInnen}{}
\providecommand{\dateiname}{\jobname}

\vspace{3cm}

\vfill

\footnotesize
\textsc{Quelle}: \titel. Herausgegeben von {\editorInnen}. In: \emph{Arthur Schnitzler: Briefwechsel mit Autorinnen und Autoren}.
 Digitale Edition, https://schnitzler-briefe.acdh.oeaw.ac.at/{\dateiname}.html (Stand \today)
\fi

\end{document}


