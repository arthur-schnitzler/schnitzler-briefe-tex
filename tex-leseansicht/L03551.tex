%% latex-korrekturansicht-vorspann.tex
%% Vorspann für die Korrekturansicht.
%% Lädt die gemeinsame Datei latex-vorspann.tex mit gesetztem Schalter.

\newif\ifkorrekturansicht
\korrekturansichttrue

\input{../tex-inputs/latex-vorspann}


\section[ Felix Salten an Arthur Schnitzler, 17. 8. 1910]{L03551 Felix Salten an Arthur Schnitzler, 17. 8. 1910}
\nopagebreak\mylabel{L03551v}
\rehead{ }\normalsize\beginnumbering\briefempfaengerindex{Schnitzler, Arthur@\textsc{Schnitzler, Arthur}!zzzSalten, Felix@\emph{von Felix Salten}!1910-08-171@{17. 8. 1910}|(be}
\toendnotes[C]{\smallbreak\pagebreak[2]}\Standort{CUL, Schnitzler, B 89, B 2.}
\physDesc{Brief, 1 Blatt, 1 Seite, 563 Zeichen
\newline{}Handschrift: schwarze Tinte, lateinische Kurrent
\newline{}Schnitzler: mit Bleistift Vermerk: »\textsc{Salte{[}n{]}}« 
\newline{}Ordnung: mit Bleistift von unbekannter Hand nummeriert: »266« }\toendnotes[C]{\smallbreak}
\pstart
           \raggedleft{}{\pb}Unterach\oindex{Unterach am Attersee@\textbf{Unterach am Attersee}, \emph{P.PPL}|pw}, Berghof\oindex{Berghof@\textbf{Berghof}, \emph{Wohngebäude (K.WHS)}|pw}.\pend
           
\pstart
           \raggedleft{}17. VIII. 10\pend
           
\pstart{}Lieber,\pend\vspace{0.5em}
\pstart
           wir bleiben, denk’ ich, bis gegen den 10. September{ }hier\oindex{Unterach am Attersee@\textbf{Unterach am Attersee}, \emph{P.PPL}|pwv}\textcolor{gray}{,} und Fischers\pwindex{Fischer, Samuel 24.12.1859 – 15.10.1934@\textsc{Fischer, Samuel} (24.12.1859 – 15.10.1934), \emph{Verleger/Verlegerin}|pw}\pwindex{Fischer, Hedwig 08.09.1871 – 11.04.1952@\textsc{Fischer, Hedwig} (08.09.1871 – 11.04.1952)|pw},
               die zur \label{K_L03551-1v}\edtext{Mahler-Symphonie\pwindex{8. Sinfonie in Es-Dur@\emph{8. Sinfonie in Es-Dur}|pwv} nach München\oindex{Muenchen@\textbf{München}, \emph{P.PPLA}|pw}}{\lemma{\textnormal{\emph{Mahler-Symphonie nach München}}}\Cendnote{\textnormal{Am 12. 9. 1910 fand in der Neuen
                     Musik-Halle\oindex{Neue Musik-Festhalle@\textbf{Neue Musik-Festhalle}, \emph{Gebäude (K.GBD)}|pwk} die Uraufführung der \emph{8.
                     Sinfonie}\pwindex{8. Sinfonie in Es-Dur@\emph{8. Sinfonie in Es-Dur}|pwk} unter der Leitung Gustav
                     Mahlers\pwindex{Mahler, Gustav 07.07.1860 – 18.05.1911@\textsc{Mahler, Gustav} (07.07.1860 – 18.05.1911), \emph{Theaterleiter/Theaterleiterin, Komponist/Komponistin, Dirigent/Dirigentin}|pwk} statt.}}}\label{K_L03551-1} wollen, werden wol auch so lange da sein. Wenn wir
               Aussicht hätten, Sie Beide\pwindex{Schnitzler, Olga 17.01.1882 – 13.01.1970@\textsc{Schnitzler, Olga} (17.01.1882 – 13.01.1970), \emph{Schauspieler/Schauspielerin, Sänger/Sängerin}|pwv}
               hier \label{K_L03551-2v}\edtext{auf dem Berghof\oindex{Berghof@\textbf{Berghof}, \emph{Wohngebäude (K.WHS)}|pw} zu begrüßen}{\lemma{\textnormal{\emph{auf … begrüßen}}}\Cendnote{\textnormal{Zu Schnitzlers Verhältnis zum Berghof\oindex{Berghof@\textbf{Berghof}, \emph{Wohngebäude (K.WHS)}|pwk}{ }siehe Felix Salten an Arthur Schnitzler, [25.? 8. 1892].}}}\label{K_L03551-2}, würden wir uns herzlich freuen. Wann glauben Sie, dass
               Sie hierher kommen könnten? In der Zeitung lese ich, dass Sie mit dem \label{K_L03551-3v}\edtext{Burgtheater\orgindex{Burgtheater@Burgtheater|pw} einig}{\lemma{\textnormal{\emph{Burgtheater einig}}}\Cendnote{\textnormal{Am 14. 8. 1910 schrieb die \emph{Neue Freie Presse}\pwindex{Neue Freie Presse@\emph{Neue Freie Presse}|pwk}: »[\so{Artur Schnitzler}\so{{ }im{ }}\so{Hofburgtheater}\orgindex{Burgtheater@Burgtheater|pw}] In der kommenden Saiſon des Hofburgtheaters\orgindex{Burgtheater@Burgtheater|pw}, welches am 1.{ }September mit ›Sappho\pwindex{Sappho. Trauerspiel in fuenf Aufzuegen@\emph{Sappho. Trauerspiel in fünf Aufzügen}|pw}‹
                     eröffnet wird, werden zwei neue Werke Artur
                        Schnitzler zur Aufführung gelangen. Als zweite Novität des Burgtheaters\orgindex{Burgtheater@Burgtheater|pw} geht ›\textsc{Der junge Herr Medardus}\pwindex{junge Medardus. Dramatische Historie in einem Vorspiel und fuenf Aufzuegen@\emph{Der junge Medardus. Dramatische Historie in einem Vorspiel und fünf Aufzügen}|pw}‹ in Szene. {[}\ldots{]} Außer
                     dieſem Werke hat Direktor Alfred Freiherr v.
                           \so{Berger}\pwindex{Berger, Alfred von 30.04.1853 – 24.08.1912@\textsc{Berger, Alfred von} (30.04.1853 – 24.08.1912), \emph{Schriftsteller/Schriftstellerin, Journalist/Journalistin, Theaterleiter/Theaterleiterin}|pw} auch Schnitzlers Schauſpiel ›\so{Das weite Land}\pwindex{weite Land. Tragikomoedie in fuenf Akten@\emph{Das weite Land. Tragikomödie in fünf Akten}|pw}‹, das zum Teil in Baden bei Wien\oindex{Baden bei Wien@\textbf{Baden bei Wien}, \emph{P.PPLA3}|pw}, zum
                     Teil in Tirol\oindex{Suedtirol@\textbf{Südtirol}, \emph{A.ADM2}|pw} ſpielt, zur Aufführung
                     angenommen. Die männliche Hauptrolle wird Herr \so{Kainz}\pwindex{Kainz, Josef 02.01.1858 – 20.09.1910@\textsc{Kainz, Josef} (02.01.1858 – 20.09.1910), \emph{Schauspieler/Schauspielerin}|pw} ſpielen.« [O. V.]: \emph{Artur
                        Schnitzler im Hofburgtheater}\pwindex{Artur Schnitzler im Hofburgtheater@\emph{Artur Schnitzler im Hofburgtheater}|pwk}. In: \emph{Neue
                        Freie Presse}\pwindex{Neue Freie Presse@\emph{Neue Freie Presse}|pwk}, Nr. 16.515, 14. 8. 1910, Morgenblatt,
                     S. 15.}}}\label{K_L03551-3} sind, was mich sehr freut. Was ist »das weite Land\pwindex{weite Land. Tragikomoedie in fuenf Akten@\emph{Das weite Land. Tragikomödie in fünf Akten}|pw}«{\dots}?\pend
           
\pstart
           Viele Grüße von uns\pwindex{Salten, Ottilie 07.03.1868 – 22.06.1942@\textsc{Salten, Ottilie} (07.03.1868 – 22.06.1942), \emph{Schauspieler/Schauspielerin}|pwv} zu
               Ihnen, und die Bitte, uns \uline{bald} Nachricht zu geben,
                  \label{K_L03551-4v}\edtext{wie es Ihrer Schwägerin\pwindex{Steinrueck, Elisabeth 19.11.1885 – 07.04.1920@\textsc{Steinrück, Elisabeth} (19.11.1885 – 07.04.1920)|pwv} geht}{\lemma{\textnormal{\emph{wie … geht}}}\Cendnote{\textnormal{Vgl. Arthur Schnitzler an Felix Salten, 8. 8. 191[0].
               }}}\label{K_L03551-4}! Herzlichst {\\[\baselineskip]}Ihr {\\[\baselineskip]}\spacefill\mbox{Felix Salten}\pend
           \leftskip=0em{}\selectlanguage{ngerman}\endnumbering\briefempfaengerindex{Schnitzler, Arthur@\textsc{Schnitzler, Arthur}!zzzSalten, Felix@\emph{von Felix Salten}!1910-08-171@{17. 8. 1910}|)be}\mylabel{L03551h}  \normalsize

\doendnotes{C}
\bigskip
\vfill

\clearpage

\footnotesize

\lohead{\textsc{register}}

% Definiere theindex-Environment komplett neu ohne reledmac
\makeatletter
\renewenvironment{theindex}{%
  \section*{\indexname}%
  \setlength{\parindent}{0pt}%
  \setlength{\parskip}{0pt plus 0.3pt}%
  \let\item\@idxitem
}{%
  \clearpage
}
\makeatother

\IfFileExists{\jobname-pw.ind}{\input{\jobname-pw.ind}}{}

\end{document}

      