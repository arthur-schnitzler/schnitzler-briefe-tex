%% latex-leseansicht-vorspann.tex
%% Vorspann für die Leseansicht.
%% Lädt die gemeinsame Datei latex-vorspann.tex mit nicht gesetztem Schalter.

\newif\ifkorrekturansicht
\korrekturansichtfalse

\input{../tex-inputs/latex-vorspann}


\section[Arthur Schnitzler an Berta Zuckerkandl, 20. 11. 1926]{L03969 Arthur Schnitzler an Berta Zuckerkandl, 20. 11. 1926}
\nopagebreak\mylabel{L03969v}
\rehead{ }\normalsize\beginnumbering\briefempfaengerindex{Zuckerkandl, Berta@\textsc{Zuckerkandl, Berta}!zzzSchnitzler, Arthur@\emph{von Arthur Schnitzler}!1926-11-201@{20. 11. 1926}|(be}
\toendnotes[C]{\smallbreak\pagebreak[2]}
\correspDesc{Versand  durch Arthur Schnitzler am 20. 11. 1926 in Wien
\newline{}Erhalt  durch Berta Zuckerkandl im Zeitraum [21. 11. 1926 – 25. 11. 1926?] in Paris}\toendnotes[C]{\smallbreak}
\Standort{DLA, HS.1985.1.2282.}
\physDesc{Brief, Durchschlag, 1 Blatt, 1 Seite, 1236 Zeichen
\newline{}Schreibmaschine
\newline{}Handschrift: 1) roter Buntstift, lateinische Kurrent (\noindent{}beschriftet: »\uline{Zuckerkandl}«, »Hofrätin« und »\uline{Paris}«, neun Unterstreichungen)\hspace{1em}2) Bleistift, lateinische Kurrent (\noindent{}Korrekturen)\hspace{1em}}\toendnotes[C]{\smallbreak}
\pstart
           \raggedleft{}{\pb}20. 11. 1926.\pend
           
\pstart{}Liebe und verehrte Frau Hofrätin.\pend\vspace{0.5em}
\pstart
           Vielen Dank für Ihre lieben \label{K_L03969-1v}\edtext{Zeilen}{\lemma{\textnormal{\emph{Zeilen}}}\Cendnote{\textnormal{nicht überliefert}}}\label{K_L03969-1}. An Frau P.\pwindex{Pollaczek, Clara Katharina 15.\,1.\,1875 Wien – 22.\,7.\,1951 ebd.@\textsc{Pollaczek, Clara Katharina} (15.\,1.\,1875 Wien – 22.\,7.\,1951 ebd.), \emph{Schriftstellerin}|pw}
               habe ich Ihre B{[}e{]}stellung weitergegeben, sie dankt und grüsst
               bestens. Was Sie mir von 
               Paul G.\pwindex{Géraldy, Paul 6.\,3.\,1885 Paris – 9.\,3.\,1983 Neuilly-sur-Seine@\textsc{Géraldy, Paul} (6.\,3.\,1885 Paris – 9.\,3.\,1983 Neuilly-sur-Seine), \emph{Schriftsteller}|pw}’s \label{K_L03969-2v}\edtext{\label{T_L03969-1v}\edtext{persönlichem}{\lemma{\textnormal{\emph{persönlichem}}}\Cendnote{\textnormal{In der Vorlage steht: »persönliches«.}}}\label{T_L03969-1}
               Schicksal}{\lemma{\textnormal{\emph{persönlichem
               Schicksal}}}\Cendnote{\textnormal{1926 endete die Ehe von Paul Géraldy und der Opernsängerin Germaine Lubin\pwindex{Lubin, Germaine 1.\,2.\,1890 Paris – 17.\,10.\,1979 ebd.@\textsc{Lubin, Germaine} (1.\,2.\,1890 Paris – 17.\,10.\,1979 ebd.), \emph{Sängerin}|pwk}.}}}\label{K_L03969-2} schreiben betrübt mich sehr. Es gibt in solchen Fällen immer tausend Gründe
               oder keinen. Bitte grüssen Sie ihn sehr herzlich von mir.\pend
           
\pstart
           Von Delamain\pwindex{Delamain, Maurice 28.\,4.\,1883 Jarnac – 2.\,5.\,1974 Paris@\textsc{Delamain, Maurice} (28.\,4.\,1883 Jarnac – 2.\,5.\,1974 Paris), \emph{Kritiker, Rechtsanwalt, Verleger}|pw} habe ich einen sehr
               liebenswürdigen und ausführlichen \label{K_L03969-3v}\edtext{Brief}{\lemma{\textnormal{\emph{Brief}}}\Cendnote{\textnormal{nicht überliefert}}}\label{K_L03969-3}
               bekommen, in dem er auf die einzelnen Varianten der »Else\pwindex{Schnitzler, Arthur 15. 5. 1862 Wien – 21. 10. 1931 ebd.@\textsc{Schnitzler, Arthur} (15. 5. 1862 Wien – 21. 10. 1931 ebd.), \emph{Schriftsteller, Mediziner}!Madmoiselle Else@\strich\emph{Madmoiselle Else}|pw}«-Übersetzung\pwindex{Schnitzler, Arthur 15. 5. 1862 Wien – 21. 10. 1931 ebd.@\textsc{Schnitzler, Arthur} (15. 5. 1862 Wien – 21. 10. 1931 ebd.), \emph{Schriftsteller, Mediziner}!Fräulein Else@\strich\emph{Fräulein Else}|pwv} mit viel Verständnis,
               wenn auch grös\introOben{}s\introOben{}tenteils ablehnend, zu sprechen kommt. Aus
               seinem Brief erst entnahm ich, dass er eine sozusagen doppelsprachige Gattin\pwindex{Delamain-Rickmers, Etha 3.\,8.\,1886 Bad Essen – 1943 Paris@\textsc{Delamain-Rickmers, Etha} (3.\,8.\,1886 Bad Essen – 1943 Paris), \emph{Übersetzerin}|pwv} hat. Ich habe ihm
               nahegelegt die bei ihm, resp. bei Stock\orgindex{Éditions Stock@Éditions Stock|pw} vor
               ungefähr 14 Jahren erschienenen \label{K_L03969-4v}\edtext{Uebersetzungen\pwindex{Schnitzler, Arthur 15. 5. 1862 Wien – 21. 10. 1931 ebd.@\textsc{Schnitzler, Arthur} (15. 5. 1862 Wien – 21. 10. 1931 ebd.), \emph{Schriftsteller, Mediziner}!Anatole. Suivi de La Compagne@\strich\emph{Anatole. Suivi de La Compagne}|pwv}\pwindex{Schnitzler, Arthur 15. 5. 1862 Wien – 21. 10. 1931 ebd.@\textsc{Schnitzler, Arthur} (15. 5. 1862 Wien – 21. 10. 1931 ebd.), \emph{Schriftsteller, Mediziner}!ronde. Dix scènes dialoguées@\strich\emph{La ronde. Dix scènes dialoguées}|pwv} von
                  »Anatol\pwindex{Schnitzler, Arthur 15. 5. 1862 Wien – 21. 10. 1931 ebd.@\textsc{Schnitzler, Arthur} (15. 5. 1862 Wien – 21. 10. 1931 ebd.), \emph{Schriftsteller, Mediziner}!Anatol@\strich\emph{Anatol}|pw}« und »Reigen\pwindex{Schnitzler, Arthur 15. 5. 1862 Wien – 21. 10. 1931 ebd.@\textsc{Schnitzler, Arthur} (15. 5. 1862 Wien – 21. 10. 1931 ebd.), \emph{Schriftsteller, Mediziner}!Reigen. Zehn Dialoge@\strich\emph{Reigen. Zehn Dialoge}|pw}«}{\lemma{\textnormal{\emph{Uebersetzungen … »Reigen«}}}\Cendnote{\textnormal{Arthur Schnitzler: \emph{La
                        ronde. Dix scènes dialoguées}\pwindex{Schnitzler, Arthur 15. 5. 1862 Wien – 21. 10. 1931 ebd.@\textsc{Schnitzler, Arthur} (15. 5. 1862 Wien – 21. 10. 1931 ebd.), \emph{Schriftsteller, Mediziner}!ronde. Dix scènes dialoguées@\strich\emph{La ronde. Dix scènes dialoguées}|pwk}. Traduction de Maurice Rémon\pwindex{Rémon, Maurice 27.\,11.\,1861 Paris – 20.\,6.\,1945 Mérignac@\textsc{Rémon, Maurice} (27.\,11.\,1861 Paris – 20.\,6.\,1945 Mérignac), \emph{Übersetzer}|pwk}{ }{\kaufmannsund}{ }Wilhelm Bauer\pwindex{Bauer, Wilhelm 27.\,11.\,1854 Zollingen – 11.\,9.\,1923 Paris@\textsc{Bauer, Wilhelm} (27.\,11.\,1854 Zollingen – 11.\,9.\,1923 Paris)|pwk},
                     Paris: \emph{Stock}\orgindex{Éditions Stock@Éditions Stock|pwk}{ }1912 und Arthur Schnitzler: \emph{Anatole. Suivi de La
                                 Compagne}\pwindex{Schnitzler, Arthur 15. 5. 1862 Wien – 21. 10. 1931 ebd.@\textsc{Schnitzler, Arthur} (15. 5. 1862 Wien – 21. 10. 1931 ebd.), \emph{Schriftsteller, Mediziner}!Anatole. Suivi de La Compagne@\strich\emph{Anatole. Suivi de La Compagne}|pwk}. Traduction de Maurice Rémon\pwindex{Rémon, Maurice 27.\,11.\,1861 Paris – 20.\,6.\,1945 Mérignac@\textsc{Rémon, Maurice} (27.\,11.\,1861 Paris – 20.\,6.\,1945 Mérignac), \emph{Übersetzer}|pwk} et Maurice Vaucaire\pwindex{Vaucaire, Maurice 2.\,7.\,1863 Versailles – 10.\,2.\,1918 Neuilly-sur-Seine@\textsc{Vaucaire, Maurice} (2.\,7.\,1863 Versailles – 10.\,2.\,1918 Neuilly-sur-Seine), \emph{Schriftsteller, Schauspieler, Übersetzer}|pwk}, Paris: \emph{Stock}\orgindex{Éditions Stock@Éditions Stock|pwk}{ }1913.}}}\label{K_L03969-4} einer sorgfältigen Durchsicht unterziehen zu lassen und
               eventuell neu herauszugeben.\pend
           
\pstart
           Auf Gemier\pwindex{Gémier, Firmin 21.\,2.\,1865 Aubervilliers – 26.\,11.\,1933 Paris@\textsc{Gémier, Firmin} (21.\,2.\,1865 Aubervilliers – 26.\,11.\,1933 Paris), \emph{Theaterleiter, Schauspieler, Drehbuchautor}|pw} setze ich nach wie vor wenig
               Hoffnungen. So bleiben mir in jedem Fall Enttäuschungen erspart.\pend
           
\pstart
           Sie schreiben kein Wort davon, wann Sie wieder zurückzukommen gedenken. Hoffentlich
               lassen Sie uns nicht zu lange mehr warten und kehren Ihrem pessimistischen
               Schlussabsatz zu Trotz mit günstigen Resultaten vor allem für sich selbst nach Wien\oindex{Wien@\textbf{Wien}, \emph{Verwaltungsgebiet}|pw} zurück.\pend
           
\pstart
           Mit den herzlichsten Grüssen{\\[\baselineskip]} Ihr getreuer\pend
           \leftskip=0em{}\selectlanguage{ngerman}\endnumbering\briefempfaengerindex{Zuckerkandl, Berta@\textsc{Zuckerkandl, Berta}!zzzSchnitzler, Arthur@\emph{von Arthur Schnitzler}!1926-11-201@{20. 11. 1926}|)be}\mylabel{L03969h}
\begin{anhang}
\end{anhang}\newcommand{\dateiname}{L03969}\newcommand{\titel}{Arthur Schnitzler an Berta Zuckerkandl, 20. 11. 1926}\newcommand{\editorInnen}{Herausgegeben von Jahnke, SelmaMüller, Martin Anton}%% latex-leseansicht-abspann.tex
%% Abspann für die Leseansicht.
%% Der Schalter \ifkorrekturansicht ist bereits durch den Vorspann gesetzt.

%% latex-abspann.tex
%% Gemeinsamer Abspann für Korrekturansicht und Leseansicht.
%% Setzt den Schalter \ifkorrekturansicht voraus (gesetzt in den
%% einbindenden Dateien latex-korrekturansicht-abspann.tex bzw.
%% latex-leseansicht-abspann.tex).
%% ---------------------------------------------------------------

\normalsize

% Das esempio-Environment wird nur in der Leseansicht benötigt
\ifkorrekturansicht\else
\newenvironment{esempio}[3]%
{
    \vspace{1.5ex}
    \rlap{\underline{#1}}
    \par
    \setlength{\parindent}{0cm}
    \nopagebreak
    \leftskip=#2cm
    \rightskip=#3cm
}
{
    \par
}
\fi

\doendnotes{C}
\bigskip
\vfill

\clearpage

\footnotesize

\ifkorrekturansicht
  \lohead{\textsc{register}}
\fi

% theindex-Environment neu definieren ohne reledmac
\makeatletter
\renewenvironment{theindex}{%
  \ifkorrekturansicht
    \section*{\indexname}%
  \else
    \subsubsection*{Index der erwähnten Entitäten}%
  \fi
  \setlength{\parindent}{0pt}%
  \setlength{\parskip}{0pt plus 0.3pt}%
  \let\item\@idxitem
}{%
  \ifkorrekturansicht\clearpage\fi
}
\makeatother

\IfFileExists{\jobname-pw.ind}{\input{\jobname-pw.ind}}{}

% Quellenangabe nur in der Leseansicht
\ifkorrekturansicht\else
% Fallback-Definitionen, falls die .tex-Datei \titel etc. nicht gesetzt hat
\providecommand{\titel}{}
\providecommand{\editorInnen}{}
\providecommand{\dateiname}{\jobname}

\vspace{3cm}

\vfill

\footnotesize
\textsc{Quelle}: \titel. Herausgegeben von {\editorInnen}. In: \emph{Arthur Schnitzler: Briefwechsel mit Autorinnen und Autoren}.
 Digitale Edition, https://schnitzler-briefe.acdh.oeaw.ac.at/{\dateiname}.html (Stand \today)
\fi

\end{document}


