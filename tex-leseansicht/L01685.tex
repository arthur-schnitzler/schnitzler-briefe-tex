%% latex-leseansicht-vorspann.tex
%% Vorspann für die Leseansicht.
%% Lädt die gemeinsame Datei latex-vorspann.tex mit nicht gesetztem Schalter.

\newif\ifkorrekturansicht
\korrekturansichtfalse

\input{../tex-inputs/latex-vorspann}


         
         \renewcommand{\erwaehntePersonen}{Personen: Jacobo Tintoretto}
         \renewcommand{\erwaehnteOrte}{Orte: Edmund-Weiß-Gasse, Lido, Venedig, Wien, Österreich}
         \renewcommand{\erwaehnteWerke}{Werke: Bacchus und Ariadne}
               \section[Hugo von Hofmannsthal an Arthur Schnitzler, 23. 6. 1907]{ Hugo von Hofmannsthal an Arthur Schnitzler, 23. 6. 1907}\nopagebreak\mylabel{v}\rehead{ }\begin{ledgroupsized}[t]{13cm}\normalsize\beginnumbering \toendnotes[C]{\smallbreak\pagebreak[2]} \Standort{CUL, Schnitzler, B 43.}
\physDesc{Bildpostkarte, 199 Zeichen
\newline{}Handschrift: schwarze Tinte, deutsche Kurrent
\newline{}Versand: Stempel: »\nobreak{}\oindex{Venedig@\textbf{Venedig}|pwk}Venezia Sezioni Riunite, 23 6–07, 6S\nobreak{}«.  
\newline{}Ordnung: 1) mit Bleistift von unbekannter Hand nummeriert: »\strikeout{278}«  2) mit Bleistift von unbekannter Hand nummeriert:
                                    »281«}\buchAbdrucke{\weitereDrucke{Hugo von Hofmannsthal, Arthur Schnitzler: \emph{Briefwechsel}. Hg. Therese Nickl und Heinrich Schnitzler. Frankfurt am Main: \emph{S. Fischer} 1964, S. 229.} }\pstart{}{\pb}\textsc{Herrn}\pend{}\pstart{}\textsc{D\textsuperscript{r} Arthur Schnitzler}\pend{}\pstart{}\textsc{Wien}\oindex{Wien@\textbf{Wien}|pw}\pend{}\pstart{}\textsc{XVII Spöttelgasse 7}.\oindex{Edmund-Weiss-Gasse@\textbf{Edmund-Weiß-Gasse}|pw}\pend{}\pstart{}\textsc{Austria\oindex{Oesterreich@\textbf{Österreich}|pw}}\pend{}{\bigskip}\pstart
           \noindent{}\centering{}\textcolor{gray}{\textbf{{\pb}Venezia\oindex{Venedig@\textbf{Venedig}|pw}– Tintoretto\pwindex{Tintoretto, Jacobo 29.9.1518 – 31.05.1594@\textsc{Tintoretto, Jacobo} (29.9.1518 – 31.05.1594), \emph{Bildender Künstler}|pw} – Arianna e
                     Bacco\pwindex{Tintoretto, Jacobo 29.9.1518 – 31.05.1594@\textsc{Tintoretto, Jacobo} (29.9.1518 – 31.05.1594), \emph{Bildender Künstler}!Bacchus und Ariadne1576 – 1577@\strich\emph{Bacchus und Ariadne} {[}1576 – 1577{]}|pw}.}}\pend
           \pstart
           \raggedleft{}{\pb}Lido\oindex{Lido@\textbf{Lido}|pw}{ }23\textsuperscript{\textcolor{gray}{ten}}\pend
           \pstart
           Zu ſelten ſieht man ſich – und zu lange habe ich Sie nicht etwas vorleſen gehört! Auf
               Wiederſehen. Wir haben ſehr angenehme Tage!\pend
           \pstart \spacefill\mbox{Hugo.}\pend{}
         
         \endnumbering\mylabel{h}\end{ledgroupsized}  \newcommand{\dateiname}{L01685}\newcommand{\titel}{Hugo von Hofmannsthal an Arthur Schnitzler, 23. 6. 1907}\newcommand{\editorInnen}{Martin Anton Müller und Gerd-Hermann Susen}%% latex-leseansicht-abspann.tex
%% Abspann für die Leseansicht.
%% Der Schalter \ifkorrekturansicht ist bereits durch den Vorspann gesetzt.

%% latex-abspann.tex
%% Gemeinsamer Abspann für Korrekturansicht und Leseansicht.
%% Setzt den Schalter \ifkorrekturansicht voraus (gesetzt in den
%% einbindenden Dateien latex-korrekturansicht-abspann.tex bzw.
%% latex-leseansicht-abspann.tex).
%% ---------------------------------------------------------------

\normalsize

% Das esempio-Environment wird nur in der Leseansicht benötigt
\ifkorrekturansicht\else
\newenvironment{esempio}[3]%
{
    \vspace{1.5ex}
    \rlap{\underline{#1}}
    \par
    \setlength{\parindent}{0cm}
    \nopagebreak
    \leftskip=#2cm
    \rightskip=#3cm
}
{
    \par
}
\fi

\doendnotes{C}
\bigskip
\vfill

\clearpage

\footnotesize

\ifkorrekturansicht
  \lohead{\textsc{register}}
\fi

% theindex-Environment neu definieren ohne reledmac
\makeatletter
\renewenvironment{theindex}{%
  \ifkorrekturansicht
    \section*{\indexname}%
  \else
    \subsubsection*{Index der erwähnten Entitäten}%
  \fi
  \setlength{\parindent}{0pt}%
  \setlength{\parskip}{0pt plus 0.3pt}%
  \let\item\@idxitem
}{%
  \ifkorrekturansicht\clearpage\fi
}
\makeatother

\IfFileExists{\jobname-pw.ind}{\input{\jobname-pw.ind}}{}

% Quellenangabe nur in der Leseansicht
\ifkorrekturansicht\else
% Fallback-Definitionen, falls die .tex-Datei \titel etc. nicht gesetzt hat
\providecommand{\titel}{}
\providecommand{\editorInnen}{}
\providecommand{\dateiname}{\jobname}

\vspace{3cm}

\vfill

\footnotesize
\textsc{Quelle}: \titel. Herausgegeben von {\editorInnen}. In: \emph{Arthur Schnitzler: Briefwechsel mit Autorinnen und Autoren}.
 Digitale Edition, https://schnitzler-briefe.acdh.oeaw.ac.at/{\dateiname}.html (Stand \today)
\fi

\end{document}


      