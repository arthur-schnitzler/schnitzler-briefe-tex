%% latex-korrekturansicht-vorspann.tex
%% Vorspann für die Korrekturansicht.
%% Lädt die gemeinsame Datei latex-vorspann.tex mit gesetztem Schalter.

\newif\ifkorrekturansicht
\korrekturansichttrue

\input{../tex-inputs/latex-vorspann}


\section[Hugo von Hofmannsthal an Arthur Schnitzler, 24. 5. 1903]{L01293 Hugo von Hofmannsthal an Arthur Schnitzler, 24. 5. 1903}
\nopagebreak\mylabel{L01293v}
\rehead{ }\normalsize\beginnumbering\briefempfaengerindex{Schnitzler, Arthur@\textsc{Schnitzler, Arthur}!zzzHofmannsthal, Hugo von@\emph{von Hugo von Hofmannsthal}!1903-05-241@{24. 5. 1903}|(be}
\toendnotes[C]{\smallbreak\pagebreak[2]}\Standort{CUL, Schnitzler, B 43.}
\physDesc{Postkarte, 285 Zeichen
\newline{}Handschrift: schwarze Tinte, deutsche Kurrent
\newline{}Versand: 1) Stempel: »\nobreak{}\oindex{Rodaun@\textbf{Rodaun}, \emph{A.ADM4}|pwk}Rodaun, 25. 5. 03, 9|V\nobreak{}«.   2) Stempel: »\nobreak{}\oindex{IX., Alsergrund@\textbf{IX., Alsergrund}, \emph{A.ADM3}|pwk}Wien 9/3, 25. 5. 03, 5.N, Bestellt\nobreak{}«. 
\newline{}Schnitzler: mit Bleistift datiert: »25. 5. 903.« 
\newline{}Ordnung: 1) mit Bleistift von unbekannter Hand nummeriert: »\strikeout{213}«  2) mit Bleistift von unbekannter Hand nummeriert: »196«}
\buchAbdrucke{\weitereDrucke{Hugo von Hofmannsthal, Arthur Schnitzler: \emph{Briefwechsel}. Frankfurt am Main: \emph{S. Fischer} 1964, S. 169.} }\toendnotes[C]{\smallbreak}\pstart{}{\pb}\textsc{Herrn D\textsuperscript{r} Arthur Schnitzler}\pend{}\pstart{}\textsc{Wien}\oindex{Wien@\textbf{Wien}, \emph{A.ADM2}|pw}\pend{}\pstart{}\textsc{IX. Franckgasse 1}.\oindex{Frankgasse 1@\textbf{Frankgasse 1}, \emph{Wohngebäude (K.WHS)}|pw}\pend{}{\bigskip}\vspace{1em}
\pstart
           \noindent{}{\pb}lieber Arthur, ich
               ſtelle dem lieben Weſen\pwindex{Luggin, Marie 01.07.1867 – 11.02.1945@\textsc{Luggin, Marie} (01.07.1867 – 11.02.1945), \emph{Rezitator/Rezitatorin, Sekretär/Sekretärin, Vorleser/Vorleserin}|pwv}{ }\label{K_L01293-1v}\edtext{alles beliebige von mir zur
               Verfügung}{\lemma{\textnormal{\emph{alles … Verfügung}}}\Cendnote{\textnormal{Schnitzler schrieb am 25. 5. 1903 an
                  Maria Luggin\pwindex{Luggin, Marie 01.07.1867 – 11.02.1945@\textsc{Luggin, Marie} (01.07.1867 – 11.02.1945), \emph{Rezitator/Rezitatorin, Sekretär/Sekretärin, Vorleser/Vorleserin}|pwk}: »Sehr geehrtes
                     Fräulein, Hofmannsthal\pwindex{Hofmannsthal, Hugo von 1874-02-01 – 1929-07-15@\textsc{Hofmannsthal, Hugo von} (1874-02-01 – 1929-07-15), \emph{Schriftsteller/Schriftstellerin}|pw} ſowie Salten\pwindex{Salten, Felix 06.09.1869 – 08.10.1945@\textsc{Salten, Felix} (06.09.1869 – 08.10.1945), \emph{Schriftsteller/Schriftstellerin, Journalist/Journalistin, Chefredakteur/Chefredakteurin}|pw} ſtellen Ihnen alles beliebige für die von Ihnen
                     für Herbſt projektirte Vorleſung zur Verfügung. Wenden Sie sich nur
                     freundlichſt zur gegebenen Zeit mit Ihren Wünſchen an die Jenen; falls es Ihnen
                     unbequem iſt, ſo können Sie die Sachen auch auf dem Umweg über mich ſehen. Mit
                     verbindlichſtem Gruß verehrtes Fräulein bin ich Ihr ſehr ergebener Arthur Schnitzler« (Zitiert nach
                        dem Auktionskatalog des \emph{Dorotheum}, Autographen, Handschriften, Urkunden, 4. 6. 2018.)}}}\label{K_L01293-1}. Sie ſoll nur ſeinerzeit an mich ſchreiben, was ſie
               haben will.\pend
           
\pstart
           Glückliche Reise! \pend
           
\pstart
           Von Herzen{\\[\baselineskip]}\spacefill\mbox{Hugo}\pend
           \leftskip=0em{}
\pstart
           \noindent{}Sonntag.\pend
           
\pstart
           \label{T_L01293-1v}\edtext{\textsc{Bitte vielmals um ein Exemplar »Reigen\pwindex{Reigen. Zehn Dialoge@\emph{Reigen. Zehn Dialoge}|pw}«}{\\}und der Richard\pwindex{Beer-Hofmann, Richard 1866-07-11 – 1945-09-26@\textsc{Beer-Hofmann, Richard} (1866-07-11 – 1945-09-26), \emph{Schriftsteller/Schriftstellerin}|pw} auch.}{\lemma{\textnormal{\emph{Bitte … auch.}}}\Cendnote{\textnormal{quer am rechten Rand}}}\label{T_L01293-1}\pend
           \selectlanguage{ngerman}\endnumbering\briefempfaengerindex{Schnitzler, Arthur@\textsc{Schnitzler, Arthur}!zzzHofmannsthal, Hugo von@\emph{von Hugo von Hofmannsthal}!1903-05-241@{24. 5. 1903}|)be}\mylabel{L01293h}  \normalsize

\doendnotes{C}
\bigskip
\vfill

\clearpage

\footnotesize

\lohead{\textsc{register}}

% Definiere theindex-Environment komplett neu ohne reledmac
\makeatletter
\renewenvironment{theindex}{%
  \section*{\indexname}%
  \setlength{\parindent}{0pt}%
  \setlength{\parskip}{0pt plus 0.3pt}%
  \let\item\@idxitem
}{%
  \clearpage
}
\makeatother

\IfFileExists{\jobname-pw.ind}{\input{\jobname-pw.ind}}{}

\end{document}

      