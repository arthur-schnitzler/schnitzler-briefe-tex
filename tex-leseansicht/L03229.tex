%% latex-leseansicht-vorspann.tex
%% Vorspann für die Leseansicht.
%% Lädt die gemeinsame Datei latex-vorspann.tex mit nicht gesetztem Schalter.

\newif\ifkorrekturansicht
\korrekturansichtfalse

\input{../tex-inputs/latex-vorspann}

\begin{center}
            \textcolor{red}{ENTWURF, NICHT FERTIG KORRIGIERT}
                      \end{center}
            
         
         \renewcommand{\erwaehntePersonen}{Personen: Max Bernstein, Otto Brahm, Eleonora Duse, Gerhart Hauptmann, Hugo von Hofmannsthal, Alfred Kerr, Olga Schnitzler, Heinrich Schnitzler, Elisabeth Steinrück, Hermann Sudermann}
         \renewcommand{\erwaehnteInstitutionen}{Institutionen: Schiller-Theater}
         \renewcommand{\erwaehnteOrte}{Orte: Berlin, Dessauer Straße, Deutsches Theater Berlin, Wien}
         \renewcommand{\erwaehnteWerke}{Werke: Berliner Tageblatt, Der Schleier der Beatrice. Schauspiel in fünf Akten, Der Tag, Die neue Richtung von Paul Goldman. Wien 1903. Verlag L. Rosner, D’Mali. Schauspiel in vier Akten, Ein Brief, Neues Wiener Tagblatt, Verrohung in der Theaterkritik, Verrohung in der Theaterkritik [Teil I]}
               \section[ Paul Goldmann an Arthur Schnitzler, 10. 11. {[}1902{]}]{ Paul Goldmann an Arthur Schnitzler, 10. 11. {[}1902{]}}\nopagebreak\mylabel{v}\rehead{ }\begin{ledgroupsized}[t]{13cm}\normalsize\beginnumbering \toendnotes[C]{\smallbreak\pagebreak[2]} \Standort{DLA, A:Schnitzler, HS.NZ85.1.3172.}
\physDesc{Brief, 1 Blatt, 4 Seiten
\newline{}Handschrift: blaue Tinte, deutsche Kurrent
\newline{}Schnitzler: 1) mit Bleistift das Jahr »{[}1{]}902« vermerkt  2) mit rotem Buntstift sechs Unterstreichungen}\toendnotes[C]{\smallbreak}\pstart
           \noindent{}\raggedleft{}{\pb}\textcolor{gray}{\textbf{DESSAUERSTRASSE 19}}\oindex{Dessauer Strasse@\textbf{Dessauer Straße}|pw}\pend
           \pstart
           Berlin\oindex{Berlin@\textbf{Berlin}|pw}, 10. November.\pend
           \pstart\center{}Mein lieber Freund,\pend\pstart
           Ich habe fürchterlich viel zu thun u. komme erſt heut
               dazu, Dir vielmals für den \label{K_L03229-1v}\edtext{Ausſchnitt\pwindex{?? Werk@Nicht ermittelte Verfasserinnen und Verfasser!neue Richtung von Paul Goldman. Wien 1903. Verlag L. Rosner1902-11-01@\emph{Die neue Richtung von Paul Goldman. Wien 1903. Verlag L. Rosner} {[}1902-11-01{]}|pwv}}{\lemma{\textnormal{\emph{Ausſchnitt}}}\Cendnote{\textnormal{[O. V.:] \emph{Die neue Richtung von Paul
                        Goldman. Wien 1903. Verlag L. Rosner}\pwindex{?? Werk@Nicht ermittelte Verfasserinnen und Verfasser!neue Richtung von Paul Goldman. Wien 1903. Verlag L. Rosner1902-11-01@\emph{Die neue Richtung von Paul Goldman. Wien 1903. Verlag L. Rosner} {[}1902-11-01{]}|pwk}. In: \emph{Neues Wiener Tagblatt}\pwindex{?? Werk@Nicht ermittelte Verfasserinnen und Verfasser!Neues Wiener Tagblatt1867 – 1945@\emph{Neues Wiener Tagblatt} {[}1867 – 1945{]}|pwk}, Jg. 36, Nr. 301, 1. 11. 1902, S. 35.}}}\label{K_L03229-1h} aus dem N. W. T.\pwindex{?? Werk@Nicht ermittelte Verfasserinnen und Verfasser!Neues Wiener Tagblatt1867 – 1945@\emph{Neues Wiener Tagblatt} {[}1867 – 1945{]}|pw} und Deinen lieben Brief zu danken.\pend
           \pstart
           Die guten Nachrichten von \textsc{Olga\pwindex{Schnitzler, Olga 17.01.1882 – 13.01.1970@\textsc{Schnitzler, Olga} (17.01.1882 – 13.01.1970), \emph{Schauspielerin, Sängerin}|pw}} und Deinem Sohne\pwindex{Schnitzler, Heinrich 09.08.1902 – 12.07.1982@\textsc{Schnitzler, Heinrich} (09.08.1902 – 12.07.1982), \emph{Regisseur, Schauspieler}|pwv} haben
               mich ſehr erfreut. Grüße ſie alle Beide recht herzlich. Wie denkt \textsc{Heinrich Schnitzler\pwindex{Schnitzler, Heinrich 09.08.1902 – 12.07.1982@\textsc{Schnitzler, Heinrich} (09.08.1902 – 12.07.1982), \emph{Regisseur, Schauspieler}|pw}} über \textsc{Gerhart Hauptmann\pwindex{Hauptmann, Gerhart 15.11.1862 – 06.06.1946@\textsc{Hauptmann, Gerhart} (15.11.1862 – 06.06.1946), \emph{Schriftsteller}|pw}}?\pend
           \pstart
           Mit \textsc{Brahm\pwindex{Brahm, Otto 05.02.1856 – 28.11.1912@\textsc{Brahm, Otto} (05.02.1856 – 28.11.1912), \emph{Theaterleiter, Regisseur}|pw}} wirſt Du wohl {\pb}inzwiſchen \label{K_L03229-2v}\edtext{einig}{\lemma{\textnormal{\emph{einig}}}\Cendnote{\textnormal{Bezug auf die Aufführung von \emph{Der Schleier der Beatrice}\pwindex{Schnitzler, Arthur 15.05.1862 – 21.10.1931@\textsc{Schnitzler, Arthur} (15.05.1862 – 21.10.1931), \emph{Schriftsteller, Mediziner}!Schleier der Beatrice. Schauspiel in fuenf Akten1900-12-01@\strich\emph{Der Schleier der Beatrice. Schauspiel in fünf Akten} {[}1900-12-01{]}|pwk} am Deutschen
                     Theater Berlin\oindex{Deutsches Theater Berlin@\textbf{Deutsches Theater Berlin}|pwk}}}}\label{K_L03229-2h} geworden ſein. Er hat ſich in der letzten \label{K_L03229-3v}\edtext{Cenſur-Affaire}{\lemma{\textnormal{\emph{Cenſur-Affaire}}}\Cendnote{\textnormal{rund um Max Bernstein\pwindex{Bernstein, Max 12.05.1854 – 08.03.1925@\textsc{Bernstein, Max} (12.05.1854 – 08.03.1925), \emph{Schriftsteller, Kritiker, Rechtsanwalt}|pwk}s vieraktiges
                  Schauspiel \emph{D’Mali}\pwindex{Bernstein, Max 12.05.1854 – 08.03.1925@\textsc{Bernstein, Max} (12.05.1854 – 08.03.1925), \emph{Schriftsteller, Kritiker, Rechtsanwalt}!DMali. Schauspiel in vier Akten1901@\strich\emph{D’Mali. Schauspiel in vier Akten} {[}1901{]}|pwk} wenige Tage zuvor}}}\label{K_L03229-3h}
               recht männlich und ſympathiſch benommen.\pend
           \pstart
           \textsc{Sudermann\pwindex{Sudermann, Hermann 30.09.1857 – 21.11.1928@\textsc{Sudermann, Hermann} (30.09.1857 – 21.11.1928), \emph{Schriftsteller}|pw}} miſcht in ſeinen \label{K_L03229-4v}\edtext{Artikel\pwindex{Sudermann, Hermann 30.09.1857 – 21.11.1928@\textsc{Sudermann, Hermann} (30.09.1857 – 21.11.1928), \emph{Schriftsteller}!Verrohung in der Theaterkritik [Teil I]1902-10-30@\strich\emph{Verrohung in der Theaterkritik [Teil I]} {[}1902-10-30{]}|pwv}}{\lemma{\textnormal{\emph{Artikel}}}\Cendnote{\textnormal{Gemeint war der erste Teil\pwindex{Sudermann, Hermann 30.09.1857 – 21.11.1928@\textsc{Sudermann, Hermann} (30.09.1857 – 21.11.1928), \emph{Schriftsteller}!Verrohung in der Theaterkritik [Teil I]1902-10-30@\strich\emph{Verrohung in der Theaterkritik [Teil I]} {[}1902-10-30{]}|pwkv} der fünfteiligen, am 30. 10., 7. 11., 17. 11., 25. 11. und
                     1. 12. 1902 in Abendausgaben des \emph{Berliner Tageblatt}\pwindex{?? Werk@Nicht ermittelte Verfasserinnen und Verfasser!Berliner Tageblatt1872 – 1939@\emph{Berliner Tageblatt} {[}1872 – 1939{]}|pwk}s erschienenen Feuilletonreihe \emph{Verrohung in der Theaterkritik}\pwindex{Sudermann, Hermann 30.09.1857 – 21.11.1928@\textsc{Sudermann, Hermann} (30.09.1857 – 21.11.1928), \emph{Schriftsteller}!Verrohung in der Theaterkritik1902-10-30 – 1902-12-01@\strich\emph{Verrohung in der Theaterkritik} {[}1902-10-30 – 1902-12-01{]}|pwk}: Hermann Sudermann\pwindex{Sudermann, Hermann 30.09.1857 – 21.11.1928@\textsc{Sudermann, Hermann} (30.09.1857 – 21.11.1928), \emph{Schriftsteller}|pwk}: \emph{Verrohung in der Theaterkritik}\pwindex{Sudermann, Hermann 30.09.1857 – 21.11.1928@\textsc{Sudermann, Hermann} (30.09.1857 – 21.11.1928), \emph{Schriftsteller}!Verrohung in der Theaterkritik [Teil I]1902-10-30@\strich\emph{Verrohung in der Theaterkritik [Teil I]} {[}1902-10-30{]}|pwk}. In: \emph{Berliner Tageblatt und Handels-Zeitung}\pwindex{?? Werk@Nicht ermittelte Verfasserinnen und Verfasser!Berliner Tageblatt1872 – 1939@\emph{Berliner Tageblatt} {[}1872 – 1939{]}|pwk}, Jg. 31,
                     Nr. 553, 30. 10. 1902, Abend-Ausgabe,
                     S. 1–3.}}}\label{K_L03229-4h} Wahres mit Albernem. Was er über den Gebrauch des Wortes
                  \label{K_L03229-123v}\edtext{»unliterariſch«}{\lemma{\textnormal{\emph{»unliterariſch«}}}\Cendnote{\textnormal{vgl. ebd.\pwindex{Sudermann, Hermann 30.09.1857 – 21.11.1928@\textsc{Sudermann, Hermann} (30.09.1857 – 21.11.1928), \emph{Schriftsteller}!Verrohung in der Theaterkritik [Teil I]1902-10-30@\strich\emph{Verrohung in der Theaterkritik [Teil I]} {[}1902-10-30{]}|pwkv}, S. 2}}}\label{K_L03229-123h} ſagte, war ſehr richtig. Auch die
                  \label{K_L03229-5v}\edtext{\textsc{\begin{otherlanguage}{french}gaminerie\end{otherlanguage}}}{\lemma{\textnormal{\emph{gaminerie}}}\Cendnote{\textnormal{französisch: Kinderei}}}\label{K_L03229-5h} unſeres
               Freundes \textsc{Kerr\pwindex{Kerr, Alfred 25.12.1867 – 12.10.1948@\textsc{Kerr, Alfred} (25.12.1867 – 12.10.1948), \emph{Schriftsteller, Kritiker}|pw}}, die er in ſeinem letzten \label{K_L03229-6v}\edtext{Feuilleton\pwindex{Sudermann, Hermann 30.09.1857 – 21.11.1928@\textsc{Sudermann, Hermann} (30.09.1857 – 21.11.1928), \emph{Schriftsteller}!Verrohung in der Theaterkritik1902-10-30 – 1902-12-01@\strich\emph{Verrohung in der Theaterkritik} {[}1902-10-30 – 1902-12-01{]}|pwv}}{\lemma{\textnormal{\emph{Feuilleton}}}\Cendnote{\textnormal{In Teil II behandelte Sudermann\pwindex{Sudermann, Hermann 30.09.1857 – 21.11.1928@\textsc{Sudermann, Hermann} (30.09.1857 – 21.11.1928), \emph{Schriftsteller}|pwk} Themen und verschiedene Kritiker, darunter Kerr\pwindex{Kerr, Alfred 25.12.1867 – 12.10.1948@\textsc{Kerr, Alfred} (25.12.1867 – 12.10.1948), \emph{Schriftsteller, Kritiker}|pwk}, dem er eine Aussage über Eleonora Duse\pwindex{Duse, Eleonora 03.10.1858 – 21.04.1924@\textsc{Duse, Eleonora} (03.10.1858 – 21.04.1924), \emph{Schauspielerin}|pwk} vorhielt.
                     (Nr. 568, S. 3.) }}}\label{K_L03229-6h} anführt, war recht garſtig. Vieles aber ließe
               ſich leicht widerlegen.\pend
           \pstart
           {\pb}Haſt Du den \label{K_L03229-7v}\edtext{»Brief\pwindex{Hofmannsthal, Hugo von 1874-02-01 – 1929-07-15@\textsc{Hofmannsthal, Hugo von} (1874-02-01 – 1929-07-15), \emph{Schriftsteller}!Brief1902-10-18 – 1902-10-19@\strich\emph{Ein Brief} {[}1902-10-18 – 1902-10-19{]}|pw}«}{\lemma{\textnormal{\emph{»Brief«}}}\Cendnote{\textnormal{Hugo von Hofmannsthal\pwindex{Hofmannsthal, Hugo von 1874-02-01 – 1929-07-15@\textsc{Hofmannsthal, Hugo von} (1874-02-01 – 1929-07-15), \emph{Schriftsteller}|pwk}: \emph{Ein Brief}\pwindex{Hofmannsthal, Hugo von 1874-02-01 – 1929-07-15@\textsc{Hofmannsthal, Hugo von} (1874-02-01 – 1929-07-15), \emph{Schriftsteller}!Brief1902-10-18 – 1902-10-19@\strich\emph{Ein Brief} {[}1902-10-18 – 1902-10-19{]}|pwk}. In: \emph{Der Tag.
                        Erster Teil: Illustrierte Zeitung}\pwindex{?? Werk@Nicht ermittelte Verfasserinnen und Verfasser!Tag19.12.1900 – 1934@\emph{Der Tag} {[}19.12.1900 – 1934{]}|pwk}, Nr. 489, 18. 10. 1902, S. [1–3] und Nr. 491, 19. 10. 1902, S. [1–3]. Eine Lektüre durch Schnitzler\pwindex{Schnitzler, Arthur 15.05.1862 – 21.10.1931@\textsc{Schnitzler, Arthur} (15.05.1862 – 21.10.1931), \emph{Schriftsteller, Mediziner}|pwk} ist nicht belegt, aber nicht zuletzt durch diesen Hinweis sehr wahrscheinlich.}}}\label{K_L03229-7h} von \textsc{Hoffmannsthal\pwindex{Hofmannsthal, Hugo von 1874-02-01 – 1929-07-15@\textsc{Hofmannsthal, Hugo von} (1874-02-01 – 1929-07-15), \emph{Schriftsteller}|pw}} geleſen, der vor einigen Wochen im »Tag\pwindex{?? Werk@Nicht ermittelte Verfasserinnen und Verfasser!Tag19.12.1900 – 1934@\emph{Der Tag} {[}19.12.1900 – 1934{]}|pw}«
               erſchienen iſt?\pend
           \pstart
           Geſtern{ }Nachmittag kam ich endlich dazu, \textsc{Liesl\pwindex{Steinrueck, Elisabeth 19.11.1885 – 07.04.1920@\textsc{Steinrück, Elisabeth} (19.11.1885 – 07.04.1920)|pw}} in ihrem \textsc{Boudoir} zu beſuchen. Sie wohnt recht
               ärmlich, das arme Ding, – aber ſie iſt ſehr vergnügt und ſpielt ſogar ſchon größere
                  \label{K_L03229-11v}\edtext{Rollen}{\lemma{\textnormal{\emph{Rollen}}}\Cendnote{\textnormal{am \emph{Schiller-Theater}\orgindex{Schiller-Theater@Schiller-Theater|pwk}, wo
                     Elisabeth Gussmann\pwindex{Steinrueck, Elisabeth 19.11.1885 – 07.04.1920@\textsc{Steinrück, Elisabeth} (19.11.1885 – 07.04.1920)|pwk} seit 1. 9. 1902 unter Vertrag stand}}}\label{K_L03229-11h}.\pend
           \pstart
           Ich bin wieder einmal durch Verſchiedenes (Schlafloſigkeit, nervöſe Störungen) ſehr
                  {\pb}niedergedrückt. Daher für heut nur dieſe wenigen Zeilen.\pend
           \pstart
           Laß’ bald von Dir hören und ſei vielmals und herzlichſt gegrüßt von {\\[\baselineskip]}Deinem {\\[\baselineskip]}\spacefill\mbox{Paul Goldm}\pend
           \leftskip=0em{}
         
         \endnumbering\mylabel{h}\end{ledgroupsized}  \newcommand{\dateiname}{L03229}\newcommand{\titel}{Paul Goldmann an Arthur Schnitzler, 10. 11. [1902]}\newcommand{\editorInnen}{Martin Anton Müller und Laura Untner}%% latex-leseansicht-abspann.tex
%% Abspann für die Leseansicht.
%% Der Schalter \ifkorrekturansicht ist bereits durch den Vorspann gesetzt.

%% latex-abspann.tex
%% Gemeinsamer Abspann für Korrekturansicht und Leseansicht.
%% Setzt den Schalter \ifkorrekturansicht voraus (gesetzt in den
%% einbindenden Dateien latex-korrekturansicht-abspann.tex bzw.
%% latex-leseansicht-abspann.tex).
%% ---------------------------------------------------------------

\normalsize

% Das esempio-Environment wird nur in der Leseansicht benötigt
\ifkorrekturansicht\else
\newenvironment{esempio}[3]%
{
    \vspace{1.5ex}
    \rlap{\underline{#1}}
    \par
    \setlength{\parindent}{0cm}
    \nopagebreak
    \leftskip=#2cm
    \rightskip=#3cm
}
{
    \par
}
\fi

\doendnotes{C}
\bigskip
\vfill

\clearpage

\footnotesize

\ifkorrekturansicht
  \lohead{\textsc{register}}
\fi

% theindex-Environment neu definieren ohne reledmac
\makeatletter
\renewenvironment{theindex}{%
  \ifkorrekturansicht
    \section*{\indexname}%
  \else
    \subsubsection*{Index der erwähnten Entitäten}%
  \fi
  \setlength{\parindent}{0pt}%
  \setlength{\parskip}{0pt plus 0.3pt}%
  \let\item\@idxitem
}{%
  \ifkorrekturansicht\clearpage\fi
}
\makeatother

\IfFileExists{\jobname-pw.ind}{\input{\jobname-pw.ind}}{}

% Quellenangabe nur in der Leseansicht
\ifkorrekturansicht\else
% Fallback-Definitionen, falls die .tex-Datei \titel etc. nicht gesetzt hat
\providecommand{\titel}{}
\providecommand{\editorInnen}{}
\providecommand{\dateiname}{\jobname}

\vspace{3cm}

\vfill

\footnotesize
\textsc{Quelle}: \titel. Herausgegeben von {\editorInnen}. In: \emph{Arthur Schnitzler: Briefwechsel mit Autorinnen und Autoren}.
 Digitale Edition, https://schnitzler-briefe.acdh.oeaw.ac.at/{\dateiname}.html (Stand \today)
\fi

\end{document}


      