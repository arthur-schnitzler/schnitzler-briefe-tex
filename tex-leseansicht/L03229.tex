%% latex-leseansicht-vorspann.tex
%% Vorspann für die Leseansicht.
%% Lädt die gemeinsame Datei latex-vorspann.tex mit nicht gesetztem Schalter.

\newif\ifkorrekturansicht
\korrekturansichtfalse

\input{../tex-inputs/latex-vorspann}


\section[ Paul Goldmann an Arthur Schnitzler, 10. 11. [1902]]{L03229 Paul Goldmann an Arthur Schnitzler,  10. 11. [1902]}
\nopagebreak\mylabel{L03229v}
\rehead{ }\normalsize\beginnumbering\briefempfaengerindex{Schnitzler, Arthur@\textsc{Schnitzler, Arthur}!zzzGoldmann, Paul@\emph{von Paul Goldmann}!1902-11-101@{10. 11. [1902]}|(be}
\toendnotes[C]{\smallbreak\pagebreak[2]}
\correspDesc{Versand  durch Paul Goldmann am 10. 11. [1902] in Berlin
\newline{}Erhalt  durch Arthur Schnitzler im Zeitraum [11. 11. 1902 – 15. 11. 1902?] in Wien}\toendnotes[C]{\smallbreak}
\Standort{DLA, A:Schnitzler, HS.NZ85.1.3172.}
\physDesc{Brief, 1 Blatt, 4 Seiten, 1221 Zeichen
\newline{}Handschrift: blaue Tinte, deutsche Kurrent
\newline{}Schnitzler: 1) mit Bleistift das Jahr »902« vermerkt  2) mit rotem Buntstift sechs Unterstreichungen}\toendnotes[C]{\smallbreak}
\pstart
           \raggedleft{}{\pb}\textcolor{gray}{\textbf{DESSAUERSTRASSE 19}}\oindex{Dessauer Straße@\textbf{Dessauer Straße}, \emph{Straße}|pw}\pend
           
\pstart
           Berlin\oindex{Berlin@\textbf{Berlin}, \emph{Hauptstadt}|pw}, 10. November.\pend
           
\pstart\center{}Mein lieber Freund,\pend\vspace{0.5em}
\pstart
           Ich habe fürchterlich viel zu thun u. komme erſt heut
               dazu, Dir vielmals für den \label{K_L03229-1v}\edtext{Ausſchnitt\pwindex{neue Richtung von Paul Goldman. Wien 1903. Verlag L. Rosner@\emph{Die neue Richtung von Paul Goldman. Wien 1903. Verlag L. Rosner}|pwv}}{\lemma{\textnormal{\emph{Ausschnitt}}}\Cendnote{\textnormal{[O. V.]: \emph{Die neue Richtung von Paul
                        Goldman. Wien 1903. Verlag L. Rosner}\pwindex{neue Richtung von Paul Goldman. Wien 1903. Verlag L. Rosner@\emph{Die neue Richtung von Paul Goldman. Wien 1903. Verlag L. Rosner}|pwk}. In: \emph{Neues Wiener Tagblatt}\pwindex{Neues Wiener Tagblatt@\emph{Neues Wiener Tagblatt}|pwk}, Jg. 36, Nr. 301, 1. 11. 1902, S. 35.}}}\label{K_L03229-1} aus dem N. W. T.\pwindex{Neues Wiener Tagblatt@\emph{Neues Wiener Tagblatt}|pw} und Deinen lieben Brief zu danken.\pend
           
\pstart
           Die guten Nachrichten von \textsc{Olga\pwindex{Schnitzler, Olga 17.\,1.\,1882 Wien – 13.\,1.\,1970 Lugano@\textsc{Schnitzler, Olga} (17.\,1.\,1882 Wien – 13.\,1.\,1970 Lugano), \emph{Schauspielerin, Sängerin}|pw}} und Deinem Sohne\pwindex{Schnitzler, Heinrich 9.\,8.\,1902 Hinterbrühl – 12.\,7.\,1982 Wien@\textsc{Schnitzler, Heinrich} (9.\,8.\,1902 Hinterbrühl – 12.\,7.\,1982 Wien), \emph{Regisseur, Schauspieler}|pwv} haben
               mich{ }ſehr erfreut. Grüße{ }ſie alle Beide recht herzlich. Wie denkt \textsc{Heinrich Schnitzler\pwindex{Schnitzler, Heinrich 9.\,8.\,1902 Hinterbrühl – 12.\,7.\,1982 Wien@\textsc{Schnitzler, Heinrich} (9.\,8.\,1902 Hinterbrühl – 12.\,7.\,1982 Wien), \emph{Regisseur, Schauspieler}|pw}} über \textsc{Gerhart Hauptmann\pwindex{Hauptmann, Gerhart 15.\,11.\,1862 Szczawno-Zdrój – 6.\,6.\,1946 Jagniątków@\textsc{Hauptmann, Gerhart} (15.\,11.\,1862 Szczawno-Zdrój – 6.\,6.\,1946 Jagniątków), \emph{Schriftsteller}|pw}}?\pend
           
\pstart
           Mit \textsc{Brahm\pwindex{Brahm, Otto 5.\,2.\,1856 Hamburg – 28.\,11.\,1912 Berlin@\textsc{Brahm, Otto} (5.\,2.\,1856 Hamburg – 28.\,11.\,1912 Berlin), \emph{Theaterleiter, Regisseur}|pw}} wirſt Du wohl {\pb}inzwiſchen \label{K_L03229-2v}\edtext{einig}{\lemma{\textnormal{\emph{einig}}}\Cendnote{\textnormal{Bezug auf die Aufführung von \emph{Der Schleier der Beatrice}\pwindex{Schnitzler, Arthur 15.\,5.\,1862 Wien – 21.\,10.\,1931 ebd.@\textsc{Schnitzler, Arthur} (15.\,5.\,1862 Wien – 21.\,10.\,1931 ebd.), \emph{Schriftsteller, Mediziner}!Schleier der Beatrice. Schauspiel in fünf Akten@\strich\emph{Der Schleier der Beatrice. Schauspiel in fünf Akten}|pwk} am Deutschen
                     Theater Berlin\oindex{Deutsches Theater Berlin@\textbf{Deutsches Theater Berlin}, \emph{Theater}|pwk}}}}\label{K_L03229-2} geworden{ }ſein. Er hat{ }ſich in der letzten \label{K_L03229-3v}\edtext{Cenſur-Affaire}{\lemma{\textnormal{\emph{Censur-Affaire}}}\Cendnote{\textnormal{rund um Max Bernsteins\pwindex{Bernstein, Max 12.\,5.\,1854 Fürth – 8.\,3.\,1925 München@\textsc{Bernstein, Max} (12.\,5.\,1854 Fürth – 8.\,3.\,1925 München), \emph{Schriftsteller, Kritiker, Rechtsanwalt}|pwk} vieraktiges
                  Schauspiel \emph{D’Mali}\pwindex{Bernstein, Max 12.\,5.\,1854 Fürth – 8.\,3.\,1925 München@\textsc{Bernstein, Max} (12.\,5.\,1854 Fürth – 8.\,3.\,1925 München), \emph{Schriftsteller, Kritiker, Rechtsanwalt}!D’Mali. Schauspiel in vier Akten@\strich\emph{D’Mali. Schauspiel in vier Akten}|pwk} wenige Tage zuvor}}}\label{K_L03229-3}
               recht männlich und{ }ſympathiſch benommen.\pend
           
\pstart
           \textsc{Sudermann\pwindex{Sudermann, Hermann 30.\,9.\,1857 Macikai – 21.\,11.\,1928 Berlin@\textsc{Sudermann, Hermann} (30.\,9.\,1857 Macikai – 21.\,11.\,1928 Berlin), \emph{Schriftsteller}|pw}} miſcht in{ }ſeinen \label{K_L03229-4v}\edtext{Artikel\pwindex{Sudermann, Hermann 30.\,9.\,1857 Macikai – 21.\,11.\,1928 Berlin@\textsc{Sudermann, Hermann} (30.\,9.\,1857 Macikai – 21.\,11.\,1928 Berlin), \emph{Schriftsteller}!Verrohung in der Theaterkritik [Teil I]@\strich\emph{Verrohung in der Theaterkritik [Teil I]}|pwv}}{\lemma{\textnormal{\emph{Artikel}}}\Cendnote{\textnormal{Gemeint war der erste Teil von \emph{Verrohung in der Theaterkritik}\pwindex{Sudermann, Hermann 30.\,9.\,1857 Macikai – 21.\,11.\,1928 Berlin@\textsc{Sudermann, Hermann} (30.\,9.\,1857 Macikai – 21.\,11.\,1928 Berlin), \emph{Schriftsteller}!Verrohung in der Theaterkritik@\strich\emph{Verrohung in der Theaterkritik}|pwk}, eine fünfteilige Feuilletonreihe, die in Abendausgaben des \emph{Berliner Tageblatts}\pwindex{Berliner Tageblatt@\emph{Berliner Tageblatt}|pwk} erschien: Hermann Sudermann\pwindex{Sudermann, Hermann 30.\,9.\,1857 Macikai – 21.\,11.\,1928 Berlin@\textsc{Sudermann, Hermann} (30.\,9.\,1857 Macikai – 21.\,11.\,1928 Berlin), \emph{Schriftsteller}|pwk}: \emph{Verrohung in der Theaterkritik}\pwindex{Sudermann, Hermann 30.\,9.\,1857 Macikai – 21.\,11.\,1928 Berlin@\textsc{Sudermann, Hermann} (30.\,9.\,1857 Macikai – 21.\,11.\,1928 Berlin), \emph{Schriftsteller}!Verrohung in der Theaterkritik [Teil I]@\strich\emph{Verrohung in der Theaterkritik [Teil I]}|pwk}. In: \emph{Berliner Tageblatt und Handels-Zeitung}\pwindex{Berliner Tageblatt@\emph{Berliner Tageblatt}|pwk}, Jg. 31,
                     Nr. 553, 30. 10. 1902, Abend-Ausgabe,
                  S. 1–3. Die weiteren Beiträge erschienen am 7. 11. 1902, am 17. 11. 1902, am 25. 11. 1902 und am
                  1. 12. 1902.}}}\label{K_L03229-4} Wahres mit Albernem. Was er über den Gebrauch des Wortes
                  \label{K_L03229-5v}\edtext{»unliterariſch«}{\lemma{\textnormal{\emph{»unliterarisch«}}}\Cendnote{\textnormal{Vgl. ebd.\pwindex{Sudermann, Hermann 30.\,9.\,1857 Macikai – 21.\,11.\,1928 Berlin@\textsc{Sudermann, Hermann} (30.\,9.\,1857 Macikai – 21.\,11.\,1928 Berlin), \emph{Schriftsteller}!Verrohung in der Theaterkritik [Teil I]@\strich\emph{Verrohung in der Theaterkritik [Teil I]}|pwkv}, S. 2.}}}\label{K_L03229-5}{ }ſagte, war{ }ſehr richtig. Auch die
                  \label{K_L03229-6v}\edtext{\textsc{\begin{otherlanguage}{french}gaminerie\end{otherlanguage}}}{\lemma{\textnormal{\emph{gaminerie}}}\Cendnote{\textnormal{französisch: Kinderei}}}\label{K_L03229-6} unſeres
               Freundes \textsc{Kerr\pwindex{Kerr, Alfred 25.\,12.\,1867 Breslau – 12.\,10.\,1948 Hamburg@\textsc{Kerr, Alfred} (25.\,12.\,1867 Breslau – 12.\,10.\,1948 Hamburg), \emph{Schriftsteller, Kritiker}|pw}}, die er in{ }ſeinem letzten \label{K_L03229-7v}\edtext{Feuilleton\pwindex{Sudermann, Hermann 30.\,9.\,1857 Macikai – 21.\,11.\,1928 Berlin@\textsc{Sudermann, Hermann} (30.\,9.\,1857 Macikai – 21.\,11.\,1928 Berlin), \emph{Schriftsteller}!Verrohung in der Theaterkritik@\strich\emph{Verrohung in der Theaterkritik}|pwv}}{\lemma{\textnormal{\emph{Feuilleton}}}\Cendnote{\textnormal{In Teil II der Feuilletonreihe \emph{Verrohung in der Theaterkritik}\pwindex{Sudermann, Hermann 30.\,9.\,1857 Macikai – 21.\,11.\,1928 Berlin@\textsc{Sudermann, Hermann} (30.\,9.\,1857 Macikai – 21.\,11.\,1928 Berlin), \emph{Schriftsteller}!Verrohung in der Theaterkritik@\strich\emph{Verrohung in der Theaterkritik}|pwk} behandelte Sudermann\pwindex{Sudermann, Hermann 30.\,9.\,1857 Macikai – 21.\,11.\,1928 Berlin@\textsc{Sudermann, Hermann} (30.\,9.\,1857 Macikai – 21.\,11.\,1928 Berlin), \emph{Schriftsteller}|pwk} Themen und verschiedene Kritiker,
                  darunter Kerr\pwindex{Kerr, Alfred 25.\,12.\,1867 Breslau – 12.\,10.\,1948 Hamburg@\textsc{Kerr, Alfred} (25.\,12.\,1867 Breslau – 12.\,10.\,1948 Hamburg), \emph{Schriftsteller, Kritiker}|pwk}, dem er eine Aussage über Eleonora Duse\pwindex{Duse, Eleonora 3.\,10.\,1858 Vigevano – 21.\,4.\,1924 Pittsburgh@\textsc{Duse, Eleonora} (3.\,10.\,1858 Vigevano – 21.\,4.\,1924 Pittsburgh), \emph{Schauspielerin}|pwk} vorhielt. 
                  Hermann Sudermann\pwindex{Sudermann, Hermann 30.\,9.\,1857 Macikai – 21.\,11.\,1928 Berlin@\textsc{Sudermann, Hermann} (30.\,9.\,1857 Macikai – 21.\,11.\,1928 Berlin), \emph{Schriftsteller}|pwk}: \emph{Verrohung in der Theaterkritik. II}\pwindex{Sudermann, Hermann 30.\,9.\,1857 Macikai – 21.\,11.\,1928 Berlin@\textsc{Sudermann, Hermann} (30.\,9.\,1857 Macikai – 21.\,11.\,1928 Berlin), \emph{Schriftsteller}!Verrohung in der Theaterkritik. II@\strich\emph{Verrohung in der Theaterkritik. II}|pwk}. In: \emph{Berliner Tageblatt und Handels-Zeitung}\pwindex{Berliner Tageblatt@\emph{Berliner Tageblatt}|pwk}, Jg. 31,
                     Nr. 568, 7. 11. 1902, Abend-Ausgabe,
                     S. 3–4.}}}\label{K_L03229-7} anführt, war recht garſtig. Vieles aber ließe{ }ſich leicht
               widerlegen.\pend
           
\pstart
           {\pb}Haſt Du den \label{K_L03229-8v}\edtext{»Brief\pwindex{Hofmannsthal, Hugo von 1.\,2.\,1874 Wien – 15.\,7.\,1929 Rodaun@\textsc{Hofmannsthal, Hugo von} (1.\,2.\,1874 Wien – 15.\,7.\,1929 Rodaun), \emph{Schriftsteller}!Brief@\strich\emph{Ein Brief}|pw}«}{\lemma{\textnormal{\emph{»Brief«}}}\Cendnote{\textnormal{Hugo von Hofmannsthal\pwindex{Hofmannsthal, Hugo von 1.\,2.\,1874 Wien – 15.\,7.\,1929 Rodaun@\textsc{Hofmannsthal, Hugo von} (1.\,2.\,1874 Wien – 15.\,7.\,1929 Rodaun), \emph{Schriftsteller}|pwk}: \emph{Ein Brief}\pwindex{Hofmannsthal, Hugo von 1.\,2.\,1874 Wien – 15.\,7.\,1929 Rodaun@\textsc{Hofmannsthal, Hugo von} (1.\,2.\,1874 Wien – 15.\,7.\,1929 Rodaun), \emph{Schriftsteller}!Brief@\strich\emph{Ein Brief}|pwk}. In: \emph{Der Tag.
                        Erster Teil: Illustrierte Zeitung}\pwindex{Tag@\emph{Der Tag}|pwk}, Nr. 489, 18. 10. 1902, S. [1–3] und Nr. 491, 19. 10. 1902, S. [1–3]. Eine Lektüre durch Schnitzler ist nicht belegt, aber nicht zuletzt durch
                  diesen Hinweis sehr wahrscheinlich.}}}\label{K_L03229-8} von \textsc{Hoffmannsthal\pwindex{Hofmannsthal, Hugo von 1.\,2.\,1874 Wien – 15.\,7.\,1929 Rodaun@\textsc{Hofmannsthal, Hugo von} (1.\,2.\,1874 Wien – 15.\,7.\,1929 Rodaun), \emph{Schriftsteller}|pw}} geleſen, der vor einigen Wochen im »Tag\pwindex{Tag@\emph{Der Tag}|pw}«
               erſchienen iſt?\pend
           
\pstart
           Geſtern{ }Nachmittag kam ich endlich dazu, \textsc{Liesl\pwindex{Steinrück, Elisabeth 19.\,11.\,1885 – 7.\,4.\,1920 Partenkirchen@\textsc{Steinrück, Elisabeth} (19.\,11.\,1885 – 7.\,4.\,1920 Partenkirchen)|pw}} in ihrem \textsc{Boudoir} zu beſuchen. Sie wohnt recht
               ärmlich, das arme Ding, – aber{ }ſie iſt{ }ſehr vergnügt und{ }ſpielt{ }ſogar{ }ſchon größere
                  \label{K_L03229-9v}\edtext{Rollen}{\lemma{\textnormal{\emph{Rollen}}}\Cendnote{\textnormal{am \emph{Schiller-Theater}\orgindex{Schiller-Theater@Schiller-Theater|pwk}, wo
                     Elisabeth Gussmann\pwindex{Steinrück, Elisabeth 19.\,11.\,1885 – 7.\,4.\,1920 Partenkirchen@\textsc{Steinrück, Elisabeth} (19.\,11.\,1885 – 7.\,4.\,1920 Partenkirchen)|pwk} seit 1. 9. 1902 unter Vertrag stand}}}\label{K_L03229-9}.\pend
           
\pstart
           Ich bin wieder einmal durch Verſchiedenes (Schlafloſigkeit, nervöſe Störungen){ }ſehr
                  {\pb}niedergedrückt. Daher für heut nur dieſe wenigen Zeilen.\pend
           
\pstart
           Laß’ bald von Dir hören und{ }ſei vielmals und herzlichſt gegrüßt von {\\[\baselineskip]}Deinem {\\[\baselineskip]}\spacefill\mbox{Paul Goldm}\pend
           \leftskip=0em{}\selectlanguage{ngerman}\endnumbering\briefempfaengerindex{Schnitzler, Arthur@\textsc{Schnitzler, Arthur}!zzzGoldmann, Paul@\emph{von Paul Goldmann}!1902-11-101@{10. 11. [1902]}|)be}\mylabel{L03229h}  \newcommand{\dateiname}{L03229}\newcommand{\titel}{Paul Goldmann an Arthur Schnitzler, 10. 11. [1902]}\newcommand{\editorInnen}{Martin Anton Müller und Laura Untner}%% latex-leseansicht-abspann.tex
%% Abspann für die Leseansicht.
%% Der Schalter \ifkorrekturansicht ist bereits durch den Vorspann gesetzt.

%% latex-abspann.tex
%% Gemeinsamer Abspann für Korrekturansicht und Leseansicht.
%% Setzt den Schalter \ifkorrekturansicht voraus (gesetzt in den
%% einbindenden Dateien latex-korrekturansicht-abspann.tex bzw.
%% latex-leseansicht-abspann.tex).
%% ---------------------------------------------------------------

\normalsize

% Das esempio-Environment wird nur in der Leseansicht benötigt
\ifkorrekturansicht\else
\newenvironment{esempio}[3]%
{
    \vspace{1.5ex}
    \rlap{\underline{#1}}
    \par
    \setlength{\parindent}{0cm}
    \nopagebreak
    \leftskip=#2cm
    \rightskip=#3cm
}
{
    \par
}
\fi

\doendnotes{C}
\bigskip
\vfill

\clearpage

\footnotesize

\ifkorrekturansicht
  \lohead{\textsc{register}}
\fi

% theindex-Environment neu definieren ohne reledmac
\makeatletter
\renewenvironment{theindex}{%
  \ifkorrekturansicht
    \section*{\indexname}%
  \else
    \subsubsection*{Index der erwähnten Entitäten}%
  \fi
  \setlength{\parindent}{0pt}%
  \setlength{\parskip}{0pt plus 0.3pt}%
  \let\item\@idxitem
}{%
  \ifkorrekturansicht\clearpage\fi
}
\makeatother

\IfFileExists{\jobname-pw.ind}{\input{\jobname-pw.ind}}{}

% Quellenangabe nur in der Leseansicht
\ifkorrekturansicht\else
% Fallback-Definitionen, falls die .tex-Datei \titel etc. nicht gesetzt hat
\providecommand{\titel}{}
\providecommand{\editorInnen}{}
\providecommand{\dateiname}{\jobname}

\vspace{3cm}

\vfill

\footnotesize
\textsc{Quelle}: \titel. Herausgegeben von {\editorInnen}. In: \emph{Arthur Schnitzler: Briefwechsel mit Autorinnen und Autoren}.
 Digitale Edition, https://schnitzler-briefe.acdh.oeaw.ac.at/{\dateiname}.html (Stand \today)
\fi

\end{document}


