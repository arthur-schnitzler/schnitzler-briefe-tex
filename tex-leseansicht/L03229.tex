%% latex-leseansicht-vorspann.tex
%% Vorspann für die Leseansicht.
%% Lädt die gemeinsame Datei latex-vorspann.tex mit nicht gesetztem Schalter.

\newif\ifkorrekturansicht
\korrekturansichtfalse

\input{../tex-inputs/latex-vorspann}

\begin{center}
            \textcolor{red}{ENTWURF, NICHT FERTIG KORRIGIERT}
                      \end{center}
            
         
         \renewcommand{\erwaehntePersonen}{Personen: Otto Brahm, Olga Schnitzler}
         \renewcommand{\erwaehnteOrte}{Orte: Berlin, Dessauer Straße, Wien}
         \renewcommand{\erwaehnteWerke}{}
               \section[ Paul Goldmann an Arthur Schnitzler, 10. 11. {[}1902{]}]{ Paul Goldmann an Arthur Schnitzler, 10. 11. {[}1902{]}}\nopagebreak\mylabel{v}\rehead{ }\begin{ledgroupsized}[t]{13cm}\normalsize\beginnumbering \toendnotes[C]{\smallbreak\pagebreak[2]} \Standort{DLA, A:Schnitzler, HS.NZ85.1.3172.}
\physDesc{Brief, 1 Blatt, 4 Seiten
\newline{}Handschrift: blaue Tinte, deutsche Kurrent
\newline{}Schnitzler: 1) mit Bleistift das Jahr »{[}1{]}902«
                                            vermerkt  2) mit rotem Buntstift sechs Unterstreichungen}\toendnotes[C]{\smallbreak}\pstart
           \noindent{}\raggedleft{}{\pb}\textcolor{gray}{\textbf{DESSAUERSTRASSE 19}}\oindex{Dessauer Strasse@\textbf{Dessauer Straße}|pw}\pend
           \pstart
           Berlin\oindex{Berlin@\textbf{Berlin}|pw}, 10. November.\pend
           \pstart\center{}Mein lieber Freund,\pend\pstart
           Ich habe fürchterlich viel zu thun u. komme erſt heut dazu, Dir
                    vielmals für den Ausſchnitt\textcolor{red}{\textsuperscript{\textbf{KEY}}} aus dem N. W. T.\textcolor{red}{\textsuperscript{\textbf{KEY}}} und Deinen lieben Brief zu danken. \pend
           \pstart
           Die guten Nachrichten von \textsc{Olga\pwindex{Schnitzler, Olga 17.01.1882 – 13.01.1970@\textsc{Schnitzler, Olga} (17.01.1882 – 13.01.1970), \emph{Schauspielerin, Sängerin}|pw}} und Deinem Sohne\textcolor{red}{\textsuperscript{\textbf{KEY}}} haben mich ſehr erfreut.
                    Grüße ſie alle Beide recht herzlich. Wie denkt \textsc{Heinrich\textcolor{red}{\textsuperscript{\textbf{KEY}}}}\textsc{Schnitzler\textcolor{red}{\textsuperscript{\textbf{KEY}}}} über \textsc{Gerhart\textcolor{red}{\textsuperscript{\textbf{KEY}}}}\textsc{Hauptmann\textcolor{red}{\textsuperscript{\textbf{KEY}}}}? \pend
           \pstart
           Mit \textsc{Brahm\pwindex{Brahm, Otto 05.02.1856 – 28.11.1912@\textsc{Brahm, Otto} (05.02.1856 – 28.11.1912), \emph{Theaterleiter, Regisseur}|pw}} wirſt Du wohl {\pb} inzwiſchen einig geworden
                    ſein. Er hat ſich in der letzten CenſurAffaire recht\label{XXXXv}\edtext{\textcolor{gray}{n}[Kommentar:
                        nachdem\u0020Goldmann\u0020das\u0020kleine\u0020d\u0020im\u0020Laufe\u0020der\u0020Zeit\u0020immer\u0020schlampiger\u0020schreibt,\u0020könnte\u0020hier\u0020auch\u0020dämlich\u0020stehen]}{\lemma{\textnormal{\emph{XXXX Lemmafehler}}}\Cendnote{\textnormal{}}}\label{XXXX}ämlich und ſympathiſch benommen. \pend
           \pstart
           \textsc{Sudermann\textcolor{red}{\textsuperscript{\textbf{KEY}}}} miſcht in ſeinen Artikel\textcolor{red}{\textsuperscript{\textbf{KEY}}} Wahres mit Albernem.
                    Was er über den Gebrauch des Wortes »unliterariſch« ſagte, war ſehr richtig.
                    Auch die \textsc{gaminerie} unſeres Freundes \textsc{Kerr\textcolor{red}{\textsuperscript{\textbf{KEY}}}}, die er in ſeinem letzten Feuilleton\textcolor{red}{\textsuperscript{\textbf{KEY}}} anführt,
                    war recht garſtig. Vieles aber ließe ſich leicht widerlegen. {\pb}\pend
           \pstart
           Haſt Du den »Brief\textcolor{red}{\textsuperscript{\textbf{KEY}}}« von \textsc{Hoffmannsthal\textcolor{red}{\textsuperscript{\textbf{KEY}}}} geleſen, der vor einigen Wochen im »Tag\textcolor{red}{\textsuperscript{\textbf{KEY}}}«
                    erſchienen iſt? \pend
           \pstart
           GeſternNachmittag kam ich endlich dazu, \textsc{Liesl\textcolor{red}{\textsuperscript{\textbf{KEY}}}} in ihrem \textsc{Boudoir\textcolor{red}{\textsuperscript{\textbf{KEY}}}} zu beſuchen. Sie wohnt recht ärmlich, das arme Ding, – aber ſie iſt
                    vergnügt und ſpielt ſogar ſchon größere Rollen. \pend
           \pstart
           Ich bin wieder einmal durch Verſchiedenes (Schlafloſigkeit, nervöſe Störungen)
                    ſehr {\pb} niedergedrückt. Daher für
                        heut nur dieſe wenigen Zeilen. \pend
           \pstart
           {\\[\baselineskip]}Laß’ bald von Dir\pend
           \leftskip=0em{}\pstart
           {\\[\baselineskip]}hören und ſei vielmals\pend
           \leftskip=0em{}\pstart
           {\\[\baselineskip]}und herzlichſt gegrüßt von\pend
           \leftskip=0em{}\pstart
           {\\[\baselineskip]}Deinem\pend
           \leftskip=0em{}\pstart
           {\\[\baselineskip]}\spacefill\mbox{Paul Goldm }\pend
           \leftskip=0em{}
         
         \endnumbering\mylabel{h}\end{ledgroupsized}\begin{anhang}\end{anhang}\newcommand{\dateiname}{L03229}\newcommand{\titel}{Paul Goldmann an Arthur Schnitzler, 10. 11. [1902]}\newcommand{\editorInnen}{Martin Anton Müller und Laura Untner}%% latex-leseansicht-abspann.tex
%% Abspann für die Leseansicht.
%% Der Schalter \ifkorrekturansicht ist bereits durch den Vorspann gesetzt.

%% latex-abspann.tex
%% Gemeinsamer Abspann für Korrekturansicht und Leseansicht.
%% Setzt den Schalter \ifkorrekturansicht voraus (gesetzt in den
%% einbindenden Dateien latex-korrekturansicht-abspann.tex bzw.
%% latex-leseansicht-abspann.tex).
%% ---------------------------------------------------------------

\normalsize

% Das esempio-Environment wird nur in der Leseansicht benötigt
\ifkorrekturansicht\else
\newenvironment{esempio}[3]%
{
    \vspace{1.5ex}
    \rlap{\underline{#1}}
    \par
    \setlength{\parindent}{0cm}
    \nopagebreak
    \leftskip=#2cm
    \rightskip=#3cm
}
{
    \par
}
\fi

\doendnotes{C}
\bigskip
\vfill

\clearpage

\footnotesize

\ifkorrekturansicht
  \lohead{\textsc{register}}
\fi

% theindex-Environment neu definieren ohne reledmac
\makeatletter
\renewenvironment{theindex}{%
  \ifkorrekturansicht
    \section*{\indexname}%
  \else
    \subsubsection*{Index der erwähnten Entitäten}%
  \fi
  \setlength{\parindent}{0pt}%
  \setlength{\parskip}{0pt plus 0.3pt}%
  \let\item\@idxitem
}{%
  \ifkorrekturansicht\clearpage\fi
}
\makeatother

\IfFileExists{\jobname-pw.ind}{\input{\jobname-pw.ind}}{}

% Quellenangabe nur in der Leseansicht
\ifkorrekturansicht\else
% Fallback-Definitionen, falls die .tex-Datei \titel etc. nicht gesetzt hat
\providecommand{\titel}{}
\providecommand{\editorInnen}{}
\providecommand{\dateiname}{\jobname}

\vspace{3cm}

\vfill

\footnotesize
\textsc{Quelle}: \titel. Herausgegeben von {\editorInnen}. In: \emph{Arthur Schnitzler: Briefwechsel mit Autorinnen und Autoren}.
 Digitale Edition, https://schnitzler-briefe.acdh.oeaw.ac.at/{\dateiname}.html (Stand \today)
\fi

\end{document}


      