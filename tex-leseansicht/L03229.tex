%% latex-korrekturansicht-vorspann.tex
%% Vorspann für die Korrekturansicht.
%% Lädt die gemeinsame Datei latex-vorspann.tex mit gesetztem Schalter.

\newif\ifkorrekturansicht
\korrekturansichttrue

\input{../tex-inputs/latex-vorspann}


\section[ Paul Goldmann an Arthur Schnitzler, 10. 11. {[}1902{]}]{L03229 Paul Goldmann an Arthur Schnitzler, 10. 11. {[}1902{]}}
\nopagebreak\mylabel{L03229v}
\rehead{ }\normalsize\beginnumbering\briefempfaengerindex{Schnitzler, Arthur@\textsc{Schnitzler, Arthur}!zzzGoldmann, Paul@\emph{von Paul Goldmann}!1902-11-101@{10. 11. {[}1902{]}}|(be}
\toendnotes[C]{\smallbreak\pagebreak[2]}\Standort{DLA, A:Schnitzler, HS.NZ85.1.3172.}
\physDesc{Brief, 1 Blatt, 4 Seiten, 1221 Zeichen
\newline{}Handschrift: blaue Tinte, deutsche Kurrent
\newline{}Schnitzler: 1) mit Bleistift das Jahr »902« vermerkt  2) mit rotem Buntstift sechs Unterstreichungen}\toendnotes[C]{\smallbreak}
\pstart
           \raggedleft{}{\pb}\textcolor{gray}{\textbf{DESSAUERSTRASSE 19}}\oindex{Dessauer Strasse@\textbf{Dessauer Straße}, \emph{Straße (K.STR)}|pw}\pend
           
\pstart
           Berlin\oindex{Berlin@\textbf{Berlin}, \emph{P.PPLC}|pw}, 10. November.\pend
           
\pstart\center{}Mein lieber Freund,\pend\vspace{0.5em}
\pstart
           Ich habe fürchterlich viel zu thun u. komme erſt heut
               dazu, Dir vielmals für den \label{K_L03229-1v}\edtext{Ausſchnitt\pwindex{neue Richtung von Paul Goldman. Wien 1903. Verlag L. Rosner@\emph{Die neue Richtung von Paul Goldman. Wien 1903. Verlag L. Rosner}|pwv}}{\lemma{\textnormal{\emph{Ausſchnitt}}}\Cendnote{\textnormal{[O. V.]: \emph{Die neue Richtung von Paul
                        Goldman. Wien 1903. Verlag L. Rosner}\pwindex{neue Richtung von Paul Goldman. Wien 1903. Verlag L. Rosner@\emph{Die neue Richtung von Paul Goldman. Wien 1903. Verlag L. Rosner}|pwk}. In: \emph{Neues Wiener Tagblatt}\pwindex{Neues Wiener Tagblatt@\emph{Neues Wiener Tagblatt}|pwk}, Jg. 36, Nr. 301, 1. 11. 1902, S. 35.}}}\label{K_L03229-1} aus dem N. W. T.\pwindex{Neues Wiener Tagblatt@\emph{Neues Wiener Tagblatt}|pw} und Deinen lieben Brief zu danken.\pend
           
\pstart
           Die guten Nachrichten von \textsc{Olga\pwindex{Schnitzler, Olga 17.01.1882 – 13.01.1970@\textsc{Schnitzler, Olga} (17.01.1882 – 13.01.1970), \emph{Schauspieler/Schauspielerin, Sänger/Sängerin}|pw}} und Deinem Sohne\pwindex{Schnitzler, Heinrich 09.08.1902 – 12.07.1982@\textsc{Schnitzler, Heinrich} (09.08.1902 – 12.07.1982), \emph{Regisseur/Regisseurin, Schauspieler/Schauspielerin}|pwv} haben
               mich ſehr erfreut. Grüße ſie alle Beide recht herzlich. Wie denkt \textsc{Heinrich Schnitzler\pwindex{Schnitzler, Heinrich 09.08.1902 – 12.07.1982@\textsc{Schnitzler, Heinrich} (09.08.1902 – 12.07.1982), \emph{Regisseur/Regisseurin, Schauspieler/Schauspielerin}|pw}} über \textsc{Gerhart Hauptmann\pwindex{Hauptmann, Gerhart 15.11.1862 – 06.06.1946@\textsc{Hauptmann, Gerhart} (15.11.1862 – 06.06.1946), \emph{Schriftsteller/Schriftstellerin}|pw}}?\pend
           
\pstart
           Mit \textsc{Brahm\pwindex{Brahm, Otto 05.02.1856 – 28.11.1912@\textsc{Brahm, Otto} (05.02.1856 – 28.11.1912), \emph{Theaterleiter/Theaterleiterin, Regisseur/Regisseurin}|pw}} wirſt Du wohl {\pb}inzwiſchen \label{K_L03229-2v}\edtext{einig}{\lemma{\textnormal{\emph{einig}}}\Cendnote{\textnormal{Bezug auf die Aufführung von \emph{Der Schleier der Beatrice}\pwindex{Schleier der Beatrice. Schauspiel in fuenf Akten@\emph{Der Schleier der Beatrice. Schauspiel in fünf Akten}|pwk} am Deutschen
                     Theater Berlin\oindex{Deutsches Theater Berlin@\textbf{Deutsches Theater Berlin}, \emph{Theater (K.THE)}|pwk}}}}\label{K_L03229-2} geworden ſein. Er hat ſich in der letzten \label{K_L03229-3v}\edtext{Cenſur-Affaire}{\lemma{\textnormal{\emph{Cenſur-Affaire}}}\Cendnote{\textnormal{rund um Max Bernsteins\pwindex{Bernstein, Max 12.05.1854 – 08.03.1925@\textsc{Bernstein, Max} (12.05.1854 – 08.03.1925), \emph{Schriftsteller/Schriftstellerin, Kritiker/Kritikerin, Rechtsanwalt/Rechtsanwältin}|pwk} vieraktiges
                  Schauspiel \emph{D’Mali}\pwindex{DMali. Schauspiel in vier Akten@\emph{D’Mali. Schauspiel in vier Akten}|pwk} wenige Tage zuvor}}}\label{K_L03229-3}
               recht männlich und ſympathiſch benommen.\pend
           
\pstart
           \textsc{Sudermann\pwindex{Sudermann, Hermann 30.09.1857 – 21.11.1928@\textsc{Sudermann, Hermann} (30.09.1857 – 21.11.1928), \emph{Schriftsteller/Schriftstellerin}|pw}} miſcht in ſeinen \label{K_L03229-4v}\edtext{Artikel\pwindex{Verrohung in der Theaterkritik [Teil I]@\emph{Verrohung in der Theaterkritik [Teil I]}|pwv}}{\lemma{\textnormal{\emph{Artikel}}}\Cendnote{\textnormal{Gemeint war der erste Teil von \emph{Verrohung in der Theaterkritik}\pwindex{Verrohung in der Theaterkritik@\emph{Verrohung in der Theaterkritik}|pwk}, eine fünfteilige Feuilletonreihe, die in Abendausgaben des \emph{Berliner Tageblatts}\pwindex{Berliner Tageblatt@\emph{Berliner Tageblatt}|pwk} erschien: Hermann Sudermann\pwindex{Sudermann, Hermann 30.09.1857 – 21.11.1928@\textsc{Sudermann, Hermann} (30.09.1857 – 21.11.1928), \emph{Schriftsteller/Schriftstellerin}|pwk}: \emph{Verrohung in der Theaterkritik}\pwindex{Verrohung in der Theaterkritik [Teil I]@\emph{Verrohung in der Theaterkritik [Teil I]}|pwk}. In: \emph{Berliner Tageblatt und Handels-Zeitung}\pwindex{Berliner Tageblatt@\emph{Berliner Tageblatt}|pwk}, Jg. 31,
                     Nr. 553, 30. 10. 1902, Abend-Ausgabe,
                  S. 1–3. Die weiteren Beiträge erschienen am 7. 11. 1902, am 17. 11. 1902, am 25. 11. 1902 und am
                  1. 12. 1902.}}}\label{K_L03229-4} Wahres mit Albernem. Was er über den Gebrauch des Wortes
                  \label{K_L03229-5v}\edtext{»unliterariſch«}{\lemma{\textnormal{\emph{»unliterariſch«}}}\Cendnote{\textnormal{Vgl. ebd.\pwindex{Verrohung in der Theaterkritik [Teil I]@\emph{Verrohung in der Theaterkritik [Teil I]}|pwkv}, S. 2.}}}\label{K_L03229-5} ſagte, war ſehr richtig. Auch die
                  \label{K_L03229-6v}\edtext{\textsc{\begin{otherlanguage}{french}gaminerie\end{otherlanguage}}}{\lemma{\textnormal{\emph{gaminerie}}}\Cendnote{\textnormal{französisch: Kinderei}}}\label{K_L03229-6} unſeres
               Freundes \textsc{Kerr\pwindex{Kerr, Alfred 25.12.1867 – 12.10.1948@\textsc{Kerr, Alfred} (25.12.1867 – 12.10.1948), \emph{Schriftsteller/Schriftstellerin, Kritiker/Kritikerin}|pw}}, die er in ſeinem letzten \label{K_L03229-7v}\edtext{Feuilleton\pwindex{Verrohung in der Theaterkritik@\emph{Verrohung in der Theaterkritik}|pwv}}{\lemma{\textnormal{\emph{Feuilleton}}}\Cendnote{\textnormal{In Teil II der Feuilletonreihe \emph{Verrohung in der Theaterkritik}\pwindex{Verrohung in der Theaterkritik@\emph{Verrohung in der Theaterkritik}|pwk} behandelte Sudermann\pwindex{Sudermann, Hermann 30.09.1857 – 21.11.1928@\textsc{Sudermann, Hermann} (30.09.1857 – 21.11.1928), \emph{Schriftsteller/Schriftstellerin}|pwk} Themen und verschiedene Kritiker,
                  darunter Kerr\pwindex{Kerr, Alfred 25.12.1867 – 12.10.1948@\textsc{Kerr, Alfred} (25.12.1867 – 12.10.1948), \emph{Schriftsteller/Schriftstellerin, Kritiker/Kritikerin}|pwk}, dem er eine Aussage über Eleonora Duse\pwindex{Duse, Eleonora 03.10.1858 – 21.04.1924@\textsc{Duse, Eleonora} (03.10.1858 – 21.04.1924), \emph{Schauspieler/Schauspielerin}|pwk} vorhielt. 
                  Hermann Sudermann\pwindex{Sudermann, Hermann 30.09.1857 – 21.11.1928@\textsc{Sudermann, Hermann} (30.09.1857 – 21.11.1928), \emph{Schriftsteller/Schriftstellerin}|pwk}: \emph{Verrohung in der Theaterkritik. II}\pwindex{Verrohung in der Theaterkritik. II@\emph{Verrohung in der Theaterkritik. II}|pwk}. In: \emph{Berliner Tageblatt und Handels-Zeitung}\pwindex{Berliner Tageblatt@\emph{Berliner Tageblatt}|pwk}, Jg. 31,
                     Nr. 568, 7. 11. 1902, Abend-Ausgabe,
                     S. 3–4.}}}\label{K_L03229-7} anführt, war recht garſtig. Vieles aber ließe ſich leicht
               widerlegen.\pend
           
\pstart
           {\pb}Haſt Du den \label{K_L03229-8v}\edtext{»Brief\pwindex{Brief@\emph{Ein Brief}|pw}«}{\lemma{\textnormal{\emph{»Brief«}}}\Cendnote{\textnormal{Hugo von Hofmannsthal\pwindex{Hofmannsthal, Hugo von 1874-02-01 – 1929-07-15@\textsc{Hofmannsthal, Hugo von} (1874-02-01 – 1929-07-15), \emph{Schriftsteller/Schriftstellerin}|pwk}: \emph{Ein Brief}\pwindex{Brief@\emph{Ein Brief}|pwk}. In: \emph{Der Tag.
                        Erster Teil: Illustrierte Zeitung}\pwindex{Tag@\emph{Der Tag}|pwk}, Nr. 489, 18. 10. 1902, S. [1–3] und Nr. 491, 19. 10. 1902, S. [1–3]. Eine Lektüre durch Schnitzler ist nicht belegt, aber nicht zuletzt durch
                  diesen Hinweis sehr wahrscheinlich.}}}\label{K_L03229-8} von \textsc{Hoffmannsthal\pwindex{Hofmannsthal, Hugo von 1874-02-01 – 1929-07-15@\textsc{Hofmannsthal, Hugo von} (1874-02-01 – 1929-07-15), \emph{Schriftsteller/Schriftstellerin}|pw}} geleſen, der vor einigen Wochen im »Tag\pwindex{Tag@\emph{Der Tag}|pw}«
               erſchienen iſt?\pend
           
\pstart
           Geſtern{ }Nachmittag kam ich endlich dazu, \textsc{Liesl\pwindex{Steinrueck, Elisabeth 19.11.1885 – 07.04.1920@\textsc{Steinrück, Elisabeth} (19.11.1885 – 07.04.1920)|pw}} in ihrem \textsc{Boudoir} zu beſuchen. Sie wohnt recht
               ärmlich, das arme Ding, – aber ſie iſt ſehr vergnügt und ſpielt ſogar ſchon größere
                  \label{K_L03229-9v}\edtext{Rollen}{\lemma{\textnormal{\emph{Rollen}}}\Cendnote{\textnormal{am \emph{Schiller-Theater}\orgindex{Schiller-Theater@Schiller-Theater|pwk}, wo
                     Elisabeth Gussmann\pwindex{Steinrueck, Elisabeth 19.11.1885 – 07.04.1920@\textsc{Steinrück, Elisabeth} (19.11.1885 – 07.04.1920)|pwk} seit 1. 9. 1902 unter Vertrag stand}}}\label{K_L03229-9}.\pend
           
\pstart
           Ich bin wieder einmal durch Verſchiedenes (Schlafloſigkeit, nervöſe Störungen) ſehr
                  {\pb}niedergedrückt. Daher für heut nur dieſe wenigen Zeilen.\pend
           
\pstart
           Laß’ bald von Dir hören und ſei vielmals und herzlichſt gegrüßt von {\\[\baselineskip]}Deinem {\\[\baselineskip]}\spacefill\mbox{Paul Goldm}\pend
           \leftskip=0em{}\selectlanguage{ngerman}\endnumbering\briefempfaengerindex{Schnitzler, Arthur@\textsc{Schnitzler, Arthur}!zzzGoldmann, Paul@\emph{von Paul Goldmann}!1902-11-101@{10. 11. {[}1902{]}}|)be}\mylabel{L03229h}  \normalsize

\doendnotes{C}
\bigskip
\vfill

\clearpage

\footnotesize

\lohead{\textsc{register}}

% Definiere theindex-Environment komplett neu ohne reledmac
\makeatletter
\renewenvironment{theindex}{%
  \section*{\indexname}%
  \setlength{\parindent}{0pt}%
  \setlength{\parskip}{0pt plus 0.3pt}%
  \let\item\@idxitem
}{%
  \clearpage
}
\makeatother

\IfFileExists{\jobname-pw.ind}{\input{\jobname-pw.ind}}{}

\end{document}

      