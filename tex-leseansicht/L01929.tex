%% latex-korrekturansicht-vorspann.tex
%% Vorspann für die Korrekturansicht.
%% Lädt die gemeinsame Datei latex-vorspann.tex mit gesetztem Schalter.

\newif\ifkorrekturansicht
\korrekturansichttrue

\input{../tex-inputs/latex-vorspann}


\section[Richard Beer-Hofmann an Arthur Schnitzler, 14. 5. 1910]{L01929 Richard Beer-Hofmann an Arthur Schnitzler, 14. 5. 1910}
\nopagebreak\mylabel{L01929v}
\rehead{ }\normalsize\beginnumbering\briefempfaengerindex{Schnitzler, Arthur@\textsc{Schnitzler, Arthur}!zzzBeer-Hofmann, Richard@\emph{von Richard Beer-Hofmann}!1910-05-141@{14. 5. 1910}|(be}
\toendnotes[C]{\smallbreak\pagebreak[2]}\Standort{YCGL, MSS 31.}
\physDesc{Brief, 1 Blatt, 1 Seite, 237 Zeichen
\newline{}Handschrift: blauer Buntstift, lateinische Kurrent
\newline{}Schnitzler: mit Bleistift beschriftet: »\textsc{BH}« 
\newline{}Ordnung: mit Bleistift von unbekannter Hand nummeriert:
                                    »230« }
\buchAbdrucke{\weitereDrucke{Arthur Schnitzler, Richard Beer-Hofmann: \emph{Briefwechsel 1891–1931}. Wien, Zürich: \emph{Europaverlag} 1992, S. 207.} }\toendnotes[C]{\smallbreak}\stanza{}{\pb}\label{K_L01929-1v}\edtext{Und lest ihr}{\lemma{\textnormal{\emph{Und lest ihr}}}\Cendnote{\textnormal{Das Gedicht begleitete eine Dose, die
                           Schnitzler am Vorabend seines
                        Geburtstages von Beer-Hofmann\pwindex{Beer-Hofmann, Richard 1866-07-11 – 1945-09-26@\textsc{Beer-Hofmann, Richard} (1866-07-11 – 1945-09-26), \emph{Schriftsteller/Schriftstellerin}|pwk}
                        geschenkt bekommen hat.}}}\label{K_L01929-1}: »H. Meister«,Und ruft ihr: »So heisst erJa nicht, dem man’s schenkt«!\stanzaend{}\stanza{}So sag ich: »Voreilig erscheint das Gekrittel,Ist’s auch nicht sein Name, so ist’s doch ein Titel,Der wol ihm gebührt – dies, Krittler, bedenkt!{[}«{]}\stanzaend{}\pstart \spacefill\mbox{R.}\pend{}
\pstart
           \raggedleft{}14/V 10\pend
           \selectlanguage{ngerman}\endnumbering\briefempfaengerindex{Schnitzler, Arthur@\textsc{Schnitzler, Arthur}!zzzBeer-Hofmann, Richard@\emph{von Richard Beer-Hofmann}!1910-05-141@{14. 5. 1910}|)be}\mylabel{L01929h}  \normalsize

\doendnotes{C}
\bigskip
\vfill

\clearpage

\footnotesize

\lohead{\textsc{register}}

% Definiere theindex-Environment komplett neu ohne reledmac
\makeatletter
\renewenvironment{theindex}{%
  \section*{\indexname}%
  \setlength{\parindent}{0pt}%
  \setlength{\parskip}{0pt plus 0.3pt}%
  \let\item\@idxitem
}{%
  \clearpage
}
\makeatother

\IfFileExists{\jobname-pw.ind}{\input{\jobname-pw.ind}}{}

\end{document}

      