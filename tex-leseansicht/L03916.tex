%% latex-leseansicht-vorspann.tex
%% Vorspann für die Leseansicht.
%% Lädt die gemeinsame Datei latex-vorspann.tex mit nicht gesetztem Schalter.

\newif\ifkorrekturansicht
\korrekturansichtfalse

\input{../tex-inputs/latex-vorspann}


\section[Arthur Schnitzler an Theodor Herzl, {[}zwischen 31. 12. 1900 und 7. 1. 1901?{]}]{L03916 Arthur Schnitzler an Theodor Herzl, {[}zwischen 31. 12. 1900 und 7. 1. 1901?{]}}
\nopagebreak\mylabel{L03916v}
\rehead{ }\normalsize\beginnumbering\briefempfaengerindex{Herzl, Theodor@\textsc{Herzl, Theodor}!zzzSchnitzler, Arthur@\emph{von Arthur Schnitzler}!1901-01-071@{{[}zwischen 31. 12. 1900 und 7. 1. 1901?{]}}|(be}
\toendnotes[C]{\smallbreak\pagebreak[2]}
\correspDesc{Versand  durch Arthur Schnitzler im Zeitraum [zwischen 31. 12. 1900 und 7. 1. 1901?] in Wien
\newline{}Erhalt  durch Theodor Herzl in Wien}\toendnotes[C]{\smallbreak}
\Standort{Jerusalem, Central Zionist Archives, H1:1926-8.}
\physDesc{Briefkarte, 389 Zeichen
\newline{}Handschrift: schwarze Tinte, deutsche Kurrent}\toendnotes[C]{\smallbreak}
\pstart
           \noindent{}{\pb}lieber Doctor Herzl, ſeien Sie nicht
               ungehalten, dſs ich \uline{Sie} Frage, aber ich wüßte nicht
               wen ſonſt, da ich nur mit Ihnen in dieſer Sache verhandelt habe: an wen hab ich mich \substVorne{}\textsuperscript{wegen}\substDazwischen{}mit\substHinten{} meinen \label{K_L03916-1v}\edtext{Honoraranſprüchen\pwindex{Schnitzler, Arthur 15.\,5.\,1862 Wien – 21.\,10.\,1931 ebd.@\textsc{Schnitzler, Arthur} (15.\,5.\,1862 Wien – 21.\,10.\,1931 ebd.), \emph{Schriftsteller, Mediziner}!Lieutenant Gustl. Novelle@\strich\emph{Lieutenant Gustl. Novelle}|pwv}}{\lemma{\textnormal{\emph{Honoraransprüchen}}}\Cendnote{\textnormal{Das vorliegende Korrespondenzstück ist undatiert, lässt
                  sich aber durch den Inhalt – die für die Zeitung ungewöhnliche Länge von \emph{Lieutenant Gustl}\pwindex{Schnitzler, Arthur 15.\,5.\,1862 Wien – 21.\,10.\,1931 ebd.@\textsc{Schnitzler, Arthur} (15.\,5.\,1862 Wien – 21.\,10.\,1931 ebd.), \emph{Schriftsteller, Mediziner}!Lieutenant Gustl. Novelle@\strich\emph{Lieutenant Gustl. Novelle}|pwk} in den Zeitraum nach ihrem Erscheinen am 25. 12. 1900 verorten. Die Bezahlung dürfte erst nach den Feiertagen
                  erfolgt sein, so dass eine Beschwerde Schnitzlers frühestens am Montag, den 31. 12. 1900 sinnvoll erscheint. Da Schnitzler am
                  XXXX Auszeichnungsfehler: Dokument L03940 nicht gefunden die vorliegende Karte für verloren gegangen hält, muss sie einige Tage vorher abgefasst sein, in denen er eine Reaktion Herzls\pwindex{Herzl, Theodor 2.\,5.\,1860 Budapest – 3.\,7.\,1904 Edlach@\textsc{Herzl, Theodor} (2.\,5.\,1860 Budapest – 3.\,7.\,1904 Edlach), \emph{Schriftsteller, Journalist}|pwk}
                  erwartete.}}}\label{K_L03916-1} zu wenden? Man hat nemlich voll{\pb}ko{\geminationm}en vergeſſen, die Länge meiner Novelle\pwindex{Schnitzler, Arthur 15.\,5.\,1862 Wien – 21.\,10.\,1931 ebd.@\textsc{Schnitzler, Arthur} (15.\,5.\,1862 Wien – 21.\,10.\,1931 ebd.), \emph{Schriftsteller, Mediziner}!Lieutenant Gustl. Novelle@\strich\emph{Lieutenant Gustl. Novelle}|pwv} in Betracht zu ziehen, was ich bei
               Einſendg ausdrücklich betonte.\pend
           
\pstart
           Verbindlichſt grüßend{\\[\baselineskip]}Ihr ergebn\textcolor{gray}{er}{\\[\baselineskip]}\spacefill\mbox{Arthur Schnitzler}\pend
           \leftskip=0em{}\selectlanguage{ngerman}\endnumbering\briefempfaengerindex{Herzl, Theodor@\textsc{Herzl, Theodor}!zzzSchnitzler, Arthur@\emph{von Arthur Schnitzler}!1900-12-311@{{[}zwischen 31. 12. 1900 und 7. 1. 1901?{]}}|)be}\mylabel{L03916h}
\begin{anhang}
\end{anhang}\newcommand{\dateiname}{L03916}\newcommand{\titel}{Arthur Schnitzler an Theodor Herzl, [zwischen 31. 12. 1900 und 7. 1. 1901?]}\newcommand{\editorInnen}{Herausgegeben von Jahnke, SelmaMüller, Martin Anton}%% latex-leseansicht-abspann.tex
%% Abspann für die Leseansicht.
%% Der Schalter \ifkorrekturansicht ist bereits durch den Vorspann gesetzt.

%% latex-abspann.tex
%% Gemeinsamer Abspann für Korrekturansicht und Leseansicht.
%% Setzt den Schalter \ifkorrekturansicht voraus (gesetzt in den
%% einbindenden Dateien latex-korrekturansicht-abspann.tex bzw.
%% latex-leseansicht-abspann.tex).
%% ---------------------------------------------------------------

\normalsize

% Das esempio-Environment wird nur in der Leseansicht benötigt
\ifkorrekturansicht\else
\newenvironment{esempio}[3]%
{
    \vspace{1.5ex}
    \rlap{\underline{#1}}
    \par
    \setlength{\parindent}{0cm}
    \nopagebreak
    \leftskip=#2cm
    \rightskip=#3cm
}
{
    \par
}
\fi

\doendnotes{C}
\bigskip
\vfill

\clearpage

\footnotesize

\ifkorrekturansicht
  \lohead{\textsc{register}}
\fi

% theindex-Environment neu definieren ohne reledmac
\makeatletter
\renewenvironment{theindex}{%
  \ifkorrekturansicht
    \section*{\indexname}%
  \else
    \subsubsection*{Index der erwähnten Entitäten}%
  \fi
  \setlength{\parindent}{0pt}%
  \setlength{\parskip}{0pt plus 0.3pt}%
  \let\item\@idxitem
}{%
  \ifkorrekturansicht\clearpage\fi
}
\makeatother

\IfFileExists{\jobname-pw.ind}{\input{\jobname-pw.ind}}{}

% Quellenangabe nur in der Leseansicht
\ifkorrekturansicht\else
% Fallback-Definitionen, falls die .tex-Datei \titel etc. nicht gesetzt hat
\providecommand{\titel}{}
\providecommand{\editorInnen}{}
\providecommand{\dateiname}{\jobname}

\vspace{3cm}

\vfill

\footnotesize
\textsc{Quelle}: \titel. Herausgegeben von {\editorInnen}. In: \emph{Arthur Schnitzler: Briefwechsel mit Autorinnen und Autoren}.
 Digitale Edition, https://schnitzler-briefe.acdh.oeaw.ac.at/{\dateiname}.html (Stand \today)
\fi

\end{document}


