%% latex-leseansicht-vorspann.tex
%% Vorspann für die Leseansicht.
%% Lädt die gemeinsame Datei latex-vorspann.tex mit nicht gesetztem Schalter.

\newif\ifkorrekturansicht
\korrekturansichtfalse

\input{../tex-inputs/latex-vorspann}


\section[Paul Goldmann an Arthur Schnitzler, 19. 7. {[}1892{]}]{L02700 Paul Goldmann an Arthur Schnitzler, 19. 7. [1892]}
\nopagebreak\mylabel{L02700v}
\rehead{ }\normalsize\beginnumbering\briefempfaengerindex{Schnitzler, Arthur@\textsc{Schnitzler, Arthur}!zzzGoldmann, Paul@\emph{von Paul Goldmann}!1892-07-192@{19. 7. [1892]}|(be}
\toendnotes[C]{\smallbreak\pagebreak[2]}
\correspDesc{Versand  durch Paul Goldmann am 19. 7. [1892] in Paris
\newline{}Erhalt  durch Arthur Schnitzler im Zeitraum [20. 7. 1892
                  – 24. 7. 1892?] in Wien}\toendnotes[C]{\smallbreak}
\Standort{DLA, A:Schnitzler, HS.NZ85.1.3163.}
\physDesc{Briefkarte, 608 Zeichen
\newline{}Handschrift: schwarze Tinte, deutsche Kurrent
\newline{}Schnitzler: mit Bleistift das Jahr »92« vermerkt }\toendnotes[C]{\smallbreak}
\pstart
           \raggedleft{}{\pb}\textcolor{gray}{\textbf{75, Rue de Richelieu\oindex{rue Richelieu@\textbf{rue Richelieu}, \emph{Straße}|pw}.}}\pend
           \vspace{0.5em}
\pstart
           
\pstart
           \textsc{Paris\oindex{Paris@\textbf{Paris}, \emph{Hauptstadt}|pw}}, 19. Juli.\pend
           
\pstart
           \raggedleft{}Mein lieber Arthur!\pend
           \pend
           
\pstart
           Soeben antwortet mir mein Onkel\pwindex{Mamroth, Fedor 21.\,2.\,1851 Breslau – 25.\,6.\,1907 Frankfurt am Main@\textsc{Mamroth, Fedor} (21.\,2.\,1851 Breslau – 25.\,6.\,1907 Frankfurt am Main), \emph{Journalist, Kritiker}|pwv}, daß er{ }ſich mit{ }ſeinem Verleger\pwindex{Schottlaender, Salo 19.\,6.\,1844 Ziębice – 2.\,4.\,1920 Breslau@\textsc{Schottlaender, Salo} (19.\,6.\,1844 Ziębice – 2.\,4.\,1920 Breslau), \emph{Verleger}|pwv} zerſtritten, weil er ihn betrogen (der Verleger\pwindex{Schottlaender, Salo 19.\,6.\,1844 Ziębice – 2.\,4.\,1920 Breslau@\textsc{Schottlaender, Salo} (19.\,6.\,1844 Ziębice – 2.\,4.\,1920 Breslau), \emph{Verleger}|pwv} meinen Onkel\pwindex{Mamroth, Fedor 21.\,2.\,1851 Breslau – 25.\,6.\,1907 Frankfurt am Main@\textsc{Mamroth, Fedor} (21.\,2.\,1851 Breslau – 25.\,6.\,1907 Frankfurt am Main), \emph{Journalist, Kritiker}|pwv} nämlich) und daß er{ }ſonſt keine \label{K_L02700-1v}\edtext{Beziehungen zu
                  Verlegern}{\lemma{\textnormal{\emph{Beziehungen zu
                  Verlegern}}}\Cendnote{\textnormal{Schnitzler war auf der Suche nach einem
                  Verlag für \emph{Anatol}\pwindex{Schnitzler, Arthur 15.\,5.\,1862 Wien – 21.\,10.\,1931 ebd.@\textsc{Schnitzler, Arthur} (15.\,5.\,1862 Wien – 21.\,10.\,1931 ebd.), \emph{Schriftsteller, Mediziner}!Anatol@\strich\emph{Anatol}|pwk}, nachdem ihm die meisten
                  Verlage abgesagt hatten, ohne das Manuskript eingesehen zu haben. Aus Goldmanns\pwindex{Goldmann, Paul 31.\,1.\,1865 Breslau – 25.\,9.\,1935 Wien@\textsc{Goldmann, Paul} (31.\,1.\,1865 Breslau – 25.\,9.\,1935 Wien), \emph{Schriftsteller, Journalist}|pwk} Vermittlungen wurde nichts, das Buch erschien im
                  Herbst mit Kostenbeteiligung Schnitzlers im
                     \emph{Bibliographischen Bureau}\orgindex{Bibliographisches Bureau@Bibliographisches Bureau|pwk}.}}}\label{K_L02700-1} habe. Ich
               verſuche jetzt noch einen andern Weg, über den ich Dir{ }ſeinerzeit berichten werde.
               Ich{ }ſchicke {\pb}Dir nur dieſe eiligen Zeilen, damit Du
               nicht glaubſt, ich{ }ſei in der Sache \strikeout{unthath} unthätig.
               – \textsc{Herzl\pwindex{Herzl, Theodor 2.\,5.\,1860 Budapest – 3.\,7.\,1904 Edlach@\textsc{Herzl, Theodor} (2.\,5.\,1860 Budapest – 3.\,7.\,1904 Edlach), \emph{Schriftsteller, Journalist}|pw}} läßt Dich erſuchen, Du möchteſt ihm noch etwas von Deinen Sachen{ }ſchicken (\textsc{8. Rue\oindex{rue Monceau@\textbf{rue Monceau}, \emph{Straße}|pw}}{ }\strikeout{Monc\oindex{rue Monceau@\textbf{rue Monceau}, \emph{Straße}|pw}}{ }\label{T_L02700-1v}\edtext{Monceau\oindex{rue Monceau@\textbf{rue Monceau}, \emph{Straße}|pw}}{\lemma{\textnormal{\emph{Monceau}}}\Cendnote{\textnormal{Zur Verdeutlichung des undeutlich
                  geschriebenen ›o‹ wurde von Goldmann\pwindex{Goldmann, Paul 31.\,1.\,1865 Breslau – 25.\,9.\,1935 Wien@\textsc{Goldmann, Paul} (31.\,1.\,1865 Breslau – 25.\,9.\,1935 Wien), \emph{Schriftsteller, Journalist}|pwk} ›Monceau\oindex{rue Monceau@\textbf{rue Monceau}, \emph{Straße}|pwv}‹ ein zweites Mal direkt darunter geschrieben.}}}\label{T_L02700-1}). Auch meine Adreſſe
               iſt nicht mehr \textsc{R. Vivienne\oindex{rue Vivienne@\textbf{rue Vivienne}, \emph{Straße}|pw}},{ }ſondern die oben
                  gedruckte\oindex{rue Richelieu@\textbf{rue Richelieu}, \emph{Straße}|pwv}.\pend
           
\pstart
           Grüß’ Dich Gott! {\\[\baselineskip]}Dein {\\[\baselineskip]}\spacefill\mbox{Paul Goldm}\pend
           \leftskip=0em{}\selectlanguage{ngerman}\endnumbering\briefempfaengerindex{Schnitzler, Arthur@\textsc{Schnitzler, Arthur}!zzzGoldmann, Paul@\emph{von Paul Goldmann}!1892-07-192@{19. 7. [1892]}|)be}\mylabel{L02700h}  \newcommand{\dateiname}{L02700}\newcommand{\titel}{Paul Goldmann an Arthur Schnitzler, 19. 7. [1892]}\newcommand{\editorInnen}{Martin Anton Müller und Laura Untner}%% latex-leseansicht-abspann.tex
%% Abspann für die Leseansicht.
%% Der Schalter \ifkorrekturansicht ist bereits durch den Vorspann gesetzt.

%% latex-abspann.tex
%% Gemeinsamer Abspann für Korrekturansicht und Leseansicht.
%% Setzt den Schalter \ifkorrekturansicht voraus (gesetzt in den
%% einbindenden Dateien latex-korrekturansicht-abspann.tex bzw.
%% latex-leseansicht-abspann.tex).
%% ---------------------------------------------------------------

\normalsize

% Das esempio-Environment wird nur in der Leseansicht benötigt
\ifkorrekturansicht\else
\newenvironment{esempio}[3]%
{
    \vspace{1.5ex}
    \rlap{\underline{#1}}
    \par
    \setlength{\parindent}{0cm}
    \nopagebreak
    \leftskip=#2cm
    \rightskip=#3cm
}
{
    \par
}
\fi

\doendnotes{C}
\bigskip
\vfill

\clearpage

\footnotesize

\ifkorrekturansicht
  \lohead{\textsc{register}}
\fi

% theindex-Environment neu definieren ohne reledmac
\makeatletter
\renewenvironment{theindex}{%
  \ifkorrekturansicht
    \section*{\indexname}%
  \else
    \subsubsection*{Index der erwähnten Entitäten}%
  \fi
  \setlength{\parindent}{0pt}%
  \setlength{\parskip}{0pt plus 0.3pt}%
  \let\item\@idxitem
}{%
  \ifkorrekturansicht\clearpage\fi
}
\makeatother

\IfFileExists{\jobname-pw.ind}{\input{\jobname-pw.ind}}{}

% Quellenangabe nur in der Leseansicht
\ifkorrekturansicht\else
% Fallback-Definitionen, falls die .tex-Datei \titel etc. nicht gesetzt hat
\providecommand{\titel}{}
\providecommand{\editorInnen}{}
\providecommand{\dateiname}{\jobname}

\vspace{3cm}

\vfill

\footnotesize
\textsc{Quelle}: \titel. Herausgegeben von {\editorInnen}. In: \emph{Arthur Schnitzler: Briefwechsel mit Autorinnen und Autoren}.
 Digitale Edition, https://schnitzler-briefe.acdh.oeaw.ac.at/{\dateiname}.html (Stand \today)
\fi

\end{document}


