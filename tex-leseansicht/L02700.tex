%% latex-korrekturansicht-vorspann.tex
%% Vorspann für die Korrekturansicht.
%% Lädt die gemeinsame Datei latex-vorspann.tex mit gesetztem Schalter.

\newif\ifkorrekturansicht
\korrekturansichttrue

\input{../tex-inputs/latex-vorspann}


\section[Paul Goldmann an Arthur Schnitzler, 19. 7. {[}1892{]}]{L02700 Paul Goldmann an Arthur Schnitzler, 19. 7. {[}1892{]}}
\nopagebreak\mylabel{L02700v}
\rehead{ }\normalsize\beginnumbering\briefempfaengerindex{Schnitzler, Arthur@\textsc{Schnitzler, Arthur}!zzzGoldmann, Paul@\emph{von Paul Goldmann}!1892-07-192@{19. 7. {[}1892{]}}|(be}
\toendnotes[C]{\smallbreak\pagebreak[2]}\Standort{DLA, A:Schnitzler, HS.NZ85.1.3163.}
\physDesc{Briefkarte, 608 Zeichen
\newline{}Handschrift: schwarze Tinte, deutsche Kurrent
\newline{}Schnitzler: mit Bleistift das Jahr »92« vermerkt }\toendnotes[C]{\smallbreak}
\pstart
           \raggedleft{}{\pb}\textcolor{gray}{\textbf{75, Rue de Richelieu\oindex{rue Richelieu@\textbf{rue Richelieu}, \emph{Straße (K.STR)}|pw}.}}\pend
           \vspace{0.5em}
\pstart
           
\pstart
           \textsc{Paris\oindex{Paris@\textbf{Paris}, \emph{P.PPLC}|pw}}, 19. Juli.\pend
           
\pstart
           \raggedleft{}Mein lieber Arthur!\pend
           \pend
           
\pstart
           Soeben antwortet mir mein Onkel\pwindex{Mamroth, Fedor 21.02.1851 – 25.06.1907@\textsc{Mamroth, Fedor} (21.02.1851 – 25.06.1907), \emph{Journalist/Journalistin, Kritiker/Kritikerin}|pwv}, daß er ſich mit ſeinem Verleger\pwindex{Schottlaender, Salo 1844-06-19 – 1920-04-02@\textsc{Schottlaender, Salo} (1844-06-19 – 1920-04-02), \emph{Verleger/Verlegerin}|pwv} zerſtritten, weil er ihn betrogen (der Verleger\pwindex{Schottlaender, Salo 1844-06-19 – 1920-04-02@\textsc{Schottlaender, Salo} (1844-06-19 – 1920-04-02), \emph{Verleger/Verlegerin}|pwv} meinen Onkel\pwindex{Mamroth, Fedor 21.02.1851 – 25.06.1907@\textsc{Mamroth, Fedor} (21.02.1851 – 25.06.1907), \emph{Journalist/Journalistin, Kritiker/Kritikerin}|pwv} nämlich) und daß er
               ſonſt keine \label{K_L02700-1v}\edtext{Beziehungen zu
                  Verlegern}{\lemma{\textnormal{\emph{Beziehungen zu
                  Verlegern}}}\Cendnote{\textnormal{Schnitzler war auf der Suche nach einem
                  Verlag für \emph{Anatol}\pwindex{Anatol@\emph{Anatol}|pwk}, nachdem ihm die meisten
                  Verlage abgesagt hatten, ohne das Manuskript eingesehen zu haben. Aus Goldmanns\pwindex{Goldmann, Paul 31.01.1865 – 25.09.1935@\textsc{Goldmann, Paul} (31.01.1865 – 25.09.1935), \emph{Schriftsteller/Schriftstellerin, Journalist/Journalistin}|pwk} Vermittlungen wurde nichts, das Buch erschien im
                  Herbst mit Kostenbeteiligung Schnitzlers im
                     \emph{Bibliographischen Bureau}\orgindex{Bibliographisches Bureau@Bibliographisches Bureau|pwk}.}}}\label{K_L02700-1} habe. Ich
               verſuche jetzt noch einen andern Weg, über den ich Dir ſeinerzeit berichten werde.
               Ich ſchicke {\pb}Dir nur dieſe eiligen Zeilen, damit Du
               nicht glaubſt, ich ſei in der Sache \strikeout{unthath} unthätig.
               – \textsc{Herzl\pwindex{Herzl, Theodor 1860-05-02 – 1904-07-03@\textsc{Herzl, Theodor} (1860-05-02 – 1904-07-03), \emph{Schriftsteller/Schriftstellerin, Journalist/Journalistin}|pw}} läßt Dich erſuchen, Du möchteſt ihm noch etwas von Deinen Sachen ſchicken (\textsc{8. Rue\oindex{rue Monceau@\textbf{rue Monceau}, \emph{Straße (K.STR)}|pw}}{ }\strikeout{Monc\oindex{rue Monceau@\textbf{rue Monceau}, \emph{Straße (K.STR)}|pw}}{ }\label{T_L02700-1v}\edtext{Monceau\oindex{rue Monceau@\textbf{rue Monceau}, \emph{Straße (K.STR)}|pw}}{\lemma{\textnormal{\emph{Monceau}}}\Cendnote{\textnormal{Zur Verdeutlichung des undeutlich
                  geschriebenen ›o‹ wurde von Goldmann\pwindex{Goldmann, Paul 31.01.1865 – 25.09.1935@\textsc{Goldmann, Paul} (31.01.1865 – 25.09.1935), \emph{Schriftsteller/Schriftstellerin, Journalist/Journalistin}|pwk} ›Monceau\oindex{rue Monceau@\textbf{rue Monceau}, \emph{Straße (K.STR)}|pwv}‹ ein zweites Mal direkt darunter geschrieben.}}}\label{T_L02700-1}). Auch meine Adreſſe
               iſt nicht mehr \textsc{R. Vivienne\oindex{rue Vivienne@\textbf{rue Vivienne}, \emph{Straße (K.STR)}|pw}}, ſondern die oben
                  gedruckte\oindex{rue Richelieu@\textbf{rue Richelieu}, \emph{Straße (K.STR)}|pwv}. \pend
           
\pstart
           Grüß’ Dich Gott! {\\[\baselineskip]}Dein {\\[\baselineskip]}\spacefill\mbox{Paul Goldm}\pend
           \leftskip=0em{}\selectlanguage{ngerman}\endnumbering\briefempfaengerindex{Schnitzler, Arthur@\textsc{Schnitzler, Arthur}!zzzGoldmann, Paul@\emph{von Paul Goldmann}!1892-07-192@{19. 7. {[}1892{]}}|)be}\mylabel{L02700h}  \normalsize

\doendnotes{C}
\bigskip
\vfill

\clearpage

\footnotesize

\lohead{\textsc{register}}

% Definiere theindex-Environment komplett neu ohne reledmac
\makeatletter
\renewenvironment{theindex}{%
  \section*{\indexname}%
  \setlength{\parindent}{0pt}%
  \setlength{\parskip}{0pt plus 0.3pt}%
  \let\item\@idxitem
}{%
  \clearpage
}
\makeatother

\IfFileExists{\jobname-pw.ind}{\input{\jobname-pw.ind}}{}

\end{document}

      