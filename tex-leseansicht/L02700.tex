%% latex-leseansicht-vorspann.tex
%% Vorspann für die Leseansicht.
%% Lädt die gemeinsame Datei latex-vorspann.tex mit nicht gesetztem Schalter.

\newif\ifkorrekturansicht
\korrekturansichtfalse

\input{../tex-inputs/latex-vorspann}

\begin{center}
            \textcolor{red}{ENTWURF, NICHT FERTIG KORRIGIERT}
                      \end{center}
            
               \section[Paul Goldmann an Arthur Schnitzler, 19. 7. {[}1892{]}]{ Paul Goldmann an Arthur Schnitzler, 19. 7. {[}1892{]}}\nopagebreak\mylabel{v}\rehead{ }\begin{ledgroupsized}[t]{13cm}\normalsize\beginnumbering\briefempfaengerindex{Schnitzler, Arthur@\textsc{Schnitzler, Arthur}!zzzGoldmann, Paul@\emph{von Paul Goldmann}!1892-07-192@{19. 7. {[}1892{]}}|(be} \toendnotes[C]{\smallbreak\pagebreak[2]} \Standort{DLA, A:Schnitzler, HS.NZ85.1.3163.}
\physDesc{Briefkarte
\newline{}Handschrift: schwarze Tinte, deutsche Kurrent
\newline{}Schnitzler: mit Bleistift das Jahr »92« vermerkt }\toendnotes[C]{\smallbreak}\pstart
           \noindent{}\raggedleft{}{\pb}\textcolor{gray}{\textbf{75, Rue de Richelieu\oindex{rue Richelieu@\textbf{rue Richelieu}|pw}.}}\pend
           \pstart
           \textsc{Paris\oindex{Paris@\textbf{Paris}|pw}}, 19. Juli.\hfill Mein lieber Arthur!\pend
           \pstart
           Soeben antwortet mir mein Onkel\pwindex{Mamroth, Fedor 21.02.1851 – 25.06.1907@\textsc{Mamroth, Fedor} (21.02.1851 – 25.06.1907), \emph{Journalist, Kritiker}|pwv}, daß er ſich mit ſeinem Verleger\pwindex{Schottlaender, Salo 1844-06-19 – 1920-04-02@\textsc{Schottlaender, Salo} (1844-06-19 – 1920-04-02), \emph{Verleger}|pwv} zerſtritten, weil er ihn betrogen (der Verleger\pwindex{Schottlaender, Salo 1844-06-19 – 1920-04-02@\textsc{Schottlaender, Salo} (1844-06-19 – 1920-04-02), \emph{Verleger}|pwv} meinen Onkel\pwindex{Mamroth, Fedor 21.02.1851 – 25.06.1907@\textsc{Mamroth, Fedor} (21.02.1851 – 25.06.1907), \emph{Journalist, Kritiker}|pwv} nämlich) und daß er
               ſonſt keine \label{K_L02700-1v}\edtext{Beziehungen zu
                  Verlegern}{\lemma{\textnormal{\emph{Beziehungen zu
                  Verlegern}}}\Cendnote{\textnormal{Schnitzler\pwindex{Schnitzler, Arthur 15.05.1862 – 21.10.1931@\textsc{Schnitzler, Arthur} (15.05.1862 – 21.10.1931), \emph{Schriftsteller, Mediziner}|pwk} war auf der Suche nach einem Verlag für \emph{Anatol}\pwindex{Schnitzler, Arthur 15.05.1862 – 21.10.1931@\textsc{Schnitzler, Arthur} (15.05.1862 – 21.10.1931), \emph{Schriftsteller, Mediziner}!Anatol1892-10-29 – 1892-10-29@\strich\emph{Anatol} {[}1892-10-29 – 1892-10-29{]}|pwk}, nachdem ihm die meisten Verlage absagten
                  ohne das Manuskript eingesehen zu haben. Aus Goldmann\pwindex{Goldmann, Paul 31.01.1865 – 25.09.1935@\textsc{Goldmann, Paul} (31.01.1865 – 25.09.1935), \emph{Schriftsteller, Journalist}|pwk}s Vermittlungen wurde nichts, das Buch erschien im Herbst mit
                  Kostenbeteiligung Schnitzler\pwindex{Schnitzler, Arthur 15.05.1862 – 21.10.1931@\textsc{Schnitzler, Arthur} (15.05.1862 – 21.10.1931), \emph{Schriftsteller, Mediziner}|pwk}s im \emph{Bibliographischen Bureau}\orgindex{Bibliographisches Bureau@Bibliographisches Bureau|pwk}.}}}\label{K_L02700-1h} habe. Ich
               verſuche jetzt noch einen andern Weg über den ich Dir ſeinerzeit berichten werde. Ich
               ſchick {\pb}Dir nur dieſe eiligen Zeilen, damit Du nicht
               glaubſt, ich ſei in der Sache \strikeout{unthath} unthätig. – \textsc{Herzl\pwindex{Herzl, Theodor 02.05.1860 – 03.07.1904@\textsc{Herzl, Theodor} (02.05.1860 – 03.07.1904), \emph{Schriftsteller, Journalist}|pw}} läßt Dich erſuchen, Du möchteſt ihm noch etwas von Deinen Sachen ſchicken (\textsc{8. Rue\oindex{rue Monceau@\textbf{rue Monceau}|pw}}{ }\substVorne{}\textsuperscript{Monc\oindex{rue Monceau@\textbf{rue Monceau}|pw}}\substDazwischen{}\label{T_L02700-1v}\edtext{Monceau\oindex{rue Monceau@\textbf{rue Monceau}|pw}}{\lemma{\textnormal{\emph{Monceau}}}\Cendnote{\textnormal{Zur Verdeutlichung des undeutlich
                        geschriebenen »o« wurde von Goldmann »Monceau\oindex{rue Monceau@\textbf{rue Monceau}|pwv}« ein zweites Mal direkt darunter geschrieben.}}}\label{T_L02700-1h}\substHinten{}). Auch meine Adreſſe iſt nicht mehr \textsc{R. Vivienne\oindex{rue Vivienne@\textbf{rue Vivienne}|pw}}, ſondern die oben
                  gedruckte\oindex{rue Richelieu@\textbf{rue Richelieu}|pwv}. \pend
           \pstart
           Grüß’ Dich Gott! {\\[\baselineskip]}Dein {\\[\baselineskip]}\spacefill\mbox{Paul Goldm}\pend
           \leftskip=0em{}\endnumbering\briefempfaengerindex{Schnitzler, Arthur@\textsc{Schnitzler, Arthur}!zzzGoldmann, Paul@\emph{von Paul Goldmann}!1892-07-192@{19. 7. {[}1892{]}}|)be}\mylabel{h}\end{ledgroupsized}  \newcommand{\dateiname}{L02700}\newcommand{\titel}{Paul Goldmann an Arthur Schnitzler, 19. 7. [1892]}\newcommand{\editorInnen}{Martin Anton Müller und Laura Untner}
            \footnotesize
\begin{ledgroupsized}[t]{11.5cm}
\doendnotes{C}
\end{ledgroupsized}
         %% latex-leseansicht-abspann.tex
%% Abspann für die Leseansicht.
%% Der Schalter \ifkorrekturansicht ist bereits durch den Vorspann gesetzt.

%% latex-abspann.tex
%% Gemeinsamer Abspann für Korrekturansicht und Leseansicht.
%% Setzt den Schalter \ifkorrekturansicht voraus (gesetzt in den
%% einbindenden Dateien latex-korrekturansicht-abspann.tex bzw.
%% latex-leseansicht-abspann.tex).
%% ---------------------------------------------------------------

\normalsize

% Das esempio-Environment wird nur in der Leseansicht benötigt
\ifkorrekturansicht\else
\newenvironment{esempio}[3]%
{
    \vspace{1.5ex}
    \rlap{\underline{#1}}
    \par
    \setlength{\parindent}{0cm}
    \nopagebreak
    \leftskip=#2cm
    \rightskip=#3cm
}
{
    \par
}
\fi

\doendnotes{C}
\bigskip
\vfill

\clearpage

\footnotesize

\ifkorrekturansicht
  \lohead{\textsc{register}}
\fi

% theindex-Environment neu definieren ohne reledmac
\makeatletter
\renewenvironment{theindex}{%
  \ifkorrekturansicht
    \section*{\indexname}%
  \else
    \subsubsection*{Index der erwähnten Entitäten}%
  \fi
  \setlength{\parindent}{0pt}%
  \setlength{\parskip}{0pt plus 0.3pt}%
  \let\item\@idxitem
}{%
  \ifkorrekturansicht\clearpage\fi
}
\makeatother

\IfFileExists{\jobname-pw.ind}{\input{\jobname-pw.ind}}{}

% Quellenangabe nur in der Leseansicht
\ifkorrekturansicht\else
% Fallback-Definitionen, falls die .tex-Datei \titel etc. nicht gesetzt hat
\providecommand{\titel}{}
\providecommand{\editorInnen}{}
\providecommand{\dateiname}{\jobname}

\vspace{3cm}

\vfill

\footnotesize
\textsc{Quelle}: \titel. Herausgegeben von {\editorInnen}. In: \emph{Arthur Schnitzler: Briefwechsel mit Autorinnen und Autoren}.
 Digitale Edition, https://schnitzler-briefe.acdh.oeaw.ac.at/{\dateiname}.html (Stand \today)
\fi

\end{document}


      