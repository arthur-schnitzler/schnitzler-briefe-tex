%% latex-leseansicht-vorspann.tex
%% Vorspann für die Leseansicht.
%% Lädt die gemeinsame Datei latex-vorspann.tex mit nicht gesetztem Schalter.

\newif\ifkorrekturansicht
\korrekturansichtfalse

\input{../tex-inputs/latex-vorspann}


\section[Hugo von Hofmannsthal an Arthur Schnitzler, 23. 12. [1892]]{L00144 Hugo von Hofmannsthal an Arthur Schnitzler, 23. 12. [1892]}
\nopagebreak\mylabel{L00144v}
\rehead{ }\normalsize\beginnumbering\briefempfaengerindex{Schnitzler, Arthur@\textsc{Schnitzler, Arthur}!zzzHofmannsthal, Hugo von@\emph{von Hugo von Hofmannsthal}!1892-12-231@{23. 12. [1892]}|(be}
\toendnotes[C]{\smallbreak\pagebreak[2]}
\correspDesc{Versand  durch Hugo von Hofmannsthal am 23. 12. [1892] in Wien
\newline{}Erhalt  durch Arthur Schnitzler im Zeitraum [23. 12. 1892 – 27. 12. 1892?] in Wien}\toendnotes[C]{\smallbreak}
\Standort{CUL, Schnitzler, B 43.}
\physDesc{Brief, 1 Blatt, 2 Seiten, 748 Zeichen (aufgeprägtes Wappen)
\newline{}Handschrift: schwarze Tinte, deutsche Kurrent
\newline{}Schnitzler: mit Bleistift nummeriert: »35« und mit einer Jahreszahl
                                 versehen: »92« }
\buchAbdrucke{\weitereDrucke{Hugo von Hofmannsthal, Arthur Schnitzler: \emph{Briefwechsel}. Herausgegeben von Therese Nickl und Heinrich Schnitzler. Frankfurt am Main: \emph{S. Fischer} 1964, S. 32–33.} }\toendnotes[C]{\smallbreak}
\pstart
           \raggedleft{}{\pb}23 December.\pend
           
\pstart{}mein lieber Arthur.\pend\vspace{0.5em}
\pstart
           Ich glaube, ich werde beſſer nicht über Anatol\pwindex{Schnitzler, Arthur 15.\,5.\,1862 Wien – 21.\,10.\,1931 ebd.@\textsc{Schnitzler, Arthur} (15.\,5.\,1862 Wien – 21.\,10.\,1931 ebd.), \emph{Schriftsteller, Mediziner}!Anatol@\strich\emph{Anatol}|pw}{ }ſchreiben. Die Mühe, beinahe Überwindung, die es
               mich koſtet, macht mich{ }ſtutzig. Sich dem Vorwurf der tactloſen Camaraderie ausſetzen
               und nichts dabei erzielen als eine gequälte mühſam gedehnte Beſprechung?\pend
           
\pstart
           Ich weiß offenbar zu viel von dem Buch\pwindex{Schnitzler, Arthur 15.\,5.\,1862 Wien – 21.\,10.\,1931 ebd.@\textsc{Schnitzler, Arthur} (15.\,5.\,1862 Wien – 21.\,10.\,1931 ebd.), \emph{Schriftsteller, Mediziner}!Anatol@\strich\emph{Anatol}|pwv} und{ }ſehe daher nicht klar. Oder Gott weiß, was es{ }ſonſt iſt. Vielleicht
               erlauben {\pb}Sie mir, Ihnen nächſtens
               die 50 Zeilen mitzubringen, die ich zuſammengebracht habe; vielleicht können wir die
               Kritik der Kritik machen und dabei etwas lernen. Wann in der Weihnachtswoche werden
               wir uns ausgiebig{ }ſehen? und was machen die Proben mit Paul Horn\pwindex{Horn, Paul 13.\,2.\,1867 Wien – 18.\,1.\,1936 Menton@\textsc{Horn, Paul} (13.\,2.\,1867 Wien – 18.\,1.\,1936 Menton), \emph{Fabrikant}|pw} und \label{K_L00144-1v}\edtext{\textsc{Aspasia\pwindex{Schroeder, Carl 18.\,12.\,1848 Quedlinburg – 22.\,9.\,1935 Bremen@\textsc{Schroeder, Carl} (18.\,12.\,1848 Quedlinburg – 22.\,9.\,1935 Bremen), \emph{Komponist, Dirigent, Cellist}!Aspasia@\strich\emph{Aspasia}|pwu}–Dora\pwindex{Kohnberger, Dorothea 1.\,8.\,1855 Lviv – 23.\,12.\,1933 Stockholm@\textsc{Kohnberger, Dorothea} (1.\,8.\,1855 Lviv – 23.\,12.\,1933 Stockholm)|pwu}}}{\lemma{\textnormal{\emph{Aspasia–Dora}}}\Cendnote{\textnormal{Bei \emph{Aspasia}\pwindex{Schroeder, Carl 18.\,12.\,1848 Quedlinburg – 22.\,9.\,1935 Bremen@\textsc{Schroeder, Carl} (18.\,12.\,1848 Quedlinburg – 22.\,9.\,1935 Bremen), \emph{Komponist, Dirigent, Cellist}!Aspasia@\strich\emph{Aspasia}|pwk} könnte es sich um die gleichnamige Oper von Carl Schroeder\pwindex{Schroeder, Carl 18.\,12.\,1848 Quedlinburg – 22.\,9.\,1935 Bremen@\textsc{Schroeder, Carl} (18.\,12.\,1848 Quedlinburg – 22.\,9.\,1935 Bremen), \emph{Komponist, Dirigent, Cellist}|pwk} handeln, die am 3. 3. 1892
                  uraufgeführt worden war. Möglicherweise wurden Partien daraus von Dora Kohnberger\pwindex{Kohnberger, Dorothea 1.\,8.\,1855 Lviv – 23.\,12.\,1933 Stockholm@\textsc{Kohnberger, Dorothea} (1.\,8.\,1855 Lviv – 23.\,12.\,1933 Stockholm)|pwk} im Zuge einer
                  Privataufführung bei Bertha Flegmann\pwindex{Flegmann, Bertha 27.\,5.\,1852 Dubrovsky, Polen – 24.\,6.\,1933 Bad Ischl@\textsc{Flegmann, Bertha} (27.\,5.\,1852 Dubrovsky, Polen – 24.\,6.\,1933 Bad Ischl), \emph{Salonnière}|pwk}
                  einstudiert.}}}\label{K_L00144-1}?\pend
           
\pstart
           Allerherzlichſt Ihr immer dankbar und aufrichtig ergebener (4\textsuperscript{ter} Grad){\\[\baselineskip]}\spacefill\mbox{Loris}\pend
           \leftskip=0em{}\selectlanguage{ngerman}\endnumbering\briefempfaengerindex{Schnitzler, Arthur@\textsc{Schnitzler, Arthur}!zzzHofmannsthal, Hugo von@\emph{von Hugo von Hofmannsthal}!1892-12-231@{23. 12. [1892]}|)be}\mylabel{L00144h}  \newcommand{\dateiname}{L00144}\newcommand{\titel}{Hugo von Hofmannsthal an Arthur Schnitzler, 23. 12. [1892]}\newcommand{\editorInnen}{Martin Anton Müller und Gerd-Hermann Susen}%% latex-leseansicht-abspann.tex
%% Abspann für die Leseansicht.
%% Der Schalter \ifkorrekturansicht ist bereits durch den Vorspann gesetzt.

%% latex-abspann.tex
%% Gemeinsamer Abspann für Korrekturansicht und Leseansicht.
%% Setzt den Schalter \ifkorrekturansicht voraus (gesetzt in den
%% einbindenden Dateien latex-korrekturansicht-abspann.tex bzw.
%% latex-leseansicht-abspann.tex).
%% ---------------------------------------------------------------

\normalsize

% Das esempio-Environment wird nur in der Leseansicht benötigt
\ifkorrekturansicht\else
\newenvironment{esempio}[3]%
{
    \vspace{1.5ex}
    \rlap{\underline{#1}}
    \par
    \setlength{\parindent}{0cm}
    \nopagebreak
    \leftskip=#2cm
    \rightskip=#3cm
}
{
    \par
}
\fi

\doendnotes{C}
\bigskip
\vfill

\clearpage

\footnotesize

\ifkorrekturansicht
  \lohead{\textsc{register}}
\fi

% theindex-Environment neu definieren ohne reledmac
\makeatletter
\renewenvironment{theindex}{%
  \ifkorrekturansicht
    \section*{\indexname}%
  \else
    \subsubsection*{Index der erwähnten Entitäten}%
  \fi
  \setlength{\parindent}{0pt}%
  \setlength{\parskip}{0pt plus 0.3pt}%
  \let\item\@idxitem
}{%
  \ifkorrekturansicht\clearpage\fi
}
\makeatother

\IfFileExists{\jobname-pw.ind}{\input{\jobname-pw.ind}}{}

% Quellenangabe nur in der Leseansicht
\ifkorrekturansicht\else
% Fallback-Definitionen, falls die .tex-Datei \titel etc. nicht gesetzt hat
\providecommand{\titel}{}
\providecommand{\editorInnen}{}
\providecommand{\dateiname}{\jobname}

\vspace{3cm}

\vfill

\footnotesize
\textsc{Quelle}: \titel. Herausgegeben von {\editorInnen}. In: \emph{Arthur Schnitzler: Briefwechsel mit Autorinnen und Autoren}.
 Digitale Edition, https://schnitzler-briefe.acdh.oeaw.ac.at/{\dateiname}.html (Stand \today)
\fi

\end{document}


