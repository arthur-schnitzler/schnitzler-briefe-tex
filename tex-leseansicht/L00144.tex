%% latex-leseansicht-vorspann.tex
%% Vorspann für die Leseansicht.
%% Lädt die gemeinsame Datei latex-vorspann.tex mit nicht gesetztem Schalter.

\newif\ifkorrekturansicht
\korrekturansichtfalse

\input{../tex-inputs/latex-vorspann}


         
         \renewcommand{\erwaehntePersonen}{Personen: Bertha Flegmann, Paul Horn, Dorothea Kohnberger, Carl Schroeder}
         \renewcommand{\erwaehnteOrte}{Orte: Wien}
         \renewcommand{\erwaehnteWerke}{Werke: Anatol, Aspasia}
               \section[Hugo von Hofmannsthal an Arthur Schnitzler, 23. 12. {[}1892{]}]{ Hugo von Hofmannsthal an Arthur Schnitzler, 23. 12. {[}1892{]}}\nopagebreak\mylabel{v}\rehead{ }\begin{ledgroupsized}[t]{13cm}\normalsize\beginnumbering \toendnotes[C]{\smallbreak\pagebreak[2]} \Standort{CUL, Schnitzler, B 43.}
\physDesc{Brief, 1 Blatt (Briefpapier mit aufgeprägtem Wappen), 2 Seiten
\newline{}Handschrift: schwarze Tinte, deutsche Kurrent
\newline{}Schnitzler: mit Bleistift nummeriert: »35« und mit einer
                                 Jahreszahl versehen: »92« }\buchAbdrucke{\weitereDrucke{Hugo von Hofmannsthal, Arthur Schnitzler: \emph{Briefwechsel}. Hg. Therese Nickl und Heinrich Schnitzler. Frankfurt am Main: \emph{S. Fischer} 1964, S. 32–33.} }\toendnotes[C]{\smallbreak}\pstart
           \raggedleft{}{\pb}23 December.\pend
           \pstart{}mein lieber Arthur.\pend\pstart
           Ich glaube, ich werde beſſer nicht über Anatol\pwindex{Schnitzler, Arthur 15.05.1862 – 21.10.1931@\textsc{Schnitzler, Arthur} (15.05.1862 – 21.10.1931), \emph{Schriftsteller, Mediziner}!Anatol1892-10-29@\strich\emph{Anatol} {[}1892-10-29{]}|pw}{ }ſchreiben. Die Mühe, beinahe Überwindung, die es
               mich koſtet, macht mich ſtutzig. Sich dem Vorwurf der tactloſen Camaraderie ausſetzen
               und nichts dabei erzielen als eine gequälte mühſam gedehnte Beſprechung?\pend
           \pstart
           Ich weiß offenbar zu viel von dem Buch\pwindex{Schnitzler, Arthur 15.05.1862 – 21.10.1931@\textsc{Schnitzler, Arthur} (15.05.1862 – 21.10.1931), \emph{Schriftsteller, Mediziner}!Anatol1892-10-29@\strich\emph{Anatol} {[}1892-10-29{]}|pwv} und ſehe daher nicht klar. Oder Gott weiß, was es ſonſt iſt. Vielleicht
               erlauben {\pb}Sie mir, Ihnen nächſtens
               die 50 Zeilen mitzubringen, die ich zuſammengebracht habe; vielleicht können wir die
               Kritik der Kritik machen und dabei etwas lernen. Wann in der Weihnachtswoche werden
               wir uns ausgiebig ſehen? und was machen die Proben mit Paul Horn\pwindex{Horn, Paul 13.02.1867 – 18.01.1936@\textsc{Horn, Paul} (13.02.1867 – 18.01.1936), \emph{Fabrikant}|pw} und \label{K_L00144_1v}\edtext{\textsc{Aspasia\pwindex{Schroeder, Carl 18.12.1848 – 22.09.1935@\textsc{Schroeder, Carl} (18.12.1848 – 22.09.1935), \emph{Komponist, Dirigent, Cellist}!Aspasia3. 3. 1892@\strich\emph{Aspasia} {[}3. 3. 1892{]}|pwu}–Dora\pwindex{Kohnberger, Dorothea 01.08.1855 – 23.12.1933@\textsc{Kohnberger, Dorothea} (01.08.1855 – 23.12.1933)|pwu}}}{\lemma{\textnormal{\emph{Aspasia–Dora}}}\Cendnote{\textnormal{Bei \emph{Aspasia}\pwindex{Schroeder, Carl 18.12.1848 – 22.09.1935@\textsc{Schroeder, Carl} (18.12.1848 – 22.09.1935), \emph{Komponist, Dirigent, Cellist}!Aspasia3. 3. 1892@\strich\emph{Aspasia} {[}3. 3. 1892{]}|pwk} könnte es sich um die gleichnamige Oper von Carl Schroeder\pwindex{Schroeder, Carl 18.12.1848 – 22.09.1935@\textsc{Schroeder, Carl} (18.12.1848 – 22.09.1935), \emph{Komponist, Dirigent, Cellist}|pwk} handeln, die am 3. 3. 1892
                  uraufgeführt worden war. Möglicherweise wurden Partien aus ihr von Dora Kohnberger\pwindex{Kohnberger, Dorothea 01.08.1855 – 23.12.1933@\textsc{Kohnberger, Dorothea} (01.08.1855 – 23.12.1933)|pwk} im Zuge einer Privataufführung
                  bei Bertha Flegmann\pwindex{Flegmann, Bertha 27.05.1852 – 24.6.1933@\textsc{Flegmann, Bertha} (27.05.1852 – 24.6.1933), \emph{Salonnière}|pwk} einstudiert.}}}\label{K_L00144_1h}?\pend
           \pstart
           Allerherzlichſt Ihr immer dankbar und aufrichtig ergebener (4\textsuperscript{ter} Grad){\\[\baselineskip]}\spacefill\mbox{Loris}\pend
           \leftskip=0em{}
         
         \endnumbering\mylabel{h}\end{ledgroupsized}  \newcommand{\dateiname}{L00144}\newcommand{\titel}{Hugo von Hofmannsthal an Arthur Schnitzler, 23. 12. [1892]}\newcommand{\editorInnen}{Martin Anton Müller und Gerd-Hermann Susen}%% latex-leseansicht-abspann.tex
%% Abspann für die Leseansicht.
%% Der Schalter \ifkorrekturansicht ist bereits durch den Vorspann gesetzt.

%% latex-abspann.tex
%% Gemeinsamer Abspann für Korrekturansicht und Leseansicht.
%% Setzt den Schalter \ifkorrekturansicht voraus (gesetzt in den
%% einbindenden Dateien latex-korrekturansicht-abspann.tex bzw.
%% latex-leseansicht-abspann.tex).
%% ---------------------------------------------------------------

\normalsize

% Das esempio-Environment wird nur in der Leseansicht benötigt
\ifkorrekturansicht\else
\newenvironment{esempio}[3]%
{
    \vspace{1.5ex}
    \rlap{\underline{#1}}
    \par
    \setlength{\parindent}{0cm}
    \nopagebreak
    \leftskip=#2cm
    \rightskip=#3cm
}
{
    \par
}
\fi

\doendnotes{C}
\bigskip
\vfill

\clearpage

\footnotesize

\ifkorrekturansicht
  \lohead{\textsc{register}}
\fi

% theindex-Environment neu definieren ohne reledmac
\makeatletter
\renewenvironment{theindex}{%
  \ifkorrekturansicht
    \section*{\indexname}%
  \else
    \subsubsection*{Index der erwähnten Entitäten}%
  \fi
  \setlength{\parindent}{0pt}%
  \setlength{\parskip}{0pt plus 0.3pt}%
  \let\item\@idxitem
}{%
  \ifkorrekturansicht\clearpage\fi
}
\makeatother

\IfFileExists{\jobname-pw.ind}{\input{\jobname-pw.ind}}{}

% Quellenangabe nur in der Leseansicht
\ifkorrekturansicht\else
% Fallback-Definitionen, falls die .tex-Datei \titel etc. nicht gesetzt hat
\providecommand{\titel}{}
\providecommand{\editorInnen}{}
\providecommand{\dateiname}{\jobname}

\vspace{3cm}

\vfill

\footnotesize
\textsc{Quelle}: \titel. Herausgegeben von {\editorInnen}. In: \emph{Arthur Schnitzler: Briefwechsel mit Autorinnen und Autoren}.
 Digitale Edition, https://schnitzler-briefe.acdh.oeaw.ac.at/{\dateiname}.html (Stand \today)
\fi

\end{document}


      