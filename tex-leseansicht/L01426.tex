%% latex-leseansicht-vorspann.tex
%% Vorspann für die Leseansicht.
%% Lädt die gemeinsame Datei latex-vorspann.tex mit nicht gesetztem Schalter.

\newif\ifkorrekturansicht
\korrekturansichtfalse

\input{../tex-inputs/latex-vorspann}


         
         \renewcommand{\erwaehntePersonen}{Personen: Hermine von Schaffgotsch, Franziska Schlesinger, Louise Schnitzler, Christopher St. John, Robert Gilbert Vansittart}
         \renewcommand{\erwaehnteInstitutionen}{Institutionen: Eton College}
         \renewcommand{\erwaehnteOrte}{Orte: Bad Aussee, Bad Ischl, Botschaft von Großbritannien in Paris, England, Ramgut, Villa Franckenstein, Wien, Wolfgangsee}
         \renewcommand{\erwaehnteWerke}{Werke: Anatol, Das Vermächtnis. Schauspiel in drei Akten, Der Schleier der Beatrice. Schauspiel in fünf Akten, Sterben. Novelle}
               \section[Hugo von Hofmannsthal an Arthur Schnitzler, 12. 8. 1904]{ Hugo von Hofmannsthal an Arthur Schnitzler, 12. 8. 1904}\nopagebreak\mylabel{v}\rehead{ }\begin{ledgroupsized}[t]{13cm}\normalsize\beginnumbering \toendnotes[C]{\smallbreak\pagebreak[2]} \Standort{CUL, Schnitzler, B 43.}
\physDesc{Brief, 2 Blätter, 8 Seiten, 2805 Zeichen
\newline{}Handschrift: schwarze Tinte, deutsche Kurrent
\newline{}Schnitzler: mit Bleistift die Jahreszahl ergänzt: »904« 
\newline{}Ordnung: 1) mit Bleistift von unbekannter Hand nummeriert: »\strikeout{254}«  2) mit Bleistift von unbekannter Hand nummeriert:
                                    »232.1« bzw. »232.2«}\buchAbdrucke{\weitereDrucke{Hugo von Hofmannsthal, Arthur Schnitzler: \emph{Briefwechsel}. Hg. Therese Nickl und Heinrich Schnitzler. Frankfurt am Main: \emph{S. Fischer} 1964, S. 196.} }\toendnotes[C]{\smallbreak}\pstart
           \raggedleft{}{\pb}Markt Auſſee, Ramgut\oindex{Ramgut@\textbf{Ramgut}|pw}{\\}12 VIII.\pend
           \pstart{}Lieber,\pend\pstart
           Ich ging gegen Abend vom Markt\oindex{Bad Aussee@\textbf{Bad Aussee}|pw} herauf, begegnete
               drei Frauen deren Geſichter ich nicht ſehen konnte. Hinter mir ſagte eine davon, ihr
               Gespräch fortſetzend: »und dann ſind wir mit ihnen auseinandergekommen, das war zu
               der Zeit wie ſie mit dem Arthur Schnitzler verlobt war«{\dots}
               und die andere ſagte beſtätigend: »ja, zu der Zeit war ſie mit {\pb}dem Arthur Schnitzler verlobt«.
               Von wem kann da die Rede geweſen ſein? Vielleicht von der ewigen Minnie\pwindex{Schaffgotsch, Hermine von 25.11.1871 – 25.11.1928@\textsc{Schaffgotsch, Hermine von} (25.11.1871 – 25.11.1928)|pw}?\pend
           \pstart
           \numberlinefalse{}\centering{}–\numberlinetrue{}\pend
           \pstart
           \noindent{}Eine Stunde ſpäter ſoupierte ich mit Leuten: da hörte ich mir gegenüber einen \strikeout{zus} zu ſeinem Nachbar ſagen, auf engliſch: »und dann
               hat mir der Manager geſagt, wenn Schnitzler fortfährt, {\pb}ſolche Sachen zu machen, wird man
               ihn als einen litterariſchen Pariah behandeln (wörtlich.)« Das intereſſierte mich
               doch ſehr und ich habe nach Tisch den Betreffenden angeredet: es iſt der \textsc{attaché} bei der engliſchen
                  Botſchaft in Paris\oindex{Botschaft von Grossbritannien in Paris@\textbf{Botschaft von Großbritannien in Paris}|pw}{ }\textsc{Mr. van Sittard\pwindex{Vansittart, Robert Gilbert 25.06.1881 – 14.02.1957@\textsc{Vansittart, Robert Gilbert} (25.06.1881 – 14.02.1957), \emph{Diplomat}|pw}}, ein ungewöhnlicher junger Menſch, ganz jung, 23, ein Spieler, ſehr elegant,
               hat die beſte Prüfung gemacht, die in {\pb}der engliſchen Diplomatie ſeit
               vielen Jahren vorgekommen iſt, war \textsc{head-boy} von \textsc{Eton}\orgindex{Eton College@Eton College|pw}, ſchreibt auf franzöſiſch Theaterſtücke und hat was das netteſte iſt, eine
               unglaublich intenſive Liebe für Ihre Sachen. Er findet ſie weit beſſer als alles was
               auf allen engliſchen und franzöſiſchen Theatern zuſa{\geminationm}en
               aufgeführt wird, worin er ja Recht haben dürfte.\hspace*{1.5em}Als
               ich ihn beſuchte (er {\pb}iſt bis
                     23\textsuperscript{ten}{ }Altauſſee, \textsc{Villa
                     Franckenſtein}\oindex{Villa Franckenstein@\textbf{Villa Franckenstein}|pw}) lag auf dem Tisch Vermächtnis\pwindex{Schnitzler, Arthur 15.05.1862 – 21.10.1931@\textsc{Schnitzler, Arthur} (15.05.1862 – 21.10.1931), \emph{Schriftsteller, Mediziner}!Vermaechtnis. Schauspiel in drei Akten1898@\strich\emph{Das Vermächtnis. Schauspiel in drei Akten} {[}1898{]}|pw}, Beatrice\pwindex{Schnitzler, Arthur 15.05.1862 – 21.10.1931@\textsc{Schnitzler, Arthur} (15.05.1862 – 21.10.1931), \emph{Schriftsteller, Mediziner}!Schleier der Beatrice. Schauspiel in fuenf Akten1900-12-01@\strich\emph{Der Schleier der Beatrice. Schauspiel in fünf Akten} {[}1900-12-01{]}|pw}, Sterben\pwindex{Schnitzler, Arthur 15.05.1862 – 21.10.1931@\textsc{Schnitzler, Arthur} (15.05.1862 – 21.10.1931), \emph{Schriftsteller, Mediziner}!Sterben. Novelle1894-10-01 – 1894-12-01@\strich\emph{Sterben. Novelle} {[}1894-10-01 – 1894-12-01{]}|pw}. Diese 3 waren das einzige was er nicht kannte und nachzuholen
               hatte. Er ſagt alſo: es geſchieht ihm nun ſchon das zweitemal das er ganz auf dem
               Punkt iſt, ſeine von Ihnen autoriſierte Überſetzung von 3-4 Anatol\pwindex{Schnitzler, Arthur 15.05.1862 – 21.10.1931@\textsc{Schnitzler, Arthur} (15.05.1862 – 21.10.1931), \emph{Schriftsteller, Mediziner}!Anatol1892-10-29@\strich\emph{Anatol} {[}1892-10-29{]}|pw}ſachen auf eine {\pb}gute Bühne zu bringen und daſs im
               letzten Moment Einſpruch erhoben wird von Leuten, denen Sie \uline{auch} die Autoriſation erteilt haben. Sonderbarerweiſe kam \uline{während} ich mit ihm redete ein Brief, in dem abermals ein
                  Regiſſeur\pwindex{St. John, Christopher 24.10.1871 – 20.10.1960@\textsc{St. John, Christopher} (24.10.1871 – 20.10.1960), \emph{Regisseurin, Übersetzerin}|pwv}{ }ſchreibt: »wenn \textsc{Mr.
                  Schnitzler} fortfährt, ſich ſo \uline{außerordentlich}
               zu benehmen, wird niemand in England\oindex{England@\textbf{England}|pw} mehr etwas
               von ihm wiſſen {\pb}wollen.« Was liegt
               da vor? ich kenne Ihre ungewöhnliche Exactheit und habe \textsc{van Sittard}\pwindex{Vansittart, Robert Gilbert 25.06.1881 – 14.02.1957@\textsc{Vansittart, Robert Gilbert} (25.06.1881 – 14.02.1957), \emph{Diplomat}|pw} verſichert, es muſs da ein Schwindel vorliegen. Bitte klären Sie ſogleich ihn
               oder mich auf, damit er nöthigenfalls durch einen Proceſs da Klarheit ſchafft und
               ſeinen ſo ſchönen und ziemlich ungewöhnlichen Eifer nicht verliert. Es iſt ein recht
               intereſſanter Mensch.\pend
           \pstart
           \centering{}{\pb}–\pend
           \pstart
           \noindent{}Ich bin alſo von der Waffenübung befreit, d. h. ſie iſt auf den November
               verſchoben, wo ſie mich nicht ſehr geniert.\hspace*{1.5em}So
               treffen wir uns hoffentlich. Wo? Iſchl\oindex{Bad Ischl@\textbf{Bad Ischl}|pw}, ich meine
               der Fleck Iſchl\oindex{Bad Ischl@\textbf{Bad Ischl}|pw}{ }ſelbſt, wird mir vielleicht dadurch unmöglich, daſs
               meine Schwiegermutter\pwindex{Schlesinger, Franziska 17.08.1851 – 11.08.1932@\textsc{Schlesinger, Franziska} (17.08.1851 – 11.08.1932)|pwv}
               hingeht. Da käme ich eventuell an den Wolfgangſee\oindex{Wolfgangsee@\textbf{Wolfgangsee}|pw}, jedenfalls rechne ich auf Zuſa{\geminationm}enſein, d. h. für den Fall daſs Sie die Mutter\pwindex{Schnitzler, Louise 1840-07-08 – 1911-09-09@\textsc{Schnitzler, Louise} (1840-07-08 – 1911-09-09)|pwv}{ }\uline{nicht} mithaben.\pend
           \pstart
           Von Herzen Ihr{\\[\baselineskip]}\spacefill\mbox{Hugo}\pend
           \leftskip=0em{}
         
         \endnumbering\mylabel{h}\end{ledgroupsized}  \newcommand{\dateiname}{L01426}\newcommand{\titel}{Hugo von Hofmannsthal an Arthur Schnitzler, 12. 8. 1904}\newcommand{\editorInnen}{Martin Anton Müller und Gerd-Hermann Susen}%% latex-leseansicht-abspann.tex
%% Abspann für die Leseansicht.
%% Der Schalter \ifkorrekturansicht ist bereits durch den Vorspann gesetzt.

%% latex-abspann.tex
%% Gemeinsamer Abspann für Korrekturansicht und Leseansicht.
%% Setzt den Schalter \ifkorrekturansicht voraus (gesetzt in den
%% einbindenden Dateien latex-korrekturansicht-abspann.tex bzw.
%% latex-leseansicht-abspann.tex).
%% ---------------------------------------------------------------

\normalsize

% Das esempio-Environment wird nur in der Leseansicht benötigt
\ifkorrekturansicht\else
\newenvironment{esempio}[3]%
{
    \vspace{1.5ex}
    \rlap{\underline{#1}}
    \par
    \setlength{\parindent}{0cm}
    \nopagebreak
    \leftskip=#2cm
    \rightskip=#3cm
}
{
    \par
}
\fi

\doendnotes{C}
\bigskip
\vfill

\clearpage

\footnotesize

\ifkorrekturansicht
  \lohead{\textsc{register}}
\fi

% theindex-Environment neu definieren ohne reledmac
\makeatletter
\renewenvironment{theindex}{%
  \ifkorrekturansicht
    \section*{\indexname}%
  \else
    \subsubsection*{Index der erwähnten Entitäten}%
  \fi
  \setlength{\parindent}{0pt}%
  \setlength{\parskip}{0pt plus 0.3pt}%
  \let\item\@idxitem
}{%
  \ifkorrekturansicht\clearpage\fi
}
\makeatother

\IfFileExists{\jobname-pw.ind}{\input{\jobname-pw.ind}}{}

% Quellenangabe nur in der Leseansicht
\ifkorrekturansicht\else
% Fallback-Definitionen, falls die .tex-Datei \titel etc. nicht gesetzt hat
\providecommand{\titel}{}
\providecommand{\editorInnen}{}
\providecommand{\dateiname}{\jobname}

\vspace{3cm}

\vfill

\footnotesize
\textsc{Quelle}: \titel. Herausgegeben von {\editorInnen}. In: \emph{Arthur Schnitzler: Briefwechsel mit Autorinnen und Autoren}.
 Digitale Edition, https://schnitzler-briefe.acdh.oeaw.ac.at/{\dateiname}.html (Stand \today)
\fi

\end{document}


      