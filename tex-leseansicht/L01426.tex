%% latex-korrekturansicht-vorspann.tex
%% Vorspann für die Korrekturansicht.
%% Lädt die gemeinsame Datei latex-vorspann.tex mit gesetztem Schalter.

\newif\ifkorrekturansicht
\korrekturansichttrue

\input{../tex-inputs/latex-vorspann}


\section[Hugo von Hofmannsthal an Arthur Schnitzler, 12. 8. 1904]{L01426 Hugo von Hofmannsthal an Arthur Schnitzler, 12. 8. 1904}
\nopagebreak\mylabel{L01426v}
\rehead{ }\normalsize\beginnumbering\briefempfaengerindex{Schnitzler, Arthur@\textsc{Schnitzler, Arthur}!zzzHofmannsthal, Hugo von@\emph{von Hugo von Hofmannsthal}!1904-08-121@{12. 8. 1904}|(be}
\toendnotes[C]{\smallbreak\pagebreak[2]}\Standort{CUL, Schnitzler, B 43.}
\physDesc{Brief, 2 Blätter, 8 Seiten, 2805 Zeichen
\newline{}Handschrift: schwarze Tinte, deutsche Kurrent
\newline{}Schnitzler: mit Bleistift die Jahreszahl ergänzt: »904« 
\newline{}Ordnung: 1) mit Bleistift von unbekannter Hand nummeriert: »\strikeout{254}«  2) mit Bleistift von unbekannter Hand nummeriert:
                                    »232.1« bzw. »232.2«}
\buchAbdrucke{\weitereDrucke{Hugo von Hofmannsthal, Arthur Schnitzler: \emph{Briefwechsel}. Frankfurt am Main: \emph{S. Fischer} 1964, S. 196.} }\toendnotes[C]{\smallbreak}
\pstart
           \raggedleft{}{\pb}Markt Auſſee, Ramgut\oindex{Ramgut@\textbf{Ramgut}, \emph{Schloss (K.SLS)}|pw}{\\}12 VIII.\pend
           
\pstart{}Lieber,\pend\vspace{0.5em}
\pstart
           Ich ging gegen Abend vom Markt\oindex{Bad Aussee@\textbf{Bad Aussee}, \emph{P.PPLA3}|pw} herauf, begegnete
               drei Frauen deren Geſichter ich nicht ſehen konnte. Hinter mir ſagte eine davon, ihr
               Gespräch fortſetzend: »und dann ſind wir mit ihnen auseinandergekommen, das war zu
               der Zeit wie ſie mit dem Arthur Schnitzler verlobt war«{\dots}
               und die andere ſagte beſtätigend: »ja, zu der Zeit war ſie mit {\pb}dem Arthur Schnitzler verlobt«.
               Von wem kann da die Rede geweſen ſein? Vielleicht von der ewigen Minnie\pwindex{Schaffgotsch, Hermine von 25.11.1871 – 25.11.1928@\textsc{Schaffgotsch, Hermine von} (25.11.1871 – 25.11.1928)|pw}?\pend
           
\pstart
           \numberlinefalse{}\centering{}–\numberlinetrue{}\pend
           
\pstart
           Eine Stunde ſpäter ſoupierte ich mit Leuten: da hörte ich mir gegenüber einen \strikeout{zus} zu ſeinem Nachbar ſagen, auf engliſch: »und dann
               hat mir der Manager geſagt, wenn Schnitzler fortfährt, {\pb}ſolche Sachen zu machen, wird man
               ihn als einen litterariſchen Pariah behandeln (wörtlich.)« Das intereſſierte mich
               doch ſehr und ich habe nach Tisch den Betreffenden angeredet: es iſt der \textsc{attaché} bei der engliſchen
                  Botſchaft in Paris\oindex{Botschaft von Grossbritannien in Paris@\textbf{Botschaft von Großbritannien in Paris}, \emph{Botschaft (K.BTS)}|pw}{ }\textsc{Mr. van Sittard\pwindex{Vansittart, Robert Gilbert 25.06.1881 – 14.02.1957@\textsc{Vansittart, Robert Gilbert} (25.06.1881 – 14.02.1957), \emph{Diplomat/Diplomatin}|pw}}, ein ungewöhnlicher junger Menſch, ganz jung, 23, ein Spieler, ſehr elegant,
               hat die beſte Prüfung gemacht, die in {\pb}der engliſchen Diplomatie ſeit
               vielen Jahren vorgekommen iſt, war \textsc{head-boy} von \textsc{Eton}\orgindex{Eton College@Eton College|pw}, ſchreibt auf franzöſiſch Theaterſtücke und hat was das netteſte iſt, eine
               unglaublich intenſive Liebe für Ihre Sachen. Er findet ſie weit beſſer als alles was
               auf allen engliſchen und franzöſiſchen Theatern zuſa{\geminationm}en
               aufgeführt wird, worin er ja Recht haben dürfte.\hspace*{1.5em}Als
               ich ihn beſuchte (er {\pb}iſt bis
                     23\textsuperscript{ten}{ }Altauſſee, \textsc{Villa
                     Franckenſtein}\oindex{Villa Franckenstein@\textbf{Villa Franckenstein}, \emph{Gebäude (K.GBD)}|pw}) lag auf dem Tisch Vermächtnis\pwindex{Vermaechtnis. Schauspiel in drei Akten@\emph{Das Vermächtnis. Schauspiel in drei Akten}|pw}, Beatrice\pwindex{Schleier der Beatrice. Schauspiel in fuenf Akten@\emph{Der Schleier der Beatrice. Schauspiel in fünf Akten}|pw}, Sterben\pwindex{Sterben. Novelle@\emph{Sterben. Novelle}|pw}. Diese 3 waren das einzige was er nicht kannte und nachzuholen
               hatte. Er ſagt alſo: es geſchieht ihm nun ſchon das zweitemal das er ganz auf dem
               Punkt iſt, ſeine von Ihnen autoriſierte Überſetzung von 3-4 Anatol\pwindex{Anatol@\emph{Anatol}|pw}ſachen auf eine {\pb}gute Bühne zu bringen und daſs im
               letzten Moment Einſpruch erhoben wird von Leuten, denen Sie \uline{auch} die Autoriſation erteilt haben. Sonderbarerweiſe kam \uline{während} ich mit ihm redete ein Brief, in dem abermals ein
                  Regiſſeur\pwindex{St. John, Christopher 24.10.1871 – 20.10.1960@\textsc{St. John, Christopher} (24.10.1871 – 20.10.1960), \emph{Regisseur/Regisseurin, Übersetzer/Übersetzerin}|pwv}{ }ſchreibt: »wenn \textsc{Mr.
                  Schnitzler} fortfährt, ſich ſo \uline{außerordentlich}
               zu benehmen, wird niemand in England\oindex{England@\textbf{England}, \emph{A.ADM1}|pw} mehr etwas
               von ihm wiſſen {\pb}wollen.« Was liegt
               da vor? ich kenne Ihre ungewöhnliche Exactheit und habe \textsc{van Sittard}\pwindex{Vansittart, Robert Gilbert 25.06.1881 – 14.02.1957@\textsc{Vansittart, Robert Gilbert} (25.06.1881 – 14.02.1957), \emph{Diplomat/Diplomatin}|pw} verſichert, es muſs da ein Schwindel vorliegen. Bitte klären Sie ſogleich ihn
               oder mich auf, damit er nöthigenfalls durch einen Proceſs da Klarheit ſchafft und
               ſeinen ſo ſchönen und ziemlich ungewöhnlichen Eifer nicht verliert. Es iſt ein recht
               intereſſanter Mensch.\pend
           
\pstart
           \centering{}{\pb}–\pend
           
\pstart
           Ich bin alſo von der Waffenübung befreit, d. h. ſie iſt auf den November
               verſchoben, wo ſie mich nicht ſehr geniert.\hspace*{1.5em}So
               treffen wir uns hoffentlich. Wo? Iſchl\oindex{Bad Ischl@\textbf{Bad Ischl}, \emph{P.PPL}|pw}, ich meine
               der Fleck Iſchl\oindex{Bad Ischl@\textbf{Bad Ischl}, \emph{P.PPL}|pw}{ }ſelbſt, wird mir vielleicht dadurch unmöglich, daſs
               meine Schwiegermutter\pwindex{Schlesinger, Franziska 17.08.1851 – 11.08.1932@\textsc{Schlesinger, Franziska} (17.08.1851 – 11.08.1932)|pwv}
               hingeht. Da käme ich eventuell an den Wolfgangſee\oindex{Wolfgangsee@\textbf{Wolfgangsee}, \emph{See (N.SEE)}|pw}, jedenfalls rechne ich auf Zuſa{\geminationm}enſein, d. h. für den Fall daſs Sie die Mutter\pwindex{Schnitzler, Louise 1840-07-08 – 1911-09-09@\textsc{Schnitzler, Louise} (1840-07-08 – 1911-09-09)|pwv}{ }\uline{nicht} mithaben.\pend
           
\pstart
           Von Herzen Ihr{\\[\baselineskip]}\spacefill\mbox{Hugo}\pend
           \leftskip=0em{}\selectlanguage{ngerman}\endnumbering\briefempfaengerindex{Schnitzler, Arthur@\textsc{Schnitzler, Arthur}!zzzHofmannsthal, Hugo von@\emph{von Hugo von Hofmannsthal}!1904-08-121@{12. 8. 1904}|)be}\mylabel{L01426h}  \normalsize

\doendnotes{C}
\bigskip
\vfill

\clearpage

\footnotesize

\lohead{\textsc{register}}

% Definiere theindex-Environment komplett neu ohne reledmac
\makeatletter
\renewenvironment{theindex}{%
  \section*{\indexname}%
  \setlength{\parindent}{0pt}%
  \setlength{\parskip}{0pt plus 0.3pt}%
  \let\item\@idxitem
}{%
  \clearpage
}
\makeatother

\IfFileExists{\jobname-pw.ind}{\input{\jobname-pw.ind}}{}

\end{document}

      