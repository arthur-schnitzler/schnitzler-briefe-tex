%% latex-korrekturansicht-vorspann.tex
%% Vorspann für die Korrekturansicht.
%% Lädt die gemeinsame Datei latex-vorspann.tex mit gesetztem Schalter.

\newif\ifkorrekturansicht
\korrekturansichttrue

\input{../tex-inputs/latex-vorspann}


\section[Hugo von Hofmannsthal an Arthur Schnitzler, 6. 10. 1905]{L01557 Hugo von Hofmannsthal an Arthur Schnitzler, 6. 10. 1905}
\nopagebreak\mylabel{L01557v}
\rehead{ }\normalsize\beginnumbering\briefempfaengerindex{Schnitzler, Arthur@\textsc{Schnitzler, Arthur}!zzzHofmannsthal, Hugo von@\emph{von Hugo von Hofmannsthal}!1905-10-061@{6. 10. 1905}|(be}
\toendnotes[C]{\smallbreak\pagebreak[2]}\Standort{CUL, Schnitzler, B 43.}
\physDesc{Postkarte, 367 Zeichen
\newline{}Handschrift: schwarze Tinte, deutsche Kurrent
\newline{}Versand: Stempel: »\nobreak{}\oindex{Rodaun@\textbf{Rodaun}, \emph{A.ADM4}|pwk}Ro{[}da{]}un, 6. 10. \textcolor{gray}{0}5, 9–12V\nobreak{}«.  
\newline{}Schnitzler: mit Bleistift datiert: »2/10 90\textcolor{gray}{5}« 
\newline{}Ordnung: 1) mit Bleistift von unbekannter Hand nummeriert:
                                    »255«  2) mit Bleistift von unbekannter Hand nummeriert:
                                    »258b«}
\buchAbdrucke{\weitereDrucke{Hugo von Hofmannsthal, Arthur Schnitzler: \emph{Briefwechsel}. Frankfurt am Main: \emph{S. Fischer} 1964, S. 216.} }\toendnotes[C]{\smallbreak}\pstart{}{\pb}\textsc{Herrn D\textsuperscript{r} Arthur Schnitzler}\pend{}\pstart{}\textsc{Wien}\oindex{Wien@\textbf{Wien}, \emph{A.ADM2}|pw}\pend{}\pstart{}\textsc{XVIII Spöttelgasse 7}.\oindex{Edmund-Weiss-Gasse 7@\textbf{Edmund-Weiß-Gasse 7}, \emph{Wohngebäude (K.WHS)}|pw}\pend{}{\bigskip}\vspace{1em}
\pstart
           \noindent{}{\pb}lieber, freuen uns ja doch trotz der W.\pwindex{Witt, Lotte 24.04.1870 – 28.12.1938@\textsc{Witt, Lotte} (24.04.1870 – 28.12.1938), \emph{Schauspieler/Schauspielerin}|pw}{ }ſehr auf eine \label{K_L01557-1v}\edtext{Première\pwindex{Zwischenspiel. Komoedie in drei Akten@\emph{Zwischenspiel. Komödie in drei Akten}|pwv}}{\lemma{\textnormal{\emph{Première}}}\Cendnote{\textnormal{von \emph{Zwischenspiel}\pwindex{Zwischenspiel. Komoedie in drei Akten@\emph{Zwischenspiel. Komödie in drei Akten}|pwk} am 12. 10. 1905}}}\label{K_L01557-1} von Ihnen. Bitten um 2 mittlere Parkettſitze oder vordere (nicht
                  rückwärtige){[}.{]} Wegen ſchlechter Poſt ſchicken Sie ſie bitte an
                  \textsc{Schlesinger}\pwindex{Schlesinger, Franziska 17.08.1851 – 11.08.1932@\textsc{Schlesinger, Franziska} (17.08.1851 – 11.08.1932)|pw} für \textsc{Hofma{\geminationn}sthal}, I. \textsc{Elisabethstrasse 6}\oindex{Elisabethstrasse [Wien]@\textbf{Elisabethstraße [Wien]}, \emph{Straße (K.STR)}|pw}. Bitte bezahlen Sie ſie indeſſen für mich. Herzlich, und auf Wiederſehen
               nachher!\pend
           \pstart \spacefill\mbox{Hugo.}\pend{}\selectlanguage{ngerman}\endnumbering\briefempfaengerindex{Schnitzler, Arthur@\textsc{Schnitzler, Arthur}!zzzHofmannsthal, Hugo von@\emph{von Hugo von Hofmannsthal}!1905-10-061@{6. 10. 1905}|)be}\mylabel{L01557h}  \normalsize

\doendnotes{C}
\bigskip
\vfill

\clearpage

\footnotesize

\lohead{\textsc{register}}

% Definiere theindex-Environment komplett neu ohne reledmac
\makeatletter
\renewenvironment{theindex}{%
  \section*{\indexname}%
  \setlength{\parindent}{0pt}%
  \setlength{\parskip}{0pt plus 0.3pt}%
  \let\item\@idxitem
}{%
  \clearpage
}
\makeatother

\IfFileExists{\jobname-pw.ind}{\input{\jobname-pw.ind}}{}

\end{document}

      