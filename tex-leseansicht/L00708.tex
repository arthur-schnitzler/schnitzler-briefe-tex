%% latex-leseansicht-vorspann.tex
%% Vorspann für die Leseansicht.
%% Lädt die gemeinsame Datei latex-vorspann.tex mit nicht gesetztem Schalter.

\newif\ifkorrekturansicht
\korrekturansichtfalse

\input{../tex-inputs/latex-vorspann}


\section[Hugo von Hofmannsthal an Arthur Schnitzler, {[}20. 7. 1897{]}]{L00708 Hugo von Hofmannsthal an Arthur Schnitzler, {[}20. 7. 1897{]}}
\nopagebreak\mylabel{L00708v}
\rehead{ }\normalsize\beginnumbering\briefempfaengerindex{Schnitzler, Arthur@\textsc{Schnitzler, Arthur}!zzzHofmannsthal, Hugo von@\emph{von Hugo von Hofmannsthal}!1897-07-202@{{[}20. 7. 1897{]}}|(be}
\toendnotes[C]{\smallbreak\pagebreak[2]}
\correspDesc{Versand  durch Hugo von Hofmannsthal am [20. 7. 1897] in Bad Fusch
\newline{}Erhalt  durch Arthur Schnitzler im Zeitraum [21. 7. 1897
                  – 25. 7. 1897?] in Wien}\toendnotes[C]{\smallbreak}
\Standort{CUL, Schnitzler, B 43.}
\physDesc{Brief, 1 Blatt, 3 Seiten, 622 Zeichen
\newline{}Handschrift: Bleistift, deutsche Kurrent
\newline{}Schnitzler: mit Bleistift datiert: »et\textcolor{gray}{w}{ }20 Juli 97« 
\newline{}Ordnung: 1) mit Bleistift von unbekannter Hand nummeriert: »\strikeout{99}«  2) mit Bleistift von unbekannter Hand nummeriert:
                                    »101«}
\buchAbdrucke{\weitereDrucke{Hugo von Hofmannsthal, Arthur Schnitzler: \emph{Briefwechsel}. Herausgegeben von Therese Nickl und Heinrich Schnitzler. Frankfurt am Main: \emph{S. Fischer} 1964, S. 93.} }
\pstart
           \raggedleft{}{\pb}Dienstag\pend
           
\pstart{}lieber Arthur\pend\vspace{0.5em}
\pstart
           bitte{ }ſeien Sie noch vor Ihrer Abreiſe{ }ſo gut mir hierher den Namen und die Adreſſe
               des Iſchl\oindex{Bad Ischl@\textbf{Bad Ischl}|pw}er Arztes zu{ }ſchreiben, den Sie für den
               beſten halten (\uline{neben}{ }Widerhofer\pwindex{Widerhofer, Hermann 24.\,3.\,1832 – 28.\,7.\,1901@\textsc{Widerhofer, Hermann} (24.\,3.\,1832 – 28.\,7.\,1901), \emph{Mediziner}|pw}.) Poldy\pwindex{Andrian-Werburg, Leopold von 9.\,5.\,1875 Berlin – 19.\,11.\,1951 Fribourg@\textsc{Andrian-Werburg, Leopold von} (9.\,5.\,1875 Berlin – 19.\,11.\,1951 Fribourg), \emph{Schriftsteller, Diplomat}|pw}’s Nervoſität hat{ }ſich nämlich in eine unausgeſetzte martervolle Angſt
               vor Schwindſucht {\pb}verwandelt, zum
               Theil hervorgerufen durch eine unvorſichtige aber gar nicht wirklich beängſtigende
               Äußerung Schrötters\pwindex{Schrötter von Kristelli, Leopold 5.\,2.\,1837 Graz – 20.\,4.\,1908 Wien@\textsc{Schrötter von Kristelli, Leopold} (5.\,2.\,1837 Graz – 20.\,4.\,1908 Wien), \emph{Mediziner, Arzt}|pw}. Er muſs alſo von Auſſee\oindex{Altaussee@\textbf{Altaussee}, \emph{Verwaltungsgebiet}|pw} aus die Möglichkeit haben,{ }ſooft er will
               einen Arzt zu{ }ſehen, der ihm die Unſchädlichkeit {\pb}des betreffenden Symptomes, das er{ }ſich von Tag zu Tag wechſelnd einredet, nachweist.\pend
           
\pstart
           Im voraus dankt Ihnen{\\[\baselineskip]} Ihr\spacefill\mbox{Hugo.}\pend
           \leftskip=0em{}\selectlanguage{ngerman}\endnumbering\briefempfaengerindex{Schnitzler, Arthur@\textsc{Schnitzler, Arthur}!zzzHofmannsthal, Hugo von@\emph{von Hugo von Hofmannsthal}!1897-07-202@{{[}20. 7. 1897{]}}|)be}\mylabel{L00708h}  \newcommand{\dateiname}{L00708}\newcommand{\titel}{Hugo von Hofmannsthal an Arthur Schnitzler, [20. 7. 1897]}\newcommand{\editorInnen}{Martin Anton Müller und Gerd-Hermann Susen}%% latex-leseansicht-abspann.tex
%% Abspann für die Leseansicht.
%% Der Schalter \ifkorrekturansicht ist bereits durch den Vorspann gesetzt.

%% latex-abspann.tex
%% Gemeinsamer Abspann für Korrekturansicht und Leseansicht.
%% Setzt den Schalter \ifkorrekturansicht voraus (gesetzt in den
%% einbindenden Dateien latex-korrekturansicht-abspann.tex bzw.
%% latex-leseansicht-abspann.tex).
%% ---------------------------------------------------------------

\normalsize

% Das esempio-Environment wird nur in der Leseansicht benötigt
\ifkorrekturansicht\else
\newenvironment{esempio}[3]%
{
    \vspace{1.5ex}
    \rlap{\underline{#1}}
    \par
    \setlength{\parindent}{0cm}
    \nopagebreak
    \leftskip=#2cm
    \rightskip=#3cm
}
{
    \par
}
\fi

\doendnotes{C}
\bigskip
\vfill

\clearpage

\footnotesize

\ifkorrekturansicht
  \lohead{\textsc{register}}
\fi

% theindex-Environment neu definieren ohne reledmac
\makeatletter
\renewenvironment{theindex}{%
  \ifkorrekturansicht
    \section*{\indexname}%
  \else
    \subsubsection*{Index der erwähnten Entitäten}%
  \fi
  \setlength{\parindent}{0pt}%
  \setlength{\parskip}{0pt plus 0.3pt}%
  \let\item\@idxitem
}{%
  \ifkorrekturansicht\clearpage\fi
}
\makeatother

\IfFileExists{\jobname-pw.ind}{\input{\jobname-pw.ind}}{}

% Quellenangabe nur in der Leseansicht
\ifkorrekturansicht\else
% Fallback-Definitionen, falls die .tex-Datei \titel etc. nicht gesetzt hat
\providecommand{\titel}{}
\providecommand{\editorInnen}{}
\providecommand{\dateiname}{\jobname}

\vspace{3cm}

\vfill

\footnotesize
\textsc{Quelle}: \titel. Herausgegeben von {\editorInnen}. In: \emph{Arthur Schnitzler: Briefwechsel mit Autorinnen und Autoren}.
 Digitale Edition, https://schnitzler-briefe.acdh.oeaw.ac.at/{\dateiname}.html (Stand \today)
\fi

\end{document}


