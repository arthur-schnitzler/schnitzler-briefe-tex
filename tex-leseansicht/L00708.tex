%% latex-korrekturansicht-vorspann.tex
%% Vorspann für die Korrekturansicht.
%% Lädt die gemeinsame Datei latex-vorspann.tex mit gesetztem Schalter.

\newif\ifkorrekturansicht
\korrekturansichttrue

\input{../tex-inputs/latex-vorspann}


\section[Hugo von Hofmannsthal an Arthur Schnitzler, {[}20. 7. 1897{]}]{L00708 Hugo von Hofmannsthal an Arthur Schnitzler, {[}20. 7. 1897{]}}
\nopagebreak\mylabel{L00708v}
\rehead{ }\normalsize\beginnumbering\briefempfaengerindex{Schnitzler, Arthur@\textsc{Schnitzler, Arthur}!zzzHofmannsthal, Hugo von@\emph{von Hugo von Hofmannsthal}!1897-07-202@{{[}20. 7. 1897{]}}|(be}
\toendnotes[C]{\smallbreak\pagebreak[2]}\Standort{CUL, Schnitzler, B 43.}
\physDesc{Brief, 1 Blatt, 3 Seiten, 622 Zeichen
\newline{}Handschrift: Bleistift, deutsche Kurrent
\newline{}Schnitzler: mit Bleistift datiert: »et\textcolor{gray}{w}{ }20 Juli 97« 
\newline{}Ordnung: 1) mit Bleistift von unbekannter Hand nummeriert: »\strikeout{99}«  2) mit Bleistift von unbekannter Hand nummeriert:
                                    »101«}
\buchAbdrucke{\weitereDrucke{Hugo von Hofmannsthal, Arthur Schnitzler: \emph{Briefwechsel}. Frankfurt am Main: \emph{S. Fischer} 1964, S. 93.} }
\pstart
           \raggedleft{}{\pb}Dienstag\pend
           
\pstart{}lieber Arthur\pend\vspace{0.5em}
\pstart
           bitte ſeien Sie noch vor Ihrer Abreiſe ſo gut mir hierher den Namen und die Adreſſe
               des Iſchl\oindex{Bad Ischl@\textbf{Bad Ischl}, \emph{P.PPL}|pw}er Arztes zu ſchreiben, den Sie für den
               beſten halten (\uline{neben}{ }Widerhofer\pwindex{Widerhofer, Hermann 24.03.1832 – 28.07.1901@\textsc{Widerhofer, Hermann} (24.03.1832 – 28.07.1901), \emph{Mediziner/Medizinerin}|pw}.) Poldy\pwindex{Andrian-Werburg, Leopold von 09.05.1875 – 19.11.1951@\textsc{Andrian-Werburg, Leopold von} (09.05.1875 – 19.11.1951), \emph{Schriftsteller/Schriftstellerin, Diplomat/Diplomatin}|pw}’s Nervoſität hat ſich nämlich in eine unausgeſetzte martervolle Angſt
               vor Schwindſucht {\pb}verwandelt, zum
               Theil hervorgerufen durch eine unvorſichtige aber gar nicht wirklich beängſtigende
               Äußerung Schrötters\pwindex{Schroetter von Kristelli, Leopold 1837-02-05 – 1908-04-20@\textsc{Schrötter von Kristelli, Leopold} (1837-02-05 – 1908-04-20), \emph{Mediziner/Medizinerin, Arzt/Ärztin}|pw}. Er muſs alſo von Auſſee\oindex{Altaussee@\textbf{Altaussee}, \emph{A.ADM3}|pw} aus die Möglichkeit haben, ſooft er will
               einen Arzt zu ſehen, der ihm die Unſchädlichkeit {\pb}des betreffenden Symptomes, das er
               ſich von Tag zu Tag wechſelnd einredet, nachweist.\pend
           
\pstart
           Im voraus dankt Ihnen{\\[\baselineskip]} Ihr\spacefill\mbox{Hugo.}\pend
           \leftskip=0em{}\selectlanguage{ngerman}\endnumbering\briefempfaengerindex{Schnitzler, Arthur@\textsc{Schnitzler, Arthur}!zzzHofmannsthal, Hugo von@\emph{von Hugo von Hofmannsthal}!1897-07-202@{{[}20. 7. 1897{]}}|)be}\mylabel{L00708h}  \normalsize

\doendnotes{C}
\bigskip
\vfill

\clearpage

\footnotesize

\lohead{\textsc{register}}

% Definiere theindex-Environment komplett neu ohne reledmac
\makeatletter
\renewenvironment{theindex}{%
  \section*{\indexname}%
  \setlength{\parindent}{0pt}%
  \setlength{\parskip}{0pt plus 0.3pt}%
  \let\item\@idxitem
}{%
  \clearpage
}
\makeatother

\IfFileExists{\jobname-pw.ind}{\input{\jobname-pw.ind}}{}

\end{document}

      