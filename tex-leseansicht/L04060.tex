%% latex-leseansicht-vorspann.tex
%% Vorspann für die Leseansicht.
%% Lädt die gemeinsame Datei latex-vorspann.tex mit nicht gesetztem Schalter.

\newif\ifkorrekturansicht
\korrekturansichtfalse

\input{../tex-inputs/latex-vorspann}


\section[Arthur Schnitzler an Gustav Schwarzkopf, 12. 4. 1903]{L04060 Arthur Schnitzler an Gustav Schwarzkopf, 12. 4. 1903}
\nopagebreak\mylabel{L04060v}
\rehead{ }\normalsize\beginnumbering\briefempfaengerindex{Schwarzkopf, Gustav@\textsc{Schwarzkopf, Gustav}!zzzSchnitzler, Arthur@\emph{von Arthur Schnitzler}!1903-04-121@{12. 4. 1903}|(be}
\toendnotes[C]{\smallbreak\pagebreak[2]}
\correspDesc{Versand  durch Arthur Schnitzler am 12. 4. 1903 in Wien
\newline{}Erhalt  durch Gustav Schwarzkopf im Zeitraum [12. 4. 1903 – 15. 4. 1903?] in Wien}\toendnotes[C]{\smallbreak}
\Standort{CUL, Schnitzler, B 96.}
\physDesc{Brief, 1 Blatt, 3 Seiten, 325 Zeichen
\newline{}Handschrift: Bleistift, deutsche Kurrent}\toendnotes[C]{\smallbreak}
\pstart
           \raggedleft{}{\pb}So{\geminationn}tag .{\\}12. 4. 903.\pend
           \vspace{0.5em}
\pstart
           lieber Guſtav, natürlich bin ich \label{K_L04060-1v}\edtext{längſt wieder in Wien\oindex{Wien@\textbf{Wien}, \emph{Verwaltungsgebiet}|pw}}{\lemma{\textnormal{\emph{längst wieder in Wien}}}\Cendnote{\textnormal{Er war am 10. 3. 1903 aus Berlin\oindex{Berlin@\textbf{Berlin}, \emph{Hauptstadt}|pwk}
                  zurückgekehrt.}}}\label{K_L04060-1}; das Wetter iſt ungeheuerlich geweſen. –\pend
           
\pstart
           Wollen Sie morgen Oſtermontag bei uns (Gentzgasse\oindex{Wien@\textbf{Wien}!XVIII., Währing@\textbf{XVIII., Währing}!Gentzgasse 110@\textbf{Gentzgasse 110}, \emph{Wohngebäude}|pw}) ½ 8{ }{\pb}\label{K_L04060-2v}\edtext{Nachtmahlen}{\lemma{\textnormal{\emph{Nachtmahlen}}}\Cendnote{\textnormal{Vgl. A. S.: \emph{Tagebuch}, 13. 4. 1903.}}}\label{K_L04060-2}, ſo ſind Sie
               willkommen.\pend
           
\pstart
           – Ich habe Ihrem Rath gefolgt – und den \textsc{Observer\orgindex{Observer. Alexander Weigl’s Unternehmen für Zeitungssausschnitte@Observer. Alexander Weigl’s Unternehmen für Zeitungssausschnitte|pw}} bis auf weiteres abgeſchrieben – befinde mich{ }ſehr wohl dabei.\pend
           
\pstart
           {\pb}HerzlGruſs{\\[\baselineskip]} Ihr{\\[\baselineskip]}\spacefill\mbox{Arthur}\pend
           \leftskip=0em{}\selectlanguage{ngerman}\endnumbering\briefempfaengerindex{Schwarzkopf, Gustav@\textsc{Schwarzkopf, Gustav}!zzzSchnitzler, Arthur@\emph{von Arthur Schnitzler}!1903-04-121@{12. 4. 1903}|)be}\mylabel{L04060h}
\begin{anhang}
\end{anhang}\newcommand{\dateiname}{L04060}\newcommand{\titel}{Arthur Schnitzler an Gustav Schwarzkopf, 12. 4. 1903}\newcommand{\editorInnen}{Herausgegeben von Jahnke, SelmaMüller, Martin Anton}%% latex-leseansicht-abspann.tex
%% Abspann für die Leseansicht.
%% Der Schalter \ifkorrekturansicht ist bereits durch den Vorspann gesetzt.

%% latex-abspann.tex
%% Gemeinsamer Abspann für Korrekturansicht und Leseansicht.
%% Setzt den Schalter \ifkorrekturansicht voraus (gesetzt in den
%% einbindenden Dateien latex-korrekturansicht-abspann.tex bzw.
%% latex-leseansicht-abspann.tex).
%% ---------------------------------------------------------------

\normalsize

% Das esempio-Environment wird nur in der Leseansicht benötigt
\ifkorrekturansicht\else
\newenvironment{esempio}[3]%
{
    \vspace{1.5ex}
    \rlap{\underline{#1}}
    \par
    \setlength{\parindent}{0cm}
    \nopagebreak
    \leftskip=#2cm
    \rightskip=#3cm
}
{
    \par
}
\fi

\doendnotes{C}
\bigskip
\vfill

\clearpage

\footnotesize

\ifkorrekturansicht
  \lohead{\textsc{register}}
\fi

% theindex-Environment neu definieren ohne reledmac
\makeatletter
\renewenvironment{theindex}{%
  \ifkorrekturansicht
    \section*{\indexname}%
  \else
    \subsubsection*{Index der erwähnten Entitäten}%
  \fi
  \setlength{\parindent}{0pt}%
  \setlength{\parskip}{0pt plus 0.3pt}%
  \let\item\@idxitem
}{%
  \ifkorrekturansicht\clearpage\fi
}
\makeatother

\IfFileExists{\jobname-pw.ind}{\input{\jobname-pw.ind}}{}

% Quellenangabe nur in der Leseansicht
\ifkorrekturansicht\else
% Fallback-Definitionen, falls die .tex-Datei \titel etc. nicht gesetzt hat
\providecommand{\titel}{}
\providecommand{\editorInnen}{}
\providecommand{\dateiname}{\jobname}

\vspace{3cm}

\vfill

\footnotesize
\textsc{Quelle}: \titel. Herausgegeben von {\editorInnen}. In: \emph{Arthur Schnitzler: Briefwechsel mit Autorinnen und Autoren}.
 Digitale Edition, https://schnitzler-briefe.acdh.oeaw.ac.at/{\dateiname}.html (Stand \today)
\fi

\end{document}


