%% latex-korrekturansicht-vorspann.tex
%% Vorspann für die Korrekturansicht.
%% Lädt die gemeinsame Datei latex-vorspann.tex mit gesetztem Schalter.

\newif\ifkorrekturansicht
\korrekturansichttrue

\input{../tex-inputs/latex-vorspann}


\section[ Felix Salten: Widmungsexemplar Der Schrei der Liebe für Arthur Schnitzler, 22. 10. 1904]{L03053 Felix Salten: Widmungsexemplar Der Schrei der Liebe für Arthur
               Schnitzler, 22. 10. 1904}
\nopagebreak\mylabel{L03053v}
\rehead{ }\normalsize\beginnumbering\briefempfaengerindex{Schnitzler, Arthur@\textsc{Schnitzler, Arthur}!zzzSalten, Felix@\emph{von Felix Salten}!1904-10-221@{22. 10. 1904}|(be}
\toendnotes[C]{\smallbreak\pagebreak[2]}\Standort{DLA, G:Schnitzler, Arthur (Sammlung Heinrich Schnitzler).}
\physDesc{, 67 Zeichen
\newline{}Handschrift: schwarze Tinte, lateinische Kurrent}
\pstart
           \noindent{}\centering{}{\pb}\textcolor{gray}{\textbf{Der {\\}Schrei der Liebe\pwindex{Schrei der Liebe. Novelle@\emph{Der Schrei der Liebe. Novelle}|pw}}}\pend
           {\vspace{1\baselineskip}}
\pstart
           Meinem lieben Arthur Schnitzler\pend
           \pstart herzl. \spacefill\mbox{Felix Salten}\pend{}
\pstart
           Wien\oindex{Wien@\textbf{Wien}, \emph{A.ADM2}|pw}, 22. X. 04.\pend
           {\vspace{1\baselineskip}}
\pstart
           \textcolor{gray}{\textbf{Bibl. mod. deutſcher Autoren\pwindex{Bibliothek moderner deutscher Autoren@\emph{Bibliothek moderner deutscher Autoren}|pw}. Band 5.}}\pend
           {\vspace{1\baselineskip}}
\pstart
           \centering{}{\pb}\textcolor{gray}{\textbf{Felix}}\pend
           
\pstart
           \centering{}\textcolor{gray}{\textbf{Salten}}\pend
           
\pstart
           \centering{}{\pb}\textcolor{gray}{\textbf{Der Schrei {\\}der Liebe\pwindex{Schrei der Liebe. Novelle@\emph{Der Schrei der Liebe. Novelle}|pw}}}\pend
           
\pstart
           \centering{}\textcolor{gray}{\textbf{Novelle}}\pend
           
\pstart
           \centering{}\textcolor{gray}{\textbf{Umſchlag von Richard
                  Lux\pwindex{Lux, Richard 1877-08-22 – 1939-11-23@\textsc{Lux, Richard} (1877-08-22 – 1939-11-23), \emph{Maler/Malerin, Grafiker/Grafikerin, Radierer/Radiererin}|pw}}}\pend
           
\pstart
           \centering{}\textcolor{gray}{\textbf{\textbf{Wiener Verlag\orgindex{Wiener Verlag@Wiener Verlag|pw}}}}\pend
           
\pstart
           \centering{}\textcolor{gray}{\textbf{Wien\oindex{Wien@\textbf{Wien}, \emph{A.ADM2}|pw} und Leipzig\oindex{Leipzig@\textbf{Leipzig}, \emph{P.PPLA3}|pw}}}\pend
           
\pstart
           \centering{}\textcolor{gray}{\textbf{1905}}\pend
           \selectlanguage{ngerman}\endnumbering\briefempfaengerindex{Schnitzler, Arthur@\textsc{Schnitzler, Arthur}!zzzSalten, Felix@\emph{von Felix Salten}!1904-10-221@{22. 10. 1904}|)be}\mylabel{L03053h}  \normalsize

\doendnotes{C}
\bigskip
\vfill

\clearpage

\footnotesize

\lohead{\textsc{register}}

% Definiere theindex-Environment komplett neu ohne reledmac
\makeatletter
\renewenvironment{theindex}{%
  \section*{\indexname}%
  \setlength{\parindent}{0pt}%
  \setlength{\parskip}{0pt plus 0.3pt}%
  \let\item\@idxitem
}{%
  \clearpage
}
\makeatother

\IfFileExists{\jobname-pw.ind}{\input{\jobname-pw.ind}}{}

\end{document}

      