%% latex-leseansicht-vorspann.tex
%% Vorspann für die Leseansicht.
%% Lädt die gemeinsame Datei latex-vorspann.tex mit nicht gesetztem Schalter.

\newif\ifkorrekturansicht
\korrekturansichtfalse

\input{../tex-inputs/latex-vorspann}


\section[Arthur Schnitzler an Stefan Zweig, 10. 5. 1928]{L03757 Arthur Schnitzler an Stefan Zweig, 10. 5. 1928}
\nopagebreak\mylabel{L03757v}
\rehead{ }\normalsize\beginnumbering\briefempfaengerindex{Zweig, Stefan@\textsc{Zweig, Stefan}!zzzSchnitzler, Arthur@\emph{von Arthur Schnitzler}!1928-05-101@{10. 5. 1928}|(be}
\toendnotes[C]{\smallbreak\pagebreak[2]}
\correspDesc{Versand  durch Arthur Schnitzler am 10. 5. 1928 in Wien
\newline{}Erhalt  durch Stefan Zweig im Zeitraum [11. 5. 1928
                  – 15. 5. 1928?] in Salzburg}\toendnotes[C]{\smallbreak}
\Standort{Jerusalem, National Library of Israel, ARC. Ms. Var. 305 1 58 Stefan Zweig Collection.}
\physDesc{Postkarte, 1026 Zeichen
\newline{}Handschrift: schwarze Tinte, lateinische Kurrent
\newline{}Versand: Stempel: »\nobreak{}\oindex{XVIII., Währing@\textbf{XVIII., Währing}, \emph{Verwaltungsgebiet}|pwk}Wien 110, 10. \textcolor{gray}{V}. 2\textcolor{gray}{8}, 17\nobreak{}«.  }\toendnotes[C]{\smallbreak}\pstart{}{\pb}\label{T_L03757-1v}\edtext{\textcolor{gray}{\textbf{A. S.}}}{\lemma{\textnormal{\emph{A. S.}}}\Cendnote{\textnormal{ovaler Absenderkleber}}}\label{T_L03757-1}\pend{}\pstart{}\textcolor{gray}{\textbf{WIEN, XVIII.}}\oindex{XVIII., Währing@\textbf{XVIII., Währing}, \emph{Verwaltungsgebiet}|pw}\pend{}\pstart{}\textcolor{gray}{\textbf{STERNWARTESTR. 71}}\oindex{Wien@\textbf{Wien}!XVIII., Währing@\textbf{XVIII., Währing}!Sternwartestraße 71@\textbf{Sternwartestraße 71}, \emph{Wohngebäude}|pw}\pend{}{\bigskip}\pstart{}Hrn Dr Stefan Zweig,\pend{}\pstart{}Salzburg\oindex{Salzburg@\textbf{Salzburg}, \emph{Verwaltungsgebiet}|pw}. \pend{}\pstart{}Kapuzinerberg 5\oindex{Paschinger Schlössl@\textbf{Paschinger Schlössl}, \emph{Wohngebäude}|pw}.\pend{}{\bigskip}\vspace{1em}
\pstart
           \raggedleft{}{\pb}Wien\oindex{Wien@\textbf{Wien}, \emph{Verwaltungsgebiet}|pw}, 10. 5. 28\pend
           \vspace{0.5em}
\pstart
           lieber Doktor Stefan Zweig, ich kam vor \label{K_L03757-1v}\edtext{einigen Tagen}{\lemma{\textnormal{\emph{einigen Tagen}}}\Cendnote{\textnormal{Er
                  war am 3. 5. 1928
                  zurückgekommen.}}}\label{K_L03757-1} von einer Reise zurück (Athen\oindex{Athen@\textbf{Athen}, \emph{Hauptstadt}|pw}{ }Konstantinopel\oindex{Istanbul@\textbf{Istanbul}, \emph{Land}|pw}{ }Rhodos\oindex{Rhodos@\textbf{Rhodos}, \emph{Hauptstadt}|pw}{ }Venedig\oindex{Venedig@\textbf{Venedig}|pw}) und Ihr neues Buch\pwindex{Zweig, Stefan 28.\,11.\,1881 Wien – 23.\,2.\,1942 Petrópolis@\textsc{Zweig, Stefan} (28.\,11.\,1881 Wien – 23.\,2.\,1942 Petrópolis), \emph{Schriftsteller}!Drei Dichter ihres Lebens. Casanova – Stendhal – Tolstoi@\strich\emph{Drei Dichter ihres Lebens. Casanova – Stendhal – Tolstoi}|pwv}, von dem ich etliche \label{K_L03757-2v}\edtext{Partien}{\lemma{\textnormal{\emph{Partien}}}\Cendnote{\textnormal{Die Texte erschienen bereits in Auszügen in mehren Zeitungen
                  und Zeitschriften vorab. Folgende Aufstellung umfasst jene, die in
                  Periodika erschienen, die Schnitzler
                  regelmäßig rezipierte: Stefan Zweig\pwindex{Zweig, Stefan 28.\,11.\,1881 Wien – 23.\,2.\,1942 Petrópolis@\textsc{Zweig, Stefan} (28.\,11.\,1881 Wien – 23.\,2.\,1942 Petrópolis), \emph{Schriftsteller}|pwk}: \emph{Die Heldenzeit der Abenteurer. (Aus einem größeren Essay
                        über Casanova)}\pwindex{Zweig, Stefan 28.\,11.\,1881 Wien – 23.\,2.\,1942 Petrópolis@\textsc{Zweig, Stefan} (28.\,11.\,1881 Wien – 23.\,2.\,1942 Petrópolis), \emph{Schriftsteller}!Heldenzeit der Abenteurer. (Aus einem größeren Essay über Casanova)@\strich\emph{Die Heldenzeit der Abenteurer. (Aus einem größeren Essay über Casanova)}|pwk}. In: \emph{Neue Freie
                        Presse}\pwindex{Neue Freie Presse@\emph{Neue Freie Presse}|pwk}, Nr. 22.754, 21. 1. 1928, S. 1–3. Stefan Zweig\pwindex{Zweig, Stefan 28.\,11.\,1881 Wien – 23.\,2.\,1942 Petrópolis@\textsc{Zweig, Stefan} (28.\,11.\,1881 Wien – 23.\,2.\,1942 Petrópolis), \emph{Schriftsteller}|pwk}: \emph{Casanova}\pwindex{Zweig, Stefan 28.\,11.\,1881 Wien – 23.\,2.\,1942 Petrópolis@\textsc{Zweig, Stefan} (28.\,11.\,1881 Wien – 23.\,2.\,1942 Petrópolis), \emph{Schriftsteller}!Casanova [Inselschiff]@\strich\emph{Casanova [Inselschiff]}|pwk}. In: \emph{Das
                        Inselschiff}\pwindex{Inselschiff. Eine Zweimonatsschrift@\emph{Das Inselschiff. Eine Zweimonatsschrift}|pwk}, Jg. 9, H. 2, Frühling 1928,
                  S. 120–125. Stefan Zweig\pwindex{Zweig, Stefan 28.\,11.\,1881 Wien – 23.\,2.\,1942 Petrópolis@\textsc{Zweig, Stefan} (28.\,11.\,1881 Wien – 23.\,2.\,1942 Petrópolis), \emph{Schriftsteller}|pwk}: \emph{Lebensbildnis Stendhals}\pwindex{Zweig, Stefan 28.\,11.\,1881 Wien – 23.\,2.\,1942 Petrópolis@\textsc{Zweig, Stefan} (28.\,11.\,1881 Wien – 23.\,2.\,1942 Petrópolis), \emph{Schriftsteller}!Lebensbildnis Stendhals@\strich\emph{Lebensbildnis Stendhals}|pwk}. In: \emph{Neue Freie Presse}\pwindex{Neue Freie Presse@\emph{Neue Freie Presse}|pwk}, Nr. 22.794, 1. 3. 1928,
                     Morgenblatt, S. 1–3; Nr. 22.796, 3. 3. 1928,
                     Morgenblatt, S. 1–3; Nr. 22.802, 9. 3. 1928,
                     Morgenblatt, S. 1–2; Nr. 22.807, 14. 3. 1928,
                     Morgenblatt, S. 1–3. Stefan Zweig\pwindex{Zweig, Stefan 28.\,11.\,1881 Wien – 23.\,2.\,1942 Petrópolis@\textsc{Zweig, Stefan} (28.\,11.\,1881 Wien – 23.\,2.\,1942 Petrópolis), \emph{Schriftsteller}|pwk}: \emph{Bildnis Stendhals}\pwindex{Zweig, Stefan 28.\,11.\,1881 Wien – 23.\,2.\,1942 Petrópolis@\textsc{Zweig, Stefan} (28.\,11.\,1881 Wien – 23.\,2.\,1942 Petrópolis), \emph{Schriftsteller}!Bildnis Stendhals@\strich\emph{Bildnis Stendhals}|pwk}. In: \emph{Berliner Tageblatt}\pwindex{Berliner Tageblatt@\emph{Berliner Tageblatt}|pwk}, Jg. 57, Nr. 145, 25. 3. 1928,
                     Morgen-Ausgabe, S. 2. }}}\label{K_L03757-2} schon gelesen hatte (insbesonders Stendhal\pwindex{Zweig, Stefan 28.\,11.\,1881 Wien – 23.\,2.\,1942 Petrópolis@\textsc{Zweig, Stefan} (28.\,11.\,1881 Wien – 23.\,2.\,1942 Petrópolis), \emph{Schriftsteller}!Stendhal@\strich\emph{Stendhal}|pwv}\pwindex{Stendhal 23.\,1.\,1783 Grenoble – 23.\,3.\,1842 Paris@\textsc{Stendhal} (23.\,1.\,1783 Grenoble – 23.\,3.\,1842 Paris), \emph{Schriftsteller}|pw}) und bin nun daran, es vom ersten bis zum letzten Worte durchzugehn. Schon
               heute will ich Ihnen danken, de{\geminationn} ich bin nicht nur
               angeregt und gefesselt, ich bin auch ergriffen in Geist und Seele, schon lang hab ich
               nichts mit solchem wirklichem Genuſs gelesen und freue mich nicht nur für mich, auch
               für Sie, der in dieser lauwerdenden Welt etwas ganz außerordentliches gegeben, ja
               fast eine neue Form der philosophisch-dichterischen {\pb}Geschichtschreibung geschaffen hat. Zugleich freu ich mich der stetig steigenden
               hohen Anerke{\geminationn}ung (ich wähle aus Bescheidenheit für Sie
               ein mildes Wort) die Ihr Werk findet; wenige haben in den letzten Jahren innerlich
               und äußerlich einen so schönen Weg zurückgelegt. Dank, Grüße, u hoffenlich auf
               Wiedersehen. –\pend
           \pstart Herzlichst Ihr \spacefill\mbox{ArthSchnitzler}\pend{}\selectlanguage{ngerman}\endnumbering\briefempfaengerindex{Zweig, Stefan@\textsc{Zweig, Stefan}!zzzSchnitzler, Arthur@\emph{von Arthur Schnitzler}!1928-05-101@{10. 5. 1928}|)be}\mylabel{L03757h}  \newcommand{\dateiname}{L03757}\newcommand{\titel}{Arthur Schnitzler an Stefan Zweig, 10. 5. 1928}\newcommand{\editorInnen}{Selma Jahnke und Martin Anton Müller}%% latex-leseansicht-abspann.tex
%% Abspann für die Leseansicht.
%% Der Schalter \ifkorrekturansicht ist bereits durch den Vorspann gesetzt.

%% latex-abspann.tex
%% Gemeinsamer Abspann für Korrekturansicht und Leseansicht.
%% Setzt den Schalter \ifkorrekturansicht voraus (gesetzt in den
%% einbindenden Dateien latex-korrekturansicht-abspann.tex bzw.
%% latex-leseansicht-abspann.tex).
%% ---------------------------------------------------------------

\normalsize

% Das esempio-Environment wird nur in der Leseansicht benötigt
\ifkorrekturansicht\else
\newenvironment{esempio}[3]%
{
    \vspace{1.5ex}
    \rlap{\underline{#1}}
    \par
    \setlength{\parindent}{0cm}
    \nopagebreak
    \leftskip=#2cm
    \rightskip=#3cm
}
{
    \par
}
\fi

\doendnotes{C}
\bigskip
\vfill

\clearpage

\footnotesize

\ifkorrekturansicht
  \lohead{\textsc{register}}
\fi

% theindex-Environment neu definieren ohne reledmac
\makeatletter
\renewenvironment{theindex}{%
  \ifkorrekturansicht
    \section*{\indexname}%
  \else
    \subsubsection*{Index der erwähnten Entitäten}%
  \fi
  \setlength{\parindent}{0pt}%
  \setlength{\parskip}{0pt plus 0.3pt}%
  \let\item\@idxitem
}{%
  \ifkorrekturansicht\clearpage\fi
}
\makeatother

\IfFileExists{\jobname-pw.ind}{\input{\jobname-pw.ind}}{}

% Quellenangabe nur in der Leseansicht
\ifkorrekturansicht\else
% Fallback-Definitionen, falls die .tex-Datei \titel etc. nicht gesetzt hat
\providecommand{\titel}{}
\providecommand{\editorInnen}{}
\providecommand{\dateiname}{\jobname}

\vspace{3cm}

\vfill

\footnotesize
\textsc{Quelle}: \titel. Herausgegeben von {\editorInnen}. In: \emph{Arthur Schnitzler: Briefwechsel mit Autorinnen und Autoren}.
 Digitale Edition, https://schnitzler-briefe.acdh.oeaw.ac.at/{\dateiname}.html (Stand \today)
\fi

\end{document}


