%% latex-korrekturansicht-vorspann.tex
%% Vorspann für die Korrekturansicht.
%% Lädt die gemeinsame Datei latex-vorspann.tex mit gesetztem Schalter.

\newif\ifkorrekturansicht
\korrekturansichttrue

\input{../tex-inputs/latex-vorspann}


\section[Arthur Schnitzler an Stefan Zweig, 10. 5. 1928]{L03757 Arthur Schnitzler an Stefan Zweig, 10. 5. 1928}
\nopagebreak\mylabel{L03757v}
\rehead{ }\normalsize\beginnumbering\briefempfaengerindex{Zweig, Stefan@\textsc{Zweig, Stefan}!zzzSchnitzler, Arthur@\emph{von Arthur Schnitzler}!1928-05-101@{10. 5. 1928}|(be}
\toendnotes[C]{\smallbreak\pagebreak[2]}\Standort{Jerusalem, National Library of Israel, ARC. Ms. Var. 305 1 58 Stefan Zweig Collection.}
\physDesc{Postkarte, 1 Blatt, 2 Seiten, 1025 Zeichen
\newline{}Handschrift: schwarze Tinte, lateinische Kurrent
\newline{}Versand: Stempel: »\nobreak{}\oindex{XVIII., Waehring@\textbf{XVIII., Währing}, \emph{A.ADM3}|pwk}Wien 110, 10. \textcolor{gray}{V}. 2\textcolor{gray}{8}, 17\nobreak{}«.  }\toendnotes[C]{\smallbreak}\pstart{}{\pb}\label{T_L03757-1v}\edtext{\textcolor{gray}{\textbf{A. S.}}}{\lemma{\textnormal{\emph{A. S.}}}\Cendnote{\textnormal{ovaler Absenderkleber}}}\label{T_L03757-1}\pend{}\pstart{}\textcolor{gray}{\textbf{WIEN, XVIII.}}\oindex{XVIII., Waehring@\textbf{XVIII., Währing}, \emph{A.ADM3}|pw}\pend{}\pstart{}\textcolor{gray}{\textbf{STERNWARTESTR. 71}}\oindex{Sternwartestrasse 71@\textbf{Sternwartestraße 71}, \emph{Wohngebäude (K.WHS)}|pw}\pend{}{\bigskip}\pstart{}Hrn Dr Stefan Zweig,\pend{}\pstart{}Salzburg\oindex{Salzburg@\textbf{Salzburg}, \emph{A.ADM2}|pw}. \pend{}\pstart{}Kapuzinerberg 5\oindex{Paschinger Schloessl@\textbf{Paschinger Schlössl}, \emph{Wohngebäude (K.WHS)}|pw}.\pend{}{\bigskip}\vspace{1em}
\pstart
           {\pb}Wien\oindex{Wien@\textbf{Wien}, \emph{A.ADM2}|pw}, 10. 5. 28\pend
           \vspace{0.5em}
\pstart
           lieber Doktor Stefan Zweig, ich kam vor \label{K_L03757-1v}\edtext{einigen Tagen}{\lemma{\textnormal{\emph{einigen Tagen}}}\Cendnote{\textnormal{Er
                  war am 3. 5. 1928
                  zurückgekommen.}}}\label{K_L03757-1} von einer Reise zurück (Athen\oindex{Athen@\textbf{Athen}, \emph{P.PPLC}|pw}{ }Konstantinopel\oindex{Istanbul@\textbf{Istanbul}, \emph{A.ADM1}|pw}{ }Rhodos\oindex{Rhodos@\textbf{Rhodos}, \emph{P.PPLA2}|pw}Venedig\oindex{Venedig@\textbf{Venedig}, \emph{P.PPLA}|pw}) und Ihr neues Buch\pwindex{Drei Dichter ihres Lebens. Casanova – Stendhal – Tolstoi@\emph{Drei Dichter ihres Lebens. Casanova – Stendhal – Tolstoi}|pwv}, von dem ich etliche \label{K_L03757-2v}\edtext{Partien}{\lemma{\textnormal{\emph{Partien}}}\Cendnote{\textnormal{Die Texte erschienen bereits in Auszügen in mehren Zeitungen
                  und Zeitschriften vorab. Folgende Aufstellung dürfte jene umfassen, die in
                  Periodika erschienen, die Schnitzler
                  regelmäßig rezipierte: Stefan Zweig\pwindex{Zweig, Stefan 28.11.1881 – 23.02.1942@\textsc{Zweig, Stefan} (28.11.1881 – 23.02.1942), \emph{Schriftsteller/Schriftstellerin}|pwk}:
                        \emph{Die Heldenzeit der Abenteurer. (Aus einem
                        größeren Essay über Casanova)}\pwindex{Heldenzeit der Abenteurer. (Aus einem groesseren Essay ueber Casanova)@\emph{Die Heldenzeit der Abenteurer. (Aus einem größeren Essay über Casanova)}|pwk}. In: \emph{Neue
                        Freie Presse}\pwindex{Neue Freie Presse@\emph{Neue Freie Presse}|pwk}, Nr. 22.754, 21. 1. 1928,
                     S. 1–3. Stefan Zweig\pwindex{Zweig, Stefan 28.11.1881 – 23.02.1942@\textsc{Zweig, Stefan} (28.11.1881 – 23.02.1942), \emph{Schriftsteller/Schriftstellerin}|pwk}:
                        \emph{Casanova}\pwindex{Casanova [Inselschiff]@\emph{Casanova [Inselschiff]}|pwk}. In: Das Inselschiff\textcolor{red}{\textsuperscript{\textbf{KEY}}}, Jg. 9, H. 2, Frühling 1928,
                     S. 120–125. Stefan
                        Zweig\pwindex{Zweig, Stefan 28.11.1881 – 23.02.1942@\textsc{Zweig, Stefan} (28.11.1881 – 23.02.1942), \emph{Schriftsteller/Schriftstellerin}|pwk}: \emph{Lebensbildnis Stendhals}\pwindex{Lebensbildnis Stendhals@\emph{Lebensbildnis Stendhals}|pwk}.
                     In: \emph{Neue Freie Presse}\pwindex{Neue Freie Presse@\emph{Neue Freie Presse}|pwk}, Nr. 22.794,
                        1. 3. 1928, Morgenblatt, S. 1–3;
                     Nr. 22.796, 3. 3. 1928, Morgenblatt,
                     S. 1–3; Nr. 22.802, 9. 3. 1928,
                     Morgenblatt, S. 1–2; Nr. 22.807,
                        14. 3. 1928, Morgenblatt, S. 1–3. Stefan Zweig\pwindex{Zweig, Stefan 28.11.1881 – 23.02.1942@\textsc{Zweig, Stefan} (28.11.1881 – 23.02.1942), \emph{Schriftsteller/Schriftstellerin}|pwk}: \emph{Bildnis Stendhals}\pwindex{Bildnis Stendhals@\emph{Bildnis Stendhals}|pwk}. In: \emph{Berliner Tageblatt}\pwindex{Berliner Tageblatt@\emph{Berliner Tageblatt}|pwk}, Jg. 57, Nr. 145, 25. 3. 1928,
                     Morgen-Ausgabe, S. 2. }}}\label{K_L03757-2} schon gelesen hatte, (insbesonders Stendhal\pwindex{Stendhal@\emph{Stendhal}|pwv}\pwindex{Stendhal 1783-01-23 – 1842-03-23@\textsc{Stendhal} (1783-01-23 – 1842-03-23), \emph{Schriftsteller/Schriftstellerin}|pw}) und bin nun daran, es vom ersten bis zum letzten Worte durchzugehn. Schon
               heute will ich Ihnen danken, de{\geminationn} ich bin nicht nur
               angeregt und gefesselt, ich bin auch ergriffen in Geist und Seele, schon lang hab ich
               nichts mit solchem wirklichem Genuſs gelesen und freue mich nicht nur für mich, auch
               für Sie, der in dieser lauwerdenden Welt etwas ganz außerordentliches gegeben, ja
               fast eine neue Form der philosophisch-dichterischen {\pb}Geschichtschreibung geschaffen hat. Zugleich freu ich mich der stetig steigenden
               hohen Anerke{\geminationn}ung (ich wähle aus Bescheidenheit für Sie
               ein lindes Wort) die Ihr Werk findet; wenige haben in den letzten Jahren innerlich
               und äußerlich einen so schönen Weg zurückgelegt. Dank, Grüße, u hoffenlich auf
               Wiedersehen,\pend
           \pstart Herzlichst Ihr \spacefill\mbox{ArthSchnitzler}\pend{}\selectlanguage{ngerman}\endnumbering\briefempfaengerindex{Zweig, Stefan@\textsc{Zweig, Stefan}!zzzSchnitzler, Arthur@\emph{von Arthur Schnitzler}!1928-05-101@{10. 5. 1928}|)be}\mylabel{L03757h}
\begin{anhang}
\end{anhang}\normalsize

\doendnotes{C}
\bigskip
\vfill

\clearpage

\footnotesize

\lohead{\textsc{register}}

% Definiere theindex-Environment komplett neu ohne reledmac
\makeatletter
\renewenvironment{theindex}{%
  \section*{\indexname}%
  \setlength{\parindent}{0pt}%
  \setlength{\parskip}{0pt plus 0.3pt}%
  \let\item\@idxitem
}{%
  \clearpage
}
\makeatother

\IfFileExists{\jobname-pw.ind}{\input{\jobname-pw.ind}}{}

\end{document}

      