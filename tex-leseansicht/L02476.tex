%% latex-leseansicht-vorspann.tex
%% Vorspann für die Leseansicht.
%% Lädt die gemeinsame Datei latex-vorspann.tex mit nicht gesetztem Schalter.

\newif\ifkorrekturansicht
\korrekturansichtfalse

\input{../tex-inputs/latex-vorspann}


\section[Arthur Schnitzler an Stefan Großmann, 31. 5. 1926]{L02476 Arthur Schnitzler an Stefan Großmann, 31. 5. 1926}
\nopagebreak\mylabel{L02476v}
\rehead{ }\normalsize\beginnumbering\briefempfaengerindex{Großmann, Stefan@\textsc{Großmann, Stefan}!zzzSchnitzler, Arthur@\emph{von Arthur Schnitzler}!1926-05-311@{31. 5. 1926}|(be}
\toendnotes[C]{\smallbreak\pagebreak[2]}
\correspDesc{Versand  durch Arthur Schnitzler am 31. 5. 1926 in Wien
\newline{}Erhalt  durch Stefan Großmann im Zeitraum [1. 6. 1926
                  – 5. 6. 1926?] in Berlin}\toendnotes[C]{\smallbreak}
\Standort{DLA, A:Schnitzler, HS.NZ85.1.896.}
\physDesc{Brief, Durchschlag, 1 Blatt, 1 Seite, 2193 Zeichen
\newline{}Schreibmaschine
\newline{}Handschrift: roter Buntstift, lateinische Kurrent (\noindent{}»Grossmann« und »Berlin\oindex{Berlin@\textbf{Berlin}, \emph{Hauptstadt}|pw}« sowie »Urteiln\textcolor{gray}{ahme}«. Zahlreiche Unterstreichungen)}\toendnotes[C]{\smallbreak}
\pstart
           \raggedleft{}{\pb}31. 5. 1926.\pend
           
\pstart{}Verehrter Herr Grossmann.\pend\vspace{0.5em}
\pstart
           Sie haben meine Zustimmung zu dem Nachdruck\pwindex{Bemerkungen@\emph{Bemerkungen}|pwv} der in der Neuen Freien
                  Presse\pwindex{Neue Freie Presse@\emph{Neue Freie Presse}|pw} zu Pfingsten veröffentlichten »Bemerkungen\pwindex{Schnitzler, Arthur 15.\,5.\,1862 Wien – 21.\,10.\,1931 ebd.@\textsc{Schnitzler, Arthur} (15.\,5.\,1862 Wien – 21.\,10.\,1931 ebd.), \emph{Schriftsteller, Mediziner}!Bemerkungen. (Aus dem noch unveröffentlichten „Buch der Sprüche und Bedenken“.)@\strich\emph{Bemerkungen. (Aus dem noch unveröffentlichten „Buch der Sprüche und Bedenken“.)}|pw}« nicht abgewartet, doch da ich in jedem Fall
               bereit gewesen wäre Ihnen diese Zustimmung zu erteilen, so habe ich auch nachträglich
               nichts einzuwenden. Höchst ärgerlich aber ist mir, dass das vorletzte Aphorisma nur
               zur Hälfte abgedruckt und dadurch zu einer pretentiösen Plattheit geworden ist.
               Offenbar ist die 3. Spalte des Originaldruckes\pwindex{Schnitzler, Arthur 15.\,5.\,1862 Wien – 21.\,10.\,1931 ebd.@\textsc{Schnitzler, Arthur} (15.\,5.\,1862 Wien – 21.\,10.\,1931 ebd.), \emph{Schriftsteller, Mediziner}!Bemerkungen. (Aus dem noch unveröffentlichten „Buch der Sprüche und Bedenken“.)@\strich\emph{Bemerkungen. (Aus dem noch unveröffentlichten „Buch der Sprüche und Bedenken“.)}|pwv} der Neuen Freien Presse\pwindex{Neue Freie Presse@\emph{Neue Freie Presse}|pw}
               dem Setzer in Verlust geraten und er hat meine »Bemerkung« aus eigener
               Machtvollkommenheit durch Hinzufügung eines Wortes zu Ende gedichtet. Sie lautet
               daher im »Tagebuch\pwindex{Tage-Buch@\emph{Das Tage-Buch}|pw}«: »Ob ein Mensch dich
               bestohlen, betrogen, verleumdet habe – es könnte immer noch die Möglichkeit einer
               Versöhnung, ja selbst eines späteren reinen Verhältnisses zwischen dir und ihm
               bestehen. Ja, wenn es sich praktisch durchführen lässt –« (!!!)\pend
           
\pstart
           In Wirklichkaut lautet die »Bemerkung{[}«{]} wie folgt:\pend
           
\pstart
           »Ob ein Mensch dich betrogen, bestohlen, \strikeout{verlejde}
               verleumdet habe, es könnte immer noch die Möglichkeit einer Versöhnung, ja selbst
               eines späteren reinen Verhältnisses zwischen dir und ihm bestehen. Ja, wenn es sich
               praktisch durchführen liesse –\strikeout{ä}: selbst mit deinem
               Mörder könntest du dich nach geschehener Tat vielleicht trefflich verstehen, am
               ehesten vielleicht mit ihm! Nur {\pb}zu einem Menschen, der
               nicht weiss, was er dir getan hat, führt, selbst wenn du dieses Tun persönlich längst
               verschmerztest, in aller Ewigkeit kein Weg zurück.«\pend
           
\pstart
           (Es folgt dann noch ein Aphorisma, das dem Setzer selbstverständlich völlig entgehen
               musste, da es auf der 3. Spalte stand.)\pend
           
\pstart
           Ich bitte Sie sehr das Versehen \label{K_L02476-1v}\edtext{richtig zu stellen}{\lemma{\textnormal{\emph{richtig zu stellen}}}\Cendnote{\textnormal{Die Richtigstellung
                  erschien am 5. 6. 1926 (Jg. 7, H. 23, S. 915)
                  einschließlich des zusätzlichen von Schnitzler vorgeschlagenen Aphorismus.}}}\label{K_L02476-1} und meine »Bemerkung\pwindex{Schnitzler, Arthur 15.\,5.\,1862 Wien – 21.\,10.\,1931 ebd.@\textsc{Schnitzler, Arthur} (15.\,5.\,1862 Wien – 21.\,10.\,1931 ebd.), \emph{Schriftsteller, Mediziner}!Bemerkungen. (Aus dem noch unveröffentlichten „Buch der Sprüche und Bedenken“.)@\strich\emph{Bemerkungen. (Aus dem noch unveröffentlichten „Buch der Sprüche und Bedenken“.)}|pw}« in Gänze dem Original gemäss abdrucken zu
               wollen\pend
           
\pstart
           Den Empfang des Nachdruckshonorars im Betrage von S. 85.– bestätige ich mit bestem
               Dank und bin mit den verbindlichsten Grüssen\pend
           \pstart Ihr sehr ergebener\pend{}{\vspace{1\baselineskip}}
\pstart
           Das letzte Aphorisma, wenn Sie es vielleicht noch nachträglich drucken wollen,
                  lautet:\pend
           
\pstart
           »Es ist schon oft genug vorgekommen, dass ein Bösewicht aus Klugheit etwas Gutes,
                  aber noch nie, dass ein Dummkopf aus Güte etwas Kluges getan hat.«\pend
           {\vspace{1\baselineskip}}
\pstart
           Herrn Stefan Grossmann,{\\}Herausgeber des »Tagebuch\orgindex{Tage-Buch@Das Tage-Buch|pw}«,{\\}Berlin\oindex{Berlin@\textbf{Berlin}, \emph{Hauptstadt}|pw}.\pend
           \selectlanguage{ngerman}\endnumbering\briefempfaengerindex{Großmann, Stefan@\textsc{Großmann, Stefan}!zzzSchnitzler, Arthur@\emph{von Arthur Schnitzler}!1926-05-311@{31. 5. 1926}|)be}\mylabel{L02476h}  \newcommand{\dateiname}{L02476}\newcommand{\titel}{Arthur Schnitzler an Stefan Großmann, 31. 5. 1926}\newcommand{\editorInnen}{Martin Anton Müller und Gerd-Hermann Susen}%% latex-leseansicht-abspann.tex
%% Abspann für die Leseansicht.
%% Der Schalter \ifkorrekturansicht ist bereits durch den Vorspann gesetzt.

%% latex-abspann.tex
%% Gemeinsamer Abspann für Korrekturansicht und Leseansicht.
%% Setzt den Schalter \ifkorrekturansicht voraus (gesetzt in den
%% einbindenden Dateien latex-korrekturansicht-abspann.tex bzw.
%% latex-leseansicht-abspann.tex).
%% ---------------------------------------------------------------

\normalsize

% Das esempio-Environment wird nur in der Leseansicht benötigt
\ifkorrekturansicht\else
\newenvironment{esempio}[3]%
{
    \vspace{1.5ex}
    \rlap{\underline{#1}}
    \par
    \setlength{\parindent}{0cm}
    \nopagebreak
    \leftskip=#2cm
    \rightskip=#3cm
}
{
    \par
}
\fi

\doendnotes{C}
\bigskip
\vfill

\clearpage

\footnotesize

\ifkorrekturansicht
  \lohead{\textsc{register}}
\fi

% theindex-Environment neu definieren ohne reledmac
\makeatletter
\renewenvironment{theindex}{%
  \ifkorrekturansicht
    \section*{\indexname}%
  \else
    \subsubsection*{Index der erwähnten Entitäten}%
  \fi
  \setlength{\parindent}{0pt}%
  \setlength{\parskip}{0pt plus 0.3pt}%
  \let\item\@idxitem
}{%
  \ifkorrekturansicht\clearpage\fi
}
\makeatother

\IfFileExists{\jobname-pw.ind}{\input{\jobname-pw.ind}}{}

% Quellenangabe nur in der Leseansicht
\ifkorrekturansicht\else
% Fallback-Definitionen, falls die .tex-Datei \titel etc. nicht gesetzt hat
\providecommand{\titel}{}
\providecommand{\editorInnen}{}
\providecommand{\dateiname}{\jobname}

\vspace{3cm}

\vfill

\footnotesize
\textsc{Quelle}: \titel. Herausgegeben von {\editorInnen}. In: \emph{Arthur Schnitzler: Briefwechsel mit Autorinnen und Autoren}.
 Digitale Edition, https://schnitzler-briefe.acdh.oeaw.ac.at/{\dateiname}.html (Stand \today)
\fi

\end{document}


