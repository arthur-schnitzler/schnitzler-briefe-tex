%% latex-korrekturansicht-vorspann.tex
%% Vorspann für die Korrekturansicht.
%% Lädt die gemeinsame Datei latex-vorspann.tex mit gesetztem Schalter.

\newif\ifkorrekturansicht
\korrekturansichttrue

\input{../tex-inputs/latex-vorspann}


\section[ Paul Goldmann an Arthur Schnitzler, {[}16./17.?{]} 9. {[}1897{]}]{L02824 Paul Goldmann an Arthur Schnitzler, {[}16./17.?{]} 9. {[}1897{]}}
\nopagebreak\mylabel{L02824v}
\rehead{ }\normalsize\beginnumbering\briefempfaengerindex{Schnitzler, Arthur@\textsc{Schnitzler, Arthur}!zzzGoldmann, Paul@\emph{von Paul Goldmann}!1897-09-171@{{[}16./17.?{]} 9. {[}1897{]}}|(be}
\toendnotes[C]{\smallbreak\pagebreak[2]}\Standort{DLA, A:Schnitzler, HS.NZ85.1.3167.}
\physDesc{Brief, 1 Blatt, 3 Seiten, 1098 Zeichen
\newline{}Handschrift: schwarze Tinte, deutsche Kurrent
\newline{}Schnitzler: 1) mit Bleistift das Jahr »97« vermerkt  2) mit rotem Buntstift fünf Unterstreichungen}\toendnotes[C]{\smallbreak}
\pstart
           \raggedleft{}{\pb}\textsc{Frankfurt\oindex{Frankfurt am Main@\textbf{Frankfurt am Main}, \emph{P.PPLA3}|pw}}{ }\label{K_L02824-1v}\edtext{17. September}{\lemma{\textnormal{\emph{17. September}}}\Cendnote{\textnormal{Dieser und der folgende Brief (Paul Goldmann an Arthur Schnitzler, [17./18.?] 9. 1897) sind auf den gleichen Tag
                     datiert, im zweiten Brief wird aber auf den vorliegenden als »geſtrigen
                        Brief« verwiesen, wodurch entweder der vorliegende auf den 16. 9. 1897 oder andernfalls der folgende auf den
                        18. 9. 1897 zu datieren
                  wäre.}}}\label{K_L02824-1}.\pend
           
\pstart\center{}Mein lieber Freund,\pend\vspace{0.5em}
\pstart
           Übermorgen gehe ich nach \textsc{Paris\oindex{Paris@\textbf{Paris}, \emph{P.PPLC}|pw}} zurück. Ich gehe mit ſchwerem Herzen. Der Arbeit, die mich dort erwartet, fühle
               ich mich kaum mehr gewachſen; und niederdrückend iſt das Bewußtſein, daß alle die
               harte Mühe nicht vowärts hilft und daß das einzige Reſultat meiner Thätigkeit iſt,
               mich von Jahr zu Jahr fortzufriſten. Und darüber geht das Leben \strikeout{ſo} hin. Es war hier wieder die Rede davon, mich nach
                  \textsc{Berlin\oindex{Berlin@\textbf{Berlin}, \emph{P.PPLC}|pw}} zu ſchicken, aber Gott weiß, ob etwas daraus wird.\pend
           
\pstart
           Bitte, ſchreibe mir ſofort nach \textsc{Paris\oindex{Paris@\textbf{Paris}, \emph{P.PPLC}|pw}}, wie es mit \textsc{Richard Klein\pwindex{Klein, Richard *~07.08.1873@\textsc{Klein, Richard} (*~07.08.1873), \emph{Maler/Malerin}|pw}} ſteht? Was weiß man {\pb}über den Grund des
                  \label{K_L02824-2v}\edtext{Selbſtmord-Verſuches}{\lemma{\textnormal{\emph{Selbſtmord-Verſuches}}}\Cendnote{\textnormal{Am 25. 8. 1897 hatte sich der
                  seit Februar 1897 als Maler in Paris\oindex{Paris@\textbf{Paris}, \emph{P.PPLC}|pwk} lebende Richard Klein\pwindex{Klein, Richard *~07.08.1873@\textsc{Klein, Richard} (*~07.08.1873), \emph{Maler/Malerin}|pwk} in
                  seinem Atelier eine Pistolenkugel in den Kopf geschossen. Obzwar die Zeitungen bereits seinen
                  Tod gemeldet hatten, überlebte er und wurde Ende September zur weiteren
                  Genesung nach Wien\oindex{Wien@\textbf{Wien}, \emph{A.ADM2}|pwk} übersiedelt.}}}\label{K_L02824-2}? Wird er
               mit dem Leben davon kommen?\pend
           
\pstart
           Bitte, frage auch \label{K_L02824-3v}\edtext{\textsc{Arthur Klein\pwindex{Klein, Arthur 27.11.1868 – 28.07.1943@\textsc{Klein, Arthur} (27.11.1868 – 28.07.1943)|pw}}}{\lemma{\textnormal{\emph{Arthur Klein}}}\Cendnote{\textnormal{Richard Kleins\pwindex{Klein, Richard *~07.08.1873@\textsc{Klein, Richard} (*~07.08.1873), \emph{Maler/Malerin}|pwk}{ }Bruder\pwindex{Klein, Arthur 27.11.1868 – 28.07.1943@\textsc{Klein, Arthur} (27.11.1868 – 28.07.1943)|pwkv}}}}\label{K_L02824-3}, ob ich nicht irgendwie in \textsc{Paris\oindex{Paris@\textbf{Paris}, \emph{P.PPLC}|pw}} mich des armen Burſchen\pwindex{Klein, Richard *~07.08.1873@\textsc{Klein, Richard} (*~07.08.1873), \emph{Maler/Malerin}|pwv} annehmen kann (wenn \strikeout{\textcolor{gray}{×}} er noch dort iſt). Ich höre, daß \textsc{Frischauer\pwindex{Frischauer, Berthold 1851-09-09 – 1924-02-04@\textsc{Frischauer, Berthold} (1851-09-09 – 1924-02-04), \emph{Journalist/Journalistin}|pw}} in \textsc{Paris\oindex{Paris@\textbf{Paris}, \emph{P.PPLC}|pw}} mit dem Vater \textsc{Klein}\pwindex{Klein, Johann 1838-10-15 – 1927-05-18@\textsc{Klein, Johann} (1838-10-15 – 1927-05-18), \emph{Großindustrieller/Großindustrielle, Bankier/Bankierin}|pwv} verkehrt hat. Er könnte da vielleicht gegen mich geſtänkert \strikeout{hab} und den unglücklichen \label{K_L02824-4v}\edtext{Zwiſchenfall}{\lemma{\textnormal{\emph{Zwiſchenfall}}}\Cendnote{\textnormal{Siehe Paul Goldmann an Arthur Schnitzler, [25. – 28.? 2. 1897].
               }}}\label{K_L02824-4}, in den ich verwickelt war, lügenhaft {\pb}dargeſtellt haben. Suche doch der Sache auf den Grund zu gehen u., im Nothfalle,
               den Thatbeſtand richtigzuſtellen.\pend
           
\pstart
           Ich begrüße Dich von Herzen{\\[\baselineskip]}Dein {\\[\baselineskip]}\spacefill\mbox{Paul Goldm}\pend
           \leftskip=0em{}\selectlanguage{ngerman}\endnumbering\briefempfaengerindex{Schnitzler, Arthur@\textsc{Schnitzler, Arthur}!zzzGoldmann, Paul@\emph{von Paul Goldmann}!1897-09-161@{{[}16./17.?{]} 9. {[}1897{]}}|)be}\mylabel{L02824h}  \normalsize

\doendnotes{C}
\bigskip
\vfill

\clearpage

\footnotesize

\lohead{\textsc{register}}

% Definiere theindex-Environment komplett neu ohne reledmac
\makeatletter
\renewenvironment{theindex}{%
  \section*{\indexname}%
  \setlength{\parindent}{0pt}%
  \setlength{\parskip}{0pt plus 0.3pt}%
  \let\item\@idxitem
}{%
  \clearpage
}
\makeatother

\IfFileExists{\jobname-pw.ind}{\input{\jobname-pw.ind}}{}

\end{document}

      