%% latex-leseansicht-vorspann.tex
%% Vorspann für die Leseansicht.
%% Lädt die gemeinsame Datei latex-vorspann.tex mit nicht gesetztem Schalter.

\newif\ifkorrekturansicht
\korrekturansichtfalse

\input{../tex-inputs/latex-vorspann}


\section[ Paul Goldmann an Arthur Schnitzler, {[}16./17.?{]} 9. {[}1897{]}]{L02824 Paul Goldmann an Arthur Schnitzler,  [16./17.?] 9. [1897]}
\nopagebreak\mylabel{L02824v}
\rehead{ }\normalsize\beginnumbering\briefempfaengerindex{Schnitzler, Arthur@\textsc{Schnitzler, Arthur}!zzzGoldmann, Paul@\emph{von Paul Goldmann}!1897-09-171@{{[}16./17.?{]}}|(be}
\toendnotes[C]{\smallbreak\pagebreak[2]}
\correspDesc{Versand  durch Paul Goldmann im Zeitraum [16./17.?] 9. [1897] in Frankfurt am Main
\newline{}Erhalt  durch Arthur Schnitzler im Zeitraum [17. 9. 1897
                  – 22. 9. 1897?] in Wien}\toendnotes[C]{\smallbreak}
\Standort{DLA, A:Schnitzler, HS.NZ85.1.3167.}
\physDesc{Brief, 1 Blatt, 3 Seiten, 1098 Zeichen
\newline{}Handschrift: schwarze Tinte, deutsche Kurrent
\newline{}Schnitzler: 1) mit Bleistift das Jahr »97« vermerkt  2) mit rotem Buntstift fünf Unterstreichungen}\toendnotes[C]{\smallbreak}
\pstart
           \raggedleft{}{\pb}\textsc{Frankfurt\oindex{Frankfurt am Main@\textbf{Frankfurt am Main}, \emph{Hauptstadt}|pw}}{ }\label{K_L02824-1v}\edtext{17. September}{\lemma{\textnormal{\emph{17. September}}}\Cendnote{\textnormal{Dieser und der folgende Brief (XXXX Auszeichnungsfehler: Dokument L02825 nicht gefunden) sind auf den gleichen Tag
                     datiert, im zweiten Brief wird aber auf den vorliegenden als »geſtrigen
                        Brief« verwiesen, wodurch entweder der vorliegende auf den 16. 9. 1897 oder andernfalls der folgende auf den
                        18. 9. 1897 zu datieren
                  wäre.}}}\label{K_L02824-1}.\pend
           
\pstart\center{}Mein lieber Freund,\pend\vspace{0.5em}
\pstart
           Übermorgen gehe ich nach \textsc{Paris\oindex{Paris@\textbf{Paris}, \emph{Hauptstadt}|pw}} zurück. Ich gehe mit{ }ſchwerem Herzen. Der Arbeit, die mich dort erwartet, fühle
               ich mich kaum mehr gewachſen; und niederdrückend iſt das Bewußtſein, daß alle die
               harte Mühe nicht vowärts hilft und daß das einzige Reſultat meiner Thätigkeit iſt,
               mich von Jahr zu Jahr fortzufriſten. Und darüber geht das Leben \strikeout{ſo} hin. Es war hier wieder die Rede davon, mich nach
                  \textsc{Berlin\oindex{Berlin@\textbf{Berlin}, \emph{Hauptstadt}|pw}} zu{ }ſchicken, aber Gott weiß, ob etwas daraus wird.\pend
           
\pstart
           Bitte,{ }ſchreibe mir{ }ſofort nach \textsc{Paris\oindex{Paris@\textbf{Paris}, \emph{Hauptstadt}|pw}}, wie es mit \textsc{Richard Klein\pwindex{Klein, Richard *~7.\,8.\,1873 Baden bei Wien@\textsc{Klein, Richard} (*~7.\,8.\,1873 Baden bei Wien), \emph{Maler}|pw}}{ }ſteht? Was weiß man {\pb}über den Grund des
                  \label{K_L02824-2v}\edtext{Selbſtmord-Verſuches}{\lemma{\textnormal{\emph{Selbstmord-Versuches}}}\Cendnote{\textnormal{Am 25. 8. 1897 hatte sich der
                  seit Februar 1897 als Maler in Paris\oindex{Paris@\textbf{Paris}, \emph{Hauptstadt}|pwk} lebende Richard Klein\pwindex{Klein, Richard *~7.\,8.\,1873 Baden bei Wien@\textsc{Klein, Richard} (*~7.\,8.\,1873 Baden bei Wien), \emph{Maler}|pwk} in
                  seinem Atelier eine Pistolenkugel in den Kopf geschossen. Obzwar die Zeitungen bereits seinen
                  Tod gemeldet hatten, überlebte er und wurde Ende September zur weiteren
                  Genesung nach Wien\oindex{Wien@\textbf{Wien}, \emph{Verwaltungsgebiet}|pwk} übersiedelt.}}}\label{K_L02824-2}? Wird er
               mit dem Leben davon kommen?\pend
           
\pstart
           Bitte, frage auch \label{K_L02824-3v}\edtext{\textsc{Arthur Klein\pwindex{Klein, Arthur 27.\,11.\,1868 Wien – 28.\,7.\,1943@\textsc{Klein, Arthur} (27.\,11.\,1868 Wien – 28.\,7.\,1943)|pw}}}{\lemma{\textnormal{\emph{Arthur Klein}}}\Cendnote{\textnormal{Richard Kleins\pwindex{Klein, Richard *~7.\,8.\,1873 Baden bei Wien@\textsc{Klein, Richard} (*~7.\,8.\,1873 Baden bei Wien), \emph{Maler}|pwk}{ }Bruder\pwindex{Klein, Arthur 27.\,11.\,1868 Wien – 28.\,7.\,1943@\textsc{Klein, Arthur} (27.\,11.\,1868 Wien – 28.\,7.\,1943)|pwkv}}}}\label{K_L02824-3}, ob ich nicht irgendwie in \textsc{Paris\oindex{Paris@\textbf{Paris}, \emph{Hauptstadt}|pw}} mich des armen Burſchen\pwindex{Klein, Richard *~7.\,8.\,1873 Baden bei Wien@\textsc{Klein, Richard} (*~7.\,8.\,1873 Baden bei Wien), \emph{Maler}|pwv} annehmen kann (wenn \strikeout{\textcolor{gray}{×}} er noch dort iſt). Ich höre, daß \textsc{Frischauer\pwindex{Frischauer, Berthold 9.\,9.\,1851 Brünn – 4.\,2.\,1924 Wien@\textsc{Frischauer, Berthold} (9.\,9.\,1851 Brünn – 4.\,2.\,1924 Wien), \emph{Journalist}|pw}} in \textsc{Paris\oindex{Paris@\textbf{Paris}, \emph{Hauptstadt}|pw}} mit dem Vater \textsc{Klein}\pwindex{Klein, Johann 15.\,10.\,1838 Kaposvár – 18.\,5.\,1927 Wien@\textsc{Klein, Johann} (15.\,10.\,1838 Kaposvár – 18.\,5.\,1927 Wien), \emph{Großindustrieller, Bankier}|pwv} verkehrt hat. Er könnte da vielleicht gegen mich geſtänkert \strikeout{hab} und den unglücklichen \label{K_L02824-4v}\edtext{Zwiſchenfall}{\lemma{\textnormal{\emph{Zwischenfall}}}\Cendnote{\textnormal{Siehe XXXX Auszeichnungsfehler: Dokument L02638 nicht gefunden.
               }}}\label{K_L02824-4}, in den ich verwickelt war, lügenhaft {\pb}dargeſtellt haben. Suche doch der Sache auf den Grund zu gehen u., im Nothfalle,
               den Thatbeſtand richtigzuſtellen.\pend
           
\pstart
           Ich begrüße Dich von Herzen{\\[\baselineskip]}Dein {\\[\baselineskip]}\spacefill\mbox{Paul Goldm}\pend
           \leftskip=0em{}\selectlanguage{ngerman}\endnumbering\briefempfaengerindex{Schnitzler, Arthur@\textsc{Schnitzler, Arthur}!zzzGoldmann, Paul@\emph{von Paul Goldmann}!1897-09-161@{{[}16./17.?{]}}|)be}\mylabel{L02824h}  \newcommand{\dateiname}{L02824}\newcommand{\titel}{Paul Goldmann an Arthur Schnitzler, [16./17.?] 9. [1897]}\newcommand{\editorInnen}{Martin Anton Müller und Laura Untner}%% latex-leseansicht-abspann.tex
%% Abspann für die Leseansicht.
%% Der Schalter \ifkorrekturansicht ist bereits durch den Vorspann gesetzt.

%% latex-abspann.tex
%% Gemeinsamer Abspann für Korrekturansicht und Leseansicht.
%% Setzt den Schalter \ifkorrekturansicht voraus (gesetzt in den
%% einbindenden Dateien latex-korrekturansicht-abspann.tex bzw.
%% latex-leseansicht-abspann.tex).
%% ---------------------------------------------------------------

\normalsize

% Das esempio-Environment wird nur in der Leseansicht benötigt
\ifkorrekturansicht\else
\newenvironment{esempio}[3]%
{
    \vspace{1.5ex}
    \rlap{\underline{#1}}
    \par
    \setlength{\parindent}{0cm}
    \nopagebreak
    \leftskip=#2cm
    \rightskip=#3cm
}
{
    \par
}
\fi

\doendnotes{C}
\bigskip
\vfill

\clearpage

\footnotesize

\ifkorrekturansicht
  \lohead{\textsc{register}}
\fi

% theindex-Environment neu definieren ohne reledmac
\makeatletter
\renewenvironment{theindex}{%
  \ifkorrekturansicht
    \section*{\indexname}%
  \else
    \subsubsection*{Index der erwähnten Entitäten}%
  \fi
  \setlength{\parindent}{0pt}%
  \setlength{\parskip}{0pt plus 0.3pt}%
  \let\item\@idxitem
}{%
  \ifkorrekturansicht\clearpage\fi
}
\makeatother

\IfFileExists{\jobname-pw.ind}{\input{\jobname-pw.ind}}{}

% Quellenangabe nur in der Leseansicht
\ifkorrekturansicht\else
% Fallback-Definitionen, falls die .tex-Datei \titel etc. nicht gesetzt hat
\providecommand{\titel}{}
\providecommand{\editorInnen}{}
\providecommand{\dateiname}{\jobname}

\vspace{3cm}

\vfill

\footnotesize
\textsc{Quelle}: \titel. Herausgegeben von {\editorInnen}. In: \emph{Arthur Schnitzler: Briefwechsel mit Autorinnen und Autoren}.
 Digitale Edition, https://schnitzler-briefe.acdh.oeaw.ac.at/{\dateiname}.html (Stand \today)
\fi

\end{document}


