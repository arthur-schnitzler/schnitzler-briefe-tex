%% latex-korrekturansicht-vorspann.tex
%% Vorspann für die Korrekturansicht.
%% Lädt die gemeinsame Datei latex-vorspann.tex mit gesetztem Schalter.

\newif\ifkorrekturansicht
\korrekturansichttrue

\input{../tex-inputs/latex-vorspann}


\section[Hugo von Hofmannsthal an Arthur Schnitzler, {[}nach dem 11.? 1. 1899{]}]{L00879 Hugo von Hofmannsthal an Arthur Schnitzler, {[}nach dem
               11.? 1. 1899{]}}
\nopagebreak\mylabel{L00879v}
\rehead{ }\normalsize\beginnumbering\briefempfaengerindex{Schnitzler, Arthur@\textsc{Schnitzler, Arthur}!zzzHofmannsthal, Hugo von@\emph{von Hugo von Hofmannsthal}!1899-01-111@{{[}nach dem 11.? 1. 1899{]}}|(be}
\toendnotes[C]{\smallbreak\pagebreak[2]}\Standort{CUL, Schnitzler, B 43.}
\physDesc{Brief, 1 Blatt, 1 Seite, 52 Zeichen
\newline{}Handschrift: Bleistift, lateinische Kurrent
\newline{}Schnitzler: mit Bleistift datiert: »Jänner 99« 
\newline{}Ordnung: 1) mit Bleistift von unbekannter Hand nummeriert: »\strikeout{133}«  2) mit Bleistift von unbekannter Hand nummeriert: »132«}
\buchAbdrucke{\weitereDrucke{Hugo von Hofmannsthal, Arthur Schnitzler: \emph{Briefwechsel}. Frankfurt am Main: \emph{S. Fischer} 1964, S. 117.} }\toendnotes[C]{\smallbreak}
\pstart
           \noindent{}\centering{}{\pb}D\textsuperscript{r}
               Arthur Schnitzler\pend
           
\pstart
           \centering{}Frankgasse 1\oindex{Frankgasse 1@\textbf{Frankgasse 1}, \emph{Wohngebäude (K.WHS)}|pw}\pend
           
\pstart
           \centering{}--------------\pend
           
\pstart
           \centering{}\label{K_L00879-1v}\edtext{Kürzen!}{\lemma{\textnormal{\emph{Kürzen!}}}\Cendnote{\textnormal{Worauf sich dieser Zettel bezieht, ist unklar. Da die sonstige
                  Kommunikation keinen Anhaltspunkt bietet und Hofmannsthal\pwindex{Hofmannsthal, Hugo von 1874-02-01 – 1929-07-15@\textsc{Hofmannsthal, Hugo von} (1874-02-01 – 1929-07-15), \emph{Schriftsteller/Schriftstellerin}|pwk} die ersten zehn Tage des Monats nicht in Wien\oindex{Wien@\textbf{Wien}, \emph{A.ADM2}|pwk} war, könnte es sich um eine schriftlich nachgereichte
                  Antwort in Folge eines persönlichen Treffens handeln. Diese fanden am 11. 1. 1899
                  und am 17. 1. 1899 statt.}}}\label{K_L00879-1}\pend
           \selectlanguage{ngerman}\endnumbering\briefempfaengerindex{Schnitzler, Arthur@\textsc{Schnitzler, Arthur}!zzzHofmannsthal, Hugo von@\emph{von Hugo von Hofmannsthal}!1899-01-111@{{[}nach dem 11.? 1. 1899{]}}|)be}\mylabel{L00879h}  \normalsize

\doendnotes{C}
\bigskip
\vfill

\clearpage

\footnotesize

\lohead{\textsc{register}}

% Definiere theindex-Environment komplett neu ohne reledmac
\makeatletter
\renewenvironment{theindex}{%
  \section*{\indexname}%
  \setlength{\parindent}{0pt}%
  \setlength{\parskip}{0pt plus 0.3pt}%
  \let\item\@idxitem
}{%
  \clearpage
}
\makeatother

\IfFileExists{\jobname-pw.ind}{\input{\jobname-pw.ind}}{}

\end{document}

      