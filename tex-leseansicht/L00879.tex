\input{../tex-inputs/latex-pdf-vorspann}
\begin{center}
            \textcolor{red}{ENTWURF. ENTZIFFERUNG NOCH NICHT KORREKTURGELESEN}
                      \end{center}
            
               \section[Hugo von Hofmannsthal an Arthur Schnitzler, {[}nach dem 11.? 1. 1899{]}]{ Hugo von Hofmannsthal an Arthur Schnitzler, {[}nach dem
               11.? 1. 1899{]}}\nopagebreak\mylabel{v}\rehead{ }\begin{ledgroupsized}[t]{13cm}\normalsize\beginnumbering\briefempfaengerindex{Schnitzler, Arthur@\textsc{Schnitzler, Arthur}!zzzHofmannsthal, Hugo von@\emph{von Hugo von Hofmannsthal}!1899-01-111@{{[}nach dem 11.? 1. 1899{]}}|(be} \toendnotes[C]{\smallbreak\pagebreak[2]} \Standort{CUL, Schnitzler, B 43.}
\physDesc{Brief, 1 Blatt, 1 Seite
\newline{}Handschrift: Bleistift, lateinische Kurrent
\newline{}Schnitzler: mit Bleistift datiert: »Jänner 99« \newline{}Ordnung: 1) mit Bleistift von unbekannter Hand nummeriert: »\strikeout{133}« 2) mit Bleistift von unbekannter Hand nummeriert:
                                    »132«}\buchAbdrucke{\weitereDrucke{Hugo von Hofmannsthal, Arthur Schnitzler: \emph{Briefwechsel}. Hg. Therese Nickl und Heinrich Schnitzler. Frankfurt am Main: \emph{S. Fischer} 1964, S. 117.} }\toendnotes[C]{\smallbreak}\pstart
           \noindent{}\centering{}{\pb}D\textsuperscript{r}
               Arthur Schnitzler\pend
           \pstart
           \noindent{}\centering{}Frankgasse 1\oindex{Frankgasse@\textbf{Frankgasse}|pw}\pend
           \pstart
           \noindent{}\centering{}--------------\pend
           \pstart
           \noindent{}\centering{}\label{K_L00879_1v}\edtext{Kürzen!}{\lemma{\textnormal{\emph{Kürzen!}}}\Cendnote{\textnormal{Worauf sich dieser Zettel bezieht, ist unklar. Da die sonstige
                  Kommunikation keinen Anhaltspunkt bietet und Hofmannsthal\pwindex{Hofmannsthal, Hugo von 01.02.1874 – 15.07.1929@\textsc{Hofmannsthal, Hugo von} (01.02.1874 – 15.07.1929), \emph{Schriftsteller}|pwk} die ersten zehn Tage des Monats nicht in Wien\oindex{Wien@\textbf{Wien}|pwk} war, könnte es sich um eine schriftlich nachgereichte
                  Antwort nach einem persönlichen Treffen handeln. Diese fanden am 11.
                  und am 17. 1. 1899 statt.}}}\label{K_L00879_1h}\pend
           \endnumbering\briefempfaengerindex{Schnitzler, Arthur@\textsc{Schnitzler, Arthur}!zzzHofmannsthal, Hugo von@\emph{von Hugo von Hofmannsthal}!1899-01-111@{{[}nach dem 11.? 1. 1899{]}}|)be}\mylabel{h}\end{ledgroupsized}  \newcommand{\dateiname}{L00879}\newcommand{\titel}{Hugo von Hofmannsthal an Arthur Schnitzler, [nach dem 11.? 1. 1899]}\newcommand{\editorInnen}{Martin Anton Müller und Gerd-Hermann Susen}\input{../tex-inputs/latex-pdf-abspann}
      