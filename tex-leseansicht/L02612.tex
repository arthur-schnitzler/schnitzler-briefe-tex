%% latex-leseansicht-vorspann.tex
%% Vorspann für die Leseansicht.
%% Lädt die gemeinsame Datei latex-vorspann.tex mit nicht gesetztem Schalter.

\newif\ifkorrekturansicht
\korrekturansichtfalse

\input{../tex-inputs/latex-vorspann}

\begin{center}
            \textcolor{red}{ENTWURF. ENTZIFFERUNG NOCH NICHT KORREKTURGELESEN}
                      \end{center}
            
               \section[Paul Goldmann an Arthur Schnitzler, 8. 9. {[}1894{]}]{ Paul Goldmann an Arthur Schnitzler, 8. 9. {[}1894{]}}\nopagebreak\mylabel{v}\rehead{ }\begin{ledgroupsized}[t]{13cm}\normalsize\beginnumbering\briefempfaengerindex{Schnitzler, Arthur@\textsc{Schnitzler, Arthur}!zzzGoldmann, Paul@\emph{von Paul Goldmann}!1894-09-081@{8. 9. {[}1894{]}}|(be} \toendnotes[C]{\smallbreak\pagebreak[2]} \Standort{DLA, A:Schnitzler, HS.NZ85.1.3164.}
\physDesc{Brief, 2 Blätter, 8 Seiten
\newline{}Handschrift: schwarze Tinte, deutsche Kurrent
\newline{}Schnitzler: 1) mit Bleistift auf dem ersten Blatt die Jahreszahl
                                       »94« vermerkt 2) mit rotem Buntstift eine Unterstreichung}\toendnotes[C]{\smallbreak}\pstart
           \raggedleft{}{\pb}Frankfurt\oindex{Frankfurt am Main@\textbf{Frankfurt am Main}|pw}{ }8. September.\pend
           \pstart\center{}Mein lieber Freund,\pend\pstart
           Ich danke Dir noch von Herzen für die köſtlichen Tage in \label{K_L02612-1v}\edtext{\textsc{Ischl\oindex{Bad Ischl@\textbf{Bad Ischl}|pw}}}{\lemma{\textnormal{\emph{Ischl}}}\Cendnote{\textnormal{Von 23. 8. 1894 bis 3. 9. 1894 verbrachten
                  Schnitzler und Goldmann einige Zeit gemeinsam in Bad
                     Ischl\oindex{Bad Ischl@\textbf{Bad Ischl}|pwk} und Bad Aussee\oindex{Bad Aussee@\textbf{Bad Aussee}|pwk}.}}}\label{K_L02612-1h}. Ich bin
               ruhig und froh geweſen, wie ſchon lange nicht. Ich danke Euch\pwindex{Beer-Hofmann, Richard 11.07.1866 – 26.09.1945@\textsc{Beer-Hofmann, Richard} (11.07.1866 – 26.09.1945), \emph{Schriftsteller}|pwv}, daß Ihr mir meine
               Geſpenſter auf ein paar Stunden geſcheucht habt, daß Ihr mich Treue und Gute habt
               fühlen laſſen, {\pb}daß Ihr mir gar – oh
                  Wunder\textcolor{gray}{,} – ein wenig Glauben an mich ſelbſt gegeben habt. Ich
               bin heut muthig und beinahe heiter. Das iſt Euer Werk!
               Und ich bin Euch\pwindex{Beer-Hofmann, Richard 11.07.1866 – 26.09.1945@\textsc{Beer-Hofmann, Richard} (11.07.1866 – 26.09.1945), \emph{Schriftsteller}|pwv}
               tief dafür \strikeout{\textcolor{gray}{v}} verpflichtet{\dotsfive}\pend
           \pstart
           Bei dem Regen wirſt Du kaum Deine \textsc{Bicycle}-Partie gemacht
               haben, und Du biſt gewiß ſchon in Wien\oindex{Wien@\textbf{Wien}|pw} für den Winter
               inſtalliert und ſitzeſt über der Arbeit. Der \label{K_mets_Goldmann_94-partII-999v}\edtext{Artikel\pwindex{Marholm, Laura 19.04.1854 – 06.10.1928@\textsc{Marholm, Laura} (19.04.1854 – 06.10.1928), \emph{Schriftstellerin}!Maerchen25.8.1894 – 25.8.1894@\strich\emph{Ein Märchen} {[}25.8.1894 – 25.8.1894{]}|pwv}}{\lemma{\textnormal{\emph{Artikel}}}\Cendnote{\textnormal{Laura
                        Marholm\pwindex{Marholm, Laura 19.04.1854 – 06.10.1928@\textsc{Marholm, Laura} (19.04.1854 – 06.10.1928), \emph{Schriftstellerin}|pwk}: \emph{Ein Märchen}\pwindex{Marholm, Laura 19.04.1854 – 06.10.1928@\textsc{Marholm, Laura} (19.04.1854 – 06.10.1928), \emph{Schriftstellerin}!Maerchen25.8.1894 – 25.8.1894@\strich\emph{Ein Märchen} {[}25.8.1894 – 25.8.1894{]}|pwk}. In: \emph{Die Zukunft}\pwindex{Zukunft1892 – 1922@\emph{Die Zukunft}|pwk}, Jg. 8,
                        25. 8. 1894, S. 368–371.}}}\label{K_mets_Goldmann_94-partII-999h}{ }{\pb}von der \textsc{Marholm\pwindex{Marholm, Laura 19.04.1854 – 06.10.1928@\textsc{Marholm, Laura} (19.04.1854 – 06.10.1928), \emph{Schriftstellerin}|pw}}, den ich mit
               Hochgenuß gleich in \textsc{Nuernberg\oindex{Nuernberg@\textbf{Nürnberg}|pw}} geleſen habe, iſt \strikeout{w} wie eine
               Antwort auf unſer letztes Geſpräch gekommen. Jetzt wirſt Du hoffentlich lange nicht
               mehr daran zweifeln, daß \textsc{Arthur Schnitzler} eine
               Individualität iſt. Ich beglückwünſche Dich zu dieſem schönen Erfolge.\pend
           \pstart
           Mit \strikeout{M} meinem Onkel\pwindex{Mamroth, Fedor 21.02.1851 – 25.06.1907@\textsc{Mamroth, Fedor} (21.02.1851 – 25.06.1907), \emph{Journalist, Kritiker}|pwv}{ }{\pb}habe ich ſofort geſprochen. Ich habe ihn unerwartet
               liebevoll und warm vorgefunden, auch voll ſreundſchaftlichen Intereſſes für Dich. Er
               ging ſofort auf meinen Vorſchlag ein, Dir einen Theil des \label{K_L02612-2v}\edtext{Bücher-Reſerats}{\lemma{\textnormal{\emph{Bücher-Reſerats}}}\Cendnote{\textnormal{XXXX}}}\label{K_L02612-2h} zu übertragen. Das iſt nur ein Anſang. Wenn Du regelmäßig arbeiteſt,
               kann noch {\pb}allerlei Anderes daraus werden. Die
               Hauptſache iſt, wie geſagt, daß Du die Sachen regelmäßig erledigſt – nicht ſür
               beſtimmte Termine, aber doch in beſtimmten nicht allzu langen Friſten. Mach’ ruhig
               den Verſuch; ich bin überzeugt, daß es ſo gehen wird. Das Feuilleton bringt, {\pb}glaube ich, \textsc{40 Mark}.\pend
           \pstart
           Ich bleibe noch bis nächſten Samſtag hier. Haſt Du
               Zeit, ſo ſchreib’ mir ein Wort hierher (Adreſſe: \textsc{Frau Clementine Goldmann\pwindex{Goldmann, Clementine 1842-05-15 – 1924-02-24@\textsc{Goldmann, Clementine} (1842-05-15 – 1924-02-24)|pw}}, \textsc{Lindenstraße 1\oindex{Lindenstrasse@\textbf{Lindenstraße}|pw}}). Vor
               Allem: Wie geht es mit Deiner Arbeit? Hat \textsc{Richard\pwindex{Beer-Hofmann, Richard 11.07.1866 – 26.09.1945@\textsc{Beer-Hofmann, Richard} (11.07.1866 – 26.09.1945), \emph{Schriftsteller}|pw}}{ }{\pb} ſeine Reiſe angetreten? Was hört man von der neuen
                  \textsc{Revue\orgindex{Zeit. Wiener Wochenschrift@Die Zeit. Wiener Wochenschrift|pwv}}?\pend
           \pstart
           Die Meinigen\pwindex{Goldmann, Clementine 1842-05-15 – 1924-02-24@\textsc{Goldmann, Clementine} (1842-05-15 – 1924-02-24)|pwuv}
               grüßen Dich herzlichſt. Bitte, empfiehl’ mich Deiner Frau Mutter\pwindex{Schnitzler, Louise 08.07.1840 – 09.09.1911@\textsc{Schnitzler, Louise} (08.07.1840 – 09.09.1911)|pwv} und \label{K_L02612-11v}\edtext{Danke auch ihr}{\lemma{\textnormal{\emph{Danke auch ihr}}}\Cendnote{\textnormal{Schnitzler\pwindex{Schnitzler, Arthur 15.05.1862 – 21.10.1931@\textsc{Schnitzler, Arthur} (15.05.1862 – 21.10.1931), \emph{Schriftsteller, Mediziner}|pwk} urlaubte mit seiner Familie in
                     Ischl\oindex{Bad Ischl@\textbf{Bad Ischl}|pwk}; die hier angesprochene Danksagung
                  dürfte auf eine Form der Gastfreundschaft bezogen sein, die Louise Schnitzler\pwindex{Schnitzler, Louise 08.07.1840 – 09.09.1911@\textsc{Schnitzler, Louise} (08.07.1840 – 09.09.1911)|pwk}{ }Paul Goldmann\pwindex{Goldmann, Paul 31.01.1865 – 25.09.1935@\textsc{Goldmann, Paul} (31.01.1865 – 25.09.1935), \emph{Schriftsteller, Journalist}|pwk} bei seinem Besuch zukommen
                  ließ.}}}\label{K_L02612-11h} nochmals in meinem Namen. Grüß’ mir Deinen Bruder\pwindex{Schnitzler, Julius 13.07.1865 – 29.06.1939@\textsc{Schnitzler, Julius} (13.07.1865 – 29.06.1939), \emph{Chirurg}|pwv} u. Deine Schwägerin\pwindex{Schnitzler, Helene 16.07.1871 – September 1941@\textsc{Schnitzler, Helene} (16.07.1871 – September 1941)|pwv}.\pend
           \pstart
           {\pb}Und ſei Du ſelbſt von Herzen und in Treue
               gegrüßt von{\\[\baselineskip]} Deinem{\\[\baselineskip]}\spacefill\mbox{Paul Goldmann}\pend
           \leftskip=0em{}\endnumbering\briefempfaengerindex{Schnitzler, Arthur@\textsc{Schnitzler, Arthur}!zzzGoldmann, Paul@\emph{von Paul Goldmann}!1894-09-081@{8. 9. {[}1894{]}}|)be}\mylabel{h}\end{ledgroupsized}  \newcommand{\dateiname}{L02612}\newcommand{\titel}{Paul Goldmann an Arthur Schnitzler, 8. 9. [1894]}\newcommand{\editorInnen}{Martin Anton Müller und Laura Untner}%% latex-leseansicht-abspann.tex
%% Abspann für die Leseansicht.
%% Der Schalter \ifkorrekturansicht ist bereits durch den Vorspann gesetzt.

%% latex-abspann.tex
%% Gemeinsamer Abspann für Korrekturansicht und Leseansicht.
%% Setzt den Schalter \ifkorrekturansicht voraus (gesetzt in den
%% einbindenden Dateien latex-korrekturansicht-abspann.tex bzw.
%% latex-leseansicht-abspann.tex).
%% ---------------------------------------------------------------

\normalsize

% Das esempio-Environment wird nur in der Leseansicht benötigt
\ifkorrekturansicht\else
\newenvironment{esempio}[3]%
{
    \vspace{1.5ex}
    \rlap{\underline{#1}}
    \par
    \setlength{\parindent}{0cm}
    \nopagebreak
    \leftskip=#2cm
    \rightskip=#3cm
}
{
    \par
}
\fi

\doendnotes{C}
\bigskip
\vfill

\clearpage

\footnotesize

\ifkorrekturansicht
  \lohead{\textsc{register}}
\fi

% theindex-Environment neu definieren ohne reledmac
\makeatletter
\renewenvironment{theindex}{%
  \ifkorrekturansicht
    \section*{\indexname}%
  \else
    \subsubsection*{Index der erwähnten Entitäten}%
  \fi
  \setlength{\parindent}{0pt}%
  \setlength{\parskip}{0pt plus 0.3pt}%
  \let\item\@idxitem
}{%
  \ifkorrekturansicht\clearpage\fi
}
\makeatother

\IfFileExists{\jobname-pw.ind}{\input{\jobname-pw.ind}}{}

% Quellenangabe nur in der Leseansicht
\ifkorrekturansicht\else
% Fallback-Definitionen, falls die .tex-Datei \titel etc. nicht gesetzt hat
\providecommand{\titel}{}
\providecommand{\editorInnen}{}
\providecommand{\dateiname}{\jobname}

\vspace{3cm}

\vfill

\footnotesize
\textsc{Quelle}: \titel. Herausgegeben von {\editorInnen}. In: \emph{Arthur Schnitzler: Briefwechsel mit Autorinnen und Autoren}.
 Digitale Edition, https://schnitzler-briefe.acdh.oeaw.ac.at/{\dateiname}.html (Stand \today)
\fi

\end{document}


      