%% latex-leseansicht-vorspann.tex
%% Vorspann für die Leseansicht.
%% Lädt die gemeinsame Datei latex-vorspann.tex mit nicht gesetztem Schalter.

\newif\ifkorrekturansicht
\korrekturansichtfalse

\input{../tex-inputs/latex-vorspann}


\section[ Paul Goldmann an Arthur Schnitzler, 8. 9. [1894]]{L02612 Paul Goldmann an Arthur Schnitzler,  8. 9. [1894]}
\nopagebreak\mylabel{L02612v}
\rehead{ }\normalsize\beginnumbering\briefempfaengerindex{Schnitzler, Arthur@\textsc{Schnitzler, Arthur}!zzzGoldmann, Paul@\emph{von Paul Goldmann}!1894-09-081@{8. 9. [1894]}|(be}
\toendnotes[C]{\smallbreak\pagebreak[2]}
\correspDesc{Versand  durch Paul Goldmann am 8. 9. [1894] in Frankfurt am Main
\newline{}Erhalt  durch Arthur Schnitzler im Zeitraum [9. 9. 1894
                  – 13. 9. 1894?] in Wien}\toendnotes[C]{\smallbreak}
\Standort{DLA, A:Schnitzler, HS.NZ85.1.3164.}
\physDesc{Brief, 2 Blätter, 8 Seiten, 1982 Zeichen
\newline{}Handschrift: blaue Tinte, deutsche Kurrent
\newline{}Schnitzler: 1) mit Bleistift auf dem ersten Blatt die Jahreszahl »94« vermerkt  2) mit rotem Buntstift zwei Unterstreichungen}\toendnotes[C]{\smallbreak}
\pstart
           \raggedleft{}{\pb}Frankfurt\oindex{Frankfurt am Main@\textbf{Frankfurt am Main}, \emph{Hauptstadt}|pw}{ }8. September.\pend
           
\pstart\center{}Mein lieber Freund,\pend\vspace{0.5em}
\pstart
           Ich danke Dir noch von Herzen für die köſtlichen Tage in \label{K_L02612-1v}\edtext{\textsc{Ischl\oindex{Bad Ischl@\textbf{Bad Ischl}|pw}}}{\lemma{\textnormal{\emph{Ischl}}}\Cendnote{\textnormal{Vom 23. 8. 1894 bis zum 3. 9. 1894 verbrachten Schnitzler und Goldmann\pwindex{Goldmann, Paul 31.\,1.\,1865 Breslau – 25.\,9.\,1935 Wien@\textsc{Goldmann, Paul} (31.\,1.\,1865 Breslau – 25.\,9.\,1935 Wien), \emph{Schriftsteller, Journalist}|pwk} einige Tage gemeinsam in Bad
                     Ischl\oindex{Bad Ischl@\textbf{Bad Ischl}|pwk} und Bad Aussee\oindex{Bad Aussee@\textbf{Bad Aussee}, \emph{Hauptstadt}|pwk}.}}}\label{K_L02612-1}. Ich bin
               ruhig und froh geweſen, wie{ }ſchon lange nicht. Ich danke Euch\pwindex{Beer-Hofmann, Richard 11.\,7.\,1866 Wien – 26.\,9.\,1945 New York City@\textsc{Beer-Hofmann, Richard} (11.\,7.\,1866 Wien – 26.\,9.\,1945 New York City), \emph{Schriftsteller}|pwv}, daß Ihr mir meine
               Geſpenſter auf ein paar Stunden geſcheucht habt, daß Ihr mich Treue und \label{K_L02612-2v}\edtext{G{[}ü{]}te}{\lemma{\textnormal{\emph{Güte}}}\Cendnote{\textnormal{Goldmann\pwindex{Goldmann, Paul 31.\,1.\,1865 Breslau – 25.\,9.\,1935 Wien@\textsc{Goldmann, Paul} (31.\,1.\,1865 Breslau – 25.\,9.\,1935 Wien), \emph{Schriftsteller, Journalist}|pwk} schreibt
                  »Gute«.}}}\label{K_L02612-2} habt fühlen laſſen, {\pb}daß Ihr mir gar – oh Wunder, – ein wenig Glauben an mich{ }ſelbſt gegeben habt. Ich
               bin heut muthig und beinahe heiter. Das iſt Euer Werk!
               Und ich bin Euch\pwindex{Beer-Hofmann, Richard 11.\,7.\,1866 Wien – 26.\,9.\,1945 New York City@\textsc{Beer-Hofmann, Richard} (11.\,7.\,1866 Wien – 26.\,9.\,1945 New York City), \emph{Schriftsteller}|pwv} tief dafür \strikeout{\textcolor{gray}{v}} verpflichtet{\dotsfive}\pend
           
\pstart
           Bei dem Regen wirſt Du kaum Deine \textsc{Bicycle}-Partie gemacht
               haben, und Du biſt gewiß{ }ſchon in Wien\oindex{Wien@\textbf{Wien}, \emph{Verwaltungsgebiet}|pw} für den
               Winter inſtallirt und{ }ſitzeſt über der Arbeit. Der \label{K_L02612-3v}\edtext{Artikel\pwindex{Marholm, Laura 19.\,4.\,1854 Riga – 6.\,10.\,1928 Jūrmala@\textsc{Marholm, Laura} (19.\,4.\,1854 Riga – 6.\,10.\,1928 Jūrmala), \emph{Schriftstellerin}!Märchen@\strich\emph{Ein Märchen}|pwv}}{\lemma{\textnormal{\emph{Artikel}}}\Cendnote{\textnormal{Laura Marholm\pwindex{Marholm, Laura 19.\,4.\,1854 Riga – 6.\,10.\,1928 Jūrmala@\textsc{Marholm, Laura} (19.\,4.\,1854 Riga – 6.\,10.\,1928 Jūrmala), \emph{Schriftstellerin}|pwk}: \emph{Ein Märchen}\pwindex{Marholm, Laura 19.\,4.\,1854 Riga – 6.\,10.\,1928 Jūrmala@\textsc{Marholm, Laura} (19.\,4.\,1854 Riga – 6.\,10.\,1928 Jūrmala), \emph{Schriftstellerin}!Märchen@\strich\emph{Ein Märchen}|pwk}. In: \emph{Die
                        Zukunft}\pwindex{Zukunft@\emph{Die Zukunft}|pwk}, Jg. 8, 25. 8. 1894, S. 368–371.}}}\label{K_L02612-3}{ }{\pb}von der \textsc{Marholm\pwindex{Marholm, Laura 19.\,4.\,1854 Riga – 6.\,10.\,1928 Jūrmala@\textsc{Marholm, Laura} (19.\,4.\,1854 Riga – 6.\,10.\,1928 Jūrmala), \emph{Schriftstellerin}|pw}}, den ich mit Hochgenuß gleich in \textsc{Nuernberg\oindex{Nürnberg@\textbf{Nürnberg}|pw}} geleſen habe, iſt \strikeout{w} wie eine Antwort auf unſer
               letztes Geſpräch gekommen. Jetzt wirſt Du hoffentlich lange nicht mehr daran
               zweifeln, daß \textsc{Arthur Schnitzler} eine Individualität iſt.
               Ich beglückwünſche Dich zu dieſem schönen Erfolge.\pend
           
\pstart
           Mit \strikeout{M} meinem Onkel\pwindex{Mamroth, Fedor 21.\,2.\,1851 Breslau – 25.\,6.\,1907 Frankfurt am Main@\textsc{Mamroth, Fedor} (21.\,2.\,1851 Breslau – 25.\,6.\,1907 Frankfurt am Main), \emph{Journalist, Kritiker}|pwv}{ }{\pb}habe ich{ }ſofort geſprochen. Ich habe ihn unerwartet
               liebevoll und warm vorgefunden, auch voll{ }ſreundſchaftlichen Intereſſes für Dich. Er
               ging{ }ſofort auf meinen Vorſchlag ein, Dir einen Theil des \label{K_L02612-4v}\edtext{Bücher-Reſerats}{\lemma{\textnormal{\emph{Bücher-Reserats}}}\Cendnote{\textnormal{Schnitzler dürfte überlegt haben, wie er
                  seine medizinische Praxis aufgeben konnte. Der hier skizzierte Plan der
                  Mitarbeit am Kulturfeuilleton der \emph{Frankfurter
                     Zeitung}\orgindex{Frankfurter Zeitung@Frankfurter Zeitung|pwk} wurde nicht realisiert.}}}\label{K_L02612-4} zu übertragen. Das iſt nur ein
               Anſang. Wenn Du regelmäßig arbeiteſt, kann noch {\pb}allerlei Anderes daraus werden. Die Hauptſache iſt, wie geſagt, daß Du die Sachen
               regelmäßig erledigſt – nicht{ }ſür beſtimmte Termine, aber doch in beſtimmten nicht
               allzu langen Friſten. Mach’ ruhig den Verſuch; ich bin überzeugt, daß es{ }ſo gehen
               wird. Das Feuilleton bringt, {\pb}glaube ich, \textsc{40 Mark}.\pend
           
\pstart
           Ich bleibe noch bis nächſten Samſtag hier. Haſt Du
               Zeit,{ }ſo{ }ſchreib’ mir ein Wort hierher (Adreſſe: \textsc{Frau Clementine Goldmann\pwindex{Goldmann, Clementine 15.\,5.\,1842 Breslau – 24.\,2.\,1924 Frankfurt am Main@\textsc{Goldmann, Clementine} (15.\,5.\,1842 Breslau – 24.\,2.\,1924 Frankfurt am Main)|pw}}, \textsc{Lindenstraſse 1\oindex{Lindenstraße@\textbf{Lindenstraße}, \emph{Straße}|pw}}). Vor Allem: Wie geht es mit Deiner Arbeit? Hat \textsc{Richard\pwindex{Beer-Hofmann, Richard 11.\,7.\,1866 Wien – 26.\,9.\,1945 New York City@\textsc{Beer-Hofmann, Richard} (11.\,7.\,1866 Wien – 26.\,9.\,1945 New York City), \emph{Schriftsteller}|pw}}{ }{\pb}ſeine Reiſe angetreten? Was hört man von der neuen
                  \textsc{Revue\orgindex{Zeit. Wiener Wochenschrift@Die Zeit. Wiener Wochenschrift|pwv}}?\pend
           
\pstart
           Die Meinigen\pwindex{Goldmann, Clementine 15.\,5.\,1842 Breslau – 24.\,2.\,1924 Frankfurt am Main@\textsc{Goldmann, Clementine} (15.\,5.\,1842 Breslau – 24.\,2.\,1924 Frankfurt am Main)|pwuv}
               grüßen Dich herzlichſt. Bitte, empfiehl’ mich Deiner Frau Mutter\pwindex{Schnitzler, Louise 8.\,7.\,1840 Kőszeg – 9.\,9.\,1911 Wien@\textsc{Schnitzler, Louise} (8.\,7.\,1840 Kőszeg – 9.\,9.\,1911 Wien)|pwv} und \label{K_L02612-5v}\edtext{danke auch ihr}{\lemma{\textnormal{\emph{danke auch ihr}}}\Cendnote{\textnormal{Schnitzler urlaubte mit seiner Familie in
                     Ischl\oindex{Bad Ischl@\textbf{Bad Ischl}|pwk}; die hier angesprochene Danksagung
                  dürfte auf eine Form der Gastfreundschaft bezogen sein, die Louise Schnitzler\pwindex{Schnitzler, Louise 8.\,7.\,1840 Kőszeg – 9.\,9.\,1911 Wien@\textsc{Schnitzler, Louise} (8.\,7.\,1840 Kőszeg – 9.\,9.\,1911 Wien)|pwk}{ }Paul Goldmann\pwindex{Goldmann, Paul 31.\,1.\,1865 Breslau – 25.\,9.\,1935 Wien@\textsc{Goldmann, Paul} (31.\,1.\,1865 Breslau – 25.\,9.\,1935 Wien), \emph{Schriftsteller, Journalist}|pwk} bei seinem Besuch hatte zukommen
                  lassen.}}}\label{K_L02612-5} nochmals in meinem Namen. Grüß’ mir Deinen Bruder\pwindex{Schnitzler, Julius 13.\,7.\,1865 Wien – 29.\,6.\,1939 ebd.@\textsc{Schnitzler, Julius} (13.\,7.\,1865 Wien – 29.\,6.\,1939 ebd.), \emph{Chirurg}|pwv} u. Deine Schwägerin\pwindex{Schnitzler, Helene 16.\,7.\,1871 Budapest – September 1941 Atlantischer Ozean@\textsc{Schnitzler, Helene} (16.\,7.\,1871 Budapest – September 1941 Atlantischer Ozean)|pwv}.\pend
           
\pstart
           {\pb}Und{ }ſei Du{ }ſelbſt von Herzen und in Treue gegrüßt
               von{\\[\baselineskip]} Deinem{\\[\baselineskip]}\spacefill\mbox{Paul Goldmann}\pend
           \leftskip=0em{}\selectlanguage{ngerman}\endnumbering\briefempfaengerindex{Schnitzler, Arthur@\textsc{Schnitzler, Arthur}!zzzGoldmann, Paul@\emph{von Paul Goldmann}!1894-09-081@{8. 9. [1894]}|)be}\mylabel{L02612h}  \newcommand{\dateiname}{L02612}\newcommand{\titel}{Paul Goldmann an Arthur Schnitzler, 8. 9. [1894]}\newcommand{\editorInnen}{Martin Anton Müller und Laura Untner}%% latex-leseansicht-abspann.tex
%% Abspann für die Leseansicht.
%% Der Schalter \ifkorrekturansicht ist bereits durch den Vorspann gesetzt.

%% latex-abspann.tex
%% Gemeinsamer Abspann für Korrekturansicht und Leseansicht.
%% Setzt den Schalter \ifkorrekturansicht voraus (gesetzt in den
%% einbindenden Dateien latex-korrekturansicht-abspann.tex bzw.
%% latex-leseansicht-abspann.tex).
%% ---------------------------------------------------------------

\normalsize

% Das esempio-Environment wird nur in der Leseansicht benötigt
\ifkorrekturansicht\else
\newenvironment{esempio}[3]%
{
    \vspace{1.5ex}
    \rlap{\underline{#1}}
    \par
    \setlength{\parindent}{0cm}
    \nopagebreak
    \leftskip=#2cm
    \rightskip=#3cm
}
{
    \par
}
\fi

\doendnotes{C}
\bigskip
\vfill

\clearpage

\footnotesize

\ifkorrekturansicht
  \lohead{\textsc{register}}
\fi

% theindex-Environment neu definieren ohne reledmac
\makeatletter
\renewenvironment{theindex}{%
  \ifkorrekturansicht
    \section*{\indexname}%
  \else
    \subsubsection*{Index der erwähnten Entitäten}%
  \fi
  \setlength{\parindent}{0pt}%
  \setlength{\parskip}{0pt plus 0.3pt}%
  \let\item\@idxitem
}{%
  \ifkorrekturansicht\clearpage\fi
}
\makeatother

\IfFileExists{\jobname-pw.ind}{\input{\jobname-pw.ind}}{}

% Quellenangabe nur in der Leseansicht
\ifkorrekturansicht\else
% Fallback-Definitionen, falls die .tex-Datei \titel etc. nicht gesetzt hat
\providecommand{\titel}{}
\providecommand{\editorInnen}{}
\providecommand{\dateiname}{\jobname}

\vspace{3cm}

\vfill

\footnotesize
\textsc{Quelle}: \titel. Herausgegeben von {\editorInnen}. In: \emph{Arthur Schnitzler: Briefwechsel mit Autorinnen und Autoren}.
 Digitale Edition, https://schnitzler-briefe.acdh.oeaw.ac.at/{\dateiname}.html (Stand \today)
\fi

\end{document}


