%% latex-korrekturansicht-vorspann.tex
%% Vorspann für die Korrekturansicht.
%% Lädt die gemeinsame Datei latex-vorspann.tex mit gesetztem Schalter.

\newif\ifkorrekturansicht
\korrekturansichttrue

\input{../tex-inputs/latex-vorspann}


\section[ Paul Goldmann an Arthur Schnitzler, 8. 9. {[}1894{]}]{L02612 Paul Goldmann an Arthur Schnitzler, 8. 9. {[}1894{]}}
\nopagebreak\mylabel{L02612v}
\rehead{ }\normalsize\beginnumbering\briefempfaengerindex{Schnitzler, Arthur@\textsc{Schnitzler, Arthur}!zzzGoldmann, Paul@\emph{von Paul Goldmann}!1894-09-081@{8. 9. {[}1894{]}}|(be}
\toendnotes[C]{\smallbreak\pagebreak[2]}\Standort{DLA, A:Schnitzler, HS.NZ85.1.3164.}
\physDesc{Brief, 2 Blätter, 8 Seiten, 1982 Zeichen
\newline{}Handschrift: blaue Tinte, deutsche Kurrent
\newline{}Schnitzler: 1) mit Bleistift auf dem ersten Blatt die Jahreszahl »94« vermerkt  2) mit rotem Buntstift zwei Unterstreichungen}\toendnotes[C]{\smallbreak}
\pstart
           \raggedleft{}{\pb}Frankfurt\oindex{Frankfurt am Main@\textbf{Frankfurt am Main}, \emph{P.PPLA3}|pw}{ }8. September.\pend
           
\pstart\center{}Mein lieber Freund,\pend\vspace{0.5em}
\pstart
           Ich danke Dir noch von Herzen für die köſtlichen Tage in \label{K_L02612-1v}\edtext{\textsc{Ischl\oindex{Bad Ischl@\textbf{Bad Ischl}, \emph{P.PPL}|pw}}}{\lemma{\textnormal{\emph{Ischl}}}\Cendnote{\textnormal{Vom 23. 8. 1894 bis zum 3. 9. 1894 verbrachten Schnitzler und Goldmann\pwindex{Goldmann, Paul 31.01.1865 – 25.09.1935@\textsc{Goldmann, Paul} (31.01.1865 – 25.09.1935), \emph{Schriftsteller/Schriftstellerin, Journalist/Journalistin}|pwk} einige Tage gemeinsam in Bad
                     Ischl\oindex{Bad Ischl@\textbf{Bad Ischl}, \emph{P.PPL}|pwk} und Bad Aussee\oindex{Bad Aussee@\textbf{Bad Aussee}, \emph{P.PPLA3}|pwk}.}}}\label{K_L02612-1}. Ich bin
               ruhig und froh geweſen, wie ſchon lange nicht. Ich danke Euch\pwindex{Beer-Hofmann, Richard 1866-07-11 – 1945-09-26@\textsc{Beer-Hofmann, Richard} (1866-07-11 – 1945-09-26), \emph{Schriftsteller/Schriftstellerin}|pwv}, daß Ihr mir meine
               Geſpenſter auf ein paar Stunden geſcheucht habt, daß Ihr mich Treue und \label{K_L02612-2v}\edtext{G{[}ü{]}te}{\lemma{\textnormal{\emph{Güte}}}\Cendnote{\textnormal{Goldmann\pwindex{Goldmann, Paul 31.01.1865 – 25.09.1935@\textsc{Goldmann, Paul} (31.01.1865 – 25.09.1935), \emph{Schriftsteller/Schriftstellerin, Journalist/Journalistin}|pwk} schreibt
                  »Gute«.}}}\label{K_L02612-2} habt fühlen laſſen, {\pb}daß Ihr mir gar – oh Wunder, – ein wenig Glauben an mich ſelbſt gegeben habt. Ich
               bin heut muthig und beinahe heiter. Das iſt Euer Werk!
               Und ich bin Euch\pwindex{Beer-Hofmann, Richard 1866-07-11 – 1945-09-26@\textsc{Beer-Hofmann, Richard} (1866-07-11 – 1945-09-26), \emph{Schriftsteller/Schriftstellerin}|pwv} tief dafür \strikeout{\textcolor{gray}{v}} verpflichtet{\dotsfive}\pend
           
\pstart
           Bei dem Regen wirſt Du kaum Deine \textsc{Bicycle}-Partie gemacht
               haben, und Du biſt gewiß ſchon in Wien\oindex{Wien@\textbf{Wien}, \emph{A.ADM2}|pw} für den
               Winter inſtallirt und ſitzeſt über der Arbeit. Der \label{K_L02612-3v}\edtext{Artikel\pwindex{Maerchen@\emph{Ein Märchen}|pwv}}{\lemma{\textnormal{\emph{Artikel}}}\Cendnote{\textnormal{Laura Marholm\pwindex{Marholm, Laura 19.04.1854 – 06.10.1928@\textsc{Marholm, Laura} (19.04.1854 – 06.10.1928), \emph{Schriftsteller/Schriftstellerin}|pwk}: \emph{Ein Märchen}\pwindex{Maerchen@\emph{Ein Märchen}|pwk}. In: \emph{Die
                        Zukunft}\pwindex{Zukunft@\emph{Die Zukunft}|pwk}, Jg. 8, 25. 8. 1894, S. 368–371.}}}\label{K_L02612-3}{ }{\pb}von der \textsc{Marholm\pwindex{Marholm, Laura 19.04.1854 – 06.10.1928@\textsc{Marholm, Laura} (19.04.1854 – 06.10.1928), \emph{Schriftsteller/Schriftstellerin}|pw}}, den ich mit Hochgenuß gleich in \textsc{Nuernberg\oindex{Nuernberg@\textbf{Nürnberg}, \emph{P.PPL}|pw}} geleſen habe, iſt \strikeout{w} wie eine Antwort auf unſer
               letztes Geſpräch gekommen. Jetzt wirſt Du hoffentlich lange nicht mehr daran
               zweifeln, daß \textsc{Arthur Schnitzler} eine Individualität iſt.
               Ich beglückwünſche Dich zu dieſem schönen Erfolge.\pend
           
\pstart
           Mit \strikeout{M} meinem Onkel\pwindex{Mamroth, Fedor 21.02.1851 – 25.06.1907@\textsc{Mamroth, Fedor} (21.02.1851 – 25.06.1907), \emph{Journalist/Journalistin, Kritiker/Kritikerin}|pwv}{ }{\pb}habe ich ſofort geſprochen. Ich habe ihn unerwartet
               liebevoll und warm vorgefunden, auch voll ſreundſchaftlichen Intereſſes für Dich. Er
               ging ſofort auf meinen Vorſchlag ein, Dir einen Theil des \label{K_L02612-4v}\edtext{Bücher-Reſerats}{\lemma{\textnormal{\emph{Bücher-Reſerats}}}\Cendnote{\textnormal{Schnitzler dürfte überlegt haben, wie er
                  seine medizinische Praxis aufgeben konnte. Der hier skizzierte Plan der
                  Mitarbeit am Kulturfeuilleton der \emph{Frankfurter
                     Zeitung}\orgindex{Frankfurter Zeitung@Frankfurter Zeitung|pwk} wurde nicht realisiert.}}}\label{K_L02612-4} zu übertragen. Das iſt nur ein
               Anſang. Wenn Du regelmäßig arbeiteſt, kann noch {\pb}allerlei Anderes daraus werden. Die Hauptſache iſt, wie geſagt, daß Du die Sachen
               regelmäßig erledigſt – nicht ſür beſtimmte Termine, aber doch in beſtimmten nicht
               allzu langen Friſten. Mach’ ruhig den Verſuch; ich bin überzeugt, daß es ſo gehen
               wird. Das Feuilleton bringt, {\pb}glaube ich, \textsc{40 Mark}.\pend
           
\pstart
           Ich bleibe noch bis nächſten Samſtag hier. Haſt Du
               Zeit, ſo ſchreib’ mir ein Wort hierher (Adreſſe: \textsc{Frau Clementine Goldmann\pwindex{Goldmann, Clementine 1842-05-15 – 1924-02-24@\textsc{Goldmann, Clementine} (1842-05-15 – 1924-02-24)|pw}}, \textsc{Lindenstraſse 1\oindex{Lindenstrasse@\textbf{Lindenstraße}, \emph{Straße (K.STR)}|pw}}). Vor Allem: Wie geht es mit Deiner Arbeit? Hat \textsc{Richard\pwindex{Beer-Hofmann, Richard 1866-07-11 – 1945-09-26@\textsc{Beer-Hofmann, Richard} (1866-07-11 – 1945-09-26), \emph{Schriftsteller/Schriftstellerin}|pw}}{ }{\pb}ſeine Reiſe angetreten? Was hört man von der neuen
                  \textsc{Revue\orgindex{Zeit. Wiener Wochenschrift@Die Zeit. Wiener Wochenschrift|pwv}}?\pend
           
\pstart
           Die Meinigen\pwindex{Goldmann, Clementine 1842-05-15 – 1924-02-24@\textsc{Goldmann, Clementine} (1842-05-15 – 1924-02-24)|pwuv}
               grüßen Dich herzlichſt. Bitte, empfiehl’ mich Deiner Frau Mutter\pwindex{Schnitzler, Louise 1840-07-08 – 1911-09-09@\textsc{Schnitzler, Louise} (1840-07-08 – 1911-09-09)|pwv} und \label{K_L02612-5v}\edtext{danke auch ihr}{\lemma{\textnormal{\emph{danke auch ihr}}}\Cendnote{\textnormal{Schnitzler urlaubte mit seiner Familie in
                     Ischl\oindex{Bad Ischl@\textbf{Bad Ischl}, \emph{P.PPL}|pwk}; die hier angesprochene Danksagung
                  dürfte auf eine Form der Gastfreundschaft bezogen sein, die Louise Schnitzler\pwindex{Schnitzler, Louise 1840-07-08 – 1911-09-09@\textsc{Schnitzler, Louise} (1840-07-08 – 1911-09-09)|pwk}{ }Paul Goldmann\pwindex{Goldmann, Paul 31.01.1865 – 25.09.1935@\textsc{Goldmann, Paul} (31.01.1865 – 25.09.1935), \emph{Schriftsteller/Schriftstellerin, Journalist/Journalistin}|pwk} bei seinem Besuch hatte zukommen
                  lassen.}}}\label{K_L02612-5} nochmals in meinem Namen. Grüß’ mir Deinen Bruder\pwindex{Schnitzler, Julius 13.07.1865 – 29.06.1939@\textsc{Schnitzler, Julius} (13.07.1865 – 29.06.1939), \emph{Chirurg/Chirurgin}|pwv} u. Deine Schwägerin\pwindex{Schnitzler, Helene 16.07.1871 – September 1941@\textsc{Schnitzler, Helene} (16.07.1871 – September 1941)|pwv}.\pend
           
\pstart
           {\pb}Und ſei Du ſelbſt von Herzen und in Treue gegrüßt
               von{\\[\baselineskip]} Deinem{\\[\baselineskip]}\spacefill\mbox{Paul Goldmann}\pend
           \leftskip=0em{}\selectlanguage{ngerman}\endnumbering\briefempfaengerindex{Schnitzler, Arthur@\textsc{Schnitzler, Arthur}!zzzGoldmann, Paul@\emph{von Paul Goldmann}!1894-09-081@{8. 9. {[}1894{]}}|)be}\mylabel{L02612h}  \normalsize

\doendnotes{C}
\bigskip
\vfill

\clearpage

\footnotesize

\lohead{\textsc{register}}

% Definiere theindex-Environment komplett neu ohne reledmac
\makeatletter
\renewenvironment{theindex}{%
  \section*{\indexname}%
  \setlength{\parindent}{0pt}%
  \setlength{\parskip}{0pt plus 0.3pt}%
  \let\item\@idxitem
}{%
  \clearpage
}
\makeatother

\IfFileExists{\jobname-pw.ind}{\input{\jobname-pw.ind}}{}

\end{document}

      