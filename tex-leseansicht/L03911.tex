%% latex-leseansicht-vorspann.tex
%% Vorspann für die Leseansicht.
%% Lädt die gemeinsame Datei latex-vorspann.tex mit nicht gesetztem Schalter.

\newif\ifkorrekturansicht
\korrekturansichtfalse

\input{../tex-inputs/latex-vorspann}


\section[Arthur Schnitzler an Theodor Herzl, 30. 11. 1894]{L03911 Arthur Schnitzler an Theodor Herzl, 30. 11. 1894}
\nopagebreak\mylabel{L03911v}
\rehead{ }\normalsize\beginnumbering\briefempfaengerindex{Herzl, Theodor@\textsc{Herzl, Theodor}!zzzSchnitzler, Arthur@\emph{von Arthur Schnitzler}!1894-11-301@{30. 11. 1894}|(be}
\toendnotes[C]{\smallbreak\pagebreak[2]}
\correspDesc{Versand  durch Arthur Schnitzler am 30. 11. 1894 in Wien
\newline{}Erhalt  durch Theodor Herzl in Wien}\toendnotes[C]{\smallbreak}
\Standort{Jerusalem, Central Zionist Archives, H1:1924-16.}
\physDesc{,  Blätter,  Seiten
\newline{}Handschrift: , deutsche Kurrent}
\buchAbdrucke{\weitereDrucke{Arthur Schnitzler: \emph{Briefe 1875–1912}. Herausgegeben von Therese Nickl und Heinrich Schnitzler. Frankfurt am Main: \emph{S. Fischer} 1981, S. 239–241.} }\toendnotes[C]{\smallbreak}
\pstart{}{\pb}Lieber Freund,\pend\vspace{0.5em}
\pstart
           ich danke Ihnen ſehr, daſs Sie die Novelle\pwindex{Schnitzler, Arthur 15.\,5.\,1862 Wien – 21.\,10.\,1931 ebd.@\textsc{Schnitzler, Arthur} (15.\,5.\,1862 Wien – 21.\,10.\,1931 ebd.), \emph{Schriftsteller, Mediziner}!Sterben. Novelle@\strich\emph{Sterben. Novelle}|pwv}{ }ſo bald geleſen und freue mich, daſs Sie ſo viel gutes darin
               gefunden. Ihre Einwendung gegen den Beginn halte ich für gerechtfertigt. Mir ſelbſt
               misfiel er, nachdem ich ihn geſchrieben hatte, ſoſehr, daſs ich ein paar Wochen
               ausſetzte, weil ſich der rechte Muth nicht zum Weiterſchreiben finden wollte. Erst
               allmälig kam ich hinein; {\pb}es geht mir übrigens faſt
               immer ſo. Vor der Veröffentlichung wollte ich den Anfang – bis zu Mariens\pwindex{Schnitzler, Arthur 15.\,5.\,1862 Wien – 21.\,10.\,1931 ebd.@\textsc{Schnitzler, Arthur} (15.\,5.\,1862 Wien – 21.\,10.\,1931 ebd.), \emph{Schriftsteller, Mediziner}!Sterben. Novelle@\strich\emph{Sterben. Novelle}|pwv} Beſuch bei Alfred\pwindex{Schnitzler, Arthur 15.\,5.\,1862 Wien – 21.\,10.\,1931 ebd.@\textsc{Schnitzler, Arthur} (15.\,5.\,1862 Wien – 21.\,10.\,1931 ebd.), \emph{Schriftsteller, Mediziner}!Sterben. Novelle@\strich\emph{Sterben. Novelle}|pwv}{ }\textsc{esclusive} – einfach wegſtreichen; aber man rieth mir ab. Sie ſind der erſte, der ſeiner
               Antipathie gegen den Anfang ſo gründlich Ausdruck gibt – nach mir. Soweit ich bisher
               urtheilen kann, hat er niemanden ſo empfindlich geſtört wie uns zwei. Aber ich glaube
               nicht, daſs der Grund im »grauen, allzu grauen« liegt. Ich habe nur hier {\pb}das Grau nicht künstleriſch bewältigt. Der Stoff\pwindex{Schnitzler, Arthur 15.\,5.\,1862 Wien – 21.\,10.\,1931 ebd.@\textsc{Schnitzler, Arthur} (15.\,5.\,1862 Wien – 21.\,10.\,1931 ebd.), \emph{Schriftsteller, Mediziner}!Sterben. Novelle@\strich\emph{Sterben. Novelle}|pwv} wäre ja als \uline{ganzes} noch anders zu faſſen geweſen – wenn ich eben die erſten vier
               Akte zu dem fünften, der vorliegt, geſchrieben hätte. Da wäre das Liebesbachanal, das
               Sie wünſchen, im dritten gekommen. Es iſt möglich, daſs nur die Kraft zu der \uline{ganzen} Tragöde gefehlt hätte; die Wahrheit iſt
               jedenfalls, daſs nur dieſes fünfte Akt in meinen Abſichten lag. {\pb}Auch daſs noch irgendwo im Buch\pwindex{Schnitzler, Arthur 15.\,5.\,1862 Wien – 21.\,10.\,1931 ebd.@\textsc{Schnitzler, Arthur} (15.\,5.\,1862 Wien – 21.\,10.\,1931 ebd.), \emph{Schriftsteller, Mediziner}!Sterben. Novelle@\strich\emph{Sterben. Novelle}|pwv} Längen{ }ſind, hab’ ich beim Durchleſen der Correcturen
               geſpürt. Was die Manier in der Naturſchilderung anbelangt,{ }ſo wären mir Details
               erwünſcht; da ich \strikeout{h} mich hier unſchuldig fühle. Ich
               müſſte mich da an irgend ein Wollen erinnern, und ich weiſs doch, daſs ich alle dieſe
               Dinge ganz einfach hingeſchrieben habe. Sagen Sie mir doch, wo Sie die Manier
               entdeckt haben. Ihr künſtleriſch-kritiſches {\pb}Auge iſt hier
               maßgebender als mein Gedächtnis. Umſomehr als die Eri{\geminationn}erung an Stunden des Schaffens täuſchend iſt wie die an Träume.–\pend
           
\pstart
           Nun das Stück\pwindex{Herzl, Theodor 2.\,5.\,1860 Budapest – 3.\,7.\,1904 Edlach@\textsc{Herzl, Theodor} (2.\,5.\,1860 Budapest – 3.\,7.\,1904 Edlach), \emph{Schriftsteller, Journalist}!neue Ghetto. Schauspiel in vier Acten@\strich\emph{Das neue Ghetto. Schauspiel in vier Acten}|pwv}. Auf meine
               Bemerkung betreffs neuer und insbeſondre jüdiſch-ſympathiſchen Figuren gingen Sie
               nicht ein. Und jemehr ich überlege, umſo weſentlicher{ }ſcheint mir das. Ich glaube
               auch, daſs die Frau des Helden um nichts weniger ins Ghetto zurückdrängt, wenn ſie
               auch ihr etwas vom Opfer geben. – An der Ein{\pb}führung B.’\pwindex{Herzl, Theodor 2.\,5.\,1860 Budapest – 3.\,7.\,1904 Edlach@\textsc{Herzl, Theodor} (2.\,5.\,1860 Budapest – 3.\,7.\,1904 Edlach), \emph{Schriftsteller, Journalist}!neue Ghetto. Schauspiel in vier Acten@\strich\emph{Das neue Ghetto. Schauspiel in vier Acten}|pwv}s hat mich das Geſpräch mit den
               Dienſtboten geſtört, das mir zu abſichtlich und ſelbſt theatraliſch unangehm ſcheint.
               Ich bin{ }ſehr begierig zu wiſſen, wie Sie ſich gegenüber meinen Ideen über
               den Schluſs verhalten.– – Daſs mir perſönlich der Begleitbrief zuſagt, brauchen nicht
               zu verſichern. Wie{ }ſich – »jene« – dazu verhalten werden, weiſs ich nicht. Ich
               denke, ſie werden das Stück\pwindex{Herzl, Theodor 2.\,5.\,1860 Budapest – 3.\,7.\,1904 Edlach@\textsc{Herzl, Theodor} (2.\,5.\,1860 Budapest – 3.\,7.\,1904 Edlach), \emph{Schriftsteller, Journalist}!neue Ghetto. Schauspiel in vier Acten@\strich\emph{Das neue Ghetto. Schauspiel in vier Acten}|pwv} ſehr raſch leſen – aber mit der ſtillen Hoffnung, ein{ }ſchlechtes zu
               finden. Sie wollen aber vor allem erreichen, daſs ſie aufmerkſam werden – das dürfte
               gelingen.– Mit \textsc{Sch.\pwindex{Schik, Friedrich *~6.\,9.\,1857 Wien@\textsc{Schik, Friedrich} (*~6.\,9.\,1857 Wien), \emph{Notar, Journalist, Dramaturg}|pw}} ſprach ich; {\pb}er iſt geneigt. Die Privatadreſſe iſt
                  \textsc{III. Reisnerstr. 25\oindex{Wien@\textbf{Wien}!III., Landstraße@\textbf{III., Landstraße}!Reisnerstraße 35@\textbf{Reisnerstraße 35}, \emph{Wohngebäude}|pwv}\oindex{Wien@\textbf{Wien}!III., Landstraße@\textbf{III., Landstraße}!Reisnerstraße 25@\textbf{Reisnerstraße 25}, \emph{Wohngebäude}|pw}}. Er iſt in der Kanzlei\oindex{Wien@\textbf{Wien}!I., Innere Stadt@\textbf{I., Innere Stadt}!Schulhof 6@\textbf{Schulhof 6}, \emph{Gebäude}|pwv}
               seines Vaters\pwindex{Schik, Josef Anton 1813/1814 – 6.\,10.\,1898 Wien@\textsc{Schik, Josef Anton} (1813/1814 – 6.\,10.\,1898 Wien), \emph{Notar}|pwv} beschäftigt –
               (deren Adreſſe neulich erſt gewechſelt hat und mir augenblicklich entfallen ist) –
               vielleicht iſt aber die Privatadreſſe vorzuziehen?– \textsc{Sch.\pwindex{Schik, Friedrich *~6.\,9.\,1857 Wien@\textsc{Schik, Friedrich} (*~6.\,9.\,1857 Wien), \emph{Notar, Journalist, Dramaturg}|pw}} gegenüber ſprach ich zur größern Vorſicht von einem in Berlin\oindex{Berlin@\textbf{Berlin}, \emph{Hauptstadt}|pw} anſäſſigen Autor. Ich hoffe Ihnen nun das abgeſchriebene
                  Manuscr\pwindex{Herzl, Theodor 2.\,5.\,1860 Budapest – 3.\,7.\,1904 Edlach@\textsc{Herzl, Theodor} (2.\,5.\,1860 Budapest – 3.\,7.\,1904 Edlach), \emph{Schriftsteller, Journalist}!neue Ghetto. Schauspiel in vier Acten@\strich\emph{Das neue Ghetto. Schauspiel in vier Acten}|pwv}. bald ſenden zu
               können; nicht wahr? –\pend
           
\pstart
           Ich kann dieſen Brief nicht ſchließen, ohne Ihnen für Ihr köſtliches \label{K_L03911-45v}\edtext{Feu{[}i{]}lleton\pwindex{Herzl, Theodor 2.\,5.\,1860 Budapest – 3.\,7.\,1904 Edlach@\textsc{Herzl, Theodor} (2.\,5.\,1860 Budapest – 3.\,7.\,1904 Edlach), \emph{Schriftsteller, Journalist}!Palais Bourbon. IV. Die Apotheke von Roubaix@\strich\emph{Das Palais Bourbon. IV. Die Apotheke von Roubaix}|pwv}}{\lemma{\textnormal{\emph{Feuilleton}}}\Cendnote{\textnormal{Theodor Herzl\pwindex{Herzl, Theodor 2.\,5.\,1860 Budapest – 3.\,7.\,1904 Edlach@\textsc{Herzl, Theodor} (2.\,5.\,1860 Budapest – 3.\,7.\,1904 Edlach), \emph{Schriftsteller, Journalist}|pwk}:
                        \emph{Das Palais Bourbon. IV. Die Apotheke von
                        Roubaix}\pwindex{Herzl, Theodor 2.\,5.\,1860 Budapest – 3.\,7.\,1904 Edlach@\textsc{Herzl, Theodor} (2.\,5.\,1860 Budapest – 3.\,7.\,1904 Edlach), \emph{Schriftsteller, Journalist}!Palais Bourbon. IV. Die Apotheke von Roubaix@\strich\emph{Das Palais Bourbon. IV. Die Apotheke von Roubaix}|pwk}. In: \emph{Neue Freie Presse}\pwindex{Neue Freie Presse@\emph{Neue Freie Presse}|pwk},
                     Nr. 10.874, 30. 11. 1894, Morgenblatt,
                     S. 1–4.}}}\label{K_L03911-45}{ }{\pb}die Hand zu drücken. Sie haben übrigens in den letzten
               Jahren kaum eines veröffentlicht, wo ich dieſes Bedürfnis nicht gehabt hätte. Man hat
               da ſo eine gewiſſe dumme Scheu. Und da fügt es ſich heute gut, daſs ich Ihnen über
               verſchiedenes andre{ }ſchreiben mußte und nur ſo beiläufig hinzufügen kann, daſs ich
               auf das Buch warte, in welchem ich dieſe kleinen Kunſtwerke geſa{\geminationm}elt finden werde. Wenn {\pb}ichs
               nicht von Ihnen geſchickt bekomme, ſo werd ich’s mir kaufen – was ich mit der \textsc{Glosse\pwindex{Herzl, Theodor 2.\,5.\,1860 Budapest – 3.\,7.\,1904 Edlach@\textsc{Herzl, Theodor} (2.\,5.\,1860 Budapest – 3.\,7.\,1904 Edlach), \emph{Schriftsteller, Journalist}!Glosse. Lustspiel in einem Act@\strich\emph{Die Glosse. Lustspiel in einem Act}|pw}} noch i{\geminationm}er nicht gethan habe. Und Sie können
               verſichert ſein – es iſt nicht wegen der ſechzig Kreuzer! –\pend
           
\pstart
           Herzlich Ihr ſehr ergebener{\\[\baselineskip]}\spacefill\mbox{Arthur Schnitzler}\pend
           \leftskip=0em{}
\pstart
           Wien\oindex{Wien@\textbf{Wien}, \emph{Verwaltungsgebiet}|pw}, 30. Nov. 94.\pend
           \selectlanguage{ngerman}\endnumbering\briefempfaengerindex{Herzl, Theodor@\textsc{Herzl, Theodor}!zzzSchnitzler, Arthur@\emph{von Arthur Schnitzler}!1894-11-301@{30. 11. 1894}|)be}\mylabel{L03911h}
\begin{anhang}
\end{anhang}\newcommand{\dateiname}{L03911}\newcommand{\titel}{Arthur Schnitzler an Theodor Herzl, 30. 11. 1894}\newcommand{\editorInnen}{Herausgegeben von Jahnke, SelmaMüller, Martin Anton}%% latex-leseansicht-abspann.tex
%% Abspann für die Leseansicht.
%% Der Schalter \ifkorrekturansicht ist bereits durch den Vorspann gesetzt.

%% latex-abspann.tex
%% Gemeinsamer Abspann für Korrekturansicht und Leseansicht.
%% Setzt den Schalter \ifkorrekturansicht voraus (gesetzt in den
%% einbindenden Dateien latex-korrekturansicht-abspann.tex bzw.
%% latex-leseansicht-abspann.tex).
%% ---------------------------------------------------------------

\normalsize

% Das esempio-Environment wird nur in der Leseansicht benötigt
\ifkorrekturansicht\else
\newenvironment{esempio}[3]%
{
    \vspace{1.5ex}
    \rlap{\underline{#1}}
    \par
    \setlength{\parindent}{0cm}
    \nopagebreak
    \leftskip=#2cm
    \rightskip=#3cm
}
{
    \par
}
\fi

\doendnotes{C}
\bigskip
\vfill

\clearpage

\footnotesize

\ifkorrekturansicht
  \lohead{\textsc{register}}
\fi

% theindex-Environment neu definieren ohne reledmac
\makeatletter
\renewenvironment{theindex}{%
  \ifkorrekturansicht
    \section*{\indexname}%
  \else
    \subsubsection*{Index der erwähnten Entitäten}%
  \fi
  \setlength{\parindent}{0pt}%
  \setlength{\parskip}{0pt plus 0.3pt}%
  \let\item\@idxitem
}{%
  \ifkorrekturansicht\clearpage\fi
}
\makeatother

\IfFileExists{\jobname-pw.ind}{\input{\jobname-pw.ind}}{}

% Quellenangabe nur in der Leseansicht
\ifkorrekturansicht\else
% Fallback-Definitionen, falls die .tex-Datei \titel etc. nicht gesetzt hat
\providecommand{\titel}{}
\providecommand{\editorInnen}{}
\providecommand{\dateiname}{\jobname}

\vspace{3cm}

\vfill

\footnotesize
\textsc{Quelle}: \titel. Herausgegeben von {\editorInnen}. In: \emph{Arthur Schnitzler: Briefwechsel mit Autorinnen und Autoren}.
 Digitale Edition, https://schnitzler-briefe.acdh.oeaw.ac.at/{\dateiname}.html (Stand \today)
\fi

\end{document}


