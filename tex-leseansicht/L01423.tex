%% latex-leseansicht-vorspann.tex
%% Vorspann für die Leseansicht.
%% Lädt die gemeinsame Datei latex-vorspann.tex mit nicht gesetztem Schalter.

\newif\ifkorrekturansicht
\korrekturansichtfalse

\input{../tex-inputs/latex-vorspann}


\section[Hugo von Hofmannsthal an Arthur Schnitzler, 8. 8. 1904]{L01423 Hugo von Hofmannsthal an Arthur Schnitzler, 8. 8. 1904}
\nopagebreak\mylabel{L01423v}
\rehead{ }\normalsize\beginnumbering\briefempfaengerindex{Schnitzler, Arthur@\textsc{Schnitzler, Arthur}!zzzHofmannsthal, Hugo von@\emph{von Hugo von Hofmannsthal}!1904-08-082@{8. 8. 1904}|(be}
\toendnotes[C]{\smallbreak\pagebreak[2]}
\correspDesc{Versand  durch Hugo von Hofmannsthal am 8. 8. 1904 in Bad Aussee
\newline{}Erhalt  durch Arthur Schnitzler im Zeitraum [9. 8. 1904
                  – 13. 8. 1904?] in Wien}\toendnotes[C]{\smallbreak}
\Standort{CUL, Schnitzler, B 43b/1.}
\physDesc{Brief, 1 Blatt, 3 Seiten, 726 Zeichen
\newline{}Handschrift: schwarze Tinte, deutsche Kurrent
\newline{}Ordnung: 1) mit Bleistift von unbekannter Hand nummeriert: »\strikeout{233}«  2) mit Bleistift von unbekannter Hand nummeriert:
                                    »231«}
\buchAbdrucke{\weitereDrucke{1) Hugo von Hofmannsthal, Arthur Schnitzler: \emph{Briefwechsel}. Herausgegeben von Therese Nickl und Heinrich Schnitzler. Frankfurt am Main: \emph{S. Fischer} 1964, S. 194.} \weitereDrucke{2) Hermann Bahr, Arthur Schnitzler: \emph{Briefwechsel, Aufzeichnungen, Dokumente (1891–1931)}. Herausgegeben von Kurt Ifkovits und Martin Anton Müller. Göttingen: \emph{Wallstein} 2018, S. 313.} }
\pstart
           
\pstart
           {\pb}\textsc{Markt Aussee, \uline{Ramgut}\oindex{Ramgut@\textbf{Ramgut}, \emph{Schloss}|pw}.}\pend
           
\pstart
           \raggedleft{}8 VIII 1904.\pend
           \pend
           \vspace{0.5em}
\pstart
           lieber, wir beko{\geminationm}en aus St. Veit\oindex{Wien@\textbf{Wien}!XIII., Hietzing@\textbf{XIII., Hietzing}!Sankt Veit@\textbf{Sankt Veit}|pw} von Bahr\pwindex{Bahr, Hermann 19.\,7.\,1863 Linz – 15.\,1.\,1934 München@\textsc{Bahr, Hermann} (19.\,7.\,1863 Linz – 15.\,1.\,1934 München), \emph{Schriftsteller, Kritiker}|pw} der durch Monate in der beſten Verfaſſung war, auf einmal{ }ſehr{ }ſchlimme
                  Briefe.\hspace*{1.5em}Es{ }ſcheint eine – hoffentlich nicht zu{ }ſchwere – objective Verſchlimmerung{ }ſeines Befindens zuſa{\geminationm}enzufallen mit einer{ }ſchweren nach langer guter Arbeitszeit {\pb}einfallenden Depreſſion. Wir{ }ſind{ }ſehr ängſtlich. Bitte{ }ſuchen Sie ihn baldigſt auf, ohne dieſen Brief zu erwähnen, und
               ohne daſs er \substVorne{}\textsuperscript{ſ}\substDazwischen{}S\substHinten{}ie einlädt: denn je{ }ſchlimmer ihm iſt, deſto mehr{ }ſchließt er{ }ſich gern ab,
               und{ }ſchreiben mir dann ein Wort.\pend
           
\pstart
           Ich bin bis heute noch nicht verſtändigt ob ich am {\pb}14\textsuperscript{ten}
               einzurücken habe oder dispenſiert bin und hier bleiben kann.\hspace*{1.5em}Sobald es entſchieden iſt,{ }ſchreib ich wieder.\pend
           
\pstart
           Herzlich Ihr{\\[\baselineskip]}\spacefill\mbox{Hugo.}\pend
           \leftskip=0em{}\selectlanguage{ngerman}\endnumbering\briefempfaengerindex{Schnitzler, Arthur@\textsc{Schnitzler, Arthur}!zzzHofmannsthal, Hugo von@\emph{von Hugo von Hofmannsthal}!1904-08-082@{8. 8. 1904}|)be}\mylabel{L01423h}  \newcommand{\dateiname}{L01423}\newcommand{\titel}{Hugo von Hofmannsthal an Arthur Schnitzler, 8. 8. 1904}\newcommand{\editorInnen}{Herausgegeben von Martin Anton Müller}%% latex-leseansicht-abspann.tex
%% Abspann für die Leseansicht.
%% Der Schalter \ifkorrekturansicht ist bereits durch den Vorspann gesetzt.

%% latex-abspann.tex
%% Gemeinsamer Abspann für Korrekturansicht und Leseansicht.
%% Setzt den Schalter \ifkorrekturansicht voraus (gesetzt in den
%% einbindenden Dateien latex-korrekturansicht-abspann.tex bzw.
%% latex-leseansicht-abspann.tex).
%% ---------------------------------------------------------------

\normalsize

% Das esempio-Environment wird nur in der Leseansicht benötigt
\ifkorrekturansicht\else
\newenvironment{esempio}[3]%
{
    \vspace{1.5ex}
    \rlap{\underline{#1}}
    \par
    \setlength{\parindent}{0cm}
    \nopagebreak
    \leftskip=#2cm
    \rightskip=#3cm
}
{
    \par
}
\fi

\doendnotes{C}
\bigskip
\vfill

\clearpage

\footnotesize

\ifkorrekturansicht
  \lohead{\textsc{register}}
\fi

% theindex-Environment neu definieren ohne reledmac
\makeatletter
\renewenvironment{theindex}{%
  \ifkorrekturansicht
    \section*{\indexname}%
  \else
    \subsubsection*{Index der erwähnten Entitäten}%
  \fi
  \setlength{\parindent}{0pt}%
  \setlength{\parskip}{0pt plus 0.3pt}%
  \let\item\@idxitem
}{%
  \ifkorrekturansicht\clearpage\fi
}
\makeatother

\IfFileExists{\jobname-pw.ind}{\input{\jobname-pw.ind}}{}

% Quellenangabe nur in der Leseansicht
\ifkorrekturansicht\else
% Fallback-Definitionen, falls die .tex-Datei \titel etc. nicht gesetzt hat
\providecommand{\titel}{}
\providecommand{\editorInnen}{}
\providecommand{\dateiname}{\jobname}

\vspace{3cm}

\vfill

\footnotesize
\textsc{Quelle}: \titel. Herausgegeben von {\editorInnen}. In: \emph{Arthur Schnitzler: Briefwechsel mit Autorinnen und Autoren}.
 Digitale Edition, https://schnitzler-briefe.acdh.oeaw.ac.at/{\dateiname}.html (Stand \today)
\fi

\end{document}


