%% latex-korrekturansicht-vorspann.tex
%% Vorspann für die Korrekturansicht.
%% Lädt die gemeinsame Datei latex-vorspann.tex mit gesetztem Schalter.

\newif\ifkorrekturansicht
\korrekturansichttrue

\input{../tex-inputs/latex-vorspann}


\section[Richard Beer-Hofmann an Arthur Schnitzler, 14. 9. 1931]{L02548 Richard Beer-Hofmann an Arthur Schnitzler, 14. 9. 1931}
\nopagebreak\mylabel{L02548v}
\rehead{ }\normalsize\beginnumbering\briefempfaengerindex{Schnitzler, Arthur@\textsc{Schnitzler, Arthur}!zzzBeer-Hofmann, Richard@\emph{von Richard Beer-Hofmann}!1931-09-141@{14. 9. 1931}|(be}
\toendnotes[C]{\smallbreak\pagebreak[2]}\Standort{CUL, Schnitzler, B 8.}
\physDesc{Bildpostkarte, 305 Zeichen
\newline{}Handschrift: Bleistift, lateinische Kurrent
\newline{}Versand: 1) Stempel: »\nobreak{}Gebrauchet die heilkräftigen Solbäder Bad
                                       Ischls\nobreak{}«.   2) Stempel: »\nobreak{}\oindex{Bad Ischl@\textbf{Bad Ischl}, \emph{P.PPL}|pwk}Bad Isch\textcolor{gray}{l}, 15. IX. 31, 13\nobreak{}«. 
\newline{}Schnitzler: mit rotem Buntstift datiert: »13/9 31« 
\newline{}Ordnung: mit Bleistift von unbekannter Hand nummeriert:
                                    »278« }
\buchAbdrucke{\weitereDrucke{Arthur Schnitzler, Richard Beer-Hofmann: \emph{Briefwechsel 1891–1931}. Wien, Zürich: \emph{Europaverlag} 1992, S. 232.} }\toendnotes[C]{\smallbreak}\pstart{}{\pb}Herrn\pend{}\pstart{}D\textsuperscript{r} Arthur Schnitzler\pend{}\pstart{}Wien XVIII.\oindex{XVIII., Waehring@\textbf{XVIII., Währing}, \emph{A.ADM3}|pw}\pend{}\pstart{}Sternwartstrasse 71\oindex{Sternwartestrasse 71@\textbf{Sternwartestraße 71}, \emph{Wohngebäude (K.WHS)}|pw}\pend{}{\bigskip}\selectlanguage{ngerman}
\pstart
           \noindent{}\centering{}\textcolor{gray}{\textbf{{\pb}Bad Ischl\oindex{Bad Ischl@\textbf{Bad Ischl}, \emph{P.PPL}|pw}}}\pend
           \vspace{1em}
\pstart
           \centering{}{\pb}14. IX. 31\pend
           \vspace{0.5em}
\pstart
           Wir sind seit 1 von Wien\oindex{Wien@\textbf{Wien}, \emph{A.ADM2}|pw} weg, seit
                  2. – hier und werden in 2–3 Tagen wieder in Wien\oindex{Wien@\textbf{Wien}, \emph{A.ADM2}|pw} sein. Wir haben 10 Tage schönes Wetter gehabt – müssen also
               zufrieden sein. Auf allen Wegen sind hier Erinnerungen, wehmütige, aber auch \label{T_L02548-1v}\edtext{schöne!}{\lemma{\textnormal{\emph{schöne!}}}\Cendnote{\textnormal{weiter quer am rechten Rand}}}\label{T_L02548-1}\pend
           \pstart Herzlichst Ihr\spacefill\mbox{Richard.}\pend{}\selectlanguage{ngerman}\endnumbering\briefempfaengerindex{Schnitzler, Arthur@\textsc{Schnitzler, Arthur}!zzzBeer-Hofmann, Richard@\emph{von Richard Beer-Hofmann}!1931-09-141@{14. 9. 1931}|)be}\mylabel{L02548h}  \normalsize

\doendnotes{C}
\bigskip
\vfill

\clearpage

\footnotesize

\lohead{\textsc{register}}

% Definiere theindex-Environment komplett neu ohne reledmac
\makeatletter
\renewenvironment{theindex}{%
  \section*{\indexname}%
  \setlength{\parindent}{0pt}%
  \setlength{\parskip}{0pt plus 0.3pt}%
  \let\item\@idxitem
}{%
  \clearpage
}
\makeatother

\IfFileExists{\jobname-pw.ind}{\input{\jobname-pw.ind}}{}

\end{document}

      