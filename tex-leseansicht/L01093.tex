%% latex-korrekturansicht-vorspann.tex
%% Vorspann für die Korrekturansicht.
%% Lädt die gemeinsame Datei latex-vorspann.tex mit gesetztem Schalter.

\newif\ifkorrekturansicht
\korrekturansichttrue

\input{../tex-inputs/latex-vorspann}


\section[Hermann Bahr an Arthur Schnitzler, 23. 1. {[}1901{]}]{L01093 Hermann Bahr an Arthur Schnitzler, 23. 1. {[}1901{]}}
\nopagebreak\mylabel{L01093v}
\rehead{ }\normalsize\beginnumbering\briefempfaengerindex{Schnitzler, Arthur@\textsc{Schnitzler, Arthur}!zzzBahr, Hermann@\emph{von Hermann Bahr}!1901-01-232@{23. 1. {[}1901{]}}|(be}
\toendnotes[C]{\smallbreak\pagebreak[2]}\Standort{CUL, Schnitzler, B 5b.}
\physDesc{Brief, 1 Blatt, 1 Seite, 403 Zeichen
\newline{}Handschrift: schwarze Tinte, deutsche Kurrent
\newline{}Schnitzler: mit Bleistift die Jahreszahl »901« ergänzt 
\newline{}Ordnung: mit Bleistift von unbekannter Hand nummeriert:
                                    »72« }
\buchAbdrucke{\weitereDrucke{Hermann Bahr, Arthur Schnitzler: \emph{Briefwechsel, Aufzeichnungen, Dokumente (1891–1931)}. Göttingen: \emph{Wallstein} 2018, S. 192.} }\toendnotes[C]{\smallbreak}
\pstart
           \centering{}{\pb}\textcolor{gray}{\textbf{Redaktion des Neuen Wiener Tagblatt\orgindex{Neues Wiener Tagblatt@Neues Wiener Tagblatt|pw}}}\pend
           
\pstart
           \centering{}\textcolor{gray}{\textbf{\textsc{Wien, I., Rothenturmstrasse,
                        Steyrerhof\oindex{Steyrerhof@\textbf{Steyrerhof}, \emph{Gebäude (K.GBD)}|pw}.}}}\pend
           
\pstart
           \centering{}\textcolor{gray}{\textbf{Telegramm-Adresse: Tagblatt\orgindex{Neues Wiener Tagblatt@Neues Wiener Tagblatt|pw}, Steyrerhof, Wien\oindex{Steyrerhof@\textbf{Steyrerhof}, \emph{Gebäude (K.GBD)}|pw}. –
                     Telephon Nr. 384. Staats-Telephon Nr. 36.}}\pend
           
\pstart
           \raggedleft{}23/1\pend
           
\pstart\center{}Lieber Arthur!\pend\vspace{0.5em}
\pstart
           Ich habe die »\label{K_L01093-1v}\edtext{Marionetten\pwindex{Zum grossen Wurstel. Burleske in einem Akt@\emph{Zum großen Wurstel. Burleske in einem Akt}|pw}}{\lemma{\textnormal{\emph{Marionetten}}}\Cendnote{\textnormal{Erste Fassung von \emph{Zum großen Wurstel}\pwindex{Zum grossen Wurstel. Burleske in einem Akt@\emph{Zum großen Wurstel. Burleske in einem Akt}|pwk}, die am 8. 3. 1901 von Wolzogens\pwindex{Wolzogen, Ernst von 23.04.1855 – 30.07.1934@\textsc{Wolzogen, Ernst von} (23.04.1855 – 30.07.1934), \emph{Schriftsteller/Schriftstellerin}|pwk}{ }Überbrettl\oindex{Ueberbrettl@\textbf{Überbrettl}, \emph{Kabarett (K.KBR)}|pwk} aufgeführt wurde. Erst in die Umarbeitung von
                     1905, die vor allem eine Erweiterung der illusionsbrechenden
                  Figuren vornahm, wurde die Hauptfigur von Bahrs\pwindex{Bahr, Hermann 19.07.1863 – 15.01.1934@\textsc{Bahr, Hermann} (19.07.1863 – 15.01.1934), \emph{Schriftsteller/Schriftstellerin, Kritiker/Kritikerin}|pwk}{ }\emph{Der Meister}\pwindex{Meister. Komoedie in drei Akten@\emph{Der Meister. Komödie in drei Akten}|pwk}
                  eingearbeitet.}}}\label{K_L01093-1}« gestern nachts ſogleich geleſen und mich diebiſch amüſiert.
               Sie ſind einfach großartig. Bei einer Vorleſung oder in einem kleinen Theater bürge
               ich für einen ſehr ſtarken Erfolg. Im Volkstheater\oindex{Volkstheater@\textbf{Volkstheater}, \emph{Theater (K.THE)}|pw}
               iſt allerdings der Raum dafür ſehr ekelhaft und noch ekelhafter ja unſere \label{K_L01093-2v}\edtext{Premièrenjuden}{\lemma{\textnormal{\emph{Premièrenjuden}}}\Cendnote{\textnormal{Vgl. Hermann Bahr, Arthur Schnitzler: \emph{Briefwechsel, Aufzeichnungen, Dokumente (1891–1931)}, Hermann Bahr: Tagebuch. 13. Oktober, 28. 10. 1905.
               }}}\label{K_L01093-2} – aber man muß es halt wagen. \textsc{Manuscript} in ein
               paar Tagen.\pend
           
\pstart
           Herzlichſt{\\[\baselineskip]}Dein{\\[\baselineskip]}\spacefill\mbox{Hermann}\pend
           \leftskip=0em{}\selectlanguage{ngerman}\endnumbering\briefempfaengerindex{Schnitzler, Arthur@\textsc{Schnitzler, Arthur}!zzzBahr, Hermann@\emph{von Hermann Bahr}!1901-01-232@{23. 1. {[}1901{]}}|)be}\mylabel{L01093h}  \normalsize

\doendnotes{C}
\bigskip
\vfill

\clearpage

\footnotesize

\lohead{\textsc{register}}

% Definiere theindex-Environment komplett neu ohne reledmac
\makeatletter
\renewenvironment{theindex}{%
  \section*{\indexname}%
  \setlength{\parindent}{0pt}%
  \setlength{\parskip}{0pt plus 0.3pt}%
  \let\item\@idxitem
}{%
  \clearpage
}
\makeatother

\IfFileExists{\jobname-pw.ind}{\input{\jobname-pw.ind}}{}

\end{document}

      