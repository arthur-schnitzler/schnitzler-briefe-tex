%% latex-leseansicht-vorspann.tex
%% Vorspann für die Leseansicht.
%% Lädt die gemeinsame Datei latex-vorspann.tex mit nicht gesetztem Schalter.

\newif\ifkorrekturansicht
\korrekturansichtfalse

\input{../tex-inputs/latex-vorspann}


\section[Stefan Großmann an Arthur Schnitzler, 25. 5. 1926]{L02475 Stefan Großmann an Arthur Schnitzler, 25. 5. 1926}
\nopagebreak\mylabel{L02475v}
\rehead{ }\normalsize\beginnumbering\briefempfaengerindex{Schnitzler, Arthur@\textsc{Schnitzler, Arthur}!zzzGroßmann, Stefan@\emph{von Stefan Großmann}!1926-05-251@{25. 5. 1926}|(be}
\toendnotes[C]{\smallbreak\pagebreak[2]}
\correspDesc{Versand  durch Stefan Großmann am 25. 5. 1926 in Berlin
\newline{}Erhalt  durch Arthur Schnitzler im Zeitraum [26. 5. 1926
                  – 30. 5. 1926?] in Wien}\toendnotes[C]{\smallbreak}
\Standort{DLA, A:Schnitzler, HS.NZ85.1.3232.}
\physDesc{Brief, 1 Blatt, 2 Seiten, 1126 Zeichen
\newline{}Schreibmaschine
\newline{}Handschrift: roter Buntstift (\noindent{}Unterschrift, eine Streichung)
\newline{}Schnitzler: mit Bleistift handschriftliche, nicht verlässlich zu
                                 entziffernde Antwortskizze auf der 2. Seite }\toendnotes[C]{\smallbreak}
\pstart
           \centering{}{\pb}\textcolor{gray}{\textbf{Das Tage-Buch\orgindex{Tage-Buch@Das Tage-Buch|pw}}}\pend
           
\pstart
           \centering{}\textcolor{gray}{\textbf{\emph{Herausgeber: Stefan Großmann und Leopold Schwarzschild\pwindex{Schwarzschild, Leopold 8.\,12.\,1891 Frankfurt am Main – 2.\,10.\,1950 Santa Margherita Ligure@\textsc{Schwarzschild, Leopold} (8.\,12.\,1891 Frankfurt am Main – 2.\,10.\,1950 Santa Margherita Ligure), \emph{Publizist}|pw}}}}\pend
           
\pstart
           \centering{}\textcolor{gray}{\textbf{Tagebuchverlag m. b. H., Berlin SW 19\oindex{Berlin@\textbf{Berlin}, \emph{Hauptstadt}|pw}}}\pend
           
\pstart
           \centering{}\textcolor{gray}{\textbf{BEUTHSTRASSE 19\oindex{Beuthstrasse@\textbf{Beuthstrasse}, \emph{Straße}|pw}}}\pend
           
\pstart
           \centering{}\textcolor{gray}{\textbf{\emph{Telegramm-Adresse: Tagebuch Berlin\oindex{Berlin@\textbf{Berlin}, \emph{Hauptstadt}|pw} ⋅ Fernsprecher: Merkur 8790–8792}}}\pend
           
\pstart
           \centering{}\textcolor{gray}{\textbf{\emph{\so{Sprechstunde der Redaktion: 12–1 Uhr}}}}\pend
           
\pstart
           \centering{}\textcolor{gray}{\textbf{*}}\pend
           
\pstart
           Tgb./Gr./Schl.\pend
           
\pstart
           \raggedleft{}Berlin\oindex{Berlin@\textbf{Berlin}, \emph{Hauptstadt}|pw}, den
                  25. Mai 1926.\pend
           
\pstart
           \raggedleft{}Herrn\pend
           
\pstart
           \raggedleft{}Dr. Arthur \so{Schnitzler}\pend
           
\pstart
           \raggedleft{}\so{Wien} XVIII\oindex{XVIII., Währing@\textbf{XVIII., Währing}, \emph{Verwaltungsgebiet}|pw}\pend
           
\pstart
           \raggedleft{}Sternwartestr. 71.\oindex{Wien@\textbf{Wien}!XVIII., Währing@\textbf{XVIII., Währing}!Sternwartestraße 71@\textbf{Sternwartestraße 71}, \emph{Wohngebäude}|pw}\pend
           
\pstart\center{}Verehrter Herr Doktor Schnitzler!\pend\vspace{0.5em}
\pstart
           Mit der Beharrlichkeit eines unangenehmen Menschen und eines guten Redakteurs stelle
               ich mich wieder bei Ihnen ein.\pend
           
\pstart
           Ich las in der »Neuen Freien Presse\pwindex{Neue Freie Presse@\emph{Neue Freie Presse}|pw}« die \label{K_L02475-1v}\edtext{Bruchstücke\pwindex{Schnitzler, Arthur 15.\,5.\,1862 Wien – 21.\,10.\,1931 ebd.@\textsc{Schnitzler, Arthur} (15.\,5.\,1862 Wien – 21.\,10.\,1931 ebd.), \emph{Schriftsteller, Mediziner}!Bemerkungen. (Aus dem noch unveröffentlichten „Buch der Sprüche und Bedenken“.)@\strich\emph{Bemerkungen. (Aus dem noch unveröffentlichten „Buch der Sprüche und Bedenken“.)}|pwv}}{\lemma{\textnormal{\emph{Bruchstücke}}}\Cendnote{\textnormal{Arthur Schnitzler: \emph{Bemerkungen. (Aus dem noch unveröffentlichten »Buch der
                        Sprüche und Bedenken«.)}\pwindex{Schnitzler, Arthur 15.\,5.\,1862 Wien – 21.\,10.\,1931 ebd.@\textsc{Schnitzler, Arthur} (15.\,5.\,1862 Wien – 21.\,10.\,1931 ebd.), \emph{Schriftsteller, Mediziner}!Bemerkungen. (Aus dem noch unveröffentlichten „Buch der Sprüche und Bedenken“.)@\strich\emph{Bemerkungen. (Aus dem noch unveröffentlichten „Buch der Sprüche und Bedenken“.)}|pwk} In: \emph{Neue Freie
                        Presse}\pwindex{Neue Freie Presse@\emph{Neue Freie Presse}|pwk}, Nr. 22.158, 23. 5. 1923, S. 33.}}}\label{K_L02475-1} aus
               Ihrem unveröffentlichten Buch\pwindex{Schnitzler, Arthur 15.\,5.\,1862 Wien – 21.\,10.\,1931 ebd.@\textsc{Schnitzler, Arthur} (15.\,5.\,1862 Wien – 21.\,10.\,1931 ebd.), \emph{Schriftsteller, Mediziner}!Buch der Sprüche und Bedenken@\strich\emph{Buch der Sprüche und Bedenken}|pwv},
               die Sie in der Pfingstnummer publizieren liessen und dachte dabei, das sind doch
               eigentlich die Beiträge, um die ich Arthur Schnitzler seit langem bedränge, und die
               er mir prinzipiell \strikeout{eigentlich} zugesagt hat. Es ist in
               diesen Bemerkungen\pwindex{Schnitzler, Arthur 15.\,5.\,1862 Wien – 21.\,10.\,1931 ebd.@\textsc{Schnitzler, Arthur} (15.\,5.\,1862 Wien – 21.\,10.\,1931 ebd.), \emph{Schriftsteller, Mediziner}!Bemerkungen. (Aus dem noch unveröffentlichten „Buch der Sprüche und Bedenken“.)@\strich\emph{Bemerkungen. (Aus dem noch unveröffentlichten „Buch der Sprüche und Bedenken“.)}|pw} so viel Weisheit und so viel
               verborgener Humor enthalten, dass Sie mir sicher gestatten werden, dass ich vorerst
               diese Bemerkungen\pwindex{Schnitzler, Arthur 15.\,5.\,1862 Wien – 21.\,10.\,1931 ebd.@\textsc{Schnitzler, Arthur} (15.\,5.\,1862 Wien – 21.\,10.\,1931 ebd.), \emph{Schriftsteller, Mediziner}!Bemerkungen. (Aus dem noch unveröffentlichten „Buch der Sprüche und Bedenken“.)@\strich\emph{Bemerkungen. (Aus dem noch unveröffentlichten „Buch der Sprüche und Bedenken“.)}|pw} im TAGE-BUCH\pwindex{Tage-Buch@\emph{Das Tage-Buch}|pw}{ }\label{K_L02475-2v}\edtext{nachdrucke}{\lemma{\textnormal{\emph{nachdrucke}}}\Cendnote{\textnormal{Arthur Schnitzler: \emph{Bemerkungen}\pwindex{Bemerkungen@\emph{Bemerkungen}|pwk}. In: \emph{Das
                        Tage-Buch}\pwindex{Tage-Buch@\emph{Das Tage-Buch}|pwk}, Jg. 7, H. 22, 29. 5. 1926,
                  S. 747–748.}}}\label{K_L02475-2}. Die »Neue Freie
                  Presse\pwindex{Neue Freie Presse@\emph{Neue Freie Presse}|pw}« wird ja seit dem Umsturz in Deutschland\oindex{Deutschland@\textbf{Deutschland}|pw} wenig gelesen und selbst wenn es ein Leser ein zweites Mal im
                  TAGE-BUCH\pwindex{Tage-Buch@\emph{Das Tage-Buch}|pw} findet, so kann {\pb}er aus einem zweiten Lesen nur
               noch weiteren Gewinn ziehen.\pend
           
\pstart
           Ich wäre sehr glücklich, wenn Sie mir aus diesen unveröffentlichten Reichtümern noch
               einiges anderes zum Erstdruck anvertrauen wollten und begrüsse Sie in dieser Hoffnung
               als\pend
           
\pstart
           Ihr dankbar ergebener{\\[\baselineskip]}\spacefill\mbox{{[}hs.:{]} Stefan Großmann}\pend
           \leftskip=0em{}\selectlanguage{ngerman}\endnumbering\briefempfaengerindex{Schnitzler, Arthur@\textsc{Schnitzler, Arthur}!zzzGroßmann, Stefan@\emph{von Stefan Großmann}!1926-05-251@{25. 5. 1926}|)be}\mylabel{L02475h}  \newcommand{\dateiname}{L02475}\newcommand{\titel}{Stefan Großmann an Arthur Schnitzler, 25. 5. 1926}\newcommand{\editorInnen}{Herausgegeben von Martin Anton Müller}%% latex-leseansicht-abspann.tex
%% Abspann für die Leseansicht.
%% Der Schalter \ifkorrekturansicht ist bereits durch den Vorspann gesetzt.

%% latex-abspann.tex
%% Gemeinsamer Abspann für Korrekturansicht und Leseansicht.
%% Setzt den Schalter \ifkorrekturansicht voraus (gesetzt in den
%% einbindenden Dateien latex-korrekturansicht-abspann.tex bzw.
%% latex-leseansicht-abspann.tex).
%% ---------------------------------------------------------------

\normalsize

% Das esempio-Environment wird nur in der Leseansicht benötigt
\ifkorrekturansicht\else
\newenvironment{esempio}[3]%
{
    \vspace{1.5ex}
    \rlap{\underline{#1}}
    \par
    \setlength{\parindent}{0cm}
    \nopagebreak
    \leftskip=#2cm
    \rightskip=#3cm
}
{
    \par
}
\fi

\doendnotes{C}
\bigskip
\vfill

\clearpage

\footnotesize

\ifkorrekturansicht
  \lohead{\textsc{register}}
\fi

% theindex-Environment neu definieren ohne reledmac
\makeatletter
\renewenvironment{theindex}{%
  \ifkorrekturansicht
    \section*{\indexname}%
  \else
    \subsubsection*{Index der erwähnten Entitäten}%
  \fi
  \setlength{\parindent}{0pt}%
  \setlength{\parskip}{0pt plus 0.3pt}%
  \let\item\@idxitem
}{%
  \ifkorrekturansicht\clearpage\fi
}
\makeatother

\IfFileExists{\jobname-pw.ind}{\input{\jobname-pw.ind}}{}

% Quellenangabe nur in der Leseansicht
\ifkorrekturansicht\else
% Fallback-Definitionen, falls die .tex-Datei \titel etc. nicht gesetzt hat
\providecommand{\titel}{}
\providecommand{\editorInnen}{}
\providecommand{\dateiname}{\jobname}

\vspace{3cm}

\vfill

\footnotesize
\textsc{Quelle}: \titel. Herausgegeben von {\editorInnen}. In: \emph{Arthur Schnitzler: Briefwechsel mit Autorinnen und Autoren}.
 Digitale Edition, https://schnitzler-briefe.acdh.oeaw.ac.at/{\dateiname}.html (Stand \today)
\fi

\end{document}


