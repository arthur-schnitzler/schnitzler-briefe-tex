%% latex-leseansicht-vorspann.tex
%% Vorspann für die Leseansicht.
%% Lädt die gemeinsame Datei latex-vorspann.tex mit nicht gesetztem Schalter.

\newif\ifkorrekturansicht
\korrekturansichtfalse

\input{../tex-inputs/latex-vorspann}


         
         \renewcommand{\erwaehntePersonen}{Personen: Leopold Schwarzschild}
         \renewcommand{\erwaehnteInstitutionen}{Institutionen: Das Tage-Buch}
         \renewcommand{\erwaehnteOrte}{Orte: Berlin, Beuthstrasse, Deutschland, Sternwartestraße, Wien, XVIII., Währing}
         \renewcommand{\erwaehnteWerke}{Werke: Bemerkungen, Bemerkungen. (Aus dem noch unveröffentlichten „Buch der Sprüche und Bedenken“.), Buch der Sprüche und Bedenken, Das Tage-Buch, Neue Freie Presse}
               \section[Stefan Großmann an Arthur Schnitzler, 25. 5. 1926]{ Stefan Großmann an Arthur Schnitzler, 25. 5. 1926}\nopagebreak\mylabel{v}\rehead{ }\begin{ledgroupsized}[t]{13cm}\normalsize\beginnumbering \toendnotes[C]{\smallbreak\pagebreak[2]} \Standort{DLA, A:Schnitzler, HS.NZ85.1.3232.}
\physDesc{Brief, 1 Blatt, 2 Seiten, 1126 Zeichen
\newline{}Schreibmaschine
\newline{}Handschrift: roter Buntstift (\noindent{}Unterschrift, eine Streichung)
\newline{}Schnitzler: mit Bleistift handschriftliche, nicht verlässlich zu
                                 entziffernde Antwortskizze auf der 2. Seite }\toendnotes[C]{\smallbreak}\pstart
           \noindent{}\centering{}{\pb}\textcolor{gray}{\textbf{Das Tage-Buch\orgindex{Tage-Buch@Das Tage-Buch|pw}}}\pend
           \pstart
           \noindent{}\centering{}\textcolor{gray}{\textbf{\emph{Herausgeber: Stefan Großmann und Leopold Schwarzschild\pwindex{Schwarzschild, Leopold 1891-12-08 – 1950-10-02@\textsc{Schwarzschild, Leopold} (1891-12-08 – 1950-10-02), \emph{Publizist}|pw}}}}\pend
           \pstart
           \noindent{}\centering{}\textcolor{gray}{\textbf{Tagebuchverlag m. b. H., Berlin SW 19\oindex{Berlin@\textbf{Berlin}|pw}}}\pend
           \pstart
           \noindent{}\centering{}\textcolor{gray}{\textbf{BEUTHSTRASSE 19\oindex{Beuthstrasse@\textbf{Beuthstrasse}|pw}}}\pend
           \pstart
           \noindent{}\centering{}\textcolor{gray}{\textbf{\emph{Telegramm-Adresse: Tagebuch Berlin\oindex{Berlin@\textbf{Berlin}|pw} ⋅ Fernsprecher: Merkur 8790–8792}}}\pend
           \pstart
           \noindent{}\centering{}\textcolor{gray}{\textbf{\emph{\so{Sprechstunde der Redaktion: 12–1 Uhr}}}}\pend
           \pstart
           \noindent{}\centering{}\textcolor{gray}{\textbf{*}}\pend
           \pstart
           \noindent{}Tgb./Gr./Schl.\pend
           \pstart
           \raggedleft{}Berlin\oindex{Berlin@\textbf{Berlin}|pw}, den
                  25. Mai 1926.\pend
           \pstart
           \raggedleft{}Herrn\pend
           \pstart
           \noindent{}\raggedleft{}Dr. Arthur \so{Schnitzler}\pend
           \pstart
           \noindent{}\raggedleft{}\so{Wien} XVIII\oindex{XVIII., Waehring@\textbf{XVIII., Währing}|pw}\pend
           \pstart
           \noindent{}\raggedleft{}Sternwartestr. 71.\oindex{Sternwartestrasse@\textbf{Sternwartestraße}|pw}\pend
           \pstart\center{}Verehrter Herr Doktor Schnitzler!\pend\pstart
           Mit der Beharrlichkeit eines unangenehmen Menschen und eines guten Redakteurs stelle
               ich mich wieder bei Ihnen ein.\pend
           \pstart
           Ich las in der »Neuen Freien Presse\pwindex{Neue Freie Presse1864 – 1939@\emph{Neue Freie Presse} {[}1864 – 1939{]}|pw}« die \label{K_L02475_1v}\edtext{Bruchstücke\pwindex{Schnitzler, Arthur 15.05.1862 – 21.10.1931@\textsc{Schnitzler, Arthur} (15.05.1862 – 21.10.1931), \emph{Schriftsteller, Mediziner}!Bemerkungen. (Aus dem noch unveroeffentlichten „Buch der Sprueche und
                  Bedenken“.)1926-05-23@\strich\emph{Bemerkungen. (Aus dem noch unveröffentlichten „Buch der Sprüche und Bedenken“.)} {[}1926-05-23{]}|pwv}}{\lemma{\textnormal{\emph{Bruchstücke}}}\Cendnote{\textnormal{Arthur Schnitzler\pwindex{Schnitzler, Arthur 15.05.1862 – 21.10.1931@\textsc{Schnitzler, Arthur} (15.05.1862 – 21.10.1931), \emph{Schriftsteller, Mediziner}|pwk}: \emph{Bemerkungen. (Aus dem noch unveröffentlichten »Buch der
                        Sprüche und Bedenken«.)}\pwindex{Schnitzler, Arthur 15.05.1862 – 21.10.1931@\textsc{Schnitzler, Arthur} (15.05.1862 – 21.10.1931), \emph{Schriftsteller, Mediziner}!Bemerkungen. (Aus dem noch unveroeffentlichten „Buch der Sprueche und
                  Bedenken“.)1926-05-23@\strich\emph{Bemerkungen. (Aus dem noch unveröffentlichten „Buch der Sprüche und Bedenken“.)} {[}1926-05-23{]}|pwk} In: \emph{Neue Freie
                        Presse}\pwindex{Neue Freie Presse1864 – 1939@\emph{Neue Freie Presse} {[}1864 – 1939{]}|pwk}, Nr. 22158, 23. 5. 1923, S. 33.}}}\label{K_L02475_1h} aus
               Ihrem unveröffentlichten Buch\pwindex{Schnitzler, Arthur 15.05.1862 – 21.10.1931@\textsc{Schnitzler, Arthur} (15.05.1862 – 21.10.1931), \emph{Schriftsteller, Mediziner}!Buch der Sprueche und Bedenken1927@\strich\emph{Buch der Sprüche und Bedenken} {[}1927{]}|pwv},
               die Sie in der Pfingstnummer publizieren liessen und dachte dabei, das sind doch
               eigentlich die Beiträge, um die ich Arthur Schnitzler seit langem bedränge, und die
               er mir prinzipiell \strikeout{eigentlich} zugesagt hat. Es ist in
               diesen Bemerkungen\pwindex{Schnitzler, Arthur 15.05.1862 – 21.10.1931@\textsc{Schnitzler, Arthur} (15.05.1862 – 21.10.1931), \emph{Schriftsteller, Mediziner}!Bemerkungen. (Aus dem noch unveroeffentlichten „Buch der Sprueche und
                  Bedenken“.)1926-05-23@\strich\emph{Bemerkungen. (Aus dem noch unveröffentlichten „Buch der Sprüche und Bedenken“.)} {[}1926-05-23{]}|pw} so viel Weisheit und so viel
               verborgener Humor enthalten, dass Sie mir sicher gestatten werden, dass ich vorerst
               diese Bemerkungen\pwindex{Schnitzler, Arthur 15.05.1862 – 21.10.1931@\textsc{Schnitzler, Arthur} (15.05.1862 – 21.10.1931), \emph{Schriftsteller, Mediziner}!Bemerkungen. (Aus dem noch unveroeffentlichten „Buch der Sprueche und
                  Bedenken“.)1926-05-23@\strich\emph{Bemerkungen. (Aus dem noch unveröffentlichten „Buch der Sprüche und Bedenken“.)} {[}1926-05-23{]}|pw} im TAGE-BUCH\pwindex{Tage-Buch1920-01-01 – 1933-01-01@\emph{Das Tage-Buch} {[}1920-01-01 – 1933-01-01{]}|pw}{ }\label{K_L02475_2v}\edtext{nachdrucke}{\lemma{\textnormal{\emph{nachdrucke}}}\Cendnote{\textnormal{Arthur Schnitzler\pwindex{Schnitzler, Arthur 15.05.1862 – 21.10.1931@\textsc{Schnitzler, Arthur} (15.05.1862 – 21.10.1931), \emph{Schriftsteller, Mediziner}|pwk}: \emph{Bemerkungen}\pwindex{?? Werk@Nicht ermittelte Verfasserinnen und Verfasser!Bemerkungen1926-05-29@\emph{Bemerkungen} {[}1926-05-29{]}|pwk}. In: \emph{Das
                        Tage-Buch}\pwindex{Tage-Buch1920-01-01 – 1933-01-01@\emph{Das Tage-Buch} {[}1920-01-01 – 1933-01-01{]}|pwk}, Jg. 7, H. 22, 29. 5. 1926,
                  S. 747–748.}}}\label{K_L02475_2h}. Die »Neue Freie
                  Presse\pwindex{Neue Freie Presse1864 – 1939@\emph{Neue Freie Presse} {[}1864 – 1939{]}|pw}« wird ja seit dem Umsturz in Deutschland\oindex{Deutschland@\textbf{Deutschland}|pw} wenig gelesen und selbst wenn es ein Leser ein zweites Mal im
                  TAGE-BUCH\pwindex{Tage-Buch1920-01-01 – 1933-01-01@\emph{Das Tage-Buch} {[}1920-01-01 – 1933-01-01{]}|pw} findet, so kann {\pb}er aus einem zweiten Lesen nur
               noch weiteren Gewinn ziehen.\pend
           \pstart
           Ich wäre sehr glücklich, wenn Sie mir aus diesen unveröffentlichten Reichtümern noch
               einiges anderes zum Erstdruck anvertrauen wollten und begrüsse Sie in dieser Hoffnung
               als\pend
           \pstart
           Ihr dankbar ergebener{\\[\baselineskip]}\spacefill\mbox{{[}hs.:{]} Stefan Großmann}\pend
           \leftskip=0em{}
         
         \endnumbering\mylabel{h}\end{ledgroupsized}  \newcommand{\dateiname}{L02475}\newcommand{\titel}{Stefan Großmann an Arthur Schnitzler, 25. 5. 1926}\newcommand{\editorInnen}{ Martin Anton Müller und Gerd-Hermann Susen}%% latex-leseansicht-abspann.tex
%% Abspann für die Leseansicht.
%% Der Schalter \ifkorrekturansicht ist bereits durch den Vorspann gesetzt.

%% latex-abspann.tex
%% Gemeinsamer Abspann für Korrekturansicht und Leseansicht.
%% Setzt den Schalter \ifkorrekturansicht voraus (gesetzt in den
%% einbindenden Dateien latex-korrekturansicht-abspann.tex bzw.
%% latex-leseansicht-abspann.tex).
%% ---------------------------------------------------------------

\normalsize

% Das esempio-Environment wird nur in der Leseansicht benötigt
\ifkorrekturansicht\else
\newenvironment{esempio}[3]%
{
    \vspace{1.5ex}
    \rlap{\underline{#1}}
    \par
    \setlength{\parindent}{0cm}
    \nopagebreak
    \leftskip=#2cm
    \rightskip=#3cm
}
{
    \par
}
\fi

\doendnotes{C}
\bigskip
\vfill

\clearpage

\footnotesize

\ifkorrekturansicht
  \lohead{\textsc{register}}
\fi

% theindex-Environment neu definieren ohne reledmac
\makeatletter
\renewenvironment{theindex}{%
  \ifkorrekturansicht
    \section*{\indexname}%
  \else
    \subsubsection*{Index der erwähnten Entitäten}%
  \fi
  \setlength{\parindent}{0pt}%
  \setlength{\parskip}{0pt plus 0.3pt}%
  \let\item\@idxitem
}{%
  \ifkorrekturansicht\clearpage\fi
}
\makeatother

\IfFileExists{\jobname-pw.ind}{\input{\jobname-pw.ind}}{}

% Quellenangabe nur in der Leseansicht
\ifkorrekturansicht\else
% Fallback-Definitionen, falls die .tex-Datei \titel etc. nicht gesetzt hat
\providecommand{\titel}{}
\providecommand{\editorInnen}{}
\providecommand{\dateiname}{\jobname}

\vspace{3cm}

\vfill

\footnotesize
\textsc{Quelle}: \titel. Herausgegeben von {\editorInnen}. In: \emph{Arthur Schnitzler: Briefwechsel mit Autorinnen und Autoren}.
 Digitale Edition, https://schnitzler-briefe.acdh.oeaw.ac.at/{\dateiname}.html (Stand \today)
\fi

\end{document}


      