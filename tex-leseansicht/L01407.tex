%% latex-leseansicht-vorspann.tex
%% Vorspann für die Leseansicht.
%% Lädt die gemeinsame Datei latex-vorspann.tex mit nicht gesetztem Schalter.

\newif\ifkorrekturansicht
\korrekturansichtfalse

\input{../tex-inputs/latex-vorspann}


\section[Arthur Schnitzler an Hugo von Hofmannsthal, {[}19.? 6. 1904{]}]{L01407 Arthur Schnitzler an Hugo von Hofmannsthal, {[}19.? 6. 1904{]}}
\nopagebreak\mylabel{L01407v}
\rehead{ }\normalsize\beginnumbering\briefempfaengerindex{Hofmannsthal, Hugo von@\textsc{Hofmannsthal, Hugo von}!zzzSchnitzler, Arthur@\emph{von Arthur Schnitzler}!1904-06-192@{{[}19.? 6. 1904{]}}|(be}
\toendnotes[C]{\smallbreak\pagebreak[2]}
\correspDesc{Versand  durch Arthur Schnitzler am [19.? 6. 1904] in Wien
\newline{}Erhalt  durch Hugo von Hofmannsthal im Zeitraum [19. 6. 1904
                  – 23. 6. 1904?] in Wien}\toendnotes[C]{\smallbreak}
\Standort{FDH, Hs-30885,107.}
\physDesc{Brief, 1 Blatt, 3 Seiten, 1193 Zeichen
\newline{}Handschrift: schwarze Tinte, deutsche Kurrent
\newline{}Ordnung: mit Bleistift von Schnitzler – mutmaßlich bei der Durchsicht der Briefe
                                    1929 – datiert: »1904« }
\buchAbdrucke{\weitereDrucke{Hugo von Hofmannsthal, Arthur Schnitzler: \emph{Briefwechsel}. Herausgegeben von Therese Nickl und Heinrich Schnitzler. Frankfurt am Main: \emph{S. Fischer} 1964, S. 188.} }\toendnotes[C]{\smallbreak}
\pstart{}{\pb}mein lieber Hugo,\pend\vspace{0.5em}
\pstart
           unter den jetzigen Witterungsverhältniſſen empfiehlt es{ }ſich jedenfalls, unſern
               Spaziergang erſt gegen Abend, etwa von 5 ½ Uhr an zu
               machen, und irgendwo draußen (Salmannsdorf\oindex{Wien@\textbf{Wien}!XIX., Döbling@\textbf{XIX., Döbling}!Salmannsdorf@\textbf{Salmannsdorf}, \emph{Ehemaliger Ort}|pw}, \textsc{etc}) zu nachtmahlen. Richten Sie{ }ſichs alſo mit \textsc{Gerty}\pwindex{Hofmannsthal, Gertrude von 16.\,3.\,1880 Wien – 9.\,11.\,1959 Paddington@\textsc{Hofmannsthal, Gertrude von} (16.\,3.\,1880 Wien – 9.\,11.\,1959 Paddington)|pw} lieber{ }ſo ein, dſs Sie an dem betreffenden Tag nicht mehr nach Rodaun\oindex{Wien@\textbf{Wien}!XXIII., Liesing@\textbf{XXIII., Liesing}!Rodaun@\textbf{Rodaun}, \emph{Region}|pw} hinausmüſſen. Unſre Gegend (worunter ich
                  Pötzldorf\oindex{Wien@\textbf{Wien}!XVIII., Währing@\textbf{XVIII., Währing}!Pötzleinsdorf@\textbf{Pötzleinsdorf}, \emph{Ehemaliger Ort}|pw}, Neuwaldegg\oindex{Wien@\textbf{Wien}!XVII., Hernals@\textbf{XVII., Hernals}!Neuwaldegg@\textbf{Neuwaldegg}, \emph{Ehemaliger Ort}|pw}, {\pb}Weidlingbach\oindex{Weidlingbach@\textbf{Weidlingbach}|pw}{ }\textsc{etc} kurz alles zwiſchen der alten Tullner Reichſtraße\oindex{Wien@\textbf{Wien}!XVII., Hernals@\textbf{XVII., Hernals}!Exelbergstraße@\textbf{Exelbergstraße}, \emph{Straße}|pw} bis zur Donau\oindex{Donau@\textbf{Donau}, \emph{Fluss}|pw} verſtehe) iſt wirklich wundervoll, ich radle manchmal (zu{ }ſelten) nur in den Wald zwiſchen Pötzleinsdorf\oindex{Wien@\textbf{Wien}!XVIII., Währing@\textbf{XVIII., Währing}!Pötzleinsdorf@\textbf{Pötzleinsdorf}, \emph{Ehemaliger Ort}|pw} u
                  Neuwaldegg\oindex{Wien@\textbf{Wien}!XVII., Hernals@\textbf{XVII., Hernals}!Neuwaldegg@\textbf{Neuwaldegg}, \emph{Ehemaliger Ort}|pw} und bin immer wieder von neuem
               entzückt. Schade dſs man nirgends angenehme oder nur mögliche Hotels findet. Ich{ }ſchlage Ihnen den \label{K_L01407-1v}\edtext{Mittwoch}{\lemma{\textnormal{\emph{Mittwoch}}}\Cendnote{\textnormal{Die Datierung des Briefes gelingt durch die
                  inhaltliche Mittelstellung zwischen dem vorangehenden (XXXX Auszeichnungsfehler: Dokument L01406 nicht gefunden) und dem nachfolgenden (XXXX Auszeichnungsfehler: Dokument L01408 nicht gefunden) Brief  der
                  Korrespondenz mit Hofmannsthal\pwindex{Hofmannsthal, Hugo von 1.\,2.\,1874 Wien – 15.\,7.\,1929 Rodaun@\textsc{Hofmannsthal, Hugo von} (1.\,2.\,1874 Wien – 15.\,7.\,1929 Rodaun), \emph{Schriftsteller}|pwk}.}}}\label{K_L01407-1} vor,
               an welchem Tag wir Sie mit \textsc{Gerty}\pwindex{Hofmannsthal, Gertrude von 16.\,3.\,1880 Wien – 9.\,11.\,1959 Paddington@\textsc{Hofmannsthal, Gertrude von} (16.\,3.\,1880 Wien – 9.\,11.\,1959 Paddington)|pw} um 5 erwarten. Sind Sie aber {\pb}ſchon
                  Vormittag in Wien\oindex{Wien@\textbf{Wien}, \emph{Verwaltungsgebiet}|pw},{ }ſo wäre es
               ausnehmend nett, we{\geminationn}{ }Sie bei uns{ }ſchon{ }ſpeiſten (gegen ½ 2)
               – wir ruhen uns da{\geminationn} in der Nachmittagshitze aus, und
               gehen fort, wann’s uns beliebt. Viel liegt in der Zeit, in der man{ }ſich nicht geſehen
               hat – Sicilien\oindex{Sizilien@\textbf{Sizilien}, \emph{Land}|pw} und Holland\oindex{Niederlande@\textbf{Niederlande}|pw} – was mir beinahe noch wichtiger{ }ſcheint als der
               kleine Kraus\pwindex{Kraus, Karl 28.\,4.\,1874 Jičín – 12.\,6.\,1936 Wien@\textsc{Kraus, Karl} (28.\,4.\,1874 Jičín – 12.\,6.\,1936 Wien), \emph{Schriftsteller, Publizist, Schriftsteller}|pw}{ }\substVorne{}\textsuperscript{oder}\substDazwischen{}der Sie zu früh, und\substHinten{} der große Graus, der Sie zu{ }ſpät gepackt hat. –\pend
           
\pstart
           Auf Wiederſehen. Antwort erbeten.{\\[\baselineskip]}Herzlichſt{\\[\baselineskip]}Ihr\spacefill\mbox{A.}\pend
           \leftskip=0em{}\selectlanguage{ngerman}\endnumbering\briefempfaengerindex{Hofmannsthal, Hugo von@\textsc{Hofmannsthal, Hugo von}!zzzSchnitzler, Arthur@\emph{von Arthur Schnitzler}!1904-06-192@{{[}19.? 6. 1904{]}}|)be}\mylabel{L01407h}  \newcommand{\dateiname}{L01407}\newcommand{\titel}{Arthur Schnitzler an Hugo von Hofmannsthal, [19.? 6. 1904]}\newcommand{\editorInnen}{Martin Anton Müller und Gerd-Hermann Susen}%% latex-leseansicht-abspann.tex
%% Abspann für die Leseansicht.
%% Der Schalter \ifkorrekturansicht ist bereits durch den Vorspann gesetzt.

%% latex-abspann.tex
%% Gemeinsamer Abspann für Korrekturansicht und Leseansicht.
%% Setzt den Schalter \ifkorrekturansicht voraus (gesetzt in den
%% einbindenden Dateien latex-korrekturansicht-abspann.tex bzw.
%% latex-leseansicht-abspann.tex).
%% ---------------------------------------------------------------

\normalsize

% Das esempio-Environment wird nur in der Leseansicht benötigt
\ifkorrekturansicht\else
\newenvironment{esempio}[3]%
{
    \vspace{1.5ex}
    \rlap{\underline{#1}}
    \par
    \setlength{\parindent}{0cm}
    \nopagebreak
    \leftskip=#2cm
    \rightskip=#3cm
}
{
    \par
}
\fi

\doendnotes{C}
\bigskip
\vfill

\clearpage

\footnotesize

\ifkorrekturansicht
  \lohead{\textsc{register}}
\fi

% theindex-Environment neu definieren ohne reledmac
\makeatletter
\renewenvironment{theindex}{%
  \ifkorrekturansicht
    \section*{\indexname}%
  \else
    \subsubsection*{Index der erwähnten Entitäten}%
  \fi
  \setlength{\parindent}{0pt}%
  \setlength{\parskip}{0pt plus 0.3pt}%
  \let\item\@idxitem
}{%
  \ifkorrekturansicht\clearpage\fi
}
\makeatother

\IfFileExists{\jobname-pw.ind}{\input{\jobname-pw.ind}}{}

% Quellenangabe nur in der Leseansicht
\ifkorrekturansicht\else
% Fallback-Definitionen, falls die .tex-Datei \titel etc. nicht gesetzt hat
\providecommand{\titel}{}
\providecommand{\editorInnen}{}
\providecommand{\dateiname}{\jobname}

\vspace{3cm}

\vfill

\footnotesize
\textsc{Quelle}: \titel. Herausgegeben von {\editorInnen}. In: \emph{Arthur Schnitzler: Briefwechsel mit Autorinnen und Autoren}.
 Digitale Edition, https://schnitzler-briefe.acdh.oeaw.ac.at/{\dateiname}.html (Stand \today)
\fi

\end{document}


