%% latex-leseansicht-vorspann.tex
%% Vorspann für die Leseansicht.
%% Lädt die gemeinsame Datei latex-vorspann.tex mit nicht gesetztem Schalter.

\newif\ifkorrekturansicht
\korrekturansichtfalse

\input{../tex-inputs/latex-vorspann}


         
         \renewcommand{\erwaehntePersonen}{Personen: Hugo von Hofmannsthal, Gertrude von Hofmannsthal, Karl Kraus}
         \renewcommand{\erwaehnteOrte}{Orte: Exelbergstraße, Neuwaldegg, Niederlande, Pötzleinsdorf, Rodaun, Salmannsdorf, Sizilien, Weidlingbach, Wien}
         \renewcommand{\erwaehnteWerke}{
               \section[Arthur Schnitzler an Hugo von Hofmannsthal, {[}19.? 6. 1904{]}]{ Arthur Schnitzler an Hugo von Hofmannsthal, {[}19.? 6. 1904{]}}\nopagebreak\mylabel{v}\rehead{ }\begin{ledgroupsized}[t]{13cm}\normalsize\beginnumbering \toendnotes[C]{\smallbreak\pagebreak[2]} \Standort{FDH, Hs-30885,107.}
\physDesc{Brief, 1 Blatt, 3 Seiten
\newline{}Handschrift: schwarze Tinte, deutsche Kurrent\newline{}Ordnung: von Schnitzler – mutmaßlich bei
                           der Durchsicht der Briefe 1929 – mit Bleistift datiert: »1904« }\buchAbdrucke{\weitereDrucke{Hugo von Hofmannsthal, Arthur Schnitzler: \emph{Briefwechsel}. Hg. Therese Nickl und Heinrich Schnitzler. Frankfurt am Main: \emph{S. Fischer} 1964, S. 188.} }\toendnotes[C]{\smallbreak}\pstart{}{\pb}mein lieber Hugo, \pend\pstart
           unter den jetzigen Witterungsverhältniſſen empfiehlt es ſich jedenfalls, unſern
               Spaziergang erſt gegen Abend, etwa von 5 ½ Uhr an zu
               machen, und irgendwo draußen (Salmannsdorf\oindex{Salmannsdorf@\textbf{Salmannsdorf}|pw}, \textsc{etc}) zu nachtmahlen. Richten Sie ſichs alſo mit \textsc{Gerty}\pwindex{Hofmannsthal, Gertrude von 16.03.1880 – 09.11.1959@\textsc{Hofmannsthal, Gertrude von} (16.03.1880 – 09.11.1959)|pw} lieber ſo ein, dſs Sie an dem betreffenden Tag nicht mehr nach Rodaun\oindex{Rodaun@\textbf{Rodaun}|pw} hinausmüſſen. Unſre Gegend (worunter ich Pötzldorf\oindex{Poetzleinsdorf@\textbf{Pötzleinsdorf}|pw}, Neuwaldegg\oindex{Neuwaldegg@\textbf{Neuwaldegg}|pw}, {\pb}Weidlingbach\oindex{Weidlingbach@\textbf{Weidlingbach}|pw}{ }\textsc{etc} kurz alles zwiſchen der alten Tullner Reichſtraße\oindex{Exelbergstrasse@\textbf{Exelbergstraße}|pw} bis zur Donau verſtehe) iſt wirklich wundervoll, ich radle manchmal (zu
               ſelten) nur in den Wald zwiſchen Pötzleinsdorf\oindex{Poetzleinsdorf@\textbf{Pötzleinsdorf}|pw} u Neuwaldegg\oindex{Neuwaldegg@\textbf{Neuwaldegg}|pw} und bin immer wieder von neuem entzückt.
               Schade dſs man nirgends angenehme oder nur mögliche Hotels findet. Ich ſchlage Ihnen
               den \label{K_L01407_1v}\edtext{Mittwoch}{\lemma{\textnormal{\emph{Mittwoch}}}\Cendnote{\textnormal{Die Datierung des Briefes geht über die
                  inhaltliche Mittelstellung zwischen dem vorangehenden und dem folgenden Brief der
                  Korrespondenz mit Hofmannsthal\pwindex{Hofmannsthal, Hugo von 1874-02-01 – 1929-07-15@\textsc{Hofmannsthal, Hugo von} (1874-02-01 – 1929-07-15), \emph{Schriftsteller}|pwk}.}}}\label{K_L01407_1h} vor, an
               welchem Tag wir Sie mit \textsc{Gerty}\pwindex{Hofmannsthal, Gertrude von 16.03.1880 – 09.11.1959@\textsc{Hofmannsthal, Gertrude von} (16.03.1880 – 09.11.1959)|pw} um 5 erwarten. Sind Sie aber {\pb}ſchon
                  Vormittag in Wien\oindex{Wien@\textbf{Wien}|pw}, ſo wäre es
               ausnehmend nett, we{\geminationn}{ }Sie bei uns ſchon ſpeiſten (gegen ½ 2)
               – wir ruhen uns da{\geminationn} in der Nachmittagshitze aus, und
               gehen fort, wann’s uns beliebt. Viel liegt in der Zeit, in der man ſich nicht geſehen
               hat – Sicilien\oindex{Sizilien@\textbf{Sizilien}|pw} und Holland\oindex{Niederlande@\textbf{Niederlande}|pw} – was mir beinahe noch wichtiger ſcheint als der kleine Kraus\pwindex{Kraus, Karl 28.04.1874 – 12.06.1936@\textsc{Kraus, Karl} (28.04.1874 – 12.06.1936), \emph{Schriftsteller, Publizist}|pw}{ }\substVorne{}\textsuperscript{oder}\substDazwischen{}der Sie zu früh, und\substHinten{} der große Graus, der Sie zu ſpät gepackt hat. –\pend
           \pstart
           Auf Wiederſehen. Antwort erbeten.{\\[\baselineskip]}Herzlichſt{\\[\baselineskip]}Ihr\spacefill\mbox{A.}\pend
           \leftskip=0em{}
         
         \endnumbering\mylabel{h}\end{ledgroupsized}  \newcommand{\dateiname}{L01407}\newcommand{\titel}{Arthur Schnitzler an Hugo von Hofmannsthal, [19.? 6. 1904]}\newcommand{\editorInnen}{Martin Anton Müller und Gerd-Hermann Susen}%% latex-leseansicht-abspann.tex
%% Abspann für die Leseansicht.
%% Der Schalter \ifkorrekturansicht ist bereits durch den Vorspann gesetzt.

%% latex-abspann.tex
%% Gemeinsamer Abspann für Korrekturansicht und Leseansicht.
%% Setzt den Schalter \ifkorrekturansicht voraus (gesetzt in den
%% einbindenden Dateien latex-korrekturansicht-abspann.tex bzw.
%% latex-leseansicht-abspann.tex).
%% ---------------------------------------------------------------

\normalsize

% Das esempio-Environment wird nur in der Leseansicht benötigt
\ifkorrekturansicht\else
\newenvironment{esempio}[3]%
{
    \vspace{1.5ex}
    \rlap{\underline{#1}}
    \par
    \setlength{\parindent}{0cm}
    \nopagebreak
    \leftskip=#2cm
    \rightskip=#3cm
}
{
    \par
}
\fi

\doendnotes{C}
\bigskip
\vfill

\clearpage

\footnotesize

\ifkorrekturansicht
  \lohead{\textsc{register}}
\fi

% theindex-Environment neu definieren ohne reledmac
\makeatletter
\renewenvironment{theindex}{%
  \ifkorrekturansicht
    \section*{\indexname}%
  \else
    \subsubsection*{Index der erwähnten Entitäten}%
  \fi
  \setlength{\parindent}{0pt}%
  \setlength{\parskip}{0pt plus 0.3pt}%
  \let\item\@idxitem
}{%
  \ifkorrekturansicht\clearpage\fi
}
\makeatother

\IfFileExists{\jobname-pw.ind}{\input{\jobname-pw.ind}}{}

% Quellenangabe nur in der Leseansicht
\ifkorrekturansicht\else
% Fallback-Definitionen, falls die .tex-Datei \titel etc. nicht gesetzt hat
\providecommand{\titel}{}
\providecommand{\editorInnen}{}
\providecommand{\dateiname}{\jobname}

\vspace{3cm}

\vfill

\footnotesize
\textsc{Quelle}: \titel. Herausgegeben von {\editorInnen}. In: \emph{Arthur Schnitzler: Briefwechsel mit Autorinnen und Autoren}.
 Digitale Edition, https://schnitzler-briefe.acdh.oeaw.ac.at/{\dateiname}.html (Stand \today)
\fi

\end{document}


      