%% latex-korrekturansicht-vorspann.tex
%% Vorspann für die Korrekturansicht.
%% Lädt die gemeinsame Datei latex-vorspann.tex mit gesetztem Schalter.

\newif\ifkorrekturansicht
\korrekturansichttrue

\input{../tex-inputs/latex-vorspann}


\section[Arthur Schnitzler an Hugo von Hofmannsthal, {[}19.? 6. 1904{]}]{L01407 Arthur Schnitzler an Hugo von Hofmannsthal, {[}19.? 6. 1904{]}}
\nopagebreak\mylabel{L01407v}
\rehead{ }\normalsize\beginnumbering\briefempfaengerindex{Hofmannsthal, Hugo von@\textsc{Hofmannsthal, Hugo von}!zzzSchnitzler, Arthur@\emph{von Arthur Schnitzler}!1904-06-192@{{[}19.? 6. 1904{]}}|(be}
\toendnotes[C]{\smallbreak\pagebreak[2]}\Standort{FDH, Hs-30885,107.}
\physDesc{Brief, 1 Blatt, 3 Seiten, 1193 Zeichen
\newline{}Handschrift: schwarze Tinte, deutsche Kurrent
\newline{}Ordnung: mit Bleistift von Schnitzler – mutmaßlich bei der Durchsicht der Briefe
                                    1929 – datiert: »1904« }
\buchAbdrucke{\weitereDrucke{Hugo von Hofmannsthal, Arthur Schnitzler: \emph{Briefwechsel}. Frankfurt am Main: \emph{S. Fischer} 1964, S. 188.} }\toendnotes[C]{\smallbreak}
\pstart{}{\pb}mein lieber Hugo, \pend\vspace{0.5em}
\pstart
           unter den jetzigen Witterungsverhältniſſen empfiehlt es ſich jedenfalls, unſern
               Spaziergang erſt gegen Abend, etwa von 5 ½ Uhr an zu
               machen, und irgendwo draußen (Salmannsdorf\oindex{Salmannsdorf@\textbf{Salmannsdorf}, \emph{P.PPLX}|pw}, \textsc{etc}) zu nachtmahlen. Richten Sie ſichs alſo mit \textsc{Gerty}\pwindex{Hofmannsthal, Gertrude von 16.03.1880 – 09.11.1959@\textsc{Hofmannsthal, Gertrude von} (16.03.1880 – 09.11.1959)|pw} lieber ſo ein, dſs Sie an dem betreffenden Tag nicht mehr nach Rodaun\oindex{Rodaun@\textbf{Rodaun}, \emph{A.ADM4}|pw} hinausmüſſen. Unſre Gegend (worunter ich
                  Pötzldorf\oindex{Poetzleinsdorf@\textbf{Pötzleinsdorf}, \emph{P.PPLX}|pw}, Neuwaldegg\oindex{Neuwaldegg@\textbf{Neuwaldegg}, \emph{P.PPLX}|pw}, {\pb}Weidlingbach\oindex{Weidlingbach@\textbf{Weidlingbach}, \emph{P.PPL}|pw}{ }\textsc{etc} kurz alles zwiſchen der alten Tullner Reichſtraße\oindex{Exelbergstrasse@\textbf{Exelbergstraße}, \emph{Straße (K.STR)}|pw} bis zur Donau\oindex{Donau@\textbf{Donau}, \emph{Fluss (N.FLS)}|pw} verſtehe) iſt wirklich wundervoll, ich radle manchmal (zu
               ſelten) nur in den Wald zwiſchen Pötzleinsdorf\oindex{Poetzleinsdorf@\textbf{Pötzleinsdorf}, \emph{P.PPLX}|pw} u
                  Neuwaldegg\oindex{Neuwaldegg@\textbf{Neuwaldegg}, \emph{P.PPLX}|pw} und bin immer wieder von neuem
               entzückt. Schade dſs man nirgends angenehme oder nur mögliche Hotels findet. Ich
               ſchlage Ihnen den \label{K_L01407-1v}\edtext{Mittwoch}{\lemma{\textnormal{\emph{Mittwoch}}}\Cendnote{\textnormal{Die Datierung des Briefes gelingt durch die
                  inhaltliche Mittelstellung zwischen dem vorangehenden (Hugo von Hofmannsthal an Arthur Schnitzler, 1[9?]. 6. [1904]) und dem nachfolgenden (Hugo von Hofmannsthal an Arthur Schnitzler, 20. 6. 1904) Brief  der
                  Korrespondenz mit Hofmannsthal\pwindex{Hofmannsthal, Hugo von 1874-02-01 – 1929-07-15@\textsc{Hofmannsthal, Hugo von} (1874-02-01 – 1929-07-15), \emph{Schriftsteller/Schriftstellerin}|pwk}.}}}\label{K_L01407-1} vor,
               an welchem Tag wir Sie mit \textsc{Gerty}\pwindex{Hofmannsthal, Gertrude von 16.03.1880 – 09.11.1959@\textsc{Hofmannsthal, Gertrude von} (16.03.1880 – 09.11.1959)|pw} um 5 erwarten. Sind Sie aber {\pb}ſchon
                  Vormittag in Wien\oindex{Wien@\textbf{Wien}, \emph{A.ADM2}|pw}, ſo wäre es
               ausnehmend nett, we{\geminationn}{ }Sie bei uns ſchon ſpeiſten (gegen ½ 2)
               – wir ruhen uns da{\geminationn} in der Nachmittagshitze aus, und
               gehen fort, wann’s uns beliebt. Viel liegt in der Zeit, in der man ſich nicht geſehen
               hat – Sicilien\oindex{Sizilien@\textbf{Sizilien}, \emph{A.ADM1}|pw} und Holland\oindex{Niederlande@\textbf{Niederlande}, \emph{A.PCLI}|pw} – was mir beinahe noch wichtiger ſcheint als der
               kleine Kraus\pwindex{Kraus, Karl 28.04.1874 – 12.06.1936@\textsc{Kraus, Karl} (28.04.1874 – 12.06.1936), \emph{Schriftsteller/Schriftstellerin, Publizist/Publizistin, Schriftsteller/Schriftstellerin}|pw}{ }\substVorne{}\textsuperscript{oder}\substDazwischen{}der Sie zu früh, und\substHinten{} der große Graus, der Sie zu ſpät gepackt hat. –\pend
           
\pstart
           Auf Wiederſehen. Antwort erbeten.{\\[\baselineskip]}Herzlichſt{\\[\baselineskip]}Ihr\spacefill\mbox{A.}\pend
           \leftskip=0em{}\selectlanguage{ngerman}\endnumbering\briefempfaengerindex{Hofmannsthal, Hugo von@\textsc{Hofmannsthal, Hugo von}!zzzSchnitzler, Arthur@\emph{von Arthur Schnitzler}!1904-06-192@{{[}19.? 6. 1904{]}}|)be}\mylabel{L01407h}  \normalsize

\doendnotes{C}
\bigskip
\vfill

\clearpage

\footnotesize

\lohead{\textsc{register}}

% Definiere theindex-Environment komplett neu ohne reledmac
\makeatletter
\renewenvironment{theindex}{%
  \section*{\indexname}%
  \setlength{\parindent}{0pt}%
  \setlength{\parskip}{0pt plus 0.3pt}%
  \let\item\@idxitem
}{%
  \clearpage
}
\makeatother

\IfFileExists{\jobname-pw.ind}{\input{\jobname-pw.ind}}{}

\end{document}

      