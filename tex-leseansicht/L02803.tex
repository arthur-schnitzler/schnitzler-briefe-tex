%% latex-leseansicht-vorspann.tex
%% Vorspann für die Leseansicht.
%% Lädt die gemeinsame Datei latex-vorspann.tex mit nicht gesetztem Schalter.

\newif\ifkorrekturansicht
\korrekturansichtfalse

\input{../tex-inputs/latex-vorspann}


         
         \renewcommand{\erwaehntePersonen}{Personen: Leopold Sonnemann}
         \renewcommand{\erwaehnteInstitutionen}{Institutionen: Frankfurter Zeitung}
         \renewcommand{\erwaehnteOrte}{Orte: Griechenland, Kreta, Panama, Paris, Riviera, Türkei, Wien, rue Feydeau}
         \renewcommand{\erwaehnteWerke}{Werke: Das Aktionsprogramm der Mächte [Privattelegramm], Die Ereignisse auf Kreta, Die Ereignisse auf Kreta [2 Privattelegramme], Die Ereignisse auf Kreta [2 Privattelegramme], Die Ereignisse auf Kreta [2 Privattelegramme], Die Ereignisse auf Kreta [2 Privattelegramme], Die Ereignisse auf Kreta [Privattelegramm], Die Ereignisse auf Kreta [Privattelegramm], Die Ereignisse auf Kreta [Privattelegramm], Die Ereignisse auf Kreta [Privattelegramm], Die Ereignisse auf Kreta [Privattelegramm], Die Ereignisse auf Kreta [Privattelegramm], Die Ereignisse auf Kreta [Privattelegramm], Die Ereignisse auf Kreta [Privattelegramm], Die Ereignisse auf Kreta [Privattelegramm], Die Interpellation Cochins in der französischen Kammer [Privattelegramm], Die Kretafrage in der französischen Kammer [Privattelegramm], Die Kretafrage und die Uneinigkeit der Mächte [Privattelegramm], Die Kämpfe auf Kreta. [Privattelegramm], Die Occupation Kreta’s [3 Privattelegramme], Die Occupation Kreta’s [Privattelegramm], Die griechischen Minister, Die kretische Frage und die Mächte [Privattelegramm], Frankfurter Zeitung, Kreta, Griechenland und die Mächte, Kreta, Griechenland und die Mächte, Kreta, Griechenland und die Mächte, Kreta, Griechenland und die Mächte [2 Privattelegramme], Kreta, Griechenland und die Mächte [2 Privattelegramme], Kreta, Griechenland und die Mächte [2 Privattelegramme], Kreta, Griechenland und die Mächte [2 Privattelegramme], Kreta, Griechenland und die Mächte [3 Privattelegramme], Kreta, Griechenland und die Mächte [Bericht und Privattelegramm], Kreta, Griechenland und die Mächte [Privattelegramm], Kreta, Griechenland und die Mächte [Privattelegramm], Kreta, Griechenland und die Mächte [Privattelegramm], Kreta, Griechenland und die Mächte [Privattelegramm], Kreta, Griechenland und die Mächte [Privattelegramm], Kreta, Griechenland und die Mächte [Privattelegramm], Kreta, Griechenland und die Mächte [Privattelegramm], Kreta, Griechenland und die Mächte [Privattelegramm], Kreta, Griechenland und die Mächte [Privattelegramm], Kreta, Griechenland und die Mächte [Privattelegramm], Kreta, Griechenland und die Mächte [Privattelegramm], Kreta, Griechenland und die Mächte [Privattelegramm], Kreta, Griechenland und die Mächte [Privattelegramm], Kreta, Griechenland und die Mächte [Privattelegramm], Zu den Wirren auf Kreta [Privattelegramm]}
               \section[ Paul Goldmann an Arthur Schnitzler, 16. 2. {[}1897{]}]{ Paul Goldmann an Arthur Schnitzler, 16. 2. {[}1897{]}}\nopagebreak\mylabel{v}\rehead{ }\begin{ledgroupsized}[t]{13cm}\normalsize\beginnumbering \toendnotes[C]{\smallbreak\pagebreak[2]} \Standort{DLA, A:Schnitzler, HS.NZ85.1.3167.}
\physDesc{Brief, 1 Blatt, 4 Seiten
\newline{}Handschrift: blaue Tinte, deutsche Kurrent
\newline{}Schnitzler: mit Bleistift das Jahr »97« vermerkt }\toendnotes[C]{\smallbreak}\pstart
           \noindent{}{\pb}\textcolor{gray}{\textbf{\textbf{Frankfurter Zeitung\orgindex{Frankfurter Zeitung@Frankfurter Zeitung|pw}}}}\pend
           \pstart
           \textcolor{gray}{\textbf{(\begin{otherlanguage}{french}Gazette de Francfort\end{otherlanguage}\orgindex{Frankfurter Zeitung@Frankfurter Zeitung|pw}).}}\pend
           \pstart
           \textcolor{gray}{\textbf{\textbf{\begin{otherlanguage}{french}Fondateur M.\end{otherlanguage}{ }L. Sonnemann\pwindex{Sonnemann, Leopold 1831-10-29 – 1909-10-30@\textsc{Sonnemann, Leopold} (1831-10-29 – 1909-10-30), \emph{Journalist, Herausgeber}|pw}.}}}\pend
           \pstart
           \begin{otherlanguage}{french}\textcolor{gray}{\textbf{Journal politique, financier,}}\end{otherlanguage}\pend
           \pstart
           \begin{otherlanguage}{french}\textcolor{gray}{\textbf{commercial et littéraire.}}\end{otherlanguage}\pend
           \pstart
           \begin{otherlanguage}{french}\textcolor{gray}{\textbf{\textbf{Paraissant trois fois par jour.}}}\end{otherlanguage}\hfill \textsc{Paris\oindex{Paris@\textbf{Paris}|pw}}, 16. Februar.\pend
           \pstart
           \begin{otherlanguage}{french}\textcolor{gray}{\textbf{\textbf{Bureau à Paris\oindex{Paris@\textbf{Paris}|pw}}}}\end{otherlanguage}\pend
           \pstart
           \begin{otherlanguage}{french}\textcolor{gray}{\textbf{\textbf{24. Rue Feydeau\oindex{rue Feydeau@\textbf{rue Feydeau}|pw}.}}}\end{otherlanguage}\pend
           \pstart\center{}Mein lieber Freund,\pend\pstart
           Ich ſtecke mitten in den \label{K_L02803-1v}\edtext{\textsc{Kreta\oindex{Kreta@\textbf{Kreta}|pw}}-Geſchichten\pwindex{Goldmann, Paul 31.01.1865 – 25.09.1935@\textsc{Goldmann, Paul} (31.01.1865 – 25.09.1935), \emph{Schriftsteller, Journalist}!Ereignisse auf Kreta1897-02-10@\strich\emph{Die Ereignisse auf Kreta} {[}1897-02-10{]}|pwv}\pwindex{Goldmann, Paul 31.01.1865 – 25.09.1935@\textsc{Goldmann, Paul} (31.01.1865 – 25.09.1935), \emph{Schriftsteller, Journalist}!Ereignisse auf Kreta [Privattelegramm]1897-02-11@\strich\emph{Die Ereignisse auf Kreta [Privattelegramm]} {[}1897-02-11{]}|pwv}\pwindex{Goldmann, Paul 31.01.1865 – 25.09.1935@\textsc{Goldmann, Paul} (31.01.1865 – 25.09.1935), \emph{Schriftsteller, Journalist}!Ereignisse auf Kreta [2 Privattelegramme]1897-02-12@\strich\emph{Die Ereignisse auf Kreta [2 Privattelegramme]} {[}1897-02-12{]}|pwv}\pwindex{Goldmann, Paul 31.01.1865 – 25.09.1935@\textsc{Goldmann, Paul} (31.01.1865 – 25.09.1935), \emph{Schriftsteller, Journalist}!Ereignisse auf Kreta [Privattelegramm]1897-02-12@\strich\emph{Die Ereignisse auf Kreta [Privattelegramm]} {[}1897-02-12{]}|pwv}\pwindex{Goldmann, Paul 31.01.1865 – 25.09.1935@\textsc{Goldmann, Paul} (31.01.1865 – 25.09.1935), \emph{Schriftsteller, Journalist}!Ereignisse auf Kreta [Privattelegramm]1897-02-13@\strich\emph{Die Ereignisse auf Kreta [Privattelegramm]} {[}1897-02-13{]}|pwv}\pwindex{Goldmann, Paul 31.01.1865 – 25.09.1935@\textsc{Goldmann, Paul} (31.01.1865 – 25.09.1935), \emph{Schriftsteller, Journalist}!Ereignisse auf Kreta [Privattelegramm]1897-02-13@\strich\emph{Die Ereignisse auf Kreta [Privattelegramm]} {[}1897-02-13{]}|pwv}\pwindex{Goldmann, Paul 31.01.1865 – 25.09.1935@\textsc{Goldmann, Paul} (31.01.1865 – 25.09.1935), \emph{Schriftsteller, Journalist}!Ereignisse auf Kreta [Privattelegramm]1897-02-14@\strich\emph{Die Ereignisse auf Kreta [Privattelegramm]} {[}1897-02-14{]}|pwv}\pwindex{Goldmann, Paul 31.01.1865 – 25.09.1935@\textsc{Goldmann, Paul} (31.01.1865 – 25.09.1935), \emph{Schriftsteller, Journalist}!Ereignisse auf Kreta [Privattelegramm]1897-02-15@\strich\emph{Die Ereignisse auf Kreta [Privattelegramm]} {[}1897-02-15{]}|pwv}\pwindex{Goldmann, Paul 31.01.1865 – 25.09.1935@\textsc{Goldmann, Paul} (31.01.1865 – 25.09.1935), \emph{Schriftsteller, Journalist}!Ereignisse auf Kreta [2 Privattelegramme]1897-02-15@\strich\emph{Die Ereignisse auf Kreta [2 Privattelegramme]} {[}1897-02-15{]}|pwv}\pwindex{Goldmann, Paul 31.01.1865 – 25.09.1935@\textsc{Goldmann, Paul} (31.01.1865 – 25.09.1935), \emph{Schriftsteller, Journalist}!Ereignisse auf Kreta [2 Privattelegramme]1897-02-16@\strich\emph{Die Ereignisse auf Kreta [2 Privattelegramme]} {[}1897-02-16{]}|pwv}\pwindex{Goldmann, Paul 31.01.1865 – 25.09.1935@\textsc{Goldmann, Paul} (31.01.1865 – 25.09.1935), \emph{Schriftsteller, Journalist}!Ereignisse auf Kreta [2 Privattelegramme]1897-02-16@\strich\emph{Die Ereignisse auf Kreta [2 Privattelegramme]} {[}1897-02-16{]}|pwv}\pwindex{Goldmann, Paul 31.01.1865 – 25.09.1935@\textsc{Goldmann, Paul} (31.01.1865 – 25.09.1935), \emph{Schriftsteller, Journalist}!Occupation Kreta s [3 Privattelegramme]1897-02-16@\strich\emph{Die Occupation Kreta’s [3 Privattelegramme]} {[}1897-02-16{]}|pwv}\pwindex{Goldmann, Paul 31.01.1865 – 25.09.1935@\textsc{Goldmann, Paul} (31.01.1865 – 25.09.1935), \emph{Schriftsteller, Journalist}!Occupation Kreta s [Privattelegramm]1897-02-17@\strich\emph{Die Occupation Kreta’s [Privattelegramm]} {[}1897-02-17{]}|pwv}\pwindex{Goldmann, Paul 31.01.1865 – 25.09.1935@\textsc{Goldmann, Paul} (31.01.1865 – 25.09.1935), \emph{Schriftsteller, Journalist}!Kaempfe auf Kreta. [Privattelegramm]1897-02-17@\strich\emph{Die Kämpfe auf Kreta. [Privattelegramm]} {[}1897-02-17{]}|pwv}\pwindex{Goldmann, Paul 31.01.1865 – 25.09.1935@\textsc{Goldmann, Paul} (31.01.1865 – 25.09.1935), \emph{Schriftsteller, Journalist}!Ereignisse auf Kreta [Privattelegramm]1897-02-18@\strich\emph{Die Ereignisse auf Kreta [Privattelegramm]} {[}1897-02-18{]}|pwv}\pwindex{Goldmann, Paul 31.01.1865 – 25.09.1935@\textsc{Goldmann, Paul} (31.01.1865 – 25.09.1935), \emph{Schriftsteller, Journalist}!Ereignisse auf Kreta [Privattelegramm]1897-02-18@\strich\emph{Die Ereignisse auf Kreta [Privattelegramm]} {[}1897-02-18{]}|pwv}\pwindex{Goldmann, Paul 31.01.1865 – 25.09.1935@\textsc{Goldmann, Paul} (31.01.1865 – 25.09.1935), \emph{Schriftsteller, Journalist}!Ereignisse auf Kreta [Privattelegramm]1897-02-18@\strich\emph{Die Ereignisse auf Kreta [Privattelegramm]} {[}1897-02-18{]}|pwv}\pwindex{Goldmann, Paul 31.01.1865 – 25.09.1935@\textsc{Goldmann, Paul} (31.01.1865 – 25.09.1935), \emph{Schriftsteller, Journalist}!kretische Frage und die Maechte [Privattelegramm]1897-02-19@\strich\emph{Die kretische Frage und die Mächte [Privattelegramm]} {[}1897-02-19{]}|pwv}\pwindex{Goldmann, Paul 31.01.1865 – 25.09.1935@\textsc{Goldmann, Paul} (31.01.1865 – 25.09.1935), \emph{Schriftsteller, Journalist}!Kreta, Griechenland und die Maechte [Bericht und Privattelegramm]1897-02-19@\strich\emph{Kreta, Griechenland und die Mächte [Bericht und Privattelegramm]} {[}1897-02-19{]}|pwv}\pwindex{Goldmann, Paul 31.01.1865 – 25.09.1935@\textsc{Goldmann, Paul} (31.01.1865 – 25.09.1935), \emph{Schriftsteller, Journalist}!Kreta, Griechenland und die Maechte [Privattelegramm]1897-02-20@\strich\emph{Kreta, Griechenland und die Mächte [Privattelegramm]} {[}1897-02-20{]}|pwv}\pwindex{Goldmann, Paul 31.01.1865 – 25.09.1935@\textsc{Goldmann, Paul} (31.01.1865 – 25.09.1935), \emph{Schriftsteller, Journalist}!Kretafrage und die Uneinigkeit der Maechte [Privattelegramm]1897-02-20@\strich\emph{Die Kretafrage und die Uneinigkeit der Mächte [Privattelegramm]} {[}1897-02-20{]}|pwv}\pwindex{Goldmann, Paul 31.01.1865 – 25.09.1935@\textsc{Goldmann, Paul} (31.01.1865 – 25.09.1935), \emph{Schriftsteller, Journalist}!Zu den Wirren auf Kreta [Privattelegramm]1897-02-20@\strich\emph{Zu den Wirren auf Kreta [Privattelegramm]} {[}1897-02-20{]}|pwv}\pwindex{Goldmann, Paul 31.01.1865 – 25.09.1935@\textsc{Goldmann, Paul} (31.01.1865 – 25.09.1935), \emph{Schriftsteller, Journalist}!griechischen Minister1897-02-20@\strich\emph{Die griechischen Minister} {[}1897-02-20{]}|pwv}\pwindex{Goldmann, Paul 31.01.1865 – 25.09.1935@\textsc{Goldmann, Paul} (31.01.1865 – 25.09.1935), \emph{Schriftsteller, Journalist}!Kreta, Griechenland und die Maechte [3 Privattelegramme]1897-02-22@\strich\emph{Kreta, Griechenland und die Mächte [3 Privattelegramme]} {[}1897-02-22{]}|pwv}\pwindex{Goldmann, Paul 31.01.1865 – 25.09.1935@\textsc{Goldmann, Paul} (31.01.1865 – 25.09.1935), \emph{Schriftsteller, Journalist}!Kreta, Griechenland und die Maechte [Privattelegramm]1897-02-23@\strich\emph{Kreta, Griechenland und die Mächte [Privattelegramm]} {[}1897-02-23{]}|pwv}\pwindex{Goldmann, Paul 31.01.1865 – 25.09.1935@\textsc{Goldmann, Paul} (31.01.1865 – 25.09.1935), \emph{Schriftsteller, Journalist}!Interpellation Cochins in der franzoesischen Kammer [Privattelegramm]1897-02-23@\strich\emph{Die Interpellation Cochins in der französischen Kammer [Privattelegramm]} {[}1897-02-23{]}|pwv}\pwindex{Goldmann, Paul 31.01.1865 – 25.09.1935@\textsc{Goldmann, Paul} (31.01.1865 – 25.09.1935), \emph{Schriftsteller, Journalist}!Kreta, Griechenland und die Maechte [Privattelegramm]1897-02-24@\strich\emph{Kreta, Griechenland und die Mächte [Privattelegramm]} {[}1897-02-24{]}|pwv}\pwindex{Goldmann, Paul 31.01.1865 – 25.09.1935@\textsc{Goldmann, Paul} (31.01.1865 – 25.09.1935), \emph{Schriftsteller, Journalist}!Kreta, Griechenland und die Maechte [2 Privattelegramme]1897-02-26@\strich\emph{Kreta, Griechenland und die Mächte [2 Privattelegramme]} {[}1897-02-26{]}|pwv}\pwindex{Goldmann, Paul 31.01.1865 – 25.09.1935@\textsc{Goldmann, Paul} (31.01.1865 – 25.09.1935), \emph{Schriftsteller, Journalist}!Kreta, Griechenland und die Maechte [Privattelegramm]1897-03-05@\strich\emph{Kreta, Griechenland und die Mächte [Privattelegramm]} {[}1897-03-05{]}|pwv}\pwindex{Goldmann, Paul 31.01.1865 – 25.09.1935@\textsc{Goldmann, Paul} (31.01.1865 – 25.09.1935), \emph{Schriftsteller, Journalist}!Kreta, Griechenland und die Maechte [Privattelegramm]1897-03-06@\strich\emph{Kreta, Griechenland und die Mächte [Privattelegramm]} {[}1897-03-06{]}|pwv}\pwindex{Goldmann, Paul 31.01.1865 – 25.09.1935@\textsc{Goldmann, Paul} (31.01.1865 – 25.09.1935), \emph{Schriftsteller, Journalist}!Kreta, Griechenland und die Maechte1897-03-06@\strich\emph{Kreta, Griechenland und die Mächte} {[}1897-03-06{]}|pwv}\pwindex{Goldmann, Paul 31.01.1865 – 25.09.1935@\textsc{Goldmann, Paul} (31.01.1865 – 25.09.1935), \emph{Schriftsteller, Journalist}!Kreta, Griechenland und die Maechte [2 Privattelegramme]1897-03-07@\strich\emph{Kreta, Griechenland und die Mächte [2 Privattelegramme]} {[}1897-03-07{]}|pwv}\pwindex{Goldmann, Paul 31.01.1865 – 25.09.1935@\textsc{Goldmann, Paul} (31.01.1865 – 25.09.1935), \emph{Schriftsteller, Journalist}!Kreta, Griechenland und die Maechte [Privattelegramm]1897-03-07@\strich\emph{Kreta, Griechenland und die Mächte [Privattelegramm]} {[}1897-03-07{]}|pwv}\pwindex{Goldmann, Paul 31.01.1865 – 25.09.1935@\textsc{Goldmann, Paul} (31.01.1865 – 25.09.1935), \emph{Schriftsteller, Journalist}!Kreta, Griechenland und die Maechte [Privattelegramm]1897-03-08@\strich\emph{Kreta, Griechenland und die Mächte [Privattelegramm]} {[}1897-03-08{]}|pwv}\pwindex{Goldmann, Paul 31.01.1865 – 25.09.1935@\textsc{Goldmann, Paul} (31.01.1865 – 25.09.1935), \emph{Schriftsteller, Journalist}!Kreta, Griechenland und die Maechte [Privattelegramm]1897-03-09@\strich\emph{Kreta, Griechenland und die Mächte [Privattelegramm]} {[}1897-03-09{]}|pwv}\pwindex{Goldmann, Paul 31.01.1865 – 25.09.1935@\textsc{Goldmann, Paul} (31.01.1865 – 25.09.1935), \emph{Schriftsteller, Journalist}!Kreta, Griechenland und die Maechte [Privattelegramm]1897-03-09@\strich\emph{Kreta, Griechenland und die Mächte [Privattelegramm]} {[}1897-03-09{]}|pwv}\pwindex{Goldmann, Paul 31.01.1865 – 25.09.1935@\textsc{Goldmann, Paul} (31.01.1865 – 25.09.1935), \emph{Schriftsteller, Journalist}!Kreta, Griechenland und die Maechte [Privattelegramm]1897-03-10@\strich\emph{Kreta, Griechenland und die Mächte [Privattelegramm]} {[}1897-03-10{]}|pwv}\pwindex{Goldmann, Paul 31.01.1865 – 25.09.1935@\textsc{Goldmann, Paul} (31.01.1865 – 25.09.1935), \emph{Schriftsteller, Journalist}!Kreta, Griechenland und die Maechte [2 Privattelegramme]1897-03-10@\strich\emph{Kreta, Griechenland und die Mächte [2 Privattelegramme]} {[}1897-03-10{]}|pwv}\pwindex{Goldmann, Paul 31.01.1865 – 25.09.1935@\textsc{Goldmann, Paul} (31.01.1865 – 25.09.1935), \emph{Schriftsteller, Journalist}!Kreta, Griechenland und die Maechte1897-03-11@\strich\emph{Kreta, Griechenland und die Mächte} {[}1897-03-11{]}|pwv}\pwindex{Goldmann, Paul 31.01.1865 – 25.09.1935@\textsc{Goldmann, Paul} (31.01.1865 – 25.09.1935), \emph{Schriftsteller, Journalist}!Kreta, Griechenland und die Maechte [Privattelegramm]1897-03-13@\strich\emph{Kreta, Griechenland und die Mächte [Privattelegramm]} {[}1897-03-13{]}|pwv}\pwindex{Goldmann, Paul 31.01.1865 – 25.09.1935@\textsc{Goldmann, Paul} (31.01.1865 – 25.09.1935), \emph{Schriftsteller, Journalist}!Kreta, Griechenland und die Maechte [Privattelegramm]1897-03-14@\strich\emph{Kreta, Griechenland und die Mächte [Privattelegramm]} {[}1897-03-14{]}|pwv}\pwindex{Goldmann, Paul 31.01.1865 – 25.09.1935@\textsc{Goldmann, Paul} (31.01.1865 – 25.09.1935), \emph{Schriftsteller, Journalist}!Kreta, Griechenland und die Maechte [2 Privattelegramme]1897-03-15@\strich\emph{Kreta, Griechenland und die Mächte [2 Privattelegramme]} {[}1897-03-15{]}|pwv}\pwindex{Goldmann, Paul 31.01.1865 – 25.09.1935@\textsc{Goldmann, Paul} (31.01.1865 – 25.09.1935), \emph{Schriftsteller, Journalist}!Kreta, Griechenland und die Maechte [Privattelegramm]1897-03-16@\strich\emph{Kreta, Griechenland und die Mächte [Privattelegramm]} {[}1897-03-16{]}|pwv}\pwindex{Goldmann, Paul 31.01.1865 – 25.09.1935@\textsc{Goldmann, Paul} (31.01.1865 – 25.09.1935), \emph{Schriftsteller, Journalist}!Aktionsprogramm der Maechte [Privattelegramm]1897-03-16@\strich\emph{Das Aktionsprogramm der Mächte [Privattelegramm]} {[}1897-03-16{]}|pwv}\pwindex{Goldmann, Paul 31.01.1865 – 25.09.1935@\textsc{Goldmann, Paul} (31.01.1865 – 25.09.1935), \emph{Schriftsteller, Journalist}!Kretafrage in der franzoesischen Kammer [Privattelegramm]1897-03-16@\strich\emph{Die Kretafrage in der französischen Kammer [Privattelegramm]} {[}1897-03-16{]}|pwv}\pwindex{Goldmann, Paul 31.01.1865 – 25.09.1935@\textsc{Goldmann, Paul} (31.01.1865 – 25.09.1935), \emph{Schriftsteller, Journalist}!Kreta, Griechenland und die Maechte [Privattelegramm]1897-03-16@\strich\emph{Kreta, Griechenland und die Mächte [Privattelegramm]} {[}1897-03-16{]}|pwv}\pwindex{Goldmann, Paul 31.01.1865 – 25.09.1935@\textsc{Goldmann, Paul} (31.01.1865 – 25.09.1935), \emph{Schriftsteller, Journalist}!Kreta, Griechenland und die Maechte1897-03-17@\strich\emph{Kreta, Griechenland und die Mächte} {[}1897-03-17{]}|pwv}}{\lemma{\textnormal{\emph{Kreta-Geſchichten}}}\Cendnote{\textnormal{Am 6. 2. 1897 waren erste griech\oindex{Griechenland@\textbf{Griechenland}|pwk}ische Kriegsschiffe auf Kreta\oindex{Kreta@\textbf{Kreta}|pwk}
                  gelandet, um die unzufriedene Bevölkerung gegen die türk\oindex{Tuerkei@\textbf{Türkei}|pwkv}ische Regierung zu unterstützen. In
                  Folge kam es zwischen 18. 4. und 20. 5. 1897 zum Türk\oindex{Tuerkei@\textbf{Türkei}|pwk}isch-Griech\oindex{Griechenland@\textbf{Griechenland}|pwk}ischen Krieg. Goldmann\pwindex{Goldmann, Paul 31.01.1865 – 25.09.1935@\textsc{Goldmann, Paul} (31.01.1865 – 25.09.1935), \emph{Schriftsteller, Journalist}|pwk} berichtete darüber in der \emph{Frankfurter Zeitung}\pwindex{?? Werk@Nicht ermittelte Verfasserinnen und Verfasser!Frankfurter Zeitung1856 – 1943@\emph{Frankfurter Zeitung} {[}1856 – 1943{]}|pwk} (manchmal mehrmals täglich
                  und zumeist in der Form von Privattelegrammen) am 10. 2.\pwindex{Goldmann, Paul 31.01.1865 – 25.09.1935@\textsc{Goldmann, Paul} (31.01.1865 – 25.09.1935), \emph{Schriftsteller, Journalist}!Ereignisse auf Kreta1897-02-10@\strich\emph{Die Ereignisse auf Kreta} {[}1897-02-10{]}|pwkv}, 11. 2.\pwindex{Goldmann, Paul 31.01.1865 – 25.09.1935@\textsc{Goldmann, Paul} (31.01.1865 – 25.09.1935), \emph{Schriftsteller, Journalist}!Ereignisse auf Kreta [Privattelegramm]1897-02-11@\strich\emph{Die Ereignisse auf Kreta [Privattelegramm]} {[}1897-02-11{]}|pwkv}, 12. 2.\pwindex{Goldmann, Paul 31.01.1865 – 25.09.1935@\textsc{Goldmann, Paul} (31.01.1865 – 25.09.1935), \emph{Schriftsteller, Journalist}!Ereignisse auf Kreta [2 Privattelegramme]1897-02-12@\strich\emph{Die Ereignisse auf Kreta [2 Privattelegramme]} {[}1897-02-12{]}|pwkv}\pwindex{Goldmann, Paul 31.01.1865 – 25.09.1935@\textsc{Goldmann, Paul} (31.01.1865 – 25.09.1935), \emph{Schriftsteller, Journalist}!Ereignisse auf Kreta [Privattelegramm]1897-02-12@\strich\emph{Die Ereignisse auf Kreta [Privattelegramm]} {[}1897-02-12{]}|pwkv}, 13. 2.\pwindex{Goldmann, Paul 31.01.1865 – 25.09.1935@\textsc{Goldmann, Paul} (31.01.1865 – 25.09.1935), \emph{Schriftsteller, Journalist}!Ereignisse auf Kreta [Privattelegramm]1897-02-13@\strich\emph{Die Ereignisse auf Kreta [Privattelegramm]} {[}1897-02-13{]}|pwkv}\pwindex{Goldmann, Paul 31.01.1865 – 25.09.1935@\textsc{Goldmann, Paul} (31.01.1865 – 25.09.1935), \emph{Schriftsteller, Journalist}!Ereignisse auf Kreta [Privattelegramm]1897-02-13@\strich\emph{Die Ereignisse auf Kreta [Privattelegramm]} {[}1897-02-13{]}|pwkv}, 14. 2.\pwindex{Goldmann, Paul 31.01.1865 – 25.09.1935@\textsc{Goldmann, Paul} (31.01.1865 – 25.09.1935), \emph{Schriftsteller, Journalist}!Ereignisse auf Kreta [Privattelegramm]1897-02-14@\strich\emph{Die Ereignisse auf Kreta [Privattelegramm]} {[}1897-02-14{]}|pwkv}, 15. 2.\pwindex{Goldmann, Paul 31.01.1865 – 25.09.1935@\textsc{Goldmann, Paul} (31.01.1865 – 25.09.1935), \emph{Schriftsteller, Journalist}!Ereignisse auf Kreta [Privattelegramm]1897-02-15@\strich\emph{Die Ereignisse auf Kreta [Privattelegramm]} {[}1897-02-15{]}|pwkv}\pwindex{Goldmann, Paul 31.01.1865 – 25.09.1935@\textsc{Goldmann, Paul} (31.01.1865 – 25.09.1935), \emph{Schriftsteller, Journalist}!Ereignisse auf Kreta [2 Privattelegramme]1897-02-15@\strich\emph{Die Ereignisse auf Kreta [2 Privattelegramme]} {[}1897-02-15{]}|pwkv}, 16. 2.\pwindex{Goldmann, Paul 31.01.1865 – 25.09.1935@\textsc{Goldmann, Paul} (31.01.1865 – 25.09.1935), \emph{Schriftsteller, Journalist}!Ereignisse auf Kreta [2 Privattelegramme]1897-02-16@\strich\emph{Die Ereignisse auf Kreta [2 Privattelegramme]} {[}1897-02-16{]}|pwkv}\pwindex{Goldmann, Paul 31.01.1865 – 25.09.1935@\textsc{Goldmann, Paul} (31.01.1865 – 25.09.1935), \emph{Schriftsteller, Journalist}!Ereignisse auf Kreta [2 Privattelegramme]1897-02-16@\strich\emph{Die Ereignisse auf Kreta [2 Privattelegramme]} {[}1897-02-16{]}|pwkv}\pwindex{Goldmann, Paul 31.01.1865 – 25.09.1935@\textsc{Goldmann, Paul} (31.01.1865 – 25.09.1935), \emph{Schriftsteller, Journalist}!Occupation Kreta s [3 Privattelegramme]1897-02-16@\strich\emph{Die Occupation Kreta’s [3 Privattelegramme]} {[}1897-02-16{]}|pwkv}, 17. 2.\pwindex{Goldmann, Paul 31.01.1865 – 25.09.1935@\textsc{Goldmann, Paul} (31.01.1865 – 25.09.1935), \emph{Schriftsteller, Journalist}!Occupation Kreta s [Privattelegramm]1897-02-17@\strich\emph{Die Occupation Kreta’s [Privattelegramm]} {[}1897-02-17{]}|pwkv}\pwindex{Goldmann, Paul 31.01.1865 – 25.09.1935@\textsc{Goldmann, Paul} (31.01.1865 – 25.09.1935), \emph{Schriftsteller, Journalist}!Kaempfe auf Kreta. [Privattelegramm]1897-02-17@\strich\emph{Die Kämpfe auf Kreta. [Privattelegramm]} {[}1897-02-17{]}|pwkv}, 18. 2.\pwindex{Goldmann, Paul 31.01.1865 – 25.09.1935@\textsc{Goldmann, Paul} (31.01.1865 – 25.09.1935), \emph{Schriftsteller, Journalist}!Ereignisse auf Kreta [Privattelegramm]1897-02-18@\strich\emph{Die Ereignisse auf Kreta [Privattelegramm]} {[}1897-02-18{]}|pwkv}\pwindex{Goldmann, Paul 31.01.1865 – 25.09.1935@\textsc{Goldmann, Paul} (31.01.1865 – 25.09.1935), \emph{Schriftsteller, Journalist}!Ereignisse auf Kreta [Privattelegramm]1897-02-18@\strich\emph{Die Ereignisse auf Kreta [Privattelegramm]} {[}1897-02-18{]}|pwkv}\pwindex{Goldmann, Paul 31.01.1865 – 25.09.1935@\textsc{Goldmann, Paul} (31.01.1865 – 25.09.1935), \emph{Schriftsteller, Journalist}!Ereignisse auf Kreta [Privattelegramm]1897-02-18@\strich\emph{Die Ereignisse auf Kreta [Privattelegramm]} {[}1897-02-18{]}|pwkv}, 19. 2.\pwindex{Goldmann, Paul 31.01.1865 – 25.09.1935@\textsc{Goldmann, Paul} (31.01.1865 – 25.09.1935), \emph{Schriftsteller, Journalist}!kretische Frage und die Maechte [Privattelegramm]1897-02-19@\strich\emph{Die kretische Frage und die Mächte [Privattelegramm]} {[}1897-02-19{]}|pwkv}\pwindex{Goldmann, Paul 31.01.1865 – 25.09.1935@\textsc{Goldmann, Paul} (31.01.1865 – 25.09.1935), \emph{Schriftsteller, Journalist}!Kreta, Griechenland und die Maechte [Bericht und Privattelegramm]1897-02-19@\strich\emph{Kreta, Griechenland und die Mächte [Bericht und Privattelegramm]} {[}1897-02-19{]}|pwkv}, 20. 2.\pwindex{Goldmann, Paul 31.01.1865 – 25.09.1935@\textsc{Goldmann, Paul} (31.01.1865 – 25.09.1935), \emph{Schriftsteller, Journalist}!Kreta, Griechenland und die Maechte [Privattelegramm]1897-02-20@\strich\emph{Kreta, Griechenland und die Mächte [Privattelegramm]} {[}1897-02-20{]}|pwkv}\pwindex{Goldmann, Paul 31.01.1865 – 25.09.1935@\textsc{Goldmann, Paul} (31.01.1865 – 25.09.1935), \emph{Schriftsteller, Journalist}!Kretafrage und die Uneinigkeit der Maechte [Privattelegramm]1897-02-20@\strich\emph{Die Kretafrage und die Uneinigkeit der Mächte [Privattelegramm]} {[}1897-02-20{]}|pwkv}\pwindex{Goldmann, Paul 31.01.1865 – 25.09.1935@\textsc{Goldmann, Paul} (31.01.1865 – 25.09.1935), \emph{Schriftsteller, Journalist}!Zu den Wirren auf Kreta [Privattelegramm]1897-02-20@\strich\emph{Zu den Wirren auf Kreta [Privattelegramm]} {[}1897-02-20{]}|pwkv}\pwindex{Goldmann, Paul 31.01.1865 – 25.09.1935@\textsc{Goldmann, Paul} (31.01.1865 – 25.09.1935), \emph{Schriftsteller, Journalist}!griechischen Minister1897-02-20@\strich\emph{Die griechischen Minister} {[}1897-02-20{]}|pwkv}, 22. 2.\pwindex{Goldmann, Paul 31.01.1865 – 25.09.1935@\textsc{Goldmann, Paul} (31.01.1865 – 25.09.1935), \emph{Schriftsteller, Journalist}!Kreta, Griechenland und die Maechte [3 Privattelegramme]1897-02-22@\strich\emph{Kreta, Griechenland und die Mächte [3 Privattelegramme]} {[}1897-02-22{]}|pwkv}, 23. 2.\pwindex{Goldmann, Paul 31.01.1865 – 25.09.1935@\textsc{Goldmann, Paul} (31.01.1865 – 25.09.1935), \emph{Schriftsteller, Journalist}!Kreta, Griechenland und die Maechte [Privattelegramm]1897-02-23@\strich\emph{Kreta, Griechenland und die Mächte [Privattelegramm]} {[}1897-02-23{]}|pwkv}\pwindex{Goldmann, Paul 31.01.1865 – 25.09.1935@\textsc{Goldmann, Paul} (31.01.1865 – 25.09.1935), \emph{Schriftsteller, Journalist}!Interpellation Cochins in der franzoesischen Kammer [Privattelegramm]1897-02-23@\strich\emph{Die Interpellation Cochins in der französischen Kammer [Privattelegramm]} {[}1897-02-23{]}|pwkv}, 24. 2.\pwindex{Goldmann, Paul 31.01.1865 – 25.09.1935@\textsc{Goldmann, Paul} (31.01.1865 – 25.09.1935), \emph{Schriftsteller, Journalist}!Kreta, Griechenland und die Maechte [Privattelegramm]1897-02-24@\strich\emph{Kreta, Griechenland und die Mächte [Privattelegramm]} {[}1897-02-24{]}|pwkv}, 26. 2.\pwindex{Goldmann, Paul 31.01.1865 – 25.09.1935@\textsc{Goldmann, Paul} (31.01.1865 – 25.09.1935), \emph{Schriftsteller, Journalist}!Kreta, Griechenland und die Maechte [2 Privattelegramme]1897-02-26@\strich\emph{Kreta, Griechenland und die Mächte [2 Privattelegramme]} {[}1897-02-26{]}|pwkv}, 5. 3.\pwindex{Goldmann, Paul 31.01.1865 – 25.09.1935@\textsc{Goldmann, Paul} (31.01.1865 – 25.09.1935), \emph{Schriftsteller, Journalist}!Kreta, Griechenland und die Maechte [Privattelegramm]1897-03-05@\strich\emph{Kreta, Griechenland und die Mächte [Privattelegramm]} {[}1897-03-05{]}|pwkv}, 6. 3.\pwindex{Goldmann, Paul 31.01.1865 – 25.09.1935@\textsc{Goldmann, Paul} (31.01.1865 – 25.09.1935), \emph{Schriftsteller, Journalist}!Kreta, Griechenland und die Maechte [Privattelegramm]1897-03-06@\strich\emph{Kreta, Griechenland und die Mächte [Privattelegramm]} {[}1897-03-06{]}|pwkv}\pwindex{Goldmann, Paul 31.01.1865 – 25.09.1935@\textsc{Goldmann, Paul} (31.01.1865 – 25.09.1935), \emph{Schriftsteller, Journalist}!Kreta, Griechenland und die Maechte1897-03-06@\strich\emph{Kreta, Griechenland und die Mächte} {[}1897-03-06{]}|pwkv}, 7. 3.\pwindex{Goldmann, Paul 31.01.1865 – 25.09.1935@\textsc{Goldmann, Paul} (31.01.1865 – 25.09.1935), \emph{Schriftsteller, Journalist}!Kreta, Griechenland und die Maechte [2 Privattelegramme]1897-03-07@\strich\emph{Kreta, Griechenland und die Mächte [2 Privattelegramme]} {[}1897-03-07{]}|pwkv}\pwindex{Goldmann, Paul 31.01.1865 – 25.09.1935@\textsc{Goldmann, Paul} (31.01.1865 – 25.09.1935), \emph{Schriftsteller, Journalist}!Kreta, Griechenland und die Maechte [Privattelegramm]1897-03-07@\strich\emph{Kreta, Griechenland und die Mächte [Privattelegramm]} {[}1897-03-07{]}|pwkv}, 8. 3.\pwindex{Goldmann, Paul 31.01.1865 – 25.09.1935@\textsc{Goldmann, Paul} (31.01.1865 – 25.09.1935), \emph{Schriftsteller, Journalist}!Kreta, Griechenland und die Maechte [Privattelegramm]1897-03-08@\strich\emph{Kreta, Griechenland und die Mächte [Privattelegramm]} {[}1897-03-08{]}|pwkv}, 9. 3.\pwindex{Goldmann, Paul 31.01.1865 – 25.09.1935@\textsc{Goldmann, Paul} (31.01.1865 – 25.09.1935), \emph{Schriftsteller, Journalist}!Kreta, Griechenland und die Maechte [Privattelegramm]1897-03-09@\strich\emph{Kreta, Griechenland und die Mächte [Privattelegramm]} {[}1897-03-09{]}|pwkv}\pwindex{Goldmann, Paul 31.01.1865 – 25.09.1935@\textsc{Goldmann, Paul} (31.01.1865 – 25.09.1935), \emph{Schriftsteller, Journalist}!Kreta, Griechenland und die Maechte [Privattelegramm]1897-03-09@\strich\emph{Kreta, Griechenland und die Mächte [Privattelegramm]} {[}1897-03-09{]}|pwkv}, 10. 3.\pwindex{Goldmann, Paul 31.01.1865 – 25.09.1935@\textsc{Goldmann, Paul} (31.01.1865 – 25.09.1935), \emph{Schriftsteller, Journalist}!Kreta, Griechenland und die Maechte [Privattelegramm]1897-03-10@\strich\emph{Kreta, Griechenland und die Mächte [Privattelegramm]} {[}1897-03-10{]}|pwkv}\pwindex{Goldmann, Paul 31.01.1865 – 25.09.1935@\textsc{Goldmann, Paul} (31.01.1865 – 25.09.1935), \emph{Schriftsteller, Journalist}!Kreta, Griechenland und die Maechte [2 Privattelegramme]1897-03-10@\strich\emph{Kreta, Griechenland und die Mächte [2 Privattelegramme]} {[}1897-03-10{]}|pwkv}, 11. 3.\pwindex{Goldmann, Paul 31.01.1865 – 25.09.1935@\textsc{Goldmann, Paul} (31.01.1865 – 25.09.1935), \emph{Schriftsteller, Journalist}!Kreta, Griechenland und die Maechte1897-03-11@\strich\emph{Kreta, Griechenland und die Mächte} {[}1897-03-11{]}|pwkv}, 13. 3.\pwindex{Goldmann, Paul 31.01.1865 – 25.09.1935@\textsc{Goldmann, Paul} (31.01.1865 – 25.09.1935), \emph{Schriftsteller, Journalist}!Kreta, Griechenland und die Maechte [Privattelegramm]1897-03-13@\strich\emph{Kreta, Griechenland und die Mächte [Privattelegramm]} {[}1897-03-13{]}|pwkv}, 14. 3.\pwindex{Goldmann, Paul 31.01.1865 – 25.09.1935@\textsc{Goldmann, Paul} (31.01.1865 – 25.09.1935), \emph{Schriftsteller, Journalist}!Kreta, Griechenland und die Maechte [Privattelegramm]1897-03-14@\strich\emph{Kreta, Griechenland und die Mächte [Privattelegramm]} {[}1897-03-14{]}|pwkv}, 15. 3.\pwindex{Goldmann, Paul 31.01.1865 – 25.09.1935@\textsc{Goldmann, Paul} (31.01.1865 – 25.09.1935), \emph{Schriftsteller, Journalist}!Kreta, Griechenland und die Maechte [2 Privattelegramme]1897-03-15@\strich\emph{Kreta, Griechenland und die Mächte [2 Privattelegramme]} {[}1897-03-15{]}|pwkv}, 16. 3.\pwindex{Goldmann, Paul 31.01.1865 – 25.09.1935@\textsc{Goldmann, Paul} (31.01.1865 – 25.09.1935), \emph{Schriftsteller, Journalist}!Kreta, Griechenland und die Maechte [Privattelegramm]1897-03-16@\strich\emph{Kreta, Griechenland und die Mächte [Privattelegramm]} {[}1897-03-16{]}|pwkv}\pwindex{Goldmann, Paul 31.01.1865 – 25.09.1935@\textsc{Goldmann, Paul} (31.01.1865 – 25.09.1935), \emph{Schriftsteller, Journalist}!Aktionsprogramm der Maechte [Privattelegramm]1897-03-16@\strich\emph{Das Aktionsprogramm der Mächte [Privattelegramm]} {[}1897-03-16{]}|pwkv}\pwindex{Goldmann, Paul 31.01.1865 – 25.09.1935@\textsc{Goldmann, Paul} (31.01.1865 – 25.09.1935), \emph{Schriftsteller, Journalist}!Kretafrage in der franzoesischen Kammer [Privattelegramm]1897-03-16@\strich\emph{Die Kretafrage in der französischen Kammer [Privattelegramm]} {[}1897-03-16{]}|pwkv}\pwindex{Goldmann, Paul 31.01.1865 – 25.09.1935@\textsc{Goldmann, Paul} (31.01.1865 – 25.09.1935), \emph{Schriftsteller, Journalist}!Kreta, Griechenland und die Maechte [Privattelegramm]1897-03-16@\strich\emph{Kreta, Griechenland und die Mächte [Privattelegramm]} {[}1897-03-16{]}|pwkv} und 17. 3. 1897\pwindex{Goldmann, Paul 31.01.1865 – 25.09.1935@\textsc{Goldmann, Paul} (31.01.1865 – 25.09.1935), \emph{Schriftsteller, Journalist}!Kreta, Griechenland und die Maechte1897-03-17@\strich\emph{Kreta, Griechenland und die Mächte} {[}1897-03-17{]}|pwkv}. Ab Ende März 1897 scheint Goldmann\pwindex{Goldmann, Paul 31.01.1865 – 25.09.1935@\textsc{Goldmann, Paul} (31.01.1865 – 25.09.1935), \emph{Schriftsteller, Journalist}|pwk} nicht mehr – oder nur noch vereinzelt – darüber
                  berichtet und sich stattdessen, insbesondere ab dem 28. 3. 1897, vermehrt der Panama\oindex{Panama@\textbf{Panama}|pwk}-Affäre gewidmet zu haben.}}}\label{K_L02803-1h} und kann Dir heut nur kurz meine Befriedigung über all’ das Beruhigende, das Dein
               lieber Brief enthält, – und mein Entzücken über die Ausſicht melden, Dich hier zu
               haben. Es iſt vielleicht ſehr egoiſtiſch, daß ich in all’ Deinem Kummer nur die große
               Freude ſehe, die für mich herauswächſt. Aber auch Dir wird \textsc{Paris\oindex{Paris@\textbf{Paris}|pw}} gut thun, ich bin deſſen ſicher. {\pb}Du wirſt
               hier Alles von fern und von hoch ſehen und wirſt leicht darüber hinwegkommen – im
               Rauſch eines Pariſ\oindex{Paris@\textbf{Paris}|pw}er Frühlings.\pend
           \pstart
           Wirſt Du \label{K_L02803-2v}\edtext{bald kommen}{\lemma{\textnormal{\emph{bald kommen}}}\Cendnote{\textnormal{Schnitzler\pwindex{Schnitzler, Arthur 15.05.1862 – 21.10.1931@\textsc{Schnitzler, Arthur} (15.05.1862 – 21.10.1931), \emph{Schriftsteller, Mediziner}|pwk} kam am 12. 4. 1897 in Paris\oindex{Paris@\textbf{Paris}|pwk} an.}}}\label{K_L02803-2h}? Es kann geſchehen, daß ich
               Anfang März oder Ende Februar
               auf vierzehn Tage nach der \textsc{Riviera\oindex{Riviera@\textbf{Riviera}|pw}} gehen muß, um \label{K_L02803-3v}\edtext{Saiſon-Feuilletons}{\lemma{\textnormal{\emph{Saiſon-Feuilletons}}}\Cendnote{\textnormal{nicht geschehen,
                     vgl. Paul Goldmann an Arthur Schnitzler, 22. 3. [1897]}}}\label{K_L02803-3h} zu ſchreiben. Wenn ich Dir alſo Wohnung beſorgen ſoll, gib’ mir \uline{umgehend} ſchriftlichen oder telegraphiſchen Auftrag.
               Und laß’ mich nur tüchtig für Dich arbeiten. {\pb}Das
               wird die erſte Pariſ\oindex{Paris@\textbf{Paris}|pw}er Wohnung ſein, die ich mit
               Vergnügen ſuchen werde.\pend
           \pstart
           Nun bleib’ aber auch bei dem Plan. Glaub’ mir, nirgends biſt Du ſo aus der Welt, wie
               in \textsc{Paris\oindex{Paris@\textbf{Paris}|pw}}. Daß Du zugleich zum Genuſſe der Stadt\oindex{Paris@\textbf{Paris}|pwv} kommſt, dafür laß’ mich nur \strikeout{ſor} ſorgen.\pend
           \pstart
           Grüß’ Dich Gott, Liebſter! Laß’ Dich nicht von den äußeren Unannehmlichkeiten
               niederdrücken. »\label{K_L02803-5v}\edtext{\begin{otherlanguage}{french}\textsc{Tout s’arrange}\end{otherlanguage}}{\lemma{\textnormal{\emph{Tout s’arrange}}}\Cendnote{\textnormal{französisch: Alles wird sich
                  richten}}}\label{K_L02803-5h}« ſagt einer meiner hieſigen Freunde, und das iſt wahr. {\pb}Es gibt nur \uline{ein}
               wirkliches Unglück: Die Krankheit. Was von Menſchen kommt, iſt nicht gefährlich.\pend
           \pstart
           Dein treuer {\\[\baselineskip]}\spacefill\mbox{Paul Goldmnn}\pend
           \leftskip=0em{}
         
         \endnumbering\mylabel{h}\end{ledgroupsized}  \newcommand{\dateiname}{L02803}\newcommand{\titel}{Paul Goldmann an Arthur Schnitzler, 16. 2. [1897]}\newcommand{\editorInnen}{Martin Anton Müller und Laura Untner}%% latex-leseansicht-abspann.tex
%% Abspann für die Leseansicht.
%% Der Schalter \ifkorrekturansicht ist bereits durch den Vorspann gesetzt.

%% latex-abspann.tex
%% Gemeinsamer Abspann für Korrekturansicht und Leseansicht.
%% Setzt den Schalter \ifkorrekturansicht voraus (gesetzt in den
%% einbindenden Dateien latex-korrekturansicht-abspann.tex bzw.
%% latex-leseansicht-abspann.tex).
%% ---------------------------------------------------------------

\normalsize

% Das esempio-Environment wird nur in der Leseansicht benötigt
\ifkorrekturansicht\else
\newenvironment{esempio}[3]%
{
    \vspace{1.5ex}
    \rlap{\underline{#1}}
    \par
    \setlength{\parindent}{0cm}
    \nopagebreak
    \leftskip=#2cm
    \rightskip=#3cm
}
{
    \par
}
\fi

\doendnotes{C}
\bigskip
\vfill

\clearpage

\footnotesize

\ifkorrekturansicht
  \lohead{\textsc{register}}
\fi

% theindex-Environment neu definieren ohne reledmac
\makeatletter
\renewenvironment{theindex}{%
  \ifkorrekturansicht
    \section*{\indexname}%
  \else
    \subsubsection*{Index der erwähnten Entitäten}%
  \fi
  \setlength{\parindent}{0pt}%
  \setlength{\parskip}{0pt plus 0.3pt}%
  \let\item\@idxitem
}{%
  \ifkorrekturansicht\clearpage\fi
}
\makeatother

\IfFileExists{\jobname-pw.ind}{\input{\jobname-pw.ind}}{}

% Quellenangabe nur in der Leseansicht
\ifkorrekturansicht\else
% Fallback-Definitionen, falls die .tex-Datei \titel etc. nicht gesetzt hat
\providecommand{\titel}{}
\providecommand{\editorInnen}{}
\providecommand{\dateiname}{\jobname}

\vspace{3cm}

\vfill

\footnotesize
\textsc{Quelle}: \titel. Herausgegeben von {\editorInnen}. In: \emph{Arthur Schnitzler: Briefwechsel mit Autorinnen und Autoren}.
 Digitale Edition, https://schnitzler-briefe.acdh.oeaw.ac.at/{\dateiname}.html (Stand \today)
\fi

\end{document}


      