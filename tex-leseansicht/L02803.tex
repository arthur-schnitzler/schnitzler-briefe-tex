%% latex-leseansicht-vorspann.tex
%% Vorspann für die Leseansicht.
%% Lädt die gemeinsame Datei latex-vorspann.tex mit nicht gesetztem Schalter.

\newif\ifkorrekturansicht
\korrekturansichtfalse

\input{../tex-inputs/latex-vorspann}


\section[ Paul Goldmann an Arthur Schnitzler, 16. 2. [1897]]{L02803 Paul Goldmann an Arthur Schnitzler,  16. 2. [1897]}
\nopagebreak\mylabel{L02803v}
\rehead{ }\normalsize\beginnumbering\briefempfaengerindex{Schnitzler, Arthur@\textsc{Schnitzler, Arthur}!zzzGoldmann, Paul@\emph{von Paul Goldmann}!1897-02-161@{16. 2. [1897]}|(be}
\toendnotes[C]{\smallbreak\pagebreak[2]}
\correspDesc{Versand  durch Paul Goldmann am 16. 2. [1897] in Paris
\newline{}Erhalt  durch Arthur Schnitzler im Zeitraum [17. 2. 1897
                  – 21. 2. 1897?] in Wien}\toendnotes[C]{\smallbreak}
\Standort{DLA, A:Schnitzler, HS.NZ85.1.3167.}
\physDesc{Brief, 1 Blatt, 4 Seiten, 1361 Zeichen
\newline{}Handschrift: blaue Tinte, deutsche Kurrent
\newline{}Schnitzler: mit Bleistift das Jahr »97« vermerkt }\toendnotes[C]{\smallbreak}
\pstart
           {\pb}\textcolor{gray}{\textbf{\textbf{Frankfurter Zeitung\orgindex{Frankfurter Zeitung@Frankfurter Zeitung|pw}}}}\pend
           
\pstart
           \textcolor{gray}{\textbf{(\begin{otherlanguage}{french}Gazette de Francfort\end{otherlanguage}\orgindex{Frankfurter Zeitung@Frankfurter Zeitung|pw}).}}\pend
           
\pstart
           \textcolor{gray}{\textbf{\textbf{\begin{otherlanguage}{french}Fondateur M.\end{otherlanguage}{ }L. Sonnemann\pwindex{Sonnemann, Leopold 29.\,10.\,1831 Höchberg – 30.\,10.\,1909 Frankfurt am Main@\textsc{Sonnemann, Leopold} (29.\,10.\,1831 Höchberg – 30.\,10.\,1909 Frankfurt am Main), \emph{Journalist, Herausgeber}|pw}.}}}\pend
           
\pstart
           \begin{otherlanguage}{french}\textcolor{gray}{\textbf{Journal politique, financier,}}\end{otherlanguage}\pend
           
\pstart
           \begin{otherlanguage}{french}\textcolor{gray}{\textbf{commercial et littéraire.}}\end{otherlanguage}\pend
           
\pstart
           \begin{otherlanguage}{french}\textcolor{gray}{\textbf{\textbf{Paraissant trois fois par jour.}}}\end{otherlanguage}\hfill \textsc{Paris\oindex{Paris@\textbf{Paris}, \emph{Hauptstadt}|pw}}, 16. Februar.\pend
           
\pstart
           \begin{otherlanguage}{french}\textcolor{gray}{\textbf{\textbf{Bureau à Paris\oindex{Paris@\textbf{Paris}, \emph{Hauptstadt}|pw}}}}\end{otherlanguage}\pend
           
\pstart
           \begin{otherlanguage}{french}\textcolor{gray}{\textbf{\textbf{24. Rue Feydeau\oindex{rue Feydeau@\textbf{rue Feydeau}, \emph{Straße}|pw}.}}}\end{otherlanguage}\pend
           
\pstart\center{}Mein lieber Freund,\pend\vspace{0.5em}
\pstart
           Ich{ }ſtecke mitten in den \label{K_L02803-1v}\edtext{\textsc{Kreta\oindex{Kreta@\textbf{Kreta}, \emph{Insel}|pw}}-Geſchichten\pwindex{Goldmann, Paul 31.\,1.\,1865 Breslau – 25.\,9.\,1935 Wien@\textsc{Goldmann, Paul} (31.\,1.\,1865 Breslau – 25.\,9.\,1935 Wien), \emph{Schriftsteller, Journalist}!Ereignisse auf Kreta@\strich\emph{Die Ereignisse auf Kreta}|pwv}\pwindex{Goldmann, Paul 31.\,1.\,1865 Breslau – 25.\,9.\,1935 Wien@\textsc{Goldmann, Paul} (31.\,1.\,1865 Breslau – 25.\,9.\,1935 Wien), \emph{Schriftsteller, Journalist}!Ereignisse auf Kreta [Privattelegramm]@\strich\emph{Die Ereignisse auf Kreta [Privattelegramm]}|pwv}\pwindex{Goldmann, Paul 31.\,1.\,1865 Breslau – 25.\,9.\,1935 Wien@\textsc{Goldmann, Paul} (31.\,1.\,1865 Breslau – 25.\,9.\,1935 Wien), \emph{Schriftsteller, Journalist}!Ereignisse auf Kreta [2 Privattelegramme]@\strich\emph{Die Ereignisse auf Kreta [2 Privattelegramme]}|pwv}\pwindex{Goldmann, Paul 31.\,1.\,1865 Breslau – 25.\,9.\,1935 Wien@\textsc{Goldmann, Paul} (31.\,1.\,1865 Breslau – 25.\,9.\,1935 Wien), \emph{Schriftsteller, Journalist}!Ereignisse auf Kreta [Privattelegramm]@\strich\emph{Die Ereignisse auf Kreta [Privattelegramm]}|pwv}\pwindex{Goldmann, Paul 31.\,1.\,1865 Breslau – 25.\,9.\,1935 Wien@\textsc{Goldmann, Paul} (31.\,1.\,1865 Breslau – 25.\,9.\,1935 Wien), \emph{Schriftsteller, Journalist}!Ereignisse auf Kreta [Privattelegramm]@\strich\emph{Die Ereignisse auf Kreta [Privattelegramm]}|pwv}\pwindex{Goldmann, Paul 31.\,1.\,1865 Breslau – 25.\,9.\,1935 Wien@\textsc{Goldmann, Paul} (31.\,1.\,1865 Breslau – 25.\,9.\,1935 Wien), \emph{Schriftsteller, Journalist}!Ereignisse auf Kreta [Privattelegramm]@\strich\emph{Die Ereignisse auf Kreta [Privattelegramm]}|pwv}\pwindex{Goldmann, Paul 31.\,1.\,1865 Breslau – 25.\,9.\,1935 Wien@\textsc{Goldmann, Paul} (31.\,1.\,1865 Breslau – 25.\,9.\,1935 Wien), \emph{Schriftsteller, Journalist}!Ereignisse auf Kreta [Privattelegramm]@\strich\emph{Die Ereignisse auf Kreta [Privattelegramm]}|pwv}\pwindex{Goldmann, Paul 31.\,1.\,1865 Breslau – 25.\,9.\,1935 Wien@\textsc{Goldmann, Paul} (31.\,1.\,1865 Breslau – 25.\,9.\,1935 Wien), \emph{Schriftsteller, Journalist}!Ereignisse auf Kreta [Privattelegramm]@\strich\emph{Die Ereignisse auf Kreta [Privattelegramm]}|pwv}\pwindex{Goldmann, Paul 31.\,1.\,1865 Breslau – 25.\,9.\,1935 Wien@\textsc{Goldmann, Paul} (31.\,1.\,1865 Breslau – 25.\,9.\,1935 Wien), \emph{Schriftsteller, Journalist}!Ereignisse auf Kreta [2 Privattelegramme]@\strich\emph{Die Ereignisse auf Kreta [2 Privattelegramme]}|pwv}\pwindex{Goldmann, Paul 31.\,1.\,1865 Breslau – 25.\,9.\,1935 Wien@\textsc{Goldmann, Paul} (31.\,1.\,1865 Breslau – 25.\,9.\,1935 Wien), \emph{Schriftsteller, Journalist}!Ereignisse auf Kreta [2 Privattelegramme]@\strich\emph{Die Ereignisse auf Kreta [2 Privattelegramme]}|pwv}\pwindex{Goldmann, Paul 31.\,1.\,1865 Breslau – 25.\,9.\,1935 Wien@\textsc{Goldmann, Paul} (31.\,1.\,1865 Breslau – 25.\,9.\,1935 Wien), \emph{Schriftsteller, Journalist}!Ereignisse auf Kreta [2 Privattelegramme]@\strich\emph{Die Ereignisse auf Kreta [2 Privattelegramme]}|pwv}\pwindex{Goldmann, Paul 31.\,1.\,1865 Breslau – 25.\,9.\,1935 Wien@\textsc{Goldmann, Paul} (31.\,1.\,1865 Breslau – 25.\,9.\,1935 Wien), \emph{Schriftsteller, Journalist}!Occupation Kreta’s [3 Privattelegramme]@\strich\emph{Die Occupation Kreta’s [3 Privattelegramme]}|pwv}\pwindex{Goldmann, Paul 31.\,1.\,1865 Breslau – 25.\,9.\,1935 Wien@\textsc{Goldmann, Paul} (31.\,1.\,1865 Breslau – 25.\,9.\,1935 Wien), \emph{Schriftsteller, Journalist}!Occupation Kreta’s [Privattelegramm]@\strich\emph{Die Occupation Kreta’s [Privattelegramm]}|pwv}\pwindex{Goldmann, Paul 31.\,1.\,1865 Breslau – 25.\,9.\,1935 Wien@\textsc{Goldmann, Paul} (31.\,1.\,1865 Breslau – 25.\,9.\,1935 Wien), \emph{Schriftsteller, Journalist}!Kämpfe auf Kreta. [Privattelegramm]@\strich\emph{Die Kämpfe auf Kreta. [Privattelegramm]}|pwv}\pwindex{Goldmann, Paul 31.\,1.\,1865 Breslau – 25.\,9.\,1935 Wien@\textsc{Goldmann, Paul} (31.\,1.\,1865 Breslau – 25.\,9.\,1935 Wien), \emph{Schriftsteller, Journalist}!Ereignisse auf Kreta [Privattelegramm]@\strich\emph{Die Ereignisse auf Kreta [Privattelegramm]}|pwv}\pwindex{Goldmann, Paul 31.\,1.\,1865 Breslau – 25.\,9.\,1935 Wien@\textsc{Goldmann, Paul} (31.\,1.\,1865 Breslau – 25.\,9.\,1935 Wien), \emph{Schriftsteller, Journalist}!Ereignisse auf Kreta [Privattelegramm]@\strich\emph{Die Ereignisse auf Kreta [Privattelegramm]}|pwv}\pwindex{Goldmann, Paul 31.\,1.\,1865 Breslau – 25.\,9.\,1935 Wien@\textsc{Goldmann, Paul} (31.\,1.\,1865 Breslau – 25.\,9.\,1935 Wien), \emph{Schriftsteller, Journalist}!Ereignisse auf Kreta [Privattelegramm]@\strich\emph{Die Ereignisse auf Kreta [Privattelegramm]}|pwv}\pwindex{Goldmann, Paul 31.\,1.\,1865 Breslau – 25.\,9.\,1935 Wien@\textsc{Goldmann, Paul} (31.\,1.\,1865 Breslau – 25.\,9.\,1935 Wien), \emph{Schriftsteller, Journalist}!kretische Frage und die Mächte [Privattelegramm]@\strich\emph{Die kretische Frage und die Mächte [Privattelegramm]}|pwv}\pwindex{Goldmann, Paul 31.\,1.\,1865 Breslau – 25.\,9.\,1935 Wien@\textsc{Goldmann, Paul} (31.\,1.\,1865 Breslau – 25.\,9.\,1935 Wien), \emph{Schriftsteller, Journalist}!Kreta, Griechenland und die Mächte [Bericht und Privattelegramm]@\strich\emph{Kreta, Griechenland und die Mächte [Bericht und Privattelegramm]}|pwv}\pwindex{Goldmann, Paul 31.\,1.\,1865 Breslau – 25.\,9.\,1935 Wien@\textsc{Goldmann, Paul} (31.\,1.\,1865 Breslau – 25.\,9.\,1935 Wien), \emph{Schriftsteller, Journalist}!Kreta, Griechenland und die Mächte [Privattelegramm]@\strich\emph{Kreta, Griechenland und die Mächte [Privattelegramm]}|pwv}\pwindex{Goldmann, Paul 31.\,1.\,1865 Breslau – 25.\,9.\,1935 Wien@\textsc{Goldmann, Paul} (31.\,1.\,1865 Breslau – 25.\,9.\,1935 Wien), \emph{Schriftsteller, Journalist}!Kretafrage und die Uneinigkeit der Mächte [Privattelegramm]@\strich\emph{Die Kretafrage und die Uneinigkeit der Mächte [Privattelegramm]}|pwv}\pwindex{Goldmann, Paul 31.\,1.\,1865 Breslau – 25.\,9.\,1935 Wien@\textsc{Goldmann, Paul} (31.\,1.\,1865 Breslau – 25.\,9.\,1935 Wien), \emph{Schriftsteller, Journalist}!Zu den Wirren auf Kreta [Privattelegramm]@\strich\emph{Zu den Wirren auf Kreta [Privattelegramm]}|pwv}\pwindex{Goldmann, Paul 31.\,1.\,1865 Breslau – 25.\,9.\,1935 Wien@\textsc{Goldmann, Paul} (31.\,1.\,1865 Breslau – 25.\,9.\,1935 Wien), \emph{Schriftsteller, Journalist}!griechischen Minister@\strich\emph{Die griechischen Minister}|pwv}\pwindex{Goldmann, Paul 31.\,1.\,1865 Breslau – 25.\,9.\,1935 Wien@\textsc{Goldmann, Paul} (31.\,1.\,1865 Breslau – 25.\,9.\,1935 Wien), \emph{Schriftsteller, Journalist}!Kreta, Griechenland und die Mächte [3 Privattelegramme]@\strich\emph{Kreta, Griechenland und die Mächte [3 Privattelegramme]}|pwv}\pwindex{Goldmann, Paul 31.\,1.\,1865 Breslau – 25.\,9.\,1935 Wien@\textsc{Goldmann, Paul} (31.\,1.\,1865 Breslau – 25.\,9.\,1935 Wien), \emph{Schriftsteller, Journalist}!Kreta, Griechenland und die Mächte [Privattelegramm]@\strich\emph{Kreta, Griechenland und die Mächte [Privattelegramm]}|pwv}\pwindex{Goldmann, Paul 31.\,1.\,1865 Breslau – 25.\,9.\,1935 Wien@\textsc{Goldmann, Paul} (31.\,1.\,1865 Breslau – 25.\,9.\,1935 Wien), \emph{Schriftsteller, Journalist}!Interpellation Cochins in der französischen Kammer [Privattelegramm]@\strich\emph{Die Interpellation Cochins in der französischen Kammer [Privattelegramm]}|pwv}\pwindex{Goldmann, Paul 31.\,1.\,1865 Breslau – 25.\,9.\,1935 Wien@\textsc{Goldmann, Paul} (31.\,1.\,1865 Breslau – 25.\,9.\,1935 Wien), \emph{Schriftsteller, Journalist}!Kreta, Griechenland und die Mächte [Privattelegramm]@\strich\emph{Kreta, Griechenland und die Mächte [Privattelegramm]}|pwv}\pwindex{Goldmann, Paul 31.\,1.\,1865 Breslau – 25.\,9.\,1935 Wien@\textsc{Goldmann, Paul} (31.\,1.\,1865 Breslau – 25.\,9.\,1935 Wien), \emph{Schriftsteller, Journalist}!Kreta, Griechenland und die Mächte [2 Privattelegramme]@\strich\emph{Kreta, Griechenland und die Mächte [2 Privattelegramme]}|pwv}\pwindex{Goldmann, Paul 31.\,1.\,1865 Breslau – 25.\,9.\,1935 Wien@\textsc{Goldmann, Paul} (31.\,1.\,1865 Breslau – 25.\,9.\,1935 Wien), \emph{Schriftsteller, Journalist}!Kreta, Griechenland und die Mächte [Privattelegramm]@\strich\emph{Kreta, Griechenland und die Mächte [Privattelegramm]}|pwv}\pwindex{Goldmann, Paul 31.\,1.\,1865 Breslau – 25.\,9.\,1935 Wien@\textsc{Goldmann, Paul} (31.\,1.\,1865 Breslau – 25.\,9.\,1935 Wien), \emph{Schriftsteller, Journalist}!Kreta, Griechenland und die Mächte [Privattelegramm]@\strich\emph{Kreta, Griechenland und die Mächte [Privattelegramm]}|pwv}\pwindex{Goldmann, Paul 31.\,1.\,1865 Breslau – 25.\,9.\,1935 Wien@\textsc{Goldmann, Paul} (31.\,1.\,1865 Breslau – 25.\,9.\,1935 Wien), \emph{Schriftsteller, Journalist}!Kreta, Griechenland und die Mächte@\strich\emph{Kreta, Griechenland und die Mächte}|pwv}\pwindex{Goldmann, Paul 31.\,1.\,1865 Breslau – 25.\,9.\,1935 Wien@\textsc{Goldmann, Paul} (31.\,1.\,1865 Breslau – 25.\,9.\,1935 Wien), \emph{Schriftsteller, Journalist}!Kreta, Griechenland und die Mächte [2 Privattelegramme]@\strich\emph{Kreta, Griechenland und die Mächte [2 Privattelegramme]}|pwv}\pwindex{Goldmann, Paul 31.\,1.\,1865 Breslau – 25.\,9.\,1935 Wien@\textsc{Goldmann, Paul} (31.\,1.\,1865 Breslau – 25.\,9.\,1935 Wien), \emph{Schriftsteller, Journalist}!Kreta, Griechenland und die Mächte [Privattelegramm]@\strich\emph{Kreta, Griechenland und die Mächte [Privattelegramm]}|pwv}\pwindex{Goldmann, Paul 31.\,1.\,1865 Breslau – 25.\,9.\,1935 Wien@\textsc{Goldmann, Paul} (31.\,1.\,1865 Breslau – 25.\,9.\,1935 Wien), \emph{Schriftsteller, Journalist}!Kreta, Griechenland und die Mächte [Privattelegramm]@\strich\emph{Kreta, Griechenland und die Mächte [Privattelegramm]}|pwv}\pwindex{Goldmann, Paul 31.\,1.\,1865 Breslau – 25.\,9.\,1935 Wien@\textsc{Goldmann, Paul} (31.\,1.\,1865 Breslau – 25.\,9.\,1935 Wien), \emph{Schriftsteller, Journalist}!Kreta, Griechenland und die Mächte [Privattelegramm]@\strich\emph{Kreta, Griechenland und die Mächte [Privattelegramm]}|pwv}\pwindex{Goldmann, Paul 31.\,1.\,1865 Breslau – 25.\,9.\,1935 Wien@\textsc{Goldmann, Paul} (31.\,1.\,1865 Breslau – 25.\,9.\,1935 Wien), \emph{Schriftsteller, Journalist}!Kreta, Griechenland und die Mächte [Privattelegramm]@\strich\emph{Kreta, Griechenland und die Mächte [Privattelegramm]}|pwv}\pwindex{Goldmann, Paul 31.\,1.\,1865 Breslau – 25.\,9.\,1935 Wien@\textsc{Goldmann, Paul} (31.\,1.\,1865 Breslau – 25.\,9.\,1935 Wien), \emph{Schriftsteller, Journalist}!Kreta, Griechenland und die Mächte [Privattelegramm]@\strich\emph{Kreta, Griechenland und die Mächte [Privattelegramm]}|pwv}\pwindex{Goldmann, Paul 31.\,1.\,1865 Breslau – 25.\,9.\,1935 Wien@\textsc{Goldmann, Paul} (31.\,1.\,1865 Breslau – 25.\,9.\,1935 Wien), \emph{Schriftsteller, Journalist}!Kreta, Griechenland und die Mächte [2 Privattelegramme]@\strich\emph{Kreta, Griechenland und die Mächte [2 Privattelegramme]}|pwv}\pwindex{Goldmann, Paul 31.\,1.\,1865 Breslau – 25.\,9.\,1935 Wien@\textsc{Goldmann, Paul} (31.\,1.\,1865 Breslau – 25.\,9.\,1935 Wien), \emph{Schriftsteller, Journalist}!Kreta, Griechenland und die Mächte@\strich\emph{Kreta, Griechenland und die Mächte}|pwv}\pwindex{Goldmann, Paul 31.\,1.\,1865 Breslau – 25.\,9.\,1935 Wien@\textsc{Goldmann, Paul} (31.\,1.\,1865 Breslau – 25.\,9.\,1935 Wien), \emph{Schriftsteller, Journalist}!Kreta, Griechenland und die Mächte [Privattelegramm]@\strich\emph{Kreta, Griechenland und die Mächte [Privattelegramm]}|pwv}\pwindex{Goldmann, Paul 31.\,1.\,1865 Breslau – 25.\,9.\,1935 Wien@\textsc{Goldmann, Paul} (31.\,1.\,1865 Breslau – 25.\,9.\,1935 Wien), \emph{Schriftsteller, Journalist}!Kreta, Griechenland und die Mächte [Privattelegramm]@\strich\emph{Kreta, Griechenland und die Mächte [Privattelegramm]}|pwv}\pwindex{Goldmann, Paul 31.\,1.\,1865 Breslau – 25.\,9.\,1935 Wien@\textsc{Goldmann, Paul} (31.\,1.\,1865 Breslau – 25.\,9.\,1935 Wien), \emph{Schriftsteller, Journalist}!Kreta, Griechenland und die Mächte [2 Privattelegramme]@\strich\emph{Kreta, Griechenland und die Mächte [2 Privattelegramme]}|pwv}\pwindex{Goldmann, Paul 31.\,1.\,1865 Breslau – 25.\,9.\,1935 Wien@\textsc{Goldmann, Paul} (31.\,1.\,1865 Breslau – 25.\,9.\,1935 Wien), \emph{Schriftsteller, Journalist}!Kreta, Griechenland und die Mächte [Privattelegramm]@\strich\emph{Kreta, Griechenland und die Mächte [Privattelegramm]}|pwv}\pwindex{Goldmann, Paul 31.\,1.\,1865 Breslau – 25.\,9.\,1935 Wien@\textsc{Goldmann, Paul} (31.\,1.\,1865 Breslau – 25.\,9.\,1935 Wien), \emph{Schriftsteller, Journalist}!Aktionsprogramm der Mächte [Privattelegramm]@\strich\emph{Das Aktionsprogramm der Mächte [Privattelegramm]}|pwv}\pwindex{Goldmann, Paul 31.\,1.\,1865 Breslau – 25.\,9.\,1935 Wien@\textsc{Goldmann, Paul} (31.\,1.\,1865 Breslau – 25.\,9.\,1935 Wien), \emph{Schriftsteller, Journalist}!Kretafrage in der französischen Kammer [Privattelegramm]@\strich\emph{Die Kretafrage in der französischen Kammer [Privattelegramm]}|pwv}\pwindex{Goldmann, Paul 31.\,1.\,1865 Breslau – 25.\,9.\,1935 Wien@\textsc{Goldmann, Paul} (31.\,1.\,1865 Breslau – 25.\,9.\,1935 Wien), \emph{Schriftsteller, Journalist}!Kreta, Griechenland und die Mächte [Privattelegramm]@\strich\emph{Kreta, Griechenland und die Mächte [Privattelegramm]}|pwv}\pwindex{Goldmann, Paul 31.\,1.\,1865 Breslau – 25.\,9.\,1935 Wien@\textsc{Goldmann, Paul} (31.\,1.\,1865 Breslau – 25.\,9.\,1935 Wien), \emph{Schriftsteller, Journalist}!Kreta, Griechenland und die Mächte@\strich\emph{Kreta, Griechenland und die Mächte}|pwv}}{\lemma{\textnormal{\emph{Kreta-Geschichten}}}\Cendnote{\textnormal{Am 6. 2. 1897 waren erste griech\oindex{Griechenland@\textbf{Griechenland}|pwk}ische Kriegsschiffe auf Kreta\oindex{Kreta@\textbf{Kreta}, \emph{Insel}|pwk}
                  gelandet, um die unzufriedene Bevölkerung gegen die türk\oindex{Türkei@\textbf{Türkei}|pwkv}ische Regierung zu unterstützen. In
                  Folge kam es zwischen 18. 4. und 20. 5. 1897 zum Türk\oindex{Türkei@\textbf{Türkei}|pwk}isch-Griech\oindex{Griechenland@\textbf{Griechenland}|pwk}ischen Krieg. Goldmann\pwindex{Goldmann, Paul 31.\,1.\,1865 Breslau – 25.\,9.\,1935 Wien@\textsc{Goldmann, Paul} (31.\,1.\,1865 Breslau – 25.\,9.\,1935 Wien), \emph{Schriftsteller, Journalist}|pwk} berichtete darüber in der \emph{Frankfurter Zeitung}\pwindex{Frankfurter Zeitung@\emph{Frankfurter Zeitung}|pwk} (manchmal mehrmals täglich
                  und zumeist in der Form von Privattelegrammen) am 10. 2. 1897\pwindex{Goldmann, Paul 31.\,1.\,1865 Breslau – 25.\,9.\,1935 Wien@\textsc{Goldmann, Paul} (31.\,1.\,1865 Breslau – 25.\,9.\,1935 Wien), \emph{Schriftsteller, Journalist}!Ereignisse auf Kreta@\strich\emph{Die Ereignisse auf Kreta}|pwkv}, 11. 2. 1897\pwindex{Goldmann, Paul 31.\,1.\,1865 Breslau – 25.\,9.\,1935 Wien@\textsc{Goldmann, Paul} (31.\,1.\,1865 Breslau – 25.\,9.\,1935 Wien), \emph{Schriftsteller, Journalist}!Ereignisse auf Kreta [Privattelegramm]@\strich\emph{Die Ereignisse auf Kreta [Privattelegramm]}|pwkv}, 12. 2. 1897\pwindex{Goldmann, Paul 31.\,1.\,1865 Breslau – 25.\,9.\,1935 Wien@\textsc{Goldmann, Paul} (31.\,1.\,1865 Breslau – 25.\,9.\,1935 Wien), \emph{Schriftsteller, Journalist}!Ereignisse auf Kreta [2 Privattelegramme]@\strich\emph{Die Ereignisse auf Kreta [2 Privattelegramme]}|pwkv}\pwindex{Goldmann, Paul 31.\,1.\,1865 Breslau – 25.\,9.\,1935 Wien@\textsc{Goldmann, Paul} (31.\,1.\,1865 Breslau – 25.\,9.\,1935 Wien), \emph{Schriftsteller, Journalist}!Ereignisse auf Kreta [Privattelegramm]@\strich\emph{Die Ereignisse auf Kreta [Privattelegramm]}|pwkv}, 13. 2. 1897\pwindex{Goldmann, Paul 31.\,1.\,1865 Breslau – 25.\,9.\,1935 Wien@\textsc{Goldmann, Paul} (31.\,1.\,1865 Breslau – 25.\,9.\,1935 Wien), \emph{Schriftsteller, Journalist}!Ereignisse auf Kreta [Privattelegramm]@\strich\emph{Die Ereignisse auf Kreta [Privattelegramm]}|pwkv}\pwindex{Goldmann, Paul 31.\,1.\,1865 Breslau – 25.\,9.\,1935 Wien@\textsc{Goldmann, Paul} (31.\,1.\,1865 Breslau – 25.\,9.\,1935 Wien), \emph{Schriftsteller, Journalist}!Ereignisse auf Kreta [Privattelegramm]@\strich\emph{Die Ereignisse auf Kreta [Privattelegramm]}|pwkv}, 14. 2. 1897\pwindex{Goldmann, Paul 31.\,1.\,1865 Breslau – 25.\,9.\,1935 Wien@\textsc{Goldmann, Paul} (31.\,1.\,1865 Breslau – 25.\,9.\,1935 Wien), \emph{Schriftsteller, Journalist}!Ereignisse auf Kreta [Privattelegramm]@\strich\emph{Die Ereignisse auf Kreta [Privattelegramm]}|pwkv}, 15. 2. 1897\pwindex{Goldmann, Paul 31.\,1.\,1865 Breslau – 25.\,9.\,1935 Wien@\textsc{Goldmann, Paul} (31.\,1.\,1865 Breslau – 25.\,9.\,1935 Wien), \emph{Schriftsteller, Journalist}!Ereignisse auf Kreta [Privattelegramm]@\strich\emph{Die Ereignisse auf Kreta [Privattelegramm]}|pwkv}\pwindex{Goldmann, Paul 31.\,1.\,1865 Breslau – 25.\,9.\,1935 Wien@\textsc{Goldmann, Paul} (31.\,1.\,1865 Breslau – 25.\,9.\,1935 Wien), \emph{Schriftsteller, Journalist}!Ereignisse auf Kreta [2 Privattelegramme]@\strich\emph{Die Ereignisse auf Kreta [2 Privattelegramme]}|pwkv}, 16. 2. 1897\pwindex{Goldmann, Paul 31.\,1.\,1865 Breslau – 25.\,9.\,1935 Wien@\textsc{Goldmann, Paul} (31.\,1.\,1865 Breslau – 25.\,9.\,1935 Wien), \emph{Schriftsteller, Journalist}!Ereignisse auf Kreta [2 Privattelegramme]@\strich\emph{Die Ereignisse auf Kreta [2 Privattelegramme]}|pwkv}\pwindex{Goldmann, Paul 31.\,1.\,1865 Breslau – 25.\,9.\,1935 Wien@\textsc{Goldmann, Paul} (31.\,1.\,1865 Breslau – 25.\,9.\,1935 Wien), \emph{Schriftsteller, Journalist}!Ereignisse auf Kreta [2 Privattelegramme]@\strich\emph{Die Ereignisse auf Kreta [2 Privattelegramme]}|pwkv}\pwindex{Goldmann, Paul 31.\,1.\,1865 Breslau – 25.\,9.\,1935 Wien@\textsc{Goldmann, Paul} (31.\,1.\,1865 Breslau – 25.\,9.\,1935 Wien), \emph{Schriftsteller, Journalist}!Occupation Kreta’s [3 Privattelegramme]@\strich\emph{Die Occupation Kreta’s [3 Privattelegramme]}|pwkv}, 17. 2. 1897\pwindex{Goldmann, Paul 31.\,1.\,1865 Breslau – 25.\,9.\,1935 Wien@\textsc{Goldmann, Paul} (31.\,1.\,1865 Breslau – 25.\,9.\,1935 Wien), \emph{Schriftsteller, Journalist}!Occupation Kreta’s [Privattelegramm]@\strich\emph{Die Occupation Kreta’s [Privattelegramm]}|pwkv}\pwindex{Goldmann, Paul 31.\,1.\,1865 Breslau – 25.\,9.\,1935 Wien@\textsc{Goldmann, Paul} (31.\,1.\,1865 Breslau – 25.\,9.\,1935 Wien), \emph{Schriftsteller, Journalist}!Kämpfe auf Kreta. [Privattelegramm]@\strich\emph{Die Kämpfe auf Kreta. [Privattelegramm]}|pwkv}, 18. 2. 1897\pwindex{Goldmann, Paul 31.\,1.\,1865 Breslau – 25.\,9.\,1935 Wien@\textsc{Goldmann, Paul} (31.\,1.\,1865 Breslau – 25.\,9.\,1935 Wien), \emph{Schriftsteller, Journalist}!Ereignisse auf Kreta [Privattelegramm]@\strich\emph{Die Ereignisse auf Kreta [Privattelegramm]}|pwkv}\pwindex{Goldmann, Paul 31.\,1.\,1865 Breslau – 25.\,9.\,1935 Wien@\textsc{Goldmann, Paul} (31.\,1.\,1865 Breslau – 25.\,9.\,1935 Wien), \emph{Schriftsteller, Journalist}!Ereignisse auf Kreta [Privattelegramm]@\strich\emph{Die Ereignisse auf Kreta [Privattelegramm]}|pwkv}\pwindex{Goldmann, Paul 31.\,1.\,1865 Breslau – 25.\,9.\,1935 Wien@\textsc{Goldmann, Paul} (31.\,1.\,1865 Breslau – 25.\,9.\,1935 Wien), \emph{Schriftsteller, Journalist}!Ereignisse auf Kreta [Privattelegramm]@\strich\emph{Die Ereignisse auf Kreta [Privattelegramm]}|pwkv}, 19. 2. 1897\pwindex{Goldmann, Paul 31.\,1.\,1865 Breslau – 25.\,9.\,1935 Wien@\textsc{Goldmann, Paul} (31.\,1.\,1865 Breslau – 25.\,9.\,1935 Wien), \emph{Schriftsteller, Journalist}!kretische Frage und die Mächte [Privattelegramm]@\strich\emph{Die kretische Frage und die Mächte [Privattelegramm]}|pwkv}\pwindex{Goldmann, Paul 31.\,1.\,1865 Breslau – 25.\,9.\,1935 Wien@\textsc{Goldmann, Paul} (31.\,1.\,1865 Breslau – 25.\,9.\,1935 Wien), \emph{Schriftsteller, Journalist}!Kreta, Griechenland und die Mächte [Bericht und Privattelegramm]@\strich\emph{Kreta, Griechenland und die Mächte [Bericht und Privattelegramm]}|pwkv}, 20. 2. 1897\pwindex{Goldmann, Paul 31.\,1.\,1865 Breslau – 25.\,9.\,1935 Wien@\textsc{Goldmann, Paul} (31.\,1.\,1865 Breslau – 25.\,9.\,1935 Wien), \emph{Schriftsteller, Journalist}!Kreta, Griechenland und die Mächte [Privattelegramm]@\strich\emph{Kreta, Griechenland und die Mächte [Privattelegramm]}|pwkv}\pwindex{Goldmann, Paul 31.\,1.\,1865 Breslau – 25.\,9.\,1935 Wien@\textsc{Goldmann, Paul} (31.\,1.\,1865 Breslau – 25.\,9.\,1935 Wien), \emph{Schriftsteller, Journalist}!Kretafrage und die Uneinigkeit der Mächte [Privattelegramm]@\strich\emph{Die Kretafrage und die Uneinigkeit der Mächte [Privattelegramm]}|pwkv}\pwindex{Goldmann, Paul 31.\,1.\,1865 Breslau – 25.\,9.\,1935 Wien@\textsc{Goldmann, Paul} (31.\,1.\,1865 Breslau – 25.\,9.\,1935 Wien), \emph{Schriftsteller, Journalist}!Zu den Wirren auf Kreta [Privattelegramm]@\strich\emph{Zu den Wirren auf Kreta [Privattelegramm]}|pwkv}\pwindex{Goldmann, Paul 31.\,1.\,1865 Breslau – 25.\,9.\,1935 Wien@\textsc{Goldmann, Paul} (31.\,1.\,1865 Breslau – 25.\,9.\,1935 Wien), \emph{Schriftsteller, Journalist}!griechischen Minister@\strich\emph{Die griechischen Minister}|pwkv}, 22. 2. 1897\pwindex{Goldmann, Paul 31.\,1.\,1865 Breslau – 25.\,9.\,1935 Wien@\textsc{Goldmann, Paul} (31.\,1.\,1865 Breslau – 25.\,9.\,1935 Wien), \emph{Schriftsteller, Journalist}!Kreta, Griechenland und die Mächte [3 Privattelegramme]@\strich\emph{Kreta, Griechenland und die Mächte [3 Privattelegramme]}|pwkv}, 23. 2. 1897\pwindex{Goldmann, Paul 31.\,1.\,1865 Breslau – 25.\,9.\,1935 Wien@\textsc{Goldmann, Paul} (31.\,1.\,1865 Breslau – 25.\,9.\,1935 Wien), \emph{Schriftsteller, Journalist}!Kreta, Griechenland und die Mächte [Privattelegramm]@\strich\emph{Kreta, Griechenland und die Mächte [Privattelegramm]}|pwkv}\pwindex{Goldmann, Paul 31.\,1.\,1865 Breslau – 25.\,9.\,1935 Wien@\textsc{Goldmann, Paul} (31.\,1.\,1865 Breslau – 25.\,9.\,1935 Wien), \emph{Schriftsteller, Journalist}!Interpellation Cochins in der französischen Kammer [Privattelegramm]@\strich\emph{Die Interpellation Cochins in der französischen Kammer [Privattelegramm]}|pwkv}, 24. 2. 1897\pwindex{Goldmann, Paul 31.\,1.\,1865 Breslau – 25.\,9.\,1935 Wien@\textsc{Goldmann, Paul} (31.\,1.\,1865 Breslau – 25.\,9.\,1935 Wien), \emph{Schriftsteller, Journalist}!Kreta, Griechenland und die Mächte [Privattelegramm]@\strich\emph{Kreta, Griechenland und die Mächte [Privattelegramm]}|pwkv}, 26. 2. 1897\pwindex{Goldmann, Paul 31.\,1.\,1865 Breslau – 25.\,9.\,1935 Wien@\textsc{Goldmann, Paul} (31.\,1.\,1865 Breslau – 25.\,9.\,1935 Wien), \emph{Schriftsteller, Journalist}!Kreta, Griechenland und die Mächte [2 Privattelegramme]@\strich\emph{Kreta, Griechenland und die Mächte [2 Privattelegramme]}|pwkv}, 5. 3. 1897\pwindex{Goldmann, Paul 31.\,1.\,1865 Breslau – 25.\,9.\,1935 Wien@\textsc{Goldmann, Paul} (31.\,1.\,1865 Breslau – 25.\,9.\,1935 Wien), \emph{Schriftsteller, Journalist}!Kreta, Griechenland und die Mächte [Privattelegramm]@\strich\emph{Kreta, Griechenland und die Mächte [Privattelegramm]}|pwkv}, 6. 3. 1897\pwindex{Goldmann, Paul 31.\,1.\,1865 Breslau – 25.\,9.\,1935 Wien@\textsc{Goldmann, Paul} (31.\,1.\,1865 Breslau – 25.\,9.\,1935 Wien), \emph{Schriftsteller, Journalist}!Kreta, Griechenland und die Mächte [Privattelegramm]@\strich\emph{Kreta, Griechenland und die Mächte [Privattelegramm]}|pwkv}\pwindex{Goldmann, Paul 31.\,1.\,1865 Breslau – 25.\,9.\,1935 Wien@\textsc{Goldmann, Paul} (31.\,1.\,1865 Breslau – 25.\,9.\,1935 Wien), \emph{Schriftsteller, Journalist}!Kreta, Griechenland und die Mächte@\strich\emph{Kreta, Griechenland und die Mächte}|pwkv}, 7. 3. 1897\pwindex{Goldmann, Paul 31.\,1.\,1865 Breslau – 25.\,9.\,1935 Wien@\textsc{Goldmann, Paul} (31.\,1.\,1865 Breslau – 25.\,9.\,1935 Wien), \emph{Schriftsteller, Journalist}!Kreta, Griechenland und die Mächte [2 Privattelegramme]@\strich\emph{Kreta, Griechenland und die Mächte [2 Privattelegramme]}|pwkv}\pwindex{Goldmann, Paul 31.\,1.\,1865 Breslau – 25.\,9.\,1935 Wien@\textsc{Goldmann, Paul} (31.\,1.\,1865 Breslau – 25.\,9.\,1935 Wien), \emph{Schriftsteller, Journalist}!Kreta, Griechenland und die Mächte [Privattelegramm]@\strich\emph{Kreta, Griechenland und die Mächte [Privattelegramm]}|pwkv}, 8. 3. 1897\pwindex{Goldmann, Paul 31.\,1.\,1865 Breslau – 25.\,9.\,1935 Wien@\textsc{Goldmann, Paul} (31.\,1.\,1865 Breslau – 25.\,9.\,1935 Wien), \emph{Schriftsteller, Journalist}!Kreta, Griechenland und die Mächte [Privattelegramm]@\strich\emph{Kreta, Griechenland und die Mächte [Privattelegramm]}|pwkv}, 9. 3. 1897\pwindex{Goldmann, Paul 31.\,1.\,1865 Breslau – 25.\,9.\,1935 Wien@\textsc{Goldmann, Paul} (31.\,1.\,1865 Breslau – 25.\,9.\,1935 Wien), \emph{Schriftsteller, Journalist}!Kreta, Griechenland und die Mächte [Privattelegramm]@\strich\emph{Kreta, Griechenland und die Mächte [Privattelegramm]}|pwkv}\pwindex{Goldmann, Paul 31.\,1.\,1865 Breslau – 25.\,9.\,1935 Wien@\textsc{Goldmann, Paul} (31.\,1.\,1865 Breslau – 25.\,9.\,1935 Wien), \emph{Schriftsteller, Journalist}!Kreta, Griechenland und die Mächte [Privattelegramm]@\strich\emph{Kreta, Griechenland und die Mächte [Privattelegramm]}|pwkv}, 10. 3. 1897\pwindex{Goldmann, Paul 31.\,1.\,1865 Breslau – 25.\,9.\,1935 Wien@\textsc{Goldmann, Paul} (31.\,1.\,1865 Breslau – 25.\,9.\,1935 Wien), \emph{Schriftsteller, Journalist}!Kreta, Griechenland und die Mächte [Privattelegramm]@\strich\emph{Kreta, Griechenland und die Mächte [Privattelegramm]}|pwkv}\pwindex{Goldmann, Paul 31.\,1.\,1865 Breslau – 25.\,9.\,1935 Wien@\textsc{Goldmann, Paul} (31.\,1.\,1865 Breslau – 25.\,9.\,1935 Wien), \emph{Schriftsteller, Journalist}!Kreta, Griechenland und die Mächte [2 Privattelegramme]@\strich\emph{Kreta, Griechenland und die Mächte [2 Privattelegramme]}|pwkv}, 11. 3. 1897\pwindex{Goldmann, Paul 31.\,1.\,1865 Breslau – 25.\,9.\,1935 Wien@\textsc{Goldmann, Paul} (31.\,1.\,1865 Breslau – 25.\,9.\,1935 Wien), \emph{Schriftsteller, Journalist}!Kreta, Griechenland und die Mächte@\strich\emph{Kreta, Griechenland und die Mächte}|pwkv}, 13. 3. 1897\pwindex{Goldmann, Paul 31.\,1.\,1865 Breslau – 25.\,9.\,1935 Wien@\textsc{Goldmann, Paul} (31.\,1.\,1865 Breslau – 25.\,9.\,1935 Wien), \emph{Schriftsteller, Journalist}!Kreta, Griechenland und die Mächte [Privattelegramm]@\strich\emph{Kreta, Griechenland und die Mächte [Privattelegramm]}|pwkv}, 14. 3. 1897\pwindex{Goldmann, Paul 31.\,1.\,1865 Breslau – 25.\,9.\,1935 Wien@\textsc{Goldmann, Paul} (31.\,1.\,1865 Breslau – 25.\,9.\,1935 Wien), \emph{Schriftsteller, Journalist}!Kreta, Griechenland und die Mächte [Privattelegramm]@\strich\emph{Kreta, Griechenland und die Mächte [Privattelegramm]}|pwkv}, 15. 3. 1897\pwindex{Goldmann, Paul 31.\,1.\,1865 Breslau – 25.\,9.\,1935 Wien@\textsc{Goldmann, Paul} (31.\,1.\,1865 Breslau – 25.\,9.\,1935 Wien), \emph{Schriftsteller, Journalist}!Kreta, Griechenland und die Mächte [2 Privattelegramme]@\strich\emph{Kreta, Griechenland und die Mächte [2 Privattelegramme]}|pwkv}, 16. 3. 1897\pwindex{Goldmann, Paul 31.\,1.\,1865 Breslau – 25.\,9.\,1935 Wien@\textsc{Goldmann, Paul} (31.\,1.\,1865 Breslau – 25.\,9.\,1935 Wien), \emph{Schriftsteller, Journalist}!Kreta, Griechenland und die Mächte [Privattelegramm]@\strich\emph{Kreta, Griechenland und die Mächte [Privattelegramm]}|pwkv}\pwindex{Goldmann, Paul 31.\,1.\,1865 Breslau – 25.\,9.\,1935 Wien@\textsc{Goldmann, Paul} (31.\,1.\,1865 Breslau – 25.\,9.\,1935 Wien), \emph{Schriftsteller, Journalist}!Aktionsprogramm der Mächte [Privattelegramm]@\strich\emph{Das Aktionsprogramm der Mächte [Privattelegramm]}|pwkv}\pwindex{Goldmann, Paul 31.\,1.\,1865 Breslau – 25.\,9.\,1935 Wien@\textsc{Goldmann, Paul} (31.\,1.\,1865 Breslau – 25.\,9.\,1935 Wien), \emph{Schriftsteller, Journalist}!Kretafrage in der französischen Kammer [Privattelegramm]@\strich\emph{Die Kretafrage in der französischen Kammer [Privattelegramm]}|pwkv}\pwindex{Goldmann, Paul 31.\,1.\,1865 Breslau – 25.\,9.\,1935 Wien@\textsc{Goldmann, Paul} (31.\,1.\,1865 Breslau – 25.\,9.\,1935 Wien), \emph{Schriftsteller, Journalist}!Kreta, Griechenland und die Mächte [Privattelegramm]@\strich\emph{Kreta, Griechenland und die Mächte [Privattelegramm]}|pwkv} und 17. 3. 1897\pwindex{Goldmann, Paul 31.\,1.\,1865 Breslau – 25.\,9.\,1935 Wien@\textsc{Goldmann, Paul} (31.\,1.\,1865 Breslau – 25.\,9.\,1935 Wien), \emph{Schriftsteller, Journalist}!Kreta, Griechenland und die Mächte@\strich\emph{Kreta, Griechenland und die Mächte}|pwkv}. Ab Ende März 1897 scheint Goldmann\pwindex{Goldmann, Paul 31.\,1.\,1865 Breslau – 25.\,9.\,1935 Wien@\textsc{Goldmann, Paul} (31.\,1.\,1865 Breslau – 25.\,9.\,1935 Wien), \emph{Schriftsteller, Journalist}|pwk} nicht mehr – oder nur noch vereinzelt – darüber
                  berichtet und sich stattdessen, insbesondere ab dem 28. 3. 1897, vermehrt der Panama\oindex{Panama@\textbf{Panama}|pwk}-Affäre gewidmet zu haben.}}}\label{K_L02803-1} und kann Dir heut nur kurz meine Befriedigung über all’ das Beruhigende, das Dein
               lieber Brief enthält, – und mein Entzücken über die Ausſicht melden, Dich hier zu
               haben. Es iſt vielleicht{ }ſehr egoiſtiſch, daß ich in all’ Deinem Kummer nur die große
               Freude{ }ſehe, die für mich herauswächſt. Aber auch Dir wird \textsc{Paris\oindex{Paris@\textbf{Paris}, \emph{Hauptstadt}|pw}} gut thun, ich bin deſſen{ }ſicher. {\pb}Du wirſt
               hier Alles von fern und von hoch{ }ſehen und wirſt leicht darüber hinwegkommen – im
               Rauſch eines Pariſ\oindex{Paris@\textbf{Paris}, \emph{Hauptstadt}|pw}er Frühlings.\pend
           
\pstart
           Wirſt Du \label{K_L02803-2v}\edtext{bald kommen}{\lemma{\textnormal{\emph{bald kommen}}}\Cendnote{\textnormal{Schnitzler kam am 12. 4. 1897 in Paris\oindex{Paris@\textbf{Paris}, \emph{Hauptstadt}|pwk} an.}}}\label{K_L02803-2}? Es kann geſchehen, daß ich
               Anfang März oder Ende Februar
               auf vierzehn Tage nach der \textsc{Riviera\oindex{Riviera@\textbf{Riviera}|pw}} gehen muß, um \label{K_L02803-3v}\edtext{Saiſon-Feuilletons}{\lemma{\textnormal{\emph{Saison-Feuilletons}}}\Cendnote{\textnormal{Dazu kam es nicht,
                     vgl. XXXX Auszeichnungsfehler: Dokument L02806 nicht gefunden.
               }}}\label{K_L02803-3} zu{ }ſchreiben. Wenn ich Dir alſo Wohnung beſorgen{ }ſoll, gib’ mir \uline{umgehend}{ }ſchriftlichen oder telegraphiſchen Auftrag.
               Und laß’ mich nur tüchtig für Dich arbeiten. {\pb}Das
               wird die erſte Pariſ\oindex{Paris@\textbf{Paris}, \emph{Hauptstadt}|pw}er Wohnung{ }ſein, die ich mit
               Vergnügen{ }ſuchen werde.\pend
           
\pstart
           Nun bleib’ aber auch bei dem Plan. Glaub’ mir, nirgends biſt Du{ }ſo aus der Welt, wie
               in \textsc{Paris\oindex{Paris@\textbf{Paris}, \emph{Hauptstadt}|pw}}. Daß Du zugleich zum Genuſſe der Stadt\oindex{Paris@\textbf{Paris}, \emph{Hauptstadt}|pwv} kommſt, dafür laß’ mich nur \strikeout{ſor}{ }ſorgen.\pend
           
\pstart
           Grüß’ Dich Gott, Liebſter! Laß’ Dich nicht von den äußeren Unannehmlichkeiten
               niederdrücken. »\label{K_L02803-4v}\edtext{\begin{otherlanguage}{french}\textsc{Tout s’arrange}\end{otherlanguage}}{\lemma{\textnormal{\emph{Tout s’arrange}}}\Cendnote{\textnormal{französisch: Alles wird sich
                  richten}}}\label{K_L02803-4}«{ }ſagt einer meiner hieſigen Freunde, und das iſt wahr. {\pb}Es gibt nur \uline{ein}
               wirkliches Unglück: Die Krankheit. Was von Menſchen kommt, iſt nicht gefährlich.\pend
           
\pstart
           Dein treuer {\\[\baselineskip]}\spacefill\mbox{Paul Goldmnn}\pend
           \leftskip=0em{}\selectlanguage{ngerman}\endnumbering\briefempfaengerindex{Schnitzler, Arthur@\textsc{Schnitzler, Arthur}!zzzGoldmann, Paul@\emph{von Paul Goldmann}!1897-02-161@{16. 2. [1897]}|)be}\mylabel{L02803h}  \newcommand{\dateiname}{L02803}\newcommand{\titel}{Paul Goldmann an Arthur Schnitzler, 16. 2. [1897]}\newcommand{\editorInnen}{Martin Anton Müller und Laura Untner}%% latex-leseansicht-abspann.tex
%% Abspann für die Leseansicht.
%% Der Schalter \ifkorrekturansicht ist bereits durch den Vorspann gesetzt.

%% latex-abspann.tex
%% Gemeinsamer Abspann für Korrekturansicht und Leseansicht.
%% Setzt den Schalter \ifkorrekturansicht voraus (gesetzt in den
%% einbindenden Dateien latex-korrekturansicht-abspann.tex bzw.
%% latex-leseansicht-abspann.tex).
%% ---------------------------------------------------------------

\normalsize

% Das esempio-Environment wird nur in der Leseansicht benötigt
\ifkorrekturansicht\else
\newenvironment{esempio}[3]%
{
    \vspace{1.5ex}
    \rlap{\underline{#1}}
    \par
    \setlength{\parindent}{0cm}
    \nopagebreak
    \leftskip=#2cm
    \rightskip=#3cm
}
{
    \par
}
\fi

\doendnotes{C}
\bigskip
\vfill

\clearpage

\footnotesize

\ifkorrekturansicht
  \lohead{\textsc{register}}
\fi

% theindex-Environment neu definieren ohne reledmac
\makeatletter
\renewenvironment{theindex}{%
  \ifkorrekturansicht
    \section*{\indexname}%
  \else
    \subsubsection*{Index der erwähnten Entitäten}%
  \fi
  \setlength{\parindent}{0pt}%
  \setlength{\parskip}{0pt plus 0.3pt}%
  \let\item\@idxitem
}{%
  \ifkorrekturansicht\clearpage\fi
}
\makeatother

\IfFileExists{\jobname-pw.ind}{\input{\jobname-pw.ind}}{}

% Quellenangabe nur in der Leseansicht
\ifkorrekturansicht\else
% Fallback-Definitionen, falls die .tex-Datei \titel etc. nicht gesetzt hat
\providecommand{\titel}{}
\providecommand{\editorInnen}{}
\providecommand{\dateiname}{\jobname}

\vspace{3cm}

\vfill

\footnotesize
\textsc{Quelle}: \titel. Herausgegeben von {\editorInnen}. In: \emph{Arthur Schnitzler: Briefwechsel mit Autorinnen und Autoren}.
 Digitale Edition, https://schnitzler-briefe.acdh.oeaw.ac.at/{\dateiname}.html (Stand \today)
\fi

\end{document}


