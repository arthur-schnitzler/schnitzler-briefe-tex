%% latex-korrekturansicht-vorspann.tex
%% Vorspann für die Korrekturansicht.
%% Lädt die gemeinsame Datei latex-vorspann.tex mit gesetztem Schalter.

\newif\ifkorrekturansicht
\korrekturansichttrue

\input{../tex-inputs/latex-vorspann}


\section[ Paul Goldmann an Arthur Schnitzler, 16. 2. {[}1897{]}]{L02803 Paul Goldmann an Arthur Schnitzler, 16. 2. {[}1897{]}}
\nopagebreak\mylabel{L02803v}
\rehead{ }\normalsize\beginnumbering\briefempfaengerindex{Schnitzler, Arthur@\textsc{Schnitzler, Arthur}!zzzGoldmann, Paul@\emph{von Paul Goldmann}!1897-02-161@{16. 2. {[}1897{]}}|(be}
\toendnotes[C]{\smallbreak\pagebreak[2]}\Standort{DLA, A:Schnitzler, HS.NZ85.1.3167.}
\physDesc{Brief, 1 Blatt, 4 Seiten, 1361 Zeichen
\newline{}Handschrift: blaue Tinte, deutsche Kurrent
\newline{}Schnitzler: mit Bleistift das Jahr »97« vermerkt }\toendnotes[C]{\smallbreak}
\pstart
           {\pb}\textcolor{gray}{\textbf{\textbf{Frankfurter Zeitung\orgindex{Frankfurter Zeitung@Frankfurter Zeitung|pw}}}}\pend
           
\pstart
           \textcolor{gray}{\textbf{(\begin{otherlanguage}{french}Gazette de Francfort\end{otherlanguage}\orgindex{Frankfurter Zeitung@Frankfurter Zeitung|pw}).}}\pend
           
\pstart
           \textcolor{gray}{\textbf{\textbf{\begin{otherlanguage}{french}Fondateur M.\end{otherlanguage}{ }L. Sonnemann\pwindex{Sonnemann, Leopold 1831-10-29 – 1909-10-30@\textsc{Sonnemann, Leopold} (1831-10-29 – 1909-10-30), \emph{Journalist/Journalistin, Herausgeber/Herausgeberin}|pw}.}}}\pend
           
\pstart
           \begin{otherlanguage}{french}\textcolor{gray}{\textbf{Journal politique, financier,}}\end{otherlanguage}\pend
           
\pstart
           \begin{otherlanguage}{french}\textcolor{gray}{\textbf{commercial et littéraire.}}\end{otherlanguage}\pend
           
\pstart
           \begin{otherlanguage}{french}\textcolor{gray}{\textbf{\textbf{Paraissant trois fois par jour.}}}\end{otherlanguage}\hfill \textsc{Paris\oindex{Paris@\textbf{Paris}, \emph{P.PPLC}|pw}}, 16. Februar.\pend
           
\pstart
           \begin{otherlanguage}{french}\textcolor{gray}{\textbf{\textbf{Bureau à Paris\oindex{Paris@\textbf{Paris}, \emph{P.PPLC}|pw}}}}\end{otherlanguage}\pend
           
\pstart
           \begin{otherlanguage}{french}\textcolor{gray}{\textbf{\textbf{24. Rue Feydeau\oindex{rue Feydeau@\textbf{rue Feydeau}, \emph{Straße (K.STR)}|pw}.}}}\end{otherlanguage}\pend
           
\pstart\center{}Mein lieber Freund,\pend\vspace{0.5em}
\pstart
           Ich ſtecke mitten in den \label{K_L02803-1v}\edtext{\textsc{Kreta\oindex{Kreta@\textbf{Kreta}, \emph{T.ISL}|pw}}-Geſchichten\pwindex{Ereignisse auf Kreta@\emph{Die Ereignisse auf Kreta}|pwv}\pwindex{Ereignisse auf Kreta [Privattelegramm]@\emph{Die Ereignisse auf Kreta [Privattelegramm]}|pwv}\pwindex{Ereignisse auf Kreta [2 Privattelegramme]@\emph{Die Ereignisse auf Kreta [2 Privattelegramme]}|pwv}\pwindex{Ereignisse auf Kreta [Privattelegramm]@\emph{Die Ereignisse auf Kreta [Privattelegramm]}|pwv}\pwindex{Ereignisse auf Kreta [Privattelegramm]@\emph{Die Ereignisse auf Kreta [Privattelegramm]}|pwv}\pwindex{Ereignisse auf Kreta [Privattelegramm]@\emph{Die Ereignisse auf Kreta [Privattelegramm]}|pwv}\pwindex{Ereignisse auf Kreta [Privattelegramm]@\emph{Die Ereignisse auf Kreta [Privattelegramm]}|pwv}\pwindex{Ereignisse auf Kreta [Privattelegramm]@\emph{Die Ereignisse auf Kreta [Privattelegramm]}|pwv}\pwindex{Ereignisse auf Kreta [2 Privattelegramme]@\emph{Die Ereignisse auf Kreta [2 Privattelegramme]}|pwv}\pwindex{Ereignisse auf Kreta [2 Privattelegramme]@\emph{Die Ereignisse auf Kreta [2 Privattelegramme]}|pwv}\pwindex{Ereignisse auf Kreta [2 Privattelegramme]@\emph{Die Ereignisse auf Kreta [2 Privattelegramme]}|pwv}\pwindex{Occupation Kreta s [3 Privattelegramme]@\emph{Die Occupation Kreta’s [3 Privattelegramme]}|pwv}\pwindex{Occupation Kreta s [Privattelegramm]@\emph{Die Occupation Kreta’s [Privattelegramm]}|pwv}\pwindex{Kaempfe auf Kreta. [Privattelegramm]@\emph{Die Kämpfe auf Kreta. [Privattelegramm]}|pwv}\pwindex{Ereignisse auf Kreta [Privattelegramm]@\emph{Die Ereignisse auf Kreta [Privattelegramm]}|pwv}\pwindex{Ereignisse auf Kreta [Privattelegramm]@\emph{Die Ereignisse auf Kreta [Privattelegramm]}|pwv}\pwindex{Ereignisse auf Kreta [Privattelegramm]@\emph{Die Ereignisse auf Kreta [Privattelegramm]}|pwv}\pwindex{kretische Frage und die Maechte [Privattelegramm]@\emph{Die kretische Frage und die Mächte [Privattelegramm]}|pwv}\pwindex{Kreta, Griechenland und die Maechte [Bericht und Privattelegramm]@\emph{Kreta, Griechenland und die Mächte [Bericht und Privattelegramm]}|pwv}\pwindex{Kreta, Griechenland und die Maechte [Privattelegramm]@\emph{Kreta, Griechenland und die Mächte [Privattelegramm]}|pwv}\pwindex{Kretafrage und die Uneinigkeit der Maechte [Privattelegramm]@\emph{Die Kretafrage und die Uneinigkeit der Mächte [Privattelegramm]}|pwv}\pwindex{Zu den Wirren auf Kreta [Privattelegramm]@\emph{Zu den Wirren auf Kreta [Privattelegramm]}|pwv}\pwindex{griechischen Minister@\emph{Die griechischen Minister}|pwv}\pwindex{Kreta, Griechenland und die Maechte [3 Privattelegramme]@\emph{Kreta, Griechenland und die Mächte [3 Privattelegramme]}|pwv}\pwindex{Kreta, Griechenland und die Maechte [Privattelegramm]@\emph{Kreta, Griechenland und die Mächte [Privattelegramm]}|pwv}\pwindex{Interpellation Cochins in der franzoesischen Kammer [Privattelegramm]@\emph{Die Interpellation Cochins in der französischen Kammer [Privattelegramm]}|pwv}\pwindex{Kreta, Griechenland und die Maechte [Privattelegramm]@\emph{Kreta, Griechenland und die Mächte [Privattelegramm]}|pwv}\pwindex{Kreta, Griechenland und die Maechte [2 Privattelegramme]@\emph{Kreta, Griechenland und die Mächte [2 Privattelegramme]}|pwv}\pwindex{Kreta, Griechenland und die Maechte [Privattelegramm]@\emph{Kreta, Griechenland und die Mächte [Privattelegramm]}|pwv}\pwindex{Kreta, Griechenland und die Maechte [Privattelegramm]@\emph{Kreta, Griechenland und die Mächte [Privattelegramm]}|pwv}\pwindex{Kreta, Griechenland und die Maechte@\emph{Kreta, Griechenland und die Mächte}|pwv}\pwindex{Kreta, Griechenland und die Maechte [2 Privattelegramme]@\emph{Kreta, Griechenland und die Mächte [2 Privattelegramme]}|pwv}\pwindex{Kreta, Griechenland und die Maechte [Privattelegramm]@\emph{Kreta, Griechenland und die Mächte [Privattelegramm]}|pwv}\pwindex{Kreta, Griechenland und die Maechte [Privattelegramm]@\emph{Kreta, Griechenland und die Mächte [Privattelegramm]}|pwv}\pwindex{Kreta, Griechenland und die Maechte [Privattelegramm]@\emph{Kreta, Griechenland und die Mächte [Privattelegramm]}|pwv}\pwindex{Kreta, Griechenland und die Maechte [Privattelegramm]@\emph{Kreta, Griechenland und die Mächte [Privattelegramm]}|pwv}\pwindex{Kreta, Griechenland und die Maechte [Privattelegramm]@\emph{Kreta, Griechenland und die Mächte [Privattelegramm]}|pwv}\pwindex{Kreta, Griechenland und die Maechte [2 Privattelegramme]@\emph{Kreta, Griechenland und die Mächte [2 Privattelegramme]}|pwv}\pwindex{Kreta, Griechenland und die Maechte@\emph{Kreta, Griechenland und die Mächte}|pwv}\pwindex{Kreta, Griechenland und die Maechte [Privattelegramm]@\emph{Kreta, Griechenland und die Mächte [Privattelegramm]}|pwv}\pwindex{Kreta, Griechenland und die Maechte [Privattelegramm]@\emph{Kreta, Griechenland und die Mächte [Privattelegramm]}|pwv}\pwindex{Kreta, Griechenland und die Maechte [2 Privattelegramme]@\emph{Kreta, Griechenland und die Mächte [2 Privattelegramme]}|pwv}\pwindex{Kreta, Griechenland und die Maechte [Privattelegramm]@\emph{Kreta, Griechenland und die Mächte [Privattelegramm]}|pwv}\pwindex{Aktionsprogramm der Maechte [Privattelegramm]@\emph{Das Aktionsprogramm der Mächte [Privattelegramm]}|pwv}\pwindex{Kretafrage in der franzoesischen Kammer [Privattelegramm]@\emph{Die Kretafrage in der französischen Kammer [Privattelegramm]}|pwv}\pwindex{Kreta, Griechenland und die Maechte [Privattelegramm]@\emph{Kreta, Griechenland und die Mächte [Privattelegramm]}|pwv}\pwindex{Kreta, Griechenland und die Maechte@\emph{Kreta, Griechenland und die Mächte}|pwv}}{\lemma{\textnormal{\emph{Kreta-Geſchichten}}}\Cendnote{\textnormal{Am 6. 2. 1897 waren erste griech\oindex{Griechenland@\textbf{Griechenland}, \emph{A.PCLI}|pwk}ische Kriegsschiffe auf Kreta\oindex{Kreta@\textbf{Kreta}, \emph{T.ISL}|pwk}
                  gelandet, um die unzufriedene Bevölkerung gegen die türk\oindex{Tuerkei@\textbf{Türkei}, \emph{A.PCLI}|pwkv}ische Regierung zu unterstützen. In
                  Folge kam es zwischen 18. 4. und 20. 5. 1897 zum Türk\oindex{Tuerkei@\textbf{Türkei}, \emph{A.PCLI}|pwk}isch-Griech\oindex{Griechenland@\textbf{Griechenland}, \emph{A.PCLI}|pwk}ischen Krieg. Goldmann\pwindex{Goldmann, Paul 31.01.1865 – 25.09.1935@\textsc{Goldmann, Paul} (31.01.1865 – 25.09.1935), \emph{Schriftsteller/Schriftstellerin, Journalist/Journalistin}|pwk} berichtete darüber in der \emph{Frankfurter Zeitung}\pwindex{Frankfurter Zeitung@\emph{Frankfurter Zeitung}|pwk} (manchmal mehrmals täglich
                  und zumeist in der Form von Privattelegrammen) am 10. 2. 1897\pwindex{Ereignisse auf Kreta@\emph{Die Ereignisse auf Kreta}|pwkv}, 11. 2. 1897\pwindex{Ereignisse auf Kreta [Privattelegramm]@\emph{Die Ereignisse auf Kreta [Privattelegramm]}|pwkv}, 12. 2. 1897\pwindex{Ereignisse auf Kreta [2 Privattelegramme]@\emph{Die Ereignisse auf Kreta [2 Privattelegramme]}|pwkv}\pwindex{Ereignisse auf Kreta [Privattelegramm]@\emph{Die Ereignisse auf Kreta [Privattelegramm]}|pwkv}, 13. 2. 1897\pwindex{Ereignisse auf Kreta [Privattelegramm]@\emph{Die Ereignisse auf Kreta [Privattelegramm]}|pwkv}\pwindex{Ereignisse auf Kreta [Privattelegramm]@\emph{Die Ereignisse auf Kreta [Privattelegramm]}|pwkv}, 14. 2. 1897\pwindex{Ereignisse auf Kreta [Privattelegramm]@\emph{Die Ereignisse auf Kreta [Privattelegramm]}|pwkv}, 15. 2. 1897\pwindex{Ereignisse auf Kreta [Privattelegramm]@\emph{Die Ereignisse auf Kreta [Privattelegramm]}|pwkv}\pwindex{Ereignisse auf Kreta [2 Privattelegramme]@\emph{Die Ereignisse auf Kreta [2 Privattelegramme]}|pwkv}, 16. 2. 1897\pwindex{Ereignisse auf Kreta [2 Privattelegramme]@\emph{Die Ereignisse auf Kreta [2 Privattelegramme]}|pwkv}\pwindex{Ereignisse auf Kreta [2 Privattelegramme]@\emph{Die Ereignisse auf Kreta [2 Privattelegramme]}|pwkv}\pwindex{Occupation Kreta s [3 Privattelegramme]@\emph{Die Occupation Kreta’s [3 Privattelegramme]}|pwkv}, 17. 2. 1897\pwindex{Occupation Kreta s [Privattelegramm]@\emph{Die Occupation Kreta’s [Privattelegramm]}|pwkv}\pwindex{Kaempfe auf Kreta. [Privattelegramm]@\emph{Die Kämpfe auf Kreta. [Privattelegramm]}|pwkv}, 18. 2. 1897\pwindex{Ereignisse auf Kreta [Privattelegramm]@\emph{Die Ereignisse auf Kreta [Privattelegramm]}|pwkv}\pwindex{Ereignisse auf Kreta [Privattelegramm]@\emph{Die Ereignisse auf Kreta [Privattelegramm]}|pwkv}\pwindex{Ereignisse auf Kreta [Privattelegramm]@\emph{Die Ereignisse auf Kreta [Privattelegramm]}|pwkv}, 19. 2. 1897\pwindex{kretische Frage und die Maechte [Privattelegramm]@\emph{Die kretische Frage und die Mächte [Privattelegramm]}|pwkv}\pwindex{Kreta, Griechenland und die Maechte [Bericht und Privattelegramm]@\emph{Kreta, Griechenland und die Mächte [Bericht und Privattelegramm]}|pwkv}, 20. 2. 1897\pwindex{Kreta, Griechenland und die Maechte [Privattelegramm]@\emph{Kreta, Griechenland und die Mächte [Privattelegramm]}|pwkv}\pwindex{Kretafrage und die Uneinigkeit der Maechte [Privattelegramm]@\emph{Die Kretafrage und die Uneinigkeit der Mächte [Privattelegramm]}|pwkv}\pwindex{Zu den Wirren auf Kreta [Privattelegramm]@\emph{Zu den Wirren auf Kreta [Privattelegramm]}|pwkv}\pwindex{griechischen Minister@\emph{Die griechischen Minister}|pwkv}, 22. 2. 1897\pwindex{Kreta, Griechenland und die Maechte [3 Privattelegramme]@\emph{Kreta, Griechenland und die Mächte [3 Privattelegramme]}|pwkv}, 23. 2. 1897\pwindex{Kreta, Griechenland und die Maechte [Privattelegramm]@\emph{Kreta, Griechenland und die Mächte [Privattelegramm]}|pwkv}\pwindex{Interpellation Cochins in der franzoesischen Kammer [Privattelegramm]@\emph{Die Interpellation Cochins in der französischen Kammer [Privattelegramm]}|pwkv}, 24. 2. 1897\pwindex{Kreta, Griechenland und die Maechte [Privattelegramm]@\emph{Kreta, Griechenland und die Mächte [Privattelegramm]}|pwkv}, 26. 2. 1897\pwindex{Kreta, Griechenland und die Maechte [2 Privattelegramme]@\emph{Kreta, Griechenland und die Mächte [2 Privattelegramme]}|pwkv}, 5. 3. 1897\pwindex{Kreta, Griechenland und die Maechte [Privattelegramm]@\emph{Kreta, Griechenland und die Mächte [Privattelegramm]}|pwkv}, 6. 3. 1897\pwindex{Kreta, Griechenland und die Maechte [Privattelegramm]@\emph{Kreta, Griechenland und die Mächte [Privattelegramm]}|pwkv}\pwindex{Kreta, Griechenland und die Maechte@\emph{Kreta, Griechenland und die Mächte}|pwkv}, 7. 3. 1897\pwindex{Kreta, Griechenland und die Maechte [2 Privattelegramme]@\emph{Kreta, Griechenland und die Mächte [2 Privattelegramme]}|pwkv}\pwindex{Kreta, Griechenland und die Maechte [Privattelegramm]@\emph{Kreta, Griechenland und die Mächte [Privattelegramm]}|pwkv}, 8. 3. 1897\pwindex{Kreta, Griechenland und die Maechte [Privattelegramm]@\emph{Kreta, Griechenland und die Mächte [Privattelegramm]}|pwkv}, 9. 3. 1897\pwindex{Kreta, Griechenland und die Maechte [Privattelegramm]@\emph{Kreta, Griechenland und die Mächte [Privattelegramm]}|pwkv}\pwindex{Kreta, Griechenland und die Maechte [Privattelegramm]@\emph{Kreta, Griechenland und die Mächte [Privattelegramm]}|pwkv}, 10. 3. 1897\pwindex{Kreta, Griechenland und die Maechte [Privattelegramm]@\emph{Kreta, Griechenland und die Mächte [Privattelegramm]}|pwkv}\pwindex{Kreta, Griechenland und die Maechte [2 Privattelegramme]@\emph{Kreta, Griechenland und die Mächte [2 Privattelegramme]}|pwkv}, 11. 3. 1897\pwindex{Kreta, Griechenland und die Maechte@\emph{Kreta, Griechenland und die Mächte}|pwkv}, 13. 3. 1897\pwindex{Kreta, Griechenland und die Maechte [Privattelegramm]@\emph{Kreta, Griechenland und die Mächte [Privattelegramm]}|pwkv}, 14. 3. 1897\pwindex{Kreta, Griechenland und die Maechte [Privattelegramm]@\emph{Kreta, Griechenland und die Mächte [Privattelegramm]}|pwkv}, 15. 3. 1897\pwindex{Kreta, Griechenland und die Maechte [2 Privattelegramme]@\emph{Kreta, Griechenland und die Mächte [2 Privattelegramme]}|pwkv}, 16. 3. 1897\pwindex{Kreta, Griechenland und die Maechte [Privattelegramm]@\emph{Kreta, Griechenland und die Mächte [Privattelegramm]}|pwkv}\pwindex{Aktionsprogramm der Maechte [Privattelegramm]@\emph{Das Aktionsprogramm der Mächte [Privattelegramm]}|pwkv}\pwindex{Kretafrage in der franzoesischen Kammer [Privattelegramm]@\emph{Die Kretafrage in der französischen Kammer [Privattelegramm]}|pwkv}\pwindex{Kreta, Griechenland und die Maechte [Privattelegramm]@\emph{Kreta, Griechenland und die Mächte [Privattelegramm]}|pwkv} und 17. 3. 1897\pwindex{Kreta, Griechenland und die Maechte@\emph{Kreta, Griechenland und die Mächte}|pwkv}. Ab Ende März 1897 scheint Goldmann\pwindex{Goldmann, Paul 31.01.1865 – 25.09.1935@\textsc{Goldmann, Paul} (31.01.1865 – 25.09.1935), \emph{Schriftsteller/Schriftstellerin, Journalist/Journalistin}|pwk} nicht mehr – oder nur noch vereinzelt – darüber
                  berichtet und sich stattdessen, insbesondere ab dem 28. 3. 1897, vermehrt der Panama\oindex{Panama@\textbf{Panama}, \emph{A.PCLI}|pwk}-Affäre gewidmet zu haben.}}}\label{K_L02803-1} und kann Dir heut nur kurz meine Befriedigung über all’ das Beruhigende, das Dein
               lieber Brief enthält, – und mein Entzücken über die Ausſicht melden, Dich hier zu
               haben. Es iſt vielleicht ſehr egoiſtiſch, daß ich in all’ Deinem Kummer nur die große
               Freude ſehe, die für mich herauswächſt. Aber auch Dir wird \textsc{Paris\oindex{Paris@\textbf{Paris}, \emph{P.PPLC}|pw}} gut thun, ich bin deſſen ſicher. {\pb}Du wirſt
               hier Alles von fern und von hoch ſehen und wirſt leicht darüber hinwegkommen – im
               Rauſch eines Pariſ\oindex{Paris@\textbf{Paris}, \emph{P.PPLC}|pw}er Frühlings.\pend
           
\pstart
           Wirſt Du \label{K_L02803-2v}\edtext{bald kommen}{\lemma{\textnormal{\emph{bald kommen}}}\Cendnote{\textnormal{Schnitzler kam am 12. 4. 1897 in Paris\oindex{Paris@\textbf{Paris}, \emph{P.PPLC}|pwk} an.}}}\label{K_L02803-2}? Es kann geſchehen, daß ich
               Anfang März oder Ende Februar
               auf vierzehn Tage nach der \textsc{Riviera\oindex{Riviera@\textbf{Riviera}, \emph{Strand (N.STR)}|pw}} gehen muß, um \label{K_L02803-3v}\edtext{Saiſon-Feuilletons}{\lemma{\textnormal{\emph{Saiſon-Feuilletons}}}\Cendnote{\textnormal{Dazu kam es nicht,
                     vgl. Paul Goldmann an Arthur Schnitzler, 22. 3. [1897].
               }}}\label{K_L02803-3} zu ſchreiben. Wenn ich Dir alſo Wohnung beſorgen ſoll, gib’ mir \uline{umgehend} ſchriftlichen oder telegraphiſchen Auftrag.
               Und laß’ mich nur tüchtig für Dich arbeiten. {\pb}Das
               wird die erſte Pariſ\oindex{Paris@\textbf{Paris}, \emph{P.PPLC}|pw}er Wohnung ſein, die ich mit
               Vergnügen ſuchen werde.\pend
           
\pstart
           Nun bleib’ aber auch bei dem Plan. Glaub’ mir, nirgends biſt Du ſo aus der Welt, wie
               in \textsc{Paris\oindex{Paris@\textbf{Paris}, \emph{P.PPLC}|pw}}. Daß Du zugleich zum Genuſſe der Stadt\oindex{Paris@\textbf{Paris}, \emph{P.PPLC}|pwv} kommſt, dafür laß’ mich nur \strikeout{ſor} ſorgen.\pend
           
\pstart
           Grüß’ Dich Gott, Liebſter! Laß’ Dich nicht von den äußeren Unannehmlichkeiten
               niederdrücken. »\label{K_L02803-4v}\edtext{\begin{otherlanguage}{french}\textsc{Tout s’arrange}\end{otherlanguage}}{\lemma{\textnormal{\emph{Tout s’arrange}}}\Cendnote{\textnormal{französisch: Alles wird sich
                  richten}}}\label{K_L02803-4}« ſagt einer meiner hieſigen Freunde, und das iſt wahr. {\pb}Es gibt nur \uline{ein}
               wirkliches Unglück: Die Krankheit. Was von Menſchen kommt, iſt nicht gefährlich.\pend
           
\pstart
           Dein treuer {\\[\baselineskip]}\spacefill\mbox{Paul Goldmnn}\pend
           \leftskip=0em{}\selectlanguage{ngerman}\endnumbering\briefempfaengerindex{Schnitzler, Arthur@\textsc{Schnitzler, Arthur}!zzzGoldmann, Paul@\emph{von Paul Goldmann}!1897-02-161@{16. 2. {[}1897{]}}|)be}\mylabel{L02803h}  \normalsize

\doendnotes{C}
\bigskip
\vfill

\clearpage

\footnotesize

\lohead{\textsc{register}}

% Definiere theindex-Environment komplett neu ohne reledmac
\makeatletter
\renewenvironment{theindex}{%
  \section*{\indexname}%
  \setlength{\parindent}{0pt}%
  \setlength{\parskip}{0pt plus 0.3pt}%
  \let\item\@idxitem
}{%
  \clearpage
}
\makeatother

\IfFileExists{\jobname-pw.ind}{\input{\jobname-pw.ind}}{}

\end{document}

      