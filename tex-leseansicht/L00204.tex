%% latex-korrekturansicht-vorspann.tex
%% Vorspann für die Korrekturansicht.
%% Lädt die gemeinsame Datei latex-vorspann.tex mit gesetztem Schalter.

\newif\ifkorrekturansicht
\korrekturansichttrue

\input{../tex-inputs/latex-vorspann}


\section[Arthur Schnitzler an Richard Beer-Hofmann, {[}29. 4. 1893?{]}]{L00204 Arthur Schnitzler an Richard Beer-Hofmann, {[}29. 4. 1893?{]}}
\nopagebreak\mylabel{L00204v}
\rehead{ }\normalsize\beginnumbering\briefempfaengerindex{Beer-Hofmann, Richard@\textsc{Beer-Hofmann, Richard}!zzzSchnitzler, Arthur@\emph{von Arthur Schnitzler}!1893-04-292@{{[}29. 4. 1893?{]}}|(be}
\toendnotes[C]{\smallbreak\pagebreak[2]}\Standort{YCGL, MSS 31.}
\physDesc{Brief, 1 Blatt, 3 Seiten, Umschlag, 672 Zeichen
\newline{}Handschrift: Bleistift, deutsche Kurrent
\newline{}Versand: ohne postalischen Übermittlungsvermerk }
\buchAbdrucke{\weitereDrucke{Arthur Schnitzler, Richard Beer-Hofmann: \emph{Briefwechsel 1891–1931}. Wien, Zürich: \emph{Europaverlag} 1992, S. 44.} }\toendnotes[C]{\smallbreak}\pstart{}{\pb}\textsc{Herrn Dr. Rich Beer-Hofmann}\pend{}\pstart{}Wien\oindex{Wien@\textbf{Wien}, \emph{A.ADM2}|pw}
                  .
               \pend{}\pstart{}\textsc{I Wollzeile 15\oindex{Wollzeile@\textbf{Wollzeile}, \emph{Straße (K.STR)}|pw}}
                  .
               \pend{}{\bigskip}\vspace{1em}
\pstart
           \noindent{}{\pb}Lieber Richard,
                hier iſt der Sitz, Sie
               bringen ihn ſicher noch leicht an 
               \introOben{}
                  (
                  \substVorne{}\textsuperscript{\textcolor{gray}{womö}}\substDazwischen{}
                        ſchli
                        {\geminationm}\substHinten{}
                  ſtenfalls an d
                  \textcolor{gray}{er}{ }\textsc{Casse}
                  )
               \introOben{}
               . – Ich ka
               {\geminationn}
               
               nicht gehen, wegen 
               Papa\pwindex{Schnitzler, Johann 10.04.1835 – 02.05.1893@\textsc{Schnitzler, Johann} (10.04.1835 – 02.05.1893), \emph{Laryngologe/Laryngologin}|pwv}
               , der
               ſtark fiebert und meinetwegen, der, Abends wenigſtens, ſchwach fiebert. Ich werde
               ſehen, ob ich heute um 
               10
                ins Cafè 
               {\pb}
               ko
               {\geminationm}
               en kann – ich hoffe! –
            \pend
           
\pstart
           
               – Von 
               \textsc{Fels}\pwindex{Fels, Friedrich Michael *~1864@\textsc{Fels, Friedrich Michael} (*~1864), \emph{Journalist/Journalistin}|pw}
                kam Telegra
               {\geminationm}
               : er bittet um 25 fl, um abreiſen zu
               können. Ich ſandte ihm die 15 von 
               \textsc{Loris}\pwindex{Hofmannsthal, Hugo von 1874-02-01 – 1929-07-15@\textsc{Hofmannsthal, Hugo von} (1874-02-01 – 1929-07-15), \emph{Schriftsteller/Schriftstellerin}|pw}{ }\textsc{resp}{ }Fiſcher\pwindex{Fischer, Robert 07.10.1860 – 27.05.1939@\textsc{Fischer, Robert} (07.10.1860 – 27.05.1939), \emph{Rechtsanwalt/Rechtsanwältin}|pw}
               , u. von mir zehn. – –
            \pend
           
\pstart
           \textsc{Specht}\pwindex{Specht, Richard 07.12.1870 – 18.03.1932@\textsc{Specht, Richard} (07.12.1870 – 18.03.1932), \emph{Schriftsteller/Schriftstellerin, Journalist/Journalistin, Kritiker/Kritikerin}|pw}
                geht vielleicht zum 
               \label{K_L00204-1v}\edtext{ledigen Hof\pwindex{ledige Hof. Schauspiel in 4 Akten@\emph{Der ledige Hof. Schauspiel in 4 Akten}|pw}}{\lemma{\textnormal{\emph{ledigen Hof}}}\Cendnote{\textnormal{
                  Mehrere Stellen des undatierten Briefes
                  erlauben gemeinsam eine zeitliche Einordnung. Am 
                  29. 4. 1893
                   fand im
                  Zuge eines Gastspiels die Aufführung von 
                  Ludwig
                     Anzengrubers\pwindex{Anzengruber, Ludwig 29.11.1839 – 10.12.1889@\textsc{Anzengruber, Ludwig} (29.11.1839 – 10.12.1889), \emph{Schriftsteller/Schriftstellerin}|pwk}{ }\emph{Der ledige Hof}\pwindex{ledige Hof. Schauspiel in 4 Akten@\emph{Der ledige Hof. Schauspiel in 4 Akten}|pwk}
                   im 
                  Carl-Theater\oindex{Carl-Theater@\textbf{Carl-Theater}, \emph{Theater (K.THE)}|pwk}
                   statt. Am Vortag vermerkte 
                  Schnitzler
                   im 
                  \emph{Tagebuch}\pwindex{Tagebuch@\emph{Tagebuch}|pwk}
                  , dass sein 
                  Vater\pwindex{Schnitzler, Johann 10.04.1835 – 02.05.1893@\textsc{Schnitzler, Johann} (10.04.1835 – 02.05.1893), \emph{Laryngologe/Laryngologin}|pwkv}
                   krank sei und er es werde. Die Verortung vor dem Sonntag spricht
                  gleichfalls für den Samstag.
               }}}\label{K_L00204-1}
               ? –
            \pend
           
\pstart
           {\pb}
               Vielleicht theilen Sie mir irgendwie mit, was für
                     
               
                  So
                  {\geminationn}
                  tag
               
                morgen 
               Nachmittag
               
               projektirt ift; ka
               {\geminationn}
                ich auf ein paar Stunden mit Euch
               ſein, möcht ichs gerne. –
            \pend
           
\pstart
           
               Herzlich der Ihre
               {\\[\baselineskip]}\spacefill\mbox{Arthur}\pend
           \leftskip=0em{}\selectlanguage{ngerman}\endnumbering\briefempfaengerindex{Beer-Hofmann, Richard@\textsc{Beer-Hofmann, Richard}!zzzSchnitzler, Arthur@\emph{von Arthur Schnitzler}!1893-04-292@{{[}29. 4. 1893?{]}}|)be}\mylabel{L00204h}  \normalsize

\doendnotes{C}
\bigskip
\vfill

\clearpage

\footnotesize

\lohead{\textsc{register}}

% Definiere theindex-Environment komplett neu ohne reledmac
\makeatletter
\renewenvironment{theindex}{%
  \section*{\indexname}%
  \setlength{\parindent}{0pt}%
  \setlength{\parskip}{0pt plus 0.3pt}%
  \let\item\@idxitem
}{%
  \clearpage
}
\makeatother

\IfFileExists{\jobname-pw.ind}{\input{\jobname-pw.ind}}{}

\end{document}

      