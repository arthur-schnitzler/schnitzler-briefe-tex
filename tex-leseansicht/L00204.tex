%% latex-leseansicht-vorspann.tex
%% Vorspann für die Leseansicht.
%% Lädt die gemeinsame Datei latex-vorspann.tex mit nicht gesetztem Schalter.

\newif\ifkorrekturansicht
\korrekturansichtfalse

\input{../tex-inputs/latex-vorspann}


\section[Arthur Schnitzler an Richard Beer-Hofmann, {{[}}29. 4. 1893?{{]}}]{L00204 Arthur Schnitzler an Richard Beer-Hofmann, {[}29. 4. 1893?{]}}
\nopagebreak\mylabel{L00204v}
\rehead{ }\normalsize\beginnumbering\briefempfaengerindex{Beer-Hofmann, Richard@\textsc{Beer-Hofmann, Richard}!zzzSchnitzler, Arthur@\emph{von Arthur Schnitzler}!1893-04-292@{{[}29. 4. 1893?{]}}|(be}
\toendnotes[C]{\smallbreak\pagebreak[2]}
\correspDesc{Versand  durch Arthur Schnitzler am [29. 4. 1893?] in Wien
\newline{}Erhalt  durch Richard Beer-Hofmann im Zeitraum [29. 4. 1893 – 3. 5. 1893?] \textbf{Ort fehlend} }\toendnotes[C]{\smallbreak}
\Standort{YCGL, MSS 31.}
\physDesc{Brief, 1 Blatt, 3 Seiten, Kuvert, 672 Zeichen
\newline{}Handschrift: Bleistift, deutsche Kurrent
\newline{}Versand: ohne postalischen Übermittlungsvermerk }
\buchAbdrucke{\weitereDrucke{Arthur Schnitzler, Richard Beer-Hofmann: \emph{Briefwechsel 1891–1931}. Herausgegeben von Konstanze Fliedl. Wien, Zürich: \emph{Europaverlag} 1992, S. 44.} }\toendnotes[C]{\smallbreak}\pstart{}{\pb}\textsc{Herrn Dr. Rich Beer-Hofmann}\pend{}\pstart{}Wien\oindex{Wien@\textbf{Wien}, \emph{Verwaltungsgebiet}|pw}.\pend{}\pstart{}\textsc{I Wollzeile 15\oindex{Wien@\textbf{Wien}!I., Innere Stadt@\textbf{I., Innere Stadt}!Wollzeile 15 (»Berthahof«)@\textbf{Wollzeile 15 (»Berthahof«)}, \emph{Wohngebäude}|pw}}. \pend{}{\bigskip}\vspace{1em}
\pstart
           \noindent{}{\pb}Lieber Richard, hier iſt der Sitz, Sie bringen ihn{ }ſicher noch
                    leicht an \introOben{}(\substVorne{}\textsuperscript{\textcolor{gray}{womö}}\substDazwischen{}ſchli{\geminationm}\substHinten{}ſtenfalls an d\textcolor{gray}{er}{ }\textsc{Casse})\introOben{}. – Ich ka{\geminationn} nicht gehen, wegen Papa\pwindex{Schnitzler, Johann 10.\,4.\,1835 Nagykanizsa – 2.\,5.\,1893 Wien@\textsc{Schnitzler, Johann} (10.\,4.\,1835 Nagykanizsa – 2.\,5.\,1893 Wien), \emph{Laryngologe}|pwv}, der{ }ſtark fiebert und
                    meinetwegen, der, Abends wenigſtens,{ }ſchwach fiebert. Ich werde{ }ſehen, ob ich
                    heute um 10 ins Cafè {\pb}ko{\geminationm}en kann – ich hoffe! –\pend
           
\pstart
           – Von \textsc{Fels}\pwindex{Fels, Friedrich Michael *~1864 Bad Dürkheim@\textsc{Fels, Friedrich Michael} (*~1864 Bad Dürkheim), \emph{Journalist}|pw} kam \label{K_L00204-11v}\edtext{Telegra{\geminationm}}{\lemma{\textnormal{\emph{Telegramm}}}\Cendnote{\textnormal{XXXX Auszeichnungsfehler: Dokument L00203 nicht gefunden.}}}\label{K_L00204-11}: er bittet um
                    25 fl, um abreiſen zu können. Ich{ }ſandte ihm die 15 von \textsc{Loris}\pwindex{Hofmannsthal, Hugo von 1.\,2.\,1874 Wien – 15.\,7.\,1929 Rodaun@\textsc{Hofmannsthal, Hugo von} (1.\,2.\,1874 Wien – 15.\,7.\,1929 Rodaun), \emph{Schriftsteller}|pw}{ }\textsc{resp}{ }Fiſcher\pwindex{Fischer, Robert 7.\,10.\,1860 Lomnice – 27.\,5.\,1939 Wien@\textsc{Fischer, Robert} (7.\,10.\,1860 Lomnice – 27.\,5.\,1939 Wien), \emph{Rechtsanwalt}|pw}, u. von mir zehn. – –\pend
           
\pstart
           \textsc{Specht}\pwindex{Specht, Richard 7.\,12.\,1870 Wien – 18.\,3.\,1932 ebd.@\textsc{Specht, Richard} (7.\,12.\,1870 Wien – 18.\,3.\,1932 ebd.), \emph{Schriftsteller, Journalist, Kritiker}|pw} geht vielleicht zum \label{K_L00204-1v}\edtext{ledigen Hof\pwindex{Anzengruber, Ludwig 29.\,11.\,1839 Wien – 10.\,12.\,1889 ebd.@\textsc{Anzengruber, Ludwig} (29.\,11.\,1839 Wien – 10.\,12.\,1889 ebd.), \emph{Schriftsteller}!ledige Hof. Schauspiel in 4 Akten@\strich\emph{Der ledige Hof. Schauspiel in 4 Akten}|pw}}{\lemma{\textnormal{\emph{ledigen Hof}}}\Cendnote{\textnormal{Mehrere Stellen des undatierten
                        Briefes erlauben gemeinsam eine zeitliche Einordnung. Am
                            29. 4. 1893 fand im Zuge eines Gastspiels die Aufführung
                        von Ludwig Anzengrubers\pwindex{Anzengruber, Ludwig 29.\,11.\,1839 Wien – 10.\,12.\,1889 ebd.@\textsc{Anzengruber, Ludwig} (29.\,11.\,1839 Wien – 10.\,12.\,1889 ebd.), \emph{Schriftsteller}|pwk}{ }\emph{Der ledige Hof}\pwindex{Anzengruber, Ludwig 29.\,11.\,1839 Wien – 10.\,12.\,1889 ebd.@\textsc{Anzengruber, Ludwig} (29.\,11.\,1839 Wien – 10.\,12.\,1889 ebd.), \emph{Schriftsteller}!ledige Hof. Schauspiel in 4 Akten@\strich\emph{Der ledige Hof. Schauspiel in 4 Akten}|pwk} im Carl-Theater\oindex{Wien@\textbf{Wien}!II., Leopoldstadt@\textbf{II., Leopoldstadt}!Carl-Theater@\textbf{Carl-Theater}, \emph{Theater}|pwk} statt. Am Vortag vermerkte Schnitzler im \emph{Tagebuch}\pwindex{Schnitzler, Arthur 15.\,5.\,1862 Wien – 21.\,10.\,1931 ebd.@\textsc{Schnitzler, Arthur} (15.\,5.\,1862 Wien – 21.\,10.\,1931 ebd.), \emph{Schriftsteller, Mediziner}!Tagebuch@\strich\emph{Tagebuch}|pwk}, dass sein Vater\pwindex{Schnitzler, Johann 10.\,4.\,1835 Nagykanizsa – 2.\,5.\,1893 Wien@\textsc{Schnitzler, Johann} (10.\,4.\,1835 Nagykanizsa – 2.\,5.\,1893 Wien), \emph{Laryngologe}|pwkv} krank sei und er es werde. Die Verortung
                        vor dem Sonntag spricht gleichfalls für den Samstag.}}}\label{K_L00204-1}? –\pend
           
\pstart
           {\pb}Vielleicht theilen Sie mir irgendwie mit, was
                    für So {\geminationn}tag morgen
                        Nachmittag projektirt ift; ka{\geminationn} ich
                    auf ein paar Stunden mit Euch{ }ſein, möcht ichs gerne. –\pend
           
\pstart
           Herzlich der Ihre {\\[\baselineskip]}\spacefill\mbox{Arthur}\pend
           \leftskip=0em{}\selectlanguage{ngerman}\endnumbering\briefempfaengerindex{Beer-Hofmann, Richard@\textsc{Beer-Hofmann, Richard}!zzzSchnitzler, Arthur@\emph{von Arthur Schnitzler}!1893-04-292@{{[}29. 4. 1893?{]}}|)be}\mylabel{L00204h}  \newcommand{\dateiname}{L00204}\newcommand{\titel}{Arthur Schnitzler an Richard Beer-Hofmann, [29. 4. 1893?]}\newcommand{\editorInnen}{Martin Anton Müller und Gerd-Hermann Susen}%% latex-leseansicht-abspann.tex
%% Abspann für die Leseansicht.
%% Der Schalter \ifkorrekturansicht ist bereits durch den Vorspann gesetzt.

%% latex-abspann.tex
%% Gemeinsamer Abspann für Korrekturansicht und Leseansicht.
%% Setzt den Schalter \ifkorrekturansicht voraus (gesetzt in den
%% einbindenden Dateien latex-korrekturansicht-abspann.tex bzw.
%% latex-leseansicht-abspann.tex).
%% ---------------------------------------------------------------

\normalsize

% Das esempio-Environment wird nur in der Leseansicht benötigt
\ifkorrekturansicht\else
\newenvironment{esempio}[3]%
{
    \vspace{1.5ex}
    \rlap{\underline{#1}}
    \par
    \setlength{\parindent}{0cm}
    \nopagebreak
    \leftskip=#2cm
    \rightskip=#3cm
}
{
    \par
}
\fi

\doendnotes{C}
\bigskip
\vfill

\clearpage

\footnotesize

\ifkorrekturansicht
  \lohead{\textsc{register}}
\fi

% theindex-Environment neu definieren ohne reledmac
\makeatletter
\renewenvironment{theindex}{%
  \ifkorrekturansicht
    \section*{\indexname}%
  \else
    \subsubsection*{Index der erwähnten Entitäten}%
  \fi
  \setlength{\parindent}{0pt}%
  \setlength{\parskip}{0pt plus 0.3pt}%
  \let\item\@idxitem
}{%
  \ifkorrekturansicht\clearpage\fi
}
\makeatother

\IfFileExists{\jobname-pw.ind}{\input{\jobname-pw.ind}}{}

% Quellenangabe nur in der Leseansicht
\ifkorrekturansicht\else
% Fallback-Definitionen, falls die .tex-Datei \titel etc. nicht gesetzt hat
\providecommand{\titel}{}
\providecommand{\editorInnen}{}
\providecommand{\dateiname}{\jobname}

\vspace{3cm}

\vfill

\footnotesize
\textsc{Quelle}: \titel. Herausgegeben von {\editorInnen}. In: \emph{Arthur Schnitzler: Briefwechsel mit Autorinnen und Autoren}.
 Digitale Edition, https://schnitzler-briefe.acdh.oeaw.ac.at/{\dateiname}.html (Stand \today)
\fi

\end{document}


