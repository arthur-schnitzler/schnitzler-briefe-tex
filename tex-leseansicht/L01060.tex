%% latex-leseansicht-vorspann.tex
%% Vorspann für die Leseansicht.
%% Lädt die gemeinsame Datei latex-vorspann.tex mit nicht gesetztem Schalter.

\newif\ifkorrekturansicht
\korrekturansichtfalse

\input{../tex-inputs/latex-vorspann}


\section[Arthur Schnitzler an Richard Beer-Hofmann, 27. 7. 1900]{L01060 Arthur Schnitzler an Richard Beer-Hofmann, 27. 7. 1900}
\nopagebreak\mylabel{L01060v}
\rehead{ }\normalsize\beginnumbering\briefempfaengerindex{Beer-Hofmann, Richard@\textsc{Beer-Hofmann, Richard}!zzzSchnitzler, Arthur@\emph{von Arthur Schnitzler}!1900-07-271@{27. 7. 1900}|(be}
\toendnotes[C]{\smallbreak\pagebreak[2]}
\correspDesc{Versand  durch Arthur Schnitzler am 27. 7. 1900 in Wien
\newline{}Erhalt  durch Richard Beer-Hofmann am 26. 7. 1900 in Altaussee}\toendnotes[C]{\smallbreak}
\Standort{YCGL, MSS 31.}
\physDesc{Telegramm, 146 Zeichen
\newline{}HandschriftX2 einer Schreibkraft: Bleistift, lateinische Kurrent
\newline{}Versand: »\noindent{}\textcolor{gray}{\textbf{Eingelangt von}}{ }\textcolor{gray}{DLi}{ / }\textcolor{gray}{\textbf{auf Leitung Nr.}}{ }10{ / }\textcolor{gray}{\textbf{am}}{ }2\textcolor{gray}{7}/7 \textcolor{gray}{\textbf{190}}{\dots}{ }\textcolor{gray}{\textbf{um}} 8 \textcolor{gray}{\textbf{Uhr}}
                                          55 \textcolor{gray}{\textbf{Min.}}{ }\textcolor{gray}{V}\textcolor{gray}{\textbf{Mittag}}{ / }\textcolor{gray}{\textbf{Aufgenommen durch}}{ }\textcolor{gray}{A}{ / }\textcolor{gray}{\textbf{Von}}{ }Wien 12\oindex{Wien@\textbf{Wien}, \emph{Verwaltungsgebiet}|pw}{ / }\textcolor{gray}{\textbf{Aufgabe-Nr.}}{ }20{ }\textcolor{gray}{\textbf{mit}} 12\textcolor{gray}{×}\textcolor{gray}{\textbf{Taxworten (}}19 \textcolor{gray}{\textbf{Worten {\dotsfive} Chiffern)}}{ / }\textcolor{gray}{\textbf{Aufgegeben am}}{ }27/7 \textcolor{gray}{\textbf{190}}{\dots}{ / }\textcolor{gray}{\textbf{um}}{ }7 \textcolor{gray}{\textbf{Uhr}}{ }/ \textcolor{gray}{\textbf{Min.}}{ }n \textcolor{gray}{\textbf{Mittag}}« 
\newline{}Ordnung: mit Bleistift von unbekannter Hand das Datum vermerkt:»27. 7. 1900« }\pstart{}{\pb}Richard Beerhofmann\pend{}\pstart{}\textcolor{gray}{\textbf{\textit{Alt Auſſee\oindex{Altaussee@\textbf{Altaussee}, \emph{Verwaltungsgebiet}|pw}}}}\pend{}{\bigskip}\vspace{1em}
\pstart
           \noindent{}{\pb}Gedenke morgen Nachmittag Aussee Post\oindex{Gasthaus Post@\textbf{Gasthaus Post}, \emph{Hotel}|pw} wo ich Wohnung bestellt einzutreffen radle gleich
               zu Ihnen herzlichst\pend
           \pstart \spacefill\mbox{Arthur}\pend{}\selectlanguage{ngerman}\endnumbering\briefempfaengerindex{Beer-Hofmann, Richard@\textsc{Beer-Hofmann, Richard}!zzzSchnitzler, Arthur@\emph{von Arthur Schnitzler}!1900-07-271@{27. 7. 1900}|)be}\mylabel{L01060h}  \newcommand{\dateiname}{L01060}\newcommand{\titel}{Arthur Schnitzler an Richard Beer-Hofmann, 27. 7. 1900}\newcommand{\editorInnen}{Martin Anton Müller und Gerd-Hermann Susen}%% latex-leseansicht-abspann.tex
%% Abspann für die Leseansicht.
%% Der Schalter \ifkorrekturansicht ist bereits durch den Vorspann gesetzt.

%% latex-abspann.tex
%% Gemeinsamer Abspann für Korrekturansicht und Leseansicht.
%% Setzt den Schalter \ifkorrekturansicht voraus (gesetzt in den
%% einbindenden Dateien latex-korrekturansicht-abspann.tex bzw.
%% latex-leseansicht-abspann.tex).
%% ---------------------------------------------------------------

\normalsize

% Das esempio-Environment wird nur in der Leseansicht benötigt
\ifkorrekturansicht\else
\newenvironment{esempio}[3]%
{
    \vspace{1.5ex}
    \rlap{\underline{#1}}
    \par
    \setlength{\parindent}{0cm}
    \nopagebreak
    \leftskip=#2cm
    \rightskip=#3cm
}
{
    \par
}
\fi

\doendnotes{C}
\bigskip
\vfill

\clearpage

\footnotesize

\ifkorrekturansicht
  \lohead{\textsc{register}}
\fi

% theindex-Environment neu definieren ohne reledmac
\makeatletter
\renewenvironment{theindex}{%
  \ifkorrekturansicht
    \section*{\indexname}%
  \else
    \subsubsection*{Index der erwähnten Entitäten}%
  \fi
  \setlength{\parindent}{0pt}%
  \setlength{\parskip}{0pt plus 0.3pt}%
  \let\item\@idxitem
}{%
  \ifkorrekturansicht\clearpage\fi
}
\makeatother

\IfFileExists{\jobname-pw.ind}{\input{\jobname-pw.ind}}{}

% Quellenangabe nur in der Leseansicht
\ifkorrekturansicht\else
% Fallback-Definitionen, falls die .tex-Datei \titel etc. nicht gesetzt hat
\providecommand{\titel}{}
\providecommand{\editorInnen}{}
\providecommand{\dateiname}{\jobname}

\vspace{3cm}

\vfill

\footnotesize
\textsc{Quelle}: \titel. Herausgegeben von {\editorInnen}. In: \emph{Arthur Schnitzler: Briefwechsel mit Autorinnen und Autoren}.
 Digitale Edition, https://schnitzler-briefe.acdh.oeaw.ac.at/{\dateiname}.html (Stand \today)
\fi

\end{document}


