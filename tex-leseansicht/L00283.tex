%% latex-korrekturansicht-vorspann.tex
%% Vorspann für die Korrekturansicht.
%% Lädt die gemeinsame Datei latex-vorspann.tex mit gesetztem Schalter.

\newif\ifkorrekturansicht
\korrekturansichttrue

\input{../tex-inputs/latex-vorspann}


\section[Richard Beer-Hofmann an Arthur Schnitzler, {[}15. 11. 1893?{]}]{L00283 Richard Beer-Hofmann an Arthur Schnitzler, {[}15. 11. 1893?{]}}
\nopagebreak\mylabel{L00283v}
\rehead{ }\normalsize\beginnumbering\briefempfaengerindex{Schnitzler, Arthur@\textsc{Schnitzler, Arthur}!zzzBeer-Hofmann, Richard@\emph{von Richard Beer-Hofmann}!1893-11-151@{{[}15. 11. 1893?{]}}|(be}
\toendnotes[C]{\smallbreak\pagebreak[2]}\Standort{CUL, Schnitzler, B 8.}
\physDesc{Briefkarte, 273 Zeichen
\newline{}Handschrift: Bleistift, lateinische Kurrent
\newline{}Schnitzler: mit Bleistift datiert: »93« 
\newline{}Ordnung: mit Bleistift von unbekannter Hand nummeriert:
                                    »25.« }
\buchAbdrucke{\weitereDrucke{Arthur Schnitzler, Richard Beer-Hofmann: \emph{Briefwechsel 1891–1931}. Wien, Zürich: \emph{Europaverlag} 1992, S. 53.} }\toendnotes[C]{\smallbreak}
\pstart
           \noindent{}{\pb}Lieber Arthur!{ }Hugo\pwindex{Hofmannsthal, Hugo von 1874-02-01 – 1929-07-15@\textsc{Hofmannsthal, Hugo von} (1874-02-01 – 1929-07-15), \emph{Schriftsteller/Schriftstellerin}|pw} ko{\geminationm}t heute
               um circa 7{ }½ 8 ins Caffée und nachtmahlt da{\geminationn} mit mir
                  zusa{\geminationm}en (\label{K_L00283-1v}\edtext{bei mir zu Hause}{\lemma{\textnormal{\emph{bei mir zu Hause}}}\Cendnote{\textnormal{Von den Besuchen, die Schnitzler{ }1893 bei Beer-Hofmann\pwindex{Beer-Hofmann, Richard 1866-07-11 – 1945-09-26@\textsc{Beer-Hofmann, Richard} (1866-07-11 – 1945-09-26), \emph{Schriftsteller/Schriftstellerin}|pwk} in
                  seinem \emph{Tagebuch}\pwindex{Tagebuch@\emph{Tagebuch}|pwk} verzeichnet, ist
                  ausschließlich jener am 15. 11. 1893 auf die Teilnahme von Hofmannsthal\pwindex{Hofmannsthal, Hugo von 1874-02-01 – 1929-07-15@\textsc{Hofmannsthal, Hugo von} (1874-02-01 – 1929-07-15), \emph{Schriftsteller/Schriftstellerin}|pwk} beschränkt.}}}\label{K_L00283-1}). Wenn Sie auch zu mir kämen, könnten wir
               dann endlich einmal unser x x x in Angriff nehmen.\pend
           
\pstart
           {\pb}Herzlichst{\\[\baselineskip]}Ihr{\\[\baselineskip]}\spacefill\mbox{Richard}\pend
           \leftskip=0em{}
\pstart
           Vielleicht treffen wir uns also alle im Caffée – \uline{vorher}!\pend
           \selectlanguage{ngerman}\endnumbering\briefempfaengerindex{Schnitzler, Arthur@\textsc{Schnitzler, Arthur}!zzzBeer-Hofmann, Richard@\emph{von Richard Beer-Hofmann}!1893-11-151@{{[}15. 11. 1893?{]}}|)be}\mylabel{L00283h}  \normalsize

\doendnotes{C}
\bigskip
\vfill

\clearpage

\footnotesize

\lohead{\textsc{register}}

% Definiere theindex-Environment komplett neu ohne reledmac
\makeatletter
\renewenvironment{theindex}{%
  \section*{\indexname}%
  \setlength{\parindent}{0pt}%
  \setlength{\parskip}{0pt plus 0.3pt}%
  \let\item\@idxitem
}{%
  \clearpage
}
\makeatother

\IfFileExists{\jobname-pw.ind}{\input{\jobname-pw.ind}}{}

\end{document}

      