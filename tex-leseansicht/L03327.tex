%% latex-leseansicht-vorspann.tex
%% Vorspann für die Leseansicht.
%% Lädt die gemeinsame Datei latex-vorspann.tex mit nicht gesetztem Schalter.

\newif\ifkorrekturansicht
\korrekturansichtfalse

\input{../tex-inputs/latex-vorspann}


         
         \renewcommand{\erwaehntePersonen}{Personen:  ?? [Kusine von Felix Salten], Felix Salten}
         \renewcommand{\erwaehnteInstitutionen}{Institutionen: Die Zeit}
         \renewcommand{\erwaehnteOrte}{Orte: Volkstheater, Wien}
         \renewcommand{\erwaehnteWerke}{Werke: Das vierte Gebot. Volksstück in vier Acten, Empfängnis}
               \section[ Felix Salten an Arthur Schnitzler, {[}24. 3. 1902{]}]{ Felix Salten an Arthur Schnitzler, {[}24. 3. 1902{]}}\nopagebreak\mylabel{v}\rehead{ }\begin{ledgroupsized}[t]{13cm}\normalsize\beginnumbering\briefempfaengerindex{Schnitzler, Arthur@\textsc{Schnitzler, Arthur}!zzzSalten, Felix@\emph{von Felix Salten}!1902-03-241@{{[}24. 3. 1902{]}}|(be} \toendnotes[C]{\smallbreak\pagebreak[2]} \Standort{CUL, Schnitzler, B 89, A 2.}
\physDesc{Karte, 252 Zeichen
\newline{}Handschrift: Bleistift, lateinische Kurrent
\newline{}Schnitzler: mit Bleistift datiert: »24/3 902.« 
\newline{}Ordnung: mit Bleistift von unbekannter Hand nummeriert: »151« }\toendnotes[C]{\smallbreak}\pstart
           \noindent{}{\pb}Lieber, hier der \label{K_L03327-1v}\edtext{Sitz
               zum »IV. Gebot\pwindex{\textcolor{red}{\textsuperscript{XXXX1 indx}}!vierte Gebot. Volksstueck in vier Acten1878@\strich\emph{Das vierte Gebot. Volksstück in vier Acten} {[}1878{]}|pw}«}{\lemma{\textnormal{\emph{Sitz
               zum »IV. Gebot«}}}\Cendnote{\textnormal{im Volkstheater\oindex{Volkstheater@\textbf{Volkstheater}|pwk}}}}\label{K_L03327-1h} – ich werde wol spät kommen, weil ich bei der »Zeit\orgindex{Zeit@Die Zeit|pw}« bin.\pend
           \pstart
           Die »Empfängnis\pwindex{Salten, Felix 06.09.1869 – 08.10.1945@\textsc{Salten, Felix} (06.09.1869 – 08.10.1945), \emph{Schriftsteller, Journalist}!Empfaengnis@\strich\emph{Empfängnis}|pw}« bring ich zum \label{K_L03327-2v}\edtext{Vorlesen}{\lemma{\textnormal{\emph{Vorlesen}}}\Cendnote{\textnormal{siehe A. S.: \emph{Tagebuch}, 24. 3. 1902}}}\label{K_L03327-2h} nachher mit.\pend
           \pstart
           Entschuldigen Sie das »\label{K_L03327-3v}\edtext{Rosa-Brieferl}{\lemma{\textnormal{\emph{Rosa-Brieferl}}}\Cendnote{\textnormal{Bezug auf die
                  Papierfarbe der Karte}}}\label{K_L03327-3h}\textcolor{gray}{«}, aber meine \label{K_L03327-4v}\edtext{Cousine\pwindex{?? [Kusine von Felix Salten] @\textsc{?? [Kusine von Felix Salten]}|pwv}}{\lemma{\textnormal{\emph{Cousine}}}\Cendnote{\textnormal{Salten\pwindex{Salten, Felix 06.09.1869 – 08.10.1945@\textsc{Salten, Felix} (06.09.1869 – 08.10.1945), \emph{Schriftsteller, Journalist}|pwk} hatte nur Cousinen väterlicherseits.
                  Welche genau gemeint war, kann nicht mit Bestimmtheit gesagt werden.}}}\label{K_L03327-4h}, bei
               der ich schreibe, ist so poetisch\pend
           \pstart
           Herzlichst {\\[\baselineskip]}\spacefill\mbox{Salten}\pend
           \leftskip=0em{}
         
         \endnumbering\mylabel{h}\end{ledgroupsized}  \newcommand{\dateiname}{L03327}\newcommand{\titel}{Felix Salten an Arthur Schnitzler, [24. 3. 1902]}\newcommand{\editorInnen}{Martin Anton Müller und Laura Untner}%% latex-leseansicht-abspann.tex
%% Abspann für die Leseansicht.
%% Der Schalter \ifkorrekturansicht ist bereits durch den Vorspann gesetzt.

%% latex-abspann.tex
%% Gemeinsamer Abspann für Korrekturansicht und Leseansicht.
%% Setzt den Schalter \ifkorrekturansicht voraus (gesetzt in den
%% einbindenden Dateien latex-korrekturansicht-abspann.tex bzw.
%% latex-leseansicht-abspann.tex).
%% ---------------------------------------------------------------

\normalsize

% Das esempio-Environment wird nur in der Leseansicht benötigt
\ifkorrekturansicht\else
\newenvironment{esempio}[3]%
{
    \vspace{1.5ex}
    \rlap{\underline{#1}}
    \par
    \setlength{\parindent}{0cm}
    \nopagebreak
    \leftskip=#2cm
    \rightskip=#3cm
}
{
    \par
}
\fi

\doendnotes{C}
\bigskip
\vfill

\clearpage

\footnotesize

\ifkorrekturansicht
  \lohead{\textsc{register}}
\fi

% theindex-Environment neu definieren ohne reledmac
\makeatletter
\renewenvironment{theindex}{%
  \ifkorrekturansicht
    \section*{\indexname}%
  \else
    \subsubsection*{Index der erwähnten Entitäten}%
  \fi
  \setlength{\parindent}{0pt}%
  \setlength{\parskip}{0pt plus 0.3pt}%
  \let\item\@idxitem
}{%
  \ifkorrekturansicht\clearpage\fi
}
\makeatother

\IfFileExists{\jobname-pw.ind}{\input{\jobname-pw.ind}}{}

% Quellenangabe nur in der Leseansicht
\ifkorrekturansicht\else
% Fallback-Definitionen, falls die .tex-Datei \titel etc. nicht gesetzt hat
\providecommand{\titel}{}
\providecommand{\editorInnen}{}
\providecommand{\dateiname}{\jobname}

\vspace{3cm}

\vfill

\footnotesize
\textsc{Quelle}: \titel. Herausgegeben von {\editorInnen}. In: \emph{Arthur Schnitzler: Briefwechsel mit Autorinnen und Autoren}.
 Digitale Edition, https://schnitzler-briefe.acdh.oeaw.ac.at/{\dateiname}.html (Stand \today)
\fi

\end{document}


      