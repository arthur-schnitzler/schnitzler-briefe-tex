%% latex-korrekturansicht-vorspann.tex
%% Vorspann für die Korrekturansicht.
%% Lädt die gemeinsame Datei latex-vorspann.tex mit gesetztem Schalter.

\newif\ifkorrekturansicht
\korrekturansichttrue

\input{../tex-inputs/latex-vorspann}


\section[Arthur Schnitzler an Hermann Bahr, 30. 7. 1905]{L01534 Arthur Schnitzler an Hermann Bahr, 30. 7. 1905}
\nopagebreak\mylabel{L01534v}
\rehead{ }\normalsize\beginnumbering\briefempfaengerindex{Bahr, Hermann@\textsc{Bahr, Hermann}!zzzSchnitzler, Arthur@\emph{von Arthur Schnitzler}!1905-07-302@{30. 7. 1905}|(be}
\toendnotes[C]{\smallbreak\pagebreak[2]}\Standort{TMW, HS AM 23375 Ba.}
\physDesc{Brief, 2 Blätter, 8 Seiten, 3313 Zeichen
\newline{}Handschrift: schwarze Tinte, deutsche Kurrent
\newline{}Ordnung: Lochung }
\buchAbdrucke{\weitereDrucke{1) Arthur Schnitzler: \emph{Briefe 1875–1912}. Frankfurt am Main: \emph{S. Fischer} 1981, S. 515–516.} \weitereDrucke{2) Arthur Schnitzler: \emph{The Letters of Arthur Schnitzler to Hermann Bahr}. Chapel Hill: \emph{The University of North Carolina Press} 1978, S. 89–90.} \weitereDrucke{3) Hermann Bahr, Arthur Schnitzler: \emph{Briefwechsel, Aufzeichnungen, Dokumente (1891–1931)}. Göttingen: \emph{Wallstein} 2018, S. 347–348.} }\toendnotes[C]{\smallbreak}
\pstart
           \centering{}\textsc{{\pb}Wien\oindex{Wien@\textbf{Wien}, \emph{A.ADM2}|pw}}{ }30. 7. 905\pend
           \vspace{0.5em}
\pstart
           lieber Hermann, dein neues Stück\pwindex{Andere@\emph{Die Andere}|pwv} hab ich in Reichenau\oindex{Reichenau an der Rax@\textbf{Reichenau an der Rax}, \emph{A.ADM3}|pw} geleſen u \damage{an}{ }Richard\pwindex{Beer-Hofmann, Richard 1866-07-11 – 1945-09-26@\textsc{Beer-Hofmann, Richard} (1866-07-11 – 1945-09-26), \emph{Schriftsteller/Schriftstellerin}|pw} abgeſandt. – Es hat mich durchaus
               intereſſirt, und allerlei menſchliches hat mich tief bewegt – gegen das Stück\pwindex{Andere@\emph{Die Andere}|pwv}, d. h. gegen das
               fünfactige Ge\damage{bi}lde, das von zweitauſend Menschen zugleich angehört u verſtanden werden
               ſoll, hab ich manches Bedenken. In wenig Worten ausgedrückt: {\pb}es mangelt dem Ganzen
               zuweilen an künſtlerischer Oekonomie. Nehmen wir an, du hätteſt mir nur den fünften
               Act zu leſen gegeben. Da hätt ich gesagt: Donnerwetter, iſt das ein merkwürdigs Ding
               – und hätte mir allerlei erſte vier Akte dazu gedacht, die vielleicht alle nicht ſo
               gut geweſen wären als deine \strikeout{oder} aber beſſer zu\strikeout{m} deinem fünften (wie ich ihn empfinde) gepaſſt hätten.
               Von deinem fünften Akt\pwindex{Andere@\emph{Die Andere}|pwv}{ }{\pb}geht ein Licht aus,
               das mir nach vorwärts deutet, aber den Herweg im Dunkel läßt. Man darf immer
               behaupten 2 × 2 = 4 – aber wenn man ſagt: \textsc{\uline{Ergo}} iſt 2 × 2 = 4, ſo verpflichtet dieſes Ergo zu einer vorhergegangenen Rechnung.
               Natürlich fühlſt du dieſes Ergo ſehr gut – aber du haſt es mich nicht dramatiſch
               nachfühlen laſſen. Etwas ähnliches hab ich zum 1. Akt zu bemerken. \textsc{Besenius\pwindex{Andere@\emph{Die Andere}|pwv}}. {\pb}Ich bediene
               mich Wörter eines Vergleichs (um das Recht zu haben etwas falſches zu behaupten!)
               Wenn ſich ein Muſiker zum Flügel ſetzt, ſo beginnt er zu praeludiren (\damage{m}anchmal) eh er ſein eigentliches Stück ſpielt. Er deutet die Sti{\geminationm}ung u die Harmonie des Stückes, – vielleicht auch nur
               ſeine eigne Laune an. Deine \textsc{Besenius\pwindex{Andere@\emph{Die Andere}|pwv}}-Scene ist ſolch ein Praeludiren, das du ſchon als Beginn des wirklichen Stückes
               ausgibſt. Man {\pb}glaubt
               dir lang {\dotstwo} 1, 2, 3, 4 Akte hindurch – denn, wenn Dein \textsc{Besenius\pwindex{Andere@\emph{Die Andere}|pwv}} noch einmal aufträte, behielteſt du vielleicht recht. Damit daſs ſeine Ideen
               ſozusagen wieder erſcheinen, iſt nichts gethan: hier war ein Menſch, der innerhalb
               der Oekonomie des ganzen zu mehr beſti{\geminationm}t ſchien, als
               einige ſchöne Dinge auszuſprechen, und er \damage{\strikeout{giebt}}\strikeout{{ }ſich}{ }ſchminkt ſich nach der erſten Scene ab. Das
               verzeihſt mir du ſo wenig wie die \label{K_L01534-1v}\edtext{bekannte ungela{\pb}dene
                  Flinte}{\lemma{\textnormal{\emph{bekannte … Flinte}}}\Cendnote{\textnormal{Čechov\pwindex{Cechov, Anton Pavlovic 1860-01-17 – 1904-07-15@\textsc{Čechov, Anton Pavlovič} (1860-01-17 – 1904-07-15), \emph{Schriftsteller/Schriftstellerin}|pwk} an Aleksandr Lazarev\pwindex{Lazarev, Alexandr S. 1861 – 1927@\textsc{Lazarev, Alexandr S.} (1861 – 1927), \emph{Schriftsteller/Schriftstellerin}|pwk}, 1. 11. 1889:
                     »Man kann nicht ein geladenes Gewehr auf die Bühne stellen, wenn niemand
                     die Absicht hat, einen Schuß daraus abzugeben.« Anton Čechov:
                        \emph{Briefe 1889–1892}. Herausgegeben und übersetzt von Peter
                     Urban. Zürich: \emph{Diogenes}{ }1998,
               S. 73.}}}\label{K_L01534-1}.\pend
           
\pstart
           Daſs \textsc{Amschel\pwindex{Andere@\emph{Die Andere}|pwv}} iſt wie er iſt, das iſt dein Wille und dein gutes Recht. Ich glaub an ihn. Ob
               man ihn, aus rein praktiſchen Gründen, nicht von einigen Widrigkeiten befreien
               ſollte, \strikeout{iſt} wäre zu überlegen. Wäre ich eine große
               Violinvirtuoſin, nicht um die Welt ließ ich mich von einem K\damage{er}l anrühren, der öfter als 6 Mal in der Minute \label{K_L01534-2v}\edtext{Schnudelchen}{\lemma{\textnormal{\emph{Schnudelchen}}}\Cendnote{\textnormal{Vgl.
                     \emph{Die Andere}\pwindex{Andere@\emph{Die Andere}|pwk}, 3. Akt.}}}\label{K_L01534-2}{ }ſagt. Aber das iſt ja Geschmackſache. Wie oft aber
               ſtört uns an einer {\pb}Frau nur der Gedanke an den der sie beseſſen hat. Und iſt das Publikum nicht gerade so\substVorne{}\textsuperscript{!}\substDazwischen{}?\substHinten{} Das Problem (»Die andere\pwindex{Andere@\emph{Die Andere}|pw}«) wird nicht
               im geringſten touchirt, wenn \textsc{Amschel\pwindex{Andere@\emph{Die Andere}|pwv}} ein wenig umgänglicher erſcheint. Die ganze Sti{\geminationm}ung des letzten Aktes iſt höchſt ſeltſam, beſonders merkwürdg die 2 neuen Personen
               – wie Lida\pwindex{Andere@\emph{Die Andere}|pwv} in die Umgebung
               geräth, iſt mir nicht ſehr klar geworden, \strikeout{das} ihr
               Hierſein hat was melodramatiſches {\pb}wenn auch ringsum alles
                  \textcolor{gray}{in}{[}s{]} Groteſkphantaſtiſche geht. Die Sterbeſcene, die zwei Männer
               bei ihr – das iſt kühn. Kühn gewiſs. Ob es noch mehr iſt, weiſs ich heute nicht. Von
               mittheilender Qual die Scene\pwindex{Andere@\emph{Die Andere}|pwv}
               zwiſchen Heinrich\pwindex{Andere@\emph{Die Andere}|pwv} und der Frau \textcolor{gray}{v} Jello\pwindex{Andere@\emph{Die Andere}|pwv}
               im 4. Akt. Wenn ich heute an das Stück denke, das ich vor 8 Tagen geleſen, ſo iſt es
               mir wie die Erinnerung an zuckende menſchliche Herzen.\pend
           
\pstart
           Ich hoffe es geht dir gut. Von mir hörſt du bald mehr. Meine Frau\pwindex{Schnitzler, Olga 17.01.1882 – 13.01.1970@\textsc{Schnitzler, Olga} (17.01.1882 – 13.01.1970), \emph{Schauspieler/Schauspielerin, Sänger/Sängerin}|pwv}, die das Stück\pwindex{Andere@\emph{Die Andere}|pwv} auch mit tiefſter Antheilnahme
               geleſen, grüßt dich vielmals\pend
           \pstart Von Herzen dein \spacefill\mbox{Arthur}\pend{}\selectlanguage{ngerman}\endnumbering\briefempfaengerindex{Bahr, Hermann@\textsc{Bahr, Hermann}!zzzSchnitzler, Arthur@\emph{von Arthur Schnitzler}!1905-07-302@{30. 7. 1905}|)be}\mylabel{L01534h}  \normalsize

\doendnotes{C}
\bigskip
\vfill

\clearpage

\footnotesize

\lohead{\textsc{register}}

% Definiere theindex-Environment komplett neu ohne reledmac
\makeatletter
\renewenvironment{theindex}{%
  \section*{\indexname}%
  \setlength{\parindent}{0pt}%
  \setlength{\parskip}{0pt plus 0.3pt}%
  \let\item\@idxitem
}{%
  \clearpage
}
\makeatother

\IfFileExists{\jobname-pw.ind}{\input{\jobname-pw.ind}}{}

\end{document}

      