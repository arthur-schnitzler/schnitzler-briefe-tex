%% latex-korrekturansicht-vorspann.tex
%% Vorspann für die Korrekturansicht.
%% Lädt die gemeinsame Datei latex-vorspann.tex mit gesetztem Schalter.

\newif\ifkorrekturansicht
\korrekturansichttrue

\input{../tex-inputs/latex-vorspann}


\section[Hermann Bahr an Arthur Schnitzler, {[}20. 4. 1894{]}]{L00314 Hermann Bahr an Arthur Schnitzler, {[}20. 4. 1894{]}}
\nopagebreak\mylabel{L00314v}
\rehead{ }\normalsize\beginnumbering\briefempfaengerindex{Schnitzler, Arthur@\textsc{Schnitzler, Arthur}!zzzBahr, Hermann@\emph{von Hermann Bahr}!1894-04-201@{{[}20. 4. 1894{]}}|(be}
\toendnotes[C]{\smallbreak\pagebreak[2]}\Standort{CUL, Schnitzler, B 5b.}
\physDesc{Brief, 1 Blatt, 2 Seiten, 544 Zeichen
\newline{}Handschrift: schwarze Tinte, deutsche Kurrent
\newline{}Schnitzler: mit Bleistift datiert: »20/4 94« 
\newline{}Ordnung: 1) mit rotem Buntstift von unbekannter Hand nummeriert:
                                    »19«  2) mit Bleistift von unbekannter Hand nummeriert:
                                    »19«}
\buchAbdrucke{\weitereDrucke{Hermann Bahr, Arthur Schnitzler: \emph{Briefwechsel, Aufzeichnungen, Dokumente (1891–1931)}. Göttingen: \emph{Wallstein} 2018, S. 70.} }
\pstart\center{}{\pb}Lieber Arthur!\pend\vspace{0.5em}
\pstart
           Adele Sandrock\pwindex{Sandrock, Adele 1863-08-19 – 1937-08-30@\textsc{Sandrock, Adele} (1863-08-19 – 1937-08-30), \emph{Schauspieler/Schauspielerin}|pw} erzählte mir geſtern von einer
               für Sonntag geplanten Partie mit \textsc{Rendezvous} in Rodaun\oindex{Rodaun@\textbf{Rodaun}, \emph{A.ADM4}|pw}. Ich möchte ſehr gern mit und könnte
               vielleicht ſchon in der Früh mit Dir hinaus. Allerdings unter der Vorausſetzung, daß
               wir ganz unter uns ſind, dh. Du, \textsc{Loris}\pwindex{Hofmannsthal, Hugo von 1874-02-01 – 1929-07-15@\textsc{Hofmannsthal, Hugo von} (1874-02-01 – 1929-07-15), \emph{Schriftsteller/Schriftstellerin}|pw} und \textsc{Richard}\pwindex{Beer-Hofmann, Richard 1866-07-11 – 1945-09-26@\textsc{Beer-Hofmann, Richard} (1866-07-11 – 1945-09-26), \emph{Schriftsteller/Schriftstellerin}|pw}, wozu dann Nachmittags ſich noch {\pb}\textsc{Dilly}\pwindex{Sandrock, Adele 1863-08-19 – 1937-08-30@\textsc{Sandrock, Adele} (1863-08-19 – 1937-08-30), \emph{Schauspieler/Schauspielerin}|pw} und \substVorne{}\textsuperscript{der}\substDazwischen{}etwa\substHinten{} der \textsc{Baumgartl}\pwindex{Baumgarten, Theodor 27.04.1863 – 28.08.1934@\textsc{Baumgarten, Theodor} (27.04.1863 – 28.08.1934), \emph{Rechtsanwalt/Rechtsanwältin}|pw} geſellen. Größere Horden ſind mir unſympathiſch; am liebſten wäre es mir zu
               viert; kommt außer den Genannten noch wer, ſo bitte, ſchreib mir das – dann gehe ich
               lieber ganz allein.\pend
           
\pstart
           Herzlichst{\\[\baselineskip]}Dein{\\[\baselineskip]}\spacefill\mbox{HermannBahr}\pend
           \leftskip=0em{}\selectlanguage{ngerman}\endnumbering\briefempfaengerindex{Schnitzler, Arthur@\textsc{Schnitzler, Arthur}!zzzBahr, Hermann@\emph{von Hermann Bahr}!1894-04-201@{{[}20. 4. 1894{]}}|)be}\mylabel{L00314h}  \normalsize

\doendnotes{C}
\bigskip
\vfill

\clearpage

\footnotesize

\lohead{\textsc{register}}

% Definiere theindex-Environment komplett neu ohne reledmac
\makeatletter
\renewenvironment{theindex}{%
  \section*{\indexname}%
  \setlength{\parindent}{0pt}%
  \setlength{\parskip}{0pt plus 0.3pt}%
  \let\item\@idxitem
}{%
  \clearpage
}
\makeatother

\IfFileExists{\jobname-pw.ind}{\input{\jobname-pw.ind}}{}

\end{document}

      