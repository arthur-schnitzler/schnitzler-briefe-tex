%% latex-korrekturansicht-vorspann.tex
%% Vorspann für die Korrekturansicht.
%% Lädt die gemeinsame Datei latex-vorspann.tex mit gesetztem Schalter.

\newif\ifkorrekturansicht
\korrekturansichttrue

\input{../tex-inputs/latex-vorspann}


\section[Arthur Schnitzler an Hugo von Hofmannsthal, 29. 3. 1902]{L01211 Arthur Schnitzler an Hugo von Hofmannsthal, 29. 3. 1902}
\nopagebreak\mylabel{L01211v}
\rehead{ }\normalsize\beginnumbering\briefempfaengerindex{Hofmannsthal, Hugo von@\textsc{Hofmannsthal, Hugo von}!zzzSchnitzler, Arthur@\emph{von Arthur Schnitzler}!1902-03-291@{29. 3. 1902}|(be}
\toendnotes[C]{\smallbreak\pagebreak[2]}\Standort{FDH, Hs-30885,97.}
\physDesc{Brief, 3 Blätter, 10 Seiten, 2451 Zeichen
\newline{}Handschrift: Bleistift, deutsche Kurrent}
\buchAbdrucke{\weitereDrucke{Hugo von Hofmannsthal, Arthur Schnitzler: \emph{Briefwechsel}. Frankfurt am Main: \emph{S. Fischer} 1964, S. 155.} }
\pstart
           \raggedleft{}{\pb}29. 3. 902\pend
           \vspace{0.5em}
\pstart
           mein lieber Hugo, da ich aus Ton u Inhalt Ihres Briefes entnehme,
               dſs Sie wirklich, we{\geminationn} auch in einer von mir nicht
               geahnten, nicht für möglich gehaltnen Weiſe und wahrhaftig nicht ganz berechtigten
               Weiſe verletzt ſind, ſo liegt es mir vor {\pb}allem am Herzen
               Ihnen zu ſagen daſs mir das beinah weh, nicht \introOben{}nur\introOben{} leid thut.
               Hätte ich eine Ahnung gehabt, daſs Sie auf dieſe Frühſtückſache irgend einen
               beträchtlichen Werth legen, hätten Sie mir z. B. geſchrieben: es wäre mir angenehm –
                  {\pb}es iſt mein ſpezieller Wunſch \textsc{etc}. ich hätte natürlich kein Abſagetelegramm an Sie geſchickt, obzwar das
               mit der Unbequemlichkeit in den nächſten Tagen wahrhaftig keine Ausrede war. Ich
               glaube auch dſs ich nicht abgeſagt hätte, we{\geminationn} Sie mich
               zu {\pb}ſich geladen hätten, aber ſo ſpielte auch, halb
               unbewußt die Überlegung mit: »ein neues Haus, – ich, der gar nirgends hingeht«. Das
               letztere ſoll natürlich keine Entſchuldg ſein ſondern \strikeout{\textcolor{gray}{aus}} wird hier nur beigefügt, da\strikeout{ſs} es zur
               Vollſtändigkeit gehört. {\pb}Sie werden mir gewiſs erwidern,
               daſs \substVorne{}\textsuperscript{was}\substDazwischen{}ich\substHinten{}{ }ſchon aus dem Umstand, dſs Sie mir \uline{überhaupt} geſchrieben haben, entnehmen mußte, es
               handelte ſich um einen herzlichen Wunſch von Ihnen. Bei näherer Überlegung ſehe ich
               das ein, und es war Unrecht {\pb}von mir, ſo raſch, ohne
               Würdigung dieſes Umſtands, Ihnen abzutelegrafiren. Aber die Hoffnung einer Beka{\geminationn}tſchaft für nächſtens, die ich am Schluſs ausgeſprochen
               habe, war keine Phraſe, und dſs Ihr Aerger über mich geſchwunden iſt, werden Sie bei
               unſerm nächſten {\pb}Zuſa{\geminationm}enſein
                  \introOben{}am beſten\introOben{} dadurch beweiſen, daſs Sie lieber auf den
               letzten als auf den erſten Satz meines Telegra{\geminationm}s
               zurückgreifen. Denken Sie freundlichſt noch einmal dran, daſs ich ſeit ſehr vielen
               Jahren kein mir fremdes Haus betreten habe und Sie {\pb}werden vielleicht ſpüren, daſs ich mit dem Wort von der Unbequemlichkeit mich
               ſelber mehr ins Unrecht geſetzt habe, als nothwendig war. Das weſentliche ist u
               bleibt: mir kam Ihr \introOben{}heutiger\introOben{} Brief ſo überraſchend wie
               möglich – \introOben{}de{\geminationn}\introOben{} als ich \substVorne{}\textsuperscript{Ih}\substDazwischen{}mein\substHinten{} Telegramm absandte, {\pb}war ich mir abſolut nicht
               bewußt, daſs ich Ihnen damit einen Wunſch abſchlage, auf deſſen Erfüllung in den
               nächſten Tagen Sie Werth legen. Aus Ihrem heutigen Briefe ſehe ich, daſs ich Sie
               verletzt habe; reichen Sie mir die Hand und ſeien Sie mir nicht böſe.\pend
           
\pstart
           Von Herzen Ihr{\\[\baselineskip]}\spacefill\mbox{Arthur}\pend
           \leftskip=0em{}
\pstart
           \noindent{}{\pb}Es wäre denkbar, dſs ich an einem der
                     Oſtertage bei Ihnen Vormittag vorbei radle, aber es iſt recht
                  unſicher.\pend
           
\pstart
           Mittwoch bin ich übrigens bei der »Kraft«probe\pwindex{Ueber unsere Kraft. Zweiter Teil@\emph{Über unsere Kraft. Zweiter Teil}|pw}, Sie wohl auch.\pend
           
\pstart
           Das Geld an Frau v. P.\pwindex{Pollanetz, Malvine von 15.2.1840 – 10.7.1926@\textsc{Pollanetz, Malvine von} (15.2.1840 – 10.7.1926)|pw} habe ich geſandt.{\\}\spacefill\mbox{A.}\pend
           \selectlanguage{ngerman}\endnumbering\briefempfaengerindex{Hofmannsthal, Hugo von@\textsc{Hofmannsthal, Hugo von}!zzzSchnitzler, Arthur@\emph{von Arthur Schnitzler}!1902-03-291@{29. 3. 1902}|)be}\mylabel{L01211h}  \normalsize

\doendnotes{C}
\bigskip
\vfill

\clearpage

\footnotesize

\lohead{\textsc{register}}

% Definiere theindex-Environment komplett neu ohne reledmac
\makeatletter
\renewenvironment{theindex}{%
  \section*{\indexname}%
  \setlength{\parindent}{0pt}%
  \setlength{\parskip}{0pt plus 0.3pt}%
  \let\item\@idxitem
}{%
  \clearpage
}
\makeatother

\IfFileExists{\jobname-pw.ind}{\input{\jobname-pw.ind}}{}

\end{document}

      