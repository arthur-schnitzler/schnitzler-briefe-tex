%% latex-leseansicht-vorspann.tex
%% Vorspann für die Leseansicht.
%% Lädt die gemeinsame Datei latex-vorspann.tex mit nicht gesetztem Schalter.

\newif\ifkorrekturansicht
\korrekturansichtfalse

\input{../tex-inputs/latex-vorspann}


\section[Mirjam Beer-Hofmann an Arthur Schnitzler, 8. 6. 1927]{L02488 Mirjam Beer-Hofmann an Arthur Schnitzler, 8. 6. 1927}
\nopagebreak\mylabel{L02488v}
\rehead{ }\normalsize\beginnumbering\briefempfaengerindex{Schnitzler, Arthur@\textsc{Schnitzler, Arthur}!zzzBeer-Hofmann, Mirjam@\emph{von Mirjam Beer-Hofmann}!1927-06-081@{8. 6. 1927}|(be}
\toendnotes[C]{\smallbreak\pagebreak[2]}
\correspDesc{Versand  durch Miriam Beer-Hofmann am 8. 6. 1927 in Berlin
\newline{}Erhalt  durch Arthur Schnitzler im Zeitraum [9. 6. 1927
                  – 13. 6. 1927?] in Wien}\toendnotes[C]{\smallbreak}
\Standort{CUL, Schnitzler, B 8.}
\physDesc{Brief, 1 Blatt, 3 Seiten, 1004 Zeichen
\newline{}Handschrift: schwarze Tinte, lateinische Kurrent
\newline{}Schnitzler: 1) mit Bleistift beschriftet: »\textsc{BH Mirjam}«  2) mit rotem Buntstift mehrere Unterstreichungen
\newline{}Ordnung: mit Bleistift von unbekannter Hand nummeriert:
                                    »273« }
\buchAbdrucke{\weitereDrucke{Arthur Schnitzler, Richard Beer-Hofmann: \emph{Briefwechsel 1891–1931}. Herausgegeben von Konstanze Fliedl. Wien, Zürich: \emph{Europaverlag} 1992, S. 230.} }\toendnotes[C]{\smallbreak}
\pstart
           \raggedleft{}{\pb}Berlin\oindex{Berlin@\textbf{Berlin}, \emph{Hauptstadt}|pw}{ }8. 6. 27\pend
           
\pstart{}Liebster Arthur!\pend\vspace{0.5em}
\pstart
           So war ich wieder in Wien\oindex{Wien@\textbf{Wien}, \emph{Verwaltungsgebiet}|pw} und Du warst nicht da
               und wenn Du in Berlin\oindex{Berlin@\textbf{Berlin}, \emph{Hauptstadt}|pw} bist, hab’ ich Dich auch
               nur höchstens die Rückfahrt von Michaelis\pwindex{Michaelis, Dora 23.\,5.\,1881 Wien – 22.\,1.\,1946 New York City@\textsc{Michaelis, Dora} (23.\,5.\,1881 Wien – 22.\,1.\,1946 New York City)|pw}\pwindex{Michaelis, Karl 5.\,1.\,1872 Berlin – 4.\,11.\,1958 Nyon@\textsc{Michaelis, Karl} (5.\,1.\,1872 Berlin – 4.\,11.\,1958 Nyon), \emph{Rechtsanwalt, Chemiker, Patentanwalt}|pw} und das sind höchstens 15 Minuten. Was soll man da machen? Und ich
               hätte Dir oft viel zu sagen und will es Dir halt jetzt schreiben. Ich freue {\pb}mich sehr, dass Lily\pwindex{Cappellini, Lili 13.\,9.\,1909 Wien – 26.\,7.\,1928 Venedig@\textsc{Cappellini, Lili} (13.\,9.\,1909 Wien – 26.\,7.\,1928 Venedig)|pw} heiratet und Du damit zufrieden bist und ihren Mann\pwindex{Cappellini, Arnoldo 10.\,8.\,1889 Venedig – 8.\,5.\,1954 Rom@\textsc{Cappellini, Arnoldo} (10.\,8.\,1889 Venedig – 8.\,5.\,1954 Rom)|pwv} gern hast. Das hat man
               mir erzählt und zwar von glaubwürdiger Stelle, so dass ich es annehme und Dir doch
               darüber schreiben darf. Weisst Du, es ist sehr gut, wenn man sehr jung heiratet, es
               bleibt einem unendlich viel erspart. Ich weiss zwar nicht, wann Lily\pwindex{Cappellini, Lili 13.\,9.\,1909 Wien – 26.\,7.\,1928 Venedig@\textsc{Cappellini, Lili} (13.\,9.\,1909 Wien – 26.\,7.\,1928 Venedig)|pw} heiratet, jedenfalls {\pb}aber sag’ ihr schon heute viel
               Liebes von mir. Und Dir wünsch’ ich i{\geminationm}er, auch ohne
               Gelegenheit nur viel Schönes und Frohes.\pend
           
\pstart
           Ko{\geminationm}st Du nicht wieder nach Berlin\oindex{Berlin@\textbf{Berlin}, \emph{Hauptstadt}|pw}?\pend
           
\pstart
           Innigst{\\[\baselineskip]}Deine{\\[\baselineskip]}\spacefill\mbox{Mirjam}\pend
           \leftskip=0em{}
\pstart
           \noindent{}Viele herzliche Grüsse und Wünsche von meinem Mann\pwindex{Lens, Ernst 19.\,11.\,1890 Wien – 1962 New York City@\textsc{Lens, Ernst} (19.\,11.\,1890 Wien – 1962 New York City), \emph{Unternehmer}|pwv}.\pend
           
\pstart
           Der Brief ist nur für Dich, denn Dir gegenüber bin ich doch nie erwachsen
                     \textcolor{gray}{u} geniere mich daher Dir zu sagen, wie lieb ich Dich
                  habe.\pend
           \selectlanguage{ngerman}\endnumbering\briefempfaengerindex{Schnitzler, Arthur@\textsc{Schnitzler, Arthur}!zzzBeer-Hofmann, Mirjam@\emph{von Mirjam Beer-Hofmann}!1927-06-081@{8. 6. 1927}|)be}\mylabel{L02488h}  \newcommand{\dateiname}{L02488}\newcommand{\titel}{Mirjam Beer-Hofmann an Arthur Schnitzler, 8. 6. 1927}\newcommand{\editorInnen}{Martin Anton Müller und Gerd-Hermann Susen}%% latex-leseansicht-abspann.tex
%% Abspann für die Leseansicht.
%% Der Schalter \ifkorrekturansicht ist bereits durch den Vorspann gesetzt.

%% latex-abspann.tex
%% Gemeinsamer Abspann für Korrekturansicht und Leseansicht.
%% Setzt den Schalter \ifkorrekturansicht voraus (gesetzt in den
%% einbindenden Dateien latex-korrekturansicht-abspann.tex bzw.
%% latex-leseansicht-abspann.tex).
%% ---------------------------------------------------------------

\normalsize

% Das esempio-Environment wird nur in der Leseansicht benötigt
\ifkorrekturansicht\else
\newenvironment{esempio}[3]%
{
    \vspace{1.5ex}
    \rlap{\underline{#1}}
    \par
    \setlength{\parindent}{0cm}
    \nopagebreak
    \leftskip=#2cm
    \rightskip=#3cm
}
{
    \par
}
\fi

\doendnotes{C}
\bigskip
\vfill

\clearpage

\footnotesize

\ifkorrekturansicht
  \lohead{\textsc{register}}
\fi

% theindex-Environment neu definieren ohne reledmac
\makeatletter
\renewenvironment{theindex}{%
  \ifkorrekturansicht
    \section*{\indexname}%
  \else
    \subsubsection*{Index der erwähnten Entitäten}%
  \fi
  \setlength{\parindent}{0pt}%
  \setlength{\parskip}{0pt plus 0.3pt}%
  \let\item\@idxitem
}{%
  \ifkorrekturansicht\clearpage\fi
}
\makeatother

\IfFileExists{\jobname-pw.ind}{\input{\jobname-pw.ind}}{}

% Quellenangabe nur in der Leseansicht
\ifkorrekturansicht\else
% Fallback-Definitionen, falls die .tex-Datei \titel etc. nicht gesetzt hat
\providecommand{\titel}{}
\providecommand{\editorInnen}{}
\providecommand{\dateiname}{\jobname}

\vspace{3cm}

\vfill

\footnotesize
\textsc{Quelle}: \titel. Herausgegeben von {\editorInnen}. In: \emph{Arthur Schnitzler: Briefwechsel mit Autorinnen und Autoren}.
 Digitale Edition, https://schnitzler-briefe.acdh.oeaw.ac.at/{\dateiname}.html (Stand \today)
\fi

\end{document}


