%% latex-leseansicht-vorspann.tex
%% Vorspann für die Leseansicht.
%% Lädt die gemeinsame Datei latex-vorspann.tex mit nicht gesetztem Schalter.

\newif\ifkorrekturansicht
\korrekturansichtfalse

\input{../tex-inputs/latex-vorspann}


         
         \renewcommand{\erwaehntePersonen}{Personen: Richard Beer-Hofmann, Arnoldo Cappellini, Ernst Lens, Dora Michaelis, Karl Michaelis, Lili Schnitzler}
         \renewcommand{\erwaehnteOrte}{Orte: Berlin, Wien}
         \renewcommand{\erwaehnteWerke}{}
               \section[Mirjam Beer-Hofmann an Arthur Schnitzler, 8. 6. 1927]{ Mirjam Beer-Hofmann an Arthur Schnitzler,
                    8. 6. 1927}\nopagebreak\mylabel{v}\rehead{ }\begin{ledgroupsized}[t]{13cm}\normalsize\beginnumbering \toendnotes[C]{\smallbreak\pagebreak[2]} \Standort{CUL, Schnitzler, B 8.}
\physDesc{Brief, 1 Blatt, 3 Seiten
\newline{}Handschrift: schwarze Tinte, lateinische Kurrent
\newline{}Schnitzler: 1) mit Bleistift beschriftet: »\textsc{BH Mirjam}«  2) mit rotem Buntstift mehrere Unterstreichungen\newline{}Ordnung: mit Bleistift von unbekannter Hand
                                    nummeriert: »273« }\buchAbdrucke{\weitereDrucke{Arthur Schnitzler, Richard Beer-Hofmann: \emph{Briefwechsel 1891–1931}. Hg. Konstanze Fliedl. Wien, Zürich: \emph{Europaverlag} 1992, S. 230.} }\toendnotes[C]{\smallbreak}\pstart
           \raggedleft{}{\pb}Berlin\oindex{Berlin@\textbf{Berlin}|pw}{ }8. 6. 27\pend
           \pstart{}Liebster Arthur!\pend\pstart
           So war ich wieder in Wien\oindex{Wien@\textbf{Wien}|pw} und Du warst nicht da
                    und wenn Du in Berlin\oindex{Berlin@\textbf{Berlin}|pw} bist, hab’ ich Dich auch
                    nur höchstens die Rückfahrt von Michaelis\pwindex{Michaelis, Dora 23.05.1881 – 22.01.1946@\textsc{Michaelis, Dora} (23.05.1881 – 22.01.1946)|pw}\pwindex{Michaelis, Karl 05.01.1872 – 04.11.1958@\textsc{Michaelis, Karl} (05.01.1872 – 04.11.1958), \emph{Rechtsanwalt, Chemiker, Patentanwalt}|pw} und das sind höchstens 15 Minuten. Was soll man da machen?
                    Und ich hätte Dir oft viel zu sagen und will es Dir halt jetzt schreiben. Ich
                    freue {\pb}mich sehr, dass Lily\pwindex{Schnitzler, Lili 13.09.1909 – 26.07.1928@\textsc{Schnitzler, Lili} (13.09.1909 – 26.07.1928)|pw} heiratet und Du damit zufrieden bist und
                    ihren Mann\pwindex{Cappellini, Arnoldo 10.08.1889 – 08.05.1954@\textsc{Cappellini, Arnoldo} (10.08.1889 – 08.05.1954)|pwv} gern hast. Das
                    hat man mir erzählt und zwar von glaubwürdiger Stelle, so dass ich es annehme
                    und Dir doch darüber schreiben darf. Weisst Du, es ist sehr gut, wenn man sehr
                    jung heiratet, es bleibt einem unendlich viel erspart. Ich weiss zwar nicht,
                    wann Lily\pwindex{Schnitzler, Lili 13.09.1909 – 26.07.1928@\textsc{Schnitzler, Lili} (13.09.1909 – 26.07.1928)|pw} heiratet, jedenfalls {\pb}aber sag’ ihr schon heute
                    viel Liebes von mir. Und Dir wünsch’ ich i{\geminationm}er, auch
                    ohne Gelegenheit nur viel Schönes und Frohes.\pend
           \pstart
           Ko{\geminationm}st Du nicht wieder nach Berlin\oindex{Berlin@\textbf{Berlin}|pw}?\pend
           \pstart
           Innigst{\\[\baselineskip]}Deine{\\[\baselineskip]}\spacefill\mbox{Mirjam}\pend
           \leftskip=0em{}\pstart
           \noindent{}Viele herzliche Grüsse und Wünsche von meinem Mann\pwindex{Lens, Ernst 19.11.1890 – 1962@\textsc{Lens, Ernst} (19.11.1890 – 1962), \emph{Unternehmer}|pwv}.\pend
           \pstart
           Der Brief ist nur für Dich, denn Dir gegenüber bin ich doch nie erwachsen \textcolor{gray}{u}
                        geniere mich daher Dir zu sagen, wie lieb ich Dich habe.\pend
           
         
         \endnumbering\mylabel{h}\end{ledgroupsized}  \newcommand{\dateiname}{L02488}\newcommand{\titel}{Mirjam Beer-Hofmann an Arthur Schnitzler, 8. 6. 1927}\newcommand{\editorInnen}{Martin Anton Müller und Gerd-Hermann Susen}%% latex-leseansicht-abspann.tex
%% Abspann für die Leseansicht.
%% Der Schalter \ifkorrekturansicht ist bereits durch den Vorspann gesetzt.

%% latex-abspann.tex
%% Gemeinsamer Abspann für Korrekturansicht und Leseansicht.
%% Setzt den Schalter \ifkorrekturansicht voraus (gesetzt in den
%% einbindenden Dateien latex-korrekturansicht-abspann.tex bzw.
%% latex-leseansicht-abspann.tex).
%% ---------------------------------------------------------------

\normalsize

% Das esempio-Environment wird nur in der Leseansicht benötigt
\ifkorrekturansicht\else
\newenvironment{esempio}[3]%
{
    \vspace{1.5ex}
    \rlap{\underline{#1}}
    \par
    \setlength{\parindent}{0cm}
    \nopagebreak
    \leftskip=#2cm
    \rightskip=#3cm
}
{
    \par
}
\fi

\doendnotes{C}
\bigskip
\vfill

\clearpage

\footnotesize

\ifkorrekturansicht
  \lohead{\textsc{register}}
\fi

% theindex-Environment neu definieren ohne reledmac
\makeatletter
\renewenvironment{theindex}{%
  \ifkorrekturansicht
    \section*{\indexname}%
  \else
    \subsubsection*{Index der erwähnten Entitäten}%
  \fi
  \setlength{\parindent}{0pt}%
  \setlength{\parskip}{0pt plus 0.3pt}%
  \let\item\@idxitem
}{%
  \ifkorrekturansicht\clearpage\fi
}
\makeatother

\IfFileExists{\jobname-pw.ind}{\input{\jobname-pw.ind}}{}

% Quellenangabe nur in der Leseansicht
\ifkorrekturansicht\else
% Fallback-Definitionen, falls die .tex-Datei \titel etc. nicht gesetzt hat
\providecommand{\titel}{}
\providecommand{\editorInnen}{}
\providecommand{\dateiname}{\jobname}

\vspace{3cm}

\vfill

\footnotesize
\textsc{Quelle}: \titel. Herausgegeben von {\editorInnen}. In: \emph{Arthur Schnitzler: Briefwechsel mit Autorinnen und Autoren}.
 Digitale Edition, https://schnitzler-briefe.acdh.oeaw.ac.at/{\dateiname}.html (Stand \today)
\fi

\end{document}


      