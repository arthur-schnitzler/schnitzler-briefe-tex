%% latex-leseansicht-vorspann.tex
%% Vorspann für die Leseansicht.
%% Lädt die gemeinsame Datei latex-vorspann.tex mit nicht gesetztem Schalter.

\newif\ifkorrekturansicht
\korrekturansichtfalse

\input{../tex-inputs/latex-vorspann}


\section[Arthur Schnitzler an Richard Beer-Hofmann, 6. 6. 1891]{L00018 Arthur Schnitzler an Richard Beer-Hofmann, 6. 6. 1891}
\nopagebreak\mylabel{L00018v}
\rehead{ }\normalsize\beginnumbering\briefempfaengerindex{Beer-Hofmann, Richard@\textsc{Beer-Hofmann, Richard}!zzzSchnitzler, Arthur@\emph{von Arthur Schnitzler}!1891-06-061@{6. 6. 1891}|(be}
\toendnotes[C]{\smallbreak\pagebreak[2]}
\correspDesc{Versand  durch Arthur Schnitzler am 6. 6. 1891 in Wien
\newline{}Erhalt  durch Richard Beer-Hofmann im Zeitraum [7. 6. 1891
                  – 11. 6. 1891?] in Brünn}\toendnotes[C]{\smallbreak}
\Standort{YCGL, MSS 31.}
\physDesc{Brief, 1 Blatt, 4 Seiten, Kuvert, 1395 Zeichen
\newline{}Handschrift: schwarze Tinte, deutsche Kurrent
\newline{}Versand: Stempel: »\nobreak{}\oindex{Wien@\textbf{Wien}, \emph{Verwaltungsgebiet}|pwk}Wien, 6 6 91, 4–5 N\nobreak{}«.  }
\buchAbdrucke{\weitereDrucke{1) Arthur Schnitzler: \emph{Briefe 1875–1912}. Herausgegeben von Therese Nickl und Heinrich Schnitzler. Frankfurt am Main: \emph{S. Fischer} 1981, S. 117.} \weitereDrucke{2) Arthur Schnitzler, Richard Beer-Hofmann: \emph{Briefwechsel 1891–1931}. Herausgegeben von Konstanze Fliedl. Wien, Zürich: \emph{Europaverlag} 1992, S. 30–31.} }\toendnotes[C]{\smallbreak}\pstart{}{\pb}\textcolor{gray}{\textbf{\textit{\label{T_L00018-1v}\edtext{AS}{\lemma{\textnormal{\emph{AS}}}\Cendnote{\textnormal{rotes Wachssiegel}}}\label{T_L00018-1}}}}\pend{}{\bigskip}\pstart{}{\pb}\textsc{Herrn Dr. Rich. Beer-Hofmann}\pend{}\pstart{}\textsc{Brünn}\oindex{Brünn@\textbf{Brünn}|pw}\pend{}\pstart{}\textsc{Hotel Neuhauser}\oindex{Hotel Neuhauser@\textbf{Hotel Neuhauser}, \emph{Hotel}|pw}\pend{}\pstart{}\textsc{Mähren}\oindex{Mähren@\textbf{Mähren}, \emph{Region}|pw}\pend{}{\bigskip}\vspace{1em}
\pstart
           \raggedleft{}{\pb}Wien\oindex{Wien@\textbf{Wien}, \emph{Verwaltungsgebiet}|pw}{ }6. 6. 91.\pend
           \vspace{0.5em}
\pstart
           Lieber Richard, ich grüße Sie vielmals und danke Ihnen für Ihre
               liebenswürdigen Zeilen. Nächſtens werden Sie etwas{ }ſchreiben müſſen; das{ }ſteht feſt.
               Ich habe die Idee angeregt, zuſa{\geminationm}en ein Buch zu ediren
               (was \label{K_L00018-1v}\edtext{nicht von Edi = Kafka\pwindex{Kafka, Eduard Michael 11.\,3.\,1869 Wien – 6.\,8.\,1893 Brünn@\textsc{Kafka, Eduard Michael} (11.\,3.\,1869 Wien – 6.\,8.\,1893 Brünn), \emph{Redakteur}|pw}}{\lemma{\textnormal{\emph{nicht von Edi = Kafka}}}\Cendnote{\textnormal{Kafka\pwindex{Kafka, Eduard Michael 11.\,3.\,1869 Wien – 6.\,8.\,1893 Brünn@\textsc{Kafka, Eduard Michael} (11.\,3.\,1869 Wien – 6.\,8.\,1893 Brünn), \emph{Redakteur}|pwk} forderte Schnitzler erst Ende August 1891 auf, an einem
                  »Oesterreichischen Jahrbuch für moderne Literatur« mitzuarbeiten; vgl. XXXX Auszeichnungsfehler: Dokument L00037 nicht gefunden.}}}\label{K_L00018-1} ko{\geminationm}t) Titel: \label{K_L00018-2v}\edtext{Aus der Kaffehausecke\pwindex{Schnitzler, Arthur 15.\,5.\,1862 Wien – 21.\,10.\,1931 ebd.@\textsc{Schnitzler, Arthur} (15.\,5.\,1862 Wien – 21.\,10.\,1931 ebd.), \emph{Schriftsteller, Mediziner}!Aus der Kaffeehausecke@\strich\emph{Aus der Kaffeehausecke}|pw}}{\lemma{\textnormal{\emph{Aus der Kaffehausecke}}}\Cendnote{\textnormal{Diesen Titel trug die von Bölsche\pwindex{Bölsche, Wilhelm 2.\,1.\,1861 Köln – 31.\,8.\,1939 Szklarska Poręba@\textsc{Bölsche, Wilhelm} (2.\,1.\,1861 Köln – 31.\,8.\,1939 Szklarska Poręba), \emph{Schriftsteller, Publizist}|pwk} vor Jahresfrist abgelehnte Novelle\pwindex{Schnitzler, Arthur 15.\,5.\,1862 Wien – 21.\,10.\,1931 ebd.@\textsc{Schnitzler, Arthur} (15.\,5.\,1862 Wien – 21.\,10.\,1931 ebd.), \emph{Schriftsteller, Mediziner}!Aus der Kaffeehausecke@\strich\emph{Aus der Kaffeehausecke}|pwkv}, die bislang
                  unveröffentlicht geblieben war; vgl. XXXX Auszeichnungsfehler: Dokument L00004 nicht gefunden.}}}\label{K_L00018-2}. Sa{\geminationm}lung von
               Skizzen, Noveletten, Impreſſionen, Aphorismen – {\pb}jeder
               hat möglichſt individuell zu{ }ſein – außerdem würde ich einen erhöhten Wien\oindex{Wien@\textbf{Wien}, \emph{Verwaltungsgebiet}|pw}er Ton (jenen Ton, der nicht im Dialekt
               beſteht) bevorzugen\strikeout{)}.\pend
           
\pstart
           Ich{ }ſpreche noch näher mit Ihnen drüber; Sie haben meiner Idee nach{ }ſehr viel damit
               zu{ }ſchaffen. Intereſſant iſt, wie einige, als Ihr Name gena{\geminationn}t wurde, mit einer gewiſſen Wehmut{ }ſagten: »Ja, we{\geminationn}{ }{\pb}man von dem was kriegen könnte« –\pend
           
\pstart
           – In Ihnen muſs ja{ }ſchließlich die Poeſie \uline{herangeglaubt} werden. Ich mache Sie auf dieſes Wort ganz beſonders
               aufmerkſam. – Die Zwiſchengeſpräche und Zwiſchengeſchichten der Kaffehausecke,
               bedürfen beſondrer Ueberlegung – ich freue mich{ }ſehr, mit Ihnen drüber plaudern zu
               können. Darüber u über andres, {\pb}bitte recht{ }ſehr,
               deſertiren Sie ehebaldigſt. Wie lang wird man Sie denn da{\geminationn} in Wien\oindex{Wien@\textbf{Wien}, \emph{Verwaltungsgebiet}|pw} genießen können? Man{ }ſehnt{ }ſich nach
               Ihnen, und die meiſten grüßen Sie herzlichſt. Haben Sie wirklich gar{ }ſo viel zu
               thun?\pend
           
\pstart
           Schreiben Sie mir,{ }ſobald Sie wieder hier{ }ſind, d. h. lieber früher, we{\geminationn}{ }Sie Laune haben u{ }ſobald Sie da, ko{\geminationm}en Sie zu\pend
           \pstart Ihrem \spacefill\mbox{Arthur S}\pend{}\selectlanguage{ngerman}\endnumbering\briefempfaengerindex{Beer-Hofmann, Richard@\textsc{Beer-Hofmann, Richard}!zzzSchnitzler, Arthur@\emph{von Arthur Schnitzler}!1891-06-061@{6. 6. 1891}|)be}\mylabel{L00018h}  \newcommand{\dateiname}{L00018}\newcommand{\titel}{Arthur Schnitzler an Richard Beer-Hofmann, 6. 6. 1891}\newcommand{\editorInnen}{Martin Anton Müller und Gerd-Hermann Susen}%% latex-leseansicht-abspann.tex
%% Abspann für die Leseansicht.
%% Der Schalter \ifkorrekturansicht ist bereits durch den Vorspann gesetzt.

%% latex-abspann.tex
%% Gemeinsamer Abspann für Korrekturansicht und Leseansicht.
%% Setzt den Schalter \ifkorrekturansicht voraus (gesetzt in den
%% einbindenden Dateien latex-korrekturansicht-abspann.tex bzw.
%% latex-leseansicht-abspann.tex).
%% ---------------------------------------------------------------

\normalsize

% Das esempio-Environment wird nur in der Leseansicht benötigt
\ifkorrekturansicht\else
\newenvironment{esempio}[3]%
{
    \vspace{1.5ex}
    \rlap{\underline{#1}}
    \par
    \setlength{\parindent}{0cm}
    \nopagebreak
    \leftskip=#2cm
    \rightskip=#3cm
}
{
    \par
}
\fi

\doendnotes{C}
\bigskip
\vfill

\clearpage

\footnotesize

\ifkorrekturansicht
  \lohead{\textsc{register}}
\fi

% theindex-Environment neu definieren ohne reledmac
\makeatletter
\renewenvironment{theindex}{%
  \ifkorrekturansicht
    \section*{\indexname}%
  \else
    \subsubsection*{Index der erwähnten Entitäten}%
  \fi
  \setlength{\parindent}{0pt}%
  \setlength{\parskip}{0pt plus 0.3pt}%
  \let\item\@idxitem
}{%
  \ifkorrekturansicht\clearpage\fi
}
\makeatother

\IfFileExists{\jobname-pw.ind}{\input{\jobname-pw.ind}}{}

% Quellenangabe nur in der Leseansicht
\ifkorrekturansicht\else
% Fallback-Definitionen, falls die .tex-Datei \titel etc. nicht gesetzt hat
\providecommand{\titel}{}
\providecommand{\editorInnen}{}
\providecommand{\dateiname}{\jobname}

\vspace{3cm}

\vfill

\footnotesize
\textsc{Quelle}: \titel. Herausgegeben von {\editorInnen}. In: \emph{Arthur Schnitzler: Briefwechsel mit Autorinnen und Autoren}.
 Digitale Edition, https://schnitzler-briefe.acdh.oeaw.ac.at/{\dateiname}.html (Stand \today)
\fi

\end{document}


