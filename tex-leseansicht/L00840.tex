%% latex-korrekturansicht-vorspann.tex
%% Vorspann für die Korrekturansicht.
%% Lädt die gemeinsame Datei latex-vorspann.tex mit gesetztem Schalter.

\newif\ifkorrekturansicht
\korrekturansichttrue

\input{../tex-inputs/latex-vorspann}


\section[Arthur Schnitzler an Hugo von Hofmannsthal, 27. 8. 1898]{L00840 Arthur Schnitzler an Hugo von Hofmannsthal, 27. 8. 1898}
\nopagebreak\mylabel{L00840v}
\rehead{ }\normalsize\beginnumbering\briefempfaengerindex{Hofmannsthal, Hugo von@\textsc{Hofmannsthal, Hugo von}!zzzSchnitzler, Arthur@\emph{von Arthur Schnitzler}!1898-08-271@{27. 8. 1898}|(be}
\toendnotes[C]{\smallbreak\pagebreak[2]}\Standort{FDH, Hs-30885,75.}
\physDesc{Postkarte, 611 Zeichen
\newline{}Handschrift: Bleistift, deutsche Kurrent
\newline{}Versand: 1) Stempel: »\nobreak{}\oindex{Alpnachstad@\textbf{Alpnachstad}, \emph{P.PPLX}|pwk}Alpnach Stad, 27. VIII. 98\nobreak{}«.   2) Stempel: »\nobreak{}\oindex{Lugano@\textbf{Lugano}, \emph{P.PPLA2}|pwk}L{[}ugano{]}
                                       Lettere, 27. VIII. 98\nobreak{}«. 
\newline{}Ordnung: mit Bleistift von Schnitzler mutmaßlich bei der Durchsicht der Briefe
                                    1929 datiert: »27/8 98« }
\buchAbdrucke{\weitereDrucke{Hugo von Hofmannsthal, Arthur Schnitzler: \emph{Briefwechsel}. Frankfurt am Main: \emph{S. Fischer} 1964, S. 111.} }\toendnotes[C]{\smallbreak}\pstart{}{\pb}\textsc{Herrn Hugo von Hofmannsthal}, \pend{}\pstart{}\textsc{Lugano}\oindex{Lugano@\textbf{Lugano}, \emph{P.PPLA2}|pw}\pend{}\pstart{}\textsc{Hotel du parc\oindex{Hôtel du Parc {\kaufmannsund} Bristol@\textbf{Hôtel du Parc {\kaufmannsund} Bristol}, \emph{Hotel (K.HTL)}|pw}}.\pend{}{\bigskip}\vspace{1em}
\pstart
           \noindent{}{\pb}Mein lieber Hugo, ich bin hieher \textsc{per} Rad
               gefahren; will \textsc{per} Bahn auf den Pilatus\oindex{Pilatus@\textbf{Pilatus}, \emph{Berg (N.BRG)}|pw}. Morgen denk ich Luzern\oindex{Luzern@\textbf{Luzern}, \emph{P.PPLA}|pw} zu verlaſſen, in dem ich mich ganz wohl behagt, nur phyſiſch nicht ſo
                  beiſa{\geminationm}en war als ich gewünſcht. Ich will die Route
                  \textsc{Mailand\oindex{Mailand@\textbf{Mailand}, \emph{P.PPLA}|pw} – (Pavia\oindex{Pavia@\textbf{Pavia}, \emph{P.PPLA2}|pw} –)}{ }Piazenza\oindex{Piacenza@\textbf{Piacenza}, \emph{P.PPLA2}|pw} – (\textsc{Parma\oindex{Parma@\textbf{Parma}, \emph{P.PPLA2}|pw}) Modena\oindex{Modena@\textbf{Modena}, \emph{P.PPLA2}|pw} – Bologna\oindex{Bologna@\textbf{Bologna}, \emph{P.PPLA}|pw} – Ferrara\oindex{Ferrara@\textbf{Ferrara}, \emph{A.ADM3}|pw} – Padua\oindex{Padua@\textbf{Padua}, \emph{P.PPLA2}|pw} – Vicenza\oindex{Vicenza@\textbf{Vicenza}, \emph{P.PPLA2}|pw} – \strikeout{(Ve}{ }Verona\oindex{Verona@\textbf{Verona}, \emph{P.PPLA2}|pw}} – Wien\oindex{Wien@\textbf{Wien}, \emph{A.ADM2}|pw} einſchlagen. – Geſtern hab ich eine
               kleine Geſchichte\pwindex{Excentric@\emph{Excentric}|pwv} zu ſchreiben
               angefangen. Schreiben Sie mir ein Wort nach \textsc{Bologna\oindex{Bologna@\textbf{Bologna}, \emph{P.PPLA}|pw} post rest}. Grüßen Sie Richard\pwindex{Beer-Hofmann, Richard 1866-07-11 – 1945-09-26@\textsc{Beer-Hofmann, Richard} (1866-07-11 – 1945-09-26), \emph{Schriftsteller/Schriftstellerin}|pw} von mir, we{\geminationn} er ko{\geminationm}t. Ich hoffe Sie gut gelaunt und heiter und bin
               von Herzen \pend
           \pstart Ihr \spacefill\mbox{Arthur}\pend{}
\pstart
           \textsc{Alpnach Stad}\oindex{Alpnachstad@\textbf{Alpnachstad}, \emph{P.PPLX}|pw}, \substVorne{}\textsuperscript{Frei}\substDazwischen{}Samſ\substHinten{}tag{ }früh.\pend
           \selectlanguage{ngerman}\endnumbering\briefempfaengerindex{Hofmannsthal, Hugo von@\textsc{Hofmannsthal, Hugo von}!zzzSchnitzler, Arthur@\emph{von Arthur Schnitzler}!1898-08-271@{27. 8. 1898}|)be}\mylabel{L00840h}  \normalsize

\doendnotes{C}
\bigskip
\vfill

\clearpage

\footnotesize

\lohead{\textsc{register}}

% Definiere theindex-Environment komplett neu ohne reledmac
\makeatletter
\renewenvironment{theindex}{%
  \section*{\indexname}%
  \setlength{\parindent}{0pt}%
  \setlength{\parskip}{0pt plus 0.3pt}%
  \let\item\@idxitem
}{%
  \clearpage
}
\makeatother

\IfFileExists{\jobname-pw.ind}{\input{\jobname-pw.ind}}{}

\end{document}

      