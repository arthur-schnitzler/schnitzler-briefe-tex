%% latex-leseansicht-vorspann.tex
%% Vorspann für die Leseansicht.
%% Lädt die gemeinsame Datei latex-vorspann.tex mit nicht gesetztem Schalter.

\newif\ifkorrekturansicht
\korrekturansichtfalse

\input{../tex-inputs/latex-vorspann}


         
         \renewcommand{\erwaehntePersonen}{Personen: Richard Beer-Hofmann, Hugo von Hofmannsthal}
         \renewcommand{\erwaehnteOrte}{Orte: Alpnachstad, Bologna, Ferrara, Hôtel du Parc {\kaufmannsund}  Bristol, Lugano, Luzern, Mailand, Modena, Padua, Parma, Pavia, Piacenza, Pilatus, Verona, Vicenza, Wien}
         \renewcommand{\erwaehnteWerke}{Werke: Excentric}
               \section[Arthur Schnitzler an Hugo von Hofmannsthal, 27. 8. 1898]{ Arthur Schnitzler an Hugo von Hofmannsthal, 27. 8. 1898}\nopagebreak\mylabel{v}\rehead{ }\begin{ledgroupsized}[t]{13cm}\normalsize\beginnumbering \toendnotes[C]{\smallbreak\pagebreak[2]} \Standort{FDH, Hs-30885,75.}
\physDesc{Postkarte
\newline{}Handschrift: Bleistift, deutsche Kurrent\newline{}Versand: 1) Stempel: »\nobreak{}\oindex{Alpnachstad@\textbf{Alpnachstad}|pwk}Alpnach Stad, 27. VIII. 98\nobreak{}«.   2) Stempel: »\nobreak{}\oindex{Lugano@\textbf{Lugano}|pwk}L{[}ugano{]}
                                        Lettere, 27. VIII. 98\nobreak{}«. \newline{}Ordnung: von Schnitzler mumaßlich bei der Durchsicht der Briefe 1929 mit Bleistift datiert: »27/8 98« }\buchAbdrucke{\weitereDrucke{Hugo von Hofmannsthal, Arthur Schnitzler: \emph{Briefwechsel}. Hg. Therese Nickl und Heinrich Schnitzler. Frankfurt am Main: \emph{S. Fischer} 1964, S. 111.} }\toendnotes[C]{\smallbreak}\pstart{}{\pb}\textsc{Herrn Hugo von Hofmannsthal}, \pend{}\pstart{}\textsc{Lugano}\oindex{Lugano@\textbf{Lugano}|pw}\pend{}\pstart{}\textsc{Hotel du parc\oindex{Hôtel du Parc {\kaufmannsund} Bristol@\textbf{Hôtel du Parc {\kaufmannsund} Bristol}|pw}}.\pend{}{\bigskip}\pstart
           \noindent{}{\pb}Mein lieber Hugo, ich bin hieher \textsc{per}
                    Rad gefahren; will \textsc{per} Bahn auf den Pilatus\oindex{Pilatus@\textbf{Pilatus}|pw}. Morgen denk ich Luzern\oindex{Luzern@\textbf{Luzern}|pw} zu verlaſſen, in dem ich mich ganz wohl behagt, nur phyſiſch
                    nicht ſo beiſa{\geminationm}en war als ich gewünſcht. Ich will
                    die Route \textsc{Mailand\oindex{Mailand@\textbf{Mailand}|pw} – (Pavia\oindex{Pavia@\textbf{Pavia}|pw} –)}{ }Piazenza\oindex{Piacenza@\textbf{Piacenza}|pw} – (\textsc{Parma\oindex{Parma@\textbf{Parma}|pw}) Modena\oindex{Modena@\textbf{Modena}|pw} – Bologna\oindex{Bologna@\textbf{Bologna}|pw} – Ferrara\oindex{Ferrara@\textbf{Ferrara}|pw} – Padua\oindex{Padua@\textbf{Padua}|pw} – Vicenza\oindex{Vicenza@\textbf{Vicenza}|pw} – \strikeout{(Ve}{ }Verona\oindex{Verona@\textbf{Verona}|pw}} – Wien\oindex{Wien@\textbf{Wien}|pw} einſchlagen. – Geſtern hab ich eine
                    kleine Geſchichte\pwindex{Schnitzler, Arthur 15.05.1862 – 21.10.1931@\textsc{Schnitzler, Arthur} (15.05.1862 – 21.10.1931), \emph{Schriftsteller, Mediziner}!Excentric16. 07. 1902@\strich\emph{Excentric} {[}16. 07. 1902{]}|pwv} zu
                    ſchreiben angefangen. Schreiben Sie mir ein Wort nach \textsc{Bologna\oindex{Bologna@\textbf{Bologna}|pw} post rest}. Grüßen Sie Richard\pwindex{Beer-Hofmann, Richard 1866-07-11 – 1945-09-26@\textsc{Beer-Hofmann, Richard} (1866-07-11 – 1945-09-26), \emph{Schriftsteller}|pw} von mir, we{\geminationn} er ko{\geminationm}t. Ich hoffe
                    Sie gut gelaunt und heiter und bin von Herzen \pend
           \pstart Ihr \spacefill\mbox{Arthur}\pend{}\pstart
           \textsc{Alpnach Stad}\oindex{Alpnachstad@\textbf{Alpnachstad}|pw}, \substVorne{}\textsuperscript{Frei}\substDazwischen{}Samſ\substHinten{}tag{ }früh.\pend
           
         
         \endnumbering\mylabel{h}\end{ledgroupsized}  \newcommand{\dateiname}{L00840}\newcommand{\titel}{Arthur Schnitzler an Hugo von Hofmannsthal, 27. 8. 1898}\newcommand{\editorInnen}{Martin Anton Müller und Gerd-Hermann Susen}%% latex-leseansicht-abspann.tex
%% Abspann für die Leseansicht.
%% Der Schalter \ifkorrekturansicht ist bereits durch den Vorspann gesetzt.

%% latex-abspann.tex
%% Gemeinsamer Abspann für Korrekturansicht und Leseansicht.
%% Setzt den Schalter \ifkorrekturansicht voraus (gesetzt in den
%% einbindenden Dateien latex-korrekturansicht-abspann.tex bzw.
%% latex-leseansicht-abspann.tex).
%% ---------------------------------------------------------------

\normalsize

% Das esempio-Environment wird nur in der Leseansicht benötigt
\ifkorrekturansicht\else
\newenvironment{esempio}[3]%
{
    \vspace{1.5ex}
    \rlap{\underline{#1}}
    \par
    \setlength{\parindent}{0cm}
    \nopagebreak
    \leftskip=#2cm
    \rightskip=#3cm
}
{
    \par
}
\fi

\doendnotes{C}
\bigskip
\vfill

\clearpage

\footnotesize

\ifkorrekturansicht
  \lohead{\textsc{register}}
\fi

% theindex-Environment neu definieren ohne reledmac
\makeatletter
\renewenvironment{theindex}{%
  \ifkorrekturansicht
    \section*{\indexname}%
  \else
    \subsubsection*{Index der erwähnten Entitäten}%
  \fi
  \setlength{\parindent}{0pt}%
  \setlength{\parskip}{0pt plus 0.3pt}%
  \let\item\@idxitem
}{%
  \ifkorrekturansicht\clearpage\fi
}
\makeatother

\IfFileExists{\jobname-pw.ind}{\input{\jobname-pw.ind}}{}

% Quellenangabe nur in der Leseansicht
\ifkorrekturansicht\else
% Fallback-Definitionen, falls die .tex-Datei \titel etc. nicht gesetzt hat
\providecommand{\titel}{}
\providecommand{\editorInnen}{}
\providecommand{\dateiname}{\jobname}

\vspace{3cm}

\vfill

\footnotesize
\textsc{Quelle}: \titel. Herausgegeben von {\editorInnen}. In: \emph{Arthur Schnitzler: Briefwechsel mit Autorinnen und Autoren}.
 Digitale Edition, https://schnitzler-briefe.acdh.oeaw.ac.at/{\dateiname}.html (Stand \today)
\fi

\end{document}


      