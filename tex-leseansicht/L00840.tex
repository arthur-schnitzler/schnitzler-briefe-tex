\input{../tex-inputs/latex-pdf-vorspann}
\begin{center}
            \textcolor{red}{ENTWURF. ENTZIFFERUNG NOCH NICHT KORREKTURGELESEN}
                      \end{center}
            
               \section[Arthur Schnitzler an Hugo von Hofmannsthal, 27. 8. 1898]{ Arthur Schnitzler an Hugo von Hofmannsthal, 27. 8. 1898}\nopagebreak\mylabel{v}\rehead{ }\begin{ledgroupsized}[t]{13cm}\normalsize\beginnumbering\briefempfaengerindex{Hofmannsthal, Hugo von@\textsc{Hofmannsthal, Hugo von}!zzzSchnitzler, Arthur@\emph{von Arthur Schnitzler}!1898-08-271@{27. 8. 1898}|(be} \toendnotes[C]{\smallbreak\pagebreak[2]} \Standort{FDH, Hs-30885,75.}
\physDesc{Postkarte
\newline{}Handschrift: Bleistift, deutsche Kurrent\newline{}Versand: 1) Stempel: »\nobreak{}\oindex{Alpnachstad@\textbf{Alpnachstad}|pwk}Alpnach Stad, 27. VIII. 98\nobreak{}«.  2) Stempel: »\nobreak{}\oindex{Lugano@\textbf{Lugano}|pwk}L{[}ugano{]}
                                        Lettere, 27. VIII. 98\nobreak{}«. \newline{}Ordnung: von Schnitzler mumaßlich bei der Durchsicht der Briefe 1929 mit Bleistift datiert: »27/8 98« }\buchAbdrucke{\weitereDrucke{Hugo von Hofmannsthal, Arthur Schnitzler: \emph{Briefwechsel}. Hg. Therese Nickl und Heinrich Schnitzler. Frankfurt am Main: \emph{S. Fischer} 1964, S. 111.} }\toendnotes[C]{\smallbreak}\pstart{}{\pb}\textsc{Herrn Hugo von Hofmannsthal}, \pend{}\pstart{}\textsc{Lugano}\oindex{Lugano@\textbf{Lugano}|pw}\pend{}\pstart{}\textsc{Hotel du parc\oindex{Hôtel du Parc {\kaufmannsund} Bristol@\textbf{Hôtel du Parc {\kaufmannsund} Bristol}|pw}}.\pend{}{\bigskip}\pstart
           \noindent{}{\pb}Mein lieber Hugo, ich bin hieher \textsc{per}
                    Rad gefahren; will \textsc{per} Bahn auf den Pilatus\oindex{Pilatus@\textbf{Pilatus}|pw}. Morgen denk ich Luzern\oindex{Luzern@\textbf{Luzern}|pw} zu verlaſſen, in dem ich mich ganz wohl behagt, nur phyſiſch
                    nicht ſo beiſa{\geminationm}en war als ich gewünſcht. Ich will
                    die Route \textsc{Mailand\oindex{Mailand@\textbf{Mailand}|pw} – (Pavia\oindex{Pavia@\textbf{Pavia}|pw} –)}{ }Piazenza\oindex{Piacenza@\textbf{Piacenza}|pw} – (\textsc{Parma\oindex{Parma@\textbf{Parma}|pw}) Modena\oindex{Modena@\textbf{Modena}|pw} – Bologna\oindex{Bologna@\textbf{Bologna}|pw} – Ferrara\oindex{Ferrara@\textbf{Ferrara}|pw} – Padua\oindex{Padua@\textbf{Padua}|pw} – Vicenza\oindex{Vicenza@\textbf{Vicenza}|pw} – \strikeout{(Ve}{ }Verona\oindex{Verona@\textbf{Verona}|pw}} – Wien\oindex{Wien@\textbf{Wien}|pw} einſchlagen. – Geſtern hab ich eine
                    kleine Geſchichte\pwindex{Schnitzler, Arthur 15.05.1862 – 21.10.1931@\textsc{Schnitzler, Arthur} (15.05.1862 – 21.10.1931), \emph{Schriftsteller, Mediziner}!Excentric16. 07. 1902@\strich\emph{Excentric} {[}16. 07. 1902{]}|pwv} zu
                    ſchreiben angefangen. Schreiben Sie mir ein Wort nach \textsc{Bologna\oindex{Bologna@\textbf{Bologna}|pw} post rest}. Grüßen Sie Richard\pwindex{Beer-Hofmann, Richard 11.07.1866 – 26.09.1945@\textsc{Beer-Hofmann, Richard} (11.07.1866 – 26.09.1945), \emph{Schriftsteller}|pw} von mir, we{\geminationn} er ko{\geminationm}t. Ich hoffe
                    Sie gut gelaunt und heiter und bin von Herzen \pend
           \pstart Ihr \spacefill\mbox{Arthur}\pend{}\pstart
           \textsc{Alpnach Stad}\oindex{Alpnachstad@\textbf{Alpnachstad}|pw}, \substVorne{}\textsuperscript{Frei}\substDazwischen{}Samſ\substHinten{}tag{ }früh.\pend
           \endnumbering\briefempfaengerindex{Hofmannsthal, Hugo von@\textsc{Hofmannsthal, Hugo von}!zzzSchnitzler, Arthur@\emph{von Arthur Schnitzler}!1898-08-271@{27. 8. 1898}|)be}\mylabel{h}\end{ledgroupsized}  \newcommand{\dateiname}{L00840}\newcommand{\titel}{Arthur Schnitzler an Hugo von Hofmannsthal, 27. 8. 1898}\newcommand{\editorInnen}{Martin Anton Müller und Gerd-Hermann Susen}\input{../tex-inputs/latex-pdf-abspann}
      