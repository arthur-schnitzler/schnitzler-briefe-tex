%% latex-leseansicht-vorspann.tex
%% Vorspann für die Leseansicht.
%% Lädt die gemeinsame Datei latex-vorspann.tex mit nicht gesetztem Schalter.

\newif\ifkorrekturansicht
\korrekturansichtfalse

\input{../tex-inputs/latex-vorspann}


\section[Arthur Schnitzler an Hugo von Hofmannsthal, 27. 8. 1898]{L00840 Arthur Schnitzler an Hugo von Hofmannsthal, 27. 8. 1898}
\nopagebreak\mylabel{L00840v}
\rehead{ }\normalsize\beginnumbering\briefempfaengerindex{Hofmannsthal, Hugo von@\textsc{Hofmannsthal, Hugo von}!zzzSchnitzler, Arthur@\emph{von Arthur Schnitzler}!1898-08-271@{27. 8. 1898}|(be}
\toendnotes[C]{\smallbreak\pagebreak[2]}
\correspDesc{Versand  durch Arthur Schnitzler am 27. 8. 1898 in Alpnachstad
\newline{}Erhalt  durch Hugo von Hofmannsthal im Zeitraum [28. 8. 1898
                  – 1. 9. 1898?] in Lugano}\toendnotes[C]{\smallbreak}
\Standort{FDH, Hs-30885,75.}
\physDesc{Postkarte, 611 Zeichen
\newline{}Handschrift: Bleistift, deutsche Kurrent
\newline{}Versand: 1) Stempel: »\nobreak{}\oindex{Alpnachstad@\textbf{Alpnachstad}, \emph{Ehemaliger Ort}|pwk}Alpnach Stad, 27. VIII. 98\nobreak{}«.   2) Stempel: »\nobreak{}\oindex{Lugano@\textbf{Lugano}, \emph{Hauptstadt}|pwk}L{[}ugano{]}
                                       Lettere, 27. VIII. 98\nobreak{}«. 
\newline{}Ordnung: mit Bleistift von Schnitzler mutmaßlich bei der Durchsicht der Briefe
                                    1929 datiert: »27/8 98« }
\buchAbdrucke{\weitereDrucke{Hugo von Hofmannsthal, Arthur Schnitzler: \emph{Briefwechsel}. Herausgegeben von Therese Nickl und Heinrich Schnitzler. Frankfurt am Main: \emph{S. Fischer} 1964, S. 111.} }\toendnotes[C]{\smallbreak}\pstart{}{\pb}\textsc{Herrn Hugo von Hofmannsthal}, \pend{}\pstart{}\textsc{Lugano}\oindex{Lugano@\textbf{Lugano}, \emph{Hauptstadt}|pw}\pend{}\pstart{}\textsc{Hotel du parc\oindex{Hôtel du Parc {\kaufmannsund} Bristol@\textbf{Hôtel du Parc {\kaufmannsund} Bristol}, \emph{Hotel}|pw}}.\pend{}{\bigskip}\vspace{1em}
\pstart
           \noindent{}{\pb}Mein lieber Hugo, ich bin hieher \textsc{per} Rad
               gefahren; will \textsc{per} Bahn auf den Pilatus\oindex{Pilatus@\textbf{Pilatus}, \emph{Berg}|pw}. Morgen denk ich Luzern\oindex{Luzern@\textbf{Luzern}|pw} zu verlaſſen, in dem ich mich ganz wohl behagt, nur phyſiſch nicht{ }ſo
                  beiſa{\geminationm}en war als ich gewünſcht. Ich will die Route
                  \textsc{Mailand\oindex{Mailand@\textbf{Mailand}|pw} – (Pavia\oindex{Pavia@\textbf{Pavia}, \emph{Hauptstadt}|pw} –)}{ }Piazenza\oindex{Piacenza@\textbf{Piacenza}, \emph{Hauptstadt}|pw} – (\textsc{Parma\oindex{Parma@\textbf{Parma}, \emph{Hauptstadt}|pw}) Modena\oindex{Modena@\textbf{Modena}, \emph{Hauptstadt}|pw} – Bologna\oindex{Bologna@\textbf{Bologna}|pw} – Ferrara\oindex{Ferrara@\textbf{Ferrara}, \emph{Verwaltungsgebiet}|pw} – Padua\oindex{Padua@\textbf{Padua}, \emph{Hauptstadt}|pw} – Vicenza\oindex{Vicenza@\textbf{Vicenza}, \emph{Hauptstadt}|pw} – \strikeout{(Ve}{ }Verona\oindex{Verona@\textbf{Verona}, \emph{Hauptstadt}|pw}} – Wien\oindex{Wien@\textbf{Wien}, \emph{Verwaltungsgebiet}|pw} einſchlagen. – Geſtern hab ich eine
               kleine Geſchichte\pwindex{Schnitzler, Arthur 15.\,5.\,1862 Wien – 21.\,10.\,1931 ebd.@\textsc{Schnitzler, Arthur} (15.\,5.\,1862 Wien – 21.\,10.\,1931 ebd.), \emph{Schriftsteller, Mediziner}!Excentric@\strich\emph{Excentric}|pwv} zu{ }ſchreiben
               angefangen. Schreiben Sie mir ein Wort nach \textsc{Bologna\oindex{Bologna@\textbf{Bologna}|pw} post rest}. Grüßen Sie Richard\pwindex{Beer-Hofmann, Richard 11.\,7.\,1866 Wien – 26.\,9.\,1945 New York City@\textsc{Beer-Hofmann, Richard} (11.\,7.\,1866 Wien – 26.\,9.\,1945 New York City), \emph{Schriftsteller}|pw} von mir, we{\geminationn} er ko{\geminationm}t. Ich hoffe Sie gut gelaunt und heiter und bin
               von Herzen\pend
           \pstart Ihr \spacefill\mbox{Arthur}\pend{}
\pstart
           \textsc{Alpnach Stad}\oindex{Alpnachstad@\textbf{Alpnachstad}, \emph{Ehemaliger Ort}|pw}, \substVorne{}\textsuperscript{Frei}\substDazwischen{}Samſ\substHinten{}tag{ }früh.\pend
           \selectlanguage{ngerman}\endnumbering\briefempfaengerindex{Hofmannsthal, Hugo von@\textsc{Hofmannsthal, Hugo von}!zzzSchnitzler, Arthur@\emph{von Arthur Schnitzler}!1898-08-271@{27. 8. 1898}|)be}\mylabel{L00840h}  \newcommand{\dateiname}{L00840}\newcommand{\titel}{Arthur Schnitzler an Hugo von Hofmannsthal, 27. 8. 1898}\newcommand{\editorInnen}{Martin Anton Müller und Gerd-Hermann Susen}%% latex-leseansicht-abspann.tex
%% Abspann für die Leseansicht.
%% Der Schalter \ifkorrekturansicht ist bereits durch den Vorspann gesetzt.

%% latex-abspann.tex
%% Gemeinsamer Abspann für Korrekturansicht und Leseansicht.
%% Setzt den Schalter \ifkorrekturansicht voraus (gesetzt in den
%% einbindenden Dateien latex-korrekturansicht-abspann.tex bzw.
%% latex-leseansicht-abspann.tex).
%% ---------------------------------------------------------------

\normalsize

% Das esempio-Environment wird nur in der Leseansicht benötigt
\ifkorrekturansicht\else
\newenvironment{esempio}[3]%
{
    \vspace{1.5ex}
    \rlap{\underline{#1}}
    \par
    \setlength{\parindent}{0cm}
    \nopagebreak
    \leftskip=#2cm
    \rightskip=#3cm
}
{
    \par
}
\fi

\doendnotes{C}
\bigskip
\vfill

\clearpage

\footnotesize

\ifkorrekturansicht
  \lohead{\textsc{register}}
\fi

% theindex-Environment neu definieren ohne reledmac
\makeatletter
\renewenvironment{theindex}{%
  \ifkorrekturansicht
    \section*{\indexname}%
  \else
    \subsubsection*{Index der erwähnten Entitäten}%
  \fi
  \setlength{\parindent}{0pt}%
  \setlength{\parskip}{0pt plus 0.3pt}%
  \let\item\@idxitem
}{%
  \ifkorrekturansicht\clearpage\fi
}
\makeatother

\IfFileExists{\jobname-pw.ind}{\input{\jobname-pw.ind}}{}

% Quellenangabe nur in der Leseansicht
\ifkorrekturansicht\else
% Fallback-Definitionen, falls die .tex-Datei \titel etc. nicht gesetzt hat
\providecommand{\titel}{}
\providecommand{\editorInnen}{}
\providecommand{\dateiname}{\jobname}

\vspace{3cm}

\vfill

\footnotesize
\textsc{Quelle}: \titel. Herausgegeben von {\editorInnen}. In: \emph{Arthur Schnitzler: Briefwechsel mit Autorinnen und Autoren}.
 Digitale Edition, https://schnitzler-briefe.acdh.oeaw.ac.at/{\dateiname}.html (Stand \today)
\fi

\end{document}


