%% latex-leseansicht-vorspann.tex
%% Vorspann für die Leseansicht.
%% Lädt die gemeinsame Datei latex-vorspann.tex mit nicht gesetztem Schalter.

\newif\ifkorrekturansicht
\korrekturansichtfalse

\input{../tex-inputs/latex-vorspann}


         
         \newcommand{\erwaehntePersonen}{Personen: Alfred Bernau, Alexander Engel, Heinrich Glücksmann,  Landerberg, Viktor Franz Patzner, Maria Pollak, Friedrich Rosenthal, Wolfgang Waniek}
         \newcommand{\erwaehnteOrte}{Orte: Burgtheater, Karlsbad, Volkstheater, Wegscheid, Wels, Wien}
         \newcommand{\erwaehnteWerke}{Werke: Der Fremde, Märchenkomödie, Yppl. Idylle in fünf Akten}
               \section[Robert Adam an Arthur Schnitzler, 19. 8. 1919]{ Robert Adam an Arthur Schnitzler, 19. 8. 1919}\nopagebreak\mylabel{v}\rehead{ }\begin{ledgroupsized}[t]{13cm}\normalsize\beginnumbering \toendnotes[C]{\smallbreak\pagebreak[2]} \Standort{CUL, Schnitzler, B 1.}
\physDesc{Brief, 1 Blatt, 3 Seiten
\newline{}Handschrift: blaue Tinte, deutsche Kurrent
\newline{}Schnitzler: 1) mit Bleistift beschriftet: »\textsc{Adam}«  2) mit rotem Buntstift zwei Unterstreichungen\newline{}Ordnung: mit Bleistift von unbekannter Hand nummeriert: »13« }\Standort{Wien, Österreichische Nationalbibliothek, Cod.ser. 52.268, 21 recto und 23.}
\physDesc{handschriftliche Abschrift
\newline{}Handschrift: schwarze Tinte, Gabelsberger Kurzschrift}\Standort{Wien, Österreichische Nationalbibliothek, Cod.ser. 52.268, 21 recto und 23.}
\physDesc{maschinelle Abschrift
\newline{}Schreibmaschine}\toendnotes[C]{\smallbreak}\pstart
           \raggedleft{}{\pb}Wien\oindex{Wien@\textbf{Wien}|pw}, am 19. August 1919\pend
           \pstart\center{}Hochverehrter Herr Doktor!\pend\pstart
           Von Wegſcheid bei Maria Zell\oindex{Wegscheid@\textbf{Wegscheid}|pw} zurückgekehrt, wo
                    ich nach vollbrachter Karlsbad\oindex{Karlsbad@\textbf{Karlsbad}|pw}er Kur Frau\pwindex{Pollak, Maria 06.10.1889 – 27.03.1948@\textsc{Pollak, Maria} (06.10.1889 – 27.03.1948)|pwv} und Kind\pwindex{Patzner, Viktor Franz 13.09.1916 – 21.12.1982@\textsc{Patzner, Viktor Franz} (13.09.1916 – 21.12.1982), \emph{Rechtsanwalt}|pwv} aufſuchte, um ſie glücklich
                    heimzubringen, finde ich Ihre Karte vor, die mir nach Karlsbad\oindex{Karlsbad@\textbf{Karlsbad}|pw} nachgeſchickt und von dort zurückgeſendet worden
                    war. Ich freue mich darauf, Ihnen über meine Schickſale bei Ihrer Rückkehr
                    mündlich berichten zu können; erfreulich sind ſie ſchließlich nicht. Wenn Ärger,
                    wie die Ärzte behaupten, auf die Folgeerſcheinungen von Magengeſchwüren
                    ungünſtig einwirkt, so trägt das Deutſche
                        Volkstheater\oindex{Volkstheater@\textbf{Volkstheater}|pw} zum guten Teile Schuld daran, daß ich mich durch vier
                    Wochen in Karlsbad\oindex{Karlsbad@\textbf{Karlsbad}|pw} mit Felſenquelle und
                    Moorumſchlägen abgeben mußte. Der »Fremde\pwindex{Adam, Robert 20.04.1877 – 16.10.1961@\textsc{Adam, Robert} (20.04.1877 – 16.10.1961), \emph{Schriftsteller, Richter}!FremdeNone@\strich\emph{Der Fremde} {[}None{]}|pw}« hat
                    alle intereſſiert: den D\textsuperscript{r}{ }\textsc{Glücksmann}\pwindex{Gluecksmann, Heinrich 08.07.1864 – 01.03.1943@\textsc{Glücksmann, Heinrich} (08.07.1864 – 01.03.1943), \emph{Schriftsteller, Journalist, Dramaturg}|pw}, {\pb}den D\textsuperscript{r}{ }\textsc{Waniek}\pwindex{Waniek, Wolfgang 26.05.1878 – 27.03.1950@\textsc{Waniek, Wolfgang} (26.05.1878 – 27.03.1950), \emph{Dramaturg, Beamter}|pw}, den D\textsuperscript{r}{ }\textsc{Rosenthal}\pwindex{Rosenthal, Friedrich 20.07.1885 – 31.08.1942@\textsc{Rosenthal, Friedrich} (20.07.1885 – 31.08.1942), \emph{Regisseur, Dramaturg}|pw} und den Direktor\pwindex{Bernau, Alfred 06.03.1879 – 20.08.1950@\textsc{Bernau, Alfred} (06.03.1879 – 20.08.1950), \emph{Theaterleiter, Schauspieler}|pwv}, und
                    ich war ſchon faſt meiner Sache ſicher: bis der Direktor\pwindex{Bernau, Alfred 06.03.1879 – 20.08.1950@\textsc{Bernau, Alfred} (06.03.1879 – 20.08.1950), \emph{Theaterleiter, Schauspieler}|pwv} mir ſeinen Entſchluß bekanntgab, das Stück
                    doch nicht zu geben, da es keine ſich ſteigernde Handlung und daher keine
                    Ausſicht auf Erfolg habe. Seither war der »Fremde\pwindex{Adam, Robert 20.04.1877 – 16.10.1961@\textsc{Adam, Robert} (20.04.1877 – 16.10.1961), \emph{Schriftsteller, Richter}!FremdeNone@\strich\emph{Der Fremde} {[}None{]}|pw}« auch ſchon im Burgtheater\oindex{Burgtheater@\textbf{Burgtheater}|pw} und
                    wurde mit anerkennenswerter Eile und einem Formular retourniert. Von dem Welſ\oindex{Wels@\textbf{Wels}|pw}er Stück\pwindex{Adam, Robert 20.04.1877 – 16.10.1961@\textsc{Adam, Robert} (20.04.1877 – 16.10.1961), \emph{Schriftsteller, Richter}!Yppl. Idylle in fuenf AktenNone@\strich\emph{Yppl. Idylle in fünf Akten} {[}None{]}|pwv} wollte D\textsuperscript{r}{ }\textsc{Waniek}\pwindex{Waniek, Wolfgang 26.05.1878 – 27.03.1950@\textsc{Waniek, Wolfgang} (26.05.1878 – 27.03.1950), \emph{Dramaturg, Beamter}|pw} ohne Umarbeitung, die er am liebſten von einem Kompagnon – \textsc{Engel}\pwindex{Engel, Alexander 10.04.1868 – 17.11.1940@\textsc{Engel, Alexander} (10.04.1868 – 17.11.1940), \emph{Schriftsteller, Journalist}|pwu}{ }\strikeout{oder Landerberg\pwindex{Landerberg @\textsc{Landerberg}, \emph{Schriftsteller/Schriftstellerin}|pw}} oder ſonſt wem – vorgenommen wüßte, überhaupt nichts wiſſen; und zu einer
                    ſolchen Arbeit fehlte es mir bisher an Luſt und an Stimmung. –\pend
           \pstart
           Es iſt ſehr traurig, daß auch die Märchenkomödie\pwindex{Adam, Robert 20.04.1877 – 16.10.1961@\textsc{Adam, Robert} (20.04.1877 – 16.10.1961), \emph{Schriftsteller, Richter}!MaerchenkomoedieNone@\strich\emph{Märchenkomödie} {[}None{]}|pw}, die ich in Karlsbad\oindex{Karlsbad@\textbf{Karlsbad}|pw}
                    fleißig ſkizziert habe, keine Bühne finden wird, da der Stoff derart iſt, daß
                    überhaupt nur wenige begreifen werden, wie man zu ihm habe gelangen können: was
                    mich aber nicht abhalten ſoll, die Arbeit, die mich perſönlich intereſſiert, {\pb}zu Ende zu bringen, obwohl ſie mich,
                    der Anlage nach, viel Zeit und Mühe koſten wird. Ich hoffe, daß Sie,
                    hochverehrter Herr Doktor, dereinſt meine Stoffwahl nicht allzuſehr ſchelten
                    werden.\pend
           \pstart
           Indem ich Ihnen angenehmen Abſchluß des Sommeraufenthalts wünſche, bin ich mit
                    den herzlichſten Grüßen Ihr ſehr ergebener\pend
           \pstart \spacefill\mbox{D\textsuperscript{r}RAdam}\pend{}
         
         \endnumbering\mylabel{h}\end{ledgroupsized}  \newcommand{\dateiname}{L02325}\newcommand{\titel}{Robert Adam an Arthur Schnitzler, 19. 8. 1919}\newcommand{\editorInnen}{Martin Anton Müller und Gerd-Hermann Susen}%% latex-leseansicht-abspann.tex
%% Abspann für die Leseansicht.
%% Der Schalter \ifkorrekturansicht ist bereits durch den Vorspann gesetzt.

%% latex-abspann.tex
%% Gemeinsamer Abspann für Korrekturansicht und Leseansicht.
%% Setzt den Schalter \ifkorrekturansicht voraus (gesetzt in den
%% einbindenden Dateien latex-korrekturansicht-abspann.tex bzw.
%% latex-leseansicht-abspann.tex).
%% ---------------------------------------------------------------

\normalsize

% Das esempio-Environment wird nur in der Leseansicht benötigt
\ifkorrekturansicht\else
\newenvironment{esempio}[3]%
{
    \vspace{1.5ex}
    \rlap{\underline{#1}}
    \par
    \setlength{\parindent}{0cm}
    \nopagebreak
    \leftskip=#2cm
    \rightskip=#3cm
}
{
    \par
}
\fi

\doendnotes{C}
\bigskip
\vfill

\clearpage

\footnotesize

\ifkorrekturansicht
  \lohead{\textsc{register}}
\fi

% theindex-Environment neu definieren ohne reledmac
\makeatletter
\renewenvironment{theindex}{%
  \ifkorrekturansicht
    \section*{\indexname}%
  \else
    \subsubsection*{Index der erwähnten Entitäten}%
  \fi
  \setlength{\parindent}{0pt}%
  \setlength{\parskip}{0pt plus 0.3pt}%
  \let\item\@idxitem
}{%
  \ifkorrekturansicht\clearpage\fi
}
\makeatother

\IfFileExists{\jobname-pw.ind}{\input{\jobname-pw.ind}}{}

% Quellenangabe nur in der Leseansicht
\ifkorrekturansicht\else
% Fallback-Definitionen, falls die .tex-Datei \titel etc. nicht gesetzt hat
\providecommand{\titel}{}
\providecommand{\editorInnen}{}
\providecommand{\dateiname}{\jobname}

\vspace{3cm}

\vfill

\footnotesize
\textsc{Quelle}: \titel. Herausgegeben von {\editorInnen}. In: \emph{Arthur Schnitzler: Briefwechsel mit Autorinnen und Autoren}.
 Digitale Edition, https://schnitzler-briefe.acdh.oeaw.ac.at/{\dateiname}.html (Stand \today)
\fi

\end{document}


      