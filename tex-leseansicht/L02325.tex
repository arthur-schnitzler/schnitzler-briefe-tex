%% latex-korrekturansicht-vorspann.tex
%% Vorspann für die Korrekturansicht.
%% Lädt die gemeinsame Datei latex-vorspann.tex mit gesetztem Schalter.

\newif\ifkorrekturansicht
\korrekturansichttrue

\input{../tex-inputs/latex-vorspann}


\section[Robert Adam an Arthur Schnitzler, 19. 8. 1919]{L02325 Robert Adam an Arthur Schnitzler, 19. 8. 1919}
\nopagebreak\mylabel{L02325v}
\rehead{ }\normalsize\beginnumbering\briefempfaengerindex{Schnitzler, Arthur@\textsc{Schnitzler, Arthur}!zzzAdam, Robert@\emph{von Robert Adam}!1919-08-191@{19. 8. 1919}|(be}
\toendnotes[C]{\smallbreak\pagebreak[2]}\Standort{CUL, Schnitzler, B 1.}
\physDesc{Brief, 1 Blatt, 3 Seiten, 1971 Zeichen
\newline{}Handschrift: blaue Tinte, deutsche Kurrent
\newline{}Schnitzler: 1) mit Bleistift beschriftet: »\textsc{Adam}«  2) mit rotem Buntstift zwei Unterstreichungen
\newline{}Ordnung: mit Bleistift von unbekannter Hand nummeriert:
                                    »13« }\Standort{Wien, Österreichische Nationalbibliothek, Cod.ser. 52.268, 21 recto und 23.}
\physDesc{handschriftliche Abschrift2 Blätter, 2 Seiten, 1971 Zeichen
\newline{}Handschrift: schwarze Tinte, Gabelsberger Kurzschrift}\Standort{Wien, Österreichische Nationalbibliothek, Cod.ser. 52.268, 21 recto und 23.}
\physDesc{maschinenschriftliche Abschrift2 Blätter, 2 Seiten, 1971 Zeichen
\newline{}Schreibmaschine}\toendnotes[C]{\smallbreak}
\pstart
           \raggedleft{}{\pb}Wien\oindex{Wien@\textbf{Wien}, \emph{A.ADM2}|pw}, am 19. August 1919\pend
           
\pstart\center{}Hochverehrter Herr Doktor!\pend\vspace{0.5em}
\pstart
           Von Wegſcheid bei Maria Zell\oindex{Wegscheid@\textbf{Wegscheid}, \emph{P.PPL}|pw} zurückgekehrt, wo
               ich nach vollbrachter Karlsbad\oindex{Karlsbad@\textbf{Karlsbad}, \emph{P.PPLA}|pw}er Kur Frau\pwindex{Pollak, Maria 06.10.1889 – 27.03.1948@\textsc{Pollak, Maria} (06.10.1889 – 27.03.1948)|pwv} und Kind\pwindex{Patzner, Viktor Franz 13.09.1916 – 21.12.1982@\textsc{Patzner, Viktor Franz} (13.09.1916 – 21.12.1982), \emph{Rechtsanwalt/Rechtsanwältin}|pwv} aufſuchte, um ſie glücklich
               heimzubringen, finde ich Ihre Karte vor, die mir nach Karlsbad\oindex{Karlsbad@\textbf{Karlsbad}, \emph{P.PPLA}|pw} nachgeſchickt und von dort zurückgeſendet worden war. Ich freue
               mich darauf, Ihnen über meine Schickſale bei Ihrer Rückkehr mündlich berichten zu
               können; erfreulich sind ſie ſchließlich nicht. Wenn Ärger, wie die Ärzte behaupten,
               auf die Folgeerſcheinungen von Magengeſchwüren ungünſtig einwirkt, so trägt das Deutſche Volkstheater\oindex{Volkstheater@\textbf{Volkstheater}, \emph{Theater (K.THE)}|pw} zum guten Teile Schuld daran,
               daß ich mich durch vier Wochen in Karlsbad\oindex{Karlsbad@\textbf{Karlsbad}, \emph{P.PPLA}|pw} mit
               Felſenquelle und Moorumſchlägen abgeben mußte. Der »Fremde\pwindex{Fremde@\emph{Der Fremde}|pw}« hat alle intereſſiert: den D\textsuperscript{r}{ }\textsc{Glücksmann}\pwindex{Gluecksmann, Heinrich 08.07.1864 – 01.03.1943@\textsc{Glücksmann, Heinrich} (08.07.1864 – 01.03.1943), \emph{Schriftsteller/Schriftstellerin, Journalist/Journalistin, Dramaturg/Dramaturgin}|pw}, {\pb}den D\textsuperscript{r}{ }\textsc{Waniek}\pwindex{Waniek, Wolfgang 26.05.1878 – 27.03.1950@\textsc{Waniek, Wolfgang} (26.05.1878 – 27.03.1950), \emph{Dramaturg/Dramaturgin, Beamter/Beamte}|pw}, den D\textsuperscript{r}{ }\textsc{Rosenthal}\pwindex{Rosenthal, Friedrich 20.07.1885 – 31.08.1942@\textsc{Rosenthal, Friedrich} (20.07.1885 – 31.08.1942), \emph{Regisseur/Regisseurin, Dramaturg/Dramaturgin}|pw} und den Direktor\pwindex{Bernau, Alfred 06.03.1879 – 20.08.1950@\textsc{Bernau, Alfred} (06.03.1879 – 20.08.1950), \emph{Theaterleiter/Theaterleiterin, Schauspieler/Schauspielerin}|pwv}, und
               ich war ſchon faſt meiner Sache ſicher: bis der Direktor\pwindex{Bernau, Alfred 06.03.1879 – 20.08.1950@\textsc{Bernau, Alfred} (06.03.1879 – 20.08.1950), \emph{Theaterleiter/Theaterleiterin, Schauspieler/Schauspielerin}|pwv} mir ſeinen Entſchluß bekanntgab, das Stück doch
               nicht zu geben, da es keine ſich ſteigernde Handlung und daher keine Ausſicht auf
               Erfolg habe. Seither war der »Fremde\pwindex{Fremde@\emph{Der Fremde}|pw}« auch ſchon
               im Burgtheater\oindex{Burgtheater@\textbf{Burgtheater}, \emph{S.THTR}|pw} und wurde mit anerkennenswerter
               Eile und einem Formular retourniert. Von dem Welſ\oindex{Wels@\textbf{Wels}, \emph{P.PPLA2}|pw}er Stück\pwindex{Yppl. Idylle in fuenf Akten@\emph{Yppl. Idylle in fünf Akten}|pwv} wollte D\textsuperscript{r}{ }\textsc{Waniek}\pwindex{Waniek, Wolfgang 26.05.1878 – 27.03.1950@\textsc{Waniek, Wolfgang} (26.05.1878 – 27.03.1950), \emph{Dramaturg/Dramaturgin, Beamter/Beamte}|pw} ohne Umarbeitung, die er am liebſten von einem Kompagnon – \textsc{Engel}\pwindex{Engel, Alexander 10.04.1868 – 17.11.1940@\textsc{Engel, Alexander} (10.04.1868 – 17.11.1940), \emph{Schriftsteller/Schriftstellerin, Journalist/Journalistin}|pwu}{ }\strikeout{oder Landerberg\pwindex{Landerberg @\textsc{Landerberg}, \emph{Schriftsteller/Schriftstellerin}|pw}} oder ſonſt wem – vorgenommen wüßte, überhaupt nichts wiſſen; und zu einer
               ſolchen Arbeit fehlte es mir bisher an Luſt und an Stimmung. –\pend
           
\pstart
           Es iſt ſehr traurig, daß auch die Märchenkomödie\pwindex{Maerchenkomoedie@\emph{Märchenkomödie}|pw}, die ich in Karlsbad\oindex{Karlsbad@\textbf{Karlsbad}, \emph{P.PPLA}|pw} fleißig
               ſkizziert habe, keine Bühne finden wird, da der Stoff derart iſt, daß überhaupt nur
               wenige begreifen werden, wie man zu ihm habe gelangen können: was mich aber nicht
               abhalten ſoll, die Arbeit, die mich perſönlich intereſſiert, {\pb}zu Ende zu bringen, obwohl ſie mich, der
               Anlage nach, viel Zeit und Mühe koſten wird. Ich hoffe, daß Sie, hochverehrter Herr
               Doktor, dereinſt meine Stoffwahl nicht allzuſehr ſchelten werden.\pend
           
\pstart
           Indem ich Ihnen angenehmen Abſchluß des Sommeraufenthalts wünſche, bin ich mit den
               herzlichſten Grüßen Ihr ſehr ergebener\pend
           \pstart \spacefill\mbox{D\textsuperscript{r}RAdam}\pend{}\selectlanguage{ngerman}\endnumbering\briefempfaengerindex{Schnitzler, Arthur@\textsc{Schnitzler, Arthur}!zzzAdam, Robert@\emph{von Robert Adam}!1919-08-191@{19. 8. 1919}|)be}\mylabel{L02325h}  \normalsize

\doendnotes{C}
\bigskip
\vfill

\clearpage

\footnotesize

\lohead{\textsc{register}}

% Definiere theindex-Environment komplett neu ohne reledmac
\makeatletter
\renewenvironment{theindex}{%
  \section*{\indexname}%
  \setlength{\parindent}{0pt}%
  \setlength{\parskip}{0pt plus 0.3pt}%
  \let\item\@idxitem
}{%
  \clearpage
}
\makeatother

\IfFileExists{\jobname-pw.ind}{\input{\jobname-pw.ind}}{}

\end{document}

      