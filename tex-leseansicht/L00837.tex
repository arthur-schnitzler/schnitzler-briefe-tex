\input{../tex-inputs/latex-pdf-vorspann}
\begin{center}
            \textcolor{red}{ENTWURF. ENTZIFFERUNG NOCH NICHT KORREKTURGELESEN}
                      \end{center}
            
               \section[Arthur Schnitzler an Hugo von Hofmannsthal, 22. 8. 1898]{ Arthur Schnitzler an Hugo von Hofmannsthal,
                    22. 8. 1898}\nopagebreak\mylabel{v}\rehead{ }\begin{ledgroupsized}[t]{13cm}\normalsize\beginnumbering\briefempfaengerindex{Hofmannsthal, Hugo von@\textsc{Hofmannsthal, Hugo von}!zzzSchnitzler, Arthur@\emph{von Arthur Schnitzler}!1898-08-222@{22. 8. 1898}|(be} \toendnotes[C]{\smallbreak\pagebreak[2]} \Standort{FDH, Hs-30885,74.}
\physDesc{Postkarte
\newline{}Handschrift: Bleistift, deutsche Kurrent\newline{}Versand: 1) Stempel: »\nobreak{}\oindex{Luzern@\textbf{Luzern}|pwk}Luzern Bre. Aufg., 22 VIII 98, XI\nobreak{}«.  2) Stempel: »\nobreak{}\oindex{Lugano@\textbf{Lugano}|pwk}Lugano Lettere, 22 VIII 98, 6\nobreak{}«. \newline{}Ordnung: von Schnitzler mit Bleistift auch auf
                                    der Anschriftenseite datiert:
                                        »22/8 98«, mutmaßlich bei der Durchsicht der Briefe 1929 }\buchAbdrucke{\weitereDrucke{Hugo von Hofmannsthal, Arthur Schnitzler: \emph{Briefwechsel}. Hg. Therese Nickl und Heinrich Schnitzler. Frankfurt am Main: \emph{S. Fischer} 1964, S. 110.} }\pstart{}{\pb}Herrn Hugo von \textsc{Hofmannsthal}\pend{}\pstart{}\textsc{Lugano\oindex{Lugano@\textbf{Lugano}|pw}}\pend{}\pstart{}\textsc{Hotel du
                            parc\oindex{Hôtel du Parc@\textbf{Hôtel du Parc}|pw}}\pend{}{\bigskip}\pstart
           \raggedleft{}{\pb}Montag{ }Früh.\pend
           \pstart
           Mein Lieber Hugo, das Recepiſs hab ich noch am ſelben Abend
                        (Mittwoch?) an Sie abgeſandt, im Couvert des Hotels;
                    hoffentlich haben Sie’s ſchon. –\pend
           \pstart
           Habe eine wunderſchöne Reiſe gemacht, werd jetzt vielleicht in \textsc{Luzern}\oindex{Luzern@\textbf{Luzern}|pw} bleiben, von hier aus
                    Partien machen \substVorne{}\textsuperscript{oder}\substDazwischen{}und\substHinten{} arbeiten, womöglich. Möchte einen Theil der Heimreiſe \textsc{per} Rad machen.\pend
           \pstart
           Bitte Nachrichten hieher. Herzliche Grüße Ihr\spacefill\mbox{Arth}\pend
           \endnumbering\briefempfaengerindex{Hofmannsthal, Hugo von@\textsc{Hofmannsthal, Hugo von}!zzzSchnitzler, Arthur@\emph{von Arthur Schnitzler}!1898-08-222@{22. 8. 1898}|)be}\mylabel{h}\end{ledgroupsized}  \newcommand{\dateiname}{L00837}\newcommand{\titel}{Arthur Schnitzler an Hugo von Hofmannsthal, 22. 8. 1898}\newcommand{\editorInnen}{Martin Anton Müller und Gerd-Hermann Susen}\input{../tex-inputs/latex-pdf-abspann}
      