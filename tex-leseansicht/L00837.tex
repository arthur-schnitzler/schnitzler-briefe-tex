%% latex-korrekturansicht-vorspann.tex
%% Vorspann für die Korrekturansicht.
%% Lädt die gemeinsame Datei latex-vorspann.tex mit gesetztem Schalter.

\newif\ifkorrekturansicht
\korrekturansichttrue

\input{../tex-inputs/latex-vorspann}


\section[Arthur Schnitzler an Hugo von Hofmannsthal, 22. 8. 1898]{L00837 Arthur Schnitzler an Hugo von Hofmannsthal, 22. 8. 1898}
\nopagebreak\mylabel{L00837v}
\rehead{ }\normalsize\beginnumbering\briefempfaengerindex{Hofmannsthal, Hugo von@\textsc{Hofmannsthal, Hugo von}!zzzSchnitzler, Arthur@\emph{von Arthur Schnitzler}!1898-08-222@{22. 8. 1898}|(be}
\toendnotes[C]{\smallbreak\pagebreak[2]}\Standort{FDH, Hs-30885,74.}
\physDesc{Postkarte, 428 Zeichen
\newline{}Handschrift: Bleistift, deutsche Kurrent
\newline{}Versand: 1) Stempel: »\nobreak{}\oindex{Luzern@\textbf{Luzern}, \emph{P.PPLA}|pwk}Luzern Bre. Aufg., 22 VIII 98, XI\nobreak{}«.   2) Stempel: »\nobreak{}\oindex{Lugano@\textbf{Lugano}, \emph{P.PPLA2}|pwk}Lugano Lettere, 22 VIII 98, 6\nobreak{}«. 
\newline{}Ordnung: mit Bleistift von Schnitzler auch auf der Anschriftenseite
                                 datiert: »22/8 98«, mutmaßlich bei der Durchsicht der Briefe
                                    1929 }
\buchAbdrucke{\weitereDrucke{Hugo von Hofmannsthal, Arthur Schnitzler: \emph{Briefwechsel}. Frankfurt am Main: \emph{S. Fischer} 1964, S. 110.} }\pstart{}{\pb}Herrn Hugo von \textsc{Hofmannsthal}\pend{}\pstart{}\textsc{Lugano\oindex{Lugano@\textbf{Lugano}, \emph{P.PPLA2}|pw}}\pend{}\pstart{}\textsc{Hotel du parc\oindex{Hôtel du Parc@\textbf{Hôtel du Parc}, \emph{Hotel (K.HTL)}|pw}}\pend{}{\bigskip}\vspace{1em}
\pstart
           \raggedleft{}{\pb}Montag{ }Früh.\pend
           \vspace{0.5em}
\pstart
           Mein Lieber Hugo, das Recepiſs hab ich noch am ſelben Abend
                  (Mittwoch?) an Sie abgeſandt, im Couvert des Hotels; hoffentlich
               haben Sie’s ſchon. –\pend
           
\pstart
           Habe eine wunderſchöne Reiſe gemacht, werd jetzt vielleicht in \textsc{Luzern}\oindex{Luzern@\textbf{Luzern}, \emph{P.PPLA}|pw} bleiben, von hier aus Partien machen \substVorne{}\textsuperscript{oder}\substDazwischen{}und\substHinten{} arbeiten, womöglich. Möchte einen Theil der Heimreiſe \textsc{per} Rad machen.\pend
           
\pstart
           Bitte Nachrichten hieher. Herzliche Grüße Ihr\spacefill\mbox{Arth}\pend
           \selectlanguage{ngerman}\endnumbering\briefempfaengerindex{Hofmannsthal, Hugo von@\textsc{Hofmannsthal, Hugo von}!zzzSchnitzler, Arthur@\emph{von Arthur Schnitzler}!1898-08-222@{22. 8. 1898}|)be}\mylabel{L00837h}  \normalsize

\doendnotes{C}
\bigskip
\vfill

\clearpage

\footnotesize

\lohead{\textsc{register}}

% Definiere theindex-Environment komplett neu ohne reledmac
\makeatletter
\renewenvironment{theindex}{%
  \section*{\indexname}%
  \setlength{\parindent}{0pt}%
  \setlength{\parskip}{0pt plus 0.3pt}%
  \let\item\@idxitem
}{%
  \clearpage
}
\makeatother

\IfFileExists{\jobname-pw.ind}{\input{\jobname-pw.ind}}{}

\end{document}

      