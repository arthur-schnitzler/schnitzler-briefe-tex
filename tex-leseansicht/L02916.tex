%% latex-leseansicht-vorspann.tex
%% Vorspann für die Leseansicht.
%% Lädt die gemeinsame Datei latex-vorspann.tex mit nicht gesetztem Schalter.

\newif\ifkorrekturansicht
\korrekturansichtfalse

\input{../tex-inputs/latex-vorspann}


         
         \renewcommand{\erwaehntePersonen}{Personen:  ?? [in Berlin lebender Wiener Journalist], Hermann Bahr, Richard Beer-Hofmann, Ludwig Fulda, Julius von Gans-Ludassy, Carl Karlweis, Helene Keßler, Isidor Landau, Rudolf Lothar, Fritz Mauthner, Erich von Monbart, Felix Salten, Ottilie Salten}
         \renewcommand{\erwaehnteInstitutionen}{Institutionen: Berliner Börsen-Courier, Deutsches Theater Berlin, Freie Bühne, Neue Freie Presse, Volkstheater}
         \renewcommand{\erwaehnteOrte}{Orte: Berlin, Dessauer Straße, Lessing-Theater, Puchberg am Schneeberg, Volkstheater, Wien}
         \renewcommand{\erwaehnteWerke}{Werke: Berliner Tageblatt, Der Star. Ein Wiener Stück in vier Akten, Der letzte Knopf, Die Familie von Barchwitz. Roman, Kleine Chronik. [Theater.] [Onkel Toni], König Harlekin. Maskenspiel in vier Aufzügen, Neue Freie Presse, Neues Wiener Tagblatt, Onkel Toni. Eine Komödie aus der Gesellschaft in vier Aufzügen, Theater, Kunst und Literatur [König Harlekin], Theater- und Kunstnachrichten [König Harlekin], Wiener Deutsches Volkstheater. (Gastspiel im Deutschen Theater.) »König Harlekin«, ein Maskenspiel in vier Aufzügen von Rudolf Lothar}
               \section[ Paul Goldmann an Arthur Schnitzler, 27. 5. {[}1900{]}]{ Paul Goldmann an Arthur Schnitzler, 27. 5. {[}1900{]}}\nopagebreak\mylabel{v}\rehead{ }\begin{ledgroupsized}[t]{13cm}\normalsize\beginnumbering \toendnotes[C]{\smallbreak\pagebreak[2]} \Standort{DLA, A:Schnitzler, HS.NZ85.1.3170.}
\physDesc{Brief, 1 Blatt, 4 Seiten, 1915 Zeichen
\newline{}Handschrift: blaue Tinte, deutsche Kurrent
\newline{}Schnitzler: 1) mit Bleistift das Jahr »900« vermerkt  2) mit rotem Buntstift sechs Unterstreichungen}\toendnotes[C]{\smallbreak}\pstart
           \noindent{}{\pb}\textcolor{gray}{\textbf{DESSAUERSTRASSE 19}}\oindex{Dessauer Strasse@\textbf{Dessauer Straße}|pw}\hfill Berlin\oindex{Berlin@\textbf{Berlin}|pw}, 27. Mai.\pend
           \pstart
           \centering{}Mein lieber Freund,\pend
           \pstart
           \noindent{}Du biſt wieder einmal ganz verſtummt. Von Woche zu Woche warte ich auf eine
               Nachricht, aber vergebens.\pend
           \pstart
           Wann alſo wirſt Du anfangen zu \label{K_L02916-1v}\edtext{reiſen}{\lemma{\textnormal{\emph{reiſen}}}\Cendnote{\textnormal{Schnitzler\pwindex{Schnitzler, Arthur 15.05.1862 – 21.10.1931@\textsc{Schnitzler, Arthur} (15.05.1862 – 21.10.1931), \emph{Schriftsteller, Mediziner}|pwk} war bereits seit 24. 5. 1900 in Puchberg am Schneeberg\oindex{Puchberg am Schneeberg@\textbf{Puchberg am Schneeberg}|pwk}, wo er bis zum 27. 5. 1900 blieb und
                  Zeit mit Felix Salten\pwindex{Salten, Felix 06.09.1869 – 08.10.1945@\textsc{Salten, Felix} (06.09.1869 – 08.10.1945), \emph{Schriftsteller, Journalist}|pwk} und Ottilie Metzl\pwindex{Salten, Ottilie 07.03.1868 – 22.06.1942@\textsc{Salten, Ottilie} (07.03.1868 – 22.06.1942), \emph{Schauspielerin}|pwk} (später Salten\pwindex{Salten, Ottilie 07.03.1868 – 22.06.1942@\textsc{Salten, Ottilie} (07.03.1868 – 22.06.1942), \emph{Schauspielerin}|pwkv}) verbrachte.}}}\label{K_L02916-1h}? Und wohin? Intereſſant wäre es
               auch, die Frage zu ſtellen: mit wem? Aber ich ſtelle ſie lieber nicht.\pend
           \pstart
           \textsc{Rudolf Lothar\pwindex{Lothar, Rudolf 23.2.1865 – 2.10.1943@\textsc{Lothar, Rudolf} (23.2.1865 – 2.10.1943), \emph{Schriftsteller, Journalist, Theaterdirektor}|pw}} hat ſich hier hübſch benommen. Er hat ſich einen in Berlin\oindex{Berlin@\textbf{Berlin}|pw} lebenden \label{K_L02916-2v}\edtext{Wien\oindex{Wien@\textbf{Wien}|pw}er Journaliſten\pwindex{?? [in Berlin lebender Wiener Journalist] @\textsc{?? [in Berlin lebender Wiener Journalist]}|pwv} engagirt}{\lemma{\textnormal{\emph{Wiener … engagirt}}}\Cendnote{\textnormal{nicht identifiziert; siehe Paul Goldmann an Arthur Schnitzler, 10. 5. [1900]}}}\label{K_L02916-2h}, der \strikeout{b} von Berlin\oindex{Berlin@\textbf{Berlin}|pw}er Redaktionen wegen »Inkorrektheiten« entlaſſen worden iſt, und hat
               von dieſem am Abend ſeiner \textsc{Première\pwindex{Lothar, Rudolf 23.2.1865 – 2.10.1943@\textsc{Lothar, Rudolf} (23.2.1865 – 2.10.1943), \emph{Schriftsteller, Journalist, Theaterdirektor}!Koenig Harlekin. Maskenspiel in vier Aufzuegen1900@\strich\emph{König Harlekin. Maskenspiel in vier Aufzügen} {[}1900{]}|pwv}}{ }ein gefälſchtes \label{K_L02916-3v}\edtext{Telegramm\pwindex{Theater, Kunst und Literatur [Koenig Harlekin]1900-05-20@\emph{Theater, Kunst und Literatur [König Harlekin]} {[}1900-05-20{]}|pwv}}{\lemma{\textnormal{\emph{Telegramm}}}\Cendnote{\textnormal{Abgedruckt zum Beispiel im \emph{Neuen Wiener Tagblatt}\pwindex{?? Werk@Nicht ermittelte Verfasserinnen und Verfasser!Neues Wiener Tagblatt1867 – 1945@\emph{Neues Wiener Tagblatt} {[}1867 – 1945{]}|pwk}: [O. V.]\pwindex{?? [in Berlin lebender Wiener Journalist] @\textsc{?? [in Berlin lebender Wiener Journalist]}|pwkv}: \emph{Theater, Kunst und Literatur}\pwindex{Theater, Kunst und Literatur [Koenig Harlekin]1900-05-20@\emph{Theater, Kunst und Literatur [König Harlekin]} {[}1900-05-20{]}|pwk}. In: \emph{Neues Wiener Tagblatt}\pwindex{?? Werk@Nicht ermittelte Verfasserinnen und Verfasser!Neues Wiener Tagblatt1867 – 1945@\emph{Neues Wiener Tagblatt} {[}1867 – 1945{]}|pwk}, Jg. 34, Nr. 137,
                        20. 5. 1900, Tages-Ausgabe,
                  S. 8.}}}\label{K_L02916-3h} an alle Wien\oindex{Wien@\textbf{Wien}|pw}er Blätter
               ſenden laſſen. Für die N. Fr. Pr.\orgindex{Neue Freie Presse@Neue Freie Presse|pw} hat \textsc{Landau\pwindex{Landau, Isidor 1850 – 1944@\textsc{Landau, Isidor} (1850 – 1944), \emph{Journalist/Journalistin}|pw}} vom Börſencourier\orgindex{Berliner Boersen-Courier@Berliner Börsen-Courier|pw}{ }{\pb}\label{K_L02916-4v}\edtext{telegraphirt\pwindex{Theater- und Kunstnachrichten [Koenig Harlekin]1900-05-20@\emph{Theater- und Kunstnachrichten [König Harlekin]} {[}1900-05-20{]}|pwv}}{\lemma{\textnormal{\emph{telegraphirt}}}\Cendnote{\textnormal{[O. V.] [ = Isidor Landau\pwindex{Landau, Isidor 1850 – 1944@\textsc{Landau, Isidor} (1850 – 1944), \emph{Journalist/Journalistin}|pwk}]: \emph{Theater- und Kunstnachrichten}\pwindex{Theater- und Kunstnachrichten [Koenig Harlekin]1900-05-20@\emph{Theater- und Kunstnachrichten [König Harlekin]} {[}1900-05-20{]}|pwk}. In: \emph{Neue Freie Presse}\pwindex{Neue Freie Presse1864 – 1939@\emph{Neue Freie Presse} {[}1864 – 1939{]}|pwk}, Nr. 12837, 20. 5. 1900, Morgenblatt, S. 9.}}}\label{K_L02916-4h}, der
               bekanntlich die Spezialität hat, Alles zu loben. Aber ſelbſt deſſen Telegramm\pwindex{Theater- und Kunstnachrichten [Koenig Harlekin]1900-05-20@\emph{Theater- und Kunstnachrichten [König Harlekin]} {[}1900-05-20{]}|pwv} genügte noch nicht, und ſo hat
               man in der Redaktion\orgindex{Neue Freie Presse@Neue Freie Presse|pwv} dieſe
               Fälſchung durch Einfügung einiger lobender Sätze noch \strikeout{\textcolor{gray}{f}} weiter gefälſcht. Dem \textsc{Fritz Mauthner\pwindex{Mauthner, Fritz 1849-11-20 – 1923-06-29@\textsc{Mauthner, Fritz} (1849-11-20 – 1923-06-29), \emph{Schriftsteller, Journalist, Philosoph}|pw}} hat ſich \textsc{Lothar\pwindex{Lothar, Rudolf 23.2.1865 – 2.10.1943@\textsc{Lothar, Rudolf} (23.2.1865 – 2.10.1943), \emph{Schriftsteller, Journalist, Theaterdirektor}|pw}} ſeit dem Tage ſeiner Ankunft an die Rockſchöße gehangen, er hat ihn umwedelt
               und umſchmeichelt. Die Folge davon war, daß \textsc{Mauthner\pwindex{Mauthner, Fritz 1849-11-20 – 1923-06-29@\textsc{Mauthner, Fritz} (1849-11-20 – 1923-06-29), \emph{Schriftsteller, Journalist, Philosoph}|pw}} in ſeinem \label{K_L02916-5v}\edtext{Feuilleton\pwindex{Wiener Deutsches Volkstheater. (Gastspiel im Deutschen Theater.) »Koenig
                  Harlekin«, ein Maskenspiel in vier Aufzuegen von Rudolf Lothar1900-05-20@\emph{Wiener Deutsches Volkstheater. (Gastspiel im Deutschen Theater.) »König Harlekin«, ein Maskenspiel in vier Aufzügen von Rudolf Lothar} {[}1900-05-20{]}|pwv}}{\lemma{\textnormal{\emph{Feuilleton}}}\Cendnote{\textnormal{F. M.\pwindex{Mauthner, Fritz 1849-11-20 – 1923-06-29@\textsc{Mauthner, Fritz} (1849-11-20 – 1923-06-29), \emph{Schriftsteller, Journalist, Philosoph}|pwkv} [ = Fritz Mauthner\pwindex{Mauthner, Fritz 1849-11-20 – 1923-06-29@\textsc{Mauthner, Fritz} (1849-11-20 – 1923-06-29), \emph{Schriftsteller, Journalist, Philosoph}|pwk}]: \emph{Wiener Deutsches Volkstheater. (Gastspiel im Deutschen
                        Theater.) »König Harlekin«, ein Maskenspiel in vier Aufzügen von Rudolf
                        Lothar}\pwindex{Wiener Deutsches Volkstheater. (Gastspiel im Deutschen Theater.) »Koenig
                  Harlekin«, ein Maskenspiel in vier Aufzuegen von Rudolf Lothar1900-05-20@\emph{Wiener Deutsches Volkstheater. (Gastspiel im Deutschen Theater.) »König Harlekin«, ein Maskenspiel in vier Aufzügen von Rudolf Lothar} {[}1900-05-20{]}|pwk}. In: \emph{Berliner Tageblatt}\pwindex{?? Werk@Nicht ermittelte Verfasserinnen und Verfasser!Berliner Tageblatt1872 – 1939@\emph{Berliner Tageblatt} {[}1872 – 1939{]}|pwk},
                     Jg. 29, Nr. 254, 20. 5. 1900,
                  S. [3].}}}\label{K_L02916-5h} vom »Dichter{ }\textsc{Lothar\pwindex{Lothar, Rudolf 23.2.1865 – 2.10.1943@\textsc{Lothar, Rudolf} (23.2.1865 – 2.10.1943), \emph{Schriftsteller, Journalist, Theaterdirektor}|pw}}\pwindex{Wiener Deutsches Volkstheater. (Gastspiel im Deutschen Theater.) »Koenig
                  Harlekin«, ein Maskenspiel in vier Aufzuegen von Rudolf Lothar1900-05-20@\emph{Wiener Deutsches Volkstheater. (Gastspiel im Deutschen Theater.) »König Harlekin«, ein Maskenspiel in vier Aufzügen von Rudolf Lothar} {[}1900-05-20{]}|pwv}« ſprach. Damit iſt \textsc{Mauthner\pwindex{Mauthner, Fritz 1849-11-20 – 1923-06-29@\textsc{Mauthner, Fritz} (1849-11-20 – 1923-06-29), \emph{Schriftsteller, Journalist, Philosoph}|pw}} als Kritiker allerdings für mich gerichtet.\pend
           \pstart
           Als \textsc{Karlweiss\pwindex{Karlweis, Carl 23.11.1850 – 27.10.1901@\textsc{Karlweis, Carl} (23.11.1850 – 27.10.1901), \emph{Schriftsteller}|pw}}’ »Onkel Toni\pwindex{Karlweis, Carl 23.11.1850 – 27.10.1901@\textsc{Karlweis, Carl} (23.11.1850 – 27.10.1901), \emph{Schriftsteller}!Onkel Toni. Eine Komoedie aus der Gesellschaft in vier Aufzuegen1899@\strich\emph{Onkel Toni. Eine Komödie aus der Gesellschaft in vier Aufzügen} {[}1899{]}|pw}« hier\oindex{Berlin@\textbf{Berlin}|pwv}{ }\label{K_L02916-6v}\edtext{aufgeführt}{\lemma{\textnormal{\emph{aufgeführt}}}\Cendnote{\textnormal{Goldmann\pwindex{Goldmann, Paul 31.01.1865 – 25.09.1935@\textsc{Goldmann, Paul} (31.01.1865 – 25.09.1935), \emph{Schriftsteller, Journalist}|pwk} bezog sich auf das Gastspiel des
                     \emph{Volkstheater}\orgindex{Volkstheater@Volkstheater|pwk}s von \emph{Onkel Toni}\pwindex{Karlweis, Carl 23.11.1850 – 27.10.1901@\textsc{Karlweis, Carl} (23.11.1850 – 27.10.1901), \emph{Schriftsteller}!Onkel Toni. Eine Komoedie aus der Gesellschaft in vier Aufzuegen1899@\strich\emph{Onkel Toni. Eine Komödie aus der Gesellschaft in vier Aufzügen} {[}1899{]}|pwk} am 11. 5. 1900.}}}\label{K_L02916-6h} wurde, telegraphirte\pwindex{Kleine Chronik. [Theater.] [Onkel Toni]1900-05-12@\emph{Kleine Chronik. [Theater.] [Onkel Toni]} {[}1900-05-12{]}|pwv} ich {\pb}ganz
               ſanft: Die vortreffliche Aufführung habe über die ſchwachen Stellen des Stück\pwindex{Karlweis, Carl 23.11.1850 – 27.10.1901@\textsc{Karlweis, Carl} (23.11.1850 – 27.10.1901), \emph{Schriftsteller}!Onkel Toni. Eine Komoedie aus der Gesellschaft in vier Aufzuegen1899@\strich\emph{Onkel Toni. Eine Komödie aus der Gesellschaft in vier Aufzügen} {[}1899{]}|pwv}es hinweggeholfen. Der
               Satz \strikeout{wurde} wurde \label{K_L02916-7v}\edtext{geſtrichen}{\lemma{\textnormal{\emph{geſtrichen}}}\Cendnote{\textnormal{[Paul Goldmann\pwindex{Goldmann, Paul 31.01.1865 – 25.09.1935@\textsc{Goldmann, Paul} (31.01.1865 – 25.09.1935), \emph{Schriftsteller, Journalist}|pwk}]: \emph{Kleine Chronik. [Theater.]}\pwindex{Kleine Chronik. [Theater.] [Onkel Toni]1900-05-12@\emph{Kleine Chronik. [Theater.] [Onkel Toni]} {[}1900-05-12{]}|pwk}. In: \emph{Neue Freie Presse}\pwindex{Neue Freie Presse1864 – 1939@\emph{Neue Freie Presse} {[}1864 – 1939{]}|pwk}, Nr. 12829, 12. 5. 1900, Abendblatt, S. 1.}}}\label{K_L02916-7h}. Ein
                  Stück\pwindex{Karlweis, Carl 23.11.1850 – 27.10.1901@\textsc{Karlweis, Carl} (23.11.1850 – 27.10.1901), \emph{Schriftsteller}!Onkel Toni. Eine Komoedie aus der Gesellschaft in vier Aufzuegen1899@\strich\emph{Onkel Toni. Eine Komödie aus der Gesellschaft in vier Aufzügen} {[}1899{]}|pw} von \textsc{Karlweiss\pwindex{Karlweis, Carl 23.11.1850 – 27.10.1901@\textsc{Karlweis, Carl} (23.11.1850 – 27.10.1901), \emph{Schriftsteller}|pw}} darf nicht einmal ſchwache Stellen haben!\pend
           \pstart
           Der »\textsc{Star\pwindex{Bahr, Hermann 19.07.1863 – 15.01.1934@\textsc{Bahr, Hermann} (19.07.1863 – 15.01.1934), \emph{Schriftsteller, Kritiker}!Star. Ein Wiener Stueck in vier Akten10. 12. 1898@\strich\emph{Der Star. Ein Wiener Stück in vier Akten} {[}10. 12. 1898{]}|pw}\pwindex{Bahr, Hermann 19.07.1863 – 15.01.1934@\textsc{Bahr, Hermann} (19.07.1863 – 15.01.1934), \emph{Schriftsteller, Kritiker}!Star. Ein Wiener Stueck in vier Akten10. 12. 1898@\strich\emph{Der Star. Ein Wiener Stück in vier Akten} {[}10. 12. 1898{]}|pw}}« von \textsc{Bahr\pwindex{Bahr, Hermann 19.07.1863 – 15.01.1934@\textsc{Bahr, Hermann} (19.07.1863 – 15.01.1934), \emph{Schriftsteller, Kritiker}|pw}} hat mir hingegen \label{K_L02916-8v}\edtext{gefallen}{\lemma{\textnormal{\emph{gefallen}}}\Cendnote{\textnormal{Das Stück\pwindex{Bahr, Hermann 19.07.1863 – 15.01.1934@\textsc{Bahr, Hermann} (19.07.1863 – 15.01.1934), \emph{Schriftsteller, Kritiker}!Star. Ein Wiener Stueck in vier Akten10. 12. 1898@\strich\emph{Der Star. Ein Wiener Stück in vier Akten} {[}10. 12. 1898{]}|pwkv}\pwindex{Bahr, Hermann 19.07.1863 – 15.01.1934@\textsc{Bahr, Hermann} (19.07.1863 – 15.01.1934), \emph{Schriftsteller, Kritiker}!Star. Ein Wiener Stueck in vier Akten10. 12. 1898@\strich\emph{Der Star. Ein Wiener Stück in vier Akten} {[}10. 12. 1898{]}|pwkv} feierte am 25. 5. 1900 am Berlin\oindex{Berlin@\textbf{Berlin}|pwk}er Lessing-Theater\oindex{Lessing-Theater@\textbf{Lessing-Theater}|pwk} Premiere.}}}\label{K_L02916-8h}. Dieſer widerliche Burſch\pwindex{Bahr, Hermann 19.07.1863 – 15.01.1934@\textsc{Bahr, Hermann} (19.07.1863 – 15.01.1934), \emph{Schriftsteller, Kritiker}|pwv} hat doch – leider! – Humor und
               Talent.\pend
           \pstart
           Bitte, \label{K_L02916-9v}\edtext{lies’}{\lemma{\textnormal{\emph{lies’}}}\Cendnote{\textnormal{Schnitzler\pwindex{Schnitzler, Arthur 15.05.1862 – 21.10.1931@\textsc{Schnitzler, Arthur} (15.05.1862 – 21.10.1931), \emph{Schriftsteller, Mediziner}|pwk} las den Roman\pwindex{Familie von Barchwitz. Roman1899-03-03@\emph{Die Familie von Barchwitz. Roman} {[}1899-03-03{]}|pwkv} (vgl. A. S.: \emph{Lektüren}, Deutschsprachige-Literatur).}}}\label{K_L02916-9h}, wenn Du es noch
               nicht kennſt, »Die Familie von \textsc{Barchwitz}\pwindex{Familie von Barchwitz. Roman1899-03-03@\emph{Die Familie von Barchwitz. Roman} {[}1899-03-03{]}|pw}« von \textsc{Hans von Kahlenberg\pwindex{Kessler, Helene 23.02.1870 – 08.08.1957@\textsc{Keßler, Helene} (23.02.1870 – 08.08.1957), \emph{Schriftstellerin}|pwv}}. Seit Langem hat mich kein Roman ſo intereſſirt. \strikeout{V\textcolor{gray}{erg}}{ }Verfaſſerin\pwindex{Kessler, Helene 23.02.1870 – 08.08.1957@\textsc{Keßler, Helene} (23.02.1870 – 08.08.1957), \emph{Schriftstellerin}|pwv} iſt ein nicht
               mehr {\pb}ganz \strikeout{\textcolor{gray}{hu}} junges, aber \strikeout{\textcolor{gray}{r}} noch \strikeout{recht} recht hübſches Mädchen\pwindex{Kessler, Helene 23.02.1870 – 08.08.1957@\textsc{Keßler, Helene} (23.02.1870 – 08.08.1957), \emph{Schriftstellerin}|pwv}, ein Fräulein von \textsc{Montbart\pwindex{Kessler, Helene 23.02.1870 – 08.08.1957@\textsc{Keßler, Helene} (23.02.1870 – 08.08.1957), \emph{Schriftstellerin}|pwv}}, Offizier\pwindex{Monbart, Erich von 1836 – 1907@\textsc{Monbart, Erich von} (1836 – 1907), \emph{Offizier, Oberleutnant}|pwv}s-Tochter.\pend
           \pstart
           Was macht \textsc{Richard\pwindex{Beer-Hofmann, Richard 1866-07-11 – 1945-09-26@\textsc{Beer-Hofmann, Richard} (1866-07-11 – 1945-09-26), \emph{Schriftsteller}|pw}}?\pend
           \pstart
           Bitte, ſchreib’ mir bald!\pend
           \pstart
           Viele treue Grüße! {\\[\baselineskip]}Dein {\\[\baselineskip]}\spacefill\mbox{Paul Goldmann}\pend
           \leftskip=0em{}\pstart
           \noindent{}Auch \textsc{Ludassy\pwindex{Gans-Ludassy, Julius von 13.04.1858 – 30.09.1922@\textsc{Gans-Ludassy, Julius von} (13.04.1858 – 30.09.1922), \emph{Schriftsteller, Journalist, Herausgeber}|pw}} benimmt ſich abſcheulich hier und macht ſich aus dem \label{K_L02916-10v}\edtext{Verbot ſeines ſchlechten Stück\pwindex{Gans-Ludassy, Julius von 13.04.1858 – 30.09.1922@\textsc{Gans-Ludassy, Julius von} (13.04.1858 – 30.09.1922), \emph{Schriftsteller, Journalist, Herausgeber}!letzte Knopf1900@\strich\emph{Der letzte Knopf} {[}1900{]}|pwv}es}{\lemma{\textnormal{\emph{Verbot … Stückes}}}\Cendnote{\textnormal{Julius von Gans-Ludassy\pwindex{Gans-Ludassy, Julius von 13.04.1858 – 30.09.1922@\textsc{Gans-Ludassy, Julius von} (13.04.1858 – 30.09.1922), \emph{Schriftsteller, Journalist, Herausgeber}|pwk}s \emph{Der letzte Knopf}\pwindex{Gans-Ludassy, Julius von 13.04.1858 – 30.09.1922@\textsc{Gans-Ludassy, Julius von} (13.04.1858 – 30.09.1922), \emph{Schriftsteller, Journalist, Herausgeber}!letzte Knopf1900@\strich\emph{Der letzte Knopf} {[}1900{]}|pwk} war am 8. 4. 1900 am Volkstheater\oindex{Volkstheater@\textbf{Volkstheater}|pwk}
                     uraufgeführt worden. Das Stück\pwindex{Gans-Ludassy, Julius von 13.04.1858 – 30.09.1922@\textsc{Gans-Ludassy, Julius von} (13.04.1858 – 30.09.1922), \emph{Schriftsteller, Journalist, Herausgeber}!letzte Knopf1900@\strich\emph{Der letzte Knopf} {[}1900{]}|pwkv}, das für einen Skandal sorgte, sollte auch in Berlin\oindex{Berlin@\textbf{Berlin}|pwk} aufgeführt werden. Ludwig Fulda\pwindex{Fulda, Ludwig 15.07.1862 – 30.03.1939@\textsc{Fulda, Ludwig} (15.07.1862 – 30.03.1939), \emph{Schriftsteller, Übersetzer}|pwk}, der als Präsident der \emph{Freien Bühne}\orgindex{Freie Buehne@Freie Bühne|pwk} das von der Zensur verbotene Stück\pwindex{Gans-Ludassy, Julius von 13.04.1858 – 30.09.1922@\textsc{Gans-Ludassy, Julius von} (13.04.1858 – 30.09.1922), \emph{Schriftsteller, Journalist, Herausgeber}!letzte Knopf1900@\strich\emph{Der letzte Knopf} {[}1900{]}|pwkv} annahm, musste von
                     seiner Funktion zurücktreten. Schließlich wurde es vor einem geladenen Publikum
                     am 30. 5. 1900 bei einer Matinée des \emph{Deutschen Theaters}\orgindex{Deutsches Theater Berlin@Deutsches Theater Berlin|pwk} aufgeführt.}}}\label{K_L02916-10h} eine
                  unerträgliche Reklame.\pend
           
         
         \endnumbering\mylabel{h}\end{ledgroupsized}  \newcommand{\dateiname}{L02916}\newcommand{\titel}{Paul Goldmann an Arthur Schnitzler, 27. 5. [1900]}\newcommand{\editorInnen}{Martin Anton Müller und Laura Untner}%% latex-leseansicht-abspann.tex
%% Abspann für die Leseansicht.
%% Der Schalter \ifkorrekturansicht ist bereits durch den Vorspann gesetzt.

%% latex-abspann.tex
%% Gemeinsamer Abspann für Korrekturansicht und Leseansicht.
%% Setzt den Schalter \ifkorrekturansicht voraus (gesetzt in den
%% einbindenden Dateien latex-korrekturansicht-abspann.tex bzw.
%% latex-leseansicht-abspann.tex).
%% ---------------------------------------------------------------

\normalsize

% Das esempio-Environment wird nur in der Leseansicht benötigt
\ifkorrekturansicht\else
\newenvironment{esempio}[3]%
{
    \vspace{1.5ex}
    \rlap{\underline{#1}}
    \par
    \setlength{\parindent}{0cm}
    \nopagebreak
    \leftskip=#2cm
    \rightskip=#3cm
}
{
    \par
}
\fi

\doendnotes{C}
\bigskip
\vfill

\clearpage

\footnotesize

\ifkorrekturansicht
  \lohead{\textsc{register}}
\fi

% theindex-Environment neu definieren ohne reledmac
\makeatletter
\renewenvironment{theindex}{%
  \ifkorrekturansicht
    \section*{\indexname}%
  \else
    \subsubsection*{Index der erwähnten Entitäten}%
  \fi
  \setlength{\parindent}{0pt}%
  \setlength{\parskip}{0pt plus 0.3pt}%
  \let\item\@idxitem
}{%
  \ifkorrekturansicht\clearpage\fi
}
\makeatother

\IfFileExists{\jobname-pw.ind}{\input{\jobname-pw.ind}}{}

% Quellenangabe nur in der Leseansicht
\ifkorrekturansicht\else
% Fallback-Definitionen, falls die .tex-Datei \titel etc. nicht gesetzt hat
\providecommand{\titel}{}
\providecommand{\editorInnen}{}
\providecommand{\dateiname}{\jobname}

\vspace{3cm}

\vfill

\footnotesize
\textsc{Quelle}: \titel. Herausgegeben von {\editorInnen}. In: \emph{Arthur Schnitzler: Briefwechsel mit Autorinnen und Autoren}.
 Digitale Edition, https://schnitzler-briefe.acdh.oeaw.ac.at/{\dateiname}.html (Stand \today)
\fi

\end{document}


      