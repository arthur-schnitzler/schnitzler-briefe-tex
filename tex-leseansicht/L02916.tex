%% latex-leseansicht-vorspann.tex
%% Vorspann für die Leseansicht.
%% Lädt die gemeinsame Datei latex-vorspann.tex mit nicht gesetztem Schalter.

\newif\ifkorrekturansicht
\korrekturansichtfalse

\input{../tex-inputs/latex-vorspann}


\section[ Paul Goldmann an Arthur Schnitzler, 27. 5. {[}1900{]}]{L02916 Paul Goldmann an Arthur Schnitzler,  27. 5. [1900]}
\nopagebreak\mylabel{L02916v}
\rehead{ }\normalsize\beginnumbering\briefempfaengerindex{Schnitzler, Arthur@\textsc{Schnitzler, Arthur}!zzzGoldmann, Paul@\emph{von Paul Goldmann}!1900-05-271@{27. 5. [1900]}|(be}
\toendnotes[C]{\smallbreak\pagebreak[2]}
\correspDesc{Versand  durch Paul Goldmann am 27. 5. [1900] in Berlin
\newline{}Erhalt  durch Arthur Schnitzler im Zeitraum [28. 5. 1900
                  – 29. 5. 1900?] in Wien}\toendnotes[C]{\smallbreak}
\Standort{DLA, A:Schnitzler, HS.NZ85.1.3170.}
\physDesc{Brief, 1 Blatt, 4 Seiten, 1915 Zeichen
\newline{}Handschrift: blaue Tinte, deutsche Kurrent
\newline{}Schnitzler: 1) mit Bleistift das Jahr »900« vermerkt  2) mit rotem Buntstift sechs Unterstreichungen}\toendnotes[C]{\smallbreak}
\pstart
           \noindent{}
\pstart
           {\pb}\textcolor{gray}{\textbf{DESSAUERSTRASSE 19}}\oindex{Dessauer Straße@\textbf{Dessauer Straße}, \emph{Straße}|pw}\pend
           
\pstart
           \raggedleft{}Berlin\oindex{Berlin@\textbf{Berlin}, \emph{Hauptstadt}|pw}, 27. Mai.\pend
           \pend
           
\pstart
           \centering{}Mein lieber Freund,\pend
           
\pstart
           Du biſt wieder einmal ganz verſtummt. Von Woche zu Woche warte ich auf eine
               Nachricht, aber vergebens.\pend
           
\pstart
           Wann alſo wirſt Du anfangen zu \label{K_L02916-1v}\edtext{reiſen}{\lemma{\textnormal{\emph{reisen}}}\Cendnote{\textnormal{Schnitzler war bereits seit 24. 5. 1900 in Puchberg am Schneeberg\oindex{Puchberg am Schneeberg@\textbf{Puchberg am Schneeberg}, \emph{Hauptstadt}|pwk}, wo er bis zum 27. 5. 1900 blieb und
                  Zeit mit Felix Salten\pwindex{Salten, Felix 6.\,9.\,1869 Budapest – 8.\,10.\,1945 Zürich@\textsc{Salten, Felix} (6.\,9.\,1869 Budapest – 8.\,10.\,1945 Zürich), \emph{Schriftsteller, Journalist, Chefredakteur}|pwk} und Ottilie Metzl\pwindex{Salten, Ottilie 7.\,3.\,1868 Prag – 22.\,6.\,1942 Zürich@\textsc{Salten, Ottilie} (7.\,3.\,1868 Prag – 22.\,6.\,1942 Zürich), \emph{Schauspielerin}|pwk} (später Salten\pwindex{Salten, Ottilie 7.\,3.\,1868 Prag – 22.\,6.\,1942 Zürich@\textsc{Salten, Ottilie} (7.\,3.\,1868 Prag – 22.\,6.\,1942 Zürich), \emph{Schauspielerin}|pwkv}) verbrachte.}}}\label{K_L02916-1}? Und wohin? Intereſſant wäre es
               auch, die Frage zu{ }ſtellen: mit wem? Aber ich{ }ſtelle{ }ſie lieber nicht.\pend
           
\pstart
           \textsc{Rudolf Lothar\pwindex{Lothar, Rudolf 23.\,2.\,1865 Budapest – 2.\,10.\,1943 ebd.@\textsc{Lothar, Rudolf} (23.\,2.\,1865 Budapest – 2.\,10.\,1943 ebd.), \emph{Schriftsteller, Journalist, Theaterdirektor}|pw}} hat{ }ſich hier hübſch benommen. Er hat{ }ſich einen in Berlin\oindex{Berlin@\textbf{Berlin}, \emph{Hauptstadt}|pw} lebenden \label{K_L02916-2v}\edtext{Wien\oindex{Wien@\textbf{Wien}, \emph{Verwaltungsgebiet}|pw}er Journaliſten\pwindex{?? [in Berlin lebender Wiener Journalist] @\textsc{?? [in Berlin lebender Wiener Journalist]}|pwv} engagirt}{\lemma{\textnormal{\emph{Wiener … engagirt}}}\Cendnote{\textnormal{Der Journalist\pwindex{?? [in Berlin lebender Wiener Journalist] @\textsc{?? [in Berlin lebender Wiener Journalist]}|pwkv} konnte nicht identifiziert werden; vgl. XXXX Auszeichnungsfehler: Dokument L02915 nicht gefunden.
               }}}\label{K_L02916-2}, der \strikeout{b} von Berlin\oindex{Berlin@\textbf{Berlin}, \emph{Hauptstadt}|pw}er Redaktionen wegen »Inkorrektheiten« entlaſſen worden iſt, und hat
               von dieſem am Abend{ }ſeiner \textsc{Première\pwindex{Lothar, Rudolf 23.\,2.\,1865 Budapest – 2.\,10.\,1943 ebd.@\textsc{Lothar, Rudolf} (23.\,2.\,1865 Budapest – 2.\,10.\,1943 ebd.), \emph{Schriftsteller, Journalist, Theaterdirektor}!König Harlekin. Maskenspiel in vier Aufzügen@\strich\emph{König Harlekin. Maskenspiel in vier Aufzügen}|pwv}}{ }ein gefälſchtes \label{K_L02916-3v}\edtext{Telegramm\pwindex{Theater, Kunst und Literatur [König Harlekin]@\emph{Theater, Kunst und Literatur [König Harlekin]}|pwv}}{\lemma{\textnormal{\emph{Telegramm}}}\Cendnote{\textnormal{Abgedruckt zum Beispiel im \emph{Neuen Wiener Tagblatt}\pwindex{Neues Wiener Tagblatt@\emph{Neues Wiener Tagblatt}|pwk}: [O. V.]\pwindex{?? [in Berlin lebender Wiener Journalist] @\textsc{?? [in Berlin lebender Wiener Journalist]}|pwkv}: \emph{Theater, Kunst und Literatur}\pwindex{Theater, Kunst und Literatur [König Harlekin]@\emph{Theater, Kunst und Literatur [König Harlekin]}|pwk}. In: \emph{Neues Wiener Tagblatt}\pwindex{Neues Wiener Tagblatt@\emph{Neues Wiener Tagblatt}|pwk}, Jg. 34, Nr. 137,
                        20. 5. 1900, Tages-Ausgabe,
                  S. 8.}}}\label{K_L02916-3} an alle Wien\oindex{Wien@\textbf{Wien}, \emph{Verwaltungsgebiet}|pw}er Blätter{ }ſenden laſſen. Für die N. Fr. Pr.\orgindex{Neue Freie Presse@Neue Freie Presse|pw} hat \textsc{Landau\pwindex{Landau, Isidor 1850 Zbaraz – 1944 Zürich@\textsc{Landau, Isidor} (1850 Zbaraz – 1944 Zürich), \emph{Journalist}|pw}} vom Börſencourier\orgindex{Berliner Börsen-Courier@Berliner Börsen-Courier|pw}{ }{\pb}\label{K_L02916-4v}\edtext{telegraphirt\pwindex{Theater- und Kunstnachrichten [König Harlekin]@\emph{Theater- und Kunstnachrichten [König Harlekin]}|pwv}}{\lemma{\textnormal{\emph{telegraphirt}}}\Cendnote{\textnormal{[O. V.] [ = Isidor Landau\pwindex{Landau, Isidor 1850 Zbaraz – 1944 Zürich@\textsc{Landau, Isidor} (1850 Zbaraz – 1944 Zürich), \emph{Journalist}|pwk}]: \emph{Theater- und Kunstnachrichten}\pwindex{Theater- und Kunstnachrichten [König Harlekin]@\emph{Theater- und Kunstnachrichten [König Harlekin]}|pwk}. In: \emph{Neue Freie Presse}\pwindex{Neue Freie Presse@\emph{Neue Freie Presse}|pwk}, Nr. 12.837, 20. 5. 1900, Morgenblatt, S. 9.}}}\label{K_L02916-4}, der
               bekanntlich die Spezialität hat, Alles zu loben. Aber{ }ſelbſt deſſen Telegramm\pwindex{Theater- und Kunstnachrichten [König Harlekin]@\emph{Theater- und Kunstnachrichten [König Harlekin]}|pwv} genügte noch nicht, und{ }ſo hat
               man in der Redaktion\orgindex{Neue Freie Presse@Neue Freie Presse|pwv} dieſe
               Fälſchung durch Einfügung einiger lobender Sätze noch \strikeout{\textcolor{gray}{f}} weiter gefälſcht. Dem \textsc{Fritz Mauthner\pwindex{Mauthner, Fritz 20.\,11.\,1849 Hořice – 29.\,6.\,1923 Meersburg@\textsc{Mauthner, Fritz} (20.\,11.\,1849 Hořice – 29.\,6.\,1923 Meersburg), \emph{Schriftsteller, Journalist, Philosoph}|pw}} hat{ }ſich \textsc{Lothar\pwindex{Lothar, Rudolf 23.\,2.\,1865 Budapest – 2.\,10.\,1943 ebd.@\textsc{Lothar, Rudolf} (23.\,2.\,1865 Budapest – 2.\,10.\,1943 ebd.), \emph{Schriftsteller, Journalist, Theaterdirektor}|pw}}{ }ſeit dem Tage{ }ſeiner Ankunft an die Rockſchöße gehangen, er hat ihn umwedelt
               und umſchmeichelt. Die Folge davon war, daß \textsc{Mauthner\pwindex{Mauthner, Fritz 20.\,11.\,1849 Hořice – 29.\,6.\,1923 Meersburg@\textsc{Mauthner, Fritz} (20.\,11.\,1849 Hořice – 29.\,6.\,1923 Meersburg), \emph{Schriftsteller, Journalist, Philosoph}|pw}} in{ }ſeinem \label{K_L02916-5v}\edtext{Feuilleton\pwindex{Mauthner, Fritz 20.\,11.\,1849 Hořice – 29.\,6.\,1923 Meersburg@\textsc{Mauthner, Fritz} (20.\,11.\,1849 Hořice – 29.\,6.\,1923 Meersburg), \emph{Schriftsteller, Journalist, Philosoph}!Wiener Deutsches Volkstheater. (Gastspiel im Deutschen Theater.) »König Harlekin«, ein Maskenspiel in vier Aufzügen von Rudolf Lothar@\strich\emph{Wiener Deutsches Volkstheater. (Gastspiel im Deutschen Theater.) »König Harlekin«, ein Maskenspiel in vier Aufzügen von Rudolf Lothar}|pwv}}{\lemma{\textnormal{\emph{Feuilleton}}}\Cendnote{\textnormal{F. M.\pwindex{Mauthner, Fritz 20.\,11.\,1849 Hořice – 29.\,6.\,1923 Meersburg@\textsc{Mauthner, Fritz} (20.\,11.\,1849 Hořice – 29.\,6.\,1923 Meersburg), \emph{Schriftsteller, Journalist, Philosoph}|pwkv} [ = Fritz Mauthner\pwindex{Mauthner, Fritz 20.\,11.\,1849 Hořice – 29.\,6.\,1923 Meersburg@\textsc{Mauthner, Fritz} (20.\,11.\,1849 Hořice – 29.\,6.\,1923 Meersburg), \emph{Schriftsteller, Journalist, Philosoph}|pwk}]: \emph{Wiener Deutsches Volkstheater. (Gastspiel im Deutschen
                        Theater.) »König Harlekin«, ein Maskenspiel in vier Aufzügen von Rudolf
                        Lothar}\pwindex{Mauthner, Fritz 20.\,11.\,1849 Hořice – 29.\,6.\,1923 Meersburg@\textsc{Mauthner, Fritz} (20.\,11.\,1849 Hořice – 29.\,6.\,1923 Meersburg), \emph{Schriftsteller, Journalist, Philosoph}!Wiener Deutsches Volkstheater. (Gastspiel im Deutschen Theater.) »König Harlekin«, ein Maskenspiel in vier Aufzügen von Rudolf Lothar@\strich\emph{Wiener Deutsches Volkstheater. (Gastspiel im Deutschen Theater.) »König Harlekin«, ein Maskenspiel in vier Aufzügen von Rudolf Lothar}|pwk}. In: \emph{Berliner Tageblatt}\pwindex{Berliner Tageblatt@\emph{Berliner Tageblatt}|pwk},
                     Jg. 29, Nr. 254, 20. 5. 1900,
                  S. [3].}}}\label{K_L02916-5} vom »Dichter{ }\textsc{Lothar\pwindex{Lothar, Rudolf 23.\,2.\,1865 Budapest – 2.\,10.\,1943 ebd.@\textsc{Lothar, Rudolf} (23.\,2.\,1865 Budapest – 2.\,10.\,1943 ebd.), \emph{Schriftsteller, Journalist, Theaterdirektor}|pw}}\pwindex{Mauthner, Fritz 20.\,11.\,1849 Hořice – 29.\,6.\,1923 Meersburg@\textsc{Mauthner, Fritz} (20.\,11.\,1849 Hořice – 29.\,6.\,1923 Meersburg), \emph{Schriftsteller, Journalist, Philosoph}!Wiener Deutsches Volkstheater. (Gastspiel im Deutschen Theater.) »König Harlekin«, ein Maskenspiel in vier Aufzügen von Rudolf Lothar@\strich\emph{Wiener Deutsches Volkstheater. (Gastspiel im Deutschen Theater.) »König Harlekin«, ein Maskenspiel in vier Aufzügen von Rudolf Lothar}|pwv}«{ }ſprach. Damit iſt \textsc{Mauthner\pwindex{Mauthner, Fritz 20.\,11.\,1849 Hořice – 29.\,6.\,1923 Meersburg@\textsc{Mauthner, Fritz} (20.\,11.\,1849 Hořice – 29.\,6.\,1923 Meersburg), \emph{Schriftsteller, Journalist, Philosoph}|pw}} als Kritiker allerdings für mich gerichtet.\pend
           
\pstart
           Als \textsc{Karlweiss\pwindex{Karlweis, Carl 23.\,11.\,1850 Wien – 27.\,10.\,1901 ebd.@\textsc{Karlweis, Carl} (23.\,11.\,1850 Wien – 27.\,10.\,1901 ebd.), \emph{Schriftsteller}|pw}}’ »Onkel Toni\pwindex{Karlweis, Carl 23.\,11.\,1850 Wien – 27.\,10.\,1901 ebd.@\textsc{Karlweis, Carl} (23.\,11.\,1850 Wien – 27.\,10.\,1901 ebd.), \emph{Schriftsteller}!Onkel Toni. Eine Komödie aus der Gesellschaft in vier Aufzügen@\strich\emph{Onkel Toni. Eine Komödie aus der Gesellschaft in vier Aufzügen}|pw}« hier\oindex{Berlin@\textbf{Berlin}, \emph{Hauptstadt}|pwv}{ }\label{K_L02916-6v}\edtext{aufgeführt}{\lemma{\textnormal{\emph{aufgeführt}}}\Cendnote{\textnormal{Goldmann\pwindex{Goldmann, Paul 31.\,1.\,1865 Breslau – 25.\,9.\,1935 Wien@\textsc{Goldmann, Paul} (31.\,1.\,1865 Breslau – 25.\,9.\,1935 Wien), \emph{Schriftsteller, Journalist}|pwk} bezog sich auf das Gastspiel des
                  \emph{Volkstheaters}\orgindex{Volkstheater@Volkstheater|pwk}, das am am 11. 5. 1900{ }\emph{Onkel Toni}\pwindex{Karlweis, Carl 23.\,11.\,1850 Wien – 27.\,10.\,1901 ebd.@\textsc{Karlweis, Carl} (23.\,11.\,1850 Wien – 27.\,10.\,1901 ebd.), \emph{Schriftsteller}!Onkel Toni. Eine Komödie aus der Gesellschaft in vier Aufzügen@\strich\emph{Onkel Toni. Eine Komödie aus der Gesellschaft in vier Aufzügen}|pwk} gab.}}}\label{K_L02916-6} wurde, telegraphirte\pwindex{Kleine Chronik. [Theater.] [Onkel Toni]@\emph{Kleine Chronik. [Theater.] [Onkel Toni]}|pwv} ich {\pb}ganz{ }ſanft: Die vortreffliche Aufführung habe über die{ }ſchwachen Stellen des Stück\pwindex{Karlweis, Carl 23.\,11.\,1850 Wien – 27.\,10.\,1901 ebd.@\textsc{Karlweis, Carl} (23.\,11.\,1850 Wien – 27.\,10.\,1901 ebd.), \emph{Schriftsteller}!Onkel Toni. Eine Komödie aus der Gesellschaft in vier Aufzügen@\strich\emph{Onkel Toni. Eine Komödie aus der Gesellschaft in vier Aufzügen}|pwv}es hinweggeholfen. Der
               Satz \strikeout{wurde} wurde \label{K_L02916-7v}\edtext{geſtrichen}{\lemma{\textnormal{\emph{gestrichen}}}\Cendnote{\textnormal{[Paul Goldmann\pwindex{Goldmann, Paul 31.\,1.\,1865 Breslau – 25.\,9.\,1935 Wien@\textsc{Goldmann, Paul} (31.\,1.\,1865 Breslau – 25.\,9.\,1935 Wien), \emph{Schriftsteller, Journalist}|pwk}]: \emph{Kleine Chronik. [Theater]}\pwindex{Kleine Chronik. [Theater.] [Onkel Toni]@\emph{Kleine Chronik. [Theater.] [Onkel Toni]}|pwk}. In: \emph{Neue Freie Presse}\pwindex{Neue Freie Presse@\emph{Neue Freie Presse}|pwk}, Nr. 12.829, 12. 5. 1900, Abendblatt, S. 1.}}}\label{K_L02916-7}. Ein
                  Stück\pwindex{Karlweis, Carl 23.\,11.\,1850 Wien – 27.\,10.\,1901 ebd.@\textsc{Karlweis, Carl} (23.\,11.\,1850 Wien – 27.\,10.\,1901 ebd.), \emph{Schriftsteller}!Onkel Toni. Eine Komödie aus der Gesellschaft in vier Aufzügen@\strich\emph{Onkel Toni. Eine Komödie aus der Gesellschaft in vier Aufzügen}|pw} von \textsc{Karlweiss\pwindex{Karlweis, Carl 23.\,11.\,1850 Wien – 27.\,10.\,1901 ebd.@\textsc{Karlweis, Carl} (23.\,11.\,1850 Wien – 27.\,10.\,1901 ebd.), \emph{Schriftsteller}|pw}} darf nicht einmal{ }ſchwache Stellen haben!\pend
           
\pstart
           Der »\textsc{Star\pwindex{Bahr, Hermann 19.\,7.\,1863 Linz – 15.\,1.\,1934 München@\textsc{Bahr, Hermann} (19.\,7.\,1863 Linz – 15.\,1.\,1934 München), \emph{Schriftsteller, Kritiker}!Star. Ein Wiener Stück in vier Akten@\strich\emph{Der Star. Ein Wiener Stück in vier Akten}|pw}}« von \textsc{Bahr\pwindex{Bahr, Hermann 19.\,7.\,1863 Linz – 15.\,1.\,1934 München@\textsc{Bahr, Hermann} (19.\,7.\,1863 Linz – 15.\,1.\,1934 München), \emph{Schriftsteller, Kritiker}|pw}} hat mir hingegen \label{K_L02916-8v}\edtext{gefallen}{\lemma{\textnormal{\emph{gefallen}}}\Cendnote{\textnormal{Das Stück\pwindex{Bahr, Hermann 19.\,7.\,1863 Linz – 15.\,1.\,1934 München@\textsc{Bahr, Hermann} (19.\,7.\,1863 Linz – 15.\,1.\,1934 München), \emph{Schriftsteller, Kritiker}!Star. Ein Wiener Stück in vier Akten@\strich\emph{Der Star. Ein Wiener Stück in vier Akten}|pwkv} feierte am 25. 5. 1900 am Berlin\oindex{Berlin@\textbf{Berlin}, \emph{Hauptstadt}|pwk}er \emph{Lessing-Theater}\orgindex{Lessing-Theater@Lessing-Theater|pwk} Premiere.}}}\label{K_L02916-8}. Dieſer widerliche Burſch\pwindex{Bahr, Hermann 19.\,7.\,1863 Linz – 15.\,1.\,1934 München@\textsc{Bahr, Hermann} (19.\,7.\,1863 Linz – 15.\,1.\,1934 München), \emph{Schriftsteller, Kritiker}|pwv} hat doch – leider! – Humor und
               Talent.\pend
           
\pstart
           Bitte, \label{K_L02916-9v}\edtext{lies’}{\lemma{\textnormal{\emph{lies’}}}\Cendnote{\textnormal{Schnitzler las den Roman\pwindex{Keßler, Helene 23.\,2.\,1870 Heilbad Heiligenstadt – 8.\,8.\,1957 Baden-Baden@\textsc{Keßler, Helene} (23.\,2.\,1870 Heilbad Heiligenstadt – 8.\,8.\,1957 Baden-Baden), \emph{Schriftstellerin}!Familie von Barchwitz. Roman@\strich\emph{Die Familie von Barchwitz. Roman}|pwkv} (vgl. A. S.: \emph{Lektüren}, deutschsprachige Literatur).}}}\label{K_L02916-9}, wenn Du es noch
               nicht kennſt, »Die Familie von \textsc{Barchwitz}\pwindex{Keßler, Helene 23.\,2.\,1870 Heilbad Heiligenstadt – 8.\,8.\,1957 Baden-Baden@\textsc{Keßler, Helene} (23.\,2.\,1870 Heilbad Heiligenstadt – 8.\,8.\,1957 Baden-Baden), \emph{Schriftstellerin}!Familie von Barchwitz. Roman@\strich\emph{Die Familie von Barchwitz. Roman}|pw}« von \textsc{Hans von Kahlenberg\pwindex{Keßler, Helene 23.\,2.\,1870 Heilbad Heiligenstadt – 8.\,8.\,1957 Baden-Baden@\textsc{Keßler, Helene} (23.\,2.\,1870 Heilbad Heiligenstadt – 8.\,8.\,1957 Baden-Baden), \emph{Schriftstellerin}|pwv}}. Seit Langem hat mich kein Roman{ }ſo intereſſirt. \strikeout{V\textcolor{gray}{erg}}{ }Verfaſſerin\pwindex{Keßler, Helene 23.\,2.\,1870 Heilbad Heiligenstadt – 8.\,8.\,1957 Baden-Baden@\textsc{Keßler, Helene} (23.\,2.\,1870 Heilbad Heiligenstadt – 8.\,8.\,1957 Baden-Baden), \emph{Schriftstellerin}|pwv} iſt ein nicht
               mehr {\pb}ganz \strikeout{\textcolor{gray}{hu}} junges, aber \strikeout{\textcolor{gray}{r}} noch \strikeout{recht} recht hübſches Mädchen\pwindex{Keßler, Helene 23.\,2.\,1870 Heilbad Heiligenstadt – 8.\,8.\,1957 Baden-Baden@\textsc{Keßler, Helene} (23.\,2.\,1870 Heilbad Heiligenstadt – 8.\,8.\,1957 Baden-Baden), \emph{Schriftstellerin}|pwv}, ein Fräulein von \textsc{Montbart\pwindex{Keßler, Helene 23.\,2.\,1870 Heilbad Heiligenstadt – 8.\,8.\,1957 Baden-Baden@\textsc{Keßler, Helene} (23.\,2.\,1870 Heilbad Heiligenstadt – 8.\,8.\,1957 Baden-Baden), \emph{Schriftstellerin}|pwv}}, Offizier\pwindex{Monbart, Erich von 1836 – 1907@\textsc{Monbart, Erich von} (1836 – 1907), \emph{Offizier, Oberleutnant}|pwv}s-Tochter.\pend
           
\pstart
           Was macht \textsc{Richard\pwindex{Beer-Hofmann, Richard 11.\,7.\,1866 Wien – 26.\,9.\,1945 New York City@\textsc{Beer-Hofmann, Richard} (11.\,7.\,1866 Wien – 26.\,9.\,1945 New York City), \emph{Schriftsteller}|pw}}?\pend
           
\pstart
           Bitte,{ }ſchreib’ mir bald!\pend
           
\pstart
           Viele treue Grüße! {\\[\baselineskip]}Dein {\\[\baselineskip]}\spacefill\mbox{Paul Goldmann}\pend
           \leftskip=0em{}
\pstart
           \noindent{}Auch \textsc{Ludassy\pwindex{Gans-Ludassy, Julius von 13.\,4.\,1858 Wien – 30.\,9.\,1922 ebd.@\textsc{Gans-Ludassy, Julius von} (13.\,4.\,1858 Wien – 30.\,9.\,1922 ebd.), \emph{Schriftsteller, Journalist, Herausgeber}|pw}} benimmt{ }ſich abſcheulich hier und macht{ }ſich aus dem \label{K_L02916-10v}\edtext{Verbot{ }ſeines{ }ſchlechten Stück\pwindex{Gans-Ludassy, Julius von 13.\,4.\,1858 Wien – 30.\,9.\,1922 ebd.@\textsc{Gans-Ludassy, Julius von} (13.\,4.\,1858 Wien – 30.\,9.\,1922 ebd.), \emph{Schriftsteller, Journalist, Herausgeber}!letzte Knopf. Volksstück in drei Aufzügen@\strich\emph{Der letzte Knopf. Volksstück in drei Aufzügen}|pwv}es}{\lemma{\textnormal{\emph{Verbot … Stückes}}}\Cendnote{\textnormal{Julius von Gans-Ludassys\pwindex{Gans-Ludassy, Julius von 13.\,4.\,1858 Wien – 30.\,9.\,1922 ebd.@\textsc{Gans-Ludassy, Julius von} (13.\,4.\,1858 Wien – 30.\,9.\,1922 ebd.), \emph{Schriftsteller, Journalist, Herausgeber}|pwk}{ }\emph{Der letzte Knopf}\pwindex{Gans-Ludassy, Julius von 13.\,4.\,1858 Wien – 30.\,9.\,1922 ebd.@\textsc{Gans-Ludassy, Julius von} (13.\,4.\,1858 Wien – 30.\,9.\,1922 ebd.), \emph{Schriftsteller, Journalist, Herausgeber}!letzte Knopf. Volksstück in drei Aufzügen@\strich\emph{Der letzte Knopf. Volksstück in drei Aufzügen}|pwk} war am 8. 4. 1900 am \emph{Volkstheater}\orgindex{Volkstheater@Volkstheater|pwk}
                     uraufgeführt worden. Das Stück\pwindex{Gans-Ludassy, Julius von 13.\,4.\,1858 Wien – 30.\,9.\,1922 ebd.@\textsc{Gans-Ludassy, Julius von} (13.\,4.\,1858 Wien – 30.\,9.\,1922 ebd.), \emph{Schriftsteller, Journalist, Herausgeber}!letzte Knopf. Volksstück in drei Aufzügen@\strich\emph{Der letzte Knopf. Volksstück in drei Aufzügen}|pwkv}, das für einen Skandal sorgte, sollte auch in Berlin\oindex{Berlin@\textbf{Berlin}, \emph{Hauptstadt}|pwk} aufgeführt werden. Ludwig Fulda\pwindex{Fulda, Ludwig 15.\,7.\,1862 Frankfurt am Main – 30.\,3.\,1939 Berlin@\textsc{Fulda, Ludwig} (15.\,7.\,1862 Frankfurt am Main – 30.\,3.\,1939 Berlin), \emph{Schriftsteller, Übersetzer}|pwk}, der als Präsident der \emph{Freien Bühne}\orgindex{Freie Bühne@Freie Bühne|pwk} das von der Zensur verbotene Stück\pwindex{Gans-Ludassy, Julius von 13.\,4.\,1858 Wien – 30.\,9.\,1922 ebd.@\textsc{Gans-Ludassy, Julius von} (13.\,4.\,1858 Wien – 30.\,9.\,1922 ebd.), \emph{Schriftsteller, Journalist, Herausgeber}!letzte Knopf. Volksstück in drei Aufzügen@\strich\emph{Der letzte Knopf. Volksstück in drei Aufzügen}|pwkv} annahm, musste von
                     seiner Funktion zurücktreten. Schließlich wurde es vor einem geladenen Publikum
                     am 30. 5. 1900 bei einer Matinée des \emph{Deutschen Theaters}\orgindex{Deutsches Theater Berlin@Deutsches Theater Berlin|pwk} aufgeführt.}}}\label{K_L02916-10} eine
                  unerträgliche Reklame.\pend
           \selectlanguage{ngerman}\endnumbering\briefempfaengerindex{Schnitzler, Arthur@\textsc{Schnitzler, Arthur}!zzzGoldmann, Paul@\emph{von Paul Goldmann}!1900-05-271@{27. 5. [1900]}|)be}\mylabel{L02916h}  \newcommand{\dateiname}{L02916}\newcommand{\titel}{Paul Goldmann an Arthur Schnitzler, 27. 5. [1900]}\newcommand{\editorInnen}{Martin Anton Müller und Laura Untner}%% latex-leseansicht-abspann.tex
%% Abspann für die Leseansicht.
%% Der Schalter \ifkorrekturansicht ist bereits durch den Vorspann gesetzt.

%% latex-abspann.tex
%% Gemeinsamer Abspann für Korrekturansicht und Leseansicht.
%% Setzt den Schalter \ifkorrekturansicht voraus (gesetzt in den
%% einbindenden Dateien latex-korrekturansicht-abspann.tex bzw.
%% latex-leseansicht-abspann.tex).
%% ---------------------------------------------------------------

\normalsize

% Das esempio-Environment wird nur in der Leseansicht benötigt
\ifkorrekturansicht\else
\newenvironment{esempio}[3]%
{
    \vspace{1.5ex}
    \rlap{\underline{#1}}
    \par
    \setlength{\parindent}{0cm}
    \nopagebreak
    \leftskip=#2cm
    \rightskip=#3cm
}
{
    \par
}
\fi

\doendnotes{C}
\bigskip
\vfill

\clearpage

\footnotesize

\ifkorrekturansicht
  \lohead{\textsc{register}}
\fi

% theindex-Environment neu definieren ohne reledmac
\makeatletter
\renewenvironment{theindex}{%
  \ifkorrekturansicht
    \section*{\indexname}%
  \else
    \subsubsection*{Index der erwähnten Entitäten}%
  \fi
  \setlength{\parindent}{0pt}%
  \setlength{\parskip}{0pt plus 0.3pt}%
  \let\item\@idxitem
}{%
  \ifkorrekturansicht\clearpage\fi
}
\makeatother

\IfFileExists{\jobname-pw.ind}{\input{\jobname-pw.ind}}{}

% Quellenangabe nur in der Leseansicht
\ifkorrekturansicht\else
% Fallback-Definitionen, falls die .tex-Datei \titel etc. nicht gesetzt hat
\providecommand{\titel}{}
\providecommand{\editorInnen}{}
\providecommand{\dateiname}{\jobname}

\vspace{3cm}

\vfill

\footnotesize
\textsc{Quelle}: \titel. Herausgegeben von {\editorInnen}. In: \emph{Arthur Schnitzler: Briefwechsel mit Autorinnen und Autoren}.
 Digitale Edition, https://schnitzler-briefe.acdh.oeaw.ac.at/{\dateiname}.html (Stand \today)
\fi

\end{document}


