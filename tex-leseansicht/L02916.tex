%% latex-korrekturansicht-vorspann.tex
%% Vorspann für die Korrekturansicht.
%% Lädt die gemeinsame Datei latex-vorspann.tex mit gesetztem Schalter.

\newif\ifkorrekturansicht
\korrekturansichttrue

\input{../tex-inputs/latex-vorspann}


\section[ Paul Goldmann an Arthur Schnitzler, 27. 5. {[}1900{]}]{L02916 Paul Goldmann an Arthur Schnitzler, 27. 5. {[}1900{]}}
\nopagebreak\mylabel{L02916v}
\rehead{ }\normalsize\beginnumbering\briefempfaengerindex{Schnitzler, Arthur@\textsc{Schnitzler, Arthur}!zzzGoldmann, Paul@\emph{von Paul Goldmann}!1900-05-271@{27. 5. {[}1900{]}}|(be}
\toendnotes[C]{\smallbreak\pagebreak[2]}\Standort{DLA, A:Schnitzler, HS.NZ85.1.3170.}
\physDesc{Brief, 1 Blatt, 4 Seiten, 1915 Zeichen
\newline{}Handschrift: blaue Tinte, deutsche Kurrent
\newline{}Schnitzler: 1) mit Bleistift das Jahr »900« vermerkt  2) mit rotem Buntstift sechs Unterstreichungen}\toendnotes[C]{\smallbreak}
\pstart
           \noindent{}
\pstart
           {\pb}\textcolor{gray}{\textbf{DESSAUERSTRASSE 19}}\oindex{Dessauer Strasse@\textbf{Dessauer Straße}, \emph{Straße (K.STR)}|pw}\pend
           
\pstart
           \raggedleft{}Berlin\oindex{Berlin@\textbf{Berlin}, \emph{P.PPLC}|pw}, 27. Mai.\pend
           \pend
           
\pstart
           \centering{}Mein lieber Freund,\pend
           
\pstart
           Du biſt wieder einmal ganz verſtummt. Von Woche zu Woche warte ich auf eine
               Nachricht, aber vergebens.\pend
           
\pstart
           Wann alſo wirſt Du anfangen zu \label{K_L02916-1v}\edtext{reiſen}{\lemma{\textnormal{\emph{reiſen}}}\Cendnote{\textnormal{Schnitzler war bereits seit 24. 5. 1900 in Puchberg am Schneeberg\oindex{Puchberg am Schneeberg@\textbf{Puchberg am Schneeberg}, \emph{P.PPLA3}|pwk}, wo er bis zum 27. 5. 1900 blieb und
                  Zeit mit Felix Salten\pwindex{Salten, Felix 06.09.1869 – 08.10.1945@\textsc{Salten, Felix} (06.09.1869 – 08.10.1945), \emph{Schriftsteller/Schriftstellerin, Journalist/Journalistin, Chefredakteur/Chefredakteurin}|pwk} und Ottilie Metzl\pwindex{Salten, Ottilie 07.03.1868 – 22.06.1942@\textsc{Salten, Ottilie} (07.03.1868 – 22.06.1942), \emph{Schauspieler/Schauspielerin}|pwk} (später Salten\pwindex{Salten, Ottilie 07.03.1868 – 22.06.1942@\textsc{Salten, Ottilie} (07.03.1868 – 22.06.1942), \emph{Schauspieler/Schauspielerin}|pwkv}) verbrachte.}}}\label{K_L02916-1}? Und wohin? Intereſſant wäre es
               auch, die Frage zu ſtellen: mit wem? Aber ich ſtelle ſie lieber nicht.\pend
           
\pstart
           \textsc{Rudolf Lothar\pwindex{Lothar, Rudolf 23.2.1865 – 2.10.1943@\textsc{Lothar, Rudolf} (23.2.1865 – 2.10.1943), \emph{Schriftsteller/Schriftstellerin, Journalist/Journalistin, Theaterdirektor/Theaterdirektorin}|pw}} hat ſich hier hübſch benommen. Er hat ſich einen in Berlin\oindex{Berlin@\textbf{Berlin}, \emph{P.PPLC}|pw} lebenden \label{K_L02916-2v}\edtext{Wien\oindex{Wien@\textbf{Wien}, \emph{A.ADM2}|pw}er Journaliſten\pwindex{?? [in Berlin lebender Wiener Journalist] @\textsc{?? [in Berlin lebender Wiener Journalist]}|pwv} engagirt}{\lemma{\textnormal{\emph{Wiener … engagirt}}}\Cendnote{\textnormal{Der Journalist\pwindex{?? [in Berlin lebender Wiener Journalist] @\textsc{?? [in Berlin lebender Wiener Journalist]}|pwkv} konnte nicht identifiziert werden; vgl. Paul Goldmann an Arthur Schnitzler, 10. 5. [1900].
               }}}\label{K_L02916-2}, der \strikeout{b} von Berlin\oindex{Berlin@\textbf{Berlin}, \emph{P.PPLC}|pw}er Redaktionen wegen »Inkorrektheiten« entlaſſen worden iſt, und hat
               von dieſem am Abend ſeiner \textsc{Première\pwindex{Koenig Harlekin. Maskenspiel in vier Aufzuegen@\emph{König Harlekin. Maskenspiel in vier Aufzügen}|pwv}}{ }ein gefälſchtes \label{K_L02916-3v}\edtext{Telegramm\pwindex{Theater, Kunst und Literatur [Koenig Harlekin]@\emph{Theater, Kunst und Literatur [König Harlekin]}|pwv}}{\lemma{\textnormal{\emph{Telegramm}}}\Cendnote{\textnormal{Abgedruckt zum Beispiel im \emph{Neuen Wiener Tagblatt}\pwindex{Neues Wiener Tagblatt@\emph{Neues Wiener Tagblatt}|pwk}: [O. V.]\pwindex{?? [in Berlin lebender Wiener Journalist] @\textsc{?? [in Berlin lebender Wiener Journalist]}|pwkv}: \emph{Theater, Kunst und Literatur}\pwindex{Theater, Kunst und Literatur [Koenig Harlekin]@\emph{Theater, Kunst und Literatur [König Harlekin]}|pwk}. In: \emph{Neues Wiener Tagblatt}\pwindex{Neues Wiener Tagblatt@\emph{Neues Wiener Tagblatt}|pwk}, Jg. 34, Nr. 137,
                        20. 5. 1900, Tages-Ausgabe,
                  S. 8.}}}\label{K_L02916-3} an alle Wien\oindex{Wien@\textbf{Wien}, \emph{A.ADM2}|pw}er Blätter
               ſenden laſſen. Für die N. Fr. Pr.\orgindex{Neue Freie Presse@Neue Freie Presse|pw} hat \textsc{Landau\pwindex{Landau, Isidor 1850 – 1944@\textsc{Landau, Isidor} (1850 – 1944), \emph{Journalist/Journalistin}|pw}} vom Börſencourier\orgindex{Berliner Boersen-Courier@Berliner Börsen-Courier|pw}{ }{\pb}\label{K_L02916-4v}\edtext{telegraphirt\pwindex{Theater- und Kunstnachrichten [Koenig Harlekin]@\emph{Theater- und Kunstnachrichten [König Harlekin]}|pwv}}{\lemma{\textnormal{\emph{telegraphirt}}}\Cendnote{\textnormal{[O. V.] [ = Isidor Landau\pwindex{Landau, Isidor 1850 – 1944@\textsc{Landau, Isidor} (1850 – 1944), \emph{Journalist/Journalistin}|pwk}]: \emph{Theater- und Kunstnachrichten}\pwindex{Theater- und Kunstnachrichten [Koenig Harlekin]@\emph{Theater- und Kunstnachrichten [König Harlekin]}|pwk}. In: \emph{Neue Freie Presse}\pwindex{Neue Freie Presse@\emph{Neue Freie Presse}|pwk}, Nr. 12.837, 20. 5. 1900, Morgenblatt, S. 9.}}}\label{K_L02916-4}, der
               bekanntlich die Spezialität hat, Alles zu loben. Aber ſelbſt deſſen Telegramm\pwindex{Theater- und Kunstnachrichten [Koenig Harlekin]@\emph{Theater- und Kunstnachrichten [König Harlekin]}|pwv} genügte noch nicht, und ſo hat
               man in der Redaktion\orgindex{Neue Freie Presse@Neue Freie Presse|pwv} dieſe
               Fälſchung durch Einfügung einiger lobender Sätze noch \strikeout{\textcolor{gray}{f}} weiter gefälſcht. Dem \textsc{Fritz Mauthner\pwindex{Mauthner, Fritz 1849-11-20 – 1923-06-29@\textsc{Mauthner, Fritz} (1849-11-20 – 1923-06-29), \emph{Schriftsteller/Schriftstellerin, Journalist/Journalistin, Philosoph/Philosophin}|pw}} hat ſich \textsc{Lothar\pwindex{Lothar, Rudolf 23.2.1865 – 2.10.1943@\textsc{Lothar, Rudolf} (23.2.1865 – 2.10.1943), \emph{Schriftsteller/Schriftstellerin, Journalist/Journalistin, Theaterdirektor/Theaterdirektorin}|pw}} ſeit dem Tage ſeiner Ankunft an die Rockſchöße gehangen, er hat ihn umwedelt
               und umſchmeichelt. Die Folge davon war, daß \textsc{Mauthner\pwindex{Mauthner, Fritz 1849-11-20 – 1923-06-29@\textsc{Mauthner, Fritz} (1849-11-20 – 1923-06-29), \emph{Schriftsteller/Schriftstellerin, Journalist/Journalistin, Philosoph/Philosophin}|pw}} in ſeinem \label{K_L02916-5v}\edtext{Feuilleton\pwindex{Wiener Deutsches Volkstheater. (Gastspiel im Deutschen Theater.) »Koenig Harlekin«, ein Maskenspiel in vier Aufzuegen von Rudolf Lothar@\emph{Wiener Deutsches Volkstheater. (Gastspiel im Deutschen Theater.) »König Harlekin«, ein Maskenspiel in vier Aufzügen von Rudolf Lothar}|pwv}}{\lemma{\textnormal{\emph{Feuilleton}}}\Cendnote{\textnormal{F. M.\pwindex{Mauthner, Fritz 1849-11-20 – 1923-06-29@\textsc{Mauthner, Fritz} (1849-11-20 – 1923-06-29), \emph{Schriftsteller/Schriftstellerin, Journalist/Journalistin, Philosoph/Philosophin}|pwkv} [ = Fritz Mauthner\pwindex{Mauthner, Fritz 1849-11-20 – 1923-06-29@\textsc{Mauthner, Fritz} (1849-11-20 – 1923-06-29), \emph{Schriftsteller/Schriftstellerin, Journalist/Journalistin, Philosoph/Philosophin}|pwk}]: \emph{Wiener Deutsches Volkstheater. (Gastspiel im Deutschen
                        Theater.) »König Harlekin«, ein Maskenspiel in vier Aufzügen von Rudolf
                        Lothar}\pwindex{Wiener Deutsches Volkstheater. (Gastspiel im Deutschen Theater.) »Koenig Harlekin«, ein Maskenspiel in vier Aufzuegen von Rudolf Lothar@\emph{Wiener Deutsches Volkstheater. (Gastspiel im Deutschen Theater.) »König Harlekin«, ein Maskenspiel in vier Aufzügen von Rudolf Lothar}|pwk}. In: \emph{Berliner Tageblatt}\pwindex{Berliner Tageblatt@\emph{Berliner Tageblatt}|pwk},
                     Jg. 29, Nr. 254, 20. 5. 1900,
                  S. [3].}}}\label{K_L02916-5} vom »Dichter{ }\textsc{Lothar\pwindex{Lothar, Rudolf 23.2.1865 – 2.10.1943@\textsc{Lothar, Rudolf} (23.2.1865 – 2.10.1943), \emph{Schriftsteller/Schriftstellerin, Journalist/Journalistin, Theaterdirektor/Theaterdirektorin}|pw}}\pwindex{Wiener Deutsches Volkstheater. (Gastspiel im Deutschen Theater.) »Koenig Harlekin«, ein Maskenspiel in vier Aufzuegen von Rudolf Lothar@\emph{Wiener Deutsches Volkstheater. (Gastspiel im Deutschen Theater.) »König Harlekin«, ein Maskenspiel in vier Aufzügen von Rudolf Lothar}|pwv}« ſprach. Damit iſt \textsc{Mauthner\pwindex{Mauthner, Fritz 1849-11-20 – 1923-06-29@\textsc{Mauthner, Fritz} (1849-11-20 – 1923-06-29), \emph{Schriftsteller/Schriftstellerin, Journalist/Journalistin, Philosoph/Philosophin}|pw}} als Kritiker allerdings für mich gerichtet.\pend
           
\pstart
           Als \textsc{Karlweiss\pwindex{Karlweis, Carl 23.11.1850 – 27.10.1901@\textsc{Karlweis, Carl} (23.11.1850 – 27.10.1901), \emph{Schriftsteller/Schriftstellerin}|pw}}’ »Onkel Toni\pwindex{Onkel Toni. Eine Komoedie aus der Gesellschaft in vier Aufzuegen@\emph{Onkel Toni. Eine Komödie aus der Gesellschaft in vier Aufzügen}|pw}« hier\oindex{Berlin@\textbf{Berlin}, \emph{P.PPLC}|pwv}{ }\label{K_L02916-6v}\edtext{aufgeführt}{\lemma{\textnormal{\emph{aufgeführt}}}\Cendnote{\textnormal{Goldmann\pwindex{Goldmann, Paul 31.01.1865 – 25.09.1935@\textsc{Goldmann, Paul} (31.01.1865 – 25.09.1935), \emph{Schriftsteller/Schriftstellerin, Journalist/Journalistin}|pwk} bezog sich auf das Gastspiel des
                  \emph{Volkstheaters}\orgindex{Volkstheater@Volkstheater|pwk}, das am am 11. 5. 1900{ }\emph{Onkel Toni}\pwindex{Onkel Toni. Eine Komoedie aus der Gesellschaft in vier Aufzuegen@\emph{Onkel Toni. Eine Komödie aus der Gesellschaft in vier Aufzügen}|pwk} gab.}}}\label{K_L02916-6} wurde, telegraphirte\pwindex{Kleine Chronik. [Theater.] [Onkel Toni]@\emph{Kleine Chronik. [Theater.] [Onkel Toni]}|pwv} ich {\pb}ganz
               ſanft: Die vortreffliche Aufführung habe über die ſchwachen Stellen des Stück\pwindex{Onkel Toni. Eine Komoedie aus der Gesellschaft in vier Aufzuegen@\emph{Onkel Toni. Eine Komödie aus der Gesellschaft in vier Aufzügen}|pwv}es hinweggeholfen. Der
               Satz \strikeout{wurde} wurde \label{K_L02916-7v}\edtext{geſtrichen}{\lemma{\textnormal{\emph{geſtrichen}}}\Cendnote{\textnormal{[Paul Goldmann\pwindex{Goldmann, Paul 31.01.1865 – 25.09.1935@\textsc{Goldmann, Paul} (31.01.1865 – 25.09.1935), \emph{Schriftsteller/Schriftstellerin, Journalist/Journalistin}|pwk}]: \emph{Kleine Chronik. [Theater]}\pwindex{Kleine Chronik. [Theater.] [Onkel Toni]@\emph{Kleine Chronik. [Theater.] [Onkel Toni]}|pwk}. In: \emph{Neue Freie Presse}\pwindex{Neue Freie Presse@\emph{Neue Freie Presse}|pwk}, Nr. 12.829, 12. 5. 1900, Abendblatt, S. 1.}}}\label{K_L02916-7}. Ein
                  Stück\pwindex{Onkel Toni. Eine Komoedie aus der Gesellschaft in vier Aufzuegen@\emph{Onkel Toni. Eine Komödie aus der Gesellschaft in vier Aufzügen}|pw} von \textsc{Karlweiss\pwindex{Karlweis, Carl 23.11.1850 – 27.10.1901@\textsc{Karlweis, Carl} (23.11.1850 – 27.10.1901), \emph{Schriftsteller/Schriftstellerin}|pw}} darf nicht einmal ſchwache Stellen haben!\pend
           
\pstart
           Der »\textsc{Star\pwindex{Star. Ein Wiener Stueck in vier Akten@\emph{Der Star. Ein Wiener Stück in vier Akten}|pw}}« von \textsc{Bahr\pwindex{Bahr, Hermann 19.07.1863 – 15.01.1934@\textsc{Bahr, Hermann} (19.07.1863 – 15.01.1934), \emph{Schriftsteller/Schriftstellerin, Kritiker/Kritikerin}|pw}} hat mir hingegen \label{K_L02916-8v}\edtext{gefallen}{\lemma{\textnormal{\emph{gefallen}}}\Cendnote{\textnormal{Das Stück\pwindex{Star. Ein Wiener Stueck in vier Akten@\emph{Der Star. Ein Wiener Stück in vier Akten}|pwkv} feierte am 25. 5. 1900 am Berlin\oindex{Berlin@\textbf{Berlin}, \emph{P.PPLC}|pwk}er \emph{Lessing-Theater}\orgindex{Lessing-Theater@Lessing-Theater|pwk} Premiere.}}}\label{K_L02916-8}. Dieſer widerliche Burſch\pwindex{Bahr, Hermann 19.07.1863 – 15.01.1934@\textsc{Bahr, Hermann} (19.07.1863 – 15.01.1934), \emph{Schriftsteller/Schriftstellerin, Kritiker/Kritikerin}|pwv} hat doch – leider! – Humor und
               Talent.\pend
           
\pstart
           Bitte, \label{K_L02916-9v}\edtext{lies’}{\lemma{\textnormal{\emph{lies’}}}\Cendnote{\textnormal{Schnitzler las den Roman\pwindex{Familie von Barchwitz. Roman@\emph{Die Familie von Barchwitz. Roman}|pwkv} (vgl. A. S.: \emph{Lektüren}, deutschsprachige Literatur).}}}\label{K_L02916-9}, wenn Du es noch
               nicht kennſt, »Die Familie von \textsc{Barchwitz}\pwindex{Familie von Barchwitz. Roman@\emph{Die Familie von Barchwitz. Roman}|pw}« von \textsc{Hans von Kahlenberg\pwindex{Kessler, Helene 23.02.1870 – 08.08.1957@\textsc{Keßler, Helene} (23.02.1870 – 08.08.1957), \emph{Schriftsteller/Schriftstellerin}|pwv}}. Seit Langem hat mich kein Roman ſo intereſſirt. \strikeout{V\textcolor{gray}{erg}}{ }Verfaſſerin\pwindex{Kessler, Helene 23.02.1870 – 08.08.1957@\textsc{Keßler, Helene} (23.02.1870 – 08.08.1957), \emph{Schriftsteller/Schriftstellerin}|pwv} iſt ein nicht
               mehr {\pb}ganz \strikeout{\textcolor{gray}{hu}} junges, aber \strikeout{\textcolor{gray}{r}} noch \strikeout{recht} recht hübſches Mädchen\pwindex{Kessler, Helene 23.02.1870 – 08.08.1957@\textsc{Keßler, Helene} (23.02.1870 – 08.08.1957), \emph{Schriftsteller/Schriftstellerin}|pwv}, ein Fräulein von \textsc{Montbart\pwindex{Kessler, Helene 23.02.1870 – 08.08.1957@\textsc{Keßler, Helene} (23.02.1870 – 08.08.1957), \emph{Schriftsteller/Schriftstellerin}|pwv}}, Offizier\pwindex{Monbart, Erich von 1836 – 1907@\textsc{Monbart, Erich von} (1836 – 1907), \emph{Offizier/Offizierin, Oberleutnant/Oberleutnantin}|pwv}s-Tochter.\pend
           
\pstart
           Was macht \textsc{Richard\pwindex{Beer-Hofmann, Richard 1866-07-11 – 1945-09-26@\textsc{Beer-Hofmann, Richard} (1866-07-11 – 1945-09-26), \emph{Schriftsteller/Schriftstellerin}|pw}}?\pend
           
\pstart
           Bitte, ſchreib’ mir bald!\pend
           
\pstart
           Viele treue Grüße! {\\[\baselineskip]}Dein {\\[\baselineskip]}\spacefill\mbox{Paul Goldmann}\pend
           \leftskip=0em{}
\pstart
           \noindent{}Auch \textsc{Ludassy\pwindex{Gans-Ludassy, Julius von 13.04.1858 – 30.09.1922@\textsc{Gans-Ludassy, Julius von} (13.04.1858 – 30.09.1922), \emph{Schriftsteller/Schriftstellerin, Journalist/Journalistin, Herausgeber/Herausgeberin}|pw}} benimmt ſich abſcheulich hier und macht ſich aus dem \label{K_L02916-10v}\edtext{Verbot ſeines ſchlechten Stück\pwindex{letzte Knopf. Volksstueck in drei Aufzuegen@\emph{Der letzte Knopf. Volksstück in drei Aufzügen}|pwv}es}{\lemma{\textnormal{\emph{Verbot … Stückes}}}\Cendnote{\textnormal{Julius von Gans-Ludassys\pwindex{Gans-Ludassy, Julius von 13.04.1858 – 30.09.1922@\textsc{Gans-Ludassy, Julius von} (13.04.1858 – 30.09.1922), \emph{Schriftsteller/Schriftstellerin, Journalist/Journalistin, Herausgeber/Herausgeberin}|pwk}{ }\emph{Der letzte Knopf}\pwindex{letzte Knopf. Volksstueck in drei Aufzuegen@\emph{Der letzte Knopf. Volksstück in drei Aufzügen}|pwk} war am 8. 4. 1900 am \emph{Volkstheater}\orgindex{Volkstheater@Volkstheater|pwk}
                     uraufgeführt worden. Das Stück\pwindex{letzte Knopf. Volksstueck in drei Aufzuegen@\emph{Der letzte Knopf. Volksstück in drei Aufzügen}|pwkv}, das für einen Skandal sorgte, sollte auch in Berlin\oindex{Berlin@\textbf{Berlin}, \emph{P.PPLC}|pwk} aufgeführt werden. Ludwig Fulda\pwindex{Fulda, Ludwig 15.07.1862 – 30.03.1939@\textsc{Fulda, Ludwig} (15.07.1862 – 30.03.1939), \emph{Schriftsteller/Schriftstellerin, Übersetzer/Übersetzerin}|pwk}, der als Präsident der \emph{Freien Bühne}\orgindex{Freie Buehne@Freie Bühne|pwk} das von der Zensur verbotene Stück\pwindex{letzte Knopf. Volksstueck in drei Aufzuegen@\emph{Der letzte Knopf. Volksstück in drei Aufzügen}|pwkv} annahm, musste von
                     seiner Funktion zurücktreten. Schließlich wurde es vor einem geladenen Publikum
                     am 30. 5. 1900 bei einer Matinée des \emph{Deutschen Theaters}\orgindex{Deutsches Theater Berlin@Deutsches Theater Berlin|pwk} aufgeführt.}}}\label{K_L02916-10} eine
                  unerträgliche Reklame.\pend
           \selectlanguage{ngerman}\endnumbering\briefempfaengerindex{Schnitzler, Arthur@\textsc{Schnitzler, Arthur}!zzzGoldmann, Paul@\emph{von Paul Goldmann}!1900-05-271@{27. 5. {[}1900{]}}|)be}\mylabel{L02916h}  \normalsize

\doendnotes{C}
\bigskip
\vfill

\clearpage

\footnotesize

\lohead{\textsc{register}}

% Definiere theindex-Environment komplett neu ohne reledmac
\makeatletter
\renewenvironment{theindex}{%
  \section*{\indexname}%
  \setlength{\parindent}{0pt}%
  \setlength{\parskip}{0pt plus 0.3pt}%
  \let\item\@idxitem
}{%
  \clearpage
}
\makeatother

\IfFileExists{\jobname-pw.ind}{\input{\jobname-pw.ind}}{}

\end{document}

      