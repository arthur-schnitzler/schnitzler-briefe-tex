%% latex-korrekturansicht-vorspann.tex
%% Vorspann für die Korrekturansicht.
%% Lädt die gemeinsame Datei latex-vorspann.tex mit gesetztem Schalter.

\newif\ifkorrekturansicht
\korrekturansichttrue

\input{../tex-inputs/latex-vorspann}


\section[Arthur Schnitzler an Richard Beer-Hofmann, 6. 3. 1910]{L01917 Arthur Schnitzler an Richard Beer-Hofmann, 6. 3. 1910}
\nopagebreak\mylabel{L01917v}
\rehead{ }\normalsize\beginnumbering\briefempfaengerindex{Beer-Hofmann, Richard@\textsc{Beer-Hofmann, Richard}!zzzSchnitzler, Arthur@\emph{von Arthur Schnitzler}!1910-03-062@{6. 3. 1910}|(be}
\toendnotes[C]{\smallbreak\pagebreak[2]}\Standort{YCGL, MSS 31.}
\physDesc{Brief, 1 Blatt, 2 Seiten, Umschlag, 404 Zeichen
\newline{}Handschrift: Bleistift, lateinische Kurrent
\newline{}Versand: ohne postalischen Übermittlungsvermerk }
\buchAbdrucke{\weitereDrucke{Arthur Schnitzler, Richard Beer-Hofmann: \emph{Briefwechsel 1891–1931}. Wien, Zürich: \emph{Europaverlag} 1992, S. 207.} }\toendnotes[C]{\smallbreak}\pstart{}{\pb}\textcolor{gray}{\textbf{Dr. Arthur Schnitzler}}\pend{}\pstart{}\textcolor{gray}{\textbf{Wien XVIII. Spoettelgasse 7\oindex{Edmund-Weiss-Gasse 7@\textbf{Edmund-Weiß-Gasse 7}, \emph{Wohngebäude (K.WHS)}|pw}.}}\pend{}{\bigskip}\pstart{}{\pb}Herrn Dr. Richard Beer-Hofmann\pend{}\pstart{}Wien\oindex{Wien@\textbf{Wien}, \emph{A.ADM2}|pw}\pend{}{\bigskip}\vspace{1em}
\pstart
           {\pb}\textcolor{gray}{\textbf{Dr. Arthur Schnitzler}}\hfill 6. 3. 10\pend
           
\pstart
           \textcolor{gray}{\textbf{Wien XVIII. Spoettelgasse 7\oindex{Edmund-Weiss-Gasse 7@\textbf{Edmund-Weiß-Gasse 7}, \emph{Wohngebäude (K.WHS)}|pw}.}}\pend
           
\pstart{}lieber Richard,\pend\vspace{0.5em}
\pstart
           wir sind, wie Sie wis{[}s{]}en, thatsächlich geladen\pwindex{Trebitsch, Siegfried 22.12.1868 – 03.06.1956@\textsc{Trebitsch, Siegfried} (22.12.1868 – 03.06.1956), \emph{Schriftsteller/Schriftstellerin, Übersetzer/Übersetzerin}|pwv}. Und we{\geminationn}
               Sie glauben es thun zu können so telephoniren Sie Hans\pwindex{Schlesinger, Hans Bernhard 20.07.1875 – 13.3.1932@\textsc{Schlesinger, Hans Bernhard} (20.07.1875 – 13.3.1932), \emph{Maler/Malerin}|pw} dem Unglaublichen bei dieser Gelegenheit, daß wir überhaupt keine Lust
               haben, neue Leute\pwindex{Barjanski, Alexander 16.12.1883 – 06.01.1961@\textsc{Barjanski, Alexander} (16.12.1883 – 06.01.1961), \emph{Cellist/Cellistin}|pwv} kennen zu
               lernen, deren Namen {\pb}wir noch dazu niemals vernommen
               haben.\pend
           
\pstart
           Es thut mir leid, daß Sie unseretwegen telephonisch incommodirt werden.\pend
           
\pstart
           Herzlichst{\\[\baselineskip]}Ihr{\\[\baselineskip]}\spacefill\mbox{A.}\pend
           \leftskip=0em{}\selectlanguage{ngerman}\endnumbering\briefempfaengerindex{Beer-Hofmann, Richard@\textsc{Beer-Hofmann, Richard}!zzzSchnitzler, Arthur@\emph{von Arthur Schnitzler}!1910-03-062@{6. 3. 1910}|)be}\mylabel{L01917h}  \normalsize

\doendnotes{C}
\bigskip
\vfill

\clearpage

\footnotesize

\lohead{\textsc{register}}

% Definiere theindex-Environment komplett neu ohne reledmac
\makeatletter
\renewenvironment{theindex}{%
  \section*{\indexname}%
  \setlength{\parindent}{0pt}%
  \setlength{\parskip}{0pt plus 0.3pt}%
  \let\item\@idxitem
}{%
  \clearpage
}
\makeatother

\IfFileExists{\jobname-pw.ind}{\input{\jobname-pw.ind}}{}

\end{document}

      