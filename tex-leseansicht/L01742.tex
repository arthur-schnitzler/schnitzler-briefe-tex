%% latex-leseansicht-vorspann.tex
%% Vorspann für die Leseansicht.
%% Lädt die gemeinsame Datei latex-vorspann.tex mit nicht gesetztem Schalter.

\newif\ifkorrekturansicht
\korrekturansichtfalse

\input{../tex-inputs/latex-vorspann}


         
         \newcommand{\erwaehntePersonen}{Personen: Anna Bahr-Mildenburg, Max Reinhardt, Helene Ritscher, Olga Schnitzler, Richard Vallentin}
         \newcommand{\erwaehnteInstitutionen}{}
         \newcommand{\erwaehnteOrte}{Orte: Berlin, Hebbel-Theater, Wien}
         \newcommand{\erwaehnteWerke}{Werke: Der Schleier der Beatrice. Schauspiel in fünf Akten}
               \section[Hermann Bahr an Arthur Schnitzler, 18. 12. 1907]{ Hermann Bahr an Arthur Schnitzler, 18. 12. 1907}\nopagebreak\mylabel{v}\rehead{ }\begin{ledgroupsized}[t]{13cm}\normalsize\beginnumbering \toendnotes[C]{\smallbreak\pagebreak[2]} \Standort{CUL, Schnitzler, B 5b.}
\physDesc{Brief, 1 Blatt, 2 Seiten
\newline{}Handschrift: blaue Tinte, deutsche Kurrent
\newline{}Schnitzler: mit Bleistift beschriftet: »Bahr« \newline{}Ordnung: mit Bleistift von unbekannter Hand
                           nummeriert: »152« }\buchAbdrucke{\weitereDrucke{Hermann Bahr, Arthur Schnitzler: \emph{Briefwechsel, Aufzeichnungen, Dokumente (1891–1931)}. Hg. Kurt Ifkovits und Martin Anton Müller. Göttingen: \emph{Wallstein} 2018, S. 399.} }\toendnotes[C]{\smallbreak}\pstart
           \raggedleft{}{\pb}18. 12. 07\pend
           \pstart\center{}Lieber Arthur!\pend\pstart
           Vertrauen gegen Vertrauen, da ich Dir doch nur helfe, wenn ich ganz rückhaltlos
               aufrichtig bin. Also: Reinhardt\pwindex{Reinhardt, Max 09.09.1873 – 30.10.1943@\textsc{Reinhardt, Max} (09.09.1873 – 30.10.1943), \emph{Theaterleiter, Regisseur, Schauspieler}|pw} würde, wenn man
               ihm ſagt, daß Du ſonſt mit Vallentin\pwindex{Vallentin, Richard 03.02.1874 – 14.01.1908@\textsc{Vallentin, Richard} (03.02.1874 – 14.01.1908), \emph{Regisseur, Schauspieler}|pw} abſchließen
               willſt, ſicher die Beatrice\pwindex{Schnitzler, Arthur 15.05.1862 – 21.10.1931@\textsc{Schnitzler, Arthur} (15.05.1862 – 21.10.1931), \emph{Schriftsteller, Mediziner}!Schleier der Beatrice. Schauspiel in fuenf Akten1900-12-01@\strich\emph{Der Schleier der Beatrice. Schauspiel in fünf Akten} {[}1900-12-01{]}|pw} annehmen, damit nur der
               andere ſie nicht habe, dann aber liegen laſſen, ſich mahnen laſſen, Dich verzweifeln
               laſſen, endlich, gedrängt, bedroht, ſie irgendwie, ohne ſich ſelbſt darum zu kümmern,
               von irgendwem ſchnell erledigen laſſen, weil er ſelbſt kein eigentliches Verhältnis
               zu dieſem Stücke hat, und weil es ſchließlich ſeine beſte Eigenschaft iſt, daß alle
               ſeine guten Eigenſchaft{[}en{]} verſagen, wo er {\pb}nicht durch ein ſtarkes inneres Verhältnis gehalten
               wird. Ich würde Dir alſo dringend zu Vallentin\pwindex{Vallentin, Richard 03.02.1874 – 14.01.1908@\textsc{Vallentin, Richard} (03.02.1874 – 14.01.1908), \emph{Regisseur, Schauspieler}|pw}
               raten und glaube, daß die Ritſcher\pwindex{Ritscher, Helene 1888 – 1964-11-27@\textsc{Ritscher, Helene} (1888 – 1964-11-27), \emph{Schauspielerin}|pw}, wenn ſie im
               Sommer bei der Mildenburg\pwindex{Bahr-Mildenburg, Anna 29.11.1872 – 27.01.1947@\textsc{Bahr-Mildenburg, Anna} (29.11.1872 – 27.01.1947), \emph{Sängerin}|pw} und gelegentlich auch
               mit mir die Rolle lernt, ſchon was recht Merkwürdiges machen könnte.\pend
           \pstart
           Ich weiß noch nicht, wann ich wieder nach Berlin\oindex{Berlin@\textbf{Berlin}|pw}
               muß, möchte aber jedenfalls vorher zu Euch, ſo bald Deine Frau\pwindex{Schnitzler, Olga 17.01.1882 – 13.01.1970@\textsc{Schnitzler, Olga} (17.01.1882 – 13.01.1970), \emph{Schauspielerin, Sängerin}|pwv}{ }ſo weit iſt, über deren Erkrankung ich,
               ahnungslos, ſehr erschrack, weshalb ich mich ihrer Geneſung gern bald in der Nähe
               erfreuen möchte.\pend
           \pstart
           Herzlichſt{\\[\baselineskip]}Dein alter{\\[\baselineskip]}\spacefill\mbox{Hermann}\pend
           \leftskip=0em{}\pstart
           \noindent{}\label{T_L01742_1v}\edtext{\uline{Frage 1}: Reinhardt\pwindex{Reinhardt, Max 09.09.1873 – 30.10.1943@\textsc{Reinhardt, Max} (09.09.1873 – 30.10.1943), \emph{Theaterleiter, Regisseur, Schauspieler}|pw} wird B.\pwindex{Schnitzler, Arthur 15.05.1862 – 21.10.1931@\textsc{Schnitzler, Arthur} (15.05.1862 – 21.10.1931), \emph{Schriftsteller, Mediziner}!Schleier der Beatrice. Schauspiel in fuenf Akten1900-12-01@\strich\emph{Der Schleier der Beatrice. Schauspiel in fünf Akten} {[}1900-12-01{]}|pw} nehmen, wenn Du mit
                     Vallentin\pwindex{Vallentin, Richard 03.02.1874 – 14.01.1908@\textsc{Vallentin, Richard} (03.02.1874 – 14.01.1908), \emph{Regisseur, Schauspieler}|pw} drohſt. \uline{Frage 2}: Ich halte Hebbeltheater\oindex{Hebbel-Theater@\textbf{Hebbel-Theater}|pw} für
                  praktiſcher. \uline{Frage 3}: Reinhardt\pwindex{Reinhardt, Max 09.09.1873 – 30.10.1943@\textsc{Reinhardt, Max} (09.09.1873 – 30.10.1943), \emph{Theaterleiter, Regisseur, Schauspieler}|pw} müßte man eine Frist von 1\textcolor{gray}{4}
                  Tagen zur Entſcheidung geben.}{\lemma{\textnormal{\emph{Frage … geben.}}}\Cendnote{\textnormal{quer zum
                     Text neben der Grußformel}}}\label{T_L01742_1h}\pend
           
         
         \endnumbering\mylabel{h}\end{ledgroupsized}  \newcommand{\dateiname}{L01742}\newcommand{\titel}{Hermann Bahr an Arthur Schnitzler, 18. 12. 1907}\newcommand{\editorInnen}{ Kurt Ifkovits,  Martin Anton Müller}%% latex-leseansicht-abspann.tex
%% Abspann für die Leseansicht.
%% Der Schalter \ifkorrekturansicht ist bereits durch den Vorspann gesetzt.

%% latex-abspann.tex
%% Gemeinsamer Abspann für Korrekturansicht und Leseansicht.
%% Setzt den Schalter \ifkorrekturansicht voraus (gesetzt in den
%% einbindenden Dateien latex-korrekturansicht-abspann.tex bzw.
%% latex-leseansicht-abspann.tex).
%% ---------------------------------------------------------------

\normalsize

% Das esempio-Environment wird nur in der Leseansicht benötigt
\ifkorrekturansicht\else
\newenvironment{esempio}[3]%
{
    \vspace{1.5ex}
    \rlap{\underline{#1}}
    \par
    \setlength{\parindent}{0cm}
    \nopagebreak
    \leftskip=#2cm
    \rightskip=#3cm
}
{
    \par
}
\fi

\doendnotes{C}
\bigskip
\vfill

\clearpage

\footnotesize

\ifkorrekturansicht
  \lohead{\textsc{register}}
\fi

% theindex-Environment neu definieren ohne reledmac
\makeatletter
\renewenvironment{theindex}{%
  \ifkorrekturansicht
    \section*{\indexname}%
  \else
    \subsubsection*{Index der erwähnten Entitäten}%
  \fi
  \setlength{\parindent}{0pt}%
  \setlength{\parskip}{0pt plus 0.3pt}%
  \let\item\@idxitem
}{%
  \ifkorrekturansicht\clearpage\fi
}
\makeatother

\IfFileExists{\jobname-pw.ind}{\input{\jobname-pw.ind}}{}

% Quellenangabe nur in der Leseansicht
\ifkorrekturansicht\else
% Fallback-Definitionen, falls die .tex-Datei \titel etc. nicht gesetzt hat
\providecommand{\titel}{}
\providecommand{\editorInnen}{}
\providecommand{\dateiname}{\jobname}

\vspace{3cm}

\vfill

\footnotesize
\textsc{Quelle}: \titel. Herausgegeben von {\editorInnen}. In: \emph{Arthur Schnitzler: Briefwechsel mit Autorinnen und Autoren}.
 Digitale Edition, https://schnitzler-briefe.acdh.oeaw.ac.at/{\dateiname}.html (Stand \today)
\fi

\end{document}


      