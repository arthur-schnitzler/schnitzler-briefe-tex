%% latex-leseansicht-vorspann.tex
%% Vorspann für die Leseansicht.
%% Lädt die gemeinsame Datei latex-vorspann.tex mit nicht gesetztem Schalter.

\newif\ifkorrekturansicht
\korrekturansichtfalse

\input{../tex-inputs/latex-vorspann}


         
         \renewcommand{\erwaehntePersonen}{Personen: Olga Schnitzler}
         \renewcommand{\erwaehnteOrte}{Orte: Berlin, Deutsches Theater Berlin, Edmund-Weiß-Gasse 7, Semmering, Wien}
         \renewcommand{\erwaehnteWerke}{Werke: Der einsame Weg. Schauspiel in fünf Akten}
               \section[ Paul Goldmann an Arthur Schnitzler, 18. 1. 1904]{ Paul Goldmann an Arthur Schnitzler, 18. 1. 1904}\nopagebreak\mylabel{v}\rehead{ }\begin{ledgroupsized}[t]{13cm}\normalsize\beginnumbering \toendnotes[C]{\smallbreak\pagebreak[2]} \Standort{DLA, A:Schnitzler, HS.NZ85.1.3174.}
\physDesc{Postkarte, 384 Zeichen
\newline{}Handschrift: 1) blaue Tinte, deutsche Kurrent\hspace{1em}2) blaue Tinte, lateinische Kurrent (\noindent{}Adresse)\hspace{1em}
\newline{}Versand: Stempel: »\nobreak{}\oindex{Berlin@\textbf{Berlin}|pwk}Berlin, W. 10 b, 16. 1. 04, 12–1N.\nobreak{}«. Stempel: »\nobreak{}\oindex{XXXX Ortsangabe fehlt|pwk}18/1 Wien 110, 18. 1. 04, 8. V, Bestellt\nobreak{}«.  
\newline{}Schnitzler: mit Bleistift das Jahr »904« vermerkt }\toendnotes[C]{\smallbreak}\pstart{}{\pb}Herrn\pend{}\pstart{}Dr. Arthur Schnitzler\pend{}\pstart{}Wien\oindex{Wien@\textbf{Wien}|pw}\pend{}\pstart{}XVIII. Spöttelgaſse 7\oindex{Edmund-Weiss-Gasse 7@\textbf{Edmund-Weiß-Gasse 7}|pw}.\pend{}{\bigskip}\pstart
           \noindent{}{\pb}Berlin\oindex{Berlin@\textbf{Berlin}|pw}, 16. Januar. Herzlichſten Dank Dir, mein lieber
                  Freund, und Deiner Frau\pwindex{Schnitzler, Olga 17.01.1882 – 13.01.1970@\textsc{Schnitzler, Olga} (17.01.1882 – 13.01.1970), \emph{Schauspielerin, Sängerin}|pwv} für Eure Grüße vom \label{K_L03439-1v}\edtext{\textsc{Semmering\oindex{Semmering@\textbf{Semmering}|pw}}}{\lemma{\textnormal{\emph{Semmering}}}\Cendnote{\textnormal{Arthur\pwindex{Schnitzler, Arthur 15.05.1862 – 21.10.1931@\textsc{Schnitzler, Arthur} (15.05.1862 – 21.10.1931), \emph{Schriftsteller, Mediziner}|pwk} und Olga Schnitzler\pwindex{Schnitzler, Olga 17.01.1882 – 13.01.1970@\textsc{Schnitzler, Olga} (17.01.1882 – 13.01.1970), \emph{Schauspielerin, Sängerin}|pwk} waren
                  zwischen 9. 1. 1904
                  und 14. 1. 1904 am
                     Semmering\oindex{Semmering@\textbf{Semmering}|pwk} gewesen.}}}\label{K_L03439-1h}. Hoffentlich
               bleibt es bei Eurer \label{K_L03439-2v}\edtext{Berlin\oindex{Berlin@\textbf{Berlin}|pw}er Reiſe im Februar}{\lemma{\textnormal{\emph{Berliner … Februar}}}\Cendnote{\textnormal{Für die Uraufführung von \emph{Der einsame Weg}\pwindex{Schnitzler, Arthur 15.05.1862 – 21.10.1931@\textsc{Schnitzler, Arthur} (15.05.1862 – 21.10.1931), \emph{Schriftsteller, Mediziner}!einsame Weg. Schauspiel in fuenf Akten1904@\strich\emph{Der einsame Weg. Schauspiel in fünf Akten} {[}1904{]}|pwk} am 13. 2. 1904 im Deutschen Theater Berlin\oindex{Deutsches Theater Berlin@\textbf{Deutsches Theater Berlin}|pwk} war Schnitzler\pwindex{Schnitzler, Arthur 15.05.1862 – 21.10.1931@\textsc{Schnitzler, Arthur} (15.05.1862 – 21.10.1931), \emph{Schriftsteller, Mediziner}|pwk} von 5. 2. 1904 bis 17. 2. 1904 in Berlin\oindex{Berlin@\textbf{Berlin}|pwk}. Am 8. 2. 1904 kam Olga Schnitzler\pwindex{Schnitzler, Olga 17.01.1882 – 13.01.1970@\textsc{Schnitzler, Olga} (17.01.1882 – 13.01.1970), \emph{Schauspielerin, Sängerin}|pwk} nach. Goldmann\pwindex{Goldmann, Paul 31.01.1865 – 25.09.1935@\textsc{Goldmann, Paul} (31.01.1865 – 25.09.1935), \emph{Schriftsteller, Journalist}|pwk} und Schnitzler\pwindex{Schnitzler, Arthur 15.05.1862 – 21.10.1931@\textsc{Schnitzler, Arthur} (15.05.1862 – 21.10.1931), \emph{Schriftsteller, Mediziner}|pwk} trafen sich nachweislich am 6. 2. 1904, 10. 2. 1904, 11. 2. 1904 und am 16. 2. 1904.}}}\label{K_L03439-2h}.
               Ich freue mich ſehr, Euch hier zu ſehen. Entſchuldige, daß ich Deinen letzten Brief
               noch nicht beantwortet habe. Ich erſaufe in Arbeit.\pend
           \pstart
           Herzlichſt Dein getreuer {\\[\baselineskip]}\spacefill\mbox{Paul Goldmann}\pend
           \leftskip=0em{}
         
         \endnumbering\mylabel{h}\end{ledgroupsized}  \newcommand{\dateiname}{L03439}\newcommand{\titel}{Paul Goldmann an Arthur Schnitzler, 18. 1. 1904}\newcommand{\editorInnen}{Martin Anton Müller und Laura Untner}%% latex-leseansicht-abspann.tex
%% Abspann für die Leseansicht.
%% Der Schalter \ifkorrekturansicht ist bereits durch den Vorspann gesetzt.

%% latex-abspann.tex
%% Gemeinsamer Abspann für Korrekturansicht und Leseansicht.
%% Setzt den Schalter \ifkorrekturansicht voraus (gesetzt in den
%% einbindenden Dateien latex-korrekturansicht-abspann.tex bzw.
%% latex-leseansicht-abspann.tex).
%% ---------------------------------------------------------------

\normalsize

% Das esempio-Environment wird nur in der Leseansicht benötigt
\ifkorrekturansicht\else
\newenvironment{esempio}[3]%
{
    \vspace{1.5ex}
    \rlap{\underline{#1}}
    \par
    \setlength{\parindent}{0cm}
    \nopagebreak
    \leftskip=#2cm
    \rightskip=#3cm
}
{
    \par
}
\fi

\doendnotes{C}
\bigskip
\vfill

\clearpage

\footnotesize

\ifkorrekturansicht
  \lohead{\textsc{register}}
\fi

% theindex-Environment neu definieren ohne reledmac
\makeatletter
\renewenvironment{theindex}{%
  \ifkorrekturansicht
    \section*{\indexname}%
  \else
    \subsubsection*{Index der erwähnten Entitäten}%
  \fi
  \setlength{\parindent}{0pt}%
  \setlength{\parskip}{0pt plus 0.3pt}%
  \let\item\@idxitem
}{%
  \ifkorrekturansicht\clearpage\fi
}
\makeatother

\IfFileExists{\jobname-pw.ind}{\input{\jobname-pw.ind}}{}

% Quellenangabe nur in der Leseansicht
\ifkorrekturansicht\else
% Fallback-Definitionen, falls die .tex-Datei \titel etc. nicht gesetzt hat
\providecommand{\titel}{}
\providecommand{\editorInnen}{}
\providecommand{\dateiname}{\jobname}

\vspace{3cm}

\vfill

\footnotesize
\textsc{Quelle}: \titel. Herausgegeben von {\editorInnen}. In: \emph{Arthur Schnitzler: Briefwechsel mit Autorinnen und Autoren}.
 Digitale Edition, https://schnitzler-briefe.acdh.oeaw.ac.at/{\dateiname}.html (Stand \today)
\fi

\end{document}


      