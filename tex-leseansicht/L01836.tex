%% latex-korrekturansicht-vorspann.tex
%% Vorspann für die Korrekturansicht.
%% Lädt die gemeinsame Datei latex-vorspann.tex mit gesetztem Schalter.

\newif\ifkorrekturansicht
\korrekturansichttrue

\input{../tex-inputs/latex-vorspann}


\section[Arthur Schnitzler an Albert Ehrenstein, 29. 3. 1909]{L01836 Arthur Schnitzler an Albert Ehrenstein, 29. 3. 1909}
\nopagebreak\mylabel{L01836v}
\rehead{ }\normalsize\beginnumbering\briefempfaengerindex{Ehrenstein, Albert@\textsc{Ehrenstein, Albert}!zzzSchnitzler, Arthur@\emph{von Arthur Schnitzler}!1909-03-291@{29. 3. 1909}|(be}
\toendnotes[C]{\smallbreak\pagebreak[2]}\Standort{Jerusalem, The National Library of Israel, ARC. Ms. Var. 306 1 118.}
\physDesc{Briefkarte, 194 Zeichen
\newline{}Handschrift: schwarze Tinte, deutsche Kurrent}\toendnotes[C]{\smallbreak}
\pstart
           \raggedleft{}{\pb}29. 3. 09\pend
           
\pstart
           \textcolor{gray}{\textbf{Dr. Arthur Schnitzler}}{\\}\textcolor{gray}{\textbf{Wien XVIII. Spoettelgasse 7\oindex{Edmund-Weiss-Gasse 7@\textbf{Edmund-Weiß-Gasse 7}, \emph{Wohngebäude (K.WHS)}|pw}.}}\pend
           
\pstart{}lieber Herr Ehrenſtein,\pend\vspace{0.5em}
\pstart
           von Ihren rundſchauerlichen\orgindex{Oesterreichische Rundschau@Österreichische Rundschau|pwv}{ }Schickſalen zu erfahren würd mich intereſſieren u
               ich erwarte Sie we{\geminationn}s Ihnen paſſt am Mittwoch um 7 Uhr
               Abends.\pend
           \pstart beſtens grüßend\hspace*{1.5em}Ihr \spacefill\mbox{A. S.}\pend{}\selectlanguage{ngerman}\endnumbering\briefempfaengerindex{Ehrenstein, Albert@\textsc{Ehrenstein, Albert}!zzzSchnitzler, Arthur@\emph{von Arthur Schnitzler}!1909-03-291@{29. 3. 1909}|)be}\mylabel{L01836h}  \normalsize

\doendnotes{C}
\bigskip
\vfill

\clearpage

\footnotesize

\lohead{\textsc{register}}

% Definiere theindex-Environment komplett neu ohne reledmac
\makeatletter
\renewenvironment{theindex}{%
  \section*{\indexname}%
  \setlength{\parindent}{0pt}%
  \setlength{\parskip}{0pt plus 0.3pt}%
  \let\item\@idxitem
}{%
  \clearpage
}
\makeatother

\IfFileExists{\jobname-pw.ind}{\input{\jobname-pw.ind}}{}

\end{document}

      