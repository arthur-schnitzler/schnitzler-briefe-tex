%% latex-leseansicht-vorspann.tex
%% Vorspann für die Leseansicht.
%% Lädt die gemeinsame Datei latex-vorspann.tex mit nicht gesetztem Schalter.

\newif\ifkorrekturansicht
\korrekturansichtfalse

\input{../tex-inputs/latex-vorspann}


\section[Sigmund Freud an Arthur Schnitzler, 7. 5. 1928]{L03817 Sigmund Freud an Arthur Schnitzler, 7. 5. 1928}
\nopagebreak\mylabel{L03817v}
\rehead{ }\normalsize\beginnumbering\briefempfaengerindex{Schnitzler, Arthur@\textsc{Schnitzler, Arthur}!zzzFreud, Sigmund@\emph{von Sigmund Freud}!1928-05-071@{7. 5. 1928}|(be}
\toendnotes[C]{\smallbreak\pagebreak[2]}
\correspDesc{Versand  durch Sigmund Freud am 7. 5. 1928 in Wien
\newline{}Erhalt  durch Arthur Schnitzler im Zeitraum [7. 5. 1928
                  – 10. 5. 1928?] in Wien}\toendnotes[C]{\smallbreak}
\Standort{CUL, Schnitzler, B 31.}
\physDesc{Kartenbrief, 209 Zeichen
\newline{}Handschrift: schwarze Tinte, deutsche Kurrent
\newline{}Schnitzler: mit rotem Buntstift beschriftet: »Therese\pwindex{Schnitzler, Arthur 15.\,5.\,1862 Wien – 21.\,10.\,1931 ebd.@\textsc{Schnitzler, Arthur} (15.\,5.\,1862 Wien – 21.\,10.\,1931 ebd.), \emph{Schriftsteller, Mediziner}!Therese. Chronik eines Frauenlebens@\strich\emph{Therese. Chronik eines Frauenlebens}|pw}« und eine Unterstreichung }
\buchAbdrucke{\weitereDrucke{1) Sigmund Freud: \emph{Briefe an Arthur Schnitzler.}Herausgegeben von Henry Schnitzler In: \emph{Neue deutsche Rundschau}, Jg. 66 (Januar 1955) Nr. 1, S. 100.} \weitereDrucke{2) Sigmund Freud: \emph{Sigmund Freud Edition. Digitale historisch-kritische
                        Gesamtausgabe}. Herausgegeben von Christine Diercks, Arkadi Blatow und Elisabeth Skale. (2014–2025) \url{https://www.freudedition.net/briefe/freud-sigmund/schnitzler-arthur/1928/05/07}.} }\toendnotes[C]{\smallbreak}
\pstart
           \raggedleft{}{\pb}7. 5. 1928\pend
           
\pstart
           \textcolor{gray}{\textbf{PROF. D\textsuperscript{R.} FREUD }}\hfill \textcolor{gray}{\textbf{WIEN, IX., BERGGASSE 19\oindex{Wien@\textbf{Wien}!IX., Alsergrund@\textbf{IX., Alsergrund}!Berggasse 19@\textbf{Berggasse 19}, \emph{Wohngebäude}|pw}.}}\pend
           
\pstart{}Verehrter Herr Kollege\pend\vspace{0.5em}
\pstart
           Schön, daß Sie mich auch diesmal mit \label{K_L03817-1v}\edtext{einer Zuſendung\pwindex{Schnitzler, Arthur 15.\,5.\,1862 Wien – 21.\,10.\,1931 ebd.@\textsc{Schnitzler, Arthur} (15.\,5.\,1862 Wien – 21.\,10.\,1931 ebd.), \emph{Schriftsteller, Mediziner}!Therese. Chronik eines Frauenlebens@\strich\emph{Therese. Chronik eines Frauenlebens}|pwv}}{\lemma{\textnormal{\emph{einer Zusendung}}}\Cendnote{\textnormal{Dass es sich um das Ende März 1928 erschienene Werk \emph{Therese. Chronik
                     eines Frauenlebens}\pwindex{Schnitzler, Arthur 15.\,5.\,1862 Wien – 21.\,10.\,1931 ebd.@\textsc{Schnitzler, Arthur} (15.\,5.\,1862 Wien – 21.\,10.\,1931 ebd.), \emph{Schriftsteller, Mediziner}!Therese. Chronik eines Frauenlebens@\strich\emph{Therese. Chronik eines Frauenlebens}|pwk} handelte, bestätigt Schnitzlers Beschriftung über dem Brief.}}}\label{K_L03817-1} bedacht haben! Aber eine
               »Revanche« dürfte es nicht mehr geben. Ich kann \uline{nicht}
               mehr oder ich habe es ſatt.\pend
           
\pstart
           Herzlich dankend{\\[\baselineskip]}\spacefill\mbox{Ihr Freud}\pend
           \leftskip=0em{}\selectlanguage{ngerman}\endnumbering\briefempfaengerindex{Schnitzler, Arthur@\textsc{Schnitzler, Arthur}!zzzFreud, Sigmund@\emph{von Sigmund Freud}!1928-05-071@{7. 5. 1928}|)be}\mylabel{L03817h}
\begin{anhang}
\end{anhang}\newcommand{\dateiname}{L03817}\newcommand{\titel}{Sigmund Freud an Arthur Schnitzler, 7. 5. 1928}\newcommand{\editorInnen}{Selma Jahnke und Martin Anton Müller}%% latex-leseansicht-abspann.tex
%% Abspann für die Leseansicht.
%% Der Schalter \ifkorrekturansicht ist bereits durch den Vorspann gesetzt.

%% latex-abspann.tex
%% Gemeinsamer Abspann für Korrekturansicht und Leseansicht.
%% Setzt den Schalter \ifkorrekturansicht voraus (gesetzt in den
%% einbindenden Dateien latex-korrekturansicht-abspann.tex bzw.
%% latex-leseansicht-abspann.tex).
%% ---------------------------------------------------------------

\normalsize

% Das esempio-Environment wird nur in der Leseansicht benötigt
\ifkorrekturansicht\else
\newenvironment{esempio}[3]%
{
    \vspace{1.5ex}
    \rlap{\underline{#1}}
    \par
    \setlength{\parindent}{0cm}
    \nopagebreak
    \leftskip=#2cm
    \rightskip=#3cm
}
{
    \par
}
\fi

\doendnotes{C}
\bigskip
\vfill

\clearpage

\footnotesize

\ifkorrekturansicht
  \lohead{\textsc{register}}
\fi

% theindex-Environment neu definieren ohne reledmac
\makeatletter
\renewenvironment{theindex}{%
  \ifkorrekturansicht
    \section*{\indexname}%
  \else
    \subsubsection*{Index der erwähnten Entitäten}%
  \fi
  \setlength{\parindent}{0pt}%
  \setlength{\parskip}{0pt plus 0.3pt}%
  \let\item\@idxitem
}{%
  \ifkorrekturansicht\clearpage\fi
}
\makeatother

\IfFileExists{\jobname-pw.ind}{\input{\jobname-pw.ind}}{}

% Quellenangabe nur in der Leseansicht
\ifkorrekturansicht\else
% Fallback-Definitionen, falls die .tex-Datei \titel etc. nicht gesetzt hat
\providecommand{\titel}{}
\providecommand{\editorInnen}{}
\providecommand{\dateiname}{\jobname}

\vspace{3cm}

\vfill

\footnotesize
\textsc{Quelle}: \titel. Herausgegeben von {\editorInnen}. In: \emph{Arthur Schnitzler: Briefwechsel mit Autorinnen und Autoren}.
 Digitale Edition, https://schnitzler-briefe.acdh.oeaw.ac.at/{\dateiname}.html (Stand \today)
\fi

\end{document}


