%% latex-korrekturansicht-vorspann.tex
%% Vorspann für die Korrekturansicht.
%% Lädt die gemeinsame Datei latex-vorspann.tex mit gesetztem Schalter.

\newif\ifkorrekturansicht
\korrekturansichttrue

\input{../tex-inputs/latex-vorspann}


\section[Sigmund Freud an Arthur Schnitzler, 7. 5. 1928]{L03817 Sigmund Freud an Arthur Schnitzler, 7. 5. 1928}
\nopagebreak\mylabel{L03817v}
\rehead{ }\normalsize\beginnumbering\briefempfaengerindex{Schnitzler, Arthur@\textsc{Schnitzler, Arthur}!zzzFreud, Sigmund@\emph{von Sigmund Freud}!1928-05-071@{7. 5. 1928}|(be}
\toendnotes[C]{\smallbreak\pagebreak[2]}\Standort{CUL, Schnitzler, B 31.}
\physDesc{Kartenbrief, 1 Blatt, 1 Seite, 209 Zeichen
\newline{}Handschrift: schwarze Tinte, deutsche Kurrent
\newline{}Schnitzler: 1) mit rotem Buntstift eine Unterstreichung  2) mit rotem Buntstift beschriftet: »Therese«}\toendnotes[C]{\smallbreak}
\pstart
           \raggedleft{}{\pb}
                     7. 5. 1928\pend
           
\pstart
           \textcolor{gray}{\textbf{PROF. D\textsuperscript{R.} FREUD
                  }}\hfill \textcolor{gray}{\textbf{WIEN, IX., BERGGASSE 19\oindex{Berggasse 19@\textbf{Berggasse 19}, \emph{Wohngebäude (K.WHS)}|pw}.}}\pend
           
\pstart{}Verehrter Herr Kollege\pend\vspace{0.5em}
\pstart
           Schön, daß Sie mich auch diesmal mit \label{K_L03817-1v}\edtext{einer Zuſendung\pwindex{Therese. Chronik eines Frauenlebens@\emph{Therese. Chronik eines Frauenlebens}|pwv}}{\lemma{\textnormal{\emph{einer Zuſendung}}}\Cendnote{\textnormal{Dass es sich um das Ende März 1928 erschienene Werk \emph{Therese. Chronik eines Frauenlebens}\pwindex{Therese. Chronik eines Frauenlebens@\emph{Therese. Chronik eines Frauenlebens}|pwk} handelte, bestätigt Schnitzlers Beschriftung über dem Brief.}}}\label{K_L03817-1}
      bedacht haben! Aber eine
      »Revanche« dürfte es nicht
      mehr geben. Ich kann \uline{nicht}
      mehr oder ich habe es ſatt.\pend
           
\pstart
           Herzlich dankend{\\[\baselineskip]}\spacefill\mbox{Ihr Freud}\pend
           \leftskip=0em{}\selectlanguage{ngerman}\endnumbering\briefempfaengerindex{Schnitzler, Arthur@\textsc{Schnitzler, Arthur}!zzzFreud, Sigmund@\emph{von Sigmund Freud}!1928-05-071@{7. 5. 1928}|)be}\mylabel{L03817h}
\begin{anhang}
\end{anhang}\normalsize

\doendnotes{C}
\bigskip
\vfill

\clearpage

\footnotesize

\lohead{\textsc{register}}

% Definiere theindex-Environment komplett neu ohne reledmac
\makeatletter
\renewenvironment{theindex}{%
  \section*{\indexname}%
  \setlength{\parindent}{0pt}%
  \setlength{\parskip}{0pt plus 0.3pt}%
  \let\item\@idxitem
}{%
  \clearpage
}
\makeatother

\IfFileExists{\jobname-pw.ind}{\input{\jobname-pw.ind}}{}

\end{document}

      