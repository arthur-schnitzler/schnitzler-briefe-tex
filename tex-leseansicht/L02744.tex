%% latex-leseansicht-vorspann.tex
%% Vorspann für die Leseansicht.
%% Lädt die gemeinsame Datei latex-vorspann.tex mit nicht gesetztem Schalter.

\newif\ifkorrekturansicht
\korrekturansichtfalse

\input{../tex-inputs/latex-vorspann}


         
         \renewcommand{\erwaehntePersonen}{Personen: Richard Beer-Hofmann, Johann Wolfgang von Goethe, Paul Goldmann, Leopold Sonnemann}
         \renewcommand{\erwaehnteInstitutionen}{Institutionen: Frankfurter Zeitung}
         \renewcommand{\erwaehnteOrte}{Orte: Alpen, Bad Ischl, Bad Tölz, Paris, Salzburg, Wien, rue Feydeau, Österreich}
         \renewcommand{\erwaehnteWerke}{Werke: Faust. Eine Tragödie}
               \section[Paul Goldmann an Arthur Schnitzler, 13. 8. {[}1895{]}]{ Paul Goldmann an Arthur Schnitzler, 13. 8. {[}1895{]}}\nopagebreak\mylabel{v}\rehead{ }\begin{ledgroupsized}[t]{13cm}\normalsize\beginnumbering\briefempfaengerindex{Schnitzler, Arthur@\textsc{Schnitzler, Arthur}!zzzGoldmann, Paul@\emph{von Paul Goldmann}!1895-08-131@{13. 8. {[}1895{]}}|(be} \toendnotes[C]{\smallbreak\pagebreak[2]} \Standort{DLA, A:Schnitzler, HS.NZ85.1.3165.}
\physDesc{Brief, 1 Blatt, 4 Seiten, 1230 Zeichen
\newline{}Handschrift: schwarze Tinte, deutsche Kurrent
\newline{}Schnitzler: mit Bleistift das Jahr »95« vermerkt }\toendnotes[C]{\smallbreak}\pstart
           \noindent{}{\pb}\textcolor{gray}{\textbf{\textbf{Frankfurter Zeitung\orgindex{Frankfurter Zeitung@Frankfurter Zeitung|pw}}}}\pend
           \pstart
           \textcolor{gray}{\textbf{(\begin{otherlanguage}{french}Gazette de Francfort\end{otherlanguage}\orgindex{Frankfurter Zeitung@Frankfurter Zeitung|pw}).}}\hfill \textsc{Toelz\oindex{Bad Toelz@\textbf{Bad Tölz}|pw}}, 13. Auguſt.\pend
           \pstart
           \textcolor{gray}{\textbf{\textbf{\begin{otherlanguage}{french}Fondateur M. L.
                              Sonnemann\pwindex{Sonnemann, Leopold 1831-10-29 – 1909-10-30@\textsc{Sonnemann, Leopold} (1831-10-29 – 1909-10-30), \emph{Journalist, Herausgeber}|pw}\end{otherlanguage}.}}}\pend
           \pstart
           \begin{otherlanguage}{french}\textcolor{gray}{\textbf{Journal politique, financier,}}\end{otherlanguage}\pend
           \pstart
           \begin{otherlanguage}{french}\textcolor{gray}{\textbf{commercial et littéraire.}}\end{otherlanguage}\pend
           \pstart
           \begin{otherlanguage}{french}\textcolor{gray}{\textbf{\textbf{Paraissant trois fois par jour.}}}\end{otherlanguage}\pend
           \pstart
           \begin{otherlanguage}{french}\textcolor{gray}{\textbf{\textbf{Bureau à Paris\oindex{Paris@\textbf{Paris}|pw}}}}\end{otherlanguage}\pend
           \pstart
           \begin{otherlanguage}{french}\textcolor{gray}{\textbf{\textbf{24. Rue Feydeau\oindex{rue Feydeau@\textbf{rue Feydeau}|pw}.}}}\end{otherlanguage}\pend
           \pstart\center{}Mein lieber Freund,\pend\pstart
           Das wäre ſchön, wenn Du ein wenig hieher kommen wollteſt! Freilich, es wäre ein
               wahres Opfer. Denn der Ort\oindex{Bad Toelz@\textbf{Bad Tölz}|pwv}
               bietet nichts. Die Berge ſind nur von fern zu ſehen, und ſelbſt dieſe Fernſichten
               ſind in den öſterreich\oindex{Oesterreich@\textbf{Österreich}|pw}iſchen Alpen\oindex{Alpen@\textbf{Alpen}|pw} ſchöner. Man ißt ſchlecht u. wohnt ohne Comfort. Das
               Bade-Publicum iſt einfach unmöglich. Ich verkehre nur mit Bauern. {\pb}Endlich ich ſelbſt \strikeout{te\textcolor{gray}{ib}} treibe Selbſtpein und brüte Schwermuth. Wenn Du freilich trotz alledem kommen
               wollteſt, ſo wärs ſchön u. dankenswerth im höchſten Grade.\pend
           \pstart
           Nach \textsc{Salzburg\oindex{Salzburg@\textbf{Salzburg}|pw}} werde ich nicht kommen können, der Kur wegen.\pend
           \pstart
           Warum willſt Du auf einmal ſo mit aller Gewalt nach dem Norden?\pend
           \pstart
           Ich gehe ſtundenweit über Land u. leſe den »Fauſt\pwindex{Goethe, Johann Wolfgang von 1749-08-28 – 1832-03-22@\textsc{Goethe, Johann Wolfgang von} (1749-08-28 – 1832-03-22), \emph{Schriftsteller}!Faust. Eine Tragoedie1808@\strich\emph{Faust. Eine Tragödie} {[}1808{]}|pw}«. Wie man in das {\pb}Buch\pwindex{Goethe, Johann Wolfgang von 1749-08-28 – 1832-03-22@\textsc{Goethe, Johann Wolfgang von} (1749-08-28 – 1832-03-22), \emph{Schriftsteller}!Faust. Eine Tragoedie1808@\strich\emph{Faust. Eine Tragödie} {[}1808{]}|pwv} hineingewachſen iſt!
               Jetzt iſt Alles ſo einfach und klar, und das Meiſte hat man ſelbſt erlebt. Aber
               gelungen iſt ihm – dem \textsc{Goethe\pwindex{Goethe, Johann Wolfgang von 1749-08-28 – 1832-03-22@\textsc{Goethe, Johann Wolfgang von} (1749-08-28 – 1832-03-22), \emph{Schriftsteller}|pw}} – doch eigentlich nur das Menſchliche u. das Teufliſche (das iſt das ſelbe;
               denn das Teufliſche iſt nur Ironie über das Menſchliche). Aber wo er vom Himmel
               ſpricht, wird er conventionell oder rhetoriſch{\dotsfive}\pend
           \pstart
           \strikeout{\textcolor{gray}{×}\-\textcolor{gray}{×}} Ich hoffe, Du biſt wohlbehalten \label{K_L02744-1v}\edtext{von Wien\oindex{Wien@\textbf{Wien}|pw}{ }{\pb}zurückgekehrt}{\lemma{\textnormal{\emph{von Wien zurückgekehrt}}}\Cendnote{\textnormal{Zwischen 11. 8. 1895 und 14. 8. 1895 unterbrach Schnitzler\pwindex{Schnitzler, Arthur 15.05.1862 – 21.10.1931@\textsc{Schnitzler, Arthur} (15.05.1862 – 21.10.1931), \emph{Schriftsteller, Mediziner}|pwk} seinen Aufenthalt in Ischl\oindex{Bad Ischl@\textbf{Bad Ischl}|pwk}
                  und kehrte nach Wien\oindex{Wien@\textbf{Wien}|pwk} zurück.}}}\label{K_L02744-1h}. Nun
               ſchreibſt Du mir wohl bald wieder, beſonders: ob u. wann Du kommſt?\pend
           \pstart
           Viele treue Grüße Dir u. \textsc{Richard\pwindex{Beer-Hofmann, Richard 1866-07-11 – 1945-09-26@\textsc{Beer-Hofmann, Richard} (1866-07-11 – 1945-09-26), \emph{Schriftsteller}|pw}}{\\[\baselineskip]}Dein {\\[\baselineskip]}\spacefill\mbox{Paul Goldmann}\pend
           \leftskip=0em{}
         
         \endnumbering\mylabel{h}\end{ledgroupsized}  \newcommand{\dateiname}{L02744}\newcommand{\titel}{Paul Goldmann an Arthur Schnitzler, 13. 8. [1895]}\newcommand{\editorInnen}{Martin Anton Müller und Laura Untner}%% latex-leseansicht-abspann.tex
%% Abspann für die Leseansicht.
%% Der Schalter \ifkorrekturansicht ist bereits durch den Vorspann gesetzt.

%% latex-abspann.tex
%% Gemeinsamer Abspann für Korrekturansicht und Leseansicht.
%% Setzt den Schalter \ifkorrekturansicht voraus (gesetzt in den
%% einbindenden Dateien latex-korrekturansicht-abspann.tex bzw.
%% latex-leseansicht-abspann.tex).
%% ---------------------------------------------------------------

\normalsize

% Das esempio-Environment wird nur in der Leseansicht benötigt
\ifkorrekturansicht\else
\newenvironment{esempio}[3]%
{
    \vspace{1.5ex}
    \rlap{\underline{#1}}
    \par
    \setlength{\parindent}{0cm}
    \nopagebreak
    \leftskip=#2cm
    \rightskip=#3cm
}
{
    \par
}
\fi

\doendnotes{C}
\bigskip
\vfill

\clearpage

\footnotesize

\ifkorrekturansicht
  \lohead{\textsc{register}}
\fi

% theindex-Environment neu definieren ohne reledmac
\makeatletter
\renewenvironment{theindex}{%
  \ifkorrekturansicht
    \section*{\indexname}%
  \else
    \subsubsection*{Index der erwähnten Entitäten}%
  \fi
  \setlength{\parindent}{0pt}%
  \setlength{\parskip}{0pt plus 0.3pt}%
  \let\item\@idxitem
}{%
  \ifkorrekturansicht\clearpage\fi
}
\makeatother

\IfFileExists{\jobname-pw.ind}{\input{\jobname-pw.ind}}{}

% Quellenangabe nur in der Leseansicht
\ifkorrekturansicht\else
% Fallback-Definitionen, falls die .tex-Datei \titel etc. nicht gesetzt hat
\providecommand{\titel}{}
\providecommand{\editorInnen}{}
\providecommand{\dateiname}{\jobname}

\vspace{3cm}

\vfill

\footnotesize
\textsc{Quelle}: \titel. Herausgegeben von {\editorInnen}. In: \emph{Arthur Schnitzler: Briefwechsel mit Autorinnen und Autoren}.
 Digitale Edition, https://schnitzler-briefe.acdh.oeaw.ac.at/{\dateiname}.html (Stand \today)
\fi

\end{document}


      