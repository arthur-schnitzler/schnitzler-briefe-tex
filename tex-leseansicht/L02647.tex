%% latex-leseansicht-vorspann.tex
%% Vorspann für die Leseansicht.
%% Lädt die gemeinsame Datei latex-vorspann.tex mit nicht gesetztem Schalter.

\newif\ifkorrekturansicht
\korrekturansichtfalse

\input{../tex-inputs/latex-vorspann}


         \renewcommand{\erwaehnteOrte}{Orte: Wien}
         \renewcommand{\erwaehnteWerke}{Werke: Tagebuch}
               \section[Paul Goldmann an Arthur Schnitzler, {[}27. 4. 1890{]}]{ Paul Goldmann an Arthur Schnitzler, {[}27. 4. 1890{]}}\nopagebreak\mylabel{v}\rehead{ }\begin{ledgroupsized}[t]{13cm}\normalsize\beginnumbering \toendnotes[C]{\smallbreak\pagebreak[2]} \Standort{DLA, A:Schnitzler, HS.NZ85.1.3162.}
\physDesc{Brief, 1 Blatt, 3 Seiten, 1766 Zeichen
\newline{}Handschrift: Bleistift, deutsche Kurrent
\newline{}Schnitzler: 1) mit rotem Buntstift unterhalb des Textes »Paul Goldma{\geminationn}{ }{\\}2\textcolor{gray}{7}. 4. 90.« vermerkt  2) mit Bleistift seitlich auf der ersten Seite das Datum »27/ 4 90« vermerkt}\toendnotes[C]{\smallbreak}\stanza{}{\pb}Weißt Du es noch, mein liebes Kind?\newverse{}– viel’ Jahre ſeitdem verfloſſen ſind –\newverse{}Es war am \label{K_L02647-11v}\edtext{Sonntag{ }Nachmittag}{\lemma{\textnormal{\emph{Sonntag Nachmittag}}}\Cendnote{\textnormal{Das Gedicht dürfte den Besuch bei Schnitzler\pwindex{Schnitzler, Arthur 15.05.1862 – 21.10.1931@\textsc{Schnitzler, Arthur} (15.05.1862 – 21.10.1931), \emph{Schriftsteller, Mediziner}|pwk} verarbeiten, da auch der betreffende Eintrag
                     in Schnitzler\pwindex{Schnitzler, Arthur 15.05.1862 – 21.10.1931@\textsc{Schnitzler, Arthur} (15.05.1862 – 21.10.1931), \emph{Schriftsteller, Mediziner}|pwk}s \emph{Tagebuch}\pwindex{Schnitzler, Arthur 15.05.1862 – 21.10.1931@\textsc{Schnitzler, Arthur} (15.05.1862 – 21.10.1931), \emph{Schriftsteller, Mediziner}!Tagebuch1981 – 2000@\strich\emph{Tagebuch} {[}1981 – 2000{]}|pwk} vom 27. 4. 1890 – einem Sonntag – Motive
                     enthält, die im Gedicht aufgegriffen werden: »Gewitter.– Nm. Paul Goldmann\pwindex{Goldmann, Paul 31.01.1865 – 25.09.1935@\textsc{Goldmann, Paul} (31.01.1865 – 25.09.1935), \emph{Schriftsteller, Journalist}|pw},
                     Testament«.}}}\label{K_L02647-11h}\newverse{}Und ich auf Deinem Divan lag,\newverse{}Die \substVorne{}\textsuperscript{\textcolor{gray}{O}}\substDazwischen{}U\substHinten{}hren tickten hin und her,\newverse{}Sonſt war es ſtill und dumpf und ſchwer,\newverse{}\strikeout{E} Das Gl\textcolor{gray}{ü}hlicht Dir auf die
                  Haare ſchien,\newverse{}Gedämpft von des Scheines Roth und Grün,\newverse{}Ich ſeh Dir zu, Du merkteſt es nicht,\newverse{}Und haſt mit ſinnendem Geſicht,\newverse{}\substVorne{}\textsuperscript{\textcolor{gray}{D}}\substDazwischen{}M\substHinten{}it wenig Poſe und viel Bedacht\newverse{}Am Tiſche Dein Teſtament gemacht,\newverse{}Es war ein Scherz, eine dumme Idee,\newverse{}Auf daß der langweilige Sonntag vergeh’ –\newverse{}Und doch es uns kalt über den Rücken kroch –\newverse{}Wir ſtanden im Banne des »vielleicht doch« –\newverse{}Und überdies kam mit dumpfem Sch\textcolor{gray}{la}g\newverse{}Zurück das Gewitter von Vormittag –\newverse{}Ein Donner am Sonntag – fern, \label{K_L02647-1v}\edtext{ſordinirt}{\lemma{\textnormal{\emph{ſordinirt}}}\Cendnote{\textnormal{gedämpft}}}\label{K_L02647-1h} –\newverse{}Du weißt, was da für Stimmung gebiert.\newverse{}{\pb}Kurz nur, als ich aufthat meinen Hut –\newverse{}Ich kann es Dir ſagen, mir war nicht gut,\newverse{}Und als ich einſam gewandelt nach Haus\newverse{}Stak mir in den Gliedern ein frierender Graus.\newverse{}Der Teufel! Meine Naſe war gar nicht ſchlecht,\newverse{}Ich witterte Geiſterluft und hatte Recht.\newverse{}\uline{Du} haſt Dein Teſtament gemacht ohne Noth,\newverse{}Und ich war in wenigen Jahren todt,\newverse{}Am ſelben Sonntag, zur ſelben Stund’\newverse{}Da lag ich da mit zuckendem Mund\newverse{}Und der letzte Eindruck, den ich vernahm,\newverse{}Das war ein Donner, der freche Bann\textcolor{gray}{:}\newverse{}Und wieder \substVorne{}\textsuperscript{iſt es}{\allowbreak}\substDazwischen{}ſank\substHinten{} ein Sonntag herab\newverse{}Da bin ich geſtiegen aus meinem Grab –\newverse{}Hier ſitz ich, am Tiſche neben Dir\newverse{}Und glotze Dich an mit dem Augenſchein\newverse{}Das Glühlicht ſcheint Dir in’s Geſicht,\newverse{}Ich ſtarre Dich an und Du weißt es nicht,\newverse{}{\pb}Es packt Dich ein Schauder, Du \strikeout{\textcolor{gray}{a}ch} ahnſt nicht warum,\newverse{}Du möchtſeſt ſprechen und bleibſt doch ſtumm –\newverse{}Von fernher zieht der Donner heran –\newverse{}Nein, nein, bleib nur ſtill\textcolor{gray}{e}, Du armer Mann,\newverse{}Ich thue Dir nichts, ich bin nur da,\newverse{}Und jetzt, wo ich endlich Dich wiederſah,\newverse{}Jetzt kriech’ ich befriedigt zurück unter’n Stein –\newverse{}Wie gut es doch ist, geſtorben zu ſein!\stanzaend{}
         
         \endnumbering\mylabel{h}\end{ledgroupsized}  \newcommand{\dateiname}{L02647}\newcommand{\titel}{Paul Goldmann an Arthur Schnitzler, [27. 4. 1890]}\newcommand{\editorInnen}{Martin Anton Müller und Laura Untner}%% latex-leseansicht-abspann.tex
%% Abspann für die Leseansicht.
%% Der Schalter \ifkorrekturansicht ist bereits durch den Vorspann gesetzt.

%% latex-abspann.tex
%% Gemeinsamer Abspann für Korrekturansicht und Leseansicht.
%% Setzt den Schalter \ifkorrekturansicht voraus (gesetzt in den
%% einbindenden Dateien latex-korrekturansicht-abspann.tex bzw.
%% latex-leseansicht-abspann.tex).
%% ---------------------------------------------------------------

\normalsize

% Das esempio-Environment wird nur in der Leseansicht benötigt
\ifkorrekturansicht\else
\newenvironment{esempio}[3]%
{
    \vspace{1.5ex}
    \rlap{\underline{#1}}
    \par
    \setlength{\parindent}{0cm}
    \nopagebreak
    \leftskip=#2cm
    \rightskip=#3cm
}
{
    \par
}
\fi

\doendnotes{C}
\bigskip
\vfill

\clearpage

\footnotesize

\ifkorrekturansicht
  \lohead{\textsc{register}}
\fi

% theindex-Environment neu definieren ohne reledmac
\makeatletter
\renewenvironment{theindex}{%
  \ifkorrekturansicht
    \section*{\indexname}%
  \else
    \subsubsection*{Index der erwähnten Entitäten}%
  \fi
  \setlength{\parindent}{0pt}%
  \setlength{\parskip}{0pt plus 0.3pt}%
  \let\item\@idxitem
}{%
  \ifkorrekturansicht\clearpage\fi
}
\makeatother

\IfFileExists{\jobname-pw.ind}{\input{\jobname-pw.ind}}{}

% Quellenangabe nur in der Leseansicht
\ifkorrekturansicht\else
% Fallback-Definitionen, falls die .tex-Datei \titel etc. nicht gesetzt hat
\providecommand{\titel}{}
\providecommand{\editorInnen}{}
\providecommand{\dateiname}{\jobname}

\vspace{3cm}

\vfill

\footnotesize
\textsc{Quelle}: \titel. Herausgegeben von {\editorInnen}. In: \emph{Arthur Schnitzler: Briefwechsel mit Autorinnen und Autoren}.
 Digitale Edition, https://schnitzler-briefe.acdh.oeaw.ac.at/{\dateiname}.html (Stand \today)
\fi

\end{document}


      