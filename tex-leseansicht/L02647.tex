%% latex-korrekturansicht-vorspann.tex
%% Vorspann für die Korrekturansicht.
%% Lädt die gemeinsame Datei latex-vorspann.tex mit gesetztem Schalter.

\newif\ifkorrekturansicht
\korrekturansichttrue

\input{../tex-inputs/latex-vorspann}


\section[Paul Goldmann an Arthur Schnitzler, {[}27. 4. 1890{]}]{L02647 Paul Goldmann an Arthur Schnitzler, {[}27. 4. 1890{]}}
\nopagebreak\mylabel{L02647v}
\rehead{ }\normalsize\beginnumbering\briefempfaengerindex{Schnitzler, Arthur@\textsc{Schnitzler, Arthur}!zzzGoldmann, Paul@\emph{von Paul Goldmann}!1890-04-271@{{[}27. 4. 1890{]}}|(be}
\toendnotes[C]{\smallbreak\pagebreak[2]}\Standort{DLA, A:Schnitzler, HS.NZ85.1.3162.}
\physDesc{Brief, 1 Blatt, 3 Seiten, 1766 Zeichen
\newline{}Handschrift: Bleistift, deutsche Kurrent
\newline{}Schnitzler: 1) mit rotem Buntstift unterhalb des Textes »Paul Goldma{\geminationn}{ }{\\}2\textcolor{gray}{7}. 4. 90.« vermerkt  2) mit Bleistift seitlich auf der ersten Seite das Datum »27/ 4 90« vermerkt}\toendnotes[C]{\smallbreak}\stanza{}{\pb}Weißt Du es noch, mein liebes Kind?– viel’ Jahre ſeitdem verfloſſen ſind –Es war am \label{K_L02647-1v}\edtext{Sonntag{ }Nachmittag}{\lemma{\textnormal{\emph{Sonntag Nachmittag}}}\Cendnote{\textnormal{Das Gedicht dürfte den Besuch bei Schnitzler verarbeiten, da auch der betreffende Eintrag
                  in Schnitzlers{ }\emph{Tagebuch}\pwindex{Tagebuch@\emph{Tagebuch}|pwk} vom 27. 4. 1890 – einem Sonntag – Motive
                     enthält, die im Gedicht aufgegriffen werden: »Gewitter. – Nm. Paul Goldmann\pwindex{Goldmann, Paul 31.01.1865 – 25.09.1935@\textsc{Goldmann, Paul} (31.01.1865 – 25.09.1935), \emph{Schriftsteller/Schriftstellerin, Journalist/Journalistin}|pw},
                     Testament«.}}}\label{K_L02647-1}Und ich auf Deinem Divan lag,Die \substVorne{}\textsuperscript{\textcolor{gray}{O}}\substDazwischen{}U\substHinten{}hren tickten hin und her,Sonſt war es ſtill und dumpf und ſchwer,\strikeout{E} Das Gl\textcolor{gray}{ü}hlicht Dir auf die
                  Haare ſchien,Gedämpft von des Scheines Roth und Grün,Ich ſeh Dir zu, Du merkteſt es nicht,Und haſt mit ſinnendem Geſicht,\substVorne{}\textsuperscript{\textcolor{gray}{D}}\substDazwischen{}M\substHinten{}it wenig Poſe und viel BedachtAm Tiſche Dein Teſtament gemacht,Es war ein Scherz, eine dumme Idee,Auf daß der langweilige Sonntag vergeh’ –Und doch es uns kalt über den Rücken kroch –Wir ſtanden im Banne des »vielleicht doch« –Und überdies kam mit dumpfem Sch\textcolor{gray}{la}gZurück das Gewitter von Vormittag –Ein Donner am Sonntag – fern, \label{K_L02647-2v}\edtext{ſordinirt}{\lemma{\textnormal{\emph{ſordinirt}}}\Cendnote{\textnormal{gedämpft}}}\label{K_L02647-2} –Du weißt, was da für Stimmung gebiert.{\pb}Kurz nur, als ich aufthat meinen Hut –Ich kann es Dir ſagen, mir war nicht gut,Und als ich einſam gewandelt nach HausStak mir in den Gliedern ein frierender Graus.Der Teufel! Meine Naſe war gar nicht ſchlecht,Ich witterte Geiſterluft und hatte Recht.\uline{Du} haſt Dein Teſtament gemacht ohne Noth,Und ich war in wenigen Jahren todt,Am ſelben Sonntag, zur ſelben Stund’Da lag ich da mit zuckendem MundUnd der letzte Eindruck, den ich vernahm,Das war ein Donner, der freche Bann\textcolor{gray}{:}Und wieder \substVorne{}\textsuperscript{iſt es}\substDazwischen{}ſank\substHinten{} ein Sonntag herabDa bin ich geſtiegen aus meinem Grab –Hier ſitz ich, am Tiſche neben DirUnd glotze Dich an mit dem AugenſcheinDas Glühlicht ſcheint Dir in’s Geſicht,Ich ſtarre Dich an und Du weißt es nicht,{\pb}Es packt Dich ein Schauder, Du \strikeout{\textcolor{gray}{a}ch} ahnſt nicht warum,Du möchtſeſt ſprechen und bleibſt doch ſtumm –Von fernher zieht der Donner heran –Nein, nein, bleib nur ſtill\textcolor{gray}{e}, Du armer Mann,Ich thue Dir nichts, ich bin nur da,Und jetzt, wo ich endlich Dich wiederſah,Jetzt kriech’ ich befriedigt zurück unter’n Stein –Wie gut es doch ist, geſtorben zu ſein!\stanzaend{}\selectlanguage{ngerman}\endnumbering\briefempfaengerindex{Schnitzler, Arthur@\textsc{Schnitzler, Arthur}!zzzGoldmann, Paul@\emph{von Paul Goldmann}!1890-04-271@{{[}27. 4. 1890{]}}|)be}\mylabel{L02647h}  \normalsize

\doendnotes{C}
\bigskip
\vfill

\clearpage

\footnotesize

\lohead{\textsc{register}}

% Definiere theindex-Environment komplett neu ohne reledmac
\makeatletter
\renewenvironment{theindex}{%
  \section*{\indexname}%
  \setlength{\parindent}{0pt}%
  \setlength{\parskip}{0pt plus 0.3pt}%
  \let\item\@idxitem
}{%
  \clearpage
}
\makeatother

\IfFileExists{\jobname-pw.ind}{\input{\jobname-pw.ind}}{}

\end{document}

      