%% latex-leseansicht-vorspann.tex
%% Vorspann für die Leseansicht.
%% Lädt die gemeinsame Datei latex-vorspann.tex mit nicht gesetztem Schalter.

\newif\ifkorrekturansicht
\korrekturansichtfalse

\input{../tex-inputs/latex-vorspann}


\section[Hugo von Hofmannsthal an Arthur Schnitzler, 24. 9. [1914]]{L02197 Hugo von Hofmannsthal an Arthur Schnitzler, 24. 9. [1914]}
\nopagebreak\mylabel{L02197v}
\rehead{ }\normalsize\beginnumbering\briefempfaengerindex{Schnitzler, Arthur@\textsc{Schnitzler, Arthur}!zzzHofmannsthal, Hugo von@\emph{von Hugo von Hofmannsthal}!1914-09-241@{24. 9. [1914]}|(be}
\toendnotes[C]{\smallbreak\pagebreak[2]}
\correspDesc{Versand  durch Hugo von Hofmannsthal am 24. 9. [1914] in Rodaun
\newline{}Erhalt  durch Arthur Schnitzler im Zeitraum [25. 9. 1914
                  – 29. 9. 1914?] in Wien}\toendnotes[C]{\smallbreak}
\Standort{CUL, Schnitzler, B 43.}
\physDesc{Brief, 1 Blatt, 1 Seite, 833 Zeichen
\newline{}Handschrift: schwarze Tinte, deutsche Kurrent
\newline{}Beilage: Alexander Hoyos\pwindex{Hoyos, Sándor 13.\,5.\,1876 – 20.\,10.\,1937@\textsc{Hoyos, Sándor} (13.\,5.\,1876 – 20.\,10.\,1937)|pw}: Brief, 1
                                 Blatt, 3 Seiten, schwarze Tinte, Lateinschrift 
\newline{}Schnitzler: 1) mit Bleistift beschriftet: »Hugo«   2) mit rotem Buntstift eine Unterstreichung
\newline{}Ordnung: 1) mit Bleistift von unbekannter Hand nummeriert: »\strikeout{329}«  2) mit Bleistift von unbekannter Hand nummeriert:
                                    »352«}
\buchAbdrucke{\weitereDrucke{Hugo von Hofmannsthal, Arthur Schnitzler: \emph{Briefwechsel}. Herausgegeben von Therese Nickl und Heinrich Schnitzler. Frankfurt am Main: \emph{S. Fischer} 1964, S. 277.} }\toendnotes[C]{\smallbreak}
\pstart
           \raggedleft{}{\pb}R.\oindex{Wien@\textbf{Wien}!XXIII., Liesing@\textbf{XXIII., Liesing}!Rodaun@\textbf{Rodaun}, \emph{Region}|pw}{ }24. IX.{\\}abends.\pend
           
\pstart{}mein lieber Arthur\pend\vspace{0.5em}
\pstart
           hier iſt die Antwort von Alexander Hoyos\pwindex{Hoyos, Sándor 13.\,5.\,1876 – 20.\,10.\,1937@\textsc{Hoyos, Sándor} (13.\,5.\,1876 – 20.\,10.\,1937)|pw}
               (Cabinetschef) bezüglich der rumäniſchen\oindex{Rumänien@\textbf{Rumänien}|pw}
               Zeitungen. Das schwer leſerliche Wort heißt \uline{Erpreſſer}.\hspace*{1.5em}Ich bin noch ziemlich unwohl und{ }ſchwach, muſs viel erledigen, daher die Kürze.\pend
           
\pstart
           Alles Liebe an Olga\pwindex{Schnitzler, Olga 17.\,1.\,1882 Wien – 13.\,1.\,1970 Lugano@\textsc{Schnitzler, Olga} (17.\,1.\,1882 Wien – 13.\,1.\,1970 Lugano), \emph{Schauspielerin, Sängerin}|pw}.\pend
           
\pstart
           Ihr{\\[\baselineskip]}\spacefill\mbox{Hugo.}\pend
           \leftskip=0em{}\selectlanguage{ngerman}\vspace{1em}{\vspace{1\baselineskip}}
\pstart
           {\pb}\textcolor{gray}{\textbf{Ministère Imperial et Royal\orgindex{Ministerium für Äußeres@Ministerium für Äußeres|pw}}}\pend
           
\pstart
           \textcolor{gray}{\textbf{des affaires étrangères\orgindex{Ministerium für Äußeres@Ministerium für Äußeres|pw}.}}\pend
           
\pstart
           \textcolor{gray}{\textbf{\textsc{Cabinet du Ministre\pwindex{Berchtold, Leopold von 18.\,4.\,1863 Wien – 21.\,11.\,1942 Peresznye@\textsc{Berchtold, Leopold von} (18.\,4.\,1863 Wien – 21.\,11.\,1942 Peresznye), \emph{Politiker, Diplomat}|pwv}.}}}\pend
           
\pstart
           \raggedleft{}{[}hs. Hoyos:{]} 22/9 1914\pend
           
\pstart{}Lieber Freund\pend\vspace{0.5em}
\pstart
           Bitte verzeihe dass ich Dir erst heute für Deine freundliche Anregung vom
                  15. d. Mts. danke, ich war auf 2 Tage verreist und nach meiner
               Rückkehr sehr beschäftigt. Wir haben schon seit einiger Zeit eine Aktion im Sinne
               Deines Briefs {\pb}eingeleitet,
               hoffentlich wird sie von Erfolg begleitet sein{[},{]} leider sind
               unsere Feinde auch sehr auch sehr freigebig und wissen unsere Bemühungen in
               geschickter Weise auszugleichen. So werden die Erpresser immer reicher ohne ihre
               Haltung ändern zu müssen.\pend
           
\pstart
           Mit besten Grüßen bin ich {\pb}Dein sehr ergebener {\\[\baselineskip]}\spacefill\mbox{A. Hoyos.}\pend
           \leftskip=0em{}\selectlanguage{ngerman}\endnumbering\briefempfaengerindex{Schnitzler, Arthur@\textsc{Schnitzler, Arthur}!zzzHofmannsthal, Hugo von@\emph{von Hugo von Hofmannsthal}!1914-09-241@{24. 9. [1914]}|)be}\mylabel{L02197h}  \newcommand{\dateiname}{L02197}\newcommand{\titel}{Hugo von Hofmannsthal an Arthur Schnitzler, 24. 9. [1914]}\newcommand{\editorInnen}{Martin Anton Müller und Gerd-Hermann Susen}%% latex-leseansicht-abspann.tex
%% Abspann für die Leseansicht.
%% Der Schalter \ifkorrekturansicht ist bereits durch den Vorspann gesetzt.

%% latex-abspann.tex
%% Gemeinsamer Abspann für Korrekturansicht und Leseansicht.
%% Setzt den Schalter \ifkorrekturansicht voraus (gesetzt in den
%% einbindenden Dateien latex-korrekturansicht-abspann.tex bzw.
%% latex-leseansicht-abspann.tex).
%% ---------------------------------------------------------------

\normalsize

% Das esempio-Environment wird nur in der Leseansicht benötigt
\ifkorrekturansicht\else
\newenvironment{esempio}[3]%
{
    \vspace{1.5ex}
    \rlap{\underline{#1}}
    \par
    \setlength{\parindent}{0cm}
    \nopagebreak
    \leftskip=#2cm
    \rightskip=#3cm
}
{
    \par
}
\fi

\doendnotes{C}
\bigskip
\vfill

\clearpage

\footnotesize

\ifkorrekturansicht
  \lohead{\textsc{register}}
\fi

% theindex-Environment neu definieren ohne reledmac
\makeatletter
\renewenvironment{theindex}{%
  \ifkorrekturansicht
    \section*{\indexname}%
  \else
    \subsubsection*{Index der erwähnten Entitäten}%
  \fi
  \setlength{\parindent}{0pt}%
  \setlength{\parskip}{0pt plus 0.3pt}%
  \let\item\@idxitem
}{%
  \ifkorrekturansicht\clearpage\fi
}
\makeatother

\IfFileExists{\jobname-pw.ind}{\input{\jobname-pw.ind}}{}

% Quellenangabe nur in der Leseansicht
\ifkorrekturansicht\else
% Fallback-Definitionen, falls die .tex-Datei \titel etc. nicht gesetzt hat
\providecommand{\titel}{}
\providecommand{\editorInnen}{}
\providecommand{\dateiname}{\jobname}

\vspace{3cm}

\vfill

\footnotesize
\textsc{Quelle}: \titel. Herausgegeben von {\editorInnen}. In: \emph{Arthur Schnitzler: Briefwechsel mit Autorinnen und Autoren}.
 Digitale Edition, https://schnitzler-briefe.acdh.oeaw.ac.at/{\dateiname}.html (Stand \today)
\fi

\end{document}


