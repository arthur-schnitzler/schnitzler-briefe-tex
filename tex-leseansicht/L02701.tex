%% latex-korrekturansicht-vorspann.tex
%% Vorspann für die Korrekturansicht.
%% Lädt die gemeinsame Datei latex-vorspann.tex mit gesetztem Schalter.

\newif\ifkorrekturansicht
\korrekturansichttrue

\input{../tex-inputs/latex-vorspann}


\section[Paul Goldmann an Arthur Schnitzler, 5. 8. {[}1892{]}]{L02701 Paul Goldmann an Arthur Schnitzler, 5. 8. {[}1892{]}}
\nopagebreak\mylabel{L02701v}
\rehead{ }\normalsize\beginnumbering\briefempfaengerindex{Schnitzler, Arthur@\textsc{Schnitzler, Arthur}!zzzGoldmann, Paul@\emph{von Paul Goldmann}!1892-08-151@{5. 8. {[}1892{]}}|(be}
\toendnotes[C]{\smallbreak\pagebreak[2]}\Standort{DLA, A:Schnitzler, HS.NZ85.1.3163.}
\physDesc{Brief, 2 Blätter, 8 Seiten, 2581 Zeichen
\newline{}Handschrift: schwarze Tinte, deutsche Kurrent
\newline{}Schnitzler: mit Bleistift das Jahr »92« vermerkt }\toendnotes[C]{\smallbreak}
\pstart
           {\pb}\textcolor{gray}{\textbf{CASINO}}\orgindex{Casino de Biarritz@Casino de Biarritz|pw}\pend
           
\pstart
           \textcolor{gray}{\textbf{DE}}\orgindex{Casino de Biarritz@Casino de Biarritz|pw}\pend
           
\pstart
           \textcolor{gray}{\textbf{BIARRITZ}}\orgindex{Casino de Biarritz@Casino de Biarritz|pw}\hfill 5. Auguſt. \pend
           
\pstart\center{}Mein lieber Arthur!\pend\vspace{0.5em}
\pstart
           Im Abreiſefieber mußte ich Deinen letzten lieben Brief unbeantwortet laſſen. Erſt
                  heut finde ich die nöthige Zeit und Ruhe zu einer
               Zeile Antwort. Da ſitze ich in halber Schlaftrunkenheit und reibe mir die Augen. Das
               blaue, blaue Meer blinkt zum Fenſter hinein und rauſcht mir in die Ohren (Atlantiſche\substVorne{}\textsuperscript{s }\substDazwischen{}r\substHinten{}{ }{\pb}Ocean\oindex{Atlantischer Ozean@\textbf{Atlantischer Ozean}, \emph{Meer (N.MER)}|pw}, mein lieber Arthur, \textsc{Golf von Gascogne\oindex{Biskaya@\textbf{Biskaya}, \emph{Bucht (N.BCT)}|pw}}.) Und ich frage mich: wie \strikeout{ko} komme ich hierher \substVorne{}\textsuperscript{a}\substDazwischen{}in\substHinten{} den blauen, blauen Süden, \strikeout{und} an die
               Grenzmarke von Frankreich\oindex{Frankreich@\textbf{Frankreich}, \emph{A.PCLI}|pw} und Spanien\oindex{Spanien@\textbf{Spanien}, \emph{A.PCLI}|pw}{ }\strikeout{(S\textcolor{gray}{u}} (Südweſtgrenze, mein lieber Arthur) – ich, der ich \label{K_L02701-1v}\edtext{geſtern}{\lemma{\textnormal{\emph{geſtern}}}\Cendnote{\textnormal{im
                  übertragenen Sinn von »es kommt mir vor, als wäre es gestern gewesen«
                  gemeint}}}\label{K_L02701-1} noch im \textsc{Café Pfob\oindex{Cafe Pfob@\textbf{Café Pfob}, \emph{Kaffeehaus (K.KAF)}|pw}} ſaß und die bekannte \label{K_L02701-2v}\edtext{\textsc{Café}haus-Ecke}{\lemma{\textnormal{\emph{Caféhaus-Ecke}}}\Cendnote{\textnormal{Vgl. Schnitzlers Texte \emph{Aus der Kaffeehausecke}\pwindex{Aus der Kaffeehausecke@\emph{Aus der Kaffeehausecke}|pwk} und \emph{Gespräch, welches in der Kaffeehausecke nach Vorlesung der »Elixiere« geführt
                     wird}\pwindex{Gespraech, welches in der Kaffeehausecke nach Vorlesung der »Elixiere« gefuehrt wird@\emph{Gespräch, welches in der Kaffeehausecke nach Vorlesung der »Elixiere« geführt wird}|pwk}. Dass Goldmann\pwindex{Goldmann, Paul 31.01.1865 – 25.09.1935@\textsc{Goldmann, Paul} (31.01.1865 – 25.09.1935), \emph{Schriftsteller/Schriftstellerin, Journalist/Journalistin}|pwk} ebenso den
                  Begriff »Caféhaus-Ecke« benutzte, deutet darauf hin, dass er
                  allgemein im Freundeskreis verwendet wurde.}}}\label{K_L02701-2} mit Aphorismen austapezierte.
               Und da willſt Du noch \textcolor{gray}{L}achen über \label{K_L02701-3v}\edtext{»die Fäden«}{\lemma{\textnormal{\emph{»die Fäden«}}}\Cendnote{\textnormal{Möglicherweise schließt hier Goldmann\pwindex{Goldmann, Paul 31.01.1865 – 25.09.1935@\textsc{Goldmann, Paul} (31.01.1865 – 25.09.1935), \emph{Schriftsteller/Schriftstellerin, Journalist/Journalistin}|pwk} an
                  bestimmte Aussagen von Schnitzler an. In
                  seinem \emph{Tagebuch}\pwindex{Tagebuch@\emph{Tagebuch}|pwk} schreibt dieser mehrfach von
                     »Fäden«, die ihn mit der Welt und die Welt an sich
                  verknüpfen.}}}\label{K_L02701-3}?\pend
           
\pstart
           Das iſt wunderbar\substVorne{}\textsuperscript{.}\substDazwischen{},\substHinten{} all’ das. Aber Du {\pb}weißt, daß das
               Wunderbare nicht das Glückliche iſt. Und meine Reiſe, die objectiv wunderſchön iſt,
               iſt es \label{T_L02701-1v}\edtext{ſubjectiv}{\lemma{\textnormal{\emph{ſubjectiv}}}\Cendnote{\textnormal{Über dem ›e‹ befindet sich ein durchgestrichener
                  u-Strich.}}}\label{T_L02701-1} um ſo weniger. Schlaftrunken laſſe ich mich durch die Welt
               ſchleppen. Und mitten in\strikeout{\textcolor{gray}{s}} der himmliſchen Herrlichkeit des Südens ſchwirrt mir der Fledermausſchwarm
               meiner Sorgen unaufhörlich um das Haupt, Tag und Nacht, Tag und Nacht. Das Glück?
               Überall, wo ich hinkomme: »Eine Empfehlung, {\pb}und es
               iſt geſtern dageweſen«. Ich habe nur ein nervöſes Bedürfniß nach \label{K_L02701-4v}\edtext{\textsc{Locomotion}}{\lemma{\textnormal{\emph{Locomotion}}}\Cendnote{\textnormal{Fortbewegung}}}\label{K_L02701-4} in mir, halte es
               nirgends aus und habe ſtets eine Stimme in mir, die mir ſagt: »Dort drüben iſt es
               ſchöner.« Und ſo geht es weiter und weiter: übermorgen
               nach \textsc{San Sebastian\oindex{San Sebastian@\textbf{San Sebastian}, \emph{P.PPLA2}|pw}} (Nordſpanien\oindex{Spanien@\textbf{Spanien}, \emph{A.PCLI}|pw}, mein lieber Arthur), dann
               nach den Pyrenäen\oindex{Pyrenees@\textbf{Pyrenees}, \emph{Gebirge (N.GBR)}|pw}, dann wieder heim. Überall
               unterwegs bin natürlich {\pb}bitterlich allein. Kein
               Menſch zu finden in dieſem verdammten Lande. Mit dem deutſchen Accent ſcheucht man
               die Leute von ſich fort, \strikeout{als} und man ſitzt im \label{K_L02701-5v}\edtext{\textsc{Coupé}}{\lemma{\textnormal{\emph{Coupé}}}\Cendnote{\textnormal{Zugabteil}}}\label{K_L02701-5} und im \label{T_L02701-2v}\edtext{W\textcolor{gray}{i}rthshaus}{\lemma{\textnormal{\emph{Wirthshaus}}}\Cendnote{\textnormal{Ein deutlicher u-Strich macht den Vokal der
                  ersten Silbe zu einem ›u‹, doch dürfte ein Schreibirrtum vorliegen.}}}\label{T_L02701-2} ſo
               gemieden, als wäre man der Scharfrichter der zu einer Hinrichtung fährt{\dotsfour}\pend
           
\pstart
           Mein Onkel\pwindex{Mamroth, Fedor 21.02.1851 – 25.06.1907@\textsc{Mamroth, Fedor} (21.02.1851 – 25.06.1907), \emph{Journalist/Journalistin, Kritiker/Kritikerin}|pwv} iſt in \textsc{Salzburg\oindex{Salzburg@\textbf{Salzburg}, \emph{A.ADM2}|pw}} (\textsc{Faberhaus\oindex{Faberhaeuser@\textbf{Faberhäuser}, \emph{Wohngebäude (K.WHS)}|pw}}). {\pb}Wenn Du ihn einmal über den Sonntag
               beſuchen könnteſt, möcht’ er ſich rieſig mit Dir freuen. Bitte,
                  fahr\textcolor{gray}{’} doch einmal hinüber. Ich weiß Euch zwei gerne zuſammen,
               die Ihr mir die theuerſten Freunde\pwindex{Mamroth, Fedor 21.02.1851 – 25.06.1907@\textsc{Mamroth, Fedor} (21.02.1851 – 25.06.1907), \emph{Journalist/Journalistin, Kritiker/Kritikerin}|pwv} ſeid. Du kannſt all’ Deine literariſchen Angelegenheiten mit ihm
               beſprechen, und beſſeren ſachverſtändigen Rath kannſt Du Dir {\pb}nicht wünſchen. Mußt’ Dich aber vorher anmelden,
               damit er nicht etwa auf Ausflug iſt{\dotsfour}\pend
           
\pstart
           Dich im \label{K_L02701-6v}\edtext{September wiederſehen}{\lemma{\textnormal{\emph{September wiederſehen}}}\Cendnote{\textnormal{Dazu kam es nicht.}}}\label{K_L02701-6}? Schönſte aller Ausſichten! Aber
               glaubſt Du, ich glaub’s? {\dotsfour}\pend
           
\pstart
           Bitte, ſei brav’ und ſchreib’ mir eine Zeile nach \textsc{Pau\oindex{Pau@\textbf{Pau}, \emph{P.PPLA2}|pw}}, \textsc{Pyrénées\oindex{Pyrenees@\textbf{Pyrenees}, \emph{Gebirge (N.GBR)}|pw}}, \textsc{Poste restante}, wo ich Mittwoch einzutreffen gedenke. Erhältſt Du {\pb}meinen Brief zu ſpät, ſo ſchreib’ mir, bitte, nach
                  \textsc{Cauterets\oindex{Cauterets@\textbf{Cauterets}, \emph{P.PPL}|pw}}, \textsc{\strikeout{Pyree\oindex{Pyrenees@\textbf{Pyrenees}, \emph{Gebirge (N.GBR)}|pw}}}{ }\textsc{Pyrénées\oindex{Pyrenees@\textbf{Pyrenees}, \emph{Gebirge (N.GBR)}|pw}}, \textsc{Post restante}.\pend
           
\pstart
           Und was wird aus \textsc{Richard\pwindex{Beer-Hofmann, Richard 1866-07-11 – 1945-09-26@\textsc{Beer-Hofmann, Richard} (1866-07-11 – 1945-09-26), \emph{Schriftsteller/Schriftstellerin}|pw}}? Keine Zeile von ihm ſeit dreiviertel Jahren!\pend
           
\pstart
           Ich umarme Dich herzlichſt! {\\[\baselineskip]}Dein {\\[\baselineskip]}treuer {\\[\baselineskip]}\spacefill\mbox{Paul Goldmann.}\pend
           \leftskip=0em{}\selectlanguage{ngerman}\endnumbering\briefempfaengerindex{Schnitzler, Arthur@\textsc{Schnitzler, Arthur}!zzzGoldmann, Paul@\emph{von Paul Goldmann}!1892-08-151@{5. 8. {[}1892{]}}|)be}\mylabel{L02701h}  \normalsize

\doendnotes{C}
\bigskip
\vfill

\clearpage

\footnotesize

\lohead{\textsc{register}}

% Definiere theindex-Environment komplett neu ohne reledmac
\makeatletter
\renewenvironment{theindex}{%
  \section*{\indexname}%
  \setlength{\parindent}{0pt}%
  \setlength{\parskip}{0pt plus 0.3pt}%
  \let\item\@idxitem
}{%
  \clearpage
}
\makeatother

\IfFileExists{\jobname-pw.ind}{\input{\jobname-pw.ind}}{}

\end{document}

      