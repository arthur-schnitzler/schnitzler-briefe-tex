%% latex-leseansicht-vorspann.tex
%% Vorspann für die Leseansicht.
%% Lädt die gemeinsame Datei latex-vorspann.tex mit nicht gesetztem Schalter.

\newif\ifkorrekturansicht
\korrekturansichtfalse

\input{../tex-inputs/latex-vorspann}


\section[Paul Goldmann an Arthur Schnitzler, 5. 8. [1892]]{L02701 Paul Goldmann an Arthur Schnitzler, 5. 8. [1892]}
\nopagebreak\mylabel{L02701v}
\rehead{ }\normalsize\beginnumbering\briefempfaengerindex{Schnitzler, Arthur@\textsc{Schnitzler, Arthur}!zzzGoldmann, Paul@\emph{von Paul Goldmann}!1892-08-151@{5. 8. [1892]}|(be}
\toendnotes[C]{\smallbreak\pagebreak[2]}
\correspDesc{Versand  durch Paul Goldmann am 5. 8. [1892] in Biarritz
\newline{}Erhalt  durch Arthur Schnitzler im Zeitraum [16. 8. 1892
                  – 20. 8. 1892?] in Wien}\toendnotes[C]{\smallbreak}
\Standort{DLA, A:Schnitzler, HS.NZ85.1.3163.}
\physDesc{Brief, 2 Blätter, 8 Seiten, 2581 Zeichen
\newline{}Handschrift: schwarze Tinte, deutsche Kurrent
\newline{}Schnitzler: mit Bleistift das Jahr »92« vermerkt }\toendnotes[C]{\smallbreak}
\pstart
           {\pb}\textcolor{gray}{\textbf{CASINO}}\orgindex{Casino de Biarritz@Casino de Biarritz|pw}\pend
           
\pstart
           \textcolor{gray}{\textbf{DE}}\orgindex{Casino de Biarritz@Casino de Biarritz|pw}\pend
           
\pstart
           \textcolor{gray}{\textbf{BIARRITZ}}\orgindex{Casino de Biarritz@Casino de Biarritz|pw}\hfill 5. Auguſt.\pend
           
\pstart\center{}Mein lieber Arthur!\pend\vspace{0.5em}
\pstart
           Im Abreiſefieber mußte ich Deinen letzten lieben Brief unbeantwortet laſſen. Erſt
                  heut finde ich die nöthige Zeit und Ruhe zu einer
               Zeile Antwort. Da{ }ſitze ich in halber Schlaftrunkenheit und reibe mir die Augen. Das
               blaue, blaue Meer blinkt zum Fenſter hinein und rauſcht mir in die Ohren (Atlantiſche\substVorne{}\textsuperscript{s}\substDazwischen{}r\substHinten{}{ }{\pb}Ocean\oindex{Atlantischer Ozean@\textbf{Atlantischer Ozean}|pw}, mein lieber Arthur, \textsc{Golf von Gascogne\oindex{Biskaya@\textbf{Biskaya}|pw}}.) Und ich frage mich: wie \strikeout{ko} komme ich hierher \substVorne{}\textsuperscript{a}\substDazwischen{}in\substHinten{} den blauen, blauen Süden, \strikeout{und} an die
               Grenzmarke von Frankreich\oindex{Frankreich@\textbf{Frankreich}|pw} und Spanien\oindex{Spanien@\textbf{Spanien}|pw}{ }\strikeout{(S\textcolor{gray}{u}} (Südweſtgrenze, mein lieber Arthur) – ich, der ich \label{K_L02701-1v}\edtext{geſtern}{\lemma{\textnormal{\emph{gestern}}}\Cendnote{\textnormal{im
                  übertragenen Sinn von »es kommt mir vor, als wäre es gestern gewesen«
                  gemeint}}}\label{K_L02701-1} noch im \textsc{Café Pfob\oindex{Wien@\textbf{Wien}!I., Innere Stadt@\textbf{I., Innere Stadt}!Café Pfob@\textbf{Café Pfob}, \emph{Kaffeehaus}|pw}}{ }ſaß und die bekannte \label{K_L02701-2v}\edtext{\textsc{Café}haus-Ecke}{\lemma{\textnormal{\emph{Caféhaus-Ecke}}}\Cendnote{\textnormal{Vgl. Schnitzlers Texte \emph{Aus der Kaffeehausecke}\pwindex{Schnitzler, Arthur 15.\,5.\,1862 Wien – 21.\,10.\,1931 ebd.@\textsc{Schnitzler, Arthur} (15.\,5.\,1862 Wien – 21.\,10.\,1931 ebd.), \emph{Schriftsteller, Mediziner}!Aus der Kaffeehausecke@\strich\emph{Aus der Kaffeehausecke}|pwk} und \emph{Gespräch, welches in der Kaffeehausecke nach Vorlesung der »Elixiere« geführt
                     wird}\pwindex{Schnitzler, Arthur 15.\,5.\,1862 Wien – 21.\,10.\,1931 ebd.@\textsc{Schnitzler, Arthur} (15.\,5.\,1862 Wien – 21.\,10.\,1931 ebd.), \emph{Schriftsteller, Mediziner}!Gespräch, welches in der Kaffeehausecke nach Vorlesung der »Elixiere« geführt wird@\strich\emph{Gespräch, welches in der Kaffeehausecke nach Vorlesung der »Elixiere« geführt wird}|pwk}. Dass Goldmann\pwindex{Goldmann, Paul 31.\,1.\,1865 Breslau – 25.\,9.\,1935 Wien@\textsc{Goldmann, Paul} (31.\,1.\,1865 Breslau – 25.\,9.\,1935 Wien), \emph{Schriftsteller, Journalist}|pwk} ebenso den
                  Begriff »Caféhaus-Ecke« benutzte, deutet darauf hin, dass er
                  allgemein im Freundeskreis verwendet wurde.}}}\label{K_L02701-2} mit Aphorismen austapezierte.
               Und da willſt Du noch \textcolor{gray}{L}achen über \label{K_L02701-3v}\edtext{»die Fäden«}{\lemma{\textnormal{\emph{»die Fäden«}}}\Cendnote{\textnormal{Möglicherweise schließt hier Goldmann\pwindex{Goldmann, Paul 31.\,1.\,1865 Breslau – 25.\,9.\,1935 Wien@\textsc{Goldmann, Paul} (31.\,1.\,1865 Breslau – 25.\,9.\,1935 Wien), \emph{Schriftsteller, Journalist}|pwk} an
                  bestimmte Aussagen von Schnitzler an. In
                  seinem \emph{Tagebuch}\pwindex{Schnitzler, Arthur 15.\,5.\,1862 Wien – 21.\,10.\,1931 ebd.@\textsc{Schnitzler, Arthur} (15.\,5.\,1862 Wien – 21.\,10.\,1931 ebd.), \emph{Schriftsteller, Mediziner}!Tagebuch@\strich\emph{Tagebuch}|pwk} schreibt dieser mehrfach von
                     »Fäden«, die ihn mit der Welt und die Welt an sich
                  verknüpfen.}}}\label{K_L02701-3}?\pend
           
\pstart
           Das iſt wunderbar\substVorne{}\textsuperscript{.}\substDazwischen{},\substHinten{} all’ das. Aber Du {\pb}weißt, daß das
               Wunderbare nicht das Glückliche iſt. Und meine Reiſe, die objectiv wunderſchön iſt,
               iſt es \label{T_L02701-1v}\edtext{ſubjectiv}{\lemma{\textnormal{\emph{subjectiv}}}\Cendnote{\textnormal{Über dem ›e‹ befindet sich ein durchgestrichener
                  u-Strich.}}}\label{T_L02701-1} um{ }ſo weniger. Schlaftrunken laſſe ich mich durch die Welt{ }ſchleppen. Und mitten in\strikeout{\textcolor{gray}{s}} der himmliſchen Herrlichkeit des Südens{ }ſchwirrt mir der Fledermausſchwarm
               meiner Sorgen unaufhörlich um das Haupt, Tag und Nacht, Tag und Nacht. Das Glück?
               Überall, wo ich hinkomme: »Eine Empfehlung, {\pb}und es
               iſt geſtern dageweſen«. Ich habe nur ein nervöſes Bedürfniß nach \label{K_L02701-4v}\edtext{\textsc{Locomotion}}{\lemma{\textnormal{\emph{Locomotion}}}\Cendnote{\textnormal{Fortbewegung}}}\label{K_L02701-4} in mir, halte es
               nirgends aus und habe{ }ſtets eine Stimme in mir, die mir{ }ſagt: »Dort drüben iſt es{ }ſchöner.« Und{ }ſo geht es weiter und weiter: übermorgen
               nach \textsc{San Sebastian\oindex{San Sebastian@\textbf{San Sebastian}, \emph{Hauptstadt}|pw}} (Nordſpanien\oindex{Spanien@\textbf{Spanien}|pw}, mein lieber Arthur), dann
               nach den Pyrenäen\oindex{Pyrenees@\textbf{Pyrenees}, \emph{Gebirge}|pw}, dann wieder heim. Überall
               unterwegs bin natürlich {\pb}bitterlich allein. Kein
               Menſch zu finden in dieſem verdammten Lande. Mit dem deutſchen Accent{ }ſcheucht man
               die Leute von{ }ſich fort, \strikeout{als} und man{ }ſitzt im \label{K_L02701-5v}\edtext{\textsc{Coupé}}{\lemma{\textnormal{\emph{Coupé}}}\Cendnote{\textnormal{Zugabteil}}}\label{K_L02701-5} und im \label{T_L02701-2v}\edtext{W\textcolor{gray}{i}rthshaus}{\lemma{\textnormal{\emph{Wirthshaus}}}\Cendnote{\textnormal{Ein deutlicher u-Strich macht den Vokal der
                  ersten Silbe zu einem ›u‹, doch dürfte ein Schreibirrtum vorliegen.}}}\label{T_L02701-2}{ }ſo
               gemieden, als wäre man der Scharfrichter der zu einer Hinrichtung fährt{\dotsfour}\pend
           
\pstart
           Mein Onkel\pwindex{Mamroth, Fedor 21.\,2.\,1851 Breslau – 25.\,6.\,1907 Frankfurt am Main@\textsc{Mamroth, Fedor} (21.\,2.\,1851 Breslau – 25.\,6.\,1907 Frankfurt am Main), \emph{Journalist, Kritiker}|pwv} iſt in \textsc{Salzburg\oindex{Salzburg@\textbf{Salzburg}, \emph{Verwaltungsgebiet}|pw}} (\textsc{Faberhaus\oindex{Faberhäuser@\textbf{Faberhäuser}, \emph{Wohngebäude}|pw}}). {\pb}Wenn Du ihn einmal über den Sonntag
               beſuchen könnteſt, möcht’ er{ }ſich rieſig mit Dir freuen. Bitte,
                  fahr\textcolor{gray}{’} doch einmal hinüber. Ich weiß Euch zwei gerne zuſammen,
               die Ihr mir die theuerſten Freunde\pwindex{Mamroth, Fedor 21.\,2.\,1851 Breslau – 25.\,6.\,1907 Frankfurt am Main@\textsc{Mamroth, Fedor} (21.\,2.\,1851 Breslau – 25.\,6.\,1907 Frankfurt am Main), \emph{Journalist, Kritiker}|pwv}{ }ſeid. Du kannſt all’ Deine literariſchen Angelegenheiten mit ihm
               beſprechen, und beſſeren{ }ſachverſtändigen Rath kannſt Du Dir {\pb}nicht wünſchen. Mußt’ Dich aber vorher anmelden,
               damit er nicht etwa auf Ausflug iſt{\dotsfour}\pend
           
\pstart
           Dich im \label{K_L02701-6v}\edtext{September wiederſehen}{\lemma{\textnormal{\emph{September wiedersehen}}}\Cendnote{\textnormal{Dazu kam es nicht.}}}\label{K_L02701-6}? Schönſte aller Ausſichten! Aber
               glaubſt Du, ich glaub’s? {\dotsfour}\pend
           
\pstart
           Bitte,{ }ſei brav’ und{ }ſchreib’ mir eine Zeile nach \textsc{Pau\oindex{Pau@\textbf{Pau}, \emph{Hauptstadt}|pw}}, \textsc{Pyrénées\oindex{Pyrenees@\textbf{Pyrenees}, \emph{Gebirge}|pw}}, \textsc{Poste restante}, wo ich Mittwoch einzutreffen gedenke. Erhältſt Du {\pb}meinen Brief zu{ }ſpät,{ }ſo{ }ſchreib’ mir, bitte, nach
                  \textsc{Cauterets\oindex{Cauterets@\textbf{Cauterets}|pw}}, \textsc{\strikeout{Pyree\oindex{Pyrenees@\textbf{Pyrenees}, \emph{Gebirge}|pw}}}{ }\textsc{Pyrénées\oindex{Pyrenees@\textbf{Pyrenees}, \emph{Gebirge}|pw}}, \textsc{Post restante}.\pend
           
\pstart
           Und was wird aus \textsc{Richard\pwindex{Beer-Hofmann, Richard 11.\,7.\,1866 Wien – 26.\,9.\,1945 New York City@\textsc{Beer-Hofmann, Richard} (11.\,7.\,1866 Wien – 26.\,9.\,1945 New York City), \emph{Schriftsteller}|pw}}? Keine Zeile von ihm{ }ſeit dreiviertel Jahren!\pend
           
\pstart
           Ich umarme Dich herzlichſt! {\\[\baselineskip]}Dein {\\[\baselineskip]}treuer {\\[\baselineskip]}\spacefill\mbox{Paul Goldmann.}\pend
           \leftskip=0em{}\selectlanguage{ngerman}\endnumbering\briefempfaengerindex{Schnitzler, Arthur@\textsc{Schnitzler, Arthur}!zzzGoldmann, Paul@\emph{von Paul Goldmann}!1892-08-151@{5. 8. [1892]}|)be}\mylabel{L02701h}  \newcommand{\dateiname}{L02701}\newcommand{\titel}{Paul Goldmann an Arthur Schnitzler, 5. 8. [1892]}\newcommand{\editorInnen}{Martin Anton Müller und Laura Untner}%% latex-leseansicht-abspann.tex
%% Abspann für die Leseansicht.
%% Der Schalter \ifkorrekturansicht ist bereits durch den Vorspann gesetzt.

%% latex-abspann.tex
%% Gemeinsamer Abspann für Korrekturansicht und Leseansicht.
%% Setzt den Schalter \ifkorrekturansicht voraus (gesetzt in den
%% einbindenden Dateien latex-korrekturansicht-abspann.tex bzw.
%% latex-leseansicht-abspann.tex).
%% ---------------------------------------------------------------

\normalsize

% Das esempio-Environment wird nur in der Leseansicht benötigt
\ifkorrekturansicht\else
\newenvironment{esempio}[3]%
{
    \vspace{1.5ex}
    \rlap{\underline{#1}}
    \par
    \setlength{\parindent}{0cm}
    \nopagebreak
    \leftskip=#2cm
    \rightskip=#3cm
}
{
    \par
}
\fi

\doendnotes{C}
\bigskip
\vfill

\clearpage

\footnotesize

\ifkorrekturansicht
  \lohead{\textsc{register}}
\fi

% theindex-Environment neu definieren ohne reledmac
\makeatletter
\renewenvironment{theindex}{%
  \ifkorrekturansicht
    \section*{\indexname}%
  \else
    \subsubsection*{Index der erwähnten Entitäten}%
  \fi
  \setlength{\parindent}{0pt}%
  \setlength{\parskip}{0pt plus 0.3pt}%
  \let\item\@idxitem
}{%
  \ifkorrekturansicht\clearpage\fi
}
\makeatother

\IfFileExists{\jobname-pw.ind}{\input{\jobname-pw.ind}}{}

% Quellenangabe nur in der Leseansicht
\ifkorrekturansicht\else
% Fallback-Definitionen, falls die .tex-Datei \titel etc. nicht gesetzt hat
\providecommand{\titel}{}
\providecommand{\editorInnen}{}
\providecommand{\dateiname}{\jobname}

\vspace{3cm}

\vfill

\footnotesize
\textsc{Quelle}: \titel. Herausgegeben von {\editorInnen}. In: \emph{Arthur Schnitzler: Briefwechsel mit Autorinnen und Autoren}.
 Digitale Edition, https://schnitzler-briefe.acdh.oeaw.ac.at/{\dateiname}.html (Stand \today)
\fi

\end{document}


