%% latex-korrekturansicht-vorspann.tex
%% Vorspann für die Korrekturansicht.
%% Lädt die gemeinsame Datei latex-vorspann.tex mit gesetztem Schalter.

\newif\ifkorrekturansicht
\korrekturansichttrue

\input{../tex-inputs/latex-vorspann}


\section[Arthur Schnitzler an Richard Beer-Hofmann, 28. 7. 1922]{L02554 Arthur Schnitzler an Richard Beer-Hofmann, 28. 7. 1922}
\nopagebreak\mylabel{L02554v}
\rehead{ }\normalsize\beginnumbering\briefempfaengerindex{Beer-Hofmann, Richard@\textsc{Beer-Hofmann, Richard}!zzzSchnitzler, Arthur@\emph{von Arthur Schnitzler}!1922-07-281@{28. 7. 1922}|(be}
\toendnotes[C]{\smallbreak\pagebreak[2]}\Standort{DLA, A:Schnitzler, HS.NZ85.1.342, S. 156.}
\physDesc{Brief, maschinenschriftliche Abschrift1 Blatt, 1 Seite, 374 Zeichen
\newline{}Schreibmaschine}\toendnotes[C]{\smallbreak}
\pstart
           \raggedleft{}{\pb}Feldafing\oindex{Feldafing@\textbf{Feldafing}, \emph{P.PPLA4}|pw},
                  28. 7. 1922.\pend
           
\pstart
           \raggedleft{}Kaiserin Elisabeth\oindex{Hotel Kaiserin Elisabeth [Feldafing]@\textbf{Hotel Kaiserin Elisabeth [Feldafing]}, \emph{Hotel (K.HTL)}|pw}\pend
           \vspace{0.5em}
\pstart
           Lieber Richard – leider konnte ich Sie nicht vor meiner \label{K_L02554-1v}\edtext{Abreise}{\lemma{\textnormal{\emph{Abreise}}}\Cendnote{\textnormal{Siehe A. S.: \emph{Tagebuch}, 25. 7. 1922.
               }}}\label{K_L02554-1} noch einmal sehn. Hier \label{K_L02554-2v}\edtext{sitze
               ich zwar unbesorgt}{\lemma{\textnormal{\emph{sitze
               ich zwar unbesorgt}}}\Cendnote{\textnormal{Siehe Arthur Schnitzler an Richard Beer-Hofmann, 27. 8. 1895, Arthur Schnitzler an Richard Beer-Hofmann, 5. 8. 1912.
               }}}\label{K_L02554-2} (soweit es das gibt), aber in Regen und Nebeln. Doch ist es ein höchst
               behagliches Hotel\oindex{Hotel Kaiserin Elisabeth [Feldafing]@\textbf{Hotel Kaiserin Elisabeth [Feldafing]}, \emph{Hotel (K.HTL)}|pwv} – und
               keineswegs sehnt man sich in die verhängten Berge. Seien Sie und die Ihren herzlichst
               gegrüsst, eine glückliche Reise! Ihr \spacefill\mbox{A.}\pend
           
\pstart
           \noindent{}(nach Wien\oindex{Wien@\textbf{Wien}, \emph{A.ADM2}|pw})\pend
           \selectlanguage{ngerman}\endnumbering\briefempfaengerindex{Beer-Hofmann, Richard@\textsc{Beer-Hofmann, Richard}!zzzSchnitzler, Arthur@\emph{von Arthur Schnitzler}!1922-07-281@{28. 7. 1922}|)be}\mylabel{L02554h}  \normalsize

\doendnotes{C}
\bigskip
\vfill

\clearpage

\footnotesize

\lohead{\textsc{register}}

% Definiere theindex-Environment komplett neu ohne reledmac
\makeatletter
\renewenvironment{theindex}{%
  \section*{\indexname}%
  \setlength{\parindent}{0pt}%
  \setlength{\parskip}{0pt plus 0.3pt}%
  \let\item\@idxitem
}{%
  \clearpage
}
\makeatother

\IfFileExists{\jobname-pw.ind}{\input{\jobname-pw.ind}}{}

\end{document}

      