%% latex-leseansicht-vorspann.tex
%% Vorspann für die Leseansicht.
%% Lädt die gemeinsame Datei latex-vorspann.tex mit nicht gesetztem Schalter.

\newif\ifkorrekturansicht
\korrekturansichtfalse

\input{../tex-inputs/latex-vorspann}


\section[Arthur Schnitzler an Richard Beer-Hofmann, 28. 7. 1922]{L02554 Arthur Schnitzler an Richard Beer-Hofmann, 28. 7. 1922}
\nopagebreak\mylabel{L02554v}
\rehead{ }\normalsize\beginnumbering\briefempfaengerindex{Beer-Hofmann, Richard@\textsc{Beer-Hofmann, Richard}!zzzSchnitzler, Arthur@\emph{von Arthur Schnitzler}!1922-07-281@{28. 7. 1922}|(be}
\toendnotes[C]{\smallbreak\pagebreak[2]}
\correspDesc{Versand  durch Arthur Schnitzler am 28. 7. 1922 in Feldafing
\newline{}Erhalt  durch Richard Beer-Hofmann im Zeitraum [29. 7. 1922
                  – 2. 8. 1922?] in Wien}\toendnotes[C]{\smallbreak}
\Standort{DLA, A:Schnitzler, HS.NZ85.1.342, S. 156.}
\physDesc{Brief, maschinenschriftliche Abschrift, 1 Blatt, 1 Seite, 374 Zeichen
\newline{}Schreibmaschine}\toendnotes[C]{\smallbreak}
\pstart
           \raggedleft{}{\pb}Feldafing\oindex{Feldafing@\textbf{Feldafing}, \emph{Hauptstadt}|pw},
                  28. 7. 1922.\pend
           
\pstart
           \raggedleft{}Kaiserin Elisabeth\oindex{Hotel Kaiserin Elisabeth [Feldafing]@\textbf{Hotel Kaiserin Elisabeth [Feldafing]}, \emph{Hotel}|pw}\pend
           \vspace{0.5em}
\pstart
           Lieber Richard – leider konnte ich Sie nicht vor meiner \label{K_L02554-1v}\edtext{Abreise}{\lemma{\textnormal{\emph{Abreise}}}\Cendnote{\textnormal{Siehe A. S.: \emph{Tagebuch}, 25. 7. 1922.
               }}}\label{K_L02554-1} noch einmal sehn. Hier \label{K_L02554-2v}\edtext{sitze
               ich zwar unbesorgt}{\lemma{\textnormal{\emph{sitze
               ich zwar unbesorgt}}}\Cendnote{\textnormal{Siehe XXXX Auszeichnungsfehler: Dokument L00478 nicht gefunden, XXXX Auszeichnungsfehler: Dokument L02082 nicht gefunden.
               }}}\label{K_L02554-2} (soweit es das gibt), aber in Regen und Nebeln. Doch ist es ein höchst
               behagliches Hotel\oindex{Hotel Kaiserin Elisabeth [Feldafing]@\textbf{Hotel Kaiserin Elisabeth [Feldafing]}, \emph{Hotel}|pwv} – und
               keineswegs sehnt man sich in die verhängten Berge. Seien Sie und die Ihren herzlichst
               gegrüsst, eine glückliche Reise! Ihr \spacefill\mbox{A.}\pend
           
\pstart
           \noindent{}(nach Wien\oindex{Wien@\textbf{Wien}, \emph{Verwaltungsgebiet}|pw})\pend
           \selectlanguage{ngerman}\endnumbering\briefempfaengerindex{Beer-Hofmann, Richard@\textsc{Beer-Hofmann, Richard}!zzzSchnitzler, Arthur@\emph{von Arthur Schnitzler}!1922-07-281@{28. 7. 1922}|)be}\mylabel{L02554h}  \newcommand{\dateiname}{L02554}\newcommand{\titel}{Arthur Schnitzler an Richard Beer-Hofmann, 28. 7. 1922}\newcommand{\editorInnen}{Martin Anton Müller und Gerd-Hermann Susen}%% latex-leseansicht-abspann.tex
%% Abspann für die Leseansicht.
%% Der Schalter \ifkorrekturansicht ist bereits durch den Vorspann gesetzt.

%% latex-abspann.tex
%% Gemeinsamer Abspann für Korrekturansicht und Leseansicht.
%% Setzt den Schalter \ifkorrekturansicht voraus (gesetzt in den
%% einbindenden Dateien latex-korrekturansicht-abspann.tex bzw.
%% latex-leseansicht-abspann.tex).
%% ---------------------------------------------------------------

\normalsize

% Das esempio-Environment wird nur in der Leseansicht benötigt
\ifkorrekturansicht\else
\newenvironment{esempio}[3]%
{
    \vspace{1.5ex}
    \rlap{\underline{#1}}
    \par
    \setlength{\parindent}{0cm}
    \nopagebreak
    \leftskip=#2cm
    \rightskip=#3cm
}
{
    \par
}
\fi

\doendnotes{C}
\bigskip
\vfill

\clearpage

\footnotesize

\ifkorrekturansicht
  \lohead{\textsc{register}}
\fi

% theindex-Environment neu definieren ohne reledmac
\makeatletter
\renewenvironment{theindex}{%
  \ifkorrekturansicht
    \section*{\indexname}%
  \else
    \subsubsection*{Index der erwähnten Entitäten}%
  \fi
  \setlength{\parindent}{0pt}%
  \setlength{\parskip}{0pt plus 0.3pt}%
  \let\item\@idxitem
}{%
  \ifkorrekturansicht\clearpage\fi
}
\makeatother

\IfFileExists{\jobname-pw.ind}{\input{\jobname-pw.ind}}{}

% Quellenangabe nur in der Leseansicht
\ifkorrekturansicht\else
% Fallback-Definitionen, falls die .tex-Datei \titel etc. nicht gesetzt hat
\providecommand{\titel}{}
\providecommand{\editorInnen}{}
\providecommand{\dateiname}{\jobname}

\vspace{3cm}

\vfill

\footnotesize
\textsc{Quelle}: \titel. Herausgegeben von {\editorInnen}. In: \emph{Arthur Schnitzler: Briefwechsel mit Autorinnen und Autoren}.
 Digitale Edition, https://schnitzler-briefe.acdh.oeaw.ac.at/{\dateiname}.html (Stand \today)
\fi

\end{document}


