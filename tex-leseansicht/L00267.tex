%% latex-leseansicht-vorspann.tex
%% Vorspann für die Leseansicht.
%% Lädt die gemeinsame Datei latex-vorspann.tex mit nicht gesetztem Schalter.

\newif\ifkorrekturansicht
\korrekturansichtfalse

\input{../tex-inputs/latex-vorspann}


               \section[Karl Kraus u. a. an Arthur Schnitzler, 30. 9. 1893]{ Karl Kraus u. a. an Arthur Schnitzler, 30. 9. 1893}\nopagebreak\mylabel{v}\rehead{ }\begin{ledgroupsized}[t]{13cm}\normalsize\beginnumbering\briefempfaengerindex{Schnitzler, Arthur@\textsc{Schnitzler, Arthur}!zzzSchaumberger, Julius@\emph{von Julius Schaumberger}!1893-09-301@{30. 9. 1893}|(be}\briefempfaengerindex{Schnitzler, Arthur@\textsc{Schnitzler, Arthur}!zzzRosner, Karl Peter@\emph{von Karl Peter Rosner}!1893-09-301@{30. 9. 1893}|(be}\briefempfaengerindex{Schnitzler, Arthur@\textsc{Schnitzler, Arthur}!zzzKraus, Karl@\emph{von Karl Kraus}!1893-09-301@{30. 9. 1893}|(be} \toendnotes[C]{\smallbreak\pagebreak[2]} \Standort{CUL, Schnitzler, B 55.}
\physDesc{Postkarte
\newline{}Handschrift Karl Kraus: Bleistift, lateinische Kurrent\newline{}Handschrift Karl Peter Rosner: Bleistift, lateinische Kurrent\newline{}Handschrift Julius Schaumberger: Bleistift, deutsche Kurrent\newline{}Versand: Stempel: »\nobreak{}\oindex{Muenchen@\textbf{München}|pwk}München II, 8–9\nobreak{}«.  \newline{}Ordnung: mit Bleistift von unbekannter Hand nummeriert:
                                                »4« }\buchAbdrucke{\weitereDrucke{1) \emph{Karl Kraus und Arthur Schnitzler. Eine Dokumentation.} Hg. Reinhard Urbach. In: \emph{Literatur und Kritik}, Bd. 49, Oktober 1970, S. 515–516.} \weitereDrucke{2) Hermann Bahr, Arthur Schnitzler: \emph{Briefwechsel, Aufzeichnungen, Dokumente
                                (1891–1931)}. Hg. Kurt Ifkovits und Martin Anton Müller. Göttingen: \emph{Wallstein} 2018, S. 43.} }\pstart{}{\pb}\textsc{Herrn D\textsuperscript{r.} Arthur
                            Schnitzler}\pend{}\pstart{}\textsc{Wien I}\oindex{I., Innere Stadt@\textbf{I., Innere Stadt}|pw}\pend{}\pstart{}\textsc{Grillparzerstraße 7. I}\oindex{Grillparzerstrasse@\textbf{Grillparzerstraße}|pw}\pend{}{\bigskip}\pstart
           {\pb}München, Café Luitpold\oindex{Cafe Luitpold@\textbf{Café Luitpold}|pw},
                            30/9 93.\pend
           \pstart
           Liebster Doktor, herzlichſte Grüße.\pend
           \pstart
           Grüßen Sie beſtens auch Beer-Hofmann\pwindex{Beer-Hofmann, Richard 11.07.1866 – 26.09.1945@\textsc{Beer-Hofmann, Richard} (11.07.1866 – 26.09.1945), \emph{Schriftsteller}|pw}{ }\introOben{}Loris\pwindex{Hofmannsthal, Hugo von 01.02.1874 – 15.07.1929@\textsc{Hofmannsthal, Hugo von} (01.02.1874 – 15.07.1929), \emph{Schriftsteller}|pw}\introOben{}. Ich habe Ihnen \uuline{vieles}{ }Sie \substVorne{}\textsuperscript{i}\substDazwischen{}I\substHinten{}ntereſſierende zu ſagen.\pend
           \pstart Ihr \spacefill\mbox{Kraus}\hspace*{1.5em}poste restante\pend{}\pstart
           \noindent{}{[}hs. Rosner:{]} \textsc{Viele innige Grüße an Sie, Hoffmann\pwindex{Beer-Hofmann, Richard 11.07.1866 – 26.09.1945@\textsc{Beer-Hofmann, Richard} (11.07.1866 – 26.09.1945), \emph{Schriftsteller}|pw}, Loris\pwindex{Hofmannsthal, Hugo von 01.02.1874 – 15.07.1929@\textsc{Hofmannsthal, Hugo von} (01.02.1874 – 15.07.1929), \emph{Schriftsteller}|pw}, Bahr\pwindex{Bahr, Hermann 19.07.1863 – 15.01.1934@\textsc{Bahr, Hermann} (19.07.1863 – 15.01.1934), \emph{Schriftsteller, Kritiker}|pw}}\pend
           \pstart
           \textsc{Ihr treuer}{\\[\baselineskip]}\spacefill\mbox{Karl Rosner.}\pend
           \leftskip=0em{}\pstart
           \noindent{}{[}hs. Kraus:{]} Dieſer Mensch hat ſich \uline{ſehr} gebeſſert, alle Poſen ſich abgewöhnt. \spacefill\mbox{Kraus}\pend
           \pstart
           \noindent{}\centering{}–------\pend
           \pstart
           \noindent{}{[}hs. Schaumberger:{]} \textsc{Prosit}\spacefill\mbox{JSchaumberger}\pend
                     \endnumbering\briefempfaengerindex{Schnitzler, Arthur@\textsc{Schnitzler, Arthur}!zzzSchaumberger, Julius@\emph{von Julius Schaumberger}!1893-09-301@{30. 9. 1893}|)be}\briefempfaengerindex{Schnitzler, Arthur@\textsc{Schnitzler, Arthur}!zzzRosner, Karl Peter@\emph{von Karl Peter Rosner}!1893-09-301@{30. 9. 1893}|)be}\briefempfaengerindex{Schnitzler, Arthur@\textsc{Schnitzler, Arthur}!zzzKraus, Karl@\emph{von Karl Kraus}!1893-09-301@{30. 9. 1893}|)be}\mylabel{h}\end{ledgroupsized}  \newcommand{\dateiname}{L00267}\newcommand{\titel}{Karl Kraus u. a. an Arthur Schnitzler, 30. 9. 1893}\newcommand{\editorInnen}{ Martin Anton Müller und Gerd-Hermann Susen}%% latex-leseansicht-abspann.tex
%% Abspann für die Leseansicht.
%% Der Schalter \ifkorrekturansicht ist bereits durch den Vorspann gesetzt.

%% latex-abspann.tex
%% Gemeinsamer Abspann für Korrekturansicht und Leseansicht.
%% Setzt den Schalter \ifkorrekturansicht voraus (gesetzt in den
%% einbindenden Dateien latex-korrekturansicht-abspann.tex bzw.
%% latex-leseansicht-abspann.tex).
%% ---------------------------------------------------------------

\normalsize

% Das esempio-Environment wird nur in der Leseansicht benötigt
\ifkorrekturansicht\else
\newenvironment{esempio}[3]%
{
    \vspace{1.5ex}
    \rlap{\underline{#1}}
    \par
    \setlength{\parindent}{0cm}
    \nopagebreak
    \leftskip=#2cm
    \rightskip=#3cm
}
{
    \par
}
\fi

\doendnotes{C}
\bigskip
\vfill

\clearpage

\footnotesize

\ifkorrekturansicht
  \lohead{\textsc{register}}
\fi

% theindex-Environment neu definieren ohne reledmac
\makeatletter
\renewenvironment{theindex}{%
  \ifkorrekturansicht
    \section*{\indexname}%
  \else
    \subsubsection*{Index der erwähnten Entitäten}%
  \fi
  \setlength{\parindent}{0pt}%
  \setlength{\parskip}{0pt plus 0.3pt}%
  \let\item\@idxitem
}{%
  \ifkorrekturansicht\clearpage\fi
}
\makeatother

\IfFileExists{\jobname-pw.ind}{\input{\jobname-pw.ind}}{}

% Quellenangabe nur in der Leseansicht
\ifkorrekturansicht\else
% Fallback-Definitionen, falls die .tex-Datei \titel etc. nicht gesetzt hat
\providecommand{\titel}{}
\providecommand{\editorInnen}{}
\providecommand{\dateiname}{\jobname}

\vspace{3cm}

\vfill

\footnotesize
\textsc{Quelle}: \titel. Herausgegeben von {\editorInnen}. In: \emph{Arthur Schnitzler: Briefwechsel mit Autorinnen und Autoren}.
 Digitale Edition, https://schnitzler-briefe.acdh.oeaw.ac.at/{\dateiname}.html (Stand \today)
\fi

\end{document}


      