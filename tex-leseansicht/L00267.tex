%% latex-korrekturansicht-vorspann.tex
%% Vorspann für die Korrekturansicht.
%% Lädt die gemeinsame Datei latex-vorspann.tex mit gesetztem Schalter.

\newif\ifkorrekturansicht
\korrekturansichttrue

\input{../tex-inputs/latex-vorspann}


\section[Karl Kraus u. a. an Arthur Schnitzler, 30. 9. 1893]{L00267 Karl Kraus u. a. an Arthur Schnitzler, 30. 9. 1893}
\nopagebreak\mylabel{L00267v}
\rehead{ }\normalsize\beginnumbering\briefempfaengerindex{Schnitzler, Arthur@\textsc{Schnitzler, Arthur}!zzzSchaumberger, Julius@\emph{von Julius Schaumberger}!1893-09-301@{30. 9. 1893}|(be}\briefempfaengerindex{Schnitzler, Arthur@\textsc{Schnitzler, Arthur}!zzzRosner, Karl Peter@\emph{von Karl Peter Rosner}!1893-09-301@{30. 9. 1893}|(be}\briefempfaengerindex{Schnitzler, Arthur@\textsc{Schnitzler, Arthur}!zzzKraus, Karl@\emph{von Karl Kraus}!1893-09-301@{30. 9. 1893}|(be}
\toendnotes[C]{\smallbreak\pagebreak[2]}\Standort{CUL, Schnitzler, B 55.}
\physDesc{Postkarte, 396 Zeichen
\newline{}Handschrift Karl Kraus: 1) Bleistift, deutsche Kurrent\hspace{1em}2) Bleistift, lateinische Kurrent (\noindent{}Adresse)\hspace{1em}
\newline{}Handschrift Karl Peter Rosner: Bleistift, lateinische Kurrent
\newline{}Handschrift Julius Schaumberger: Bleistift, deutsche Kurrent
\newline{}Versand: Stempel: »\nobreak{}\oindex{Muenchen@\textbf{München}, \emph{P.PPLA}|pwk}München II, 8–9\nobreak{}«.  
\newline{}Ordnung: mit Bleistift von unbekannter Hand nummeriert:
                                 »4« }
\buchAbdrucke{\weitereDrucke{1) \emph{Literatur und Kritik}, Bd. 49, Oktober 1970, S. 515–516.} \weitereDrucke{2) Hermann Bahr, Arthur Schnitzler: \emph{Briefwechsel, Aufzeichnungen, Dokumente (1891–1931)}. Göttingen: \emph{Wallstein} 2018, S. 43.} }\pstart{}{\pb}Herrn D\textsuperscript{r.}
                  Arthur Schnitzler\pend{}\pstart{}Wien I\oindex{I., Innere Stadt@\textbf{I., Innere Stadt}, \emph{A.ADM3}|pw}\pend{}\pstart{}Grillparzerstraße 7. I\oindex{Grillparzerstrasse@\textbf{Grillparzerstraße}, \emph{R.ST}|pw}\pend{}{\bigskip}\vspace{1em}
\pstart
           {\pb}München, Café Luitpold\oindex{Cafe Luitpold@\textbf{Café Luitpold}, \emph{Kaffeehaus (K.KAF)}|pw},
                  30/9 93.\pend
           \vspace{0.5em}
\pstart
           Liebster Doktor, herzlichſte Grüße.\pend
           
\pstart
           Grüßen Sie beſtens auch Beer-Hofmann\pwindex{Beer-Hofmann, Richard 1866-07-11 – 1945-09-26@\textsc{Beer-Hofmann, Richard} (1866-07-11 – 1945-09-26), \emph{Schriftsteller/Schriftstellerin}|pw}{ }\introOben{}Loris\pwindex{Hofmannsthal, Hugo von 1874-02-01 – 1929-07-15@\textsc{Hofmannsthal, Hugo von} (1874-02-01 – 1929-07-15), \emph{Schriftsteller/Schriftstellerin}|pw}\introOben{}. Ich habe Ihnen \uuline{vieles}{ }Sie \substVorne{}\textsuperscript{i}\substDazwischen{}I\substHinten{}ntereſſierende zu ſagen.\pend
           \pstart Ihr \spacefill\mbox{Kraus}\hspace*{1.5em}poste restante\pend{}\selectlanguage{ngerman}\vspace{1em}
\pstart
           \noindent{}{[}hs. :{]} \textsc{Viele innige Grüße an Sie, Hoffmann\pwindex{Beer-Hofmann, Richard 1866-07-11 – 1945-09-26@\textsc{Beer-Hofmann, Richard} (1866-07-11 – 1945-09-26), \emph{Schriftsteller/Schriftstellerin}|pw}, Loris\pwindex{Hofmannsthal, Hugo von 1874-02-01 – 1929-07-15@\textsc{Hofmannsthal, Hugo von} (1874-02-01 – 1929-07-15), \emph{Schriftsteller/Schriftstellerin}|pw}, Bahr\pwindex{Bahr, Hermann 19.07.1863 – 15.01.1934@\textsc{Bahr, Hermann} (19.07.1863 – 15.01.1934), \emph{Schriftsteller/Schriftstellerin, Kritiker/Kritikerin}|pw}}\pend
           
\pstart
           \textsc{Ihr treuer}{\\[\baselineskip]}\spacefill\mbox{Karl Rosner.}\pend
           \leftskip=0em{}\selectlanguage{ngerman}\vspace{1em}
\pstart
           \noindent{}{[}hs. :{]} Dieſer Mensch hat ſich \uline{ſehr} gebeſſert, alle Poſen ſich abgewöhnt. \spacefill\mbox{Kraus}\pend
           \selectlanguage{ngerman}\vspace{1em}
\pstart
           \noindent{}\centering{}–------\pend
           
\pstart
           {[}hs. :{]} \textsc{Prosit}\spacefill\mbox{JSchaumberger}\pend
           \selectlanguage{ngerman}\endnumbering\briefempfaengerindex{Schnitzler, Arthur@\textsc{Schnitzler, Arthur}!zzzSchaumberger, Julius@\emph{von Julius Schaumberger}!1893-09-301@{30. 9. 1893}|)be}\briefempfaengerindex{Schnitzler, Arthur@\textsc{Schnitzler, Arthur}!zzzRosner, Karl Peter@\emph{von Karl Peter Rosner}!1893-09-301@{30. 9. 1893}|)be}\briefempfaengerindex{Schnitzler, Arthur@\textsc{Schnitzler, Arthur}!zzzKraus, Karl@\emph{von Karl Kraus}!1893-09-301@{30. 9. 1893}|)be}\mylabel{L00267h}  \normalsize

\doendnotes{C}
\bigskip
\vfill

\clearpage

\footnotesize

\lohead{\textsc{register}}

% Definiere theindex-Environment komplett neu ohne reledmac
\makeatletter
\renewenvironment{theindex}{%
  \section*{\indexname}%
  \setlength{\parindent}{0pt}%
  \setlength{\parskip}{0pt plus 0.3pt}%
  \let\item\@idxitem
}{%
  \clearpage
}
\makeatother

\IfFileExists{\jobname-pw.ind}{\input{\jobname-pw.ind}}{}

\end{document}

      