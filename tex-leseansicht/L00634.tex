%% latex-korrekturansicht-vorspann.tex
%% Vorspann für die Korrekturansicht.
%% Lädt die gemeinsame Datei latex-vorspann.tex mit gesetztem Schalter.

\newif\ifkorrekturansicht
\korrekturansichttrue

\input{../tex-inputs/latex-vorspann}


\section[Arthur Schnitzler an Richard Beer-Hofmann, {[}1897?{]}]{L00634 Arthur Schnitzler an Richard Beer-Hofmann, {[}1897?{]}}
\nopagebreak\mylabel{L00634v}
\rehead{ }\normalsize\beginnumbering\briefempfaengerindex{Beer-Hofmann, Richard@\textsc{Beer-Hofmann, Richard}!zzzSchnitzler, Arthur@\emph{von Arthur Schnitzler}!1897-12-311@{{[}1897?{]}}|(be}
\toendnotes[C]{\smallbreak\pagebreak[2]}\Standort{YCGL, MSS 31.}
\physDesc{Visitenkarte, 97 Zeichen
\newline{}Handschrift: Bleistift, deutsche Kurrent}\toendnotes[C]{\smallbreak}
\pstart
           \noindent{}{\pb}Lieber Richard,{ }\label{K_L00634-1v}\edtext{ich bin}{\lemma{\textnormal{\emph{ich bin}}}\Cendnote{\textnormal{Die undatierte Visitenkarte wird zwischen Korrespondenzstücken des
                  Jahres 1897 aufbewahrt, was den einzigen Hinweis auf die zeitliche
                  Einordnung gibt.}}}\label{K_L00634-1} im Theater, I Rang, Loge, 3, rechts.\pend
           
\pstart
           We{\geminationn} Sie Luſt haben, {\pb}kommen Sie!\pend
           
\pstart
           \centering{}Herzlichſt\pend
           
\pstart
           \centering{}\textcolor{gray}{\textbf{D\textsuperscript{r} Arthur Schnitzler}}\pend
           \selectlanguage{ngerman}\endnumbering\briefempfaengerindex{Beer-Hofmann, Richard@\textsc{Beer-Hofmann, Richard}!zzzSchnitzler, Arthur@\emph{von Arthur Schnitzler}!1897-01-011@{{[}1897?{]}}|)be}\mylabel{L00634h}  \normalsize

\doendnotes{C}
\bigskip
\vfill

\clearpage

\footnotesize

\lohead{\textsc{register}}

% Definiere theindex-Environment komplett neu ohne reledmac
\makeatletter
\renewenvironment{theindex}{%
  \section*{\indexname}%
  \setlength{\parindent}{0pt}%
  \setlength{\parskip}{0pt plus 0.3pt}%
  \let\item\@idxitem
}{%
  \clearpage
}
\makeatother

\IfFileExists{\jobname-pw.ind}{\input{\jobname-pw.ind}}{}

\end{document}

      