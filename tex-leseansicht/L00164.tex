%% latex-leseansicht-vorspann.tex
%% Vorspann für die Leseansicht.
%% Lädt die gemeinsame Datei latex-vorspann.tex mit nicht gesetztem Schalter.

\newif\ifkorrekturansicht
\korrekturansichtfalse

\input{../tex-inputs/latex-vorspann}


\section[Karl Kraus an Arthur Schnitzler, 26. 1. 1893]{L00164 Karl Kraus an Arthur Schnitzler, 26. 1. 1893}
\nopagebreak\mylabel{L00164v}
\rehead{ }\normalsize\beginnumbering\briefempfaengerindex{Schnitzler, Arthur@\textsc{Schnitzler, Arthur}!zzzKraus, Karl@\emph{von Karl Kraus}!1893-01-261@{26. 1. 1893}|(be}
\toendnotes[C]{\smallbreak\pagebreak[2]}
\correspDesc{Versand  durch Karl Kraus am 26. 1. 1893 in Wien
\newline{}Erhalt  durch Arthur Schnitzler am 26. 1. 1893 in Wien}\toendnotes[C]{\smallbreak}
\Standort{CUL, Schnitzler, B 55.}
\physDesc{Kartenbrief, 563 Zeichen
\newline{}Handschrift: schwarze Tinte, deutsche Kurrent
\newline{}Versand: 1) Stempel: »\nobreak{}\oindex{Wien@\textbf{Wien}, \emph{Verwaltungsgebiet}|pwk}Wien 1/1, 26. 1 {[}93{]}, 11–12V\nobreak{}«.   2) Stempel: »\nobreak{}\oindex{Wien@\textbf{Wien}, \emph{Verwaltungsgebiet}|pwk}Wien 1/1, 26/1. 93, 1–2½ N\nobreak{}«. 
\newline{}Schnitzler: mit Bleistift auf der Textseite beschriftet: »Wienerh\oindex{Wiener Hof@\textbf{Wiener Hof}, \emph{Hotel}|pw}{ }Mauerst 20\oindex{Mauerstraße@\textbf{Mauerstraße}, \emph{Straße}|pw}« }
\buchAbdrucke{\weitereDrucke{\emph{Karl Kraus und Arthur Schnitzler. Eine Dokumentation.}Herausgegeben von Reinhard Urbach In: \emph{Literatur und Kritik}, Bd. 49, Oktober 1970, S. 515.} }\toendnotes[C]{\smallbreak}\pstart{}{\pb}Herrn Schriftſteller\pend{}\pstart{}D\textsuperscript{r} med Arthur Schnitzler\pend{}\pstart{}Grillparzerstr. 7\oindex{Wien@\textbf{Wien}!I., Innere Stadt@\textbf{I., Innere Stadt}!Grillparzerstraße@\textbf{Grillparzerstraße}, \emph{Straße}|pw}\pend{}\pstart{}Wien I.\oindex{I., Innere Stadt@\textbf{I., Innere Stadt}, \emph{Verwaltungsgebiet}|pw}\pend{}{\bigskip}\vspace{1em}
\pstart{}{\pb}Lieber Doctor Schnitzler!\pend\vspace{0.5em}
\pstart
           Otto Julius Bierbaum\pwindex{Bierbaum, Otto Julius 28.\,6.\,1865 Zielona Góra – 1.\,2.\,1910 Dresden@\textsc{Bierbaum, Otto Julius} (28.\,6.\,1865 Zielona Góra – 1.\,2.\,1910 Dresden)|pw} fordert Sie durch mich
               auf, ihm was für{ }ſeinen Mod. Muſen-Almanach 1894\pwindex{Moderner Musen-Almanach auf das Jahr 1894. Ein Jahrbuch deutscher Kunst@\emph{Moderner Musen-Almanach auf das Jahr 1894. Ein Jahrbuch deutscher Kunst}|pw}
               zukommen zu laſſen. Der Almanach erſcheint 1. Septemb. 93. Endtermin für
               die Einſendung\pwindex{Schnitzler, Arthur 15.\,5.\,1862 Wien – 21.\,10.\,1931 ebd.@\textsc{Schnitzler, Arthur} (15.\,5.\,1862 Wien – 21.\,10.\,1931 ebd.), \emph{Schriftsteller, Mediziner}!drei Elixire@\strich\emph{Die drei Elixire}|pwv}{ }\uline{1. Juli}. Adreſſe: O. J. Bierbaum\pwindex{Bierbaum, Otto Julius 28.\,6.\,1865 Zielona Góra – 1.\,2.\,1910 Dresden@\textsc{Bierbaum, Otto Julius} (28.\,6.\,1865 Zielona Góra – 1.\,2.\,1910 Dresden)|pw}, \uline{Oberbayern}\oindex{Oberbayern@\textbf{Oberbayern}, \emph{Verwaltungsgebiet}|pw}: \uline{Post Beuerberg\oindex{Beuerberg@\textbf{Beuerberg}|pw}}; \uline{Auf der Öd}\oindex{Auf der Öd@\textbf{Auf der Öd}, \emph{Straße}|pw}.\pend
           
\pstart
           Über Ihren Anatol\pwindex{Schnitzler, Arthur 15.\,5.\,1862 Wien – 21.\,10.\,1931 ebd.@\textsc{Schnitzler, Arthur} (15.\,5.\,1862 Wien – 21.\,10.\,1931 ebd.), \emph{Schriftsteller, Mediziner}!Anatol@\strich\emph{Anatol}|pw}{ }ſchreibe ich einige \label{K_L00164-1v}\edtext{Zeilen}{\lemma{\textnormal{\emph{Zeilen}}}\Cendnote{\textnormal{nicht
                  erschienen}}}\label{K_L00164-1} für\strikeout{’s}{ }N. l. Bl.\orgindex{Neue litterarische Blätter@Neue litterarische Blätter|pw} (Bremen\oindex{Bremen@\textbf{Bremen}|pw}) 1. März, welche N\textsuperscript{r.} in
               4–5000 Ex. erſcheinen wird. Demnächſt erhalten Sie von mir Druckſorte: Aufforderung
               zur Satirenanthologie.\pend
           \pstart Gruß u. Handſchlag. Ihr \spacefill\mbox{Karl Kraus.}\pend{}\selectlanguage{ngerman}\endnumbering\briefempfaengerindex{Schnitzler, Arthur@\textsc{Schnitzler, Arthur}!zzzKraus, Karl@\emph{von Karl Kraus}!1893-01-261@{26. 1. 1893}|)be}\mylabel{L00164h}  \newcommand{\dateiname}{L00164}\newcommand{\titel}{Karl Kraus an Arthur Schnitzler, 26. 1. 1893}\newcommand{\editorInnen}{Martin Anton Müller und Gerd-Hermann Susen}%% latex-leseansicht-abspann.tex
%% Abspann für die Leseansicht.
%% Der Schalter \ifkorrekturansicht ist bereits durch den Vorspann gesetzt.

%% latex-abspann.tex
%% Gemeinsamer Abspann für Korrekturansicht und Leseansicht.
%% Setzt den Schalter \ifkorrekturansicht voraus (gesetzt in den
%% einbindenden Dateien latex-korrekturansicht-abspann.tex bzw.
%% latex-leseansicht-abspann.tex).
%% ---------------------------------------------------------------

\normalsize

% Das esempio-Environment wird nur in der Leseansicht benötigt
\ifkorrekturansicht\else
\newenvironment{esempio}[3]%
{
    \vspace{1.5ex}
    \rlap{\underline{#1}}
    \par
    \setlength{\parindent}{0cm}
    \nopagebreak
    \leftskip=#2cm
    \rightskip=#3cm
}
{
    \par
}
\fi

\doendnotes{C}
\bigskip
\vfill

\clearpage

\footnotesize

\ifkorrekturansicht
  \lohead{\textsc{register}}
\fi

% theindex-Environment neu definieren ohne reledmac
\makeatletter
\renewenvironment{theindex}{%
  \ifkorrekturansicht
    \section*{\indexname}%
  \else
    \subsubsection*{Index der erwähnten Entitäten}%
  \fi
  \setlength{\parindent}{0pt}%
  \setlength{\parskip}{0pt plus 0.3pt}%
  \let\item\@idxitem
}{%
  \ifkorrekturansicht\clearpage\fi
}
\makeatother

\IfFileExists{\jobname-pw.ind}{\input{\jobname-pw.ind}}{}

% Quellenangabe nur in der Leseansicht
\ifkorrekturansicht\else
% Fallback-Definitionen, falls die .tex-Datei \titel etc. nicht gesetzt hat
\providecommand{\titel}{}
\providecommand{\editorInnen}{}
\providecommand{\dateiname}{\jobname}

\vspace{3cm}

\vfill

\footnotesize
\textsc{Quelle}: \titel. Herausgegeben von {\editorInnen}. In: \emph{Arthur Schnitzler: Briefwechsel mit Autorinnen und Autoren}.
 Digitale Edition, https://schnitzler-briefe.acdh.oeaw.ac.at/{\dateiname}.html (Stand \today)
\fi

\end{document}


