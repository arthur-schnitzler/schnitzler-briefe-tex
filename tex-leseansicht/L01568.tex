%% latex-leseansicht-vorspann.tex
%% Vorspann für die Leseansicht.
%% Lädt die gemeinsame Datei latex-vorspann.tex mit nicht gesetztem Schalter.

\newif\ifkorrekturansicht
\korrekturansichtfalse

\input{../tex-inputs/latex-vorspann}


\section[Max Burckhard an Arthur Schnitzler, 30. 11. 1905]{L01568 Max Burckhard an Arthur Schnitzler, 30. 11. 1905}
\nopagebreak\mylabel{L01568v}
\rehead{ }\normalsize\beginnumbering\briefempfaengerindex{Schnitzler, Arthur@\textsc{Schnitzler, Arthur}!zzzBurckhard, Max Eugen@\emph{von Max Eugen Burckhard}!1905-11-301@{30. 11. 1905}|(be}
\toendnotes[C]{\smallbreak\pagebreak[2]}
\correspDesc{Versand  durch Max Burckhard am 30. 11. 1905 in St. Gilgen
\newline{}Erhalt  durch Arthur Schnitzler im Zeitraum [1. 12. 1905
                  – 5. 12. 1905?] in Wien}\toendnotes[C]{\smallbreak}
\Standort{CUL, Schnitzler, B 20.}
\physDesc{Brief, 1 Blatt, 2 Seiten, 528 Zeichen
\newline{}Handschrift: schwarze Tinte, deutsche Kurrent
\newline{}Schnitzler: mit Bleistift beschriftet: »B« und datiert: »1905?« 
\newline{}Ordnung: mit Bleistift von unbekannter Hand nummeriert: »15« }\toendnotes[C]{\smallbreak}
\pstart
           \raggedleft{}{\pb}St. Gilgen\oindex{St. Gilgen@\textbf{St. Gilgen}, \emph{Verwaltungsgebiet}|pw}{ }30/11 05\pend
           
\pstart{}Sehr verehrter lieber Herr Doctor!\pend\vspace{0.5em}
\pstart
           Herzlichſten Dank für das »Zwischenſpiel\pwindex{Schnitzler, Arthur 15.\,5.\,1862 Wien – 21.\,10.\,1931 ebd.@\textsc{Schnitzler, Arthur} (15.\,5.\,1862 Wien – 21.\,10.\,1931 ebd.), \emph{Schriftsteller, Mediziner}!Zwischenspiel. Komödie in drei Akten@\strich\emph{Zwischenspiel. Komödie in drei Akten}|pw}«, das
               ich noch nicht gekannt hatte und das einen außerordentlich tiefen Eindruck auf mich
               gemacht hat – beſonders dadurch vielleicht, daſs die eigenthümliche Sti{\geminationm}ung, {\pb}mit
               der es{ }ſchon einſetzt,{ }ſo außerordentlich feſtgehalten iſt bis zum letzten
               Augenblick.\pend
           
\pstart
           Auf baldiges Wiederſehen, denn jetzt geht der Sommer zur Neige.\pend
           
\pstart
           Mit Handkuſs an Ihre verehrte Gattin\pwindex{Schnitzler, Olga 17.\,1.\,1882 Wien – 13.\,1.\,1970 Lugano@\textsc{Schnitzler, Olga} (17.\,1.\,1882 Wien – 13.\,1.\,1970 Lugano), \emph{Schauspielerin, Sängerin}|pwv} u herzlichſte Grüße\pend
           
\pstart
           Ihr getreuer{\\[\baselineskip]}\spacefill\mbox{D\textsuperscript{r}Burckhard}\pend
           \leftskip=0em{}
\pstart
           \noindent{}Ich gratuliere noch zum \label{K_L01568-1v}\edtext{Berlin\oindex{Berlin@\textbf{Berlin}, \emph{Hauptstadt}|pw}er Erfolg}{\lemma{\textnormal{\emph{Berliner Erfolg}}}\Cendnote{\textnormal{Am 25. 11. 1905 hatte die Premiere\eventindex{Lessing-Theater@\textbf{Lessing-Theater}!Premiere von Zwischenspiel, 25.11.1905@Premiere von Zwischenspiel, 25.11.1905|pwkv} von \emph{Zwischenspiel}\pwindex{Schnitzler, Arthur 15.\,5.\,1862 Wien – 21.\,10.\,1931 ebd.@\textsc{Schnitzler, Arthur} (15.\,5.\,1862 Wien – 21.\,10.\,1931 ebd.), \emph{Schriftsteller, Mediziner}!Zwischenspiel. Komödie in drei Akten@\strich\emph{Zwischenspiel. Komödie in drei Akten}|pwk} am \emph{Lessing-Theater}\orgindex{Lessing-Theater@Lessing-Theater|pwk} stattgefunden, etwas über einen Monat nach der Wien\oindex{Wien@\textbf{Wien}, \emph{Verwaltungsgebiet}|pwk}er Uraufführung\eventindex{Burgtheater@\textbf{Burgtheater}!Uraufführung von Zwischenspiel, 12.10.1905@Uraufführung von Zwischenspiel, 12.10.1905|pwkv}.}}}\label{K_L01568-1}\pend
           \selectlanguage{ngerman}\endnumbering\briefempfaengerindex{Schnitzler, Arthur@\textsc{Schnitzler, Arthur}!zzzBurckhard, Max Eugen@\emph{von Max Eugen Burckhard}!1905-11-301@{30. 11. 1905}|)be}\mylabel{L01568h}  \newcommand{\dateiname}{L01568}\newcommand{\titel}{Max Burckhard an Arthur Schnitzler, 30. 11. 1905}\newcommand{\editorInnen}{Martin Anton Müller und Gerd-Hermann Susen}%% latex-leseansicht-abspann.tex
%% Abspann für die Leseansicht.
%% Der Schalter \ifkorrekturansicht ist bereits durch den Vorspann gesetzt.

%% latex-abspann.tex
%% Gemeinsamer Abspann für Korrekturansicht und Leseansicht.
%% Setzt den Schalter \ifkorrekturansicht voraus (gesetzt in den
%% einbindenden Dateien latex-korrekturansicht-abspann.tex bzw.
%% latex-leseansicht-abspann.tex).
%% ---------------------------------------------------------------

\normalsize

% Das esempio-Environment wird nur in der Leseansicht benötigt
\ifkorrekturansicht\else
\newenvironment{esempio}[3]%
{
    \vspace{1.5ex}
    \rlap{\underline{#1}}
    \par
    \setlength{\parindent}{0cm}
    \nopagebreak
    \leftskip=#2cm
    \rightskip=#3cm
}
{
    \par
}
\fi

\doendnotes{C}
\bigskip
\vfill

\clearpage

\footnotesize

\ifkorrekturansicht
  \lohead{\textsc{register}}
\fi

% theindex-Environment neu definieren ohne reledmac
\makeatletter
\renewenvironment{theindex}{%
  \ifkorrekturansicht
    \section*{\indexname}%
  \else
    \subsubsection*{Index der erwähnten Entitäten}%
  \fi
  \setlength{\parindent}{0pt}%
  \setlength{\parskip}{0pt plus 0.3pt}%
  \let\item\@idxitem
}{%
  \ifkorrekturansicht\clearpage\fi
}
\makeatother

\IfFileExists{\jobname-pw.ind}{\input{\jobname-pw.ind}}{}

% Quellenangabe nur in der Leseansicht
\ifkorrekturansicht\else
% Fallback-Definitionen, falls die .tex-Datei \titel etc. nicht gesetzt hat
\providecommand{\titel}{}
\providecommand{\editorInnen}{}
\providecommand{\dateiname}{\jobname}

\vspace{3cm}

\vfill

\footnotesize
\textsc{Quelle}: \titel. Herausgegeben von {\editorInnen}. In: \emph{Arthur Schnitzler: Briefwechsel mit Autorinnen und Autoren}.
 Digitale Edition, https://schnitzler-briefe.acdh.oeaw.ac.at/{\dateiname}.html (Stand \today)
\fi

\end{document}


