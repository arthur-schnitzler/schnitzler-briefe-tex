\input{../tex-inputs/latex-pdf-vorspann}
\begin{center}
            \textcolor{red}{ENTWURF. ENTZIFFERUNG NOCH NICHT KORREKTURGELESEN}
                      \end{center}
            
               \section[Stefan Großmann an Arthur Schnitzler, 29. 9. 1907]{ Stefan Großmann an Arthur Schnitzler, 29. 9. 1907}\nopagebreak\mylabel{v}\rehead{ }\begin{ledgroupsized}[t]{13cm}\normalsize\beginnumbering\briefempfaengerindex{Schnitzler, Arthur@\textsc{Schnitzler, Arthur}!zzzGrossmann, Stefan@\emph{von Stefan Großmann}!1907-09-292@{29. 9. 1907}|(be} \toendnotes[C]{\smallbreak\pagebreak[2]} \Standort{CUL, Schnitzler, B 34.}
\physDesc{Brief, 1 Blatt (Briefpapier mit Trauerrand), 1 Seite
\newline{}Handschrift: schwarze Tinte, deutsche Kurrent\newline{}Ordnung: mit Bleistift von unbekannter Hand nummeriert:
                                 »5« }\toendnotes[C]{\smallbreak}\pstart
           \noindent{}{\pb}\textcolor{gray}{\textbf{Freie Volksbühne\orgindex{Wiener Freie Volksbuehne@Wiener Freie Volksbühne|pw}}}\pend
           \pstart
           \textcolor{gray}{\textbf{Wien VI/\textsubscript{1}\oindex{VI., Mariahilf@\textbf{VI., Mariahilf}|pw}.}}\pend
           \pstart
           \textcolor{gray}{\textbf{Mariahilferſtraße Nr. 89.\oindex{Mariahilferstrasse@\textbf{Mariahilferstraße}|pw}}}\hfill \textcolor{gray}{\textbf{Wien\oindex{Wien@\textbf{Wien}|pw}, am}}{ }29. \damage{\textcolor{gray}{Se}}pt. \textcolor{gray}{\textbf{190}}7\pend
           \pstart
           \textcolor{gray}{\textbf{Poſtſparkaſſen-Konto Nr. 87.544.}}\pend
           \pstart\center{}Verehrter Herr,\pend\pstart
           Danke für Ihre Bereitwilligkeit.\pend
           \pstart
           Wir werden alle Ihre Wünſche berückſichtigen u den Abend am \uline{17. Oktober} in einem kleinen (500 Leute faſſenden) Saal veranſtalten. Herr Abg \uline{\textsc{Pernerstorfer\pwindex{Pernerstorfer, Engelbert 27.04.1850 – 06.01.1918@\textsc{Pernerstorfer, Engelbert} (27.04.1850 – 06.01.1918), \emph{Politiker/Politikerin, Journalist/Journalistin}|pw}}} wird den Abend mit einem kleinen Vortrag eröffnen. Dann leſen Sie, verehrter
               Herr, vielleicht den »\label{T_L01714_1v}\edtext{\textsc{\uline{Lieutenant}}}{\lemma{\textnormal{\emph{Lieutenant}}}\Cendnote{\textnormal{Er schreibt:
                        »Leuitenant«.}}}\label{T_L01714_1h}\textsc{\uline{{ }Gustl}}\pwindex{Schnitzler, Arthur 15.05.1862 – 21.10.1931@\textsc{Schnitzler, Arthur} (15.05.1862 – 21.10.1931), \emph{Schriftsteller, Mediziner}!Lieutenant Gustl. Novelle25. 12. 1900@\strich\emph{Lieutenant Gustl. Novelle} {[}25. 12. 1900{]}|pw}« und irgendeine kleine Arbeit\pwindex{Schnitzler, Arthur 15.05.1862 – 21.10.1931@\textsc{Schnitzler, Arthur} (15.05.1862 – 21.10.1931), \emph{Schriftsteller, Mediziner}!neue Lied23. 04. 1905@\strich\emph{Das neue Lied} {[}23. 04. 1905{]}|pwv}.\pend
           \pstart
           Wir bitten uns \uline{recht bald} Ihre Zust\textcolor{gray}{i{\geminationm}un}g definitiv zu übermitteln, da wir
               14 Tage vorher die Ankündigung in unſeren Vereinsmittheilungen loslaſſen müſſen.\pend
           \pstart
           Aufrichtig dankend{\\[\baselineskip]}ſehr ergeben{\\[\baselineskip]}f. d. Fr. V.\orgindex{Wiener Freie Volksbuehne@Wiener Freie Volksbühne|pw}{\\[\baselineskip]}\spacefill\mbox{GroßSt}\pend
           \leftskip=0em{}\endnumbering\briefempfaengerindex{Schnitzler, Arthur@\textsc{Schnitzler, Arthur}!zzzGrossmann, Stefan@\emph{von Stefan Großmann}!1907-09-292@{29. 9. 1907}|)be}\mylabel{h}\end{ledgroupsized}  \newcommand{\dateiname}{L01714}\newcommand{\titel}{Stefan Großmann an Arthur Schnitzler, 29. 9. 1907}\newcommand{\editorInnen}{ Martin Anton Müller und Gerd-Hermann Susen}\input{../tex-inputs/latex-pdf-abspann}
      