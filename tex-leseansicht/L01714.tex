%% latex-leseansicht-vorspann.tex
%% Vorspann für die Leseansicht.
%% Lädt die gemeinsame Datei latex-vorspann.tex mit nicht gesetztem Schalter.

\newif\ifkorrekturansicht
\korrekturansichtfalse

\input{../tex-inputs/latex-vorspann}


         
         \renewcommand{\erwaehntePersonen}{Personen: Engelbert Pernerstorfer}
         \renewcommand{\erwaehnteInstitutionen}{Institutionen: Wiener Freie Volksbühne}
         \renewcommand{\erwaehnteOrte}{Orte: Mariahilferstraße, VI., Mariahilf, Wien}
         \renewcommand{\erwaehnteWerke}{Werke: Das neue Lied, Lieutenant Gustl. Novelle}
               \section[Stefan Großmann an Arthur Schnitzler, 29. 9. 1907]{ Stefan Großmann an Arthur Schnitzler, 29. 9. 1907}\nopagebreak\mylabel{v}\rehead{ }\begin{ledgroupsized}[t]{13cm}\normalsize\beginnumbering \toendnotes[C]{\smallbreak\pagebreak[2]} \Standort{CUL, Schnitzler, B 34.}
\physDesc{Brief, 1 Blatt, 1 Seite, 551 Zeichen (Briefpapier mit Trauerrand)
\newline{}Handschrift: schwarze Tinte, deutsche Kurrent
\newline{}Ordnung: mit Bleistift von unbekannter Hand nummeriert:
                                 »5« }\toendnotes[C]{\smallbreak}\pstart
           \noindent{}{\pb}\textcolor{gray}{\textbf{Freie Volksbühne\orgindex{Wiener Freie Volksbuehne@Wiener Freie Volksbühne|pw}}}\pend
           \pstart
           \textcolor{gray}{\textbf{Wien VI/\textsubscript{1}\oindex{VI., Mariahilf@\textbf{VI., Mariahilf}|pw}.}}\pend
           \pstart
           \textcolor{gray}{\textbf{Mariahilferſtraße Nr. 89.\oindex{Mariahilferstrasse@\textbf{Mariahilferstraße}|pw}}}\hfill \textcolor{gray}{\textbf{Wien\oindex{Wien@\textbf{Wien}|pw}, am}}{ }29. \damage{\textcolor{gray}{Se}}pt. \textcolor{gray}{\textbf{190}}7\pend
           \pstart
           \textcolor{gray}{\textbf{Poſtſparkaſſen-Konto Nr. 87.544.}}\pend
           \pstart\center{}Verehrter Herr,\pend\pstart
           Danke für Ihre Bereitwilligkeit.\pend
           \pstart
           Wir werden alle Ihre Wünſche berückſichtigen u den Abend am \uline{17. Oktober} in einem kleinen (500 Leute faſſenden) Saal veranſtalten. Herr Abg \uline{\textsc{Pernerstorfer\pwindex{Pernerstorfer, Engelbert 27.04.1850 – 06.01.1918@\textsc{Pernerstorfer, Engelbert} (27.04.1850 – 06.01.1918), \emph{Politiker, Journalist}|pw}}} wird den Abend mit einem kleinen Vortrag eröffnen. Dann leſen Sie, verehrter
               Herr, vielleicht den »\label{T_L01714-1v}\edtext{\textsc{\uline{Lieutenant}}}{\lemma{\textnormal{\emph{Lieutenant}}}\Cendnote{\textnormal{Er schreibt:
                        »Leuitenant«.}}}\label{T_L01714-1h}\textsc{\uline{{ }Gustl}}\pwindex{Schnitzler, Arthur 15.05.1862 – 21.10.1931@\textsc{Schnitzler, Arthur} (15.05.1862 – 21.10.1931), \emph{Schriftsteller, Mediziner}!Lieutenant Gustl. Novelle1900-12-25@\strich\emph{Lieutenant Gustl. Novelle} {[}1900-12-25{]}|pw}« und irgendeine kleine Arbeit\pwindex{Schnitzler, Arthur 15.05.1862 – 21.10.1931@\textsc{Schnitzler, Arthur} (15.05.1862 – 21.10.1931), \emph{Schriftsteller, Mediziner}!neue Lied23. 04. 1905@\strich\emph{Das neue Lied} {[}23. 04. 1905{]}|pwv}.\pend
           \pstart
           Wir bitten uns \uline{recht bald} Ihre Zust\textcolor{gray}{i{\geminationm}un}g definitiv zu übermitteln, da wir
               14 Tage vorher die Ankündigung in unſeren Vereinsmittheilungen loslaſſen müſſen.\pend
           \pstart
           Aufrichtig dankend{\\[\baselineskip]}ſehr ergeben{\\[\baselineskip]}f. d. Fr. V.\orgindex{Wiener Freie Volksbuehne@Wiener Freie Volksbühne|pw}{\\[\baselineskip]}\spacefill\mbox{GroßSt}\pend
           \leftskip=0em{}
         
         \endnumbering\mylabel{h}\end{ledgroupsized}  \newcommand{\dateiname}{L01714}\newcommand{\titel}{Stefan Großmann an Arthur Schnitzler, 29. 9. 1907}\newcommand{\editorInnen}{ Martin Anton Müller und Gerd-Hermann Susen}%% latex-leseansicht-abspann.tex
%% Abspann für die Leseansicht.
%% Der Schalter \ifkorrekturansicht ist bereits durch den Vorspann gesetzt.

%% latex-abspann.tex
%% Gemeinsamer Abspann für Korrekturansicht und Leseansicht.
%% Setzt den Schalter \ifkorrekturansicht voraus (gesetzt in den
%% einbindenden Dateien latex-korrekturansicht-abspann.tex bzw.
%% latex-leseansicht-abspann.tex).
%% ---------------------------------------------------------------

\normalsize

% Das esempio-Environment wird nur in der Leseansicht benötigt
\ifkorrekturansicht\else
\newenvironment{esempio}[3]%
{
    \vspace{1.5ex}
    \rlap{\underline{#1}}
    \par
    \setlength{\parindent}{0cm}
    \nopagebreak
    \leftskip=#2cm
    \rightskip=#3cm
}
{
    \par
}
\fi

\doendnotes{C}
\bigskip
\vfill

\clearpage

\footnotesize

\ifkorrekturansicht
  \lohead{\textsc{register}}
\fi

% theindex-Environment neu definieren ohne reledmac
\makeatletter
\renewenvironment{theindex}{%
  \ifkorrekturansicht
    \section*{\indexname}%
  \else
    \subsubsection*{Index der erwähnten Entitäten}%
  \fi
  \setlength{\parindent}{0pt}%
  \setlength{\parskip}{0pt plus 0.3pt}%
  \let\item\@idxitem
}{%
  \ifkorrekturansicht\clearpage\fi
}
\makeatother

\IfFileExists{\jobname-pw.ind}{\input{\jobname-pw.ind}}{}

% Quellenangabe nur in der Leseansicht
\ifkorrekturansicht\else
% Fallback-Definitionen, falls die .tex-Datei \titel etc. nicht gesetzt hat
\providecommand{\titel}{}
\providecommand{\editorInnen}{}
\providecommand{\dateiname}{\jobname}

\vspace{3cm}

\vfill

\footnotesize
\textsc{Quelle}: \titel. Herausgegeben von {\editorInnen}. In: \emph{Arthur Schnitzler: Briefwechsel mit Autorinnen und Autoren}.
 Digitale Edition, https://schnitzler-briefe.acdh.oeaw.ac.at/{\dateiname}.html (Stand \today)
\fi

\end{document}


      