%% latex-korrekturansicht-vorspann.tex
%% Vorspann für die Korrekturansicht.
%% Lädt die gemeinsame Datei latex-vorspann.tex mit gesetztem Schalter.

\newif\ifkorrekturansicht
\korrekturansichttrue

\input{../tex-inputs/latex-vorspann}


\section[Stefan Großmann an Arthur Schnitzler, 29. 9. 1907]{L01714 Stefan Großmann an Arthur Schnitzler, 29. 9. 1907}
\nopagebreak\mylabel{L01714v}
\rehead{ }\normalsize\beginnumbering\briefempfaengerindex{Schnitzler, Arthur@\textsc{Schnitzler, Arthur}!zzzGrossmann, Stefan@\emph{von Stefan Großmann}!1907-09-292@{29. 9. 1907}|(be}
\toendnotes[C]{\smallbreak\pagebreak[2]}\Standort{CUL, Schnitzler, B 34.}
\physDesc{Brief, 1 Blatt, 1 Seite, 551 Zeichen (Briefpapier mit Trauerrand)
\newline{}Handschrift: schwarze Tinte, deutsche Kurrent
\newline{}Ordnung: mit Bleistift von unbekannter Hand nummeriert:
                                 »5« }\toendnotes[C]{\smallbreak}
\pstart
           {\pb}\textcolor{gray}{\textbf{Freie Volksbühne\orgindex{Wiener Freie Volksbuehne@Wiener Freie Volksbühne|pw}}}\pend
           
\pstart
           \textcolor{gray}{\textbf{Wien VI/\textsubscript{1}\oindex{VI., Mariahilf@\textbf{VI., Mariahilf}, \emph{A.ADM3}|pw}.}}\pend
           
\pstart
           \textcolor{gray}{\textbf{Mariahilferſtraße Nr. 89.\oindex{Mariahilfer Strasse@\textbf{Mariahilfer Straße}, \emph{Straße (K.STR)}|pw}}}\hfill \textcolor{gray}{\textbf{Wien\oindex{Wien@\textbf{Wien}, \emph{A.ADM2}|pw}, am}}{ }29. \damage{\textcolor{gray}{Se}}pt. \textcolor{gray}{\textbf{190}}7\pend
           
\pstart
           \textcolor{gray}{\textbf{Poſtſparkaſſen-Konto Nr. 87.544.}}\pend
           
\pstart\center{}Verehrter Herr,\pend\vspace{0.5em}
\pstart
           Danke für Ihre Bereitwilligkeit.\pend
           
\pstart
           Wir werden alle Ihre Wünſche berückſichtigen u den Abend am \uline{17. Oktober} in einem kleinen (500 Leute faſſenden) Saal veranſtalten. Herr Abg \uline{\textsc{Pernerstorfer\pwindex{Pernerstorfer, Engelbert 27.04.1850 – 06.01.1918@\textsc{Pernerstorfer, Engelbert} (27.04.1850 – 06.01.1918), \emph{Politiker/Politikerin, Journalist/Journalistin}|pw}}} wird den Abend mit einem kleinen Vortrag eröffnen. Dann leſen Sie, verehrter
               Herr, vielleicht den »\label{T_L01714-1v}\edtext{\textsc{\uline{Lieutenant}}}{\lemma{\textnormal{\emph{Lieutenant}}}\Cendnote{\textnormal{Er schreibt:
                        »Leuitenant«.}}}\label{T_L01714-1}\textsc{\uline{{ }Gustl}}\pwindex{Lieutenant Gustl. Novelle@\emph{Lieutenant Gustl. Novelle}|pw}« und irgendeine kleine Arbeit\pwindex{neue Lied@\emph{Das neue Lied}|pwv}.\pend
           
\pstart
           Wir bitten uns \uline{recht bald} Ihre Zust\textcolor{gray}{i{\geminationm}un}g definitiv zu übermitteln, da wir
               14 Tage vorher die Ankündigung in unſeren Vereinsmittheilungen loslaſſen müſſen.\pend
           
\pstart
           Aufrichtig dankend{\\[\baselineskip]}ſehr ergeben{\\[\baselineskip]}f. d. Fr. V.\orgindex{Wiener Freie Volksbuehne@Wiener Freie Volksbühne|pw}{\\[\baselineskip]}\spacefill\mbox{GroßSt}\pend
           \leftskip=0em{}\selectlanguage{ngerman}\endnumbering\briefempfaengerindex{Schnitzler, Arthur@\textsc{Schnitzler, Arthur}!zzzGrossmann, Stefan@\emph{von Stefan Großmann}!1907-09-292@{29. 9. 1907}|)be}\mylabel{L01714h}  \normalsize

\doendnotes{C}
\bigskip
\vfill

\clearpage

\footnotesize

\lohead{\textsc{register}}

% Definiere theindex-Environment komplett neu ohne reledmac
\makeatletter
\renewenvironment{theindex}{%
  \section*{\indexname}%
  \setlength{\parindent}{0pt}%
  \setlength{\parskip}{0pt plus 0.3pt}%
  \let\item\@idxitem
}{%
  \clearpage
}
\makeatother

\IfFileExists{\jobname-pw.ind}{\input{\jobname-pw.ind}}{}

\end{document}

      