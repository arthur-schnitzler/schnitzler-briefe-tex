%% latex-leseansicht-vorspann.tex
%% Vorspann für die Leseansicht.
%% Lädt die gemeinsame Datei latex-vorspann.tex mit nicht gesetztem Schalter.

\newif\ifkorrekturansicht
\korrekturansichtfalse

\input{../tex-inputs/latex-vorspann}


\section[Hugo von Hofmannsthal an Arthur Schnitzler, 4. 10. 1900]{L01075 Hugo von Hofmannsthal an Arthur Schnitzler, 4. 10. 1900}
\nopagebreak\mylabel{L01075v}
\rehead{ }\normalsize\beginnumbering\briefempfaengerindex{Schnitzler, Arthur@\textsc{Schnitzler, Arthur}!zzzHofmannsthal, Hugo von@\emph{von Hugo von Hofmannsthal}!1900-10-041@{4. 10. 1900}|(be}
\toendnotes[C]{\smallbreak\pagebreak[2]}
\correspDesc{Versand  durch Hugo von Hofmannsthal am 4. 10. 1900 in Ouchy
\newline{}Erhalt  durch Arthur Schnitzler am 6. 10. 1900 in Wien}\toendnotes[C]{\smallbreak}
\Standort{CUL, Schnitzler, B 43.}
\physDesc{Bildpostkarte, 230 Zeichen
\newline{}Handschrift: schwarze Tinte, deutsche Kurrent
\newline{}Versand: 1) Stempel: »\nobreak{}\oindex{Ouchy@\textbf{Ouchy}|pwk}Ouchy, 4. X. 00, 4\nobreak{}«.   2) Stempel: »\nobreak{}\oindex{IX., Alsergrund@\textbf{IX., Alsergrund}, \emph{Verwaltungsgebiet}|pwk}Wien 9/3 72, 6. 10. 00, 8.V, Bestellt\nobreak{}«. 
\newline{}Ordnung: 1) mit Bleistift von unbekannter Hand nummeriert: »167«  2) mit Bleistift von unbekannter Hand nummeriert: »174«}
\buchAbdrucke{\weitereDrucke{Hugo von Hofmannsthal, Arthur Schnitzler: \emph{Briefwechsel}. Herausgegeben von Therese Nickl und Heinrich Schnitzler. Frankfurt am Main: \emph{S. Fischer} 1964, S. 144.} }\toendnotes[C]{\smallbreak}\pstart{}\textsc{{\pb}Herrn D\textsuperscript{r} Arthur Schnitzler}\pend{}\pstart{}\textsc{Wien\oindex{Wien@\textbf{Wien}, \emph{Verwaltungsgebiet}|pw}}\pend{}\pstart{}\textsc{IX. Franckgasse 1.\oindex{Wien@\textbf{Wien}!IX., Alsergrund@\textbf{IX., Alsergrund}!Frankgasse 1@\textbf{Frankgasse 1}, \emph{Wohngebäude}|pw}}\pend{}{\bigskip}
\pstart
           \noindent{}\centering{}{\pb}\textcolor{gray}{\textbf{Ouchy\oindex{Ouchy@\textbf{Ouchy}|pw} – Hôtel Beau Rivage\oindex{Beau-Rivage Palace@\textbf{Beau-Rivage Palace}, \emph{Hotel}|pw}.}}\pend
           \vspace{1em}
\pstart
           {\pb}4. X. 1900.\pend
           \vspace{0.5em}
\pstart
           In \textsc{Ouchy}\oindex{Ouchy@\textbf{Ouchy}|pw} haben wir einmal zuſa{\geminationm}en aufs Dampfſchiff \label{K_L01075-1v}\edtext{gewartet}{\lemma{\textnormal{\emph{gewartet}}}\Cendnote{\textnormal{Siehe A. S.: \emph{Tagebuch}, 14. 8. 1898. }}}\label{K_L01075-1} und
               über färbige Strümpfe geſprochen. Dann war der{ }ſchöne Abend in \textsc{Chillon}\oindex{Schloss Chillon@\textbf{Schloss Chillon}, \emph{Schloss}|pw} und in \textsc{Glion}\oindex{Glion@\textbf{Glion}|pw}.\pend
           
\pstart
           Auf baldiges Wiederſehen{\\[\baselineskip]}Ihr\spacefill\mbox{Hugo}\pend
           \leftskip=0em{}\selectlanguage{ngerman}\endnumbering\briefempfaengerindex{Schnitzler, Arthur@\textsc{Schnitzler, Arthur}!zzzHofmannsthal, Hugo von@\emph{von Hugo von Hofmannsthal}!1900-10-041@{4. 10. 1900}|)be}\mylabel{L01075h}  \newcommand{\dateiname}{L01075}\newcommand{\titel}{Hugo von Hofmannsthal an Arthur Schnitzler, 4. 10. 1900}\newcommand{\editorInnen}{Martin Anton Müller und Gerd-Hermann Susen}%% latex-leseansicht-abspann.tex
%% Abspann für die Leseansicht.
%% Der Schalter \ifkorrekturansicht ist bereits durch den Vorspann gesetzt.

%% latex-abspann.tex
%% Gemeinsamer Abspann für Korrekturansicht und Leseansicht.
%% Setzt den Schalter \ifkorrekturansicht voraus (gesetzt in den
%% einbindenden Dateien latex-korrekturansicht-abspann.tex bzw.
%% latex-leseansicht-abspann.tex).
%% ---------------------------------------------------------------

\normalsize

% Das esempio-Environment wird nur in der Leseansicht benötigt
\ifkorrekturansicht\else
\newenvironment{esempio}[3]%
{
    \vspace{1.5ex}
    \rlap{\underline{#1}}
    \par
    \setlength{\parindent}{0cm}
    \nopagebreak
    \leftskip=#2cm
    \rightskip=#3cm
}
{
    \par
}
\fi

\doendnotes{C}
\bigskip
\vfill

\clearpage

\footnotesize

\ifkorrekturansicht
  \lohead{\textsc{register}}
\fi

% theindex-Environment neu definieren ohne reledmac
\makeatletter
\renewenvironment{theindex}{%
  \ifkorrekturansicht
    \section*{\indexname}%
  \else
    \subsubsection*{Index der erwähnten Entitäten}%
  \fi
  \setlength{\parindent}{0pt}%
  \setlength{\parskip}{0pt plus 0.3pt}%
  \let\item\@idxitem
}{%
  \ifkorrekturansicht\clearpage\fi
}
\makeatother

\IfFileExists{\jobname-pw.ind}{\input{\jobname-pw.ind}}{}

% Quellenangabe nur in der Leseansicht
\ifkorrekturansicht\else
% Fallback-Definitionen, falls die .tex-Datei \titel etc. nicht gesetzt hat
\providecommand{\titel}{}
\providecommand{\editorInnen}{}
\providecommand{\dateiname}{\jobname}

\vspace{3cm}

\vfill

\footnotesize
\textsc{Quelle}: \titel. Herausgegeben von {\editorInnen}. In: \emph{Arthur Schnitzler: Briefwechsel mit Autorinnen und Autoren}.
 Digitale Edition, https://schnitzler-briefe.acdh.oeaw.ac.at/{\dateiname}.html (Stand \today)
\fi

\end{document}


