%% latex-korrekturansicht-vorspann.tex
%% Vorspann für die Korrekturansicht.
%% Lädt die gemeinsame Datei latex-vorspann.tex mit gesetztem Schalter.

\newif\ifkorrekturansicht
\korrekturansichttrue

\input{../tex-inputs/latex-vorspann}


\section[Hugo von Hofmannsthal an Arthur Schnitzler, 4. 10. 1900]{L01075 Hugo von Hofmannsthal an Arthur Schnitzler, 4. 10. 1900}
\nopagebreak\mylabel{L01075v}
\rehead{ }\normalsize\beginnumbering\briefempfaengerindex{Schnitzler, Arthur@\textsc{Schnitzler, Arthur}!zzzHofmannsthal, Hugo von@\emph{von Hugo von Hofmannsthal}!1900-10-041@{4. 10. 1900}|(be}
\toendnotes[C]{\smallbreak\pagebreak[2]}\Standort{CUL, Schnitzler, B 43.}
\physDesc{Bildpostkarte, 230 Zeichen
\newline{}Handschrift: 1) schwarze Tinte, deutsche Kurrent\hspace{1em}2) schwarze Tinte, lateinische Kurrent (\noindent{}Adresse)\hspace{1em}
\newline{}Versand: 1) Stempel: »\nobreak{}\oindex{Ouchy@\textbf{Ouchy}, \emph{P.PPL}|pwk}Ouchy, 4. X. 00, 4\nobreak{}«.   2) Stempel: »\nobreak{}\oindex{IX., Alsergrund@\textbf{IX., Alsergrund}, \emph{A.ADM3}|pwk}Wien 9/3 72, 6. 10. 00, 8.V, Bestellt\nobreak{}«. 
\newline{}Ordnung: 1) mit Bleistift von unbekannter Hand nummeriert: »167«  2) mit Bleistift von unbekannter Hand nummeriert: »174«}
\buchAbdrucke{\weitereDrucke{Hugo von Hofmannsthal, Arthur Schnitzler: \emph{Briefwechsel}. Frankfurt am Main: \emph{S. Fischer} 1964, S. 144.} }\toendnotes[C]{\smallbreak}\pstart{}{\pb}Herrn D\textsuperscript{r} Arthur Schnitzler\pend{}\pstart{}Wien\oindex{Wien@\textbf{Wien}, \emph{A.ADM2}|pw}\pend{}\pstart{}IX. Franckgasse 1.\oindex{Frankgasse 1@\textbf{Frankgasse 1}, \emph{Wohngebäude (K.WHS)}|pw}\pend{}{\bigskip}
\pstart
           \noindent{}\centering{}{\pb}\textcolor{gray}{\textbf{Ouchy\oindex{Ouchy@\textbf{Ouchy}, \emph{P.PPL}|pw} – Hôtel Beau Rivage\oindex{Beau-Rivage Palace@\textbf{Beau-Rivage Palace}, \emph{Hotel (K.HTL)}|pw}.}}\pend
           \vspace{1em}
\pstart
           {\pb}4. X. 1900.\pend
           \vspace{0.5em}
\pstart
           In \textsc{Ouchy}\oindex{Ouchy@\textbf{Ouchy}, \emph{P.PPL}|pw} haben wir einmal zuſa{\geminationm}en aufs Dampfſchiff \label{K_L01075-1v}\edtext{gewartet}{\lemma{\textnormal{\emph{gewartet}}}\Cendnote{\textnormal{Siehe A. S.: \emph{Tagebuch}, 14. 8. 1898. }}}\label{K_L01075-1} und
               über färbige Strümpfe geſprochen. Dann war der ſchöne Abend in \textsc{Chillon}\oindex{Schloss Chillon@\textbf{Schloss Chillon}, \emph{Schloss (K.SLS)}|pw} und in \textsc{Glion}\oindex{Glion@\textbf{Glion}, \emph{P.PPL}|pw}.\pend
           
\pstart
           Auf baldiges Wiederſehen{\\[\baselineskip]}Ihr\spacefill\mbox{Hugo}\pend
           \leftskip=0em{}\selectlanguage{ngerman}\endnumbering\briefempfaengerindex{Schnitzler, Arthur@\textsc{Schnitzler, Arthur}!zzzHofmannsthal, Hugo von@\emph{von Hugo von Hofmannsthal}!1900-10-041@{4. 10. 1900}|)be}\mylabel{L01075h}  \normalsize

\doendnotes{C}
\bigskip
\vfill

\clearpage

\footnotesize

\lohead{\textsc{register}}

% Definiere theindex-Environment komplett neu ohne reledmac
\makeatletter
\renewenvironment{theindex}{%
  \section*{\indexname}%
  \setlength{\parindent}{0pt}%
  \setlength{\parskip}{0pt plus 0.3pt}%
  \let\item\@idxitem
}{%
  \clearpage
}
\makeatother

\IfFileExists{\jobname-pw.ind}{\input{\jobname-pw.ind}}{}

\end{document}

      