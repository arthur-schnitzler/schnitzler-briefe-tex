%% latex-korrekturansicht-vorspann.tex
%% Vorspann für die Korrekturansicht.
%% Lädt die gemeinsame Datei latex-vorspann.tex mit gesetztem Schalter.

\newif\ifkorrekturansicht
\korrekturansichttrue

\input{../tex-inputs/latex-vorspann}


\section[Stefan Zweig an Arthur Schnitzler, 25. 9. 1916]{L03659 Stefan Zweig an Arthur Schnitzler, 25. 9. 1916}
\nopagebreak\mylabel{L03659v}
\rehead{ }\normalsize\beginnumbering\briefempfaengerindex{Schnitzler, Arthur@\textsc{Schnitzler, Arthur}!zzzZweig, Stefan@\emph{von Stefan Zweig}!1916-09-251@{25. 9. 1916}|(be}
\toendnotes[C]{\smallbreak\pagebreak[2]}\Standort{CUL, Schnitzler, B 118.}
\physDesc{Brief, 1 Blatt, 3 Seiten, 2292 Zeichen
\newline{}Handschrift: schwarze Tinte, lateinische Kurrent
\newline{}Schnitzler: 1) mit Bleistift »\textsc{Zweig}«  2) mit rotem Buntstift eine Unterstreichung}
\buchAbdrucke{\weitereDrucke{1) Stefan Zweig: \emph{Briefwechsel mit Hermann Bahr, Sigmund Freud, Rainer Maria
                        Rilke und Arthur Schnitzler}. Frankfurt am Main: \emph{S. Fischer} 1987, S. 400–401.} \weitereDrucke{2) Stefan Zweig: \emph{Briefe. Bd. II: 1914–1919}. Frankfurt am Main: \emph{S. Fischer} 1998, S. 116–117.} }\toendnotes[C]{\smallbreak}
\pstart
           {\pb}Wien\oindex{Wien@\textbf{Wien}, \emph{A.ADM2}|pw}{ }25. September 1916\pend
           \vspace{0.5em}
\pstart
           Lieber verehrter Herr Doktor, ich weiss nicht, ob Sie schon wieder in Wien\oindex{Wien@\textbf{Wien}, \emph{A.ADM2}|pw} sind, möchte mich doch aber bei Ihnen
               anfragen, weil ich eigentlich recht ungeduldig bin, Sie wiederzusehen. So lange habe
               ich mir es versagen müssen, von Ihnen zu hören, aber meine Existenz in dieser Zeit
               ist ja eine so zusammengepresste, dass selbst die erwünschtesten Dinge aussen bleiben
               mussten. Seit bald zwei Jahren bin ich nun eine Art \label{K_L03659-1v}\edtext{Diurnist}{\lemma{\textnormal{\emph{Diurnist}}}\Cendnote{\textnormal{österreichisch: Beamter mit zeitlich beschränktem Vertrag. Zweig\pwindex{Zweig, Stefan 28.11.1881 – 23.02.1942@\textsc{Zweig, Stefan} (28.11.1881 – 23.02.1942), \emph{Schriftsteller/Schriftstellerin}|pwk} war seit 1. 12. 1914 im \emph{Kriegsarchiv}\orgindex{Kriegsarchiv@Kriegsarchiv|pwk} eingesetzt.}}}\label{K_L03659-1}, habe in zwei
               Jahren keine acht Tage für mich gehabt und muss selbst dieses kleine Stück Leben
               tagtäglich fast vor dem Absturz hüten, dazu kam in den letzten acht Monaten eine
               grosse Arbeit\pwindex{Jeremias. Ein dramatische Dichtung in neun Bildern@\emph{Jeremias. Ein dramatische Dichtung in neun Bildern}|pwv}, die ich mit
               zusammengebissenen Zähnen vorwärts{\pb}treibe, die Angst im Nacken, sie nicht vollenden zu können, die Bangnis innen, ob
               nicht meine Kraft durch die innere Gegenwehr wider die Zeit nicht gebrochen sei. Sie
               mögen denken, was mir es bedeutet, da wieder einen Abend in der guten Atmosphäre
               schaffender, ans eigene Werk gestellter Menschen zu verbringen, wie viel mirs wäre,
               wieder von \label{K_L03659-2v}\edtext{Ihrer gütigen Gegenwart
                  Freude}{\lemma{\textnormal{\emph{Ihrer … Freude}}}\Cendnote{\textnormal{Das nächste Treffen fand am
                     1. 10. 1916
                  statt.}}}\label{K_L03659-2} haben zu dürfen.\pend
           
\pstart
           Fürchten Sie nicht, dass ich zu trübe Stimmung Ihnen bringe: im Gegenteil, ich bin
               jetzt durch Resignation gegen Alles irgendwie gepanzert. Etwas Anderes hätte ich
               Ihnen lieber mitgebracht, nämlich einen Akt oder zwei aus meiner grossen Arbeit\pwindex{Jeremias. Ein dramatische Dichtung in neun Bildern@\emph{Jeremias. Ein dramatische Dichtung in neun Bildern}|pwv} und Ihnen vorgelesen.
               Ich glaube nur zu wissen, dass Sie nicht gerne sich lesen lassen, andrerseits ist
               mein Stück\pwindex{Jeremias. Ein dramatische Dichtung in neun Bildern@\emph{Jeremias. Ein dramatische Dichtung in neun Bildern}|pwv} eben noch weit von
               der Vollendung und ich hätte es {\pb}gerne
               Ihnen näher gebracht. Es ist eigentlich meine erste wirkliche Arbeit, die erste, die
               ich innerlich ganz anerkenne, weil sie über das Mass meines Willens so
               hinausgewachsen ist, weil sie – wohl aussichtslos in jedem zweckdienlichen Sinne –
               nur die ganzen innern Probleme der Zeit und meines persönlichen Erlebens erlösend
               aufgelöst hat. Es war in den letzten acht Monaten die innerlichste Zwiesprache die
               ich führen konnte, besser als mit allen Menschen.\pend
           
\pstart
           Ich habe in diesen Zeilen gar nicht Ihrer lieben Frau\pwindex{Schnitzler, Olga 17.01.1882 – 13.01.1970@\textsc{Schnitzler, Olga} (17.01.1882 – 13.01.1970), \emph{Schauspieler/Schauspielerin, Sänger/Sängerin}|pwv} gedacht und doch gehen auch meine Worte an sie. Es ist
               so gut jetzt an wirkliche an menschliche Menschen denken zu dürfen und ich tue es
               gern und oft, um mich über die andern zu trösten. Wann immer Sie mir es erlauben
               wollen, komme ich zu Ihnen. Meine herzlichsten Grüsse voraus! Ihr getreuer\pend
           \pstart \spacefill\mbox{Stefan Zweig}\pend{}\selectlanguage{ngerman}\endnumbering\briefempfaengerindex{Schnitzler, Arthur@\textsc{Schnitzler, Arthur}!zzzZweig, Stefan@\emph{von Stefan Zweig}!1916-09-251@{25. 9. 1916}|)be}\mylabel{L03659h}
\begin{anhang}
\end{anhang}\normalsize

\doendnotes{C}
\bigskip
\vfill

\clearpage

\footnotesize

\lohead{\textsc{register}}

% Definiere theindex-Environment komplett neu ohne reledmac
\makeatletter
\renewenvironment{theindex}{%
  \section*{\indexname}%
  \setlength{\parindent}{0pt}%
  \setlength{\parskip}{0pt plus 0.3pt}%
  \let\item\@idxitem
}{%
  \clearpage
}
\makeatother

\IfFileExists{\jobname-pw.ind}{\input{\jobname-pw.ind}}{}

\end{document}

      