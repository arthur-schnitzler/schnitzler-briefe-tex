%% latex-korrekturansicht-vorspann.tex
%% Vorspann für die Korrekturansicht.
%% Lädt die gemeinsame Datei latex-vorspann.tex mit gesetztem Schalter.

\newif\ifkorrekturansicht
\korrekturansichttrue

\input{../tex-inputs/latex-vorspann}


\section[Richard Beer-Hofmann an Arthur Schnitzler, 29. 7. 1896]{L00573 Richard Beer-Hofmann an Arthur Schnitzler, 29. 7. 1896}
\nopagebreak\mylabel{L00573v}
\rehead{ }\normalsize\beginnumbering\briefempfaengerindex{Schnitzler, Arthur@\textsc{Schnitzler, Arthur}!zzzBeer-Hofmann, Richard@\emph{von Richard Beer-Hofmann}!1896-07-293@{29. 7. 1896}|(be}
\toendnotes[C]{\smallbreak\pagebreak[2]}\Standort{CUL, Schnitzler, B 8.}
\physDesc{Telegramm, 150 Zeichen
\newline{}Handschrift einer Schreibkraft: blaue Tinte, lateinische Kurrent
\newline{}Ordnung: mit Bleistift von unbekannter Hand nummeriert:
                                    »79« }\pstart{}{\pb}Doktor Arthur\pend{}\pstart{}Schnitzler\pend{}\pstart{}poste restante Stcklm\oindex{Stockholm@\textbf{Stockholm}, \emph{P.PPLC}|pw}\pend{}{\bigskip}\vspace{1em}
\pstart
           {\pb}\textcolor{gray}{\textbf{Inlemnadt i}}{ }Köpenhamn\oindex{Kopenhagen@\textbf{Kopenhagen}, \emph{P.PPLC}|pw}{ }\textcolor{gray}{\textbf{Nr}} 44/2206{ }\textcolor{gray}{\textbf{Ord}} 18{ }\textcolor{gray}{\textbf{År}}96{ }\textcolor{gray}{\textbf{Datum}}{ }29/7{ }\textcolor{gray}{\textbf{Kl.}} 2e\pend
           \vspace{0.5em}
\pstart
           Wäre mir lieb, wenn sie schon 31 kämen bitte telegrafisch Antwort\pend
           \pstart \spacefill\mbox{Richard}\pend{}\selectlanguage{ngerman}\endnumbering\briefempfaengerindex{Schnitzler, Arthur@\textsc{Schnitzler, Arthur}!zzzBeer-Hofmann, Richard@\emph{von Richard Beer-Hofmann}!1896-07-293@{29. 7. 1896}|)be}\mylabel{L00573h}  \normalsize

\doendnotes{C}
\bigskip
\vfill

\clearpage

\footnotesize

\lohead{\textsc{register}}

% Definiere theindex-Environment komplett neu ohne reledmac
\makeatletter
\renewenvironment{theindex}{%
  \section*{\indexname}%
  \setlength{\parindent}{0pt}%
  \setlength{\parskip}{0pt plus 0.3pt}%
  \let\item\@idxitem
}{%
  \clearpage
}
\makeatother

\IfFileExists{\jobname-pw.ind}{\input{\jobname-pw.ind}}{}

\end{document}

      