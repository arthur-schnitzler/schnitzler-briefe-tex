\input{../tex-inputs/latex-pdf-vorspann}
\begin{center}
            \textcolor{red}{ENTWURF. ENTZIFFERUNG NOCH NICHT KORREKTURGELESEN}
                      \end{center}
            
               \section[Richard Beer-Hofmann an Arthur Schnitzler, 29. 7. 1896]{ Richard Beer-Hofmann an Arthur Schnitzler, 29. 7. 1896}\nopagebreak\mylabel{v}\rehead{ }\begin{ledgroupsized}[t]{13cm}\normalsize\beginnumbering\briefempfaengerindex{Schnitzler, Arthur@\textsc{Schnitzler, Arthur}!zzzBeer-Hofmann, Richard@\emph{von Richard Beer-Hofmann}!1896-07-293@{29. 7. 1896}|(be} \toendnotes[C]{\smallbreak\pagebreak[2]} \Standort{CUL, Schnitzler, B 8.}
\physDesc{Telegramm
\newline{}Handschrift einer Schreibkraft: blaue Tinte, lateinische Kurrent\newline{}Ordnung: mit Bleistift von unbekannter Hand nummeriert: »79« }\pstart{}{\pb}\textsc{Doktor Arthur}\pend{}\pstart{}\textsc{Schnitzler}\pend{}\pstart{}\textsc{poste restante Stcklm\oindex{Stockholm@\textbf{Stockholm}|pw}}\pend{}{\bigskip}\pstart
           {\pb}\textcolor{gray}{\textbf{Inlemnadt i}}{ }Köpenhamn\oindex{Kopenhagen@\textbf{Kopenhagen}|pw}{ }\textcolor{gray}{\textbf{Nr}} 44/2206{ }\textcolor{gray}{\textbf{Ord}} 18{ }\textcolor{gray}{\textbf{År}} 96{ }\textcolor{gray}{\textbf{Datum}}{ }29/7{ }\textcolor{gray}{\textbf{Kl.}} 2e\pend
           \pstart
           Wäre mir lieb, wenn sie schon 31 kämen bitte telegrafisch Antwort\pend
           \pstart \spacefill\mbox{Richard}\pend{}\endnumbering\briefempfaengerindex{Schnitzler, Arthur@\textsc{Schnitzler, Arthur}!zzzBeer-Hofmann, Richard@\emph{von Richard Beer-Hofmann}!1896-07-293@{29. 7. 1896}|)be}\mylabel{h}\end{ledgroupsized}  \newcommand{\dateiname}{L00573}\newcommand{\titel}{Richard Beer-Hofmann an Arthur Schnitzler, 29. 7. 1896}\newcommand{\editorInnen}{Martin Anton Müller und Gerd-Hermann Susen}\input{../tex-inputs/latex-pdf-abspann}
      