%% latex-leseansicht-vorspann.tex
%% Vorspann für die Leseansicht.
%% Lädt die gemeinsame Datei latex-vorspann.tex mit nicht gesetztem Schalter.

\newif\ifkorrekturansicht
\korrekturansichtfalse

\input{../tex-inputs/latex-vorspann}


         
         \renewcommand{\erwaehntePersonen}{Personen: Leo Feld, Paul Goldmann, Clementine Goldmann, Carl Karlweis, Fedor Mamroth, Marie Reinhard, Josef Rosengart, Vally Rosengart, Ferdinand von Saar}
         \renewcommand{\erwaehnteInstitutionen}{Institutionen: Bauernfeld-Preis}
         \renewcommand{\erwaehnteOrte}{Orte: Darmstadt, Frankfurt am Main, Karlsruhe, Mainz, Wien, Wiesbaden}
         \renewcommand{\erwaehnteWerke}{}
               \section[ Paul Goldmann an Arthur Schnitzler, 31. 3. {[}1899{]}]{ Paul Goldmann an Arthur Schnitzler, 31. 3. {[}1899{]}}\nopagebreak\mylabel{v}\rehead{ }\begin{ledgroupsized}[t]{13cm}\normalsize\beginnumbering\briefempfaengerindex{Schnitzler, Arthur@\textsc{Schnitzler, Arthur}!zzzGoldmann, Paul@\emph{von Paul Goldmann}!1899-03-311@{31. 3. {[}1899{]}}|(be} \toendnotes[C]{\smallbreak\pagebreak[2]} \Standort{DLA, A:Schnitzler, HS.NZ85.1.3169.}
\physDesc{Brief, 1 Blatt, 4 Seiten, 3635 Zeichen
\newline{}Handschrift: schwarze Tinte, deutsche Kurrent}\toendnotes[C]{\smallbreak}\pstart
           \raggedleft{}{\pb}Frankfurt\oindex{Frankfurt am Main@\textbf{Frankfurt am Main}|pw}{ }31. März.\pend
           \pstart\center{}Mein lieber Freund,\pend\pstart
           Gerade in dieſen Tagen werde ich herumgehetzt, wie ein Hund. Miſſionen nach Mainz\oindex{Mainz@\textbf{Mainz}|pw}, Karlsruhe\oindex{Karlsruhe@\textbf{Karlsruhe}|pw}, Darmſtadt\oindex{Darmstadt@\textbf{Darmstadt}|pw}, – hier berichten,
               dort berichten. Ich habe keinen Augenblick freie Zeit und habe den heutigen \label{K_L02871-1v}\edtext{Feiertag}{\lemma{\textnormal{\emph{Feiertag}}}\Cendnote{\textnormal{Karfreitag}}}\label{K_L02871-1h} abwarten müſſen, um Dir endlich einmal auch ein Wort zu ſchreiben, nachdem
               ich \strikeout{\textcolor{gray}{×}\-\textcolor{gray}{×}\-\textcolor{gray}{×}} alle dieſe Tage \strikeout{\textcolor{gray}{×}\-\textcolor{gray}{×}\-\textcolor{gray}{×}}{ }\label{K_L02871-2v}\edtext{bekümmert}{\lemma{\textnormal{\emph{bekümmert}}}\Cendnote{\textnormal{Er trauerte um Marie
                     Reinhard\pwindex{Reinhard, Marie 1871-03-13 – 1899-03-18@\textsc{Reinhard, Marie} (1871-03-13 – 1899-03-18), \emph{Gesangspädagogin}|pwk}, die am 18. 3. 1899 gestorben war. }}}\label{K_L02871-2h} Deiner gedacht.\pend
           \pstart
           Dein Brief, in dem Du das Fürchterliche ſchilderſt, hat mich tief ergriffen. Es iſt
               ein wahres \label{K_L02871-3v}\edtext{\begin{otherlanguage}{french}Raffinement\end{otherlanguage}}{\lemma{\textnormal{\emph{Raffinement}}}\Cendnote{\textnormal{französisch: Vervollkommnung}}}\label{K_L02871-3h} von
               Qual geweſen. Das Herz preßt ſich zuſammen, wenn man das lieſt. Und nun gar das
                  \label{K_L02871-4v}\edtext{miterleben}{\lemma{\textnormal{\emph{miterleben}}}\Cendnote{\textnormal{Schnitzler\pwindex{Schnitzler, Arthur 15.05.1862 – 21.10.1931@\textsc{Schnitzler, Arthur} (15.05.1862 – 21.10.1931), \emph{Schriftsteller, Mediziner}|pwk} war beim Tod von Marie Reinhard\pwindex{Reinhard, Marie 1871-03-13 – 1899-03-18@\textsc{Reinhard, Marie} (1871-03-13 – 1899-03-18), \emph{Gesangspädagogin}|pwk} anwesend, vgl. A. S.: \emph{Tagebuch}, 18. 3. 1899.}}}\label{K_L02871-4h}! Du
               Ärmſter, was mußt Du gelitten haben! Ich will auch gar nicht verſuchen, Dir Troſt zu
               ſpenden. Es gibt \strikeout{d\textcolor{gray}{a}} da nichts zu tröſten. Und außer der Zeit kann nichts und Niemand helfen. Auf
               die Zeit rechne ich allerdings. Auch das wird ſich ſchließlich mildern. \strikeout{D\textcolor{gray}{a}s\textcolor{gray}{,}} In dem, was Du über {\pb}Dein Alter ſchreibſt,
               haſt Du Unrecht. Gerade in Deinem Alter kann man ſelbſt eine ſolche Schickung noch
               tragen, – ſpäter nicht mehr. Du biſt noch jung, und in Deinem Leben iſt noch Kraft
               genug, um ſelbſt dieſe ſchreckliche Leere, die ſich auf einmal aufgethan hat, wieder
               auszufüllen und \strikeout{\textcolor{gray}{zu}\textcolor{gray}{×}} langſsam \introOben{}zu\introOben{} verdecken. Das iſt \strikeout{\textcolor{gray}{×}} in dieſem Unglück meine einzige, aber auch meine feſte und ſichere Hoffnung,
               Du mußt freilich ſelbſt etwas dazuthun und mußt Dich gewaltſam herausreißen. Du mußt
               Dich zu der Erkenntniß durchringen, daß in der Beziehung zu einer Frau, und ſei es
               die beſte und liebſte, das Leben ſich nicht erſchöpft. Glaube mir, das iſt die
               Wahrheit. Es gibt Anderes, viel Anderes noch. Es gibt auch wieder einmal neues Glück!
               Nur leben bleiben – leben und warten!\pend
           \pstart
           Ich empfinde es bitter und ſchmerzlich, daß ich nicht bei Dir ſein kann. Mir kommt es
               vor, als ließe ich Dich im Stich, wenn ich hier fern von Dir bin und Dich allein weiß
               mit Deinem Kummer. Eines wäre \strikeout{d\textcolor{gray}{r}i}
               dringend nöthig, und ich komme immer wieder darauf zurück: {\pb}Du müßteſt fort aus Wien\oindex{Wien@\textbf{Wien}|pw}, ſo raſch als möglich, – ein paar Wochen reiſen. Komm’ auf einige Tage
               nach \label{K_L02871-5v}\edtext{Frankfurt\oindex{Frankfurt am Main@\textbf{Frankfurt am Main}|pw}}{\lemma{\textnormal{\emph{Frankfurt}}}\Cendnote{\textnormal{Dazu kam es nicht.}}}\label{K_L02871-5h}! Wenn nicht, ſo
               gehe anderswohin, – irgendwohin, wo Du Geſellſchaft haſt. Allein reiſen dürfteſt Du
               auch nicht.\pend
           \pstart
           Bitte, lieber Freund, ſchreib’ mir bald einmal, wenigſtens eine Zeile, \strikeout{d\textcolor{gray}{a}} damit ich weiß, wie es Dir geht. Es braucht nicht viel zu ſein, – nur ein
               Lebenszeichen.\pend
           \pstart
           Mit meinem Schwager\pwindex{Rosengart, Josef 1860-02-08 – 1927-08-04@\textsc{Rosengart, Josef} (1860-02-08 – 1927-08-04), \emph{Arzt}|pwv} habe ich
               über einiges Mediziniſche geſprochen. Er meint, ob es denn nicht möglich geweſen
               wäre, noch eine Operation zu verſuchen? Dein \label{K_L02871-6v}\edtext{Ohrenleiden}{\lemma{\textnormal{\emph{Ohrenleiden}}}\Cendnote{\textnormal{Schnitzler\pwindex{Schnitzler, Arthur 15.05.1862 – 21.10.1931@\textsc{Schnitzler, Arthur} (15.05.1862 – 21.10.1931), \emph{Schriftsteller, Mediziner}|pwk} litt seit Herbst 1896 an Otosklerose – einer Verknöcherung des Innenohrs mit
                  zunehmender Schwerhörigkeit.}}}\label{K_L02871-6h} aber kann er ſich abſolut nicht entſchließen
               ernſtzunehmen. Er hat ſich viel mit dieſen Dingen beſchäftigt und vermag in allen
               Symptomen, die ich ihm ſchildere, nichts Bedenkliches zu entdecken. Er, meine Schweſter\pwindex{Rosengart, Vally *~1866-12-29@\textsc{Rosengart, Vally} (*~1866-12-29)|pwv} und mein Onkel\pwindex{Mamroth, Fedor 21.02.1851 – 25.06.1907@\textsc{Mamroth, Fedor} (21.02.1851 – 25.06.1907), \emph{Journalist, Kritiker}|pwv}, denen ich von dem
               Schlage, der Dich betroffen, Mittheilung gemacht habe, nehmen warmen Antheil an
               Deinen Schmerzen, {\pb}haben aber nicht gewagt, Dir
               ſelbſt zu ſchreiben. Meine Mutter\pwindex{Goldmann, Clementine 1842-05-15 – 1924-02-24@\textsc{Goldmann, Clementine} (1842-05-15 – 1924-02-24)|pwv} iſt gegenwärtig in Wiesbaden\oindex{Wiesbaden@\textbf{Wiesbaden}|pw}.\pend
           \pstart
           Daß Dir der \label{K_L02871-7v}\edtext{\textsc{Bauernfeld}-Preis\orgindex{Bauernfeld-Preis@Bauernfeld-Preis|pw}}{\lemma{\textnormal{\emph{Bauernfeld-Preis}}}\Cendnote{\textnormal{Der \emph{Bauernfeld-Preis}\orgindex{Bauernfeld-Preis@Bauernfeld-Preis|pwk}, dotiert als Ehrengabe von 1000 Gulden, wurde Schnitzler\pwindex{Schnitzler, Arthur 15.05.1862 – 21.10.1931@\textsc{Schnitzler, Arthur} (15.05.1862 – 21.10.1931), \emph{Schriftsteller, Mediziner}|pwk} am 27. 3. 1899 für seine Dramen und Novellen verliehen. Den gleichen
                  Betrag erhielten im selben Jahr auch Ferdinand
                     von Saar\pwindex{Saar, Ferdinand von 30.09.1833 – 24.07.1906@\textsc{Saar, Ferdinand von} (30.09.1833 – 24.07.1906), \emph{Schriftsteller}|pwk} und Carl Karlweis\pwindex{Karlweis, Carl 23.11.1850 – 27.10.1901@\textsc{Karlweis, Carl} (23.11.1850 – 27.10.1901), \emph{Schriftsteller}|pwk}, 500
                  Gulden gingen an Leo Feld\pwindex{Feld, Leo 14.02.1869 – 05.09.1924@\textsc{Feld, Leo} (14.02.1869 – 05.09.1924), \emph{Schriftsteller, Übersetzer, Dirigent}|pwk}. Das war der
                  erste Literaturpreis, den Schnitzler\pwindex{Schnitzler, Arthur 15.05.1862 – 21.10.1931@\textsc{Schnitzler, Arthur} (15.05.1862 – 21.10.1931), \emph{Schriftsteller, Mediziner}|pwk}
                  erhielt.}}}\label{K_L02871-7h} zu Theil geworden, hat uns Alle hier ſehr gefreut. Das iſt ſchön
               und ehrenvoll{\dotsfive}\pend
           \pstart
           Liebſter Freund, Du mußt ſtark ſein und mußt Dich in das Unabänderliche fügen! Es iſt
               viel verloren, \uline{und doch iſt nichts zu Ende}! Und dann
               haſt Du \label{K_L02871-8v}\edtext{vier Jahre}{\lemma{\textnormal{\emph{vier Jahre}}}\Cendnote{\textnormal{Schnitzler\pwindex{Schnitzler, Arthur 15.05.1862 – 21.10.1931@\textsc{Schnitzler, Arthur} (15.05.1862 – 21.10.1931), \emph{Schriftsteller, Mediziner}|pwk} hatte Marie Reinhard\pwindex{Reinhard, Marie 1871-03-13 – 1899-03-18@\textsc{Reinhard, Marie} (1871-03-13 – 1899-03-18), \emph{Gesangspädagogin}|pwk} am 12. 7. 1894 kennengelernt, als sie seine
                  Patientin war. Bereits im darauffolgenden Herbst begann ihre intime Beziehung, die
                  bis zu Reinhards\pwindex{Reinhard, Marie 1871-03-13 – 1899-03-18@\textsc{Reinhard, Marie} (1871-03-13 – 1899-03-18), \emph{Gesangspädagogin}|pwk} Tod anhielt.}}}\label{K_L02871-8h}
               glücklich ſein dürfen, wie Wenige. Ich verſichere Dich: wenn das Schickſal mir vier
               Jahre ſolchen Glückes geben wollte, um den Preis, daß ich dann einen Schmerz
               durchmachen müßte, wie Du ihn jetzt erlebſt, – ich würde ohneweiters zuſtimmen. Dieſe
               arme Frau\pwindex{Reinhard, Marie 1871-03-13 – 1899-03-18@\textsc{Reinhard, Marie} (1871-03-13 – 1899-03-18), \emph{Gesangspädagogin}|pwv} iſt dahingegangen,
               nachdem ſie Dir das Beſte gegeben hatte, was ſie geben konnte. Sie hat ihr volles Maß
               ausgeſchüttet. Dann iſt ſie für immer geſchieden, auch darin vielleicht ſelbſtlos und
                  \strikeout{\textcolor{gray}{×}\-\textcolor{gray}{×}\-\textcolor{gray}{×}} rührend, wie ſie ſtets war{\dotsfive}\pend
           \pstart
           Ich grüße Dich von Herzen und in Treue {\\[\baselineskip]}Dein {\\[\baselineskip]}\spacefill\mbox{Paul Goldmann.}\pend
           \leftskip=0em{}
         
         \endnumbering\mylabel{h}\end{ledgroupsized}  \newcommand{\dateiname}{L02871}\newcommand{\titel}{Paul Goldmann an Arthur Schnitzler, 31. 3. [1899]}\newcommand{\editorInnen}{Martin Anton Müller und Laura Untner}%% latex-leseansicht-abspann.tex
%% Abspann für die Leseansicht.
%% Der Schalter \ifkorrekturansicht ist bereits durch den Vorspann gesetzt.

%% latex-abspann.tex
%% Gemeinsamer Abspann für Korrekturansicht und Leseansicht.
%% Setzt den Schalter \ifkorrekturansicht voraus (gesetzt in den
%% einbindenden Dateien latex-korrekturansicht-abspann.tex bzw.
%% latex-leseansicht-abspann.tex).
%% ---------------------------------------------------------------

\normalsize

% Das esempio-Environment wird nur in der Leseansicht benötigt
\ifkorrekturansicht\else
\newenvironment{esempio}[3]%
{
    \vspace{1.5ex}
    \rlap{\underline{#1}}
    \par
    \setlength{\parindent}{0cm}
    \nopagebreak
    \leftskip=#2cm
    \rightskip=#3cm
}
{
    \par
}
\fi

\doendnotes{C}
\bigskip
\vfill

\clearpage

\footnotesize

\ifkorrekturansicht
  \lohead{\textsc{register}}
\fi

% theindex-Environment neu definieren ohne reledmac
\makeatletter
\renewenvironment{theindex}{%
  \ifkorrekturansicht
    \section*{\indexname}%
  \else
    \subsubsection*{Index der erwähnten Entitäten}%
  \fi
  \setlength{\parindent}{0pt}%
  \setlength{\parskip}{0pt plus 0.3pt}%
  \let\item\@idxitem
}{%
  \ifkorrekturansicht\clearpage\fi
}
\makeatother

\IfFileExists{\jobname-pw.ind}{\input{\jobname-pw.ind}}{}

% Quellenangabe nur in der Leseansicht
\ifkorrekturansicht\else
% Fallback-Definitionen, falls die .tex-Datei \titel etc. nicht gesetzt hat
\providecommand{\titel}{}
\providecommand{\editorInnen}{}
\providecommand{\dateiname}{\jobname}

\vspace{3cm}

\vfill

\footnotesize
\textsc{Quelle}: \titel. Herausgegeben von {\editorInnen}. In: \emph{Arthur Schnitzler: Briefwechsel mit Autorinnen und Autoren}.
 Digitale Edition, https://schnitzler-briefe.acdh.oeaw.ac.at/{\dateiname}.html (Stand \today)
\fi

\end{document}


      