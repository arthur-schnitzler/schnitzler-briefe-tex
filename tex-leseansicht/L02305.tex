%% latex-leseansicht-vorspann.tex
%% Vorspann für die Leseansicht.
%% Lädt die gemeinsame Datei latex-vorspann.tex mit nicht gesetztem Schalter.

\newif\ifkorrekturansicht
\korrekturansichtfalse

\input{../tex-inputs/latex-vorspann}


         
         \renewcommand{\erwaehntePersonen}{Personen: Leopold von Andrian-Werburg, Hermann Bahr, Robert Michel}
         \renewcommand{\erwaehnteInstitutionen}{Institutionen: Burgtheater}
         \renewcommand{\erwaehnteOrte}{Orte: Wien}
         \renewcommand{\erwaehnteWerke}{}
               \section[Erhard Buschbeck an Arthur Schnitzler, 24. 9. 1918]{ Erhard Buschbeck an Arthur Schnitzler, 24. 9. 1918}\nopagebreak\mylabel{v}\rehead{ }\begin{ledgroupsized}[t]{13cm}\normalsize\beginnumbering \toendnotes[C]{\smallbreak\pagebreak[2]} \Standort{CUL, Schnitzler, B 5b.}
\physDesc{Brief, 1 Blatt, 1 Seite, 851 Zeichen
\newline{}Handschrift: schwarze Tinte, lateinische Kurrent
\newline{}Schnitzler: 1) mit Bleistift ergänzt: »Bahr.« und Vermerk »\textsc{A}«, vermutlich für »Abzuschreiben«/»Abschrift«  2) mit rotem Buntstift eine Unterstreichung
\newline{}Ordnung: mit Bleistift von unbekannter Hand nummeriert:
                                    »183« }\buchAbdrucke{\weitereDrucke{Hermann Bahr, Arthur Schnitzler: \emph{Briefwechsel, Aufzeichnungen, Dokumente (1891–1931)}. Hg. Kurt Ifkovits und Martin Anton Müller. Göttingen: \emph{Wallstein} 2018, S. 520.} }\toendnotes[C]{\smallbreak}\pstart
           \noindent{}{\pb}\textcolor{gray}{\textbf{\textit{\label{T_L02305-1v}\edtext{k. k. Hofburgtheater\orgindex{Burgtheater@Burgtheater|pw}}{\lemma{\textnormal{\emph{k. k. Hofburgtheater}}}\Cendnote{\textnormal{Prägestempel}}}\label{T_L02305-1h}}}}\hfill Wien\oindex{Wien@\textbf{Wien}|pw}, 24. Sept. 1918.\pend
           \pstart
           \textcolor{gray}{\textbf{Direction}}\pend
           \pstart{}Sehr geehrter Herr Doktor,\pend\pstart
           Hermann Bahr\pwindex{Bahr, Hermann 19.07.1863 – 15.01.1934@\textsc{Bahr, Hermann} (19.07.1863 – 15.01.1934), \emph{Schriftsteller, Kritiker}|pw} hat mich gebeten, Ihnen zu sagen,
               daß ein Beschluss vorliegt, die Generalproben vorläufig nicht mehr öffentlich
               abzuhalten und nur die Vertreter der Wiener\oindex{Wien@\textbf{Wien}|pw}
               Tagespresse und Mitglieder des Hauses einzulassen. Es ist ihm sehr schmerzlich, daß
               er infolge der Verreisung des General-Intendanten und Major Michels\pwindex{Michel, Robert 24.02.1876 – 12.02.1957@\textsc{Michel, Robert} (24.02.1876 – 12.02.1957), \emph{Schriftsteller}|pw} bis zu diesem Freitag eine Ausnahme für Sie,
               hochgeehrter Herr Doktor, wird nicht mehr erreichen können. Bahr\pwindex{Bahr, Hermann 19.07.1863 – 15.01.1934@\textsc{Bahr, Hermann} (19.07.1863 – 15.01.1934), \emph{Schriftsteller, Kritiker}|pw} glaubt aber sicher, daß das für die kommenden Male nach
               einer Intervention bei Exc. Andrian\pwindex{Andrian-Werburg, Leopold von 09.05.1875 – 19.11.1951@\textsc{Andrian-Werburg, Leopold von} (09.05.1875 – 19.11.1951), \emph{Schriftsteller, Diplomat}|pw} ohne
               weiteres wird geschehen können. Daß es ganz seinen Wünschen entspricht und es ihm
               natürlich sehr lieb \introOben{}und wertvoll\introOben{} wäre, Arthur Schnitzler
               dabei zu wissen, soll ich Ihnen, sehr geehrter Herr Dr., noch ganz besonders
               sagen.\pend
           \pstart
           In größter Hochachtung{\\[\baselineskip]}ergebenst{\\[\baselineskip]}\spacefill\mbox{ErhardBuschbeck}\pend
           \leftskip=0em{}
         
         \endnumbering\mylabel{h}\end{ledgroupsized}  \newcommand{\dateiname}{L02305}\newcommand{\titel}{Erhard Buschbeck an Arthur Schnitzler, 24. 9. 1918}\newcommand{\editorInnen}{ Martin Anton Müller und Gerd-Hermann Susen}%% latex-leseansicht-abspann.tex
%% Abspann für die Leseansicht.
%% Der Schalter \ifkorrekturansicht ist bereits durch den Vorspann gesetzt.

%% latex-abspann.tex
%% Gemeinsamer Abspann für Korrekturansicht und Leseansicht.
%% Setzt den Schalter \ifkorrekturansicht voraus (gesetzt in den
%% einbindenden Dateien latex-korrekturansicht-abspann.tex bzw.
%% latex-leseansicht-abspann.tex).
%% ---------------------------------------------------------------

\normalsize

% Das esempio-Environment wird nur in der Leseansicht benötigt
\ifkorrekturansicht\else
\newenvironment{esempio}[3]%
{
    \vspace{1.5ex}
    \rlap{\underline{#1}}
    \par
    \setlength{\parindent}{0cm}
    \nopagebreak
    \leftskip=#2cm
    \rightskip=#3cm
}
{
    \par
}
\fi

\doendnotes{C}
\bigskip
\vfill

\clearpage

\footnotesize

\ifkorrekturansicht
  \lohead{\textsc{register}}
\fi

% theindex-Environment neu definieren ohne reledmac
\makeatletter
\renewenvironment{theindex}{%
  \ifkorrekturansicht
    \section*{\indexname}%
  \else
    \subsubsection*{Index der erwähnten Entitäten}%
  \fi
  \setlength{\parindent}{0pt}%
  \setlength{\parskip}{0pt plus 0.3pt}%
  \let\item\@idxitem
}{%
  \ifkorrekturansicht\clearpage\fi
}
\makeatother

\IfFileExists{\jobname-pw.ind}{\input{\jobname-pw.ind}}{}

% Quellenangabe nur in der Leseansicht
\ifkorrekturansicht\else
% Fallback-Definitionen, falls die .tex-Datei \titel etc. nicht gesetzt hat
\providecommand{\titel}{}
\providecommand{\editorInnen}{}
\providecommand{\dateiname}{\jobname}

\vspace{3cm}

\vfill

\footnotesize
\textsc{Quelle}: \titel. Herausgegeben von {\editorInnen}. In: \emph{Arthur Schnitzler: Briefwechsel mit Autorinnen und Autoren}.
 Digitale Edition, https://schnitzler-briefe.acdh.oeaw.ac.at/{\dateiname}.html (Stand \today)
\fi

\end{document}


      