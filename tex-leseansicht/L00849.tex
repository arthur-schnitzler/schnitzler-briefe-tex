%% latex-korrekturansicht-vorspann.tex
%% Vorspann für die Korrekturansicht.
%% Lädt die gemeinsame Datei latex-vorspann.tex mit gesetztem Schalter.

\newif\ifkorrekturansicht
\korrekturansichttrue

\input{../tex-inputs/latex-vorspann}


\section[Hugo von Hofmannsthal an Arthur Schnitzler, 2. 10. {[}1898{]}]{L00849 Hugo von Hofmannsthal an Arthur Schnitzler, 2. 10. {[}1898{]}}
\nopagebreak\mylabel{L00849v}
\rehead{ }\normalsize\beginnumbering\briefempfaengerindex{Schnitzler, Arthur@\textsc{Schnitzler, Arthur}!zzzHofmannsthal, Hugo von@\emph{von Hugo von Hofmannsthal}!1898-10-021@{2. 10. {[}1898{]}}|(be}
\toendnotes[C]{\smallbreak\pagebreak[2]}\Standort{CUL, Schnitzler, B 43.}
\physDesc{Brief, 1 Blatt, 3 Seiten, 1001 Zeichen
\newline{}Handschrift: schwarze Tinte, deutsche Kurrent
\newline{}Schnitzler: mit Bleistift die Jahreszahl ergänzt: »98« 
\newline{}Ordnung: 1) mit Bleistift von unbekannter Hand nummeriert: »\strikeout{127}«  2) mit Bleistift von unbekannter Hand nummeriert:
                                    »124«}
\buchAbdrucke{\weitereDrucke{Hugo von Hofmannsthal, Arthur Schnitzler: \emph{Briefwechsel}. Frankfurt am Main: \emph{S. Fischer} 1964, S. 112.} }\toendnotes[C]{\smallbreak}
\pstart
           \centering{}{\pb}\textcolor{gray}{\textbf{Hôtel de l’Europe Venise\oindex{Hotel de l Europe [Venedig]@\textbf{Hotel de l’Europe [Venedig]}, \emph{Hotel (K.HTL)}|pw}}}\pend
           
\pstart
           \centering{}\hspace*{5em}\textcolor{gray}{\textbf{sur le Grand Canal\oindex{Canal Grande@\textbf{Canal Grande}, \emph{Fluss (N.FLS)}|pw}}}\pend
           
\pstart
           \centering{}\textcolor{gray}{\textbf{Marseille Frères\pwindex{Marseille, Augusto @\textsc{Marseille, Augusto}, \emph{Hotelier/Hotelière}|pw}, Prop\textsuperscript{res}}}\pend
           
\pstart
           \centering{}\textcolor{gray}{\textbf{Vue prise de l’hôtel}}\pend
           
\pstart
           \raggedleft{}Venedig\oindex{Venedig@\textbf{Venedig}, \emph{P.PPLA}|pw}{ }2\textsuperscript{ten} X.\pend
           
\pstart{}mein lieber Arthur\pend\vspace{0.5em}
\pstart
           ſo hör ich auf einmal von meinen Eltern\pwindex{Hofmannsthal, Hugo August von 21.12.1841 – 08.12.1915@\textsc{Hofmannsthal, Hugo August von} (21.12.1841 – 08.12.1915), \emph{Bankdirektor/Bankdirektorin}|pwv}\pwindex{Hofmannsthal, Anna von 27.01.1849 – 22.03.1904@\textsc{Hofmannsthal, Anna von} (27.01.1849 – 22.03.1904)|pwv}, daſs die Aufführung vom »Vermächtnis\pwindex{Vermaechtnis. Schauspiel in drei Akten@\emph{Das Vermächtnis. Schauspiel in drei Akten}|pw}« unmittelbar bevorſteht und denke Sie auf den Proben,
               in dem halbfinſteren Theater, u der Luft die Sie ſo gern haben und die ich auch ſehr
               gern zu haben anfange. Dann kommen mir Wien\oindex{Wien@\textbf{Wien}, \emph{A.ADM2}|pw}er
               Sommerabende ins Gedächtnis, das Bad im Neufchatelerſee\oindex{Neuenburgersee@\textbf{Neuenburgersee}, \emph{See (N.SEE)}|pw}, der letzte {\pb}Tag am Dampfſchiff und ich denke
               mir, wie ſchön und gut es iſt, was für ein großes Glück, daſs ich Menſchen wie Sie ſo
               früh hab finden und behalten dürfen.\pend
           
\pstart
           Ich war bei den Thürmen\oindex{Bologna@\textbf{Bologna}, \emph{P.PPLA}|pwv}, von
               denen Sie mir einen geſchenkt\strikeout{en} haben, dann in Florenz\oindex{Florenz@\textbf{Florenz}, \emph{P.PPLA}|pw}, worüber mehr als viel zu erzählen iſt und
               ſitze nun ſeit 14 Tagen hier ſo fieberhaft fleißig wie ichs manchmal und leider ſo
               ſelten ſein kann. Etwa den 10\textsuperscript{ten} bin ich in Wien\oindex{Wien@\textbf{Wien}, \emph{A.ADM2}|pw}, höre von Berlin\oindex{Berlin@\textbf{Berlin}, \emph{P.PPLC}|pw}, höre endlich den »Kakadu\pwindex{gruene Kakadu – Paracelsus – Die Gefaehrtin. Drei Einakter@\emph{Der grüne Kakadu – Paracelsus – Die Gefährtin. Drei Einakter}|pw}«,
                  \label{K_L00849-1v}\edtext{leſe wohl eine venezianiſche Comödie\pwindex{Abenteurer und die Saengerin oder Die Geschenke des Lebens@\emph{Der Abenteurer und die Sängerin oder Die Geschenke des Lebens}|pwv} vor}{\lemma{\textnormal{\emph{leſe … vor}}}\Cendnote{\textnormal{Am 30. 10. 1898 las er \emph{Der Abenteurer und die Sängerin}\pwindex{Abenteurer und die Saengerin oder Die Geschenke des Lebens@\emph{Der Abenteurer und die Sängerin oder Die Geschenke des Lebens}|pwk}{ }Schnitzler und Beer-Hofmann\pwindex{Beer-Hofmann, Richard 1866-07-11 – 1945-09-26@\textsc{Beer-Hofmann, Richard} (1866-07-11 – 1945-09-26), \emph{Schriftsteller/Schriftstellerin}|pwk} vor.}}}\label{K_L00849-1}, erzähle von \textsc{d’Annunzio}\pwindex{DAnnunzio, Gabriele 12.03.1863 – 01.03.1938@\textsc{D’Annunzio, Gabriele} (12.03.1863 – 01.03.1938), \emph{Schriftsteller/Schriftstellerin}|pw}, und ſage wie {\pb}alle Herbſte
               aber noch mit viel tieferer Überzeugung als früher, daſs man ſich öfter ſehen
               muſs.\pend
           
\pstart
           Herzlich Ihr{\\[\baselineskip]}\spacefill\mbox{Hugo.}\pend
           \leftskip=0em{}\selectlanguage{ngerman}\endnumbering\briefempfaengerindex{Schnitzler, Arthur@\textsc{Schnitzler, Arthur}!zzzHofmannsthal, Hugo von@\emph{von Hugo von Hofmannsthal}!1898-10-021@{2. 10. {[}1898{]}}|)be}\mylabel{L00849h}  \normalsize

\doendnotes{C}
\bigskip
\vfill

\clearpage

\footnotesize

\lohead{\textsc{register}}

% Definiere theindex-Environment komplett neu ohne reledmac
\makeatletter
\renewenvironment{theindex}{%
  \section*{\indexname}%
  \setlength{\parindent}{0pt}%
  \setlength{\parskip}{0pt plus 0.3pt}%
  \let\item\@idxitem
}{%
  \clearpage
}
\makeatother

\IfFileExists{\jobname-pw.ind}{\input{\jobname-pw.ind}}{}

\end{document}

      