%% latex-leseansicht-vorspann.tex
%% Vorspann für die Leseansicht.
%% Lädt die gemeinsame Datei latex-vorspann.tex mit nicht gesetztem Schalter.

\newif\ifkorrekturansicht
\korrekturansichtfalse

\input{../tex-inputs/latex-vorspann}


         
         \newcommand{\erwaehntePersonen}{Personen: }
         \newcommand{\erwaehnteInstitutionen}{}
         \newcommand{\erwaehnteOrte}{}
         \newcommand{\erwaehnteWerke}{
               \section[Hugo von Hofmannsthal an Arthur Schnitzler, 2. 10. {[}1898{]}]{ Hugo von Hofmannsthal an Arthur Schnitzler, 2. 10. {[}1898{]}}\nopagebreak\mylabel{v}\rehead{ }\begin{ledgroupsized}[t]{13cm}\normalsize\beginnumbering \toendnotes[C]{\smallbreak\pagebreak[2]} \Standort{CUL, Schnitzler, B 43.}
\physDesc{Brief, 1 Blatt, 3 Seiten
\newline{}Handschrift: schwarze Tinte, deutsche Kurrent
\newline{}Schnitzler: mit Bleistift die Jahreszahl ergänzt: »98« \newline{}Ordnung: 1) mit Bleistift von unbekannter Hand nummeriert: »\strikeout{127}«  2) mit Bleistift von unbekannter Hand nummeriert: »124«}\buchAbdrucke{\weitereDrucke{Hugo von Hofmannsthal, Arthur Schnitzler: \emph{Briefwechsel}. Hg. Therese Nickl und Heinrich Schnitzler. Frankfurt am Main: \emph{S. Fischer} 1964, S. 112.} }\toendnotes[C]{\smallbreak}\pstart
           \noindent{}\centering{}{\pb}\textcolor{gray}{\textbf{Hôtel de l’Europe Venise\oindex{XXXX Ortsangabe fehlt|pw}}}\pend
           \pstart
           \noindent{}\centering{}\hspace*{5em}\textcolor{gray}{\textbf{sur le Grand Canal\oindex{XXXX Ortsangabe fehlt|pw}}}\pend
           \pstart
           \noindent{}\centering{}\textcolor{gray}{\textbf{Marseille Frères\pwindex{\textcolor{red}{\textsuperscript{XXXX1 indx}}|pw}, Prop\textsuperscript{res}}}\pend
           \pstart
           \noindent{}\centering{}\textcolor{gray}{\textbf{Vue prise de l’hôtel}}\pend
           \pstart
           \raggedleft{}Venedig\oindex{XXXX Ortsangabe fehlt|pw}{ }2\textsuperscript{ten} X.\pend
           \pstart{}mein lieber Arthur\pend\pstart
           ſo hör ich auf einmal von meinen Eltern\pwindex{\textcolor{red}{\textsuperscript{XXXX1 indx}}|pwv}\pwindex{\textcolor{red}{\textsuperscript{XXXX1 indx}}|pwv}, daſs die Aufführung vom »Vermächtnis\textcolor{red}{\textsuperscript{XXXX indx}}« unmittelbar bevorſteht und denke Sie auf den
                    Proben, in dem halbfinſteren Theater, u der Luft die Sie ſo gern haben und die
                    ich auch ſehr gern zu haben anfange. Dann kommen mir Wien\oindex{XXXX Ortsangabe fehlt|pw}er Sommerabende ins Gedächtnis, das Bad im Neufchatelerſee\oindex{XXXX Ortsangabe fehlt|pw}, der letzte {\pb}Tag am Dampfſchiff und ich
                    denke mir, wie ſchön und gut es iſt, was für ein großes Glück, daſs ich Menſchen
                    wie Sie ſo früh hab finden und behalten dürfen.\pend
           \pstart
           Ich war bei den Thürmen\oindex{XXXX Ortsangabe fehlt|pwv}, von
                    denen Sie mir einen geſchenkt\strikeout{en} haben, dann in
                        Florenz\oindex{XXXX Ortsangabe fehlt|pw}, worüber mehr als viel zu erzählen
                    iſt und ſitze nun ſeit 14 Tagen hier ſo fieberhaft fleißig wie ichs manchmal und
                    leider ſo ſelten ſein kann. Etwa den 10\textsuperscript{ten} bin ich
                    in Wien\oindex{XXXX Ortsangabe fehlt|pw}, höre von Berlin\oindex{XXXX Ortsangabe fehlt|pw}, höre endlich den »Kakadu\textcolor{red}{\textsuperscript{XXXX indx}}«,
                        \label{K_L00849_1v}\edtext{leſe wohl eine venezianiſche Comödie\textcolor{red}{\textsuperscript{XXXX indx}}
                        vor}{\lemma{\textnormal{\emph{leſe … vor}}}\Cendnote{\textnormal{Am 30. 10. 1898
                        las er \emph{Der Abenteurer und die Sängerin}\textcolor{red}{\textsuperscript{XXXX indx}}{ }Schnitzler\pwindex{\textcolor{red}{\textsuperscript{XXXX1 indx}}|pwk} und Beer-Hofmann\pwindex{\textcolor{red}{\textsuperscript{XXXX1 indx}}|pwk} vor.}}}\label{K_L00849_1h}, erzähle von \textsc{d’Annunzio}\pwindex{\textcolor{red}{\textsuperscript{XXXX1 indx}}|pw}, und ſage wie {\pb}alle
                    Herbſte aber noch mit viel tieferer Überzeugung als früher, daſs man ſich öfter
                    ſehen muſs.\pend
           \pstart
           Herzlich Ihr{\\[\baselineskip]}\spacefill\mbox{Hugo.}\pend
           \leftskip=0em{}
         
         \endnumbering\mylabel{h}\end{ledgroupsized}  \newcommand{\dateiname}{L00849}\newcommand{\titel}{Hugo von Hofmannsthal an Arthur Schnitzler, 2. 10. [1898]}\newcommand{\editorInnen}{Martin Anton Müller und Gerd-Hermann Susen}%% latex-leseansicht-abspann.tex
%% Abspann für die Leseansicht.
%% Der Schalter \ifkorrekturansicht ist bereits durch den Vorspann gesetzt.

%% latex-abspann.tex
%% Gemeinsamer Abspann für Korrekturansicht und Leseansicht.
%% Setzt den Schalter \ifkorrekturansicht voraus (gesetzt in den
%% einbindenden Dateien latex-korrekturansicht-abspann.tex bzw.
%% latex-leseansicht-abspann.tex).
%% ---------------------------------------------------------------

\normalsize

% Das esempio-Environment wird nur in der Leseansicht benötigt
\ifkorrekturansicht\else
\newenvironment{esempio}[3]%
{
    \vspace{1.5ex}
    \rlap{\underline{#1}}
    \par
    \setlength{\parindent}{0cm}
    \nopagebreak
    \leftskip=#2cm
    \rightskip=#3cm
}
{
    \par
}
\fi

\doendnotes{C}
\bigskip
\vfill

\clearpage

\footnotesize

\ifkorrekturansicht
  \lohead{\textsc{register}}
\fi

% theindex-Environment neu definieren ohne reledmac
\makeatletter
\renewenvironment{theindex}{%
  \ifkorrekturansicht
    \section*{\indexname}%
  \else
    \subsubsection*{Index der erwähnten Entitäten}%
  \fi
  \setlength{\parindent}{0pt}%
  \setlength{\parskip}{0pt plus 0.3pt}%
  \let\item\@idxitem
}{%
  \ifkorrekturansicht\clearpage\fi
}
\makeatother

\IfFileExists{\jobname-pw.ind}{\input{\jobname-pw.ind}}{}

% Quellenangabe nur in der Leseansicht
\ifkorrekturansicht\else
% Fallback-Definitionen, falls die .tex-Datei \titel etc. nicht gesetzt hat
\providecommand{\titel}{}
\providecommand{\editorInnen}{}
\providecommand{\dateiname}{\jobname}

\vspace{3cm}

\vfill

\footnotesize
\textsc{Quelle}: \titel. Herausgegeben von {\editorInnen}. In: \emph{Arthur Schnitzler: Briefwechsel mit Autorinnen und Autoren}.
 Digitale Edition, https://schnitzler-briefe.acdh.oeaw.ac.at/{\dateiname}.html (Stand \today)
\fi

\end{document}


      