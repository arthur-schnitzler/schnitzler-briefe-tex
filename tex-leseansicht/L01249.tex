%% latex-korrekturansicht-vorspann.tex
%% Vorspann für die Korrekturansicht.
%% Lädt die gemeinsame Datei latex-vorspann.tex mit gesetztem Schalter.

\newif\ifkorrekturansicht
\korrekturansichttrue

\input{../tex-inputs/latex-vorspann}


\section[Hermann Bahr an Arthur Schnitzler, 20. 11. {[}1902?{]}]{L01249 Hermann Bahr an Arthur Schnitzler, 20. 11. {[}1902?{]}}
\nopagebreak\mylabel{L01249v}
\rehead{ }\normalsize\beginnumbering\briefempfaengerindex{Schnitzler, Arthur@\textsc{Schnitzler, Arthur}!zzzBahr, Hermann@\emph{von Hermann Bahr}!1902-11-201@{20. {[}11. 1902?{]}}|(be}
\toendnotes[C]{\smallbreak\pagebreak[2]}\Standort{CUL, Schnitzler, B 5b.}
\physDesc{Brief, 1 Blatt, 1 Seite, 181 Zeichen
\newline{}Handschrift: schwarze Tinte, deutsche Kurrent
\newline{}Ordnung: mit Bleistift von unbekannter Hand nummeriert:
                                    »92« }
\buchAbdrucke{\weitereDrucke{Hermann Bahr, Arthur Schnitzler: \emph{Briefwechsel, Aufzeichnungen, Dokumente (1891–1931)}. Göttingen: \emph{Wallstein} 2018, S. 244.} }\toendnotes[C]{\smallbreak}
\pstart
           \raggedleft{}{\pb}\label{K_L01249-1v}\edtext{20/\textcolor{gray}{11}}{\lemma{\textnormal{\emph{20/11}}}\Cendnote{\textnormal{Das Jahr ist unklar, als
                        Monatsangabe wäre auch eine lateinische Zahl ›II‹ (für Februar) möglich. Das
                        Briefpapier deckt sich mit dem am 15. 3. 1903 verwendeten. Wir
                        folgen der Einordnung der Abschrift.}}}\label{K_L01249-1}\pend
           
\pstart\center{}Lieber Arthur!\pend\vspace{0.5em}
\pstart
           Herzlichen Dank – das war wirklich ſehr lieb von Dir. Ich will immer einmal zu Dir
               kommen, aber, aber! Der Journalismus frißt mich auf.\pend
           
\pstart
           Nochmals dankend{\\[\baselineskip]}Dein{\\[\baselineskip]}\spacefill\mbox{Hermann}\pend
           \leftskip=0em{}\selectlanguage{ngerman}\endnumbering\briefempfaengerindex{Schnitzler, Arthur@\textsc{Schnitzler, Arthur}!zzzBahr, Hermann@\emph{von Hermann Bahr}!1902-11-201@{20. {[}11. 1902?{]}}|)be}\mylabel{L01249h}  \normalsize

\doendnotes{C}
\bigskip
\vfill

\clearpage

\footnotesize

\lohead{\textsc{register}}

% Definiere theindex-Environment komplett neu ohne reledmac
\makeatletter
\renewenvironment{theindex}{%
  \section*{\indexname}%
  \setlength{\parindent}{0pt}%
  \setlength{\parskip}{0pt plus 0.3pt}%
  \let\item\@idxitem
}{%
  \clearpage
}
\makeatother

\IfFileExists{\jobname-pw.ind}{\input{\jobname-pw.ind}}{}

\end{document}

      