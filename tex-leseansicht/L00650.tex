%% latex-korrekturansicht-vorspann.tex
%% Vorspann für die Korrekturansicht.
%% Lädt die gemeinsame Datei latex-vorspann.tex mit gesetztem Schalter.

\newif\ifkorrekturansicht
\korrekturansichttrue

\input{../tex-inputs/latex-vorspann}


\section[Hugo von Hofmannsthal an Arthur Schnitzler, {[}12. 3. 1897{]}]{L00650 Hugo von Hofmannsthal an Arthur Schnitzler, {[}12. 3. 1897{]}}
\nopagebreak\mylabel{L00650v}
\rehead{ }\normalsize\beginnumbering\briefempfaengerindex{Schnitzler, Arthur@\textsc{Schnitzler, Arthur}!zzzHofmannsthal, Hugo von@\emph{von Hugo von Hofmannsthal}!1897-03-122@{{[}12. 3. 1897{]}}|(be}
\toendnotes[C]{\smallbreak\pagebreak[2]}\Standort{CUL, Schnitzler, B 43b/1.}
\physDesc{Brief, 1 Blatt, 3 Seiten, 418 Zeichen (gedrucktes Wappen in blauer Farbe)
\newline{}Handschrift: Bleistift, deutsche Kurrent
\newline{}Schnitzler: mit Bleistift das Datum ergänzt: »12/\textcolor{gray}{3} 97« 
\newline{}Ordnung: 1) mit Bleistift von unbekannter Hand nummeriert: »\strikeout{87}«  2) mit Bleistift von unbekannter Hand nummeriert:
                                    »86«}
\buchAbdrucke{\weitereDrucke{1) Hugo von Hofmannsthal, Arthur Schnitzler: \emph{Briefwechsel}. Frankfurt am Main: \emph{S. Fischer} 1964, S. 79.} \weitereDrucke{2) Hermann Bahr, Arthur Schnitzler: \emph{Briefwechsel, Aufzeichnungen, Dokumente (1891–1931)}. Göttingen: \emph{Wallstein} 2018, S. 137.} }
\pstart{}{\pb}lieber Arthur\pend\vspace{0.5em}
\pstart
           wenn die Geſchichte noch lang dauert werd ich ja vielleicht etwas haben was mir zum
               Vorleſen paſst, dann werd ichs ja gern thuen. Wenn ich aber keine paſſend\substVorne{}\textsuperscript{en}\substDazwischen{}eren\substHinten{}{ }Sachen habe als {\pb}jetzt, ſo \uuline{nicht}.\pend
           
\pstart
           Daß man meinen Namen vorläufig aufs Programm ſetzt, wenn Ihrer etc. drauf ſteht, iſt
               mir natürlich ganz recht, aber Ihnen und Bahr\pwindex{Bahr, Hermann 19.07.1863 – 15.01.1934@\textsc{Bahr, Hermann} (19.07.1863 – 15.01.1934), \emph{Schriftsteller/Schriftstellerin, Kritiker/Kritikerin}|pw}
               gegenüber verpflichte ich mich eben abſolut nur unter der obigen {\pb}Bedingung.\pend
           
\pstart
           Ihr{\\[\baselineskip]}\spacefill\mbox{Hugo.}\pend
           \leftskip=0em{}\selectlanguage{ngerman}\endnumbering\briefempfaengerindex{Schnitzler, Arthur@\textsc{Schnitzler, Arthur}!zzzHofmannsthal, Hugo von@\emph{von Hugo von Hofmannsthal}!1897-03-122@{{[}12. 3. 1897{]}}|)be}\mylabel{L00650h}  \normalsize

\doendnotes{C}
\bigskip
\vfill

\clearpage

\footnotesize

\lohead{\textsc{register}}

% Definiere theindex-Environment komplett neu ohne reledmac
\makeatletter
\renewenvironment{theindex}{%
  \section*{\indexname}%
  \setlength{\parindent}{0pt}%
  \setlength{\parskip}{0pt plus 0.3pt}%
  \let\item\@idxitem
}{%
  \clearpage
}
\makeatother

\IfFileExists{\jobname-pw.ind}{\input{\jobname-pw.ind}}{}

\end{document}

      