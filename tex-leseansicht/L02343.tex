%% latex-leseansicht-vorspann.tex
%% Vorspann für die Leseansicht.
%% Lädt die gemeinsame Datei latex-vorspann.tex mit nicht gesetztem Schalter.

\newif\ifkorrekturansicht
\korrekturansichtfalse

\input{../tex-inputs/latex-vorspann}


         
         \newcommand{\erwaehntePersonen}{Personen: Viktor Franz Patzner, Maria Pollak}
         \newcommand{\erwaehnteOrte}{Orte: Erholungsheim der Bundesbeamten, Gutenstein, Schweden, Wien}
         \newcommand{\erwaehnteWerke}{Werke: Über Rechtsprinzipien. Eine analytische Untersuchung}
               \section[Robert Adam an Arthur Schnitzler, 17. 6. 1920]{ Robert Adam an Arthur Schnitzler, 17. 6. 1920}\nopagebreak\mylabel{v}\rehead{ }\begin{ledgroupsized}[t]{13cm}\normalsize\beginnumbering \toendnotes[C]{\smallbreak\pagebreak[2]} \Standort{CUL, Schnitzler, B 1.}
\physDesc{Brief, 1 Blatt, 3 Seiten
\newline{}Handschrift: blaue Tinte, deutsche Kurrent
\newline{}Schnitzler: 1) mit Bleistift beschriftet: »\textsc{Adam}«  2) mit rotem Buntstift mehrere Unterstreichungen\newline{}Ordnung: mit Bleistift von unbekannter Hand nummeriert:
                                        »15« }\Standort{Wien, Österreichische Nationalbibliothek, Cod.ser. 52.268, 74 recto und 73 recto.}
\physDesc{Brief, maschinelle Abschrift
\newline{}Schreibmaschine}\toendnotes[C]{\smallbreak}\pstart
           \raggedleft{}{\pb}Wien\oindex{Wien@\textbf{Wien}|pw}, am 17. Juni 1920\pend
           \pstart\center{}Hochverehrter Herr Doktor!\pend\pstart
           Beſten Dank für Ihre Karte! Daß Sie ſich mit der Lektüre meines Aufſatzes\pwindex{Adam, Robert 20.04.1877 – 16.10.1961@\textsc{Adam, Robert} (20.04.1877 – 16.10.1961), \emph{Schriftsteller, Richter}!Ueber Rechtsprinzipien. Eine analytische Untersuchung1920@\strich\emph{Über Rechtsprinzipien. Eine analytische Untersuchung} {[}1920{]}|pwv} plagen, darf ich gar nicht
                    verlangen!\pend
           \pstart
           Ich habe meinem Magenleiden, das mich ſeit mehr als einem Jahre quälte und faſt
                    arbeitsunfähig, jedenfalls aber lebensunluſtig machte, endlich dadurch ein Ende
                    gemacht, daß ich mich – Mitte Mai – operieren ließ. Ich bin noch
                    immer ſehr ſchwach, gehe aber doch ſchon aus und würde ſehr gerne {\pb}im Laufe der nächſten Woche – den
                        26. muß ich ausnehmen – zu Ihnen kommen; bitte mir einen Tag zu
                    beſtimmen.\pend
           \pstart
           Am 3. Juli fahre ich mit Frau\pwindex{Pollak, Maria 06.10.1889 – 27.03.1948@\textsc{Pollak, Maria} (06.10.1889 – 27.03.1948)|pwv} und Kind\pwindex{Patzner, Viktor Franz 13.09.1916 – 21.12.1982@\textsc{Patzner, Viktor Franz} (13.09.1916 – 21.12.1982), \emph{Rechtsanwalt}|pwv} nach Gutenſtein\oindex{Gutenstein@\textbf{Gutenstein}|pw}, wo uns ein von den Schweden\oindex{Schweden@\textbf{Schweden}|pw} beliefertes Richtererholungsheim, das den verſprechenden
                    Namen: »Heim der Ruhe\oindex{Erholungsheim der Bundesbeamten@\textbf{Erholungsheim der Bundesbeamten}|pw}« führt, für wenig Geld
                    durch 4 Wochen verpflegen ſoll. Was dann geſchieht, hängt davon ab, ob ich mich
                        anfangs Auguſt bereits zur Wiederaufnahme des Dienſtes ſtark
                    genug fühlen werde oder noch irgendwo Erholungsmöglichkeit ſuchen muß.\pend
           \pstart
           Gearbeitet habe ich \strikeout{die}{ }ſeit dem Herbſt gar nichts, aber viel Lehrreiches
                    geleſen, vor allem vieles Lateiniſche.\pend
           \pstart
           {\pb}Mit den ergebenſten Grüßen\pend
           \pstart
           Ihr{\\[\baselineskip]}\spacefill\mbox{D\textsuperscript{r}RAdam}\pend
           \leftskip=0em{}
         
         \endnumbering\mylabel{h}\end{ledgroupsized}  \newcommand{\dateiname}{L02343}\newcommand{\titel}{Robert Adam an Arthur Schnitzler, 17. 6. 1920}\newcommand{\editorInnen}{Martin Anton Müller und Gerd-Hermann Susen}%% latex-leseansicht-abspann.tex
%% Abspann für die Leseansicht.
%% Der Schalter \ifkorrekturansicht ist bereits durch den Vorspann gesetzt.

%% latex-abspann.tex
%% Gemeinsamer Abspann für Korrekturansicht und Leseansicht.
%% Setzt den Schalter \ifkorrekturansicht voraus (gesetzt in den
%% einbindenden Dateien latex-korrekturansicht-abspann.tex bzw.
%% latex-leseansicht-abspann.tex).
%% ---------------------------------------------------------------

\normalsize

% Das esempio-Environment wird nur in der Leseansicht benötigt
\ifkorrekturansicht\else
\newenvironment{esempio}[3]%
{
    \vspace{1.5ex}
    \rlap{\underline{#1}}
    \par
    \setlength{\parindent}{0cm}
    \nopagebreak
    \leftskip=#2cm
    \rightskip=#3cm
}
{
    \par
}
\fi

\doendnotes{C}
\bigskip
\vfill

\clearpage

\footnotesize

\ifkorrekturansicht
  \lohead{\textsc{register}}
\fi

% theindex-Environment neu definieren ohne reledmac
\makeatletter
\renewenvironment{theindex}{%
  \ifkorrekturansicht
    \section*{\indexname}%
  \else
    \subsubsection*{Index der erwähnten Entitäten}%
  \fi
  \setlength{\parindent}{0pt}%
  \setlength{\parskip}{0pt plus 0.3pt}%
  \let\item\@idxitem
}{%
  \ifkorrekturansicht\clearpage\fi
}
\makeatother

\IfFileExists{\jobname-pw.ind}{\input{\jobname-pw.ind}}{}

% Quellenangabe nur in der Leseansicht
\ifkorrekturansicht\else
% Fallback-Definitionen, falls die .tex-Datei \titel etc. nicht gesetzt hat
\providecommand{\titel}{}
\providecommand{\editorInnen}{}
\providecommand{\dateiname}{\jobname}

\vspace{3cm}

\vfill

\footnotesize
\textsc{Quelle}: \titel. Herausgegeben von {\editorInnen}. In: \emph{Arthur Schnitzler: Briefwechsel mit Autorinnen und Autoren}.
 Digitale Edition, https://schnitzler-briefe.acdh.oeaw.ac.at/{\dateiname}.html (Stand \today)
\fi

\end{document}


      