%% latex-leseansicht-vorspann.tex
%% Vorspann für die Leseansicht.
%% Lädt die gemeinsame Datei latex-vorspann.tex mit nicht gesetztem Schalter.

\newif\ifkorrekturansicht
\korrekturansichtfalse

\input{../tex-inputs/latex-vorspann}


\section[Arthur Schnitzler an Albert Ehrenstein, 5. 6. 1906]{L02796 Arthur Schnitzler an Albert Ehrenstein, 5. 6. 1906}
\nopagebreak\mylabel{L02796v}
\rehead{ }\normalsize\beginnumbering\briefempfaengerindex{Ehrenstein, Albert@\textsc{Ehrenstein, Albert}!zzzSchnitzler, Arthur@\emph{von Arthur Schnitzler}!1906-06-051@{5. 6. 1906}|(be}
\toendnotes[C]{\smallbreak\pagebreak[2]}
\correspDesc{Versand  durch Arthur Schnitzler am 5. 6. 1906 in Wien
\newline{}Übermittlung  am 6. 6. 1906 in Wien
\newline{}Erhalt  durch Albert Ehrenstein am 6. 6. 1906 in Wien}\toendnotes[C]{\smallbreak}
\Standort{New York, Leo Baeck Institute, Gertrude Lobbenberg Collection (AR 11130 / MF 1582), Autograph letters, 1846–1937.}
\physDesc{Kartenbrief, 212 Zeichen
\newline{}Handschrift: schwarze Tinte, deutsche Kurrent
\newline{}Versand: 1) Stempel: »\nobreak{}\oindex{XVIII., Währing@\textbf{XVIII., Währing}, \emph{Verwaltungsgebiet}|pwk}18/1 Wien 110, 6. VI. 06, XI\nobreak{}«.   2) Stempel: »\nobreak{}\oindex{XVIII., Währing@\textbf{XVIII., Währing}, \emph{Verwaltungsgebiet}|pwk}Wien 18/1, \textcolor{gray}{6}. 6. 06, 3. N, Bestellt\nobreak{}«. 
\newline{}Ordnung: mit Bleistift von unbekannter Hand nummeriert: »2« }\toendnotes[C]{\smallbreak}\pstart{}{\pb}Herrn \textsc{stud.
                     phil}\pend{}\pstart{}\textsc{Albert Ehrenstein}\pend{}\pstart{}Wien XVII.\oindex{XVII., Hernals@\textbf{XVII., Hernals}, \emph{Verwaltungsgebiet}|pw}\pend{}\pstart{}\textsc{Ottakringerstr. 114\oindex{Wien@\textbf{Wien}!XVI., Ottakring@\textbf{XVI., Ottakring}!Ottakringer Straße@\textbf{Ottakringer Straße}, \emph{Straße}|pw}\oindex{Wien@\textbf{Wien}!XVII., Hernals@\textbf{XVII., Hernals}!Ottakringer Straße@\textbf{Ottakringer Straße}, \emph{Straße}|pw}}.\pend{}{\bigskip}\vspace{1em}
\pstart{}{\pb}ſehr geehrter Herr Ehrenſtein,\pend\vspace{0.5em}
\pstart
           wollen Sie{ }ſich am \label{K_L02796-1v}\edtext{Freitag}{\lemma{\textnormal{\emph{Freitag}}}\Cendnote{\textnormal{Siehe A. S.: \emph{Tagebuch}, 8. 6. 1906.
               }}}\label{K_L02796-1}{ }zwiſchen ½ 4 und 4 Ihre Gedichte von mir abholen?\pend
           
\pstart
           mit beſten Grüßen Ihres{\\[\baselineskip]}\spacefill\mbox{ArthSchnitzler}\pend
           \leftskip=0em{}
\pstart
           5. 6. 906.\pend
           \selectlanguage{ngerman}\endnumbering\briefempfaengerindex{Ehrenstein, Albert@\textsc{Ehrenstein, Albert}!zzzSchnitzler, Arthur@\emph{von Arthur Schnitzler}!1906-06-051@{5. 6. 1906}|)be}\mylabel{L02796h}  \newcommand{\dateiname}{L02796}\newcommand{\titel}{Arthur Schnitzler an Albert Ehrenstein, 5. 6. 1906}\newcommand{\editorInnen}{Martin Anton Müller und Gerd-Hermann Susen}%% latex-leseansicht-abspann.tex
%% Abspann für die Leseansicht.
%% Der Schalter \ifkorrekturansicht ist bereits durch den Vorspann gesetzt.

%% latex-abspann.tex
%% Gemeinsamer Abspann für Korrekturansicht und Leseansicht.
%% Setzt den Schalter \ifkorrekturansicht voraus (gesetzt in den
%% einbindenden Dateien latex-korrekturansicht-abspann.tex bzw.
%% latex-leseansicht-abspann.tex).
%% ---------------------------------------------------------------

\normalsize

% Das esempio-Environment wird nur in der Leseansicht benötigt
\ifkorrekturansicht\else
\newenvironment{esempio}[3]%
{
    \vspace{1.5ex}
    \rlap{\underline{#1}}
    \par
    \setlength{\parindent}{0cm}
    \nopagebreak
    \leftskip=#2cm
    \rightskip=#3cm
}
{
    \par
}
\fi

\doendnotes{C}
\bigskip
\vfill

\clearpage

\footnotesize

\ifkorrekturansicht
  \lohead{\textsc{register}}
\fi

% theindex-Environment neu definieren ohne reledmac
\makeatletter
\renewenvironment{theindex}{%
  \ifkorrekturansicht
    \section*{\indexname}%
  \else
    \subsubsection*{Index der erwähnten Entitäten}%
  \fi
  \setlength{\parindent}{0pt}%
  \setlength{\parskip}{0pt plus 0.3pt}%
  \let\item\@idxitem
}{%
  \ifkorrekturansicht\clearpage\fi
}
\makeatother

\IfFileExists{\jobname-pw.ind}{\input{\jobname-pw.ind}}{}

% Quellenangabe nur in der Leseansicht
\ifkorrekturansicht\else
% Fallback-Definitionen, falls die .tex-Datei \titel etc. nicht gesetzt hat
\providecommand{\titel}{}
\providecommand{\editorInnen}{}
\providecommand{\dateiname}{\jobname}

\vspace{3cm}

\vfill

\footnotesize
\textsc{Quelle}: \titel. Herausgegeben von {\editorInnen}. In: \emph{Arthur Schnitzler: Briefwechsel mit Autorinnen und Autoren}.
 Digitale Edition, https://schnitzler-briefe.acdh.oeaw.ac.at/{\dateiname}.html (Stand \today)
\fi

\end{document}


