%% latex-leseansicht-vorspann.tex
%% Vorspann für die Leseansicht.
%% Lädt die gemeinsame Datei latex-vorspann.tex mit nicht gesetztem Schalter.

\newif\ifkorrekturansicht
\korrekturansichtfalse

\input{../tex-inputs/latex-vorspann}


\section[Arthur Schnitzler an Gustav Schwarzkopf, 5. 7. 1893]{L04217 Arthur Schnitzler an Gustav Schwarzkopf, 5. 7. 1893}
\nopagebreak\mylabel{L04217v}
\rehead{ }\normalsize\beginnumbering\briefempfaengerindex{Schwarzkopf, Gustav@\textsc{Schwarzkopf, Gustav}!zzzSchnitzler, Arthur@\emph{von Arthur Schnitzler}!1893-07-053@{5. 7. 1893}|(be}
\toendnotes[C]{\smallbreak\pagebreak[2]}
\correspDesc{Versand  durch Arthur Schnitzler am 5. 7. 1893 in Bad Ischl
\newline{}Erhalt  durch Gustav Schwarzkopf im Zeitraum [6. 7. 1893 – 10. 7. 1893?] in Mödling}\toendnotes[C]{\smallbreak}
\Standort{CUL, Schnitzler, B 96.}
\physDesc{Brief, 1 Blatt, 4 Seiten, 938 Zeichen (Briefpapier mit Trauerrand)
\newline{}Handschrift: schwarze Tinte, deutsche Kurrent}\toendnotes[C]{\smallbreak}
\pstart{}{\pb}Mein verehrteſter Herr
                  Schwarzkopf,\pend\vspace{0.5em}
\pstart
           wie geht es Ihnen, Ihren Brüdern\pwindex{Schwarzkopf, Emil 17.\,9.\,1851 Wien – 28.\,1.\,1928 ebd.@\textsc{Schwarzkopf, Emil} (17.\,9.\,1851 Wien – 28.\,1.\,1928 ebd.), \emph{Übersetzer, Komponist, Musiklehrer}|pwv}\pwindex{Schwarzkopf, Max 12.\,6.\,1857 Wien – 14.\,4.\,1928 ebd.@\textsc{Schwarzkopf, Max} (12.\,6.\,1857 Wien – 14.\,4.\,1928 ebd.), \emph{Rechtsanwalt}|pwv}\pwindex{Schwarzkopf, Rudolf 25.\,5.\,1861 Wien – 13.\,10.\,1893 Meran@\textsc{Schwarzkopf, Rudolf} (25.\,5.\,1861 Wien – 13.\,10.\,1893 Meran), \emph{Schriftsteller}|pwv}, insbesondere Rudolf\pwindex{Schwarzkopf, Rudolf 25.\,5.\,1861 Wien – 13.\,10.\,1893 Meran@\textsc{Schwarzkopf, Rudolf} (25.\,5.\,1861 Wien – 13.\,10.\,1893 Meran), \emph{Schriftsteller}|pw}? Sie wären ſehr gütig, we{\geminationn} Sie mich in
               ein paar Zeilen benachrichtigten!–\pend
           
\pstart
           Von mir iſt wenig andres mitzutheilen, als daſs ich \textsc{Bicycle}
               fahre, mir das {\pb}Hinken ſchon
               abgewöhnt habe, einige ſchlecht Wien\oindex{Wien@\textbf{Wien}, \emph{Verwaltungsgebiet}|pw}er Verse\pwindex{Schnitzler, Arthur 15. 5. 1862 Wien – 21. 10. 1931 ebd.@\textsc{Schnitzler, Arthur} (15. 5. 1862 Wien – 21. 10. 1931 ebd.), \emph{Schriftsteller, Mediziner}!Artifex@\strich\emph{Artifex}|pwv} zu verbeſſern und eine kleine
               Novelle\pwindex{Schnitzler, Arthur 15. 5. 1862 Wien – 21. 10. 1931 ebd.@\textsc{Schnitzler, Arthur} (15. 5. 1862 Wien – 21. 10. 1931 ebd.), \emph{Schriftsteller, Mediziner}!kleine Komödie@\strich\emph{Die kleine Komödie}|pwv} zu beenden – trachte, noch
                  i{\geminationm}er huſte, und merkwürdigr Weiſe auch nicht beim Leopold\oindex{Hotel und Pension Rudolfshöhe (Leopold Petter)@\textbf{Hotel und Pension Rudolfshöhe (Leopold Petter)}, \emph{Hotel}|pw}, jetzt, 3 Uhr Nachmittag, bei einer
               Cigarre über meine Verhältniſſe, {\pb}und
               die berühmte »Natur« vor meinem Fenſtern (Aufgeſchnitten, ich habe nur ein Fenſter),
               daſs ich auch da nicht zu der Empfindg reinen Glückes ko{\geminationm}e. –\pend
           
\pstart
           – Ich ſage Ihnen alles dies, mein lieber und verehrter Freund, um eine Antwort zu
               erpreſſen, ſelbſt we{\geminationn}{\pb}es nur 2 Zeilen wären. Bitte grüßen
               Sie Ihre Brüder\pwindex{Schwarzkopf, Emil 17.\,9.\,1851 Wien – 28.\,1.\,1928 ebd.@\textsc{Schwarzkopf, Emil} (17.\,9.\,1851 Wien – 28.\,1.\,1928 ebd.), \emph{Übersetzer, Komponist, Musiklehrer}|pwv}\pwindex{Schwarzkopf, Max 12.\,6.\,1857 Wien – 14.\,4.\,1928 ebd.@\textsc{Schwarzkopf, Max} (12.\,6.\,1857 Wien – 14.\,4.\,1928 ebd.), \emph{Rechtsanwalt}|pwv}\pwindex{Schwarzkopf, Rudolf 25.\,5.\,1861 Wien – 13.\,10.\,1893 Meran@\textsc{Schwarzkopf, Rudolf} (25.\,5.\,1861 Wien – 13.\,10.\,1893 Meran), \emph{Schriftsteller}|pwv} aufs herzlichſte, Herrn \textsc{Wachtel\pwindex{Wachtel, Felix Viktor *~15.\,6.\,1859 Lamnitz@\textsc{Wachtel, Felix Viktor} (*~15.\,6.\,1859 Lamnitz), \emph{Regisseur, Schauspieler}|pw}} kühl, die reizenden Tini\pwindex{Kepert, Christine 17.\,11.\,1875 – 3.\,2.\,1971 Wien@\textsc{Kepert, Christine} (17.\,11.\,1875 – 3.\,2.\,1971 Wien), \emph{Gastwirtin}|pw} mit frivoler
               Wärme. –\pend
           
\pstart
           Ganz de\textcolor{gray}{r} Ihre{\\[\baselineskip]}\spacefill\mbox{Arthur Schnitzler}\pend
           \leftskip=0em{}
\pstart
           5/7. 93\pend
           
\pstart
           \textsc{I\textcolor{gray}{sc}hl, Pens Leopold\oindex{Hotel und Pension Rudolfshöhe (Leopold Petter)@\textbf{Hotel und Pension Rudolfshöhe (Leopold Petter)}, \emph{Hotel}|pw}}\pend
           
\pstart
           \noindent{}{\pb}\label{T_L04102-1v}\edtext{Von \textsc{Rich. B. Hofm.\pwindex{Beer-Hofmann, Richard 11.\,7.\,1866 Wien – 26.\,9.\,1945 New York City@\textsc{Beer-Hofmann, Richard} (11.\,7.\,1866 Wien – 26.\,9.\,1945 New York City), \emph{Schriftsteller}|pw}} beſte Grüße. –}{\lemma{\textnormal{\emph{Von … Grüße. –}}}\Cendnote{\textnormal{Am Kopf der
                     ersten Seite, verkehrt zum Text.}}}\label{T_L04102-1}\pend
           \selectlanguage{ngerman}\endnumbering\briefempfaengerindex{Schwarzkopf, Gustav@\textsc{Schwarzkopf, Gustav}!zzzSchnitzler, Arthur@\emph{von Arthur Schnitzler}!1893-07-053@{5. 7. 1893}|)be}\mylabel{L04217h}
\begin{anhang}
\end{anhang}\newcommand{\dateiname}{L04217}\newcommand{\titel}{Arthur Schnitzler an Gustav Schwarzkopf, 5. 7. 1893}\newcommand{\editorInnen}{Herausgegeben von Jahnke, SelmaMüller, Martin Anton}%% latex-leseansicht-abspann.tex
%% Abspann für die Leseansicht.
%% Der Schalter \ifkorrekturansicht ist bereits durch den Vorspann gesetzt.

%% latex-abspann.tex
%% Gemeinsamer Abspann für Korrekturansicht und Leseansicht.
%% Setzt den Schalter \ifkorrekturansicht voraus (gesetzt in den
%% einbindenden Dateien latex-korrekturansicht-abspann.tex bzw.
%% latex-leseansicht-abspann.tex).
%% ---------------------------------------------------------------

\normalsize

% Das esempio-Environment wird nur in der Leseansicht benötigt
\ifkorrekturansicht\else
\newenvironment{esempio}[3]%
{
    \vspace{1.5ex}
    \rlap{\underline{#1}}
    \par
    \setlength{\parindent}{0cm}
    \nopagebreak
    \leftskip=#2cm
    \rightskip=#3cm
}
{
    \par
}
\fi

\doendnotes{C}
\bigskip
\vfill

\clearpage

\footnotesize

\ifkorrekturansicht
  \lohead{\textsc{register}}
\fi

% theindex-Environment neu definieren ohne reledmac
\makeatletter
\renewenvironment{theindex}{%
  \ifkorrekturansicht
    \section*{\indexname}%
  \else
    \subsubsection*{Index der erwähnten Entitäten}%
  \fi
  \setlength{\parindent}{0pt}%
  \setlength{\parskip}{0pt plus 0.3pt}%
  \let\item\@idxitem
}{%
  \ifkorrekturansicht\clearpage\fi
}
\makeatother

\IfFileExists{\jobname-pw.ind}{\input{\jobname-pw.ind}}{}

% Quellenangabe nur in der Leseansicht
\ifkorrekturansicht\else
% Fallback-Definitionen, falls die .tex-Datei \titel etc. nicht gesetzt hat
\providecommand{\titel}{}
\providecommand{\editorInnen}{}
\providecommand{\dateiname}{\jobname}

\vspace{3cm}

\vfill

\footnotesize
\textsc{Quelle}: \titel. Herausgegeben von {\editorInnen}. In: \emph{Arthur Schnitzler: Briefwechsel mit Autorinnen und Autoren}.
 Digitale Edition, https://schnitzler-briefe.acdh.oeaw.ac.at/{\dateiname}.html (Stand \today)
\fi

\end{document}


