%% latex-leseansicht-vorspann.tex
%% Vorspann für die Leseansicht.
%% Lädt die gemeinsame Datei latex-vorspann.tex mit nicht gesetztem Schalter.

\newif\ifkorrekturansicht
\korrekturansichtfalse

\input{../tex-inputs/latex-vorspann}


\section[Richard Beer-Hofmann an Arthur Schnitzler, 24. 9. 1895]{L00492 Richard Beer-Hofmann an Arthur Schnitzler, 24. 9. 1895}
\nopagebreak\mylabel{L00492v}
\rehead{ }\normalsize\beginnumbering\briefempfaengerindex{Schnitzler, Arthur@\textsc{Schnitzler, Arthur}!zzzBeer-Hofmann, Richard@\emph{von Richard Beer-Hofmann}!1895-09-241@{24. 9. 1895}|(be}
\toendnotes[C]{\smallbreak\pagebreak[2]}
\correspDesc{Versand  durch Richard Beer-Hofmann am 24. 9. 1895 in Gardone Riviera
\newline{}Erhalt  durch Arthur Schnitzler am 26. 9. 1895 in Wien}\toendnotes[C]{\smallbreak}
\Standort{CUL, Schnitzler, B 8.}
\physDesc{Brief, 1 Blatt, 4 Seiten, 1356 Zeichen
\newline{}Handschrift: Bleistift, lateinische Kurrent
\newline{}Schnitzler: mit Bleistift nummeriert: »64« }
\buchAbdrucke{\weitereDrucke{1) Arthur Schnitzler, Richard Beer-Hofmann: \emph{Briefwechsel 1891–1931}. Herausgegeben von Konstanze Fliedl. Wien, Zürich: \emph{Europaverlag} 1992, S. 84–85.} \weitereDrucke{2) Hermann Bahr, Arthur Schnitzler: \emph{Briefwechsel, Aufzeichnungen, Dokumente (1891–1931)}. Herausgegeben von Kurt Ifkovits und Martin Anton Müller. Göttingen: \emph{Wallstein} 2018.} }\toendnotes[C]{\smallbreak}
\pstart
           \raggedleft{}{\pb}Gardone\oindex{Gardone Riviera@\textbf{Gardone Riviera}, \emph{Verwaltungsgebiet}|pw}, Dienstag 24/IX 95\pend
           \vspace{0.5em}
\pstart
           Lieber Arthur! Soeben erhalte ich von Riva\oindex{Riva del Garda@\textbf{Riva del Garda}, \emph{Hauptstadt}|pw} nachgesandt Ihren Brief vom 21/IX. \uline{Fels\pwindex{Fels, Friedrich Michael *~1864 Bad Dürkheim@\textsc{Fels, Friedrich Michael} (*~1864 Bad Dürkheim), \emph{Journalist}|pw} – \label{K_L00492-1v}\edtext{Hekuba}{\lemma{\textnormal{\emph{Hekuba}}}\Cendnote{\textnormal{sprichwörtlicher Ausruf, der »Ist mir gleichgültig« bedeutet}}}\label{K_L00492-1}} senden Sie bitte für mich ebensoviel als Sie bereits gesandt haben. Wie zuwider
               müssen wir ihm sein! Später oder früher werden wir es auch merken.\pend
           
\pstart
           Hier ist{[}’s{]} wunderschön; der See 20 Grad Wärme – und etwas zu
               heiß, wodurch mein Arbeiten wieder stockt.\pend
           
\pstart
           {\pb}Das mit dem »Blaßwerden guter
               Stücke« hat auch mich immer sehr traurig gemacht.\pend
           \stanza{}»Alles entführet die Zeit; die
                     flüchtigen Jahre verändern\pwindex{Epigramme@\emph{Epigramme}|pwv}\newverse{}Ganz allmählich Gestalt, Namen
                     und Glück und Natur.\pwindex{Epigramme@\emph{Epigramme}|pwv}{[}«{]}\stanzaend{}
\pstart
           Das ist aber nicht von mir sondern von Plato\pwindex{Platon 427? v.\,u.\,Z. Athen – 347/348 v.\,u.\,Z. ebd.@\textsc{Platon} (427? v.\,u.\,Z. Athen – 347/348 v.\,u.\,Z. ebd.), \emph{Philosoph}|pw}!
               Wirklich!\pend
           
\pstart
           Schreiben Sie mir doch recht viel oder zumindest oft, Sie sehen wie pünktlich ich
               antworte. Sagen Sie, sind in Wien\oindex{Wien@\textbf{Wien}, \emph{Verwaltungsgebiet}|pw} auch alle Frauen
               jetzt läufig (l-ä-u-f-i-g)? {\pb}Hier
                  \strikeout{au} oder viel mehr auf der Reise schien es so.
               Manchmal angenehm, manchmal komisch und manchmal widerlich.\pend
           
\pstart
           Daß Burkhardt\pwindex{Burckhard, Max Eugen 14.\,7.\,1854 Korneuburg – 16.\,3.\,1912 Wien@\textsc{Burckhard, Max Eugen} (14.\,7.\,1854 Korneuburg – 16.\,3.\,1912 Wien), \emph{Schriftsteller, Rechtswissenschaftler, Theaterleiter}|pw} die »Enthüllung von Frl. Dandler\pwindex{Dandler, Anna 14.\,3.\,1862 Stuttgart – 17.\,9.\,1930 Wiesbaden@\textsc{Dandler, Anna} (14.\,3.\,1862 Stuttgart – 17.\,9.\,1930 Wiesbaden), \emph{Schauspielerin}|pw}« (München\oindex{München@\textbf{München}|pw}?) lieber wäre als \label{K_L00492-2v}\edtext{die
                  Laubes\pwindex{Laube, Heinrich 18.\,9.\,1806 Sprottau – 1.\,8.\,1884 Wien@\textsc{Laube, Heinrich} (18.\,9.\,1806 Sprottau – 1.\,8.\,1884 Wien), \emph{Schriftsteller, Theaterleiter}|pw}}{\lemma{\textnormal{\emph{die
                  Laubes}}}\Cendnote{\textnormal{Am 18. 9. 1895 wurde im
                  Geburtsort Heinrich Laubes\pwindex{Laube, Heinrich 18.\,9.\,1806 Sprottau – 1.\,8.\,1884 Wien@\textsc{Laube, Heinrich} (18.\,9.\,1806 Sprottau – 1.\,8.\,1884 Wien), \emph{Schriftsteller, Theaterleiter}|pwk}, in Sprottau\oindex{Sprottau@\textbf{Sprottau}|pwk}, ein Denkmal für diesen
                  eingeweiht.}}}\label{K_L00492-2} begreife ich. Die Dandler\pwindex{Dandler, Anna 14.\,3.\,1862 Stuttgart – 17.\,9.\,1930 Wiesbaden@\textsc{Dandler, Anna} (14.\,3.\,1862 Stuttgart – 17.\,9.\,1930 Wiesbaden), \emph{Schauspielerin}|pw}
               ist übrigens {\pb}auch Bahrs\pwindex{Bahr, Hermann 19.\,7.\,1863 Linz – 15.\,1.\,1934 München@\textsc{Bahr, Hermann} (19.\,7.\,1863 Linz – 15.\,1.\,1934 München), \emph{Schriftsteller, Kritiker}|pw} Geschmack, voraussichtlich auch der
               Doctor Luegers\pwindex{Lueger, Karl 24.\,10.\,1844 Wien – 10.\,3.\,1910 ebd.@\textsc{Lueger, Karl} (24.\,10.\,1844 Wien – 10.\,3.\,1910 ebd.), \emph{Politiker}|pw}. Das{[}s{]} die
                  Kallina\pwindex{Kallina, Anna 31.\,3.\,1874 Wien – 4.\,1.\,1948 ebd.@\textsc{Kallina, Anna} (31.\,3.\,1874 Wien – 4.\,1.\,1948 ebd.), \emph{Schauspielerin}|pw} überraschen wird, freut mich,
               vielleicht überrascht sie auch mich; jedenfalls grüßen Sie sie von mir – sie hat
               wirklich schöne Augen. Übrigens ist sie Ihnen so sympathisch weil Bahr\pwindex{Bahr, Hermann 19.\,7.\,1863 Linz – 15.\,1.\,1934 München@\textsc{Bahr, Hermann} (19.\,7.\,1863 Linz – 15.\,1.\,1934 München), \emph{Schriftsteller, Kritiker}|pw} sie gar nicht mag – was? Wann ist Liebelei\pwindex{Schnitzler, Arthur 15.\,5.\,1862 Wien – 21.\,10.\,1931 ebd.@\textsc{Schnitzler, Arthur} (15.\,5.\,1862 Wien – 21.\,10.\,1931 ebd.), \emph{Schriftsteller, Mediziner}!Liebelei. Schauspiel in drei Akten@\strich\emph{Liebelei. Schauspiel in drei Akten}|pw}? Das muß ich nämlich genau wissen, wegen meiner
               Ankunft!\pend
           \pstart Herzlichst Ihr \spacefill\mbox{Richard}\pend{}\selectlanguage{ngerman}\endnumbering\briefempfaengerindex{Schnitzler, Arthur@\textsc{Schnitzler, Arthur}!zzzBeer-Hofmann, Richard@\emph{von Richard Beer-Hofmann}!1895-09-241@{24. 9. 1895}|)be}\mylabel{L00492h}  \newcommand{\dateiname}{L00492}\newcommand{\titel}{Richard Beer-Hofmann an Arthur Schnitzler, 24. 9. 1895}\newcommand{\editorInnen}{Herausgegeben von Martin Anton Müller}%% latex-leseansicht-abspann.tex
%% Abspann für die Leseansicht.
%% Der Schalter \ifkorrekturansicht ist bereits durch den Vorspann gesetzt.

%% latex-abspann.tex
%% Gemeinsamer Abspann für Korrekturansicht und Leseansicht.
%% Setzt den Schalter \ifkorrekturansicht voraus (gesetzt in den
%% einbindenden Dateien latex-korrekturansicht-abspann.tex bzw.
%% latex-leseansicht-abspann.tex).
%% ---------------------------------------------------------------

\normalsize

% Das esempio-Environment wird nur in der Leseansicht benötigt
\ifkorrekturansicht\else
\newenvironment{esempio}[3]%
{
    \vspace{1.5ex}
    \rlap{\underline{#1}}
    \par
    \setlength{\parindent}{0cm}
    \nopagebreak
    \leftskip=#2cm
    \rightskip=#3cm
}
{
    \par
}
\fi

\doendnotes{C}
\bigskip
\vfill

\clearpage

\footnotesize

\ifkorrekturansicht
  \lohead{\textsc{register}}
\fi

% theindex-Environment neu definieren ohne reledmac
\makeatletter
\renewenvironment{theindex}{%
  \ifkorrekturansicht
    \section*{\indexname}%
  \else
    \subsubsection*{Index der erwähnten Entitäten}%
  \fi
  \setlength{\parindent}{0pt}%
  \setlength{\parskip}{0pt plus 0.3pt}%
  \let\item\@idxitem
}{%
  \ifkorrekturansicht\clearpage\fi
}
\makeatother

\IfFileExists{\jobname-pw.ind}{\input{\jobname-pw.ind}}{}

% Quellenangabe nur in der Leseansicht
\ifkorrekturansicht\else
% Fallback-Definitionen, falls die .tex-Datei \titel etc. nicht gesetzt hat
\providecommand{\titel}{}
\providecommand{\editorInnen}{}
\providecommand{\dateiname}{\jobname}

\vspace{3cm}

\vfill

\footnotesize
\textsc{Quelle}: \titel. Herausgegeben von {\editorInnen}. In: \emph{Arthur Schnitzler: Briefwechsel mit Autorinnen und Autoren}.
 Digitale Edition, https://schnitzler-briefe.acdh.oeaw.ac.at/{\dateiname}.html (Stand \today)
\fi

\end{document}


