%% latex-korrekturansicht-vorspann.tex
%% Vorspann für die Korrekturansicht.
%% Lädt die gemeinsame Datei latex-vorspann.tex mit gesetztem Schalter.

\newif\ifkorrekturansicht
\korrekturansichttrue

\input{../tex-inputs/latex-vorspann}


\section[Richard Beer-Hofmann an Arthur Schnitzler, 24. 9. 1895]{L00492 Richard Beer-Hofmann an Arthur Schnitzler, 24. 9. 1895}
\nopagebreak\mylabel{L00492v}
\rehead{ }\normalsize\beginnumbering\briefempfaengerindex{Schnitzler, Arthur@\textsc{Schnitzler, Arthur}!zzzBeer-Hofmann, Richard@\emph{von Richard Beer-Hofmann}!1895-09-241@{24. 9. 1895}|(be}
\toendnotes[C]{\smallbreak\pagebreak[2]}\Standort{CUL, Schnitzler, B 8.}
\physDesc{Brief, 1 Blatt, 4 Seiten, 1356 Zeichen
\newline{}Handschrift: Bleistift, lateinische Kurrent
\newline{}Schnitzler: mit Bleistift nummeriert: »64« }
\buchAbdrucke{\weitereDrucke{1) Arthur Schnitzler, Richard Beer-Hofmann: \emph{Briefwechsel 1891–1931}. Wien, Zürich: \emph{Europaverlag} 1992, S. 84–85.} \weitereDrucke{2) Hermann Bahr, Arthur Schnitzler: \emph{Briefwechsel, Aufzeichnungen, Dokumente (1891–1931)}. Göttingen: \emph{Wallstein} 2018.} }\toendnotes[C]{\smallbreak}
\pstart
           \raggedleft{}{\pb}Gardone\oindex{Gardone Riviera@\textbf{Gardone Riviera}, \emph{A.ADM3}|pw}, Dienstag 24/IX 95\pend
           \vspace{0.5em}
\pstart
           Lieber Arthur! Soeben erhalte ich von Riva\oindex{Riva del Garda@\textbf{Riva del Garda}, \emph{P.PPLA3}|pw} nachgesandt Ihren Brief vom 21/IX. \uline{Fels\pwindex{Fels, Friedrich Michael *~1864@\textsc{Fels, Friedrich Michael} (*~1864), \emph{Journalist/Journalistin}|pw} – \label{K_L00492-1v}\edtext{Hekuba}{\lemma{\textnormal{\emph{Hekuba}}}\Cendnote{\textnormal{sprichwörtlicher Ausruf, der »Ist mir gleichgültig« bedeutet}}}\label{K_L00492-1}} senden Sie bitte für mich ebensoviel als Sie bereits gesandt haben. Wie zuwider
               müssen wir ihm sein! Später oder früher werden wir es auch merken.\pend
           
\pstart
           Hier ist{[}’s{]} wunderschön; der See 20 Grad Wärme – und etwas zu
               heiß, wodurch mein Arbeiten wieder stockt.\pend
           
\pstart
           {\pb}Das mit dem »Blaßwerden guter
               Stücke« hat auch mich immer sehr traurig gemacht.\pend
           \stanza{}»Alles entführet die Zeit; die
                     flüchtigen Jahre verändern\pwindex{Epigramme@\emph{Epigramme}|pwv}Ganz allmählich Gestalt, Namen
                     und Glück und Natur.\pwindex{Epigramme@\emph{Epigramme}|pwv}{[}«{]}\stanzaend{}
\pstart
           Das ist aber nicht von mir sondern von Plato\pwindex{Platon 427? v. u. Z. – 347/348 v. u. Z.@\textsc{Platon} (427? v. u. Z. – 347/348 v. u. Z.), \emph{Philosoph/Philosophin}|pw}!
               Wirklich! \pend
           
\pstart
           Schreiben Sie mir doch recht viel oder zumindest oft, Sie sehen wie pünktlich ich
               antworte. Sagen Sie, sind in Wien\oindex{Wien@\textbf{Wien}, \emph{A.ADM2}|pw} auch alle Frauen
               jetzt läufig (l-ä-u-f-i-g)? {\pb}Hier
                  \strikeout{au} oder viel mehr auf der Reise schien es so.
               Manchmal angenehm, manchmal komisch und manchmal widerlich.\pend
           
\pstart
           Daß Burkhardt\pwindex{Burckhard, Max Eugen 14.07.1854 – 16.03.1912@\textsc{Burckhard, Max Eugen} (14.07.1854 – 16.03.1912), \emph{Schriftsteller/Schriftstellerin, Rechtswissenschaftler/Rechtswissenschaftlerin, Theaterleiter/Theaterleiterin}|pw} die »Enthüllung von Frl. Dandler\pwindex{Dandler, Anna 1862-03-14 – 1930-09-17@\textsc{Dandler, Anna} (1862-03-14 – 1930-09-17), \emph{Schauspieler/Schauspielerin}|pw}« (München\oindex{Muenchen@\textbf{München}, \emph{P.PPLA}|pw}?) lieber wäre als \label{K_L00492-2v}\edtext{die
                  Laubes\pwindex{Laube, Heinrich 1806-09-18 – 1884-08-01@\textsc{Laube, Heinrich} (1806-09-18 – 1884-08-01), \emph{Schriftsteller/Schriftstellerin, Theaterleiter/Theaterleiterin}|pw}}{\lemma{\textnormal{\emph{die
                  Laubes}}}\Cendnote{\textnormal{Am 18. 9. 1895 wurde im
                  Geburtsort Heinrich Laubes\pwindex{Laube, Heinrich 1806-09-18 – 1884-08-01@\textsc{Laube, Heinrich} (1806-09-18 – 1884-08-01), \emph{Schriftsteller/Schriftstellerin, Theaterleiter/Theaterleiterin}|pwk}, in Sprottau\oindex{Sprottau@\textbf{Sprottau}, \emph{P.PPL}|pwk}, ein Denkmal für diesen
                  eingeweiht.}}}\label{K_L00492-2} begreife ich. Die Dandler\pwindex{Dandler, Anna 1862-03-14 – 1930-09-17@\textsc{Dandler, Anna} (1862-03-14 – 1930-09-17), \emph{Schauspieler/Schauspielerin}|pw}
               ist übrigens {\pb}auch Bahrs\pwindex{Bahr, Hermann 19.07.1863 – 15.01.1934@\textsc{Bahr, Hermann} (19.07.1863 – 15.01.1934), \emph{Schriftsteller/Schriftstellerin, Kritiker/Kritikerin}|pw} Geschmack, voraussichtlich auch der
               Doctor Luegers\pwindex{Lueger, Karl 24.10.1844 – 10.03.1910@\textsc{Lueger, Karl} (24.10.1844 – 10.03.1910), \emph{Politiker/Politikerin}|pw}. Das{[}s{]} die
                  Kallina\pwindex{Kallina, Anna 31.03.1874 – 04.01.1948@\textsc{Kallina, Anna} (31.03.1874 – 04.01.1948), \emph{Schauspieler/Schauspielerin}|pw} überraschen wird, freut mich,
               vielleicht überrascht sie auch mich; jedenfalls grüßen Sie sie von mir – sie hat
               wirklich schöne Augen. Übrigens ist sie Ihnen so sympathisch weil Bahr\pwindex{Bahr, Hermann 19.07.1863 – 15.01.1934@\textsc{Bahr, Hermann} (19.07.1863 – 15.01.1934), \emph{Schriftsteller/Schriftstellerin, Kritiker/Kritikerin}|pw} sie gar nicht mag – was? Wann ist Liebelei\pwindex{Liebelei. Schauspiel in drei Akten@\emph{Liebelei. Schauspiel in drei Akten}|pw}? Das muß ich nämlich genau wissen, wegen meiner
               Ankunft!\pend
           \pstart Herzlichst Ihr \spacefill\mbox{Richard}\pend{}\selectlanguage{ngerman}\endnumbering\briefempfaengerindex{Schnitzler, Arthur@\textsc{Schnitzler, Arthur}!zzzBeer-Hofmann, Richard@\emph{von Richard Beer-Hofmann}!1895-09-241@{24. 9. 1895}|)be}\mylabel{L00492h}  \normalsize

\doendnotes{C}
\bigskip
\vfill

\clearpage

\footnotesize

\lohead{\textsc{register}}

% Definiere theindex-Environment komplett neu ohne reledmac
\makeatletter
\renewenvironment{theindex}{%
  \section*{\indexname}%
  \setlength{\parindent}{0pt}%
  \setlength{\parskip}{0pt plus 0.3pt}%
  \let\item\@idxitem
}{%
  \clearpage
}
\makeatother

\IfFileExists{\jobname-pw.ind}{\input{\jobname-pw.ind}}{}

\end{document}

      