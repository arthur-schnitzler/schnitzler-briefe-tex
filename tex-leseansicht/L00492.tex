%% latex-leseansicht-vorspann.tex
%% Vorspann für die Leseansicht.
%% Lädt die gemeinsame Datei latex-vorspann.tex mit nicht gesetztem Schalter.

\newif\ifkorrekturansicht
\korrekturansichtfalse

\input{../tex-inputs/latex-vorspann}


         
         \newcommand{\erwaehntePersonen}{Personen: Hermann Bahr, Max Eugen Burckhard, Anna Dandler, Friedrich Michael Fels, Anna Kallina, Heinrich Laube, Karl Lueger,  Platon}
         \newcommand{\erwaehnteOrte}{Orte: Gardone Riviera, München, Riva del Garda, Sprottau, Wien}
         \newcommand{\erwaehnteWerke}{Werke: Epigramme, Liebelei. Schauspiel in drei Akten}
               \section[Richard Beer-Hofmann an Arthur Schnitzler, 24. 9. 1895]{ Richard Beer-Hofmann an Arthur Schnitzler, 24. 9. 1895}\nopagebreak\mylabel{v}\rehead{ }\begin{ledgroupsized}[t]{13cm}\normalsize\beginnumbering \toendnotes[C]{\smallbreak\pagebreak[2]} \Standort{CUL, Schnitzler, B 8.}
\physDesc{Brief, 1 Blatt, 4 Seiten
\newline{}Handschrift: Bleistift, lateinische Kurrent
\newline{}Schnitzler: mit Bleistift nummeriert: »64« }\buchAbdrucke{\weitereDrucke{1) Arthur Schnitzler, Richard Beer-Hofmann: \emph{Briefwechsel 1891–1931}. Hg. Konstanze Fliedl. Wien, Zürich: \emph{Europaverlag} 1992, S. 84–85.} \weitereDrucke{2) Hermann Bahr, Arthur Schnitzler: \emph{Briefwechsel, Aufzeichnungen, Dokumente (1891–1931)}. Hg. Kurt Ifkovits und Martin Anton Müller. Göttingen: \emph{Wallstein} 2018.} }\toendnotes[C]{\smallbreak}\pstart
           \raggedleft{}{\pb}Gardone\oindex{Gardone Riviera@\textbf{Gardone Riviera}|pw}, Dienstag 24/IX 95\pend
           \pstart
           Lieber Arthur! Soeben erhalte ich von Riva\oindex{Riva del Garda@\textbf{Riva del Garda}|pw} nachgesandt Ihren Brief vom 21/IX. \uline{Fels\pwindex{Fels, Friedrich Michael *~1864@\textsc{Fels, Friedrich Michael} (*~1864), \emph{Journalist}|pw} – \label{K_L00492_1v}\edtext{Hekuba}{\lemma{\textnormal{\emph{Hekuba}}}\Cendnote{\textnormal{sprichwörtlicher Ausruf, der »Ist mir gleichgültig« bedeutet}}}\label{K_L00492_1h}} senden Sie bitte für mich ebensoviel als Sie bereits gesandt haben. Wie zuwider
               müssen wir ihm sein! Später oder früher werden wir es auch merken.\pend
           \pstart
           Hier ist{[}’s{]} wunderschön; der See 20 Grad Wärme – und etwas zu
               heiß, wodurch mein Arbeiten wieder stockt.\pend
           \pstart
           {\pb}Das mit dem »Blaßwerden guter
               Stücke« hat auch mich immer sehr traurig gemacht.\pend
           \stanza{}»Alles entführet die Zeit; die
                     flüchtigen Jahre verändern\pwindex{EpigrammeNone@\emph{Epigramme} {[}None{]}|pwv}\newverse{}Ganz allmählich Gestalt, Namen und
                     Glück und Natur.\pwindex{EpigrammeNone@\emph{Epigramme} {[}None{]}|pwv}{[}«{]}\stanzaend{}\pstart
           Das ist aber nicht von mir sondern von Plato\pwindex{Platon 427? v. u. Z. – 347/348 v. u. Z.@\textsc{Platon} (427? v. u. Z. – 347/348 v. u. Z.), \emph{Philosoph}|pw}!
               Wirklich! \pend
           \pstart
           Schreiben Sie mir doch recht viel oder zumindest oft, Sie sehen wie pünktlich ich
               antworte. Sagen Sie, sind in Wien\oindex{Wien@\textbf{Wien}|pw} auch alle Frauen
               jetzt läufig (l-ä-u-f-i-g)? {\pb}Hier
                  \strikeout{au} oder viel mehr auf der Reise schien es so.
               Manchmal angenehm, manchmal komisch und manchmal widerlich.\pend
           \pstart
           Daß Burkhardt\pwindex{Burckhard, Max Eugen 14.07.1854 – 16.03.1912@\textsc{Burckhard, Max Eugen} (14.07.1854 – 16.03.1912), \emph{Schriftsteller, Rechtswissenschaftler, Theaterleiter}|pw} die »Enthüllung von Frl. Dandler\pwindex{Dandler, Anna 1862-03-14 – 1930-09-17@\textsc{Dandler, Anna} (1862-03-14 – 1930-09-17), \emph{Schauspielerin}|pw}« (München\oindex{Muenchen@\textbf{München}|pw}?) lieber wäre als \label{K_L00492_2v}\edtext{die
                  Laubes\pwindex{Laube, Heinrich 1806-09-18 – 1884-08-01@\textsc{Laube, Heinrich} (1806-09-18 – 1884-08-01), \emph{Schriftsteller, Theaterleiter}|pw}}{\lemma{\textnormal{\emph{die
                  Laubes}}}\Cendnote{\textnormal{Am 18. 9. 1895 wurde im
                  Geburtsort Heinrich Laubes\pwindex{Laube, Heinrich 1806-09-18 – 1884-08-01@\textsc{Laube, Heinrich} (1806-09-18 – 1884-08-01), \emph{Schriftsteller, Theaterleiter}|pwk}, in Sprottau\oindex{Sprottau@\textbf{Sprottau}|pwk}, ein Denkmal für diesen eingeweiht.}}}\label{K_L00492_2h} begreife
               ich. Die Dandler\pwindex{Dandler, Anna 1862-03-14 – 1930-09-17@\textsc{Dandler, Anna} (1862-03-14 – 1930-09-17), \emph{Schauspielerin}|pw} ist übrigens {\pb}auch Bahr\pwindex{Bahr, Hermann 19.07.1863 – 15.01.1934@\textsc{Bahr, Hermann} (19.07.1863 – 15.01.1934), \emph{Schriftsteller, Kritiker}|pw}s Geschmack, voraussichtlich auch der Doctor Lueger\pwindex{Lueger, Karl 24.10.1844 – 10.03.1910@\textsc{Lueger, Karl} (24.10.1844 – 10.03.1910), \emph{Politiker}|pw}s. Das{[}s{]} die Kallina\pwindex{Kallina, Anna 31.03.1874 – 04.01.1948@\textsc{Kallina, Anna} (31.03.1874 – 04.01.1948), \emph{Schauspielerin}|pw} überraschen wird, freut mich, vielleicht überrascht sie
               auch mich; jedenfalls grüßen Sie sie von mir – sie hat wirklich schöne Augen.
               Übrigens ist sie Ihnen so sympathisch weil Bahr\pwindex{Bahr, Hermann 19.07.1863 – 15.01.1934@\textsc{Bahr, Hermann} (19.07.1863 – 15.01.1934), \emph{Schriftsteller, Kritiker}|pw}
               sie gar nicht mag – was? Wann ist Liebelei\pwindex{Schnitzler, Arthur 15.05.1862 – 21.10.1931@\textsc{Schnitzler, Arthur} (15.05.1862 – 21.10.1931), \emph{Schriftsteller, Mediziner}!Liebelei. Schauspiel in drei Akten1895-10-09@\strich\emph{Liebelei. Schauspiel in drei Akten} {[}1895-10-09{]}|pw}? Das muß
               ich nämlich genau wissen, wegen meiner Ankunft!\pend
           \pstart Herzlichst Ihr \spacefill\mbox{Richard}\pend{}
         
         \endnumbering\mylabel{h}\end{ledgroupsized}  \newcommand{\dateiname}{L00492}\newcommand{\titel}{Richard Beer-Hofmann an Arthur Schnitzler, 24. 9. 1895}\newcommand{\editorInnen}{ Martin Anton Müller und Gerd-Hermann Susen}%% latex-leseansicht-abspann.tex
%% Abspann für die Leseansicht.
%% Der Schalter \ifkorrekturansicht ist bereits durch den Vorspann gesetzt.

%% latex-abspann.tex
%% Gemeinsamer Abspann für Korrekturansicht und Leseansicht.
%% Setzt den Schalter \ifkorrekturansicht voraus (gesetzt in den
%% einbindenden Dateien latex-korrekturansicht-abspann.tex bzw.
%% latex-leseansicht-abspann.tex).
%% ---------------------------------------------------------------

\normalsize

% Das esempio-Environment wird nur in der Leseansicht benötigt
\ifkorrekturansicht\else
\newenvironment{esempio}[3]%
{
    \vspace{1.5ex}
    \rlap{\underline{#1}}
    \par
    \setlength{\parindent}{0cm}
    \nopagebreak
    \leftskip=#2cm
    \rightskip=#3cm
}
{
    \par
}
\fi

\doendnotes{C}
\bigskip
\vfill

\clearpage

\footnotesize

\ifkorrekturansicht
  \lohead{\textsc{register}}
\fi

% theindex-Environment neu definieren ohne reledmac
\makeatletter
\renewenvironment{theindex}{%
  \ifkorrekturansicht
    \section*{\indexname}%
  \else
    \subsubsection*{Index der erwähnten Entitäten}%
  \fi
  \setlength{\parindent}{0pt}%
  \setlength{\parskip}{0pt plus 0.3pt}%
  \let\item\@idxitem
}{%
  \ifkorrekturansicht\clearpage\fi
}
\makeatother

\IfFileExists{\jobname-pw.ind}{\input{\jobname-pw.ind}}{}

% Quellenangabe nur in der Leseansicht
\ifkorrekturansicht\else
% Fallback-Definitionen, falls die .tex-Datei \titel etc. nicht gesetzt hat
\providecommand{\titel}{}
\providecommand{\editorInnen}{}
\providecommand{\dateiname}{\jobname}

\vspace{3cm}

\vfill

\footnotesize
\textsc{Quelle}: \titel. Herausgegeben von {\editorInnen}. In: \emph{Arthur Schnitzler: Briefwechsel mit Autorinnen und Autoren}.
 Digitale Edition, https://schnitzler-briefe.acdh.oeaw.ac.at/{\dateiname}.html (Stand \today)
\fi

\end{document}


      