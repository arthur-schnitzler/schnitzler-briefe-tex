%% latex-leseansicht-vorspann.tex
%% Vorspann für die Leseansicht.
%% Lädt die gemeinsame Datei latex-vorspann.tex mit nicht gesetztem Schalter.

\newif\ifkorrekturansicht
\korrekturansichtfalse

\input{../tex-inputs/latex-vorspann}


         
         \renewcommand{\erwaehntePersonen}{Personen: Anselm Feuerbach, Paul Goldmann, Hugo von Hofmannsthal, Marie Reinhard, Paul Schubring}
         \renewcommand{\erwaehnteInstitutionen}{Institutionen: Anping Maru, Frankfurter Zeitung}
         \renewcommand{\erwaehnteOrte}{Orte: Gelbes Meer, Taku Shi, Wien, Yantai}
         \renewcommand{\erwaehnteWerke}{Werke: Der Schleier der Beatrice. Schauspiel in fünf Akten, Frankfurter Zeitung, Giotto in Assisi. Zum 600. Geburtstag seiner Fresken in der Oberkirche San Francesco}
               \section[ Paul Goldmann an Arthur Schnitzler, 18. 10. {[}1898{]}]{ Paul Goldmann an Arthur Schnitzler, 18. 10. {[}1898{]}}\nopagebreak\mylabel{v}\rehead{ }\begin{ledgroupsized}[t]{13cm}\normalsize\beginnumbering\briefempfaengerindex{Schnitzler, Arthur@\textsc{Schnitzler, Arthur}!zzzGoldmann, Paul@\emph{von Paul Goldmann}!1898-10-181@{18. 10. {[}1898{]}}|(be} \toendnotes[C]{\smallbreak\pagebreak[2]} \Standort{DLA, A:Schnitzler, HS.NZ85.1.3168.}
\physDesc{Brief, 1 Blatt, 2 Seiten, 923 Zeichen
\newline{}Handschrift: blaue Tinte, deutsche Kurrent
\newline{}Schnitzler: 1) mit Bleistift das Jahr »98« vermerkt  2) mit rotem Buntstift eine Unterstreichung}\toendnotes[C]{\smallbreak}\pstart
           \raggedleft{}{\pb}18. Oktober. An Bord der »\textsc{Anping\orgindex{Anping Maru@Anping Maru|pw}}«, zwiſchen \textsc{Taku\oindex{Taku Shi@\textbf{Taku Shi}|pw}} und \textsc{Tschifu\oindex{Yantai@\textbf{Yantai}|pw}}.\pend
           \pstart\center{}Mein lieber Freund,\pend\pstart
           Da ich fürchte, daß Dir beifolgendes \label{K_L02862-1v}\edtext{Feuilleton\pwindex{Schubring, Paul 1869-01-28 – 1935-11-07@\textsc{Schubring, Paul} (1869-01-28 – 1935-11-07), \emph{Kunsthistoriker}!Giotto in Assisi. Zum 600. Geburtstag seiner Fresken in der Oberkirche San
                  Francesco1898-09-10@\strich\emph{Giotto in Assisi. Zum 600. Geburtstag seiner Fresken in der Oberkirche San Francesco} {[}1898-09-10{]}|pwv}}{\lemma{\textnormal{\emph{Feuilleton}}}\Cendnote{\textnormal{Paul Schubring\pwindex{Schubring, Paul 1869-01-28 – 1935-11-07@\textsc{Schubring, Paul} (1869-01-28 – 1935-11-07), \emph{Kunsthistoriker}|pwk}: \emph{Giotto in Assisi. Zum 600. Geburtstag seiner Fresken in der
                        Oberkirche San Francesco}\pwindex{Schubring, Paul 1869-01-28 – 1935-11-07@\textsc{Schubring, Paul} (1869-01-28 – 1935-11-07), \emph{Kunsthistoriker}!Giotto in Assisi. Zum 600. Geburtstag seiner Fresken in der Oberkirche San
                  Francesco1898-09-10@\strich\emph{Giotto in Assisi. Zum 600. Geburtstag seiner Fresken in der Oberkirche San Francesco} {[}1898-09-10{]}|pwk}. In: \emph{Frankfurter Zeitung}\pwindex{?? Werk@Nicht ermittelte Verfasserinnen und Verfasser!Frankfurter Zeitung1856 – 1943@\emph{Frankfurter Zeitung} {[}1856 – 1943{]}|pwk}, Jg. 43, Nr. 250, 10. 9. 1898, Erstes
                     Morgenblatt, S. 1–3.}}}\label{K_L02862-1h} entgangen iſt, ſende ich es Dir der
               Sicherheit halber zu. Ich denke mir, es wird Dir recht kommen jetzt wo Du mit einer
                  Arbeit\pwindex{Schnitzler, Arthur 15.05.1862 – 21.10.1931@\textsc{Schnitzler, Arthur} (15.05.1862 – 21.10.1931), \emph{Schriftsteller, Mediziner}!Schleier der Beatrice. Schauspiel in fuenf Akten1900-12-01@\strich\emph{Der Schleier der Beatrice. Schauspiel in fünf Akten} {[}1900-12-01{]}|pwv} über die \textsc{Renaissance} beſchäftigt biſt. Ich habe ſeit Langem nichts ſo
               Schönes über dieſe Zeit geleſen. Auch iſt eine Definition des »Styls« von \textsc{Feuerbach\pwindex{Feuerbach, Anselm 1829-09-12 – 1880-01-04@\textsc{Feuerbach, Anselm} (1829-09-12 – 1880-01-04), \emph{Bildender Künstler, Künstler}|pw}} darin citirt, derentwegen allein es ſich ſchon verlohnt, Dir dieſes Feuilleton\pwindex{Schubring, Paul 1869-01-28 – 1935-11-07@\textsc{Schubring, Paul} (1869-01-28 – 1935-11-07), \emph{Kunsthistoriker}!Giotto in Assisi. Zum 600. Geburtstag seiner Fresken in der Oberkirche San
                  Francesco1898-09-10@\strich\emph{Giotto in Assisi. Zum 600. Geburtstag seiner Fresken in der Oberkirche San Francesco} {[}1898-09-10{]}|pwv} der Frankfurter Zeitung\orgindex{Frankfurter Zeitung@Frankfurter Zeitung|pw} auf Dem Umweg über das Gelbe Meer\oindex{Gelbes Meer@\textbf{Gelbes Meer}|pw} nach Wien\oindex{Wien@\textbf{Wien}|pw} zu
               ſchicken. {\pb}Vergleiche insbeſondere die einfache und
               tiefe Schreibweiſe dieſes unbekannten Gelehrten\pwindex{Schubring, Paul 1869-01-28 – 1935-11-07@\textsc{Schubring, Paul} (1869-01-28 – 1935-11-07), \emph{Kunsthistoriker}|pwv} mit dem \strikeout{\textcolor{gray}{unv}} unverſtändlichen Kauderwelſch, das die »Dichter« \label{K_L02862-2v}\edtext{\textsc{Loris\pwindex{Hofmannsthal, Hugo von 1874-02-01 – 1929-07-15@\textsc{Hofmannsthal, Hugo von} (1874-02-01 – 1929-07-15), \emph{Schriftsteller}|pw}} und Genoſſen anzuwenden ſich befleißen, wenn ſie über die \textsc{Renaissance}}{\lemma{\textnormal{\emph{Loris … Renaissance}}}\Cendnote{\textnormal{Vgl. Paul Goldmann an Arthur Schnitzler, 24. 8. [1898].
               }}}\label{K_L02862-2h} ſchreiben.\pend
           \pstart
           Ich werde in einer halben Stunde wieder ſehr ſeekrank ſein.\pend
           \pstart
           Grüß’ Dich Gott, liebſter Freund!\pend
           \pstart
           Dein treuer {\\[\baselineskip]}\spacefill\mbox{Paul Goldmann}\pend
           \leftskip=0em{}\pstart
           \noindent{}Empfehlungen an Deine Freundin\pwindex{Reinhard, Marie 1871-03-13 – 1899-03-18@\textsc{Reinhard, Marie} (1871-03-13 – 1899-03-18), \emph{Gesangspädagogin}|pwv}!\pend
           
         
         \endnumbering\mylabel{h}\end{ledgroupsized}  \newcommand{\dateiname}{L02862}\newcommand{\titel}{Paul Goldmann an Arthur Schnitzler, 18. 10. [1898]}\newcommand{\editorInnen}{Martin Anton Müller und Laura Untner}%% latex-leseansicht-abspann.tex
%% Abspann für die Leseansicht.
%% Der Schalter \ifkorrekturansicht ist bereits durch den Vorspann gesetzt.

%% latex-abspann.tex
%% Gemeinsamer Abspann für Korrekturansicht und Leseansicht.
%% Setzt den Schalter \ifkorrekturansicht voraus (gesetzt in den
%% einbindenden Dateien latex-korrekturansicht-abspann.tex bzw.
%% latex-leseansicht-abspann.tex).
%% ---------------------------------------------------------------

\normalsize

% Das esempio-Environment wird nur in der Leseansicht benötigt
\ifkorrekturansicht\else
\newenvironment{esempio}[3]%
{
    \vspace{1.5ex}
    \rlap{\underline{#1}}
    \par
    \setlength{\parindent}{0cm}
    \nopagebreak
    \leftskip=#2cm
    \rightskip=#3cm
}
{
    \par
}
\fi

\doendnotes{C}
\bigskip
\vfill

\clearpage

\footnotesize

\ifkorrekturansicht
  \lohead{\textsc{register}}
\fi

% theindex-Environment neu definieren ohne reledmac
\makeatletter
\renewenvironment{theindex}{%
  \ifkorrekturansicht
    \section*{\indexname}%
  \else
    \subsubsection*{Index der erwähnten Entitäten}%
  \fi
  \setlength{\parindent}{0pt}%
  \setlength{\parskip}{0pt plus 0.3pt}%
  \let\item\@idxitem
}{%
  \ifkorrekturansicht\clearpage\fi
}
\makeatother

\IfFileExists{\jobname-pw.ind}{\input{\jobname-pw.ind}}{}

% Quellenangabe nur in der Leseansicht
\ifkorrekturansicht\else
% Fallback-Definitionen, falls die .tex-Datei \titel etc. nicht gesetzt hat
\providecommand{\titel}{}
\providecommand{\editorInnen}{}
\providecommand{\dateiname}{\jobname}

\vspace{3cm}

\vfill

\footnotesize
\textsc{Quelle}: \titel. Herausgegeben von {\editorInnen}. In: \emph{Arthur Schnitzler: Briefwechsel mit Autorinnen und Autoren}.
 Digitale Edition, https://schnitzler-briefe.acdh.oeaw.ac.at/{\dateiname}.html (Stand \today)
\fi

\end{document}


      