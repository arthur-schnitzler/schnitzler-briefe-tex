%% latex-korrekturansicht-vorspann.tex
%% Vorspann für die Korrekturansicht.
%% Lädt die gemeinsame Datei latex-vorspann.tex mit gesetztem Schalter.

\newif\ifkorrekturansicht
\korrekturansichttrue

\input{../tex-inputs/latex-vorspann}


\section[ Paul Goldmann an Arthur Schnitzler, 18. 10. {[}1898{]}]{L02862 Paul Goldmann an Arthur Schnitzler, 18. 10. {[}1898{]}}
\nopagebreak\mylabel{L02862v}
\rehead{ }\normalsize\beginnumbering\briefempfaengerindex{Schnitzler, Arthur@\textsc{Schnitzler, Arthur}!zzzGoldmann, Paul@\emph{von Paul Goldmann}!1898-10-181@{18. 10. {[}1898{]}}|(be}
\toendnotes[C]{\smallbreak\pagebreak[2]}\Standort{DLA, A:Schnitzler, HS.NZ85.1.3168.}
\physDesc{Brief, 1 Blatt, 2 Seiten, 923 Zeichen
\newline{}Handschrift: blaue Tinte, deutsche Kurrent
\newline{}Schnitzler: 1) mit Bleistift das Jahr »98« vermerkt  2) mit rotem Buntstift eine Unterstreichung}\toendnotes[C]{\smallbreak}
\pstart
           \raggedleft{}{\pb}18. Oktober. An Bord der »\textsc{Anping\orgindex{Anping Maru@Anping Maru|pw}}«, zwiſchen \textsc{Taku\oindex{Taku Shi@\textbf{Taku Shi}, \emph{A.ADM2}|pw}} und \textsc{Tschifu\oindex{Yantai@\textbf{Yantai}, \emph{Besiedelter Ort (A.BSO)}|pw}}.\pend
           
\pstart\center{}Mein lieber Freund,\pend\vspace{0.5em}
\pstart
           Da ich fürchte, daß Dir beifolgendes \label{K_L02862-1v}\edtext{Feuilleton\pwindex{Giotto in Assisi. Zum 600. Geburtstag seiner Fresken in der Oberkirche San Francesco@\emph{Giotto in Assisi. Zum 600. Geburtstag seiner Fresken in der Oberkirche San Francesco}|pwv}}{\lemma{\textnormal{\emph{Feuilleton}}}\Cendnote{\textnormal{Paul Schubring\pwindex{Schubring, Paul 1869-01-28 – 1935-11-07@\textsc{Schubring, Paul} (1869-01-28 – 1935-11-07), \emph{Kunsthistoriker/Kunsthistorikerin}|pwk}: \emph{Giotto in Assisi. Zum 600. Geburtstag seiner Fresken in der
                        Oberkirche San Francesco}\pwindex{Giotto in Assisi. Zum 600. Geburtstag seiner Fresken in der Oberkirche San Francesco@\emph{Giotto in Assisi. Zum 600. Geburtstag seiner Fresken in der Oberkirche San Francesco}|pwk}. In: \emph{Frankfurter Zeitung}\pwindex{Frankfurter Zeitung@\emph{Frankfurter Zeitung}|pwk}, Jg. 43, Nr. 250, 10. 9. 1898, Erstes
                     Morgenblatt, S. 1–3.}}}\label{K_L02862-1} entgangen iſt, ſende ich es Dir der
               Sicherheit halber zu. Ich denke mir, es wird Dir recht kommen jetzt wo Du mit einer
                  Arbeit\pwindex{Schleier der Beatrice. Schauspiel in fuenf Akten@\emph{Der Schleier der Beatrice. Schauspiel in fünf Akten}|pwv} über die \textsc{Renaissance} beſchäftigt biſt. Ich habe ſeit Langem nichts ſo
               Schönes über dieſe Zeit geleſen. Auch iſt eine Definition des »Styls« von \textsc{Feuerbach\pwindex{Feuerbach, Anselm 1829-09-12 – 1880-01-04@\textsc{Feuerbach, Anselm} (1829-09-12 – 1880-01-04), \emph{Maler/Malerin, Künstler/Künstlerin}|pw}} darin citirt, derentwegen allein es ſich ſchon verlohnt, Dir dieſes Feuilleton\pwindex{Giotto in Assisi. Zum 600. Geburtstag seiner Fresken in der Oberkirche San Francesco@\emph{Giotto in Assisi. Zum 600. Geburtstag seiner Fresken in der Oberkirche San Francesco}|pwv} der Frankfurter Zeitung\orgindex{Frankfurter Zeitung@Frankfurter Zeitung|pw} auf Dem Umweg über das Gelbe Meer\oindex{Gelbes Meer@\textbf{Gelbes Meer}, \emph{Meer (N.MER)}|pw} nach Wien\oindex{Wien@\textbf{Wien}, \emph{A.ADM2}|pw} zu
               ſchicken. {\pb}Vergleiche insbeſondere die einfache und
               tiefe Schreibweiſe dieſes unbekannten Gelehrten\pwindex{Schubring, Paul 1869-01-28 – 1935-11-07@\textsc{Schubring, Paul} (1869-01-28 – 1935-11-07), \emph{Kunsthistoriker/Kunsthistorikerin}|pwv} mit dem \strikeout{\textcolor{gray}{unv}} unverſtändlichen Kauderwelſch, das die »Dichter« \label{K_L02862-2v}\edtext{\textsc{Loris\pwindex{Hofmannsthal, Hugo von 1874-02-01 – 1929-07-15@\textsc{Hofmannsthal, Hugo von} (1874-02-01 – 1929-07-15), \emph{Schriftsteller/Schriftstellerin}|pw}} und Genoſſen anzuwenden ſich befleißen, wenn ſie über die \textsc{Renaissance}}{\lemma{\textnormal{\emph{Loris … Renaissance}}}\Cendnote{\textnormal{Vgl. Paul Goldmann an Arthur Schnitzler, 24. 8. [1898].
               }}}\label{K_L02862-2} ſchreiben.\pend
           
\pstart
           Ich werde in einer halben Stunde wieder ſehr ſeekrank ſein.\pend
           
\pstart
           Grüß’ Dich Gott, liebſter Freund!\pend
           
\pstart
           Dein treuer {\\[\baselineskip]}\spacefill\mbox{Paul Goldmann}\pend
           \leftskip=0em{}
\pstart
           \noindent{}Empfehlungen an Deine Freundin\pwindex{Reinhard, Marie 1871-03-13 – 1899-03-18@\textsc{Reinhard, Marie} (1871-03-13 – 1899-03-18), \emph{Gesangspädagoge/Gesangspädagogin}|pwv}!\pend
           \selectlanguage{ngerman}\endnumbering\briefempfaengerindex{Schnitzler, Arthur@\textsc{Schnitzler, Arthur}!zzzGoldmann, Paul@\emph{von Paul Goldmann}!1898-10-181@{18. 10. {[}1898{]}}|)be}\mylabel{L02862h}  \normalsize

\doendnotes{C}
\bigskip
\vfill

\clearpage

\footnotesize

\lohead{\textsc{register}}

% Definiere theindex-Environment komplett neu ohne reledmac
\makeatletter
\renewenvironment{theindex}{%
  \section*{\indexname}%
  \setlength{\parindent}{0pt}%
  \setlength{\parskip}{0pt plus 0.3pt}%
  \let\item\@idxitem
}{%
  \clearpage
}
\makeatother

\IfFileExists{\jobname-pw.ind}{\input{\jobname-pw.ind}}{}

\end{document}

      