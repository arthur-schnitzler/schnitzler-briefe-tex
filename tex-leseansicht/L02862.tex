%% latex-leseansicht-vorspann.tex
%% Vorspann für die Leseansicht.
%% Lädt die gemeinsame Datei latex-vorspann.tex mit nicht gesetztem Schalter.

\newif\ifkorrekturansicht
\korrekturansichtfalse

\input{../tex-inputs/latex-vorspann}


\section[ Paul Goldmann an Arthur Schnitzler, 18. 10. [1898]]{L02862 Paul Goldmann an Arthur Schnitzler,  18. 10. [1898]}
\nopagebreak\mylabel{L02862v}
\rehead{ }\normalsize\beginnumbering\briefempfaengerindex{Schnitzler, Arthur@\textsc{Schnitzler, Arthur}!zzzGoldmann, Paul@\emph{von Paul Goldmann}!1898-10-181@{18. 10. [1898]}|(be}
\toendnotes[C]{\smallbreak\pagebreak[2]}
\correspDesc{Versand  durch Paul Goldmann am 18. 10. [1898] in Gelbes Meer
\newline{}Erhalt  durch Arthur Schnitzler im Zeitraum [18. 11. 1898 – 19. 12. 1898?] in Wien}\toendnotes[C]{\smallbreak}
\Standort{DLA, A:Schnitzler, HS.NZ85.1.3168.}
\physDesc{Brief, 1 Blatt, 2 Seiten, 923 Zeichen
\newline{}Handschrift: blaue Tinte, deutsche Kurrent
\newline{}Schnitzler: 1) mit Bleistift das Jahr »98« vermerkt  2) mit rotem Buntstift eine Unterstreichung}\toendnotes[C]{\smallbreak}
\pstart
           \raggedleft{}{\pb}18. Oktober. An Bord der »\textsc{Anping\orgindex{Anping Maru@Anping Maru|pw}}«, zwiſchen \textsc{Taku\oindex{Taku Shi@\textbf{Taku Shi}, \emph{Verwaltungsgebiet}|pw}} und \textsc{Tschifu\oindex{Yantai@\textbf{Yantai}|pw}}.\pend
           
\pstart\center{}Mein lieber Freund,\pend\vspace{0.5em}
\pstart
           Da ich fürchte, daß Dir beifolgendes \label{K_L02862-1v}\edtext{Feuilleton\pwindex{Schubring, Paul 28.\,1.\,1869 Bad Godesberg – 7.\,11.\,1935 Hannover@\textsc{Schubring, Paul} (28.\,1.\,1869 Bad Godesberg – 7.\,11.\,1935 Hannover), \emph{Kunsthistoriker}!Giotto in Assisi. Zum 600. Geburtstag seiner Fresken in der Oberkirche San Francesco@\strich\emph{Giotto in Assisi. Zum 600. Geburtstag seiner Fresken in der Oberkirche San Francesco}|pwv}}{\lemma{\textnormal{\emph{Feuilleton}}}\Cendnote{\textnormal{Paul Schubring\pwindex{Schubring, Paul 28.\,1.\,1869 Bad Godesberg – 7.\,11.\,1935 Hannover@\textsc{Schubring, Paul} (28.\,1.\,1869 Bad Godesberg – 7.\,11.\,1935 Hannover), \emph{Kunsthistoriker}|pwk}: \emph{Giotto in Assisi. Zum 600. Geburtstag seiner Fresken in der
                        Oberkirche San Francesco}\pwindex{Schubring, Paul 28.\,1.\,1869 Bad Godesberg – 7.\,11.\,1935 Hannover@\textsc{Schubring, Paul} (28.\,1.\,1869 Bad Godesberg – 7.\,11.\,1935 Hannover), \emph{Kunsthistoriker}!Giotto in Assisi. Zum 600. Geburtstag seiner Fresken in der Oberkirche San Francesco@\strich\emph{Giotto in Assisi. Zum 600. Geburtstag seiner Fresken in der Oberkirche San Francesco}|pwk}. In: \emph{Frankfurter Zeitung}\pwindex{Frankfurter Zeitung@\emph{Frankfurter Zeitung}|pwk}, Jg. 43, Nr. 250, 10. 9. 1898, Erstes
                     Morgenblatt, S. 1–3.}}}\label{K_L02862-1} entgangen iſt,{ }ſende ich es Dir der
               Sicherheit halber zu. Ich denke mir, es wird Dir recht kommen jetzt wo Du mit einer
                  Arbeit\pwindex{Schnitzler, Arthur 15.\,5.\,1862 Wien – 21.\,10.\,1931 ebd.@\textsc{Schnitzler, Arthur} (15.\,5.\,1862 Wien – 21.\,10.\,1931 ebd.), \emph{Schriftsteller, Mediziner}!Schleier der Beatrice. Schauspiel in fünf Akten@\strich\emph{Der Schleier der Beatrice. Schauspiel in fünf Akten}|pwv} über die \textsc{Renaissance} beſchäftigt biſt. Ich habe{ }ſeit Langem nichts{ }ſo
               Schönes über dieſe Zeit geleſen. Auch iſt eine Definition des »Styls« von \textsc{Feuerbach\pwindex{Feuerbach, Anselm 12.\,9.\,1829 Speyer – 4.\,1.\,1880 Venedig@\textsc{Feuerbach, Anselm} (12.\,9.\,1829 Speyer – 4.\,1.\,1880 Venedig), \emph{Maler, Künstler}|pw}} darin citirt, derentwegen allein es{ }ſich{ }ſchon verlohnt, Dir dieſes Feuilleton\pwindex{Schubring, Paul 28.\,1.\,1869 Bad Godesberg – 7.\,11.\,1935 Hannover@\textsc{Schubring, Paul} (28.\,1.\,1869 Bad Godesberg – 7.\,11.\,1935 Hannover), \emph{Kunsthistoriker}!Giotto in Assisi. Zum 600. Geburtstag seiner Fresken in der Oberkirche San Francesco@\strich\emph{Giotto in Assisi. Zum 600. Geburtstag seiner Fresken in der Oberkirche San Francesco}|pwv} der Frankfurter Zeitung\orgindex{Frankfurter Zeitung@Frankfurter Zeitung|pw} auf Dem Umweg über das Gelbe Meer\oindex{Gelbes Meer@\textbf{Gelbes Meer}|pw} nach Wien\oindex{Wien@\textbf{Wien}, \emph{Verwaltungsgebiet}|pw} zu{ }ſchicken. {\pb}Vergleiche insbeſondere die einfache und
               tiefe Schreibweiſe dieſes unbekannten Gelehrten\pwindex{Schubring, Paul 28.\,1.\,1869 Bad Godesberg – 7.\,11.\,1935 Hannover@\textsc{Schubring, Paul} (28.\,1.\,1869 Bad Godesberg – 7.\,11.\,1935 Hannover), \emph{Kunsthistoriker}|pwv} mit dem \strikeout{\textcolor{gray}{unv}} unverſtändlichen Kauderwelſch, das die »Dichter« \label{K_L02862-2v}\edtext{\textsc{Loris\pwindex{Hofmannsthal, Hugo von 1.\,2.\,1874 Wien – 15.\,7.\,1929 Rodaun@\textsc{Hofmannsthal, Hugo von} (1.\,2.\,1874 Wien – 15.\,7.\,1929 Rodaun), \emph{Schriftsteller}|pw}} und Genoſſen anzuwenden{ }ſich befleißen, wenn{ }ſie über die \textsc{Renaissance}}{\lemma{\textnormal{\emph{Loris … Renaissance}}}\Cendnote{\textnormal{Vgl. XXXX Auszeichnungsfehler: Dokument L02854 nicht gefunden.
               }}}\label{K_L02862-2}{ }ſchreiben.\pend
           
\pstart
           Ich werde in einer halben Stunde wieder{ }ſehr{ }ſeekrank{ }ſein.\pend
           
\pstart
           Grüß’ Dich Gott, liebſter Freund!\pend
           
\pstart
           Dein treuer {\\[\baselineskip]}\spacefill\mbox{Paul Goldmann}\pend
           \leftskip=0em{}
\pstart
           \noindent{}Empfehlungen an Deine Freundin\pwindex{Reinhard, Marie 13.\,3.\,1871 Wien – 18.\,3.\,1899 ebd.@\textsc{Reinhard, Marie} (13.\,3.\,1871 Wien – 18.\,3.\,1899 ebd.), \emph{Gesangspädagogin}|pwv}!\pend
           \selectlanguage{ngerman}\endnumbering\briefempfaengerindex{Schnitzler, Arthur@\textsc{Schnitzler, Arthur}!zzzGoldmann, Paul@\emph{von Paul Goldmann}!1898-10-181@{18. 10. [1898]}|)be}\mylabel{L02862h}  \newcommand{\dateiname}{L02862}\newcommand{\titel}{Paul Goldmann an Arthur Schnitzler, 18. 10. [1898]}\newcommand{\editorInnen}{Martin Anton Müller und Laura Untner}%% latex-leseansicht-abspann.tex
%% Abspann für die Leseansicht.
%% Der Schalter \ifkorrekturansicht ist bereits durch den Vorspann gesetzt.

%% latex-abspann.tex
%% Gemeinsamer Abspann für Korrekturansicht und Leseansicht.
%% Setzt den Schalter \ifkorrekturansicht voraus (gesetzt in den
%% einbindenden Dateien latex-korrekturansicht-abspann.tex bzw.
%% latex-leseansicht-abspann.tex).
%% ---------------------------------------------------------------

\normalsize

% Das esempio-Environment wird nur in der Leseansicht benötigt
\ifkorrekturansicht\else
\newenvironment{esempio}[3]%
{
    \vspace{1.5ex}
    \rlap{\underline{#1}}
    \par
    \setlength{\parindent}{0cm}
    \nopagebreak
    \leftskip=#2cm
    \rightskip=#3cm
}
{
    \par
}
\fi

\doendnotes{C}
\bigskip
\vfill

\clearpage

\footnotesize

\ifkorrekturansicht
  \lohead{\textsc{register}}
\fi

% theindex-Environment neu definieren ohne reledmac
\makeatletter
\renewenvironment{theindex}{%
  \ifkorrekturansicht
    \section*{\indexname}%
  \else
    \subsubsection*{Index der erwähnten Entitäten}%
  \fi
  \setlength{\parindent}{0pt}%
  \setlength{\parskip}{0pt plus 0.3pt}%
  \let\item\@idxitem
}{%
  \ifkorrekturansicht\clearpage\fi
}
\makeatother

\IfFileExists{\jobname-pw.ind}{\input{\jobname-pw.ind}}{}

% Quellenangabe nur in der Leseansicht
\ifkorrekturansicht\else
% Fallback-Definitionen, falls die .tex-Datei \titel etc. nicht gesetzt hat
\providecommand{\titel}{}
\providecommand{\editorInnen}{}
\providecommand{\dateiname}{\jobname}

\vspace{3cm}

\vfill

\footnotesize
\textsc{Quelle}: \titel. Herausgegeben von {\editorInnen}. In: \emph{Arthur Schnitzler: Briefwechsel mit Autorinnen und Autoren}.
 Digitale Edition, https://schnitzler-briefe.acdh.oeaw.ac.at/{\dateiname}.html (Stand \today)
\fi

\end{document}


