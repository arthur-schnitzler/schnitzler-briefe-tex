%% latex-leseansicht-vorspann.tex
%% Vorspann für die Leseansicht.
%% Lädt die gemeinsame Datei latex-vorspann.tex mit nicht gesetztem Schalter.

\newif\ifkorrekturansicht
\korrekturansichtfalse

\input{../tex-inputs/latex-vorspann}


\section[Richard Beer-Hofmann an Arthur Schnitzler, 22. 6. 1897]{L00689 Richard Beer-Hofmann an Arthur Schnitzler, 22. 6. 1897}
\nopagebreak\mylabel{L00689v}
\rehead{ }\normalsize\beginnumbering\briefempfaengerindex{Schnitzler, Arthur@\textsc{Schnitzler, Arthur}!zzzBeer-Hofmann, Richard@\emph{von Richard Beer-Hofmann}!1897-06-221@{22. 6. 1897}|(be}
\toendnotes[C]{\smallbreak\pagebreak[2]}
\correspDesc{Versand  durch Richard Beer-Hofmann am 22. 6. 1897 in Bad Ischl
\newline{}Erhalt  durch Arthur Schnitzler am 23. 6. 1897 in Wien}\toendnotes[C]{\smallbreak}
\Standort{CUL, Schnitzler, B 8.}
\physDesc{Brief, 1 Blatt, 4 Seiten, 627 Zeichen
\newline{}Handschrift: blauer Buntstift, lateinische Kurrent
\newline{}Ordnung: mit Bleistift von unbekannter Hand nummeriert:
                                    »100« }
\buchAbdrucke{\weitereDrucke{Arthur Schnitzler, Richard Beer-Hofmann: \emph{Briefwechsel 1891–1931}. Herausgegeben von Konstanze Fliedl. Wien, Zürich: \emph{Europaverlag} 1992, S. 110.} }\toendnotes[C]{\smallbreak}
\pstart
           \centering{}{\pb}Ischl\oindex{Bad Ischl@\textbf{Bad Ischl}|pw}{ }22/VI 97\pend
           \vspace{0.5em}
\pstart
           Lieber Arthur, sie haben meinen letzten Brief nicht
               beantwortet und ko{\geminationm}en daher wol sehr bald. Bitte
               besorgen Sie mir – ohne Nervosität Folgendes:\pend
           
\pstart
           I. Eine Pincette – vernickelt oder {\pb}in Silber.\pend
           
\pstart
           2.) Im Durchhaus in der Wollzeile\oindex{Wien@\textbf{Wien}!I., Innere Stadt@\textbf{I., Innere Stadt}!Wollzeile@\textbf{Wollzeile}, \emph{Straße}|pw} das auf den
               alten Universitätsplatz\oindex{Dr.-Ignaz-Seipel-Platz@\textbf{Dr.-Ignaz-Seipel-Platz}, \emph{Platz}|pw} führt ist ein Tierhändler\orgindex{G. Findeis@G. Findeis|pwv}; dort kaufen Sie um
               circa \label{K_L00689-1v}\edtext{50 xr}{\lemma{\textnormal{\emph{50 xr}}}\Cendnote{\textnormal{50 Kreuzer}}}\label{K_L00689-1} Vogelfutter für Wellenpapageie.\pend
           
\pstart
           3.) Im Durchhaus Graben\oindex{Wien@\textbf{Wien}!I., Innere Stadt@\textbf{I., Innere Stadt}!Graben@\textbf{Graben}, \emph{Straße}|pw}{ }Goldschmidt{\pb}gasse\oindex{Goldschmiedgasse@\textbf{Goldschmiedgasse}, \emph{Straße}|pw} die \label{K_L00689-2v}\edtext{Dinge}{\lemma{\textnormal{\emph{Dinge}}}\Cendnote{\textnormal{Kondome}}}\label{K_L00689-2} die Sie auch dort kaufen.\pend
           
\pstart
           \strikeout{4.) \strikeout{Wi} Im Verlag der »Wiener Mode\orgindex{Wiener Mode@Wiener Mode|pw}« ist ein Pro\pwindex{\textcolor{red}{\textsuperscript{XXXX indx1}}!Pro und Contra. Eine hygienische Studie über das Radfahren@\strich\emph{Pro und Contra. Eine hygienische Studie über das Radfahren}|pwu}} überflüssig.\pend
           
\pstart
           Ich bin da es viel regnet erst einmal auf der Strasse gefahren. Hoffe wenn Sie ko{\geminationm}en {\pb}öfters. Schwarzkopf\pwindex{Schwarzkopf, Gustav 7.\,11.\,1853 Wien – 13.\,11.\,1939 ebd.@\textsc{Schwarzkopf, Gustav} (7.\,11.\,1853 Wien – 13.\,11.\,1939 ebd.), \emph{Schriftsteller}|pw} viele Grüße – ko{\geminationm}t er?\pend
           
\pstart
           Auf Wiedersehen{\\[\baselineskip]}\spacefill\mbox{Richard}\pend
           \leftskip=0em{}\selectlanguage{ngerman}\endnumbering\briefempfaengerindex{Schnitzler, Arthur@\textsc{Schnitzler, Arthur}!zzzBeer-Hofmann, Richard@\emph{von Richard Beer-Hofmann}!1897-06-221@{22. 6. 1897}|)be}\mylabel{L00689h}  \newcommand{\dateiname}{L00689}\newcommand{\titel}{Richard Beer-Hofmann an Arthur Schnitzler, 22. 6. 1897}\newcommand{\editorInnen}{Martin Anton Müller und Gerd-Hermann Susen}%% latex-leseansicht-abspann.tex
%% Abspann für die Leseansicht.
%% Der Schalter \ifkorrekturansicht ist bereits durch den Vorspann gesetzt.

%% latex-abspann.tex
%% Gemeinsamer Abspann für Korrekturansicht und Leseansicht.
%% Setzt den Schalter \ifkorrekturansicht voraus (gesetzt in den
%% einbindenden Dateien latex-korrekturansicht-abspann.tex bzw.
%% latex-leseansicht-abspann.tex).
%% ---------------------------------------------------------------

\normalsize

% Das esempio-Environment wird nur in der Leseansicht benötigt
\ifkorrekturansicht\else
\newenvironment{esempio}[3]%
{
    \vspace{1.5ex}
    \rlap{\underline{#1}}
    \par
    \setlength{\parindent}{0cm}
    \nopagebreak
    \leftskip=#2cm
    \rightskip=#3cm
}
{
    \par
}
\fi

\doendnotes{C}
\bigskip
\vfill

\clearpage

\footnotesize

\ifkorrekturansicht
  \lohead{\textsc{register}}
\fi

% theindex-Environment neu definieren ohne reledmac
\makeatletter
\renewenvironment{theindex}{%
  \ifkorrekturansicht
    \section*{\indexname}%
  \else
    \subsubsection*{Index der erwähnten Entitäten}%
  \fi
  \setlength{\parindent}{0pt}%
  \setlength{\parskip}{0pt plus 0.3pt}%
  \let\item\@idxitem
}{%
  \ifkorrekturansicht\clearpage\fi
}
\makeatother

\IfFileExists{\jobname-pw.ind}{\input{\jobname-pw.ind}}{}

% Quellenangabe nur in der Leseansicht
\ifkorrekturansicht\else
% Fallback-Definitionen, falls die .tex-Datei \titel etc. nicht gesetzt hat
\providecommand{\titel}{}
\providecommand{\editorInnen}{}
\providecommand{\dateiname}{\jobname}

\vspace{3cm}

\vfill

\footnotesize
\textsc{Quelle}: \titel. Herausgegeben von {\editorInnen}. In: \emph{Arthur Schnitzler: Briefwechsel mit Autorinnen und Autoren}.
 Digitale Edition, https://schnitzler-briefe.acdh.oeaw.ac.at/{\dateiname}.html (Stand \today)
\fi

\end{document}


