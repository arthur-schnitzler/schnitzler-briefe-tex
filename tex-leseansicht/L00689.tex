%% latex-korrekturansicht-vorspann.tex
%% Vorspann für die Korrekturansicht.
%% Lädt die gemeinsame Datei latex-vorspann.tex mit gesetztem Schalter.

\newif\ifkorrekturansicht
\korrekturansichttrue

\input{../tex-inputs/latex-vorspann}


\section[Richard Beer-Hofmann an Arthur Schnitzler, 22. 6. 1897]{L00689 Richard Beer-Hofmann an Arthur Schnitzler, 22. 6. 1897}
\nopagebreak\mylabel{L00689v}
\rehead{ }\normalsize\beginnumbering\briefempfaengerindex{Schnitzler, Arthur@\textsc{Schnitzler, Arthur}!zzzBeer-Hofmann, Richard@\emph{von Richard Beer-Hofmann}!1897-06-221@{22. 6. 1897}|(be}
\toendnotes[C]{\smallbreak\pagebreak[2]}\Standort{CUL, Schnitzler, B 8.}
\physDesc{Brief, 1 Blatt, 4 Seiten, 627 Zeichen
\newline{}Handschrift: blauer Buntstift, lateinische Kurrent
\newline{}Ordnung: mit Bleistift von unbekannter Hand nummeriert:
                                    »100« }
\buchAbdrucke{\weitereDrucke{Arthur Schnitzler, Richard Beer-Hofmann: \emph{Briefwechsel 1891–1931}. Wien, Zürich: \emph{Europaverlag} 1992, S. 110.} }\toendnotes[C]{\smallbreak}
\pstart
           \centering{}{\pb}Ischl\oindex{Bad Ischl@\textbf{Bad Ischl}, \emph{P.PPL}|pw}{ }22/VI 97\pend
           \vspace{0.5em}
\pstart
           Lieber Arthur, sie haben meinen letzten Brief nicht
               beantwortet und ko{\geminationm}en daher wol sehr bald. Bitte
               besorgen Sie mir – ohne Nervosität Folgendes:\pend
           
\pstart
           I. Eine Pincette – vernickelt oder {\pb}in Silber.\pend
           
\pstart
           2.) Im Durchhaus in der Wollzeile\oindex{Wollzeile@\textbf{Wollzeile}, \emph{Straße (K.STR)}|pw} das auf den
               alten Universitätsplatz\oindex{Dr.-Ignaz-Seipel-Platz@\textbf{Dr.-Ignaz-Seipel-Platz}, \emph{Platz (K.PLT)}|pw} führt ist ein Tierhändler\orgindex{G. Findeis@G. Findeis|pwv}; dort kaufen Sie um
               circa \label{K_L00689-1v}\edtext{50 xr}{\lemma{\textnormal{\emph{50 xr}}}\Cendnote{\textnormal{50 Kreuzer}}}\label{K_L00689-1} Vogelfutter für Wellenpapageie.\pend
           
\pstart
           3.) Im Durchhaus Graben\oindex{Graben@\textbf{Graben}, \emph{Straße (K.STR)}|pw}{ }Goldschmidt{\pb}gasse\oindex{Goldschmiedgasse@\textbf{Goldschmiedgasse}, \emph{Straße (K.STR)}|pw} die \label{K_L00689-2v}\edtext{Dinge}{\lemma{\textnormal{\emph{Dinge}}}\Cendnote{\textnormal{Kondome}}}\label{K_L00689-2} die Sie auch dort kaufen.\pend
           
\pstart
           \strikeout{4.) \strikeout{Wi} Im Verlag der »Wiener Mode\orgindex{Wiener Mode@Wiener Mode|pw}« ist ein Pro\pwindex{Pro und Contra. Eine hygienische Studie ueber das Radfahren@\emph{Pro und Contra. Eine hygienische Studie über das Radfahren}|pwu}} überflüssig.\pend
           
\pstart
           Ich bin da es viel regnet erst einmal auf der Strasse gefahren. Hoffe wenn Sie ko{\geminationm}en {\pb}öfters. Schwarzkopf\pwindex{Schwarzkopf, Gustav 07.11.1853 – 13.11.1939@\textsc{Schwarzkopf, Gustav} (07.11.1853 – 13.11.1939), \emph{Schriftsteller/Schriftstellerin}|pw} viele Grüße – ko{\geminationm}t er?\pend
           
\pstart
           Auf Wiedersehen{\\[\baselineskip]}\spacefill\mbox{Richard}\pend
           \leftskip=0em{}\selectlanguage{ngerman}\endnumbering\briefempfaengerindex{Schnitzler, Arthur@\textsc{Schnitzler, Arthur}!zzzBeer-Hofmann, Richard@\emph{von Richard Beer-Hofmann}!1897-06-221@{22. 6. 1897}|)be}\mylabel{L00689h}  \normalsize

\doendnotes{C}
\bigskip
\vfill

\clearpage

\footnotesize

\lohead{\textsc{register}}

% Definiere theindex-Environment komplett neu ohne reledmac
\makeatletter
\renewenvironment{theindex}{%
  \section*{\indexname}%
  \setlength{\parindent}{0pt}%
  \setlength{\parskip}{0pt plus 0.3pt}%
  \let\item\@idxitem
}{%
  \clearpage
}
\makeatother

\IfFileExists{\jobname-pw.ind}{\input{\jobname-pw.ind}}{}

\end{document}

      