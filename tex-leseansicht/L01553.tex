%% latex-leseansicht-vorspann.tex
%% Vorspann für die Leseansicht.
%% Lädt die gemeinsame Datei latex-vorspann.tex mit nicht gesetztem Schalter.

\newif\ifkorrekturansicht
\korrekturansichtfalse

\input{../tex-inputs/latex-vorspann}


\section[Hugo von Hofmannsthal an Arthur Schnitzler, 30. 9. 1905]{L01553 Hugo von Hofmannsthal an Arthur Schnitzler, 30. 9. 1905}
\nopagebreak\mylabel{L01553v}
\rehead{ }\normalsize\beginnumbering\briefempfaengerindex{Schnitzler, Arthur@\textsc{Schnitzler, Arthur}!zzzHofmannsthal, Hugo von@\emph{von Hugo von Hofmannsthal}!1905-09-301@{30. 9. 1905}|(be}
\toendnotes[C]{\smallbreak\pagebreak[2]}
\correspDesc{Versand  durch Hugo von Hofmannsthal am 30. 9. 1905 in Rodaun
\newline{}Erhalt  durch Arthur Schnitzler am 2. 10. 1905 in Wien}\toendnotes[C]{\smallbreak}
\Standort{CUL, Schnitzler, B 43.}
\physDesc{Postkarte, 550 Zeichen
\newline{}Handschrift: schwarze Tinte, deutsche Kurrent
\newline{}Versand: 1) Stempel: »\nobreak{}\oindex{Wien@\textbf{Wien}!XXIII., Liesing@\textbf{XXIII., Liesing}!Rodaun@\textbf{Rodaun}, \emph{Region}|pwk}{[}Rodau{]}n, 1 10 \textcolor{gray}{05}\nobreak{}«.   2) Stempel: »\nobreak{}\oindex{XVIII., Währing@\textbf{XVIII., Währing}, \emph{Verwaltungsgebiet}|pwk}18/1 Wien 110, 2 X 05, VIII, Bestellt\nobreak{}«. 
\newline{}Ordnung: 1) mit Bleistift von unbekannter Hand nummeriert:
                                    »253«  2) mit Bleistift von unbekannter Hand nummeriert:
                                    »258a«}
\buchAbdrucke{\weitereDrucke{Hugo von Hofmannsthal, Arthur Schnitzler: \emph{Briefwechsel}. Herausgegeben von Therese Nickl und Heinrich Schnitzler. Frankfurt am Main: \emph{S. Fischer} 1964, S. 215.} }\toendnotes[C]{\smallbreak}\pstart{}{\pb}\textsc{Herrn D\textsuperscript{r} Arthur Schnitzler}\pend{}\pstart{}\textsc{Wien}\oindex{Wien@\textbf{Wien}, \emph{Verwaltungsgebiet}|pw}\pend{}\pstart{}\textsc{XVIII Spöttelgasse 7}.\oindex{Wien@\textbf{Wien}!XVIII., Währing@\textbf{XVIII., Währing}!Edmund-Weiß-Gasse 7@\textbf{Edmund-Weiß-Gasse 7}, \emph{Wohngebäude}|pw}\pend{}{\bigskip}\vspace{1em}
\pstart
           \raggedleft{}{\pb}Samstg 30/9 905\pend
           \vspace{0.5em}
\pstart
           lieber, ich bin{ }ſchon über eine Woche zurück, arbeite aber vor- und
               nachmittg, wenn ich nicht, wie zufällig heute, unwohl bin. Ich höre von Bahr, daſs
               der »Ruf des Lebens\pwindex{Schnitzler, Arthur 15.\,5.\,1862 Wien – 21.\,10.\,1931 ebd.@\textsc{Schnitzler, Arthur} (15.\,5.\,1862 Wien – 21.\,10.\,1931 ebd.), \emph{Schriftsteller, Mediziner}!Ruf des Lebens. Schauspiel in drei Akten@\strich\emph{Der Ruf des Lebens. Schauspiel in drei Akten}|pw}«{ }ſchon in irgend einer Form
               lesbar vorliegt. Ich wäre{ }ſehr froh, es im Ganzen zu leſen. Dem »Zwiſchenſpiel\pwindex{Schnitzler, Arthur 15.\,5.\,1862 Wien – 21.\,10.\,1931 ebd.@\textsc{Schnitzler, Arthur} (15.\,5.\,1862 Wien – 21.\,10.\,1931 ebd.), \emph{Schriftsteller, Mediziner}!Zwischenspiel. Komödie in drei Akten@\strich\emph{Zwischenspiel. Komödie in drei Akten}|pw}« bewahre ich die{ }ſchönſte Erinnerung und würde mich
                auf die \label{K_L01553-1v}\edtext{Aufführung\eventindex{Burgtheater@\textbf{Burgtheater}!Uraufführung von Zwischenspiel, 12.10.1905@Uraufführung von Zwischenspiel, 12.10.1905|pwv}}{\lemma{\textnormal{\emph{Aufführung}}}\Cendnote{\textnormal{Die Uraufführung\eventindex{Burgtheater@\textbf{Burgtheater}!Uraufführung von Zwischenspiel, 12.10.1905@Uraufführung von Zwischenspiel, 12.10.1905|pwkv} fand am
                     12. 10. 1905 statt.}}}\label{K_L01553-1}{ }ſehr freuen, wäre nicht \uuline{Witt}\pwindex{Witt, Lotte 24.\,4.\,1870 Berlin – 28.\,12.\,1938 Wien@\textsc{Witt, Lotte} (24.\,4.\,1870 Berlin – 28.\,12.\,1938 Wien), \emph{Schauspielerin}|pw}! Unbegreiflich! Unerklärlich!\pend
           \pstart Ihr \spacefill\mbox{Hugo}\pend{}
\pstart
           \noindent{}\label{T_L01553-1v}\edtext{\textsc{Frl. W.\pwindex{Witt, Lotte 24.\,4.\,1870 Berlin – 28.\,12.\,1938 Wien@\textsc{Witt, Lotte} (24.\,4.\,1870 Berlin – 28.\,12.\,1938 Wien), \emph{Schauspielerin}|pw}} iſt für mich eines der unangenehmſten Geſchöpfe der deutſchen Bühnen.}{\lemma{\textnormal{\emph{Frl. … Bühnen.}}}\Cendnote{\textnormal{quer am linken Rand}}}\label{T_L01553-1}\pend
           \selectlanguage{ngerman}\endnumbering\briefempfaengerindex{Schnitzler, Arthur@\textsc{Schnitzler, Arthur}!zzzHofmannsthal, Hugo von@\emph{von Hugo von Hofmannsthal}!1905-09-301@{30. 9. 1905}|)be}\mylabel{L01553h}  \newcommand{\dateiname}{L01553}\newcommand{\titel}{Hugo von Hofmannsthal an Arthur Schnitzler, 30. 9. 1905}\newcommand{\editorInnen}{Martin Anton Müller und Gerd-Hermann Susen}%% latex-leseansicht-abspann.tex
%% Abspann für die Leseansicht.
%% Der Schalter \ifkorrekturansicht ist bereits durch den Vorspann gesetzt.

%% latex-abspann.tex
%% Gemeinsamer Abspann für Korrekturansicht und Leseansicht.
%% Setzt den Schalter \ifkorrekturansicht voraus (gesetzt in den
%% einbindenden Dateien latex-korrekturansicht-abspann.tex bzw.
%% latex-leseansicht-abspann.tex).
%% ---------------------------------------------------------------

\normalsize

% Das esempio-Environment wird nur in der Leseansicht benötigt
\ifkorrekturansicht\else
\newenvironment{esempio}[3]%
{
    \vspace{1.5ex}
    \rlap{\underline{#1}}
    \par
    \setlength{\parindent}{0cm}
    \nopagebreak
    \leftskip=#2cm
    \rightskip=#3cm
}
{
    \par
}
\fi

\doendnotes{C}
\bigskip
\vfill

\clearpage

\footnotesize

\ifkorrekturansicht
  \lohead{\textsc{register}}
\fi

% theindex-Environment neu definieren ohne reledmac
\makeatletter
\renewenvironment{theindex}{%
  \ifkorrekturansicht
    \section*{\indexname}%
  \else
    \subsubsection*{Index der erwähnten Entitäten}%
  \fi
  \setlength{\parindent}{0pt}%
  \setlength{\parskip}{0pt plus 0.3pt}%
  \let\item\@idxitem
}{%
  \ifkorrekturansicht\clearpage\fi
}
\makeatother

\IfFileExists{\jobname-pw.ind}{\input{\jobname-pw.ind}}{}

% Quellenangabe nur in der Leseansicht
\ifkorrekturansicht\else
% Fallback-Definitionen, falls die .tex-Datei \titel etc. nicht gesetzt hat
\providecommand{\titel}{}
\providecommand{\editorInnen}{}
\providecommand{\dateiname}{\jobname}

\vspace{3cm}

\vfill

\footnotesize
\textsc{Quelle}: \titel. Herausgegeben von {\editorInnen}. In: \emph{Arthur Schnitzler: Briefwechsel mit Autorinnen und Autoren}.
 Digitale Edition, https://schnitzler-briefe.acdh.oeaw.ac.at/{\dateiname}.html (Stand \today)
\fi

\end{document}


