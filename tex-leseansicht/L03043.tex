%% latex-korrekturansicht-vorspann.tex
%% Vorspann für die Korrekturansicht.
%% Lädt die gemeinsame Datei latex-vorspann.tex mit gesetztem Schalter.

\newif\ifkorrekturansicht
\korrekturansichttrue

\input{../tex-inputs/latex-vorspann}


\section[ Felix Salten: Widmungsexemplar Simson für Arthur Schnitzler, 1. 10. 1928]{L03043 Felix Salten: Widmungsexemplar Simson für Arthur
               Schnitzler, 1. 10. 1928}
\nopagebreak\mylabel{L03043v}
\rehead{ }\normalsize\beginnumbering\briefempfaengerindex{Schnitzler, Arthur@\textsc{Schnitzler, Arthur}!zzzSalten, Felix@\emph{von Felix Salten}!1928-10-011@{1. 10. 1928}|(be}
\toendnotes[C]{\smallbreak\pagebreak[2]}\Standort{DLA, G:Schnitzler, Arthur (Sammlung Heinrich Schnitzler).}
\physDesc{Widmung am Schmutztitel, 54 Zeichen
\newline{}Handschrift: schwarze Tinte, lateinische Kurrent}
\pstart
           \noindent{}\centering{}{\pb}\textcolor{gray}{\textbf{\textsc{\so{FELIX SALTEN}}}}\pend
           
\pstart
           \centering{}\textcolor{gray}{\textbf{\so{Geſammelte Werke}}}\pend
           
\pstart
           \centering{}\textcolor{gray}{\textbf{\so{in Einzelausgaben}}}\pend
           
\pstart
           Arthur Schnitzler\pend
           
\pstart
           herzlichst {\\[\baselineskip]}\spacefill\mbox{Felix Salten}\pend
           \leftskip=0em{}
\pstart
           Wien\oindex{Wien@\textbf{Wien}, \emph{A.ADM2}|pw}, 1. X. 28\pend
           {\vspace{1\baselineskip}}
\pstart
           \centering{}{\pb}\textcolor{gray}{\textbf{\textsc{\so{FELIX SALTEN}}}}\pend
           
\pstart
           \centering{}\textcolor{gray}{\textbf{\textbf{\so{Simſon}}\pwindex{Simson. Das Schicksal eines Erwaehlten@\emph{Simson. Das Schicksal eines Erwählten}|pw}}}\pend
           
\pstart
           \centering{}\textcolor{gray}{\textbf{\so{Das Schickſal eines Erwählten}}}\pend
           
\pstart
           \centering{}\textcolor{gray}{\textbf{\textsc{\so{ROMAN}}}}\pend
           {\vspace{1\baselineskip}}
\pstart
           \centering{}\textcolor{gray}{\textbf{\so{1928}}}\pend
           
\pstart
           \centering{}\textcolor{gray}{\textbf{\textsc{\so{PAUL ZSOLNAY VERLAG}\orgindex{Paul Zsolnay Verlag@Paul Zsolnay Verlag|pw}}}}\pend
           
\pstart
           \centering{}\textcolor{gray}{\textbf{\so{BERLIN}\oindex{Berlin@\textbf{Berlin}, \emph{P.PPLC}|pw}{ }\so{/}{ }\so{WIEN}\oindex{Wien@\textbf{Wien}, \emph{A.ADM2}|pw}{ }\so{/}{ }\so{LEIPZIG}\oindex{Leipzig@\textbf{Leipzig}, \emph{P.PPLA3}|pw}}}\pend
           \selectlanguage{ngerman}\endnumbering\briefempfaengerindex{Schnitzler, Arthur@\textsc{Schnitzler, Arthur}!zzzSalten, Felix@\emph{von Felix Salten}!1928-10-011@{1. 10. 1928}|)be}\mylabel{L03043h}  \normalsize

\doendnotes{C}
\bigskip
\vfill

\clearpage

\footnotesize

\lohead{\textsc{register}}

% Definiere theindex-Environment komplett neu ohne reledmac
\makeatletter
\renewenvironment{theindex}{%
  \section*{\indexname}%
  \setlength{\parindent}{0pt}%
  \setlength{\parskip}{0pt plus 0.3pt}%
  \let\item\@idxitem
}{%
  \clearpage
}
\makeatother

\IfFileExists{\jobname-pw.ind}{\input{\jobname-pw.ind}}{}

\end{document}

      