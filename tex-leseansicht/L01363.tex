%% latex-leseansicht-vorspann.tex
%% Vorspann für die Leseansicht.
%% Lädt die gemeinsame Datei latex-vorspann.tex mit nicht gesetztem Schalter.

\newif\ifkorrekturansicht
\korrekturansichtfalse

\input{../tex-inputs/latex-vorspann}


         
         \newcommand{\erwaehntePersonen}{Personen: Michael Georg Conrad, Marie Knorr-Schmidt}
         \newcommand{\erwaehnteInstitutionen}{}
         \newcommand{\erwaehnteOrte}{Orte: Edmund-Weiß-Gasse, München, Steinsdorfstraße, Wien}
         \newcommand{\erwaehnteWerke}{Werke: Evoë! Ein Schritt zur Lichtung des Seelenlebens}
               \section[Arthur Schnitzler an Michael Georg Conrad, 24. 1. 1904]{ Arthur Schnitzler an Michael Georg Conrad, 24. 1. 1904}\nopagebreak\mylabel{v}\rehead{ }\begin{ledgroupsized}[t]{13cm}\normalsize\beginnumbering \toendnotes[C]{\smallbreak\pagebreak[2]} \Standort{München, Monacensia, Schnitzler, Arthur A I/2.}
\physDesc{Kartenbrief
\newline{}Handschrift: schwarze Tinte, deutsche Kurrent\newline{}Versand: 1) Stempel: »\nobreak{}Wien, 4–5 N\nobreak{}«.   2) Stempel: »\nobreak{}\oindex{Muenchen@\textbf{München}|pwk}München 2. B.Z., 25. Jan. 04, V. 7–8\nobreak{}«. 
\newline{}Conrad: mit Bleistift beschriftet: »Artur
                                 Schnitzler« }\toendnotes[C]{\smallbreak}\pstart{}{\pb}Herrn\pend{}\pstart{}\textsc{M. G. Conrad}\pend{}\pstart{}\textsc{München\oindex{Muenchen@\textbf{München}|pw}}\pend{}\pstart{}Steinsdorfſtraße 7\oindex{Steinsdorfstrasse@\textbf{Steinsdorfstraße}|pw}.\pend{}{\bigskip}\pstart
           \raggedleft{}{\pb}\textsc{XVIII. Spöttelgasse 7}\oindex{Edmund-Weiss-Gasse@\textbf{Edmund-Weiß-Gasse}|pw}{\\}Wien\oindex{Wien@\textbf{Wien}|pw}{ }24./1 0\textcolor{gray}{4}\pend
           \pstart
           lieber Herr Conrad, ich habe das Buch\pwindex{Knorr-Schmidt, Marie 1861-08-20 – 1942?@\textsc{Knorr-Schmidt, Marie} (1861-08-20 – 1942?), \emph{Schriftstellerin}!Evoe Ein Schritt zur Lichtung des Seelenlebens1903@\strich\emph{Evoë{\rufezeichen} Ein Schritt zur Lichtung des Seelenlebens} {[}1903{]}|pwv} der Frau \textsc{Knorr Schmidt}\pwindex{Knorr-Schmidt, Marie 1861-08-20 – 1942?@\textsc{Knorr-Schmidt, Marie} (1861-08-20 – 1942?), \emph{Schriftstellerin}|pw} erhalten u mancherlei Blicke hinein gethan – der Arzt in mir regt ſich und
               meint: man dürfe über dergleichen \textsc{Imanationen} nicht
               urtheilen, ehe man mehr über das betreffende Individium erfährt oder es wirklich als
               ganzes kennen lernt. Als »Fall« wäre die Sache vielleicht intereſſant.\pend
           \pstart
           Im übrigen: Können Sie ſich talentloſe \label{K_L01363-1v}\edtext{Geiſter}{\lemma{\textnormal{\emph{Geiſter}}}\Cendnote{\textnormal{Beim folgenden Brief, der
                  als »Aus der Korrespondenz mit unbekannten Autoren« veröffentlicht wurde, könnte
                  es sich um Schnitzler\pwindex{Schnitzler, Arthur 15.05.1862 – 21.10.1931@\textsc{Schnitzler, Arthur} (15.05.1862 – 21.10.1931), \emph{Schriftsteller, Mediziner}|pwk}s Schreiben an Marie Knorr-Schmidt\pwindex{Knorr-Schmidt, Marie 1861-08-20 – 1942?@\textsc{Knorr-Schmidt, Marie} (1861-08-20 – 1942?), \emph{Schriftstellerin}|pwk} handeln: »Sehr
                        geehrte gnädige Frau!{ / }Ich bin keineswegs befugt, die Frage zu entscheiden, ob es Geister gibt oder
                        nicht. Nicht leugnen will ich indes, daß ich mich einer gewissen vorgefaßten
                        Meinung, wenn auch keiner unbegründeten, schuldig weiß. Sollte es aber
                        Geister geben, so flößen mir diejenigen, welche Ihnen Ihre Gedichte
                        diktieren, durch ihren auffallenden Mangel an Geschmack und Talent nicht
                        genügend Interesse ein, um der Erforschung des Problems vorläufig
                        näherzutreten.{ / }Mit vorzüglicher Hochachtung{ / }Arthur Schnitzler«. (\emph{Aus der
                        Korrespondenz mit unbekannten Autoren}. In: \emph{Literatur und Kritik}, Jg. 12, März 1967,
                     S. 87.)}}}\label{K_L01363-1h} vorſtellen? Oder Geſpenſter, die abgeſtandene Witze
               machen? We{\geminationn} ich mich entſchließen ſollte, an einen Geiſt
               zu glauben, ſo müßte er ſich ſchon die Mühe machen, ein Genie zu ſein. –\pend
           \pstart
           Herzliche Grüße.\hspace*{1.5em}Ihr{\\[\baselineskip]}\spacefill\mbox{ArthurSchnitzler}\pend
           \leftskip=0em{}
         
         \endnumbering\mylabel{h}\end{ledgroupsized}  \newcommand{\dateiname}{L01363}\newcommand{\titel}{Arthur Schnitzler an Michael Georg Conrad, 24. 1. 1904}\newcommand{\editorInnen}{Martin Anton Müller und Gerd-Hermann Susen}%% latex-leseansicht-abspann.tex
%% Abspann für die Leseansicht.
%% Der Schalter \ifkorrekturansicht ist bereits durch den Vorspann gesetzt.

%% latex-abspann.tex
%% Gemeinsamer Abspann für Korrekturansicht und Leseansicht.
%% Setzt den Schalter \ifkorrekturansicht voraus (gesetzt in den
%% einbindenden Dateien latex-korrekturansicht-abspann.tex bzw.
%% latex-leseansicht-abspann.tex).
%% ---------------------------------------------------------------

\normalsize

% Das esempio-Environment wird nur in der Leseansicht benötigt
\ifkorrekturansicht\else
\newenvironment{esempio}[3]%
{
    \vspace{1.5ex}
    \rlap{\underline{#1}}
    \par
    \setlength{\parindent}{0cm}
    \nopagebreak
    \leftskip=#2cm
    \rightskip=#3cm
}
{
    \par
}
\fi

\doendnotes{C}
\bigskip
\vfill

\clearpage

\footnotesize

\ifkorrekturansicht
  \lohead{\textsc{register}}
\fi

% theindex-Environment neu definieren ohne reledmac
\makeatletter
\renewenvironment{theindex}{%
  \ifkorrekturansicht
    \section*{\indexname}%
  \else
    \subsubsection*{Index der erwähnten Entitäten}%
  \fi
  \setlength{\parindent}{0pt}%
  \setlength{\parskip}{0pt plus 0.3pt}%
  \let\item\@idxitem
}{%
  \ifkorrekturansicht\clearpage\fi
}
\makeatother

\IfFileExists{\jobname-pw.ind}{\input{\jobname-pw.ind}}{}

% Quellenangabe nur in der Leseansicht
\ifkorrekturansicht\else
% Fallback-Definitionen, falls die .tex-Datei \titel etc. nicht gesetzt hat
\providecommand{\titel}{}
\providecommand{\editorInnen}{}
\providecommand{\dateiname}{\jobname}

\vspace{3cm}

\vfill

\footnotesize
\textsc{Quelle}: \titel. Herausgegeben von {\editorInnen}. In: \emph{Arthur Schnitzler: Briefwechsel mit Autorinnen und Autoren}.
 Digitale Edition, https://schnitzler-briefe.acdh.oeaw.ac.at/{\dateiname}.html (Stand \today)
\fi

\end{document}


      