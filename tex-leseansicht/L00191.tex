%% latex-leseansicht-vorspann.tex
%% Vorspann für die Leseansicht.
%% Lädt die gemeinsame Datei latex-vorspann.tex mit nicht gesetztem Schalter.

\newif\ifkorrekturansicht
\korrekturansichtfalse

\input{../tex-inputs/latex-vorspann}


         
         \renewcommand{\erwaehntePersonen}{Personen: Richard Beer-Hofmann, Felix Dörmann, Alexander Engel, Otto Erich Hartleben, Karl Kraus, Alexander Landesberg, Detlev von Liliencron, Anton Lindner, Gilbert Otto Neumann-Hofer, Paul Schettler, Josef Schmid-Braunfels, Richard Specht, Leopold Weiß}
         \renewcommand{\erwaehnteInstitutionen}{Institutionen: Magazin für die Literatur des Auslandes, Neue litterarische Blätter, Wiener Tagblatt}
         \renewcommand{\erwaehnteOrte}{Orte: Berlin, Bremen, Café Griensteidl, Mahlerstraße, Wien}
         \renewcommand{\erwaehnteWerke}{Werke: Anatol, Arthur Schnitzler: Anatol, Gedichte, Magazin für die Literatur des Auslandes, Neue litterarische Blätter, Sensationen, Wiener Lyriker}
               \section[Karl Kraus an Arthur Schnitzler, 19. 3. 1893]{ Karl Kraus an Arthur Schnitzler, 19. 3. 1893}\nopagebreak\mylabel{v}\rehead{ }\begin{ledgroupsized}[t]{13cm}\normalsize\beginnumbering \toendnotes[C]{\smallbreak\pagebreak[2]} \Standort{CUL, Schnitzler, B 55.}
\physDesc{Brief, 1 Blatt, 4 Seiten, 4248 Zeichen
\newline{}Handschrift: schwarze Tinte, deutsche Kurrent}\buchAbdrucke{\weitereDrucke{\emph{Karl Kraus und Arthur Schnitzler. Eine Dokumentation.} Hg. Reinhard Urbach. In: \emph{Literatur und Kritik}, Bd. 49, Oktober 1970, S. 516–517.} }\toendnotes[C]{\smallbreak}\pstart
           \noindent{}{\pb}\textcolor{gray}{\textbf{Karl Kraus}}\hfill \textcolor{gray}{\textbf{Wien\oindex{Wien@\textbf{Wien}|pw}, am}}{ }19. 3. \textcolor{gray}{\textbf{189}}3\pend
           \pstart
           \textcolor{gray}{\textbf{Wien\oindex{Wien@\textbf{Wien}|pw}}}\pend
           \pstart
           \textcolor{gray}{\textbf{I., Maximilianstrasse 13\oindex{Mahlerstrasse@\textbf{Mahlerstraße}|pw}.}}\pend
           \pstart{}Sehr verehrter Herr Doctor!\pend\pstart
           Leider ſehe ich mich genöthigt, mich in einer Angelegenheit an Sie zu wenden, mit der
               Sie gewiss nicht gerne belästigt werden. Aber, da ich \uline{Sie}, lieber Herr, ſtets hochgeſchätzt und geachtet habe, ſo will ich \introOben{}mich\introOben{} auch Ihnen \strikeout{mich} ganz offenbaren.
               Sie können ermeſſen, wie ſehr es mich kränkten muſste, daſs Sie mir vorgeſtern im Grienſteidl\oindex{Cafe Griensteidl@\textbf{Café Griensteidl}|pw}, nachdem wir uns 4 Wochen nicht geſehen
               hatten, mit ſichtlicher Kälte und – ich möchte ſagen – »ceremonieller« Höflichkeit
               begegneten.\pend
           \pstart
           Und weil es mir nun ganz enorm furchtbar und rieſig daran liegt, daſs \uline{Sie}, liebſter Herr D\textsuperscript{r.} Schnitzler, von mir \uline{gut} denken oder ſo
               denken, wie über mich zu denken iſt, ſo will ich \uline{Ihnen}, damit \uline{Sie}{ }ſich \introOben{}nicht\introOben{} durch nichtige
               Redereien beſtimmen laſſen, mir böſe zu ſein und mich quasi für einen »Ausſätzigen«
               anzuſehen, folgende Thatſachen mittheilen:\pend
           \pstart
           Meine in N\textsuperscript{o} 8 des »Magazin\pwindex{?? Werk@Nicht ermittelte Verfasserinnen und Verfasser!Magazin fuer die Literatur des Auslandes1832 – 1915@\emph{Magazin für die Literatur des Auslandes} {[}1832 – 1915{]}|pw}« enthaltene »Dörmann\pwindex{Doermann, Felix 29.05.1870 – 26.10.1928@\textsc{Dörmann, Felix} (29.05.1870 – 26.10.1928), \emph{Schriftsteller}|pw}–Specht\pwindex{Specht, Richard 07.12.1870 – 18.03.1932@\textsc{Specht, Richard} (07.12.1870 – 18.03.1932), \emph{Schriftsteller, Journalist, Kritiker}|pw}«-Recenſion\pwindex{Kraus, Karl 28.04.1874 – 12.06.1936@\textsc{Kraus, Karl} (28.04.1874 – 12.06.1936), \emph{Schriftsteller, Publizist}!Wiener Lyriker25. 02. 1893@\strich\emph{Wiener Lyriker} {[}25. 02. 1893{]}|pwv} iſt \uline{in dieſer Form}
               bereits vor Monaten entſtanden. Herr Richard
                  Specht\pwindex{Specht, Richard 07.12.1870 – 18.03.1932@\textsc{Specht, Richard} (07.12.1870 – 18.03.1932), \emph{Schriftsteller, Journalist, Kritiker}|pw}{ }ſandte mir im November od.
                  December, (ich weiß nicht genau, wann) ſeine Gedichte\pwindex{Specht, Richard 07.12.1870 – 18.03.1932@\textsc{Specht, Richard} (07.12.1870 – 18.03.1932), \emph{Schriftsteller, Journalist, Kritiker}!Gedichte1893@\strich\emph{Gedichte} {[}1893{]}|pw}. Ich ſchrieb ſofort (nach 2–3 Tagen) eine Kritik, \uline{diese Kritik\pwindex{Kraus, Karl 28.04.1874 – 12.06.1936@\textsc{Kraus, Karl} (28.04.1874 – 12.06.1936), \emph{Schriftsteller, Publizist}!Wiener Lyriker25. 02. 1893@\strich\emph{Wiener Lyriker} {[}25. 02. 1893{]}|pwv}} (mit Dörmann\pwindex{Doermann, Felix 29.05.1870 – 26.10.1928@\textsc{Dörmann, Felix} (29.05.1870 – 26.10.1928), \emph{Schriftsteller}|pw} zuſammen beſprach ich ihn;
                  F. D.\pwindex{Doermann, Felix 29.05.1870 – 26.10.1928@\textsc{Dörmann, Felix} (29.05.1870 – 26.10.1928), \emph{Schriftsteller}|pw} »Senſationen\pwindex{Doermann, Felix 29.05.1870 – 26.10.1928@\textsc{Dörmann, Felix} (29.05.1870 – 26.10.1928), \emph{Schriftsteller}!Sensationen1892@\strich\emph{Sensationen} {[}1892{]}|pw}« ſandte mir gerade vorher L.
                  Weiß\pwindex{Weiss, Leopold *~21.11.1853@\textsc{Weiß, Leopold} (*~21.11.1853), \emph{Verleger, Buchhändler}|pw} zur Recenſion). Dörmann\pwindex{Doermann, Felix 29.05.1870 – 26.10.1928@\textsc{Dörmann, Felix} (29.05.1870 – 26.10.1928), \emph{Schriftsteller}|pw}{ }\uline{kannte ich damals} noch nicht; den lernte ich erſt
               ſpäter durch Vermittelung D\textsuperscript{r.} Beer-Hofmann\pwindex{Beer-Hofmann, Richard 1866-07-11 – 1945-09-26@\textsc{Beer-Hofmann, Richard} (1866-07-11 – 1945-09-26), \emph{Schriftsteller}|pw}’s perſönlich kennen.\pend
           \pstart
           Die Kritik gab ich dem »Tagblatt\orgindex{Wiener Tagblatt@Wiener Tagblatt|pw}«. Alexander Landesberg\pwindex{Landesberg, Alexander 15.07.1848 – 14.6.1916@\textsc{Landesberg, Alexander} (15.07.1848 – 14.6.1916), \emph{Schriftsteller, Journalist}|pw} behielt ſie volle 2 Monate
               bei ſich, ohne ſich zu entſcheiden. Endlich gieng ich hin. Er erklärte, dieſer Sache
               keinen ſo breiten Raum gewähren zu können. Er ſuchte sie heraus, fand ſie nach langem
               Suchen und gab ſie mir – {\pb}Nun ſchickte ich
               die Arbeit \introOben{}(\uline{Dieſelbe!! In dieſer
                     Form!!})\introOben{} – auf’s Geratewohl – an’s »Magazin\orgindex{Magazin fuer die Literatur des Auslandes@Magazin für die Literatur des Auslandes|pw}«. Nach 8 Tagen ſchrieb mir Paul
                     Sch\strikeout{l}ettler\pwindex{Schettler, Paul 11.11.1864 – 01.02.1948@\textsc{Schettler, Paul} (11.11.1864 – 01.02.1948), \emph{Redakteur}|pw} für die Redaction: »Ihre
               Besprechung der beiden Wien\oindex{Wien@\textbf{Wien}|pw}er ›Neurotiker‹
               acceptiert das ›Magazin\orgindex{Magazin fuer die Literatur des Auslandes@Magazin für die Literatur des Auslandes|pw}‹ mit Vergnügen.«\pend
           \pstart
           Als ich nach Berlin\oindex{Berlin@\textbf{Berlin}|pw} kam, machte man mich auf die
               bereits erſchienene Kritik\pwindex{Kraus, Karl 28.04.1874 – 12.06.1936@\textsc{Kraus, Karl} (28.04.1874 – 12.06.1936), \emph{Schriftsteller, Publizist}!Wiener Lyriker25. 02. 1893@\strich\emph{Wiener Lyriker} {[}25. 02. 1893{]}|pwv}
               aufmerkſam. Ich war dem Tgbl.\orgindex{Wiener Tagblatt@Wiener Tagblatt|pw} vom Herzen dankbar,
               daſs es die Kritik\pwindex{Kraus, Karl 28.04.1874 – 12.06.1936@\textsc{Kraus, Karl} (28.04.1874 – 12.06.1936), \emph{Schriftsteller, Publizist}!Wiener Lyriker25. 02. 1893@\strich\emph{Wiener Lyriker} {[}25. 02. 1893{]}|pwv}{ }retournierte. Denn durch dieſe Kritik\pwindex{Kraus, Karl 28.04.1874 – 12.06.1936@\textsc{Kraus, Karl} (28.04.1874 – 12.06.1936), \emph{Schriftsteller, Publizist}!Wiener Lyriker25. 02. 1893@\strich\emph{Wiener Lyriker} {[}25. 02. 1893{]}|pwv}, die Otto Neumann-Hofer\pwindex{Neumann-Hofer, Gilbert Otto 04.02.1857 – 14.04.1941@\textsc{Neumann-Hofer, Gilbert Otto} (04.02.1857 – 14.04.1941), \emph{Kritiker, Theaterleiter}|pw} und die andern Herren \introOben{}(\uline{auch Baron Liliencron\pwindex{Liliencron, Detlev von 03.06.1844 – 22.07.1909@\textsc{Liliencron, Detlev von} (03.06.1844 – 22.07.1909)|pw}})\introOben{} außerordentlich lobten, ſchuf ich mir feſte Position im »Magazin\orgindex{Magazin fuer die Literatur des Auslandes@Magazin für die Literatur des Auslandes|pw}«. Die Sache wurde ſofort honoriert und
               weitere Artikel (über Wien\oindex{Wien@\textbf{Wien}|pw}er Litteratur,
               »Decadence« etc) – ſozuſagen – »beſtellt«.\pend
           \pstart
           Ich glaube, es ſind ſchon 4 Monate her, daſs mir Herr Specht\pwindex{Specht, Richard 07.12.1870 – 18.03.1932@\textsc{Specht, Richard} (07.12.1870 – 18.03.1932), \emph{Schriftsteller, Journalist, Kritiker}|pw}{ }ſein Büchlein\pwindex{Specht, Richard 07.12.1870 – 18.03.1932@\textsc{Specht, Richard} (07.12.1870 – 18.03.1932), \emph{Schriftsteller, Journalist, Kritiker}!Gedichte1893@\strich\emph{Gedichte} {[}1893{]}|pwv}{ }ſchickte, circa{ }\uline{4 Monate} alſo ſeit Abfaſſung des vor 2–3 Wochen
               erſchienenen Artikels\pwindex{Kraus, Karl 28.04.1874 – 12.06.1936@\textsc{Kraus, Karl} (28.04.1874 – 12.06.1936), \emph{Schriftsteller, Publizist}!Wiener Lyriker25. 02. 1893@\strich\emph{Wiener Lyriker} {[}25. 02. 1893{]}|pwv}!!
               Deshalb iſt entſtanden\strikeout{,}{ }\uline{lange, lange}, bevor ich Herrn Specht\pwindex{Specht, Richard 07.12.1870 – 18.03.1932@\textsc{Specht, Richard} (07.12.1870 – 18.03.1932), \emph{Schriftsteller, Journalist, Kritiker}|pw} den wirklich mit Müh und Not beſchafften
               »Sündentraum«beleg ſchickte und da\substVorne{}\textsuperscript{bei}\substDazwischen{}zu\substHinten{} jenen ominösen, aber durch und durch freundlichen Brief ſchrieb, der den
               harmloſen Witz (»Dör-mannbar\pwindex{Doermann, Felix 29.05.1870 – 26.10.1928@\textsc{Dörmann, Felix} (29.05.1870 – 26.10.1928), \emph{Schriftsteller}|pwv}«
               enthielt) ſie iſt entſtanden, \uline{lange} bevor ich Herrn
                  Dörmann\pwindex{Doermann, Felix 29.05.1870 – 26.10.1928@\textsc{Dörmann, Felix} (29.05.1870 – 26.10.1928), \emph{Schriftsteller}|pw} perſönlich kennen lernte, ſo daſs
               alſo weder von einem perſönlichen Gefühle {\pb}Herrn Specht\pwindex{Specht, Richard 07.12.1870 – 18.03.1932@\textsc{Specht, Richard} (07.12.1870 – 18.03.1932), \emph{Schriftsteller, Journalist, Kritiker}|pw} gegenüber noch von einer
               »Beeinfluſſung durch Dörmann\pwindex{Doermann, Felix 29.05.1870 – 26.10.1928@\textsc{Dörmann, Felix} (29.05.1870 – 26.10.1928), \emph{Schriftsteller}|pw}« die Rede ſein
               kann!\pend
           \pstart
           \uuline{Das beſchwöre ich}!\pend
           \pstart
           \uline{Alexander Landesberg}\pwindex{Landesberg, Alexander 15.07.1848 – 14.6.1916@\textsc{Landesberg, Alexander} (15.07.1848 – 14.6.1916), \emph{Schriftsteller, Journalist}|pw}, Alexander Engel\pwindex{Engel, Alexander 10.04.1868 – 17.11.1940@\textsc{Engel, Alexander} (10.04.1868 – 17.11.1940), \emph{Schriftsteller, Journalist}|pw}, Anton Lindner\pwindex{Lindner, Anton 14.12.1874 – 30.12.1928@\textsc{Lindner, Anton} (14.12.1874 – 30.12.1928), \emph{Schriftsteller}|pw} etc etc andere Freunde ſind Zeugen!!\pend
           \pstart
           Die Kritik\pwindex{Kraus, Karl 28.04.1874 – 12.06.1936@\textsc{Kraus, Karl} (28.04.1874 – 12.06.1936), \emph{Schriftsteller, Publizist}!Wiener Lyriker25. 02. 1893@\strich\emph{Wiener Lyriker} {[}25. 02. 1893{]}|pwv} (\uline{ganz} in der jetzigen Geſtalt!!) iſt – vor Monaten –
               aus einer ehrlichen, vollſten, ureigenſten Überzeugung heraus entſtanden. Nichts
               liegt mir ferner als Unehrlichkeit, als »Rachegefühl« und jüdiſches
               Tagſschreiberthum. Man hüte ſich, mich in dieſer niederträchtigen Weise zu
               verleumden!!\pend
           \pstart
           Ich haſſe und haſste diese falſche, erlogene »Decadence«, die artig mit ſich ſelbst
               coquettiert; ich bekämpfe und werde immer bekämpfen: die posierte, krankhafte,
               onanierte Poeſie! {\pb}Und \uline{dieſer Haſs} war das Kritikmotiv!\pend
           \pstart
           \strikeout{Glauben} Sie werden vielleicht, verehrter Herr D\textsuperscript{r.}, ſich denken: Aha, wer ſich \uline{ſo} vertheidigt, \uline{muſs}{ }ſich wohl verteidigen!? \strikeout{und} Nein, ſeien Sie versichert, die ganze Litanei hab ich auch nur \uline{Ihnen}\footnote{\noindent{}Auch dem verehrten Herrn D\textsuperscript{r.}{ }B-Hofmann\pwindex{Beer-Hofmann, Richard 1866-07-11 – 1945-09-26@\textsc{Beer-Hofmann, Richard} (1866-07-11 – 1945-09-26), \emph{Schriftsteller}|pw} hätte ich’s geſagt!} hergeſagt, weil mir an \uline{Ihrer} Meinung \strikeout{etw} viel liegt. Den andern gegenüber hab’ ich es
               Gottſseidank nicht nöthig, mich zu vertheidigen!\pend
           \pstart
           Wenn ich Sie beläſtigt habe, verzeihen Sie.\pend
           \pstart
           Otto Erich Hartleben\pwindex{Hartleben, Otto Erich 03.06.1864 – 11.02.1905@\textsc{Hartleben, Otto Erich} (03.06.1864 – 11.02.1905), \emph{Schriftsteller}|pw} grüßt Sie durch mich.\pend
           \pstart
           Für »Neue litt. Bl\orgindex{Neue litterarische Blaetter@Neue litterarische Blätter|pw}« \introOben{}(Bremen\oindex{Bremen@\textbf{Bremen}|pw})\introOben{} wäre ich mit \strikeout{mit}{ }Anatol\pwindex{Schnitzler, Arthur 15.05.1862 – 21.10.1931@\textsc{Schnitzler, Arthur} (15.05.1862 – 21.10.1931), \emph{Schriftsteller, Mediziner}!Anatol1892-10-29@\strich\emph{Anatol} {[}1892-10-29{]}|pw} zu ſpät gekommen, da das dort in \label{K_L00191-1v}\edtext{Einläufe}{\lemma{\textnormal{\emph{Einläufe}}}\Cendnote{\textnormal{\emph{Neue litterarische Blätter}\pwindex{Neue litterarische Blaetter1893@\emph{Neue litterarische Blätter} {[}1893{]}|pwk}, Jg. 1, H. 5/6,
                        1. 3. 1893, S. 66.}}}\label{K_L00191-1h} verzeichnete Buch\pwindex{Schnitzler, Arthur 15.05.1862 – 21.10.1931@\textsc{Schnitzler, Arthur} (15.05.1862 – 21.10.1931), \emph{Schriftsteller, Mediziner}!Anatol1892-10-29@\strich\emph{Anatol} {[}1892-10-29{]}|pwv} bereits an einen andern Mitarbeiter\pwindex{Schmid-Braunfels, Josef 29.11.1871 – 22.11.1911@\textsc{Schmid-Braunfels, Josef} (29.11.1871 – 22.11.1911), \emph{Schriftsteller, Veterinärmediziner}|pwv} zur Recension\pwindex{Schmid-Braunfels, Josef 29.11.1871 – 22.11.1911@\textsc{Schmid-Braunfels, Josef} (29.11.1871 – 22.11.1911), \emph{Schriftsteller, Veterinärmediziner}!Arthur Schnitzler: Anatol01. 04. 1893@\strich\emph{Arthur Schnitzler: Anatol} {[}01. 04. 1893{]}|pwv} abgegeben wurde.\pend
           \pstart
           Sonſt ſtehe ich Ihnen mit aufrichtigem Vergnügen ſtets zu Dienſten u bin (Sie noch
                  \uline{um paar Zeilen bittend}!) Ihr \uline{Sie vollkommen hochachtender}\pend
           \pstart
           Herzlichſt grüſſend{\\[\baselineskip]}\spacefill\mbox{Karl Kraus}\pend
           \leftskip=0em{}
         
         \endnumbering\mylabel{h}\end{ledgroupsized}  \newcommand{\dateiname}{L00191}\newcommand{\titel}{Karl Kraus an Arthur Schnitzler, 19. 3. 1893}\newcommand{\editorInnen}{Martin Anton Müller und Gerd-Hermann Susen}%% latex-leseansicht-abspann.tex
%% Abspann für die Leseansicht.
%% Der Schalter \ifkorrekturansicht ist bereits durch den Vorspann gesetzt.

%% latex-abspann.tex
%% Gemeinsamer Abspann für Korrekturansicht und Leseansicht.
%% Setzt den Schalter \ifkorrekturansicht voraus (gesetzt in den
%% einbindenden Dateien latex-korrekturansicht-abspann.tex bzw.
%% latex-leseansicht-abspann.tex).
%% ---------------------------------------------------------------

\normalsize

% Das esempio-Environment wird nur in der Leseansicht benötigt
\ifkorrekturansicht\else
\newenvironment{esempio}[3]%
{
    \vspace{1.5ex}
    \rlap{\underline{#1}}
    \par
    \setlength{\parindent}{0cm}
    \nopagebreak
    \leftskip=#2cm
    \rightskip=#3cm
}
{
    \par
}
\fi

\doendnotes{C}
\bigskip
\vfill

\clearpage

\footnotesize

\ifkorrekturansicht
  \lohead{\textsc{register}}
\fi

% theindex-Environment neu definieren ohne reledmac
\makeatletter
\renewenvironment{theindex}{%
  \ifkorrekturansicht
    \section*{\indexname}%
  \else
    \subsubsection*{Index der erwähnten Entitäten}%
  \fi
  \setlength{\parindent}{0pt}%
  \setlength{\parskip}{0pt plus 0.3pt}%
  \let\item\@idxitem
}{%
  \ifkorrekturansicht\clearpage\fi
}
\makeatother

\IfFileExists{\jobname-pw.ind}{\input{\jobname-pw.ind}}{}

% Quellenangabe nur in der Leseansicht
\ifkorrekturansicht\else
% Fallback-Definitionen, falls die .tex-Datei \titel etc. nicht gesetzt hat
\providecommand{\titel}{}
\providecommand{\editorInnen}{}
\providecommand{\dateiname}{\jobname}

\vspace{3cm}

\vfill

\footnotesize
\textsc{Quelle}: \titel. Herausgegeben von {\editorInnen}. In: \emph{Arthur Schnitzler: Briefwechsel mit Autorinnen und Autoren}.
 Digitale Edition, https://schnitzler-briefe.acdh.oeaw.ac.at/{\dateiname}.html (Stand \today)
\fi

\end{document}


      