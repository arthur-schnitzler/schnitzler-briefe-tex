%% latex-korrekturansicht-vorspann.tex
%% Vorspann für die Korrekturansicht.
%% Lädt die gemeinsame Datei latex-vorspann.tex mit gesetztem Schalter.

\newif\ifkorrekturansicht
\korrekturansichttrue

\input{../tex-inputs/latex-vorspann}


\section[Karl Kraus an Arthur Schnitzler, 19. 3. 1893]{L00191 Karl Kraus an Arthur Schnitzler, 19. 3. 1893}
\nopagebreak\mylabel{L00191v}
\rehead{ }\normalsize\beginnumbering\briefempfaengerindex{Schnitzler, Arthur@\textsc{Schnitzler, Arthur}!zzzKraus, Karl@\emph{von Karl Kraus}!1893-03-191@{19. 3. 1893}|(be}
\toendnotes[C]{\smallbreak\pagebreak[2]}\Standort{CUL, Schnitzler, B 55.}
\physDesc{Brief, 1 Blatt, 4 Seiten, 4248 Zeichen
\newline{}Handschrift: schwarze Tinte, deutsche Kurrent}
\buchAbdrucke{\weitereDrucke{\emph{Literatur und Kritik}, Bd. 49, Oktober 1970, S. 516–517.} }\toendnotes[C]{\smallbreak}
\pstart
           {\pb}\textcolor{gray}{\textbf{Karl Kraus}}\hfill \textcolor{gray}{\textbf{Wien\oindex{Wien@\textbf{Wien}, \emph{A.ADM2}|pw}
                        , am
                     }}{ }
                        19. 3. 
                        \textcolor{gray}{\textbf{189}}
                        3
                     \pend
           
\pstart
           \textcolor{gray}{\textbf{Wien\oindex{Wien@\textbf{Wien}, \emph{A.ADM2}|pw}}}\pend
           
\pstart
           \textcolor{gray}{\textbf{I., Maximilianstrasse 13\oindex{Mahlerstrasse@\textbf{Mahlerstraße}, \emph{Straße (K.STR)}|pw}
                     .
                  }}\pend
           
\pstart{}Sehr verehrter Herr Doctor!\pend\vspace{0.5em}
\pstart
           
               Leider ſehe ich mich genöthigt, mich in einer Angelegenheit an Sie zu wenden, mit der
               Sie gewiss nicht gerne belästigt werden. Aber, da ich 
               \uline{Sie}
               , lieber Herr, ſtets hochgeſchätzt und geachtet habe, ſo will ich 
               \introOben{}mich\introOben{}
                auch Ihnen 
               \strikeout{mich}
                ganz offenbaren.
               Sie können ermeſſen, wie ſehr es mich kränkten muſste, daſs Sie mir vorgeſtern im 
               Grienſteidl\oindex{Cafe Griensteidl@\textbf{Café Griensteidl}, \emph{Kaffeehaus (K.KAF)}|pw}
               , nachdem wir uns 4 Wochen nicht geſehen
               hatten, mit ſichtlicher Kälte und – ich möchte ſagen – »ceremonieller« Höflichkeit
               begegneten.
            \pend
           
\pstart
           
               Und weil es mir nun ganz enorm furchtbar und rieſig daran liegt, daſs 
               \uline{Sie}
               , liebſter Herr D
               \textsuperscript{r.}
                Schnitzler, von mir 
               \uline{gut}
                denken oder ſo
               denken, wie über mich zu denken iſt, ſo will ich 
               \uline{Ihnen}
               , damit 
               \uline{Sie}{ }
               ſich 
               \introOben{}nicht\introOben{}
                durch nichtige
               Redereien beſtimmen laſſen, mir böſe zu ſein und mich quasi für einen »Ausſätzigen«
               anzuſehen, folgende Thatſachen mittheilen:
            \pend
           
\pstart
           
               Meine in N
               \textsuperscript{o}
                8 des »
               Magazin\pwindex{Magazin fuer die Literatur des Auslandes@\emph{Magazin für die Literatur des Auslandes}|pw}
               « enthaltene »
               Dörmann\pwindex{Doermann, Felix 29.05.1870 – 26.10.1928@\textsc{Dörmann, Felix} (29.05.1870 – 26.10.1928), \emph{Schriftsteller/Schriftstellerin}|pw}
                  –
                  Specht\pwindex{Specht, Richard 07.12.1870 – 18.03.1932@\textsc{Specht, Richard} (07.12.1870 – 18.03.1932), \emph{Schriftsteller/Schriftstellerin, Journalist/Journalistin, Kritiker/Kritikerin}|pw}
                  «-Recenſion
               \pwindex{Wiener Lyriker@\emph{Wiener Lyriker}|pwv}
                iſt 
               \uline{in dieſer Form}
               
               bereits vor Monaten entſtanden. Herr 
               Richard
                  Specht\pwindex{Specht, Richard 07.12.1870 – 18.03.1932@\textsc{Specht, Richard} (07.12.1870 – 18.03.1932), \emph{Schriftsteller/Schriftstellerin, Journalist/Journalistin, Kritiker/Kritikerin}|pw}{ }
               ſandte mir im 
               November
                od.
                  
               December
               , (ich weiß nicht genau, wann) ſeine 
               Gedichte\pwindex{Gedichte@\emph{Gedichte}|pw}
               . Ich ſchrieb ſofort (nach 2–3 Tagen) eine Kritik, 
               \uline{
                  diese 
                  Kritik\pwindex{Wiener Lyriker@\emph{Wiener Lyriker}|pwv}}
                (mit 
               Dörmann\pwindex{Doermann, Felix 29.05.1870 – 26.10.1928@\textsc{Dörmann, Felix} (29.05.1870 – 26.10.1928), \emph{Schriftsteller/Schriftstellerin}|pw}
                zuſammen beſprach ich ihn;
                  
               F. D.\pwindex{Doermann, Felix 29.05.1870 – 26.10.1928@\textsc{Dörmann, Felix} (29.05.1870 – 26.10.1928), \emph{Schriftsteller/Schriftstellerin}|pw}
                »
               Senſationen\pwindex{Sensationen@\emph{Sensationen}|pw}
               « ſandte mir gerade vorher 
               L.
                  Weiß\pwindex{Weiss, Leopold *~21.11.1853@\textsc{Weiß, Leopold} (*~21.11.1853), \emph{Verleger/Verlegerin, Buchhändler/Buchhändlerin}|pw}
                zur Recenſion). 
               Dörmann\pwindex{Doermann, Felix 29.05.1870 – 26.10.1928@\textsc{Dörmann, Felix} (29.05.1870 – 26.10.1928), \emph{Schriftsteller/Schriftstellerin}|pw}{ }\uline{kannte ich damals}
                noch nicht; den lernte ich erſt
               ſpäter durch Vermittelung D
               \textsuperscript{r.}Beer-Hofmann\pwindex{Beer-Hofmann, Richard 1866-07-11 – 1945-09-26@\textsc{Beer-Hofmann, Richard} (1866-07-11 – 1945-09-26), \emph{Schriftsteller/Schriftstellerin}|pw}
               ’s perſönlich kennen.
            \pend
           
\pstart
           
               Die Kritik gab ich dem »
               Tagblatt\orgindex{Wiener Tagblatt@Wiener Tagblatt|pw}
               «. 
               Alexander Landesberg\pwindex{Landesberg, Alexander 15.07.1848 – 14.6.1916@\textsc{Landesberg, Alexander} (15.07.1848 – 14.6.1916), \emph{Schriftsteller/Schriftstellerin, Journalist/Journalistin}|pw}
                behielt ſie volle 2 Monate
               bei ſich, ohne ſich zu entſcheiden. Endlich gieng ich hin. Er erklärte, dieſer Sache
               keinen ſo breiten Raum gewähren zu können. Er ſuchte sie heraus, fand ſie nach langem
               Suchen und gab ſie mir – 
               {\pb}
               Nun ſchickte ich
               die Arbeit 
               \introOben{}
                  (
                  \uline{Dieſelbe!! In dieſer
                     Form!!}
                  )
               \introOben{}
                – auf’s Geratewohl – an’s »
               Magazin\orgindex{Magazin fuer die Literatur des Auslandes@Magazin für die Literatur des Auslandes|pw}
               «. Nach 8 Tagen ſchrieb mir 
               
                  Paul
                     Sch
                  \strikeout{l}
                  ettler
               \pwindex{Schettler, Paul 11.11.1864 – 01.02.1948@\textsc{Schettler, Paul} (11.11.1864 – 01.02.1948), \emph{Redakteur/Redakteurin}|pw}
                für die Redaction: »Ihre
               Besprechung der beiden 
               Wien\oindex{Wien@\textbf{Wien}, \emph{A.ADM2}|pw}
               er ›Neurotiker‹
               acceptiert das ›
               Magazin\orgindex{Magazin fuer die Literatur des Auslandes@Magazin für die Literatur des Auslandes|pw}
               ‹ mit Vergnügen.«
            \pend
           
\pstart
           
               Als ich nach 
               Berlin\oindex{Berlin@\textbf{Berlin}, \emph{P.PPLC}|pw}
                kam, machte man mich auf die
               bereits erſchienene 
               Kritik\pwindex{Wiener Lyriker@\emph{Wiener Lyriker}|pwv}
               
               aufmerkſam. Ich war dem 
               Tgbl.\orgindex{Wiener Tagblatt@Wiener Tagblatt|pw}
                vom Herzen dankbar,
               daſs es die 
               Kritik\pwindex{Wiener Lyriker@\emph{Wiener Lyriker}|pwv}{ }
               retournierte. Denn durch dieſe 
               Kritik\pwindex{Wiener Lyriker@\emph{Wiener Lyriker}|pwv}
               , die 
               Otto Neumann-Hofer\pwindex{Neumann-Hofer, Gilbert Otto 04.02.1857 – 14.04.1941@\textsc{Neumann-Hofer, Gilbert Otto} (04.02.1857 – 14.04.1941), \emph{Kritiker/Kritikerin, Theaterleiter/Theaterleiterin}|pw}
                und die andern Herren 
               \introOben{}
                  (
                  \uline{
                     auch Baron 
                     Liliencron\pwindex{Liliencron, Detlev von 03.06.1844 – 22.07.1909@\textsc{Liliencron, Detlev von} (03.06.1844 – 22.07.1909), \emph{Schriftsteller/Schriftstellerin, Dichter/Dichterin, Dramatiker/Dramatikerin}|pw}}
                  )
               \introOben{}
                außerordentlich lobten, ſchuf ich mir feſte Position im »
               Magazin\orgindex{Magazin fuer die Literatur des Auslandes@Magazin für die Literatur des Auslandes|pw}
               «. Die Sache wurde ſofort honoriert und
               weitere Artikel (über 
               Wien\oindex{Wien@\textbf{Wien}, \emph{A.ADM2}|pw}
               er Litteratur,
               »Decadence« etc) – ſozuſagen – »beſtellt«.
            \pend
           
\pstart
           
               Ich glaube, es ſind ſchon 4 Monate her, daſs mir Herr 
               Specht\pwindex{Specht, Richard 07.12.1870 – 18.03.1932@\textsc{Specht, Richard} (07.12.1870 – 18.03.1932), \emph{Schriftsteller/Schriftstellerin, Journalist/Journalistin, Kritiker/Kritikerin}|pw}{ }
               ſein 
               Büchlein\pwindex{Gedichte@\emph{Gedichte}|pwv}{ }
               ſchickte, circa
               { }\uline{4 Monate}
                alſo ſeit Abfaſſung des vor 2–3 Wochen
               erſchienenen 
               Artikels\pwindex{Wiener Lyriker@\emph{Wiener Lyriker}|pwv}
               !!
               Deshalb iſt entſtanden
               \strikeout{,}{ }\uline{lange, lange}
               , bevor ich Herrn 
               Specht\pwindex{Specht, Richard 07.12.1870 – 18.03.1932@\textsc{Specht, Richard} (07.12.1870 – 18.03.1932), \emph{Schriftsteller/Schriftstellerin, Journalist/Journalistin, Kritiker/Kritikerin}|pw}
                den wirklich mit Müh und Not beſchafften
               »Sündentraum«beleg ſchickte und da
               \substVorne{}\textsuperscript{bei}\substDazwischen{}zu\substHinten{}
                jenen ominösen, aber durch und durch freundlichen Brief ſchrieb, der den
               harmloſen Witz (»
               Dör-mannbar\pwindex{Doermann, Felix 29.05.1870 – 26.10.1928@\textsc{Dörmann, Felix} (29.05.1870 – 26.10.1928), \emph{Schriftsteller/Schriftstellerin}|pwv}
               «
               enthielt) ſie iſt entſtanden, 
               \uline{lange}
                bevor ich Herrn
                  
               Dörmann\pwindex{Doermann, Felix 29.05.1870 – 26.10.1928@\textsc{Dörmann, Felix} (29.05.1870 – 26.10.1928), \emph{Schriftsteller/Schriftstellerin}|pw}
                perſönlich kennen lernte, ſo daſs
               alſo weder von einem perſönlichen Gefühle 
               {\pb}
               Herrn 
               Specht\pwindex{Specht, Richard 07.12.1870 – 18.03.1932@\textsc{Specht, Richard} (07.12.1870 – 18.03.1932), \emph{Schriftsteller/Schriftstellerin, Journalist/Journalistin, Kritiker/Kritikerin}|pw}
                gegenüber noch von einer
               »Beeinfluſſung durch 
               Dörmann\pwindex{Doermann, Felix 29.05.1870 – 26.10.1928@\textsc{Dörmann, Felix} (29.05.1870 – 26.10.1928), \emph{Schriftsteller/Schriftstellerin}|pw}
               « die Rede ſein
               kann!
            \pend
           
\pstart
           \uuline{Das beſchwöre ich}
               !
            \pend
           
\pstart
           \uline{Alexander Landesberg}\pwindex{Landesberg, Alexander 15.07.1848 – 14.6.1916@\textsc{Landesberg, Alexander} (15.07.1848 – 14.6.1916), \emph{Schriftsteller/Schriftstellerin, Journalist/Journalistin}|pw}
               , 
               Alexander Engel\pwindex{Engel, Alexander 10.04.1868 – 17.11.1940@\textsc{Engel, Alexander} (10.04.1868 – 17.11.1940), \emph{Schriftsteller/Schriftstellerin, Journalist/Journalistin}|pw}
               , 
               Anton Lindner\pwindex{Lindner, Anton 14.12.1874 – 30.12.1928@\textsc{Lindner, Anton} (14.12.1874 – 30.12.1928), \emph{Schriftsteller/Schriftstellerin}|pw}
                etc etc andere Freunde ſind Zeugen!!
            \pend
           
\pstart
           
               Die 
               Kritik\pwindex{Wiener Lyriker@\emph{Wiener Lyriker}|pwv}
                (
               \uline{ganz}
                in der jetzigen Geſtalt!!) iſt – vor Monaten –
               aus einer ehrlichen, vollſten, ureigenſten Überzeugung heraus entſtanden. Nichts
               liegt mir ferner als Unehrlichkeit, als »Rachegefühl« und jüdiſches
               Tagſschreiberthum. Man hüte ſich, mich in dieſer niederträchtigen Weise zu
               verleumden!!
            \pend
           
\pstart
           
               Ich haſſe und haſste diese falſche, erlogene »Decadence«, die artig mit ſich ſelbst
               coquettiert; ich bekämpfe und werde immer bekämpfen: die posierte, krankhafte,
               onanierte Poeſie! 
               {\pb}
               Und 
               \uline{dieſer Haſs}
                war das Kritikmotiv!
            \pend
           
\pstart
           \strikeout{Glauben}
                Sie werden vielleicht, verehrter Herr D
               \textsuperscript{r.}
               , ſich denken: Aha, wer ſich 
               \uline{ſo}
                vertheidigt, 
               \uline{muſs}{ }
               ſich wohl verteidigen!? 
               \strikeout{und}
                Nein, ſeien Sie versichert, die ganze Litanei hab ich auch nur 
               \uline{Ihnen}\noindent{}
                     Auch dem verehrten Herrn D
                     \textsuperscript{r.}{ }B-Hofmann\pwindex{Beer-Hofmann, Richard 1866-07-11 – 1945-09-26@\textsc{Beer-Hofmann, Richard} (1866-07-11 – 1945-09-26), \emph{Schriftsteller/Schriftstellerin}|pw}
                      hätte ich’s geſagt!
                  
                hergeſagt, weil mir an 
               \uline{Ihrer}
                Meinung 
               \strikeout{etw}
                viel liegt. Den andern gegenüber hab’ ich es
               Gottſseidank nicht nöthig, mich zu vertheidigen!
            \pend
           
\pstart
           Wenn ich Sie beläſtigt habe, verzeihen Sie.\pend
           
\pstart
           Otto Erich Hartleben\pwindex{Hartleben, Otto Erich 03.06.1864 – 11.02.1905@\textsc{Hartleben, Otto Erich} (03.06.1864 – 11.02.1905), \emph{Schriftsteller/Schriftstellerin}|pw}
                grüßt Sie durch mich.
            \pend
           
\pstart
           
               Für »
               Neue litt. Bl\orgindex{Neue litterarische Blaetter@Neue litterarische Blätter|pw}
               « 
               \introOben{}
                  (
                  Bremen\oindex{Bremen@\textbf{Bremen}, \emph{P.PPLA}|pw}
                  )
               \introOben{}
                wäre ich mit 
               \strikeout{mit}{ }Anatol\pwindex{Anatol@\emph{Anatol}|pw}
                zu ſpät gekommen, da das dort in 
               \label{K_L00191-1v}\edtext{
               Einläufe
               }{\lemma{\textnormal{\emph{
               Einläufe
               }}}\Cendnote{\textnormal{\emph{Neue litterarische Blätter}\pwindex{Neue litterarische Blaetter@\emph{Neue litterarische Blätter}|pwk}
                     , Jg. 1, H. 5/6,
                        
                     1. 3. 1893
                     , S. 66
                  
                  .
               }}}\label{K_L00191-1}
                verzeichnete 
               Buch\pwindex{Anatol@\emph{Anatol}|pwv}
                bereits an einen andern 
               Mitarbeiter\pwindex{Schmid-Braunfels, Josef 29.11.1871 – 22.11.1911@\textsc{Schmid-Braunfels, Josef} (29.11.1871 – 22.11.1911), \emph{Schriftsteller/Schriftstellerin, Veterinärmediziner/Veterinärmedizinerin}|pwv}
                zur 
               Recension\pwindex{Arthur Schnitzler: Anatol@\emph{Arthur Schnitzler: Anatol}|pwv}
                abgegeben wurde.
            \pend
           
\pstart
           
               Sonſt ſtehe ich Ihnen mit aufrichtigem Vergnügen ſtets zu Dienſten u bin (Sie noch
                  
               \uline{um paar Zeilen bittend}
               !) Ihr 
               \uline{Sie vollkommen hochachtender}\pend
           
\pstart
           
               Herzlichſt grüſſend
               {\\[\baselineskip]}\spacefill\mbox{Karl Kraus}\pend
           \leftskip=0em{}\selectlanguage{ngerman}\endnumbering\briefempfaengerindex{Schnitzler, Arthur@\textsc{Schnitzler, Arthur}!zzzKraus, Karl@\emph{von Karl Kraus}!1893-03-191@{19. 3. 1893}|)be}\mylabel{L00191h}  \normalsize

\doendnotes{C}
\bigskip
\vfill

\clearpage

\footnotesize

\lohead{\textsc{register}}

% Definiere theindex-Environment komplett neu ohne reledmac
\makeatletter
\renewenvironment{theindex}{%
  \section*{\indexname}%
  \setlength{\parindent}{0pt}%
  \setlength{\parskip}{0pt plus 0.3pt}%
  \let\item\@idxitem
}{%
  \clearpage
}
\makeatother

\IfFileExists{\jobname-pw.ind}{\input{\jobname-pw.ind}}{}

\end{document}

      