%% latex-korrekturansicht-vorspann.tex
%% Vorspann für die Korrekturansicht.
%% Lädt die gemeinsame Datei latex-vorspann.tex mit gesetztem Schalter.

\newif\ifkorrekturansicht
\korrekturansichttrue

\input{../tex-inputs/latex-vorspann}


\section[Hermann Bahr an Arthur Schnitzler, 21. 5. 1903]{L01291 Hermann Bahr an Arthur Schnitzler, 21. 5. 1903}
\nopagebreak\mylabel{L01291v}
\rehead{ }\normalsize\beginnumbering\briefempfaengerindex{Schnitzler, Arthur@\textsc{Schnitzler, Arthur}!zzzBahr, Hermann@\emph{von Hermann Bahr}!1903-05-211@{21. 5. 1903}|(be}
\toendnotes[C]{\smallbreak\pagebreak[2]}\Standort{CUL, Schnitzler, B 5b.}
\physDesc{Postkarte, 400 Zeichen
\newline{}Handschrift: schwarze Tinte, deutsche Kurrent
\newline{}Versand: 1) Stempel: »\nobreak{}\oindex{Edlach@\textbf{Edlach}, \emph{P.PPL}|pwk}Edlach b. Reichenau in
                                       N.OE., 22 5 03, 8–12V\nobreak{}«.   2) Stempel: »\nobreak{}\oindex{IX., Alsergrund@\textbf{IX., Alsergrund}, \emph{A.ADM3}|pwk}Wien 9/3, 22 5. 03, 1.N, Bestellt\nobreak{}«. 
\newline{}Schnitzler: mit Bleistift die Jahreszahl »903.« ergänzt 
\newline{}Ordnung: mit Bleistift von unbekannter Hand nummeriert:
                                    »99« }
\buchAbdrucke{\weitereDrucke{Hermann Bahr, Arthur Schnitzler: \emph{Briefwechsel, Aufzeichnungen, Dokumente (1891–1931)}. Göttingen: \emph{Wallstein} 2018, S. 265.} }\toendnotes[C]{\smallbreak}\pstart{}{\pb}Herrn \textsc{D\textsuperscript{r} Arthur Schnitzler}\pend{}\pstart{}Wien IX\oindex{IX., Alsergrund@\textbf{IX., Alsergrund}, \emph{A.ADM3}|pw}\pend{}\pstart{}Frankgaſſe 1\oindex{Frankgasse 1@\textbf{Frankgasse 1}, \emph{Wohngebäude (K.WHS)}|pw}.\pend{}{\bigskip}\vspace{1em}
\pstart
           
\pstart
           {\pb}Edlach Anſtalt D\textsuperscript{r}
                        Konried\oindex{Kuranstalt Dr. Konried@\textbf{Kuranstalt Dr. Konried}, \emph{Sanatorium (K.SAN)}|pw}\pend
           
\pstart
           \raggedleft{}21. 5.\pend
           \pend
           \vspace{0.5em}
\pstart
           Lieber Arthur! Ich habe keine Ahnung, was Du eigentlich meinſt. Ich
               bin ſeit drei Jahren Mitglied des \label{K_L01291-1v}\edtext{Münchner\oindex{Muenchen@\textbf{München}, \emph{P.PPLA}|pw}{ }Penſionsfonds\orgindex{Pensionsanstalt deutscher Journalisten und Schriftsteller@Pensionsanstalt deutscher Journalisten und Schriftsteller|pwv}}{\lemma{\textnormal{\emph{Münchner Penſionsfonds}}}\Cendnote{\textnormal{Bahr\pwindex{Bahr, Hermann 19.07.1863 – 15.01.1934@\textsc{Bahr, Hermann} (19.07.1863 – 15.01.1934), \emph{Schriftsteller/Schriftstellerin, Kritiker/Kritikerin}|pwk} meint denselben Pensionsfonds wie Schnitzler, dieser hatte seinen Sitz in München\oindex{Muenchen@\textbf{München}, \emph{P.PPLA}|pwk}.}}}\label{K_L01291-1} und zahle dafür ſehr wenig;
               ich glaube 6 oder 8 Mark pro Quartal. Von einer anderen »Zeichnung« iſt mir nichts
               bekannt. Ich komme übrigens Montag zurück u. werde mich dann erkundigen.\pend
           
\pstart
           Herzlichſt{\\[\baselineskip]}Dein{\\[\baselineskip]}\spacefill\mbox{Hermann}\pend
           \leftskip=0em{}\selectlanguage{ngerman}\endnumbering\briefempfaengerindex{Schnitzler, Arthur@\textsc{Schnitzler, Arthur}!zzzBahr, Hermann@\emph{von Hermann Bahr}!1903-05-211@{21. 5. 1903}|)be}\mylabel{L01291h}  \normalsize

\doendnotes{C}
\bigskip
\vfill

\clearpage

\footnotesize

\lohead{\textsc{register}}

% Definiere theindex-Environment komplett neu ohne reledmac
\makeatletter
\renewenvironment{theindex}{%
  \section*{\indexname}%
  \setlength{\parindent}{0pt}%
  \setlength{\parskip}{0pt plus 0.3pt}%
  \let\item\@idxitem
}{%
  \clearpage
}
\makeatother

\IfFileExists{\jobname-pw.ind}{\input{\jobname-pw.ind}}{}

\end{document}

      