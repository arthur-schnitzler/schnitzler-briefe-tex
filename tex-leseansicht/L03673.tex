%% latex-leseansicht-vorspann.tex
%% Vorspann für die Leseansicht.
%% Lädt die gemeinsame Datei latex-vorspann.tex mit nicht gesetztem Schalter.

\newif\ifkorrekturansicht
\korrekturansichtfalse

\input{../tex-inputs/latex-vorspann}


\section[Stefan Zweig an Arthur Schnitzler, 4. 2. 1927]{L03673 Stefan Zweig an Arthur Schnitzler, 4. 2. 1927}
\nopagebreak\mylabel{L03673v}
\rehead{ }\normalsize\beginnumbering\briefempfaengerindex{Schnitzler, Arthur@\textsc{Schnitzler, Arthur}!zzzZweig, Stefan@\emph{von Stefan Zweig}!1927-02-041@{4. 2. 1927}|(be}
\toendnotes[C]{\smallbreak\pagebreak[2]}
\correspDesc{Versand  durch Stefan Zweig am 4. 2. 1927 in Salzburg
\newline{}Erhalt  durch Arthur Schnitzler im Zeitraum [5. 2. 1927 – 9. 2. 1927?] in Wien}\toendnotes[C]{\smallbreak}
\Standort{CUL, Schnitzler, B 118.}
\physDesc{Brief, 1 Blatt, 2 Seiten, 2462 Zeichen
\newline{}Handschrift: blaue Tinte, lateinische Kurrent
\newline{}Schnitzler: mit rotem Buntstift 13 Unterstreichungen und beschriftet: »\noindent{}\textsc{Zweig}{ / }(Diagram\pwindex{Schnitzler, Arthur 15.\,5.\,1862 Wien – 21.\,10.\,1931 ebd.@\textsc{Schnitzler, Arthur} (15.\,5.\,1862 Wien – 21.\,10.\,1931 ebd.), \emph{Schriftsteller, Mediziner}!Geist im Wort und der Geist in der Tat@\strich\emph{Der Geist im Wort und der Geist in der Tat}|pw}« }
\buchAbdrucke{\weitereDrucke{Stefan Zweig: \emph{Briefwechsel mit Hermann Bahr, Sigmund Freud, Rainer Maria
                        Rilke und Arthur Schnitzler}. Herausgegeben von Jeffrey B. Berlin, Hans-Ulrich Lindken und Donald A. Prater. Frankfurt am Main: \emph{S. Fischer} 1987, S. 425–426.} }\toendnotes[C]{\smallbreak}
\pstart
           {\pb}\textcolor{gray}{\textbf{SZ}}\hfill \textcolor{gray}{\textbf{SALZBURG\oindex{Salzburg@\textbf{Salzburg}, \emph{Verwaltungsgebiet}|pw}}}{ }4. II 1927\pend
           
\pstart
           \raggedleft{}\textcolor{gray}{\textbf{KAPUZINERBERG 5\oindex{Paschinger Schlössl@\textbf{Paschinger Schlössl}, \emph{Wohngebäude}|pw}}}\pend
           \vspace{0.5em}
\pstart
           Lieber verehrter Herr Doktor, ich war, wie wohl alle, erst
               überrascht, von Ihnen ein characterologisches Buch\pwindex{Schnitzler, Arthur 15.\,5.\,1862 Wien – 21.\,10.\,1931 ebd.@\textsc{Schnitzler, Arthur} (15.\,5.\,1862 Wien – 21.\,10.\,1931 ebd.), \emph{Schriftsteller, Mediziner}!Geist im Wort und der Geist in der Tat@\strich\emph{Der Geist im Wort und der Geist in der Tat}|pw} zu empfangen, doch gleichzeitig sehr neugierig gereizt, wie ein so
               verantwortliches Problem bei Ihnen Lösung finde. Sie wissen ja, dass mein
               essayistisches Hauptwerk\pwindex{Zweig, Stefan 28.\,11.\,1881 Wien – 23.\,2.\,1942 Petrópolis@\textsc{Zweig, Stefan} (28.\,11.\,1881 Wien – 23.\,2.\,1942 Petrópolis), \emph{Schriftsteller}!Baumeister der Welt. Versuch einer Typologie des Geistes@\strich\emph{Die Baumeister der Welt. Versuch einer Typologie des Geistes}|pwv} von
               dem nur zwei Bände\pwindex{Zweig, Stefan 28.\,11.\,1881 Wien – 23.\,2.\,1942 Petrópolis@\textsc{Zweig, Stefan} (28.\,11.\,1881 Wien – 23.\,2.\,1942 Petrópolis), \emph{Schriftsteller}!Drei Meister. Balzac – Dickens – Dostojewski@\strich\emph{Drei Meister. Balzac – Dickens – Dostojewski}|pwv}\pwindex{Zweig, Stefan 28.\,11.\,1881 Wien – 23.\,2.\,1942 Petrópolis@\textsc{Zweig, Stefan} (28.\,11.\,1881 Wien – 23.\,2.\,1942 Petrópolis), \emph{Schriftsteller}!Kampf mit dem Dämon. Hölderlin – Kleist – Nietzsche@\strich\emph{Der Kampf mit dem Dämon. Hölderlin – Kleist – Nietzsche}|pwv}
               bisher erschienen sind, eine »\label{K_L03673-1v}\edtext{Typologie des Geistes\pwindex{Zweig, Stefan 28.\,11.\,1881 Wien – 23.\,2.\,1942 Petrópolis@\textsc{Zweig, Stefan} (28.\,11.\,1881 Wien – 23.\,2.\,1942 Petrópolis), \emph{Schriftsteller}!Baumeister der Welt. Versuch einer Typologie des Geistes@\strich\emph{Die Baumeister der Welt. Versuch einer Typologie des Geistes}|pwv}}{\lemma{\textnormal{\emph{Typologie des Geistes}}}\Cendnote{\textnormal{Der Reihentitel lautet vollständig:
                     »Die Baumeister der Welt. Versuch einer Typologie des Geistes«
                  und wurde 1928 noch um den Band \emph{Drei
                     Dichter ihres Lebens. Casanova – Stendhal – Tolstoi}\pwindex{Zweig, Stefan 28.\,11.\,1881 Wien – 23.\,2.\,1942 Petrópolis@\textsc{Zweig, Stefan} (28.\,11.\,1881 Wien – 23.\,2.\,1942 Petrópolis), \emph{Schriftsteller}!Drei Dichter ihres Lebens. Casanova – Stendhal – Tolstoi@\strich\emph{Drei Dichter ihres Lebens. Casanova – Stendhal – Tolstoi}|pwk} erweitert.}}}\label{K_L03673-1}«
               sein will, also die ganzen Identitäten in Varianten aufzeigen; so war Ihre
               Formulierung mir eine Art Bestätigung und insbesondere jener tragende Gedanke, dass
               jede Erscheinung ihren Schatten wirft wie ein organisches Gebilde, dass jeder Sinn
               tätig seinen Widersinn, seine Verzerrung in der irdischen Erscheinung erschafft, will
               mir ausserordentlich fruchtbar erscheinen. Dazu formt sich die Abwandlung durchaus
               klar: \label{K_L03673-11v}\edtext{more geometrico}{\lemma{\textnormal{\emph{more geometrico}}}\Cendnote{\textnormal{lateinisch: nach der geometrischen Methode. Spinoza\pwindex{Spinoza, Baruch de 24.\,11.\,1632 Amsterdam – 21.\,2.\,1677 Den Haag@\textsc{Spinoza, Baruch de} (24.\,11.\,1632 Amsterdam – 21.\,2.\,1677 Den Haag), \emph{Philosoph, Augenoptiker}|pwk} 
                  verwendete den Begriff um, in Analogie zur Geometrie, den Zusammenhang von Grundsätzen und Axiomen zu benennen, die ein gemeinsames, ableitbares System bilden.}}}\label{K_L03673-11} im Sinne unseres Spinoza\pwindex{Spinoza, Baruch de 24.\,11.\,1632 Amsterdam – 21.\,2.\,1677 Den Haag@\textsc{Spinoza, Baruch de} (24.\,11.\,1632 Amsterdam – 21.\,2.\,1677 Den Haag), \emph{Philosoph, Augenoptiker}|pw}
               und auch das Widerspiel fehlt nicht, \label{K_L03673-4v}\edtext{amor intellectualis}{\lemma{\textnormal{\emph{amor intellectualis}}}\Cendnote{\textnormal{lateinisch: geistige Liebe. Bei Spinoza\pwindex{Spinoza, Baruch de 24.\,11.\,1632 Amsterdam – 21.\,2.\,1677 Den Haag@\textsc{Spinoza, Baruch de} (24.\,11.\,1632 Amsterdam – 21.\,2.\,1677 Den Haag), \emph{Philosoph, Augenoptiker}|pwk}
                in Zusammenhang mit Gott: »amor intellectualis dei«, in dem Sinne gebraucht, dass es eine intuitive Erkenntnis (Gottes) gäbe.}}}\label{K_L03673-4}, die rein geistige Liebe zur
               beinahe metaphysischen Problematik. Ich bin für Sie dieses kleinen Büchleins\pwindex{Schnitzler, Arthur 15.\,5.\,1862 Wien – 21.\,10.\,1931 ebd.@\textsc{Schnitzler, Arthur} (15.\,5.\,1862 Wien – 21.\,10.\,1931 ebd.), \emph{Schriftsteller, Mediziner}!Geist im Wort und der Geist in der Tat@\strich\emph{Der Geist im Wort und der Geist in der Tat}|pw} sehr froh, denn die Menschen nehmen den Künstler am
               liebsten dort, wo er leicht und locker wird, in ihre Wertung auf. Hier werden manche
               über den sachlichen Ernst erstaunen, der in Ihnen die Urmacht ist – ich freilich
               erstaune nicht, ich weiss ja auch von Ihren verstreuten und leider noch nicht
               gesammelten Reflexionen über die Kunst, wie sehr Sie die innerliche Mechanik dessen
               beschäftigt, was nach aussen hin als Selbstverständlich-Wirkendes erscheint. Es wäre
               mir innige Freude, einmal ausführlich mit Ihnen über diese Probleme sprechen zu
               dürfen: im tiefsten Grunde sind Sie damit dem Sinn der Zeit nahegekommen, die endlich
               – endlich! – müde wird der collectiven Typenlehre von den »Rassen« und »Nationen« wie
               sie Gobineau\pwindex{Gobineau, Joseph Arthur de 14.\,7.\,1816 Ville-d'Avray – 13.\,10.\,1882 Turin@\textsc{Gobineau, Joseph Arthur de} (14.\,7.\,1816 Ville-d'Avray – 13.\,10.\,1882 Turin), \emph{Schriftsteller, Diplomat, Orientalist}|pw} in die Welt setzte und die \strikeout{individueller} Einordnung in den \introOben{}Individual-\introOben{}Typus begehrt. Das haben Sie mit {\pb}dieser kleinen Studie\pwindex{Schnitzler, Arthur 15.\,5.\,1862 Wien – 21.\,10.\,1931 ebd.@\textsc{Schnitzler, Arthur} (15.\,5.\,1862 Wien – 21.\,10.\,1931 ebd.), \emph{Schriftsteller, Mediziner}!Geist im Wort und der Geist in der Tat@\strich\emph{Der Geist im Wort und der Geist in der Tat}|pw}, die nur den Rand zu berühren scheint, in Wahrheit
               aber auf das Wesentliche zielt, sehr gefördert.\pend
           
\pstart
           Ich fahre jetzt ein wenig nach Süden, hoffentlich in neue Arbeit hinein. Das letzte
               Jahr war äusserlich so gut zu mir, dass ich nun doppelt anpruchsvoll wider mich sein
               muss, um den unerwarteten Erfolg nicht zu dementieren. Aber je schwerer sie wird,
               desto lieber hat man die Arbeit: ich weiss, es geht Ihnen ebenso und nie war Ihr
               geistiger Ertrag fülliger und bedeutsamer als in den letzten Jahren.\pend
           
\pstart
           Muss ich noch besonders sagen, wie sehr und innig ich Ihnen anhänge? Ich hoffe,
               Sie wissen’s und gedenken freundlich Ihres getreuen{\\[\baselineskip]}\spacefill\mbox{Stefan Zweig}\pend
           \leftskip=0em{}\selectlanguage{ngerman}\endnumbering\briefempfaengerindex{Schnitzler, Arthur@\textsc{Schnitzler, Arthur}!zzzZweig, Stefan@\emph{von Stefan Zweig}!1927-02-041@{4. 2. 1927}|)be}\mylabel{L03673h}
\begin{anhang}
\end{anhang}\newcommand{\dateiname}{L03673}\newcommand{\titel}{Stefan Zweig an Arthur Schnitzler, 4. 2. 1927}\newcommand{\editorInnen}{Selma Jahnke und Martin Anton Müller}%% latex-leseansicht-abspann.tex
%% Abspann für die Leseansicht.
%% Der Schalter \ifkorrekturansicht ist bereits durch den Vorspann gesetzt.

%% latex-abspann.tex
%% Gemeinsamer Abspann für Korrekturansicht und Leseansicht.
%% Setzt den Schalter \ifkorrekturansicht voraus (gesetzt in den
%% einbindenden Dateien latex-korrekturansicht-abspann.tex bzw.
%% latex-leseansicht-abspann.tex).
%% ---------------------------------------------------------------

\normalsize

% Das esempio-Environment wird nur in der Leseansicht benötigt
\ifkorrekturansicht\else
\newenvironment{esempio}[3]%
{
    \vspace{1.5ex}
    \rlap{\underline{#1}}
    \par
    \setlength{\parindent}{0cm}
    \nopagebreak
    \leftskip=#2cm
    \rightskip=#3cm
}
{
    \par
}
\fi

\doendnotes{C}
\bigskip
\vfill

\clearpage

\footnotesize

\ifkorrekturansicht
  \lohead{\textsc{register}}
\fi

% theindex-Environment neu definieren ohne reledmac
\makeatletter
\renewenvironment{theindex}{%
  \ifkorrekturansicht
    \section*{\indexname}%
  \else
    \subsubsection*{Index der erwähnten Entitäten}%
  \fi
  \setlength{\parindent}{0pt}%
  \setlength{\parskip}{0pt plus 0.3pt}%
  \let\item\@idxitem
}{%
  \ifkorrekturansicht\clearpage\fi
}
\makeatother

\IfFileExists{\jobname-pw.ind}{\input{\jobname-pw.ind}}{}

% Quellenangabe nur in der Leseansicht
\ifkorrekturansicht\else
% Fallback-Definitionen, falls die .tex-Datei \titel etc. nicht gesetzt hat
\providecommand{\titel}{}
\providecommand{\editorInnen}{}
\providecommand{\dateiname}{\jobname}

\vspace{3cm}

\vfill

\footnotesize
\textsc{Quelle}: \titel. Herausgegeben von {\editorInnen}. In: \emph{Arthur Schnitzler: Briefwechsel mit Autorinnen und Autoren}.
 Digitale Edition, https://schnitzler-briefe.acdh.oeaw.ac.at/{\dateiname}.html (Stand \today)
\fi

\end{document}


