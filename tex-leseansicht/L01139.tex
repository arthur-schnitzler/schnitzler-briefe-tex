%% latex-korrekturansicht-vorspann.tex
%% Vorspann für die Korrekturansicht.
%% Lädt die gemeinsame Datei latex-vorspann.tex mit gesetztem Schalter.

\newif\ifkorrekturansicht
\korrekturansichttrue

\input{../tex-inputs/latex-vorspann}


\section[Arthur Schnitzler an Hermann Bahr, 3. 7. 1901]{L01139 Arthur Schnitzler an Hermann Bahr, 3. 7. 1901}
\nopagebreak\mylabel{L01139v}
\rehead{ }\normalsize\beginnumbering\briefempfaengerindex{Bahr, Hermann@\textsc{Bahr, Hermann}!zzzSchnitzler, Arthur@\emph{von Arthur Schnitzler}!1901-07-031@{3. 7. 1901}|(be}
\toendnotes[C]{\smallbreak\pagebreak[2]}\Standort{TMW, HS AM 60180 Ba.}
\physDesc{Postkarte, 215 Zeichen
\newline{}Handschrift: Bleistift, deutsche Kurrent
\newline{}Versand: 1) Stempel: »\nobreak{}\oindex{St. Anton am Arlberg@\textbf{St. Anton am Arlberg}, \emph{A.ADM3}|pwk}St. Anton am Arlberge, 4 7 01\nobreak{}«.   2) Stempel: »\nobreak{}Wien, {[}5. 7.{]} 01, B{[}estell{]}t\nobreak{}«. 
\newline{}Ordnung: Lochung }
\buchAbdrucke{\weitereDrucke{1) Arthur Schnitzler: \emph{The Letters of Arthur Schnitzler to Hermann Bahr}. Chapel Hill: \emph{The University of North Carolina Press} 1978, S. 69.} \weitereDrucke{2) Hermann Bahr, Arthur Schnitzler: \emph{Briefwechsel, Aufzeichnungen, Dokumente (1891–1931)}. Göttingen: \emph{Wallstein} 2018, S. 212.} }\toendnotes[C]{\smallbreak}\pstart{}{\pb}\textsc{Herrn Hermann Bahr}\pend{}\pstart{}Wien – \textsc{Ob St Veit}\oindex{Ober Sankt Veit@\textbf{Ober Sankt Veit}, \emph{P.PPLX}|pw}\pend{}\pstart{}\textsc{Veitlissengasse\oindex{Veitlissengasse@\textbf{Veitlissengasse}, \emph{Straße (K.STR)}|pw}}\pend{}{\bigskip}\vspace{1em}
\pstart
           \noindent{}{\pb}mein lieber Hermann, hieher beko{\geminationm} ich
               dein Feuilleton\pwindex{Erotisch@\emph{Erotisch}|pwv} nachgeſandt;
               ich hatte es aber ſchon vorher mit großer Freude geleſen \pend
           
\pstart
           Herzlichſt dein{\\[\baselineskip]}\spacefill\mbox{ArthSch}\pend
           \leftskip=0em{}
\pstart
           \noindent{}\textsc{St. Anton Arlberg\oindex{St. Anton am Arlberg@\textbf{St. Anton am Arlberg}, \emph{A.ADM3}|pw}}{ }3/7 901\pend
           \selectlanguage{ngerman}\endnumbering\briefempfaengerindex{Bahr, Hermann@\textsc{Bahr, Hermann}!zzzSchnitzler, Arthur@\emph{von Arthur Schnitzler}!1901-07-031@{3. 7. 1901}|)be}\mylabel{L01139h}  \normalsize

\doendnotes{C}
\bigskip
\vfill

\clearpage

\footnotesize

\lohead{\textsc{register}}

% Definiere theindex-Environment komplett neu ohne reledmac
\makeatletter
\renewenvironment{theindex}{%
  \section*{\indexname}%
  \setlength{\parindent}{0pt}%
  \setlength{\parskip}{0pt plus 0.3pt}%
  \let\item\@idxitem
}{%
  \clearpage
}
\makeatother

\IfFileExists{\jobname-pw.ind}{\input{\jobname-pw.ind}}{}

\end{document}

      