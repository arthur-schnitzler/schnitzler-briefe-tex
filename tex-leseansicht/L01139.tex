%% latex-leseansicht-vorspann.tex
%% Vorspann für die Leseansicht.
%% Lädt die gemeinsame Datei latex-vorspann.tex mit nicht gesetztem Schalter.

\newif\ifkorrekturansicht
\korrekturansichtfalse

\input{../tex-inputs/latex-vorspann}


\section[Arthur Schnitzler an Hermann Bahr, 3. 7. 1901]{L01139 Arthur Schnitzler an Hermann Bahr, 3. 7. 1901}
\nopagebreak\mylabel{L01139v}
\rehead{ }\normalsize\beginnumbering\briefempfaengerindex{Bahr, Hermann@\textsc{Bahr, Hermann}!zzzSchnitzler, Arthur@\emph{von Arthur Schnitzler}!1901-07-031@{3. 7. 1901}|(be}
\toendnotes[C]{\smallbreak\pagebreak[2]}
\correspDesc{Versand  durch Arthur Schnitzler am 3. 7. 1901 in St. Anton am Arlberg
\newline{}Erhalt  durch Hermann Bahr am 5. 7. 1901 in Wien}\toendnotes[C]{\smallbreak}
\Standort{TMW, HS AM 60180 Ba.}
\physDesc{Postkarte, 215 Zeichen
\newline{}Handschrift: Bleistift, deutsche Kurrent
\newline{}Versand: 1) Stempel: »\nobreak{}\oindex{St. Anton am Arlberg@\textbf{St. Anton am Arlberg}, \emph{Verwaltungsgebiet}|pwk}St. Anton am Arlberge, 4 7 01\nobreak{}«.   2) Stempel: »\nobreak{}\oindex{Wien@\textbf{Wien}, \emph{Verwaltungsgebiet}|pwk}Wien, {[}5. 7.{]} 01, B{[}estell{]}t\nobreak{}«. 
\newline{}Ordnung: Lochung }
\buchAbdrucke{\weitereDrucke{1) \emph{3. 7. 1901, Abschrift.} In: Arthur Schnitzler: \emph{The Letters of Arthur Schnitzler to Hermann Bahr}. Edited, annotated, and with an introduction, by Donald G. Daviau. Chapel Hill: \emph{The University of North Carolina Press} 1978, S. 69 (University of North Carolina studies in the Germanic languages
                        and literatures, 89).} \weitereDrucke{2) Hermann Bahr, Arthur Schnitzler: \emph{Briefwechsel, Aufzeichnungen, Dokumente (1891–1931)}. Herausgegeben von Kurt Ifkovits und Martin Anton Müller. Göttingen: \emph{Wallstein} 2018, S. 212.} }\toendnotes[C]{\smallbreak}\pstart{}{\pb}\textsc{Herrn Hermann Bahr}\pend{}\pstart{}Wien – \textsc{Ob St Veit}\oindex{Wien@\textbf{Wien}!XIII., Hietzing@\textbf{XIII., Hietzing}!Ober Sankt Veit@\textbf{Ober Sankt Veit}, \emph{Ehemaliger Ort}|pw}\pend{}\pstart{}\textsc{Veitlissengasse\oindex{Wien@\textbf{Wien}!XIII., Hietzing@\textbf{XIII., Hietzing}!Veitlissengasse@\textbf{Veitlissengasse}, \emph{Straße}|pw}}\pend{}{\bigskip}\vspace{1em}
\pstart
           \noindent{}{\pb}mein lieber Hermann, hieher beko{\geminationm} ich
               dein Feuilleton\pwindex{Bahr, Hermann 19.\,7.\,1863 Linz – 15.\,1.\,1934 München@\textsc{Bahr, Hermann} (19.\,7.\,1863 Linz – 15.\,1.\,1934 München), \emph{Schriftsteller, Kritiker}!Erotisch@\strich\emph{Erotisch}|pwv} nachgeſandt;
               ich hatte es aber{ }ſchon vorher mit großer Freude geleſen\pend
           
\pstart
           Herzlichſt dein{\\[\baselineskip]}\spacefill\mbox{ArthSch}\pend
           \leftskip=0em{}
\pstart
           \noindent{}\textsc{St. Anton Arlberg\oindex{St. Anton am Arlberg@\textbf{St. Anton am Arlberg}, \emph{Verwaltungsgebiet}|pw}}{ }3/7 901\pend
           \selectlanguage{ngerman}\endnumbering\briefempfaengerindex{Bahr, Hermann@\textsc{Bahr, Hermann}!zzzSchnitzler, Arthur@\emph{von Arthur Schnitzler}!1901-07-031@{3. 7. 1901}|)be}\mylabel{L01139h}  \newcommand{\dateiname}{L01139}\newcommand{\titel}{Arthur Schnitzler an Hermann Bahr, 3. 7. 1901}\newcommand{\editorInnen}{Herausgegeben von Martin Anton Müller}%% latex-leseansicht-abspann.tex
%% Abspann für die Leseansicht.
%% Der Schalter \ifkorrekturansicht ist bereits durch den Vorspann gesetzt.

%% latex-abspann.tex
%% Gemeinsamer Abspann für Korrekturansicht und Leseansicht.
%% Setzt den Schalter \ifkorrekturansicht voraus (gesetzt in den
%% einbindenden Dateien latex-korrekturansicht-abspann.tex bzw.
%% latex-leseansicht-abspann.tex).
%% ---------------------------------------------------------------

\normalsize

% Das esempio-Environment wird nur in der Leseansicht benötigt
\ifkorrekturansicht\else
\newenvironment{esempio}[3]%
{
    \vspace{1.5ex}
    \rlap{\underline{#1}}
    \par
    \setlength{\parindent}{0cm}
    \nopagebreak
    \leftskip=#2cm
    \rightskip=#3cm
}
{
    \par
}
\fi

\doendnotes{C}
\bigskip
\vfill

\clearpage

\footnotesize

\ifkorrekturansicht
  \lohead{\textsc{register}}
\fi

% theindex-Environment neu definieren ohne reledmac
\makeatletter
\renewenvironment{theindex}{%
  \ifkorrekturansicht
    \section*{\indexname}%
  \else
    \subsubsection*{Index der erwähnten Entitäten}%
  \fi
  \setlength{\parindent}{0pt}%
  \setlength{\parskip}{0pt plus 0.3pt}%
  \let\item\@idxitem
}{%
  \ifkorrekturansicht\clearpage\fi
}
\makeatother

\IfFileExists{\jobname-pw.ind}{\input{\jobname-pw.ind}}{}

% Quellenangabe nur in der Leseansicht
\ifkorrekturansicht\else
% Fallback-Definitionen, falls die .tex-Datei \titel etc. nicht gesetzt hat
\providecommand{\titel}{}
\providecommand{\editorInnen}{}
\providecommand{\dateiname}{\jobname}

\vspace{3cm}

\vfill

\footnotesize
\textsc{Quelle}: \titel. Herausgegeben von {\editorInnen}. In: \emph{Arthur Schnitzler: Briefwechsel mit Autorinnen und Autoren}.
 Digitale Edition, https://schnitzler-briefe.acdh.oeaw.ac.at/{\dateiname}.html (Stand \today)
\fi

\end{document}


