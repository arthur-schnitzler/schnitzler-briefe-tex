\input{../tex-inputs/latex-pdf-vorspann}
\begin{center}
            \textcolor{red}{ENTWURF. ENTZIFFERUNG NOCH NICHT KORREKTURGELESEN}
                      \end{center}
            
               \section[Arthur Schnitzler an Hermann Bahr, 3. 7. 1901]{ Arthur Schnitzler an Hermann Bahr, 3. 7. 1901}\nopagebreak\mylabel{v}\rehead{ }\begin{ledgroupsized}[t]{13cm}\normalsize\beginnumbering\briefempfaengerindex{Bahr, Hermann@\textsc{Bahr, Hermann}!zzzSchnitzler, Arthur@\emph{von Arthur Schnitzler}!1901-07-031@{3. 7. 1901}|(be} \toendnotes[C]{\smallbreak\pagebreak[2]} \Standort{TMW, HS AM 60180 Ba.}
\physDesc{Postkarte
\newline{}Handschrift: Bleistift, deutsche Kurrent\newline{}Versand: 1) Stempel: »\nobreak{}\oindex{St. Anton am Arlberg@\textbf{St. Anton am Arlberg}|pwk}St. Anton am Arlberge, 4 7 01\nobreak{}«.  2) Stempel: »\nobreak{}Wien, {[}5. 7.{]} 01, B{[}estell{]}t\nobreak{}«. \newline{}Ordnung: Lochung }\buchAbdrucke{\weitereDrucke{1) \emph{3. 7. 1901, Abschrift.} In: Arthur Schnitzler: \emph{The Letters of Arthur Schnitzler to Hermann Bahr}. Edited, annotated, and with an introduction, by Donald G.
                        Daviau. Chapel Hill: \emph{The University of North Carolina Press} 1978, S. 69 (University of North Carolina studies in the Germanic languages
                        and literatures, 89).} \weitereDrucke{2) Hermann Bahr, Arthur Schnitzler: \emph{Briefwechsel, Aufzeichnungen, Dokumente (1891–1931)}. Hg. Kurt Ifkovits und Martin Anton Müller. Göttingen: \emph{Wallstein} 2018, S. 212.} }\toendnotes[C]{\smallbreak}\pstart{}{\pb}\textsc{Herrn Hermann Bahr}\pend{}\pstart{}Wien – \textsc{Ob St Veit}\oindex{Ober Sankt Veit@\textbf{Ober Sankt Veit}|pw}\pend{}\pstart{}\textsc{Veitlissengasse\oindex{Veitlissengasse@\textbf{Veitlissengasse}|pw}}\pend{}{\bigskip}\pstart
           \noindent{}{\pb}mein lieber Hermann, hieher beko{\geminationm} ich
               dein Feuilleton\pwindex{Bahr, Hermann 19.07.1863 – 15.01.1934@\textsc{Bahr, Hermann} (19.07.1863 – 15.01.1934), \emph{Schriftsteller, Kritiker}!Erotisch22. 06. 1901@\strich\emph{Erotisch} {[}22. 06. 1901{]}|pwv}\pwindex{Bahr, Hermann 19.07.1863 – 15.01.1934@\textsc{Bahr, Hermann} (19.07.1863 – 15.01.1934), \emph{Schriftsteller, Kritiker}!Erotisch22. 06. 1901@\strich\emph{Erotisch} {[}22. 06. 1901{]}|pwv} nachgeſandt;
               ich hatte es aber ſchon vorher mit großer Freude geleſen \pend
           \pstart
           Herzlichſt dein{\\[\baselineskip]}\spacefill\mbox{ArthSch}\pend
           \leftskip=0em{}\pstart
           \noindent{}\textsc{St. Anton Arlberg\oindex{St. Anton am Arlberg@\textbf{St. Anton am Arlberg}|pw}}{ }3/7 901\pend
           \endnumbering\briefempfaengerindex{Bahr, Hermann@\textsc{Bahr, Hermann}!zzzSchnitzler, Arthur@\emph{von Arthur Schnitzler}!1901-07-031@{3. 7. 1901}|)be}\mylabel{h}\end{ledgroupsized}  \newcommand{\dateiname}{L01139}\newcommand{\titel}{Arthur Schnitzler an Hermann Bahr, 3. 7. 1901}\newcommand{\editorInnen}{ Kurt Ifkovits,  Martin Anton Müller}\input{../tex-inputs/latex-pdf-abspann}
      