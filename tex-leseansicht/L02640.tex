%% latex-leseansicht-vorspann.tex
%% Vorspann für die Leseansicht.
%% Lädt die gemeinsame Datei latex-vorspann.tex mit nicht gesetztem Schalter.

\newif\ifkorrekturansicht
\korrekturansichtfalse

\input{../tex-inputs/latex-vorspann}


\section[Paul Goldmann an Arthur Schnitzler, 18. 6. 1889]{L02640 Paul Goldmann an Arthur Schnitzler, 18. 6. 1889}
\nopagebreak\mylabel{L02640v}
\rehead{ }\normalsize\beginnumbering\briefempfaengerindex{Schnitzler, Arthur@\textsc{Schnitzler, Arthur}!zzzGoldmann, Paul@\emph{von Paul Goldmann}!1889-06-181@{18. 6. 1889}|(be}
\toendnotes[C]{\smallbreak\pagebreak[2]}
\correspDesc{Versand  durch Paul Goldmann am 18. 6. 1889 in Wien
\newline{}Erhalt  durch Arthur Schnitzler im Zeitraum [18. 6. 1889
                  – 22. 6. 1889?] in Wien}\toendnotes[C]{\smallbreak}
\Standort{DLA, A:Schnitzler, HS.NZ85.1.3162.}
\physDesc{Brief, 1 Blatt, 2 Seiten, 857 Zeichen
\newline{}Handschrift: blaue Tinte, deutsche Kurrent}\toendnotes[C]{\smallbreak}
\pstart
           \centering{}{\pb}\textcolor{gray}{\textbf{\textbf{Adminiſtration: VII.
                           Seidengaſſe 7\oindex{Wien@\textbf{Wien}!VII., Neubau@\textbf{VII., Neubau}!Seidengasse@\textbf{Seidengasse}, \emph{Straße}|pw}} (Jos. Eberle {\kaufmannsund} Co.\orgindex{Josef Eberle Stein-, Buch und Musikaliendruckerei@Josef Eberle Stein-, Buch und Musikaliendruckerei|pw})}}\pend
           
\pstart
           \centering{}\textcolor{gray}{\textbf{An der Schönen Blauen Donau\orgindex{der schönen blauen Donau@An der schönen blauen Donau|pw}}}\pend
           
\pstart
           \centering{}\textcolor{gray}{\textbf{Chef-Redacteur: Dr. F.
                        Mamroth\pwindex{Mamroth, Fedor 21.\,2.\,1851 Breslau – 25.\,6.\,1907 Frankfurt am Main@\textsc{Mamroth, Fedor} (21.\,2.\,1851 Breslau – 25.\,6.\,1907 Frankfurt am Main), \emph{Journalist, Kritiker}|pw}. – Redaction: IX.,
                        Berggaſſe 31\oindex{Wien@\textbf{Wien}!IX., Alsergrund@\textbf{IX., Alsergrund}!Berggasse@\textbf{Berggasse}, \emph{Straße}|pw}.}}\pend
           
\pstart
           \raggedleft{}\textcolor{gray}{\textbf{Wien\oindex{Wien@\textbf{Wien}, \emph{Verwaltungsgebiet}|pw}, den}}{ }18. Juni \textcolor{gray}{\textbf{18}}89.\pend
           
\pstart\center{}Sehr geehrter Herr Doctor!\pend\vspace{0.5em}
\pstart
           Die zwei vermißten \label{K_L02640-1v}\edtext{Gedichte\pwindex{Schnitzler, Arthur 15.\,5.\,1862 Wien – 21.\,10.\,1931 ebd.@\textsc{Schnitzler, Arthur} (15.\,5.\,1862 Wien – 21.\,10.\,1931 ebd.), \emph{Schriftsteller, Mediziner}!Lieder eines Nervösen@\strich\emph{Lieder eines Nervösen}|pwv}}{\lemma{\textnormal{\emph{Gedichte}}}\Cendnote{\textnormal{Unter dem Pseudonym »Anatol« und mit dem
                  Titel \emph{Lieder eines Nervösen}\pwindex{Schnitzler, Arthur 15.\,5.\,1862 Wien – 21.\,10.\,1931 ebd.@\textsc{Schnitzler, Arthur} (15.\,5.\,1862 Wien – 21.\,10.\,1931 ebd.), \emph{Schriftsteller, Mediziner}!Lieder eines Nervösen@\strich\emph{Lieder eines Nervösen}|pwk} erschienen im
                  ersten Juli-Heft von \emph{An der schönen blauen
                     Donau}\pwindex{der schönen blauen Donau@\emph{An der schönen blauen Donau}|pwk} fünf Gedichte Schnitzlers (Jg. 4, H. 13, S. 297). Welche davon kurzzeitig vermisst waren,
                  ist nicht geklärt.}}}\label{K_L02640-1} und noch eine Anzahl anderer haben{ }ſich bereits
               gefunden. Ich hatte dieſelben in jenes beſondere Fach unſeres Manuſkripten-Kaſtens
               gelegt, in dem die zum Setzen zu gebenden Beiträge aufbewahrt werden und{ }ſofort,
               nachdem ich dies gethan, daran vergeſſen (wie ich dies mit {\pb}Vorliebe zu thun pflege). Die Sachen hätten{ }ſich
               ohnedies dann bei den Vorabeiten für das nächſte Heft\pwindex{der schönen blauen Donau@\emph{An der schönen blauen Donau}|pwv} wieder an’s Tageslicht emporgearbeitet. Es
               thut mir nur leid, daß ich Ihnen durch meine Zerſtreutheit einige Stunden der Sorge
               bereitet habe. Ich bitte Sie alſo, vollſtändig beruhigt \introOben{}zu\introOben{}{ }ſein. Wenn Sie mir das nächſte Mal wieder das Vergnügen Ihres Beſuches machen werden,
               werden Sie die Kinder ihrer
                  Muſe\pwindex{Schnitzler, Arthur 15.\,5.\,1862 Wien – 21.\,10.\,1931 ebd.@\textsc{Schnitzler, Arthur} (15.\,5.\,1862 Wien – 21.\,10.\,1931 ebd.), \emph{Schriftsteller, Mediziner}!Lieder eines Nervösen@\strich\emph{Lieder eines Nervösen}|pwv} friſch, geſund und unbeſchädigt von Angeſicht zu Angeſicht begrüßen
               können. Hochachtungsvoll\pend
           
\pstart
           Ihr ergeb\textcolor{gray}{ner}{\\[\baselineskip]}\spacefill\mbox{Dr. Paul Goldmann}\pend
           \leftskip=0em{}\selectlanguage{ngerman}\endnumbering\briefempfaengerindex{Schnitzler, Arthur@\textsc{Schnitzler, Arthur}!zzzGoldmann, Paul@\emph{von Paul Goldmann}!1889-06-181@{18. 6. 1889}|)be}\mylabel{L02640h}  \newcommand{\dateiname}{L02640}\newcommand{\titel}{Paul Goldmann an Arthur Schnitzler, 18. 6. 1889}\newcommand{\editorInnen}{Martin Anton Müller und Laura Untner}%% latex-leseansicht-abspann.tex
%% Abspann für die Leseansicht.
%% Der Schalter \ifkorrekturansicht ist bereits durch den Vorspann gesetzt.

%% latex-abspann.tex
%% Gemeinsamer Abspann für Korrekturansicht und Leseansicht.
%% Setzt den Schalter \ifkorrekturansicht voraus (gesetzt in den
%% einbindenden Dateien latex-korrekturansicht-abspann.tex bzw.
%% latex-leseansicht-abspann.tex).
%% ---------------------------------------------------------------

\normalsize

% Das esempio-Environment wird nur in der Leseansicht benötigt
\ifkorrekturansicht\else
\newenvironment{esempio}[3]%
{
    \vspace{1.5ex}
    \rlap{\underline{#1}}
    \par
    \setlength{\parindent}{0cm}
    \nopagebreak
    \leftskip=#2cm
    \rightskip=#3cm
}
{
    \par
}
\fi

\doendnotes{C}
\bigskip
\vfill

\clearpage

\footnotesize

\ifkorrekturansicht
  \lohead{\textsc{register}}
\fi

% theindex-Environment neu definieren ohne reledmac
\makeatletter
\renewenvironment{theindex}{%
  \ifkorrekturansicht
    \section*{\indexname}%
  \else
    \subsubsection*{Index der erwähnten Entitäten}%
  \fi
  \setlength{\parindent}{0pt}%
  \setlength{\parskip}{0pt plus 0.3pt}%
  \let\item\@idxitem
}{%
  \ifkorrekturansicht\clearpage\fi
}
\makeatother

\IfFileExists{\jobname-pw.ind}{\input{\jobname-pw.ind}}{}

% Quellenangabe nur in der Leseansicht
\ifkorrekturansicht\else
% Fallback-Definitionen, falls die .tex-Datei \titel etc. nicht gesetzt hat
\providecommand{\titel}{}
\providecommand{\editorInnen}{}
\providecommand{\dateiname}{\jobname}

\vspace{3cm}

\vfill

\footnotesize
\textsc{Quelle}: \titel. Herausgegeben von {\editorInnen}. In: \emph{Arthur Schnitzler: Briefwechsel mit Autorinnen und Autoren}.
 Digitale Edition, https://schnitzler-briefe.acdh.oeaw.ac.at/{\dateiname}.html (Stand \today)
\fi

\end{document}


