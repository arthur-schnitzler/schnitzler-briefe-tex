%% latex-korrekturansicht-vorspann.tex
%% Vorspann für die Korrekturansicht.
%% Lädt die gemeinsame Datei latex-vorspann.tex mit gesetztem Schalter.

\newif\ifkorrekturansicht
\korrekturansichttrue

\input{../tex-inputs/latex-vorspann}


\section[Paul Goldmann an Arthur Schnitzler, 18. 6. 1889]{L02640 Paul Goldmann an Arthur Schnitzler, 18. 6. 1889}
\nopagebreak\mylabel{L02640v}
\rehead{ }\normalsize\beginnumbering\briefempfaengerindex{Schnitzler, Arthur@\textsc{Schnitzler, Arthur}!zzzGoldmann, Paul@\emph{von Paul Goldmann}!1889-06-181@{18. 6. 1889}|(be}
\toendnotes[C]{\smallbreak\pagebreak[2]}\Standort{DLA, A:Schnitzler, HS.NZ85.1.3162.}
\physDesc{Brief, 1 Blatt, 2 Seiten, 857 Zeichen
\newline{}Handschrift: blaue Tinte, deutsche Kurrent}\toendnotes[C]{\smallbreak}
\pstart
           \centering{}{\pb}\textcolor{gray}{\textbf{\textbf{Adminiſtration: VII.
                           Seidengaſſe 7\oindex{Seidengasse@\textbf{Seidengasse}, \emph{Straße (K.STR)}|pw}} (Jos. Eberle {\kaufmannsund} Co.\orgindex{Josef Eberle Stein-, Buch und Musikaliendruckerei@Josef Eberle Stein-, Buch und Musikaliendruckerei|pw})}}\pend
           
\pstart
           \centering{}\textcolor{gray}{\textbf{An der Schönen Blauen Donau\orgindex{der schoenen blauen Donau@An der schönen blauen Donau|pw}}}\pend
           
\pstart
           \centering{}\textcolor{gray}{\textbf{Chef-Redacteur: Dr. F.
                        Mamroth\pwindex{Mamroth, Fedor 21.02.1851 – 25.06.1907@\textsc{Mamroth, Fedor} (21.02.1851 – 25.06.1907), \emph{Journalist/Journalistin, Kritiker/Kritikerin}|pw}. – Redaction: IX.,
                        Berggaſſe 31\oindex{Berggasse@\textbf{Berggasse}, \emph{Straße (K.STR)}|pw}.}}\pend
           
\pstart
           \raggedleft{}\textcolor{gray}{\textbf{Wien\oindex{Wien@\textbf{Wien}, \emph{A.ADM2}|pw}, den}}{ }18. Juni \textcolor{gray}{\textbf{18}}89.\pend
           
\pstart\center{}Sehr geehrter Herr Doctor!\pend\vspace{0.5em}
\pstart
           Die zwei vermißten \label{K_L02640-1v}\edtext{Gedichte\pwindex{Lieder eines Nervoesen@\emph{Lieder eines Nervösen}|pwv}}{\lemma{\textnormal{\emph{Gedichte}}}\Cendnote{\textnormal{Unter dem Pseudonym »Anatol« und mit dem
                  Titel \emph{Lieder eines Nervösen}\pwindex{Lieder eines Nervoesen@\emph{Lieder eines Nervösen}|pwk} erschienen im
                  ersten Juli-Heft von \emph{An der schönen blauen
                     Donau}\pwindex{der schoenen blauen Donau@\emph{An der schönen blauen Donau}|pwk} fünf Gedichte Schnitzlers (Jg. 4, H. 13, S. 297). Welche davon kurzzeitig vermisst waren,
                  ist nicht geklärt.}}}\label{K_L02640-1} und noch eine Anzahl anderer haben ſich bereits
               gefunden. Ich hatte dieſelben in jenes beſondere Fach unſeres Manuſkripten-Kaſtens
               gelegt, in dem die zum Setzen zu gebenden Beiträge aufbewahrt werden und ſofort,
               nachdem ich dies gethan, daran vergeſſen (wie ich dies mit {\pb}Vorliebe zu thun pflege). Die Sachen hätten ſich
               ohnedies dann bei den Vorabeiten für das nächſte Heft\pwindex{der schoenen blauen Donau@\emph{An der schönen blauen Donau}|pwv} wieder an’s Tageslicht emporgearbeitet. Es
               thut mir nur leid, daß ich Ihnen durch meine Zerſtreutheit einige Stunden der Sorge
               bereitet habe. Ich bitte Sie alſo, vollſtändig beruhigt \introOben{}zu\introOben{}
               ſein. Wenn Sie mir das nächſte Mal wieder das Vergnügen Ihres Beſuches machen werden,
               werden Sie die Kinder ihrer
                  Muſe\pwindex{Lieder eines Nervoesen@\emph{Lieder eines Nervösen}|pwv} friſch, geſund und unbeſchädigt von Angeſicht zu Angeſicht begrüßen
               können. Hochachtungsvoll\pend
           
\pstart
           Ihr ergeb\textcolor{gray}{ner}{\\[\baselineskip]}\spacefill\mbox{Dr. Paul Goldmann}\pend
           \leftskip=0em{}\selectlanguage{ngerman}\endnumbering\briefempfaengerindex{Schnitzler, Arthur@\textsc{Schnitzler, Arthur}!zzzGoldmann, Paul@\emph{von Paul Goldmann}!1889-06-181@{18. 6. 1889}|)be}\mylabel{L02640h}  \normalsize

\doendnotes{C}
\bigskip
\vfill

\clearpage

\footnotesize

\lohead{\textsc{register}}

% Definiere theindex-Environment komplett neu ohne reledmac
\makeatletter
\renewenvironment{theindex}{%
  \section*{\indexname}%
  \setlength{\parindent}{0pt}%
  \setlength{\parskip}{0pt plus 0.3pt}%
  \let\item\@idxitem
}{%
  \clearpage
}
\makeatother

\IfFileExists{\jobname-pw.ind}{\input{\jobname-pw.ind}}{}

\end{document}

      