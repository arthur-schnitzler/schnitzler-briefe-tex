%% latex-leseansicht-vorspann.tex
%% Vorspann für die Leseansicht.
%% Lädt die gemeinsame Datei latex-vorspann.tex mit nicht gesetztem Schalter.

\newif\ifkorrekturansicht
\korrekturansichtfalse

\input{../tex-inputs/latex-vorspann}


         
         \newcommand{\erwaehntePersonen}{Personen: Gertrude von Hofmannsthal, Olga Schnitzler}
         \newcommand{\erwaehnteInstitutionen}{}
         \newcommand{\erwaehnteOrte}{Orte: Edmund-Weiß-Gasse, Kaštiel Kuffnerovcov, Sládkovičovo, Wien}
         \newcommand{\erwaehnteWerke}{
               \section[Hugo von Hofmannsthal an Arthur Schnitzler, 8. 1. 190{[}8?{]}]{ Hugo von Hofmannsthal an Arthur Schnitzler, 8. 1. 190{[}8?{]}}\nopagebreak\mylabel{v}\rehead{ }\begin{ledgroupsized}[t]{13cm}\normalsize\beginnumbering \toendnotes[C]{\smallbreak\pagebreak[2]} \Standort{CUL, Schnitzler, B 43.}
\physDesc{Bildpostkarte
\newline{}Handschrift: schwarze Tinte, deutsche Kurrent\newline{}Versand: Stempel: »\nobreak{}\oindex{Sládkovicovo@\textbf{Sládkovičovo}|pwk}Diószeg, 9\textcolor{gray}{08} {[}JAN{]} 10\nobreak{}«.  
\newline{}Schnitzler: mit Bleistift die Jahreszahl ergänzt: »08« und mit »\textsc{Ho}« beschriftet \newline{}Ordnung: 1) mit Bleistift von unbekannter Hand nummeriert: »\strikeout{280}«  2) mit Bleistift von unbekannter Hand nummeriert: »291«}\buchAbdrucke{\weitereDrucke{Hugo von Hofmannsthal, Arthur Schnitzler: \emph{Briefwechsel}. Hg. Therese Nickl und Heinrich Schnitzler. Frankfurt am Main: \emph{S. Fischer} 1964, S. 235.} }\toendnotes[C]{\smallbreak}\pstart{}{\pb}\textsc{Herrn}\pend{}\pstart{}\textsc{D\textsuperscript{r} Arthur Schnitzler}\pend{}\pstart{}Wien\oindex{Wien@\textbf{Wien}|pw}\pend{}\pstart{}XVIII Spöttelgasse 7\oindex{Edmund-Weiss-Gasse@\textbf{Edmund-Weiß-Gasse}|pw}.\pend{}{\bigskip}\pstart
           \noindent{}\centering{}\textcolor{gray}{\textbf{{\pb}\label{K_L01749_1v}\edtext{Ädvözlet
                     Diószegről\oindex{Sládkovicovo@\textbf{Sládkovičovo}|pw}}{\lemma{\textnormal{\emph{Ädvözlet
                     Diószegről}}}\Cendnote{\textnormal{ungarisch:
                        Schöne Grüße aus Diószeg\oindex{Sládkovicovo@\textbf{Sládkovičovo}|pwk} (heute Sládkovičovo\oindex{Sládkovicovo@\textbf{Sládkovičovo}|pwk}).}}}\label{K_L01749_1h}.}}\pend
           \pstart
           \noindent{}\centering{}\textcolor{gray}{\textbf{\label{K_L01749_2v}\edtext{Kastély}{\lemma{\textnormal{\emph{Kastély}}}\Cendnote{\textnormal{Im Kuffner-Schloss\oindex{Kaštiel Kuffnerovcov@\textbf{Kaštiel Kuffnerovcov}|pwk} lebten Verwandte Hofmannsthal\pwindex{Hofmannsthal, Hugo von 1874-02-01 – 1929-07-15@\textsc{Hofmannsthal, Hugo von} (1874-02-01 – 1929-07-15), \emph{Schriftsteller}|pwk}s mütterlicherseits.}}}\label{K_L01749_2h}\oindex{Kaštiel Kuffnerovcov@\textbf{Kaštiel Kuffnerovcov}|pw}.}}\pend
           \pstart
           {\pb}8 I\pend
           \pstart
           lieber, ich Scheuſal denke ja oft an Sie u. ſchreibe Ihnen nie! Wird
               man nicht jetzt bald miteinander ſpazierengehen können?\pend
           \pstart
           Alles Liebe Olga\pwindex{Schnitzler, Olga 17.01.1882 – 13.01.1970@\textsc{Schnitzler, Olga} (17.01.1882 – 13.01.1970), \emph{Schauspielerin, Sängerin}|pw} von uns beiden\pwindex{Hofmannsthal, Gertrude von 16.03.1880 – 09.11.1959@\textsc{Hofmannsthal, Gertrude von} (16.03.1880 – 09.11.1959)|pwv}.\pend
           \pstart Ihr \spacefill\mbox{Hugo.}\pend{}
         
         \endnumbering\mylabel{h}\end{ledgroupsized}  \newcommand{\dateiname}{L01749}\newcommand{\titel}{Hugo von Hofmannsthal an Arthur Schnitzler, 8. 1. 190[8?]}\newcommand{\editorInnen}{Martin Anton Müller und Gerd-Hermann Susen}%% latex-leseansicht-abspann.tex
%% Abspann für die Leseansicht.
%% Der Schalter \ifkorrekturansicht ist bereits durch den Vorspann gesetzt.

%% latex-abspann.tex
%% Gemeinsamer Abspann für Korrekturansicht und Leseansicht.
%% Setzt den Schalter \ifkorrekturansicht voraus (gesetzt in den
%% einbindenden Dateien latex-korrekturansicht-abspann.tex bzw.
%% latex-leseansicht-abspann.tex).
%% ---------------------------------------------------------------

\normalsize

% Das esempio-Environment wird nur in der Leseansicht benötigt
\ifkorrekturansicht\else
\newenvironment{esempio}[3]%
{
    \vspace{1.5ex}
    \rlap{\underline{#1}}
    \par
    \setlength{\parindent}{0cm}
    \nopagebreak
    \leftskip=#2cm
    \rightskip=#3cm
}
{
    \par
}
\fi

\doendnotes{C}
\bigskip
\vfill

\clearpage

\footnotesize

\ifkorrekturansicht
  \lohead{\textsc{register}}
\fi

% theindex-Environment neu definieren ohne reledmac
\makeatletter
\renewenvironment{theindex}{%
  \ifkorrekturansicht
    \section*{\indexname}%
  \else
    \subsubsection*{Index der erwähnten Entitäten}%
  \fi
  \setlength{\parindent}{0pt}%
  \setlength{\parskip}{0pt plus 0.3pt}%
  \let\item\@idxitem
}{%
  \ifkorrekturansicht\clearpage\fi
}
\makeatother

\IfFileExists{\jobname-pw.ind}{\input{\jobname-pw.ind}}{}

% Quellenangabe nur in der Leseansicht
\ifkorrekturansicht\else
% Fallback-Definitionen, falls die .tex-Datei \titel etc. nicht gesetzt hat
\providecommand{\titel}{}
\providecommand{\editorInnen}{}
\providecommand{\dateiname}{\jobname}

\vspace{3cm}

\vfill

\footnotesize
\textsc{Quelle}: \titel. Herausgegeben von {\editorInnen}. In: \emph{Arthur Schnitzler: Briefwechsel mit Autorinnen und Autoren}.
 Digitale Edition, https://schnitzler-briefe.acdh.oeaw.ac.at/{\dateiname}.html (Stand \today)
\fi

\end{document}


      