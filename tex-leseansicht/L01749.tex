%% latex-korrekturansicht-vorspann.tex
%% Vorspann für die Korrekturansicht.
%% Lädt die gemeinsame Datei latex-vorspann.tex mit gesetztem Schalter.

\newif\ifkorrekturansicht
\korrekturansichttrue

\input{../tex-inputs/latex-vorspann}


\section[Hugo von Hofmannsthal an Arthur Schnitzler, 8. 1. 190{[}8?{]}]{L01749 Hugo von Hofmannsthal an Arthur Schnitzler, 8. 1. 190{[}8?{]}}
\nopagebreak\mylabel{L01749v}
\rehead{ }\normalsize\beginnumbering\briefempfaengerindex{Schnitzler, Arthur@\textsc{Schnitzler, Arthur}!zzzHofmannsthal, Hugo von@\emph{von Hugo von Hofmannsthal}!1908-01-081@{8. 1. 190{[}8?{]}}|(be}
\toendnotes[C]{\smallbreak\pagebreak[2]}\Standort{CUL, Schnitzler, B 43.}
\physDesc{Bildpostkarte, 212 Zeichen
\newline{}Handschrift: schwarze Tinte, deutsche Kurrent
\newline{}Versand: Stempel: »\nobreak{}\oindex{Sládkovicovo@\textbf{Sládkovičovo}, \emph{Besiedelter Ort (A.BSO)}|pwk}Diószeg, 9\textcolor{gray}{08}{[}JAN{]} 10\nobreak{}«.  
\newline{}Schnitzler: mit Bleistift die Jahreszahl ergänzt: »08« und mit »\textsc{Ho}« beschriftet 
\newline{}Ordnung: 1) mit Bleistift von unbekannter Hand nummeriert: »\strikeout{280}«  2) mit Bleistift von unbekannter Hand nummeriert: »291«}
\buchAbdrucke{\weitereDrucke{Hugo von Hofmannsthal, Arthur Schnitzler: \emph{Briefwechsel}. Frankfurt am Main: \emph{S. Fischer} 1964, S. 235.} }\toendnotes[C]{\smallbreak}\pstart{}{\pb}\textsc{Herrn}\pend{}\pstart{}\textsc{D\textsuperscript{r} Arthur Schnitzler}\pend{}\pstart{}Wien\oindex{Wien@\textbf{Wien}, \emph{A.ADM2}|pw}\pend{}\pstart{}XVIII Spöttelgasse 7\oindex{Edmund-Weiss-Gasse 7@\textbf{Edmund-Weiß-Gasse 7}, \emph{Wohngebäude (K.WHS)}|pw}.\pend{}{\bigskip}
\pstart
           \noindent{}\centering{}{\pb}\textcolor{gray}{\textbf{\label{K_L01749-1v}\edtext{Ädvözlet Diószegről\oindex{Sládkovicovo@\textbf{Sládkovičovo}, \emph{Besiedelter Ort (A.BSO)}|pw}}{\lemma{\textnormal{\emph{Ädvözlet Diószegről}}}\Cendnote{\textnormal{ungarisch: Schöne Grüße aus Diószeg\oindex{Sládkovicovo@\textbf{Sládkovičovo}, \emph{Besiedelter Ort (A.BSO)}|pwk} (heute Sládkovičovo\oindex{Sládkovicovo@\textbf{Sládkovičovo}, \emph{Besiedelter Ort (A.BSO)}|pwk})}}}\label{K_L01749-1}.}}\pend
           
\pstart
           \centering{}\textcolor{gray}{\textbf{\label{K_L01749-2v}\edtext{Kastély}{\lemma{\textnormal{\emph{Kastély}}}\Cendnote{\textnormal{Im Kuffner-Schloss\oindex{Kaštiel Kuffnerovcov@\textbf{Kaštiel Kuffnerovcov}, \emph{Schloss (K.SLS)}|pwk} lebten Verwandte Hofmannsthals\pwindex{Hofmannsthal, Hugo von 1874-02-01 – 1929-07-15@\textsc{Hofmannsthal, Hugo von} (1874-02-01 – 1929-07-15), \emph{Schriftsteller/Schriftstellerin}|pwk}s mütterlicherseits.}}}\label{K_L01749-2}\oindex{Kaštiel Kuffnerovcov@\textbf{Kaštiel Kuffnerovcov}, \emph{Schloss (K.SLS)}|pw}.}}\pend
           \vspace{1em}
\pstart
           {\pb}8 I\pend
           \vspace{0.5em}
\pstart
           lieber, ich Scheuſal denke ja oft an Sie u. ſchreibe Ihnen nie! Wird
               man nicht jetzt bald miteinander ſpazierengehen können?\pend
           
\pstart
           Alles Liebe Olga\pwindex{Schnitzler, Olga 17.01.1882 – 13.01.1970@\textsc{Schnitzler, Olga} (17.01.1882 – 13.01.1970), \emph{Schauspieler/Schauspielerin, Sänger/Sängerin}|pw} von uns beiden\pwindex{Hofmannsthal, Gertrude von 16.03.1880 – 09.11.1959@\textsc{Hofmannsthal, Gertrude von} (16.03.1880 – 09.11.1959)|pwv}.\pend
           \pstart Ihr \spacefill\mbox{Hugo.}\pend{}\selectlanguage{ngerman}\endnumbering\briefempfaengerindex{Schnitzler, Arthur@\textsc{Schnitzler, Arthur}!zzzHofmannsthal, Hugo von@\emph{von Hugo von Hofmannsthal}!1908-01-081@{8. 1. 190{[}8?{]}}|)be}\mylabel{L01749h}  \normalsize

\doendnotes{C}
\bigskip
\vfill

\clearpage

\footnotesize

\lohead{\textsc{register}}

% Definiere theindex-Environment komplett neu ohne reledmac
\makeatletter
\renewenvironment{theindex}{%
  \section*{\indexname}%
  \setlength{\parindent}{0pt}%
  \setlength{\parskip}{0pt plus 0.3pt}%
  \let\item\@idxitem
}{%
  \clearpage
}
\makeatother

\IfFileExists{\jobname-pw.ind}{\input{\jobname-pw.ind}}{}

\end{document}

      