%% latex-korrekturansicht-vorspann.tex
%% Vorspann für die Korrekturansicht.
%% Lädt die gemeinsame Datei latex-vorspann.tex mit gesetztem Schalter.

\newif\ifkorrekturansicht
\korrekturansichttrue

\input{../tex-inputs/latex-vorspann}


\section[ Felix Salten an Arthur Schnitzler, 3. 6. 1904]{L03397 Felix Salten an Arthur Schnitzler, 3. 6. 1904}
\nopagebreak\mylabel{L03397v}
\rehead{ }\normalsize\beginnumbering\briefempfaengerindex{Schnitzler, Arthur@\textsc{Schnitzler, Arthur}!zzzSalten, Felix@\emph{von Felix Salten}!1904-06-031@{3. 6. 1904}|(be}
\toendnotes[C]{\smallbreak\pagebreak[2]}\Standort{CUL, Schnitzler, B 89, B 1.}
\physDesc{Brief, 1 Blatt, 1 Seite, 250 Zeichen
\newline{}Handschrift: Bleistift, lateinische Kurrent
\newline{}Schnitzler: mit Bleistift Saltens Adresse vermerkt: »Starkfriedg 12\oindex{Starkfriedgassse@\textbf{Starkfriedgassse}, \emph{Straße (K.STR)}|pw}« 
\newline{}Ordnung: mit Bleistift von unbekannter Hand nummeriert: »189« }\toendnotes[C]{\smallbreak}
\pstart
           \raggedleft{}{\pb}3. VI. 04\pend
           \vspace{0.5em}
\pstart
           Lieber, wir könnten, wenn es Ihnen recht ist, \label{K_L03397-1v}\edtext{an einem der nächsten Nachmittage in
               unserem Garten sein, oder im Wald spazieren gehen und dann beim Straßer\oindex{Zum weissen Lamm@\textbf{Zum weißen Lamm}, \emph{Gastgewerbegebäude (K.GGW)}|pw} (lieber aber bei uns\pwindex{Salten, Ottilie 07.03.1868 – 22.06.1942@\textsc{Salten, Ottilie} (07.03.1868 – 22.06.1942), \emph{Schauspieler/Schauspielerin}|pwv}) nachtmahlen}{\lemma{\textnormal{\emph{an … nachtmahlen}}}\Cendnote{\textnormal{In der Starkfriedgasse 12\oindex{Starkfriedgassse@\textbf{Starkfriedgassse}, \emph{Straße (K.STR)}|pwk} befand sich
                     1904 der Sommersitz von Felix\pwindex{Salten, Felix 06.09.1869 – 08.10.1945@\textsc{Salten, Felix} (06.09.1869 – 08.10.1945), \emph{Schriftsteller/Schriftstellerin, Journalist/Journalistin, Chefredakteur/Chefredakteurin}|pwk} und Ottilie Salten\pwindex{Salten, Ottilie 07.03.1868 – 22.06.1942@\textsc{Salten, Ottilie} (07.03.1868 – 22.06.1942), \emph{Schauspieler/Schauspielerin}|pwk}. Am
                  Vormittag des 5. 6. 1904 kam Schnitzler zu
                  Besuch, dürfte dort aber nur Ottilie Salten\pwindex{Salten, Ottilie 07.03.1868 – 22.06.1942@\textsc{Salten, Ottilie} (07.03.1868 – 22.06.1942), \emph{Schauspieler/Schauspielerin}|pwk}
                  angetroffen haben. Am Nachmittag war Schnitzler neuerlich in der unmittelbaren Nähe: Er war mit seiner Frau
                     Olga\pwindex{Schnitzler, Olga 17.01.1882 – 13.01.1970@\textsc{Schnitzler, Olga} (17.01.1882 – 13.01.1970), \emph{Schauspieler/Schauspielerin, Sänger/Sängerin}|pwk} im Weißen Lamm\oindex{Zum weissen Lamm@\textbf{Zum weißen Lamm}, \emph{Gastgewerbegebäude (K.GGW)}|pwk} (auch als Straßer-Wirt\oindex{Zum weissen Lamm@\textbf{Zum weißen Lamm}, \emph{Gastgewerbegebäude (K.GGW)}|pwk}
                  bekannt). Wahrscheinlich erfolgte das ohne Salten\pwindex{Salten, Felix 06.09.1869 – 08.10.1945@\textsc{Salten, Felix} (06.09.1869 – 08.10.1945), \emph{Schriftsteller/Schriftstellerin, Journalist/Journalistin, Chefredakteur/Chefredakteurin}|pwk}.}}}\label{K_L03397-1}. Schreiben Sie mir nur vorher eine Zeile.\pend
           
\pstart
           herzlichst {\\[\baselineskip]}Ihr \spacefill\mbox{S.}\pend
           \leftskip=0em{}\selectlanguage{ngerman}\endnumbering\briefempfaengerindex{Schnitzler, Arthur@\textsc{Schnitzler, Arthur}!zzzSalten, Felix@\emph{von Felix Salten}!1904-06-031@{3. 6. 1904}|)be}\mylabel{L03397h}  \normalsize

\doendnotes{C}
\bigskip
\vfill

\clearpage

\footnotesize

\lohead{\textsc{register}}

% Definiere theindex-Environment komplett neu ohne reledmac
\makeatletter
\renewenvironment{theindex}{%
  \section*{\indexname}%
  \setlength{\parindent}{0pt}%
  \setlength{\parskip}{0pt plus 0.3pt}%
  \let\item\@idxitem
}{%
  \clearpage
}
\makeatother

\IfFileExists{\jobname-pw.ind}{\input{\jobname-pw.ind}}{}

\end{document}

      