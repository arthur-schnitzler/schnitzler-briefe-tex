%% latex-leseansicht-vorspann.tex
%% Vorspann für die Leseansicht.
%% Lädt die gemeinsame Datei latex-vorspann.tex mit nicht gesetztem Schalter.

\newif\ifkorrekturansicht
\korrekturansichtfalse

\input{../tex-inputs/latex-vorspann}


\section[Robert Adam an Arthur Schnitzler, 21. 10. 1915]{L02220 Robert Adam an Arthur Schnitzler, 21. 10. 1915}
\nopagebreak\mylabel{L02220v}
\rehead{ }\normalsize\beginnumbering\briefempfaengerindex{Schnitzler, Arthur@\textsc{Schnitzler, Arthur}!zzzAdam, Robert@\emph{von Robert Adam}!1915-10-211@{21. 10. 1915}|(be}
\toendnotes[C]{\smallbreak\pagebreak[2]}
\correspDesc{Versand  durch Robert Adam am 21. 10. 1915 in Wien
\newline{}Erhalt  durch Arthur Schnitzler im Zeitraum [21. 10. 1915 – 25. 10. 1915?] in Wien}\toendnotes[C]{\smallbreak}
\Standort{DLA, A:Schnitzler, HS.NZ85.1.4230,12.}
\physDesc{Brief, 1 Blatt, 2 Seiten, 915 Zeichen
\newline{}Handschrift: schwarze Tinte, deutsche Kurrent
\newline{}Schnitzler: 1) mit Bleistift beschriftet: »\textsc{Adam}«  2) mit rotem Buntstift eine Unterstreichung}\Standort{Wien, Österreichische Nationalbibliothek, Cod.ser. 52.267, 119.}
\physDesc{Briefentwurf, maschinenschriftliche Abschrift, 1 Blatt, 1 Seite, 915 Zeichen
\newline{}Schreibmaschine}\toendnotes[C]{\smallbreak}
\pstart
           \raggedleft{}{\pb}Wien\oindex{Wien@\textbf{Wien}, \emph{Verwaltungsgebiet}|pw}, am 21. Oktober 1915\pend
           
\pstart{}Hochverehrter Herr Doktor!\pend\vspace{0.5em}
\pstart
           Vom Büreau heimkehrend, finde ich Ihre »Komödie der
                  Worte\pwindex{Schnitzler, Arthur 15.\,5.\,1862 Wien – 21.\,10.\,1931 ebd.@\textsc{Schnitzler, Arthur} (15.\,5.\,1862 Wien – 21.\,10.\,1931 ebd.), \emph{Schriftsteller, Mediziner}!Komödie der Worte. Drei Einakter@\strich\emph{Komödie der Worte. Drei Einakter}|pw}« mit Ihren mich hocherfreuenden Zeile vor.\pend
           
\pstart
           Ich beeile mich, Ihnen für Widmung und Buch auf’s Herzlichſte zu danken.\pend
           
\pstart
           Ich glaube in der Überſendung nicht bloß ein liebenswürdiges Zeichen dafür erblicken
               zu dürfen, daß Sie meiner gedenken,{ }ſondern auch dafür, daß Sie an meinem
               Dichterſchickſal noch nicht verzweifeln: und dies iſt mir juſt in dieſen Tagen, da
               ich in allem, was ich bisher{ }ſchaffte, nur die Beſtätigung eines ruheloſen und der
               richtigen Selbſtkritik {\pb}entſtehenden Dilettantismus
               erblicken zu müſſen meinte, Ermunterung und Tröſtung.\pend
           
\pstart
           Möge Ihrer Komödie\pwindex{Schnitzler, Arthur 15.\,5.\,1862 Wien – 21.\,10.\,1931 ebd.@\textsc{Schnitzler, Arthur} (15.\,5.\,1862 Wien – 21.\,10.\,1931 ebd.), \emph{Schriftsteller, Mediziner}!Komödie der Worte. Drei Einakter@\strich\emph{Komödie der Worte. Drei Einakter}|pwv} trotz
               dieser kunſt- und kulturfeindlichen Zeit ein freundliches Geſchick zuteil
               werden! –\pend
           
\pstart
           Ich werde mir erlauben, Ihnen für Ihre Liebenswürdigkeit, wenn Sie es geſtatten,
               demnächſt perſönlich zu danken.\pend
           
\pstart
           Mit den beſten Grüßen Ihr{ }ſehr ergebener{\\[\baselineskip]}\spacefill\mbox{D\textsuperscript{r}RAdam}\pend
           \leftskip=0em{}\selectlanguage{ngerman}\endnumbering\briefempfaengerindex{Schnitzler, Arthur@\textsc{Schnitzler, Arthur}!zzzAdam, Robert@\emph{von Robert Adam}!1915-10-211@{21. 10. 1915}|)be}\mylabel{L02220h}  \newcommand{\dateiname}{L02220}\newcommand{\titel}{Robert Adam an Arthur Schnitzler, 21. 10. 1915}\newcommand{\editorInnen}{Martin Anton Müller und Gerd-Hermann Susen}%% latex-leseansicht-abspann.tex
%% Abspann für die Leseansicht.
%% Der Schalter \ifkorrekturansicht ist bereits durch den Vorspann gesetzt.

%% latex-abspann.tex
%% Gemeinsamer Abspann für Korrekturansicht und Leseansicht.
%% Setzt den Schalter \ifkorrekturansicht voraus (gesetzt in den
%% einbindenden Dateien latex-korrekturansicht-abspann.tex bzw.
%% latex-leseansicht-abspann.tex).
%% ---------------------------------------------------------------

\normalsize

% Das esempio-Environment wird nur in der Leseansicht benötigt
\ifkorrekturansicht\else
\newenvironment{esempio}[3]%
{
    \vspace{1.5ex}
    \rlap{\underline{#1}}
    \par
    \setlength{\parindent}{0cm}
    \nopagebreak
    \leftskip=#2cm
    \rightskip=#3cm
}
{
    \par
}
\fi

\doendnotes{C}
\bigskip
\vfill

\clearpage

\footnotesize

\ifkorrekturansicht
  \lohead{\textsc{register}}
\fi

% theindex-Environment neu definieren ohne reledmac
\makeatletter
\renewenvironment{theindex}{%
  \ifkorrekturansicht
    \section*{\indexname}%
  \else
    \subsubsection*{Index der erwähnten Entitäten}%
  \fi
  \setlength{\parindent}{0pt}%
  \setlength{\parskip}{0pt plus 0.3pt}%
  \let\item\@idxitem
}{%
  \ifkorrekturansicht\clearpage\fi
}
\makeatother

\IfFileExists{\jobname-pw.ind}{\input{\jobname-pw.ind}}{}

% Quellenangabe nur in der Leseansicht
\ifkorrekturansicht\else
% Fallback-Definitionen, falls die .tex-Datei \titel etc. nicht gesetzt hat
\providecommand{\titel}{}
\providecommand{\editorInnen}{}
\providecommand{\dateiname}{\jobname}

\vspace{3cm}

\vfill

\footnotesize
\textsc{Quelle}: \titel. Herausgegeben von {\editorInnen}. In: \emph{Arthur Schnitzler: Briefwechsel mit Autorinnen und Autoren}.
 Digitale Edition, https://schnitzler-briefe.acdh.oeaw.ac.at/{\dateiname}.html (Stand \today)
\fi

\end{document}


