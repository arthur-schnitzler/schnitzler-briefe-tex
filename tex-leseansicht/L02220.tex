\input{../tex-inputs/latex-pdf-vorspann}
\begin{center}
            \textcolor{red}{ENTWURF. ENTZIFFERUNG NOCH NICHT KORREKTURGELESEN}
                      \end{center}
            
               \section[Robert Adam an Arthur Schnitzler, 21. 10. 1915]{ Robert Adam an Arthur Schnitzler, 21. 10. 1915}\nopagebreak\mylabel{v}\rehead{ }\begin{ledgroupsized}[t]{13cm}\normalsize\beginnumbering\briefempfaengerindex{Schnitzler, Arthur@\textsc{Schnitzler, Arthur}!zzzAdam, Robert@\emph{von Robert Adam}!1915-10-211@{21. 10. 1915}|(be} \toendnotes[C]{\smallbreak\pagebreak[2]} \Standort{DLA, A:Schnitzler, HS.NZ85.1.4230,12.}
\physDesc{Brief, 1 Blatt, 2 Seiten
\newline{}Handschrift: schwarze Tinte, deutsche Kurrent
\newline{}Schnitzler: 1) mit Bleistift beschriftet: »\textsc{Adam}« 2) mit rotem Buntstift eine Unterstreichung}\Standort{Wien, Österreichische Nationalbibliothek, Cod.ser. 52.267, 119.}
\physDesc{Brief, maschinelle Abschrift, Entwurf}\toendnotes[C]{\smallbreak}\pstart
           \raggedleft{}{\pb}Wien\oindex{Wien@\textbf{Wien}|pw}, am 21. Oktober 1915\pend
           \pstart{}Hochverehrter Herr Doktor!\pend\pstart
           Vom Büreau heimkehrend, finde ich Ihre »Komödie der
                        Worte\pwindex{Schnitzler, Arthur 15.05.1862 – 21.10.1931@\textsc{Schnitzler, Arthur} (15.05.1862 – 21.10.1931), \emph{Schriftsteller, Mediziner}!Komoedie der Worte. Drei Einakter1915@\strich\emph{Komödie der Worte. Drei Einakter} {[}1915{]}|pw}« mit Ihren mich hocherfreuenden Zeile vor.\pend
           \pstart
           Ich beeile mich, Ihnen für Widmung und Buch auf’s Herzlichſte zu danken.\pend
           \pstart
           Ich glaube in der Überſendung nicht bloß ein liebenswürdiges Zeichen dafür
                    erblicken zu dürfen, daß Sie meiner gedenken, ſondern auch dafür, daß Sie an
                    meinem Dichterſchickſal noch nicht verzweifeln: und dies iſt mir juſt in dieſen
                    Tagen, da ich in allem, was ich bisher ſchaffte, nur die Beſtätigung eines
                    ruheloſen und der richtigen Selbſtkritik {\pb}entſtehenden Dilettantismus erblicken zu müſſen meinte, Ermunterung und
                    Tröſtung.\pend
           \pstart
           Möge Ihrer Komödie\pwindex{Schnitzler, Arthur 15.05.1862 – 21.10.1931@\textsc{Schnitzler, Arthur} (15.05.1862 – 21.10.1931), \emph{Schriftsteller, Mediziner}!Komoedie der Worte. Drei Einakter1915@\strich\emph{Komödie der Worte. Drei Einakter} {[}1915{]}|pwv} trotz
                    dieser kunſt- und kulturfeindlichen Zeit ein freundliches Geſchick zuteil
                    werden! –\pend
           \pstart
           Ich werde mir erlauben, Ihnen für Ihre Liebenswürdigkeit, wenn Sie es geſtatten,
                    demnächſt perſönlich zu danken.\pend
           \pstart
           Mit den beſten Grüßen Ihr ſehr ergebener\pend
           \pstart D\textsuperscript{r}RAdam \pend{}\endnumbering\briefempfaengerindex{Schnitzler, Arthur@\textsc{Schnitzler, Arthur}!zzzAdam, Robert@\emph{von Robert Adam}!1915-10-211@{21. 10. 1915}|)be}\mylabel{h}\end{ledgroupsized}  \newcommand{\dateiname}{L02220}\newcommand{\titel}{Robert Adam an Arthur Schnitzler, 21. 10. 1915}\newcommand{\editorInnen}{Martin Anton Müller und Gerd-Hermann Susen}\input{../tex-inputs/latex-pdf-abspann}
      