%% latex-leseansicht-vorspann.tex
%% Vorspann für die Leseansicht.
%% Lädt die gemeinsame Datei latex-vorspann.tex mit nicht gesetztem Schalter.

\newif\ifkorrekturansicht
\korrekturansichtfalse

\input{../tex-inputs/latex-vorspann}


\section[Theodor Herzl an Arthur Schnitzler, 20. 7. 1895]{L03863 Theodor Herzl an Arthur Schnitzler, 20. 7. 1895}
\nopagebreak\mylabel{L03863v}
\rehead{ }\normalsize\beginnumbering\briefempfaengerindex{Schnitzler, Arthur@\textsc{Schnitzler, Arthur}!zzzHerzl, Theodor@\emph{von Theodor Herzl}!1895-07-201@{20. 7. 1895}|(be}
\toendnotes[C]{\smallbreak\pagebreak[2]}
\correspDesc{Versand  durch Theodor Herzl am 20. 7. 1895 in Paris
\newline{}Erhalt  durch Arthur Schnitzler im Zeitraum [21. 7. 1895 – 25. 7. 1895?] in Wien}\toendnotes[C]{\smallbreak}
\Standort{CUL, Schnitzler, B 39.}
\physDesc{Brief, 1 Blatt, 1 Seite, 331 Zeichen
\newline{}Handschrift: blaue Tinte, lateinische Kurrent
\newline{}Ordnung: mit Bleistift von unbekannter Hand nummeriert: »42« }
\buchAbdrucke{\weitereDrucke{Theodor Herzl: \emph{Briefe Anfang Mai 1895 – Anfang Dezember 1898}. Bearbeitet von Barbara Schäfer in Zusammenarbeit mit Sofia Gelmann, Chaya Harel, Ines Rubin und Daisy Ticho. Berlin, Frankfurt am Main, Wien: \emph{Propyläen} 1990, S. 59 (Briefe und Tagebücher. Herausgegeben von Alex Bein, Hermann Greive, Moshe Schaerf, Julius H. Schoeps und Johannes Wachten, 4).} }\toendnotes[C]{\smallbreak}
\pstart
           \raggedleft{}{\pb}Paris\oindex{Paris@\textbf{Paris}, \emph{Hauptstadt}|pw}{ }20. VII. 95\pend
           
\pstart{}Mein lieber Schnitzler,\pend\vspace{0.5em}
\pstart
           Dank für Ihre lieben Zeilen.
      Ich hoffe Anfangs August in
               Aussee\oindex{Bad Aussee@\textbf{Bad Aussee}, \emph{Hauptstadt}|pw} zu sein u. werde
      Ihnen dann einen Wink
      mit dem Zaunpfahl geben.
      Sie können ja dort leichter
      zu mir kommen als ich zu
      Ihnen, denn ich bin ein
      schwerfälliger \label{K_L03863-1v}\edtext{bonus
               paterfamilias}{\lemma{\textnormal{\emph{bonus
               paterfamilias}}}\Cendnote{\textnormal{lateinisch bonus pater familias: guter Familienvater}}}\label{K_L03863-1}.\pend
           
\pstart
           Wir plauschen uns mündlich
      aus\pend
           
\pstart
           herzlich der Ihrige{\\[\baselineskip]}\spacefill\mbox{Th Herzl}\pend
           \leftskip=0em{}\selectlanguage{ngerman}\endnumbering\briefempfaengerindex{Schnitzler, Arthur@\textsc{Schnitzler, Arthur}!zzzHerzl, Theodor@\emph{von Theodor Herzl}!1895-07-201@{20. 7. 1895}|)be}\mylabel{L03863h}
\begin{anhang}
\end{anhang}\newcommand{\dateiname}{L03863}\newcommand{\titel}{Theodor Herzl an Arthur Schnitzler, 20. 7. 1895}\newcommand{\editorInnen}{Selma Jahnke und Martin Anton Müller}%% latex-leseansicht-abspann.tex
%% Abspann für die Leseansicht.
%% Der Schalter \ifkorrekturansicht ist bereits durch den Vorspann gesetzt.

%% latex-abspann.tex
%% Gemeinsamer Abspann für Korrekturansicht und Leseansicht.
%% Setzt den Schalter \ifkorrekturansicht voraus (gesetzt in den
%% einbindenden Dateien latex-korrekturansicht-abspann.tex bzw.
%% latex-leseansicht-abspann.tex).
%% ---------------------------------------------------------------

\normalsize

% Das esempio-Environment wird nur in der Leseansicht benötigt
\ifkorrekturansicht\else
\newenvironment{esempio}[3]%
{
    \vspace{1.5ex}
    \rlap{\underline{#1}}
    \par
    \setlength{\parindent}{0cm}
    \nopagebreak
    \leftskip=#2cm
    \rightskip=#3cm
}
{
    \par
}
\fi

\doendnotes{C}
\bigskip
\vfill

\clearpage

\footnotesize

\ifkorrekturansicht
  \lohead{\textsc{register}}
\fi

% theindex-Environment neu definieren ohne reledmac
\makeatletter
\renewenvironment{theindex}{%
  \ifkorrekturansicht
    \section*{\indexname}%
  \else
    \subsubsection*{Index der erwähnten Entitäten}%
  \fi
  \setlength{\parindent}{0pt}%
  \setlength{\parskip}{0pt plus 0.3pt}%
  \let\item\@idxitem
}{%
  \ifkorrekturansicht\clearpage\fi
}
\makeatother

\IfFileExists{\jobname-pw.ind}{\input{\jobname-pw.ind}}{}

% Quellenangabe nur in der Leseansicht
\ifkorrekturansicht\else
% Fallback-Definitionen, falls die .tex-Datei \titel etc. nicht gesetzt hat
\providecommand{\titel}{}
\providecommand{\editorInnen}{}
\providecommand{\dateiname}{\jobname}

\vspace{3cm}

\vfill

\footnotesize
\textsc{Quelle}: \titel. Herausgegeben von {\editorInnen}. In: \emph{Arthur Schnitzler: Briefwechsel mit Autorinnen und Autoren}.
 Digitale Edition, https://schnitzler-briefe.acdh.oeaw.ac.at/{\dateiname}.html (Stand \today)
\fi

\end{document}


