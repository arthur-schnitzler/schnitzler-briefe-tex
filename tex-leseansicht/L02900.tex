%% latex-leseansicht-vorspann.tex
%% Vorspann für die Leseansicht.
%% Lädt die gemeinsame Datei latex-vorspann.tex mit nicht gesetztem Schalter.

\newif\ifkorrekturansicht
\korrekturansichtfalse

\input{../tex-inputs/latex-vorspann}


\section[ Paul Goldmann an Arthur Schnitzler, 23. 12. [1899]]{L02900 Paul Goldmann an Arthur Schnitzler,  23. 12. [1899]}
\nopagebreak\mylabel{L02900v}
\rehead{ }\normalsize\beginnumbering\briefempfaengerindex{Schnitzler, Arthur@\textsc{Schnitzler, Arthur}!zzzGoldmann, Paul@\emph{von Paul Goldmann}!1899-12-231@{23. 12. [1899]}|(be}
\toendnotes[C]{\smallbreak\pagebreak[2]}
\correspDesc{Versand  durch Paul Goldmann am 23. 12. [1899] in Frankfurt am Main
\newline{}Erhalt  durch Arthur Schnitzler im Zeitraum [24. 12. 1899 – 28. 12. 1899?] in Wien}\toendnotes[C]{\smallbreak}
\Standort{DLA, A:Schnitzler, HS.NZ85.1.3169.}
\physDesc{Brief, 1 Blatt, 4 Seiten, 3956 Zeichen
\newline{}Handschrift: blaue Tinte, deutsche Kurrent
\newline{}Schnitzler: mit rotem Buntstift das Jahr »99« vermerkt und vier Unterstreichungen }\toendnotes[C]{\smallbreak}
\pstart
           \centering{}{\pb}Frankfurt\oindex{Frankfurt am Main@\textbf{Frankfurt am Main}, \emph{Hauptstadt}|pw}, 23. Dezember.\pend
           
\pstart{}Mein lieber Freund,\pend\vspace{0.5em}
\pstart
           Ich habe Deine lieben Nachrichten lange vermißt und war{ }ſehr froh, wieder ausführlich
               von Dir zu hören.\pend
           
\pstart
           Wenn Du die \label{K_L02900-1v}\edtext{»\textsc{Beatrice\pwindex{Schnitzler, Arthur 15.\,5.\,1862 Wien – 21.\,10.\,1931 ebd.@\textsc{Schnitzler, Arthur} (15.\,5.\,1862 Wien – 21.\,10.\,1931 ebd.), \emph{Schriftsteller, Mediziner}!Schleier der Beatrice. Schauspiel in fünf Akten@\strich\emph{Der Schleier der Beatrice. Schauspiel in fünf Akten}|pw}}« drucken}{\lemma{\textnormal{\emph{»Beatrice« drucken}}}\Cendnote{\textnormal{\emph{Der Schleier der Beatrice}\pwindex{Schnitzler, Arthur 15.\,5.\,1862 Wien – 21.\,10.\,1931 ebd.@\textsc{Schnitzler, Arthur} (15.\,5.\,1862 Wien – 21.\,10.\,1931 ebd.), \emph{Schriftsteller, Mediziner}!Schleier der Beatrice. Schauspiel in fünf Akten@\strich\emph{Der Schleier der Beatrice. Schauspiel in fünf Akten}|pwk} wurde 1900 zuerst für die Bühnen gedruckt (bei \emph{A. Entsch}\orgindex{A. Entsch@A. Entsch|pwk}), mit Jahresbeginn 1901 war es dann bei \emph{S. Fischer}\orgindex{S. Fischer Verlag@S. Fischer Verlag|pwk}
                  verfügbar.}}}\label{K_L02900-1} läßt, werde ich{ }ſie hoffentlich bald zu leſen bekommen. Wie{ }ſtehen die \label{K_L02900-2v}\edtext{Aufführungs-Chancen beim
                  Burgtheater\orgindex{Burgtheater@Burgtheater|pw}}{\lemma{\textnormal{\emph{Aufführungs-Chancen beim Burgtheater}}}\Cendnote{\textnormal{Siehe XXXX Auszeichnungsfehler: Dokument L02893 nicht gefunden.
               }}}\label{K_L02900-2}? Und wie in \label{K_L02900-3v}\edtext{Berlin\oindex{Berlin@\textbf{Berlin}, \emph{Hauptstadt}|pw}}{\lemma{\textnormal{\emph{Berlin}}}\Cendnote{\textnormal{Am Deutschen Theater\oindex{Deutsches Theater Berlin@\textbf{Deutsches Theater Berlin}, \emph{Theater}|pwk} feierte \emph{Der Schleier
                     der Beatrice}\pwindex{Schnitzler, Arthur 15.\,5.\,1862 Wien – 21.\,10.\,1931 ebd.@\textsc{Schnitzler, Arthur} (15.\,5.\,1862 Wien – 21.\,10.\,1931 ebd.), \emph{Schriftsteller, Mediziner}!Schleier der Beatrice. Schauspiel in fünf Akten@\strich\emph{Der Schleier der Beatrice. Schauspiel in fünf Akten}|pwk} – obwohl Otto Brahm\pwindex{Brahm, Otto 5.\,2.\,1856 Hamburg – 28.\,11.\,1912 Berlin@\textsc{Brahm, Otto} (5.\,2.\,1856 Hamburg – 28.\,11.\,1912 Berlin), \emph{Theaterleiter, Regisseur}|pwk}
                  das Stück\pwindex{Schnitzler, Arthur 15.\,5.\,1862 Wien – 21.\,10.\,1931 ebd.@\textsc{Schnitzler, Arthur} (15.\,5.\,1862 Wien – 21.\,10.\,1931 ebd.), \emph{Schriftsteller, Mediziner}!Schleier der Beatrice. Schauspiel in fünf Akten@\strich\emph{Der Schleier der Beatrice. Schauspiel in fünf Akten}|pwkv} seit einer
                  persönlichen Lesung durch den Autor am 7. 10. 1899 kannte – erst am 7. 3. 1903 Premiere.}}}\label{K_L02900-3}? Deutſches Theater\orgindex{Deutsches Theater Berlin@Deutsches Theater Berlin|pw} oder Schauſpielhaus\orgindex{Schauspielhaus Berlin@Schauspielhaus Berlin|pw}? Vielleicht wird es eine meiner erſten Aufgaben{ }ſein, über
               eine \begin{otherlanguage}{french}\textsc{Première}\end{otherlanguage}{ }von Dir zu berichten. Iſt der \label{K_L02900-4v}\edtext{»Reigen\pwindex{Schnitzler, Arthur 15.\,5.\,1862 Wien – 21.\,10.\,1931 ebd.@\textsc{Schnitzler, Arthur} (15.\,5.\,1862 Wien – 21.\,10.\,1931 ebd.), \emph{Schriftsteller, Mediziner}!Reigen. Zehn Dialoge@\strich\emph{Reigen. Zehn Dialoge}|pw}«{ }ſchon gedruckt?}{\lemma{\textnormal{\emph{»Reigen« schon gedruckt?}}}\Cendnote{\textnormal{Ein
                  Privatdruck des \emph{Reigen}\pwindex{Schnitzler, Arthur 15.\,5.\,1862 Wien – 21.\,10.\,1931 ebd.@\textsc{Schnitzler, Arthur} (15.\,5.\,1862 Wien – 21.\,10.\,1931 ebd.), \emph{Schriftsteller, Mediziner}!Reigen. Zehn Dialoge@\strich\emph{Reigen. Zehn Dialoge}|pwk} für die Verteilung an
                  Freundinnen und Freunde in der Auflage von 200 Stück wurde – betreut vom Verleger
                     Samuel Fischer\pwindex{Fischer, Samuel 24.\,12.\,1859 Liptovský Mikuláš – 15.\,10.\,1934 Berlin@\textsc{Fischer, Samuel} (24.\,12.\,1859 Liptovský Mikuláš – 15.\,10.\,1934 Berlin), \emph{Verleger}|pwk} – zwischen November 1899 und 12. 2. 1900 gedruckt.}}}\label{K_L02900-4}{\dotsfour}\pend
           
\pstart
           In den \label{K_L02900-5v}\edtext{Fragen \textsc{Wassermann\pwindex{Wassermann, Jakob 10.\,3.\,1873 Fürth – 1.\,1.\,1934 Altaussee@\textsc{Wassermann, Jakob} (10.\,3.\,1873 Fürth – 1.\,1.\,1934 Altaussee), \emph{Schriftsteller}|pw}} und \textsc{Schwarzkopf\pwindex{Schwarzkopf, Gustav 7.\,11.\,1853 Wien – 13.\,11.\,1939 ebd.@\textsc{Schwarzkopf, Gustav} (7.\,11.\,1853 Wien – 13.\,11.\,1939 ebd.), \emph{Schriftsteller}|pw}}}{\lemma{\textnormal{\emph{Fragen … Schwarzkopf}}}\Cendnote{\textnormal{Siehe XXXX Auszeichnungsfehler: Dokument L02892 nicht gefunden, XXXX Auszeichnungsfehler: Dokument L02897 nicht gefunden und XXXX Auszeichnungsfehler: Dokument L02898 nicht gefunden.
               }}}\label{K_L02900-5} beharre ich durchaus auf meinem Standpunkte. \textsc{Wassermann}\pwindex{Wassermann, Jakob 10.\,3.\,1873 Fürth – 1.\,1.\,1934 Altaussee@\textsc{Wassermann, Jakob} (10.\,3.\,1873 Fürth – 1.\,1.\,1934 Altaussee), \emph{Schriftsteller}|pw} brauchte das betr. Concert\pwindex{Franckenstein, Clemens von 14.\,7.\,1875 Wiesentheid – 19.\,8.\,1942 Hechendorf am Pilsensee@\textsc{Franckenstein, Clemens von} (14.\,7.\,1875 Wiesentheid – 19.\,8.\,1942 Hechendorf am Pilsensee), \emph{Theaterleiter, Komponist, Dirigent}|pwv} nicht zu übergehen, wenn er{ }ſonſt die Gewohnheit gehabt hätte, über
               Concerte zu berichten. Da er das aber faſt nie thut,{ }ſo iſt die Herausgreifung dieſes
                  \strikeout{\textcolor{gray}{×}} unbedeutenden Concertes\pwindex{Franckenstein, Clemens von 14.\,7.\,1875 Wiesentheid – 19.\,8.\,1942 Hechendorf am Pilsensee@\textsc{Franckenstein, Clemens von} (14.\,7.\,1875 Wiesentheid – 19.\,8.\,1942 Hechendorf am Pilsensee), \emph{Theaterleiter, Komponist, Dirigent}|pwv} aus der ungeheuren Fülle der Wien\oindex{Wien@\textbf{Wien}, \emph{Verwaltungsgebiet}|pw}er Concerte{ }ſchon \strikeout{\textcolor{gray}{an{ }ſich}} eine ungerechte Bevorzugung; und wenn auch das Lob, das er dem Concertgeber\pwindex{Franckenstein, Clemens von 14.\,7.\,1875 Wiesentheid – 19.\,8.\,1942 Hechendorf am Pilsensee@\textsc{Franckenstein, Clemens von} (14.\,7.\,1875 Wiesentheid – 19.\,8.\,1942 Hechendorf am Pilsensee), \emph{Theaterleiter, Komponist, Dirigent}|pwv}{ }ſpendet, an{ }ſich
               nicht übertrieben iſt,{ }ſo wird es übertrieben durch den Tadel gegenüber einem anderen
                  \strikeout{\textcolor{gray}{×}\-\textcolor{gray}{×}} viel bedeutenderen Concertgeber\pwindex{d’Albert, Eugen 10.\,4.\,1864 Glasgow – 3.\,3.\,1932 Riga@\textsc{d’Albert, Eugen} (10.\,4.\,1864 Glasgow – 3.\,3.\,1932 Riga), \emph{Komponist}|pwv}, mit dem \textsc{W}.\pwindex{Wassermann, Jakob 10.\,3.\,1873 Fürth – 1.\,1.\,1934 Altaussee@\textsc{Wassermann, Jakob} (10.\,3.\,1873 Fürth – 1.\,1.\,1934 Altaussee), \emph{Schriftsteller}|pwv} es verbunden hat. Was \textsc{Schwarzkopf\pwindex{Schwarzkopf, Gustav 7.\,11.\,1853 Wien – 13.\,11.\,1939 ebd.@\textsc{Schwarzkopf, Gustav} (7.\,11.\,1853 Wien – 13.\,11.\,1939 ebd.), \emph{Schriftsteller}|pw}} anlangt,{ }ſo kenne ich{ }ſeine bedeutenden Vorzüge. \textsc{Hirschfeld\pwindex{Hirschfeld, Georg 11.\,2.\,1873 Berlin – 17.\,1.\,1942 München@\textsc{Hirschfeld, Georg} (11.\,2.\,1873 Berlin – 17.\,1.\,1942 München), \emph{Schriftsteller}|pw}} wäre trotzdem der beſſere Berichterſtatter, weil er zu allem Anderen \uline{auch} die Muſik umfaßt und weil er {\pb}etwas lebendiger und farbiger{ }ſchreibt als \textsc{Schw}.\pwindex{Schwarzkopf, Gustav 7.\,11.\,1853 Wien – 13.\,11.\,1939 ebd.@\textsc{Schwarzkopf, Gustav} (7.\,11.\,1853 Wien – 13.\,11.\,1939 ebd.), \emph{Schriftsteller}|pwv} Eine Theilung der Berichterſtattung unter die
                  Beiden\pwindex{Schwarzkopf, Gustav 7.\,11.\,1853 Wien – 13.\,11.\,1939 ebd.@\textsc{Schwarzkopf, Gustav} (7.\,11.\,1853 Wien – 13.\,11.\,1939 ebd.), \emph{Schriftsteller}|pwv}\pwindex{Hirschfeld, Georg 11.\,2.\,1873 Berlin – 17.\,1.\,1942 München@\textsc{Hirschfeld, Georg} (11.\,2.\,1873 Berlin – 17.\,1.\,1942 München), \emph{Schriftsteller}|pwv} iſt,
               nach den bei der Frankf. Zeit.\orgindex{Frankfurter Zeitung@Frankfurter Zeitung|pw} beſtehenden
               Einrichtungen, unmöglich. Daß ich die Intereſſen der Frankf. Zeit.\orgindex{Frankfurter Zeitung@Frankfurter Zeitung|pw} vor Allem zu vertreten habe, weiß ich, auch ohne daß Du es mir{ }ſagſt, und ich würde \strikeout{\textcolor{gray}{f}\textcolor{gray}{×}\-\textcolor{gray}{×}}{ }\textsc{Schw}.\pwindex{Schwarzkopf, Gustav 7.\,11.\,1853 Wien – 13.\,11.\,1939 ebd.@\textsc{Schwarzkopf, Gustav} (7.\,11.\,1853 Wien – 13.\,11.\,1939 ebd.), \emph{Schriftsteller}|pwv} niemals \strikeout{\textcolor{gray}{einget}} empfohlen haben, wenn ich \strikeout{\textcolor{gray}{irg}} auch nur einen Augenblick hätte annehmen müſſen, er würde als Correſpondent
               den Intereſſen der Zeitung\orgindex{Frankfurter Zeitung@Frankfurter Zeitung|pwv}
               nicht entſprechen. Es handelt{ }ſich hier um zwei ungefähr gleich würdige Candidaten\pwindex{Schwarzkopf, Gustav 7.\,11.\,1853 Wien – 13.\,11.\,1939 ebd.@\textsc{Schwarzkopf, Gustav} (7.\,11.\,1853 Wien – 13.\,11.\,1939 ebd.), \emph{Schriftsteller}|pwv}\pwindex{Hirschfeld, Georg 11.\,2.\,1873 Berlin – 17.\,1.\,1942 München@\textsc{Hirschfeld, Georg} (11.\,2.\,1873 Berlin – 17.\,1.\,1942 München), \emph{Schriftsteller}|pwv}, und
               wenn irgendwo,{ }ſo \strikeout{\textcolor{gray}{×}\textcolor{gray}{i}\textcolor{gray}{×}} kann hier das Perſönliche interveniren. Ich perſönlich fühle mich, bei aller
               Sympathie und Freundſchaft{ }ſür \textsc{Schw}.\pwindex{Schwarzkopf, Gustav 7.\,11.\,1853 Wien – 13.\,11.\,1939 ebd.@\textsc{Schwarzkopf, Gustav} (7.\,11.\,1853 Wien – 13.\,11.\,1939 ebd.), \emph{Schriftsteller}|pwv}, doch mehr zu \textsc{H}.\pwindex{Hirschfeld, Georg 11.\,2.\,1873 Berlin – 17.\,1.\,1942 München@\textsc{Hirschfeld, Georg} (11.\,2.\,1873 Berlin – 17.\,1.\,1942 München), \emph{Schriftsteller}|pwv} hingezogen. Von Dir weiß ich das Umgekehrte.
               Oder vielmehr ich weiß, daß es Dir lieb wäre, wenn \textsc{Schw}.\pwindex{Schwarzkopf, Gustav 7.\,11.\,1853 Wien – 13.\,11.\,1939 ebd.@\textsc{Schwarzkopf, Gustav} (7.\,11.\,1853 Wien – 13.\,11.\,1939 ebd.), \emph{Schriftsteller}|pwv} die Stelle\orgindex{Frankfurter Zeitung@Frankfurter Zeitung|pwv} bekäme. Darum{ }ſchrieb ich Dir, ich würde »\label{K_L02900-6v}\edtext{Dir zuliebe}{\lemma{\textnormal{\emph{Dir zuliebe}}}\Cendnote{\textnormal{Vgl. XXXX Auszeichnungsfehler: Dokument L02893 nicht gefunden und XXXX Auszeichnungsfehler: Dokument L02898 nicht gefunden.
               }}}\label{K_L02900-6}« in dieſer Richtung wirken. Nachdem Du dieſes »Dir zuliebe« abgelehnt haſt,
               habe ich, wie ich Dir{ }ſchon{ }ſchrieb, \strikeout{\textcolor{gray}{×}\-\textcolor{gray}{×}\-\textcolor{gray}{×}} mich jeder weiteren Einwirkung auf die Angelegenheit enthalten{\dotsfour}\pend
           
\pstart
           Nächſte Woche gehe ich nach Berlin\oindex{Berlin@\textbf{Berlin}, \emph{Hauptstadt}|pw}. Das heißt,
               wenn ich Geld aus Wien\oindex{Wien@\textbf{Wien}, \emph{Verwaltungsgebiet}|pw} bekomme. Die N. Fr. Pr.\orgindex{Neue Freie Presse@Neue Freie Presse|pw} benimmt{ }ſich {\pb}(im
               Vertrauen geſagt) in{ }ſkandalöſer Weiſe. Ich habe den Leuten\orgindex{Neue Freie Presse@Neue Freie Presse|pwv} geſchrieben, daß ich von der Frankf. Zeit.\orgindex{Frankfurter Zeitung@Frankfurter Zeitung|pw} keinen Gehalt mehr beziehe und daß{ }ſie mir
               infolgedeſſen meinen Januar-Gehalt vorauszahlen möchten.
               Das iſt vor zehn Tagen geſchehen, und ich habe bis heut nicht einmal eine Antwort bekommen. So{ }ſitze ich hier ohne Geld in
               den abſcheulichſten Schwierigkeiten, die durch die Weihnachtszeit und das Jahresende
               nur noch vermehrt werden. Wenn ich das{ }ſehe und auf der andern Seite das Bedauern
               conſtatire, mit dem die Redaktion\orgindex{Frankfurter Zeitung@Frankfurter Zeitung|pwv} der Frankf. Zeit.\orgindex{Frankfurter Zeitung@Frankfurter Zeitung|pw} und das
               Publikum meinen Weggang begleiten, –{ }ſo reut mich bereits der gefaßte Entſchluß. Auch
               graut mir vor der neuen{ }ſchweren Arbeit, – vor dem neuen Blatte\orgindex{Neue Freie Presse@Neue Freie Presse|pwv} und dem neuen Publikum. Ich bin{ }ſo
               müde! Und in dieſer Muthloſigkeit habe ich nur den \uline{einen} Wunſch: mich aus all’ \strikeout{dieſ\textcolor{gray}{e}} den endloſen Kämpfen und Sorgen durch eine reiche Heirath zu retten. Aber auch
               dazu iſt es leider{ }ſchon zu{ }ſpät.\pend
           
\pstart
           Meine Mutter\pwindex{Goldmann, Clementine 15.\,5.\,1842 Breslau – 24.\,2.\,1924 Frankfurt am Main@\textsc{Goldmann, Clementine} (15.\,5.\,1842 Breslau – 24.\,2.\,1924 Frankfurt am Main)|pwv} zieht mit mir.
               Sie muß mitziehen, weil ich{ }ſonſt nicht für ihren Unterhalt{ }ſorgen könnte. Und{ }ſie
               wäre{ }ſo gern hier geblieben bei ihrem \label{K_L02900-7v}\edtext{Enkelchen\pwindex{Rosengart, Paul 2.\,6.\,1896 Frankfurt am Main – 21.\,3.\,1962 Straßburg@\textsc{Rosengart, Paul} (2.\,6.\,1896 Frankfurt am Main – 21.\,3.\,1962 Straßburg)|pwv}}{\lemma{\textnormal{\emph{Enkelchen}}}\Cendnote{\textnormal{Paul Rosengart\pwindex{Rosengart, Paul 2.\,6.\,1896 Frankfurt am Main – 21.\,3.\,1962 Straßburg@\textsc{Rosengart, Paul} (2.\,6.\,1896 Frankfurt am Main – 21.\,3.\,1962 Straßburg)|pwk}, Goldmanns\pwindex{Goldmann, Paul 31.\,1.\,1865 Breslau – 25.\,9.\,1935 Wien@\textsc{Goldmann, Paul} (31.\,1.\,1865 Breslau – 25.\,9.\,1935 Wien), \emph{Schriftsteller, Journalist}|pwk}{ }Neffe\pwindex{Rosengart, Paul 2.\,6.\,1896 Frankfurt am Main – 21.\,3.\,1962 Straßburg@\textsc{Rosengart, Paul} (2.\,6.\,1896 Frankfurt am Main – 21.\,3.\,1962 Straßburg)|pwkv}, Tochter seiner Schwester Vally\pwindex{Rosengart, Vally 29.\,12.\,1866 Breslau – nach 1926@\textsc{Rosengart, Vally} (29.\,12.\,1866 Breslau – nach 1926)|pwk} und deren Mann Josef\pwindex{Rosengart, Josef 8.\,2.\,1860 Laupheim – 4.\,8.\,1927 Frankfurt am Main@\textsc{Rosengart, Josef} (8.\,2.\,1860 Laupheim – 4.\,8.\,1927 Frankfurt am Main), \emph{Arzt}|pwk}, geb. am 2. 6. 1896}}}\label{K_L02900-7}, in der{ }ſtillen freundlichen Stadt\oindex{Frankfurt am Main@\textbf{Frankfurt am Main}, \emph{Hauptstadt}|pwv}.\pend
           
\pstart
           {\pb}Bitte, theile mir die Berlin\oindex{Berlin@\textbf{Berlin}, \emph{Hauptstadt}|pw}er Adreſſe von Fräulein \textsc{G}.\pwindex{Glümer, Marie 3.\,7.\,1867 Wien – 16.\,11.\,1925 München@\textsc{Glümer, Marie} (3.\,7.\,1867 Wien – 16.\,11.\,1925 München), \emph{Schauspielerin}|pwv} mit, – wenn Du
               wünſcheſt, daß ich{ }ſie aufſuche.\pend
           
\pstart
           Ich hoffe Dich bald in \label{K_L02900-8v}\edtext{Berlin\oindex{Berlin@\textbf{Berlin}, \emph{Hauptstadt}|pw}}{\lemma{\textnormal{\emph{Berlin}}}\Cendnote{\textnormal{Das nächste Mal war Schnitzler zwischen 24. 11. 1900 und 28. 11. 1900 in Berlin\oindex{Berlin@\textbf{Berlin}, \emph{Hauptstadt}|pwk}. Goldmann\pwindex{Goldmann, Paul 31.\,1.\,1865 Breslau – 25.\,9.\,1935 Wien@\textsc{Goldmann, Paul} (31.\,1.\,1865 Breslau – 25.\,9.\,1935 Wien), \emph{Schriftsteller, Journalist}|pwk} traf er
                  täglich.}}}\label{K_L02900-8} zu{ }ſehen.\pend
           
\pstart
           Heute wünſche ich Dir von Herzen frohe Weihnachten und
               ein glückliches neues Jahr.\pend
           
\pstart
           Meine Mutter\pwindex{Goldmann, Clementine 15.\,5.\,1842 Breslau – 24.\,2.\,1924 Frankfurt am Main@\textsc{Goldmann, Clementine} (15.\,5.\,1842 Breslau – 24.\,2.\,1924 Frankfurt am Main)|pwv} und meine
               andern Verwandten erwidern Deine Grüße und bitten mich, Dir ihre Feiertagswünſche zu
               übermitteln.\pend
           
\pstart
           Ebenſo bitte ich Dich, mich Deiner Frau Mutter\pwindex{Schnitzler, Louise 8.\,7.\,1840 Kőszeg – 9.\,9.\,1911 Wien@\textsc{Schnitzler, Louise} (8.\,7.\,1840 Kőszeg – 9.\,9.\,1911 Wien)|pwv}, Deiner Frau Schweſter\pwindex{Hajek, Gisela 20.\,12.\,1867 Wien – 3.\,2.\,1953 Cambridge@\textsc{Hajek, Gisela} (20.\,12.\,1867 Wien – 3.\,2.\,1953 Cambridge)|pwv}, Deinem Bruder\pwindex{Schnitzler, Julius 13.\,7.\,1865 Wien – 29.\,6.\,1939 ebd.@\textsc{Schnitzler, Julius} (13.\,7.\,1865 Wien – 29.\,6.\,1939 ebd.), \emph{Chirurg}|pwv} und Deinem Schwager\pwindex{Hajek, Markus 25.\,11.\,1861 Vršac – 4.\,4.\,1941 London@\textsc{Hajek, Markus} (25.\,11.\,1861 Vršac – 4.\,4.\,1941 London), \emph{Mediziner, Laryngologe}|pwv} zu empfehlen und ihnen ein frohes Fest zu
               wünſchen.\pend
           
\pstart
           Von Herzen Dein {\\[\baselineskip]}\spacefill\mbox{Paul Goldmann}\pend
           \leftskip=0em{}\selectlanguage{ngerman}\endnumbering\briefempfaengerindex{Schnitzler, Arthur@\textsc{Schnitzler, Arthur}!zzzGoldmann, Paul@\emph{von Paul Goldmann}!1899-12-231@{23. 12. [1899]}|)be}\mylabel{L02900h}  \newcommand{\dateiname}{L02900}\newcommand{\titel}{Paul Goldmann an Arthur Schnitzler, 23. 12. [1899]}\newcommand{\editorInnen}{Martin Anton Müller und Laura Untner}%% latex-leseansicht-abspann.tex
%% Abspann für die Leseansicht.
%% Der Schalter \ifkorrekturansicht ist bereits durch den Vorspann gesetzt.

%% latex-abspann.tex
%% Gemeinsamer Abspann für Korrekturansicht und Leseansicht.
%% Setzt den Schalter \ifkorrekturansicht voraus (gesetzt in den
%% einbindenden Dateien latex-korrekturansicht-abspann.tex bzw.
%% latex-leseansicht-abspann.tex).
%% ---------------------------------------------------------------

\normalsize

% Das esempio-Environment wird nur in der Leseansicht benötigt
\ifkorrekturansicht\else
\newenvironment{esempio}[3]%
{
    \vspace{1.5ex}
    \rlap{\underline{#1}}
    \par
    \setlength{\parindent}{0cm}
    \nopagebreak
    \leftskip=#2cm
    \rightskip=#3cm
}
{
    \par
}
\fi

\doendnotes{C}
\bigskip
\vfill

\clearpage

\footnotesize

\ifkorrekturansicht
  \lohead{\textsc{register}}
\fi

% theindex-Environment neu definieren ohne reledmac
\makeatletter
\renewenvironment{theindex}{%
  \ifkorrekturansicht
    \section*{\indexname}%
  \else
    \subsubsection*{Index der erwähnten Entitäten}%
  \fi
  \setlength{\parindent}{0pt}%
  \setlength{\parskip}{0pt plus 0.3pt}%
  \let\item\@idxitem
}{%
  \ifkorrekturansicht\clearpage\fi
}
\makeatother

\IfFileExists{\jobname-pw.ind}{\input{\jobname-pw.ind}}{}

% Quellenangabe nur in der Leseansicht
\ifkorrekturansicht\else
% Fallback-Definitionen, falls die .tex-Datei \titel etc. nicht gesetzt hat
\providecommand{\titel}{}
\providecommand{\editorInnen}{}
\providecommand{\dateiname}{\jobname}

\vspace{3cm}

\vfill

\footnotesize
\textsc{Quelle}: \titel. Herausgegeben von {\editorInnen}. In: \emph{Arthur Schnitzler: Briefwechsel mit Autorinnen und Autoren}.
 Digitale Edition, https://schnitzler-briefe.acdh.oeaw.ac.at/{\dateiname}.html (Stand \today)
\fi

\end{document}


