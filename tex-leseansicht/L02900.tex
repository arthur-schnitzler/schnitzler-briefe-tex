%% latex-leseansicht-vorspann.tex
%% Vorspann für die Leseansicht.
%% Lädt die gemeinsame Datei latex-vorspann.tex mit nicht gesetztem Schalter.

\newif\ifkorrekturansicht
\korrekturansichtfalse

\input{../tex-inputs/latex-vorspann}

\begin{center}
            \textcolor{red}{ENTWURF, NICHT FERTIG KORRIGIERT}
                      \end{center}
            
         
         \newcommand{\erwaehntePersonen}{Personen: Otto Brahm, Samuel Fischer, Clemens von Franckenstein, Marie Glümer, Clementine Goldmann, Gisela Hajek, Markus Hajek, Georg Hirschfeld, Paul Rosengart, Vally Rosengart, Josef Rosengart, Louise Schnitzler, Julius Schnitzler, Gustav Schwarzkopf, Jakob Wassermann, Eugen d’Albert}
         \newcommand{\erwaehnteInstitutionen}{Institutionen: A. Entsch, Burgtheater, Deutsches Theater Berlin, Frankfurter Zeitung, Neue Freie Presse, Schauspielhaus Berlin}
         \newcommand{\erwaehnteOrte}{Orte: Berlin, Deutsches Theater Berlin, Frankfurt am Main, Wien}
         \newcommand{\erwaehnteWerke}{Werke: Der Schleier der Beatrice. Schauspiel in fünf Akten, Reigen. Zehn Dialoge}
               \section[ Paul Goldmann an Arthur Schnitzler, 23. 12. {[}1899{]}]{ Paul Goldmann an Arthur Schnitzler, 23. 12. {[}1899{]}}\nopagebreak\mylabel{v}\rehead{ }\begin{ledgroupsized}[t]{13cm}\normalsize\beginnumbering \toendnotes[C]{\smallbreak\pagebreak[2]} \Standort{DLA, A:Schnitzler, HS.NZ85.1.3169.}
\physDesc{Brief, 1 Blatt, 4 Seiten
\newline{}Handschrift: blaue Tinte, deutsche Kurrent
\newline{}Schnitzler: mit rotem Buntstift das Jahr »99« vermerkt und vier Unterstreichungen }\toendnotes[C]{\smallbreak}\pstart
           \centering{}{\pb}Frankfurt\oindex{Frankfurt am Main@\textbf{Frankfurt am Main}|pw}, 23. Dezember.\pend
           \pstart{}Mein lieber Freund,\pend\pstart
           Ich habe Deine lieben Nachrichten lange vermißt und war ſehr froh, wieder ausführlich
               von Dir zu hören.\pend
           \pstart
           Wenn Du die \label{K_L02900-1v}\edtext{»Beatrice\pwindex{Schnitzler, Arthur 15.05.1862 – 21.10.1931@\textsc{Schnitzler, Arthur} (15.05.1862 – 21.10.1931), \emph{Schriftsteller, Mediziner}!Schleier der Beatrice. Schauspiel in fuenf Akten1900-12-01@\strich\emph{Der Schleier der Beatrice. Schauspiel in fünf Akten} {[}1900-12-01{]}|pw}« drucken}{\lemma{\textnormal{\emph{»Beatrice« drucken}}}\Cendnote{\textnormal{\emph{Der Schleier der Beatrice}\pwindex{Schnitzler, Arthur 15.05.1862 – 21.10.1931@\textsc{Schnitzler, Arthur} (15.05.1862 – 21.10.1931), \emph{Schriftsteller, Mediziner}!Schleier der Beatrice. Schauspiel in fuenf Akten1900-12-01@\strich\emph{Der Schleier der Beatrice. Schauspiel in fünf Akten} {[}1900-12-01{]}|pwk} wurde 1900 zuerst für die Bühnen gedruckt (bei \emph{A. Entsch}\orgindex{A. Entsch@A. Entsch|pwk}), mit Jahresbeginn
                     1901 war es dann bei \emph{S. Fischer}\textcolor{red}{\textsuperscript{XXXX indx}}
                  verfügbar.}}}\label{K_L02900-1h} läßt, werde ich ſie hoffentlich bald zu leſen bekommen. Wie
               ſtehen die \label{K_L02900-2v}\edtext{Aufführungs-Chancen beim
                  Burgtheater\orgindex{Burgtheater@Burgtheater|pw}}{\lemma{\textnormal{\emph{Aufführungs-Chancen beim Burgtheater}}}\Cendnote{\textnormal{siehe Paul Goldmann an Arthur Schnitzler, 12. 11. [1899]}}}\label{K_L02900-2h}? Und wie in \label{K_L02900-3v}\edtext{Berlin\oindex{Berlin@\textbf{Berlin}|pw}}{\lemma{\textnormal{\emph{Berlin}}}\Cendnote{\textnormal{Am Deutschen Theater\oindex{Deutsches Theater Berlin@\textbf{Deutsches Theater Berlin}|pwk} feierte \emph{Der Schleier
                     der Beatrice}\pwindex{Schnitzler, Arthur 15.05.1862 – 21.10.1931@\textsc{Schnitzler, Arthur} (15.05.1862 – 21.10.1931), \emph{Schriftsteller, Mediziner}!Schleier der Beatrice. Schauspiel in fuenf Akten1900-12-01@\strich\emph{Der Schleier der Beatrice. Schauspiel in fünf Akten} {[}1900-12-01{]}|pwk} – trotzdem Otto Brahm\pwindex{Brahm, Otto 05.02.1856 – 28.11.1912@\textsc{Brahm, Otto} (05.02.1856 – 28.11.1912), \emph{Theaterleiter, Regisseur}|pwk}
                  das Stück\pwindex{Schnitzler, Arthur 15.05.1862 – 21.10.1931@\textsc{Schnitzler, Arthur} (15.05.1862 – 21.10.1931), \emph{Schriftsteller, Mediziner}!Schleier der Beatrice. Schauspiel in fuenf Akten1900-12-01@\strich\emph{Der Schleier der Beatrice. Schauspiel in fünf Akten} {[}1900-12-01{]}|pwkv} seit einer
                  persönlichen Lesung durch den Autor am 7. 10. 1899 kannte – erst am 7. 3. 1903
                  Premiere.}}}\label{K_L02900-3h}? Deutſches Theater\orgindex{Deutsches Theater Berlin@Deutsches Theater Berlin|pw} oder Schauſpielhaus\orgindex{Schauspielhaus Berlin@Schauspielhaus Berlin|pw}? Vielleicht wird es eine meiner
               erſten Aufgaben ſein, über eine \begin{otherlanguage}{french}\textsc{Première}\end{otherlanguage}{ }von Dir zu berichten. Iſt der \label{K_L02900-4v}\edtext{»Reigen\pwindex{Schnitzler, Arthur 15.05.1862 – 21.10.1931@\textsc{Schnitzler, Arthur} (15.05.1862 – 21.10.1931), \emph{Schriftsteller, Mediziner}!Reigen. Zehn Dialoge1900@\strich\emph{Reigen. Zehn Dialoge} {[}1900{]}|pw}« ſchon gedruckt?}{\lemma{\textnormal{\emph{»Reigen« ſchon gedruckt?}}}\Cendnote{\textnormal{Ein
                  Privatdruck für die Verteilung an Freundinnen und Freunde in der Auflage von 200
                  Stück wurde, betreut vom Verleger S.
                     Fischer\pwindex{Fischer, Samuel 24.12.1859 – 15.10.1934@\textsc{Fischer, Samuel} (24.12.1859 – 15.10.1934), \emph{Verleger}|pwk}, zwischen November 1899 und 12. 2. 1900
                  gedruckt.}}}\label{K_L02900-4h}{\dotsfour}\pend
           \pstart
           In den \label{K_L02900-5v}\edtext{Fragen \textsc{Wassermann\pwindex{Wassermann, Jakob 10.03.1873 – 01.01.1934@\textsc{Wassermann, Jakob} (10.03.1873 – 01.01.1934), \emph{Schriftsteller}|pw}} und \textsc{Schwarzkopf\pwindex{Schwarzkopf, Gustav 07.11.1853 – 13.11.1939@\textsc{Schwarzkopf, Gustav} (07.11.1853 – 13.11.1939), \emph{Schriftsteller}|pw}}}{\lemma{\textnormal{\emph{Fragen … Schwarzkopf}}}\Cendnote{\textnormal{siehe Paul Goldmann an Arthur Schnitzler, 26. 10. 1899, 6. 12. [1899] und 11. 12. [1899]}}}\label{K_L02900-5h} beharre ich durchaus auf meinem Standpunkte. \textsc{Wassermann}\pwindex{Wassermann, Jakob 10.03.1873 – 01.01.1934@\textsc{Wassermann, Jakob} (10.03.1873 – 01.01.1934), \emph{Schriftsteller}|pw} brauchte das betr. Concert\pwindex{Franckenstein, Clemens von 14.07.1875 – 19.08.1942@\textsc{Franckenstein, Clemens von} (14.07.1875 – 19.08.1942), \emph{Theaterleiter, Komponist, Dirigent}|pwv} nicht zu übergehen, wenn er ſonſt die Gewohnheit gehabt hätte, über
               Concerte zu berichten. Da er das aber faſt nie thut, ſo iſt die Herausgreifung dieſes
                  \strikeout{\textcolor{gray}{×}} unbedeutenden Concertes\pwindex{Franckenstein, Clemens von 14.07.1875 – 19.08.1942@\textsc{Franckenstein, Clemens von} (14.07.1875 – 19.08.1942), \emph{Theaterleiter, Komponist, Dirigent}|pwv} aus der ungeheuren Fülle der Wien\oindex{Wien@\textbf{Wien}|pw}er Concerte ſchon \strikeout{\textcolor{gray}{an ſich}} eine ungerechte Bevorzugung; und wenn auch das Lob, das er dem Concertgeber\pwindex{Franckenstein, Clemens von 14.07.1875 – 19.08.1942@\textsc{Franckenstein, Clemens von} (14.07.1875 – 19.08.1942), \emph{Theaterleiter, Komponist, Dirigent}|pwv} ſpendet, an ſich
               nicht übertrieben iſt, ſo wird es übertrieben durch den Tadel gegenüber einem anderen
                  \strikeout{\textcolor{gray}{×}\-\textcolor{gray}{×}} viel bedeutenderen Concertgeber\pwindex{DAlbert, Eugen 10.04.1864 – 03.03.1932@\textsc{d’Albert, Eugen} (10.04.1864 – 03.03.1932), \emph{Komponist}|pwv}, mit dem \textsc{W}.\pwindex{Wassermann, Jakob 10.03.1873 – 01.01.1934@\textsc{Wassermann, Jakob} (10.03.1873 – 01.01.1934), \emph{Schriftsteller}|pwv} es verbunden hat. Was \textsc{Schwarzkopf\pwindex{Schwarzkopf, Gustav 07.11.1853 – 13.11.1939@\textsc{Schwarzkopf, Gustav} (07.11.1853 – 13.11.1939), \emph{Schriftsteller}|pw}} anlangt, ſo kenne ich ſeine bedeutenden Vorzüge. \textsc{Hirschfeld\pwindex{Hirschfeld, Georg 11.02.1873 – 17.01.1942@\textsc{Hirschfeld, Georg} (11.02.1873 – 17.01.1942), \emph{Schriftsteller}|pw}} wäre trotzdem der beſſere Berichterſtatter, weil er zu allem Anderen \uline{auch} die Muſik umfaßt und weil er {\pb}etwas lebendiger und farbiger ſchreibt als \textsc{Schw}.\pwindex{Schwarzkopf, Gustav 07.11.1853 – 13.11.1939@\textsc{Schwarzkopf, Gustav} (07.11.1853 – 13.11.1939), \emph{Schriftsteller}|pwv} Eine Theilung der Berichterſtattung unter die
               Beiden iſt, nach den bei der Frankf. Zeit.\orgindex{Frankfurter Zeitung@Frankfurter Zeitung|pw}
               beſtehenden Einrichtungen, unmöglich. Daß ich die Intereſſen der Frankf. Zeit.\orgindex{Frankfurter Zeitung@Frankfurter Zeitung|pw} vor Allem zu vertreten habe, weiß ich, auch
               ohne daß Du es mir ſagſt, und ich würde \strikeout{\textcolor{gray}{×}\-\textcolor{gray}{×}\-\textcolor{gray}{×}}{ }\textsc{Schw}.\pwindex{Schwarzkopf, Gustav 07.11.1853 – 13.11.1939@\textsc{Schwarzkopf, Gustav} (07.11.1853 – 13.11.1939), \emph{Schriftsteller}|pwv} niemals \strikeout{\textcolor{gray}{eigent}} empfohlen haben, wenn ich \strikeout{\textcolor{gray}{irg}} auch nur einen Augenblick hätte annehmen müſſen, er würde als Correſpondent
               den Intereſſen der Zeitung\orgindex{Frankfurter Zeitung@Frankfurter Zeitung|pwv}
               nicht entſprechen. Es handelt ſich hier um zwei ungefähr gleich würdige Candidaten,
               und wenn irgendwo, ſo \strikeout{\textcolor{gray}{wie}} kann hier das Perſönliche interveniren. Ich perſönlich fühle mich, bei aller
               Sympathie und Freundſchaft ſür \textsc{Schw}.\pwindex{Schwarzkopf, Gustav 07.11.1853 – 13.11.1939@\textsc{Schwarzkopf, Gustav} (07.11.1853 – 13.11.1939), \emph{Schriftsteller}|pwv}, doch mehr zu \textsc{H}.\pwindex{Hirschfeld, Georg 11.02.1873 – 17.01.1942@\textsc{Hirschfeld, Georg} (11.02.1873 – 17.01.1942), \emph{Schriftsteller}|pwv} hingezogen. Von Dir weiß ich das Umgekehrte.
               Oder vielmehr ich weiß, daß es Dir lieb wäre, wenn \textsc{Schw}.\pwindex{Schwarzkopf, Gustav 07.11.1853 – 13.11.1939@\textsc{Schwarzkopf, Gustav} (07.11.1853 – 13.11.1939), \emph{Schriftsteller}|pwv} die Stelle\orgindex{Frankfurter Zeitung@Frankfurter Zeitung|pwv} bekäme. Darum ſchrieb ich Dir, ich würde »\label{K_L02900-7v}\edtext{Dir zuliebe}{\lemma{\textnormal{\emph{Dir zuliebe}}}\Cendnote{\textnormal{vgl. Paul Goldmann an Arthur Schnitzler, 12. 11. [1899] und 11. 12. [1899]}}}\label{K_L02900-7h}« in dieſer Richtung wirken. Nachdem Du dieſes »Dir zuliebe« abgelehnt haſt,
               habe ich, wie ich Dir ſchon ſchrieb, \strikeout{\textcolor{gray}{×}\-\textcolor{gray}{×}\-\textcolor{gray}{×}} mich jeder weiteren Einwirkung auf die Angelegenheit enthalten{\dotsfour}\pend
           \pstart
           Nächſte Woche gehe ich nach Berlin\oindex{Berlin@\textbf{Berlin}|pw}. Das heißt,
               wenn ich Geld aus Wien\oindex{Wien@\textbf{Wien}|pw} bekomme. Die N. Fr. Pr.\orgindex{Neue Freie Presse@Neue Freie Presse|pw} benimmt ſich {\pb}(im
               Vertrauen geſagt) in ſkandalöſer Weiſe. Ich habe den Leuten\orgindex{Neue Freie Presse@Neue Freie Presse|pwv} geſchrieben, daß ich von der Frankf. Zeit.\orgindex{Frankfurter Zeitung@Frankfurter Zeitung|pw} keinen Gehalt mehr beziehe und daß ſie mir
               infolgedeſſen meinen Januar-Gehalt vorauszahlen möchten.
               Das iſt vor zehn Tagen geſchehen, und ich habe bis heut nicht einmal eine Antwort bekommen. So ſitze ich hier ohne Geld in
               den abſcheulichſten Schwierigkeiten, die durch die Weihnachtszeit und das Jahresende
               nur noch vermehrt werden. Wenn ich das ſehe und auf der andern Seite das Bedauern
               conſtatire, mit dem die Redaktion\orgindex{Frankfurter Zeitung@Frankfurter Zeitung|pwv} der Frankf. Zeit.\orgindex{Frankfurter Zeitung@Frankfurter Zeitung|pw} und das
               Publikum meinen Weggang begleiten, – ſo reut mich bereits der gefaßte Entſchluß. Auch
               graut mir vor der neuen ſchweren Arbeit, – vor dem neuen Blatte\orgindex{Neue Freie Presse@Neue Freie Presse|pwv} und dem neuen Publikum. Ich bin ſo
               müde! Und in dieſer Muthloſigkeit habe ich nur den \uline{einen} Wunſch: mich aus all’ \strikeout{dieſ\textcolor{gray}{e}} den endloſen Kämpfen und Sorgen durch eine reiche Heirath zu retten. Aber auch
               dazu iſt es leider ſchon zu ſpät.\pend
           \pstart
           Meine Mutter\pwindex{Goldmann, Clementine 1842-05-15 – 1924-02-24@\textsc{Goldmann, Clementine} (1842-05-15 – 1924-02-24)|pwv} zieht mit mir.
               Sie muß mitziehen, weil ich ſonſt nicht für ihren Unterhalt ſorgen könnte. Und ſie
               wäre ſo gern hier geblieben bei ihrem \label{K_L02900-16v}\edtext{Enkelchen\pwindex{Rosengart, Paul 1896-06-02 – 1962-03-21@\textsc{Rosengart, Paul} (1896-06-02 – 1962-03-21)|pwv}}{\lemma{\textnormal{\emph{Enkelchen}}}\Cendnote{\textnormal{Paul Rosengart\pwindex{Rosengart, Paul 1896-06-02 – 1962-03-21@\textsc{Rosengart, Paul} (1896-06-02 – 1962-03-21)|pwk}, Goldmann\pwindex{Goldmann, Paul 31.01.1865 – 25.09.1935@\textsc{Goldmann, Paul} (31.01.1865 – 25.09.1935), \emph{Schriftsteller, Journalist}|pwk}s Neffe\pwindex{Rosengart, Paul 1896-06-02 – 1962-03-21@\textsc{Rosengart, Paul} (1896-06-02 – 1962-03-21)|pwkv}, Tochter seiner Schwester Vally\pwindex{Rosengart, Vally *~1866-12-29@\textsc{Rosengart, Vally} (*~1866-12-29)|pwk} und deren Mann Josef\pwindex{Rosengart, Josef 1860-02-08 – 1927-08-04@\textsc{Rosengart, Josef} (1860-02-08 – 1927-08-04), \emph{Arzt}|pwk}, geb. am 2. 6. 1896}}}\label{K_L02900-16h}, in der ſtillen freundlichen Stadt\oindex{Frankfurt am Main@\textbf{Frankfurt am Main}|pwv}.\pend
           \pstart
           {\pb}Bitte, theile mir die Berlin\oindex{Berlin@\textbf{Berlin}|pw}er Adreſſe von Fräulein \textsc{G}.\pwindex{Gluemer, Marie 03.07.1867 – 16.11.1925@\textsc{Glümer, Marie} (03.07.1867 – 16.11.1925), \emph{Schauspielerin}|pwv} mit, – wenn Du
               wünſcheſt, daß ich ſie aufſuche.\pend
           \pstart
           Ich hoffe Dich bald in \label{K_L02900-14v}\edtext{Berlin\oindex{Berlin@\textbf{Berlin}|pw}}{\lemma{\textnormal{\emph{Berlin}}}\Cendnote{\textnormal{Das nächste Mal war Schnitzler\pwindex{Schnitzler, Arthur 15.05.1862 – 21.10.1931@\textsc{Schnitzler, Arthur} (15.05.1862 – 21.10.1931), \emph{Schriftsteller, Mediziner}|pwk} zwischen 24. 11. 1900 und 28. 11. 1900 in Berlin\oindex{Berlin@\textbf{Berlin}|pwk}. Goldmann\pwindex{Goldmann, Paul 31.01.1865 – 25.09.1935@\textsc{Goldmann, Paul} (31.01.1865 – 25.09.1935), \emph{Schriftsteller, Journalist}|pwk} traf er
                  täglich.}}}\label{K_L02900-14h} zu ſehen.\pend
           \pstart
           Heute wünſche ich Dir von Herzen frohe Weihnachten und
               ein glückliches neues Jahr.\pend
           \pstart
           Meine Mutter\pwindex{Goldmann, Clementine 1842-05-15 – 1924-02-24@\textsc{Goldmann, Clementine} (1842-05-15 – 1924-02-24)|pwv} und meine
               andern Verwandten erwidern Deine Grüße und bitten mich, Dir ihre Feiertagswünſche zu
               übermitteln.\pend
           \pstart
           Ebenſo bitte ich Dich, mich Deiner Frau Mutter\pwindex{Schnitzler, Louise 1840-07-08 – 1911-09-09@\textsc{Schnitzler, Louise} (1840-07-08 – 1911-09-09)|pwv}, Deiner Frau Schweſter\pwindex{Hajek, Gisela 20.12.1867 – 03.02.1953@\textsc{Hajek, Gisela} (20.12.1867 – 03.02.1953)|pwv}, Deinem Bruder\pwindex{Schnitzler, Julius 13.07.1865 – 29.06.1939@\textsc{Schnitzler, Julius} (13.07.1865 – 29.06.1939), \emph{Chirurg}|pwv} und Deinem Schwager\pwindex{Hajek, Markus 25.11.1861 – 04.04.1941@\textsc{Hajek, Markus} (25.11.1861 – 04.04.1941), \emph{Mediziner, Laryngologe}|pwv} zu empfehlen und ihnen ein frohes Fest zu
               wünſchen.\pend
           \pstart
           Von Herzen Dein {\\[\baselineskip]}\spacefill\mbox{Paul Goldmann}\pend
           \leftskip=0em{}
         
         \endnumbering\mylabel{h}\end{ledgroupsized}  \newcommand{\dateiname}{L02900}\newcommand{\titel}{Paul Goldmann an Arthur Schnitzler, 23. 12. [1899]}\newcommand{\editorInnen}{Martin Anton Müller und Laura Untner}%% latex-leseansicht-abspann.tex
%% Abspann für die Leseansicht.
%% Der Schalter \ifkorrekturansicht ist bereits durch den Vorspann gesetzt.

%% latex-abspann.tex
%% Gemeinsamer Abspann für Korrekturansicht und Leseansicht.
%% Setzt den Schalter \ifkorrekturansicht voraus (gesetzt in den
%% einbindenden Dateien latex-korrekturansicht-abspann.tex bzw.
%% latex-leseansicht-abspann.tex).
%% ---------------------------------------------------------------

\normalsize

% Das esempio-Environment wird nur in der Leseansicht benötigt
\ifkorrekturansicht\else
\newenvironment{esempio}[3]%
{
    \vspace{1.5ex}
    \rlap{\underline{#1}}
    \par
    \setlength{\parindent}{0cm}
    \nopagebreak
    \leftskip=#2cm
    \rightskip=#3cm
}
{
    \par
}
\fi

\doendnotes{C}
\bigskip
\vfill

\clearpage

\footnotesize

\ifkorrekturansicht
  \lohead{\textsc{register}}
\fi

% theindex-Environment neu definieren ohne reledmac
\makeatletter
\renewenvironment{theindex}{%
  \ifkorrekturansicht
    \section*{\indexname}%
  \else
    \subsubsection*{Index der erwähnten Entitäten}%
  \fi
  \setlength{\parindent}{0pt}%
  \setlength{\parskip}{0pt plus 0.3pt}%
  \let\item\@idxitem
}{%
  \ifkorrekturansicht\clearpage\fi
}
\makeatother

\IfFileExists{\jobname-pw.ind}{\input{\jobname-pw.ind}}{}

% Quellenangabe nur in der Leseansicht
\ifkorrekturansicht\else
% Fallback-Definitionen, falls die .tex-Datei \titel etc. nicht gesetzt hat
\providecommand{\titel}{}
\providecommand{\editorInnen}{}
\providecommand{\dateiname}{\jobname}

\vspace{3cm}

\vfill

\footnotesize
\textsc{Quelle}: \titel. Herausgegeben von {\editorInnen}. In: \emph{Arthur Schnitzler: Briefwechsel mit Autorinnen und Autoren}.
 Digitale Edition, https://schnitzler-briefe.acdh.oeaw.ac.at/{\dateiname}.html (Stand \today)
\fi

\end{document}


      