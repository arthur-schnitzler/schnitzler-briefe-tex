%% latex-korrekturansicht-vorspann.tex
%% Vorspann für die Korrekturansicht.
%% Lädt die gemeinsame Datei latex-vorspann.tex mit gesetztem Schalter.

\newif\ifkorrekturansicht
\korrekturansichttrue

\input{../tex-inputs/latex-vorspann}


\section[Arthur Schnitzler an Hermann Bahr, 11. 12. 1901]{L01189 Arthur Schnitzler an Hermann Bahr, 11. 12. 1901}
\nopagebreak\mylabel{L01189v}
\rehead{ }\normalsize\beginnumbering\briefempfaengerindex{Bahr, Hermann@\textsc{Bahr, Hermann}!zzzSchnitzler, Arthur@\emph{von Arthur Schnitzler}!1901-12-112@{11. 12. 1901}|(be}
\toendnotes[C]{\smallbreak\pagebreak[2]}\Standort{TMW, HS AM 23346 Ba.}
\physDesc{Brief, 1 Blatt, 3 Seiten, 567 Zeichen
\newline{}Handschrift: Bleistift, deutsche Kurrent
\newline{}Ordnung: Lochung }
\buchAbdrucke{\weitereDrucke{1) Arthur Schnitzler: \emph{The Letters of Arthur Schnitzler to Hermann Bahr}. Chapel Hill: \emph{The University of North Carolina Press} 1978, S. 73.} \weitereDrucke{2) Hermann Bahr, Arthur Schnitzler: \emph{Briefwechsel, Aufzeichnungen, Dokumente (1891–1931)}. Göttingen: \emph{Wallstein} 2018, S. 220.} }\toendnotes[C]{\smallbreak}
\pstart
           \raggedleft{}{\pb}11. 12. 901\pend
           
\pstart{}mein lieber Hermann,\pend\vspace{0.5em}
\pstart
           ich nehme an, Direktor \textsc{Bukovics\pwindex{Bukovics, Emerich von 28.02.1844 – 04.07.1905@\textsc{Bukovics, Emerich von} (28.02.1844 – 04.07.1905), \emph{Journalist/Journalistin, Theaterleiter/Theaterleiterin}|pw}} wird dir den \label{K_L01189-1v}\edtext{Brief}{\lemma{\textnormal{\emph{Brief}}}\Cendnote{\textnormal{Siehe Hermann Bahr, Arthur Schnitzler: \emph{Briefwechsel, Aufzeichnungen, Dokumente (1891–1931)}, Arthur Schnitzler an Emerich von Bukovics, 11. 12. 1901.
               }}}\label{K_L01189-1} zeigen, den ich heute an ihn
               geſchrieben, um die Sache endgiltig abzuſchließen und etliche ſonderbare Auffaſſungen
               ſeinerſeits richtigzuſtellen. We{\geminationn} nicht, ſteht dir
               gelegentlich {\pb}eine
               Abſchrift zur Verfügung.\pend
           
\pstart
           – Jedenfalls habe ich dir für deine wiederholten Verſuche, \textsc{Bukovics\pwindex{Bukovics, Emerich von 28.02.1844 – 04.07.1905@\textsc{Bukovics, Emerich von} (28.02.1844 – 04.07.1905), \emph{Journalist/Journalistin, Theaterleiter/Theaterleiterin}|pw}} auf ſeine Höflichkeitsverpflichtungen (ich ſehe von den andern ab, die
               vielleicht ein Theaterdirektor gegen einen Autor haben kö{\geminationn}te) aufmerkſam zu machen, herzlichſt zu danken.{\pb}\pend
           
\pstart
           {\pb}Auf baldgs
               Wiederſehen{\\[\baselineskip]}dein treuer{\\[\baselineskip]}\spacefill\mbox{Arth Sch}\pend
           \leftskip=0em{}\selectlanguage{ngerman}\endnumbering\briefempfaengerindex{Bahr, Hermann@\textsc{Bahr, Hermann}!zzzSchnitzler, Arthur@\emph{von Arthur Schnitzler}!1901-12-112@{11. 12. 1901}|)be}\mylabel{L01189h}  \normalsize

\doendnotes{C}
\bigskip
\vfill

\clearpage

\footnotesize

\lohead{\textsc{register}}

% Definiere theindex-Environment komplett neu ohne reledmac
\makeatletter
\renewenvironment{theindex}{%
  \section*{\indexname}%
  \setlength{\parindent}{0pt}%
  \setlength{\parskip}{0pt plus 0.3pt}%
  \let\item\@idxitem
}{%
  \clearpage
}
\makeatother

\IfFileExists{\jobname-pw.ind}{\input{\jobname-pw.ind}}{}

\end{document}

      