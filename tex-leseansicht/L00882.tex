%% latex-leseansicht-vorspann.tex
%% Vorspann für die Leseansicht.
%% Lädt die gemeinsame Datei latex-vorspann.tex mit nicht gesetztem Schalter.

\newif\ifkorrekturansicht
\korrekturansichtfalse

\input{../tex-inputs/latex-vorspann}


\section[Georg Brandes an Arthur Schnitzler, 22. 1. 1899]{L00882 Georg Brandes an Arthur Schnitzler, 22. 1. 1899}
\nopagebreak\mylabel{L00882v}
\rehead{ }\normalsize\beginnumbering\briefempfaengerindex{Schnitzler, Arthur@\textsc{Schnitzler, Arthur}!zzzBrandes, Georg@\emph{von Georg Brandes}!1899-01-221@{22. 1. 1899}|(be}
\toendnotes[C]{\smallbreak\pagebreak[2]}
\correspDesc{Versand  durch Georg Brandes am 22. 1. 1899 in Kopenhagen
\newline{}Erhalt  durch Arthur Schnitzler am 24. 1. 1899 in Wien}\toendnotes[C]{\smallbreak}
\Standort{CUL, Schnitzler, B 17.}
\physDesc{Postkarte, 1113 Zeichen
\newline{}Handschrift: blaue Tinte, lateinische Kurrent
\newline{}Versand: 1) Stempel: »\nobreak{}\oindex{Kopenhagen@\textbf{Kopenhagen}, \emph{Hauptstadt}|pwk}Kobenhavn, 22. 1. 99, 3–4 E\nobreak{}«.   2) Stempel: »\nobreak{}\oindex{IX., Alsergrund@\textbf{IX., Alsergrund}, \emph{Verwaltungsgebiet}|pwk}Wien 9/3, 24. 1. 99, 8. V, Bestellt\nobreak{}«. 
\newline{}Ordnung: mit Bleistift von unbekannter Hand nummeriert:
                                    »13« }
\buchAbdrucke{\weitereDrucke{Georg Brandes, Arthur Schnitzler: \emph{Ein Briefwechsel}. Herausgegeben von Kurt Bergel. Bern: \emph{Francke} 1956, S. 72–73.} }\toendnotes[C]{\smallbreak}\pstart{}{\pb}Herrn Dr. Arthur
                  Schnitzler\pend{}\pstart{}Frankgasse 1\oindex{Wien@\textbf{Wien}!IX., Alsergrund@\textbf{IX., Alsergrund}!Frankgasse 1@\textbf{Frankgasse 1}, \emph{Wohngebäude}|pw}\pend{}\pstart{}Wien IX\oindex{IX., Alsergrund@\textbf{IX., Alsergrund}, \emph{Verwaltungsgebiet}|pw}\pend{}{\bigskip}\vspace{1em}
\pstart
           \raggedleft{}{\pb}22 Januar 99\pend
           \vspace{0.5em}
\pstart
           Lieber Herr Doctor! Es war ein Fehler von mir dass ich nicht für die
                  Novellensammlung\pwindex{Schnitzler, Arthur 15.\,5.\,1862 Wien – 21.\,10.\,1931 ebd.@\textsc{Schnitzler, Arthur} (15.\,5.\,1862 Wien – 21.\,10.\,1931 ebd.), \emph{Schriftsteller, Mediziner}!Frau des Weisen. Novelletten@\strich\emph{Die Frau des Weisen. Novelletten}|pwv} dankte.
               ich habe sie mit grosser Aufmerksamkeit gelesen. Für mich ist die Novelle\pwindex{Schnitzler, Arthur 15.\,5.\,1862 Wien – 21.\,10.\,1931 ebd.@\textsc{Schnitzler, Arthur} (15.\,5.\,1862 Wien – 21.\,10.\,1931 ebd.), \emph{Schriftsteller, Mediziner}!Toten schweigen@\strich\emph{Die Toten schweigen}|pwv} die zuerst in Cosmopolis\pwindex{Cosmopolis@\emph{Cosmopolis}|pw} stand – ich erinnere mich nicht des Titels – ein \uline{Meisterwerk} erstaunlich wahr und packend; nur ein
               (sehr kleiner) Fehler gegen den Schluss, dass die Frau zuletzt alles gesteht. Als ob
               Frauen je geständen, wenn keine Beweise vorliegen, und wenn sie keinem absolut
               überlegenen Mann gegenüber stehen! Ein wahres Meisterwerk ist es dennoch.\pend
           
\pstart
           Meine Gedichte\pwindex{Brandes, Georg 4.\,2.\,1842 Kopenhagen – 19.\,2.\,1927 ebd.@\textsc{Brandes, Georg} (4.\,2.\,1842 Kopenhagen – 19.\,2.\,1927 ebd.)!Ungdomsvers [Jugendgedichte]@\strich\emph{Ungdomsvers [Jugendgedichte]}|pwv}! Was soll ich
               darüber sagen. Lesen Sie Dänisch\oindex{Dänemark@\textbf{Dänemark}|pw}, so werden Sie
               einräumen dass zwei oder drei sehr gut sind, »Reconvalescent-Besuch\pwindex{Brandes, Georg 4.\,2.\,1842 Kopenhagen – 19.\,2.\,1927 ebd.@\textsc{Brandes, Georg} (4.\,2.\,1842 Kopenhagen – 19.\,2.\,1927 ebd.)!Reconvalescent-Besuch@\strich\emph{Reconvalescent-Besuch}|pw}« und »Harald Haarfager
                  in Finmarken\pwindex{Brandes, Georg 4.\,2.\,1842 Kopenhagen – 19.\,2.\,1927 ebd.@\textsc{Brandes, Georg} (4.\,2.\,1842 Kopenhagen – 19.\,2.\,1927 ebd.)!Harald Haarfager in Finmarken@\strich\emph{Harald Haarfager in Finmarken}|pw}«. Es ist eine Art Jugend-Tagebuch. – Ich liege noch immer zu
               Bett, schon 5 Wochen, Sie wissen ja was Venenentzündung ist. Doch ist es diesmal
               anscheinend nicht schlimm. Beste Grüsse \spacefill\mbox{G. B.}\pend
           
\pstart
           \noindent{}\label{T_L00882-1v}\edtext{Sie haben wohl meinen Protest gegen die Ausweisungen
                     der Dänen\pwindex{Köllers Erfolge@\emph{Köllers Erfolge}|pwv} gelesen, oder auch nicht. 100 Zeitungen aller Länder haben ihn
                  abgedruckt aber die Neue Freie\orgindex{Neue Freie Presse@Neue Freie Presse|pw} ist ja preussisch\oindex{Preußen@\textbf{Preußen}|pw}.}{\lemma{\textnormal{\emph{Sie … preussisch.}}}\Cendnote{\textnormal{am linken Rand}}}\label{T_L00882-1}\pend
           \selectlanguage{ngerman}\endnumbering\briefempfaengerindex{Schnitzler, Arthur@\textsc{Schnitzler, Arthur}!zzzBrandes, Georg@\emph{von Georg Brandes}!1899-01-221@{22. 1. 1899}|)be}\mylabel{L00882h}  \newcommand{\dateiname}{L00882}\newcommand{\titel}{Georg Brandes an Arthur Schnitzler, 22. 1. 1899}\newcommand{\editorInnen}{Martin Anton Müller und Gerd-Hermann Susen}%% latex-leseansicht-abspann.tex
%% Abspann für die Leseansicht.
%% Der Schalter \ifkorrekturansicht ist bereits durch den Vorspann gesetzt.

%% latex-abspann.tex
%% Gemeinsamer Abspann für Korrekturansicht und Leseansicht.
%% Setzt den Schalter \ifkorrekturansicht voraus (gesetzt in den
%% einbindenden Dateien latex-korrekturansicht-abspann.tex bzw.
%% latex-leseansicht-abspann.tex).
%% ---------------------------------------------------------------

\normalsize

% Das esempio-Environment wird nur in der Leseansicht benötigt
\ifkorrekturansicht\else
\newenvironment{esempio}[3]%
{
    \vspace{1.5ex}
    \rlap{\underline{#1}}
    \par
    \setlength{\parindent}{0cm}
    \nopagebreak
    \leftskip=#2cm
    \rightskip=#3cm
}
{
    \par
}
\fi

\doendnotes{C}
\bigskip
\vfill

\clearpage

\footnotesize

\ifkorrekturansicht
  \lohead{\textsc{register}}
\fi

% theindex-Environment neu definieren ohne reledmac
\makeatletter
\renewenvironment{theindex}{%
  \ifkorrekturansicht
    \section*{\indexname}%
  \else
    \subsubsection*{Index der erwähnten Entitäten}%
  \fi
  \setlength{\parindent}{0pt}%
  \setlength{\parskip}{0pt plus 0.3pt}%
  \let\item\@idxitem
}{%
  \ifkorrekturansicht\clearpage\fi
}
\makeatother

\IfFileExists{\jobname-pw.ind}{\input{\jobname-pw.ind}}{}

% Quellenangabe nur in der Leseansicht
\ifkorrekturansicht\else
% Fallback-Definitionen, falls die .tex-Datei \titel etc. nicht gesetzt hat
\providecommand{\titel}{}
\providecommand{\editorInnen}{}
\providecommand{\dateiname}{\jobname}

\vspace{3cm}

\vfill

\footnotesize
\textsc{Quelle}: \titel. Herausgegeben von {\editorInnen}. In: \emph{Arthur Schnitzler: Briefwechsel mit Autorinnen und Autoren}.
 Digitale Edition, https://schnitzler-briefe.acdh.oeaw.ac.at/{\dateiname}.html (Stand \today)
\fi

\end{document}


