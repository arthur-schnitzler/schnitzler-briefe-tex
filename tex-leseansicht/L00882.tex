%% latex-leseansicht-vorspann.tex
%% Vorspann für die Leseansicht.
%% Lädt die gemeinsame Datei latex-vorspann.tex mit nicht gesetztem Schalter.

\newif\ifkorrekturansicht
\korrekturansichtfalse

\input{../tex-inputs/latex-vorspann}


         \renewcommand{\erwaehnteInstitutionen}{Institutionen: Neue Freie Presse}
         \renewcommand{\erwaehnteOrte}{Orte: Dänemark, Frankgasse 1, IX., Alsergrund, Kopenhagen, Preußen, Wien}
         \renewcommand{\erwaehnteWerke}{Werke: Cosmopolis, Die Frau des Weisen. Novelletten, Die Toten schweigen, Harald Haarfager in Finmarken, Köllers Erfolge, Reconvalescent-Besuch, Ungdomsvers [Jugendgedichte]}
               \section[Georg Brandes an Arthur Schnitzler, 22. 1. 1899]{ Georg Brandes an Arthur Schnitzler, 22. 1. 1899}\nopagebreak\mylabel{v}\rehead{ }\begin{ledgroupsized}[t]{13cm}\normalsize\beginnumbering \toendnotes[C]{\smallbreak\pagebreak[2]} \Standort{CUL, Schnitzler, B 17.}
\physDesc{Postkarte, 1113 Zeichen
\newline{}Handschrift: blaue Tinte, lateinische Kurrent
\newline{}Versand: 1) Stempel: »\nobreak{}\oindex{Kopenhagen@\textbf{Kopenhagen}|pwk}Kobenhavn, 22. 1. 99, 3–4 E\nobreak{}«.   2) Stempel: »\nobreak{}\oindex{IX., Alsergrund@\textbf{IX., Alsergrund}|pwk}Wien 9/3, 24. 1. 99, 8. V, Bestellt\nobreak{}«. 
\newline{}Ordnung: mit Bleistift von unbekannter Hand nummeriert:
                                    »13« }\buchAbdrucke{\weitereDrucke{Georg Brandes, Arthur Schnitzler: \emph{Ein Briefwechsel}. Hg. Kurt Bergel. Bern: \emph{Francke} 1956, S. 72–73.} }\toendnotes[C]{\smallbreak}\pstart{}{\pb}Herrn Dr. Arthur
                  Schnitzler\pend{}\pstart{}Frankgasse 1\oindex{Frankgasse 1@\textbf{Frankgasse 1}|pw}\pend{}\pstart{}Wien IX\oindex{IX., Alsergrund@\textbf{IX., Alsergrund}|pw}\pend{}{\bigskip}\pstart
           \raggedleft{}{\pb}22 Januar 99\pend
           \pstart
           Lieber Herr Doctor! Es war ein Fehler von mir dass ich nicht für die
                  Novellensammlung\pwindex{Schnitzler, Arthur 15.05.1862 – 21.10.1931@\textsc{Schnitzler, Arthur} (15.05.1862 – 21.10.1931), \emph{Schriftsteller, Mediziner}!Frau des Weisen. Novelletten1898-05-03@\strich\emph{Die Frau des Weisen. Novelletten} {[}1898-05-03{]}|pwv} dankte.
               ich habe sie mit grosser Aufmerksamkeit gelesen. Für mich ist die Novelle\pwindex{Schnitzler, Arthur 15.05.1862 – 21.10.1931@\textsc{Schnitzler, Arthur} (15.05.1862 – 21.10.1931), \emph{Schriftsteller, Mediziner}!Toten schweigen01. 10. 1897@\strich\emph{Die Toten schweigen} {[}01. 10. 1897{]}|pwv} die zuerst in Cosmopolis\pwindex{?? Werk@Nicht ermittelte Verfasserinnen und Verfasser!Cosmopolis1896 – 1898@\emph{Cosmopolis} {[}1896 – 1898{]}|pw} stand – ich erinnere mich nicht des Titels – ein \uline{Meisterwerk} erstaunlich wahr und packend; nur ein
               (sehr kleiner) Fehler gegen den Schluss, dass die Frau zuletzt alles gesteht. Als ob
               Frauen je geständen, wenn keine Beweise vorliegen, und wenn sie keinem absolut
               überlegenen Mann gegenüber stehen! Ein wahres Meisterwerk ist es dennoch.\pend
           \pstart
           Meine Gedichte\pwindex{Brandes, Georg 04.02.1842 – 19.02.1927@\textsc{Brandes, Georg} (04.02.1842 – 19.02.1927)!Ungdomsvers [Jugendgedichte]1898@\strich\emph{Ungdomsvers [Jugendgedichte]} {[}1898{]}|pwv}! Was soll ich
               darüber sagen. Lesen Sie Dänisch\oindex{Daenemark@\textbf{Dänemark}|pw}, so werden Sie
               einräumen dass zwei oder drei sehr gut sind, »Reconvalescent-Besuch\pwindex{Brandes, Georg 04.02.1842 – 19.02.1927@\textsc{Brandes, Georg} (04.02.1842 – 19.02.1927)!Reconvalescent-Besuch1898@\strich\emph{Reconvalescent-Besuch} {[}1898{]}|pw}« und »Harald Haarfager
                  in Finmarken\pwindex{Brandes, Georg 04.02.1842 – 19.02.1927@\textsc{Brandes, Georg} (04.02.1842 – 19.02.1927)!Harald Haarfager in Finmarken1898@\strich\emph{Harald Haarfager in Finmarken} {[}1898{]}|pw}«. Es ist eine Art Jugend-Tagebuch. – Ich liege noch immer zu
               Bett, schon 5 Wochen, Sie wissen ja was Venenentzündung ist. Doch ist es diesmal
               anscheinend nicht schlimm. Beste Grüsse \spacefill\mbox{G. B.}\pend
           \pstart
           \noindent{}\label{T_L00882-1v}\edtext{Sie haben wohl meinen Protest gegen die Ausweisungen
                     der Dänen\pwindex{?? Werk@Nicht ermittelte Verfasserinnen und Verfasser!Koellers Erfolge05. 01. 1899@\emph{Köllers Erfolge} {[}05. 01. 1899{]}|pwv} gelesen, oder auch nicht. 100 Zeitungen aller Länder haben ihn
                  abgedruckt aber die Neue Freie\orgindex{Neue Freie Presse@Neue Freie Presse|pw} ist ja preussisch\oindex{Preussen@\textbf{Preußen}|pw}.}{\lemma{\textnormal{\emph{Sie … preussisch.}}}\Cendnote{\textnormal{am linken Rand}}}\label{T_L00882-1h}\pend
           
         
         \endnumbering\mylabel{h}\end{ledgroupsized}  \newcommand{\dateiname}{L00882}\newcommand{\titel}{Georg Brandes an Arthur Schnitzler, 22. 1. 1899}\newcommand{\editorInnen}{Martin Anton Müller und Gerd-Hermann Susen}%% latex-leseansicht-abspann.tex
%% Abspann für die Leseansicht.
%% Der Schalter \ifkorrekturansicht ist bereits durch den Vorspann gesetzt.

%% latex-abspann.tex
%% Gemeinsamer Abspann für Korrekturansicht und Leseansicht.
%% Setzt den Schalter \ifkorrekturansicht voraus (gesetzt in den
%% einbindenden Dateien latex-korrekturansicht-abspann.tex bzw.
%% latex-leseansicht-abspann.tex).
%% ---------------------------------------------------------------

\normalsize

% Das esempio-Environment wird nur in der Leseansicht benötigt
\ifkorrekturansicht\else
\newenvironment{esempio}[3]%
{
    \vspace{1.5ex}
    \rlap{\underline{#1}}
    \par
    \setlength{\parindent}{0cm}
    \nopagebreak
    \leftskip=#2cm
    \rightskip=#3cm
}
{
    \par
}
\fi

\doendnotes{C}
\bigskip
\vfill

\clearpage

\footnotesize

\ifkorrekturansicht
  \lohead{\textsc{register}}
\fi

% theindex-Environment neu definieren ohne reledmac
\makeatletter
\renewenvironment{theindex}{%
  \ifkorrekturansicht
    \section*{\indexname}%
  \else
    \subsubsection*{Index der erwähnten Entitäten}%
  \fi
  \setlength{\parindent}{0pt}%
  \setlength{\parskip}{0pt plus 0.3pt}%
  \let\item\@idxitem
}{%
  \ifkorrekturansicht\clearpage\fi
}
\makeatother

\IfFileExists{\jobname-pw.ind}{\input{\jobname-pw.ind}}{}

% Quellenangabe nur in der Leseansicht
\ifkorrekturansicht\else
% Fallback-Definitionen, falls die .tex-Datei \titel etc. nicht gesetzt hat
\providecommand{\titel}{}
\providecommand{\editorInnen}{}
\providecommand{\dateiname}{\jobname}

\vspace{3cm}

\vfill

\footnotesize
\textsc{Quelle}: \titel. Herausgegeben von {\editorInnen}. In: \emph{Arthur Schnitzler: Briefwechsel mit Autorinnen und Autoren}.
 Digitale Edition, https://schnitzler-briefe.acdh.oeaw.ac.at/{\dateiname}.html (Stand \today)
\fi

\end{document}


      