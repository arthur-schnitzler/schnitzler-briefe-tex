%% latex-korrekturansicht-vorspann.tex
%% Vorspann für die Korrekturansicht.
%% Lädt die gemeinsame Datei latex-vorspann.tex mit gesetztem Schalter.

\newif\ifkorrekturansicht
\korrekturansichttrue

\input{../tex-inputs/latex-vorspann}


\section[Arthur Schnitzler an Hermann Bahr, 20. 5. 1907]{L01677 Arthur Schnitzler an Hermann Bahr, 20. 5. 1907}
\nopagebreak\mylabel{L01677v}
\rehead{ }\normalsize\beginnumbering\briefempfaengerindex{Bahr, Hermann@\textsc{Bahr, Hermann}!zzzSchnitzler, Arthur@\emph{von Arthur Schnitzler}!1907-05-201@{20. 5. 1907}|(be}
\toendnotes[C]{\smallbreak\pagebreak[2]}\Standort{TMW, HS AM 23385 Ba.}
\physDesc{Kartenbrief, 370 Zeichen
\newline{}Handschrift: schwarze Tinte, deutsche Kurrent
\newline{}Versand: 1) Stempel: »\nobreak{}Wien, 20.V{[}.07{]}, 7–8\nobreak{}«.   2) Stempel: »\nobreak{}Wien, 21. V. 07\nobreak{}«. 
\newline{}Ordnung: Lochung }
\buchAbdrucke{\weitereDrucke{1) Arthur Schnitzler: \emph{The Letters of Arthur Schnitzler to Hermann Bahr}. Chapel Hill: \emph{The University of North Carolina Press} 1978, S. 98.} \weitereDrucke{2) Hermann Bahr, Arthur Schnitzler: \emph{Briefwechsel, Aufzeichnungen, Dokumente (1891–1931)}. Göttingen: \emph{Wallstein} 2018, S. 393.} }\toendnotes[C]{\smallbreak}\pstart{}{\pb}Herrn \textsc{Hermann Bahr,}\pend{}\pstart{}\textsc{Wien Ob St Veit\oindex{Ober Sankt Veit@\textbf{Ober Sankt Veit}, \emph{P.PPLX}|pw}}\pend{}\pstart{}\textsc{Veitlissengasse\oindex{Veitlissengasse@\textbf{Veitlissengasse}, \emph{Straße (K.STR)}|pw}.}\pend{}{\bigskip}\vspace{1em}
\pstart
           \raggedleft{}{\pb}20/5 907\pend
           
\pstart{}lieber Hermann, \pend\vspace{0.5em}
\pstart
           gar nichts wichtiges. Wollte dich nur wieder einmal ſehn. Schreib mir, wann du wieder
               aus deiner Welt emportauchſt. Vielleicht fahren wir Ende \substVorne{}\textsuperscript{nächſter}\substDazwischen{}der\substHinten{} Woche auf ein paar Tage in die Brühl\oindex{Bruehl@\textbf{Brühl}, \emph{Tal (N.TAL)}|pw}.
               Du haſt hoffentlich deine \label{K_L01677-1v}\edtext{Meeresvilla}{\lemma{\textnormal{\emph{Meeresvilla}}}\Cendnote{\textnormal{Den Sommer verbrachten
                     Bahr\pwindex{Bahr, Hermann 19.07.1863 – 15.01.1934@\textsc{Bahr, Hermann} (19.07.1863 – 15.01.1934), \emph{Schriftsteller/Schriftstellerin, Kritiker/Kritikerin}|pwk} und Mildenburg\pwindex{Bahr-Mildenburg, Anna 29.11.1872 – 27.01.1947@\textsc{Bahr-Mildenburg, Anna} (29.11.1872 – 27.01.1947), \emph{Sänger/Sängerin}|pwk} jedoch in einem Hotel am Lido\oindex{Lido@\textbf{Lido}, \emph{P.PPL}|pwk}.}}}\label{K_L01677-1} gefunden. Brehm\pwindex{Brehms Tierleben@\emph{Brehms Tierleben}|pw} behalte natürlich ſo lang du willst.\pend
           
\pstart
           Von Herzen dein{\\[\baselineskip]}\spacefill\mbox{Arthur.}\pend
           \leftskip=0em{}\selectlanguage{ngerman}\endnumbering\briefempfaengerindex{Bahr, Hermann@\textsc{Bahr, Hermann}!zzzSchnitzler, Arthur@\emph{von Arthur Schnitzler}!1907-05-201@{20. 5. 1907}|)be}\mylabel{L01677h}  \normalsize

\doendnotes{C}
\bigskip
\vfill

\clearpage

\footnotesize

\lohead{\textsc{register}}

% Definiere theindex-Environment komplett neu ohne reledmac
\makeatletter
\renewenvironment{theindex}{%
  \section*{\indexname}%
  \setlength{\parindent}{0pt}%
  \setlength{\parskip}{0pt plus 0.3pt}%
  \let\item\@idxitem
}{%
  \clearpage
}
\makeatother

\IfFileExists{\jobname-pw.ind}{\input{\jobname-pw.ind}}{}

\end{document}

      