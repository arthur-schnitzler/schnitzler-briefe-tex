%% latex-leseansicht-vorspann.tex
%% Vorspann für die Leseansicht.
%% Lädt die gemeinsame Datei latex-vorspann.tex mit nicht gesetztem Schalter.

\newif\ifkorrekturansicht
\korrekturansichtfalse

\input{../tex-inputs/latex-vorspann}


         
         \renewcommand{\erwaehntePersonen}{Personen: Hermann Bahr, Anna Bahr-Mildenburg}
         \renewcommand{\erwaehnteOrte}{Orte: Brühl, Lido, Ober Sankt Veit, Veitlissengasse, Wien}
         \renewcommand{\erwaehnteWerke}{Werke: Brehms Tierleben}
               \section[Arthur Schnitzler an Hermann Bahr, 20. 5. 1907]{ Arthur Schnitzler an Hermann Bahr, 20. 5. 1907}\nopagebreak\mylabel{v}\rehead{ }\begin{ledgroupsized}[t]{13cm}\normalsize\beginnumbering \toendnotes[C]{\smallbreak\pagebreak[2]} \Standort{TMW, HS AM 23385 Ba.}
\physDesc{Kartenbrief, 370 Zeichen
\newline{}Handschrift: schwarze Tinte, deutsche Kurrent
\newline{}Versand: 1) Stempel: »\nobreak{}Wien, 20.V{[}.07{]}, 7–8\nobreak{}«.   2) Stempel: »\nobreak{}Wien, 21. V. 07\nobreak{}«. 
\newline{}Ordnung: Lochung }\buchAbdrucke{\weitereDrucke{1) \emph{20. 5. 1907.} In: Arthur Schnitzler: \emph{The Letters of Arthur Schnitzler to Hermann Bahr}. Edited, annotated, and with an introduction, by Donald G.
                        Daviau. Chapel Hill: \emph{The University of North Carolina Press} 1978, S. 98 (University of North Carolina studies in the Germanic languages
                        and literatures, 89).} \weitereDrucke{2) Hermann Bahr, Arthur Schnitzler: \emph{Briefwechsel, Aufzeichnungen, Dokumente (1891–1931)}. Hg. Kurt Ifkovits und Martin Anton Müller. Göttingen: \emph{Wallstein} 2018, S. 393.} }\toendnotes[C]{\smallbreak}\pstart{}{\pb}Herrn \textsc{Hermann Bahr,}\pend{}\pstart{}\textsc{Wien Ob St Veit\oindex{Ober Sankt Veit@\textbf{Ober Sankt Veit}|pw}}\pend{}\pstart{}\textsc{Veitlissengasse\oindex{Veitlissengasse@\textbf{Veitlissengasse}|pw}.}\pend{}{\bigskip}\pstart
           \raggedleft{}{\pb}20/5 907\pend
           \pstart{}lieber Hermann, \pend\pstart
           gar nichts wichtiges. Wollte dich nur wieder einmal ſehn. Schreib mir, wann du wieder
               aus deiner Welt emportauchſt. Vielleicht fahren wir Ende \substVorne{}\textsuperscript{nächſter}{\allowbreak}\substDazwischen{}der\substHinten{} Woche auf ein paar Tage in die Brühl\oindex{Bruehl@\textbf{Brühl}|pw}.
               Du haſt hoffentlich deine \label{K_L01677-1v}\edtext{Meeresvilla}{\lemma{\textnormal{\emph{Meeresvilla}}}\Cendnote{\textnormal{Den Sommer verbrachten
                     Bahr\pwindex{Bahr, Hermann 19.07.1863 – 15.01.1934@\textsc{Bahr, Hermann} (19.07.1863 – 15.01.1934), \emph{Schriftsteller, Kritiker}|pwk} und Mildenburg\pwindex{Bahr-Mildenburg, Anna 29.11.1872 – 27.01.1947@\textsc{Bahr-Mildenburg, Anna} (29.11.1872 – 27.01.1947), \emph{Sängerin}|pwk} jedoch in einem Hotel am Lido\oindex{Lido@\textbf{Lido}|pwk}.}}}\label{K_L01677-1h} gefunden. Brehm\pwindex{\textcolor{red}{\textsuperscript{XXXX1 indx}}!Brehms Tierleben1863 – 1869@\strich\emph{Brehms Tierleben} {[}1863 – 1869{]}|pw} behalte natürlich ſo lang du willst.\pend
           \pstart
           Von Herzen dein{\\[\baselineskip]}\spacefill\mbox{Arthur.}\pend
           \leftskip=0em{}
         
         \endnumbering\mylabel{h}\end{ledgroupsized}  \newcommand{\dateiname}{L01677}\newcommand{\titel}{Arthur Schnitzler an Hermann Bahr, 20. 5. 1907}\newcommand{\editorInnen}{ Kurt Ifkovits,  Martin Anton Müller}%% latex-leseansicht-abspann.tex
%% Abspann für die Leseansicht.
%% Der Schalter \ifkorrekturansicht ist bereits durch den Vorspann gesetzt.

%% latex-abspann.tex
%% Gemeinsamer Abspann für Korrekturansicht und Leseansicht.
%% Setzt den Schalter \ifkorrekturansicht voraus (gesetzt in den
%% einbindenden Dateien latex-korrekturansicht-abspann.tex bzw.
%% latex-leseansicht-abspann.tex).
%% ---------------------------------------------------------------

\normalsize

% Das esempio-Environment wird nur in der Leseansicht benötigt
\ifkorrekturansicht\else
\newenvironment{esempio}[3]%
{
    \vspace{1.5ex}
    \rlap{\underline{#1}}
    \par
    \setlength{\parindent}{0cm}
    \nopagebreak
    \leftskip=#2cm
    \rightskip=#3cm
}
{
    \par
}
\fi

\doendnotes{C}
\bigskip
\vfill

\clearpage

\footnotesize

\ifkorrekturansicht
  \lohead{\textsc{register}}
\fi

% theindex-Environment neu definieren ohne reledmac
\makeatletter
\renewenvironment{theindex}{%
  \ifkorrekturansicht
    \section*{\indexname}%
  \else
    \subsubsection*{Index der erwähnten Entitäten}%
  \fi
  \setlength{\parindent}{0pt}%
  \setlength{\parskip}{0pt plus 0.3pt}%
  \let\item\@idxitem
}{%
  \ifkorrekturansicht\clearpage\fi
}
\makeatother

\IfFileExists{\jobname-pw.ind}{\input{\jobname-pw.ind}}{}

% Quellenangabe nur in der Leseansicht
\ifkorrekturansicht\else
% Fallback-Definitionen, falls die .tex-Datei \titel etc. nicht gesetzt hat
\providecommand{\titel}{}
\providecommand{\editorInnen}{}
\providecommand{\dateiname}{\jobname}

\vspace{3cm}

\vfill

\footnotesize
\textsc{Quelle}: \titel. Herausgegeben von {\editorInnen}. In: \emph{Arthur Schnitzler: Briefwechsel mit Autorinnen und Autoren}.
 Digitale Edition, https://schnitzler-briefe.acdh.oeaw.ac.at/{\dateiname}.html (Stand \today)
\fi

\end{document}


      