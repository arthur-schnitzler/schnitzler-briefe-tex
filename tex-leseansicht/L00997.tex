%% latex-korrekturansicht-vorspann.tex
%% Vorspann für die Korrekturansicht.
%% Lädt die gemeinsame Datei latex-vorspann.tex mit gesetztem Schalter.

\newif\ifkorrekturansicht
\korrekturansichttrue

\input{../tex-inputs/latex-vorspann}


\section[Arthur Schnitzler an Hugo von Hofmannsthal, {[}18. 11. 1899?{]}]{L00997 Arthur Schnitzler an Hugo von Hofmannsthal, {[}18. 11. 1899?{]}}
\nopagebreak\mylabel{L00997v}
\rehead{ }\normalsize\beginnumbering\briefempfaengerindex{Hofmannsthal, Hugo von@\textsc{Hofmannsthal, Hugo von}!zzzSchnitzler, Arthur@\emph{von Arthur Schnitzler}!1899-11-181@{{[}18. 11. 1899?{]}}|(be}
\toendnotes[C]{\smallbreak\pagebreak[2]}\Standort{FDH, Hs-30885,89.}
\physDesc{Brief, 1 Blatt, 2 Seiten, 176 Zeichen
\newline{}Handschrift: Bleistift, deutsche Kurrent
\newline{}Ordnung: mit Bleistift von Schnitzler mutmaßlich bei der Durchsicht der Briefe
                                    1929 datiert: »99?« }
\buchAbdrucke{\weitereDrucke{Hugo von Hofmannsthal, Arthur Schnitzler: \emph{Briefwechsel}. Frankfurt am Main: \emph{S. Fischer} 1964, S. 117.} }\toendnotes[C]{\smallbreak}
\pstart{}{\pb}Mein lieber Hugo,\pend\vspace{0.5em}
\pstart
           Sie sehen, ich \label{K_L00997-1v}\edtext{ka{\geminationn} nicht ko{\geminationm}en}{\lemma{\textnormal{\emph{kann nicht kommen}}}\Cendnote{\textnormal{Die Datierung dieses Briefes ist mit vielen
                  Zweifeln behaftet. Sofern die handschriftlich von Schnitzler angebrachte Jahresangabe zutrifft – sie ist mit Fragezeichen
                  versehen – ist dies die beste Platzierung innerhalb der überlieferten Dokumente
                  dieses Jahres. Hofmannsthal\pwindex{Hofmannsthal, Hugo von 1874-02-01 – 1929-07-15@\textsc{Hofmannsthal, Hugo von} (1874-02-01 – 1929-07-15), \emph{Schriftsteller/Schriftstellerin}|pwk} bat am 17. 11. 1899 um ein Treffen
                  für den Folgetag, das bei Beer-Hofmann\pwindex{Beer-Hofmann, Richard 1866-07-11 – 1945-09-26@\textsc{Beer-Hofmann, Richard} (1866-07-11 – 1945-09-26), \emph{Schriftsteller/Schriftstellerin}|pwk}
                  begonnen und dann ins Kaffeehaus geführt hätte. Das Treffen kam nicht zustande
                  und dieses Schreiben könnte die Absage darstellen. Unbeantwortet bleibt damit
                  aber, warum Schnitzler{ }Beer-Hofmann\pwindex{Beer-Hofmann, Richard 1866-07-11 – 1945-09-26@\textsc{Beer-Hofmann, Richard} (1866-07-11 – 1945-09-26), \emph{Schriftsteller/Schriftstellerin}|pwk} anzurufen
                  gedenkt.}}}\label{K_L00997-1}, auch nicht ins Café{\dots}\pend
           
\pstart
           Alles Gute Ihnen!\pend
           
\pstart
           – Ich werde möglicherweiſe Richard\pwindex{Beer-Hofmann, Richard 1866-07-11 – 1945-09-26@\textsc{Beer-Hofmann, Richard} (1866-07-11 – 1945-09-26), \emph{Schriftsteller/Schriftstellerin}|pw} ſpät Nachts
               im Café te{\pb}lephoniſch anrufen.\pend
           
\pstart
           Ihr treuer{\\}\spacefill\mbox{Arthur}\pend
           \selectlanguage{ngerman}\endnumbering\briefempfaengerindex{Hofmannsthal, Hugo von@\textsc{Hofmannsthal, Hugo von}!zzzSchnitzler, Arthur@\emph{von Arthur Schnitzler}!1899-11-181@{{[}18. 11. 1899?{]}}|)be}\mylabel{L00997h}  \normalsize

\doendnotes{C}
\bigskip
\vfill

\clearpage

\footnotesize

\lohead{\textsc{register}}

% Definiere theindex-Environment komplett neu ohne reledmac
\makeatletter
\renewenvironment{theindex}{%
  \section*{\indexname}%
  \setlength{\parindent}{0pt}%
  \setlength{\parskip}{0pt plus 0.3pt}%
  \let\item\@idxitem
}{%
  \clearpage
}
\makeatother

\IfFileExists{\jobname-pw.ind}{\input{\jobname-pw.ind}}{}

\end{document}

      