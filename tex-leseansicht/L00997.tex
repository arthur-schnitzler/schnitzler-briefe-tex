\input{../tex-inputs/latex-pdf-vorspann}
\begin{center}
            \textcolor{red}{ENTWURF. ENTZIFFERUNG NOCH NICHT KORREKTURGELESEN}
                      \end{center}
            
               \section[Arthur Schnitzler an Hugo von Hofmannsthal, {[}18. 11. 1899?{]}]{ Arthur Schnitzler an Hugo von Hofmannsthal,
                    {[}18. 11. 1899?{]}}\nopagebreak\mylabel{v}\rehead{ }\begin{ledgroupsized}[t]{13cm}\normalsize\beginnumbering\briefempfaengerindex{Hofmannsthal, Hugo von@\textsc{Hofmannsthal, Hugo von}!zzzSchnitzler, Arthur@\emph{von Arthur Schnitzler}!1899-11-181@{{[}18. 11. 1899?{]}}|(be} \toendnotes[C]{\smallbreak\pagebreak[2]} \Standort{FDH, Hs-30885,89.}
\physDesc{Brief, 1 Blatt, 2 Seiten
\newline{}Handschrift: Bleistift, deutsche Kurrent\newline{}Ordnung: von Schnitzler mutmaßlich bei der Durchsicht der Briefe
                                   1929 mit
                                    Bleistift datiert: »99?« }\buchAbdrucke{\weitereDrucke{Hugo von Hofmannsthal, Arthur Schnitzler: \emph{Briefwechsel}. Hg. Therese Nickl und Heinrich Schnitzler. Frankfurt am Main: \emph{S. Fischer} 1964, S. 117.} }\toendnotes[C]{\smallbreak}\pstart{}{\pb}Mein lieber Hugo,\pend\pstart
           Sie sehen, ich \label{K_L00997_1v}\edtext{ka{\geminationn}
                    nicht ko{\geminationm}en}{\lemma{\textnormal{\emph{ka
                    nicht koen}}}\Cendnote{\textnormal{Die Datierung dieses Briefes ist
                        mit vielen Zweifeln behaftet. Sofern die handschriftlich von Schnitzler\pwindex{Schnitzler, Arthur 15.05.1862 – 21.10.1931@\textsc{Schnitzler, Arthur} (15.05.1862 – 21.10.1931), \emph{Schriftsteller, Mediziner}|pwk} angebrachte Jahresangabe
                        zutrifft – sie ist mit Fragezeichen versehen – ist dies die beste
                        Platzierung innerhalb der überlieferten Dokumente dieses Jahres. Hofmannsthal\pwindex{Hofmannsthal, Hugo von 01.02.1874 – 15.07.1929@\textsc{Hofmannsthal, Hugo von} (01.02.1874 – 15.07.1929), \emph{Schriftsteller}|pwk} bat am 17. 11. 1899 um ein Treffen für den Folgetag, das bei Beer-Hofmann\pwindex{Beer-Hofmann, Richard 11.07.1866 – 26.09.1945@\textsc{Beer-Hofmann, Richard} (11.07.1866 – 26.09.1945), \emph{Schriftsteller}|pwk} begonnen und dann ins
                        Kaffeehaus geführt hätte. Das Treffen kam nicht zu Stande und dieses
                        Schreiben könnte die Absage darstellen. Unbeantwortet bleibt damit aber,
                        warum er Beer-Hofmann\pwindex{Beer-Hofmann, Richard 11.07.1866 – 26.09.1945@\textsc{Beer-Hofmann, Richard} (11.07.1866 – 26.09.1945), \emph{Schriftsteller}|pwk} anzurufen
                        gedenkt.}}}\label{K_L00997_1h}, auch nicht ins Café{\dots}\pend
           \pstart
           Alles Gute Ihnen!\pend
           \pstart
           – Ich werde möglicherweiſe Richard\pwindex{Beer-Hofmann, Richard 11.07.1866 – 26.09.1945@\textsc{Beer-Hofmann, Richard} (11.07.1866 – 26.09.1945), \emph{Schriftsteller}|pw} ſpät
                    Nachts im Café te{\pb}lephoniſch anrufen.\pend
           \pstart
           Ihr treuer{\\}\spacefill\mbox{Arthur}\pend
           \endnumbering\briefempfaengerindex{Hofmannsthal, Hugo von@\textsc{Hofmannsthal, Hugo von}!zzzSchnitzler, Arthur@\emph{von Arthur Schnitzler}!1899-11-181@{{[}18. 11. 1899?{]}}|)be}\mylabel{h}\end{ledgroupsized}  \newcommand{\dateiname}{L00997}\newcommand{\titel}{Arthur Schnitzler an Hugo von Hofmannsthal, [18. 11. 1899?]}\newcommand{\editorInnen}{Martin Anton Müller und Gerd-Hermann Susen}\input{../tex-inputs/latex-pdf-abspann}
      