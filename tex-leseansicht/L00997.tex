%% latex-leseansicht-vorspann.tex
%% Vorspann für die Leseansicht.
%% Lädt die gemeinsame Datei latex-vorspann.tex mit nicht gesetztem Schalter.

\newif\ifkorrekturansicht
\korrekturansichtfalse

\input{../tex-inputs/latex-vorspann}


\section[Arthur Schnitzler an Hugo von Hofmannsthal, {{[}}18. 11. 1899?{{]}}]{L00997 Arthur Schnitzler an Hugo von Hofmannsthal, {[}18. 11. 1899?{]}}
\nopagebreak\mylabel{L00997v}
\rehead{ }\normalsize\beginnumbering\briefempfaengerindex{Hofmannsthal, Hugo von@\textsc{Hofmannsthal, Hugo von}!zzzSchnitzler, Arthur@\emph{von Arthur Schnitzler}!1899-11-181@{{[}18. 11. 1899?{]}}|(be}
\toendnotes[C]{\smallbreak\pagebreak[2]}
\correspDesc{Versand  durch Arthur Schnitzler am [18. 11. 1899?] in Wien
\newline{}Erhalt  durch Hugo von Hofmannsthal im Zeitraum [18. 11. 1899 – 22. 11. 1899?] in Wien}\toendnotes[C]{\smallbreak}
\Standort{FDH, Hs-30885,89.}
\physDesc{Brief, 1 Blatt, 2 Seiten, 176 Zeichen
\newline{}Handschrift: Bleistift, deutsche Kurrent
\newline{}Ordnung: mit Bleistift von Schnitzler mutmaßlich bei der Durchsicht der Briefe
                                    1929 datiert: »99?« }
\buchAbdrucke{\weitereDrucke{Hugo von Hofmannsthal, Arthur Schnitzler: \emph{Briefwechsel}. Herausgegeben von Therese Nickl und Heinrich Schnitzler. Frankfurt am Main: \emph{S. Fischer} 1964, S. 117.} }\toendnotes[C]{\smallbreak}
\pstart{}{\pb}Mein lieber Hugo,\pend\vspace{0.5em}
\pstart
           Sie sehen, ich \label{K_L00997-1v}\edtext{ka{\geminationn} nicht ko{\geminationm}en}{\lemma{\textnormal{\emph{kann nicht kommen}}}\Cendnote{\textnormal{Die Datierung dieses Briefes ist mit vielen
                  Zweifeln behaftet. Sofern die handschriftlich von Schnitzler angebrachte Jahresangabe zutrifft – sie ist mit Fragezeichen
                  versehen – ist dies die beste Platzierung innerhalb der überlieferten Dokumente
                  dieses Jahres. Hofmannsthal\pwindex{Hofmannsthal, Hugo von 1.\,2.\,1874 Wien – 15.\,7.\,1929 Rodaun@\textsc{Hofmannsthal, Hugo von} (1.\,2.\,1874 Wien – 15.\,7.\,1929 Rodaun), \emph{Schriftsteller}|pwk} bat am XXXX Auszeichnungsfehler: Dokument L00996 nicht gefunden um ein Treffen
                  für den Folgetag, das bei Beer-Hofmann\pwindex{Beer-Hofmann, Richard 11.\,7.\,1866 Wien – 26.\,9.\,1945 New York City@\textsc{Beer-Hofmann, Richard} (11.\,7.\,1866 Wien – 26.\,9.\,1945 New York City), \emph{Schriftsteller}|pwk}
                  begonnen und dann ins Kaffeehaus geführt hätte. Das Treffen kam nicht zustande
                  und dieses Schreiben könnte die Absage darstellen. Unbeantwortet bleibt damit
                  aber, warum Schnitzler{ }Beer-Hofmann\pwindex{Beer-Hofmann, Richard 11.\,7.\,1866 Wien – 26.\,9.\,1945 New York City@\textsc{Beer-Hofmann, Richard} (11.\,7.\,1866 Wien – 26.\,9.\,1945 New York City), \emph{Schriftsteller}|pwk} anzurufen
                  gedenkt.}}}\label{K_L00997-1}, auch nicht ins Café{\dots}\pend
           
\pstart
           Alles Gute Ihnen!\pend
           
\pstart
           – Ich werde möglicherweiſe Richard\pwindex{Beer-Hofmann, Richard 11.\,7.\,1866 Wien – 26.\,9.\,1945 New York City@\textsc{Beer-Hofmann, Richard} (11.\,7.\,1866 Wien – 26.\,9.\,1945 New York City), \emph{Schriftsteller}|pw}{ }ſpät Nachts
               im Café te{\pb}lephoniſch anrufen.\pend
           
\pstart
           Ihr treuer{\\}\spacefill\mbox{Arthur}\pend
           \selectlanguage{ngerman}\endnumbering\briefempfaengerindex{Hofmannsthal, Hugo von@\textsc{Hofmannsthal, Hugo von}!zzzSchnitzler, Arthur@\emph{von Arthur Schnitzler}!1899-11-181@{{[}18. 11. 1899?{]}}|)be}\mylabel{L00997h}  \newcommand{\dateiname}{L00997}\newcommand{\titel}{Arthur Schnitzler an Hugo von Hofmannsthal, [18. 11. 1899?]}\newcommand{\editorInnen}{Martin Anton Müller und Gerd-Hermann Susen}%% latex-leseansicht-abspann.tex
%% Abspann für die Leseansicht.
%% Der Schalter \ifkorrekturansicht ist bereits durch den Vorspann gesetzt.

%% latex-abspann.tex
%% Gemeinsamer Abspann für Korrekturansicht und Leseansicht.
%% Setzt den Schalter \ifkorrekturansicht voraus (gesetzt in den
%% einbindenden Dateien latex-korrekturansicht-abspann.tex bzw.
%% latex-leseansicht-abspann.tex).
%% ---------------------------------------------------------------

\normalsize

% Das esempio-Environment wird nur in der Leseansicht benötigt
\ifkorrekturansicht\else
\newenvironment{esempio}[3]%
{
    \vspace{1.5ex}
    \rlap{\underline{#1}}
    \par
    \setlength{\parindent}{0cm}
    \nopagebreak
    \leftskip=#2cm
    \rightskip=#3cm
}
{
    \par
}
\fi

\doendnotes{C}
\bigskip
\vfill

\clearpage

\footnotesize

\ifkorrekturansicht
  \lohead{\textsc{register}}
\fi

% theindex-Environment neu definieren ohne reledmac
\makeatletter
\renewenvironment{theindex}{%
  \ifkorrekturansicht
    \section*{\indexname}%
  \else
    \subsubsection*{Index der erwähnten Entitäten}%
  \fi
  \setlength{\parindent}{0pt}%
  \setlength{\parskip}{0pt plus 0.3pt}%
  \let\item\@idxitem
}{%
  \ifkorrekturansicht\clearpage\fi
}
\makeatother

\IfFileExists{\jobname-pw.ind}{\input{\jobname-pw.ind}}{}

% Quellenangabe nur in der Leseansicht
\ifkorrekturansicht\else
% Fallback-Definitionen, falls die .tex-Datei \titel etc. nicht gesetzt hat
\providecommand{\titel}{}
\providecommand{\editorInnen}{}
\providecommand{\dateiname}{\jobname}

\vspace{3cm}

\vfill

\footnotesize
\textsc{Quelle}: \titel. Herausgegeben von {\editorInnen}. In: \emph{Arthur Schnitzler: Briefwechsel mit Autorinnen und Autoren}.
 Digitale Edition, https://schnitzler-briefe.acdh.oeaw.ac.at/{\dateiname}.html (Stand \today)
\fi

\end{document}


