%% latex-korrekturansicht-vorspann.tex
%% Vorspann für die Korrekturansicht.
%% Lädt die gemeinsame Datei latex-vorspann.tex mit gesetztem Schalter.

\newif\ifkorrekturansicht
\korrekturansichttrue

\input{../tex-inputs/latex-vorspann}


\section[ Paul Goldmann an Arthur Schnitzler, 3. 6. {[}1906{]}]{L03246 Paul Goldmann an Arthur Schnitzler, 3. 6. {[}1906{]}}
\nopagebreak\mylabel{L03246v}
\rehead{ }\normalsize\beginnumbering\briefempfaengerindex{Schnitzler, Arthur@\textsc{Schnitzler, Arthur}!zzzGoldmann, Paul@\emph{von Paul Goldmann}!1906-06-031@{3. 6. {[}1906{]}}|(be}
\toendnotes[C]{\smallbreak\pagebreak[2]}\Standort{DLA, A:Schnitzler, HS.NZ85.1.3175.}
\physDesc{Brief, 1 Blatt, 1 Seite, 152 Zeichen
\newline{}Handschrift: schwarze Tinte, deutsche Kurrent
\newline{}Schnitzler: mit Bleistift das Jahr »906« vermerkt }\toendnotes[C]{\smallbreak}
\pstart
           \raggedleft{}{\pb}\textcolor{gray}{\textbf{\textbf{GRAND HÔTEL\oindex{Grand Hotel Wien@\textbf{Grand Hotel Wien}, \emph{Hotel (K.HTL)}|pw}, \begin{otherlanguage}{french}VIENNE\oindex{Wien@\textbf{Wien}, \emph{A.ADM2}|pw}\end{otherlanguage}}}}\pend
           
\pstart
           \raggedleft{}\textcolor{gray}{\textbf{I., KÄRNTNERRING 9\oindex{Kaerntnerring@\textbf{Kärntnerring}, \emph{Straße (K.STR)}|pw}}}.\pend
           
\pstart
           Wien\oindex{Wien@\textbf{Wien}, \emph{A.ADM2}|pw}, 3. Juni\pend
           \vspace{0.5em}
\pstart
           Herzlichen Dank, lieber Freund! Ich komme alſo \label{K_L03246-1v}\edtext{Montag}{\lemma{\textnormal{\emph{Montag}}}\Cendnote{\textnormal{Siehe A. S.: \emph{Tagebuch}, 4. 6. 1906.
               }}}\label{K_L03246-1}{ }zwiſchen 7 u. 8. Uhr.\pend
           
\pstart
           Herzliche Grüße Dir und Deiner Frau\pwindex{Schnitzler, Olga 17.01.1882 – 13.01.1970@\textsc{Schnitzler, Olga} (17.01.1882 – 13.01.1970), \emph{Schauspieler/Schauspielerin, Sänger/Sängerin}|pwv} von Deinem getreuen{\\[\baselineskip]}\spacefill\mbox{Paul Goldmann}\pend
           \leftskip=0em{}\selectlanguage{ngerman}\endnumbering\briefempfaengerindex{Schnitzler, Arthur@\textsc{Schnitzler, Arthur}!zzzGoldmann, Paul@\emph{von Paul Goldmann}!1906-06-031@{3. 6. {[}1906{]}}|)be}\mylabel{L03246h}  \normalsize

\doendnotes{C}
\bigskip
\vfill

\clearpage

\footnotesize

\lohead{\textsc{register}}

% Definiere theindex-Environment komplett neu ohne reledmac
\makeatletter
\renewenvironment{theindex}{%
  \section*{\indexname}%
  \setlength{\parindent}{0pt}%
  \setlength{\parskip}{0pt plus 0.3pt}%
  \let\item\@idxitem
}{%
  \clearpage
}
\makeatother

\IfFileExists{\jobname-pw.ind}{\input{\jobname-pw.ind}}{}

\end{document}

      