%% latex-leseansicht-vorspann.tex
%% Vorspann für die Leseansicht.
%% Lädt die gemeinsame Datei latex-vorspann.tex mit nicht gesetztem Schalter.

\newif\ifkorrekturansicht
\korrekturansichtfalse

\input{../tex-inputs/latex-vorspann}


         
         \renewcommand{\erwaehntePersonen}{Personen: Richard Beer-Hofmann, Betty Ernst, Marie Glümer, Paul Goldmann, Emil Granichstaedten, Gisela Hajek, Markus Hajek, Friedrich Kapper, Adele Kapper, Fedor Mamroth, Hermine von Mauthner, Hans Johann von Mauthner, Stephan von Mauthner, Elise Pserhofer, Ignaz Pserhofer, Emma Pserhofer, Max von Rosenberg, Leo N. von Tolstoi, Leo Van-Jung}
         \renewcommand{\erwaehnteInstitutionen}{Institutionen: An der schönen blauen Donau, Die Presse, Josef Eberle Stein-, Buch und Musikaliendruckerei}
         \renewcommand{\erwaehnteOrte}{Orte: Berggasse, Capua, Kärnten, Pörtschach, Seidengasse, Wien}
         \renewcommand{\erwaehnteWerke}{Werke: Anatol, Bibel, Die Kreutzersonate, Tagebuch, Weihnachts-Einkäufe}
               \section[Paul Goldmann an Arthur Schnitzler, 18. 8. 1890]{ Paul Goldmann an Arthur Schnitzler, 18. 8. 1890}\nopagebreak\mylabel{v}\rehead{ }\begin{ledgroupsized}[t]{13cm}\normalsize\beginnumbering\briefempfaengerindex{Schnitzler, Arthur@\textsc{Schnitzler, Arthur}!zzzGoldmann, Paul@\emph{von Paul Goldmann}!1890-08-181@{18. 8. 1890}|(be} \toendnotes[C]{\smallbreak\pagebreak[2]} \Standort{DLA, A:Schnitzler, HS.NZ85.1.3162.}
\physDesc{Brief, 2 Blätter, 7 Seiten, 5774 Zeichen
\newline{}Handschrift: schwarze Tinte, deutsche Kurrent
\newline{}Schnitzler: mit rotem Buntstift eine Unterstreichung }\toendnotes[C]{\smallbreak}\pstart
           \noindent{}\centering{}{\pb}\textcolor{gray}{\textbf{\textbf{Adminiſtration: VII.
                           Seidengaſſe 7\oindex{Seidengasse@\textbf{Seidengasse}|pw}} (Jos. Eberle {\kaufmannsund} Co.\orgindex{Josef Eberle Stein-, Buch und Musikaliendruckerei@Josef Eberle Stein-, Buch und Musikaliendruckerei|pw})}}\pend
           \pstart
           \noindent{}\centering{}\textcolor{gray}{\textbf{An der Schönen Blauen Donau\orgindex{der schoenen blauen Donau@An der schönen blauen Donau|pw}}}\pend
           \pstart
           \noindent{}\centering{}\textcolor{gray}{\textbf{Chef-Redacteur: Dr. F.
                        Mamroth\pwindex{Mamroth, Fedor 21.02.1851 – 25.06.1907@\textsc{Mamroth, Fedor} (21.02.1851 – 25.06.1907), \emph{Journalist, Kritiker}|pw}. – Redaction: IX.,
                        Berggaſſe 31\oindex{Berggasse@\textbf{Berggasse}|pw}.}}\pend
           \pstart
           \raggedleft{}\textsc{Pörtschach\oindex{Poertschach@\textbf{Pörtschach}|pw}}{ }\textcolor{gray}{\textbf{\strikeout{Wien\oindex{Wien@\textbf{Wien}|pw}}, den}}{ }18. August \textcolor{gray}{\textbf{18}}90.\pend
           \pstart\center{}Mein lieber Arthur!\pend\pstart
           Viel Dank für Deinen lieben Brief! Ich habe mich ehrlich damit gefreut, wenigſtens
               inſoweit, als ich ſehe, daß Du meiner in Treuen gedenkſt. Was Dich angeht, freilich –
               die Nachrichten über Deine Perſon, die die Epiſtel bringt, – bin ich wenig zufrieden.
               Wenig – nein, gar nicht! Kind, Kind – ſei geſcheit! Laß’ Dich nicht ſo willenlos
               untergehen in der \label{K_L02649-1v}\edtext{Geſchichte}{\lemma{\textnormal{\emph{Geſchichte}}}\Cendnote{\textnormal{Er spielt auf die Beziehung Schnitzler\pwindex{Schnitzler, Arthur 15.05.1862 – 21.10.1931@\textsc{Schnitzler, Arthur} (15.05.1862 – 21.10.1931), \emph{Schriftsteller, Mediziner}|pwk}s mit Marie Glümer\pwindex{Gluemer, Marie 03.07.1867 – 16.11.1925@\textsc{Glümer, Marie} (03.07.1867 – 16.11.1925), \emph{Schauspielerin}|pwk} seit Juni 1889
                  an. Am 13. 7. 1889
                  nannte er sie im \emph{Tagebuch}\pwindex{Schnitzler, Arthur 15.05.1862 – 21.10.1931@\textsc{Schnitzler, Arthur} (15.05.1862 – 21.10.1931), \emph{Schriftsteller, Mediziner}!Tagebuch1981 – 2000@\strich\emph{Tagebuch} {[}1981 – 2000{]}|pwk} »das Ideal
                     des ›süßen Mädels‹, wie ichs geträumt«.}}}\label{K_L02649-1h}! Fühlen, Stimmung
               empfinden iſt gut; aber ein wenig Denken und Wollen iſt auch vonnöthen. Du brauchſt
               kein raſches Ende – \begin{otherlanguage}{french}pardon\end{otherlanguage}! – zu machen; aber
                  \textcolor{gray}{da} das Ende von ſelbst kommen wird, {\pb}wäre es Wahnſinn, ſich nicht bei Zeiten damit
               abzufinden. Jetzt haſt Du das Mädel\pwindex{Gluemer, Marie 03.07.1867 – 16.11.1925@\textsc{Glümer, Marie} (03.07.1867 – 16.11.1925), \emph{Schauspielerin}|pwv} – \textsc{\begin{otherlanguage}{french}bon\end{otherlanguage}}! – aber wenn Du das Mädel\pwindex{Gluemer, Marie 03.07.1867 – 16.11.1925@\textsc{Glümer, Marie} (03.07.1867 – 16.11.1925), \emph{Schauspielerin}|pwv} nicht mehr haſt, wirſt Du etwas viel Beſſeres wieder haben – Dich
               ſelbſt. Der Tauſch iſt, weiß Gott, kein ſchlechter. Überleg’ Dir das! Und denk’ nur
               an meine Spießbürger-Philoſophie, die aber doch die einzig geſcheite iſt: der Menſch
               iſt nicht zum Lieben allein da. Dieſes Taumeln von Rauſch zu Rauſch,
                  d\textcolor{gray}{i}eſes Selbſtzerquälen um ein Nichts iſt verderblich und
               zerrüttend. Beſonders dieſe Quälereien. Ich ſehe das ſo klar: in Dir iſt eine große
               Kunſt vorhanden, und da Du ſie nirgends hin ableiteſt, kehrt ſie ſich gegen Dich
               ſelbſt. Dieſe \label{K_L02649-2v}\edtext{Eiferſucht auf die
                  Vergangenheit}{\lemma{\textnormal{\emph{Eiferſucht … Vergangenheit}}}\Cendnote{\textnormal{Schnitzler\pwindex{Schnitzler, Arthur 15.05.1862 – 21.10.1931@\textsc{Schnitzler, Arthur} (15.05.1862 – 21.10.1931), \emph{Schriftsteller, Mediziner}|pwk} war nicht der erste Liebhaber von
                     Marie Glümer\pwindex{Gluemer, Marie 03.07.1867 – 16.11.1925@\textsc{Glümer, Marie} (03.07.1867 – 16.11.1925), \emph{Schauspielerin}|pwk} gewesen: »Ich bin
                     nie völlig glücklich mit ihr; weil ich eben das gewesene nie los werde. Sie
                     sagt, sie liebe mich unendlich mehr, ganz anders u. s. w.– Natürlich sagt sies.
                     Ja, natürlich glaubt sie’s. Es ist sonderbar, daß ich absolut nicht darüber weg
                     kann.« (A. S.: \emph{Tagebuch}, 10. 8. 1890)}}}\label{K_L02649-2h}
               iſt vielleicht nichts, als die Eiferſucht \uline{der}
               Vergangenheit, \uline{Deiner} Vergangenheit, jener Stunden,
               in denen Du geſchafft und geſtrebt haſt, jener hohen Ziele, denen Du zugeſtaunt, und
               die Dich jetzt wieder haben wollen. Nun, ſie \uline{werden}
               Dich wieder haben; und ich, der ich Dein Beſtes ſehe und will, kann das »Ende« nicht
               erwarten. Übrigens, glaube ich, es wird Dir nicht gar ſo weh thun. Dieſe tollen
               Schmerzen, die Du vorausempfindeſt, {\pb}ſtumpfen das
               Empfindungsvermögen ab, ſo daß es ſicherlich gegenüber dem großen Schmerze, wenn er
               wirklich eintritt, verſagen wird. Alſo, nochmals, ſei geſcheit: Du lebſt in \label{K_L02649-3v}\edtext{\textsc{Capua\oindex{Capua@\textbf{Capua}|pw}}}{\lemma{\textnormal{\emph{Capua}}}\Cendnote{\textnormal{Synonym für Luxus, Komfort etc.}}}\label{K_L02649-3h},
               und mußt ſroh ſein, wenn Du herauskommſt. Oder, wenn Du willſt, Du biſt im Paradieſe;
               aber, als ſrommer Bibel\pwindex{?? Werk@Nicht ermittelte Verfasserinnen und Verfasser!BibelNone@\emph{Bibel} {[}None{]}|pw}leſer, \strikeout{ist \textcolor{gray}{d}} weißt Du, daß wir Alle da nicht hineingehören; und Du wirſt Dich doch wieder
               mit der Erde befreunden müſſen, auf der zu leben ſchließlich auch nicht ohne Reiz
               iſt.\pend
           \pstart
           Dies die Moralpredigt eines Menſchen, der ſelbst nichts dringender brauchte, als eine
               ſolche. In Kurzem: auch mich hat’s wieder, mein Sohn! Das \label{K_L02649-4v}\edtext{ſüße Mädel\pwindex{Pserhofer, Elise 1872-09-18 – 1938-06-28@\textsc{Pserhofer, Elise} (1872-09-18 – 1938-06-28)|pwv}}{\lemma{\textnormal{\emph{ſüße Mädel}}}\Cendnote{\textnormal{Es handelt sich hierbei um eine frühe
                  Verwendung des von Schnitzler\pwindex{Schnitzler, Arthur 15.05.1862 – 21.10.1931@\textsc{Schnitzler, Arthur} (15.05.1862 – 21.10.1931), \emph{Schriftsteller, Mediziner}|pwk} populär
                  gemachten Begriffs. Im \emph{Tagebuch}\pwindex{Schnitzler, Arthur 15.05.1862 – 21.10.1931@\textsc{Schnitzler, Arthur} (15.05.1862 – 21.10.1931), \emph{Schriftsteller, Mediziner}!Tagebuch1981 – 2000@\strich\emph{Tagebuch} {[}1981 – 2000{]}|pwk} findet sich
                  der Begriff bereits am 19. 10. 1887. In einem veröffentlichten literarischen Text gebrauchte
                     Schnitzler\pwindex{Schnitzler, Arthur 15.05.1862 – 21.10.1931@\textsc{Schnitzler, Arthur} (15.05.1862 – 21.10.1931), \emph{Schriftsteller, Mediziner}|pwk} den Ausdruck »süßes Mädel«
                  erstmals im \emph{Anatol}\pwindex{Schnitzler, Arthur 15.05.1862 – 21.10.1931@\textsc{Schnitzler, Arthur} (15.05.1862 – 21.10.1931), \emph{Schriftsteller, Mediziner}!Anatol1892-10-29@\strich\emph{Anatol} {[}1892-10-29{]}|pwk}-Einakter \emph{Weihnachts-Einkäufe}\pwindex{Schnitzler, Arthur 15.05.1862 – 21.10.1931@\textsc{Schnitzler, Arthur} (15.05.1862 – 21.10.1931), \emph{Schriftsteller, Mediziner}!Weihnachts-Einkaeufe24. 12. 1891@\strich\emph{Weihnachts-Einkäufe} {[}24. 12. 1891{]}|pwk} (erschienen 24. 12. 1891).}}}\label{K_L02649-4h} – geſcheit, wahrhaftig und nicht \begin{otherlanguage}{french}coquett\end{otherlanguage}, das ich ſo lange mit der Laterne geſucht – mir
               ſcheint, ich hab’s gefunden. Seit geſtern ſind in mir
               wieder alle Teufel los. Und ich ſehe, es wird wieder genau die alte Geſchichte. Eine
               wahnſinnige Sehnſucht, das erblickte Glück zu faſſen, ein toller Geſühlsüberſchwang,
               ein Mich-Unwürdig-Fühlen gegenüber der Auserwählten\pwindex{Pserhofer, Elise 1872-09-18 – 1938-06-28@\textsc{Pserhofer, Elise} (1872-09-18 – 1938-06-28)|pwv} – dieſe drei Sachen, die es mir ſchon einmal
               verdorben haben, werden es mir wieder verderben. Da ſteh’ ich {\pb}nu\textcolor{gray}{n} mit meinem weltumfaſſenden
               Geiſte, und kann das praktiſche Problem nicht löſen, wie ich ein kleines Mädchen\pwindex{Pserhofer, Elise 1872-09-18 – 1938-06-28@\textsc{Pserhofer, Elise} (1872-09-18 – 1938-06-28)|pwv}herz lehren ſoll, mich
               gern zu haben. Dich quält das bevorſtehende Ende des Glücks, mich bringt es zur
               Verzweiflung, daß ich ſeinen Anfang nicht herbeiführen kann. So bin ich geſtern{ }Abend geſeſſen, den Kopf in beide Hände geſtützt und die Stirne heiß von
               Rauſch und Sehnſucht, und es hat in mir gewühlt und gewühlt und ich habe geſehen, daß
               ich ein hoffnungslos unglücklicher Menſch bin. Hab’ ich’s alſo wieder einmal mit dem
               Beten verſucht – Du weißt, ich gedenke gern des lieben Gottes, wenn ich ihn brauche –
               und warte nun ab, ob mir das vielleicht nutzen wird. Ich habe mir bei alledem ſo heiß
               gewünſcht, Du zu ſein, mit all' Deinen Reizen und \strikeout{\textcolor{gray}{Lüſten}} Liſten, Du, der Du die große Kunſt verſtehſt: geliebt zu werden. Vielleicht
               theilſt Du mir ein oder das andere \label{K_L02649-5v}\edtext{\textsc{arcanum}}{\lemma{\textnormal{\emph{arcanum}}}\Cendnote{\textnormal{lateinisch: Geheimnis}}}\label{K_L02649-5h} mit. Wie
               geſagt: mir ſcheint, ich habe das Richtige gefunden, und ich wäre außer mir vor
               Schmerz, wenn ich es wieder nicht faſſen könnte.\pend
           \pstart
           Thatſächliches – unter Discretion, würde \textsc{Fritz Kapper\pwindex{Kapper, Friedrich 21.04.1861 – 22.07.1939@\textsc{Kapper, Friedrich} (21.04.1861 – 22.07.1939), \emph{Mediziner}|pw}} ſagen. Das Richtige heißt: {\pb}\textsc{Lisi Pserhofer\pwindex{Pserhofer, Elise 1872-09-18 – 1938-06-28@\textsc{Pserhofer, Elise} (1872-09-18 – 1938-06-28)|pw}}, Tochter\pwindex{Pserhofer, Elise 1872-09-18 – 1938-06-28@\textsc{Pserhofer, Elise} (1872-09-18 – 1938-06-28)|pwv} des bekannten
                  Apothekers\pwindex{Pserhofer, Ignaz 22.04.1835 – 23.11.1910@\textsc{Pserhofer, Ignaz} (22.04.1835 – 23.11.1910), \emph{Apotheker}|pwv}, \label{K_L02649-6v}\edtext{Familie \textsc{Mautner},
                  \textsc{Ernst}}{\lemma{\textnormal{\emph{Familie Mautner,
                  Ernst}}}\Cendnote{\textnormal{Die drei genannten Familien Pserhofer,
                  von Mauthner und Ernst werden durch drei Schwestern\pwindex{Pserhofer, Emma 1844-12-01 – 1937-06-15@\textsc{Pserhofer, Emma} (1844-12-01 – 1937-06-15)|pwkv}\pwindex{Ernst, Betty 1841 – 1895-11-04@\textsc{Ernst, Betty} (1841 – 1895-11-04)|pwkv}\pwindex{Mauthner, Hermine von 1843-10-12 – 1927-09-22@\textsc{Mauthner, Hermine von} (1843-10-12 – 1927-09-22)|pwkv} verbunden, alle geborene
                  Benedikt: Emma\pwindex{Pserhofer, Emma 1844-12-01 – 1937-06-15@\textsc{Pserhofer, Emma} (1844-12-01 – 1937-06-15)|pwk}, die Mutter\pwindex{Pserhofer, Emma 1844-12-01 – 1937-06-15@\textsc{Pserhofer, Emma} (1844-12-01 – 1937-06-15)|pwkv} von Elise Pserhofer\pwindex{Pserhofer, Elise 1872-09-18 – 1938-06-28@\textsc{Pserhofer, Elise} (1872-09-18 – 1938-06-28)|pwk} und Ehefrau\pwindex{Pserhofer, Emma 1844-12-01 – 1937-06-15@\textsc{Pserhofer, Emma} (1844-12-01 – 1937-06-15)|pwkv} von Ignaz
                     Pserhofer\pwindex{Pserhofer, Ignaz 22.04.1835 – 23.11.1910@\textsc{Pserhofer, Ignaz} (22.04.1835 – 23.11.1910), \emph{Apotheker}|pwk}, Betty Ernst\pwindex{Ernst, Betty 1841 – 1895-11-04@\textsc{Ernst, Betty} (1841 – 1895-11-04)|pwk} und Hermine von Mauthner\pwindex{Mauthner, Hermine von 1843-10-12 – 1927-09-22@\textsc{Mauthner, Hermine von} (1843-10-12 – 1927-09-22)|pwk}, die Mutter\pwindex{Mauthner, Hermine von 1843-10-12 – 1927-09-22@\textsc{Mauthner, Hermine von} (1843-10-12 – 1927-09-22)|pwkv} der beiden in Folge genannten Söhne\pwindex{Mauthner, Hans Johann von 1866-12-29 – 1933-08-12@\textsc{Mauthner, Hans Johann von} (1866-12-29 – 1933-08-12)|pwkv}\pwindex{Mauthner, Stephan von 1871-03-19 – 1917-11-21@\textsc{Mauthner, Stephan von} (1871-03-19 – 1917-11-21), \emph{Militär}|pwkv}.}}}\label{K_L02649-6h}{ }\textsc{etc}. Noch iſt es mir nicht gelungen, in den intimen Kreis
               dieſer Leute einzudringen, die ſich hier vollkommen reſervirt verhalten\strikeout{,} und den einzig erſtrebenswerthen Verkehr \strikeout{\textcolor{gray}{di}} repräſentiren. Kennſt du nicht die beiden \textsc{Mautner\pwindex{Mauthner, Hans Johann von 1866-12-29 – 1933-08-12@\textsc{Mauthner, Hans Johann von} (1866-12-29 – 1933-08-12)|pwv}\pwindex{Mauthner, Stephan von 1871-03-19 – 1917-11-21@\textsc{Mauthner, Stephan von} (1871-03-19 – 1917-11-21), \emph{Militär}|pwv}}’s\strikeout{,}{ }\textsc{Hans\pwindex{Mauthner, Hans Johann von 1866-12-29 – 1933-08-12@\textsc{Mauthner, Hans Johann von} (1866-12-29 – 1933-08-12)|pwu}} und \textsc{Stephan\pwindex{Mauthner, Stephan von 1871-03-19 – 1917-11-21@\textsc{Mauthner, Stephan von} (1871-03-19 – 1917-11-21), \emph{Militär}|pw}}? Und kannſt Du mir nicht ein wenig helfen? Den Leuten ein Wort ſchreiben, daß
               ich ein anſtändiger Menſch bin ober ſo was? \textsc{Max Rosenberg\pwindex{Rosenberg, Max von 09.10.1867 – 3.4.1923?@\textsc{Rosenberg, Max von} (09.10.1867 – 3.4.1923?), \emph{Journalist}|pw}} kennt ſie, wie mir ſcheint, ſehr gut; aber der iſt wohl nicht in Wien\oindex{Wien@\textbf{Wien}|pw}. Das ſind nur ſo akademiſche Fragen. Ich ſehne
               mich nach irgend einer Hilfe von Außen, da ich mich ſelbſt ſo unendlich ſchwach
               fühle. Oder kennſt Du das {\pb}Mädel\pwindex{Pserhofer, Elise 1872-09-18 – 1938-06-28@\textsc{Pserhofer, Elise} (1872-09-18 – 1938-06-28)|pwv} ſelber und weißt etwas
               von ihr? Vielleicht etwas Ungünſtiges? Noch wäre es Zeit, ſich die Geſchichte aus dem
               Herzen zu reißen.\pend
           \pstart
           Sonſt \label{K_L02649-7v}\edtext{wimmelt der Ort\oindex{Poertschach@\textbf{Pörtschach}|pwv} wohl von Menſchen}{\lemma{\textnormal{\emph{wimmelt … Menſchen}}}\Cendnote{\textnormal{Beer-Hofmann\pwindex{Beer-Hofmann, Richard 1866-07-11 – 1945-09-26@\textsc{Beer-Hofmann, Richard} (1866-07-11 – 1945-09-26), \emph{Schriftsteller}|pwk} war in diesem Sommer ebenfalls
                  in Pörtschach\oindex{Poertschach@\textbf{Pörtschach}|pwk} und lernte hier Goldmann\pwindex{Goldmann, Paul 31.01.1865 – 25.09.1935@\textsc{Goldmann, Paul} (31.01.1865 – 25.09.1935), \emph{Schriftsteller, Journalist}|pwk} und Leo Van-Jung\pwindex{Van-Jung, Leo 15.10.1866 – 02.07.1939@\textsc{Van-Jung, Leo} (15.10.1866 – 02.07.1939), \emph{Gesangspädagoge, Mathematiker}|pwk} kennen, sodass auch eine Bekanntschaft
                  zwischen den letzteren beiden anzunehmen ist.}}}\label{K_L02649-7h}, aber es iſt Alles das
               gewöhnliche Börſenjuden-Niveau, blöd, frech, unſympathiſch, die Landſchaſt iſt
               großartig, aber Du weißt, wie ſehr ich auf \textcolor{gray}{»}die
                  Landſchaft{[}«{]} pfeife, wenn ich nicht bei ihrem Anblick am Abend
               eine weiche Hand drücken kann und dabei ſagen: »Süßes Mädel!«\pend
           \pstart
           Geleſen: die Kreutzer-Sonate\pwindex{Tolstoi, Leo N. von 9.09.1828 – 20.11.1910@\textsc{Tolstoi, Leo N. von} (9.09.1828 – 20.11.1910), \emph{Schriftsteller}!KreutzersonateNone@\strich\emph{Die Kreutzersonate} {[}None{]}|pw}. Kritiſch
                  großartig\strikeout{\textcolor{gray}{e}}, das Poſitive aber wahnſinnig und pervers. Aber Alles in Allem ein echter \textsc{Tolstoi\pwindex{Tolstoi, Leo N. von 9.09.1828 – 20.11.1910@\textsc{Tolstoi, Leo N. von} (9.09.1828 – 20.11.1910), \emph{Schriftsteller}|pw}} und höchſt leſenswerth. Sonſt nichts. Geſchrieben auch nichts. Von der »Preſſe\orgindex{Presse@Die Presse|pw}« höre ich allerlei Sorgenvolles. \textsc{Granichstaedten\pwindex{Granichstaedten, Emil 1847-07-08 – 1904-07-02@\textsc{Granichstaedten, Emil} (1847-07-08 – 1904-07-02), \emph{Journalist, Wissenschaftler}|pw}} ſoll fortgehen, und man ſucht einen Erſatz, aber nicht mich. Hierbleiben werde
               ich ſo lange als möglich, zumindeſt eine Woche. Könnteſt du nicht auf einen Sprung
                  \label{K_L02649-8v}\edtext{herkommmen}{\lemma{\textnormal{\emph{herkommmen}}}\Cendnote{\textnormal{Schnitzler\pwindex{Schnitzler, Arthur 15.05.1862 – 21.10.1931@\textsc{Schnitzler, Arthur} (15.05.1862 – 21.10.1931), \emph{Schriftsteller, Mediziner}|pwk} kam 1890 nicht nach
                     Pörtschach\oindex{Poertschach@\textbf{Pörtschach}|pwk}.}}}\label{K_L02649-8h}? Jedenfalls \strikeout{ſ\textcolor{gray}{c}h} ſchreib’ mir bald über all’ das
               Wichtige, das ich Dich gefragt. Wieder \textsc{Poste restante}.\pend
           \pstart
           {\pb}Viele herzliche Grüße an Herrn\pwindex{Kapper, Friedrich 21.04.1861 – 22.07.1939@\textsc{Kapper, Friedrich} (21.04.1861 – 22.07.1939), \emph{Mediziner}|pwv} und Frau\pwindex{Kapper, Adele 25.01.1870 – 1941@\textsc{Kapper, Adele} (25.01.1870 – 1941)|pw}{ }\textsc{Fritz}. Ebenſo an Dich!\pend
           \pstart
           Dein {\\[\baselineskip]}\spacefill\mbox{Paul Goldmann.}\pend
           \leftskip=0em{}\pstart
           \noindent{}Empfehlungen an Deine\strikeout{\textcolor{gray}{n}}{ }Schweſter\pwindex{Hajek, Gisela 20.12.1867 – 03.02.1953@\textsc{Hajek, Gisela} (20.12.1867 – 03.02.1953)|pwv} und deinen Schwager\pwindex{Hajek, Markus 25.11.1861 – 04.04.1941@\textsc{Hajek, Markus} (25.11.1861 – 04.04.1941), \emph{Mediziner, Mediziner}|pwv}, die ſich wie
                  befinden?\pend
           \pstart
           Bitte, antworte raſch! Mir ſcheint übrigens, ich hab’ das ſchon oben irgendwo
                  geſagt.\pend
           \pstart
           Unter Discretion: \textsc{Pörtschach\oindex{Poertschach@\textbf{Pörtschach}|pw}} liegt in \textsc{\uline{Kärnthen}\oindex{Kaernten@\textbf{Kärnten}|pw}}.\pend
           
         
         \endnumbering\mylabel{h}\end{ledgroupsized}  \newcommand{\dateiname}{L02649}\newcommand{\titel}{Paul Goldmann an Arthur Schnitzler, 18. 8. 1890}\newcommand{\editorInnen}{Martin Anton Müller und Laura Untner}%% latex-leseansicht-abspann.tex
%% Abspann für die Leseansicht.
%% Der Schalter \ifkorrekturansicht ist bereits durch den Vorspann gesetzt.

%% latex-abspann.tex
%% Gemeinsamer Abspann für Korrekturansicht und Leseansicht.
%% Setzt den Schalter \ifkorrekturansicht voraus (gesetzt in den
%% einbindenden Dateien latex-korrekturansicht-abspann.tex bzw.
%% latex-leseansicht-abspann.tex).
%% ---------------------------------------------------------------

\normalsize

% Das esempio-Environment wird nur in der Leseansicht benötigt
\ifkorrekturansicht\else
\newenvironment{esempio}[3]%
{
    \vspace{1.5ex}
    \rlap{\underline{#1}}
    \par
    \setlength{\parindent}{0cm}
    \nopagebreak
    \leftskip=#2cm
    \rightskip=#3cm
}
{
    \par
}
\fi

\doendnotes{C}
\bigskip
\vfill

\clearpage

\footnotesize

\ifkorrekturansicht
  \lohead{\textsc{register}}
\fi

% theindex-Environment neu definieren ohne reledmac
\makeatletter
\renewenvironment{theindex}{%
  \ifkorrekturansicht
    \section*{\indexname}%
  \else
    \subsubsection*{Index der erwähnten Entitäten}%
  \fi
  \setlength{\parindent}{0pt}%
  \setlength{\parskip}{0pt plus 0.3pt}%
  \let\item\@idxitem
}{%
  \ifkorrekturansicht\clearpage\fi
}
\makeatother

\IfFileExists{\jobname-pw.ind}{\input{\jobname-pw.ind}}{}

% Quellenangabe nur in der Leseansicht
\ifkorrekturansicht\else
% Fallback-Definitionen, falls die .tex-Datei \titel etc. nicht gesetzt hat
\providecommand{\titel}{}
\providecommand{\editorInnen}{}
\providecommand{\dateiname}{\jobname}

\vspace{3cm}

\vfill

\footnotesize
\textsc{Quelle}: \titel. Herausgegeben von {\editorInnen}. In: \emph{Arthur Schnitzler: Briefwechsel mit Autorinnen und Autoren}.
 Digitale Edition, https://schnitzler-briefe.acdh.oeaw.ac.at/{\dateiname}.html (Stand \today)
\fi

\end{document}


      