%% latex-leseansicht-vorspann.tex
%% Vorspann für die Leseansicht.
%% Lädt die gemeinsame Datei latex-vorspann.tex mit nicht gesetztem Schalter.

\newif\ifkorrekturansicht
\korrekturansichtfalse

\input{../tex-inputs/latex-vorspann}

\begin{center}
            \textcolor{red}{ENTWURF, NICHT FERTIG KORRIGIERT}
                      \end{center}
            
         
         \renewcommand{\erwaehntePersonen}{Personen: Hans Rehmann, Anna Katharina Rehmann, Felix Salten, Johann Strauss}
         \renewcommand{\erwaehnteOrte}{Orte: Berlin, Grossmünster, Stadttheater Zürich, Sternwartestraße 71, Wasserkirche, Wien, Zürich}
         \renewcommand{\erwaehnteWerke}{}
               \section[Felix Salten an Arthur Schnitzler, 6. 11. 1929]{ Felix Salten an Arthur Schnitzler, 6. 11. 1929}\nopagebreak\mylabel{v}\rehead{ }\begin{ledgroupsized}[t]{13cm}\normalsize\beginnumbering \toendnotes[C]{\smallbreak\pagebreak[2]} \Standort{CUL, Schnitzler, B 89, B 2.}
\physDesc{Bildpostkarte, 356 Zeichen
\newline{}Handschrift: blaue Tinte, lateinische Kurrent
\newline{}Versand: Stempel: »\nobreak{}\oindex{Zuerich@\textbf{Zürich}|pwk}Zürich 1, 6. IX 929, 21–22, Briefversand\nobreak{}«.  
\newline{}Schnitzler: mit Bleistift datiert: »6/11 92\textcolor{gray}{9}« und zwei Unterstreichungen 
\newline{}Ordnung: mit Bleistift von unbekannter Hand nummeriert:
                                    »301« }\toendnotes[C]{\smallbreak}\pstart{}{\pb}Herrn D\textsuperscript{r} Arthur Schnitzler\pend{}\pstart{}Wien\oindex{Wien@\textbf{Wien}|pw}\pend{}\pstart{}XVIII. Sternwartestrasse 71\oindex{Sternwartestrasse 71@\textbf{Sternwartestraße 71}|pw}\pend{}{\bigskip}\pstart
           \noindent{}\centering{}{\pb}\textcolor{gray}{\textbf{Zürich\oindex{Zuerich@\textbf{Zürich}|pw}. Großmünster\oindex{Grossmuenster@\textbf{Grossmünster}|pw} und Wasserkirche\oindex{Wasserkirche@\textbf{Wasserkirche}|pw}}}\pend
           \pstart{}{\pb}Lieber,\pend\pstart
           Berlin\oindex{Berlin@\textbf{Berlin}|pw} war diesmal sehr angenehm. Denn \label{K_L03588-1v}\edtext{Hans Rehmann\pwindex{Rehmann, Hans 20.03.1900 – 30.08.1939@\textsc{Rehmann, Hans} (20.03.1900 – 30.08.1939), \emph{Schauspieler}|pw} gefiel mir ungemein}{\lemma{\textnormal{\emph{Hans … ungemein}}}\Cendnote{\textnormal{der zukunftige Ehemann der Tochter Anna Katharina Salten\pwindex{Rehmann, Anna Katharina 18.08.1904 – 27.03.1977@\textsc{Rehmann, Anna Katharina} (18.08.1904 – 27.03.1977), \emph{Schauspielerin, Übersetzerin}|pwk}}}}\label{K_L03588-1h} und wir verstanden einander bald. Ich glaube, er ist ein wirklicher Mensch
               und bin natürlich froh! Hier muss ich bis Sonntag bleiben, um die \label{K_L03588-2v}\edtext{Johann-Strauss\pwindex{Strauss, Johann 25.10.1825 – 03.06.1899@\textsc{Strauss, Johann} (25.10.1825 – 03.06.1899), \emph{Komponist, Dirigent}|pw}-Rede am Samstag zu wiederholen}{\lemma{\textnormal{\emph{Johann-Strauss-Rede … wiederholen}}}\Cendnote{\textnormal{Am 4. 11. 1929 hatte
                     Salten\pwindex{Salten, Felix 06.09.1869 – 08.10.1945@\textsc{Salten, Felix} (06.09.1869 – 08.10.1945), \emph{Schriftsteller, Journalist}|pwk} im Stadttheater\oindex{Stadttheater Zuerich@\textbf{Stadttheater Zürich}|pwk} eine Gedenkrede für Johann-Strauss\pwindex{Strauss, Johann 25.10.1825 – 03.06.1899@\textsc{Strauss, Johann} (25.10.1825 – 03.06.1899), \emph{Komponist, Dirigent}|pwk} gehalten. Am 9. 9. 1929 wurde die
                  Veranstaltung wiederholt.}}}\label{K_L03588-2h}.\pend
           \pstart
           Herzlichst {\\[\baselineskip]}Ihr {\\[\baselineskip]}\spacefill\mbox{Felix Salten}\pend
           \leftskip=0em{}\pstart
           Zürich\oindex{Zuerich@\textbf{Zürich}|pw}{ }\textcolor{gray}{6}. XI. 29\pend
           
         
         \endnumbering\mylabel{h}\end{ledgroupsized}\begin{anhang}\end{anhang}\newcommand{\dateiname}{L03588}\newcommand{\titel}{Felix Salten an Arthur Schnitzler, 6. 11. 1929}\newcommand{\editorInnen}{Martin Anton Müller und Laura Untner}%% latex-leseansicht-abspann.tex
%% Abspann für die Leseansicht.
%% Der Schalter \ifkorrekturansicht ist bereits durch den Vorspann gesetzt.

%% latex-abspann.tex
%% Gemeinsamer Abspann für Korrekturansicht und Leseansicht.
%% Setzt den Schalter \ifkorrekturansicht voraus (gesetzt in den
%% einbindenden Dateien latex-korrekturansicht-abspann.tex bzw.
%% latex-leseansicht-abspann.tex).
%% ---------------------------------------------------------------

\normalsize

% Das esempio-Environment wird nur in der Leseansicht benötigt
\ifkorrekturansicht\else
\newenvironment{esempio}[3]%
{
    \vspace{1.5ex}
    \rlap{\underline{#1}}
    \par
    \setlength{\parindent}{0cm}
    \nopagebreak
    \leftskip=#2cm
    \rightskip=#3cm
}
{
    \par
}
\fi

\doendnotes{C}
\bigskip
\vfill

\clearpage

\footnotesize

\ifkorrekturansicht
  \lohead{\textsc{register}}
\fi

% theindex-Environment neu definieren ohne reledmac
\makeatletter
\renewenvironment{theindex}{%
  \ifkorrekturansicht
    \section*{\indexname}%
  \else
    \subsubsection*{Index der erwähnten Entitäten}%
  \fi
  \setlength{\parindent}{0pt}%
  \setlength{\parskip}{0pt plus 0.3pt}%
  \let\item\@idxitem
}{%
  \ifkorrekturansicht\clearpage\fi
}
\makeatother

\IfFileExists{\jobname-pw.ind}{\input{\jobname-pw.ind}}{}

% Quellenangabe nur in der Leseansicht
\ifkorrekturansicht\else
% Fallback-Definitionen, falls die .tex-Datei \titel etc. nicht gesetzt hat
\providecommand{\titel}{}
\providecommand{\editorInnen}{}
\providecommand{\dateiname}{\jobname}

\vspace{3cm}

\vfill

\footnotesize
\textsc{Quelle}: \titel. Herausgegeben von {\editorInnen}. In: \emph{Arthur Schnitzler: Briefwechsel mit Autorinnen und Autoren}.
 Digitale Edition, https://schnitzler-briefe.acdh.oeaw.ac.at/{\dateiname}.html (Stand \today)
\fi

\end{document}


      