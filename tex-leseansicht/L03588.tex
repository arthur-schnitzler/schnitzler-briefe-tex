%% latex-korrekturansicht-vorspann.tex
%% Vorspann für die Korrekturansicht.
%% Lädt die gemeinsame Datei latex-vorspann.tex mit gesetztem Schalter.

\newif\ifkorrekturansicht
\korrekturansichttrue

\input{../tex-inputs/latex-vorspann}


\section[ Felix Salten an Arthur Schnitzler, 6. 11. 1929]{L03588 Felix Salten an Arthur Schnitzler, 6. 11. 1929}
\nopagebreak\mylabel{L03588v}
\rehead{ }\normalsize\beginnumbering\briefempfaengerindex{Schnitzler, Arthur@\textsc{Schnitzler, Arthur}!zzzSalten, Felix@\emph{von Felix Salten}!1929-11-061@{6. 11. 1929}|(be}
\toendnotes[C]{\smallbreak\pagebreak[2]}\Standort{CUL, Schnitzler, B 89, B 2.}
\physDesc{Bildpostkarte, 354 Zeichen
\newline{}Handschrift: schwarze Tinte, lateinische Kurrent
\newline{}Versand: Stempel: »\nobreak{}\oindex{Zuerich@\textbf{Zürich}, \emph{P.PPLA}|pwk}Zürich 1, 6 · IX 929, 21–22, Briefversand\nobreak{}«.  
\newline{}Schnitzler: mit Bleistift datiert: »6/11 92\textcolor{gray}{9}« und zwei Unterstreichungen 
\newline{}Ordnung: mit Bleistift von unbekannter Hand nummeriert: »301« }\toendnotes[C]{\smallbreak}\pstart{}{\pb}Herrn D\textsuperscript{r} Arthur Schnitzler\pend{}\pstart{}Wien\oindex{Wien@\textbf{Wien}, \emph{A.ADM2}|pw}\pend{}\pstart{}XVIII. Sternwartestrasse 71\oindex{Sternwartestrasse 71@\textbf{Sternwartestraße 71}, \emph{Wohngebäude (K.WHS)}|pw}\pend{}{\bigskip}
\pstart
           \noindent{}\centering{}{\pb}\textcolor{gray}{\textbf{Zürich\oindex{Zuerich@\textbf{Zürich}, \emph{P.PPLA}|pw}. Großmünster\oindex{Grossmuenster@\textbf{Grossmünster}, \emph{S.CH}|pw} und Wasserkirche\oindex{Wasserkirche@\textbf{Wasserkirche}, \emph{Kirche (K.KRC)}|pw}}}\pend
           \vspace{1em}
\pstart{}{\pb}Lieber,\pend\vspace{0.5em}
\pstart
           Berlin\oindex{Berlin@\textbf{Berlin}, \emph{P.PPLC}|pw} war diesmal sehr angenehm. Denn \label{K_L03588-1v}\edtext{Hans Rehmann\pwindex{Rehmann, Hans 20.03.1900 – 30.08.1939@\textsc{Rehmann, Hans} (20.03.1900 – 30.08.1939), \emph{Schauspieler/Schauspielerin}|pw} gefiel mir ungemein}{\lemma{\textnormal{\emph{Hans … ungemein}}}\Cendnote{\textnormal{Hans Rehmann\pwindex{Rehmann, Hans 20.03.1900 – 30.08.1939@\textsc{Rehmann, Hans} (20.03.1900 – 30.08.1939), \emph{Schauspieler/Schauspielerin}|pwk} war der zukünftige Ehemann der
                  Tochter Anna Katharina Salten\pwindex{Rehmann, Anna Katharina 18.08.1904 – 27.03.1977@\textsc{Rehmann, Anna Katharina} (18.08.1904 – 27.03.1977), \emph{Schauspieler/Schauspielerin, Übersetzer/Übersetzerin}|pwk}.}}}\label{K_L03588-1} und
               wir verstanden einander bald. Ich glaube, er ist ein wirklicher Mensch und bin
               natürlich froh! Hier muss ich bis Sonntag bleiben, um
               die \label{K_L03588-2v}\edtext{Johann-Strauss\pwindex{Strauss, Johann 25.10.1825 – 03.06.1899@\textsc{Strauss, Johann} (25.10.1825 – 03.06.1899), \emph{Komponist/Komponistin, Dirigent/Dirigentin}|pw}-Rede am Samstag zu wiederholen}{\lemma{\textnormal{\emph{Johann-Strauss-Rede … wiederholen}}}\Cendnote{\textnormal{Am 4. 11. 1929 hatte Salten\pwindex{Salten, Felix 06.09.1869 – 08.10.1945@\textsc{Salten, Felix} (06.09.1869 – 08.10.1945), \emph{Schriftsteller/Schriftstellerin, Journalist/Journalistin, Chefredakteur/Chefredakteurin}|pwk} im Stadttheater\oindex{Stadttheater [Zuerich]@\textbf{Stadttheater [Zürich]}, \emph{Theater (K.THE)}|pwk}
                  eine Gedenkrede für Johann Strauss\pwindex{Strauss, Johann 25.10.1825 – 03.06.1899@\textsc{Strauss, Johann} (25.10.1825 – 03.06.1899), \emph{Komponist/Komponistin, Dirigent/Dirigentin}|pwk}
                  gehalten; am 9. 11. 1929 wurde die Veranstaltung
                  wiederholt.}}}\label{K_L03588-2}.\pend
           
\pstart
           Herzlichst {\\[\baselineskip]}Ihr {\\[\baselineskip]}\spacefill\mbox{Felix Salten}\pend
           \leftskip=0em{}
\pstart
           Zürich\oindex{Zuerich@\textbf{Zürich}, \emph{P.PPLA}|pw}{ }6. XI. 29\pend
           \selectlanguage{ngerman}\endnumbering\briefempfaengerindex{Schnitzler, Arthur@\textsc{Schnitzler, Arthur}!zzzSalten, Felix@\emph{von Felix Salten}!1929-11-061@{6. 11. 1929}|)be}\mylabel{L03588h}  \normalsize

\doendnotes{C}
\bigskip
\vfill

\clearpage

\footnotesize

\lohead{\textsc{register}}

% Definiere theindex-Environment komplett neu ohne reledmac
\makeatletter
\renewenvironment{theindex}{%
  \section*{\indexname}%
  \setlength{\parindent}{0pt}%
  \setlength{\parskip}{0pt plus 0.3pt}%
  \let\item\@idxitem
}{%
  \clearpage
}
\makeatother

\IfFileExists{\jobname-pw.ind}{\input{\jobname-pw.ind}}{}

\end{document}

      