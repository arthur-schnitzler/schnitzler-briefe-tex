%% latex-leseansicht-vorspann.tex
%% Vorspann für die Leseansicht.
%% Lädt die gemeinsame Datei latex-vorspann.tex mit nicht gesetztem Schalter.

\newif\ifkorrekturansicht
\korrekturansichtfalse

\input{../tex-inputs/latex-vorspann}


\section[ Felix Salten an Arthur Schnitzler, 6. 11. 1929]{L03588 Felix Salten an Arthur Schnitzler,  6. 11. 1929}
\nopagebreak\mylabel{L03588v}
\rehead{ }\normalsize\beginnumbering\briefempfaengerindex{Schnitzler, Arthur@\textsc{Schnitzler, Arthur}!zzzSalten, Felix@\emph{von Felix Salten}!1929-11-061@{6. 11. 1929}|(be}
\toendnotes[C]{\smallbreak\pagebreak[2]}
\correspDesc{Versand  durch Felix Salten am 6. 11. 1929 in Zürich
\newline{}Erhalt  durch Arthur Schnitzler im Zeitraum [7. 11. 1929
                  – 11. 11. 1929?] in Wien}\toendnotes[C]{\smallbreak}
\Standort{CUL, Schnitzler, B 89, B 2.}
\physDesc{Bildpostkarte, 354 Zeichen
\newline{}Handschrift: schwarze Tinte, lateinische Kurrent
\newline{}Versand: Stempel: »\nobreak{}\oindex{Zürich@\textbf{Zürich}|pwk}Zürich 1, 6 · IX 929, 21–22, Briefversand\nobreak{}«.  
\newline{}Schnitzler: mit Bleistift datiert: »6/11 92\textcolor{gray}{9}« und zwei Unterstreichungen 
\newline{}Ordnung: mit Bleistift von unbekannter Hand nummeriert: »301« }\toendnotes[C]{\smallbreak}\pstart{}{\pb}Herrn D\textsuperscript{r} Arthur Schnitzler\pend{}\pstart{}Wien\oindex{Wien@\textbf{Wien}, \emph{Verwaltungsgebiet}|pw}\pend{}\pstart{}XVIII. Sternwartestrasse 71\oindex{Wien@\textbf{Wien}!XVIII., Währing@\textbf{XVIII., Währing}!Sternwartestraße 71@\textbf{Sternwartestraße 71}, \emph{Wohngebäude}|pw}\pend{}{\bigskip}
\pstart
           \noindent{}\centering{}{\pb}\textcolor{gray}{\textbf{Zürich\oindex{Zürich@\textbf{Zürich}|pw}. Großmünster\oindex{Grossmünster@\textbf{Grossmünster}, \emph{Kirche}|pw} und Wasserkirche\oindex{Wasserkirche@\textbf{Wasserkirche}, \emph{Kirche}|pw}}}\pend
           \vspace{1em}
\pstart{}{\pb}Lieber,\pend\vspace{0.5em}
\pstart
           Berlin\oindex{Berlin@\textbf{Berlin}, \emph{Hauptstadt}|pw} war diesmal sehr angenehm. Denn \label{K_L03588-1v}\edtext{Hans Rehmann\pwindex{Rehmann, Hans 20.\,3.\,1900 Zürich – 30.\,8.\,1939 Langenthal@\textsc{Rehmann, Hans} (20.\,3.\,1900 Zürich – 30.\,8.\,1939 Langenthal), \emph{Schauspieler}|pw} gefiel mir ungemein}{\lemma{\textnormal{\emph{Hans … ungemein}}}\Cendnote{\textnormal{Hans Rehmann\pwindex{Rehmann, Hans 20.\,3.\,1900 Zürich – 30.\,8.\,1939 Langenthal@\textsc{Rehmann, Hans} (20.\,3.\,1900 Zürich – 30.\,8.\,1939 Langenthal), \emph{Schauspieler}|pwk} war der zukünftige Ehemann der
                  Tochter Anna Katharina Salten\pwindex{Rehmann, Anna Katharina 18.\,8.\,1904 Wien – 27.\,3.\,1977 Zürich@\textsc{Rehmann, Anna Katharina} (18.\,8.\,1904 Wien – 27.\,3.\,1977 Zürich), \emph{Schauspielerin, Übersetzerin}|pwk}.}}}\label{K_L03588-1} und
               wir verstanden einander bald. Ich glaube, er ist ein wirklicher Mensch und bin
               natürlich froh! Hier muss ich bis Sonntag bleiben, um
               die \label{K_L03588-2v}\edtext{Johann-Strauss\pwindex{Strauss, Johann 25.\,10.\,1825 Wien – 3.\,6.\,1899 ebd.@\textsc{Strauss, Johann} (25.\,10.\,1825 Wien – 3.\,6.\,1899 ebd.), \emph{Komponist, Dirigent}|pw}-Rede am Samstag zu wiederholen}{\lemma{\textnormal{\emph{Johann-Strauss-Rede … wiederholen}}}\Cendnote{\textnormal{Am 4. 11. 1929 hatte Salten\pwindex{Salten, Felix 6.\,9.\,1869 Budapest – 8.\,10.\,1945 Zürich@\textsc{Salten, Felix} (6.\,9.\,1869 Budapest – 8.\,10.\,1945 Zürich), \emph{Schriftsteller, Journalist, Chefredakteur}|pwk} im Stadttheater\oindex{Stadttheater [Zürich]@\textbf{Stadttheater [Zürich]}, \emph{Theater}|pwk}
                  eine Gedenkrede für Johann Strauss\pwindex{Strauss, Johann 25.\,10.\,1825 Wien – 3.\,6.\,1899 ebd.@\textsc{Strauss, Johann} (25.\,10.\,1825 Wien – 3.\,6.\,1899 ebd.), \emph{Komponist, Dirigent}|pwk}
                  gehalten; am 9. 11. 1929 wurde die Veranstaltung
                  wiederholt.}}}\label{K_L03588-2}.\pend
           
\pstart
           Herzlichst {\\[\baselineskip]}Ihr {\\[\baselineskip]}\spacefill\mbox{Felix Salten}\pend
           \leftskip=0em{}
\pstart
           Zürich\oindex{Zürich@\textbf{Zürich}|pw}{ }6. XI. 29\pend
           \selectlanguage{ngerman}\endnumbering\briefempfaengerindex{Schnitzler, Arthur@\textsc{Schnitzler, Arthur}!zzzSalten, Felix@\emph{von Felix Salten}!1929-11-061@{6. 11. 1929}|)be}\mylabel{L03588h}  \newcommand{\dateiname}{L03588}\newcommand{\titel}{Felix Salten an Arthur Schnitzler, 6. 11. 1929}\newcommand{\editorInnen}{Martin Anton Müller und Laura Untner}%% latex-leseansicht-abspann.tex
%% Abspann für die Leseansicht.
%% Der Schalter \ifkorrekturansicht ist bereits durch den Vorspann gesetzt.

%% latex-abspann.tex
%% Gemeinsamer Abspann für Korrekturansicht und Leseansicht.
%% Setzt den Schalter \ifkorrekturansicht voraus (gesetzt in den
%% einbindenden Dateien latex-korrekturansicht-abspann.tex bzw.
%% latex-leseansicht-abspann.tex).
%% ---------------------------------------------------------------

\normalsize

% Das esempio-Environment wird nur in der Leseansicht benötigt
\ifkorrekturansicht\else
\newenvironment{esempio}[3]%
{
    \vspace{1.5ex}
    \rlap{\underline{#1}}
    \par
    \setlength{\parindent}{0cm}
    \nopagebreak
    \leftskip=#2cm
    \rightskip=#3cm
}
{
    \par
}
\fi

\doendnotes{C}
\bigskip
\vfill

\clearpage

\footnotesize

\ifkorrekturansicht
  \lohead{\textsc{register}}
\fi

% theindex-Environment neu definieren ohne reledmac
\makeatletter
\renewenvironment{theindex}{%
  \ifkorrekturansicht
    \section*{\indexname}%
  \else
    \subsubsection*{Index der erwähnten Entitäten}%
  \fi
  \setlength{\parindent}{0pt}%
  \setlength{\parskip}{0pt plus 0.3pt}%
  \let\item\@idxitem
}{%
  \ifkorrekturansicht\clearpage\fi
}
\makeatother

\IfFileExists{\jobname-pw.ind}{\input{\jobname-pw.ind}}{}

% Quellenangabe nur in der Leseansicht
\ifkorrekturansicht\else
% Fallback-Definitionen, falls die .tex-Datei \titel etc. nicht gesetzt hat
\providecommand{\titel}{}
\providecommand{\editorInnen}{}
\providecommand{\dateiname}{\jobname}

\vspace{3cm}

\vfill

\footnotesize
\textsc{Quelle}: \titel. Herausgegeben von {\editorInnen}. In: \emph{Arthur Schnitzler: Briefwechsel mit Autorinnen und Autoren}.
 Digitale Edition, https://schnitzler-briefe.acdh.oeaw.ac.at/{\dateiname}.html (Stand \today)
\fi

\end{document}


