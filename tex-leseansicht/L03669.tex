%% latex-leseansicht-vorspann.tex
%% Vorspann für die Leseansicht.
%% Lädt die gemeinsame Datei latex-vorspann.tex mit nicht gesetztem Schalter.

\newif\ifkorrekturansicht
\korrekturansichtfalse

\input{../tex-inputs/latex-vorspann}


\section[Stefan Zweig an Arthur Schnitzler, 4. 11. 1924]{L03669 Stefan Zweig an Arthur Schnitzler, 4. 11. 1924}
\nopagebreak\mylabel{L03669v}
\rehead{ }\normalsize\beginnumbering\briefempfaengerindex{Schnitzler, Arthur@\textsc{Schnitzler, Arthur}!zzzZweig, Stefan@\emph{von Stefan Zweig}!1924-11-041@{4. 11. 1924}|(be}
\toendnotes[C]{\smallbreak\pagebreak[2]}
\correspDesc{Versand  durch Stefan Zweig am 4. 11. 1924 in Salzburg
\newline{}Erhalt  durch Arthur Schnitzler im Zeitraum [5. 11. 1924
                  – 9. 11. 1924?] in Wien}\toendnotes[C]{\smallbreak}
\Standort{CUL, Schnitzler, B 118.}
\physDesc{Brief, 1 Blatt, 2 Seiten, 1370 Zeichen
\newline{}Handschrift: blaue Tinte, lateinische Kurrent
\newline{}Schnitzler: 1) mit Bleistift »\textsc{Zweig}«  2) mit rotem Buntstift drei Unterstreichungen}
\buchAbdrucke{\weitereDrucke{Stefan Zweig: \emph{Briefwechsel mit Hermann Bahr, Sigmund Freud, Rainer Maria
                        Rilke und Arthur Schnitzler}. Herausgegeben von Jeffrey B. Berlin, Hans-Ulrich Lindken und Donald A. Prater. Frankfurt am Main: \emph{S. Fischer} 1987, S. 419.} }\toendnotes[C]{\smallbreak}
\pstart
           {\pb}\textcolor{gray}{\textbf{SZ}}\hfill \textcolor{gray}{\textbf{KAPUZINERBERG 5\oindex{Paschinger Schlössl@\textbf{Paschinger Schlössl}, \emph{Wohngebäude}|pw}}}\pend
           
\pstart
           \raggedleft{}\textcolor{gray}{\textbf{SALZBURG\oindex{Salzburg@\textbf{Salzburg}, \emph{Verwaltungsgebiet}|pw}}}{ }4. Nov. 1924\pend
           \vspace{0.5em}
\pstart
           Lieber verehrter Herr Doktor, ich bin schwer in Arbeit – aber ich
               muss mich für eine Minute unterbrechen, um Ihnen zu sagen, \uline{wie ausserordentlich} ich ihre \label{K_L03669-1v}\edtext{Novelle\pwindex{Schnitzler, Arthur 15.\,5.\,1862 Wien – 21.\,10.\,1931 ebd.@\textsc{Schnitzler, Arthur} (15.\,5.\,1862 Wien – 21.\,10.\,1931 ebd.), \emph{Schriftsteller, Mediziner}!Fräulein Else@\strich\emph{Fräulein Else}|pwv}}{\lemma{\textnormal{\emph{Novelle}}}\Cendnote{\textnormal{\emph{Fräulein Else}\pwindex{Schnitzler, Arthur 15.\,5.\,1862 Wien – 21.\,10.\,1931 ebd.@\textsc{Schnitzler, Arthur} (15.\,5.\,1862 Wien – 21.\,10.\,1931 ebd.), \emph{Schriftsteller, Mediziner}!Fräulein Else@\strich\emph{Fräulein Else}|pwk}. Novelle von Arthur
                        Schnitzler. In: \emph{Die neue
                     Rundschau}\pwindex{neue Rundschau@\emph{Die neue Rundschau}|pwk}, Jg. 35, H. 10, Oktober 1924,
                     S. 993–1051.}}}\label{K_L03669-1} in der »Neuen
                  Rundschau\pwindex{neue Rundschau@\emph{Die neue Rundschau}|pw}« finde: eine \label{K_L03669-2v}\edtext{trouvaille}{\lemma{\textnormal{\emph{trouvaille}}}\Cendnote{\textnormal{französisch: wertvolles Fundstück, Schmuckstück}}}\label{K_L03669-2} in der Technik der Novelle, spannend,
               aufwühlend, ganz ins Tragische aus kleinen Präludium aufsteigend. Ich wüsste kein
               Wort darin zu ändern – einzig für die Buchausgabe eine \uline{Zahl}. 50 000 Gulden 100.000 Friedenskronen – das war eine Summe, die ein
               Rotschild kaum seinem Brüder \label{K_L03669-3v}\edtext{a fond
                  perdu}{\lemma{\textnormal{\emph{a fond
                  perdu}}}\Cendnote{\textnormal{französisch: ohne
                  Rückzahlungsverpflichtung}}}\label{K_L03669-3} lieh. Gieng es Ihnen nicht da wie Jacob Wassermann\pwindex{Wassermann, Jakob 10.\,3.\,1873 Fürth – 1.\,1.\,1934 Altaussee@\textsc{Wassermann, Jakob} (10.\,3.\,1873 Fürth – 1.\,1.\,1934 Altaussee), \emph{Schriftsteller}|pw} in der Ulrike Woytech\pwindex{Wassermann, Jakob 10.\,3.\,1873 Fürth – 1.\,1.\,1934 Altaussee@\textsc{Wassermann, Jakob} (10.\,3.\,1873 Fürth – 1.\,1.\,1934 Altaussee), \emph{Schriftsteller}!Ulrike Woytich. Roman@\strich\emph{Ulrike Woytich. Roman}|pw}, dass unsere Erinnerungsgefühle an Geld auch
               schon inflationiert sind? Gerade weil es ein entfernter Bekannter aus dem \uline{Mittelstand} ist, schien mir die Summe grotest hoch –
               ich verstehe, dass {\pb}Sie für die seelische
               Motivation eine \uline{hohe} Summe brauchten – uns klingt
               10 000 Kronen heute wie ein »Fetzen« war aber doch schon als Leihgeld unerhört viel.
               Ich kam auf diesen Kleinkram zu reden, weil ich selbst bei einer (unveröffentlichten)
                  \label{K_L03669-4v}\edtext{Arbeit\pwindex{Zweig, Stefan 28.\,11.\,1881 Wien – 23.\,2.\,1942 Petrópolis@\textsc{Zweig, Stefan} (28.\,11.\,1881 Wien – 23.\,2.\,1942 Petrópolis), \emph{Schriftsteller}!?? [Novelle, in der ein kleiner Geldbetrag ein Schicksal entscheidet]@\strich\emph{?? [Novelle, in der ein kleiner Geldbetrag ein Schicksal entscheidet]}|pwv}}{\lemma{\textnormal{\emph{Arbeit}}}\Cendnote{\textnormal{nicht
                  identifiziert}}}\label{K_L03669-4} den Widersinn spürte, zehn Kronen zu einer Entscheidung über
               ein Schicksal zu machen: aber es gab damals Katastrofen wegen fünfzig Heller. Wo ist
               die Zeit!\pend
           
\pstart
           Wie habe ich mich gefreut an Ihrem Werk\pwindex{Schnitzler, Arthur 15.\,5.\,1862 Wien – 21.\,10.\,1931 ebd.@\textsc{Schnitzler, Arthur} (15.\,5.\,1862 Wien – 21.\,10.\,1931 ebd.), \emph{Schriftsteller, Mediziner}!Fräulein Else@\strich\emph{Fräulein Else}|pwv}, wie an der Überraschung, die mir trotz aller alten
               Liebe, alles guten Vertrauens, dieser Aufstieg war!\pend
           
\pstart
           Seien Sie innigst beglückwünscht von Ihrem ergebenen{\\[\baselineskip]}\spacefill\mbox{Stefan Zweig}\pend
           \leftskip=0em{}\selectlanguage{ngerman}\endnumbering\briefempfaengerindex{Schnitzler, Arthur@\textsc{Schnitzler, Arthur}!zzzZweig, Stefan@\emph{von Stefan Zweig}!1924-11-041@{4. 11. 1924}|)be}\mylabel{L03669h}  \newcommand{\dateiname}{L03669}\newcommand{\titel}{Stefan Zweig an Arthur Schnitzler, 4. 11. 1924}\newcommand{\editorInnen}{Selma Jahnke und Martin Anton Müller}%% latex-leseansicht-abspann.tex
%% Abspann für die Leseansicht.
%% Der Schalter \ifkorrekturansicht ist bereits durch den Vorspann gesetzt.

%% latex-abspann.tex
%% Gemeinsamer Abspann für Korrekturansicht und Leseansicht.
%% Setzt den Schalter \ifkorrekturansicht voraus (gesetzt in den
%% einbindenden Dateien latex-korrekturansicht-abspann.tex bzw.
%% latex-leseansicht-abspann.tex).
%% ---------------------------------------------------------------

\normalsize

% Das esempio-Environment wird nur in der Leseansicht benötigt
\ifkorrekturansicht\else
\newenvironment{esempio}[3]%
{
    \vspace{1.5ex}
    \rlap{\underline{#1}}
    \par
    \setlength{\parindent}{0cm}
    \nopagebreak
    \leftskip=#2cm
    \rightskip=#3cm
}
{
    \par
}
\fi

\doendnotes{C}
\bigskip
\vfill

\clearpage

\footnotesize

\ifkorrekturansicht
  \lohead{\textsc{register}}
\fi

% theindex-Environment neu definieren ohne reledmac
\makeatletter
\renewenvironment{theindex}{%
  \ifkorrekturansicht
    \section*{\indexname}%
  \else
    \subsubsection*{Index der erwähnten Entitäten}%
  \fi
  \setlength{\parindent}{0pt}%
  \setlength{\parskip}{0pt plus 0.3pt}%
  \let\item\@idxitem
}{%
  \ifkorrekturansicht\clearpage\fi
}
\makeatother

\IfFileExists{\jobname-pw.ind}{\input{\jobname-pw.ind}}{}

% Quellenangabe nur in der Leseansicht
\ifkorrekturansicht\else
% Fallback-Definitionen, falls die .tex-Datei \titel etc. nicht gesetzt hat
\providecommand{\titel}{}
\providecommand{\editorInnen}{}
\providecommand{\dateiname}{\jobname}

\vspace{3cm}

\vfill

\footnotesize
\textsc{Quelle}: \titel. Herausgegeben von {\editorInnen}. In: \emph{Arthur Schnitzler: Briefwechsel mit Autorinnen und Autoren}.
 Digitale Edition, https://schnitzler-briefe.acdh.oeaw.ac.at/{\dateiname}.html (Stand \today)
\fi

\end{document}


