%% latex-leseansicht-vorspann.tex
%% Vorspann für die Leseansicht.
%% Lädt die gemeinsame Datei latex-vorspann.tex mit nicht gesetztem Schalter.

\newif\ifkorrekturansicht
\korrekturansichtfalse

\input{../tex-inputs/latex-vorspann}

\begin{center}
            \textcolor{red}{ENTWURF, NICHT FERTIG KORRIGIERT}
                      \end{center}
            
         
         \renewcommand{\erwaehntePersonen}{Personen: Shelomoh ben Mosheh Alḳabets}
         \renewcommand{\erwaehnteOrte}{Orte: Dr. Ludwig Leber-Straße, Edmund-Weiß-Gasse, Mariazell, Wien, XVIII., Währing}
         \renewcommand{\erwaehnteWerke}{Werke: Lecha Dodi, Spätgotische Marienstatue mit Strahlenkranz, Tagebuch}
               \section[Felix Salten und Richard Metzl an Arthur Schnitzler, {[}30. 7. 1905?{]}]{ Felix Salten und Richard Metzl an Arthur Schnitzler,
               {[}30. 7. 1905?{]}}\nopagebreak\mylabel{v}\rehead{ }\begin{ledgroupsized}[t]{13cm}\normalsize\beginnumbering \toendnotes[C]{\smallbreak\pagebreak[2]} \Standort{CUL, Schnitzler, B 89, B 1.}
\physDesc{Bildpostkarte
\newline{}Handschrift Felix Salten: Bleistift, lateinische Kurrent\newline{}Handschrift Richard Metzl: Bleistift, lateinische Kurrent\newline{}Versand: Stempel: »\nobreak{}\oindex{Mariazell@\textbf{Mariazell}|pwk}\textcolor{gray}{Mariazell}, \textcolor{gray}{3}0 7 \textcolor{gray}{05}\nobreak{}«.  \newline{}Ordnung: mit Bleistift von unbekannter Hand nummeriert:
                                    »202« }\toendnotes[C]{\smallbreak}\pstart{}{\pb}Herrn D\textsuperscript{r} Arthur Schnitzler\pend{}\pstart{}Wien XVIII.\oindex{XVIII., Waehring@\textbf{XVIII., Währing}|pw}\pend{}\pstart{}Spöttelgaſse 7\oindex{Edmund-Weiss-Gasse@\textbf{Edmund-Weiß-Gasse}|pw}\pend{}{\bigskip}\pstart
           \noindent{}\centering{}{\pb}\textcolor{gray}{\textbf{\textsc{Gruss aus \label{K_L03410-11v}\edtext{Mariazell\oindex{Mariazell@\textbf{Mariazell}|pw}}{\lemma{\textnormal{\emph{Mariazell}}}\Cendnote{\textnormal{Die am 18. 7. 1905 erwähnte »Maria Zell\oindex{Mariazell@\textbf{Mariazell}|pw}er Partie« dürfte
                           sich bis Monatsende verschoben haben, vgl. Felix Salten an Arthur Schnitzler, 18. 7. 1905. Am 31. 7. 1931 war Salten\pwindex{Salten, Felix 06.09.1869 – 08.10.1945@\textsc{Salten, Felix} (06.09.1869 – 08.10.1945), \emph{Schriftsteller, Journalist}|pwk} wieder in Wien\oindex{Wien@\textbf{Wien}|pwk}, »aus Mariazell\oindex{Mariazell@\textbf{Mariazell}|pw},
                              angeekelt«, wie Schnitzler\pwindex{Schnitzler, Arthur 15.05.1862 – 21.10.1931@\textsc{Schnitzler, Arthur} (15.05.1862 – 21.10.1931), \emph{Schriftsteller, Mediziner}|pwk} im \emph{Tagebuch}\pwindex{Schnitzler, Arthur 15.05.1862 – 21.10.1931@\textsc{Schnitzler, Arthur} (15.05.1862 – 21.10.1931), \emph{Schriftsteller, Mediziner}!Tagebuch1981 – 2000@\strich\emph{Tagebuch} {[}1981 – 2000{]}|pwk}
                           festhält.}}}\label{K_L03410-11h}}}}\pend
           \pstart
           \noindent{}\centering{}\textcolor{gray}{\textbf{MARIENSTATUE\pwindex{?? Werk@Nicht ermittelte Verfasserinnen und Verfasser!Spaetgotische Marienstatue mit StrahlenkranzNone@\emph{Spätgotische Marienstatue mit Strahlenkranz} {[}None{]}|pw}}}\pend
           \pstart
           \noindent{}\centering{}\textcolor{gray}{\textbf{Wienergasse\oindex{Dr. Ludwig Leber-Strasse@\textbf{Dr. Ludwig Leber-Straße}|pw}}}\pend
           \pstart
           Das \label{K_L03410-1v}\edtext{Lechodaudi\pwindex{Alḳabets, Shelomoh ben Mosheh 1505 – 1576@\textsc{Alḳabets, Shelomoh ben Mosheh} (1505 – 1576), \emph{Mystiker, Dichter}!Lecha DodiNone@\strich\emph{Lecha Dodi} {[}None{]}|pw}}{\lemma{\textnormal{\emph{Lechodaudi}}}\Cendnote{\textnormal{Lecha Dodi (L’kha Dodi) – sind die
                  ersten beiden Worte einer Hymne von Shelomoh
                     ben Mosheh Alḳabets\pwindex{Alḳabets, Shelomoh ben Mosheh 1505 – 1576@\textsc{Alḳabets, Shelomoh ben Mosheh} (1505 – 1576), \emph{Mystiker, Dichter}|pwk}, mit der der Sabbat eingeläutet wird.}}}\label{K_L03410-1h} singend, \pend
           \pstart herzlich Ihr \spacefill\mbox{Salten}\pend{}\pstart
           \noindent{}{[}hs. Metzl:{]} Beſten Gruß\pend
           \pstart \spacefill\mbox{R. Metzl}\pend{}
         
         \endnumbering\mylabel{h}\end{ledgroupsized}\begin{anhang}\end{anhang}\newcommand{\dateiname}{L03410}\newcommand{\titel}{Felix Salten und Richard Metzl an Arthur Schnitzler, [30. 7. 1905?]}\newcommand{\editorInnen}{Martin Anton Müller und Laura Untner}%% latex-leseansicht-abspann.tex
%% Abspann für die Leseansicht.
%% Der Schalter \ifkorrekturansicht ist bereits durch den Vorspann gesetzt.

%% latex-abspann.tex
%% Gemeinsamer Abspann für Korrekturansicht und Leseansicht.
%% Setzt den Schalter \ifkorrekturansicht voraus (gesetzt in den
%% einbindenden Dateien latex-korrekturansicht-abspann.tex bzw.
%% latex-leseansicht-abspann.tex).
%% ---------------------------------------------------------------

\normalsize

% Das esempio-Environment wird nur in der Leseansicht benötigt
\ifkorrekturansicht\else
\newenvironment{esempio}[3]%
{
    \vspace{1.5ex}
    \rlap{\underline{#1}}
    \par
    \setlength{\parindent}{0cm}
    \nopagebreak
    \leftskip=#2cm
    \rightskip=#3cm
}
{
    \par
}
\fi

\doendnotes{C}
\bigskip
\vfill

\clearpage

\footnotesize

\ifkorrekturansicht
  \lohead{\textsc{register}}
\fi

% theindex-Environment neu definieren ohne reledmac
\makeatletter
\renewenvironment{theindex}{%
  \ifkorrekturansicht
    \section*{\indexname}%
  \else
    \subsubsection*{Index der erwähnten Entitäten}%
  \fi
  \setlength{\parindent}{0pt}%
  \setlength{\parskip}{0pt plus 0.3pt}%
  \let\item\@idxitem
}{%
  \ifkorrekturansicht\clearpage\fi
}
\makeatother

\IfFileExists{\jobname-pw.ind}{\input{\jobname-pw.ind}}{}

% Quellenangabe nur in der Leseansicht
\ifkorrekturansicht\else
% Fallback-Definitionen, falls die .tex-Datei \titel etc. nicht gesetzt hat
\providecommand{\titel}{}
\providecommand{\editorInnen}{}
\providecommand{\dateiname}{\jobname}

\vspace{3cm}

\vfill

\footnotesize
\textsc{Quelle}: \titel. Herausgegeben von {\editorInnen}. In: \emph{Arthur Schnitzler: Briefwechsel mit Autorinnen und Autoren}.
 Digitale Edition, https://schnitzler-briefe.acdh.oeaw.ac.at/{\dateiname}.html (Stand \today)
\fi

\end{document}


      