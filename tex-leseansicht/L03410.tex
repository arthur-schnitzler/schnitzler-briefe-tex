%% latex-leseansicht-vorspann.tex
%% Vorspann für die Leseansicht.
%% Lädt die gemeinsame Datei latex-vorspann.tex mit nicht gesetztem Schalter.

\newif\ifkorrekturansicht
\korrekturansichtfalse

\input{../tex-inputs/latex-vorspann}


         
         \renewcommand{\erwaehntePersonen}{Personen: Shelomoh ben Mosheh Alḳabets, Richard Metzl, Felix Salten, Olga Schnitzler}
         \renewcommand{\erwaehnteOrte}{Orte: Dr. Ludwig Leber-Straße, Edmund-Weiß-Gasse 7, Mariazell, Reichenau an der Rax, Wien, XVIII., Währing}
         \renewcommand{\erwaehnteWerke}{Werke: Die Zeit, Lecha Dodi, Mariazell, Spätgotische Marienstatue mit Strahlenkranz, Tagebuch}
               \section[ Felix Salten und Richard Metzl an Arthur Schnitzler, {[}30. 7. 1905?{]}]{ Felix Salten und Richard Metzl an Arthur
               Schnitzler, {[}30. 7. 1905?{]}}\nopagebreak\mylabel{v}\rehead{ }\begin{ledgroupsized}[t]{13cm}\normalsize\beginnumbering\briefempfaengerindex{Schnitzler, Arthur@\textsc{Schnitzler, Arthur}!zzzMetzl, Richard@\emph{von Richard Metzl}!1905-07-303@{{[}30. 7. 1905?{]}}|(be}\briefempfaengerindex{Schnitzler, Arthur@\textsc{Schnitzler, Arthur}!zzzSalten, Felix@\emph{von Felix Salten}!1905-07-303@{{[}30. 7. 1905?{]}}|(be} \toendnotes[C]{\smallbreak\pagebreak[2]} \Standort{CUL, Schnitzler, B 89, B 1.}
\physDesc{Bildpostkarte, 107 Zeichen
\newline{}Handschrift Felix Salten: Bleistift, lateinische Kurrent\newline{}Handschrift Richard Metzl: Bleistift, deutsche Kurrent
\newline{}Versand: Stempel: »\nobreak{}\oindex{Mariazell@\textbf{Mariazell}|pwk}\textcolor{gray}{Mariazell}, 30 7 \textcolor{gray}{05}\nobreak{}«.  
\newline{}Ordnung: mit Bleistift von unbekannter Hand nummeriert: »202« }\toendnotes[C]{\smallbreak}\pstart{}{\pb}Herrn D\textsuperscript{r} Arthur Schnitzler\pend{}\pstart{}Wien XVIII.\oindex{XVIII., Waehring@\textbf{XVIII., Währing}|pw}\pend{}\pstart{}Spöttelgaſse 7\oindex{Edmund-Weiss-Gasse 7@\textbf{Edmund-Weiß-Gasse 7}|pw}\pend{}{\bigskip}\pstart
           \noindent{}\centering{}{\pb}\textcolor{gray}{\textbf{\textsc{Gruss aus \label{K_L03410-1v}\edtext{Mariazell\oindex{Mariazell@\textbf{Mariazell}|pw}}{\lemma{\textnormal{\emph{Mariazell}}}\Cendnote{\textnormal{Die am 6. 5. 1905
                           erwähnte »Maria Zell\oindex{Mariazell@\textbf{Mariazell}|pw}er Partie« fand
                           aus nicht überlieferten Gründen letztlich ohne Beteiligung Schnitzlers\pwindex{Schnitzler, Arthur 15.05.1862 – 21.10.1931@\textsc{Schnitzler, Arthur} (15.05.1862 – 21.10.1931), \emph{Schriftsteller, Mediziner}|pwk} und seiner Frau\pwindex{Schnitzler, Olga 17.01.1882 – 13.01.1970@\textsc{Schnitzler, Olga} (17.01.1882 – 13.01.1970), \emph{Schauspielerin, Sängerin}|pwkv} statt und lässt
                           sich auf ein Zeitfenster eingrenzen. Am 28. 7. 1905
                           sahen sich Schnitzler\pwindex{Schnitzler, Arthur 15.05.1862 – 21.10.1931@\textsc{Schnitzler, Arthur} (15.05.1862 – 21.10.1931), \emph{Schriftsteller, Mediziner}|pwk} und Salten\pwindex{Salten, Felix 06.09.1869 – 08.10.1945@\textsc{Salten, Felix} (06.09.1869 – 08.10.1945), \emph{Schriftsteller, Journalist, Chefredakteur}|pwk}
                           in Reichenau an der Rax\oindex{Reichenau an der Rax@\textbf{Reichenau an der Rax}|pwk}, am
                              31. 7. 1931 war Salten\pwindex{Salten, Felix 06.09.1869 – 08.10.1945@\textsc{Salten, Felix} (06.09.1869 – 08.10.1945), \emph{Schriftsteller, Journalist, Chefredakteur}|pwk}
                           wieder in Wien\oindex{Wien@\textbf{Wien}|pwk} – »aus Mariazell\oindex{Mariazell@\textbf{Mariazell}|pw}, angeekelt«, wie
                           Schnitzler\pwindex{Schnitzler, Arthur 15.05.1862 – 21.10.1931@\textsc{Schnitzler, Arthur} (15.05.1862 – 21.10.1931), \emph{Schriftsteller, Mediziner}|pwk} im \emph{Tagebuch}\pwindex{\textcolor{red}{\textsuperscript{XXXX1 indx}}!Tagebuch1981 – 2000@\strich\emph{Tagebuch} {[}Hrsg., 1981 – 2000{]}|pwk} festhielt. Vgl. Martin Finder\pwindex{Salten, Felix 06.09.1869 – 08.10.1945@\textsc{Salten, Felix} (06.09.1869 – 08.10.1945), \emph{Schriftsteller, Journalist, Chefredakteur}|pwk} [ = Felix Salten\pwindex{Salten, Felix 06.09.1869 – 08.10.1945@\textsc{Salten, Felix} (06.09.1869 – 08.10.1945), \emph{Schriftsteller, Journalist, Chefredakteur}|pwk}]: \emph{Mariazell}\pwindex{Mariazell1905-08-20@\emph{Mariazell} {[}1905-08-20{]}|pwk}. In: 
                              \emph{Die Zeit}\pwindex{Zeit1902-09-27 – 1919@\emph{Die Zeit} {[}1902-09-27 – 1919{]}|pwk}, Jg. 4, Nr. 1042, 20. 8. 1905, S. 1–2.}}}\label{K_L03410-1h}}}}\pend
           \pstart
           \noindent{}\centering{}\textcolor{gray}{\textbf{MARIENSTATUE\pwindex{?? Werk@Nicht ermittelte Verfasserinnen und Verfasser!Spaetgotische Marienstatue mit Strahlenkranz@\emph{Spätgotische Marienstatue mit Strahlenkranz}|pw}}}\pend
           \pstart
           \noindent{}\centering{}\textcolor{gray}{\textbf{WIENERGASSE\oindex{Dr. Ludwig Leber-Strasse@\textbf{Dr. Ludwig Leber-Straße}|pw}}}\pend
           \pstart
           Das \label{K_L03410-2v}\edtext{Lechodaudi\pwindex{Alḳabets, Shelomoh ben Mosheh 1505 – 1576@\textsc{Alḳabets, Shelomoh ben Mosheh} (1505 – 1576), \emph{Mystiker, Dichter}!Lecha Dodi@\strich\emph{Lecha Dodi}|pw}}{\lemma{\textnormal{\emph{Lechodaudi}}}\Cendnote{\textnormal{Lecha Dodi (L’kha Dodi)\pwindex{Alḳabets, Shelomoh ben Mosheh 1505 – 1576@\textsc{Alḳabets, Shelomoh ben Mosheh} (1505 – 1576), \emph{Mystiker, Dichter}!Lecha Dodi@\strich\emph{Lecha Dodi}|pwkv}
                  sind die ersten beiden Wörter einer Hymne von Shelomoh ben Mosheh Alḳabets\pwindex{Alḳabets, Shelomoh ben Mosheh 1505 – 1576@\textsc{Alḳabets, Shelomoh ben Mosheh} (1505 – 1576), \emph{Mystiker, Dichter}|pwk}, mit der der Sabbat eingeläutet wird. Salten\pwindex{Salten, Felix 06.09.1869 – 08.10.1945@\textsc{Salten, Felix} (06.09.1869 – 08.10.1945), \emph{Schriftsteller, Journalist, Chefredakteur}|pwk} dürfte hier dem  
                  Vergnügen Ausdruck verleihen, in einem katholischen Wallfahrtsort ein jüdisches Lied 
                  zu singen. Um tatsächlich mit dem Beginn des Sabbats übereinzustimmen, müsste die Karte
                  am Freitag Abend verfasst worden sein. Der Poststempel
                  weist aber auf Sonntag, den 30. sodass Salten\pwindex{Salten, Felix 06.09.1869 – 08.10.1945@\textsc{Salten, Felix} (06.09.1869 – 08.10.1945), \emph{Schriftsteller, Journalist, Chefredakteur}|pwk} hier
                  nicht versuchen dürfte, in der Aussage eine Datums- und Uhrzeitangabe zu verstecken.}}}\label{K_L03410-2h}
                  singend,
            \pend
           \pstart herzlich Ihr \spacefill\mbox{Salten}\pend{}\pstart
           {[}hs. Metzl:{]} Beſten Gruß {\\[\baselineskip]}\spacefill\mbox{R Metzl}\pend
           \leftskip=0em{}
         
         \endnumbering\mylabel{h}\end{ledgroupsized}  \newcommand{\dateiname}{L03410}\newcommand{\titel}{Felix Salten und Richard Metzl an Arthur Schnitzler, [30. 7. 1905?]}\newcommand{\editorInnen}{Martin Anton Müller und Laura Untner}%% latex-leseansicht-abspann.tex
%% Abspann für die Leseansicht.
%% Der Schalter \ifkorrekturansicht ist bereits durch den Vorspann gesetzt.

%% latex-abspann.tex
%% Gemeinsamer Abspann für Korrekturansicht und Leseansicht.
%% Setzt den Schalter \ifkorrekturansicht voraus (gesetzt in den
%% einbindenden Dateien latex-korrekturansicht-abspann.tex bzw.
%% latex-leseansicht-abspann.tex).
%% ---------------------------------------------------------------

\normalsize

% Das esempio-Environment wird nur in der Leseansicht benötigt
\ifkorrekturansicht\else
\newenvironment{esempio}[3]%
{
    \vspace{1.5ex}
    \rlap{\underline{#1}}
    \par
    \setlength{\parindent}{0cm}
    \nopagebreak
    \leftskip=#2cm
    \rightskip=#3cm
}
{
    \par
}
\fi

\doendnotes{C}
\bigskip
\vfill

\clearpage

\footnotesize

\ifkorrekturansicht
  \lohead{\textsc{register}}
\fi

% theindex-Environment neu definieren ohne reledmac
\makeatletter
\renewenvironment{theindex}{%
  \ifkorrekturansicht
    \section*{\indexname}%
  \else
    \subsubsection*{Index der erwähnten Entitäten}%
  \fi
  \setlength{\parindent}{0pt}%
  \setlength{\parskip}{0pt plus 0.3pt}%
  \let\item\@idxitem
}{%
  \ifkorrekturansicht\clearpage\fi
}
\makeatother

\IfFileExists{\jobname-pw.ind}{\input{\jobname-pw.ind}}{}

% Quellenangabe nur in der Leseansicht
\ifkorrekturansicht\else
% Fallback-Definitionen, falls die .tex-Datei \titel etc. nicht gesetzt hat
\providecommand{\titel}{}
\providecommand{\editorInnen}{}
\providecommand{\dateiname}{\jobname}

\vspace{3cm}

\vfill

\footnotesize
\textsc{Quelle}: \titel. Herausgegeben von {\editorInnen}. In: \emph{Arthur Schnitzler: Briefwechsel mit Autorinnen und Autoren}.
 Digitale Edition, https://schnitzler-briefe.acdh.oeaw.ac.at/{\dateiname}.html (Stand \today)
\fi

\end{document}


      