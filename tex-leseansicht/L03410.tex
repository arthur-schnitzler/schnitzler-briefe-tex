%% latex-leseansicht-vorspann.tex
%% Vorspann für die Leseansicht.
%% Lädt die gemeinsame Datei latex-vorspann.tex mit nicht gesetztem Schalter.

\newif\ifkorrekturansicht
\korrekturansichtfalse

\input{../tex-inputs/latex-vorspann}


\section[ Felix Salten und Richard Metzl an Arthur Schnitzler, [30. 7. 1905?]]{L03410 Felix Salten und Richard Metzl an Arthur
               Schnitzler,  [30. 7. 1905?]}
\nopagebreak\mylabel{L03410v}
\rehead{ }\normalsize\beginnumbering\briefempfaengerindex{Schnitzler, Arthur@\textsc{Schnitzler, Arthur}!zzzMetzl, Richard@\emph{von Richard Metzl}!1905-07-303@{{[}30. 7. 1905?{]}}|(be}\briefempfaengerindex{Schnitzler, Arthur@\textsc{Schnitzler, Arthur}!zzzSalten, Felix@\emph{von Felix Salten}!1905-07-303@{{[}30. 7. 1905?{]}}|(be}
\toendnotes[C]{\smallbreak\pagebreak[2]}
\correspDesc{Versand  durch Felix Salten, Richard Metzl am [30. 7. 1905?] in Mariazell
\newline{}Erhalt  durch Arthur Schnitzler im Zeitraum [31. 7. 1905
                  – 3. 8. 1905?] in Wien}\toendnotes[C]{\smallbreak}
\Standort{CUL, Schnitzler, B 89, B 1.}
\physDesc{Bildpostkarte, 107 Zeichen
\newline{}Handschrift Felix Salten: Bleistift, lateinische Kurrent
\newline{}Handschrift Richard Metzl: Bleistift, deutsche Kurrent
\newline{}Versand: Stempel: »\nobreak{}\oindex{Mariazell@\textbf{Mariazell}, \emph{Hauptstadt}|pwk}\textcolor{gray}{Mariazell}, 30 7 \textcolor{gray}{05}\nobreak{}«.  
\newline{}Ordnung: mit Bleistift von unbekannter Hand nummeriert: »202« }\toendnotes[C]{\smallbreak}\pstart{}{\pb}Herrn D\textsuperscript{r} Arthur Schnitzler\pend{}\pstart{}Wien XVIII.\oindex{XVIII., Währing@\textbf{XVIII., Währing}, \emph{Verwaltungsgebiet}|pw}\pend{}\pstart{}Spöttelgaſse 7\oindex{Wien@\textbf{Wien}!XVIII., Währing@\textbf{XVIII., Währing}!Edmund-Weiß-Gasse 7@\textbf{Edmund-Weiß-Gasse 7}, \emph{Wohngebäude}|pw}\pend{}{\bigskip}
\pstart
           \noindent{}\centering{}{\pb}\textcolor{gray}{\textbf{\textsc{Gruss aus \label{K_L03410-1v}\edtext{Mariazell\oindex{Mariazell@\textbf{Mariazell}, \emph{Hauptstadt}|pw}}{\lemma{\textnormal{\emph{Mariazell}}}\Cendnote{\textnormal{Die am XXXX Auszeichnungsfehler: Dokument L03408 nicht gefunden erwähnte »Maria Zell\oindex{Mariazell@\textbf{Mariazell}, \emph{Hauptstadt}|pw}er Partie« fand aus
                        nicht überlieferten Gründen letztlich ohne Beteiligung Schnitzlers und seiner Frau\pwindex{Schnitzler, Olga 17.\,1.\,1882 Wien – 13.\,1.\,1970 Lugano@\textsc{Schnitzler, Olga} (17.\,1.\,1882 Wien – 13.\,1.\,1970 Lugano), \emph{Schauspielerin, Sängerin}|pwkv} statt und lässt sich auf ein
                        Zeitfenster eingrenzen. Am 28. 7. 1905 sahen sich Schnitzler und Salten\pwindex{Salten, Felix 6.\,9.\,1869 Budapest – 8.\,10.\,1945 Zürich@\textsc{Salten, Felix} (6.\,9.\,1869 Budapest – 8.\,10.\,1945 Zürich), \emph{Schriftsteller, Journalist, Chefredakteur}|pwk} in Reichenau an der Rax\oindex{Reichenau an der Rax@\textbf{Reichenau an der Rax}, \emph{Verwaltungsgebiet}|pwk},
                        am 31. 7. 1931
                        war Salten\pwindex{Salten, Felix 6.\,9.\,1869 Budapest – 8.\,10.\,1945 Zürich@\textsc{Salten, Felix} (6.\,9.\,1869 Budapest – 8.\,10.\,1945 Zürich), \emph{Schriftsteller, Journalist, Chefredakteur}|pwk} wieder in Wien\oindex{Wien@\textbf{Wien}, \emph{Verwaltungsgebiet}|pwk} – »aus Mariazell\oindex{Mariazell@\textbf{Mariazell}, \emph{Hauptstadt}|pw}, angeekelt«, wie Schnitzler im \emph{Tagebuch}\pwindex{Schnitzler, Arthur 15.\,5.\,1862 Wien – 21.\,10.\,1931 ebd.@\textsc{Schnitzler, Arthur} (15.\,5.\,1862 Wien – 21.\,10.\,1931 ebd.), \emph{Schriftsteller, Mediziner}!Tagebuch@\strich\emph{Tagebuch}|pwk} festhielt. Vgl. Martin Finder\pwindex{Salten, Felix 6.\,9.\,1869 Budapest – 8.\,10.\,1945 Zürich@\textsc{Salten, Felix} (6.\,9.\,1869 Budapest – 8.\,10.\,1945 Zürich), \emph{Schriftsteller, Journalist, Chefredakteur}|pwk} [ = Felix
                              Salten\pwindex{Salten, Felix 6.\,9.\,1869 Budapest – 8.\,10.\,1945 Zürich@\textsc{Salten, Felix} (6.\,9.\,1869 Budapest – 8.\,10.\,1945 Zürich), \emph{Schriftsteller, Journalist, Chefredakteur}|pwk}]: \emph{Mariazell}\pwindex{Salten, Felix 6.\,9.\,1869 Budapest – 8.\,10.\,1945 Zürich@\textsc{Salten, Felix} (6.\,9.\,1869 Budapest – 8.\,10.\,1945 Zürich), \emph{Schriftsteller, Journalist, Chefredakteur}!Mariazell@\strich\emph{Mariazell}|pwk}. In: \emph{Die Zeit}\pwindex{Zeit@\emph{Die Zeit}|pwk}, Jg. 4, Nr. 1042,
                              20. 8. 1905, S. 1–2.}}}\label{K_L03410-1}}}}\pend
           
\pstart
           \centering{}\textcolor{gray}{\textbf{MARIENSTATUE\pwindex{Spätgotische Marienstatue mit Strahlenkranz@\emph{Spätgotische Marienstatue mit Strahlenkranz}|pw}}}\pend
           
\pstart
           \centering{}\textcolor{gray}{\textbf{WIENERGASSE\oindex{Dr. Ludwig Leber-Straße@\textbf{Dr. Ludwig Leber-Straße}, \emph{Straße}|pw}}}\pend
           \vspace{1em}
\pstart
           \noindent{}{\pb}Das \label{K_L03410-2v}\edtext{Lechodaudi\pwindex{Alḳabets, Shelomoh ben Mosheh 1505 Thessaloniki – 1576 Safed@\textsc{Alḳabets, Shelomoh ben Mosheh} (1505 Thessaloniki – 1576 Safed), \emph{Mystiker, Dichter}!Lecha Dodi@\strich\emph{Lecha Dodi}|pw}}{\lemma{\textnormal{\emph{Lechodaudi}}}\Cendnote{\textnormal{Lecha Dodi (L’kha Dodi)\pwindex{Alḳabets, Shelomoh ben Mosheh 1505 Thessaloniki – 1576 Safed@\textsc{Alḳabets, Shelomoh ben Mosheh} (1505 Thessaloniki – 1576 Safed), \emph{Mystiker, Dichter}!Lecha Dodi@\strich\emph{Lecha Dodi}|pwkv}
                  sind die ersten beiden Wörter einer Hymne von Shelomoh ben Mosheh Alḳabets\pwindex{Alḳabets, Shelomoh ben Mosheh 1505 Thessaloniki – 1576 Safed@\textsc{Alḳabets, Shelomoh ben Mosheh} (1505 Thessaloniki – 1576 Safed), \emph{Mystiker, Dichter}|pwk}, mit der der Sabbat eingeläutet wird. Salten\pwindex{Salten, Felix 6.\,9.\,1869 Budapest – 8.\,10.\,1945 Zürich@\textsc{Salten, Felix} (6.\,9.\,1869 Budapest – 8.\,10.\,1945 Zürich), \emph{Schriftsteller, Journalist, Chefredakteur}|pwk} dürfte hier dem Vergnügen Ausdruck
                  verleihen, in einem katholischen Wallfahrtsort ein jüdisches Lied zu singen. Um
                  tatsächlich mit dem Beginn des Sabbats übereinzustimmen, müsste die Karte am
                  Freitag Abend verfasst worden sein. Der Poststempel weist aber auf Sonntag, den 30. sodass Salten\pwindex{Salten, Felix 6.\,9.\,1869 Budapest – 8.\,10.\,1945 Zürich@\textsc{Salten, Felix} (6.\,9.\,1869 Budapest – 8.\,10.\,1945 Zürich), \emph{Schriftsteller, Journalist, Chefredakteur}|pwk} hier nicht
                  versuchen dürfte, in der Aussage eine Datums- und Uhrzeitangabe zu
                  verstecken.}}}\label{K_L03410-2} singend,\pend
           \pstart herzlich Ihr \spacefill\mbox{Salten}\pend{}\selectlanguage{ngerman}\vspace{1em}
\pstart
           {[}hs. Metzl:{]} Beſten Gruß {\\[\baselineskip]}\spacefill\mbox{R Metzl}\pend
           \leftskip=0em{}\selectlanguage{ngerman}\endnumbering\briefempfaengerindex{Schnitzler, Arthur@\textsc{Schnitzler, Arthur}!zzzMetzl, Richard@\emph{von Richard Metzl}!1905-07-303@{{[}30. 7. 1905?{]}}|)be}\briefempfaengerindex{Schnitzler, Arthur@\textsc{Schnitzler, Arthur}!zzzSalten, Felix@\emph{von Felix Salten}!1905-07-303@{{[}30. 7. 1905?{]}}|)be}\mylabel{L03410h}  \newcommand{\dateiname}{L03410}\newcommand{\titel}{Felix Salten und Richard Metzl an Arthur Schnitzler, [30. 7. 1905?]}\newcommand{\editorInnen}{Martin Anton Müller und Laura Untner}%% latex-leseansicht-abspann.tex
%% Abspann für die Leseansicht.
%% Der Schalter \ifkorrekturansicht ist bereits durch den Vorspann gesetzt.

%% latex-abspann.tex
%% Gemeinsamer Abspann für Korrekturansicht und Leseansicht.
%% Setzt den Schalter \ifkorrekturansicht voraus (gesetzt in den
%% einbindenden Dateien latex-korrekturansicht-abspann.tex bzw.
%% latex-leseansicht-abspann.tex).
%% ---------------------------------------------------------------

\normalsize

% Das esempio-Environment wird nur in der Leseansicht benötigt
\ifkorrekturansicht\else
\newenvironment{esempio}[3]%
{
    \vspace{1.5ex}
    \rlap{\underline{#1}}
    \par
    \setlength{\parindent}{0cm}
    \nopagebreak
    \leftskip=#2cm
    \rightskip=#3cm
}
{
    \par
}
\fi

\doendnotes{C}
\bigskip
\vfill

\clearpage

\footnotesize

\ifkorrekturansicht
  \lohead{\textsc{register}}
\fi

% theindex-Environment neu definieren ohne reledmac
\makeatletter
\renewenvironment{theindex}{%
  \ifkorrekturansicht
    \section*{\indexname}%
  \else
    \subsubsection*{Index der erwähnten Entitäten}%
  \fi
  \setlength{\parindent}{0pt}%
  \setlength{\parskip}{0pt plus 0.3pt}%
  \let\item\@idxitem
}{%
  \ifkorrekturansicht\clearpage\fi
}
\makeatother

\IfFileExists{\jobname-pw.ind}{\input{\jobname-pw.ind}}{}

% Quellenangabe nur in der Leseansicht
\ifkorrekturansicht\else
% Fallback-Definitionen, falls die .tex-Datei \titel etc. nicht gesetzt hat
\providecommand{\titel}{}
\providecommand{\editorInnen}{}
\providecommand{\dateiname}{\jobname}

\vspace{3cm}

\vfill

\footnotesize
\textsc{Quelle}: \titel. Herausgegeben von {\editorInnen}. In: \emph{Arthur Schnitzler: Briefwechsel mit Autorinnen und Autoren}.
 Digitale Edition, https://schnitzler-briefe.acdh.oeaw.ac.at/{\dateiname}.html (Stand \today)
\fi

\end{document}


