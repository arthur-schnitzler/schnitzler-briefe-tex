%% latex-korrekturansicht-vorspann.tex
%% Vorspann für die Korrekturansicht.
%% Lädt die gemeinsame Datei latex-vorspann.tex mit gesetztem Schalter.

\newif\ifkorrekturansicht
\korrekturansichttrue

\input{../tex-inputs/latex-vorspann}


\section[Hermann Bahr an Arthur Schnitzler, {[}12. 10. 1895{]}]{L00505 Hermann Bahr an Arthur Schnitzler, {[}12. 10. 1895{]}}
\nopagebreak\mylabel{L00505v}
\rehead{ }\normalsize\beginnumbering\briefempfaengerindex{Schnitzler, Arthur@\textsc{Schnitzler, Arthur}!zzzBahr, Hermann@\emph{von Hermann Bahr}!1895-10-121@{{[}12. 10. 1895{]}}|(be}
\toendnotes[C]{\smallbreak\pagebreak[2]}\Standort{CUL, Schnitzler, B 5b.}
\physDesc{Brief, 1 Blatt, 1 Seite, 130 Zeichen
\newline{}Handschrift: Bleistift, deutsche Kurrent
\newline{}Schnitzler: mit rotem Buntstift datiert: »10/10 95 / ev 12/10« 
\newline{}Ordnung: 1) mit rotem Buntstift von unbekannter Hand nummeriert:
                                    »34«  2) mit Bleistift von unbekannter Hand nummeriert:
                                    »34«}
\buchAbdrucke{\weitereDrucke{Hermann Bahr, Arthur Schnitzler: \emph{Briefwechsel, Aufzeichnungen, Dokumente (1891–1931)}. Göttingen: \emph{Wallstein} 2018, S. 109.} }
\pstart
           \noindent{}{\pb}Lieber Thuri! Ich war neulich ſchon hier und bins jetzt wieder, um
               Dir mündlich von ganzem Herzen zu gratulieren.\pend
           
\pstart
           Dein treuer{\\[\baselineskip]}\spacefill\mbox{Hermann}\pend
           \leftskip=0em{}\selectlanguage{ngerman}\endnumbering\briefempfaengerindex{Schnitzler, Arthur@\textsc{Schnitzler, Arthur}!zzzBahr, Hermann@\emph{von Hermann Bahr}!1895-10-121@{{[}12. 10. 1895{]}}|)be}\mylabel{L00505h}  \normalsize

\doendnotes{C}
\bigskip
\vfill

\clearpage

\footnotesize

\lohead{\textsc{register}}

% Definiere theindex-Environment komplett neu ohne reledmac
\makeatletter
\renewenvironment{theindex}{%
  \section*{\indexname}%
  \setlength{\parindent}{0pt}%
  \setlength{\parskip}{0pt plus 0.3pt}%
  \let\item\@idxitem
}{%
  \clearpage
}
\makeatother

\IfFileExists{\jobname-pw.ind}{\input{\jobname-pw.ind}}{}

\end{document}

      