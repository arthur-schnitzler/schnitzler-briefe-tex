%% latex-korrekturansicht-vorspann.tex
%% Vorspann für die Korrekturansicht.
%% Lädt die gemeinsame Datei latex-vorspann.tex mit gesetztem Schalter.

\newif\ifkorrekturansicht
\korrekturansichttrue

\input{../tex-inputs/latex-vorspann}


\section[Arthur Schnitzler an Richard Beer-Hofmann, 23. 7. 1913]{L02145 Arthur Schnitzler an Richard Beer-Hofmann, 23. 7. 1913}
\nopagebreak\mylabel{L02145v}
\rehead{ }\normalsize\beginnumbering\briefempfaengerindex{Beer-Hofmann, Richard@\textsc{Beer-Hofmann, Richard}!zzzSchnitzler, Arthur@\emph{von Arthur Schnitzler}!1913-07-231@{23. 7. 1913}|(be}
\toendnotes[C]{\smallbreak\pagebreak[2]}\Standort{YCGL, MSS 31.}
\physDesc{Bildpostkarte, 367 Zeichen
\newline{}Handschrift: schwarze Tinte, deutsche Kurrent
\newline{}Versand: Stempel: »\nobreak{}Wien 110, {[}23.{]} VII. 13, 3\nobreak{}«.  
\newline{}Zusatz: Postkartenmotiv mit Olga\pwindex{Schnitzler, Olga 17.01.1882 – 13.01.1970@\textsc{Schnitzler, Olga} (17.01.1882 – 13.01.1970), \emph{Schauspieler/Schauspielerin, Sänger/Sängerin}|pw}
                                 und Heinrich\pwindex{Schnitzler, Heinrich 09.08.1902 – 12.07.1982@\textsc{Schnitzler, Heinrich} (09.08.1902 – 12.07.1982), \emph{Regisseur/Regisseurin, Schauspieler/Schauspielerin}|pw} links vor dem
                                 Haus und Schnitzler und Lili\pwindex{Cappellini, Lili 13.09.1909 – 26.07.1928@\textsc{Cappellini, Lili} (13.09.1909 – 26.07.1928)|pw}
                                 auf dem Söller }
\buchAbdrucke{\weitereDrucke{Arthur Schnitzler, Richard Beer-Hofmann: \emph{Briefwechsel 1891–1931}. Wien, Zürich: \emph{Europaverlag} 1992, S. 218.} }\pstart{}{\pb}Herrn \textsc{Dr. Richard Beer
                     Hofmann}\pend{}\pstart{}aus Wien\oindex{Wien@\textbf{Wien}, \emph{A.ADM2}|pw}\pend{}\pstart{}\textsc{St. Moritz\oindex{St. Moritz@\textbf{St. Moritz}, \emph{P.PPL}|pw}}\pend{}\pstart{}\textsc{im Engadin\oindex{Engadin@\textbf{Engadin}, \emph{T.VAL}|pw}}\pend{}\pstart{}\textsc{Hotel du lac\oindex{Hotel du Lac@\textbf{Hotel du Lac}, \emph{Hotel (K.HTL)}|pw}.}\pend{}{\bigskip}
\pstart
           \noindent{}\centering{}{\pb}\textcolor{gray}{\textbf{Wien, XVIII, Sternwartestr. 71\oindex{Sternwartestrasse 71@\textbf{Sternwartestraße 71}, \emph{Wohngebäude (K.WHS)}|pw}.}}\pend
           \vspace{1em}
\pstart
           {\pb}23. 7. 913\pend
           
\pstart{}lieber Richard,\pend\vspace{0.5em}
\pstart
           heute fahren wir nach Brioni\oindex{Brijuni@\textbf{Brijuni}, \emph{P.PPL}|pw}, Lili\pwindex{Cappellini, Lili 13.09.1909 – 26.07.1928@\textsc{Cappellini, Lili} (13.09.1909 – 26.07.1928)|pw} iſt ſchon ſeit ein paar Tagen dort –
               ſchreiben Sie mir doch ein Wort dahin. Es war eine etwas unruhig-ruhige Zeit. Sie
               haben wohl kein ſchönes Wetter gehabt bis heut – {\pb}nun
                  wir\textcolor{gray}{ds} hoffentlich beſſer, – überall. Wir grüßen Sie alle
               herzlichſt\pend
           
\pstart
           Ihr{\\[\baselineskip]}\spacefill\mbox{A.}\pend
           \leftskip=0em{}\selectlanguage{ngerman}\endnumbering\briefempfaengerindex{Beer-Hofmann, Richard@\textsc{Beer-Hofmann, Richard}!zzzSchnitzler, Arthur@\emph{von Arthur Schnitzler}!1913-07-231@{23. 7. 1913}|)be}\mylabel{L02145h}  \normalsize

\doendnotes{C}
\bigskip
\vfill

\clearpage

\footnotesize

\lohead{\textsc{register}}

% Definiere theindex-Environment komplett neu ohne reledmac
\makeatletter
\renewenvironment{theindex}{%
  \section*{\indexname}%
  \setlength{\parindent}{0pt}%
  \setlength{\parskip}{0pt plus 0.3pt}%
  \let\item\@idxitem
}{%
  \clearpage
}
\makeatother

\IfFileExists{\jobname-pw.ind}{\input{\jobname-pw.ind}}{}

\end{document}

      