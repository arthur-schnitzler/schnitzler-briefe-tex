%% latex-leseansicht-vorspann.tex
%% Vorspann für die Leseansicht.
%% Lädt die gemeinsame Datei latex-vorspann.tex mit nicht gesetztem Schalter.

\newif\ifkorrekturansicht
\korrekturansichtfalse

\input{../tex-inputs/latex-vorspann}


\section[Arthur Schnitzler an Gustav Schwarzkopf, {{[}}19.? 5. 1899{{]}}]{L04210 Arthur Schnitzler an Gustav Schwarzkopf, {[}19.? 5. 1899{]}}
\nopagebreak\mylabel{L04210v}
\rehead{ }\normalsize\beginnumbering\briefempfaengerindex{Schwarzkopf, Gustav@\textsc{Schwarzkopf, Gustav}!zzzSchnitzler, Arthur@\emph{von Arthur Schnitzler}!1899-05-192@{{[}19.? 5. 1899{]}}|(be}
\toendnotes[C]{\smallbreak\pagebreak[2]}
\correspDesc{Versand  durch Arthur Schnitzler am [19.? 5. 1899] in Wien
\newline{}Erhalt  durch Gustav Schwarzkopf im Zeitraum [19. 5. 1899 – 22. 5. 1899?] in Wien}\toendnotes[C]{\smallbreak}
\Standort{CUL, Schnitzler, B 96.}
\physDesc{Brief, 1 Blatt, 2 Seiten, 369 Zeichen
\newline{}Handschrift: Bleistift, deutsche Kurrent}\toendnotes[C]{\smallbreak}
\pstart
           \raggedleft{}{\pb}\label{K_L04210-1v}\edtext{Freitag}{\lemma{\textnormal{\emph{Freitag}}}\Cendnote{\textnormal{ Der Brief ist undatiert. Durch den
                     Inhalt ist klar, dass er vor der Aufführung
                        von \emph{Die von Strebersdorf}\pwindex{Kolloden, M. 2.\,9.\,1847 Gmina Ostrzeszów – 25.\,11.\,1924 Krems@\textsc{Kolloden, M.} (2.\,9.\,1847 Gmina Ostrzeszów – 25.\,11.\,1924 Krems), \emph{Schriftsteller, Dramaturg, Landwirt}!von Strebersdorf. Komödie in drei Acten@\strich\emph{Die von Strebersdorf. Komödie in drei Acten}|pwk}\eventindex{Carl-Theater@\textbf{Carl-Theater}!Aufführung von Die von Strebersdorf, 21.5.1899@Aufführung von Die von Strebersdorf, 21.5.1899|pwk} am 21. 5. 1899 verfasst
                     wurde. Gegen den davorliegenden Freitage spricht ein Detail, nämlich dass sich
                        Schnitzler und Schwarzkopf\pwindex{Schwarzkopf, Gustav 7.\,11.\,1853 Wien – 13.\,11.\,1939 ebd.@\textsc{Schwarzkopf, Gustav} (7.\,11.\,1853 Wien – 13.\,11.\,1939 ebd.), \emph{Schriftsteller}|pwk} am Samstag in der Früh statt am Abend
                     trafen (13. 5. 1899). }}}\label{K_L04210-1}\pend
           
\pstart{}lieber Guſtav,\pend\vspace{0.5em}
\pstart
           für \label{K_L04210-2v}\edtext{Pfingſtſonntag}{\lemma{\textnormal{\emph{Pfingstsonntag}}}\Cendnote{\textnormal{ Pfingstsonntag war in diesem Jahr am 21. 5. 1899.
                     }}}\label{K_L04210-2} beko{\geminationm}’ ich eine Loge zu »Strebersdorf\pwindex{Kolloden, M. 2.\,9.\,1847 Gmina Ostrzeszów – 25.\,11.\,1924 Krems@\textsc{Kolloden, M.} (2.\,9.\,1847 Gmina Ostrzeszów – 25.\,11.\,1924 Krems), \emph{Schriftsteller, Dramaturg, Landwirt}!von Strebersdorf. Komödie in drei Acten@\strich\emph{Die von Strebersdorf. Komödie in drei Acten}|pw}\eventindex{Carl-Theater@\textbf{Carl-Theater}!Aufführung von Die von Strebersdorf, 21.5.1899@Aufführung von Die von Strebersdorf, 21.5.1899|pwv}«, wenn Sie nirgends anders verſagt,
               bitte \label{K_L04210-3v}\edtext{ko{\geminationm}en Sie mit}{\lemma{\textnormal{\emph{kommen Sie mit}}}\Cendnote{\textnormal{Eine Teilnahme von Schwarzkopf\pwindex{Schwarzkopf, Gustav 7.\,11.\,1853 Wien – 13.\,11.\,1939 ebd.@\textsc{Schwarzkopf, Gustav} (7.\,11.\,1853 Wien – 13.\,11.\,1939 ebd.), \emph{Schriftsteller}|pwk} ist nicht belegt.}}}\label{K_L04210-3}
               mir.\pend
           
\pstart
           Ich bin den ganzen Nachmittg zu Hauſe, Sie ko{\geminationm}en
               vielleicht, wann Ihnen angenehm. {\pb}Jedenfalls ſeh ich Sie hoffentlich Samſtg Abds im \textsc{Café};– we{\geminationn} ich nicht radfahren{ }ſollte,{ }ſchau ich
               vielleicht vor 6 oder ½ 7 zu Ihnen hinauf.\pend
           
\pstart
           Herzlichſt Ihr{\\[\baselineskip]}\spacefill\mbox{Arthur}\pend
           \leftskip=0em{}\selectlanguage{ngerman}\endnumbering\briefempfaengerindex{Schwarzkopf, Gustav@\textsc{Schwarzkopf, Gustav}!zzzSchnitzler, Arthur@\emph{von Arthur Schnitzler}!1899-05-192@{{[}19.? 5. 1899{]}}|)be}\mylabel{L04210h}
\begin{anhang}
\end{anhang}\newcommand{\dateiname}{L04210}\newcommand{\titel}{Arthur Schnitzler an Gustav Schwarzkopf, [19.? 5. 1899]}\newcommand{\editorInnen}{Herausgegeben von Jahnke, SelmaMüller, Martin Anton}%% latex-leseansicht-abspann.tex
%% Abspann für die Leseansicht.
%% Der Schalter \ifkorrekturansicht ist bereits durch den Vorspann gesetzt.

%% latex-abspann.tex
%% Gemeinsamer Abspann für Korrekturansicht und Leseansicht.
%% Setzt den Schalter \ifkorrekturansicht voraus (gesetzt in den
%% einbindenden Dateien latex-korrekturansicht-abspann.tex bzw.
%% latex-leseansicht-abspann.tex).
%% ---------------------------------------------------------------

\normalsize

% Das esempio-Environment wird nur in der Leseansicht benötigt
\ifkorrekturansicht\else
\newenvironment{esempio}[3]%
{
    \vspace{1.5ex}
    \rlap{\underline{#1}}
    \par
    \setlength{\parindent}{0cm}
    \nopagebreak
    \leftskip=#2cm
    \rightskip=#3cm
}
{
    \par
}
\fi

\doendnotes{C}
\bigskip
\vfill

\clearpage

\footnotesize

\ifkorrekturansicht
  \lohead{\textsc{register}}
\fi

% theindex-Environment neu definieren ohne reledmac
\makeatletter
\renewenvironment{theindex}{%
  \ifkorrekturansicht
    \section*{\indexname}%
  \else
    \subsubsection*{Index der erwähnten Entitäten}%
  \fi
  \setlength{\parindent}{0pt}%
  \setlength{\parskip}{0pt plus 0.3pt}%
  \let\item\@idxitem
}{%
  \ifkorrekturansicht\clearpage\fi
}
\makeatother

\IfFileExists{\jobname-pw.ind}{\input{\jobname-pw.ind}}{}

% Quellenangabe nur in der Leseansicht
\ifkorrekturansicht\else
% Fallback-Definitionen, falls die .tex-Datei \titel etc. nicht gesetzt hat
\providecommand{\titel}{}
\providecommand{\editorInnen}{}
\providecommand{\dateiname}{\jobname}

\vspace{3cm}

\vfill

\footnotesize
\textsc{Quelle}: \titel. Herausgegeben von {\editorInnen}. In: \emph{Arthur Schnitzler: Briefwechsel mit Autorinnen und Autoren}.
 Digitale Edition, https://schnitzler-briefe.acdh.oeaw.ac.at/{\dateiname}.html (Stand \today)
\fi

\end{document}


