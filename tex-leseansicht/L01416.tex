%% latex-korrekturansicht-vorspann.tex
%% Vorspann für die Korrekturansicht.
%% Lädt die gemeinsame Datei latex-vorspann.tex mit gesetztem Schalter.

\newif\ifkorrekturansicht
\korrekturansichttrue

\input{../tex-inputs/latex-vorspann}


\section[Hugo von Hofmannsthal an Arthur Schnitzler, 10. 7. 1904]{L01416 Hugo von Hofmannsthal an Arthur Schnitzler, 10. 7. 1904}
\nopagebreak\mylabel{L01416v}
\rehead{ }\normalsize\beginnumbering\briefempfaengerindex{Schnitzler, Arthur@\textsc{Schnitzler, Arthur}!zzzHofmannsthal, Hugo von@\emph{von Hugo von Hofmannsthal}!1904-07-101@{10. 7. 1904}|(be}
\toendnotes[C]{\smallbreak\pagebreak[2]}\Standort{CUL, Schnitzler, B 43.}
\physDesc{Postkarte, 265 Zeichen
\newline{}Handschrift: schwarze Tinte, deutsche Kurrent
\newline{}Versand: 1) Stempel: »\nobreak{}\oindex{Rodaun@\textbf{Rodaun}, \emph{A.ADM4}|pwk}Rodaun, 10. 7. 04\nobreak{}«.   2) Stempel: »\nobreak{}\oindex{XVIII., Waehring@\textbf{XVIII., Währing}, \emph{A.ADM3}|pwk}18/1 Wien, 11. 7. 04, 8.V, Bestellt\nobreak{}«. 
\newline{}Schnitzler: mit Bleistift datiert: »11. 7 904« 
\newline{}Ordnung: 1) mit Bleistift von unbekannter Hand nummeriert: »\strikeout{237}«  2) mit Bleistift von unbekannter Hand nummeriert:
                                    »228«}
\buchAbdrucke{\weitereDrucke{Hugo von Hofmannsthal, Arthur Schnitzler: \emph{Briefwechsel}. Frankfurt am Main: \emph{S. Fischer} 1964, S. 191.} }\toendnotes[C]{\smallbreak}\pstart{}{\pb}\textsc{Herrn D\textsuperscript{r} Arthur Schnitzler}\pend{}\pstart{}\textsc{Wien}\oindex{Wien@\textbf{Wien}, \emph{A.ADM2}|pw}\pend{}\pstart{}\textsc{XVIII. Spöttelgasse 7}.\oindex{Edmund-Weiss-Gasse 7@\textbf{Edmund-Weiß-Gasse 7}, \emph{Wohngebäude (K.WHS)}|pw}\pend{}{\bigskip}\vspace{1em}
\pstart
           \noindent{}{\pb}Vielleicht »\textsc{\label{K_L01416-1v}\edtext{chasse libre}{\lemma{\textnormal{\emph{chasse libre}}}\Cendnote{\textnormal{französisch wörtlich: freie Jagd; 
                     es dürfte um die Suche nach einem passenden Titel für die französische Übersetzung von \emph{Freiwild}\pwindex{Freiwild. Schauspiel in 3 Akten@\emph{Freiwild. Schauspiel in 3 Akten}|pwk} gehen,
                     woran Stephan Epstein\pwindex{Epstein, Stephan 12.11.1866 – 1941@\textsc{Epstein, Stephan} (12.11.1866 – 1941), \emph{Schriftsteller/Schriftstellerin, Dramaturg/Dramaturgin, Übersetzer/Übersetzerin}|pwk} gerade arbeitete. In 
                     der im Nachlass Schnitzlers überlieferten Fassung (\emph{CUL}, A 245)
                     wird er als \emph{Le privilège. Trois actes}\pwindex{Le Privilege. Trois actes@\emph{Le Privilège. Trois actes}|pwk} angegeben, aber die Vorläufigkeit 
                     kenntlich gemacht: »\begin{otherlanguage}{french}titre provisoire\end{otherlanguage}«.}}}\label{K_L01416-1}}«, das giebt den Begriff treu wieder und klingt nicht ſchlecht.\hspace*{1.5em}Ich denke Dienstag oder
                  Mittwoch{ }abends zu \label{K_L01416-2v}\edtext{fahren}{\lemma{\textnormal{\emph{fahren}}}\Cendnote{\textnormal{Der genaue Abreisezeitpunkt konnte nicht
                  ermittelt werden. Vom 15. 7. 1904. bis zum 29. 7. 1904 war er als
                  erste Station seines Sommerurlaubs in Bad
                     Fusch\oindex{Bad Fusch@\textbf{Bad Fusch}, \emph{A.ADM3}|pwk}. Er und Schnitzler sahen sich
                  erst am 3. 9. 1904 wieder.}}}\label{K_L01416-2}.\pend
           
\pstart
           So ſehen wir uns wohl nicht wieder? Aber im Herbſt! Ich hoffe ſehr.\pend
           
\pstart
           Von Herzen{\\[\baselineskip]}\spacefill\mbox{Hugo.}\pend
           \leftskip=0em{}\selectlanguage{ngerman}\endnumbering\briefempfaengerindex{Schnitzler, Arthur@\textsc{Schnitzler, Arthur}!zzzHofmannsthal, Hugo von@\emph{von Hugo von Hofmannsthal}!1904-07-101@{10. 7. 1904}|)be}\mylabel{L01416h}  \normalsize

\doendnotes{C}
\bigskip
\vfill

\clearpage

\footnotesize

\lohead{\textsc{register}}

% Definiere theindex-Environment komplett neu ohne reledmac
\makeatletter
\renewenvironment{theindex}{%
  \section*{\indexname}%
  \setlength{\parindent}{0pt}%
  \setlength{\parskip}{0pt plus 0.3pt}%
  \let\item\@idxitem
}{%
  \clearpage
}
\makeatother

\IfFileExists{\jobname-pw.ind}{\input{\jobname-pw.ind}}{}

\end{document}

      