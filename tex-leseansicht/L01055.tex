%% latex-leseansicht-vorspann.tex
%% Vorspann für die Leseansicht.
%% Lädt die gemeinsame Datei latex-vorspann.tex mit nicht gesetztem Schalter.

\newif\ifkorrekturansicht
\korrekturansichtfalse

\input{../tex-inputs/latex-vorspann}


         
         \renewcommand{\erwaehntePersonen}{Personen: Richard Beer-Hofmann, Oskar Mayer, Hermine von Schaffgotsch}
         \renewcommand{\erwaehnteInstitutionen}{Institutionen: Wenzel Swatek}
         \renewcommand{\erwaehnteOrte}{Orte: Altaussee, Bad Aussee, Bad Ischl, Innsbruck, Klösterle, Levico Terme, Payerbach}
         \renewcommand{\erwaehnteWerke}{
               \section[Arthur Schnitzler an Richard Beer-Hofmann, 15. 7. 1900]{ Arthur Schnitzler an Richard Beer-Hofmann, 15. 7. 1900}\nopagebreak\mylabel{v}\rehead{ }\begin{ledgroupsized}[t]{13cm}\normalsize\beginnumbering \toendnotes[C]{\smallbreak\pagebreak[2]} \Standort{YCGL, MSS 31.}
\physDesc{Postkarte
\newline{}Handschrift: Bleistift, deutsche Kurrent\newline{}Versand: 1) Stempel: »\nobreak{}\oindex{Payerbach@\textbf{Payerbach}|pwk}Payerbach, 15. 7. \textcolor{gray}{00}, 7–12V\nobreak{}«.   2) Stempel: »\nobreak{}\oindex{Altaussee@\textbf{Altaussee}|pwk}Alt-Aussee, 16/7 00\nobreak{}«. \newline{}Ordnung: mit Bleistift von unbekannter Hand datiert:
                                 »15. 7.« }\buchAbdrucke{\weitereDrucke{Arthur Schnitzler, Richard Beer-Hofmann: \emph{Briefwechsel 1891–1931}. Hg. Konstanze Fliedl. Wien, Zürich: \emph{Europaverlag} 1992, S. 148–149.} }\pstart{}{\pb}Hrn \textsc{Dr. Richard
                     Beer-Hofmann}\pend{}\pstart{}\textsc{Altaussee\oindex{Altaussee@\textbf{Altaussee}|pw}}\pend{}{\bigskip}\pstart
           \noindent{}{\pb}lieber Richard, eben ko{\geminationm}t Ihr Brief, alſo nehm ich meine geſtrige Karte
               zurück. – M.\pwindex{Mayer, Oskar 1876 – 15.05.1915@\textsc{Mayer, Oskar} (1876 – 15.05.1915), \emph{Schriftsteller, Beamter}|pw} hat mir aus \textsc{Lev}.\oindex{Levico Terme@\textbf{Levico Terme}|pw} bereits vor 8 Tagen eine längere begeiſterte
               Sauce über B.\pwindex{Schaffgotsch, Hermine von 25.11.1871 – 25.11.1928@\textsc{Schaffgotsch, Hermine von} (25.11.1871 – 25.11.1928)|pw} geſchrieben. Es iſt wirklich
               ziemlich egal. Denken Sie doch nach\strikeout{,} wie wir von Klöſterle\oindex{Kloesterle@\textbf{Klösterle}|pw} oder wie das heißt weiter ko{\geminationm}en
               ſollen. – Vielleicht fahren wir zusa{\geminationm}en von Auſſee\oindex{Bad Aussee@\textbf{Bad Aussee}|pw}{ }\textsc{resp}. Iſchl\oindex{Bad Ischl@\textbf{Bad Ischl}|pw} nach Insbruck\oindex{Innsbruck@\textbf{Innsbruck}|pw} (über \textsc{Svatek}\orgindex{Wenzel Swatek@Wenzel Swatek|pw})\pend
           \pstart
           – Von der nächſten Zeit weiſs ich noch i{\geminationm}er nichts. (Es geht den meisten Menschen so.) –\pend
           \pstart
           Herzlichst{\\[\baselineskip]}Ihr\spacefill\mbox{A.}\pend
           \leftskip=0em{}
         
         \endnumbering\mylabel{h}\end{ledgroupsized}  \newcommand{\dateiname}{L01055}\newcommand{\titel}{Arthur Schnitzler an Richard Beer-Hofmann, 15. 7. 1900}\newcommand{\editorInnen}{Martin Anton Müller und Gerd-Hermann Susen}%% latex-leseansicht-abspann.tex
%% Abspann für die Leseansicht.
%% Der Schalter \ifkorrekturansicht ist bereits durch den Vorspann gesetzt.

%% latex-abspann.tex
%% Gemeinsamer Abspann für Korrekturansicht und Leseansicht.
%% Setzt den Schalter \ifkorrekturansicht voraus (gesetzt in den
%% einbindenden Dateien latex-korrekturansicht-abspann.tex bzw.
%% latex-leseansicht-abspann.tex).
%% ---------------------------------------------------------------

\normalsize

% Das esempio-Environment wird nur in der Leseansicht benötigt
\ifkorrekturansicht\else
\newenvironment{esempio}[3]%
{
    \vspace{1.5ex}
    \rlap{\underline{#1}}
    \par
    \setlength{\parindent}{0cm}
    \nopagebreak
    \leftskip=#2cm
    \rightskip=#3cm
}
{
    \par
}
\fi

\doendnotes{C}
\bigskip
\vfill

\clearpage

\footnotesize

\ifkorrekturansicht
  \lohead{\textsc{register}}
\fi

% theindex-Environment neu definieren ohne reledmac
\makeatletter
\renewenvironment{theindex}{%
  \ifkorrekturansicht
    \section*{\indexname}%
  \else
    \subsubsection*{Index der erwähnten Entitäten}%
  \fi
  \setlength{\parindent}{0pt}%
  \setlength{\parskip}{0pt plus 0.3pt}%
  \let\item\@idxitem
}{%
  \ifkorrekturansicht\clearpage\fi
}
\makeatother

\IfFileExists{\jobname-pw.ind}{\input{\jobname-pw.ind}}{}

% Quellenangabe nur in der Leseansicht
\ifkorrekturansicht\else
% Fallback-Definitionen, falls die .tex-Datei \titel etc. nicht gesetzt hat
\providecommand{\titel}{}
\providecommand{\editorInnen}{}
\providecommand{\dateiname}{\jobname}

\vspace{3cm}

\vfill

\footnotesize
\textsc{Quelle}: \titel. Herausgegeben von {\editorInnen}. In: \emph{Arthur Schnitzler: Briefwechsel mit Autorinnen und Autoren}.
 Digitale Edition, https://schnitzler-briefe.acdh.oeaw.ac.at/{\dateiname}.html (Stand \today)
\fi

\end{document}


      