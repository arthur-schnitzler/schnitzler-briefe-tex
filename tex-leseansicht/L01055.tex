%% latex-korrekturansicht-vorspann.tex
%% Vorspann für die Korrekturansicht.
%% Lädt die gemeinsame Datei latex-vorspann.tex mit gesetztem Schalter.

\newif\ifkorrekturansicht
\korrekturansichttrue

\input{../tex-inputs/latex-vorspann}


\section[Arthur Schnitzler an Richard Beer-Hofmann, 15. 7. 1900]{L01055 Arthur Schnitzler an Richard Beer-Hofmann, 15. 7. 1900}
\nopagebreak\mylabel{L01055v}
\rehead{ }\normalsize\beginnumbering\briefempfaengerindex{Beer-Hofmann, Richard@\textsc{Beer-Hofmann, Richard}!zzzSchnitzler, Arthur@\emph{von Arthur Schnitzler}!1900-07-151@{15. 7. 1900}|(be}
\toendnotes[C]{\smallbreak\pagebreak[2]}\Standort{YCGL, MSS 31.}
\physDesc{Postkarte, 491 Zeichen
\newline{}Handschrift: Bleistift, deutsche Kurrent
\newline{}Versand: 1) Stempel: »\nobreak{}\oindex{Payerbach@\textbf{Payerbach}, \emph{P.PPLA3}|pwk}Payerbach, 15. 7. \textcolor{gray}{00}, 7–12V\nobreak{}«.   2) Stempel: »\nobreak{}\oindex{Altaussee@\textbf{Altaussee}, \emph{A.ADM3}|pwk}Alt-Aussee, 16/7 00\nobreak{}«. 
\newline{}Ordnung: mit Bleistift von unbekannter Hand datiert: »15. 7.« }
\buchAbdrucke{\weitereDrucke{Arthur Schnitzler, Richard Beer-Hofmann: \emph{Briefwechsel 1891–1931}. Wien, Zürich: \emph{Europaverlag} 1992, S. 148–149.} }\toendnotes[C]{\smallbreak}\pstart{}{\pb}Hrn \textsc{Dr. Richard
                     Beer-Hofmann}\pend{}\pstart{}\textsc{Altaussee\oindex{Altaussee@\textbf{Altaussee}, \emph{A.ADM3}|pw}}\pend{}{\bigskip}\vspace{1em}
\pstart
           \noindent{}{\pb}lieber Richard, eben ko{\geminationm}t Ihr \label{K_L01055-1v}\edtext{Brief}{\lemma{\textnormal{\emph{Brief}}}\Cendnote{\textnormal{Richard Beer-Hofmann an Arthur Schnitzler, 13. 7. 1900.
               }}}\label{K_L01055-1}, alſo nehm ich meine geſtrige \label{K_L01055-2v}\edtext{Karte}{\lemma{\textnormal{\emph{Karte}}}\Cendnote{\textnormal{Arthur Schnitzler an Richard Beer-Hofmann, 14. 7. 1900.
               }}}\label{K_L01055-2}
               zurück. – M.\pwindex{Mayer, Oskar 1876 – 15.05.1915@\textsc{Mayer, Oskar} (1876 – 15.05.1915), \emph{Schriftsteller/Schriftstellerin, Beamter/Beamte}|pw} hat mir aus \textsc{Lev}.\oindex{Levico Terme@\textbf{Levico Terme}, \emph{Besiedelter Ort (A.BSO)}|pw} bereits vor 8 Tagen eine längere begeiſterte
               Sauce über B.\pwindex{Schaffgotsch, Hermine von 25.11.1871 – 25.11.1928@\textsc{Schaffgotsch, Hermine von} (25.11.1871 – 25.11.1928)|pw} geſchrieben. Es iſt wirklich
               ziemlich egal. Denken Sie doch nach\strikeout{,} wie wir von Klöſterle\oindex{Kloesterle@\textbf{Klösterle}, \emph{A.ADM3}|pw} oder wie das heißt weiter ko{\geminationm}en ſollen. – Vielleicht fahren wir zusa{\geminationm}en von Auſſee\oindex{Bad Aussee@\textbf{Bad Aussee}, \emph{P.PPLA3}|pw}{ }\textsc{resp}. Iſchl\oindex{Bad Ischl@\textbf{Bad Ischl}, \emph{P.PPL}|pw} nach Insbruck\oindex{Innsbruck@\textbf{Innsbruck}, \emph{A.ADM2}|pw} (über \textsc{Svatek}\orgindex{Wenzel Swatek@Wenzel Swatek|pw})\pend
           
\pstart
           – Von der nächſten Zeit weiſs ich noch i{\geminationm}er nichts. (Es
               geht den meisten Menschen so.) –\pend
           
\pstart
           Herzlichst{\\[\baselineskip]}Ihr\spacefill\mbox{A.}\pend
           \leftskip=0em{}\selectlanguage{ngerman}\endnumbering\briefempfaengerindex{Beer-Hofmann, Richard@\textsc{Beer-Hofmann, Richard}!zzzSchnitzler, Arthur@\emph{von Arthur Schnitzler}!1900-07-151@{15. 7. 1900}|)be}\mylabel{L01055h}  \normalsize

\doendnotes{C}
\bigskip
\vfill

\clearpage

\footnotesize

\lohead{\textsc{register}}

% Definiere theindex-Environment komplett neu ohne reledmac
\makeatletter
\renewenvironment{theindex}{%
  \section*{\indexname}%
  \setlength{\parindent}{0pt}%
  \setlength{\parskip}{0pt plus 0.3pt}%
  \let\item\@idxitem
}{%
  \clearpage
}
\makeatother

\IfFileExists{\jobname-pw.ind}{\input{\jobname-pw.ind}}{}

\end{document}

      