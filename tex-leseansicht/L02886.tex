%% latex-korrekturansicht-vorspann.tex
%% Vorspann für die Korrekturansicht.
%% Lädt die gemeinsame Datei latex-vorspann.tex mit gesetztem Schalter.

\newif\ifkorrekturansicht
\korrekturansichttrue

\input{../tex-inputs/latex-vorspann}


\section[ Paul Goldmann an Arthur Schnitzler, 12. 9. {[}1899{]}]{L02886 Paul Goldmann an Arthur Schnitzler, 12. 9. {[}1899{]}}
\nopagebreak\mylabel{L02886v}
\rehead{ }\normalsize\beginnumbering\briefempfaengerindex{Schnitzler, Arthur@\textsc{Schnitzler, Arthur}!zzzGoldmann, Paul@\emph{von Paul Goldmann}!1899-09-122@{12. 9. {[}1899{]}}|(be}
\toendnotes[C]{\smallbreak\pagebreak[2]}\Standort{DLA, A:Schnitzler, HS.NZ85.1.3169.}
\physDesc{Brief, 1 Blatt, 2 Seiten, 728 Zeichen
\newline{}Handschrift: schwarze Tinte, deutsche Kurrent
\newline{}Schnitzler: mit Bleistift das Jahr »99« vermerkt }\toendnotes[C]{\smallbreak}
\pstart
           \raggedleft{}{\pb}\textsc{Paris\oindex{Paris@\textbf{Paris}, \emph{P.PPLC}|pw}}, 12. September.\pend
           
\pstart\center{}Liebſter Freund,\pend\vspace{0.5em}
\pstart
           Ich bekomme Deinen lieben Brief erſt heut, Dienſtag,
               in \textsc{Paris\oindex{Paris@\textbf{Paris}, \emph{P.PPLC}|pw}}. Hoffentlich erreicht Dich meine Antwort noch Donnerſtag in \label{K_L02886-1v}\edtext{\textsc{Muenchen\oindex{Muenchen@\textbf{München}, \emph{P.PPLA}|pw}}}{\lemma{\textnormal{\emph{Muenchen}}}\Cendnote{\textnormal{Schnitzler war seit 12. 9. 1899 in München\oindex{Muenchen@\textbf{München}, \emph{P.PPLA}|pwk}. Am 16. 9. 1899 reiste er nach Nürnberg\oindex{Nuernberg@\textbf{Nürnberg}, \emph{P.PPL}|pwk} weiter.}}}\label{K_L02886-1}. Ich habe auch hier noch immer
               raſend zu thun und kann Dir daher nur einen Gruß in aller Eile ſchicken. \strikeout{Wi} Wie es mit meinem Urlaub wird und mit der Reiſe nach
                  Florenz\oindex{Florenz@\textbf{Florenz}, \emph{P.PPLA}|pw}, erfahre ich in Frankfurt\oindex{Frankfurt am Main@\textbf{Frankfurt am Main}, \emph{P.PPLA3}|pw}, wo ich Ende der Woche eintreffe. Es wäre entzückend,
               wenn Du in nächſter \strikeout{Wo} Woche auch \label{K_L02886-2v}\edtext{hinkämeſt}{\lemma{\textnormal{\emph{hinkämeſt}}}\Cendnote{\textnormal{Schnitzler war vom 19. 9. 1899 bis zum 24. 9. 1899 in Frankfurt am Main\oindex{Frankfurt am Main@\textbf{Frankfurt am Main}, \emph{P.PPLA3}|pwk}.}}}\label{K_L02886-2}. Von München\oindex{Muenchen@\textbf{München}, \emph{P.PPLA}|pw} iſts ja nicht allzuweit. {\pb}Jedenfalls theile mir ſofort nach \uline{Frankfurt\oindex{Frankfurt am Main@\textbf{Frankfurt am Main}, \emph{P.PPLA3}|pw}} Deine weitere Adreſſe mit, damit ich \strikeout{Dir} mit
               Dir die erforderlichen Verabredungen treffen kann.\pend
           
\pstart
           Viele treue Grüße! {\\[\baselineskip]}Dein treuer {\\[\baselineskip]}\spacefill\mbox{Paul Goldmann}\pend
           \leftskip=0em{}
\pstart
           \noindent{}In München\oindex{Muenchen@\textbf{München}, \emph{P.PPLA}|pw} findeſt Du hoffentlich
                  Zerſtreuung und einige frohe Stunden.\pend
           \selectlanguage{ngerman}\endnumbering\briefempfaengerindex{Schnitzler, Arthur@\textsc{Schnitzler, Arthur}!zzzGoldmann, Paul@\emph{von Paul Goldmann}!1899-09-122@{12. 9. {[}1899{]}}|)be}\mylabel{L02886h}  \normalsize

\doendnotes{C}
\bigskip
\vfill

\clearpage

\footnotesize

\lohead{\textsc{register}}

% Definiere theindex-Environment komplett neu ohne reledmac
\makeatletter
\renewenvironment{theindex}{%
  \section*{\indexname}%
  \setlength{\parindent}{0pt}%
  \setlength{\parskip}{0pt plus 0.3pt}%
  \let\item\@idxitem
}{%
  \clearpage
}
\makeatother

\IfFileExists{\jobname-pw.ind}{\input{\jobname-pw.ind}}{}

\end{document}

      