%% latex-leseansicht-vorspann.tex
%% Vorspann für die Leseansicht.
%% Lädt die gemeinsame Datei latex-vorspann.tex mit nicht gesetztem Schalter.

\newif\ifkorrekturansicht
\korrekturansichtfalse

\input{../tex-inputs/latex-vorspann}


         \renewcommand{\erwaehnteOrte}{Orte: Florenz, Frankfurt am Main, München, Nürnberg, Paris}
         \renewcommand{\erwaehnteWerke}{}
               \section[ Paul Goldmann an Arthur Schnitzler, 12. 9. {[}1899{]}]{ Paul Goldmann an Arthur Schnitzler, 12. 9. {[}1899{]}}\nopagebreak\mylabel{v}\rehead{ }\begin{ledgroupsized}[t]{13cm}\normalsize\beginnumbering \toendnotes[C]{\smallbreak\pagebreak[2]} \Standort{DLA, A:Schnitzler, HS.NZ85.1.3169.}
\physDesc{Brief, 1 Blatt, 2 Seiten
\newline{}Handschrift: schwarze Tinte, deutsche Kurrent
\newline{}Schnitzler: mit Bleistift das Jahr »99« vermerkt }\toendnotes[C]{\smallbreak}\pstart
           \raggedleft{}{\pb}\textsc{Paris\oindex{Paris@\textbf{Paris}|pw}}, 12. September.\pend
           \pstart\center{}Liebſter Freund,\pend\pstart
           Ich bekomme Deinen lieben Brief erſt heut, Dienſtag,
               in \textsc{Paris\oindex{Paris@\textbf{Paris}|pw}}. Hoffentlich erreicht Dich meine Antwort noch Donnerſtag in \label{K_L02886-1v}\edtext{\textsc{Muenchen\oindex{Muenchen@\textbf{München}|pw}}}{\lemma{\textnormal{\emph{Muenchen}}}\Cendnote{\textnormal{Schnitzler\pwindex{Schnitzler, Arthur 15.05.1862 – 21.10.1931@\textsc{Schnitzler, Arthur} (15.05.1862 – 21.10.1931), \emph{Schriftsteller, Mediziner}|pwk} war seit 12. 9. 1899 in München\oindex{Muenchen@\textbf{München}|pwk}. Am 16. 9. 1899 reiste er nach Nürnberg\oindex{Nuernberg@\textbf{Nürnberg}|pwk} weiter.}}}\label{K_L02886-1h}. Ich habe auch hier noch immer
               raſend zu thun und kann Dir daher nur einen Gruß in aller Eile ſchicken. \strikeout{Wi} Wie es mit meinem Urlaub wird und mit der Reiſe nach
                  Florenz\oindex{Florenz@\textbf{Florenz}|pw}, erfahre ich in Frankfurt\oindex{Frankfurt am Main@\textbf{Frankfurt am Main}|pw}, wo ich Ende der Woche eintreffe. Es wäre entzückend,
               wenn Du in nächſter \strikeout{Wo} Woche auch \label{K_L02886-2v}\edtext{hinkämeſt}{\lemma{\textnormal{\emph{hinkämeſt}}}\Cendnote{\textnormal{Schnitzler\pwindex{Schnitzler, Arthur 15.05.1862 – 21.10.1931@\textsc{Schnitzler, Arthur} (15.05.1862 – 21.10.1931), \emph{Schriftsteller, Mediziner}|pwk} war von 19. 9. 1899 bis 24. 9. 1899 in Frankfurt am Main\oindex{Frankfurt am Main@\textbf{Frankfurt am Main}|pwk}.}}}\label{K_L02886-2h}. Von München\oindex{Muenchen@\textbf{München}|pw} iſts ja nicht allzuweit. {\pb}Jedenfalls theile mir ſofort nach \uline{Frankfurt\oindex{Frankfurt am Main@\textbf{Frankfurt am Main}|pw}} Deine weitere Adreſſe mit, damit ich \strikeout{Dir} mit
               Dir die erforderlichen Verabredungen treffen kann.\pend
           \pstart
           Viele treue Grüße! {\\[\baselineskip]}Dein treuer {\\[\baselineskip]}\spacefill\mbox{Paul Goldmann}\pend
           \leftskip=0em{}\pstart
           \noindent{}In München\oindex{Muenchen@\textbf{München}|pw} findeſt Du hoffentlich
                  Zerſtreuung und einige frohe Stunden.\pend
           
         
         \endnumbering\mylabel{h}\end{ledgroupsized}  \newcommand{\dateiname}{L02886}\newcommand{\titel}{Paul Goldmann an Arthur Schnitzler, 12. 9. [1899]}\newcommand{\editorInnen}{Martin Anton Müller und Laura Untner}%% latex-leseansicht-abspann.tex
%% Abspann für die Leseansicht.
%% Der Schalter \ifkorrekturansicht ist bereits durch den Vorspann gesetzt.

%% latex-abspann.tex
%% Gemeinsamer Abspann für Korrekturansicht und Leseansicht.
%% Setzt den Schalter \ifkorrekturansicht voraus (gesetzt in den
%% einbindenden Dateien latex-korrekturansicht-abspann.tex bzw.
%% latex-leseansicht-abspann.tex).
%% ---------------------------------------------------------------

\normalsize

% Das esempio-Environment wird nur in der Leseansicht benötigt
\ifkorrekturansicht\else
\newenvironment{esempio}[3]%
{
    \vspace{1.5ex}
    \rlap{\underline{#1}}
    \par
    \setlength{\parindent}{0cm}
    \nopagebreak
    \leftskip=#2cm
    \rightskip=#3cm
}
{
    \par
}
\fi

\doendnotes{C}
\bigskip
\vfill

\clearpage

\footnotesize

\ifkorrekturansicht
  \lohead{\textsc{register}}
\fi

% theindex-Environment neu definieren ohne reledmac
\makeatletter
\renewenvironment{theindex}{%
  \ifkorrekturansicht
    \section*{\indexname}%
  \else
    \subsubsection*{Index der erwähnten Entitäten}%
  \fi
  \setlength{\parindent}{0pt}%
  \setlength{\parskip}{0pt plus 0.3pt}%
  \let\item\@idxitem
}{%
  \ifkorrekturansicht\clearpage\fi
}
\makeatother

\IfFileExists{\jobname-pw.ind}{\input{\jobname-pw.ind}}{}

% Quellenangabe nur in der Leseansicht
\ifkorrekturansicht\else
% Fallback-Definitionen, falls die .tex-Datei \titel etc. nicht gesetzt hat
\providecommand{\titel}{}
\providecommand{\editorInnen}{}
\providecommand{\dateiname}{\jobname}

\vspace{3cm}

\vfill

\footnotesize
\textsc{Quelle}: \titel. Herausgegeben von {\editorInnen}. In: \emph{Arthur Schnitzler: Briefwechsel mit Autorinnen und Autoren}.
 Digitale Edition, https://schnitzler-briefe.acdh.oeaw.ac.at/{\dateiname}.html (Stand \today)
\fi

\end{document}


      