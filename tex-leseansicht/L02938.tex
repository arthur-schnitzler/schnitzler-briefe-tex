%% latex-korrekturansicht-vorspann.tex
%% Vorspann für die Korrekturansicht.
%% Lädt die gemeinsame Datei latex-vorspann.tex mit gesetztem Schalter.

\newif\ifkorrekturansicht
\korrekturansichttrue

\input{../tex-inputs/latex-vorspann}


\section[ Paul Goldmann an Arthur Schnitzler, 12. 11. {[}1900{]}]{L02938 Paul Goldmann an Arthur Schnitzler, 12. 11. {[}1900{]}}
\nopagebreak\mylabel{L02938v}
\rehead{ }\normalsize\beginnumbering\briefempfaengerindex{Schnitzler, Arthur@\textsc{Schnitzler, Arthur}!zzzGoldmann, Paul@\emph{von Paul Goldmann}!1900-11-121@{12. 11. {[}1900{]}}|(be}
\toendnotes[C]{\smallbreak\pagebreak[2]}\Standort{DLA, A:Schnitzler, HS.NZ85.1.3170.}
\physDesc{Brief, 1 Blatt, 2 Seiten, 557 Zeichen
\newline{}Handschrift: blaue Tinte, deutsche Kurrent
\newline{}Schnitzler: mit Bleistift das Jahr »90\textcolor{gray}{0}« vermerkt }\toendnotes[C]{\smallbreak}
\pstart
           {\pb}Berlin\oindex{Berlin@\textbf{Berlin}, \emph{P.PPLC}|pw}, 12. November.\hfill \textcolor{gray}{\textbf{DESSAUERSTRASSE 19}}\oindex{Dessauer Strasse@\textbf{Dessauer Straße}, \emph{Straße (K.STR)}|pw}\pend
           
\pstart\center{}Mein lieber Freund,\pend\vspace{0.5em}
\pstart
           Ich will Dir nur in aller Eile Glück zur \label{K_L02938-1v}\edtext{Reiſe}{\lemma{\textnormal{\emph{Reiſe}}}\Cendnote{\textnormal{Schnitzler hielt sich vom 22. 11. 1900 bis zum 24. 11. 1900 und vom
                     29. 11. 1900 bis zum
                     2. 12. 1900 in
                     Breslau\oindex{Breslau@\textbf{Breslau}, \emph{P.PPLA}|pwk} auf.}}}\label{K_L02938-1} wünſchen. Es iſt
               wirklich ſehr beklagenswerth, daß ich nicht nach Breslau\oindex{Breslau@\textbf{Breslau}, \emph{P.PPLA}|pw} kommen kann. Wo wirſt Du in Breslau\oindex{Breslau@\textbf{Breslau}, \emph{P.PPLA}|pw} wohnen? Willſt Du ſo lieb ſein, mir am \label{K_L02938-2v}\edtext{Tage nach der \begin{otherlanguage}{french}\textsc{Première\pwindex{Schleier der Beatrice. Schauspiel in fuenf Akten@\emph{Der Schleier der Beatrice. Schauspiel in fünf Akten}|pwv}}\end{otherlanguage}}{\lemma{\textnormal{\emph{Tage nach der Première}}}\Cendnote{\textnormal{Anfänglich war die Uraufführung von \emph{Der Schleier der Beatrice}\pwindex{Schleier der Beatrice. Schauspiel in fuenf Akten@\emph{Der Schleier der Beatrice. Schauspiel in fünf Akten}|pwk} für den 17. 11. 1900 geplant. Sie wurde jedoch auf den 1. 12. 1900
                  verschoben.}}}\label{K_L02938-2} ein Wort zu telegraphiren?\pend
           
\pstart
           Die N. Fr. Pr.\orgindex{Neue Freie Presse@Neue Freie Presse|pw} hat meinen Vorſchlag, das \label{K_L02938-3v}\edtext{Referat}{\lemma{\textnormal{\emph{Referat}}}\Cendnote{\textnormal{Siehe Paul Goldmann an Arthur Schnitzler, 30. 10. [1900] und 3. 12. [1900].
               }}}\label{K_L02938-3} dem \textsc{Dr.}{ }{\pb}\textsc{Erich Freund\pwindex{Freund, Erich 1866-08-13 – 1940@\textsc{Freund, Erich} (1866-08-13 – 1940), \emph{Kritiker/Kritikerin, Musikjournalist/Musikjournalistin}|pw}} zu übertragen, angenommen. So wird wenigſtens ein anſtändiger Menſch über Dich\pwindex{Schleier der Beatrice. Schauspiel in fuenf Akten@\emph{Der Schleier der Beatrice. Schauspiel in fünf Akten}|pwv} berichten. Das iſt
               einſtweilen Alles, was ich thun konnte.\pend
           
\pstart
           Auf frohes Wiederſehn in Berlin\oindex{Berlin@\textbf{Berlin}, \emph{P.PPLC}|pw}!\pend
           
\pstart
           Sei von Herzen gegrüßt von Deinem {\\[\baselineskip]}treuen {\\[\baselineskip]}\spacefill\mbox{Paul Goldmann.}\pend
           \leftskip=0em{}\selectlanguage{ngerman}\endnumbering\briefempfaengerindex{Schnitzler, Arthur@\textsc{Schnitzler, Arthur}!zzzGoldmann, Paul@\emph{von Paul Goldmann}!1900-11-121@{12. 11. {[}1900{]}}|)be}\mylabel{L02938h}  \normalsize

\doendnotes{C}
\bigskip
\vfill

\clearpage

\footnotesize

\lohead{\textsc{register}}

% Definiere theindex-Environment komplett neu ohne reledmac
\makeatletter
\renewenvironment{theindex}{%
  \section*{\indexname}%
  \setlength{\parindent}{0pt}%
  \setlength{\parskip}{0pt plus 0.3pt}%
  \let\item\@idxitem
}{%
  \clearpage
}
\makeatother

\IfFileExists{\jobname-pw.ind}{\input{\jobname-pw.ind}}{}

\end{document}

      