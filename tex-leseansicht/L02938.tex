%% latex-leseansicht-vorspann.tex
%% Vorspann für die Leseansicht.
%% Lädt die gemeinsame Datei latex-vorspann.tex mit nicht gesetztem Schalter.

\newif\ifkorrekturansicht
\korrekturansichtfalse

\input{../tex-inputs/latex-vorspann}


\section[ Paul Goldmann an Arthur Schnitzler, 12. 11. {[}1900{]}]{L02938 Paul Goldmann an Arthur Schnitzler,  12. 11. [1900]}
\nopagebreak\mylabel{L02938v}
\rehead{ }\normalsize\beginnumbering\briefempfaengerindex{Schnitzler, Arthur@\textsc{Schnitzler, Arthur}!zzzGoldmann, Paul@\emph{von Paul Goldmann}!1900-11-121@{12. 11. [1900]}|(be}
\toendnotes[C]{\smallbreak\pagebreak[2]}
\correspDesc{Versand  durch Paul Goldmann am 12. 11. [1900] in Berlin
\newline{}Erhalt  durch Arthur Schnitzler im Zeitraum [13. 11. 1900 – 17. 11. 1900?] in Wien}\toendnotes[C]{\smallbreak}
\Standort{DLA, A:Schnitzler, HS.NZ85.1.3170.}
\physDesc{Brief, 1 Blatt, 2 Seiten, 557 Zeichen
\newline{}Handschrift: blaue Tinte, deutsche Kurrent
\newline{}Schnitzler: mit Bleistift das Jahr »90\textcolor{gray}{0}« vermerkt }\toendnotes[C]{\smallbreak}
\pstart
           {\pb}Berlin\oindex{Berlin@\textbf{Berlin}, \emph{Hauptstadt}|pw}, 12. November.\hfill \textcolor{gray}{\textbf{DESSAUERSTRASSE 19}}\oindex{Dessauer Straße@\textbf{Dessauer Straße}, \emph{Straße}|pw}\pend
           
\pstart\center{}Mein lieber Freund,\pend\vspace{0.5em}
\pstart
           Ich will Dir nur in aller Eile Glück zur \label{K_L02938-1v}\edtext{Reiſe}{\lemma{\textnormal{\emph{Reise}}}\Cendnote{\textnormal{Schnitzler hielt sich vom 22. 11. 1900 bis zum 24. 11. 1900 und vom
                     29. 11. 1900 bis zum
                     2. 12. 1900 in
                     Breslau\oindex{Breslau@\textbf{Breslau}|pwk} auf.}}}\label{K_L02938-1} wünſchen. Es iſt
               wirklich{ }ſehr beklagenswerth, daß ich nicht nach Breslau\oindex{Breslau@\textbf{Breslau}|pw} kommen kann. Wo wirſt Du in Breslau\oindex{Breslau@\textbf{Breslau}|pw} wohnen? Willſt Du{ }ſo lieb{ }ſein, mir am \label{K_L02938-2v}\edtext{Tage nach der \begin{otherlanguage}{french}\textsc{Première\eventindex{Lobe-Theater@\textbf{Lobe-Theater}!Uraufführung von Der Schleier der Beatrice, 1.12.1900@Uraufführung von Der Schleier der Beatrice, 1.12.1900|pwv}\pwindex{Schnitzler, Arthur 15.\,5.\,1862 Wien – 21.\,10.\,1931 ebd.@\textsc{Schnitzler, Arthur} (15.\,5.\,1862 Wien – 21.\,10.\,1931 ebd.), \emph{Schriftsteller, Mediziner}!Schleier der Beatrice. Schauspiel in fünf Akten@\strich\emph{Der Schleier der Beatrice. Schauspiel in fünf Akten}|pwv}}\end{otherlanguage}}{\lemma{\textnormal{\emph{Tage nach der Première}}}\Cendnote{\textnormal{Anfänglich war die Uraufführung von \emph{Der Schleier der Beatrice}\pwindex{Schnitzler, Arthur 15.\,5.\,1862 Wien – 21.\,10.\,1931 ebd.@\textsc{Schnitzler, Arthur} (15.\,5.\,1862 Wien – 21.\,10.\,1931 ebd.), \emph{Schriftsteller, Mediziner}!Schleier der Beatrice. Schauspiel in fünf Akten@\strich\emph{Der Schleier der Beatrice. Schauspiel in fünf Akten}|pwk}\eventindex{Lobe-Theater@\textbf{Lobe-Theater}!Uraufführung von Der Schleier der Beatrice, 1.12.1900@Uraufführung von Der Schleier der Beatrice, 1.12.1900|pwk} für den 17. 11. 1900 geplant. Sie wurde jedoch auf den 1. 12. 1900
                  verschoben.}}}\label{K_L02938-2} ein Wort zu telegraphiren?\pend
           
\pstart
           Die N. Fr. Pr.\orgindex{Neue Freie Presse@Neue Freie Presse|pw} hat meinen Vorſchlag, das \label{K_L02938-3v}\edtext{Referat}{\lemma{\textnormal{\emph{Referat}}}\Cendnote{\textnormal{Siehe XXXX Auszeichnungsfehler: Dokument L02937 nicht gefunden und XXXX Auszeichnungsfehler: Dokument L02943 nicht gefunden.
               }}}\label{K_L02938-3} dem \textsc{Dr.}{ }{\pb}\textsc{Erich Freund\pwindex{Freund, Erich 13.\,8.\,1866 Breslau – 1940 Berlin@\textsc{Freund, Erich} (13.\,8.\,1866 Breslau – 1940 Berlin), \emph{Kritiker, Musikjournalist}|pw}} zu übertragen, angenommen. So wird wenigſtens ein anſtändiger Menſch über Dich\pwindex{Schnitzler, Arthur 15.\,5.\,1862 Wien – 21.\,10.\,1931 ebd.@\textsc{Schnitzler, Arthur} (15.\,5.\,1862 Wien – 21.\,10.\,1931 ebd.), \emph{Schriftsteller, Mediziner}!Schleier der Beatrice. Schauspiel in fünf Akten@\strich\emph{Der Schleier der Beatrice. Schauspiel in fünf Akten}|pwv} berichten. Das iſt
               einſtweilen Alles, was ich thun konnte.\pend
           
\pstart
           Auf frohes Wiederſehn in Berlin\oindex{Berlin@\textbf{Berlin}, \emph{Hauptstadt}|pw}!\pend
           
\pstart
           Sei von Herzen gegrüßt von Deinem {\\[\baselineskip]}treuen {\\[\baselineskip]}\spacefill\mbox{Paul Goldmann.}\pend
           \leftskip=0em{}\selectlanguage{ngerman}\endnumbering\briefempfaengerindex{Schnitzler, Arthur@\textsc{Schnitzler, Arthur}!zzzGoldmann, Paul@\emph{von Paul Goldmann}!1900-11-121@{12. 11. [1900]}|)be}\mylabel{L02938h}  \newcommand{\dateiname}{L02938}\newcommand{\titel}{Paul Goldmann an Arthur Schnitzler, 12. 11. [1900]}\newcommand{\editorInnen}{Martin Anton Müller und Laura Untner}%% latex-leseansicht-abspann.tex
%% Abspann für die Leseansicht.
%% Der Schalter \ifkorrekturansicht ist bereits durch den Vorspann gesetzt.

%% latex-abspann.tex
%% Gemeinsamer Abspann für Korrekturansicht und Leseansicht.
%% Setzt den Schalter \ifkorrekturansicht voraus (gesetzt in den
%% einbindenden Dateien latex-korrekturansicht-abspann.tex bzw.
%% latex-leseansicht-abspann.tex).
%% ---------------------------------------------------------------

\normalsize

% Das esempio-Environment wird nur in der Leseansicht benötigt
\ifkorrekturansicht\else
\newenvironment{esempio}[3]%
{
    \vspace{1.5ex}
    \rlap{\underline{#1}}
    \par
    \setlength{\parindent}{0cm}
    \nopagebreak
    \leftskip=#2cm
    \rightskip=#3cm
}
{
    \par
}
\fi

\doendnotes{C}
\bigskip
\vfill

\clearpage

\footnotesize

\ifkorrekturansicht
  \lohead{\textsc{register}}
\fi

% theindex-Environment neu definieren ohne reledmac
\makeatletter
\renewenvironment{theindex}{%
  \ifkorrekturansicht
    \section*{\indexname}%
  \else
    \subsubsection*{Index der erwähnten Entitäten}%
  \fi
  \setlength{\parindent}{0pt}%
  \setlength{\parskip}{0pt plus 0.3pt}%
  \let\item\@idxitem
}{%
  \ifkorrekturansicht\clearpage\fi
}
\makeatother

\IfFileExists{\jobname-pw.ind}{\input{\jobname-pw.ind}}{}

% Quellenangabe nur in der Leseansicht
\ifkorrekturansicht\else
% Fallback-Definitionen, falls die .tex-Datei \titel etc. nicht gesetzt hat
\providecommand{\titel}{}
\providecommand{\editorInnen}{}
\providecommand{\dateiname}{\jobname}

\vspace{3cm}

\vfill

\footnotesize
\textsc{Quelle}: \titel. Herausgegeben von {\editorInnen}. In: \emph{Arthur Schnitzler: Briefwechsel mit Autorinnen und Autoren}.
 Digitale Edition, https://schnitzler-briefe.acdh.oeaw.ac.at/{\dateiname}.html (Stand \today)
\fi

\end{document}


