\input{../tex-inputs/latex-pdf-vorspann}
\begin{center}
            \textcolor{red}{ENTWURF. ENTZIFFERUNG NOCH NICHT KORREKTURGELESEN}
                      \end{center}
            
               \section[Richard Beer-Hofmann an Arthur Schnitzler, 1. 10. 1899]{ Richard Beer-Hofmann an Arthur Schnitzler, 1. 10. 1899}\nopagebreak\mylabel{v}\rehead{ }\begin{ledgroupsized}[t]{13cm}\normalsize\beginnumbering\briefempfaengerindex{Schnitzler, Arthur@\textsc{Schnitzler, Arthur}!zzzBeer-Hofmann, Richard@\emph{von Richard Beer-Hofmann}!1899-10-011@{1. 10. 1899}|(be} \toendnotes[C]{\smallbreak\pagebreak[2]} \Standort{CUL, Schnitzler, B 8.}
\physDesc{Bildpostkarte
\newline{}Handschrift: schwarze Tinte, lateinische Kurrent\newline{}Versand: 1) Stempel: »\nobreak{}\oindex{Sankt Michael@\textbf{Sankt Michael}|pwk}St. Michael in Eppan, 2 10 99\nobreak{}«.  2) Stempel: »\nobreak{}\oindex{Wiesbaden@\textbf{Wiesbaden}|pwk}Wiesbaden, 3. 10. 99, 9–10V\nobreak{}«. \newline{}Ordnung: mit Bleistift von unbekannter Hand nummeriert: »142« }\buchAbdrucke{\weitereDrucke{Arthur Schnitzler, Richard Beer-Hofmann: \emph{Briefwechsel 1891–1931}. Hg. Konstanze Fliedl. Wien, Zürich: \emph{Europaverlag} 1992, S. 138.} }\toendnotes[C]{\smallbreak}\pstart{}{\pb}D\textsuperscript{r} Arthur Schnitzler\pend{}\pstart{}Wiesbaden\oindex{Wiesbaden@\textbf{Wiesbaden}|pw}\pend{}\pstart{}Park-Hôtel\oindex{Hôtel du Parc {\kaufmannsund} Bristol@\textbf{Hôtel du Parc {\kaufmannsund} Bristol}|pw}\pend{}{\bigskip}\pstart
           \noindent{}\centering{}\textcolor{gray}{\textbf{{\pb}Hotel und
                  Pension Eppaner Hof\oindex{Eppaner Hof@\textbf{Eppaner Hof}|pw} in
                  Eppan\oindex{Eppan an der Weinstrasse@\textbf{Eppan an der Weinstraße}|pw}.}}\pend
           \pstart
           \raggedleft{}1/\strikeout{I}X 1899\pend
           \pstart
           Die × Fenster bewohnen wir\pwindex{Beer-Hofmann, Paula 25.02.1879 – 30.10.1939@\textsc{Beer-Hofmann, Paula} (25.02.1879 – 30.10.1939)|pwv}\pwindex{Beer-Hofmann, Mirjam 04.09.1897 – 24.12.1984@\textsc{Beer-Hofmann, Mirjam} (04.09.1897 – 24.12.1984)|pwv}\pwindex{Beer-Hofmann, Naemah 20.12.1898 – 10.11.1971@\textsc{Beer-Hofmann, Naëmah} (20.12.1898 – 10.11.1971)|pwv}. Die
               zwei rechts, ich. (Historisch).\pend
           \pstart
           Ich bin leider schon beim 420\textsuperscript{ten} Vers\pwindex{Beer-Hofmann, Richard 11.07.1866 – 26.09.1945@\textsc{Beer-Hofmann, Richard} (11.07.1866 – 26.09.1945), \emph{Schriftsteller}!Graf von Charolais. Ein Trauerspiel1904-12-23 – 1904-12-23@\strich\emph{Der Graf von Charolais. Ein Trauerspiel} {[}1904-12-23 – 1904-12-23{]}|pwv} angelangt und noch i{\geminationm}er in der
               ersten Verwandl. des I Aktes\pwindex{Beer-Hofmann, Richard 11.07.1866 – 26.09.1945@\textsc{Beer-Hofmann, Richard} (11.07.1866 – 26.09.1945), \emph{Schriftsteller}!Graf von Charolais. Ein Trauerspiel1904-12-23 – 1904-12-23@\strich\emph{Der Graf von Charolais. Ein Trauerspiel} {[}1904-12-23 – 1904-12-23{]}|pwv}. Das wird ein
               den Abend überfüllendes Stück\pwindex{Beer-Hofmann, Richard 11.07.1866 – 26.09.1945@\textsc{Beer-Hofmann, Richard} (11.07.1866 – 26.09.1945), \emph{Schriftsteller}!Graf von Charolais. Ein Trauerspiel1904-12-23 – 1904-12-23@\strich\emph{Der Graf von Charolais. Ein Trauerspiel} {[}1904-12-23 – 1904-12-23{]}|pwv}! Ihr\spacefill\mbox{
                  R.}\pend
           \endnumbering\briefempfaengerindex{Schnitzler, Arthur@\textsc{Schnitzler, Arthur}!zzzBeer-Hofmann, Richard@\emph{von Richard Beer-Hofmann}!1899-10-011@{1. 10. 1899}|)be}\mylabel{h}\end{ledgroupsized}  \newcommand{\dateiname}{L00985}\newcommand{\titel}{Richard Beer-Hofmann an Arthur Schnitzler, 1. 10. 1899}\newcommand{\editorInnen}{Martin Anton Müller und Gerd-Hermann Susen}\input{../tex-inputs/latex-pdf-abspann}
      