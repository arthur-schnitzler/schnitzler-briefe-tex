%% latex-korrekturansicht-vorspann.tex
%% Vorspann für die Korrekturansicht.
%% Lädt die gemeinsame Datei latex-vorspann.tex mit gesetztem Schalter.

\newif\ifkorrekturansicht
\korrekturansichttrue

\input{../tex-inputs/latex-vorspann}


\section[Hugo von Hofmannsthal an Arthur Schnitzler, 12. 10. 1897]{L00731 Hugo von Hofmannsthal an Arthur Schnitzler, 12. 10. 1897}
\nopagebreak\mylabel{L00731v}
\rehead{ }\normalsize\beginnumbering\briefempfaengerindex{Schnitzler, Arthur@\textsc{Schnitzler, Arthur}!zzzHofmannsthal, Hugo von@\emph{von Hugo von Hofmannsthal}!1897-10-121@{12. 10. 1897}|(be}
\toendnotes[C]{\smallbreak\pagebreak[2]}\Standort{CUL, Schnitzler, B 43.}
\physDesc{Postkarte, 352 Zeichen
\newline{}Handschrift: 1) schwarze Tinte, deutsche Kurrent\hspace{1em}2) schwarze Tinte, lateinische Kurrent (\noindent{}Adresse)\hspace{1em}
\newline{}Versand: 1) Stempel: »\nobreak{}\oindex{Hinterbruehl@\textbf{Hinterbrühl}, \emph{P.PPLA3}|pwk}Hinterbrühl, 12. 10. 97, 6–7 N\nobreak{}«.   2) Stempel: »\nobreak{}\oindex{IX., Alsergrund@\textbf{IX., Alsergrund}, \emph{A.ADM3}|pwk}Wien 9/3 72, 13. 10. 97, 8 . V, Bestellt\nobreak{}«. 
\newline{}Schnitzler: mit Bleistift datiert: »10. 97« 
\newline{}Ordnung: 1) mit Bleistift von unbekannter Hand nummeriert: »\strikeout{103}«  2) mit Bleistift von unbekannter Hand nummeriert:
                                    »97«}
\buchAbdrucke{\weitereDrucke{Hugo von Hofmannsthal, Arthur Schnitzler: \emph{Briefwechsel}. Frankfurt am Main: \emph{S. Fischer} 1964, S. 97.} }\toendnotes[C]{\smallbreak}\pstart{}{\pb}Herrn D\textsuperscript{r} Arthur Schnitzler\pend{}\pstart{}Wien\oindex{Wien@\textbf{Wien}, \emph{A.ADM2}|pw}\pend{}\pstart{}XI. Franckgasse 1.\oindex{Frankgasse 1@\textbf{Frankgasse 1}, \emph{Wohngebäude (K.WHS)}|pw}\pend{}{\bigskip}\vspace{1em}
\pstart
           \raggedleft{}{\pb}12\textsuperscript{ten}\pend
           
\pstart{}Mein lieber Arthur\pend\vspace{0.5em}
\pstart
           ich bin von morgen Mittwoch{ }abend an in Wien\oindex{Wien@\textbf{Wien}, \emph{A.ADM2}|pw}. Falls Sie ſich zu
               einer \label{K_L00731-1v}\edtext{Kainz\pwindex{Kainz, Josef 02.01.1858 – 20.09.1910@\textsc{Kainz, Josef} (02.01.1858 – 20.09.1910), \emph{Schauspieler/Schauspielerin}|pw}vorſtellung\pwindex{Juedin von Toledo@\emph{Die Jüdin von Toledo}|pwv}}{\lemma{\textnormal{\emph{Kainzvorſtellung}}}\Cendnote{\textnormal{\emph{Die Jüdin von Toledo}\pwindex{Juedin von Toledo@\emph{Die Jüdin von Toledo}|pwk} von Franz Grillparzer\pwindex{Grillparzer, Franz 15.01.1791 – 21.01.1872@\textsc{Grillparzer, Franz} (15.01.1791 – 21.01.1872), \emph{Schriftsteller/Schriftstellerin, Beamter/Beamte}|pwk} wurde im Burgtheater\oindex{Burgtheater@\textbf{Burgtheater}, \emph{S.THTR}|pwk} gegeben.}}}\label{K_L00731-1}, Donnerstag oder
                  Freitag einen Sitz nehmen und noch Zeit haben, einen gleichen für
               mich zu nehmen bitte thuen Sie es und ſchreiben mir vielleicht eine Zeile wo ich Sie
               für’s Theater abholen kann.\pend
           \pstart Ihr\spacefill\mbox{Hugo.}\pend{}\selectlanguage{ngerman}\endnumbering\briefempfaengerindex{Schnitzler, Arthur@\textsc{Schnitzler, Arthur}!zzzHofmannsthal, Hugo von@\emph{von Hugo von Hofmannsthal}!1897-10-121@{12. 10. 1897}|)be}\mylabel{L00731h}  \normalsize

\doendnotes{C}
\bigskip
\vfill

\clearpage

\footnotesize

\lohead{\textsc{register}}

% Definiere theindex-Environment komplett neu ohne reledmac
\makeatletter
\renewenvironment{theindex}{%
  \section*{\indexname}%
  \setlength{\parindent}{0pt}%
  \setlength{\parskip}{0pt plus 0.3pt}%
  \let\item\@idxitem
}{%
  \clearpage
}
\makeatother

\IfFileExists{\jobname-pw.ind}{\input{\jobname-pw.ind}}{}

\end{document}

      