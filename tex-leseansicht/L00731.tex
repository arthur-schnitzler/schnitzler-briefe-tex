%% latex-leseansicht-vorspann.tex
%% Vorspann für die Leseansicht.
%% Lädt die gemeinsame Datei latex-vorspann.tex mit nicht gesetztem Schalter.

\newif\ifkorrekturansicht
\korrekturansichtfalse

\input{../tex-inputs/latex-vorspann}


               \section[Hugo von Hofmannsthal an Arthur Schnitzler, 12. 10. 1897]{ Hugo von Hofmannsthal an Arthur Schnitzler, 12. 10. 1897}\nopagebreak\mylabel{v}\rehead{ }\begin{ledgroupsized}[t]{13cm}\normalsize\beginnumbering\briefempfaengerindex{Schnitzler, Arthur@\textsc{Schnitzler, Arthur}!zzzHofmannsthal, Hugo von@\emph{von Hugo von Hofmannsthal}!1897-10-121@{12. 10. 1897}|(be} \toendnotes[C]{\smallbreak\pagebreak[2]} \Standort{CUL, Schnitzler, B 43.}
\physDesc{Postkarte
\newline{}Handschrift: schwarze Tinte, deutsche Kurrent\newline{}Versand: 1) Stempel: »\nobreak{}\oindex{Hinterbruehl@\textbf{Hinterbrühl}|pwk}Hinterbrühl, 12. 10. 97, 6–7 N\nobreak{}«.  2) Stempel: »\nobreak{}\oindex{IX., Alsergrund@\textbf{IX., Alsergrund}|pwk}Wien 9/3 72, 13. 10. 97, 8 . V, Bestellt\nobreak{}«. 
\newline{}Schnitzler: mit Bleistift datiert: »10. 97« \newline{}Ordnung: 1) mit Bleistift von unbekannter Hand nummeriert: »\strikeout{103}« 2) mit Bleistift von unbekannter Hand nummeriert:
                                    »97«}\buchAbdrucke{\weitereDrucke{Hugo von Hofmannsthal, Arthur Schnitzler: \emph{Briefwechsel}. Hg. Therese Nickl und Heinrich Schnitzler. Frankfurt am Main: \emph{S. Fischer} 1964, S. 97.} }\toendnotes[C]{\smallbreak}\pstart{}{\pb}\textsc{Herrn D\textsuperscript{r} Arthur Schnitzler}\pend{}\pstart{}\textsc{Wien}\oindex{Wien@\textbf{Wien}|pw}\pend{}\pstart{}\textsc{XI. Franckgasse 1.}\oindex{Frankgasse@\textbf{Frankgasse}|pw}\pend{}{\bigskip}\pstart
           \raggedleft{}{\pb}12\textsuperscript{ten}\pend
           \pstart{}Mein lieber Arthur\pend\pstart
           ich bin von morgen Mittwoch{ }abend an in Wien\oindex{Wien@\textbf{Wien}|pw}. Falls Sie ſich zu
               einer \label{K_L00731_1v}\edtext{Kainz\pwindex{Kainz, Josef 02.01.1858 – 20.09.1910@\textsc{Kainz, Josef} (02.01.1858 – 20.09.1910), \emph{Schauspieler}|pw}vorſtellung\pwindex{Grillparzer, Franz 15.01.1791 – 21.01.1872@\textsc{Grillparzer, Franz} (15.01.1791 – 21.01.1872), \emph{Schriftsteller, Beamter}!Juedin von Toledo22. 11. 1872@\strich\emph{Die Jüdin von Toledo} {[}22. 11. 1872{]}|pwv}}{\lemma{\textnormal{\emph{Kainzvorſtellung}}}\Cendnote{\textnormal{\emph{Die Jüdin von Toledo}\pwindex{Grillparzer, Franz 15.01.1791 – 21.01.1872@\textsc{Grillparzer, Franz} (15.01.1791 – 21.01.1872), \emph{Schriftsteller, Beamter}!Juedin von Toledo22. 11. 1872@\strich\emph{Die Jüdin von Toledo} {[}22. 11. 1872{]}|pwk} von Franz Grillparzer\pwindex{Grillparzer, Franz 15.01.1791 – 21.01.1872@\textsc{Grillparzer, Franz} (15.01.1791 – 21.01.1872), \emph{Schriftsteller, Beamter}|pwk} wurde im Burgtheater\oindex{Burgtheater@\textbf{Burgtheater}|pwk} gegeben.}}}\label{K_L00731_1h}, Donnerstag oder
                  Freitag einen Sitz nehmen und noch Zeit haben, einen gleichen für
               mich zu nehmen bitte thuen Sie es und ſchreiben mir vielleicht eine Zeile wo ich Sie
               für’s Theater abholen kann.\pend
           \pstart Ihr\spacefill\mbox{Hugo.}\pend{}\endnumbering\briefempfaengerindex{Schnitzler, Arthur@\textsc{Schnitzler, Arthur}!zzzHofmannsthal, Hugo von@\emph{von Hugo von Hofmannsthal}!1897-10-121@{12. 10. 1897}|)be}\mylabel{h}\end{ledgroupsized}  \newcommand{\dateiname}{L00731}\newcommand{\titel}{Hugo von Hofmannsthal an Arthur Schnitzler, 12. 10. 1897}\newcommand{\editorInnen}{Martin Anton Müller und Gerd-Hermann Susen}%% latex-leseansicht-abspann.tex
%% Abspann für die Leseansicht.
%% Der Schalter \ifkorrekturansicht ist bereits durch den Vorspann gesetzt.

%% latex-abspann.tex
%% Gemeinsamer Abspann für Korrekturansicht und Leseansicht.
%% Setzt den Schalter \ifkorrekturansicht voraus (gesetzt in den
%% einbindenden Dateien latex-korrekturansicht-abspann.tex bzw.
%% latex-leseansicht-abspann.tex).
%% ---------------------------------------------------------------

\normalsize

% Das esempio-Environment wird nur in der Leseansicht benötigt
\ifkorrekturansicht\else
\newenvironment{esempio}[3]%
{
    \vspace{1.5ex}
    \rlap{\underline{#1}}
    \par
    \setlength{\parindent}{0cm}
    \nopagebreak
    \leftskip=#2cm
    \rightskip=#3cm
}
{
    \par
}
\fi

\doendnotes{C}
\bigskip
\vfill

\clearpage

\footnotesize

\ifkorrekturansicht
  \lohead{\textsc{register}}
\fi

% theindex-Environment neu definieren ohne reledmac
\makeatletter
\renewenvironment{theindex}{%
  \ifkorrekturansicht
    \section*{\indexname}%
  \else
    \subsubsection*{Index der erwähnten Entitäten}%
  \fi
  \setlength{\parindent}{0pt}%
  \setlength{\parskip}{0pt plus 0.3pt}%
  \let\item\@idxitem
}{%
  \ifkorrekturansicht\clearpage\fi
}
\makeatother

\IfFileExists{\jobname-pw.ind}{\input{\jobname-pw.ind}}{}

% Quellenangabe nur in der Leseansicht
\ifkorrekturansicht\else
% Fallback-Definitionen, falls die .tex-Datei \titel etc. nicht gesetzt hat
\providecommand{\titel}{}
\providecommand{\editorInnen}{}
\providecommand{\dateiname}{\jobname}

\vspace{3cm}

\vfill

\footnotesize
\textsc{Quelle}: \titel. Herausgegeben von {\editorInnen}. In: \emph{Arthur Schnitzler: Briefwechsel mit Autorinnen und Autoren}.
 Digitale Edition, https://schnitzler-briefe.acdh.oeaw.ac.at/{\dateiname}.html (Stand \today)
\fi

\end{document}


      