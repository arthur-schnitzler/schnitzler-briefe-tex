%% latex-leseansicht-vorspann.tex
%% Vorspann für die Leseansicht.
%% Lädt die gemeinsame Datei latex-vorspann.tex mit nicht gesetztem Schalter.

\newif\ifkorrekturansicht
\korrekturansichtfalse

\input{../tex-inputs/latex-vorspann}


\section[Hugo von Hofmannsthal an Arthur Schnitzler, 12. 10. 1897]{L00731 Hugo von Hofmannsthal an Arthur Schnitzler, 12. 10. 1897}
\nopagebreak\mylabel{L00731v}
\rehead{ }\normalsize\beginnumbering\briefempfaengerindex{Schnitzler, Arthur@\textsc{Schnitzler, Arthur}!zzzHofmannsthal, Hugo von@\emph{von Hugo von Hofmannsthal}!1897-10-121@{12. 10. 1897}|(be}
\toendnotes[C]{\smallbreak\pagebreak[2]}
\correspDesc{Versand  durch Hugo von Hofmannsthal am 12. 10. 1897 in Hinterbrühl
\newline{}Erhalt  durch Arthur Schnitzler am 13. 10. 1897 in Wien}\toendnotes[C]{\smallbreak}
\Standort{CUL, Schnitzler, B 43.}
\physDesc{Postkarte, 352 Zeichen
\newline{}Handschrift: schwarze Tinte, deutsche Kurrent
\newline{}Versand: 1) Stempel: »\nobreak{}\oindex{Hinterbrühl@\textbf{Hinterbrühl}, \emph{Hauptstadt}|pwk}Hinterbrühl, 12. 10. 97, 6–7 N\nobreak{}«.   2) Stempel: »\nobreak{}\oindex{IX., Alsergrund@\textbf{IX., Alsergrund}, \emph{Verwaltungsgebiet}|pwk}Wien 9/3 72, 13. 10. 97, 8 . V, Bestellt\nobreak{}«. 
\newline{}Schnitzler: mit Bleistift datiert: »10. 97« 
\newline{}Ordnung: 1) mit Bleistift von unbekannter Hand nummeriert: »\strikeout{103}«  2) mit Bleistift von unbekannter Hand nummeriert:
                                    »97«}
\buchAbdrucke{\weitereDrucke{Hugo von Hofmannsthal, Arthur Schnitzler: \emph{Briefwechsel}. Herausgegeben von Therese Nickl und Heinrich Schnitzler. Frankfurt am Main: \emph{S. Fischer} 1964, S. 97.} }\toendnotes[C]{\smallbreak}\pstart{}\textsc{{\pb}Herrn D\textsuperscript{r} Arthur Schnitzler}\pend{}\pstart{}\textsc{Wien\oindex{Wien@\textbf{Wien}, \emph{Verwaltungsgebiet}|pw}}\pend{}\pstart{}\textsc{XI. Franckgasse 1.\oindex{Wien@\textbf{Wien}!IX., Alsergrund@\textbf{IX., Alsergrund}!Frankgasse 1@\textbf{Frankgasse 1}, \emph{Wohngebäude}|pw}}\pend{}{\bigskip}\vspace{1em}
\pstart
           \raggedleft{}{\pb}12\textsuperscript{ten}\pend
           
\pstart{}Mein lieber Arthur\pend\vspace{0.5em}
\pstart
           ich bin von morgen Mittwoch{ }abend an in Wien\oindex{Wien@\textbf{Wien}, \emph{Verwaltungsgebiet}|pw}. Falls Sie{ }ſich zu
               einer \label{K_L00731-1v}\edtext{Kainz\pwindex{Kainz, Josef 2.\,1.\,1858 Mosonmagyaróvár – 20.\,9.\,1910 Wien@\textsc{Kainz, Josef} (2.\,1.\,1858 Mosonmagyaróvár – 20.\,9.\,1910 Wien), \emph{Schauspieler}|pw}vorſtellung\pwindex{Grillparzer, Franz 15.\,1.\,1791 Wien – 21.\,1.\,1872 ebd.@\textsc{Grillparzer, Franz} (15.\,1.\,1791 Wien – 21.\,1.\,1872 ebd.), \emph{Schriftsteller, Beamter}!Jüdin von Toledo@\strich\emph{Die Jüdin von Toledo}|pwv}}{\lemma{\textnormal{\emph{Kainzvorstellung}}}\Cendnote{\textnormal{\emph{Die Jüdin von Toledo}\pwindex{Grillparzer, Franz 15.\,1.\,1791 Wien – 21.\,1.\,1872 ebd.@\textsc{Grillparzer, Franz} (15.\,1.\,1791 Wien – 21.\,1.\,1872 ebd.), \emph{Schriftsteller, Beamter}!Jüdin von Toledo@\strich\emph{Die Jüdin von Toledo}|pwk} von Franz Grillparzer\pwindex{Grillparzer, Franz 15.\,1.\,1791 Wien – 21.\,1.\,1872 ebd.@\textsc{Grillparzer, Franz} (15.\,1.\,1791 Wien – 21.\,1.\,1872 ebd.), \emph{Schriftsteller, Beamter}|pwk} wurde im Burgtheater\oindex{Wien@\textbf{Wien}!I., Innere Stadt@\textbf{I., Innere Stadt}!Burgtheater@\textbf{Burgtheater}, \emph{Theater}|pwk} gegeben.}}}\label{K_L00731-1}, Donnerstag oder
                  Freitag einen Sitz nehmen und noch Zeit haben, einen gleichen für
               mich zu nehmen bitte thuen Sie es und{ }ſchreiben mir vielleicht eine Zeile wo ich Sie
               für’s Theater abholen kann.\pend
           \pstart Ihr\spacefill\mbox{Hugo.}\pend{}\selectlanguage{ngerman}\endnumbering\briefempfaengerindex{Schnitzler, Arthur@\textsc{Schnitzler, Arthur}!zzzHofmannsthal, Hugo von@\emph{von Hugo von Hofmannsthal}!1897-10-121@{12. 10. 1897}|)be}\mylabel{L00731h}  \newcommand{\dateiname}{L00731}\newcommand{\titel}{Hugo von Hofmannsthal an Arthur Schnitzler, 12. 10. 1897}\newcommand{\editorInnen}{Martin Anton Müller und Gerd-Hermann Susen}%% latex-leseansicht-abspann.tex
%% Abspann für die Leseansicht.
%% Der Schalter \ifkorrekturansicht ist bereits durch den Vorspann gesetzt.

%% latex-abspann.tex
%% Gemeinsamer Abspann für Korrekturansicht und Leseansicht.
%% Setzt den Schalter \ifkorrekturansicht voraus (gesetzt in den
%% einbindenden Dateien latex-korrekturansicht-abspann.tex bzw.
%% latex-leseansicht-abspann.tex).
%% ---------------------------------------------------------------

\normalsize

% Das esempio-Environment wird nur in der Leseansicht benötigt
\ifkorrekturansicht\else
\newenvironment{esempio}[3]%
{
    \vspace{1.5ex}
    \rlap{\underline{#1}}
    \par
    \setlength{\parindent}{0cm}
    \nopagebreak
    \leftskip=#2cm
    \rightskip=#3cm
}
{
    \par
}
\fi

\doendnotes{C}
\bigskip
\vfill

\clearpage

\footnotesize

\ifkorrekturansicht
  \lohead{\textsc{register}}
\fi

% theindex-Environment neu definieren ohne reledmac
\makeatletter
\renewenvironment{theindex}{%
  \ifkorrekturansicht
    \section*{\indexname}%
  \else
    \subsubsection*{Index der erwähnten Entitäten}%
  \fi
  \setlength{\parindent}{0pt}%
  \setlength{\parskip}{0pt plus 0.3pt}%
  \let\item\@idxitem
}{%
  \ifkorrekturansicht\clearpage\fi
}
\makeatother

\IfFileExists{\jobname-pw.ind}{\input{\jobname-pw.ind}}{}

% Quellenangabe nur in der Leseansicht
\ifkorrekturansicht\else
% Fallback-Definitionen, falls die .tex-Datei \titel etc. nicht gesetzt hat
\providecommand{\titel}{}
\providecommand{\editorInnen}{}
\providecommand{\dateiname}{\jobname}

\vspace{3cm}

\vfill

\footnotesize
\textsc{Quelle}: \titel. Herausgegeben von {\editorInnen}. In: \emph{Arthur Schnitzler: Briefwechsel mit Autorinnen und Autoren}.
 Digitale Edition, https://schnitzler-briefe.acdh.oeaw.ac.at/{\dateiname}.html (Stand \today)
\fi

\end{document}


