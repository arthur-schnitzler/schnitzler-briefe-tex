%% latex-korrekturansicht-vorspann.tex
%% Vorspann für die Korrekturansicht.
%% Lädt die gemeinsame Datei latex-vorspann.tex mit gesetztem Schalter.

\newif\ifkorrekturansicht
\korrekturansichttrue

\input{../tex-inputs/latex-vorspann}


\section[Felix Braun an Arthur Schnitzler, 9. 12. 1924]{L02421 Felix Braun an Arthur Schnitzler, 9. 12. 1924}
\nopagebreak\mylabel{L02421v}
\rehead{ }\normalsize\beginnumbering\briefempfaengerindex{Schnitzler, Arthur@\textsc{Schnitzler, Arthur}!zzzBraun, Felix@\emph{von Felix Braun}!1924-12-091@{9. 12. 1924}|(be}
\toendnotes[C]{\smallbreak\pagebreak[2]}\Standort{DLA, A:Schnitzler, HS.NZ85.1.2604,4.}
\physDesc{Brief, 1 Blatt, 2 Seiten, 913 Zeichen
\newline{}Handschrift: schwarze Tinte, deutsche Kurrent
\newline{}Schnitzler: 1) mit Bleistift beschriftet: »\textsc{Braun}«  2) mit rotem Buntstift mehrere Unterstreichungen}\toendnotes[C]{\smallbreak}
\pstart
           \centering{}{\pb}Salzburg\oindex{Salzburg@\textbf{Salzburg}, \emph{A.ADM2}|pw} / 9. XII. 1924\pend
           
\pstart{}Verehrter Herr Doktor!\pend\vspace{0.5em}
\pstart
           Statt Ihnen für die liebe Gabe Ihres neuen Buches\pwindex{Fraeulein Else@\emph{Fräulein Else}|pwv} zu danken, komme ich mit einer Bitte, die nun wohl
               eben dieſes Buch\pwindex{Fraeulein Else@\emph{Fräulein Else}|pwv} betrifft. Ich
               habe es nämlich – nicht erhalten, man hat es mir von Wien\oindex{Wien@\textbf{Wien}, \emph{A.ADM2}|pw} hieher, wo ich für einige Tage Stefan
                  Zweigs\pwindex{Zweig, Stefan 28.11.1881 – 23.02.1942@\textsc{Zweig, Stefan} (28.11.1881 – 23.02.1942), \emph{Schriftsteller/Schriftstellerin}|pw} Stellvertreter war, nachgeſandt und da hat es ein
               autographenſammelnder Poſtbeamter an ſich genommen – ich hoffe leider nicht mehr auf
               den Wiedergewinn des mir durch Ihre Inſchrift doppelt wertvollen Buchs\pwindex{Fraeulein Else@\emph{Fräulein Else}|pwv}. Darf ich Ihnen nun die Bitte
               unterbreiten, in das Exemplar, das ich Ihnen ſenden werde, mir wieder eine Widmung
               einzuſchreiben? Ich wäre Ihnen ſehr, ſehr dankbar dafür. In einer Woche etwa bin ich
               wieder zu Hauſe. {\pb}So käme mir das erneute Geſchenk
               gerade als Weihnachtsgabe recht.\pend
           
\pstart
           Für die ehrenvolle Freude, die Sie mir zugedacht haben, ſage ich Ihnen, verehrter
               Herr Doktor, beſten Dank und ſo bleibe ich Ihr herzlich ergebener\pend
           \pstart \spacefill\mbox{Felix Braun.}\pend{}\selectlanguage{ngerman}\endnumbering\briefempfaengerindex{Schnitzler, Arthur@\textsc{Schnitzler, Arthur}!zzzBraun, Felix@\emph{von Felix Braun}!1924-12-091@{9. 12. 1924}|)be}\mylabel{L02421h}  \normalsize

\doendnotes{C}
\bigskip
\vfill

\clearpage

\footnotesize

\lohead{\textsc{register}}

% Definiere theindex-Environment komplett neu ohne reledmac
\makeatletter
\renewenvironment{theindex}{%
  \section*{\indexname}%
  \setlength{\parindent}{0pt}%
  \setlength{\parskip}{0pt plus 0.3pt}%
  \let\item\@idxitem
}{%
  \clearpage
}
\makeatother

\IfFileExists{\jobname-pw.ind}{\input{\jobname-pw.ind}}{}

\end{document}

      