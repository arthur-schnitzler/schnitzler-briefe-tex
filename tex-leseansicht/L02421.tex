%% latex-leseansicht-vorspann.tex
%% Vorspann für die Leseansicht.
%% Lädt die gemeinsame Datei latex-vorspann.tex mit nicht gesetztem Schalter.

\newif\ifkorrekturansicht
\korrekturansichtfalse

\input{../tex-inputs/latex-vorspann}


\section[Felix Braun an Arthur Schnitzler, 9. 12. 1924]{L02421 Felix Braun an Arthur Schnitzler, 9. 12. 1924}
\nopagebreak\mylabel{L02421v}
\rehead{ }\normalsize\beginnumbering\briefempfaengerindex{Schnitzler, Arthur@\textsc{Schnitzler, Arthur}!zzzBraun, Felix@\emph{von Felix Braun}!1924-12-091@{9. 12. 1924}|(be}
\toendnotes[C]{\smallbreak\pagebreak[2]}
\correspDesc{Versand  durch Felix Braun am 9. 12. 1924 in Salzburg
\newline{}Erhalt  durch Arthur Schnitzler im Zeitraum [10. 12. 1924 – 14. 12. 1924?] in Wien}\toendnotes[C]{\smallbreak}
\Standort{DLA, A:Schnitzler, HS.NZ85.1.2604,4.}
\physDesc{Brief, 1 Blatt, 2 Seiten, 913 Zeichen
\newline{}Handschrift: schwarze Tinte, deutsche Kurrent
\newline{}Schnitzler: 1) mit Bleistift beschriftet: »\textsc{Braun}«  2) mit rotem Buntstift mehrere Unterstreichungen}\toendnotes[C]{\smallbreak}
\pstart
           \centering{}{\pb}Salzburg\oindex{Salzburg@\textbf{Salzburg}, \emph{Verwaltungsgebiet}|pw} / 9. XII. 1924\pend
           
\pstart{}Verehrter Herr Doktor!\pend\vspace{0.5em}
\pstart
           Statt Ihnen für die liebe Gabe Ihres neuen Buches\pwindex{Schnitzler, Arthur 15.\,5.\,1862 Wien – 21.\,10.\,1931 ebd.@\textsc{Schnitzler, Arthur} (15.\,5.\,1862 Wien – 21.\,10.\,1931 ebd.), \emph{Schriftsteller, Mediziner}!Fräulein Else@\strich\emph{Fräulein Else}|pwv} zu danken, komme ich mit einer Bitte, die nun wohl
               eben dieſes Buch\pwindex{Schnitzler, Arthur 15.\,5.\,1862 Wien – 21.\,10.\,1931 ebd.@\textsc{Schnitzler, Arthur} (15.\,5.\,1862 Wien – 21.\,10.\,1931 ebd.), \emph{Schriftsteller, Mediziner}!Fräulein Else@\strich\emph{Fräulein Else}|pwv} betrifft. Ich
               habe es nämlich – nicht erhalten, man hat es mir von Wien\oindex{Wien@\textbf{Wien}, \emph{Verwaltungsgebiet}|pw} hieher, wo ich für einige Tage Stefan
                  Zweigs\pwindex{Zweig, Stefan 28.\,11.\,1881 Wien – 23.\,2.\,1942 Petrópolis@\textsc{Zweig, Stefan} (28.\,11.\,1881 Wien – 23.\,2.\,1942 Petrópolis), \emph{Schriftsteller}|pw} Stellvertreter war, nachgeſandt und da hat es ein
               autographenſammelnder Poſtbeamter an{ }ſich genommen – ich hoffe leider nicht mehr auf
               den Wiedergewinn des mir durch Ihre Inſchrift doppelt wertvollen Buchs\pwindex{Schnitzler, Arthur 15.\,5.\,1862 Wien – 21.\,10.\,1931 ebd.@\textsc{Schnitzler, Arthur} (15.\,5.\,1862 Wien – 21.\,10.\,1931 ebd.), \emph{Schriftsteller, Mediziner}!Fräulein Else@\strich\emph{Fräulein Else}|pwv}. Darf ich Ihnen nun die Bitte
               unterbreiten, in das Exemplar, das ich Ihnen{ }ſenden werde, mir wieder eine Widmung
               einzuſchreiben? Ich wäre Ihnen{ }ſehr,{ }ſehr dankbar dafür. In einer Woche etwa bin ich
               wieder zu Hauſe. {\pb}So käme mir das erneute Geſchenk
               gerade als Weihnachtsgabe recht.\pend
           
\pstart
           Für die ehrenvolle Freude, die Sie mir zugedacht haben,{ }ſage ich Ihnen, verehrter
               Herr Doktor, beſten Dank und{ }ſo bleibe ich Ihr herzlich ergebener\pend
           \pstart \spacefill\mbox{Felix Braun.}\pend{}\selectlanguage{ngerman}\endnumbering\briefempfaengerindex{Schnitzler, Arthur@\textsc{Schnitzler, Arthur}!zzzBraun, Felix@\emph{von Felix Braun}!1924-12-091@{9. 12. 1924}|)be}\mylabel{L02421h}  \newcommand{\dateiname}{L02421}\newcommand{\titel}{Felix Braun an Arthur Schnitzler, 9. 12. 1924}\newcommand{\editorInnen}{Martin Anton Müller und Gerd-Hermann Susen}%% latex-leseansicht-abspann.tex
%% Abspann für die Leseansicht.
%% Der Schalter \ifkorrekturansicht ist bereits durch den Vorspann gesetzt.

%% latex-abspann.tex
%% Gemeinsamer Abspann für Korrekturansicht und Leseansicht.
%% Setzt den Schalter \ifkorrekturansicht voraus (gesetzt in den
%% einbindenden Dateien latex-korrekturansicht-abspann.tex bzw.
%% latex-leseansicht-abspann.tex).
%% ---------------------------------------------------------------

\normalsize

% Das esempio-Environment wird nur in der Leseansicht benötigt
\ifkorrekturansicht\else
\newenvironment{esempio}[3]%
{
    \vspace{1.5ex}
    \rlap{\underline{#1}}
    \par
    \setlength{\parindent}{0cm}
    \nopagebreak
    \leftskip=#2cm
    \rightskip=#3cm
}
{
    \par
}
\fi

\doendnotes{C}
\bigskip
\vfill

\clearpage

\footnotesize

\ifkorrekturansicht
  \lohead{\textsc{register}}
\fi

% theindex-Environment neu definieren ohne reledmac
\makeatletter
\renewenvironment{theindex}{%
  \ifkorrekturansicht
    \section*{\indexname}%
  \else
    \subsubsection*{Index der erwähnten Entitäten}%
  \fi
  \setlength{\parindent}{0pt}%
  \setlength{\parskip}{0pt plus 0.3pt}%
  \let\item\@idxitem
}{%
  \ifkorrekturansicht\clearpage\fi
}
\makeatother

\IfFileExists{\jobname-pw.ind}{\input{\jobname-pw.ind}}{}

% Quellenangabe nur in der Leseansicht
\ifkorrekturansicht\else
% Fallback-Definitionen, falls die .tex-Datei \titel etc. nicht gesetzt hat
\providecommand{\titel}{}
\providecommand{\editorInnen}{}
\providecommand{\dateiname}{\jobname}

\vspace{3cm}

\vfill

\footnotesize
\textsc{Quelle}: \titel. Herausgegeben von {\editorInnen}. In: \emph{Arthur Schnitzler: Briefwechsel mit Autorinnen und Autoren}.
 Digitale Edition, https://schnitzler-briefe.acdh.oeaw.ac.at/{\dateiname}.html (Stand \today)
\fi

\end{document}


