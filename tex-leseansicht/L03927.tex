%% latex-leseansicht-vorspann.tex
%% Vorspann für die Leseansicht.
%% Lädt die gemeinsame Datei latex-vorspann.tex mit nicht gesetztem Schalter.

\newif\ifkorrekturansicht
\korrekturansichtfalse

\input{../tex-inputs/latex-vorspann}


\section[Arthur Schnitzler an Theodor Herzl, 27. 3. 1895]{L03927 Arthur Schnitzler an Theodor Herzl, 27. 3. 1895}
\nopagebreak\mylabel{L03927v}
\rehead{ }\normalsize\beginnumbering\briefempfaengerindex{Herzl, Theodor@\textsc{Herzl, Theodor}!zzzSchnitzler, Arthur@\emph{von Arthur Schnitzler}!1895-04-271@{27. 4. 1895}|(be}
\toendnotes[C]{\smallbreak\pagebreak[2]}
\correspDesc{Versand  durch Arthur Schnitzler am 27. 4. 1895 in Wien
\newline{}Erhalt  durch Theodor Herzl in Wien}\toendnotes[C]{\smallbreak}
\Standort{Jerusalem, Central Zionist Archives, H1:1925-12.}
\physDesc{,  Blätter,  Seiten
\newline{}Handschrift: , deutsche Kurrent
\newline{}Beilage: H1/1925-2, XXXX }
\buchAbdrucke{\weitereDrucke{Arthur Schnitzler: \emph{Briefe 1875–1912}. Herausgegeben von Therese Nickl und Heinrich Schnitzler. Frankfurt am Main: \emph{S. Fischer} 1981, S. 254.} }\toendnotes[C]{\smallbreak}
\pstart{}{\pb}Lieber Freund,\pend\vspace{0.5em}
\pstart
           ich ſende Ihnen hier den \label{K_L03927-1v}\edtext{Brief}{\lemma{\textnormal{\emph{Brief}}}\Cendnote{\textnormal{}}}\label{K_L03927-1}, welchen ich
               geſtern Abends vorgefunden habe. Es würde ſich nun doch vielleicht empfehlen, wenn
               Sie Ihre Anweſenheit in Wien\oindex{Wien@\textbf{Wien}, \emph{Verwaltungsgebiet}|pw} benützten, um der
               Angelegenheit einen Ruck nach vorwärts zu geben. Daſs dieſe erſte erfreuliche
               Kundgebung mit Ihrem Eintreffen {\pb}in Wien\oindex{Wien@\textbf{Wien}, \emph{Verwaltungsgebiet}|pw} zuſammenfällt, wollen wir als günstige Vorbedeutung nehmen,
               und ich kann mich der Hoffnung nicht erwehren, daſs eine perſönliche Rückſprache von
               Ihnen mit M. G.\pwindex{Müller-Guttenbrunn, Adam 22.\,10.\,1852 Zăbrani – 5.\,1.\,1923 Wien@\textsc{Müller-Guttenbrunn, Adam} (22.\,10.\,1852 Zăbrani – 5.\,1.\,1923 Wien), \emph{Schriftsteller, Theaterleiter, Beamter}|pw} der Sache eine raſche und
               glückliche Wendung gäbe.\pend
           
\pstart
           Seien Sie vielmals herzlich gegrüßt von Ihrem treu ergeb\textcolor{gray}{nen}{\\[\baselineskip]}\spacefill\mbox{ArthSch}\pend
           \leftskip=0em{}
\pstart
           27. 3. 95\pend
           \selectlanguage{ngerman}\vspace{1em}{\vspace{1\baselineskip}}
\pstart
           \centering{}{\pb}\textcolor{gray}{\textbf{Raimund Theater\orgindex{Raimund-Theater@Raimund-Theater|pw}.}}\pend
           
\pstart
           \centering{}\textcolor{gray}{\textbf{Direction: A.
                        Müller-Guttenbrunn\pwindex{Müller-Guttenbrunn, Adam 22.\,10.\,1852 Zăbrani – 5.\,1.\,1923 Wien@\textsc{Müller-Guttenbrunn, Adam} (22.\,10.\,1852 Zăbrani – 5.\,1.\,1923 Wien), \emph{Schriftsteller, Theaterleiter, Beamter}|pw}.}}\pend
           
\pstart
           \raggedleft{}\textcolor{gray}{\textbf{Wien\oindex{Wien@\textbf{Wien}, \emph{Verwaltungsgebiet}|pw}, am}}{ }26. III \textcolor{gray}{\textbf{189}}5\pend
           
\pstart{}Verehrter Herr Dr Schnitzler!\pend\vspace{0.5em}
\pstart
           Ich habe das Schauſpiel ›Ghetto\pwindex{Herzl, Theodor 2.\,5.\,1860 Budapest – 3.\,7.\,1904 Edlach@\textsc{Herzl, Theodor} (2.\,5.\,1860 Budapest – 3.\,7.\,1904 Edlach), \emph{Schriftsteller, Journalist}!neue Ghetto. Schauspiel in vier Acten@\strich\emph{Das neue Ghetto. Schauspiel in vier Acten}|pw}‹ mit
                  außerordentlichem Intereſſe geleſen u. halte das Stück\pwindex{Herzl, Theodor 2.\,5.\,1860 Budapest – 3.\,7.\,1904 Edlach@\textsc{Herzl, Theodor} (2.\,5.\,1860 Budapest – 3.\,7.\,1904 Edlach), \emph{Schriftsteller, Journalist}!neue Ghetto. Schauspiel in vier Acten@\strich\emph{Das neue Ghetto. Schauspiel in vier Acten}|pwv}, obwohl mich die Löſung
                  nicht befriedigte u. ich dem Helden mehr Spielraum gegönnt hätte, für eine
                  der intereſſanteſten Arbeiten, die mir ſeit Langem unter\strikeout{k}geko{\geminationm}en. Das Stück hat frappante Züge von
                  Lebenswahrheit, es iſt reich an feinem Detail u. es wird getragen {\pb}von einer Idee, der man weder die Natürkichkeit,
                  noch die tiefere Bedeutung abſprechen kann.\pend
           
\pstart
           Und trotz alledem – würden \uline{Sie} es aufführen?
                  Und glauben Sie, daß ſich irgendwo in deutſchen Landen ein großes Theater
                  findet, welches »Ghetto\pwindex{Herzl, Theodor 2.\,5.\,1860 Budapest – 3.\,7.\,1904 Edlach@\textsc{Herzl, Theodor} (2.\,5.\,1860 Budapest – 3.\,7.\,1904 Edlach), \emph{Schriftsteller, Journalist}!neue Ghetto. Schauspiel in vier Acten@\strich\emph{Das neue Ghetto. Schauspiel in vier Acten}|pw}« aufführt? Ich
                  glaub es nicht! Sie können das Stück\pwindex{Herzl, Theodor 2.\,5.\,1860 Budapest – 3.\,7.\,1904 Edlach@\textsc{Herzl, Theodor} (2.\,5.\,1860 Budapest – 3.\,7.\,1904 Edlach), \emph{Schriftsteller, Journalist}!neue Ghetto. Schauspiel in vier Acten@\strich\emph{Das neue Ghetto. Schauspiel in vier Acten}|pwv} alſo ruhig noch einige Tage hier liegen laſſen, ich will es
                  noch von andern, ganz unbetheiligten Perſonen, deren Urtheil mir werthvoll
                  iſt, leſen laſſen. \uline{Wenn} das Stück\pwindex{Herzl, Theodor 2.\,5.\,1860 Budapest – 3.\,7.\,1904 Edlach@\textsc{Herzl, Theodor} (2.\,5.\,1860 Budapest – 3.\,7.\,1904 Edlach), \emph{Schriftsteller, Journalist}!neue Ghetto. Schauspiel in vier Acten@\strich\emph{Das neue Ghetto. Schauspiel in vier Acten}|pwv} in Wien\oindex{Wien@\textbf{Wien}, \emph{Verwaltungsgebiet}|pw} jemals aufgeführt \strikeout{iſ} wird,{ }ſo kann dies nur im R.
                     Th.\orgindex{Raimund-Theater@Raimund-Theater|pw} geſchehen. \uline{Dank} werden wir kaum
                  dafür ernten, weder von den Juden, noch von den \textsc{Antisemiten}! Den Herrn \textsc{Schnabel} würde ich
                  gerne{ }ſprechen.\pend
           
\pstart
           Ihr ergebenſter \textcolor{gray}{MGuttenbrunn}\pend
           \selectlanguage{ngerman}\endnumbering\briefempfaengerindex{Herzl, Theodor@\textsc{Herzl, Theodor}!zzzSchnitzler, Arthur@\emph{von Arthur Schnitzler}!1895-04-271@{27. 4. 1895}|)be}\mylabel{L03927h}
\begin{anhang}
\end{anhang}\newcommand{\dateiname}{L03927}\newcommand{\titel}{Arthur Schnitzler an Theodor Herzl, 27. 3. 1895}\newcommand{\editorInnen}{Herausgegeben von Jahnke, SelmaMüller, Martin Anton}%% latex-leseansicht-abspann.tex
%% Abspann für die Leseansicht.
%% Der Schalter \ifkorrekturansicht ist bereits durch den Vorspann gesetzt.

%% latex-abspann.tex
%% Gemeinsamer Abspann für Korrekturansicht und Leseansicht.
%% Setzt den Schalter \ifkorrekturansicht voraus (gesetzt in den
%% einbindenden Dateien latex-korrekturansicht-abspann.tex bzw.
%% latex-leseansicht-abspann.tex).
%% ---------------------------------------------------------------

\normalsize

% Das esempio-Environment wird nur in der Leseansicht benötigt
\ifkorrekturansicht\else
\newenvironment{esempio}[3]%
{
    \vspace{1.5ex}
    \rlap{\underline{#1}}
    \par
    \setlength{\parindent}{0cm}
    \nopagebreak
    \leftskip=#2cm
    \rightskip=#3cm
}
{
    \par
}
\fi

\doendnotes{C}
\bigskip
\vfill

\clearpage

\footnotesize

\ifkorrekturansicht
  \lohead{\textsc{register}}
\fi

% theindex-Environment neu definieren ohne reledmac
\makeatletter
\renewenvironment{theindex}{%
  \ifkorrekturansicht
    \section*{\indexname}%
  \else
    \subsubsection*{Index der erwähnten Entitäten}%
  \fi
  \setlength{\parindent}{0pt}%
  \setlength{\parskip}{0pt plus 0.3pt}%
  \let\item\@idxitem
}{%
  \ifkorrekturansicht\clearpage\fi
}
\makeatother

\IfFileExists{\jobname-pw.ind}{\input{\jobname-pw.ind}}{}

% Quellenangabe nur in der Leseansicht
\ifkorrekturansicht\else
% Fallback-Definitionen, falls die .tex-Datei \titel etc. nicht gesetzt hat
\providecommand{\titel}{}
\providecommand{\editorInnen}{}
\providecommand{\dateiname}{\jobname}

\vspace{3cm}

\vfill

\footnotesize
\textsc{Quelle}: \titel. Herausgegeben von {\editorInnen}. In: \emph{Arthur Schnitzler: Briefwechsel mit Autorinnen und Autoren}.
 Digitale Edition, https://schnitzler-briefe.acdh.oeaw.ac.at/{\dateiname}.html (Stand \today)
\fi

\end{document}


