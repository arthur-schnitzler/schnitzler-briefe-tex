%% latex-leseansicht-vorspann.tex
%% Vorspann für die Leseansicht.
%% Lädt die gemeinsame Datei latex-vorspann.tex mit nicht gesetztem Schalter.

\newif\ifkorrekturansicht
\korrekturansichtfalse

\input{../tex-inputs/latex-vorspann}


\section[ Paul Goldmann an Arthur Schnitzler, 2. 7. 1904]{L03447 Paul Goldmann an Arthur Schnitzler,  2. 7. 1904}
\nopagebreak\mylabel{L03447v}
\rehead{ }\normalsize\beginnumbering\briefempfaengerindex{Schnitzler, Arthur@\textsc{Schnitzler, Arthur}!zzzGoldmann, Paul@\emph{von Paul Goldmann}!1904-07-021@{2. 7. 1904}|(be}
\toendnotes[C]{\smallbreak\pagebreak[2]}
\correspDesc{Versand  durch Paul Goldmann am 2. 7. 1904 in Hamburg
\newline{}Erhalt  durch Arthur Schnitzler im Zeitraum [3. 7. 1904
                  – 7. 7. 1904?] in Wien}\toendnotes[C]{\smallbreak}
\Standort{DLA, A:Schnitzler, HS.NZ85.1.3174.}
\physDesc{Bildpostkarte, 146 Zeichen
\newline{}Handschrift: Bleistift, deutsche Kurrent
\newline{}Versand: Stempel: »\nobreak{}\oindex{Hamburg@\textbf{Hamburg}|pwk}Hamburg 1 af, 2. 7. 04., 12–1N.\nobreak{}«.  }\toendnotes[C]{\smallbreak}\pstart{}\textsc{{\pb}Herrn}\pend{}\pstart{}\textsc{Dr. Arthur Schnitzler}\pend{}\pstart{}\textsc{Wien\oindex{Wien@\textbf{Wien}, \emph{Verwaltungsgebiet}|pw}}\pend{}\pstart{}\textsc{XVIII. Spöttelgaſse 7\oindex{Wien@\textbf{Wien}!XVIII., Währing@\textbf{XVIII., Währing}!Edmund-Weiß-Gasse 7@\textbf{Edmund-Weiß-Gasse 7}, \emph{Wohngebäude}|pw}}\pend{}{\bigskip}
\pstart
           \noindent{}\centering{}{\pb}{[}\label{K_L03447-1v}\edtext{Kunstdruck einer Wäscherin mit
                  schwarzer Hautfarbe von Bonte\pwindex{Quarles Bonte, Louise *~10.\,10.\,1873@\textsc{Quarles Bonte, Louise} (*~10.\,10.\,1873), \emph{Illustratorin}|pw}\pwindex{Bonte, George Willard *~16.\,5.\,1873 Cincinnati@\textsc{Bonte, George Willard} (*~16.\,5.\,1873 Cincinnati), \emph{Illustrator}|pw}}{\lemma{\textnormal{\emph{Kunstdruck … Bonte}}}\Cendnote{\textnormal{Die Postkartenserie, aus der das
                     Motiv dieser Karte stammt, ist eine rassistische Seltsamkeit mit Stereotypen
                     schwarzer Plantagenarbeiterinnen und -arbeiter. Die meisten Darstellungen
                     fanden leicht modifiziert Verwendung in einem Kinderbuch\pwindex{Bonte, George Willard *~16.\,5.\,1873 Cincinnati@\textsc{Bonte, George Willard} (*~16.\,5.\,1873 Cincinnati), \emph{Illustrator}!ABC in Dixie. A Plantation Alphabet@\strich\emph{ABC in Dixie. A Plantation Alphabet}|pwkv}\pwindex{Quarles Bonte, Louise *~10.\,10.\,1873@\textsc{Quarles Bonte, Louise} (*~10.\,10.\,1873), \emph{Illustratorin}!ABC in Dixie. A Plantation Alphabet@\strich\emph{ABC in Dixie. A Plantation Alphabet}|pwkv} von Louise Quarles Bonte\pwindex{Quarles Bonte, Louise *~10.\,10.\,1873@\textsc{Quarles Bonte, Louise} (*~10.\,10.\,1873), \emph{Illustratorin}|pwk} und Georges Willard Bonte\pwindex{Bonte, George Willard *~16.\,5.\,1873 Cincinnati@\textsc{Bonte, George Willard} (*~16.\,5.\,1873 Cincinnati), \emph{Illustrator}|pwk}: \emph{ABC in Dixie. A Plantation Alphabet}\pwindex{Bonte, George Willard *~16.\,5.\,1873 Cincinnati@\textsc{Bonte, George Willard} (*~16.\,5.\,1873 Cincinnati), \emph{Illustrator}!ABC in Dixie. A Plantation Alphabet@\strich\emph{ABC in Dixie. A Plantation Alphabet}|pwk}\pwindex{Quarles Bonte, Louise *~10.\,10.\,1873@\textsc{Quarles Bonte, Louise} (*~10.\,10.\,1873), \emph{Illustratorin}!ABC in Dixie. A Plantation Alphabet@\strich\emph{ABC in Dixie. A Plantation Alphabet}|pwk}, das um diese
                     Zeit in London\oindex{London@\textbf{London}, \emph{Hauptstadt}|pwk} und New York\oindex{New York City@\textbf{New York City}|pwk} erschien, jedoch in Bayern\oindex{Bayern@\textbf{Bayern}, \emph{Land}|pwk} gedruckt wurde. Das deutet auf den aus Ober-Klingen\oindex{Ober-Klingen@\textbf{Ober-Klingen}|pwk} stammenden Verleger Ernest Nister\pwindex{Nister, Ernest 22.\,12.\,1842 Ober-Klingen – 26.\,5.\,1909 Nürnberg@\textsc{Nister, Ernest} (22.\,12.\,1842 Ober-Klingen – 26.\,5.\,1909 Nürnberg), \emph{Verleger}|pwk} als Vermittler
                     hin.}}}\label{K_L03447-1}{]}\pend
           \vspace{1em}
\pstart
           {\pb}\textsc{Hamburg}\oindex{Hamburg@\textbf{Hamburg}|pw}{ }2. Juli\pend
           \vspace{0.5em}
\pstart
           Herzliche Grüße!\pend
           \pstart \spacefill\mbox{P. G.}\pend{}\selectlanguage{ngerman}\endnumbering\briefempfaengerindex{Schnitzler, Arthur@\textsc{Schnitzler, Arthur}!zzzGoldmann, Paul@\emph{von Paul Goldmann}!1904-07-021@{2. 7. 1904}|)be}\mylabel{L03447h}  \newcommand{\dateiname}{L03447}\newcommand{\titel}{Paul Goldmann an Arthur Schnitzler, 2. 7. 1904}\newcommand{\editorInnen}{Martin Anton Müller und Laura Untner}%% latex-leseansicht-abspann.tex
%% Abspann für die Leseansicht.
%% Der Schalter \ifkorrekturansicht ist bereits durch den Vorspann gesetzt.

%% latex-abspann.tex
%% Gemeinsamer Abspann für Korrekturansicht und Leseansicht.
%% Setzt den Schalter \ifkorrekturansicht voraus (gesetzt in den
%% einbindenden Dateien latex-korrekturansicht-abspann.tex bzw.
%% latex-leseansicht-abspann.tex).
%% ---------------------------------------------------------------

\normalsize

% Das esempio-Environment wird nur in der Leseansicht benötigt
\ifkorrekturansicht\else
\newenvironment{esempio}[3]%
{
    \vspace{1.5ex}
    \rlap{\underline{#1}}
    \par
    \setlength{\parindent}{0cm}
    \nopagebreak
    \leftskip=#2cm
    \rightskip=#3cm
}
{
    \par
}
\fi

\doendnotes{C}
\bigskip
\vfill

\clearpage

\footnotesize

\ifkorrekturansicht
  \lohead{\textsc{register}}
\fi

% theindex-Environment neu definieren ohne reledmac
\makeatletter
\renewenvironment{theindex}{%
  \ifkorrekturansicht
    \section*{\indexname}%
  \else
    \subsubsection*{Index der erwähnten Entitäten}%
  \fi
  \setlength{\parindent}{0pt}%
  \setlength{\parskip}{0pt plus 0.3pt}%
  \let\item\@idxitem
}{%
  \ifkorrekturansicht\clearpage\fi
}
\makeatother

\IfFileExists{\jobname-pw.ind}{\input{\jobname-pw.ind}}{}

% Quellenangabe nur in der Leseansicht
\ifkorrekturansicht\else
% Fallback-Definitionen, falls die .tex-Datei \titel etc. nicht gesetzt hat
\providecommand{\titel}{}
\providecommand{\editorInnen}{}
\providecommand{\dateiname}{\jobname}

\vspace{3cm}

\vfill

\footnotesize
\textsc{Quelle}: \titel. Herausgegeben von {\editorInnen}. In: \emph{Arthur Schnitzler: Briefwechsel mit Autorinnen und Autoren}.
 Digitale Edition, https://schnitzler-briefe.acdh.oeaw.ac.at/{\dateiname}.html (Stand \today)
\fi

\end{document}


