%% latex-korrekturansicht-vorspann.tex
%% Vorspann für die Korrekturansicht.
%% Lädt die gemeinsame Datei latex-vorspann.tex mit gesetztem Schalter.

\newif\ifkorrekturansicht
\korrekturansichttrue

\input{../tex-inputs/latex-vorspann}


\section[ Paul Goldmann an Arthur Schnitzler, 2. 7. 1904]{L03447 Paul Goldmann an Arthur Schnitzler, 2. 7. 1904}
\nopagebreak\mylabel{L03447v}
\rehead{ }\normalsize\beginnumbering\briefempfaengerindex{Schnitzler, Arthur@\textsc{Schnitzler, Arthur}!zzzGoldmann, Paul@\emph{von Paul Goldmann}!1904-07-021@{2. 7. 1904}|(be}
\toendnotes[C]{\smallbreak\pagebreak[2]}\Standort{DLA, A:Schnitzler, HS.NZ85.1.3174.}
\physDesc{Bildpostkarte, 146 Zeichen
\newline{}Handschrift: 1) Bleistift, deutsche Kurrent\hspace{1em}2) Bleistift, lateinische Kurrent (\noindent{}Adresse)\hspace{1em}
\newline{}Versand: Stempel: »\nobreak{}\oindex{Hamburg@\textbf{Hamburg}, \emph{P.PPLA}|pwk}Hamburg 1 af, 2. 7. 04., 12–1N.\nobreak{}«.  }\toendnotes[C]{\smallbreak}\pstart{}{\pb}Herrn\pend{}\pstart{}Dr. Arthur Schnitzler\pend{}\pstart{}Wien\oindex{Wien@\textbf{Wien}, \emph{A.ADM2}|pw}\pend{}\pstart{}XVIII. Spöttelgaſse 7\oindex{Edmund-Weiss-Gasse 7@\textbf{Edmund-Weiß-Gasse 7}, \emph{Wohngebäude (K.WHS)}|pw}\pend{}{\bigskip}
\pstart
           \noindent{}\centering{}{\pb}{[}\label{K_L03447-1v}\edtext{Kunstdruck einer Wäscherin mit
                  schwarzer Hautfarbe von Bonte\pwindex{Quarles Bonte, Louise *~1873-10-10@\textsc{Quarles Bonte, Louise} (*~1873-10-10), \emph{Illustrator/Illustratorin}|pw}\pwindex{Bonte, George Willard *~1873-05-16@\textsc{Bonte, George Willard} (*~1873-05-16), \emph{Illustrator/Illustratorin}|pw}}{\lemma{\textnormal{\emph{Kunstdruck … Bonte}}}\Cendnote{\textnormal{Die Postkartenserie, aus der das
                     Motiv dieser Karte stammt, ist eine rassistische Seltsamkeit mit Stereotypen
                     schwarzer Plantagenarbeiterinnen und -arbeiter. Die meisten Darstellungen
                     fanden leicht modifiziert Verwendung in einem Kinderbuch\pwindex{ABC in Dixie. A Plantation Alphabet@\emph{ABC in Dixie. A Plantation Alphabet}|pwkv} von Louise Quarles Bonte\pwindex{Quarles Bonte, Louise *~1873-10-10@\textsc{Quarles Bonte, Louise} (*~1873-10-10), \emph{Illustrator/Illustratorin}|pwk} und Georges Willard Bonte\pwindex{Bonte, George Willard *~1873-05-16@\textsc{Bonte, George Willard} (*~1873-05-16), \emph{Illustrator/Illustratorin}|pwk}: \emph{ABC in Dixie. A Plantation Alphabet}\pwindex{ABC in Dixie. A Plantation Alphabet@\emph{ABC in Dixie. A Plantation Alphabet}|pwk}, das um diese
                     Zeit in London\oindex{London@\textbf{London}, \emph{P.PPLC}|pwk} und New York\oindex{New York City@\textbf{New York City}, \emph{P.PPL}|pwk} erschien, jedoch in Bayern\oindex{Bayern@\textbf{Bayern}, \emph{A.ADM1}|pwk} gedruckt wurde. Das deutet auf den aus Ober-Klingen\oindex{Ober-Klingen@\textbf{Ober-Klingen}, \emph{P.PPL}|pwk} stammenden Verleger Ernest Nister\pwindex{Nister, Ernest 1842-12-22 – 1909-05-26@\textsc{Nister, Ernest} (1842-12-22 – 1909-05-26), \emph{Verleger/Verlegerin}|pwk} als Vermittler
                     hin.}}}\label{K_L03447-1}{]}\pend
           \vspace{1em}
\pstart
           {\pb}\textsc{Hamburg}\oindex{Hamburg@\textbf{Hamburg}, \emph{P.PPLA}|pw}{ }2. Juli\pend
           \vspace{0.5em}
\pstart
           Herzliche Grüße!\pend
           \pstart \spacefill\mbox{P. G.}\pend{}\selectlanguage{ngerman}\endnumbering\briefempfaengerindex{Schnitzler, Arthur@\textsc{Schnitzler, Arthur}!zzzGoldmann, Paul@\emph{von Paul Goldmann}!1904-07-021@{2. 7. 1904}|)be}\mylabel{L03447h}  \normalsize

\doendnotes{C}
\bigskip
\vfill

\clearpage

\footnotesize

\lohead{\textsc{register}}

% Definiere theindex-Environment komplett neu ohne reledmac
\makeatletter
\renewenvironment{theindex}{%
  \section*{\indexname}%
  \setlength{\parindent}{0pt}%
  \setlength{\parskip}{0pt plus 0.3pt}%
  \let\item\@idxitem
}{%
  \clearpage
}
\makeatother

\IfFileExists{\jobname-pw.ind}{\input{\jobname-pw.ind}}{}

\end{document}

      