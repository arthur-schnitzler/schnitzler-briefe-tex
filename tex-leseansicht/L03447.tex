%% latex-leseansicht-vorspann.tex
%% Vorspann für die Leseansicht.
%% Lädt die gemeinsame Datei latex-vorspann.tex mit nicht gesetztem Schalter.

\newif\ifkorrekturansicht
\korrekturansichtfalse

\input{../tex-inputs/latex-vorspann}


         
         \renewcommand{\erwaehntePersonen}{Personen: George Willard Bonte, Ernest Nister, Louise Quarles Bonte}
         \renewcommand{\erwaehnteOrte}{Orte: Bayern, Edmund-Weiß-Gasse 7, Hamburg, London, New York City, Ober-Klingen, Wien}
         \renewcommand{\erwaehnteWerke}{Werke: ABC in Dixie. A Plantation Alphabet}
               \section[ Paul Goldmann an Arthur Schnitzler, 2. 7. 1904]{ Paul Goldmann an Arthur Schnitzler, 2. 7. 1904}\nopagebreak\mylabel{v}\rehead{ }\begin{ledgroupsized}[t]{13cm}\normalsize\beginnumbering \toendnotes[C]{\smallbreak\pagebreak[2]} \Standort{DLA, A:Schnitzler, HS.NZ85.1.3174.}
\physDesc{Bildpostkarte, 146 Zeichen
\newline{}Handschrift: 1) Bleistift, deutsche Kurrent\hspace{1em}2) Bleistift, lateinische Kurrent (\noindent{}Adresse)\hspace{1em}
\newline{}Versand: Stempel: »\nobreak{}\oindex{Hamburg@\textbf{Hamburg}|pwk}Hamburg 1 af, 2. 7. 04., 12–1N.\nobreak{}«.  }\toendnotes[C]{\smallbreak}\pstart{}{\pb}Herrn\pend{}\pstart{}Dr. Arthur Schnitzler\pend{}\pstart{}Wien\oindex{Wien@\textbf{Wien}|pw}\pend{}\pstart{}XVIII. Spöttelgaſse 7\oindex{Edmund-Weiss-Gasse 7@\textbf{Edmund-Weiß-Gasse 7}|pw}\pend{}{\bigskip}\pstart
           \noindent{}\centering{}{\pb}{[}\label{K_L03447-1v}\edtext{Kunstdruck einer Wäscherin mit
                     schwarzer Hautfarbe von Bonte\pwindex{Quarles Bonte, Louise *~1873-10-10@\textsc{Quarles Bonte, Louise} (*~1873-10-10), \emph{Illustratorin}|pw}\pwindex{Bonte, George Willard *~1873-05-16@\textsc{Bonte, George Willard} (*~1873-05-16), \emph{Illustrator}|pw}}{\lemma{\textnormal{\emph{Kunstdruck … Bonte}}}\Cendnote{\textnormal{Die Postkartenserie, aus der das
                        Motiv dieser Karte stammt, ist eine rassistische Seltsamkeit mit Stereotypen
                        schwarzer Plantagenarbeiterinnen und -arbeiter. Die meisten Darstellungen
                        fanden leicht modifiziert Verwendung in einem Kinderbuch\pwindex{Bonte, George Willard *~1873-05-16@\textsc{Bonte, George Willard} (*~1873-05-16), \emph{Illustrator}!ABC in Dixie. A Plantation Alphabet@\strich\emph{ABC in Dixie. A Plantation Alphabet}|pwkv}\pwindex{Quarles Bonte, Louise *~1873-10-10@\textsc{Quarles Bonte, Louise} (*~1873-10-10), \emph{Illustratorin}!ABC in Dixie. A Plantation Alphabet@\strich\emph{ABC in Dixie. A Plantation Alphabet}|pwkv} von Louise Quarles Bonte\pwindex{Quarles Bonte, Louise *~1873-10-10@\textsc{Quarles Bonte, Louise} (*~1873-10-10), \emph{Illustratorin}|pwk} und Georges Willard Bonte\pwindex{Bonte, George Willard *~1873-05-16@\textsc{Bonte, George Willard} (*~1873-05-16), \emph{Illustrator}|pwk}: \emph{ABC in Dixie. A Plantation Alphabet}\pwindex{Bonte, George Willard *~1873-05-16@\textsc{Bonte, George Willard} (*~1873-05-16), \emph{Illustrator}!ABC in Dixie. A Plantation Alphabet@\strich\emph{ABC in Dixie. A Plantation Alphabet}|pwk}\pwindex{Quarles Bonte, Louise *~1873-10-10@\textsc{Quarles Bonte, Louise} (*~1873-10-10), \emph{Illustratorin}!ABC in Dixie. A Plantation Alphabet@\strich\emph{ABC in Dixie. A Plantation Alphabet}|pwk}, das um die
                        Zeit in London\oindex{London@\textbf{London}|pwk} und New York\oindex{New York City@\textbf{New York City}|pwk} erschien, jedoch in Bayern\oindex{Bayern@\textbf{Bayern}|pwk} gedruckt wurde. Das deutet auf den aus Ober-Klingen\oindex{Ober-Klingen@\textbf{Ober-Klingen}|pwk} stammenden Verleger Ernest Nister\pwindex{Nister, Ernest 1842-12-22 – 1909-05-26@\textsc{Nister, Ernest} (1842-12-22 – 1909-05-26), \emph{Verleger}|pwk} als Vermittler
                        hin.}}}\label{K_L03447-1h}{]}\pend
           \pstart
           \textsc{Hamburg}\oindex{Hamburg@\textbf{Hamburg}|pw}{ }2. Juli\pend
           \pstart
           Herzliche Grüße!\pend
           \pstart \spacefill\mbox{P. G.}\pend{}
         
         \endnumbering\mylabel{h}\end{ledgroupsized}  \newcommand{\dateiname}{L03447}\newcommand{\titel}{Paul Goldmann an Arthur Schnitzler, 2. 7. 1904}\newcommand{\editorInnen}{Martin Anton Müller und Laura Untner}%% latex-leseansicht-abspann.tex
%% Abspann für die Leseansicht.
%% Der Schalter \ifkorrekturansicht ist bereits durch den Vorspann gesetzt.

%% latex-abspann.tex
%% Gemeinsamer Abspann für Korrekturansicht und Leseansicht.
%% Setzt den Schalter \ifkorrekturansicht voraus (gesetzt in den
%% einbindenden Dateien latex-korrekturansicht-abspann.tex bzw.
%% latex-leseansicht-abspann.tex).
%% ---------------------------------------------------------------

\normalsize

% Das esempio-Environment wird nur in der Leseansicht benötigt
\ifkorrekturansicht\else
\newenvironment{esempio}[3]%
{
    \vspace{1.5ex}
    \rlap{\underline{#1}}
    \par
    \setlength{\parindent}{0cm}
    \nopagebreak
    \leftskip=#2cm
    \rightskip=#3cm
}
{
    \par
}
\fi

\doendnotes{C}
\bigskip
\vfill

\clearpage

\footnotesize

\ifkorrekturansicht
  \lohead{\textsc{register}}
\fi

% theindex-Environment neu definieren ohne reledmac
\makeatletter
\renewenvironment{theindex}{%
  \ifkorrekturansicht
    \section*{\indexname}%
  \else
    \subsubsection*{Index der erwähnten Entitäten}%
  \fi
  \setlength{\parindent}{0pt}%
  \setlength{\parskip}{0pt plus 0.3pt}%
  \let\item\@idxitem
}{%
  \ifkorrekturansicht\clearpage\fi
}
\makeatother

\IfFileExists{\jobname-pw.ind}{\input{\jobname-pw.ind}}{}

% Quellenangabe nur in der Leseansicht
\ifkorrekturansicht\else
% Fallback-Definitionen, falls die .tex-Datei \titel etc. nicht gesetzt hat
\providecommand{\titel}{}
\providecommand{\editorInnen}{}
\providecommand{\dateiname}{\jobname}

\vspace{3cm}

\vfill

\footnotesize
\textsc{Quelle}: \titel. Herausgegeben von {\editorInnen}. In: \emph{Arthur Schnitzler: Briefwechsel mit Autorinnen und Autoren}.
 Digitale Edition, https://schnitzler-briefe.acdh.oeaw.ac.at/{\dateiname}.html (Stand \today)
\fi

\end{document}


      