%% latex-korrekturansicht-vorspann.tex
%% Vorspann für die Korrekturansicht.
%% Lädt die gemeinsame Datei latex-vorspann.tex mit gesetztem Schalter.

\newif\ifkorrekturansicht
\korrekturansichttrue

\input{../tex-inputs/latex-vorspann}


\section[ Paul Goldmann an Arthur Schnitzler, 19. 4. 1910]{L03470 Paul Goldmann an Arthur Schnitzler, 19. 4. 1910}
\nopagebreak\mylabel{L03470v}
\rehead{ }\normalsize\beginnumbering\briefempfaengerindex{Schnitzler, Arthur@\textsc{Schnitzler, Arthur}!zzzGoldmann, Paul@\emph{von Paul Goldmann}!1910-04-191@{19. 4. 1910}|(be}
\toendnotes[C]{\smallbreak\pagebreak[2]}\Standort{DLA, A:Schnitzler, HS.NZ85.1.3175.}
\physDesc{Brief, 1 Blatt, 1 Seite, 481 Zeichen
\newline{}Handschrift Schreibkraft: blaue Tinte, lateinische Kurrent
\newline{}Handschrift Paul Goldmann: blaue Tinte, lateinische Kurrent (\noindent{}Schlussformel und Unterschrift)}\toendnotes[C]{\smallbreak}
\pstart
           \raggedleft{}{\pb}19. 4. 10\pend
           
\pstart
           \raggedleft{}\textcolor{gray}{\textbf{W. Schöneberger-Ufer 34\oindex{Schoeneberger Ufer@\textbf{Schöneberger Ufer}, \emph{Straße (K.STR)}|pw}.}}\pend
           
\pstart{}Lieber Freund,\pend\vspace{0.5em}
\pstart
           Herr Victor Klemperer\pwindex{Klemperer, Victor 09.10.1881 – 11.02.1960@\textsc{Klemperer, Victor} (09.10.1881 – 11.02.1960), \emph{Romanist/Romanistin}|pw}, der Dir aus seinen
               vortrefflichen literarischen Essais ja schon literarisch bekannt ist, möchte Dich
               auch persönlich \label{K_L03470-1v}\edtext{kennen lernen}{\lemma{\textnormal{\emph{kennen lernen}}}\Cendnote{\textnormal{Am 27. 4. 1910 suchte Victor Klemperer\pwindex{Klemperer, Victor 09.10.1881 – 11.02.1960@\textsc{Klemperer, Victor} (09.10.1881 – 11.02.1960), \emph{Romanist/Romanistin}|pwk}{ }Schnitzler zu Hause auf. Klemperer\pwindex{Klemperer, Victor 09.10.1881 – 11.02.1960@\textsc{Klemperer, Victor} (09.10.1881 – 11.02.1960), \emph{Romanist/Romanistin}|pwk} veröffentlichte mehrere Aufsätze über Schnitzlers Werk, aber keine
                  Monografie.}}}\label{K_L03470-1} und hat mich um eine Einführung bei Dir gebeten, die ich ihm
               mit Vergnügen gebe. Ich höre, daß er beabsichtigt, auch ein Buch über Dich zu
               schreiben, und ich würde mich freuen, wenn dieses Werk zustande käme. Ich bitte Dich,
               Herrn Klemperer\pwindex{Klemperer, Victor 09.10.1881 – 11.02.1960@\textsc{Klemperer, Victor} (09.10.1881 – 11.02.1960), \emph{Romanist/Romanistin}|pw} freundlich aufzunehmen, und
               begrüße Dich herzlich.\pend
           
\pstart
           {[}hs. :{]} Dein {\\[\baselineskip]}\spacefill\mbox{Paul Goldmann.}\pend
           \leftskip=0em{}\selectlanguage{ngerman}\endnumbering\briefempfaengerindex{Schnitzler, Arthur@\textsc{Schnitzler, Arthur}!zzzGoldmann, Paul@\emph{von Paul Goldmann}!1910-04-191@{19. 4. 1910}|)be}\mylabel{L03470h}  \normalsize

\doendnotes{C}
\bigskip
\vfill

\clearpage

\footnotesize

\lohead{\textsc{register}}

% Definiere theindex-Environment komplett neu ohne reledmac
\makeatletter
\renewenvironment{theindex}{%
  \section*{\indexname}%
  \setlength{\parindent}{0pt}%
  \setlength{\parskip}{0pt plus 0.3pt}%
  \let\item\@idxitem
}{%
  \clearpage
}
\makeatother

\IfFileExists{\jobname-pw.ind}{\input{\jobname-pw.ind}}{}

\end{document}

      