%% latex-leseansicht-vorspann.tex
%% Vorspann für die Leseansicht.
%% Lädt die gemeinsame Datei latex-vorspann.tex mit nicht gesetztem Schalter.

\newif\ifkorrekturansicht
\korrekturansichtfalse

\input{../tex-inputs/latex-vorspann}

\begin{center}
            \textcolor{red}{ENTWURF, NICHT FERTIG KORRIGIERT}
                      \end{center}
            
         
         \renewcommand{\erwaehntePersonen}{Personen: Victor Klemperer}
         \renewcommand{\erwaehnteOrte}{Orte: Berlin, Schöneberger Ufer, Wien}
         \renewcommand{\erwaehnteWerke}{}
               \section[ Paul Goldmann an Arthur Schnitzler, 19. 4. 1910]{ Paul Goldmann an Arthur Schnitzler, 19. 4. 1910}\nopagebreak\mylabel{v}\rehead{ }\begin{ledgroupsized}[t]{13cm}\normalsize\beginnumbering \toendnotes[C]{\smallbreak\pagebreak[2]} \Standort{DLA, A:Schnitzler, HS.NZ85.1.3175.}
\physDesc{Brief, 1 Blatt, 1 Seite, 481 Zeichen
\newline{}Handschrift Schreibkraft: blaue Tinte, lateinische Kurrent\newline{}Handschrift Paul Goldmann: blaue Tinte, lateinische Kurrent (\noindent{}Schlussformel und Unterschrift)}\toendnotes[C]{\smallbreak}\pstart
           \raggedleft{}{\pb}19. 4. 10\pend
           \pstart
           \raggedleft{}\textcolor{gray}{\textbf{W. Schöneberger-Ufer 34\oindex{Schoeneberger Ufer@\textbf{Schöneberger Ufer}|pw}.}}\pend
           \pstart{}Lieber Freund,\pend\pstart
           Herr Victor Klemperer\pwindex{Klemperer, Victor 09.10.1881 – 11.02.1960@\textsc{Klemperer, Victor} (09.10.1881 – 11.02.1960), \emph{Romanist}|pw}, der Dir aus seinen
               vortrefflichen literarischen Essais ja schon literarisch bekannt ist, möchte Dich
               auch persönlich \label{K_L03470-1v}\edtext{kennen lernen}{\lemma{\textnormal{\emph{kennen lernen}}}\Cendnote{\textnormal{Am 27. 4. 1910 suchte Victor Klemperer\pwindex{Klemperer, Victor 09.10.1881 – 11.02.1960@\textsc{Klemperer, Victor} (09.10.1881 – 11.02.1960), \emph{Romanist}|pwk}{ }Schnitzler\pwindex{Schnitzler, Arthur 15.05.1862 – 21.10.1931@\textsc{Schnitzler, Arthur} (15.05.1862 – 21.10.1931), \emph{Schriftsteller, Mediziner}|pwk} zuhause auf. Klemperer\pwindex{Klemperer, Victor 09.10.1881 – 11.02.1960@\textsc{Klemperer, Victor} (09.10.1881 – 11.02.1960), \emph{Romanist}|pwk} veröffentlichte mehrere Aufsätze über Schnitzler\pwindex{Schnitzler, Arthur 15.05.1862 – 21.10.1931@\textsc{Schnitzler, Arthur} (15.05.1862 – 21.10.1931), \emph{Schriftsteller, Mediziner}|pwk}s Werk, aber keine
                  Monografie.}}}\label{K_L03470-1h} und hat mich um eine Einführung bei Dir gebeten, die ich ihm
               mit Vergnügen gebe. Ich höre, daß er beabsichtigt, auch ein Buch über Dich zu
               schreiben, und ich würde mich freuen, wenn dieses Werk zustande käme. Ich bitte Dich,
               Herrn Klemperer\pwindex{Klemperer, Victor 09.10.1881 – 11.02.1960@\textsc{Klemperer, Victor} (09.10.1881 – 11.02.1960), \emph{Romanist}|pw} freundlich aufzunehmen, und
               begrüße Dich herzlich.\pend
           \pstart
           {[}hs. Goldmann:{]} Dein {\\[\baselineskip]}\spacefill\mbox{Paul Goldmann.}\pend
           \leftskip=0em{}
         
         \endnumbering\mylabel{h}\end{ledgroupsized}  \newcommand{\dateiname}{L03470}\newcommand{\titel}{Paul Goldmann an Arthur Schnitzler, 19. 4. 1910}\newcommand{\editorInnen}{Martin Anton Müller und Laura Untner}%% latex-leseansicht-abspann.tex
%% Abspann für die Leseansicht.
%% Der Schalter \ifkorrekturansicht ist bereits durch den Vorspann gesetzt.

%% latex-abspann.tex
%% Gemeinsamer Abspann für Korrekturansicht und Leseansicht.
%% Setzt den Schalter \ifkorrekturansicht voraus (gesetzt in den
%% einbindenden Dateien latex-korrekturansicht-abspann.tex bzw.
%% latex-leseansicht-abspann.tex).
%% ---------------------------------------------------------------

\normalsize

% Das esempio-Environment wird nur in der Leseansicht benötigt
\ifkorrekturansicht\else
\newenvironment{esempio}[3]%
{
    \vspace{1.5ex}
    \rlap{\underline{#1}}
    \par
    \setlength{\parindent}{0cm}
    \nopagebreak
    \leftskip=#2cm
    \rightskip=#3cm
}
{
    \par
}
\fi

\doendnotes{C}
\bigskip
\vfill

\clearpage

\footnotesize

\ifkorrekturansicht
  \lohead{\textsc{register}}
\fi

% theindex-Environment neu definieren ohne reledmac
\makeatletter
\renewenvironment{theindex}{%
  \ifkorrekturansicht
    \section*{\indexname}%
  \else
    \subsubsection*{Index der erwähnten Entitäten}%
  \fi
  \setlength{\parindent}{0pt}%
  \setlength{\parskip}{0pt plus 0.3pt}%
  \let\item\@idxitem
}{%
  \ifkorrekturansicht\clearpage\fi
}
\makeatother

\IfFileExists{\jobname-pw.ind}{\input{\jobname-pw.ind}}{}

% Quellenangabe nur in der Leseansicht
\ifkorrekturansicht\else
% Fallback-Definitionen, falls die .tex-Datei \titel etc. nicht gesetzt hat
\providecommand{\titel}{}
\providecommand{\editorInnen}{}
\providecommand{\dateiname}{\jobname}

\vspace{3cm}

\vfill

\footnotesize
\textsc{Quelle}: \titel. Herausgegeben von {\editorInnen}. In: \emph{Arthur Schnitzler: Briefwechsel mit Autorinnen und Autoren}.
 Digitale Edition, https://schnitzler-briefe.acdh.oeaw.ac.at/{\dateiname}.html (Stand \today)
\fi

\end{document}


      