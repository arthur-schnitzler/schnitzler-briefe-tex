%% latex-korrekturansicht-vorspann.tex
%% Vorspann für die Korrekturansicht.
%% Lädt die gemeinsame Datei latex-vorspann.tex mit gesetztem Schalter.

\newif\ifkorrekturansicht
\korrekturansichttrue

\input{../tex-inputs/latex-vorspann}


\section[ Paul Goldmann an Arthur Schnitzler, 22. 3. {[}1906{]}]{L03241 Paul Goldmann an Arthur Schnitzler, 22. 3. {[}1906{]}}
\nopagebreak\mylabel{L03241v}
\rehead{ }\normalsize\beginnumbering\briefempfaengerindex{Schnitzler, Arthur@\textsc{Schnitzler, Arthur}!zzzGoldmann, Paul@\emph{von Paul Goldmann}!1906-03-221@{22. 3. {[}1906{]}}|(be}
\toendnotes[C]{\smallbreak\pagebreak[2]}\Standort{DLA, A:Schnitzler, HS.NZ85.1.3175.}
\physDesc{Brief, 1 Blatt, 2 Seiten, 500 Zeichen
\newline{}Handschrift: blaue Tinte, deutsche Kurrent
\newline{}Schnitzler: mit Bleistift das Jahr »906« vermerkt }\toendnotes[C]{\smallbreak}
\pstart
           \raggedleft{}{\pb}\textcolor{gray}{\textbf{DESSAUERSTRASSE 19}}\oindex{Dessauer Strasse@\textbf{Dessauer Straße}, \emph{Straße (K.STR)}|pw}\pend
           
\pstart
           Berlin\oindex{Berlin@\textbf{Berlin}, \emph{P.PPLC}|pw}, 22. März.\pend
           
\pstart{}Mein lieber Freund,\pend\vspace{0.5em}
\pstart
           Ich habe mich ſehr über die Zuſendung Deines neuen Werk\pwindex{Marionetten. Drei Einakter@\emph{Marionetten. Drei Einakter}|pwv}es gefreut und danke Dir von Herzen für das Buch\pwindex{Marionetten. Drei Einakter@\emph{Marionetten. Drei Einakter}|pwv} und ganz beſonders für die
                  \label{K_L03241-1v}\edtext{Widmung}{\lemma{\textnormal{\emph{Widmung}}}\Cendnote{\textnormal{Auch Hermann Bahr\pwindex{Bahr, Hermann 19.07.1863 – 15.01.1934@\textsc{Bahr, Hermann} (19.07.1863 – 15.01.1934), \emph{Schriftsteller/Schriftstellerin, Kritiker/Kritikerin}|pwk} und
                     Hugo von Hofmannsthal\pwindex{Hofmannsthal, Hugo von 1874-02-01 – 1929-07-15@\textsc{Hofmannsthal, Hugo von} (1874-02-01 – 1929-07-15), \emph{Schriftsteller/Schriftstellerin}|pwk} erhielten
                  Widmungsexemplare von Schnitzlers Einakterband \emph{Marionetten}\pwindex{Marionetten. Drei Einakter@\emph{Marionetten. Drei Einakter}|pwk}, vgl. Arthur Schnitzler: Widmungsexemplar Marionetten für Hermann Bahr,
               23. 3. 1906 und Arthur Schnitzler: Widmungsexemplar Marionetten für Hugo von
               Hofmannsthal, [23.?] 3. 1906.}}}\label{K_L03241-1}.\pend
           
\pstart
           Ob ich Dir werde Oſtern in Wien\oindex{Wien@\textbf{Wien}, \emph{A.ADM2}|pw} die Hand drücken können, iſt \substVorne{}\textsuperscript{doch}\substDazwischen{}wieder\substHinten{} ſehr ungewiß geworden. Wahrſcheinlich komme ich {\pb}zu Oſtern überhaupt
               nicht von hier\oindex{Berlin@\textbf{Berlin}, \emph{P.PPLC}|pwv}{ }\label{K_L03241-2v}\edtext{fort}{\lemma{\textnormal{\emph{fort}}}\Cendnote{\textnormal{Goldmann\pwindex{Goldmann, Paul 31.01.1865 – 25.09.1935@\textsc{Goldmann, Paul} (31.01.1865 – 25.09.1935), \emph{Schriftsteller/Schriftstellerin, Journalist/Journalistin}|pwk} reiste zu Ostern 1906 nicht nach Wien\oindex{Wien@\textbf{Wien}, \emph{A.ADM2}|pwk}. Er und Schnitzler sahen sich dort erst am 4. 6. 1906 und am
                     10. 6. 1906
                  wieder.}}}\label{K_L03241-2}.\pend
           
\pstart
           Es hat mich ſehr gefreut, vom Erfolg des »Großen
                  Wurſtl\pwindex{Zum grossen Wurstel. Burleske in einem Akt@\emph{Zum großen Wurstel. Burleske in einem Akt}|pw}« in der N. Fr. Pr.\pwindex{Neue Freie Presse@\emph{Neue Freie Presse}|pw} zu \label{K_L03241-3v}\edtext{leſen}{\lemma{\textnormal{\emph{leſen}}}\Cendnote{\textnormal{R. A.\pwindex{Auernheimer, Raoul 15.04.1876 – 06.01.1948@\textsc{Auernheimer, Raoul} (15.04.1876 – 06.01.1948), \emph{Schriftsteller/Schriftstellerin, Journalist/Journalistin, Kritiker/Kritikerin}|pwk} [ = Raoul Auernheimer\pwindex{Auernheimer, Raoul 15.04.1876 – 06.01.1948@\textsc{Auernheimer, Raoul} (15.04.1876 – 06.01.1948), \emph{Schriftsteller/Schriftstellerin, Journalist/Journalistin, Kritiker/Kritikerin}|pwk}]: \emph{Theater- und Kunstnachrichten. [Lustspieltheater, literarischer
                        Einakterabend]}\pwindex{Theater- und Kunstnachrichten. [Lustspieltheater, literarischer Einakterabend.]@\emph{Theater- und Kunstnachrichten. [Lustspieltheater, literarischer Einakterabend.]}|pwk}. In: \emph{Neue Freie
                        Presse}\pwindex{Neue Freie Presse@\emph{Neue Freie Presse}|pwk}, Nr. 14.930, 17. 3. 1906,
                     Morgenblatt, S. 13.}}}\label{K_L03241-3}.\pend
           
\pstart
           Alſo nochmals herzlichſten Dank und viele Grüße an Dich, Frau\pwindex{Schnitzler, Olga 17.01.1882 – 13.01.1970@\textsc{Schnitzler, Olga} (17.01.1882 – 13.01.1970), \emph{Schauspieler/Schauspielerin, Sänger/Sängerin}|pwv} und Kind\pwindex{Schnitzler, Heinrich 09.08.1902 – 12.07.1982@\textsc{Schnitzler, Heinrich} (09.08.1902 – 12.07.1982), \emph{Regisseur/Regisseurin, Schauspieler/Schauspielerin}|pwv} von {\\[\baselineskip]}Deinem getreuen {\\[\baselineskip]}\spacefill\mbox{Paul Goldmann.}\pend
           \leftskip=0em{}\selectlanguage{ngerman}\endnumbering\briefempfaengerindex{Schnitzler, Arthur@\textsc{Schnitzler, Arthur}!zzzGoldmann, Paul@\emph{von Paul Goldmann}!1906-03-221@{22. 3. {[}1906{]}}|)be}\mylabel{L03241h}  \normalsize

\doendnotes{C}
\bigskip
\vfill

\clearpage

\footnotesize

\lohead{\textsc{register}}

% Definiere theindex-Environment komplett neu ohne reledmac
\makeatletter
\renewenvironment{theindex}{%
  \section*{\indexname}%
  \setlength{\parindent}{0pt}%
  \setlength{\parskip}{0pt plus 0.3pt}%
  \let\item\@idxitem
}{%
  \clearpage
}
\makeatother

\IfFileExists{\jobname-pw.ind}{\input{\jobname-pw.ind}}{}

\end{document}

      