%% latex-korrekturansicht-vorspann.tex
%% Vorspann für die Korrekturansicht.
%% Lädt die gemeinsame Datei latex-vorspann.tex mit gesetztem Schalter.

\newif\ifkorrekturansicht
\korrekturansichttrue

\input{../tex-inputs/latex-vorspann}


\section[Robert Adam an Arthur Schnitzler, 25. 2. 1909]{L01830 Robert Adam an Arthur Schnitzler, 25. 2. 1909}
\nopagebreak\mylabel{L01830v}
\rehead{ }\normalsize\beginnumbering\briefempfaengerindex{Schnitzler, Arthur@\textsc{Schnitzler, Arthur}!zzzAdam, Robert@\emph{von Robert Adam}!1909-02-251@{25. 2. 1909}|(be}
\toendnotes[C]{\smallbreak\pagebreak[2]}\Standort{DLA, A:Schnitzler, HS.NZ85.1.4230,1.}
\physDesc{Brief, 1 Blatt, 1 Seite, 313 Zeichen
\newline{}Handschrift: schwarze Tinte, deutsche Kurrent
\newline{}Schnitzler: 1) mit Bleistift beschriftet: »\textsc{Ada\textcolor{gray}{m}}«  2) mit rotem Buntstift zwei Unterstreichungen}\toendnotes[C]{\smallbreak}
\pstart
           \raggedleft{}{\pb}Wien\oindex{Wien@\textbf{Wien}, \emph{A.ADM2}|pw}, am 25. Febr. 1909\pend
           
\pstart{}Hochverehrter Herr Doktor!\pend\vspace{0.5em}
\pstart
           Ich bin ſo frei, Ihnen als Zeichen meiner Hochſchätzung meine Komödie: »\label{K_L01830-1v}\edtext{Die Geſchichte des Alî ibn Bekkâr mit Schams
                  an-Nahâr\pwindex{Geschichte des Alî ibn Bekkâr mit Schams an-Nahâr@\emph{Die Geschichte des Alî ibn Bekkâr mit Schams an-Nahâr}|pw}}{\lemma{\textnormal{\emph{Die … an-Nahâr}}}\Cendnote{\textnormal{\emph{Die Geschichte des Alî ibn Bekkâr mit Schams
                        an-Nahâr}\pwindex{Geschichte des Alî ibn Bekkâr mit Schams an-Nahâr@\emph{Die Geschichte des Alî ibn Bekkâr mit Schams an-Nahâr}|pwk}. Eine Komödie von Robert
                        Adam\pwindex{Adam, Robert 20.04.1877 – 16.10.1961@\textsc{Adam, Robert} (20.04.1877 – 16.10.1961), \emph{Schriftsteller/Schriftstellerin, Richter/Richterin}|pwk}. Wien und Leipzig: \emph{Hugo Heller {\kaufmannsund} Cie.}\orgindex{Hugo Heller@Hugo Heller|pwk}{ }1909 (erschienen im Februar).}}}\label{K_L01830-1}« zu überſenden und bitte Sie,
               mein Buch Ihrer Aufmerkſamkeit für wert zu erachten.\pend
           
\pstart
           Ihr ergebener{\\[\baselineskip]}\spacefill\mbox{Robert Adam}\pend
           \leftskip=0em{}
\pstart
           \noindent{}Wien XII/\textsubscript{1} Meidlinger
                     Hauptſtr. 56\oindex{Meidlinger Hauptstrasse@\textbf{Meidlinger Hauptstraße}, \emph{Straße (K.STR)}|pw}\pend
           \selectlanguage{ngerman}\endnumbering\briefempfaengerindex{Schnitzler, Arthur@\textsc{Schnitzler, Arthur}!zzzAdam, Robert@\emph{von Robert Adam}!1909-02-251@{25. 2. 1909}|)be}\mylabel{L01830h}  \normalsize

\doendnotes{C}
\bigskip
\vfill

\clearpage

\footnotesize

\lohead{\textsc{register}}

% Definiere theindex-Environment komplett neu ohne reledmac
\makeatletter
\renewenvironment{theindex}{%
  \section*{\indexname}%
  \setlength{\parindent}{0pt}%
  \setlength{\parskip}{0pt plus 0.3pt}%
  \let\item\@idxitem
}{%
  \clearpage
}
\makeatother

\IfFileExists{\jobname-pw.ind}{\input{\jobname-pw.ind}}{}

\end{document}

      