%% latex-korrekturansicht-vorspann.tex
%% Vorspann für die Korrekturansicht.
%% Lädt die gemeinsame Datei latex-vorspann.tex mit gesetztem Schalter.

\newif\ifkorrekturansicht
\korrekturansichttrue

\input{../tex-inputs/latex-vorspann}


\section[Arthur Schnitzler an Georg Brandes, 18. 7. 1897]{L00705 Arthur Schnitzler an Georg Brandes, 18. 7. 1897}
\nopagebreak\mylabel{L00705v}
\rehead{ }\normalsize\beginnumbering\briefempfaengerindex{Brandes, Georg@\textsc{Brandes, Georg}!zzzSchnitzler, Arthur@\emph{von Arthur Schnitzler}!1897-07-181@{18. 7. 1897}|(be}
\toendnotes[C]{\smallbreak\pagebreak[2]}\Standort{Kopenhagen, Det Kongelige Bibliotek, Georg Brandes Arkiv, box 125.}
\physDesc{Brief, 2 Blätter, 7 Seiten, 2406 Zeichen
\newline{}Handschrift: schwarze Tinte, deutsche Kurrent
\newline{}Ordnung: mit Bleistift von unbekannter Hand nummeriert: »9.
                                    Schnitzler«, das zweite Blatt mit »18/7 97« gekennzeichnet }
\buchAbdrucke{\weitereDrucke{Georg Brandes, Arthur Schnitzler: \emph{Ein Briefwechsel}. Bern: \emph{Francke} 1956, S. 64–65.} }
\pstart
           \raggedleft{}{\pb}\textsc{Ischl}\oindex{Bad Ischl@\textbf{Bad Ischl}, \emph{P.PPL}|pw}, 18. 7. 97. \pend
           
\pstart{}Verehrteſter Herr Brandes,\pend\vspace{0.5em}
\pstart
           Ich danke Ihnen herzlich, dſs Sie mir ſo ſchnell eine Nachricht haben zugehen laſſen.
               Vor allem entnehme ich ihr, daſs jede Gefahr vorüber iſt, und das iſt ja das
               weſentliche. Auch ſcheint es, dſs Sie ſchon wieder arbeiten dürfen – und ſogar ſich
               aergern – we{\geminationn} das mit aerztlicher {\pb}Erlaubnis geſchieht? Aber mir ſcheint wirklich,
               Sie ſind mit den deutſchen Überſetzungen ein bischen gar zu ſtreng – die Leute, die
               nicht das Glück haben, Überſetzungen Ihrer Bücher mit dem Urtext vergleichen zu
               können, finden auch in dieſen Überſetzungen irgend was und ſogar ſehr viel, das \introOben{}ihnen\introOben{} trotz Misverſtändniſſen u Flüchtigkeiten (die ja uns
                  \introOben{}großentheils\introOben{} entgehen) der ganze Georg Brandes zu ſein
               ſcheint. {\pb}Freilich ahnt man oft, daſs hier ein
               Zauber verloren gegangen iſt, der unwiederbringlich iſt; – aber glauben Sie mir, es
               bleibt noch i{\geminationm}er ſo viel Zauber übrig, daſs die meiſten
               gar nicht dazu ko{\geminationm}en, den fehlenden zu vermiſſen. Ich
               gehöre ja leider auch zu denen, die nicht däniſch\oindex{Daenemark@\textbf{Dänemark}, \emph{A.PCLI}|pw} verſtehn – und Sie haben mir noch jedesmal, durch die ſchwächſten
               Übertragungen hindurch, wahrhaftig {\pb}viel
               gegeben!\pend
           
\pstart
           Ich wuſste nicht, dſs Paul Goldmann\pwindex{Goldmann, Paul 31.01.1865 – 25.09.1935@\textsc{Goldmann, Paul} (31.01.1865 – 25.09.1935), \emph{Schriftsteller/Schriftstellerin, Journalist/Journalistin}|pw} Ihnen
               ſchon lange Zeit nicht geſchrieben hat. Aber Sie können kaum ahnen, was dieſer Mann
               zu thun hat. Ich bin im Frühjahr in Paris\oindex{Paris@\textbf{Paris}, \emph{P.PPLC}|pw}
               geweſen, und habe manche Tage mit ihm verbracht; er ko{\geminationm}t
               überhaupt kaum je eine Viertelſtunde zur Ruhe. Allerdings hat er etwas zu viel
               Gewiſſen und opfert meiner An{\pb}ſicht nach der Frankf. Zeitg\orgindex{Frankfurter Zeitung@Frankfurter Zeitung|pw} mehr von dem beſten ſeines Lebens
               auf, als ſie ihm je danken wird. Da der Gruſs an meine Freunde wohl ihm und Dr. \textsc{Beer-Hofma{\geminationn}}\pwindex{Beer-Hofmann, Richard 1866-07-11 – 1945-09-26@\textsc{Beer-Hofmann, Richard} (1866-07-11 – 1945-09-26), \emph{Schriftsteller/Schriftstellerin}|pw} gilt, hab ich ihn beiden mitgetheilt. Dr \textsc{B. H.}\pwindex{Beer-Hofmann, Richard 1866-07-11 – 1945-09-26@\textsc{Beer-Hofmann, Richard} (1866-07-11 – 1945-09-26), \emph{Schriftsteller/Schriftstellerin}|pw} iſt hier und dankt Ihnen vielmals; er verbindet ſeine beſten Wünſche für Ihre
               baldige vollko{\geminationm}ene Geneſung mit den meinen.\pend
           
\pstart
           {\pb}Eine Frage an Sie hatte ich mir ſchon neulich
               vorgenommen: Haben Sie die Skizzen von \textsc{Altenberg}\pwindex{Altenberg, Peter 09.03.1859 – 08.01.1919@\textsc{Altenberg, Peter} (09.03.1859 – 08.01.1919), \emph{Schriftsteller/Schriftstellerin}|pw} geleſen? (Es iſt ein Buch: »Wie ich es
                  ſehe\pwindex{Wie ich es sehe@\emph{Wie ich es sehe}|pw},« der Autor hat es Ihnen wohl geſchickt.)\pend
           
\pstart
           Ich ſchreibe jetzt, nach einigen kleinern Erzählungen, wieder ein Stück und habe mehr
               Freude daran als von meinem letzten. Ob es beſſer wird, \strikeout{f} weiſs ich freilich {\pb}noch nicht. Aber
               das Freudhaben iſt ja doch das wichtigere. –\pend
           
\pstart
           In wenigen Tagen fahre ich wieder nach Wien\oindex{Wien@\textbf{Wien}, \emph{A.ADM2}|pw}
               zurück; vielleicht erfreuen Sie mich bald wieder durch ein Wort; und wär es auch nur
               das eine »Geſundheit.«\pend
           \pstart Ich grüße Sie, hochverehrter Herr Brandes, in herzlichſter Ergebenheit.
                  \spacefill\mbox{ArthurSchnitzler}\pend{}\selectlanguage{ngerman}\endnumbering\briefempfaengerindex{Brandes, Georg@\textsc{Brandes, Georg}!zzzSchnitzler, Arthur@\emph{von Arthur Schnitzler}!1897-07-181@{18. 7. 1897}|)be}\mylabel{L00705h}  \normalsize

\doendnotes{C}
\bigskip
\vfill

\clearpage

\footnotesize

\lohead{\textsc{register}}

% Definiere theindex-Environment komplett neu ohne reledmac
\makeatletter
\renewenvironment{theindex}{%
  \section*{\indexname}%
  \setlength{\parindent}{0pt}%
  \setlength{\parskip}{0pt plus 0.3pt}%
  \let\item\@idxitem
}{%
  \clearpage
}
\makeatother

\IfFileExists{\jobname-pw.ind}{\input{\jobname-pw.ind}}{}

\end{document}

      