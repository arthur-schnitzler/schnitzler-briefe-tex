\input{../tex-inputs/latex-pdf-vorspann}
\begin{center}
            \textcolor{red}{ENTWURF. ENTZIFFERUNG NOCH NICHT KORREKTURGELESEN}
                      \end{center}
            
               \section[Richard Beer-Hofmann an Arthur Schnitzler, 30. {[}10. 1896{]}]{ Richard Beer-Hofmann an Arthur Schnitzler, 30. {[}10. 1896{]}}\nopagebreak\mylabel{v}\rehead{ }\begin{ledgroupsized}[t]{13cm}\normalsize\beginnumbering\briefempfaengerindex{Schnitzler, Arthur@\textsc{Schnitzler, Arthur}!zzzBeer-Hofmann, Richard@\emph{von Richard Beer-Hofmann}!1896-10-302@{30. {[}10. 1896{]}}|(be} \toendnotes[C]{\smallbreak\pagebreak[2]} \Standort{CUL, Schnitzler, B 8.}
\physDesc{Telegramm
\newline{}maschinell\newline{}Ordnung: mit Bleistift von unbekannter Hand
                              nummeriert: »88« }\toendnotes[C]{\smallbreak}\pstart
           \noindent{}{\pb}b\oindex{Berlin@\textbf{Berlin}|pw}{ }de{ }wien\oindex{Wien@\textbf{Wien}|pw} 111.–529 16 6{ }30–\pend
           \pstart
           den schoensten erfolg\pwindex{Schnitzler, Arthur 15.05.1862 – 21.10.1931@\textsc{Schnitzler, Arthur} (15.05.1862 – 21.10.1931), \emph{Schriftsteller, Mediziner}!Freiwild. Schauspiel in 3 Akten1896@\strich\emph{Freiwild. Schauspiel in 3 Akten} {[}1896{]}|pwv} und herzliche gruesse von
               dem halbwahren aus upsala\oindex{Uppsala@\textbf{Uppsala}|pw} +\pend
           \endnumbering\briefempfaengerindex{Schnitzler, Arthur@\textsc{Schnitzler, Arthur}!zzzBeer-Hofmann, Richard@\emph{von Richard Beer-Hofmann}!1896-10-302@{30. {[}10. 1896{]}}|)be}\mylabel{h}\end{ledgroupsized}  \newcommand{\dateiname}{L00613}\newcommand{\titel}{Richard Beer-Hofmann an Arthur Schnitzler, 30. [10. 1896]}\newcommand{\editorInnen}{Martin Anton Müller und Gerd-Hermann Susen}\input{../tex-inputs/latex-pdf-abspann}
      