%% latex-korrekturansicht-vorspann.tex
%% Vorspann für die Korrekturansicht.
%% Lädt die gemeinsame Datei latex-vorspann.tex mit gesetztem Schalter.

\newif\ifkorrekturansicht
\korrekturansichttrue

\input{../tex-inputs/latex-vorspann}


\section[Richard Beer-Hofmann an Arthur Schnitzler, 30. {[}10. 1896{]}]{L00613 Richard Beer-Hofmann an Arthur Schnitzler, 30. {[}10. 1896{]}}
\nopagebreak\mylabel{L00613v}
\rehead{ }\normalsize\beginnumbering\briefempfaengerindex{Schnitzler, Arthur@\textsc{Schnitzler, Arthur}!zzzBeer-Hofmann, Richard@\emph{von Richard Beer-Hofmann}!1896-10-302@{30. {[}10. 1896{]}}|(be}
\toendnotes[C]{\smallbreak\pagebreak[2]}\Standort{CUL, Schnitzler, B 8.}
\physDesc{Telegramm, 96 Zeichen
\newline{}maschinell
\newline{}Ordnung: mit Bleistift von unbekannter Hand
                                    nummeriert: »88« }\toendnotes[C]{\smallbreak}
\pstart
           \noindent{}{\pb}b\oindex{Berlin@\textbf{Berlin}, \emph{P.PPLC}|pw}{ }de{ }wien\oindex{Wien@\textbf{Wien}, \emph{A.ADM2}|pw} 111.–529 16 6{ }30–\pend
           
\pstart
           den schoensten erfolg\pwindex{Freiwild. Schauspiel in 3 Akten@\emph{Freiwild. Schauspiel in 3 Akten}|pwv} und
               herzliche gruesse von dem halbwahren aus upsala\oindex{Uppsala@\textbf{Uppsala}, \emph{A.ADM4}|pw}
               +\pend
           \selectlanguage{ngerman}\endnumbering\briefempfaengerindex{Schnitzler, Arthur@\textsc{Schnitzler, Arthur}!zzzBeer-Hofmann, Richard@\emph{von Richard Beer-Hofmann}!1896-10-302@{30. {[}10. 1896{]}}|)be}\mylabel{L00613h}  \normalsize

\doendnotes{C}
\bigskip
\vfill

\clearpage

\footnotesize

\lohead{\textsc{register}}

% Definiere theindex-Environment komplett neu ohne reledmac
\makeatletter
\renewenvironment{theindex}{%
  \section*{\indexname}%
  \setlength{\parindent}{0pt}%
  \setlength{\parskip}{0pt plus 0.3pt}%
  \let\item\@idxitem
}{%
  \clearpage
}
\makeatother

\IfFileExists{\jobname-pw.ind}{\input{\jobname-pw.ind}}{}

\end{document}

      