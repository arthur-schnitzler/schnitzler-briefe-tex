%% latex-leseansicht-vorspann.tex
%% Vorspann für die Leseansicht.
%% Lädt die gemeinsame Datei latex-vorspann.tex mit nicht gesetztem Schalter.

\newif\ifkorrekturansicht
\korrekturansichtfalse

\input{../tex-inputs/latex-vorspann}


         
         \renewcommand{\erwaehntePersonen}{Personen: Richard Beer-Hofmann, Johann Wolfgang von Goethe, Friedrich Maximilian von Klinger, Friedrich von Schiller, William Shakespeare}
         \renewcommand{\erwaehnteInstitutionen}{Institutionen: Manz’sche Verlags- und Universitätsbuchhandlung}
         \renewcommand{\erwaehnteOrte}{Orte: Bad Fusch, Bad Ischl, Karlsbad, Kohlmarkt}
         \renewcommand{\erwaehnteWerke}{Werke: Anatol, Briefwechsel zwischen Schiller und Goethe, Gestern. Dramatische Studie in einem Akt in Versen}
               \section[Hugo von Hofmannsthal an Richard Beer-Hofmann und Arthur Schnitzler, 8. 7. 1893]{ Hugo von Hofmannsthal an Richard Beer-Hofmann und Arthur Schnitzler,
               8. 7. 1893}\nopagebreak\mylabel{v}\rehead{ }\begin{ledgroupsized}[t]{13cm}\normalsize\beginnumbering \toendnotes[C]{\smallbreak\pagebreak[2]} \Standort{YCGL, MSS 32.}
\physDesc{Brief, 1 Blatt, 3 Seiten
\newline{}Handschrift: Bleistift, deutsche Kurrent\newline{}Ordnung: mit rotem Buntstift von unbekannter Hand datiert: »8. VII. 1893–13« }\buchAbdrucke{\weitereDrucke{Hugo von Hofmannsthal, Richard Beer-Hofmann: \emph{Briefwechsel}. Hg. Eugene Weber. Frankfurt am Main: \emph{S. Fischer} 1972, S. 23.} }\pstart
           \raggedleft{}{\pb}Fuſch\oindex{Bad Fusch@\textbf{Bad Fusch}|pw}, 8 Juli 93.\pend
           \pstart{}lieber Richard und Arthur!\pend\pstart
           Ich brauch Euch wohl nicht zu ſagen, wie ich mich freue, daſs endlich einmal ein paar
               von den graciöſen Schatten aus dem Anatolbuch\pwindex{Schnitzler, Arthur 15.05.1862 – 21.10.1931@\textsc{Schnitzler, Arthur} (15.05.1862 – 21.10.1931), \emph{Schriftsteller, Mediziner}!Anatol1892-10-29@\strich\emph{Anatol} {[}1892-10-29{]}|pw} bei
               Sommerſonne und Lampenlicht lebendig werden ſollen. Ich käme hin, wäre ich nicht
               gerade beim zaghaften Anfang einer Erholung meines etwas in Unordnung gerathenen ſog.
               Nervenſyſtems.\pend
           \pstart
           Es thut mir merkwürdig wohl, ohne Kaffeehaus, ohne Geſelligkeit, ohne etwas das
               treibt oder bindet, ſo vor mich hin zu dämmern, {\pb}in
               lauen Bädern beinahe einzuſchlafen und \textsc{Shakespeare\pwindex{Shakespeare, William 23.4.1564? – 03.05.1616@\textsc{Shakespeare, William} (23.4.1564? – 03.05.1616), \emph{Schauspieler, Dramatiker}|pw}’sche Comödien} zu leſen,
               während kleine Katzen in der Sonne mit einem Knäuel Wolle ſpielen. Am liebſten war
               mir, Ihr möchtet am \substVorne{}\textsuperscript{m}\substDazwischen{}M\substHinten{}orgen drauf telegrafieren; jedenfalls ſchickt mir, was Ihr an \strikeout{ſonſti} localen und ſonſtigen Recenſionen bekommt,
               wenigſtens zum Anſehen hierher; ich ſchicke Euch doch auch immer alles von mir.\pend
           \pstart
           »Geſtern\pwindex{Gestern. Dramatische Studie in einem Akt in Versen15. 10. 1891@\emph{Gestern. Dramatische Studie in einem Akt in Versen} {[}15. 10. 1891{]}|pw}« hab ich nicht mit; wenn Richard es
               braucht, soll er an Manz\orgindex{Manz sche Verlags- und Universitaetsbuchhandlung@Manz’sche Verlags- und Universitätsbuchhandlung|pw} (\textsc{Kohlmarkt}\oindex{Kohlmarkt@\textbf{Kohlmarkt}|pw}) {\pb}telegrafieren.\pend
           \pstart
           Ich tröſte mich am Goethe\pwindex{Goethe, Johann Wolfgang von 1749-08-28 – 1832-03-22@\textsc{Goethe, Johann Wolfgang von} (1749-08-28 – 1832-03-22), \emph{Schriftsteller}|pw}–Schiller\pwindex{Schiller, Friedrich von 10.11.1759 – 09.05.1805@\textsc{Schiller, Friedrich von} (10.11.1759 – 09.05.1805), \emph{Schriftsteller, Historiker, Philosoph}|pw}'ſchen Briefwechſel\pwindex{Schiller, Friedrich von 10.11.1759 – 09.05.1805@\textsc{Schiller, Friedrich von} (10.11.1759 – 09.05.1805), \emph{Schriftsteller, Historiker, Philosoph}!Briefwechsel zwischen Schiller und Goethe1791 – 1805@\strich\emph{Briefwechsel zwischen Schiller und Goethe} {[}1791 – 1805{]}|pw}\pwindex{Goethe, Johann Wolfgang von 1749-08-28 – 1832-03-22@\textsc{Goethe, Johann Wolfgang von} (1749-08-28 – 1832-03-22), \emph{Schriftsteller}!Briefwechsel zwischen Schiller und Goethe1791 – 1805@\strich\emph{Briefwechsel zwischen Schiller und Goethe} {[}1791 – 1805{]}|pw}
               über unſere \strikeout{mannigfache} mangelhafte Berühmtheit (Goethe\pwindex{Goethe, Johann Wolfgang von 1749-08-28 – 1832-03-22@\textsc{Goethe, Johann Wolfgang von} (1749-08-28 – 1832-03-22), \emph{Schriftsteller}|pw} mit \uline{46} Jahren in Karlsbad\oindex{Karlsbad@\textbf{Karlsbad}|pw} wird mit \uline{\textsc{Klinger}}\pwindex{Klinger, Friedrich Maximilian von 17.02.1752 – 09.03.1831@\textsc{Klinger, Friedrich Maximilian von} (17.02.1752 – 09.03.1831), \emph{Schriftsteller}|pw} verwechſelt) und habe Euch ſehr gern.\pend
           \pstart \spacefill\mbox{Hugo.}\pend{}
         
         \endnumbering\mylabel{h}\end{ledgroupsized}  \newcommand{\dateiname}{L00235}\newcommand{\titel}{Hugo von Hofmannsthal an Richard Beer-Hofmann und Arthur Schnitzler, 8. 7. 1893}\newcommand{\editorInnen}{Martin Anton Müller und Gerd-Hermann Susen}%% latex-leseansicht-abspann.tex
%% Abspann für die Leseansicht.
%% Der Schalter \ifkorrekturansicht ist bereits durch den Vorspann gesetzt.

%% latex-abspann.tex
%% Gemeinsamer Abspann für Korrekturansicht und Leseansicht.
%% Setzt den Schalter \ifkorrekturansicht voraus (gesetzt in den
%% einbindenden Dateien latex-korrekturansicht-abspann.tex bzw.
%% latex-leseansicht-abspann.tex).
%% ---------------------------------------------------------------

\normalsize

% Das esempio-Environment wird nur in der Leseansicht benötigt
\ifkorrekturansicht\else
\newenvironment{esempio}[3]%
{
    \vspace{1.5ex}
    \rlap{\underline{#1}}
    \par
    \setlength{\parindent}{0cm}
    \nopagebreak
    \leftskip=#2cm
    \rightskip=#3cm
}
{
    \par
}
\fi

\doendnotes{C}
\bigskip
\vfill

\clearpage

\footnotesize

\ifkorrekturansicht
  \lohead{\textsc{register}}
\fi

% theindex-Environment neu definieren ohne reledmac
\makeatletter
\renewenvironment{theindex}{%
  \ifkorrekturansicht
    \section*{\indexname}%
  \else
    \subsubsection*{Index der erwähnten Entitäten}%
  \fi
  \setlength{\parindent}{0pt}%
  \setlength{\parskip}{0pt plus 0.3pt}%
  \let\item\@idxitem
}{%
  \ifkorrekturansicht\clearpage\fi
}
\makeatother

\IfFileExists{\jobname-pw.ind}{\input{\jobname-pw.ind}}{}

% Quellenangabe nur in der Leseansicht
\ifkorrekturansicht\else
% Fallback-Definitionen, falls die .tex-Datei \titel etc. nicht gesetzt hat
\providecommand{\titel}{}
\providecommand{\editorInnen}{}
\providecommand{\dateiname}{\jobname}

\vspace{3cm}

\vfill

\footnotesize
\textsc{Quelle}: \titel. Herausgegeben von {\editorInnen}. In: \emph{Arthur Schnitzler: Briefwechsel mit Autorinnen und Autoren}.
 Digitale Edition, https://schnitzler-briefe.acdh.oeaw.ac.at/{\dateiname}.html (Stand \today)
\fi

\end{document}


      