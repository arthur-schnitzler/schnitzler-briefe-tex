%% latex-leseansicht-vorspann.tex
%% Vorspann für die Leseansicht.
%% Lädt die gemeinsame Datei latex-vorspann.tex mit nicht gesetztem Schalter.

\newif\ifkorrekturansicht
\korrekturansichtfalse

\input{../tex-inputs/latex-vorspann}


\section[Hugo von Hofmannsthal an Richard Beer-Hofmann und Arthur Schnitzler, 8. 7. 1893]{L00235 Hugo von Hofmannsthal an Richard Beer-Hofmann und Arthur Schnitzler, 8. 7. 1893}
\nopagebreak\mylabel{L00235v}
\rehead{ }\normalsize\beginnumbering\briefempfaengerindex{Beer-Hofmann, Richard@\textsc{Beer-Hofmann, Richard}!zzzHofmannsthal, Hugo von@\emph{von Hugo von Hofmannsthal}!1893-07-082@{8. 7. 1893}|(be}\briefempfaengerindex{Schnitzler, Arthur@\textsc{Schnitzler, Arthur}!zzzHofmannsthal, Hugo von@\emph{von Hugo von Hofmannsthal}!1893-07-082@{8. 7. 1893}|(be}
\toendnotes[C]{\smallbreak\pagebreak[2]}
\correspDesc{Versand  durch Hugo von Hofmannsthal am 8. 7. 1893 in Bad Fusch
\newline{}Erhalt  durch Arthur Schnitzler, Richard Beer-Hofmann im Zeitraum [9. 7. 1893
                  – 13. 7. 1893?] in Bad Ischl}\toendnotes[C]{\smallbreak}
\Standort{YCGL, MSS 32.}
\physDesc{Brief, 1 Blatt, 3 Seiten, 1130 Zeichen
\newline{}Handschrift: Bleistift, deutsche Kurrent
\newline{}Ordnung: mit rotem Buntstift von unbekannter Hand datiert: »8. VII. 1893–13« }
\buchAbdrucke{\weitereDrucke{Hugo von Hofmannsthal, Richard Beer-Hofmann: \emph{Briefwechsel}. Herausgegeben von Eugene Weber. Frankfurt am Main: \emph{S. Fischer} 1972, S. 23.} }
\pstart
           \raggedleft{}{\pb}Fuſch\oindex{Bad Fusch@\textbf{Bad Fusch}|pw}, 8 Juli 93.\pend
           
\pstart{}lieber Richard und Arthur!\pend\vspace{0.5em}
\pstart
           Ich brauch Euch wohl nicht zu{ }ſagen, wie ich mich freue, daſs endlich einmal ein paar
               von den graciöſen Schatten aus dem Anatolbuch\pwindex{Schnitzler, Arthur 15.\,5.\,1862 Wien – 21.\,10.\,1931 ebd.@\textsc{Schnitzler, Arthur} (15.\,5.\,1862 Wien – 21.\,10.\,1931 ebd.), \emph{Schriftsteller, Mediziner}!Anatol@\strich\emph{Anatol}|pw}
               bei Sommerſonne und Lampenlicht lebendig werden{ }ſollen. Ich käme hin, wäre ich nicht
               gerade beim zaghaften Anfang einer Erholung meines etwas in Unordnung gerathenen{ }ſog.
               Nervenſyſtems.\pend
           
\pstart
           Es thut mir merkwürdig wohl, ohne Kaffeehaus, ohne Geſelligkeit, ohne etwas das
               treibt oder bindet,{ }ſo vor mich hin zu dämmern, {\pb}in
               lauen Bädern beinahe einzuſchlafen und \textsc{Shakespeare\pwindex{Shakespeare, William 23.\,4.\,1564? Stratford-upon-Avon – 3.\,5.\,1616 ebd.@\textsc{Shakespeare, William} (23.\,4.\,1564? Stratford-upon-Avon – 3.\,5.\,1616 ebd.), \emph{Schauspieler, Dramatiker}|pw}’sche Comödien} zu leſen,
               während kleine Katzen in der Sonne mit einem Knäuel Wolle{ }ſpielen. Am liebſten war
               mir, Ihr möchtet am \substVorne{}\textsuperscript{m}\substDazwischen{}M\substHinten{}orgen drauf telegrafieren; jedenfalls{ }ſchickt mir, was Ihr an \strikeout{ſonſti} localen und{ }ſonſtigen Recenſionen bekommt,
               wenigſtens zum Anſehen hierher; ich{ }ſchicke Euch doch auch immer alles von mir.\pend
           
\pstart
           »Geſtern\pwindex{Hofmannsthal, Hugo von 1.\,2.\,1874 Wien – 15.\,7.\,1929 Rodaun@\textsc{Hofmannsthal, Hugo von} (1.\,2.\,1874 Wien – 15.\,7.\,1929 Rodaun), \emph{Schriftsteller}!Gestern. Dramatische Studie in einem Akt in Versen@\strich\emph{Gestern. Dramatische Studie in einem Akt in Versen}|pw}« hab ich nicht mit; wenn Richard es
               braucht, soll er an Manz\orgindex{Manz’sche Verlags- und Universitätsbuchhandlung@Manz’sche Verlags- und Universitätsbuchhandlung|pw} (\textsc{Kohlmarkt}\oindex{Wien@\textbf{Wien}!I., Innere Stadt@\textbf{I., Innere Stadt}!Kohlmarkt@\textbf{Kohlmarkt}, \emph{Straße}|pw}) {\pb}telegrafieren.\pend
           
\pstart
           Ich tröſte mich am Goethe\pwindex{Goethe, Johann Wolfgang von 28.\,8.\,1749 Frankfurt am Main – 22.\,3.\,1832 Weimar@\textsc{Goethe, Johann Wolfgang von} (28.\,8.\,1749 Frankfurt am Main – 22.\,3.\,1832 Weimar), \emph{Schriftsteller}|pw}–Schiller\pwindex{Schiller, Friedrich von 10.\,11.\,1759 Marbach am Neckar – 9.\,5.\,1805 Weimar@\textsc{Schiller, Friedrich von} (10.\,11.\,1759 Marbach am Neckar – 9.\,5.\,1805 Weimar), \emph{Schriftsteller, Historiker, Philosoph}|pw}’ſchen Briefwechſel\pwindex{Schiller, Friedrich von 10.\,11.\,1759 Marbach am Neckar – 9.\,5.\,1805 Weimar@\textsc{Schiller, Friedrich von} (10.\,11.\,1759 Marbach am Neckar – 9.\,5.\,1805 Weimar), \emph{Schriftsteller, Historiker, Philosoph}!Briefwechsel zwischen Schiller und Goethe@\strich\emph{Briefwechsel zwischen Schiller und Goethe}|pw}\pwindex{Goethe, Johann Wolfgang von 28.\,8.\,1749 Frankfurt am Main – 22.\,3.\,1832 Weimar@\textsc{Goethe, Johann Wolfgang von} (28.\,8.\,1749 Frankfurt am Main – 22.\,3.\,1832 Weimar), \emph{Schriftsteller}!Briefwechsel zwischen Schiller und Goethe@\strich\emph{Briefwechsel zwischen Schiller und Goethe}|pw} über unſere \strikeout{mannigfache}
               mangelhafte Berühmtheit (Goethe\pwindex{Goethe, Johann Wolfgang von 28.\,8.\,1749 Frankfurt am Main – 22.\,3.\,1832 Weimar@\textsc{Goethe, Johann Wolfgang von} (28.\,8.\,1749 Frankfurt am Main – 22.\,3.\,1832 Weimar), \emph{Schriftsteller}|pw} mit \uline{46} Jahren in Karlsbad\oindex{Karlsbad@\textbf{Karlsbad}|pw} wird mit \uline{\textsc{Klinger}}\pwindex{Klinger, Friedrich Maximilian von 17.\,2.\,1752 Frankfurt am Main – 9.\,3.\,1831 Tartu@\textsc{Klinger, Friedrich Maximilian von} (17.\,2.\,1752 Frankfurt am Main – 9.\,3.\,1831 Tartu), \emph{Schriftsteller}|pw} verwechſelt) und habe Euch{ }ſehr gern.\pend
           \pstart \spacefill\mbox{Hugo.}\pend{}\selectlanguage{ngerman}\endnumbering\briefempfaengerindex{Beer-Hofmann, Richard@\textsc{Beer-Hofmann, Richard}!zzzHofmannsthal, Hugo von@\emph{von Hugo von Hofmannsthal}!1893-07-082@{8. 7. 1893}|)be}\briefempfaengerindex{Schnitzler, Arthur@\textsc{Schnitzler, Arthur}!zzzHofmannsthal, Hugo von@\emph{von Hugo von Hofmannsthal}!1893-07-082@{8. 7. 1893}|)be}\mylabel{L00235h}  \newcommand{\dateiname}{L00235}\newcommand{\titel}{Hugo von Hofmannsthal an Richard Beer-Hofmann und Arthur Schnitzler, 8. 7. 1893}\newcommand{\editorInnen}{Martin Anton Müller und Gerd-Hermann Susen}%% latex-leseansicht-abspann.tex
%% Abspann für die Leseansicht.
%% Der Schalter \ifkorrekturansicht ist bereits durch den Vorspann gesetzt.

%% latex-abspann.tex
%% Gemeinsamer Abspann für Korrekturansicht und Leseansicht.
%% Setzt den Schalter \ifkorrekturansicht voraus (gesetzt in den
%% einbindenden Dateien latex-korrekturansicht-abspann.tex bzw.
%% latex-leseansicht-abspann.tex).
%% ---------------------------------------------------------------

\normalsize

% Das esempio-Environment wird nur in der Leseansicht benötigt
\ifkorrekturansicht\else
\newenvironment{esempio}[3]%
{
    \vspace{1.5ex}
    \rlap{\underline{#1}}
    \par
    \setlength{\parindent}{0cm}
    \nopagebreak
    \leftskip=#2cm
    \rightskip=#3cm
}
{
    \par
}
\fi

\doendnotes{C}
\bigskip
\vfill

\clearpage

\footnotesize

\ifkorrekturansicht
  \lohead{\textsc{register}}
\fi

% theindex-Environment neu definieren ohne reledmac
\makeatletter
\renewenvironment{theindex}{%
  \ifkorrekturansicht
    \section*{\indexname}%
  \else
    \subsubsection*{Index der erwähnten Entitäten}%
  \fi
  \setlength{\parindent}{0pt}%
  \setlength{\parskip}{0pt plus 0.3pt}%
  \let\item\@idxitem
}{%
  \ifkorrekturansicht\clearpage\fi
}
\makeatother

\IfFileExists{\jobname-pw.ind}{\input{\jobname-pw.ind}}{}

% Quellenangabe nur in der Leseansicht
\ifkorrekturansicht\else
% Fallback-Definitionen, falls die .tex-Datei \titel etc. nicht gesetzt hat
\providecommand{\titel}{}
\providecommand{\editorInnen}{}
\providecommand{\dateiname}{\jobname}

\vspace{3cm}

\vfill

\footnotesize
\textsc{Quelle}: \titel. Herausgegeben von {\editorInnen}. In: \emph{Arthur Schnitzler: Briefwechsel mit Autorinnen und Autoren}.
 Digitale Edition, https://schnitzler-briefe.acdh.oeaw.ac.at/{\dateiname}.html (Stand \today)
\fi

\end{document}


