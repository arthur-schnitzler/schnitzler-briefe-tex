%% latex-korrekturansicht-vorspann.tex
%% Vorspann für die Korrekturansicht.
%% Lädt die gemeinsame Datei latex-vorspann.tex mit gesetztem Schalter.

\newif\ifkorrekturansicht
\korrekturansichttrue

\input{../tex-inputs/latex-vorspann}


\section[Hugo von Hofmannsthal an Richard Beer-Hofmann und Arthur Schnitzler, 8. 7. 1893]{L00235 Hugo von Hofmannsthal an Richard Beer-Hofmann und Arthur Schnitzler,
               8. 7. 1893}
\nopagebreak\mylabel{L00235v}
\rehead{ }\normalsize\beginnumbering\briefempfaengerindex{Beer-Hofmann, Richard@\textsc{Beer-Hofmann, Richard}!zzzHofmannsthal, Hugo von@\emph{von Hugo von Hofmannsthal}!1893-07-082@{8. 7. 1893}|(be}\briefempfaengerindex{Schnitzler, Arthur@\textsc{Schnitzler, Arthur}!zzzHofmannsthal, Hugo von@\emph{von Hugo von Hofmannsthal}!1893-07-082@{8. 7. 1893}|(be}
\toendnotes[C]{\smallbreak\pagebreak[2]}\Standort{YCGL, MSS 32.}
\physDesc{Brief, 1 Blatt, 3 Seiten, 1130 Zeichen
\newline{}Handschrift: Bleistift, deutsche Kurrent
\newline{}Ordnung: mit rotem Buntstift von unbekannter Hand datiert: »8. VII. 1893–13« }
\buchAbdrucke{\weitereDrucke{Hugo von Hofmannsthal, Richard Beer-Hofmann: \emph{Briefwechsel}. Frankfurt am Main: \emph{S. Fischer} 1972, S. 23.} }
\pstart
           \raggedleft{}{\pb}Fuſch\oindex{Bad Fusch@\textbf{Bad Fusch}, \emph{A.ADM3}|pw}, 8 Juli 93.\pend
           
\pstart{}lieber Richard und Arthur!\pend\vspace{0.5em}
\pstart
           Ich brauch Euch wohl nicht zu ſagen, wie ich mich freue, daſs endlich einmal ein paar
               von den graciöſen Schatten aus dem Anatolbuch\pwindex{Anatol@\emph{Anatol}|pw}
               bei Sommerſonne und Lampenlicht lebendig werden ſollen. Ich käme hin, wäre ich nicht
               gerade beim zaghaften Anfang einer Erholung meines etwas in Unordnung gerathenen ſog.
               Nervenſyſtems.\pend
           
\pstart
           Es thut mir merkwürdig wohl, ohne Kaffeehaus, ohne Geſelligkeit, ohne etwas das
               treibt oder bindet, ſo vor mich hin zu dämmern, {\pb}in
               lauen Bädern beinahe einzuſchlafen und \textsc{Shakespeare\pwindex{Shakespeare, William 23.4.1564? – 03.05.1616@\textsc{Shakespeare, William} (23.4.1564? – 03.05.1616), \emph{Schauspieler/Schauspielerin, Dramatiker/Dramatikerin}|pw}’sche Comödien} zu leſen,
               während kleine Katzen in der Sonne mit einem Knäuel Wolle ſpielen. Am liebſten war
               mir, Ihr möchtet am \substVorne{}\textsuperscript{m}\substDazwischen{}M\substHinten{}orgen drauf telegrafieren; jedenfalls ſchickt mir, was Ihr an \strikeout{ſonſti} localen und ſonſtigen Recenſionen bekommt,
               wenigſtens zum Anſehen hierher; ich ſchicke Euch doch auch immer alles von mir.\pend
           
\pstart
           »Geſtern\pwindex{Gestern. Dramatische Studie in einem Akt in Versen@\emph{Gestern. Dramatische Studie in einem Akt in Versen}|pw}« hab ich nicht mit; wenn Richard es
               braucht, soll er an Manz\orgindex{Manz sche Verlags- und Universitaetsbuchhandlung@Manz’sche Verlags- und Universitätsbuchhandlung|pw} (\textsc{Kohlmarkt}\oindex{Kohlmarkt@\textbf{Kohlmarkt}, \emph{Straße (K.STR)}|pw}) {\pb}telegrafieren.\pend
           
\pstart
           Ich tröſte mich am Goethe\pwindex{Goethe, Johann Wolfgang von 1749-08-28 – 1832-03-22@\textsc{Goethe, Johann Wolfgang von} (1749-08-28 – 1832-03-22), \emph{Schriftsteller/Schriftstellerin}|pw}–Schiller\pwindex{Schiller, Friedrich von 10.11.1759 – 09.05.1805@\textsc{Schiller, Friedrich von} (10.11.1759 – 09.05.1805), \emph{Schriftsteller/Schriftstellerin, Historiker/Historikerin, Philosoph/Philosophin}|pw}'ſchen Briefwechſel\pwindex{Briefwechsel zwischen Schiller und Goethe@\emph{Briefwechsel zwischen Schiller und Goethe}|pw} über unſere \strikeout{mannigfache}
               mangelhafte Berühmtheit (Goethe\pwindex{Goethe, Johann Wolfgang von 1749-08-28 – 1832-03-22@\textsc{Goethe, Johann Wolfgang von} (1749-08-28 – 1832-03-22), \emph{Schriftsteller/Schriftstellerin}|pw} mit \uline{46} Jahren in Karlsbad\oindex{Karlsbad@\textbf{Karlsbad}, \emph{P.PPLA}|pw} wird mit \uline{\textsc{Klinger}}\pwindex{Klinger, Friedrich Maximilian von 17.02.1752 – 09.03.1831@\textsc{Klinger, Friedrich Maximilian von} (17.02.1752 – 09.03.1831), \emph{Schriftsteller/Schriftstellerin}|pw} verwechſelt) und habe Euch ſehr gern.\pend
           \pstart \spacefill\mbox{Hugo.}\pend{}\selectlanguage{ngerman}\endnumbering\briefempfaengerindex{Beer-Hofmann, Richard@\textsc{Beer-Hofmann, Richard}!zzzHofmannsthal, Hugo von@\emph{von Hugo von Hofmannsthal}!1893-07-082@{8. 7. 1893}|)be}\briefempfaengerindex{Schnitzler, Arthur@\textsc{Schnitzler, Arthur}!zzzHofmannsthal, Hugo von@\emph{von Hugo von Hofmannsthal}!1893-07-082@{8. 7. 1893}|)be}\mylabel{L00235h}  \normalsize

\doendnotes{C}
\bigskip
\vfill

\clearpage

\footnotesize

\lohead{\textsc{register}}

% Definiere theindex-Environment komplett neu ohne reledmac
\makeatletter
\renewenvironment{theindex}{%
  \section*{\indexname}%
  \setlength{\parindent}{0pt}%
  \setlength{\parskip}{0pt plus 0.3pt}%
  \let\item\@idxitem
}{%
  \clearpage
}
\makeatother

\IfFileExists{\jobname-pw.ind}{\input{\jobname-pw.ind}}{}

\end{document}

      