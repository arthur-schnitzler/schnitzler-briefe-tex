%% latex-leseansicht-vorspann.tex
%% Vorspann für die Leseansicht.
%% Lädt die gemeinsame Datei latex-vorspann.tex mit nicht gesetztem Schalter.

\newif\ifkorrekturansicht
\korrekturansichtfalse

\input{../tex-inputs/latex-vorspann}


\section[Richard Beer-Hofmann an Arthur Schnitzler, 18. 10. 1894]{L00384 Richard Beer-Hofmann an Arthur Schnitzler, 18. 10. 1894}
\nopagebreak\mylabel{L00384v}
\rehead{ }\normalsize\beginnumbering\briefempfaengerindex{Schnitzler, Arthur@\textsc{Schnitzler, Arthur}!zzzBeer-Hofmann, Richard@\emph{von Richard Beer-Hofmann}!1894-10-181@{18. 10. 1894}|(be}
\toendnotes[C]{\smallbreak\pagebreak[2]}
\correspDesc{Versand  durch Richard Beer-Hofmann am 18. 10. 1894 in Neapel
\newline{}Erhalt  durch Arthur Schnitzler im Zeitraum [19. 10. 1894 – 23. 10. 1894?] in Wien}\toendnotes[C]{\smallbreak}
\Standort{CUL, Schnitzler, B 8.}
\physDesc{Brief, 2 Blätter, 8 Seiten, 2127 Zeichen
\newline{}Handschrift: Bleistift, lateinische Kurrent
\newline{}Schnitzler: mit Bleistift datiert: »17/10 94« und nummeriert: »49«, Datum auf dem
                                 zweiten Blatt wiederholt }
\buchAbdrucke{\weitereDrucke{Arthur Schnitzler, Richard Beer-Hofmann: \emph{Briefwechsel 1891–1931}. Herausgegeben von Konstanze Fliedl. Wien, Zürich: \emph{Europaverlag} 1992, S. 64–65.} }\toendnotes[C]{\smallbreak}
\pstart
           \noindent{}{\pb}Lieber Arthur! Ich verdiene es nicht – aber schreiben Sie – ich
               meine Briefe an mich. Ich bin furchtbar neugierig auf Ihr Stück. Sie werden es mir
               separat vorlesen müssen, und Salten\pwindex{Salten, Felix 6.\,9.\,1869 Budapest – 8.\,10.\,1945 Zürich@\textsc{Salten, Felix} (6.\,9.\,1869 Budapest – 8.\,10.\,1945 Zürich), \emph{Schriftsteller, Journalist, Chefredakteur}|pw} und Hugo\pwindex{Hofmannsthal, Hugo von 1.\,2.\,1874 Wien – 15.\,7.\,1929 Rodaun@\textsc{Hofmannsthal, Hugo von} (1.\,2.\,1874 Wien – 15.\,7.\,1929 Rodaun), \emph{Schriftsteller}|pw} werden bitten es nochmal hören zu dürfen.
               Wenn Kraus\pwindex{Kraus, Karl 28.\,4.\,1874 Jičín – 12.\,6.\,1936 Wien@\textsc{Kraus, Karl} (28.\,4.\,1874 Jičín – 12.\,6.\,1936 Wien), \emph{Schriftsteller, Publizist, Schriftsteller}|pw} sich überni{\geminationm}t, sagen Sie {\pb}ihm die Worte: »\uline{Musenalmanach}\pwindex{Moderner Musen-Almanach auf das Jahr 1894. Ein Jahrbuch deutscher Kunst@\emph{Moderner Musen-Almanach auf das Jahr 1894. Ein Jahrbuch deutscher Kunst}|pw} – \uline{Herodot}\pwindex{Herodot 490? v. Chr. Bodrum – 425/420? v. Chr.@\textsc{Herodot} (490? v. Chr. Bodrum – 425/420? v. Chr.), \emph{Historiker}|pw}« und er wird \label{K_L00384-1v}\edtext{erbleichen}{\lemma{\textnormal{\emph{erbleichen}}}\Cendnote{\textnormal{Kraus\pwindex{Kraus, Karl 28.\,4.\,1874 Jičín – 12.\,6.\,1936 Wien@\textsc{Kraus, Karl} (28.\,4.\,1874 Jičín – 12.\,6.\,1936 Wien), \emph{Schriftsteller, Publizist, Schriftsteller}|pwk} hatte Beer-Hofmann\pwindex{Beer-Hofmann, Richard 11.\,7.\,1866 Wien – 26.\,9.\,1945 New York City@\textsc{Beer-Hofmann, Richard} (11.\,7.\,1866 Wien – 26.\,9.\,1945 New York City), \emph{Schriftsteller}|pwk} an Stelle des ausgeliehenen \emph{Modernen Musenalmanachs}\pwindex{Moderner Musen-Almanach auf das Jahr 1894. Ein Jahrbuch deutscher Kunst@\emph{Moderner Musen-Almanach auf das Jahr 1894. Ein Jahrbuch deutscher Kunst}|pwk} ein Kinderbuch
                  retourniert. Auch eine Herodot\pwindex{Herodot 490? v. Chr. Bodrum – 425/420? v. Chr.@\textsc{Herodot} (490? v. Chr. Bodrum – 425/420? v. Chr.), \emph{Historiker}|pwk}-Ausgabe hätte er
                   von Goldmann\pwindex{Goldmann, Paul 31.\,1.\,1865 Breslau – 25.\,9.\,1935 Wien@\textsc{Goldmann, Paul} (31.\,1.\,1865 Breslau – 25.\,9.\,1935 Wien), \emph{Schriftsteller, Journalist}|pwk} übermitteln
                  sollen, aber darauf vergessen. Siehe Karl Kraus\pwindex{Kraus, Karl 28.\,4.\,1874 Jičín – 12.\,6.\,1936 Wien@\textsc{Kraus, Karl} (28.\,4.\,1874 Jičín – 12.\,6.\,1936 Wien), \emph{Schriftsteller, Publizist, Schriftsteller}|pwk}: \emph{Ein Brief an
                        Richard}. Kommentiert von Leo A. Lensing. In: \emph{Kraus-Hefte}, Nr. 41, Januar 1987,
                  S. 5–7.}}}\label{K_L00384-1}.\pend
           
\pstart
           Ich habe gestern eine Karte an Sie geschrieben. Wegen »Saubermänner\orgindex{Saubermänner. Tischgesellschaft@Die Saubermänner. Tischgesellschaft|pw}«, suchen Sie es zu vereiteln, daß Schönthan\pwindex{Schönthan-Pernwald, Paul von 19.\,3.\,1853 Wien – 4.\,8.\,1905 ebd.@\textsc{Schönthan-Pernwald, Paul von} (19.\,3.\,1853 Wien – 4.\,8.\,1905 ebd.), \emph{Schriftsteller, Journalist}|pw} an mich eine Aufforderung richtet beizutreten. Refus
               wäre Beleidigung, und es ist genug, daß Sie beitre{\pb}ten mussten. »Ikarus Ikarus, Ja{\geminationm}ers genug\pwindex{\textcolor{red}{\textsuperscript{XXXX indx1}}!Faust. Eine Tragödie@\strich\emph{Faust. Eine Tragödie}|pwv}« – (Mir ko{\geminationm}t vor ich citire \label{K_L00384-2v}\edtext{ungenau}{\lemma{\textnormal{\emph{ungenau}}}\Cendnote{\textnormal{richtig: »Jammer«}}}\label{K_L00384-2} – oder genau – oder – ungenau sagt
               A. S.)\pend
           
\pstart
           Denken Sie, ich erhalte gleichzeitig mit \uline{Ihrem} Brief
               einen von S. Fischer\pwindex{Fischer, Samuel 24.\,12.\,1859 Liptovský Mikuláš – 15.\,10.\,1934 Berlin@\textsc{Fischer, Samuel} (24.\,12.\,1859 Liptovský Mikuláš – 15.\,10.\,1934 Berlin), \emph{Verleger}|pw}, der vor kurzem wie er
               schreibt meine Novellen gelesen hat und er hegt »seit jener Zeit den lebhaften Wunsch
                  {\pb}falls Sie betreffs Ihrer
               zukünftigen Production mit einem andern Verlag noch nichts vereinbart haben Ihre
               Werke in meinem Verlage zu publiciren« folgt eine Schilderung seines Verlages und die
               inhaltsschwere Phrase: »mannigfache Vorteile bieten zu können«. Zum Schluss
               Aufforderung eine Novelle bei ihm zu publiciren (freie Bühne\pwindex{Neue Deutsche Rundschau@\emph{Neue Deutsche Rundschau}|pw}). »Sollten Sie {\pb}etwas fertig haben, so würden Sie uns durch die Einsendung sehr erfreuen«: Dem
               »erfreuten u. lebhaftwünschenden« Verlag werde ich natürlich furchtbar frech
               antworten, oder besser vornehm reservirt – schon weil ich – (ich weiss es ist
               peinlich, für meine Freunde, ich fange an lächerliche Figur zu werden, ich soll doch
               was fer{\pb}tig machen, –  oder nein
               ich soll mir Zeit lassen) nichts fertig habe. –\pend
           
\pstart
           Ich bin längstens 5ten Nov. in Wien\oindex{Wien@\textbf{Wien}, \emph{Verwaltungsgebiet}|pw}.
               Ich fange an meine Aufnahmsfähigkeit zu verlieren – \uline{zu
                  viel}, – zu viel stürmt auf einen, Landschaft Kunst und manchmal {\pb}auch eigne Gedanken über all das,
               und über anderes, – durch Associationen verrücktester Art hervorgerufen.\pend
           
\pstart
           Ich freue mich sehr auf Euch und Wien\oindex{Wien@\textbf{Wien}, \emph{Verwaltungsgebiet}|pw}. Hier in \uline{Italien}\oindex{Italien@\textbf{Italien}|pw} – in \uline{Rom}\oindex{Rom@\textbf{Rom}, \emph{Hauptstadt}|pw} in \uline{Neapel}\oindex{Neapel@\textbf{Neapel}|pw} empfinde ich es daß die einzige Stadt wo ich leben {\pb}und – bitte nicht zu lachen –
               arbeiten kann doch nur Wien\oindex{Wien@\textbf{Wien}, \emph{Verwaltungsgebiet}|pw} ist. Was aber kein
               Coupletrefrain sein soll. Schreiben Sie mir bald, – \uline{Neapel}\oindex{Neapel@\textbf{Neapel}|pw}.\pend
           
\pstart
           Herzlichst Ihr{\\[\baselineskip]}\spacefill\mbox{Richard}\pend
           \leftskip=0em{}
\pstart
           Donnerstag\hspace*{1.5em}\uline{Neapel}\oindex{Neapel@\textbf{Neapel}|pw}\pend
           
\pstart
           18/10 94.\pend
           \selectlanguage{ngerman}\endnumbering\briefempfaengerindex{Schnitzler, Arthur@\textsc{Schnitzler, Arthur}!zzzBeer-Hofmann, Richard@\emph{von Richard Beer-Hofmann}!1894-10-181@{18. 10. 1894}|)be}\mylabel{L00384h}  \newcommand{\dateiname}{L00384}\newcommand{\titel}{Richard Beer-Hofmann an Arthur Schnitzler, 18. 10. 1894}\newcommand{\editorInnen}{Martin Anton Müller und Gerd-Hermann Susen}%% latex-leseansicht-abspann.tex
%% Abspann für die Leseansicht.
%% Der Schalter \ifkorrekturansicht ist bereits durch den Vorspann gesetzt.

%% latex-abspann.tex
%% Gemeinsamer Abspann für Korrekturansicht und Leseansicht.
%% Setzt den Schalter \ifkorrekturansicht voraus (gesetzt in den
%% einbindenden Dateien latex-korrekturansicht-abspann.tex bzw.
%% latex-leseansicht-abspann.tex).
%% ---------------------------------------------------------------

\normalsize

% Das esempio-Environment wird nur in der Leseansicht benötigt
\ifkorrekturansicht\else
\newenvironment{esempio}[3]%
{
    \vspace{1.5ex}
    \rlap{\underline{#1}}
    \par
    \setlength{\parindent}{0cm}
    \nopagebreak
    \leftskip=#2cm
    \rightskip=#3cm
}
{
    \par
}
\fi

\doendnotes{C}
\bigskip
\vfill

\clearpage

\footnotesize

\ifkorrekturansicht
  \lohead{\textsc{register}}
\fi

% theindex-Environment neu definieren ohne reledmac
\makeatletter
\renewenvironment{theindex}{%
  \ifkorrekturansicht
    \section*{\indexname}%
  \else
    \subsubsection*{Index der erwähnten Entitäten}%
  \fi
  \setlength{\parindent}{0pt}%
  \setlength{\parskip}{0pt plus 0.3pt}%
  \let\item\@idxitem
}{%
  \ifkorrekturansicht\clearpage\fi
}
\makeatother

\IfFileExists{\jobname-pw.ind}{\input{\jobname-pw.ind}}{}

% Quellenangabe nur in der Leseansicht
\ifkorrekturansicht\else
% Fallback-Definitionen, falls die .tex-Datei \titel etc. nicht gesetzt hat
\providecommand{\titel}{}
\providecommand{\editorInnen}{}
\providecommand{\dateiname}{\jobname}

\vspace{3cm}

\vfill

\footnotesize
\textsc{Quelle}: \titel. Herausgegeben von {\editorInnen}. In: \emph{Arthur Schnitzler: Briefwechsel mit Autorinnen und Autoren}.
 Digitale Edition, https://schnitzler-briefe.acdh.oeaw.ac.at/{\dateiname}.html (Stand \today)
\fi

\end{document}


