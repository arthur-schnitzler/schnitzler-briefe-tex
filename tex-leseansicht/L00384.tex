%% latex-korrekturansicht-vorspann.tex
%% Vorspann für die Korrekturansicht.
%% Lädt die gemeinsame Datei latex-vorspann.tex mit gesetztem Schalter.

\newif\ifkorrekturansicht
\korrekturansichttrue

\input{../tex-inputs/latex-vorspann}


\section[Richard Beer-Hofmann an Arthur Schnitzler, 18. 10. 1894]{L00384 Richard Beer-Hofmann an Arthur Schnitzler, 18. 10. 1894}
\nopagebreak\mylabel{L00384v}
\rehead{ }\normalsize\beginnumbering\briefempfaengerindex{Schnitzler, Arthur@\textsc{Schnitzler, Arthur}!zzzBeer-Hofmann, Richard@\emph{von Richard Beer-Hofmann}!1894-10-181@{18. 10. 1894}|(be}
\toendnotes[C]{\smallbreak\pagebreak[2]}\Standort{CUL, Schnitzler, B 8.}
\physDesc{Brief, 2 Blätter, 8 Seiten, 2127 Zeichen
\newline{}Handschrift: Bleistift, lateinische Kurrent
\newline{}Schnitzler: mit Bleistift datiert: »17/10 94« und nummeriert: »49«, Datum auf dem
                                 zweiten Blatt wiederholt }
\buchAbdrucke{\weitereDrucke{Arthur Schnitzler, Richard Beer-Hofmann: \emph{Briefwechsel 1891–1931}. Wien, Zürich: \emph{Europaverlag} 1992, S. 64–65.} }\toendnotes[C]{\smallbreak}
\pstart
           \noindent{}{\pb}Lieber Arthur! Ich verdiene es nicht – aber schreiben Sie – ich
               meine Briefe an mich. Ich bin furchtbar neugierig auf Ihr Stück. Sie werden es mir
               separat vorlesen müssen, und Salten\pwindex{Salten, Felix 06.09.1869 – 08.10.1945@\textsc{Salten, Felix} (06.09.1869 – 08.10.1945), \emph{Schriftsteller/Schriftstellerin, Journalist/Journalistin, Chefredakteur/Chefredakteurin}|pw} und Hugo\pwindex{Hofmannsthal, Hugo von 1874-02-01 – 1929-07-15@\textsc{Hofmannsthal, Hugo von} (1874-02-01 – 1929-07-15), \emph{Schriftsteller/Schriftstellerin}|pw} werden bitten es nochmal hören zu dürfen.
               Wenn Kraus\pwindex{Kraus, Karl 28.04.1874 – 12.06.1936@\textsc{Kraus, Karl} (28.04.1874 – 12.06.1936), \emph{Schriftsteller/Schriftstellerin, Publizist/Publizistin, Schriftsteller/Schriftstellerin}|pw} sich überni{\geminationm}t, sagen Sie {\pb}ihm die Worte: »\uline{Musenalmanach}\pwindex{Moderner Musen-Almanach auf das Jahr 1894. Ein Jahrbuch deutscher Kunst@\emph{Moderner Musen-Almanach auf das Jahr 1894. Ein Jahrbuch deutscher Kunst}|pw} – \uline{Herodot}\pwindex{Herodot 490? v. Chr. – 425/420? v. Chr.@\textsc{Herodot} (490? v. Chr. – 425/420? v. Chr.), \emph{Historiker/Historikerin}|pw}« und er wird \label{K_L00384-1v}\edtext{erbleichen}{\lemma{\textnormal{\emph{erbleichen}}}\Cendnote{\textnormal{Kraus\pwindex{Kraus, Karl 28.04.1874 – 12.06.1936@\textsc{Kraus, Karl} (28.04.1874 – 12.06.1936), \emph{Schriftsteller/Schriftstellerin, Publizist/Publizistin, Schriftsteller/Schriftstellerin}|pwk} hatte Beer-Hofmann\pwindex{Beer-Hofmann, Richard 1866-07-11 – 1945-09-26@\textsc{Beer-Hofmann, Richard} (1866-07-11 – 1945-09-26), \emph{Schriftsteller/Schriftstellerin}|pwk} an Stelle des ausgeliehenen \emph{Modernen Musenalmanachs}\pwindex{Moderner Musen-Almanach auf das Jahr 1894. Ein Jahrbuch deutscher Kunst@\emph{Moderner Musen-Almanach auf das Jahr 1894. Ein Jahrbuch deutscher Kunst}|pwk} ein Kinderbuch
                  retourniert. Auch eine Herodot\pwindex{Herodot 490? v. Chr. – 425/420? v. Chr.@\textsc{Herodot} (490? v. Chr. – 425/420? v. Chr.), \emph{Historiker/Historikerin}|pwk}-Ausgabe hätte er
                   von Goldmann\pwindex{Goldmann, Paul 31.01.1865 – 25.09.1935@\textsc{Goldmann, Paul} (31.01.1865 – 25.09.1935), \emph{Schriftsteller/Schriftstellerin, Journalist/Journalistin}|pwk} übermitteln
                  sollen, aber darauf vergessen. Siehe Karl Kraus\pwindex{Kraus, Karl 28.04.1874 – 12.06.1936@\textsc{Kraus, Karl} (28.04.1874 – 12.06.1936), \emph{Schriftsteller/Schriftstellerin, Publizist/Publizistin, Schriftsteller/Schriftstellerin}|pwk}: \emph{Ein Brief an
                        Richard}. Kommentiert von Leo A. Lensing. In: \emph{Kraus-Hefte}, Nr. 41, Januar 1987,
                  S. 5–7.}}}\label{K_L00384-1}.\pend
           
\pstart
           Ich habe gestern eine Karte an Sie geschrieben. Wegen »Saubermänner\orgindex{Saubermaenner. Tischgesellschaft@Die Saubermänner. Tischgesellschaft|pw}«, suchen Sie es zu vereiteln, daß Schönthan\pwindex{Schoenthan-Pernwald, Paul von 19.03.1853 – 04.08.1905@\textsc{Schönthan-Pernwald, Paul von} (19.03.1853 – 04.08.1905), \emph{Schriftsteller/Schriftstellerin, Journalist/Journalistin}|pw} an mich eine Aufforderung richtet beizutreten. Refus
               wäre Beleidigung, und es ist genug, daß Sie beitre{\pb}ten mussten. »Ikarus Ikarus, Ja{\geminationm}ers genug\pwindex{Faust. Eine Tragoedie@\emph{Faust. Eine Tragödie}|pwv}« – (Mir ko{\geminationm}t vor ich citire \label{K_L00384-2v}\edtext{ungenau}{\lemma{\textnormal{\emph{ungenau}}}\Cendnote{\textnormal{richtig: »Jammer«}}}\label{K_L00384-2} – oder genau – oder – ungenau sagt
               A. S.)\pend
           
\pstart
           Denken Sie, ich erhalte gleichzeitig mit \uline{Ihrem} Brief
               einen von S. Fischer\pwindex{Fischer, Samuel 24.12.1859 – 15.10.1934@\textsc{Fischer, Samuel} (24.12.1859 – 15.10.1934), \emph{Verleger/Verlegerin}|pw}, der vor kurzem wie er
               schreibt meine Novellen gelesen hat und er hegt »seit jener Zeit den lebhaften Wunsch
                  {\pb}falls Sie betreffs Ihrer
               zukünftigen Production mit einem andern Verlag noch nichts vereinbart haben Ihre
               Werke in meinem Verlage zu publiciren« folgt eine Schilderung seines Verlages und die
               inhaltsschwere Phrase: »mannigfache Vorteile bieten zu können«. Zum Schluss
               Aufforderung eine Novelle bei ihm zu publiciren (freie Bühne\pwindex{Neue Deutsche Rundschau@\emph{Neue Deutsche Rundschau}|pw}). »Sollten Sie {\pb}etwas fertig haben, so würden Sie uns durch die Einsendung sehr erfreuen«: Dem
               »erfreuten u. lebhaftwünschenden« Verlag werde ich natürlich furchtbar frech
               antworten, oder besser vornehm reservirt – schon weil ich – (ich weiss es ist
               peinlich, für meine Freunde, ich fange an lächerliche Figur zu werden, ich soll doch
               was fer{\pb}tig machen, –  oder nein
               ich soll mir Zeit lassen) nichts fertig habe. –\pend
           
\pstart
           Ich bin längstens 5ten Nov. in Wien\oindex{Wien@\textbf{Wien}, \emph{A.ADM2}|pw}.
               Ich fange an meine Aufnahmsfähigkeit zu verlieren – \uline{zu
                  viel}, – zu viel stürmt auf einen, Landschaft Kunst und manchmal {\pb}auch eigne Gedanken über all das,
               und über anderes, – durch Associationen verrücktester Art hervorgerufen.\pend
           
\pstart
           Ich freue mich sehr auf Euch und Wien\oindex{Wien@\textbf{Wien}, \emph{A.ADM2}|pw}. Hier in \uline{Italien}\oindex{Italien@\textbf{Italien}, \emph{A.PCLI}|pw} – in \uline{Rom}\oindex{Rom@\textbf{Rom}, \emph{P.PPLC}|pw} in \uline{Neapel}\oindex{Neapel@\textbf{Neapel}, \emph{P.PPLA}|pw} empfinde ich es daß die einzige Stadt wo ich leben {\pb}und – bitte nicht zu lachen –
               arbeiten kann doch nur Wien\oindex{Wien@\textbf{Wien}, \emph{A.ADM2}|pw} ist. Was aber kein
               Coupletrefrain sein soll. Schreiben Sie mir bald, – \uline{Neapel}\oindex{Neapel@\textbf{Neapel}, \emph{P.PPLA}|pw}.\pend
           
\pstart
           Herzlichst Ihr{\\[\baselineskip]}\spacefill\mbox{Richard}\pend
           \leftskip=0em{}
\pstart
           Donnerstag\hspace*{1.5em}\uline{Neapel}\oindex{Neapel@\textbf{Neapel}, \emph{P.PPLA}|pw}\pend
           
\pstart
           18/10 94.\pend
           \selectlanguage{ngerman}\endnumbering\briefempfaengerindex{Schnitzler, Arthur@\textsc{Schnitzler, Arthur}!zzzBeer-Hofmann, Richard@\emph{von Richard Beer-Hofmann}!1894-10-181@{18. 10. 1894}|)be}\mylabel{L00384h}  \normalsize

\doendnotes{C}
\bigskip
\vfill

\clearpage

\footnotesize

\lohead{\textsc{register}}

% Definiere theindex-Environment komplett neu ohne reledmac
\makeatletter
\renewenvironment{theindex}{%
  \section*{\indexname}%
  \setlength{\parindent}{0pt}%
  \setlength{\parskip}{0pt plus 0.3pt}%
  \let\item\@idxitem
}{%
  \clearpage
}
\makeatother

\IfFileExists{\jobname-pw.ind}{\input{\jobname-pw.ind}}{}

\end{document}

      