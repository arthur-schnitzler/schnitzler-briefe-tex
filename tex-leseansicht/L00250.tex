%% latex-korrekturansicht-vorspann.tex
%% Vorspann für die Korrekturansicht.
%% Lädt die gemeinsame Datei latex-vorspann.tex mit gesetztem Schalter.

\newif\ifkorrekturansicht
\korrekturansichttrue

\input{../tex-inputs/latex-vorspann}


\section[Arthur Schnitzler an Richard Beer-Hofmann, 11. 8. 1893]{L00250 Arthur Schnitzler an Richard Beer-Hofmann, 11. 8. 1893}
\nopagebreak\mylabel{L00250v}
\rehead{ }\normalsize\beginnumbering\briefempfaengerindex{Beer-Hofmann, Richard@\textsc{Beer-Hofmann, Richard}!zzzSchnitzler, Arthur@\emph{von Arthur Schnitzler}!1893-08-111@{11. 8. 1893}|(be}
\toendnotes[C]{\smallbreak\pagebreak[2]}\Standort{YCGL, MSS 31.}
\physDesc{Brief, 1 Blatt, 1 Seite, 662 Zeichen (Briefpapier mit Trauerrand)
\newline{}Handschrift: schwarze Tinte, deutsche Kurrent}
\buchAbdrucke{\weitereDrucke{Arthur Schnitzler, Richard Beer-Hofmann: \emph{Briefwechsel 1891–1931}. Wien, Zürich: \emph{Europaverlag} 1992, S. 50.} }\toendnotes[C]{\smallbreak}
\pstart
           \noindent{}{\pb}Lieber Richard, warum ſchreiben Sie mir
               nicht? – – Haben Sie Ihre Novelle\pwindex{Camelias@\emph{Camelias}|pwv} vorgeleſen? – Was macht der Götterliebling\pwindex{Tod Georgs@\emph{Der Tod Georgs}|pw}? – Erfuhren Sie was über Freund u \textsc{Jäckel}\orgindex{Freund und Jeckel@Freund {\kaufmannsund}  Jeckel|pw}? – Sehen Sie Benedikt’s\pwindex{Benedict, Marianne 01.01.1848 – 12.05.1930@\textsc{Benedict, Marianne} (01.01.1848 – 12.05.1930)|pw}\pwindex{Benedict, Markus 17.09.1834 – 26.2.1909@\textsc{Benedict, Markus} (17.09.1834 – 26.2.1909), \emph{Industrieller/Industrielle}|pw}? –
               Haben Sie gehört, wie ſchauerlich und wie du{\geminationm} die Abendpoſt\pwindex{Wiener Abendpost@\emph{Wiener Abendpost}|pw} den Anatol\pwindex{Anatol@\emph{Anatol}|pw}{ }verriſs\pwindex{Literatur. »Bunte Reihe.« Ein Geschichtenbuch von Moritz Goldschmidt. »Anatol« von Arthur Schnitzler@\emph{Literatur. »Bunte Reihe.« Ein Geschichtenbuch von Moritz Goldschmidt. »Anatol« von Arthur Schnitzler}|pwv}? – Wa{\geminationn} rücken Sie ein? Wann sind Sie in Wien\oindex{Wien@\textbf{Wien}, \emph{A.ADM2}|pw}? – Ich reiſe vielleicht am 19. oder
                  20. ab. – Sind Sie glücklich? – Sind Sie arrogant? – Wiſſen Sie, daſs
               Sie noch im Herbſt \textsc{Bic}. fahren lernen werden? Was macht
               Frau \textsc{Flegm}.\pwindex{Flegmann, Bertha 27.05.1852 – 24.6.1933@\textsc{Flegmann, Bertha} (27.05.1852 – 24.6.1933), \emph{männliche Salonnière/Salonnière}|pw}? Was das Theater? – Sprachen Sie \textsc{Jarno}\pwindex{Jarno, Josef 24.08.1865 – 11.01.1932@\textsc{Jarno, Josef} (24.08.1865 – 11.01.1932), \emph{Theaterleiter/Theaterleiterin, Schauspieler/Schauspielerin}|pw}? – Die \textsc{Wreden}\pwindex{Wreden, Grethe @\textsc{Wreden, Grethe}, \emph{Schauspieler/Schauspielerin}|pw}? – Stand was in der Iſchler
                  Ztg.\pwindex{Ischler Wochenblatt@\emph{Ischler Wochenblatt}|pwv} über mein Stück\pwindex{Anatol@\emph{Anatol}|pwv}? – Senden Sie – ich vertrage alles\substVorne{}\textsuperscript{?}\substDazwischen{}. –\substHinten{}{ }Goldmann\pwindex{Goldmann, Paul 31.01.1865 – 25.09.1935@\textsc{Goldmann, Paul} (31.01.1865 – 25.09.1935), \emph{Schriftsteller/Schriftstellerin, Journalist/Journalistin}|pw} ko{\geminationm}t im
                  September nach Salzburg\oindex{Salzburg@\textbf{Salzburg}, \emph{A.ADM2}|pw}. –\pend
           \pstart Herzlich der Ihre \spacefill\mbox{Arthur}\pend{}\selectlanguage{ngerman}\endnumbering\briefempfaengerindex{Beer-Hofmann, Richard@\textsc{Beer-Hofmann, Richard}!zzzSchnitzler, Arthur@\emph{von Arthur Schnitzler}!1893-08-111@{11. 8. 1893}|)be}\mylabel{L00250h}  \normalsize

\doendnotes{C}
\bigskip
\vfill

\clearpage

\footnotesize

\lohead{\textsc{register}}

% Definiere theindex-Environment komplett neu ohne reledmac
\makeatletter
\renewenvironment{theindex}{%
  \section*{\indexname}%
  \setlength{\parindent}{0pt}%
  \setlength{\parskip}{0pt plus 0.3pt}%
  \let\item\@idxitem
}{%
  \clearpage
}
\makeatother

\IfFileExists{\jobname-pw.ind}{\input{\jobname-pw.ind}}{}

\end{document}

      