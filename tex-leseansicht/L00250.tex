%% latex-leseansicht-vorspann.tex
%% Vorspann für die Leseansicht.
%% Lädt die gemeinsame Datei latex-vorspann.tex mit nicht gesetztem Schalter.

\newif\ifkorrekturansicht
\korrekturansichtfalse

\input{../tex-inputs/latex-vorspann}


\section[Arthur Schnitzler an Richard Beer-Hofmann, 11. 8. 1893]{L00250 Arthur Schnitzler an Richard Beer-Hofmann, 11. 8. 1893}
\nopagebreak\mylabel{L00250v}
\rehead{ }\normalsize\beginnumbering\briefempfaengerindex{Beer-Hofmann, Richard@\textsc{Beer-Hofmann, Richard}!zzzSchnitzler, Arthur@\emph{von Arthur Schnitzler}!1893-08-111@{11. 8. 1893}|(be}
\toendnotes[C]{\smallbreak\pagebreak[2]}
\correspDesc{Versand  durch Arthur Schnitzler am 11. 8. 1893 in Wien
\newline{}Erhalt  durch Richard Beer-Hofmann im Zeitraum [12. 8. 1893
                  – 16. 8. 1893?] in Bad Ischl}\toendnotes[C]{\smallbreak}
\Standort{YCGL, MSS 31.}
\physDesc{Brief, 1 Blatt, 1 Seite, 662 Zeichen (Briefpapier mit Trauerrand)
\newline{}Handschrift: schwarze Tinte, deutsche Kurrent}
\buchAbdrucke{\weitereDrucke{Arthur Schnitzler, Richard Beer-Hofmann: \emph{Briefwechsel 1891–1931}. Herausgegeben von Konstanze Fliedl. Wien, Zürich: \emph{Europaverlag} 1992, S. 50.} }\toendnotes[C]{\smallbreak}
\pstart
           \noindent{}{\pb}Lieber Richard, warum{ }ſchreiben Sie mir
               nicht? – – Haben Sie Ihre Novelle\pwindex{Beer-Hofmann, Richard 11.\,7.\,1866 Wien – 26.\,9.\,1945 New York City@\textsc{Beer-Hofmann, Richard} (11.\,7.\,1866 Wien – 26.\,9.\,1945 New York City), \emph{Schriftsteller}!Camelias@\strich\emph{Camelias}|pwv} vorgeleſen? – Was macht der Götterliebling\pwindex{Beer-Hofmann, Richard 11.\,7.\,1866 Wien – 26.\,9.\,1945 New York City@\textsc{Beer-Hofmann, Richard} (11.\,7.\,1866 Wien – 26.\,9.\,1945 New York City), \emph{Schriftsteller}!Tod Georgs@\strich\emph{Der Tod Georgs}|pw}? – Erfuhren Sie was über Freund u \textsc{Jäckel}\orgindex{Freund und Jeckel@Freund {\kaufmannsund}  Jeckel|pw}? – Sehen Sie Benedikt’s\pwindex{Benedict, Marianne 1.\,1.\,1848 Bratislava – 12.\,5.\,1930 Wien@\textsc{Benedict, Marianne} (1.\,1.\,1848 Bratislava – 12.\,5.\,1930 Wien)|pw}\pwindex{Benedict, Markus 17.\,9.\,1834 Mikulov – 26.\,2.\,1909 Kärntnerring 13@\textsc{Benedict, Markus} (17.\,9.\,1834 Mikulov – 26.\,2.\,1909 Kärntnerring 13), \emph{Industrieller}|pw}? –
               Haben Sie gehört, wie{ }ſchauerlich und wie du{\geminationm} die Abendpoſt\pwindex{Wiener Abendpost@\emph{Wiener Abendpost}|pw} den Anatol\pwindex{Schnitzler, Arthur 15.\,5.\,1862 Wien – 21.\,10.\,1931 ebd.@\textsc{Schnitzler, Arthur} (15.\,5.\,1862 Wien – 21.\,10.\,1931 ebd.), \emph{Schriftsteller, Mediziner}!Anatol@\strich\emph{Anatol}|pw}{ }verriſs\pwindex{\textcolor{red}{\textsuperscript{XXXX indx1}}!Literatur. »Bunte Reihe.« Ein Geschichtenbuch von Moritz Goldschmidt. »Anatol« von Arthur Schnitzler@\strich\emph{Literatur. »Bunte Reihe.« Ein Geschichtenbuch von Moritz Goldschmidt. »Anatol« von Arthur Schnitzler}|pwv}? – Wa{\geminationn} rücken Sie ein? Wann sind Sie in Wien\oindex{Wien@\textbf{Wien}, \emph{Verwaltungsgebiet}|pw}? – Ich reiſe vielleicht am 19. oder
                  20. ab. – Sind Sie glücklich? – Sind Sie arrogant? – Wiſſen Sie, daſs
               Sie noch im Herbſt \textsc{Bic}. fahren lernen werden? Was macht
               Frau \textsc{Flegm}.\pwindex{Flegmann, Bertha 27.\,5.\,1852 Dubrovsky, Polen – 24.\,6.\,1933 Bad Ischl@\textsc{Flegmann, Bertha} (27.\,5.\,1852 Dubrovsky, Polen – 24.\,6.\,1933 Bad Ischl), \emph{Salonnière}|pw}? Was das Theater? – Sprachen Sie \textsc{Jarno}\pwindex{Jarno, Josef 24.\,8.\,1865 Budapest – 11.\,1.\,1932 Wien@\textsc{Jarno, Josef} (24.\,8.\,1865 Budapest – 11.\,1.\,1932 Wien), \emph{Theaterleiter, Schauspieler}|pw}? – Die \textsc{Wreden}\pwindex{Wreden, Grethe @\textsc{Wreden, Grethe}, \emph{Schauspielerin}|pw}? – Stand was in der Iſchler
                  Ztg.\pwindex{Ischler Wochenblatt@\emph{Ischler Wochenblatt}|pwv} über mein Stück\pwindex{Schnitzler, Arthur 15.\,5.\,1862 Wien – 21.\,10.\,1931 ebd.@\textsc{Schnitzler, Arthur} (15.\,5.\,1862 Wien – 21.\,10.\,1931 ebd.), \emph{Schriftsteller, Mediziner}!Anatol@\strich\emph{Anatol}|pwv}? – Senden Sie – ich vertrage alles\substVorne{}\textsuperscript{?}\substDazwischen{}. –\substHinten{}{ }Goldmann\pwindex{Goldmann, Paul 31.\,1.\,1865 Breslau – 25.\,9.\,1935 Wien@\textsc{Goldmann, Paul} (31.\,1.\,1865 Breslau – 25.\,9.\,1935 Wien), \emph{Schriftsteller, Journalist}|pw} ko{\geminationm}t im
                  September nach Salzburg\oindex{Salzburg@\textbf{Salzburg}, \emph{Verwaltungsgebiet}|pw}. –\pend
           \pstart Herzlich der Ihre \spacefill\mbox{Arthur}\pend{}\selectlanguage{ngerman}\endnumbering\briefempfaengerindex{Beer-Hofmann, Richard@\textsc{Beer-Hofmann, Richard}!zzzSchnitzler, Arthur@\emph{von Arthur Schnitzler}!1893-08-111@{11. 8. 1893}|)be}\mylabel{L00250h}  \newcommand{\dateiname}{L00250}\newcommand{\titel}{Arthur Schnitzler an Richard Beer-Hofmann, 11. 8. 1893}\newcommand{\editorInnen}{Martin Anton Müller und Gerd-Hermann Susen}%% latex-leseansicht-abspann.tex
%% Abspann für die Leseansicht.
%% Der Schalter \ifkorrekturansicht ist bereits durch den Vorspann gesetzt.

%% latex-abspann.tex
%% Gemeinsamer Abspann für Korrekturansicht und Leseansicht.
%% Setzt den Schalter \ifkorrekturansicht voraus (gesetzt in den
%% einbindenden Dateien latex-korrekturansicht-abspann.tex bzw.
%% latex-leseansicht-abspann.tex).
%% ---------------------------------------------------------------

\normalsize

% Das esempio-Environment wird nur in der Leseansicht benötigt
\ifkorrekturansicht\else
\newenvironment{esempio}[3]%
{
    \vspace{1.5ex}
    \rlap{\underline{#1}}
    \par
    \setlength{\parindent}{0cm}
    \nopagebreak
    \leftskip=#2cm
    \rightskip=#3cm
}
{
    \par
}
\fi

\doendnotes{C}
\bigskip
\vfill

\clearpage

\footnotesize

\ifkorrekturansicht
  \lohead{\textsc{register}}
\fi

% theindex-Environment neu definieren ohne reledmac
\makeatletter
\renewenvironment{theindex}{%
  \ifkorrekturansicht
    \section*{\indexname}%
  \else
    \subsubsection*{Index der erwähnten Entitäten}%
  \fi
  \setlength{\parindent}{0pt}%
  \setlength{\parskip}{0pt plus 0.3pt}%
  \let\item\@idxitem
}{%
  \ifkorrekturansicht\clearpage\fi
}
\makeatother

\IfFileExists{\jobname-pw.ind}{\input{\jobname-pw.ind}}{}

% Quellenangabe nur in der Leseansicht
\ifkorrekturansicht\else
% Fallback-Definitionen, falls die .tex-Datei \titel etc. nicht gesetzt hat
\providecommand{\titel}{}
\providecommand{\editorInnen}{}
\providecommand{\dateiname}{\jobname}

\vspace{3cm}

\vfill

\footnotesize
\textsc{Quelle}: \titel. Herausgegeben von {\editorInnen}. In: \emph{Arthur Schnitzler: Briefwechsel mit Autorinnen und Autoren}.
 Digitale Edition, https://schnitzler-briefe.acdh.oeaw.ac.at/{\dateiname}.html (Stand \today)
\fi

\end{document}


