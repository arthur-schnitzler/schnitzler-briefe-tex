\input{../tex-inputs/latex-pdf-vorspann}
\begin{center}
            \textcolor{red}{ENTWURF. ENTZIFFERUNG NOCH NICHT KORREKTURGELESEN}
                      \end{center}
            
               \section[Arthur Schnitzler an Richard Beer-Hofmann, 11. 8. 1893]{ Arthur Schnitzler an Richard Beer-Hofmann, 11. 8. 1893}\nopagebreak\mylabel{v}\rehead{ }\begin{ledgroupsized}[t]{13cm}\normalsize\beginnumbering\briefempfaengerindex{Beer-Hofmann, Richard@\textsc{Beer-Hofmann, Richard}!zzzSchnitzler, Arthur@\emph{von Arthur Schnitzler}!1893-08-111@{11. 8. 1893}|(be} \toendnotes[C]{\smallbreak\pagebreak[2]} \Standort{YCGL, MSS 31.}
\physDesc{Brief, 1 Blatt (Briefpapier mit Trauerrand), 1 Seite
\newline{}Handschrift: schwarze Tinte, deutsche Kurrent}\buchAbdrucke{\weitereDrucke{Arthur Schnitzler, Richard Beer-Hofmann: \emph{Briefwechsel 1891–1931}. Hg. Konstanze Fliedl. Wien, Zürich: \emph{Europaverlag} 1992, S. 50.} }\toendnotes[C]{\smallbreak}\pstart
           \noindent{}{\pb}Lieber Richard, warum ſchreiben Sie
               mir nicht? – – Haben Sie Ihre Novelle\pwindex{Beer-Hofmann, Richard 11.07.1866 – 26.09.1945@\textsc{Beer-Hofmann, Richard} (11.07.1866 – 26.09.1945), \emph{Schriftsteller}!Camelias1893@\strich\emph{Camelias} {[}1893{]}|pwv}
               vorgeleſen? – Was macht der Götterliebling\pwindex{Beer-Hofmann, Richard 11.07.1866 – 26.09.1945@\textsc{Beer-Hofmann, Richard} (11.07.1866 – 26.09.1945), \emph{Schriftsteller}!Tod Georgs1900@\strich\emph{Der Tod Georgs} {[}1900{]}|pw}? – Erfuhren Sie was über Freund u \textsc{Jäckel}\orgindex{Freund und Jeckel@Freund {\kaufmannsund}  Jeckel|pw}? – Sehen Sie Benedikt’s\pwindex{Benedict, Marianne 01.01.1848 – 12.05.1930@\textsc{Benedict, Marianne} (01.01.1848 – 12.05.1930)|pw}\pwindex{Benedict, Markus 17.09.1834 – 26.2.1909@\textsc{Benedict, Markus} (17.09.1834 – 26.2.1909), \emph{Industrieller}|pw}? – Haben Sie gehört, wie ſchauerlich und wie du{\geminationm} die Abendpoſt\pwindex{Wiener Abendpost1.7.1863 – 31.12.1921@\emph{Wiener Abendpost}|pw} den
                  Anatol\pwindex{Schnitzler, Arthur 15.05.1862 – 21.10.1931@\textsc{Schnitzler, Arthur} (15.05.1862 – 21.10.1931), \emph{Schriftsteller, Mediziner}!Anatol1892-10-29 – 1892-10-29@\strich\emph{Anatol} {[}1892-10-29 – 1892-10-29{]}|pw}{ }verriſs\pwindex{Literatur. »Bunte Reihe.« Ein Geschichtenbuch von Moritz Goldschmidt. »Anatol« von Arthur Schnitzler3.8.1893 – 3.8.1893@\emph{Literatur. »Bunte Reihe.« Ein Geschichtenbuch von Moritz Goldschmidt. »Anatol« von Arthur Schnitzler} {[}3.8.1893 – 3.8.1893{]}|pwv}? – Wa{\geminationn}
               rücken Sie ein? Wann sind Sie in Wien\oindex{Wien@\textbf{Wien}|pw}? – Ich
               reiſe vielleicht am 19. oder 20. ab. – Sind Sie
               glücklich? – Sind Sie arrogant? – Wiſſen Sie, daſs Sie noch im Herbſt \textsc{Bic}. fahren lernen werden? Was macht Frau \textsc{Flegm}.\pwindex{Flegmann, Bertha 27.05.1852 – 24.6.1933@\textsc{Flegmann, Bertha} (27.05.1852 – 24.6.1933), \emph{Salonnière}|pw}? Was das Theater? – Sprachen Sie
                  \textsc{Jarno}\pwindex{Jarno, Josef 24.08.1865 – 11.01.1932@\textsc{Jarno, Josef} (24.08.1865 – 11.01.1932), \emph{Theaterleiter, Schauspieler}|pw}? – Die \textsc{Wreden}\pwindex{Wreden, Grethe @\textsc{Wreden, Grethe}, \emph{Schauspielerin}|pw}? – Stand was in der Iſchler
                  Ztg.\pwindex{Ischler Wochenblatt1876 – 1915@\emph{Ischler Wochenblatt}|pwv} über mein Stück\pwindex{Schnitzler, Arthur 15.05.1862 – 21.10.1931@\textsc{Schnitzler, Arthur} (15.05.1862 – 21.10.1931), \emph{Schriftsteller, Mediziner}!Anatol1892-10-29 – 1892-10-29@\strich\emph{Anatol} {[}1892-10-29 – 1892-10-29{]}|pwv}? – Senden Sie
               – ich vertrage alles\substVorne{}\textsuperscript{?}\substDazwischen{}. –\substHinten{}{ }Goldmann\pwindex{Goldmann, Paul 31.01.1865 – 25.09.1935@\textsc{Goldmann, Paul} (31.01.1865 – 25.09.1935), \emph{Schriftsteller, Journalist}|pw} ko{\geminationm}t im September
               nach Salzburg\oindex{Salzburg@\textbf{Salzburg}|pw}. –\pend
           \pstart Herzlich der Ihre \spacefill\mbox{Arthur}\pend{}\endnumbering\briefempfaengerindex{Beer-Hofmann, Richard@\textsc{Beer-Hofmann, Richard}!zzzSchnitzler, Arthur@\emph{von Arthur Schnitzler}!1893-08-111@{11. 8. 1893}|)be}\mylabel{h}\end{ledgroupsized}  \newcommand{\dateiname}{L00250}\newcommand{\titel}{Arthur Schnitzler an Richard Beer-Hofmann, 11. 8. 1893}\newcommand{\editorInnen}{Martin Anton Müller und Gerd-Hermann Susen}\input{../tex-inputs/latex-pdf-abspann}
      