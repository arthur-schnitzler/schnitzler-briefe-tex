%% latex-leseansicht-vorspann.tex
%% Vorspann für die Leseansicht.
%% Lädt die gemeinsame Datei latex-vorspann.tex mit nicht gesetztem Schalter.

\newif\ifkorrekturansicht
\korrekturansichtfalse

\input{../tex-inputs/latex-vorspann}


         
         \renewcommand{\erwaehntePersonen}{Personen: Richard Beer-Hofmann, Marianne Benedict, Markus Benedict, Bertha Flegmann, Paul Goldmann, Josef Jarno, Grethe Wreden}
         \renewcommand{\erwaehnteInstitutionen}{Institutionen: Freund {\kaufmannsund}  Jeckel}
         \renewcommand{\erwaehnteOrte}{Orte: Bad Ischl, Salzburg, Wien}
         \renewcommand{\erwaehnteWerke}{Werke: Anatol, Camelias, Der Tod Georgs, Ischler Wochenblatt, Literatur. »Bunte Reihe.« Ein Geschichtenbuch von Moritz Goldschmidt. »Anatol« von Arthur Schnitzler, Wiener Abendpost}
               \section[Arthur Schnitzler an Richard Beer-Hofmann, 11. 8. 1893]{ Arthur Schnitzler an Richard Beer-Hofmann, 11. 8. 1893}\nopagebreak\mylabel{v}\rehead{ }\begin{ledgroupsized}[t]{13cm}\normalsize\beginnumbering \toendnotes[C]{\smallbreak\pagebreak[2]} \Standort{YCGL, MSS 31.}
\physDesc{Brief, 1 Blatt, 1 Seite, 662 Zeichen (Briefpapier mit Trauerrand)
\newline{}Handschrift: schwarze Tinte, deutsche Kurrent}\buchAbdrucke{\weitereDrucke{Arthur Schnitzler, Richard Beer-Hofmann: \emph{Briefwechsel 1891–1931}. Hg. Konstanze Fliedl. Wien, Zürich: \emph{Europaverlag} 1992, S. 50.} }\toendnotes[C]{\smallbreak}\pstart
           \noindent{}{\pb}Lieber Richard, warum ſchreiben Sie mir
               nicht? – – Haben Sie Ihre Novelle\pwindex{Beer-Hofmann, Richard 1866-07-11 – 1945-09-26@\textsc{Beer-Hofmann, Richard} (1866-07-11 – 1945-09-26), \emph{Schriftsteller}!Camelias1893@\strich\emph{Camelias} {[}1893{]}|pwv} vorgeleſen? – Was macht der Götterliebling\pwindex{Beer-Hofmann, Richard 1866-07-11 – 1945-09-26@\textsc{Beer-Hofmann, Richard} (1866-07-11 – 1945-09-26), \emph{Schriftsteller}!Tod Georgs1900@\strich\emph{Der Tod Georgs} {[}1900{]}|pw}? – Erfuhren Sie was über Freund u \textsc{Jäckel}\orgindex{Freund und Jeckel@Freund {\kaufmannsund}  Jeckel|pw}? – Sehen Sie Benedikt’s\pwindex{Benedict, Marianne 01.01.1848 – 12.05.1930@\textsc{Benedict, Marianne} (01.01.1848 – 12.05.1930)|pw}\pwindex{Benedict, Markus 17.09.1834 – 26.2.1909@\textsc{Benedict, Markus} (17.09.1834 – 26.2.1909), \emph{Industrieller}|pw}? –
               Haben Sie gehört, wie ſchauerlich und wie du{\geminationm} die Abendpoſt\pwindex{?? Werk@Nicht ermittelte Verfasserinnen und Verfasser!Wiener Abendpost1.7.1863 – 31.12.1921@\emph{Wiener Abendpost} {[}1.7.1863 – 31.12.1921{]}|pw} den Anatol\pwindex{Schnitzler, Arthur 15.05.1862 – 21.10.1931@\textsc{Schnitzler, Arthur} (15.05.1862 – 21.10.1931), \emph{Schriftsteller, Mediziner}!Anatol1892-10-29@\strich\emph{Anatol} {[}1892-10-29{]}|pw}{ }verriſs\pwindex{Literatur. »Bunte Reihe.« Ein Geschichtenbuch von Moritz Goldschmidt. »Anatol«
                  von Arthur Schnitzler3. 8. 1893@\emph{Literatur. »Bunte Reihe.« Ein Geschichtenbuch von Moritz Goldschmidt. »Anatol« von Arthur Schnitzler} {[}3. 8. 1893{]}|pwv}? – Wa{\geminationn} rücken Sie ein? Wann sind Sie in Wien\oindex{Wien@\textbf{Wien}|pw}? – Ich reiſe vielleicht am 19. oder
                  20. ab. – Sind Sie glücklich? – Sind Sie arrogant? – Wiſſen Sie, daſs
               Sie noch im Herbſt \textsc{Bic}. fahren lernen werden? Was macht
               Frau \textsc{Flegm}.\pwindex{Flegmann, Bertha 27.05.1852 – 24.6.1933@\textsc{Flegmann, Bertha} (27.05.1852 – 24.6.1933), \emph{Salonnière}|pw}? Was das Theater? – Sprachen Sie \textsc{Jarno}\pwindex{Jarno, Josef 24.08.1865 – 11.01.1932@\textsc{Jarno, Josef} (24.08.1865 – 11.01.1932), \emph{Theaterleiter, Schauspieler}|pw}? – Die \textsc{Wreden}\pwindex{Wreden, Grethe @\textsc{Wreden, Grethe}, \emph{Schauspielerin}|pw}? – Stand was in der Iſchler
                  Ztg.\pwindex{?? Werk@Nicht ermittelte Verfasserinnen und Verfasser!Ischler Wochenblatt1876 – 1915@\emph{Ischler Wochenblatt} {[}1876 – 1915{]}|pwv} über mein Stück\pwindex{Schnitzler, Arthur 15.05.1862 – 21.10.1931@\textsc{Schnitzler, Arthur} (15.05.1862 – 21.10.1931), \emph{Schriftsteller, Mediziner}!Anatol1892-10-29@\strich\emph{Anatol} {[}1892-10-29{]}|pwv}? – Senden Sie – ich vertrage alles\substVorne{}\textsuperscript{?}\substDazwischen{}. –\substHinten{}{ }Goldmann\pwindex{Goldmann, Paul 31.01.1865 – 25.09.1935@\textsc{Goldmann, Paul} (31.01.1865 – 25.09.1935), \emph{Schriftsteller, Journalist}|pw} ko{\geminationm}t im
                  September nach Salzburg\oindex{Salzburg@\textbf{Salzburg}|pw}. –\pend
           \pstart Herzlich der Ihre \spacefill\mbox{Arthur}\pend{}
         
         \endnumbering\mylabel{h}\end{ledgroupsized}  \newcommand{\dateiname}{L00250}\newcommand{\titel}{Arthur Schnitzler an Richard Beer-Hofmann, 11. 8. 1893}\newcommand{\editorInnen}{Martin Anton Müller und Gerd-Hermann Susen}%% latex-leseansicht-abspann.tex
%% Abspann für die Leseansicht.
%% Der Schalter \ifkorrekturansicht ist bereits durch den Vorspann gesetzt.

%% latex-abspann.tex
%% Gemeinsamer Abspann für Korrekturansicht und Leseansicht.
%% Setzt den Schalter \ifkorrekturansicht voraus (gesetzt in den
%% einbindenden Dateien latex-korrekturansicht-abspann.tex bzw.
%% latex-leseansicht-abspann.tex).
%% ---------------------------------------------------------------

\normalsize

% Das esempio-Environment wird nur in der Leseansicht benötigt
\ifkorrekturansicht\else
\newenvironment{esempio}[3]%
{
    \vspace{1.5ex}
    \rlap{\underline{#1}}
    \par
    \setlength{\parindent}{0cm}
    \nopagebreak
    \leftskip=#2cm
    \rightskip=#3cm
}
{
    \par
}
\fi

\doendnotes{C}
\bigskip
\vfill

\clearpage

\footnotesize

\ifkorrekturansicht
  \lohead{\textsc{register}}
\fi

% theindex-Environment neu definieren ohne reledmac
\makeatletter
\renewenvironment{theindex}{%
  \ifkorrekturansicht
    \section*{\indexname}%
  \else
    \subsubsection*{Index der erwähnten Entitäten}%
  \fi
  \setlength{\parindent}{0pt}%
  \setlength{\parskip}{0pt plus 0.3pt}%
  \let\item\@idxitem
}{%
  \ifkorrekturansicht\clearpage\fi
}
\makeatother

\IfFileExists{\jobname-pw.ind}{\input{\jobname-pw.ind}}{}

% Quellenangabe nur in der Leseansicht
\ifkorrekturansicht\else
% Fallback-Definitionen, falls die .tex-Datei \titel etc. nicht gesetzt hat
\providecommand{\titel}{}
\providecommand{\editorInnen}{}
\providecommand{\dateiname}{\jobname}

\vspace{3cm}

\vfill

\footnotesize
\textsc{Quelle}: \titel. Herausgegeben von {\editorInnen}. In: \emph{Arthur Schnitzler: Briefwechsel mit Autorinnen und Autoren}.
 Digitale Edition, https://schnitzler-briefe.acdh.oeaw.ac.at/{\dateiname}.html (Stand \today)
\fi

\end{document}


      