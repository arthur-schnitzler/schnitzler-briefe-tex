%% latex-leseansicht-vorspann.tex
%% Vorspann für die Leseansicht.
%% Lädt die gemeinsame Datei latex-vorspann.tex mit nicht gesetztem Schalter.

\newif\ifkorrekturansicht
\korrekturansichtfalse

\input{../tex-inputs/latex-vorspann}

\begin{center}
            \textcolor{red}{ENTWURF. ENTZIFFERUNG NOCH NICHT KORREKTURGELESEN}
                      \end{center}
            
               \section[Arthur Schnitzler an Hugo von Hofmannsthal, 28. 3. 1914]{ Arthur Schnitzler an Hugo von Hofmannsthal, 28. 3. 1914}\nopagebreak\mylabel{v}\rehead{ }\begin{ledgroupsized}[t]{13cm}\normalsize\beginnumbering\briefempfaengerindex{Hofmannsthal, Hugo von@\textsc{Hofmannsthal, Hugo von}!zzzSchnitzler, Arthur@\emph{von Arthur Schnitzler}!1914-03-281@{28. 3. 1914}|(be} \toendnotes[C]{\smallbreak\pagebreak[2]} \Standort{FDH, Hs-30885,147.}
\physDesc{Brief, 1 Blatt, 2 Seiten
\newline{}Handschrift: schwarze Tinte, deutsche Kurrent}\buchAbdrucke{\weitereDrucke{1) Hugo von Hofmannsthal, Arthur Schnitzler: \emph{Briefwechsel}. Hg. Therese Nickl und Heinrich Schnitzler. Frankfurt am Main: \emph{S. Fischer} 1964, S. 273–274.} \weitereDrucke{2) Arthur Schnitzler: \emph{Briefe 1913–1931}. Hg. Peter Michael Braunwarth, Richard Miklin, Susanne Pertlik und Heinrich Schnitzler. Frankfurt am Main: \emph{S. Fischer} 1984, S. 36–37.} }\toendnotes[C]{\smallbreak}\pstart
           \raggedleft{}{\pb}Wien\oindex{Wien@\textbf{Wien}|pw}, 28/3 914\pend
           \pstart
           mein lieber Hugo, ich danke Ihnen ſehr für Ihre Gratulation zum \textsc{Raimund}preis\orgindex{Raimund-Preis@Raimund-Preis|pw}; und will Ihnen für alle Fälle gleich
               ſagen, daſs \uline{Sie}{ }\uline{mir} gewiſs nicht zum Schatten geworden ſind und es
               niemals werden können. We{\geminationn} unſre Beziehungen ein wenig
               loſer geworden ſind, oder beſſer geſagt, ſich \introOben{}eben\introOben{} in einer
               loſeren Epoche befinden, ſo iſt daran wohl mehr äußeres als inneres ſchuld, \strikeout{\textcolor{gray}{i{\geminationm}}} und
               daſs Sie eher geneigt ſind, nach mir zu rufen als ich nach Ihnen liegt wohl
               hauptſächlich daran, daſs Sie oft »ſowieſo« nach Wien\oindex{Wien@\textbf{Wien}|pw}
                  ko{\geminationm}en, ich aber nie »ſowieſo« nach Rodaun\oindex{Rodaun@\textbf{Rodaun}|pw} – ferner daran: daſs wir’s uns beide, wohl aus unſrer
               Natur heraus ſo und nicht anders eingerichtet haben. Und ſo käm ich jetzt wohl auch
               auf den Semmering\oindex{Semmering@\textbf{Semmering}|pw} – we{\geminationn}
               mir die Wetterverhältniſſe um dieſe Zeit oben nicht ſo unangenehm wären. Ändert ſichs
               noch beträchtlich, ſo meld ich mich vielleicht. Andernfalls möcht ich Sie im Thal ſo
               bald es angeht, ſehn; denn ich glaube, {\pb}Sie haben das
               Bedürfnis mir von Ihrer neuen Arbeit\pwindex{Hofmannsthal, Hugo von 01.02.1874 – 15.07.1929@\textsc{Hofmannsthal, Hugo von} (01.02.1874 – 15.07.1929), \emph{Schriftsteller}!Frau ohne Schatten. Erzaehlung1919 – 1919@\strich\emph{Die Frau ohne Schatten. Erzählung} {[}1919 – 1919{]}|pwv} was zu erzählen – und ich rechne es wie Ihnen nicht unbekannt iſt,
               immer zu meinen beſten Stunden, we{\geminationn} Sie ſich zu mir über
               Ihre Sachen ausſprechen. Und aus ſolchen Stunden ſcheiden wir, wie Sie wohl auch
               ſchon oft gefühlt haben, ſo in beſten Si{\geminationn}en verbunden,
               daſs ein Auseinanderlaufen äußerer Lebenslinien für das weſentliche unſrer
               Beziehungen \substVorne{}\textsuperscript{hin}\substDazwischen{}auf\substHinten{} längre Zeit \introOben{}hin\introOben{} ohne Bedeutung, we{\geminationn} auch oft mit einiger Wehmut zu empfinden bleibt. Im
               ganzen aber glaub ich, trotz aller Ehrfurcht vor dem Geſetz der Entwicklung, immer
               mehr an die Conſtanz der \introOben{}menſchlichen\introOben{} Beziehungen \introOben{}ſo\introOben{}wie an die der Menſchen: was aus uns und aus andern wird,
               hat Ahnung längst vorausempfunden, und jeder Wolkendunſt unſrer Jugend, der ſich
               harmlos zu verziehen ſchien, ko{\geminationm}t irgend einmal als
               Gewitter wieder. Von dieſem Ausflug ins Allgemeinere oder Halbwahre kehre ich in die
               Realität gerne wieder, wo ich Sie ſehr bald, und ich hoffe in beſſerer Sti{\geminationm}ung als Ihr Brief mir vertraut, zu ſehn u ſprechen
               wünſche.\pend
           \pstart Herzlichſt Ihr \spacefill\mbox{Arthur.}\pend{}\endnumbering\briefempfaengerindex{Hofmannsthal, Hugo von@\textsc{Hofmannsthal, Hugo von}!zzzSchnitzler, Arthur@\emph{von Arthur Schnitzler}!1914-03-281@{28. 3. 1914}|)be}\mylabel{h}\end{ledgroupsized}  \newcommand{\dateiname}{L02168}\newcommand{\titel}{Arthur Schnitzler an Hugo von Hofmannsthal, 28. 3. 1914}\newcommand{\editorInnen}{Martin Anton Müller und Gerd-Hermann Susen}%% latex-leseansicht-abspann.tex
%% Abspann für die Leseansicht.
%% Der Schalter \ifkorrekturansicht ist bereits durch den Vorspann gesetzt.

%% latex-abspann.tex
%% Gemeinsamer Abspann für Korrekturansicht und Leseansicht.
%% Setzt den Schalter \ifkorrekturansicht voraus (gesetzt in den
%% einbindenden Dateien latex-korrekturansicht-abspann.tex bzw.
%% latex-leseansicht-abspann.tex).
%% ---------------------------------------------------------------

\normalsize

% Das esempio-Environment wird nur in der Leseansicht benötigt
\ifkorrekturansicht\else
\newenvironment{esempio}[3]%
{
    \vspace{1.5ex}
    \rlap{\underline{#1}}
    \par
    \setlength{\parindent}{0cm}
    \nopagebreak
    \leftskip=#2cm
    \rightskip=#3cm
}
{
    \par
}
\fi

\doendnotes{C}
\bigskip
\vfill

\clearpage

\footnotesize

\ifkorrekturansicht
  \lohead{\textsc{register}}
\fi

% theindex-Environment neu definieren ohne reledmac
\makeatletter
\renewenvironment{theindex}{%
  \ifkorrekturansicht
    \section*{\indexname}%
  \else
    \subsubsection*{Index der erwähnten Entitäten}%
  \fi
  \setlength{\parindent}{0pt}%
  \setlength{\parskip}{0pt plus 0.3pt}%
  \let\item\@idxitem
}{%
  \ifkorrekturansicht\clearpage\fi
}
\makeatother

\IfFileExists{\jobname-pw.ind}{\input{\jobname-pw.ind}}{}

% Quellenangabe nur in der Leseansicht
\ifkorrekturansicht\else
% Fallback-Definitionen, falls die .tex-Datei \titel etc. nicht gesetzt hat
\providecommand{\titel}{}
\providecommand{\editorInnen}{}
\providecommand{\dateiname}{\jobname}

\vspace{3cm}

\vfill

\footnotesize
\textsc{Quelle}: \titel. Herausgegeben von {\editorInnen}. In: \emph{Arthur Schnitzler: Briefwechsel mit Autorinnen und Autoren}.
 Digitale Edition, https://schnitzler-briefe.acdh.oeaw.ac.at/{\dateiname}.html (Stand \today)
\fi

\end{document}


      