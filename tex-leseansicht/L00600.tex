%% latex-korrekturansicht-vorspann.tex
%% Vorspann für die Korrekturansicht.
%% Lädt die gemeinsame Datei latex-vorspann.tex mit gesetztem Schalter.

\newif\ifkorrekturansicht
\korrekturansichttrue

\input{../tex-inputs/latex-vorspann}


\section[Georg Brandes an Arthur Schnitzler, 6. 10. 1896]{L00600 Georg Brandes an Arthur Schnitzler, 6. 10. 1896}
\nopagebreak\mylabel{L00600v}
\rehead{ }\normalsize\beginnumbering\briefempfaengerindex{Schnitzler, Arthur@\textsc{Schnitzler, Arthur}!zzzBrandes, Georg@\emph{von Georg Brandes}!1896-10-061@{6. 10. 1896}|(be}
\toendnotes[C]{\smallbreak\pagebreak[2]}\Standort{CUL, Schnitzler, B 17.}
\physDesc{Postkarte, 857 Zeichen
\newline{}Handschrift: blaue Tinte, lateinische Kurrent
\newline{}Versand: 1) Stempel: »\nobreak{}\oindex{Kopenhagen@\textbf{Kopenhagen}, \emph{P.PPLC}|pwk}Kjobenhavn, 6. 10.96, 5–5E\nobreak{}«.   2) Stempel: »\nobreak{}\oindex{III., Landstrasse@\textbf{III., Landstraße}, \emph{A.ADM3}|pwk}Wien 3/3, 8. 10.96, 8.V\nobreak{}«. 
\newline{}Ordnung: von unbekannter Hand nummeriert: »3« }
\buchAbdrucke{\weitereDrucke{Georg Brandes, Arthur Schnitzler: \emph{Ein Briefwechsel}. Bern: \emph{Francke} 1956, S. 58.} }\toendnotes[C]{\smallbreak}\pstart{}{\pb}Herrn Dr. Arthur
                  Schnitzler\pend{}\pstart{}Frankgasse 1\oindex{Frankgasse 1@\textbf{Frankgasse 1}, \emph{Wohngebäude (K.WHS)}|pw}\pend{}\pstart{}Wien IX\oindex{IX., Alsergrund@\textbf{IX., Alsergrund}, \emph{A.ADM3}|pw}\pend{}{\bigskip}\vspace{1em}
\pstart
           \raggedleft{}{\pb}Kopenhagen\oindex{Kopenhagen@\textbf{Kopenhagen}, \emph{P.PPLC}|pw}{ }6 Oct\pend
           \vspace{0.5em}
\pstart
           Lieber Herr Schnitzler! Könnten Sie mir nicht ein Bischen zu Hülfe
               kommen. Mir wird ein Numero der \uline{Zeit}\orgindex{Zeit. Wiener Wochenschrift@Die Zeit. Wiener Wochenschrift|pw} geschickt, worin als von mir eingesandt ein Bruchstück\pwindex{Censur in Polen@\emph{Censur in Polen}|pwv} meines alten Buches\pwindex{Polen@\emph{Polen}|pwv} über Polen\oindex{Polen@\textbf{Polen}, \emph{A.PCLI}|pw} sich
               findet. Es ist vor \uline{10} Jahren herausgegeben, und die
               Zeitangaben passen darauf; nun steht es da als von \uline{heute} stammend. Wenn ich doch wenigstens eine Correctur dieser Sachen sähe!
               Es wimmelt von Missverständnissen. Die Fehler sind derart dass das dänische\oindex{Daenemark@\textbf{Dänemark}, \emph{A.PCLI}|pw} Wort \uline{Rædsel} (horror, horreur, Schrecken) übersetzt ist \uline{Räthsel}.\hspace*{5em} Ich erfahre, dass kürzlich in
                  Berlin\oindex{Berlin@\textbf{Berlin}, \emph{P.PPLC}|pw} ein Buch mit meinem Namen versehen
               erschienen ist \uline{Aus dem Reiche des Absolutismus}\pwindex{Eindruecke aus Russland@\emph{Eindrücke aus Russland}|pw} (!) Welcher Titel. Es sind wohl meine »Eindrücke aus Rusland\pwindex{Eindruecke aus Russland@\emph{Eindrücke aus Russland}|pw}«. Es ist mir nicht geschickt worden. \introOben{}Es ist der 9\textsuperscript{te} nicht autorisirte Band
                  von mir in Einem Jahre.\introOben{}\pend
           \pstart Ihr ergebener \spacefill\mbox{Georg Brandes}\pend{}\selectlanguage{ngerman}\endnumbering\briefempfaengerindex{Schnitzler, Arthur@\textsc{Schnitzler, Arthur}!zzzBrandes, Georg@\emph{von Georg Brandes}!1896-10-061@{6. 10. 1896}|)be}\mylabel{L00600h}  \normalsize

\doendnotes{C}
\bigskip
\vfill

\clearpage

\footnotesize

\lohead{\textsc{register}}

% Definiere theindex-Environment komplett neu ohne reledmac
\makeatletter
\renewenvironment{theindex}{%
  \section*{\indexname}%
  \setlength{\parindent}{0pt}%
  \setlength{\parskip}{0pt plus 0.3pt}%
  \let\item\@idxitem
}{%
  \clearpage
}
\makeatother

\IfFileExists{\jobname-pw.ind}{\input{\jobname-pw.ind}}{}

\end{document}

      