%% latex-leseansicht-vorspann.tex
%% Vorspann für die Leseansicht.
%% Lädt die gemeinsame Datei latex-vorspann.tex mit nicht gesetztem Schalter.

\newif\ifkorrekturansicht
\korrekturansichtfalse

\input{../tex-inputs/latex-vorspann}


         
         \renewcommand{\erwaehntePersonen}{Personen: Georg Brandes}
         \renewcommand{\erwaehnteInstitutionen}{Institutionen: Die Zeit. Wiener Wochenschrift}
         \renewcommand{\erwaehnteOrte}{Orte: Berlin, Dänemark, Frankgasse 1, III., Landstraße, IX., Alsergrund, Kopenhagen, Polen, Wien}
         \renewcommand{\erwaehnteWerke}{Werke: Censur in Polen, Eindrücke aus Russland, Polen}
               \section[Georg Brandes an Arthur Schnitzler, 6. 10. 1896]{ Georg Brandes an Arthur Schnitzler, 6. 10. 1896}\nopagebreak\mylabel{v}\rehead{ }\begin{ledgroupsized}[t]{13cm}\normalsize\beginnumbering \toendnotes[C]{\smallbreak\pagebreak[2]} \Standort{CUL, Schnitzler, B 17.}
\physDesc{Postkarte, 857 Zeichen
\newline{}Handschrift: blaue Tinte, lateinische Kurrent
\newline{}Versand: 1) Stempel: »\nobreak{}\oindex{Kopenhagen@\textbf{Kopenhagen}|pwk}Kjobenhavn, 6. 10.96, 5–5E\nobreak{}«.   2) Stempel: »\nobreak{}\oindex{III., Landstrasse@\textbf{III., Landstraße}|pwk}Wien 3/3, 8. 10.96, 8.V\nobreak{}«. 
\newline{}Ordnung: von unbekannter Hand nummeriert: »3« }\buchAbdrucke{\weitereDrucke{Georg Brandes, Arthur Schnitzler: \emph{Ein Briefwechsel}. Hg. Kurt Bergel. Bern: \emph{Francke} 1956, S. 58.} }\toendnotes[C]{\smallbreak}\pstart{}{\pb}Herrn Dr. Arthur
                  Schnitzler\pend{}\pstart{}Frankgasse 1\oindex{Frankgasse 1@\textbf{Frankgasse 1}|pw}\pend{}\pstart{}Wien IX\oindex{IX., Alsergrund@\textbf{IX., Alsergrund}|pw}\pend{}{\bigskip}\pstart
           \raggedleft{}{\pb}Kopenhagen\oindex{Kopenhagen@\textbf{Kopenhagen}|pw}{ }6 Oct\pend
           \pstart
           Lieber Herr Schnitzler! Könnten Sie mir nicht ein Bischen zu Hülfe
               kommen. Mir wird ein Numero der \uline{Zeit}\orgindex{Zeit. Wiener Wochenschrift@Die Zeit. Wiener Wochenschrift|pw} geschickt, worin als von mir eingesandt ein Bruchstück\pwindex{Brandes, Georg 04.02.1842 – 19.02.1927@\textsc{Brandes, Georg} (04.02.1842 – 19.02.1927)!Censur in Polen03. 10. 1896@\strich\emph{Censur in Polen} {[}03. 10. 1896{]}|pwv} meines alten Buches\pwindex{Brandes, Georg 04.02.1842 – 19.02.1927@\textsc{Brandes, Georg} (04.02.1842 – 19.02.1927)!Polen1888@\strich\emph{Polen} {[}1888{]}|pwv} über Polen\oindex{Polen@\textbf{Polen}|pw} sich
               findet. Es ist vor \uline{10} Jahren herausgegeben, und die
               Zeitangaben passen darauf; nun steht es da als von \uline{heute} stammend. Wenn ich doch wenigstens eine Correctur dieser Sachen sähe!
               Es wimmelt von Missverständnissen. Die Fehler sind derart dass das dänische\oindex{Daenemark@\textbf{Dänemark}|pw} Wort \uline{Rædsel} (horror, horreur, Schrecken) übersetzt ist \uline{Räthsel}.\hspace*{5em} Ich erfahre, dass kürzlich in
                  Berlin\oindex{Berlin@\textbf{Berlin}|pw} ein Buch mit meinem Namen versehen
               erschienen ist \uline{Aus dem Reiche des Absolutismus}\pwindex{Brandes, Georg 04.02.1842 – 19.02.1927@\textsc{Brandes, Georg} (04.02.1842 – 19.02.1927)!Eindruecke aus Russland1888@\strich\emph{Eindrücke aus Russland} {[}1888{]}|pw} (!) Welcher Titel. Es sind wohl meine »Eindrücke aus Rusland\pwindex{Brandes, Georg 04.02.1842 – 19.02.1927@\textsc{Brandes, Georg} (04.02.1842 – 19.02.1927)!Eindruecke aus Russland1888@\strich\emph{Eindrücke aus Russland} {[}1888{]}|pw}«. Es ist mir nicht geschickt worden. \introOben{}Es ist der 9\textsuperscript{te} nicht autorisirte Band
                  von mir in Einem Jahre.\introOben{}\pend
           \pstart Ihr ergebener \spacefill\mbox{Georg Brandes}\pend{}
         
         \endnumbering\mylabel{h}\end{ledgroupsized}  \newcommand{\dateiname}{L00600}\newcommand{\titel}{Georg Brandes an Arthur Schnitzler, 6. 10. 1896}\newcommand{\editorInnen}{Martin Anton Müller und Gerd-Hermann Susen}%% latex-leseansicht-abspann.tex
%% Abspann für die Leseansicht.
%% Der Schalter \ifkorrekturansicht ist bereits durch den Vorspann gesetzt.

%% latex-abspann.tex
%% Gemeinsamer Abspann für Korrekturansicht und Leseansicht.
%% Setzt den Schalter \ifkorrekturansicht voraus (gesetzt in den
%% einbindenden Dateien latex-korrekturansicht-abspann.tex bzw.
%% latex-leseansicht-abspann.tex).
%% ---------------------------------------------------------------

\normalsize

% Das esempio-Environment wird nur in der Leseansicht benötigt
\ifkorrekturansicht\else
\newenvironment{esempio}[3]%
{
    \vspace{1.5ex}
    \rlap{\underline{#1}}
    \par
    \setlength{\parindent}{0cm}
    \nopagebreak
    \leftskip=#2cm
    \rightskip=#3cm
}
{
    \par
}
\fi

\doendnotes{C}
\bigskip
\vfill

\clearpage

\footnotesize

\ifkorrekturansicht
  \lohead{\textsc{register}}
\fi

% theindex-Environment neu definieren ohne reledmac
\makeatletter
\renewenvironment{theindex}{%
  \ifkorrekturansicht
    \section*{\indexname}%
  \else
    \subsubsection*{Index der erwähnten Entitäten}%
  \fi
  \setlength{\parindent}{0pt}%
  \setlength{\parskip}{0pt plus 0.3pt}%
  \let\item\@idxitem
}{%
  \ifkorrekturansicht\clearpage\fi
}
\makeatother

\IfFileExists{\jobname-pw.ind}{\input{\jobname-pw.ind}}{}

% Quellenangabe nur in der Leseansicht
\ifkorrekturansicht\else
% Fallback-Definitionen, falls die .tex-Datei \titel etc. nicht gesetzt hat
\providecommand{\titel}{}
\providecommand{\editorInnen}{}
\providecommand{\dateiname}{\jobname}

\vspace{3cm}

\vfill

\footnotesize
\textsc{Quelle}: \titel. Herausgegeben von {\editorInnen}. In: \emph{Arthur Schnitzler: Briefwechsel mit Autorinnen und Autoren}.
 Digitale Edition, https://schnitzler-briefe.acdh.oeaw.ac.at/{\dateiname}.html (Stand \today)
\fi

\end{document}


      