%% latex-korrekturansicht-vorspann.tex
%% Vorspann für die Korrekturansicht.
%% Lädt die gemeinsame Datei latex-vorspann.tex mit gesetztem Schalter.

\newif\ifkorrekturansicht
\korrekturansichttrue

\input{../tex-inputs/latex-vorspann}


\section[Jaques Joachim und Eduard Michael Kafka an Arthur Schnitzler, 6.\,7.\,1891]{L00022 Jaques Joachim und Eduard Michael Kafka an Arthur Schnitzler,6.\,7.\,1891}
\nopagebreak\mylabel{L00022v}
\rehead{ }\normalsize\beginnumbering\briefempfaengerindex{Schnitzler, Arthur@\textsc{Schnitzler, Arthur}!zzzKafka, Eduard Michael@\emph{von Eduard Michael Kafka}!1891-07-061@{6.\,7.\,1891}|(be}\briefempfaengerindex{Schnitzler, Arthur@\textsc{Schnitzler, Arthur}!zzzJoachim, Jaques@\emph{von Jaques Joachim}!1891-07-061@{6.\,7.\,1891}|(be}
\toendnotes[C]{\smallbreak\pagebreak[2]}\Standort{DLA, A:Schnitzler, HS.NZ85.1.3571,2.}
\physDesc{Postkarte, 459 Zeichen
\newline{}Handschrift Jaques Joachim: schwarze Tinte, deutsche Kurrent
\newline{}Handschrift Eduard Michael Kafka: schwarze Tinte, deutsche Kurrent
\newline{}Versand: 1) Stempel: »\nobreak{}\oindex{Wipplingerstrasse@\textbf{Wipplingerstraße}|pwk}Wipplingerstrasse Wien, 6 7 91, 5–6 A\nobreak{}«.   2) Stempel: »\nobreak{}Wien 1/1, 7./7. 91, 8–9½ V., Bestellt\nobreak{}«. }\toendnotes[C]{\smallbreak}\pstart{}{\pb}\textsc{Herrn D \textsuperscript{r}Arthur Schnitzler }\pend{}\pstart{}I. Giselastr 11\oindex{Ordination Arthur Schnitzler [Boesendorferstrasse 11]@\textbf{Ordination Arthur Schnitzler [Bösendorferstraße 11]}|pw}. \pend{}{\bigskip}\vspace{1em}
\pstart
           \centering{}{\pb}\textsc{Wien\oindex{Wien@\textbf{Wien}|pw}} am 6. Juli \textcolor{gray}{1891}. \pend
           
\pstart\center{}Euer Wolgeboren\pend\vspace{0.5em}
\pstart
           werden hiemit höflichſt eingeladen, – falls Sie dem \pend
           
\pstart
           \centering{}\textsc{Verein für moderne Literatur (»Wiener Freie Bühne«)}\orgindex{»Freie Buehne« Verein fuer moderne Literatur@»Freie Bühne« Verein für moderne Literatur|pw}\pend
           
\pstart
           als Mitglied beizutreten beabſichtigen –, an der Dienſtag, den 7. Juli d. J.im \textsc{Souterrainlocale} des \textsc{Hotel de France}\oindex{Hotel de France@\textbf{Hotel de France}|pw} (Eingang: Maria-Theresienſtraße\oindex{Maria-Theresienstrasse [Wien]@\textbf{Maria-Theresienstraße [Wien]}|pw}){ }ſtattfindenden \pend
           
\pstart
           \centering{}\textsc{Constituirenden Versa{\geminationm}lung}\eventindex{Hotel de France@\textbf{Hotel de France}!Gruendungsveranstaltung Freie Buehne, Verein fuer moderne Literatur, 7.7.1891@Gründungsveranstaltung Freie Bühne, Verein für moderne Literatur, 7.7.1891|pw}\pend
           
\pstart
           theilnehmen zu wollen.\pend
           \pstart \spacefill\mbox{D\textsuperscript{r}Joachim}\pend{}\selectlanguage{ngerman}\vspace{1em}
\pstart
           \noindent{}\centering{}{[}hs. :{]} 7 ½ Uhr Abends\pend
           \pstart \spacefill\mbox{E. M \textcolor{gray}{.}Kafka}\pend{}
\pstart
           \noindent{}\label{T_L00022-1v}\edtext{Dieſe Einladung gilt zugleich als
                     \textsc{Legitimation}.}{\lemma{\textnormal{\emph{Diese … Legitimation.}}}\Cendnote{\textnormal{quer am linken Rand}}}\label{T_L00022-1}\pend
           \selectlanguage{ngerman}\endnumbering\briefempfaengerindex{Schnitzler, Arthur@\textsc{Schnitzler, Arthur}!zzzKafka, Eduard Michael@\emph{von Eduard Michael Kafka}!1891-07-061@{6.\,7.\,1891}|)be}\briefempfaengerindex{Schnitzler, Arthur@\textsc{Schnitzler, Arthur}!zzzJoachim, Jaques@\emph{von Jaques Joachim}!1891-07-061@{6.\,7.\,1891}|)be}\mylabel{L00022h}  \normalsize

\doendnotes{C}
\bigskip
\vfill

\clearpage

\footnotesize

\lohead{\textsc{register}}

% Definiere theindex-Environment komplett neu ohne reledmac
\makeatletter
\renewenvironment{theindex}{%
  \section*{\indexname}%
  \setlength{\parindent}{0pt}%
  \setlength{\parskip}{0pt plus 0.3pt}%
  \let\item\@idxitem
}{%
  \clearpage
}
\makeatother

\IfFileExists{\jobname-pw.ind}{\input{\jobname-pw.ind}}{}

\end{document}

      