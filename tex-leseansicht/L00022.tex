%% latex-leseansicht-vorspann.tex
%% Vorspann für die Leseansicht.
%% Lädt die gemeinsame Datei latex-vorspann.tex mit nicht gesetztem Schalter.

\newif\ifkorrekturansicht
\korrekturansichtfalse

\input{../tex-inputs/latex-vorspann}


         \renewcommand{\erwaehnteInstitutionen}{Institutionen: »Freie Bühne« Verein für moderne Literatur}
         \renewcommand{\erwaehnteOrte}{Orte: Hotel de France, Maria-Theresienstraße, Ordination Dr. Arthur Schnitzler Giselastraße 11, Wien, Wipplingerstraße}
         \renewcommand{\erwaehnteWerke}{}
               \section[Jaques Joachim und Eduard Michael Kafka an Arthur Schnitzler, 6. 7. 1891]{ Jaques Joachim und Eduard Michael Kafka an Arthur Schnitzler, 6. 7. 1891}\nopagebreak\mylabel{v}\rehead{ }\begin{ledgroupsized}[t]{13cm}\normalsize\beginnumbering \toendnotes[C]{\smallbreak\pagebreak[2]} \Standort{DLA, A:Schnitzler, HS.NZ85.1.3571,2.}
\physDesc{Postkarte, 459 Zeichen
\newline{}Handschrift Jaques Joachim: schwarze Tinte, deutsche Kurrent\newline{}Handschrift Eduard Michael Kafka: schwarze Tinte, deutsche Kurrent
\newline{}Versand: 1) Stempel: »\nobreak{}\oindex{Wipplingerstrasse@\textbf{Wipplingerstraße}|pwk}Wipplingerstrasse Wien, 6 7 91, 5–6 A\nobreak{}«.   2) Stempel: »\nobreak{}Wien 1/1, 7./7. 91, 8–9½ V., Bestellt\nobreak{}«. }\toendnotes[C]{\smallbreak}\pstart{}{\pb}\textsc{Herrn D \textsuperscript{r}Arthur Schnitzler }\pend{}\pstart{}I. Giselastr 11\oindex{Ordination Dr. Arthur Schnitzler Giselastrasse 11@\textbf{Ordination Dr. Arthur Schnitzler Giselastraße 11}|pw}. \pend{}{\bigskip}\pstart
           \centering{}{\pb}\textsc{Wien\oindex{Wien@\textbf{Wien}|pw}} am 6. Juli \textcolor{gray}{1891}. \pend
           \pstart\center{}Euer Wolgeboren\pend\pstart
           werden hiemit höflichſt eingeladen, – falls Sie dem \pend
           \pstart
           \centering{}\textsc{Verein für moderne Literatur (»Wiener Freie Bühne«)}\orgindex{»Freie Buehne« Verein fuer moderne Literatur@»Freie Bühne« Verein für moderne Literatur|pw}\pend
           \pstart
           \noindent{}als Mitglied beizutreten beabſichtigen –, an der Dienſtag, den 7.
                  Juli d. J.im \textsc{Souterrainlocale} des \textsc{Hotel de France}\oindex{Hotel de France@\textbf{Hotel de France}|pw} (Eingang: Maria-Theresienſtraße\oindex{Maria-Theresienstrasse@\textbf{Maria-Theresienstraße}|pw}){ }ſtattfindenden \pend
           \pstart
           \centering{}\textsc{Constituirenden Versa{\geminationm}lung }\pend
           \pstart
           \noindent{}theilnehmen zu wollen.\pend
           \pstart \spacefill\mbox{D \textsuperscript{r}Joachim}\pend{}\pstart
           \noindent{}\centering{}{[}hs. Kafka:{]} 7 ½ Uhr Abends\pend
           \pstart \spacefill\mbox{E. M \textcolor{gray}{.}Kafka}\pend{}\pstart
           \noindent{}\label{T_L00022-1v}\edtext{Dieſe Einladung gilt zugleich als
                     \textsc{Legitimation}. }{\lemma{\textnormal{\emph{Dieſe … Legitimation.}}}\Cendnote{\textnormal{quer am linken Rand}}}\label{T_L00022-1h}\pend
           
         
         \endnumbering\mylabel{h}\end{ledgroupsized}  \newcommand{\dateiname}{L00022}\newcommand{\titel}{Jaques Joachim und Eduard Michael Kafka an Arthur Schnitzler, 6. 7. 1891}\newcommand{\editorInnen}{Martin Anton Müller und Gerd-Hermann Susen}%% latex-leseansicht-abspann.tex
%% Abspann für die Leseansicht.
%% Der Schalter \ifkorrekturansicht ist bereits durch den Vorspann gesetzt.

%% latex-abspann.tex
%% Gemeinsamer Abspann für Korrekturansicht und Leseansicht.
%% Setzt den Schalter \ifkorrekturansicht voraus (gesetzt in den
%% einbindenden Dateien latex-korrekturansicht-abspann.tex bzw.
%% latex-leseansicht-abspann.tex).
%% ---------------------------------------------------------------

\normalsize

% Das esempio-Environment wird nur in der Leseansicht benötigt
\ifkorrekturansicht\else
\newenvironment{esempio}[3]%
{
    \vspace{1.5ex}
    \rlap{\underline{#1}}
    \par
    \setlength{\parindent}{0cm}
    \nopagebreak
    \leftskip=#2cm
    \rightskip=#3cm
}
{
    \par
}
\fi

\doendnotes{C}
\bigskip
\vfill

\clearpage

\footnotesize

\ifkorrekturansicht
  \lohead{\textsc{register}}
\fi

% theindex-Environment neu definieren ohne reledmac
\makeatletter
\renewenvironment{theindex}{%
  \ifkorrekturansicht
    \section*{\indexname}%
  \else
    \subsubsection*{Index der erwähnten Entitäten}%
  \fi
  \setlength{\parindent}{0pt}%
  \setlength{\parskip}{0pt plus 0.3pt}%
  \let\item\@idxitem
}{%
  \ifkorrekturansicht\clearpage\fi
}
\makeatother

\IfFileExists{\jobname-pw.ind}{\input{\jobname-pw.ind}}{}

% Quellenangabe nur in der Leseansicht
\ifkorrekturansicht\else
% Fallback-Definitionen, falls die .tex-Datei \titel etc. nicht gesetzt hat
\providecommand{\titel}{}
\providecommand{\editorInnen}{}
\providecommand{\dateiname}{\jobname}

\vspace{3cm}

\vfill

\footnotesize
\textsc{Quelle}: \titel. Herausgegeben von {\editorInnen}. In: \emph{Arthur Schnitzler: Briefwechsel mit Autorinnen und Autoren}.
 Digitale Edition, https://schnitzler-briefe.acdh.oeaw.ac.at/{\dateiname}.html (Stand \today)
\fi

\end{document}


      