\input{../tex-inputs/latex-pdf-vorspann}
\begin{center}
            \textcolor{red}{ENTWURF. ENTZIFFERUNG NOCH NICHT KORREKTURGELESEN}
                      \end{center}
            
               \section[Jaques Joachim und Eduard Michael Kafka an Arthur Schnitzler, 6. 7. 1891]{ Jaques Joachim und Eduard Michael Kafka an Arthur Schnitzler, 6. 7. 1891}\nopagebreak\mylabel{v}\rehead{ }\begin{ledgroupsized}[t]{13cm}\normalsize\beginnumbering\briefempfaengerindex{Schnitzler, Arthur@\textsc{Schnitzler, Arthur}!zzzKafka, Eduard Michael@\emph{von Eduard Michael Kafka}!1891-07-061@{6. 7. 1891}|(be}\briefempfaengerindex{Schnitzler, Arthur@\textsc{Schnitzler, Arthur}!zzzJoachim, Jaques@\emph{von Jaques Joachim}!1891-07-061@{6. 7. 1891}|(be} \toendnotes[C]{\smallbreak\pagebreak[2]} \Standort{DLA, A:Schnitzler, HS.NZ85.1.3571,2.}
\physDesc{Postkarte
\newline{}Handschrift Jaques Joachim: schwarze Tinte, deutsche Kurrent\newline{}Handschrift Eduard Michael Kafka: schwarze Tinte, deutsche Kurrent\newline{}Versand: 1) Stempel: »\nobreak{}\oindex{Wipplingerstrasse@\textbf{Wipplingerstraße}|pwk}Wipplingerstrasse Wien, 6 7 91, 5–6 A\nobreak{}«.  2) Stempel: »\nobreak{}Wien 1/1, 7./7. 91, 8–9½ V., Bestellt\nobreak{}«. }\toendnotes[C]{\smallbreak}\pstart{}{\pb}\textsc{Herrn D \textsuperscript{r}Arthur Schnitzler }\pend{}\pstart{}I. Giselastr 11\oindex{Boesendorferstrasse@\textbf{Bösendorferstraße}|pw}. \pend{}{\bigskip}\pstart
           \centering{}{\pb}\textsc{Wien\oindex{Wien@\textbf{Wien}|pw}}am 6. Juli
                        \textcolor{gray}{1891}. \pend
           \pstart\center{}Euer Wolgeboren\pend\pstart
           werden hiemit höflichſt eingeladen, – falls Sie dem \pend
           \pstart
           \centering{}\textsc{Verein für moderne
                     Literatur (»Wiener Freie Bühne«) }\orgindex{»Freie Buehne« Verein fuer moderne Literatur@»Freie Bühne« Verein für moderne Literatur|pw}\pend
           \pstart
           \noindent{}als Mitglied beizutreten beabſichtigen –, an der
                  Dienſtag , den 7. Juli d.
                  J.im \textsc{Souterrainlocale}des \textsc{Hotel de France}\oindex{Hotel de France@\textbf{Hotel de France}|pw}(Eingang: Maria-Theresienſtraße
               \oindex{Maria-Theresienstrasse@\textbf{Maria-Theresienstraße}|pw}) { }ſtattfindenden \pend
           \pstart
           \centering{}\textsc{Constituirenden Versa {\geminationm}lung }\pend
           \pstart
           \noindent{}theilnehmen zu wollen.\pend
           \pstart \spacefill\mbox{D \textsuperscript{r}Joachim }\pend{}\pstart
           \noindent{}\centering{}{[}hs. Kafka:{]} 7 ½ Uhr Abends\pend
           \pstart \spacefill\mbox{E. M \textcolor{gray}{.}Kafka }\pend{}\pstart
           \noindent{}\label{T_L00022_1v}\edtext{Dieſe
                  Einladung gilt zugleich als \textsc{Legitimation}. }{\lemma{\textnormal{\emph{Dieſe … Legitimation.}}}\Cendnote{\textnormal{quer am linken Rand}}}\label{T_L00022_1h}\pend
           \endnumbering\briefempfaengerindex{Schnitzler, Arthur@\textsc{Schnitzler, Arthur}!zzzKafka, Eduard Michael@\emph{von Eduard Michael Kafka}!1891-07-061@{6. 7. 1891}|)be}\briefempfaengerindex{Schnitzler, Arthur@\textsc{Schnitzler, Arthur}!zzzJoachim, Jaques@\emph{von Jaques Joachim}!1891-07-061@{6. 7. 1891}|)be}\mylabel{h}\end{ledgroupsized}  \newcommand{\dateiname}{L00022}\newcommand{\titel}{Jaques Joachim und Eduard Michael Kafka an Arthur Schnitzler, 6. 7. 1891}\newcommand{\editorInnen}{Martin
                  Anton Müller und Gerd-Hermann Susen}\input{../tex-inputs/latex-pdf-abspann}
      