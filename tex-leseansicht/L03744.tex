%% latex-korrekturansicht-vorspann.tex
%% Vorspann für die Korrekturansicht.
%% Lädt die gemeinsame Datei latex-vorspann.tex mit gesetztem Schalter.

\newif\ifkorrekturansicht
\korrekturansichttrue

\input{../tex-inputs/latex-vorspann}


\section[Arthur Schnitzler an Stefan Zweig, 28. 5. 1927]{L03744 Arthur Schnitzler an Stefan Zweig, 28. 5. 1927}
\nopagebreak\mylabel{L03744v}
\rehead{ }\normalsize\beginnumbering\briefempfaengerindex{Zweig, Stefan@\textsc{Zweig, Stefan}!zzzSchnitzler, Arthur@\emph{von Arthur Schnitzler}!1927-05-282@{28. 5. 1927}|(be}
\toendnotes[C]{\smallbreak\pagebreak[2]}\Standort{Jerusalem, National Library of Israel, ARC. Ms. Var. 305 1 58 Stefan Zweig Collection.}
\physDesc{Brief, 1 Blatt, 2 Seiten, 1258 Zeichen
\newline{}Handschrift: schwarze Tinte, lateinische Kurrent}\toendnotes[C]{\smallbreak}
\pstart
           {\pb}Wien\oindex{Wien@\textbf{Wien}, \emph{A.ADM2}|pw},
                        28. 5. 927\pend
           \vspace{0.5em}
\pstart
           lieber Stefan Zweig, daſs und wie Sie mir bei jeder Gelegenheit Ihre
               Sympathie und Ihre Antheilnahme kundgeben – anläßlich Älterwerdens,
               Novellenschreibens und Nichtaufgeführtwerdens, rührt mich geradezu und so hab ich
               Ihnen auch für Ihren letzten lieben \label{K_L03744-1v}\edtext{Brief}{\lemma{\textnormal{\emph{Brief}}}\Cendnote{\textnormal{Stefan Zweig an Arthur Schnitzler, 18. 5. 1927.
               }}}\label{K_L03744-1} wärmstens zu danken.\pend
           
\pstart
           Mit Ihrem Bedenken gegen die Höhe des Betrags haben Sie wahrschreinlich recht, wie im
               Fall Else\pwindex{Fraeulein Else@\emph{Fräulein Else}|pw}; nach der Aufführung des »Gangs\pwindex{Gang zum Weiher. Dramatische Dichtung@\emph{Der Gang zum Weiher. Dramatische Dichtung}|pw}« sehn ich mich, unter den gegenwärtigen
               Umständen, selbst nicht sonderlich; – und \strikeout{daſs} das
               Alter – um nicht zu sagen Altwerden ist (wie die Sandrock\pwindex{Sandrock, Adele 1863-08-19 – 1937-08-30@\textsc{Sandrock, Adele} (1863-08-19 – 1937-08-30), \emph{Schauspieler/Schauspielerin}|pw} einmal vom Tod behauptet hat) ein Element gegen das sich nichts
               sagen läßt. {\pb}Pathetisch oder resignirt genommen – unsere
               Erwiderung bleibt immer nur »\label{K_L03744-2v}\edtext{Allons travailler}{\lemma{\textnormal{\emph{Allons travailler}}}\Cendnote{\textnormal{französisch: machen wir uns an die
                     Arbeit. Es handelt sich um die letzten Worte von \emph{L’Œuvre}\pwindex{œuvre@\emph{L’œuvre}|pwk} (1886) von Émile Zola\pwindex{Zola, Emile 02.04.1840 – 29.09.1902@\textsc{Zola, Émile} (02.04.1840 – 29.09.1902), \emph{Schriftsteller/Schriftstellerin, Journalist/Journalistin}|pwk}.}}}\label{K_L03744-2}\pwindex{œuvre@\emph{L’œuvre}|pwv}« (wer\pwindex{Zola, Emile 02.04.1840 – 29.09.1902@\textsc{Zola, Émile} (02.04.1840 – 29.09.1902), \emph{Schriftsteller/Schriftstellerin, Journalist/Journalistin}|pwv} hat es nur gesagt?)\pend
           
\pstart
           Ich bleibe vorläufig in Wien\oindex{Wien@\textbf{Wien}, \emph{A.ADM2}|pw} (we{\geminationn} nicht das
               Wetter zu ausgedehnten Ausflügen locken sollte) vor dem So{\geminationm}er noch, Sie haben es
               wohl gelesen, \label{K_L03744-11v}\edtext{heiratet meine Tochter\pwindex{Cappellini, Lili 13.09.1909 – 26.07.1928@\textsc{Cappellini, Lili} (13.09.1909 – 26.07.1928)|pwv}}{\lemma{\textnormal{\emph{heiratet meine Tochter}}}\Cendnote{\textnormal{Vgl. A. S.: \emph{Kulturveranstaltungen}, 30. 6. 1927.}}}\label{K_L03744-11} nach Venedig\oindex{Venedig@\textbf{Venedig}, \emph{P.PPLA}|pw} (die Wohnung\oindex{Campo Sant Agostin 2545@\textbf{Campo Sant’Agostin 2545}, \emph{Wohngebäude (K.WHS)}|pwv} dort, in Fari\oindex{Santa Maria Gloriosa dei Frari@\textbf{Santa Maria Gloriosa dei Frari}, \emph{S.CH}|pw}-Nähe steht schon bereit) die Eintheilung
               meiner »Ferien« (die oft meine beste Arbeitszeit sind) wird dazu ein wenig abhängen.
               Noch steht mein Progra{\geminationm} nicht fest – in jedem Fall hoff ich wir begegnen einander
               bald wieder – es ist mir immer eine Freude wie Sie wissen.\pend
           
\pstart
           Herzlichst grüßt Sie Ihr{\\[\baselineskip]}\spacefill\mbox{ArthurSchnitzler}\pend
           \leftskip=0em{}\selectlanguage{ngerman}\endnumbering\briefempfaengerindex{Zweig, Stefan@\textsc{Zweig, Stefan}!zzzSchnitzler, Arthur@\emph{von Arthur Schnitzler}!1927-05-282@{28. 5. 1927}|)be}\mylabel{L03744h}
\begin{anhang}
\end{anhang}\normalsize

\doendnotes{C}
\bigskip
\vfill

\clearpage

\footnotesize

\lohead{\textsc{register}}

% Definiere theindex-Environment komplett neu ohne reledmac
\makeatletter
\renewenvironment{theindex}{%
  \section*{\indexname}%
  \setlength{\parindent}{0pt}%
  \setlength{\parskip}{0pt plus 0.3pt}%
  \let\item\@idxitem
}{%
  \clearpage
}
\makeatother

\IfFileExists{\jobname-pw.ind}{\input{\jobname-pw.ind}}{}

\end{document}

      