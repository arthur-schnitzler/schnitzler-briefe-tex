%% latex-leseansicht-vorspann.tex
%% Vorspann für die Leseansicht.
%% Lädt die gemeinsame Datei latex-vorspann.tex mit nicht gesetztem Schalter.

\newif\ifkorrekturansicht
\korrekturansichtfalse

\input{../tex-inputs/latex-vorspann}


               \section[Olga Schnitzler an Paula Beer-Hofmann, {[}23. 10. 1908?{]}]{ Olga Schnitzler an Paula Beer-Hofmann, {[}23. 10. 1908?{]}}\nopagebreak\mylabel{v}\rehead{ }\begin{ledgroupsized}[t]{13cm}\normalsize\beginnumbering\briefempfaengerindex{Beer-Hofmann, Paula@\textsc{Beer-Hofmann, Paula}!zzzSchnitzler, Olga@\emph{von Olga Schnitzler}!1908-10-231@{{[}23. 10. 1908?{]}}|(be} \toendnotes[C]{\smallbreak\pagebreak[2]} \Standort{YCGL, MSS 31.}
\physDesc{Brief, 1 Blatt, 2 Seiten (die zweite Seite über den Mittelfalz geschrieben), Umschlag, mit rotem Wachssiegel verschlossen
\newline{}Handschrift: schwarze Tinte, lateinische Kurrent\newline{}Versand: ohne postalischen Übermittlungsvermerk }\toendnotes[C]{\smallbreak}\pstart{}{\pb}Herrn D\textsuperscript{r} Richard
                  Beer-Hofmann\pend{}{\bigskip}\pstart
           \noindent{}{\pb}\textcolor{gray}{\textbf{O. S.}}\pend
           \pstart
           Liebe Paula, eine grosse Bitte! ich glaube Sie haben mehrere
               Pelzjacken, ich soll morgen bis Sonntag auf den Semmering\oindex{Semmering@\textbf{Semmering}|pw} – Brahm\pwindex{Brahm, Otto 05.02.1856 – 28.11.1912@\textsc{Brahm, Otto} (05.02.1856 – 28.11.1912), \emph{Theaterleiter, Regisseur}|pw} ist oben –
               mein Schneider\pwindex{?? [Schneider von Olga Schnitzler] 1908 – 1908@\textsc{?? [Schneider von Olga Schnitzler]} (1908 – 1908)|pwv} hat meine
               Pelzjacke nicht fertig, würden Sie mir eine der Ihren auf \label{K_L01795_1v}\edtext{2 Tage}{\lemma{\textnormal{\emph{2 Tage}}}\Cendnote{\textnormal{Ein solcher
                  Kurzaufenthalt lässt sich nicht nachweisen. Mutmaßlich war er für den Aufenthalt
                     Brahm\pwindex{Brahm, Otto 05.02.1856 – 28.11.1912@\textsc{Brahm, Otto} (05.02.1856 – 28.11.1912), \emph{Theaterleiter, Regisseur}|pwk}s vom 22. 10. 1908 bis zum
                     27. 10. 1908 geplant? Eine alternative Datierung wäre der 9. 11. 1906, wenngleich es
                  damit das erste überlieferte Dokument nachbarschaftlicher Korrespondenz direkt
                  nach dem Einzug wäre.}}}\label{K_L01795_1h} leihen? nur wenn es Ihnen gar keine Umstände
               verursacht.\pend
           \pstart
           Seien Sie nicht bös, lassen Sie {\pb}von sich hören und
               seien Sie alle herzlich gegrüsst von Ihrer\pend
           \pstart \spacefill\mbox{Olga.}\pend{}\pstart
           Freitag.\pend
           \endnumbering\briefempfaengerindex{Beer-Hofmann, Paula@\textsc{Beer-Hofmann, Paula}!zzzSchnitzler, Olga@\emph{von Olga Schnitzler}!1908-10-231@{{[}23. 10. 1908?{]}}|)be}\mylabel{h}\end{ledgroupsized}  \newcommand{\dateiname}{L01795}\newcommand{\titel}{Olga Schnitzler an Paula Beer-Hofmann, [23. 10. 1908?]}\newcommand{\editorInnen}{Martin Anton Müller und Gerd-Hermann Susen}
            \footnotesize
\begin{ledgroupsized}[t]{11.5cm}
\doendnotes{C}
\end{ledgroupsized}
         %% latex-leseansicht-abspann.tex
%% Abspann für die Leseansicht.
%% Der Schalter \ifkorrekturansicht ist bereits durch den Vorspann gesetzt.

%% latex-abspann.tex
%% Gemeinsamer Abspann für Korrekturansicht und Leseansicht.
%% Setzt den Schalter \ifkorrekturansicht voraus (gesetzt in den
%% einbindenden Dateien latex-korrekturansicht-abspann.tex bzw.
%% latex-leseansicht-abspann.tex).
%% ---------------------------------------------------------------

\normalsize

% Das esempio-Environment wird nur in der Leseansicht benötigt
\ifkorrekturansicht\else
\newenvironment{esempio}[3]%
{
    \vspace{1.5ex}
    \rlap{\underline{#1}}
    \par
    \setlength{\parindent}{0cm}
    \nopagebreak
    \leftskip=#2cm
    \rightskip=#3cm
}
{
    \par
}
\fi

\doendnotes{C}
\bigskip
\vfill

\clearpage

\footnotesize

\ifkorrekturansicht
  \lohead{\textsc{register}}
\fi

% theindex-Environment neu definieren ohne reledmac
\makeatletter
\renewenvironment{theindex}{%
  \ifkorrekturansicht
    \section*{\indexname}%
  \else
    \subsubsection*{Index der erwähnten Entitäten}%
  \fi
  \setlength{\parindent}{0pt}%
  \setlength{\parskip}{0pt plus 0.3pt}%
  \let\item\@idxitem
}{%
  \ifkorrekturansicht\clearpage\fi
}
\makeatother

\IfFileExists{\jobname-pw.ind}{\input{\jobname-pw.ind}}{}

% Quellenangabe nur in der Leseansicht
\ifkorrekturansicht\else
% Fallback-Definitionen, falls die .tex-Datei \titel etc. nicht gesetzt hat
\providecommand{\titel}{}
\providecommand{\editorInnen}{}
\providecommand{\dateiname}{\jobname}

\vspace{3cm}

\vfill

\footnotesize
\textsc{Quelle}: \titel. Herausgegeben von {\editorInnen}. In: \emph{Arthur Schnitzler: Briefwechsel mit Autorinnen und Autoren}.
 Digitale Edition, https://schnitzler-briefe.acdh.oeaw.ac.at/{\dateiname}.html (Stand \today)
\fi

\end{document}


      