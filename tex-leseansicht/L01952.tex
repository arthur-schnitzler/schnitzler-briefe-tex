%% latex-leseansicht-vorspann.tex
%% Vorspann für die Leseansicht.
%% Lädt die gemeinsame Datei latex-vorspann.tex mit nicht gesetztem Schalter.

\newif\ifkorrekturansicht
\korrekturansichtfalse

\input{../tex-inputs/latex-vorspann}


         
         \renewcommand{\erwaehntePersonen}{Personen: Otto Brahm, Rose Friedmann, Louis Philipp Friedmann, Hugo von Hofmannsthal, Gertrude von Hofmannsthal, Clara Jonas, Josef Kainz, Olga Schnitzler, Heinrich Schnitzler}
         \renewcommand{\erwaehnteOrte}{Orte: Aspang-Markt, Bad Ischl, Edmund-Weiß-Gasse, Glion, Kirchberg am Wechsel, Mönichkirchen, Otterthal, Rodaun, Semmering, Sonnwendstein, St. Gilgen, Sternwartestraße, Welsberg-Taisten, Wien, Österreich}
         \renewcommand{\erwaehnteWerke}{Werke: Der junge Medardus. Dramatische Historie in einem Vorspiel und fünf Aufzügen}
               \section[Arthur Schnitzler an Hugo von Hofmannsthal, 30. 7. 1910]{ Arthur Schnitzler an Hugo von Hofmannsthal, 30. 7. 1910}\nopagebreak\mylabel{v}\rehead{ }\begin{ledgroupsized}[t]{13cm}\normalsize\beginnumbering \toendnotes[C]{\smallbreak\pagebreak[2]} \Standort{FDH, Hs-30885,138.}
\physDesc{Brief, 1 Blatt, 4 Seiten
\newline{}Handschrift: schwarze Tinte, deutsche Kurrent}\buchAbdrucke{\weitereDrucke{Hugo von Hofmannsthal, Arthur Schnitzler: \emph{Briefwechsel}. Hg. Therese Nickl und Heinrich Schnitzler. Frankfurt am Main: \emph{S. Fischer} 1964, S. 252.} }\toendnotes[C]{\smallbreak}\pstart
           \noindent{}{\pb}\textcolor{gray}{\textbf{Dr. Arthur Schnitzler}}\hfill XVIII. \textsc{Sternwartestr}. 71.\oindex{Sternwartestrasse@\textbf{Sternwartestraße}|pw}\pend
           \pstart
           \textcolor{gray}{\textbf{\strikeout{Wien XVIII. Spoettelgasse 7.\oindex{Edmund-Weiss-Gasse@\textbf{Edmund-Weiß-Gasse}|pw}}}}\hfill 30. 7. 1910!\pend
           \pstart
           mein lieber Hugo, Sie ſehen: wir ſind ſchon \label{K_L01952_1v}\edtext{überſiedelt}{\lemma{\textnormal{\emph{überſiedelt}}}\Cendnote{\textnormal{siehe A. S.: \emph{Tagebuch}, 14. 7. 1910}}}\label{K_L01952_1h} – und das ſind auch ſchon wieder faſt drei Wochen her, natürlich gings recht
               allmälig, und auch jetzt ſind wir noch nicht in völliger Ordnung. Aber mein
               Arbeitszimmer iſt längſt ſo wohnlich, daſs es kaum einen rechten Grund gibt das
               Stückeſchreiben länger hinauszuſchieben. Übrigens war ich \label{K_L01952_2v}\edtext{zweimal fort}{\lemma{\textnormal{\emph{zweimal fort}}}\Cendnote{\textnormal{zuerst
                     vom 6. 7. 1910 bis zum 10. 7. 1910, dann vom 26. 7. 1910
                     bis zum 28. 7. 1910}}}\label{K_L01952_2h}, auf dem Se{\geminationm}ering\oindex{Semmering@\textbf{Semmering}|pw}, mit Olga\pwindex{Schnitzler, Olga 17.01.1882 – 13.01.1970@\textsc{Schnitzler, Olga} (17.01.1882 – 13.01.1970), \emph{Schauspielerin, Sängerin}|pw} u Heini\pwindex{Schnitzler, Heinrich 09.08.1902 – 12.07.1982@\textsc{Schnitzler, Heinrich} (09.08.1902 – 12.07.1982), \emph{Regisseur, Schauspieler}|pw}, knapp vor dem Umzug; und
               jetzt wieder ein paar Tage allein auf {\pb}dem Se{\geminationm}ering\oindex{Semmering@\textbf{Semmering}|pw}, viel mit Brahm\pwindex{Brahm, Otto 05.02.1856 – 28.11.1912@\textsc{Brahm, Otto} (05.02.1856 – 28.11.1912), \emph{Theaterleiter, Regisseur}|pw} zuſammen; mit Frau \textsc{Jonas}\pwindex{Jonas, Clara 29.6.1863 – 17.1.1922@\textsc{Jonas, Clara} (29.6.1863 – 17.1.1922)|pw}, mit Kainz\pwindex{Kainz, Josef 02.01.1858 – 20.09.1910@\textsc{Kainz, Josef} (02.01.1858 – 20.09.1910), \emph{Schauspieler}|pw} (der, we{\geminationn} alles gut geht, bald wieder eine neue Rolle von mir
               ſpielen dürfte.) Von Se{\geminationm}ering\oindex{Semmering@\textbf{Semmering}|pw} aus hab ich eine \label{K_L01952_3v}\edtext{Fußpartie}{\lemma{\textnormal{\emph{Fußpartie}}}\Cendnote{\textnormal{siehe A. S.: \emph{Tagebuch}, 28. 7. 1910}}}\label{K_L01952_3h} gemacht (denken Sie, mein Rad hab ich – verſchenkt{\dotstwo}), über den So{\geminationn}wendſtein\oindex{Sonnwendstein@\textbf{Sonnwendstein}|pw}, ins Otterthal\oindex{Otterthal@\textbf{Otterthal}|pw}, über Kirchberg\oindex{Kirchberg am Wechsel@\textbf{Kirchberg am Wechsel}|pw}, Aspang\oindex{Aspang-Markt@\textbf{Aspang-Markt}|pw} nach Mönichkirchen\oindex{Moenichkirchen@\textbf{Mönichkirchen}|pw} – etwas ganz
               beſonders ſchönes, von oeſterreichiſcher\oindex{Oesterreich@\textbf{Österreich}|pw}
               Unberühmtheit; ich hatte mich jahrelange geſehnt, es kennen zu lernen, ſo daſs es ein
               Witzwort unſres Hauſes, beſonders Heini\pwindex{Schnitzler, Heinrich 09.08.1902 – 12.07.1982@\textsc{Schnitzler, Heinrich} (09.08.1902 – 12.07.1982), \emph{Regisseur, Schauspieler}|pw}s zu
               werden anfing; – und als {\pb}ich es endlich, nach etwa
               zehnſtündiger Wanderung erreichte, – gab es kein Bett im ganzen Ort, ſo daſs ich
               gleich wieder hinunter fahren mußte – (was in jüngern Jahren gewiſs ſymboliſch
               empfunden worden wäre.)\pend
           \pstart
           Ich hoffe wir reiſen heuer doch noch einmal weg, gegen Ende Auguſt, –
                  \textsc{St. Gilgen}\oindex{St. Gilgen@\textbf{St. Gilgen}|pw} vielleicht, oder Iſchl\oindex{Bad Ischl@\textbf{Bad Ischl}|pw}, aber kaum auf lang, da
               die \textsc{Medardus}\pwindex{Schnitzler, Arthur 15.05.1862 – 21.10.1931@\textsc{Schnitzler, Arthur} (15.05.1862 – 21.10.1931), \emph{Schriftsteller, Mediziner}!junge Medardus. Dramatische Historie in einem Vorspiel und fuenf Aufzuegen1910-10-26@\strich\emph{Der junge Medardus. Dramatische Historie in einem Vorspiel und fünf Aufzügen} {[}1910-10-26{]}|pw} Proben ſehr früh beginnen dürften. \strikeout{\textcolor{gray}{Also}} Es wäre wirklich ſchön, wieder einmal ein paar So{\geminationm}ertage miteinander zu verleben; aber daſs man ſich in Wien\oindex{Wien@\textbf{Wien}|pw}{ }ſo ſelten, ja nahezu ſchon gar nicht ſieht, iſt
               wahrhaftig nicht {\pb}meine Schuld allein. Erſtens reiſen Sie
               viel zu viel – und we{\geminationn} Sie von Rodaun\oindex{Rodaun@\textbf{Rodaun}|pw} nach Wien\oindex{Wien@\textbf{Wien}|pw} ko{\geminationm}en, erfährt man es doch meiſtens nur ganz zufällig oder
               gar nicht. Entſchließen Sie ſich doch wieder öfter telegrafiſch oder ſonſtwie ſich
               anzuſagen oder anzufragen – da{\geminationn} ſollen Sie mich ke{\geminationn}en lernen! Eine hiſtoriſche Berichtigung: \textsc{Welsberg}\oindex{Welsberg-Taisten@\textbf{Welsberg-Taisten}|pw} ist nicht \substVorne{}\textsuperscript{3}\substDazwischen{}4\substHinten{}, sondern 3 Jahre her – auch lang genug! Haben Sie meine \label{K_L01952_4v}\edtext{Karte aus Glion\oindex{Glion@\textbf{Glion}|pw}}{\lemma{\textnormal{\emph{Karte aus Glion}}}\Cendnote{\textnormal{vgl. A. S.: \emph{Tagebuch}, 28. 5. 1910}}}\label{K_L01952_4h} beko{\geminationm}en – was \label{K_L01952_5v}\edtext{12 Jahre her}{\lemma{\textnormal{\emph{12 Jahre her}}}\Cendnote{\textnormal{siehe A. S.: \emph{Tagebuch}, 14. 8. 1898}}}\label{K_L01952_5h} iſt! – Man ka{\geminationn} den Feuilletoniſten nicht Unrecht
               geben: die Zeit verrinnt{\dots}\pend
           \pstart
           Schönen Dank für die gemeinſame Karte mit Friedmanns\pwindex{Friedmann, Rose 12.02.1864 – 14.01.1919@\textsc{Friedmann, Rose} (12.02.1864 – 14.01.1919)|pw}\pwindex{Friedmann, Louis Philipp 29.06.1861 – 01.04.1939@\textsc{Friedmann, Louis Philipp} (29.06.1861 – 01.04.1939), \emph{Industrieller, Bergsteiger}|pw}, u Grüße auch an dieſe ſowie \label{T_L01952_1v}\edtext{an Sie u Gerty\pwindex{Hofmannsthal, Gertrude von 16.03.1880 – 09.11.1959@\textsc{Hofmannsthal, Gertrude von} (16.03.1880 – 09.11.1959)|pw}}{\lemma{\textnormal{\emph{an Sie u Gerty}}}\Cendnote{\textnormal{weiter quer am rechten Rand}}}\label{T_L01952_1h} von
               uns Beiden\pwindex{Schnitzler, Olga 17.01.1882 – 13.01.1970@\textsc{Schnitzler, Olga} (17.01.1882 – 13.01.1970), \emph{Schauspielerin, Sängerin}|pwv}. Herzlichſt Ihr
                  \spacefill\mbox{A.}\pend
           
         
         \endnumbering\mylabel{h}\end{ledgroupsized}  \newcommand{\dateiname}{L01952}\newcommand{\titel}{Arthur Schnitzler an Hugo von Hofmannsthal, 30. 7. 1910}\newcommand{\editorInnen}{Martin Anton Müller und Gerd-Hermann Susen}%% latex-leseansicht-abspann.tex
%% Abspann für die Leseansicht.
%% Der Schalter \ifkorrekturansicht ist bereits durch den Vorspann gesetzt.

%% latex-abspann.tex
%% Gemeinsamer Abspann für Korrekturansicht und Leseansicht.
%% Setzt den Schalter \ifkorrekturansicht voraus (gesetzt in den
%% einbindenden Dateien latex-korrekturansicht-abspann.tex bzw.
%% latex-leseansicht-abspann.tex).
%% ---------------------------------------------------------------

\normalsize

% Das esempio-Environment wird nur in der Leseansicht benötigt
\ifkorrekturansicht\else
\newenvironment{esempio}[3]%
{
    \vspace{1.5ex}
    \rlap{\underline{#1}}
    \par
    \setlength{\parindent}{0cm}
    \nopagebreak
    \leftskip=#2cm
    \rightskip=#3cm
}
{
    \par
}
\fi

\doendnotes{C}
\bigskip
\vfill

\clearpage

\footnotesize

\ifkorrekturansicht
  \lohead{\textsc{register}}
\fi

% theindex-Environment neu definieren ohne reledmac
\makeatletter
\renewenvironment{theindex}{%
  \ifkorrekturansicht
    \section*{\indexname}%
  \else
    \subsubsection*{Index der erwähnten Entitäten}%
  \fi
  \setlength{\parindent}{0pt}%
  \setlength{\parskip}{0pt plus 0.3pt}%
  \let\item\@idxitem
}{%
  \ifkorrekturansicht\clearpage\fi
}
\makeatother

\IfFileExists{\jobname-pw.ind}{\input{\jobname-pw.ind}}{}

% Quellenangabe nur in der Leseansicht
\ifkorrekturansicht\else
% Fallback-Definitionen, falls die .tex-Datei \titel etc. nicht gesetzt hat
\providecommand{\titel}{}
\providecommand{\editorInnen}{}
\providecommand{\dateiname}{\jobname}

\vspace{3cm}

\vfill

\footnotesize
\textsc{Quelle}: \titel. Herausgegeben von {\editorInnen}. In: \emph{Arthur Schnitzler: Briefwechsel mit Autorinnen und Autoren}.
 Digitale Edition, https://schnitzler-briefe.acdh.oeaw.ac.at/{\dateiname}.html (Stand \today)
\fi

\end{document}


      