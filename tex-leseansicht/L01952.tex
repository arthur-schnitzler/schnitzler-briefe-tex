%% latex-leseansicht-vorspann.tex
%% Vorspann für die Leseansicht.
%% Lädt die gemeinsame Datei latex-vorspann.tex mit nicht gesetztem Schalter.

\newif\ifkorrekturansicht
\korrekturansichtfalse

\input{../tex-inputs/latex-vorspann}


\section[Arthur Schnitzler an Hugo von Hofmannsthal, 30. 7. 1910]{L01952 Arthur Schnitzler an Hugo von Hofmannsthal, 30. 7. 1910}
\nopagebreak\mylabel{L01952v}
\rehead{ }\normalsize\beginnumbering\briefempfaengerindex{Hofmannsthal, Hugo von@\textsc{Hofmannsthal, Hugo von}!zzzSchnitzler, Arthur@\emph{von Arthur Schnitzler}!1910-07-301@{30. 7. 1910}|(be}
\toendnotes[C]{\smallbreak\pagebreak[2]}
\correspDesc{Versand  durch Arthur Schnitzler am 30. 7. 1910 in Wien
\newline{}Erhalt  durch Hugo von Hofmannsthal im Zeitraum [30. 7. 1910
                  – 3. 8. 1910?] \textbf{Ort fehlend} }\toendnotes[C]{\smallbreak}
\Standort{FDH, Hs-30885,138.}
\physDesc{Brief, 1 Blatt, 4 Seiten, 2176 Zeichen
\newline{}Handschrift: schwarze Tinte, deutsche Kurrent}
\buchAbdrucke{\weitereDrucke{Hugo von Hofmannsthal, Arthur Schnitzler: \emph{Briefwechsel}. Herausgegeben von Therese Nickl und Heinrich Schnitzler. Frankfurt am Main: \emph{S. Fischer} 1964, S. 252.} }\toendnotes[C]{\smallbreak}
\pstart
           {\pb}\textcolor{gray}{\textbf{Dr. Arthur Schnitzler}}\hfill XVIII. \textsc{Sternwartestr}. 71.\oindex{Wien@\textbf{Wien}!XVIII., Währing@\textbf{XVIII., Währing}!Sternwartestraße 71@\textbf{Sternwartestraße 71}, \emph{Wohngebäude}|pw}\pend
           
\pstart
           \textcolor{gray}{\textbf{\strikeout{Wien XVIII. Spoettelgasse 7.\oindex{Wien@\textbf{Wien}!XVIII., Währing@\textbf{XVIII., Währing}!Edmund-Weiß-Gasse 7@\textbf{Edmund-Weiß-Gasse 7}, \emph{Wohngebäude}|pw}}}}\hfill 30. 7. 1910!\pend
           \vspace{0.5em}
\pstart
           mein lieber Hugo, Sie{ }ſehen: wir{ }ſind{ }ſchon \label{K_L01952-1v}\edtext{überſiedelt}{\lemma{\textnormal{\emph{übersiedelt}}}\Cendnote{\textnormal{Siehe A. S.: \emph{Tagebuch}, 14. 7. 1910.
               }}}\label{K_L01952-1} – und das{ }ſind auch{ }ſchon wieder faſt drei Wochen her, natürlich gings recht
               allmälig, und auch jetzt{ }ſind wir noch nicht in völliger Ordnung. Aber mein
               Arbeitszimmer iſt längſt{ }ſo wohnlich, daſs es kaum einen rechten Grund gibt das
               Stückeſchreiben länger hinauszuſchieben. Übrigens war ich \label{K_L01952-2v}\edtext{zweimal fort}{\lemma{\textnormal{\emph{zweimal fort}}}\Cendnote{\textnormal{zuerst vom 6. 7. 1910 bis zum 10. 7. 1910, dann vom 26. 7. 1910 bis zum
                     28. 7. 1910}}}\label{K_L01952-2}, auf dem Se{\geminationm}ering\oindex{Semmering@\textbf{Semmering}, \emph{Verwaltungsgebiet}|pw}, mit Olga\pwindex{Schnitzler, Olga 17.\,1.\,1882 Wien – 13.\,1.\,1970 Lugano@\textsc{Schnitzler, Olga} (17.\,1.\,1882 Wien – 13.\,1.\,1970 Lugano), \emph{Schauspielerin, Sängerin}|pw} u Heini\pwindex{Schnitzler, Heinrich 9.\,8.\,1902 Hinterbrühl – 12.\,7.\,1982 Wien@\textsc{Schnitzler, Heinrich} (9.\,8.\,1902 Hinterbrühl – 12.\,7.\,1982 Wien), \emph{Regisseur, Schauspieler}|pw}, knapp vor dem Umzug; und jetzt wieder ein paar Tage
               allein auf {\pb}dem Se{\geminationm}ering\oindex{Semmering@\textbf{Semmering}, \emph{Verwaltungsgebiet}|pw}, viel mit Brahm\pwindex{Brahm, Otto 5.\,2.\,1856 Hamburg – 28.\,11.\,1912 Berlin@\textsc{Brahm, Otto} (5.\,2.\,1856 Hamburg – 28.\,11.\,1912 Berlin), \emph{Theaterleiter, Regisseur}|pw} zuſammen; mit Frau \textsc{Jonas}\pwindex{Jonas, Clara 29.\,6.\,1863 Mannheim – 17.\,1.\,1922 Berlin@\textsc{Jonas, Clara} (29.\,6.\,1863 Mannheim – 17.\,1.\,1922 Berlin)|pw}, mit Kainz\pwindex{Kainz, Josef 2.\,1.\,1858 Mosonmagyaróvár – 20.\,9.\,1910 Wien@\textsc{Kainz, Josef} (2.\,1.\,1858 Mosonmagyaróvár – 20.\,9.\,1910 Wien), \emph{Schauspieler}|pw} (der, we{\geminationn} alles gut geht, bald wieder eine neue Rolle von mir{ }ſpielen dürfte.) Von Se{\geminationm}ering\oindex{Semmering@\textbf{Semmering}, \emph{Verwaltungsgebiet}|pw} aus hab ich eine \label{K_L01952-3v}\edtext{Fußpartie}{\lemma{\textnormal{\emph{Fußpartie}}}\Cendnote{\textnormal{Siehe A. S.: \emph{Tagebuch}, 28. 7. 1910.
               }}}\label{K_L01952-3} gemacht (denken Sie, mein Rad hab ich – verſchenkt{\dotstwo}), über den So{\geminationn}wendſtein\oindex{Sonnwendstein@\textbf{Sonnwendstein}, \emph{Berg}|pw}, ins Otterthal\oindex{Otterthal@\textbf{Otterthal}, \emph{Verwaltungsgebiet}|pw}, über Kirchberg\oindex{Kirchberg am Wechsel@\textbf{Kirchberg am Wechsel}, \emph{Hauptstadt}|pw}, Aspang\oindex{Aspang-Markt@\textbf{Aspang-Markt}, \emph{Verwaltungsgebiet}|pw} nach Mönichkirchen\oindex{Mönichkirchen [Niederösterreich]@\textbf{Mönichkirchen [Niederösterreich]}, \emph{Verwaltungsgebiet}|pw} – etwas ganz
               beſonders{ }ſchönes, von oeſterreichiſcher\oindex{Österreich@\textbf{Österreich}|pw}
               Unberühmtheit; ich hatte mich jahrelange geſehnt, es kennen zu lernen,{ }ſo daſs es ein
               Witzwort unſres Hauſes, beſonders Heinis\pwindex{Schnitzler, Heinrich 9.\,8.\,1902 Hinterbrühl – 12.\,7.\,1982 Wien@\textsc{Schnitzler, Heinrich} (9.\,8.\,1902 Hinterbrühl – 12.\,7.\,1982 Wien), \emph{Regisseur, Schauspieler}|pw} zu
               werden anfing; – und als {\pb}ich es endlich, nach etwa
               zehnſtündiger Wanderung erreichte, – gab es kein Bett im ganzen Ort,{ }ſo daſs ich
               gleich wieder hinunter fahren mußte – (was in jüngern Jahren gewiſs{ }ſymboliſch
               empfunden worden wäre.)\pend
           
\pstart
           Ich hoffe wir reiſen heuer doch noch einmal weg, gegen Ende Auguſt, –
                  \textsc{St. Gilgen}\oindex{St. Gilgen@\textbf{St. Gilgen}, \emph{Verwaltungsgebiet}|pw} vielleicht, oder Iſchl\oindex{Bad Ischl@\textbf{Bad Ischl}|pw}, aber kaum auf lang,
               da die \textsc{Medardus}\pwindex{Schnitzler, Arthur 15.\,5.\,1862 Wien – 21.\,10.\,1931 ebd.@\textsc{Schnitzler, Arthur} (15.\,5.\,1862 Wien – 21.\,10.\,1931 ebd.), \emph{Schriftsteller, Mediziner}!junge Medardus. Dramatische Historie in einem Vorspiel und fünf Aufzügen@\strich\emph{Der junge Medardus. Dramatische Historie in einem Vorspiel und fünf Aufzügen}|pw} Proben{ }ſehr früh beginnen dürften. \strikeout{\textcolor{gray}{Also}} Es wäre wirklich{ }ſchön, wieder einmal ein paar So{\geminationm}ertage miteinander zu verleben; aber daſs man{ }ſich in Wien\oindex{Wien@\textbf{Wien}, \emph{Verwaltungsgebiet}|pw}{ }ſo{ }ſelten, ja nahezu{ }ſchon gar nicht{ }ſieht, iſt
               wahrhaftig nicht {\pb}meine Schuld allein. Erſtens reiſen Sie
               viel zu viel – und we{\geminationn} Sie von Rodaun\oindex{Wien@\textbf{Wien}!XXIII., Liesing@\textbf{XXIII., Liesing}!Rodaun@\textbf{Rodaun}, \emph{Region}|pw} nach Wien\oindex{Wien@\textbf{Wien}, \emph{Verwaltungsgebiet}|pw} ko{\geminationm}en, erfährt man es doch meiſtens nur ganz zufällig oder
               gar nicht. Entſchließen Sie{ }ſich doch wieder öfter telegrafiſch oder{ }ſonſtwie{ }ſich
               anzuſagen oder anzufragen – da{\geminationn}{ }ſollen Sie mich ke{\geminationn}en lernen! Eine hiſtoriſche Berichtigung: \textsc{Welsberg}\oindex{Welsberg-Taisten@\textbf{Welsberg-Taisten}, \emph{Verwaltungsgebiet}|pw} ist nicht \substVorne{}\textsuperscript{3}\substDazwischen{}4\substHinten{}, sondern 3 Jahre her – auch lang genug! Haben Sie meine \label{K_L01952-4v}\edtext{Karte aus Glion\oindex{Glion@\textbf{Glion}|pw}}{\lemma{\textnormal{\emph{Karte aus Glion}}}\Cendnote{\textnormal{Vgl. A. S.: \emph{Tagebuch}, 28. 5. 1910. Das
                     Korrespondenzstück ist nicht überliefert.
               }}}\label{K_L01952-4} beko{\geminationm}en – was \label{K_L01952-5v}\edtext{12 Jahre her}{\lemma{\textnormal{\emph{12 Jahre her}}}\Cendnote{\textnormal{Siehe A. S.: \emph{Tagebuch}, 14. 8. 1898.
               }}}\label{K_L01952-5} iſt! – Man ka{\geminationn} den Feuilletoniſten nicht Unrecht
               geben: die Zeit verrinnt{\dots}\pend
           
\pstart
           Schönen Dank für die gemeinſame Karte mit Friedmanns\pwindex{Friedmann, Rose 12.\,2.\,1864 – 14.\,1.\,1919 Baden bei Wien@\textsc{Friedmann, Rose} (12.\,2.\,1864 – 14.\,1.\,1919 Baden bei Wien)|pw}\pwindex{Friedmann, Louis Philipp 29.\,6.\,1861 Paris – 1.\,4.\,1939 Wien@\textsc{Friedmann, Louis Philipp} (29.\,6.\,1861 Paris – 1.\,4.\,1939 Wien), \emph{Industrieller, Bergsteiger}|pw}, u Grüße auch an dieſe{ }ſowie \label{T_L01952-1v}\edtext{an Sie u Gerty\pwindex{Hofmannsthal, Gertrude von 16.\,3.\,1880 Wien – 9.\,11.\,1959 Paddington@\textsc{Hofmannsthal, Gertrude von} (16.\,3.\,1880 Wien – 9.\,11.\,1959 Paddington)|pw}}{\lemma{\textnormal{\emph{an Sie u Gerty}}}\Cendnote{\textnormal{weiter quer am rechten Rand}}}\label{T_L01952-1} von
               uns Beiden\pwindex{Schnitzler, Olga 17.\,1.\,1882 Wien – 13.\,1.\,1970 Lugano@\textsc{Schnitzler, Olga} (17.\,1.\,1882 Wien – 13.\,1.\,1970 Lugano), \emph{Schauspielerin, Sängerin}|pwv}. Herzlichſt Ihr
                  \spacefill\mbox{A.}\pend
           \selectlanguage{ngerman}\endnumbering\briefempfaengerindex{Hofmannsthal, Hugo von@\textsc{Hofmannsthal, Hugo von}!zzzSchnitzler, Arthur@\emph{von Arthur Schnitzler}!1910-07-301@{30. 7. 1910}|)be}\mylabel{L01952h}  \newcommand{\dateiname}{L01952}\newcommand{\titel}{Arthur Schnitzler an Hugo von Hofmannsthal, 30. 7. 1910}\newcommand{\editorInnen}{Martin Anton Müller und Gerd-Hermann Susen}%% latex-leseansicht-abspann.tex
%% Abspann für die Leseansicht.
%% Der Schalter \ifkorrekturansicht ist bereits durch den Vorspann gesetzt.

%% latex-abspann.tex
%% Gemeinsamer Abspann für Korrekturansicht und Leseansicht.
%% Setzt den Schalter \ifkorrekturansicht voraus (gesetzt in den
%% einbindenden Dateien latex-korrekturansicht-abspann.tex bzw.
%% latex-leseansicht-abspann.tex).
%% ---------------------------------------------------------------

\normalsize

% Das esempio-Environment wird nur in der Leseansicht benötigt
\ifkorrekturansicht\else
\newenvironment{esempio}[3]%
{
    \vspace{1.5ex}
    \rlap{\underline{#1}}
    \par
    \setlength{\parindent}{0cm}
    \nopagebreak
    \leftskip=#2cm
    \rightskip=#3cm
}
{
    \par
}
\fi

\doendnotes{C}
\bigskip
\vfill

\clearpage

\footnotesize

\ifkorrekturansicht
  \lohead{\textsc{register}}
\fi

% theindex-Environment neu definieren ohne reledmac
\makeatletter
\renewenvironment{theindex}{%
  \ifkorrekturansicht
    \section*{\indexname}%
  \else
    \subsubsection*{Index der erwähnten Entitäten}%
  \fi
  \setlength{\parindent}{0pt}%
  \setlength{\parskip}{0pt plus 0.3pt}%
  \let\item\@idxitem
}{%
  \ifkorrekturansicht\clearpage\fi
}
\makeatother

\IfFileExists{\jobname-pw.ind}{\input{\jobname-pw.ind}}{}

% Quellenangabe nur in der Leseansicht
\ifkorrekturansicht\else
% Fallback-Definitionen, falls die .tex-Datei \titel etc. nicht gesetzt hat
\providecommand{\titel}{}
\providecommand{\editorInnen}{}
\providecommand{\dateiname}{\jobname}

\vspace{3cm}

\vfill

\footnotesize
\textsc{Quelle}: \titel. Herausgegeben von {\editorInnen}. In: \emph{Arthur Schnitzler: Briefwechsel mit Autorinnen und Autoren}.
 Digitale Edition, https://schnitzler-briefe.acdh.oeaw.ac.at/{\dateiname}.html (Stand \today)
\fi

\end{document}


