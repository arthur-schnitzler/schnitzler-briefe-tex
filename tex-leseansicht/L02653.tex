%% latex-leseansicht-vorspann.tex
%% Vorspann für die Leseansicht.
%% Lädt die gemeinsame Datei latex-vorspann.tex mit nicht gesetztem Schalter.

\newif\ifkorrekturansicht
\korrekturansichtfalse

\input{../tex-inputs/latex-vorspann}


         
         \renewcommand{\erwaehntePersonen}{Personen:  ?? [Freund 1 von Paul Lasker Schüler],  ?? [Freund 2 von Paul Lasker Schüler], Albert Einstein, Paul Lasker-Schüler, Elise Schiedlbauer, Frank Wedekind}
         \renewcommand{\erwaehnteOrte}{Orte: Berlin, Café Central, Florianigasse, Gastwirtschaft Schiedlbauer, Haberlandstraße, Hotel Koschel, Motzstraße, München, Wien, Zürich}
         \renewcommand{\erwaehnteWerke}{}
               \section[Else Lasker-Schüler an Arthur Schnitzler, 10. 12. 1924]{ Else Lasker-Schüler an Arthur Schnitzler, 10. 12. 1924}\nopagebreak\mylabel{v}\rehead{ }\begin{ledgroupsized}[t]{13cm}\normalsize\beginnumbering \toendnotes[C]{\smallbreak\pagebreak[2]} \Standort{unbekannt, Privatbesitz, \emph{ohne Signatur}.}
\physDesc{Brief, 5 Blätter, 6 Seiten (Paginierung 2–5)
\newline{}Handschrift: schwarze Tinte, lateinische Kurrent
\newline{}Schnitzler: 1) mit Bleistift auf der ersten Seite Vermerk: »\textsc{Laske\textcolor{gray}{r} Schüler}« und »L ſch\pwindex{Lasker-Schueler, Else 11.02.1869 – 22.01.1945@\textsc{Lasker-Schüler, Else} (11.02.1869 – 22.01.1945), \emph{Dichterin}|pwv}«, auf der zweiten Seite »2/1«, auf den
                                 Seiten zwei bis vier außerdem die Datierung »10/12 24«  2) mit rotem Buntstift Vermerk: »(\textsc{Ihr
                                          Sohn\pwindex{Lasker-Schueler, Paul 1899-08-24 – 1927-12-14@\textsc{Lasker-Schüler, Paul} (1899-08-24 – 1927-12-14)|pwv}})«\newline{}Ordnung: von unbekannter Hand mit rotem Buntstift zwölf
                                 Unterstreichungen \newline{}Zusatz: Der Brief lässt sich 2002 im Besitz des
                                 Antiquariats Eberhard Köstler in Tutzing nachweisen.
                                    2006 wurde er an das Antiquariat Inlibris in Wien
                                 verkauft. Der weitere Verbleib ist ungeklärt. Ebenso ungeklärt
                                 bleibt, warum das Original des Briefes nicht im Nachlass
                                 Schnitzlers überliefert ist. Die im Nachlass befindliche Abschrift
                                 weist handschriftliche Spuren Schnitzler\pwindex{Schnitzler, Arthur 15.05.1862 – 21.10.1931@\textsc{Schnitzler, Arthur} (15.05.1862 – 21.10.1931), \emph{Schriftsteller, Mediziner}|pw}s aus, wurde aber nicht mit einer von
                                 Schnitzlers Schreibmaschinen getippt. Eine wahrscheinliche (obzwar
                                 nicht häufiger nachweisbare) Erklärung ist, dass Schnitzler\pwindex{Schnitzler, Arthur 15.05.1862 – 21.10.1931@\textsc{Schnitzler, Arthur} (15.05.1862 – 21.10.1931), \emph{Schriftsteller, Mediziner}|pw} selbst den Brief an eine
                                 Autografensammlerin oder einen -sammler schenkte. Grundlage unserer
                                 Transkription stellt eine Kopie dar. }\Standort{DLA, A:Schnitzler, HS.1985.1.3875.}
\physDesc{Brief, Maschinenschriftliche Abschrift, 1 Blatt, 4 Seiten
\newline{}Schreibmaschine
\newline{}Schnitzler: mit rotem Buntstift Vermerk: »\uline{\textsc{Else Lasker Schule{[}r{]}}}« }\buchAbdrucke{\weitereDrucke{Else Lasker-Schüler: \emph{Werke und Briefe. Kritische Ausgabe. Band 7: Briefe
                        1914–1924}. Hg. Karl Jürgen Skrodzki. Frankfurt am Main: \emph{Jüdischer Verlag Verlag} 2004, S. 315–316.} }\toendnotes[C]{\smallbreak}\pstart
           \noindent{}10. XII. 24\hfill Berlin W Motztr. 78\oindex{Motzstrasse@\textbf{Motzstraße}|pw}{ }\pend
           \pstart
           \raggedleft{}Hôtel Koschel\oindex{Hotel Koschel@\textbf{Hotel Koschel}|pw}\pend
           \pstart\center{}Hochzuverehrender Herr Doktor und lieber Dichter\pend\pstart
           Ich fühle es mit Bestimmtheit, daß ich diesen Brief nicht nur in \introOben{}den\introOben{} Wind schreibe. Wenn man wenigstens immer in den Wind schriebe, aber man
               schreibt ja nicht an kleinherzige Menschen. Es hat mir kein Mensch geraten an Sie,
               lieber Dichter, zu schreiben, es überkam mich \strikeout{\textcolor{gray}{×}}, Sie um eine große Gefälligkeit zu bitten, nämlich mit {\pb}meinem geliebten
                  Kinde\pwindex{Lasker-Schueler, Paul 1899-08-24 – 1927-12-14@\textsc{Lasker-Schüler, Paul} (1899-08-24 – 1927-12-14)|pwv}, meinem Sohn\pwindex{Lasker-Schueler, Paul 1899-08-24 – 1927-12-14@\textsc{Lasker-Schüler, Paul} (1899-08-24 – 1927-12-14)|pwv} zu sprechen. Ich bin
               Else Lasker-Schüler; mein Junge\pwindex{Lasker-Schueler, Paul 1899-08-24 – 1927-12-14@\textsc{Lasker-Schüler, Paul} (1899-08-24 – 1927-12-14)|pwv} wohnt in Wien \introOben{}VIII.\introOben{} Florianigasse 47/49 Stiege II. \introOben{}Thüre 25\introOben{}\oindex{Florianigasse@\textbf{Florianigasse}|pw} in einem grossen Zimmer bei einer netten Wirtin\pwindex{Schiedlbauer, Elise @\textsc{Schiedlbauer, Elise}, \emph{Gastwirtin}|pwv}\oindex{Gastwirtschaft Schiedlbauer@\textbf{Gastwirtschaft Schiedlbauer}|pwv}. Wenn Sie ihm schreiben lassen, kommt er zur angegebenen Zeit, Herr Doktor. Ich
               möchte Ihnen so viel sagen; schon wie ich im Januar in
                  Wien\oindex{Wien@\textbf{Wien}|pw} war. Ich bekam dort Scharlach und
               Diphteritie, saß dabei vier Wochen in Flanell gehüllt im Cafe Central\oindex{Cafe Central@\textbf{Café Central}|pw} am Fenster und ich glaube das herrliche Wien\oindex{Wien@\textbf{Wien}|pw}er Trinkwasser heilte mich. Ich habe in München\oindex{Muenchen@\textbf{München}|pw} jetzt Gelegenheit gehabt, meinen Paul\pwindex{Lasker-Schueler, Paul 1899-08-24 – 1927-12-14@\textsc{Lasker-Schüler, Paul} (1899-08-24 – 1927-12-14)|pw} zeichnerisch anzubringen {\pb}aber er
               liebt Wien\oindex{Wien@\textbf{Wien}|pw} so und bat mich doch dort bleiben zu
               können. Zunächst versuchte er mit einem \label{K_L02653-2v}\edtext{Freund\pwindex{?? [Freund 1 von Paul Lasker Schueler] @\textsc{?? [Freund 1 von Paul Lasker Schüler]}|pwv}}{\lemma{\textnormal{\emph{Freund}}}\Cendnote{\textnormal{nicht identifiziert}}}\label{K_L02653-2h} Plakate zu
               zeichnen für Geschäfte. Einen Monat ging das, aber nun ist grosser Stillstand.
                  Nu\textcolor{gray}{n} möchte ich so gern, hochzuverehrender Herr Doktor, daß Sie
               mein liebes Kind\pwindex{Lasker-Schueler, Paul 1899-08-24 – 1927-12-14@\textsc{Lasker-Schüler, Paul} (1899-08-24 – 1927-12-14)|pwv} kennen
               lernen; er ist der liebste \uline{kindlichste}{ }Junge\pwindex{Lasker-Schueler, Paul 1899-08-24 – 1927-12-14@\textsc{Lasker-Schüler, Paul} (1899-08-24 – 1927-12-14)|pwv}, den ich fast kenne –
               im Grunde;– aber was man \uline{mir} nicht antut – vielleicht
               aus Feigheit, – muß der arme Junge\pwindex{Lasker-Schueler, Paul 1899-08-24 – 1927-12-14@\textsc{Lasker-Schüler, Paul} (1899-08-24 – 1927-12-14)|pwv} erleiden. Ich weiß \uline{wie} unerhört er
                  \label{K_L02653-3v}\edtext{in Wien\oindex{Wien@\textbf{Wien}|pw} angeschwärzt}{\lemma{\textnormal{\emph{in Wien angeschwärzt}}}\Cendnote{\textnormal{nicht
                  ermittelt}}}\label{K_L02653-3h} wurde; niemand spricht von seiner Bescheidenheit, auch in
               künstlerischen Dingen. {\pb}\uline{Darum} wird er sich alleine nie durchsetzen, ich meine
               – weiterkommen – äußerlich – was doch \introOben{}hier\introOben{} sein muss. Er
               giebt sich \strikeout{\textcolor{gray}{so}} Mühe, aber es gelingt ihm nicht und ich tue ja alles was in meiner Kraft
               liegt. Danach muß er stets genug zu essen und Anzuziehen haben und wenn er \uline{nicht} charmant seinen \label{K_L02653-5v}\edtext{Besuch}{\lemma{\textnormal{\emph{Besuch}}}\Cendnote{\textnormal{kein
                  Zusammentreffen Paul Lasker-Schüler\pwindex{Lasker-Schueler, Paul 1899-08-24 – 1927-12-14@\textsc{Lasker-Schüler, Paul} (1899-08-24 – 1927-12-14)|pwk}s mit
                     Schnitzler\pwindex{Schnitzler, Arthur 15.05.1862 – 21.10.1931@\textsc{Schnitzler, Arthur} (15.05.1862 – 21.10.1931), \emph{Schriftsteller, Mediziner}|pwk} ist belegt}}}\label{K_L02653-5h} bei Ihnen
               machen sollte, so kann ich nichts dafür. Wirklich es \textcolor{gray}{l}eben nicht
               zwei Menschen mehr, die \label{K_L02653-6v}\edtext{verfolgter}{\lemma{\textnormal{\emph{verfolgter}}}\Cendnote{\textnormal{Womöglich deutete
                  Lasker-Schüler hier antisemitische Anfeindungen an.}}}\label{K_L02653-6h} sind wie wir zwei, mein
                  Junge\pwindex{Lasker-Schueler, Paul 1899-08-24 – 1927-12-14@\textsc{Lasker-Schüler, Paul} (1899-08-24 – 1927-12-14)|pwv} und ich. Herr
               Doktor, ich bitte Sie herzlich als Mensch und als Dichterin, \introOben{}(\introOben{}und nie werde ich es Ihnen vergessen) meinen Jungen\pwindex{Lasker-Schueler, Paul 1899-08-24 – 1927-12-14@\textsc{Lasker-Schüler, Paul} (1899-08-24 – 1927-12-14)|pwv} einmal einzuladen. Wedekind\pwindex{Wedekind, Frank 24.07.1864 – 09.03.1918@\textsc{Wedekind, Frank} (24.07.1864 – 09.03.1918), \emph{Schriftsteller, Schauspieler}|pw}{ }\introOben{}war direkt begeistert von ihm in Zürich\oindex{Zuerich@\textbf{Zürich}|pw}\introOben{} und Prof. \label{K_L02653-7v}\edtext{Einstein\pwindex{Einstein, Albert 14.03.1879 – 18.04.1955@\textsc{Einstein, Albert} (14.03.1879 – 18.04.1955), \emph{Physiker}|pw}}{\lemma{\textnormal{\emph{Einstein}}}\Cendnote{\textnormal{Albert Einstein\pwindex{Einstein, Albert 14.03.1879 – 18.04.1955@\textsc{Einstein, Albert} (14.03.1879 – 18.04.1955), \emph{Physiker}|pwk} und Else Lasker-Schüler
                  lebten beide in der Haberlandstraße 5\oindex{Haberlandstrasse@\textbf{Haberlandstraße}|pwk} (heute 3) in Berlin\oindex{Berlin@\textbf{Berlin}|pwk}.}}}\label{K_L02653-7h} fand ihn prachtvoll{[}.{]}{ }{\pb}Vielleicht können Sie ihm raten, wohin er sich wenden soll, Ihr Wort in Wien\oindex{Wien@\textbf{Wien}|pw} gilt ja. Was kann ich für Sie je tun? Kommen
               Sie bald nach Berlin\oindex{Berlin@\textbf{Berlin}|pw}? Sehe ich Sie? Denken Sie,
               ich kenne nur ein \label{K_L02653-8v}\edtext{Schauspiel}{\lemma{\textnormal{\emph{Schauspiel}}}\Cendnote{\textnormal{nicht ermittelt}}}\label{K_L02653-8h} von Ihnen; ich gehe
               so selten ins Theater, so erschöpft bin ich am Abend. Ich bitte Sie mir die Freude zu
               machen, Herr Doktor, und es wäre so schön mein Junge\pwindex{Lasker-Schueler, Paul 1899-08-24 – 1927-12-14@\textsc{Lasker-Schüler, Paul} (1899-08-24 – 1927-12-14)|pwv} und seine \label{K_L02653-9v}\edtext{Freunde\pwindex{?? [Freund 1 von Paul Lasker Schueler] @\textsc{?? [Freund 1 von Paul Lasker Schüler]}|pwv}\pwindex{?? [Freund 2 von Paul Lasker Schueler] @\textsc{?? [Freund 2 von Paul Lasker Schüler]}|pwv}}{\lemma{\textnormal{\emph{Freunde}}}\Cendnote{\textnormal{nicht identifiziert}}}\label{K_L02653-9h} würden mal wo
               eingeladen in Familie, alle drei, entzückende Bengels. Als wir noch in Berlin\oindex{Berlin@\textbf{Berlin}|pw} waren, gingen wir oft zusammen ins Kino,
               mein Sanatorium. Ich grüße Sie, hochwerter lieber Dichter, Ihre {\\}\spacefill\mbox{Else Lasker-Schüler}{ }{\\}der Prinz von Theben {[}Mondsichel mit
                  Stern{]}\pend
           \pstart
           \noindent{}Was kann ich je für Sie tun?\pend
           \pstart
           {\pb}\label{T_L02653-1v}\edtext{Motzstr. 78 Berlin W.\oindex{Motzstrasse@\textbf{Motzstraße}|pw}{\\}Hôtel Koschel\oindex{Hotel Koschel@\textbf{Hotel Koschel}|pw}{\\}{[}Segelschiff auf Wasser{]}{\\}mit lieben Grüßen}{\lemma{\textnormal{\emph{Motzstr. 78 … Grüßen}}}\Cendnote{\textnormal{auf der Rückseite
                  des letzten Blattes, dieses ins Querformat gedreht und in der linken oberen Ecke
                  beschrieben}}}\label{T_L02653-1h}\pend
           
         
         \endnumbering\mylabel{h}\end{ledgroupsized}  \newcommand{\dateiname}{L02653}\newcommand{\titel}{Else Lasker-Schüler an Arthur Schnitzler, 10. 12. 1924}\newcommand{\editorInnen}{Martin Anton Müller und Laura Untner}%% latex-leseansicht-abspann.tex
%% Abspann für die Leseansicht.
%% Der Schalter \ifkorrekturansicht ist bereits durch den Vorspann gesetzt.

%% latex-abspann.tex
%% Gemeinsamer Abspann für Korrekturansicht und Leseansicht.
%% Setzt den Schalter \ifkorrekturansicht voraus (gesetzt in den
%% einbindenden Dateien latex-korrekturansicht-abspann.tex bzw.
%% latex-leseansicht-abspann.tex).
%% ---------------------------------------------------------------

\normalsize

% Das esempio-Environment wird nur in der Leseansicht benötigt
\ifkorrekturansicht\else
\newenvironment{esempio}[3]%
{
    \vspace{1.5ex}
    \rlap{\underline{#1}}
    \par
    \setlength{\parindent}{0cm}
    \nopagebreak
    \leftskip=#2cm
    \rightskip=#3cm
}
{
    \par
}
\fi

\doendnotes{C}
\bigskip
\vfill

\clearpage

\footnotesize

\ifkorrekturansicht
  \lohead{\textsc{register}}
\fi

% theindex-Environment neu definieren ohne reledmac
\makeatletter
\renewenvironment{theindex}{%
  \ifkorrekturansicht
    \section*{\indexname}%
  \else
    \subsubsection*{Index der erwähnten Entitäten}%
  \fi
  \setlength{\parindent}{0pt}%
  \setlength{\parskip}{0pt plus 0.3pt}%
  \let\item\@idxitem
}{%
  \ifkorrekturansicht\clearpage\fi
}
\makeatother

\IfFileExists{\jobname-pw.ind}{\input{\jobname-pw.ind}}{}

% Quellenangabe nur in der Leseansicht
\ifkorrekturansicht\else
% Fallback-Definitionen, falls die .tex-Datei \titel etc. nicht gesetzt hat
\providecommand{\titel}{}
\providecommand{\editorInnen}{}
\providecommand{\dateiname}{\jobname}

\vspace{3cm}

\vfill

\footnotesize
\textsc{Quelle}: \titel. Herausgegeben von {\editorInnen}. In: \emph{Arthur Schnitzler: Briefwechsel mit Autorinnen und Autoren}.
 Digitale Edition, https://schnitzler-briefe.acdh.oeaw.ac.at/{\dateiname}.html (Stand \today)
\fi

\end{document}


      