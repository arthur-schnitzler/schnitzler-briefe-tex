%% latex-leseansicht-vorspann.tex
%% Vorspann für die Leseansicht.
%% Lädt die gemeinsame Datei latex-vorspann.tex mit nicht gesetztem Schalter.

\newif\ifkorrekturansicht
\korrekturansichtfalse

\input{../tex-inputs/latex-vorspann}


\section[Richard Beer-Hofmann an Arthur Schnitzler, 5. 8. 1906]{L01622 Richard Beer-Hofmann an Arthur Schnitzler, 5. 8. 1906}
\nopagebreak\mylabel{L01622v}
\rehead{ }\normalsize\beginnumbering\briefempfaengerindex{Schnitzler, Arthur@\textsc{Schnitzler, Arthur}!zzzBeer-Hofmann, Richard@\emph{von Richard Beer-Hofmann}!1906-08-051@{5. 8. 1906}|(be}
\toendnotes[C]{\smallbreak\pagebreak[2]}
\correspDesc{Versand  durch Richard Beer-Hofmann am 5. 8. 1906 in Bayreuth
\newline{}Erhalt  durch Arthur Schnitzler am 6. 8. 1906 in Helsingør}\toendnotes[C]{\smallbreak}
\Standort{CUL, Schnitzler, B 8.}
\physDesc{Bildpostkarte, 116 Zeichen
\newline{}Handschrift: Bleistift, lateinische Kurrent
\newline{}Versand: 1) Stempel: »\nobreak{}\oindex{Bayreuth@\textbf{Bayreuth}, \emph{Hauptstadt}|pwk}Bayreuth, 5 Aug 06, 12-1N\nobreak{}«.   2) Stempel: »\nobreak{}\oindex{Helsingør@\textbf{Helsingør}, \emph{Hauptstadt}|pwk}Helsingør, 6. 8. 06, 10-11E\nobreak{}«. 
\newline{}Ordnung: mit Bleistift von unbekannter Hand nummeriert:
                                    »207« }\toendnotes[C]{\smallbreak}\pstart{}{\pb}Herrn D\textsuperscript{r} Arthur Schnitzler\pend{}\pstart{}Marienlyst\oindex{Marienlyst@\textbf{Marienlyst}, \emph{Gut}|pw}\pend{}\pstart{}Curhaus\oindex{Kurhotellet@\textbf{Kurhotellet}, \emph{Hotel}|pw}\pend{}\pstart{}Dänemark\oindex{Dänemark@\textbf{Dänemark}|pw}\pend{}{\bigskip}
\pstart
           \noindent{}\centering{}{\pb}\textcolor{gray}{\textbf{Bayreuth\oindex{Bayreuth@\textbf{Bayreuth}, \emph{Hauptstadt}|pw}, Kgl. \label{T_L01622-1v}\edtext{Opern}{\lemma{\textnormal{\emph{Opern}}}\Cendnote{\textnormal{Umrahmt, ein Pfeil weist auf »Das
                        andere«.}}}\label{T_L01622-1}haus\oindex{Markgräfliches Opernhaus@\textbf{Markgräfliches Opernhaus}, \emph{Oper}|pw}.}}\pend
           
\pstart
           \centering{}\textcolor{gray}{\textbf{Trompeterloge}}\pend
           \vspace{1em}
\pstart
           {\pb}5/VIII 06\pend
           \vspace{0.5em}
\pstart
           Das andere\oindex{Richard-Wagner-Festspielhaus@\textbf{Richard-Wagner-Festspielhaus}, \emph{Oper}|pwv} ist \label{K_L01622-1v}\edtext{auch eins}{\lemma{\textnormal{\emph{auch eins}}}\Cendnote{\textnormal{Er spielt auf das Festspielhaus\oindex{Richard-Wagner-Festspielhaus@\textbf{Richard-Wagner-Festspielhaus}, \emph{Oper}|pwk} an, in dem die Wagner\pwindex{Wagner, Richard 22.\,5.\,1813 Leipzig – 13.\,2.\,1883 Venedig@\textsc{Wagner, Richard} (22.\,5.\,1813 Leipzig – 13.\,2.\,1883 Venedig), \emph{Komponist}|pwk}-Inszenierungen stattfinden.}}}\label{K_L01622-1}. Ich war gestern \uuline{im}{ }Parsifal\pwindex{Wagner, Richard 22.\,5.\,1813 Leipzig – 13.\,2.\,1883 Venedig@\textsc{Wagner, Richard} (22.\,5.\,1813 Leipzig – 13.\,2.\,1883 Venedig), \emph{Komponist}!Parsifal@\strich\emph{Parsifal}|pw}.\pend
           \pstart \spacefill\mbox{Richard}\pend{}\selectlanguage{ngerman}\endnumbering\briefempfaengerindex{Schnitzler, Arthur@\textsc{Schnitzler, Arthur}!zzzBeer-Hofmann, Richard@\emph{von Richard Beer-Hofmann}!1906-08-051@{5. 8. 1906}|)be}\mylabel{L01622h}  \newcommand{\dateiname}{L01622}\newcommand{\titel}{Richard Beer-Hofmann an Arthur Schnitzler, 5. 8. 1906}\newcommand{\editorInnen}{Martin Anton Müller und Gerd-Hermann Susen}%% latex-leseansicht-abspann.tex
%% Abspann für die Leseansicht.
%% Der Schalter \ifkorrekturansicht ist bereits durch den Vorspann gesetzt.

%% latex-abspann.tex
%% Gemeinsamer Abspann für Korrekturansicht und Leseansicht.
%% Setzt den Schalter \ifkorrekturansicht voraus (gesetzt in den
%% einbindenden Dateien latex-korrekturansicht-abspann.tex bzw.
%% latex-leseansicht-abspann.tex).
%% ---------------------------------------------------------------

\normalsize

% Das esempio-Environment wird nur in der Leseansicht benötigt
\ifkorrekturansicht\else
\newenvironment{esempio}[3]%
{
    \vspace{1.5ex}
    \rlap{\underline{#1}}
    \par
    \setlength{\parindent}{0cm}
    \nopagebreak
    \leftskip=#2cm
    \rightskip=#3cm
}
{
    \par
}
\fi

\doendnotes{C}
\bigskip
\vfill

\clearpage

\footnotesize

\ifkorrekturansicht
  \lohead{\textsc{register}}
\fi

% theindex-Environment neu definieren ohne reledmac
\makeatletter
\renewenvironment{theindex}{%
  \ifkorrekturansicht
    \section*{\indexname}%
  \else
    \subsubsection*{Index der erwähnten Entitäten}%
  \fi
  \setlength{\parindent}{0pt}%
  \setlength{\parskip}{0pt plus 0.3pt}%
  \let\item\@idxitem
}{%
  \ifkorrekturansicht\clearpage\fi
}
\makeatother

\IfFileExists{\jobname-pw.ind}{\input{\jobname-pw.ind}}{}

% Quellenangabe nur in der Leseansicht
\ifkorrekturansicht\else
% Fallback-Definitionen, falls die .tex-Datei \titel etc. nicht gesetzt hat
\providecommand{\titel}{}
\providecommand{\editorInnen}{}
\providecommand{\dateiname}{\jobname}

\vspace{3cm}

\vfill

\footnotesize
\textsc{Quelle}: \titel. Herausgegeben von {\editorInnen}. In: \emph{Arthur Schnitzler: Briefwechsel mit Autorinnen und Autoren}.
 Digitale Edition, https://schnitzler-briefe.acdh.oeaw.ac.at/{\dateiname}.html (Stand \today)
\fi

\end{document}


