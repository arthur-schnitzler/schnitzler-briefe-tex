%% latex-leseansicht-vorspann.tex
%% Vorspann für die Leseansicht.
%% Lädt die gemeinsame Datei latex-vorspann.tex mit nicht gesetztem Schalter.

\newif\ifkorrekturansicht
\korrekturansichtfalse

\input{../tex-inputs/latex-vorspann}


\section[Richard Dehmel an Arthur Schnitzler, {[}1907{]}]{L01647 Richard Dehmel an Arthur Schnitzler, {[}1907{]}}
\nopagebreak\mylabel{L01647v}
\rehead{ }\normalsize\beginnumbering\briefempfaengerindex{Schnitzler, Arthur@\textsc{Schnitzler, Arthur}!zzzDehmel, Richard@\emph{von Richard Dehmel}!1907-12-312@{{[}1907{]}}|(be}
\toendnotes[C]{\smallbreak\pagebreak[2]}
\correspDesc{Versand  durch Richard Dehmel im Zeitraum [1907] \textbf{Ort fehlend} 
\newline{}Erhalt  durch Arthur Schnitzler in Wien}\toendnotes[C]{\smallbreak}
\Standort{CUL, Schnitzler, B 26.}
\physDesc{Brief, 1 Blatt, 1 Seite, 680 Zeichen
\newline{}Druck}
\pstart\center{}{\pb}EURER WOHLGEBOREN\pend\vspace{0.5em}
\pstart
           erhalten anbei ein Exemplar meiner »Verwandlungen der
                  Venus\pwindex{Dehmel, Richard 18.\,11.\,1863 Hermsdorf – 8.\,2.\,1920 Blankenese@\textsc{Dehmel, Richard} (18.\,11.\,1863 Hermsdorf – 8.\,2.\,1920 Blankenese), \emph{Schriftsteller, Schriftsteller, Krimiautor}!Verwandlungen der Venus@\strich\emph{Die Verwandlungen der Venus}|pw}« im \uline{vollständigen} Wortlaut. Ich sende
               es Ihnen, weil ich annehmen darf, daß Sie der genannten Dichtung, deren öffentliche
               Ausgabe an einer wichtigen Stelle (Venus Consolatrix) auf gerichtlichen Befehl
               verstümmelt werden mußte, ein rein ästhetisches oder ideelles Interesse
               entgegenbringen. Deshalb darf ich auch glauben, daß Sie dieses private Exemplar,
               welches ich Ihnen als \uline{vertrauliche} Gabe überreiche,
               nicht in falsche Hände geraten lassen werden. Meine Absicht dabei ist lediglich die,
               einige vollständige Exemplare des Textes dem Urteil der Nachlebenden zuzuführen.\pend
           
\pstart
           Mit besonderer Hochachtung{\\[\baselineskip]}\spacefill\mbox{R. DEHMEL.}\pend
           \leftskip=0em{}\selectlanguage{ngerman}\endnumbering\briefempfaengerindex{Schnitzler, Arthur@\textsc{Schnitzler, Arthur}!zzzDehmel, Richard@\emph{von Richard Dehmel}!1907-01-012@{{[}1907{]}}|)be}\mylabel{L01647h}  \newcommand{\dateiname}{L01647}\newcommand{\titel}{Richard Dehmel an Arthur Schnitzler, [1907]}\newcommand{\editorInnen}{Martin Anton Müller und Gerd-Hermann Susen}%% latex-leseansicht-abspann.tex
%% Abspann für die Leseansicht.
%% Der Schalter \ifkorrekturansicht ist bereits durch den Vorspann gesetzt.

%% latex-abspann.tex
%% Gemeinsamer Abspann für Korrekturansicht und Leseansicht.
%% Setzt den Schalter \ifkorrekturansicht voraus (gesetzt in den
%% einbindenden Dateien latex-korrekturansicht-abspann.tex bzw.
%% latex-leseansicht-abspann.tex).
%% ---------------------------------------------------------------

\normalsize

% Das esempio-Environment wird nur in der Leseansicht benötigt
\ifkorrekturansicht\else
\newenvironment{esempio}[3]%
{
    \vspace{1.5ex}
    \rlap{\underline{#1}}
    \par
    \setlength{\parindent}{0cm}
    \nopagebreak
    \leftskip=#2cm
    \rightskip=#3cm
}
{
    \par
}
\fi

\doendnotes{C}
\bigskip
\vfill

\clearpage

\footnotesize

\ifkorrekturansicht
  \lohead{\textsc{register}}
\fi

% theindex-Environment neu definieren ohne reledmac
\makeatletter
\renewenvironment{theindex}{%
  \ifkorrekturansicht
    \section*{\indexname}%
  \else
    \subsubsection*{Index der erwähnten Entitäten}%
  \fi
  \setlength{\parindent}{0pt}%
  \setlength{\parskip}{0pt plus 0.3pt}%
  \let\item\@idxitem
}{%
  \ifkorrekturansicht\clearpage\fi
}
\makeatother

\IfFileExists{\jobname-pw.ind}{\input{\jobname-pw.ind}}{}

% Quellenangabe nur in der Leseansicht
\ifkorrekturansicht\else
% Fallback-Definitionen, falls die .tex-Datei \titel etc. nicht gesetzt hat
\providecommand{\titel}{}
\providecommand{\editorInnen}{}
\providecommand{\dateiname}{\jobname}

\vspace{3cm}

\vfill

\footnotesize
\textsc{Quelle}: \titel. Herausgegeben von {\editorInnen}. In: \emph{Arthur Schnitzler: Briefwechsel mit Autorinnen und Autoren}.
 Digitale Edition, https://schnitzler-briefe.acdh.oeaw.ac.at/{\dateiname}.html (Stand \today)
\fi

\end{document}


