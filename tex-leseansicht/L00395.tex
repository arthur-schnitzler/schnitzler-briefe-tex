%% latex-korrekturansicht-vorspann.tex
%% Vorspann für die Korrekturansicht.
%% Lädt die gemeinsame Datei latex-vorspann.tex mit gesetztem Schalter.

\newif\ifkorrekturansicht
\korrekturansichttrue

\input{../tex-inputs/latex-vorspann}


\section[Max Burckhard an Arthur Schnitzler, {[}31. 10. 1894{]}]{L00395 Max Burckhard an Arthur Schnitzler, {[}31. 10. 1894{]}}
\nopagebreak\mylabel{L00395v}
\rehead{ }\normalsize\beginnumbering\briefempfaengerindex{Schnitzler, Arthur@\textsc{Schnitzler, Arthur}!zzzBurckhard, Max Eugen@\emph{von Max Eugen Burckhard}!1894-10-311@{{[}31. 10. 1894{]}}|(be}
\toendnotes[C]{\smallbreak\pagebreak[2]}\Standort{CUL, Schnitzler, B 20.}
\physDesc{Telegramm, 271 Zeichen
\newline{}maschinell
\newline{}Schnitzler: mit Bleistift datiert: »31/X 94« 
\newline{}Ordnung: 1) beschnitten  2) von unbekannter Hand nummeriert: »2«}\toendnotes[C]{\smallbreak}
\pstart
           \noindent{}{\pb}gratuliere herzlichst. stueck\pwindex{Liebelei. Schauspiel in drei Akten@\emph{Liebelei. Schauspiel in drei Akten}|pwv} hat tiefen eindruck auf
               mich gemacht. lebenswahr und poetische wirkung freilich in anderem sinne als dem
               gewoehnlichen zugleich. werde wegen factischem vorgehen puncto censur muendlich
               naeheres besprechen. herzliche empfehlung \spacefill\mbox{doctor \label{T_L00395-1v}\edtext{burckhard.+}{\lemma{\textnormal{\emph{burckhard.+}}}\Cendnote{\textnormal{Satzfehler: »burchhard«}}}\label{T_L00395-1}}\pend
           \selectlanguage{ngerman}\endnumbering\briefempfaengerindex{Schnitzler, Arthur@\textsc{Schnitzler, Arthur}!zzzBurckhard, Max Eugen@\emph{von Max Eugen Burckhard}!1894-10-311@{{[}31. 10. 1894{]}}|)be}\mylabel{L00395h}  \normalsize

\doendnotes{C}
\bigskip
\vfill

\clearpage

\footnotesize

\lohead{\textsc{register}}

% Definiere theindex-Environment komplett neu ohne reledmac
\makeatletter
\renewenvironment{theindex}{%
  \section*{\indexname}%
  \setlength{\parindent}{0pt}%
  \setlength{\parskip}{0pt plus 0.3pt}%
  \let\item\@idxitem
}{%
  \clearpage
}
\makeatother

\IfFileExists{\jobname-pw.ind}{\input{\jobname-pw.ind}}{}

\end{document}

      