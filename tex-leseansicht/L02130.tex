%% latex-leseansicht-vorspann.tex
%% Vorspann für die Leseansicht.
%% Lädt die gemeinsame Datei latex-vorspann.tex mit nicht gesetztem Schalter.

\newif\ifkorrekturansicht
\korrekturansichtfalse

\input{../tex-inputs/latex-vorspann}


               \section[Hermann Bahr an Arthur Schnitzler, 23. 4. 1913]{ Hermann Bahr an Arthur Schnitzler, 23. 4. 1913}\nopagebreak\mylabel{v}\rehead{ }\begin{ledgroupsized}[t]{13cm}\normalsize\beginnumbering\briefempfaengerindex{Schnitzler, Arthur@\textsc{Schnitzler, Arthur}!zzzBahr, Hermann@\emph{von Hermann Bahr}!1913-04-231@{23. 4. 1913}|(be} \toendnotes[C]{\smallbreak\pagebreak[2]} \Standort{CUL, Schnitzler, B 5b.}
\physDesc{Brief, 1 Blatt, 2 Seiten
\newline{}Handschrift: schwarze Tinte, deutsche Kurrent
\newline{}Schnitzler: mit Bleistift ergänzt »Bahr« \newline{}Ordnung: mit Bleistift von unbekannter Hand nummeriert:
                                    »177« }\buchAbdrucke{\weitereDrucke{Hermann Bahr, Arthur Schnitzler: \emph{Briefwechsel, Aufzeichnungen, Dokumente (1891–1931)}. Hg. Kurt Ifkovits und Martin Anton Müller. Göttingen: \emph{Wallstein} 2018, S. 485.} }\toendnotes[C]{\smallbreak}\pstart
           \raggedleft{}{\pb}23. 4. 13\pend
           \pstart{}Lieber Arthur,\pend\pstart
           herzlichen Dank! Ich bin ſehr froh, den armen Peter\pwindex{Altenberg, Peter 09.03.1859 – 08.01.1919@\textsc{Altenberg, Peter} (09.03.1859 – 08.01.1919), \emph{Schriftsteller}|pw} bald wieder »draußen« zu wiſſen.\pend
           \pstart
           Und nun noch was. Ich \label{K_L02130_1v}\edtext{ſchrieb Dir im
                  Dezember}{\lemma{\textnormal{\emph{ſchrieb Dir im
                  Dezember}}}\Cendnote{\textnormal{Hermann Bahr an Arthur Schnitzler, 7. 12. 1912}}}\label{K_L02130_1h}, daß ich keine Luſt habe, Geld für ihn herzugeben. Ich glaube nemlich
               beſtimmt zu wiſſen, daß er es nicht braucht und daß ich es alſo beſſer verwenden
               kann. Sollteſt Du aber einmal den Eindruck haben, daß es \uline{notwendig} iſt, ſo bitte ſchreib mir das, da geb ich natürlich gleich, was
               ich entbehren kann. Aber bitte {\pb}dies ganz unter
               uns.\pend
           \pstart
           Ich erfuhr jetzt erſt, daß Du einem \label{K_L02130_2v}\edtext{»Comité« für meinen 50. Geburtstag}{\lemma{\textnormal{\emph{»Comité« … Geburtstag}}}\Cendnote{\textnormal{vgl. \emph{Briefwechsel}
                  Bahr/Schnitzler 483.}}}\label{K_L02130_2h} uſw. Ich danke Dir dafür ſehr.\pend
           \pstart
           Zur \label{K_L02130_3v}\edtext{»Götterdämmerung\pwindex{Wagner, Richard 22.05.1813 – 13.02.1883@\textsc{Wagner, Richard} (22.05.1813 – 13.02.1883), \emph{Komponist}!Goetterdaemmerung1876@\strich\emph{Götterdämmerung} {[}1876{]}|pw}« war ich neulich in Wien\oindex{Wien@\textbf{Wien}|pw},
               komme wol zum »Triſtan\pwindex{Wagner, Richard 22.05.1813 – 13.02.1883@\textsc{Wagner, Richard} (22.05.1813 – 13.02.1883), \emph{Komponist}!Tristan und Isolde1865@\strich\emph{Tristan und Isolde} {[}1865{]}|pw}« wieder}{\lemma{\textnormal{\emph{»Götterdämmerung« … wieder}}}\Cendnote{\textnormal{ Die Hofoper\oindex{Oper@\textbf{Oper}|pwk} gab Wagner\pwindex{Wagner, Richard 22.05.1813 – 13.02.1883@\textsc{Wagner, Richard} (22.05.1813 – 13.02.1883), \emph{Komponist}|pwk}s \emph{Götterdämmerung}\pwindex{Wagner, Richard 22.05.1813 – 13.02.1883@\textsc{Wagner, Richard} (22.05.1813 – 13.02.1883), \emph{Komponist}!Goetterdaemmerung1876@\strich\emph{Götterdämmerung} {[}1876{]}|pwk} am 13. 4. 1913, \emph{Tristan und Isolde}\pwindex{Wagner, Richard 22.05.1813 – 13.02.1883@\textsc{Wagner, Richard} (22.05.1813 – 13.02.1883), \emph{Komponist}!Tristan und Isolde1865@\strich\emph{Tristan und Isolde} {[}1865{]}|pwk} am 5. 5. 1913, beide Male mit
                     Anna Bahr-Mildenburg\pwindex{Bahr-Mildenburg, Anna 29.11.1872 – 27.01.1947@\textsc{Bahr-Mildenburg, Anna} (29.11.1872 – 27.01.1947), \emph{Sängerin}|pwk}.}}}\label{K_L02130_3h}, aber immer
               knapp zur Vorſtellung und nachher in aller Früh wieder weg, denn ich
                  bi\textcolor{gray}{n} mitten in einem neuen \label{K_L02130_4v}\edtext{Stück\pwindex{Bahr, Hermann 19.07.1863 – 15.01.1934@\textsc{Bahr, Hermann} (19.07.1863 – 15.01.1934), \emph{Schriftsteller, Kritiker}!Phantom1913@\strich\emph{Das Phantom} {[}1913{]}|pwv}}{\lemma{\textnormal{\emph{Stück}}}\Cendnote{\textnormal{\emph{Das Phantom}\pwindex{Bahr, Hermann 19.07.1863 – 15.01.1934@\textsc{Bahr, Hermann} (19.07.1863 – 15.01.1934), \emph{Schriftsteller, Kritiker}!Phantom1913@\strich\emph{Das Phantom} {[}1913{]}|pwk} (Komödie in drei Akten. Mit
                     Dekorationsskizzen von Koloman Moser\pwindex{Moser, Koloman 1868-03-30 – 1918-10-18@\textsc{Moser, Koloman} (1868-03-30 – 1918-10-18), \emph{Maler, Grafiker}|pwk}.
                     Berlin: \emph{S. Fischer}\orgindex{S. Fischer Verlag@S. Fischer Verlag|pwk}{ }1913).}}}\label{K_L02130_4h}. Aber, wohin Du ſommers auch gehſt, Du kommſt doch über Salzburg\oindex{Salzburg@\textbf{Salzburg}|pw} und wir freuen uns Beide\pwindex{Bahr-Mildenburg, Anna 29.11.1872 – 27.01.1947@\textsc{Bahr-Mildenburg, Anna} (29.11.1872 – 27.01.1947), \emph{Sängerin}|pwv}{ }ſehr, ſehr, ſehr darauf, Euch\pwindex{Schnitzler, Olga 17.01.1882 – 13.01.1970@\textsc{Schnitzler, Olga} (17.01.1882 – 13.01.1970), \emph{Schauspielerin, Sängerin}|pwv} dann hier zu haben und einmal ausgiebig
               mit Euch\pwindex{Schnitzler, Olga 17.01.1882 – 13.01.1970@\textsc{Schnitzler, Olga} (17.01.1882 – 13.01.1970), \emph{Schauspielerin, Sängerin}|pwv} zuſammen zu ſein.\pend
           \pstart
           \label{K_L02130_5v}\edtext{Immer derſelbe}{\lemma{\textnormal{\emph{Immer derſelbe}}}\Cendnote{\textnormal{Hier lässt sich eine Verbindung zu einem
                  zentralen Motto Bahrs\pwindex{Bahr, Hermann 19.07.1863 – 15.01.1934@\textsc{Bahr, Hermann} (19.07.1863 – 15.01.1934), \emph{Schriftsteller, Kritiker}|pwk} herstellen, das er
                     1911{ }so begründete: »In ein Stammbuch schrieb
                     einer stolz: Immer derselbe! Ich darunter keck: Niemals derselbe! Spät erst
                     ging mir auf, das Rechte wäre wohl Beides: Niemals derselbe und eben darin doch
                     immer derselbe zu sein!« (\emph{[Stammbuch-Spruch]}\pwindex{Bahr, Hermann 19.07.1863 – 15.01.1934@\textsc{Bahr, Hermann} (19.07.1863 – 15.01.1934), \emph{Schriftsteller, Kritiker}!Stammbuch-Spruch]1910@\strich\emph{[Stammbuch-Spruch]} {[}1910{]}|pwk} In: \emph{Musen-Almanach 1911.} Berlin: \emph{Verein Berliner
                        Presse} 1910, S. 39) Im Jahr darauf knüpfte er im Text \emph{Selbstinventur}\pwindex{Bahr, Hermann 19.07.1863 – 15.01.1934@\textsc{Bahr, Hermann} (19.07.1863 – 15.01.1934), \emph{Schriftsteller, Kritiker}!Selbstinventur01. 09. 1912@\strich\emph{Selbstinventur} {[}01. 09. 1912{]}|pwk} (\emph{Die neue Rundschau}\pwindex{neue Rundschau1904@\emph{Die neue Rundschau}|pwk}, Jg. 23, H. 9,
                     S. 1287–1303) längere Überlegungen daran an.}}}\label{K_L02130_5h}{\\[\baselineskip]}\spacefill\mbox{Hermann}\pend
           \leftskip=0em{}\endnumbering\briefempfaengerindex{Schnitzler, Arthur@\textsc{Schnitzler, Arthur}!zzzBahr, Hermann@\emph{von Hermann Bahr}!1913-04-231@{23. 4. 1913}|)be}\mylabel{h}\end{ledgroupsized}  \newcommand{\dateiname}{L02130}\newcommand{\titel}{Hermann Bahr an Arthur Schnitzler, 23. 4. 1913}\newcommand{\editorInnen}{ Kurt Ifkovits,  Martin Anton Müller}%% latex-leseansicht-abspann.tex
%% Abspann für die Leseansicht.
%% Der Schalter \ifkorrekturansicht ist bereits durch den Vorspann gesetzt.

%% latex-abspann.tex
%% Gemeinsamer Abspann für Korrekturansicht und Leseansicht.
%% Setzt den Schalter \ifkorrekturansicht voraus (gesetzt in den
%% einbindenden Dateien latex-korrekturansicht-abspann.tex bzw.
%% latex-leseansicht-abspann.tex).
%% ---------------------------------------------------------------

\normalsize

% Das esempio-Environment wird nur in der Leseansicht benötigt
\ifkorrekturansicht\else
\newenvironment{esempio}[3]%
{
    \vspace{1.5ex}
    \rlap{\underline{#1}}
    \par
    \setlength{\parindent}{0cm}
    \nopagebreak
    \leftskip=#2cm
    \rightskip=#3cm
}
{
    \par
}
\fi

\doendnotes{C}
\bigskip
\vfill

\clearpage

\footnotesize

\ifkorrekturansicht
  \lohead{\textsc{register}}
\fi

% theindex-Environment neu definieren ohne reledmac
\makeatletter
\renewenvironment{theindex}{%
  \ifkorrekturansicht
    \section*{\indexname}%
  \else
    \subsubsection*{Index der erwähnten Entitäten}%
  \fi
  \setlength{\parindent}{0pt}%
  \setlength{\parskip}{0pt plus 0.3pt}%
  \let\item\@idxitem
}{%
  \ifkorrekturansicht\clearpage\fi
}
\makeatother

\IfFileExists{\jobname-pw.ind}{\input{\jobname-pw.ind}}{}

% Quellenangabe nur in der Leseansicht
\ifkorrekturansicht\else
% Fallback-Definitionen, falls die .tex-Datei \titel etc. nicht gesetzt hat
\providecommand{\titel}{}
\providecommand{\editorInnen}{}
\providecommand{\dateiname}{\jobname}

\vspace{3cm}

\vfill

\footnotesize
\textsc{Quelle}: \titel. Herausgegeben von {\editorInnen}. In: \emph{Arthur Schnitzler: Briefwechsel mit Autorinnen und Autoren}.
 Digitale Edition, https://schnitzler-briefe.acdh.oeaw.ac.at/{\dateiname}.html (Stand \today)
\fi

\end{document}


      