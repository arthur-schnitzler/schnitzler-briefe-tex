%% latex-leseansicht-vorspann.tex
%% Vorspann für die Leseansicht.
%% Lädt die gemeinsame Datei latex-vorspann.tex mit nicht gesetztem Schalter.

\newif\ifkorrekturansicht
\korrekturansichtfalse

\input{../tex-inputs/latex-vorspann}


\section[Hermann Bahr an Arthur Schnitzler, 23. 4. 1913]{L02130 Hermann Bahr an Arthur Schnitzler, 23. 4. 1913}
\nopagebreak\mylabel{L02130v}
\rehead{ }\normalsize\beginnumbering\briefempfaengerindex{Schnitzler, Arthur@\textsc{Schnitzler, Arthur}!zzzBahr, Hermann@\emph{von Hermann Bahr}!1913-04-231@{23. 4. 1913}|(be}
\toendnotes[C]{\smallbreak\pagebreak[2]}
\correspDesc{Versand  durch Hermann Bahr am 23. 4. 1913 in Salzburg
\newline{}Erhalt  durch Arthur Schnitzler im Zeitraum [24. 4. 1913
                  – 28. 4. 1913?] in Wien}\toendnotes[C]{\smallbreak}
\Standort{CUL, Schnitzler, B 5b.}
\physDesc{Brief, 1 Blatt, 2 Seiten, 966 Zeichen
\newline{}Handschrift: schwarze Tinte, deutsche Kurrent
\newline{}Schnitzler: mit Bleistift ergänzt »Bahr« 
\newline{}Ordnung: mit Bleistift von unbekannter Hand nummeriert: »177« }
\buchAbdrucke{\weitereDrucke{Hermann Bahr, Arthur Schnitzler: \emph{Briefwechsel, Aufzeichnungen, Dokumente (1891–1931)}. Herausgegeben von Kurt Ifkovits und Martin Anton Müller. Göttingen: \emph{Wallstein} 2018, S. 485.} }\toendnotes[C]{\smallbreak}
\pstart
           \raggedleft{}{\pb}23. 4. 13\pend
           
\pstart{}Lieber Arthur,\pend\vspace{0.5em}
\pstart
           herzlichen Dank! Ich bin{ }ſehr froh, den armen Peter\pwindex{Altenberg, Peter 9.\,3.\,1859 Wien – 8.\,1.\,1919 ebd.@\textsc{Altenberg, Peter} (9.\,3.\,1859 Wien – 8.\,1.\,1919 ebd.), \emph{Schriftsteller}|pw} bald wieder »draußen« zu wiſſen.\pend
           
\pstart
           Und nun noch was. Ich \label{K_L02130-1v}\edtext{ſchrieb Dir im
                  Dezember}{\lemma{\textnormal{\emph{schrieb Dir im
                  Dezember}}}\Cendnote{\textnormal{XXXX Auszeichnungsfehler: Dokument L02107 nicht gefunden. }}}\label{K_L02130-1}, daß ich keine Luſt habe,
               Geld für ihn herzugeben. Ich glaube nemlich beſtimmt zu wiſſen, daß er es nicht
               braucht und daß ich es alſo beſſer verwenden kann. Sollteſt Du aber einmal den
               Eindruck haben, daß es \uline{notwendig} iſt,{ }ſo bitte{ }ſchreib mir das, da geb ich natürlich gleich, was ich entbehren kann. Aber bitte {\pb}dies ganz unter uns.\pend
           
\pstart
           Ich erfuhr jetzt erſt, daß Du einem \label{K_L02130-2v}\edtext{»Comité« für meinen 50. Geburtstag}{\lemma{\textnormal{\emph{»Comité« … Geburtstag}}}\Cendnote{\textnormal{Siehe Hermann Bahr, Arthur Schnitzler: \emph{Briefwechsel, Aufzeichnungen, Dokumente (1891–1931)}, [Aufruf für Hermann Bahr], Fremden-Blatt, 22. 4. 1913.}}}\label{K_L02130-2} uſw. Ich danke Dir dafür{ }ſehr.\pend
           
\pstart
           Zur \label{K_L02130-3v}\edtext{»Götterdämmerung\pwindex{Wagner, Richard 22.\,5.\,1813 Leipzig – 13.\,2.\,1883 Venedig@\textsc{Wagner, Richard} (22.\,5.\,1813 Leipzig – 13.\,2.\,1883 Venedig), \emph{Komponist}!Ring des Nibelungen. Dritter Tag: Götterdämmerung@\strich\emph{Der Ring des Nibelungen. Dritter Tag: Götterdämmerung}|pw}« war ich neulich in Wien\oindex{Wien@\textbf{Wien}, \emph{Verwaltungsgebiet}|pw},
               komme wol zum »Triſtan\pwindex{Wagner, Richard 22.\,5.\,1813 Leipzig – 13.\,2.\,1883 Venedig@\textsc{Wagner, Richard} (22.\,5.\,1813 Leipzig – 13.\,2.\,1883 Venedig), \emph{Komponist}!Tristan und Isolde@\strich\emph{Tristan und Isolde}|pw}« wieder}{\lemma{\textnormal{\emph{»Götterdämmerung« … wieder}}}\Cendnote{\textnormal{ Die Hofoper\oindex{Wien@\textbf{Wien}!I., Innere Stadt@\textbf{I., Innere Stadt}!Oper@\textbf{Oper}, \emph{Oper}|pwk} gab Wagners\pwindex{Wagner, Richard 22.\,5.\,1813 Leipzig – 13.\,2.\,1883 Venedig@\textsc{Wagner, Richard} (22.\,5.\,1813 Leipzig – 13.\,2.\,1883 Venedig), \emph{Komponist}|pwk}{ }\emph{Götterdämmerung}\pwindex{Wagner, Richard 22.\,5.\,1813 Leipzig – 13.\,2.\,1883 Venedig@\textsc{Wagner, Richard} (22.\,5.\,1813 Leipzig – 13.\,2.\,1883 Venedig), \emph{Komponist}!Ring des Nibelungen. Dritter Tag: Götterdämmerung@\strich\emph{Der Ring des Nibelungen. Dritter Tag: Götterdämmerung}|pwk} am 13. 4. 1913,
                     \emph{Tristan und Isolde}\pwindex{Wagner, Richard 22.\,5.\,1813 Leipzig – 13.\,2.\,1883 Venedig@\textsc{Wagner, Richard} (22.\,5.\,1813 Leipzig – 13.\,2.\,1883 Venedig), \emph{Komponist}!Tristan und Isolde@\strich\emph{Tristan und Isolde}|pwk} am
                     5. 5. 1913, beide Male mit Anna
                     Bahr-Mildenburg\pwindex{Bahr-Mildenburg, Anna 29.\,11.\,1872 Wien – 27.\,1.\,1947 ebd.@\textsc{Bahr-Mildenburg, Anna} (29.\,11.\,1872 Wien – 27.\,1.\,1947 ebd.), \emph{Sängerin}|pwk}.}}}\label{K_L02130-3}, aber immer knapp zur Vorſtellung und nachher in
               aller Früh wieder weg, denn ich bi\textcolor{gray}{n} mitten in einem neuen \label{K_L02130-4v}\edtext{Stück\pwindex{Bahr, Hermann 19.\,7.\,1863 Linz – 15.\,1.\,1934 München@\textsc{Bahr, Hermann} (19.\,7.\,1863 Linz – 15.\,1.\,1934 München), \emph{Schriftsteller, Kritiker}!Phantom@\strich\emph{Das Phantom}|pwv}}{\lemma{\textnormal{\emph{Stück}}}\Cendnote{\textnormal{\emph{Das Phantom}\pwindex{Bahr, Hermann 19.\,7.\,1863 Linz – 15.\,1.\,1934 München@\textsc{Bahr, Hermann} (19.\,7.\,1863 Linz – 15.\,1.\,1934 München), \emph{Schriftsteller, Kritiker}!Phantom@\strich\emph{Das Phantom}|pwk} (Komödie in drei Akten. Mit
                     Dekorationsskizzen von Koloman Moser\pwindex{Moser, Koloman 30.\,3.\,1868 Wien – 18.\,10.\,1918 ebd.@\textsc{Moser, Koloman} (30.\,3.\,1868 Wien – 18.\,10.\,1918 ebd.), \emph{Maler, Grafiker}|pwk}.
                     Berlin: \emph{S. Fischer}\orgindex{S. Fischer Verlag@S. Fischer Verlag|pwk}{ }1913).}}}\label{K_L02130-4}. Aber, wohin Du{ }ſommers auch gehſt, Du kommſt doch über Salzburg\oindex{Salzburg@\textbf{Salzburg}, \emph{Verwaltungsgebiet}|pw} und wir freuen uns Beide\pwindex{Bahr-Mildenburg, Anna 29.\,11.\,1872 Wien – 27.\,1.\,1947 ebd.@\textsc{Bahr-Mildenburg, Anna} (29.\,11.\,1872 Wien – 27.\,1.\,1947 ebd.), \emph{Sängerin}|pwv}{ }ſehr,{ }ſehr,{ }ſehr darauf, Euch\pwindex{Schnitzler, Olga 17.\,1.\,1882 Wien – 13.\,1.\,1970 Lugano@\textsc{Schnitzler, Olga} (17.\,1.\,1882 Wien – 13.\,1.\,1970 Lugano), \emph{Schauspielerin, Sängerin}|pwv} dann hier zu haben und einmal ausgiebig
               mit Euch\pwindex{Schnitzler, Olga 17.\,1.\,1882 Wien – 13.\,1.\,1970 Lugano@\textsc{Schnitzler, Olga} (17.\,1.\,1882 Wien – 13.\,1.\,1970 Lugano), \emph{Schauspielerin, Sängerin}|pwv} zuſammen zu{ }ſein.\pend
           
\pstart
           \label{K_L02130-5v}\edtext{Immer derſelbe}{\lemma{\textnormal{\emph{Immer derselbe}}}\Cendnote{\textnormal{Hier lässt sich eine Verbindung zu einem
                  zentralen Motto Bahrs\pwindex{Bahr, Hermann 19.\,7.\,1863 Linz – 15.\,1.\,1934 München@\textsc{Bahr, Hermann} (19.\,7.\,1863 Linz – 15.\,1.\,1934 München), \emph{Schriftsteller, Kritiker}|pwk} herstellen, das er
                     1911{ }so begründete: »In ein Stammbuch schrieb
                     einer stolz: Immer derselbe! Ich darunter keck: Niemals derselbe! Spät erst
                     ging mir auf, das Rechte wäre wohl Beides: Niemals derselbe und eben darin doch
                     immer derselbe zu sein!« (\emph{[Stammbuch-Spruch]}\pwindex{Bahr, Hermann 19.\,7.\,1863 Linz – 15.\,1.\,1934 München@\textsc{Bahr, Hermann} (19.\,7.\,1863 Linz – 15.\,1.\,1934 München), \emph{Schriftsteller, Kritiker}!Stammbuch-Spruch]@\strich\emph{[Stammbuch-Spruch]}|pwk} In: \emph{Musen-Almanach 1911.} Berlin: \emph{Verein Berliner
                        Presse} 1910, S. 39) Im Jahr darauf knüpfte er im Text \emph{Selbstinventur}\pwindex{Bahr, Hermann 19.\,7.\,1863 Linz – 15.\,1.\,1934 München@\textsc{Bahr, Hermann} (19.\,7.\,1863 Linz – 15.\,1.\,1934 München), \emph{Schriftsteller, Kritiker}!Selbstinventur@\strich\emph{Selbstinventur}|pwk} (\emph{Die neue Rundschau}\pwindex{neue Rundschau@\emph{Die neue Rundschau}|pwk}, Jg. 23, H. 9,
                     S. 1287–1303) längere Überlegungen daran an.}}}\label{K_L02130-5}{\\[\baselineskip]}\spacefill\mbox{Hermann}\pend
           \leftskip=0em{}\selectlanguage{ngerman}\endnumbering\briefempfaengerindex{Schnitzler, Arthur@\textsc{Schnitzler, Arthur}!zzzBahr, Hermann@\emph{von Hermann Bahr}!1913-04-231@{23. 4. 1913}|)be}\mylabel{L02130h}  \newcommand{\dateiname}{L02130}\newcommand{\titel}{Hermann Bahr an Arthur Schnitzler, 23. 4. 1913}\newcommand{\editorInnen}{Herausgegeben von Martin Anton Müller}%% latex-leseansicht-abspann.tex
%% Abspann für die Leseansicht.
%% Der Schalter \ifkorrekturansicht ist bereits durch den Vorspann gesetzt.

%% latex-abspann.tex
%% Gemeinsamer Abspann für Korrekturansicht und Leseansicht.
%% Setzt den Schalter \ifkorrekturansicht voraus (gesetzt in den
%% einbindenden Dateien latex-korrekturansicht-abspann.tex bzw.
%% latex-leseansicht-abspann.tex).
%% ---------------------------------------------------------------

\normalsize

% Das esempio-Environment wird nur in der Leseansicht benötigt
\ifkorrekturansicht\else
\newenvironment{esempio}[3]%
{
    \vspace{1.5ex}
    \rlap{\underline{#1}}
    \par
    \setlength{\parindent}{0cm}
    \nopagebreak
    \leftskip=#2cm
    \rightskip=#3cm
}
{
    \par
}
\fi

\doendnotes{C}
\bigskip
\vfill

\clearpage

\footnotesize

\ifkorrekturansicht
  \lohead{\textsc{register}}
\fi

% theindex-Environment neu definieren ohne reledmac
\makeatletter
\renewenvironment{theindex}{%
  \ifkorrekturansicht
    \section*{\indexname}%
  \else
    \subsubsection*{Index der erwähnten Entitäten}%
  \fi
  \setlength{\parindent}{0pt}%
  \setlength{\parskip}{0pt plus 0.3pt}%
  \let\item\@idxitem
}{%
  \ifkorrekturansicht\clearpage\fi
}
\makeatother

\IfFileExists{\jobname-pw.ind}{\input{\jobname-pw.ind}}{}

% Quellenangabe nur in der Leseansicht
\ifkorrekturansicht\else
% Fallback-Definitionen, falls die .tex-Datei \titel etc. nicht gesetzt hat
\providecommand{\titel}{}
\providecommand{\editorInnen}{}
\providecommand{\dateiname}{\jobname}

\vspace{3cm}

\vfill

\footnotesize
\textsc{Quelle}: \titel. Herausgegeben von {\editorInnen}. In: \emph{Arthur Schnitzler: Briefwechsel mit Autorinnen und Autoren}.
 Digitale Edition, https://schnitzler-briefe.acdh.oeaw.ac.at/{\dateiname}.html (Stand \today)
\fi

\end{document}


