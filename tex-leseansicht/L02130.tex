%% latex-korrekturansicht-vorspann.tex
%% Vorspann für die Korrekturansicht.
%% Lädt die gemeinsame Datei latex-vorspann.tex mit gesetztem Schalter.

\newif\ifkorrekturansicht
\korrekturansichttrue

\input{../tex-inputs/latex-vorspann}


\section[Hermann Bahr an Arthur Schnitzler, 23. 4. 1913]{L02130 Hermann Bahr an Arthur Schnitzler, 23. 4. 1913}
\nopagebreak\mylabel{L02130v}
\rehead{ }\normalsize\beginnumbering\briefempfaengerindex{Schnitzler, Arthur@\textsc{Schnitzler, Arthur}!zzzBahr, Hermann@\emph{von Hermann Bahr}!1913-04-231@{23. 4. 1913}|(be}
\toendnotes[C]{\smallbreak\pagebreak[2]}\Standort{CUL, Schnitzler, B 5b.}
\physDesc{Brief, 1 Blatt, 2 Seiten, 966 Zeichen
\newline{}Handschrift: schwarze Tinte, deutsche Kurrent
\newline{}Schnitzler: mit Bleistift ergänzt »Bahr« 
\newline{}Ordnung: mit Bleistift von unbekannter Hand nummeriert: »177« }
\buchAbdrucke{\weitereDrucke{Hermann Bahr, Arthur Schnitzler: \emph{Briefwechsel, Aufzeichnungen, Dokumente (1891–1931)}. Göttingen: \emph{Wallstein} 2018, S. 485.} }\toendnotes[C]{\smallbreak}
\pstart
           \raggedleft{}{\pb}23. 4. 13\pend
           
\pstart{}Lieber Arthur,\pend\vspace{0.5em}
\pstart
           herzlichen Dank! Ich bin ſehr froh, den armen Peter\pwindex{Altenberg, Peter 09.03.1859 – 08.01.1919@\textsc{Altenberg, Peter} (09.03.1859 – 08.01.1919), \emph{Schriftsteller/Schriftstellerin}|pw} bald wieder »draußen« zu wiſſen.\pend
           
\pstart
           Und nun noch was. Ich \label{K_L02130-1v}\edtext{ſchrieb Dir im
                  Dezember}{\lemma{\textnormal{\emph{ſchrieb Dir im
                  Dezember}}}\Cendnote{\textnormal{Hermann Bahr an Arthur Schnitzler, 7. 12. 1912. }}}\label{K_L02130-1}, daß ich keine Luſt habe,
               Geld für ihn herzugeben. Ich glaube nemlich beſtimmt zu wiſſen, daß er es nicht
               braucht und daß ich es alſo beſſer verwenden kann. Sollteſt Du aber einmal den
               Eindruck haben, daß es \uline{notwendig} iſt, ſo bitte
               ſchreib mir das, da geb ich natürlich gleich, was ich entbehren kann. Aber bitte {\pb}dies ganz unter uns.\pend
           
\pstart
           Ich erfuhr jetzt erſt, daß Du einem \label{K_L02130-2v}\edtext{»Comité« für meinen 50. Geburtstag}{\lemma{\textnormal{\emph{»Comité« … Geburtstag}}}\Cendnote{\textnormal{Siehe Hermann Bahr, Arthur Schnitzler: \emph{Briefwechsel, Aufzeichnungen, Dokumente (1891–1931)}, [Aufruf für Hermann Bahr], Fremden-Blatt, 22. 4. 1913.}}}\label{K_L02130-2} uſw. Ich danke Dir dafür ſehr.\pend
           
\pstart
           Zur \label{K_L02130-3v}\edtext{»Götterdämmerung\pwindex{Goetterdaemmerung@\emph{Götterdämmerung}|pw}« war ich neulich in Wien\oindex{Wien@\textbf{Wien}, \emph{A.ADM2}|pw},
               komme wol zum »Triſtan\pwindex{Tristan und Isolde@\emph{Tristan und Isolde}|pw}« wieder}{\lemma{\textnormal{\emph{»Götterdämmerung« … wieder}}}\Cendnote{\textnormal{ Die Hofoper\oindex{Oper@\textbf{Oper}, \emph{Oper (K.OPR)}|pwk} gab Wagners\pwindex{Wagner, Richard 22.05.1813 – 13.02.1883@\textsc{Wagner, Richard} (22.05.1813 – 13.02.1883), \emph{Komponist/Komponistin}|pwk}{ }\emph{Götterdämmerung}\pwindex{Goetterdaemmerung@\emph{Götterdämmerung}|pwk} am 13. 4. 1913,
                     \emph{Tristan und Isolde}\pwindex{Tristan und Isolde@\emph{Tristan und Isolde}|pwk} am
                     5. 5. 1913, beide Male mit Anna
                     Bahr-Mildenburg\pwindex{Bahr-Mildenburg, Anna 29.11.1872 – 27.01.1947@\textsc{Bahr-Mildenburg, Anna} (29.11.1872 – 27.01.1947), \emph{Sänger/Sängerin}|pwk}.}}}\label{K_L02130-3}, aber immer knapp zur Vorſtellung und nachher in
               aller Früh wieder weg, denn ich bi\textcolor{gray}{n} mitten in einem neuen \label{K_L02130-4v}\edtext{Stück\pwindex{Phantom@\emph{Das Phantom}|pwv}}{\lemma{\textnormal{\emph{Stück}}}\Cendnote{\textnormal{\emph{Das Phantom}\pwindex{Phantom@\emph{Das Phantom}|pwk} (Komödie in drei Akten. Mit
                     Dekorationsskizzen von Koloman Moser\pwindex{Moser, Koloman 1868-03-30 – 1918-10-18@\textsc{Moser, Koloman} (1868-03-30 – 1918-10-18), \emph{Maler/Malerin, Grafiker/Grafikerin}|pwk}.
                     Berlin: \emph{S. Fischer}\orgindex{S. Fischer Verlag@S. Fischer Verlag|pwk}{ }1913).}}}\label{K_L02130-4}. Aber, wohin Du ſommers auch gehſt, Du kommſt doch über Salzburg\oindex{Salzburg@\textbf{Salzburg}, \emph{A.ADM2}|pw} und wir freuen uns Beide\pwindex{Bahr-Mildenburg, Anna 29.11.1872 – 27.01.1947@\textsc{Bahr-Mildenburg, Anna} (29.11.1872 – 27.01.1947), \emph{Sänger/Sängerin}|pwv}{ }ſehr, ſehr, ſehr darauf, Euch\pwindex{Schnitzler, Olga 17.01.1882 – 13.01.1970@\textsc{Schnitzler, Olga} (17.01.1882 – 13.01.1970), \emph{Schauspieler/Schauspielerin, Sänger/Sängerin}|pwv} dann hier zu haben und einmal ausgiebig
               mit Euch\pwindex{Schnitzler, Olga 17.01.1882 – 13.01.1970@\textsc{Schnitzler, Olga} (17.01.1882 – 13.01.1970), \emph{Schauspieler/Schauspielerin, Sänger/Sängerin}|pwv} zuſammen zu
               ſein.\pend
           
\pstart
           \label{K_L02130-5v}\edtext{Immer derſelbe}{\lemma{\textnormal{\emph{Immer derſelbe}}}\Cendnote{\textnormal{Hier lässt sich eine Verbindung zu einem
                  zentralen Motto Bahrs\pwindex{Bahr, Hermann 19.07.1863 – 15.01.1934@\textsc{Bahr, Hermann} (19.07.1863 – 15.01.1934), \emph{Schriftsteller/Schriftstellerin, Kritiker/Kritikerin}|pwk} herstellen, das er
                     1911{ }so begründete: »In ein Stammbuch schrieb
                     einer stolz: Immer derselbe! Ich darunter keck: Niemals derselbe! Spät erst
                     ging mir auf, das Rechte wäre wohl Beides: Niemals derselbe und eben darin doch
                     immer derselbe zu sein!« (\emph{[Stammbuch-Spruch]}\pwindex{Stammbuch-Spruch]@\emph{[Stammbuch-Spruch]}|pwk} In: \emph{Musen-Almanach 1911.} Berlin: \emph{Verein Berliner
                        Presse} 1910, S. 39) Im Jahr darauf knüpfte er im Text \emph{Selbstinventur}\pwindex{Selbstinventur@\emph{Selbstinventur}|pwk} (\emph{Die neue Rundschau}\pwindex{neue Rundschau@\emph{Die neue Rundschau}|pwk}, Jg. 23, H. 9,
                     S. 1287–1303) längere Überlegungen daran an.}}}\label{K_L02130-5}{\\[\baselineskip]}\spacefill\mbox{Hermann}\pend
           \leftskip=0em{}\selectlanguage{ngerman}\endnumbering\briefempfaengerindex{Schnitzler, Arthur@\textsc{Schnitzler, Arthur}!zzzBahr, Hermann@\emph{von Hermann Bahr}!1913-04-231@{23. 4. 1913}|)be}\mylabel{L02130h}  \normalsize

\doendnotes{C}
\bigskip
\vfill

\clearpage

\footnotesize

\lohead{\textsc{register}}

% Definiere theindex-Environment komplett neu ohne reledmac
\makeatletter
\renewenvironment{theindex}{%
  \section*{\indexname}%
  \setlength{\parindent}{0pt}%
  \setlength{\parskip}{0pt plus 0.3pt}%
  \let\item\@idxitem
}{%
  \clearpage
}
\makeatother

\IfFileExists{\jobname-pw.ind}{\input{\jobname-pw.ind}}{}

\end{document}

      