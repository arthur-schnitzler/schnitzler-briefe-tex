%% latex-korrekturansicht-vorspann.tex
%% Vorspann für die Korrekturansicht.
%% Lädt die gemeinsame Datei latex-vorspann.tex mit gesetztem Schalter.

\newif\ifkorrekturansicht
\korrekturansichttrue

\input{../tex-inputs/latex-vorspann}


\section[Stefan Zweig an Arthur Schnitzler, 18. 5. 1927]{L03674 Stefan Zweig an Arthur Schnitzler, 18. 5. 1927}
\nopagebreak\mylabel{L03674v}
\rehead{ }\normalsize\beginnumbering\briefempfaengerindex{Schnitzler, Arthur@\textsc{Schnitzler, Arthur}!zzzZweig, Stefan@\emph{von Stefan Zweig}!1927-05-181@{18. 5. 1927}|(be}
\toendnotes[C]{\smallbreak\pagebreak[2]}\Standort{CUL, Schnitzler, B 118.}
\physDesc{Brief, 1 Blatt, 2 Seiten, 3020 Zeichen
\newline{}Handschrift: blaue Tinte, lateinische Kurrent
\newline{}Schnitzler: 1) mit Bleistift beschriftet: »\textsc{Zweig}«  2) mit rotem Buntstift 13 Unterstreichungen}
\buchAbdrucke{\weitereDrucke{1) Stefan Zweig: \emph{Briefwechsel mit Hermann Bahr, Sigmund Freud, Rainer Maria
                        Rilke und Arthur Schnitzler}. Frankfurt am Main: \emph{S. Fischer} 1987, S. 427–429.} \weitereDrucke{2) Stefan Zweig: \emph{Briefe. Bd. III: 1920–1931}. Frankfurt am Main: \emph{S. Fischer} 2000, S. 186–188.} }\toendnotes[C]{\smallbreak}
\pstart
           {\pb}\textcolor{gray}{\textbf{SZ}}\hfill 18. Mai 1927\pend
           
\pstart
           \raggedleft{}\textcolor{gray}{\textbf{SALZBURG\oindex{Salzburg@\textbf{Salzburg}, \emph{A.ADM2}|pw}}}\pend
           
\pstart
           \raggedleft{}\textcolor{gray}{\textbf{KAPUZINERBERG\oindex{Paschinger Schloessl@\textbf{Paschinger Schlössl}, \emph{Wohngebäude (K.WHS)}|pw}}}\pend
           \vspace{0.5em}
\pstart
           Lieber verehrter Herr Doktor, man sagt mir, irgend eine Zeitung
               hätte Ihren \label{K_L03674-1v}\edtext{fünfundsechzigsten
                  Geburtstag}{\lemma{\textnormal{\emph{fünfundsechzigsten Geburtstag}}}\Cendnote{\textnormal{Schnitzler wurde am 15. 5. 1927 65.
                  Mehrere Zeitungen berichteten.}}}\label{K_L03674-1} gemeldet: nun, den habe ich gründlich
               verschlafen und komme mit einem späten, darum nicht minder herzlichen Glückwunsch.
               Und sage Ihnen gleichzeitig Dank für Ihre ausserordentliche \label{K_L03674-2v}\edtext{Novelle\pwindex{Spiel im Morgengrauen. Novelle@\emph{Spiel im Morgengrauen. Novelle}|pwv}}{\lemma{\textnormal{\emph{Novelle}}}\Cendnote{\textnormal{\emph{Spiel im Morgengrauen}\pwindex{Spiel im Morgengrauen. Novelle@\emph{Spiel im Morgengrauen. Novelle}|pwk} erschien zwischen dem
                     5. 12. 1926 und dem 9. 1. 1927 in sechs
                  Fortsetzungen. Die Buchausgabe bei \emph{S. Fischer}\orgindex{S. Fischer Verlag@S. Fischer Verlag|pwk}
                  wurde am 9. 3. 1927 erstmals im \emph{Börsenblatt für den deutschen Buchhandel}\pwindex{Boersenblatt fuer den Deutschen Buchhandel@\emph{Börsenblatt für den Deutschen Buchhandel}|pwk} angekündigt und in den Tagen vor
                  dem 1. 5. 1927
                  ausgegeben. }}}\label{K_L03674-2}. Ich hatte seinerzeit die ersten drei Fortsetzungen in der B. I.\pwindex{Berliner Illustrirte Zeitung@\emph{Berliner Illustrirte Zeitung}|pw} gelesen, die andern versäumte ich und war
               schon \substVorne{}\textsuperscript{z}\substDazwischen{}–\substHinten{} besser und wichtiger! – ich \uline{blieb} gespannt
               und hatte so deutlich alle Figuren und Situationen im Gedächtnis, dass ich im Buche\pwindex{Spiel im Morgengrauen. Novelle@\emph{Spiel im Morgengrauen. Novelle}|pwv} gleich dort weiterlas, wo
               mir die Continuität genommen war. Ich erzähle Ihnen das, um Ihnen am lebendigen
               Obiect die Plastik Ihrer Figuren zu erweisen: ich hatte nicht einen Zug von ihnen
               vergessen, so scharfkantig waren sie in mein Gedächtnis eingeprägt. Ja, das ist
               wieder eine ausserordentliche Novelle\pwindex{Spiel im Morgengrauen. Novelle@\emph{Spiel im Morgengrauen. Novelle}|pwv}, geradling im Ablauf und doch kreisförmig rund, rein abgeschlossen
               und vier Menschen voll erfüllend, fünf eigentlich, denn auch der Consul\pwindex{Spiel im Morgengrauen. Novelle@\emph{Spiel im Morgengrauen. Novelle}|pwv}, den Sie bewusst abdunkeln, erfüllt
               sich als Character. Vorbildlich bleibt mir die Ruhe Ihres Erzählens erregter und
               erregender Zustände: ich fühle, wie viel mir da von Ihnen zu lernen nottut und ich
               schäme mich nicht, willig dies Vorbildliche Ihrer ruhig referierenden und dabei den
               Atem der andern festhaltenden Kunst einzugestehen. Möge Sie {\pb}dies grossartige Gelingen Ihrer epischen
               Ausdrucks entschäd\substVorne{}\textsuperscript{en}\substDazwischen{}igen\substHinten{} für die hässliche, mesquine und undankbare Art, mit der man gegen Ihre
               dramatische Production verfährt. Ich empfinde es als ein Unrecht gegen uns alle im
               Sinne der geistigen Gemeinschaft, dass Ihrem letzten \label{K_L03674-3v}\edtext{Stück\pwindex{Gang zum Weiher. Dramatische Dichtung@\emph{Der Gang zum Weiher. Dramatische Dichtung}|pwv} sich das würdige Theater}{\lemma{\textnormal{\emph{Stück … Theater}}}\Cendnote{\textnormal{Bereits Anfang 1926 war die
                     Buchausgabe von \emph{Der Gang zum Weiher}\pwindex{Gang zum Weiher. Dramatische Dichtung@\emph{Der Gang zum Weiher. Dramatische Dichtung}|pwk} erschienen, kein Theater wollte sich aber zur Uraufführung verpflichten. Diese
                     fand schließlich erst am 14. 2. 1931 am \emph{Burgtheater}\orgindex{Burgtheater@Burgtheater|pwk}
                     statt.}}}\label{K_L03674-3} nicht gefunden hat, dass der
               erbärmlichste französische Dreck meisterlich insceniert und interpretiert wird,
               indess man wagt, ein ed\substVorne{}\textsuperscript{les}\substDazwischen{}el\substHinten{} geformtes und geistig ergreifendes Werk von Ihnen so einfach zur Seite zu
               legen. Ich \substVorne{}\textsuperscript{empfinde}\substDazwischen{}spüre\substHinten{} diese Art Kränkung vehementer als eine mir selbst zugefügte.\pend
           
\pstart
           Sonderbar: in der Novelle\pwindex{Spiel im Morgengrauen. Novelle@\emph{Spiel im Morgengrauen. Novelle}|pwv}
               erhob sich mir jener Einwand, den ich bei Fräulein
                  Else\pwindex{Fraeulein Else@\emph{Fräulein Else}|pw} schon verspürt hatte. Sie scheinen mir, Sie, der im Leben so
               Bescheidene, in der Kunst verschwenderisch mit dem Gelde. Ich habe, obwohl aus
               reichem Hause, einen Tausendguldenschein \introOben{}bei meinem Vater\introOben{}
               nie gesehen und kann mir kaum ausdenken wo die Leopoldine\pwindex{Spiel im Morgengrauen. Novelle@\emph{Spiel im Morgengrauen. Novelle}|pwv}{ }\substVorne{}\textsuperscript{ihn}\substDazwischen{}diese schwer zu beschaffende Note so rasch\substHinten{} aufgetrieben hat. Mir wäre es tragischer erschienen, wenn ein amer Teufel
               von Leutnant schon um einer jämmerlichen Summe von 800 Gulden zu Grunde gienge.
               Elftausend, das war damals schon eine \introOben{}kleine\introOben{} Villa in Hietzing\oindex{XIII., Hietzing@\textbf{XIII., Hietzing}, \emph{A.ADM3}|pw}. Seien Sie nicht böse, dass ich \introOben{}auch\introOben{} auf solche Kleinigkeiten sehe: ich glaube nur rein
               technisch, dass es wichtig ist zu zeigen, wie im Leben oft an einem Hosenknopf ein
               Schicksal scheitert. Die grosse Summe steigert den Leichtsinn des Leutnants und
               entschuldigt das Zögern seiner Verwandten: ich hätte als junger Mensch bei all meinen
               reichen Verwandten um 1000 Gulden schon vergeblich gebeten. Dies wahrhaftig mein
               einziger Einwand inmitten leidenschaftlich dankbarer Zustimmung.\pend
           
\pstart
           Ihr getreu ergebener{\\[\baselineskip]}\spacefill\mbox{Stefan Zweig}\pend
           \leftskip=0em{}\selectlanguage{ngerman}\endnumbering\briefempfaengerindex{Schnitzler, Arthur@\textsc{Schnitzler, Arthur}!zzzZweig, Stefan@\emph{von Stefan Zweig}!1927-05-181@{18. 5. 1927}|)be}\mylabel{L03674h}
\begin{anhang}
\end{anhang}\normalsize

\doendnotes{C}
\bigskip
\vfill

\clearpage

\footnotesize

\lohead{\textsc{register}}

% Definiere theindex-Environment komplett neu ohne reledmac
\makeatletter
\renewenvironment{theindex}{%
  \section*{\indexname}%
  \setlength{\parindent}{0pt}%
  \setlength{\parskip}{0pt plus 0.3pt}%
  \let\item\@idxitem
}{%
  \clearpage
}
\makeatother

\IfFileExists{\jobname-pw.ind}{\input{\jobname-pw.ind}}{}

\end{document}

      