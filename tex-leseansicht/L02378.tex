%% latex-korrekturansicht-vorspann.tex
%% Vorspann für die Korrekturansicht.
%% Lädt die gemeinsame Datei latex-vorspann.tex mit gesetztem Schalter.

\newif\ifkorrekturansicht
\korrekturansichttrue

\input{../tex-inputs/latex-vorspann}


\section[Stefan Großmann an Arthur Schnitzler, 26. 4. 1922]{L02378 Stefan Großmann an Arthur Schnitzler, 26. 4. 1922}
\nopagebreak\mylabel{L02378v}
\rehead{ }\normalsize\beginnumbering\briefempfaengerindex{Schnitzler, Arthur@\textsc{Schnitzler, Arthur}!zzzGrossmann, Stefan@\emph{von Stefan Großmann}!1922-04-261@{26. 4. 1922}|(be}
\toendnotes[C]{\smallbreak\pagebreak[2]}\Standort{CUL, Schnitzler, B 34.}
\physDesc{Brief, 1 Blatt, 2 Seiten, 1034 Zeichen
\newline{}Schreibmaschine
\newline{}Handschrift: schwarze Tinte, deutsche Kurrent (\noindent{}Einfügung von »Doktor«, Schlussformel und
                                 Unterschrift)
\newline{}Schnitzler: 1) mit Bleistift beschriftet: »\textsc{Großman}« und »b. sein«  2) mit rotem Buntstift vier Unterstreichungen
\newline{}Ordnung: mit Bleistift von unbekannter Hand nummeriert:
                                    »16« }\toendnotes[C]{\smallbreak}
\pstart
           \centering{}{\pb}\textcolor{gray}{\textbf{Das Tage-Buch\orgindex{Tage-Buch@Das Tage-Buch|pw}}}\pend
           
\pstart
           \centering{}\textcolor{gray}{\textbf{Erscheint jeden Sonnabend ⋅ Herausgeber: Stefan Großmann}}\pend
           
\pstart
           \centering{}\textcolor{gray}{\textbf{Ernst Rowohlt Verlag\orgindex{Ernst Rowohlt Verlag@Ernst Rowohlt Verlag|pw} ⋅ Berlin W 35\oindex{Berlin@\textbf{Berlin}, \emph{P.PPLC}|pw}}}\pend
           
\pstart
           \centering{}\textcolor{gray}{\textbf{POTSDAMER STRASSE 123\textsuperscript{B} ⋅ AN DER POTSDAMER BRÜCKE\oindex{Potsdamer Strasse@\textbf{Potsdamer Straße}, \emph{Straße (K.STR)}|pw}}}\pend
           
\pstart
           \centering{}\textcolor{gray}{\textbf{TELEGRAMM-ADRESSE: TAGEBUCH
                        BERLIN\orgindex{Tage-Buch@Das Tage-Buch|pw} ⋅ FERNSPRECHER: AMT LÜTZOW\orgindex{Fernsprechamt Lietzow@Fernsprechamt Lietzow|pw}
                     Nr. 4931}}\pend
           
\pstart
           \centering{}\textcolor{gray}{\textbf{SPRECHSTUNDE DER REDAKTION: 12–1 UHR}}\pend
           
\pstart
           Gr/Sch\pend
           
\pstart
           \centering{}26. April 1922\pend
           
\pstart
           \textcolor{gray}{\textbf{\emph{REDAKTION}}}\pend
           
\pstart
           Herrn\pend
           
\pstart
           Dr. Arthur \so{Schnitzler}\pend
           
\pstart
           \so{Wien}\oindex{Wien@\textbf{Wien}, \emph{A.ADM2}|pw}\pend
           
\pstart\center{}Verehrter Herr \introOben{}Doktor\introOben{}
                  Schnitzler!\pend\vspace{0.5em}
\pstart
           Sie wissen vielleicht, dass ich mich in Wien\oindex{Wien@\textbf{Wien}, \emph{A.ADM2}|pw} nie
               so sehr als Österreich\oindex{Oesterreich@\textbf{Österreich}, \emph{A.PCLI}|pw}er gefühlt habe, wie ich
               es in Norddeutschland\oindex{Deutschland@\textbf{Deutschland}, \emph{A.PCLI}|pw} tue. Das hat mein ganzes
               Verhältnis zur Heimat wesentlich geändert. Deshalb glaube ich keine Fehlbitte zu tun,
               wenn ich Ihnen mitteile, dass wir Mitte Mai ein \label{K_L02378-1v}\edtext{Heft}{\lemma{\textnormal{\emph{Heft}}}\Cendnote{\textnormal{Die Nummer 20 des \emph{Tage-Buchs}\pwindex{Tage-Buch@\emph{Das Tage-Buch}|pwk} vom 20. 5. 1922 enthält zwar mehrere Beiträge, die
                  sich mit Österreich\oindex{Oesterreich@\textbf{Österreich}, \emph{A.PCLI}|pwk} beschäftigen, aber nur
                  einen kleinen Gruß zum 60. Geburtstag Schnitzlers ([Stefan Großmann\pwindex{Grossmann, Stefan 19.05.1875 – 03.01.1935@\textsc{Großmann, Stefan} (19.05.1875 – 03.01.1935), \emph{Schriftsteller/Schriftstellerin, Journalist/Journalistin}|pwk}?]: \emph{Von der kleinen Liebe}\pwindex{Von der kleinen Liebe@\emph{Von der kleinen Liebe}|pwk}, Jg. 3, H. 20, S. 766–767).}}}\label{K_L02378-1} des
                  »Tage-Buch\orgindex{Tage-Buch@Das Tage-Buch|pw}« herausgeben wollen, das ein österreich\oindex{Oesterreich@\textbf{Österreich}, \emph{A.PCLI}|pw}isches Heft, ein Schnitzlerheft werden
               soll. Ich habe auch heute dieserhalb an Felix
                  Salten\pwindex{Salten, Felix 06.09.1869 – 08.10.1945@\textsc{Salten, Felix} (06.09.1869 – 08.10.1945), \emph{Schriftsteller/Schriftstellerin, Journalist/Journalistin, Chefredakteur/Chefredakteurin}|pw} geschrieben, und ich wäre Ihnen zu Dank verpflichtet, wenn Sie mir
               dafür eine Ihrer ungedruckten Arbeiten, seien es nur Aphorismen oder eine andere
               ungerecht verschollene Arbeit aus früheren Zeiten, überlassen wollten. Ich sende
               Ihnen die letzten Nummern des »Tage-Buch\orgindex{Tage-Buch@Das Tage-Buch|pw}«, aus
               denen Sie ersehen wollen, dass die Zeitschrift die besten deutschen Autoren zu ihren
               Mitarbeitern hat, sodass sie sich sehen lassen kann.\pend
           
\pstart
           {\pb}Da die Zeit drängt, bitte ich Sie um eine
               möglichst rasche Antwort und bin\pend
           
\pstart
           mit herzlichsten Grüssen{\\[\baselineskip]}Ihr{\\[\baselineskip]}{[}hs.:{]} dankbarer{\\[\baselineskip]}\spacefill\mbox{Stefan Großmann}\pend
           \leftskip=0em{}\selectlanguage{ngerman}\endnumbering\briefempfaengerindex{Schnitzler, Arthur@\textsc{Schnitzler, Arthur}!zzzGrossmann, Stefan@\emph{von Stefan Großmann}!1922-04-261@{26. 4. 1922}|)be}\mylabel{L02378h}  \normalsize

\doendnotes{C}
\bigskip
\vfill

\clearpage

\footnotesize

\lohead{\textsc{register}}

% Definiere theindex-Environment komplett neu ohne reledmac
\makeatletter
\renewenvironment{theindex}{%
  \section*{\indexname}%
  \setlength{\parindent}{0pt}%
  \setlength{\parskip}{0pt plus 0.3pt}%
  \let\item\@idxitem
}{%
  \clearpage
}
\makeatother

\IfFileExists{\jobname-pw.ind}{\input{\jobname-pw.ind}}{}

\end{document}

      