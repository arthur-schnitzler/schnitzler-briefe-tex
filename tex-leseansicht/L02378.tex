%% latex-leseansicht-vorspann.tex
%% Vorspann für die Leseansicht.
%% Lädt die gemeinsame Datei latex-vorspann.tex mit nicht gesetztem Schalter.

\newif\ifkorrekturansicht
\korrekturansichtfalse

\input{../tex-inputs/latex-vorspann}

\begin{center}
            \textcolor{red}{ENTWURF. ENTZIFFERUNG NOCH NICHT KORREKTURGELESEN}
                      \end{center}
            
               \section[Stefan Großmann an Arthur Schnitzler, 26. 4. 1922]{ Stefan Großmann an Arthur Schnitzler, 26. 4. 1922}\nopagebreak\mylabel{v}\rehead{ }\begin{ledgroupsized}[t]{13cm}\normalsize\beginnumbering\briefempfaengerindex{Schnitzler, Arthur@\textsc{Schnitzler, Arthur}!zzzGrossmann, Stefan@\emph{von Stefan Großmann}!1922-04-261@{26. 4. 1922}|(be} \toendnotes[C]{\smallbreak\pagebreak[2]} \Standort{CUL, Schnitzler, B 34.}
\physDesc{Brief, 1 Blatt, 2 Seiten
\newline{}Schreibmaschine
\newline{}Handschrift: schwarze Tinte, deutsche Kurrent (\noindent{}Einfügung von »Doktor«,
                           Schlussformel und Unterschrift)
\newline{}Schnitzler: 1) mit Bleistift beschriftet: »\textsc{Großman}« und »b. sein« 2) mit rotem Buntstift vier Unterstreichungen\newline{}Ordnung: mit Bleistift von unbekannter Hand
                           nummeriert: »16« }\toendnotes[C]{\smallbreak}\pstart
           \noindent{}\centering{}{\pb}\textcolor{gray}{\textbf{Das Tage-Buch\orgindex{Tage-Buch@Das Tage-Buch|pw}}}\pend
           \pstart
           \noindent{}\centering{}\textcolor{gray}{\textbf{Erscheint jeden Sonnabend ⋅ Herausgeber: Stefan Großmann}}\pend
           \pstart
           \noindent{}\centering{}\textcolor{gray}{\textbf{Ernst Rowohlt Verlag\orgindex{Ernst Rowohlt Verlag@Ernst Rowohlt Verlag|pw} ⋅ Berlin W 35\oindex{Berlin@\textbf{Berlin}|pw}}}\pend
           \pstart
           \noindent{}\centering{}\textcolor{gray}{\textbf{POTSDAMER STRASSE 123\textsuperscript{B}
                        ⋅ AN DER POTSDAMER BRÜCKE\oindex{Potsdamerstrasse@\textbf{Potsdamerstraße}|pw}}}\pend
           \pstart
           \noindent{}\centering{}\textcolor{gray}{\textbf{TELEGRAMM-ADRESSE: TAGEBUCH
                        BERLIN\orgindex{Tage-Buch@Das Tage-Buch|pw} ⋅ FERNSPRECHER: AMT LÜTZOW\orgindex{Fernsprechamt Lietzow@Fernsprechamt Lietzow|pw}
                     Nr. 4931}}\pend
           \pstart
           \noindent{}\centering{}\textcolor{gray}{\textbf{SPRECHSTUNDE DER REDAKTION: 12–1 UHR}}\pend
           \pstart
           \noindent{}Gr/Sch\pend
           \pstart
           \centering{}26. April 1922\pend
           \pstart
           \textcolor{gray}{\textbf{\emph{REDAKTION}}}\pend
           \pstart
           Herrn\pend
           \leftskip=3em{}\pstart
           \noindent{}Dr. Arthur \so{Schnitzler}\pend
           \leftskip=0em{}\leftskip=3em{}\pstart
           \so{Wien}\oindex{Wien@\textbf{Wien}|pw}\pend
           \leftskip=0em{}\pstart\center{}Verehrter Herr \introOben{}Doktor\introOben{}
                  Schnitzler!\pend\pstart
           Sie wissen vielleicht, dass ich mich in Wien\oindex{Wien@\textbf{Wien}|pw} nie so
               sehr als Österreich\oindex{Oesterreich@\textbf{Österreich}|pw}er gefühlt habe, wie ich es in
                  Norddeutschland\oindex{Deutschland@\textbf{Deutschland}|pw} tue. Das hat mein ganzes
               Verhältnis zur Heimat wesentlich geändert. Deshalb glaube ich keine Fehlbitte zu tun,
               wenn ich Ihnen mitteile, dass wir Mitte Mai ein \label{K_L02378_1v}\edtext{Heft}{\lemma{\textnormal{\emph{Heft}}}\Cendnote{\textnormal{Die Nummer 20 des \emph{Tage-Buch}\pwindex{Tage-Buch1920-01-01 – 1933-01-01@\emph{Das Tage-Buch}|pwk}s vom 20. 5. 1922
                  enthält zwar mehrere Beiträge, die sich mit Österreich\oindex{Oesterreich@\textbf{Österreich}|pwk} beschäftigen, aber nur einen kleinen Gruss zum
                  60. Geburtstag Schnitzler\pwindex{Schnitzler, Arthur 15.05.1862 – 21.10.1931@\textsc{Schnitzler, Arthur} (15.05.1862 – 21.10.1931), \emph{Schriftsteller, Mediziner}|pwk}s ([O.V. =
                     Großmann?]: \emph{Von der kleinen Liebe}\pwindex{Von der kleinen Liebe1922-05-20@\emph{Von der kleinen Liebe} {[}1922-05-20{]}|pwk}, Jg. 3, H. 20, S. 766–767).}}}\label{K_L02378_1h} des »Tage-Buch\orgindex{Tage-Buch@Das Tage-Buch|pw}«
               herausgeben wollen, das ein österreich\oindex{Oesterreich@\textbf{Österreich}|pw}isches Heft,
               ein Schnitzlerheft werden soll. Ich habe auch heute dieserhalb an Felix Salten\pwindex{Salten, Felix 06.09.1869 – 08.10.1945@\textsc{Salten, Felix} (06.09.1869 – 08.10.1945), \emph{Schriftsteller, Journalist}|pw} geschrieben, und ich wäre Ihnen zu Dank
               verpflichtet, wenn Sie mir dafür eine Ihrer ungedruckten Arbeiten, seien es nur
               Aphorismen oder eine andere ungerecht verschollene Arbeit aus früheren Zeiten,
               überlassen wollten. Ich sende Ihnen die letzten Nummern des »Tage-Buch\orgindex{Tage-Buch@Das Tage-Buch|pw}«, aus denen Sie ersehen wollen, dass die Zeitschrift die
               besten deutschen Autoren zu ihren Mitarbeitern hat, sodass sie sich sehen lassen
               kann.\pend
           \pstart
           {\pb}Da die Zeit drängt, bitte ich Sie um eine
               möglichst rasche Antwort und bin\pend
           \pstart
           mit herzlichsten Grüssen{\\[\baselineskip]}Ihr{\\[\baselineskip]}{[}hs.:{]} dankbarer{\\[\baselineskip]}\spacefill\mbox{Stefan Großmann}\pend
           \leftskip=0em{}\endnumbering\briefempfaengerindex{Schnitzler, Arthur@\textsc{Schnitzler, Arthur}!zzzGrossmann, Stefan@\emph{von Stefan Großmann}!1922-04-261@{26. 4. 1922}|)be}\mylabel{h}\end{ledgroupsized}  \newcommand{\dateiname}{L02378}\newcommand{\titel}{Stefan Großmann an Arthur Schnitzler, 26. 4. 1922}\newcommand{\editorInnen}{ Martin Anton Müller und Gerd-Hermann Susen}%% latex-leseansicht-abspann.tex
%% Abspann für die Leseansicht.
%% Der Schalter \ifkorrekturansicht ist bereits durch den Vorspann gesetzt.

%% latex-abspann.tex
%% Gemeinsamer Abspann für Korrekturansicht und Leseansicht.
%% Setzt den Schalter \ifkorrekturansicht voraus (gesetzt in den
%% einbindenden Dateien latex-korrekturansicht-abspann.tex bzw.
%% latex-leseansicht-abspann.tex).
%% ---------------------------------------------------------------

\normalsize

% Das esempio-Environment wird nur in der Leseansicht benötigt
\ifkorrekturansicht\else
\newenvironment{esempio}[3]%
{
    \vspace{1.5ex}
    \rlap{\underline{#1}}
    \par
    \setlength{\parindent}{0cm}
    \nopagebreak
    \leftskip=#2cm
    \rightskip=#3cm
}
{
    \par
}
\fi

\doendnotes{C}
\bigskip
\vfill

\clearpage

\footnotesize

\ifkorrekturansicht
  \lohead{\textsc{register}}
\fi

% theindex-Environment neu definieren ohne reledmac
\makeatletter
\renewenvironment{theindex}{%
  \ifkorrekturansicht
    \section*{\indexname}%
  \else
    \subsubsection*{Index der erwähnten Entitäten}%
  \fi
  \setlength{\parindent}{0pt}%
  \setlength{\parskip}{0pt plus 0.3pt}%
  \let\item\@idxitem
}{%
  \ifkorrekturansicht\clearpage\fi
}
\makeatother

\IfFileExists{\jobname-pw.ind}{\input{\jobname-pw.ind}}{}

% Quellenangabe nur in der Leseansicht
\ifkorrekturansicht\else
% Fallback-Definitionen, falls die .tex-Datei \titel etc. nicht gesetzt hat
\providecommand{\titel}{}
\providecommand{\editorInnen}{}
\providecommand{\dateiname}{\jobname}

\vspace{3cm}

\vfill

\footnotesize
\textsc{Quelle}: \titel. Herausgegeben von {\editorInnen}. In: \emph{Arthur Schnitzler: Briefwechsel mit Autorinnen und Autoren}.
 Digitale Edition, https://schnitzler-briefe.acdh.oeaw.ac.at/{\dateiname}.html (Stand \today)
\fi

\end{document}


      