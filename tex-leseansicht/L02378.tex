%% latex-leseansicht-vorspann.tex
%% Vorspann für die Leseansicht.
%% Lädt die gemeinsame Datei latex-vorspann.tex mit nicht gesetztem Schalter.

\newif\ifkorrekturansicht
\korrekturansichtfalse

\input{../tex-inputs/latex-vorspann}


\section[Stefan Großmann an Arthur Schnitzler, 26. 4. 1922]{L02378 Stefan Großmann an Arthur Schnitzler, 26. 4. 1922}
\nopagebreak\mylabel{L02378v}
\rehead{ }\normalsize\beginnumbering\briefempfaengerindex{Schnitzler, Arthur@\textsc{Schnitzler, Arthur}!zzzGroßmann, Stefan@\emph{von Stefan Großmann}!1922-04-261@{26. 4. 1922}|(be}
\toendnotes[C]{\smallbreak\pagebreak[2]}
\correspDesc{Versand  durch Stefan Großmann am 26. 4. 1922 in Berlin
\newline{}Erhalt  durch Arthur Schnitzler im Zeitraum [27. 4. 1922
                  – 1. 5. 1922?] in Wien}\toendnotes[C]{\smallbreak}
\Standort{CUL, Schnitzler, B 34.}
\physDesc{Brief, 1 Blatt, 2 Seiten, 1034 Zeichen
\newline{}Schreibmaschine
\newline{}Handschrift: schwarze Tinte, deutsche Kurrent (\noindent{}Einfügung von »Doktor«, Schlussformel und
                                 Unterschrift)
\newline{}Schnitzler: 1) mit Bleistift beschriftet: »\textsc{Großman}« und »b. sein«  2) mit rotem Buntstift vier Unterstreichungen
\newline{}Ordnung: mit Bleistift von unbekannter Hand nummeriert:
                                    »16« }\toendnotes[C]{\smallbreak}
\pstart
           \centering{}{\pb}\textcolor{gray}{\textbf{Das Tage-Buch\orgindex{Tage-Buch@Das Tage-Buch|pw}}}\pend
           
\pstart
           \centering{}\textcolor{gray}{\textbf{Erscheint jeden Sonnabend ⋅ Herausgeber: Stefan Großmann}}\pend
           
\pstart
           \centering{}\textcolor{gray}{\textbf{Ernst Rowohlt Verlag\orgindex{Ernst Rowohlt Verlag@Ernst Rowohlt Verlag|pw} ⋅ Berlin W 35\oindex{Berlin@\textbf{Berlin}, \emph{Hauptstadt}|pw}}}\pend
           
\pstart
           \centering{}\textcolor{gray}{\textbf{POTSDAMER STRASSE 123\textsuperscript{B} ⋅ AN DER POTSDAMER BRÜCKE\oindex{Potsdamer Straße@\textbf{Potsdamer Straße}, \emph{Straße}|pw}}}\pend
           
\pstart
           \centering{}\textcolor{gray}{\textbf{TELEGRAMM-ADRESSE: TAGEBUCH
                        BERLIN\orgindex{Tage-Buch@Das Tage-Buch|pw} ⋅ FERNSPRECHER: AMT LÜTZOW\orgindex{Fernsprechamt Lietzow@Fernsprechamt Lietzow|pw}
                     Nr. 4931}}\pend
           
\pstart
           \centering{}\textcolor{gray}{\textbf{SPRECHSTUNDE DER REDAKTION: 12–1 UHR}}\pend
           
\pstart
           Gr/Sch\pend
           
\pstart
           \centering{}26. April 1922\pend
           
\pstart
           \textcolor{gray}{\textbf{\emph{REDAKTION}}}\pend
           
\pstart
           Herrn\pend
           
\pstart
           Dr. Arthur \so{Schnitzler}\pend
           
\pstart
           \so{Wien}\oindex{Wien@\textbf{Wien}, \emph{Verwaltungsgebiet}|pw}\pend
           
\pstart\center{}Verehrter Herr \introOben{}Doktor\introOben{}
                  Schnitzler!\pend\vspace{0.5em}
\pstart
           Sie wissen vielleicht, dass ich mich in Wien\oindex{Wien@\textbf{Wien}, \emph{Verwaltungsgebiet}|pw} nie
               so sehr als Österreich\oindex{Österreich@\textbf{Österreich}|pw}er gefühlt habe, wie ich
               es in Norddeutschland\oindex{Deutschland@\textbf{Deutschland}|pw} tue. Das hat mein ganzes
               Verhältnis zur Heimat wesentlich geändert. Deshalb glaube ich keine Fehlbitte zu tun,
               wenn ich Ihnen mitteile, dass wir Mitte Mai ein \label{K_L02378-1v}\edtext{Heft}{\lemma{\textnormal{\emph{Heft}}}\Cendnote{\textnormal{Die Nummer 20 des \emph{Tage-Buchs}\pwindex{Tage-Buch@\emph{Das Tage-Buch}|pwk} vom 20. 5. 1922 enthält zwar mehrere Beiträge, die
                  sich mit Österreich\oindex{Österreich@\textbf{Österreich}|pwk} beschäftigen, aber nur
                  einen kleinen Gruß zum 60. Geburtstag Schnitzlers ([Stefan Großmann\pwindex{Großmann, Stefan 19.\,5.\,1875 Wien – 3.\,1.\,1935 ebd.@\textsc{Großmann, Stefan} (19.\,5.\,1875 Wien – 3.\,1.\,1935 ebd.), \emph{Schriftsteller, Journalist}|pwk}?]: \emph{Von der kleinen Liebe}\pwindex{Von der kleinen Liebe@\emph{Von der kleinen Liebe}|pwk}, Jg. 3, H. 20, S. 766–767).}}}\label{K_L02378-1} des
                  »Tage-Buch\orgindex{Tage-Buch@Das Tage-Buch|pw}« herausgeben wollen, das ein österreich\oindex{Österreich@\textbf{Österreich}|pw}isches Heft, ein Schnitzlerheft werden
               soll. Ich habe auch heute dieserhalb an Felix
                  Salten\pwindex{Salten, Felix 6.\,9.\,1869 Budapest – 8.\,10.\,1945 Zürich@\textsc{Salten, Felix} (6.\,9.\,1869 Budapest – 8.\,10.\,1945 Zürich), \emph{Schriftsteller, Journalist, Chefredakteur}|pw} geschrieben, und ich wäre Ihnen zu Dank verpflichtet, wenn Sie mir
               dafür eine Ihrer ungedruckten Arbeiten, seien es nur Aphorismen oder eine andere
               ungerecht verschollene Arbeit aus früheren Zeiten, überlassen wollten. Ich sende
               Ihnen die letzten Nummern des »Tage-Buch\orgindex{Tage-Buch@Das Tage-Buch|pw}«, aus
               denen Sie ersehen wollen, dass die Zeitschrift die besten deutschen Autoren zu ihren
               Mitarbeitern hat, sodass sie sich sehen lassen kann.\pend
           
\pstart
           {\pb}Da die Zeit drängt, bitte ich Sie um eine
               möglichst rasche Antwort und bin\pend
           
\pstart
           mit herzlichsten Grüssen{\\[\baselineskip]}Ihr{\\[\baselineskip]}{[}hs.:{]} dankbarer{\\[\baselineskip]}\spacefill\mbox{Stefan Großmann}\pend
           \leftskip=0em{}\selectlanguage{ngerman}\endnumbering\briefempfaengerindex{Schnitzler, Arthur@\textsc{Schnitzler, Arthur}!zzzGroßmann, Stefan@\emph{von Stefan Großmann}!1922-04-261@{26. 4. 1922}|)be}\mylabel{L02378h}  \newcommand{\dateiname}{L02378}\newcommand{\titel}{Stefan Großmann an Arthur Schnitzler, 26. 4. 1922}\newcommand{\editorInnen}{Herausgegeben von Martin Anton Müller}%% latex-leseansicht-abspann.tex
%% Abspann für die Leseansicht.
%% Der Schalter \ifkorrekturansicht ist bereits durch den Vorspann gesetzt.

%% latex-abspann.tex
%% Gemeinsamer Abspann für Korrekturansicht und Leseansicht.
%% Setzt den Schalter \ifkorrekturansicht voraus (gesetzt in den
%% einbindenden Dateien latex-korrekturansicht-abspann.tex bzw.
%% latex-leseansicht-abspann.tex).
%% ---------------------------------------------------------------

\normalsize

% Das esempio-Environment wird nur in der Leseansicht benötigt
\ifkorrekturansicht\else
\newenvironment{esempio}[3]%
{
    \vspace{1.5ex}
    \rlap{\underline{#1}}
    \par
    \setlength{\parindent}{0cm}
    \nopagebreak
    \leftskip=#2cm
    \rightskip=#3cm
}
{
    \par
}
\fi

\doendnotes{C}
\bigskip
\vfill

\clearpage

\footnotesize

\ifkorrekturansicht
  \lohead{\textsc{register}}
\fi

% theindex-Environment neu definieren ohne reledmac
\makeatletter
\renewenvironment{theindex}{%
  \ifkorrekturansicht
    \section*{\indexname}%
  \else
    \subsubsection*{Index der erwähnten Entitäten}%
  \fi
  \setlength{\parindent}{0pt}%
  \setlength{\parskip}{0pt plus 0.3pt}%
  \let\item\@idxitem
}{%
  \ifkorrekturansicht\clearpage\fi
}
\makeatother

\IfFileExists{\jobname-pw.ind}{\input{\jobname-pw.ind}}{}

% Quellenangabe nur in der Leseansicht
\ifkorrekturansicht\else
% Fallback-Definitionen, falls die .tex-Datei \titel etc. nicht gesetzt hat
\providecommand{\titel}{}
\providecommand{\editorInnen}{}
\providecommand{\dateiname}{\jobname}

\vspace{3cm}

\vfill

\footnotesize
\textsc{Quelle}: \titel. Herausgegeben von {\editorInnen}. In: \emph{Arthur Schnitzler: Briefwechsel mit Autorinnen und Autoren}.
 Digitale Edition, https://schnitzler-briefe.acdh.oeaw.ac.at/{\dateiname}.html (Stand \today)
\fi

\end{document}


