%% latex-korrekturansicht-vorspann.tex
%% Vorspann für die Korrekturansicht.
%% Lädt die gemeinsame Datei latex-vorspann.tex mit gesetztem Schalter.

\newif\ifkorrekturansicht
\korrekturansichttrue

\input{../tex-inputs/latex-vorspann}


\section[Hugo von Hofmannsthal an Arthur Schnitzler, 1{[}1?{]}. 6. 1902]{L01222 Hugo von Hofmannsthal an Arthur Schnitzler, 1{[}1?{]}. 6. 1902}
\nopagebreak\mylabel{L01222v}
\rehead{ }\normalsize\beginnumbering\briefempfaengerindex{Schnitzler, Arthur@\textsc{Schnitzler, Arthur}!zzzHofmannsthal, Hugo von@\emph{von Hugo von Hofmannsthal}!1902-06-111@{1{[}1?{]}. 6. 1902}|(be}
\toendnotes[C]{\smallbreak\pagebreak[2]}\Standort{CUL, Schnitzler, B 43.}
\physDesc{Brief, 1 Blatt, 4 Seiten, 930 Zeichen
\newline{}Handschrift: schwarze Tinte, deutsche Kurrent
\newline{}Schnitzler: mit Bleistift datiert: »10/6 902« 
\newline{}Ordnung: 1) mit Bleistift von unbekannter Hand nummeriert: »\strikeout{196}«  2) mit Bleistift von unbekannter Hand nummeriert:
                                    »189«}
\buchAbdrucke{\weitereDrucke{Hugo von Hofmannsthal, Arthur Schnitzler: \emph{Briefwechsel}. Frankfurt am Main: \emph{S. Fischer} 1964, S. 158–159.} }\toendnotes[C]{\smallbreak}
\pstart
           \raggedleft{}{\pb}\label{K_L01222-1v}\edtext{Mittwoch}{\lemma{\textnormal{\emph{Mittwoch}}}\Cendnote{\textnormal{Schnitzlers Datierung verweist auf
                        einen Dienstag. Unter der Annahme, dass er – und nicht Hofmannsthal – sich
                        geirrt hat, wurde auf den Folgetag datiert.}}}\label{K_L01222-1}\pend
           
\pstart{}lieber Arthur\pend\vspace{0.5em}
\pstart
           wenn nächſten Sonntag (15.\textsuperscript{ten}) ſchönes Wetter iſt, möchten Richard\pwindex{Beer-Hofmann, Richard 1866-07-11 – 1945-09-26@\textsc{Beer-Hofmann, Richard} (1866-07-11 – 1945-09-26), \emph{Schriftsteller/Schriftstellerin}|pw}
               und ich ſehr gern um 11\textsuperscript{h} vormittag auf dem Friedhof in \textsc{Gutenſtein}\oindex{Bergfriedhof@\textbf{Bergfriedhof}, \emph{Friedhof (K.FRH)}|pw} bei der Beſtattung von \label{K_L01222-2v}\edtext{Raimund\pwindex{Raimund, Ferdinand 01.06.1790 – 05.09.1836@\textsc{Raimund, Ferdinand} (01.06.1790 – 05.09.1836), \emph{Schauspieler/Schauspielerin, Dramatiker/Dramatikerin}|pw} im neuen Grab}{\lemma{\textnormal{\emph{Raimund im neuen Grab}}}\Cendnote{\textnormal{Die Wiederbestattung in der renovierten
                  Gruft fand am 15. 6. 1902 um 11 Uhr vormittags statt.
                  Einige kulturelle Prominenz aus Wien\oindex{Wien@\textbf{Wien}, \emph{A.ADM2}|pwk} war dafür
                  angereist, Schnitzler aber nicht.}}}\label{K_L01222-2}
               dabei ſein. Mir ſagt ein für gewöhnlich bei {\pb}mir nicht ſo lebhaftes Gefühl,
               daſs ich es thuen ſoll.\pend
           
\pstart
           Wir würden in \textsc{Mödling}\oindex{Moedling@\textbf{Mödling}, \emph{P.PPLA3}|pw} in den Schnellzug einſteigen der in \textsc{Mödling}\oindex{Moedling@\textbf{Mödling}, \emph{P.PPLA3}|pw}{ }7\textsuperscript{h}15 durchfährt, in Wien\oindex{Wien@\textbf{Wien}, \emph{A.ADM2}|pw} geht er 6\textsuperscript{h}50 ab.
               Ich möchte dann in Guthenſtein\oindex{Gutenstein@\textbf{Gutenstein}, \emph{A.ADM3}|pw} mittageſſen {\pb}und den ſchönen Weg über \textsc{Vöslau}\oindex{Bad Voeslau@\textbf{Bad Vöslau}, \emph{P.PPLA3}|pw} etc. nachmittag mit dem Rad zurück-machen. Ich hoffe, mit Ihnen.\pend
           
\pstart
           Wenn Sie nichts ſagen laſſen und \uline{es kein Regentag}
               iſt, ſo hoffen wir, Sie ſind im Zug oder ſteigen in \textsc{Mödling}\oindex{Moedling@\textbf{Mödling}, \emph{P.PPLA3}|pw} in ihn ein.\pend
           
\pstart
           {\pb}Iſt das Wetter zweifelhaft ſo
               kann man ſich noch Samstag bis 9\textsuperscript{h} abends im Telephon ſprechen.\pend
           
\pstart
           Von Herzen Ihr{\\[\baselineskip]}\spacefill\mbox{Hugo.}\pend
           \leftskip=0em{}
\pstart
           \noindent{}\textsc{Circa} 20\textsuperscript{ten} hoffe ich wir
                  fahren \textsc{Salzburg\oindex{Salzburg@\textbf{Salzburg}, \emph{A.ADM2}|pw} – Lofer\oindex{Lofer@\textbf{Lofer}, \emph{P.PPLA3}|pw} – Innsbruck\oindex{Innsbruck@\textbf{Innsbruck}, \emph{A.ADM2}|pw} – (Seitenausflug
                        Stubaithal\oindex{Stubaital@\textbf{Stubaital}, \emph{T.VAL}|pw}) – Brenner\oindex{Brenner@\textbf{Brenner}, \emph{T.PASS}|pw} – Toblach\oindex{Toblach@\textbf{Toblach}, \emph{A.ADM3}|pw}
                     (Seitenausflug Ampezzothal\oindex{Ampezzo@\textbf{Ampezzo}, \emph{P.PPLA3}|pw}) – Spital a. Drau\oindex{Spittal an der Drau@\textbf{Spittal an der Drau}, \emph{P.PPLA3}|pw} – Radstadt\oindex{Radstadt@\textbf{Radstadt}, \emph{A.ADM3}|pw} – Bischofshofen\oindex{Bischofshofen@\textbf{Bischofshofen}, \emph{P.PPLA3}|pw} – Salzburg\oindex{Salzburg@\textbf{Salzburg}, \emph{A.ADM2}|pw}, circa
                     12 Tage}.\pend
           \selectlanguage{ngerman}\endnumbering\briefempfaengerindex{Schnitzler, Arthur@\textsc{Schnitzler, Arthur}!zzzHofmannsthal, Hugo von@\emph{von Hugo von Hofmannsthal}!1902-06-111@{1{[}1?{]}. 6. 1902}|)be}\mylabel{L01222h}  \normalsize

\doendnotes{C}
\bigskip
\vfill

\clearpage

\footnotesize

\lohead{\textsc{register}}

% Definiere theindex-Environment komplett neu ohne reledmac
\makeatletter
\renewenvironment{theindex}{%
  \section*{\indexname}%
  \setlength{\parindent}{0pt}%
  \setlength{\parskip}{0pt plus 0.3pt}%
  \let\item\@idxitem
}{%
  \clearpage
}
\makeatother

\IfFileExists{\jobname-pw.ind}{\input{\jobname-pw.ind}}{}

\end{document}

      