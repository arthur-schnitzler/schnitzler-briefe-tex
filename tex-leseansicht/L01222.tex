%% latex-leseansicht-vorspann.tex
%% Vorspann für die Leseansicht.
%% Lädt die gemeinsame Datei latex-vorspann.tex mit nicht gesetztem Schalter.

\newif\ifkorrekturansicht
\korrekturansichtfalse

\input{../tex-inputs/latex-vorspann}


\section[Hugo von Hofmannsthal an Arthur Schnitzler, 1[1?]. 6. 1902]{L01222 Hugo von Hofmannsthal an Arthur Schnitzler, 1[1?]. 6. 1902}
\nopagebreak\mylabel{L01222v}
\rehead{ }\normalsize\beginnumbering\briefempfaengerindex{Schnitzler, Arthur@\textsc{Schnitzler, Arthur}!zzzHofmannsthal, Hugo von@\emph{von Hugo von Hofmannsthal}!1902-06-111@{1[1?]. 6. 1902}|(be}
\toendnotes[C]{\smallbreak\pagebreak[2]}
\correspDesc{Versand  durch Hugo von Hofmannsthal am 1[1?]. 6. 1902 \textbf{Ort fehlend} 
\newline{}Erhalt  durch Arthur Schnitzler im Zeitraum [11. 6. 1902
                  – 15. 6. 1902?] in Wien}\toendnotes[C]{\smallbreak}
\Standort{CUL, Schnitzler, B 43.}
\physDesc{Brief, 1 Blatt, 4 Seiten, 930 Zeichen
\newline{}Handschrift: schwarze Tinte, deutsche Kurrent
\newline{}Schnitzler: mit Bleistift datiert: »10/6 902« 
\newline{}Ordnung: 1) mit Bleistift von unbekannter Hand nummeriert: »\strikeout{196}«  2) mit Bleistift von unbekannter Hand nummeriert:
                                    »189«}
\buchAbdrucke{\weitereDrucke{Hugo von Hofmannsthal, Arthur Schnitzler: \emph{Briefwechsel}. Herausgegeben von Therese Nickl und Heinrich Schnitzler. Frankfurt am Main: \emph{S. Fischer} 1964, S. 158–159.} }\toendnotes[C]{\smallbreak}
\pstart
           \raggedleft{}{\pb}\label{K_L01222-1v}\edtext{Mittwoch}{\lemma{\textnormal{\emph{Mittwoch}}}\Cendnote{\textnormal{Schnitzlers Datierung verweist auf
                        einen Dienstag. Unter der Annahme, dass er – und nicht Hofmannsthal – sich
                        geirrt hat, wurde auf den Folgetag datiert.}}}\label{K_L01222-1}\pend
           
\pstart{}lieber Arthur\pend\vspace{0.5em}
\pstart
           wenn nächſten Sonntag (15.\textsuperscript{ten}){ }ſchönes Wetter iſt, möchten Richard\pwindex{Beer-Hofmann, Richard 11.\,7.\,1866 Wien – 26.\,9.\,1945 New York City@\textsc{Beer-Hofmann, Richard} (11.\,7.\,1866 Wien – 26.\,9.\,1945 New York City), \emph{Schriftsteller}|pw}
               und ich{ }ſehr gern um 11\textsuperscript{h} vormittag auf dem Friedhof in \textsc{Gutenſtein}\oindex{Bergfriedhof@\textbf{Bergfriedhof}, \emph{Friedhof}|pw} bei der Beſtattung von \label{K_L01222-2v}\edtext{Raimund\pwindex{Raimund, Ferdinand 1.\,6.\,1790 Wien – 5.\,9.\,1836 Pottenstein@\textsc{Raimund, Ferdinand} (1.\,6.\,1790 Wien – 5.\,9.\,1836 Pottenstein), \emph{Schauspieler, Dramatiker}|pw} im neuen Grab}{\lemma{\textnormal{\emph{Raimund im neuen Grab}}}\Cendnote{\textnormal{Die Wiederbestattung in der renovierten
                  Gruft fand am 15. 6. 1902 um 11 Uhr vormittags statt.
                  Einige kulturelle Prominenz aus Wien\oindex{Wien@\textbf{Wien}, \emph{Verwaltungsgebiet}|pwk} war dafür
                  angereist, Schnitzler aber nicht.}}}\label{K_L01222-2}
               dabei{ }ſein. Mir{ }ſagt ein für gewöhnlich bei {\pb}mir nicht{ }ſo lebhaftes Gefühl,
               daſs ich es thuen{ }ſoll.\pend
           
\pstart
           Wir würden in \textsc{Mödling}\oindex{Mödling@\textbf{Mödling}, \emph{Hauptstadt}|pw} in den Schnellzug einſteigen der in \textsc{Mödling}\oindex{Mödling@\textbf{Mödling}, \emph{Hauptstadt}|pw}{ }7\textsuperscript{h}15 durchfährt, in Wien\oindex{Wien@\textbf{Wien}, \emph{Verwaltungsgebiet}|pw} geht er 6\textsuperscript{h}50 ab.
               Ich möchte dann in Guthenſtein\oindex{Gutenstein@\textbf{Gutenstein}, \emph{Verwaltungsgebiet}|pw} mittageſſen {\pb}und den{ }ſchönen Weg über \textsc{Vöslau}\oindex{Bad Vöslau@\textbf{Bad Vöslau}, \emph{Hauptstadt}|pw} etc. nachmittag mit dem Rad zurück-machen. Ich hoffe, mit Ihnen.\pend
           
\pstart
           Wenn Sie nichts{ }ſagen laſſen und \uline{es kein Regentag}
               iſt,{ }ſo hoffen wir, Sie{ }ſind im Zug oder{ }ſteigen in \textsc{Mödling}\oindex{Mödling@\textbf{Mödling}, \emph{Hauptstadt}|pw} in ihn ein.\pend
           
\pstart
           {\pb}Iſt das Wetter zweifelhaft{ }ſo
               kann man{ }ſich noch Samstag bis 9\textsuperscript{h} abends im Telephon{ }ſprechen.\pend
           
\pstart
           Von Herzen Ihr{\\[\baselineskip]}\spacefill\mbox{Hugo.}\pend
           \leftskip=0em{}
\pstart
           \noindent{}\textsc{Circa} 20\textsuperscript{ten} hoffe ich wir
                  fahren \textsc{Salzburg\oindex{Salzburg@\textbf{Salzburg}, \emph{Verwaltungsgebiet}|pw} – Lofer\oindex{Lofer@\textbf{Lofer}, \emph{Hauptstadt}|pw} – Innsbruck\oindex{Innsbruck@\textbf{Innsbruck}, \emph{Verwaltungsgebiet}|pw} – (Seitenausflug
                        Stubaithal\oindex{Stubaital@\textbf{Stubaital}, \emph{Tal}|pw}) – Brenner\oindex{Brenner@\textbf{Brenner}, \emph{Pass}|pw} – Toblach\oindex{Toblach@\textbf{Toblach}, \emph{Verwaltungsgebiet}|pw}
                     (Seitenausflug Ampezzothal\oindex{Ampezzo@\textbf{Ampezzo}, \emph{Hauptstadt}|pw}) – Spital a. Drau\oindex{Spittal an der Drau@\textbf{Spittal an der Drau}, \emph{Hauptstadt}|pw} – Radstadt\oindex{Radstadt@\textbf{Radstadt}, \emph{Verwaltungsgebiet}|pw} – Bischofshofen\oindex{Bischofshofen@\textbf{Bischofshofen}, \emph{Hauptstadt}|pw} – Salzburg\oindex{Salzburg@\textbf{Salzburg}, \emph{Verwaltungsgebiet}|pw}, circa
                     12 Tage}.\pend
           \selectlanguage{ngerman}\endnumbering\briefempfaengerindex{Schnitzler, Arthur@\textsc{Schnitzler, Arthur}!zzzHofmannsthal, Hugo von@\emph{von Hugo von Hofmannsthal}!1902-06-111@{1[1?]. 6. 1902}|)be}\mylabel{L01222h}  \newcommand{\dateiname}{L01222}\newcommand{\titel}{Hugo von Hofmannsthal an Arthur Schnitzler, 1[1?]. 6. 1902}\newcommand{\editorInnen}{Martin Anton Müller und Gerd-Hermann Susen}%% latex-leseansicht-abspann.tex
%% Abspann für die Leseansicht.
%% Der Schalter \ifkorrekturansicht ist bereits durch den Vorspann gesetzt.

%% latex-abspann.tex
%% Gemeinsamer Abspann für Korrekturansicht und Leseansicht.
%% Setzt den Schalter \ifkorrekturansicht voraus (gesetzt in den
%% einbindenden Dateien latex-korrekturansicht-abspann.tex bzw.
%% latex-leseansicht-abspann.tex).
%% ---------------------------------------------------------------

\normalsize

% Das esempio-Environment wird nur in der Leseansicht benötigt
\ifkorrekturansicht\else
\newenvironment{esempio}[3]%
{
    \vspace{1.5ex}
    \rlap{\underline{#1}}
    \par
    \setlength{\parindent}{0cm}
    \nopagebreak
    \leftskip=#2cm
    \rightskip=#3cm
}
{
    \par
}
\fi

\doendnotes{C}
\bigskip
\vfill

\clearpage

\footnotesize

\ifkorrekturansicht
  \lohead{\textsc{register}}
\fi

% theindex-Environment neu definieren ohne reledmac
\makeatletter
\renewenvironment{theindex}{%
  \ifkorrekturansicht
    \section*{\indexname}%
  \else
    \subsubsection*{Index der erwähnten Entitäten}%
  \fi
  \setlength{\parindent}{0pt}%
  \setlength{\parskip}{0pt plus 0.3pt}%
  \let\item\@idxitem
}{%
  \ifkorrekturansicht\clearpage\fi
}
\makeatother

\IfFileExists{\jobname-pw.ind}{\input{\jobname-pw.ind}}{}

% Quellenangabe nur in der Leseansicht
\ifkorrekturansicht\else
% Fallback-Definitionen, falls die .tex-Datei \titel etc. nicht gesetzt hat
\providecommand{\titel}{}
\providecommand{\editorInnen}{}
\providecommand{\dateiname}{\jobname}

\vspace{3cm}

\vfill

\footnotesize
\textsc{Quelle}: \titel. Herausgegeben von {\editorInnen}. In: \emph{Arthur Schnitzler: Briefwechsel mit Autorinnen und Autoren}.
 Digitale Edition, https://schnitzler-briefe.acdh.oeaw.ac.at/{\dateiname}.html (Stand \today)
\fi

\end{document}


