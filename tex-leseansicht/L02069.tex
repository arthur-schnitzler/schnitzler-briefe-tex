\input{../tex-inputs/latex-pdf-vorspann}
\begin{center}
            \textcolor{red}{ENTWURF. ENTZIFFERUNG NOCH NICHT KORREKTURGELESEN}
                      \end{center}
            
               \section[Richard Beer-Hofmann an Arthur Schnitzler, 23. 5. 1912]{ Richard Beer-Hofmann an Arthur Schnitzler,
               23. 5. 1912}\nopagebreak\mylabel{v}\rehead{ }\begin{ledgroupsized}[t]{13cm}\normalsize\beginnumbering\briefempfaengerindex{Schnitzler, Arthur@\textsc{Schnitzler, Arthur}!zzzBeer-Hofmann, Richard@\emph{von Richard Beer-Hofmann}!1912-05-231@{23. 5. 1912}|(be} \toendnotes[C]{\smallbreak\pagebreak[2]} \Standort{CUL, Schnitzler, B 8.}
\physDesc{Visitenkarte
\newline{}Handschrift: schwarze Tinte, lateinische Kurrent\newline{}Ordnung: mit Bleistift von unbekannter Hand nummeriert: »244a« }\pstart
           \noindent{}{\pb}Lieber Arthur! Für Ihre abendliche Virginia!\pend
           \pstart
           Herzlichst\pend
           \pstart
           {\pb}Ihr\pend
           \pstart
           \centering{}\textcolor{gray}{\textbf{\textsc{Richard}{ }\strikeout{\textsc{Beer-Hofmann}}}}\pend
           \pstart
           \noindent{}Wien\oindex{Wien@\textbf{Wien}|pw}\hspace*{1.5em}23./V.
                     12.\pend
           \endnumbering\briefempfaengerindex{Schnitzler, Arthur@\textsc{Schnitzler, Arthur}!zzzBeer-Hofmann, Richard@\emph{von Richard Beer-Hofmann}!1912-05-231@{23. 5. 1912}|)be}\mylabel{h}\end{ledgroupsized}  \newcommand{\dateiname}{L02069}\newcommand{\titel}{Richard Beer-Hofmann an Arthur Schnitzler, 23. 5. 1912}\newcommand{\editorInnen}{Martin Anton Müller und Gerd-Hermann Susen}\input{../tex-inputs/latex-pdf-abspann}
      