%% latex-korrekturansicht-vorspann.tex
%% Vorspann für die Korrekturansicht.
%% Lädt die gemeinsame Datei latex-vorspann.tex mit gesetztem Schalter.

\newif\ifkorrekturansicht
\korrekturansichttrue

\input{../tex-inputs/latex-vorspann}


\section[Richard Beer-Hofmann an Arthur Schnitzler, 23. 5. 1912]{L02069 Richard Beer-Hofmann an Arthur Schnitzler, 23. 5. 1912}
\nopagebreak\mylabel{L02069v}
\rehead{ }\normalsize\beginnumbering\briefempfaengerindex{Schnitzler, Arthur@\textsc{Schnitzler, Arthur}!zzzBeer-Hofmann, Richard@\emph{von Richard Beer-Hofmann}!1912-05-231@{23. 5. 1912}|(be}
\toendnotes[C]{\smallbreak\pagebreak[2]}\Standort{CUL, Schnitzler, B 8.}
\physDesc{Visitenkarte, 71 Zeichen
\newline{}Handschrift: schwarze Tinte, lateinische Kurrent
\newline{}Ordnung: mit Bleistift von unbekannter Hand nummeriert:
                                    »244a« }
\pstart
           \noindent{}{\pb}Lieber Arthur! Für Ihre abendliche Virginia!\pend
           
\pstart
           Herzlichst\pend
           
\pstart
           {\pb}Ihr\pend
           
\pstart
           \centering{}\textcolor{gray}{\textbf{\textsc{Richard}{ }\strikeout{\textsc{Beer-Hofmann}}}}\pend
           
\pstart
           \noindent{}Wien\oindex{Wien@\textbf{Wien}, \emph{A.ADM2}|pw}\hspace*{1.5em}23./V. 12.\pend
           \selectlanguage{ngerman}\endnumbering\briefempfaengerindex{Schnitzler, Arthur@\textsc{Schnitzler, Arthur}!zzzBeer-Hofmann, Richard@\emph{von Richard Beer-Hofmann}!1912-05-231@{23. 5. 1912}|)be}\mylabel{L02069h}  \normalsize

\doendnotes{C}
\bigskip
\vfill

\clearpage

\footnotesize

\lohead{\textsc{register}}

% Definiere theindex-Environment komplett neu ohne reledmac
\makeatletter
\renewenvironment{theindex}{%
  \section*{\indexname}%
  \setlength{\parindent}{0pt}%
  \setlength{\parskip}{0pt plus 0.3pt}%
  \let\item\@idxitem
}{%
  \clearpage
}
\makeatother

\IfFileExists{\jobname-pw.ind}{\input{\jobname-pw.ind}}{}

\end{document}

      