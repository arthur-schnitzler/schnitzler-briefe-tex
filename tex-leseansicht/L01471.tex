%% latex-leseansicht-vorspann.tex
%% Vorspann für die Leseansicht.
%% Lädt die gemeinsame Datei latex-vorspann.tex mit nicht gesetztem Schalter.

\newif\ifkorrekturansicht
\korrekturansichtfalse

\input{../tex-inputs/latex-vorspann}


\section[Arthur Schnitzler an Richard Beer-Hofmann, 15. 11. 1904]{L01471 Arthur Schnitzler an Richard Beer-Hofmann, 15. 11. 1904}
\nopagebreak\mylabel{L01471v}
\rehead{ }\normalsize\beginnumbering\briefempfaengerindex{Beer-Hofmann, Richard@\textsc{Beer-Hofmann, Richard}!zzzSchnitzler, Arthur@\emph{von Arthur Schnitzler}!1904-11-151@{15. 11. 1904}|(be}
\toendnotes[C]{\smallbreak\pagebreak[2]}
\correspDesc{Versand  durch Arthur Schnitzler am 15. 11. 1904 in Berlin
\newline{}Erhalt  durch Richard Beer-Hofmann am 16. 11. 1904 in Rodaun}\toendnotes[C]{\smallbreak}
\Standort{YCGL, MSS 31.}
\physDesc{Briefkarte, 2 Karten, Kuvert, 1215 Zeichen
\newline{}Handschrift: Bleistift, deutsche Kurrent
\newline{}Versand: 1) Stempel: »\nobreak{}\oindex{Berlin@\textbf{Berlin}, \emph{Hauptstadt}|pwk}Berlin W 64, 15. 11. 04, 11–12V\nobreak{}«.   2) Stempel: »\nobreak{}\oindex{Wien@\textbf{Wien}!XXIII., Liesing@\textbf{XXIII., Liesing}!Rodaun@\textbf{Rodaun}, \emph{Region}|pwk}{\pb}Ro\textcolor{gray}{d}aun, 16 \textcolor{gray}{11} 04\nobreak{}«. }
\buchAbdrucke{\weitereDrucke{1) Arthur Schnitzler: \emph{Briefe.} In: \emph{Die Neue Rundschau}, Bd. 68 (1957) Nr. 1, S. 93.} \weitereDrucke{2) Arthur Schnitzler: \emph{Briefe 1875–1912}. Herausgegeben von Therese Nickl und Heinrich Schnitzler. Frankfurt am Main: \emph{S. Fischer} 1981, S. 493.} \weitereDrucke{3) Arthur Schnitzler, Richard Beer-Hofmann: \emph{Briefwechsel 1891–1931}. Herausgegeben von Konstanze Fliedl. Wien, Zürich: \emph{Europaverlag} 1992, S. 170.} }\toendnotes[C]{\smallbreak}\pstart{}{\pb}\textcolor{gray}{\textbf{ICH WACH!}}\pend{}\pstart{}\textcolor{gray}{\textbf{CONRAD UHL’S HOTEL
                           BRISTOL\oindex{Hotel Bristol Berlin@\textbf{Hotel Bristol Berlin}, \emph{Hotel}|pw}}}\pend{}\pstart{}\textcolor{gray}{\textbf{BERLIN U. D. LINDEN\oindex{Unter den Linden@\textbf{Unter den Linden}, \emph{Ehemaliger Ort}|pw} 5 u. 6}}\pend{}{\bigskip}\pstart{}\textsc{{\pb}Herrn Dr. Richard Beer-Hofmann}\pend{}\pstart{}\textsc{Rodaun\oindex{Wien@\textbf{Wien}!XXIII., Liesing@\textbf{XXIII., Liesing}!Rodaun@\textbf{Rodaun}, \emph{Region}|pw}}\pend{}\pstart{}\textsc{bei Wien\oindex{Wien@\textbf{Wien}, \emph{Verwaltungsgebiet}|pw}}\pend{}\pstart{}\textsc{Liesingerstraße 1}\oindex{Liesingerstraße@\textbf{Liesingerstraße}, \emph{Straße}|pw}\pend{}{\bigskip}\vspace{1em}
\pstart
           \raggedleft{}{\pb}15/11 904\pend
           
\pstart
           \textcolor{gray}{\textbf{ICH WACH!}}\hfill \textcolor{gray}{\textbf{CONRAD UHL’S HOTEL
                              BRISTOL\oindex{Hotel Bristol Berlin@\textbf{Hotel Bristol Berlin}, \emph{Hotel}|pw}}}\pend
           
\pstart
           \raggedleft{}\textcolor{gray}{\textbf{BERLIN U. D. LINDEN\oindex{Unter den Linden@\textbf{Unter den Linden}, \emph{Ehemaliger Ort}|pw} 5 u. 6}}\pend
           \vspace{0.5em}
\pstart
           lieber Richard, telegram haben Sie wohl vom Theater aus bekommen:
                  Freitag{ }Samſtag Arrangirprobe. Meine \textsc{Premiere}\pwindex{Schnitzler, Arthur 15.\,5.\,1862 Wien – 21.\,10.\,1931 ebd.@\textsc{Schnitzler, Arthur} (15.\,5.\,1862 Wien – 21.\,10.\,1931 ebd.), \emph{Schriftsteller, Mediziner}!tapfere Cassian. Puppenspiel in einem Akt@\strich\emph{Der tapfere Cassian. Puppenspiel in einem Akt}|pwv}{ }Dinſtag; ich lieſs es Ihnen auch telegraphiren weil Sie am Ende, wenn es
               bei Freitag geblieben wäre, um einen Tag früher gekommen wären. –\pend
           
\pstart
           \textsc{Carlton Hotel}\oindex{Carlton Hotel [Berlin]@\textbf{Carlton Hotel [Berlin]}, \emph{Hotel}|pw}{ }ſoll, wie mir {\pb}\textsc{Reinhardt}\pwindex{Reinhardt, Max 9.\,9.\,1873 Baden bei Wien – 30.\,10.\,1943 New York City@\textsc{Reinhardt, Max} (9.\,9.\,1873 Baden bei Wien – 30.\,10.\,1943 New York City), \emph{Theaterleiter, Regisseur, Schauspieler}|pw}, der dort wohnt,{ }ſagt, nichts rechtes{ }ſein; räth es Ihnen nicht.\pend
           
\pstart
           Ich wohne \textsc{Bristol}\oindex{Hotel Bristol Berlin@\textbf{Hotel Bristol Berlin}, \emph{Hotel}|pw}, es befriedigt mich von allen Berlin\oindex{Berlin@\textbf{Berlin}, \emph{Hauptstadt}|pw}er
               Hotels doch am meiſten. Hoffentlich auf Wiederſehen.\pend
           
\pstart
           \textsc{\uline{Moissi}}\pwindex{Moissi, Alexander 2.\,4.\,1879 Triest – 22.\,3.\,1935 Wien@\textsc{Moissi, Alexander} (2.\,4.\,1879 Triest – 22.\,3.\,1935 Wien), \emph{Schauspieler}|pw}, den ich geſtern zum erſten Mal im Kakadu\pwindex{Schnitzler, Arthur 15.\,5.\,1862 Wien – 21.\,10.\,1931 ebd.@\textsc{Schnitzler, Arthur} (15.\,5.\,1862 Wien – 21.\,10.\,1931 ebd.), \emph{Schriftsteller, Mediziner}!grüne Kakadu. Groteske in einem Akt@\strich\emph{Der grüne Kakadu. Groteske in einem Akt}|pw}
               proben{ }ſah, \uline{eins der augenfälligſten Talente}, das mir
               in der {\pb}letzten Zeit untergeko{\geminationm}en iſt\substVorne{}\textsuperscript{dſs}\substDazwischen{}. Als\substHinten{}{ }\textsc{Henri}\pwindex{Schnitzler, Arthur 15.\,5.\,1862 Wien – 21.\,10.\,1931 ebd.@\textsc{Schnitzler, Arthur} (15.\,5.\,1862 Wien – 21.\,10.\,1931 ebd.), \emph{Schriftsteller, Mediziner}!grüne Kakadu. Groteske in einem Akt@\strich\emph{Der grüne Kakadu. Groteske in einem Akt}|pwv} ka{\geminationn} er übrigens{ }ſeine Fehler zu Tugenden ausnützen
               (was übrigens auch ein Talent iſt.). Für den \textsc{Filipp}\pwindex{Schnitzler, Arthur 15.\,5.\,1862 Wien – 21.\,10.\,1931 ebd.@\textsc{Schnitzler, Arthur} (15.\,5.\,1862 Wien – 21.\,10.\,1931 ebd.), \emph{Schriftsteller, Mediziner}!grüne Kakadu. Groteske in einem Akt@\strich\emph{Der grüne Kakadu. Groteske in einem Akt}|pw} dürfte ihm wohl das wie{ }ſoll ich{ }ſagen Höfiſche fehlen; aber er iſt{ }ſehr
               lenkſam, und das abſolute{ }ſeiner Begabung innerhalb {\pb}des hier (und anderswo) graſſirenden Mittelmaßes \substVorne{}\textsuperscript{\textcolor{gray}{thut}}\substDazwischen{}müßte\substHinten{} jedem Vernünftigen wohlthun. Seine Ausſprache iſt ja{ }ſehr fremdartig – aber{ }ſobald man{ }ſie gewöhnt, wirkt{ }ſie (auf mich wenigſtens) beinah als ein Reiz mehr.
               Natürlich iſt es denkbar, daſs ihn das Publikum anfangs auslacht. Mit dieſem Troſt
               will ich{ }ſchließen.\pend
           \pstart Ihr \spacefill\mbox{A.}\pend{}\selectlanguage{ngerman}\endnumbering\briefempfaengerindex{Beer-Hofmann, Richard@\textsc{Beer-Hofmann, Richard}!zzzSchnitzler, Arthur@\emph{von Arthur Schnitzler}!1904-11-151@{15. 11. 1904}|)be}\mylabel{L01471h}  \newcommand{\dateiname}{L01471}\newcommand{\titel}{Arthur Schnitzler an Richard Beer-Hofmann, 15. 11. 1904}\newcommand{\editorInnen}{Martin Anton Müller und Gerd-Hermann Susen}%% latex-leseansicht-abspann.tex
%% Abspann für die Leseansicht.
%% Der Schalter \ifkorrekturansicht ist bereits durch den Vorspann gesetzt.

%% latex-abspann.tex
%% Gemeinsamer Abspann für Korrekturansicht und Leseansicht.
%% Setzt den Schalter \ifkorrekturansicht voraus (gesetzt in den
%% einbindenden Dateien latex-korrekturansicht-abspann.tex bzw.
%% latex-leseansicht-abspann.tex).
%% ---------------------------------------------------------------

\normalsize

% Das esempio-Environment wird nur in der Leseansicht benötigt
\ifkorrekturansicht\else
\newenvironment{esempio}[3]%
{
    \vspace{1.5ex}
    \rlap{\underline{#1}}
    \par
    \setlength{\parindent}{0cm}
    \nopagebreak
    \leftskip=#2cm
    \rightskip=#3cm
}
{
    \par
}
\fi

\doendnotes{C}
\bigskip
\vfill

\clearpage

\footnotesize

\ifkorrekturansicht
  \lohead{\textsc{register}}
\fi

% theindex-Environment neu definieren ohne reledmac
\makeatletter
\renewenvironment{theindex}{%
  \ifkorrekturansicht
    \section*{\indexname}%
  \else
    \subsubsection*{Index der erwähnten Entitäten}%
  \fi
  \setlength{\parindent}{0pt}%
  \setlength{\parskip}{0pt plus 0.3pt}%
  \let\item\@idxitem
}{%
  \ifkorrekturansicht\clearpage\fi
}
\makeatother

\IfFileExists{\jobname-pw.ind}{\input{\jobname-pw.ind}}{}

% Quellenangabe nur in der Leseansicht
\ifkorrekturansicht\else
% Fallback-Definitionen, falls die .tex-Datei \titel etc. nicht gesetzt hat
\providecommand{\titel}{}
\providecommand{\editorInnen}{}
\providecommand{\dateiname}{\jobname}

\vspace{3cm}

\vfill

\footnotesize
\textsc{Quelle}: \titel. Herausgegeben von {\editorInnen}. In: \emph{Arthur Schnitzler: Briefwechsel mit Autorinnen und Autoren}.
 Digitale Edition, https://schnitzler-briefe.acdh.oeaw.ac.at/{\dateiname}.html (Stand \today)
\fi

\end{document}


