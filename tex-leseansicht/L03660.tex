%% latex-korrekturansicht-vorspann.tex
%% Vorspann für die Korrekturansicht.
%% Lädt die gemeinsame Datei latex-vorspann.tex mit gesetztem Schalter.

\newif\ifkorrekturansicht
\korrekturansichttrue

\input{../tex-inputs/latex-vorspann}


\section[Stefan Zweig an Arthur Schnitzler, 25. 2. 1917]{L03660 Stefan Zweig an Arthur Schnitzler, 25. 2. 1917}
\nopagebreak\mylabel{L03660v}
\rehead{ }\normalsize\beginnumbering\briefempfaengerindex{Schnitzler, Arthur@\textsc{Schnitzler, Arthur}!zzzZweig, Stefan@\emph{von Stefan Zweig}!1917-02-251@{25. 2. 1917}|(be}
\toendnotes[C]{\smallbreak\pagebreak[2]}\Standort{CUL, Schnitzler, B 118.}
\physDesc{Postkarte, 1 Blatt, 2 Seiten, 883 Zeichen
\newline{}Handschrift: blaue Tinte, lateinische Kurrent
\newline{}Versand: 1) Kleber »Express«  2) Stempel: »\nobreak{}\oindex{Bern@\textbf{Bern}, \emph{P.PPLC}|pwk}Bern
                                       {[}Bri{]}efaufgabe, 25. II. 17, 7\nobreak{}«.  3) Stempel: »\nobreak{}\oindex{Feldkirch@\textbf{Feldkirch}, \emph{P.PPLA2}|pwk}Feldkirch, Zensiert K. u. k. \textcolor{gray}{Zensur} 9\nobreak{}«.  4) Stempel: »\nobreak{}\oindex{XVIII., Waehring@\textbf{XVIII., Währing}, \emph{A.ADM3}|pwk}\textcolor{gray}{18/1} Wien 110, \textcolor{gray}{3}. III. 17, VI\textcolor{gray}{I}\nobreak{}«.  5) Stempel: »\nobreak{}\oindex{XVIII., Waehring@\textbf{XVIII., Währing}, \emph{A.ADM3}|pwk}\textcolor{gray}{18/1} Wien 111, \textcolor{gray}{3}. III. 17, VIII\nobreak{}«. }\toendnotes[C]{\smallbreak}\pstart{}{\pb}Herrn D\textsuperscript{r}
                  Artur Schnitzler\pend{}\pstart{}XVIII. Sternwartestrasse 71\oindex{Sternwartestrasse 71@\textbf{Sternwartestraße 71}, \emph{Wohngebäude (K.WHS)}|pw}.\pend{}\pstart{}Wien\oindex{Wien@\textbf{Wien}, \emph{A.ADM2}|pw}\pend{}\pstart{}Österreich\oindex{Oesterreich@\textbf{Österreich}, \emph{A.PCLI}|pw}.\pend{}{\bigskip}\vspace{1em}
\pstart
           \raggedleft{}{\pb}25. F. 1917.\pend
           \vspace{0.5em}
\pstart
           Verehrter Freund! Nur ein Kartengruss – weil dieser sicherer anko{\geminationm}t. Hier über Erwarten interessantes Leben. Anregungen
               künstlerischer u menschlicher Art. Viele Beziehungen angeknüpft. Von unserer österreichisch-ungarischen Gesandtschaft\orgindex{k.u.k. Oesterreichisch-ungarische Gesandtschaft in der Schweiz@k.u.k. Österreichisch-ungarische Gesandtschaft in der Schweiz|pw}
               ausserordentlich aufgeno{\geminationm}en worden. Vorläufig bleibe ich
               hier. Es ist aber möglich dass ich wenn das Wetter so herrlich schön u warm bleibt
               für einige Zeit an den Genfer See\oindex{Genfer See@\textbf{Genfer See}, \emph{H.LK}|pw} gehe.
                  Gestern fuhr ich nur auf einige Stunden nachOuchy\oindex{Ouchy@\textbf{Ouchy}, \emph{P.PPL}|pw} ein So{\geminationn}enbad nehmen. Aber
               bestimmt weiss ich noch nichts. Über {\pb}die
               Lösung Ihres neuen Dramas\pwindex{Fink und Fliederbusch. Komoedie in drei Akten@\emph{Fink und Fliederbusch. Komödie in drei Akten}|pwv} habe
               ich nachgedacht. Ich fürchte Sie müssen auf den raschen Schluss verzichten und ein
               Jahr zwischen dem \label{K_L03660-1v}\edtext{dritten und den
               letzten Akt}{\lemma{\textnormal{\emph{dritten … Akt}}}\Cendnote{\textnormal{In der publizierten Fassung
                  gibt es nur drei Akte, die ohne größeren Zeitsprung auskommen.}}}\label{K_L03660-1} noch
               verstreichen lassen. Aber selbstverständlich werden Sie als Meister dies besser
               wissen.\pend
           \pstart Olga\pwindex{Schnitzler, Olga 17.01.1882 – 13.01.1970@\textsc{Schnitzler, Olga} (17.01.1882 – 13.01.1970), \emph{Schauspieler/Schauspielerin, Sänger/Sängerin}|pw} u Ihnen \textcolor{gray}{herzlich},
                  \spacefill\mbox{S. Z.}\pend{}\selectlanguage{ngerman}\endnumbering\briefempfaengerindex{Schnitzler, Arthur@\textsc{Schnitzler, Arthur}!zzzZweig, Stefan@\emph{von Stefan Zweig}!1917-02-251@{25. 2. 1917}|)be}\mylabel{L03660h}
\begin{anhang}
\end{anhang}\normalsize

\doendnotes{C}
\bigskip
\vfill

\clearpage

\footnotesize

\lohead{\textsc{register}}

% Definiere theindex-Environment komplett neu ohne reledmac
\makeatletter
\renewenvironment{theindex}{%
  \section*{\indexname}%
  \setlength{\parindent}{0pt}%
  \setlength{\parskip}{0pt plus 0.3pt}%
  \let\item\@idxitem
}{%
  \clearpage
}
\makeatother

\IfFileExists{\jobname-pw.ind}{\input{\jobname-pw.ind}}{}

\end{document}

      