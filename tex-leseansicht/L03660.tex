%% latex-leseansicht-vorspann.tex
%% Vorspann für die Leseansicht.
%% Lädt die gemeinsame Datei latex-vorspann.tex mit nicht gesetztem Schalter.

\newif\ifkorrekturansicht
\korrekturansichtfalse

\input{../tex-inputs/latex-vorspann}


\section[Berta Zuckerkandl an Arthur Schnitzler, 25. 2. 1917]{L03660 Berta Zuckerkandl an Arthur Schnitzler, 25. 2. 1917}
\nopagebreak\mylabel{L03660v}
\rehead{ }\normalsize\beginnumbering\briefempfaengerindex{Schnitzler, Arthur@\textsc{Schnitzler, Arthur}!zzzZuckerkandl, Berta@\emph{von Berta Zuckerkandl}!1917-02-251@{25. 2. 1917}|(be}
\toendnotes[C]{\smallbreak\pagebreak[2]}
\correspDesc{Versand  durch Berta Zuckerkandl am 25. 2. 1917 in Bern
\newline{}Erhalt  durch Arthur Schnitzler am [3. 3. 1917?] in Wien}\toendnotes[C]{\smallbreak}
\Standort{CUL, Schnitzler, B 118.}
\physDesc{Postkarte, 877 Zeichen
\newline{}Handschrift: blaue Tinte, lateinische Kurrent
\newline{}Versand: 1) Kleber »Express«  2) Stempel: »\nobreak{}\oindex{Bern@\textbf{Bern}, \emph{Hauptstadt}|pwk}Bern
                                       {[}Bri{]}efaufgabe, 25. II. 17, 7\nobreak{}«.  3) Stempel: »\nobreak{}\oindex{Feldkirch@\textbf{Feldkirch}, \emph{Hauptstadt}|pwk}Feldkirch, Zensiert K. u. k. \textcolor{gray}{Zensur} 9\nobreak{}«.  4) Stempel: »\nobreak{}\oindex{XVIII., Währing@\textbf{XVIII., Währing}, \emph{Verwaltungsgebiet}|pwk}\textcolor{gray}{18/1} Wien 110, \textcolor{gray}{3}. III. 17, VI\textcolor{gray}{I}\nobreak{}«.  5) Stempel: »\nobreak{}\oindex{XVIII., Währing@\textbf{XVIII., Währing}, \emph{Verwaltungsgebiet}|pwk}\textcolor{gray}{18/1} Wien 111, \textcolor{gray}{3}. III. 17, VIII\nobreak{}«. 
\newline{}Ordnung: Diese Karte findet sich im Nachlass Schnitzlers unter den
                                 Korrespondenzstücken von Stefan Zweig. }\toendnotes[C]{\smallbreak}\pstart{}{\pb}Herrn D\textsuperscript{r}
                  Artur Schnitzler\pend{}\pstart{}XVIII. Sternwartestrasse 71\oindex{Wien@\textbf{Wien}!XVIII., Währing@\textbf{XVIII., Währing}!Sternwartestraße 71@\textbf{Sternwartestraße 71}, \emph{Wohngebäude}|pw}.\pend{}\pstart{}Wien\oindex{Wien@\textbf{Wien}, \emph{Verwaltungsgebiet}|pw}\pend{}\pstart{}Österreich\oindex{Österreich@\textbf{Österreich}|pw}.\pend{}{\bigskip}\vspace{1em}
\pstart
           \raggedleft{}{\pb}25. F. 1917.\pend
           \vspace{0.5em}
\pstart
           Verehrter Freund! Nur ein Kartengruss – weil dieser sicherer anko{\geminationm}t. \label{K_L03660-1v}\edtext{Hier}{\lemma{\textnormal{\emph{Hier}}}\Cendnote{\textnormal{Zweig\pwindex{Zweig, Stefan 28.\,11.\,1881 Wien – 23.\,2.\,1942 Petrópolis@\textsc{Zweig, Stefan} (28.\,11.\,1881 Wien – 23.\,2.\,1942 Petrópolis), \emph{Schriftsteller}|pwk} schrieb am 18. 2. 1917 an Romain Rolland\pwindex{Rolland, Romain 29.\,1.\,1866 Clamecy – 30.\,12.\,1944 Vézelay@\textsc{Rolland, Romain} (29.\,1.\,1866 Clamecy – 30.\,12.\,1944 Vézelay), \emph{Schriftsteller}|pwk}: »Eine liebe Freundin
                     von mir, Frau Berta Zuckerkandl\pwindex{Zuckerkandl, Berta 13.\,4.\,1864 Wien – 16.\,10.\,1945 Paris@\textsc{Zuckerkandl, Berta} (13.\,4.\,1864 Wien – 16.\,10.\,1945 Paris), \emph{Schriftstellerin, Journalistin, Übersetzerin}|pw}, die
                     Witwe des berühmten Anatomen\pwindex{Zuckerkandl, Emil 1.\,9.\,1849 Győr – 28.\,5.\,1910 Wien@\textsc{Zuckerkandl, Emil} (1.\,9.\,1849 Győr – 28.\,5.\,1910 Wien), \emph{Anatom}|pwv}, eine der vollendetsten, gütigsten Frauen, die ich kenne, ist
                     jetzt in der Schweiz\oindex{Schweiz@\textbf{Schweiz}|pw} zur Erholung.
                     Vielleicht begegnen Sie ihr, sie wird mit Oskar Fried\pwindex{Fried, Oscar 1.\,8.\,1871 Berlin – 5.\,7.\,1941 Moskau@\textsc{Fried, Oscar} (1.\,8.\,1871 Berlin – 5.\,7.\,1941 Moskau), \emph{Komponist, Dirigent, Arrangeur}|pw} sein und auch in Genf\oindex{Genf@\textbf{Genf}|pw}.
                     Ich würde glücklich sein, lernten Sie sie kennen: sie ist voll Hingebung für
                     alle großen Dinge und eine Kennerin der Kunst wie wenige: übrigens eine
                     Freundin Rodins\pwindex{Rodin, Auguste 12.\,11.\,1840 Paris – 17.\,11.\,1917 Meudon@\textsc{Rodin, Auguste} (12.\,11.\,1840 Paris – 17.\,11.\,1917 Meudon), \emph{Bildhauer}|pw} und Carrières\pwindex{Carrière, Eugène 16.\,1.\,1849 Gournay-sur-Marne – 27.\,3.\,1906 Paris@\textsc{Carrière, Eugène} (16.\,1.\,1849 Gournay-sur-Marne – 27.\,3.\,1906 Paris), \emph{Maler, Lithograf}|pw}, der sie gemalt hat.« Romain Rolland\pwindex{Rolland, Romain 29.\,1.\,1866 Clamecy – 30.\,12.\,1944 Vézelay@\textsc{Rolland, Romain} (29.\,1.\,1866 Clamecy – 30.\,12.\,1944 Vézelay), \emph{Schriftsteller}|pwk}, Stefan Zweig\pwindex{Zweig, Stefan 28.\,11.\,1881 Wien – 23.\,2.\,1942 Petrópolis@\textsc{Zweig, Stefan} (28.\,11.\,1881 Wien – 23.\,2.\,1942 Petrópolis), \emph{Schriftsteller}|pwk}: \emph{Von Welt zu Welt. Briefe
                        einer Freundschaft 1914–1918}. Mit einem Begleitwort von Peter
                     Handke. Aus dem Französischen von Eva und Gerhard Schwewe (Briefe Rollands) und
                     Christel Gersch (Briefe Zweigs). Berlin: \emph{Aufbau
                        Verlag}{ }2014. Schnitzler traf Zuckerkandl\pwindex{Zuckerkandl, Berta 13.\,4.\,1864 Wien – 16.\,10.\,1945 Paris@\textsc{Zuckerkandl, Berta} (13.\,4.\,1864 Wien – 16.\,10.\,1945 Paris), \emph{Schriftstellerin, Journalistin, Übersetzerin}|pwk} am Tag ihrer Abreise (15. 2. 1917) und zwei Monate später kurz nach ihrer Heimkehr und notierte im \emph{Tagebuch}\pwindex{Schnitzler, Arthur 15. 5. 1862 Wien – 21. 10. 1931 ebd.@\textsc{Schnitzler, Arthur} (15. 5. 1862 Wien – 21. 10. 1931 ebd.), \emph{Schriftsteller, Mediziner}!Tagebuch@\strich\emph{Tagebuch}|pwk} Diverses aus ihren Berichten, vgl. A. S.: \emph{Tagebuch}, 26. 4. 1917.}}}\label{K_L03660-1} über Erwarten interessantes Leben. Anregungen künstlerischer u
               menschlicher Art. Viele Beziehungen angeknüpft. Von unserer österreichisch-ungarischen Gesandtschaft\orgindex{k.u.k. Österreichisch-ungarische Gesandtschaft in der Schweiz@k.u.k. Österreichisch-ungarische Gesandtschaft in der Schweiz|pw} ausserordentlich
                  aufgeno{\geminationm}en worden. Vorläufig bleibe ich hier. Es ist
               aber möglich dass ich wenn das Wetter so herrlich schön u warm bleibt für einige Zeit
               an den Genfer See\oindex{Genfer See@\textbf{Genfer See}, \emph{See}|pw} gehe. Gestern fuhr
               ich nur auf ein paar Stunden nach Ouchy\oindex{Ouchy@\textbf{Ouchy}|pw} ein So{\geminationn}enbad nehmen. Aber bestimmt weiss ich noch nichts. Über
                  {\pb}die Lösung Ihres neuen Drama’s\pwindex{Schnitzler, Arthur 15. 5. 1862 Wien – 21. 10. 1931 ebd.@\textsc{Schnitzler, Arthur} (15. 5. 1862 Wien – 21. 10. 1931 ebd.), \emph{Schriftsteller, Mediziner}!Fink und Fliederbusch. Komödie in drei Akten@\strich\emph{Fink und Fliederbusch. Komödie in drei Akten}|pwv} habe ich nachgedacht. Ich fürchte
               Sie müssen auf den raschen Schluss verzichten und ein Jahr zwischen den \label{K_L03660-2v}\edtext{dritten u den letzten Akt}{\lemma{\textnormal{\emph{dritten … Akt}}}\Cendnote{\textnormal{In der publizierten Fassung von \emph{Fink und Fliederbusch}\pwindex{Schnitzler, Arthur 15. 5. 1862 Wien – 21. 10. 1931 ebd.@\textsc{Schnitzler, Arthur} (15. 5. 1862 Wien – 21. 10. 1931 ebd.), \emph{Schriftsteller, Mediziner}!Fink und Fliederbusch. Komödie in drei Akten@\strich\emph{Fink und Fliederbusch. Komödie in drei Akten}|pwk} gibt es nur drei Akte,
                  die ohne größeren Zeitsprung auskommen.}}}\label{K_L03660-2} noch verstreichen lassen. Aber
               selbstverständlich werden Sie als Meister dies besser wissen.\pend
           \pstart Olga\pwindex{Schnitzler, Olga 17.\,1.\,1882 Wien – 13.\,1.\,1970 Lugano@\textsc{Schnitzler, Olga} (17.\,1.\,1882 Wien – 13.\,1.\,1970 Lugano), \emph{Schauspielerin, Sängerin}|pw} u Ihnen \textcolor{gray}{herzlich},
                  \spacefill\mbox{B. Z.}\pend{}\selectlanguage{ngerman}\endnumbering\briefempfaengerindex{Schnitzler, Arthur@\textsc{Schnitzler, Arthur}!zzzZuckerkandl, Berta@\emph{von Berta Zuckerkandl}!1917-02-251@{25. 2. 1917}|)be}\mylabel{L03660h}
\begin{anhang}
\end{anhang}\newcommand{\dateiname}{L03660}\newcommand{\titel}{Berta Zuckerkandl an Arthur Schnitzler, 25. 2. 1917}\newcommand{\editorInnen}{Selma Jahnke und Martin Anton Müller}%% latex-leseansicht-abspann.tex
%% Abspann für die Leseansicht.
%% Der Schalter \ifkorrekturansicht ist bereits durch den Vorspann gesetzt.

%% latex-abspann.tex
%% Gemeinsamer Abspann für Korrekturansicht und Leseansicht.
%% Setzt den Schalter \ifkorrekturansicht voraus (gesetzt in den
%% einbindenden Dateien latex-korrekturansicht-abspann.tex bzw.
%% latex-leseansicht-abspann.tex).
%% ---------------------------------------------------------------

\normalsize

% Das esempio-Environment wird nur in der Leseansicht benötigt
\ifkorrekturansicht\else
\newenvironment{esempio}[3]%
{
    \vspace{1.5ex}
    \rlap{\underline{#1}}
    \par
    \setlength{\parindent}{0cm}
    \nopagebreak
    \leftskip=#2cm
    \rightskip=#3cm
}
{
    \par
}
\fi

\doendnotes{C}
\bigskip
\vfill

\clearpage

\footnotesize

\ifkorrekturansicht
  \lohead{\textsc{register}}
\fi

% theindex-Environment neu definieren ohne reledmac
\makeatletter
\renewenvironment{theindex}{%
  \ifkorrekturansicht
    \section*{\indexname}%
  \else
    \subsubsection*{Index der erwähnten Entitäten}%
  \fi
  \setlength{\parindent}{0pt}%
  \setlength{\parskip}{0pt plus 0.3pt}%
  \let\item\@idxitem
}{%
  \ifkorrekturansicht\clearpage\fi
}
\makeatother

\IfFileExists{\jobname-pw.ind}{\input{\jobname-pw.ind}}{}

% Quellenangabe nur in der Leseansicht
\ifkorrekturansicht\else
% Fallback-Definitionen, falls die .tex-Datei \titel etc. nicht gesetzt hat
\providecommand{\titel}{}
\providecommand{\editorInnen}{}
\providecommand{\dateiname}{\jobname}

\vspace{3cm}

\vfill

\footnotesize
\textsc{Quelle}: \titel. Herausgegeben von {\editorInnen}. In: \emph{Arthur Schnitzler: Briefwechsel mit Autorinnen und Autoren}.
 Digitale Edition, https://schnitzler-briefe.acdh.oeaw.ac.at/{\dateiname}.html (Stand \today)
\fi

\end{document}


