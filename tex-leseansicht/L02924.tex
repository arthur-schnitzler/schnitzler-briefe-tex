%% latex-leseansicht-vorspann.tex
%% Vorspann für die Leseansicht.
%% Lädt die gemeinsame Datei latex-vorspann.tex mit nicht gesetztem Schalter.

\newif\ifkorrekturansicht
\korrekturansichtfalse

\input{../tex-inputs/latex-vorspann}


         
         \renewcommand{\erwaehntePersonen}{Personen: Paul Goldmann, Georg Hirschfeld, Alfred Kerr}
         \renewcommand{\erwaehnteInstitutionen}{Institutionen: Berliner Theater, Schauspielhaus Berlin}
         \renewcommand{\erwaehnteOrte}{Orte: Berlin, China, Dessauer Straße, Innsbruck, Lago di Garda, Reichenau an der Rax, Riva del Garda}
         \renewcommand{\erwaehnteWerke}{Werke: Der Schleier der Beatrice. Schauspiel in fünf Akten}
               \section[ Paul Goldmann an Arthur Schnitzler, 18. 7. {[}1900{]}]{ Paul Goldmann an Arthur Schnitzler, 18. 7. {[}1900{]}}\nopagebreak\mylabel{v}\rehead{ }\begin{ledgroupsized}[t]{13cm}\normalsize\beginnumbering\briefempfaengerindex{Schnitzler, Arthur@\textsc{Schnitzler, Arthur}!zzzGoldmann, Paul@\emph{von Paul Goldmann}!1900-07-182@{18. 7. {[}1900{]}}|(be} \toendnotes[C]{\smallbreak\pagebreak[2]} \Standort{DLA, A:Schnitzler, HS.NZ85.1.3170.}
\physDesc{Brief, 1 Blatt, 3 Seiten, 1025 Zeichen
\newline{}Handschrift: blaue Tinte, deutsche Kurrent
\newline{}Schnitzler: 1) mit Bleistift das Jahr »90\textcolor{gray}{0}« vermerkt  2) mit rotem Buntstift zwei Unterstreichungen}\toendnotes[C]{\smallbreak}\pstart
           \noindent{}\raggedleft{}{\pb}\textcolor{gray}{\textbf{DESSAUERSTRASSE 19}}\oindex{Dessauer Strasse@\textbf{Dessauer Straße}|pw}\pend
           \pstart
           Berlin\oindex{Berlin@\textbf{Berlin}|pw}, 18. Juli.\pend
           \pstart\center{}Mein lieber Freund,\pend\pstart
           Mit der \label{K_L02924-1v}\edtext{Fußparthie}{\lemma{\textnormal{\emph{Fußparthie}}}\Cendnote{\textnormal{siehe Paul Goldmann an Arthur Schnitzler, 16. 6. [1900]}}}\label{K_L02924-1h}, wie Du ſie entworfen haſt, und mit dem \label{K_L02924-2v}\edtext{Zuſammentreffen in \textsc{Innsbruck\oindex{Innsbruck@\textbf{Innsbruck}|pw}}}{\lemma{\textnormal{\emph{Zuſammentreffen in Innsbruck}}}\Cendnote{\textnormal{siehe A. S.: \emph{Tagebuch}, 16. 8. 1900}}}\label{K_L02924-2h} bin ich einverſtanden, – vorausgeſetzt, daß ich überhaupt fortkomme, was
               durch die \label{K_L02924-3v}\edtext{chin\oindex{China@\textbf{China}|pwv}eſiſchen Ereigniſſe}{\lemma{\textnormal{\emph{chineſiſchen Ereigniſſe}}}\Cendnote{\textnormal{siehe Paul Goldmann an Arthur Schnitzler, 5. 7. [1900]}}}\label{K_L02924-3h} immer fraglicher wird. Ich habe noch nicht einmal um Urlaub geſchrieben.
               Immerhin hoffe ich, zum 15. Auguſt fortzukommen. Laß’
               mich Deine Adreſſe wiſſen, damit ich Dir das Nähere telegraphiſch oder brieflich
               mittheilen kann.\pend
           \pstart
           Von \textsc{Kerr\pwindex{Kerr, Alfred 25.12.1867 – 12.10.1948@\textsc{Kerr, Alfred} (25.12.1867 – 12.10.1948), \emph{Schriftsteller, Kritiker}|pw}} hatte ich heut eine Karte mit der Bitte, ihm
               nach \textsc{Riva\oindex{Riva del Garda@\textbf{Riva del Garda}|pw}} (Gardaſee\oindex{Lago di Garda@\textbf{Lago di Garda}|pw}) zu {\pb}ſchreiben. Er ſagt, er erwarte von Dir Nachricht,
               und wird jedenfalls pünktlich beim \textsc{Rendezvous} in \textsc{Innsbruck\oindex{Innsbruck@\textbf{Innsbruck}|pw}} ſein. \strikeout{\textcolor{gray}{×}\-\textcolor{gray}{×}\-\textcolor{gray}{×}\-\textcolor{gray}{×}\-\textcolor{gray}{×}\-\textcolor{gray}{×}\-\textcolor{gray}{×}\-\textcolor{gray}{×}\-\textcolor{gray}{×}\-\textcolor{gray}{×}\-\textcolor{gray}{×}\-\textcolor{gray}{×}\-\textcolor{gray}{×}\-\textcolor{gray}{×}\-\textcolor{gray}{×}\-\textcolor{gray}{×}} Bitte, \label{K_L02924-4v}\edtext{ſchreib’ ihm
                  ſofort}{\lemma{\textnormal{\emph{ſchreib’ ihm
                  ſofort}}}\Cendnote{\textnormal{nicht überliefert}}}\label{K_L02924-4h}.\pend
           \pstart
           Daß \label{K_L02924-5v}\edtext{\textsc{Hirschfeld\pwindex{Hirschfeld, Georg 11.02.1873 – 17.01.1942@\textsc{Hirschfeld, Georg} (11.02.1873 – 17.01.1942), \emph{Schriftsteller}|pw}} mitgeht}{\lemma{\textnormal{\emph{Hirschfeld mitgeht}}}\Cendnote{\textnormal{Das ist nicht geschehen,
                     vgl. Arthur Schnitzler an Richard Beer-Hofmann, 3. 8. 1900. Schnitzler\pwindex{Schnitzler, Arthur 15.05.1862 – 21.10.1931@\textsc{Schnitzler, Arthur} (15.05.1862 – 21.10.1931), \emph{Schriftsteller, Mediziner}|pwk} hatte Georg
                     Hirschfeld\pwindex{Hirschfeld, Georg 11.02.1873 – 17.01.1942@\textsc{Hirschfeld, Georg} (11.02.1873 – 17.01.1942), \emph{Schriftsteller}|pwk} am 28. 6. 1900 und am 29. 6. 1900 getroffen und dabei wohl eine
                  mögliche Teilnahme an der gemeinsamen Wanderung angesprochen.}}}\label{K_L02924-5h}, iſt mir
               nicht ſympathiſch. Er ſoll doch lieber zu Hauſe bleiben und \label{K_L02924-6v}\edtext{»\textsc{Milieu}-Stücke«}{\lemma{\textnormal{\emph{»Milieu-Stücke«}}}\Cendnote{\textnormal{siehe Paul Goldmann an Arthur Schnitzler, 21. 6. [1900]}}}\label{K_L02924-6h} ſchreiben.\pend
           \pstart
           Wenn das \label{K_L02924-7v}\edtext{Schauſpielhaus\orgindex{Schauspielhaus Berlin@Schauspielhaus Berlin|pw} Dein Stück\pwindex{Schnitzler, Arthur 15.05.1862 – 21.10.1931@\textsc{Schnitzler, Arthur} (15.05.1862 – 21.10.1931), \emph{Schriftsteller, Mediziner}!Schleier der Beatrice. Schauspiel in fuenf Akten1900-12-01@\strich\emph{Der Schleier der Beatrice. Schauspiel in fünf Akten} {[}1900-12-01{]}|pwv} refüſiren}{\lemma{\textnormal{\emph{Schauſpielhaus … refüſiren}}}\Cendnote{\textnormal{siehe Paul Goldmann an Arthur Schnitzler, 5. 7. [1900]}}}\label{K_L02924-7h} ſollte, was noch gar nicht ausgemacht iſt, ſo verſuchen wir es beim Berliner Theater\orgindex{Berliner Theater@Berliner Theater|pw}, wo ich die Annahme für ſicher
               halte.\pend
           \pstart
           {\pb}Für heut nur dieſes
               Wenige. Ich habe unmenſchlich viel zu thun.\pend
           \pstart
           Viele treue Grüße! {\\[\baselineskip]}Dein {\\[\baselineskip]}\spacefill\mbox{Paul Goldmann.}\pend
           \leftskip=0em{}
         
         \endnumbering\mylabel{h}\end{ledgroupsized}  \newcommand{\dateiname}{L02924}\newcommand{\titel}{Paul Goldmann an Arthur Schnitzler, 18. 7. [1900]}\newcommand{\editorInnen}{Martin Anton Müller und Laura Untner}%% latex-leseansicht-abspann.tex
%% Abspann für die Leseansicht.
%% Der Schalter \ifkorrekturansicht ist bereits durch den Vorspann gesetzt.

%% latex-abspann.tex
%% Gemeinsamer Abspann für Korrekturansicht und Leseansicht.
%% Setzt den Schalter \ifkorrekturansicht voraus (gesetzt in den
%% einbindenden Dateien latex-korrekturansicht-abspann.tex bzw.
%% latex-leseansicht-abspann.tex).
%% ---------------------------------------------------------------

\normalsize

% Das esempio-Environment wird nur in der Leseansicht benötigt
\ifkorrekturansicht\else
\newenvironment{esempio}[3]%
{
    \vspace{1.5ex}
    \rlap{\underline{#1}}
    \par
    \setlength{\parindent}{0cm}
    \nopagebreak
    \leftskip=#2cm
    \rightskip=#3cm
}
{
    \par
}
\fi

\doendnotes{C}
\bigskip
\vfill

\clearpage

\footnotesize

\ifkorrekturansicht
  \lohead{\textsc{register}}
\fi

% theindex-Environment neu definieren ohne reledmac
\makeatletter
\renewenvironment{theindex}{%
  \ifkorrekturansicht
    \section*{\indexname}%
  \else
    \subsubsection*{Index der erwähnten Entitäten}%
  \fi
  \setlength{\parindent}{0pt}%
  \setlength{\parskip}{0pt plus 0.3pt}%
  \let\item\@idxitem
}{%
  \ifkorrekturansicht\clearpage\fi
}
\makeatother

\IfFileExists{\jobname-pw.ind}{\input{\jobname-pw.ind}}{}

% Quellenangabe nur in der Leseansicht
\ifkorrekturansicht\else
% Fallback-Definitionen, falls die .tex-Datei \titel etc. nicht gesetzt hat
\providecommand{\titel}{}
\providecommand{\editorInnen}{}
\providecommand{\dateiname}{\jobname}

\vspace{3cm}

\vfill

\footnotesize
\textsc{Quelle}: \titel. Herausgegeben von {\editorInnen}. In: \emph{Arthur Schnitzler: Briefwechsel mit Autorinnen und Autoren}.
 Digitale Edition, https://schnitzler-briefe.acdh.oeaw.ac.at/{\dateiname}.html (Stand \today)
\fi

\end{document}


      