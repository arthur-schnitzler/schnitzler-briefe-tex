%% latex-korrekturansicht-vorspann.tex
%% Vorspann für die Korrekturansicht.
%% Lädt die gemeinsame Datei latex-vorspann.tex mit gesetztem Schalter.

\newif\ifkorrekturansicht
\korrekturansichttrue

\input{../tex-inputs/latex-vorspann}


\section[Johann Schnitzler an Karl Emil Franzos, 4. 4. 1888]{L03617 Johann Schnitzler an Karl Emil Franzos, 4. 4. 1888}
\nopagebreak\mylabel{L03617v}
\rehead{ }\normalsize\beginnumbering\briefempfaengerindex{Franzos, Karl Emil@\textsc{Franzos, Karl Emil}!zzzSchnitzler, Johann@\emph{von Johann Schnitzler}!1888-04-041@{4. 4. 1888}|(be}
\toendnotes[C]{\smallbreak\pagebreak[2]}\Standort{Wienbibliothek im Rathaus, H.I.N.-110.962.}
\physDesc{Brief, 1 Blatt, 1 Seite, 292 Zeichen
\newline{}Handschrift: schwarze Tinte, deutsche Kurrent}\toendnotes[C]{\smallbreak}
\pstart
           {\pb}\textcolor{gray}{\textbf{Professor}}{\\}\textcolor{gray}{\textbf{D\textsuperscript{r.} Joh. Schnitzler}}{\\}\textcolor{gray}{\textbf{Wien}}\oindex{Wien@\textbf{Wien}, \emph{A.ADM2}|pw}{\\}\textcolor{gray}{\textbf{I., Burgring 1\oindex{Wohnung und Ordination Johann Schnitzler Burgring 1@\textbf{Wohnung und Ordination Johann Schnitzler Burgring 1}, \emph{Ordination}|pw}. }}\pend
           
\pstart{}Sehr geehrter Herr u Freund!\pend\vspace{0.5em}
\pstart
           Ich erlaube mir hiemit Ihnen mein{[}en{]} Sohn – \textsc{Arthur} – der zur Vervollſtändigung ſeiner Studien \label{K_L03617-1v}\edtext{einige Monate}{\lemma{\textnormal{\emph{einige Monate}}}\Cendnote{\textnormal{Schnitzler reiste am 5. 4. 1888 nach Berlin\oindex{Berlin@\textbf{Berlin}, \emph{P.PPLC}|pwk}, blieb dort allerdings nur bis zum 12. 5. 1888.}}}\label{K_L03617-1} in Berlin\oindex{Berlin@\textbf{Berlin}, \emph{P.PPLC}|pw} zubringen ſoll, vorzuſtellen u denſelben
               Ihrer freundlichen Aufnahme beſtens zu empfehlen.\pend
           
\pstart
           Mit herzlichſten Grüßen{\\[\baselineskip]}Ihr aufrichtig ergebener{\\[\baselineskip]}\spacefill\mbox{D\textsuperscript{r}Schnitzler}\pend
           \leftskip=0em{}
\pstart
           4/4 88\pend
           \selectlanguage{ngerman}\endnumbering\briefempfaengerindex{Franzos, Karl Emil@\textsc{Franzos, Karl Emil}!zzzSchnitzler, Johann@\emph{von Johann Schnitzler}!1888-04-041@{4. 4. 1888}|)be}\mylabel{L03617h}  \normalsize

\doendnotes{C}
\bigskip
\vfill

\clearpage

\footnotesize

\lohead{\textsc{register}}

% Definiere theindex-Environment komplett neu ohne reledmac
\makeatletter
\renewenvironment{theindex}{%
  \section*{\indexname}%
  \setlength{\parindent}{0pt}%
  \setlength{\parskip}{0pt plus 0.3pt}%
  \let\item\@idxitem
}{%
  \clearpage
}
\makeatother

\IfFileExists{\jobname-pw.ind}{\input{\jobname-pw.ind}}{}

\end{document}

      