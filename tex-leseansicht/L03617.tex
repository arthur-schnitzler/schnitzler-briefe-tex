%% latex-leseansicht-vorspann.tex
%% Vorspann für die Leseansicht.
%% Lädt die gemeinsame Datei latex-vorspann.tex mit nicht gesetztem Schalter.

\newif\ifkorrekturansicht
\korrekturansichtfalse

\input{../tex-inputs/latex-vorspann}


\section[Johann Schnitzler an Karl Emil Franzos, 4. 4. 1888]{L03617 Johann Schnitzler an Karl Emil Franzos, 4. 4. 1888}
\nopagebreak\mylabel{L03617v}
\rehead{ }\normalsize\beginnumbering\briefempfaengerindex{Franzos, Karl Emil@\textsc{Franzos, Karl Emil}!zzzSchnitzler, Johann@\emph{von Johann Schnitzler}!1888-04-041@{4. 4. 1888}|(be}
\toendnotes[C]{\smallbreak\pagebreak[2]}
\correspDesc{Versand  durch Johann Schnitzler am 4. 4. 1888 in Wien
\newline{}Erhalt  durch Karl Emil Franzos am 15. 4. 1888 in Berlin}\toendnotes[C]{\smallbreak}
\Standort{Wienbibliothek im Rathaus, H.I.N.-110.962.}
\physDesc{Brief, 1 Blatt, 1 Seite, 292 Zeichen
\newline{}Handschrift: schwarze Tinte, deutsche Kurrent}\toendnotes[C]{\smallbreak}
\pstart
           {\pb}\textcolor{gray}{\textbf{Professor}}{\\}\textcolor{gray}{\textbf{D\textsuperscript{r.} Joh. Schnitzler}}{\\}\textcolor{gray}{\textbf{Wien}}\oindex{Wien@\textbf{Wien}, \emph{Verwaltungsgebiet}|pw}{\\}\textcolor{gray}{\textbf{I., Burgring 1\oindex{Wien@\textbf{Wien}!I., Innere Stadt@\textbf{I., Innere Stadt}!Wohnung und Ordination Johann Schnitzler Burgring 1@\textbf{Wohnung und Ordination Johann Schnitzler Burgring 1}, \emph{Ordination}|pw}.}}\pend
           
\pstart{}Sehr geehrter Herr u Freund!\pend\vspace{0.5em}
\pstart
           Ich erlaube mir hiemit Ihnen mein{[}en{]} Sohn – \textsc{Arthur} – der zur Vervollſtändigung{ }ſeiner Studien \label{K_L03617-1v}\edtext{einige Monate}{\lemma{\textnormal{\emph{einige Monate}}}\Cendnote{\textnormal{Schnitzler reiste am 5. 4. 1888 nach Berlin\oindex{Berlin@\textbf{Berlin}, \emph{Hauptstadt}|pwk}, blieb dort allerdings nur bis zum 12. 5. 1888.}}}\label{K_L03617-1} in Berlin\oindex{Berlin@\textbf{Berlin}, \emph{Hauptstadt}|pw} zubringen{ }ſoll, vorzuſtellen u denſelben
               Ihrer freundlichen Aufnahme beſtens zu empfehlen.\pend
           
\pstart
           Mit herzlichſten Grüßen{\\[\baselineskip]}Ihr aufrichtig ergebener{\\[\baselineskip]}\spacefill\mbox{D\textsuperscript{r}Schnitzler}\pend
           \leftskip=0em{}
\pstart
           4/4 88\pend
           \selectlanguage{ngerman}\endnumbering\briefempfaengerindex{Franzos, Karl Emil@\textsc{Franzos, Karl Emil}!zzzSchnitzler, Johann@\emph{von Johann Schnitzler}!1888-04-041@{4. 4. 1888}|)be}\mylabel{L03617h}  \newcommand{\dateiname}{L03617}\newcommand{\titel}{Johann Schnitzler an Karl Emil Franzos, 4. 4. 1888}\newcommand{\editorInnen}{Selma Jahnke und Martin Anton Müller}%% latex-leseansicht-abspann.tex
%% Abspann für die Leseansicht.
%% Der Schalter \ifkorrekturansicht ist bereits durch den Vorspann gesetzt.

%% latex-abspann.tex
%% Gemeinsamer Abspann für Korrekturansicht und Leseansicht.
%% Setzt den Schalter \ifkorrekturansicht voraus (gesetzt in den
%% einbindenden Dateien latex-korrekturansicht-abspann.tex bzw.
%% latex-leseansicht-abspann.tex).
%% ---------------------------------------------------------------

\normalsize

% Das esempio-Environment wird nur in der Leseansicht benötigt
\ifkorrekturansicht\else
\newenvironment{esempio}[3]%
{
    \vspace{1.5ex}
    \rlap{\underline{#1}}
    \par
    \setlength{\parindent}{0cm}
    \nopagebreak
    \leftskip=#2cm
    \rightskip=#3cm
}
{
    \par
}
\fi

\doendnotes{C}
\bigskip
\vfill

\clearpage

\footnotesize

\ifkorrekturansicht
  \lohead{\textsc{register}}
\fi

% theindex-Environment neu definieren ohne reledmac
\makeatletter
\renewenvironment{theindex}{%
  \ifkorrekturansicht
    \section*{\indexname}%
  \else
    \subsubsection*{Index der erwähnten Entitäten}%
  \fi
  \setlength{\parindent}{0pt}%
  \setlength{\parskip}{0pt plus 0.3pt}%
  \let\item\@idxitem
}{%
  \ifkorrekturansicht\clearpage\fi
}
\makeatother

\IfFileExists{\jobname-pw.ind}{\input{\jobname-pw.ind}}{}

% Quellenangabe nur in der Leseansicht
\ifkorrekturansicht\else
% Fallback-Definitionen, falls die .tex-Datei \titel etc. nicht gesetzt hat
\providecommand{\titel}{}
\providecommand{\editorInnen}{}
\providecommand{\dateiname}{\jobname}

\vspace{3cm}

\vfill

\footnotesize
\textsc{Quelle}: \titel. Herausgegeben von {\editorInnen}. In: \emph{Arthur Schnitzler: Briefwechsel mit Autorinnen und Autoren}.
 Digitale Edition, https://schnitzler-briefe.acdh.oeaw.ac.at/{\dateiname}.html (Stand \today)
\fi

\end{document}


