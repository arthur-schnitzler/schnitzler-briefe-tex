%% latex-leseansicht-vorspann.tex
%% Vorspann für die Leseansicht.
%% Lädt die gemeinsame Datei latex-vorspann.tex mit nicht gesetztem Schalter.

\newif\ifkorrekturansicht
\korrekturansichtfalse

\input{../tex-inputs/latex-vorspann}


         
         \renewcommand{\erwaehntePersonen}{Personen: Ludwig Anzengruber, Hermann Bahr, Rosa Bahr, Richard Beer-Hofmann, Josef von Bezecný, Lou Brion, Max Eugen Burckhard, Jakob Julius David, Leo Ebermann, Alfred Gold, Olga von Golovin, Hugo August von Hofmannsthal, Anna von Hofmannsthal, Hugo von Hofmannsthal, Paul Horn, Heinrich Laube, Felix Salten, Adele Sandrock, Gustav Schwarzkopf, Ludwig Speidel}
         \renewcommand{\erwaehnteOrte}{Orte: Bruck an der Mur, Burgtheater, Hodonín, IX., Alsergrund, Riva del Garda, Russland, Schönberg im Stubaital, Tegetthoff-Denkmal, Tirol, Venedig, Volkstheater, Wien}
         \renewcommand{\erwaehnteWerke}{Werke: Alte Wiener, Anatol, Böse Zungen, Der Tod Georgs, Ein Regentag. Charakterbild, Liebelei. Schauspiel in drei Akten}
               \section[Arthur Schnitzler an Richard Beer-Hofmann, 15. 9. 1895]{ Arthur Schnitzler an Richard Beer-Hofmann, 15. 9. 1895}\nopagebreak\mylabel{v}\rehead{ }\begin{ledgroupsized}[t]{13cm}\normalsize\beginnumbering \toendnotes[C]{\smallbreak\pagebreak[2]} \Standort{YCGL, MSS 31.}
\physDesc{Brief, 2 Blätter, 7 Seiten, , , , Umschlag
\newline{}Handschrift: 1) Bleistift, deutsche Kurrent\hspace{1em}2) schwarze Tinte, deutsche Kurrent (\noindent{}Adressierung)\hspace{1em}\newline{}Versand: 1) Stempel: »\nobreak{}\oindex{IX., Alsergrund@\textbf{IX., Alsergrund}|pwk}Wien 9/3, 16. 9. 95, 6–7 V\nobreak{}«.   2) Stempel: »\nobreak{}\oindex{Schoenberg im Stubaital@\textbf{Schönberg im Stubaital}|pwk}{\pb}{[}Sch{]}önb{[}e{]}rg\nobreak{}«. }\buchAbdrucke{\weitereDrucke{1) Arthur Schnitzler: \emph{Briefe 1875–1912}. Hg. Therese Nickl und Heinrich Schnitzler. Frankfurt am Main: \emph{S. Fischer} 1981, S. 277–278.} \weitereDrucke{2) Arthur Schnitzler, Richard Beer-Hofmann: \emph{Briefwechsel 1891–1931}. Hg. Konstanze Fliedl. Wien, Zürich: \emph{Europaverlag} 1992, S. 80–81.} \weitereDrucke{3) Hermann Bahr, Arthur Schnitzler: \emph{Briefwechsel, Aufzeichnungen, Dokumente (1891–1931)}. Hg. Kurt Ifkovits und Martin Anton Müller. Göttingen: \emph{Wallstein} 2018.} }\toendnotes[C]{\smallbreak}\pstart{}{\pb}Herrn Dr. \textsc{Richard
                     Beer-Hofmann}\pend{}\pstart{}\textsc{Schönberg im Stubaithal\oindex{Schoenberg im Stubaital@\textbf{Schönberg im Stubaital}|pw}}\pend{}\pstart{}\textsc{Tirol\oindex{Tirol@\textbf{Tirol}|pw}}\pend{}{\bigskip}\pstart
           \raggedleft{}{\pb}So{\geminationn}tg 15. 9. 95.\pend
           \pstart
           Lieber Richard. Ich freue mich, daſs Sie in guter Sti{\geminationm}ung ſind. Wahrscheinlich werden Sie bald südlicher
               gehn; kennen Sie \textsc{Riva\oindex{Riva del Garda@\textbf{Riva del Garda}|pw}}? Es iſt ſchön, war \introOben{}mir\introOben{} aber nicht ſympathiſch. Ich bin
               von dort nach Venedig\oindex{Venedig@\textbf{Venedig}|pw} gegangen; es iſt so nah. Sie
               haben \uline{mich} falſch verſtanden; ich wußte, dſs Sie Ende
               Sept. in Wien\oindex{Wien@\textbf{Wien}|pw}{ }ſein wollten. An dieſes Wien\oindex{Wien@\textbf{Wien}|pw} hab ich mich noch nicht ganz gewöhnt; empfinde gleich wieder, jetzt wo
               die alten Verhältniſſe sich aufdrängen, das vielfach unzulängliche, unter dem man zu
               leiden hat. Dünne Fäden, mit denen {\pb}man an mancherlei
               gebunden iſt – dünn, aber doch Fäden. Denken Sie, ſeit ich hier bin, bin ich bereits
               2mal in der früh \introOben{}(um 6 oder ½ 7)\introOben{} geweckt worden – von
               Patienten, nicht vom Burgtheater\oindex{Burgtheater@\textbf{Burgtheater}|pw}. – Am Mittwoch 18.
               ſoll Leſeprobe\pwindex{Schnitzler, Arthur 15.05.1862 – 21.10.1931@\textsc{Schnitzler, Arthur} (15.05.1862 – 21.10.1931), \emph{Schriftsteller, Mediziner}!Liebelei. Schauspiel in drei Akten1895-10-09@\strich\emph{Liebelei. Schauspiel in drei Akten} {[}1895-10-09{]}|pwv}{ }ſein; wenigſtens ist sie angesetzt.\pend
           \pstart
           – Die S.\pwindex{Sandrock, Adele 1863-08-19 – 1937-08-30@\textsc{Sandrock, Adele} (1863-08-19 – 1937-08-30), \emph{Schauspielerin}|pw} verhält ſich ſtille; ihre Feindſeligkeit
               hat ſie vorläufig nur dadurch ausgedrückt, daſs ſie ihrer ruſſiſchen\oindex{Russland@\textbf{Russland}|pw}{ }Freundin\pwindex{Golovin, Olga von @\textsc{Golovin, Olga von}|pwv} einen Brief ſchrieb,
               ſie dürfe \uline{mich} nicht mehr als Arzt nehmen, wenn ſie
               mit ihr verkehren wolle. Die ruſſiſche\oindex{Russland@\textbf{Russland}|pw}{ }Freundin\pwindex{Golovin, Olga von @\textsc{Golovin, Olga von}|pwv} kümmert ſich nicht
               drum {\pb}und läßt ſich mit Begeiſterung von mir
               behandeln. – \textsc{Bckhrd}\pwindex{Burckhard, Max Eugen 14.07.1854 – 16.03.1912@\textsc{Burckhard, Max Eugen} (14.07.1854 – 16.03.1912), \emph{Schriftsteller, Rechtswissenschaftler, Theaterleiter}|pw}{ }ſprach neulich das erſte Mal von der Sache: »Ich
               hab ja nur zufällig durch den Bahr\pwindex{Bahr, Hermann 19.07.1863 – 15.01.1934@\textsc{Bahr, Hermann} (19.07.1863 – 15.01.1934), \emph{Schriftsteller, Kritiker}|pw} von der Sache
               erfahren {\dotstwo} aber ich werd ihr ſchon begreiflich machen,
               daſs das beim Burgtheater\oindex{Burgtheater@\textbf{Burgtheater}|pw} nicht geht – beſonders \uline{ſie}{\dots} Freilich mit Ketten kann ich ſie nicht auf die Bühne
               zerren.« – Man war bei \textsc{Besezny}\pwindex{Bezecný, Josef von 05.02.1829 – 17.06.1904@\textsc{Bezecný, Josef von} (05.02.1829 – 17.06.1904), \emph{Theaterintendant}|pw}, ihm erzählen, wie du{\geminationm} und ordinär mein Stück\pwindex{Schnitzler, Arthur 15.05.1862 – 21.10.1931@\textsc{Schnitzler, Arthur} (15.05.1862 – 21.10.1931), \emph{Schriftsteller, Mediziner}!Liebelei. Schauspiel in drei Akten1895-10-09@\strich\emph{Liebelei. Schauspiel in drei Akten} {[}1895-10-09{]}|pwv}{ }ſei. – Unser Freund J. J. David\pwindex{David, Jakob Julius 1859-02-06 – 1906-11-20@\textsc{David, Jakob Julius} (1859-02-06 – 1906-11-20), \emph{Schriftsteller, Journalist}|pw}: Ich werde \label{K_L00483_1v}\edtext{vielleicht durch{\pb}fallen\pwindex{David, Jakob Julius 1859-02-06 – 1906-11-20@\textsc{David, Jakob Julius} (1859-02-06 – 1906-11-20), \emph{Schriftsteller, Journalist}!Regentag. Charakterbild12. 10. 1895@\strich\emph{Ein Regentag. Charakterbild} {[}12. 10. 1895{]}|pwv}}{\lemma{\textnormal{\emph{vielleicht durchfallen}}}\Cendnote{\textnormal{\emph{Ein Regentag}\pwindex{David, Jakob Julius 1859-02-06 – 1906-11-20@\textsc{David, Jakob Julius} (1859-02-06 – 1906-11-20), \emph{Schriftsteller, Journalist}!Regentag. Charakterbild12. 10. 1895@\strich\emph{Ein Regentag. Charakterbild} {[}12. 10. 1895{]}|pwk}; Uraufführung im Deutschen Volkstheater\oindex{Volkstheater@\textbf{Volkstheater}|pwk} am 12. 10. 1895}}}\label{K_L00483_1h},
               der \textsc{Schnitzler} aber doch ganz gewiſs. –\pend
           \pstart
           – \textsc{Speidel}\pwindex{Speidel, Ludwig 1830-04-11 – 1906-02-03@\textsc{Speidel, Ludwig} (1830-04-11 – 1906-02-03), \emph{Journalist, Kritiker}|pw} zu \textsc{Eberma{\geminationn}}\pwindex{Ebermann, Leo 16.07.1863 – 09.10.1914@\textsc{Ebermann, Leo} (16.07.1863 – 09.10.1914), \emph{Schriftsteller, Journalist, Rechtswissenschaftler}|pw} über die Liebelei\pwindex{Schnitzler, Arthur 15.05.1862 – 21.10.1931@\textsc{Schnitzler, Arthur} (15.05.1862 – 21.10.1931), \emph{Schriftsteller, Mediziner}!Liebelei. Schauspiel in drei Akten1895-10-09@\strich\emph{Liebelei. Schauspiel in drei Akten} {[}1895-10-09{]}|pw} – »Da werden die Wiener\oindex{Wien@\textbf{Wien}|pw}{ }ſchaun!« – Iſt vom Anatol\pwindex{Schnitzler, Arthur 15.05.1862 – 21.10.1931@\textsc{Schnitzler, Arthur} (15.05.1862 – 21.10.1931), \emph{Schriftsteller, Mediziner}!Anatol1892-10-29@\strich\emph{Anatol} {[}1892-10-29{]}|pw} äußerst – (ich genire mich »entzückt« zu ſchreiben.) – Theater: Alte Wiener\pwindex{Anzengruber, Ludwig 29.11.1839 – 10.12.1889@\textsc{Anzengruber, Ludwig} (29.11.1839 – 10.12.1889), \emph{Schriftsteller}!Alte Wiener1879@\strich\emph{Alte Wiener} {[}1879{]}|pw}, ſchlechtes Stück von Anzengruber\pwindex{Anzengruber, Ludwig 29.11.1839 – 10.12.1889@\textsc{Anzengruber, Ludwig} (29.11.1839 – 10.12.1889), \emph{Schriftsteller}|pw}. Böſe Zungen\pwindex{Laube, Heinrich 1806-09-18 – 1884-08-01@\textsc{Laube, Heinrich} (1806-09-18 – 1884-08-01), \emph{Schriftsteller, Theaterleiter}!Boese Zungen1868@\strich\emph{Böse Zungen} {[}1868{]}|pw},
               lächerliches Stück von \textsc{Laube}\pwindex{Laube, Heinrich 1806-09-18 – 1884-08-01@\textsc{Laube, Heinrich} (1806-09-18 – 1884-08-01), \emph{Schriftsteller, Theaterleiter}|pw}. –\pend
           \pstart
           Die Eltern\pwindex{Hofmannsthal, Hugo August von 21.12.1841 – 08.12.1915@\textsc{Hofmannsthal, Hugo August von} (21.12.1841 – 08.12.1915), \emph{Bankdirektor}|pwv}\pwindex{Hofmannsthal, Anna von 27.01.1849 – 22.03.1904@\textsc{Hofmannsthal, Anna von} (27.01.1849 – 22.03.1904)|pwv}{ }\textsc{Hugo}\pwindex{Hofmannsthal, Hugo von 1874-02-01 – 1929-07-15@\textsc{Hofmannsthal, Hugo von} (1874-02-01 – 1929-07-15), \emph{Schriftsteller}|pw}s \label{K_L00483_2v}\edtext{neulich im Kaffeehaus}{\lemma{\textnormal{\emph{neulich im Kaffeehaus}}}\Cendnote{\textnormal{am 12. 9. 1895}}}\label{K_L00483_2h}. \textsc{Hugo}\pwindex{Hofmannsthal, Hugo von 1874-02-01 – 1929-07-15@\textsc{Hofmannsthal, Hugo von} (1874-02-01 – 1929-07-15), \emph{Schriftsteller}|pw} ritt durch Wien\oindex{Wien@\textbf{Wien}|pw}; ſie ſtanden beim Tegethoffmonument\oindex{Tegetthoff-Denkmal@\textbf{Tegetthoff-Denkmal}|pw} und ſchauten zu. Er war in Göding\oindex{Hodonín@\textbf{Hodonín}|pw}{ }ſehr unglücklich; die Manöver ſollen {\pb}ihm enorm gefallen haben. Jetzt iſt er in Bruck\oindex{Bruck an der Mur@\textbf{Bruck an der Mur}|pw}. –\hspace*{1.5em}Geſprochen:
                  \textsc{Salten}\pwindex{Salten, Felix 06.09.1869 – 08.10.1945@\textsc{Salten, Felix} (06.09.1869 – 08.10.1945), \emph{Schriftsteller, Journalist}|pw} oft, \textsc{Schwarzkopf}\pwindex{Schwarzkopf, Gustav 07.11.1853 – 13.11.1939@\textsc{Schwarzkopf, Gustav} (07.11.1853 – 13.11.1939), \emph{Schriftsteller}|pw} einige Mal, \textsc{Gold}\pwindex{Gold, Alfred 28.06.1874 – 24.10.1958@\textsc{Gold, Alfred} (28.06.1874 – 24.10.1958), \emph{Schriftsteller, Journalist, Kunsthändler}|pw}{ }ſelten, \textsc{Bahr}\pwindex{Bahr, Hermann 19.07.1863 – 15.01.1934@\textsc{Bahr, Hermann} (19.07.1863 – 15.01.1934), \emph{Schriftsteller, Kritiker}|pw} (Guten Tag, wie gehts dir denn?) Seine Frau\pwindex{Bahr, Rosa 26.10.1871 – 17.02.1940@\textsc{Bahr, Rosa} (26.10.1871 – 17.02.1940), \emph{Schauspielerin}|pwv} heute ein Stück begleitet, mich dringlich zum Beſuche
               aufgefordert. Auch \uline{er} fährt ſchon \textsc{bicycle}. –\pend
           \pstart
           – Gearbeitet noch gar nichts – ſchämen Sie ſich, daſs ich mich nicht vor Ihnen zu
               ſchämen brauche.\pend
           \pstart
           Die Brion\pwindex{Brion, Lou 17.12.1864 – 16.05.1942@\textsc{Brion, Lou} (17.12.1864 – 16.05.1942), \emph{Schauspielerin}|pw}{ }ſoll über uns geäußert haben: Setzen ſich in die
               Proſceniumsloge – und {\pb}man kriegt kein \textsc{Bracelet}, nicht einmal eine Einladung zum \textsc{Souper}! – Quelle unlauter, nemlich Paul Horn\pwindex{Horn, Paul 13.02.1867 – 18.01.1936@\textsc{Horn, Paul} (13.02.1867 – 18.01.1936), \emph{Fabrikant}|pw}. Dieſer tadelt an der kleinen Komödie\pwindex{Schnitzler, Arthur 15.05.1862 – 21.10.1931@\textsc{Schnitzler, Arthur} (15.05.1862 – 21.10.1931), \emph{Schriftsteller, Mediziner}!Liebelei. Schauspiel in drei Akten1895-10-09@\strich\emph{Liebelei. Schauspiel in drei Akten} {[}1895-10-09{]}|pwv} die Unmöglichkeit, daſs ſich ein Menſch
               wirklich von den Seidenſtrümpfen und den \textsc{grande marque}
               Cocotten zu einem lieben Vorſtadtmädel hingezogen fühlen ſollte. –\pend
           \pstart
           Hier regnet es i{\geminationm}er – und Sie? – Alles erkundigt ſich
               nach Ihnen; ſind Sie ſtolz? Leben Sie wohl, laſſen Sie ſchnell {\pb}wieder was von ſich hören, bringen Sie den fertigen
                  Götterliebling\pwindex{Beer-Hofmann, Richard 1866-07-11 – 1945-09-26@\textsc{Beer-Hofmann, Richard} (1866-07-11 – 1945-09-26), \emph{Schriftsteller}!Tod Georgs1900@\strich\emph{Der Tod Georgs} {[}1900{]}|pwv} und viel Luſt
               zu neuen Werken mit. Sagen Sie, wie hat denn die Lou\pwindex{Brion, Lou 17.12.1864 – 16.05.1942@\textsc{Brion, Lou} (17.12.1864 – 16.05.1942), \emph{Schauspielerin}|pw} das Alleinfahrenmüſſen aufgeno{\geminationm}en? Hier ist
               es »bekannt geworden« daſs wir miteinander nicht über Literatur reden; man findet das
               höchſt anmaßend – »ſo groß ſind ſie nicht, daß ſie nicht mehr über Literatur reden
               müßten.« – Laßt uns lächeln.\pend
           \pstart Ihr \spacefill\mbox{Arthur Sch} mit vielen herzlichen Grüßen.\pend{}
         
         \endnumbering\mylabel{h}\end{ledgroupsized}  \newcommand{\dateiname}{L00483}\newcommand{\titel}{Arthur Schnitzler an Richard Beer-Hofmann, 15. 9. 1895}\newcommand{\editorInnen}{ Martin Anton Müller und Gerd-Hermann Susen}%% latex-leseansicht-abspann.tex
%% Abspann für die Leseansicht.
%% Der Schalter \ifkorrekturansicht ist bereits durch den Vorspann gesetzt.

%% latex-abspann.tex
%% Gemeinsamer Abspann für Korrekturansicht und Leseansicht.
%% Setzt den Schalter \ifkorrekturansicht voraus (gesetzt in den
%% einbindenden Dateien latex-korrekturansicht-abspann.tex bzw.
%% latex-leseansicht-abspann.tex).
%% ---------------------------------------------------------------

\normalsize

% Das esempio-Environment wird nur in der Leseansicht benötigt
\ifkorrekturansicht\else
\newenvironment{esempio}[3]%
{
    \vspace{1.5ex}
    \rlap{\underline{#1}}
    \par
    \setlength{\parindent}{0cm}
    \nopagebreak
    \leftskip=#2cm
    \rightskip=#3cm
}
{
    \par
}
\fi

\doendnotes{C}
\bigskip
\vfill

\clearpage

\footnotesize

\ifkorrekturansicht
  \lohead{\textsc{register}}
\fi

% theindex-Environment neu definieren ohne reledmac
\makeatletter
\renewenvironment{theindex}{%
  \ifkorrekturansicht
    \section*{\indexname}%
  \else
    \subsubsection*{Index der erwähnten Entitäten}%
  \fi
  \setlength{\parindent}{0pt}%
  \setlength{\parskip}{0pt plus 0.3pt}%
  \let\item\@idxitem
}{%
  \ifkorrekturansicht\clearpage\fi
}
\makeatother

\IfFileExists{\jobname-pw.ind}{\input{\jobname-pw.ind}}{}

% Quellenangabe nur in der Leseansicht
\ifkorrekturansicht\else
% Fallback-Definitionen, falls die .tex-Datei \titel etc. nicht gesetzt hat
\providecommand{\titel}{}
\providecommand{\editorInnen}{}
\providecommand{\dateiname}{\jobname}

\vspace{3cm}

\vfill

\footnotesize
\textsc{Quelle}: \titel. Herausgegeben von {\editorInnen}. In: \emph{Arthur Schnitzler: Briefwechsel mit Autorinnen und Autoren}.
 Digitale Edition, https://schnitzler-briefe.acdh.oeaw.ac.at/{\dateiname}.html (Stand \today)
\fi

\end{document}


      