%% latex-leseansicht-vorspann.tex
%% Vorspann für die Leseansicht.
%% Lädt die gemeinsame Datei latex-vorspann.tex mit nicht gesetztem Schalter.

\newif\ifkorrekturansicht
\korrekturansichtfalse

\input{../tex-inputs/latex-vorspann}


\section[Arthur Schnitzler an Richard Beer-Hofmann, 15. 9. 1895]{L00483 Arthur Schnitzler an Richard Beer-Hofmann, 15. 9. 1895}
\nopagebreak\mylabel{L00483v}
\rehead{ }\normalsize\beginnumbering\briefempfaengerindex{Beer-Hofmann, Richard@\textsc{Beer-Hofmann, Richard}!zzzSchnitzler, Arthur@\emph{von Arthur Schnitzler}!1895-09-151@{15. 9. 1895}|(be}
\toendnotes[C]{\smallbreak\pagebreak[2]}
\correspDesc{Versand  durch Arthur Schnitzler am 15. 9. 1895 in Wien
\newline{}Erhalt  durch Richard Beer-Hofmann am 16. 9. 95 in Schönberg im Stubaital}\toendnotes[C]{\smallbreak}
\Standort{YCGL, MSS 31.}
\physDesc{Brief, 2 Blätter, 7 Seiten, Kuvert, 3228 Zeichen
\newline{}Handschrift: 1) Bleistift, deutsche Kurrent\hspace{1em}2) schwarze Tinte, deutsche Kurrent (\noindent{}Adresse)\hspace{1em}
\newline{}Versand: 1) Stempel: »\nobreak{}\oindex{IX., Alsergrund@\textbf{IX., Alsergrund}, \emph{Verwaltungsgebiet}|pwk}Wien 9/3, 16. 9. 95, 6–7 V\nobreak{}«.   2) Stempel: »\nobreak{}\oindex{Schönberg im Stubaital@\textbf{Schönberg im Stubaital}, \emph{Hauptstadt}|pwk}{\pb}{[}Sch{]}önb{[}e{]}rg\nobreak{}«. }
\buchAbdrucke{\weitereDrucke{1) Arthur Schnitzler: \emph{Briefe 1875–1912}. Herausgegeben von Therese Nickl und Heinrich Schnitzler. Frankfurt am Main: \emph{S. Fischer} 1981, S. 277–278.} \weitereDrucke{2) Arthur Schnitzler, Richard Beer-Hofmann: \emph{Briefwechsel 1891–1931}. Herausgegeben von Konstanze Fliedl. Wien, Zürich: \emph{Europaverlag} 1992, S. 80–81.} \weitereDrucke{3) Hermann Bahr, Arthur Schnitzler: \emph{Briefwechsel, Aufzeichnungen, Dokumente (1891–1931)}. Herausgegeben von Kurt Ifkovits und Martin Anton Müller. Göttingen: \emph{Wallstein} 2018.} }\toendnotes[C]{\smallbreak}\pstart{}{\pb}Herrn Dr. \textsc{Richard
                     Beer-Hofmann}\pend{}\pstart{}\textsc{Schönberg im Stubaithal\oindex{Schönberg im Stubaital@\textbf{Schönberg im Stubaital}, \emph{Hauptstadt}|pw}}\pend{}\pstart{}\textsc{Tirol\oindex{Tirol@\textbf{Tirol}, \emph{Land}|pw}}\pend{}{\bigskip}\vspace{1em}
\pstart
           \raggedleft{}{\pb}So{\geminationn}tg 15. 9. 95.\pend
           \vspace{0.5em}
\pstart
           Lieber Richard. Ich freue mich, daſs Sie in guter Sti{\geminationm}ung{ }ſind. Wahrscheinlich werden Sie bald südlicher
               gehn; kennen Sie \textsc{Riva\oindex{Riva del Garda@\textbf{Riva del Garda}, \emph{Hauptstadt}|pw}}? Es iſt{ }ſchön, war \introOben{}mir\introOben{} aber nicht{ }ſympathiſch. Ich bin
               von dort nach Venedig\oindex{Venedig@\textbf{Venedig}|pw} gegangen; es iſt so nah.
               Sie haben \uline{mich} falſch verſtanden; ich wußte, dſs Sie
               Ende Sept. in Wien\oindex{Wien@\textbf{Wien}, \emph{Verwaltungsgebiet}|pw}{ }ſein wollten. An dieſes Wien\oindex{Wien@\textbf{Wien}, \emph{Verwaltungsgebiet}|pw} hab ich mich noch nicht ganz gewöhnt; empfinde gleich
               wieder, jetzt wo die alten Verhältniſſe sich aufdrängen, das vielfach unzulängliche,
               unter dem man zu leiden hat. Dünne Fäden, mit denen {\pb}man an mancherlei gebunden iſt – dünn, aber doch Fäden. Denken Sie,{ }ſeit ich hier
               bin, bin ich bereits 2mal in der früh \introOben{}(um 6 oder ½ 7)\introOben{}
                geweckt worden – von Patienten, nicht vom Burgtheater\oindex{Wien@\textbf{Wien}!I., Innere Stadt@\textbf{I., Innere Stadt}!Burgtheater@\textbf{Burgtheater}, \emph{Theater}|pw}. – Am Mittwoch 18.{ }ſoll Leſeprobe\eventindex{Burgtheater@\textbf{Burgtheater}!Leseprobe von Liebelei, 18.9.1895@Leseprobe von Liebelei, 18.9.1895|pw}{ }ſein; wenigſtens ist sie angesetzt.\pend
           
\pstart
           – Die S.\pwindex{Sandrock, Adele 19.\,8.\,1863 Rotterdam – 30.\,8.\,1937 Berlin@\textsc{Sandrock, Adele} (19.\,8.\,1863 Rotterdam – 30.\,8.\,1937 Berlin), \emph{Schauspielerin}|pw} verhält{ }ſich{ }ſtille; ihre
               Feindſeligkeit hat{ }ſie vorläufig nur dadurch ausgedrückt, daſs{ }ſie ihrer ruſſiſchen\oindex{Russland@\textbf{Russland}|pw}{ }Freundin\pwindex{Golovin, Olga von @\textsc{Golovin, Olga von}|pwv} einen Brief{ }ſchrieb,{ }ſie dürfe \uline{mich} nicht mehr als Arzt nehmen,
               wenn{ }ſie mit ihr verkehren wolle. Die ruſſiſche\oindex{Russland@\textbf{Russland}|pw}{ }Freundin\pwindex{Golovin, Olga von @\textsc{Golovin, Olga von}|pwv} kümmert{ }ſich nicht
               drum {\pb}und läßt{ }ſich mit Begeiſterung von mir
               behandeln. – \textsc{Bckhrd}\pwindex{Burckhard, Max Eugen 14.\,7.\,1854 Korneuburg – 16.\,3.\,1912 Wien@\textsc{Burckhard, Max Eugen} (14.\,7.\,1854 Korneuburg – 16.\,3.\,1912 Wien), \emph{Schriftsteller, Rechtswissenschaftler, Theaterleiter}|pw}{ }ſprach neulich das erſte Mal von der Sache: »Ich
               hab ja nur zufällig durch den Bahr\pwindex{Bahr, Hermann 19.\,7.\,1863 Linz – 15.\,1.\,1934 München@\textsc{Bahr, Hermann} (19.\,7.\,1863 Linz – 15.\,1.\,1934 München), \emph{Schriftsteller, Kritiker}|pw} von der
               Sache erfahren {\dotstwo} aber ich werd ihr{ }ſchon begreiflich
               machen, daſs das beim Burgtheater\oindex{Wien@\textbf{Wien}!I., Innere Stadt@\textbf{I., Innere Stadt}!Burgtheater@\textbf{Burgtheater}, \emph{Theater}|pw} nicht geht –
               beſonders \uline{ſie}{\dots} Freilich mit Ketten kann ich{ }ſie nicht auf die Bühne
               zerren.« – Man war bei \textsc{Besezny}\pwindex{Bezecný, Josef von 5.\,2.\,1829 Tábor – 17.\,6.\,1904 Wien@\textsc{Bezecný, Josef von} (5.\,2.\,1829 Tábor – 17.\,6.\,1904 Wien), \emph{Pianist, Theaterintendant, Beamter}|pw}, ihm erzählen, wie du{\geminationm} und ordinär mein Stück\pwindex{Schnitzler, Arthur 15.\,5.\,1862 Wien – 21.\,10.\,1931 ebd.@\textsc{Schnitzler, Arthur} (15.\,5.\,1862 Wien – 21.\,10.\,1931 ebd.), \emph{Schriftsteller, Mediziner}!Liebelei. Schauspiel in drei Akten@\strich\emph{Liebelei. Schauspiel in drei Akten}|pwv}{ }ſei. – Unser Freund J. J. David\pwindex{David, Jakob Julius 6.\,2.\,1859 Hranice – 20.\,11.\,1906 Wien@\textsc{David, Jakob Julius} (6.\,2.\,1859 Hranice – 20.\,11.\,1906 Wien), \emph{Schriftsteller, Journalist}|pw}: Ich werde \label{K_L00483-1v}\edtext{vielleicht durch{\pb}fallen\pwindex{David, Jakob Julius 6.\,2.\,1859 Hranice – 20.\,11.\,1906 Wien@\textsc{David, Jakob Julius} (6.\,2.\,1859 Hranice – 20.\,11.\,1906 Wien), \emph{Schriftsteller, Journalist}!Regentag. Charakterbild@\strich\emph{Ein Regentag. Charakterbild}|pwv}}{\lemma{\textnormal{\emph{vielleicht durchfallen}}}\Cendnote{\textnormal{\emph{Ein Regentag}\pwindex{David, Jakob Julius 6.\,2.\,1859 Hranice – 20.\,11.\,1906 Wien@\textsc{David, Jakob Julius} (6.\,2.\,1859 Hranice – 20.\,11.\,1906 Wien), \emph{Schriftsteller, Journalist}!Regentag. Charakterbild@\strich\emph{Ein Regentag. Charakterbild}|pwk}; Uraufführung\eventindex{Volkstheater@\textbf{Volkstheater}!Uraufführung von Ein Regentag, 12.10.1895@Uraufführung von Ein Regentag, 12.10.1895|pwkv} im Deutschen Volkstheater\oindex{Wien@\textbf{Wien}!VII., Neubau@\textbf{VII., Neubau}!Volkstheater@\textbf{Volkstheater}, \emph{Theater}|pwk} am
                     12. 10. 1895}}}\label{K_L00483-1}, der \textsc{Schnitzler} aber doch ganz gewiſs. –\pend
           
\pstart
           – \textsc{Speidel}\pwindex{Speidel, Ludwig 11.\,4.\,1830 Ulm – 3.\,2.\,1906 Wien@\textsc{Speidel, Ludwig} (11.\,4.\,1830 Ulm – 3.\,2.\,1906 Wien), \emph{Journalist, Kritiker}|pw} zu \textsc{Eberma{\geminationn}}\pwindex{Ebermann, Leo 16.\,7.\,1863 Draganovka – 9.\,10.\,1914 Wien@\textsc{Ebermann, Leo} (16.\,7.\,1863 Draganovka – 9.\,10.\,1914 Wien), \emph{Schriftsteller, Journalist, Rechtswissenschaftler}|pw} über die Liebelei\pwindex{Schnitzler, Arthur 15.\,5.\,1862 Wien – 21.\,10.\,1931 ebd.@\textsc{Schnitzler, Arthur} (15.\,5.\,1862 Wien – 21.\,10.\,1931 ebd.), \emph{Schriftsteller, Mediziner}!Liebelei. Schauspiel in drei Akten@\strich\emph{Liebelei. Schauspiel in drei Akten}|pw} – »Da werden die Wiener\oindex{Wien@\textbf{Wien}, \emph{Verwaltungsgebiet}|pw}{ }ſchaun!« – Iſt vom Anatol\pwindex{Schnitzler, Arthur 15.\,5.\,1862 Wien – 21.\,10.\,1931 ebd.@\textsc{Schnitzler, Arthur} (15.\,5.\,1862 Wien – 21.\,10.\,1931 ebd.), \emph{Schriftsteller, Mediziner}!Anatol@\strich\emph{Anatol}|pw} äußerst – (ich genire mich »entzückt« zu{ }ſchreiben.) – Theater: Alte Wiener\pwindex{Anzengruber, Ludwig 29.\,11.\,1839 Wien – 10.\,12.\,1889 ebd.@\textsc{Anzengruber, Ludwig} (29.\,11.\,1839 Wien – 10.\,12.\,1889 ebd.), \emph{Schriftsteller}!Alte Wiener@\strich\emph{Alte Wiener}|pw},{ }ſchlechtes Stück von Anzengruber\pwindex{Anzengruber, Ludwig 29.\,11.\,1839 Wien – 10.\,12.\,1889 ebd.@\textsc{Anzengruber, Ludwig} (29.\,11.\,1839 Wien – 10.\,12.\,1889 ebd.), \emph{Schriftsteller}|pw}. Böſe Zungen\pwindex{Laube, Heinrich 18.\,9.\,1806 Sprottau – 1.\,8.\,1884 Wien@\textsc{Laube, Heinrich} (18.\,9.\,1806 Sprottau – 1.\,8.\,1884 Wien), \emph{Schriftsteller, Theaterleiter}!Böse Zungen@\strich\emph{Böse Zungen}|pw}, lächerliches Stück von \textsc{Laube}\pwindex{Laube, Heinrich 18.\,9.\,1806 Sprottau – 1.\,8.\,1884 Wien@\textsc{Laube, Heinrich} (18.\,9.\,1806 Sprottau – 1.\,8.\,1884 Wien), \emph{Schriftsteller, Theaterleiter}|pw}. –\pend
           
\pstart
           Die Eltern\pwindex{Hofmannsthal, Hugo August von 21.\,12.\,1841 Wien – 8.\,12.\,1915 ebd.@\textsc{Hofmannsthal, Hugo August von} (21.\,12.\,1841 Wien – 8.\,12.\,1915 ebd.), \emph{Bankdirektor}|pwv}\pwindex{Hofmannsthal, Anna von 27.\,1.\,1849 Wien – 22.\,3.\,1904 Sanatorium Fürth@\textsc{Hofmannsthal, Anna von} (27.\,1.\,1849 Wien – 22.\,3.\,1904 Sanatorium Fürth)|pwv}{ }\textsc{Hugos}\pwindex{Hofmannsthal, Hugo von 1.\,2.\,1874 Wien – 15.\,7.\,1929 Rodaun@\textsc{Hofmannsthal, Hugo von} (1.\,2.\,1874 Wien – 15.\,7.\,1929 Rodaun), \emph{Schriftsteller}|pw}{ }\label{K_L00483-2v}\edtext{neulich im Kaffeehaus}{\lemma{\textnormal{\emph{neulich im Kaffeehaus}}}\Cendnote{\textnormal{am 12. 9. 1895}}}\label{K_L00483-2}. \textsc{Hugo}\pwindex{Hofmannsthal, Hugo von 1.\,2.\,1874 Wien – 15.\,7.\,1929 Rodaun@\textsc{Hofmannsthal, Hugo von} (1.\,2.\,1874 Wien – 15.\,7.\,1929 Rodaun), \emph{Schriftsteller}|pw} ritt durch Wien\oindex{Wien@\textbf{Wien}, \emph{Verwaltungsgebiet}|pw};{ }ſie{ }ſtanden beim Tegethoffmonument\oindex{Wien@\textbf{Wien}!II., Leopoldstadt@\textbf{II., Leopoldstadt}!Tegetthoff-Denkmal@\textbf{Tegetthoff-Denkmal}, \emph{Monument}|pw} und{ }ſchauten zu. Er war in Göding\oindex{Hodonín@\textbf{Hodonín}|pw}{ }ſehr unglücklich; die Manöver{ }ſollen {\pb}ihm enorm gefallen haben. Jetzt iſt er in Bruck\oindex{Bruck an der Mur@\textbf{Bruck an der Mur}, \emph{Hauptstadt}|pw}. –\hspace*{1.5em}Geſprochen: \textsc{Salten}\pwindex{Salten, Felix 6.\,9.\,1869 Budapest – 8.\,10.\,1945 Zürich@\textsc{Salten, Felix} (6.\,9.\,1869 Budapest – 8.\,10.\,1945 Zürich), \emph{Schriftsteller, Journalist, Chefredakteur}|pw} oft, \textsc{Schwarzkopf}\pwindex{Schwarzkopf, Gustav 7.\,11.\,1853 Wien – 13.\,11.\,1939 ebd.@\textsc{Schwarzkopf, Gustav} (7.\,11.\,1853 Wien – 13.\,11.\,1939 ebd.), \emph{Schriftsteller}|pw} einige Mal, \textsc{Gold}\pwindex{Gold, Alfred 28.\,6.\,1874 Wien – 24.\,10.\,1958 New York City@\textsc{Gold, Alfred} (28.\,6.\,1874 Wien – 24.\,10.\,1958 New York City), \emph{Schriftsteller, Journalist, Kunsthändler}|pw}{ }ſelten, \textsc{Bahr}\pwindex{Bahr, Hermann 19.\,7.\,1863 Linz – 15.\,1.\,1934 München@\textsc{Bahr, Hermann} (19.\,7.\,1863 Linz – 15.\,1.\,1934 München), \emph{Schriftsteller, Kritiker}|pw} (Guten Tag, wie gehts dir denn?) Seine Frau\pwindex{Bahr, Rosa 26.\,10.\,1871 Prag – 17.\,2.\,1940 Berlin@\textsc{Bahr, Rosa} (26.\,10.\,1871 Prag – 17.\,2.\,1940 Berlin), \emph{Schauspielerin}|pwv} heute ein Stück begleitet, mich dringlich zum Beſuche
               aufgefordert. Auch \uline{er} fährt{ }ſchon \textsc{bicycle}. –\pend
           
\pstart
           – Gearbeitet noch gar nichts –{ }ſchämen Sie{ }ſich, daſs ich mich nicht vor Ihnen zu{ }ſchämen brauche.\pend
           
\pstart
           Die Brion\pwindex{Brion, Lou 17.\,12.\,1864 Besançon – 16.\,5.\,1942 Wien@\textsc{Brion, Lou} (17.\,12.\,1864 Besançon – 16.\,5.\,1942 Wien), \emph{Schauspielerin}|pw}{ }ſoll über uns geäußert haben: Setzen{ }ſich in die
               Proſceniumsloge – und {\pb}man kriegt kein \textsc{Bracelet}, nicht einmal eine Einladung zum \textsc{Souper}! – Quelle unlauter, nemlich Paul Horn\pwindex{Horn, Paul 13.\,2.\,1867 Wien – 18.\,1.\,1936 Menton@\textsc{Horn, Paul} (13.\,2.\,1867 Wien – 18.\,1.\,1936 Menton), \emph{Fabrikant}|pw}. Dieſer tadelt an der kleinen Komödie\pwindex{Schnitzler, Arthur 15.\,5.\,1862 Wien – 21.\,10.\,1931 ebd.@\textsc{Schnitzler, Arthur} (15.\,5.\,1862 Wien – 21.\,10.\,1931 ebd.), \emph{Schriftsteller, Mediziner}!Liebelei. Schauspiel in drei Akten@\strich\emph{Liebelei. Schauspiel in drei Akten}|pwv} die Unmöglichkeit, daſs{ }ſich ein Menſch
               wirklich von den Seidenſtrümpfen und den \textsc{grande marque}
               Cocotten zu einem lieben Vorſtadtmädel hingezogen fühlen{ }ſollte. –\pend
           
\pstart
           Hier regnet es i{\geminationm}er – und Sie? – Alles erkundigt{ }ſich
               nach Ihnen;{ }ſind Sie{ }ſtolz? Leben Sie wohl, laſſen Sie{ }ſchnell {\pb}wieder was von{ }ſich hören, bringen Sie den fertigen
                  Götterliebling\pwindex{Beer-Hofmann, Richard 11.\,7.\,1866 Wien – 26.\,9.\,1945 New York City@\textsc{Beer-Hofmann, Richard} (11.\,7.\,1866 Wien – 26.\,9.\,1945 New York City), \emph{Schriftsteller}!Tod Georgs@\strich\emph{Der Tod Georgs}|pwv} und viel
               Luſt zu neuen Werken mit. Sagen Sie, wie hat denn die Lou\pwindex{Brion, Lou 17.\,12.\,1864 Besançon – 16.\,5.\,1942 Wien@\textsc{Brion, Lou} (17.\,12.\,1864 Besançon – 16.\,5.\,1942 Wien), \emph{Schauspielerin}|pw} das Alleinfahrenmüſſen aufgeno{\geminationm}en? Hier ist es »bekannt geworden« daſs wir miteinander nicht über Literatur reden;
               man findet das höchſt anmaßend – »ſo groß{ }ſind{ }ſie nicht, daß{ }ſie nicht mehr über
               Literatur reden müßten.« – Laßt uns lächeln.\pend
           \pstart Ihr \spacefill\mbox{Arthur Sch} mit vielen herzlichen Grüßen.\pend{}\selectlanguage{ngerman}\endnumbering\briefempfaengerindex{Beer-Hofmann, Richard@\textsc{Beer-Hofmann, Richard}!zzzSchnitzler, Arthur@\emph{von Arthur Schnitzler}!1895-09-151@{15. 9. 1895}|)be}\mylabel{L00483h}  \newcommand{\dateiname}{L00483}\newcommand{\titel}{Arthur Schnitzler an Richard Beer-Hofmann, 15. 9. 1895}\newcommand{\editorInnen}{Herausgegeben von Martin Anton Müller}%% latex-leseansicht-abspann.tex
%% Abspann für die Leseansicht.
%% Der Schalter \ifkorrekturansicht ist bereits durch den Vorspann gesetzt.

%% latex-abspann.tex
%% Gemeinsamer Abspann für Korrekturansicht und Leseansicht.
%% Setzt den Schalter \ifkorrekturansicht voraus (gesetzt in den
%% einbindenden Dateien latex-korrekturansicht-abspann.tex bzw.
%% latex-leseansicht-abspann.tex).
%% ---------------------------------------------------------------

\normalsize

% Das esempio-Environment wird nur in der Leseansicht benötigt
\ifkorrekturansicht\else
\newenvironment{esempio}[3]%
{
    \vspace{1.5ex}
    \rlap{\underline{#1}}
    \par
    \setlength{\parindent}{0cm}
    \nopagebreak
    \leftskip=#2cm
    \rightskip=#3cm
}
{
    \par
}
\fi

\doendnotes{C}
\bigskip
\vfill

\clearpage

\footnotesize

\ifkorrekturansicht
  \lohead{\textsc{register}}
\fi

% theindex-Environment neu definieren ohne reledmac
\makeatletter
\renewenvironment{theindex}{%
  \ifkorrekturansicht
    \section*{\indexname}%
  \else
    \subsubsection*{Index der erwähnten Entitäten}%
  \fi
  \setlength{\parindent}{0pt}%
  \setlength{\parskip}{0pt plus 0.3pt}%
  \let\item\@idxitem
}{%
  \ifkorrekturansicht\clearpage\fi
}
\makeatother

\IfFileExists{\jobname-pw.ind}{\input{\jobname-pw.ind}}{}

% Quellenangabe nur in der Leseansicht
\ifkorrekturansicht\else
% Fallback-Definitionen, falls die .tex-Datei \titel etc. nicht gesetzt hat
\providecommand{\titel}{}
\providecommand{\editorInnen}{}
\providecommand{\dateiname}{\jobname}

\vspace{3cm}

\vfill

\footnotesize
\textsc{Quelle}: \titel. Herausgegeben von {\editorInnen}. In: \emph{Arthur Schnitzler: Briefwechsel mit Autorinnen und Autoren}.
 Digitale Edition, https://schnitzler-briefe.acdh.oeaw.ac.at/{\dateiname}.html (Stand \today)
\fi

\end{document}


