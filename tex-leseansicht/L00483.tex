%% latex-korrekturansicht-vorspann.tex
%% Vorspann für die Korrekturansicht.
%% Lädt die gemeinsame Datei latex-vorspann.tex mit gesetztem Schalter.

\newif\ifkorrekturansicht
\korrekturansichttrue

\input{../tex-inputs/latex-vorspann}


\section[Arthur Schnitzler an Richard Beer-Hofmann, 15. 9. 1895]{L00483 Arthur Schnitzler an Richard Beer-Hofmann, 15. 9. 1895}
\nopagebreak\mylabel{L00483v}
\rehead{ }\normalsize\beginnumbering\briefempfaengerindex{Beer-Hofmann, Richard@\textsc{Beer-Hofmann, Richard}!zzzSchnitzler, Arthur@\emph{von Arthur Schnitzler}!1895-09-151@{15. 9. 1895}|(be}
\toendnotes[C]{\smallbreak\pagebreak[2]}\Standort{YCGL, MSS 31.}
\physDesc{Brief, 2 Blätter, 7 Seiten, Umschlag, 3228 Zeichen
\newline{}Handschrift: 1) Bleistift, deutsche Kurrent\hspace{1em}2) schwarze Tinte, deutsche Kurrent (\noindent{}Adressierung)\hspace{1em}
\newline{}Versand: 1) Stempel: »\nobreak{}\oindex{IX., Alsergrund@\textbf{IX., Alsergrund}, \emph{A.ADM3}|pwk}Wien 9/3, 16. 9. 95, 6–7 V\nobreak{}«.   2) Stempel: »\nobreak{}\oindex{Schoenberg im Stubaital@\textbf{Schönberg im Stubaital}, \emph{P.PPLA3}|pwk}{\pb}{[}Sch{]}önb{[}e{]}rg\nobreak{}«. }
\buchAbdrucke{\weitereDrucke{1) Arthur Schnitzler: \emph{Briefe 1875–1912}. Frankfurt am Main: \emph{S. Fischer} 1981, S. 277–278.} \weitereDrucke{2) Arthur Schnitzler, Richard Beer-Hofmann: \emph{Briefwechsel 1891–1931}. Wien, Zürich: \emph{Europaverlag} 1992, S. 80–81.} \weitereDrucke{3) Hermann Bahr, Arthur Schnitzler: \emph{Briefwechsel, Aufzeichnungen, Dokumente (1891–1931)}. Göttingen: \emph{Wallstein} 2018.} }\toendnotes[C]{\smallbreak}\pstart{}{\pb}Herrn Dr. \textsc{Richard
                     Beer-Hofmann}\pend{}\pstart{}\textsc{Schönberg im Stubaithal\oindex{Schoenberg im Stubaital@\textbf{Schönberg im Stubaital}, \emph{P.PPLA3}|pw}}\pend{}\pstart{}\textsc{Tirol\oindex{Tirol@\textbf{Tirol}, \emph{A.ADM1}|pw}}\pend{}{\bigskip}\vspace{1em}
\pstart
           \raggedleft{}{\pb}So{\geminationn}tg 15. 9. 95.\pend
           \vspace{0.5em}
\pstart
           Lieber Richard. Ich freue mich, daſs Sie in guter Sti{\geminationm}ung ſind. Wahrscheinlich werden Sie bald südlicher
               gehn; kennen Sie \textsc{Riva\oindex{Riva del Garda@\textbf{Riva del Garda}, \emph{P.PPLA3}|pw}}? Es iſt ſchön, war \introOben{}mir\introOben{} aber nicht ſympathiſch. Ich bin
               von dort nach Venedig\oindex{Venedig@\textbf{Venedig}, \emph{P.PPLA}|pw} gegangen; es iſt so nah.
               Sie haben \uline{mich} falſch verſtanden; ich wußte, dſs Sie
               Ende Sept. in Wien\oindex{Wien@\textbf{Wien}, \emph{A.ADM2}|pw}{ }ſein wollten. An dieſes Wien\oindex{Wien@\textbf{Wien}, \emph{A.ADM2}|pw} hab ich mich noch nicht ganz gewöhnt; empfinde gleich
               wieder, jetzt wo die alten Verhältniſſe sich aufdrängen, das vielfach unzulängliche,
               unter dem man zu leiden hat. Dünne Fäden, mit denen {\pb}man an mancherlei gebunden iſt – dünn, aber doch Fäden. Denken Sie, ſeit ich hier
               bin, bin ich bereits 2mal in der früh \introOben{}(um 6 oder ½ 7)\introOben{}
                geweckt worden – von Patienten, nicht vom Burgtheater\oindex{Burgtheater@\textbf{Burgtheater}, \emph{S.THTR}|pw}. – Am Mittwoch 18. ſoll Leſeprobe\eventindex{Burgtheater@\textbf{Burgtheater}!Leseprobe von Liebelei, 18.9.1895@Leseprobe von Liebelei, 18.9.1895|pw}{ }ſein; wenigſtens ist sie angesetzt.\pend
           
\pstart
           – Die S.\pwindex{Sandrock, Adele 1863-08-19 – 1937-08-30@\textsc{Sandrock, Adele} (1863-08-19 – 1937-08-30), \emph{Schauspieler/Schauspielerin}|pw} verhält ſich ſtille; ihre
               Feindſeligkeit hat ſie vorläufig nur dadurch ausgedrückt, daſs ſie ihrer ruſſiſchen\oindex{Russland@\textbf{Russland}, \emph{A.PCLI}|pw}{ }Freundin\pwindex{Golovin, Olga von @\textsc{Golovin, Olga von}|pwv} einen Brief
               ſchrieb, ſie dürfe \uline{mich} nicht mehr als Arzt nehmen,
               wenn ſie mit ihr verkehren wolle. Die ruſſiſche\oindex{Russland@\textbf{Russland}, \emph{A.PCLI}|pw}{ }Freundin\pwindex{Golovin, Olga von @\textsc{Golovin, Olga von}|pwv} kümmert ſich nicht
               drum {\pb}und läßt ſich mit Begeiſterung von mir
               behandeln. – \textsc{Bckhrd}\pwindex{Burckhard, Max Eugen 14.07.1854 – 16.03.1912@\textsc{Burckhard, Max Eugen} (14.07.1854 – 16.03.1912), \emph{Schriftsteller/Schriftstellerin, Rechtswissenschaftler/Rechtswissenschaftlerin, Theaterleiter/Theaterleiterin}|pw}{ }ſprach neulich das erſte Mal von der Sache: »Ich
               hab ja nur zufällig durch den Bahr\pwindex{Bahr, Hermann 19.07.1863 – 15.01.1934@\textsc{Bahr, Hermann} (19.07.1863 – 15.01.1934), \emph{Schriftsteller/Schriftstellerin, Kritiker/Kritikerin}|pw} von der
               Sache erfahren {\dotstwo} aber ich werd ihr ſchon begreiflich
               machen, daſs das beim Burgtheater\oindex{Burgtheater@\textbf{Burgtheater}, \emph{S.THTR}|pw} nicht geht –
               beſonders \uline{ſie}{\dots} Freilich mit Ketten kann ich ſie nicht auf die Bühne
               zerren.« – Man war bei \textsc{Besezny}\pwindex{Bezecný, Josef von 05.02.1829 – 17.06.1904@\textsc{Bezecný, Josef von} (05.02.1829 – 17.06.1904), \emph{Pianist/Pianistin, Theaterintendant/Theaterintendantin, Beamter/Beamte}|pw}, ihm erzählen, wie du{\geminationm} und ordinär mein Stück\pwindex{Liebelei. Schauspiel in drei Akten@\emph{Liebelei. Schauspiel in drei Akten}|pwv}{ }ſei. – Unser Freund J. J. David\pwindex{David, Jakob Julius 1859-02-06 – 1906-11-20@\textsc{David, Jakob Julius} (1859-02-06 – 1906-11-20), \emph{Schriftsteller/Schriftstellerin, Journalist/Journalistin}|pw}: Ich werde \label{K_L00483-1v}\edtext{vielleicht durch{\pb}fallen\pwindex{Regentag. Charakterbild@\emph{Ein Regentag. Charakterbild}|pwv}}{\lemma{\textnormal{\emph{vielleicht durchfallen}}}\Cendnote{\textnormal{\emph{Ein Regentag}\pwindex{Regentag. Charakterbild@\emph{Ein Regentag. Charakterbild}|pwk}; Uraufführung\eventindex{Volkstheater@\textbf{Volkstheater}!Urauffuehrung von Ein Regentag, 12.10.1895@Uraufführung von Ein Regentag, 12.10.1895|pwkv} im Deutschen Volkstheater\oindex{Volkstheater@\textbf{Volkstheater}, \emph{Theater (K.THE)}|pwk} am
                     12. 10. 1895}}}\label{K_L00483-1}, der \textsc{Schnitzler} aber doch ganz gewiſs. –\pend
           
\pstart
           – \textsc{Speidel}\pwindex{Speidel, Ludwig 1830-04-11 – 1906-02-03@\textsc{Speidel, Ludwig} (1830-04-11 – 1906-02-03), \emph{Journalist/Journalistin, Kritiker/Kritikerin}|pw} zu \textsc{Eberma{\geminationn}}\pwindex{Ebermann, Leo 16.07.1863 – 09.10.1914@\textsc{Ebermann, Leo} (16.07.1863 – 09.10.1914), \emph{Schriftsteller/Schriftstellerin, Journalist/Journalistin, Rechtswissenschaftler/Rechtswissenschaftlerin}|pw} über die Liebelei\pwindex{Liebelei. Schauspiel in drei Akten@\emph{Liebelei. Schauspiel in drei Akten}|pw} – »Da werden die Wiener\oindex{Wien@\textbf{Wien}, \emph{A.ADM2}|pw}{ }ſchaun!« – Iſt vom Anatol\pwindex{Anatol@\emph{Anatol}|pw} äußerst – (ich genire mich »entzückt« zu ſchreiben.) – Theater: Alte Wiener\pwindex{Alte Wiener@\emph{Alte Wiener}|pw}, ſchlechtes Stück von Anzengruber\pwindex{Anzengruber, Ludwig 29.11.1839 – 10.12.1889@\textsc{Anzengruber, Ludwig} (29.11.1839 – 10.12.1889), \emph{Schriftsteller/Schriftstellerin}|pw}. Böſe Zungen\pwindex{Boese Zungen@\emph{Böse Zungen}|pw}, lächerliches Stück von \textsc{Laube}\pwindex{Laube, Heinrich 1806-09-18 – 1884-08-01@\textsc{Laube, Heinrich} (1806-09-18 – 1884-08-01), \emph{Schriftsteller/Schriftstellerin, Theaterleiter/Theaterleiterin}|pw}. –\pend
           
\pstart
           Die Eltern\pwindex{Hofmannsthal, Hugo August von 21.12.1841 – 08.12.1915@\textsc{Hofmannsthal, Hugo August von} (21.12.1841 – 08.12.1915), \emph{Bankdirektor/Bankdirektorin}|pwv}\pwindex{Hofmannsthal, Anna von 27.01.1849 – 22.03.1904@\textsc{Hofmannsthal, Anna von} (27.01.1849 – 22.03.1904)|pwv}{ }\textsc{Hugos}\pwindex{Hofmannsthal, Hugo von 1874-02-01 – 1929-07-15@\textsc{Hofmannsthal, Hugo von} (1874-02-01 – 1929-07-15), \emph{Schriftsteller/Schriftstellerin}|pw}{ }\label{K_L00483-2v}\edtext{neulich im Kaffeehaus}{\lemma{\textnormal{\emph{neulich im Kaffeehaus}}}\Cendnote{\textnormal{am 12. 9. 1895}}}\label{K_L00483-2}. \textsc{Hugo}\pwindex{Hofmannsthal, Hugo von 1874-02-01 – 1929-07-15@\textsc{Hofmannsthal, Hugo von} (1874-02-01 – 1929-07-15), \emph{Schriftsteller/Schriftstellerin}|pw} ritt durch Wien\oindex{Wien@\textbf{Wien}, \emph{A.ADM2}|pw}; ſie ſtanden beim Tegethoffmonument\oindex{Tegetthoff-Denkmal@\textbf{Tegetthoff-Denkmal}, \emph{Monument (K.MON)}|pw} und ſchauten zu. Er war in Göding\oindex{Hodonín@\textbf{Hodonín}, \emph{P.PPL}|pw}{ }ſehr unglücklich; die Manöver ſollen {\pb}ihm enorm gefallen haben. Jetzt iſt er in Bruck\oindex{Bruck an der Mur@\textbf{Bruck an der Mur}, \emph{P.PPLA3}|pw}. –\hspace*{1.5em}Geſprochen: \textsc{Salten}\pwindex{Salten, Felix 06.09.1869 – 08.10.1945@\textsc{Salten, Felix} (06.09.1869 – 08.10.1945), \emph{Schriftsteller/Schriftstellerin, Journalist/Journalistin, Chefredakteur/Chefredakteurin}|pw} oft, \textsc{Schwarzkopf}\pwindex{Schwarzkopf, Gustav 07.11.1853 – 13.11.1939@\textsc{Schwarzkopf, Gustav} (07.11.1853 – 13.11.1939), \emph{Schriftsteller/Schriftstellerin}|pw} einige Mal, \textsc{Gold}\pwindex{Gold, Alfred 28.06.1874 – 24.10.1958@\textsc{Gold, Alfred} (28.06.1874 – 24.10.1958), \emph{Schriftsteller/Schriftstellerin, Journalist/Journalistin, Kunsthändler/Kunsthändlerin}|pw}{ }ſelten, \textsc{Bahr}\pwindex{Bahr, Hermann 19.07.1863 – 15.01.1934@\textsc{Bahr, Hermann} (19.07.1863 – 15.01.1934), \emph{Schriftsteller/Schriftstellerin, Kritiker/Kritikerin}|pw} (Guten Tag, wie gehts dir denn?) Seine Frau\pwindex{Bahr, Rosa 26.10.1871 – 17.02.1940@\textsc{Bahr, Rosa} (26.10.1871 – 17.02.1940), \emph{Schauspieler/Schauspielerin}|pwv} heute ein Stück begleitet, mich dringlich zum Beſuche
               aufgefordert. Auch \uline{er} fährt ſchon \textsc{bicycle}. –\pend
           
\pstart
           – Gearbeitet noch gar nichts – ſchämen Sie ſich, daſs ich mich nicht vor Ihnen zu
               ſchämen brauche.\pend
           
\pstart
           Die Brion\pwindex{Brion, Lou 17.12.1864 – 16.05.1942@\textsc{Brion, Lou} (17.12.1864 – 16.05.1942), \emph{Schauspieler/Schauspielerin}|pw}{ }ſoll über uns geäußert haben: Setzen ſich in die
               Proſceniumsloge – und {\pb}man kriegt kein \textsc{Bracelet}, nicht einmal eine Einladung zum \textsc{Souper}! – Quelle unlauter, nemlich Paul Horn\pwindex{Horn, Paul 13.02.1867 – 18.01.1936@\textsc{Horn, Paul} (13.02.1867 – 18.01.1936), \emph{Fabrikant/Fabrikantin}|pw}. Dieſer tadelt an der kleinen Komödie\pwindex{Liebelei. Schauspiel in drei Akten@\emph{Liebelei. Schauspiel in drei Akten}|pwv} die Unmöglichkeit, daſs ſich ein Menſch
               wirklich von den Seidenſtrümpfen und den \textsc{grande marque}
               Cocotten zu einem lieben Vorſtadtmädel hingezogen fühlen ſollte. –\pend
           
\pstart
           Hier regnet es i{\geminationm}er – und Sie? – Alles erkundigt ſich
               nach Ihnen; ſind Sie ſtolz? Leben Sie wohl, laſſen Sie ſchnell {\pb}wieder was von ſich hören, bringen Sie den fertigen
                  Götterliebling\pwindex{Tod Georgs@\emph{Der Tod Georgs}|pwv} und viel
               Luſt zu neuen Werken mit. Sagen Sie, wie hat denn die Lou\pwindex{Brion, Lou 17.12.1864 – 16.05.1942@\textsc{Brion, Lou} (17.12.1864 – 16.05.1942), \emph{Schauspieler/Schauspielerin}|pw} das Alleinfahrenmüſſen aufgeno{\geminationm}en? Hier ist es »bekannt geworden« daſs wir miteinander nicht über Literatur reden;
               man findet das höchſt anmaßend – »ſo groß ſind ſie nicht, daß ſie nicht mehr über
               Literatur reden müßten.« – Laßt uns lächeln.\pend
           \pstart Ihr \spacefill\mbox{Arthur Sch} mit vielen herzlichen Grüßen.\pend{}\selectlanguage{ngerman}\endnumbering\briefempfaengerindex{Beer-Hofmann, Richard@\textsc{Beer-Hofmann, Richard}!zzzSchnitzler, Arthur@\emph{von Arthur Schnitzler}!1895-09-151@{15. 9. 1895}|)be}\mylabel{L00483h}  \normalsize

\doendnotes{C}
\bigskip
\vfill

\clearpage

\footnotesize

\lohead{\textsc{register}}

% Definiere theindex-Environment komplett neu ohne reledmac
\makeatletter
\renewenvironment{theindex}{%
  \section*{\indexname}%
  \setlength{\parindent}{0pt}%
  \setlength{\parskip}{0pt plus 0.3pt}%
  \let\item\@idxitem
}{%
  \clearpage
}
\makeatother

\IfFileExists{\jobname-pw.ind}{\input{\jobname-pw.ind}}{}

\end{document}

      