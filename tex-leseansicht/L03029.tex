%% latex-leseansicht-vorspann.tex
%% Vorspann für die Leseansicht.
%% Lädt die gemeinsame Datei latex-vorspann.tex mit nicht gesetztem Schalter.

\newif\ifkorrekturansicht
\korrekturansichtfalse

\input{../tex-inputs/latex-vorspann}

\begin{center}
            \textcolor{red}{ENTWURF, NICHT FERTIG KORRIGIERT}
                      \end{center}
            
         
         \newcommand{\erwaehntePersonen}{Personen: Gustav Kadelburg, Heinrich Kadelburg, Felix Salten, Adele Sandrock}
         \newcommand{\erwaehnteOrte}{Orte: Café Central, Wien}
         \newcommand{\erwaehnteWerke}{Werke: Adele Sandrock und das Volkstheater, Das Märchen. Schauspiel in drei Aufzügen, Der Fall Sandrock, Hinter den Coulissen [Sandrock im Volkstheater], Neues Wiener Journal, Theater und Kunst. [Fräulein Adele Sandrock dürfte…]}
               \section[Arthur Schnitzler an Felix Salten, {[}2. 4. 1894?{]}]{ Arthur Schnitzler an Felix Salten, {[}2. 4. 1894?{]}}\nopagebreak\mylabel{v}\rehead{ }\begin{ledgroupsized}[t]{13cm}\normalsize\beginnumbering \toendnotes[C]{\smallbreak\pagebreak[2]} \Standort{Wienbibliothek im Rathaus, ZPH 1681, 2.1.516.}
\physDesc{
\newline{}Handschrift: , deutsche Kurrent}\toendnotes[C]{\smallbreak}\pstart
           \noindent{}{\pb}Lieber Freund, Frl. S.\pwindex{Sandrock, Adele 1863-08-19 – 1937-08-30@\textsc{Sandrock, Adele} (1863-08-19 – 1937-08-30), \emph{Schauspielerin}|pw}
               telephonirt mir eben, daſs ſie zu nervös iſt, Abends u. ſ. w.– Eine mit der
                  \label{K_L03029-1v}\edtext{Kadelburg\pwindex{Kadelburg, Gustav 26.07.1851 – 11.09.1925@\textsc{Kadelburg, Gustav} (26.07.1851 – 11.09.1925), \emph{Schriftsteller, Schauspieler}|pw}aiffaire}{\lemma{\textnormal{\emph{Kadelburgaiffaire}}}\Cendnote{\textnormal{Am
                     30. 3. 1894 erschien im \emph{Neuen Wiener Journal}\pwindex{Neues Wiener Journal1893 – 1939@\emph{Neues Wiener Journal} {[}1893 – 1939{]}|pwk} in der Rubrik »Theater und Kunst« die Meldung\pwindex{?? Werk@Nicht ermittelte Verfasserinnen und Verfasser!Theater und Kunst. [Fraeulein Adele Sandrock duerfte…]1894-03-30@\emph{Theater und Kunst. [Fräulein Adele Sandrock dürfte…]} {[}1894-03-30{]}|pwkv}
                     (Nr. 154, S. 6), dass Adele Sandrock\pwindex{Sandrock, Adele 1863-08-19 – 1937-08-30@\textsc{Sandrock, Adele} (1863-08-19 – 1937-08-30), \emph{Schauspielerin}|pwk} von Auftritten ferngehalten werde und durch den Regisseur
                     Heinrich Kadelburg\pwindex{Kadelburg, Heinrich 14.02.1856 – 13.07.1910@\textsc{Kadelburg, Heinrich} (14.02.1856 – 13.07.1910), \emph{Schriftsteller, Regisseur, Schauspieler}|pwk} gemobbt worden sei.
                  An den Folgetagen erschienen mehrere Dementi (\emph{Hinter den Coulissen}\pwindex{?? Werk@Nicht ermittelte Verfasserinnen und Verfasser!Hinter den Coulissen [Sandrock im Volkstheater]31. 03. 1894@\emph{Hinter den Coulissen [Sandrock im Volkstheater]} {[}31. 03. 1894{]}|pwk},
                        31. 3. 1894, Nr. 155, S. 5;
                        \emph{Adele Sandrock und das
                        Volkstheater}\pwindex{?? Werk@Nicht ermittelte Verfasserinnen und Verfasser!Adele Sandrock und das Volkstheater01. 04. 1894@\emph{Adele Sandrock und das Volkstheater} {[}01. 04. 1894{]}|pwk}, 1. 4. 1894, Nr. 156,
                     S. 5). Am 4. 4. 1894 folgte eines von Schnitzler\pwindex{Schnitzler, Arthur 15.05.1862 – 21.10.1931@\textsc{Schnitzler, Arthur} (15.05.1862 – 21.10.1931), \emph{Schriftsteller, Mediziner}|pwk}, dass er \emph{Das Märchen}\pwindex{Schnitzler, Arthur 15.05.1862 – 21.10.1931@\textsc{Schnitzler, Arthur} (15.05.1862 – 21.10.1931), \emph{Schriftsteller, Mediziner}!Maerchen. Schauspiel in drei Aufzuegen1893-12-01@\strich\emph{Das Märchen. Schauspiel in drei Aufzügen} {[}1893-12-01{]}|pwk} nicht speziell für Sandrock\pwindex{Sandrock, Adele 1863-08-19 – 1937-08-30@\textsc{Sandrock, Adele} (1863-08-19 – 1937-08-30), \emph{Schauspielerin}|pwk} geschrieben habe. (\emph{Der Fall Sandrock}\pwindex{Schnitzler, Arthur 15.05.1862 – 21.10.1931@\textsc{Schnitzler, Arthur} (15.05.1862 – 21.10.1931), \emph{Schriftsteller, Mediziner}!Fall Sandrock03. 04. 1894@\strich\emph{Der Fall Sandrock} {[}03. 04. 1894{]}|pwk}, Nr. 158,
                     S. 5.) Das vorliegende Korrespondenzstück ist undatiert, dürfte aber in den Zeitraum
                  des Skandals fallen – und da an diesen Tagen nur am 2. 4. 1894 ein
                  Treffen mit Salten\pwindex{Salten, Felix 06.09.1869 – 08.10.1945@\textsc{Salten, Felix} (06.09.1869 – 08.10.1945), \emph{Schriftsteller, Journalist}|pwk} festgehalten ist, das sich noch dazu
                     auch im Café Central\oindex{Cafe Central@\textbf{Café Central}|pwk} zugetragen haben könnte, lässt sich
                  eine – wenngleich unsichere – Datierung erreichen.}}}\label{K_L03029-1h} zuſa{\geminationm}enhängende
               Klagegeſchichte.– Jeden{\pb}falls treffen wir, Sie, u ich uns Abends um 10 im
                  \textsc{Central\oindex{Cafe Central@\textbf{Café Central}|pw}}.– \pend
           \pstart
           – Ja richtig: Sie möchten nicht böſe ſein.– \pend
           \pstart
           Herzlichen Gruß {\\[\baselineskip]}Ihr \spacefill\mbox{ArthurSch.}\pend
           \leftskip=0em{}
         
         \endnumbering\mylabel{h}\end{ledgroupsized}\begin{anhang}\end{anhang}\newcommand{\dateiname}{L03029}\newcommand{\titel}{Arthur Schnitzler an Felix Salten, [2. 4. 1894?]}\newcommand{\editorInnen}{Martin Anton Müller und Laura Untner}%% latex-leseansicht-abspann.tex
%% Abspann für die Leseansicht.
%% Der Schalter \ifkorrekturansicht ist bereits durch den Vorspann gesetzt.

%% latex-abspann.tex
%% Gemeinsamer Abspann für Korrekturansicht und Leseansicht.
%% Setzt den Schalter \ifkorrekturansicht voraus (gesetzt in den
%% einbindenden Dateien latex-korrekturansicht-abspann.tex bzw.
%% latex-leseansicht-abspann.tex).
%% ---------------------------------------------------------------

\normalsize

% Das esempio-Environment wird nur in der Leseansicht benötigt
\ifkorrekturansicht\else
\newenvironment{esempio}[3]%
{
    \vspace{1.5ex}
    \rlap{\underline{#1}}
    \par
    \setlength{\parindent}{0cm}
    \nopagebreak
    \leftskip=#2cm
    \rightskip=#3cm
}
{
    \par
}
\fi

\doendnotes{C}
\bigskip
\vfill

\clearpage

\footnotesize

\ifkorrekturansicht
  \lohead{\textsc{register}}
\fi

% theindex-Environment neu definieren ohne reledmac
\makeatletter
\renewenvironment{theindex}{%
  \ifkorrekturansicht
    \section*{\indexname}%
  \else
    \subsubsection*{Index der erwähnten Entitäten}%
  \fi
  \setlength{\parindent}{0pt}%
  \setlength{\parskip}{0pt plus 0.3pt}%
  \let\item\@idxitem
}{%
  \ifkorrekturansicht\clearpage\fi
}
\makeatother

\IfFileExists{\jobname-pw.ind}{\input{\jobname-pw.ind}}{}

% Quellenangabe nur in der Leseansicht
\ifkorrekturansicht\else
% Fallback-Definitionen, falls die .tex-Datei \titel etc. nicht gesetzt hat
\providecommand{\titel}{}
\providecommand{\editorInnen}{}
\providecommand{\dateiname}{\jobname}

\vspace{3cm}

\vfill

\footnotesize
\textsc{Quelle}: \titel. Herausgegeben von {\editorInnen}. In: \emph{Arthur Schnitzler: Briefwechsel mit Autorinnen und Autoren}.
 Digitale Edition, https://schnitzler-briefe.acdh.oeaw.ac.at/{\dateiname}.html (Stand \today)
\fi

\end{document}


      