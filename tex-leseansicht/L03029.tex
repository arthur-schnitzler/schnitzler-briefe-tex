%% latex-korrekturansicht-vorspann.tex
%% Vorspann für die Korrekturansicht.
%% Lädt die gemeinsame Datei latex-vorspann.tex mit gesetztem Schalter.

\newif\ifkorrekturansicht
\korrekturansichttrue

\input{../tex-inputs/latex-vorspann}


\section[Arthur Schnitzler an Felix Salten, {[}2. 4. 1894?{]}]{L03029 Arthur Schnitzler an Felix Salten, {[}2. 4. 1894?{]}}
\nopagebreak\mylabel{L03029v}
\rehead{ }\normalsize\beginnumbering\briefempfaengerindex{Salten, Felix@\textsc{Salten, Felix}!zzzSchnitzler, Arthur@\emph{von Arthur Schnitzler}!1894-04-022@{{[}2. 4. 1894?{]}}|(be}
\toendnotes[C]{\smallbreak\pagebreak[2]}\Standort{Wienbibliothek im Rathaus, ZPH 1681, 2.1.516.}
\physDesc{Briefkarte, 279 Zeichen (Briefkarte mit Trauerrand)
\newline{}Handschrift: schwarze Tinte, deutsche Kurrent
\newline{}Ordnung: mit Bleistift von unbekannter Hand nummeriert: »32« }\toendnotes[C]{\smallbreak}
\pstart
           \noindent{}{\pb}Lieber Freund; Frl. S.\pwindex{Sandrock, Adele 1863-08-19 – 1937-08-30@\textsc{Sandrock, Adele} (1863-08-19 – 1937-08-30), \emph{Schauspieler/Schauspielerin}|pw} telephonirt mir eben, daſs ſie zu nervös iſt, Abends
               u. ſ. w. – Eine mit der \label{K_L03029-1v}\edtext{Kadelburg\pwindex{Kadelburg, Gustav 26.07.1851 – 11.09.1925@\textsc{Kadelburg, Gustav} (26.07.1851 – 11.09.1925), \emph{Schriftsteller/Schriftstellerin, Schauspieler/Schauspielerin}|pw}affaire}{\lemma{\textnormal{\emph{Kadelburgaffaire}}}\Cendnote{\textnormal{Am 30. 3. 1894 war im \emph{Neuen Wiener Journal}\pwindex{Neues Wiener Journal@\emph{Neues Wiener Journal}|pwk} in der Rubrik »Theater
                  und Kunst« die Meldung\pwindex{Theater und Kunst. [Fraeulein Adele Sandrock duerfte…]@\emph{Theater und Kunst. [Fräulein Adele Sandrock dürfte…]}|pwkv}
                  erschienen (Nr. 154, S. 6), dass Adele Sandrock\pwindex{Sandrock, Adele 1863-08-19 – 1937-08-30@\textsc{Sandrock, Adele} (1863-08-19 – 1937-08-30), \emph{Schauspieler/Schauspielerin}|pwk} von Auftritten ferngehalten werde und durch den Regisseur
                     Heinrich Kadelburg\pwindex{Kadelburg, Heinrich 14.02.1856 – 13.07.1910@\textsc{Kadelburg, Heinrich} (14.02.1856 – 13.07.1910), \emph{Schriftsteller/Schriftstellerin, Regisseur/Regisseurin, Schauspieler/Schauspielerin}|pwk} schikaniert worden sei.
                  An den Folgetagen erschienen mehrere Dementi (\emph{Hinter den Coulissen}\pwindex{Hinter den Coulissen [Sandrock im Volkstheater]@\emph{Hinter den Coulissen [Sandrock im Volkstheater]}|pwk}, 31. 3. 1894, Nr. 155, S. 5; \emph{Adele Sandrock und das Volkstheater}\pwindex{Adele Sandrock und das Volkstheater@\emph{Adele Sandrock und das Volkstheater}|pwk}, 1. 4. 1894, Nr. 156, S. 5). Am 4. 4. 1894 folgte eines von Schnitzler, worin er meinte, dass er \emph{Das Märchen}\pwindex{Maerchen. Schauspiel in drei Aufzuegen@\emph{Das Märchen. Schauspiel in drei Aufzügen}|pwk} nicht speziell für Sandrock\pwindex{Sandrock, Adele 1863-08-19 – 1937-08-30@\textsc{Sandrock, Adele} (1863-08-19 – 1937-08-30), \emph{Schauspieler/Schauspielerin}|pwk} geschrieben habe (vgl. \emph{Der Fall Sandrock}\pwindex{Fall Sandrock@\emph{Der Fall Sandrock}|pwk}, Nr. 158, S. 5). Das
                  vorliegende Korrespondenzstück ist undatiert, dürfte aber in den Zeitraum des
                  Skandals fallen – und da an diesen Tagen nur für den 2. 4. 1894 ein
                  Treffen mit Salten\pwindex{Salten, Felix 06.09.1869 – 08.10.1945@\textsc{Salten, Felix} (06.09.1869 – 08.10.1945), \emph{Schriftsteller/Schriftstellerin, Journalist/Journalistin, Chefredakteur/Chefredakteurin}|pwk} festgehalten ist und Schnitzler auch im Café Central\oindex{Cafe Central@\textbf{Café Central}, \emph{Kaffeehaus (K.KAF)}|pwk} war, lässt sich eine – wenngleich unsichere –
                  Datierung erreichen.}}}\label{K_L03029-1} zuſa{\geminationm}enhängende
               Klagegeſchichte. – Jeden{\pb}falls treffen wir,
               Sie, u ich uns Abends um 10 im \textsc{Central\oindex{Cafe Central@\textbf{Café Central}, \emph{Kaffeehaus (K.KAF)}|pw}}. –\pend
           
\pstart
           – Ja richtig: Sie möchten nicht böſe ſein. –\pend
           
\pstart
           Herzlichen Gruß {\\[\baselineskip]}Ihr \spacefill\mbox{ArthurSch.}\pend
           \leftskip=0em{}\selectlanguage{ngerman}\endnumbering\briefempfaengerindex{Salten, Felix@\textsc{Salten, Felix}!zzzSchnitzler, Arthur@\emph{von Arthur Schnitzler}!1894-04-022@{{[}2. 4. 1894?{]}}|)be}\mylabel{L03029h}  \normalsize

\doendnotes{C}
\bigskip
\vfill

\clearpage

\footnotesize

\lohead{\textsc{register}}

% Definiere theindex-Environment komplett neu ohne reledmac
\makeatletter
\renewenvironment{theindex}{%
  \section*{\indexname}%
  \setlength{\parindent}{0pt}%
  \setlength{\parskip}{0pt plus 0.3pt}%
  \let\item\@idxitem
}{%
  \clearpage
}
\makeatother

\IfFileExists{\jobname-pw.ind}{\input{\jobname-pw.ind}}{}

\end{document}

      