%% latex-leseansicht-vorspann.tex
%% Vorspann für die Leseansicht.
%% Lädt die gemeinsame Datei latex-vorspann.tex mit nicht gesetztem Schalter.

\newif\ifkorrekturansicht
\korrekturansichtfalse

\input{../tex-inputs/latex-vorspann}


\section[Arthur Schnitzler an Felix Salten, {[}2. 4. 1894?{]}]{L03029 Arthur Schnitzler an Felix Salten, {[}2. 4. 1894?{]}}
\nopagebreak\mylabel{L03029v}
\rehead{ }\normalsize\beginnumbering\briefempfaengerindex{Salten, Felix@\textsc{Salten, Felix}!zzzSchnitzler, Arthur@\emph{von Arthur Schnitzler}!1894-04-022@{{[}2. 4. 1894?{]}}|(be}
\toendnotes[C]{\smallbreak\pagebreak[2]}
\correspDesc{Versand  durch Arthur Schnitzler am [2. 4. 1894?] in Wien
\newline{}Erhalt  durch Felix Salten im Zeitraum [2. 4. 1894
                  – 5. 4. 1894?] in Wien}\toendnotes[C]{\smallbreak}
\Standort{Wienbibliothek im Rathaus, ZPH 1681, 2.1.516.}
\physDesc{Briefkarte, 279 Zeichen (Briefkarte mit Trauerrand)
\newline{}Handschrift: schwarze Tinte, deutsche Kurrent
\newline{}Ordnung: mit Bleistift von unbekannter Hand nummeriert: »32« }\toendnotes[C]{\smallbreak}
\pstart
           \noindent{}{\pb}Lieber Freund; Frl. S.\pwindex{Sandrock, Adele 19.\,8.\,1863 Rotterdam – 30.\,8.\,1937 Berlin@\textsc{Sandrock, Adele} (19.\,8.\,1863 Rotterdam – 30.\,8.\,1937 Berlin), \emph{Schauspielerin}|pw} telephonirt mir eben, daſs{ }ſie zu nervös iſt, Abends
               u. ſ. w. – Eine mit der \label{K_L03029-1v}\edtext{Kadelburg\pwindex{Kadelburg, Heinrich 14.\,2.\,1856 Budapest – 13.\,7.\,1910 Marienbad@\textsc{Kadelburg, Heinrich} (14.\,2.\,1856 Budapest – 13.\,7.\,1910 Marienbad), \emph{Schriftsteller, Regisseur, Schauspieler}|pw}affaire}{\lemma{\textnormal{\emph{Kadelburgaffaire}}}\Cendnote{\textnormal{Am 30. 3. 1894 war im \emph{Neuen Wiener Journal}\pwindex{Neues Wiener Journal@\emph{Neues Wiener Journal}|pwk} in der Rubrik »Theater
                  und Kunst« die Meldung\pwindex{Theater und Kunst. [Fräulein Adele Sandrock dürfte…]@\emph{Theater und Kunst. [Fräulein Adele Sandrock dürfte…]}|pwkv}
                  erschienen (Nr. 154, S. 6), dass Adele Sandrock\pwindex{Sandrock, Adele 19.\,8.\,1863 Rotterdam – 30.\,8.\,1937 Berlin@\textsc{Sandrock, Adele} (19.\,8.\,1863 Rotterdam – 30.\,8.\,1937 Berlin), \emph{Schauspielerin}|pwk} von Auftritten ferngehalten werde und durch den Regisseur
                     Heinrich Kadelburg\pwindex{Kadelburg, Heinrich 14.\,2.\,1856 Budapest – 13.\,7.\,1910 Marienbad@\textsc{Kadelburg, Heinrich} (14.\,2.\,1856 Budapest – 13.\,7.\,1910 Marienbad), \emph{Schriftsteller, Regisseur, Schauspieler}|pwk} schikaniert worden sei.
                  An den Folgetagen erschienen mehrere Dementi (\emph{Hinter den Coulissen}\pwindex{Hinter den Coulissen [Sandrock im Volkstheater]@\emph{Hinter den Coulissen [Sandrock im Volkstheater]}|pwk}, 31. 3. 1894, Nr. 155, S. 5; \emph{Adele Sandrock und das Volkstheater}\pwindex{Adele Sandrock und das Volkstheater@\emph{Adele Sandrock und das Volkstheater}|pwk}, 1. 4. 1894, Nr. 156, S. 5). Am 4. 4. 1894 folgte eines von Schnitzler, worin er meinte, dass er \emph{Das Märchen}\pwindex{Schnitzler, Arthur 15. 5. 1862 Wien – 21. 10. 1931 ebd.@\textsc{Schnitzler, Arthur} (15. 5. 1862 Wien – 21. 10. 1931 ebd.), \emph{Schriftsteller, Mediziner}!Märchen. Schauspiel in drei Aufzügen@\strich\emph{Das Märchen. Schauspiel in drei Aufzügen}|pwk} nicht speziell für Sandrock\pwindex{Sandrock, Adele 19.\,8.\,1863 Rotterdam – 30.\,8.\,1937 Berlin@\textsc{Sandrock, Adele} (19.\,8.\,1863 Rotterdam – 30.\,8.\,1937 Berlin), \emph{Schauspielerin}|pwk} geschrieben habe (vgl. \emph{Der Fall Sandrock}\pwindex{Schnitzler, Arthur 15. 5. 1862 Wien – 21. 10. 1931 ebd.@\textsc{Schnitzler, Arthur} (15. 5. 1862 Wien – 21. 10. 1931 ebd.), \emph{Schriftsteller, Mediziner}!Fall Sandrock@\strich\emph{Der Fall Sandrock}|pwk}, Nr. 158, S. 5). Das
                  vorliegende Korrespondenzstück ist undatiert, dürfte aber in den Zeitraum des
                  Skandals fallen – und da an diesen Tagen nur für den 2. 4. 1894 ein
                  Treffen mit Salten\pwindex{Salten, Felix 6.\,9.\,1869 Budapest – 8.\,10.\,1945 Zürich@\textsc{Salten, Felix} (6.\,9.\,1869 Budapest – 8.\,10.\,1945 Zürich), \emph{Schriftsteller, Journalist, Chefredakteur}|pwk} festgehalten ist und Schnitzler auch im Café Central\oindex{Wien@\textbf{Wien}!I., Innere Stadt@\textbf{I., Innere Stadt}!Café Central@\textbf{Café Central}, \emph{Kaffeehaus}|pwk} war, lässt sich eine – wenngleich unsichere –
                  Datierung erreichen.}}}\label{K_L03029-1} zuſa{\geminationm}enhängende
               Klagegeſchichte. – Jeden{\pb}falls treffen wir,
               Sie, u ich uns Abends um 10 im \textsc{Central\oindex{Wien@\textbf{Wien}!I., Innere Stadt@\textbf{I., Innere Stadt}!Café Central@\textbf{Café Central}, \emph{Kaffeehaus}|pw}}. –\pend
           
\pstart
           – Ja richtig: Sie möchten nicht böſe{ }ſein. –\pend
           
\pstart
           Herzlichen Gruß {\\[\baselineskip]}Ihr \spacefill\mbox{ArthurSch.}\pend
           \leftskip=0em{}\selectlanguage{ngerman}\endnumbering\briefempfaengerindex{Salten, Felix@\textsc{Salten, Felix}!zzzSchnitzler, Arthur@\emph{von Arthur Schnitzler}!1894-04-022@{{[}2. 4. 1894?{]}}|)be}\mylabel{L03029h}  \newcommand{\dateiname}{L03029}\newcommand{\titel}{Arthur Schnitzler an Felix Salten, [2. 4. 1894?]}\newcommand{\editorInnen}{Martin Anton Müller und Laura Untner}%% latex-leseansicht-abspann.tex
%% Abspann für die Leseansicht.
%% Der Schalter \ifkorrekturansicht ist bereits durch den Vorspann gesetzt.

%% latex-abspann.tex
%% Gemeinsamer Abspann für Korrekturansicht und Leseansicht.
%% Setzt den Schalter \ifkorrekturansicht voraus (gesetzt in den
%% einbindenden Dateien latex-korrekturansicht-abspann.tex bzw.
%% latex-leseansicht-abspann.tex).
%% ---------------------------------------------------------------

\normalsize

% Das esempio-Environment wird nur in der Leseansicht benötigt
\ifkorrekturansicht\else
\newenvironment{esempio}[3]%
{
    \vspace{1.5ex}
    \rlap{\underline{#1}}
    \par
    \setlength{\parindent}{0cm}
    \nopagebreak
    \leftskip=#2cm
    \rightskip=#3cm
}
{
    \par
}
\fi

\doendnotes{C}
\bigskip
\vfill

\clearpage

\footnotesize

\ifkorrekturansicht
  \lohead{\textsc{register}}
\fi

% theindex-Environment neu definieren ohne reledmac
\makeatletter
\renewenvironment{theindex}{%
  \ifkorrekturansicht
    \section*{\indexname}%
  \else
    \subsubsection*{Index der erwähnten Entitäten}%
  \fi
  \setlength{\parindent}{0pt}%
  \setlength{\parskip}{0pt plus 0.3pt}%
  \let\item\@idxitem
}{%
  \ifkorrekturansicht\clearpage\fi
}
\makeatother

\IfFileExists{\jobname-pw.ind}{\input{\jobname-pw.ind}}{}

% Quellenangabe nur in der Leseansicht
\ifkorrekturansicht\else
% Fallback-Definitionen, falls die .tex-Datei \titel etc. nicht gesetzt hat
\providecommand{\titel}{}
\providecommand{\editorInnen}{}
\providecommand{\dateiname}{\jobname}

\vspace{3cm}

\vfill

\footnotesize
\textsc{Quelle}: \titel. Herausgegeben von {\editorInnen}. In: \emph{Arthur Schnitzler: Briefwechsel mit Autorinnen und Autoren}.
 Digitale Edition, https://schnitzler-briefe.acdh.oeaw.ac.at/{\dateiname}.html (Stand \today)
\fi

\end{document}


