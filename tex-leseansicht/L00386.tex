%% latex-korrekturansicht-vorspann.tex
%% Vorspann für die Korrekturansicht.
%% Lädt die gemeinsame Datei latex-vorspann.tex mit gesetztem Schalter.

\newif\ifkorrekturansicht
\korrekturansichttrue

\input{../tex-inputs/latex-vorspann}


\section[Arthur Schnitzler an Richard Beer-Hofmann, 19. 10. 1894]{L00386 Arthur Schnitzler an Richard Beer-Hofmann, 19. 10. 1894}
\nopagebreak\mylabel{L00386v}
\rehead{ }\normalsize\beginnumbering\briefempfaengerindex{Beer-Hofmann, Richard@\textsc{Beer-Hofmann, Richard}!zzzSchnitzler, Arthur@\emph{von Arthur Schnitzler}!1894-10-191@{19. 10. 1894}|(be}
\toendnotes[C]{\smallbreak\pagebreak[2]}\Standort{YCGL, MSS 31.}
\physDesc{Postkarte, 557 Zeichen
\newline{}Handschrift: Bleistift, deutsche Kurrent
\newline{}Versand: 1) nachgesandt nach \textsc{Hotel Hassler}\oindex{Hôtel Hassler@\textbf{Hôtel Hassler}, \emph{Hotel (K.HTL)}|pw}  2) Stempel: »\nobreak{}\oindex{I., Innere Stadt@\textbf{I., Innere Stadt}, \emph{A.ADM3}|pwk}Wien 1/1, 19. 10. 94, 9–10 N\nobreak{}«.  3) Stempel: »\nobreak{}\oindex{Neapel@\textbf{Neapel}, \emph{P.PPLA}|pwk}Napoli, 21 10–94, 8 S\nobreak{}«. }
\buchAbdrucke{\weitereDrucke{Arthur Schnitzler, Richard Beer-Hofmann: \emph{Briefwechsel 1891–1931}. Wien, Zürich: \emph{Europaverlag} 1992, S. 65.} }\toendnotes[C]{\smallbreak}\pstart{}{\pb}Herrn \textsc{Dr. Richard Beer
                     Hofmann}\pend{}\pstart{}\textsc{Neapel\oindex{Neapel@\textbf{Neapel}, \emph{P.PPLA}|pw}}\pend{}\pstart{}\textsc{a posta ferma}\pend{}\pstart{}\textsc{\label{T_L00386-1v}\edtext{Italien\oindex{Italien@\textbf{Italien}, \emph{A.PCLI}|pw}}{\lemma{\textnormal{\emph{Italien}}}\Cendnote{\textnormal{in jede Ecke der Karte
                        geschrieben}}}\label{T_L00386-1}}\pend{}{\bigskip}\vspace{1em}
\pstart
           \noindent{}{\pb}Lieber Richard, ich habe Ihren Brief aus
                  \textsc{Frascati}\oindex{Frascati@\textbf{Frascati}, \emph{P.PPLA3}|pw} beko{\geminationm}en und danke beſtens. Sie meinen erſten nach
                  Neapel\oindex{Neapel@\textbf{Neapel}, \emph{P.PPLA}|pw} und die \uline{Zeit}\pwindex{Zeit. Wiener Wochenschrift@\emph{Die Zeit. Wiener Wochenschrift}|pw} doch wohl auch? Ihre gute und hohe Sti{\geminationm}ung iſt
               ſehr erfreulich – man kann gewiſs beſſeres von Reiſen heimbringen als Novellen – ob
               aber auch beſſeres – als \uline{Ihre} Novellen??? – Mein Stück\pwindex{Liebelei. Schauspiel in drei Akten@\emph{Liebelei. Schauspiel in drei Akten}|pwv} beim Abſchreiber\pwindex{?? [Schreibkraft fuer Arthur Schnitzler] @\textsc{?? [Schreibkraft für Arthur Schnitzler]}|pwv}; vielleicht ka{\geminationn} ich bei Ihrer Heimkehr ſchon mit Reſultaten aufwarten.
               Mache die Correcturen am Buch (Sterben\pwindex{Sterben. Novelle@\emph{Sterben. Novelle}|pw}.) – Heute
               arges Kopfweh. – Viele herzliche Grüße, bitte ſchreiben Sie mir.\pend
           \pstart Ihr \spacefill\mbox{Arth.}\pend{}\selectlanguage{ngerman}\endnumbering\briefempfaengerindex{Beer-Hofmann, Richard@\textsc{Beer-Hofmann, Richard}!zzzSchnitzler, Arthur@\emph{von Arthur Schnitzler}!1894-10-191@{19. 10. 1894}|)be}\mylabel{L00386h}  \normalsize

\doendnotes{C}
\bigskip
\vfill

\clearpage

\footnotesize

\lohead{\textsc{register}}

% Definiere theindex-Environment komplett neu ohne reledmac
\makeatletter
\renewenvironment{theindex}{%
  \section*{\indexname}%
  \setlength{\parindent}{0pt}%
  \setlength{\parskip}{0pt plus 0.3pt}%
  \let\item\@idxitem
}{%
  \clearpage
}
\makeatother

\IfFileExists{\jobname-pw.ind}{\input{\jobname-pw.ind}}{}

\end{document}

      