%% latex-leseansicht-vorspann.tex
%% Vorspann für die Leseansicht.
%% Lädt die gemeinsame Datei latex-vorspann.tex mit nicht gesetztem Schalter.

\newif\ifkorrekturansicht
\korrekturansichtfalse

\input{../tex-inputs/latex-vorspann}


         
         \renewcommand{\erwaehntePersonen}{Personen:  ?? [Schreibkraft für Arthur Schnitzler], Richard Beer-Hofmann}
         \renewcommand{\erwaehnteOrte}{Orte: Frascati, Hôtel Hassler, I., Innere Stadt, Italien, Neapel, Rom, Wien}
         \renewcommand{\erwaehnteWerke}{Werke: Die Zeit. Wiener Wochenschrift, Liebelei. Schauspiel in drei Akten, Sterben. Novelle}
               \section[Arthur Schnitzler an Richard Beer-Hofmann, 19. 10. 1894]{ Arthur Schnitzler an Richard Beer-Hofmann, 19. 10. 1894}\nopagebreak\mylabel{v}\rehead{ }\begin{ledgroupsized}[t]{13cm}\normalsize\beginnumbering\briefempfaengerindex{Beer-Hofmann, Richard@\textsc{Beer-Hofmann, Richard}!zzzSchnitzler, Arthur@\emph{von Arthur Schnitzler}!1894-10-191@{19. 10. 1894}|(be} \toendnotes[C]{\smallbreak\pagebreak[2]} \Standort{YCGL, MSS 31.}
\physDesc{Postkarte, 557 Zeichen
\newline{}Handschrift: Bleistift, deutsche Kurrent
\newline{}Versand: 1) nachgesandt nach \textsc{Hotel Hassler}\oindex{Hôtel Hassler@\textbf{Hôtel Hassler}|pw}  2) Stempel: »\nobreak{}\oindex{I., Innere Stadt@\textbf{I., Innere Stadt}|pwk}Wien 1/1, 19. 10. 94, 9–10 N\nobreak{}«.  3) Stempel: »\nobreak{}\oindex{Neapel@\textbf{Neapel}|pwk}Napoli, 21 10–94, 8 S\nobreak{}«. }\buchAbdrucke{\weitereDrucke{Arthur Schnitzler, Richard Beer-Hofmann: \emph{Briefwechsel 1891–1931}. Hg. Konstanze Fliedl. Wien, Zürich: \emph{Europaverlag} 1992, S. 65.} }\toendnotes[C]{\smallbreak}\pstart{}{\pb}Herrn \textsc{Dr. Richard Beer
                     Hofmann}\pend{}\pstart{}\textsc{Neapel\oindex{Neapel@\textbf{Neapel}|pw}}\pend{}\pstart{}\textsc{a posta ferma}\pend{}\pstart{}\textsc{\label{T_L00386-1v}\edtext{Italien\oindex{Italien@\textbf{Italien}|pw}}{\lemma{\textnormal{\emph{Italien}}}\Cendnote{\textnormal{in jede Ecke der Karte
                        geschrieben}}}\label{T_L00386-1h}}\pend{}{\bigskip}\pstart
           \noindent{}{\pb}Lieber Richard, ich habe Ihren Brief aus
                  \textsc{Frascati}\oindex{Frascati@\textbf{Frascati}|pw} beko{\geminationm}en und danke beſtens. Sie meinen erſten nach
                  Neapel\oindex{Neapel@\textbf{Neapel}|pw} und die \uline{Zeit}\pwindex{Zeit. Wiener Wochenschrift1894 – 1904@\emph{Die Zeit. Wiener Wochenschrift} {[}1894 – 1904{]}|pw} doch wohl auch? Ihre gute und hohe Sti{\geminationm}ung iſt
               ſehr erfreulich – man kann gewiſs beſſeres von Reiſen heimbringen als Novellen – ob
               aber auch beſſeres – als \uline{Ihre} Novellen??? – Mein Stück\pwindex{Schnitzler, Arthur 15.05.1862 – 21.10.1931@\textsc{Schnitzler, Arthur} (15.05.1862 – 21.10.1931), \emph{Schriftsteller, Mediziner}!Liebelei. Schauspiel in drei Akten1895-10-09@\strich\emph{Liebelei. Schauspiel in drei Akten} {[}1895-10-09{]}|pwv} beim Abſchreiber\pwindex{?? [Schreibkraft fuer Arthur Schnitzler] @\textsc{?? [Schreibkraft für Arthur Schnitzler]}|pwv}; vielleicht ka{\geminationn} ich bei Ihrer Heimkehr ſchon mit Reſultaten aufwarten.
               Mache die Correcturen am Buch (Sterben\pwindex{Schnitzler, Arthur 15.05.1862 – 21.10.1931@\textsc{Schnitzler, Arthur} (15.05.1862 – 21.10.1931), \emph{Schriftsteller, Mediziner}!Sterben. Novelle1894-10-01 – 1894-12-01@\strich\emph{Sterben. Novelle} {[}1894-10-01 – 1894-12-01{]}|pw}.) – Heute
               arges Kopfweh. – Viele herzliche Grüße, bitte ſchreiben Sie mir.\pend
           \pstart Ihr \spacefill\mbox{Arth.}\pend{}
         
         \endnumbering\mylabel{h}\end{ledgroupsized}  \newcommand{\dateiname}{L00386}\newcommand{\titel}{Arthur Schnitzler an Richard Beer-Hofmann, 19. 10. 1894}\newcommand{\editorInnen}{Martin Anton Müller und Gerd-Hermann Susen}%% latex-leseansicht-abspann.tex
%% Abspann für die Leseansicht.
%% Der Schalter \ifkorrekturansicht ist bereits durch den Vorspann gesetzt.

%% latex-abspann.tex
%% Gemeinsamer Abspann für Korrekturansicht und Leseansicht.
%% Setzt den Schalter \ifkorrekturansicht voraus (gesetzt in den
%% einbindenden Dateien latex-korrekturansicht-abspann.tex bzw.
%% latex-leseansicht-abspann.tex).
%% ---------------------------------------------------------------

\normalsize

% Das esempio-Environment wird nur in der Leseansicht benötigt
\ifkorrekturansicht\else
\newenvironment{esempio}[3]%
{
    \vspace{1.5ex}
    \rlap{\underline{#1}}
    \par
    \setlength{\parindent}{0cm}
    \nopagebreak
    \leftskip=#2cm
    \rightskip=#3cm
}
{
    \par
}
\fi

\doendnotes{C}
\bigskip
\vfill

\clearpage

\footnotesize

\ifkorrekturansicht
  \lohead{\textsc{register}}
\fi

% theindex-Environment neu definieren ohne reledmac
\makeatletter
\renewenvironment{theindex}{%
  \ifkorrekturansicht
    \section*{\indexname}%
  \else
    \subsubsection*{Index der erwähnten Entitäten}%
  \fi
  \setlength{\parindent}{0pt}%
  \setlength{\parskip}{0pt plus 0.3pt}%
  \let\item\@idxitem
}{%
  \ifkorrekturansicht\clearpage\fi
}
\makeatother

\IfFileExists{\jobname-pw.ind}{\input{\jobname-pw.ind}}{}

% Quellenangabe nur in der Leseansicht
\ifkorrekturansicht\else
% Fallback-Definitionen, falls die .tex-Datei \titel etc. nicht gesetzt hat
\providecommand{\titel}{}
\providecommand{\editorInnen}{}
\providecommand{\dateiname}{\jobname}

\vspace{3cm}

\vfill

\footnotesize
\textsc{Quelle}: \titel. Herausgegeben von {\editorInnen}. In: \emph{Arthur Schnitzler: Briefwechsel mit Autorinnen und Autoren}.
 Digitale Edition, https://schnitzler-briefe.acdh.oeaw.ac.at/{\dateiname}.html (Stand \today)
\fi

\end{document}


      