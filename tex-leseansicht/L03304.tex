%% latex-korrekturansicht-vorspann.tex
%% Vorspann für die Korrekturansicht.
%% Lädt die gemeinsame Datei latex-vorspann.tex mit gesetztem Schalter.

\newif\ifkorrekturansicht
\korrekturansichttrue

\input{../tex-inputs/latex-vorspann}


\section[ Felix Salten an Arthur Schnitzler, {[}3.? 1. 1900{]}]{L03304 Felix Salten an Arthur Schnitzler, {[}3.? 1. 1900{]}}
\nopagebreak\mylabel{L03304v}
\rehead{ }\normalsize\beginnumbering\briefempfaengerindex{Schnitzler, Arthur@\textsc{Schnitzler, Arthur}!zzzSalten, Felix@\emph{von Felix Salten}!1900-01-032@{{[}3.? 1. 1900{]}}|(be}
\toendnotes[C]{\smallbreak\pagebreak[2]}\Standort{CUL, Schnitzler, B 89, A 2.}
\physDesc{Brief, 1 Blatt, 1 Seite, 207 Zeichen
\newline{}Handschrift: Bleistift, lateinische Kurrent
\newline{}Schnitzler: mit Bleistift datiert: »\textsc{Januar 900}« 
\newline{}Ordnung: mit Bleistift von unbekannter Hand nummeriert: »128« }\toendnotes[C]{\smallbreak}
\pstart
           \noindent{}{\pb}Lieber Arthur, Sie waren gerade weg, als ich kam. Vielleicht
               schreiben Sie mir eine Zeile, wo \substVorne{}\textsuperscript{\textcolor{gray}{sie}}\substDazwischen{}Sie\substHinten{} während der \label{K_L03304-1v}\edtext{Feiertage}{\lemma{\textnormal{\emph{Feiertage}}}\Cendnote{\textnormal{Heilige Drei Könige fiel im Jahr 1900 auf einen Samstag. Der 6. 1. 1900 und der Sonntag, 7. 1. 1900,
                  waren arbeitsfrei.}}}\label{K_L03304-1} sind, \textcolor{gray}{im}{ }Club\orgindex{Wiener Schachclub@Wiener Schachclub|pwv}, ec.\pend
           
\pstart
           Herzlichst {\\[\baselineskip]}Ihr {\\[\baselineskip]}\spacefill\mbox{Salten}\pend
           \leftskip=0em{}
\pstart
           \noindent{}Ich wollte Ihnen heute auch »\label{K_L03304-2v}\edtext{das}{\lemma{\textnormal{\emph{das}}}\Cendnote{\textnormal{Unter der Annahme, dass damit das Widmungsexemplar von \emph{Der Hinterbliebene. Kurze Novellen}\pwindex{Hinterbliebene. Kurze Novellen@\emph{Der Hinterbliebene. Kurze Novellen}|pwk} (vgl. Felix Salten: Widmungsexemplar Der Hinterbliebene für Arthur
               Schnitzler, 3. 1. 1900) gemeint ist, lässt
                     sich die Datierung Schnitzlers am Blatt
                     weiter eingrenzen. }}}\label{K_L03304-2}« bringen, d. h. geben.\pend
           \selectlanguage{ngerman}\endnumbering\briefempfaengerindex{Schnitzler, Arthur@\textsc{Schnitzler, Arthur}!zzzSalten, Felix@\emph{von Felix Salten}!1900-01-032@{{[}3.? 1. 1900{]}}|)be}\mylabel{L03304h}  \normalsize

\doendnotes{C}
\bigskip
\vfill

\clearpage

\footnotesize

\lohead{\textsc{register}}

% Definiere theindex-Environment komplett neu ohne reledmac
\makeatletter
\renewenvironment{theindex}{%
  \section*{\indexname}%
  \setlength{\parindent}{0pt}%
  \setlength{\parskip}{0pt plus 0.3pt}%
  \let\item\@idxitem
}{%
  \clearpage
}
\makeatother

\IfFileExists{\jobname-pw.ind}{\input{\jobname-pw.ind}}{}

\end{document}

      