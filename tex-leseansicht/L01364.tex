%% latex-korrekturansicht-vorspann.tex
%% Vorspann für die Korrekturansicht.
%% Lädt die gemeinsame Datei latex-vorspann.tex mit gesetztem Schalter.

\newif\ifkorrekturansicht
\korrekturansichttrue

\input{../tex-inputs/latex-vorspann}


\section[Hugo von Hofmannsthal an Arthur Schnitzler, 26. 1. 1904]{L01364 Hugo von Hofmannsthal an Arthur Schnitzler, 26. 1. 1904}
\nopagebreak\mylabel{L01364v}
\rehead{ }\normalsize\beginnumbering\briefempfaengerindex{Schnitzler, Arthur@\textsc{Schnitzler, Arthur}!zzzHofmannsthal, Hugo von@\emph{von Hugo von Hofmannsthal}!1904-01-261@{26. 1. 1904}|(be}
\toendnotes[C]{\smallbreak\pagebreak[2]}\Standort{CUL, Schnitzler, B 43.}
\physDesc{Bildpostkarte, 234 Zeichen
\newline{}Handschrift: 1) schwarze Tinte, deutsche Kurrent\hspace{1em}2) schwarze Tinte, lateinische Kurrent (\noindent{}Adresse)\hspace{1em}
\newline{}Versand: 1) Stempel: »\nobreak{}\oindex{Stazione di Venezia Santa Lucia@\textbf{Stazione di Venezia Santa Lucia}, \emph{Bahnhofsgebäude (K.BHF)}|pwk}Venezia Ferrovia, 27{[}-1{]}-04, 8M\nobreak{}«.   2) Stempel: »\nobreak{}\oindex{XVIII., Waehring@\textbf{XVIII., Währing}, \emph{A.ADM3}|pwk}18/1 Wien, 28. 1. 04, 12.V, Bestellt\nobreak{}«. 
\newline{}Ordnung: mit Bleistift von unbekannter Hand nummeriert: »212« }
\buchAbdrucke{\weitereDrucke{Hugo von Hofmannsthal, Arthur Schnitzler: \emph{Briefwechsel}. Frankfurt am Main: \emph{S. Fischer} 1964, S. 182.} }\toendnotes[C]{\smallbreak}\pstart{}{\pb}Herrn D\textsuperscript{r} Arthur Schnitzler\pend{}\pstart{}Wien\oindex{Wien@\textbf{Wien}, \emph{A.ADM2}|pw}\pend{}\pstart{}XVIII Spöttelgasse 7\oindex{Edmund-Weiss-Gasse 7@\textbf{Edmund-Weiß-Gasse 7}, \emph{Wohngebäude (K.WHS)}|pw}\pend{}\pstart{}Austria\oindex{Oesterreich@\textbf{Österreich}, \emph{A.PCLI}|pw}\pend{}{\bigskip}
\pstart
           \noindent{}\centering{}{\pb}\textcolor{gray}{\textbf{Venezia – R. Accademia di Belle Arti\orgindex{Accademia di belle arti di Venezia@Accademia di belle arti di Venezia|pw}}}\pend
           
\pstart
           \centering{}\textcolor{gray}{\textbf{L’Arrivo nel Porto di Colonia della nave che
                     conduceva S. Orsola e le Vergini\pwindex{Ankuft der Pilger in Koeln@\emph{Die Ankuft der Pilger in Köln}|pw} (Carpaccio\pwindex{Carpaccio, Vittore 1465 – 1526@\textsc{Carpaccio, Vittore} (1465 – 1526), \emph{Maler/Malerin}|pw})}}\pend
           \vspace{1em}
\pstart
           \raggedleft{}{\pb}26. I.\pend
           \vspace{0.5em}
\pstart
           Hier iſt es ſchön ſtill und i{\geminationm}erfort Sonne. – S. 128 im
                  »einſ. Weg\pwindex{einsame Weg. Schauspiel in fuenf Akten@\emph{Der einsame Weg. Schauspiel in fünf Akten}|pw}« (ein ſchönes Stück!) ſteht noch
               immer die Stelle die überflüſſig an Baumeiſter \textsc{Solness}\pwindex{Baumeister Solness@\emph{Baumeister Solness}|pw}{ }\label{K_L01364-1v}\edtext{erinnert}{\lemma{\textnormal{\emph{erinnert}}}\Cendnote{\textnormal{In der Erstausgabe von \emph{Der
                     einsame Weg}\pwindex{einsame Weg. Schauspiel in fuenf Akten@\emph{Der einsame Weg. Schauspiel in fünf Akten}|pwk} (Berlin: \emph{S. Fischer}\orgindex{S. Fischer Verlag@S. Fischer Verlag|pwk}{ }1904) steht auf S. 128: »Dann bist Du vielleicht eine Prinzessin
                     geworden und ich Fürst einer versunkenen Stadt«. Das alludiert an ein
                  mit »Prinzessin« angesprochenes Mädchen, dem vom Baumeister Solness\pwindex{Baumeister Solness@\emph{Baumeister Solness}|pwkv} ein Königreich versprochen
                  wird.}}}\label{K_L01364-1}.\pend
           
\pstart
           Grüße{\\[\baselineskip]}\spacefill\mbox{Hugo.}\pend
           \leftskip=0em{}\selectlanguage{ngerman}\endnumbering\briefempfaengerindex{Schnitzler, Arthur@\textsc{Schnitzler, Arthur}!zzzHofmannsthal, Hugo von@\emph{von Hugo von Hofmannsthal}!1904-01-261@{26. 1. 1904}|)be}\mylabel{L01364h}  \normalsize

\doendnotes{C}
\bigskip
\vfill

\clearpage

\footnotesize

\lohead{\textsc{register}}

% Definiere theindex-Environment komplett neu ohne reledmac
\makeatletter
\renewenvironment{theindex}{%
  \section*{\indexname}%
  \setlength{\parindent}{0pt}%
  \setlength{\parskip}{0pt plus 0.3pt}%
  \let\item\@idxitem
}{%
  \clearpage
}
\makeatother

\IfFileExists{\jobname-pw.ind}{\input{\jobname-pw.ind}}{}

\end{document}

      