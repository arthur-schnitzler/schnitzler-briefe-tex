%% latex-korrekturansicht-vorspann.tex
%% Vorspann für die Korrekturansicht.
%% Lädt die gemeinsame Datei latex-vorspann.tex mit gesetztem Schalter.

\newif\ifkorrekturansicht
\korrekturansichttrue

\input{../tex-inputs/latex-vorspann}


\section[Arthur Schnitzler an Richard Beer-Hofmann, 11. 7. 1899]{L00937 Arthur Schnitzler an Richard Beer-Hofmann, 11. 7. 1899}
\nopagebreak\mylabel{L00937v}
\rehead{ }\normalsize\beginnumbering\briefempfaengerindex{Beer-Hofmann, Richard@\textsc{Beer-Hofmann, Richard}!zzzSchnitzler, Arthur@\emph{von Arthur Schnitzler}!1899-07-112@{11. 7. 1899}|(be}
\toendnotes[C]{\smallbreak\pagebreak[2]}\Standort{YCGL, MSS 31.}
\physDesc{Briefkarte, , Umschlag, 314 Zeichen
\newline{}Handschrift: Bleistift, deutsche Kurrent
\newline{}Versand: 1) Stempel: »\nobreak{}\oindex{I., Innere Stadt@\textbf{I., Innere Stadt}, \emph{A.ADM3}|pwk}Wien 1/1, 11. 7. 99, 8–9 N\nobreak{}«.   2) Stempel: »\nobreak{}\oindex{Seeboden@\textbf{Seeboden}, \emph{A.ADM3}|pwk}\textcolor{gray}{Seeboden}, 12. 7. 9{[}9{]}\nobreak{}«. }\pstart{}{\pb}\textsc{Dr Richard Beer-Hofmann}\pend{}\pstart{}\textsc{Villa Platzer}\oindex{Villa Platzer@\textbf{Villa Platzer}, \emph{Gebäude (K.GBD)}|pw}\pend{}\pstart{}\textsc{Seeboden am Millstätter}ſee\oindex{Seeboden@\textbf{Seeboden}, \emph{A.ADM3}|pw}\pend{}\pstart{}\textsc{Kärnthen}\oindex{Kaernten@\textbf{Kärnten}, \emph{A.ADM1}|pw}\pend{}{\bigskip}\vspace{1em}
\pstart
           \noindent{}{\pb}Lieber Richard, meine Abſicht iſt, Montag nach Velden\oindex{Velden am Woerthersee@\textbf{Velden am Wörthersee}, \emph{P.PPL}|pw} zu reiſen. Schreiben Sie mir vorher ein
               Wort. Von Velden\oindex{Velden am Woerthersee@\textbf{Velden am Wörthersee}, \emph{P.PPL}|pw} aus möcht ich Sie eventuell
                  Mittwoch oder Donnerſtg beſuchen. Sagen Sie mir – auf
               wie lang Sie {\pb}event. \textsc{Seeboden}\oindex{Seeboden@\textbf{Seeboden}, \emph{A.ADM3}|pw} verlaſſen k\textcolor{gray}{ö}nnten.\pend
           
\pstart
           Herzlich Ihr{\\[\baselineskip]}\spacefill\mbox{Arthur}\pend
           \leftskip=0em{}\selectlanguage{ngerman}\endnumbering\briefempfaengerindex{Beer-Hofmann, Richard@\textsc{Beer-Hofmann, Richard}!zzzSchnitzler, Arthur@\emph{von Arthur Schnitzler}!1899-07-112@{11. 7. 1899}|)be}\mylabel{L00937h}  \normalsize

\doendnotes{C}
\bigskip
\vfill

\clearpage

\footnotesize

\lohead{\textsc{register}}

% Definiere theindex-Environment komplett neu ohne reledmac
\makeatletter
\renewenvironment{theindex}{%
  \section*{\indexname}%
  \setlength{\parindent}{0pt}%
  \setlength{\parskip}{0pt plus 0.3pt}%
  \let\item\@idxitem
}{%
  \clearpage
}
\makeatother

\IfFileExists{\jobname-pw.ind}{\input{\jobname-pw.ind}}{}

\end{document}

      