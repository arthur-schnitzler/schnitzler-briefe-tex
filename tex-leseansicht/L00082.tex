%% latex-korrekturansicht-vorspann.tex
%% Vorspann für die Korrekturansicht.
%% Lädt die gemeinsame Datei latex-vorspann.tex mit gesetztem Schalter.

\newif\ifkorrekturansicht
\korrekturansichttrue

\input{../tex-inputs/latex-vorspann}


\section[Arthur Schnitzler an Hugo von Hofmannsthal, 16. 3. 1892]{L00082 Arthur Schnitzler an Hugo von Hofmannsthal, 16. 3. 1892}
\nopagebreak\mylabel{L00082v}
\rehead{ }\normalsize\beginnumbering\briefempfaengerindex{Hofmannsthal, Hugo von@\textsc{Hofmannsthal, Hugo von}!zzzSchnitzler, Arthur@\emph{von Arthur Schnitzler}!1892-03-161@{16. 3. 1892}|(be}
\toendnotes[C]{\smallbreak\pagebreak[2]}\Standort{FDH, Hs-30885,18.}
\physDesc{Brief, 1 Blatt, 2 Seiten, 549 Zeichen
\newline{}Handschrift: schwarze Tinte, deutsche Kurrent
\newline{}Ordnung: mit Bleistift von Schnitzler mutmaßlich bei der Durchsicht der Briefe
                                    1929 datiert: »16/\substVorne{}\textsuperscript{5}\substDazwischen{}3\substHinten{} 92«; eventuell die Korrektur der Monatsangabe
                                 von anderer Hand }
\buchAbdrucke{\weitereDrucke{Hugo von Hofmannsthal, Arthur Schnitzler: \emph{Briefwechsel}. Frankfurt am Main: \emph{S. Fischer} 1964, S. 16–17.} }\toendnotes[C]{\smallbreak}
\pstart{}{\pb}Lieber Freund,\pend\vspace{0.5em}
\pstart
           die beiliegende Karte kam an mich. Geſtern ſtellte man von derſelben Seite die \strikeout{Bedin} Frage an mich, unter welchen Bedingungen ich ev.
               mein Stück\pwindex{Maerchen. Schauspiel in drei Aufzuegen@\emph{Das Märchen. Schauspiel in drei Aufzügen}|pwuv} zum
               Abdruck überlaſſen würde. –\pend
           
\pstart
           Bèraton\pwindex{Beraton, Ferry 06.12.1859 – 11.02.1900@\textsc{Bératon, Ferry} (06.12.1859 – 11.02.1900), \emph{Schriftsteller/Schriftstellerin, Journalist/Journalistin, Maler/Malerin}|pw}{ }ſprach dieſer Tage mit mir über die materielle
               Seite des \label{K_L00082-1v}\edtext{\textsc{Maeterlinck}\pwindex{Maeterlinck, Maurice 29.08.1862 – 06.05.1949@\textsc{Maeterlinck, Maurice} (29.08.1862 – 06.05.1949), \emph{Schriftsteller/Schriftstellerin}|pw}-Abends}{\lemma{\textnormal{\emph{Maeterlinck-Abends}}}\Cendnote{\textnormal{am 2. 5. 1892.
               }}}\label{K_L00082-1}. Vorläufig habe ich ihm zehn Gulden geſchickt. Ueber dieſen Abend wäre
               manches {\pb}zu ſprechen.\pend
           
\pstart
           Möchten Sie mir die Adreſſe von \textsc{Schwarzkopf}\pwindex{Schwarzkopf, Gustav 07.11.1853 – 13.11.1939@\textsc{Schwarzkopf, Gustav} (07.11.1853 – 13.11.1939), \emph{Schriftsteller/Schriftstellerin}|pw} mittheilen? Ich möchte ihn um eine Empfehlung an \textsc{Bonz}\orgindex{Adolf Bonz und Comp.@Adolf Bonz {\kaufmannsund}  Comp.|pw} wegen meines \textsc{Anatol-Cyclus}\pwindex{Anatol@\emph{Anatol}|pw} erſuchen. Was glauben Sie? –\pend
           
\pstart
           Herzlichſt der Ihre{\\[\baselineskip]}\spacefill\mbox{Arth Sch}\pend
           \leftskip=0em{}
\pstart
           
\pstart
           16. März 92\pend
           
\pstart
           \raggedleft{}Wien\oindex{Wien@\textbf{Wien}, \emph{A.ADM2}|pw}.\pend
           \pend
           \selectlanguage{ngerman}\endnumbering\briefempfaengerindex{Hofmannsthal, Hugo von@\textsc{Hofmannsthal, Hugo von}!zzzSchnitzler, Arthur@\emph{von Arthur Schnitzler}!1892-03-161@{16. 3. 1892}|)be}\mylabel{L00082h}  \normalsize

\doendnotes{C}
\bigskip
\vfill

\clearpage

\footnotesize

\lohead{\textsc{register}}

% Definiere theindex-Environment komplett neu ohne reledmac
\makeatletter
\renewenvironment{theindex}{%
  \section*{\indexname}%
  \setlength{\parindent}{0pt}%
  \setlength{\parskip}{0pt plus 0.3pt}%
  \let\item\@idxitem
}{%
  \clearpage
}
\makeatother

\IfFileExists{\jobname-pw.ind}{\input{\jobname-pw.ind}}{}

\end{document}

      