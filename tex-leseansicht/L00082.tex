%% latex-leseansicht-vorspann.tex
%% Vorspann für die Leseansicht.
%% Lädt die gemeinsame Datei latex-vorspann.tex mit nicht gesetztem Schalter.

\newif\ifkorrekturansicht
\korrekturansichtfalse

\input{../tex-inputs/latex-vorspann}


               \section[Arthur Schnitzler an Hugo von Hofmannsthal, 16. 3. 1892]{ Arthur Schnitzler an Hugo von Hofmannsthal, 16. 3. 1892}\nopagebreak\mylabel{v}\rehead{ }\begin{ledgroupsized}[t]{13cm}\normalsize\beginnumbering\briefempfaengerindex{Hofmannsthal, Hugo von@\textsc{Hofmannsthal, Hugo von}!zzzSchnitzler, Arthur@\emph{von Arthur Schnitzler}!1892-03-161@{16. 3. 1892}|(be} \toendnotes[C]{\smallbreak\pagebreak[2]} \Standort{FDH, Hs-30885,18.}
\physDesc{Brief, 1 Blatt, 2 Seiten
\newline{}Handschrift: schwarze Tinte, deutsche Kurrent\newline{}Ordnung: von Schnitzler mutmaßlich bei der Durchsicht der Briefe
                                    1929 mit Bleistift datiert: »16/\substVorne{}\textsuperscript{5}\substDazwischen{}3\substHinten{} 92«; eventuell die Korrektur der Monatsangabe
                                 von anderer Hand }\buchAbdrucke{\weitereDrucke{Hugo von Hofmannsthal, Arthur Schnitzler: \emph{Briefwechsel}. Hg. Therese Nickl und Heinrich Schnitzler. Frankfurt am Main: \emph{S. Fischer} 1964, S. 16–17.} }\toendnotes[C]{\smallbreak}\pstart{}{\pb}Lieber Freund,\pend\pstart
           die beiliegende Karte kam an mich. Geſtern ſtellte man von derſelben Seite die \strikeout{Bedin} Frage an mich, unter welchen Bedingungen ich ev.
               mein Stück\pwindex{Schnitzler, Arthur 15.05.1862 – 21.10.1931@\textsc{Schnitzler, Arthur} (15.05.1862 – 21.10.1931), \emph{Schriftsteller, Mediziner}!Maerchen. Schauspiel in drei Aufzuegen1891 – 1891@\strich\emph{Das Märchen. Schauspiel in drei Aufzügen} {[}1891 – 1891{]}|pwuv} zum Abdruck
               überlaſſen würde. –\pend
           \pstart
           Bèraton\pwindex{Beraton, Ferry 06.12.1859 – 11.02.1900@\textsc{Bératon, Ferry} (06.12.1859 – 11.02.1900), \emph{Schriftsteller/Schriftstellerin, Journalist/Journalistin, Bildender Künstler/Bildende Künstlerin >> Maler/Malerin}|pw}{ }ſprach dieſer Tage mit mir über die materielle
               Seite des \label{K_L00082_1v}\edtext{\textsc{Maeterlinck}\pwindex{Maeterlinck, Maurice 29.08.1862 – 06.05.1949@\textsc{Maeterlinck, Maurice} (29.08.1862 – 06.05.1949), \emph{Schriftsteller}|pw}-Abends}{\lemma{\textnormal{\emph{Maeterlinck-Abends}}}\Cendnote{\textnormal{am 2. 5. 1892}}}\label{K_L00082_1h}. Vorläufig habe ich ihm
               zehn Gulden geſchickt. Ueber dieſen Abend wäre manches {\pb}zu
               ſprechen.\pend
           \pstart
           Möchten Sie mir die Adreſſe von \textsc{Schwarzkopf}\pwindex{Schwarzkopf, Gustav 07.11.1853 – 13.11.1939@\textsc{Schwarzkopf, Gustav} (07.11.1853 – 13.11.1939), \emph{Schriftsteller}|pw} mittheilen? Ich möchte ihn um eine Empfehlung an \textsc{Bonz}\orgindex{Adolf Bonz und Comp.@Adolf Bonz {\kaufmannsund}  Comp.|pw} wegen meines \textsc{Anatol-Cyclus}\pwindex{Schnitzler, Arthur 15.05.1862 – 21.10.1931@\textsc{Schnitzler, Arthur} (15.05.1862 – 21.10.1931), \emph{Schriftsteller, Mediziner}!Anatol1892-10-29 – 1892-10-29@\strich\emph{Anatol} {[}1892-10-29 – 1892-10-29{]}|pw} erſuchen. Was glauben Sie? –\pend
           \pstart
           Herzlichſt der Ihre{\\[\baselineskip]}\spacefill\mbox{Arth Sch}\pend
           \leftskip=0em{}\pstart
           16. März 92\hfill Wien\oindex{Wien@\textbf{Wien}|pw}.\pend
           \endnumbering\briefempfaengerindex{Hofmannsthal, Hugo von@\textsc{Hofmannsthal, Hugo von}!zzzSchnitzler, Arthur@\emph{von Arthur Schnitzler}!1892-03-161@{16. 3. 1892}|)be}\mylabel{h}\end{ledgroupsized}  \newcommand{\dateiname}{L00082}\newcommand{\titel}{Arthur Schnitzler an Hugo von Hofmannsthal, 16. 3. 1892}\newcommand{\editorInnen}{Martin Anton Müller und Gerd-Hermann Susen}
            \footnotesize
\begin{ledgroupsized}[t]{11.5cm}
\doendnotes{C}
\end{ledgroupsized}
         %% latex-leseansicht-abspann.tex
%% Abspann für die Leseansicht.
%% Der Schalter \ifkorrekturansicht ist bereits durch den Vorspann gesetzt.

%% latex-abspann.tex
%% Gemeinsamer Abspann für Korrekturansicht und Leseansicht.
%% Setzt den Schalter \ifkorrekturansicht voraus (gesetzt in den
%% einbindenden Dateien latex-korrekturansicht-abspann.tex bzw.
%% latex-leseansicht-abspann.tex).
%% ---------------------------------------------------------------

\normalsize

% Das esempio-Environment wird nur in der Leseansicht benötigt
\ifkorrekturansicht\else
\newenvironment{esempio}[3]%
{
    \vspace{1.5ex}
    \rlap{\underline{#1}}
    \par
    \setlength{\parindent}{0cm}
    \nopagebreak
    \leftskip=#2cm
    \rightskip=#3cm
}
{
    \par
}
\fi

\doendnotes{C}
\bigskip
\vfill

\clearpage

\footnotesize

\ifkorrekturansicht
  \lohead{\textsc{register}}
\fi

% theindex-Environment neu definieren ohne reledmac
\makeatletter
\renewenvironment{theindex}{%
  \ifkorrekturansicht
    \section*{\indexname}%
  \else
    \subsubsection*{Index der erwähnten Entitäten}%
  \fi
  \setlength{\parindent}{0pt}%
  \setlength{\parskip}{0pt plus 0.3pt}%
  \let\item\@idxitem
}{%
  \ifkorrekturansicht\clearpage\fi
}
\makeatother

\IfFileExists{\jobname-pw.ind}{\input{\jobname-pw.ind}}{}

% Quellenangabe nur in der Leseansicht
\ifkorrekturansicht\else
% Fallback-Definitionen, falls die .tex-Datei \titel etc. nicht gesetzt hat
\providecommand{\titel}{}
\providecommand{\editorInnen}{}
\providecommand{\dateiname}{\jobname}

\vspace{3cm}

\vfill

\footnotesize
\textsc{Quelle}: \titel. Herausgegeben von {\editorInnen}. In: \emph{Arthur Schnitzler: Briefwechsel mit Autorinnen und Autoren}.
 Digitale Edition, https://schnitzler-briefe.acdh.oeaw.ac.at/{\dateiname}.html (Stand \today)
\fi

\end{document}


      