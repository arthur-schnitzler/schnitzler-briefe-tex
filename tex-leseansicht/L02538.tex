%% latex-korrekturansicht-vorspann.tex
%% Vorspann für die Korrekturansicht.
%% Lädt die gemeinsame Datei latex-vorspann.tex mit gesetztem Schalter.

\newif\ifkorrekturansicht
\korrekturansichttrue

\input{../tex-inputs/latex-vorspann}


\section[Robert Adam an Arthur Schnitzler, 15. 6. 1930]{L02538 Robert Adam an Arthur Schnitzler, 15. 6. 1930}
\nopagebreak\mylabel{L02538v}
\rehead{ }\normalsize\beginnumbering\briefempfaengerindex{Schnitzler, Arthur@\textsc{Schnitzler, Arthur}!zzzAdam, Robert@\emph{von Robert Adam}!1930-06-151@{15. 6. 1930}|(be}
\toendnotes[C]{\smallbreak\pagebreak[2]}\Standort{CUL, Schnitzler, B 1.}
\physDesc{Brief, 1 Blatt, 2 Seiten, 1305 Zeichen
\newline{}Handschrift: schwarze Tinte, deutsche Kurrent
\newline{}Schnitzler: mit rotem Buntstift vereinzelte Unterstreichungen 
\newline{}Ordnung: von unbekannter Hand nummeriert: »25« }\Standort{Wien, Österreichische Nationalbibliothek, Cod.ser. 52.269, 185 recto.}
\physDesc{handschriftliche Abschrift1 Blatt, 1 Seite, 1305 Zeichen
\newline{}Handschrift: schwarze Tinte, Gabelsberger Kurzschrift}\Standort{Wien, Österreichische Nationalbibliothek, Cod.ser. 52.269, 155 verso.}
\physDesc{maschinenschriftliche Abschrift1 Blatt, 1 Seite, 1305 Zeichen
\newline{}Schreibmaschine}\toendnotes[C]{\smallbreak}
\pstart
           \raggedleft{}{\pb}Wien\oindex{Wien@\textbf{Wien}, \emph{A.ADM2}|pw}, am 15. Juni 1930\pend
           
\pstart{}Hochverehrter Herr Doktor!\pend\vspace{0.5em}
\pstart
           Nehmen Sie meinen beſten Dank für Ihre freundlichen Glückwünſche in beiden
               »Belangen«!\pend
           
\pstart
           Der literariſche Vor-Erfolg – ich weiß recht gut, daß von ihm zum Enderfolg noch ein
               weiter unſicherer Weg zurückzulegen iſt – hat mich eigentlich weit mehr erfreut als
               die \label{K_L02538-1v}\edtext{Ernennung}{\lemma{\textnormal{\emph{Ernennung}}}\Cendnote{\textnormal{Adam\pwindex{Adam, Robert 20.04.1877 – 16.10.1961@\textsc{Adam, Robert} (20.04.1877 – 16.10.1961), \emph{Schriftsteller/Schriftstellerin, Richter/Richterin}|pwk} war zum Vizepräsidenten des \emph{Handelsgerichts}\orgindex{Handelsgericht Wien@Handelsgericht Wien|pwk}{ }Wien\oindex{Wien@\textbf{Wien}, \emph{A.ADM2}|pwk} ernannt worden.}}}\label{K_L02538-1}; denn die mußte ja doch, trotz
               angeſtammter Hinderniſſe, einmal erfolgen, während ich, nach vergeblichen zähen
               Kämpfen, deren Zeitlänge Ihnen bekannt iſt, ſchon jede Hoffnung aufgegeben hatte, mit
               einer meiner Komödien an’s Rampenlicht zu kommen. Daß das vom Schickſal hiezu
               beſtimmte Stück\pwindex{Margot und das Jugendgericht@\emph{Margot und das Jugendgericht}|pwv} kein
               künſtlerisch-beſſeres iſt, muß ich achſelzuckend hinnehmen. Außerdem hat es durch die
               mir vom Berlin\oindex{Berlin@\textbf{Berlin}, \emph{P.PPLC}|pw}er Verlag\orgindex{Drei Masken-Verlag@Drei Masken-Verlag|pwv} abgeforderte Umarbeitung – ich habe einen neuen
               letzten Akt verfaßt – an geiſtigem Inhalt noch eingebüßt, mag es auch bühnenwirkſamer
               geworden ſein.\pend
           
\pstart
           Gern ſchriebe ich eine oder die andere Komödie {\pb}nieder, die mir in freien Minuten durch
               den Kopf geht: aber ich bin von Amtsarbeit derart erdrückt, daß mir die Zeit wie die
               Konzentrationsmöglichkeit vollkommen mangeln.\pend
           
\pstart
           Ich würde Sie, hochverehrter Herr Doktor, außerordentlich gern einmal aufſuchen und
               würde mir zu jeder Stunde, die Ihnen genehm wäre, die Arbeit abſchütteln.\pend
           
\pstart
           Mit bestem Dank und vielen Grüßen verbleibe ich Ihr ergebener\pend
           \pstart \spacefill\mbox{D\textsuperscript{r}RAdam}\pend{}\selectlanguage{ngerman}\endnumbering\briefempfaengerindex{Schnitzler, Arthur@\textsc{Schnitzler, Arthur}!zzzAdam, Robert@\emph{von Robert Adam}!1930-06-151@{15. 6. 1930}|)be}\mylabel{L02538h}  \normalsize

\doendnotes{C}
\bigskip
\vfill

\clearpage

\footnotesize

\lohead{\textsc{register}}

% Definiere theindex-Environment komplett neu ohne reledmac
\makeatletter
\renewenvironment{theindex}{%
  \section*{\indexname}%
  \setlength{\parindent}{0pt}%
  \setlength{\parskip}{0pt plus 0.3pt}%
  \let\item\@idxitem
}{%
  \clearpage
}
\makeatother

\IfFileExists{\jobname-pw.ind}{\input{\jobname-pw.ind}}{}

\end{document}

      