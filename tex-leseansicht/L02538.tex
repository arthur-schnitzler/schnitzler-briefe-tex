%% latex-leseansicht-vorspann.tex
%% Vorspann für die Leseansicht.
%% Lädt die gemeinsame Datei latex-vorspann.tex mit nicht gesetztem Schalter.

\newif\ifkorrekturansicht
\korrekturansichtfalse

\input{../tex-inputs/latex-vorspann}


\section[Robert Adam an Arthur Schnitzler, 15. 6. 1930]{L02538 Robert Adam an Arthur Schnitzler, 15. 6. 1930}
\nopagebreak\mylabel{L02538v}
\rehead{ }\normalsize\beginnumbering\briefempfaengerindex{Schnitzler, Arthur@\textsc{Schnitzler, Arthur}!zzzAdam, Robert@\emph{von Robert Adam}!1930-06-151@{15. 6. 1930}|(be}
\toendnotes[C]{\smallbreak\pagebreak[2]}
\correspDesc{Versand  durch Robert Adam am 15. 6. 1930 in Wien
\newline{}Erhalt  durch Arthur Schnitzler im Zeitraum [15. 6. 1930
                  – 19. 6. 1930?] in Wien}\toendnotes[C]{\smallbreak}
\Standort{CUL, Schnitzler, B 1.}
\physDesc{Brief, 1 Blatt, 2 Seiten, 1305 Zeichen
\newline{}Handschrift: schwarze Tinte, deutsche Kurrent
\newline{}Schnitzler: mit rotem Buntstift vereinzelte Unterstreichungen 
\newline{}Ordnung: von unbekannter Hand nummeriert: »25« }\Standort{Wien, Österreichische Nationalbibliothek, Cod.ser. 52.269, 185 recto.}
\physDesc{handschriftliche Abschrift. 1 Blatt, 1 Seite, 1305 Zeichen
\newline{}Handschrift: schwarze Tinte, Gabelsberger Kurzschrift}\Standort{Wien, Österreichische Nationalbibliothek, Cod.ser. 52.269, 155 verso.}
\physDesc{maschinenschriftliche Abschrift, 1 Blatt, 1 Seite, 1305 Zeichen
\newline{}Schreibmaschine}\toendnotes[C]{\smallbreak}
\pstart
           \raggedleft{}{\pb}Wien\oindex{Wien@\textbf{Wien}, \emph{Verwaltungsgebiet}|pw}, am 15. Juni 1930\pend
           
\pstart{}Hochverehrter Herr Doktor!\pend\vspace{0.5em}
\pstart
           Nehmen Sie meinen beſten Dank für Ihre freundlichen Glückwünſche in beiden
               »Belangen«!\pend
           
\pstart
           Der literariſche Vor-Erfolg – ich weiß recht gut, daß von ihm zum Enderfolg noch ein
               weiter unſicherer Weg zurückzulegen iſt – hat mich eigentlich weit mehr erfreut als
               die \label{K_L02538-1v}\edtext{Ernennung}{\lemma{\textnormal{\emph{Ernennung}}}\Cendnote{\textnormal{Adam\pwindex{Adam, Robert 20.\,4.\,1877 Wien – 16.\,10.\,1961 Baden bei Wien@\textsc{Adam, Robert} (20.\,4.\,1877 Wien – 16.\,10.\,1961 Baden bei Wien), \emph{Schriftsteller, Richter}|pwk} war zum Vizepräsidenten des \emph{Handelsgerichts}\orgindex{Handelsgericht Wien@Handelsgericht Wien|pwk}{ }Wien\oindex{Wien@\textbf{Wien}, \emph{Verwaltungsgebiet}|pwk} ernannt worden.}}}\label{K_L02538-1}; denn die mußte ja doch, trotz
               angeſtammter Hinderniſſe, einmal erfolgen, während ich, nach vergeblichen zähen
               Kämpfen, deren Zeitlänge Ihnen bekannt iſt,{ }ſchon jede Hoffnung aufgegeben hatte, mit
               einer meiner Komödien an’s Rampenlicht zu kommen. Daß das vom Schickſal hiezu
               beſtimmte Stück\pwindex{Adam, Robert 20.\,4.\,1877 Wien – 16.\,10.\,1961 Baden bei Wien@\textsc{Adam, Robert} (20.\,4.\,1877 Wien – 16.\,10.\,1961 Baden bei Wien), \emph{Schriftsteller, Richter}!Margot und das Jugendgericht@\strich\emph{Margot und das Jugendgericht}|pwv} kein
               künſtlerisch-beſſeres iſt, muß ich achſelzuckend hinnehmen. Außerdem hat es durch die
               mir vom Berlin\oindex{Berlin@\textbf{Berlin}, \emph{Hauptstadt}|pw}er Verlag\orgindex{Drei Masken-Verlag@Drei Masken-Verlag|pwv} abgeforderte Umarbeitung – ich habe einen neuen
               letzten Akt verfaßt – an geiſtigem Inhalt noch eingebüßt, mag es auch bühnenwirkſamer
               geworden{ }ſein.\pend
           
\pstart
           Gern{ }ſchriebe ich eine oder die andere Komödie {\pb}nieder, die mir in freien Minuten durch
               den Kopf geht: aber ich bin von Amtsarbeit derart erdrückt, daß mir die Zeit wie die
               Konzentrationsmöglichkeit vollkommen mangeln.\pend
           
\pstart
           Ich würde Sie, hochverehrter Herr Doktor, außerordentlich gern einmal aufſuchen und
               würde mir zu jeder Stunde, die Ihnen genehm wäre, die Arbeit abſchütteln.\pend
           
\pstart
           Mit bestem Dank und vielen Grüßen verbleibe ich Ihr ergebener\pend
           \pstart \spacefill\mbox{D\textsuperscript{r}RAdam}\pend{}\selectlanguage{ngerman}\endnumbering\briefempfaengerindex{Schnitzler, Arthur@\textsc{Schnitzler, Arthur}!zzzAdam, Robert@\emph{von Robert Adam}!1930-06-151@{15. 6. 1930}|)be}\mylabel{L02538h}  \newcommand{\dateiname}{L02538}\newcommand{\titel}{Robert Adam an Arthur Schnitzler, 15. 6. 1930}\newcommand{\editorInnen}{Martin Anton Müller und Gerd-Hermann Susen}%% latex-leseansicht-abspann.tex
%% Abspann für die Leseansicht.
%% Der Schalter \ifkorrekturansicht ist bereits durch den Vorspann gesetzt.

%% latex-abspann.tex
%% Gemeinsamer Abspann für Korrekturansicht und Leseansicht.
%% Setzt den Schalter \ifkorrekturansicht voraus (gesetzt in den
%% einbindenden Dateien latex-korrekturansicht-abspann.tex bzw.
%% latex-leseansicht-abspann.tex).
%% ---------------------------------------------------------------

\normalsize

% Das esempio-Environment wird nur in der Leseansicht benötigt
\ifkorrekturansicht\else
\newenvironment{esempio}[3]%
{
    \vspace{1.5ex}
    \rlap{\underline{#1}}
    \par
    \setlength{\parindent}{0cm}
    \nopagebreak
    \leftskip=#2cm
    \rightskip=#3cm
}
{
    \par
}
\fi

\doendnotes{C}
\bigskip
\vfill

\clearpage

\footnotesize

\ifkorrekturansicht
  \lohead{\textsc{register}}
\fi

% theindex-Environment neu definieren ohne reledmac
\makeatletter
\renewenvironment{theindex}{%
  \ifkorrekturansicht
    \section*{\indexname}%
  \else
    \subsubsection*{Index der erwähnten Entitäten}%
  \fi
  \setlength{\parindent}{0pt}%
  \setlength{\parskip}{0pt plus 0.3pt}%
  \let\item\@idxitem
}{%
  \ifkorrekturansicht\clearpage\fi
}
\makeatother

\IfFileExists{\jobname-pw.ind}{\input{\jobname-pw.ind}}{}

% Quellenangabe nur in der Leseansicht
\ifkorrekturansicht\else
% Fallback-Definitionen, falls die .tex-Datei \titel etc. nicht gesetzt hat
\providecommand{\titel}{}
\providecommand{\editorInnen}{}
\providecommand{\dateiname}{\jobname}

\vspace{3cm}

\vfill

\footnotesize
\textsc{Quelle}: \titel. Herausgegeben von {\editorInnen}. In: \emph{Arthur Schnitzler: Briefwechsel mit Autorinnen und Autoren}.
 Digitale Edition, https://schnitzler-briefe.acdh.oeaw.ac.at/{\dateiname}.html (Stand \today)
\fi

\end{document}


