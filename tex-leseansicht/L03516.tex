%% latex-leseansicht-vorspann.tex
%% Vorspann für die Leseansicht.
%% Lädt die gemeinsame Datei latex-vorspann.tex mit nicht gesetztem Schalter.

\newif\ifkorrekturansicht
\korrekturansichtfalse

\input{../tex-inputs/latex-vorspann}

\begin{center}
            \textcolor{red}{ENTWURF, NICHT FERTIG KORRIGIERT}
                      \end{center}
            
         
         \renewcommand{\erwaehntePersonen}{Personen: Paul Goldmann, Eva Marie Goldmann, Franziska Goldmann}
         \renewcommand{\erwaehnteInstitutionen}{Institutionen: Neue Freie Presse, S. Fischer Verlag}
         \renewcommand{\erwaehnteOrte}{Orte: Bendlerstraße, Berlin, Wien}
         \renewcommand{\erwaehnteWerke}{Werke: Therese. Chronik eines Frauenlebens}
               \section[ Paul Goldmann an Arthur Schnitzler, 3. 5. 1928]{ Paul Goldmann an Arthur Schnitzler, 3. 5. 1928}\nopagebreak\mylabel{v}\rehead{ }\begin{ledgroupsized}[t]{13cm}\normalsize\beginnumbering\briefempfaengerindex{Schnitzler, Arthur@\textsc{Schnitzler, Arthur}!zzzGoldmann, Paul@\emph{von Paul Goldmann}!1928-05-031@{3. 5. 1928}|(be} \toendnotes[C]{\smallbreak\pagebreak[2]} \Standort{DLA, A:Schnitzler, HS.NZ85.1.3176.}
\physDesc{Brief, 1 Blatt, 1 Seite, 489 Zeichen
\newline{}Schreibmaschine
\newline{}Handschrift: lila Tinte, lateinische Kurrent (\noindent{}eine Korrektur und Unterschrift)
\newline{}Schnitzler: mit rotem Buntstift »Theres{[}e{]}\pwindex{Schnitzler, Arthur 15.05.1862 – 21.10.1931@\textsc{Schnitzler, Arthur} (15.05.1862 – 21.10.1931), \emph{Schriftsteller, Mediziner}!Therese. Chronik eines Frauenlebens1928-03-27@\strich\emph{Therese. Chronik eines Frauenlebens} {[}1928-03-27{]}|pw}« vermerkt und eine Unterstreichung }\toendnotes[C]{\smallbreak}\pstart
           \noindent{}{\pb}\textcolor{gray}{\textbf{Dr. Paul Goldmann}}\hfill \textcolor{gray}{\textbf{Berlin W. 10\oindex{Berlin@\textbf{Berlin}|pw}}}\pend
           \pstart
           \textcolor{gray}{\textbf{Vertreter der »Neuen Freien
                           Presse\orgindex{Neue Freie Presse@Neue Freie Presse|pw}«}}\hfill \textcolor{gray}{\textbf{Bendlerſtraße 36\oindex{Bendlerstrasse@\textbf{Bendlerstraße}|pw}.}}\pend
           \pstart
           \raggedleft{}\textcolor{gray}{\textbf{Tel. Lützow 9142}}\pend
           \pstart
           \raggedleft{}3. 5. 28.\pend
           \pstart\center{}Lieber Freund,\pend\pstart
           Für die Übersendung Deines neuen \label{K_L03516-1v}\edtext{Roman\pwindex{Schnitzler, Arthur 15.05.1862 – 21.10.1931@\textsc{Schnitzler, Arthur} (15.05.1862 – 21.10.1931), \emph{Schriftsteller, Mediziner}!Therese. Chronik eines Frauenlebens1928-03-27@\strich\emph{Therese. Chronik eines Frauenlebens} {[}1928-03-27{]}|pwv}}{\lemma{\textnormal{\emph{Roman}}}\Cendnote{\textnormal{Schnitzler\pwindex{Schnitzler, Arthur 15.05.1862 – 21.10.1931@\textsc{Schnitzler, Arthur} (15.05.1862 – 21.10.1931), \emph{Schriftsteller, Mediziner}|pwk} Roman \emph{Therese. Chronik eines Frauenlebens}\pwindex{Schnitzler, Arthur 15.05.1862 – 21.10.1931@\textsc{Schnitzler, Arthur} (15.05.1862 – 21.10.1931), \emph{Schriftsteller, Mediziner}!Therese. Chronik eines Frauenlebens1928-03-27@\strich\emph{Therese. Chronik eines Frauenlebens} {[}1928-03-27{]}|pwk} war am 27. 3. 1928 im Berlin\oindex{Berlin@\textbf{Berlin}|pwk}er \emph{S. Fischer-Verlag}\orgindex{S. Fischer Verlag@S. Fischer Verlag|pwk}
                  erschienen.}}}\label{K_L03516-1h}s sagen wir alle Dir unseren herzlichsten Dank. Er geht
               gegenwärtig in meinem Haushalt von Hand zu Hand und findet den Beifall von Jung und
               Alt. Wenn Frau\pwindex{Goldmann, Eva Marie 27.10.1877 – 02.11.1937@\textsc{Goldmann, Eva Marie} (27.10.1877 – 02.11.1937)|pwv} und Tochter\pwindex{Goldmann, Franziska 1911-05-29 – 1963-08-19@\textsc{Goldmann, Franziska} (1911-05-29 – 1963-08-19), \emph{Schauspielerin}|pwv} fertig sind, darf ich
               dann das Buch\pwindex{Schnitzler, Arthur 15.05.1862 – 21.10.1931@\textsc{Schnitzler, Arthur} (15.05.1862 – 21.10.1931), \emph{Schriftsteller, Mediziner}!Therese. Chronik eines Frauenlebens1928-03-27@\strich\emph{Therese. Chronik eines Frauenlebens} {[}1928-03-27{]}|pwv} auch lesen.
               Darum kann ich einstweilen nur für die Übersendung danken.\pend
           \pstart
           Ich ho\substVorne{}\textsuperscript{gf}\substDazwischen{}ff\substHinten{}e, dass es Dir gut geht, und dass wir bald wieder einmal die Freude haben
               werden, Dich in \label{K_L03516-2v}\edtext{Berlin\oindex{Berlin@\textbf{Berlin}|pw}}{\lemma{\textnormal{\emph{Berlin}}}\Cendnote{\textnormal{In Berlin\oindex{Berlin@\textbf{Berlin}|pwk} sahen sich Goldmann\pwindex{Goldmann, Paul 31.01.1865 – 25.09.1935@\textsc{Goldmann, Paul} (31.01.1865 – 25.09.1935), \emph{Schriftsteller, Journalist}|pwk} und Schnitzler\pwindex{Schnitzler, Arthur 15.05.1862 – 21.10.1931@\textsc{Schnitzler, Arthur} (15.05.1862 – 21.10.1931), \emph{Schriftsteller, Mediziner}|pwk} erst am 11. 11. 1930 und 16. 11. 1930 wieder. Am
                     16. 5. 1930 hatte
                     Goldmann\pwindex{Goldmann, Paul 31.01.1865 – 25.09.1935@\textsc{Goldmann, Paul} (31.01.1865 – 25.09.1935), \emph{Schriftsteller, Journalist}|pwk}{ }Schnitzler\pwindex{Schnitzler, Arthur 15.05.1862 – 21.10.1931@\textsc{Schnitzler, Arthur} (15.05.1862 – 21.10.1931), \emph{Schriftsteller, Mediziner}|pwk} noch vorgeworfen, ihn nicht in
                     Berlin\oindex{Berlin@\textbf{Berlin}|pwk} zu besuchen.}}}\label{K_L03516-2h} zu sehen.\pend
           \pstart
           Alles Herzliche von uns Allen! {\\[\baselineskip]}{[}hs.:{]} Dein {\\[\baselineskip]}\spacefill\mbox{Paul Goldmann.}\pend
           \leftskip=0em{}
         
         \endnumbering\mylabel{h}\end{ledgroupsized}\begin{anhang}\end{anhang}\newcommand{\dateiname}{L03516}\newcommand{\titel}{Paul Goldmann an Arthur Schnitzler, 3. 5. 1928}\newcommand{\editorInnen}{Martin Anton Müller und Laura Untner}%% latex-leseansicht-abspann.tex
%% Abspann für die Leseansicht.
%% Der Schalter \ifkorrekturansicht ist bereits durch den Vorspann gesetzt.

%% latex-abspann.tex
%% Gemeinsamer Abspann für Korrekturansicht und Leseansicht.
%% Setzt den Schalter \ifkorrekturansicht voraus (gesetzt in den
%% einbindenden Dateien latex-korrekturansicht-abspann.tex bzw.
%% latex-leseansicht-abspann.tex).
%% ---------------------------------------------------------------

\normalsize

% Das esempio-Environment wird nur in der Leseansicht benötigt
\ifkorrekturansicht\else
\newenvironment{esempio}[3]%
{
    \vspace{1.5ex}
    \rlap{\underline{#1}}
    \par
    \setlength{\parindent}{0cm}
    \nopagebreak
    \leftskip=#2cm
    \rightskip=#3cm
}
{
    \par
}
\fi

\doendnotes{C}
\bigskip
\vfill

\clearpage

\footnotesize

\ifkorrekturansicht
  \lohead{\textsc{register}}
\fi

% theindex-Environment neu definieren ohne reledmac
\makeatletter
\renewenvironment{theindex}{%
  \ifkorrekturansicht
    \section*{\indexname}%
  \else
    \subsubsection*{Index der erwähnten Entitäten}%
  \fi
  \setlength{\parindent}{0pt}%
  \setlength{\parskip}{0pt plus 0.3pt}%
  \let\item\@idxitem
}{%
  \ifkorrekturansicht\clearpage\fi
}
\makeatother

\IfFileExists{\jobname-pw.ind}{\input{\jobname-pw.ind}}{}

% Quellenangabe nur in der Leseansicht
\ifkorrekturansicht\else
% Fallback-Definitionen, falls die .tex-Datei \titel etc. nicht gesetzt hat
\providecommand{\titel}{}
\providecommand{\editorInnen}{}
\providecommand{\dateiname}{\jobname}

\vspace{3cm}

\vfill

\footnotesize
\textsc{Quelle}: \titel. Herausgegeben von {\editorInnen}. In: \emph{Arthur Schnitzler: Briefwechsel mit Autorinnen und Autoren}.
 Digitale Edition, https://schnitzler-briefe.acdh.oeaw.ac.at/{\dateiname}.html (Stand \today)
\fi

\end{document}


      