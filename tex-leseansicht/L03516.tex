%% latex-korrekturansicht-vorspann.tex
%% Vorspann für die Korrekturansicht.
%% Lädt die gemeinsame Datei latex-vorspann.tex mit gesetztem Schalter.

\newif\ifkorrekturansicht
\korrekturansichttrue

\input{../tex-inputs/latex-vorspann}


\section[ Paul Goldmann an Arthur Schnitzler, 3. 5. 1928]{L03516 Paul Goldmann an Arthur Schnitzler, 3. 5. 1928}
\nopagebreak\mylabel{L03516v}
\rehead{ }\normalsize\beginnumbering\briefempfaengerindex{Schnitzler, Arthur@\textsc{Schnitzler, Arthur}!zzzGoldmann, Paul@\emph{von Paul Goldmann}!1928-05-031@{3. 5. 1928}|(be}
\toendnotes[C]{\smallbreak\pagebreak[2]}\Standort{DLA, A:Schnitzler, HS.NZ85.1.3176.}
\physDesc{Brief, 1 Blatt, 1 Seite, 489 Zeichen
\newline{}Schreibmaschine
\newline{}Handschrift: lila Tinte, lateinische Kurrent (\noindent{}eine Korrektur und Unterschrift)
\newline{}Schnitzler: mit rotem Buntstift »Theres{[}e{]}\pwindex{Therese. Chronik eines Frauenlebens@\emph{Therese. Chronik eines Frauenlebens}|pw}« vermerkt und eine Unterstreichung }\toendnotes[C]{\smallbreak}
\pstart
           {\pb}\textcolor{gray}{\textbf{Dr. Paul Goldmann}}\hfill \textcolor{gray}{\textbf{Berlin W. 10\oindex{Berlin@\textbf{Berlin}, \emph{P.PPLC}|pw}}}\pend
           
\pstart
           \textcolor{gray}{\textbf{Vertreter der »Neuen Freien
                           Presse\orgindex{Neue Freie Presse@Neue Freie Presse|pw}«}}\hfill \textcolor{gray}{\textbf{Bendlerſtraße 36\oindex{Stauffenbergstrasse@\textbf{Stauffenbergstraße}, \emph{Straße (K.STR)}|pw}.}}\pend
           
\pstart
           \raggedleft{}\textcolor{gray}{\textbf{Tel.: Lützow 9142}}\pend
           
\pstart
           \raggedleft{}3. 5. 28.\pend
           
\pstart\center{}Lieber Freund,\pend\vspace{0.5em}
\pstart
           Für die Übersendung Deines neuen \label{K_L03516-1v}\edtext{Romans\pwindex{Therese. Chronik eines Frauenlebens@\emph{Therese. Chronik eines Frauenlebens}|pwv}}{\lemma{\textnormal{\emph{Romans}}}\Cendnote{\textnormal{Schnitzlers Roman \emph{Therese. Chronik eines Frauenlebens}\pwindex{Therese. Chronik eines Frauenlebens@\emph{Therese. Chronik eines Frauenlebens}|pwk} war am 27. 3. 1928 im Berlin\oindex{Berlin@\textbf{Berlin}, \emph{P.PPLC}|pwk}er \emph{S. Fischer Verlag}\orgindex{S. Fischer Verlag@S. Fischer Verlag|pwk}
                  erschienen.}}}\label{K_L03516-1} sagen wir alle Dir unseren herzlichsten Dank. Er geht
               gegenwärtig in meinem Haushalt von Hand zu Hand und findet den Beifall von Jung und
               Alt. Wenn Frau\pwindex{Goldmann, Eva Marie 27.10.1877 – 02.11.1937@\textsc{Goldmann, Eva Marie} (27.10.1877 – 02.11.1937)|pwv} und Tochter\pwindex{Goldmann, Franziska 1911-05-29 – 1963-08-19@\textsc{Goldmann, Franziska} (1911-05-29 – 1963-08-19), \emph{Schauspieler/Schauspielerin}|pwv} fertig sind, darf ich
               dann das Buch\pwindex{Therese. Chronik eines Frauenlebens@\emph{Therese. Chronik eines Frauenlebens}|pwv} auch lesen.
               Darum kann ich einstweilen nur für die Übersendung danken.\pend
           
\pstart
           Ich ho\substVorne{}\textsuperscript{ff}\substDazwischen{}ff\substHinten{}e, dass es Dir gut geht, und dass wir bald wieder einmal die Freude haben
               werden, Dich in \label{K_L03516-2v}\edtext{Berlin\oindex{Berlin@\textbf{Berlin}, \emph{P.PPLC}|pw}}{\lemma{\textnormal{\emph{Berlin}}}\Cendnote{\textnormal{In Berlin\oindex{Berlin@\textbf{Berlin}, \emph{P.PPLC}|pwk} sahen sich Goldmann\pwindex{Goldmann, Paul 31.01.1865 – 25.09.1935@\textsc{Goldmann, Paul} (31.01.1865 – 25.09.1935), \emph{Schriftsteller/Schriftstellerin, Journalist/Journalistin}|pwk} und Schnitzler erst am 11. 11. 1930 und am
                     16. 11. 1930
                  wieder. Am 16. 5. 1930 hatte Goldmann\pwindex{Goldmann, Paul 31.01.1865 – 25.09.1935@\textsc{Goldmann, Paul} (31.01.1865 – 25.09.1935), \emph{Schriftsteller/Schriftstellerin, Journalist/Journalistin}|pwk}{ }Schnitzler noch vorgeworfen, ihn nicht in
                     Berlin\oindex{Berlin@\textbf{Berlin}, \emph{P.PPLC}|pwk} zu besuchen.}}}\label{K_L03516-2} zu sehen.\pend
           
\pstart
           Alles Herzliche von uns Allen! {\\[\baselineskip]}{[}hs.:{]} Dein {\\[\baselineskip]}\spacefill\mbox{Paul Goldmann.}\pend
           \leftskip=0em{}\selectlanguage{ngerman}\endnumbering\briefempfaengerindex{Schnitzler, Arthur@\textsc{Schnitzler, Arthur}!zzzGoldmann, Paul@\emph{von Paul Goldmann}!1928-05-031@{3. 5. 1928}|)be}\mylabel{L03516h}  \normalsize

\doendnotes{C}
\bigskip
\vfill

\clearpage

\footnotesize

\lohead{\textsc{register}}

% Definiere theindex-Environment komplett neu ohne reledmac
\makeatletter
\renewenvironment{theindex}{%
  \section*{\indexname}%
  \setlength{\parindent}{0pt}%
  \setlength{\parskip}{0pt plus 0.3pt}%
  \let\item\@idxitem
}{%
  \clearpage
}
\makeatother

\IfFileExists{\jobname-pw.ind}{\input{\jobname-pw.ind}}{}

\end{document}

      