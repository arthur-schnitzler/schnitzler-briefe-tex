%% latex-korrekturansicht-vorspann.tex
%% Vorspann für die Korrekturansicht.
%% Lädt die gemeinsame Datei latex-vorspann.tex mit gesetztem Schalter.

\newif\ifkorrekturansicht
\korrekturansichttrue

\input{../tex-inputs/latex-vorspann}


\section[Hugo von Hofmannsthal an Arthur Schnitzler, 27. {[}5. 1915{]}]{L02206 Hugo von Hofmannsthal an Arthur Schnitzler, 27. {[}5. 1915{]}}
\nopagebreak\mylabel{L02206v}
\rehead{ }\normalsize\beginnumbering\briefempfaengerindex{Schnitzler, Arthur@\textsc{Schnitzler, Arthur}!zzzHofmannsthal, Hugo von@\emph{von Hugo von Hofmannsthal}!1915-05-271@{27. {[}5. 1915{]}}|(be}
\toendnotes[C]{\smallbreak\pagebreak[2]}\Standort{CUL, Schnitzler, B 43.}
\physDesc{Briefkarte, 571 Zeichen
\newline{}Handschrift: schwarze Tinte, deutsche Kurrent
\newline{}Schnitzler: mit Bleistift Monat und Jahreszahl ergänzt: »/5 915« 
\newline{}Ordnung: 1) mit Bleistift von unbekannter Hand nummeriert: »\strikeout{341}«  2) mit Bleistift von unbekannter Hand nummeriert:
                                    »353«}
\buchAbdrucke{\weitereDrucke{Hugo von Hofmannsthal, Arthur Schnitzler: \emph{Briefwechsel}. Frankfurt am Main: \emph{S. Fischer} 1964, S. 277.} }
\pstart
           \raggedleft{}{\pb}Rodaun\oindex{Rodaun@\textbf{Rodaun}, \emph{A.ADM4}|pw}, 27\textsuperscript{ten}{ }abends.\pend
           \vspace{0.5em}
\pstart
           mein lieber Arthur, wir mußten damals Olga\pwindex{Schnitzler, Olga 17.01.1882 – 13.01.1970@\textsc{Schnitzler, Olga} (17.01.1882 – 13.01.1970), \emph{Schauspieler/Schauspielerin, Sänger/Sängerin}|pw} abſagen, ſo leid es uns tat, weil ich annehmen muſste, das
               ich den darauffolgenden Tag würde nach Polen\oindex{Polen@\textbf{Polen}, \emph{A.PCLI}|pw}
               abzugehen haben. Indeſſen hat ſich dies von Woche zu Woche hinausgeſchoben und nun
               erſt gehe ich fort, übermorgen, zunächſt nach Teſchen, weiterhin in die beſetzten
                  Gebiete.\hspace*{1.5em}Ich rechne in 2–3 Wochen {\pb}wieder zurückzuſein. Ob ich dann,
               wie beabſichtigt war, nach Belgien\oindex{Belgien@\textbf{Belgien}, \emph{A.PCLI}|pw} zu gehen haben
               werde, oder vielleicht in irgendwelchem Auftrage zur italien\oindex{Italien@\textbf{Italien}, \emph{A.PCLI}|pw}iſchen Armee, wird ſich ergeben.\pend
           
\pstart
           Ich grüße Sie und Olga\pwindex{Schnitzler, Olga 17.01.1882 – 13.01.1970@\textsc{Schnitzler, Olga} (17.01.1882 – 13.01.1970), \emph{Schauspieler/Schauspielerin, Sänger/Sängerin}|pw} herzlich.{\\[\baselineskip]}Ihr{\\[\baselineskip]}\spacefill\mbox{Hugo.}\pend
           \leftskip=0em{}\selectlanguage{ngerman}\endnumbering\briefempfaengerindex{Schnitzler, Arthur@\textsc{Schnitzler, Arthur}!zzzHofmannsthal, Hugo von@\emph{von Hugo von Hofmannsthal}!1915-05-271@{27. {[}5. 1915{]}}|)be}\mylabel{L02206h}  \normalsize

\doendnotes{C}
\bigskip
\vfill

\clearpage

\footnotesize

\lohead{\textsc{register}}

% Definiere theindex-Environment komplett neu ohne reledmac
\makeatletter
\renewenvironment{theindex}{%
  \section*{\indexname}%
  \setlength{\parindent}{0pt}%
  \setlength{\parskip}{0pt plus 0.3pt}%
  \let\item\@idxitem
}{%
  \clearpage
}
\makeatother

\IfFileExists{\jobname-pw.ind}{\input{\jobname-pw.ind}}{}

\end{document}

      