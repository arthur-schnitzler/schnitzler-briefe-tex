%% latex-leseansicht-vorspann.tex
%% Vorspann für die Leseansicht.
%% Lädt die gemeinsame Datei latex-vorspann.tex mit nicht gesetztem Schalter.

\newif\ifkorrekturansicht
\korrekturansichtfalse

\input{../tex-inputs/latex-vorspann}


               \section[Richard Beer-Hofmann an Arthur Schnitzler, 28. 7. 1893]{ Richard Beer-Hofmann an Arthur Schnitzler, 28. 7. 1893}\nopagebreak\mylabel{v}\rehead{ }\begin{ledgroupsized}[t]{13cm}\normalsize\beginnumbering\briefempfaengerindex{Schnitzler, Arthur@\textsc{Schnitzler, Arthur}!zzzBeer-Hofmann, Richard@\emph{von Richard Beer-Hofmann}!1893-07-281@{28. 7. 1893}|(be} \toendnotes[C]{\smallbreak\pagebreak[2]} \Standort{CUL, Schnitzler, B 8.}
\physDesc{Brief, 1 Blatt, 4 Seiten
\newline{}Handschrift: Bleistift, deutsche Kurrent
\newline{}Schnitzler: mit Bleistift nummeriert: »22« }\buchAbdrucke{\weitereDrucke{Arthur Schnitzler, Richard Beer-Hofmann: \emph{Briefwechsel 1891–1931}. Hg. Konstanze Fliedl. Wien, Zürich: \emph{Europaverlag} 1992, S. 48–49.} }\toendnotes[C]{\smallbreak}\pstart
           \raggedleft{}{\pb}Freitag{ }Mittag. \pend
           \pstart
           Lieber Arthur! Bin wieder seit vorgestern nachts hier. Las Ihren
               Brief an Frau F.\pwindex{Flegmann, Bertha 27.05.1852 – 24.6.1933@\textsc{Flegmann, Bertha} (27.05.1852 – 24.6.1933), \emph{Salonnière}|pw}; das Telegramm\pwindex{?? Werk@Nicht ermittelte Verfasserinnen und Verfasser!Aus Ischl, 14. Juli, schreibt man uns: …18.07.1893 – 18.07.1893@\emph{Aus Ischl, 14. Juli, schreibt man uns: …} {[}18.07.1893 – 18.07.1893{]}|pwv} ist nicht von ihr; von Ben.\pwindex{Benedict, Markus 17.09.1834 – 26.2.1909@\textsc{Benedict, Markus} (17.09.1834 – 26.2.1909), \emph{Industrieller}|pw}?\pend
           \pstart
           Im Börsencourir\orgindex{Berliner Boersen-Courier@Berliner Börsen-Courier|pw} von \strikeout{ge} – ? – ich höre in dem, der vorgestern hier war, – ich hoffe ihn zu
               erhalten {[}–{]} soll eine lange günstige Notiz\pwindex{?? Werk@Nicht ermittelte Verfasserinnen und Verfasser!Man schreibt uns aus Ischl]25.07.1893 – 25.07.1893@\emph{[Man schreibt uns aus Ischl]} {[}25.07.1893 – 25.07.1893{]}|pwv} stehen.\pend
           \pstart
           {\pb}Ich habe Paul Horn\pwindex{Horn, Paul 13.02.1867 – 18.01.1936@\textsc{Horn, Paul} (13.02.1867 – 18.01.1936), \emph{Fabrikant}|pw} als er hier war sämtliche Daten gegeben; auch bez.
               Lektüre durch Reicher\pwindex{Reicher, Emanuel 18.06.1849 – 15.05.1924@\textsc{Reicher, Emanuel} (18.06.1849 – 15.05.1924), \emph{Schauspieler}|pw} u. Jarno\pwindex{Jarno, Josef 24.08.1865 – 11.01.1932@\textsc{Jarno, Josef} (24.08.1865 – 11.01.1932), \emph{Theaterleiter, Schauspieler}|pw} in Berlin\oindex{Berlin@\textbf{Berlin}|pw}; dürfte also
               darin stehen. Heute wieder \label{K_L00245_1v}\edtext{Mamroth\pwindex{Mamroth, Fedor 21.02.1851 – 25.06.1907@\textsc{Mamroth, Fedor} (21.02.1851 – 25.06.1907), \emph{Journalist, Kritiker}|pw} zitirt}{\lemma{\textnormal{\emph{Mamroth zitirt}}}\Cendnote{\textnormal{Fedor Mamroth an Arthur Schnitzler, 5. 3. 1893.}}}\label{K_L00245_1h} (Tolstoi\pwindex{Tolstoi, Leo N. von 9.09.1828 – 20.11.1910@\textsc{Tolstoi, Leo N. von} (9.09.1828 – 20.11.1910), \emph{Schriftsteller}|pw}) vor Frau
                  Kalbek\pwindex{Kalbeck, Julie 1852 – 17.6.1922@\textsc{Kalbeck, Julie} (1852 – 17.6.1922), \emph{Frauenrechtlerin}|pw}.\pend
           \pstart
           Ich glaube es wird gehen. Verhalten Sie sich nur gut mit F.\pwindex{Flegmann, Bertha 27.05.1852 – 24.6.1933@\textsc{Flegmann, Bertha} (27.05.1852 – 24.6.1933), \emph{Salonnière}|pw}; sie setzt sich {\pb}wirklich für ihre Freunde ein.
               Bitte \uline{urgiren} Sie den Abschreiber\pwindex{?? [Schreibkraft fuer Arthur Schnitzler] @\textsc{?? [Schreibkraft für Arthur Schnitzler]}|pwv}; mir ist sehr darum zu thun die Sache hier
               vorlesen zu können solange Kalbeks\pwindex{Kalbeck, Max 04.01.1850 – 04.05.1921@\textsc{Kalbeck, Max} (04.01.1850 – 04.05.1921), \emph{Journalist}|pw}\pwindex{Kalbeck, Julie 1852 – 17.6.1922@\textsc{Kalbeck, Julie} (1852 – 17.6.1922), \emph{Frauenrechtlerin}|pw} u. \substVorne{}\textsuperscript{I}\substDazwischen{}i\substHinten{}hre Schwester eine Frau Lion\pwindex{Lion @\textsc{Lion}|pw} da ist.
               Bitte!\pend
           \pstart
           Heute, Freitag Mittag, – ist noch nichts eingetroffen,
               hoffentlich kreuzt {\pb}es sich mit
               meinem Brief; der Schluss des Kindes\pwindex{Beer-Hofmann, Richard 11.07.1866 – 26.09.1945@\textsc{Beer-Hofmann, Richard} (11.07.1866 – 26.09.1945), \emph{Schriftsteller}!Kind1893@\strich\emph{Das Kind} {[}1893{]}|pw} ist endgiltig
               geändert, hoffentlich gefällt er jetzt besser.\pend
           \pstart
           Grüßen Sie Schwarzkopf\pwindex{Schwarzkopf, Gustav 07.11.1853 – 13.11.1939@\textsc{Schwarzkopf, Gustav} (07.11.1853 – 13.11.1939), \emph{Schriftsteller}|pw}{ }Salten\pwindex{Salten, Felix 06.09.1869 – 08.10.1945@\textsc{Salten, Felix} (06.09.1869 – 08.10.1945), \emph{Schriftsteller, Journalist}|pw}. Herzlichst Ihr\pend
           \pstart \spacefill\mbox{Richard}\pend{}\pstart
           Ischl\oindex{Bad Ischl@\textbf{Bad Ischl}|pw}. 28 Juli 93.\pend
           \pstart
           \noindent{}Was sagen Sie zu \strikeout{Schr}{ }Wengraf\pwindex{Wengraf, Edmund 09.01.1860 – 08.12.1933@\textsc{Wengraf, Edmund} (09.01.1860 – 08.12.1933), \emph{Journalist}|pw}{ }Hirschfeld\pwindex{Hirschfeld, Robert 17.09.1857 – 02.04.1914@\textsc{Hirschfeld, Robert} (17.09.1857 – 02.04.1914), \emph{Journalist, Musikkritiker}|pw}\pwindex{Zwei Freunde Burckhards24.07.1893 – 24.07.1893@\emph{Zwei Freunde Burckhards} {[}24.07.1893 – 24.07.1893{]}|pwv}?\pend
           \pstart
           Schreiben Sie Löbl\pwindex{Loebl, Emil 05.02.1863 – 26.08.1942@\textsc{Löbl, Emil} (05.02.1863 – 26.08.1942), \emph{Journalist}|pw} ein paar Zeilen. Vide: Ischler Brief\pwindex{Ischler Brief18.07.1893 – 18.07.1893@\emph{Ischler Brief} {[}18.07.1893 – 18.07.1893{]}|pwv}.\pend
           \endnumbering\briefempfaengerindex{Schnitzler, Arthur@\textsc{Schnitzler, Arthur}!zzzBeer-Hofmann, Richard@\emph{von Richard Beer-Hofmann}!1893-07-281@{28. 7. 1893}|)be}\mylabel{h}\end{ledgroupsized}  \newcommand{\dateiname}{L00245}\newcommand{\titel}{Richard Beer-Hofmann an Arthur Schnitzler, 28. 7. 1893}\newcommand{\editorInnen}{Martin Anton Müller und Gerd-Hermann Susen}%% latex-leseansicht-abspann.tex
%% Abspann für die Leseansicht.
%% Der Schalter \ifkorrekturansicht ist bereits durch den Vorspann gesetzt.

%% latex-abspann.tex
%% Gemeinsamer Abspann für Korrekturansicht und Leseansicht.
%% Setzt den Schalter \ifkorrekturansicht voraus (gesetzt in den
%% einbindenden Dateien latex-korrekturansicht-abspann.tex bzw.
%% latex-leseansicht-abspann.tex).
%% ---------------------------------------------------------------

\normalsize

% Das esempio-Environment wird nur in der Leseansicht benötigt
\ifkorrekturansicht\else
\newenvironment{esempio}[3]%
{
    \vspace{1.5ex}
    \rlap{\underline{#1}}
    \par
    \setlength{\parindent}{0cm}
    \nopagebreak
    \leftskip=#2cm
    \rightskip=#3cm
}
{
    \par
}
\fi

\doendnotes{C}
\bigskip
\vfill

\clearpage

\footnotesize

\ifkorrekturansicht
  \lohead{\textsc{register}}
\fi

% theindex-Environment neu definieren ohne reledmac
\makeatletter
\renewenvironment{theindex}{%
  \ifkorrekturansicht
    \section*{\indexname}%
  \else
    \subsubsection*{Index der erwähnten Entitäten}%
  \fi
  \setlength{\parindent}{0pt}%
  \setlength{\parskip}{0pt plus 0.3pt}%
  \let\item\@idxitem
}{%
  \ifkorrekturansicht\clearpage\fi
}
\makeatother

\IfFileExists{\jobname-pw.ind}{\input{\jobname-pw.ind}}{}

% Quellenangabe nur in der Leseansicht
\ifkorrekturansicht\else
% Fallback-Definitionen, falls die .tex-Datei \titel etc. nicht gesetzt hat
\providecommand{\titel}{}
\providecommand{\editorInnen}{}
\providecommand{\dateiname}{\jobname}

\vspace{3cm}

\vfill

\footnotesize
\textsc{Quelle}: \titel. Herausgegeben von {\editorInnen}. In: \emph{Arthur Schnitzler: Briefwechsel mit Autorinnen und Autoren}.
 Digitale Edition, https://schnitzler-briefe.acdh.oeaw.ac.at/{\dateiname}.html (Stand \today)
\fi

\end{document}


      