%% latex-korrekturansicht-vorspann.tex
%% Vorspann für die Korrekturansicht.
%% Lädt die gemeinsame Datei latex-vorspann.tex mit gesetztem Schalter.

\newif\ifkorrekturansicht
\korrekturansichttrue

\input{../tex-inputs/latex-vorspann}


\section[Richard Beer-Hofmann an Arthur Schnitzler, 28. 7. 1893]{L00245 Richard Beer-Hofmann an Arthur Schnitzler, 28. 7. 1893}
\nopagebreak\mylabel{L00245v}
\rehead{ }\normalsize\beginnumbering\briefempfaengerindex{Schnitzler, Arthur@\textsc{Schnitzler, Arthur}!zzzBeer-Hofmann, Richard@\emph{von Richard Beer-Hofmann}!1893-07-281@{28. 7. 1893}|(be}
\toendnotes[C]{\smallbreak\pagebreak[2]}\Standort{CUL, Schnitzler, B 8.}
\physDesc{Brief, 1 Blatt, 4 Seiten, 1044 Zeichen
\newline{}Handschrift: Bleistift, deutsche Kurrent
\newline{}Schnitzler: mit Bleistift nummeriert: »22« }
\buchAbdrucke{\weitereDrucke{Arthur Schnitzler, Richard Beer-Hofmann: \emph{Briefwechsel 1891–1931}. Wien, Zürich: \emph{Europaverlag} 1992, S. 48–49.} }\toendnotes[C]{\smallbreak}
\pstart
           \raggedleft{}{\pb}Freitag{ }Mittag. \pend
           \vspace{0.5em}
\pstart
           Lieber Arthur! Bin wieder seit vorgestern nachts hier. Las Ihren
               Brief an Frau F.\pwindex{Flegmann, Bertha 27.05.1852 – 24.6.1933@\textsc{Flegmann, Bertha} (27.05.1852 – 24.6.1933), \emph{männliche Salonnière/Salonnière}|pw}; das Telegramm\pwindex{Aus Ischl, 14. Juli, schreibt man uns: …@\emph{Aus Ischl, 14. Juli, schreibt man uns: …}|pwv} ist nicht von ihr; von Ben.\pwindex{Benedict, Markus 17.09.1834 – 26.2.1909@\textsc{Benedict, Markus} (17.09.1834 – 26.2.1909), \emph{Industrieller/Industrielle}|pw}?\pend
           
\pstart
           Im Börsencourir\orgindex{Berliner Boersen-Courier@Berliner Börsen-Courier|pw} von \strikeout{ge} – ? – ich höre in dem, der vorgestern hier war, – ich hoffe ihn zu
               erhalten {[}–{]} soll eine lange günstige Notiz\pwindex{Man schreibt uns aus Ischl]@\emph{[Man schreibt uns aus Ischl]}|pwv} stehen.\pend
           
\pstart
           {\pb}Ich habe Paul Horn\pwindex{Horn, Paul 13.02.1867 – 18.01.1936@\textsc{Horn, Paul} (13.02.1867 – 18.01.1936), \emph{Fabrikant/Fabrikantin}|pw} als er hier war sämtliche Daten gegeben; auch bez.
               Lektüre durch Reicher\pwindex{Reicher, Emanuel 18.06.1849 – 15.05.1924@\textsc{Reicher, Emanuel} (18.06.1849 – 15.05.1924), \emph{Schauspieler/Schauspielerin}|pw} u. Jarno\pwindex{Jarno, Josef 24.08.1865 – 11.01.1932@\textsc{Jarno, Josef} (24.08.1865 – 11.01.1932), \emph{Theaterleiter/Theaterleiterin, Schauspieler/Schauspielerin}|pw} in Berlin\oindex{Berlin@\textbf{Berlin}, \emph{P.PPLC}|pw}; dürfte
               also darin stehen. Heute wieder \label{K_L00245-1v}\edtext{Mamroth\pwindex{Mamroth, Fedor 21.02.1851 – 25.06.1907@\textsc{Mamroth, Fedor} (21.02.1851 – 25.06.1907), \emph{Journalist/Journalistin, Kritiker/Kritikerin}|pw} zitirt}{\lemma{\textnormal{\emph{Mamroth zitirt}}}\Cendnote{\textnormal{Fedor Mamroth an Arthur Schnitzler, 5. 3. 1893.}}}\label{K_L00245-1} (Tolstoi\pwindex{Tolstoi, Leo N. von 09.09.1828 – 20.11.1910@\textsc{Tolstoi, Leo N. von} (09.09.1828 – 20.11.1910), \emph{Schriftsteller/Schriftstellerin, Schriftsteller/Schriftstellerin, Krimiautor/Krimiautorin}|pw}) vor Frau Kalbek\pwindex{Kalbeck, Julie 1852-05-25 – 1922-06-17@\textsc{Kalbeck, Julie} (1852-05-25 – 1922-06-17), \emph{Hausmann/Hausfrau}|pw}.\pend
           
\pstart
           Ich glaube es wird gehen. Verhalten Sie sich nur gut mit F.\pwindex{Flegmann, Bertha 27.05.1852 – 24.6.1933@\textsc{Flegmann, Bertha} (27.05.1852 – 24.6.1933), \emph{männliche Salonnière/Salonnière}|pw}; sie setzt sich {\pb}wirklich für ihre Freunde ein.
               Bitte \uline{urgiren} Sie den Abschreiber\pwindex{?? [Schreibkraft fuer Arthur Schnitzler] @\textsc{?? [Schreibkraft für Arthur Schnitzler]}|pwv}; mir ist sehr darum zu thun die
               Sache hier vorlesen zu können solange Kalbeks\pwindex{Kalbeck, Max 1850-01-04 – 1921-05-04@\textsc{Kalbeck, Max} (1850-01-04 – 1921-05-04), \emph{Journalist/Journalistin}|pw}\pwindex{Kalbeck, Julie 1852-05-25 – 1922-06-17@\textsc{Kalbeck, Julie} (1852-05-25 – 1922-06-17), \emph{Hausmann/Hausfrau}|pw} u. \substVorne{}\textsuperscript{I}\substDazwischen{}i\substHinten{}hre Schwester eine Frau Lion\pwindex{Lion, Marie 1854 – 1899-12-20@\textsc{Lion, Marie} (1854 – 1899-12-20)|pw} da ist.
               Bitte!\pend
           
\pstart
           Heute, FreitagMittag, – ist noch nichts eingetroffen,
               hoffentlich kreuzt {\pb}es sich mit
               meinem Brief; der Schluss des Kindes\pwindex{Kind@\emph{Das Kind}|pw} ist
               endgiltig geändert, hoffentlich gefällt er jetzt besser.\pend
           
\pstart
           Grüßen Sie Schwarzkopf\pwindex{Schwarzkopf, Gustav 07.11.1853 – 13.11.1939@\textsc{Schwarzkopf, Gustav} (07.11.1853 – 13.11.1939), \emph{Schriftsteller/Schriftstellerin}|pw}{ }Salten\pwindex{Salten, Felix 06.09.1869 – 08.10.1945@\textsc{Salten, Felix} (06.09.1869 – 08.10.1945), \emph{Schriftsteller/Schriftstellerin, Journalist/Journalistin, Chefredakteur/Chefredakteurin}|pw}. Herzlichst Ihr\pend
           \pstart \spacefill\mbox{Richard}\pend{}
\pstart
           Ischl\oindex{Bad Ischl@\textbf{Bad Ischl}, \emph{P.PPL}|pw}. 28 Juli 93.\pend
           
\pstart
           \noindent{}Was sagen Sie zu \strikeout{Schr}{ }Wengraf\pwindex{Wengraf, Edmund 09.01.1860 – 08.12.1933@\textsc{Wengraf, Edmund} (09.01.1860 – 08.12.1933), \emph{Schriftsteller/Schriftstellerin, Journalist/Journalistin, Kaufmann/Kauffrau}|pw}{ }Hirschfeld\pwindex{Hirschfeld, Robert 17.09.1857 – 02.04.1914@\textsc{Hirschfeld, Robert} (17.09.1857 – 02.04.1914), \emph{Journalist/Journalistin, Musikkritiker/Musikkritikerin}|pw}\pwindex{Zwei Freunde Burckhards@\emph{Zwei Freunde Burckhards}|pwv}?\pend
           
\pstart
           Schreiben Sie Löbl\pwindex{Loebl, Emil 05.02.1863 – 26.08.1942@\textsc{Löbl, Emil} (05.02.1863 – 26.08.1942), \emph{Schriftsteller/Schriftstellerin, Journalist/Journalistin}|pw} ein paar Zeilen. Vide:
                     Ischler Brief\pwindex{Ischler Brief@\emph{Ischler Brief}|pwv}.\pend
           \selectlanguage{ngerman}\endnumbering\briefempfaengerindex{Schnitzler, Arthur@\textsc{Schnitzler, Arthur}!zzzBeer-Hofmann, Richard@\emph{von Richard Beer-Hofmann}!1893-07-281@{28. 7. 1893}|)be}\mylabel{L00245h}  \normalsize

\doendnotes{C}
\bigskip
\vfill

\clearpage

\footnotesize

\lohead{\textsc{register}}

% Definiere theindex-Environment komplett neu ohne reledmac
\makeatletter
\renewenvironment{theindex}{%
  \section*{\indexname}%
  \setlength{\parindent}{0pt}%
  \setlength{\parskip}{0pt plus 0.3pt}%
  \let\item\@idxitem
}{%
  \clearpage
}
\makeatother

\IfFileExists{\jobname-pw.ind}{\input{\jobname-pw.ind}}{}

\end{document}

      