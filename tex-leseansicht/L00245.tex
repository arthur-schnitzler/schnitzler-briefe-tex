%% latex-leseansicht-vorspann.tex
%% Vorspann für die Leseansicht.
%% Lädt die gemeinsame Datei latex-vorspann.tex mit nicht gesetztem Schalter.

\newif\ifkorrekturansicht
\korrekturansichtfalse

\input{../tex-inputs/latex-vorspann}


\section[Richard Beer-Hofmann an Arthur Schnitzler, 28. 7. 1893]{L00245 Richard Beer-Hofmann an Arthur Schnitzler, 28. 7. 1893}
\nopagebreak\mylabel{L00245v}
\rehead{ }\normalsize\beginnumbering\briefempfaengerindex{Schnitzler, Arthur@\textsc{Schnitzler, Arthur}!zzzBeer-Hofmann, Richard@\emph{von Richard Beer-Hofmann}!1893-07-281@{28. 7. 1893}|(be}
\toendnotes[C]{\smallbreak\pagebreak[2]}
\correspDesc{Versand  durch Richard Beer-Hofmann am 28. 7. 1893 in Bad Ischl
\newline{}Erhalt  durch Arthur Schnitzler im Zeitraum [29. 7. 1893
                  – 2. 8. 1893?] in Wien}\toendnotes[C]{\smallbreak}
\Standort{CUL, Schnitzler, B 8.}
\physDesc{Brief, 1 Blatt, 4 Seiten, 1044 Zeichen
\newline{}Handschrift: Bleistift, deutsche Kurrent
\newline{}Schnitzler: mit Bleistift nummeriert: »22« }
\buchAbdrucke{\weitereDrucke{Arthur Schnitzler, Richard Beer-Hofmann: \emph{Briefwechsel 1891–1931}. Herausgegeben von Konstanze Fliedl. Wien, Zürich: \emph{Europaverlag} 1992, S. 48–49.} }\toendnotes[C]{\smallbreak}
\pstart
           \raggedleft{}{\pb}Freitag{ }Mittag.\pend
           \vspace{0.5em}
\pstart
           Lieber Arthur! Bin wieder seit vorgestern nachts hier. Las Ihren
               Brief an Frau F.\pwindex{Flegmann, Bertha 27.\,5.\,1852 Dubrovsky, Polen – 24.\,6.\,1933 Bad Ischl@\textsc{Flegmann, Bertha} (27.\,5.\,1852 Dubrovsky, Polen – 24.\,6.\,1933 Bad Ischl), \emph{Salonnière}|pw}; das Telegramm\pwindex{Aus Ischl, 14. Juli, schreibt man uns: …@\emph{Aus Ischl, 14. Juli, schreibt man uns: …}|pwv} ist nicht von ihr; von Ben.\pwindex{Benedict, Markus 17.\,9.\,1834 Mikulov – 26.\,2.\,1909 Kärntnerring 13@\textsc{Benedict, Markus} (17.\,9.\,1834 Mikulov – 26.\,2.\,1909 Kärntnerring 13), \emph{Industrieller}|pw}?\pend
           
\pstart
           Im Börsencourir\orgindex{Berliner Börsen-Courier@Berliner Börsen-Courier|pw} von \strikeout{ge} – ? – ich höre in dem, der vorgestern hier war, – ich hoffe ihn zu
               erhalten {[}–{]} soll eine lange günstige Notiz\pwindex{Man schreibt uns aus Ischl]@\emph{[Man schreibt uns aus Ischl]}|pwv} stehen.\pend
           
\pstart
           {\pb}Ich habe Paul Horn\pwindex{Horn, Paul 13.\,2.\,1867 Wien – 18.\,1.\,1936 Menton@\textsc{Horn, Paul} (13.\,2.\,1867 Wien – 18.\,1.\,1936 Menton), \emph{Fabrikant}|pw} als er hier war sämtliche Daten gegeben; auch bez.
               Lektüre durch Reicher\pwindex{Reicher, Emanuel 18.\,6.\,1849 Bochnia – 15.\,5.\,1924 Berlin@\textsc{Reicher, Emanuel} (18.\,6.\,1849 Bochnia – 15.\,5.\,1924 Berlin), \emph{Schauspieler}|pw} u. Jarno\pwindex{Jarno, Josef 24.\,8.\,1865 Budapest – 11.\,1.\,1932 Wien@\textsc{Jarno, Josef} (24.\,8.\,1865 Budapest – 11.\,1.\,1932 Wien), \emph{Theaterleiter, Schauspieler}|pw} in Berlin\oindex{Berlin@\textbf{Berlin}, \emph{Hauptstadt}|pw}; dürfte
               also darin stehen. Heute wieder \label{K_L00245-1v}\edtext{Mamroth\pwindex{Mamroth, Fedor 21.\,2.\,1851 Breslau – 25.\,6.\,1907 Frankfurt am Main@\textsc{Mamroth, Fedor} (21.\,2.\,1851 Breslau – 25.\,6.\,1907 Frankfurt am Main), \emph{Journalist, Kritiker}|pw} zitirt}{\lemma{\textnormal{\emph{Mamroth zitirt}}}\Cendnote{\textnormal{XXXX Auszeichnungsfehler: Dokument L00186 nicht gefunden.}}}\label{K_L00245-1} (Tolstoi\pwindex{Tolstoi, Lew Nikolajewitsch 9.\,9.\,1828 Yasnaya Polyana – 20.\,11.\,1910 Lev Tolstoy@\textsc{Tolstoi, Lew Nikolajewitsch} (9.\,9.\,1828 Yasnaya Polyana – 20.\,11.\,1910 Lev Tolstoy), \emph{Schriftsteller}|pw}) vor Frau Kalbek\pwindex{Kalbeck, Julie 25.\,5.\,1852 Breslau – 17.\,6.\,1922 Wien@\textsc{Kalbeck, Julie} (25.\,5.\,1852 Breslau – 17.\,6.\,1922 Wien), \emph{Hausfrau}|pw}.\pend
           
\pstart
           Ich glaube es wird gehen. Verhalten Sie sich nur gut mit F.\pwindex{Flegmann, Bertha 27.\,5.\,1852 Dubrovsky, Polen – 24.\,6.\,1933 Bad Ischl@\textsc{Flegmann, Bertha} (27.\,5.\,1852 Dubrovsky, Polen – 24.\,6.\,1933 Bad Ischl), \emph{Salonnière}|pw}; sie setzt sich {\pb}wirklich für ihre Freunde ein.
               Bitte \uline{urgiren} Sie den Abschreiber\pwindex{?? [Schreibkraft für Arthur Schnitzler] @\textsc{?? [Schreibkraft für Arthur Schnitzler]}|pwv}; mir ist sehr darum zu thun die
               Sache hier vorlesen zu können solange Kalbeks\pwindex{Kalbeck, Max 4.\,1.\,1850 Breslau – 4.\,5.\,1921 Wien@\textsc{Kalbeck, Max} (4.\,1.\,1850 Breslau – 4.\,5.\,1921 Wien), \emph{Journalist}|pw}\pwindex{Kalbeck, Julie 25.\,5.\,1852 Breslau – 17.\,6.\,1922 Wien@\textsc{Kalbeck, Julie} (25.\,5.\,1852 Breslau – 17.\,6.\,1922 Wien), \emph{Hausfrau}|pw} u. \substVorne{}\textsuperscript{I}\substDazwischen{}i\substHinten{}hre Schwester eine Frau Lion\pwindex{Lion, Marie 1854 Breslau – 20.\,12.\,1899 Berlin@\textsc{Lion, Marie} (1854 Breslau – 20.\,12.\,1899 Berlin)|pw} da ist.
               Bitte!\pend
           
\pstart
           Heute, Freitag{ }Mittag, – ist noch nichts eingetroffen,
               hoffentlich kreuzt {\pb}es sich mit
               meinem Brief; der Schluss des Kindes\pwindex{Beer-Hofmann, Richard 11.\,7.\,1866 Wien – 26.\,9.\,1945 New York City@\textsc{Beer-Hofmann, Richard} (11.\,7.\,1866 Wien – 26.\,9.\,1945 New York City), \emph{Schriftsteller}!Kind@\strich\emph{Das Kind}|pw} ist
               endgiltig geändert, hoffentlich gefällt er jetzt besser.\pend
           
\pstart
           Grüßen Sie Schwarzkopf\pwindex{Schwarzkopf, Gustav 7.\,11.\,1853 Wien – 13.\,11.\,1939 ebd.@\textsc{Schwarzkopf, Gustav} (7.\,11.\,1853 Wien – 13.\,11.\,1939 ebd.), \emph{Schriftsteller}|pw}{ }Salten\pwindex{Salten, Felix 6.\,9.\,1869 Budapest – 8.\,10.\,1945 Zürich@\textsc{Salten, Felix} (6.\,9.\,1869 Budapest – 8.\,10.\,1945 Zürich), \emph{Schriftsteller, Journalist, Chefredakteur}|pw}. Herzlichst Ihr\pend
           \pstart \spacefill\mbox{Richard}\pend{}
\pstart
           Ischl\oindex{Bad Ischl@\textbf{Bad Ischl}|pw}. 28 Juli 93.\pend
           
\pstart
           \noindent{}Was sagen Sie zu \strikeout{Schr}{ }Wengraf\pwindex{Wengraf, Edmund 9.\,1.\,1860 Mikulov – 8.\,12.\,1933 Wien@\textsc{Wengraf, Edmund} (9.\,1.\,1860 Mikulov – 8.\,12.\,1933 Wien), \emph{Schriftsteller, Journalist, Kaufmann}|pw}{ }Hirschfeld\pwindex{Hirschfeld, Robert 17.\,9.\,1857 Žďár nad Sázavou – 2.\,4.\,1914 Salzburg@\textsc{Hirschfeld, Robert} (17.\,9.\,1857 Žďár nad Sázavou – 2.\,4.\,1914 Salzburg), \emph{Journalist, Musikkritiker}|pw}\pwindex{Hirschfeld, Robert 17.\,9.\,1857 Žďár nad Sázavou – 2.\,4.\,1914 Salzburg@\textsc{Hirschfeld, Robert} (17.\,9.\,1857 Žďár nad Sázavou – 2.\,4.\,1914 Salzburg), \emph{Journalist, Musikkritiker}!Zwei Freunde Burckhards@\strich\emph{Zwei Freunde Burckhards}|pwv}?\pend
           
\pstart
           Schreiben Sie Löbl\pwindex{Löbl, Emil 5.\,2.\,1863 Wien – 26.\,8.\,1942 ebd.@\textsc{Löbl, Emil} (5.\,2.\,1863 Wien – 26.\,8.\,1942 ebd.), \emph{Schriftsteller, Journalist}|pw} ein paar Zeilen. Vide:
                     Ischler Brief\pwindex{\textcolor{red}{\textsuperscript{XXXX indx1}}!Ischler Brief@\strich\emph{Ischler Brief}|pwv}.\pend
           \selectlanguage{ngerman}\endnumbering\briefempfaengerindex{Schnitzler, Arthur@\textsc{Schnitzler, Arthur}!zzzBeer-Hofmann, Richard@\emph{von Richard Beer-Hofmann}!1893-07-281@{28. 7. 1893}|)be}\mylabel{L00245h}  \newcommand{\dateiname}{L00245}\newcommand{\titel}{Richard Beer-Hofmann an Arthur Schnitzler, 28. 7. 1893}\newcommand{\editorInnen}{Martin Anton Müller und Gerd-Hermann Susen}%% latex-leseansicht-abspann.tex
%% Abspann für die Leseansicht.
%% Der Schalter \ifkorrekturansicht ist bereits durch den Vorspann gesetzt.

%% latex-abspann.tex
%% Gemeinsamer Abspann für Korrekturansicht und Leseansicht.
%% Setzt den Schalter \ifkorrekturansicht voraus (gesetzt in den
%% einbindenden Dateien latex-korrekturansicht-abspann.tex bzw.
%% latex-leseansicht-abspann.tex).
%% ---------------------------------------------------------------

\normalsize

% Das esempio-Environment wird nur in der Leseansicht benötigt
\ifkorrekturansicht\else
\newenvironment{esempio}[3]%
{
    \vspace{1.5ex}
    \rlap{\underline{#1}}
    \par
    \setlength{\parindent}{0cm}
    \nopagebreak
    \leftskip=#2cm
    \rightskip=#3cm
}
{
    \par
}
\fi

\doendnotes{C}
\bigskip
\vfill

\clearpage

\footnotesize

\ifkorrekturansicht
  \lohead{\textsc{register}}
\fi

% theindex-Environment neu definieren ohne reledmac
\makeatletter
\renewenvironment{theindex}{%
  \ifkorrekturansicht
    \section*{\indexname}%
  \else
    \subsubsection*{Index der erwähnten Entitäten}%
  \fi
  \setlength{\parindent}{0pt}%
  \setlength{\parskip}{0pt plus 0.3pt}%
  \let\item\@idxitem
}{%
  \ifkorrekturansicht\clearpage\fi
}
\makeatother

\IfFileExists{\jobname-pw.ind}{\input{\jobname-pw.ind}}{}

% Quellenangabe nur in der Leseansicht
\ifkorrekturansicht\else
% Fallback-Definitionen, falls die .tex-Datei \titel etc. nicht gesetzt hat
\providecommand{\titel}{}
\providecommand{\editorInnen}{}
\providecommand{\dateiname}{\jobname}

\vspace{3cm}

\vfill

\footnotesize
\textsc{Quelle}: \titel. Herausgegeben von {\editorInnen}. In: \emph{Arthur Schnitzler: Briefwechsel mit Autorinnen und Autoren}.
 Digitale Edition, https://schnitzler-briefe.acdh.oeaw.ac.at/{\dateiname}.html (Stand \today)
\fi

\end{document}


