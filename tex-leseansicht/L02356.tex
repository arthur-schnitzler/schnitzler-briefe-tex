%% latex-leseansicht-vorspann.tex
%% Vorspann für die Leseansicht.
%% Lädt die gemeinsame Datei latex-vorspann.tex mit nicht gesetztem Schalter.

\newif\ifkorrekturansicht
\korrekturansichtfalse

\input{../tex-inputs/latex-vorspann}


         
         \renewcommand{\erwaehntePersonen}{Personen: Robert Adam}
         \renewcommand{\erwaehnteInstitutionen}{Institutionen: Deutschösterreichischer Autorenverband}
         \renewcommand{\erwaehnteOrte}{Orte: IX., Alsergrund, Meidlinger Hauptstraße, Sternwartestraße 71, Wien, XII., Meidling, XVIII., Währing}
         \renewcommand{\erwaehnteWerke}{}
               \section[Arthur Schnitzler an Robert Adam, 28. 10. 1920]{ Arthur Schnitzler an Robert Adam, 28. 10. 1920}\nopagebreak\mylabel{v}\rehead{ }\begin{ledgroupsized}[t]{13cm}\normalsize\beginnumbering\briefempfaengerindex{Adam, Robert@\textsc{Adam, Robert}!zzzSchnitzler, Arthur@\emph{von Arthur Schnitzler}!1920-10-281@{28. 10. 1920}|(be} \toendnotes[C]{\smallbreak\pagebreak[2]} \Standort{DLA, 96.34.2/23.}
\physDesc{Brief, 1 Blatt, 1 Seite, Umschlag, 1412 Zeichen
\newline{}Schreibmaschine
\newline{}Handschrift: schwarze Tinte, deutsche Kurrent (\noindent{}Korrektur und Unterschrift)
\newline{}Versand: Stempel: »\nobreak{}\oindex{IX., Alsergrund@\textbf{IX., Alsergrund}|pwk}Wien 72, 28. X. 20, 6\nobreak{}«.  }\Standort{DLA, A:Schnitzler, 85.1.1621.}
\physDesc{Brief, Durchschlag, 1 Blatt, 1 Seite, Umschlag, 1412 Zeichen
\newline{}Schreibmaschine
\newline{}Handschrift: Bleistift, lateinische Kurrent (\noindent{}Beschriftung »\strikeout{Pollak Adam}{ }Adam«)}\toendnotes[C]{\smallbreak}\pstart{}{\pb}\textcolor{gray}{\textbf{\textit{Dr. ARTHUR SCHNITZLER}}}\pend{}\pstart{}\textcolor{gray}{\textbf{\textit{Wien XVIII.}}}\oindex{XVIII., Waehring@\textbf{XVIII., Währing}|pw}\pend{}\pstart{}\textcolor{gray}{\textbf{\textit{STERNWARTESTRASSE 71}}}\oindex{Sternwartestrasse 71@\textbf{Sternwartestraße 71}|pw}\pend{}{\bigskip}\pstart{}{\pb}Herrn\pend{}\pstart{}Oberlandesgerichtsrat\pend{}\pstart{}Dr. Robert Adam \so{Pollak},\pend{}\pstart{}Wien XII\oindex{XII., Meidling@\textbf{XII., Meidling}|pw}.\pend{}\pstart{}Meidlinger Hauptstr. 52\oindex{Meidlinger Hauptstrasse@\textbf{Meidlinger Hauptstraße}|pw}.\pend{}{\bigskip}\pstart
           {\pb}\textcolor{gray}{\textbf{Dr. ARTHUR SCHNITZLER}}\hfill 28. 10. 1920.\pend
           \pstart
           \textcolor{gray}{\textbf{WIEN, XVIII. STERNWARTESTRASSE 71\oindex{Sternwartestrasse 71@\textbf{Sternwartestraße 71}|pw}}}\pend
           \pstart{}Verehrter Herr Doktor,\pend\pstart
           Innerhalb der dramatischen Autorengenossenschaft\orgindex{Deutschoesterreichischer Autorenverband@Deutschösterreichischer Autorenverband|pw}
               soll \strikeout{es} statutengemäss ein Schiedsgericht ins Leben
               gerufen werden, ungefähr nach den Prinzipien, wie sie in den beiliegenden Statuten
               des Bühnenschiedsgerichtes ausgesprochen sind. Als Obmann dieses Schiedsgerichtes
               soll ein Berufsrichter fungieren und ich habe mir erlaubt in der \label{K_L02356-1v}\edtext{Vorstandssitzung}{\lemma{\textnormal{\emph{Vorstandssitzung}}}\Cendnote{\textnormal{am 25. 10. 1920}}}\label{K_L02356-1h}, wo diese Angelegenheit zur Sprache kam, Sie, lieber Herr Doktor, als
               denjenigen zu nennen, der mir für ein solches Amt schon dadurch höchst geeignet
               erschiene, weil in Ihnen eben Eigenschaften eines dramatischen Autors und eines
               Richters sich vereinen. Man hat mich ersucht unverbindlich bei Ihnen anzufragen, ob
               Sie geneigt wären ein solches Amt eventuell zu übernehmen. Ist es gewissermassen auch
               ein Ehrenamt, so soll es keineswegs ein unbesoldetes sein; nach ungefährer Berechnung
               dürften im Jahre 10–20 Fälle zur schiedsrichterlichen Verhandlung kommen und es wird
               auch daran gedacht, eventuell nicht einen, sondern zwei Vorsitzende zu ernennen, so
               dass sich also die Arbeit verteilen würde. Bitte schreiben Sie mir recht bald, wie
               Sie sich prinzipiell zu dieser Angelegenheit verhalten.\pend
           \pstart
           Mit herzlichem Gruss{\\[\baselineskip]}Ihr sehr ergebener{\\[\baselineskip]}\spacefill\mbox{{[}hs.:{]} Arthur Schnitzler}\pend
           \leftskip=0em{}\pstart
           \noindent{}{[}ms.:{]} Herrn Oberlandsgerichtsrat{\\}Dr. Robert Adam Pollak,{\\}Wien\oindex{Wien@\textbf{Wien}|pw}.\pend
           
         
         \endnumbering\mylabel{h}\end{ledgroupsized}  \newcommand{\dateiname}{L02356}\newcommand{\titel}{Arthur Schnitzler an Robert Adam, 28. 10. 1920}\newcommand{\editorInnen}{Martin Anton Müller und Gerd-Hermann Susen}%% latex-leseansicht-abspann.tex
%% Abspann für die Leseansicht.
%% Der Schalter \ifkorrekturansicht ist bereits durch den Vorspann gesetzt.

%% latex-abspann.tex
%% Gemeinsamer Abspann für Korrekturansicht und Leseansicht.
%% Setzt den Schalter \ifkorrekturansicht voraus (gesetzt in den
%% einbindenden Dateien latex-korrekturansicht-abspann.tex bzw.
%% latex-leseansicht-abspann.tex).
%% ---------------------------------------------------------------

\normalsize

% Das esempio-Environment wird nur in der Leseansicht benötigt
\ifkorrekturansicht\else
\newenvironment{esempio}[3]%
{
    \vspace{1.5ex}
    \rlap{\underline{#1}}
    \par
    \setlength{\parindent}{0cm}
    \nopagebreak
    \leftskip=#2cm
    \rightskip=#3cm
}
{
    \par
}
\fi

\doendnotes{C}
\bigskip
\vfill

\clearpage

\footnotesize

\ifkorrekturansicht
  \lohead{\textsc{register}}
\fi

% theindex-Environment neu definieren ohne reledmac
\makeatletter
\renewenvironment{theindex}{%
  \ifkorrekturansicht
    \section*{\indexname}%
  \else
    \subsubsection*{Index der erwähnten Entitäten}%
  \fi
  \setlength{\parindent}{0pt}%
  \setlength{\parskip}{0pt plus 0.3pt}%
  \let\item\@idxitem
}{%
  \ifkorrekturansicht\clearpage\fi
}
\makeatother

\IfFileExists{\jobname-pw.ind}{\input{\jobname-pw.ind}}{}

% Quellenangabe nur in der Leseansicht
\ifkorrekturansicht\else
% Fallback-Definitionen, falls die .tex-Datei \titel etc. nicht gesetzt hat
\providecommand{\titel}{}
\providecommand{\editorInnen}{}
\providecommand{\dateiname}{\jobname}

\vspace{3cm}

\vfill

\footnotesize
\textsc{Quelle}: \titel. Herausgegeben von {\editorInnen}. In: \emph{Arthur Schnitzler: Briefwechsel mit Autorinnen und Autoren}.
 Digitale Edition, https://schnitzler-briefe.acdh.oeaw.ac.at/{\dateiname}.html (Stand \today)
\fi

\end{document}


      