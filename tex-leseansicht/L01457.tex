%% latex-korrekturansicht-vorspann.tex
%% Vorspann für die Korrekturansicht.
%% Lädt die gemeinsame Datei latex-vorspann.tex mit gesetztem Schalter.

\newif\ifkorrekturansicht
\korrekturansichttrue

\input{../tex-inputs/latex-vorspann}


\section[Hugo von Hofmannsthal an Arthur Schnitzler, 18. 10. 1904]{L01457 Hugo von Hofmannsthal an Arthur Schnitzler, 18. 10. 1904}
\nopagebreak\mylabel{L01457v}
\rehead{ }\normalsize\beginnumbering\briefempfaengerindex{Schnitzler, Arthur@\textsc{Schnitzler, Arthur}!zzzHofmannsthal, Hugo von@\emph{von Hugo von Hofmannsthal}!1904-10-181@{18. 10. 1904}|(be}
\toendnotes[C]{\smallbreak\pagebreak[2]}\Standort{CUL, Schnitzler, B 43.}
\physDesc{Postkarte, 99 Zeichen
\newline{}Handschrift Gertrude von Hofmannsthal: schwarze Tinte, lateinische Kurrent
\newline{}Versand: 1) Stempel: »\nobreak{}\oindex{X., Favoriten@\textbf{X., Favoriten}, \emph{A.ADM3}|pwk}Wien 10/2, 18. X. 04, 10\nobreak{}«.   2) Stempel: »\nobreak{}\oindex{VII., Neubau@\textbf{VII., Neubau}, \emph{A.ADM3}|pwk}Wien 7/3, 19. 10. 04, 8–9V, Bestellt\nobreak{}«.  3) Stempel: »\nobreak{}\oindex{VII., Neubau@\textbf{VII., Neubau}, \emph{A.ADM3}|pwk}18/1 Wien 110, 19. 10. 04, 10V, Bestellt\nobreak{}«. 
\newline{}Schnitzler: mit Bleistift datiert: »18/10 904« 
\newline{}Ordnung: 1) mit Bleistift von unbekannter Hand nummeriert: »\strikeout{227}«  2) mit Bleistift von unbekannter Hand nummeriert:
                                    »240«}
\buchAbdrucke{\weitereDrucke{Hugo von Hofmannsthal, Arthur Schnitzler: \emph{Briefwechsel}. Frankfurt am Main: \emph{S. Fischer} 1964, S. 207.} }\toendnotes[C]{\smallbreak}\pstart{}{\pb}Herrn Dr Arthur
                  Schnitzler\pend{}\pstart{}Wien\oindex{Wien@\textbf{Wien}, \emph{A.ADM2}|pw}\pend{}\pstart{}XVIII Spöttlgasse 7\oindex{Edmund-Weiss-Gasse 7@\textbf{Edmund-Weiß-Gasse 7}, \emph{Wohngebäude (K.WHS)}|pw}. \pend{}{\bigskip}\vspace{1em}
\pstart
           \noindent{}{\pb}Mit Freude \label{K_L01457-1v}\edtext{Mittwoch}{\lemma{\textnormal{\emph{Mittwoch}}}\Cendnote{\textnormal{Siehe A. S.: \emph{Tagebuch}, 19. 10. 1904.
                  }}}\label{K_L01457-1}{ }abends{ }Hietzing\oindex{XIII., Hietzing@\textbf{XIII., Hietzing}, \emph{A.ADM3}|pw}\pend
           \pstart Herzlichst \spacefill\mbox{Hugo.}\pend{}\selectlanguage{ngerman}\endnumbering\briefempfaengerindex{Schnitzler, Arthur@\textsc{Schnitzler, Arthur}!zzzHofmannsthal, Hugo von@\emph{von Hugo von Hofmannsthal}!1904-10-181@{18. 10. 1904}|)be}\mylabel{L01457h}  \normalsize

\doendnotes{C}
\bigskip
\vfill

\clearpage

\footnotesize

\lohead{\textsc{register}}

% Definiere theindex-Environment komplett neu ohne reledmac
\makeatletter
\renewenvironment{theindex}{%
  \section*{\indexname}%
  \setlength{\parindent}{0pt}%
  \setlength{\parskip}{0pt plus 0.3pt}%
  \let\item\@idxitem
}{%
  \clearpage
}
\makeatother

\IfFileExists{\jobname-pw.ind}{\input{\jobname-pw.ind}}{}

\end{document}

      