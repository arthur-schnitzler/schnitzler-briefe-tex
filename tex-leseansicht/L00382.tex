%% latex-korrekturansicht-vorspann.tex
%% Vorspann für die Korrekturansicht.
%% Lädt die gemeinsame Datei latex-vorspann.tex mit gesetztem Schalter.

\newif\ifkorrekturansicht
\korrekturansichttrue

\input{../tex-inputs/latex-vorspann}


\section[Arthur Schnitzler an Richard Beer-Hofmann, 15. 10. 1894]{L00382 Arthur Schnitzler an Richard Beer-Hofmann, 15. 10. 1894}
\nopagebreak\mylabel{L00382v}
\rehead{ }\normalsize\beginnumbering\briefempfaengerindex{Beer-Hofmann, Richard@\textsc{Beer-Hofmann, Richard}!zzzSchnitzler, Arthur@\emph{von Arthur Schnitzler}!1894-10-151@{15. 10. 1894}|(be}
\toendnotes[C]{\smallbreak\pagebreak[2]}\Standort{YCGL, MSS 31.}
\physDesc{Brief, 2 Blätter, 7 Seiten, Umschlag, 1937 Zeichen
\newline{}Handschrift: schwarze Tinte, deutsche Kurrent
\newline{}Versand: 1) nachgesandt nach \textsc{Hotel Hassler}\oindex{Hôtel Hassler@\textbf{Hôtel Hassler}, \emph{Hotel (K.HTL)}|pw}  2) Stempel: »\nobreak{}\oindex{I., Innere Stadt@\textbf{I., Innere Stadt}, \emph{A.ADM3}|pwk}Wien 1/1, 15. 10. 94, 11\textcolor{gray}{–12}N\nobreak{}«.  3) Stempel: »\nobreak{}\oindex{Neapel@\textbf{Neapel}, \emph{P.PPLA}|pwk}Napoli, \textcolor{gray}{7} 10–94, 8 S\nobreak{}«. }
\buchAbdrucke{\weitereDrucke{1) Arthur Schnitzler: \emph{Briefe 1875–1912}. Frankfurt am Main: \emph{S. Fischer} 1981, S. 231.} \weitereDrucke{2) Arthur Schnitzler, Richard Beer-Hofmann: \emph{Briefwechsel 1891–1931}. Wien, Zürich: \emph{Europaverlag} 1992, S. 63–64.} }\toendnotes[C]{\smallbreak}\pstart{}{\pb}\textsc{Dr. Arthur Schnitzler}, Wien, IX. Frankgaſſe 1.\oindex{Frankgasse 1@\textbf{Frankgasse 1}, \emph{Wohngebäude (K.WHS)}|pw}\pend{}{\bigskip}\pstart{}{\pb}Herrn \textsc{Dr. Richard
                     Beer-Hofmann}\pend{}\pstart{}\textsc{Neapel\oindex{Neapel@\textbf{Neapel}, \emph{P.PPLA}|pw}}\pend{}\pstart{}\textsc{(Napoli\oindex{Neapel@\textbf{Neapel}, \emph{P.PPLA}|pw})}\pend{}\pstart{}\textsc{a poste ferma}\pend{}\pstart{}\textsc{Italien\oindex{Italien@\textbf{Italien}, \emph{A.PCLI}|pw}}\pend{}{\bigskip}\vspace{1em}
\pstart
           \raggedleft{}{\pb}Wien\oindex{Wien@\textbf{Wien}, \emph{A.ADM2}|pw}, \uline{15. Oct. 94}.\pend
           \vspace{0.5em}
\pstart
           Lieber Richard – Sie würden es nicht verdienen, daſs man Ihnen
               ſchreibt – aber ich nehme an, Sie empfinden den Empfang eines Briefs von mir nicht
               als Glück – alſo – Sie verstehen ja dieses \label{K_L00382-1v}\edtext{linke Ohr}{\lemma{\textnormal{\emph{linke Ohr}}}\Cendnote{\textnormal{»Pollack, wo hast Du Dein linkes Ohr?« – Stehende Redewendung für den Griff mit
                  der rechten Hand über den Kopf zum linken Ohr. Ein (jüdischer) Junge, vom Lehrer
                  gefragt, wo er sein linkes Ohr habe, soll diese umständliche Geste gemacht haben.
                     Vgl. Richard Beer-Hofmann an Arthur Schnitzler, 22. 2. 1900. }}}\label{K_L00382-1}? –\pend
           
\pstart
           {\pb}Gestern hab ich dem Hugo\pwindex{Hofmannsthal, Hugo von 1874-02-01 – 1929-07-15@\textsc{Hofmannsthal, Hugo von} (1874-02-01 – 1929-07-15), \emph{Schriftsteller/Schriftstellerin}|pw} und Salten\pwindex{Salten, Felix 06.09.1869 – 08.10.1945@\textsc{Salten, Felix} (06.09.1869 – 08.10.1945), \emph{Schriftsteller/Schriftstellerin, Journalist/Journalistin, Chefredakteur/Chefredakteurin}|pw} mein Stück\pwindex{Liebelei. Schauspiel in drei Akten@\emph{Liebelei. Schauspiel in drei Akten}|pwv} vorgeleſen, – mit einem von mir nicht
               geahnten Erfolg. Es ſollen nur ein paar Wendungen drin zu ändern und ſonſt ſoll es
               ganz fertig ſein – das übrige Lob ſchäm ich mich beizufügen. Ich bin aber ſehr
               froh. – Momentan ſchreib ich {\pb}einen Einakter\pwindex{Frau mit dem Dolche@\emph{Die Frau mit dem Dolche}|pwv}. (15. Jahrhundert – aber es iſt
               eigentlich eine Fälſchung.) –\pend
           
\pstart
           Es iſt läppiſch, daſs Sie mir ſo gut wie gar nichts ſchreiben. Ich ſage läppiſch, in
               der Ueberzeugung dſs das Sie viel mehr beleidigt als infam oder schurkiſch, was man
               auch ſagen könnte. – Hugo\pwindex{Hofmannsthal, Hugo von 1874-02-01 – 1929-07-15@\textsc{Hofmannsthal, Hugo von} (1874-02-01 – 1929-07-15), \emph{Schriftsteller/Schriftstellerin}|pw}{ }ſieht als Dragoner {\pb}ausgezeichnet aus. Ein \textsc{Oberlieutn}. zum andern: »Du, ich
               hör, du haſt in deiner Abthlg einen, der Trauerſpiel dicht’ –?« –\pend
           
\pstart
           \textsc{Salten}\pwindex{Salten, Felix 06.09.1869 – 08.10.1945@\textsc{Salten, Felix} (06.09.1869 – 08.10.1945), \emph{Schriftsteller/Schriftstellerin, Journalist/Journalistin, Chefredakteur/Chefredakteurin}|pw}, hab ich Ihnen das ſchon geſchrieben?, – ist in der Redaction der allgem. Zeitung\orgindex{Wiener Allgemeine Zeitung@Wiener Allgemeine Zeitung|pw}. – Neulich hat er den \textsc{Suderma{\geminationn}}\pwindex{Sudermann, Hermann 30.09.1857 – 21.11.1928@\textsc{Sudermann, Hermann} (30.09.1857 – 21.11.1928), \emph{Schriftsteller/Schriftstellerin}|pw}{ }\label{K_L00382-2v}\edtext{\textsc{interviewt}\pwindex{Bei Hermann Sudermann@\emph{Bei Hermann Sudermann}|pwv}}{\lemma{\textnormal{\emph{interviewt}}}\Cendnote{\textnormal{–x. –n.\pwindex{Salten, Felix 06.09.1869 – 08.10.1945@\textsc{Salten, Felix} (06.09.1869 – 08.10.1945), \emph{Schriftsteller/Schriftstellerin, Journalist/Journalistin, Chefredakteur/Chefredakteurin}|pwk}: \emph{Bei Hermann Sudermann}\pwindex{Bei Hermann Sudermann@\emph{Bei Hermann Sudermann}|pwk}. In: \emph{Wiener
                        Allgemeine Zeitung}\pwindex{Wiener Allgemeine Zeitung@\emph{Wiener Allgemeine Zeitung}|pwk}, Nr. 4977, 13. 10. 1894,
                  S. 2–3.}}}\label{K_L00382-2}, und der kleine Kraus\pwindex{Kraus, Karl 28.04.1874 – 12.06.1936@\textsc{Kraus, Karl} (28.04.1874 – 12.06.1936), \emph{Schriftsteller/Schriftstellerin, Publizist/Publizistin, Schriftsteller/Schriftstellerin}|pw}
               erklärt das für unerhört charakterlos.\pend
           
\pstart
           {\pb}Wünſchen Sie auch von \textsc{Fels}\pwindex{Fels, Friedrich Michael *~1864@\textsc{Fels, Friedrich Michael} (*~1864), \emph{Journalist/Journalistin}|pw} was zu wiſſen? Ich zweifle nicht daran. Alſo: alles beim alten; – was Sie ſchon
               merken werden, wenn Sie zurückko{\geminationm}en. – Wünſchen Sie was
               von \textsc{Korff}\pwindex{Korff, Heinrich von 05.06.1868 – 18.08.1938@\textsc{Korff, Heinrich von} (05.06.1868 – 18.08.1938), \emph{Journalist/Journalistin}|pw} zu wiſſen? Er hat eine Hebamme\pwindex{Korff, Irma von @\textsc{Korff, Irma von}, \emph{männliche Hebamme/Hebamme}|pwv} geheiratet, welche aber kaum 15 Jahre älter iſt als er. – Und \textsc{Specht}\pwindex{Specht, Richard 07.12.1870 – 18.03.1932@\textsc{Specht, Richard} (07.12.1870 – 18.03.1932), \emph{Schriftsteller/Schriftstellerin, Journalist/Journalistin, Kritiker/Kritikerin}|pw}? – Er fährt nächſtens {\pb}auf ein Jahr nach \textsc{Liverpool}\oindex{Liverpool@\textbf{Liverpool}, \emph{P.PPLA2}|pw}. Und \textsc{Paul von Schönthan}\pwindex{Specht, Richard 07.12.1870 – 18.03.1932@\textsc{Specht, Richard} (07.12.1870 – 18.03.1932), \emph{Schriftsteller/Schriftstellerin, Journalist/Journalistin, Kritiker/Kritikerin}|pw}? Er wünſcht ſehnlichſt, Sie zum Saubermann\orgindex{Saubermaenner. Tischgesellschaft@Die Saubermänner. Tischgesellschaft|pw}
               zu geſtalten. – Neulich hab ich den \textsc{Julian Sternberg}\pwindex{Sternberg, Julian 08.11.1868 – 28.06. 1945@\textsc{Sternberg, Julian} (08.11.1868 – 28.06. 1945), \emph{Journalist/Journalistin}|pw} (den bei dem Sie ſich ſo einzuſchmeicheln »gewußt« haben) ke{\geminationn}en gelernt; da hat er mir ſehr gut gefallen. –\pend
           
\pstart
           {\pb}Außerdem regnets, iſt kalt, und der Winter iſt
               da. –\pend
           
\pstart
           Leben Sie wohl und ſchreiben Sie einem doch wenigſtens endlich einmal, wann man sie
               »wieder haben« wird.\pend
           
\pstart
           Herzlich der Ihre{\\[\baselineskip]}\spacefill\mbox{Arthur}\pend
           \leftskip=0em{}
\pstart
           \noindent{}»Zeit\orgindex{Zeit. Wiener Wochenschrift@Die Zeit. Wiener Wochenschrift|pw}« wird beſorgt. Sie iſt \uline{ſehr} gut\pend
           \selectlanguage{ngerman}\endnumbering\briefempfaengerindex{Beer-Hofmann, Richard@\textsc{Beer-Hofmann, Richard}!zzzSchnitzler, Arthur@\emph{von Arthur Schnitzler}!1894-10-151@{15. 10. 1894}|)be}\mylabel{L00382h}  \normalsize

\doendnotes{C}
\bigskip
\vfill

\clearpage

\footnotesize

\lohead{\textsc{register}}

% Definiere theindex-Environment komplett neu ohne reledmac
\makeatletter
\renewenvironment{theindex}{%
  \section*{\indexname}%
  \setlength{\parindent}{0pt}%
  \setlength{\parskip}{0pt plus 0.3pt}%
  \let\item\@idxitem
}{%
  \clearpage
}
\makeatother

\IfFileExists{\jobname-pw.ind}{\input{\jobname-pw.ind}}{}

\end{document}

      