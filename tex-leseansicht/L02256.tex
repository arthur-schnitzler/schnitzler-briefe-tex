%% latex-korrekturansicht-vorspann.tex
%% Vorspann für die Korrekturansicht.
%% Lädt die gemeinsame Datei latex-vorspann.tex mit gesetztem Schalter.

\newif\ifkorrekturansicht
\korrekturansichttrue

\input{../tex-inputs/latex-vorspann}


\section[Arthur Schnitzler an Hugo von Hofmannsthal, 19. 2. 1917]{L02256 Arthur Schnitzler an Hugo von Hofmannsthal, 19. 2. 1917}
\nopagebreak\mylabel{L02256v}
\rehead{ }\normalsize\beginnumbering\briefempfaengerindex{Hofmannsthal, Hugo von@\textsc{Hofmannsthal, Hugo von}!zzzSchnitzler, Arthur@\emph{von Arthur Schnitzler}!1917-02-191@{19. 2. 1917}|(be}
\toendnotes[C]{\smallbreak\pagebreak[2]}\buchAlsQuelle{Hugo von Hofmannsthal, Arthur Schnitzler: \emph{Briefwechsel}. Frankfurt am Main: \emph{S. Fischer} 1964, S. 281.}\toendnotes[C]{\smallbreak}
\pstart
           {\pb}\label{K_L02256-1v}\edtext{{[}Maschinenschrift{]}}{\lemma{\textnormal{\emph{Maschinenschrift}}}\Cendnote{\textnormal{Die Vorlage ist nicht
                        nachweisbar.}}}\label{K_L02256-1}\hfill 19. 2. 1917.\pend
           
\pstart{}Lieber Hugo.\pend\vspace{0.5em}
\pstart
           Der Anonymus\pwindex{Billiter, Jean 23.05.1877 – 01.04.1965@\textsc{Billiter, Jean} (23.05.1877 – 01.04.1965), \emph{Chemiker/Chemikerin}|pwv}, dessen zwei
                  \label{K_L02256-2v}\edtext{Einakter}{\lemma{\textnormal{\emph{Einakter}}}\Cendnote{\textnormal{nicht ermittelt}}}\label{K_L02256-2}{ }Sie mir zurückließen, ist \label{K_L02256-3v}\edtext{gestern}{\lemma{\textnormal{\emph{gestern}}}\Cendnote{\textnormal{Vgl. A. S.: \emph{Tagebuch}, 18. 2. 1917.
               }}}\label{K_L02256-3} während ich nicht zu Hause war, bei mir erschienen, hat sich, was Ihnen kein
               Geheimnis sein dürfte, als Privatdozent Dr. Jean
                  Billiter\pwindex{Billiter, Jean 23.05.1877 – 01.04.1965@\textsc{Billiter, Jean} (23.05.1877 – 01.04.1965), \emph{Chemiker/Chemikerin}|pw} entpuppt und ein drittes Stück dagelassen, das nicht besser ist als
               die zwei andern und das er sich (wie er mir auf einer Karte mitteilt) zwischen jenen
               aufgeführt denken würde. Bevor ich \label{K_L02256-4v}\edtext{ihn
               nun empfange}{\lemma{\textnormal{\emph{ihn
               nun empfange}}}\Cendnote{\textnormal{Vgl. A. S.: \emph{Tagebuch}, 20. 3. 1917.
               }}}\label{K_L02256-4} wünschte ich sehr von Ihnen zu wissen, ob Herr B.\pwindex{Billiter, Jean 23.05.1877 – 01.04.1965@\textsc{Billiter, Jean} (23.05.1877 – 01.04.1965), \emph{Chemiker/Chemikerin}|pw} etwa von einer durch mich herzustellenden Verbindung mit dem
                  Burgtheater\oindex{Burgtheater@\textbf{Burgtheater}, \emph{S.THTR}|pw} oder sonst einer Bühne träumt und
               ob er sich vielleicht schon anderweitig literarisch oder sonstwie in einer mir nicht
               bekannt gewordenen Weise betätigt oder gar hervorgetan hat.\pend
           
\pstart
           Herzlichst grüßend{\\[\baselineskip]}Ihr \spacefill\mbox{\textcolor{gray}{A. S.}}\pend
           \leftskip=0em{}\selectlanguage{ngerman}\endnumbering\briefempfaengerindex{Hofmannsthal, Hugo von@\textsc{Hofmannsthal, Hugo von}!zzzSchnitzler, Arthur@\emph{von Arthur Schnitzler}!1917-02-191@{19. 2. 1917}|)be}\mylabel{L02256h}  \normalsize

\doendnotes{C}
\bigskip
\vfill

\clearpage

\footnotesize

\lohead{\textsc{register}}

% Definiere theindex-Environment komplett neu ohne reledmac
\makeatletter
\renewenvironment{theindex}{%
  \section*{\indexname}%
  \setlength{\parindent}{0pt}%
  \setlength{\parskip}{0pt plus 0.3pt}%
  \let\item\@idxitem
}{%
  \clearpage
}
\makeatother

\IfFileExists{\jobname-pw.ind}{\input{\jobname-pw.ind}}{}

\end{document}

      