%% latex-leseansicht-vorspann.tex
%% Vorspann für die Leseansicht.
%% Lädt die gemeinsame Datei latex-vorspann.tex mit nicht gesetztem Schalter.

\newif\ifkorrekturansicht
\korrekturansichtfalse

\input{../tex-inputs/latex-vorspann}


\section[Arthur Schnitzler an Hugo von Hofmannsthal, 19. 2. 1917]{L02256 Arthur Schnitzler an Hugo von Hofmannsthal, 19. 2. 1917}
\nopagebreak\mylabel{L02256v}
\rehead{ }\normalsize\beginnumbering\briefempfaengerindex{Hofmannsthal, Hugo von@\textsc{Hofmannsthal, Hugo von}!zzzSchnitzler, Arthur@\emph{von Arthur Schnitzler}!1917-02-191@{19. 2. 1917}|(be}
\toendnotes[C]{\smallbreak\pagebreak[2]}
\correspDesc{Versand  durch Arthur Schnitzler am 19. 2. 1917 in Wien
\newline{}Erhalt  durch Hugo von Hofmannsthal im Zeitraum [19. 2. 1917
                  – 23. 2. 1917?] in Wien}\toendnotes[C]{\smallbreak}
\buchAlsQuelle{Hugo von Hofmannsthal, Arthur Schnitzler: \emph{Briefwechsel}. Herausgegeben von Therese Nickl und Heinrich Schnitzler. Frankfurt am Main: \emph{S. Fischer} 1964, S. 281.}\toendnotes[C]{\smallbreak}
\pstart
           {\pb}\label{K_L02256-1v}\edtext{{[}Maschinenschrift{]}}{\lemma{\textnormal{\emph{Maschinenschrift}}}\Cendnote{\textnormal{Die Vorlage ist nicht
                        nachweisbar.}}}\label{K_L02256-1}\hfill 19. 2. 1917.\pend
           
\pstart{}Lieber Hugo.\pend\vspace{0.5em}
\pstart
           Der Anonymus\pwindex{Billiter, Jean 23.\,5.\,1877 Paris – 1.\,4.\,1965 Salzburg@\textsc{Billiter, Jean} (23.\,5.\,1877 Paris – 1.\,4.\,1965 Salzburg), \emph{Chemiker}|pwv}, dessen zwei
                  \label{K_L02256-2v}\edtext{Einakter}{\lemma{\textnormal{\emph{Einakter}}}\Cendnote{\textnormal{nicht ermittelt}}}\label{K_L02256-2}{ }Sie mir zurückließen, ist \label{K_L02256-3v}\edtext{gestern}{\lemma{\textnormal{\emph{gestern}}}\Cendnote{\textnormal{Vgl. A. S.: \emph{Tagebuch}, 18. 2. 1917.
               }}}\label{K_L02256-3} während ich nicht zu Hause war, bei mir erschienen, hat sich, was Ihnen kein
               Geheimnis sein dürfte, als Privatdozent Dr. Jean
                  Billiter\pwindex{Billiter, Jean 23.\,5.\,1877 Paris – 1.\,4.\,1965 Salzburg@\textsc{Billiter, Jean} (23.\,5.\,1877 Paris – 1.\,4.\,1965 Salzburg), \emph{Chemiker}|pw} entpuppt und ein drittes Stück dagelassen, das nicht besser ist als
               die zwei andern und das er sich (wie er mir auf einer Karte mitteilt) zwischen jenen
               aufgeführt denken würde. Bevor ich \label{K_L02256-4v}\edtext{ihn
               nun empfange}{\lemma{\textnormal{\emph{ihn
               nun empfange}}}\Cendnote{\textnormal{Vgl. A. S.: \emph{Tagebuch}, 20. 3. 1917.
               }}}\label{K_L02256-4} wünschte ich sehr von Ihnen zu wissen, ob Herr B.\pwindex{Billiter, Jean 23.\,5.\,1877 Paris – 1.\,4.\,1965 Salzburg@\textsc{Billiter, Jean} (23.\,5.\,1877 Paris – 1.\,4.\,1965 Salzburg), \emph{Chemiker}|pw} etwa von einer durch mich herzustellenden Verbindung mit dem
                  Burgtheater\oindex{Wien@\textbf{Wien}!I., Innere Stadt@\textbf{I., Innere Stadt}!Burgtheater@\textbf{Burgtheater}, \emph{Theater}|pw} oder sonst einer Bühne träumt und
               ob er sich vielleicht schon anderweitig literarisch oder sonstwie in einer mir nicht
               bekannt gewordenen Weise betätigt oder gar hervorgetan hat.\pend
           
\pstart
           Herzlichst grüßend{\\[\baselineskip]}Ihr \spacefill\mbox{\textcolor{gray}{A. S.}}\pend
           \leftskip=0em{}\selectlanguage{ngerman}\endnumbering\briefempfaengerindex{Hofmannsthal, Hugo von@\textsc{Hofmannsthal, Hugo von}!zzzSchnitzler, Arthur@\emph{von Arthur Schnitzler}!1917-02-191@{19. 2. 1917}|)be}\mylabel{L02256h}  \newcommand{\dateiname}{L02256}\newcommand{\titel}{Arthur Schnitzler an Hugo von Hofmannsthal, 19. 2. 1917}\newcommand{\editorInnen}{Martin Anton Müller und Gerd-Hermann Susen}%% latex-leseansicht-abspann.tex
%% Abspann für die Leseansicht.
%% Der Schalter \ifkorrekturansicht ist bereits durch den Vorspann gesetzt.

%% latex-abspann.tex
%% Gemeinsamer Abspann für Korrekturansicht und Leseansicht.
%% Setzt den Schalter \ifkorrekturansicht voraus (gesetzt in den
%% einbindenden Dateien latex-korrekturansicht-abspann.tex bzw.
%% latex-leseansicht-abspann.tex).
%% ---------------------------------------------------------------

\normalsize

% Das esempio-Environment wird nur in der Leseansicht benötigt
\ifkorrekturansicht\else
\newenvironment{esempio}[3]%
{
    \vspace{1.5ex}
    \rlap{\underline{#1}}
    \par
    \setlength{\parindent}{0cm}
    \nopagebreak
    \leftskip=#2cm
    \rightskip=#3cm
}
{
    \par
}
\fi

\doendnotes{C}
\bigskip
\vfill

\clearpage

\footnotesize

\ifkorrekturansicht
  \lohead{\textsc{register}}
\fi

% theindex-Environment neu definieren ohne reledmac
\makeatletter
\renewenvironment{theindex}{%
  \ifkorrekturansicht
    \section*{\indexname}%
  \else
    \subsubsection*{Index der erwähnten Entitäten}%
  \fi
  \setlength{\parindent}{0pt}%
  \setlength{\parskip}{0pt plus 0.3pt}%
  \let\item\@idxitem
}{%
  \ifkorrekturansicht\clearpage\fi
}
\makeatother

\IfFileExists{\jobname-pw.ind}{\input{\jobname-pw.ind}}{}

% Quellenangabe nur in der Leseansicht
\ifkorrekturansicht\else
% Fallback-Definitionen, falls die .tex-Datei \titel etc. nicht gesetzt hat
\providecommand{\titel}{}
\providecommand{\editorInnen}{}
\providecommand{\dateiname}{\jobname}

\vspace{3cm}

\vfill

\footnotesize
\textsc{Quelle}: \titel. Herausgegeben von {\editorInnen}. In: \emph{Arthur Schnitzler: Briefwechsel mit Autorinnen und Autoren}.
 Digitale Edition, https://schnitzler-briefe.acdh.oeaw.ac.at/{\dateiname}.html (Stand \today)
\fi

\end{document}


