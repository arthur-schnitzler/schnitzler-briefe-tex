%% latex-leseansicht-vorspann.tex
%% Vorspann für die Leseansicht.
%% Lädt die gemeinsame Datei latex-vorspann.tex mit nicht gesetztem Schalter.

\newif\ifkorrekturansicht
\korrekturansichtfalse

\input{../tex-inputs/latex-vorspann}


         
         \renewcommand{\erwaehntePersonen}{Personen: Karl Morré}
         \renewcommand{\erwaehnteOrte}{Orte: Café Central, Hörlgasse, Raimund-Theater, Wien}
         \renewcommand{\erwaehnteWerke}{Werke: Ein Regimentsarzt. Volksstück mit Gesang in 4 Akten}
               \section[Felix Salten an Arthur Schnitzler, {[}28. 4. 1894{]}]{ Felix Salten an Arthur Schnitzler, {[}28. 4. 1894{]}}\nopagebreak\mylabel{v}\rehead{ }\begin{ledgroupsized}[t]{13cm}\normalsize\beginnumbering \toendnotes[C]{\smallbreak\pagebreak[2]} \Standort{CUL, Schnitzler, B 89, A 1.}
\physDesc{Visitenkarte, 171 Zeichen
\newline{}Handschrift: Bleistift, lateinische Kurrent
\newline{}Schnitzler: mit Bleistift datiert: »28/4 94.« 
\newline{}Ordnung: mit Bleistift von unbekannter Hand nummeriert: »36« }\toendnotes[C]{\smallbreak}\pstart
           \noindent{}\centering{}{\pb}\textcolor{gray}{\textbf{Felix Salten}}\pend
           \pstart
           \noindent{}\raggedleft{}\textcolor{gray}{\textbf{IX. Hörlgasse 16\oindex{Hoerlgasse@\textbf{Hörlgasse}|pw}.}}\pend
           \pstart
           {\pb}Lieber Frd. Es ist \uline{nichts} mit heute{ }Abd.{ }\label{K_L03135-1v}\edtext{Regimentsarzt\pwindex{Morre, Karl 1832-11-08 – 1897-02-21@\textsc{Morré, Karl} (1832-11-08 – 1897-02-21), \emph{Schriftsteller, Politiker, Dramatiker}!Regimentsarzt. Volksstueck mit Gesang in 4 Akten1887@\strich\emph{Ein Regimentsarzt. Volksstück mit Gesang in 4 Akten} {[}1887{]}|pw}}{\lemma{\textnormal{\emph{Regimentsarzt}}}\Cendnote{\textnormal{Das vieraktige Stück \emph{Ein Regimentsarzt}\pwindex{Morre, Karl 1832-11-08 – 1897-02-21@\textsc{Morré, Karl} (1832-11-08 – 1897-02-21), \emph{Schriftsteller, Politiker, Dramatiker}!Regimentsarzt. Volksstueck mit Gesang in 4 Akten1887@\strich\emph{Ein Regimentsarzt. Volksstück mit Gesang in 4 Akten} {[}1887{]}|pwk} von Karl Morré\pwindex{Morre, Karl 1832-11-08 – 1897-02-21@\textsc{Morré, Karl} (1832-11-08 – 1897-02-21), \emph{Schriftsteller, Politiker, Dramatiker}|pwk} wurde an diesem Tag im Raimund-Theater\oindex{Raimund-Theater@\textbf{Raimund-Theater}|pwk} gespielt.}}}\label{K_L03135-1h}. Sind Sie
                  Abds im Café Central\oindex{Cafe Central@\textbf{Café Central}|pw}? Es wäre
               gut, wegen des zweifelhaften Wetters für \label{K_L03135-2v}\edtext{Morgen}{\lemma{\textnormal{\emph{Morgen}}}\Cendnote{\textnormal{siehe A. S.: \emph{Tagebuch}, 29. 4. 1894}}}\label{K_L03135-2h} etwas auszumachen.\pend
           \pstart
           Herzlichst {\\[\baselineskip]}Ihr {\\[\baselineskip]}\spacefill\mbox{Salten}\pend
           \leftskip=0em{}
         
         \endnumbering\mylabel{h}\end{ledgroupsized}  \newcommand{\dateiname}{L03135}\newcommand{\titel}{Felix Salten an Arthur Schnitzler, [28. 4. 1894]}\newcommand{\editorInnen}{Martin Anton Müller und Laura Untner}%% latex-leseansicht-abspann.tex
%% Abspann für die Leseansicht.
%% Der Schalter \ifkorrekturansicht ist bereits durch den Vorspann gesetzt.

%% latex-abspann.tex
%% Gemeinsamer Abspann für Korrekturansicht und Leseansicht.
%% Setzt den Schalter \ifkorrekturansicht voraus (gesetzt in den
%% einbindenden Dateien latex-korrekturansicht-abspann.tex bzw.
%% latex-leseansicht-abspann.tex).
%% ---------------------------------------------------------------

\normalsize

% Das esempio-Environment wird nur in der Leseansicht benötigt
\ifkorrekturansicht\else
\newenvironment{esempio}[3]%
{
    \vspace{1.5ex}
    \rlap{\underline{#1}}
    \par
    \setlength{\parindent}{0cm}
    \nopagebreak
    \leftskip=#2cm
    \rightskip=#3cm
}
{
    \par
}
\fi

\doendnotes{C}
\bigskip
\vfill

\clearpage

\footnotesize

\ifkorrekturansicht
  \lohead{\textsc{register}}
\fi

% theindex-Environment neu definieren ohne reledmac
\makeatletter
\renewenvironment{theindex}{%
  \ifkorrekturansicht
    \section*{\indexname}%
  \else
    \subsubsection*{Index der erwähnten Entitäten}%
  \fi
  \setlength{\parindent}{0pt}%
  \setlength{\parskip}{0pt plus 0.3pt}%
  \let\item\@idxitem
}{%
  \ifkorrekturansicht\clearpage\fi
}
\makeatother

\IfFileExists{\jobname-pw.ind}{\input{\jobname-pw.ind}}{}

% Quellenangabe nur in der Leseansicht
\ifkorrekturansicht\else
% Fallback-Definitionen, falls die .tex-Datei \titel etc. nicht gesetzt hat
\providecommand{\titel}{}
\providecommand{\editorInnen}{}
\providecommand{\dateiname}{\jobname}

\vspace{3cm}

\vfill

\footnotesize
\textsc{Quelle}: \titel. Herausgegeben von {\editorInnen}. In: \emph{Arthur Schnitzler: Briefwechsel mit Autorinnen und Autoren}.
 Digitale Edition, https://schnitzler-briefe.acdh.oeaw.ac.at/{\dateiname}.html (Stand \today)
\fi

\end{document}


      