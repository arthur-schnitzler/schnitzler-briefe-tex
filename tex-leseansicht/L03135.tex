%% latex-korrekturansicht-vorspann.tex
%% Vorspann für die Korrekturansicht.
%% Lädt die gemeinsame Datei latex-vorspann.tex mit gesetztem Schalter.

\newif\ifkorrekturansicht
\korrekturansichttrue

\input{../tex-inputs/latex-vorspann}


\section[Felix Salten an Arthur Schnitzler, {[}28. 4. 1894{]}]{L03135 Felix Salten an Arthur Schnitzler, {[}28. 4. 1894{]}}
\nopagebreak\mylabel{L03135v}
\rehead{ }\normalsize\beginnumbering\briefempfaengerindex{Schnitzler, Arthur@\textsc{Schnitzler, Arthur}!zzzSalten, Felix@\emph{von Felix Salten}!1894-04-282@{{[}28. 4. 1894{]}}|(be}
\toendnotes[C]{\smallbreak\pagebreak[2]}\Standort{CUL, Schnitzler, B 89, A 1.}
\physDesc{Visitenkarte, 171 Zeichen
\newline{}Handschrift: Bleistift, lateinische Kurrent
\newline{}Schnitzler: mit Bleistift datiert: »28/4 94.« 
\newline{}Ordnung: mit Bleistift von unbekannter Hand nummeriert: »36« }\toendnotes[C]{\smallbreak}
\pstart
           \centering{}{\pb}\textcolor{gray}{\textbf{Felix Salten}}\pend
           
\pstart
           \raggedleft{}\textcolor{gray}{\textbf{IX. Hörlgasse 16\oindex{Hoerlgasse 16@\textbf{Hörlgasse 16}, \emph{Wohngebäude (K.WHS)}|pw}.}}\pend
           \vspace{0.5em}
\pstart
           {\pb}Lieber Frd. Es ist \uline{nichts} mit heute{ }Abd.{ }\label{K_L03135-1v}\edtext{Regimentsarzt\pwindex{Regimentsarzt. Volksstueck mit Gesang in 4 Akten@\emph{Ein Regimentsarzt. Volksstück mit Gesang in 4 Akten}|pw}}{\lemma{\textnormal{\emph{Regimentsarzt}}}\Cendnote{\textnormal{Das vieraktige Stück \emph{Ein Regimentsarzt}\pwindex{Regimentsarzt. Volksstueck mit Gesang in 4 Akten@\emph{Ein Regimentsarzt. Volksstück mit Gesang in 4 Akten}|pwk} von Karl Morré\pwindex{Morre, Karl 1832-11-08 – 1897-02-21@\textsc{Morré, Karl} (1832-11-08 – 1897-02-21), \emph{Schriftsteller/Schriftstellerin, Politiker/Politikerin, Dramatiker/Dramatikerin}|pwk} wurde an diesem Tag im Raimund-Theater\oindex{Raimund-Theater@\textbf{Raimund-Theater}, \emph{Theater (K.THE)}|pwk} gespielt.}}}\label{K_L03135-1}. Sind Sie
                  Abds im Café Central\oindex{Cafe Central@\textbf{Café Central}, \emph{Kaffeehaus (K.KAF)}|pw}? Es wäre
               gut, wegen des zweifelhaften Wetters für \label{K_L03135-2v}\edtext{Morgen}{\lemma{\textnormal{\emph{Morgen}}}\Cendnote{\textnormal{Siehe A. S.: \emph{Tagebuch}, 29. 4. 1894.
               }}}\label{K_L03135-2} etwas auszumachen.\pend
           
\pstart
           Herzlichst {\\[\baselineskip]}Ihr {\\[\baselineskip]}\spacefill\mbox{Salten}\pend
           \leftskip=0em{}\selectlanguage{ngerman}\endnumbering\briefempfaengerindex{Schnitzler, Arthur@\textsc{Schnitzler, Arthur}!zzzSalten, Felix@\emph{von Felix Salten}!1894-04-282@{{[}28. 4. 1894{]}}|)be}\mylabel{L03135h}  \normalsize

\doendnotes{C}
\bigskip
\vfill

\clearpage

\footnotesize

\lohead{\textsc{register}}

% Definiere theindex-Environment komplett neu ohne reledmac
\makeatletter
\renewenvironment{theindex}{%
  \section*{\indexname}%
  \setlength{\parindent}{0pt}%
  \setlength{\parskip}{0pt plus 0.3pt}%
  \let\item\@idxitem
}{%
  \clearpage
}
\makeatother

\IfFileExists{\jobname-pw.ind}{\input{\jobname-pw.ind}}{}

\end{document}

      