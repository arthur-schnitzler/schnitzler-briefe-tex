%% latex-leseansicht-vorspann.tex
%% Vorspann für die Leseansicht.
%% Lädt die gemeinsame Datei latex-vorspann.tex mit nicht gesetztem Schalter.

\newif\ifkorrekturansicht
\korrekturansichtfalse

\input{../tex-inputs/latex-vorspann}


\section[Georg Brandes an Arthur Schnitzler, 28. 8. 1926]{L02478 Georg Brandes an Arthur Schnitzler, 28. 8. 1926}
\nopagebreak\mylabel{L02478v}
\rehead{ }\normalsize\beginnumbering\briefempfaengerindex{Schnitzler, Arthur@\textsc{Schnitzler, Arthur}!zzzBrandes, Georg@\emph{von Georg Brandes}!1926-08-281@{28. 8. 1926}|(be}
\toendnotes[C]{\smallbreak\pagebreak[2]}
\correspDesc{Versand  durch Georg Brandes am 28. 8. 1926 in Kopenhagen
\newline{}Erhalt  durch Arthur Schnitzler im Zeitraum [29. 8. 1926
                  – 2. 9. 1926?] in Wien}\toendnotes[C]{\smallbreak}
\Standort{CUL, Schnitzler, B 17.}
\physDesc{Postkarte, 1087 Zeichen
\newline{}Handschrift: Bleistift, lateinische Kurrent
\newline{}Versand: Stempel: »\nobreak{}\oindex{Kopenhagen@\textbf{Kopenhagen}, \emph{Hauptstadt}|pwk}Københaven, 2\textcolor{gray}{8}. VIII. 1926\nobreak{}«.  
\newline{}Schnitzler: 1) mit Bleistift datiert: »28/8«  2) mit rotem Buntstift vereinzelte Unterstreichungen
\newline{}Ordnung: mit Bleistift von unbekannter Hand nummeriert:
                                    »63« }
\buchAbdrucke{\weitereDrucke{Georg Brandes, Arthur Schnitzler: \emph{Ein Briefwechsel}. Herausgegeben von Kurt Bergel. Bern: \emph{Francke} 1956, S. 153.} }\toendnotes[C]{\smallbreak}\pstart{}{\pb}Herrn Dr. Arthur
                  Schnitzler\pend{}\pstart{}Sternwartestrasse 71\oindex{Wien@\textbf{Wien}!XVIII., Währing@\textbf{XVIII., Währing}!Sternwartestraße 71@\textbf{Sternwartestraße 71}, \emph{Wohngebäude}|pw}\pend{}\pstart{}Wien XVIII\oindex{VIII., Josefstadt@\textbf{VIII., Josefstadt}, \emph{Verwaltungsgebiet}|pw}\pend{}{\bigskip}\vspace{1em}
\pstart
           \raggedleft{}{\pb}Kopenhagen\oindex{Kopenhagen@\textbf{Kopenhagen}, \emph{Hauptstadt}|pw}{ }Goethes\pwindex{Goethe, Johann Wolfgang von 28.\,8.\,1749 Frankfurt am Main – 22.\,3.\,1832 Weimar@\textsc{Goethe, Johann Wolfgang von} (28.\,8.\,1749 Frankfurt am Main – 22.\,3.\,1832 Weimar), \emph{Schriftsteller}|pw} Geburtstag 1926\pend
           \vspace{0.5em}
\pstart
           Verehrter Freund Seit April 1925 hab ich Sie nicht
               gesehen, und es ist mir, als sah ich Sie gestern. So lebhaft stehen Sie mir vor
               Augen. Seitdem haben Sie eine weite Reise nach den canarischen Inseln\oindex{Gran Canaria@\textbf{Gran Canaria}, \emph{Insel}|pw} gemacht, sich freundlich meiner erinnert, mir die
               sonderbar tiefsinnige Traumnovelle\pwindex{Schnitzler, Arthur 15.\,5.\,1862 Wien – 21.\,10.\,1931 ebd.@\textsc{Schnitzler, Arthur} (15.\,5.\,1862 Wien – 21.\,10.\,1931 ebd.), \emph{Schriftsteller, Mediziner}!Traumnovelle@\strich\emph{Traumnovelle}|pw} zugesandt,
               vermutlich noch anderes hervorgebracht. Ich bitte nur, mich nicht zu vergessen; ich
               war in Karlsbad\oindex{Karlsbad@\textbf{Karlsbad}|pw}, Prag\oindex{Prag@\textbf{Prag}, \emph{Land}|pw}, Schandau\oindex{Bad Schandau@\textbf{Bad Schandau}|pw}, meiner Gesundheit
               halber, und bin nicht krank, arbeite weiter mit Forschungen über Petrus\pwindex{Petrus, Simon †~65–67 Rom@\textsc{Petrus, Simon} (†~65–67 Rom), \emph{Wanderprediger, Religionslehrer, Papst}|pw} u. Paulus\pwindex{Paulus vor 10? Tarsus – nach 60 Rom@\textsc{Paulus} (vor 10? Tarsus – nach 60 Rom), \emph{Wanderprediger}|pw}. Ueber
                  \uline{Petrus}\pwindex{Petrus, Simon †~65–67 Rom@\textsc{Petrus, Simon} (†~65–67 Rom), \emph{Wanderprediger, Religionslehrer, Papst}|pw} erschien vor langer Zeit ein Büchlein\pwindex{Brandes, Georg 4.\,2.\,1842 Kopenhagen – 19.\,2.\,1927 ebd.@\textsc{Brandes, Georg} (4.\,2.\,1842 Kopenhagen – 19.\,2.\,1927 ebd.)!Petrus@\strich\emph{Petrus}|pwv}, aber da mein Verleger\pwindex{Reiss, Erich 24.\,1.\,1887 Berlin – 8.\,5.\,1951 New York City@\textsc{Reiss, Erich} (24.\,1.\,1887 Berlin – 8.\,5.\,1951 New York City), \emph{Verleger}|pwv} in Berlin\oindex{Berlin@\textbf{Berlin}, \emph{Hauptstadt}|pw} bankerot ist, wurde
               es nicht deutsch publicirt.\pend
           
\pstart
           Es war schön, daß ich in Wien\oindex{Wien@\textbf{Wien}, \emph{Verwaltungsgebiet}|pw} Ihr Gast sein
               durfte. Ihre junge Tochter\pwindex{Cappellini, Lili 13.\,9.\,1909 Wien – 26.\,7.\,1928 Venedig@\textsc{Cappellini, Lili} (13.\,9.\,1909 Wien – 26.\,7.\,1928 Venedig)|pwv}
               war {\pb}war Schmuck des Hauses.\pend
           
\pstart
           Ich bitte, gelegentlich Beer-Hofmann\pwindex{Beer-Hofmann, Richard 11.\,7.\,1866 Wien – 26.\,9.\,1945 New York City@\textsc{Beer-Hofmann, Richard} (11.\,7.\,1866 Wien – 26.\,9.\,1945 New York City), \emph{Schriftsteller}|pw} und seine
                  Gemahlin\pwindex{Beer-Hofmann, Paula 25.\,2.\,1879 Wien – 30.\,10.\,1939 Zürich@\textsc{Beer-Hofmann, Paula} (25.\,2.\,1879 Wien – 30.\,10.\,1939 Zürich)|pwv} sehr herzlich
               von mir zu grüssen.\pend
           
\pstart
           Ich weiss nicht, ob Sie Zeit zum Lesen haben. Sonst würde ich Ihnen Kyra Kyralina\pwindex{Istrati, Panaït 22.\,8.\,1884 Brăila – 16.\,4.\,1935 Bukarest@\textsc{Istrati, Panaït} (22.\,8.\,1884 Brăila – 16.\,4.\,1935 Bukarest), \emph{Schriftsteller}!Kyra Kyralina@\strich\emph{Kyra Kyralina}|pw} von dem Rumänen\oindex{Rumänien@\textbf{Rumänien}|pw}{ }Panit Istrati\pwindex{Istrati, Panaït 22.\,8.\,1884 Brăila – 16.\,4.\,1935 Bukarest@\textsc{Istrati, Panaït} (22.\,8.\,1884 Brăila – 16.\,4.\,1935 Bukarest), \emph{Schriftsteller}|pw} empfehlen. Er schreibt
               französisch und hat grosse Frische.\pend
           \pstart Ihr getreuer Freund \spacefill\mbox{Georg B}\pend{}\selectlanguage{ngerman}\endnumbering\briefempfaengerindex{Schnitzler, Arthur@\textsc{Schnitzler, Arthur}!zzzBrandes, Georg@\emph{von Georg Brandes}!1926-08-281@{28. 8. 1926}|)be}\mylabel{L02478h}  \newcommand{\dateiname}{L02478}\newcommand{\titel}{Georg Brandes an Arthur Schnitzler, 28. 8. 1926}\newcommand{\editorInnen}{Martin Anton Müller und Gerd-Hermann Susen}%% latex-leseansicht-abspann.tex
%% Abspann für die Leseansicht.
%% Der Schalter \ifkorrekturansicht ist bereits durch den Vorspann gesetzt.

%% latex-abspann.tex
%% Gemeinsamer Abspann für Korrekturansicht und Leseansicht.
%% Setzt den Schalter \ifkorrekturansicht voraus (gesetzt in den
%% einbindenden Dateien latex-korrekturansicht-abspann.tex bzw.
%% latex-leseansicht-abspann.tex).
%% ---------------------------------------------------------------

\normalsize

% Das esempio-Environment wird nur in der Leseansicht benötigt
\ifkorrekturansicht\else
\newenvironment{esempio}[3]%
{
    \vspace{1.5ex}
    \rlap{\underline{#1}}
    \par
    \setlength{\parindent}{0cm}
    \nopagebreak
    \leftskip=#2cm
    \rightskip=#3cm
}
{
    \par
}
\fi

\doendnotes{C}
\bigskip
\vfill

\clearpage

\footnotesize

\ifkorrekturansicht
  \lohead{\textsc{register}}
\fi

% theindex-Environment neu definieren ohne reledmac
\makeatletter
\renewenvironment{theindex}{%
  \ifkorrekturansicht
    \section*{\indexname}%
  \else
    \subsubsection*{Index der erwähnten Entitäten}%
  \fi
  \setlength{\parindent}{0pt}%
  \setlength{\parskip}{0pt plus 0.3pt}%
  \let\item\@idxitem
}{%
  \ifkorrekturansicht\clearpage\fi
}
\makeatother

\IfFileExists{\jobname-pw.ind}{\input{\jobname-pw.ind}}{}

% Quellenangabe nur in der Leseansicht
\ifkorrekturansicht\else
% Fallback-Definitionen, falls die .tex-Datei \titel etc. nicht gesetzt hat
\providecommand{\titel}{}
\providecommand{\editorInnen}{}
\providecommand{\dateiname}{\jobname}

\vspace{3cm}

\vfill

\footnotesize
\textsc{Quelle}: \titel. Herausgegeben von {\editorInnen}. In: \emph{Arthur Schnitzler: Briefwechsel mit Autorinnen und Autoren}.
 Digitale Edition, https://schnitzler-briefe.acdh.oeaw.ac.at/{\dateiname}.html (Stand \today)
\fi

\end{document}


