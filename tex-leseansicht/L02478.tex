%% latex-leseansicht-vorspann.tex
%% Vorspann für die Leseansicht.
%% Lädt die gemeinsame Datei latex-vorspann.tex mit nicht gesetztem Schalter.

\newif\ifkorrekturansicht
\korrekturansichtfalse

\input{../tex-inputs/latex-vorspann}


         
         \renewcommand{\erwaehntePersonen}{Personen: Richard Beer-Hofmann, Paula Beer-Hofmann, Georg Brandes, Johann Wolfgang von Goethe, Panaït Istrati,  Paulus, Simon Petrus, Erich Reiss, Lili Schnitzler}
         \renewcommand{\erwaehnteOrte}{Orte: Bad Schandau, Berlin, Gran Canaria, Karlsbad, Kopenhagen, Prag, Rumänien, Sternwartestraße, VIII., Josefstadt, Wien}
         \renewcommand{\erwaehnteWerke}{Werke: Kyra Kyralina, Petrus, Traumnovelle}
               \section[Georg Brandes an Arthur Schnitzler, 28. 8. 1926]{ Georg Brandes an Arthur Schnitzler, 28. 8. 1926}\nopagebreak\mylabel{v}\rehead{ }\begin{ledgroupsized}[t]{13cm}\normalsize\beginnumbering \toendnotes[C]{\smallbreak\pagebreak[2]} \Standort{CUL, Schnitzler, B 17.}
\physDesc{Postkarte, 1087 Zeichen
\newline{}Handschrift: Bleistift, lateinische Kurrent
\newline{}Versand: Stempel: »\nobreak{}\oindex{Kopenhagen@\textbf{Kopenhagen}|pwk}Københaven, 2\textcolor{gray}{8}. VIII. 1926\nobreak{}«.  
\newline{}Schnitzler: 1) mit Bleistift datiert: »28/8«  2) mit rotem Buntstift vereinzelte Unterstreichungen
\newline{}Ordnung: mit Bleistift von unbekannter Hand nummeriert:
                                    »63« }\buchAbdrucke{\weitereDrucke{Georg Brandes, Arthur Schnitzler: \emph{Ein Briefwechsel}. Hg. Kurt Bergel. Bern: \emph{Francke} 1956, S. 153.} }\toendnotes[C]{\smallbreak}\pstart{}{\pb}Herrn Dr. Arthur
                  Schnitzler\pend{}\pstart{}Sternwartestrasse 71\oindex{XXXX Ortsangabe fehlt|pw}\pend{}\pstart{}Wien XVIII\oindex{VIII., Josefstadt@\textbf{VIII., Josefstadt}|pw}\pend{}{\bigskip}\pstart
           \raggedleft{}{\pb}Kopenhagen\oindex{Kopenhagen@\textbf{Kopenhagen}|pw}{ }Goethe\pwindex{Goethe, Johann Wolfgang von 1749-08-28 – 1832-03-22@\textsc{Goethe, Johann Wolfgang von} (1749-08-28 – 1832-03-22), \emph{Schriftsteller}|pw}s Geburtstag 1926 \pend
           \pstart
           Verehrter Freund Seit April 1925 hab ich Sie nicht
               gesehen, und es ist mir, als sah ich Sie gestern. So lebhaft stehen Sie mir vor
               Augen. Seitdem haben Sie eine weite Reise nach den canarischen Inseln\oindex{Gran Canaria@\textbf{Gran Canaria}|pw} gemacht, sich freundlich meiner erinnert, mir die
               sonderbar tiefsinnige Traumnovelle\pwindex{Schnitzler, Arthur 15.05.1862 – 21.10.1931@\textsc{Schnitzler, Arthur} (15.05.1862 – 21.10.1931), \emph{Schriftsteller, Mediziner}!Traumnovelle1.12.1925 – 1.3.1926@\strich\emph{Traumnovelle} {[}1.12.1925 – 1.3.1926{]}|pw} zugesandt,
               vermutlich noch anderes hervorgebracht. Ich bitte nur, mich nicht zu vergessen; ich
               war in Karlsbad\oindex{Karlsbad@\textbf{Karlsbad}|pw}, Prag\oindex{Prag@\textbf{Prag}|pw}, Schandau\oindex{Bad Schandau@\textbf{Bad Schandau}|pw}, meiner Gesundheit
               halber, und bin nicht krank, arbeite weiter mit Forschungen über Petrus\pwindex{Petrus, Simon †~65–67@\textsc{Petrus, Simon} (†~65–67), \emph{Prediger}|pw} u. Paulus\pwindex{Paulus vor 10? – nach 60@\textsc{Paulus} (vor 10? – nach 60), \emph{Prediger}|pw}. Ueber
                  \uline{Petrus}\pwindex{Petrus, Simon †~65–67@\textsc{Petrus, Simon} (†~65–67), \emph{Prediger}|pw} erschien vor langer Zeit ein Büchlein\pwindex{Brandes, Georg 04.02.1842 – 19.02.1927@\textsc{Brandes, Georg} (04.02.1842 – 19.02.1927)!Petrus1926@\strich\emph{Petrus} {[}1926{]}|pwv}, aber da mein Verleger\pwindex{Reiss, Erich 24.01.1887 – 08.05.1951@\textsc{Reiss, Erich} (24.01.1887 – 08.05.1951), \emph{Verleger}|pwv} in Berlin\oindex{Berlin@\textbf{Berlin}|pw} bankerot ist, wurde
               es nicht deutsch publicirt.\pend
           \pstart
           Es war schön, daß ich in Wien\oindex{Wien@\textbf{Wien}|pw} Ihr Gast sein
               durfte. Ihre junge Tochter\pwindex{Schnitzler, Lili 13.09.1909 – 26.07.1928@\textsc{Schnitzler, Lili} (13.09.1909 – 26.07.1928)|pwv}
               war {\pb}war Schmuck des Hauses.\pend
           \pstart
           Ich bitte, gelegentlich Beer-Hofmann\pwindex{Beer-Hofmann, Richard 1866-07-11 – 1945-09-26@\textsc{Beer-Hofmann, Richard} (1866-07-11 – 1945-09-26), \emph{Schriftsteller}|pw} und seine
                  Gemahlin\pwindex{Beer-Hofmann, Paula 25.02.1879 – 30.10.1939@\textsc{Beer-Hofmann, Paula} (25.02.1879 – 30.10.1939)|pwv} sehr herzlich
               von mir zu grüssen.\pend
           \pstart
           Ich weiss nicht, ob Sie Zeit zum Lesen haben. Sonst würde ich Ihnen Kyra Kyralina\pwindex{Istrati, Panaït 22.08.1884 – 16.04.1935@\textsc{Istrati, Panaït} (22.08.1884 – 16.04.1935), \emph{Schriftsteller}!Kyra Kyralina1923@\strich\emph{Kyra Kyralina} {[}1923{]}|pw} von dem Rumänen\oindex{Rumaenien@\textbf{Rumänien}|pw}{ }Panit Istrati\pwindex{Istrati, Panaït 22.08.1884 – 16.04.1935@\textsc{Istrati, Panaït} (22.08.1884 – 16.04.1935), \emph{Schriftsteller}|pw} empfehlen. Er schreibt
               französisch und hat grosse Frische.\pend
           \pstart Ihr getreuer Freund \spacefill\mbox{Georg B}\pend{}
         
         \endnumbering\mylabel{h}\end{ledgroupsized}  \newcommand{\dateiname}{L02478}\newcommand{\titel}{Georg Brandes an Arthur Schnitzler, 28. 8. 1926}\newcommand{\editorInnen}{Martin Anton Müller und Gerd-Hermann Susen}%% latex-leseansicht-abspann.tex
%% Abspann für die Leseansicht.
%% Der Schalter \ifkorrekturansicht ist bereits durch den Vorspann gesetzt.

%% latex-abspann.tex
%% Gemeinsamer Abspann für Korrekturansicht und Leseansicht.
%% Setzt den Schalter \ifkorrekturansicht voraus (gesetzt in den
%% einbindenden Dateien latex-korrekturansicht-abspann.tex bzw.
%% latex-leseansicht-abspann.tex).
%% ---------------------------------------------------------------

\normalsize

% Das esempio-Environment wird nur in der Leseansicht benötigt
\ifkorrekturansicht\else
\newenvironment{esempio}[3]%
{
    \vspace{1.5ex}
    \rlap{\underline{#1}}
    \par
    \setlength{\parindent}{0cm}
    \nopagebreak
    \leftskip=#2cm
    \rightskip=#3cm
}
{
    \par
}
\fi

\doendnotes{C}
\bigskip
\vfill

\clearpage

\footnotesize

\ifkorrekturansicht
  \lohead{\textsc{register}}
\fi

% theindex-Environment neu definieren ohne reledmac
\makeatletter
\renewenvironment{theindex}{%
  \ifkorrekturansicht
    \section*{\indexname}%
  \else
    \subsubsection*{Index der erwähnten Entitäten}%
  \fi
  \setlength{\parindent}{0pt}%
  \setlength{\parskip}{0pt plus 0.3pt}%
  \let\item\@idxitem
}{%
  \ifkorrekturansicht\clearpage\fi
}
\makeatother

\IfFileExists{\jobname-pw.ind}{\input{\jobname-pw.ind}}{}

% Quellenangabe nur in der Leseansicht
\ifkorrekturansicht\else
% Fallback-Definitionen, falls die .tex-Datei \titel etc. nicht gesetzt hat
\providecommand{\titel}{}
\providecommand{\editorInnen}{}
\providecommand{\dateiname}{\jobname}

\vspace{3cm}

\vfill

\footnotesize
\textsc{Quelle}: \titel. Herausgegeben von {\editorInnen}. In: \emph{Arthur Schnitzler: Briefwechsel mit Autorinnen und Autoren}.
 Digitale Edition, https://schnitzler-briefe.acdh.oeaw.ac.at/{\dateiname}.html (Stand \today)
\fi

\end{document}


      