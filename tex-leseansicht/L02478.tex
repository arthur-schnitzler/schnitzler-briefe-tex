%% latex-korrekturansicht-vorspann.tex
%% Vorspann für die Korrekturansicht.
%% Lädt die gemeinsame Datei latex-vorspann.tex mit gesetztem Schalter.

\newif\ifkorrekturansicht
\korrekturansichttrue

\input{../tex-inputs/latex-vorspann}


\section[Georg Brandes an Arthur Schnitzler, 28. 8. 1926]{L02478 Georg Brandes an Arthur Schnitzler, 28. 8. 1926}
\nopagebreak\mylabel{L02478v}
\rehead{ }\normalsize\beginnumbering\briefempfaengerindex{Schnitzler, Arthur@\textsc{Schnitzler, Arthur}!zzzBrandes, Georg@\emph{von Georg Brandes}!1926-08-281@{28. 8. 1926}|(be}
\toendnotes[C]{\smallbreak\pagebreak[2]}\Standort{CUL, Schnitzler, B 17.}
\physDesc{Postkarte, 1087 Zeichen
\newline{}Handschrift: Bleistift, lateinische Kurrent
\newline{}Versand: Stempel: »\nobreak{}\oindex{Kopenhagen@\textbf{Kopenhagen}, \emph{P.PPLC}|pwk}Københaven, 2\textcolor{gray}{8}. VIII. 1926\nobreak{}«.  
\newline{}Schnitzler: 1) mit Bleistift datiert: »28/8«  2) mit rotem Buntstift vereinzelte Unterstreichungen
\newline{}Ordnung: mit Bleistift von unbekannter Hand nummeriert:
                                    »63« }
\buchAbdrucke{\weitereDrucke{Georg Brandes, Arthur Schnitzler: \emph{Ein Briefwechsel}. Bern: \emph{Francke} 1956, S. 153.} }\toendnotes[C]{\smallbreak}\pstart{}{\pb}Herrn Dr. Arthur
                  Schnitzler\pend{}\pstart{}Sternwartestrasse 71\oindex{Sternwartestrasse 71@\textbf{Sternwartestraße 71}, \emph{Wohngebäude (K.WHS)}|pw}\pend{}\pstart{}Wien XVIII\oindex{VIII., Josefstadt@\textbf{VIII., Josefstadt}, \emph{A.ADM3}|pw}\pend{}{\bigskip}\vspace{1em}
\pstart
           \raggedleft{}{\pb}Kopenhagen\oindex{Kopenhagen@\textbf{Kopenhagen}, \emph{P.PPLC}|pw}{ }Goethes\pwindex{Goethe, Johann Wolfgang von 1749-08-28 – 1832-03-22@\textsc{Goethe, Johann Wolfgang von} (1749-08-28 – 1832-03-22), \emph{Schriftsteller/Schriftstellerin}|pw} Geburtstag 1926 \pend
           \vspace{0.5em}
\pstart
           Verehrter Freund Seit April 1925 hab ich Sie nicht
               gesehen, und es ist mir, als sah ich Sie gestern. So lebhaft stehen Sie mir vor
               Augen. Seitdem haben Sie eine weite Reise nach den canarischen Inseln\oindex{Gran Canaria@\textbf{Gran Canaria}, \emph{T.ISL}|pw} gemacht, sich freundlich meiner erinnert, mir die
               sonderbar tiefsinnige Traumnovelle\pwindex{Traumnovelle@\emph{Traumnovelle}|pw} zugesandt,
               vermutlich noch anderes hervorgebracht. Ich bitte nur, mich nicht zu vergessen; ich
               war in Karlsbad\oindex{Karlsbad@\textbf{Karlsbad}, \emph{P.PPLA}|pw}, Prag\oindex{Prag@\textbf{Prag}, \emph{A.ADM1}|pw}, Schandau\oindex{Bad Schandau@\textbf{Bad Schandau}, \emph{P.PPL}|pw}, meiner Gesundheit
               halber, und bin nicht krank, arbeite weiter mit Forschungen über Petrus\pwindex{Petrus, Simon †~65–67@\textsc{Petrus, Simon} (†~65–67), \emph{Wanderprediger/Wanderpredigerin, Religionslehrer/Religionslehrerin, Papst/Päpstin}|pw} u. Paulus\pwindex{Paulus vor 10? – nach 60@\textsc{Paulus} (vor 10? – nach 60), \emph{Wanderprediger/Wanderpredigerin}|pw}. Ueber
                  \uline{Petrus}\pwindex{Petrus, Simon †~65–67@\textsc{Petrus, Simon} (†~65–67), \emph{Wanderprediger/Wanderpredigerin, Religionslehrer/Religionslehrerin, Papst/Päpstin}|pw} erschien vor langer Zeit ein Büchlein\pwindex{Petrus@\emph{Petrus}|pwv}, aber da mein Verleger\pwindex{Reiss, Erich 24.01.1887 – 08.05.1951@\textsc{Reiss, Erich} (24.01.1887 – 08.05.1951), \emph{Verleger/Verlegerin}|pwv} in Berlin\oindex{Berlin@\textbf{Berlin}, \emph{P.PPLC}|pw} bankerot ist, wurde
               es nicht deutsch publicirt.\pend
           
\pstart
           Es war schön, daß ich in Wien\oindex{Wien@\textbf{Wien}, \emph{A.ADM2}|pw} Ihr Gast sein
               durfte. Ihre junge Tochter\pwindex{Cappellini, Lili 13.09.1909 – 26.07.1928@\textsc{Cappellini, Lili} (13.09.1909 – 26.07.1928)|pwv}
               war {\pb}war Schmuck des Hauses.\pend
           
\pstart
           Ich bitte, gelegentlich Beer-Hofmann\pwindex{Beer-Hofmann, Richard 1866-07-11 – 1945-09-26@\textsc{Beer-Hofmann, Richard} (1866-07-11 – 1945-09-26), \emph{Schriftsteller/Schriftstellerin}|pw} und seine
                  Gemahlin\pwindex{Beer-Hofmann, Paula 25.02.1879 – 30.10.1939@\textsc{Beer-Hofmann, Paula} (25.02.1879 – 30.10.1939)|pwv} sehr herzlich
               von mir zu grüssen.\pend
           
\pstart
           Ich weiss nicht, ob Sie Zeit zum Lesen haben. Sonst würde ich Ihnen Kyra Kyralina\pwindex{Kyra Kyralina@\emph{Kyra Kyralina}|pw} von dem Rumänen\oindex{Rumaenien@\textbf{Rumänien}, \emph{A.PCLI}|pw}{ }Panit Istrati\pwindex{Istrati, Panaït 22.08.1884 – 16.04.1935@\textsc{Istrati, Panaït} (22.08.1884 – 16.04.1935), \emph{Schriftsteller/Schriftstellerin}|pw} empfehlen. Er schreibt
               französisch und hat grosse Frische.\pend
           \pstart Ihr getreuer Freund \spacefill\mbox{Georg B}\pend{}\selectlanguage{ngerman}\endnumbering\briefempfaengerindex{Schnitzler, Arthur@\textsc{Schnitzler, Arthur}!zzzBrandes, Georg@\emph{von Georg Brandes}!1926-08-281@{28. 8. 1926}|)be}\mylabel{L02478h}  \normalsize

\doendnotes{C}
\bigskip
\vfill

\clearpage

\footnotesize

\lohead{\textsc{register}}

% Definiere theindex-Environment komplett neu ohne reledmac
\makeatletter
\renewenvironment{theindex}{%
  \section*{\indexname}%
  \setlength{\parindent}{0pt}%
  \setlength{\parskip}{0pt plus 0.3pt}%
  \let\item\@idxitem
}{%
  \clearpage
}
\makeatother

\IfFileExists{\jobname-pw.ind}{\input{\jobname-pw.ind}}{}

\end{document}

      