%% latex-leseansicht-vorspann.tex
%% Vorspann für die Leseansicht.
%% Lädt die gemeinsame Datei latex-vorspann.tex mit nicht gesetztem Schalter.

\newif\ifkorrekturansicht
\korrekturansichtfalse

\input{../tex-inputs/latex-vorspann}


         
         \renewcommand{\erwaehntePersonen}{Personen: Georg Brandes, Johann Wolfgang von Goethe, William Shakespeare}
         \renewcommand{\erwaehnteOrte}{Orte: Kopenhagen, Polen, Sternwartestraße, Wien}
         \renewcommand{\erwaehnteWerke}{Werke: Tilstande i russisk Polen, William Shakespeare, Wolfgang Goethe}
               \section[Arthur Schnitzler an Georg Brandes, 8. 1. 1915]{ Arthur Schnitzler an Georg Brandes, 8. 1. 1915}\nopagebreak\mylabel{v}\rehead{ }\begin{ledgroupsized}[t]{13cm}\normalsize\beginnumbering \toendnotes[C]{\smallbreak\pagebreak[2]} \Standort{Kopenhagen, Det Kongelige Bibliotek, Georg Brandes Arkiv, box 125.}
\physDesc{Postkarte, 1043 Zeichen
\newline{}Handschrift: schwarze Tinte, lateinische Kurrent
\newline{}Versand: 1) Stempel: »\nobreak{}W{[}i{]}en
                                       110, 15 XII {[}1915{]}\nobreak{}«.   2) Stempel: »\nobreak{}Überprüft\nobreak{}«. }\buchAbdrucke{\weitereDrucke{Georg Brandes, Arthur Schnitzler: \emph{Ein Briefwechsel}. Hg. Kurt Bergel. Bern: \emph{Francke} 1956, S. 114.} }\toendnotes[C]{\smallbreak}\pstart{}{\pb}Herrn\pend{}\pstart{}\textsc{Georg Brandes}\pend{}\pstart{}\textsc{Kopenhagen}\oindex{Kopenhagen@\textbf{Kopenhagen}|pw}\pend{}{\bigskip}\pstart
           \noindent{}{\pb}\textcolor{gray}{\textbf{Dr. Arthur Schnitzler}}{\\}\textcolor{gray}{\textbf{Wien XVIII. Sternwartestrasse 71\oindex{XXXX Ortsangabe fehlt|pw}}}\pend
           \pstart
           \raggedleft{}8. 1. 15\pend
           \pstart
           verehrter lieber Freund, ich danke Ihren für Ihre Karte und freue
               mich auf Ihr Goethe\pwindex{Goethe, Johann Wolfgang von 1749-08-28 – 1832-03-22@\textsc{Goethe, Johann Wolfgang von} (1749-08-28 – 1832-03-22), \emph{Schriftsteller}|pw}buch\pwindex{Brandes, Georg 04.02.1842 – 19.02.1927@\textsc{Brandes, Georg} (04.02.1842 – 19.02.1927)!Wolfgang Goethe1915@\strich\emph{Wolfgang Goethe} {[}1915{]}|pwv}. Mit welcher Ergriffenheit
               denk ich noch heute Ihres Shakespeare\pwindex{Shakespeare, William 23.4.1564? – 03.05.1616@\textsc{Shakespeare, William} (23.4.1564? – 03.05.1616), \emph{Schauspieler, Schriftsteller}|pw}\pwindex{Brandes, Georg 04.02.1842 – 19.02.1927@\textsc{Brandes, Georg} (04.02.1842 – 19.02.1927)!William Shakespeare1895 – 1896@\strich\emph{William Shakespeare} {[}1895 – 1896{]}|pwv} – des Schlusses besonders – in dem Sie – so schien mir damals – Ihr
               Allereigenstes – viel selbstdurchlittenes hineingeheimnist hatten!\pend
           \pstart
           – Auch ich versuche meinen Kopf aus diese düster-wirren Zeit in phantastischere Lüfte
               emporzustecken; aber es gelingt nicht immer, uns rühren gar zu viele Wirbel an; man
               sieht, hört zu vieles, spricht mit Heimgekehrten, Hinausziehenden, – möchte irgendwie
               das seine thun – wärs auch nur für spätre Zeiten;– aber solange die Politik noch
               nicht Geschichte \strikeout{ist} geworden ist, ist der Blick
               nicht h\damage{el}l {\pb}genug. – Von Ihren letzten Artikeln
               ist mir nur ein\pwindex{Brandes, Georg 04.02.1842 – 19.02.1927@\textsc{Brandes, Georg} (04.02.1842 – 19.02.1927)!Tilstande i russisk Polen25. 10. 1914@\strich\emph{Tilstande i russisk Polen} {[}25. 10. 1914{]}|pwv} erschütternder
               über die Juden in Polen\oindex{Polen@\textbf{Polen}|pw} vor Augen geko{\geminationm}en. Ich wünsche Ihnen zum neuen Jahr weitre
               Arbeitsfreudigkeit, und für Ihre Lieben alles gute – und für uns alle eine bessre
               Zeit der Gerechtigkeit, der Einsicht, des Friedens! Wir grüßen Sie von Herzen! Ihr
                  \spacefill\mbox{Arthur Schnitzler}\pend
           
         
         \endnumbering\mylabel{h}\end{ledgroupsized}  \newcommand{\dateiname}{L02202}\newcommand{\titel}{Arthur Schnitzler an Georg Brandes, 8. 1. 1915}\newcommand{\editorInnen}{Martin Anton Müller und Gerd-Hermann Susen}%% latex-leseansicht-abspann.tex
%% Abspann für die Leseansicht.
%% Der Schalter \ifkorrekturansicht ist bereits durch den Vorspann gesetzt.

%% latex-abspann.tex
%% Gemeinsamer Abspann für Korrekturansicht und Leseansicht.
%% Setzt den Schalter \ifkorrekturansicht voraus (gesetzt in den
%% einbindenden Dateien latex-korrekturansicht-abspann.tex bzw.
%% latex-leseansicht-abspann.tex).
%% ---------------------------------------------------------------

\normalsize

% Das esempio-Environment wird nur in der Leseansicht benötigt
\ifkorrekturansicht\else
\newenvironment{esempio}[3]%
{
    \vspace{1.5ex}
    \rlap{\underline{#1}}
    \par
    \setlength{\parindent}{0cm}
    \nopagebreak
    \leftskip=#2cm
    \rightskip=#3cm
}
{
    \par
}
\fi

\doendnotes{C}
\bigskip
\vfill

\clearpage

\footnotesize

\ifkorrekturansicht
  \lohead{\textsc{register}}
\fi

% theindex-Environment neu definieren ohne reledmac
\makeatletter
\renewenvironment{theindex}{%
  \ifkorrekturansicht
    \section*{\indexname}%
  \else
    \subsubsection*{Index der erwähnten Entitäten}%
  \fi
  \setlength{\parindent}{0pt}%
  \setlength{\parskip}{0pt plus 0.3pt}%
  \let\item\@idxitem
}{%
  \ifkorrekturansicht\clearpage\fi
}
\makeatother

\IfFileExists{\jobname-pw.ind}{\input{\jobname-pw.ind}}{}

% Quellenangabe nur in der Leseansicht
\ifkorrekturansicht\else
% Fallback-Definitionen, falls die .tex-Datei \titel etc. nicht gesetzt hat
\providecommand{\titel}{}
\providecommand{\editorInnen}{}
\providecommand{\dateiname}{\jobname}

\vspace{3cm}

\vfill

\footnotesize
\textsc{Quelle}: \titel. Herausgegeben von {\editorInnen}. In: \emph{Arthur Schnitzler: Briefwechsel mit Autorinnen und Autoren}.
 Digitale Edition, https://schnitzler-briefe.acdh.oeaw.ac.at/{\dateiname}.html (Stand \today)
\fi

\end{document}


      