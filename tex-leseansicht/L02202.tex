%% latex-korrekturansicht-vorspann.tex
%% Vorspann für die Korrekturansicht.
%% Lädt die gemeinsame Datei latex-vorspann.tex mit gesetztem Schalter.

\newif\ifkorrekturansicht
\korrekturansichttrue

\input{../tex-inputs/latex-vorspann}


\section[Arthur Schnitzler an Georg Brandes, 8. 1. 1915]{L02202 Arthur Schnitzler an Georg Brandes, 8. 1. 1915}
\nopagebreak\mylabel{L02202v}
\rehead{ }\normalsize\beginnumbering\briefempfaengerindex{Brandes, Georg@\textsc{Brandes, Georg}!zzzSchnitzler, Arthur@\emph{von Arthur Schnitzler}!1915-01-081@{8. 1. 1915}|(be}
\toendnotes[C]{\smallbreak\pagebreak[2]}\Standort{Kopenhagen, Det Kongelige Bibliotek, Georg Brandes Arkiv, box 125.}
\physDesc{Postkarte, 1043 Zeichen
\newline{}Handschrift: schwarze Tinte, lateinische Kurrent
\newline{}Versand: 1) Stempel: »\nobreak{}W{[}i{]}en
                                       110, 15 XII {[}1915{]}\nobreak{}«.   2) Stempel: »\nobreak{}Überprüft\nobreak{}«. }
\buchAbdrucke{\weitereDrucke{Georg Brandes, Arthur Schnitzler: \emph{Ein Briefwechsel}. Bern: \emph{Francke} 1956, S. 114.} }\toendnotes[C]{\smallbreak}\pstart{}{\pb}Herrn\pend{}\pstart{}\textsc{Georg Brandes}\pend{}\pstart{}\textsc{Kopenhagen}\oindex{Kopenhagen@\textbf{Kopenhagen}, \emph{P.PPLC}|pw}\pend{}{\bigskip}\vspace{1em}
\pstart
           {\pb}\textcolor{gray}{\textbf{Dr. Arthur Schnitzler}}{\\}\textcolor{gray}{\textbf{Wien XVIII. Sternwartestrasse 71\oindex{Sternwartestrasse 71@\textbf{Sternwartestraße 71}, \emph{Wohngebäude (K.WHS)}|pw}}}\pend
           
\pstart
           \raggedleft{}8. 1. 15\pend
           \vspace{0.5em}
\pstart
           verehrter lieber Freund, ich danke Ihren für Ihre Karte und freue
               mich auf Ihr Goethe\pwindex{Goethe, Johann Wolfgang von 1749-08-28 – 1832-03-22@\textsc{Goethe, Johann Wolfgang von} (1749-08-28 – 1832-03-22), \emph{Schriftsteller/Schriftstellerin}|pw}buch\pwindex{Wolfgang Goethe@\emph{Wolfgang Goethe}|pwv}. Mit welcher Ergriffenheit
               denk ich noch heute Ihres Shakespeare\pwindex{Shakespeare, William 23.4.1564? – 03.05.1616@\textsc{Shakespeare, William} (23.4.1564? – 03.05.1616), \emph{Schauspieler/Schauspielerin, Dramatiker/Dramatikerin}|pw}\pwindex{William Shakespeare@\emph{William Shakespeare}|pwv} – des Schlusses besonders – in dem Sie – so schien mir damals – Ihr
               Allereigenstes – viel selbstdurchlittenes hineingeheimnist hatten!\pend
           
\pstart
           – Auch ich versuche meinen Kopf aus diese düster-wirren Zeit in phantastischere Lüfte
               emporzustecken; aber es gelingt nicht immer, uns rühren gar zu viele Wirbel an; man
               sieht, hört zu vieles, spricht mit Heimgekehrten, Hinausziehenden, – möchte irgendwie
               das seine thun – wärs auch nur für spätre Zeiten;– aber solange die Politik noch
               nicht Geschichte \strikeout{ist} geworden ist, ist der Blick
               nicht h\damage{el}l {\pb}genug. – Von Ihren letzten Artikeln
               ist mir nur ein\pwindex{Tilstande i russisk Polen@\emph{Tilstande i russisk Polen}|pwv} erschütternder
               über die Juden in Polen\oindex{Polen@\textbf{Polen}, \emph{A.PCLI}|pw} vor Augen geko{\geminationm}en. Ich wünsche Ihnen zum neuen Jahr weitre
               Arbeitsfreudigkeit, und für Ihre Lieben alles gute – und für uns alle eine bessre
               Zeit der Gerechtigkeit, der Einsicht, des Friedens! Wir grüßen Sie von Herzen! Ihr
                  \spacefill\mbox{Arthur Schnitzler}\pend
           \selectlanguage{ngerman}\endnumbering\briefempfaengerindex{Brandes, Georg@\textsc{Brandes, Georg}!zzzSchnitzler, Arthur@\emph{von Arthur Schnitzler}!1915-01-081@{8. 1. 1915}|)be}\mylabel{L02202h}  \normalsize

\doendnotes{C}
\bigskip
\vfill

\clearpage

\footnotesize

\lohead{\textsc{register}}

% Definiere theindex-Environment komplett neu ohne reledmac
\makeatletter
\renewenvironment{theindex}{%
  \section*{\indexname}%
  \setlength{\parindent}{0pt}%
  \setlength{\parskip}{0pt plus 0.3pt}%
  \let\item\@idxitem
}{%
  \clearpage
}
\makeatother

\IfFileExists{\jobname-pw.ind}{\input{\jobname-pw.ind}}{}

\end{document}

      