%% latex-leseansicht-vorspann.tex
%% Vorspann für die Leseansicht.
%% Lädt die gemeinsame Datei latex-vorspann.tex mit nicht gesetztem Schalter.

\newif\ifkorrekturansicht
\korrekturansichtfalse

\input{../tex-inputs/latex-vorspann}


\section[Hugo von Hofmannsthal an Arthur Schnitzler, 18. 9. [1907]]{L01708 Hugo von Hofmannsthal an Arthur Schnitzler, 18. 9. [1907]}
\nopagebreak\mylabel{L01708v}
\rehead{ }\normalsize\beginnumbering\briefempfaengerindex{Schnitzler, Arthur@\textsc{Schnitzler, Arthur}!zzzHofmannsthal, Hugo von@\emph{von Hugo von Hofmannsthal}!1907-09-181@{{[}18. 9. 1907{]}}|(be}
\toendnotes[C]{\smallbreak\pagebreak[2]}
\correspDesc{Versand  durch Hugo von Hofmannsthal am [18. 9. 1907] in Bad Aussee
\newline{}Erhalt  durch Arthur Schnitzler im Zeitraum [19. 9. 1907
                  – 23. 9. 1907?] in Wien}\toendnotes[C]{\smallbreak}
\Standort{CUL, Schnitzler, B 43.}
\physDesc{Brief, 1 Blatt, 3 Seiten, 844 Zeichen
\newline{}Handschrift: schwarze Tinte, deutsche Kurrent
\newline{}Schnitzler: mit Bleistift die Jahreszahl ergänzt: »907« 
\newline{}Ordnung: mit Bleistift von unbekannter Hand eine vorausgehende
                                 Nummerierung geändert zu: »286« }
\buchAbdrucke{\weitereDrucke{Hugo von Hofmannsthal, Arthur Schnitzler: \emph{Briefwechsel}. Herausgegeben von Therese Nickl und Heinrich Schnitzler. Frankfurt am Main: \emph{S. Fischer} 1964, S. 231.} }\toendnotes[C]{\smallbreak}
\pstart
           \raggedleft{}18 IX.\pend
           
\pstart{}{\pb}lieber\pend\vspace{0.5em}
\pstart
           Diplomatenprüfung im Alter 28/29 natürlich{ }ſehr ungewöhnlich, nur erklärlich – wie
               Sie{ }ſelbſt annehmen – durch Umſatteln aus dem \uline{inneren}
               Dienſt (Statthalterei.) allenfalls aus der Officierslaufbahn. Diplomatenprüfung{ }ſetzt
               volles jus (ohne Doctorat) voraus, hat aber mit orient. Akademie\orgindex{Orientalische Akademie@Orientalische Akademie|pw} gar nichts zu thuen; dieſe bereitet zur Conſularcarrière
                  {\pb}vor, welche dienſtlich und{ }ſocial von Diplomatie \uline{geſchieden}.\pend
           
\pstart
           Mein Stück\pwindex{Hofmannsthal, Hugo von 1.\,2.\,1874 Wien – 15.\,7.\,1929 Rodaun@\textsc{Hofmannsthal, Hugo von} (1.\,2.\,1874 Wien – 15.\,7.\,1929 Rodaun), \emph{Schriftsteller}!Silvia im »Stern«@\strich\emph{Silvia im »Stern«}|pw}{ }ſchreitet, in ungleichem tempo, vor.\hspace*{1.5em}Wir{ }ſind jedenfalls 1\textsuperscript{ten} October in Wien\oindex{Wien@\textbf{Wien}, \emph{Verwaltungsgebiet}|pw}.\pend
           
\pstart
           Herzlich Ihr{\\[\baselineskip]}\spacefill\mbox{Hugo.}\pend
           \leftskip=0em{}
\pstart
           \textsc{P. S.} Rathe dringend »Morgen\orgindex{Morgen. Wochenschrift für deutsche Kultur@Morgen. Wochenschrift für deutsche Kultur|pw}« und allen andern Reflectanten gegenüber den Preis \uuline{halten}, nicht{ }ſich eilen, nicht {\pb}Geduld verlieren, nicht{ }ſich ein
               paar Briefe mehr verdrießen laſſen. Waſſermann\pwindex{Wassermann, Jakob 10.\,3.\,1873 Fürth – 1.\,1.\,1934 Altaussee@\textsc{Wassermann, Jakob} (10.\,3.\,1873 Fürth – 1.\,1.\,1934 Altaussee), \emph{Schriftsteller}|pw}
                  beko{\geminationm}t von Über Land u
                  Meer\orgindex{Über Land und Meer@Über Land und Meer|pw}\pend
           \settowidth{\longeste}{8 Auflagen im vorhinein}\settowidth{\longestz}{= 24000 Kronen.m}\settowidth{\longestd}{}\settowidth{\longestv}{}\settowidth{\longestf}{}\addtolength\longeste{1em}
        \addtolength\longestz{1em}
      \pstart\noindent\makebox[\the\longeste][l]{für den Romanabdruck\pwindex{Wassermann, Jakob 10.\,3.\,1873 Fürth – 1.\,1.\,1934 Altaussee@\textsc{Wassermann, Jakob} (10.\,3.\,1873 Fürth – 1.\,1.\,1934 Altaussee), \emph{Schriftsteller}!Caspar Hauser oder Die Trägheit des Herzens@\strich\emph{Caspar Hauser oder Die Trägheit des Herzens}|pwv}}\makebox[\the\longestz][l]{\hspace*{1.5em}12000{ }}
                  \pend\pstart\noindent\makebox[\the\longeste][l]{8 Auflagen im vorhinein}\makebox[\the\longestz][l]{\hspace*{1.5em}\uline{{ }8000{ }}}
                  \pend\pstart\noindent\makebox[\the\longeste][l]{}\makebox[\the\longestz][l]{\hspace*{1.5em}20000M}
                  \pend\pstart\noindent\makebox[\the\longeste][l]{}\makebox[\the\longestz][l]{\hspace*{1.5em}= 24000 Kronen.}
                  \pend
\pstart
           Und Sie haben einen viel{ }ſtärkern Namen!\pend
           \selectlanguage{ngerman}\endnumbering\briefempfaengerindex{Schnitzler, Arthur@\textsc{Schnitzler, Arthur}!zzzHofmannsthal, Hugo von@\emph{von Hugo von Hofmannsthal}!1907-09-181@{{[}18. 9. 1907{]}}|)be}\mylabel{L01708h}  \newcommand{\dateiname}{L01708}\newcommand{\titel}{Hugo von Hofmannsthal an Arthur Schnitzler, 18. 9. [1907]}\newcommand{\editorInnen}{Martin Anton Müller und Gerd-Hermann Susen}%% latex-leseansicht-abspann.tex
%% Abspann für die Leseansicht.
%% Der Schalter \ifkorrekturansicht ist bereits durch den Vorspann gesetzt.

%% latex-abspann.tex
%% Gemeinsamer Abspann für Korrekturansicht und Leseansicht.
%% Setzt den Schalter \ifkorrekturansicht voraus (gesetzt in den
%% einbindenden Dateien latex-korrekturansicht-abspann.tex bzw.
%% latex-leseansicht-abspann.tex).
%% ---------------------------------------------------------------

\normalsize

% Das esempio-Environment wird nur in der Leseansicht benötigt
\ifkorrekturansicht\else
\newenvironment{esempio}[3]%
{
    \vspace{1.5ex}
    \rlap{\underline{#1}}
    \par
    \setlength{\parindent}{0cm}
    \nopagebreak
    \leftskip=#2cm
    \rightskip=#3cm
}
{
    \par
}
\fi

\doendnotes{C}
\bigskip
\vfill

\clearpage

\footnotesize

\ifkorrekturansicht
  \lohead{\textsc{register}}
\fi

% theindex-Environment neu definieren ohne reledmac
\makeatletter
\renewenvironment{theindex}{%
  \ifkorrekturansicht
    \section*{\indexname}%
  \else
    \subsubsection*{Index der erwähnten Entitäten}%
  \fi
  \setlength{\parindent}{0pt}%
  \setlength{\parskip}{0pt plus 0.3pt}%
  \let\item\@idxitem
}{%
  \ifkorrekturansicht\clearpage\fi
}
\makeatother

\IfFileExists{\jobname-pw.ind}{\input{\jobname-pw.ind}}{}

% Quellenangabe nur in der Leseansicht
\ifkorrekturansicht\else
% Fallback-Definitionen, falls die .tex-Datei \titel etc. nicht gesetzt hat
\providecommand{\titel}{}
\providecommand{\editorInnen}{}
\providecommand{\dateiname}{\jobname}

\vspace{3cm}

\vfill

\footnotesize
\textsc{Quelle}: \titel. Herausgegeben von {\editorInnen}. In: \emph{Arthur Schnitzler: Briefwechsel mit Autorinnen und Autoren}.
 Digitale Edition, https://schnitzler-briefe.acdh.oeaw.ac.at/{\dateiname}.html (Stand \today)
\fi

\end{document}


