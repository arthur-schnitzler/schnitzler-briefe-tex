%% latex-korrekturansicht-vorspann.tex
%% Vorspann für die Korrekturansicht.
%% Lädt die gemeinsame Datei latex-vorspann.tex mit gesetztem Schalter.

\newif\ifkorrekturansicht
\korrekturansichttrue

\input{../tex-inputs/latex-vorspann}


\section[Hugo von Hofmannsthal an Arthur Schnitzler, 25. 4. 1892]{L00098 Hugo von Hofmannsthal an Arthur Schnitzler, 25. 4. 1892}
\nopagebreak\mylabel{L00098v}
\rehead{ }\normalsize\beginnumbering\briefempfaengerindex{Schnitzler, Arthur@\textsc{Schnitzler, Arthur}!zzzHofmannsthal, Hugo von@\emph{von Hugo von Hofmannsthal}!1892-04-251@{25. 4. 1892}|(be}
\toendnotes[C]{\smallbreak\pagebreak[2]}\Standort{CUL, Schnitzler, B 43.}
\physDesc{Kartenbrief, 265 Zeichen
\newline{}Handschrift: Bleistift, deutsche Kurrent
\newline{}Versand: Stempel: »\nobreak{}Wien, 25. 4. 92, 10–12N\nobreak{}«.  
\newline{}Ordnung: mit Bleistift von unbekannter Hand nummeriert:
                                    »23« }
\buchAbdrucke{\weitereDrucke{Hugo von Hofmannsthal, Arthur Schnitzler: \emph{Briefwechsel}. Frankfurt am Main: \emph{S. Fischer} 1964, S. 21.} }\toendnotes[C]{\smallbreak}\pstart{}{\pb}Herrn \textsc{D\textsuperscript{r} Arthur Schnitzler}\pend{}\pstart{}\textsc{Wien\oindex{Wien@\textbf{Wien}, \emph{A.ADM2}|pw}}\pend{}\pstart{}\textsc{I Kärntnerring 12\oindex{Kaerntnerring 12/Boesendorferstrasse 11@\textbf{Kärntnerring 12/Bösendorferstraße 11}, \emph{Wohngebäude (K.WHS)}|pw}.}\pend{}{\bigskip}\vspace{1em}
\pstart{}{\pb}Lieber Arthur.\pend\vspace{0.5em}
\pstart
           Ich kann zu der Vorleſung leider abſolut nicht kommen, werde aber trachten, Sie
               nächſtens zur Ordinationsſtunde\oindex{Ordination Arthur Schnitzler [Boesendorferstrasse 11]@\textbf{Ordination Arthur Schnitzler [Bösendorferstraße 11]}, \emph{Ordination}|pwv} zu treffen.\pend
           
\pstart
           Grüßen Sie, bitte, \textsc{Salten}\pwindex{Salten, Felix 06.09.1869 – 08.10.1945@\textsc{Salten, Felix} (06.09.1869 – 08.10.1945), \emph{Schriftsteller/Schriftstellerin, Journalist/Journalistin, Chefredakteur/Chefredakteurin}|pw} und \textsc{Richard}\pwindex{Beer-Hofmann, Richard 1866-07-11 – 1945-09-26@\textsc{Beer-Hofmann, Richard} (1866-07-11 – 1945-09-26), \emph{Schriftsteller/Schriftstellerin}|pw} und ſeien Sie ſelbſt gegrüßt von Ihrem \pend
           \pstart \spacefill\mbox{Loris}\pend{}
\pstart
           Montag.\pend
           \selectlanguage{ngerman}\endnumbering\briefempfaengerindex{Schnitzler, Arthur@\textsc{Schnitzler, Arthur}!zzzHofmannsthal, Hugo von@\emph{von Hugo von Hofmannsthal}!1892-04-251@{25. 4. 1892}|)be}\mylabel{L00098h}  \normalsize

\doendnotes{C}
\bigskip
\vfill

\clearpage

\footnotesize

\lohead{\textsc{register}}

% Definiere theindex-Environment komplett neu ohne reledmac
\makeatletter
\renewenvironment{theindex}{%
  \section*{\indexname}%
  \setlength{\parindent}{0pt}%
  \setlength{\parskip}{0pt plus 0.3pt}%
  \let\item\@idxitem
}{%
  \clearpage
}
\makeatother

\IfFileExists{\jobname-pw.ind}{\input{\jobname-pw.ind}}{}

\end{document}

      