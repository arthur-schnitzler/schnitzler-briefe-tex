%% latex-leseansicht-vorspann.tex
%% Vorspann für die Leseansicht.
%% Lädt die gemeinsame Datei latex-vorspann.tex mit nicht gesetztem Schalter.

\newif\ifkorrekturansicht
\korrekturansichtfalse

\input{../tex-inputs/latex-vorspann}


\section[Arthur Schnitzler an Gustav Schwarzkopf, {{[}}20. 12. 1900?{{]}}]{L04176 Arthur Schnitzler an Gustav Schwarzkopf, {[}20. 12. 1900?{]}}
\nopagebreak\mylabel{L04176v}
\rehead{ }\normalsize\beginnumbering\briefempfaengerindex{Schwarzkopf, Gustav@\textsc{Schwarzkopf, Gustav}!zzzSchnitzler, Arthur@\emph{von Arthur Schnitzler}!1900-12-201@{{[}20. 12. 1900?{]}}|(be}
\toendnotes[C]{\smallbreak\pagebreak[2]}
\correspDesc{Versand  durch Arthur Schnitzler am [20. 12. 1900?] in Wien
\newline{}Erhalt  durch Gustav Schwarzkopf im Zeitraum [20. 12. 1900 – 21. 12. 1900?] in Wien}\toendnotes[C]{\smallbreak}
\Standort{CUL, Schnitzler, B 96.}
\physDesc{Brief, 1 Blatt, 1 Seite, 106 Zeichen
\newline{}Handschrift: Bleistift, deutsche Kurrent}\toendnotes[C]{\smallbreak}
\pstart{}{\pb}Lieber Guſtav,\pend\vspace{0.5em}
\pstart
           wollen Sie \label{K_L04176-1v}\edtext{morgen Freitg}{\lemma{\textnormal{\emph{morgen Freitg}}}\Cendnote{\textnormal{Der vorliegende Brief ist undatiert,
                  aber Schnitzler war, nachdem er Schwarzkopf\pwindex{Schwarzkopf, Gustav 7.\,11.\,1853 Wien – 13.\,11.\,1939 ebd.@\textsc{Schwarzkopf, Gustav} (7.\,11.\,1853 Wien – 13.\,11.\,1939 ebd.), \emph{Schriftsteller}|pwk} kannte, nur einmal an einem
                  Freitag bei einer Veranstaltung im Konservatorium\oindex{Wien@\textbf{Wien}!I., Innere Stadt@\textbf{I., Innere Stadt}!Konservatorium der Gesellschaft der Musikfreunde@\textbf{Konservatorium der Gesellschaft der Musikfreunde}, \emph{Konservatorium}|pwk}. Begleitet wurde er von Schwarzkopf\pwindex{Schwarzkopf, Gustav 7.\,11.\,1853 Wien – 13.\,11.\,1939 ebd.@\textsc{Schwarzkopf, Gustav} (7.\,11.\,1853 Wien – 13.\,11.\,1939 ebd.), \emph{Schriftsteller}|pwk}, demnach hat dieser zugesagt. Vgl. A. S.: \emph{Kulturveranstaltungen}, 21. 12. 1900.}}}\label{K_L04176-1} mit mir ins Conſervator.\oindex{Wien@\textbf{Wien}!I., Innere Stadt@\textbf{I., Innere Stadt}!Konservatorium der Gesellschaft der Musikfreunde@\textbf{Konservatorium der Gesellschaft der Musikfreunde}, \emph{Konservatorium}|pw}\eventindex{Konservatorium der Gesellschaft der Musikfreunde@\textbf{Konservatorium der Gesellschaft der Musikfreunde}!Vortragsübung der Schauspielschule des Konservatoriums der Gesellschaft der Musikfreunde, 21.12.1900@Vortragsübung der Schauspielschule des Konservatoriums der Gesellschaft der Musikfreunde, 21.12.1900|pwv} gehen, ſo wirds mich freuen.\pend
           
\pstart
           Herzlichſt{\\[\baselineskip]} Ihr{\\[\baselineskip]}\spacefill\mbox{Arthur}\pend
           \leftskip=0em{}\selectlanguage{ngerman}\endnumbering\briefempfaengerindex{Schwarzkopf, Gustav@\textsc{Schwarzkopf, Gustav}!zzzSchnitzler, Arthur@\emph{von Arthur Schnitzler}!1900-12-201@{{[}20. 12. 1900?{]}}|)be}\mylabel{L04176h}
\begin{anhang}
\end{anhang}\newcommand{\dateiname}{L04176}\newcommand{\titel}{Arthur Schnitzler an Gustav Schwarzkopf, [20. 12. 1900?]}\newcommand{\editorInnen}{Herausgegeben von Jahnke, SelmaMüller, Martin Anton}%% latex-leseansicht-abspann.tex
%% Abspann für die Leseansicht.
%% Der Schalter \ifkorrekturansicht ist bereits durch den Vorspann gesetzt.

%% latex-abspann.tex
%% Gemeinsamer Abspann für Korrekturansicht und Leseansicht.
%% Setzt den Schalter \ifkorrekturansicht voraus (gesetzt in den
%% einbindenden Dateien latex-korrekturansicht-abspann.tex bzw.
%% latex-leseansicht-abspann.tex).
%% ---------------------------------------------------------------

\normalsize

% Das esempio-Environment wird nur in der Leseansicht benötigt
\ifkorrekturansicht\else
\newenvironment{esempio}[3]%
{
    \vspace{1.5ex}
    \rlap{\underline{#1}}
    \par
    \setlength{\parindent}{0cm}
    \nopagebreak
    \leftskip=#2cm
    \rightskip=#3cm
}
{
    \par
}
\fi

\doendnotes{C}
\bigskip
\vfill

\clearpage

\footnotesize

\ifkorrekturansicht
  \lohead{\textsc{register}}
\fi

% theindex-Environment neu definieren ohne reledmac
\makeatletter
\renewenvironment{theindex}{%
  \ifkorrekturansicht
    \section*{\indexname}%
  \else
    \subsubsection*{Index der erwähnten Entitäten}%
  \fi
  \setlength{\parindent}{0pt}%
  \setlength{\parskip}{0pt plus 0.3pt}%
  \let\item\@idxitem
}{%
  \ifkorrekturansicht\clearpage\fi
}
\makeatother

\IfFileExists{\jobname-pw.ind}{\input{\jobname-pw.ind}}{}

% Quellenangabe nur in der Leseansicht
\ifkorrekturansicht\else
% Fallback-Definitionen, falls die .tex-Datei \titel etc. nicht gesetzt hat
\providecommand{\titel}{}
\providecommand{\editorInnen}{}
\providecommand{\dateiname}{\jobname}

\vspace{3cm}

\vfill

\footnotesize
\textsc{Quelle}: \titel. Herausgegeben von {\editorInnen}. In: \emph{Arthur Schnitzler: Briefwechsel mit Autorinnen und Autoren}.
 Digitale Edition, https://schnitzler-briefe.acdh.oeaw.ac.at/{\dateiname}.html (Stand \today)
\fi

\end{document}


