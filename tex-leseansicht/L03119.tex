%% latex-korrekturansicht-vorspann.tex
%% Vorspann für die Korrekturansicht.
%% Lädt die gemeinsame Datei latex-vorspann.tex mit gesetztem Schalter.

\newif\ifkorrekturansicht
\korrekturansichttrue

\input{../tex-inputs/latex-vorspann}


\section[Felix Salten an Arthur Schnitzler, {[}7. 2. 1893?{]}]{L03119 Felix Salten an Arthur Schnitzler, {[}7. 2. 1893?{]}}
\nopagebreak\mylabel{L03119v}
\rehead{ }\normalsize\beginnumbering\briefempfaengerindex{Schnitzler, Arthur@\textsc{Schnitzler, Arthur}!zzzSalten, Felix@\emph{von Felix Salten}!1893-02-071@{{[}7. 2. 1893?{]}}|(be}
\toendnotes[C]{\smallbreak\pagebreak[2]}\Standort{CUL, Schnitzler, B 89, A 1.}
\physDesc{Brief, 1 Blatt, 2 Seiten, 601 Zeichen
\newline{}Handschrift: Bleistift, lateinische Kurrent
\newline{}Schnitzler: mit Bleistift datiert: »92« 
\newline{}Ordnung: mit Bleistift von unbekannter Hand nummeriert: »22« }\toendnotes[C]{\smallbreak}
\pstart
           \noindent{}{\pb}Lieber Freund! Ich habe allerdings eine Verständigung
               erhalten, bin aber nicht sehr aufgelegt \label{K_L03119-1v}\edtext{hinauszufahren}{\lemma{\textnormal{\emph{hinauszufahren}}}\Cendnote{\textnormal{Das Volkstheater in Rudolfsheim\oindex{Volkstheater in Rudolfsheim@\textbf{Volkstheater in Rudolfsheim}, \emph{Theater (K.THE)}|pwk} befand sich im 15. Wiener
                     Gemeindebezirk\oindex{XV., Rudolfsheim-Fuenfhaus@\textbf{XV., Rudolfsheim-Fünfhaus}, \emph{A.ADM3}|pwk} und damit außerhalb der ›Linie‹ – dem Gürtel\oindex{Guertel@\textbf{Gürtel}, \emph{Straße (K.STR)}|pwk} –, die die inneren Wohnbezirke von den äußeren
                  trennte.}}}\label{K_L03119-1}, um so mehr als ich eine \label{K_L03119-2v}\edtext{Karte zur Joachim\pwindex{Joachim, Amalie 1839-05-10 – 1899-02-03@\textsc{Joachim, Amalie} (1839-05-10 – 1899-02-03), \emph{Sänger/Sängerin, Opernsänger/Opernsängerin, Gesangslehrer/Gesangslehrerin}|pw}}{\lemma{\textnormal{\emph{Karte zur Joachim}}}\Cendnote{\textnormal{Das Korrespondenzstück ist undatiert und
                  von Schnitzler nur grob – im Jahr 1892 – verortet. Im Oktober 1892
                  gab Amalie Joachim\pwindex{Joachim, Amalie 1839-05-10 – 1899-02-03@\textsc{Joachim, Amalie} (1839-05-10 – 1899-02-03), \emph{Sänger/Sängerin, Opernsänger/Opernsängerin, Gesangslehrer/Gesangslehrerin}|pwk} drei Konzerte in Wien\oindex{Wien@\textbf{Wien}, \emph{A.ADM2}|pwk}, am 3.,
                     5. und 7.{ }Schnitzler war bei keinem der drei und zu
                  dieser Zeit auch nicht im \emph{Volkstheater in
                     Rudolfsheim}\orgindex{Volkstheater in Rudolfsheim@Volkstheater in Rudolfsheim|pwk}. Die nächsten drei Auftritte in Wien\oindex{Wien@\textbf{Wien}, \emph{A.ADM2}|pwk} gab Joachim\pwindex{Joachim, Amalie 1839-05-10 – 1899-02-03@\textsc{Joachim, Amalie} (1839-05-10 – 1899-02-03), \emph{Sänger/Sängerin, Opernsänger/Opernsängerin, Gesangslehrer/Gesangslehrerin}|pwk} am
                     5., 7. und 11. 2. 1893. Da Schnitzler am 7. 2. 1893 im Volkstheater in Rudolfsheim\oindex{Volkstheater in Rudolfsheim@\textbf{Volkstheater in Rudolfsheim}, \emph{Theater (K.THE)}|pwk} die Aufführung von \emph{Medea}\pwindex{Medea. Trauerspiel in fuenf Aufzuegen@\emph{Medea. Trauerspiel in fünf Aufzügen}|pwk} besuchte, dürfte dies der Tag dieses Schreibens
                  sein.}}}\label{K_L03119-2} habe, wovon ich Ihnen auch eine zur Verfügung stellen kann, falls Sie
               doch nicht nach Rudolfsheim\oindex{Volkstheater in Rudolfsheim@\textbf{Volkstheater in Rudolfsheim}, \emph{Theater (K.THE)}|pw} fahren.\pend
           
\pstart
           Ich gehe jetzt zu Beer-Hofmann\pwindex{Beer-Hofmann, Richard 1866-07-11 – 1945-09-26@\textsc{Beer-Hofmann, Richard} (1866-07-11 – 1945-09-26), \emph{Schriftsteller/Schriftstellerin}|pw} und frage ihn
               was er beschließt. Auf jeden Fall {\pb}haben Sie dann bestimmte
               Nachricht im Griensteidl\oindex{Cafe Griensteidl@\textbf{Café Griensteidl}, \emph{Kaffeehaus (K.KAF)}|pw} noch vor
                  6 Uhr.\pend
           
\pstart
           Ehrlich, ist mir diese \label{K_L03119-3v}\edtext{Person\pwindex{Pichler, Marie @\textsc{Pichler, Marie}, \emph{Schauspieler/Schauspielerin}|pwuv}}{\lemma{\textnormal{\emph{Person}}}\Cendnote{\textnormal{Zuletzt waren Schnitzler und Salten\pwindex{Salten, Felix 06.09.1869 – 08.10.1945@\textsc{Salten, Felix} (06.09.1869 – 08.10.1945), \emph{Schriftsteller/Schriftstellerin, Journalist/Journalistin, Chefredakteur/Chefredakteurin}|pwk} am 14. 1. 1893 im
                  Volkstheater Rudolfsheim\oindex{Volkstheater in Rudolfsheim@\textbf{Volkstheater in Rudolfsheim}, \emph{Theater (K.THE)}|pwk} in der Aufführung von \emph{Die Räuber}\pwindex{Raeuber. Ein Schauspiel@\emph{Die Räuber. Ein Schauspiel}|pwk}, 
                  an der Karl Kraus\pwindex{Kraus, Karl 28.04.1874 – 12.06.1936@\textsc{Kraus, Karl} (28.04.1874 – 12.06.1936), \emph{Schriftsteller/Schriftstellerin, Publizist/Publizistin, Schriftsteller/Schriftstellerin}|pwk} und Max Reinhardt\pwindex{Reinhardt, Max 09.09.1873 – 30.10.1943@\textsc{Reinhardt, Max} (09.09.1873 – 30.10.1943), \emph{Theaterleiter/Theaterleiterin, Regisseur/Regisseurin, Schauspieler/Schauspielerin}|pwk} mitwirkten.
                  Wenn es sich bei der »Person« um eine Schauspielerin handeln sollte, dürfte Marie Pichler\pwindex{Pichler, Marie @\textsc{Pichler, Marie}, \emph{Schauspieler/Schauspielerin}|pwk}
                  gemeint sein, die einzige Schauspielerin, die auf dem Theaterzettel von \emph{Die Räuber}\pwindex{Raeuber. Ein Schauspiel@\emph{Die Räuber. Ein Schauspiel}|pwk} stand.}}}\label{K_L03119-3} ziemlich
               uninteressant, und glaube ich, dass wir uns ein 2\textsuperscript{tes} Mal
               sehr langweilen werden.\pend
           
\pstart
           Herzlichst Ihr {\\[\baselineskip]}treuer {\\[\baselineskip]}\spacefill\mbox{Salten}\pend
           \leftskip=0em{}
\pstart
           \noindent{}\label{K_L03119-4v}\edtext{Specht\pwindex{Specht, Richard 07.12.1870 – 18.03.1932@\textsc{Specht, Richard} (07.12.1870 – 18.03.1932), \emph{Schriftsteller/Schriftstellerin, Journalist/Journalistin, Kritiker/Kritikerin}|pw}, werde ich wegen Pfob\oindex{Cafe Pfob@\textbf{Café Pfob}, \emph{Kaffeehaus (K.KAF)}|pw} avisiren}{\lemma{\textnormal{\emph{Specht, … avisiren}}}\Cendnote{\textnormal{Für
                     diese Zeit ist kein gemeinsamer Besuch im Café
                        Pfob\oindex{Cafe Pfob@\textbf{Café Pfob}, \emph{Kaffeehaus (K.KAF)}|pwk} belegt.}}}\label{K_L03119-4}, da er \uline{gewiss} nicht
                  nach Rdlfshm\oindex{Volkstheater in Rudolfsheim@\textbf{Volkstheater in Rudolfsheim}, \emph{Theater (K.THE)}|pw} fährt.\pend
           \selectlanguage{ngerman}\endnumbering\briefempfaengerindex{Schnitzler, Arthur@\textsc{Schnitzler, Arthur}!zzzSalten, Felix@\emph{von Felix Salten}!1893-02-071@{{[}7. 2. 1893?{]}}|)be}\mylabel{L03119h}  \normalsize

\doendnotes{C}
\bigskip
\vfill

\clearpage

\footnotesize

\lohead{\textsc{register}}

% Definiere theindex-Environment komplett neu ohne reledmac
\makeatletter
\renewenvironment{theindex}{%
  \section*{\indexname}%
  \setlength{\parindent}{0pt}%
  \setlength{\parskip}{0pt plus 0.3pt}%
  \let\item\@idxitem
}{%
  \clearpage
}
\makeatother

\IfFileExists{\jobname-pw.ind}{\input{\jobname-pw.ind}}{}

\end{document}

      