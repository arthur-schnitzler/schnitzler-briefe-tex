%% latex-leseansicht-vorspann.tex
%% Vorspann für die Leseansicht.
%% Lädt die gemeinsame Datei latex-vorspann.tex mit nicht gesetztem Schalter.

\newif\ifkorrekturansicht
\korrekturansichtfalse

\input{../tex-inputs/latex-vorspann}

\begin{center}
            \textcolor{red}{ENTWURF, NICHT FERTIG KORRIGIERT}
                      \end{center}
            
         
         \renewcommand{\erwaehntePersonen}{Personen: Richard Beer-Hofmann, Amalie Joachim, Richard Specht}
         \renewcommand{\erwaehnteInstitutionen}{Institutionen: Volkstheater in Rudolfsheim}
         \renewcommand{\erwaehnteOrte}{Orte: Café Griensteidl, Café Pfob, Gürtel, Volkstheater in Rudolfsheim, Wien, XV., Rudolfsheim-Fünfhaus}
         \renewcommand{\erwaehnteWerke}{}
               \section[Felix Salten an Arthur Schnitzler, {[}5. 10.? 1892{]}]{ Felix Salten an Arthur Schnitzler, {[}5. 10.? 1892{]}}\nopagebreak\mylabel{v}\rehead{ }\begin{ledgroupsized}[t]{13cm}\normalsize\beginnumbering \toendnotes[C]{\smallbreak\pagebreak[2]} \Standort{CUL, Schnitzler, B 89, A 1.}
\physDesc{Brief, 1 Blatt, 3 Seiten, 600 Zeichen
\newline{}Handschrift: Bleistift, lateinische Kurrent
\newline{}Ordnung: mit Bleistift von unbekannter Hand nummeriert:
                                    »22« }\toendnotes[C]{\smallbreak}\pstart
           \noindent{}{\pb}Lieber Freund! Ich habe allerdings eine Verständigung erhalten, bin
               aber nicht sehr aufgelegt \label{K_L03119-1v}\edtext{hinauszufahren}{\lemma{\textnormal{\emph{hinauszufahren}}}\Cendnote{\textnormal{Das Volkstheater in Rudolphsheim\oindex{Volkstheater in Rudolfsheim@\textbf{Volkstheater in Rudolfsheim}|pwk} befand sich im 15. Wiener Gemeindebezirk\oindex{XV., Rudolfsheim-Fuenfhaus@\textbf{XV., Rudolfsheim-Fünfhaus}|pwk} und damit außerhalb der »Linie« –
                  dem Gürtel\oindex{Guertel@\textbf{Gürtel}|pwk} –, die die inneren Wohnbezirke
                  von den äußeren trennte.}}}\label{K_L03119-1h}, um so mehr als ich eine \label{K_L03119-2v}\edtext{Karte zur Joachim\pwindex{Joachim, Amalie 1839-05-10 – 1899-02-03@\textsc{Joachim, Amalie} (1839-05-10 – 1899-02-03), \emph{, , }|pw}}{\lemma{\textnormal{\emph{Karte zur Joachim}}}\Cendnote{\textnormal{Das Korrespondenzstück ist undatiert und
                  von Schnitzler\pwindex{Schnitzler, Arthur 15.05.1862 – 21.10.1931@\textsc{Schnitzler, Arthur} (15.05.1862 – 21.10.1931), \emph{Schriftsteller, Mediziner}|pwk} nur grob im Jahr
                     1892 verortet. Im Oktober 1892 gab Amalie Joachim\pwindex{Joachim, Amalie 1839-05-10 – 1899-02-03@\textsc{Joachim, Amalie} (1839-05-10 – 1899-02-03), \emph{, , }|pwk} drei Konzerte in Wien\oindex{Wien@\textbf{Wien}|pwk}, am 3., am 5. und am
                     7.Schnitzler\pwindex{Schnitzler, Arthur 15.05.1862 – 21.10.1931@\textsc{Schnitzler, Arthur} (15.05.1862 – 21.10.1931), \emph{Schriftsteller, Mediziner}|pwk} war auf keinem der drei und zu
                  dieser Zeit auch nicht im \emph{Volkstheater in
                     Rudolfsheim}\orgindex{Volkstheater in Rudolfsheim@Volkstheater in Rudolfsheim|pwk}. Durch die Aussage Salten\pwindex{Salten, Felix 06.09.1869 – 08.10.1945@\textsc{Salten, Felix} (06.09.1869 – 08.10.1945), \emph{Schriftsteller, Journalist}|pwk}s, sie bereits gesehen zu haben, ist der erste Konzerttermin für
                  dieses Schreiben nicht zu berücksichtigen. Da Salten\pwindex{Salten, Felix 06.09.1869 – 08.10.1945@\textsc{Salten, Felix} (06.09.1869 – 08.10.1945), \emph{Schriftsteller, Journalist}|pwk} am [8. 10. 1892] bereits mehrere Tage krank ist, fällt auch ist auch der
                  dritte Termin nicht heranzuziehen, weswegen das Schreiben vom
                     5. 10. 1892 stammen dürfte.}}}\label{K_L03119-2h} habe, wovon ich Ihnen auch eine
               zur Verfügung stellen kann, falls Sie doch nicht nach Rudolfsheim\oindex{Volkstheater in Rudolfsheim@\textbf{Volkstheater in Rudolfsheim}|pw} fahren.\pend
           \pstart
           Ich gehe jetzt zu Beer-Hofmann\pwindex{Beer-Hofmann, Richard 1866-07-11 – 1945-09-26@\textsc{Beer-Hofmann, Richard} (1866-07-11 – 1945-09-26), \emph{Schriftsteller}|pw} und frage ihn
               was er beschließt. Auf jeden Fall {\pb}haben Sie dann bestimmte
               Nachricht im Griensteidl\oindex{Cafe Griensteidl@\textbf{Café Griensteidl}|pw} noch vor
                  6 Uhr. \pend
           \pstart
           Ehrlich, ist mir diese Person\pwindex{Joachim, Amalie 1839-05-10 – 1899-02-03@\textsc{Joachim, Amalie} (1839-05-10 – 1899-02-03), \emph{, , }|pwv} ziemlich uninteressant, und glaube ich, dass wir uns ein 2\textsuperscript{tes} Mal sehr langweilen werden.\pend
           \pstart
           Herzlichst Ihr treuer{\\[\baselineskip]}\spacefill\mbox{Salten}\pend
           \leftskip=0em{}\pstart
           \noindent{}Specht\pwindex{Specht, Richard 07.12.1870 – 18.03.1932@\textsc{Specht, Richard} (07.12.1870 – 18.03.1932), \emph{Schriftsteller, Journalist, Kritiker}|pw} werde ich wegen Pfob\oindex{Cafe Pfob@\textbf{Café Pfob}|pw} avisiren, da er \uline{gewiss}
                  nicht nach Rdlfshm\oindex{Volkstheater in Rudolfsheim@\textbf{Volkstheater in Rudolfsheim}|pw} fährt.\pend
           
         
         \endnumbering\mylabel{h}\end{ledgroupsized}\begin{anhang}\end{anhang}\newcommand{\dateiname}{L03119}\newcommand{\titel}{Felix Salten an Arthur Schnitzler, [5. 10.? 1892]}\newcommand{\editorInnen}{Martin Anton Müller und Laura Untner}%% latex-leseansicht-abspann.tex
%% Abspann für die Leseansicht.
%% Der Schalter \ifkorrekturansicht ist bereits durch den Vorspann gesetzt.

%% latex-abspann.tex
%% Gemeinsamer Abspann für Korrekturansicht und Leseansicht.
%% Setzt den Schalter \ifkorrekturansicht voraus (gesetzt in den
%% einbindenden Dateien latex-korrekturansicht-abspann.tex bzw.
%% latex-leseansicht-abspann.tex).
%% ---------------------------------------------------------------

\normalsize

% Das esempio-Environment wird nur in der Leseansicht benötigt
\ifkorrekturansicht\else
\newenvironment{esempio}[3]%
{
    \vspace{1.5ex}
    \rlap{\underline{#1}}
    \par
    \setlength{\parindent}{0cm}
    \nopagebreak
    \leftskip=#2cm
    \rightskip=#3cm
}
{
    \par
}
\fi

\doendnotes{C}
\bigskip
\vfill

\clearpage

\footnotesize

\ifkorrekturansicht
  \lohead{\textsc{register}}
\fi

% theindex-Environment neu definieren ohne reledmac
\makeatletter
\renewenvironment{theindex}{%
  \ifkorrekturansicht
    \section*{\indexname}%
  \else
    \subsubsection*{Index der erwähnten Entitäten}%
  \fi
  \setlength{\parindent}{0pt}%
  \setlength{\parskip}{0pt plus 0.3pt}%
  \let\item\@idxitem
}{%
  \ifkorrekturansicht\clearpage\fi
}
\makeatother

\IfFileExists{\jobname-pw.ind}{\input{\jobname-pw.ind}}{}

% Quellenangabe nur in der Leseansicht
\ifkorrekturansicht\else
% Fallback-Definitionen, falls die .tex-Datei \titel etc. nicht gesetzt hat
\providecommand{\titel}{}
\providecommand{\editorInnen}{}
\providecommand{\dateiname}{\jobname}

\vspace{3cm}

\vfill

\footnotesize
\textsc{Quelle}: \titel. Herausgegeben von {\editorInnen}. In: \emph{Arthur Schnitzler: Briefwechsel mit Autorinnen und Autoren}.
 Digitale Edition, https://schnitzler-briefe.acdh.oeaw.ac.at/{\dateiname}.html (Stand \today)
\fi

\end{document}


      