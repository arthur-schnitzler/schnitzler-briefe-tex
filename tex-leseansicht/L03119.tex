%% latex-leseansicht-vorspann.tex
%% Vorspann für die Leseansicht.
%% Lädt die gemeinsame Datei latex-vorspann.tex mit nicht gesetztem Schalter.

\newif\ifkorrekturansicht
\korrekturansichtfalse

\input{../tex-inputs/latex-vorspann}


\section[Felix Salten an Arthur Schnitzler, {[}7. 2. 1893?{]}]{L03119 Felix Salten an Arthur Schnitzler, {[}7. 2. 1893?{]}}
\nopagebreak\mylabel{L03119v}
\rehead{ }\normalsize\beginnumbering\briefempfaengerindex{Schnitzler, Arthur@\textsc{Schnitzler, Arthur}!zzzSalten, Felix@\emph{von Felix Salten}!1893-02-071@{{[}7. 2. 1893?{]}}|(be}
\toendnotes[C]{\smallbreak\pagebreak[2]}
\correspDesc{Versand  durch Felix Salten am [7. 2. 1893?] in Wien
\newline{}Erhalt  durch Arthur Schnitzler am [7. 2. 1893?] in Wien}\toendnotes[C]{\smallbreak}
\Standort{CUL, Schnitzler, B 89, A 1.}
\physDesc{Brief, 1 Blatt, 2 Seiten, 601 Zeichen
\newline{}Handschrift: Bleistift, lateinische Kurrent
\newline{}Schnitzler: mit Bleistift datiert: »92« 
\newline{}Ordnung: mit Bleistift von unbekannter Hand nummeriert: »22« }\toendnotes[C]{\smallbreak}
\pstart
           \noindent{}{\pb}Lieber Freund! Ich habe allerdings eine Verständigung
               erhalten, bin aber nicht sehr aufgelegt \label{K_L03119-1v}\edtext{hinauszufahren}{\lemma{\textnormal{\emph{hinauszufahren}}}\Cendnote{\textnormal{Das Volkstheater in Rudolfsheim\oindex{Wien@\textbf{Wien}!XV., Rudolfsheim-Fünfhaus@\textbf{XV., Rudolfsheim-Fünfhaus}!Volkstheater in Rudolfsheim@\textbf{Volkstheater in Rudolfsheim}, \emph{Theater}|pwk} befand sich im 15. Wiener
                     Gemeindebezirk\oindex{XV., Rudolfsheim-Fünfhaus@\textbf{XV., Rudolfsheim-Fünfhaus}, \emph{Verwaltungsgebiet}|pwk} und damit außerhalb der ›Linie‹ – dem Gürtel\oindex{Gürtel@\textbf{Gürtel}, \emph{Straße}|pwk} –, die die inneren Wohnbezirke von den äußeren
                  trennte.}}}\label{K_L03119-1}, um so mehr als ich eine \label{K_L03119-2v}\edtext{Karte zur Joachim\pwindex{Joachim, Amalie 10.\,5.\,1839 Maribor – 3.\,2.\,1899@\textsc{Joachim, Amalie} (10.\,5.\,1839 Maribor – 3.\,2.\,1899), \emph{Sängerin, Opernsängerin, Gesangslehrerin}|pw}}{\lemma{\textnormal{\emph{Karte zur Joachim}}}\Cendnote{\textnormal{Das Korrespondenzstück ist undatiert und
                  von Schnitzler nur grob – im Jahr 1892 – verortet. Im Oktober 1892
                  gab Amalie Joachim\pwindex{Joachim, Amalie 10.\,5.\,1839 Maribor – 3.\,2.\,1899@\textsc{Joachim, Amalie} (10.\,5.\,1839 Maribor – 3.\,2.\,1899), \emph{Sängerin, Opernsängerin, Gesangslehrerin}|pwk} drei Konzerte in Wien\oindex{Wien@\textbf{Wien}, \emph{Verwaltungsgebiet}|pwk}, am 3.,
                     5. und 7.{ }Schnitzler war bei keinem der drei und zu
                  dieser Zeit auch nicht im \emph{Volkstheater in
                     Rudolfsheim}\orgindex{Volkstheater in Rudolfsheim@Volkstheater in Rudolfsheim|pwk}. Die nächsten drei Auftritte in Wien\oindex{Wien@\textbf{Wien}, \emph{Verwaltungsgebiet}|pwk} gab Joachim\pwindex{Joachim, Amalie 10.\,5.\,1839 Maribor – 3.\,2.\,1899@\textsc{Joachim, Amalie} (10.\,5.\,1839 Maribor – 3.\,2.\,1899), \emph{Sängerin, Opernsängerin, Gesangslehrerin}|pwk} am
                     5., 7. und 11. 2. 1893. Da Schnitzler am 7. 2. 1893 im Volkstheater in Rudolfsheim\oindex{Wien@\textbf{Wien}!XV., Rudolfsheim-Fünfhaus@\textbf{XV., Rudolfsheim-Fünfhaus}!Volkstheater in Rudolfsheim@\textbf{Volkstheater in Rudolfsheim}, \emph{Theater}|pwk} die Aufführung von \emph{Medea}\pwindex{\textcolor{red}{\textsuperscript{XXXX indx1}}!Medea. Trauerspiel in fünf Aufzügen@\strich\emph{Medea. Trauerspiel in fünf Aufzügen}|pwk} besuchte, dürfte dies der Tag dieses Schreibens
                  sein.}}}\label{K_L03119-2} habe, wovon ich Ihnen auch eine zur Verfügung stellen kann, falls Sie
               doch nicht nach Rudolfsheim\oindex{Wien@\textbf{Wien}!XV., Rudolfsheim-Fünfhaus@\textbf{XV., Rudolfsheim-Fünfhaus}!Volkstheater in Rudolfsheim@\textbf{Volkstheater in Rudolfsheim}, \emph{Theater}|pw} fahren.\pend
           
\pstart
           Ich gehe jetzt zu Beer-Hofmann\pwindex{Beer-Hofmann, Richard 11.\,7.\,1866 Wien – 26.\,9.\,1945 New York City@\textsc{Beer-Hofmann, Richard} (11.\,7.\,1866 Wien – 26.\,9.\,1945 New York City), \emph{Schriftsteller}|pw} und frage ihn
               was er beschließt. Auf jeden Fall {\pb}haben Sie dann bestimmte
               Nachricht im Griensteidl\oindex{Wien@\textbf{Wien}!I., Innere Stadt@\textbf{I., Innere Stadt}!Café Griensteidl@\textbf{Café Griensteidl}, \emph{Kaffeehaus}|pw} noch vor
                  6 Uhr.\pend
           
\pstart
           Ehrlich, ist mir diese \label{K_L03119-3v}\edtext{Person\pwindex{Pichler, Marie @\textsc{Pichler, Marie}, \emph{Schauspielerin}|pwuv}}{\lemma{\textnormal{\emph{Person}}}\Cendnote{\textnormal{Zuletzt waren Schnitzler und Salten\pwindex{Salten, Felix 6.\,9.\,1869 Budapest – 8.\,10.\,1945 Zürich@\textsc{Salten, Felix} (6.\,9.\,1869 Budapest – 8.\,10.\,1945 Zürich), \emph{Schriftsteller, Journalist, Chefredakteur}|pwk} am 14. 1. 1893 im
                  Volkstheater Rudolfsheim\oindex{Wien@\textbf{Wien}!XV., Rudolfsheim-Fünfhaus@\textbf{XV., Rudolfsheim-Fünfhaus}!Volkstheater in Rudolfsheim@\textbf{Volkstheater in Rudolfsheim}, \emph{Theater}|pwk} in der Aufführung von \emph{Die Räuber}\pwindex{\textcolor{red}{\textsuperscript{XXXX indx1}}!Räuber. Ein Schauspiel@\strich\emph{Die Räuber. Ein Schauspiel}|pwk}, 
                  an der Karl Kraus\pwindex{Kraus, Karl 28.\,4.\,1874 Jičín – 12.\,6.\,1936 Wien@\textsc{Kraus, Karl} (28.\,4.\,1874 Jičín – 12.\,6.\,1936 Wien), \emph{Schriftsteller, Publizist, Schriftsteller}|pwk} und Max Reinhardt\pwindex{Reinhardt, Max 9.\,9.\,1873 Baden bei Wien – 30.\,10.\,1943 New York City@\textsc{Reinhardt, Max} (9.\,9.\,1873 Baden bei Wien – 30.\,10.\,1943 New York City), \emph{Theaterleiter, Regisseur, Schauspieler}|pwk} mitwirkten.
                  Wenn es sich bei der »Person« um eine Schauspielerin handeln sollte, dürfte Marie Pichler\pwindex{Pichler, Marie @\textsc{Pichler, Marie}, \emph{Schauspielerin}|pwk}
                  gemeint sein, die einzige Schauspielerin, die auf dem Theaterzettel von \emph{Die Räuber}\pwindex{\textcolor{red}{\textsuperscript{XXXX indx1}}!Räuber. Ein Schauspiel@\strich\emph{Die Räuber. Ein Schauspiel}|pwk} stand.}}}\label{K_L03119-3} ziemlich
               uninteressant, und glaube ich, dass wir uns ein 2\textsuperscript{tes} Mal
               sehr langweilen werden.\pend
           
\pstart
           Herzlichst Ihr {\\[\baselineskip]}treuer {\\[\baselineskip]}\spacefill\mbox{Salten}\pend
           \leftskip=0em{}
\pstart
           \noindent{}\label{K_L03119-4v}\edtext{Specht\pwindex{Specht, Richard 7.\,12.\,1870 Wien – 18.\,3.\,1932 ebd.@\textsc{Specht, Richard} (7.\,12.\,1870 Wien – 18.\,3.\,1932 ebd.), \emph{Schriftsteller, Journalist, Kritiker}|pw}, werde ich wegen Pfob\oindex{Wien@\textbf{Wien}!I., Innere Stadt@\textbf{I., Innere Stadt}!Café Pfob@\textbf{Café Pfob}, \emph{Kaffeehaus}|pw} avisiren}{\lemma{\textnormal{\emph{Specht, … avisiren}}}\Cendnote{\textnormal{Für
                     diese Zeit ist kein gemeinsamer Besuch im Café
                        Pfob\oindex{Wien@\textbf{Wien}!I., Innere Stadt@\textbf{I., Innere Stadt}!Café Pfob@\textbf{Café Pfob}, \emph{Kaffeehaus}|pwk} belegt.}}}\label{K_L03119-4}, da er \uline{gewiss} nicht
                  nach Rdlfshm\oindex{Wien@\textbf{Wien}!XV., Rudolfsheim-Fünfhaus@\textbf{XV., Rudolfsheim-Fünfhaus}!Volkstheater in Rudolfsheim@\textbf{Volkstheater in Rudolfsheim}, \emph{Theater}|pw} fährt.\pend
           \selectlanguage{ngerman}\endnumbering\briefempfaengerindex{Schnitzler, Arthur@\textsc{Schnitzler, Arthur}!zzzSalten, Felix@\emph{von Felix Salten}!1893-02-071@{{[}7. 2. 1893?{]}}|)be}\mylabel{L03119h}  \newcommand{\dateiname}{L03119}\newcommand{\titel}{Felix Salten an Arthur Schnitzler, [7. 2. 1893?]}\newcommand{\editorInnen}{Martin Anton Müller und Laura Untner}%% latex-leseansicht-abspann.tex
%% Abspann für die Leseansicht.
%% Der Schalter \ifkorrekturansicht ist bereits durch den Vorspann gesetzt.

%% latex-abspann.tex
%% Gemeinsamer Abspann für Korrekturansicht und Leseansicht.
%% Setzt den Schalter \ifkorrekturansicht voraus (gesetzt in den
%% einbindenden Dateien latex-korrekturansicht-abspann.tex bzw.
%% latex-leseansicht-abspann.tex).
%% ---------------------------------------------------------------

\normalsize

% Das esempio-Environment wird nur in der Leseansicht benötigt
\ifkorrekturansicht\else
\newenvironment{esempio}[3]%
{
    \vspace{1.5ex}
    \rlap{\underline{#1}}
    \par
    \setlength{\parindent}{0cm}
    \nopagebreak
    \leftskip=#2cm
    \rightskip=#3cm
}
{
    \par
}
\fi

\doendnotes{C}
\bigskip
\vfill

\clearpage

\footnotesize

\ifkorrekturansicht
  \lohead{\textsc{register}}
\fi

% theindex-Environment neu definieren ohne reledmac
\makeatletter
\renewenvironment{theindex}{%
  \ifkorrekturansicht
    \section*{\indexname}%
  \else
    \subsubsection*{Index der erwähnten Entitäten}%
  \fi
  \setlength{\parindent}{0pt}%
  \setlength{\parskip}{0pt plus 0.3pt}%
  \let\item\@idxitem
}{%
  \ifkorrekturansicht\clearpage\fi
}
\makeatother

\IfFileExists{\jobname-pw.ind}{\input{\jobname-pw.ind}}{}

% Quellenangabe nur in der Leseansicht
\ifkorrekturansicht\else
% Fallback-Definitionen, falls die .tex-Datei \titel etc. nicht gesetzt hat
\providecommand{\titel}{}
\providecommand{\editorInnen}{}
\providecommand{\dateiname}{\jobname}

\vspace{3cm}

\vfill

\footnotesize
\textsc{Quelle}: \titel. Herausgegeben von {\editorInnen}. In: \emph{Arthur Schnitzler: Briefwechsel mit Autorinnen und Autoren}.
 Digitale Edition, https://schnitzler-briefe.acdh.oeaw.ac.at/{\dateiname}.html (Stand \today)
\fi

\end{document}


