%% latex-leseansicht-vorspann.tex
%% Vorspann für die Leseansicht.
%% Lädt die gemeinsame Datei latex-vorspann.tex mit nicht gesetztem Schalter.

\newif\ifkorrekturansicht
\korrekturansichtfalse

\input{../tex-inputs/latex-vorspann}


\section[Hugo von Hofmannsthal an Arthur Schnitzler, 1. 4. {[}1909{]}]{L01837 Hugo von Hofmannsthal an Arthur Schnitzler, 1. 4. [1909]}
\nopagebreak\mylabel{L01837v}
\rehead{ }\normalsize\beginnumbering\briefempfaengerindex{Schnitzler, Arthur@\textsc{Schnitzler, Arthur}!zzzHofmannsthal, Hugo von@\emph{von Hugo von Hofmannsthal}!1909-04-011@{1. 4. [1909]}|(be}
\toendnotes[C]{\smallbreak\pagebreak[2]}
\correspDesc{Versand  durch Hugo von Hofmannsthal am 1. 4. [1909] in Rodaun
\newline{}Erhalt  durch Arthur Schnitzler im Zeitraum [2. 4. 1909
                  – 6. 4. 1909?] in Wien}\toendnotes[C]{\smallbreak}
\Standort{CUL, Schnitzler, B 43.}
\physDesc{Brief, 2 Blätter, 6 Seiten, 1973 Zeichen
\newline{}Handschrift: schwarze Tinte, deutsche Kurrent
\newline{}Schnitzler: mit Bleistift die Jahreszahl ergänzt: »09« und beschriftet: »Hofmannsthal« 
\newline{}Ordnung: 1) mit Bleistift von unbekannter Hand nummeriert: »\strikeout{300}«  2) mit Bleistift von unbekannter Hand nummeriert:
                                    »304«}
\buchAbdrucke{\weitereDrucke{Hugo von Hofmannsthal, Arthur Schnitzler: \emph{Briefwechsel}. Herausgegeben von Therese Nickl und Heinrich Schnitzler. Frankfurt am Main: \emph{S. Fischer} 1964, S. 244.} }\toendnotes[C]{\smallbreak}
\pstart
           \raggedleft{}{\pb}1 IV.{ }Rodaun\oindex{Wien@\textbf{Wien}!XXIII., Liesing@\textbf{XXIII., Liesing}!Rodaun@\textbf{Rodaun}, \emph{Region}|pw}\pend
           
\pstart{}mein lieber Arthur\pend\vspace{0.5em}
\pstart
           ich danke Ihnen{ }ſehr für Ihre guten Worte über Elektra\pwindex{Hofmannsthal, Hugo von 1.\,2.\,1874 Wien – 15.\,7.\,1929 Rodaun@\textsc{Hofmannsthal, Hugo von} (1.\,2.\,1874 Wien – 15.\,7.\,1929 Rodaun), \emph{Schriftsteller}!Elektra. Tragödie in einem Aufzug@\strich\emph{Elektra. Tragödie in einem Aufzug}|pw}. Dies iſt die reinſte Freude, von einem Menſchen, den man{ }ſo gern
               hat. Ich habe Ihre Arbeiten immer gern gehabt, aber erſt in den letzten 4–5 Jahren
               iſt mir eigentlich der Knopf für ihren ganzen Wert aufgegangen {\pb}und{ }ſeitdem habe ich mir
               angewöhnt,{ }ſie mit{ }ſo großer Freude wiederholt zu leſen.\pend
           
\pstart
           Es iſt mir{ }ſehr hart, Sie{ }ſo gar{ }ſelten zu{ }ſehen. Nie habe ich eine Stunde mit Ihnen
               verbracht, die nicht von einem ganz beſti{\geminationm}ten poſitiven
               Wohlgefühl, mehr noch des Gemütes als des Geiſtes begleitet geweſen wäre.\hspace*{1.5em}Ich denke daran, {\pb}wenn Sie Ende Mai
               nach Tirol\oindex{Tirol@\textbf{Tirol}, \emph{Land}|pw} fahren, um Wohnung zu{ }ſuchen,
               mitzufahren, auch ohne dieſen Zweck. – Es iſt nun bald zwanzig Jahre, daſs wir uns
               kennen.\pend
           
\pstart
           \numberlinefalse{}\centering{}–\numberlinetrue{}\pend
           
\pstart
           Die Gedichte\pwindex{Winterstein, Alfred von 25.\,9.\,1885 Wien – 28.\,4.\,1958 ebd.@\textsc{Winterstein, Alfred von} (25.\,9.\,1885 Wien – 28.\,4.\,1958 ebd.), \emph{Schriftsteller, Psychoanalytiker, Beamter}!Gedichte]@\strich\emph{[Gedichte]}|pwv} von Winterſtein\pwindex{Winterstein, Alfred von 25.\,9.\,1885 Wien – 28.\,4.\,1958 ebd.@\textsc{Winterstein, Alfred von} (25.\,9.\,1885 Wien – 28.\,4.\,1958 ebd.), \emph{Schriftsteller, Psychoanalytiker, Beamter}|pw} haben mir zum Teil{ }ſehr gut
               gefallen. Ohne allen Zweifel habe ich{ }ſie damals (vor Monaten) an Sie
               zurückgeſchickt, denn ich bin in dieſem Punkt{ }ſehr {\pb}genau und an dem einzigen Platz,
               wo{ }ſie liegen könnten, liegen{ }ſie nicht mehr. – Es{ }ſchien mir eine feine, aber{ }ſchwache Perſönlichkeit{ }ſich zu äußern. –\pend
           
\pstart
           Betreffs Elektra\pwindex{Hofmannsthal, Hugo von 1.\,2.\,1874 Wien – 15.\,7.\,1929 Rodaun@\textsc{Hofmannsthal, Hugo von} (1.\,2.\,1874 Wien – 15.\,7.\,1929 Rodaun), \emph{Schriftsteller}!Elektra. Tragödie in einem Aufzug@\strich\emph{Elektra. Tragödie in einem Aufzug}|pw},{ }ſo habe ich Fiſcher\pwindex{Fischer, Samuel 24.\,12.\,1859 Liptovský Mikuláš – 15.\,10.\,1934 Berlin@\textsc{Fischer, Samuel} (24.\,12.\,1859 Liptovský Mikuláš – 15.\,10.\,1934 Berlin), \emph{Verleger}|pw} nicht ohne Mühe veranlaßt,{ }ſeine \uline{Verlagsrechte} an Fürſtner\pwindex{Fürstner, Otto 17.\,10.\,1886 Berlin – 18.\,6.\,1958 London@\textsc{Fürstner, Otto} (17.\,10.\,1886 Berlin – 18.\,6.\,1958 London), \emph{Musikverleger}|pw} abzutreten. Hiefür bezahle ich an Fiſcher\pwindex{Fischer, Samuel 24.\,12.\,1859 Liptovský Mikuláš – 15.\,10.\,1934 Berlin@\textsc{Fischer, Samuel} (24.\,12.\,1859 Liptovský Mikuláš – 15.\,10.\,1934 Berlin), \emph{Verleger}|pw} die Hälfte der von Fürſtner\pwindex{Fürstner, Otto 17.\,10.\,1886 Berlin – 18.\,6.\,1958 London@\textsc{Fürstner, Otto} (17.\,10.\,1886 Berlin – 18.\,6.\,1958 London), \emph{Musikverleger}|pw}
               mir zufließenden 25{\%}. D. h. von 10000 Exemplaren bekomme ich
                  1250\strikeout{0} Mark, Fiſcher\pwindex{Fischer, Samuel 24.\,12.\,1859 Liptovský Mikuláš – 15.\,10.\,1934 Berlin@\textsc{Fischer, Samuel} (24.\,12.\,1859 Liptovský Mikuláš – 15.\,10.\,1934 Berlin), \emph{Verleger}|pw} das gleiche.\pend
           
\pstart
           Ihr{\\[\baselineskip]}\spacefill\mbox{Hugo}\pend
           \leftskip=0em{}
\pstart
           \noindent{}{\pb}\textsc{P. S.}\label{K_L01837-1v}\edtext{In 14 Tagen}{\lemma{\textnormal{\emph{In 14 Tagen}}}\Cendnote{\textnormal{Vgl. A. S.: \emph{Tagebuch}, 16. 10. 1909.
                  }}}\label{K_L01837-1}{ }ſpielt die \textsc{Després}\pwindex{Desprès, Suzanne 18.\,12.\,1875 Verdun – 29.\,6.\,1951 Paris@\textsc{Desprès, Suzanne} (18.\,12.\,1875 Verdun – 29.\,6.\,1951 Paris), \emph{Schauspielerin}|pw} hier die \textsc{Elektra}\pwindex{Hofmannsthal, Hugo von 1.\,2.\,1874 Wien – 15.\,7.\,1929 Rodaun@\textsc{Hofmannsthal, Hugo von} (1.\,2.\,1874 Wien – 15.\,7.\,1929 Rodaun), \emph{Schriftsteller}!Elektra [op. 58]@\strich\emph{Elektra [op. 58]}|pw}. Referent über{ }ſolche \label{K_L01837-2v}\edtext{Dinge}{\lemma{\textnormal{\emph{Dinge}}}\Cendnote{\textnormal{Die Besprechung des
                     Gastspiels ist nicht gezeichnet und äußert sich nicht explizit zu \emph{Elektra}\pwindex{Hofmannsthal, Hugo von 1.\,2.\,1874 Wien – 15.\,7.\,1929 Rodaun@\textsc{Hofmannsthal, Hugo von} (1.\,2.\,1874 Wien – 15.\,7.\,1929 Rodaun), \emph{Schriftsteller}!Elektra [op. 58]@\strich\emph{Elektra [op. 58]}|pwk}, nennt aber den Auftritt von Desprès\pwindex{Desprès, Suzanne 18.\,12.\,1875 Verdun – 29.\,6.\,1951 Paris@\textsc{Desprès, Suzanne} (18.\,12.\,1875 Verdun – 29.\,6.\,1951 Paris), \emph{Schauspielerin}|pwk} im Stück das »künstlerische
                        Ereignis des Abends« (\emph{Gastspiel der Suzanne Després}\pwindex{Gastspiel Suzanne Després@\emph{Gastspiel Suzanne Després}|pwk}. In: \emph{Neue Freie Presse}\pwindex{Neue Freie Presse@\emph{Neue Freie Presse}|pwk}, Nr. 16.040,
                           17. 4. 1909, S. 12).}}}\label{K_L01837-2} iſt Auernheimer\pwindex{Auernheimer, Raoul 15.\,4.\,1876 Wien – 6.\,1.\,1948 Oakland@\textsc{Auernheimer, Raoul} (15.\,4.\,1876 Wien – 6.\,1.\,1948 Oakland), \emph{Schriftsteller, Journalist, Kritiker}|pw}. Nun iſt das ein anſtändiger und nicht
                  übelwollender Menſch und ich wäre wahrhaftig froh nicht durch eine unangenehme
                  Haltung{ }ſeinerſeits wiederum auch gegen dieſe Figur in die gewiſſe defenſive {\pb}Haltung gerathen zu müſſen. Ich
                  glaube daſs ein Geſpräch von 10 Minuten mit Ihnen hinreichen würde, ihm verſtehen
                  zu machen worin die Qualität des Stückes\pwindex{Hofmannsthal, Hugo von 1.\,2.\,1874 Wien – 15.\,7.\,1929 Rodaun@\textsc{Hofmannsthal, Hugo von} (1.\,2.\,1874 Wien – 15.\,7.\,1929 Rodaun), \emph{Schriftsteller}!Elektra. Tragödie in einem Aufzug@\strich\emph{Elektra. Tragödie in einem Aufzug}|pwv} liegt, – glaube aber auch daſs er ohne dieſes Geſpräch \uline{nicht} auf dem \textsc{niveau} iſt,{ }ſich zu dem Stück in ein loyales Verhältnis zu{ }ſetzen, beſonders in{ }ſeine
                  Atmoſphäre. Vielleicht finden Sie die Gelegenheit. –\pend
           \selectlanguage{ngerman}\endnumbering\briefempfaengerindex{Schnitzler, Arthur@\textsc{Schnitzler, Arthur}!zzzHofmannsthal, Hugo von@\emph{von Hugo von Hofmannsthal}!1909-04-011@{1. 4. [1909]}|)be}\mylabel{L01837h}  \newcommand{\dateiname}{L01837}\newcommand{\titel}{Hugo von Hofmannsthal an Arthur Schnitzler, 1. 4. [1909]}\newcommand{\editorInnen}{Martin Anton Müller und Gerd-Hermann Susen}%% latex-leseansicht-abspann.tex
%% Abspann für die Leseansicht.
%% Der Schalter \ifkorrekturansicht ist bereits durch den Vorspann gesetzt.

%% latex-abspann.tex
%% Gemeinsamer Abspann für Korrekturansicht und Leseansicht.
%% Setzt den Schalter \ifkorrekturansicht voraus (gesetzt in den
%% einbindenden Dateien latex-korrekturansicht-abspann.tex bzw.
%% latex-leseansicht-abspann.tex).
%% ---------------------------------------------------------------

\normalsize

% Das esempio-Environment wird nur in der Leseansicht benötigt
\ifkorrekturansicht\else
\newenvironment{esempio}[3]%
{
    \vspace{1.5ex}
    \rlap{\underline{#1}}
    \par
    \setlength{\parindent}{0cm}
    \nopagebreak
    \leftskip=#2cm
    \rightskip=#3cm
}
{
    \par
}
\fi

\doendnotes{C}
\bigskip
\vfill

\clearpage

\footnotesize

\ifkorrekturansicht
  \lohead{\textsc{register}}
\fi

% theindex-Environment neu definieren ohne reledmac
\makeatletter
\renewenvironment{theindex}{%
  \ifkorrekturansicht
    \section*{\indexname}%
  \else
    \subsubsection*{Index der erwähnten Entitäten}%
  \fi
  \setlength{\parindent}{0pt}%
  \setlength{\parskip}{0pt plus 0.3pt}%
  \let\item\@idxitem
}{%
  \ifkorrekturansicht\clearpage\fi
}
\makeatother

\IfFileExists{\jobname-pw.ind}{\input{\jobname-pw.ind}}{}

% Quellenangabe nur in der Leseansicht
\ifkorrekturansicht\else
% Fallback-Definitionen, falls die .tex-Datei \titel etc. nicht gesetzt hat
\providecommand{\titel}{}
\providecommand{\editorInnen}{}
\providecommand{\dateiname}{\jobname}

\vspace{3cm}

\vfill

\footnotesize
\textsc{Quelle}: \titel. Herausgegeben von {\editorInnen}. In: \emph{Arthur Schnitzler: Briefwechsel mit Autorinnen und Autoren}.
 Digitale Edition, https://schnitzler-briefe.acdh.oeaw.ac.at/{\dateiname}.html (Stand \today)
\fi

\end{document}


