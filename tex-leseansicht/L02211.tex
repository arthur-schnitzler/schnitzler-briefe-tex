%% latex-korrekturansicht-vorspann.tex
%% Vorspann für die Korrekturansicht.
%% Lädt die gemeinsame Datei latex-vorspann.tex mit gesetztem Schalter.

\newif\ifkorrekturansicht
\korrekturansichttrue

\input{../tex-inputs/latex-vorspann}


\section[Richard Beer-Hofmann an Arthur Schnitzler, 29. 6. 1915]{L02211 Richard Beer-Hofmann an Arthur Schnitzler, 29. 6. 1915}
\nopagebreak\mylabel{L02211v}
\rehead{ }\normalsize\beginnumbering\briefempfaengerindex{Schnitzler, Arthur@\textsc{Schnitzler, Arthur}!zzzBeer-Hofmann, Richard@\emph{von Richard Beer-Hofmann}!1915-06-292@{29. 6. 1915}|(be}
\toendnotes[C]{\smallbreak\pagebreak[2]}\Standort{CUL, Schnitzler, B 8.}
\physDesc{Bildpostkarte, 345 Zeichen
\newline{}Handschrift: Bleistift, lateinische Kurrent
\newline{}Versand: Stempel: »\nobreak{}\oindex{Bad Ischl@\textbf{Bad Ischl}, \emph{P.PPL}|pwk}Bad Ischl 1, 29 VI 15, 2\nobreak{}«.  
\newline{}Ordnung: mit Bleistift von unbekannter Hand nummeriert:
                                    »260« }
\buchAbdrucke{\weitereDrucke{Arthur Schnitzler, Richard Beer-Hofmann: \emph{Briefwechsel 1891–1931}. Wien, Zürich: \emph{Europaverlag} 1992, S. 221.} }\toendnotes[C]{\smallbreak}\pstart{}{\pb}S. H.\pend{}\pstart{}Herrn\pend{}\pstart{}D\textsuperscript{r} Arthur Schnitzler\pend{}\pstart{}Wien XVIII\oindex{XVIII., Waehring@\textbf{XVIII., Währing}, \emph{A.ADM3}|pw}\pend{}\pstart{}Sternwartestrasse 71\oindex{Sternwartestrasse 71@\textbf{Sternwartestraße 71}, \emph{Wohngebäude (K.WHS)}|pw}\pend{}{\bigskip}
\pstart
           \noindent{}\centering{}{\pb}\textcolor{gray}{\textbf{Salzkammergut\oindex{Salzkammergut@\textbf{Salzkammergut}, \emph{L.RGN}|pw}.\hspace*{1.5em}Bad Ischl\oindex{Bad Ischl@\textbf{Bad Ischl}, \emph{P.PPL}|pw}.}}\pend
           \vspace{1em}
\pstart
           \raggedleft{}{\pb}29/VI 15\pend
           \vspace{0.5em}
\pstart
           Lieber Arthur! Ich wollte zu Ihnen, aber Kaufmann\pwindex{Kaufmann, Arthur 04.04.1872 – 25.07.1938@\textsc{Kaufmann, Arthur} (04.04.1872 – 25.07.1938), \emph{Rechtswissenschaftler/Rechtswissenschaftlerin, Privatgelehrte/Privatgelehrte, Privatier/Privatière}|pw} sagte mir, Sie wären auf dem Se{\geminationm}ering\oindex{Semmering@\textbf{Semmering}, \emph{A.ADM3}|pw}. So wünsche ich Ihnen und
                  Olga\pwindex{Schnitzler, Olga 17.01.1882 – 13.01.1970@\textsc{Schnitzler, Olga} (17.01.1882 – 13.01.1970), \emph{Schauspieler/Schauspielerin, Sänger/Sängerin}|pw} den schönsten So{\geminationm}er – bringt er Sie nicht doch noch hieher? Bitte
               schreiben Sie mir gelegentlich D\textsuperscript{r}Reiks\pwindex{Reik, Theodor 12.05.1888 – 31.12.1969@\textsc{Reik, Theodor} (12.05.1888 – 31.12.1969), \emph{Psychoanalytiker/Psychoanalytikerin}|pw} Adresse ich muss ihm noch für einen zugesandten \label{K_L02211-1v}\edtext{Aufsatz}{\lemma{\textnormal{\emph{Aufsatz}}}\Cendnote{\textnormal{nicht ermittelt}}}\label{K_L02211-1} danken\pend
           \selectlanguage{ngerman}\endnumbering\briefempfaengerindex{Schnitzler, Arthur@\textsc{Schnitzler, Arthur}!zzzBeer-Hofmann, Richard@\emph{von Richard Beer-Hofmann}!1915-06-292@{29. 6. 1915}|)be}\mylabel{L02211h}  \normalsize

\doendnotes{C}
\bigskip
\vfill

\clearpage

\footnotesize

\lohead{\textsc{register}}

% Definiere theindex-Environment komplett neu ohne reledmac
\makeatletter
\renewenvironment{theindex}{%
  \section*{\indexname}%
  \setlength{\parindent}{0pt}%
  \setlength{\parskip}{0pt plus 0.3pt}%
  \let\item\@idxitem
}{%
  \clearpage
}
\makeatother

\IfFileExists{\jobname-pw.ind}{\input{\jobname-pw.ind}}{}

\end{document}

      