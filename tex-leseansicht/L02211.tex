%% latex-leseansicht-vorspann.tex
%% Vorspann für die Leseansicht.
%% Lädt die gemeinsame Datei latex-vorspann.tex mit nicht gesetztem Schalter.

\newif\ifkorrekturansicht
\korrekturansichtfalse

\input{../tex-inputs/latex-vorspann}


\section[Richard Beer-Hofmann an Arthur Schnitzler, 29. 6. 1915]{L02211 Richard Beer-Hofmann an Arthur Schnitzler, 29. 6. 1915}
\nopagebreak\mylabel{L02211v}
\rehead{ }\normalsize\beginnumbering\briefempfaengerindex{Schnitzler, Arthur@\textsc{Schnitzler, Arthur}!zzzBeer-Hofmann, Richard@\emph{von Richard Beer-Hofmann}!1915-06-292@{29. 6. 1915}|(be}
\toendnotes[C]{\smallbreak\pagebreak[2]}
\correspDesc{Versand  durch Richard Beer-Hofmann am 29. 6. 1915 in Bad Ischl
\newline{}Erhalt  durch Arthur Schnitzler im Zeitraum [30. 6. 1915
                  – 4. 7. 1915?] in Wien}\toendnotes[C]{\smallbreak}
\Standort{CUL, Schnitzler, B 8.}
\physDesc{Bildpostkarte, 345 Zeichen
\newline{}Handschrift: Bleistift, lateinische Kurrent
\newline{}Versand: Stempel: »\nobreak{}\oindex{Bad Ischl@\textbf{Bad Ischl}|pwk}Bad Ischl 1, 29 VI 15, 2\nobreak{}«.  
\newline{}Ordnung: mit Bleistift von unbekannter Hand nummeriert: »260« }
\buchAbdrucke{\weitereDrucke{Arthur Schnitzler, Richard Beer-Hofmann: \emph{Briefwechsel 1891–1931}. Herausgegeben von Konstanze Fliedl. Wien, Zürich: \emph{Europaverlag} 1992, S. 221.} }\toendnotes[C]{\smallbreak}\pstart{}{\pb}S. H.\pend{}\pstart{}Herrn\pend{}\pstart{}D\textsuperscript{r} Arthur Schnitzler\pend{}\pstart{}Wien XVIII\oindex{XVIII., Währing@\textbf{XVIII., Währing}, \emph{Verwaltungsgebiet}|pw}\pend{}\pstart{}Sternwartestrasse 71\oindex{Wien@\textbf{Wien}!XVIII., Währing@\textbf{XVIII., Währing}!Sternwartestraße 71@\textbf{Sternwartestraße 71}, \emph{Wohngebäude}|pw}\pend{}{\bigskip}
\pstart
           \noindent{}\centering{}{\pb}\textcolor{gray}{\textbf{Salzkammergut\oindex{Salzkammergut@\textbf{Salzkammergut}, \emph{Region}|pw}.\hspace*{1.5em}Bad Ischl\oindex{Bad Ischl@\textbf{Bad Ischl}|pw}.}}\pend
           \vspace{1em}
\pstart
           \raggedleft{}{\pb}29/VI 15\pend
           \vspace{0.5em}
\pstart
           Lieber Arthur! Ich wollte zu Ihnen, aber Kaufmann\pwindex{Kaufmann, Arthur 4.\,4.\,1872 Iași – 25.\,7.\,1938 Wien@\textsc{Kaufmann, Arthur} (4.\,4.\,1872 Iași – 25.\,7.\,1938 Wien), \emph{Rechtswissenschaftler, Privatgelehrte, Privatier}|pw} sagte mir, Sie wären auf dem Se{\geminationm}ering\oindex{Semmering@\textbf{Semmering}, \emph{Verwaltungsgebiet}|pw}. So wünsche ich Ihnen und
                  Olga\pwindex{Schnitzler, Olga 17.\,1.\,1882 Wien – 13.\,1.\,1970 Lugano@\textsc{Schnitzler, Olga} (17.\,1.\,1882 Wien – 13.\,1.\,1970 Lugano), \emph{Schauspielerin, Sängerin}|pw} den schönsten So{\geminationm}er – bringt er Sie nicht doch noch hieher? Bitte
               schreiben Sie mir gelegentlich D\textsuperscript{r}{ }Reiks\pwindex{Reik, Theodor 12.\,5.\,1888 Wien – 31.\,12.\,1969 New York City@\textsc{Reik, Theodor} (12.\,5.\,1888 Wien – 31.\,12.\,1969 New York City), \emph{Psychoanalytiker}|pw} Adresse ich muss ihm noch für einen
               zugesandten \label{K_L02211-1v}\edtext{Aufsatz}{\lemma{\textnormal{\emph{Aufsatz}}}\Cendnote{\textnormal{nicht ermittelt}}}\label{K_L02211-1} danken\pend
           \selectlanguage{ngerman}\endnumbering\briefempfaengerindex{Schnitzler, Arthur@\textsc{Schnitzler, Arthur}!zzzBeer-Hofmann, Richard@\emph{von Richard Beer-Hofmann}!1915-06-292@{29. 6. 1915}|)be}\mylabel{L02211h}  \newcommand{\dateiname}{L02211}\newcommand{\titel}{Richard Beer-Hofmann an Arthur Schnitzler, 29. 6. 1915}\newcommand{\editorInnen}{Martin Anton Müller und Gerd-Hermann Susen}%% latex-leseansicht-abspann.tex
%% Abspann für die Leseansicht.
%% Der Schalter \ifkorrekturansicht ist bereits durch den Vorspann gesetzt.

%% latex-abspann.tex
%% Gemeinsamer Abspann für Korrekturansicht und Leseansicht.
%% Setzt den Schalter \ifkorrekturansicht voraus (gesetzt in den
%% einbindenden Dateien latex-korrekturansicht-abspann.tex bzw.
%% latex-leseansicht-abspann.tex).
%% ---------------------------------------------------------------

\normalsize

% Das esempio-Environment wird nur in der Leseansicht benötigt
\ifkorrekturansicht\else
\newenvironment{esempio}[3]%
{
    \vspace{1.5ex}
    \rlap{\underline{#1}}
    \par
    \setlength{\parindent}{0cm}
    \nopagebreak
    \leftskip=#2cm
    \rightskip=#3cm
}
{
    \par
}
\fi

\doendnotes{C}
\bigskip
\vfill

\clearpage

\footnotesize

\ifkorrekturansicht
  \lohead{\textsc{register}}
\fi

% theindex-Environment neu definieren ohne reledmac
\makeatletter
\renewenvironment{theindex}{%
  \ifkorrekturansicht
    \section*{\indexname}%
  \else
    \subsubsection*{Index der erwähnten Entitäten}%
  \fi
  \setlength{\parindent}{0pt}%
  \setlength{\parskip}{0pt plus 0.3pt}%
  \let\item\@idxitem
}{%
  \ifkorrekturansicht\clearpage\fi
}
\makeatother

\IfFileExists{\jobname-pw.ind}{\input{\jobname-pw.ind}}{}

% Quellenangabe nur in der Leseansicht
\ifkorrekturansicht\else
% Fallback-Definitionen, falls die .tex-Datei \titel etc. nicht gesetzt hat
\providecommand{\titel}{}
\providecommand{\editorInnen}{}
\providecommand{\dateiname}{\jobname}

\vspace{3cm}

\vfill

\footnotesize
\textsc{Quelle}: \titel. Herausgegeben von {\editorInnen}. In: \emph{Arthur Schnitzler: Briefwechsel mit Autorinnen und Autoren}.
 Digitale Edition, https://schnitzler-briefe.acdh.oeaw.ac.at/{\dateiname}.html (Stand \today)
\fi

\end{document}


