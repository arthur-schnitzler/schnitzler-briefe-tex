%% latex-korrekturansicht-vorspann.tex
%% Vorspann für die Korrekturansicht.
%% Lädt die gemeinsame Datei latex-vorspann.tex mit gesetztem Schalter.

\newif\ifkorrekturansicht
\korrekturansichttrue

\input{../tex-inputs/latex-vorspann}


\section[Arthur Schnitzler an Richard Beer-Hofmann, 16. 7. 1899]{L00945 Arthur Schnitzler an Richard Beer-Hofmann, 16. 7. 1899}
\nopagebreak\mylabel{L00945v}
\rehead{ }\normalsize\beginnumbering\briefempfaengerindex{Beer-Hofmann, Richard@\textsc{Beer-Hofmann, Richard}!zzzSchnitzler, Arthur@\emph{von Arthur Schnitzler}!1899-07-161@{16. 7. 1899}|(be}
\toendnotes[C]{\smallbreak\pagebreak[2]}\Standort{YCGL, MSS 31.}
\physDesc{Brief, 1 Blatt, 4 Seiten, Umschlag, 547 Zeichen
\newline{}Handschrift: Bleistift, deutsche Kurrent
\newline{}Versand: 1) Stempel: »\nobreak{}Wien, 16. 7. {[}1899{]}, 5–6N\nobreak{}«.   2) Stempel: »\nobreak{}\oindex{Seeboden@\textbf{Seeboden}, \emph{A.ADM3}|pwk}{\pb}\textcolor{gray}{Seeb}oden, 17. 7. 99\nobreak{}«. }
\buchAbdrucke{\weitereDrucke{Arthur Schnitzler, Richard Beer-Hofmann: \emph{Briefwechsel 1891–1931}. Wien, Zürich: \emph{Europaverlag} 1992, S. 132–133.} }\pstart{}{\pb}\textsc{Herrn Dr Rich Beer-Hofmann}\pend{}\pstart{}\textsc{Seeboden am Millstätter}ſee\oindex{Seeboden@\textbf{Seeboden}, \emph{A.ADM3}|pw}\pend{}\pstart{}\textsc{Villa Platzer}\oindex{Villa Platzer@\textbf{Villa Platzer}, \emph{Gebäude (K.GBD)}|pw}\pend{}\pstart{}\textsc{Kärnthen}\oindex{Kaernten@\textbf{Kärnten}, \emph{A.ADM1}|pw}\pend{}{\bigskip}\vspace{1em}
\pstart
           \raggedleft{}{\pb}16/7 99\pend
           \vspace{0.5em}
\pstart
           Lieber Richard, ich will Dinſtg{ }Früh in \textsc{Velden}\oindex{Velden am Woerthersee@\textbf{Velden am Wörthersee}, \emph{P.PPL}|pw}, \textsc{Pension Pundschu}\oindex{Pension Pundschu@\textbf{Pension Pundschu}, \emph{Hotel (K.HTL)}|pw} eintreffen. Schreiben Sie mir dann, wa{\geminationn} Sie zu mir
               oder ich zu Ihnen ko{\geminationm}en ſoll. {\pb}Wollen Sie früher mit Ihrer Arbeit fertig ſein, ſo
               ſchreiben Sie mir eben, wann Sie fertig sind.\pend
           
\pstart
           \textsc{Bayreuth}\oindex{Bayreuth@\textbf{Bayreuth}, \emph{P.PPLA2}|pw} wird kaum {\pb}was zu bekommen ſein.\pend
           
\pstart
           Bin ich Ende Juli{ }ſchon in jener Gegend, ſo ko{\geminationm} ich kaum mehr nach Kärnthen\oindex{Kaernten@\textbf{Kärnten}, \emph{A.ADM1}|pw}, \textsc{resp}. Tirol\oindex{Tirol@\textbf{Tirol}, \emph{A.ADM1}|pw} zurück. Im übrigen all das läßt sich mündlich {\pb}beſſer beſprechen.\pend
           \pstart Herzlich Ihr \spacefill\mbox{Arth}\pend{}
\pstart
           \textsc{Wasserma{\geminationn}}\pwindex{Wassermann, Jakob 10.03.1873 – 01.01.1934@\textsc{Wassermann, Jakob} (10.03.1873 – 01.01.1934), \emph{Schriftsteller/Schriftstellerin}|pw} kommt Mittwoch nach Velden\oindex{Velden am Woerthersee@\textbf{Velden am Wörthersee}, \emph{P.PPL}|pw}.\pend
           \selectlanguage{ngerman}\endnumbering\briefempfaengerindex{Beer-Hofmann, Richard@\textsc{Beer-Hofmann, Richard}!zzzSchnitzler, Arthur@\emph{von Arthur Schnitzler}!1899-07-161@{16. 7. 1899}|)be}\mylabel{L00945h}  \normalsize

\doendnotes{C}
\bigskip
\vfill

\clearpage

\footnotesize

\lohead{\textsc{register}}

% Definiere theindex-Environment komplett neu ohne reledmac
\makeatletter
\renewenvironment{theindex}{%
  \section*{\indexname}%
  \setlength{\parindent}{0pt}%
  \setlength{\parskip}{0pt plus 0.3pt}%
  \let\item\@idxitem
}{%
  \clearpage
}
\makeatother

\IfFileExists{\jobname-pw.ind}{\input{\jobname-pw.ind}}{}

\end{document}

      