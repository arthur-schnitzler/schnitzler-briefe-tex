%% latex-leseansicht-vorspann.tex
%% Vorspann für die Leseansicht.
%% Lädt die gemeinsame Datei latex-vorspann.tex mit nicht gesetztem Schalter.

\newif\ifkorrekturansicht
\korrekturansichtfalse

\input{../tex-inputs/latex-vorspann}


         
         \newcommand{\erwaehntePersonen}{Personen: }
         \newcommand{\erwaehnteInstitutionen}{}
         \newcommand{\erwaehnteOrte}{}
         \newcommand{\erwaehnteWerke}{
               \section[Arthur Schnitzler an Richard Beer-Hofmann, 15. 7. 1898]{ Arthur Schnitzler an Richard Beer-Hofmann, 15. 7. 1898}\nopagebreak\mylabel{v}\rehead{ }\begin{ledgroupsized}[t]{13cm}\normalsize\beginnumbering \toendnotes[C]{\smallbreak\pagebreak[2]} \Standort{YCGL, MSS 31.}
\physDesc{Brief, 1 Blatt, 2 Seiten, Umschlag
\newline{}Handschrift: 1) Bleistift, deutsche Kurrent\hspace{1em}2) schwarze Tinte, deutsche Kurrent (\noindent{}Umschlag)\hspace{1em}\newline{}Versand: 1) Stempel: »\nobreak{}\oindex{XXXX Ortsangabe fehlt|pwk}Graz, 15/7 98, 7.A\nobreak{}«.   2) Stempel: »\nobreak{}\oindex{XXXX Ortsangabe fehlt|pwk}\textcolor{gray}{Steindorf} am Ossiacher
                              See, 16{[} 7 98{]}\nobreak{}«. }\buchAbdrucke{\weitereDrucke{Arthur Schnitzler, Richard Beer-Hofmann: \emph{Briefwechsel 1891–1931}. Hg. Konstanze Fliedl. Wien, Zürich: \emph{Europaverlag} 1992, S. 123.} }\toendnotes[C]{\smallbreak}\pstart{}{\pb}Dr. \textsc{Arthur \damage{\textcolor{gray}{Schnitz}}ler}, Wien IX.
                  Frankgaſſe 1\oindex{XXXX Ortsangabe fehlt|pw}.\pend{}{\bigskip}\pstart{}{\pb}Herrn \textsc{Dr. Richard
                     Beer-Hofmann}\pend{}\pstart{}\textsc{Steindorf\oindex{XXXX Ortsangabe fehlt|pw}}\pend{}\pstart{}\textsc{am Ossiacher\oindex{XXXX Ortsangabe fehlt|pw}-See}\pend{}\pstart{}Kärnthen\oindex{XXXX Ortsangabe fehlt|pw}.\pend{}{\bigskip}\pstart
           \raggedleft{}{\pb}Graz\oindex{XXXX Ortsangabe fehlt|pw}{ }15/7 98\pend
           \pstart
           Mein lieber Richard, So{\geminationn}tag den
                  17. verlaſſe ich Graz\oindex{XXXX Ortsangabe fehlt|pw}, komme auf mancherlei
               Art am 21. nach \textsc{\uline{Bad Gastein, Villa Wassing}}\oindex{XXXX Ortsangabe fehlt|pw}, zu meiner Mama\pwindex{\textcolor{red}{\textsuperscript{XXXX1 indx}}|pwv}, wo ich
               bis 23. bleibe und ein Wort von Ihnen erwarte. Radle dann nach Salzburg\oindex{XXXX Ortsangabe fehlt|pw}, bin ſpäteſtens Dinſtag 26. dort
               und bleibe bis 28; radle da{\geminationn} (in Geſellſchaft) {\pb}nach Tegernſee\oindex{XXXX Ortsangabe fehlt|pw}. Hugo\pwindex{\textcolor{red}{\textsuperscript{XXXX1 indx}}|pw} hat Ihnen geſchrieben – werden wir uns alſo
               am 9. Auguſt circa irgendwo treffen, um \substVorne{}\textsuperscript{\textcolor{gray}{b}}\substDazwischen{}a\substHinten{}uf 10 Tage mindeſtens zuſa{\geminationm}en zu bleiben? Machen Sie’s doch möglich. Können Sie
               zwiſchen 23 u 26. d. nach Salzburg\oindex{XXXX Ortsangabe fehlt|pw} kommen? – Arbeiten Sie was?\pend
           \pstart
           Grüßen Sie Paula\pwindex{\textcolor{red}{\textsuperscript{XXXX1 indx}}|pw} und Mirjam\pwindex{\textcolor{red}{\textsuperscript{XXXX1 indx}}|pw}.\pend
           \pstart Herzlichſt Ihr \spacefill\mbox{Arthur}\pend{}
         
         \endnumbering\mylabel{h}\end{ledgroupsized}  \newcommand{\dateiname}{L00819}\newcommand{\titel}{Arthur Schnitzler an Richard Beer-Hofmann, 15. 7. 1898}\newcommand{\editorInnen}{Martin Anton Müller und Gerd-Hermann Susen}%% latex-leseansicht-abspann.tex
%% Abspann für die Leseansicht.
%% Der Schalter \ifkorrekturansicht ist bereits durch den Vorspann gesetzt.

%% latex-abspann.tex
%% Gemeinsamer Abspann für Korrekturansicht und Leseansicht.
%% Setzt den Schalter \ifkorrekturansicht voraus (gesetzt in den
%% einbindenden Dateien latex-korrekturansicht-abspann.tex bzw.
%% latex-leseansicht-abspann.tex).
%% ---------------------------------------------------------------

\normalsize

% Das esempio-Environment wird nur in der Leseansicht benötigt
\ifkorrekturansicht\else
\newenvironment{esempio}[3]%
{
    \vspace{1.5ex}
    \rlap{\underline{#1}}
    \par
    \setlength{\parindent}{0cm}
    \nopagebreak
    \leftskip=#2cm
    \rightskip=#3cm
}
{
    \par
}
\fi

\doendnotes{C}
\bigskip
\vfill

\clearpage

\footnotesize

\ifkorrekturansicht
  \lohead{\textsc{register}}
\fi

% theindex-Environment neu definieren ohne reledmac
\makeatletter
\renewenvironment{theindex}{%
  \ifkorrekturansicht
    \section*{\indexname}%
  \else
    \subsubsection*{Index der erwähnten Entitäten}%
  \fi
  \setlength{\parindent}{0pt}%
  \setlength{\parskip}{0pt plus 0.3pt}%
  \let\item\@idxitem
}{%
  \ifkorrekturansicht\clearpage\fi
}
\makeatother

\IfFileExists{\jobname-pw.ind}{\input{\jobname-pw.ind}}{}

% Quellenangabe nur in der Leseansicht
\ifkorrekturansicht\else
% Fallback-Definitionen, falls die .tex-Datei \titel etc. nicht gesetzt hat
\providecommand{\titel}{}
\providecommand{\editorInnen}{}
\providecommand{\dateiname}{\jobname}

\vspace{3cm}

\vfill

\footnotesize
\textsc{Quelle}: \titel. Herausgegeben von {\editorInnen}. In: \emph{Arthur Schnitzler: Briefwechsel mit Autorinnen und Autoren}.
 Digitale Edition, https://schnitzler-briefe.acdh.oeaw.ac.at/{\dateiname}.html (Stand \today)
\fi

\end{document}


      