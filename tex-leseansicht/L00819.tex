%% latex-korrekturansicht-vorspann.tex
%% Vorspann für die Korrekturansicht.
%% Lädt die gemeinsame Datei latex-vorspann.tex mit gesetztem Schalter.

\newif\ifkorrekturansicht
\korrekturansichttrue

\input{../tex-inputs/latex-vorspann}


\section[Arthur Schnitzler an Richard Beer-Hofmann, 15. 7. 1898]{L00819 Arthur Schnitzler an Richard Beer-Hofmann, 15. 7. 1898}
\nopagebreak\mylabel{L00819v}
\rehead{ }\normalsize\beginnumbering\briefempfaengerindex{Beer-Hofmann, Richard@\textsc{Beer-Hofmann, Richard}!zzzSchnitzler, Arthur@\emph{von Arthur Schnitzler}!1898-07-151@{15. 7. 1898}|(be}
\toendnotes[C]{\smallbreak\pagebreak[2]}\Standort{YCGL, MSS 31.}
\physDesc{Brief, 1 Blatt, 2 Seiten, Umschlag, 679 Zeichen
\newline{}Handschrift: 1) Bleistift, deutsche Kurrent\hspace{1em}2) schwarze Tinte, deutsche Kurrent (\noindent{}Umschlag)\hspace{1em}
\newline{}Versand: 1) Stempel: »\nobreak{}\oindex{Graz@\textbf{Graz}, \emph{A.ADM2}|pwk}Graz, 15/7 98, 7.A\nobreak{}«.   2) Stempel: »\nobreak{}\oindex{Steindorf am Ossiacher See@\textbf{Steindorf am Ossiacher See}, \emph{A.ADM3}|pwk}\textcolor{gray}{Steindorf} am Ossiacher See, 16{[} 7 98{]}\nobreak{}«. }
\buchAbdrucke{\weitereDrucke{Arthur Schnitzler, Richard Beer-Hofmann: \emph{Briefwechsel 1891–1931}. Wien, Zürich: \emph{Europaverlag} 1992, S. 123.} }\toendnotes[C]{\smallbreak}\pstart{}{\pb}Dr. \textsc{Arthur \damage{\textcolor{gray}{Schnitz}}ler}, Wien IX.
                  Frankgaſſe 1\oindex{Frankgasse 1@\textbf{Frankgasse 1}, \emph{Wohngebäude (K.WHS)}|pw}.\pend{}{\bigskip}\pstart{}{\pb}Herrn \textsc{Dr. Richard
                     Beer-Hofmann}\pend{}\pstart{}\textsc{Steindorf\oindex{Steindorf am Ossiacher See@\textbf{Steindorf am Ossiacher See}, \emph{A.ADM3}|pw}}\pend{}\pstart{}\textsc{am Ossiacher\oindex{Ossiacher See@\textbf{Ossiacher See}, \emph{See (N.SEE)}|pw}-See}\pend{}\pstart{}Kärnthen\oindex{Kaernten@\textbf{Kärnten}, \emph{A.ADM1}|pw}.\pend{}{\bigskip}\vspace{1em}
\pstart
           \raggedleft{}{\pb}Graz\oindex{Graz@\textbf{Graz}, \emph{A.ADM2}|pw}{ }15/7 98\pend
           \vspace{0.5em}
\pstart
           Mein lieber Richard,{ }So{\geminationn}tag den 17. verlaſſe ich Graz\oindex{Graz@\textbf{Graz}, \emph{A.ADM2}|pw}, komme auf mancherlei Art am 21.
               nach \textsc{\uline{Bad Gastein, Villa Wassing}}\oindex{Villa Dr. Wassing@\textbf{Villa Dr. Wassing}, \emph{Sanatorium (K.SAN)}|pw}, zu meiner Mama\pwindex{Schnitzler, Louise 1840-07-08 – 1911-09-09@\textsc{Schnitzler, Louise} (1840-07-08 – 1911-09-09)|pwv}, wo
               ich bis 23. bleibe und ein Wort von Ihnen erwarte. Radle dann nach Salzburg\oindex{Salzburg@\textbf{Salzburg}, \emph{A.ADM2}|pw}, bin ſpäteſtens Dinſtag 26.
               dort und bleibe bis 28; radle da{\geminationn} (in
               Geſellſchaft) {\pb}nach Tegernſee\oindex{Tegernsee@\textbf{Tegernsee}, \emph{P.PPL}|pw}. Hugo\pwindex{Hofmannsthal, Hugo von 1874-02-01 – 1929-07-15@\textsc{Hofmannsthal, Hugo von} (1874-02-01 – 1929-07-15), \emph{Schriftsteller/Schriftstellerin}|pw} hat Ihnen geſchrieben
               – werden wir uns alſo am 9. Auguſt circa irgendwo treffen, um \substVorne{}\textsuperscript{\textcolor{gray}{b}}\substDazwischen{}a\substHinten{}uf 10 Tage mindeſtens zuſa{\geminationm}en zu bleiben? Machen
               Sie’s doch möglich. Können Sie zwiſchen 23 u 26. d. nach
                  Salzburg\oindex{Salzburg@\textbf{Salzburg}, \emph{A.ADM2}|pw} kommen? – Arbeiten Sie was?\pend
           
\pstart
           Grüßen Sie Paula\pwindex{Beer-Hofmann, Paula 25.02.1879 – 30.10.1939@\textsc{Beer-Hofmann, Paula} (25.02.1879 – 30.10.1939)|pw} und Mirjam\pwindex{Beer-Hofmann, Mirjam 04.09.1897 – 24.12.1984@\textsc{Beer-Hofmann, Mirjam} (04.09.1897 – 24.12.1984)|pw}.\pend
           \pstart Herzlichſt Ihr \spacefill\mbox{Arthur}\pend{}\selectlanguage{ngerman}\endnumbering\briefempfaengerindex{Beer-Hofmann, Richard@\textsc{Beer-Hofmann, Richard}!zzzSchnitzler, Arthur@\emph{von Arthur Schnitzler}!1898-07-151@{15. 7. 1898}|)be}\mylabel{L00819h}  \normalsize

\doendnotes{C}
\bigskip
\vfill

\clearpage

\footnotesize

\lohead{\textsc{register}}

% Definiere theindex-Environment komplett neu ohne reledmac
\makeatletter
\renewenvironment{theindex}{%
  \section*{\indexname}%
  \setlength{\parindent}{0pt}%
  \setlength{\parskip}{0pt plus 0.3pt}%
  \let\item\@idxitem
}{%
  \clearpage
}
\makeatother

\IfFileExists{\jobname-pw.ind}{\input{\jobname-pw.ind}}{}

\end{document}

      