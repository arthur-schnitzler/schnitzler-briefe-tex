%% latex-leseansicht-vorspann.tex
%% Vorspann für die Leseansicht.
%% Lädt die gemeinsame Datei latex-vorspann.tex mit nicht gesetztem Schalter.

\newif\ifkorrekturansicht
\korrekturansichtfalse

\input{../tex-inputs/latex-vorspann}


         
         \renewcommand{\erwaehntePersonen}{Personen: Julius Rodenberg}
         \renewcommand{\erwaehnteInstitutionen}{Institutionen: Deutsche Rundschau, Gebrüder Paetel Verlag}
         \renewcommand{\erwaehnteOrte}{Orte: Berlin, Lützowstraße, Margaretenstraße, Wien}
         \renewcommand{\erwaehnteWerke}{Werke: Frau Bertha Garlan. Roman}
               \section[Julius Rodenberg an Arthur Schnitzler, 23. 6. 1900]{ Julius Rodenberg an Arthur Schnitzler, 23. 6. 1900}\nopagebreak\mylabel{v}\rehead{ }\begin{ledgroupsized}[t]{13cm}\normalsize\beginnumbering \toendnotes[C]{\smallbreak\pagebreak[2]} \Standort{TMW, HS Schn 4/27/1.}
\physDesc{Brief, 1 Blatt, 2 Seiten, 1220 Zeichen
\newline{}Handschrift: schwarze Tinte, deutsche Kurrent
\newline{}Schnitzler: mit rotem Buntstift eine Unterstreichung }\toendnotes[C]{\smallbreak}\pstart
           \noindent{}\centering{}{\pb}\textcolor{gray}{\textbf{Deutsche Rundschau\orgindex{Deutsche Rundschau@Deutsche Rundschau|pw}}}\pend
           \pstart
           \noindent{}\textcolor{gray}{\textbf{Expedition u. Redaction:}}\hfill \textcolor{gray}{\textbf{Herausgeber:}}\pend
           \pstart
           \textcolor{gray}{\textbf{Gebrüder Paetel\orgindex{Gebrueder Paetel Verlag@Gebrüder Paetel Verlag|pw} in Berlin\oindex{Berlin@\textbf{Berlin}|pw}}}\hfill \textcolor{gray}{\textbf{Julius Rodenberg in Berlin\oindex{Berlin@\textbf{Berlin}|pw}}}\pend
           \pstart
           \textcolor{gray}{\textbf{W., Lützowstr. 7\oindex{Luetzowstrasse@\textbf{Lützowstraße}|pw}.}}\hfill \textcolor{gray}{\textbf{W., Margarethenstr. 1\oindex{Margaretenstrasse@\textbf{Margaretenstraße}|pw}.}}\pend
           \pstart
           \raggedleft{}\textbf{\textcolor{gray}{\textbf{Berlin W.\oindex{Berlin@\textbf{Berlin}|pw},}} den}{ }23. Juni 1900.\pend
           \pstart{}Hochgeehrter Herr Doctor!\pend\pstart
           Empfangen Sie meinen verbindlichſten Dank für Ihr freundliches Schreiben vom
                  21. d M. u. das darin enthaltene Anerbieten. Ich brauche Ihnen nicht
               zu ſagen, welchen Werth es für mich hat, Sie wißen es, wie ſehr ich mich freuen
               würde, endlich einmal eine Novelle\pwindex{Schnitzler, Arthur 15.05.1862 – 21.10.1931@\textsc{Schnitzler, Arthur} (15.05.1862 – 21.10.1931), \emph{Schriftsteller, Mediziner}!Frau Bertha Garlan. Roman15.1.1901 – 15.3.1901@\strich\emph{Frau Bertha Garlan. Roman} {[}15.1.1901 – 15.3.1901{]}|pwv} von Ihnen bringen zu können u. wie froh ich jede Hoffnung dazu
               begrüßt habe. Zu meinem größten Bedauern aber, indem Sie jetzt eben wieder mir eine
               ſolche Hoffnung machen, deuten Sie ſelber an, daß Sie auch diesmal an ihrer Erfüllung
               zweifeln. Sie kennen ja die Haltung der »\textsc{Rundschau}\orgindex{Deutsche Rundschau@Deutsche Rundschau|pw}« u. wenn Sie das von Ihnen behandelte Sujet für »bedenklich« halten, ſo kann
               ich kaum glauben, daß ich darin anderer Meinung ſein werde als Sie, u. wage deshalb
                  {\pb}gar nicht, Sie um
               Einſendung Ihrer Arbeit\pwindex{Schnitzler, Arthur 15.05.1862 – 21.10.1931@\textsc{Schnitzler, Arthur} (15.05.1862 – 21.10.1931), \emph{Schriftsteller, Mediziner}!Frau Bertha Garlan. Roman15.1.1901 – 15.3.1901@\strich\emph{Frau Bertha Garlan. Roman} {[}15.1.1901 – 15.3.1901{]}|pwv} zu
               bitten. Denn eine Ablehnung würde peinlich für mich ſein u. einen Zeitverluſt für Sie
               bedeuten. Alſo, ſehr geehrter Herr Doctor, bewahren Sie mir Ihren freundlichen guten
               Willen, u. ſobald Sie eine Novelle ſchreiben, die nach Ihrem eigenen Dafürhalten mehr
               in den Rahmen der »\textsc{Rundschau}\orgindex{Deutsche Rundschau@Deutsche Rundschau|pw}« paßt, ſenden Sie ſie und ſeien Sie überzeugt, daß ſie uns herzlich willko{\geminationm}en ſein wird.\pend
           \pstart
           In aufrichtiger Hochachtung{\\[\baselineskip]}ergebenſt{\\[\baselineskip]}Ihr{\\[\baselineskip]}\spacefill\mbox{Dr Julius Rodenberg.}\pend
           \leftskip=0em{}
         
         \endnumbering\mylabel{h}\end{ledgroupsized}  \newcommand{\dateiname}{L01049}\newcommand{\titel}{Julius Rodenberg an Arthur Schnitzler, 23. 6. 1900}\newcommand{\editorInnen}{Martin Anton Müller und Gerd-Hermann Susen}%% latex-leseansicht-abspann.tex
%% Abspann für die Leseansicht.
%% Der Schalter \ifkorrekturansicht ist bereits durch den Vorspann gesetzt.

%% latex-abspann.tex
%% Gemeinsamer Abspann für Korrekturansicht und Leseansicht.
%% Setzt den Schalter \ifkorrekturansicht voraus (gesetzt in den
%% einbindenden Dateien latex-korrekturansicht-abspann.tex bzw.
%% latex-leseansicht-abspann.tex).
%% ---------------------------------------------------------------

\normalsize

% Das esempio-Environment wird nur in der Leseansicht benötigt
\ifkorrekturansicht\else
\newenvironment{esempio}[3]%
{
    \vspace{1.5ex}
    \rlap{\underline{#1}}
    \par
    \setlength{\parindent}{0cm}
    \nopagebreak
    \leftskip=#2cm
    \rightskip=#3cm
}
{
    \par
}
\fi

\doendnotes{C}
\bigskip
\vfill

\clearpage

\footnotesize

\ifkorrekturansicht
  \lohead{\textsc{register}}
\fi

% theindex-Environment neu definieren ohne reledmac
\makeatletter
\renewenvironment{theindex}{%
  \ifkorrekturansicht
    \section*{\indexname}%
  \else
    \subsubsection*{Index der erwähnten Entitäten}%
  \fi
  \setlength{\parindent}{0pt}%
  \setlength{\parskip}{0pt plus 0.3pt}%
  \let\item\@idxitem
}{%
  \ifkorrekturansicht\clearpage\fi
}
\makeatother

\IfFileExists{\jobname-pw.ind}{\input{\jobname-pw.ind}}{}

% Quellenangabe nur in der Leseansicht
\ifkorrekturansicht\else
% Fallback-Definitionen, falls die .tex-Datei \titel etc. nicht gesetzt hat
\providecommand{\titel}{}
\providecommand{\editorInnen}{}
\providecommand{\dateiname}{\jobname}

\vspace{3cm}

\vfill

\footnotesize
\textsc{Quelle}: \titel. Herausgegeben von {\editorInnen}. In: \emph{Arthur Schnitzler: Briefwechsel mit Autorinnen und Autoren}.
 Digitale Edition, https://schnitzler-briefe.acdh.oeaw.ac.at/{\dateiname}.html (Stand \today)
\fi

\end{document}


      