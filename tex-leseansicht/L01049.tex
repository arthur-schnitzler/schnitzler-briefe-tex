%% latex-leseansicht-vorspann.tex
%% Vorspann für die Leseansicht.
%% Lädt die gemeinsame Datei latex-vorspann.tex mit nicht gesetztem Schalter.

\newif\ifkorrekturansicht
\korrekturansichtfalse

\input{../tex-inputs/latex-vorspann}


\section[Julius Rodenberg an Arthur Schnitzler, 23. 6. 1900]{L01049 Julius Rodenberg an Arthur Schnitzler, 23. 6. 1900}
\nopagebreak\mylabel{L01049v}
\rehead{ }\normalsize\beginnumbering\briefempfaengerindex{Schnitzler, Arthur@\textsc{Schnitzler, Arthur}!zzzRodenberg, Julius@\emph{von Julius Rodenberg}!1900-06-231@{23. 6. 1900}|(be}
\toendnotes[C]{\smallbreak\pagebreak[2]}
\correspDesc{Versand  durch Julius Rodenberg am 23. 6. 1900 in Berlin
\newline{}Erhalt  durch Arthur Schnitzler im Zeitraum [24. 6. 1900
                  – 28. 6. 1900?] in Wien}\toendnotes[C]{\smallbreak}
\Standort{TMW, HS Schn 4/27/1.}
\physDesc{Brief, 1 Blatt, 2 Seiten, 1220 Zeichen
\newline{}Handschrift: schwarze Tinte, deutsche Kurrent
\newline{}Schnitzler: mit rotem Buntstift eine Unterstreichung }\toendnotes[C]{\smallbreak}
\pstart
           \centering{}{\pb}\textcolor{gray}{\textbf{Deutsche Rundschau\orgindex{Deutsche Rundschau@Deutsche Rundschau|pw}}}\pend
           
\pstart
           \textcolor{gray}{\textbf{Expedition u. Redaction:}}\hfill \textcolor{gray}{\textbf{Herausgeber:}}\pend
           
\pstart
           \textcolor{gray}{\textbf{Gebrüder Paetel\orgindex{Gebrüder Paetel Verlag@Gebrüder Paetel Verlag|pw} in Berlin\oindex{Berlin@\textbf{Berlin}, \emph{Hauptstadt}|pw}}}\hfill \textcolor{gray}{\textbf{Julius Rodenberg in Berlin\oindex{Berlin@\textbf{Berlin}, \emph{Hauptstadt}|pw}}}\pend
           
\pstart
           \textcolor{gray}{\textbf{W., Lützowstr. 7\oindex{Lützowstraße@\textbf{Lützowstraße}, \emph{Straße}|pw}.}}\hfill \textcolor{gray}{\textbf{W., Margarethenstr. 1\oindex{Margaretenstraße [Berlin]@\textbf{Margaretenstraße [Berlin]}, \emph{Straße}|pw}.}}\pend
           
\pstart
           \raggedleft{}\textbf{\textcolor{gray}{\textbf{Berlin W.\oindex{Berlin@\textbf{Berlin}, \emph{Hauptstadt}|pw},}} den}{ }23. Juni 1900.\pend
           
\pstart{}Hochgeehrter Herr Doctor!\pend\vspace{0.5em}
\pstart
           Empfangen Sie meinen verbindlichſten Dank für Ihr freundliches Schreiben vom
                  21. d M. u. das darin enthaltene Anerbieten. Ich brauche Ihnen nicht
               zu{ }ſagen, welchen Werth es für mich hat, Sie wißen es, wie{ }ſehr ich mich freuen
               würde, endlich einmal eine Novelle\pwindex{Schnitzler, Arthur 15.\,5.\,1862 Wien – 21.\,10.\,1931 ebd.@\textsc{Schnitzler, Arthur} (15.\,5.\,1862 Wien – 21.\,10.\,1931 ebd.), \emph{Schriftsteller, Mediziner}!Frau Bertha Garlan. Roman@\strich\emph{Frau Bertha Garlan. Roman}|pwv} von Ihnen bringen zu können u. wie froh ich jede Hoffnung dazu
               begrüßt habe. Zu meinem größten Bedauern aber, indem Sie jetzt eben wieder mir eine{ }ſolche Hoffnung machen, deuten Sie{ }ſelber an, daß Sie auch diesmal an ihrer Erfüllung
               zweifeln. Sie kennen ja die Haltung der »\textsc{Rundschau}\orgindex{Deutsche Rundschau@Deutsche Rundschau|pw}« u. wenn Sie das von Ihnen behandelte Sujet für »bedenklich« halten,{ }ſo kann
               ich kaum glauben, daß ich darin anderer Meinung{ }ſein werde als Sie, u. wage deshalb
                  {\pb}gar nicht, Sie um
               Einſendung Ihrer Arbeit\pwindex{Schnitzler, Arthur 15.\,5.\,1862 Wien – 21.\,10.\,1931 ebd.@\textsc{Schnitzler, Arthur} (15.\,5.\,1862 Wien – 21.\,10.\,1931 ebd.), \emph{Schriftsteller, Mediziner}!Frau Bertha Garlan. Roman@\strich\emph{Frau Bertha Garlan. Roman}|pwv} zu
               bitten. Denn eine Ablehnung würde peinlich für mich{ }ſein u. einen Zeitverluſt für Sie
               bedeuten. Alſo,{ }ſehr geehrter Herr Doctor, bewahren Sie mir Ihren freundlichen guten
               Willen, u.{ }ſobald Sie eine Novelle{ }ſchreiben, die nach Ihrem eigenen Dafürhalten mehr
               in den Rahmen der »\textsc{Rundschau}\orgindex{Deutsche Rundschau@Deutsche Rundschau|pw}« paßt,{ }ſenden Sie{ }ſie und{ }ſeien Sie überzeugt, daß{ }ſie uns herzlich willko{\geminationm}en{ }ſein wird.\pend
           
\pstart
           In aufrichtiger Hochachtung{\\[\baselineskip]}ergebenſt{\\[\baselineskip]}Ihr{\\[\baselineskip]}\spacefill\mbox{Dr Julius Rodenberg.}\pend
           \leftskip=0em{}\selectlanguage{ngerman}\endnumbering\briefempfaengerindex{Schnitzler, Arthur@\textsc{Schnitzler, Arthur}!zzzRodenberg, Julius@\emph{von Julius Rodenberg}!1900-06-231@{23. 6. 1900}|)be}\mylabel{L01049h}  \newcommand{\dateiname}{L01049}\newcommand{\titel}{Julius Rodenberg an Arthur Schnitzler, 23. 6. 1900}\newcommand{\editorInnen}{Martin Anton Müller und Gerd-Hermann Susen}%% latex-leseansicht-abspann.tex
%% Abspann für die Leseansicht.
%% Der Schalter \ifkorrekturansicht ist bereits durch den Vorspann gesetzt.

%% latex-abspann.tex
%% Gemeinsamer Abspann für Korrekturansicht und Leseansicht.
%% Setzt den Schalter \ifkorrekturansicht voraus (gesetzt in den
%% einbindenden Dateien latex-korrekturansicht-abspann.tex bzw.
%% latex-leseansicht-abspann.tex).
%% ---------------------------------------------------------------

\normalsize

% Das esempio-Environment wird nur in der Leseansicht benötigt
\ifkorrekturansicht\else
\newenvironment{esempio}[3]%
{
    \vspace{1.5ex}
    \rlap{\underline{#1}}
    \par
    \setlength{\parindent}{0cm}
    \nopagebreak
    \leftskip=#2cm
    \rightskip=#3cm
}
{
    \par
}
\fi

\doendnotes{C}
\bigskip
\vfill

\clearpage

\footnotesize

\ifkorrekturansicht
  \lohead{\textsc{register}}
\fi

% theindex-Environment neu definieren ohne reledmac
\makeatletter
\renewenvironment{theindex}{%
  \ifkorrekturansicht
    \section*{\indexname}%
  \else
    \subsubsection*{Index der erwähnten Entitäten}%
  \fi
  \setlength{\parindent}{0pt}%
  \setlength{\parskip}{0pt plus 0.3pt}%
  \let\item\@idxitem
}{%
  \ifkorrekturansicht\clearpage\fi
}
\makeatother

\IfFileExists{\jobname-pw.ind}{\input{\jobname-pw.ind}}{}

% Quellenangabe nur in der Leseansicht
\ifkorrekturansicht\else
% Fallback-Definitionen, falls die .tex-Datei \titel etc. nicht gesetzt hat
\providecommand{\titel}{}
\providecommand{\editorInnen}{}
\providecommand{\dateiname}{\jobname}

\vspace{3cm}

\vfill

\footnotesize
\textsc{Quelle}: \titel. Herausgegeben von {\editorInnen}. In: \emph{Arthur Schnitzler: Briefwechsel mit Autorinnen und Autoren}.
 Digitale Edition, https://schnitzler-briefe.acdh.oeaw.ac.at/{\dateiname}.html (Stand \today)
\fi

\end{document}


