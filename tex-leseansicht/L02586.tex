%% latex-leseansicht-vorspann.tex
%% Vorspann für die Leseansicht.
%% Lädt die gemeinsame Datei latex-vorspann.tex mit nicht gesetztem Schalter.

\newif\ifkorrekturansicht
\korrekturansichtfalse

\input{../tex-inputs/latex-vorspann}


\section[ Auguste Hauschner an Arthur Schnitzler, 17. 6. 1908]{L02586 Auguste Hauschner an Arthur Schnitzler,  17. 6. 1908}
\nopagebreak\mylabel{L02586v}
\rehead{ }\normalsize\beginnumbering\briefempfaengerindex{Schnitzler, Arthur@\textsc{Schnitzler, Arthur}!zzzHauschner, Auguste@\emph{von Auguste Hauschner}!1908-06-171@{17. 6. 1908}|(be}
\toendnotes[C]{\smallbreak\pagebreak[2]}
\correspDesc{Versand  durch Auguste Hauschner am 17. 6. 1908 in Berlin
\newline{}Weiterleitung  in Wien
\newline{}Erhalt  durch Arthur Schnitzler im Zeitraum [25. 6. 1908
                  – 29. 6. 1908?] in Seis am Schlern}\toendnotes[C]{\smallbreak}
\Standort{DLA, A:Schnitzler, HS1985.1.3363.}
\physDesc{Brief, 1 Blatt, 3 Seiten, 1071 Zeichen
\newline{}Handschrift: schwarze Tinte, lateinische Kurrent
\newline{}Schnitzler: mit Bleistift Vermerk »\textsc{Hauschner\pwindex{Hauschner, Auguste 12.\,2.\,1850 Prag – 10.\,4.\,1924 Berlin@\textsc{Hauschner, Auguste} (12.\,2.\,1850 Prag – 10.\,4.\,1924 Berlin), \emph{Schriftstellerin}|pw}}« }\toendnotes[C]{\smallbreak}
\pstart
           \raggedleft{}{\pb}Berlin\oindex{Berlin@\textbf{Berlin}, \emph{Hauptstadt}|pw} d. 17. 6. 08\pend
           \vspace{0.5em}
\pstart
           Sehr geehrter Herr Doctor – ich wünschte sehr, ich
               dürfte meine Bewunderung Ihres Romans\pwindex{Schnitzler, Arthur 15.\,5.\,1862 Wien – 21.\,10.\,1931 ebd.@\textsc{Schnitzler, Arthur} (15.\,5.\,1862 Wien – 21.\,10.\,1931 ebd.), \emph{Schriftsteller, Mediziner}!Weg ins Freie. Roman@\strich\emph{Der Weg ins Freie. Roman}|pwv} öffentlich aussprechen. Aber auf dem Weg zur Buchbesprechung ist für
               mich leider gar kein Plätzchen frei. So möchte ich Ihnen wenigstens, als ein Zeichen
               meiner Verehrung mein eigenes, so eben erschienenes, Buch\pwindex{Hauschner, Auguste 12.\,2.\,1850 Prag – 10.\,4.\,1924 Berlin@\textsc{Hauschner, Auguste} (12.\,2.\,1850 Prag – 10.\,4.\,1924 Berlin), \emph{Schriftstellerin}!Familie Lowositz. Roman@\strich\emph{Die Familie Lowositz. Roman}|pwv}{ }{\pb}senden. Leider hat es mit dem Ihren nichts gemein, als
               eine Stimmung. In einem \label{K_L02586-1v}\edtext{zweiten Band\pwindex{Hauschner, Auguste 12.\,2.\,1850 Prag – 10.\,4.\,1924 Berlin@\textsc{Hauschner, Auguste} (12.\,2.\,1850 Prag – 10.\,4.\,1924 Berlin), \emph{Schriftstellerin}!Rudolf und Camilla. Roman@\strich\emph{Rudolf und Camilla. Roman}|pwv}}{\lemma{\textnormal{\emph{zweiten Band}}}\Cendnote{\textnormal{Die Fortführung erschien 1910 mit dem Titel \emph{Rudolf
                     und Camilla}\pwindex{Hauschner, Auguste 12.\,2.\,1850 Prag – 10.\,4.\,1924 Berlin@\textsc{Hauschner, Auguste} (12.\,2.\,1850 Prag – 10.\,4.\,1924 Berlin), \emph{Schriftstellerin}!Rudolf und Camilla. Roman@\strich\emph{Rudolf und Camilla. Roman}|pwk}.}}}\label{K_L02586-1} soll diese noch vertiefter werden. –\pend
           
\pstart
           Hätte ich mich an Ihrem Werk\pwindex{Schnitzler, Arthur 15.\,5.\,1862 Wien – 21.\,10.\,1931 ebd.@\textsc{Schnitzler, Arthur} (15.\,5.\,1862 Wien – 21.\,10.\,1931 ebd.), \emph{Schriftsteller, Mediziner}!Weg ins Freie. Roman@\strich\emph{Der Weg ins Freie. Roman}|pwv}
               nicht so entzückt, so könnte ich Sie darum beneiden. Wie kann man so viel können!
               Einen solchen Reichthum in sich haben und solche Kraft ihn auszumünzen. Ich liebe Maupassant\pwindex{Maupassant, Guy de 5.\,8.\,1850 Tourville-sur-Arques – 7.\,7.\,1893 Paris@\textsc{Maupassant, Guy de} (5.\,8.\,1850 Tourville-sur-Arques – 7.\,7.\,1893 Paris), \emph{Schriftsteller}|pw}, aber ich suche nicht den billigen
               Vergleich mit Ihnen. Der Sie so persönlich sind, so ganz ein Eigener. {\pb}Ganz traurig wird man doch, dass es so eine restlose
               Fähigkeit des Ausdrucks giebt, so eine Seelenkunde, so ein Verstehen des
               Menschlichen. Und Unsereins wagt sich daneben auch Schriftsteller zu nennen.
               Verzeihen Sie mir Beides. Diesen Herzensschrei und das Senden meines Buchs\pwindex{Hauschner, Auguste 12.\,2.\,1850 Prag – 10.\,4.\,1924 Berlin@\textsc{Hauschner, Auguste} (12.\,2.\,1850 Prag – 10.\,4.\,1924 Berlin), \emph{Schriftstellerin}!Familie Lowositz. Roman@\strich\emph{Die Familie Lowositz. Roman}|pwv}.\pend
           
\pstart
           In aufrichtiger Ergebenheit{\\[\baselineskip]}\spacefill\mbox{Frau Auguste Hauschner}\pend
           \leftskip=0em{}\selectlanguage{ngerman}\endnumbering\briefempfaengerindex{Schnitzler, Arthur@\textsc{Schnitzler, Arthur}!zzzHauschner, Auguste@\emph{von Auguste Hauschner}!1908-06-171@{17. 6. 1908}|)be}\mylabel{L02586h}  \newcommand{\dateiname}{L02586}\newcommand{\titel}{Auguste Hauschner an Arthur Schnitzler, 17. 6. 1908}\newcommand{\editorInnen}{Martin Anton Müller und Laura Untner}%% latex-leseansicht-abspann.tex
%% Abspann für die Leseansicht.
%% Der Schalter \ifkorrekturansicht ist bereits durch den Vorspann gesetzt.

%% latex-abspann.tex
%% Gemeinsamer Abspann für Korrekturansicht und Leseansicht.
%% Setzt den Schalter \ifkorrekturansicht voraus (gesetzt in den
%% einbindenden Dateien latex-korrekturansicht-abspann.tex bzw.
%% latex-leseansicht-abspann.tex).
%% ---------------------------------------------------------------

\normalsize

% Das esempio-Environment wird nur in der Leseansicht benötigt
\ifkorrekturansicht\else
\newenvironment{esempio}[3]%
{
    \vspace{1.5ex}
    \rlap{\underline{#1}}
    \par
    \setlength{\parindent}{0cm}
    \nopagebreak
    \leftskip=#2cm
    \rightskip=#3cm
}
{
    \par
}
\fi

\doendnotes{C}
\bigskip
\vfill

\clearpage

\footnotesize

\ifkorrekturansicht
  \lohead{\textsc{register}}
\fi

% theindex-Environment neu definieren ohne reledmac
\makeatletter
\renewenvironment{theindex}{%
  \ifkorrekturansicht
    \section*{\indexname}%
  \else
    \subsubsection*{Index der erwähnten Entitäten}%
  \fi
  \setlength{\parindent}{0pt}%
  \setlength{\parskip}{0pt plus 0.3pt}%
  \let\item\@idxitem
}{%
  \ifkorrekturansicht\clearpage\fi
}
\makeatother

\IfFileExists{\jobname-pw.ind}{\input{\jobname-pw.ind}}{}

% Quellenangabe nur in der Leseansicht
\ifkorrekturansicht\else
% Fallback-Definitionen, falls die .tex-Datei \titel etc. nicht gesetzt hat
\providecommand{\titel}{}
\providecommand{\editorInnen}{}
\providecommand{\dateiname}{\jobname}

\vspace{3cm}

\vfill

\footnotesize
\textsc{Quelle}: \titel. Herausgegeben von {\editorInnen}. In: \emph{Arthur Schnitzler: Briefwechsel mit Autorinnen und Autoren}.
 Digitale Edition, https://schnitzler-briefe.acdh.oeaw.ac.at/{\dateiname}.html (Stand \today)
\fi

\end{document}


