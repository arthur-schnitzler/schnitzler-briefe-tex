%% latex-korrekturansicht-vorspann.tex
%% Vorspann für die Korrekturansicht.
%% Lädt die gemeinsame Datei latex-vorspann.tex mit gesetztem Schalter.

\newif\ifkorrekturansicht
\korrekturansichttrue

\input{../tex-inputs/latex-vorspann}


\section[ Auguste Hauschner an Arthur Schnitzler, 17. 6. 1908]{L02586 Auguste Hauschner an Arthur Schnitzler, 17. 6. 1908}
\nopagebreak\mylabel{L02586v}
\rehead{ }\normalsize\beginnumbering\briefempfaengerindex{Schnitzler, Arthur@\textsc{Schnitzler, Arthur}!zzzHauschner, Auguste@\emph{von Auguste Hauschner}!1908-06-171@{17. 6. 1908}|(be}
\toendnotes[C]{\smallbreak\pagebreak[2]}\Standort{DLA, A:Schnitzler, HS1985.1.3363.}
\physDesc{Brief, 1 Blatt, 3 Seiten, 1071 Zeichen
\newline{}Handschrift: schwarze Tinte, lateinische Kurrent
\newline{}Schnitzler: mit Bleistift Vermerk »\textsc{Hauschner\pwindex{Hauschner, Auguste 12.02.1850 – 10.04.1924@\textsc{Hauschner, Auguste} (12.02.1850 – 10.04.1924), \emph{Schriftsteller/Schriftstellerin}|pw}}« }\toendnotes[C]{\smallbreak}
\pstart
           \raggedleft{}{\pb}Berlin\oindex{Berlin@\textbf{Berlin}, \emph{P.PPLC}|pw} d. 17. 6. 08\pend
           \vspace{0.5em}
\pstart
           Sehr geehrter Herr Doctor – ich wünschte sehr, ich
               dürfte meine Bewunderung Ihres Romans\pwindex{Weg ins Freie. Roman@\emph{Der Weg ins Freie. Roman}|pwv} öffentlich aussprechen. Aber auf dem Weg zur Buchbesprechung ist für
               mich leider gar kein Plätzchen frei. So möchte ich Ihnen wenigstens, als ein Zeichen
               meiner Verehrung mein eigenes, so eben erschienenes, Buch\pwindex{Familie Lowositz. Roman@\emph{Die Familie Lowositz. Roman}|pwv}{ }{\pb}senden. Leider hat es mit dem Ihren nichts gemein, als
               eine Stimmung. In einem \label{K_L02586-1v}\edtext{zweiten Band\pwindex{Rudolf und Camilla. Roman@\emph{Rudolf und Camilla. Roman}|pwv}}{\lemma{\textnormal{\emph{zweiten Band}}}\Cendnote{\textnormal{Die Fortführung erschien 1910 mit dem Titel \emph{Rudolf
                     und Camilla}\pwindex{Rudolf und Camilla. Roman@\emph{Rudolf und Camilla. Roman}|pwk}.}}}\label{K_L02586-1} soll diese noch vertiefter werden. –\pend
           
\pstart
           Hätte ich mich an Ihrem Werk\pwindex{Weg ins Freie. Roman@\emph{Der Weg ins Freie. Roman}|pwv}
               nicht so entzückt, so könnte ich Sie darum beneiden. Wie kann man so viel können!
               Einen solchen Reichthum in sich haben und solche Kraft ihn auszumünzen. Ich liebe Maupassant\pwindex{Maupassant, Guy de 05.08.1850 – 07.07.1893@\textsc{Maupassant, Guy de} (05.08.1850 – 07.07.1893), \emph{Schriftsteller/Schriftstellerin}|pw}, aber ich suche nicht den billigen
               Vergleich mit Ihnen. Der Sie so persönlich sind, so ganz ein Eigener. {\pb}Ganz traurig wird man doch, dass es so eine restlose
               Fähigkeit des Ausdrucks giebt, so eine Seelenkunde, so ein Verstehen des
               Menschlichen. Und Unsereins wagt sich daneben auch Schriftsteller zu nennen.
               Verzeihen Sie mir Beides. Diesen Herzensschrei und das Senden meines Buchs\pwindex{Familie Lowositz. Roman@\emph{Die Familie Lowositz. Roman}|pwv}.\pend
           
\pstart
           In aufrichtiger Ergebenheit{\\[\baselineskip]}\spacefill\mbox{Frau Auguste Hauschner}\pend
           \leftskip=0em{}\selectlanguage{ngerman}\endnumbering\briefempfaengerindex{Schnitzler, Arthur@\textsc{Schnitzler, Arthur}!zzzHauschner, Auguste@\emph{von Auguste Hauschner}!1908-06-171@{17. 6. 1908}|)be}\mylabel{L02586h}  \normalsize

\doendnotes{C}
\bigskip
\vfill

\clearpage

\footnotesize

\lohead{\textsc{register}}

% Definiere theindex-Environment komplett neu ohne reledmac
\makeatletter
\renewenvironment{theindex}{%
  \section*{\indexname}%
  \setlength{\parindent}{0pt}%
  \setlength{\parskip}{0pt plus 0.3pt}%
  \let\item\@idxitem
}{%
  \clearpage
}
\makeatother

\IfFileExists{\jobname-pw.ind}{\input{\jobname-pw.ind}}{}

\end{document}

      