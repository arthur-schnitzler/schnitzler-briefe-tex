%% latex-korrekturansicht-vorspann.tex
%% Vorspann für die Korrekturansicht.
%% Lädt die gemeinsame Datei latex-vorspann.tex mit gesetztem Schalter.

\newif\ifkorrekturansicht
\korrekturansichttrue

\input{../tex-inputs/latex-vorspann}


\section[ Arthur Schnitzler an Felix Salten, 24. 9. 1898]{L02966 Arthur Schnitzler an Felix Salten, 24. 9. 1898}
\nopagebreak\mylabel{L02966v}
\rehead{ }\normalsize\beginnumbering\briefempfaengerindex{Salten, Felix@\textsc{Salten, Felix}!zzzSchnitzler, Arthur@\emph{von Arthur Schnitzler}!1898-09-241@{24. 9. 1898}|(be}
\toendnotes[C]{\smallbreak\pagebreak[2]}\Standort{Wienbibliothek im Rathaus, ZPH 1681, 2.1.516.}
\physDesc{Brief, 1 Blatt, 3 Seiten, 512 Zeichen
\newline{}Handschrift: Bleistift, deutsche Kurrent
\newline{}Ordnung: mit Bleistift von unbekannter Hand Nummerierung der Doppelseiten des Konvoluts:
                                    »73«–»74« }
\buchAbdrucke{\weitereDrucke{Arthur Schnitzler: \emph{Briefe 1875–1912}. Frankfurt am Main: \emph{S. Fischer} 1981, S. 354.} }\toendnotes[C]{\smallbreak}
\pstart
           \raggedleft{}{\pb}24. 9. 98\pend
           
\pstart{}Lieber Freund,\pend\vspace{0.5em}
\pstart
           den \label{K_L02966-1v}\edtext{Lu\textcolor{gray}{lu}\pwindex{Vermaechtnis. Schauspiel in drei Akten@\emph{Das Vermächtnis. Schauspiel in drei Akten}|pwv}}{\lemma{\textnormal{\emph{Lulu}}}\Cendnote{\textnormal{Siehe Felix Salten an Arthur Schnitzler, 23. 9. 1898.
               }}}\label{K_L02966-1} wird die kleine Gerzhofer\pwindex{Gerzhofer, Camilla 1888-02-02 – 1961-03-08@\textsc{Gerzhofer, Camilla} (1888-02-02 – 1961-03-08), \emph{Schauspieler/Schauspielerin}|pw}, alſo ein
               wirkliches Kind ſpielen, welche Eventual. wir noch gar nicht in Betracht gezogen
               hatten, und was mir {\pb}doch das weitaus beſte
               zu ſein ſcheint. We{\geminationn} Sie das Fräulein Metzl\pwindex{Salten, Ottilie 07.03.1868 – 22.06.1942@\textsc{Salten, Ottilie} (07.03.1868 – 22.06.1942), \emph{Schauspieler/Schauspielerin}|pw} ſagen, wird ſie gewiſs nicht im mindeſten verletzt
               ſein. Sie wiſſen, daſs unter den wirklichen Schauſpielerin\textcolor{gray}{nen} für
               mich nur \textsc{Frl. Metzl\pwindex{Salten, Ottilie 07.03.1868 – 22.06.1942@\textsc{Salten, Ottilie} (07.03.1868 – 22.06.1942), \emph{Schauspieler/Schauspielerin}|pw}} in {\pb}Betracht kam; aber das wirkliche
                  \uline{Kind}, das Talent hat, iſt in der Rolle entſchieden
               vorzuziehen.\pend
           
\pstart
           Ich ſehe Sie hoffentlich \label{K_L02966-2v}\edtext{heut{ }Abd}{\lemma{\textnormal{\emph{heut Abd}}}\Cendnote{\textnormal{Schnitzler besuchte am Abend
                  des 24. 9. 1898 die
                  Premiere von Carl Karlweis\pwindex{Karlweis, Carl 23.11.1850 – 27.10.1901@\textsc{Karlweis, Carl} (23.11.1850 – 27.10.1901), \emph{Schriftsteller/Schriftstellerin}|pwk}’ \emph{Das liebe Ich}\pwindex{liebe Ich. Volksstueck in einem Vorspiel und drei Akten (fuenf Bildern)@\emph{Das liebe Ich. Volksstück in einem Vorspiel und drei Akten (fünf Bildern)}|pwk} im Volkstheater\oindex{Volkstheater@\textbf{Volkstheater}, \emph{Theater (K.THE)}|pwk}. Saltens\pwindex{Salten, Felix 06.09.1869 – 08.10.1945@\textsc{Salten, Felix} (06.09.1869 – 08.10.1945), \emph{Schriftsteller/Schriftstellerin, Journalist/Journalistin, Chefredakteur/Chefredakteurin}|pwk} Anwesenheit
                  ist nicht nachweisbar.}}}\label{K_L02966-2}\pend
           
\pstart
           Herzl Gr\textcolor{gray}{üße}{ }{\\[\baselineskip]}Ihr
                  \spacefill\mbox{ArthS.}\pend
           \leftskip=0em{}\selectlanguage{ngerman}\endnumbering\briefempfaengerindex{Salten, Felix@\textsc{Salten, Felix}!zzzSchnitzler, Arthur@\emph{von Arthur Schnitzler}!1898-09-241@{24. 9. 1898}|)be}\mylabel{L02966h}  \normalsize

\doendnotes{C}
\bigskip
\vfill

\clearpage

\footnotesize

\lohead{\textsc{register}}

% Definiere theindex-Environment komplett neu ohne reledmac
\makeatletter
\renewenvironment{theindex}{%
  \section*{\indexname}%
  \setlength{\parindent}{0pt}%
  \setlength{\parskip}{0pt plus 0.3pt}%
  \let\item\@idxitem
}{%
  \clearpage
}
\makeatother

\IfFileExists{\jobname-pw.ind}{\input{\jobname-pw.ind}}{}

\end{document}

      