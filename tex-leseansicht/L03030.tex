%% latex-korrekturansicht-vorspann.tex
%% Vorspann für die Korrekturansicht.
%% Lädt die gemeinsame Datei latex-vorspann.tex mit gesetztem Schalter.

\newif\ifkorrekturansicht
\korrekturansichttrue

\input{../tex-inputs/latex-vorspann}


\section[ Arthur Schnitzler an Felix Salten, {[}31. 12. 1899?{]}]{L03030 Arthur Schnitzler an Felix Salten, {[}31. 12. 1899?{]}}
\nopagebreak\mylabel{L03030v}
\rehead{ }\normalsize\beginnumbering\briefempfaengerindex{Salten, Felix@\textsc{Salten, Felix}!zzzSchnitzler, Arthur@\emph{von Arthur Schnitzler}!1899-12-311@{{[}31. 12. 1899?{]}}|(be}
\toendnotes[C]{\smallbreak\pagebreak[2]}\Standort{Wienbibliothek im Rathaus, ZPH 1681, 2.1.516.}
\physDesc{Karte, 203 Zeichen
\newline{}Handschrift: schwarze Tinte, deutsche Kurrent
\newline{}Ordnung: mit Bleistift von unbekannter Hand nummeriert: »33« }\toendnotes[C]{\smallbreak}
\pstart
           \noindent{}{\pb}lieber Freund, ich bin \label{K_L03030-1v}\edtext{morgen (Neujahr) Abend}{\lemma{\textnormal{\emph{morgen (Neujahr) Abend}}}\Cendnote{\textnormal{Das erlaubt die Datierung anhand von Schnitzlers{ }\emph{Tagebuch}\pwindex{Tagebuch@\emph{Tagebuch}|pwk}, vgl. A. S.: \emph{Tagebuch}, 1. 1. 1900.}}}\label{K_L03030-1}, we{\geminationn} ich frei
               bin, bei Richard\pwindex{Beer-Hofmann, Richard 1866-07-11 – 1945-09-26@\textsc{Beer-Hofmann, Richard} (1866-07-11 – 1945-09-26), \emph{Schriftsteller/Schriftstellerin}|pw}; er läſst Sie bitten, auch zu
               ihm zu ko{\geminationm}en. Hugo\pwindex{Hofmannsthal, Hugo von 1874-02-01 – 1929-07-15@\textsc{Hofmannsthal, Hugo von} (1874-02-01 – 1929-07-15), \emph{Schriftsteller/Schriftstellerin}|pw} und Guſt. Schwarzk.\pwindex{Schwarzkopf, Gustav 07.11.1853 – 13.11.1939@\textsc{Schwarzkopf, Gustav} (07.11.1853 – 13.11.1939), \emph{Schriftsteller/Schriftstellerin}|pw} ſind beſti{\geminationm}t dort.\pend
           \pstart Herzlichſt Ihr \spacefill\mbox{Arthur.}\pend{}
\pstart
           Schlenther\pwindex{Schlenther, Paul 20.08.1854 – 30.04.1916@\textsc{Schlenther, Paul} (20.08.1854 – 30.04.1916), \emph{Schriftsteller/Schriftstellerin, Kritiker/Kritikerin, Theaterleiter/Theaterleiterin}|pw} wieder \label{K_L03030-2v}\edtext{gutge\textcolor{gray}{h}äkelt}{\lemma{\textnormal{\emph{gutgehäkelt}}}\Cendnote{\textnormal{Der spezifische Bezug bleibt unklar. Womöglich geht es um die kürzlich
                  erfolgte Absetzung von \emph{Der grüne Kakadu}\pwindex{gruene Kakadu. Groteske in einem Akt@\emph{Der grüne Kakadu. Groteske in einem Akt}|pwk}.
               }}}\label{K_L03030-2}!\pend
           \selectlanguage{ngerman}\endnumbering\briefempfaengerindex{Salten, Felix@\textsc{Salten, Felix}!zzzSchnitzler, Arthur@\emph{von Arthur Schnitzler}!1899-12-311@{{[}31. 12. 1899?{]}}|)be}\mylabel{L03030h}  \normalsize

\doendnotes{C}
\bigskip
\vfill

\clearpage

\footnotesize

\lohead{\textsc{register}}

% Definiere theindex-Environment komplett neu ohne reledmac
\makeatletter
\renewenvironment{theindex}{%
  \section*{\indexname}%
  \setlength{\parindent}{0pt}%
  \setlength{\parskip}{0pt plus 0.3pt}%
  \let\item\@idxitem
}{%
  \clearpage
}
\makeatother

\IfFileExists{\jobname-pw.ind}{\input{\jobname-pw.ind}}{}

\end{document}

      