%% latex-korrekturansicht-vorspann.tex
%% Vorspann für die Korrekturansicht.
%% Lädt die gemeinsame Datei latex-vorspann.tex mit gesetztem Schalter.

\newif\ifkorrekturansicht
\korrekturansichttrue

\input{../tex-inputs/latex-vorspann}


\section[Bertha von Suttner an Arthur Schnitzler, 22. 10. 1913]{L02153 Bertha von Suttner an Arthur Schnitzler, 22. 10. 1913}
\nopagebreak\mylabel{L02153v}
\rehead{ }\normalsize\beginnumbering\briefempfaengerindex{Schnitzler, Arthur@\textsc{Schnitzler, Arthur}!zzzSuttner, Bertha von@\emph{von Bertha von Suttner}!1913-10-221@{22. 10. 1913}|(be}
\toendnotes[C]{\smallbreak\pagebreak[2]}\Standort{CUL, Schnitzler, B 104.}
\physDesc{Brief, 1 Blatt, 2 Seiten, 498 Zeichen (aufgeprägte Krone in Golddruck)
\newline{}Handschrift: schwarze Tinte, deutsche Kurrent
\newline{}Schnitzler: 1) mit Bleistift beschriftet: »\textsc{Suttner}«  2) mit rotem Buntstift eine Anstreichung}\Standort{DLA, A:Schnitzler, HS.NZ85.1.4773.}
\physDesc{maschinenschriftliche Abschrift1 Blatt, 1 Seite, 498 Zeichen
\newline{}Schreibmaschine}\toendnotes[C]{\smallbreak}
\pstart
           \centering{}{\pb}\textsc{Zedlitzgasse 7 Wien}\oindex{Zedlitzgasse@\textbf{Zedlitzgasse}, \emph{Straße (K.STR)}|pw}\pend
           
\pstart
           \raggedleft{}22/10 1913\pend
           
\pstart{}Verehrter Dichter\pend\vspace{0.5em}
\pstart
           In einer \label{K_L02153-1v}\edtext{Angelegenheit}{\lemma{\textnormal{\emph{Angelegenheit}}}\Cendnote{\textnormal{Vgl. A. S.: \emph{Tagebuch}, 29. 10. 1913.
               }}}\label{K_L02153-1}, die Sie und mich angeht, wäre mir eine Rückſprache ſehr erwünſcht.\pend
           
\pstart
           Wie ſollen wir das machen? Ich wäre auch gern bereit, zu einer Stunde, wo Sie u. Frau
                  D\textsuperscript{r}{ }Schnitzler\pwindex{Schnitzler, Olga 17.01.1882 – 13.01.1970@\textsc{Schnitzler, Olga} (17.01.1882 – 13.01.1970), \emph{Schauspieler/Schauspielerin, Sänger/Sängerin}|pw} ein paar Freunde um ſich haben, nach
               der \label{K_L02153-2v}\edtext{Sternwartegaſſe\oindex{Sternwartestrasse 71@\textbf{Sternwartestraße 71}, \emph{Wohngebäude (K.WHS)}|pw}}{\lemma{\textnormal{\emph{Sternwartegaſſe}}}\Cendnote{\textnormal{richtig: Sternwartestraße\oindex{Sternwartestrasse 71@\textbf{Sternwartestraße 71}, \emph{Wohngebäude (K.WHS)}|pwk}}}}\label{K_L02153-2} zu kommen. Da {\pb}würde ich Sie um
               nichts von Ihrer Arbeitszeit berauben, und zugleich das Vergnügen einer gemüthlichen
               Unterhaltung mit Ihnen beiden \strikeout{ge} haben.\pend
           
\pstart
           Mit ausgezeichneter Hochachtung{\\[\baselineskip]}Ihre erg.{\\[\baselineskip]}\spacefill\mbox{Bertha v. Suttner}\pend
           \leftskip=0em{}\selectlanguage{ngerman}\endnumbering\briefempfaengerindex{Schnitzler, Arthur@\textsc{Schnitzler, Arthur}!zzzSuttner, Bertha von@\emph{von Bertha von Suttner}!1913-10-221@{22. 10. 1913}|)be}\mylabel{L02153h}  \normalsize

\doendnotes{C}
\bigskip
\vfill

\clearpage

\footnotesize

\lohead{\textsc{register}}

% Definiere theindex-Environment komplett neu ohne reledmac
\makeatletter
\renewenvironment{theindex}{%
  \section*{\indexname}%
  \setlength{\parindent}{0pt}%
  \setlength{\parskip}{0pt plus 0.3pt}%
  \let\item\@idxitem
}{%
  \clearpage
}
\makeatother

\IfFileExists{\jobname-pw.ind}{\input{\jobname-pw.ind}}{}

\end{document}

      