%% latex-leseansicht-vorspann.tex
%% Vorspann für die Leseansicht.
%% Lädt die gemeinsame Datei latex-vorspann.tex mit nicht gesetztem Schalter.

\newif\ifkorrekturansicht
\korrekturansichtfalse

\input{../tex-inputs/latex-vorspann}


\section[Bertha von Suttner an Arthur Schnitzler, 22. 10. 1913]{L02153 Bertha von Suttner an Arthur Schnitzler, 22. 10. 1913}
\nopagebreak\mylabel{L02153v}
\rehead{ }\normalsize\beginnumbering\briefempfaengerindex{Schnitzler, Arthur@\textsc{Schnitzler, Arthur}!zzzSuttner, Bertha von@\emph{von Bertha von Suttner}!1913-10-221@{22. 10. 1913}|(be}
\toendnotes[C]{\smallbreak\pagebreak[2]}
\correspDesc{Versand  durch Bertha von Suttner am 22. 10. 1913 in Wien
\newline{}Erhalt  durch Arthur Schnitzler im Zeitraum [22. 10. 1913 – 26. 10. 1913?] in Wien}\toendnotes[C]{\smallbreak}
\Standort{CUL, Schnitzler, B 104.}
\physDesc{Brief, 1 Blatt, 2 Seiten, 498 Zeichen (aufgeprägte Krone in Golddruck)
\newline{}Handschrift: schwarze Tinte, deutsche Kurrent
\newline{}Schnitzler: 1) mit Bleistift beschriftet: »\textsc{Suttner}«  2) mit rotem Buntstift eine Anstreichung}\Standort{DLA, A:Schnitzler, HS.NZ85.1.4773.}
\physDesc{maschinenschriftliche Abschrift, 1 Blatt, 1 Seite, 498 Zeichen
\newline{}Schreibmaschine}\toendnotes[C]{\smallbreak}
\pstart
           \centering{}{\pb}\textsc{Zedlitzgasse 7 Wien}\oindex{Wien@\textbf{Wien}!I., Innere Stadt@\textbf{I., Innere Stadt}!Zedlitzgasse@\textbf{Zedlitzgasse}, \emph{Straße}|pw}\pend
           
\pstart
           \raggedleft{}22/10 1913\pend
           
\pstart{}Verehrter Dichter\pend\vspace{0.5em}
\pstart
           In einer \label{K_L02153-1v}\edtext{Angelegenheit}{\lemma{\textnormal{\emph{Angelegenheit}}}\Cendnote{\textnormal{Vgl. A. S.: \emph{Tagebuch}, 29. 10. 1913.
               }}}\label{K_L02153-1}, die Sie und mich angeht, wäre mir eine Rückſprache{ }ſehr erwünſcht.\pend
           
\pstart
           Wie{ }ſollen wir das machen? Ich wäre auch gern bereit, zu einer Stunde, wo Sie u. Frau
                  D\textsuperscript{r}{ }Schnitzler\pwindex{Schnitzler, Olga 17.\,1.\,1882 Wien – 13.\,1.\,1970 Lugano@\textsc{Schnitzler, Olga} (17.\,1.\,1882 Wien – 13.\,1.\,1970 Lugano), \emph{Schauspielerin, Sängerin}|pw} ein paar Freunde um{ }ſich haben, nach
               der \label{K_L02153-2v}\edtext{Sternwartegaſſe\oindex{Wien@\textbf{Wien}!XVIII., Währing@\textbf{XVIII., Währing}!Sternwartestraße 71@\textbf{Sternwartestraße 71}, \emph{Wohngebäude}|pw}}{\lemma{\textnormal{\emph{Sternwartegasse}}}\Cendnote{\textnormal{richtig: Sternwartestraße\oindex{Wien@\textbf{Wien}!XVIII., Währing@\textbf{XVIII., Währing}!Sternwartestraße 71@\textbf{Sternwartestraße 71}, \emph{Wohngebäude}|pwk}}}}\label{K_L02153-2} zu kommen. Da {\pb}würde ich Sie um
               nichts von Ihrer Arbeitszeit berauben, und zugleich das Vergnügen einer gemüthlichen
               Unterhaltung mit Ihnen beiden \strikeout{ge} haben.\pend
           
\pstart
           Mit ausgezeichneter Hochachtung{\\[\baselineskip]}Ihre erg.{\\[\baselineskip]}\spacefill\mbox{Bertha v. Suttner}\pend
           \leftskip=0em{}\selectlanguage{ngerman}\endnumbering\briefempfaengerindex{Schnitzler, Arthur@\textsc{Schnitzler, Arthur}!zzzSuttner, Bertha von@\emph{von Bertha von Suttner}!1913-10-221@{22. 10. 1913}|)be}\mylabel{L02153h}  \newcommand{\dateiname}{L02153}\newcommand{\titel}{Bertha von Suttner an Arthur Schnitzler, 22. 10. 1913}\newcommand{\editorInnen}{Martin Anton Müller und Gerd-Hermann Susen}%% latex-leseansicht-abspann.tex
%% Abspann für die Leseansicht.
%% Der Schalter \ifkorrekturansicht ist bereits durch den Vorspann gesetzt.

%% latex-abspann.tex
%% Gemeinsamer Abspann für Korrekturansicht und Leseansicht.
%% Setzt den Schalter \ifkorrekturansicht voraus (gesetzt in den
%% einbindenden Dateien latex-korrekturansicht-abspann.tex bzw.
%% latex-leseansicht-abspann.tex).
%% ---------------------------------------------------------------

\normalsize

% Das esempio-Environment wird nur in der Leseansicht benötigt
\ifkorrekturansicht\else
\newenvironment{esempio}[3]%
{
    \vspace{1.5ex}
    \rlap{\underline{#1}}
    \par
    \setlength{\parindent}{0cm}
    \nopagebreak
    \leftskip=#2cm
    \rightskip=#3cm
}
{
    \par
}
\fi

\doendnotes{C}
\bigskip
\vfill

\clearpage

\footnotesize

\ifkorrekturansicht
  \lohead{\textsc{register}}
\fi

% theindex-Environment neu definieren ohne reledmac
\makeatletter
\renewenvironment{theindex}{%
  \ifkorrekturansicht
    \section*{\indexname}%
  \else
    \subsubsection*{Index der erwähnten Entitäten}%
  \fi
  \setlength{\parindent}{0pt}%
  \setlength{\parskip}{0pt plus 0.3pt}%
  \let\item\@idxitem
}{%
  \ifkorrekturansicht\clearpage\fi
}
\makeatother

\IfFileExists{\jobname-pw.ind}{\input{\jobname-pw.ind}}{}

% Quellenangabe nur in der Leseansicht
\ifkorrekturansicht\else
% Fallback-Definitionen, falls die .tex-Datei \titel etc. nicht gesetzt hat
\providecommand{\titel}{}
\providecommand{\editorInnen}{}
\providecommand{\dateiname}{\jobname}

\vspace{3cm}

\vfill

\footnotesize
\textsc{Quelle}: \titel. Herausgegeben von {\editorInnen}. In: \emph{Arthur Schnitzler: Briefwechsel mit Autorinnen und Autoren}.
 Digitale Edition, https://schnitzler-briefe.acdh.oeaw.ac.at/{\dateiname}.html (Stand \today)
\fi

\end{document}


