%% latex-leseansicht-vorspann.tex
%% Vorspann für die Leseansicht.
%% Lädt die gemeinsame Datei latex-vorspann.tex mit nicht gesetztem Schalter.

\newif\ifkorrekturansicht
\korrekturansichtfalse

\input{../tex-inputs/latex-vorspann}


               \section[Bertha von Suttner an Arthur Schnitzler, 22. 10. 1913]{ Bertha von Suttner an Arthur Schnitzler,
                    22. 10. 1913}\nopagebreak\mylabel{v}\rehead{ }\begin{ledgroupsized}[t]{13cm}\normalsize\beginnumbering\briefempfaengerindex{Schnitzler, Arthur@\textsc{Schnitzler, Arthur}!zzzSuttner, Bertha von@\emph{von Bertha von Suttner}!1913-10-221@{22. 10. 1913}|(be} \toendnotes[C]{\smallbreak\pagebreak[2]} \Standort{CUL, Schnitzler, B 104.}
\physDesc{Brief, 1 Blatt (mit Krone in Golddruck), 2 Seiten
\newline{}Handschrift: schwarze Tinte, deutsche Kurrent
\newline{}Schnitzler: 1) mit Bleistift beschriftet: »\textsc{Suttner}« 2)  mit rotem Buntstift eine Anstreichung}\Standort{DLA, A:Schnitzler, HS.NZ85.1.4773.}
\physDesc{1 Blatt, 1 Seite, maschinelle Abschrift}\toendnotes[C]{\smallbreak}\pstart
           \noindent{}\centering{}{\pb}\textsc{Zedlitzgasse 7 Wien}\oindex{Zedlitzgasse@\textbf{Zedlitzgasse}|pw}\pend
           \pstart
           \raggedleft{}22/10 1913\pend
           \pstart{}Verehrter Dichter\pend\pstart
           In einer \label{K_L02153_1v}\edtext{Angelegenheit}{\lemma{\textnormal{\emph{Angelegenheit}}}\Cendnote{\textnormal{vgl. A. S.: \emph{Tagebuch}, 29. 10. 1913}}}\label{K_L02153_1h}, die Sie und mich angeht, wäre mir eine Rückſprache ſehr erwünſcht.\pend
           \pstart
           Wie ſollen wir das machen? Ich wäre auch gern bereit, zu einer Stunde, wo Sie u.
                    Frau D\textsuperscript{r}{ }Schnitzler\pwindex{Schnitzler, Olga 17.01.1882 – 13.01.1970@\textsc{Schnitzler, Olga} (17.01.1882 – 13.01.1970), \emph{Schauspielerin, Sängerin}|pw} ein paar Freunde um ſich haben,
                    nach der \label{K_L02153_2v}\edtext{Sternwartegaſſe\oindex{Sternwartestrasse@\textbf{Sternwartestraße}|pw}}{\lemma{\textnormal{\emph{Sternwartegaſſe}}}\Cendnote{\textnormal{richtig: Sternwartestraße\oindex{Sternwartestrasse@\textbf{Sternwartestraße}|pwk}}}}\label{K_L02153_2h} zu kommen. Da {\pb}würde ich Sie
                    um nichts von Ihrer Arbeitszeit berauben, und zugleich das Vergnügen einer
                    gemüthlichen Unterhaltung mit Ihnen beiden \strikeout{ge}
                    haben.\pend
           \pstart
           Mit ausgezeichneter Hochachtung{\\[\baselineskip]}Ihre erg.{\\[\baselineskip]}\spacefill\mbox{Bertha v. Suttner}\pend
           \leftskip=0em{}\endnumbering\briefempfaengerindex{Schnitzler, Arthur@\textsc{Schnitzler, Arthur}!zzzSuttner, Bertha von@\emph{von Bertha von Suttner}!1913-10-221@{22. 10. 1913}|)be}\mylabel{h}\end{ledgroupsized}  \newcommand{\dateiname}{L02153}\newcommand{\titel}{Bertha von Suttner an Arthur Schnitzler, 22. 10. 1913}\newcommand{\editorInnen}{Martin Anton Müller und Gerd-Hermann Susen}
            \footnotesize
\begin{ledgroupsized}[t]{11.5cm}
\doendnotes{C}
\end{ledgroupsized}
         %% latex-leseansicht-abspann.tex
%% Abspann für die Leseansicht.
%% Der Schalter \ifkorrekturansicht ist bereits durch den Vorspann gesetzt.

%% latex-abspann.tex
%% Gemeinsamer Abspann für Korrekturansicht und Leseansicht.
%% Setzt den Schalter \ifkorrekturansicht voraus (gesetzt in den
%% einbindenden Dateien latex-korrekturansicht-abspann.tex bzw.
%% latex-leseansicht-abspann.tex).
%% ---------------------------------------------------------------

\normalsize

% Das esempio-Environment wird nur in der Leseansicht benötigt
\ifkorrekturansicht\else
\newenvironment{esempio}[3]%
{
    \vspace{1.5ex}
    \rlap{\underline{#1}}
    \par
    \setlength{\parindent}{0cm}
    \nopagebreak
    \leftskip=#2cm
    \rightskip=#3cm
}
{
    \par
}
\fi

\doendnotes{C}
\bigskip
\vfill

\clearpage

\footnotesize

\ifkorrekturansicht
  \lohead{\textsc{register}}
\fi

% theindex-Environment neu definieren ohne reledmac
\makeatletter
\renewenvironment{theindex}{%
  \ifkorrekturansicht
    \section*{\indexname}%
  \else
    \subsubsection*{Index der erwähnten Entitäten}%
  \fi
  \setlength{\parindent}{0pt}%
  \setlength{\parskip}{0pt plus 0.3pt}%
  \let\item\@idxitem
}{%
  \ifkorrekturansicht\clearpage\fi
}
\makeatother

\IfFileExists{\jobname-pw.ind}{\input{\jobname-pw.ind}}{}

% Quellenangabe nur in der Leseansicht
\ifkorrekturansicht\else
% Fallback-Definitionen, falls die .tex-Datei \titel etc. nicht gesetzt hat
\providecommand{\titel}{}
\providecommand{\editorInnen}{}
\providecommand{\dateiname}{\jobname}

\vspace{3cm}

\vfill

\footnotesize
\textsc{Quelle}: \titel. Herausgegeben von {\editorInnen}. In: \emph{Arthur Schnitzler: Briefwechsel mit Autorinnen und Autoren}.
 Digitale Edition, https://schnitzler-briefe.acdh.oeaw.ac.at/{\dateiname}.html (Stand \today)
\fi

\end{document}


      