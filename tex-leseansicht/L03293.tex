%% latex-leseansicht-vorspann.tex
%% Vorspann für die Leseansicht.
%% Lädt die gemeinsame Datei latex-vorspann.tex mit nicht gesetztem Schalter.

\newif\ifkorrekturansicht
\korrekturansichtfalse

\input{../tex-inputs/latex-vorspann}

\begin{center}
            \textcolor{red}{ENTWURF, NICHT FERTIG KORRIGIERT}
                      \end{center}
            
         
         \renewcommand{\erwaehntePersonen}{Personen: Paul Goldmann, Georg Hirschfeld, Hugo von Hofmannsthal, Gustav Schwarzkopf}
         \renewcommand{\erwaehnteInstitutionen}{Institutionen: Burgtheater, Wiener Allgemeine Montagszeitung, Wiener Allgemeine Zeitung}
         \renewcommand{\erwaehnteOrte}{Orte: Berlin, Pelikangasse, Schulerstraße, Wien}
         \renewcommand{\erwaehnteWerke}{}
               \section[Felix Salten an Arthur Schnitzler, 21. 6. 1899]{ Felix Salten an Arthur Schnitzler, 21. 6. 1899}\nopagebreak\mylabel{v}\rehead{ }\begin{ledgroupsized}[t]{13cm}\normalsize\beginnumbering \toendnotes[C]{\smallbreak\pagebreak[2]} \Standort{CUL, Schnitzler, B 89, A 2.}
\physDesc{Brief, 1 Blatt, 1 Seite
\newline{}Handschrift: schwarze Tinte, lateinische Kurrent\newline{}Ordnung: mit Bleistift von unbekannter Hand nummeriert:
                                    »117« }\pstart
           \noindent{}{\pb}\textcolor{gray}{\textbf{\textbf{»Wiener Allgemeine
                        Zeitung\orgindex{Wiener Allgemeine Zeitung@Wiener Allgemeine Zeitung|pw}«}}}\pend
           \pstart
           \textcolor{gray}{\textbf{Redaction:}}\pend
           \pstart
           \textcolor{gray}{\textbf{\textbf{IX/2, Pelikangaſſe Nr. 4\oindex{Pelikangasse@\textbf{Pelikangasse}|pw}.}}}\pend
           \pstart
           \textcolor{gray}{\textbf{Administration:}}\hfill \textcolor{gray}{\textbf{Wien\oindex{Wien@\textbf{Wien}|pw}, am}}{ }21. Juni \textcolor{gray}{\textbf{189}}9\pend
           \pstart
           \textcolor{gray}{\textbf{\textbf{I. Schulerſtraße Nr. 20\oindex{Schulerstrasse@\textbf{Schulerstraße}|pw}. }}}\pend
           \pstart
           \textcolor{gray}{\textbf{Telegramm-Adreſſe: »Allgemeine, Wien\oindex{Wien@\textbf{Wien}|pw}«.}}\pend
           \pstart
           \textcolor{gray}{\textbf{Telephon der Redaction: Nr. 805 u. 2180.}}\pend
           \pstart
           \textcolor{gray}{\textbf{\hspace*{2.5em}„\hspace*{2.5em}„\hspace*{2.5em} Adminiſtration: Nr. 1024.}}\pend
           \pstart{}Lieber Arthur,\pend\pstart
           die »W\textsuperscript{r} Allg. Ztg\orgindex{Wiener Allgemeine Zeitung@Wiener Allgemeine Zeitung|pw}«
               läßt vom 3. Juli an ein Montag früh Blatt erscheinen, das mit einer
               literarischen Revue verbunden ist. Die Revue führt den Titel »W\textsuperscript{r} Allg. Rundschau\orgindex{Wiener Allgemeine Montagszeitung@Wiener Allgemeine Montagszeitung|pw}.« Sie ist etwas durchaus
               Selbstständiges, keine Rubrik im Blatt, und soll nach dem Wunsch der Unternehmer
               selbst »ersten Ranges« werden. Die Zeitung habe ich erhalten, und Sie können sich
               denken, dass ich gerne in unserem Sinne daraus machen möchte. Da mir so wenig Zeit
               zur Vorbereitung bleibt, ist die Gefahr groß, dass ich von Anfang an, in
               Schwierigkeiten (in künstlerische) gerathe. Ich bitte Sie dringend, mir was immer zur
               ersten, event. zweiten N\textsuperscript{m.} zu geben. Großes oder Kleines.
               An Hofmannsthal\pwindex{Hofmannsthal, Hugo von 1874-02-01 – 1929-07-15@\textsc{Hofmannsthal, Hugo von} (1874-02-01 – 1929-07-15), \emph{Schriftsteller}|pw} schrieb ich bereits, und bitte Sie nur,
               nochmals auch ihn zur schleunigen Einsendung zu veranlaßen. Jetzt, (1\textsuperscript{h}) besuche ichSchwarzkopf\pwindex{Schwarzkopf, Gustav 07.11.1853 – 13.11.1939@\textsc{Schwarzkopf, Gustav} (07.11.1853 – 13.11.1939), \emph{Schriftsteller}|pw}. Hirschfeld\pwindex{Hirschfeld, Georg 11.02.1873 – 17.01.1942@\textsc{Hirschfeld, Georg} (11.02.1873 – 17.01.1942), \emph{Schriftsteller}|pw}, mit dem ich heute
               Abds. nach Berlin\oindex{Berlin@\textbf{Berlin}|pw} fahre, hat die Correspondenz
               für Berlin\oindex{Berlin@\textbf{Berlin}|pw} über Theater, Kunst zu ganz bestimmten
               Terminen übernommen. \pend
           \pstart
           Montag früh bin ich wieder da, Abds im Burgtheater\orgindex{Burgtheater@Burgtheater|pw} und
               nachher kann ich Sie hoffentl. im Caféhaus sprechen. \pend
           \pstart
           Nochmals bitte, senden Sie mir was immer. Das Honorar ist gut. \pend
           \pstart
           Herzlichst Ihr {\\[\baselineskip]}\spacefill\mbox{Salten}\pend
           \leftskip=0em{}\pstart
           \noindent{}An D\textsuperscript{r}Goldmann\pwindex{Goldmann, Paul 31.01.1865 – 25.09.1935@\textsc{Goldmann, Paul} (31.01.1865 – 25.09.1935), \emph{Schriftsteller, Journalist}|pw} schreibe ich eben, bitte schreiben
                  auch Sie an ihn und reden ihm zu. Es ist vielleicht gut, dass er wieder auch für
                     Wien\oindex{Wien@\textbf{Wien}|pw} schreibt. \pend
           
         
         \endnumbering\mylabel{h}\end{ledgroupsized}\begin{anhang}\end{anhang}\newcommand{\dateiname}{L03293}\newcommand{\titel}{Felix Salten an Arthur Schnitzler, 21. 6. 1899}\newcommand{\editorInnen}{Martin Anton Müller und Laura Untner}%% latex-leseansicht-abspann.tex
%% Abspann für die Leseansicht.
%% Der Schalter \ifkorrekturansicht ist bereits durch den Vorspann gesetzt.

%% latex-abspann.tex
%% Gemeinsamer Abspann für Korrekturansicht und Leseansicht.
%% Setzt den Schalter \ifkorrekturansicht voraus (gesetzt in den
%% einbindenden Dateien latex-korrekturansicht-abspann.tex bzw.
%% latex-leseansicht-abspann.tex).
%% ---------------------------------------------------------------

\normalsize

% Das esempio-Environment wird nur in der Leseansicht benötigt
\ifkorrekturansicht\else
\newenvironment{esempio}[3]%
{
    \vspace{1.5ex}
    \rlap{\underline{#1}}
    \par
    \setlength{\parindent}{0cm}
    \nopagebreak
    \leftskip=#2cm
    \rightskip=#3cm
}
{
    \par
}
\fi

\doendnotes{C}
\bigskip
\vfill

\clearpage

\footnotesize

\ifkorrekturansicht
  \lohead{\textsc{register}}
\fi

% theindex-Environment neu definieren ohne reledmac
\makeatletter
\renewenvironment{theindex}{%
  \ifkorrekturansicht
    \section*{\indexname}%
  \else
    \subsubsection*{Index der erwähnten Entitäten}%
  \fi
  \setlength{\parindent}{0pt}%
  \setlength{\parskip}{0pt plus 0.3pt}%
  \let\item\@idxitem
}{%
  \ifkorrekturansicht\clearpage\fi
}
\makeatother

\IfFileExists{\jobname-pw.ind}{\input{\jobname-pw.ind}}{}

% Quellenangabe nur in der Leseansicht
\ifkorrekturansicht\else
% Fallback-Definitionen, falls die .tex-Datei \titel etc. nicht gesetzt hat
\providecommand{\titel}{}
\providecommand{\editorInnen}{}
\providecommand{\dateiname}{\jobname}

\vspace{3cm}

\vfill

\footnotesize
\textsc{Quelle}: \titel. Herausgegeben von {\editorInnen}. In: \emph{Arthur Schnitzler: Briefwechsel mit Autorinnen und Autoren}.
 Digitale Edition, https://schnitzler-briefe.acdh.oeaw.ac.at/{\dateiname}.html (Stand \today)
\fi

\end{document}


      