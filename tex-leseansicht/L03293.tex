%% latex-leseansicht-vorspann.tex
%% Vorspann für die Leseansicht.
%% Lädt die gemeinsame Datei latex-vorspann.tex mit nicht gesetztem Schalter.

\newif\ifkorrekturansicht
\korrekturansichtfalse

\input{../tex-inputs/latex-vorspann}


\section[ Felix Salten an Arthur Schnitzler, 21. 6. 1899]{L03293 Felix Salten an Arthur Schnitzler,  21. 6. 1899}
\nopagebreak\mylabel{L03293v}
\rehead{ }\normalsize\beginnumbering\briefempfaengerindex{Schnitzler, Arthur@\textsc{Schnitzler, Arthur}!zzzSalten, Felix@\emph{von Felix Salten}!1899-06-213@{21. 6. 1899}|(be}
\toendnotes[C]{\smallbreak\pagebreak[2]}
\correspDesc{Versand  durch Felix Salten am 21. 6. 1899 in Wien
\newline{}Erhalt  durch Arthur Schnitzler im Zeitraum [21. 6. 1899
                  – 25. 6. 1899?] in Wien?}\toendnotes[C]{\smallbreak}
\Standort{CUL, Schnitzler, B 89, A 2.}
\physDesc{Brief, 1 Blatt, 1 Seite, 1289 Zeichen
\newline{}Handschrift: schwarze Tinte, lateinische Kurrent
\newline{}Ordnung: mit Bleistift von unbekannter Hand nummeriert: »117« }\toendnotes[C]{\smallbreak}
\pstart
           {\pb}\textcolor{gray}{\textbf{\textbf{»Wiener Allgemeine
                        Zeitung\orgindex{Wiener Allgemeine Zeitung@Wiener Allgemeine Zeitung|pw}«}}}\pend
           
\pstart
           \textcolor{gray}{\textbf{Redaction:}}\pend
           
\pstart
           \textcolor{gray}{\textbf{IX/2. \textbf{Pelikangaſſe
                        Nr.} 4\oindex{Wien@\textbf{Wien}!IX., Alsergrund@\textbf{IX., Alsergrund}!Pelikangasse@\textbf{Pelikangasse}, \emph{Straße}|pw}.}}\pend
           
\pstart
           \textcolor{gray}{\textbf{Adminiſtration:}}\hfill \textcolor{gray}{\textbf{Wien\oindex{Wien@\textbf{Wien}, \emph{Verwaltungsgebiet}|pw}, am}}{ }21. Juni \textcolor{gray}{\textbf{189}}9.\pend
           
\pstart
           \textcolor{gray}{\textbf{I. \textbf{Schulerſtraße
                        Nr.} 20\oindex{Wien@\textbf{Wien}!I., Innere Stadt@\textbf{I., Innere Stadt}!Schulerstraße@\textbf{Schulerstraße}, \emph{Straße}|pw}.}}\pend
           
\pstart
           \textcolor{gray}{\textbf{Telegramm-Adreſſe: »Allgemeine, Wien\oindex{Wien@\textbf{Wien}, \emph{Verwaltungsgebiet}|pw}«.}}\pend
           
\pstart
           \textcolor{gray}{\textbf{Telephon der Redaction: Nr. 805 u. 2180.}}\pend
           
\pstart
           \textcolor{gray}{\textbf{\hspace*{1.5em}„\hspace*{1.5em}„\hspace*{1.5em} Adminiſtration: Nr. 1024.}}\pend
           
\pstart{}Lieber Arthur,\pend\vspace{0.5em}
\pstart
           die »W\textsuperscript{r} Allg. Ztg\orgindex{Wiener Allgemeine Zeitung@Wiener Allgemeine Zeitung|pw}«
               läßt vom 3. Juli an ein \label{K_L03293-1v}\edtext{Montagfrühblatt\pwindex{Wiener Allgemeine Montags-Zeitung@\emph{Wiener Allgemeine Montags-Zeitung}|pwv}}{\lemma{\textnormal{\emph{Montagfrühblatt}}}\Cendnote{\textnormal{Die \emph{Wiener Allgemeine Montags-Zeitung}\pwindex{Wiener Allgemeine Montags-Zeitung@\emph{Wiener Allgemeine Montags-Zeitung}|pwk} erschien zwischen dem 3. 7. 1899 und dem 18. 12. 1899. Chefredakteur war Julius Szeps\pwindex{Szeps, Julius 27.\,10.\,1867 Wien – 27.\,10.\,1924 ebd.@\textsc{Szeps, Julius} (27.\,10.\,1867 Wien – 27.\,10.\,1924 ebd.), \emph{Journalist}|pwk}. Die Rubrik \emph{Wiener
                     Allgemeine Rundschau}\pwindex{Wiener Allgemeine Rundschau@\emph{Wiener Allgemeine Rundschau}|pwk} leitete Salten\pwindex{Salten, Felix 6.\,9.\,1869 Budapest – 8.\,10.\,1945 Zürich@\textsc{Salten, Felix} (6.\,9.\,1869 Budapest – 8.\,10.\,1945 Zürich), \emph{Schriftsteller, Journalist, Chefredakteur}|pwk}.}}}\label{K_L03293-1} erscheinen, das mit einer literarischen Revue\pwindex{Wiener Allgemeine Rundschau@\emph{Wiener Allgemeine Rundschau}|pwv} verbunden ist. Die Revue führt den Titel
                  »W\textsuperscript{r} Allg.
                  Rundschau\pwindex{Wiener Allgemeine Rundschau@\emph{Wiener Allgemeine Rundschau}|pw}.« Sie ist etwas durchaus Selbstständiges, keine Rubrik im Blatt\pwindex{Wiener Allgemeine Montags-Zeitung@\emph{Wiener Allgemeine Montags-Zeitung}|pwv}, und soll nach dem
               Wunsch der Unternehmer\orgindex{Wiener Allgemeine Zeitung@Wiener Allgemeine Zeitung|pwv} selbst,
               »ersten Ranges« werden. Die Leitung habe ich erhalten, und Sie können sich denken,
               dass ich gerne etwas in unserem Sinne daraus machen möchte. Da mir so wenig Zeit zur
               Vorbereitung bleibt, ist die Gefahr groß, dass ich von Anfang an, in Schwierigkeiten
               (in künstlerische) gerathe. Ich bitte Sie dringend, mir was immer zur ersten, event.
               zweiten N\textsuperscript{u.} zu \label{K_L03293-2v}\edtext{geben}{\lemma{\textnormal{\emph{geben}}}\Cendnote{\textnormal{Obgleich
                  eine Veröffentlichung von \emph{Reigen}\pwindex{Schnitzler, Arthur 15.\,5.\,1862 Wien – 21.\,10.\,1931 ebd.@\textsc{Schnitzler, Arthur} (15.\,5.\,1862 Wien – 21.\,10.\,1931 ebd.), \emph{Schriftsteller, Mediziner}!Reigen. Zehn Dialoge@\strich\emph{Reigen. Zehn Dialoge}|pwk} in der \emph{Wiener Allgemeinen Montags-Zeitung}\pwindex{Wiener Allgemeine Montags-Zeitung@\emph{Wiener Allgemeine Montags-Zeitung}|pwk} angedacht
                  war (vgl. XXXX Auszeichnungsfehler: Dokument L03294 nicht gefunden), kam es zu
                  keiner Publikation Schnitzlers in dieser Zeitschrift\pwindex{Wiener Allgemeine Montags-Zeitung@\emph{Wiener Allgemeine Montags-Zeitung}|pwkv}.}}}\label{K_L03293-2}. Großes
               oder Kleines. An \label{K_L03293-3v}\edtext{Hofmannsthal\pwindex{Hofmannsthal, Hugo von 1.\,2.\,1874 Wien – 15.\,7.\,1929 Rodaun@\textsc{Hofmannsthal, Hugo von} (1.\,2.\,1874 Wien – 15.\,7.\,1929 Rodaun), \emph{Schriftsteller}|pw}}{\lemma{\textnormal{\emph{Hofmannsthal}}}\Cendnote{\textnormal{In Folge erschien Hugo von Hofmannsthal\pwindex{Hofmannsthal, Hugo von 1.\,2.\,1874 Wien – 15.\,7.\,1929 Rodaun@\textsc{Hofmannsthal, Hugo von} (1.\,2.\,1874 Wien – 15.\,7.\,1929 Rodaun), \emph{Schriftsteller}|pwk}: \emph{Scene aus der »Hochzeit der Sobeide«. (Ältere
                        Niederschrift. Wien 1897. – Ungedruckt)}\pwindex{Hofmannsthal, Hugo von 1.\,2.\,1874 Wien – 15.\,7.\,1929 Rodaun@\textsc{Hofmannsthal, Hugo von} (1.\,2.\,1874 Wien – 15.\,7.\,1929 Rodaun), \emph{Schriftsteller}!Scene aus der »Hochzeit der Sobeide«. (Ältere Niederschrift. Wien 1897. — Ungedruckt.)@\strich\emph{Scene aus der »Hochzeit der Sobeide«. (Ältere Niederschrift. Wien 1897. — Ungedruckt.)}|pwk}. In: \emph{Wiener Allgemeine Montags-Zeitung}\pwindex{Wiener Allgemeine Montags-Zeitung@\emph{Wiener Allgemeine Montags-Zeitung}|pwk}, [Jg. 1, H. 3,]
                        17. 7. 1899, S. 2–3.}}}\label{K_L03293-3} schrieb
               ich bereits, und bitte Sie nur, nochmals auch ihn zur schleunigen Einsendung zu
               veranlaßen. Jetzt, (1\textsuperscript{h.}) besuche ich \label{K_L03293-4v}\edtext{Schwarzkopf\pwindex{Schwarzkopf, Gustav 7.\,11.\,1853 Wien – 13.\,11.\,1939 ebd.@\textsc{Schwarzkopf, Gustav} (7.\,11.\,1853 Wien – 13.\,11.\,1939 ebd.), \emph{Schriftsteller}|pw}}{\lemma{\textnormal{\emph{Schwarzkopf}}}\Cendnote{\textnormal{Auch von Gustav Schwarzkopf\pwindex{Schwarzkopf, Gustav 7.\,11.\,1853 Wien – 13.\,11.\,1939 ebd.@\textsc{Schwarzkopf, Gustav} (7.\,11.\,1853 Wien – 13.\,11.\,1939 ebd.), \emph{Schriftsteller}|pwk} ist keine Publikation in der \emph{Wiener Allgemeinen Montags-Zeitung}\pwindex{Wiener Allgemeine Montags-Zeitung@\emph{Wiener Allgemeine Montags-Zeitung}|pwk}
                  nachweisbar.}}}\label{K_L03293-4}. Hirschfeld\pwindex{Hirschfeld, Georg 11.\,2.\,1873 Berlin – 17.\,1.\,1942 München@\textsc{Hirschfeld, Georg} (11.\,2.\,1873 Berlin – 17.\,1.\,1942 München), \emph{Schriftsteller}|pw}, mit dem
               ich heute{ }abds. nach Berlin\oindex{Berlin@\textbf{Berlin}, \emph{Hauptstadt}|pw} fahre, hat die
               Correspondenz für Berlin\oindex{Berlin@\textbf{Berlin}, \emph{Hauptstadt}|pw} über Theater, Kunst zu
               ganz bestimmten Terminen übernommen. Montag{ }früh bin ich wieder da, abds im Burgtheater\orgindex{Burgtheater@Burgtheater|pw} und nachher kann ich Sie hoffentl. \label{K_L03293-5v}\edtext{im Caféhaus sprechen}{\lemma{\textnormal{\emph{im Caféhaus sprechen}}}\Cendnote{\textnormal{Ein Treffen konnte nicht stattfinden, 
                  Schnitzler war zwischen 23. 6. 1899 und 28. 6. 1899 auf Reisen
                     (Slawonien\oindex{Slawonien@\textbf{Slawonien}, \emph{Region}|pwk}, Budapest\oindex{Budapest@\textbf{Budapest}, \emph{Hauptstadt}|pwk}).}}}\label{K_L03293-5}. Nochmals bitte, senden Sie mir was
               immer. Das Honorar ist gut.\pend
           
\pstart
           Herzlichst Ihr {\\[\baselineskip]}\spacefill\mbox{Salten}\pend
           \leftskip=0em{}
\pstart
           \noindent{}An D\textsuperscript{r}{ }\label{K_L03293-6v}\edtext{Goldmann\pwindex{Goldmann, Paul 31.\,1.\,1865 Breslau – 25.\,9.\,1935 Wien@\textsc{Goldmann, Paul} (31.\,1.\,1865 Breslau – 25.\,9.\,1935 Wien), \emph{Schriftsteller, Journalist}|pw} schreibe ich}{\lemma{\textnormal{\emph{Goldmann schreibe ich}}}\Cendnote{\textnormal{ In der überlieferten Korrespondenz Goldmanns\pwindex{Goldmann, Paul 31.\,1.\,1865 Breslau – 25.\,9.\,1935 Wien@\textsc{Goldmann, Paul} (31.\,1.\,1865 Breslau – 25.\,9.\,1935 Wien), \emph{Schriftsteller, Journalist}|pwk} mit Schnitzler sind keine Hinweise darauf zu finden. In der
                     Korrespondenz Schnitzlers mit Salten\pwindex{Salten, Felix 6.\,9.\,1869 Budapest – 8.\,10.\,1945 Zürich@\textsc{Salten, Felix} (6.\,9.\,1869 Budapest – 8.\,10.\,1945 Zürich), \emph{Schriftsteller, Journalist, Chefredakteur}|pwk} findet sich im Brief vom XXXX Auszeichnungsfehler: Dokument L03295 nicht gefunden die Erwähnung
                     eines mit Goldmann\pwindex{Goldmann, Paul 31.\,1.\,1865 Breslau – 25.\,9.\,1935 Wien@\textsc{Goldmann, Paul} (31.\,1.\,1865 Breslau – 25.\,9.\,1935 Wien), \emph{Schriftsteller, Journalist}|pwk} in Beziehung
                     stehenden Feuilletons\pwindex{Salten, Felix 6.\,9.\,1869 Budapest – 8.\,10.\,1945 Zürich@\textsc{Salten, Felix} (6.\,9.\,1869 Budapest – 8.\,10.\,1945 Zürich), \emph{Schriftsteller, Journalist, Chefredakteur}!?? [Feuilleton über Paul Goldmann]@\strich\emph{?? [Feuilleton über Paul Goldmann]}|pwkv}.}}}\label{K_L03293-6} eben, bitte schreiben auch Sie an ihn und reden ihm zu. Es
                  ist vielleicht gut, wenn er wieder auch für Wien\oindex{Wien@\textbf{Wien}, \emph{Verwaltungsgebiet}|pw} schreibt.\pend
           \selectlanguage{ngerman}\endnumbering\briefempfaengerindex{Schnitzler, Arthur@\textsc{Schnitzler, Arthur}!zzzSalten, Felix@\emph{von Felix Salten}!1899-06-213@{21. 6. 1899}|)be}\mylabel{L03293h}  \newcommand{\dateiname}{L03293}\newcommand{\titel}{Felix Salten an Arthur Schnitzler, 21. 6. 1899}\newcommand{\editorInnen}{Martin Anton Müller und Laura Untner}%% latex-leseansicht-abspann.tex
%% Abspann für die Leseansicht.
%% Der Schalter \ifkorrekturansicht ist bereits durch den Vorspann gesetzt.

%% latex-abspann.tex
%% Gemeinsamer Abspann für Korrekturansicht und Leseansicht.
%% Setzt den Schalter \ifkorrekturansicht voraus (gesetzt in den
%% einbindenden Dateien latex-korrekturansicht-abspann.tex bzw.
%% latex-leseansicht-abspann.tex).
%% ---------------------------------------------------------------

\normalsize

% Das esempio-Environment wird nur in der Leseansicht benötigt
\ifkorrekturansicht\else
\newenvironment{esempio}[3]%
{
    \vspace{1.5ex}
    \rlap{\underline{#1}}
    \par
    \setlength{\parindent}{0cm}
    \nopagebreak
    \leftskip=#2cm
    \rightskip=#3cm
}
{
    \par
}
\fi

\doendnotes{C}
\bigskip
\vfill

\clearpage

\footnotesize

\ifkorrekturansicht
  \lohead{\textsc{register}}
\fi

% theindex-Environment neu definieren ohne reledmac
\makeatletter
\renewenvironment{theindex}{%
  \ifkorrekturansicht
    \section*{\indexname}%
  \else
    \subsubsection*{Index der erwähnten Entitäten}%
  \fi
  \setlength{\parindent}{0pt}%
  \setlength{\parskip}{0pt plus 0.3pt}%
  \let\item\@idxitem
}{%
  \ifkorrekturansicht\clearpage\fi
}
\makeatother

\IfFileExists{\jobname-pw.ind}{\input{\jobname-pw.ind}}{}

% Quellenangabe nur in der Leseansicht
\ifkorrekturansicht\else
% Fallback-Definitionen, falls die .tex-Datei \titel etc. nicht gesetzt hat
\providecommand{\titel}{}
\providecommand{\editorInnen}{}
\providecommand{\dateiname}{\jobname}

\vspace{3cm}

\vfill

\footnotesize
\textsc{Quelle}: \titel. Herausgegeben von {\editorInnen}. In: \emph{Arthur Schnitzler: Briefwechsel mit Autorinnen und Autoren}.
 Digitale Edition, https://schnitzler-briefe.acdh.oeaw.ac.at/{\dateiname}.html (Stand \today)
\fi

\end{document}


