%% latex-leseansicht-vorspann.tex
%% Vorspann für die Leseansicht.
%% Lädt die gemeinsame Datei latex-vorspann.tex mit nicht gesetztem Schalter.

\newif\ifkorrekturansicht
\korrekturansichtfalse

\input{../tex-inputs/latex-vorspann}


         
         \renewcommand{\erwaehntePersonen}{Personen: Paul Goldmann, Georg Hirschfeld, Hugo von Hofmannsthal, Felix Salten, Gustav Schwarzkopf, Julius Szeps}
         \renewcommand{\erwaehnteInstitutionen}{Institutionen: Burgtheater, Wiener Allgemeine Zeitung}
         \renewcommand{\erwaehnteOrte}{Orte: Berlin, Budapest, Pelikangasse, Schulerstraße, Slawonien, Wien}
         \renewcommand{\erwaehnteWerke}{Werke: ?? [Feuilleton über Paul Goldmann], Reigen. Zehn Dialoge, Scene aus der »Hochzeit der Sobeide«. (Ältere Niederschrift. Wien 1897. — Ungedruckt.), Wiener Allgemeine Montags-Zeitung, Wiener Allgemeine Rundschau}
               \section[ Felix Salten an Arthur Schnitzler, 21. 6. 1899]{ Felix Salten an Arthur Schnitzler, 21. 6. 1899}\nopagebreak\mylabel{v}\rehead{ }\begin{ledgroupsized}[t]{13cm}\normalsize\beginnumbering \toendnotes[C]{\smallbreak\pagebreak[2]} \Standort{CUL, Schnitzler, B 89, A 2.}
\physDesc{Brief, 1 Blatt, 1 Seite, 1287 Zeichen
\newline{}Handschrift: schwarze Tinte, lateinische Kurrent
\newline{}Ordnung: mit Bleistift von unbekannter Hand nummeriert: »117« }\toendnotes[C]{\smallbreak}\pstart
           \noindent{}{\pb}\textcolor{gray}{\textbf{\textbf{»Wiener Allgemeine
                        Zeitung\orgindex{Wiener Allgemeine Zeitung@Wiener Allgemeine Zeitung|pw}«}}}\pend
           \pstart
           \textcolor{gray}{\textbf{Redaction:}}\pend
           \pstart
           \textcolor{gray}{\textbf{IX/2. \textbf{Pelikangaſſe
                        Nr.} 4\oindex{Pelikangasse@\textbf{Pelikangasse}|pw}. }}\pend
           \pstart
           \textcolor{gray}{\textbf{Adminiſtration:}}\hfill \textcolor{gray}{\textbf{Wien\oindex{Wien@\textbf{Wien}|pw}, am}}{ }21. Juni \textcolor{gray}{\textbf{189}}9.\pend
           \pstart
           \textcolor{gray}{\textbf{I. \textbf{Schulerſtraße
                        Nr.} 20\oindex{Schulerstrasse@\textbf{Schulerstraße}|pw}. }}\pend
           \pstart
           \textcolor{gray}{\textbf{Telegramm-Adreſſe: »Allgemeine, Wien\oindex{Wien@\textbf{Wien}|pw}«.}}\pend
           \pstart
           \textcolor{gray}{\textbf{Telephon der Redaction: Nr. 805 u. 2180.}}\pend
           \pstart
           \textcolor{gray}{\textbf{\hspace*{1.5em}„\hspace*{1.5em}„\hspace*{1.5em} Adminiſtration: Nr. 1024.}}\pend
           \pstart{}Lieber Arthur,\pend\pstart
           die »W\textsuperscript{r} Allg. Ztg\orgindex{Wiener Allgemeine Zeitung@Wiener Allgemeine Zeitung|pw}«
               läßt vom 3. Juli an ein \label{K_L03293-1v}\edtext{Montagfrühblatt\pwindex{Wiener Allgemeine Montags-Zeitung1899-07-03 – 1899-12-18@\emph{Wiener Allgemeine Montags-Zeitung} {[}1899-07-03 – 1899-12-18{]}|pwv}}{\lemma{\textnormal{\emph{Montagfrühblatt}}}\Cendnote{\textnormal{Die \emph{Wiener Allgemeine Montags-Zeitung}\pwindex{Wiener Allgemeine Montags-Zeitung1899-07-03 – 1899-12-18@\emph{Wiener Allgemeine Montags-Zeitung} {[}1899-07-03 – 1899-12-18{]}|pwk} erschien zwischen dem 3. 7. 1899 und dem 18. 12. 1899. Chefredakteur war Julius Szeps\pwindex{Szeps, Julius 1867-10-27 – 27.10.1924@\textsc{Szeps, Julius} (1867-10-27 – 27.10.1924), \emph{Journalist}|pwk}. Die Rubrik \emph{Wiener
                     Allgemeine Rundschau}\pwindex{Wiener Allgemeine Rundschau1899-07-03 – 1899-12-18@\emph{Wiener Allgemeine Rundschau} {[}1899-07-03 – 1899-12-18{]}|pwk} leitete Salten\pwindex{Salten, Felix 06.09.1869 – 08.10.1945@\textsc{Salten, Felix} (06.09.1869 – 08.10.1945), \emph{Schriftsteller, Journalist}|pwk}.}}}\label{K_L03293-1h} erscheinen, das mit einer literarischen Revue\pwindex{Wiener Allgemeine Rundschau1899-07-03 – 1899-12-18@\emph{Wiener Allgemeine Rundschau} {[}1899-07-03 – 1899-12-18{]}|pwv} verbunden ist. Die Revue führt den Titel
                  »W\textsuperscript{r} Allg.
                  Rundschau\pwindex{Wiener Allgemeine Rundschau1899-07-03 – 1899-12-18@\emph{Wiener Allgemeine Rundschau} {[}1899-07-03 – 1899-12-18{]}|pw}.« Sie ist etwas durchaus Selbstständiges, keine Rubrik im Blatt\pwindex{Wiener Allgemeine Montags-Zeitung1899-07-03 – 1899-12-18@\emph{Wiener Allgemeine Montags-Zeitung} {[}1899-07-03 – 1899-12-18{]}|pwv}, und soll nach dem
               Wunsch der Unternehmer\orgindex{Wiener Allgemeine Zeitung@Wiener Allgemeine Zeitung|pwv} selbst,
               »ersten Ranges« werden. Die Leitung habe ich erhalten, und Sie können sich denken,
               dass ich gerne etwas in unserem Sinne daraus machen möchte. Da mir so wenig Zeit zur
               Vorbereitung bleibt, ist die Gefahr groß, dass ich von Anfang an, in Schwierigkeiten
               (in künstlerische) gerathe. Ich bitte Sie dringend, mir was immer zur ersten, event.
               zweiten N\textsuperscript{u.} zu \label{K_L03293-2v}\edtext{geben}{\lemma{\textnormal{\emph{geben}}}\Cendnote{\textnormal{Obgleich
                  eine Veröffentlichung von \emph{Reigen}\pwindex{Schnitzler, Arthur 15.05.1862 – 21.10.1931@\textsc{Schnitzler, Arthur} (15.05.1862 – 21.10.1931), \emph{Schriftsteller, Mediziner}!Reigen. Zehn Dialoge1900@\strich\emph{Reigen. Zehn Dialoge} {[}1900{]}|pwk} in der \emph{Wiener Allgemeinen Montags-Zeitung}\pwindex{Wiener Allgemeine Montags-Zeitung1899-07-03 – 1899-12-18@\emph{Wiener Allgemeine Montags-Zeitung} {[}1899-07-03 – 1899-12-18{]}|pwk} angedacht
                  war (vgl. Felix Salten an Arthur Schnitzler, [27. 6. 1899]), kam es zu
                  keiner Publikation Schnitzler\pwindex{Schnitzler, Arthur 15.05.1862 – 21.10.1931@\textsc{Schnitzler, Arthur} (15.05.1862 – 21.10.1931), \emph{Schriftsteller, Mediziner}|pwk}s in dieser Zeitschrift\pwindex{Wiener Allgemeine Montags-Zeitung1899-07-03 – 1899-12-18@\emph{Wiener Allgemeine Montags-Zeitung} {[}1899-07-03 – 1899-12-18{]}|pwkv}.}}}\label{K_L03293-2h}. Großes
               oder Kleines. An \label{K_L03293-3v}\edtext{Hofmannsthal\pwindex{Hofmannsthal, Hugo von 1874-02-01 – 1929-07-15@\textsc{Hofmannsthal, Hugo von} (1874-02-01 – 1929-07-15), \emph{Schriftsteller}|pw}}{\lemma{\textnormal{\emph{Hofmannsthal}}}\Cendnote{\textnormal{In Folge erschien Hugo von Hofmannsthal\pwindex{Hofmannsthal, Hugo von 1874-02-01 – 1929-07-15@\textsc{Hofmannsthal, Hugo von} (1874-02-01 – 1929-07-15), \emph{Schriftsteller}|pwk}: \emph{Scene aus der »Hochzeit der Sobeide«. (Ältere
                        Niederschrift. Wien 1897. — Ungedruckt.)}\pwindex{?? Werk@Nicht ermittelte Verfasserinnen und Verfasser!Scene aus der »Hochzeit der Sobeide«. (Aeltere Niederschrift. Wien 1897. —
                  Ungedruckt.)1899-07-17@\emph{Scene aus der »Hochzeit der Sobeide«. (Ältere Niederschrift. Wien 1897. — Ungedruckt.)} {[}1899-07-17{]}|pwk}. In: \emph{Wiener Allgemeine Montags-Zeitung}\pwindex{Wiener Allgemeine Montags-Zeitung1899-07-03 – 1899-12-18@\emph{Wiener Allgemeine Montags-Zeitung} {[}1899-07-03 – 1899-12-18{]}|pwk}, [Jg. 1, H. 3,]
                        17. 7. 1899, S. 2–3.}}}\label{K_L03293-3h} schrieb
               ich bereits, und bitte Sie nur, nochmals auch ihn zur schleunigen Einsendung zu
               veranlaßen. Jetzt, (1\textsuperscript{h.}) besuche ich \label{K_L03293-4v}\edtext{Schwarzkopf\pwindex{Schwarzkopf, Gustav 07.11.1853 – 13.11.1939@\textsc{Schwarzkopf, Gustav} (07.11.1853 – 13.11.1939), \emph{Schriftsteller}|pw}}{\lemma{\textnormal{\emph{Schwarzkopf}}}\Cendnote{\textnormal{Auch von Gustav Schwarzkopf\pwindex{Schwarzkopf, Gustav 07.11.1853 – 13.11.1939@\textsc{Schwarzkopf, Gustav} (07.11.1853 – 13.11.1939), \emph{Schriftsteller}|pwk} ist keine Publikation in der \emph{Wiener Allgemeinen Montags-Zeitung}\pwindex{Wiener Allgemeine Montags-Zeitung1899-07-03 – 1899-12-18@\emph{Wiener Allgemeine Montags-Zeitung} {[}1899-07-03 – 1899-12-18{]}|pwk}
                  nachweisbar.}}}\label{K_L03293-4h}. Hirschfeld\pwindex{Hirschfeld, Georg 11.02.1873 – 17.01.1942@\textsc{Hirschfeld, Georg} (11.02.1873 – 17.01.1942), \emph{Schriftsteller}|pw}, mit dem
               ich heute{ }abds. nach Berlin\oindex{Berlin@\textbf{Berlin}|pw} fahre, hat die
               Correspondenz für Berlin\oindex{Berlin@\textbf{Berlin}|pw} über Theater, Kunst zu
               ganz bestimmten Terminen übernommen. Montag{ }früh bin ich wieder da, abds im Burgtheater\orgindex{Burgtheater@Burgtheater|pw} und nachher kann ich Sie hoffentl. \label{K_L03293-5v}\edtext{im Caféhaus sprechen}{\lemma{\textnormal{\emph{im Caféhaus sprechen}}}\Cendnote{\textnormal{Ein Treffen konnte nicht stattfinden, 
                  Schnitzler\pwindex{Schnitzler, Arthur 15.05.1862 – 21.10.1931@\textsc{Schnitzler, Arthur} (15.05.1862 – 21.10.1931), \emph{Schriftsteller, Mediziner}|pwk} war zwischen 23. 6. 1899 und 28. 6. 1899 auf Reisen
                     (Slawonien\oindex{Slawonien@\textbf{Slawonien}|pwk}, Budapest\oindex{Budapest@\textbf{Budapest}|pwk}).}}}\label{K_L03293-5h}. Nochmals bitte, senden Sie mir was
               immer. Das Honorar ist gut.\pend
           \pstart
           Herzlichst Ihr {\\[\baselineskip]}\spacefill\mbox{Salten}\pend
           \leftskip=0em{}\pstart
           \noindent{}An D\textsuperscript{r}{ }\label{K_L03293-6v}\edtext{Goldmann\pwindex{Goldmann, Paul 31.01.1865 – 25.09.1935@\textsc{Goldmann, Paul} (31.01.1865 – 25.09.1935), \emph{Schriftsteller, Journalist}|pw} schreibe ich}{\lemma{\textnormal{\emph{Goldmann schreibe ich}}}\Cendnote{\textnormal{ In der überlieferten Korrespondenz Goldmann\pwindex{Goldmann, Paul 31.01.1865 – 25.09.1935@\textsc{Goldmann, Paul} (31.01.1865 – 25.09.1935), \emph{Schriftsteller, Journalist}|pwk}s mit Schnitzler\pwindex{Schnitzler, Arthur 15.05.1862 – 21.10.1931@\textsc{Schnitzler, Arthur} (15.05.1862 – 21.10.1931), \emph{Schriftsteller, Mediziner}|pwk} sind keine Hinweise darauf zu finden. In der
                     Korrespondenz Schnitzler\pwindex{Schnitzler, Arthur 15.05.1862 – 21.10.1931@\textsc{Schnitzler, Arthur} (15.05.1862 – 21.10.1931), \emph{Schriftsteller, Mediziner}|pwk}s mit Salten\pwindex{Salten, Felix 06.09.1869 – 08.10.1945@\textsc{Salten, Felix} (06.09.1869 – 08.10.1945), \emph{Schriftsteller, Journalist}|pwk} findet sich im Brief vom 27. 7. 1899 die Erwähnung
                     eines mit Goldmann\pwindex{Goldmann, Paul 31.01.1865 – 25.09.1935@\textsc{Goldmann, Paul} (31.01.1865 – 25.09.1935), \emph{Schriftsteller, Journalist}|pwk} in Beziehung
                     stehenden Feuilleton\pwindex{Salten, Felix 06.09.1869 – 08.10.1945@\textsc{Salten, Felix} (06.09.1869 – 08.10.1945), \emph{Schriftsteller, Journalist}!?? [Feuilleton ueber Paul Goldmann]Ende Juli/Anfang August 1899@\strich\emph{?? [Feuilleton über Paul Goldmann]} {[}Ende Juli/Anfang August 1899{]}|pwkv}s,
                     siehe dort.}}}\label{K_L03293-6h} eben, bitte schreiben auch Sie an ihn und reden ihm zu. Es
                  ist vielleicht gut, wenn er wieder auch für Wien\oindex{Wien@\textbf{Wien}|pw} schreibt.\pend
           
         
         \endnumbering\mylabel{h}\end{ledgroupsized}  \newcommand{\dateiname}{L03293}\newcommand{\titel}{Felix Salten an Arthur Schnitzler, 21. 6. 1899}\newcommand{\editorInnen}{Martin Anton Müller und Laura Untner}%% latex-leseansicht-abspann.tex
%% Abspann für die Leseansicht.
%% Der Schalter \ifkorrekturansicht ist bereits durch den Vorspann gesetzt.

%% latex-abspann.tex
%% Gemeinsamer Abspann für Korrekturansicht und Leseansicht.
%% Setzt den Schalter \ifkorrekturansicht voraus (gesetzt in den
%% einbindenden Dateien latex-korrekturansicht-abspann.tex bzw.
%% latex-leseansicht-abspann.tex).
%% ---------------------------------------------------------------

\normalsize

% Das esempio-Environment wird nur in der Leseansicht benötigt
\ifkorrekturansicht\else
\newenvironment{esempio}[3]%
{
    \vspace{1.5ex}
    \rlap{\underline{#1}}
    \par
    \setlength{\parindent}{0cm}
    \nopagebreak
    \leftskip=#2cm
    \rightskip=#3cm
}
{
    \par
}
\fi

\doendnotes{C}
\bigskip
\vfill

\clearpage

\footnotesize

\ifkorrekturansicht
  \lohead{\textsc{register}}
\fi

% theindex-Environment neu definieren ohne reledmac
\makeatletter
\renewenvironment{theindex}{%
  \ifkorrekturansicht
    \section*{\indexname}%
  \else
    \subsubsection*{Index der erwähnten Entitäten}%
  \fi
  \setlength{\parindent}{0pt}%
  \setlength{\parskip}{0pt plus 0.3pt}%
  \let\item\@idxitem
}{%
  \ifkorrekturansicht\clearpage\fi
}
\makeatother

\IfFileExists{\jobname-pw.ind}{\input{\jobname-pw.ind}}{}

% Quellenangabe nur in der Leseansicht
\ifkorrekturansicht\else
% Fallback-Definitionen, falls die .tex-Datei \titel etc. nicht gesetzt hat
\providecommand{\titel}{}
\providecommand{\editorInnen}{}
\providecommand{\dateiname}{\jobname}

\vspace{3cm}

\vfill

\footnotesize
\textsc{Quelle}: \titel. Herausgegeben von {\editorInnen}. In: \emph{Arthur Schnitzler: Briefwechsel mit Autorinnen und Autoren}.
 Digitale Edition, https://schnitzler-briefe.acdh.oeaw.ac.at/{\dateiname}.html (Stand \today)
\fi

\end{document}


      