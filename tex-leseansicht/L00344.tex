%% latex-leseansicht-vorspann.tex
%% Vorspann für die Leseansicht.
%% Lädt die gemeinsame Datei latex-vorspann.tex mit nicht gesetztem Schalter.

\newif\ifkorrekturansicht
\korrekturansichtfalse

\input{../tex-inputs/latex-vorspann}


         
         \renewcommand{\erwaehntePersonen}{Personen: Max Eugen Burckhard, Karl Glossy}
         \renewcommand{\erwaehnteInstitutionen}{Institutionen: Burgtheater}
         \renewcommand{\erwaehnteOrte}{Orte: Wien}
         \renewcommand{\erwaehnteWerke}{Werke: Anatol, Neue Freie Presse, Schnitzlers Einzug ins Burgtheater}
               \section[Max Burckhard an Arthur Schnitzler, 2. 7. 1894]{ Max Burckhard an Arthur Schnitzler, 2. 7. 1894}\nopagebreak\mylabel{v}\rehead{ }\begin{ledgroupsized}[t]{13cm}\normalsize\beginnumbering \toendnotes[C]{\smallbreak\pagebreak[2]} \Standort{CUL, Schnitzler, B 20.}
\physDesc{Brief, 1 Blatt, 2 Seiten, 447 Zeichen
\newline{}Handschrift: schwarze Tinte, deutsche Kurrent
\newline{}Ordnung: mit rotem Buntstift von unbekannter Hand nummeriert:
                                    »3«, mutmaßlich von anderer Hand mit Bleistift
                                 überschrieben mit: »5« }\buchAbdrucke{\weitereDrucke{1) \pwindex{Glossy, Karl 07.03.1848 – 09.09.1937@\textsc{Glossy, Karl} (07.03.1848 – 09.09.1937), \emph{Schriftsteller, Museumsleiter, Zensurbeirat}!Schnitzlers Einzug ins Burgtheater19. 12. 1931@\strich\emph{Schnitzlers Einzug ins Burgtheater} {[}19. 12. 1931{]}|pwk}\pwindex{Neue Freie Presse1864 – 1939@\emph{Neue Freie Presse} {[}1864 – 1939{]}|pwk}Karl Glossy: \emph{Schnitzlers Einzug ins Burgtheater. Unbekannte Briefe des Dichters.} In: \emph{Neue Freie Presse}, Nr. 24162, 19. 12. 1931, S. 14.} \weitereDrucke{2) \pwindex{Glossy, Karl 07.03.1848 – 09.09.1937@\textsc{Glossy, Karl} (07.03.1848 – 09.09.1937), \emph{Schriftsteller, Museumsleiter, Zensurbeirat}!Schnitzlers Einzug ins Burgtheater19. 12. 1931@\strich\emph{Schnitzlers Einzug ins Burgtheater} {[}19. 12. 1931{]}|pwk}Karl Glossy: \emph{Schnitzlers Einzug ins Burgtheater. Unbekannte Briefe des Dichters.} In: \emph{Wiener Studien und Dokumente}. Zum 85. Geburtstag des Verfassers hg. von seinen Freunden. Wien: \emph{Steyrermühl} 1933, S. 166–168.} \weitereDrucke{3) Hans-Ulrich Lindken: \emph{Arthur Schnitzler. Aspekte und Akzente. Materialien zu Leben
                        und Werk}. Frankfurt am Main, Bern, Göttingen: \emph{Peter Lang} 1984, S. 243–246 (Europäische Hochschulschriften, Reihe 1, Deutsche Sprache und
                        Literatur, 754).} }\toendnotes[C]{\smallbreak}\pstart
           \noindent{}{\pb}\textcolor{gray}{\textbf{\label{T_L00344-1v}\edtext{k. k. Hofburgtheater
                              Direction}{\lemma{\textnormal{\emph{k. k. … Direction}}}\Cendnote{\textnormal{Wappen in
                              Prägedruck}}}\label{T_L00344-1h}\orgindex{Burgtheater@Burgtheater|pw}}}\hfill Wien\oindex{Wien@\textbf{Wien}|pw}{ }2. 7. 94\pend
           \pstart{}Sehr geehrter Herr Doctor!\pend\pstart
           Mit herzlichem Danke ſende ich Ihnen Anatol\pwindex{Schnitzler, Arthur 15.05.1862 – 21.10.1931@\textsc{Schnitzler, Arthur} (15.05.1862 – 21.10.1931), \emph{Schriftsteller, Mediziner}!Anatol1892-10-29@\strich\emph{Anatol} {[}1892-10-29{]}|pw}
               zurück. Alles iſt intereſſant, Vieles ganz ausgezeichnet – aber das was uns gefällt,
               mißfällt Manchen, auf deren Sti{\geminationm}e man hören muſs, \textsc{resp.} deren Sti{\geminationm}e nicht hören
               zu müßen, das beſte ist. Die Cenſur und ein Theil des Publicums wären über das
               »Milieu« in dem Alles ſpielt entrüſtet, denn {\pb}der Publicus liebt es nicht, ſich ſelbſt
               geſpielt zu ſehen.\pend
           \pstart
           Herz1ichſt{\\[\baselineskip]}\spacefill\mbox{D\textsuperscript{r}Burckhard}\pend
           \leftskip=0em{}
         
         \endnumbering\mylabel{h}\end{ledgroupsized}  \newcommand{\dateiname}{L00344}\newcommand{\titel}{Max Burckhard an Arthur Schnitzler, 2. 7. 1894}\newcommand{\editorInnen}{Martin Anton Müller und Gerd-Hermann Susen}%% latex-leseansicht-abspann.tex
%% Abspann für die Leseansicht.
%% Der Schalter \ifkorrekturansicht ist bereits durch den Vorspann gesetzt.

%% latex-abspann.tex
%% Gemeinsamer Abspann für Korrekturansicht und Leseansicht.
%% Setzt den Schalter \ifkorrekturansicht voraus (gesetzt in den
%% einbindenden Dateien latex-korrekturansicht-abspann.tex bzw.
%% latex-leseansicht-abspann.tex).
%% ---------------------------------------------------------------

\normalsize

% Das esempio-Environment wird nur in der Leseansicht benötigt
\ifkorrekturansicht\else
\newenvironment{esempio}[3]%
{
    \vspace{1.5ex}
    \rlap{\underline{#1}}
    \par
    \setlength{\parindent}{0cm}
    \nopagebreak
    \leftskip=#2cm
    \rightskip=#3cm
}
{
    \par
}
\fi

\doendnotes{C}
\bigskip
\vfill

\clearpage

\footnotesize

\ifkorrekturansicht
  \lohead{\textsc{register}}
\fi

% theindex-Environment neu definieren ohne reledmac
\makeatletter
\renewenvironment{theindex}{%
  \ifkorrekturansicht
    \section*{\indexname}%
  \else
    \subsubsection*{Index der erwähnten Entitäten}%
  \fi
  \setlength{\parindent}{0pt}%
  \setlength{\parskip}{0pt plus 0.3pt}%
  \let\item\@idxitem
}{%
  \ifkorrekturansicht\clearpage\fi
}
\makeatother

\IfFileExists{\jobname-pw.ind}{\input{\jobname-pw.ind}}{}

% Quellenangabe nur in der Leseansicht
\ifkorrekturansicht\else
% Fallback-Definitionen, falls die .tex-Datei \titel etc. nicht gesetzt hat
\providecommand{\titel}{}
\providecommand{\editorInnen}{}
\providecommand{\dateiname}{\jobname}

\vspace{3cm}

\vfill

\footnotesize
\textsc{Quelle}: \titel. Herausgegeben von {\editorInnen}. In: \emph{Arthur Schnitzler: Briefwechsel mit Autorinnen und Autoren}.
 Digitale Edition, https://schnitzler-briefe.acdh.oeaw.ac.at/{\dateiname}.html (Stand \today)
\fi

\end{document}


      