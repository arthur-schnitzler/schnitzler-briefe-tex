%% latex-korrekturansicht-vorspann.tex
%% Vorspann für die Korrekturansicht.
%% Lädt die gemeinsame Datei latex-vorspann.tex mit gesetztem Schalter.

\newif\ifkorrekturansicht
\korrekturansichttrue

\input{../tex-inputs/latex-vorspann}


\section[Arthur Schnitzler an Georg Brandes, 11. 5. 1901]{L01117 Arthur Schnitzler an Georg Brandes, 11. 5. 1901}
\nopagebreak\mylabel{L01117v}
\rehead{ }\normalsize\beginnumbering\briefempfaengerindex{Brandes, Georg@\textsc{Brandes, Georg}!zzzSchnitzler, Arthur@\emph{von Arthur Schnitzler}!1901-05-111@{11. 5. 1901}|(be}
\toendnotes[C]{\smallbreak\pagebreak[2]}\Standort{Kopenhagen, Det Kongelige Bibliotek, Georg Brandes Arkiv, box 125.}
\physDesc{Brief, 1 Blatt, 3 Seiten, 610 Zeichen
\newline{}Handschrift: schwarze Tinte, deutsche Kurrent
\newline{}Ordnung: mit Bleistift von unbekannter Hand nummeriert:
                                    »22.« und datiert: »11. 5. 01. \textsc{Schnitzler}« }
\buchAbdrucke{\weitereDrucke{Georg Brandes, Arthur Schnitzler: \emph{Ein Briefwechsel}. Bern: \emph{Francke} 1956, S. 85.} }\toendnotes[C]{\smallbreak}
\pstart
           \noindent{}{\pb}liebſter Herr Brandes, gewiſs bin ich am 16. in Wien\oindex{Wien@\textbf{Wien}, \emph{A.ADM2}|pw} und wäre ſehr froh, Sie wiederzuſehn. Ich
               ſchlage Ihnen vor, von der Bahn direct zu mir zu fahren; Sie kö{\geminationn}en da{\geminationn} bei mir ausruhn und
                  we{\geminationn} es Ihnen paſſt, vor der Abreiſe mit mir und
               meiner Mama\pwindex{Schnitzler, Louise 1840-07-08 – 1911-09-09@\textsc{Schnitzler, Louise} (1840-07-08 – 1911-09-09)|pwv}{ }ſpeiſen; wollen Sie viel{\pb}leicht Richard
                     \textsc{Beer Hofmann}\pwindex{Beer-Hofmann, Richard 1866-07-11 – 1945-09-26@\textsc{Beer-Hofmann, Richard} (1866-07-11 – 1945-09-26), \emph{Schriftsteller/Schriftstellerin}|pw}{ }ſehen, ſo wird er ſehr gern zu mir ko{\geminationm}en. Kurz richten Sie ſich alles ganz nach Ihrer
               Bequemlichkeit ein, ſchreiben Sie mir vorher nur ein Wort, insbeſondere, wa{\geminationn} Ihr Zug weggeht und um wie viel Uhr Sie bei mir eſſen
               wollen.\pend
           
\pstart
           So darf ich alſo wohl ſagen {\pb}auf baldiges
               Wiederſehen.\pend
           
\pstart
           Von Herzen\hspace*{1.5em}Ihr{\\[\baselineskip]}\spacefill\mbox{Arthur Schnitzler}\pend
           \leftskip=0em{}
\pstart
           Wien\oindex{Wien@\textbf{Wien}, \emph{A.ADM2}|pw}, 11. 5. 901. \pend
           \selectlanguage{ngerman}\endnumbering\briefempfaengerindex{Brandes, Georg@\textsc{Brandes, Georg}!zzzSchnitzler, Arthur@\emph{von Arthur Schnitzler}!1901-05-111@{11. 5. 1901}|)be}\mylabel{L01117h}  \normalsize

\doendnotes{C}
\bigskip
\vfill

\clearpage

\footnotesize

\lohead{\textsc{register}}

% Definiere theindex-Environment komplett neu ohne reledmac
\makeatletter
\renewenvironment{theindex}{%
  \section*{\indexname}%
  \setlength{\parindent}{0pt}%
  \setlength{\parskip}{0pt plus 0.3pt}%
  \let\item\@idxitem
}{%
  \clearpage
}
\makeatother

\IfFileExists{\jobname-pw.ind}{\input{\jobname-pw.ind}}{}

\end{document}

      