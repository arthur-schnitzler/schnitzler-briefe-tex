%% latex-korrekturansicht-vorspann.tex
%% Vorspann für die Korrekturansicht.
%% Lädt die gemeinsame Datei latex-vorspann.tex mit gesetztem Schalter.

\newif\ifkorrekturansicht
\korrekturansichttrue

\input{../tex-inputs/latex-vorspann}


\section[Felix Salten: Widmungsexemplar Das Buch der Könige für Arthur Schnitzler, {[}zwischen 1. und 20. 12.{]} 1905]{L03050 Felix Salten: Widmungsexemplar Das Buch der Könige für Arthur
               Schnitzler, {[}zwischen 1. und 20. 12.{]} 1905}
\nopagebreak\mylabel{L03050v}
\rehead{ }\normalsize\beginnumbering\briefempfaengerindex{Schnitzler, Arthur@\textsc{Schnitzler, Arthur}!zzzSalten, Felix@\emph{von Felix Salten}!1905-12-201@{{[}zwischen 1. und 20. 12.{]} 1905}|(be}
\toendnotes[C]{\smallbreak\pagebreak[2]}\Standort{DLA, G:Schnitzler, Arthur (Sammlung Heinrich Schnitzler).}
\physDesc{, 66 Zeichen
\newline{}Handschrift: schwarze Tinte, lateinische Kurrent
\newline{}Ordnung: 1) mit Bleistift von unbekannter Hand Vermerk: »Salzmann, Felix«  2) von der Österreichischen Nationalbibliothek nach der Enteignung einsigniert als »683010-B«}\toendnotes[C]{\smallbreak}
\pstart
           \noindent{}\centering{}{\pb}Meinem lieben Arthur Schnitzler\pend
           
\pstart
           herzlichst{\\[\baselineskip]}\spacefill\mbox{Salten}\pend
           \leftskip=0em{}
\pstart
           Wien\oindex{Wien@\textbf{Wien}, \emph{A.ADM2}|pw}, \label{K_L03050-1v}\edtext{Dezember 05}{\lemma{\textnormal{\emph{Dezember 05}}}\Cendnote{\textnormal{am 27. 10. 1905 vom \emph{Börsenblatt für
                     den deutschen Buchhandel}\pwindex{Boersenblatt fuer den Deutschen Buchhandel@\emph{Börsenblatt für den Deutschen Buchhandel}|pwk} als Neuerscheinung für Anfang November angekündigt}}}\label{K_L03050-1}\pend
           {\vspace{1\baselineskip}}
\pstart
           \centering{}\textcolor{gray}{\textbf{\textbf{\textbf{\so{DAS BUCH DER}}{ }{\\}\textbf{\so{KÖNIGE}}\pwindex{Buch der Koenige@\emph{Das Buch der Könige}|pw}}}}\pend
           
\pstart
           \centering{}\textcolor{gray}{\textbf{VON}}\pend
           
\pstart
           \centering{}\textcolor{gray}{\textbf{\textbf{\so{FELIX SALTEN}}}}\pend
           {\vspace{1\baselineskip}}
\pstart
           \centering{}\textcolor{gray}{\textbf{MIT ZEICHUNGEN}}\pend
           
\pstart
           \centering{}\textcolor{gray}{\textbf{VON}}\pend
           
\pstart
           \centering{}\textcolor{gray}{\textbf{LEO \so{KOBER}\pwindex{Kober, Leo 1876-09-24 – 1931-09-18@\textsc{Kober, Leo} (1876-09-24 – 1931-09-18), \emph{Maler/Malerin, Grafiker/Grafikerin, Illustrator/Illustratorin}|pw}}}\pend
           {\vspace{1\baselineskip}}
\pstart
           \centering{}\textcolor{gray}{\textbf{MÜNCHEN\oindex{Muenchen@\textbf{München}, \emph{P.PPLA}|pw} UND LEIPZIG\oindex{Leipzig@\textbf{Leipzig}, \emph{P.PPLA3}|pw}}}\pend
           
\pstart
           \centering{}\textcolor{gray}{\textbf{BEI GEORG MÜLLER\orgindex{Georg Mueller Verlag@Georg Müller Verlag|pw}}}\pend
           \selectlanguage{ngerman}\endnumbering\briefempfaengerindex{Schnitzler, Arthur@\textsc{Schnitzler, Arthur}!zzzSalten, Felix@\emph{von Felix Salten}!1905-12-011@{{[}zwischen 1. und 20. 12.{]} 1905}|)be}\mylabel{L03050h}  \normalsize

\doendnotes{C}
\bigskip
\vfill

\clearpage

\footnotesize

\lohead{\textsc{register}}

% Definiere theindex-Environment komplett neu ohne reledmac
\makeatletter
\renewenvironment{theindex}{%
  \section*{\indexname}%
  \setlength{\parindent}{0pt}%
  \setlength{\parskip}{0pt plus 0.3pt}%
  \let\item\@idxitem
}{%
  \clearpage
}
\makeatother

\IfFileExists{\jobname-pw.ind}{\input{\jobname-pw.ind}}{}

\end{document}

      