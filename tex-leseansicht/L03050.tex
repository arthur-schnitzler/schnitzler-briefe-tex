%% latex-leseansicht-vorspann.tex
%% Vorspann für die Leseansicht.
%% Lädt die gemeinsame Datei latex-vorspann.tex mit nicht gesetztem Schalter.

\newif\ifkorrekturansicht
\korrekturansichtfalse

\input{../tex-inputs/latex-vorspann}

\begin{center}
            \textcolor{red}{ENTWURF, NICHT FERTIG KORRIGIERT}
                      \end{center}
            
         
         \renewcommand{\erwaehntePersonen}{Personen: Leo Kober}
         \renewcommand{\erwaehnteInstitutionen}{Institutionen: Georg Müller Verlag}
         \renewcommand{\erwaehnteOrte}{Orte: Leipzig, München, Wien}
         \renewcommand{\erwaehnteWerke}{Werke: Das Buch der Könige}
               \section[Felix Salten: Widmungsexemplar Das Buch der Könige für Arthur Schnitzler, {[}zwischen 1. und 20. 12.{]} 1905]{ Felix Salten: Widmungsexemplar Das Buch der Könige für Arthur
               Schnitzler, {[}zwischen 1. und 20. 12.{]} 1905}\nopagebreak\mylabel{v}\rehead{ }\begin{ledgroupsized}[t]{13cm}\normalsize\beginnumbering \toendnotes[C]{\smallbreak\pagebreak[2]} \Standort{DLA, G:Schnitzler, Arthur (Sammlung Heinrich Schnitzler).}
\physDesc{Widmung am Titelblatt, 65 Zeichen
\newline{}Handschrift: schwarze Tinte, lateinische Kurrent}\pstart
           \noindent{}\centering{}{\pb}Meinem lieben Arthur Schnitzler\pend
           \pstart
           herlichst{\\[\baselineskip]}\spacefill\mbox{Salten}\pend
           \leftskip=0em{}\pstart
           Wien\oindex{Wien@\textbf{Wien}|pw}, Dezember 05\pend
           {\bigskip}\pstart
           \noindent{}\centering{}\textcolor{gray}{\textbf{DAS BUCH DER {\\}KÖNIGE\pwindex{Salten, Felix 06.09.1869 – 08.10.1945@\textsc{Salten, Felix} (06.09.1869 – 08.10.1945), \emph{Schriftsteller, Journalist}!Buch der Koenige1905@\strich\emph{Das Buch der Könige} {[}1905{]}|pw}}}\pend
           \pstart
           \noindent{}\centering{}\textcolor{gray}{\textbf{VON}}\pend
           \pstart
           \noindent{}\centering{}\textcolor{gray}{\textbf{FELIX SALTEN}}\pend
           {\bigskip}\pstart
           \noindent{}\centering{}\textcolor{gray}{\textbf{MIT ZEICHUNGEN}}\pend
           \pstart
           \noindent{}\centering{}\textcolor{gray}{\textbf{VON}}\pend
           \pstart
           \noindent{}\centering{}\textcolor{gray}{\textbf{LEO \so{Kober}\pwindex{Kober, Leo 1876-09-24 – 1931-09-18@\textsc{Kober, Leo} (1876-09-24 – 1931-09-18), \emph{Bildender Künstler, Bildender Künstler, Bildender Künstler}|pw}}}\pend
           {\bigskip}\pstart
           \noindent{}\centering{}\textcolor{gray}{\textbf{MÜNCHEN\oindex{Muenchen@\textbf{München}|pw} UND LEIPZIG\oindex{Leipzig@\textbf{Leipzig}|pw}}}\pend
           \pstart
           \noindent{}\centering{}\textcolor{gray}{\textbf{BEI GEORG MÜLLER\orgindex{Georg Mueller Verlag@Georg Müller Verlag|pw}}}\pend
           
         
         \endnumbering\mylabel{h}\end{ledgroupsized}\begin{anhang}\end{anhang}\newcommand{\dateiname}{L03050}\newcommand{\titel}{Felix Salten: Widmungsexemplar Das Buch der Könige für Arthur Schnitzler, [zwischen 1. und 20. 12.] 1905}\newcommand{\editorInnen}{Martin Anton Müller und Laura Untner}%% latex-leseansicht-abspann.tex
%% Abspann für die Leseansicht.
%% Der Schalter \ifkorrekturansicht ist bereits durch den Vorspann gesetzt.

%% latex-abspann.tex
%% Gemeinsamer Abspann für Korrekturansicht und Leseansicht.
%% Setzt den Schalter \ifkorrekturansicht voraus (gesetzt in den
%% einbindenden Dateien latex-korrekturansicht-abspann.tex bzw.
%% latex-leseansicht-abspann.tex).
%% ---------------------------------------------------------------

\normalsize

% Das esempio-Environment wird nur in der Leseansicht benötigt
\ifkorrekturansicht\else
\newenvironment{esempio}[3]%
{
    \vspace{1.5ex}
    \rlap{\underline{#1}}
    \par
    \setlength{\parindent}{0cm}
    \nopagebreak
    \leftskip=#2cm
    \rightskip=#3cm
}
{
    \par
}
\fi

\doendnotes{C}
\bigskip
\vfill

\clearpage

\footnotesize

\ifkorrekturansicht
  \lohead{\textsc{register}}
\fi

% theindex-Environment neu definieren ohne reledmac
\makeatletter
\renewenvironment{theindex}{%
  \ifkorrekturansicht
    \section*{\indexname}%
  \else
    \subsubsection*{Index der erwähnten Entitäten}%
  \fi
  \setlength{\parindent}{0pt}%
  \setlength{\parskip}{0pt plus 0.3pt}%
  \let\item\@idxitem
}{%
  \ifkorrekturansicht\clearpage\fi
}
\makeatother

\IfFileExists{\jobname-pw.ind}{\input{\jobname-pw.ind}}{}

% Quellenangabe nur in der Leseansicht
\ifkorrekturansicht\else
% Fallback-Definitionen, falls die .tex-Datei \titel etc. nicht gesetzt hat
\providecommand{\titel}{}
\providecommand{\editorInnen}{}
\providecommand{\dateiname}{\jobname}

\vspace{3cm}

\vfill

\footnotesize
\textsc{Quelle}: \titel. Herausgegeben von {\editorInnen}. In: \emph{Arthur Schnitzler: Briefwechsel mit Autorinnen und Autoren}.
 Digitale Edition, https://schnitzler-briefe.acdh.oeaw.ac.at/{\dateiname}.html (Stand \today)
\fi

\end{document}


      