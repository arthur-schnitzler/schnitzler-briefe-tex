%% latex-leseansicht-vorspann.tex
%% Vorspann für die Leseansicht.
%% Lädt die gemeinsame Datei latex-vorspann.tex mit nicht gesetztem Schalter.

\newif\ifkorrekturansicht
\korrekturansichtfalse

\input{../tex-inputs/latex-vorspann}


         
         \renewcommand{\erwaehntePersonen}{Personen: Richard Beer-Hofmann, Felix Salten}
         \renewcommand{\erwaehnteOrte}{Orte: Altaussee, Bad Ischl, Salzburg, Schruns}
         \renewcommand{\erwaehnteWerke}{}
               \section[Arthur Schnitzler an Richard Beer-Hofmann, 6. 8. 1900]{ Arthur Schnitzler an Richard Beer-Hofmann, 6. 8. 1900}\nopagebreak\mylabel{v}\rehead{ }\begin{ledgroupsized}[t]{13cm}\normalsize\beginnumbering\briefempfaengerindex{Beer-Hofmann, Richard@\textsc{Beer-Hofmann, Richard}!zzzSchnitzler, Arthur@\emph{von Arthur Schnitzler}!1900-08-061@{6. 8. 1900}|(be} \toendnotes[C]{\smallbreak\pagebreak[2]} \Standort{YCGL, MSS 31.}
\physDesc{Postkarte, 284 Zeichen
\newline{}Handschrift: Bleistift, deutsche Kurrent
\newline{}Versand: 1) Stempel: »\nobreak{}\oindex{Bad Ischl@\textbf{Bad Ischl}|pwk}Ischl, 6. 8. 0\textcolor{gray}{0}\nobreak{}«.   2) Stempel: »\nobreak{}\oindex{Altaussee@\textbf{Altaussee}|pwk}Alt-Aussee, 7/8 \textcolor{gray}{00}\nobreak{}«. }\toendnotes[C]{\smallbreak}\pstart{}{\pb}Herrn Dr. \textsc{Richard
                     Beer-Hofmann}\pend{}\pstart{}\textsc{Altaussee}\oindex{Altaussee@\textbf{Altaussee}|pw}\pend{}{\bigskip}\pstart
           \noindent{}{\pb}lieber Richard, ich möchte, we{\geminationn}’s Ihnen
               recht iſt, am \label{K_L01067-1v}\edtext{Mittwoch}{\lemma{\textnormal{\emph{Mittwoch}}}\Cendnote{\textnormal{Siehe A. S.: \emph{Tagebuch}, 8. 8. 1900.
               }}}\label{K_L01067-1h}{ }Mittag zu Ihnen kommen; we{\geminationn}’s Ihnen nicht
               paſſt, telegraf. Sie mir ab. Bei So{\geminationn}tag{ }Salzburg\oindex{Salzburg@\textbf{Salzburg}|pw} bleibts. \textsc{Salten}\pwindex{Salten, Felix 06.09.1869 – 08.10.1945@\textsc{Salten, Felix} (06.09.1869 – 08.10.1945), \emph{Schriftsteller, Journalist, Chefredakteur}|pw} ko{\geminationm}t vielleicht \label{K_L01067-2v}\edtext{direct nach Schruns\oindex{Schruns@\textbf{Schruns}|pw}}{\lemma{\textnormal{\emph{direct nach Schruns}}}\Cendnote{\textnormal{Vgl. Felix Salten an Arthur Schnitzler, 5. 8. 1900.
               }}}\label{K_L01067-2h}.\pend
           \pstart
           Herzlichſt Ihr{\\[\baselineskip]}\spacefill\mbox{Arthur.}\pend
           \leftskip=0em{}\pstart
           \noindent{}(Ich verlaſſe Iſchl\oindex{Bad Ischl@\textbf{Bad Ischl}|pw}{ }Freitag.)\pend
           
         
         \endnumbering\mylabel{h}\end{ledgroupsized}  \newcommand{\dateiname}{L01067}\newcommand{\titel}{Arthur Schnitzler an Richard Beer-Hofmann, 6. 8. 1900}\newcommand{\editorInnen}{Martin Anton Müller und Gerd-Hermann Susen}%% latex-leseansicht-abspann.tex
%% Abspann für die Leseansicht.
%% Der Schalter \ifkorrekturansicht ist bereits durch den Vorspann gesetzt.

%% latex-abspann.tex
%% Gemeinsamer Abspann für Korrekturansicht und Leseansicht.
%% Setzt den Schalter \ifkorrekturansicht voraus (gesetzt in den
%% einbindenden Dateien latex-korrekturansicht-abspann.tex bzw.
%% latex-leseansicht-abspann.tex).
%% ---------------------------------------------------------------

\normalsize

% Das esempio-Environment wird nur in der Leseansicht benötigt
\ifkorrekturansicht\else
\newenvironment{esempio}[3]%
{
    \vspace{1.5ex}
    \rlap{\underline{#1}}
    \par
    \setlength{\parindent}{0cm}
    \nopagebreak
    \leftskip=#2cm
    \rightskip=#3cm
}
{
    \par
}
\fi

\doendnotes{C}
\bigskip
\vfill

\clearpage

\footnotesize

\ifkorrekturansicht
  \lohead{\textsc{register}}
\fi

% theindex-Environment neu definieren ohne reledmac
\makeatletter
\renewenvironment{theindex}{%
  \ifkorrekturansicht
    \section*{\indexname}%
  \else
    \subsubsection*{Index der erwähnten Entitäten}%
  \fi
  \setlength{\parindent}{0pt}%
  \setlength{\parskip}{0pt plus 0.3pt}%
  \let\item\@idxitem
}{%
  \ifkorrekturansicht\clearpage\fi
}
\makeatother

\IfFileExists{\jobname-pw.ind}{\input{\jobname-pw.ind}}{}

% Quellenangabe nur in der Leseansicht
\ifkorrekturansicht\else
% Fallback-Definitionen, falls die .tex-Datei \titel etc. nicht gesetzt hat
\providecommand{\titel}{}
\providecommand{\editorInnen}{}
\providecommand{\dateiname}{\jobname}

\vspace{3cm}

\vfill

\footnotesize
\textsc{Quelle}: \titel. Herausgegeben von {\editorInnen}. In: \emph{Arthur Schnitzler: Briefwechsel mit Autorinnen und Autoren}.
 Digitale Edition, https://schnitzler-briefe.acdh.oeaw.ac.at/{\dateiname}.html (Stand \today)
\fi

\end{document}


      