%% latex-leseansicht-vorspann.tex
%% Vorspann für die Leseansicht.
%% Lädt die gemeinsame Datei latex-vorspann.tex mit nicht gesetztem Schalter.

\newif\ifkorrekturansicht
\korrekturansichtfalse

\input{../tex-inputs/latex-vorspann}


\section[Arthur Schnitzler an Theodor Herzl, 10. 11. 1894]{L03909 Arthur Schnitzler an Theodor Herzl, 10. 11. 1894}
\nopagebreak\mylabel{L03909v}
\rehead{ }\normalsize\beginnumbering\briefempfaengerindex{Herzl, Theodor@\textsc{Herzl, Theodor}!zzzSchnitzler, Arthur@\emph{von Arthur Schnitzler}!1894-11-102@{10. 11. 1894}|(be}
\toendnotes[C]{\smallbreak\pagebreak[2]}
\correspDesc{Versand  durch Arthur Schnitzler am 10. 11. 1894 in Wien
\newline{}Erhalt  durch Theodor Herzl in Wien}\toendnotes[C]{\smallbreak}
\Standort{Jerusalem, Central Zionist Archives, H1:1924-14.}
\physDesc{,  Blätter,  Seiten
\newline{}Handschrift: , deutsche Kurrent}
\buchAbdrucke{\weitereDrucke{Arthur Schnitzler: \emph{Briefe 1875–1912}. Herausgegeben von Therese Nickl und Heinrich Schnitzler. Frankfurt am Main: \emph{S. Fischer} 1981, S. 235–236.} }\toendnotes[C]{\smallbreak}
\pstart
           {\pb}Wien\oindex{Wien@\textbf{Wien}, \emph{Verwaltungsgebiet}|pw},
                     10. Nov. 94.\pend
           
\pstart{}Lieber und verehrter Freund!\pend\vspace{0.5em}
\pstart
           ich beglückwünſche Sie vor allem zur Vollendung Ihrer Stückes\pwindex{Herzl, Theodor 2.\,5.\,1860 Budapest – 3.\,7.\,1904 Edlach@\textsc{Herzl, Theodor} (2.\,5.\,1860 Budapest – 3.\,7.\,1904 Edlach), \emph{Schriftsteller, Journalist}!neue Ghetto. Schauspiel in vier Acten@\strich\emph{Das neue Ghetto. Schauspiel in vier Acten}|pwv}, das{ }ſchon während{ }ſeines Entſtehens die Miſſion
               erfüllt haben, welche bei{ }ſelbſtkritiſchen Geiſtern wie bei Ihnen{ }ſo{ }ſchwer zu
               erhalten iſt. Ihnen {\pb}eine hohe und{ }ſchöne Sti{\geminationm}ung zu
               geben. Ich habe alſo wohl ein Recht mich auf die Lecture derſelben aufs innigſte zu
               freuen. Es iſt{ }ſelbſtverſtändlich, daſs ich Ihnen in jeder Weiſe und mit dem größten
               Vergnügen zur Verfügung{ }ſtehe. Ich habe auch bereits an einen notar. Vertreter
               gedacht, \textsc{\uline{Schick}\pwindex{Schik, Friedrich *~6.\,9.\,1857 Wien@\textsc{Schik, Friedrich} (*~6.\,9.\,1857 Wien), \emph{Notar, Journalist, Dramaturg}|pw}}. Ke{\geminationn}en Sie ihn? Er hat vor Jahren intim mit \textsc{Ludaßy\pwindex{Gans-Ludassy, Julius von 13.\,4.\,1858 Wien – 30.\,9.\,1922 ebd.@\textsc{Gans-Ludassy, Julius von} (13.\,4.\,1858 Wien – 30.\,9.\,1922 ebd.), \emph{Schriftsteller, Journalist, Herausgeber}|pw}} verkehrt; ich {\pb}komme jetzt häufig mit ihm zuſa{\geminationm}en
               und{ }ſeine Verläßlichkeit iſt außer Zweifel. Im übrigen braucht ja auch ihm gegenüber
               Ihr Name nicht genannt zu werden.\pend
           
\pstart
           Einiges wäre immerhin zu bedenken. Nehmen wir den Fall an, die Direction\orgindex{Deutsches Theater Berlin@Deutsches Theater Berlin|pwv} entſcheidet{ }ſich wirklich binnen
               vier Wochen – wird{ }ſie dann, im Falle der Ablehnung – das Stück\pwindex{Herzl, Theodor 2.\,5.\,1860 Budapest – 3.\,7.\,1904 Edlach@\textsc{Herzl, Theodor} (2.\,5.\,1860 Budapest – 3.\,7.\,1904 Edlach), \emph{Schriftsteller, Journalist}!neue Ghetto. Schauspiel in vier Acten@\strich\emph{Das neue Ghetto. Schauspiel in vier Acten}|pwv} auch an das andere Theater\orgindex{Lessing-Theater@Lessing-Theater|pwv}{\pb}weiterbeförden? – daſs es ihr ein leichtes iſt, wiſſen wir
               ja – es iſt aber nicht zu vergeſſen, daſs es nichts nachläßigeres, rückſichtsloſeres,{ }ſchamloſeres gibt als Theaterdirectionen. Dieſe Nachläßigkeit, Rückſichtsloſigkeit,
               Schamloſigkeit{ }ſteigert{ }ſich ins ungewiſſere,{ }ſobald{ }ſie es mit einem Unbeka{\geminationn}ten zu
               thun haben. Ich glaube also, daſs man{ }ſich an eine Beförderung des Stücks\pwindex{Herzl, Theodor 2.\,5.\,1860 Budapest – 3.\,7.\,1904 Edlach@\textsc{Herzl, Theodor} (2.\,5.\,1860 Budapest – 3.\,7.\,1904 Edlach), \emph{Schriftsteller, Journalist}!neue Ghetto. Schauspiel in vier Acten@\strich\emph{Das neue Ghetto. Schauspiel in vier Acten}|pwv} von Theater zu Theater kaum recht
               verlaſſen kann. Außerdem {\pb}ko{\geminationm}t in Betracht, daſs die
               Vergangenheit eines Stückes auch eine Art Nordau\pwindex{Nordau, Max 29.\,7.\,1849 Budapest – 22.\,1.\,1923 Paris@\textsc{Nordau, Max} (29.\,7.\,1849 Budapest – 22.\,1.\,1923 Paris), \emph{Schriftsteller, Mediziner}|pw}ſcher
               Kugel ist – d. h. das neue
                  Theater\orgindex{Lessing-Theater@Lessing-Theater|pwv} z. B. kann erſtens »gekränkt«{ }ſein, daſs es später als die
               andern berückſichtigt wird u kann zugleich ein Vorurtheil gegen das ein oder zwei
               oder dreimal abgelehnte Stück haben. Ob es nicht, we{\geminationn} auch etwas mühſeliger, doch
               praktiſcher iſt, das Stück i{\geminationm}er wieder an den Notar\pwindex{Schik, Friedrich *~6.\,9.\,1857 Wien@\textsc{Schik, Friedrich} (*~6.\,9.\,1857 Wien), \emph{Notar, Journalist, Dramaturg}|pwv} zurückbefördern laſſen? {\pb}Noch eins. Ich ka{\geminationn} mir denken, daſs die betr. Direction\orgindex{Deutsches Theater Berlin@Deutsches Theater Berlin|pwv}{ }ſich wirklich bi{\geminationn}en vier Wochen entſcheidet –
               obwohl man da nur an die Anſtändigkeit der Direction, alſo eine{ }ſehr imaginäre Größe
               appelliren kann – aber daſs irgend eine Direction die Verpflichtg übernimmt, ein
               eingereichtes Stück innerhalb der nächſten 2 Monate aufzuführen, kann ich mir kaum
               vorſtellen. Man gibt ihr{ }ſogar durch dieſe Clausel eine gar zu {\pb}billige Ausrede in die Hand.\pend
           
\pstart
           – Ob die \textsc{Pseudonym}-Idee an{ }ſich Erfolg verſpricht, iſt{ }ſchwer zu entſcheiden. Sie müſſen eben annehmen, daſs das Werk{ }ſelbſt auf die
               Directoren{ }ſo mächtig wirkt, daſs u. ſ. w. u. ſ. w. Es{ }ſteht in
               dieſem Briefe{ }ſchon{ }ſo viel über die Directoren, daſs es kaum nothwendig iſt, ihnen
               das letzte und traurigſte Epitheton zu erſparen, daſs{ }ſie von der Güte eines echten
               Stücks doch wohl nicht viel verſtehen. {\pb}\textsc{Blumenthal\pwindex{Blumenthal, Oskar 13.\,3.\,1852 Berlin – 24.\,4.\,1917 ebd.@\textsc{Blumenthal, Oskar} (13.\,3.\,1852 Berlin – 24.\,4.\,1917 ebd.), \emph{Schriftsteller, Journalist, Theaterleiter}|pw}} dürfte einen gewiſſen Blick fürs theatraliſche haben. \textsc{Brahm\pwindex{Brahm, Otto 5.\,2.\,1856 Hamburg – 28.\,11.\,1912 Berlin@\textsc{Brahm, Otto} (5.\,2.\,1856 Hamburg – 28.\,11.\,1912 Berlin), \emph{Theaterleiter, Regisseur}|pw}} iſt ein Herr mit Principien und kalten Fanatismen; für einen tiefen Verſteher
               halt ich ihn nicht. \textsc{Lautenburg\pwindex{Lautenburg, Sigmund 11.\,9.\,1851 Budapest – 21.\,7.\,1918 Marienbad@\textsc{Lautenburg, Sigmund} (11.\,9.\,1851 Budapest – 21.\,7.\,1918 Marienbad), \emph{Theaterleiter, Schauspieler}|pw}} iſt einfach ein Dummkopf. Die »Freie Bühne\orgindex{Freie Bühne@Freie Bühne|pw}«
               glaub ich, exiſtirt gar nicht mehr. We{\geminationn} es ein gerades und natürliches Verhältnis
               zwiſchen dem Werth eines Stückes und der Annahme desſelben gäbe, bräuchte das alles
               freilich nicht beſprochen zu werden. Und alles, was ich da {\pb}geſagt habe, wiſſen Sie, lieber Freund,{ }ſo gut wie ich – aber man kommt{ }ſo ins
               plaudern. Daß Herr \textsc{Albert Schnabel} genau{ }ſo auf mich zählen
               kann wie Dr. \textsc{Theod. Herzl}, brauche ich wohl nicht noch
               einmal zu verſichern. Senden Sie mir Ihr Stück\pwindex{Herzl, Theodor 2.\,5.\,1860 Budapest – 3.\,7.\,1904 Edlach@\textsc{Herzl, Theodor} (2.\,5.\,1860 Budapest – 3.\,7.\,1904 Edlach), \emph{Schriftsteller, Journalist}!neue Ghetto. Schauspiel in vier Acten@\strich\emph{Das neue Ghetto. Schauspiel in vier Acten}|pwv} nur{ }ſobald wie möglich. Daſs Sie mir die \textsc{Glosse\pwindex{Herzl, Theodor 2.\,5.\,1860 Budapest – 3.\,7.\,1904 Edlach@\textsc{Herzl, Theodor} (2.\,5.\,1860 Budapest – 3.\,7.\,1904 Edlach), \emph{Schriftsteller, Journalist}!Glosse. Lustspiel in einem Act@\strich\emph{Die Glosse. Lustspiel in einem Act}|pw}} nicht geſchickt haben, iſt nicht{ }ſchön. Aber Sie haben vergeſſen. Meine Novelle\pwindex{Schnitzler, Arthur 15.\,5.\,1862 Wien – 21.\,10.\,1931 ebd.@\textsc{Schnitzler, Arthur} (15.\,5.\,1862 Wien – 21.\,10.\,1931 ebd.), \emph{Schriftsteller, Mediziner}!Sterben. Novelle@\strich\emph{Sterben. Novelle}|pwv} erſcheint in etwa 14 Tagen. Ich werde nicht
               vergeſſen. –\pend
           
\pstart
           {\pb}Ein Stück\pwindex{Schnitzler, Arthur 15.\,5.\,1862 Wien – 21.\,10.\,1931 ebd.@\textsc{Schnitzler, Arthur} (15.\,5.\,1862 Wien – 21.\,10.\,1931 ebd.), \emph{Schriftsteller, Mediziner}!Liebelei. Schauspiel in drei Akten@\strich\emph{Liebelei. Schauspiel in drei Akten}|pwv} hab ich auch geſchrieben. Vom 13. September bis
                  4. October. Und es hat nur 3 Akte. Hoffentlich kann ich
               Ihnen bald günſtiges davon{ }ſagen. Seien Sie vielmals herzlichſt gegrüßt und empfehlen
               Sie mich gütigſt Ihrer Gattin\pwindex{Herzl, Julie 1.\,2.\,1868 Budapest – 10.\,11.\,1907 Wien@\textsc{Herzl, Julie} (1.\,2.\,1868 Budapest – 10.\,11.\,1907 Wien)|pwv}.\pend
           
\pstart
           Ihr treu ergebener{\\[\baselineskip]}\spacefill\mbox{ArthurSchnitzler}\pend
           \leftskip=0em{}
\pstart
           Wien\oindex{Wien@\textbf{Wien}, \emph{Verwaltungsgebiet}|pw}, 10. Nov. 94.\pend
           \selectlanguage{ngerman}\endnumbering\briefempfaengerindex{Herzl, Theodor@\textsc{Herzl, Theodor}!zzzSchnitzler, Arthur@\emph{von Arthur Schnitzler}!1894-11-102@{10. 11. 1894}|)be}\mylabel{L03909h}
\begin{anhang}
\end{anhang}\newcommand{\dateiname}{L03909}\newcommand{\titel}{Arthur Schnitzler an Theodor Herzl, 10. 11. 1894}\newcommand{\editorInnen}{Herausgegeben von Jahnke, SelmaMüller, Martin Anton}%% latex-leseansicht-abspann.tex
%% Abspann für die Leseansicht.
%% Der Schalter \ifkorrekturansicht ist bereits durch den Vorspann gesetzt.

%% latex-abspann.tex
%% Gemeinsamer Abspann für Korrekturansicht und Leseansicht.
%% Setzt den Schalter \ifkorrekturansicht voraus (gesetzt in den
%% einbindenden Dateien latex-korrekturansicht-abspann.tex bzw.
%% latex-leseansicht-abspann.tex).
%% ---------------------------------------------------------------

\normalsize

% Das esempio-Environment wird nur in der Leseansicht benötigt
\ifkorrekturansicht\else
\newenvironment{esempio}[3]%
{
    \vspace{1.5ex}
    \rlap{\underline{#1}}
    \par
    \setlength{\parindent}{0cm}
    \nopagebreak
    \leftskip=#2cm
    \rightskip=#3cm
}
{
    \par
}
\fi

\doendnotes{C}
\bigskip
\vfill

\clearpage

\footnotesize

\ifkorrekturansicht
  \lohead{\textsc{register}}
\fi

% theindex-Environment neu definieren ohne reledmac
\makeatletter
\renewenvironment{theindex}{%
  \ifkorrekturansicht
    \section*{\indexname}%
  \else
    \subsubsection*{Index der erwähnten Entitäten}%
  \fi
  \setlength{\parindent}{0pt}%
  \setlength{\parskip}{0pt plus 0.3pt}%
  \let\item\@idxitem
}{%
  \ifkorrekturansicht\clearpage\fi
}
\makeatother

\IfFileExists{\jobname-pw.ind}{\input{\jobname-pw.ind}}{}

% Quellenangabe nur in der Leseansicht
\ifkorrekturansicht\else
% Fallback-Definitionen, falls die .tex-Datei \titel etc. nicht gesetzt hat
\providecommand{\titel}{}
\providecommand{\editorInnen}{}
\providecommand{\dateiname}{\jobname}

\vspace{3cm}

\vfill

\footnotesize
\textsc{Quelle}: \titel. Herausgegeben von {\editorInnen}. In: \emph{Arthur Schnitzler: Briefwechsel mit Autorinnen und Autoren}.
 Digitale Edition, https://schnitzler-briefe.acdh.oeaw.ac.at/{\dateiname}.html (Stand \today)
\fi

\end{document}


