\input{../tex-inputs/latex-pdf-vorspann}
\begin{center}
            \textcolor{red}{ENTWURF. ENTZIFFERUNG NOCH NICHT KORREKTURGELESEN}
                      \end{center}
            
               \section[Thomas Mann an Arthur Schnitzler, 7. 8. 1908]{ Thomas Mann an Arthur Schnitzler, 7. 8. 1908}\nopagebreak\mylabel{v}\rehead{ }\begin{ledgroupsized}[t]{13cm}\normalsize\beginnumbering\briefempfaengerindex{Schnitzler, Arthur@\textsc{Schnitzler, Arthur}!zzzMann, Thomas@\emph{von Thomas Mann}!1908-08-071@{7. 8. 1908}|(be} \toendnotes[C]{\smallbreak\pagebreak[2]} \Standort{CUL, Schnitzler, B 67.}
\physDesc{Briefkarte
\newline{}Handschrift: schwarze Tinte, deutsche Kurrent
\newline{}Schnitzler: mit Bleistift beschriftet: »\textsc{Mann}« }\buchAbdrucke{\weitereDrucke{Hertha Krotkoff: \emph{Arthur Schnitzler – Thomas Mann: Briefe.} In: \emph{Modern Austrian Literature}, Jg. 7 (1974) Nr. 1/2, S. 13–14.} }\toendnotes[C]{\smallbreak}\pstart
           {\pb}Tölz\oindex{Bad Toelz@\textbf{Bad Tölz}|pw} den 7. August 1908\pend
           \pstart{}Verehrter Herr Doctor:\pend\pstart
           Ich ſchreibe Ihnen nochmals unter Ihrer Wien\oindex{Wien@\textbf{Wien}|pw}er
                    Adreſſe, weil es mir vollkommen unmöglich iſt, die ländliche zu entziffern, –
                    woran wohl noch mehr als Ihre Handſchrift meine mangelhaften geographiſchen
                    Kenntnisse ſchuld ſind.\pend
           \pstart
           Ich habe nichts dagegen, daß Sie {\pb}»Wälſungenblut\pwindex{Mann, Thomas 06.06.1875 – 12.08.1955@\textsc{Mann, Thomas} (06.06.1875 – 12.08.1955), \emph{Schriftsteller}!Waelsungenblut1921@\strich\emph{Wälsungenblut} {[}1921{]}|pw}« Waſſermann\pwindex{Wassermann, Jakob 10.03.1873 – 01.01.1934@\textsc{Wassermann, Jakob} (10.03.1873 – 01.01.1934), \emph{Schriftsteller}|pw} zu leſen geben, geſetzt, daß er noch bei Ihnen iſt. Sagen
                    Sie ihm aber, bitte, daß ich ſie Ihnen der Sache wegen und im Hinblick auf den
                        »Weg ins Freie\pwindex{Schnitzler, Arthur 15.05.1862 – 21.10.1931@\textsc{Schnitzler, Arthur} (15.05.1862 – 21.10.1931), \emph{Schriftsteller, Mediziner}!Weg ins Freie. Roman1.1.1908 – 1.6.1908@\strich\emph{Der Weg ins Freie. Roman} {[}1.1.1908 – 1.6.1908{]}|pw}« geſchickt habe. Er könnte
                    ſich ſonſt gekränkt fühlen. Daß die Novelle\pwindex{Mann, Thomas 06.06.1875 – 12.08.1955@\textsc{Mann, Thomas} (06.06.1875 – 12.08.1955), \emph{Schriftsteller}!Waelsungenblut1921@\strich\emph{Wälsungenblut} {[}1921{]}|pwv} weiter kurſiert, möchte ich Sie bitten zu
                    verhindern.\pend
           \pstart
           Mit den verbindlichſten Grüßen bin ich, verehrter Herr Doctor, Ihr ergebener\pend
           \pstart \spacefill\mbox{Thomas Mann.}\pend{}\endnumbering\briefempfaengerindex{Schnitzler, Arthur@\textsc{Schnitzler, Arthur}!zzzMann, Thomas@\emph{von Thomas Mann}!1908-08-071@{7. 8. 1908}|)be}\mylabel{h}\end{ledgroupsized}  \newcommand{\dateiname}{L01787}\newcommand{\titel}{Thomas Mann an Arthur Schnitzler, 7. 8. 1908}\newcommand{\editorInnen}{Martin Anton Müller und Gerd-Hermann Susen}\input{../tex-inputs/latex-pdf-abspann}
      