%% latex-korrekturansicht-vorspann.tex
%% Vorspann für die Korrekturansicht.
%% Lädt die gemeinsame Datei latex-vorspann.tex mit gesetztem Schalter.

\newif\ifkorrekturansicht
\korrekturansichttrue

\input{../tex-inputs/latex-vorspann}


\section[Thomas Mann an Arthur Schnitzler, 7. 8. 1908]{L01787 Thomas Mann an Arthur Schnitzler, 7. 8. 1908}
\nopagebreak\mylabel{L01787v}
\rehead{ }\normalsize\beginnumbering\briefempfaengerindex{Schnitzler, Arthur@\textsc{Schnitzler, Arthur}!zzzMann, Thomas@\emph{von Thomas Mann}!1908-08-071@{7. 8. 1908}|(be}
\toendnotes[C]{\smallbreak\pagebreak[2]}\Standort{CUL, Schnitzler, B 67.}
\physDesc{Briefkarte, 685 Zeichen
\newline{}Handschrift: schwarze Tinte, deutsche Kurrent
\newline{}Schnitzler: mit Bleistift beschriftet: »\textsc{Mann}« }
\buchAbdrucke{\weitereDrucke{\emph{Modern Austrian Literature}, Jg. 7 (1974) Nr. 1/2, S. 13–14.} }\toendnotes[C]{\smallbreak}
\pstart
           {\pb}Tölz\oindex{Bad Toelz@\textbf{Bad Tölz}, \emph{P.PPLA3}|pw} den 7. August 1908\pend
           
\pstart{}Verehrter Herr Doctor:\pend\vspace{0.5em}
\pstart
           Ich ſchreibe Ihnen nochmals unter Ihrer Wien\oindex{Wien@\textbf{Wien}, \emph{A.ADM2}|pw}er
               Adreſſe, weil es mir vollkommen unmöglich iſt, die ländliche zu entziffern, – woran
               wohl noch mehr als Ihre Handſchrift meine mangelhaften geographiſchen Kenntnisse
               ſchuld ſind.\pend
           
\pstart
           Ich habe nichts dagegen, daß Sie {\pb}»Wälſungenblut\pwindex{Waelsungenblut@\emph{Wälsungenblut}|pw}« Waſſermann\pwindex{Wassermann, Jakob 10.03.1873 – 01.01.1934@\textsc{Wassermann, Jakob} (10.03.1873 – 01.01.1934), \emph{Schriftsteller/Schriftstellerin}|pw} zu leſen geben, geſetzt, daß er noch bei Ihnen iſt. Sagen Sie ihm
               aber, bitte, daß ich ſie Ihnen der Sache wegen und im Hinblick auf den »Weg ins Freie\pwindex{Weg ins Freie. Roman@\emph{Der Weg ins Freie. Roman}|pw}« geſchickt habe. Er könnte ſich
               ſonſt gekränkt fühlen. Daß die Novelle\pwindex{Waelsungenblut@\emph{Wälsungenblut}|pwv} weiter kurſiert, möchte ich Sie bitten zu verhindern.\pend
           
\pstart
           Mit den verbindlichſten Grüßen bin ich, verehrter Herr Doctor, Ihr ergebener\pend
           \pstart \spacefill\mbox{Thomas Mann.}\pend{}\selectlanguage{ngerman}\endnumbering\briefempfaengerindex{Schnitzler, Arthur@\textsc{Schnitzler, Arthur}!zzzMann, Thomas@\emph{von Thomas Mann}!1908-08-071@{7. 8. 1908}|)be}\mylabel{L01787h}  \normalsize

\doendnotes{C}
\bigskip
\vfill

\clearpage

\footnotesize

\lohead{\textsc{register}}

% Definiere theindex-Environment komplett neu ohne reledmac
\makeatletter
\renewenvironment{theindex}{%
  \section*{\indexname}%
  \setlength{\parindent}{0pt}%
  \setlength{\parskip}{0pt plus 0.3pt}%
  \let\item\@idxitem
}{%
  \clearpage
}
\makeatother

\IfFileExists{\jobname-pw.ind}{\input{\jobname-pw.ind}}{}

\end{document}

      