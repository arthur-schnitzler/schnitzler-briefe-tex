%% latex-korrekturansicht-vorspann.tex
%% Vorspann für die Korrekturansicht.
%% Lädt die gemeinsame Datei latex-vorspann.tex mit gesetztem Schalter.

\newif\ifkorrekturansicht
\korrekturansichttrue

\input{../tex-inputs/latex-vorspann}


\section[Arthur Schnitzler an Richard Beer-Hofmann, 25. 2. 1896]{L00536 Arthur Schnitzler an Richard Beer-Hofmann, 25. 2. 1896}
\nopagebreak\mylabel{L00536v}
\rehead{ }\normalsize\beginnumbering\briefempfaengerindex{Beer-Hofmann, Richard@\textsc{Beer-Hofmann, Richard}!zzzSchnitzler, Arthur@\emph{von Arthur Schnitzler}!1896-02-251@{25. 2. 1896}|(be}
\toendnotes[C]{\smallbreak\pagebreak[2]}\Standort{YCGL, MSS 31.}
\physDesc{Brief, 1 Blatt, 1 Seite, Umschlag, 272 Zeichen
\newline{}Handschrift: schwarze Tinte, deutsche Kurrent
\newline{}Versand: Stempel: »\nobreak{}\oindex{I., Innere Stadt@\textbf{I., Innere Stadt}, \emph{A.ADM3}|pwk}Wien 1/1, 26. 2. 96, 12–1 V\nobreak{}«.  }\toendnotes[C]{\smallbreak}\pstart{}{\pb}Herrn \textsc{Dr. Richard
                     Beer-Hofmann}\pend{}\pstart{}Wien\oindex{Wien@\textbf{Wien}, \emph{A.ADM2}|pw}\pend{}\pstart{}\textsc{I. Wollzeile 15\oindex{Wollzeile@\textbf{Wollzeile}, \emph{Straße (K.STR)}|pw}}.\pend{}{\bigskip}\vspace{1em}
\pstart
           {\pb}Abend, Dinstag,{\\}25. 2. 96.\pend
           
\pstart{}Lieber Richard.\pend\vspace{0.5em}
\pstart
           Heute erhielt ich dieſen \label{K_L00536-1v}\edtext{Brief}{\lemma{\textnormal{\emph{Brief}}}\Cendnote{\textnormal{Hugo Bettauer\pwindex{Bettauer, Hugo 18.08.1872 – 26.03.1925@\textsc{Bettauer, Hugo} (18.08.1872 – 26.03.1925), \emph{Schriftsteller/Schriftstellerin, Journalist/Journalistin}|pwk} hatte geschrieben, dass es Fels\pwindex{Fels, Friedrich Michael *~1864@\textsc{Fels, Friedrich Michael} (*~1864), \emph{Journalist/Journalistin}|pwk} in Zürich\oindex{Zuerich@\textbf{Zürich}, \emph{P.PPLA}|pwk} wieder schlecht gehe (\emph{Deutsches Literaturarchiv},
                  HS.NZ85.1.2518).}}}\label{K_L00536-1}. Ich habe ſofort telegrafiſch 25 fl. angewieſen.
               Wenn Sie können, thun Sie dasſelbe; nicht wahr?\pend
           \pstart Herzlich der Ihre, \spacefill\mbox{Arthur}\pend{}
\pstart
           \noindent{}Könnte man auch an Hugo\pwindex{Hofmannsthal, Hugo von 1874-02-01 – 1929-07-15@\textsc{Hofmannsthal, Hugo von} (1874-02-01 – 1929-07-15), \emph{Schriftsteller/Schriftstellerin}|pw} herantreten?\pend
           \selectlanguage{ngerman}\endnumbering\briefempfaengerindex{Beer-Hofmann, Richard@\textsc{Beer-Hofmann, Richard}!zzzSchnitzler, Arthur@\emph{von Arthur Schnitzler}!1896-02-251@{25. 2. 1896}|)be}\mylabel{L00536h}  \normalsize

\doendnotes{C}
\bigskip
\vfill

\clearpage

\footnotesize

\lohead{\textsc{register}}

% Definiere theindex-Environment komplett neu ohne reledmac
\makeatletter
\renewenvironment{theindex}{%
  \section*{\indexname}%
  \setlength{\parindent}{0pt}%
  \setlength{\parskip}{0pt plus 0.3pt}%
  \let\item\@idxitem
}{%
  \clearpage
}
\makeatother

\IfFileExists{\jobname-pw.ind}{\input{\jobname-pw.ind}}{}

\end{document}

      