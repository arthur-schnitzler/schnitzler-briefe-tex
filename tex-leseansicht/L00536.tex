\input{../tex-inputs/latex-pdf-vorspann}
\begin{center}
            \textcolor{red}{ENTWURF. ENTZIFFERUNG NOCH NICHT KORREKTURGELESEN}
                      \end{center}
            
               \section[Arthur Schnitzler an Richard Beer-Hofmann, 25. 2. 1896]{ Arthur Schnitzler an Richard Beer-Hofmann, 25. 2. 1896}\nopagebreak\mylabel{v}\rehead{ }\begin{ledgroupsized}[t]{13cm}\normalsize\beginnumbering\briefempfaengerindex{Beer-Hofmann, Richard@\textsc{Beer-Hofmann, Richard}!zzzSchnitzler, Arthur@\emph{von Arthur Schnitzler}!1896-02-251@{25. 2. 1896}|(be} \toendnotes[C]{\smallbreak\pagebreak[2]} \Standort{YCGL, MSS 31.}
\physDesc{Brief, 1 Blatt (Briefpapier mit Trauerrand), 1 Seite, Umschlag
\newline{}Handschrift: schwarze Tinte, deutsche Kurrent\newline{}Versand: Stempel: »\nobreak{}\oindex{I., Innere Stadt@\textbf{I., Innere Stadt}|pwk}Wien 1/1, 26. 2. 96, 12–1 V\nobreak{}«.  }\toendnotes[C]{\smallbreak}\pstart{}{\pb}Herrn \textsc{Dr. Richard
                     Beer-Hofmann}\pend{}\pstart{}Wien\oindex{Wien@\textbf{Wien}|pw}\pend{}\pstart{}\textsc{I. Wollzeile 15\oindex{Wollzeile@\textbf{Wollzeile}|pw}}.\pend{}{\bigskip}\pstart
           {\pb}Abend, Dinstag,{\\}25. 2. 96.\pend
           \pstart{}Lieber Richard.\pend\pstart
           Heute erhielt ich dieſen \label{K_L00536_1v}\edtext{Brief}{\lemma{\textnormal{\emph{Brief}}}\Cendnote{\textnormal{Hugo Bettauer\pwindex{Bettauer, Hugo 18.08.1872 – 26.03.1925@\textsc{Bettauer, Hugo} (18.08.1872 – 26.03.1925), \emph{Schriftsteller, Journalist}|pwk} hatte geschrieben, dass es Fels\pwindex{Fels, Friedrich Michael *~1864@\textsc{Fels, Friedrich Michael} (*~1864), \emph{Journalist}|pwk} in Zürich\oindex{Zuerich@\textbf{Zürich}|pwk}
                  wieder schlecht gehe (\emph{Deutsches Literaturarchiv}, HS.NZ85.1.2518).}}}\label{K_L00536_1h}. Ich habe ſofort
               telegrafiſch 25 fl. angewieſen. Wenn Sie können, thun Sie dasſelbe; nicht wahr?\pend
           \pstart Herzlich der Ihre, \spacefill\mbox{Arthur}\pend{}\pstart
           \noindent{}Könnte man auch an Hugo\pwindex{Hofmannsthal, Hugo von 01.02.1874 – 15.07.1929@\textsc{Hofmannsthal, Hugo von} (01.02.1874 – 15.07.1929), \emph{Schriftsteller}|pw} herantreten?\pend
           \endnumbering\briefempfaengerindex{Beer-Hofmann, Richard@\textsc{Beer-Hofmann, Richard}!zzzSchnitzler, Arthur@\emph{von Arthur Schnitzler}!1896-02-251@{25. 2. 1896}|)be}\mylabel{h}\end{ledgroupsized}  \newcommand{\dateiname}{L00536}\newcommand{\titel}{Arthur Schnitzler an Richard Beer-Hofmann, 25. 2. 1896}\newcommand{\editorInnen}{Martin Anton Müller und Gerd-Hermann Susen}\input{../tex-inputs/latex-pdf-abspann}
      