%% latex-korrekturansicht-vorspann.tex
%% Vorspann für die Korrekturansicht.
%% Lädt die gemeinsame Datei latex-vorspann.tex mit gesetztem Schalter.

\newif\ifkorrekturansicht
\korrekturansichttrue

\input{../tex-inputs/latex-vorspann}


\section[Olga Schnitzler an Richard Beer-Hofmann, {[}30. 11. 1911?{]}]{L02565 Olga Schnitzler an Richard Beer-Hofmann, {[}30. 11. 1911?{]}}
\nopagebreak\mylabel{L02565v}
\rehead{ }\normalsize\beginnumbering\briefempfaengerindex{Beer-Hofmann, Richard@\textsc{Beer-Hofmann, Richard}!zzzSchnitzler, Olga@\emph{von Olga Schnitzler}!1911-11-301@{{[}30. 11. 1911?{]}}|(be}
\toendnotes[C]{\smallbreak\pagebreak[2]}\Standort{YCGL, MSS 31.}
\physDesc{Briefkarte, 271 Zeichen
\newline{}Handschrift: schwarze Tinte, lateinische Kurrent
\newline{}Beer-Hofmann: mit blauem Buntstift datiert: »1911« }\toendnotes[C]{\smallbreak}
\pstart
           {\pb}\textcolor{gray}{\textbf{O. S.}}\pend
           \vspace{0.5em}
\pstart
           Lieber Herr D\textsuperscript{r}, ich antworte in Arthurs Namen, der \label{K_L02565-1v}\edtext{bei d’Albert\pwindex{DAlbert, Eugen 10.04.1864 – 03.03.1932@\textsc{d’Albert, Eugen} (10.04.1864 – 03.03.1932), \emph{Komponist/Komponistin}|pw} zum
                  Thee}{\lemma{\textnormal{\emph{bei d’Albert zum
                  Thee}}}\Cendnote{\textnormal{Das erlaubt die Datierung der
                  Karte, vgl. A. S.: \emph{Tagebuch}, 30. 11. 1911.}}}\label{K_L02565-1} ist: morgen können wir nicht kommen, haben schon bei Schmidl\pwindex{Schmidl, Hugo 06.11.1869 – 22.10.1923@\textsc{Schmidl, Hugo} (06.11.1869 – 22.10.1923), \emph{Industrieller/Industrielle}|pw}\pwindex{Schmidl, Paula 13.10.1874 – 24.09.1966@\textsc{Schmidl, Paula} (13.10.1874 – 24.09.1966)|pw}’s abgesagt, – ich bin zu Bett, {\pb}hoffentlich gehts an einem der nächsten Tage, –
               wir werden mit Vergnügen kommen.\pend
           
\pstart
           Viele Grüsse Ihnen Allen!{\\[\baselineskip]}Ihre \spacefill\mbox{OlgaS.}\pend
           \leftskip=0em{}\selectlanguage{ngerman}\endnumbering\briefempfaengerindex{Beer-Hofmann, Richard@\textsc{Beer-Hofmann, Richard}!zzzSchnitzler, Olga@\emph{von Olga Schnitzler}!1911-11-301@{{[}30. 11. 1911?{]}}|)be}\mylabel{L02565h}  \normalsize

\doendnotes{C}
\bigskip
\vfill

\clearpage

\footnotesize

\lohead{\textsc{register}}

% Definiere theindex-Environment komplett neu ohne reledmac
\makeatletter
\renewenvironment{theindex}{%
  \section*{\indexname}%
  \setlength{\parindent}{0pt}%
  \setlength{\parskip}{0pt plus 0.3pt}%
  \let\item\@idxitem
}{%
  \clearpage
}
\makeatother

\IfFileExists{\jobname-pw.ind}{\input{\jobname-pw.ind}}{}

\end{document}

      