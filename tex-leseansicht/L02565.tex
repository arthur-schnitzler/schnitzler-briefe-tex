%% latex-leseansicht-vorspann.tex
%% Vorspann für die Leseansicht.
%% Lädt die gemeinsame Datei latex-vorspann.tex mit nicht gesetztem Schalter.

\newif\ifkorrekturansicht
\korrekturansichtfalse

\input{../tex-inputs/latex-vorspann}


         
         \renewcommand{\erwaehntePersonen}{Personen: Richard Beer-Hofmann, Hugo Schmidl, Paula Schmidl, Eugen d’Albert}
         \renewcommand{\erwaehnteOrte}{Orte: Wien}
         \renewcommand{\erwaehnteWerke}{
               \section[Olga Schnitzler an Richard Beer-Hofmann, {[}30. 11. 1911?{]}]{ Olga Schnitzler an Richard Beer-Hofmann, {[}30. 11. 1911?{]}}\nopagebreak\mylabel{v}\rehead{ }\begin{ledgroupsized}[t]{13cm}\normalsize\beginnumbering \toendnotes[C]{\smallbreak\pagebreak[2]} \Standort{YCGL, MSS 31.}
\physDesc{Briefkarte
\newline{}Handschrift: schwarze Tinte, lateinische Kurrent
\newline{}Beer-Hofmann: mit blauem Buntstift datiert:
                                 »1911« }\toendnotes[C]{\smallbreak}\pstart
           \noindent{}{\pb}\textcolor{gray}{\textbf{O. S.}}\pend
           \pstart
           Lieber Herr D\textsuperscript{r}, ich antworte in Arthurs\pwindex{Schnitzler, Arthur 15.05.1862 – 21.10.1931@\textsc{Schnitzler, Arthur} (15.05.1862 – 21.10.1931), \emph{Schriftsteller, Mediziner}|pw} Namen, der \label{K_L02565-1v}\edtext{bei d’Albert\pwindex{DAlbert, Eugen 10.04.1864 – 03.03.1932@\textsc{d’Albert, Eugen} (10.04.1864 – 03.03.1932), \emph{Komponist}|pw} zum
                  Thee}{\lemma{\textnormal{\emph{bei d’Albert zum
                  Thee}}}\Cendnote{\textnormal{Das erlaubt die Datierung der Karte. Vgl. A. S.: \emph{Tagebuch}, 30. 11. 1911}}}\label{K_L02565-1h} ist: morgen
               können wir nicht kommen, haben schon bei Schmidl\pwindex{Schmidl, Hugo 06.11.1869 – 22.10.1923@\textsc{Schmidl, Hugo} (06.11.1869 – 22.10.1923), \emph{Industrieller}|pw}\pwindex{Schmidl, Paula 13.10.1874 – 24.09.1966@\textsc{Schmidl, Paula} (13.10.1874 – 24.09.1966)|pw}’s abgesagt, – ich bin zu Bett, {\pb}hoffentlich gehts an einem der nächsten Tage, – wir werden mit Vergnügen
               kommen.\pend
           \pstart
           Viele Grüsse Ihnen Allen!{\\[\baselineskip]}Ihre \spacefill\mbox{OlgaS.}\pend
           \leftskip=0em{}
         
         \endnumbering\mylabel{h}\end{ledgroupsized}  \newcommand{\dateiname}{L02565}\newcommand{\titel}{Olga Schnitzler an Richard Beer-Hofmann, [30. 11. 1911?]}\newcommand{\editorInnen}{Martin Anton Müller und Gerd-Hermann Susen}%% latex-leseansicht-abspann.tex
%% Abspann für die Leseansicht.
%% Der Schalter \ifkorrekturansicht ist bereits durch den Vorspann gesetzt.

%% latex-abspann.tex
%% Gemeinsamer Abspann für Korrekturansicht und Leseansicht.
%% Setzt den Schalter \ifkorrekturansicht voraus (gesetzt in den
%% einbindenden Dateien latex-korrekturansicht-abspann.tex bzw.
%% latex-leseansicht-abspann.tex).
%% ---------------------------------------------------------------

\normalsize

% Das esempio-Environment wird nur in der Leseansicht benötigt
\ifkorrekturansicht\else
\newenvironment{esempio}[3]%
{
    \vspace{1.5ex}
    \rlap{\underline{#1}}
    \par
    \setlength{\parindent}{0cm}
    \nopagebreak
    \leftskip=#2cm
    \rightskip=#3cm
}
{
    \par
}
\fi

\doendnotes{C}
\bigskip
\vfill

\clearpage

\footnotesize

\ifkorrekturansicht
  \lohead{\textsc{register}}
\fi

% theindex-Environment neu definieren ohne reledmac
\makeatletter
\renewenvironment{theindex}{%
  \ifkorrekturansicht
    \section*{\indexname}%
  \else
    \subsubsection*{Index der erwähnten Entitäten}%
  \fi
  \setlength{\parindent}{0pt}%
  \setlength{\parskip}{0pt plus 0.3pt}%
  \let\item\@idxitem
}{%
  \ifkorrekturansicht\clearpage\fi
}
\makeatother

\IfFileExists{\jobname-pw.ind}{\input{\jobname-pw.ind}}{}

% Quellenangabe nur in der Leseansicht
\ifkorrekturansicht\else
% Fallback-Definitionen, falls die .tex-Datei \titel etc. nicht gesetzt hat
\providecommand{\titel}{}
\providecommand{\editorInnen}{}
\providecommand{\dateiname}{\jobname}

\vspace{3cm}

\vfill

\footnotesize
\textsc{Quelle}: \titel. Herausgegeben von {\editorInnen}. In: \emph{Arthur Schnitzler: Briefwechsel mit Autorinnen und Autoren}.
 Digitale Edition, https://schnitzler-briefe.acdh.oeaw.ac.at/{\dateiname}.html (Stand \today)
\fi

\end{document}


      