%% latex-leseansicht-vorspann.tex
%% Vorspann für die Leseansicht.
%% Lädt die gemeinsame Datei latex-vorspann.tex mit nicht gesetztem Schalter.

\newif\ifkorrekturansicht
\korrekturansichtfalse

\input{../tex-inputs/latex-vorspann}


         
         \renewcommand{\erwaehntePersonen}{Personen: Lou Andreas-Salomé, Richard Beer-Hofmann}
         \renewcommand{\erwaehnteOrte}{Orte: Bad Ischl, Hotel und Pension Rudolfshöhe (Leopold Petter), München, Salzburg, Wien, Österreichischer Hof}
         \renewcommand{\erwaehnteWerke}{}
               \section[Arthur Schnitzler an Lou Andreas-Salomé, 11. 8. 1895]{ Arthur Schnitzler an Lou Andreas-Salomé, 11. 8. 1895}\nopagebreak\mylabel{v}\rehead{ }\begin{ledgroupsized}[t]{13cm}\normalsize\beginnumbering\briefempfaengerindex{Andreas-Salome, Lou@\textsc{Andreas-Salomé, Lou}!zzzSchnitzler, Arthur@\emph{von Arthur Schnitzler}!1895-08-111@{11. 8. 1895}|(be} \toendnotes[C]{\smallbreak\pagebreak[2]} \Standort{Göttingen, Lou Andreas-Salomé Archiv, Schnitzler.}
\physDesc{Brief, 1 Blatt, 3 Seiten, 858 Zeichen
\newline{}Handschrift: schwarze Tinte, deutsche Kurrent}\pstart{}{\pb}Verehrte Frau Lou,\pend\pstart
           es trifft ſich alles aufs beſte. Heute früh ko{\geminationm}’ ich in
                  Wien\oindex{Wien@\textbf{Wien}|pw} an, und \strikeout{treffe} finde Ihre lieben Zeilen, für die ich herzlich danke.\pend
           \pstart
           Ich fahre in 2 oder 3 Tagen nach Iſchl\oindex{Bad Ischl@\textbf{Bad Ischl}|pw} und ko{\geminationm}e etwa 20. oder 21. nach Salzburg\oindex{Salzburg@\textbf{Salzburg}|pw}. Dort einige Tage zugleich mit Ihnen und
               in Ihrer Geſellſchaft zu verbringen, freut {\pb}mich ganz beſonders. Von S.\oindex{Salzburg@\textbf{Salzburg}|pw} aus fahre ich, wahrſcheinlich per Rad u auf
               einem Umweg nach München\oindex{Muenchen@\textbf{München}|pw}. Es geht aus Ihrer
               Karte nicht deutlich hervor, ob Sie München\oindex{Muenchen@\textbf{München}|pw} vor
               oder nach Salzburg\oindex{Salzburg@\textbf{Salzburg}|pw} zu beſuchen denken. Sollte das
               letztere der Fall ſein, ſo wärs aber ganz beſonders ſchön.\pend
           \pstart
           In Iſchl\oindex{Bad Ischl@\textbf{Bad Ischl}|pw} wohne ich \textsc{Rudolfshöhe}\oindex{Hotel und Pension Rudolfshoehe (Leopold Petter)@\textbf{Hotel und Pension Rudolfshöhe (Leopold Petter)}|pw}, {\pb}wo ich Nachricht von Ihnen vorzufinden hoffe. In Salzb.\oindex{Salzburg@\textbf{Salzburg}|pw} werde ich wahrſcheinlich im oesterr. Hof\oindex{Oesterreichischer Hof@\textbf{Österreichischer Hof}|pw} abſteigen. Richard\pwindex{Beer-Hofmann, Richard 1866-07-11 – 1945-09-26@\textsc{Beer-Hofmann, Richard} (1866-07-11 – 1945-09-26), \emph{Schriftsteller}|pw} iſt wohl von den genauen Salzb.\oindex{Salzburg@\textbf{Salzburg}|pw}
               Daten gleichfalls in Ke{\geminationn}tnis geſetzt? – \pend
           \pstart
           Viele Grüße und auf angenehmes Wiederſehen!{\\[\baselineskip]}Ihr Sie
                  hochſch\textcolor{gray}{ätz}ender{\\[\baselineskip]}\spacefill\mbox{ArthSch}\pend
           \leftskip=0em{}\pstart
           11. 8. 95.{\\}Wien\oindex{Wien@\textbf{Wien}|pw}\pend
           
         
         \endnumbering\mylabel{h}\end{ledgroupsized}  \newcommand{\dateiname}{L00472}\newcommand{\titel}{Arthur Schnitzler an Lou Andreas-Salomé, 11. 8. 1895}\newcommand{\editorInnen}{Martin Anton Müller und Gerd-Hermann Susen}%% latex-leseansicht-abspann.tex
%% Abspann für die Leseansicht.
%% Der Schalter \ifkorrekturansicht ist bereits durch den Vorspann gesetzt.

%% latex-abspann.tex
%% Gemeinsamer Abspann für Korrekturansicht und Leseansicht.
%% Setzt den Schalter \ifkorrekturansicht voraus (gesetzt in den
%% einbindenden Dateien latex-korrekturansicht-abspann.tex bzw.
%% latex-leseansicht-abspann.tex).
%% ---------------------------------------------------------------

\normalsize

% Das esempio-Environment wird nur in der Leseansicht benötigt
\ifkorrekturansicht\else
\newenvironment{esempio}[3]%
{
    \vspace{1.5ex}
    \rlap{\underline{#1}}
    \par
    \setlength{\parindent}{0cm}
    \nopagebreak
    \leftskip=#2cm
    \rightskip=#3cm
}
{
    \par
}
\fi

\doendnotes{C}
\bigskip
\vfill

\clearpage

\footnotesize

\ifkorrekturansicht
  \lohead{\textsc{register}}
\fi

% theindex-Environment neu definieren ohne reledmac
\makeatletter
\renewenvironment{theindex}{%
  \ifkorrekturansicht
    \section*{\indexname}%
  \else
    \subsubsection*{Index der erwähnten Entitäten}%
  \fi
  \setlength{\parindent}{0pt}%
  \setlength{\parskip}{0pt plus 0.3pt}%
  \let\item\@idxitem
}{%
  \ifkorrekturansicht\clearpage\fi
}
\makeatother

\IfFileExists{\jobname-pw.ind}{\input{\jobname-pw.ind}}{}

% Quellenangabe nur in der Leseansicht
\ifkorrekturansicht\else
% Fallback-Definitionen, falls die .tex-Datei \titel etc. nicht gesetzt hat
\providecommand{\titel}{}
\providecommand{\editorInnen}{}
\providecommand{\dateiname}{\jobname}

\vspace{3cm}

\vfill

\footnotesize
\textsc{Quelle}: \titel. Herausgegeben von {\editorInnen}. In: \emph{Arthur Schnitzler: Briefwechsel mit Autorinnen und Autoren}.
 Digitale Edition, https://schnitzler-briefe.acdh.oeaw.ac.at/{\dateiname}.html (Stand \today)
\fi

\end{document}


      