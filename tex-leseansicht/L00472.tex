%% latex-korrekturansicht-vorspann.tex
%% Vorspann für die Korrekturansicht.
%% Lädt die gemeinsame Datei latex-vorspann.tex mit gesetztem Schalter.

\newif\ifkorrekturansicht
\korrekturansichttrue

\input{../tex-inputs/latex-vorspann}


\section[Arthur Schnitzler an Lou Andreas-Salomé, 11. 8. 1895]{L00472 Arthur Schnitzler an Lou Andreas-Salomé, 11. 8. 1895}
\nopagebreak\mylabel{L00472v}
\rehead{ }\normalsize\beginnumbering\briefempfaengerindex{Andreas-Salome, Lou@\textsc{Andreas-Salomé, Lou}!zzzSchnitzler, Arthur@\emph{von Arthur Schnitzler}!1895-08-111@{11. 8. 1895}|(be}
\toendnotes[C]{\smallbreak\pagebreak[2]}\Standort{Göttingen, Lou Andreas-Salomé Archiv, Schnitzler.}
\physDesc{Brief, 1 Blatt, 3 Seiten, 858 Zeichen
\newline{}Handschrift: schwarze Tinte, deutsche Kurrent}
\pstart{}{\pb}Verehrte Frau Lou,\pend\vspace{0.5em}
\pstart
           es trifft ſich alles aufs beſte. Heute früh ko{\geminationm}’ ich in
                  Wien\oindex{Wien@\textbf{Wien}, \emph{A.ADM2}|pw} an, und \strikeout{treffe} finde Ihre lieben Zeilen, für die ich herzlich danke.\pend
           
\pstart
           Ich fahre in 2 oder 3 Tagen nach Iſchl\oindex{Bad Ischl@\textbf{Bad Ischl}, \emph{P.PPL}|pw} und ko{\geminationm}e etwa 20. oder 21. nach Salzburg\oindex{Salzburg@\textbf{Salzburg}, \emph{A.ADM2}|pw}. Dort einige Tage zugleich mit Ihnen und
               in Ihrer Geſellſchaft zu verbringen, freut {\pb}mich ganz beſonders. Von S.\oindex{Salzburg@\textbf{Salzburg}, \emph{A.ADM2}|pw} aus fahre ich, wahrſcheinlich per Rad u auf
               einem Umweg nach München\oindex{Muenchen@\textbf{München}, \emph{P.PPLA}|pw}. Es geht aus Ihrer
               Karte nicht deutlich hervor, ob Sie München\oindex{Muenchen@\textbf{München}, \emph{P.PPLA}|pw} vor
               oder nach Salzburg\oindex{Salzburg@\textbf{Salzburg}, \emph{A.ADM2}|pw} zu beſuchen denken. Sollte das
               letztere der Fall ſein, ſo wärs aber ganz beſonders ſchön.\pend
           
\pstart
           In Iſchl\oindex{Bad Ischl@\textbf{Bad Ischl}, \emph{P.PPL}|pw} wohne ich \textsc{Rudolfshöhe}\oindex{Hotel und Pension Rudolfshoehe (Leopold Petter)@\textbf{Hotel und Pension Rudolfshöhe (Leopold Petter)}, \emph{Hotel (K.HTL)}|pw}, {\pb}wo ich Nachricht von Ihnen vorzufinden hoffe. In Salzb.\oindex{Salzburg@\textbf{Salzburg}, \emph{A.ADM2}|pw} werde ich wahrſcheinlich im oesterr. Hof\oindex{Oesterreichischer Hof@\textbf{Österreichischer Hof}, \emph{Hotel (K.HTL)}|pw} abſteigen. Richard\pwindex{Beer-Hofmann, Richard 1866-07-11 – 1945-09-26@\textsc{Beer-Hofmann, Richard} (1866-07-11 – 1945-09-26), \emph{Schriftsteller/Schriftstellerin}|pw} iſt wohl von den genauen Salzb.\oindex{Salzburg@\textbf{Salzburg}, \emph{A.ADM2}|pw}
               Daten gleichfalls in Ke{\geminationn}tnis geſetzt? – \pend
           
\pstart
           Viele Grüße und auf angenehmes Wiederſehen!{\\[\baselineskip]}Ihr Sie
                  hochſch\textcolor{gray}{ätz}ender{\\[\baselineskip]}\spacefill\mbox{ArthSch}\pend
           \leftskip=0em{}
\pstart
           11. 8. 95.{\\}Wien\oindex{Wien@\textbf{Wien}, \emph{A.ADM2}|pw}\pend
           \selectlanguage{ngerman}\endnumbering\briefempfaengerindex{Andreas-Salome, Lou@\textsc{Andreas-Salomé, Lou}!zzzSchnitzler, Arthur@\emph{von Arthur Schnitzler}!1895-08-111@{11. 8. 1895}|)be}\mylabel{L00472h}  \normalsize

\doendnotes{C}
\bigskip
\vfill

\clearpage

\footnotesize

\lohead{\textsc{register}}

% Definiere theindex-Environment komplett neu ohne reledmac
\makeatletter
\renewenvironment{theindex}{%
  \section*{\indexname}%
  \setlength{\parindent}{0pt}%
  \setlength{\parskip}{0pt plus 0.3pt}%
  \let\item\@idxitem
}{%
  \clearpage
}
\makeatother

\IfFileExists{\jobname-pw.ind}{\input{\jobname-pw.ind}}{}

\end{document}

      