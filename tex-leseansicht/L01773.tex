%% latex-korrekturansicht-vorspann.tex
%% Vorspann für die Korrekturansicht.
%% Lädt die gemeinsame Datei latex-vorspann.tex mit gesetztem Schalter.

\newif\ifkorrekturansicht
\korrekturansichttrue

\input{../tex-inputs/latex-vorspann}


\section[Hermann Bahr an Arthur Schnitzler, 3. {[}6.{]} 1908]{L01773 Hermann Bahr an Arthur Schnitzler, 3. {[}6.{]} 1908}
\nopagebreak\mylabel{L01773v}
\rehead{ }\normalsize\beginnumbering\briefempfaengerindex{Schnitzler, Arthur@\textsc{Schnitzler, Arthur}!zzzBahr, Hermann@\emph{von Hermann Bahr}!1908-06-031@{3. {[}6.{]} 1908}|(be}
\toendnotes[C]{\smallbreak\pagebreak[2]}\Standort{CUL, Schnitzler, B 5b.}
\physDesc{Brief, 1 Blatt, 1 Seite, 252 Zeichen
\newline{}Handschrift: blaue Tinte, deutsche Kurrent
\newline{}Schnitzler: mit Bleistift ergänzt »Bahr« 
\newline{}Ordnung: mit Bleistift von unbekannter Hand nummeriert:
                                    »154« }
\buchAbdrucke{\weitereDrucke{Hermann Bahr, Arthur Schnitzler: \emph{Briefwechsel, Aufzeichnungen, Dokumente (1891–1931)}. Göttingen: \emph{Wallstein} 2018, S. 403.} }\toendnotes[C]{\smallbreak}
\pstart
           \raggedleft{}{\pb}3. \substVorne{}\textsuperscript{\textcolor{gray}{5}}\substDazwischen{}\textcolor{gray}{6}\substHinten{}. 08\pend
           
\pstart\center{}Lieber Artur!\pend\vspace{0.5em}
\pstart
           Nur geſchwind herzlichſten Dank für \label{K_L01773-1v}\edtext{Deinen Roman\pwindex{Weg ins Freie. Roman@\emph{Der Weg ins Freie. Roman}|pwv}}{\lemma{\textnormal{\emph{Deinen Roman}}}\Cendnote{\textnormal{Schnitzler hatte den \emph{Weg ins Freie}\pwindex{Weg ins Freie. Roman@\emph{Der Weg ins Freie. Roman}|pwk} am 2. 6. 1908 versandt.}}}\label{K_L01773-1}. Darüber müſſen wir einmal
               lange reden. Bis ich erſt \label{K_L01773-2v}\edtext{mit meinem\pwindex{Rahl. Roman@\emph{Die Rahl. Roman}|pwv} fertig}{\lemma{\textnormal{\emph{mit meinem fertig}}}\Cendnote{\textnormal{Bahr\pwindex{Bahr, Hermann 19.07.1863 – 15.01.1934@\textsc{Bahr, Hermann} (19.07.1863 – 15.01.1934), \emph{Schriftsteller/Schriftstellerin, Kritiker/Kritikerin}|pwk} diktierte seinen Roman \emph{Die Rahl}\pwindex{Rahl. Roman@\emph{Die Rahl. Roman}|pwk} vom 20. 4. bis zum 14. 6. 1908 (\emph{Theatermuseum Wien}, VM 1227 Ba).}}}\label{K_L01773-2} bin, in
               dem ich jetzt über die Ohren ſtecke.\pend
           
\pstart
           Eiligſt{\\[\baselineskip]}herzlichſt{\\[\baselineskip]}mit den allerbeſten Grüßen an Deine liebe Frau\pwindex{Schnitzler, Olga 17.01.1882 – 13.01.1970@\textsc{Schnitzler, Olga} (17.01.1882 – 13.01.1970), \emph{Schauspieler/Schauspielerin, Sänger/Sängerin}|pwv}{\\[\baselineskip]}Dein{\\[\baselineskip]}\spacefill\mbox{Hermann}\pend
           \leftskip=0em{}\selectlanguage{ngerman}\endnumbering\briefempfaengerindex{Schnitzler, Arthur@\textsc{Schnitzler, Arthur}!zzzBahr, Hermann@\emph{von Hermann Bahr}!1908-06-031@{3. {[}6.{]} 1908}|)be}\mylabel{L01773h}  \normalsize

\doendnotes{C}
\bigskip
\vfill

\clearpage

\footnotesize

\lohead{\textsc{register}}

% Definiere theindex-Environment komplett neu ohne reledmac
\makeatletter
\renewenvironment{theindex}{%
  \section*{\indexname}%
  \setlength{\parindent}{0pt}%
  \setlength{\parskip}{0pt plus 0.3pt}%
  \let\item\@idxitem
}{%
  \clearpage
}
\makeatother

\IfFileExists{\jobname-pw.ind}{\input{\jobname-pw.ind}}{}

\end{document}

      