%% latex-leseansicht-vorspann.tex
%% Vorspann für die Leseansicht.
%% Lädt die gemeinsame Datei latex-vorspann.tex mit nicht gesetztem Schalter.

\newif\ifkorrekturansicht
\korrekturansichtfalse

\input{../tex-inputs/latex-vorspann}


\section[ Paul Goldmann an Arthur Schnitzler, 26. 2. [1899]]{L02867 Paul Goldmann an Arthur Schnitzler,  26. 2. [1899]}
\nopagebreak\mylabel{L02867v}
\rehead{ }\normalsize\beginnumbering\briefempfaengerindex{Schnitzler, Arthur@\textsc{Schnitzler, Arthur}!zzzGoldmann, Paul@\emph{von Paul Goldmann}!1899-02-261@{26. 2. [1899]}|(be}
\toendnotes[C]{\smallbreak\pagebreak[2]}
\correspDesc{Versand  durch Paul Goldmann am 26. 2. [1899] in Paris
\newline{}Erhalt  durch Arthur Schnitzler im Zeitraum [27. 2. 1899
                  – 3. 3. 1899?] in Wien}\toendnotes[C]{\smallbreak}
\Standort{DLA, A:Schnitzler, HS.NZ85.1.3169.}
\physDesc{Brief, 1 Blatt, 1 Seite, 316 Zeichen
\newline{}Handschrift: schwarze Tinte, deutsche Kurrent
\newline{}Schnitzler: mit Bleistift das Jahr »99« vermerkt }\toendnotes[C]{\smallbreak}
\pstart
           \raggedleft{}{\pb}\textsc{Paris\oindex{Paris@\textbf{Paris}, \emph{Hauptstadt}|pw}}, 26. Februar.\pend
           
\pstart\center{}Mein lieber Freund,\pend\vspace{0.5em}
\pstart
           Ich war acht Tage in \textsc{Paris\oindex{Paris@\textbf{Paris}, \emph{Hauptstadt}|pw}} zur \label{K_L02867-1v}\edtext{Berichterſtattung{ }\strikeout{über} über den Congreß\orgindex{Congrès du Parlement français@Congrès du Parlement français|pw} u. das Begräbniß \textsc{Faures\pwindex{Faure, Félix 30.\,1.\,1841 Paris – 16.\,2.\,1899 ebd.@\textsc{Faure, Félix} (30.\,1.\,1841 Paris – 16.\,2.\,1899 ebd.), \emph{Politiker, Präsident}|pw}}}{\lemma{\textnormal{\emph{Berichterstattung … Faures}}}\Cendnote{\textnormal{Félix Faure\pwindex{Faure, Félix 30.\,1.\,1841 Paris – 16.\,2.\,1899 ebd.@\textsc{Faure, Félix} (30.\,1.\,1841 Paris – 16.\,2.\,1899 ebd.), \emph{Politiker, Präsident}|pwk}, der bis zu seinem Tod durch
                  einen Schlaganfall am 16. 2. 1899{ }Frankreichs\oindex{Frankreich@\textbf{Frankreich}|pwk} Präsident gewesen war, wurde am 23. 2. 1899 beerdigt. Am 18. 2. 1899 war der \emph{Congrès}\orgindex{Congrès du Parlement français@Congrès du Parlement français|pwk}
                  zusammengetreten, um Émile Loubet\pwindex{Loubet, Émile 30.\,12.\,1838 Marsanne – 20.\,12.\,1929 Montélimar@\textsc{Loubet, Émile} (30.\,12.\,1838 Marsanne – 20.\,12.\,1929 Montélimar), \emph{Politiker}|pwk} zu
                  seinem Nachfolger zu wählen.}}}\label{K_L02867-1}. Nach Wien\oindex{Wien@\textbf{Wien}, \emph{Verwaltungsgebiet}|pw}
               komme ich \uuline{nicht}. Wie{ }ſich das Alles ergeben, theile
               ich Dir von Frankfurt\oindex{Frankfurt am Main@\textbf{Frankfurt am Main}, \emph{Hauptstadt}|pw} aus ausführlich mit,{ }ſobald
               ich einen freien Augenblick finde. Einſtweilen viele treue Grüße! Dein {\\}\spacefill\mbox{Paul Goldmann}\pend
           \selectlanguage{ngerman}\endnumbering\briefempfaengerindex{Schnitzler, Arthur@\textsc{Schnitzler, Arthur}!zzzGoldmann, Paul@\emph{von Paul Goldmann}!1899-02-261@{26. 2. [1899]}|)be}\mylabel{L02867h}  \newcommand{\dateiname}{L02867}\newcommand{\titel}{Paul Goldmann an Arthur Schnitzler, 26. 2. [1899]}\newcommand{\editorInnen}{Martin Anton Müller und Laura Untner}%% latex-leseansicht-abspann.tex
%% Abspann für die Leseansicht.
%% Der Schalter \ifkorrekturansicht ist bereits durch den Vorspann gesetzt.

%% latex-abspann.tex
%% Gemeinsamer Abspann für Korrekturansicht und Leseansicht.
%% Setzt den Schalter \ifkorrekturansicht voraus (gesetzt in den
%% einbindenden Dateien latex-korrekturansicht-abspann.tex bzw.
%% latex-leseansicht-abspann.tex).
%% ---------------------------------------------------------------

\normalsize

% Das esempio-Environment wird nur in der Leseansicht benötigt
\ifkorrekturansicht\else
\newenvironment{esempio}[3]%
{
    \vspace{1.5ex}
    \rlap{\underline{#1}}
    \par
    \setlength{\parindent}{0cm}
    \nopagebreak
    \leftskip=#2cm
    \rightskip=#3cm
}
{
    \par
}
\fi

\doendnotes{C}
\bigskip
\vfill

\clearpage

\footnotesize

\ifkorrekturansicht
  \lohead{\textsc{register}}
\fi

% theindex-Environment neu definieren ohne reledmac
\makeatletter
\renewenvironment{theindex}{%
  \ifkorrekturansicht
    \section*{\indexname}%
  \else
    \subsubsection*{Index der erwähnten Entitäten}%
  \fi
  \setlength{\parindent}{0pt}%
  \setlength{\parskip}{0pt plus 0.3pt}%
  \let\item\@idxitem
}{%
  \ifkorrekturansicht\clearpage\fi
}
\makeatother

\IfFileExists{\jobname-pw.ind}{\input{\jobname-pw.ind}}{}

% Quellenangabe nur in der Leseansicht
\ifkorrekturansicht\else
% Fallback-Definitionen, falls die .tex-Datei \titel etc. nicht gesetzt hat
\providecommand{\titel}{}
\providecommand{\editorInnen}{}
\providecommand{\dateiname}{\jobname}

\vspace{3cm}

\vfill

\footnotesize
\textsc{Quelle}: \titel. Herausgegeben von {\editorInnen}. In: \emph{Arthur Schnitzler: Briefwechsel mit Autorinnen und Autoren}.
 Digitale Edition, https://schnitzler-briefe.acdh.oeaw.ac.at/{\dateiname}.html (Stand \today)
\fi

\end{document}


