%% latex-leseansicht-vorspann.tex
%% Vorspann für die Leseansicht.
%% Lädt die gemeinsame Datei latex-vorspann.tex mit nicht gesetztem Schalter.

\newif\ifkorrekturansicht
\korrekturansichtfalse

\input{../tex-inputs/latex-vorspann}

\begin{center}
            \textcolor{red}{ENTWURF, NICHT FERTIG KORRIGIERT}
                      \end{center}
            
         
         \renewcommand{\erwaehntePersonen}{Personen: Félix Faure, Émile Loubet}
         \renewcommand{\erwaehnteInstitutionen}{Institutionen: Congrès du Parlement français}
         \renewcommand{\erwaehnteOrte}{Orte: Frankfurt am Main, Frankreich, Paris, Wien}
         \renewcommand{\erwaehnteWerke}{
               \section[ Paul Goldmann an Arthur Schnitzler, 26. 2. {[}1899{]}]{ Paul Goldmann an Arthur Schnitzler, 26. 2. {[}1899{]}}\nopagebreak\mylabel{v}\rehead{ }\begin{ledgroupsized}[t]{13cm}\normalsize\beginnumbering \toendnotes[C]{\smallbreak\pagebreak[2]} \Standort{DLA, A:Schnitzler, HS.NZ85.1.3169.}
\physDesc{Brief, 1 Blatt, 1 Seite
\newline{}Handschrift: schwarze Tinte, deutsche Kurrent
\newline{}Schnitzler: mit Bleistift das Jahr »99« vermerkt }\toendnotes[C]{\smallbreak}\pstart
           \raggedleft{}{\pb}\textsc{Paris\oindex{Paris@\textbf{Paris}|pw}}, 26. Februar.\pend
           \pstart\center{}Mein lieber Freund,\pend\pstart
           Ich war acht Tage in \textsc{Paris\oindex{Paris@\textbf{Paris}|pw}} zur \label{K_L02867-1v}\edtext{Berichterſtattung{ }\strikeout{über} über den Congreß u. das Begräbniß \textsc{Faure\pwindex{Faure, Felix 1841-01-30 – 1899-02-16@\textsc{Faure, Félix} (1841-01-30 – 1899-02-16), \emph{Politiker, Präsident}|pw}s}}{\lemma{\textnormal{\emph{Berichterſtattung … Faures}}}\Cendnote{\textnormal{Félix Faure\pwindex{Faure, Felix 1841-01-30 – 1899-02-16@\textsc{Faure, Félix} (1841-01-30 – 1899-02-16), \emph{Politiker, Präsident}|pwk}, der bis zu seinem Tod durch
                  einen Schlaganfall am 16. 2. 1899{ }Frankreich\oindex{Frankreich@\textbf{Frankreich}|pwk}s Präsident war, wurde am 23. 2. 1899 beerdigt. Am 18. 2. 1899 war
                  der \emph{Congrès}\orgindex{Congres du Parlement français@Congrès du Parlement français|pwk} zusammengetreten, um Émile Loubet\pwindex{Loubet, Emile 1838-12-30 – 1929-12-20@\textsc{Loubet, Émile} (1838-12-30 – 1929-12-20), \emph{Politiker}|pwk} zu seinem Nachfolger zu
                  wählen.}}}\label{K_L02867-1h}. Nach Wien\oindex{Wien@\textbf{Wien}|pw} komme ich \uuline{nicht}. Wie ſich das Alles ergeben, theile ich Dir von
                  Frankfurt\oindex{Frankfurt am Main@\textbf{Frankfurt am Main}|pw} aus ausführlich mit, ſobald ich
               einen freien Augenblick finde. Einſtweilen viele treue Grüße! \pend
           \pstart
           Dein{\\[\baselineskip]}\spacefill\mbox{Paul Goldmann}\pend
           \leftskip=0em{}
         
         \endnumbering\mylabel{h}\end{ledgroupsized}  \newcommand{\dateiname}{L02867}\newcommand{\titel}{Paul Goldmann an Arthur Schnitzler, 26. 2. [1899]}\newcommand{\editorInnen}{Martin Anton Müller und Laura Untner}%% latex-leseansicht-abspann.tex
%% Abspann für die Leseansicht.
%% Der Schalter \ifkorrekturansicht ist bereits durch den Vorspann gesetzt.

%% latex-abspann.tex
%% Gemeinsamer Abspann für Korrekturansicht und Leseansicht.
%% Setzt den Schalter \ifkorrekturansicht voraus (gesetzt in den
%% einbindenden Dateien latex-korrekturansicht-abspann.tex bzw.
%% latex-leseansicht-abspann.tex).
%% ---------------------------------------------------------------

\normalsize

% Das esempio-Environment wird nur in der Leseansicht benötigt
\ifkorrekturansicht\else
\newenvironment{esempio}[3]%
{
    \vspace{1.5ex}
    \rlap{\underline{#1}}
    \par
    \setlength{\parindent}{0cm}
    \nopagebreak
    \leftskip=#2cm
    \rightskip=#3cm
}
{
    \par
}
\fi

\doendnotes{C}
\bigskip
\vfill

\clearpage

\footnotesize

\ifkorrekturansicht
  \lohead{\textsc{register}}
\fi

% theindex-Environment neu definieren ohne reledmac
\makeatletter
\renewenvironment{theindex}{%
  \ifkorrekturansicht
    \section*{\indexname}%
  \else
    \subsubsection*{Index der erwähnten Entitäten}%
  \fi
  \setlength{\parindent}{0pt}%
  \setlength{\parskip}{0pt plus 0.3pt}%
  \let\item\@idxitem
}{%
  \ifkorrekturansicht\clearpage\fi
}
\makeatother

\IfFileExists{\jobname-pw.ind}{\input{\jobname-pw.ind}}{}

% Quellenangabe nur in der Leseansicht
\ifkorrekturansicht\else
% Fallback-Definitionen, falls die .tex-Datei \titel etc. nicht gesetzt hat
\providecommand{\titel}{}
\providecommand{\editorInnen}{}
\providecommand{\dateiname}{\jobname}

\vspace{3cm}

\vfill

\footnotesize
\textsc{Quelle}: \titel. Herausgegeben von {\editorInnen}. In: \emph{Arthur Schnitzler: Briefwechsel mit Autorinnen und Autoren}.
 Digitale Edition, https://schnitzler-briefe.acdh.oeaw.ac.at/{\dateiname}.html (Stand \today)
\fi

\end{document}


      