%% latex-korrekturansicht-vorspann.tex
%% Vorspann für die Korrekturansicht.
%% Lädt die gemeinsame Datei latex-vorspann.tex mit gesetztem Schalter.

\newif\ifkorrekturansicht
\korrekturansichttrue

\input{../tex-inputs/latex-vorspann}


\section[ Paul Goldmann an Arthur Schnitzler, 26. 2. {[}1899{]}]{L02867 Paul Goldmann an Arthur Schnitzler, 26. 2. {[}1899{]}}
\nopagebreak\mylabel{L02867v}
\rehead{ }\normalsize\beginnumbering\briefempfaengerindex{Schnitzler, Arthur@\textsc{Schnitzler, Arthur}!zzzGoldmann, Paul@\emph{von Paul Goldmann}!1899-02-261@{26. 2. {[}1899{]}}|(be}
\toendnotes[C]{\smallbreak\pagebreak[2]}\Standort{DLA, A:Schnitzler, HS.NZ85.1.3169.}
\physDesc{Brief, 1 Blatt, 1 Seite, 316 Zeichen
\newline{}Handschrift: schwarze Tinte, deutsche Kurrent
\newline{}Schnitzler: mit Bleistift das Jahr »99« vermerkt }\toendnotes[C]{\smallbreak}
\pstart
           \raggedleft{}{\pb}\textsc{Paris\oindex{Paris@\textbf{Paris}, \emph{P.PPLC}|pw}}, 26. Februar.\pend
           
\pstart\center{}Mein lieber Freund,\pend\vspace{0.5em}
\pstart
           Ich war acht Tage in \textsc{Paris\oindex{Paris@\textbf{Paris}, \emph{P.PPLC}|pw}} zur \label{K_L02867-1v}\edtext{Berichterſtattung{ }\strikeout{über} über den Congreß\orgindex{Congres du Parlement français@Congrès du Parlement français|pw} u. das Begräbniß \textsc{Faures\pwindex{Faure, Felix 1841-01-30 – 1899-02-16@\textsc{Faure, Félix} (1841-01-30 – 1899-02-16), \emph{Politiker/Politikerin, Präsident/Präsidentin}|pw}}}{\lemma{\textnormal{\emph{Berichterſtattung … Faures}}}\Cendnote{\textnormal{Félix Faure\pwindex{Faure, Felix 1841-01-30 – 1899-02-16@\textsc{Faure, Félix} (1841-01-30 – 1899-02-16), \emph{Politiker/Politikerin, Präsident/Präsidentin}|pwk}, der bis zu seinem Tod durch
                  einen Schlaganfall am 16. 2. 1899{ }Frankreichs\oindex{Frankreich@\textbf{Frankreich}, \emph{A.PCLI}|pwk} Präsident gewesen war, wurde am 23. 2. 1899 beerdigt. Am 18. 2. 1899 war der \emph{Congrès}\orgindex{Congres du Parlement français@Congrès du Parlement français|pwk}
                  zusammengetreten, um Émile Loubet\pwindex{Loubet, Emile 1838-12-30 – 1929-12-20@\textsc{Loubet, Émile} (1838-12-30 – 1929-12-20), \emph{Politiker/Politikerin}|pwk} zu
                  seinem Nachfolger zu wählen.}}}\label{K_L02867-1}. Nach Wien\oindex{Wien@\textbf{Wien}, \emph{A.ADM2}|pw}
               komme ich \uuline{nicht}. Wie ſich das Alles ergeben, theile
               ich Dir von Frankfurt\oindex{Frankfurt am Main@\textbf{Frankfurt am Main}, \emph{P.PPLA3}|pw} aus ausführlich mit, ſobald
               ich einen freien Augenblick finde. Einſtweilen viele treue Grüße! Dein {\\}\spacefill\mbox{Paul Goldmann}\pend
           \selectlanguage{ngerman}\endnumbering\briefempfaengerindex{Schnitzler, Arthur@\textsc{Schnitzler, Arthur}!zzzGoldmann, Paul@\emph{von Paul Goldmann}!1899-02-261@{26. 2. {[}1899{]}}|)be}\mylabel{L02867h}  \normalsize

\doendnotes{C}
\bigskip
\vfill

\clearpage

\footnotesize

\lohead{\textsc{register}}

% Definiere theindex-Environment komplett neu ohne reledmac
\makeatletter
\renewenvironment{theindex}{%
  \section*{\indexname}%
  \setlength{\parindent}{0pt}%
  \setlength{\parskip}{0pt plus 0.3pt}%
  \let\item\@idxitem
}{%
  \clearpage
}
\makeatother

\IfFileExists{\jobname-pw.ind}{\input{\jobname-pw.ind}}{}

\end{document}

      