%% latex-korrekturansicht-vorspann.tex
%% Vorspann für die Korrekturansicht.
%% Lädt die gemeinsame Datei latex-vorspann.tex mit gesetztem Schalter.

\newif\ifkorrekturansicht
\korrekturansichttrue

\input{../tex-inputs/latex-vorspann}


\section[Arthur Schnitzler an Hugo von Hofmannsthal, 14. 10. 1898]{L00853 Arthur Schnitzler an Hugo von Hofmannsthal, 14. 10. 1898}
\nopagebreak\mylabel{L00853v}
\rehead{ }\normalsize\beginnumbering\briefempfaengerindex{Hofmannsthal, Hugo von@\textsc{Hofmannsthal, Hugo von}!zzzSchnitzler, Arthur@\emph{von Arthur Schnitzler}!1898-10-142@{14. 10. 1898}|(be}
\toendnotes[C]{\smallbreak\pagebreak[2]}\Standort{FDH, Hs-30885,78.}
\physDesc{Brief, 1 Blatt, 3 Seiten, 1103 Zeichen
\newline{}Handschrift: schwarze Tinte, deutsche Kurrent
\newline{}Ordnung: mit Bleistift von Schnitzler mutmaßlich bei der Durchsicht der Korrespondenz
                                    1929 datiert: »14/10 98« }
\buchAbdrucke{\weitereDrucke{Hugo von Hofmannsthal, Arthur Schnitzler: \emph{Briefwechsel}. Frankfurt am Main: \emph{S. Fischer} 1964, S. 114.} }\toendnotes[C]{\smallbreak}
\pstart
           \noindent{}{\pb}mein lieber Hugo, es iſt jetzt ſo grau und kühl und feucht, und ich
               bin ſo verſchnupft und habe eine ganz geſchwollene Naſe, dſs wohl an eine Hinterbrühl\oindex{Hinterbruehl@\textbf{Hinterbrühl}, \emph{P.PPLA3}|pw}erreiſe kaum zu denken iſt, vielmehr
               vermute ich Sie ko{\geminationm}en früher nach Wien\oindex{Wien@\textbf{Wien}, \emph{A.ADM2}|pw}. Viele Grüße hab ich Ihnen von Brahm\pwindex{Brahm, Otto 05.02.1856 – 28.11.1912@\textsc{Brahm, Otto} (05.02.1856 – 28.11.1912), \emph{Theaterleiter/Theaterleiterin, Regisseur/Regisseurin}|pw}, Harden\pwindex{Harden, Maximilian 20.10.1861 – 30.10.1927@\textsc{Harden, Maximilian} (20.10.1861 – 30.10.1927), \emph{Schriftsteller/Schriftstellerin, Publizist/Publizistin}|pw} und der
                  Dumont\pwindex{Dumont, Louise 22.02.1862 – 16.05.1932@\textsc{Dumont, Louise} (22.02.1862 – 16.05.1932), \emph{Theaterleiter/Theaterleiterin, Schauspieler/Schauspielerin}|pw} zu bringen. Die Leute ſpüren doch
               ungefähr, wer Sie ſind. Man freut ſich auf Ihr Wiederko{\geminationm}en, auf Ihr neues Stück\pwindex{Abenteurer und die Saengerin oder Die Geschenke des Lebens@\emph{Der Abenteurer und die Sängerin oder Die Geschenke des Lebens}|pwv}, {\pb}– mir ſcheint, im Jänner{ }ſind einige Abende für Sie frei; (von den künftigen
               Monaten ganz zu geſchweigen.)\pend
           
\pstart
           Über meinen Berl\oindex{Berlin@\textbf{Berlin}, \emph{P.PPLC}|pw}. Aufenthalt mündlich. Der Erfolg
               nach dem 3. Akt\pwindex{Vermaechtnis. Schauspiel in drei Akten@\emph{Das Vermächtnis. Schauspiel in drei Akten}|pwv} war
               überraſchend ſtark. Während des Akts\pwindex{Vermaechtnis. Schauspiel in drei Akten@\emph{Das Vermächtnis. Schauspiel in drei Akten}|pwv} hatte ich die Empfindung, das Stück\pwindex{Vermaechtnis. Schauspiel in drei Akten@\emph{Das Vermächtnis. Schauspiel in drei Akten}|pwv} iſt hin. Da kamen die letzten paar Scenen, die
               wirkten unmittelbar und ſind ja wirklich aller Ehren wert. Aber aus welchen {\pb}Tiefen ſteigen ſie empor! –\pend
           
\pstart
           – Im übrigen wird ſich das Stück\pwindex{Vermaechtnis. Schauspiel in drei Akten@\emph{Das Vermächtnis. Schauspiel in drei Akten}|pwv} nicht lang halten; ſchon die 3. Vorſtellung war ſchwach beſucht.\pend
           
\pstart
           – Von meinen 3 Einaktern\pwindex{gruene Kakadu – Paracelsus – Die Gefaehrtin. Drei Einakter@\emph{Der grüne Kakadu – Paracelsus – Die Gefährtin. Drei Einakter}|pwv} hat
               dem Br.\pwindex{Brahm, Otto 05.02.1856 – 28.11.1912@\textsc{Brahm, Otto} (05.02.1856 – 28.11.1912), \emph{Theaterleiter/Theaterleiterin, Regisseur/Regisseurin}|pw} der gefärbte Vogel\pwindex{gruene Kakadu. Groteske in einem Akt@\emph{Der grüne Kakadu. Groteske in einem Akt}|pwv} (wie es ſcheint weitaus) am beſten gefallen.
                  \introOben{}Aufführung wahrſcheinlich Februar mit Kainz\pwindex{Kainz, Josef 02.01.1858 – 20.09.1910@\textsc{Kainz, Josef} (02.01.1858 – 20.09.1910), \emph{Schauspieler/Schauspielerin}|pw}.\introOben{}\pend
           
\pstart
           Seien Sie herzlich gegrüßt und laſſen Sie uns bald zuſa{\geminationm}en ſein.\pend
           \pstart Ihr \spacefill\mbox{Arthur}\pend{}
\pstart
           Wien\oindex{Wien@\textbf{Wien}, \emph{A.ADM2}|pw}, 14. X. 98.\pend
           \selectlanguage{ngerman}\endnumbering\briefempfaengerindex{Hofmannsthal, Hugo von@\textsc{Hofmannsthal, Hugo von}!zzzSchnitzler, Arthur@\emph{von Arthur Schnitzler}!1898-10-142@{14. 10. 1898}|)be}\mylabel{L00853h}  \normalsize

\doendnotes{C}
\bigskip
\vfill

\clearpage

\footnotesize

\lohead{\textsc{register}}

% Definiere theindex-Environment komplett neu ohne reledmac
\makeatletter
\renewenvironment{theindex}{%
  \section*{\indexname}%
  \setlength{\parindent}{0pt}%
  \setlength{\parskip}{0pt plus 0.3pt}%
  \let\item\@idxitem
}{%
  \clearpage
}
\makeatother

\IfFileExists{\jobname-pw.ind}{\input{\jobname-pw.ind}}{}

\end{document}

      