%% latex-leseansicht-vorspann.tex
%% Vorspann für die Leseansicht.
%% Lädt die gemeinsame Datei latex-vorspann.tex mit nicht gesetztem Schalter.

\newif\ifkorrekturansicht
\korrekturansichtfalse

\input{../tex-inputs/latex-vorspann}


         
         \newcommand{\erwaehntePersonen}{Personen: }
         \newcommand{\erwaehnteInstitutionen}{}
         \newcommand{\erwaehnteOrte}{}
         \newcommand{\erwaehnteWerke}{
               \section[Arthur Schnitzler an Hugo von Hofmannsthal, 14. 10. 1898]{ Arthur Schnitzler an Hugo von Hofmannsthal,
                    14. 10. 1898}\nopagebreak\mylabel{v}\rehead{ }\begin{ledgroupsized}[t]{13cm}\normalsize\beginnumbering \toendnotes[C]{\smallbreak\pagebreak[2]} \Standort{FDH, Hs-30885,78.}
\physDesc{Brief, 1 Blatt, 3 Seiten
\newline{}Handschrift: schwarze Tinte, deutsche Kurrent\newline{}Ordnung: von Schnitzler mutmaßlich bei der Durchsicht der Korrespondenz 1929
                                    mit Bleistift datiert: »14/10 98« }\buchAbdrucke{\weitereDrucke{Hugo von Hofmannsthal, Arthur Schnitzler: \emph{Briefwechsel}. Hg. Therese Nickl und Heinrich Schnitzler. Frankfurt am Main: \emph{S. Fischer} 1964, S. 114.} }\toendnotes[C]{\smallbreak}\pstart
           \noindent{}{\pb}mein lieber Hugo, es iſt jetzt ſo grau und kühl und feucht, und
                    ich bin ſo verſchnupft und habe eine ganz geſchwollene Naſe, dſs wohl an eine
                        Hinterbrühl\oindex{XXXX Ortsangabe fehlt|pw}erreiſe kaum zu denken iſt,
                    vielmehr vermute ich Sie ko{\geminationm}en früher nach Wien\oindex{XXXX Ortsangabe fehlt|pw}. Viele Grüße hab ich Ihnen von Brahm\pwindex{\textcolor{red}{\textsuperscript{XXXX1 indx}}|pw}, Harden\pwindex{\textcolor{red}{\textsuperscript{XXXX1 indx}}|pw} und der Dumont\pwindex{\textcolor{red}{\textsuperscript{XXXX1 indx}}|pw} zu
                    bringen. Die Leute ſpüren doch ungefähr, wer Sie ſind. Man freut ſich auf Ihr
                        Wiederko{\geminationm}en, auf Ihr neues Stück\textcolor{red}{\textsuperscript{XXXX indx}}, {\pb}– mir ſcheint, im Jänner{ }ſind einige Abende für Sie frei; (von
                    den künftigen Monaten ganz zu geſchweigen.)\pend
           \pstart
           Über meinen Berl\oindex{XXXX Ortsangabe fehlt|pw}. Aufenthalt mündlich. Der
                    Erfolg nach dem 3. Akt\textcolor{red}{\textsuperscript{XXXX indx}}
                    war überraſchend ſtark. Während des Akts\textcolor{red}{\textsuperscript{XXXX indx}} hatte ich die Empfindung, das Stück\textcolor{red}{\textsuperscript{XXXX indx}} iſt hin. Da kamen die letzten
                    paar Scenen, die wirkten unmittelbar und ſind ja wirklich aller Ehren wert. Aber
                    aus welchen {\pb}Tiefen ſteigen ſie empor! –\pend
           \pstart
           – Im übrigen wird ſich das Stück\textcolor{red}{\textsuperscript{XXXX indx}} nicht lang halten; ſchon die 3. Vorſtellung war ſchwach
                    beſucht.\pend
           \pstart
           – Von meinen 3 Einaktern\textcolor{red}{\textsuperscript{XXXX indx}}
                    hat dem Br.\pwindex{\textcolor{red}{\textsuperscript{XXXX1 indx}}|pw} der gefärbte Vogel\textcolor{red}{\textsuperscript{XXXX indx}} (wie es ſcheint
                    weitaus) am beſten gefallen. \introOben{}Aufführung wahrſcheinlich
                            Februar mit Kainz\pwindex{\textcolor{red}{\textsuperscript{XXXX1 indx}}|pw}.\introOben{}\pend
           \pstart
           Seien Sie herzlich gegrüßt und laſſen Sie uns bald zuſa{\geminationm}en ſein.\pend
           \pstart Ihr \spacefill\mbox{Arthur}\pend{}\pstart
           Wien\oindex{XXXX Ortsangabe fehlt|pw},
                        14. X. 98.\pend
           
         
         \endnumbering\mylabel{h}\end{ledgroupsized}  \newcommand{\dateiname}{L00853}\newcommand{\titel}{Arthur Schnitzler an Hugo von Hofmannsthal, 14. 10. 1898}\newcommand{\editorInnen}{Martin Anton Müller und Gerd-Hermann Susen}%% latex-leseansicht-abspann.tex
%% Abspann für die Leseansicht.
%% Der Schalter \ifkorrekturansicht ist bereits durch den Vorspann gesetzt.

%% latex-abspann.tex
%% Gemeinsamer Abspann für Korrekturansicht und Leseansicht.
%% Setzt den Schalter \ifkorrekturansicht voraus (gesetzt in den
%% einbindenden Dateien latex-korrekturansicht-abspann.tex bzw.
%% latex-leseansicht-abspann.tex).
%% ---------------------------------------------------------------

\normalsize

% Das esempio-Environment wird nur in der Leseansicht benötigt
\ifkorrekturansicht\else
\newenvironment{esempio}[3]%
{
    \vspace{1.5ex}
    \rlap{\underline{#1}}
    \par
    \setlength{\parindent}{0cm}
    \nopagebreak
    \leftskip=#2cm
    \rightskip=#3cm
}
{
    \par
}
\fi

\doendnotes{C}
\bigskip
\vfill

\clearpage

\footnotesize

\ifkorrekturansicht
  \lohead{\textsc{register}}
\fi

% theindex-Environment neu definieren ohne reledmac
\makeatletter
\renewenvironment{theindex}{%
  \ifkorrekturansicht
    \section*{\indexname}%
  \else
    \subsubsection*{Index der erwähnten Entitäten}%
  \fi
  \setlength{\parindent}{0pt}%
  \setlength{\parskip}{0pt plus 0.3pt}%
  \let\item\@idxitem
}{%
  \ifkorrekturansicht\clearpage\fi
}
\makeatother

\IfFileExists{\jobname-pw.ind}{\input{\jobname-pw.ind}}{}

% Quellenangabe nur in der Leseansicht
\ifkorrekturansicht\else
% Fallback-Definitionen, falls die .tex-Datei \titel etc. nicht gesetzt hat
\providecommand{\titel}{}
\providecommand{\editorInnen}{}
\providecommand{\dateiname}{\jobname}

\vspace{3cm}

\vfill

\footnotesize
\textsc{Quelle}: \titel. Herausgegeben von {\editorInnen}. In: \emph{Arthur Schnitzler: Briefwechsel mit Autorinnen und Autoren}.
 Digitale Edition, https://schnitzler-briefe.acdh.oeaw.ac.at/{\dateiname}.html (Stand \today)
\fi

\end{document}


      