%% latex-leseansicht-vorspann.tex
%% Vorspann für die Leseansicht.
%% Lädt die gemeinsame Datei latex-vorspann.tex mit nicht gesetztem Schalter.

\newif\ifkorrekturansicht
\korrekturansichtfalse

\input{../tex-inputs/latex-vorspann}


         
         \renewcommand{\erwaehntePersonen}{Personen: Otto Brahm, Louise Dumont, Maximilian Harden, Hugo von Hofmannsthal, Josef Kainz}
         \renewcommand{\erwaehnteOrte}{Orte: Berlin, Hinterbrühl, Wien}
         \renewcommand{\erwaehnteWerke}{Werke: Das Vermächtnis. Schauspiel in drei Akten, Der Abenteurer und die Sängerin oder Die Geschenke des Lebens, Der grüne Kakadu – Paracelsus – Die Gefährtin. Drei Einakter, Der grüne Kakadu. Groteske in einem Akt}
               \section[Arthur Schnitzler an Hugo von Hofmannsthal, 14. 10. 1898]{ Arthur Schnitzler an Hugo von Hofmannsthal, 14. 10. 1898}\nopagebreak\mylabel{v}\rehead{ }\begin{ledgroupsized}[t]{13cm}\normalsize\beginnumbering \toendnotes[C]{\smallbreak\pagebreak[2]} \Standort{FDH, Hs-30885,78.}
\physDesc{Brief, 1 Blatt, 3 Seiten, 1103 Zeichen
\newline{}Handschrift: schwarze Tinte, deutsche Kurrent
\newline{}Ordnung: mit Bleistift von Schnitzler mutmaßlich bei der Durchsicht der Korrespondenz
                                    1929 datiert: »14/10 98« }\buchAbdrucke{\weitereDrucke{Hugo von Hofmannsthal, Arthur Schnitzler: \emph{Briefwechsel}. Hg. Therese Nickl und Heinrich Schnitzler. Frankfurt am Main: \emph{S. Fischer} 1964, S. 114.} }\toendnotes[C]{\smallbreak}\pstart
           \noindent{}{\pb}mein lieber Hugo, es iſt jetzt ſo grau und kühl und feucht, und ich
               bin ſo verſchnupft und habe eine ganz geſchwollene Naſe, dſs wohl an eine Hinterbrühl\oindex{Hinterbruehl@\textbf{Hinterbrühl}|pw}erreiſe kaum zu denken iſt, vielmehr
               vermute ich Sie ko{\geminationm}en früher nach Wien\oindex{Wien@\textbf{Wien}|pw}. Viele Grüße hab ich Ihnen von Brahm\pwindex{Brahm, Otto 05.02.1856 – 28.11.1912@\textsc{Brahm, Otto} (05.02.1856 – 28.11.1912), \emph{Theaterleiter, Regisseur}|pw}, Harden\pwindex{Harden, Maximilian 20.10.1861 – 30.10.1927@\textsc{Harden, Maximilian} (20.10.1861 – 30.10.1927), \emph{Schriftsteller, Publizist}|pw} und der
                  Dumont\pwindex{Dumont, Louise 22.02.1862 – 16.05.1932@\textsc{Dumont, Louise} (22.02.1862 – 16.05.1932), \emph{Theaterleiterin, Schauspielerin}|pw} zu bringen. Die Leute ſpüren doch
               ungefähr, wer Sie ſind. Man freut ſich auf Ihr Wiederko{\geminationm}en, auf Ihr neues Stück\pwindex{Hofmannsthal, Hugo von 1874-02-01 – 1929-07-15@\textsc{Hofmannsthal, Hugo von} (1874-02-01 – 1929-07-15), \emph{Schriftsteller}!Abenteurer und die Saengerin oder Die Geschenke des Lebens18. 3. 1899@\strich\emph{Der Abenteurer und die Sängerin oder Die Geschenke des Lebens} {[}18. 3. 1899{]}|pwv}, {\pb}– mir ſcheint, im Jänner{ }ſind einige Abende für Sie frei; (von den künftigen
               Monaten ganz zu geſchweigen.)\pend
           \pstart
           Über meinen Berl\oindex{Berlin@\textbf{Berlin}|pw}. Aufenthalt mündlich. Der Erfolg
               nach dem 3. Akt\pwindex{Schnitzler, Arthur 15.05.1862 – 21.10.1931@\textsc{Schnitzler, Arthur} (15.05.1862 – 21.10.1931), \emph{Schriftsteller, Mediziner}!Vermaechtnis. Schauspiel in drei Akten1898@\strich\emph{Das Vermächtnis. Schauspiel in drei Akten} {[}1898{]}|pwv} war
               überraſchend ſtark. Während des Akts\pwindex{Schnitzler, Arthur 15.05.1862 – 21.10.1931@\textsc{Schnitzler, Arthur} (15.05.1862 – 21.10.1931), \emph{Schriftsteller, Mediziner}!Vermaechtnis. Schauspiel in drei Akten1898@\strich\emph{Das Vermächtnis. Schauspiel in drei Akten} {[}1898{]}|pwv} hatte ich die Empfindung, das Stück\pwindex{Schnitzler, Arthur 15.05.1862 – 21.10.1931@\textsc{Schnitzler, Arthur} (15.05.1862 – 21.10.1931), \emph{Schriftsteller, Mediziner}!Vermaechtnis. Schauspiel in drei Akten1898@\strich\emph{Das Vermächtnis. Schauspiel in drei Akten} {[}1898{]}|pwv} iſt hin. Da kamen die letzten paar Scenen, die
               wirkten unmittelbar und ſind ja wirklich aller Ehren wert. Aber aus welchen {\pb}Tiefen ſteigen ſie empor! –\pend
           \pstart
           – Im übrigen wird ſich das Stück\pwindex{Schnitzler, Arthur 15.05.1862 – 21.10.1931@\textsc{Schnitzler, Arthur} (15.05.1862 – 21.10.1931), \emph{Schriftsteller, Mediziner}!Vermaechtnis. Schauspiel in drei Akten1898@\strich\emph{Das Vermächtnis. Schauspiel in drei Akten} {[}1898{]}|pwv} nicht lang halten; ſchon die 3. Vorſtellung war ſchwach beſucht.\pend
           \pstart
           – Von meinen 3 Einaktern\pwindex{Schnitzler, Arthur 15.05.1862 – 21.10.1931@\textsc{Schnitzler, Arthur} (15.05.1862 – 21.10.1931), \emph{Schriftsteller, Mediziner}!gruene Kakadu – Paracelsus – Die Gefaehrtin. Drei Einakter1898 – 1899@\strich\emph{Der grüne Kakadu – Paracelsus – Die Gefährtin. Drei Einakter} {[}1898 – 1899{]}|pwv} hat
               dem Br.\pwindex{Brahm, Otto 05.02.1856 – 28.11.1912@\textsc{Brahm, Otto} (05.02.1856 – 28.11.1912), \emph{Theaterleiter, Regisseur}|pw} der gefärbte Vogel\pwindex{Schnitzler, Arthur 15.05.1862 – 21.10.1931@\textsc{Schnitzler, Arthur} (15.05.1862 – 21.10.1931), \emph{Schriftsteller, Mediziner}!gruene Kakadu. Groteske in einem Akt1. 3. 1899@\strich\emph{Der grüne Kakadu. Groteske in einem Akt} {[}1. 3. 1899{]}|pwv} (wie es ſcheint weitaus) am beſten gefallen.
                  \introOben{}Aufführung wahrſcheinlich Februar mit Kainz\pwindex{Kainz, Josef 02.01.1858 – 20.09.1910@\textsc{Kainz, Josef} (02.01.1858 – 20.09.1910), \emph{Schauspieler}|pw}.\introOben{}\pend
           \pstart
           Seien Sie herzlich gegrüßt und laſſen Sie uns bald zuſa{\geminationm}en ſein.\pend
           \pstart Ihr \spacefill\mbox{Arthur}\pend{}\pstart
           Wien\oindex{Wien@\textbf{Wien}|pw}, 14. X. 98.\pend
           
         
         \endnumbering\mylabel{h}\end{ledgroupsized}  \newcommand{\dateiname}{L00853}\newcommand{\titel}{Arthur Schnitzler an Hugo von Hofmannsthal, 14. 10. 1898}\newcommand{\editorInnen}{Martin Anton Müller und Gerd-Hermann Susen}%% latex-leseansicht-abspann.tex
%% Abspann für die Leseansicht.
%% Der Schalter \ifkorrekturansicht ist bereits durch den Vorspann gesetzt.

%% latex-abspann.tex
%% Gemeinsamer Abspann für Korrekturansicht und Leseansicht.
%% Setzt den Schalter \ifkorrekturansicht voraus (gesetzt in den
%% einbindenden Dateien latex-korrekturansicht-abspann.tex bzw.
%% latex-leseansicht-abspann.tex).
%% ---------------------------------------------------------------

\normalsize

% Das esempio-Environment wird nur in der Leseansicht benötigt
\ifkorrekturansicht\else
\newenvironment{esempio}[3]%
{
    \vspace{1.5ex}
    \rlap{\underline{#1}}
    \par
    \setlength{\parindent}{0cm}
    \nopagebreak
    \leftskip=#2cm
    \rightskip=#3cm
}
{
    \par
}
\fi

\doendnotes{C}
\bigskip
\vfill

\clearpage

\footnotesize

\ifkorrekturansicht
  \lohead{\textsc{register}}
\fi

% theindex-Environment neu definieren ohne reledmac
\makeatletter
\renewenvironment{theindex}{%
  \ifkorrekturansicht
    \section*{\indexname}%
  \else
    \subsubsection*{Index der erwähnten Entitäten}%
  \fi
  \setlength{\parindent}{0pt}%
  \setlength{\parskip}{0pt plus 0.3pt}%
  \let\item\@idxitem
}{%
  \ifkorrekturansicht\clearpage\fi
}
\makeatother

\IfFileExists{\jobname-pw.ind}{\input{\jobname-pw.ind}}{}

% Quellenangabe nur in der Leseansicht
\ifkorrekturansicht\else
% Fallback-Definitionen, falls die .tex-Datei \titel etc. nicht gesetzt hat
\providecommand{\titel}{}
\providecommand{\editorInnen}{}
\providecommand{\dateiname}{\jobname}

\vspace{3cm}

\vfill

\footnotesize
\textsc{Quelle}: \titel. Herausgegeben von {\editorInnen}. In: \emph{Arthur Schnitzler: Briefwechsel mit Autorinnen und Autoren}.
 Digitale Edition, https://schnitzler-briefe.acdh.oeaw.ac.at/{\dateiname}.html (Stand \today)
\fi

\end{document}


      