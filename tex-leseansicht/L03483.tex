%% latex-leseansicht-vorspann.tex
%% Vorspann für die Leseansicht.
%% Lädt die gemeinsame Datei latex-vorspann.tex mit nicht gesetztem Schalter.

\newif\ifkorrekturansicht
\korrekturansichtfalse

\input{../tex-inputs/latex-vorspann}

\begin{center}
            \textcolor{red}{ENTWURF, NICHT FERTIG KORRIGIERT}
                      \end{center}
            
         
         \renewcommand{\erwaehntePersonen}{Personen: Eva Marie Goldmann}
         \renewcommand{\erwaehnteOrte}{Orte: Berlin, Gmunden, Hotel Excelsior, Kärnten, Sternwartestraße 71, Velden am Wörthersee, Wien, XVIII., Währing}
         \renewcommand{\erwaehnteWerke}{}
               \section[ Paul Goldmann an Arthur Schnitzler, 22. 8. 1931]{ Paul Goldmann an Arthur Schnitzler, 22. 8. 1931}\nopagebreak\mylabel{v}\rehead{ }\begin{ledgroupsized}[t]{13cm}\normalsize\beginnumbering \toendnotes[C]{\smallbreak\pagebreak[2]} \Standort{DLA, A:Schnitzler, HS.NZ85.1.3176.}
\physDesc{Bildpostkarte, 579 Zeichen
\newline{}Handschrift: 1) schwarze Tinte, deutsche Kurrent\hspace{1em}2) schwarze Tinte, lateinische Kurrent (\noindent{}Adresse)\hspace{1em}
\newline{}Versand: Stempel: »\nobreak{}\oindex{Velden am Woerthersee@\textbf{Velden am Wörthersee}|pwk}Velden \textcolor{gray}{am}
                                       Wörthersee, 22. VIII. 31\nobreak{}«.  }\toendnotes[C]{\smallbreak}\pstart{}{\pb}Herrn Dr.\pend{}\pstart{}Arthur Schnitzler\pend{}\pstart{}Wien XVIII.\oindex{XVIII., Waehring@\textbf{XVIII., Währing}|pw}\pend{}\pstart{}Sternwartstr. 71\oindex{Sternwartestrasse 71@\textbf{Sternwartestraße 71}|pw}.\pend{}{\bigskip}\pstart
           \noindent{}\centering{}{\pb}\textcolor{gray}{\textbf{Blick auf Velden am
                        Wörthersee\oindex{Velden am Woerthersee@\textbf{Velden am Wörthersee}|pw}.}}\pend
           \pstart
           Velden a. Wörtherſee\oindex{Velden am Woerthersee@\textbf{Velden am Wörthersee}|pw}, \textsc{Hotel Excelsior\oindex{Hotel Excelsior@\textbf{Hotel Excelsior}|pw}}, den 22. 8.\pend
           \pstart
           Lieber Freund, Mit großer Verſpätung erreicht Deine
               nach Berlin\oindex{Berlin@\textbf{Berlin}|pw} geſandte Karte mich hier in Velden\oindex{Velden am Woerthersee@\textbf{Velden am Wörthersee}|pw}. Meine Frau\pwindex{Goldmann, Eva Marie 27.10.1877 – 02.11.1937@\textsc{Goldmann, Eva Marie} (27.10.1877 – 02.11.1937)|pwv} wollte abſolut an einen See gehen; für mich hat es wenig
               Sinn, da ich nicht ſchwimme, auch iſt mir die Luft zu lau u. zu ſchlapp. Aber das \textsc{Hotel Excelsior\oindex{Hotel Excelsior@\textbf{Hotel Excelsior}|pw}} iſt vorzüglich. Hoffentlich verbringſt Du \label{K_L03483-1v}\edtext{an einem ſchönen Ort einen angenehmen Sommer}{\lemma{\textnormal{\emph{an … Sommer}}}\Cendnote{\textnormal{Schnitzler\pwindex{Schnitzler, Arthur 15.05.1862 – 21.10.1931@\textsc{Schnitzler, Arthur} (15.05.1862 – 21.10.1931), \emph{Schriftsteller, Mediziner}|pwk} war seit 7. 8. 1931 und noch
                  bis 25. 8. 1931 in
                     Gmunden\oindex{Gmunden@\textbf{Gmunden}|pwk}.}}}\label{K_L03483-1h}. Wir\pwindex{Goldmann, Eva Marie 27.10.1877 – 02.11.1937@\textsc{Goldmann, Eva Marie} (27.10.1877 – 02.11.1937)|pwv} haben hier faſt immer schönes Wetter,
               das iſt der große Vorzug von Kärnten\oindex{Kaernten@\textbf{Kärnten}|pw}. Alles
               Herzliche von meiner Frau\pwindex{Goldmann, Eva Marie 27.10.1877 – 02.11.1937@\textsc{Goldmann, Eva Marie} (27.10.1877 – 02.11.1937)|pwv} u.
               mir! Dein \spacefill\mbox{Paul Goldmann.}\pend
           
         
         \endnumbering\mylabel{h}\end{ledgroupsized}\begin{anhang}\end{anhang}\newcommand{\dateiname}{L03483}\newcommand{\titel}{Paul Goldmann an Arthur Schnitzler, 22. 8. 1931}\newcommand{\editorInnen}{Martin Anton Müller und Laura Untner}%% latex-leseansicht-abspann.tex
%% Abspann für die Leseansicht.
%% Der Schalter \ifkorrekturansicht ist bereits durch den Vorspann gesetzt.

%% latex-abspann.tex
%% Gemeinsamer Abspann für Korrekturansicht und Leseansicht.
%% Setzt den Schalter \ifkorrekturansicht voraus (gesetzt in den
%% einbindenden Dateien latex-korrekturansicht-abspann.tex bzw.
%% latex-leseansicht-abspann.tex).
%% ---------------------------------------------------------------

\normalsize

% Das esempio-Environment wird nur in der Leseansicht benötigt
\ifkorrekturansicht\else
\newenvironment{esempio}[3]%
{
    \vspace{1.5ex}
    \rlap{\underline{#1}}
    \par
    \setlength{\parindent}{0cm}
    \nopagebreak
    \leftskip=#2cm
    \rightskip=#3cm
}
{
    \par
}
\fi

\doendnotes{C}
\bigskip
\vfill

\clearpage

\footnotesize

\ifkorrekturansicht
  \lohead{\textsc{register}}
\fi

% theindex-Environment neu definieren ohne reledmac
\makeatletter
\renewenvironment{theindex}{%
  \ifkorrekturansicht
    \section*{\indexname}%
  \else
    \subsubsection*{Index der erwähnten Entitäten}%
  \fi
  \setlength{\parindent}{0pt}%
  \setlength{\parskip}{0pt plus 0.3pt}%
  \let\item\@idxitem
}{%
  \ifkorrekturansicht\clearpage\fi
}
\makeatother

\IfFileExists{\jobname-pw.ind}{\input{\jobname-pw.ind}}{}

% Quellenangabe nur in der Leseansicht
\ifkorrekturansicht\else
% Fallback-Definitionen, falls die .tex-Datei \titel etc. nicht gesetzt hat
\providecommand{\titel}{}
\providecommand{\editorInnen}{}
\providecommand{\dateiname}{\jobname}

\vspace{3cm}

\vfill

\footnotesize
\textsc{Quelle}: \titel. Herausgegeben von {\editorInnen}. In: \emph{Arthur Schnitzler: Briefwechsel mit Autorinnen und Autoren}.
 Digitale Edition, https://schnitzler-briefe.acdh.oeaw.ac.at/{\dateiname}.html (Stand \today)
\fi

\end{document}


      