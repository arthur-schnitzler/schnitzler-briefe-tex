%% latex-leseansicht-vorspann.tex
%% Vorspann für die Leseansicht.
%% Lädt die gemeinsame Datei latex-vorspann.tex mit nicht gesetztem Schalter.

\newif\ifkorrekturansicht
\korrekturansichtfalse

\input{../tex-inputs/latex-vorspann}


\section[ Paul Goldmann an Arthur Schnitzler, 22. 8. 1931]{L03483 Paul Goldmann an Arthur Schnitzler,  22. 8. 1931}
\nopagebreak\mylabel{L03483v}
\rehead{ }\normalsize\beginnumbering\briefempfaengerindex{Schnitzler, Arthur@\textsc{Schnitzler, Arthur}!zzzGoldmann, Paul@\emph{von Paul Goldmann}!1931-08-221@{22. 8. 1931}|(be}
\toendnotes[C]{\smallbreak\pagebreak[2]}
\correspDesc{Versand  durch Paul Goldmann am 22. 8. 1931 in Velden am Wörthersee
\newline{}Erhalt  durch Arthur Schnitzler am [25. 8. 1931?] in Wien}\toendnotes[C]{\smallbreak}
\Standort{DLA, A:Schnitzler, HS.NZ85.1.3176.}
\physDesc{Bildpostkarte, 579 Zeichen
\newline{}Handschrift: schwarze Tinte, deutsche Kurrent
\newline{}Versand: Stempel: »\nobreak{}A\textcolor{gray}{LPEN}SEEB{[}AD{]} u. {[}KLIMATISC{]}\textcolor{gray}{H}ER KUR\textcolor{gray}{O}{[}RT{]}\oindex{Velden am Wörthersee@\textbf{Velden am Wörthersee}|pwv}{ }{[}am{]}{ }W\textcolor{gray}{ÄR}MSTEN ALPE\textcolor{gray}{NS}EE\oindex{Wörthersee@\textbf{Wörthersee}, \emph{See}|pwv}{ }EUROPA\oindex{Europa@\textbf{Europa}|pw}’s\nobreak{}«. Stempel: »\nobreak{}\oindex{Velden am Wörthersee@\textbf{Velden am Wörthersee}|pwk}Velden \textcolor{gray}{am}
                                       Wörthersee, 22. VIII. 31\nobreak{}«.  }\toendnotes[C]{\smallbreak}\pstart{}\textsc{{\pb}Herrn Dr.}\pend{}\pstart{}\textsc{Arthur Schnitzler}\pend{}\pstart{}\textsc{Wien XVIII.\oindex{XVIII., Währing@\textbf{XVIII., Währing}, \emph{Verwaltungsgebiet}|pw}}\pend{}\pstart{}\textsc{Sternwartstr. 71\oindex{Wien@\textbf{Wien}!XVIII., Währing@\textbf{XVIII., Währing}!Sternwartestraße 71@\textbf{Sternwartestraße 71}, \emph{Wohngebäude}|pw}.}\pend{}{\bigskip}
\pstart
           \noindent{}\centering{}{\pb}\textcolor{gray}{\textbf{Blick auf Velden am
                     Wörthersee\oindex{Velden am Wörthersee@\textbf{Velden am Wörthersee}|pw}.}}\pend
           \vspace{1em}
\pstart
           {\pb}Velden a. Wörtherſee\oindex{Velden am Wörthersee@\textbf{Velden am Wörthersee}|pw}, \textsc{Hotel Excelsior\oindex{Hotel Excelsior [Velden]@\textbf{Hotel Excelsior [Velden]}, \emph{Hotel}|pw}}, den 22. 8.\pend
           \vspace{0.5em}
\pstart
           Lieber Freund, Mit großer Verſpätung erreicht Deine nach Berlin\oindex{Berlin@\textbf{Berlin}, \emph{Hauptstadt}|pw} geſandte Karte mich hier in Velden\oindex{Velden am Wörthersee@\textbf{Velden am Wörthersee}|pw}. Meine Frau\pwindex{Goldmann, Eva Marie 27.\,10.\,1877 Wien – 2.\,11.\,1937 ebd.@\textsc{Goldmann, Eva Marie} (27.\,10.\,1877 Wien – 2.\,11.\,1937 ebd.)|pwv} wollte abſolut an einen See gehen; für mich hat es wenig
               Sinn, da ich nicht{ }ſchwimme, auch iſt mir die Luft zu lau u. zu{ }ſchlapp. Aber das \textsc{Hotel Excelsior\oindex{Hotel Excelsior [Velden]@\textbf{Hotel Excelsior [Velden]}, \emph{Hotel}|pw}} iſt vorzüglich. Hoffentlich verbringſt Du an einem{ }ſchönen \label{K_L03483-1v}\edtext{Ort einen angenehmen Sommer}{\lemma{\textnormal{\emph{Ort … Sommer}}}\Cendnote{\textnormal{Schnitzler war seit 7. 8. 1931 und noch
                  bis 25. 8. 1931 in
                     Gmunden\oindex{Gmunden@\textbf{Gmunden}|pwk}.}}}\label{K_L03483-1}. Wir\pwindex{Goldmann, Eva Marie 27.\,10.\,1877 Wien – 2.\,11.\,1937 ebd.@\textsc{Goldmann, Eva Marie} (27.\,10.\,1877 Wien – 2.\,11.\,1937 ebd.)|pwv} haben hier faſt immer schönes Wetter,
               das iſt der große Vorzug von Kärnten\oindex{Kärnten@\textbf{Kärnten}, \emph{Land}|pw}. Alles
               Herzliche von meiner Frau\pwindex{Goldmann, Eva Marie 27.\,10.\,1877 Wien – 2.\,11.\,1937 ebd.@\textsc{Goldmann, Eva Marie} (27.\,10.\,1877 Wien – 2.\,11.\,1937 ebd.)|pwv} u.
               mir! Dein \spacefill\mbox{Paul Goldmann.}\pend
           \selectlanguage{ngerman}\endnumbering\briefempfaengerindex{Schnitzler, Arthur@\textsc{Schnitzler, Arthur}!zzzGoldmann, Paul@\emph{von Paul Goldmann}!1931-08-221@{22. 8. 1931}|)be}\mylabel{L03483h}  \newcommand{\dateiname}{L03483}\newcommand{\titel}{Paul Goldmann an Arthur Schnitzler, 22. 8. 1931}\newcommand{\editorInnen}{Martin Anton Müller und Laura Untner}%% latex-leseansicht-abspann.tex
%% Abspann für die Leseansicht.
%% Der Schalter \ifkorrekturansicht ist bereits durch den Vorspann gesetzt.

%% latex-abspann.tex
%% Gemeinsamer Abspann für Korrekturansicht und Leseansicht.
%% Setzt den Schalter \ifkorrekturansicht voraus (gesetzt in den
%% einbindenden Dateien latex-korrekturansicht-abspann.tex bzw.
%% latex-leseansicht-abspann.tex).
%% ---------------------------------------------------------------

\normalsize

% Das esempio-Environment wird nur in der Leseansicht benötigt
\ifkorrekturansicht\else
\newenvironment{esempio}[3]%
{
    \vspace{1.5ex}
    \rlap{\underline{#1}}
    \par
    \setlength{\parindent}{0cm}
    \nopagebreak
    \leftskip=#2cm
    \rightskip=#3cm
}
{
    \par
}
\fi

\doendnotes{C}
\bigskip
\vfill

\clearpage

\footnotesize

\ifkorrekturansicht
  \lohead{\textsc{register}}
\fi

% theindex-Environment neu definieren ohne reledmac
\makeatletter
\renewenvironment{theindex}{%
  \ifkorrekturansicht
    \section*{\indexname}%
  \else
    \subsubsection*{Index der erwähnten Entitäten}%
  \fi
  \setlength{\parindent}{0pt}%
  \setlength{\parskip}{0pt plus 0.3pt}%
  \let\item\@idxitem
}{%
  \ifkorrekturansicht\clearpage\fi
}
\makeatother

\IfFileExists{\jobname-pw.ind}{\input{\jobname-pw.ind}}{}

% Quellenangabe nur in der Leseansicht
\ifkorrekturansicht\else
% Fallback-Definitionen, falls die .tex-Datei \titel etc. nicht gesetzt hat
\providecommand{\titel}{}
\providecommand{\editorInnen}{}
\providecommand{\dateiname}{\jobname}

\vspace{3cm}

\vfill

\footnotesize
\textsc{Quelle}: \titel. Herausgegeben von {\editorInnen}. In: \emph{Arthur Schnitzler: Briefwechsel mit Autorinnen und Autoren}.
 Digitale Edition, https://schnitzler-briefe.acdh.oeaw.ac.at/{\dateiname}.html (Stand \today)
\fi

\end{document}


