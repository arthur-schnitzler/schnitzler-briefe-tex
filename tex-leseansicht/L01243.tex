%% latex-leseansicht-vorspann.tex
%% Vorspann für die Leseansicht.
%% Lädt die gemeinsame Datei latex-vorspann.tex mit nicht gesetztem Schalter.

\newif\ifkorrekturansicht
\korrekturansichtfalse

\input{../tex-inputs/latex-vorspann}


               \section[Arthur Schnitzler an Hugo von Hofmannsthal, 21. 10. 1902]{ Arthur Schnitzler an Hugo von Hofmannsthal, 21. 10. 1902}\nopagebreak\mylabel{v}\rehead{ }\begin{ledgroupsized}[t]{13cm}\normalsize\beginnumbering\briefempfaengerindex{Hofmannsthal, Hugo von@\textsc{Hofmannsthal, Hugo von}!zzzSchnitzler, Arthur@\emph{von Arthur Schnitzler}!1902-10-211@{21. 10. 1902}|(be} \toendnotes[C]{\smallbreak\pagebreak[2]} \Standort{FDH, Hs-30885,99.}
\physDesc{Postkarte
\newline{}Handschrift: schwarze Tinte, deutsche Kurrent\newline{}Versand: Stempel: »\nobreak{}\oindex{IX., Alsergrund@\textbf{IX., Alsergrund}|pwk}9/3 Wien 72, 21. 10. 02, 8N\nobreak{}«.  \newline{}Ordnung: von Schnitzler mutmaßlich
                           bei der Durchsicht der Korrespondenz 1929 mit Bleistift beschriftet:
                                 »Rom\oindex{Rom@\textbf{Rom}|pw}{ }1903.« }\buchAbdrucke{\weitereDrucke{Hugo von Hofmannsthal, Arthur Schnitzler: \emph{Briefwechsel}. Hg. Therese Nickl und Heinrich Schnitzler. Frankfurt am Main: \emph{S. Fischer} 1964, S. 162.} }\toendnotes[C]{\smallbreak}\pstart{}{\pb}\textsc{Hrn Hugo v. Hofmannsthal}\pend{}\pstart{}\textsc{Rom}\oindex{Rom@\textbf{Rom}|pw}\pend{}\pstart{}\textsc{Hotel Hassler\oindex{Hôtel Hassler@\textbf{Hôtel Hassler}|pw}}\pend{}\pstart{}\textsc{Italia}\oindex{Italien@\textbf{Italien}|pw}\pend{}{\bigskip}\pstart
           \noindent{}{\pb}lieber, die Sandrock\pwindex{Sandrock, Adele 19.08.1863 – 30.08.1937@\textsc{Sandrock, Adele} (19.08.1863 – 30.08.1937), \emph{Schauspielerin}|pw} möchte den
                  Tod des Tizian\pwindex{Hofmannsthal, Hugo von 01.02.1874 – 15.07.1929@\textsc{Hofmannsthal, Hugo von} (01.02.1874 – 15.07.1929), \emph{Schriftsteller}!Tod des Tizian1892.10@\strich\emph{Der Tod des Tizian} {[}1892.10{]}|pw}, wohl um ihn vorzuleſen; – bitte
               ſehr laſſen Sie ihr ein Exemplar ſenden.\pend
           \pstart
           – Ich bin heute Früh aus \textsc{Agnetendorf}\oindex{Agnetendorf@\textbf{Agnetendorf}|pw} gekommen, wo ich nach 6tägigem Berlin\oindex{Berlin@\textbf{Berlin}|pw}er
               Aufenthalt, 1 Tag mit Brahm\pwindex{Brahm, Otto 05.02.1856 – 28.11.1912@\textsc{Brahm, Otto} (05.02.1856 – 28.11.1912), \emph{Theaterleiter, Regisseur}|pw} bei Hauptmann\pwindex{Hauptmann, Gerhart 15.11.1862 – 06.06.1946@\textsc{Hauptmann, Gerhart} (15.11.1862 – 06.06.1946), \emph{Schriftsteller}|pw}{ }ſehr angenehm
               verbrachte. –\pend
           \pstart
           \textsc{Beatrice}\pwindex{Schnitzler, Arthur 15.05.1862 – 21.10.1931@\textsc{Schnitzler, Arthur} (15.05.1862 – 21.10.1931), \emph{Schriftsteller, Mediziner}!Schleier der Beatrice. Schauspiel in fuenf Akten1900-12-01 – 1900-12-01@\strich\emph{Der Schleier der Beatrice. Schauspiel in fünf Akten} {[}1900-12-01 – 1900-12-01{]}|pw} dürfte im Feber am Dtſch. Th.\oindex{Deutsches Theater Berlin@\textbf{Deutsches Theater Berlin}|pw}
               geſpielt werden. –\pend
           \pstart
           \textsc{M. Vanna}\pwindex{\textcolor{red}{\textsuperscript{XXXX1 indx}}!Monna Vanna7.5.1902 – 7.5.1902@\strich\emph{Monna Vanna} {[}7.5.1902 – 7.5.1902{]}|pw} iſt ein außerordentlicher Kaſſenerfolg. Die \label{K_L01243_1v}\edtext{Aufführung}{\lemma{\textnormal{\emph{Aufführung}}}\Cendnote{\textnormal{Er besuchte die Vorstellung am 14. 10. 1902. Zum Urteil
                     Vgl. A. S.: \emph{Tagebuch}, 19. 10. 1902.}}}\label{K_L01243_1h} läßt zu
               wünſchen übrig. Haben Sie meinen Brief erhalten? – Schreiben Sie ein Wort, wie’s
               Ihnen geht.\pend
           \pstart Herzlichſt Ihr \spacefill\mbox{A.}\pend{}\endnumbering\briefempfaengerindex{Hofmannsthal, Hugo von@\textsc{Hofmannsthal, Hugo von}!zzzSchnitzler, Arthur@\emph{von Arthur Schnitzler}!1902-10-211@{21. 10. 1902}|)be}\mylabel{h}\end{ledgroupsized}  \newcommand{\dateiname}{L01243}\newcommand{\titel}{Arthur Schnitzler an Hugo von Hofmannsthal, 21. 10. 1902}\newcommand{\editorInnen}{Martin Anton Müller und Gerd-Hermann Susen}
            \footnotesize
\begin{ledgroupsized}[t]{11.5cm}
\doendnotes{C}
\end{ledgroupsized}
         %% latex-leseansicht-abspann.tex
%% Abspann für die Leseansicht.
%% Der Schalter \ifkorrekturansicht ist bereits durch den Vorspann gesetzt.

%% latex-abspann.tex
%% Gemeinsamer Abspann für Korrekturansicht und Leseansicht.
%% Setzt den Schalter \ifkorrekturansicht voraus (gesetzt in den
%% einbindenden Dateien latex-korrekturansicht-abspann.tex bzw.
%% latex-leseansicht-abspann.tex).
%% ---------------------------------------------------------------

\normalsize

% Das esempio-Environment wird nur in der Leseansicht benötigt
\ifkorrekturansicht\else
\newenvironment{esempio}[3]%
{
    \vspace{1.5ex}
    \rlap{\underline{#1}}
    \par
    \setlength{\parindent}{0cm}
    \nopagebreak
    \leftskip=#2cm
    \rightskip=#3cm
}
{
    \par
}
\fi

\doendnotes{C}
\bigskip
\vfill

\clearpage

\footnotesize

\ifkorrekturansicht
  \lohead{\textsc{register}}
\fi

% theindex-Environment neu definieren ohne reledmac
\makeatletter
\renewenvironment{theindex}{%
  \ifkorrekturansicht
    \section*{\indexname}%
  \else
    \subsubsection*{Index der erwähnten Entitäten}%
  \fi
  \setlength{\parindent}{0pt}%
  \setlength{\parskip}{0pt plus 0.3pt}%
  \let\item\@idxitem
}{%
  \ifkorrekturansicht\clearpage\fi
}
\makeatother

\IfFileExists{\jobname-pw.ind}{\input{\jobname-pw.ind}}{}

% Quellenangabe nur in der Leseansicht
\ifkorrekturansicht\else
% Fallback-Definitionen, falls die .tex-Datei \titel etc. nicht gesetzt hat
\providecommand{\titel}{}
\providecommand{\editorInnen}{}
\providecommand{\dateiname}{\jobname}

\vspace{3cm}

\vfill

\footnotesize
\textsc{Quelle}: \titel. Herausgegeben von {\editorInnen}. In: \emph{Arthur Schnitzler: Briefwechsel mit Autorinnen und Autoren}.
 Digitale Edition, https://schnitzler-briefe.acdh.oeaw.ac.at/{\dateiname}.html (Stand \today)
\fi

\end{document}


      