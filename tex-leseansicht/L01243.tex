%% latex-korrekturansicht-vorspann.tex
%% Vorspann für die Korrekturansicht.
%% Lädt die gemeinsame Datei latex-vorspann.tex mit gesetztem Schalter.

\newif\ifkorrekturansicht
\korrekturansichttrue

\input{../tex-inputs/latex-vorspann}


\section[Arthur Schnitzler an Hugo von Hofmannsthal, 21. 10. 1902]{L01243 Arthur Schnitzler an Hugo von Hofmannsthal, 21. 10. 1902}
\nopagebreak\mylabel{L01243v}
\rehead{ }\normalsize\beginnumbering\briefempfaengerindex{Hofmannsthal, Hugo von@\textsc{Hofmannsthal, Hugo von}!zzzSchnitzler, Arthur@\emph{von Arthur Schnitzler}!1902-10-211@{21. 10. 1902}|(be}
\toendnotes[C]{\smallbreak\pagebreak[2]}\Standort{FDH, Hs-30885,99.}
\physDesc{Postkarte, 529 Zeichen
\newline{}Handschrift: 1) schwarze Tinte, deutsche Kurrent\hspace{1em}2) schwarze Tinte, lateinische Kurrent (\noindent{}Adresse)\hspace{1em}
\newline{}Versand: Stempel: »\nobreak{}\oindex{IX., Alsergrund@\textbf{IX., Alsergrund}, \emph{A.ADM3}|pwk}9/3 Wien 72, 21. 10. 02, 8N\nobreak{}«.  
\newline{}Ordnung: mit Bleistift von Schnitzler mutmaßlich bei der Durchsicht der Korrespondenz
                                    1929 beschriftet: »Rom\oindex{Rom@\textbf{Rom}, \emph{P.PPLC}|pw}{ }1903.« }
\buchAbdrucke{\weitereDrucke{Hugo von Hofmannsthal, Arthur Schnitzler: \emph{Briefwechsel}. Frankfurt am Main: \emph{S. Fischer} 1964, S. 162.} }\toendnotes[C]{\smallbreak}\pstart{}{\pb}Hrn Hugo v. Hofmannsthal\pend{}\pstart{}Rom\oindex{Rom@\textbf{Rom}, \emph{P.PPLC}|pw}\pend{}\pstart{}Hotel Hassler\oindex{Hôtel Hassler@\textbf{Hôtel Hassler}, \emph{Hotel (K.HTL)}|pw}\pend{}\pstart{}Italia\oindex{Italien@\textbf{Italien}, \emph{A.PCLI}|pw}\pend{}{\bigskip}\vspace{1em}
\pstart
           \noindent{}{\pb}lieber, die Sandrock\pwindex{Sandrock, Adele 1863-08-19 – 1937-08-30@\textsc{Sandrock, Adele} (1863-08-19 – 1937-08-30), \emph{Schauspieler/Schauspielerin}|pw} möchte
               den Tod des Tizian\pwindex{Tod des Tizian. Ein Bruchstueck@\emph{Der Tod des Tizian. Ein Bruchstück}|pw}, wohl um ihn vorzuleſen; –
               bitte ſehr laſſen Sie ihr ein Exemplar ſenden.\pend
           
\pstart
           – Ich bin heute Früh aus \textsc{Agnetendorf}\oindex{Jagniątków@\textbf{Jagniątków}, \emph{P.PPL}|pw} gekommen, wo ich nach 6tägigem Berlin\oindex{Berlin@\textbf{Berlin}, \emph{P.PPLC}|pw}er
               Aufenthalt, 1 Tag mit Brahm\pwindex{Brahm, Otto 05.02.1856 – 28.11.1912@\textsc{Brahm, Otto} (05.02.1856 – 28.11.1912), \emph{Theaterleiter/Theaterleiterin, Regisseur/Regisseurin}|pw} bei Hauptmann\pwindex{Hauptmann, Gerhart 15.11.1862 – 06.06.1946@\textsc{Hauptmann, Gerhart} (15.11.1862 – 06.06.1946), \emph{Schriftsteller/Schriftstellerin}|pw}{ }ſehr angenehm verbrachte. –\pend
           
\pstart
           \textsc{Beatrice}\pwindex{Schleier der Beatrice. Schauspiel in fuenf Akten@\emph{Der Schleier der Beatrice. Schauspiel in fünf Akten}|pw} dürfte im Feber am Dtſch. Th.\oindex{Deutsches Theater Berlin@\textbf{Deutsches Theater Berlin}, \emph{Theater (K.THE)}|pw}
               geſpielt werden. –\pend
           
\pstart
           \textsc{M. Vanna}\pwindex{Monna Vanna. Schauspiel in drei Akten@\emph{Monna Vanna. Schauspiel in drei Akten}|pw} iſt ein außerordentlicher Kaſſenerfolg. Die \label{K_L01243-1v}\edtext{Aufführung}{\lemma{\textnormal{\emph{Aufführung}}}\Cendnote{\textnormal{Siehe Paul Goldmann an Arthur Schnitzler, 16. 6. [1902].
               }}}\label{K_L01243-1} läßt zu
               wünſchen übrig. Haben Sie meinen Brief erhalten? – Schreiben Sie ein Wort, wie’s
               Ihnen geht.\pend
           \pstart Herzlichſt Ihr \spacefill\mbox{A.}\pend{}\selectlanguage{ngerman}\endnumbering\briefempfaengerindex{Hofmannsthal, Hugo von@\textsc{Hofmannsthal, Hugo von}!zzzSchnitzler, Arthur@\emph{von Arthur Schnitzler}!1902-10-211@{21. 10. 1902}|)be}\mylabel{L01243h}  \normalsize

\doendnotes{C}
\bigskip
\vfill

\clearpage

\footnotesize

\lohead{\textsc{register}}

% Definiere theindex-Environment komplett neu ohne reledmac
\makeatletter
\renewenvironment{theindex}{%
  \section*{\indexname}%
  \setlength{\parindent}{0pt}%
  \setlength{\parskip}{0pt plus 0.3pt}%
  \let\item\@idxitem
}{%
  \clearpage
}
\makeatother

\IfFileExists{\jobname-pw.ind}{\input{\jobname-pw.ind}}{}

\end{document}

      