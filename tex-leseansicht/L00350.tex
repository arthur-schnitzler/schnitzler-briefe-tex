\input{../tex-inputs/latex-pdf-vorspann}
\begin{center}
            \textcolor{red}{ENTWURF. ENTZIFFERUNG NOCH NICHT KORREKTURGELESEN}
                      \end{center}
            
               \section[Arthur Schnitzler an Richard Beer-Hofmann, 12. 7. 1894]{ Arthur Schnitzler an Richard Beer-Hofmann,
               12. 7. 1894}\nopagebreak\mylabel{v}\rehead{ }\begin{ledgroupsized}[t]{13cm}\normalsize\beginnumbering\briefempfaengerindex{Beer-Hofmann, Richard@\textsc{Beer-Hofmann, Richard}!zzzSchnitzler, Arthur@\emph{von Arthur Schnitzler}!1894-07-121@{12. 7. 1894}|(be} \toendnotes[C]{\smallbreak\pagebreak[2]} \Standort{YCGL, MSS 31.}
\physDesc{Brief, 1 Blatt, 2 Seiten, Umschlag
\newline{}Handschrift: Bleistift, deutsche Kurrent\newline{}Versand: 1) Stempel: »\nobreak{}\oindex{IX., Alsergrund@\textbf{IX., Alsergrund}|pwk}Wien 9/3 72, 12 7 {[}94{]}, 6–7N\nobreak{}«.  2) Stempel: »\nobreak{}\oindex{Bad Ischl@\textbf{Bad Ischl}|pwk}{\pb}Ischl, 13 7 94, 10 F\nobreak{}«. }\buchAbdrucke{\weitereDrucke{Arthur Schnitzler, Richard Beer-Hofmann: \emph{Briefwechsel 1891–1931}. Hg. Konstanze Fliedl. Wien, Zürich: \emph{Europaverlag} 1992, S. 57.} }\toendnotes[C]{\smallbreak}\pstart{}{\pb}Hrn \textsc{Dr. Rich.
                     Beer-Hofmann}\pend{}\pstart{}\textsc{Ischl\oindex{Bad Ischl@\textbf{Bad Ischl}|pw}}\pend{}\pstart{}\textsc{Egelmoos} 22\oindex{Eglmoosgasse@\textbf{Eglmoosgasse}|pw}.\pend{}{\bigskip}\pstart
           \raggedleft{}{\pb}Donnerſtag\pend
           \pstart{}Lieber Richard, \pend\pstart
           Samſtag ko{\geminationm}en die Cigaretten, und die 2 Bände {\pb}\textsc{Becque}\pwindex{Becque, Henri 09.04.1837 – 12.05.1899@\textsc{Becque, Henri} (09.04.1837 – 12.05.1899), \emph{Schriftsteller}|pw}\pwindex{Becque, Henri 09.04.1837 – 12.05.1899@\textsc{Becque, Henri} (09.04.1837 – 12.05.1899), \emph{Schriftsteller}!Theâtre complet1889 – 1899@\strich\emph{Theâtre complet} {[}1889 – 1899{]}|pwv}.\pend
           \pstart Herzlich Ihr \spacefill\mbox{Arthur.}\pend{}\pstart
           Auch ich komme Samſtag. –\pend
           \endnumbering\briefempfaengerindex{Beer-Hofmann, Richard@\textsc{Beer-Hofmann, Richard}!zzzSchnitzler, Arthur@\emph{von Arthur Schnitzler}!1894-07-121@{12. 7. 1894}|)be}\mylabel{h}\end{ledgroupsized}  \newcommand{\dateiname}{L00350}\newcommand{\titel}{Arthur Schnitzler an Richard Beer-Hofmann, 12. 7. 1894}\newcommand{\editorInnen}{Martin Anton Müller und Gerd-Hermann Susen}\input{../tex-inputs/latex-pdf-abspann}
      