%% latex-leseansicht-vorspann.tex
%% Vorspann für die Leseansicht.
%% Lädt die gemeinsame Datei latex-vorspann.tex mit nicht gesetztem Schalter.

\newif\ifkorrekturansicht
\korrekturansichtfalse

\input{../tex-inputs/latex-vorspann}


         
         \renewcommand{\erwaehntePersonen}{Personen: Henry Becque, Richard Beer-Hofmann}
         \renewcommand{\erwaehnteOrte}{Orte: Bad Ischl, Eglmoosgasse, IX., Alsergrund, Wien}
         \renewcommand{\erwaehnteWerke}{Werke: Theâtre complet}
               \section[Arthur Schnitzler an Richard Beer-Hofmann, 12. 7. 1894]{ Arthur Schnitzler an Richard Beer-Hofmann, 12. 7. 1894}\nopagebreak\mylabel{v}\rehead{ }\begin{ledgroupsized}[t]{13cm}\normalsize\beginnumbering \toendnotes[C]{\smallbreak\pagebreak[2]} \Standort{YCGL, MSS 31.}
\physDesc{Brief, 1 Blatt, 2 Seiten, Umschlag, 159 Zeichen
\newline{}Handschrift: Bleistift, deutsche Kurrent
\newline{}Versand: 1) Stempel: »\nobreak{}\oindex{IX., Alsergrund@\textbf{IX., Alsergrund}|pwk}Wien 9/3 72, 12 7 {[}94{]}, 6–7N\nobreak{}«.   2) Stempel: »\nobreak{}\oindex{Bad Ischl@\textbf{Bad Ischl}|pwk}{\pb}Ischl, 13 7 94, 10 F\nobreak{}«. }\buchAbdrucke{\weitereDrucke{Arthur Schnitzler, Richard Beer-Hofmann: \emph{Briefwechsel 1891–1931}. Hg. Konstanze Fliedl. Wien, Zürich: \emph{Europaverlag} 1992, S. 57.} }\toendnotes[C]{\smallbreak}\pstart{}{\pb}Hrn \textsc{Dr. Rich.
                     Beer-Hofmann}\pend{}\pstart{}\textsc{Ischl\oindex{Bad Ischl@\textbf{Bad Ischl}|pw}}\pend{}\pstart{}\textsc{Egelmoos} 22\oindex{Eglmoosgasse@\textbf{Eglmoosgasse}|pw}.\pend{}{\bigskip}\pstart
           \raggedleft{}{\pb}Donnerſtag\pend
           \pstart{}Lieber Richard, \pend\pstart
           Samſtag ko{\geminationm}en die Cigaretten, und die 2 Bände {\pb}\textsc{Becque}\pwindex{Becque, Henry 1837-04-09 – 1899-05-12@\textsc{Becque, Henry} (1837-04-09 – 1899-05-12), \emph{Schriftsteller, Schriftsteller}|pw}\pwindex{Becque, Henry 1837-04-09 – 1899-05-12@\textsc{Becque, Henry} (1837-04-09 – 1899-05-12), \emph{Schriftsteller, Schriftsteller}!Theâtre complet1889 – 1899@\strich\emph{Theâtre complet} {[}1889 – 1899{]}|pwv}.\pend
           \pstart Herzlich Ihr \spacefill\mbox{Arthur.}\pend{}\pstart
           Auch ich komme Samſtag. –\pend
           
         
         \endnumbering\mylabel{h}\end{ledgroupsized}  \newcommand{\dateiname}{L00350}\newcommand{\titel}{Arthur Schnitzler an Richard Beer-Hofmann, 12. 7. 1894}\newcommand{\editorInnen}{Martin Anton Müller und Gerd-Hermann Susen}%% latex-leseansicht-abspann.tex
%% Abspann für die Leseansicht.
%% Der Schalter \ifkorrekturansicht ist bereits durch den Vorspann gesetzt.

%% latex-abspann.tex
%% Gemeinsamer Abspann für Korrekturansicht und Leseansicht.
%% Setzt den Schalter \ifkorrekturansicht voraus (gesetzt in den
%% einbindenden Dateien latex-korrekturansicht-abspann.tex bzw.
%% latex-leseansicht-abspann.tex).
%% ---------------------------------------------------------------

\normalsize

% Das esempio-Environment wird nur in der Leseansicht benötigt
\ifkorrekturansicht\else
\newenvironment{esempio}[3]%
{
    \vspace{1.5ex}
    \rlap{\underline{#1}}
    \par
    \setlength{\parindent}{0cm}
    \nopagebreak
    \leftskip=#2cm
    \rightskip=#3cm
}
{
    \par
}
\fi

\doendnotes{C}
\bigskip
\vfill

\clearpage

\footnotesize

\ifkorrekturansicht
  \lohead{\textsc{register}}
\fi

% theindex-Environment neu definieren ohne reledmac
\makeatletter
\renewenvironment{theindex}{%
  \ifkorrekturansicht
    \section*{\indexname}%
  \else
    \subsubsection*{Index der erwähnten Entitäten}%
  \fi
  \setlength{\parindent}{0pt}%
  \setlength{\parskip}{0pt plus 0.3pt}%
  \let\item\@idxitem
}{%
  \ifkorrekturansicht\clearpage\fi
}
\makeatother

\IfFileExists{\jobname-pw.ind}{\input{\jobname-pw.ind}}{}

% Quellenangabe nur in der Leseansicht
\ifkorrekturansicht\else
% Fallback-Definitionen, falls die .tex-Datei \titel etc. nicht gesetzt hat
\providecommand{\titel}{}
\providecommand{\editorInnen}{}
\providecommand{\dateiname}{\jobname}

\vspace{3cm}

\vfill

\footnotesize
\textsc{Quelle}: \titel. Herausgegeben von {\editorInnen}. In: \emph{Arthur Schnitzler: Briefwechsel mit Autorinnen und Autoren}.
 Digitale Edition, https://schnitzler-briefe.acdh.oeaw.ac.at/{\dateiname}.html (Stand \today)
\fi

\end{document}


      