%% latex-korrekturansicht-vorspann.tex
%% Vorspann für die Korrekturansicht.
%% Lädt die gemeinsame Datei latex-vorspann.tex mit gesetztem Schalter.

\newif\ifkorrekturansicht
\korrekturansichttrue

\input{../tex-inputs/latex-vorspann}


\section[Thomas Mann an Arthur Schnitzler, 22. 11. 1910]{L01986 Thomas Mann an Arthur Schnitzler, 22. 11. 1910}
\nopagebreak\mylabel{L01986v}
\rehead{ }\normalsize\beginnumbering\briefempfaengerindex{Schnitzler, Arthur@\textsc{Schnitzler, Arthur}!zzzMann, Thomas@\emph{von Thomas Mann}!1910-11-223@{22. 11. 1910}|(be}
\toendnotes[C]{\smallbreak\pagebreak[2]}\Standort{CUL, Schnitzler, B 67.}
\physDesc{Brief, 1 Blatt, 2 Seiten, 464 Zeichen
\newline{}Handschrift: schwarze Tinte, deutsche Kurrent
\newline{}Schnitzler: mit Bleistift beschriftet: »\textsc{Mann}« }
\buchAbdrucke{\weitereDrucke{\emph{Modern Austrian Literature}, Jg. 7 (1974) Nr. 1/2, S. 14.} }\toendnotes[C]{\smallbreak}
\pstart
           \raggedleft{}{\pb}\textcolor{gray}{\textbf{\textsc{München}\oindex{Muenchen@\textbf{München}, \emph{P.PPLA}|pw}\textsc{, den}}}{ }22. XI. 1910.\pend
           
\pstart
           \raggedleft{}\textcolor{gray}{\textbf{FRANZ JOSEPH-STRASSE 2\oindex{Franz-Joseph-Strasse@\textbf{Franz-Joseph-Straße}, \emph{Straße (K.STR)}|pw}.}}\pend
           
\pstart{}Sehr verehrter Herr Doctor:\pend\vspace{0.5em}
\pstart
           Der Verlag S. Fiſcher\orgindex{S. Fischer Verlag@S. Fischer Verlag|pw}{ }ſendet mir in Ihrem gütigen Auftrage Ihr neues Werk\pwindex{junge Medardus. Dramatische Historie in einem Vorspiel und fuenf Aufzuegen@\emph{Der junge Medardus. Dramatische Historie in einem Vorspiel und fünf Aufzügen}|pwv}. Ich brauche Ihnen nicht
               zu ſagen, mit welcher Freude ich es in Empfang genommen habe. Das \label{K_L01986-1v}\edtext{Bruchstück\pwindex{Vorspiel zu einem Drama »Der junge Medardus«@\emph{Vorspiel zu einem Drama »Der junge Medardus«}|pwv}}{\lemma{\textnormal{\emph{Bruchstück}}}\Cendnote{\textnormal{Arthur Schnitzler: \emph{Vorspiel zu einem Drama »Der junge Medardus«}\pwindex{Vorspiel zu einem Drama »Der junge Medardus«@\emph{Vorspiel zu einem Drama »Der junge Medardus«}|pwk}. In: \emph{Die neue Rundschau}\pwindex{neue Rundschau@\emph{Die neue Rundschau}|pwk}, Jg. 21, H. 10,
                        1. 10. 1910, S. 1385–1415.}}}\label{K_L01986-1}, das Sie in der Neuen Rundſchau\pwindex{neue Rundschau@\emph{Die neue Rundschau}|pw} daraus veroeffentlichten, kannte
               ich ſchon. Nun {\pb}iſt es mir ein Bedürfnis,
               Ihnen aus der Lektüre des kunſt- und lebensvollen Ganzen heraus, meinen herzlichen
               Dank und Glückwunſch darzubringen.\pend
           
\pstart
           Ihr ſehr ergebener{\\[\baselineskip]}\spacefill\mbox{Thomas Mann.}\pend
           \leftskip=0em{}\selectlanguage{ngerman}\endnumbering\briefempfaengerindex{Schnitzler, Arthur@\textsc{Schnitzler, Arthur}!zzzMann, Thomas@\emph{von Thomas Mann}!1910-11-223@{22. 11. 1910}|)be}\mylabel{L01986h}  \normalsize

\doendnotes{C}
\bigskip
\vfill

\clearpage

\footnotesize

\lohead{\textsc{register}}

% Definiere theindex-Environment komplett neu ohne reledmac
\makeatletter
\renewenvironment{theindex}{%
  \section*{\indexname}%
  \setlength{\parindent}{0pt}%
  \setlength{\parskip}{0pt plus 0.3pt}%
  \let\item\@idxitem
}{%
  \clearpage
}
\makeatother

\IfFileExists{\jobname-pw.ind}{\input{\jobname-pw.ind}}{}

\end{document}

      