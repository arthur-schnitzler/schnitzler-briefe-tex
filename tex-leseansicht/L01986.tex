%% latex-leseansicht-vorspann.tex
%% Vorspann für die Leseansicht.
%% Lädt die gemeinsame Datei latex-vorspann.tex mit nicht gesetztem Schalter.

\newif\ifkorrekturansicht
\korrekturansichtfalse

\input{../tex-inputs/latex-vorspann}


         \renewcommand{\erwaehnteInstitutionen}{Institutionen: S. Fischer Verlag}
         \renewcommand{\erwaehnteOrte}{Orte: Franz-Joseph-Straße, München, Wien}
         \renewcommand{\erwaehnteWerke}{Werke: Der junge Medardus. Dramatische Historie in einem Vorspiel und fünf Aufzügen, Die neue Rundschau, Vorspiel zu einem Drama »Der junge Medardus«}
               \section[Thomas Mann an Arthur Schnitzler, 22. 11. 1910]{ Thomas Mann an Arthur Schnitzler, 22. 11. 1910}\nopagebreak\mylabel{v}\rehead{ }\begin{ledgroupsized}[t]{13cm}\normalsize\beginnumbering \toendnotes[C]{\smallbreak\pagebreak[2]} \Standort{CUL, Schnitzler, B 67.}
\physDesc{Brief, 1 Blatt, 2 Seiten, 464 Zeichen
\newline{}Handschrift: schwarze Tinte, deutsche Kurrent
\newline{}Schnitzler: mit Bleistift beschriftet: »\textsc{Mann}« }\buchAbdrucke{\weitereDrucke{Hertha Krotkoff: \emph{Arthur Schnitzler – Thomas Mann: Briefe.} In: \emph{Modern Austrian Literature}, Jg. 7 (1974) Nr. 1/2, S. 14.} }\toendnotes[C]{\smallbreak}\pstart
           \noindent{}\raggedleft{}{\pb}\textcolor{gray}{\textbf{\textsc{München}\oindex{Muenchen@\textbf{München}|pw}\textsc{, den}}}{ }22. XI. 1910.\pend
           \pstart
           \noindent{}\raggedleft{}\textcolor{gray}{\textbf{FRANZ JOSEPH-STRASSE 2\oindex{Franz-Joseph-Strasse@\textbf{Franz-Joseph-Straße}|pw}.}}\pend
           \pstart{}Sehr verehrter Herr Doctor:\pend\pstart
           Der Verlag S. Fiſcher\orgindex{S. Fischer Verlag@S. Fischer Verlag|pw}{ }ſendet mir in Ihrem gütigen Auftrage Ihr neues Werk\pwindex{Schnitzler, Arthur 15.05.1862 – 21.10.1931@\textsc{Schnitzler, Arthur} (15.05.1862 – 21.10.1931), \emph{Schriftsteller, Mediziner}!junge Medardus. Dramatische Historie in einem Vorspiel und fuenf
                  Aufzuegen1910-10-26@\strich\emph{Der junge Medardus. Dramatische Historie in einem Vorspiel und fünf Aufzügen} {[}1910-10-26{]}|pwv}. Ich brauche Ihnen nicht
               zu ſagen, mit welcher Freude ich es in Empfang genommen habe. Das \label{K_L01986_1v}\edtext{Bruchstück\pwindex{Schnitzler, Arthur 15.05.1862 – 21.10.1931@\textsc{Schnitzler, Arthur} (15.05.1862 – 21.10.1931), \emph{Schriftsteller, Mediziner}!Vorspiel zu einem Drama »Der junge Medardus«01. 10. 1910@\strich\emph{Vorspiel zu einem Drama »Der junge Medardus«} {[}01. 10. 1910{]}|pwv}}{\lemma{\textnormal{\emph{Bruchstück}}}\Cendnote{\textnormal{Arthur Schnitzler\pwindex{Schnitzler, Arthur 15.05.1862 – 21.10.1931@\textsc{Schnitzler, Arthur} (15.05.1862 – 21.10.1931), \emph{Schriftsteller, Mediziner}|pwk}: \emph{Vorspiel zu einem Drama »Der junge Medardus«}\pwindex{Schnitzler, Arthur 15.05.1862 – 21.10.1931@\textsc{Schnitzler, Arthur} (15.05.1862 – 21.10.1931), \emph{Schriftsteller, Mediziner}!Vorspiel zu einem Drama »Der junge Medardus«01. 10. 1910@\strich\emph{Vorspiel zu einem Drama »Der junge Medardus«} {[}01. 10. 1910{]}|pwk}. In: \emph{Die neue Rundschau}\pwindex{?? Werk@Nicht ermittelte Verfasserinnen und Verfasser!neue Rundschau1904@\emph{Die neue Rundschau} {[}1904{]}|pwk}, Jg. 21, H. 10,
                        1. 10. 1910, S. 1385–1415.}}}\label{K_L01986_1h}, das Sie in der Neuen Rundſchau\pwindex{?? Werk@Nicht ermittelte Verfasserinnen und Verfasser!neue Rundschau1904@\emph{Die neue Rundschau} {[}1904{]}|pw} daraus veroeffentlichten, kannte
               ich ſchon. Nun {\pb}iſt es mir ein Bedürfnis,
               Ihnen aus der Lektüre des kunſt- und lebensvollen Ganzen heraus, meinen herzlichen
               Dank und Glückwunſch darzubringen.\pend
           \pstart
           Ihr ſehr ergebener{\\[\baselineskip]}\spacefill\mbox{Thomas Mann.}\pend
           \leftskip=0em{}
         
         \endnumbering\mylabel{h}\end{ledgroupsized}  \newcommand{\dateiname}{L01986}\newcommand{\titel}{Thomas Mann an Arthur Schnitzler, 22. 11. 1910}\newcommand{\editorInnen}{Martin Anton Müller und Gerd-Hermann Susen}%% latex-leseansicht-abspann.tex
%% Abspann für die Leseansicht.
%% Der Schalter \ifkorrekturansicht ist bereits durch den Vorspann gesetzt.

%% latex-abspann.tex
%% Gemeinsamer Abspann für Korrekturansicht und Leseansicht.
%% Setzt den Schalter \ifkorrekturansicht voraus (gesetzt in den
%% einbindenden Dateien latex-korrekturansicht-abspann.tex bzw.
%% latex-leseansicht-abspann.tex).
%% ---------------------------------------------------------------

\normalsize

% Das esempio-Environment wird nur in der Leseansicht benötigt
\ifkorrekturansicht\else
\newenvironment{esempio}[3]%
{
    \vspace{1.5ex}
    \rlap{\underline{#1}}
    \par
    \setlength{\parindent}{0cm}
    \nopagebreak
    \leftskip=#2cm
    \rightskip=#3cm
}
{
    \par
}
\fi

\doendnotes{C}
\bigskip
\vfill

\clearpage

\footnotesize

\ifkorrekturansicht
  \lohead{\textsc{register}}
\fi

% theindex-Environment neu definieren ohne reledmac
\makeatletter
\renewenvironment{theindex}{%
  \ifkorrekturansicht
    \section*{\indexname}%
  \else
    \subsubsection*{Index der erwähnten Entitäten}%
  \fi
  \setlength{\parindent}{0pt}%
  \setlength{\parskip}{0pt plus 0.3pt}%
  \let\item\@idxitem
}{%
  \ifkorrekturansicht\clearpage\fi
}
\makeatother

\IfFileExists{\jobname-pw.ind}{\input{\jobname-pw.ind}}{}

% Quellenangabe nur in der Leseansicht
\ifkorrekturansicht\else
% Fallback-Definitionen, falls die .tex-Datei \titel etc. nicht gesetzt hat
\providecommand{\titel}{}
\providecommand{\editorInnen}{}
\providecommand{\dateiname}{\jobname}

\vspace{3cm}

\vfill

\footnotesize
\textsc{Quelle}: \titel. Herausgegeben von {\editorInnen}. In: \emph{Arthur Schnitzler: Briefwechsel mit Autorinnen und Autoren}.
 Digitale Edition, https://schnitzler-briefe.acdh.oeaw.ac.at/{\dateiname}.html (Stand \today)
\fi

\end{document}


      