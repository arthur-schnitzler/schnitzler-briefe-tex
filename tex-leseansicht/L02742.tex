%% latex-leseansicht-vorspann.tex
%% Vorspann für die Leseansicht.
%% Lädt die gemeinsame Datei latex-vorspann.tex mit nicht gesetztem Schalter.

\newif\ifkorrekturansicht
\korrekturansichtfalse

\input{../tex-inputs/latex-vorspann}


         
         \renewcommand{\erwaehntePersonen}{Personen: Lou Andreas-Salomé, Richard Beer-Hofmann, Theodor Herzl, Hugo von Hofmannsthal, Leopold Sonnemann}
         \renewcommand{\erwaehnteInstitutionen}{Institutionen: Frankfurter Zeitung, Houghton Library}
         \renewcommand{\erwaehnteOrte}{Orte: Bad Aussee, Bad Ischl, Bad Tölz, Bayern, Hotel Marienbad, München, Paris, Salzburg, Straßburg, Ungarn, rue Feydeau}
         \renewcommand{\erwaehnteWerke}{Werke: Freiwild. Schauspiel in 3 Akten, Liebelei. Schauspiel in drei Akten, Tagebuch}
               \section[Paul Goldmann an Arthur Schnitzler, 29. 7. {[}1895{]}]{ Paul Goldmann an Arthur Schnitzler, 29. 7. {[}1895{]}}\nopagebreak\mylabel{v}\rehead{ }\begin{ledgroupsized}[t]{13cm}\normalsize\beginnumbering \toendnotes[C]{\smallbreak\pagebreak[2]} \Standort{DLA, A:Schnitzler, HS.NZ85.1.3165.}
\physDesc{Brief, 2 Blätter, 6 Seiten
\newline{}Handschrift: schwarze Tinte, deutsche Kurrent
\newline{}Schnitzler: 1) mit Bleistift das Jahr »95« vermerkt  2) mit rotem Buntstift drei Unterstreichungen}\toendnotes[C]{\smallbreak}\pstart
           \noindent{}{\pb}\textcolor{gray}{\textbf{\textbf{Frankfurter Zeitung\orgindex{Frankfurter Zeitung@Frankfurter Zeitung|pw}}}}\pend
           \pstart
           \textcolor{gray}{\textbf{(\begin{otherlanguage}{french}Gazette de Francfort\end{otherlanguage}\orgindex{Frankfurter Zeitung@Frankfurter Zeitung|pw}). }}\pend
           \pstart
           \textcolor{gray}{\textbf{\textbf{\begin{otherlanguage}{french}Fondateur M. L.
                              Sonnemann\pwindex{Sonnemann, Leopold 1831-10-29 – 1909-10-30@\textsc{Sonnemann, Leopold} (1831-10-29 – 1909-10-30), \emph{Journalist, Herausgeber}|pw}\end{otherlanguage}.}}}\pend
           \pstart
           \begin{otherlanguage}{french}\textcolor{gray}{\textbf{Journal politique, financier,}}\end{otherlanguage}\pend
           \pstart
           \begin{otherlanguage}{french}\textcolor{gray}{\textbf{commercial et littéraire.}}\end{otherlanguage}\pend
           \pstart
           \begin{otherlanguage}{french}\textcolor{gray}{\textbf{\textbf{Paraissant trois fois par jour.}}}\end{otherlanguage}\pend
           \pstart
           \begin{otherlanguage}{french}\textcolor{gray}{\textbf{\textbf{Bureau à Paris\oindex{Paris@\textbf{Paris}|pw}}}}\end{otherlanguage}\hfill \textsc{Paris\oindex{Paris@\textbf{Paris}|pw}}, 29. Juli.\pend
           \pstart
           \begin{otherlanguage}{french}\textcolor{gray}{\textbf{\textbf{24. Rue Feydeau\oindex{rue Feydeau@\textbf{rue Feydeau}|pw}.}}}\end{otherlanguage}\pend
           \pstart\center{}Mein lieber Freund,\pend\pstart
           Vielen Dank für Deinen lieben Brief! \pend
           \pstart
           Mittwoch od. Donnerſtag
               fahre ich von hier fort, gedenke einen Tag in \strikeout{\textsc{S\textcolor{gray}{t}ras}}{ }\textsc{Strassburg\oindex{Strassburg@\textbf{Straßburg}|pw}} mich aufzuhalten, dann zwei oder drei Tage in \textsc{Muenchen\oindex{Muenchen@\textbf{München}|pw}}, wo ich im »\textsc{Hotel Marienbad\oindex{Hotel Marienbad@\textbf{Hotel Marienbad}|pw}}« wohnen werde (dies für etwaige Nachrichten). Dann nach \textsc{Toelz\oindex{Bad Toelz@\textbf{Bad Tölz}|pw}}. Ich habe diesmal {\pb}fünf bis ſechs Wochen
               Urlaub. Wenns der Arzt verlangt, ſo muß ich ſie natürlich ganz auf die Kur verwenden.
               Sollten vier Wochen genügen, ſo möchte ich gern – falls ich noch Geld habe – ſo etwa
               acht Tage irgendwo in der Welt mit Euch zuſammenſein. Jedenfalls ſehe ich mit Freude,
               daß ich Ausſicht habe, Dich ſchon vorher zu ſehen. Mein Wunſch iſt nur, daß es
               möglichſt lange wäre. Nachrichten erreichen mich {\pb}\uline{nach}{ }\textsc{Muenchen\oindex{Muenchen@\textbf{München}|pw}} zunächſt \textsc{Toelz\oindex{Bad Toelz@\textbf{Bad Tölz}|pw}} (\textsc{Baiern\oindex{Bayern@\textbf{Bayern}|pw}}) \textsc{Poste-restante}. Kommt die Frau \label{K_L02742-1v}\edtext{\textsc{Andreas\pwindex{Andreas-Salome, Lou 12.02.1861 – 05.02.1937@\textsc{Andreas-Salomé, Lou} (12.02.1861 – 05.02.1937), \emph{Schriftstellerin}|pw}} nach \textsc{Salzburg\oindex{Salzburg@\textbf{Salzburg}|pw}}}{\lemma{\textnormal{\emph{Andreas nach Salzburg}}}\Cendnote{\textnormal{siehe die \emph{Tagebuch}\pwindex{Schnitzler, Arthur 15.05.1862 – 21.10.1931@\textsc{Schnitzler, Arthur} (15.05.1862 – 21.10.1931), \emph{Schriftsteller, Mediziner}!Tagebuch1981 – 2000@\strich\emph{Tagebuch} {[}1981 – 2000{]}|pwk}-Einträge zwischen 20. 8. 1895 und 6. 9. 1895}}}\label{K_L02742-1h}, ſo gehe ich vielleicht auch
               hinüber. Was Du \textsc{Richard\pwindex{Beer-Hofmann, Richard 1866-07-11 – 1945-09-26@\textsc{Beer-Hofmann, Richard} (1866-07-11 – 1945-09-26), \emph{Schriftsteller}|pw}}{ }\label{K_L02742-2v}\edtext{ſagen ſollſt}{\lemma{\textnormal{\emph{ſagen ſollſt}}}\Cendnote{\textnormal{wohl im Hinblick auf die frühere Beziehung Paul Goldmann\pwindex{Goldmann, Paul 31.01.1865 – 25.09.1935@\textsc{Goldmann, Paul} (31.01.1865 – 25.09.1935), \emph{Schriftsteller, Journalist}|pwk}s zu Lou
                     Andreas-Salomé\pwindex{Andreas-Salome, Lou 12.02.1861 – 05.02.1937@\textsc{Andreas-Salomé, Lou} (12.02.1861 – 05.02.1937), \emph{Schriftstellerin}|pwk} zu verstehen, mit der Richard Beer-Hofmann\pwindex{Beer-Hofmann, Richard 1866-07-11 – 1945-09-26@\textsc{Beer-Hofmann, Richard} (1866-07-11 – 1945-09-26), \emph{Schriftsteller}|pwk} seit wenigen Wochen intim war}}}\label{K_L02742-2h}, weiß ich
               nicht. Ich gebe Dir Vollmacht, zu ſagen, was Du willſt. Mir widerſtrebt es, ihn
               anzulügen. Ich danke Dir für die Mittheilung deſſen, was \label{K_L02742-3v}\edtext{\textsc{Loris\pwindex{Hofmannsthal, Hugo von 1874-02-01 – 1929-07-15@\textsc{Hofmannsthal, Hugo von} (1874-02-01 – 1929-07-15), \emph{Schriftsteller}|pw}}}{\lemma{\textnormal{\emph{Loris}}}\Cendnote{\textnormal{Schnitzler\pwindex{Schnitzler, Arthur 15.05.1862 – 21.10.1931@\textsc{Schnitzler, Arthur} (15.05.1862 – 21.10.1931), \emph{Schriftsteller, Mediziner}|pwk} dürfte Goldmann\pwindex{Goldmann, Paul 31.01.1865 – 25.09.1935@\textsc{Goldmann, Paul} (31.01.1865 – 25.09.1935), \emph{Schriftsteller, Journalist}|pwk} aus Hugo von
                     Hofmannsthal\pwindex{Hofmannsthal, Hugo von 1874-02-01 – 1929-07-15@\textsc{Hofmannsthal, Hugo von} (1874-02-01 – 1929-07-15), \emph{Schriftsteller}|pwk}s Brief vom 17. [7. 1895] zitiert haben, in dem dieser schrieb: »Als ein
                     besonders merkwürdiger Tag erscheint mir der, wo wir mit Goldmann\pwindex{Goldmann, Paul 31.01.1865 – 25.09.1935@\textsc{Goldmann, Paul} (31.01.1865 – 25.09.1935), \emph{Schriftsteller, Journalist}|pw}{ }{[}{\dots}{]} waren und dann ein großes Gewitter gekommen ist. Ich kann aber
                     nicht finden, warum.«}}}\label{K_L02742-3h} geſchrieben. Es iſt ſehr hübſch, – nur
               weiß man nicht recht, was eigentlich an der Sache merkwürdig war, {\pb}\textsc{Goldmann} oder das \strikeout{Gew\textcolor{gray}{i}tte\textcolor{gray}{r}} Gewitter? {\dotsfour}\pend
           \pstart
           \textsc{Herzl\pwindex{Herzl, Theodor 1860-05-02 – 1904-07-03@\textsc{Herzl, Theodor} (1860-05-02 – 1904-07-03), \emph{Schriftsteller, Journalist}|pw}} iſt vorgeſtern nach \textsc{Aussee\oindex{Bad Aussee@\textbf{Bad Aussee}|pw}} abgereiſt. Ich bin innnerlich ganz fertig mit ihm. Äußerlich hält es nur noch
               durch ein paar recht lockere Fäden zuſammen. Der \label{K_L02742-55v}\edtext{ungar\oindex{Ungarn@\textbf{Ungarn}|pwv}iſche Saujud}{\lemma{\textnormal{\emph{ungariſche Saujud}}}\Cendnote{\textnormal{Herzl\pwindex{Herzl, Theodor 1860-05-02 – 1904-07-03@\textsc{Herzl, Theodor} (1860-05-02 – 1904-07-03), \emph{Schriftsteller, Journalist}|pwk}s zunehmende Neuorientierung vom
                  literarischen Schriftsteller zum Zionisten wird hier durch Goldmann\pwindex{Goldmann, Paul 31.01.1865 – 25.09.1935@\textsc{Goldmann, Paul} (31.01.1865 – 25.09.1935), \emph{Schriftsteller, Journalist}|pwk} mit einer überraschend groben Ausdrucksweise
                  kommentiert. Dies dürfte als Hinweis zu lesen sein, dass Goldmann\pwindex{Goldmann, Paul 31.01.1865 – 25.09.1935@\textsc{Goldmann, Paul} (31.01.1865 – 25.09.1935), \emph{Schriftsteller, Journalist}|pwk} den richtigen Umgang mit der jüdischen Kultur in
                  der Assimilation sah, während Herzl\pwindex{Herzl, Theodor 1860-05-02 – 1904-07-03@\textsc{Herzl, Theodor} (1860-05-02 – 1904-07-03), \emph{Schriftsteller, Journalist}|pwk} das
                  verarmte Judentum aus dem Osten der k. k. Monarchie nicht nur nicht ablehnte,
                  sondern sich dafür begeisterte.}}}\label{K_L02742-55h} kommt immer deutlicher \strikeout{\textcolor{gray}{ut}} unter dem Literaten hervor, und das wird unerträglich. Ich glaube es wächſt
               ein \strikeout{ſold} ſolider Haß heran zwiſchen ihm u. mir.\pend
           \pstart
           Was geht mit Deinem Stücke\pwindex{Schnitzler, Arthur 15.05.1862 – 21.10.1931@\textsc{Schnitzler, Arthur} (15.05.1862 – 21.10.1931), \emph{Schriftsteller, Mediziner}!Liebelei. Schauspiel in drei Akten1895-10-09@\strich\emph{Liebelei. Schauspiel in drei Akten} {[}1895-10-09{]}|pwuv} vor, daß Du ſo reſignirt über das {\pb}Warten auf Erfolg ſprichſt? Nun, ich höre es ja nächſtens wohl mündlich. Gewiß, Du
               ſollſt den Erfolg nicht erwarten. Laß’ \strikeout{D} das nur
               gehn, das thue ich ſchon für Dich.\pend
           \pstart
           Daß Du »Freiwild\pwindex{Schnitzler, Arthur 15.05.1862 – 21.10.1931@\textsc{Schnitzler, Arthur} (15.05.1862 – 21.10.1931), \emph{Schriftsteller, Mediziner}!Freiwild. Schauspiel in 3 Akten1896@\strich\emph{Freiwild. Schauspiel in 3 Akten} {[}1896{]}|pw}« ſchreibſt, freut mich ſehr. Du
               haſt Recht: die Arbeit iſt bei dem Allen das Schönſte. Oh, wer arbeiten könnte, \substVorne{}\textsuperscript{!}\substDazwischen{},\substHinten{} wie Du! Alles gute Glück {\pb}zum Werke\pwindex{Schnitzler, Arthur 15.05.1862 – 21.10.1931@\textsc{Schnitzler, Arthur} (15.05.1862 – 21.10.1931), \emph{Schriftsteller, Mediziner}!Freiwild. Schauspiel in 3 Akten1896@\strich\emph{Freiwild. Schauspiel in 3 Akten} {[}1896{]}|pwv}! {\dotsfour}\pend
           \pstart
           Grüß’ Dich Gott, mein lieber Freund. Nun wird man ſich bald ſehen. Wie ich mich
                  freue!! {\dotstwo}\pend
           \pstart
           Dein treuer {\\[\baselineskip]}\spacefill\mbox{Paul Goldmann {\dotstwo}}\pend
           \leftskip=0em{}\pstart
           \noindent{}Ich weiß \textsc{Richard\pwindex{Beer-Hofmann, Richard 1866-07-11 – 1945-09-26@\textsc{Beer-Hofmann, Richard} (1866-07-11 – 1945-09-26), \emph{Schriftsteller}|pw}s} Adreſſe nicht. Bitte,
                     gib\textcolor{gray}{’} ihm inliegenden \label{K_L02742-4v}\edtext{Brief}{\lemma{\textnormal{\emph{Brief}}}\Cendnote{\textnormal{Der
                     sechsseitige Brief, datiert vom 29. 7. {[}1895{]}, ist im Nachlass Beer-Hofmann\pwindex{Beer-Hofmann, Richard 1866-07-11 – 1945-09-26@\textsc{Beer-Hofmann, Richard} (1866-07-11 – 1945-09-26), \emph{Schriftsteller}|pwk}s in der \emph{Houghton Library}\orgindex{Houghton Library@Houghton Library|pwk}, Harvard
                        (Signatur 825.978), überliefert. Goldmann\pwindex{Goldmann, Paul 31.01.1865 – 25.09.1935@\textsc{Goldmann, Paul} (31.01.1865 – 25.09.1935), \emph{Schriftsteller, Journalist}|pwk} bedankt sich für Fotografien, eine von Beer-Hofmann\pwindex{Beer-Hofmann, Richard 1866-07-11 – 1945-09-26@\textsc{Beer-Hofmann, Richard} (1866-07-11 – 1945-09-26), \emph{Schriftsteller}|pwk}, die andere von dessen Hund »Flirt«. Goldmann\pwindex{Goldmann, Paul 31.01.1865 – 25.09.1935@\textsc{Goldmann, Paul} (31.01.1865 – 25.09.1935), \emph{Schriftsteller, Journalist}|pwk} berichtet von seinem eigenen
                     Pudel und freut sich auf das bevorstehende Wiedersehen.}}}\label{K_L02742-4h}.\pend
           
         
         \endnumbering\mylabel{h}\end{ledgroupsized}  \newcommand{\dateiname}{L02742}\newcommand{\titel}{Paul Goldmann an Arthur Schnitzler, 29. 7. [1895]}\newcommand{\editorInnen}{Martin Anton Müller und Laura Untner}%% latex-leseansicht-abspann.tex
%% Abspann für die Leseansicht.
%% Der Schalter \ifkorrekturansicht ist bereits durch den Vorspann gesetzt.

%% latex-abspann.tex
%% Gemeinsamer Abspann für Korrekturansicht und Leseansicht.
%% Setzt den Schalter \ifkorrekturansicht voraus (gesetzt in den
%% einbindenden Dateien latex-korrekturansicht-abspann.tex bzw.
%% latex-leseansicht-abspann.tex).
%% ---------------------------------------------------------------

\normalsize

% Das esempio-Environment wird nur in der Leseansicht benötigt
\ifkorrekturansicht\else
\newenvironment{esempio}[3]%
{
    \vspace{1.5ex}
    \rlap{\underline{#1}}
    \par
    \setlength{\parindent}{0cm}
    \nopagebreak
    \leftskip=#2cm
    \rightskip=#3cm
}
{
    \par
}
\fi

\doendnotes{C}
\bigskip
\vfill

\clearpage

\footnotesize

\ifkorrekturansicht
  \lohead{\textsc{register}}
\fi

% theindex-Environment neu definieren ohne reledmac
\makeatletter
\renewenvironment{theindex}{%
  \ifkorrekturansicht
    \section*{\indexname}%
  \else
    \subsubsection*{Index der erwähnten Entitäten}%
  \fi
  \setlength{\parindent}{0pt}%
  \setlength{\parskip}{0pt plus 0.3pt}%
  \let\item\@idxitem
}{%
  \ifkorrekturansicht\clearpage\fi
}
\makeatother

\IfFileExists{\jobname-pw.ind}{\input{\jobname-pw.ind}}{}

% Quellenangabe nur in der Leseansicht
\ifkorrekturansicht\else
% Fallback-Definitionen, falls die .tex-Datei \titel etc. nicht gesetzt hat
\providecommand{\titel}{}
\providecommand{\editorInnen}{}
\providecommand{\dateiname}{\jobname}

\vspace{3cm}

\vfill

\footnotesize
\textsc{Quelle}: \titel. Herausgegeben von {\editorInnen}. In: \emph{Arthur Schnitzler: Briefwechsel mit Autorinnen und Autoren}.
 Digitale Edition, https://schnitzler-briefe.acdh.oeaw.ac.at/{\dateiname}.html (Stand \today)
\fi

\end{document}


      