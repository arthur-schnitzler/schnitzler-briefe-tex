%% latex-leseansicht-vorspann.tex
%% Vorspann für die Leseansicht.
%% Lädt die gemeinsame Datei latex-vorspann.tex mit nicht gesetztem Schalter.

\newif\ifkorrekturansicht
\korrekturansichtfalse

\input{../tex-inputs/latex-vorspann}


\section[Paul Goldmann an Arthur Schnitzler, 29. 7. [1895]]{L02742 Paul Goldmann an Arthur Schnitzler, 29. 7. [1895]}
\nopagebreak\mylabel{L02742v}
\rehead{ }\normalsize\beginnumbering\briefempfaengerindex{Schnitzler, Arthur@\textsc{Schnitzler, Arthur}!zzzGoldmann, Paul@\emph{von Paul Goldmann}!1895-07-291@{29. 7. [1895]}|(be}
\toendnotes[C]{\smallbreak\pagebreak[2]}
\correspDesc{Versand  durch Paul Goldmann am 29. 7. [1895] in Paris
\newline{}Erhalt  durch Arthur Schnitzler im Zeitraum [30. 7. 1895
                  – 3. 8. 1895?] in Bad Ischl}\toendnotes[C]{\smallbreak}
\Standort{DLA, A:Schnitzler, HS.NZ85.1.3165.}
\physDesc{Brief, 2 Blätter, 6 Seiten, 2046 Zeichen
\newline{}Handschrift: schwarze Tinte, deutsche Kurrent
\newline{}Schnitzler: 1) mit Bleistift das Jahr »95« vermerkt  2) mit rotem Buntstift drei Unterstreichungen}\toendnotes[C]{\smallbreak}
\pstart
           {\pb}\textcolor{gray}{\textbf{\textbf{Frankfurter Zeitung\orgindex{Frankfurter Zeitung@Frankfurter Zeitung|pw}}}}\pend
           
\pstart
           \textcolor{gray}{\textbf{(\begin{otherlanguage}{french}Gazette de Francfort\end{otherlanguage}\orgindex{Frankfurter Zeitung@Frankfurter Zeitung|pw}).}}\pend
           
\pstart
           \textcolor{gray}{\textbf{\textbf{\begin{otherlanguage}{french}Fondateur M. L.
                              Sonnemann\pwindex{Sonnemann, Leopold 29.\,10.\,1831 Höchberg – 30.\,10.\,1909 Frankfurt am Main@\textsc{Sonnemann, Leopold} (29.\,10.\,1831 Höchberg – 30.\,10.\,1909 Frankfurt am Main), \emph{Journalist, Herausgeber}|pw}\end{otherlanguage}.}}}\pend
           
\pstart
           \begin{otherlanguage}{french}\textcolor{gray}{\textbf{Journal politique, financier,}}\end{otherlanguage}\pend
           
\pstart
           \begin{otherlanguage}{french}\textcolor{gray}{\textbf{commercial et littéraire.}}\end{otherlanguage}\pend
           
\pstart
           \begin{otherlanguage}{french}\textcolor{gray}{\textbf{\textbf{Paraissant trois fois par jour.}}}\end{otherlanguage}\pend
           
\pstart
           \begin{otherlanguage}{french}\textcolor{gray}{\textbf{\textbf{Bureau à Paris\oindex{Paris@\textbf{Paris}, \emph{Hauptstadt}|pw}}}}\end{otherlanguage}\hfill \textsc{Paris\oindex{Paris@\textbf{Paris}, \emph{Hauptstadt}|pw}}, 29. Juli.\pend
           
\pstart
           \begin{otherlanguage}{french}\textcolor{gray}{\textbf{\textbf{24. Rue Feydeau\oindex{rue Feydeau@\textbf{rue Feydeau}, \emph{Straße}|pw}.}}}\end{otherlanguage}\pend
           
\pstart\center{}Mein lieber Freund,\pend\vspace{0.5em}
\pstart
           Vielen Dank für Deinen lieben Brief!\pend
           
\pstart
           Mittwoch od. Donnerſtag
               fahre ich von hier fort, gedenke einen Tag in \strikeout{\textsc{S\textcolor{gray}{t}ras}}{ }\textsc{Strassburg\oindex{Straßburg@\textbf{Straßburg}|pw}} mich aufzuhalten, dann zwei oder drei Tage in \textsc{Muenchen\oindex{München@\textbf{München}|pw}}, wo ich im »\textsc{Hotel Marienbad\oindex{Hotel Marienbad [München]@\textbf{Hotel Marienbad [München]}, \emph{Hotel}|pw}}« wohnen werde (dies für etwaige Nachrichten). Dann nach \textsc{Toelz\oindex{Bad Tölz@\textbf{Bad Tölz}, \emph{Hauptstadt}|pw}}. Ich habe diesmal {\pb}fünf bis{ }ſechs Wochen
               Urlaub. Wenns der Arzt verlangt,{ }ſo muß ich{ }ſie natürlich ganz auf die Kur verwenden.
               Sollten vier Wochen genügen,{ }ſo möchte ich gern – falls ich noch Geld habe –{ }ſo etwa
               acht Tage irgendwo in der Welt mit Euch zuſammenſein. Jedenfalls{ }ſehe ich mit Freude,
               daß ich Ausſicht habe, Dich{ }ſchon vorher zu{ }ſehen. Mein Wunſch iſt nur, daß es
               möglichſt lange wäre. Nachrichten erreichen mich {\pb}\uline{nach}{ }\textsc{Muenchen\oindex{München@\textbf{München}|pw}} zunächſt \textsc{Toelz\oindex{Bad Tölz@\textbf{Bad Tölz}, \emph{Hauptstadt}|pw}} (\textsc{Baiern\oindex{Bayern@\textbf{Bayern}, \emph{Land}|pw}}) \textsc{Poste-restante}. Kommt die Frau \label{K_L02742-1v}\edtext{\textsc{Andreas\pwindex{Andreas-Salomé, Lou 12.\,2.\,1861 Sankt Petersburg – 5.\,2.\,1937 Göttingen@\textsc{Andreas-Salomé, Lou} (12.\,2.\,1861 Sankt Petersburg – 5.\,2.\,1937 Göttingen), \emph{Schriftstellerin}|pw}} nach \textsc{Salzburg\oindex{Salzburg@\textbf{Salzburg}, \emph{Verwaltungsgebiet}|pw}}}{\lemma{\textnormal{\emph{Andreas nach Salzburg}}}\Cendnote{\textnormal{Siehe die \emph{Tagebuch}\pwindex{Schnitzler, Arthur 15.\,5.\,1862 Wien – 21.\,10.\,1931 ebd.@\textsc{Schnitzler, Arthur} (15.\,5.\,1862 Wien – 21.\,10.\,1931 ebd.), \emph{Schriftsteller, Mediziner}!Tagebuch@\strich\emph{Tagebuch}|pwk}-Einträge zwischen 20. 8. 1895 und 6. 9. 1895. }}}\label{K_L02742-1},{ }ſo gehe ich vielleicht auch
               hinüber. Was Du \textsc{Richard\pwindex{Beer-Hofmann, Richard 11.\,7.\,1866 Wien – 26.\,9.\,1945 New York City@\textsc{Beer-Hofmann, Richard} (11.\,7.\,1866 Wien – 26.\,9.\,1945 New York City), \emph{Schriftsteller}|pw}}{ }\label{K_L02742-2v}\edtext{ſagen{ }ſollſt}{\lemma{\textnormal{\emph{sagen sollst}}}\Cendnote{\textnormal{wohl im Hinblick auf die frühere Beziehung Paul Goldmanns\pwindex{Goldmann, Paul 31.\,1.\,1865 Breslau – 25.\,9.\,1935 Wien@\textsc{Goldmann, Paul} (31.\,1.\,1865 Breslau – 25.\,9.\,1935 Wien), \emph{Schriftsteller, Journalist}|pwk} zu Lou
                     Andreas-Salomé\pwindex{Andreas-Salomé, Lou 12.\,2.\,1861 Sankt Petersburg – 5.\,2.\,1937 Göttingen@\textsc{Andreas-Salomé, Lou} (12.\,2.\,1861 Sankt Petersburg – 5.\,2.\,1937 Göttingen), \emph{Schriftstellerin}|pwk} zu verstehen, mit der Richard Beer-Hofmann\pwindex{Beer-Hofmann, Richard 11.\,7.\,1866 Wien – 26.\,9.\,1945 New York City@\textsc{Beer-Hofmann, Richard} (11.\,7.\,1866 Wien – 26.\,9.\,1945 New York City), \emph{Schriftsteller}|pwk} seit wenigen Wochen intim war}}}\label{K_L02742-2}, weiß ich
               nicht. Ich gebe Dir Vollmacht, zu{ }ſagen, was Du willſt. Mir widerſtrebt es, ihn
               anzulügen. Ich danke Dir für die Mittheilung deſſen, was \label{K_L02742-3v}\edtext{\textsc{Loris\pwindex{Hofmannsthal, Hugo von 1.\,2.\,1874 Wien – 15.\,7.\,1929 Rodaun@\textsc{Hofmannsthal, Hugo von} (1.\,2.\,1874 Wien – 15.\,7.\,1929 Rodaun), \emph{Schriftsteller}|pw}}}{\lemma{\textnormal{\emph{Loris}}}\Cendnote{\textnormal{Schnitzler dürfte Goldmann\pwindex{Goldmann, Paul 31.\,1.\,1865 Breslau – 25.\,9.\,1935 Wien@\textsc{Goldmann, Paul} (31.\,1.\,1865 Breslau – 25.\,9.\,1935 Wien), \emph{Schriftsteller, Journalist}|pwk} aus Hugo von
                     Hofmannsthals\pwindex{Hofmannsthal, Hugo von 1.\,2.\,1874 Wien – 15.\,7.\,1929 Rodaun@\textsc{Hofmannsthal, Hugo von} (1.\,2.\,1874 Wien – 15.\,7.\,1929 Rodaun), \emph{Schriftsteller}|pwk} Brief vom XXXX Auszeichnungsfehler: Dokument L00464 nicht gefunden zitiert haben, in dem dieser geschrieben hat: »Als
                     ein besonders merkwürdiger Tag erscheint mir der, wo wir mit Goldmann\pwindex{Goldmann, Paul 31.\,1.\,1865 Breslau – 25.\,9.\,1935 Wien@\textsc{Goldmann, Paul} (31.\,1.\,1865 Breslau – 25.\,9.\,1935 Wien), \emph{Schriftsteller, Journalist}|pw}{ }{[}{\dots}{]} waren und dann ein großes Gewitter gekommen ist. Ich kann aber
                     nicht finden, warum.«}}}\label{K_L02742-3} geſchrieben. Es iſt{ }ſehr hübſch, – nur
               weiß man nicht recht, was eigentlich an der Sache merkwürdig war, {\pb}\textsc{Goldmann} oder das \strikeout{Gew\textcolor{gray}{i}tte\textcolor{gray}{r}} Gewitter? {\dotsfour}\pend
           
\pstart
           \textsc{Herzl\pwindex{Herzl, Theodor 2.\,5.\,1860 Budapest – 3.\,7.\,1904 Edlach@\textsc{Herzl, Theodor} (2.\,5.\,1860 Budapest – 3.\,7.\,1904 Edlach), \emph{Schriftsteller, Journalist}|pw}} iſt vorgeſtern nach \textsc{Aussee\oindex{Bad Aussee@\textbf{Bad Aussee}, \emph{Hauptstadt}|pw}} abgereiſt. Ich bin innnerlich ganz fertig mit ihm. Äußerlich hält es nur noch
               durch ein paar recht lockere Fäden zuſammen. Der \label{K_L02742-4v}\edtext{ungar\oindex{Ungarn@\textbf{Ungarn}|pwv}iſche Saujud}{\lemma{\textnormal{\emph{ungarische Saujud}}}\Cendnote{\textnormal{Herzls\pwindex{Herzl, Theodor 2.\,5.\,1860 Budapest – 3.\,7.\,1904 Edlach@\textsc{Herzl, Theodor} (2.\,5.\,1860 Budapest – 3.\,7.\,1904 Edlach), \emph{Schriftsteller, Journalist}|pwk} zunehmende Neuorientierung vom
                  literarischen Schriftsteller zum Zionisten wird hier durch Goldmann\pwindex{Goldmann, Paul 31.\,1.\,1865 Breslau – 25.\,9.\,1935 Wien@\textsc{Goldmann, Paul} (31.\,1.\,1865 Breslau – 25.\,9.\,1935 Wien), \emph{Schriftsteller, Journalist}|pwk} mit einer überraschend groben Ausdrucksweise
                  kommentiert. Dies dürfte als Hinweis zu lesen sein, dass Goldmann\pwindex{Goldmann, Paul 31.\,1.\,1865 Breslau – 25.\,9.\,1935 Wien@\textsc{Goldmann, Paul} (31.\,1.\,1865 Breslau – 25.\,9.\,1935 Wien), \emph{Schriftsteller, Journalist}|pwk} den richtigen Umgang mit der jüdischen Kultur in
                  der Assimilation sah, während Herzl\pwindex{Herzl, Theodor 2.\,5.\,1860 Budapest – 3.\,7.\,1904 Edlach@\textsc{Herzl, Theodor} (2.\,5.\,1860 Budapest – 3.\,7.\,1904 Edlach), \emph{Schriftsteller, Journalist}|pwk} das
                  verarmte Judentum aus dem Osten der k. k. Monarchie nicht nur nicht ablehnte,
                  sondern sich dafür begeisterte.}}}\label{K_L02742-4} kommt immer deutlicher \strikeout{\textcolor{gray}{ut}} unter dem Literaten hervor, und das wird unerträglich. Ich glaube es wächſt
               ein \strikeout{ſold}{ }ſolider Haß heran zwiſchen ihm u. mir.\pend
           
\pstart
           Was geht mit Deinem Stücke\pwindex{Schnitzler, Arthur 15.\,5.\,1862 Wien – 21.\,10.\,1931 ebd.@\textsc{Schnitzler, Arthur} (15.\,5.\,1862 Wien – 21.\,10.\,1931 ebd.), \emph{Schriftsteller, Mediziner}!Liebelei. Schauspiel in drei Akten@\strich\emph{Liebelei. Schauspiel in drei Akten}|pwuv} vor, daß Du{ }ſo reſignirt über das {\pb}Warten auf Erfolg{ }ſprichſt? Nun, ich höre es ja nächſtens wohl mündlich. Gewiß, Du{ }ſollſt den Erfolg nicht erwarten. Laß’ \strikeout{D} das nur
               gehn, das thue ich{ }ſchon für Dich.\pend
           
\pstart
           Daß Du »Freiwild\pwindex{Schnitzler, Arthur 15.\,5.\,1862 Wien – 21.\,10.\,1931 ebd.@\textsc{Schnitzler, Arthur} (15.\,5.\,1862 Wien – 21.\,10.\,1931 ebd.), \emph{Schriftsteller, Mediziner}!Freiwild. Schauspiel in 3 Akten@\strich\emph{Freiwild. Schauspiel in 3 Akten}|pw}«{ }ſchreibſt, freut mich{ }ſehr. Du
               haſt Recht: die Arbeit iſt bei dem Allen das Schönſte. Oh, wer arbeiten könnte, \substVorne{}\textsuperscript{!}\substDazwischen{},\substHinten{} wie Du! Alles gute Glück {\pb}zum Werke\pwindex{Schnitzler, Arthur 15.\,5.\,1862 Wien – 21.\,10.\,1931 ebd.@\textsc{Schnitzler, Arthur} (15.\,5.\,1862 Wien – 21.\,10.\,1931 ebd.), \emph{Schriftsteller, Mediziner}!Freiwild. Schauspiel in 3 Akten@\strich\emph{Freiwild. Schauspiel in 3 Akten}|pwv}! {\dotsfour}\pend
           
\pstart
           Grüß’ Dich Gott, mein lieber Freund. Nun wird man{ }ſich bald{ }ſehen. Wie ich mich
                  freue!! {\dotstwo}\pend
           
\pstart
           Dein treuer {\\[\baselineskip]}\spacefill\mbox{Paul Goldmann {\dotstwo}}\pend
           \leftskip=0em{}
\pstart
           \noindent{}Ich weiß \textsc{Richards\pwindex{Beer-Hofmann, Richard 11.\,7.\,1866 Wien – 26.\,9.\,1945 New York City@\textsc{Beer-Hofmann, Richard} (11.\,7.\,1866 Wien – 26.\,9.\,1945 New York City), \emph{Schriftsteller}|pw}} Adreſſe nicht. Bitte,
                     gib\textcolor{gray}{’} ihm inliegenden \label{K_L02742-5v}\edtext{Brief}{\lemma{\textnormal{\emph{Brief}}}\Cendnote{\textnormal{Der
                     sechsseitige Brief, datiert vom 29. 7. [1895], ist im Nachlass Beer-Hofmanns\pwindex{Beer-Hofmann, Richard 11.\,7.\,1866 Wien – 26.\,9.\,1945 New York City@\textsc{Beer-Hofmann, Richard} (11.\,7.\,1866 Wien – 26.\,9.\,1945 New York City), \emph{Schriftsteller}|pwk} in der \emph{Houghton Library}\orgindex{Houghton Library@Houghton Library|pwk},
                        Harvard (Signatur 825.978), überliefert. Goldmann\pwindex{Goldmann, Paul 31.\,1.\,1865 Breslau – 25.\,9.\,1935 Wien@\textsc{Goldmann, Paul} (31.\,1.\,1865 Breslau – 25.\,9.\,1935 Wien), \emph{Schriftsteller, Journalist}|pwk} bedankt sich für Fotografien,
                     eine von Beer-Hofmann\pwindex{Beer-Hofmann, Richard 11.\,7.\,1866 Wien – 26.\,9.\,1945 New York City@\textsc{Beer-Hofmann, Richard} (11.\,7.\,1866 Wien – 26.\,9.\,1945 New York City), \emph{Schriftsteller}|pwk}, die andere von
                     dessen Hund »Flirt«. Goldmann\pwindex{Goldmann, Paul 31.\,1.\,1865 Breslau – 25.\,9.\,1935 Wien@\textsc{Goldmann, Paul} (31.\,1.\,1865 Breslau – 25.\,9.\,1935 Wien), \emph{Schriftsteller, Journalist}|pwk} berichtet
                     von seinem eigenen Pudel und freut sich auf das bevorstehende
                     Wiedersehen.}}}\label{K_L02742-5}.\pend
           \selectlanguage{ngerman}\endnumbering\briefempfaengerindex{Schnitzler, Arthur@\textsc{Schnitzler, Arthur}!zzzGoldmann, Paul@\emph{von Paul Goldmann}!1895-07-291@{29. 7. [1895]}|)be}\mylabel{L02742h}  \newcommand{\dateiname}{L02742}\newcommand{\titel}{Paul Goldmann an Arthur Schnitzler, 29. 7. [1895]}\newcommand{\editorInnen}{Martin Anton Müller und Laura Untner}%% latex-leseansicht-abspann.tex
%% Abspann für die Leseansicht.
%% Der Schalter \ifkorrekturansicht ist bereits durch den Vorspann gesetzt.

%% latex-abspann.tex
%% Gemeinsamer Abspann für Korrekturansicht und Leseansicht.
%% Setzt den Schalter \ifkorrekturansicht voraus (gesetzt in den
%% einbindenden Dateien latex-korrekturansicht-abspann.tex bzw.
%% latex-leseansicht-abspann.tex).
%% ---------------------------------------------------------------

\normalsize

% Das esempio-Environment wird nur in der Leseansicht benötigt
\ifkorrekturansicht\else
\newenvironment{esempio}[3]%
{
    \vspace{1.5ex}
    \rlap{\underline{#1}}
    \par
    \setlength{\parindent}{0cm}
    \nopagebreak
    \leftskip=#2cm
    \rightskip=#3cm
}
{
    \par
}
\fi

\doendnotes{C}
\bigskip
\vfill

\clearpage

\footnotesize

\ifkorrekturansicht
  \lohead{\textsc{register}}
\fi

% theindex-Environment neu definieren ohne reledmac
\makeatletter
\renewenvironment{theindex}{%
  \ifkorrekturansicht
    \section*{\indexname}%
  \else
    \subsubsection*{Index der erwähnten Entitäten}%
  \fi
  \setlength{\parindent}{0pt}%
  \setlength{\parskip}{0pt plus 0.3pt}%
  \let\item\@idxitem
}{%
  \ifkorrekturansicht\clearpage\fi
}
\makeatother

\IfFileExists{\jobname-pw.ind}{\input{\jobname-pw.ind}}{}

% Quellenangabe nur in der Leseansicht
\ifkorrekturansicht\else
% Fallback-Definitionen, falls die .tex-Datei \titel etc. nicht gesetzt hat
\providecommand{\titel}{}
\providecommand{\editorInnen}{}
\providecommand{\dateiname}{\jobname}

\vspace{3cm}

\vfill

\footnotesize
\textsc{Quelle}: \titel. Herausgegeben von {\editorInnen}. In: \emph{Arthur Schnitzler: Briefwechsel mit Autorinnen und Autoren}.
 Digitale Edition, https://schnitzler-briefe.acdh.oeaw.ac.at/{\dateiname}.html (Stand \today)
\fi

\end{document}


