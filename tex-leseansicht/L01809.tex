%% latex-leseansicht-vorspann.tex
%% Vorspann für die Leseansicht.
%% Lädt die gemeinsame Datei latex-vorspann.tex mit nicht gesetztem Schalter.

\newif\ifkorrekturansicht
\korrekturansichtfalse

\input{../tex-inputs/latex-vorspann}


         
         \renewcommand{\erwaehntePersonen}{Personen: Hermann Bahr, Oskar Bie, Ernst von Dohnányi, Anton Edthofer, Heinrich Heine, Hugo von Hofmannsthal, Gertrude von Hofmannsthal, Thomas Mann, Jakob Wassermann, Gustav Wied, Alfred von Winterstein}
         \renewcommand{\erwaehnteInstitutionen}{Institutionen: Neue Rundschau, Neue Deutsche Rundschau, Freie Bühne}
         \renewcommand{\erwaehnteOrte}{Orte: Edmund-Weiß-Gasse, Semmering, Wien}
         \renewcommand{\erwaehnteWerke}{Werke: 2 × 2 = 5, Caph. Novellen, [Gedichte]}
               \section[Arthur Schnitzler an Hugo von Hofmannsthal, 26. 11. 1908]{ Arthur Schnitzler an Hugo von Hofmannsthal, 26. 11. 1908}\nopagebreak\mylabel{v}\rehead{ }\begin{ledgroupsized}[t]{13cm}\normalsize\beginnumbering\briefempfaengerindex{Hofmannsthal, Hugo von@\textsc{Hofmannsthal, Hugo von}!zzzSchnitzler, Arthur@\emph{von Arthur Schnitzler}!1908-11-261@{26. 11. 1908}|(be} \toendnotes[C]{\smallbreak\pagebreak[2]} \Standort{FDH, Hs-30885,133.}
\physDesc{Brief, 1 Blatt, 3 Seiten, 976 Zeichen
\newline{}Handschrift: schwarze Tinte, deutsche Kurrent}\buchAbdrucke{\weitereDrucke{1) Hugo von Hofmannsthal, Arthur Schnitzler: \emph{Briefwechsel}. Hg. Therese Nickl und Heinrich Schnitzler. Frankfurt am Main: \emph{S. Fischer} 1964, S. 242.} \weitereDrucke{2) Hermann Bahr, Arthur Schnitzler: \emph{Briefwechsel, Aufzeichnungen, Dokumente (1891–1931)}. Hg. Kurt Ifkovits und Martin Anton Müller. Göttingen: \emph{Wallstein} 2018, S. 411.} }\toendnotes[C]{\smallbreak}\pstart
           \raggedleft{}{\pb}2\substVorne{}\textsuperscript{5}\substDazwischen{}6\substHinten{}. 11. 08\pend
           \pstart
           \textcolor{gray}{\textbf{Dr. Arthur Schnitzler}}{\\}\textcolor{gray}{\textbf{Wien XVIII. Spoettelgasse 7\oindex{XXXX Ortsangabe fehlt|pw}.}}\pend
           \pstart
           mein lieber Hugo,  geſtern waren wir in \label{K_L01809-1v}\edtext{2 × 2 = 5\pwindex{Wied, Gustav 06.03.1858 – 24.10.1914@\textsc{Wied, Gustav} (06.03.1858 – 24.10.1914), \emph{Schriftsteller}!2 × 2 = 51906@\strich\emph{2 × 2 = 5} {[}1906{]}|pw}}{\lemma{\textnormal{\emph{2 × 2 = 5}}}\Cendnote{\textnormal{von Gustav Wied\pwindex{Wied, Gustav 06.03.1858 – 24.10.1914@\textsc{Wied, Gustav} (06.03.1858 – 24.10.1914), \emph{Schriftsteller}|pwk}}}}\label{K_L01809-1h} (\uline{unbedingt} anzuſehen, ſchon, u. beſonders
               wegen Ethofer\pwindex{Edthofer, Anton 18.09.1883 – 25.02.1971@\textsc{Edthofer, Anton} (18.09.1883 – 25.02.1971), \emph{Schauspieler}|pw}) vorgeſtern beim Krampus\pwindex{Bahr, Hermann 19.07.1863 – 15.01.1934@\textsc{Bahr, Hermann} (19.07.1863 – 15.01.1934), \emph{Schriftsteller, Kritiker}!Caph. Novellen1894@\strich\emph{Caph. Novellen} {[}1894{]}|pw}, heut gehn wir ins Tonkünſtlerconcert,
               Samſtag zum \textsc{Dohnanyi}\pwindex{Dohnányi, Ernst von 27.07.1877 – 09.02.1960@\textsc{Dohnányi, Ernst von} (27.07.1877 – 09.02.1960), \emph{Komponist, Pianist}|pw}, So{\geminationn}tag zum \textsc{Heine\pwindex{Heine, Heinrich 13.12.1797 – 17.02.1856@\textsc{Heine, Heinrich} (13.12.1797 – 17.02.1856), \emph{Schriftsteller}|pw} Abend} – es gibt ſo verhexte
               Wochen; hingegen wollen wir am Montag oder Dinſtag für 2 Tage auf den Se{\geminationm}ering\oindex{Semmering@\textbf{Semmering}|pw}, es wäre
               ſehr ſchön, we{\geminationn} Sie u Gerty\pwindex{Hofmannsthal, Gertrude von 16.03.1880 – 09.11.1959@\textsc{Hofmannsthal, Gertrude von} (16.03.1880 – 09.11.1959)|pw} auch hinauf kämen; schrei{\pb}ben Sie mir ein
               Wort. (Nicht unmöglich, dſs auch Waſſermann\pwindex{Wassermann, Jakob 10.03.1873 – 01.01.1934@\textsc{Wassermann, Jakob} (10.03.1873 – 01.01.1934), \emph{Schriftsteller}|pw} u
                  Thomas Ma{\geminationn}\pwindex{Mann, Thomas 06.06.1875 – 12.08.1955@\textsc{Mann, Thomas} (06.06.1875 – 12.08.1955), \emph{Schriftsteller}|pw} (mit dem wir geſtern Mittag bei W.\pwindex{Wassermann, Jakob 10.03.1873 – 01.01.1934@\textsc{Wassermann, Jakob} (10.03.1873 – 01.01.1934), \emph{Schriftsteller}|pw}
                  zuſa{\geminationm}en waren) hinaufkommen.)\pend
           \pstart
           – Es freut mich, dſs Sie meine Anſicht von den Winterſtein\pwindex{Winterstein, Alfred von 25.09.1885 – 28.04.1958@\textsc{Winterstein, Alfred von} (25.09.1885 – 28.04.1958), \emph{Schriftsteller, Psychoanalytiker, Beamter}|pw}’ſchen Gedichten\pwindex{Winterstein, Alfred von 25.09.1885 – 28.04.1958@\textsc{Winterstein, Alfred von} (25.09.1885 – 28.04.1958), \emph{Schriftsteller, Psychoanalytiker, Beamter}!Gedichte]None@\strich\emph{[Gedichte]} {[}None{]}|pwv} theilen. \label{K_L01809-2v}\edtext{Einmal}{\lemma{\textnormal{\emph{Einmal}}}\Cendnote{\textnormal{vgl. A. S.: \emph{Tagebuch}, 13. 12. 1906}}}\label{K_L01809-2h} hab ich ſchon an Bie\pwindex{Bie, Oskar 09.02.1864 – 21.04.1938@\textsc{Bie, Oskar} (09.02.1864 – 21.04.1938), \emph{Schriftsteller, Journalist, Redakteur}|pw} geſchrieben u ihm
                  Gedichte\pwindex{Winterstein, Alfred von 25.09.1885 – 28.04.1958@\textsc{Winterstein, Alfred von} (25.09.1885 – 28.04.1958), \emph{Schriftsteller, Psychoanalytiker, Beamter}!Gedichte]None@\strich\emph{[Gedichte]} {[}None{]}|pwv} von W.\pwindex{Winterstein, Alfred von 25.09.1885 – 28.04.1958@\textsc{Winterstein, Alfred von} (25.09.1885 – 28.04.1958), \emph{Schriftsteller, Psychoanalytiker, Beamter}|pw} geſchickt, es waren aber viel ſchwächere als
               diesmal; we{\geminationn} Sie glauben, ſo könnte man doch die \textsc{N. Rdsch}\orgindex{Neue Rundschau, Neue Deutsche Rundschau, Freie Buehne@Neue Rundschau, Neue Deutsche Rundschau, Freie Bühne|pw} noch einmal {\pb}verſuchen; ein paar Zeilen von Ihnen
               denk ich wären von allergrößtem Werth. Übrigens ſchreib ich auch an den Baron W.\pwindex{Winterstein, Alfred von 25.09.1885 – 28.04.1958@\textsc{Winterstein, Alfred von} (25.09.1885 – 28.04.1958), \emph{Schriftsteller, Psychoanalytiker, Beamter}|pw}, vielleicht hat er eine andre Bitte an
               Sie. –\pend
           \pstart
           Also auf ſehr baldiges Wiederſehen; u herzliche Grüße.\pend
           \pstart
           Ihr{\\[\baselineskip]}\spacefill\mbox{Arthur}\pend
           \leftskip=0em{}
         
         \endnumbering\mylabel{h}\end{ledgroupsized}  \newcommand{\dateiname}{L01809}\newcommand{\titel}{Arthur Schnitzler an Hugo von Hofmannsthal, 26. 11. 1908}\newcommand{\editorInnen}{ Martin Anton Müller und Gerd-Hermann Susen}%% latex-leseansicht-abspann.tex
%% Abspann für die Leseansicht.
%% Der Schalter \ifkorrekturansicht ist bereits durch den Vorspann gesetzt.

%% latex-abspann.tex
%% Gemeinsamer Abspann für Korrekturansicht und Leseansicht.
%% Setzt den Schalter \ifkorrekturansicht voraus (gesetzt in den
%% einbindenden Dateien latex-korrekturansicht-abspann.tex bzw.
%% latex-leseansicht-abspann.tex).
%% ---------------------------------------------------------------

\normalsize

% Das esempio-Environment wird nur in der Leseansicht benötigt
\ifkorrekturansicht\else
\newenvironment{esempio}[3]%
{
    \vspace{1.5ex}
    \rlap{\underline{#1}}
    \par
    \setlength{\parindent}{0cm}
    \nopagebreak
    \leftskip=#2cm
    \rightskip=#3cm
}
{
    \par
}
\fi

\doendnotes{C}
\bigskip
\vfill

\clearpage

\footnotesize

\ifkorrekturansicht
  \lohead{\textsc{register}}
\fi

% theindex-Environment neu definieren ohne reledmac
\makeatletter
\renewenvironment{theindex}{%
  \ifkorrekturansicht
    \section*{\indexname}%
  \else
    \subsubsection*{Index der erwähnten Entitäten}%
  \fi
  \setlength{\parindent}{0pt}%
  \setlength{\parskip}{0pt plus 0.3pt}%
  \let\item\@idxitem
}{%
  \ifkorrekturansicht\clearpage\fi
}
\makeatother

\IfFileExists{\jobname-pw.ind}{\input{\jobname-pw.ind}}{}

% Quellenangabe nur in der Leseansicht
\ifkorrekturansicht\else
% Fallback-Definitionen, falls die .tex-Datei \titel etc. nicht gesetzt hat
\providecommand{\titel}{}
\providecommand{\editorInnen}{}
\providecommand{\dateiname}{\jobname}

\vspace{3cm}

\vfill

\footnotesize
\textsc{Quelle}: \titel. Herausgegeben von {\editorInnen}. In: \emph{Arthur Schnitzler: Briefwechsel mit Autorinnen und Autoren}.
 Digitale Edition, https://schnitzler-briefe.acdh.oeaw.ac.at/{\dateiname}.html (Stand \today)
\fi

\end{document}


      