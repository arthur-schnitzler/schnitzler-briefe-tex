%% latex-leseansicht-vorspann.tex
%% Vorspann für die Leseansicht.
%% Lädt die gemeinsame Datei latex-vorspann.tex mit nicht gesetztem Schalter.

\newif\ifkorrekturansicht
\korrekturansichtfalse

\input{../tex-inputs/latex-vorspann}


\section[Arthur Schnitzler an Hugo von Hofmannsthal, 26. 11. 1908]{L01809 Arthur Schnitzler an Hugo von Hofmannsthal, 26. 11. 1908}
\nopagebreak\mylabel{L01809v}
\rehead{ }\normalsize\beginnumbering\briefempfaengerindex{Hofmannsthal, Hugo von@\textsc{Hofmannsthal, Hugo von}!zzzSchnitzler, Arthur@\emph{von Arthur Schnitzler}!1908-11-261@{26. 11. 1908}|(be}
\toendnotes[C]{\smallbreak\pagebreak[2]}
\correspDesc{Versand  durch Arthur Schnitzler am 26. 11. 1908 in Wien
\newline{}Erhalt  durch Hugo von Hofmannsthal im Zeitraum [26. 11. 1908 – 30. 11. 1908?] \textbf{Ort fehlend} }\toendnotes[C]{\smallbreak}
\Standort{FDH, Hs-30885,133.}
\physDesc{Brief, 1 Blatt, 3 Seiten, 976 Zeichen
\newline{}Handschrift: schwarze Tinte, deutsche Kurrent}
\buchAbdrucke{\weitereDrucke{1) Hugo von Hofmannsthal, Arthur Schnitzler: \emph{Briefwechsel}. Herausgegeben von Therese Nickl und Heinrich Schnitzler. Frankfurt am Main: \emph{S. Fischer} 1964, S. 242.} \weitereDrucke{2) Hermann Bahr, Arthur Schnitzler: \emph{Briefwechsel, Aufzeichnungen, Dokumente (1891–1931)}. Herausgegeben von Kurt Ifkovits und Martin Anton Müller. Göttingen: \emph{Wallstein} 2018, S. 411.} }\toendnotes[C]{\smallbreak}
\pstart
           \raggedleft{}{\pb}2\substVorne{}\textsuperscript{5}\substDazwischen{}6\substHinten{}. 11. 08\pend
           
\pstart
           \textcolor{gray}{\textbf{Dr. Arthur Schnitzler}}{\\}\textcolor{gray}{\textbf{Wien XVIII. Spoettelgasse 7\oindex{Wien@\textbf{Wien}!XVIII., Währing@\textbf{XVIII., Währing}!Edmund-Weiß-Gasse 7@\textbf{Edmund-Weiß-Gasse 7}, \emph{Wohngebäude}|pw}.}}\pend
           \vspace{0.5em}
\pstart
           mein lieber Hugo,  geſtern waren wir in \label{K_L01809-1v}\edtext{2 × 2 = 5\pwindex{Wied, Gustav 6.\,3.\,1858 Branderslev – 24.\,10.\,1914 Roskilde@\textsc{Wied, Gustav} (6.\,3.\,1858 Branderslev – 24.\,10.\,1914 Roskilde), \emph{Schriftsteller}!2 × 2 = 5@\strich\emph{2 × 2 = 5}|pw}}{\lemma{\textnormal{\emph{2 × 2 = 5}}}\Cendnote{\textnormal{\emph{2 × 2 = 5}\pwindex{Wied, Gustav 6.\,3.\,1858 Branderslev – 24.\,10.\,1914 Roskilde@\textsc{Wied, Gustav} (6.\,3.\,1858 Branderslev – 24.\,10.\,1914 Roskilde), \emph{Schriftsteller}!2 × 2 = 5@\strich\emph{2 × 2 = 5}|pwk} ist ein Theaterstück von Gustav Wied\pwindex{Wied, Gustav 6.\,3.\,1858 Branderslev – 24.\,10.\,1914 Roskilde@\textsc{Wied, Gustav} (6.\,3.\,1858 Branderslev – 24.\,10.\,1914 Roskilde), \emph{Schriftsteller}|pwk}.
               }}}\label{K_L01809-1} (\uline{unbedingt} anzuſehen,{ }ſchon, u. beſonders
               wegen Ethofer\pwindex{Edthofer, Anton 18.\,9.\,1883 Wien – 25.\,2.\,1971 ebd.@\textsc{Edthofer, Anton} (18.\,9.\,1883 Wien – 25.\,2.\,1971 ebd.), \emph{Schauspieler}|pw}) vorgeſtern beim Krampus\pwindex{Bahr, Hermann 19.\,7.\,1863 Linz – 15.\,1.\,1934 München@\textsc{Bahr, Hermann} (19.\,7.\,1863 Linz – 15.\,1.\,1934 München), \emph{Schriftsteller, Kritiker}!Caph. Novellen@\strich\emph{Caph. Novellen}|pw}, heut gehn wir ins Tonkünſtlerconcert,
               Samſtag zum \textsc{Dohnanyi}\pwindex{Dohnányi, Ernst von 27.\,7.\,1877 Bratislava – 9.\,2.\,1960 New York City@\textsc{Dohnányi, Ernst von} (27.\,7.\,1877 Bratislava – 9.\,2.\,1960 New York City), \emph{Komponist, Pianist}|pw}, So{\geminationn}tag zum \textsc{Heine\pwindex{Heine, Heinrich 13.\,12.\,1797 Düsseldorf – 17.\,2.\,1856 Paris@\textsc{Heine, Heinrich} (13.\,12.\,1797 Düsseldorf – 17.\,2.\,1856 Paris), \emph{Schriftsteller}|pw} Abend} – es gibt{ }ſo verhexte
               Wochen; hingegen wollen wir am Montag oder Dinſtag für 2 Tage auf den Se{\geminationm}ering\oindex{Semmering@\textbf{Semmering}, \emph{Verwaltungsgebiet}|pw}, es wäre{ }ſehr{ }ſchön, we{\geminationn} Sie u Gerty\pwindex{Hofmannsthal, Gertrude von 16.\,3.\,1880 Wien – 9.\,11.\,1959 Paddington@\textsc{Hofmannsthal, Gertrude von} (16.\,3.\,1880 Wien – 9.\,11.\,1959 Paddington)|pw} auch hinauf kämen; schrei{\pb}ben Sie mir ein
               Wort. (Nicht unmöglich, dſs auch Waſſermann\pwindex{Wassermann, Jakob 10.\,3.\,1873 Fürth – 1.\,1.\,1934 Altaussee@\textsc{Wassermann, Jakob} (10.\,3.\,1873 Fürth – 1.\,1.\,1934 Altaussee), \emph{Schriftsteller}|pw} u
                  Thomas Ma{\geminationn}\pwindex{Mann, Thomas 6.\,6.\,1875 Lübeck – 12.\,8.\,1955 Zürich@\textsc{Mann, Thomas} (6.\,6.\,1875 Lübeck – 12.\,8.\,1955 Zürich), \emph{Schriftsteller}|pw} (mit dem wir geſtern Mittag bei W.\pwindex{Wassermann, Jakob 10.\,3.\,1873 Fürth – 1.\,1.\,1934 Altaussee@\textsc{Wassermann, Jakob} (10.\,3.\,1873 Fürth – 1.\,1.\,1934 Altaussee), \emph{Schriftsteller}|pw}
                  zuſa{\geminationm}en waren) hinaufkommen.)\pend
           
\pstart
           – Es freut mich, dſs Sie meine Anſicht von den Winterſtein\pwindex{Winterstein, Alfred von 25.\,9.\,1885 Wien – 28.\,4.\,1958 ebd.@\textsc{Winterstein, Alfred von} (25.\,9.\,1885 Wien – 28.\,4.\,1958 ebd.), \emph{Schriftsteller, Psychoanalytiker, Beamter}|pw}’ſchen Gedichten\pwindex{Winterstein, Alfred von 25.\,9.\,1885 Wien – 28.\,4.\,1958 ebd.@\textsc{Winterstein, Alfred von} (25.\,9.\,1885 Wien – 28.\,4.\,1958 ebd.), \emph{Schriftsteller, Psychoanalytiker, Beamter}!Gedichte]@\strich\emph{[Gedichte]}|pwv} theilen. \label{K_L01809-2v}\edtext{Einmal}{\lemma{\textnormal{\emph{Einmal}}}\Cendnote{\textnormal{Vgl. A. S.: \emph{Tagebuch}, 13. 12. 1906.
               }}}\label{K_L01809-2} hab ich{ }ſchon an Bie\pwindex{Bie, Oskar 9.\,2.\,1864 Breslau – 21.\,4.\,1938 Berlin@\textsc{Bie, Oskar} (9.\,2.\,1864 Breslau – 21.\,4.\,1938 Berlin), \emph{Schriftsteller, Journalist, Redakteur}|pw} geſchrieben u ihm
                  Gedichte\pwindex{Winterstein, Alfred von 25.\,9.\,1885 Wien – 28.\,4.\,1958 ebd.@\textsc{Winterstein, Alfred von} (25.\,9.\,1885 Wien – 28.\,4.\,1958 ebd.), \emph{Schriftsteller, Psychoanalytiker, Beamter}!Gedichte]@\strich\emph{[Gedichte]}|pwv} von W.\pwindex{Winterstein, Alfred von 25.\,9.\,1885 Wien – 28.\,4.\,1958 ebd.@\textsc{Winterstein, Alfred von} (25.\,9.\,1885 Wien – 28.\,4.\,1958 ebd.), \emph{Schriftsteller, Psychoanalytiker, Beamter}|pw} geſchickt, es waren aber viel{ }ſchwächere als
               diesmal; we{\geminationn} Sie glauben,{ }ſo könnte man doch die \textsc{N. Rdsch}\orgindex{Neue Rundschau, Neue Deutsche Rundschau, Freie Bühne@Neue Rundschau, Neue Deutsche Rundschau, Freie Bühne|pw} noch einmal {\pb}verſuchen; ein paar Zeilen von Ihnen
               denk ich wären von allergrößtem Werth. Übrigens{ }ſchreib ich auch an den Baron W.\pwindex{Winterstein, Alfred von 25.\,9.\,1885 Wien – 28.\,4.\,1958 ebd.@\textsc{Winterstein, Alfred von} (25.\,9.\,1885 Wien – 28.\,4.\,1958 ebd.), \emph{Schriftsteller, Psychoanalytiker, Beamter}|pw}, vielleicht hat er eine andre Bitte an
               Sie. –\pend
           
\pstart
           Also auf{ }ſehr baldiges Wiederſehen; u herzliche Grüße.\pend
           
\pstart
           Ihr{\\[\baselineskip]}\spacefill\mbox{Arthur}\pend
           \leftskip=0em{}\selectlanguage{ngerman}\endnumbering\briefempfaengerindex{Hofmannsthal, Hugo von@\textsc{Hofmannsthal, Hugo von}!zzzSchnitzler, Arthur@\emph{von Arthur Schnitzler}!1908-11-261@{26. 11. 1908}|)be}\mylabel{L01809h}  \newcommand{\dateiname}{L01809}\newcommand{\titel}{Arthur Schnitzler an Hugo von Hofmannsthal, 26. 11. 1908}\newcommand{\editorInnen}{Herausgegeben von Martin Anton Müller}%% latex-leseansicht-abspann.tex
%% Abspann für die Leseansicht.
%% Der Schalter \ifkorrekturansicht ist bereits durch den Vorspann gesetzt.

%% latex-abspann.tex
%% Gemeinsamer Abspann für Korrekturansicht und Leseansicht.
%% Setzt den Schalter \ifkorrekturansicht voraus (gesetzt in den
%% einbindenden Dateien latex-korrekturansicht-abspann.tex bzw.
%% latex-leseansicht-abspann.tex).
%% ---------------------------------------------------------------

\normalsize

% Das esempio-Environment wird nur in der Leseansicht benötigt
\ifkorrekturansicht\else
\newenvironment{esempio}[3]%
{
    \vspace{1.5ex}
    \rlap{\underline{#1}}
    \par
    \setlength{\parindent}{0cm}
    \nopagebreak
    \leftskip=#2cm
    \rightskip=#3cm
}
{
    \par
}
\fi

\doendnotes{C}
\bigskip
\vfill

\clearpage

\footnotesize

\ifkorrekturansicht
  \lohead{\textsc{register}}
\fi

% theindex-Environment neu definieren ohne reledmac
\makeatletter
\renewenvironment{theindex}{%
  \ifkorrekturansicht
    \section*{\indexname}%
  \else
    \subsubsection*{Index der erwähnten Entitäten}%
  \fi
  \setlength{\parindent}{0pt}%
  \setlength{\parskip}{0pt plus 0.3pt}%
  \let\item\@idxitem
}{%
  \ifkorrekturansicht\clearpage\fi
}
\makeatother

\IfFileExists{\jobname-pw.ind}{\input{\jobname-pw.ind}}{}

% Quellenangabe nur in der Leseansicht
\ifkorrekturansicht\else
% Fallback-Definitionen, falls die .tex-Datei \titel etc. nicht gesetzt hat
\providecommand{\titel}{}
\providecommand{\editorInnen}{}
\providecommand{\dateiname}{\jobname}

\vspace{3cm}

\vfill

\footnotesize
\textsc{Quelle}: \titel. Herausgegeben von {\editorInnen}. In: \emph{Arthur Schnitzler: Briefwechsel mit Autorinnen und Autoren}.
 Digitale Edition, https://schnitzler-briefe.acdh.oeaw.ac.at/{\dateiname}.html (Stand \today)
\fi

\end{document}


