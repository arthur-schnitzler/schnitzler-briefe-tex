%% latex-korrekturansicht-vorspann.tex
%% Vorspann für die Korrekturansicht.
%% Lädt die gemeinsame Datei latex-vorspann.tex mit gesetztem Schalter.

\newif\ifkorrekturansicht
\korrekturansichttrue

\input{../tex-inputs/latex-vorspann}


\section[Arthur Schnitzler an Hugo von Hofmannsthal, 26. 11. 1908]{L01809 Arthur Schnitzler an Hugo von Hofmannsthal, 26. 11. 1908}
\nopagebreak\mylabel{L01809v}
\rehead{ }\normalsize\beginnumbering\briefempfaengerindex{Hofmannsthal, Hugo von@\textsc{Hofmannsthal, Hugo von}!zzzSchnitzler, Arthur@\emph{von Arthur Schnitzler}!1908-11-261@{26. 11. 1908}|(be}
\toendnotes[C]{\smallbreak\pagebreak[2]}\Standort{FDH, Hs-30885,133.}
\physDesc{Brief, 1 Blatt, 3 Seiten, 976 Zeichen
\newline{}Handschrift: schwarze Tinte, deutsche Kurrent}
\buchAbdrucke{\weitereDrucke{1) Hugo von Hofmannsthal, Arthur Schnitzler: \emph{Briefwechsel}. Frankfurt am Main: \emph{S. Fischer} 1964, S. 242.} \weitereDrucke{2) Hermann Bahr, Arthur Schnitzler: \emph{Briefwechsel, Aufzeichnungen, Dokumente (1891–1931)}. Göttingen: \emph{Wallstein} 2018, S. 411.} }\toendnotes[C]{\smallbreak}
\pstart
           \raggedleft{}{\pb}2\substVorne{}\textsuperscript{5}\substDazwischen{}6\substHinten{}. 11. 08\pend
           
\pstart
           \textcolor{gray}{\textbf{Dr. Arthur Schnitzler}}{\\}\textcolor{gray}{\textbf{Wien XVIII. Spoettelgasse 7\oindex{Edmund-Weiss-Gasse 7@\textbf{Edmund-Weiß-Gasse 7}, \emph{Wohngebäude (K.WHS)}|pw}.}}\pend
           \vspace{0.5em}
\pstart
           mein lieber Hugo,  geſtern waren wir in \label{K_L01809-1v}\edtext{2 × 2 = 5\pwindex{2 × 2 = 5@\emph{2 × 2 = 5}|pw}}{\lemma{\textnormal{\emph{2 × 2 = 5}}}\Cendnote{\textnormal{\emph{2 × 2 = 5}\pwindex{2 × 2 = 5@\emph{2 × 2 = 5}|pwk} ist ein Theaterstück von Gustav Wied\pwindex{Wied, Gustav 06.03.1858 – 24.10.1914@\textsc{Wied, Gustav} (06.03.1858 – 24.10.1914), \emph{Schriftsteller/Schriftstellerin}|pwk}.
               }}}\label{K_L01809-1} (\uline{unbedingt} anzuſehen, ſchon, u. beſonders
               wegen Ethofer\pwindex{Edthofer, Anton 18.09.1883 – 25.02.1971@\textsc{Edthofer, Anton} (18.09.1883 – 25.02.1971), \emph{Schauspieler/Schauspielerin}|pw}) vorgeſtern beim Krampus\pwindex{Caph. Novellen@\emph{Caph. Novellen}|pw}, heut gehn wir ins Tonkünſtlerconcert,
               Samſtag zum \textsc{Dohnanyi}\pwindex{Dohnányi, Ernst von 27.07.1877 – 09.02.1960@\textsc{Dohnányi, Ernst von} (27.07.1877 – 09.02.1960), \emph{Komponist/Komponistin, Pianist/Pianistin}|pw}, So{\geminationn}tag zum \textsc{Heine\pwindex{Heine, Heinrich 13.12.1797 – 17.02.1856@\textsc{Heine, Heinrich} (13.12.1797 – 17.02.1856), \emph{Schriftsteller/Schriftstellerin}|pw} Abend} – es gibt ſo verhexte
               Wochen; hingegen wollen wir am Montag oder Dinſtag für 2 Tage auf den Se{\geminationm}ering\oindex{Semmering@\textbf{Semmering}, \emph{A.ADM3}|pw}, es wäre
               ſehr ſchön, we{\geminationn} Sie u Gerty\pwindex{Hofmannsthal, Gertrude von 16.03.1880 – 09.11.1959@\textsc{Hofmannsthal, Gertrude von} (16.03.1880 – 09.11.1959)|pw} auch hinauf kämen; schrei{\pb}ben Sie mir ein
               Wort. (Nicht unmöglich, dſs auch Waſſermann\pwindex{Wassermann, Jakob 10.03.1873 – 01.01.1934@\textsc{Wassermann, Jakob} (10.03.1873 – 01.01.1934), \emph{Schriftsteller/Schriftstellerin}|pw} u
                  Thomas Ma{\geminationn}\pwindex{Mann, Thomas 06.06.1875 – 12.08.1955@\textsc{Mann, Thomas} (06.06.1875 – 12.08.1955), \emph{Schriftsteller/Schriftstellerin}|pw} (mit dem wir geſtern Mittag bei W.\pwindex{Wassermann, Jakob 10.03.1873 – 01.01.1934@\textsc{Wassermann, Jakob} (10.03.1873 – 01.01.1934), \emph{Schriftsteller/Schriftstellerin}|pw}
                  zuſa{\geminationm}en waren) hinaufkommen.)\pend
           
\pstart
           – Es freut mich, dſs Sie meine Anſicht von den Winterſtein\pwindex{Winterstein, Alfred von 25.09.1885 – 28.04.1958@\textsc{Winterstein, Alfred von} (25.09.1885 – 28.04.1958), \emph{Schriftsteller/Schriftstellerin, Psychoanalytiker/Psychoanalytikerin, Beamter/Beamte}|pw}’ſchen Gedichten\pwindex{Gedichte]@\emph{[Gedichte]}|pwv} theilen. \label{K_L01809-2v}\edtext{Einmal}{\lemma{\textnormal{\emph{Einmal}}}\Cendnote{\textnormal{Vgl. A. S.: \emph{Tagebuch}, 13. 12. 1906.
               }}}\label{K_L01809-2} hab ich ſchon an Bie\pwindex{Bie, Oskar 09.02.1864 – 21.04.1938@\textsc{Bie, Oskar} (09.02.1864 – 21.04.1938), \emph{Schriftsteller/Schriftstellerin, Journalist/Journalistin, Redakteur/Redakteurin}|pw} geſchrieben u ihm
                  Gedichte\pwindex{Gedichte]@\emph{[Gedichte]}|pwv} von W.\pwindex{Winterstein, Alfred von 25.09.1885 – 28.04.1958@\textsc{Winterstein, Alfred von} (25.09.1885 – 28.04.1958), \emph{Schriftsteller/Schriftstellerin, Psychoanalytiker/Psychoanalytikerin, Beamter/Beamte}|pw} geſchickt, es waren aber viel ſchwächere als
               diesmal; we{\geminationn} Sie glauben, ſo könnte man doch die \textsc{N. Rdsch}\orgindex{Neue Rundschau, Neue Deutsche Rundschau, Freie Buehne@Neue Rundschau, Neue Deutsche Rundschau, Freie Bühne|pw} noch einmal {\pb}verſuchen; ein paar Zeilen von Ihnen
               denk ich wären von allergrößtem Werth. Übrigens ſchreib ich auch an den Baron W.\pwindex{Winterstein, Alfred von 25.09.1885 – 28.04.1958@\textsc{Winterstein, Alfred von} (25.09.1885 – 28.04.1958), \emph{Schriftsteller/Schriftstellerin, Psychoanalytiker/Psychoanalytikerin, Beamter/Beamte}|pw}, vielleicht hat er eine andre Bitte an
               Sie. –\pend
           
\pstart
           Also auf ſehr baldiges Wiederſehen; u herzliche Grüße.\pend
           
\pstart
           Ihr{\\[\baselineskip]}\spacefill\mbox{Arthur}\pend
           \leftskip=0em{}\selectlanguage{ngerman}\endnumbering\briefempfaengerindex{Hofmannsthal, Hugo von@\textsc{Hofmannsthal, Hugo von}!zzzSchnitzler, Arthur@\emph{von Arthur Schnitzler}!1908-11-261@{26. 11. 1908}|)be}\mylabel{L01809h}  \normalsize

\doendnotes{C}
\bigskip
\vfill

\clearpage

\footnotesize

\lohead{\textsc{register}}

% Definiere theindex-Environment komplett neu ohne reledmac
\makeatletter
\renewenvironment{theindex}{%
  \section*{\indexname}%
  \setlength{\parindent}{0pt}%
  \setlength{\parskip}{0pt plus 0.3pt}%
  \let\item\@idxitem
}{%
  \clearpage
}
\makeatother

\IfFileExists{\jobname-pw.ind}{\input{\jobname-pw.ind}}{}

\end{document}

      