%% latex-korrekturansicht-vorspann.tex
%% Vorspann für die Korrekturansicht.
%% Lädt die gemeinsame Datei latex-vorspann.tex mit gesetztem Schalter.

\newif\ifkorrekturansicht
\korrekturansichttrue

\input{../tex-inputs/latex-vorspann}


\section[Hermann Bahr an Arthur Schnitzler, 5. 8. 1905]{L01538 Hermann Bahr an Arthur Schnitzler, 5. 8. 1905}
\nopagebreak\mylabel{L01538v}
\rehead{ }\normalsize\beginnumbering\briefempfaengerindex{Schnitzler, Arthur@\textsc{Schnitzler, Arthur}!zzzBahr, Hermann@\emph{von Hermann Bahr}!1905-08-051@{5. 8. 1905}|(be}
\toendnotes[C]{\smallbreak\pagebreak[2]}\Standort{CUL, Schnitzler, B 5b.}
\physDesc{Bildpostkarte, 389 Zeichen
\newline{}Handschrift: 1) Bleistift, deutsche Kurrent\hspace{1em}2) Bleistift, lateinische Kurrent (\noindent{}Adresse)\hspace{1em}
\newline{}Versand: Stempel: »\nobreak{}\oindex{Muenchen@\textbf{München}, \emph{P.PPLA}|pwk}München–Glaspalast, 5 Aug 05, 12–1\nobreak{}«.  
\newline{}Ordnung: mit Bleistift von unbekannter Hand nummeriert:
                                    »131« }
\buchAbdrucke{\weitereDrucke{Hermann Bahr, Arthur Schnitzler: \emph{Briefwechsel, Aufzeichnungen, Dokumente (1891–1931)}. Göttingen: \emph{Wallstein} 2018, S. 349.} }\toendnotes[C]{\smallbreak}\pstart{}{\pb}D\textsuperscript{r} Artur
                  Schnitzler\pend{}\pstart{}Wien XVIII\oindex{XVIII., Waehring@\textbf{XVIII., Währing}, \emph{A.ADM3}|pw}\pend{}\pstart{}Spöttelgasse 7\oindex{Edmund-Weiss-Gasse 7@\textbf{Edmund-Weiß-Gasse 7}, \emph{Wohngebäude (K.WHS)}|pw}\pend{}{\bigskip}
\pstart
           \noindent{}\centering{}{\pb}\textcolor{gray}{\textbf{München. Glaspalast\oindex{Glaspalast@\textbf{Glaspalast}, \emph{Gebäude (K.GBD)}|pw}}}\pend
           \vspace{1em}
\pstart
           {\pb}5. 8.\pend
           \vspace{0.5em}
\pstart
           Einſtweilen herzlichſten Dank für Deinen lieben Brief. Mit allem anderen magſt Du
               recht haben, mit \textsc{Besenius}\pwindex{Andere@\emph{Die Andere}|pwv} nicht. Für mich müßte das Stück\pwindex{Andere@\emph{Die Andere}|pwv} eigentlich \textsc{Besenius}\pwindex{Andere@\emph{Die Andere}|pwv} heißen, da ſein Thema iſt: 1) Was kann ein wirklicher Menſch heute werden?
               Antwort: \textsc{Besenius}\pwindex{Andere@\emph{Die Andere}|pwv}. 2) Wie wird man \textsc{Besenius}\pwindex{Andere@\emph{Die Andere}|pwv}? Wenn man Heinrich\pwindex{Andere@\emph{Die Andere}|pwv} iſt
               und dies erlebt.\pend
           \pstart Herzlichſt \spacefill\mbox{H.}\pend{}
\pstart
           \noindent{}Viele Grüße Deiner Frau\pwindex{Schnitzler, Olga 17.01.1882 – 13.01.1970@\textsc{Schnitzler, Olga} (17.01.1882 – 13.01.1970), \emph{Schauspieler/Schauspielerin, Sänger/Sängerin}|pwv}\pend
           \selectlanguage{ngerman}\endnumbering\briefempfaengerindex{Schnitzler, Arthur@\textsc{Schnitzler, Arthur}!zzzBahr, Hermann@\emph{von Hermann Bahr}!1905-08-051@{5. 8. 1905}|)be}\mylabel{L01538h}  \normalsize

\doendnotes{C}
\bigskip
\vfill

\clearpage

\footnotesize

\lohead{\textsc{register}}

% Definiere theindex-Environment komplett neu ohne reledmac
\makeatletter
\renewenvironment{theindex}{%
  \section*{\indexname}%
  \setlength{\parindent}{0pt}%
  \setlength{\parskip}{0pt plus 0.3pt}%
  \let\item\@idxitem
}{%
  \clearpage
}
\makeatother

\IfFileExists{\jobname-pw.ind}{\input{\jobname-pw.ind}}{}

\end{document}

      