%% latex-leseansicht-vorspann.tex
%% Vorspann für die Leseansicht.
%% Lädt die gemeinsame Datei latex-vorspann.tex mit nicht gesetztem Schalter.

\newif\ifkorrekturansicht
\korrekturansichtfalse

\input{../tex-inputs/latex-vorspann}

\begin{center}
            \textcolor{red}{ENTWURF, NICHT FERTIG KORRIGIERT}
                      \end{center}
            
         
         \newcommand{\erwaehntePersonen}{Personen: Eberhard König, Agnes Sorma}
         \newcommand{\erwaehnteOrte}{Orte: Berlin, Dessauer Straße, Frankfurt am Main, Lessing-Theater, Schauspielhaus, Wien}
         \newcommand{\erwaehnteWerke}{Werke: Gevatter Tod. Ein Märchen von der Menschheit. Drama in fünf Aufzügen, Liebelei. Schauspiel in drei Akten}
               \section[ Paul Goldmann an Arthur Schnitzler, 13. 4. {[}1900{]}]{ Paul Goldmann an Arthur Schnitzler, 13. 4. {[}1900{]}}\nopagebreak\mylabel{v}\rehead{ }\begin{ledgroupsized}[t]{13cm}\normalsize\beginnumbering \toendnotes[C]{\smallbreak\pagebreak[2]} \Standort{DLA, A:Schnitzler, HS.NZ85.1.3170.}
\physDesc{Brief, 1 Blatt, 2 Seiten
\newline{}Handschrift: blaue Tinte, deutsche Kurrent
\newline{}Schnitzler: mit Bleistift das Jahr »{[}1{]}900« vermerkt }\toendnotes[C]{\smallbreak}\pstart{}{\pb}\textcolor{gray}{\textbf{DESSAUERSTRASSE 19}}\oindex{Dessauer Strasse@\textbf{Dessauer Straße}|pw}\pend{}{\bigskip}\pstart
           \raggedleft{}Berlin\oindex{Berlin@\textbf{Berlin}|pw}, 13. April.\pend
           \pstart{}Mein lieber Freund,\pend\pstart
           Warum höre ich denn ſo gar nichts von Dir? Die zwei Anſichtspoſtkarten habe ich wohl
               erhalten, aber ſie geben mir mehr Aufſchluß über die \label{K_L02910-1v}\edtext{Gegend}{\lemma{\textnormal{\emph{Gegend}}}\Cendnote{\textnormal{siehe Paul Goldmann an Arthur Schnitzler, 29. 3. [1900]}}}\label{K_L02910-1h}, als über Dein Ergehen. Haſt Du unterwegs nicht einmal eine Viertelſtunde, um
               mir etwas ausführlicher zu berichten, was Du erlebſt und wie Du Dich fühlſt? Ich weiß
               nicht einmal, ob Du ſchon zurück biſt. Und wann kommſt Du nach \label{K_L02910-2v}\edtext{Berlin\oindex{Berlin@\textbf{Berlin}|pw}}{\lemma{\textnormal{\emph{Berlin}}}\Cendnote{\textnormal{Schnitzler\pwindex{Schnitzler, Arthur 15.05.1862 – 21.10.1931@\textsc{Schnitzler, Arthur} (15.05.1862 – 21.10.1931), \emph{Schriftsteller, Mediziner}|pwk} kam erst am 24. 11. 1900 wieder
                  nach Berlin\oindex{Berlin@\textbf{Berlin}|pwk}. Er blieb dort bis zum 28. 11. 1900.}}}\label{K_L02910-2h}?
               Hätte ich gewußt, ob Du bereits wieder {\pb}heimgekehrt
               biſt, ſo \strikeout{hätte} wäre ich vielleicht über Oſtern nach
                  Wien\oindex{Wien@\textbf{Wien}|pw} gekommen. Aber bei dieſer
               Nachrichtenloſigkeit habe ich mich zu einem Entſchluß nicht aufſchwingen können.
               Bitte, ſchreib’ mir bald!\pend
           \pstart
           Ich hätte gern über das \label{K_L02910-3v}\edtext{Gaſtſpiel der
                  \textsc{Sorma\pwindex{Sorma, Agnes 17.05.1862 – 10.02.1927@\textsc{Sorma, Agnes} (17.05.1862 – 10.02.1927), \emph{Schauspielerin}|pw}} in »Liebelei\pwindex{Schnitzler, Arthur 15.05.1862 – 21.10.1931@\textsc{Schnitzler, Arthur} (15.05.1862 – 21.10.1931), \emph{Schriftsteller, Mediziner}!Liebelei. Schauspiel in drei Akten1895-10-09@\strich\emph{Liebelei. Schauspiel in drei Akten} {[}1895-10-09{]}|pw}«}{\lemma{\textnormal{\emph{Gaſtſpiel … »Liebelei«}}}\Cendnote{\textnormal{Agnes Sorma\pwindex{Sorma, Agnes 17.05.1862 – 10.02.1927@\textsc{Sorma, Agnes} (17.05.1862 – 10.02.1927), \emph{Schauspielerin}|pwk} gastierte am 4. 6. 1900 und am 12. 4. 1900 als Christine\pwindex{Schnitzler, Arthur 15.05.1862 – 21.10.1931@\textsc{Schnitzler, Arthur} (15.05.1862 – 21.10.1931), \emph{Schriftsteller, Mediziner}!Liebelei. Schauspiel in drei Akten1895-10-09@\strich\emph{Liebelei. Schauspiel in drei Akten} {[}1895-10-09{]}|pwkv} in den \emph{Liebelei}\pwindex{Schnitzler, Arthur 15.05.1862 – 21.10.1931@\textsc{Schnitzler, Arthur} (15.05.1862 – 21.10.1931), \emph{Schriftsteller, Mediziner}!Liebelei. Schauspiel in drei Akten1895-10-09@\strich\emph{Liebelei. Schauspiel in drei Akten} {[}1895-10-09{]}|pwk}-Aufführungen am Berlin\oindex{Berlin@\textbf{Berlin}|pwk}er Lessing-Theater\oindex{Lessing-Theater@\textbf{Lessing-Theater}|pwk}.}}}\label{K_L02910-3h} berichtet. Aber am erſten Abend war eine blödſinnige \label{K_L02910-52v}\edtext{\begin{otherlanguage}{french}\textsc{Première\pwindex{Koenig, Eberhard 1871-06-18 – 1949-12-26@\textsc{König, Eberhard} (1871-06-18 – 1949-12-26), \emph{Schriftsteller}!Gevatter Tod. Ein Maerchen von der Menschheit. Drama in fuenf Aufzuegen1900@\strich\emph{Gevatter Tod. Ein Märchen von der Menschheit. Drama in fünf Aufzügen} {[}1900{]}|pwv}}\end{otherlanguage}}{\lemma{\textnormal{\emph{Première}}}\Cendnote{\textnormal{von Eberhard König\pwindex{Koenig, Eberhard 1871-06-18 – 1949-12-26@\textsc{König, Eberhard} (1871-06-18 – 1949-12-26), \emph{Schriftsteller}|pwk}s Fünfakter \emph{Gevatter Tod.
                     Ein Märchen von der Menschheit}\pwindex{Koenig, Eberhard 1871-06-18 – 1949-12-26@\textsc{König, Eberhard} (1871-06-18 – 1949-12-26), \emph{Schriftsteller}!Gevatter Tod. Ein Maerchen von der Menschheit. Drama in fuenf Aufzuegen1900@\strich\emph{Gevatter Tod. Ein Märchen von der Menschheit. Drama in fünf Aufzügen} {[}1900{]}|pwk}}}}\label{K_L02910-52h} im Schauſpielhauſe\oindex{Schauspielhaus@\textbf{Schauspielhaus}|pw}; und am zweiten konnte ich auch nicht hineingehen. Es ſteht in
               den Sternen geſchrieben, daß ich \label{K_L02910-4v}\edtext{nie
               ein Stück von Dir auf der Bühne ſehen}{\lemma{\textnormal{\emph{nie … ſehen}}}\Cendnote{\textnormal{Goldmann\pwindex{Goldmann, Paul 31.01.1865 – 25.09.1935@\textsc{Goldmann, Paul} (31.01.1865 – 25.09.1935), \emph{Schriftsteller, Journalist}|pwk} hatte bereits ein Stück (?) von
                     Schnitzler\pwindex{Schnitzler, Arthur 15.05.1862 – 21.10.1931@\textsc{Schnitzler, Arthur} (15.05.1862 – 21.10.1931), \emph{Schriftsteller, Mediziner}|pwk} in Frankfurt am Main\oindex{Frankfurt am Main@\textbf{Frankfurt am Main}|pwk} (?) gesehen. XXXX Konkretisierung und
                  Verweis auf den Brief, wenn wir ihn wiederfinden}}}\label{K_L02910-4h} ſoll.\pend
           \pstart
           Viele treue Grüße! {\\[\baselineskip]}Dein {\\[\baselineskip]}\spacefill\mbox{Paul Goldmann.}\pend
           \leftskip=0em{}
         
         \endnumbering\mylabel{h}\end{ledgroupsized}\begin{anhang}\end{anhang}\newcommand{\dateiname}{L02910}\newcommand{\titel}{Paul Goldmann an Arthur Schnitzler, 13. 4. [1900]}\newcommand{\editorInnen}{Martin Anton Müller und Laura Untner}%% latex-leseansicht-abspann.tex
%% Abspann für die Leseansicht.
%% Der Schalter \ifkorrekturansicht ist bereits durch den Vorspann gesetzt.

%% latex-abspann.tex
%% Gemeinsamer Abspann für Korrekturansicht und Leseansicht.
%% Setzt den Schalter \ifkorrekturansicht voraus (gesetzt in den
%% einbindenden Dateien latex-korrekturansicht-abspann.tex bzw.
%% latex-leseansicht-abspann.tex).
%% ---------------------------------------------------------------

\normalsize

% Das esempio-Environment wird nur in der Leseansicht benötigt
\ifkorrekturansicht\else
\newenvironment{esempio}[3]%
{
    \vspace{1.5ex}
    \rlap{\underline{#1}}
    \par
    \setlength{\parindent}{0cm}
    \nopagebreak
    \leftskip=#2cm
    \rightskip=#3cm
}
{
    \par
}
\fi

\doendnotes{C}
\bigskip
\vfill

\clearpage

\footnotesize

\ifkorrekturansicht
  \lohead{\textsc{register}}
\fi

% theindex-Environment neu definieren ohne reledmac
\makeatletter
\renewenvironment{theindex}{%
  \ifkorrekturansicht
    \section*{\indexname}%
  \else
    \subsubsection*{Index der erwähnten Entitäten}%
  \fi
  \setlength{\parindent}{0pt}%
  \setlength{\parskip}{0pt plus 0.3pt}%
  \let\item\@idxitem
}{%
  \ifkorrekturansicht\clearpage\fi
}
\makeatother

\IfFileExists{\jobname-pw.ind}{\input{\jobname-pw.ind}}{}

% Quellenangabe nur in der Leseansicht
\ifkorrekturansicht\else
% Fallback-Definitionen, falls die .tex-Datei \titel etc. nicht gesetzt hat
\providecommand{\titel}{}
\providecommand{\editorInnen}{}
\providecommand{\dateiname}{\jobname}

\vspace{3cm}

\vfill

\footnotesize
\textsc{Quelle}: \titel. Herausgegeben von {\editorInnen}. In: \emph{Arthur Schnitzler: Briefwechsel mit Autorinnen und Autoren}.
 Digitale Edition, https://schnitzler-briefe.acdh.oeaw.ac.at/{\dateiname}.html (Stand \today)
\fi

\end{document}


      