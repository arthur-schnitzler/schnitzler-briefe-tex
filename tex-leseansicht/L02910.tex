%% latex-korrekturansicht-vorspann.tex
%% Vorspann für die Korrekturansicht.
%% Lädt die gemeinsame Datei latex-vorspann.tex mit gesetztem Schalter.

\newif\ifkorrekturansicht
\korrekturansichttrue

\input{../tex-inputs/latex-vorspann}


\section[ Paul Goldmann an Arthur Schnitzler, 13. 4. {[}1900{]}]{L02910 Paul Goldmann an Arthur Schnitzler, 13. 4. {[}1900{]}}
\nopagebreak\mylabel{L02910v}
\rehead{ }\normalsize\beginnumbering\briefempfaengerindex{Schnitzler, Arthur@\textsc{Schnitzler, Arthur}!zzzGoldmann, Paul@\emph{von Paul Goldmann}!1900-04-131@{13. 4. {[}1900{]}}|(be}
\toendnotes[C]{\smallbreak\pagebreak[2]}\Standort{DLA, A:Schnitzler, HS.NZ85.1.3170.}
\physDesc{Brief, 1 Blatt, 2 Seiten, 956 Zeichen
\newline{}Handschrift: blaue Tinte, deutsche Kurrent
\newline{}Schnitzler: mit Bleistift das Jahr »900« vermerkt }\toendnotes[C]{\smallbreak}
\pstart
           {\pb}\textcolor{gray}{\textbf{DESSAUERSTRASSE 19}}\oindex{Dessauer Strasse@\textbf{Dessauer Straße}, \emph{Straße (K.STR)}|pw}\pend
           
\pstart
           \raggedleft{}Berlin\oindex{Berlin@\textbf{Berlin}, \emph{P.PPLC}|pw}, 13. April.\pend
           
\pstart{}Mein lieber Freund,\pend\vspace{0.5em}
\pstart
           Warum höre ich denn ſo gar nichts von Dir? Die zwei Anſichtspoſtkarten habe ich wohl
               erhalten, aber ſie geben mir mehr Aufſchluß über die \label{K_L02910-1v}\edtext{Gegend}{\lemma{\textnormal{\emph{Gegend}}}\Cendnote{\textnormal{Siehe Paul Goldmann an Arthur Schnitzler, 29. 3. [1900].
               }}}\label{K_L02910-1}, als über Dein Ergehen. Haſt Du unterwegs nicht einmal eine Viertelſtunde, um
               mir etwas ausführlicher zu berichten, was Du erlebſt und wie Du Dich fühlſt? Ich weiß
               nicht einmal, ob Du ſchon zurück biſt. Und wann kommſt Du nach \label{K_L02910-2v}\edtext{Berlin\oindex{Berlin@\textbf{Berlin}, \emph{P.PPLC}|pw}}{\lemma{\textnormal{\emph{Berlin}}}\Cendnote{\textnormal{Schnitzler kam erst am 24. 11. 1900 wieder
                  nach Berlin\oindex{Berlin@\textbf{Berlin}, \emph{P.PPLC}|pwk}. Er blieb dort bis zum 28. 11. 1900.}}}\label{K_L02910-2}?
               Hätte ich gewußt, ob Du bereits wieder {\pb}heimgekehrt
               biſt, ſo \strikeout{hätte} wäre ich vielleicht über Oſtern nach
                  Wien\oindex{Wien@\textbf{Wien}, \emph{A.ADM2}|pw} gekommen. Aber bei dieſer
               Nachrichtenloſigkeit habe ich mich zu einem Entſchluß nicht aufſchwingen können.
               Bitte, ſchreib’ mir bald!\pend
           
\pstart
           Ich hätte gern über das \label{K_L02910-3v}\edtext{Gaſtſpiel der
                  \textsc{Sorma\pwindex{Sorma, Agnes 17.05.1862 – 10.02.1927@\textsc{Sorma, Agnes} (17.05.1862 – 10.02.1927), \emph{Schauspieler/Schauspielerin}|pw}} in »Liebelei\pwindex{Liebelei. Schauspiel in drei Akten@\emph{Liebelei. Schauspiel in drei Akten}|pw}«}{\lemma{\textnormal{\emph{Gaſtſpiel … »Liebelei«}}}\Cendnote{\textnormal{Agnes Sorma\pwindex{Sorma, Agnes 17.05.1862 – 10.02.1927@\textsc{Sorma, Agnes} (17.05.1862 – 10.02.1927), \emph{Schauspieler/Schauspielerin}|pwk} gastierte am 4. 6. 1900 und am 12. 4. 1900 als Christine\pwindex{Liebelei. Schauspiel in drei Akten@\emph{Liebelei. Schauspiel in drei Akten}|pwkv} in den \emph{Liebelei}\pwindex{Liebelei. Schauspiel in drei Akten@\emph{Liebelei. Schauspiel in drei Akten}|pwk}-Aufführungen am Berlin\oindex{Berlin@\textbf{Berlin}, \emph{P.PPLC}|pwk}er \emph{Lessing-Theater}\orgindex{Lessing-Theater@Lessing-Theater|pwk}.}}}\label{K_L02910-3} berichtet. Aber am
               erſten Abend war eine blödſinnige \label{K_L02910-4v}\edtext{\begin{otherlanguage}{french}\textsc{Première\pwindex{Gevatter Tod. Ein Maerchen von der Menschheit. Drama in fuenf Aufzuegen@\emph{Gevatter Tod. Ein Märchen von der Menschheit. Drama in fünf Aufzügen}|pwv}}\end{otherlanguage}}{\lemma{\textnormal{\emph{Première}}}\Cendnote{\textnormal{von Eberhard Königs\pwindex{Koenig, Eberhard 1871-06-18 – 1949-12-26@\textsc{König, Eberhard} (1871-06-18 – 1949-12-26), \emph{Schriftsteller/Schriftstellerin}|pwk} Fünfakter \emph{Gevatter Tod.
                     Ein Märchen von der Menschheit}\pwindex{Gevatter Tod. Ein Maerchen von der Menschheit. Drama in fuenf Aufzuegen@\emph{Gevatter Tod. Ein Märchen von der Menschheit. Drama in fünf Aufzügen}|pwk}}}}\label{K_L02910-4} im Schauſpielhauſe\oindex{Schauspielhaus Berlin@\textbf{Schauspielhaus Berlin}, \emph{Theater (K.THE)}|pw}; und am zweiten konnte ich auch nicht hineingehen. Es ſteht in
               den Sternen geſchrieben, daß ich nie ein Stück von Dir auf der Bühne ſehen ſoll.\pend
           
\pstart
           Viele treue Grüße! {\\[\baselineskip]}Dein {\\[\baselineskip]}\spacefill\mbox{Paul Goldmann.}\pend
           \leftskip=0em{}\selectlanguage{ngerman}\endnumbering\briefempfaengerindex{Schnitzler, Arthur@\textsc{Schnitzler, Arthur}!zzzGoldmann, Paul@\emph{von Paul Goldmann}!1900-04-131@{13. 4. {[}1900{]}}|)be}\mylabel{L02910h}  \normalsize

\doendnotes{C}
\bigskip
\vfill

\clearpage

\footnotesize

\lohead{\textsc{register}}

% Definiere theindex-Environment komplett neu ohne reledmac
\makeatletter
\renewenvironment{theindex}{%
  \section*{\indexname}%
  \setlength{\parindent}{0pt}%
  \setlength{\parskip}{0pt plus 0.3pt}%
  \let\item\@idxitem
}{%
  \clearpage
}
\makeatother

\IfFileExists{\jobname-pw.ind}{\input{\jobname-pw.ind}}{}

\end{document}

      