%% latex-leseansicht-vorspann.tex
%% Vorspann für die Leseansicht.
%% Lädt die gemeinsame Datei latex-vorspann.tex mit nicht gesetztem Schalter.

\newif\ifkorrekturansicht
\korrekturansichtfalse

\input{../tex-inputs/latex-vorspann}


\section[ Paul Goldmann an Arthur Schnitzler, 13. 4. {[}1900{]}]{L02910 Paul Goldmann an Arthur Schnitzler,  13. 4. [1900]}
\nopagebreak\mylabel{L02910v}
\rehead{ }\normalsize\beginnumbering\briefempfaengerindex{Schnitzler, Arthur@\textsc{Schnitzler, Arthur}!zzzGoldmann, Paul@\emph{von Paul Goldmann}!1900-04-131@{13. 4. [1900]}|(be}
\toendnotes[C]{\smallbreak\pagebreak[2]}
\correspDesc{Versand  durch Paul Goldmann am 13. 4. [1900] in Berlin
\newline{}Erhalt  durch Arthur Schnitzler im Zeitraum [14. 4. 1900
                  – 18. 4. 1900?] in Wien}\toendnotes[C]{\smallbreak}
\Standort{DLA, A:Schnitzler, HS.NZ85.1.3170.}
\physDesc{Brief, 1 Blatt, 2 Seiten, 956 Zeichen
\newline{}Handschrift: blaue Tinte, deutsche Kurrent
\newline{}Schnitzler: mit Bleistift das Jahr »900« vermerkt }\toendnotes[C]{\smallbreak}
\pstart
           {\pb}\textcolor{gray}{\textbf{DESSAUERSTRASSE 19}}\oindex{Dessauer Straße@\textbf{Dessauer Straße}, \emph{Straße}|pw}\pend
           
\pstart
           \raggedleft{}Berlin\oindex{Berlin@\textbf{Berlin}, \emph{Hauptstadt}|pw}, 13. April.\pend
           
\pstart{}Mein lieber Freund,\pend\vspace{0.5em}
\pstart
           Warum höre ich denn{ }ſo gar nichts von Dir? Die zwei Anſichtspoſtkarten habe ich wohl
               erhalten, aber{ }ſie geben mir mehr Aufſchluß über die \label{K_L02910-1v}\edtext{Gegend}{\lemma{\textnormal{\emph{Gegend}}}\Cendnote{\textnormal{Siehe XXXX Auszeichnungsfehler: Dokument L02909 nicht gefunden.
               }}}\label{K_L02910-1}, als über Dein Ergehen. Haſt Du unterwegs nicht einmal eine Viertelſtunde, um
               mir etwas ausführlicher zu berichten, was Du erlebſt und wie Du Dich fühlſt? Ich weiß
               nicht einmal, ob Du{ }ſchon zurück biſt. Und wann kommſt Du nach \label{K_L02910-2v}\edtext{Berlin\oindex{Berlin@\textbf{Berlin}, \emph{Hauptstadt}|pw}}{\lemma{\textnormal{\emph{Berlin}}}\Cendnote{\textnormal{Schnitzler kam erst am 24. 11. 1900 wieder
                  nach Berlin\oindex{Berlin@\textbf{Berlin}, \emph{Hauptstadt}|pwk}. Er blieb dort bis zum 28. 11. 1900.}}}\label{K_L02910-2}?
               Hätte ich gewußt, ob Du bereits wieder {\pb}heimgekehrt
               biſt,{ }ſo \strikeout{hätte} wäre ich vielleicht über Oſtern nach
                  Wien\oindex{Wien@\textbf{Wien}, \emph{Verwaltungsgebiet}|pw} gekommen. Aber bei dieſer
               Nachrichtenloſigkeit habe ich mich zu einem Entſchluß nicht aufſchwingen können.
               Bitte,{ }ſchreib’ mir bald!\pend
           
\pstart
           Ich hätte gern über das \label{K_L02910-3v}\edtext{Gaſtſpiel der
                  \textsc{Sorma\pwindex{Sorma, Agnes 17.\,5.\,1862 Breslau – 10.\,2.\,1927 Crown King@\textsc{Sorma, Agnes} (17.\,5.\,1862 Breslau – 10.\,2.\,1927 Crown King), \emph{Schauspielerin}|pw}} in »Liebelei\pwindex{Schnitzler, Arthur 15.\,5.\,1862 Wien – 21.\,10.\,1931 ebd.@\textsc{Schnitzler, Arthur} (15.\,5.\,1862 Wien – 21.\,10.\,1931 ebd.), \emph{Schriftsteller, Mediziner}!Liebelei. Schauspiel in drei Akten@\strich\emph{Liebelei. Schauspiel in drei Akten}|pw}«}{\lemma{\textnormal{\emph{Gastspiel … »Liebelei«}}}\Cendnote{\textnormal{Agnes Sorma\pwindex{Sorma, Agnes 17.\,5.\,1862 Breslau – 10.\,2.\,1927 Crown King@\textsc{Sorma, Agnes} (17.\,5.\,1862 Breslau – 10.\,2.\,1927 Crown King), \emph{Schauspielerin}|pwk} gastierte am 4. 6. 1900 und am 12. 4. 1900 als Christine\pwindex{Schnitzler, Arthur 15.\,5.\,1862 Wien – 21.\,10.\,1931 ebd.@\textsc{Schnitzler, Arthur} (15.\,5.\,1862 Wien – 21.\,10.\,1931 ebd.), \emph{Schriftsteller, Mediziner}!Liebelei. Schauspiel in drei Akten@\strich\emph{Liebelei. Schauspiel in drei Akten}|pwkv} in den \emph{Liebelei}\pwindex{Schnitzler, Arthur 15.\,5.\,1862 Wien – 21.\,10.\,1931 ebd.@\textsc{Schnitzler, Arthur} (15.\,5.\,1862 Wien – 21.\,10.\,1931 ebd.), \emph{Schriftsteller, Mediziner}!Liebelei. Schauspiel in drei Akten@\strich\emph{Liebelei. Schauspiel in drei Akten}|pwk}-Aufführungen am Berlin\oindex{Berlin@\textbf{Berlin}, \emph{Hauptstadt}|pwk}er \emph{Lessing-Theater}\orgindex{Lessing-Theater@Lessing-Theater|pwk}.}}}\label{K_L02910-3} berichtet. Aber am
               erſten Abend war eine blödſinnige \label{K_L02910-4v}\edtext{\begin{otherlanguage}{french}\textsc{Première\pwindex{König, Eberhard 18.\,6.\,1871 Grünberg – 26.\,12.\,1949 Berlin@\textsc{König, Eberhard} (18.\,6.\,1871 Grünberg – 26.\,12.\,1949 Berlin), \emph{Schriftsteller}!Gevatter Tod. Ein Märchen von der Menschheit. Drama in fünf Aufzügen@\strich\emph{Gevatter Tod. Ein Märchen von der Menschheit. Drama in fünf Aufzügen}|pwv}}\end{otherlanguage}}{\lemma{\textnormal{\emph{Première}}}\Cendnote{\textnormal{von Eberhard Königs\pwindex{König, Eberhard 18.\,6.\,1871 Grünberg – 26.\,12.\,1949 Berlin@\textsc{König, Eberhard} (18.\,6.\,1871 Grünberg – 26.\,12.\,1949 Berlin), \emph{Schriftsteller}|pwk} Fünfakter \emph{Gevatter Tod.
                     Ein Märchen von der Menschheit}\pwindex{König, Eberhard 18.\,6.\,1871 Grünberg – 26.\,12.\,1949 Berlin@\textsc{König, Eberhard} (18.\,6.\,1871 Grünberg – 26.\,12.\,1949 Berlin), \emph{Schriftsteller}!Gevatter Tod. Ein Märchen von der Menschheit. Drama in fünf Aufzügen@\strich\emph{Gevatter Tod. Ein Märchen von der Menschheit. Drama in fünf Aufzügen}|pwk}}}}\label{K_L02910-4} im Schauſpielhauſe\oindex{Schauspielhaus Berlin@\textbf{Schauspielhaus Berlin}, \emph{Theater}|pw}; und am zweiten konnte ich auch nicht hineingehen. Es{ }ſteht in
               den Sternen geſchrieben, daß ich nie ein Stück von Dir auf der Bühne{ }ſehen{ }ſoll.\pend
           
\pstart
           Viele treue Grüße! {\\[\baselineskip]}Dein {\\[\baselineskip]}\spacefill\mbox{Paul Goldmann.}\pend
           \leftskip=0em{}\selectlanguage{ngerman}\endnumbering\briefempfaengerindex{Schnitzler, Arthur@\textsc{Schnitzler, Arthur}!zzzGoldmann, Paul@\emph{von Paul Goldmann}!1900-04-131@{13. 4. [1900]}|)be}\mylabel{L02910h}  \newcommand{\dateiname}{L02910}\newcommand{\titel}{Paul Goldmann an Arthur Schnitzler, 13. 4. [1900]}\newcommand{\editorInnen}{Martin Anton Müller und Laura Untner}%% latex-leseansicht-abspann.tex
%% Abspann für die Leseansicht.
%% Der Schalter \ifkorrekturansicht ist bereits durch den Vorspann gesetzt.

%% latex-abspann.tex
%% Gemeinsamer Abspann für Korrekturansicht und Leseansicht.
%% Setzt den Schalter \ifkorrekturansicht voraus (gesetzt in den
%% einbindenden Dateien latex-korrekturansicht-abspann.tex bzw.
%% latex-leseansicht-abspann.tex).
%% ---------------------------------------------------------------

\normalsize

% Das esempio-Environment wird nur in der Leseansicht benötigt
\ifkorrekturansicht\else
\newenvironment{esempio}[3]%
{
    \vspace{1.5ex}
    \rlap{\underline{#1}}
    \par
    \setlength{\parindent}{0cm}
    \nopagebreak
    \leftskip=#2cm
    \rightskip=#3cm
}
{
    \par
}
\fi

\doendnotes{C}
\bigskip
\vfill

\clearpage

\footnotesize

\ifkorrekturansicht
  \lohead{\textsc{register}}
\fi

% theindex-Environment neu definieren ohne reledmac
\makeatletter
\renewenvironment{theindex}{%
  \ifkorrekturansicht
    \section*{\indexname}%
  \else
    \subsubsection*{Index der erwähnten Entitäten}%
  \fi
  \setlength{\parindent}{0pt}%
  \setlength{\parskip}{0pt plus 0.3pt}%
  \let\item\@idxitem
}{%
  \ifkorrekturansicht\clearpage\fi
}
\makeatother

\IfFileExists{\jobname-pw.ind}{\input{\jobname-pw.ind}}{}

% Quellenangabe nur in der Leseansicht
\ifkorrekturansicht\else
% Fallback-Definitionen, falls die .tex-Datei \titel etc. nicht gesetzt hat
\providecommand{\titel}{}
\providecommand{\editorInnen}{}
\providecommand{\dateiname}{\jobname}

\vspace{3cm}

\vfill

\footnotesize
\textsc{Quelle}: \titel. Herausgegeben von {\editorInnen}. In: \emph{Arthur Schnitzler: Briefwechsel mit Autorinnen und Autoren}.
 Digitale Edition, https://schnitzler-briefe.acdh.oeaw.ac.at/{\dateiname}.html (Stand \today)
\fi

\end{document}


