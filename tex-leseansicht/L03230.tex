%% latex-leseansicht-vorspann.tex
%% Vorspann für die Leseansicht.
%% Lädt die gemeinsame Datei latex-vorspann.tex mit nicht gesetztem Schalter.

\newif\ifkorrekturansicht
\korrekturansichtfalse

\input{../tex-inputs/latex-vorspann}

\begin{center}
            \textcolor{red}{ENTWURF, NICHT FERTIG KORRIGIERT}
                      \end{center}
            
         \renewcommand{\erwaehnteOrte}{Orte: Berlin, Dessauer Straße, Wien}
         \renewcommand{\erwaehnteWerke}{Werke: Der Schleier der Beatrice. Schauspiel in fünf Akten}
               \section[ Paul Goldmann an Arthur Schnitzler, 24. 11. {[}1902{]}]{ Paul Goldmann an Arthur Schnitzler, 24. 11. {[}1902{]}}\nopagebreak\mylabel{v}\rehead{ }\begin{ledgroupsized}[t]{13cm}\normalsize\beginnumbering \toendnotes[C]{\smallbreak\pagebreak[2]} \Standort{DLA, A:Schnitzler, HS.NZ85.1.3172.}
\physDesc{Brief, 1 Blatt, 4 Seiten
\newline{}Handschrift: schwarze Tinte, deutsche Kurrent
\newline{}Schnitzler: 1) mit Bleistift das Jahr »{[}1{]}902«
                                            vermerkt  2) mit rotem Buntstift vier Unterstreichungen}\pstart
           \noindent{}\raggedleft{}{\pb}\textcolor{gray}{\textbf{DESSAUERSTRASSE 19}}\oindex{Dessauer Strasse@\textbf{Dessauer Straße}|pw}\pend
           \pstart
           Berlin\oindex{Berlin@\textbf{Berlin}|pw}, 24. November.\pend
           \pstart\center{}Mein lieber Freund,\pend\pstart
           Der Beifall, den Du in ſo gütigen Worten meinem Feuilleton\textcolor{red}{\textsuperscript{\textbf{KEY}}} ſpendeſt, hat mich innig erfreut, und ich Danke Dir von
                    Herzen dafür. \pend
           \pstart
           Dein lieber Brief, den ich Samſtag empfing, iſt nicht beſonders
                    erfreulich. Warum ſo mißgelaunt? Wer wird ſich ſo vom Wetter abhängig machen?
                    Und wenn es gegenwärtig mit dem Produziren nicht recht geht, ſo wird ſchon {\pb} wieder ein produktiver Zuſtand kommen. Der
                    Geiſt ſammelt eben neue Kraft. \pend
           \pstart
           Was iſt mit der »\textsc{Beatrice\pwindex{Schnitzler, Arthur 15.05.1862 – 21.10.1931@\textsc{Schnitzler, Arthur} (15.05.1862 – 21.10.1931), \emph{Schriftsteller, Mediziner}!Schleier der Beatrice. Schauspiel in fuenf Akten1900-12-01@\strich\emph{Der Schleier der Beatrice. Schauspiel in fünf Akten} {[}1900-12-01{]}|pw}}« und dem »Deutſchen Theater\textcolor{red}{\textsuperscript{\textbf{KEY}}}«? \pend
           \pstart
           Die Bücher, die Du mir empfiehlſt, möchte ich gern leſen; nur wird die Erfüllung
                    dieſes Wunſches an dem Umſtande ſcheitern, daß ich die Namen zumeiſt nicht leſen
                    kann. Insbeſondere von Demjenigen\textcolor{red}{\textsuperscript{\textbf{KEY}}}, den Du mir ans
                    Herz legſt, habe ich trotz eifriger Bemühung nicht mehr herausbekommen können,
                    als daß er mit \textsc{L.}{\pb} anfängt. \pend
           \pstart
           Haſt Du Dir die »\textsc{Maximes de la\textcolor{red}{\textsuperscript{\textbf{KEY}}}}der\textsc{Vie\textcolor{red}{\textsuperscript{\textbf{KEY}}}}« \textsc{\strikeout{der}}\textsc{Comtesse Diane\textcolor{red}{\textsuperscript{\textbf{KEY}}}} kommen laſſen? Noch ſchöner vielleicht iſt das \textsc{Livre d’or\textcolor{red}{\textsuperscript{\textbf{KEY}}}} von derſelben, – ein entzückendes Spiel des Geiſtes und zugleich eine
                    Quelle tiefer Lebensweisheit. \pend
           \pstart
           Was \textsc{Sudermann\textcolor{red}{\textsuperscript{\textbf{KEY}}}} anlangt, bin ich durchaus Deiner Anſicht. Vielleicht ergreife ich in dem
                    Streit noch das Wort, obwohl mir Andere gerade das, was ich ſagen möchte,
                    weggeſchrieben {\pb} haben. \textsc{Kerr\textcolor{red}{\textsuperscript{\textbf{KEY}}}s}Erwiderung\textcolor{red}{\textsuperscript{\textbf{KEY}}} war zum Theil hübſch in der Form, aber
                    der Geſinnung nach lausbübiſch, wie überhaupt ein Lausbuben-Zug immer ſtärker
                    bei ihm hervortritt. \textsc{Harden\textcolor{red}{\textsuperscript{\textbf{KEY}}}} war, im erſten Theil ſeiner Erwiderung\textcolor{red}{\textsuperscript{\textbf{KEY}}}, viel
                    bedeutender; im zweiten ſpricht er zu viel und zu eitel von ſich. \pend
           \pstart
           Fräulein \textsc{Eva F.\textcolor{red}{\textsuperscript{\textbf{KEY}}}} iſt hier. Ich habe ſie einmal geſehen und in den erſten fünf Minnten den
                    Eindruck gehabt: »Es iſt unmöglich.« Es iſt beinahe eine phyſiſche Antipathie,
                    die ich nicht werde überwinden können. {\\[\baselineskip]}Grüße \textsc{Heinrich\textcolor{red}{\textsuperscript{\textbf{KEY}}}} und ſeine Mutter\textcolor{red}{\textsuperscript{\textbf{KEY}}}\pend
           \leftskip=0em{}\pstart
           {\\[\baselineskip]}und ſei Du ſelbſt vielmals gegrüßt\pend
           \leftskip=0em{}\pstart
           {\\[\baselineskip]}von Deinem \spacefill\mbox{Paul Goldmn }\pend
           \leftskip=0em{}
         
         \endnumbering\mylabel{h}\end{ledgroupsized}\begin{anhang}\end{anhang}\newcommand{\dateiname}{L03230}\newcommand{\titel}{Paul Goldmann an Arthur Schnitzler, 24. 11. [1902]}\newcommand{\editorInnen}{Martin Anton Müller und Laura Untner}%% latex-leseansicht-abspann.tex
%% Abspann für die Leseansicht.
%% Der Schalter \ifkorrekturansicht ist bereits durch den Vorspann gesetzt.

%% latex-abspann.tex
%% Gemeinsamer Abspann für Korrekturansicht und Leseansicht.
%% Setzt den Schalter \ifkorrekturansicht voraus (gesetzt in den
%% einbindenden Dateien latex-korrekturansicht-abspann.tex bzw.
%% latex-leseansicht-abspann.tex).
%% ---------------------------------------------------------------

\normalsize

% Das esempio-Environment wird nur in der Leseansicht benötigt
\ifkorrekturansicht\else
\newenvironment{esempio}[3]%
{
    \vspace{1.5ex}
    \rlap{\underline{#1}}
    \par
    \setlength{\parindent}{0cm}
    \nopagebreak
    \leftskip=#2cm
    \rightskip=#3cm
}
{
    \par
}
\fi

\doendnotes{C}
\bigskip
\vfill

\clearpage

\footnotesize

\ifkorrekturansicht
  \lohead{\textsc{register}}
\fi

% theindex-Environment neu definieren ohne reledmac
\makeatletter
\renewenvironment{theindex}{%
  \ifkorrekturansicht
    \section*{\indexname}%
  \else
    \subsubsection*{Index der erwähnten Entitäten}%
  \fi
  \setlength{\parindent}{0pt}%
  \setlength{\parskip}{0pt plus 0.3pt}%
  \let\item\@idxitem
}{%
  \ifkorrekturansicht\clearpage\fi
}
\makeatother

\IfFileExists{\jobname-pw.ind}{\input{\jobname-pw.ind}}{}

% Quellenangabe nur in der Leseansicht
\ifkorrekturansicht\else
% Fallback-Definitionen, falls die .tex-Datei \titel etc. nicht gesetzt hat
\providecommand{\titel}{}
\providecommand{\editorInnen}{}
\providecommand{\dateiname}{\jobname}

\vspace{3cm}

\vfill

\footnotesize
\textsc{Quelle}: \titel. Herausgegeben von {\editorInnen}. In: \emph{Arthur Schnitzler: Briefwechsel mit Autorinnen und Autoren}.
 Digitale Edition, https://schnitzler-briefe.acdh.oeaw.ac.at/{\dateiname}.html (Stand \today)
\fi

\end{document}


      