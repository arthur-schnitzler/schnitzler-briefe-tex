%% latex-leseansicht-vorspann.tex
%% Vorspann für die Leseansicht.
%% Lädt die gemeinsame Datei latex-vorspann.tex mit nicht gesetztem Schalter.

\newif\ifkorrekturansicht
\korrekturansichtfalse

\input{../tex-inputs/latex-vorspann}


         
         \renewcommand{\erwaehntePersonen}{Personen: Marie Suin Beausacq, Paul Goldmann, Eva Marie Goldmann, Maximilian Harden, Alfred Kerr, Heinrich Schnitzler, Olga Schnitzler, Hermann Sudermann}
         \renewcommand{\erwaehnteInstitutionen}{Institutionen: Deutsches Theater Berlin, Helianthus}
         \renewcommand{\erwaehnteOrte}{Orte: Berlin, Dessauer Straße, Paris, Wien}
         \renewcommand{\erwaehnteWerke}{Werke: Berliner Theater. (»König Laurin« von Ernst v. Wildenbruch.), Der Schleier der Beatrice. Schauspiel in fünf Akten, Der Tag, Die Kritik und Herr Sudermann, Die Zukunft, Herr Sudermann, der D .. Di .. Dichter. Ein kritisches Vademecum, Livre d’or de la comtesse Diane, préface par Gaston Bergeret, Maximes de la vie. Préface par Sully Prud’homme, Neue Freie Presse, Theater [Erwiderung auf Sudermanns Verrohung in der Literaturkritik, II], Theater [Erwiderung auf Sudermanns Verrohung in der Literaturkritik], Verrohung in der Theaterkritik}
               \section[ Paul Goldmann an Arthur Schnitzler, 24. 11. {[}1902{]}]{ Paul Goldmann an Arthur Schnitzler, 24. 11. {[}1902{]}}\nopagebreak\mylabel{v}\rehead{ }\begin{ledgroupsized}[t]{13cm}\normalsize\beginnumbering \toendnotes[C]{\smallbreak\pagebreak[2]} \Standort{DLA, A:Schnitzler, HS.NZ85.1.3172.}
\physDesc{Brief, 1 Blatt, 4 Seiten, 1764 Zeichen
\newline{}Handschrift: blaue Tinte, deutsche Kurrent
\newline{}Schnitzler: 1) mit Bleistift das Jahr »902« vermerkt  2) mit rotem Buntstift vier Unterstreichungen}\toendnotes[C]{\smallbreak}\pstart
           \noindent{}\raggedleft{}{\pb}\textcolor{gray}{\textbf{DESSAUERSTRASSE 19}}\oindex{Dessauer Strasse@\textbf{Dessauer Straße}|pw}\pend
           \pstart
           Berlin\oindex{Berlin@\textbf{Berlin}|pw}, 24. November.\pend
           \pstart{}Mein lieber Freund,\pend\pstart
           Der Beifall, den Du in ſo gütigen Worten meinem \label{K_L03230-1v}\edtext{Feuilleton\pwindex{Goldmann, Paul 31.01.1865 – 25.09.1935@\textsc{Goldmann, Paul} (31.01.1865 – 25.09.1935), \emph{Schriftsteller, Journalist}!Berliner Theater. (»Koenig Laurin« von Ernst v. Wildenbruch.)1902-11-22@\strich\emph{Berliner Theater. (»König Laurin« von Ernst v. Wildenbruch.)} {[}1902-11-22{]}|pwv}}{\lemma{\textnormal{\emph{Feuilleton}}}\Cendnote{\textnormal{Paul Goldmann\pwindex{Goldmann, Paul 31.01.1865 – 25.09.1935@\textsc{Goldmann, Paul} (31.01.1865 – 25.09.1935), \emph{Schriftsteller, Journalist}|pwk}: \emph{Berliner Theater. (»König Laurin« von Ernst v.
                        Wildenbruch.)}\pwindex{Goldmann, Paul 31.01.1865 – 25.09.1935@\textsc{Goldmann, Paul} (31.01.1865 – 25.09.1935), \emph{Schriftsteller, Journalist}!Berliner Theater. (»Koenig Laurin« von Ernst v. Wildenbruch.)1902-11-22@\strich\emph{Berliner Theater. (»König Laurin« von Ernst v. Wildenbruch.)} {[}1902-11-22{]}|pwk}. In: \emph{Neue Freie
                        Presse}\pwindex{Neue Freie Presse1864 – 1939@\emph{Neue Freie Presse} {[}1864 – 1939{]}|pwk}, Nr. 13.737, 22. 11. 1902,
                     Morgenblatt, S. 1–4. Die Reihenfolge, in der der Dank in diesem Brief
                  ausgesprochen wurde, legt nahe, dass Schnitzler\pwindex{Schnitzler, Arthur 15.05.1862 – 21.10.1931@\textsc{Schnitzler, Arthur} (15.05.1862 – 21.10.1931), \emph{Schriftsteller, Mediziner}|pwk} seine Gratulation in einem separaten Schreiben, möglicherweise
                  einem Telegramm oder einer Karte ausdrückte.}}}\label{K_L03230-1h} ſpendeſt, hat mich innig
               erfreut, und ich Danke Dir von Herzen dafür.\pend
           \pstart
           Dein lieber Brief, den ich \label{K_L03230-2v}\edtext{Samſtag}{\lemma{\textnormal{\emph{Samſtag}}}\Cendnote{\textnormal{22. 11. 1902}}}\label{K_L03230-2h} empfing, iſt nicht beſonders erfreulich. Warum ſo \label{K_L03230-3v}\edtext{mißgelaunt}{\lemma{\textnormal{\emph{mißgelaunt}}}\Cendnote{\textnormal{Schnitzler\pwindex{Schnitzler, Arthur 15.05.1862 – 21.10.1931@\textsc{Schnitzler, Arthur} (15.05.1862 – 21.10.1931), \emph{Schriftsteller, Mediziner}|pwk} plagten in dieser Zeit
                  Nervosität, Arbeitsunfähigkeit und Zukunftsängste, vgl. A. S.: \emph{Tagebuch}, 12. 11. 1902, 13. 11. 1902, 14. 11. 1902, 20. 11. 1902 und 23. 11. 1902.}}}\label{K_L03230-3h}? Wer wird ſich ſo vom Wetter
               abhängig machen? Und wenn es gegenwärtig mit dem Produziren nicht recht geht, ſo wird
               ſchon {\pb}wieder ein produktiver Zuſtand kommen. Der
               Geiſt ſammelt eben neue Kraft.\pend
           \pstart
           Was iſt mit der \label{K_L03230-4v}\edtext{»\textsc{Beatrice\pwindex{Schnitzler, Arthur 15.05.1862 – 21.10.1931@\textsc{Schnitzler, Arthur} (15.05.1862 – 21.10.1931), \emph{Schriftsteller, Mediziner}!Schleier der Beatrice. Schauspiel in fuenf Akten1900-12-01@\strich\emph{Der Schleier der Beatrice. Schauspiel in fünf Akten} {[}1900-12-01{]}|pw}}« und dem »Deutſchen Theater\orgindex{Deutsches Theater Berlin@Deutsches Theater Berlin|pw}«}{\lemma{\textnormal{\emph{»Beatrice« … Theater«}}}\Cendnote{\textnormal{siehe Paul Goldmann an Arthur Schnitzler, 16. 6. [1902]}}}\label{K_L03230-4h}?\pend
           \pstart
           Die \label{K_L03230-5v}\edtext{Bücher}{\lemma{\textnormal{\emph{Bücher}}}\Cendnote{\textnormal{nicht ermittelt}}}\label{K_L03230-5h}, die Du mir empfiehlſt, möchte ich
               gern leſen; nur wird die Erfüllung dieſes Wunſches an dem Umſtande ſcheitern, daß ich
               die Namen zumeiſt nicht leſen kann. Insbeſondere von Demjenigen, den Du mir ans Herz
               legſt, habe ich trotz eifriger Bemühung nicht mehr herausbekommen können, als daß er
               mit \textsc{L.}{ }{\pb}anfängt.\pend
           \pstart
           Haſt Du Dir die »\textsc{Maximes de la Vie\pwindex{Beausacq, Marie Suin 1829-10-03 – 1899-12-19@\textsc{Beausacq, Marie Suin} (1829-10-03 – 1899-12-19), \emph{Schriftstellerin}!Maximes de la vie. Preface par Sully PruDhomme1883@\strich\emph{Maximes de la vie. Préface par Sully Prud’homme} {[}1883{]}|pw}}« \substVorne{}\textsuperscript{\textsc{der}}\substDazwischen{}der\substHinten{}{ }\label{K_L03230-6v}\edtext{\textsc{Comtesse Diane\pwindex{Beausacq, Marie Suin 1829-10-03 – 1899-12-19@\textsc{Beausacq, Marie Suin} (1829-10-03 – 1899-12-19), \emph{Schriftstellerin}|pw}}}{\lemma{\textnormal{\emph{Comtesse Diane}}}\Cendnote{\textnormal{Zu \emph{Maximes de la vie}\pwindex{Beausacq, Marie Suin 1829-10-03 – 1899-12-19@\textsc{Beausacq, Marie Suin} (1829-10-03 – 1899-12-19), \emph{Schriftstellerin}!Maximes de la vie. Preface par Sully PruDhomme1883@\strich\emph{Maximes de la vie. Préface par Sully Prud’homme} {[}1883{]}|pwk}{ }siehe Paul Goldmann an Arthur Schnitzler, 2. [10. 1902]. Auch eine Lektüre
                  von \emph{Livre d’or}\pwindex{Beausacq, Marie Suin 1829-10-03 – 1899-12-19@\textsc{Beausacq, Marie Suin} (1829-10-03 – 1899-12-19), \emph{Schriftstellerin}!Livre Dor de la comtesse Diane, preface par Gaston Bergeret1886@\strich\emph{Livre d’or de la comtesse Diane, préface par Gaston Bergeret} {[}1886{]}|pwk} (Paris\oindex{Paris@\textbf{Paris}|pwk}{ }1886) ist nicht nachweisbar.}}}\label{K_L03230-6h} kommen laſſen? Noch
               ſchöner vielleicht iſt das \textsc{Livre d’or\pwindex{Beausacq, Marie Suin 1829-10-03 – 1899-12-19@\textsc{Beausacq, Marie Suin} (1829-10-03 – 1899-12-19), \emph{Schriftstellerin}!Livre Dor de la comtesse Diane, preface par Gaston Bergeret1886@\strich\emph{Livre d’or de la comtesse Diane, préface par Gaston Bergeret} {[}1886{]}|pw}} von derſelben, – ein entzückendes Spiel des Geiſtes und zugleich eine Quelle
               tiefer Lebensweisheit.\pend
           \pstart
           Was \label{K_L03230-7v}\edtext{\textsc{Sudermann\pwindex{Sudermann, Hermann 30.09.1857 – 21.11.1928@\textsc{Sudermann, Hermann} (30.09.1857 – 21.11.1928), \emph{Schriftsteller}!Verrohung in der Theaterkritik1902-10-30 – 1902-12-01@\strich\emph{Verrohung in der Theaterkritik} {[}1902-10-30 – 1902-12-01{]}|pwv}\pwindex{Sudermann, Hermann 30.09.1857 – 21.11.1928@\textsc{Sudermann, Hermann} (30.09.1857 – 21.11.1928), \emph{Schriftsteller}|pw}}}{\lemma{\textnormal{\emph{Sudermann}}}\Cendnote{\textnormal{siehe Paul Goldmann an Arthur Schnitzler, 10. 11. [1902]}}}\label{K_L03230-7h} anlangt, bin ich durchaus Deiner Anſicht. Vielleicht ergreife ich in dem
               Streit noch das \label{K_L03230-8v}\edtext{Wort}{\lemma{\textnormal{\emph{Wort}}}\Cendnote{\textnormal{ein solches Feuilleton ist nicht
                  bekannt}}}\label{K_L03230-8h}, obwohl mir Andere gerade das, was ich ſagen möchte, weggeſchrieben
                  {\pb}haben. \label{K_L03230-9v}\edtext{\textsc{Kerr\pwindex{Kerr, Alfred 25.12.1867 – 12.10.1948@\textsc{Kerr, Alfred} (25.12.1867 – 12.10.1948), \emph{Schriftsteller, Kritiker}|pw}s}{ }Erwiderung\pwindex{Kerr, Alfred 25.12.1867 – 12.10.1948@\textsc{Kerr, Alfred} (25.12.1867 – 12.10.1948), \emph{Schriftsteller, Kritiker}!Kritik und Herr Sudermann1902-11-21@\strich\emph{Die Kritik und Herr Sudermann} {[}1902-11-21{]}|pwv}}{\lemma{\textnormal{\emph{Kerrs Erwiderung}}}\Cendnote{\textnormal{Alfred Kerr\pwindex{Kerr, Alfred 25.12.1867 – 12.10.1948@\textsc{Kerr, Alfred} (25.12.1867 – 12.10.1948), \emph{Schriftsteller, Kritiker}|pwk}: \emph{Die Kritik und Herr Sudermann}\pwindex{Kerr, Alfred 25.12.1867 – 12.10.1948@\textsc{Kerr, Alfred} (25.12.1867 – 12.10.1948), \emph{Schriftsteller, Kritiker}!Kritik und Herr Sudermann1902-11-21@\strich\emph{Die Kritik und Herr Sudermann} {[}1902-11-21{]}|pwk}. In: \emph{Der Tag}\pwindex{?? Werk@Nicht ermittelte Verfasserinnen und Verfasser!Tag19.12.1900 – 1934@\emph{Der Tag} {[}19.12.1900 – 1934{]}|pwk}, Nr. 545, 21. 11. 1902, S. [1–3]. Weitgehend parallel dazu, wenngleich
                  auf 1903 vordatiert, erschien dieser Text\pwindex{Kerr, Alfred 25.12.1867 – 12.10.1948@\textsc{Kerr, Alfred} (25.12.1867 – 12.10.1948), \emph{Schriftsteller, Kritiker}!Kritik und Herr Sudermann1902-11-21@\strich\emph{Die Kritik und Herr Sudermann} {[}1902-11-21{]}|pwkv} gemeinsam mit gesammelten Kritiken
                     Kerr\pwindex{Kerr, Alfred 25.12.1867 – 12.10.1948@\textsc{Kerr, Alfred} (25.12.1867 – 12.10.1948), \emph{Schriftsteller, Kritiker}|pwk}s zu Sudermann\pwindex{Sudermann, Hermann 30.09.1857 – 21.11.1928@\textsc{Sudermann, Hermann} (30.09.1857 – 21.11.1928), \emph{Schriftsteller}|pwk}s Stücken als Broschüre: Alfred Kerr\pwindex{Kerr, Alfred 25.12.1867 – 12.10.1948@\textsc{Kerr, Alfred} (25.12.1867 – 12.10.1948), \emph{Schriftsteller, Kritiker}|pwk}: \emph{Herr Sudermann, der D .. Di .. Dichter. Ein kritisches
                        Vademecum}\pwindex{Kerr, Alfred 25.12.1867 – 12.10.1948@\textsc{Kerr, Alfred} (25.12.1867 – 12.10.1948), \emph{Schriftsteller, Kritiker}!Herr Sudermann, der D .. Di .. Dichter. Ein kritisches VademecumEnde November 1902@\strich\emph{Herr Sudermann, der D .. Di .. Dichter. Ein kritisches Vademecum} {[}Ende November 1902{]}|pwk}. Berlin\oindex{Berlin@\textbf{Berlin}|pwk}: \emph{Helianthus}\orgindex{Helianthus@Helianthus|pwk}{ }1903. Die Vorbemerkung zur dritten Auflage – wohl zu lesen als 3. und 4.
                  Tausend – ist mit dem 6. 12. 1902 datiert. }}}\label{K_L03230-9h}
               war zum Theil hübſch in der Form, aber der Geſinnung nach lausbübiſch, wie überhaupt
               ein Lausbuben-Zug immer ſtärker bei ihm hervortritt. \textsc{Harden\pwindex{Harden, Maximilian 20.10.1861 – 30.10.1927@\textsc{Harden, Maximilian} (20.10.1861 – 30.10.1927), \emph{Schriftsteller, Publizist}|pw}} war, im erſten Theil ſeiner \label{K_L03230-10v}\edtext{Erwiderung\pwindex{Theater [Erwiderung auf Sudermanns Verrohung in der Literaturkritik]1902-11-22@\emph{Theater [Erwiderung auf Sudermanns Verrohung in der Literaturkritik]} {[}1902-11-22{]}|pwv}}{\lemma{\textnormal{\emph{Erwiderung}}}\Cendnote{\textnormal{M. H.\pwindex{Harden, Maximilian 20.10.1861 – 30.10.1927@\textsc{Harden, Maximilian} (20.10.1861 – 30.10.1927), \emph{Schriftsteller, Publizist}|pwkv} [ = Maximilian Harden\pwindex{Harden, Maximilian 20.10.1861 – 30.10.1927@\textsc{Harden, Maximilian} (20.10.1861 – 30.10.1927), \emph{Schriftsteller, Publizist}|pwk}]: \emph{Theater}\pwindex{Theater [Erwiderung auf Sudermanns Verrohung in der Literaturkritik]1902-11-22@\emph{Theater [Erwiderung auf Sudermanns Verrohung in der Literaturkritik]} {[}1902-11-22{]}|pwk}. In: \emph{Die
                        Zukunft}\pwindex{Zukunft1892 – 1922@\emph{Die Zukunft} {[}1892 – 1922{]}|pwk}, Bd. 41, 22. 11. 1902,
                     S. 311–326. (Der zweite Teil\pwindex{Harden, Maximilian 20.10.1861 – 30.10.1927@\textsc{Harden, Maximilian} (20.10.1861 – 30.10.1927), \emph{Schriftsteller, Publizist}!Theater [Erwiderung auf Sudermanns Verrohung in der Literaturkritik, II]1902-11-29@\strich\emph{Theater [Erwiderung auf Sudermanns Verrohung in der Literaturkritik, II]} {[}1902-11-29{]}|pwkv} erschien in der Folgewoche, 29. 11. 1902, S. 356–370.)}}}\label{K_L03230-10h}, viel bedeutender; im
               zweiten ſpricht er zu viel und zu eitel von ſich.\pend
           \pstart
           Fräulein \label{K_L03230-11v}\edtext{\textsc{Eva F.\pwindex{Goldmann, Eva Marie 27.10.1877 – 02.11.1937@\textsc{Goldmann, Eva Marie} (27.10.1877 – 02.11.1937)|pw}}}{\lemma{\textnormal{\emph{Eva F.}}}\Cendnote{\textnormal{Eva Fränkel\pwindex{Goldmann, Eva Marie 27.10.1877 – 02.11.1937@\textsc{Goldmann, Eva Marie} (27.10.1877 – 02.11.1937)|pwk}, Goldmann\pwindex{Goldmann, Paul 31.01.1865 – 25.09.1935@\textsc{Goldmann, Paul} (31.01.1865 – 25.09.1935), \emph{Schriftsteller, Journalist}|pwk}s spätere Ehefrau, die Schnitzler\pwindex{Schnitzler, Arthur 15.05.1862 – 21.10.1931@\textsc{Schnitzler, Arthur} (15.05.1862 – 21.10.1931), \emph{Schriftsteller, Mediziner}|pwk} bereits kannte}}}\label{K_L03230-11h} iſt hier. Ich habe ſie
               einmal geſehen und in den erſten fünf Minuten den Eindruck gehabt: »Es iſt
               unmöglich.« Es iſt beinahe eine phyſiſche Antipathie, die ich nicht werde überwinden
               können.\pend
           \pstart
           Grüße \textsc{Heinrich\pwindex{Schnitzler, Heinrich 09.08.1902 – 12.07.1982@\textsc{Schnitzler, Heinrich} (09.08.1902 – 12.07.1982), \emph{Regisseur, Schauspieler}|pw}} und ſeine Mutter\pwindex{Schnitzler, Olga 17.01.1882 – 13.01.1970@\textsc{Schnitzler, Olga} (17.01.1882 – 13.01.1970), \emph{Schauspielerin, Sängerin}|pwv} und
               ſei Du ſelbſt vielmals gegrüßt {\\[\baselineskip]}von Deinem \spacefill\mbox{Paul Goldmn}\pend
           \leftskip=0em{}
         
         \endnumbering\mylabel{h}\end{ledgroupsized}  \newcommand{\dateiname}{L03230}\newcommand{\titel}{Paul Goldmann an Arthur Schnitzler, 24. 11. [1902]}\newcommand{\editorInnen}{Martin Anton Müller und Laura Untner}%% latex-leseansicht-abspann.tex
%% Abspann für die Leseansicht.
%% Der Schalter \ifkorrekturansicht ist bereits durch den Vorspann gesetzt.

%% latex-abspann.tex
%% Gemeinsamer Abspann für Korrekturansicht und Leseansicht.
%% Setzt den Schalter \ifkorrekturansicht voraus (gesetzt in den
%% einbindenden Dateien latex-korrekturansicht-abspann.tex bzw.
%% latex-leseansicht-abspann.tex).
%% ---------------------------------------------------------------

\normalsize

% Das esempio-Environment wird nur in der Leseansicht benötigt
\ifkorrekturansicht\else
\newenvironment{esempio}[3]%
{
    \vspace{1.5ex}
    \rlap{\underline{#1}}
    \par
    \setlength{\parindent}{0cm}
    \nopagebreak
    \leftskip=#2cm
    \rightskip=#3cm
}
{
    \par
}
\fi

\doendnotes{C}
\bigskip
\vfill

\clearpage

\footnotesize

\ifkorrekturansicht
  \lohead{\textsc{register}}
\fi

% theindex-Environment neu definieren ohne reledmac
\makeatletter
\renewenvironment{theindex}{%
  \ifkorrekturansicht
    \section*{\indexname}%
  \else
    \subsubsection*{Index der erwähnten Entitäten}%
  \fi
  \setlength{\parindent}{0pt}%
  \setlength{\parskip}{0pt plus 0.3pt}%
  \let\item\@idxitem
}{%
  \ifkorrekturansicht\clearpage\fi
}
\makeatother

\IfFileExists{\jobname-pw.ind}{\input{\jobname-pw.ind}}{}

% Quellenangabe nur in der Leseansicht
\ifkorrekturansicht\else
% Fallback-Definitionen, falls die .tex-Datei \titel etc. nicht gesetzt hat
\providecommand{\titel}{}
\providecommand{\editorInnen}{}
\providecommand{\dateiname}{\jobname}

\vspace{3cm}

\vfill

\footnotesize
\textsc{Quelle}: \titel. Herausgegeben von {\editorInnen}. In: \emph{Arthur Schnitzler: Briefwechsel mit Autorinnen und Autoren}.
 Digitale Edition, https://schnitzler-briefe.acdh.oeaw.ac.at/{\dateiname}.html (Stand \today)
\fi

\end{document}


      