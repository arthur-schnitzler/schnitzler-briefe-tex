%% latex-leseansicht-vorspann.tex
%% Vorspann für die Leseansicht.
%% Lädt die gemeinsame Datei latex-vorspann.tex mit nicht gesetztem Schalter.

\newif\ifkorrekturansicht
\korrekturansichtfalse

\input{../tex-inputs/latex-vorspann}


\section[ Paul Goldmann an Arthur Schnitzler, 24. 11. {[}1902{]}]{L03230 Paul Goldmann an Arthur Schnitzler,  24. 11. [1902]}
\nopagebreak\mylabel{L03230v}
\rehead{ }\normalsize\beginnumbering\briefempfaengerindex{Schnitzler, Arthur@\textsc{Schnitzler, Arthur}!zzzGoldmann, Paul@\emph{von Paul Goldmann}!1902-11-242@{24. 11. [1902]}|(be}
\toendnotes[C]{\smallbreak\pagebreak[2]}
\correspDesc{Versand  durch Paul Goldmann am 24. 11. [1902] in Berlin
\newline{}Erhalt  durch Arthur Schnitzler im Zeitraum [25. 11. 1902 – 29. 11. 1902?] in Wien}\toendnotes[C]{\smallbreak}
\Standort{DLA, A:Schnitzler, HS.NZ85.1.3172.}
\physDesc{Brief, 1 Blatt, 4 Seiten, 1764 Zeichen
\newline{}Handschrift: blaue Tinte, deutsche Kurrent
\newline{}Schnitzler: 1) mit Bleistift das Jahr »902« vermerkt  2) mit rotem Buntstift vier Unterstreichungen}\toendnotes[C]{\smallbreak}
\pstart
           \raggedleft{}{\pb}\textcolor{gray}{\textbf{DESSAUERSTRASSE 19}}\oindex{Dessauer Straße@\textbf{Dessauer Straße}, \emph{Straße}|pw}\pend
           
\pstart
           Berlin\oindex{Berlin@\textbf{Berlin}, \emph{Hauptstadt}|pw}, 24. November.\pend
           
\pstart{}Mein lieber Freund,\pend\vspace{0.5em}
\pstart
           Der Beifall, den Du in{ }ſo gütigen Worten meinem \label{K_L03230-1v}\edtext{Feuilleton\pwindex{Goldmann, Paul 31.\,1.\,1865 Breslau – 25.\,9.\,1935 Wien@\textsc{Goldmann, Paul} (31.\,1.\,1865 Breslau – 25.\,9.\,1935 Wien), \emph{Schriftsteller, Journalist}!Berliner Theater. (»König Laurin« von Ernst v. Wildenbruch.)@\strich\emph{Berliner Theater. (»König Laurin« von Ernst v. Wildenbruch.)}|pwv}}{\lemma{\textnormal{\emph{Feuilleton}}}\Cendnote{\textnormal{Paul Goldmann\pwindex{Goldmann, Paul 31.\,1.\,1865 Breslau – 25.\,9.\,1935 Wien@\textsc{Goldmann, Paul} (31.\,1.\,1865 Breslau – 25.\,9.\,1935 Wien), \emph{Schriftsteller, Journalist}|pwk}: \emph{Berliner Theater. (»König Laurin« von Ernst v.
                        Wildenbruch)}\pwindex{Goldmann, Paul 31.\,1.\,1865 Breslau – 25.\,9.\,1935 Wien@\textsc{Goldmann, Paul} (31.\,1.\,1865 Breslau – 25.\,9.\,1935 Wien), \emph{Schriftsteller, Journalist}!Berliner Theater. (»König Laurin« von Ernst v. Wildenbruch.)@\strich\emph{Berliner Theater. (»König Laurin« von Ernst v. Wildenbruch.)}|pwk}. In: \emph{Neue Freie
                        Presse}\pwindex{Neue Freie Presse@\emph{Neue Freie Presse}|pwk}, Nr. 13.737, 22. 11. 1902,
                     Morgenblatt, S. 1–4. Die Reihenfolge, in der der Dank in diesem Brief
                  ausgesprochen wurde, legt nahe, dass Schnitzler seine Gratulation in einem separaten Schreiben, möglicherweise
                  einem Telegramm oder einer Karte ausdrückte.}}}\label{K_L03230-1}{ }ſpendeſt, hat mich innig
               erfreut, und ich Danke Dir von Herzen dafür.\pend
           
\pstart
           Dein lieber Brief, den ich \label{K_L03230-2v}\edtext{Samſtag}{\lemma{\textnormal{\emph{Samstag}}}\Cendnote{\textnormal{22. 11. 1902}}}\label{K_L03230-2} empfing, iſt nicht beſonders erfreulich. Warum{ }ſo \label{K_L03230-3v}\edtext{mißgelaunt}{\lemma{\textnormal{\emph{mißgelaunt}}}\Cendnote{\textnormal{Schnitzler plagten in dieser Zeit
                  Nervosität, Arbeitsunfähigkeit und Zukunftsängste, vgl. A. S.: \emph{Tagebuch}, 12. 11. 1902, 13. 11. 1902, 14. 11. 1902, 20. 11. 1902 und 23. 11. 1902.}}}\label{K_L03230-3}? Wer wird{ }ſich{ }ſo vom Wetter
               abhängig machen? Und wenn es gegenwärtig mit dem Produziren nicht recht geht,{ }ſo wird{ }ſchon {\pb}wieder ein produktiver Zuſtand kommen. Der
               Geiſt{ }ſammelt eben neue Kraft.\pend
           
\pstart
           Was iſt mit der \label{K_L03230-4v}\edtext{»\textsc{Beatrice\pwindex{Schnitzler, Arthur 15.\,5.\,1862 Wien – 21.\,10.\,1931 ebd.@\textsc{Schnitzler, Arthur} (15.\,5.\,1862 Wien – 21.\,10.\,1931 ebd.), \emph{Schriftsteller, Mediziner}!Schleier der Beatrice. Schauspiel in fünf Akten@\strich\emph{Der Schleier der Beatrice. Schauspiel in fünf Akten}|pw}}« und dem »Deutſchen Theater\orgindex{Deutsches Theater Berlin@Deutsches Theater Berlin|pw}«}{\lemma{\textnormal{\emph{»Beatrice« … Theater«}}}\Cendnote{\textnormal{Siehe XXXX Auszeichnungsfehler: Dokument L03211 nicht gefunden.
               }}}\label{K_L03230-4}?\pend
           
\pstart
           Die \label{K_L03230-5v}\edtext{Bücher}{\lemma{\textnormal{\emph{Bücher}}}\Cendnote{\textnormal{nicht ermittelt}}}\label{K_L03230-5}, die Du mir empfiehlſt, möchte ich
               gern leſen; nur wird die Erfüllung dieſes Wunſches an dem Umſtande{ }ſcheitern, daß ich
               die Namen zumeiſt nicht leſen kann. Insbeſondere von Demjenigen, den Du mir ans Herz
               legſt, habe ich trotz eifriger Bemühung nicht mehr herausbekommen können, als daß er
               mit \textsc{L.}{ }{\pb}anfängt.\pend
           
\pstart
           Haſt Du Dir die »\textsc{Maximes de la Vie\pwindex{Beausacq, Marie Suin 3.\,10.\,1829 Cherbourg-Octeville – 19.\,12.\,1899 Paris@\textsc{Beausacq, Marie Suin} (3.\,10.\,1829 Cherbourg-Octeville – 19.\,12.\,1899 Paris), \emph{Schriftstellerin}!Maximes de la vie. Préface par Sully Prud’homme@\strich\emph{Maximes de la vie. Préface par Sully Prud’homme}|pw}}« \substVorne{}\textsuperscript{\textsc{der}}\substDazwischen{}der\substHinten{}{ }\label{K_L03230-6v}\edtext{\textsc{Comtesse Diane\pwindex{Beausacq, Marie Suin 3.\,10.\,1829 Cherbourg-Octeville – 19.\,12.\,1899 Paris@\textsc{Beausacq, Marie Suin} (3.\,10.\,1829 Cherbourg-Octeville – 19.\,12.\,1899 Paris), \emph{Schriftstellerin}|pw}}}{\lemma{\textnormal{\emph{Comtesse Diane}}}\Cendnote{\textnormal{Zu \emph{Maximes de la vie}\pwindex{Beausacq, Marie Suin 3.\,10.\,1829 Cherbourg-Octeville – 19.\,12.\,1899 Paris@\textsc{Beausacq, Marie Suin} (3.\,10.\,1829 Cherbourg-Octeville – 19.\,12.\,1899 Paris), \emph{Schriftstellerin}!Maximes de la vie. Préface par Sully Prud’homme@\strich\emph{Maximes de la vie. Préface par Sully Prud’homme}|pwk}{ }siehe XXXX Auszeichnungsfehler: Dokument L03223 nicht gefunden. Auch eine Lektüre
                  von \emph{Livre d’or}\pwindex{Beausacq, Marie Suin 3.\,10.\,1829 Cherbourg-Octeville – 19.\,12.\,1899 Paris@\textsc{Beausacq, Marie Suin} (3.\,10.\,1829 Cherbourg-Octeville – 19.\,12.\,1899 Paris), \emph{Schriftstellerin}!Livre d’or de la comtesse Diane, préface par Gaston Bergeret@\strich\emph{Livre d’or de la comtesse Diane, préface par Gaston Bergeret}|pwk} (Paris\oindex{Paris@\textbf{Paris}, \emph{Hauptstadt}|pwk}{ }1886) ist nicht nachweisbar.}}}\label{K_L03230-6} kommen laſſen? Noch{ }ſchöner vielleicht iſt das \textsc{Livre d’or\pwindex{Beausacq, Marie Suin 3.\,10.\,1829 Cherbourg-Octeville – 19.\,12.\,1899 Paris@\textsc{Beausacq, Marie Suin} (3.\,10.\,1829 Cherbourg-Octeville – 19.\,12.\,1899 Paris), \emph{Schriftstellerin}!Livre d’or de la comtesse Diane, préface par Gaston Bergeret@\strich\emph{Livre d’or de la comtesse Diane, préface par Gaston Bergeret}|pw}} von derſelben, – ein entzückendes Spiel des Geiſtes und zugleich eine Quelle
               tiefer Lebensweisheit.\pend
           
\pstart
           Was \label{K_L03230-7v}\edtext{\textsc{Sudermann\pwindex{Sudermann, Hermann 30.\,9.\,1857 Macikai – 21.\,11.\,1928 Berlin@\textsc{Sudermann, Hermann} (30.\,9.\,1857 Macikai – 21.\,11.\,1928 Berlin), \emph{Schriftsteller}!Verrohung in der Theaterkritik@\strich\emph{Verrohung in der Theaterkritik}|pwv}\pwindex{Sudermann, Hermann 30.\,9.\,1857 Macikai – 21.\,11.\,1928 Berlin@\textsc{Sudermann, Hermann} (30.\,9.\,1857 Macikai – 21.\,11.\,1928 Berlin), \emph{Schriftsteller}|pw}}}{\lemma{\textnormal{\emph{Sudermann}}}\Cendnote{\textnormal{Siehe XXXX Auszeichnungsfehler: Dokument L03229 nicht gefunden.
               }}}\label{K_L03230-7} anlangt, bin ich durchaus Deiner Anſicht. Vielleicht ergreife ich in dem
               Streit noch das \label{K_L03230-8v}\edtext{Wort}{\lemma{\textnormal{\emph{Wort}}}\Cendnote{\textnormal{Ein solches Feuilleton ist nicht
                  bekannt.}}}\label{K_L03230-8}, obwohl mir Andere gerade das, was ich{ }ſagen möchte, weggeſchrieben
                  {\pb}haben. \label{K_L03230-9v}\edtext{\textsc{Kerrs\pwindex{Kerr, Alfred 25.\,12.\,1867 Breslau – 12.\,10.\,1948 Hamburg@\textsc{Kerr, Alfred} (25.\,12.\,1867 Breslau – 12.\,10.\,1948 Hamburg), \emph{Schriftsteller, Kritiker}|pw}}{ }Erwiderung\pwindex{Kerr, Alfred 25.\,12.\,1867 Breslau – 12.\,10.\,1948 Hamburg@\textsc{Kerr, Alfred} (25.\,12.\,1867 Breslau – 12.\,10.\,1948 Hamburg), \emph{Schriftsteller, Kritiker}!Kritik und Herr Sudermann@\strich\emph{Die Kritik und Herr Sudermann}|pwv}}{\lemma{\textnormal{\emph{Kerrs Erwiderung}}}\Cendnote{\textnormal{Alfred Kerr\pwindex{Kerr, Alfred 25.\,12.\,1867 Breslau – 12.\,10.\,1948 Hamburg@\textsc{Kerr, Alfred} (25.\,12.\,1867 Breslau – 12.\,10.\,1948 Hamburg), \emph{Schriftsteller, Kritiker}|pwk}: \emph{Die Kritik und Herr Sudermann}\pwindex{Kerr, Alfred 25.\,12.\,1867 Breslau – 12.\,10.\,1948 Hamburg@\textsc{Kerr, Alfred} (25.\,12.\,1867 Breslau – 12.\,10.\,1948 Hamburg), \emph{Schriftsteller, Kritiker}!Kritik und Herr Sudermann@\strich\emph{Die Kritik und Herr Sudermann}|pwk}. In: \emph{Der Tag}\pwindex{Tag@\emph{Der Tag}|pwk}, Nr. 545, 21. 11. 1902, S. [1–3]. Weitgehend parallel dazu, wenngleich
                  auf 1903 vordatiert, erschien dieser Text\pwindex{Kerr, Alfred 25.\,12.\,1867 Breslau – 12.\,10.\,1948 Hamburg@\textsc{Kerr, Alfred} (25.\,12.\,1867 Breslau – 12.\,10.\,1948 Hamburg), \emph{Schriftsteller, Kritiker}!Kritik und Herr Sudermann@\strich\emph{Die Kritik und Herr Sudermann}|pwkv} gemeinsam mit gesammelten Kritiken
                     Kerrs\pwindex{Kerr, Alfred 25.\,12.\,1867 Breslau – 12.\,10.\,1948 Hamburg@\textsc{Kerr, Alfred} (25.\,12.\,1867 Breslau – 12.\,10.\,1948 Hamburg), \emph{Schriftsteller, Kritiker}|pwk} zu Sudermanns\pwindex{Sudermann, Hermann 30.\,9.\,1857 Macikai – 21.\,11.\,1928 Berlin@\textsc{Sudermann, Hermann} (30.\,9.\,1857 Macikai – 21.\,11.\,1928 Berlin), \emph{Schriftsteller}|pwk} Stücken als Broschüre: Alfred Kerr\pwindex{Kerr, Alfred 25.\,12.\,1867 Breslau – 12.\,10.\,1948 Hamburg@\textsc{Kerr, Alfred} (25.\,12.\,1867 Breslau – 12.\,10.\,1948 Hamburg), \emph{Schriftsteller, Kritiker}|pwk}: \emph{Herr Sudermann, der D .. Di .. Dichter. Ein kritisches
                        Vademecum}\pwindex{Kerr, Alfred 25.\,12.\,1867 Breslau – 12.\,10.\,1948 Hamburg@\textsc{Kerr, Alfred} (25.\,12.\,1867 Breslau – 12.\,10.\,1948 Hamburg), \emph{Schriftsteller, Kritiker}!Herr Sudermann, der D .. Di .. Dichter. Ein kritisches Vademecum@\strich\emph{Herr Sudermann, der D .. Di .. Dichter. Ein kritisches Vademecum}|pwk}. Berlin\oindex{Berlin@\textbf{Berlin}, \emph{Hauptstadt}|pwk}: \emph{Helianthus}\orgindex{Helianthus@Helianthus|pwk}{ }1903. Die Vorbemerkung zur dritten Auflage – wohl zu lesen als 3. und 4.
                  Tausend – ist mit dem 6. 12. 1902 datiert. }}}\label{K_L03230-9}
               war zum Theil hübſch in der Form, aber der Geſinnung nach lausbübiſch, wie überhaupt
               ein Lausbuben-Zug immer{ }ſtärker bei ihm hervortritt. \textsc{Harden\pwindex{Harden, Maximilian 20.\,10.\,1861 Berlin – 30.\,10.\,1927 Montana@\textsc{Harden, Maximilian} (20.\,10.\,1861 Berlin – 30.\,10.\,1927 Montana), \emph{Schriftsteller, Publizist}|pw}} war, im erſten Theil{ }ſeiner \label{K_L03230-10v}\edtext{Erwiderung\pwindex{Harden, Maximilian 20.\,10.\,1861 Berlin – 30.\,10.\,1927 Montana@\textsc{Harden, Maximilian} (20.\,10.\,1861 Berlin – 30.\,10.\,1927 Montana), \emph{Schriftsteller, Publizist}!Theater [Erwiderung auf Sudermanns Verrohung in der Literaturkritik]@\strich\emph{Theater [Erwiderung auf Sudermanns Verrohung in der Literaturkritik]}|pwv}}{\lemma{\textnormal{\emph{Erwiderung}}}\Cendnote{\textnormal{M. H.\pwindex{Harden, Maximilian 20.\,10.\,1861 Berlin – 30.\,10.\,1927 Montana@\textsc{Harden, Maximilian} (20.\,10.\,1861 Berlin – 30.\,10.\,1927 Montana), \emph{Schriftsteller, Publizist}|pwkv} [ = Maximilian Harden\pwindex{Harden, Maximilian 20.\,10.\,1861 Berlin – 30.\,10.\,1927 Montana@\textsc{Harden, Maximilian} (20.\,10.\,1861 Berlin – 30.\,10.\,1927 Montana), \emph{Schriftsteller, Publizist}|pwk}]: \emph{Theater}\pwindex{Harden, Maximilian 20.\,10.\,1861 Berlin – 30.\,10.\,1927 Montana@\textsc{Harden, Maximilian} (20.\,10.\,1861 Berlin – 30.\,10.\,1927 Montana), \emph{Schriftsteller, Publizist}!Theater [Erwiderung auf Sudermanns Verrohung in der Literaturkritik]@\strich\emph{Theater [Erwiderung auf Sudermanns Verrohung in der Literaturkritik]}|pwk}. In: \emph{Die
                        Zukunft}\pwindex{Zukunft@\emph{Die Zukunft}|pwk}, Bd. 41, 22. 11. 1902,
                     S. 311–326. (Der zweite Teil\pwindex{Harden, Maximilian 20.\,10.\,1861 Berlin – 30.\,10.\,1927 Montana@\textsc{Harden, Maximilian} (20.\,10.\,1861 Berlin – 30.\,10.\,1927 Montana), \emph{Schriftsteller, Publizist}!Theater [Erwiderung auf Sudermanns Verrohung in der Literaturkritik, II]@\strich\emph{Theater [Erwiderung auf Sudermanns Verrohung in der Literaturkritik, II]}|pwkv} erschien in der Folgewoche, 29. 11. 1902, S. 356–370.)}}}\label{K_L03230-10}, viel bedeutender; im
               zweiten{ }ſpricht er zu viel und zu eitel von{ }ſich.\pend
           
\pstart
           Fräulein \label{K_L03230-11v}\edtext{\textsc{Eva F.\pwindex{Goldmann, Eva Marie 27.\,10.\,1877 Wien – 2.\,11.\,1937 ebd.@\textsc{Goldmann, Eva Marie} (27.\,10.\,1877 Wien – 2.\,11.\,1937 ebd.)|pw}}}{\lemma{\textnormal{\emph{Eva F.}}}\Cendnote{\textnormal{Eva Fränkel\pwindex{Goldmann, Eva Marie 27.\,10.\,1877 Wien – 2.\,11.\,1937 ebd.@\textsc{Goldmann, Eva Marie} (27.\,10.\,1877 Wien – 2.\,11.\,1937 ebd.)|pwk}, Goldmanns\pwindex{Goldmann, Paul 31.\,1.\,1865 Breslau – 25.\,9.\,1935 Wien@\textsc{Goldmann, Paul} (31.\,1.\,1865 Breslau – 25.\,9.\,1935 Wien), \emph{Schriftsteller, Journalist}|pwk} spätere Ehefrau, die Schnitzler bereits kannte}}}\label{K_L03230-11} iſt hier. Ich habe{ }ſie
               einmal geſehen und in den erſten fünf Minuten den Eindruck gehabt: »Es iſt
               unmöglich.« Es iſt beinahe eine phyſiſche Antipathie, die ich nicht werde überwinden
               können.\pend
           
\pstart
           Grüße \textsc{Heinrich\pwindex{Schnitzler, Heinrich 9.\,8.\,1902 Hinterbrühl – 12.\,7.\,1982 Wien@\textsc{Schnitzler, Heinrich} (9.\,8.\,1902 Hinterbrühl – 12.\,7.\,1982 Wien), \emph{Regisseur, Schauspieler}|pw}} und{ }ſeine Mutter\pwindex{Schnitzler, Olga 17.\,1.\,1882 Wien – 13.\,1.\,1970 Lugano@\textsc{Schnitzler, Olga} (17.\,1.\,1882 Wien – 13.\,1.\,1970 Lugano), \emph{Schauspielerin, Sängerin}|pwv} und{ }ſei Du{ }ſelbſt vielmals gegrüßt {\\[\baselineskip]}von Deinem \spacefill\mbox{Paul Goldmn}\pend
           \leftskip=0em{}\selectlanguage{ngerman}\endnumbering\briefempfaengerindex{Schnitzler, Arthur@\textsc{Schnitzler, Arthur}!zzzGoldmann, Paul@\emph{von Paul Goldmann}!1902-11-242@{24. 11. [1902]}|)be}\mylabel{L03230h}  \newcommand{\dateiname}{L03230}\newcommand{\titel}{Paul Goldmann an Arthur Schnitzler, 24. 11. [1902]}\newcommand{\editorInnen}{Martin Anton Müller und Laura Untner}%% latex-leseansicht-abspann.tex
%% Abspann für die Leseansicht.
%% Der Schalter \ifkorrekturansicht ist bereits durch den Vorspann gesetzt.

%% latex-abspann.tex
%% Gemeinsamer Abspann für Korrekturansicht und Leseansicht.
%% Setzt den Schalter \ifkorrekturansicht voraus (gesetzt in den
%% einbindenden Dateien latex-korrekturansicht-abspann.tex bzw.
%% latex-leseansicht-abspann.tex).
%% ---------------------------------------------------------------

\normalsize

% Das esempio-Environment wird nur in der Leseansicht benötigt
\ifkorrekturansicht\else
\newenvironment{esempio}[3]%
{
    \vspace{1.5ex}
    \rlap{\underline{#1}}
    \par
    \setlength{\parindent}{0cm}
    \nopagebreak
    \leftskip=#2cm
    \rightskip=#3cm
}
{
    \par
}
\fi

\doendnotes{C}
\bigskip
\vfill

\clearpage

\footnotesize

\ifkorrekturansicht
  \lohead{\textsc{register}}
\fi

% theindex-Environment neu definieren ohne reledmac
\makeatletter
\renewenvironment{theindex}{%
  \ifkorrekturansicht
    \section*{\indexname}%
  \else
    \subsubsection*{Index der erwähnten Entitäten}%
  \fi
  \setlength{\parindent}{0pt}%
  \setlength{\parskip}{0pt plus 0.3pt}%
  \let\item\@idxitem
}{%
  \ifkorrekturansicht\clearpage\fi
}
\makeatother

\IfFileExists{\jobname-pw.ind}{\input{\jobname-pw.ind}}{}

% Quellenangabe nur in der Leseansicht
\ifkorrekturansicht\else
% Fallback-Definitionen, falls die .tex-Datei \titel etc. nicht gesetzt hat
\providecommand{\titel}{}
\providecommand{\editorInnen}{}
\providecommand{\dateiname}{\jobname}

\vspace{3cm}

\vfill

\footnotesize
\textsc{Quelle}: \titel. Herausgegeben von {\editorInnen}. In: \emph{Arthur Schnitzler: Briefwechsel mit Autorinnen und Autoren}.
 Digitale Edition, https://schnitzler-briefe.acdh.oeaw.ac.at/{\dateiname}.html (Stand \today)
\fi

\end{document}


