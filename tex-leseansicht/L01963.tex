%% latex-korrekturansicht-vorspann.tex
%% Vorspann für die Korrekturansicht.
%% Lädt die gemeinsame Datei latex-vorspann.tex mit gesetztem Schalter.

\newif\ifkorrekturansicht
\korrekturansichttrue

\input{../tex-inputs/latex-vorspann}


\section[Arthur Schnitzler an Hermann Bahr, 8. 10. 1910]{L01963 Arthur Schnitzler an Hermann Bahr, 8. 10. 1910}
\nopagebreak\mylabel{L01963v}
\rehead{ }\normalsize\beginnumbering\briefempfaengerindex{Bahr, Hermann@\textsc{Bahr, Hermann}!zzzSchnitzler, Arthur@\emph{von Arthur Schnitzler}!1910-10-081@{8. 10. 1910}|(be}
\toendnotes[C]{\smallbreak\pagebreak[2]}\Standort{TMW, HS AM 60144 Ba.}
\physDesc{Postkarte, 455 Zeichen
\newline{}Schreibmaschine
\newline{}Handschrift: schwarze Tinte, lateinische Kurrent (\noindent{}Anschrift, Unterschrift und Korrekturen)
\newline{}Versand: 1) Stempel: »\nobreak{}8. X. 10, 3\nobreak{}«.   2) Stempel: »\nobreak{}\oindex{London@\textbf{London}, \emph{P.PPLC}|pwk}London\nobreak{}«. }
\buchAbdrucke{\weitereDrucke{1) Arthur Schnitzler: \emph{The Letters of Arthur Schnitzler to Hermann Bahr}. Chapel Hill: \emph{The University of North Carolina Press} 1978, S. 106.} \weitereDrucke{2) Hermann Bahr, Arthur Schnitzler: \emph{Briefwechsel, Aufzeichnungen, Dokumente (1891–1931)}. Göttingen: \emph{Wallstein} 2018, S. 439.} }\toendnotes[C]{\smallbreak}\pstart{}{\pb}\textcolor{gray}{\textbf{Dr. Arthur Schnitzler}}\pend{}\pstart{}\textcolor{gray}{\textbf{Wien XVIII. Sternwartestrasse 71\oindex{Sternwartestrasse 71@\textbf{Sternwartestraße 71}, \emph{Wohngebäude (K.WHS)}|pw}}}\pend{}{\bigskip}\pstart{}\textsc{Herrn Hermann Bahr}\pend{}\pstart{}\textsc{London\oindex{London@\textbf{London}, \emph{P.PPLC}|pw} E. C.}\pend{}\pstart{}\textsc{Victoria Embankment\oindex{Victoria Embankment@\textbf{Victoria Embankment}, \emph{Straße (K.STR)}|pw}}\pend{}\pstart{}\textsc{D\textsuperscript{r} Kaysers
                        Hotel\oindex{De Keysers Royal Hotel@\textbf{De Keysers Royal Hotel}, \emph{Hotel (K.HTL)}|pw}}\pend{}\pstart{}\textsc{England\oindex{England@\textbf{England}, \emph{A.ADM1}|pw}.}\pend{}{\bigskip}\vspace{1em}
\pstart
           \raggedleft{}{\pb}8. 10. 1910.\pend
           \vspace{0.5em}
\pstart
           Lieber Hermann. Ein gewisser D\introOben{}r\introOben{}. Cesare Levi\pwindex{Levi, Cesare 1874 – 18.07.1926@\textsc{Levi, Cesare} (1874 – 18.07.1926), \emph{Journalist/Journalistin, Übersetzer/Übersetzerin, Theaterkritiker/Theaterkritikerin}|pw} möchte Dein Konzert\pwindex{Konzert. Lustspiel in drei Akten@\emph{Das Konzert. Lustspiel in drei Akten}|pw} ins Italienische\oindex{Italien@\textbf{Italien}, \emph{A.PCLI}|pw}
               übersetzen. Zu seiner Empfehlung kann ich nur sagen, dass in seiner \label{K_L01963-1v}\edtext{Uebersetzung}{\lemma{\textnormal{\emph{Uebersetzung}}}\Cendnote{\textnormal{Levi\pwindex{Levi, Cesare 1874 – 18.07.1926@\textsc{Levi, Cesare} (1874 – 18.07.1926), \emph{Journalist/Journalistin, Übersetzer/Übersetzerin, Theaterkritiker/Theaterkritikerin}|pwk} hatte mehrere Übersetzungen von Einaktern Schnitzlers angefertigt:
                  \emph{Il matrimonio d’Anatolio}\pwindex{matrimonio DAnatolio@\emph{Il matrimonio d’Anatolio}|pwk} (\emph{Anatols
                     Hochzeitsmorgen}\pwindex{Anatols Hochzeitsmorgen@\emph{Anatols Hochzeitsmorgen}|pwk}), \emph{Cena d’addio}\pwindex{Cena Daddio@\emph{Cena d’addio}|pwk}
                  (\emph{Abschiedssouper}\pwindex{Abschiedssouper@\emph{Abschiedssouper}|pwk}), \emph{Letteratura}\pwindex{Letteratura@\emph{Letteratura}|pwk}
                  (\emph{Literatur}\pwindex{Literatur@\emph{Literatur}|pwk}), \emph{Il burattinaio}\pwindex{burattinaio@\emph{Il burattinaio}|pwk} (\emph{Der
                     Puppenspieler}\pwindex{Puppenspieler. Studie in einem Aufzuge@\emph{Der Puppenspieler. Studie in einem Aufzuge}|pwk}) und \emph{L’ultime maschere}\pwindex{Le ultime maschere. Drama in un atto@\emph{Le ultime maschere. Drama in un atto}|pwk} (\emph{Die
                     letzten Masken}\pwindex{letzten Masken@\emph{Die letzten Masken}|pwk}).}}}\label{K_L01963-1}{ }einige meiner Einakter\pwindex{Anatols Hochzeitsmorgen@\emph{Anatols Hochzeitsmorgen}|pwv}\pwindex{Puppenspieler. Studie in einem Aufzuge@\emph{Der Puppenspieler. Studie in einem Aufzuge}|pwv}\pwindex{Abschiedssouper@\emph{Abschiedssouper}|pwv}\pwindex{letzten Masken@\emph{Die letzten Masken}|pwv}\pwindex{Literatur@\emph{Literatur}|pwv} in Italien\oindex{Italien@\textbf{Italien}, \emph{A.PCLI}|pw} aufgeführt worden sind und seither eine wahre Flut von Lire auf mich
               niederströmt. \substVorne{}\textsuperscript{Neulich}\substDazwischen{}Im letzten Vierteljahr\substHinten{} waren es vierzehn.\pend
           
\pstart
           Nächstens bekommst Du den Medardus\pwindex{junge Medardus. Dramatische Historie in einem Vorspiel und fuenf Aufzuegen@\emph{Der junge Medardus. Dramatische Historie in einem Vorspiel und fünf Aufzügen}|pw}.\pend
           
\pstart
           Herzlichst Dein{\\[\baselineskip]}\spacefill\mbox{{[}hs.:{]} Arthur.}\pend
           \leftskip=0em{}\selectlanguage{ngerman}\endnumbering\briefempfaengerindex{Bahr, Hermann@\textsc{Bahr, Hermann}!zzzSchnitzler, Arthur@\emph{von Arthur Schnitzler}!1910-10-081@{8. 10. 1910}|)be}\mylabel{L01963h}  \normalsize

\doendnotes{C}
\bigskip
\vfill

\clearpage

\footnotesize

\lohead{\textsc{register}}

% Definiere theindex-Environment komplett neu ohne reledmac
\makeatletter
\renewenvironment{theindex}{%
  \section*{\indexname}%
  \setlength{\parindent}{0pt}%
  \setlength{\parskip}{0pt plus 0.3pt}%
  \let\item\@idxitem
}{%
  \clearpage
}
\makeatother

\IfFileExists{\jobname-pw.ind}{\input{\jobname-pw.ind}}{}

\end{document}

      