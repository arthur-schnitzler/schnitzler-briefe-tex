%% latex-leseansicht-vorspann.tex
%% Vorspann für die Leseansicht.
%% Lädt die gemeinsame Datei latex-vorspann.tex mit nicht gesetztem Schalter.

\newif\ifkorrekturansicht
\korrekturansichtfalse

\input{../tex-inputs/latex-vorspann}


         
         \renewcommand{\erwaehntePersonen}{Personen: Hermann Bahr, Cesare Levi}
         \renewcommand{\erwaehnteOrte}{Orte: De Keysers Royal Hotel, England, Italien, London, Sternwartestraße, Victoria Embankment, Wien}
         \renewcommand{\erwaehnteWerke}{Werke: Abschiedssouper, Anatols Hochzeitsmorgen, Das Konzert. Lustspiel in drei Akten, Der Puppenspieler, Der junge Medardus. Dramatische Historie in einem Vorspiel und fünf Aufzügen, Die letzten Masken, Literatur}
               \section[Arthur Schnitzler an Hermann Bahr, 8. 10. 1910]{ Arthur Schnitzler an Hermann Bahr, 8. 10. 1910}\nopagebreak\mylabel{v}\rehead{ }\begin{ledgroupsized}[t]{13cm}\normalsize\beginnumbering \toendnotes[C]{\smallbreak\pagebreak[2]} \Standort{TMW, HS AM 60144 Ba.}
\physDesc{Postkarte, 455 Zeichen
\newline{}Schreibmaschine
\newline{}Handschrift: schwarze Tinte, lateinische Kurrent (\noindent{}Anschrift, Unterschrift und Korrekturen)
\newline{}Versand: 1) Stempel: »\nobreak{}8. X. 10, 3\nobreak{}«.   2) Stempel: »\nobreak{}\oindex{London@\textbf{London}|pwk}London\nobreak{}«. }\buchAbdrucke{\weitereDrucke{1) \emph{8. 10. 1910, Abschrift.} In: Arthur Schnitzler: \emph{The Letters of Arthur Schnitzler to Hermann Bahr}. Edited, annotated, and with an introduction, by Donald G.
                        Daviau. Chapel Hill: \emph{The University of North Carolina Press} 1978, S. 106 (University of North Carolina studies in the Germanic languages
                        and literatures, 89).} \weitereDrucke{2) Hermann Bahr, Arthur Schnitzler: \emph{Briefwechsel, Aufzeichnungen, Dokumente (1891–1931)}. Hg. Kurt Ifkovits und Martin Anton Müller. Göttingen: \emph{Wallstein} 2018, S. 439.} }\toendnotes[C]{\smallbreak}\pstart{}{\pb}\textcolor{gray}{\textbf{Dr. Arthur Schnitzler}}\pend{}\pstart{}\textcolor{gray}{\textbf{Wien XVIII. Sternwartestrasse 71\oindex{Sternwartestrasse@\textbf{Sternwartestraße}|pw}}}\pend{}{\bigskip}\pstart{}\textsc{Herrn Hermann Bahr}\pend{}\pstart{}\textsc{London\oindex{London@\textbf{London}|pw} E. C.}\pend{}\pstart{}\textsc{Victoria Embankment\oindex{Victoria Embankment@\textbf{Victoria Embankment}|pw}}\pend{}\pstart{}\textsc{D\textsuperscript{r } Kaysers
                        Hotel\oindex{De Keysers Royal Hotel@\textbf{De Keysers Royal Hotel}|pw}}\pend{}\pstart{}\textsc{England\oindex{England@\textbf{England}|pw}.}\pend{}{\bigskip}\pstart
           \raggedleft{}{\pb}8. 10. 1910.\pend
           \pstart
           Lieber Hermann. Ein gewisser D\introOben{}r\introOben{}. Cesare Levi\pwindex{Levi, Cesare 1874 – 18.07.1926@\textsc{Levi, Cesare} (1874 – 18.07.1926), \emph{Journalist, Übersetzer, Theaterkritiker}|pw} möchte Dein Konzert\pwindex{Bahr, Hermann 19.07.1863 – 15.01.1934@\textsc{Bahr, Hermann} (19.07.1863 – 15.01.1934), \emph{Schriftsteller, Kritiker}!Konzert. Lustspiel in drei Akten1909@\strich\emph{Das Konzert. Lustspiel in drei Akten} {[}1909{]}|pw} ins Italienische\oindex{Italien@\textbf{Italien}|pw}
               übersetzen. Zu seiner Empfehlung kann ich nur sagen, dass in seiner \label{K_L01963_1v}\edtext{Uebersetzung}{\lemma{\textnormal{\emph{Uebersetzung}}}\Cendnote{\textnormal{\emph{Il matrimonio d’Anatolio (Anatols
                     Hochzeitsmorgen)}\pwindex{Schnitzler, Arthur 15.05.1862 – 21.10.1931@\textsc{Schnitzler, Arthur} (15.05.1862 – 21.10.1931), \emph{Schriftsteller, Mediziner}!Anatols Hochzeitsmorgen01. 07. 1890@\strich\emph{Anatols Hochzeitsmorgen} {[}01. 07. 1890{]}|pwk}, \emph{Cena d’addio
                     (Abschiedssouper)}\pwindex{Schnitzler, Arthur 15.05.1862 – 21.10.1931@\textsc{Schnitzler, Arthur} (15.05.1862 – 21.10.1931), \emph{Schriftsteller, Mediziner}!Abschiedssouper1892@\strich\emph{Abschiedssouper} {[}1892{]}|pwk}, \emph{Letteratura
                     (Literatur)}\pwindex{Schnitzler, Arthur 15.05.1862 – 21.10.1931@\textsc{Schnitzler, Arthur} (15.05.1862 – 21.10.1931), \emph{Schriftsteller, Mediziner}!Literatur1901@\strich\emph{Literatur} {[}1901{]}|pwk}, \emph{Il burattinaio (Der
                     Puppenspieler)}\pwindex{Schnitzler, Arthur 15.05.1862 – 21.10.1931@\textsc{Schnitzler, Arthur} (15.05.1862 – 21.10.1931), \emph{Schriftsteller, Mediziner}!Puppenspieler31. 05. 1903@\strich\emph{Der Puppenspieler} {[}31. 05. 1903{]}|pwk} und \emph{L’ultime maschere (Die
                     letzten Masken)}\pwindex{Schnitzler, Arthur 15.05.1862 – 21.10.1931@\textsc{Schnitzler, Arthur} (15.05.1862 – 21.10.1931), \emph{Schriftsteller, Mediziner}!letzten Masken1901@\strich\emph{Die letzten Masken} {[}1901{]}|pwk}.}}}\label{K_L01963_1h}{ }einige meiner Einakter\pwindex{Schnitzler, Arthur 15.05.1862 – 21.10.1931@\textsc{Schnitzler, Arthur} (15.05.1862 – 21.10.1931), \emph{Schriftsteller, Mediziner}!Anatols Hochzeitsmorgen01. 07. 1890@\strich\emph{Anatols Hochzeitsmorgen} {[}01. 07. 1890{]}|pwv}\pwindex{Schnitzler, Arthur 15.05.1862 – 21.10.1931@\textsc{Schnitzler, Arthur} (15.05.1862 – 21.10.1931), \emph{Schriftsteller, Mediziner}!Puppenspieler31. 05. 1903@\strich\emph{Der Puppenspieler} {[}31. 05. 1903{]}|pwv}\pwindex{Schnitzler, Arthur 15.05.1862 – 21.10.1931@\textsc{Schnitzler, Arthur} (15.05.1862 – 21.10.1931), \emph{Schriftsteller, Mediziner}!Abschiedssouper1892@\strich\emph{Abschiedssouper} {[}1892{]}|pwv}\pwindex{Schnitzler, Arthur 15.05.1862 – 21.10.1931@\textsc{Schnitzler, Arthur} (15.05.1862 – 21.10.1931), \emph{Schriftsteller, Mediziner}!letzten Masken1901@\strich\emph{Die letzten Masken} {[}1901{]}|pwv}\pwindex{Schnitzler, Arthur 15.05.1862 – 21.10.1931@\textsc{Schnitzler, Arthur} (15.05.1862 – 21.10.1931), \emph{Schriftsteller, Mediziner}!Literatur1901@\strich\emph{Literatur} {[}1901{]}|pwv} in Italien\oindex{Italien@\textbf{Italien}|pw} aufgeführt worden sind und seither eine wahre Flut von Lire auf mich
               niederströmt. \substVorne{}\textsuperscript{Neulich}{\allowbreak}\substDazwischen{}Im letzten Vierteljahr\substHinten{} waren es vierzehn.\pend
           \pstart
           Nächstens bekommst Du den Medardus\pwindex{Schnitzler, Arthur 15.05.1862 – 21.10.1931@\textsc{Schnitzler, Arthur} (15.05.1862 – 21.10.1931), \emph{Schriftsteller, Mediziner}!junge Medardus. Dramatische Historie in einem Vorspiel und fuenf
                  Aufzuegen1910-10-26@\strich\emph{Der junge Medardus. Dramatische Historie in einem Vorspiel und fünf Aufzügen} {[}1910-10-26{]}|pw}.\pend
           \pstart
           Herzlichst Dein{\\[\baselineskip]}\spacefill\mbox{{[}hs.:{]} Arthur.}\pend
           \leftskip=0em{}
         
         \endnumbering\mylabel{h}\end{ledgroupsized}  \newcommand{\dateiname}{L01963}\newcommand{\titel}{Arthur Schnitzler an Hermann Bahr, 8. 10. 1910}\newcommand{\editorInnen}{ Kurt Ifkovits,  Martin Anton Müller}%% latex-leseansicht-abspann.tex
%% Abspann für die Leseansicht.
%% Der Schalter \ifkorrekturansicht ist bereits durch den Vorspann gesetzt.

%% latex-abspann.tex
%% Gemeinsamer Abspann für Korrekturansicht und Leseansicht.
%% Setzt den Schalter \ifkorrekturansicht voraus (gesetzt in den
%% einbindenden Dateien latex-korrekturansicht-abspann.tex bzw.
%% latex-leseansicht-abspann.tex).
%% ---------------------------------------------------------------

\normalsize

% Das esempio-Environment wird nur in der Leseansicht benötigt
\ifkorrekturansicht\else
\newenvironment{esempio}[3]%
{
    \vspace{1.5ex}
    \rlap{\underline{#1}}
    \par
    \setlength{\parindent}{0cm}
    \nopagebreak
    \leftskip=#2cm
    \rightskip=#3cm
}
{
    \par
}
\fi

\doendnotes{C}
\bigskip
\vfill

\clearpage

\footnotesize

\ifkorrekturansicht
  \lohead{\textsc{register}}
\fi

% theindex-Environment neu definieren ohne reledmac
\makeatletter
\renewenvironment{theindex}{%
  \ifkorrekturansicht
    \section*{\indexname}%
  \else
    \subsubsection*{Index der erwähnten Entitäten}%
  \fi
  \setlength{\parindent}{0pt}%
  \setlength{\parskip}{0pt plus 0.3pt}%
  \let\item\@idxitem
}{%
  \ifkorrekturansicht\clearpage\fi
}
\makeatother

\IfFileExists{\jobname-pw.ind}{\input{\jobname-pw.ind}}{}

% Quellenangabe nur in der Leseansicht
\ifkorrekturansicht\else
% Fallback-Definitionen, falls die .tex-Datei \titel etc. nicht gesetzt hat
\providecommand{\titel}{}
\providecommand{\editorInnen}{}
\providecommand{\dateiname}{\jobname}

\vspace{3cm}

\vfill

\footnotesize
\textsc{Quelle}: \titel. Herausgegeben von {\editorInnen}. In: \emph{Arthur Schnitzler: Briefwechsel mit Autorinnen und Autoren}.
 Digitale Edition, https://schnitzler-briefe.acdh.oeaw.ac.at/{\dateiname}.html (Stand \today)
\fi

\end{document}


      