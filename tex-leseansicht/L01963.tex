%% latex-leseansicht-vorspann.tex
%% Vorspann für die Leseansicht.
%% Lädt die gemeinsame Datei latex-vorspann.tex mit nicht gesetztem Schalter.

\newif\ifkorrekturansicht
\korrekturansichtfalse

\input{../tex-inputs/latex-vorspann}


         
         \renewcommand{\erwaehntePersonen}{Personen: Hermann Bahr, Cesare Levi}
         \renewcommand{\erwaehnteOrte}{Orte: De Keysers Royal Hotel, England, Italien, London, Sternwartestraße 71, Victoria Embankment, Wien}
         \renewcommand{\erwaehnteWerke}{Werke: Abschiedssouper, Anatols Hochzeitsmorgen, Cena d’addio, Das Konzert. Lustspiel in drei Akten, Der Puppenspieler. Studie in einem Aufzuge, Der junge Medardus. Dramatische Historie in einem Vorspiel und fünf Aufzügen, Die letzten Masken, Il burattinaio, Il matrimonio d’Anatolio, Le ultime maschere. Drama in un atto, Letteratura, Literatur}
               \section[Arthur Schnitzler an Hermann Bahr, 8. 10. 1910]{ Arthur Schnitzler an Hermann Bahr, 8. 10. 1910}\nopagebreak\mylabel{v}\rehead{ }\begin{ledgroupsized}[t]{13cm}\normalsize\beginnumbering\briefempfaengerindex{Bahr, Hermann@\textsc{Bahr, Hermann}!zzzSchnitzler, Arthur@\emph{von Arthur Schnitzler}!1910-10-081@{8. 10. 1910}|(be} \toendnotes[C]{\smallbreak\pagebreak[2]} \Standort{TMW, HS AM 60144 Ba.}
\physDesc{Postkarte, 455 Zeichen
\newline{}Schreibmaschine
\newline{}Handschrift: schwarze Tinte, lateinische Kurrent (\noindent{}Anschrift, Unterschrift und Korrekturen)
\newline{}Versand: 1) Stempel: »\nobreak{}8. X. 10, 3\nobreak{}«.   2) Stempel: »\nobreak{}\oindex{London@\textbf{London}|pwk}London\nobreak{}«. }\buchAbdrucke{\weitereDrucke{1) \emph{8. 10. 1910, Abschrift.} In: Arthur Schnitzler: \emph{The Letters of Arthur Schnitzler to Hermann Bahr}. Edited, annotated, and with an introduction, by Donald G.
                        Daviau. Chapel Hill: \emph{The University of North Carolina Press} 1978, S. 106 (University of North Carolina studies in the Germanic languages
                        and literatures, 89).} \weitereDrucke{2) Hermann Bahr, Arthur Schnitzler: \emph{Briefwechsel, Aufzeichnungen, Dokumente (1891–1931)}. Hg. Kurt Ifkovits und Martin Anton Müller. Göttingen: \emph{Wallstein} 2018, S. 439.} }\toendnotes[C]{\smallbreak}\pstart{}{\pb}\textcolor{gray}{\textbf{Dr. Arthur Schnitzler}}\pend{}\pstart{}\textcolor{gray}{\textbf{Wien XVIII. Sternwartestrasse 71\oindex{Sternwartestrasse 71@\textbf{Sternwartestraße 71}|pw}}}\pend{}{\bigskip}\pstart{}\textsc{Herrn Hermann Bahr}\pend{}\pstart{}\textsc{London\oindex{London@\textbf{London}|pw} E. C.}\pend{}\pstart{}\textsc{Victoria Embankment\oindex{Victoria Embankment@\textbf{Victoria Embankment}|pw}}\pend{}\pstart{}\textsc{D\textsuperscript{r} Kaysers
                        Hotel\oindex{De Keysers Royal Hotel@\textbf{De Keysers Royal Hotel}|pw}}\pend{}\pstart{}\textsc{England\oindex{England@\textbf{England}|pw}.}\pend{}{\bigskip}\pstart
           \raggedleft{}{\pb}8. 10. 1910.\pend
           \pstart
           Lieber Hermann. Ein gewisser D\introOben{}r\introOben{}. Cesare Levi\pwindex{Levi, Cesare 1874 – 18.07.1926@\textsc{Levi, Cesare} (1874 – 18.07.1926), \emph{Journalist, Übersetzer, Theaterkritiker}|pw} möchte Dein Konzert\pwindex{Bahr, Hermann 19.07.1863 – 15.01.1934@\textsc{Bahr, Hermann} (19.07.1863 – 15.01.1934), \emph{Schriftsteller, Kritiker}!Konzert. Lustspiel in drei Akten1909@\strich\emph{Das Konzert. Lustspiel in drei Akten} {[}1909{]}|pw} ins Italienische\oindex{Italien@\textbf{Italien}|pw}
               übersetzen. Zu seiner Empfehlung kann ich nur sagen, dass in seiner \label{K_L01963-1v}\edtext{Uebersetzung}{\lemma{\textnormal{\emph{Uebersetzung}}}\Cendnote{\textnormal{Levi\pwindex{Levi, Cesare 1874 – 18.07.1926@\textsc{Levi, Cesare} (1874 – 18.07.1926), \emph{Journalist, Übersetzer, Theaterkritiker}|pwk} hatte mehrere Übersetzungen von Einaktern Schnitzler\pwindex{Schnitzler, Arthur 15.05.1862 – 21.10.1931@\textsc{Schnitzler, Arthur} (15.05.1862 – 21.10.1931), \emph{Schriftsteller, Mediziner}|pwk}s angefertigt:
                  \emph{Il matrimonio d’Anatolio}\pwindex{Schnitzler, Arthur 15.05.1862 – 21.10.1931@\textsc{Schnitzler, Arthur} (15.05.1862 – 21.10.1931), \emph{Schriftsteller, Mediziner}!matrimonio DAnatolio@\strich\emph{Il matrimonio d’Anatolio}|pwk} (\emph{Anatols
                     Hochzeitsmorgen}\pwindex{Schnitzler, Arthur 15.05.1862 – 21.10.1931@\textsc{Schnitzler, Arthur} (15.05.1862 – 21.10.1931), \emph{Schriftsteller, Mediziner}!Anatols Hochzeitsmorgen01. 07. 1890@\strich\emph{Anatols Hochzeitsmorgen} {[}01. 07. 1890{]}|pwk}), \emph{Cena d’addio}\pwindex{Schnitzler, Arthur 15.05.1862 – 21.10.1931@\textsc{Schnitzler, Arthur} (15.05.1862 – 21.10.1931), \emph{Schriftsteller, Mediziner}!Cena Daddio1908@\strich\emph{Cena d’addio} {[}1908{]}|pwk}
                  (\emph{Abschiedssouper}\pwindex{Schnitzler, Arthur 15.05.1862 – 21.10.1931@\textsc{Schnitzler, Arthur} (15.05.1862 – 21.10.1931), \emph{Schriftsteller, Mediziner}!Abschiedssouper1892@\strich\emph{Abschiedssouper} {[}1892{]}|pwk}), \emph{Letteratura}\pwindex{Levi, Cesare 1874 – 18.07.1926@\textsc{Levi, Cesare} (1874 – 18.07.1926), \emph{Journalist, Übersetzer, Theaterkritiker}!Letteratura1908@\strich\emph{Letteratura} {[}Übersetzung, 1908{]}|pwk}
                  (\emph{Literatur}\pwindex{Schnitzler, Arthur 15.05.1862 – 21.10.1931@\textsc{Schnitzler, Arthur} (15.05.1862 – 21.10.1931), \emph{Schriftsteller, Mediziner}!Literatur1901@\strich\emph{Literatur} {[}1901{]}|pwk}), \emph{Il burattinaio}\pwindex{Levi, Cesare 1874 – 18.07.1926@\textsc{Levi, Cesare} (1874 – 18.07.1926), \emph{Journalist, Übersetzer, Theaterkritiker}!burattinaio1908@\strich\emph{Il burattinaio} {[}Übersetzung, 1908{]}|pwk} (\emph{Der
                     Puppenspieler}\pwindex{Schnitzler, Arthur 15.05.1862 – 21.10.1931@\textsc{Schnitzler, Arthur} (15.05.1862 – 21.10.1931), \emph{Schriftsteller, Mediziner}!Puppenspieler. Studie in einem Aufzuge31. 05. 1903@\strich\emph{Der Puppenspieler. Studie in einem Aufzuge} {[}31. 05. 1903{]}|pwk}) und \emph{L’ultime maschere}\pwindex{Levi, Cesare 1874 – 18.07.1926@\textsc{Levi, Cesare} (1874 – 18.07.1926), \emph{Journalist, Übersetzer, Theaterkritiker}!Le ultime maschere. Drama in un atto1908@\strich\emph{Le ultime maschere. Drama in un atto} {[}Übersetzung, 1908{]}|pwk} (\emph{Die
                     letzten Masken}\pwindex{Schnitzler, Arthur 15.05.1862 – 21.10.1931@\textsc{Schnitzler, Arthur} (15.05.1862 – 21.10.1931), \emph{Schriftsteller, Mediziner}!letzten Masken1901@\strich\emph{Die letzten Masken} {[}1901{]}|pwk}).}}}\label{K_L01963-1h}{ }einige meiner Einakter\pwindex{Schnitzler, Arthur 15.05.1862 – 21.10.1931@\textsc{Schnitzler, Arthur} (15.05.1862 – 21.10.1931), \emph{Schriftsteller, Mediziner}!Anatols Hochzeitsmorgen01. 07. 1890@\strich\emph{Anatols Hochzeitsmorgen} {[}01. 07. 1890{]}|pwv}\pwindex{Schnitzler, Arthur 15.05.1862 – 21.10.1931@\textsc{Schnitzler, Arthur} (15.05.1862 – 21.10.1931), \emph{Schriftsteller, Mediziner}!Puppenspieler. Studie in einem Aufzuge31. 05. 1903@\strich\emph{Der Puppenspieler. Studie in einem Aufzuge} {[}31. 05. 1903{]}|pwv}\pwindex{Schnitzler, Arthur 15.05.1862 – 21.10.1931@\textsc{Schnitzler, Arthur} (15.05.1862 – 21.10.1931), \emph{Schriftsteller, Mediziner}!Abschiedssouper1892@\strich\emph{Abschiedssouper} {[}1892{]}|pwv}\pwindex{Schnitzler, Arthur 15.05.1862 – 21.10.1931@\textsc{Schnitzler, Arthur} (15.05.1862 – 21.10.1931), \emph{Schriftsteller, Mediziner}!letzten Masken1901@\strich\emph{Die letzten Masken} {[}1901{]}|pwv}\pwindex{Schnitzler, Arthur 15.05.1862 – 21.10.1931@\textsc{Schnitzler, Arthur} (15.05.1862 – 21.10.1931), \emph{Schriftsteller, Mediziner}!Literatur1901@\strich\emph{Literatur} {[}1901{]}|pwv} in Italien\oindex{Italien@\textbf{Italien}|pw} aufgeführt worden sind und seither eine wahre Flut von Lire auf mich
               niederströmt. \substVorne{}\textsuperscript{Neulich}{\allowbreak}\substDazwischen{}Im letzten Vierteljahr\substHinten{} waren es vierzehn.\pend
           \pstart
           Nächstens bekommst Du den Medardus\pwindex{Schnitzler, Arthur 15.05.1862 – 21.10.1931@\textsc{Schnitzler, Arthur} (15.05.1862 – 21.10.1931), \emph{Schriftsteller, Mediziner}!junge Medardus. Dramatische Historie in einem Vorspiel und fuenf Aufzuegen1910-10-26@\strich\emph{Der junge Medardus. Dramatische Historie in einem Vorspiel und fünf Aufzügen} {[}1910-10-26{]}|pw}.\pend
           \pstart
           Herzlichst Dein{\\[\baselineskip]}\spacefill\mbox{{[}hs.:{]} Arthur.}\pend
           \leftskip=0em{}
         
         \endnumbering\mylabel{h}\end{ledgroupsized}  \newcommand{\dateiname}{L01963}\newcommand{\titel}{Arthur Schnitzler an Hermann Bahr, 8. 10. 1910}\newcommand{\editorInnen}{ Kurt Ifkovits,  Martin Anton Müller}%% latex-leseansicht-abspann.tex
%% Abspann für die Leseansicht.
%% Der Schalter \ifkorrekturansicht ist bereits durch den Vorspann gesetzt.

%% latex-abspann.tex
%% Gemeinsamer Abspann für Korrekturansicht und Leseansicht.
%% Setzt den Schalter \ifkorrekturansicht voraus (gesetzt in den
%% einbindenden Dateien latex-korrekturansicht-abspann.tex bzw.
%% latex-leseansicht-abspann.tex).
%% ---------------------------------------------------------------

\normalsize

% Das esempio-Environment wird nur in der Leseansicht benötigt
\ifkorrekturansicht\else
\newenvironment{esempio}[3]%
{
    \vspace{1.5ex}
    \rlap{\underline{#1}}
    \par
    \setlength{\parindent}{0cm}
    \nopagebreak
    \leftskip=#2cm
    \rightskip=#3cm
}
{
    \par
}
\fi

\doendnotes{C}
\bigskip
\vfill

\clearpage

\footnotesize

\ifkorrekturansicht
  \lohead{\textsc{register}}
\fi

% theindex-Environment neu definieren ohne reledmac
\makeatletter
\renewenvironment{theindex}{%
  \ifkorrekturansicht
    \section*{\indexname}%
  \else
    \subsubsection*{Index der erwähnten Entitäten}%
  \fi
  \setlength{\parindent}{0pt}%
  \setlength{\parskip}{0pt plus 0.3pt}%
  \let\item\@idxitem
}{%
  \ifkorrekturansicht\clearpage\fi
}
\makeatother

\IfFileExists{\jobname-pw.ind}{\input{\jobname-pw.ind}}{}

% Quellenangabe nur in der Leseansicht
\ifkorrekturansicht\else
% Fallback-Definitionen, falls die .tex-Datei \titel etc. nicht gesetzt hat
\providecommand{\titel}{}
\providecommand{\editorInnen}{}
\providecommand{\dateiname}{\jobname}

\vspace{3cm}

\vfill

\footnotesize
\textsc{Quelle}: \titel. Herausgegeben von {\editorInnen}. In: \emph{Arthur Schnitzler: Briefwechsel mit Autorinnen und Autoren}.
 Digitale Edition, https://schnitzler-briefe.acdh.oeaw.ac.at/{\dateiname}.html (Stand \today)
\fi

\end{document}


      