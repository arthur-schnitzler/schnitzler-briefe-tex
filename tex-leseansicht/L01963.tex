%% latex-leseansicht-vorspann.tex
%% Vorspann für die Leseansicht.
%% Lädt die gemeinsame Datei latex-vorspann.tex mit nicht gesetztem Schalter.

\newif\ifkorrekturansicht
\korrekturansichtfalse

\input{../tex-inputs/latex-vorspann}


\section[Arthur Schnitzler an Hermann Bahr, 8. 10. 1910]{L01963 Arthur Schnitzler an Hermann Bahr, 8. 10. 1910}
\nopagebreak\mylabel{L01963v}
\rehead{ }\normalsize\beginnumbering\briefempfaengerindex{Bahr, Hermann@\textsc{Bahr, Hermann}!zzzSchnitzler, Arthur@\emph{von Arthur Schnitzler}!1910-10-081@{8. 10. 1910}|(be}
\toendnotes[C]{\smallbreak\pagebreak[2]}
\correspDesc{Versand  durch Arthur Schnitzler am 8. 10. 1910 in Wien
\newline{}Erhalt  durch Hermann Bahr im Zeitraum [9. 10. 1910
                  – 13. 10. 1910?] in London}\toendnotes[C]{\smallbreak}
\Standort{TMW, HS AM 60144 Ba.}
\physDesc{Postkarte, 455 Zeichen
\newline{}Schreibmaschine
\newline{}Handschrift: schwarze Tinte, lateinische Kurrent (\noindent{}Anschrift, Unterschrift und Korrekturen)
\newline{}Versand: 1) Stempel: »\nobreak{}8. X. 10, 3\nobreak{}«.   2) Stempel: »\nobreak{}\oindex{London@\textbf{London}, \emph{Hauptstadt}|pwk}London\nobreak{}«. }
\buchAbdrucke{\weitereDrucke{1) \emph{8. 10. 1910, Abschrift.} In: Arthur Schnitzler: \emph{The Letters of Arthur Schnitzler to Hermann Bahr}. Edited, annotated, and with an introduction, by Donald G. Daviau. Chapel Hill: \emph{The University of North Carolina Press} 1978, S. 106 (University of North Carolina studies in the Germanic languages
                        and literatures, 89).} \weitereDrucke{2) Hermann Bahr, Arthur Schnitzler: \emph{Briefwechsel, Aufzeichnungen, Dokumente (1891–1931)}. Herausgegeben von Kurt Ifkovits und Martin Anton Müller. Göttingen: \emph{Wallstein} 2018, S. 439.} }\toendnotes[C]{\smallbreak}\pstart{}{\pb}\textcolor{gray}{\textbf{Dr. Arthur Schnitzler}}\pend{}\pstart{}\textcolor{gray}{\textbf{Wien XVIII. Sternwartestrasse 71\oindex{Wien@\textbf{Wien}!XVIII., Währing@\textbf{XVIII., Währing}!Sternwartestraße 71@\textbf{Sternwartestraße 71}, \emph{Wohngebäude}|pw}}}\pend{}{\bigskip}\pstart{}\textsc{Herrn Hermann Bahr}\pend{}\pstart{}\textsc{London\oindex{London@\textbf{London}, \emph{Hauptstadt}|pw} E. C.}\pend{}\pstart{}\textsc{Victoria Embankment\oindex{Victoria Embankment@\textbf{Victoria Embankment}, \emph{Straße}|pw}}\pend{}\pstart{}\textsc{D\textsuperscript{r} Kaysers
                        Hotel\oindex{De Keysers Royal Hotel@\textbf{De Keysers Royal Hotel}, \emph{Hotel}|pw}}\pend{}\pstart{}\textsc{England\oindex{England@\textbf{England}, \emph{Land}|pw}.}\pend{}{\bigskip}\vspace{1em}
\pstart
           \raggedleft{}{\pb}8. 10. 1910.\pend
           \vspace{0.5em}
\pstart
           Lieber Hermann. Ein gewisser D\introOben{}r\introOben{}. Cesare Levi\pwindex{Levi, Cesare 1874 Triest – 18.\,7.\,1926 Sand in Taufers@\textsc{Levi, Cesare} (1874 Triest – 18.\,7.\,1926 Sand in Taufers), \emph{Journalist, Übersetzer, Theaterkritiker}|pw} möchte Dein Konzert\pwindex{Bahr, Hermann 19.\,7.\,1863 Linz – 15.\,1.\,1934 München@\textsc{Bahr, Hermann} (19.\,7.\,1863 Linz – 15.\,1.\,1934 München), \emph{Schriftsteller, Kritiker}!Konzert. Lustspiel in drei Akten@\strich\emph{Das Konzert. Lustspiel in drei Akten}|pw} ins Italienische\oindex{Italien@\textbf{Italien}|pw}
               übersetzen. Zu seiner Empfehlung kann ich nur sagen, dass in seiner \label{K_L01963-1v}\edtext{Uebersetzung}{\lemma{\textnormal{\emph{Uebersetzung}}}\Cendnote{\textnormal{Levi\pwindex{Levi, Cesare 1874 Triest – 18.\,7.\,1926 Sand in Taufers@\textsc{Levi, Cesare} (1874 Triest – 18.\,7.\,1926 Sand in Taufers), \emph{Journalist, Übersetzer, Theaterkritiker}|pwk} hatte mehrere Übersetzungen von Einaktern Schnitzlers angefertigt:
                  \emph{Il matrimonio d’Anatolio}\pwindex{Schnitzler, Arthur 15.\,5.\,1862 Wien – 21.\,10.\,1931 ebd.@\textsc{Schnitzler, Arthur} (15.\,5.\,1862 Wien – 21.\,10.\,1931 ebd.), \emph{Schriftsteller, Mediziner}!matrimonio d’Anatolio@\strich\emph{Il matrimonio d’Anatolio}|pwk} (\emph{Anatols
                     Hochzeitsmorgen}\pwindex{Schnitzler, Arthur 15.\,5.\,1862 Wien – 21.\,10.\,1931 ebd.@\textsc{Schnitzler, Arthur} (15.\,5.\,1862 Wien – 21.\,10.\,1931 ebd.), \emph{Schriftsteller, Mediziner}!Anatols Hochzeitsmorgen@\strich\emph{Anatols Hochzeitsmorgen}|pwk}), \emph{Cena d’addio}\pwindex{Schnitzler, Arthur 15.\,5.\,1862 Wien – 21.\,10.\,1931 ebd.@\textsc{Schnitzler, Arthur} (15.\,5.\,1862 Wien – 21.\,10.\,1931 ebd.), \emph{Schriftsteller, Mediziner}!Cena d’addio@\strich\emph{Cena d’addio}|pwk}
                  (\emph{Abschiedssouper}\pwindex{Schnitzler, Arthur 15.\,5.\,1862 Wien – 21.\,10.\,1931 ebd.@\textsc{Schnitzler, Arthur} (15.\,5.\,1862 Wien – 21.\,10.\,1931 ebd.), \emph{Schriftsteller, Mediziner}!Abschiedssouper@\strich\emph{Abschiedssouper}|pwk}), \emph{Letteratura}\pwindex{Schnitzler, Arthur 15.\,5.\,1862 Wien – 21.\,10.\,1931 ebd.@\textsc{Schnitzler, Arthur} (15.\,5.\,1862 Wien – 21.\,10.\,1931 ebd.), \emph{Schriftsteller, Mediziner}!Letteratura@\strich\emph{Letteratura}|pwk}
                  (\emph{Literatur}\pwindex{Schnitzler, Arthur 15.\,5.\,1862 Wien – 21.\,10.\,1931 ebd.@\textsc{Schnitzler, Arthur} (15.\,5.\,1862 Wien – 21.\,10.\,1931 ebd.), \emph{Schriftsteller, Mediziner}!Literatur@\strich\emph{Literatur}|pwk}), \emph{Il burattinaio}\pwindex{Schnitzler, Arthur 15.\,5.\,1862 Wien – 21.\,10.\,1931 ebd.@\textsc{Schnitzler, Arthur} (15.\,5.\,1862 Wien – 21.\,10.\,1931 ebd.), \emph{Schriftsteller, Mediziner}!burattinaio@\strich\emph{Il burattinaio}|pwk} (\emph{Der
                     Puppenspieler}\pwindex{Schnitzler, Arthur 15.\,5.\,1862 Wien – 21.\,10.\,1931 ebd.@\textsc{Schnitzler, Arthur} (15.\,5.\,1862 Wien – 21.\,10.\,1931 ebd.), \emph{Schriftsteller, Mediziner}!Puppenspieler. Studie in einem Aufzuge@\strich\emph{Der Puppenspieler. Studie in einem Aufzuge}|pwk}) und \emph{L’ultime maschere}\pwindex{Schnitzler, Arthur 15.\,5.\,1862 Wien – 21.\,10.\,1931 ebd.@\textsc{Schnitzler, Arthur} (15.\,5.\,1862 Wien – 21.\,10.\,1931 ebd.), \emph{Schriftsteller, Mediziner}!Le ultime maschere. Drama in un atto@\strich\emph{Le ultime maschere. Drama in un atto}|pwk} (\emph{Die
                     letzten Masken}\pwindex{Schnitzler, Arthur 15.\,5.\,1862 Wien – 21.\,10.\,1931 ebd.@\textsc{Schnitzler, Arthur} (15.\,5.\,1862 Wien – 21.\,10.\,1931 ebd.), \emph{Schriftsteller, Mediziner}!letzten Masken@\strich\emph{Die letzten Masken}|pwk}).}}}\label{K_L01963-1}{ }einige meiner Einakter\pwindex{Schnitzler, Arthur 15.\,5.\,1862 Wien – 21.\,10.\,1931 ebd.@\textsc{Schnitzler, Arthur} (15.\,5.\,1862 Wien – 21.\,10.\,1931 ebd.), \emph{Schriftsteller, Mediziner}!Anatols Hochzeitsmorgen@\strich\emph{Anatols Hochzeitsmorgen}|pwv}\pwindex{Schnitzler, Arthur 15.\,5.\,1862 Wien – 21.\,10.\,1931 ebd.@\textsc{Schnitzler, Arthur} (15.\,5.\,1862 Wien – 21.\,10.\,1931 ebd.), \emph{Schriftsteller, Mediziner}!Puppenspieler. Studie in einem Aufzuge@\strich\emph{Der Puppenspieler. Studie in einem Aufzuge}|pwv}\pwindex{Schnitzler, Arthur 15.\,5.\,1862 Wien – 21.\,10.\,1931 ebd.@\textsc{Schnitzler, Arthur} (15.\,5.\,1862 Wien – 21.\,10.\,1931 ebd.), \emph{Schriftsteller, Mediziner}!Abschiedssouper@\strich\emph{Abschiedssouper}|pwv}\pwindex{Schnitzler, Arthur 15.\,5.\,1862 Wien – 21.\,10.\,1931 ebd.@\textsc{Schnitzler, Arthur} (15.\,5.\,1862 Wien – 21.\,10.\,1931 ebd.), \emph{Schriftsteller, Mediziner}!letzten Masken@\strich\emph{Die letzten Masken}|pwv}\pwindex{Schnitzler, Arthur 15.\,5.\,1862 Wien – 21.\,10.\,1931 ebd.@\textsc{Schnitzler, Arthur} (15.\,5.\,1862 Wien – 21.\,10.\,1931 ebd.), \emph{Schriftsteller, Mediziner}!Literatur@\strich\emph{Literatur}|pwv} in Italien\oindex{Italien@\textbf{Italien}|pw} aufgeführt worden sind und seither eine wahre Flut von Lire auf mich
               niederströmt. \substVorne{}\textsuperscript{Neulich}\substDazwischen{}Im letzten Vierteljahr\substHinten{} waren es vierzehn.\pend
           
\pstart
           Nächstens bekommst Du den Medardus\pwindex{Schnitzler, Arthur 15.\,5.\,1862 Wien – 21.\,10.\,1931 ebd.@\textsc{Schnitzler, Arthur} (15.\,5.\,1862 Wien – 21.\,10.\,1931 ebd.), \emph{Schriftsteller, Mediziner}!junge Medardus. Dramatische Historie in einem Vorspiel und fünf Aufzügen@\strich\emph{Der junge Medardus. Dramatische Historie in einem Vorspiel und fünf Aufzügen}|pw}.\pend
           
\pstart
           Herzlichst Dein{\\[\baselineskip]}\spacefill\mbox{{[}hs.:{]} Arthur.}\pend
           \leftskip=0em{}\selectlanguage{ngerman}\endnumbering\briefempfaengerindex{Bahr, Hermann@\textsc{Bahr, Hermann}!zzzSchnitzler, Arthur@\emph{von Arthur Schnitzler}!1910-10-081@{8. 10. 1910}|)be}\mylabel{L01963h}  \newcommand{\dateiname}{L01963}\newcommand{\titel}{Arthur Schnitzler an Hermann Bahr, 8. 10. 1910}\newcommand{\editorInnen}{Herausgegeben von Martin Anton Müller}%% latex-leseansicht-abspann.tex
%% Abspann für die Leseansicht.
%% Der Schalter \ifkorrekturansicht ist bereits durch den Vorspann gesetzt.

%% latex-abspann.tex
%% Gemeinsamer Abspann für Korrekturansicht und Leseansicht.
%% Setzt den Schalter \ifkorrekturansicht voraus (gesetzt in den
%% einbindenden Dateien latex-korrekturansicht-abspann.tex bzw.
%% latex-leseansicht-abspann.tex).
%% ---------------------------------------------------------------

\normalsize

% Das esempio-Environment wird nur in der Leseansicht benötigt
\ifkorrekturansicht\else
\newenvironment{esempio}[3]%
{
    \vspace{1.5ex}
    \rlap{\underline{#1}}
    \par
    \setlength{\parindent}{0cm}
    \nopagebreak
    \leftskip=#2cm
    \rightskip=#3cm
}
{
    \par
}
\fi

\doendnotes{C}
\bigskip
\vfill

\clearpage

\footnotesize

\ifkorrekturansicht
  \lohead{\textsc{register}}
\fi

% theindex-Environment neu definieren ohne reledmac
\makeatletter
\renewenvironment{theindex}{%
  \ifkorrekturansicht
    \section*{\indexname}%
  \else
    \subsubsection*{Index der erwähnten Entitäten}%
  \fi
  \setlength{\parindent}{0pt}%
  \setlength{\parskip}{0pt plus 0.3pt}%
  \let\item\@idxitem
}{%
  \ifkorrekturansicht\clearpage\fi
}
\makeatother

\IfFileExists{\jobname-pw.ind}{\input{\jobname-pw.ind}}{}

% Quellenangabe nur in der Leseansicht
\ifkorrekturansicht\else
% Fallback-Definitionen, falls die .tex-Datei \titel etc. nicht gesetzt hat
\providecommand{\titel}{}
\providecommand{\editorInnen}{}
\providecommand{\dateiname}{\jobname}

\vspace{3cm}

\vfill

\footnotesize
\textsc{Quelle}: \titel. Herausgegeben von {\editorInnen}. In: \emph{Arthur Schnitzler: Briefwechsel mit Autorinnen und Autoren}.
 Digitale Edition, https://schnitzler-briefe.acdh.oeaw.ac.at/{\dateiname}.html (Stand \today)
\fi

\end{document}


