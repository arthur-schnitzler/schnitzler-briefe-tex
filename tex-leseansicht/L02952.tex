%% latex-korrekturansicht-vorspann.tex
%% Vorspann für die Korrekturansicht.
%% Lädt die gemeinsame Datei latex-vorspann.tex mit gesetztem Schalter.

\newif\ifkorrekturansicht
\korrekturansichttrue

\input{../tex-inputs/latex-vorspann}


\section[Arthur Schnitzler an Felix Salten, 14. 9. 1891]{L02952 Arthur Schnitzler an Felix Salten, 14. 9. 1891}
\nopagebreak\mylabel{L02952v}
\rehead{ }\normalsize\beginnumbering\briefempfaengerindex{Salten, Felix@\textsc{Salten, Felix}!zzzSchnitzler, Arthur@\emph{von Arthur Schnitzler}!1891-09-141@{14. 9. 1891}|(be}
\toendnotes[C]{\smallbreak\pagebreak[2]}\Standort{Wienbibliothek im Rathaus, ZPH 1681, 2.1.516.}
\physDesc{Brief, 1 Blatt, 3 Seiten, 679 Zeichen
\newline{}Handschrift: schwarze Tinte, deutsche Kurrent
\newline{}Ordnung: mit Bleistift von unbekannter Hand Nummerierung der Doppelseiten des Konvoluts:
                                    »87«–»88« }\toendnotes[C]{\smallbreak}
\pstart
           \noindent{}{\pb}Mein lieber Freund, ich werde wahrscheinlich doch nach
                  \label{K_L02952-1v}\edtext{Halle\oindex{Halle (Saale)@\textbf{Halle (Saale)}, \emph{P.PPL}|pw}}{\lemma{\textnormal{\emph{Halle}}}\Cendnote{\textnormal{Schnitzler war vom 19. 9. 1891 bis 24. 9. 1891 in Halle an der Saale\oindex{Halle (Saale)@\textbf{Halle (Saale)}, \emph{P.PPL}|pwk}, wo er die Versammlung der
                     \emph{Gesellschaft Deutscher Naturforscher und
                     Ärzte}\orgindex{Gesellschaft Deutscher Naturforscher und Aerzte@Gesellschaft Deutscher Naturforscher und Ärzte|pwk} besuchte. Siehe Arthur Schnitzler an Richard Beer-Hofmann, 16. 9. 1891.}}}\label{K_L02952-1} müſſen, entſetzlich! – Bitte, machen Sie \label{K_L02952-2v}\edtext{\textsc{Italien\oindex{Italien@\textbf{Italien}, \emph{A.PCLI}|pw}}}{\lemma{\textnormal{\emph{Italien}}}\Cendnote{\textnormal{Siehe Arthur Schnitzler an Felix Salten, [10.? 9. 1891].
               }}}\label{K_L02952-2} möglich – wenigſtens 14 Tage für Venedig\oindex{Venedig@\textbf{Venedig}, \emph{P.PPLA}|pw}
               und die \textsc{ob. ital.} Seen\oindex{Oberitalienische Seen@\textbf{Oberitalienische Seen}, \emph{Region}|pw} – das muſs doch gehn{\dotstwo} Wa{\geminationn}{ }\label{K_L02952-3v}\edtext{ko{\geminationm}en}{\lemma{\textnormal{\emph{kommen}}}\Cendnote{\textnormal{Siehe Felix Salten an Arthur Schnitzler, [28. 9. 1891?].
               }}}\label{K_L02952-3} Sie? We{\geminationn} ich nach Halle\oindex{Halle (Saale)@\textbf{Halle (Saale)}, \emph{P.PPL}|pw} muſs, dürft ich wohl Freitag weg. Und
                  we{\geminationn} ich zurückko{\geminationm}e, iſt
                  {\pb}man ſchon in \label{K_L02952-4v}\edtext{T.\oindex{Opava@\textbf{Opava}, \emph{P.PPL}|pw}}{\lemma{\textnormal{\emph{T.}}}\Cendnote{\textnormal{Siehe Felix Salten an Arthur Schnitzler, 12. 9. 1891.
               }}}\label{K_L02952-4}, wo man am 23. eintreffen muſs. – Wie, das
               Brief ſchreiben iſt ein recht matter Ersatz fürs Plaudern! Man ſollte einen Secretair
               haben, der ſofort nachſchreibt und dabei nichts verſteht. Aviſiren Sie mich, ſobald
               Sie kommen. – Hoffentlich schüttl’ ich diese verda{\geminationm}te
               Naturforſcherversammlung noch ab. Sie glauben nicht, wie die mir’s ſtiert.\pend
           
\pstart
           {\pb}Auf Wiederſehen, bald, ja?\pend
           \pstart Ihr \spacefill\mbox{ArthSchnitzl}\pend{}
\pstart
           14. 9. 91.\pend
           \selectlanguage{ngerman}\endnumbering\briefempfaengerindex{Salten, Felix@\textsc{Salten, Felix}!zzzSchnitzler, Arthur@\emph{von Arthur Schnitzler}!1891-09-141@{14. 9. 1891}|)be}\mylabel{L02952h}  \normalsize

\doendnotes{C}
\bigskip
\vfill

\clearpage

\footnotesize

\lohead{\textsc{register}}

% Definiere theindex-Environment komplett neu ohne reledmac
\makeatletter
\renewenvironment{theindex}{%
  \section*{\indexname}%
  \setlength{\parindent}{0pt}%
  \setlength{\parskip}{0pt plus 0.3pt}%
  \let\item\@idxitem
}{%
  \clearpage
}
\makeatother

\IfFileExists{\jobname-pw.ind}{\input{\jobname-pw.ind}}{}

\end{document}

      