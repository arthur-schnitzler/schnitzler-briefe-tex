%% latex-leseansicht-vorspann.tex
%% Vorspann für die Leseansicht.
%% Lädt die gemeinsame Datei latex-vorspann.tex mit nicht gesetztem Schalter.

\newif\ifkorrekturansicht
\korrekturansichtfalse

\input{../tex-inputs/latex-vorspann}

\begin{center}
            \textcolor{red}{ENTWURF, NICHT FERTIG KORRIGIERT}
                      \end{center}
            
         
         \renewcommand{\erwaehntePersonen}{Personen: Felix Salten}
         \renewcommand{\erwaehnteInstitutionen}{Institutionen: Gesellschaft Deutscher Naturforscher und Ärzte}
         \renewcommand{\erwaehnteOrte}{Orte: Halle an der Saale, Italien, Miskolc, Oberitalienische Seen, Opava, Venedig, Wien}
         \renewcommand{\erwaehnteWerke}{}
               \section[Arthur Schnitzler an Felix Salten, 14. 9. 1891]{ Arthur Schnitzler an Felix Salten, 14. 9. 1891}\nopagebreak\mylabel{v}\rehead{ }\begin{ledgroupsized}[t]{13cm}\normalsize\beginnumbering \toendnotes[C]{\smallbreak\pagebreak[2]} \Standort{Wienbibliothek im Rathaus, ZPH 1681, 2.1.516.}
\physDesc{Brief, 1 Blatt, 3 Seiten, 679 Zeichen
\newline{}Handschrift: schwarze Tinte, deutsche Kurrent
\newline{}Ordnung: mit Bleistift von unbekannter Hand Nummerierung der Doppelseiten des Konvoluts:
                                    »87«–»88« }\toendnotes[C]{\smallbreak}\pstart
           \noindent{}{\pb}Mein lieber Freund, ich werde wahrscheinlich doch nach
                  \label{K_L02952-1v}\edtext{Halle\oindex{Halle an der Saale@\textbf{Halle an der Saale}|pw}}{\lemma{\textnormal{\emph{Halle}}}\Cendnote{\textnormal{Schnitzler\pwindex{Schnitzler, Arthur 15.05.1862 – 21.10.1931@\textsc{Schnitzler, Arthur} (15.05.1862 – 21.10.1931), \emph{Schriftsteller, Mediziner}|pwk} war von 19. 9. 1891 bis 24. 9. 1891 in Halle an der Saale\oindex{Halle an der Saale@\textbf{Halle an der Saale}|pwk}, wo er die Versammlung der
                     \emph{Gesellschaft Deutscher Naturforscher und
                     Ärzte}\orgindex{Gesellschaft Deutscher Naturforscher und Aerzte@Gesellschaft Deutscher Naturforscher und Ärzte|pwk} besuchte. Siehe Arthur Schnitzler an Richard Beer-Hofmann, 16. 9. 1891.}}}\label{K_L02952-1h} müſſen, entſetzlich! – Bitte, machen Sie \label{K_L02952-2v}\edtext{\textsc{Italien\oindex{Italien@\textbf{Italien}|pw}}}{\lemma{\textnormal{\emph{Italien}}}\Cendnote{\textnormal{siehe Arthur Schnitzler an Felix Salten, [10.? 9. 1891]}}}\label{K_L02952-2h} möglich – wenigſtens 14 Tage für Venedig\oindex{Venedig@\textbf{Venedig}|pw}
               und die \textsc{ob. ital.} Seen\oindex{Oberitalienische Seen@\textbf{Oberitalienische Seen}|pw} – das muſs doch gehn{\dotstwo} Wa{\geminationn}{ }\label{K_L02952-3v}\edtext{ko{\geminationm}en}{\lemma{\textnormal{\emph{kommen}}}\Cendnote{\textnormal{siehe Felix Salten an Arthur Schnitzler, [28. 9. 1891?]}}}\label{K_L02952-3h} Sie? We{\geminationn} ich nach Halle\oindex{Halle an der Saale@\textbf{Halle an der Saale}|pw} muſs, dürft ich wohl Freitag weg. Und
                  we{\geminationn} ich zurückko{\geminationm}e, iſt
                  {\pb}man ſchon in \label{K_L02952-4v}\edtext{T.\oindex{Opava@\textbf{Opava}|pw}}{\lemma{\textnormal{\emph{T.}}}\Cendnote{\textnormal{siehe Felix Salten an Arthur Schnitzler, 12. 9. 1891}}}\label{K_L02952-4h}, wo man am 23. eintreffen muſs. – Wie, das
               Brief ſchreiben iſt ein recht matter Ersatz fürs Plaudern! Man ſollte einen Secretair
               haben, der ſofort nachſchreibt und dabei nichts verſteht. Aviſiren Sie mich, ſobald
               Sie kommen. – Hoffentlich schüttl’ ich diese verda{\geminationm}te
               Naturforſcherversammlung noch ab. Sie glauben nicht, wie die mir’s ſtiert.\pend
           \pstart
           {\pb}Auf Wiederſehen, bald, ja?\pend
           \pstart Ihr \spacefill\mbox{ArthSchnitzl}\pend{}\pstart
           14. 9. 91.\pend
           
         
         \endnumbering\mylabel{h}\end{ledgroupsized}  \newcommand{\dateiname}{L02952}\newcommand{\titel}{Arthur Schnitzler an Felix Salten, 14. 9. 1891}\newcommand{\editorInnen}{Martin Anton Müller und Laura Untner}%% latex-leseansicht-abspann.tex
%% Abspann für die Leseansicht.
%% Der Schalter \ifkorrekturansicht ist bereits durch den Vorspann gesetzt.

%% latex-abspann.tex
%% Gemeinsamer Abspann für Korrekturansicht und Leseansicht.
%% Setzt den Schalter \ifkorrekturansicht voraus (gesetzt in den
%% einbindenden Dateien latex-korrekturansicht-abspann.tex bzw.
%% latex-leseansicht-abspann.tex).
%% ---------------------------------------------------------------

\normalsize

% Das esempio-Environment wird nur in der Leseansicht benötigt
\ifkorrekturansicht\else
\newenvironment{esempio}[3]%
{
    \vspace{1.5ex}
    \rlap{\underline{#1}}
    \par
    \setlength{\parindent}{0cm}
    \nopagebreak
    \leftskip=#2cm
    \rightskip=#3cm
}
{
    \par
}
\fi

\doendnotes{C}
\bigskip
\vfill

\clearpage

\footnotesize

\ifkorrekturansicht
  \lohead{\textsc{register}}
\fi

% theindex-Environment neu definieren ohne reledmac
\makeatletter
\renewenvironment{theindex}{%
  \ifkorrekturansicht
    \section*{\indexname}%
  \else
    \subsubsection*{Index der erwähnten Entitäten}%
  \fi
  \setlength{\parindent}{0pt}%
  \setlength{\parskip}{0pt plus 0.3pt}%
  \let\item\@idxitem
}{%
  \ifkorrekturansicht\clearpage\fi
}
\makeatother

\IfFileExists{\jobname-pw.ind}{\input{\jobname-pw.ind}}{}

% Quellenangabe nur in der Leseansicht
\ifkorrekturansicht\else
% Fallback-Definitionen, falls die .tex-Datei \titel etc. nicht gesetzt hat
\providecommand{\titel}{}
\providecommand{\editorInnen}{}
\providecommand{\dateiname}{\jobname}

\vspace{3cm}

\vfill

\footnotesize
\textsc{Quelle}: \titel. Herausgegeben von {\editorInnen}. In: \emph{Arthur Schnitzler: Briefwechsel mit Autorinnen und Autoren}.
 Digitale Edition, https://schnitzler-briefe.acdh.oeaw.ac.at/{\dateiname}.html (Stand \today)
\fi

\end{document}


      