%% latex-korrekturansicht-vorspann.tex
%% Vorspann für die Korrekturansicht.
%% Lädt die gemeinsame Datei latex-vorspann.tex mit gesetztem Schalter.

\newif\ifkorrekturansicht
\korrekturansichttrue

\input{../tex-inputs/latex-vorspann}


\section[Arthur Schnitzler an Richard Beer-Hofmann, 2. 11. 1904]{L01464 Arthur Schnitzler an Richard Beer-Hofmann, 2. 11. 1904}
\nopagebreak\mylabel{L01464v}
\rehead{ }\normalsize\beginnumbering\briefempfaengerindex{Beer-Hofmann, Richard@\textsc{Beer-Hofmann, Richard}!zzzSchnitzler, Arthur@\emph{von Arthur Schnitzler}!1904-11-021@{2. 11. 1904}|(be}
\toendnotes[C]{\smallbreak\pagebreak[2]}\Standort{YCGL, MSS 31.}
\physDesc{Brief, 1 Blatt, 3 Seiten, Umschlag, 657 Zeichen
\newline{}Handschrift: Bleistift, deutsche Kurrent
\newline{}Versand: 1) Stempel: »\nobreak{}Wien, 2. XI. 04, 7\nobreak{}«.   2) Stempel: »\nobreak{}\oindex{Rodaun@\textbf{Rodaun}, \emph{A.ADM4}|pwk}Rodau\textcolor{gray}{n}, 3 \textcolor{gray}{11 04}\nobreak{}«. 
\newline{}Beer-Hofmann: mit schwarzer Tinte das Datum der Beantwortung notiert: »4/XI. b.« }
\buchAbdrucke{\weitereDrucke{Arthur Schnitzler, Richard Beer-Hofmann: \emph{Briefwechsel 1891–1931}. Wien, Zürich: \emph{Europaverlag} 1992, S. 167.} }\toendnotes[C]{\smallbreak}\pstart{}{\pb}Wien\oindex{Wien@\textbf{Wien}, \emph{A.ADM2}|pw}\pend{}\pstart{}\textsc{Arthur Schnitzler XIII Spoettelg\oindex{Edmund-Weiss-Gasse 7@\textbf{Edmund-Weiß-Gasse 7}, \emph{Wohngebäude (K.WHS)}|pw}}\pend{}{\bigskip}\pstart{}{\pb}\textsc{Herrn Dr Rich. Beer-Hofmann}\pend{}\pstart{}\textsc{Rodaun\oindex{Rodaun@\textbf{Rodaun}, \emph{A.ADM4}|pw}}\pend{}\pstart{}L\pend{}\pstart{}\textsc{Liesingerstraße 2}\oindex{Liesingerstrasse@\textbf{Liesingerstraße}, \emph{Straße (K.STR)}|pw}.\pend{}{\bigskip}\vspace{1em}
\pstart
           \raggedleft{}{\pb}2. 11. 904\pend
           \vspace{0.5em}
\pstart
           lieber Richard, ich bekomme heute beiliegendes \label{K_L01464-1v}\edtext{Telegra{\geminationm}}{\lemma{\textnormal{\emph{Telegramm}}}\Cendnote{\textnormal{Im Telegramm vom
                     1. 11. 1904 schrieb Max
                     Reinhardt\pwindex{Reinhardt, Max 09.09.1873 – 30.10.1943@\textsc{Reinhardt, Max} (09.09.1873 – 30.10.1943), \emph{Theaterleiter/Theaterleiterin, Regisseur/Regisseurin, Schauspieler/Schauspielerin}|pwk}, dass sich die Inszenierung von \emph{Der grüne Kakadu}\pwindex{gruene Kakadu. Groteske in einem Akt@\emph{Der grüne Kakadu. Groteske in einem Akt}|pwk}, \emph{Der tapfere
                     Cassian}\pwindex{tapfere Cassian. Puppenspiel in einem Akt@\emph{Der tapfere Cassian. Puppenspiel in einem Akt}|pwk} und \emph{Das Haus Delorme}\pwindex{Haus Delorme. Eine Familienszene@\emph{Das Haus Delorme. Eine Familienszene}|pwk} wegen
                  Erkrankung von Agnes Sorma\pwindex{Sorma, Agnes 17.05.1862 – 10.02.1927@\textsc{Sorma, Agnes} (17.05.1862 – 10.02.1927), \emph{Schauspieler/Schauspielerin}|pwk} weiter verzögere (\emph{Der Briefwechsel Arthur Schnitzlers mit Max Reinhardt und
                        dessen Mitarbeitern}. Herausgegeben von  Renate Wagner. Salzburg: \emph{Otto
                        Müller Verlag}{ }1971, S. 44).}}}\label{K_L01464-1}. Mir ſehr ärgerlich, weil auf mein
               Erſuchen im Volkstheater\orgindex{Volkstheater@Volkstheater|pw}{ }\label{K_L01464-2v}\edtext{\textsc{Freiwild}\pwindex{Freiwild. Schauspiel in 3 Akten@\emph{Freiwild. Schauspiel in 3 Akten}|pw}{ }\textsc{Premiere}\eventindex{Volkstheater@\textbf{Volkstheater}!Premiere von Freiwild, 28.1.1905@Premiere von Freiwild, 28.1.1905|pw}}{\lemma{\textnormal{\emph{Freiwild Premiere}}}\Cendnote{\textnormal{Die Premiere\eventindex{Volkstheater@\textbf{Volkstheater}!Premiere von Freiwild, 28.1.1905@Premiere von Freiwild, 28.1.1905|pwkv} fand letztlich am
                     28. 1. 1905 statt.}}}\label{K_L01464-2} wegen meiner Berlin\oindex{Berlin@\textbf{Berlin}, \emph{P.PPLC}|pw}er \label{K_L01464-3v}\edtext{\textsc{Premiere}\pwindex{gruene Kakadu. Groteske in einem Akt@\emph{Der grüne Kakadu. Groteske in einem Akt}|pwv}\pwindex{tapfere Cassian. Puppenspiel in einem Akt@\emph{Der tapfere Cassian. Puppenspiel in einem Akt}|pwv}}{\lemma{\textnormal{\emph{Premiere}}}\Cendnote{\textnormal{Die Uraufführung\eventindex{Kleines Theater@\textbf{Kleines Theater}!Urauffuehrung von Der tapfere Cassian, Premiere von Der gruene Kakadu, 22.11.1904@Uraufführung von Der tapfere Cassian, Premiere von Der grüne Kakadu, 22.11.1904|pwkv} von \emph{Der tapfere Cassian}\pwindex{tapfere Cassian. Puppenspiel in einem Akt@\emph{Der tapfere Cassian. Puppenspiel in einem Akt}|pwk} zusammen mit einer Neueinstudierung
                    von \emph{Der grüne Kakadu}\pwindex{gruene Kakadu. Groteske in einem Akt@\emph{Der grüne Kakadu. Groteske in einem Akt}|pwk} ging letztlich am 22. 11. 1904
                  vonstatten.}}}\label{K_L01464-3} hinausgeſchoben wurde u es jetzt erſt recht zu einer Colliſion
               kommen dürfte. Ich {\pb}nehme an, daſs nun der Graf v \textsc{Charolais}\pwindex{Graf von Charolais. Ein Trauerspiel@\emph{Der Graf von Charolais. Ein Trauerspiel}|pw} gleich (\label{K_L01464-4v}\edtext{vor \textsc{Ruederer}\pwindex{Morgenroethe. Komoedie aus dem Jahre 1848@\emph{Die Morgenröthe. Komödie aus dem Jahre 1848}|pwv}\pwindex{Ruederer, Josef 15.10.1861 – 20.10.1915@\textsc{Ruederer, Josef} (15.10.1861 – 20.10.1915), \emph{Schriftsteller/Schriftstellerin}|pw}}{\lemma{\textnormal{\emph{vor Ruederer}}}\Cendnote{\textnormal{\emph{Vor Morgenröthe}\pwindex{Morgenroethe. Komoedie aus dem Jahre 1848@\emph{Die Morgenröthe. Komödie aus dem Jahre 1848}|pwk} erlebte am
                  15. 11. 1904 die Uraufführung\eventindex{Neues Theater@\textbf{Neues Theater}!Urauffuehrung von Die Morgenroethe, 15.11.1904@Uraufführung von Die Morgenröthe, 15.11.1904|pwkv}.}}}\label{K_L01464-4}) \label{K_L01464-5v}\edtext{drankommt}{\lemma{\textnormal{\emph{drankommt}}}\Cendnote{\textnormal{\emph{Der Graf von Charolais}\pwindex{Graf von Charolais. Ein Trauerspiel@\emph{Der Graf von Charolais. Ein Trauerspiel}|pwk} hatte am
                     23. 12. 1904{ }Uraufführung\eventindex{Neues Theater@\textbf{Neues Theater}!Urauffuehrung von Der Graf von Charolais, 23.12.1904@Uraufführung von Der Graf von Charolais, 23.12.1904|pwkv}.}}}\label{K_L01464-5} (wobei ich allerdings noch immer
               nicht verſtehe, weshalb er\pwindex{Brahm, Otto 05.02.1856 – 28.11.1912@\textsc{Brahm, Otto} (05.02.1856 – 28.11.1912), \emph{Theaterleiter/Theaterleiterin, Regisseur/Regisseurin}|pwv}
               plötzlich meine Sachen nicht beſetzen kann) – jedenfalls bitte ich Sie mir ein Wort
               zu ſchrei{\pb}ben ſobald Sie aus Berlin\oindex{Berlin@\textbf{Berlin}, \emph{P.PPLC}|pw} eine Nachricht haben u mir auch dieſes Telegr.
               zurückzuſchicken.\pend
           
\pstart
           Herzlichſt Ihr{\\[\baselineskip]}\spacefill\mbox{A.}\pend
           \leftskip=0em{}\selectlanguage{ngerman}\endnumbering\briefempfaengerindex{Beer-Hofmann, Richard@\textsc{Beer-Hofmann, Richard}!zzzSchnitzler, Arthur@\emph{von Arthur Schnitzler}!1904-11-021@{2. 11. 1904}|)be}\mylabel{L01464h}  \normalsize

\doendnotes{C}
\bigskip
\vfill

\clearpage

\footnotesize

\lohead{\textsc{register}}

% Definiere theindex-Environment komplett neu ohne reledmac
\makeatletter
\renewenvironment{theindex}{%
  \section*{\indexname}%
  \setlength{\parindent}{0pt}%
  \setlength{\parskip}{0pt plus 0.3pt}%
  \let\item\@idxitem
}{%
  \clearpage
}
\makeatother

\IfFileExists{\jobname-pw.ind}{\input{\jobname-pw.ind}}{}

\end{document}

      