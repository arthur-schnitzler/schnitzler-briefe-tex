\input{../tex-inputs/latex-pdf-vorspann}
\begin{center}
            \textcolor{red}{ENTWURF. ENTZIFFERUNG NOCH NICHT KORREKTURGELESEN}
                      \end{center}
            
               \section[Arthur Schnitzler an Richard Beer-Hofmann, 2. 11. 1904]{ Arthur Schnitzler an Richard Beer-Hofmann, 2. 11. 1904}\nopagebreak\mylabel{v}\rehead{ }\begin{ledgroupsized}[t]{13cm}\normalsize\beginnumbering\briefempfaengerindex{Beer-Hofmann, Richard@\textsc{Beer-Hofmann, Richard}!zzzSchnitzler, Arthur@\emph{von Arthur Schnitzler}!1904-11-021@{2. 11. 1904}|(be} \toendnotes[C]{\smallbreak\pagebreak[2]} \Standort{YCGL, MSS 31.}
\physDesc{Brief, 1 Blatt, 3 Seiten, Umschlag
\newline{}Handschrift: Bleistift, deutsche Kurrent\newline{}Versand: 1) Stempel: »\nobreak{}Wien, 2. XI. 04, 7\nobreak{}«.  2) Stempel: »\nobreak{}\oindex{Rodaun@\textbf{Rodaun}|pwk}Rodau\textcolor{gray}{n}, 3 \textcolor{gray}{11 04}\nobreak{}«. 
\newline{}Beer-Hofmann: mit Tinte den Zeitpunkt der Beantwortung notiert: »4/XI. b.« }\buchAbdrucke{\weitereDrucke{Arthur Schnitzler, Richard Beer-Hofmann: \emph{Briefwechsel 1891–1931}. Hg. Konstanze Fliedl. Wien, Zürich: \emph{Europaverlag} 1992, S. 167.} }\toendnotes[C]{\smallbreak}\pstart{}{\pb}Wien\oindex{Wien@\textbf{Wien}|pw}\pend{}\pstart{}\textsc{Arthur Schnitzler XIII Spoettelg\oindex{Edmund-Weiss-Gasse@\textbf{Edmund-Weiß-Gasse}|pw}}\pend{}{\bigskip}\pstart{}{\pb}\textsc{Herrn Dr Rich. Beer-Hofmann}\pend{}\pstart{}\textsc{Rodaun\oindex{Rodaun@\textbf{Rodaun}|pw}}\pend{}\pstart{}L\pend{}\pstart{}\textsc{Liesingerstraße 2}\oindex{Liesingerstrasse@\textbf{Liesingerstraße}|pw}.\pend{}{\bigskip}\pstart
           \raggedleft{}{\pb}2. 11. 904\pend
           \pstart
           lieber Richard, ich bekomme heute beiliegendes \label{K_L01464_1v}\edtext{Telegra{\geminationm}}{\lemma{\textnormal{\emph{Telegra}}}\Cendnote{\textnormal{Im Telegramm vom
                     1. 11. 1904 schreibt Max
                     Reinhardt\pwindex{Reinhardt, Max 09.09.1873 – 30.10.1943@\textsc{Reinhardt, Max} (09.09.1873 – 30.10.1943), \emph{Theaterleiter, Regisseur, Schauspieler}|pwk}, dass sich die Inszenierung von \emph{Der
                     grüne Kakadu}\pwindex{Schnitzler, Arthur 15.05.1862 – 21.10.1931@\textsc{Schnitzler, Arthur} (15.05.1862 – 21.10.1931), \emph{Schriftsteller, Mediziner}!gruene Kakadu. Groteske in einem Akt1.3.1899 – 1.3.1899@\strich\emph{Der grüne Kakadu. Groteske in einem Akt} {[}1.3.1899 – 1.3.1899{]}|pwk}, \emph{Der tapfere Cassian}\pwindex{Schnitzler, Arthur 15.05.1862 – 21.10.1931@\textsc{Schnitzler, Arthur} (15.05.1862 – 21.10.1931), \emph{Schriftsteller, Mediziner}!tapfere Cassian. Puppenspiel in einem Akt01. 02. 1904@\strich\emph{Der tapfere Cassian. Puppenspiel in einem Akt} {[}01. 02. 1904{]}|pwk} und
                     \emph{Das Haus Delorme}\pwindex{Schnitzler, Arthur 15.05.1862 – 21.10.1931@\textsc{Schnitzler, Arthur} (15.05.1862 – 21.10.1931), \emph{Schriftsteller, Mediziner}!Haus Delorme. Eine Familienszene1977@\strich\emph{Das Haus Delorme. Eine Familienszene} {[}1977{]}|pwk} wegen Erkrankung von Agnes Sorma\pwindex{Sorma, Agnes 17.05.1862 – 10.02.1927@\textsc{Sorma, Agnes} (17.05.1862 – 10.02.1927), \emph{Schauspielerin}|pwk} weiter verzögere. (\emph{Der Briefwechsel Arthur Schnitzlers mit Max Reinhardt und
                        dessen Mitarbeitern}. Hg. Renate Wagner. Salzburg: \emph{Otto
                        Müller Verlag}{ }1971, S. 44.)}}}\label{K_L01464_1h}. Mir ſehr ärgerlich, weil auf mein
               Erſuchen im Volkstheater\orgindex{Volkstheater@Volkstheater|pw}{ }\label{K_L01464_2v}\edtext{\textsc{Freiwild}{ }\textsc{Premiere}\pwindex{Schnitzler, Arthur 15.05.1862 – 21.10.1931@\textsc{Schnitzler, Arthur} (15.05.1862 – 21.10.1931), \emph{Schriftsteller, Mediziner}!Freiwild. Schauspiel in 3 Akten1896@\strich\emph{Freiwild. Schauspiel in 3 Akten} {[}1896{]}|pw}}{\lemma{\textnormal{\emph{Freiwild Premiere}}}\Cendnote{\textnormal{Diese fand letztlich am
                     28. 1. 1905 statt.}}}\label{K_L01464_2h} wegen meiner Berlin\oindex{Berlin@\textbf{Berlin}|pw}er \label{K_L01464_3v}\edtext{\textsc{Premiere}\pwindex{Schnitzler, Arthur 15.05.1862 – 21.10.1931@\textsc{Schnitzler, Arthur} (15.05.1862 – 21.10.1931), \emph{Schriftsteller, Mediziner}!gruene Kakadu. Groteske in einem Akt1.3.1899 – 1.3.1899@\strich\emph{Der grüne Kakadu. Groteske in einem Akt} {[}1.3.1899 – 1.3.1899{]}|pwv}\pwindex{Schnitzler, Arthur 15.05.1862 – 21.10.1931@\textsc{Schnitzler, Arthur} (15.05.1862 – 21.10.1931), \emph{Schriftsteller, Mediziner}!tapfere Cassian. Puppenspiel in einem Akt01. 02. 1904@\strich\emph{Der tapfere Cassian. Puppenspiel in einem Akt} {[}01. 02. 1904{]}|pwv}}{\lemma{\textnormal{\emph{Premiere}}}\Cendnote{\textnormal{Die Uraufführung von \emph{Der tapfere Cassian}\pwindex{Schnitzler, Arthur 15.05.1862 – 21.10.1931@\textsc{Schnitzler, Arthur} (15.05.1862 – 21.10.1931), \emph{Schriftsteller, Mediziner}!gruene Kakadu. Groteske in einem Akt1.3.1899 – 1.3.1899@\strich\emph{Der grüne Kakadu. Groteske in einem Akt} {[}1.3.1899 – 1.3.1899{]}|pwk} zusammen mit einer Neueinstudierung von
                     \emph{Der grüne Kakadu}\pwindex{Schnitzler, Arthur 15.05.1862 – 21.10.1931@\textsc{Schnitzler, Arthur} (15.05.1862 – 21.10.1931), \emph{Schriftsteller, Mediziner}!tapfere Cassian. Puppenspiel in einem Akt01. 02. 1904@\strich\emph{Der tapfere Cassian. Puppenspiel in einem Akt} {[}01. 02. 1904{]}|pwk} ging letztlich am
                     22. 11. 1904 vonstatten.}}}\label{K_L01464_3h} hinausgeſchoben wurde u es jetzt
               erſt recht zu einer Colliſion kommen dürfte. Ich {\pb}nehme an, daſs nun der Graf v \textsc{Charolais}\pwindex{Beer-Hofmann, Richard 11.07.1866 – 26.09.1945@\textsc{Beer-Hofmann, Richard} (11.07.1866 – 26.09.1945), \emph{Schriftsteller}!Graf von Charolais. Ein Trauerspiel1904-12-23 – 1904-12-23@\strich\emph{Der Graf von Charolais. Ein Trauerspiel} {[}1904-12-23 – 1904-12-23{]}|pw} gleich (\label{K_L01464_4v}\edtext{vor \textsc{Ruederer}\pwindex{Ruederer, Josef 15.10.1861 – 20.10.1915@\textsc{Ruederer, Josef} (15.10.1861 – 20.10.1915), \emph{Schriftsteller}!Morgenroethe. Komoedie aus dem Jahre 184815. 11. 1904@\strich\emph{Die Morgenröthe. Komödie aus dem Jahre 1848} {[}15. 11. 1904{]}|pwv}\pwindex{Ruederer, Josef 15.10.1861 – 20.10.1915@\textsc{Ruederer, Josef} (15.10.1861 – 20.10.1915), \emph{Schriftsteller}|pw}}{\lemma{\textnormal{\emph{vor Ruederer}}}\Cendnote{\textnormal{\emph{Vor Morgenröthe}\pwindex{Ruederer, Josef 15.10.1861 – 20.10.1915@\textsc{Ruederer, Josef} (15.10.1861 – 20.10.1915), \emph{Schriftsteller}!Morgenroethe. Komoedie aus dem Jahre 184815. 11. 1904@\strich\emph{Die Morgenröthe. Komödie aus dem Jahre 1848} {[}15. 11. 1904{]}|pwk} erlebte am
                     15. 11. 1904 die Uraufführung.}}}\label{K_L01464_4h}) \label{K_L01464_5v}\edtext{drankommt}{\lemma{\textnormal{\emph{drankommt}}}\Cendnote{\textnormal{\emph{Der Graf von Charolais}\pwindex{Beer-Hofmann, Richard 11.07.1866 – 26.09.1945@\textsc{Beer-Hofmann, Richard} (11.07.1866 – 26.09.1945), \emph{Schriftsteller}!Graf von Charolais. Ein Trauerspiel1904-12-23 – 1904-12-23@\strich\emph{Der Graf von Charolais. Ein Trauerspiel} {[}1904-12-23 – 1904-12-23{]}|pwk} hatte am
                     23. 12. 1904 Uraufführung.}}}\label{K_L01464_5h} (wobei ich allerdings noch immer
               nicht verſtehe, weshalb er\pwindex{Brahm, Otto 05.02.1856 – 28.11.1912@\textsc{Brahm, Otto} (05.02.1856 – 28.11.1912), \emph{Theaterleiter, Regisseur}|pwv}
               plötzlich meine Sachen nicht beſetzen kann) – jedenfalls bitte ich Sie mir ein Wort
               zu ſchrei{\pb}ben ſobald Sie aus Berlin\oindex{Berlin@\textbf{Berlin}|pw} eine Nachricht haben u mir auch dieſes Telegr.
               zurückzuſchicken.\pend
           \pstart
           Herzlichſt Ihr{\\[\baselineskip]}\spacefill\mbox{A.}\pend
           \leftskip=0em{}\endnumbering\briefempfaengerindex{Beer-Hofmann, Richard@\textsc{Beer-Hofmann, Richard}!zzzSchnitzler, Arthur@\emph{von Arthur Schnitzler}!1904-11-021@{2. 11. 1904}|)be}\mylabel{h}\end{ledgroupsized}  \newcommand{\dateiname}{L01464}\newcommand{\titel}{Arthur Schnitzler an Richard Beer-Hofmann, 2. 11. 1904}\newcommand{\editorInnen}{Martin Anton Müller und Gerd-Hermann Susen}\input{../tex-inputs/latex-pdf-abspann}
      