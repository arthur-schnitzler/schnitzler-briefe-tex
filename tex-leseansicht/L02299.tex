%% latex-leseansicht-vorspann.tex
%% Vorspann für die Leseansicht.
%% Lädt die gemeinsame Datei latex-vorspann.tex mit nicht gesetztem Schalter.

\newif\ifkorrekturansicht
\korrekturansichtfalse

\input{../tex-inputs/latex-vorspann}


               \section[Arthur Schnitzler an Hermann Bahr, 24. 8. 1918]{ Arthur Schnitzler an Hermann Bahr, 24. 8. 1918}\nopagebreak\mylabel{v}\rehead{ }\begin{ledgroupsized}[t]{13cm}\normalsize\beginnumbering\briefempfaengerindex{Bahr, Hermann@\textsc{Bahr, Hermann}!zzzSchnitzler, Arthur@\emph{von Arthur Schnitzler}!1918-08-241@{24. 8. 1918}|(be} \toendnotes[C]{\smallbreak\pagebreak[2]} \Standort{TMW, HS AM 39902 Ba.}
\physDesc{Briefkarte
\newline{}Handschrift: schwarze Tinte, lateinische Kurrent\newline{}Ordnung: Lochung }\buchAbdrucke{\weitereDrucke{1) \emph{24. 8. 1918.} In: Arthur Schnitzler: \emph{The Letters of Arthur Schnitzler to Hermann Bahr}. Edited, annotated, and with an introduction, by Donald G.
                        Daviau. Chapel Hill: \emph{The University of North Carolina Press} 1978, S. 114 (University of North Carolina studies in the Germanic languages
                        and literatures, 89).} \weitereDrucke{2) Hermann Bahr, Arthur Schnitzler: \emph{Briefwechsel, Aufzeichnungen, Dokumente (1891–1931)}. Hg. Kurt Ifkovits und Martin Anton Müller. Göttingen: \emph{Wallstein} 2018, S. 511–512.} }\toendnotes[C]{\smallbreak}\pstart
           \noindent{}{\pb}\textcolor{gray}{\textbf{Dr. Arthur Schnitzler}}\hfill Wien\oindex{Wien@\textbf{Wien}|pw}, 24. 8. 18\pend
           \pstart
           \textcolor{gray}{\textbf{Wien XVIII. Sternwartestrasse 71\oindex{Sternwartestrasse@\textbf{Sternwartestraße}|pw}}}\pend
           \pstart
           lieber Hermann, ein begabter junger Componist, Musikdirector, (mein
                  Sohn\pwindex{Schnitzler, Heinrich 09.08.1902 – 12.07.1982@\textsc{Schnitzler, Heinrich} (09.08.1902 – 12.07.1982), \emph{Regisseur, Schauspieler}|pwv}
               studirt Harmonielehre
               u. Clarinette bei ihm) hat deine Pantomime vom braven Mann\pwindex{Bahr, Hermann 19.07.1863 – 15.01.1934@\textsc{Bahr, Hermann} (19.07.1863 – 15.01.1934), \emph{Schriftsteller, Kritiker}!Pantomime vom braven Manne11. 02. 1893@\strich\emph{Die Pantomime vom braven Manne} {[}11. 02. 1893{]}|pw}\pwindex{Scholz, Arthur Johannes 16.11.1883 – 03.04.1945@\textsc{Scholz, Arthur Johannes} (16.11.1883 – 03.04.1945), \emph{Komponist/Komponistin, Musikpädagoge/Musikpädagogin, Dirigent/Dirigentin}!Pantomime vom braven Mann op. 321918@\strich\emph{Die Pantomime vom braven Mann op. 32} {[}1918{]}|pw} in einer mir sehr interessant erscheinenden Weise vertont und möchte nicht nur
               deine nachträgliche Autorisation erbitten sondern hegt den begreiflichen Wunsch, dir
               die Sache einmal vorzuspielen. Vielleicht bist du so gütig und gibst dem jungen
               Künstler (sein Name ist {\pb}Arthur Johannes Scholz\pwindex{Scholz, Arthur Johannes 16.11.1883 – 03.04.1945@\textsc{Scholz, Arthur Johannes} (16.11.1883 – 03.04.1945), \emph{Komponist/Komponistin, Musikpädagoge/Musikpädagogin, Dirigent/Dirigentin}|pw} – Gelegenheit dazu, wenn du
               dich, was ja (– wenn die \label{K_L02299_1v}\edtext{Zeitungsnachrichten}{\lemma{\textnormal{\emph{Zeitungsnachrichten}}}\Cendnote{\textnormal{Ab
                     18. 8. 1918 wurde mehrfach gemeldet, Bahr\pwindex{Bahr, Hermann 19.07.1863 – 15.01.1934@\textsc{Bahr, Hermann} (19.07.1863 – 15.01.1934), \emph{Schriftsteller, Kritiker}|pwk} gehe nicht als Direktor, sondern als künstlerischer
                  Beirat für ein Jahr ans Burgtheater\oindex{Burgtheater@\textbf{Burgtheater}|pwk}. Die
                  offizielle Bestätigung erfolgte erst nach diesem Brief.}}}\label{K_L02299_1h}
               sti{\geminationm}en) nun öfters der Fall sein dürfte, für ein paar Tage
               in Wien\oindex{Wien@\textbf{Wien}|pw} aufhältst?\pend
           \pstart
           Wie lang hab ich dich nun schon nicht gesehn und gesprochen. Nun wirds hoffentlich
               nicht mehr so lange dauern wie seit dem letzten Mal!\pend
           \pstart Sei herzlichst gegrüßt von Deinem alten \spacefill\mbox{Art}\pend{}          \endnumbering\briefempfaengerindex{Bahr, Hermann@\textsc{Bahr, Hermann}!zzzSchnitzler, Arthur@\emph{von Arthur Schnitzler}!1918-08-241@{24. 8. 1918}|)be}\mylabel{h}\end{ledgroupsized}  \newcommand{\dateiname}{L02299}\newcommand{\titel}{Arthur Schnitzler an Hermann Bahr, 24. 8. 1918}\newcommand{\editorInnen}{ Kurt Ifkovits,  Martin Anton Müller}
            \footnotesize
\begin{ledgroupsized}[t]{11.5cm}
\doendnotes{C}
\end{ledgroupsized}
         %% latex-leseansicht-abspann.tex
%% Abspann für die Leseansicht.
%% Der Schalter \ifkorrekturansicht ist bereits durch den Vorspann gesetzt.

%% latex-abspann.tex
%% Gemeinsamer Abspann für Korrekturansicht und Leseansicht.
%% Setzt den Schalter \ifkorrekturansicht voraus (gesetzt in den
%% einbindenden Dateien latex-korrekturansicht-abspann.tex bzw.
%% latex-leseansicht-abspann.tex).
%% ---------------------------------------------------------------

\normalsize

% Das esempio-Environment wird nur in der Leseansicht benötigt
\ifkorrekturansicht\else
\newenvironment{esempio}[3]%
{
    \vspace{1.5ex}
    \rlap{\underline{#1}}
    \par
    \setlength{\parindent}{0cm}
    \nopagebreak
    \leftskip=#2cm
    \rightskip=#3cm
}
{
    \par
}
\fi

\doendnotes{C}
\bigskip
\vfill

\clearpage

\footnotesize

\ifkorrekturansicht
  \lohead{\textsc{register}}
\fi

% theindex-Environment neu definieren ohne reledmac
\makeatletter
\renewenvironment{theindex}{%
  \ifkorrekturansicht
    \section*{\indexname}%
  \else
    \subsubsection*{Index der erwähnten Entitäten}%
  \fi
  \setlength{\parindent}{0pt}%
  \setlength{\parskip}{0pt plus 0.3pt}%
  \let\item\@idxitem
}{%
  \ifkorrekturansicht\clearpage\fi
}
\makeatother

\IfFileExists{\jobname-pw.ind}{\input{\jobname-pw.ind}}{}

% Quellenangabe nur in der Leseansicht
\ifkorrekturansicht\else
% Fallback-Definitionen, falls die .tex-Datei \titel etc. nicht gesetzt hat
\providecommand{\titel}{}
\providecommand{\editorInnen}{}
\providecommand{\dateiname}{\jobname}

\vspace{3cm}

\vfill

\footnotesize
\textsc{Quelle}: \titel. Herausgegeben von {\editorInnen}. In: \emph{Arthur Schnitzler: Briefwechsel mit Autorinnen und Autoren}.
 Digitale Edition, https://schnitzler-briefe.acdh.oeaw.ac.at/{\dateiname}.html (Stand \today)
\fi

\end{document}


      