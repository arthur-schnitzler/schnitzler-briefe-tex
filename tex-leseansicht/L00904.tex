%% latex-korrekturansicht-vorspann.tex
%% Vorspann für die Korrekturansicht.
%% Lädt die gemeinsame Datei latex-vorspann.tex mit gesetztem Schalter.

\newif\ifkorrekturansicht
\korrekturansichttrue

\input{../tex-inputs/latex-vorspann}


\section[Julius Rodenberg an Arthur Schnitzler, 9. 3. 1899]{L00904 Julius Rodenberg an Arthur Schnitzler, 9. 3. 1899}
\nopagebreak\mylabel{L00904v}
\rehead{ }\normalsize\beginnumbering\briefempfaengerindex{Schnitzler, Arthur@\textsc{Schnitzler, Arthur}!zzzRodenberg, Julius@\emph{von Julius Rodenberg}!1899-03-092@{9. 3. 1899}|(be}
\toendnotes[C]{\smallbreak\pagebreak[2]}\Standort{CUL, Schnitzler, B 85.}
\physDesc{Brief, 1 Blatt, 1 Seite, 895 Zeichen
\newline{}Handschrift: schwarze Tinte, deutsche Kurrent
\newline{}Schnitzler: mit rotem Buntstift eine Unterstreichung }\toendnotes[C]{\smallbreak}
\pstart
           \centering{}{\pb}\textcolor{gray}{\textbf{DEUTSCHE RUNDSCHAU\orgindex{Deutsche Rundschau@Deutsche Rundschau|pw}}}\pend
           
\pstart
           \textcolor{gray}{\textbf{Expedition u. Redaction:}}\hfill \textcolor{gray}{\textbf{Herausgeber:}}\pend
           
\pstart
           \textcolor{gray}{\textbf{Gebrüder Paetel\orgindex{Gebrueder Paetel Verlag@Gebrüder Paetel Verlag|pw} in Berlin\oindex{Berlin@\textbf{Berlin}, \emph{P.PPLC}|pw}}}\hfill \textcolor{gray}{\textbf{Julius Rodenberg in Berlin\oindex{Berlin@\textbf{Berlin}, \emph{P.PPLC}|pw}}}\pend
           
\pstart
           \textcolor{gray}{\textbf{(Elwin Paetel\pwindex{Paetel, Elwin 13.11.1847 – 04.10.1907@\textsc{Paetel, Elwin} (13.11.1847 – 04.10.1907), \emph{Verleger/Verlegerin}|pw})}}\hfill \textcolor{gray}{\textbf{W., Margarethenstr. 1\oindex{Margaretenstrasse [Berlin]@\textbf{Margaretenstraße [Berlin]}, \emph{Straße (K.STR)}|pw}.}}\pend
           
\pstart
           \textcolor{gray}{\textbf{W., Lützowstr. 7\oindex{Luetzowstrasse@\textbf{Lützowstraße}, \emph{Straße (K.STR)}|pw}.}}\pend
           
\pstart
           \raggedleft{}\textbf{\textcolor{gray}{\textbf{Berlin W.\oindex{Berlin@\textbf{Berlin}, \emph{P.PPLC}|pw},}} den}{ }9. März \textcolor{gray}{\textbf{189}}9.\pend
           
\pstart{}Hochgeehrter Herr Doctor!\pend\vspace{0.5em}
\pstart
           Für Ihr freundliches Anerbieten bin ich Ihnen aufrichtig dankbar, doch vermuthen Sie
               mit Recht, daß die »\textsc{Rundschau}\orgindex{Deutsche Rundschau@Deutsche Rundschau|pw}« dramatiſche Dichtungen grundſätzlich nicht bringt. Wir haben wohl, in weiten
               Abſtänden, einmal eine Ausnahme gemacht, aber i{\geminationm}er nur,
               um wieder zu der Regel zurückzukehren; u. ſo gern ich Ihren geiſtvollen Einakter\pwindex{Gefaehrtin. Schauspiel in einem Akt@\emph{Die Gefährtin. Schauspiel in einem Akt}|pwv} in unſerer Zeitſchrift
               ſähe, ſo kann ich es doch nicht, ohne inconſequent gegen Andere zu erſcheinen – um ſo
               weniger, als ich vor Jahr und Tag ſchon eine ſzeniſche Kleinigkeit von einem unſerer
               berühmten Mitarbeiter angenommen habe, die doch zuerſt publiciert werden müßte. Sie
               werden es unter dieſen Umſtänden entſchuldbar finden, wenn ich mit wiederholtem Dank
               ablehne, dagegen hoffe, recht bald durch eine Novelle ſchadlos gehalten zu werden,
               die des Willkomms ſicher ſein darf.\pend
           
\pstart
           Hochachtungsvoll ergeben{\\[\baselineskip]}Ihr{\\[\baselineskip]}\spacefill\mbox{Dr Julius Rodenberg.}\pend
           \leftskip=0em{}\selectlanguage{ngerman}\endnumbering\briefempfaengerindex{Schnitzler, Arthur@\textsc{Schnitzler, Arthur}!zzzRodenberg, Julius@\emph{von Julius Rodenberg}!1899-03-092@{9. 3. 1899}|)be}\mylabel{L00904h}  \normalsize

\doendnotes{C}
\bigskip
\vfill

\clearpage

\footnotesize

\lohead{\textsc{register}}

% Definiere theindex-Environment komplett neu ohne reledmac
\makeatletter
\renewenvironment{theindex}{%
  \section*{\indexname}%
  \setlength{\parindent}{0pt}%
  \setlength{\parskip}{0pt plus 0.3pt}%
  \let\item\@idxitem
}{%
  \clearpage
}
\makeatother

\IfFileExists{\jobname-pw.ind}{\input{\jobname-pw.ind}}{}

\end{document}

      