%% latex-leseansicht-vorspann.tex
%% Vorspann für die Leseansicht.
%% Lädt die gemeinsame Datei latex-vorspann.tex mit nicht gesetztem Schalter.

\newif\ifkorrekturansicht
\korrekturansichtfalse

\input{../tex-inputs/latex-vorspann}


\section[Julius Rodenberg an Arthur Schnitzler, 9. 3. 1899]{L00904 Julius Rodenberg an Arthur Schnitzler, 9. 3. 1899}
\nopagebreak\mylabel{L00904v}
\rehead{ }\normalsize\beginnumbering\briefempfaengerindex{Schnitzler, Arthur@\textsc{Schnitzler, Arthur}!zzzRodenberg, Julius@\emph{von Julius Rodenberg}!1899-03-092@{9. 3. 1899}|(be}
\toendnotes[C]{\smallbreak\pagebreak[2]}
\correspDesc{Versand  durch Julius Rodenberg am 9. 3. 1899 in Berlin
\newline{}Erhalt  durch Arthur Schnitzler im Zeitraum [10. 3. 1899
                  – 14. 3. 1899?] in Wien}\toendnotes[C]{\smallbreak}
\Standort{CUL, Schnitzler, B 85.}
\physDesc{Brief, 1 Blatt, 1 Seite, 895 Zeichen
\newline{}Handschrift: schwarze Tinte, deutsche Kurrent
\newline{}Schnitzler: mit rotem Buntstift eine Unterstreichung }\toendnotes[C]{\smallbreak}
\pstart
           \centering{}{\pb}\textcolor{gray}{\textbf{DEUTSCHE RUNDSCHAU\orgindex{Deutsche Rundschau@Deutsche Rundschau|pw}}}\pend
           
\pstart
           \textcolor{gray}{\textbf{Expedition u. Redaction:}}\hfill \textcolor{gray}{\textbf{Herausgeber:}}\pend
           
\pstart
           \textcolor{gray}{\textbf{Gebrüder Paetel\orgindex{Gebrüder Paetel Verlag@Gebrüder Paetel Verlag|pw} in Berlin\oindex{Berlin@\textbf{Berlin}, \emph{Hauptstadt}|pw}}}\hfill \textcolor{gray}{\textbf{Julius Rodenberg in Berlin\oindex{Berlin@\textbf{Berlin}, \emph{Hauptstadt}|pw}}}\pend
           
\pstart
           \textcolor{gray}{\textbf{(Elwin Paetel\pwindex{Paetel, Elwin 13.\,11.\,1847 Berlin – 4.\,10.\,1907 ebd.@\textsc{Paetel, Elwin} (13.\,11.\,1847 Berlin – 4.\,10.\,1907 ebd.), \emph{Verleger}|pw})}}\hfill \textcolor{gray}{\textbf{W., Margarethenstr. 1\oindex{Margaretenstraße [Berlin]@\textbf{Margaretenstraße [Berlin]}, \emph{Straße}|pw}.}}\pend
           
\pstart
           \textcolor{gray}{\textbf{W., Lützowstr. 7\oindex{Lützowstraße@\textbf{Lützowstraße}, \emph{Straße}|pw}.}}\pend
           
\pstart
           \raggedleft{}\textbf{\textcolor{gray}{\textbf{Berlin W.\oindex{Berlin@\textbf{Berlin}, \emph{Hauptstadt}|pw},}} den}{ }9. März \textcolor{gray}{\textbf{189}}9.\pend
           
\pstart{}Hochgeehrter Herr Doctor!\pend\vspace{0.5em}
\pstart
           Für Ihr freundliches Anerbieten bin ich Ihnen aufrichtig dankbar, doch vermuthen Sie
               mit Recht, daß die »\textsc{Rundschau}\orgindex{Deutsche Rundschau@Deutsche Rundschau|pw}« dramatiſche Dichtungen grundſätzlich nicht bringt. Wir haben wohl, in weiten
               Abſtänden, einmal eine Ausnahme gemacht, aber i{\geminationm}er nur,
               um wieder zu der Regel zurückzukehren; u.{ }ſo gern ich Ihren geiſtvollen Einakter\pwindex{Schnitzler, Arthur 15.\,5.\,1862 Wien – 21.\,10.\,1931 ebd.@\textsc{Schnitzler, Arthur} (15.\,5.\,1862 Wien – 21.\,10.\,1931 ebd.), \emph{Schriftsteller, Mediziner}!Gefährtin. Schauspiel in einem Akt@\strich\emph{Die Gefährtin. Schauspiel in einem Akt}|pwv} in unſerer Zeitſchrift{ }ſähe,{ }ſo kann ich es doch nicht, ohne inconſequent gegen Andere zu erſcheinen – um{ }ſo
               weniger, als ich vor Jahr und Tag{ }ſchon eine{ }ſzeniſche Kleinigkeit von einem unſerer
               berühmten Mitarbeiter angenommen habe, die doch zuerſt publiciert werden müßte. Sie
               werden es unter dieſen Umſtänden entſchuldbar finden, wenn ich mit wiederholtem Dank
               ablehne, dagegen hoffe, recht bald durch eine Novelle{ }ſchadlos gehalten zu werden,
               die des Willkomms{ }ſicher{ }ſein darf.\pend
           
\pstart
           Hochachtungsvoll ergeben{\\[\baselineskip]}Ihr{\\[\baselineskip]}\spacefill\mbox{Dr Julius Rodenberg.}\pend
           \leftskip=0em{}\selectlanguage{ngerman}\endnumbering\briefempfaengerindex{Schnitzler, Arthur@\textsc{Schnitzler, Arthur}!zzzRodenberg, Julius@\emph{von Julius Rodenberg}!1899-03-092@{9. 3. 1899}|)be}\mylabel{L00904h}  \newcommand{\dateiname}{L00904}\newcommand{\titel}{Julius Rodenberg an Arthur Schnitzler, 9. 3. 1899}\newcommand{\editorInnen}{Martin Anton Müller und Gerd-Hermann Susen}%% latex-leseansicht-abspann.tex
%% Abspann für die Leseansicht.
%% Der Schalter \ifkorrekturansicht ist bereits durch den Vorspann gesetzt.

%% latex-abspann.tex
%% Gemeinsamer Abspann für Korrekturansicht und Leseansicht.
%% Setzt den Schalter \ifkorrekturansicht voraus (gesetzt in den
%% einbindenden Dateien latex-korrekturansicht-abspann.tex bzw.
%% latex-leseansicht-abspann.tex).
%% ---------------------------------------------------------------

\normalsize

% Das esempio-Environment wird nur in der Leseansicht benötigt
\ifkorrekturansicht\else
\newenvironment{esempio}[3]%
{
    \vspace{1.5ex}
    \rlap{\underline{#1}}
    \par
    \setlength{\parindent}{0cm}
    \nopagebreak
    \leftskip=#2cm
    \rightskip=#3cm
}
{
    \par
}
\fi

\doendnotes{C}
\bigskip
\vfill

\clearpage

\footnotesize

\ifkorrekturansicht
  \lohead{\textsc{register}}
\fi

% theindex-Environment neu definieren ohne reledmac
\makeatletter
\renewenvironment{theindex}{%
  \ifkorrekturansicht
    \section*{\indexname}%
  \else
    \subsubsection*{Index der erwähnten Entitäten}%
  \fi
  \setlength{\parindent}{0pt}%
  \setlength{\parskip}{0pt plus 0.3pt}%
  \let\item\@idxitem
}{%
  \ifkorrekturansicht\clearpage\fi
}
\makeatother

\IfFileExists{\jobname-pw.ind}{\input{\jobname-pw.ind}}{}

% Quellenangabe nur in der Leseansicht
\ifkorrekturansicht\else
% Fallback-Definitionen, falls die .tex-Datei \titel etc. nicht gesetzt hat
\providecommand{\titel}{}
\providecommand{\editorInnen}{}
\providecommand{\dateiname}{\jobname}

\vspace{3cm}

\vfill

\footnotesize
\textsc{Quelle}: \titel. Herausgegeben von {\editorInnen}. In: \emph{Arthur Schnitzler: Briefwechsel mit Autorinnen und Autoren}.
 Digitale Edition, https://schnitzler-briefe.acdh.oeaw.ac.at/{\dateiname}.html (Stand \today)
\fi

\end{document}


