%% latex-leseansicht-vorspann.tex
%% Vorspann für die Leseansicht.
%% Lädt die gemeinsame Datei latex-vorspann.tex mit nicht gesetztem Schalter.

\newif\ifkorrekturansicht
\korrekturansichtfalse

\input{../tex-inputs/latex-vorspann}


\section[Robert Adam an Arthur Schnitzler, 2. 12. 1918]{L02314 Robert Adam an Arthur Schnitzler, 2. 12. 1918}
\nopagebreak\mylabel{L02314v}
\rehead{ }\normalsize\beginnumbering\briefempfaengerindex{Schnitzler, Arthur@\textsc{Schnitzler, Arthur}!zzzAdam, Robert@\emph{von Robert Adam}!1918-12-021@{2. 12. 1918}|(be}
\toendnotes[C]{\smallbreak\pagebreak[2]}
\correspDesc{Versand  durch Robert Adam am 2. 12. 1918 in Wien
\newline{}Erhalt  durch Arthur Schnitzler im Zeitraum [2. 12. 1918
                  – 6. 12. 1918?] in Wien}\toendnotes[C]{\smallbreak}
\Standort{CUL, Schnitzler, B 1.}
\physDesc{Brief, 1 Blatt, 3 Seiten, 2211 Zeichen
\newline{}Handschrift: schwarze Tinte, deutsche Kurrent
\newline{}Schnitzler: 1) mit Bleistift beschriftet: »\textsc{Adam}«  2) mit rotem Buntstift zwei Unterstreichungen
\newline{}Ordnung: von unbekannter Hand nummeriert: »10« }\Standort{Wien, Österreichische Nationalbibliothek, Cod. ser. 52.263.}
\physDesc{Briefentwurf, 1 Blatt, 4 Seiten, 2211 Zeichen
\newline{}Handschrift: schwarze Tinte, deutsche Kurrent
\newline{}Zusatz: Entwurf des Briefes, datiert auf den 1. 12. 1918
                                 und mit leichten sprachlichen Variationen }\Standort{Wien, Österreichische Nationalbibliothek, Cod.ser. 52.269, 225 verso.}
\physDesc{Brief, maschinenschriftliche Abschrift, 1 Blatt, 1 Seite, 2211 Zeichen
\newline{}Schreibmaschine}\toendnotes[C]{\smallbreak}
\pstart
           \raggedleft{}{\pb}Wien\oindex{Wien@\textbf{Wien}, \emph{Verwaltungsgebiet}|pw}, am 2. Dezember 1918\pend
           
\pstart\center{}Hochverehrter Herr Doktor!\pend\vspace{0.5em}
\pstart
           Verzeihen Sie es meiner bangen Ungeduld, daß ich, obwohl nicht viel mehr als zwei
               Wochen verſtrichen{ }ſind,{ }ſeit ich dem Deutſchen
                  Volkstheater\oindex{Wien@\textbf{Wien}!VII., Neubau@\textbf{VII., Neubau}!Volkstheater@\textbf{Volkstheater}, \emph{Theater}|pw} meine zwei Stücke\pwindex{Adam, Robert 20.\,4.\,1877 Wien – 16.\,10.\,1961 Baden bei Wien@\textsc{Adam, Robert} (20.\,4.\,1877 Wien – 16.\,10.\,1961 Baden bei Wien), \emph{Schriftsteller, Richter}!Yppl. Idylle in fünf Akten@\strich\emph{Yppl. Idylle in fünf Akten}|pwv}\pwindex{Adam, Robert 20.\,4.\,1877 Wien – 16.\,10.\,1961 Baden bei Wien@\textsc{Adam, Robert} (20.\,4.\,1877 Wien – 16.\,10.\,1961 Baden bei Wien), \emph{Schriftsteller, Richter}!Fremde@\strich\emph{Der Fremde}|pwv} überreichte, bei Ihnen anfrage, ob Ihnen von dem
               Schickſal, das ihrer harrt,{ }ſchon etwas bekannt geworden iſt? Ich bin ohne jede
               Nachricht und weiß nicht recht, ob ich wieder im Theater\oindex{Wien@\textbf{Wien}!VII., Neubau@\textbf{VII., Neubau}!Volkstheater@\textbf{Volkstheater}, \emph{Theater}|pwv} vorſprechen{ }ſoll und an wen ich mich am beſten
               wenden{ }ſollte; ich beſorge, mir durch Zudringlichkeit und Zurſchautragen von Ungeduld
               Chancen, die ich etwa hätte, zu verderben, anderſeits aber wieder,{ }ſtilles Zuwarten
               möchte auch nicht das {\pb}richtige Vorgehen{ }ſein. Könnten Sie mir, bitte, hierin einen Rat geben?\pend
           
\pstart
           Mir hilft jetzt über viele Unannehmlichkeiten der deutſchöſterreichiſchen\oindex{Österreich@\textbf{Österreich}|pw} Epoche – Amtsarbeit, Verkühlung, Fett- und
               Fleiſchhunger, kühle Zimmer – die Lektüre eines wundervollen Buches hinweg, das ich
               neulich in der Bibliothek der Juſtizbeamten\orgindex{Privatbibliothek der Wiener Justizbeamten@Privatbibliothek der Wiener Justizbeamten|pw}
               aufſtöberte und das mir bis jetzt vollkommen unbekannt war (obwohl es in den
                     80\textsuperscript{er} Jahren einiges Aufſehen erregt
               haben muß). Es heißt: »Briefe eines Unbekannten\pwindex{Villers, Alexander von 12.\,4.\,1812 Moskau – 16.\,2.\,1880 Neulengbach@\textsc{Villers, Alexander von} (12.\,4.\,1812 Moskau – 16.\,2.\,1880 Neulengbach), \emph{Schriftsteller, Diplomat}!Briefe eines Unbekannten@\strich\emph{Briefe eines Unbekannten}|pw}«
               und wurde von dem Grafen Rudolf \textsc{Hoyos}\pwindex{Hoyos, Rudolf von 9.\,11.\,1821 Horn – 8.\,11.\,1896 Bolków@\textsc{Hoyos, Rudolf von} (9.\,11.\,1821 Horn – 8.\,11.\,1896 Bolków), \emph{Schriftsteller}|pw} bei Gerold\orgindex{Carl Gerold’s Sohn@Carl Gerold’s Sohn|pw} in Wien\oindex{Wien@\textbf{Wien}, \emph{Verwaltungsgebiet}|pw} herausgegeben, 1887 in zweiter Auflage. Der Briefſchreiber
               war ein Herr von \textsc{Villers}\pwindex{Villers, Alexander von 12.\,4.\,1812 Moskau – 16.\,2.\,1880 Neulengbach@\textsc{Villers, Alexander von} (12.\,4.\,1812 Moskau – 16.\,2.\,1880 Neulengbach), \emph{Schriftsteller, Diplomat}|pw}, penſionierter ſächſiſcher\oindex{Sachsen@\textbf{Sachsen}, \emph{Land}|pw} Legationsrat,
               ein Mann von höchſter Kultur. Wie konnte es kommen, daß ich von dieſem Buch nie etwas
               las oder hörte? Es gehört, will mich dünken, nicht nur zu den vornehmſten,{ }ſondern zu
               den geiſtvollſten und liebenswürdigſten Büchern der deutſchen {\pb}Literatur. Ich muß mich zurückhalten,
               Ihnen nicht Stücke auszuſchreiben, um Ihnen davon – falls Sie dieſe Briefe nicht
               ohnehin kennen{ }ſollten – Proben zu geben; aber vielleicht kennen Sie, was ich
               entdeckt oder wiederentdeckt zu haben glaubte, ohnehin und meine Begeiſterung{ }ſcheint
               Ihnen zwar nicht lächerlich – denn ich glaube kaum, daß ein für Literatur
               Empfänglicher dieſen Briefen gegenüber kalt bleiben könnte –, aber doch unnütz. –\pend
           
\pstart
           Zu{ }ſchriftſtelleriſcher Betätigung komme ich jetzt gar nicht; mir iſt, als müßte ich
               alle mir nach viereinhalb Kriegsjahren verbliebene Energie dazu aufbrauchen, nicht
               allzuſehr zu frieren, und als bliebe für’s Denken keine mehr übrig.\pend
           
\pstart
           Mit den ergebenſten Grüßen{\\[\baselineskip]}Ihr{\\[\baselineskip]}\spacefill\mbox{D\textsuperscript{r}RAdam}\pend
           \leftskip=0em{}\selectlanguage{ngerman}\endnumbering\briefempfaengerindex{Schnitzler, Arthur@\textsc{Schnitzler, Arthur}!zzzAdam, Robert@\emph{von Robert Adam}!1918-12-021@{2. 12. 1918}|)be}\mylabel{L02314h}  \newcommand{\dateiname}{L02314}\newcommand{\titel}{Robert Adam an Arthur Schnitzler, 2. 12. 1918}\newcommand{\editorInnen}{Martin Anton Müller und Gerd-Hermann Susen}%% latex-leseansicht-abspann.tex
%% Abspann für die Leseansicht.
%% Der Schalter \ifkorrekturansicht ist bereits durch den Vorspann gesetzt.

%% latex-abspann.tex
%% Gemeinsamer Abspann für Korrekturansicht und Leseansicht.
%% Setzt den Schalter \ifkorrekturansicht voraus (gesetzt in den
%% einbindenden Dateien latex-korrekturansicht-abspann.tex bzw.
%% latex-leseansicht-abspann.tex).
%% ---------------------------------------------------------------

\normalsize

% Das esempio-Environment wird nur in der Leseansicht benötigt
\ifkorrekturansicht\else
\newenvironment{esempio}[3]%
{
    \vspace{1.5ex}
    \rlap{\underline{#1}}
    \par
    \setlength{\parindent}{0cm}
    \nopagebreak
    \leftskip=#2cm
    \rightskip=#3cm
}
{
    \par
}
\fi

\doendnotes{C}
\bigskip
\vfill

\clearpage

\footnotesize

\ifkorrekturansicht
  \lohead{\textsc{register}}
\fi

% theindex-Environment neu definieren ohne reledmac
\makeatletter
\renewenvironment{theindex}{%
  \ifkorrekturansicht
    \section*{\indexname}%
  \else
    \subsubsection*{Index der erwähnten Entitäten}%
  \fi
  \setlength{\parindent}{0pt}%
  \setlength{\parskip}{0pt plus 0.3pt}%
  \let\item\@idxitem
}{%
  \ifkorrekturansicht\clearpage\fi
}
\makeatother

\IfFileExists{\jobname-pw.ind}{\input{\jobname-pw.ind}}{}

% Quellenangabe nur in der Leseansicht
\ifkorrekturansicht\else
% Fallback-Definitionen, falls die .tex-Datei \titel etc. nicht gesetzt hat
\providecommand{\titel}{}
\providecommand{\editorInnen}{}
\providecommand{\dateiname}{\jobname}

\vspace{3cm}

\vfill

\footnotesize
\textsc{Quelle}: \titel. Herausgegeben von {\editorInnen}. In: \emph{Arthur Schnitzler: Briefwechsel mit Autorinnen und Autoren}.
 Digitale Edition, https://schnitzler-briefe.acdh.oeaw.ac.at/{\dateiname}.html (Stand \today)
\fi

\end{document}


