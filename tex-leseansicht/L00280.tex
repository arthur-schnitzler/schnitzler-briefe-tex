%% latex-leseansicht-vorspann.tex
%% Vorspann für die Leseansicht.
%% Lädt die gemeinsame Datei latex-vorspann.tex mit nicht gesetztem Schalter.

\newif\ifkorrekturansicht
\korrekturansichtfalse

\input{../tex-inputs/latex-vorspann}


\section[Arthur Schnitzler an Hermann Bahr, 7. 11. 1893]{L00280 Arthur Schnitzler an Hermann Bahr, 7. 11. 1893}
\nopagebreak\mylabel{L00280v}
\rehead{ }\normalsize\beginnumbering\briefempfaengerindex{Bahr, Hermann@\textsc{Bahr, Hermann}!zzzSchnitzler, Arthur@\emph{von Arthur Schnitzler}!1893-11-071@{7. 11. 1893}|(be}
\toendnotes[C]{\smallbreak\pagebreak[2]}
\correspDesc{Versand  durch Arthur Schnitzler am 7. 11. 1893 in Wien
\newline{}Erhalt  durch Hermann Bahr im Zeitraum [7. 11. 1893
                  – 11. 11. 1893?] in Wien}\toendnotes[C]{\smallbreak}
\Standort{TMW, HS AM 23323 Ba.}
\physDesc{Brief, 1 Blatt, 3 Seiten, 789 Zeichen (Briefpapier mit Trauerrand)
\newline{}Handschrift: schwarze Tinte, deutsche Kurrent
\newline{}Ordnung: Lochung }
\buchAbdrucke{\weitereDrucke{1) \emph{7. 11. 1893.} In: Arthur Schnitzler: \emph{The Letters of Arthur Schnitzler to Hermann Bahr}. Edited, annotated, and with an introduction, by Donald G. Daviau. Chapel Hill: \emph{The University of North Carolina Press} 1978, S. 57–58 (University of North Carolina studies in the Germanic languages
                        and literatures, 89).} \weitereDrucke{2) Hermann Bahr, Arthur Schnitzler: \emph{Briefwechsel, Aufzeichnungen, Dokumente (1891–1931)}. Herausgegeben von Kurt Ifkovits und Martin Anton Müller. Göttingen: \emph{Wallstein} 2018, S. 47.} }\toendnotes[C]{\smallbreak}
\pstart{}{\pb}Lieber
                  Freund,\pend\vspace{0.5em}
\pstart
           hier iſt also etwas, was{ }ſich möglicherweiſe als Eingangsfeuilleton eignet. Ich habe
               ihm vorläufig keinen Namen gegeben – eventuell könnte man das Ding »\label{K_L00280-1v}\edtext{Abendſpaziergang\pwindex{Schnitzler, Arthur 15.\,5.\,1862 Wien – 21.\,10.\,1931 ebd.@\textsc{Schnitzler, Arthur} (15.\,5.\,1862 Wien – 21.\,10.\,1931 ebd.), \emph{Schriftsteller, Mediziner}!Spaziergang@\strich\emph{Spaziergang}|pw}}{\lemma{\textnormal{\emph{Abendspaziergang}}}\Cendnote{\textnormal{Am Vortag hatte Schnitzler den Text
                  vollendet, am 15. 11. 1893 las er ihn Beer-Hofmann\pwindex{Beer-Hofmann, Richard 11.\,7.\,1866 Wien – 26.\,9.\,1945 New York City@\textsc{Beer-Hofmann, Richard} (11.\,7.\,1866 Wien – 26.\,9.\,1945 New York City), \emph{Schriftsteller}|pwk} und Hofmannsthal\pwindex{Hofmannsthal, Hugo von 1.\,2.\,1874 Wien – 15.\,7.\,1929 Rodaun@\textsc{Hofmannsthal, Hugo von} (1.\,2.\,1874 Wien – 15.\,7.\,1929 Rodaun), \emph{Schriftsteller}|pwk} vor,
                     »der viel getadelt wurde«. Am selben Tag korrigierte er ihn
                  noch. Am 6. 12. 1893 erschien der Text als \emph{Spaziergang}\pwindex{Schnitzler, Arthur 15.\,5.\,1862 Wien – 21.\,10.\,1931 ebd.@\textsc{Schnitzler, Arthur} (15.\,5.\,1862 Wien – 21.\,10.\,1931 ebd.), \emph{Schriftsteller, Mediziner}!Spaziergang@\strich\emph{Spaziergang}|pwk}.}}}\label{K_L00280-1}« heißen. Vortheilhaft erſcheint mir, daſs
               in den vier Freunden \label{LL001-2v}Typen\label{LL001-2h} angedeutet{ }ſind, die{ }ſich vielleicht {\pb}weiterhin für die Reihe
               noch irgendwie werden verwenden laſſen. –\pend
           
\pstart
           Ich{ }ſchicke Ihnen da gleich auch eine andre kleine \label{K_L00280-2v}\edtext{Geſchichte}{\lemma{\textnormal{\emph{Geschichte}}}\Cendnote{\textnormal{eventuell \emph{Die Braut}\pwindex{Schnitzler, Arthur 15.\,5.\,1862 Wien – 21.\,10.\,1931 ebd.@\textsc{Schnitzler, Arthur} (15.\,5.\,1862 Wien – 21.\,10.\,1931 ebd.), \emph{Schriftsteller, Mediziner}!Braut@\strich\emph{Die Braut}|pwk}}}}\label{K_L00280-2} mit, die, wenn{ }ſie nicht am Ende zu »frivol« iſt, ganz ohne Praetenſion
               gelegentlich unter den Skizzen gebracht werden könnte.\pend
           
\pstart
           Ich hoffe Ihnen nun aber bald was vernünftiges{ }ſchicken {\pb}zu können. \label{LL001-1v}Schließlich werde ich doch wohl auch das Feuilleton{ }ſchreiben lernen – vorläufig fehlt mir noch manches dazu\label{LL001-1h}.\pend
           
\pstart
           – Mit herzlichen Grüßen{\\[\baselineskip]}Ihr{ }ſehr ergebner{\\[\baselineskip]}\spacefill\mbox{Arthur Schnitzler}\pend
           \leftskip=0em{}
\pstart
           Wien\oindex{Wien@\textbf{Wien}, \emph{Verwaltungsgebiet}|pw}, 7. November 93.\pend
           \selectlanguage{ngerman}\endnumbering\briefempfaengerindex{Bahr, Hermann@\textsc{Bahr, Hermann}!zzzSchnitzler, Arthur@\emph{von Arthur Schnitzler}!1893-11-071@{7. 11. 1893}|)be}\mylabel{L00280h}  \newcommand{\dateiname}{L00280}\newcommand{\titel}{Arthur Schnitzler an Hermann Bahr, 7. 11. 1893}\newcommand{\editorInnen}{Herausgegeben von Martin Anton Müller}%% latex-leseansicht-abspann.tex
%% Abspann für die Leseansicht.
%% Der Schalter \ifkorrekturansicht ist bereits durch den Vorspann gesetzt.

%% latex-abspann.tex
%% Gemeinsamer Abspann für Korrekturansicht und Leseansicht.
%% Setzt den Schalter \ifkorrekturansicht voraus (gesetzt in den
%% einbindenden Dateien latex-korrekturansicht-abspann.tex bzw.
%% latex-leseansicht-abspann.tex).
%% ---------------------------------------------------------------

\normalsize

% Das esempio-Environment wird nur in der Leseansicht benötigt
\ifkorrekturansicht\else
\newenvironment{esempio}[3]%
{
    \vspace{1.5ex}
    \rlap{\underline{#1}}
    \par
    \setlength{\parindent}{0cm}
    \nopagebreak
    \leftskip=#2cm
    \rightskip=#3cm
}
{
    \par
}
\fi

\doendnotes{C}
\bigskip
\vfill

\clearpage

\footnotesize

\ifkorrekturansicht
  \lohead{\textsc{register}}
\fi

% theindex-Environment neu definieren ohne reledmac
\makeatletter
\renewenvironment{theindex}{%
  \ifkorrekturansicht
    \section*{\indexname}%
  \else
    \subsubsection*{Index der erwähnten Entitäten}%
  \fi
  \setlength{\parindent}{0pt}%
  \setlength{\parskip}{0pt plus 0.3pt}%
  \let\item\@idxitem
}{%
  \ifkorrekturansicht\clearpage\fi
}
\makeatother

\IfFileExists{\jobname-pw.ind}{\input{\jobname-pw.ind}}{}

% Quellenangabe nur in der Leseansicht
\ifkorrekturansicht\else
% Fallback-Definitionen, falls die .tex-Datei \titel etc. nicht gesetzt hat
\providecommand{\titel}{}
\providecommand{\editorInnen}{}
\providecommand{\dateiname}{\jobname}

\vspace{3cm}

\vfill

\footnotesize
\textsc{Quelle}: \titel. Herausgegeben von {\editorInnen}. In: \emph{Arthur Schnitzler: Briefwechsel mit Autorinnen und Autoren}.
 Digitale Edition, https://schnitzler-briefe.acdh.oeaw.ac.at/{\dateiname}.html (Stand \today)
\fi

\end{document}


