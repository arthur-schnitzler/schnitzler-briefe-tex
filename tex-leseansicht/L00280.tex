%% latex-korrekturansicht-vorspann.tex
%% Vorspann für die Korrekturansicht.
%% Lädt die gemeinsame Datei latex-vorspann.tex mit gesetztem Schalter.

\newif\ifkorrekturansicht
\korrekturansichttrue

\input{../tex-inputs/latex-vorspann}


\section[Arthur Schnitzler an Hermann Bahr, 7. 11. 1893]{L00280 Arthur Schnitzler an Hermann Bahr, 7. 11. 1893}
\nopagebreak\mylabel{L00280v}
\rehead{ }\normalsize\beginnumbering\briefempfaengerindex{Bahr, Hermann@\textsc{Bahr, Hermann}!zzzSchnitzler, Arthur@\emph{von Arthur Schnitzler}!1893-11-071@{7. 11. 1893}|(be}
\toendnotes[C]{\smallbreak\pagebreak[2]}\Standort{TMW, HS AM 23323 Ba.}
\physDesc{Brief, 1 Blatt, 3 Seiten, 789 Zeichen (Briefpapier mit Trauerrand)
\newline{}Handschrift: schwarze Tinte, deutsche Kurrent
\newline{}Ordnung: Lochung }
\buchAbdrucke{\weitereDrucke{1) Arthur Schnitzler: \emph{The Letters of Arthur Schnitzler to Hermann Bahr}. Chapel Hill: \emph{The University of North Carolina Press} 1978, S. 57–58.} \weitereDrucke{2) Hermann Bahr, Arthur Schnitzler: \emph{Briefwechsel, Aufzeichnungen, Dokumente (1891–1931)}. Göttingen: \emph{Wallstein} 2018, S. 47.} }\toendnotes[C]{\smallbreak}
\pstart{}{\pb}Lieber
                  Freund,\pend\vspace{0.5em}
\pstart
           hier iſt also etwas, was ſich möglicherweiſe als Eingangsfeuilleton eignet. Ich habe
               ihm vorläufig keinen Namen gegeben – eventuell könnte man das Ding »\label{K_L00280-1v}\edtext{Abendſpaziergang\pwindex{Spaziergang@\emph{Spaziergang}|pw}}{\lemma{\textnormal{\emph{Abendſpaziergang}}}\Cendnote{\textnormal{Am Vortag hatte Schnitzler den Text
                  vollendet, am 15. 11. 1893 las er ihn Beer-Hofmann\pwindex{Beer-Hofmann, Richard 1866-07-11 – 1945-09-26@\textsc{Beer-Hofmann, Richard} (1866-07-11 – 1945-09-26), \emph{Schriftsteller/Schriftstellerin}|pwk} und Hofmannsthal\pwindex{Hofmannsthal, Hugo von 1874-02-01 – 1929-07-15@\textsc{Hofmannsthal, Hugo von} (1874-02-01 – 1929-07-15), \emph{Schriftsteller/Schriftstellerin}|pwk} vor,
                     »der viel getadelt wurde«. Am selben Tag korrigierte er ihn
                  noch. Am 6. 12. 1893 erschien der Text als \emph{Spaziergang}\pwindex{Spaziergang@\emph{Spaziergang}|pwk}.}}}\label{K_L00280-1}« heißen. Vortheilhaft erſcheint mir, daſs
               in den vier Freunden \label{LL001-2v}Typen\label{LL001-2h} angedeutet
               ſind, die ſich vielleicht {\pb}weiterhin für die Reihe
               noch irgendwie werden verwenden laſſen. –\pend
           
\pstart
           Ich ſchicke Ihnen da gleich auch eine andre kleine \label{K_L00280-2v}\edtext{Geſchichte}{\lemma{\textnormal{\emph{Geſchichte}}}\Cendnote{\textnormal{eventuell \emph{Die Braut}\pwindex{Braut@\emph{Die Braut}|pwk}}}}\label{K_L00280-2} mit, die, wenn ſie nicht am Ende zu »frivol« iſt, ganz ohne Praetenſion
               gelegentlich unter den Skizzen gebracht werden könnte.\pend
           
\pstart
           Ich hoffe Ihnen nun aber bald was vernünftiges ſchicken {\pb}zu können. \label{LL001-1v}Schließlich werde ich doch wohl auch das Feuilleton
                  ſchreiben lernen – vorläufig fehlt mir noch manches dazu\label{LL001-1h}.\pend
           
\pstart
           – Mit herzlichen Grüßen{\\[\baselineskip]}Ihr ſehr ergebner{\\[\baselineskip]}\spacefill\mbox{Arthur Schnitzler}\pend
           \leftskip=0em{}
\pstart
           Wien\oindex{Wien@\textbf{Wien}, \emph{A.ADM2}|pw}, 7. November
                     93.\pend
           \selectlanguage{ngerman}\endnumbering\briefempfaengerindex{Bahr, Hermann@\textsc{Bahr, Hermann}!zzzSchnitzler, Arthur@\emph{von Arthur Schnitzler}!1893-11-071@{7. 11. 1893}|)be}\mylabel{L00280h}  \normalsize

\doendnotes{C}
\bigskip
\vfill

\clearpage

\footnotesize

\lohead{\textsc{register}}

% Definiere theindex-Environment komplett neu ohne reledmac
\makeatletter
\renewenvironment{theindex}{%
  \section*{\indexname}%
  \setlength{\parindent}{0pt}%
  \setlength{\parskip}{0pt plus 0.3pt}%
  \let\item\@idxitem
}{%
  \clearpage
}
\makeatother

\IfFileExists{\jobname-pw.ind}{\input{\jobname-pw.ind}}{}

\end{document}

      