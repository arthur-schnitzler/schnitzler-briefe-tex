%% latex-korrekturansicht-vorspann.tex
%% Vorspann für die Korrekturansicht.
%% Lädt die gemeinsame Datei latex-vorspann.tex mit gesetztem Schalter.

\newif\ifkorrekturansicht
\korrekturansichttrue

\input{../tex-inputs/latex-vorspann}


\section[Hugo von Hofmannsthal an Arthur Schnitzler, 15. 7. {[}1899{]}]{L00944 Hugo von Hofmannsthal an Arthur Schnitzler, 15. 7. {[}1899{]}}
\nopagebreak\mylabel{L00944v}
\rehead{ }\normalsize\beginnumbering\briefempfaengerindex{Schnitzler, Arthur@\textsc{Schnitzler, Arthur}!zzzHofmannsthal, Hugo von@\emph{von Hugo von Hofmannsthal}!1899-07-152@{15. 7. {[}1899{]}}|(be}
\toendnotes[C]{\smallbreak\pagebreak[2]}\Standort{CUL, Schnitzler, B 43.}
\physDesc{Brief, 1 Blatt, 4 Seiten, 902 Zeichen
\newline{}Handschrift: schwarze Tinte, deutsche Kurrent
\newline{}Schnitzler: mit Bleistift die Jahreszahl ergänzt: »99« 
\newline{}Ordnung: 1) mit Bleistift von unbekannter Hand nummeriert:
                                    »151«  2) mit Bleistift von unbekannter Hand nummeriert: »\strikeout{155}«}
\buchAbdrucke{\weitereDrucke{1) Hugo von Hofmannsthal: \emph{Briefe. 1890–1901}. Berlin: \emph{S. Fischer} 1935, S. 287.} \weitereDrucke{2) Hugo von Hofmannsthal, Arthur Schnitzler: \emph{Briefwechsel}. Frankfurt am Main: \emph{S. Fischer} 1964, S. 125–126.} }\toendnotes[C]{\smallbreak}
\pstart
           \raggedleft{}{\pb}15 VII.\pend
           \vspace{0.5em}
\pstart
           lieber, bitte ſehen Sie keinen Eigenſinn darin, wenn ich Sie
               nochmals bitte nicht darauf zu rechnen, daſs ich unſere Radtour \introOben{}(auf die ich mich ſehr freue)\introOben{} vor dem 1\textsuperscript{ten}{ }Sept. anzutreten im Stande ſein werde.
               Viel eher wird es mir möglich ſein im Laufe des Auguſt{ }ſonſt mit Ihnen zuſa{\geminationm}en zu ſein aber {\pb}an einem Ort,
               ſodaſs ich weiterarbeiten kann. Ich hoffe hier ungefähr die beiden erſten Acte eines
               neuen Stückes\pwindex{Bergwerk zu Falun@\emph{Das Bergwerk zu Falun}|pwv} in Verſen fertig
               zu bringen, dann – etwa in Salzburg\oindex{Salzburg@\textbf{Salzburg}, \emph{A.ADM2}|pw}{ }1–10 Auguſt – noch einen Act\pwindex{Bergwerk zu Falun@\emph{Das Bergwerk zu Falun}|pwv}. Die beiden letzten laſſen ſich \uline{vielleicht} verſchieben, kaum aber {\pb}werden ſie eine ſo radicale
               Unterbrechung der Sti{\geminationm}ung vertragen wie eine Reiſe.\pend
           
\pstart
           Jedenfalls bleiben wir in Verbindung. \uuline{Bitte} fahren
               Sie zu Richard\pwindex{Beer-Hofmann, Richard 1866-07-11 – 1945-09-26@\textsc{Beer-Hofmann, Richard} (1866-07-11 – 1945-09-26), \emph{Schriftsteller/Schriftstellerin}|pw}, nicht nur auf Stunden, sondern
               für mehrere Tage; bringen Sie bitte ſeinem Zuſtand denſelben Ernſt aber mehr {\pb}Vernunft entgegen als er ſelber.
               Ich werde auch im Auguſt hinzuko{\geminationm}en
               trachten.\pend
           
\pstart
           Bitte ſchreiben!\pend
           
\pstart
           Ihr{\\[\baselineskip]}\spacefill\mbox{Hugo.}\pend
           \leftskip=0em{}\selectlanguage{ngerman}\endnumbering\briefempfaengerindex{Schnitzler, Arthur@\textsc{Schnitzler, Arthur}!zzzHofmannsthal, Hugo von@\emph{von Hugo von Hofmannsthal}!1899-07-152@{15. 7. {[}1899{]}}|)be}\mylabel{L00944h}  \normalsize

\doendnotes{C}
\bigskip
\vfill

\clearpage

\footnotesize

\lohead{\textsc{register}}

% Definiere theindex-Environment komplett neu ohne reledmac
\makeatletter
\renewenvironment{theindex}{%
  \section*{\indexname}%
  \setlength{\parindent}{0pt}%
  \setlength{\parskip}{0pt plus 0.3pt}%
  \let\item\@idxitem
}{%
  \clearpage
}
\makeatother

\IfFileExists{\jobname-pw.ind}{\input{\jobname-pw.ind}}{}

\end{document}

      