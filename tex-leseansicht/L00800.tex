%% latex-leseansicht-vorspann.tex
%% Vorspann für die Leseansicht.
%% Lädt die gemeinsame Datei latex-vorspann.tex mit nicht gesetztem Schalter.

\newif\ifkorrekturansicht
\korrekturansichtfalse

\input{../tex-inputs/latex-vorspann}


               \section[Hugo von Hofmannsthal an Arthur Schnitzler, {[}3.? 6. 1898{]}]{ Hugo von Hofmannsthal an Arthur Schnitzler, {[}3.? 6. 1898{]}}\nopagebreak\mylabel{v}\rehead{ }\begin{ledgroupsized}[t]{13cm}\normalsize\beginnumbering\briefempfaengerindex{Schnitzler, Arthur@\textsc{Schnitzler, Arthur}!zzzHofmannsthal, Hugo von@\emph{von Hugo von Hofmannsthal}!1898-06-031@{{[}3.? 6. 1898{]}}|(be} \toendnotes[C]{\smallbreak\pagebreak[2]} \Standort{CUL, Schnitzler, B 43b/1.}
\physDesc{Brief, 1 Blatt, 3 Seiten
\newline{}Handschrift: schwarze Tinte, deutsche Kurrent
\newline{}Schnitzler: mit Bleistift datiert: »Mai? 98« \newline{}Ordnung: mit Bleistift von unbekannter Hand nummeriert:
                                    »113« }\buchAbdrucke{\weitereDrucke{Hugo von Hofmannsthal, Arthur Schnitzler: \emph{Briefwechsel}. Hg. Therese Nickl und Heinrich Schnitzler. Frankfurt am Main: \emph{S. Fischer} 1964, S. 101.} }\toendnotes[C]{\smallbreak}\pstart
           \raggedleft{}{\pb}Hinterbrühl\oindex{Hinterbruehl@\textbf{Hinterbrühl}|pw}, Freitag.\pend
           \pstart{}mein lieber Arthur\pend\pstart
           \label{K_L00800_1v}\edtext{Dienstag}{\lemma{\textnormal{\emph{Dienstag}}}\Cendnote{\textnormal{Durch die privaten Aufzeichungen Hofmannsthal\pwindex{Hofmannsthal, Hugo von 01.02.1874 – 15.07.1929@\textsc{Hofmannsthal, Hugo von} (01.02.1874 – 15.07.1929), \emph{Schriftsteller}|pwk}s (S. 397–398) ergibt
                  sich für die Maturalernzeit nur ein Freitag in Hinterbrühl\oindex{Hinterbruehl@\textbf{Hinterbrühl}|pwk}, an dem er am Dienstag und Mittwoch zuvor in Wien\oindex{Wien@\textbf{Wien}|pwk} war, nämlich der 3. 6. 1898.}}}\label{K_L00800_1h} war ich
               im Café bin aber um ½ 11{ }ſehr müd geworden und Mittwoch war ich
               überhaupt von der Lernerei ſehr müd. Auch davon iſt man ein biſſel niedergeſchlagen,
               daſs es gar {\pb}nicht So{\geminationm}er werden kann und ſo wenig Sonne iſt.\pend
           \pstart
           Bitte gehen Sie nur gleich fort nach Kärnten\oindex{Kaernten@\textbf{Kärnten}|pw}{ }ſobald es ſchön iſt, es giebt doch Möglichkeiten,
               ohne Betrug, einer ſo tiefen Verſtimmung entgegenzuarbeiten.\pend
           \pstart
           {\pb}Aber bitte laſſen Sie mich nicht
               ganz ohne Verſtändigung, es freut einen i{\geminationm}er ſo die
               Menſchen die man gern hat, in irgend einer Landſchaft zu denken.\pend
           \pstart
           Von Herzen\hspace*{2em}Ihr{\\[\baselineskip]}\spacefill\mbox{Hugo}\pend
           \leftskip=0em{}          \endnumbering\briefempfaengerindex{Schnitzler, Arthur@\textsc{Schnitzler, Arthur}!zzzHofmannsthal, Hugo von@\emph{von Hugo von Hofmannsthal}!1898-06-031@{{[}3.? 6. 1898{]}}|)be}\mylabel{h}\end{ledgroupsized}  \newcommand{\dateiname}{L00800}\newcommand{\titel}{Hugo von Hofmannsthal an Arthur Schnitzler, [3.? 6. 1898]}\newcommand{\editorInnen}{Martin Anton Müller und Gerd-Hermann Susen}
            \footnotesize
\begin{ledgroupsized}[t]{11.5cm}
\doendnotes{C}
\end{ledgroupsized}
         %% latex-leseansicht-abspann.tex
%% Abspann für die Leseansicht.
%% Der Schalter \ifkorrekturansicht ist bereits durch den Vorspann gesetzt.

%% latex-abspann.tex
%% Gemeinsamer Abspann für Korrekturansicht und Leseansicht.
%% Setzt den Schalter \ifkorrekturansicht voraus (gesetzt in den
%% einbindenden Dateien latex-korrekturansicht-abspann.tex bzw.
%% latex-leseansicht-abspann.tex).
%% ---------------------------------------------------------------

\normalsize

% Das esempio-Environment wird nur in der Leseansicht benötigt
\ifkorrekturansicht\else
\newenvironment{esempio}[3]%
{
    \vspace{1.5ex}
    \rlap{\underline{#1}}
    \par
    \setlength{\parindent}{0cm}
    \nopagebreak
    \leftskip=#2cm
    \rightskip=#3cm
}
{
    \par
}
\fi

\doendnotes{C}
\bigskip
\vfill

\clearpage

\footnotesize

\ifkorrekturansicht
  \lohead{\textsc{register}}
\fi

% theindex-Environment neu definieren ohne reledmac
\makeatletter
\renewenvironment{theindex}{%
  \ifkorrekturansicht
    \section*{\indexname}%
  \else
    \subsubsection*{Index der erwähnten Entitäten}%
  \fi
  \setlength{\parindent}{0pt}%
  \setlength{\parskip}{0pt plus 0.3pt}%
  \let\item\@idxitem
}{%
  \ifkorrekturansicht\clearpage\fi
}
\makeatother

\IfFileExists{\jobname-pw.ind}{\input{\jobname-pw.ind}}{}

% Quellenangabe nur in der Leseansicht
\ifkorrekturansicht\else
% Fallback-Definitionen, falls die .tex-Datei \titel etc. nicht gesetzt hat
\providecommand{\titel}{}
\providecommand{\editorInnen}{}
\providecommand{\dateiname}{\jobname}

\vspace{3cm}

\vfill

\footnotesize
\textsc{Quelle}: \titel. Herausgegeben von {\editorInnen}. In: \emph{Arthur Schnitzler: Briefwechsel mit Autorinnen und Autoren}.
 Digitale Edition, https://schnitzler-briefe.acdh.oeaw.ac.at/{\dateiname}.html (Stand \today)
\fi

\end{document}


      