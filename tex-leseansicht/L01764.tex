%% latex-korrekturansicht-vorspann.tex
%% Vorspann für die Korrekturansicht.
%% Lädt die gemeinsame Datei latex-vorspann.tex mit gesetztem Schalter.

\newif\ifkorrekturansicht
\korrekturansichttrue

\input{../tex-inputs/latex-vorspann}


\section[Hugo von Hofmannsthal an Arthur Schnitzler, 7. 4. 1908]{L01764 Hugo von Hofmannsthal an Arthur Schnitzler, 7. 4. 1908}
\nopagebreak\mylabel{L01764v}
\rehead{ }\normalsize\beginnumbering\briefempfaengerindex{Schnitzler, Arthur@\textsc{Schnitzler, Arthur}!zzzHofmannsthal, Hugo von@\emph{von Hugo von Hofmannsthal}!1908-04-071@{7. 4. 1908}|(be}
\toendnotes[C]{\smallbreak\pagebreak[2]}\Standort{CUL, Schnitzler, B 43.}
\physDesc{Postkarte, 299 Zeichen
\newline{}Handschrift: 1) schwarze Tinte, deutsche Kurrent\hspace{1em}2) schwarze Tinte, lateinische Kurrent (\noindent{}Adresse)\hspace{1em}
\newline{}Versand: 1) Rohrpost  2) Stempel: »\nobreak{}\oindex{I., Innere Stadt@\textbf{I., Innere Stadt}, \emph{A.ADM3}|pwk}1/1 Wien 15, 7 IV 08, 5\textsuperscript{50}\nobreak{}«.  3) Stempel: »\nobreak{}\oindex{XVIII., Waehring@\textbf{XVIII., Währing}, \emph{A.ADM3}|pwk}18/1 Wien 111, 7 IV 08, 6\textsuperscript{50}\nobreak{}«. 
\newline{}Schnitzler: mit Bleistift datiert: »7/4 908« und beschriftet: »\textsc{Hugo H.}« 
\newline{}Ordnung: 1) mit Bleistift von unbekannter Hand nummeriert: »\strikeout{292}«  2) mit Bleistift von unbekannter Hand nummeriert:
                                    »296«}
\buchAbdrucke{\weitereDrucke{Hugo von Hofmannsthal, Arthur Schnitzler: \emph{Briefwechsel}. Frankfurt am Main: \emph{S. Fischer} 1964, S. 237.} }\toendnotes[C]{\smallbreak}\pstart{}{\pb}Herrn\pend{}\pstart{}D\textsuperscript{r} Arthur Schnitzler\pend{}\pstart{}Wien\oindex{Wien@\textbf{Wien}, \emph{A.ADM2}|pw}\pend{}\pstart{}XVIII Spöttelgasse 7\oindex{Edmund-Weiss-Gasse 7@\textbf{Edmund-Weiß-Gasse 7}, \emph{Wohngebäude (K.WHS)}|pw}\pend{}\pstart{}Pneumatis\damage{ch}\pend{}{\bigskip}\vspace{1em}
\pstart
           \raggedleft{}{\pb}Dinstg\pend
           \vspace{0.5em}
\pstart
           Ich bin nur mehr paar Tage hier gehe Montag nach Griechenland\oindex{Griechenland@\textbf{Griechenland}, \emph{A.PCLI}|pw} deshalb wir möchten \label{K_L01764-1v}\edtext{morgigen (=Mittwoch)
               Abend}{\lemma{\textnormal{\emph{morgigen … Abend}}}\Cendnote{\textnormal{Vgl. A. S.: \emph{Tagebuch}, 8. 4. 1908.
               }}}\label{K_L01764-1} bei Euch ſein. \uuline{Hoffentlich
                  gehts}. Wenn nicht, ſo gienge noch Freitag abends oder Do{\geminationn}erstg mittg. Erbitten ſofort Depeſche \uline{Rodaun}\oindex{Rodaun@\textbf{Rodaun}, \emph{A.ADM4}|pw}.\pend
           \pstart Ihr \spacefill\mbox{Hugo}\pend{}\selectlanguage{ngerman}\endnumbering\briefempfaengerindex{Schnitzler, Arthur@\textsc{Schnitzler, Arthur}!zzzHofmannsthal, Hugo von@\emph{von Hugo von Hofmannsthal}!1908-04-071@{7. 4. 1908}|)be}\mylabel{L01764h}  \normalsize

\doendnotes{C}
\bigskip
\vfill

\clearpage

\footnotesize

\lohead{\textsc{register}}

% Definiere theindex-Environment komplett neu ohne reledmac
\makeatletter
\renewenvironment{theindex}{%
  \section*{\indexname}%
  \setlength{\parindent}{0pt}%
  \setlength{\parskip}{0pt plus 0.3pt}%
  \let\item\@idxitem
}{%
  \clearpage
}
\makeatother

\IfFileExists{\jobname-pw.ind}{\input{\jobname-pw.ind}}{}

\end{document}

      