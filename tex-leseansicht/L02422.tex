%% latex-leseansicht-vorspann.tex
%% Vorspann für die Leseansicht.
%% Lädt die gemeinsame Datei latex-vorspann.tex mit nicht gesetztem Schalter.

\newif\ifkorrekturansicht
\korrekturansichtfalse

\input{../tex-inputs/latex-vorspann}


\section[Georg Brandes an Arthur Schnitzler, 10. 12. 1924]{L02422 Georg Brandes an Arthur Schnitzler, 10. 12. 1924}
\nopagebreak\mylabel{L02422v}
\rehead{ }\normalsize\beginnumbering\briefempfaengerindex{Schnitzler, Arthur@\textsc{Schnitzler, Arthur}!zzzBrandes, Georg@\emph{von Georg Brandes}!1924-12-101@{10. 12. 1924}|(be}
\toendnotes[C]{\smallbreak\pagebreak[2]}
\correspDesc{Versand  durch Georg Brandes am 10. 12. 1924 in Kopenhagen
\newline{}Erhalt  durch Arthur Schnitzler im Zeitraum [11. 12. 1924 – 15. 12. 1924?] in Wien}\toendnotes[C]{\smallbreak}
\Standort{CUL, Schnitzler, B 17.}
\physDesc{Brief, 1 Blatt, 4 Seiten, 2758 Zeichen
\newline{}Handschrift: schwarze Tinte, lateinische Kurrent
\newline{}Schnitzler: mit rotem Buntstift mehrere Unterstreichungen 
\newline{}Ordnung: mit Bleistift von unbekannter Hand nummeriert:
                                    »55« }
\buchAbdrucke{\weitereDrucke{Georg Brandes, Arthur Schnitzler: \emph{Ein Briefwechsel}. Herausgegeben von Kurt Bergel. Bern: \emph{Francke} 1956, S. 140–141.} }\toendnotes[C]{\smallbreak}
\pstart
           \raggedleft{}{\pb}Kopenhagen\oindex{Kopenhagen@\textbf{Kopenhagen}, \emph{Hauptstadt}|pw}{ }10 December 24\pend
           \vspace{0.5em}
\pstart
           Mein liebster Schnitzler\hspace*{3.5em}Viel Arbeit und lang dauernde wenn auch nicht
               schwere Krankheit, die noch nicht vorüber ist, haben mich verhindert, Ihnen in Dank
               mein Herz auszuschütten. Irgend jemand, der von Ihnen kam oder auf Sie sich berief,
               war neulich bei mir. Wie er hiess, habe ich vergessen.\pend
           
\pstart
           Ich habe Ihnen für zwei Bücher\pwindex{Schnitzler, Arthur 15.\,5.\,1862 Wien – 21.\,10.\,1931 ebd.@\textsc{Schnitzler, Arthur} (15.\,5.\,1862 Wien – 21.\,10.\,1931 ebd.), \emph{Schriftsteller, Mediziner}!Fräulein Else@\strich\emph{Fräulein Else}|pwv}\pwindex{Schnitzler, Arthur 15.\,5.\,1862 Wien – 21.\,10.\,1931 ebd.@\textsc{Schnitzler, Arthur} (15.\,5.\,1862 Wien – 21.\,10.\,1931 ebd.), \emph{Schriftsteller, Mediziner}!Komödie der Verführung. In drei Akten@\strich\emph{Komödie der Verführung. In drei Akten}|pwv} zu danken. Besonders das erstere die Komoedie der Verführung\pwindex{Schnitzler, Arthur 15.\,5.\,1862 Wien – 21.\,10.\,1931 ebd.@\textsc{Schnitzler, Arthur} (15.\,5.\,1862 Wien – 21.\,10.\,1931 ebd.), \emph{Schriftsteller, Mediziner}!Komödie der Verführung. In drei Akten@\strich\emph{Komödie der Verführung. In drei Akten}|pw} gibt viel zu denken über den Reichtum und die Tiefe
               Ihrer Erfahrungen, vielleicht noch mehr über die Fülle und Geschmeidigkeit Ihrer
               Erfindungskraft, die ich am meisten bewundere, weil sie mir völlig fehlt. Man
               bewundert {\pb}wol immer am meisten
               Fähigkeiten, die uns verweigert sind.\pend
           
\pstart
           Ich habe mit Ueberraschung gesehen dass Ihre paar kurzen Aufenthalte in unserem
               kleinen langweiligen Land\oindex{Dänemark@\textbf{Dänemark}|pwv}
               Ihre Phantasie in Bewegung gesetzt hat, und dass sogar die Nordküste von Seeland\oindex{Seeland@\textbf{Seeland}, \emph{Insel}|pw}\oindex{Gilleleje@\textbf{Gilleleje}|pwv} unter Ihren Händen einen Zauberschimmer erhalten hat.\pend
           
\pstart
           Sie sind ein grosser Menschenkenner, besonders ein Frauenkenner wie wenige. Meine
               Erfahrungen stimmen nicht immer mit den Ihrigen überein. Aber der Menschenschlag war
               verschieden, ich habe meistens Skandinavinnen\oindex{Skandinavien@\textbf{Skandinavien}|pw} und
                  Russinnen\oindex{Russland@\textbf{Russland}|pw} gekannt, nie Oesterreicherinnen\oindex{Österreich@\textbf{Österreich}|pw}. Die wenigen dieser Nation, die ich getroffen
               habe, waren sehr prosaisch; alle Ihre Frauen haben eine poetische Aureole.\pend
           
\pstart
           Das andere Buch\pwindex{Schnitzler, Arthur 15.\,5.\,1862 Wien – 21.\,10.\,1931 ebd.@\textsc{Schnitzler, Arthur} (15.\,5.\,1862 Wien – 21.\,10.\,1931 ebd.), \emph{Schriftsteller, Mediziner}!Fräulein Else@\strich\emph{Fräulein Else}|pwv} dessen
               erzählende Form an Ihr Meisterwerk über den {\pb}Lieutenant Gustel\pwindex{Schnitzler, Arthur 15.\,5.\,1862 Wien – 21.\,10.\,1931 ebd.@\textsc{Schnitzler, Arthur} (15.\,5.\,1862 Wien – 21.\,10.\,1931 ebd.), \emph{Schriftsteller, Mediziner}!Lieutenant Gustl. Novelle@\strich\emph{Lieutenant Gustl. Novelle}|pw} erinnert, ist ganz einfach
               aufgebaut, durch traurige Wahrheit \uline{ergreifend}. Sie
               haben den tragischen Ausgang gewollt, haben dem armen Mädchen die Auswege versperrt.
               Am feinsten scheint mir in der Erzählung die Lebenslust, die das junge Mädchen\pwindex{Schnitzler, Arthur 15.\,5.\,1862 Wien – 21.\,10.\,1931 ebd.@\textsc{Schnitzler, Arthur} (15.\,5.\,1862 Wien – 21.\,10.\,1931 ebd.), \emph{Schriftsteller, Mediziner}!Fräulein Else@\strich\emph{Fräulein Else}|pwv} an den Vetter\pwindex{Schnitzler, Arthur 15.\,5.\,1862 Wien – 21.\,10.\,1931 ebd.@\textsc{Schnitzler, Arthur} (15.\,5.\,1862 Wien – 21.\,10.\,1931 ebd.), \emph{Schriftsteller, Mediziner}!Fräulein Else@\strich\emph{Fräulein Else}|pwv} und an den Fred\pwindex{Schnitzler, Arthur 15.\,5.\,1862 Wien – 21.\,10.\,1931 ebd.@\textsc{Schnitzler, Arthur} (15.\,5.\,1862 Wien – 21.\,10.\,1931 ebd.), \emph{Schriftsteller, Mediziner}!Fräulein Else@\strich\emph{Fräulein Else}|pwv} zieht. Warum sind Sie so hart gewesen,
               sie sterben zu lassen! – –\pend
           
\pstart
           Sie werden bemerkt haben, dass die Jahre zwischen 80 und 90 nicht die Blüthezeit der
               Weiber ist. Sie ist ja leider auch nicht die der Männer, wenn man sich auch gern
               Illusionen macht.\pend
           
\pstart
           Ich habe ein paar Bücher\pwindex{Brandes, Georg 4.\,2.\,1842 Kopenhagen – 19.\,2.\,1927 ebd.@\textsc{Brandes, Georg} (4.\,2.\,1842 Kopenhagen – 19.\,2.\,1927 ebd.)!Hertuginden af Dino og Fyrsten af Talleyrand@\strich\emph{Hertuginden af Dino og Fyrsten af Talleyrand}|pwv}\pwindex{Brandes, Georg 4.\,2.\,1842 Kopenhagen – 19.\,2.\,1927 ebd.@\textsc{Brandes, Georg} (4.\,2.\,1842 Kopenhagen – 19.\,2.\,1927 ebd.)!Uimodstaaelige: (Attende aarhundrede, Frankrig)@\strich\emph{Uimodstaaelige: (Attende aarhundrede, Frankrig)}|pwv} über das 18. Jahrhundert in Frankreich\oindex{Frankreich@\textbf{Frankreich}|pw} herausgegeben über \label{K_L02422-1v}\edtext{Talleyrand\pwindex{Talleyrand-Périgord, Charles Maurice de 13.\,2.\,1754 Paris – 17.\,5.\,1838 ebd.@\textsc{Talleyrand-Périgord, Charles Maurice de} (13.\,2.\,1754 Paris – 17.\,5.\,1838 ebd.), \emph{Politiker, Bischof}|pw}}{\lemma{\textnormal{\emph{Talleyrand}}}\Cendnote{\textnormal{Georg Brandes\pwindex{Brandes, Georg 4.\,2.\,1842 Kopenhagen – 19.\,2.\,1927 ebd.@\textsc{Brandes, Georg} (4.\,2.\,1842 Kopenhagen – 19.\,2.\,1927 ebd.)|pwk}: \emph{Hertuginden af Dino og Fyrsten af Talleyrand}\pwindex{Brandes, Georg 4.\,2.\,1842 Kopenhagen – 19.\,2.\,1927 ebd.@\textsc{Brandes, Georg} (4.\,2.\,1842 Kopenhagen – 19.\,2.\,1927 ebd.)!Hertuginden af Dino og Fyrsten af Talleyrand@\strich\emph{Hertuginden af Dino og Fyrsten af Talleyrand}|pwk}. Kopenhagen:
                        \emph{Gyldendalske Boghandel, Nordisk forlag}{ }1923.}}}\label{K_L02422-1}, über \label{K_L02422-2v}\edtext{Lauzun\pwindex{Biron, Armand-Louis de Gontaut de 13.\,4.\,1747 Paris – 31.\,12.\,1793 ebd.@\textsc{Biron, Armand-Louis de Gontaut de} (13.\,4.\,1747 Paris – 31.\,12.\,1793 ebd.), \emph{Militär}|pw}}{\lemma{\textnormal{\emph{Lauzun}}}\Cendnote{\textnormal{Georg Brandes\pwindex{Brandes, Georg 4.\,2.\,1842 Kopenhagen – 19.\,2.\,1927 ebd.@\textsc{Brandes, Georg} (4.\,2.\,1842 Kopenhagen – 19.\,2.\,1927 ebd.)|pwk}: \emph{Uimodstaaelige. Attende aarhundrede. Frankrig}\pwindex{Brandes, Georg 4.\,2.\,1842 Kopenhagen – 19.\,2.\,1927 ebd.@\textsc{Brandes, Georg} (4.\,2.\,1842 Kopenhagen – 19.\,2.\,1927 ebd.)!Uimodstaaelige: (Attende aarhundrede, Frankrig)@\strich\emph{Uimodstaaelige: (Attende aarhundrede, Frankrig)}|pwk}. Kopenhagen:
                        \emph{Gyldendalske Boghandel, Nordisk forlag}{ }1924.}}}\label{K_L02422-2} etc. aber ich habe bisher die Uebersetzung verhindert da die Form
               noch nicht endgültig ist.\pend
           
\pstart
           {\pb}In der letzten Zeit habe ich ein
                  Buch\pwindex{Brandes, Georg 4.\,2.\,1842 Kopenhagen – 19.\,2.\,1927 ebd.@\textsc{Brandes, Georg} (4.\,2.\,1842 Kopenhagen – 19.\,2.\,1927 ebd.)!Urkristendom@\strich\emph{Urkristendom}|pwv} auf dem Stapel, \strikeout{von} das beweisen will dass das Leben Jesu\pwindex{Jesus 7–4 v.\,u.\,Z. Nazareth – 30/31 Jerusalem@\textsc{Jesus} (7–4 v.\,u.\,Z. Nazareth – 30/31 Jerusalem), \emph{Wanderprediger}|pw} (ungefähr wie das Leben Wilhelm Tells) nur Sage ist. Ich
               habe ein paar Kapitel schon veröffentlicht und werde bald damit zu Ende sein, erwarte
               nur Rückkehr der Gesundheit. Es wird leider viel Geheul verursachen.\pend
           
\pstart
           Dieser Brief ist ein sehr schwacher Ausdruck meiner freundschaftlichen Gefühle. Mit
               den Jahren blieben wenige zurück, denen man sich geistig verwandt fühlt und von denen
               man etwas lernt. Sie sind einer von diesen ganz wenigen für mich.\pend
           
\pstart
           Jemand sagte mir, ein Buch\pwindex{Brandes, Georg 4.\,2.\,1842 Kopenhagen – 19.\,2.\,1927 ebd.@\textsc{Brandes, Georg} (4.\,2.\,1842 Kopenhagen – 19.\,2.\,1927 ebd.)!Gaius Julius Cæsar@\strich\emph{Gaius Julius Cæsar}|pwv} das
               ich 1918 über \uline{Cäsar}\pwindex{Caesar, Gaius Iulius 13.7.100? v. Chr. Rom – 15.3.44 v. Chr. ebd.@\textsc{Caesar, Gaius Iulius} (13.7.100? v. Chr. Rom – 15.3.44 v. Chr. ebd.), \emph{Politiker, Kaiser, Heerführer}|pw} schrieb sei \label{K_L02422-3v}\edtext{deutsch
                  erschienen}{\lemma{\textnormal{\emph{deutsch
                  erschienen}}}\Cendnote{\textnormal{Georg Brandes\pwindex{Brandes, Georg 4.\,2.\,1842 Kopenhagen – 19.\,2.\,1927 ebd.@\textsc{Brandes, Georg} (4.\,2.\,1842 Kopenhagen – 19.\,2.\,1927 ebd.)|pwk}: \emph{Cajus Julius Caesar}\pwindex{Brandes, Georg 4.\,2.\,1842 Kopenhagen – 19.\,2.\,1927 ebd.@\textsc{Brandes, Georg} (4.\,2.\,1842 Kopenhagen – 19.\,2.\,1927 ebd.)!Gaius Julius Cæsar@\strich\emph{Gaius Julius Cæsar}|pwk}. Autorisierte Übersetzung von Erwin Magnus\pwindex{Magnus, Erwin 24.\,11.\,1881 Hamburg – 31.\,3.\,1947 Kopenhagen@\textsc{Magnus, Erwin} (24.\,11.\,1881 Hamburg – 31.\,3.\,1947 Kopenhagen), \emph{Übersetzer}|pwk}. Berlin: \emph{Erich Reiss}\orgindex{Erich-Reiss-Verlag@Erich-Reiss-Verlag|pwk}{ }1925 (erschienen September 1924).}}}\label{K_L02422-3}. Ich
               habe weder ein Exemplar noch ein Honorar gesehen.\pend
           \pstart Ihr\hspace*{3.5em}\spacefill\mbox{Georg Brandes}\pend{}\selectlanguage{ngerman}\endnumbering\briefempfaengerindex{Schnitzler, Arthur@\textsc{Schnitzler, Arthur}!zzzBrandes, Georg@\emph{von Georg Brandes}!1924-12-101@{10. 12. 1924}|)be}\mylabel{L02422h}  \newcommand{\dateiname}{L02422}\newcommand{\titel}{Georg Brandes an Arthur Schnitzler, 10. 12. 1924}\newcommand{\editorInnen}{Martin Anton Müller und Gerd-Hermann Susen}%% latex-leseansicht-abspann.tex
%% Abspann für die Leseansicht.
%% Der Schalter \ifkorrekturansicht ist bereits durch den Vorspann gesetzt.

%% latex-abspann.tex
%% Gemeinsamer Abspann für Korrekturansicht und Leseansicht.
%% Setzt den Schalter \ifkorrekturansicht voraus (gesetzt in den
%% einbindenden Dateien latex-korrekturansicht-abspann.tex bzw.
%% latex-leseansicht-abspann.tex).
%% ---------------------------------------------------------------

\normalsize

% Das esempio-Environment wird nur in der Leseansicht benötigt
\ifkorrekturansicht\else
\newenvironment{esempio}[3]%
{
    \vspace{1.5ex}
    \rlap{\underline{#1}}
    \par
    \setlength{\parindent}{0cm}
    \nopagebreak
    \leftskip=#2cm
    \rightskip=#3cm
}
{
    \par
}
\fi

\doendnotes{C}
\bigskip
\vfill

\clearpage

\footnotesize

\ifkorrekturansicht
  \lohead{\textsc{register}}
\fi

% theindex-Environment neu definieren ohne reledmac
\makeatletter
\renewenvironment{theindex}{%
  \ifkorrekturansicht
    \section*{\indexname}%
  \else
    \subsubsection*{Index der erwähnten Entitäten}%
  \fi
  \setlength{\parindent}{0pt}%
  \setlength{\parskip}{0pt plus 0.3pt}%
  \let\item\@idxitem
}{%
  \ifkorrekturansicht\clearpage\fi
}
\makeatother

\IfFileExists{\jobname-pw.ind}{\input{\jobname-pw.ind}}{}

% Quellenangabe nur in der Leseansicht
\ifkorrekturansicht\else
% Fallback-Definitionen, falls die .tex-Datei \titel etc. nicht gesetzt hat
\providecommand{\titel}{}
\providecommand{\editorInnen}{}
\providecommand{\dateiname}{\jobname}

\vspace{3cm}

\vfill

\footnotesize
\textsc{Quelle}: \titel. Herausgegeben von {\editorInnen}. In: \emph{Arthur Schnitzler: Briefwechsel mit Autorinnen und Autoren}.
 Digitale Edition, https://schnitzler-briefe.acdh.oeaw.ac.at/{\dateiname}.html (Stand \today)
\fi

\end{document}


