%% latex-leseansicht-vorspann.tex
%% Vorspann für die Leseansicht.
%% Lädt die gemeinsame Datei latex-vorspann.tex mit nicht gesetztem Schalter.

\newif\ifkorrekturansicht
\korrekturansichtfalse

\input{../tex-inputs/latex-vorspann}


\section[Arthur Schnitzler an Hugo von Hofmannsthal, 10. 7. 1898]{L00816 Arthur Schnitzler an Hugo von Hofmannsthal, 10. 7. 1898}
\nopagebreak\mylabel{L00816v}
\rehead{ }\normalsize\beginnumbering\briefempfaengerindex{Hofmannsthal, Hugo von@\textsc{Hofmannsthal, Hugo von}!zzzSchnitzler, Arthur@\emph{von Arthur Schnitzler}!1898-07-102@{10. 7. 1898}|(be}
\toendnotes[C]{\smallbreak\pagebreak[2]}
\correspDesc{Versand  durch Arthur Schnitzler am 10. 7. 1898 in Wien
\newline{}Erhalt  durch Hugo von Hofmannsthal im Zeitraum [11. 7. 1898
                  – 15. 7. 1898?] in Tschortkiw}\toendnotes[C]{\smallbreak}
\Standort{FDH, Hs-30885,69.}
\physDesc{Brief, 1 Blatt, 2 Seiten, 417 Zeichen
\newline{}Handschrift: schwarze Tinte, deutsche Kurrent}
\buchAbdrucke{\weitereDrucke{Hugo von Hofmannsthal, Arthur Schnitzler: \emph{Briefwechsel}. Herausgegeben von Therese Nickl und Heinrich Schnitzler. Frankfurt am Main: \emph{S. Fischer} 1964, S. 105.} }\toendnotes[C]{\smallbreak}
\pstart
           \raggedleft{}{\pb}So{\geminationn}tag, 10. 7. 98.\pend
           
\pstart{}Mein lieber Hugo,\pend\vspace{0.5em}
\pstart
           morgen Früh reiſe ich ab. Bis Ende der Woche (16.) treffen mich
               Nachrichten in Graz, Hotel zum Elefanten\oindex{Hotel Elefant@\textbf{Hotel Elefant}, \emph{Hotel}|pw}. Für
               das neue Stück\pwindex{Schnitzler, Arthur 15.\,5.\,1862 Wien – 21.\,10.\,1931 ebd.@\textsc{Schnitzler, Arthur} (15.\,5.\,1862 Wien – 21.\,10.\,1931 ebd.), \emph{Schriftsteller, Mediziner}!Schleier der Beatrice. Schauspiel in fünf Akten@\strich\emph{Der Schleier der Beatrice. Schauspiel in fünf Akten}|pwv} iſt mir viel
               und gutes eingefallen; doch werd ich es vor Auguſt kaum beginnen, da ich
               ein bischen \textsc{Burckhard}\pwindex{Burckhardt, Jacob 25.\,5.\,1818 Basel – 8.\,8.\,1897 ebd.@\textsc{Burckhardt, Jacob} (25.\,5.\,1818 Basel – 8.\,8.\,1897 ebd.), \emph{Historiker, Kunsthistoriker}|pw}, \textsc{Gregorovius}\pwindex{Gregorovius, Ferdinand 19.\,1.\,1821 Nidzica – 1.\,5.\,1891 München@\textsc{Gregorovius, Ferdinand} (19.\,1.\,1821 Nidzica – 1.\,5.\,1891 München), \emph{Schriftsteller, Historiker}|pw}, {\pb}\textsc{Geiger}\pwindex{Geiger, Ludwig 5.\,6.\,1848 Breslau – 9.\,2.\,1919 Berlin@\textsc{Geiger, Ludwig} (5.\,6.\,1848 Breslau – 9.\,2.\,1919 Berlin)|pw} leſen will (dazu.)\pend
           
\pstart
           – Meine Sti{\geminationm}ung iſt recht düſter; entko{\geminationm}en werd ich ihr nicht.\pend
           
\pstart
           Laſſen Sie doch bald von{ }ſich hören.\pend
           \pstart Von Herzen Ihr \spacefill\mbox{Arthur.}\pend{}\selectlanguage{ngerman}\endnumbering\briefempfaengerindex{Hofmannsthal, Hugo von@\textsc{Hofmannsthal, Hugo von}!zzzSchnitzler, Arthur@\emph{von Arthur Schnitzler}!1898-07-102@{10. 7. 1898}|)be}\mylabel{L00816h}  \newcommand{\dateiname}{L00816}\newcommand{\titel}{Arthur Schnitzler an Hugo von Hofmannsthal, 10. 7. 1898}\newcommand{\editorInnen}{Martin Anton Müller und Gerd-Hermann Susen}%% latex-leseansicht-abspann.tex
%% Abspann für die Leseansicht.
%% Der Schalter \ifkorrekturansicht ist bereits durch den Vorspann gesetzt.

%% latex-abspann.tex
%% Gemeinsamer Abspann für Korrekturansicht und Leseansicht.
%% Setzt den Schalter \ifkorrekturansicht voraus (gesetzt in den
%% einbindenden Dateien latex-korrekturansicht-abspann.tex bzw.
%% latex-leseansicht-abspann.tex).
%% ---------------------------------------------------------------

\normalsize

% Das esempio-Environment wird nur in der Leseansicht benötigt
\ifkorrekturansicht\else
\newenvironment{esempio}[3]%
{
    \vspace{1.5ex}
    \rlap{\underline{#1}}
    \par
    \setlength{\parindent}{0cm}
    \nopagebreak
    \leftskip=#2cm
    \rightskip=#3cm
}
{
    \par
}
\fi

\doendnotes{C}
\bigskip
\vfill

\clearpage

\footnotesize

\ifkorrekturansicht
  \lohead{\textsc{register}}
\fi

% theindex-Environment neu definieren ohne reledmac
\makeatletter
\renewenvironment{theindex}{%
  \ifkorrekturansicht
    \section*{\indexname}%
  \else
    \subsubsection*{Index der erwähnten Entitäten}%
  \fi
  \setlength{\parindent}{0pt}%
  \setlength{\parskip}{0pt plus 0.3pt}%
  \let\item\@idxitem
}{%
  \ifkorrekturansicht\clearpage\fi
}
\makeatother

\IfFileExists{\jobname-pw.ind}{\input{\jobname-pw.ind}}{}

% Quellenangabe nur in der Leseansicht
\ifkorrekturansicht\else
% Fallback-Definitionen, falls die .tex-Datei \titel etc. nicht gesetzt hat
\providecommand{\titel}{}
\providecommand{\editorInnen}{}
\providecommand{\dateiname}{\jobname}

\vspace{3cm}

\vfill

\footnotesize
\textsc{Quelle}: \titel. Herausgegeben von {\editorInnen}. In: \emph{Arthur Schnitzler: Briefwechsel mit Autorinnen und Autoren}.
 Digitale Edition, https://schnitzler-briefe.acdh.oeaw.ac.at/{\dateiname}.html (Stand \today)
\fi

\end{document}


