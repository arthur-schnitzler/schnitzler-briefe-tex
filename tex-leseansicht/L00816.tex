%% latex-korrekturansicht-vorspann.tex
%% Vorspann für die Korrekturansicht.
%% Lädt die gemeinsame Datei latex-vorspann.tex mit gesetztem Schalter.

\newif\ifkorrekturansicht
\korrekturansichttrue

\input{../tex-inputs/latex-vorspann}


\section[Arthur Schnitzler an Hugo von Hofmannsthal, 10. 7. 1898]{L00816 Arthur Schnitzler an Hugo von Hofmannsthal, 10. 7. 1898}
\nopagebreak\mylabel{L00816v}
\rehead{ }\normalsize\beginnumbering\briefempfaengerindex{Hofmannsthal, Hugo von@\textsc{Hofmannsthal, Hugo von}!zzzSchnitzler, Arthur@\emph{von Arthur Schnitzler}!1898-07-102@{10. 7. 1898}|(be}
\toendnotes[C]{\smallbreak\pagebreak[2]}\Standort{FDH, Hs-30885,69.}
\physDesc{Brief, 1 Blatt, 2 Seiten, 417 Zeichen
\newline{}Handschrift: schwarze Tinte, deutsche Kurrent}
\buchAbdrucke{\weitereDrucke{Hugo von Hofmannsthal, Arthur Schnitzler: \emph{Briefwechsel}. Frankfurt am Main: \emph{S. Fischer} 1964, S. 105.} }\toendnotes[C]{\smallbreak}
\pstart
           \raggedleft{}{\pb}So{\geminationn}tag, 10. 7. 98.\pend
           
\pstart{}Mein lieber Hugo,\pend\vspace{0.5em}
\pstart
           morgen Früh reiſe ich ab. Bis Ende der Woche (16.) treffen mich
               Nachrichten in Graz, Hotel zum Elefanten\oindex{Hotel Elefant@\textbf{Hotel Elefant}, \emph{Hotel (K.HTL)}|pw}. Für
               das neue Stück\pwindex{Schleier der Beatrice. Schauspiel in fuenf Akten@\emph{Der Schleier der Beatrice. Schauspiel in fünf Akten}|pwv} iſt mir viel
               und gutes eingefallen; doch werd ich es vor Auguſt kaum beginnen, da ich
               ein bischen \textsc{Burckhard}\pwindex{Burckhardt, Jacob 25.05.1818 – 08.08.1897@\textsc{Burckhardt, Jacob} (25.05.1818 – 08.08.1897), \emph{Historiker/Historikerin, Kunsthistoriker/Kunsthistorikerin}|pw}, \textsc{Gregorovius}\pwindex{Gregorovius, Ferdinand 19.01.1821 – 01.05.1891@\textsc{Gregorovius, Ferdinand} (19.01.1821 – 01.05.1891), \emph{Schriftsteller/Schriftstellerin, Historiker/Historikerin}|pw}, {\pb}\textsc{Geiger}\pwindex{Geiger, Ludwig 05.06.1848 – 09.02.1919@\textsc{Geiger, Ludwig} (05.06.1848 – 09.02.1919)|pw} leſen will (dazu.)\pend
           
\pstart
           – Meine Sti{\geminationm}ung iſt recht düſter; entko{\geminationm}en werd ich ihr nicht.\pend
           
\pstart
           Laſſen Sie doch bald von ſich hören.\pend
           \pstart Von Herzen Ihr \spacefill\mbox{Arthur.}\pend{}\selectlanguage{ngerman}\endnumbering\briefempfaengerindex{Hofmannsthal, Hugo von@\textsc{Hofmannsthal, Hugo von}!zzzSchnitzler, Arthur@\emph{von Arthur Schnitzler}!1898-07-102@{10. 7. 1898}|)be}\mylabel{L00816h}  \normalsize

\doendnotes{C}
\bigskip
\vfill

\clearpage

\footnotesize

\lohead{\textsc{register}}

% Definiere theindex-Environment komplett neu ohne reledmac
\makeatletter
\renewenvironment{theindex}{%
  \section*{\indexname}%
  \setlength{\parindent}{0pt}%
  \setlength{\parskip}{0pt plus 0.3pt}%
  \let\item\@idxitem
}{%
  \clearpage
}
\makeatother

\IfFileExists{\jobname-pw.ind}{\input{\jobname-pw.ind}}{}

\end{document}

      