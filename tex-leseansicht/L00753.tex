%% latex-leseansicht-vorspann.tex
%% Vorspann für die Leseansicht.
%% Lädt die gemeinsame Datei latex-vorspann.tex mit nicht gesetztem Schalter.

\newif\ifkorrekturansicht
\korrekturansichtfalse

\input{../tex-inputs/latex-vorspann}


\section[Max Burckhard an Arthur Schnitzler, {{[}}27.? 12. 1897{{]}}]{L00753 Max Burckhard an Arthur Schnitzler, {[}27.? 12. 1897{]}}
\nopagebreak\mylabel{L00753v}
\rehead{ }\normalsize\beginnumbering\briefempfaengerindex{Schnitzler, Arthur@\textsc{Schnitzler, Arthur}!zzzBurckhard, Max Eugen@\emph{von Max Eugen Burckhard}!1897-12-272@{{[}27.? 12. 1897{]}}|(be}
\toendnotes[C]{\smallbreak\pagebreak[2]}
\correspDesc{Versand  durch Max Burckhard am [27.? 12. 1897] in Wien
\newline{}Erhalt  durch Arthur Schnitzler im Zeitraum [27. 12. 1897 – 31. 12. 1897?] in Wien}\toendnotes[C]{\smallbreak}
\Standort{CUL, Schnitzler, B 20.}
\physDesc{Visitenkarte, 212 Zeichen
\newline{}Handschrift: schwarze Tinte, deutsche Kurrent
\newline{}Ordnung: von Schnitzler datiert: »\strikeout{Anf 98}{ }Dez 97«, von unbekannter Hand nummeriert:
                                 »10« }\toendnotes[C]{\smallbreak}
\pstart
           \centering{}{\pb}\textcolor{gray}{\textbf{\textsc{D\textsuperscript{r.} Max Eugen Burckhard}}}\pend
           
\pstart
           \centering{}\textcolor{gray}{\textbf{\textsc{K. u. K. Director des K. K. Hofburgtheaters\oindex{Wien@\textbf{Wien}!I., Innere Stadt@\textbf{I., Innere Stadt}!Burgtheater@\textbf{Burgtheater}, \emph{Theater}|pw}}}}\pend
           
\pstart{}{\pb}Sehr verehrter Herr Doctor!\pend\vspace{0.5em}
\pstart
           Ich komme um ½ 3 zum Speiſen nachhause – würde es Ihnen \uline{nach Tisch}{ }\introOben{}(\introOben{}also{ }\strikeout{(}circa
                  ¼ 4) genehm{ }ſein,{ }ſo komme ich \label{K_L00753-1v}\edtext{um diese Stunde}{\lemma{\textnormal{\emph{um diese Stunde}}}\Cendnote{\textnormal{Im
                     Dezember 1897 notierte sich Schnitzler nur ein Treffen mit Burckhard\pwindex{Burckhard, Max Eugen 14.\,7.\,1854 Korneuburg – 16.\,3.\,1912 Wien@\textsc{Burckhard, Max Eugen} (14.\,7.\,1854 Korneuburg – 16.\,3.\,1912 Wien), \emph{Schriftsteller, Rechtswissenschaftler, Theaterleiter}|pwk}. Dieses fand am 27. 12. 1897 am Nachmittag statt. Beim Treffen las
                  er \emph{Das Vermächtnis}\pwindex{Schnitzler, Arthur 15.\,5.\,1862 Wien – 21.\,10.\,1931 ebd.@\textsc{Schnitzler, Arthur} (15.\,5.\,1862 Wien – 21.\,10.\,1931 ebd.), \emph{Schriftsteller, Mediziner}!Vermächtnis. Schauspiel in drei Akten@\strich\emph{Das Vermächtnis. Schauspiel in drei Akten}|pwk} vor.}}}\label{K_L00753-1} hinab – oder
               Sie zu mir wie es Ihnen lieber ist.\pend
           
\pstart
           Herzlichst{\\[\baselineskip]}\spacefill\mbox{DrBurc}\pend
           \leftskip=0em{}\selectlanguage{ngerman}\endnumbering\briefempfaengerindex{Schnitzler, Arthur@\textsc{Schnitzler, Arthur}!zzzBurckhard, Max Eugen@\emph{von Max Eugen Burckhard}!1897-12-272@{{[}27.? 12. 1897{]}}|)be}\mylabel{L00753h}  \newcommand{\dateiname}{L00753}\newcommand{\titel}{Max Burckhard an Arthur Schnitzler, [27.? 12. 1897]}\newcommand{\editorInnen}{Martin Anton Müller und Gerd-Hermann Susen}%% latex-leseansicht-abspann.tex
%% Abspann für die Leseansicht.
%% Der Schalter \ifkorrekturansicht ist bereits durch den Vorspann gesetzt.

%% latex-abspann.tex
%% Gemeinsamer Abspann für Korrekturansicht und Leseansicht.
%% Setzt den Schalter \ifkorrekturansicht voraus (gesetzt in den
%% einbindenden Dateien latex-korrekturansicht-abspann.tex bzw.
%% latex-leseansicht-abspann.tex).
%% ---------------------------------------------------------------

\normalsize

% Das esempio-Environment wird nur in der Leseansicht benötigt
\ifkorrekturansicht\else
\newenvironment{esempio}[3]%
{
    \vspace{1.5ex}
    \rlap{\underline{#1}}
    \par
    \setlength{\parindent}{0cm}
    \nopagebreak
    \leftskip=#2cm
    \rightskip=#3cm
}
{
    \par
}
\fi

\doendnotes{C}
\bigskip
\vfill

\clearpage

\footnotesize

\ifkorrekturansicht
  \lohead{\textsc{register}}
\fi

% theindex-Environment neu definieren ohne reledmac
\makeatletter
\renewenvironment{theindex}{%
  \ifkorrekturansicht
    \section*{\indexname}%
  \else
    \subsubsection*{Index der erwähnten Entitäten}%
  \fi
  \setlength{\parindent}{0pt}%
  \setlength{\parskip}{0pt plus 0.3pt}%
  \let\item\@idxitem
}{%
  \ifkorrekturansicht\clearpage\fi
}
\makeatother

\IfFileExists{\jobname-pw.ind}{\input{\jobname-pw.ind}}{}

% Quellenangabe nur in der Leseansicht
\ifkorrekturansicht\else
% Fallback-Definitionen, falls die .tex-Datei \titel etc. nicht gesetzt hat
\providecommand{\titel}{}
\providecommand{\editorInnen}{}
\providecommand{\dateiname}{\jobname}

\vspace{3cm}

\vfill

\footnotesize
\textsc{Quelle}: \titel. Herausgegeben von {\editorInnen}. In: \emph{Arthur Schnitzler: Briefwechsel mit Autorinnen und Autoren}.
 Digitale Edition, https://schnitzler-briefe.acdh.oeaw.ac.at/{\dateiname}.html (Stand \today)
\fi

\end{document}


