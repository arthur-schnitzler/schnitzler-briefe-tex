%% latex-korrekturansicht-vorspann.tex
%% Vorspann für die Korrekturansicht.
%% Lädt die gemeinsame Datei latex-vorspann.tex mit gesetztem Schalter.

\newif\ifkorrekturansicht
\korrekturansichttrue

\input{../tex-inputs/latex-vorspann}


\section[Max Burckhard an Arthur Schnitzler, {[}27.? 12. 1897{]}]{L00753 Max Burckhard an Arthur Schnitzler, {[}27.? 12. 1897{]}}
\nopagebreak\mylabel{L00753v}
\rehead{ }\normalsize\beginnumbering\briefempfaengerindex{Schnitzler, Arthur@\textsc{Schnitzler, Arthur}!zzzBurckhard, Max Eugen@\emph{von Max Eugen Burckhard}!1897-12-272@{{[}27.? 12. 1897{]}}|(be}
\toendnotes[C]{\smallbreak\pagebreak[2]}\Standort{CUL, Schnitzler, B 20.}
\physDesc{Visitenkarte, 212 Zeichen
\newline{}Handschrift: schwarze Tinte, deutsche Kurrent
\newline{}Ordnung: von Schnitzler datiert: »\strikeout{Anf 98}{ }Dez 97«, von unbekannter Hand nummeriert:
                                 »10« }\toendnotes[C]{\smallbreak}
\pstart
           \centering{}{\pb}\textcolor{gray}{\textbf{\textsc{D\textsuperscript{r.} Max Eugen Burckhard}}}\pend
           
\pstart
           \centering{}\textcolor{gray}{\textbf{\textsc{K. u. K. Director des K. K. Hofburgtheaters\oindex{Burgtheater@\textbf{Burgtheater}, \emph{S.THTR}|pw}}}}\pend
           
\pstart{}{\pb}Sehr verehrter Herr Doctor! \pend\vspace{0.5em}
\pstart
           Ich komme um ½ 3 zum Speiſen nachhause – würde es Ihnen \uline{nach Tisch}{ }\introOben{}(\introOben{}also{ }\strikeout{(}circa
                  ¼ 4) genehm ſein, ſo komme ich \label{K_L00753-1v}\edtext{um diese Stunde}{\lemma{\textnormal{\emph{um diese Stunde}}}\Cendnote{\textnormal{Im
                     Dezember 1897 notierte sich Schnitzler nur ein Treffen mit Burckhard\pwindex{Burckhard, Max Eugen 14.07.1854 – 16.03.1912@\textsc{Burckhard, Max Eugen} (14.07.1854 – 16.03.1912), \emph{Schriftsteller/Schriftstellerin, Rechtswissenschaftler/Rechtswissenschaftlerin, Theaterleiter/Theaterleiterin}|pwk}. Dieses fand am 27. 12. 1897 am Nachmittag statt. Beim Treffen las
                  er \emph{Das Vermächtnis}\pwindex{Vermaechtnis. Schauspiel in drei Akten@\emph{Das Vermächtnis. Schauspiel in drei Akten}|pwk} vor.}}}\label{K_L00753-1} hinab – oder
               Sie zu mir wie es Ihnen lieber ist.\pend
           
\pstart
           Herzlichst{\\[\baselineskip]}\spacefill\mbox{DrBurc}\pend
           \leftskip=0em{}\selectlanguage{ngerman}\endnumbering\briefempfaengerindex{Schnitzler, Arthur@\textsc{Schnitzler, Arthur}!zzzBurckhard, Max Eugen@\emph{von Max Eugen Burckhard}!1897-12-272@{{[}27.? 12. 1897{]}}|)be}\mylabel{L00753h}  \normalsize

\doendnotes{C}
\bigskip
\vfill

\clearpage

\footnotesize

\lohead{\textsc{register}}

% Definiere theindex-Environment komplett neu ohne reledmac
\makeatletter
\renewenvironment{theindex}{%
  \section*{\indexname}%
  \setlength{\parindent}{0pt}%
  \setlength{\parskip}{0pt plus 0.3pt}%
  \let\item\@idxitem
}{%
  \clearpage
}
\makeatother

\IfFileExists{\jobname-pw.ind}{\input{\jobname-pw.ind}}{}

\end{document}

      