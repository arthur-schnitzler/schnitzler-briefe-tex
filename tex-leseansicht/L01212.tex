%% latex-leseansicht-vorspann.tex
%% Vorspann für die Leseansicht.
%% Lädt die gemeinsame Datei latex-vorspann.tex mit nicht gesetztem Schalter.

\newif\ifkorrekturansicht
\korrekturansichtfalse

\input{../tex-inputs/latex-vorspann}


         
         \renewcommand{\erwaehntePersonen}{Personen: Hermann Bahr, Bjørnstjerne Bjørnson, Gustav Modry, Aur. St.}
         \renewcommand{\erwaehnteOrte}{Orte: Venedig, Volkstheater, Wien}
         \renewcommand{\erwaehnteWerke}{Werke: Über unsere Kraft. Zweiter Teil}
               \section[Arthur Schnitzler an Hermann Bahr, 30. 3. 1902]{ Arthur Schnitzler an Hermann Bahr, 30. 3. 1902}\nopagebreak\mylabel{v}\rehead{ }\begin{ledgroupsized}[t]{13cm}\normalsize\beginnumbering\briefempfaengerindex{Bahr, Hermann@\textsc{Bahr, Hermann}!zzzSchnitzler, Arthur@\emph{von Arthur Schnitzler}!1902-03-301@{30. 3. 1902}|(be} \toendnotes[C]{\smallbreak\pagebreak[2]} \Standort{TMW, HS AM 23350 Ba.}
\physDesc{Brief, 1 Blatt, 3 Seiten, 962 Zeichen
\newline{}Handschrift: schwarze Tinte, deutsche Kurrent
\newline{}Ordnung: 1) Lochung  2) mit Bleistift von unbekannter Hand ergänzt: »\textsc{Charfreitag}«}\buchAbdrucke{\weitereDrucke{1) \emph{[28. 3.] 1902.} In: Arthur Schnitzler: \emph{The Letters of Arthur Schnitzler to Hermann Bahr}. Edited, annotated, and with an introduction, by Donald G.
                        Daviau. Chapel Hill: \emph{The University of North Carolina Press} 1978, S. 74–75 (University of North Carolina studies in the Germanic languages
                        and literatures, 89).} \weitereDrucke{2) Hermann Bahr, Arthur Schnitzler: \emph{Briefwechsel, Aufzeichnungen, Dokumente (1891–1931)}. Hg. Kurt Ifkovits und Martin Anton Müller. Göttingen: \emph{Wallstein} 2018, S. 227–228.} }\toendnotes[C]{\smallbreak}\pstart
           \raggedleft{}{\pb}Oſterſo{\geminationn}tag 1902\pend
           \pstart{}lieber Hermann,\pend\pstart
           eine \label{K_L01212-1v}\edtext{Dame\pwindex{St., Aur. @\textsc{St., Aur.}|pwv}}{\lemma{\textnormal{\emph{Dame}}}\Cendnote{\textnormal{Vgl. A. S.: \emph{Tagebuch}, 30. 3. 1902: »Aur. St.\pwindex{St., Aur. @\textsc{St., Aur.}|pw}«.}}}\label{K_L01212-1h} bringt mir beiliegende 2 Skizzen{[},{]} der Verfaſſer\pwindex{Modry, Gustav 01.09.1875 – 16.01.1928@\textsc{Modry, Gustav} (01.09.1875 – 16.01.1928), \emph{Rechtsanwalt}|pwv} hat die Abſicht
               Journaliſt zu werden. Ich ſoll ihn protegiren. Was anders ſoll er noch nicht
               geſchrieben haben. Auf dich hab ich ſo viel Einfluſs, ich ſoll’s dir doch einfach
               ſchicken.\pend
           \pstart
           Ich thue das, nicht ohne mich für diese Inanſpruchnahme deiner Zeit gebührend zu
               entſchuldigen. Aber ich denke, in 3 Minuten haſt du die Werke des jungen Manns
               geleſen, und wir ſind \introOben{}(bis auf weiteres)\introOben{} von dem Verdacht
               befreit, {\pb}die »Jungen«
               zu unterdrücken.\pend
           \pstart
           Wenn du mir überdies in 3 Worten dein Urtheil über die Leiſtungen dieſes Herrn
               kundgibſt, in einem Brief, den ich der Dame gleich zeigen ka{\geminationn}, u. mit \substVorne{}\textsuperscript{g}\substDazwischen{}d\substHinten{}einer \introOben{}ganzen\introOben{} Aufrichtigkeit, die in dieſem Fall
               beſonders nützlich, ja nothwendig wäre, ſo bin ich dir ſehr verbunden. –\pend
           \pstart
           In Venedig\oindex{Venedig@\textbf{Venedig}|pw}{ }ſollen die Blattern ſein. Man müßte ſich für alle
               Fälle impfen laſſen, eh man hinunter{\pb}radelt.\pend
           \pstart
           Ich ſeh dich übrigens bei der \label{K_L01212-2v}\edtext{»Kraft«probe\pwindex{Bjørnson, Bjørnstjerne 1832-12-08 – 1910-04-26@\textsc{Bjørnson, Bjørnstjerne} (1832-12-08 – 1910-04-26), \emph{Schriftsteller}!Ueber unsere Kraft. Zweiter Teil1896@\strich\emph{Über unsere Kraft. Zweiter Teil} {[}1896{]}|pw}}{\lemma{\textnormal{\emph{»Kraft«probe}}}\Cendnote{\textnormal{\emph{Über unsere Kraft}\pwindex{Bjørnson, Bjørnstjerne 1832-12-08 – 1910-04-26@\textsc{Bjørnson, Bjørnstjerne} (1832-12-08 – 1910-04-26), \emph{Schriftsteller}!Ueber unsere Kraft. Zweiter Teil1896@\strich\emph{Über unsere Kraft. Zweiter Teil} {[}1896{]}|pwk} von Bjørnson\pwindex{Bjørnson, Bjørnstjerne 1832-12-08 – 1910-04-26@\textsc{Bjørnson, Bjørnstjerne} (1832-12-08 – 1910-04-26), \emph{Schriftsteller}|pwk} wurde im Deutschen
                     Volkstheater\oindex{Volkstheater@\textbf{Volkstheater}|pwk} in zwei Teilen gegeben, der erste am 4., der zweite am 5. 4. 1902. Ob auch
                  die Generalprobe auf zwei Tage aufgeteilt war, ist unklar.}}}\label{K_L01212-2h}.\pend
           \pstart
           Herzlichſt der Deine{\\[\baselineskip]}\spacefill\mbox{Arth Sch}\pend
           \leftskip=0em{}
         
         \endnumbering\mylabel{h}\end{ledgroupsized}  \newcommand{\dateiname}{L01212}\newcommand{\titel}{Arthur Schnitzler an Hermann Bahr, 30. 3. 1902}\newcommand{\editorInnen}{ Kurt Ifkovits,  Martin Anton Müller}%% latex-leseansicht-abspann.tex
%% Abspann für die Leseansicht.
%% Der Schalter \ifkorrekturansicht ist bereits durch den Vorspann gesetzt.

%% latex-abspann.tex
%% Gemeinsamer Abspann für Korrekturansicht und Leseansicht.
%% Setzt den Schalter \ifkorrekturansicht voraus (gesetzt in den
%% einbindenden Dateien latex-korrekturansicht-abspann.tex bzw.
%% latex-leseansicht-abspann.tex).
%% ---------------------------------------------------------------

\normalsize

% Das esempio-Environment wird nur in der Leseansicht benötigt
\ifkorrekturansicht\else
\newenvironment{esempio}[3]%
{
    \vspace{1.5ex}
    \rlap{\underline{#1}}
    \par
    \setlength{\parindent}{0cm}
    \nopagebreak
    \leftskip=#2cm
    \rightskip=#3cm
}
{
    \par
}
\fi

\doendnotes{C}
\bigskip
\vfill

\clearpage

\footnotesize

\ifkorrekturansicht
  \lohead{\textsc{register}}
\fi

% theindex-Environment neu definieren ohne reledmac
\makeatletter
\renewenvironment{theindex}{%
  \ifkorrekturansicht
    \section*{\indexname}%
  \else
    \subsubsection*{Index der erwähnten Entitäten}%
  \fi
  \setlength{\parindent}{0pt}%
  \setlength{\parskip}{0pt plus 0.3pt}%
  \let\item\@idxitem
}{%
  \ifkorrekturansicht\clearpage\fi
}
\makeatother

\IfFileExists{\jobname-pw.ind}{\input{\jobname-pw.ind}}{}

% Quellenangabe nur in der Leseansicht
\ifkorrekturansicht\else
% Fallback-Definitionen, falls die .tex-Datei \titel etc. nicht gesetzt hat
\providecommand{\titel}{}
\providecommand{\editorInnen}{}
\providecommand{\dateiname}{\jobname}

\vspace{3cm}

\vfill

\footnotesize
\textsc{Quelle}: \titel. Herausgegeben von {\editorInnen}. In: \emph{Arthur Schnitzler: Briefwechsel mit Autorinnen und Autoren}.
 Digitale Edition, https://schnitzler-briefe.acdh.oeaw.ac.at/{\dateiname}.html (Stand \today)
\fi

\end{document}


      