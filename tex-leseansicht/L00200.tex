%% latex-leseansicht-vorspann.tex
%% Vorspann für die Leseansicht.
%% Lädt die gemeinsame Datei latex-vorspann.tex mit nicht gesetztem Schalter.

\newif\ifkorrekturansicht
\korrekturansichtfalse

\input{../tex-inputs/latex-vorspann}


         
         \renewcommand{\erwaehntePersonen}{Personen: Hermann Bahr, Richard Beer-Hofmann, Jakob Julius David, Friedrich Michael Fels, Felix Salten, Gustav Schwarzkopf}
         \renewcommand{\erwaehnteOrte}{Orte: Wien}
         \renewcommand{\erwaehnteWerke}{Werke: Der Thor und der Tod}
               \section[Hugo von Hofmannsthal an Arthur Schnitzler, {[}21. 4. 1893{]}]{ Hugo von Hofmannsthal an Arthur Schnitzler, {[}21. 4. 1893{]}}\nopagebreak\mylabel{v}\rehead{ }\begin{ledgroupsized}[t]{13cm}\normalsize\beginnumbering \toendnotes[C]{\smallbreak\pagebreak[2]} \Standort{CUL, Schnitzler, B 43.}
\physDesc{Briefkarte, 726 Zeichen (aufgeprägtes Wappen )
\newline{}Handschrift: schwarze Tinte, deutsche Kurrent
\newline{}Schnitzler: mit Bleistift das Datum ergänzt: »21/4 93« und nummeriert: »46« }\buchAbdrucke{\weitereDrucke{1) Hugo von Hofmannsthal, Arthur Schnitzler: \emph{Briefwechsel}. Hg. Therese Nickl und Heinrich Schnitzler. Frankfurt am Main: \emph{S. Fischer} 1964, S. 38.} \weitereDrucke{2) Hermann Bahr, Arthur Schnitzler: \emph{Briefwechsel, Aufzeichnungen, Dokumente (1891–1931)}. Hg. Kurt Ifkovits und Martin Anton Müller. Göttingen: \emph{Wallstein} 2018, S. 35.} }\toendnotes[C]{\smallbreak}\pstart
           \raggedleft{}{\pb}Freitag, abend.\pend
           \pstart{}Lieber Arthur!\pend\pstart
           Ich finde das Benehmen des Fels\pwindex{Fels, Friedrich Michael *~1864@\textsc{Fels, Friedrich Michael} (*~1864), \emph{Journalist}|pw} nicht recht
               verſtändlich. Ich habe die ganze Geſchichte sogleich an \textsc{J. J. David}\pwindex{David, Jakob Julius 1859-02-06 – 1906-11-20@\textsc{David, Jakob Julius} (1859-02-06 – 1906-11-20), \emph{Schriftsteller, Journalist}|pw} geſchrieben, und von ſeiner größeren Routine in Journalſachen einen Rath
               erbeten. Er antwortet mir: er kann nichts thuen, iſt übrigens durch das »frevelhafte
               Stillſchweigen des Fels\pwindex{Fels, Friedrich Michael *~1864@\textsc{Fels, Friedrich Michael} (*~1864), \emph{Journalist}|pw} vollkommen
               disguſtiert«. Heute Nacht ſpreche ich Bahr\pwindex{Bahr, Hermann 19.07.1863 – 15.01.1934@\textsc{Bahr, Hermann} (19.07.1863 – 15.01.1934), \emph{Schriftsteller, Kritiker}|pw} und
               ſchreibe Ihnen pneumatiſch das Reſultat.\pend
           \pstart
           \numberlinefalse{}\centering{}–\numberlinetrue{}\pend
           \pstart
           \noindent{}Ich werde mit meinem Einacter\pwindex{Hofmannsthal, Hugo von 1874-02-01 – 1929-07-15@\textsc{Hofmannsthal, Hugo von} (1874-02-01 – 1929-07-15), \emph{Schriftsteller}!Thor und der Tod1893@\strich\emph{Der Thor und der Tod} {[}1893{]}|pwv}{ }Sonntag fertig und möchte daß wir den
                  nachmittag 4–9 miteinander verbringen, Land oder Stadt, damit ich ihn
               vorleſen kann, natürlich nur unter uns 5\pwindex{Beer-Hofmann, Richard 1866-07-11 – 1945-09-26@\textsc{Beer-Hofmann, Richard} (1866-07-11 – 1945-09-26), \emph{Schriftsteller}|pwv}\pwindex{Salten, Felix 06.09.1869 – 08.10.1945@\textsc{Salten, Felix} (06.09.1869 – 08.10.1945), \emph{Schriftsteller, Journalist}|pwv}\pwindex{Schwarzkopf, Gustav 07.11.1853 – 13.11.1939@\textsc{Schwarzkopf, Gustav} (07.11.1853 – 13.11.1939), \emph{Schriftsteller}|pwv}\pwindex{Hofmannsthal, Hugo von 1874-02-01 – 1929-07-15@\textsc{Hofmannsthal, Hugo von} (1874-02-01 – 1929-07-15), \emph{Schriftsteller}|pwv} (die
               Hex mitgerechnet). Bei dieſer Gelegenheit beſprechen wir wohl am
               beſten das unmittelbar {\pb}nötige in
               der ekelhaften obigen Affaire.\pend
           \pstart Ihr \spacefill\mbox{Hugo}\pend{}
         
         \endnumbering\mylabel{h}\end{ledgroupsized}  \newcommand{\dateiname}{L00200}\newcommand{\titel}{Hugo von Hofmannsthal an Arthur Schnitzler, [21. 4. 1893]}\newcommand{\editorInnen}{ Martin Anton Müller und Gerd-Hermann Susen}%% latex-leseansicht-abspann.tex
%% Abspann für die Leseansicht.
%% Der Schalter \ifkorrekturansicht ist bereits durch den Vorspann gesetzt.

%% latex-abspann.tex
%% Gemeinsamer Abspann für Korrekturansicht und Leseansicht.
%% Setzt den Schalter \ifkorrekturansicht voraus (gesetzt in den
%% einbindenden Dateien latex-korrekturansicht-abspann.tex bzw.
%% latex-leseansicht-abspann.tex).
%% ---------------------------------------------------------------

\normalsize

% Das esempio-Environment wird nur in der Leseansicht benötigt
\ifkorrekturansicht\else
\newenvironment{esempio}[3]%
{
    \vspace{1.5ex}
    \rlap{\underline{#1}}
    \par
    \setlength{\parindent}{0cm}
    \nopagebreak
    \leftskip=#2cm
    \rightskip=#3cm
}
{
    \par
}
\fi

\doendnotes{C}
\bigskip
\vfill

\clearpage

\footnotesize

\ifkorrekturansicht
  \lohead{\textsc{register}}
\fi

% theindex-Environment neu definieren ohne reledmac
\makeatletter
\renewenvironment{theindex}{%
  \ifkorrekturansicht
    \section*{\indexname}%
  \else
    \subsubsection*{Index der erwähnten Entitäten}%
  \fi
  \setlength{\parindent}{0pt}%
  \setlength{\parskip}{0pt plus 0.3pt}%
  \let\item\@idxitem
}{%
  \ifkorrekturansicht\clearpage\fi
}
\makeatother

\IfFileExists{\jobname-pw.ind}{\input{\jobname-pw.ind}}{}

% Quellenangabe nur in der Leseansicht
\ifkorrekturansicht\else
% Fallback-Definitionen, falls die .tex-Datei \titel etc. nicht gesetzt hat
\providecommand{\titel}{}
\providecommand{\editorInnen}{}
\providecommand{\dateiname}{\jobname}

\vspace{3cm}

\vfill

\footnotesize
\textsc{Quelle}: \titel. Herausgegeben von {\editorInnen}. In: \emph{Arthur Schnitzler: Briefwechsel mit Autorinnen und Autoren}.
 Digitale Edition, https://schnitzler-briefe.acdh.oeaw.ac.at/{\dateiname}.html (Stand \today)
\fi

\end{document}


      