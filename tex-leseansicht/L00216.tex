%% latex-leseansicht-vorspann.tex
%% Vorspann für die Leseansicht.
%% Lädt die gemeinsame Datei latex-vorspann.tex mit nicht gesetztem Schalter.

\newif\ifkorrekturansicht
\korrekturansichtfalse

\input{../tex-inputs/latex-vorspann}


               \section[Fedor Mamroth an Arthur Schnitzler, 4. 6. 1893]{ Fedor Mamroth an Arthur Schnitzler, 4. 6. 1893}\nopagebreak\mylabel{v}\rehead{ }\begin{ledgroupsized}[t]{13cm}\normalsize\beginnumbering\briefempfaengerindex{Schnitzler, Arthur@\textsc{Schnitzler, Arthur}!zzzMamroth, Fedor@\emph{von Fedor Mamroth}!1893-06-041@{4. 6. 1893}|(be} \toendnotes[C]{\smallbreak\pagebreak[2]} \Standort{CUL, Schnitzler, B 68.}
\physDesc{Brief, 1 Blatt, 2 Seiten
\newline{}Handschrift: blaue Tinte, deutsche Kurrent
\newline{}Schnitzler: mit rotem Buntstift eine Unterstreichung \newline{}Ordnung: mit Bleistift von unbekannter Hand nummeriert: »5« }\toendnotes[C]{\smallbreak}\pstart
           \noindent{}{\pb}\textcolor{gray}{\textbf{\textsc{Frankfurter Zeitung}}}{\\}\textsc{\textcolor{gray}{\textbf{und}}}{\\}\textcolor{gray}{\textbf{\textsc{Handelsblatt.}}}\orgindex{Frankfurter Zeitung@Frankfurter Zeitung|pw}\pend
           \pstart
           \textcolor{gray}{\textbf{\textsc{Redaktion.\footnote{\noindent{}\textcolor{gray}{\textbf{\textsc{Für die Redaktion bestimmte
                                                  Briefe und Sendungen wolle man \so{nicht} an die Person eines
                                                  Redakteurs, sondern stets \textbf{an die
                                                  Redaktion der Frankfurter Zeitung}
                                                  adressiren}}}.}}}}\hfill \textcolor{gray}{\textbf{\textsc{Frankfurt a. M.\oindex{Frankfurt am Main@\textbf{Frankfurt am Main}|pw},}}}{ }4. Juni \textsc{\textcolor{gray}{\textbf{189}}}3\pend
           \pstart
           \textcolor{gray}{\textbf{\textsc{Telegramm-Adresse:}}}\pend
           \pstart
           \textcolor{gray}{\textbf{\textsc{Zeitung Frankfurt Main.}}}\pend
           \pstart{}Sehr verehrter Herr Doctor!\pend\pstart
           Ich habe Ihren Roman »Der ſterbende Herr\pwindex{Schnitzler, Arthur 15.05.1862 – 21.10.1931@\textsc{Schnitzler, Arthur} (15.05.1862 – 21.10.1931), \emph{Schriftsteller, Mediziner}!Sterben. Novelle1.10.1894 – 1.12.1894@\strich\emph{Sterben. Novelle} {[}1.10.1894 – 1.12.1894{]}|pw}« mit
                    einer Theilnahme geleſen, die mir noch ſelten eine eingereichte Arbeit
                    eingeflößt hat. Ich beglückwünſche Sie zu dieſer Dichtung, in der ſie
                    den feinen Geiſt eines Poeten und \introOben{}die\introOben{}{ }ſcharfe Beobachtungsgabe des Arztes mit merkwürdiger Ergänzungskunſt
                    verſchmolzen haben. Allein »Der ſterbende
                        Herr\pwindex{Schnitzler, Arthur 15.05.1862 – 21.10.1931@\textsc{Schnitzler, Arthur} (15.05.1862 – 21.10.1931), \emph{Schriftsteller, Mediziner}!Sterben. Novelle1.10.1894 – 1.12.1894@\strich\emph{Sterben. Novelle} {[}1.10.1894 – 1.12.1894{]}|pw}« iſt kein Zeitungs- ſondern ein Buchroman; erſtens nicht aus
                    Gründen, die ich an dieſer Stelle nicht zu erörtern vermag. Darf ich mir
                    erlauben, Ihnen einen Rath zu ertheilen, ſo würde ich Ihnen dringend empfehlen,
                    für die Veröffentlichung Ihrer ſchönen Arbeit, die Ihnen einen verdienten Erfolg
                    einbringen wird, ohne Verzug einen Verleger zu ſuchen. Mein Intereſſe daran iſt
                    ein ſo aufrichtiges, daß es mir ein Vergnügen wäre, Ihnen auch perſönlich in
                    dieſer Richtung zu dienen, wenn ich dem Kreiſe der deutſchen Verleger leider
                    nicht völlig fernſtünde. Aber ich kann mir nicht denken, daß Ihnen eine
                    Placirung der Arbeit Schwierigkeiten bereiten ſollte. Es gibt doch gewiß
                    Unternehmer von Urtheil u. Geſchmack, die den Werth einer ſo hervorragenden
                    Compoſition zu ſchätzen wiſſen! Eine Änderung des Titels würde ich Ihnen
                    ernſtlich {\pb}in Vorſchlag bringen. Wie
                    denken Sie über »Das letzte Jahr« oder »Ende« oder »Ein Todesurtheil« oder »Der
                    Wille zum Leben« u. ſ. w. All das heißt auch nicht viel, aber es ſcheint mir
                    doch beſſer als der gewählte Titel.\pend
           \pstart
           Verſäumen Sie nicht, mir Nachricht zu geben, ſobald der Roman\pwindex{Schnitzler, Arthur 15.05.1862 – 21.10.1931@\textsc{Schnitzler, Arthur} (15.05.1862 – 21.10.1931), \emph{Schriftsteller, Mediziner}!Sterben. Novelle1.10.1894 – 1.12.1894@\strich\emph{Sterben. Novelle} {[}1.10.1894 – 1.12.1894{]}|pwv} unter Dach u. Fach gelangt.\pend
           \pstart
           Hochachtungsvoll{\\[\baselineskip]}Ihr{\\[\baselineskip]}ergebener{\\[\baselineskip]}\spacefill\mbox{D\textsuperscript{r} FMamroth.}\pend
           \leftskip=0em{}          \endnumbering\briefempfaengerindex{Schnitzler, Arthur@\textsc{Schnitzler, Arthur}!zzzMamroth, Fedor@\emph{von Fedor Mamroth}!1893-06-041@{4. 6. 1893}|)be}\mylabel{h}\end{ledgroupsized}  \newcommand{\dateiname}{L00216}\newcommand{\titel}{Fedor Mamroth an Arthur Schnitzler, 4. 6. 1893}\newcommand{\editorInnen}{Martin Anton Müller und Gerd-Hermann Susen}%% latex-leseansicht-abspann.tex
%% Abspann für die Leseansicht.
%% Der Schalter \ifkorrekturansicht ist bereits durch den Vorspann gesetzt.

%% latex-abspann.tex
%% Gemeinsamer Abspann für Korrekturansicht und Leseansicht.
%% Setzt den Schalter \ifkorrekturansicht voraus (gesetzt in den
%% einbindenden Dateien latex-korrekturansicht-abspann.tex bzw.
%% latex-leseansicht-abspann.tex).
%% ---------------------------------------------------------------

\normalsize

% Das esempio-Environment wird nur in der Leseansicht benötigt
\ifkorrekturansicht\else
\newenvironment{esempio}[3]%
{
    \vspace{1.5ex}
    \rlap{\underline{#1}}
    \par
    \setlength{\parindent}{0cm}
    \nopagebreak
    \leftskip=#2cm
    \rightskip=#3cm
}
{
    \par
}
\fi

\doendnotes{C}
\bigskip
\vfill

\clearpage

\footnotesize

\ifkorrekturansicht
  \lohead{\textsc{register}}
\fi

% theindex-Environment neu definieren ohne reledmac
\makeatletter
\renewenvironment{theindex}{%
  \ifkorrekturansicht
    \section*{\indexname}%
  \else
    \subsubsection*{Index der erwähnten Entitäten}%
  \fi
  \setlength{\parindent}{0pt}%
  \setlength{\parskip}{0pt plus 0.3pt}%
  \let\item\@idxitem
}{%
  \ifkorrekturansicht\clearpage\fi
}
\makeatother

\IfFileExists{\jobname-pw.ind}{\input{\jobname-pw.ind}}{}

% Quellenangabe nur in der Leseansicht
\ifkorrekturansicht\else
% Fallback-Definitionen, falls die .tex-Datei \titel etc. nicht gesetzt hat
\providecommand{\titel}{}
\providecommand{\editorInnen}{}
\providecommand{\dateiname}{\jobname}

\vspace{3cm}

\vfill

\footnotesize
\textsc{Quelle}: \titel. Herausgegeben von {\editorInnen}. In: \emph{Arthur Schnitzler: Briefwechsel mit Autorinnen und Autoren}.
 Digitale Edition, https://schnitzler-briefe.acdh.oeaw.ac.at/{\dateiname}.html (Stand \today)
\fi

\end{document}


      