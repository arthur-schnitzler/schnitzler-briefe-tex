%% latex-korrekturansicht-vorspann.tex
%% Vorspann für die Korrekturansicht.
%% Lädt die gemeinsame Datei latex-vorspann.tex mit gesetztem Schalter.

\newif\ifkorrekturansicht
\korrekturansichttrue

\input{../tex-inputs/latex-vorspann}


\section[Fedor Mamroth an Arthur Schnitzler, 4. 6. 1893]{L00216 Fedor Mamroth an Arthur Schnitzler, 4. 6. 1893}
\nopagebreak\mylabel{L00216v}
\rehead{ }\normalsize\beginnumbering\briefempfaengerindex{Schnitzler, Arthur@\textsc{Schnitzler, Arthur}!zzzMamroth, Fedor@\emph{von Fedor Mamroth}!1893-06-041@{4. 6. 1893}|(be}
\toendnotes[C]{\smallbreak\pagebreak[2]}\Standort{CUL, Schnitzler, B 68.}
\physDesc{Brief, 1 Blatt, 2 Seiten, 1553 Zeichen
\newline{}Handschrift: blaue Tinte, deutsche Kurrent
\newline{}Schnitzler: mit rotem Buntstift eine Unterstreichung 
\newline{}Ordnung: mit Bleistift von unbekannter Hand nummeriert:
                                 »5« }\toendnotes[C]{\smallbreak}
\pstart
           {\pb}\textcolor{gray}{\textbf{\textsc{Frankfurter Zeitung}}}{\\}\textsc{\textcolor{gray}{\textbf{und}}}{\\}\textcolor{gray}{\textbf{\textsc{Handelsblatt.}}}\orgindex{Frankfurter Zeitung@Frankfurter Zeitung|pw}\pend
           
\pstart
           
\pstart
           \textcolor{gray}{\textbf{\textsc{Redaktion.\noindent{}\textcolor{gray}{\textbf{\textsc{Für die Redaktion bestimmte Briefe und
                                       Sendungen wolle man \so{nicht} an die
                                       Person eines Redakteurs, sondern stets \textbf{an die
                                          Redaktion der Frankfurter Zeitung} adressiren}}}.}}}\pend
           
\pstart
           \raggedleft{}\textcolor{gray}{\textbf{\textsc{Frankfurt a. M.\oindex{Frankfurt am Main@\textbf{Frankfurt am Main}, \emph{P.PPLA3}|pw},}}}{ }4. Juni \textsc{\textcolor{gray}{\textbf{189}}}3\pend
           \pend
           
\pstart
           \textcolor{gray}{\textbf{\textsc{Telegramm-Adresse:}}}\pend
           
\pstart
           \textcolor{gray}{\textbf{\textsc{Zeitung Frankfurt Main.}}}\pend
           
\pstart{}Sehr verehrter Herr Doctor!\pend\vspace{0.5em}
\pstart
           Ich habe Ihren Roman »Der ſterbende Herr\pwindex{Sterben. Novelle@\emph{Sterben. Novelle}|pw}« mit
               einer Theilnahme geleſen, die mir noch ſelten eine eingereichte Arbeit eingeflößt
               hat. Ich beglückwünſche Sie zu dieſer Dichtung, in der ſie den feinen
               Geiſt eines Poeten und \introOben{}die\introOben{}{ }ſcharfe Beobachtungsgabe des Arztes mit
               merkwürdiger Ergänzungskunſt verſchmolzen haben. Allein »Der ſterbende Herr\pwindex{Sterben. Novelle@\emph{Sterben. Novelle}|pw}« iſt kein Zeitungs- ſondern ein Buchroman;
               erſtens nicht aus Gründen, die ich an dieſer Stelle nicht zu erörtern vermag. Darf
               ich mir erlauben, Ihnen einen Rath zu ertheilen, ſo würde ich Ihnen dringend
               empfehlen, für die Veröffentlichung Ihrer ſchönen Arbeit, die Ihnen einen verdienten
               Erfolg einbringen wird, ohne Verzug einen Verleger zu ſuchen. Mein Intereſſe daran
               iſt ein ſo aufrichtiges, daß es mir ein Vergnügen wäre, Ihnen auch perſönlich in
               dieſer Richtung zu dienen, wenn ich dem Kreiſe der deutſchen Verleger leider nicht
               völlig fernſtünde. Aber ich kann mir nicht denken, daß Ihnen eine Placirung der
               Arbeit Schwierigkeiten bereiten ſollte. Es gibt doch gewiß Unternehmer von Urtheil u.
               Geſchmack, die den Werth einer ſo hervorragenden Compoſition zu ſchätzen wiſſen! Eine
               Änderung des Titels würde ich Ihnen ernſtlich {\pb}in Vorſchlag bringen. Wie denken Sie über
               »Das letzte Jahr« oder »Ende« oder »Ein Todesurtheil« oder »Der Wille zum Leben«
               u. ſ. w. All das heißt auch nicht viel, aber es ſcheint mir doch beſſer als der
               gewählte Titel.\pend
           
\pstart
           Verſäumen Sie nicht, mir Nachricht zu geben, ſobald der Roman\pwindex{Sterben. Novelle@\emph{Sterben. Novelle}|pwv} unter Dach u. Fach gelangt.\pend
           
\pstart
           Hochachtungsvoll{\\[\baselineskip]}Ihr{\\[\baselineskip]}ergebener{\\[\baselineskip]}\spacefill\mbox{D\textsuperscript{r} FMamroth.}\pend
           \leftskip=0em{}\selectlanguage{ngerman}\endnumbering\briefempfaengerindex{Schnitzler, Arthur@\textsc{Schnitzler, Arthur}!zzzMamroth, Fedor@\emph{von Fedor Mamroth}!1893-06-041@{4. 6. 1893}|)be}\mylabel{L00216h}  \normalsize

\doendnotes{C}
\bigskip
\vfill

\clearpage

\footnotesize

\lohead{\textsc{register}}

% Definiere theindex-Environment komplett neu ohne reledmac
\makeatletter
\renewenvironment{theindex}{%
  \section*{\indexname}%
  \setlength{\parindent}{0pt}%
  \setlength{\parskip}{0pt plus 0.3pt}%
  \let\item\@idxitem
}{%
  \clearpage
}
\makeatother

\IfFileExists{\jobname-pw.ind}{\input{\jobname-pw.ind}}{}

\end{document}

      