%% latex-korrekturansicht-vorspann.tex
%% Vorspann für die Korrekturansicht.
%% Lädt die gemeinsame Datei latex-vorspann.tex mit gesetztem Schalter.

\newif\ifkorrekturansicht
\korrekturansichttrue

\input{../tex-inputs/latex-vorspann}


\section[Richard Beer-Hofmann an Arthur Schnitzler, {[}20. 12. 1896{]}]{L00631 Richard Beer-Hofmann an Arthur Schnitzler, {[}20. 12. 1896{]}}
\nopagebreak\mylabel{L00631v}
\rehead{ }\normalsize\beginnumbering\briefempfaengerindex{Schnitzler, Arthur@\textsc{Schnitzler, Arthur}!zzzBeer-Hofmann, Richard@\emph{von Richard Beer-Hofmann}!1896-12-201@{{[}20. 12. 1896{]}}|(be}
\toendnotes[C]{\smallbreak\pagebreak[2]}\Standort{CUL, Schnitzler, B 8.}
\physDesc{Brief, 1 Blatt, 2 Seiten, 277 Zeichen
\newline{}Handschrift: blauer Buntstift, lateinische Kurrent
\newline{}Schnitzler: mit Bleistift datiert: »20/12 96« 
\newline{}Ordnung: mit Bleistift von unbekannter Hand nummeriert:
                                    »91« }
\pstart
           \noindent{}{\pb}Lieber Arthur! Hier der beste Sitz, der noch zu haben war. (Er
               kostet 3.50 also 4.) Ich ko{\geminationm}e morgen nicht \introOben{}ins Theater u. Stefanskeller\oindex{Stephanskeller@\textbf{Stephanskeller}, \emph{Gastgewerbegebäude (K.GGW)}|pw}\introOben{}, bin nicht aufgelegt will arbeiten\pend
           
\pstart
           Sagen Sie ich wäre sehr erkältet, bedaure sehr, u. s. w. und {\pb}das ist nur zur Hälfte unwahr.\pend
           
\pstart
           Herzlichst Ihr {\\[\baselineskip]}\spacefill\mbox{Richard}\pend
           \leftskip=0em{}\selectlanguage{ngerman}\endnumbering\briefempfaengerindex{Schnitzler, Arthur@\textsc{Schnitzler, Arthur}!zzzBeer-Hofmann, Richard@\emph{von Richard Beer-Hofmann}!1896-12-201@{{[}20. 12. 1896{]}}|)be}\mylabel{L00631h}  \normalsize

\doendnotes{C}
\bigskip
\vfill

\clearpage

\footnotesize

\lohead{\textsc{register}}

% Definiere theindex-Environment komplett neu ohne reledmac
\makeatletter
\renewenvironment{theindex}{%
  \section*{\indexname}%
  \setlength{\parindent}{0pt}%
  \setlength{\parskip}{0pt plus 0.3pt}%
  \let\item\@idxitem
}{%
  \clearpage
}
\makeatother

\IfFileExists{\jobname-pw.ind}{\input{\jobname-pw.ind}}{}

\end{document}

      