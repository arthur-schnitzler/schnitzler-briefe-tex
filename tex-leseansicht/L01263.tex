%% latex-korrekturansicht-vorspann.tex
%% Vorspann für die Korrekturansicht.
%% Lädt die gemeinsame Datei latex-vorspann.tex mit gesetztem Schalter.

\newif\ifkorrekturansicht
\korrekturansichttrue

\input{../tex-inputs/latex-vorspann}


\section[Arthur Schnitzler an Hugo von Hofmannsthal, 7. 1. 1903]{L01263 Arthur Schnitzler an Hugo von Hofmannsthal, 7. 1. 1903}
\nopagebreak\mylabel{L01263v}
\rehead{ }\normalsize\beginnumbering\briefempfaengerindex{Hofmannsthal, Hugo von@\textsc{Hofmannsthal, Hugo von}!zzzSchnitzler, Arthur@\emph{von Arthur Schnitzler}!1903-01-071@{7. 1. 1903}|(be}
\toendnotes[C]{\smallbreak\pagebreak[2]}\Standort{FDH, Hs-30885,101.}
\physDesc{Brief, 1 Blatt, 4 Seiten, 1936 Zeichen
\newline{}Handschrift: schwarze Tinte, deutsche Kurrent
\newline{}Ordnung: eine längere Unterstreichung von unbekannter Hand mit
                                 Bleistift }
\buchAbdrucke{\weitereDrucke{1) Hugo von Hofmannsthal, Arthur Schnitzler: \emph{Briefwechsel}. Frankfurt am Main: \emph{S. Fischer} 1964, S. 165.} \weitereDrucke{2) Arthur Schnitzler: \emph{Briefe 1875–1912}. Frankfurt am Main: \emph{S. Fischer} 1981, S. 453–454.} }\toendnotes[C]{\smallbreak}
\pstart
           {\pb}7. 1. 903.\pend
           \vspace{0.5em}
\pstart
           mein lieber Hugo, zum 2. Akt\pwindex{gerettete Venedig. Trauerspiel in fuenf Aufzuegen@\emph{Das gerettete Venedig. Trauerspiel in fünf Aufzügen}|pwv} wäre vielleicht noch folgendes zu bemerken: wie wenn
               die Angſt der Verſchworenen nicht \uline{ganz} ohne concrete
               Urſache geweſen wäre? Unter den harmloſen Spaziergängern könnte irgend ein nicht \uline{ganz} harmloſer ſein; – einer von den Verſchworenen auf
               die Straße, greift ihn auf, thut ihn ab \introOben{}(wie, iſt Ihre
                  Sache)\introOben{}. Ich glaube, mit 5–6 Zeilen iſt das zu machen und Sie gewinnen für
               den Zuſchauer den Eindruck der wirklichen Gefahr, befreien ihn von {\pb}dem unbewußten, aber dem fernern Intereſſe nicht ganz
               unſchädlichen Aerger, \uline{mit} den Verſchworenen
               aufgeſeſſen zu ſein. »Wer einmal lügt, dem glaubt man nicht« gehört in gewiſſem Sinne
               zu den dramatiſchen Warnungen.\pend
           
\pstart
           – Ferner: Sie ſteigern u vereinfachen den 3. Akt\pwindex{gerettete Venedig. Trauerspiel in fuenf Aufzuegen@\emph{Das gerettete Venedig. Trauerspiel in fünf Aufzügen}|pwv} – wenn Sie das Motiv der leidenſchaftlichen Liebe
                  Jaffiers\pwindex{gerettete Venedig. Trauerspiel in fuenf Aufzuegen@\emph{Das gerettete Venedig. Trauerspiel in fünf Aufzügen}|pwv} für ſeine Frau
               mindeſtens für einige Augenblicke mit meinethalben übertriebener Heftigkeit {\pb}durchblitzen laſſen. Es iſt ja da, ich weiſs, – aber
               entſpricht es nicht ſogar dem Weſen Ihres Jaffiers\pwindex{gerettete Venedig. Trauerspiel in fuenf Aufzuegen@\emph{Das gerettete Venedig. Trauerspiel in fünf Aufzügen}|pwv} beſonders, wenn er dieſe Empfindg, zu ſeiner
               eignen Rechtfertigung, in einer hiſtoriſch-komödiantenhaften Weiſe, ausdrückt,
               vorträgt, ja die Scene damit erfüllt?\pend
           
\pstart
           – \label{T_L01263-1v}\edtext{Es war geſtern wahrhaftig ſo viel von
               den paar techniſchen Zweifeln die Rede, die rege wurden, daſs man nicht dazugeko{\geminationm}en iſt, das viele gute ja außerordentliche zu
                  begrüßen,}{\lemma{\textnormal{\emph{Es … begrüßen,}}}\Cendnote{\textnormal{mit Bleistift von unbekannter
                  Hand unterstrichen}}}\label{T_L01263-1}{ }{\pb}das Sie uns gegeben haben. Aber ich bin heute mit der
               Erinnerung an etwas prangendes, flutendes, kraftvolles aufgewacht, als das ſich Ihr
                  Stück\pwindex{gerettete Venedig. Trauerspiel in fuenf Aufzuegen@\emph{Das gerettete Venedig. Trauerspiel in fünf Aufzügen}|pwv} im Nachgenuſs meldet;
               und finde insbeſonders, daſs Sie diesmal Ihrem Vers, ohne daſs er an Schönheit das
               geringſte verloren, das dramatiſch\substVorne{}\textsuperscript{e}\substDazwischen{}hinſtürmende\substHinten{} verliehen haben wie noch nie zuvor. Ich glaube an die Zukunft dieſes Stücks\pwindex{gerettete Venedig. Trauerspiel in fuenf Aufzuegen@\emph{Das gerettete Venedig. Trauerspiel in fünf Aufzügen}|pwv} auf dem Theater. Leben
               Sie wohl und freuen Sie ſich nur, daſs Sie ſowas geſchrieben haben. So gut wie man
               ſich ſelber \label{T_L01263-2v}\edtext{freut, – freut ſich doch
               kein andrer – de{\geminationn} das beſste an dieſer Freude ſind die
               Schaffenserinnerungen, die im geheimen mitzittern. Ihr \spacefill\mbox{A.}}{\lemma{\textnormal{\emph{freut, – … A.}}}\Cendnote{\textnormal{über der Anrede auf dem Kopf}}}\label{T_L01263-2}\pend
           \selectlanguage{ngerman}\endnumbering\briefempfaengerindex{Hofmannsthal, Hugo von@\textsc{Hofmannsthal, Hugo von}!zzzSchnitzler, Arthur@\emph{von Arthur Schnitzler}!1903-01-071@{7. 1. 1903}|)be}\mylabel{L01263h}  \normalsize

\doendnotes{C}
\bigskip
\vfill

\clearpage

\footnotesize

\lohead{\textsc{register}}

% Definiere theindex-Environment komplett neu ohne reledmac
\makeatletter
\renewenvironment{theindex}{%
  \section*{\indexname}%
  \setlength{\parindent}{0pt}%
  \setlength{\parskip}{0pt plus 0.3pt}%
  \let\item\@idxitem
}{%
  \clearpage
}
\makeatother

\IfFileExists{\jobname-pw.ind}{\input{\jobname-pw.ind}}{}

\end{document}

      