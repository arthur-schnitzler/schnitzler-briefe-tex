%% latex-korrekturansicht-vorspann.tex
%% Vorspann für die Korrekturansicht.
%% Lädt die gemeinsame Datei latex-vorspann.tex mit gesetztem Schalter.

\newif\ifkorrekturansicht
\korrekturansichttrue

\input{../tex-inputs/latex-vorspann}


\section[Richard Beer-Hofmann an Arthur Schnitzler, {[}19. 8. 1893?{]}]{L00257 Richard Beer-Hofmann an Arthur Schnitzler, {[}19. 8. 1893?{]}}
\nopagebreak\mylabel{L00257v}
\rehead{ }\normalsize\beginnumbering\briefempfaengerindex{Schnitzler, Arthur@\textsc{Schnitzler, Arthur}!zzzBeer-Hofmann, Richard@\emph{von Richard Beer-Hofmann}!1893-08-191@{{[}19. 8. 1893?{]}}|(be}
\toendnotes[C]{\smallbreak\pagebreak[2]}\Standort{CUL, Schnitzler, B 8.}
\physDesc{Brief, 1 Blatt, 2 Seiten, 527 Zeichen
\newline{}Handschrift: Bleistift, deutsche Kurrent
\newline{}Schnitzler: mit Bleistift nummeriert: »23« }
\buchAbdrucke{\weitereDrucke{Arthur Schnitzler, Richard Beer-Hofmann: \emph{Briefwechsel 1891–1931}. Wien, Zürich: \emph{Europaverlag} 1992, S. 51.} }
\pstart
           \noindent{}{\pb}Lieber Arthur! Verzeihen Sie meine Nachlässigkeit; war in den
               letzten Tagen stark beschäftigt. Ich ko{\geminationm}{ }Montag{ }Abends{ }\introOben{}gegen\introOben{}{ }8 Uhr in Wien\oindex{Wien@\textbf{Wien}, \emph{A.ADM2}|pw} an. \damage{\textcolor{gray}{H}}abe mit Ihnen zu sprechen; und \damage{\textcolor{gray}{w}}erde Ihnen dann mündlich Alles beantworten. Schreiben Sie zwei Zeilen wo Sie
                  Montag{ }8 Uhr Abends sind, oder besser noch erwarten Sie mich zwischen
                  8 u ½ 9{ }Caffée Europe\oindex{Cafe de l Europe@\textbf{Café de l’Europe}, \emph{Kaffeehaus (K.KAF)}|pw}{ }Stefansplatz\oindex{Stephansplatz@\textbf{Stephansplatz}, \emph{S.SQR}|pw}. Ich war in Marienbad\oindex{Marienbad@\textbf{Marienbad}, \emph{P.PPL}|pw} bei Freund\pwindex{Freund, Carl @\textsc{Freund, Carl}, \emph{Verleger/Verlegerin}|pw} –
               Nichts Positives erreicht. Näheres mündlich. Vielleicht kann ich auch Schwarzkopf\pwindex{Schwarzkopf, Gustav 07.11.1853 – 13.11.1939@\textsc{Schwarzkopf, Gustav} (07.11.1853 – 13.11.1939), \emph{Schriftsteller/Schriftstellerin}|pw}{ }{\pb}sehen. Ich reise
                  Mittwoch{ }Früh nach Znaim\oindex{Znaim@\textbf{Znaim}, \emph{P.PPLA2}|pw}.\pend
           
\pstart
           Herzlichst{\\[\baselineskip]}\spacefill\mbox{Richard}\pend
           \leftskip=0em{}
\pstart
           \uline{Samstag Mittag}\pend
           \selectlanguage{ngerman}\endnumbering\briefempfaengerindex{Schnitzler, Arthur@\textsc{Schnitzler, Arthur}!zzzBeer-Hofmann, Richard@\emph{von Richard Beer-Hofmann}!1893-08-191@{{[}19. 8. 1893?{]}}|)be}\mylabel{L00257h}  \normalsize

\doendnotes{C}
\bigskip
\vfill

\clearpage

\footnotesize

\lohead{\textsc{register}}

% Definiere theindex-Environment komplett neu ohne reledmac
\makeatletter
\renewenvironment{theindex}{%
  \section*{\indexname}%
  \setlength{\parindent}{0pt}%
  \setlength{\parskip}{0pt plus 0.3pt}%
  \let\item\@idxitem
}{%
  \clearpage
}
\makeatother

\IfFileExists{\jobname-pw.ind}{\input{\jobname-pw.ind}}{}

\end{document}

      