%% latex-leseansicht-vorspann.tex
%% Vorspann für die Leseansicht.
%% Lädt die gemeinsame Datei latex-vorspann.tex mit nicht gesetztem Schalter.

\newif\ifkorrekturansicht
\korrekturansichtfalse

\input{../tex-inputs/latex-vorspann}


         
         \newcommand{\erwaehntePersonen}{Personen: Carl Freund, Gustav Schwarzkopf}
         \newcommand{\erwaehnteOrte}{Orte: Bad Ischl, Café de l’Europe, Marienbad, Stephansplatz, Wien, Znaim}
         \newcommand{\erwaehnteWerke}{
               \section[Richard Beer-Hofmann an Arthur Schnitzler, {[}19. 8. 1893?{]}]{ Richard Beer-Hofmann an Arthur Schnitzler, {[}19. 8. 1893?{]}}\nopagebreak\mylabel{v}\rehead{ }\begin{ledgroupsized}[t]{13cm}\normalsize\beginnumbering \toendnotes[C]{\smallbreak\pagebreak[2]} \Standort{CUL, Schnitzler, B 8.}
\physDesc{Brief, 1 Blatt, 2 Seiten
\newline{}Handschrift: Bleistift, deutsche Kurrent
\newline{}Schnitzler: mit Bleistift nummeriert: »23« }\buchAbdrucke{\weitereDrucke{Arthur Schnitzler, Richard Beer-Hofmann: \emph{Briefwechsel 1891–1931}. Hg. Konstanze Fliedl. Wien, Zürich: \emph{Europaverlag} 1992, S. 51.} }\pstart
           \noindent{}{\pb}Lieber Arthur! Verzeihen Sie meine Nachlässigkeit; war in den
               letzten Tagen stark beschäftigt. Ich ko{\geminationm}{ }Montag{ }Abends{ }\introOben{}gegen\introOben{}{ }8 Uhr in Wien\oindex{Wien@\textbf{Wien}|pw} an. \damage{\textcolor{gray}{H}}abe mit Ihnen zu sprechen; und \damage{\textcolor{gray}{w}}erde Ihnen dann mündlich Alles beantworten. Schreiben Sie zwei Zeilen wo Sie
                  Montag{ }8 Uhr Abends
               sind, oder besser noch erwarten Sie mich zwischen 8 u ½ 9{ }Caffée Europe\oindex{Cafe de l Europe@\textbf{Café de l’Europe}|pw}{ }Stefansplatz\oindex{Stephansplatz@\textbf{Stephansplatz}|pw}. Ich war in Marienbad\oindex{Marienbad@\textbf{Marienbad}|pw} bei Freund\pwindex{Freund, Carl @\textsc{Freund, Carl}, \emph{Verleger}|pw} – Nichts
               Positives erreicht. Näheres mündlich. Vielleicht kann ich auch Schwarzkopf\pwindex{Schwarzkopf, Gustav 07.11.1853 – 13.11.1939@\textsc{Schwarzkopf, Gustav} (07.11.1853 – 13.11.1939), \emph{Schriftsteller}|pw}{ }{\pb}sehen. Ich reise
                  Mittwoch{ }Früh nach Znaim\oindex{Znaim@\textbf{Znaim}|pw}.\pend
           \pstart
           Herzlichst{\\[\baselineskip]}\spacefill\mbox{Richard}\pend
           \leftskip=0em{}\pstart
           \uline{Samstag Mittag}\pend
           
         
         \endnumbering\mylabel{h}\end{ledgroupsized}  \newcommand{\dateiname}{L00257}\newcommand{\titel}{Richard Beer-Hofmann an Arthur Schnitzler, [19. 8. 1893?]}\newcommand{\editorInnen}{Martin Anton Müller und Gerd-Hermann Susen}%% latex-leseansicht-abspann.tex
%% Abspann für die Leseansicht.
%% Der Schalter \ifkorrekturansicht ist bereits durch den Vorspann gesetzt.

%% latex-abspann.tex
%% Gemeinsamer Abspann für Korrekturansicht und Leseansicht.
%% Setzt den Schalter \ifkorrekturansicht voraus (gesetzt in den
%% einbindenden Dateien latex-korrekturansicht-abspann.tex bzw.
%% latex-leseansicht-abspann.tex).
%% ---------------------------------------------------------------

\normalsize

% Das esempio-Environment wird nur in der Leseansicht benötigt
\ifkorrekturansicht\else
\newenvironment{esempio}[3]%
{
    \vspace{1.5ex}
    \rlap{\underline{#1}}
    \par
    \setlength{\parindent}{0cm}
    \nopagebreak
    \leftskip=#2cm
    \rightskip=#3cm
}
{
    \par
}
\fi

\doendnotes{C}
\bigskip
\vfill

\clearpage

\footnotesize

\ifkorrekturansicht
  \lohead{\textsc{register}}
\fi

% theindex-Environment neu definieren ohne reledmac
\makeatletter
\renewenvironment{theindex}{%
  \ifkorrekturansicht
    \section*{\indexname}%
  \else
    \subsubsection*{Index der erwähnten Entitäten}%
  \fi
  \setlength{\parindent}{0pt}%
  \setlength{\parskip}{0pt plus 0.3pt}%
  \let\item\@idxitem
}{%
  \ifkorrekturansicht\clearpage\fi
}
\makeatother

\IfFileExists{\jobname-pw.ind}{\input{\jobname-pw.ind}}{}

% Quellenangabe nur in der Leseansicht
\ifkorrekturansicht\else
% Fallback-Definitionen, falls die .tex-Datei \titel etc. nicht gesetzt hat
\providecommand{\titel}{}
\providecommand{\editorInnen}{}
\providecommand{\dateiname}{\jobname}

\vspace{3cm}

\vfill

\footnotesize
\textsc{Quelle}: \titel. Herausgegeben von {\editorInnen}. In: \emph{Arthur Schnitzler: Briefwechsel mit Autorinnen und Autoren}.
 Digitale Edition, https://schnitzler-briefe.acdh.oeaw.ac.at/{\dateiname}.html (Stand \today)
\fi

\end{document}


      