%% latex-korrekturansicht-vorspann.tex
%% Vorspann für die Korrekturansicht.
%% Lädt die gemeinsame Datei latex-vorspann.tex mit gesetztem Schalter.

\newif\ifkorrekturansicht
\korrekturansichttrue

\input{../tex-inputs/latex-vorspann}


\section[Oscar Blumenthal an Arthur Schnitzler, 16. 1. 1892]{L00062 Oscar Blumenthal an Arthur Schnitzler, 16. 1. 1892}
\nopagebreak\mylabel{L00062v}
\rehead{ }\normalsize\beginnumbering\briefempfaengerindex{Schnitzler, Arthur@\textsc{Schnitzler, Arthur}!zzzBlumenthal, Oskar@\emph{von Oskar Blumenthal}!1892-01-161@{16. 1. 1892}|(be}
\toendnotes[C]{\smallbreak\pagebreak[2]}\Standort{CUL, Schnitzler, B 15.}
\physDesc{Brief, 1 Blatt, 1 Seite, 466 Zeichen
\newline{}Handschrift Schreibkraft: schwarze Tinte, deutsche Kurrent
\newline{}Handschrift Oskar Blumenthal: schwarze Tinte, deutsche Kurrent
\newline{}Schnitzler: mit rotem Buntstift nummeriert: »2« 
\newline{}Ordnung: mit Bleistift von unbekannter Hand nummeriert:
                                 »2« }\toendnotes[C]{\smallbreak}
\pstart
           \centering{}{\pb}\textcolor{gray}{\textbf{LESSING-THEATER\orgindex{Lessing-Theater@Lessing-Theater|pw}}}\pend
           
\pstart
           \centering{}\textcolor{gray}{\textbf{Director:}}{\\}\textcolor{gray}{\textbf{Dr. Oscar Blumenthal.}}\pend
           
\pstart
           \raggedleft{}\textcolor{gray}{\textbf{Berlin N.W.\oindex{Berlin@\textbf{Berlin}, \emph{P.PPLC}|pw}, den}}{ }16. Januar \textcolor{gray}{\textbf{189}}2.{\\}\textcolor{gray}{\textbf{Friedrich-Carl-Ufer\oindex{Kapelle-Ufer@\textbf{Kapelle-Ufer}, \emph{Straße (K.STR)}|pw}}}.\pend
           
\pstart\center{}Sehr geehrter Herr Doktor!\pend\vspace{0.5em}
\pstart
           Die von Ihnen gewünſchte kritiſche Gloſſirung Ihres intereſſanten Schauſpiels\pwindex{Maerchen. Schauspiel in drei Aufzuegen@\emph{Das Märchen. Schauspiel in drei Aufzügen}|pwv} muß ich mir für den
                  Sommer aufſparen, da ich gegenwärtig durch eine Fülle von anderen
               dringenden Arbeiten zu ſehr in Anſpruch genommen bin. Jedenfalls rathe ich Ihnen
               nochmals, ſich \label{K_L00062-1v}\edtext{mit Herrn \textsc{Emanuel Reicher\pwindex{Reicher, Emanuel 18.06.1849 – 15.05.1924@\textsc{Reicher, Emanuel} (18.06.1849 – 15.05.1924), \emph{Schauspieler/Schauspielerin}|pw}}}{\lemma{\textnormal{\emph{mit … Reicher}}}\Cendnote{\textnormal{Laut \emph{Tagebuch}\pwindex{Tagebuch@\emph{Tagebuch}|pwk} schrieb Schnitzler am 24. 1. 1892 an Reicher\pwindex{Reicher, Emanuel 18.06.1849 – 15.05.1924@\textsc{Reicher, Emanuel} (18.06.1849 – 15.05.1924), \emph{Schauspieler/Schauspielerin}|pwk}.}}}\label{K_L00062-1} (\textsc{Berlin O.,} Alexanderſtraße 30\oindex{Alexanderstrasse@\textbf{Alexanderstraße}, \emph{Straße (K.STR)}|pw}) wegen der Aufnahme
               des Werks\pwindex{Maerchen. Schauspiel in drei Aufzuegen@\emph{Das Märchen. Schauspiel in drei Aufzügen}|pwv} in ſein Ausſtellungsrepertoire\orgindex{Emanuel Reicher s Deutsche Gastspielgesellschaft@Emanuel Reicher’s Deutsche Gastspielgesellschaft|pwv} in
               Verbindung zu ſetzen.\pend
           
\pstart
           Hochachtungsvoll{\\[\baselineskip]}\spacefill\mbox{{[}hs. :{]} Dr. Osc. Blumenthal}\pend
           \leftskip=0em{}\selectlanguage{ngerman}\endnumbering\briefempfaengerindex{Schnitzler, Arthur@\textsc{Schnitzler, Arthur}!zzzBlumenthal, Oskar@\emph{von Oskar Blumenthal}!1892-01-161@{16. 1. 1892}|)be}\mylabel{L00062h}  \normalsize

\doendnotes{C}
\bigskip
\vfill

\clearpage

\footnotesize

\lohead{\textsc{register}}

% Definiere theindex-Environment komplett neu ohne reledmac
\makeatletter
\renewenvironment{theindex}{%
  \section*{\indexname}%
  \setlength{\parindent}{0pt}%
  \setlength{\parskip}{0pt plus 0.3pt}%
  \let\item\@idxitem
}{%
  \clearpage
}
\makeatother

\IfFileExists{\jobname-pw.ind}{\input{\jobname-pw.ind}}{}

\end{document}

      