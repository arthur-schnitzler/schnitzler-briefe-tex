%% latex-korrekturansicht-vorspann.tex
%% Vorspann für die Korrekturansicht.
%% Lädt die gemeinsame Datei latex-vorspann.tex mit gesetztem Schalter.

\newif\ifkorrekturansicht
\korrekturansichttrue

\input{../tex-inputs/latex-vorspann}


\section[ Paul Goldmann an Arthur Schnitzler, 20. 5. {[}1899{]}]{L02875 Paul Goldmann an Arthur Schnitzler, 20. 5. {[}1899{]}}
\nopagebreak\mylabel{L02875v}
\rehead{ }\normalsize\beginnumbering\briefempfaengerindex{Schnitzler, Arthur@\textsc{Schnitzler, Arthur}!zzzGoldmann, Paul@\emph{von Paul Goldmann}!1899-05-201@{20. 5. {[}1899{]}}|(be}
\toendnotes[C]{\smallbreak\pagebreak[2]}\Standort{DLA, A:Schnitzler, HS.NZ85.1.3169.}
\physDesc{Brief, 1 Blatt, 4 Seiten, 2060 Zeichen
\newline{}Handschrift: schwarze Tinte, deutsche Kurrent
\newline{}Schnitzler: 1) mit Bleistift das Jahr »99« vermerkt  2) mit rotem Buntstift eine Unterstreichung}\toendnotes[C]{\smallbreak}
\pstart
           \raggedleft{}{\pb}\textsc{Haag\oindex{Den Haag@\textbf{Den Haag}, \emph{P.PPLG}|pw}}, 20. Mai.\pend
           
\pstart{}Mein lieber Freund,\pend\vspace{0.5em}
\pstart
           Von Berlin\oindex{Berlin@\textbf{Berlin}, \emph{P.PPLC}|pw} bin ich nach dem \textsc{Haag\oindex{Den Haag@\textbf{Den Haag}, \emph{P.PPLG}|pw}} beordert worden zur \label{K_L02875-1v}\edtext{Friedensconferenz}{\lemma{\textnormal{\emph{Friedensconferenz}}}\Cendnote{\textnormal{Die Haag\oindex{Den Haag@\textbf{Den Haag}, \emph{P.PPLG}|pwk}er Friedenskonferenz fand von 18. 5. 1899 bis 29. 7. 1899 statt.}}}\label{K_L02875-1}. Seit zehn Tage{[}n{]} lebe
               ich in einer unbeſchreiblichen Hetzjagd, und endlich heut finde ich fünf Minuten Zeit, um Dir von Herzen für Deinen lieben
               Brief zu danken, der mir nach Berlin\oindex{Berlin@\textbf{Berlin}, \emph{P.PPLC}|pw}
               nachgeſchickt wurde. Aber wichtiger wäre es mir, zu wiſſen, wie es Dir geht? Ich
               hoffe, nächſter Tage nach Frankfurt\oindex{Frankfurt am Main@\textbf{Frankfurt am Main}, \emph{P.PPLA3}|pw}
               zurückzukehren, und bitte Dich, mir \uuline{ſofort} eine Zeile
               dorthin zu ſenden, um mir zu ſagen, {\pb}wie Du Dich
               befindeſt?\pend
           
\pstart
           In Berlin\oindex{Berlin@\textbf{Berlin}, \emph{P.PPLC}|pw} habe ich natürlich den »Grünen Kakadu\pwindex{gruene Kakadu. Groteske in einem Akt@\emph{Der grüne Kakadu. Groteske in einem Akt}|pw}« geſehen. Ich kann Dir nur offen ſagen, mit
               jenem Freimuth, der zwiſchen uns Gebot iſt: Ich habe das Stück\pwindex{gruene Kakadu. Groteske in einem Akt@\emph{Der grüne Kakadu. Groteske in einem Akt}|pwv} nicht ſehr lieb. Es iſt ein
               glänzendes und ein geiſtreiches Stück\pwindex{gruene Kakadu. Groteske in einem Akt@\emph{Der grüne Kakadu. Groteske in einem Akt}|pw}, das
               ſeinen großen Erfolg wohl verdient; aber mir fehlt etwas darin, und ich habe die
               Empfindung, daß Du weit, weit höher ſtehſt, als dieſes Stück\pwindex{gruene Kakadu. Groteske in einem Akt@\emph{Der grüne Kakadu. Groteske in einem Akt}|pw}. Und d\textcolor{gray}{a}nn bleibe ich dabei: die fran\oindex{Frankreich@\textbf{Frankreich}, \emph{A.PCLI}|pwv}zöſiſche Revolution iſt
               nicht \uline{in} dem Stück\pwindex{gruene Kakadu. Groteske in einem Akt@\emph{Der grüne Kakadu. Groteske in einem Akt}|pwv}, in der Stimmung, ſondern ſie wird nur \strikeout{zum Schluß} als Effekt von draußen, als Aktſchluß
               verwendet. Sei mir nicht bös, {\pb}ich habe vielleicht
               Unrecht, aber jedenfalls iſt’s meine ehrliche, wohl erwogene Meinung{\dotssix}\pend
           
\pstart
           Vor meiner Abreiſe aus Frankfurt\oindex{Frankfurt am Main@\textbf{Frankfurt am Main}, \emph{P.PPLA3}|pw} habe ich \label{K_L02875-2v}\edtext{etwas}{\lemma{\textnormal{\emph{etwas}}}\Cendnote{\textnormal{Eventuell wurde hier auf den Beginn der intimen Beziehung
                  mit der verheirateten Theodore Rottenberg\pwindex{Rottenberg, Theodore 1875-09-07 – 1945-04-05@\textsc{Rottenberg, Theodore} (1875-09-07 – 1945-04-05)|pwk}
                  angespielt (siehe Paul Goldmann an Arthur Schnitzler, 8. 10. [1899]).}}}\label{K_L02875-2}
               erlebt, das für jeden Menſchen den Gipfel des Glücks bedeuten würde. Für mich iſts
               durch meine an Wahnſinn grenzende Nervoſität, die in dieſem Augenblick noch durch
               Krankheit complicirt iſt, zu einer der größten ſeeliſchen Kataſtrophen ausgeſchlagen
                  \strikeout{haben}, die ich noch durchgemacht habe. Niemals
               habe ich dem Selbſtmord ſo nahe geſtanden, – niemals auch hätte ich Deines Troſtes
               und Rathes \strikeout{b\textcolor{gray}{e}} mehr bedurft. Aber es ſteht geſchrieben, daß wir von einander getrennt ſein
               müſſen, wenn wir einander {\pb}am Meiſten nöthig haben.
               Schon daß ich an Dich ſchreibe, beruhigt mich ein wenig. Wie hätte es mich erſt
               beruhigt, mit Dir zu ſprechen!\pend
           
\pstart
           Grüß’ Dich Gott, liebſter Freund! Schreib’ mir umgehend, was Du machſt!\pend
           
\pstart
           In Treue {\\[\baselineskip]}Dein {\\[\baselineskip]}\spacefill\mbox{Paul Goldmann.}\pend
           \leftskip=0em{}
\pstart
           \noindent{}In Berlin\oindex{Berlin@\textbf{Berlin}, \emph{P.PPLC}|pw} ſah ich \textsc{Kerr\pwindex{Kerr, Alfred 25.12.1867 – 12.10.1948@\textsc{Kerr, Alfred} (25.12.1867 – 12.10.1948), \emph{Schriftsteller/Schriftstellerin, Kritiker/Kritikerin}|pw}}. Er hat mir diesmal ſehr gefallen; von Dir ſpricht er mit echter Wärme. Es
                  iſt ein gutes Zeichen für ihn, daß er Dich verſteht.\pend
           \selectlanguage{ngerman}\endnumbering\briefempfaengerindex{Schnitzler, Arthur@\textsc{Schnitzler, Arthur}!zzzGoldmann, Paul@\emph{von Paul Goldmann}!1899-05-201@{20. 5. {[}1899{]}}|)be}\mylabel{L02875h}  \normalsize

\doendnotes{C}
\bigskip
\vfill

\clearpage

\footnotesize

\lohead{\textsc{register}}

% Definiere theindex-Environment komplett neu ohne reledmac
\makeatletter
\renewenvironment{theindex}{%
  \section*{\indexname}%
  \setlength{\parindent}{0pt}%
  \setlength{\parskip}{0pt plus 0.3pt}%
  \let\item\@idxitem
}{%
  \clearpage
}
\makeatother

\IfFileExists{\jobname-pw.ind}{\input{\jobname-pw.ind}}{}

\end{document}

      