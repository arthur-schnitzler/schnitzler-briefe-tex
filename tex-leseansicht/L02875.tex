%% latex-leseansicht-vorspann.tex
%% Vorspann für die Leseansicht.
%% Lädt die gemeinsame Datei latex-vorspann.tex mit nicht gesetztem Schalter.

\newif\ifkorrekturansicht
\korrekturansichtfalse

\input{../tex-inputs/latex-vorspann}

\begin{center}
            \textcolor{red}{ENTWURF, NICHT FERTIG KORRIGIERT}
                      \end{center}
            
         
         \newcommand{\erwaehntePersonen}{Personen: Alfred Kerr, Theodore Rottenberg}
         \newcommand{\erwaehnteInstitutionen}{}
         \newcommand{\erwaehnteOrte}{Orte: Berlin, Den Haag, Frankfurt am Main, Frankreich, Wien}
         \newcommand{\erwaehnteWerke}{Werke: Der grüne Kakadu. Groteske in einem Akt}
               \section[ Paul Goldmann an Arthur Schnitzler, 20. 5. {[}1899{]}]{ Paul Goldmann an Arthur Schnitzler, 20. 5. {[}1899{]}}\nopagebreak\mylabel{v}\rehead{ }\begin{ledgroupsized}[t]{13cm}\normalsize\beginnumbering \toendnotes[C]{\smallbreak\pagebreak[2]} \Standort{DLA, A:Schnitzler, HS.NZ85.1.3169.}
\physDesc{Brief, 1 Blatt, 4 Seiten
\newline{}Handschrift: schwarze Tinte, deutsche Kurrent
\newline{}Schnitzler: 1) mit Bleistift das Jahr »99« vermerkt  2) mit rotem Buntstift eine Unterstreichung}\toendnotes[C]{\smallbreak}\pstart
           \raggedleft{}{\pb}\textsc{Haag\oindex{Den Haag@\textbf{Den Haag}|pw}}, 20. Mai.\pend
           \pstart{}Mein lieber Freund,\pend\pstart
           Von Berlin\oindex{Berlin@\textbf{Berlin}|pw} bin ich nach dem \textsc{Haag\oindex{Den Haag@\textbf{Den Haag}|pw}} beordert worden zur \label{K_L02875-1v}\edtext{Friedensconferenz}{\lemma{\textnormal{\emph{Friedensconferenz}}}\Cendnote{\textnormal{Die Haag\oindex{Den Haag@\textbf{Den Haag}|pwk}er Friedenskonferenz fand von 18. 5. 1899 bis 29. 7. 1899 statt.}}}\label{K_L02875-1h}. Seit zehn Tage{[}n{]} lebe
               ich in einer unbeſchreiblichen Hetzjagd, und endlich heut finde ich fünf Minuten Zeit, um Dir von Herzen für Deinen lieben
               Brief zu danken, der mir nach Berlin\oindex{Berlin@\textbf{Berlin}|pw}
               nachgeſchickt wurde. Aber wichtiger wäre es mir, zu wiſſen, wie es Dir geht? Ich
               hoffe, nächſter Tage nach Frankfurt\oindex{Frankfurt am Main@\textbf{Frankfurt am Main}|pw}
               zurückzukehren, und bitte Dich, mir \uuline{ſofort} eine Zeile
               dorthin zu ſenden, um mir zu ſagen, {\pb}wie Du Dich
               befindeſt?\pend
           \pstart
           In Berlin\oindex{Berlin@\textbf{Berlin}|pw} habe ich natürlich den »Grünen Kakadu\pwindex{Schnitzler, Arthur 15.05.1862 – 21.10.1931@\textsc{Schnitzler, Arthur} (15.05.1862 – 21.10.1931), \emph{Schriftsteller, Mediziner}!gruene Kakadu. Groteske in einem Akt1. 3. 1899@\strich\emph{Der grüne Kakadu. Groteske in einem Akt} {[}1. 3. 1899{]}|pw}« geſehen. Ich kann Dir nur offen ſagen, mit
               jenem Freimuth, der zwiſchen uns Gebot iſt: Ich habe das Stück\pwindex{Schnitzler, Arthur 15.05.1862 – 21.10.1931@\textsc{Schnitzler, Arthur} (15.05.1862 – 21.10.1931), \emph{Schriftsteller, Mediziner}!gruene Kakadu. Groteske in einem Akt1. 3. 1899@\strich\emph{Der grüne Kakadu. Groteske in einem Akt} {[}1. 3. 1899{]}|pwv} nicht ſehr lieb. Es iſt ein
               glänzendes und ein geiſtreifes Stück\pwindex{Schnitzler, Arthur 15.05.1862 – 21.10.1931@\textsc{Schnitzler, Arthur} (15.05.1862 – 21.10.1931), \emph{Schriftsteller, Mediziner}!gruene Kakadu. Groteske in einem Akt1. 3. 1899@\strich\emph{Der grüne Kakadu. Groteske in einem Akt} {[}1. 3. 1899{]}|pw}, das ſeinen
               großen Erfolg wohl verdient; aber mir fehlt etwas darin, und ich habe die Empfindung,
               daß Du weit, weit höher ſtehſt, als dieſes Stück\pwindex{Schnitzler, Arthur 15.05.1862 – 21.10.1931@\textsc{Schnitzler, Arthur} (15.05.1862 – 21.10.1931), \emph{Schriftsteller, Mediziner}!gruene Kakadu. Groteske in einem Akt1. 3. 1899@\strich\emph{Der grüne Kakadu. Groteske in einem Akt} {[}1. 3. 1899{]}|pw}. Und d\textcolor{gray}{a}nn bleibe ich dabei: die fran\oindex{Frankreich@\textbf{Frankreich}|pwv}zöſiſche Revolution iſt nicht \uline{in} dem Stück\pwindex{Schnitzler, Arthur 15.05.1862 – 21.10.1931@\textsc{Schnitzler, Arthur} (15.05.1862 – 21.10.1931), \emph{Schriftsteller, Mediziner}!gruene Kakadu. Groteske in einem Akt1. 3. 1899@\strich\emph{Der grüne Kakadu. Groteske in einem Akt} {[}1. 3. 1899{]}|pwv}, in der Stimmung, ſondern ſie wird nur \strikeout{zum Schluß} als Effekt von draußen, als Aktſchluß
               verwendet. Sei mir nicht bös, {\pb}ich habe vielleicht
               Unrecht, aber jedenfalls iſt’s meine ehrliche, wohl erwogene Meinung{\dotssix}\pend
           \pstart
           Vor meiner Abreiſe aus Frankfurt\oindex{Frankfurt am Main@\textbf{Frankfurt am Main}|pw} habe ich \label{K_L02875-11v}\edtext{etwas}{\lemma{\textnormal{\emph{etwas}}}\Cendnote{\textnormal{Eventuell wird hier auf den Beginn der intimen Beziehung mit
                  der verheirateten Theodore Rottenberg\pwindex{Rottenberg, Theodore 1875-09-07 – 1945-04-05@\textsc{Rottenberg, Theodore} (1875-09-07 – 1945-04-05)|pwk}
                  angespielt, vgl. Paul Goldmann an Arthur Schnitzler, 8. 10. [1899]}}}\label{K_L02875-11h} erlebt, das für
               jeden Menſchen den Gipfel des Glücks bedeuten würde. Für mich iſts durch meine an
               Wahnſinn grenzende Nervoſität, die in dieſem Augenblick noch durch Krankheit
               complicirt iſt, zu einer der größten ſeeliſchen Kataſtrophen ausgeſchlagen \strikeout{haben}, die ich noch durchgemacht habe. Niemals habe ich
               dem Selbſtmord ſo nahe geſtanden, – niemals auch hätte ich Deines Troſtes und Rathes
                  \strikeout{b\textcolor{gray}{e}} mehr bedurft. Aber es ſteht geſchrieben, daß wir von einander getrennt ſein
               müſſen, wenn wir einander {\pb}am Meiſten nöthig haben.
               Schon daß ich an Dich ſchreibe, beruhigt mich ein wenig. Wie hätte es mich erſt
               beruhigt, mit Dir zu ſprechen!\pend
           \pstart
           Grüß’ Dich Gott, liebſter Freund! Schreib’ mir umgehend, was Du machſt!\pend
           \pstart
           In Treue {\\[\baselineskip]}Dein {\\[\baselineskip]}\spacefill\mbox{Paul Goldmann.}\pend
           \leftskip=0em{}\pstart
           \noindent{}In Berlin\oindex{Berlin@\textbf{Berlin}|pw} ſah ich \textsc{Kerr\pwindex{Kerr, Alfred 25.12.1867 – 12.10.1948@\textsc{Kerr, Alfred} (25.12.1867 – 12.10.1948), \emph{Schriftsteller, Kritiker}|pw}}. Er hat mir diesmal ſehr gefallen; von Dir ſpricht er mit echter Wärme. Es
                  iſt ein gutes Zeichen für ihn, daß er Dich verſteht.\pend
           
         
         \endnumbering\mylabel{h}\end{ledgroupsized}  \newcommand{\dateiname}{L02875}\newcommand{\titel}{Paul Goldmann an Arthur Schnitzler, 20. 5. [1899]}\newcommand{\editorInnen}{Martin Anton Müller und Laura Untner}%% latex-leseansicht-abspann.tex
%% Abspann für die Leseansicht.
%% Der Schalter \ifkorrekturansicht ist bereits durch den Vorspann gesetzt.

%% latex-abspann.tex
%% Gemeinsamer Abspann für Korrekturansicht und Leseansicht.
%% Setzt den Schalter \ifkorrekturansicht voraus (gesetzt in den
%% einbindenden Dateien latex-korrekturansicht-abspann.tex bzw.
%% latex-leseansicht-abspann.tex).
%% ---------------------------------------------------------------

\normalsize

% Das esempio-Environment wird nur in der Leseansicht benötigt
\ifkorrekturansicht\else
\newenvironment{esempio}[3]%
{
    \vspace{1.5ex}
    \rlap{\underline{#1}}
    \par
    \setlength{\parindent}{0cm}
    \nopagebreak
    \leftskip=#2cm
    \rightskip=#3cm
}
{
    \par
}
\fi

\doendnotes{C}
\bigskip
\vfill

\clearpage

\footnotesize

\ifkorrekturansicht
  \lohead{\textsc{register}}
\fi

% theindex-Environment neu definieren ohne reledmac
\makeatletter
\renewenvironment{theindex}{%
  \ifkorrekturansicht
    \section*{\indexname}%
  \else
    \subsubsection*{Index der erwähnten Entitäten}%
  \fi
  \setlength{\parindent}{0pt}%
  \setlength{\parskip}{0pt plus 0.3pt}%
  \let\item\@idxitem
}{%
  \ifkorrekturansicht\clearpage\fi
}
\makeatother

\IfFileExists{\jobname-pw.ind}{\input{\jobname-pw.ind}}{}

% Quellenangabe nur in der Leseansicht
\ifkorrekturansicht\else
% Fallback-Definitionen, falls die .tex-Datei \titel etc. nicht gesetzt hat
\providecommand{\titel}{}
\providecommand{\editorInnen}{}
\providecommand{\dateiname}{\jobname}

\vspace{3cm}

\vfill

\footnotesize
\textsc{Quelle}: \titel. Herausgegeben von {\editorInnen}. In: \emph{Arthur Schnitzler: Briefwechsel mit Autorinnen und Autoren}.
 Digitale Edition, https://schnitzler-briefe.acdh.oeaw.ac.at/{\dateiname}.html (Stand \today)
\fi

\end{document}


      