%% latex-leseansicht-vorspann.tex
%% Vorspann für die Leseansicht.
%% Lädt die gemeinsame Datei latex-vorspann.tex mit nicht gesetztem Schalter.

\newif\ifkorrekturansicht
\korrekturansichtfalse

\input{../tex-inputs/latex-vorspann}


\section[Hugo von Hofmannsthal an Arthur Schnitzler, {[}zwischen 24. und 27. 6. 1906{]}]{L01605 Hugo von Hofmannsthal an Arthur Schnitzler, {[}zwischen 24. und 27. 6. 1906{]}}
\nopagebreak\mylabel{L01605v}
\rehead{ }\normalsize\beginnumbering\briefempfaengerindex{Schnitzler, Arthur@\textsc{Schnitzler, Arthur}!zzzHofmannsthal, Hugo von@\emph{von Hugo von Hofmannsthal}!1906-06-244@{{[}zwischen 24. und 27. 6. 1906{]}}|(be}
\toendnotes[C]{\smallbreak\pagebreak[2]}
\correspDesc{Versand  durch Hugo von Hofmannsthal am [zwischen 24. und 27. 6. 1906] in Semmering
\newline{}Erhalt  durch Arthur Schnitzler im Zeitraum [25. 6. 1906
                  – 29. 6. 1906?] in Wien}\toendnotes[C]{\smallbreak}
\Standort{CUL, Schnitzler, B 43.}
\physDesc{Brief, 1 Blatt, 2 Seiten, 451 Zeichen
\newline{}Handschrift: schwarze Tinte, deutsche Kurrent
\newline{}Schnitzler: mit Bleistift (falsch) datiert: »Ende Juni 901« 
\newline{}Ordnung: 1) mit Bleistift von unbekannter Hand nummeriert: »\strikeout{180}«  2) mit Bleistift von unbekannter Hand nummeriert:
                                    »176«}
\buchAbdrucke{\weitereDrucke{Hugo von Hofmannsthal, Arthur Schnitzler: \emph{Briefwechsel}. Herausgegeben von Therese Nickl und Heinrich Schnitzler. Frankfurt am Main: \emph{S. Fischer} 1964, S. 148.} }\toendnotes[C]{\smallbreak}
\pstart
           \centering{}{\pb}\textcolor{gray}{\textbf{Südbahn-Hôtel Semmering\oindex{Südbahnhotel [Semmering]@\textbf{Südbahnhotel [Semmering]}, \emph{Hotel}|pw}.}}\pend
           
\pstart
           \textcolor{gray}{\textbf{TELEGRAMME:}}\pend
           
\pstart
           \textcolor{gray}{\textbf{SÜDBAHNHÔTEL SEMMERING\oindex{Südbahnhotel [Semmering]@\textbf{Südbahnhotel [Semmering]}, \emph{Hotel}|pw}}}.\pend
           
\pstart
           \textcolor{gray}{\textbf{TELEPHON:}}\pend
           
\pstart
           \textcolor{gray}{\textbf{HÔTEL{ }{\dotsfour} Nr. 5.}}.\pend
           
\pstart
           \textcolor{gray}{\textbf{DEPENDANCE Nr. 6.}}\pend
           \vspace{0.5em}
\pstart
           lieber\hspace*{1.5em}bitte{ }ſchicken Sie mir \label{K_L01605-1v}\edtext{nach Rodaun\oindex{Wien@\textbf{Wien}!XXIII., Liesing@\textbf{XXIII., Liesing}!Rodaun@\textbf{Rodaun}, \emph{Region}|pw}}{\lemma{\textnormal{\emph{nach Rodaun}}}\Cendnote{\textnormal{Der Aufenthalt am Semmering\oindex{Semmering@\textbf{Semmering}, \emph{Verwaltungsgebiet}|pwk} fand vom 23. 6. 1906 bis zum 27. 6. 1906
                  statt, was die Datierung ermöglicht.}}}\label{K_L01605-1} die Selbſtbiografie\pwindex{Castelli, Ignaz Franz 6.\,3.\,1781 Wien – 5.\,2.\,1862 ebd.@\textsc{Castelli, Ignaz Franz} (6.\,3.\,1781 Wien – 5.\,2.\,1862 ebd.), \emph{Schriftsteller, Beamter}!Memoiren meines Lebens@\strich\emph{Memoiren meines Lebens}|pwv} von Caſtelli\pwindex{Castelli, Ignaz Franz 6.\,3.\,1781 Wien – 5.\,2.\,1862 ebd.@\textsc{Castelli, Ignaz Franz} (6.\,3.\,1781 Wien – 5.\,2.\,1862 ebd.), \emph{Schriftsteller, Beamter}|pw}. Ferner wenn Sie eine gute Biographie von \uline{Raimund}\pwindex{Raimund, Ferdinand 1.\,6.\,1790 Wien – 5.\,9.\,1836 Pottenstein@\textsc{Raimund, Ferdinand} (1.\,6.\,1790 Wien – 5.\,9.\,1836 Pottenstein), \emph{Schauspieler, Dramatiker}|pw} haben,{ }ſowie Briefe oder Tagebücher von Raimund\pwindex{Raimund, Ferdinand 1.\,6.\,1790 Wien – 5.\,9.\,1836 Pottenstein@\textsc{Raimund, Ferdinand} (1.\,6.\,1790 Wien – 5.\,9.\,1836 Pottenstein), \emph{Schauspieler, Dramatiker}|pw}. Ferner wenn Sie etwas dergleichen das näheres über Raimund\pwindex{Raimund, Ferdinand 1.\,6.\,1790 Wien – 5.\,9.\,1836 Pottenstein@\textsc{Raimund, Ferdinand} (1.\,6.\,1790 Wien – 5.\,9.\,1836 Pottenstein), \emph{Schauspieler, Dramatiker}|pw} enthält, nicht haben aber \uline{wiſſen},{ }ſo{ }ſchreiben Sie mir bitte den Titel gleich. Bitte{ }ſchicken Sie alles möglichſt {\pb}bald. Ich bin herzlich dankbar dafür.\pend
           
\pstart
           Den Pöhnl\pwindex{Pöhnl, Hans 1.\,1.\,1849 Wien – 1914 Znojmo@\textsc{Pöhnl, Hans} (1.\,1.\,1849 Wien – 1914 Znojmo), \emph{Schriftsteller, Regisseur, Dramaturg}|pw}\pwindex{Deutsche Volksbühnenspiele@\emph{Deutsche Volksbühnenspiele}|pwv}{ }ſchick ich per Poſt an Sie zurück.\pend
           
\pstart
           Wie lange{ }ſind Sie noch da?{\\[\baselineskip]}Ihr\spacefill\mbox{Hugo.}\pend
           \leftskip=0em{}\selectlanguage{ngerman}\endnumbering\briefempfaengerindex{Schnitzler, Arthur@\textsc{Schnitzler, Arthur}!zzzHofmannsthal, Hugo von@\emph{von Hugo von Hofmannsthal}!1906-06-244@{{[}zwischen 24. und 27. 6. 1906{]}}|)be}\mylabel{L01605h}  \newcommand{\dateiname}{L01605}\newcommand{\titel}{Hugo von Hofmannsthal an Arthur Schnitzler, [zwischen 24. und 27. 6. 1906]}\newcommand{\editorInnen}{Martin Anton Müller und Gerd-Hermann Susen}%% latex-leseansicht-abspann.tex
%% Abspann für die Leseansicht.
%% Der Schalter \ifkorrekturansicht ist bereits durch den Vorspann gesetzt.

%% latex-abspann.tex
%% Gemeinsamer Abspann für Korrekturansicht und Leseansicht.
%% Setzt den Schalter \ifkorrekturansicht voraus (gesetzt in den
%% einbindenden Dateien latex-korrekturansicht-abspann.tex bzw.
%% latex-leseansicht-abspann.tex).
%% ---------------------------------------------------------------

\normalsize

% Das esempio-Environment wird nur in der Leseansicht benötigt
\ifkorrekturansicht\else
\newenvironment{esempio}[3]%
{
    \vspace{1.5ex}
    \rlap{\underline{#1}}
    \par
    \setlength{\parindent}{0cm}
    \nopagebreak
    \leftskip=#2cm
    \rightskip=#3cm
}
{
    \par
}
\fi

\doendnotes{C}
\bigskip
\vfill

\clearpage

\footnotesize

\ifkorrekturansicht
  \lohead{\textsc{register}}
\fi

% theindex-Environment neu definieren ohne reledmac
\makeatletter
\renewenvironment{theindex}{%
  \ifkorrekturansicht
    \section*{\indexname}%
  \else
    \subsubsection*{Index der erwähnten Entitäten}%
  \fi
  \setlength{\parindent}{0pt}%
  \setlength{\parskip}{0pt plus 0.3pt}%
  \let\item\@idxitem
}{%
  \ifkorrekturansicht\clearpage\fi
}
\makeatother

\IfFileExists{\jobname-pw.ind}{\input{\jobname-pw.ind}}{}

% Quellenangabe nur in der Leseansicht
\ifkorrekturansicht\else
% Fallback-Definitionen, falls die .tex-Datei \titel etc. nicht gesetzt hat
\providecommand{\titel}{}
\providecommand{\editorInnen}{}
\providecommand{\dateiname}{\jobname}

\vspace{3cm}

\vfill

\footnotesize
\textsc{Quelle}: \titel. Herausgegeben von {\editorInnen}. In: \emph{Arthur Schnitzler: Briefwechsel mit Autorinnen und Autoren}.
 Digitale Edition, https://schnitzler-briefe.acdh.oeaw.ac.at/{\dateiname}.html (Stand \today)
\fi

\end{document}


