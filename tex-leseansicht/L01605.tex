\input{../tex-inputs/latex-pdf-vorspann}
\begin{center}
            \textcolor{red}{ENTWURF. ENTZIFFERUNG NOCH NICHT KORREKTURGELESEN}
                      \end{center}
            
               \section[Hugo von Hofmannsthal an Arthur Schnitzler, {[}zwischen 24. und 27. 6. 1906{]}]{ Hugo von Hofmannsthal an Arthur Schnitzler, {[}zwischen 24. und
               27. 6. 1906{]}}\nopagebreak\mylabel{v}\rehead{ }\begin{ledgroupsized}[t]{13cm}\normalsize\beginnumbering\briefempfaengerindex{Schnitzler, Arthur@\textsc{Schnitzler, Arthur}!zzzHofmannsthal, Hugo von@\emph{von Hugo von Hofmannsthal}!1906-06-243@{{[}zwischen 24. und 27. 6. 1906{]}}|(be} \toendnotes[C]{\smallbreak\pagebreak[2]} \Standort{CUL, Schnitzler, B 43.}
\physDesc{Brief, 1 Blatt, 2 Seiten
\newline{}Handschrift: schwarze Tinte, deutsche Kurrent
\newline{}Schnitzler: mit Bleistift (falsch) datiert: »Ende Juni 901« \newline{}Ordnung: 1) mit Bleistift von unbekannter Hand nummeriert: »\strikeout{180}« 2) mit Bleistift von unbekannter Hand nummeriert:
                                    »176«}\buchAbdrucke{\weitereDrucke{Hugo von Hofmannsthal, Arthur Schnitzler: \emph{Briefwechsel}. Hg. Therese Nickl und Heinrich Schnitzler. Frankfurt am Main: \emph{S. Fischer} 1964, S. 148.} }\toendnotes[C]{\smallbreak}\pstart
           \noindent{}\centering{}{\pb}\textcolor{gray}{\textbf{Südbahn-Hôtel Semmering\oindex{Suedbahnhotel@\textbf{Südbahnhotel}|pw}.}}\pend
           \pstart
           \noindent{}\textcolor{gray}{\textbf{TELEGRAMME:}}\pend
           \pstart
           \textcolor{gray}{\textbf{SÜDBAHNHÔTEL SEMMERING\oindex{Suedbahnhotel@\textbf{Südbahnhotel}|pw}}}. \pend
           \pstart
           \textcolor{gray}{\textbf{TELEPHON:}}\pend
           \pstart
           \textcolor{gray}{\textbf{HÔTEL{ }{\dotsfour} Nr. 5.}}. \pend
           \pstart
           \textcolor{gray}{\textbf{DEPENDANCE Nr. 6.}}\pend
           \pstart
           lieber\hspace*{1.5em}bitte ſchicken Sie mir \label{K_L01605_1v}\edtext{nach Rodaun\oindex{Rodaun@\textbf{Rodaun}|pw}}{\lemma{\textnormal{\emph{nach Rodaun}}}\Cendnote{\textnormal{Der Aufenthalt am Semmering\oindex{Semmering@\textbf{Semmering}|pwk} fand von 23.–27. 6. 1906
                  statt, was die Datierung ermöglicht.}}}\label{K_L01605_1h} die Selbſtbiografie\pwindex{Castelli, Ignaz Franz 06.03.1781 – 05.02.1862@\textsc{Castelli, Ignaz Franz} (06.03.1781 – 05.02.1862), \emph{Schriftsteller, Beamter}!Memoiren meines Lebens1861 – 1861@\strich\emph{Memoiren meines Lebens} {[}1861 – 1861{]}|pwv} von Caſtelli\pwindex{Castelli, Ignaz Franz 06.03.1781 – 05.02.1862@\textsc{Castelli, Ignaz Franz} (06.03.1781 – 05.02.1862), \emph{Schriftsteller, Beamter}|pw}. Ferner wenn Sie eine gute Biographie von \uline{Raimund}\pwindex{Raimund, Ferdinand 01.06.1790 – 05.09.1836@\textsc{Raimund, Ferdinand} (01.06.1790 – 05.09.1836), \emph{Schauspieler, Dramatiker}|pw} haben, ſowie Briefe oder Tagebücher von Raimund\pwindex{Raimund, Ferdinand 01.06.1790 – 05.09.1836@\textsc{Raimund, Ferdinand} (01.06.1790 – 05.09.1836), \emph{Schauspieler, Dramatiker}|pw}. Ferner wenn Sie etwas dergleichen das näheres über Raimund\pwindex{Raimund, Ferdinand 01.06.1790 – 05.09.1836@\textsc{Raimund, Ferdinand} (01.06.1790 – 05.09.1836), \emph{Schauspieler, Dramatiker}|pw} enthält, nicht haben aber \uline{wiſſen}, ſo ſchreiben Sie mir bitte den Titel gleich. Bitte ſchicken Sie
               alles möglichſt {\pb}bald. Ich bin
               herzlich dankbar dafür.\pend
           \pstart
           Den Pöhnl\pwindex{Poehnl, Hans 03.05.1849 – 1914@\textsc{Pöhnl, Hans} (03.05.1849 – 1914), \emph{Schriftsteller, Regisseur, Dramaturg}|pw}\pwindex{Deutsche Volksbuehnenspiele1887 – 1887@\emph{Deutsche Volksbühnenspiele} {[}1887 – 1887{]}|pwv}{ }ſchick ich per Poſt an Sie zurück.\pend
           \pstart
           Wie lange ſind Sie noch da?{\\[\baselineskip]}Ihr\spacefill\mbox{Hugo.}\pend
           \leftskip=0em{}\endnumbering\briefempfaengerindex{Schnitzler, Arthur@\textsc{Schnitzler, Arthur}!zzzHofmannsthal, Hugo von@\emph{von Hugo von Hofmannsthal}!1906-06-243@{{[}zwischen 24. und 27. 6. 1906{]}}|)be}\mylabel{h}\end{ledgroupsized}  \newcommand{\dateiname}{L01605}\newcommand{\titel}{Hugo von Hofmannsthal an Arthur Schnitzler, [zwischen 24. und 27. 6. 1906]}\newcommand{\editorInnen}{Martin Anton Müller und Gerd-Hermann Susen}\input{../tex-inputs/latex-pdf-abspann}
      