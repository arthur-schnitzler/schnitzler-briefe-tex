%% latex-korrekturansicht-vorspann.tex
%% Vorspann für die Korrekturansicht.
%% Lädt die gemeinsame Datei latex-vorspann.tex mit gesetztem Schalter.

\newif\ifkorrekturansicht
\korrekturansichttrue

\input{../tex-inputs/latex-vorspann}


\section[Hugo von Hofmannsthal an Arthur Schnitzler, {[}zwischen 24. und 27. 6. 1906{]}]{L01605 Hugo von Hofmannsthal an Arthur Schnitzler, {[}zwischen 24. und
               27. 6. 1906{]}}
\nopagebreak\mylabel{L01605v}
\rehead{ }\normalsize\beginnumbering\briefempfaengerindex{Schnitzler, Arthur@\textsc{Schnitzler, Arthur}!zzzHofmannsthal, Hugo von@\emph{von Hugo von Hofmannsthal}!1906-06-243@{{[}zwischen 24. und 27. 6. 1906{]}}|(be}
\toendnotes[C]{\smallbreak\pagebreak[2]}\Standort{CUL, Schnitzler, B 43.}
\physDesc{Brief, 1 Blatt, 2 Seiten, 451 Zeichen
\newline{}Handschrift: schwarze Tinte, deutsche Kurrent
\newline{}Schnitzler: mit Bleistift (falsch) datiert: »Ende Juni 901« 
\newline{}Ordnung: 1) mit Bleistift von unbekannter Hand nummeriert: »\strikeout{180}«  2) mit Bleistift von unbekannter Hand nummeriert:
                                    »176«}
\buchAbdrucke{\weitereDrucke{Hugo von Hofmannsthal, Arthur Schnitzler: \emph{Briefwechsel}. Frankfurt am Main: \emph{S. Fischer} 1964, S. 148.} }\toendnotes[C]{\smallbreak}
\pstart
           \centering{}{\pb}\textcolor{gray}{\textbf{Südbahn-Hôtel Semmering\oindex{Suedbahnhotel [Semmering]@\textbf{Südbahnhotel [Semmering]}, \emph{Hotel (K.HTL)}|pw}.}}\pend
           
\pstart
           \textcolor{gray}{\textbf{TELEGRAMME:}}\pend
           
\pstart
           \textcolor{gray}{\textbf{SÜDBAHNHÔTEL SEMMERING\oindex{Suedbahnhotel [Semmering]@\textbf{Südbahnhotel [Semmering]}, \emph{Hotel (K.HTL)}|pw}}}. \pend
           
\pstart
           \textcolor{gray}{\textbf{TELEPHON:}}\pend
           
\pstart
           \textcolor{gray}{\textbf{HÔTEL{ }{\dotsfour} Nr. 5.}}. \pend
           
\pstart
           \textcolor{gray}{\textbf{DEPENDANCE Nr. 6.}}\pend
           \vspace{0.5em}
\pstart
           lieber\hspace*{1.5em}bitte ſchicken Sie mir \label{K_L01605-1v}\edtext{nach Rodaun\oindex{Rodaun@\textbf{Rodaun}, \emph{A.ADM4}|pw}}{\lemma{\textnormal{\emph{nach Rodaun}}}\Cendnote{\textnormal{Der Aufenthalt am Semmering\oindex{Semmering@\textbf{Semmering}, \emph{A.ADM3}|pwk} fand vom 23. 6. 1906 bis zum 27. 6. 1906
                  statt, was die Datierung ermöglicht.}}}\label{K_L01605-1} die Selbſtbiografie\pwindex{Memoiren meines Lebens@\emph{Memoiren meines Lebens}|pwv} von Caſtelli\pwindex{Castelli, Ignaz Franz 06.03.1781 – 05.02.1862@\textsc{Castelli, Ignaz Franz} (06.03.1781 – 05.02.1862), \emph{Schriftsteller/Schriftstellerin, Beamter/Beamte}|pw}. Ferner wenn Sie eine gute Biographie von \uline{Raimund}\pwindex{Raimund, Ferdinand 01.06.1790 – 05.09.1836@\textsc{Raimund, Ferdinand} (01.06.1790 – 05.09.1836), \emph{Schauspieler/Schauspielerin, Dramatiker/Dramatikerin}|pw} haben, ſowie Briefe oder Tagebücher von Raimund\pwindex{Raimund, Ferdinand 01.06.1790 – 05.09.1836@\textsc{Raimund, Ferdinand} (01.06.1790 – 05.09.1836), \emph{Schauspieler/Schauspielerin, Dramatiker/Dramatikerin}|pw}. Ferner wenn Sie etwas dergleichen das näheres über Raimund\pwindex{Raimund, Ferdinand 01.06.1790 – 05.09.1836@\textsc{Raimund, Ferdinand} (01.06.1790 – 05.09.1836), \emph{Schauspieler/Schauspielerin, Dramatiker/Dramatikerin}|pw} enthält, nicht haben aber \uline{wiſſen}, ſo ſchreiben Sie mir bitte den Titel gleich. Bitte
               ſchicken Sie alles möglichſt {\pb}bald. Ich bin herzlich dankbar dafür.\pend
           
\pstart
           Den Pöhnl\pwindex{Poehnl, Hans 1849-01-01 – 1914@\textsc{Pöhnl, Hans} (1849-01-01 – 1914), \emph{Schriftsteller/Schriftstellerin, Regisseur/Regisseurin, Dramaturg/Dramaturgin}|pw}\pwindex{Deutsche Volksbuehnenspiele@\emph{Deutsche Volksbühnenspiele}|pwv}{ }ſchick ich per Poſt an Sie zurück.\pend
           
\pstart
           Wie lange ſind Sie noch da?{\\[\baselineskip]}Ihr\spacefill\mbox{Hugo.}\pend
           \leftskip=0em{}\selectlanguage{ngerman}\endnumbering\briefempfaengerindex{Schnitzler, Arthur@\textsc{Schnitzler, Arthur}!zzzHofmannsthal, Hugo von@\emph{von Hugo von Hofmannsthal}!1906-06-243@{{[}zwischen 24. und 27. 6. 1906{]}}|)be}\mylabel{L01605h}  \normalsize

\doendnotes{C}
\bigskip
\vfill

\clearpage

\footnotesize

\lohead{\textsc{register}}

% Definiere theindex-Environment komplett neu ohne reledmac
\makeatletter
\renewenvironment{theindex}{%
  \section*{\indexname}%
  \setlength{\parindent}{0pt}%
  \setlength{\parskip}{0pt plus 0.3pt}%
  \let\item\@idxitem
}{%
  \clearpage
}
\makeatother

\IfFileExists{\jobname-pw.ind}{\input{\jobname-pw.ind}}{}

\end{document}

      