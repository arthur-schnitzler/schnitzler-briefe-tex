%% latex-leseansicht-vorspann.tex
%% Vorspann für die Leseansicht.
%% Lädt die gemeinsame Datei latex-vorspann.tex mit nicht gesetztem Schalter.

\newif\ifkorrekturansicht
\korrekturansichtfalse

\input{../tex-inputs/latex-vorspann}


\section[Gerty Hofmannsthal an Arthur Schnitzler, 12. 8. 1931]{L02545 Gerty Hofmannsthal an Arthur Schnitzler, 12. 8. 1931}
\nopagebreak\mylabel{L02545v}
\rehead{ }\normalsize\beginnumbering\briefempfaengerindex{Schnitzler, Arthur@\textsc{Schnitzler, Arthur}!zzzHofmannsthal, Gertrude von@\emph{von Gertrude von Hofmannsthal}!1931-08-121@{12. 8. 1931}|(be}
\toendnotes[C]{\smallbreak\pagebreak[2]}
\correspDesc{Versand  durch Gerty Hofmannsthal am 12. 8. 1931 \textbf{Ort fehlend} 
\newline{}Erhalt  durch Arthur Schnitzler im Zeitraum [12. 8. 1931
                  – 16. 8. 1931?] in Wien}\toendnotes[C]{\smallbreak}
\Standort{DLA, A:Schnitzler, HS.NZ85.1.3482.}
\physDesc{Brief, maschinenschriftliche Abschrift, 1 Blatt, 1 Seite, 519 Zeichen
\newline{}Schreibmaschine}\toendnotes[C]{\smallbreak}
\pstart
           {\pb}Hofman{[}n{]}sthal.\hfill 12. 8. 31.\pend
           \vspace{0.5em}
\pstart
           Lieber Arthur, Ihre Teilnahme hat mir so wohl getan, vielen Dank! Es
               ist ein grosser Schmerz für mich, das \label{K_L02545-1v}\edtext{Kinderl\pwindex{Zimmer, Christoph 7.\,2.\,1929 Heidelberg – 26.\,7.\,1931 ebd.@\textsc{Zimmer, Christoph} (7.\,2.\,1929 Heidelberg – 26.\,7.\,1931 ebd.)|pwv}}{\lemma{\textnormal{\emph{Kinderl}}}\Cendnote{\textnormal{Ihr Enkel, Christiane Zimmers\pwindex{Zimmer, Christiane 14.\,5.\,1902 Rodaun – 5.\,1.\,1987 New York City@\textsc{Zimmer, Christiane} (14.\,5.\,1902 Rodaun – 5.\,1.\,1987 New York City)|pwk} erstes Kind Christoph\pwindex{Zimmer, Christoph 7.\,2.\,1929 Heidelberg – 26.\,7.\,1931 ebd.@\textsc{Zimmer, Christoph} (7.\,2.\,1929 Heidelberg – 26.\,7.\,1931 ebd.)|pwk}, war am 26. 7. 1929 im Alter von
                  zwei Jahren gestorben.}}}\label{K_L02545-1} zu verlieren – es war für alle meine traurigen
               Gedanken sonst der einzige glückliche Ruhepunkt zu wissen dass Christiane\pwindex{Zimmer, Christiane 14.\,5.\,1902 Rodaun – 5.\,1.\,1987 New York City@\textsc{Zimmer, Christiane} (14.\,5.\,1902 Rodaun – 5.\,1.\,1987 New York City)|pw} glücklich ist und mit viel Hoffnungen begleitet man
               so ein kleines Wesen! Seit Christiane\pwindex{Zimmer, Christiane 14.\,5.\,1902 Rodaun – 5.\,1.\,1987 New York City@\textsc{Zimmer, Christiane} (14.\,5.\,1902 Rodaun – 5.\,1.\,1987 New York City)|pw} mit dem
               lieben kl. Andreas\pwindex{Zimmer, Andreas 14.\,5.\,1930 Heidelberg – 21.\,6.\,2003@\textsc{Zimmer, Andreas} (14.\,5.\,1930 Heidelberg – 21.\,6.\,2003)|pw} da ist, bin ich viel
               ruhiger und wir versuchen gegenseitig unsern Kummer zu verstecken. Also nochmals Dank
               für Ihre Freundschaft, die so wohl tut.\pend
           \pstart \spacefill\mbox{Gerty.}\pend{}\selectlanguage{ngerman}\endnumbering\briefempfaengerindex{Schnitzler, Arthur@\textsc{Schnitzler, Arthur}!zzzHofmannsthal, Gertrude von@\emph{von Gertrude von Hofmannsthal}!1931-08-121@{12. 8. 1931}|)be}\mylabel{L02545h}  \newcommand{\dateiname}{L02545}\newcommand{\titel}{Gerty Hofmannsthal an Arthur Schnitzler, 12. 8. 1931}\newcommand{\editorInnen}{Martin Anton Müller und Gerd-Hermann Susen}%% latex-leseansicht-abspann.tex
%% Abspann für die Leseansicht.
%% Der Schalter \ifkorrekturansicht ist bereits durch den Vorspann gesetzt.

%% latex-abspann.tex
%% Gemeinsamer Abspann für Korrekturansicht und Leseansicht.
%% Setzt den Schalter \ifkorrekturansicht voraus (gesetzt in den
%% einbindenden Dateien latex-korrekturansicht-abspann.tex bzw.
%% latex-leseansicht-abspann.tex).
%% ---------------------------------------------------------------

\normalsize

% Das esempio-Environment wird nur in der Leseansicht benötigt
\ifkorrekturansicht\else
\newenvironment{esempio}[3]%
{
    \vspace{1.5ex}
    \rlap{\underline{#1}}
    \par
    \setlength{\parindent}{0cm}
    \nopagebreak
    \leftskip=#2cm
    \rightskip=#3cm
}
{
    \par
}
\fi

\doendnotes{C}
\bigskip
\vfill

\clearpage

\footnotesize

\ifkorrekturansicht
  \lohead{\textsc{register}}
\fi

% theindex-Environment neu definieren ohne reledmac
\makeatletter
\renewenvironment{theindex}{%
  \ifkorrekturansicht
    \section*{\indexname}%
  \else
    \subsubsection*{Index der erwähnten Entitäten}%
  \fi
  \setlength{\parindent}{0pt}%
  \setlength{\parskip}{0pt plus 0.3pt}%
  \let\item\@idxitem
}{%
  \ifkorrekturansicht\clearpage\fi
}
\makeatother

\IfFileExists{\jobname-pw.ind}{\input{\jobname-pw.ind}}{}

% Quellenangabe nur in der Leseansicht
\ifkorrekturansicht\else
% Fallback-Definitionen, falls die .tex-Datei \titel etc. nicht gesetzt hat
\providecommand{\titel}{}
\providecommand{\editorInnen}{}
\providecommand{\dateiname}{\jobname}

\vspace{3cm}

\vfill

\footnotesize
\textsc{Quelle}: \titel. Herausgegeben von {\editorInnen}. In: \emph{Arthur Schnitzler: Briefwechsel mit Autorinnen und Autoren}.
 Digitale Edition, https://schnitzler-briefe.acdh.oeaw.ac.at/{\dateiname}.html (Stand \today)
\fi

\end{document}


