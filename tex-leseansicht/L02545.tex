%% latex-korrekturansicht-vorspann.tex
%% Vorspann für die Korrekturansicht.
%% Lädt die gemeinsame Datei latex-vorspann.tex mit gesetztem Schalter.

\newif\ifkorrekturansicht
\korrekturansichttrue

\input{../tex-inputs/latex-vorspann}


\section[Gerty Hofmannsthal an Arthur Schnitzler, 12. 8. 1931]{L02545 Gerty Hofmannsthal an Arthur Schnitzler, 12. 8. 1931}
\nopagebreak\mylabel{L02545v}
\rehead{ }\normalsize\beginnumbering\briefempfaengerindex{Schnitzler, Arthur@\textsc{Schnitzler, Arthur}!zzzHofmannsthal, Gertrude von@\emph{von Gertrude von Hofmannsthal}!1931-08-121@{12. 8. 1931}|(be}
\toendnotes[C]{\smallbreak\pagebreak[2]}\Standort{DLA, A:Schnitzler, HS.NZ85.1.3482.}
\physDesc{Brief, maschinenschriftliche Abschrift1 Blatt, 1 Seite, 519 Zeichen
\newline{}Schreibmaschine}\toendnotes[C]{\smallbreak}
\pstart
           {\pb}Hofman{[}n{]}sthal.\hfill 12. 8. 31.\pend
           \vspace{0.5em}
\pstart
           Lieber Arthur, Ihre Teilnahme hat mir so wohl getan, vielen Dank! Es
               ist ein grosser Schmerz für mich, das \label{K_L02545-1v}\edtext{Kinderl\pwindex{Zimmer, Christoph 07.02.1929 – 26.07.1931@\textsc{Zimmer, Christoph} (07.02.1929 – 26.07.1931)|pwv}}{\lemma{\textnormal{\emph{Kinderl}}}\Cendnote{\textnormal{Ihr Enkel, Christiane Zimmers\pwindex{Zimmer, Christiane 14.05.1902 – 05.01.1987@\textsc{Zimmer, Christiane} (14.05.1902 – 05.01.1987)|pwk} erstes Kind Christoph\pwindex{Zimmer, Christoph 07.02.1929 – 26.07.1931@\textsc{Zimmer, Christoph} (07.02.1929 – 26.07.1931)|pwk}, war am 26. 7. 1929 im Alter von
                  zwei Jahren gestorben.}}}\label{K_L02545-1} zu verlieren – es war für alle meine traurigen
               Gedanken sonst der einzige glückliche Ruhepunkt zu wissen dass Christiane\pwindex{Zimmer, Christiane 14.05.1902 – 05.01.1987@\textsc{Zimmer, Christiane} (14.05.1902 – 05.01.1987)|pw} glücklich ist und mit viel Hoffnungen begleitet man
               so ein kleines Wesen! Seit Christiane\pwindex{Zimmer, Christiane 14.05.1902 – 05.01.1987@\textsc{Zimmer, Christiane} (14.05.1902 – 05.01.1987)|pw} mit dem
               lieben kl. Andreas\pwindex{Zimmer, Andreas 1930-05-14 – 2003-06-21@\textsc{Zimmer, Andreas} (1930-05-14 – 2003-06-21)|pw} da ist, bin ich viel
               ruhiger und wir versuchen gegenseitig unsern Kummer zu verstecken. Also nochmals Dank
               für Ihre Freundschaft, die so wohl tut.\pend
           \pstart \spacefill\mbox{Gerty.}\pend{}\selectlanguage{ngerman}\endnumbering\briefempfaengerindex{Schnitzler, Arthur@\textsc{Schnitzler, Arthur}!zzzHofmannsthal, Gertrude von@\emph{von Gertrude von Hofmannsthal}!1931-08-121@{12. 8. 1931}|)be}\mylabel{L02545h}  \normalsize

\doendnotes{C}
\bigskip
\vfill

\clearpage

\footnotesize

\lohead{\textsc{register}}

% Definiere theindex-Environment komplett neu ohne reledmac
\makeatletter
\renewenvironment{theindex}{%
  \section*{\indexname}%
  \setlength{\parindent}{0pt}%
  \setlength{\parskip}{0pt plus 0.3pt}%
  \let\item\@idxitem
}{%
  \clearpage
}
\makeatother

\IfFileExists{\jobname-pw.ind}{\input{\jobname-pw.ind}}{}

\end{document}

      