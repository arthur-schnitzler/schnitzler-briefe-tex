%% latex-korrekturansicht-vorspann.tex
%% Vorspann für die Korrekturansicht.
%% Lädt die gemeinsame Datei latex-vorspann.tex mit gesetztem Schalter.

\newif\ifkorrekturansicht
\korrekturansichttrue

\input{../tex-inputs/latex-vorspann}


\section[Arthur Schnitzler an Richard Dehmel, 13. 1. 1902]{L01198 Arthur Schnitzler an Richard Dehmel, 13. 1. 1902}
\nopagebreak\mylabel{L01198v}
\rehead{ }\normalsize\beginnumbering\briefempfaengerindex{Dehmel, Richard@\textsc{Dehmel, Richard}!zzzSchnitzler, Arthur@\emph{von Arthur Schnitzler}!1902-01-131@{13. 1. 1902}|(be}
\toendnotes[C]{\smallbreak\pagebreak[2]}\Standort{Hamburg, Staats- und Universitätsbibliothek, DA:Br:S:616/20.}
\physDesc{Brief, 1 Blatt, 2 Seiten, 451 Zeichen
\newline{}Handschrift: schwarze Tinte, deutsche Kurrent}
\pstart{}{\pb}Verehrteſter Herr Dehmel,\pend\vspace{0.5em}
\pstart
           ich danke verbindlichſt für Ihre freundliche Aufforderung zur Mitarbeiterſchaft am
                  »Buntſcheck\pwindex{Buntscheck. Ein Sammelbuch herzhafter Kunst fuer Ohr und Auge deutscher Kinder@\emph{Der Buntscheck. Ein Sammelbuch herzhafter Kunst für Ohr und Auge deutscher Kinder}|pw}«. Eine beſtimmte Zuſage ka{\geminationn} ich aber erſt machen, we{\geminationn}
               ich etwas für Ihr Buch geeignetes {\pb}geſchrieben haben
               werde. Sollte das bis zum September dJ. geſchehen ſein, ſo ſende ich
               Ihnen den Beitrag ſelbſtverſtändlich mit beſonderm Vergnügen zu.\pend
           
\pstart
           Mit beſondrer Hochſchätzung{\\[\baselineskip]}Ihr aufrichtg ergebner{\\[\baselineskip]}\spacefill\mbox{Arthur Schnitzler}\pend
           \leftskip=0em{}
\pstart
           Wien IX. Frankg. 1\oindex{Frankgasse 1@\textbf{Frankgasse 1}, \emph{Wohngebäude (K.WHS)}|pw}.{\\}13. 1. 902.\pend
           \selectlanguage{ngerman}\endnumbering\briefempfaengerindex{Dehmel, Richard@\textsc{Dehmel, Richard}!zzzSchnitzler, Arthur@\emph{von Arthur Schnitzler}!1902-01-131@{13. 1. 1902}|)be}\mylabel{L01198h}  \normalsize

\doendnotes{C}
\bigskip
\vfill

\clearpage

\footnotesize

\lohead{\textsc{register}}

% Definiere theindex-Environment komplett neu ohne reledmac
\makeatletter
\renewenvironment{theindex}{%
  \section*{\indexname}%
  \setlength{\parindent}{0pt}%
  \setlength{\parskip}{0pt plus 0.3pt}%
  \let\item\@idxitem
}{%
  \clearpage
}
\makeatother

\IfFileExists{\jobname-pw.ind}{\input{\jobname-pw.ind}}{}

\end{document}

      