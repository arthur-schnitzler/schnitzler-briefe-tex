%% latex-korrekturansicht-vorspann.tex
%% Vorspann für die Korrekturansicht.
%% Lädt die gemeinsame Datei latex-vorspann.tex mit gesetztem Schalter.

\newif\ifkorrekturansicht
\korrekturansichttrue

\input{../tex-inputs/latex-vorspann}


\section[ Felix Salten an Arthur Schnitzler, 18. 7. 1909]{L03503 Felix Salten an Arthur Schnitzler, 18. 7. 1909}
\nopagebreak\mylabel{L03503v}
\rehead{ }\normalsize\beginnumbering\briefempfaengerindex{Schnitzler, Arthur@\textsc{Schnitzler, Arthur}!zzzSalten, Felix@\emph{von Felix Salten}!1909-07-182@{18. 7. 1909}|(be}
\toendnotes[C]{\smallbreak\pagebreak[2]}\Standort{CUL, Schnitzler, B 89, B 1.}
\physDesc{Bildpostkarte, 402 Zeichen
\newline{}Handschrift: schwarze Tinte, lateinische Kurrent
\newline{}Versand: Stempel: »\nobreak{}\oindex{Hoehlenstein@\textbf{Höhlenstein}, \emph{P.PPLQ}|pwk}{[}L{]}an\textcolor{gray}{dr}o\nobreak{}«.  
\newline{}Schnitzler: mit Bleistift Vermerk: »\textsc{Salten}« 
\newline{}Ordnung: mit Bleistift von unbekannter Hand nummeriert: »253« }\toendnotes[C]{\smallbreak}\pstart{}{\pb}Herrn\pend{}\pstart{}D\textsuperscript{r} Arthur Schnitzler\pend{}\pstart{}Edlach \textsuperscript{b}/Reichenau\oindex{Edlach@\textbf{Edlach}, \emph{P.PPL}|pw}\pend{}\pstart{}Nied. Öst.\oindex{Niederoesterreich@\textbf{Niederösterreich}, \emph{A.ADM1}|pw}\pend{}{\bigskip}
\pstart
           \noindent{}\centering{}{\pb}\textcolor{gray}{\textbf{Dürrensee\oindex{Lago di Landro@\textbf{Lago di Landro}, \emph{H.LK}|pw} (1410 m) mit Monte Cristallo\oindex{Monte Cristallo@\textbf{Monte Cristallo}, \emph{Berg (N.BRG)}|pw} (3199 m) Ampezzo\oindex{Ampezzo@\textbf{Ampezzo}, \emph{P.PPLA3}|pw}{\dotstwo}{ }Tirol\oindex{Suedtirol@\textbf{Südtirol}, \emph{A.ADM2}|pw}.}}\pend
           \vspace{1em}
\pstart
           \noindent{}{\pb}Lieber,{ }\uline{sehr} erfreut, dass es dem Heini\pwindex{Schnitzler, Heinrich 09.08.1902 – 12.07.1982@\textsc{Schnitzler, Heinrich} (09.08.1902 – 12.07.1982), \emph{Regisseur/Regisseurin, Schauspieler/Schauspielerin}|pw} schon besser geht. Auch Annerle\pwindex{Rehmann, Anna Katharina 18.08.1904 – 27.03.1977@\textsc{Rehmann, Anna Katharina} (18.08.1904 – 27.03.1977), \emph{Schauspieler/Schauspielerin, Übersetzer/Übersetzerin}|pw} ist wieder munter, und die drohende Malaria gott sei
               dank nicht eingetroffen. Uns geht’s hier\oindex{Hoehlenstein@\textbf{Höhlenstein}, \emph{P.PPLQ}|pwv} ganz gut, die Leute stören nicht, das Hotel ist angenehm; das Wetter allein von einer kalten Freundlichkeit.
               Alles Gute Ihnen Dreien\pwindex{Schnitzler, Olga 17.01.1882 – 13.01.1970@\textsc{Schnitzler, Olga} (17.01.1882 – 13.01.1970), \emph{Schauspieler/Schauspielerin, Sänger/Sängerin}|pwv}\pwindex{Schnitzler, Heinrich 09.08.1902 – 12.07.1982@\textsc{Schnitzler, Heinrich} (09.08.1902 – 12.07.1982), \emph{Regisseur/Regisseurin, Schauspieler/Schauspielerin}|pwv}! \pend
           
\pstart
           Herzliche Grüße von uns zu Ihnen {\\[\baselineskip]}Ihr {\\[\baselineskip]}\spacefill\mbox{Salten}\pend
           \leftskip=0em{}
\pstart
           Landro\oindex{Hoehlenstein@\textbf{Höhlenstein}, \emph{P.PPLQ}|pw}, 18. VII. 09.\pend
           \selectlanguage{ngerman}\endnumbering\briefempfaengerindex{Schnitzler, Arthur@\textsc{Schnitzler, Arthur}!zzzSalten, Felix@\emph{von Felix Salten}!1909-07-182@{18. 7. 1909}|)be}\mylabel{L03503h}  \normalsize

\doendnotes{C}
\bigskip
\vfill

\clearpage

\footnotesize

\lohead{\textsc{register}}

% Definiere theindex-Environment komplett neu ohne reledmac
\makeatletter
\renewenvironment{theindex}{%
  \section*{\indexname}%
  \setlength{\parindent}{0pt}%
  \setlength{\parskip}{0pt plus 0.3pt}%
  \let\item\@idxitem
}{%
  \clearpage
}
\makeatother

\IfFileExists{\jobname-pw.ind}{\input{\jobname-pw.ind}}{}

\end{document}

      