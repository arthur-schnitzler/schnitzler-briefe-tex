\input{../tex-inputs/latex-pdf-vorspann}
\begin{center}
            \textcolor{red}{ENTWURF. ENTZIFFERUNG NOCH NICHT KORREKTURGELESEN}
                      \end{center}
            
               \section[Richard Beer-Hofmann an Arthur Schnitzler, 28. 8. 1918]{ Richard Beer-Hofmann an Arthur Schnitzler,
               28. 8. 1918}\nopagebreak\mylabel{v}\rehead{ }\begin{ledgroupsized}[t]{13cm}\normalsize\beginnumbering\briefempfaengerindex{Schnitzler, Arthur@\textsc{Schnitzler, Arthur}!zzzBeer-Hofmann, Richard@\emph{von Richard Beer-Hofmann}!1918-08-281@{28. 8. 1918}|(be} \toendnotes[C]{\smallbreak\pagebreak[2]} \Standort{CUL, Schnitzler, B 8.}
\physDesc{Postkarte
\newline{}Handschrift: Bleistift, lateinische Kurrent\newline{}Versand: Stempel: »\nobreak{}\oindex{Bad Ischl@\textbf{Bad Ischl}|pwk}Bad Ischl, 29. VIII. 18, 5\nobreak{}«.  \newline{}Ordnung: mit Bleistift von unbekannter Hand nummeriert: »267« }\buchAbdrucke{\weitereDrucke{Arthur Schnitzler, Richard Beer-Hofmann: \emph{Briefwechsel 1891–1931}. Hg. Konstanze Fliedl. Wien, Zürich: \emph{Europaverlag} 1992, S. 226.} }\toendnotes[C]{\smallbreak}\pstart{}{\pb}Herrn\pend{}\pstart{}D\textsc{r} Arthur Schnitzler\pend{}\pstart{}Partenkirchen\oindex{Partenkirchen@\textbf{Partenkirchen}|pw}\pend{}\pstart{}Haus Tannenberg\oindex{Haus Tannenberg@\textbf{Haus Tannenberg}|pw}\pend{}{\bigskip}\pstart
           \raggedleft{}{\pb}Bad-Ischl\oindex{Bad Ischl@\textbf{Bad Ischl}|pw}{ }28. VIII. 18.\pend
           \pstart
           Lieber Arthur! Schade, dass Sie nicht nach Salzburg\oindex{Salzburg@\textbf{Salzburg}|pw} kamen. Über meinen Aufführungstermin wurde erst –
               nachdem wir 10 Tage beisa{\geminationm}en waren, gesprochen, da ich
               nicht fragte. Fest steht \strikeout{erst} nur \introOben{}(wenn es fest steht!\introOben{}): Als erstes: »Wie es
                  Euch gefällt\pwindex{\textcolor{red}{\textsuperscript{XXXX1 indx}}!Wie es euch gefaellt1599@\strich\emph{Wie es euch gefällt} {[}1599{]}|pw}«. Als zweites »Jaakobs
               Traum\pwindex{Beer-Hofmann, Richard 11.07.1866 – 26.09.1945@\textsc{Beer-Hofmann, Richard} (11.07.1866 – 26.09.1945), \emph{Schriftsteller}!Jaákobs Traum. Ein Vorspiel1918-04-05@\strich\emph{Jaákobs Traum. Ein Vorspiel} {[}1918-04-05{]}|pw}«. Alles andere noch unbesti{\geminationm}t.
               Wann wollen Sie wieder in Wien\oindex{Wien@\textbf{Wien}|pw} sein? Ich dürfte {\pb}16. od. 17 Sept. kommen. Herzliche Grüsse Ihnen und Ihrer
                  Frau\pwindex{Schnitzler, Olga 17.01.1882 – 13.01.1970@\textsc{Schnitzler, Olga} (17.01.1882 – 13.01.1970), \emph{Schauspielerin, Sängerin}|pwv}, und auch Ihrer Schwägerin\pwindex{Steinrueck, Elisabeth 19.11.1885 – 07.04.1920@\textsc{Steinrück, Elisabeth} (19.11.1885 – 07.04.1920)|pwv} und Steinrück\pwindex{Steinrueck, Albert 20.05.1872 – 11.02.1929@\textsc{Steinrück, Albert} (20.05.1872 – 11.02.1929), \emph{Schauspieler}|pw}.\pend
           \pstart
           Ihr{\\[\baselineskip]}\spacefill\mbox{Richard}\pend
           \leftskip=0em{}\endnumbering\briefempfaengerindex{Schnitzler, Arthur@\textsc{Schnitzler, Arthur}!zzzBeer-Hofmann, Richard@\emph{von Richard Beer-Hofmann}!1918-08-281@{28. 8. 1918}|)be}\mylabel{h}\end{ledgroupsized}  \newcommand{\dateiname}{L02302}\newcommand{\titel}{Richard Beer-Hofmann an Arthur Schnitzler, 28. 8. 1918}\newcommand{\editorInnen}{Martin Anton Müller und Gerd-Hermann Susen}\input{../tex-inputs/latex-pdf-abspann}
      