%% latex-leseansicht-vorspann.tex
%% Vorspann für die Leseansicht.
%% Lädt die gemeinsame Datei latex-vorspann.tex mit nicht gesetztem Schalter.

\newif\ifkorrekturansicht
\korrekturansichtfalse

\input{../tex-inputs/latex-vorspann}


         
         \renewcommand{\erwaehntePersonen}{Personen: Olga Schnitzler, Elisabeth Steinrück, Albert Steinrück}
         \renewcommand{\erwaehnteOrte}{Orte: Bad Ischl, Garmisch-Partenkirchen, Haus Tannenberg, Partenkirchen, Salzburg, Wien}
         \renewcommand{\erwaehnteWerke}{Werke: Jaákobs Traum. Ein Vorspiel, Wie es euch gefällt}
               \section[Richard Beer-Hofmann an Arthur Schnitzler, 28. 8. 1918]{ Richard Beer-Hofmann an Arthur Schnitzler, 28. 8. 1918}\nopagebreak\mylabel{v}\rehead{ }\begin{ledgroupsized}[t]{13cm}\normalsize\beginnumbering \toendnotes[C]{\smallbreak\pagebreak[2]} \Standort{CUL, Schnitzler, B 8.}
\physDesc{Postkarte, 528 Zeichen
\newline{}Handschrift: Bleistift, lateinische Kurrent
\newline{}Versand: Stempel: »\nobreak{}\oindex{Bad Ischl@\textbf{Bad Ischl}|pwk}Bad Ischl, 29. VIII. 18, 5\nobreak{}«.  
\newline{}Ordnung: mit Bleistift von unbekannter Hand nummeriert:
                                    »267« }\buchAbdrucke{\weitereDrucke{Arthur Schnitzler, Richard Beer-Hofmann: \emph{Briefwechsel 1891–1931}. Hg. Konstanze Fliedl. Wien, Zürich: \emph{Europaverlag} 1992, S. 226.} }\toendnotes[C]{\smallbreak}\pstart{}{\pb}Herrn\pend{}\pstart{}D\textsc{r} Arthur Schnitzler\pend{}\pstart{}Partenkirchen\oindex{Partenkirchen@\textbf{Partenkirchen}|pw}\pend{}\pstart{}Haus Tannenberg\oindex{Haus Tannenberg@\textbf{Haus Tannenberg}|pw}\pend{}{\bigskip}\pstart
           \raggedleft{}{\pb}Bad-Ischl\oindex{Bad Ischl@\textbf{Bad Ischl}|pw}{ }28. VIII. 18.\pend
           \pstart
           Lieber Arthur! Schade, dass Sie nicht nach Salzburg\oindex{Salzburg@\textbf{Salzburg}|pw} kamen. Über meinen Aufführungstermin wurde erst –
               nachdem wir 10 Tage beisa{\geminationm}en waren, gesprochen, da ich
               nicht fragte. Fest steht \strikeout{erst} nur \introOben{}(wenn es fest steht!\introOben{}): Als erstes: »Wie es
                  Euch gefällt\pwindex{\textcolor{red}{\textsuperscript{XXXX1 indx}}!Wie es euch gefaellt1599@\strich\emph{Wie es euch gefällt} {[}1599{]}|pw}«. Als zweites »Jaakobs
                  Traum\pwindex{Beer-Hofmann, Richard 1866-07-11 – 1945-09-26@\textsc{Beer-Hofmann, Richard} (1866-07-11 – 1945-09-26), \emph{Schriftsteller}!Jaákobs Traum. Ein Vorspiel1918-04-05@\strich\emph{Jaákobs Traum. Ein Vorspiel} {[}1918-04-05{]}|pw}«. Alles andere noch unbesti{\geminationm}t. Wann wollen Sie wieder in Wien\oindex{Wien@\textbf{Wien}|pw} sein? Ich
               dürfte {\pb}16. od. 17 Sept. kommen. Herzliche Grüsse Ihnen und Ihrer
                  Frau\pwindex{Schnitzler, Olga 17.01.1882 – 13.01.1970@\textsc{Schnitzler, Olga} (17.01.1882 – 13.01.1970), \emph{Schauspielerin, Sängerin}|pwv}, und auch Ihrer Schwägerin\pwindex{Steinrueck, Elisabeth 19.11.1885 – 07.04.1920@\textsc{Steinrück, Elisabeth} (19.11.1885 – 07.04.1920)|pwv} und Steinrück\pwindex{Steinrueck, Albert 20.05.1872 – 11.02.1929@\textsc{Steinrück, Albert} (20.05.1872 – 11.02.1929), \emph{Schauspieler}|pw}.\pend
           \pstart
           Ihr{\\[\baselineskip]}\spacefill\mbox{Richard}\pend
           \leftskip=0em{}
         
         \endnumbering\mylabel{h}\end{ledgroupsized}  \newcommand{\dateiname}{L02302}\newcommand{\titel}{Richard Beer-Hofmann an Arthur Schnitzler, 28. 8. 1918}\newcommand{\editorInnen}{Martin Anton Müller und Gerd-Hermann Susen}%% latex-leseansicht-abspann.tex
%% Abspann für die Leseansicht.
%% Der Schalter \ifkorrekturansicht ist bereits durch den Vorspann gesetzt.

%% latex-abspann.tex
%% Gemeinsamer Abspann für Korrekturansicht und Leseansicht.
%% Setzt den Schalter \ifkorrekturansicht voraus (gesetzt in den
%% einbindenden Dateien latex-korrekturansicht-abspann.tex bzw.
%% latex-leseansicht-abspann.tex).
%% ---------------------------------------------------------------

\normalsize

% Das esempio-Environment wird nur in der Leseansicht benötigt
\ifkorrekturansicht\else
\newenvironment{esempio}[3]%
{
    \vspace{1.5ex}
    \rlap{\underline{#1}}
    \par
    \setlength{\parindent}{0cm}
    \nopagebreak
    \leftskip=#2cm
    \rightskip=#3cm
}
{
    \par
}
\fi

\doendnotes{C}
\bigskip
\vfill

\clearpage

\footnotesize

\ifkorrekturansicht
  \lohead{\textsc{register}}
\fi

% theindex-Environment neu definieren ohne reledmac
\makeatletter
\renewenvironment{theindex}{%
  \ifkorrekturansicht
    \section*{\indexname}%
  \else
    \subsubsection*{Index der erwähnten Entitäten}%
  \fi
  \setlength{\parindent}{0pt}%
  \setlength{\parskip}{0pt plus 0.3pt}%
  \let\item\@idxitem
}{%
  \ifkorrekturansicht\clearpage\fi
}
\makeatother

\IfFileExists{\jobname-pw.ind}{\input{\jobname-pw.ind}}{}

% Quellenangabe nur in der Leseansicht
\ifkorrekturansicht\else
% Fallback-Definitionen, falls die .tex-Datei \titel etc. nicht gesetzt hat
\providecommand{\titel}{}
\providecommand{\editorInnen}{}
\providecommand{\dateiname}{\jobname}

\vspace{3cm}

\vfill

\footnotesize
\textsc{Quelle}: \titel. Herausgegeben von {\editorInnen}. In: \emph{Arthur Schnitzler: Briefwechsel mit Autorinnen und Autoren}.
 Digitale Edition, https://schnitzler-briefe.acdh.oeaw.ac.at/{\dateiname}.html (Stand \today)
\fi

\end{document}


      