%% latex-korrekturansicht-vorspann.tex
%% Vorspann für die Korrekturansicht.
%% Lädt die gemeinsame Datei latex-vorspann.tex mit gesetztem Schalter.

\newif\ifkorrekturansicht
\korrekturansichttrue

\input{../tex-inputs/latex-vorspann}


\section[Arthur Schnitzler an Hermann Bahr, 2{[}5{]}. 4. 1913]{L02132 Arthur Schnitzler an Hermann Bahr, 2{[}5{]}. 4. 1913}
\nopagebreak\mylabel{L02132v}
\rehead{ }\normalsize\beginnumbering\briefempfaengerindex{Bahr, Hermann@\textsc{Bahr, Hermann}!zzzSchnitzler, Arthur@\emph{von Arthur Schnitzler}!1913-04-252@{2{[}5{]}. 4. 1913}|(be}
\toendnotes[C]{\smallbreak\pagebreak[2]}\Standort{TMW, HS AM 60139 Ba.}
\physDesc{Briefkarte, 338 Zeichen
\newline{}Handschrift: schwarze Tinte, deutsche Kurrent
\newline{}Ordnung: Lochung }
\buchAbdrucke{\weitereDrucke{1) Arthur Schnitzler: \emph{The Letters of Arthur Schnitzler to Hermann Bahr}. Chapel Hill: \emph{The University of North Carolina Press} 1978, S. 112.} \weitereDrucke{2) Hermann Bahr, Arthur Schnitzler: \emph{Briefwechsel, Aufzeichnungen, Dokumente (1891–1931)}. Göttingen: \emph{Wallstein} 2018, S. 485.} }\toendnotes[C]{\smallbreak}
\pstart
           {\pb}\textcolor{gray}{\textbf{Dr. Arthur Schnitzler}}\hfill 2\substVorne{}\textsuperscript{\textcolor{gray}{4}}\substDazwischen{}\textcolor{gray}{5}\substHinten{}/4 913.\pend
           
\pstart
           \textcolor{gray}{\textbf{Wien XVIII. Sternwartestrasse 71\oindex{Sternwartestrasse 71@\textbf{Sternwartestraße 71}, \emph{Wohngebäude (K.WHS)}|pw}}}\pend
           
\pstart{}lieber Hermann, \pend\vspace{0.5em}
\pstart
           für heute nur die Mittheilg, daſs \label{K_L02132-1v}\edtext{\textsc{P. A.}\pwindex{Altenberg, Peter 09.03.1859 – 08.01.1919@\textsc{Altenberg, Peter} (09.03.1859 – 08.01.1919), \emph{Schriftsteller/Schriftstellerin}|pw} Montag mit ſeinem Bruder\pwindex{Englaender, Georg 03.04.1862 – 10.04.1927@\textsc{Engländer, Georg} (03.04.1862 – 10.04.1927), \emph{Privatbeamter/Privatbeamtin}|pwv} auf den Se{\geminationm}ering\oindex{Semmering@\textbf{Semmering}, \emph{A.ADM3}|pw}, zuerſt zu \textsc{Hansy\pwindex{Hansy, Franz 23.07.1865 – 25.05.1944@\textsc{Hansy, Franz} (23.07.1865 – 25.05.1944), \emph{Mediziner/Medizinerin}|pw}}, hinauffährt}{\lemma{\textnormal{\emph{P. A. … hinauffährt}}}\Cendnote{\textnormal{Schnitzler hatte das
                     Kurhaus\oindex{Kurhaus Semmering@\textbf{Kurhaus Semmering}, \emph{Hotel (K.HTL)}|pwkv} von Dr. Franz Hansy\pwindex{Hansy, Franz 23.07.1865 – 25.05.1944@\textsc{Hansy, Franz} (23.07.1865 – 25.05.1944), \emph{Mediziner/Medizinerin}|pwk} vorgeschlagen (vgl. Arthur Schnitzler an Peter Altenberg, 22. 4. 1913); Georg Engländer\pwindex{Englaender, Georg 03.04.1862 – 10.04.1927@\textsc{Engländer, Georg} (03.04.1862 – 10.04.1927), \emph{Privatbeamter/Privatbeamtin}|pwk} schrieb Schnitzler am 25. 4. 1913, dass das umgesetzt werde.}}}\label{K_L02132-1}.\pend
           
\pstart
           Für deinen Brief herzlichen Dank. Wa{\geminationn}{\pb}wir nach Salzburg\oindex{Salzburg@\textbf{Salzburg}, \emph{A.ADM2}|pw} kommen, weiſs ich noch nicht, aber
               hoffentlich noch in dieſem Jahr. Zu welcher Zeit ſeid Ihr
               \textcolor{gray}{dort}?\pend
           
\pstart
           Auf Wiederſehen, u alles gute von Haus zu Haus.{\\[\baselineskip]}Dein{\\[\baselineskip]}\spacefill\mbox{Arthur}\pend
           \leftskip=0em{}\selectlanguage{ngerman}\endnumbering\briefempfaengerindex{Bahr, Hermann@\textsc{Bahr, Hermann}!zzzSchnitzler, Arthur@\emph{von Arthur Schnitzler}!1913-04-252@{2{[}5{]}. 4. 1913}|)be}\mylabel{L02132h}  \normalsize

\doendnotes{C}
\bigskip
\vfill

\clearpage

\footnotesize

\lohead{\textsc{register}}

% Definiere theindex-Environment komplett neu ohne reledmac
\makeatletter
\renewenvironment{theindex}{%
  \section*{\indexname}%
  \setlength{\parindent}{0pt}%
  \setlength{\parskip}{0pt plus 0.3pt}%
  \let\item\@idxitem
}{%
  \clearpage
}
\makeatother

\IfFileExists{\jobname-pw.ind}{\input{\jobname-pw.ind}}{}

\end{document}

      