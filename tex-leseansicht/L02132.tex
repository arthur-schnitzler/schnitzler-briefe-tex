%% latex-leseansicht-vorspann.tex
%% Vorspann für die Leseansicht.
%% Lädt die gemeinsame Datei latex-vorspann.tex mit nicht gesetztem Schalter.

\newif\ifkorrekturansicht
\korrekturansichtfalse

\input{../tex-inputs/latex-vorspann}


               \section[Arthur Schnitzler an Hermann Bahr, 2{[}5{]}. 4. 1913]{ Arthur Schnitzler an Hermann Bahr, 2{[}5{]}. 4. 1913}\nopagebreak\mylabel{v}\rehead{ }\begin{ledgroupsized}[t]{13cm}\normalsize\beginnumbering\briefempfaengerindex{Bahr, Hermann@\textsc{Bahr, Hermann}!zzzSchnitzler, Arthur@\emph{von Arthur Schnitzler}!1913-04-252@{2{[}5{]}. 4. 1913}|(be} \toendnotes[C]{\smallbreak\pagebreak[2]} \Standort{TMW, HS AM 60139 Ba.}
\physDesc{Briefkarte
\newline{}Handschrift: schwarze Tinte, deutsche Kurrent\newline{}Ordnung: Lochung }\buchAbdrucke{\weitereDrucke{1) \emph{25. 4. 1913, Abschrift.} In: Arthur Schnitzler: \emph{The Letters of Arthur Schnitzler to Hermann Bahr}. Edited, annotated, and with an introduction, by Donald G.
                        Daviau. Chapel Hill: \emph{The University of North Carolina Press} 1978, S. 112 (University of North Carolina studies in the Germanic languages
                        and literatures, 89).} \weitereDrucke{2) Hermann Bahr, Arthur Schnitzler: \emph{Briefwechsel, Aufzeichnungen, Dokumente (1891–1931)}. Hg. Kurt Ifkovits und Martin Anton Müller. Göttingen: \emph{Wallstein} 2018, S. 485.} }\toendnotes[C]{\smallbreak}\pstart
           \noindent{}{\pb}\textcolor{gray}{\textbf{Dr. Arthur Schnitzler}}\hfill 2\substVorne{}\textsuperscript{\textcolor{gray}{4}}\substDazwischen{}\textcolor{gray}{5}\substHinten{}/4 913.\pend
           \pstart
           \textcolor{gray}{\textbf{Wien XVIII. Sternwartestrasse 71\oindex{Sternwartestrasse@\textbf{Sternwartestraße}|pw}}}\pend
           \pstart{}lieber Hermann, \pend\pstart
           für heute nur die Mittheilg, daſs \label{K_L02132_1v}\edtext{\textsc{P. A.}\pwindex{Altenberg, Peter 09.03.1859 – 08.01.1919@\textsc{Altenberg, Peter} (09.03.1859 – 08.01.1919), \emph{Schriftsteller}|pw} Montag mit ſeinem Bruder\pwindex{Englaender, Georg 03.04.1862 – 10.04.1927@\textsc{Engländer, Georg} (03.04.1862 – 10.04.1927), \emph{Privatbeamter}|pwv}
               auf den Se{\geminationm}ering\oindex{Semmering@\textbf{Semmering}|pw},
               zuerſt zu \textsc{Hansy\pwindex{Hansy, Franz 23.07.1865 – 25.05.1944@\textsc{Hansy, Franz} (23.07.1865 – 25.05.1944), \emph{Mediziner}|pw}}, hinauffährt}{\lemma{\textnormal{\emph{P. A. … hinauffährt}}}\Cendnote{\textnormal{Schnitzler hatte das
                     Kurhaus\oindex{Kurhaus Semmering@\textbf{Kurhaus Semmering}|pwkv} von Dr. Franz Hansy\pwindex{Hansy, Franz 23.07.1865 – 25.05.1944@\textsc{Hansy, Franz} (23.07.1865 – 25.05.1944), \emph{Mediziner}|pwk} vorgeschlagen (vgl. Arthur Schnitzler an Peter Altenberg, 22. 4. 1913); Georg
                     Engländer\pwindex{Englaender, Georg 03.04.1862 – 10.04.1927@\textsc{Engländer, Georg} (03.04.1862 – 10.04.1927), \emph{Privatbeamter}|pwk} schrieb Schnitzler\pwindex{Schnitzler, Arthur 15.05.1862 – 21.10.1931@\textsc{Schnitzler, Arthur} (15.05.1862 – 21.10.1931), \emph{Schriftsteller, Mediziner}|pwk} am 25. 4. 1913, dass das umgesetzt
                  werde.}}}\label{K_L02132_1h}.\pend
           \pstart
           Für deinen Brief herzlichen Dank. Wa{\geminationn}{\pb}wir nach Salzburg\oindex{Salzburg@\textbf{Salzburg}|pw} kommen, weiſs ich noch nicht, aber hoffentlich noch in
               dieſem Jahr. Zu welcher Zeit ſeid Ihr \textcolor{gray}{dort}?\pend
           \pstart
           Auf Wiederſehen, u alles gute von Haus zu Haus.{\\[\baselineskip]}Dein{\\[\baselineskip]}\spacefill\mbox{Arthur}\pend
           \leftskip=0em{}\endnumbering\briefempfaengerindex{Bahr, Hermann@\textsc{Bahr, Hermann}!zzzSchnitzler, Arthur@\emph{von Arthur Schnitzler}!1913-04-252@{2{[}5{]}. 4. 1913}|)be}\mylabel{h}\end{ledgroupsized}  \newcommand{\dateiname}{L02132}\newcommand{\titel}{Arthur Schnitzler an Hermann Bahr, 2[5]. 4. 1913}\newcommand{\editorInnen}{ Kurt Ifkovits,  Martin Anton Müller}
            \footnotesize
\begin{ledgroupsized}[t]{11.5cm}
\doendnotes{C}
\end{ledgroupsized}
         %% latex-leseansicht-abspann.tex
%% Abspann für die Leseansicht.
%% Der Schalter \ifkorrekturansicht ist bereits durch den Vorspann gesetzt.

%% latex-abspann.tex
%% Gemeinsamer Abspann für Korrekturansicht und Leseansicht.
%% Setzt den Schalter \ifkorrekturansicht voraus (gesetzt in den
%% einbindenden Dateien latex-korrekturansicht-abspann.tex bzw.
%% latex-leseansicht-abspann.tex).
%% ---------------------------------------------------------------

\normalsize

% Das esempio-Environment wird nur in der Leseansicht benötigt
\ifkorrekturansicht\else
\newenvironment{esempio}[3]%
{
    \vspace{1.5ex}
    \rlap{\underline{#1}}
    \par
    \setlength{\parindent}{0cm}
    \nopagebreak
    \leftskip=#2cm
    \rightskip=#3cm
}
{
    \par
}
\fi

\doendnotes{C}
\bigskip
\vfill

\clearpage

\footnotesize

\ifkorrekturansicht
  \lohead{\textsc{register}}
\fi

% theindex-Environment neu definieren ohne reledmac
\makeatletter
\renewenvironment{theindex}{%
  \ifkorrekturansicht
    \section*{\indexname}%
  \else
    \subsubsection*{Index der erwähnten Entitäten}%
  \fi
  \setlength{\parindent}{0pt}%
  \setlength{\parskip}{0pt plus 0.3pt}%
  \let\item\@idxitem
}{%
  \ifkorrekturansicht\clearpage\fi
}
\makeatother

\IfFileExists{\jobname-pw.ind}{\input{\jobname-pw.ind}}{}

% Quellenangabe nur in der Leseansicht
\ifkorrekturansicht\else
% Fallback-Definitionen, falls die .tex-Datei \titel etc. nicht gesetzt hat
\providecommand{\titel}{}
\providecommand{\editorInnen}{}
\providecommand{\dateiname}{\jobname}

\vspace{3cm}

\vfill

\footnotesize
\textsc{Quelle}: \titel. Herausgegeben von {\editorInnen}. In: \emph{Arthur Schnitzler: Briefwechsel mit Autorinnen und Autoren}.
 Digitale Edition, https://schnitzler-briefe.acdh.oeaw.ac.at/{\dateiname}.html (Stand \today)
\fi

\end{document}


      