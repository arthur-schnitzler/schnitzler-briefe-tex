%% latex-leseansicht-vorspann.tex
%% Vorspann für die Leseansicht.
%% Lädt die gemeinsame Datei latex-vorspann.tex mit nicht gesetztem Schalter.

\newif\ifkorrekturansicht
\korrekturansichtfalse

\input{../tex-inputs/latex-vorspann}


\section[Arthur Schnitzler an Hermann Bahr, 2[5]. 4. 1913]{L02132 Arthur Schnitzler an Hermann Bahr, 2[5]. 4. 1913}
\nopagebreak\mylabel{L02132v}
\rehead{ }\normalsize\beginnumbering\briefempfaengerindex{Bahr, Hermann@\textsc{Bahr, Hermann}!zzzSchnitzler, Arthur@\emph{von Arthur Schnitzler}!1913-04-252@{2[5]. 4. 1913}|(be}
\toendnotes[C]{\smallbreak\pagebreak[2]}
\correspDesc{Versand  durch Arthur Schnitzler am 2[5]. 4. 1913 in Wien
\newline{}Erhalt  durch Hermann Bahr im Zeitraum [25. 4. 1913
                  – 29. 4. 1913?] \textbf{Ort fehlend} }\toendnotes[C]{\smallbreak}
\Standort{TMW, HS AM 60139 Ba.}
\physDesc{Briefkarte, 338 Zeichen
\newline{}Handschrift: schwarze Tinte, deutsche Kurrent
\newline{}Ordnung: Lochung }
\buchAbdrucke{\weitereDrucke{1) \emph{25. 4. 1913, Abschrift.} In: Arthur Schnitzler: \emph{The Letters of Arthur Schnitzler to Hermann Bahr}. Edited, annotated, and with an introduction, by Donald G. Daviau. Chapel Hill: \emph{The University of North Carolina Press} 1978, S. 112 (University of North Carolina studies in the Germanic languages
                        and literatures, 89).} \weitereDrucke{2) Hermann Bahr, Arthur Schnitzler: \emph{Briefwechsel, Aufzeichnungen, Dokumente (1891–1931)}. Herausgegeben von Kurt Ifkovits und Martin Anton Müller. Göttingen: \emph{Wallstein} 2018, S. 485.} }\toendnotes[C]{\smallbreak}
\pstart
           {\pb}\textcolor{gray}{\textbf{Dr. Arthur Schnitzler}}\hfill 2\substVorne{}\textsuperscript{\textcolor{gray}{4}}\substDazwischen{}\textcolor{gray}{5}\substHinten{}/4 913.\pend
           
\pstart
           \textcolor{gray}{\textbf{Wien XVIII. Sternwartestrasse 71\oindex{Wien@\textbf{Wien}!XVIII., Währing@\textbf{XVIII., Währing}!Sternwartestraße 71@\textbf{Sternwartestraße 71}, \emph{Wohngebäude}|pw}}}\pend
           
\pstart{}lieber Hermann,\pend\vspace{0.5em}
\pstart
           für heute nur die Mittheilg, daſs \label{K_L02132-1v}\edtext{\textsc{P. A.}\pwindex{Altenberg, Peter 9.\,3.\,1859 Wien – 8.\,1.\,1919 ebd.@\textsc{Altenberg, Peter} (9.\,3.\,1859 Wien – 8.\,1.\,1919 ebd.), \emph{Schriftsteller}|pw} Montag mit{ }ſeinem Bruder\pwindex{Engländer, Georg 3.\,4.\,1862 Wien – 10.\,4.\,1927 ebd.@\textsc{Engländer, Georg} (3.\,4.\,1862 Wien – 10.\,4.\,1927 ebd.), \emph{Privatbeamter}|pwv} auf den Se{\geminationm}ering\oindex{Semmering@\textbf{Semmering}, \emph{Verwaltungsgebiet}|pw}, zuerſt zu \textsc{Hansy\pwindex{Hansy, Franz 23.\,7.\,1865 Baden bei Wien – 25.\,5.\,1944 Wien@\textsc{Hansy, Franz} (23.\,7.\,1865 Baden bei Wien – 25.\,5.\,1944 Wien), \emph{Mediziner}|pw}}, hinauffährt}{\lemma{\textnormal{\emph{P. A. … hinauffährt}}}\Cendnote{\textnormal{Schnitzler hatte das
                     Kurhaus\oindex{Kurhaus Semmering@\textbf{Kurhaus Semmering}, \emph{Hotel}|pwkv} von Dr. Franz Hansy\pwindex{Hansy, Franz 23.\,7.\,1865 Baden bei Wien – 25.\,5.\,1944 Wien@\textsc{Hansy, Franz} (23.\,7.\,1865 Baden bei Wien – 25.\,5.\,1944 Wien), \emph{Mediziner}|pwk} vorgeschlagen (vgl. XXXX Auszeichnungsfehler: Dokument L02128 nicht gefunden); Georg Engländer\pwindex{Engländer, Georg 3.\,4.\,1862 Wien – 10.\,4.\,1927 ebd.@\textsc{Engländer, Georg} (3.\,4.\,1862 Wien – 10.\,4.\,1927 ebd.), \emph{Privatbeamter}|pwk} schrieb Schnitzler am XXXX Auszeichnungsfehler: Dokument L02131 nicht gefunden, dass das umgesetzt werde.}}}\label{K_L02132-1}.\pend
           
\pstart
           Für deinen Brief herzlichen Dank. Wa{\geminationn}{ }{\pb}wir nach Salzburg\oindex{Salzburg@\textbf{Salzburg}, \emph{Verwaltungsgebiet}|pw} kommen, weiſs ich noch nicht, aber
               hoffentlich noch in dieſem Jahr. Zu welcher Zeit{ }ſeid Ihr
               \textcolor{gray}{dort}?\pend
           
\pstart
           Auf Wiederſehen, u alles gute von Haus zu Haus.{\\[\baselineskip]}Dein{\\[\baselineskip]}\spacefill\mbox{Arthur}\pend
           \leftskip=0em{}\selectlanguage{ngerman}\endnumbering\briefempfaengerindex{Bahr, Hermann@\textsc{Bahr, Hermann}!zzzSchnitzler, Arthur@\emph{von Arthur Schnitzler}!1913-04-252@{2[5]. 4. 1913}|)be}\mylabel{L02132h}  \newcommand{\dateiname}{L02132}\newcommand{\titel}{Arthur Schnitzler an Hermann Bahr, 2[5]. 4. 1913}\newcommand{\editorInnen}{Herausgegeben von Martin Anton Müller}%% latex-leseansicht-abspann.tex
%% Abspann für die Leseansicht.
%% Der Schalter \ifkorrekturansicht ist bereits durch den Vorspann gesetzt.

%% latex-abspann.tex
%% Gemeinsamer Abspann für Korrekturansicht und Leseansicht.
%% Setzt den Schalter \ifkorrekturansicht voraus (gesetzt in den
%% einbindenden Dateien latex-korrekturansicht-abspann.tex bzw.
%% latex-leseansicht-abspann.tex).
%% ---------------------------------------------------------------

\normalsize

% Das esempio-Environment wird nur in der Leseansicht benötigt
\ifkorrekturansicht\else
\newenvironment{esempio}[3]%
{
    \vspace{1.5ex}
    \rlap{\underline{#1}}
    \par
    \setlength{\parindent}{0cm}
    \nopagebreak
    \leftskip=#2cm
    \rightskip=#3cm
}
{
    \par
}
\fi

\doendnotes{C}
\bigskip
\vfill

\clearpage

\footnotesize

\ifkorrekturansicht
  \lohead{\textsc{register}}
\fi

% theindex-Environment neu definieren ohne reledmac
\makeatletter
\renewenvironment{theindex}{%
  \ifkorrekturansicht
    \section*{\indexname}%
  \else
    \subsubsection*{Index der erwähnten Entitäten}%
  \fi
  \setlength{\parindent}{0pt}%
  \setlength{\parskip}{0pt plus 0.3pt}%
  \let\item\@idxitem
}{%
  \ifkorrekturansicht\clearpage\fi
}
\makeatother

\IfFileExists{\jobname-pw.ind}{\input{\jobname-pw.ind}}{}

% Quellenangabe nur in der Leseansicht
\ifkorrekturansicht\else
% Fallback-Definitionen, falls die .tex-Datei \titel etc. nicht gesetzt hat
\providecommand{\titel}{}
\providecommand{\editorInnen}{}
\providecommand{\dateiname}{\jobname}

\vspace{3cm}

\vfill

\footnotesize
\textsc{Quelle}: \titel. Herausgegeben von {\editorInnen}. In: \emph{Arthur Schnitzler: Briefwechsel mit Autorinnen und Autoren}.
 Digitale Edition, https://schnitzler-briefe.acdh.oeaw.ac.at/{\dateiname}.html (Stand \today)
\fi

\end{document}


