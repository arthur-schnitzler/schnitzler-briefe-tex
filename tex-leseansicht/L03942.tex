%% latex-leseansicht-vorspann.tex
%% Vorspann für die Leseansicht.
%% Lädt die gemeinsame Datei latex-vorspann.tex mit nicht gesetztem Schalter.

\newif\ifkorrekturansicht
\korrekturansichtfalse

\input{../tex-inputs/latex-vorspann}


\section[Arthur Schnitzler an Theodor Herzl, 30. 12. 1892]{L03942 Arthur Schnitzler an Theodor Herzl, 30. 12. 1892}
\nopagebreak\mylabel{L03942v}
\rehead{ }\normalsize\beginnumbering\briefempfaengerindex{Herzl, Theodor@\textsc{Herzl, Theodor}!zzzSchnitzler, Arthur@\emph{von Arthur Schnitzler}!1892-12-301@{30. 12. 1892}|(be}
\toendnotes[C]{\smallbreak\pagebreak[2]}
\correspDesc{Versand  durch Arthur Schnitzler am 30. 12. 1892 in Wien
\newline{}Erhalt  durch Theodor Herzl im Zeitraum [31. 12. 1892
                  – 2. 1. 1893] in Paris}\toendnotes[C]{\smallbreak}
\Standort{Jerusalem, Central Zionist Archives, H1:1924-4.}
\physDesc{Brief, 2 Blätter, 6 Seiten, 2027 Zeichen
\newline{}Handschrift: schwarze Tinte, deutsche Kurrent
\newline{}Ordnung: mit Bleistift von unbekannter Hand innerhalb das Konvoluts paginiert:
                                    »15«–»18« }
\buchAbdrucke{\weitereDrucke{Arthur Schnitzler: \emph{Briefe 1875–1912}. Herausgegeben von Therese Nickl und Heinrich Schnitzler. Frankfurt am Main: \emph{S. Fischer} 1981, S. 161–162.} }\toendnotes[C]{\smallbreak}
\pstart\center{}{\pb}Verehrteſter Freund,\pend\vspace{0.5em}
\pstart
           nehmen Sie meine herzlichſten Neujahrsgrüße entgegen! Ich{ }ſende Ihnen dieſelben mit
               beſondrer Freude, denn we{\geminationn} ich{ }ſo die Ergebniſſe des
               heurigen Jahres überſchaue,{ }ſo finde ich, daſs jener \label{K_L03942-1v}\edtext{Brief}{\lemma{\textnormal{\emph{Brief}}}\Cendnote{\textnormal{XXXX Auszeichnungsfehler: Dokument L03823 nicht gefunden. }}}\label{K_L03942-1}, mit welchem Sie{ }ſich als
               einen{ }ſo liebenswürdig\substVorne{}\textsuperscript{ch}\substDazwischen{}en\substHinten{} Betrachter des Märchen\pwindex{Schnitzler, Arthur 15.\,5.\,1862 Wien – 21.\,10.\,1931 ebd.@\textsc{Schnitzler, Arthur} (15.\,5.\,1862 Wien – 21.\,10.\,1931 ebd.), \emph{Schriftsteller, Mediziner}!Märchen. Schauspiel in drei Aufzügen@\strich\emph{Das Märchen. Schauspiel in drei Aufzügen}|pw} zu erke{\geminationn}en gaben, und zugleich manche Misverständniſſe unſrer
               bisherigen {\pb}Beziehungen löſten, zu den wärmſten und
               wohltuendſten Erlebniſſen \strikeout{de} meines 92er Jahres
               gehören. Ich{ }ſtehe in meiner eigenen Anerke{\geminationn}ung noch
               nicht feſt genug, um eine Liebenswürdigkeit wie die Ihre nicht beſonders{ }ſtark zu
               empfinden. Es wundert mich umſomehr, daſs Sie mir noch bis zu einem gewiſſen Grad zu
               mistrauen{ }ſcheinen. Die Gründe, mit welchen Sie mein Erſuchen um einige Ihrer
               Arbeiten ableh\substVorne{}\textsuperscript{en}\substDazwischen{}nen\substHinten{}, {\pb}veranlaßten mich zu dieſer Bemerkung. Sie, mein
               lieber und verehrter Freund,{ }ſtehen auf meine »\textsc{reciproke}«
                  Anerke{\geminationn}ung gewiſs nicht an, und ich meinerſeits
               glaube vor dem Verdacht{ }ſicher zu{ }ſein, aus dem Bedürfnis \textsc{Revanche}freundlichkeiten auszutheilen mich für Ihre Manuscripte zu
               intereſſieren. Daſs Sie manches Dramatiſche geſchrieben haben, daß Sie auch jetzt für
               gut halten, geht aus \label{K_L03942-2v}\edtext{einem {\pb}Ihrer Briefe}{\lemma{\textnormal{\emph{einem Ihrer Briefe}}}\Cendnote{\textnormal{XXXX Auszeichnungsfehler: Dokument L03823 nicht gefunden.}}}\label{K_L03942-2} mit Sicherheit hervor,
               und we{\geminationn} Sie \label{K_L03942-3v}\edtext{vor zehn oder zwölf Jahren}{\lemma{\textnormal{\emph{vor … Jahren}}}\Cendnote{\textnormal{Schnitzler las \emph{Tabarin}\pwindex{Herzl, Theodor 2.\,5.\,1860 Budapest – 3.\,7.\,1904 Edlach@\textsc{Herzl, Theodor} (2.\,5.\,1860 Budapest – 3.\,7.\,1904 Edlach), \emph{Schriftsteller, Journalist}!Tabarin. Schauspiel in einem Act. Frei nach Catulle Mendès@\strich\emph{Tabarin. Schauspiel in einem Act. Frei nach Catulle Mendès}|pwk} (siehe A. S.: \emph{Tagebuch}, 1. 2. 1886) und vermutlich \emph{Die Enttäuschten}\pwindex{Herzl, Theodor 2.\,5.\,1860 Budapest – 3.\,7.\,1904 Edlach@\textsc{Herzl, Theodor} (2.\,5.\,1860 Budapest – 3.\,7.\,1904 Edlach), \emph{Schriftsteller, Journalist}!Enttäuschten. Komödie in vier Acten@\strich\emph{Die Enttäuschten. Komödie in vier Acten}|pwk} im Manuskript. (Vgl. XXXX Auszeichnungsfehler: Dokument L03901 nicht gefunden.)}}}\label{K_L03942-3} nicht
               bezweifelt haben, daſs ich mich für Ihre Stücke intereſſire,{ }ſo liegt heute wohl auch
               kein Grund dafür vor. Es wäre doch ganz{ }ſchön, we{\geminationn} aus der Formel, welche wir beide
               über den Anfang unſrer Briefe{ }ſetzen, auch ein Inhalt flöße. Einigen wir uns dahin,
               daſs wir durchaus keinen Grund haben, in Phraſen {\pb}miteinander zu correſpondiren, und
               daſs jeder Satz, welcher einer dem andern{ }ſchreibt dieſen verbindlich macht, jenem
               Satze zu glauben. Das iſt natürlich keine Erpreſſung, als we{\geminationn} Sie mir nun unbedingt
               was{ }ſchicken müſſten; aber ein Erſuchen iſt es, in meinen Worten an Sie mehr als
               Höflichkeit{ }ſehen zu wollen. Ich war ja{ }ſo frei, auch die Ihren als etwas
               beſſeres zu nehmen. – Und nun leben {\pb}Sie wohl und{ }ſeien Sie meiner aufrichtigen und
               wärmſten Hochſchätzung verſichert.\pend
           
\pstart
           Ihr{\\[\baselineskip]}\spacefill\mbox{ArthSchnitzler}\pend
           \leftskip=0em{}
\pstart
           30/12 92\pend
           \selectlanguage{ngerman}\endnumbering\briefempfaengerindex{Herzl, Theodor@\textsc{Herzl, Theodor}!zzzSchnitzler, Arthur@\emph{von Arthur Schnitzler}!1892-12-301@{30. 12. 1892}|)be}\mylabel{L03942h}
\begin{anhang}
\end{anhang}\newcommand{\dateiname}{L03942}\newcommand{\titel}{Arthur Schnitzler an Theodor Herzl, 30. 12. 1892}\newcommand{\editorInnen}{Herausgegeben von Jahnke, SelmaMüller, Martin Anton}%% latex-leseansicht-abspann.tex
%% Abspann für die Leseansicht.
%% Der Schalter \ifkorrekturansicht ist bereits durch den Vorspann gesetzt.

%% latex-abspann.tex
%% Gemeinsamer Abspann für Korrekturansicht und Leseansicht.
%% Setzt den Schalter \ifkorrekturansicht voraus (gesetzt in den
%% einbindenden Dateien latex-korrekturansicht-abspann.tex bzw.
%% latex-leseansicht-abspann.tex).
%% ---------------------------------------------------------------

\normalsize

% Das esempio-Environment wird nur in der Leseansicht benötigt
\ifkorrekturansicht\else
\newenvironment{esempio}[3]%
{
    \vspace{1.5ex}
    \rlap{\underline{#1}}
    \par
    \setlength{\parindent}{0cm}
    \nopagebreak
    \leftskip=#2cm
    \rightskip=#3cm
}
{
    \par
}
\fi

\doendnotes{C}
\bigskip
\vfill

\clearpage

\footnotesize

\ifkorrekturansicht
  \lohead{\textsc{register}}
\fi

% theindex-Environment neu definieren ohne reledmac
\makeatletter
\renewenvironment{theindex}{%
  \ifkorrekturansicht
    \section*{\indexname}%
  \else
    \subsubsection*{Index der erwähnten Entitäten}%
  \fi
  \setlength{\parindent}{0pt}%
  \setlength{\parskip}{0pt plus 0.3pt}%
  \let\item\@idxitem
}{%
  \ifkorrekturansicht\clearpage\fi
}
\makeatother

\IfFileExists{\jobname-pw.ind}{\input{\jobname-pw.ind}}{}

% Quellenangabe nur in der Leseansicht
\ifkorrekturansicht\else
% Fallback-Definitionen, falls die .tex-Datei \titel etc. nicht gesetzt hat
\providecommand{\titel}{}
\providecommand{\editorInnen}{}
\providecommand{\dateiname}{\jobname}

\vspace{3cm}

\vfill

\footnotesize
\textsc{Quelle}: \titel. Herausgegeben von {\editorInnen}. In: \emph{Arthur Schnitzler: Briefwechsel mit Autorinnen und Autoren}.
 Digitale Edition, https://schnitzler-briefe.acdh.oeaw.ac.at/{\dateiname}.html (Stand \today)
\fi

\end{document}


