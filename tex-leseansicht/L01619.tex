%% latex-korrekturansicht-vorspann.tex
%% Vorspann für die Korrekturansicht.
%% Lädt die gemeinsame Datei latex-vorspann.tex mit gesetztem Schalter.

\newif\ifkorrekturansicht
\korrekturansichttrue

\input{../tex-inputs/latex-vorspann}


\section[Arthur Schnitzler an Richard Beer-Hofmann, 1. 8. 1906]{L01619 Arthur Schnitzler an Richard Beer-Hofmann, 1. 8. 1906}
\nopagebreak\mylabel{L01619v}
\rehead{ }\normalsize\beginnumbering\briefempfaengerindex{Beer-Hofmann, Richard@\textsc{Beer-Hofmann, Richard}!zzzSchnitzler, Arthur@\emph{von Arthur Schnitzler}!1906-08-011@{1. 8. 1906}|(be}
\toendnotes[C]{\smallbreak\pagebreak[2]}\Standort{YCGL, MSS 31.}
\physDesc{Bildpostkarte, 107 Zeichen
\newline{}Handschrift: 1) Bleistift, deutsche Kurrent\hspace{1em}2) Bleistift, lateinische Kurrent (\noindent{}Adresse)\hspace{1em}
\newline{}Versand: 1) Stempel: »\nobreak{}\oindex{Skodsborg@\textbf{Skodsborg}, \emph{P.PPL}|pwk}Skodsborg, 1. 8. 06, \textcolor{gray}{D}17F\nobreak{}«.   2) Stempel: »\nobreak{}\oindex{Rodaun@\textbf{Rodaun}, \emph{A.ADM4}|pwk}Rodaun\nobreak{}«. }\toendnotes[C]{\smallbreak}\pstart{}{\pb}Dr. Richard Beer-Hofmann\pend{}\pstart{}Rodaun bei Wien\oindex{Rodaun@\textbf{Rodaun}, \emph{A.ADM4}|pw}\pend{}\pstart{}Liesingerstr. –\oindex{Liesingerstrasse@\textbf{Liesingerstraße}, \emph{Straße (K.STR)}|pw}\pend{}\pstart{}Austria\oindex{Oesterreich@\textbf{Österreich}, \emph{A.PCLI}|pw}\pend{}{\bigskip}
\pstart
           \noindent{}{\pb}\textcolor{gray}{\textbf{Skodsborg Badehotel\oindex{Badehotellet@\textbf{Badehotellet}, \emph{Hotel (K.HTL)}|pw}}}\pend
           \vspace{1em}
\pstart
           \centering{}{\pb}1/8 906.\pend
           \vspace{0.5em}
\pstart
           \label{K_L01619-1v}\edtext{Domino.}{\lemma{\textnormal{\emph{Domino.}}}\Cendnote{\textnormal{Einzelne Begriffe sind mit bestimmten Stellen des
                  Ansichtkartenmotivs verbunden, bestimmte Themen des gemeinsamen Aufenthalts
                     1896 evozierend (vgl. A. S.: \emph{Tagebuch}, 14. 8. 1896). Nachdem Schnitzler damals in der Dependance des Hotels gewohnt hat, dürfte dieser
                  Begriff die Terrasse vor seiner Unterkunft markieren, während das folgende
                     »Jugend?« die Zimmer von Beer-Hofmann\pwindex{Beer-Hofmann, Richard 1866-07-11 – 1945-09-26@\textsc{Beer-Hofmann, Richard} (1866-07-11 – 1945-09-26), \emph{Schriftsteller/Schriftstellerin}|pwk} und Paula Lissy\pwindex{Beer-Hofmann, Paula 25.02.1879 – 30.10.1939@\textsc{Beer-Hofmann, Paula} (25.02.1879 – 30.10.1939)|pwk}
                  einzeichnet.}}}\label{K_L01619-1}\pend
           
\pstart
           Jugend?\pend
           
\pstart
           \label{T_L01619-1v}\edtext{Bad}{\lemma{\textnormal{\emph{Bad}}}\Cendnote{\textnormal{Verbindungslinie zum Wasser}}}\label{T_L01619-1}\pend
           
\pstart
           {\pb}Herzliche Grüße,\pend
           
\pstart
           Ihr{\\[\baselineskip]}\spacefill\mbox{A.}\pend
           \leftskip=0em{}\selectlanguage{ngerman}\endnumbering\briefempfaengerindex{Beer-Hofmann, Richard@\textsc{Beer-Hofmann, Richard}!zzzSchnitzler, Arthur@\emph{von Arthur Schnitzler}!1906-08-011@{1. 8. 1906}|)be}\mylabel{L01619h}  \normalsize

\doendnotes{C}
\bigskip
\vfill

\clearpage

\footnotesize

\lohead{\textsc{register}}

% Definiere theindex-Environment komplett neu ohne reledmac
\makeatletter
\renewenvironment{theindex}{%
  \section*{\indexname}%
  \setlength{\parindent}{0pt}%
  \setlength{\parskip}{0pt plus 0.3pt}%
  \let\item\@idxitem
}{%
  \clearpage
}
\makeatother

\IfFileExists{\jobname-pw.ind}{\input{\jobname-pw.ind}}{}

\end{document}

      