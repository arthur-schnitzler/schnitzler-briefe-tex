%% latex-korrekturansicht-vorspann.tex
%% Vorspann für die Korrekturansicht.
%% Lädt die gemeinsame Datei latex-vorspann.tex mit gesetztem Schalter.

\newif\ifkorrekturansicht
\korrekturansichttrue

\input{../tex-inputs/latex-vorspann}


\section[Hermann Bahr an Arthur Schnitzler, 30. 12. {[}1901{]}]{L01192 Hermann Bahr an Arthur Schnitzler, 30. 12. {[}1901{]}}
\nopagebreak\mylabel{L01192v}
\rehead{ }\normalsize\beginnumbering\briefempfaengerindex{Schnitzler, Arthur@\textsc{Schnitzler, Arthur}!zzzBahr, Hermann@\emph{von Hermann Bahr}!1901-12-301@{30. 12. 1901}|(be}
\toendnotes[C]{\smallbreak\pagebreak[2]}\Standort{CUL, Schnitzler, B 5b.}
\physDesc{Brief, 1 Blatt, 4 Seiten, 1649 Zeichen
\newline{}Handschrift: schwarze Tinte, deutsche Kurrent
\newline{}Schnitzler: mit Bleistift die Jahreszahl »901« ergänzt 
\newline{}Ordnung: mit Bleistift von unbekannter Hand nummeriert:
                                    »84« }
\buchAbdrucke{\weitereDrucke{Hermann Bahr, Arthur Schnitzler: \emph{Briefwechsel, Aufzeichnungen, Dokumente (1891–1931)}. Göttingen: \emph{Wallstein} 2018, S. 220–221.} }\toendnotes[C]{\smallbreak}
\pstart
           \centering{}{\pb}\textcolor{gray}{\textbf{Redaktion des Neuen Wiener Tagblatt\orgindex{Neues Wiener Tagblatt@Neues Wiener Tagblatt|pw}}}\pend
           
\pstart
           \centering{}\textcolor{gray}{\textbf{\textsc{Wien, I., Rothenturmstrasse,
                        Steyrerhof\oindex{Steyrerhof@\textbf{Steyrerhof}, \emph{Gebäude (K.GBD)}|pw}.}}}\pend
           
\pstart
           \centering{}\textcolor{gray}{\textbf{Telegramm-Adresse: Tagblatt\orgindex{Neues Wiener Tagblatt@Neues Wiener Tagblatt|pw}, Steyrerhof, Wien\oindex{Steyrerhof@\textbf{Steyrerhof}, \emph{Gebäude (K.GBD)}|pw}. –
                     Telephon Nr. 384. Staats-Telephon Nr. 36.}}\pend
           
\pstart
           30. 12.\pend
           
\pstart\center{}Lieber Arthur!\pend\vspace{0.5em}
\pstart
           Danke ſehr für Deine liebe Karte. Du könnteſt mir allerdings in Berlin\oindex{Berlin@\textbf{Berlin}, \emph{P.PPLC}|pw} einen ſehr, ſehr großen Dienst erweiſen, wenn Du
               gelegentlich mit Brahm\pwindex{Brahm, Otto 05.02.1856 – 28.11.1912@\textsc{Brahm, Otto} (05.02.1856 – 28.11.1912), \emph{Theaterleiter/Theaterleiterin, Regisseur/Regisseurin}|pw} über mich ſprechen und
               ihm klar machen würdeſt, daß ich, bei allem, was man gegen mich ſagen kann, doch
               ſchließlich auch Jemand bin und daß ich gern in ein, wenn auch kühles, doch
               anſtändiges Verhältnis gegenſeitiger Duldung und beding{\pb}ter Anerkennung \introOben{}zu ihm\introOben{}
               kommen möchte. Ich leide ſehr unter meiner Erfolgloſigkeit in Deutſchland\oindex{Deutschland@\textbf{Deutschland}, \emph{A.PCLI}|pw} und bin ſchon ſo beſcheiden geworden, daß ich es als
               einen großen Erfolg empfinden würde, wenn er ſich nur entschließen könnte, ein Stück
               von mir anzunehmen und aufzuführen, meinetwegen in der ſchlechteſten Zeit, weil es
               mir dabei gar nicht auf die Tantièmen ankommt, ſondern auf den »literarischen
               Stempel«, {\pb}den nun das Deutſche Theater\oindex{Deutsches Theater Berlin@\textbf{Deutsches Theater Berlin}, \emph{Theater (K.THE)}|pw} einmal ſeinen Autoren gibt und der mir noch immer fehlt,
               und darauf, von ſeiner »Clique« ernſt genommen zu werden. Er hat mir über den
                  »\label{K_L01192-1v}\edtext{Krampus\pwindex{Krampus. Lustspiel in drei Aufzuegen@\emph{Der Krampus. Lustspiel in drei Aufzügen}|pw}}{\lemma{\textnormal{\emph{Krampus}}}\Cendnote{\textnormal{Hermann Bahr\pwindex{Bahr, Hermann 19.07.1863 – 15.01.1934@\textsc{Bahr, Hermann} (19.07.1863 – 15.01.1934), \emph{Schriftsteller/Schriftstellerin, Kritiker/Kritikerin}|pwk}: \emph{Der Krampus. Lustspiel in drei Aufzügen}\pwindex{Krampus. Lustspiel in drei Aufzuegen@\emph{Der Krampus. Lustspiel in drei Aufzügen}|pwk}. München: \emph{Albert Langen}\orgindex{Albert Langen@Albert Langen|pwk}{ }1902 (vordatiert von Dezember 1901).}}}\label{K_L01192-1}«
               ſehr anerkennend geſprochen, ihn aber ſchließlich leider doch abgelehnt; ich werde
               ihn nun einladen, der Hamburger\oindex{Hamburg@\textbf{Hamburg}, \emph{P.PPLA}|pw}{ }\label{K_L01192-2v}\edtext{Première\eventindex{Hamburg@\textbf{Hamburg}!Premiere von Der Krampus, 14.1.1902@Premiere von Der Krampus, 14.1.1902|pw}}{\lemma{\textnormal{\emph{Première}}}\Cendnote{\textnormal{Letztlich erfolgte die Aufführung\eventindex{Hamburg@\textbf{Hamburg}!Premiere von Der Krampus, 14.1.1902@Premiere von Der Krampus, 14.1.1902|pwkv} in Hamburg\oindex{Hamburg@\textbf{Hamburg}, \emph{P.PPLA}|pwk} am 14. 1. 1902 unter dem Titel \emph{Der Herr Hofrat}\pwindex{Krampus. Lustspiel in drei Aufzuegen@\emph{Der Krampus. Lustspiel in drei Aufzügen}|pwk}.}}}\label{K_L01192-2} (am 12 oder 13 Januar)
               beizuwohnen; freilich ohne viel Hoffnung, \strikeout{\textcolor{gray}{ohne}} ihn noch {\pb}umzuſtimmen. Aber vielleicht
               bringſt Du ihn doch ſo weit, daß er ſich, wenn ich ihm wieder ein Stück ſchicke, es
               wenigſtens mit nicht im Vorhinein feindlichen Augen anſieht.\pend
           
\pstart
           Aber bitte, thu das nur, wenn es ſich leicht machen läßt, ohne Dir unbequem zu
               ſein.\pend
           
\pstart
           Ich bin rieſig neugierig auf \label{K_L01192-3v}\edtext{Samſtag}{\lemma{\textnormal{\emph{Samſtag}}}\Cendnote{\textnormal{Die Uraufführung\eventindex{Deutsches Theater Berlin@\textbf{Deutsches Theater Berlin}!Urauffuehrung von Lebendige Stunden, 4.1.1902@Uraufführung von Lebendige Stunden, 4.1.1902|pwkv} von \emph{Lebendige Stunden}\pwindex{Lebendige Stunden. Vier Einakter@\emph{Lebendige Stunden. Vier Einakter}|pwk} fand am 4. 1. 1902 am \emph{Deutschen Theater}\orgindex{Deutsches Theater Berlin@Deutsches Theater Berlin|pwk}  in Berlin\oindex{Berlin@\textbf{Berlin}, \emph{P.PPLC}|pwk} statt.
               }}}\label{K_L01192-3}; mehr auszuſprechen verbietet mir mein Aberglaube.\pend
           
\pstart
           Herzlichſt{\\[\baselineskip]}Dein alter{\\[\baselineskip]}\spacefill\mbox{HermannB}\pend
           \leftskip=0em{}
\pstart
           \noindent{}\textsc{Prost Neujahr!}\pend
           
\pstart
           \label{T_L01192-1v}\edtext{\label{K_L01192-4v}\edtext{Den Novelli\pwindex{Novelli, Ermete 05.03.1851 – 29.01.1919@\textsc{Novelli, Ermete} (05.03.1851 – 29.01.1919), \emph{Schauspieler/Schauspielerin}|pw}, der über den »Kakadu\pwindex{gruene Kakadu. Groteske in einem Akt@\emph{Der grüne Kakadu. Groteske in einem Akt}|pw}« noch immer nichts hören ließ, habe ich gestern \substVorne{}\textsuperscript{d}\substDazwischen{}D\substHinten{}ringend gemahnt.}{\lemma{\textnormal{\emph{Den … gemahnt.}}}\Cendnote{\textnormal{In den
                     Korrespondenzstücken, die von Novelli\pwindex{Novelli, Ermete 05.03.1851 – 29.01.1919@\textsc{Novelli, Ermete} (05.03.1851 – 29.01.1919), \emph{Schauspieler/Schauspielerin}|pwk} im
                     Nachlass Bahrs\pwindex{Bahr, Hermann 19.07.1863 – 15.01.1934@\textsc{Bahr, Hermann} (19.07.1863 – 15.01.1934), \emph{Schriftsteller/Schriftstellerin, Kritiker/Kritikerin}|pwk} überliefert sind, findet
                     sich darüber kein näherer Aufschluss.}}}\label{K_L01192-4}}{\lemma{\textnormal{\emph{Den … gemahnt.}}}\Cendnote{\textnormal{quer am rechten Rand}}}\label{T_L01192-1}\pend
           \selectlanguage{ngerman}\endnumbering\briefempfaengerindex{Schnitzler, Arthur@\textsc{Schnitzler, Arthur}!zzzBahr, Hermann@\emph{von Hermann Bahr}!1901-12-301@{30. 12. 1901}|)be}\mylabel{L01192h}  \normalsize

\doendnotes{C}
\bigskip
\vfill

\clearpage

\footnotesize

\lohead{\textsc{register}}

% Definiere theindex-Environment komplett neu ohne reledmac
\makeatletter
\renewenvironment{theindex}{%
  \section*{\indexname}%
  \setlength{\parindent}{0pt}%
  \setlength{\parskip}{0pt plus 0.3pt}%
  \let\item\@idxitem
}{%
  \clearpage
}
\makeatother

\IfFileExists{\jobname-pw.ind}{\input{\jobname-pw.ind}}{}

\end{document}

      