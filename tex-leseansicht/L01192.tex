%% latex-leseansicht-vorspann.tex
%% Vorspann für die Leseansicht.
%% Lädt die gemeinsame Datei latex-vorspann.tex mit nicht gesetztem Schalter.

\newif\ifkorrekturansicht
\korrekturansichtfalse

\input{../tex-inputs/latex-vorspann}


\section[Hermann Bahr an Arthur Schnitzler, 30. 12. {[}1901{]}]{L01192 Hermann Bahr an Arthur Schnitzler, 30. 12. [1901]}
\nopagebreak\mylabel{L01192v}
\rehead{ }\normalsize\beginnumbering\briefempfaengerindex{Schnitzler, Arthur@\textsc{Schnitzler, Arthur}!zzzBahr, Hermann@\emph{von Hermann Bahr}!1901-12-301@{30. 12. 1901}|(be}
\toendnotes[C]{\smallbreak\pagebreak[2]}
\correspDesc{Versand  durch Hermann Bahr am 30. 12. 1901 in Wien
\newline{}Erhalt  durch Arthur Schnitzler im Zeitraum [30. 12. 1901 – 3. 1. 1902?] in Wien}\toendnotes[C]{\smallbreak}
\Standort{CUL, Schnitzler, B 5b.}
\physDesc{Brief, 1 Blatt, 4 Seiten, 1649 Zeichen
\newline{}Handschrift: schwarze Tinte, deutsche Kurrent
\newline{}Schnitzler: mit Bleistift die Jahreszahl »901« ergänzt 
\newline{}Ordnung: mit Bleistift von unbekannter Hand nummeriert:
                                    »84« }
\buchAbdrucke{\weitereDrucke{Hermann Bahr, Arthur Schnitzler: \emph{Briefwechsel, Aufzeichnungen, Dokumente (1891–1931)}. Herausgegeben von Kurt Ifkovits und Martin Anton Müller. Göttingen: \emph{Wallstein} 2018, S. 220–221.} }\toendnotes[C]{\smallbreak}
\pstart
           \centering{}{\pb}\textcolor{gray}{\textbf{Redaktion des Neuen Wiener Tagblatt\orgindex{Neues Wiener Tagblatt@Neues Wiener Tagblatt|pw}}}\pend
           
\pstart
           \centering{}\textcolor{gray}{\textbf{\textsc{Wien, I., Rothenturmstrasse,
                        Steyrerhof\oindex{Wien@\textbf{Wien}!I., Innere Stadt@\textbf{I., Innere Stadt}!Steyrerhof@\textbf{Steyrerhof}, \emph{Gebäude}|pw}.}}}\pend
           
\pstart
           \centering{}\textcolor{gray}{\textbf{Telegramm-Adresse: Tagblatt\orgindex{Neues Wiener Tagblatt@Neues Wiener Tagblatt|pw}, Steyrerhof, Wien\oindex{Wien@\textbf{Wien}!I., Innere Stadt@\textbf{I., Innere Stadt}!Steyrerhof@\textbf{Steyrerhof}, \emph{Gebäude}|pw}. –
                     Telephon Nr. 384. Staats-Telephon Nr. 36.}}\pend
           
\pstart
           30. 12.\pend
           
\pstart\center{}Lieber Arthur!\pend\vspace{0.5em}
\pstart
           Danke{ }ſehr für Deine liebe Karte. Du könnteſt mir allerdings in Berlin\oindex{Berlin@\textbf{Berlin}, \emph{Hauptstadt}|pw} einen{ }ſehr,{ }ſehr großen Dienst erweiſen, wenn Du
               gelegentlich mit Brahm\pwindex{Brahm, Otto 5.\,2.\,1856 Hamburg – 28.\,11.\,1912 Berlin@\textsc{Brahm, Otto} (5.\,2.\,1856 Hamburg – 28.\,11.\,1912 Berlin), \emph{Theaterleiter, Regisseur}|pw} über mich{ }ſprechen und
               ihm klar machen würdeſt, daß ich, bei allem, was man gegen mich{ }ſagen kann, doch{ }ſchließlich auch Jemand bin und daß ich gern in ein, wenn auch kühles, doch
               anſtändiges Verhältnis gegenſeitiger Duldung und beding{\pb}ter Anerkennung \introOben{}zu ihm\introOben{}
               kommen möchte. Ich leide{ }ſehr unter meiner Erfolgloſigkeit in Deutſchland\oindex{Deutschland@\textbf{Deutschland}|pw} und bin{ }ſchon{ }ſo beſcheiden geworden, daß ich es als
               einen großen Erfolg empfinden würde, wenn er{ }ſich nur entschließen könnte, ein Stück
               von mir anzunehmen und aufzuführen, meinetwegen in der{ }ſchlechteſten Zeit, weil es
               mir dabei gar nicht auf die Tantièmen ankommt,{ }ſondern auf den »literarischen
               Stempel«, {\pb}den nun das Deutſche Theater\oindex{Deutsches Theater Berlin@\textbf{Deutsches Theater Berlin}, \emph{Theater}|pw} einmal{ }ſeinen Autoren gibt und der mir noch immer fehlt,
               und darauf, von{ }ſeiner »Clique« ernſt genommen zu werden. Er hat mir über den
                  »\label{K_L01192-1v}\edtext{Krampus\pwindex{Bahr, Hermann 19.\,7.\,1863 Linz – 15.\,1.\,1934 München@\textsc{Bahr, Hermann} (19.\,7.\,1863 Linz – 15.\,1.\,1934 München), \emph{Schriftsteller, Kritiker}!Krampus. Lustspiel in drei Aufzügen@\strich\emph{Der Krampus. Lustspiel in drei Aufzügen}|pw}}{\lemma{\textnormal{\emph{Krampus}}}\Cendnote{\textnormal{Hermann Bahr\pwindex{Bahr, Hermann 19.\,7.\,1863 Linz – 15.\,1.\,1934 München@\textsc{Bahr, Hermann} (19.\,7.\,1863 Linz – 15.\,1.\,1934 München), \emph{Schriftsteller, Kritiker}|pwk}: \emph{Der Krampus. Lustspiel in drei Aufzügen}\pwindex{Bahr, Hermann 19.\,7.\,1863 Linz – 15.\,1.\,1934 München@\textsc{Bahr, Hermann} (19.\,7.\,1863 Linz – 15.\,1.\,1934 München), \emph{Schriftsteller, Kritiker}!Krampus. Lustspiel in drei Aufzügen@\strich\emph{Der Krampus. Lustspiel in drei Aufzügen}|pwk}. München: \emph{Albert Langen}\orgindex{Albert Langen@Albert Langen|pwk}{ }1902 (vordatiert von Dezember 1901).}}}\label{K_L01192-1}«{ }ſehr anerkennend geſprochen, ihn aber{ }ſchließlich leider doch abgelehnt; ich werde
               ihn nun einladen, der Hamburger\oindex{Hamburg@\textbf{Hamburg}|pw}{ }\label{K_L01192-2v}\edtext{Première\eventindex{Hamburg@\textbf{Hamburg}!Premiere von Der Krampus, 14.1.1902@Premiere von Der Krampus, 14.1.1902|pw}}{\lemma{\textnormal{\emph{Première}}}\Cendnote{\textnormal{Letztlich erfolgte die Aufführung\eventindex{Hamburg@\textbf{Hamburg}!Premiere von Der Krampus, 14.1.1902@Premiere von Der Krampus, 14.1.1902|pwkv} in Hamburg\oindex{Hamburg@\textbf{Hamburg}|pwk} am 14. 1. 1902 unter dem Titel \emph{Der Herr Hofrat}\pwindex{Bahr, Hermann 19.\,7.\,1863 Linz – 15.\,1.\,1934 München@\textsc{Bahr, Hermann} (19.\,7.\,1863 Linz – 15.\,1.\,1934 München), \emph{Schriftsteller, Kritiker}!Krampus. Lustspiel in drei Aufzügen@\strich\emph{Der Krampus. Lustspiel in drei Aufzügen}|pwk}.}}}\label{K_L01192-2} (am 12 oder 13 Januar)
               beizuwohnen; freilich ohne viel Hoffnung, \strikeout{\textcolor{gray}{ohne}} ihn noch {\pb}umzuſtimmen. Aber vielleicht
               bringſt Du ihn doch{ }ſo weit, daß er{ }ſich, wenn ich ihm wieder ein Stück{ }ſchicke, es
               wenigſtens mit nicht im Vorhinein feindlichen Augen anſieht.\pend
           
\pstart
           Aber bitte, thu das nur, wenn es{ }ſich leicht machen läßt, ohne Dir unbequem zu{ }ſein.\pend
           
\pstart
           Ich bin rieſig neugierig auf \label{K_L01192-3v}\edtext{Samſtag}{\lemma{\textnormal{\emph{Samstag}}}\Cendnote{\textnormal{Die Uraufführung\eventindex{Deutsches Theater Berlin@\textbf{Deutsches Theater Berlin}!Uraufführung von Lebendige Stunden, 4.1.1902@Uraufführung von Lebendige Stunden, 4.1.1902|pwkv} von \emph{Lebendige Stunden}\pwindex{Schnitzler, Arthur 15.\,5.\,1862 Wien – 21.\,10.\,1931 ebd.@\textsc{Schnitzler, Arthur} (15.\,5.\,1862 Wien – 21.\,10.\,1931 ebd.), \emph{Schriftsteller, Mediziner}!Lebendige Stunden. Vier Einakter@\strich\emph{Lebendige Stunden. Vier Einakter}|pwk} fand am 4. 1. 1902 am \emph{Deutschen Theater}\orgindex{Deutsches Theater Berlin@Deutsches Theater Berlin|pwk}  in Berlin\oindex{Berlin@\textbf{Berlin}, \emph{Hauptstadt}|pwk} statt.
               }}}\label{K_L01192-3}; mehr auszuſprechen verbietet mir mein Aberglaube.\pend
           
\pstart
           Herzlichſt{\\[\baselineskip]}Dein alter{\\[\baselineskip]}\spacefill\mbox{HermannB}\pend
           \leftskip=0em{}
\pstart
           \noindent{}\textsc{Prost Neujahr!}\pend
           
\pstart
           \label{T_L01192-1v}\edtext{\label{K_L01192-4v}\edtext{Den Novelli\pwindex{Novelli, Ermete 5.\,3.\,1851 Lucca – 29.\,1.\,1919 Neapel@\textsc{Novelli, Ermete} (5.\,3.\,1851 Lucca – 29.\,1.\,1919 Neapel), \emph{Schauspieler}|pw}, der über den »Kakadu\pwindex{Schnitzler, Arthur 15.\,5.\,1862 Wien – 21.\,10.\,1931 ebd.@\textsc{Schnitzler, Arthur} (15.\,5.\,1862 Wien – 21.\,10.\,1931 ebd.), \emph{Schriftsteller, Mediziner}!grüne Kakadu. Groteske in einem Akt@\strich\emph{Der grüne Kakadu. Groteske in einem Akt}|pw}« noch immer nichts hören ließ, habe ich gestern \substVorne{}\textsuperscript{d}\substDazwischen{}D\substHinten{}ringend gemahnt.}{\lemma{\textnormal{\emph{Den … gemahnt.}}}\Cendnote{\textnormal{In den
                     Korrespondenzstücken, die von Novelli\pwindex{Novelli, Ermete 5.\,3.\,1851 Lucca – 29.\,1.\,1919 Neapel@\textsc{Novelli, Ermete} (5.\,3.\,1851 Lucca – 29.\,1.\,1919 Neapel), \emph{Schauspieler}|pwk} im
                     Nachlass Bahrs\pwindex{Bahr, Hermann 19.\,7.\,1863 Linz – 15.\,1.\,1934 München@\textsc{Bahr, Hermann} (19.\,7.\,1863 Linz – 15.\,1.\,1934 München), \emph{Schriftsteller, Kritiker}|pwk} überliefert sind, findet
                     sich darüber kein näherer Aufschluss.}}}\label{K_L01192-4}}{\lemma{\textnormal{\emph{Den … gemahnt.}}}\Cendnote{\textnormal{quer am rechten Rand}}}\label{T_L01192-1}\pend
           \selectlanguage{ngerman}\endnumbering\briefempfaengerindex{Schnitzler, Arthur@\textsc{Schnitzler, Arthur}!zzzBahr, Hermann@\emph{von Hermann Bahr}!1901-12-301@{30. 12. 1901}|)be}\mylabel{L01192h}  \newcommand{\dateiname}{L01192}\newcommand{\titel}{Hermann Bahr an Arthur Schnitzler, 30. 12. [1901]}\newcommand{\editorInnen}{Herausgegeben von Martin Anton Müller}%% latex-leseansicht-abspann.tex
%% Abspann für die Leseansicht.
%% Der Schalter \ifkorrekturansicht ist bereits durch den Vorspann gesetzt.

%% latex-abspann.tex
%% Gemeinsamer Abspann für Korrekturansicht und Leseansicht.
%% Setzt den Schalter \ifkorrekturansicht voraus (gesetzt in den
%% einbindenden Dateien latex-korrekturansicht-abspann.tex bzw.
%% latex-leseansicht-abspann.tex).
%% ---------------------------------------------------------------

\normalsize

% Das esempio-Environment wird nur in der Leseansicht benötigt
\ifkorrekturansicht\else
\newenvironment{esempio}[3]%
{
    \vspace{1.5ex}
    \rlap{\underline{#1}}
    \par
    \setlength{\parindent}{0cm}
    \nopagebreak
    \leftskip=#2cm
    \rightskip=#3cm
}
{
    \par
}
\fi

\doendnotes{C}
\bigskip
\vfill

\clearpage

\footnotesize

\ifkorrekturansicht
  \lohead{\textsc{register}}
\fi

% theindex-Environment neu definieren ohne reledmac
\makeatletter
\renewenvironment{theindex}{%
  \ifkorrekturansicht
    \section*{\indexname}%
  \else
    \subsubsection*{Index der erwähnten Entitäten}%
  \fi
  \setlength{\parindent}{0pt}%
  \setlength{\parskip}{0pt plus 0.3pt}%
  \let\item\@idxitem
}{%
  \ifkorrekturansicht\clearpage\fi
}
\makeatother

\IfFileExists{\jobname-pw.ind}{\input{\jobname-pw.ind}}{}

% Quellenangabe nur in der Leseansicht
\ifkorrekturansicht\else
% Fallback-Definitionen, falls die .tex-Datei \titel etc. nicht gesetzt hat
\providecommand{\titel}{}
\providecommand{\editorInnen}{}
\providecommand{\dateiname}{\jobname}

\vspace{3cm}

\vfill

\footnotesize
\textsc{Quelle}: \titel. Herausgegeben von {\editorInnen}. In: \emph{Arthur Schnitzler: Briefwechsel mit Autorinnen und Autoren}.
 Digitale Edition, https://schnitzler-briefe.acdh.oeaw.ac.at/{\dateiname}.html (Stand \today)
\fi

\end{document}


