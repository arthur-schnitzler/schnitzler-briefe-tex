%% latex-korrekturansicht-vorspann.tex
%% Vorspann für die Korrekturansicht.
%% Lädt die gemeinsame Datei latex-vorspann.tex mit gesetztem Schalter.

\newif\ifkorrekturansicht
\korrekturansichttrue

\input{../tex-inputs/latex-vorspann}


\section[Elsa Plessner an Arthur Schnitzler, 2. 1. 1899]{L03718 Elsa Plessner an Arthur Schnitzler, 2. 1. 1899}
\nopagebreak\mylabel{L03718v}
\rehead{ }\normalsize\beginnumbering\briefempfaengerindex{Schnitzler, Arthur@\textsc{Schnitzler, Arthur}!zzzPlessner, Elsa@\emph{von Elsa Plessner}!1899-01-021@{2. 1. 1899}|(be}
\toendnotes[C]{\smallbreak\pagebreak[2]}\Standort{DLA, A:Schnitzler, HS.1985.1.419.}
\physDesc{Brief, 1 Blatt, 3 Seiten, 1269 Zeichen (Briefpapier mit Blumenmotiv (Mohn) auf S. 1)
\newline{}Handschrift: , lateinische Kurrent}\toendnotes[C]{\smallbreak}
\pstart
           \raggedleft{}{\pb}Wien I. Spiegelg. 2\oindex{Spiegelgasse 2@\textbf{Spiegelgasse 2}, \emph{Wohngebäude (K.WHS)}|pw}\pend
           
\pstart
           \raggedleft{} den 2. I. 99. \pend
           
\pstart{}Verehrter Herr Doctor!\pend\vspace{0.5em}
\pstart
           Neugierig und zudringlich, wie ich einmal bin, bitte ich Sie heute
               wieder einmal um Ihre Meinung über eine Arbeit\pwindex{neue Lehrer. Novelle@\emph{Der neue Lehrer. Novelle}|pwv}. \label{K_L03718-1v}\edtext{Beiliegende Novelle\pwindex{neue Lehrer. Novelle@\emph{Der neue Lehrer. Novelle}|pwv}}{\lemma{\textnormal{\emph{Beiliegende Novelle}}}\Cendnote{\textnormal{Vermutlich lag dem Brief die Novelle \emph{Der neue Lehrer}\pwindex{neue Lehrer. Novelle@\emph{Der neue Lehrer. Novelle}|pwk} bei. Im Brief vom 19. 1. 1899 nennt Elsa Plessner\pwindex{Plessner, Elsa 22.08.1875 – 01.05.1932@\textsc{Plessner, Elsa} (22.08.1875 – 01.05.1932), \emph{Schriftsteller/Schriftstellerin}|pwk}
                  erstmals den Titel ihres längsten erhaltenen Prosatextes.}}}\label{K_L03718-1} habe ich vor 14
               Tagen aus der »Wage\pwindex{Wage. Eine Wiener Wochenschrift@\emph{Die Wage. Eine Wiener Wochenschrift}|pw}« zurückgezogen da ich mir
               keine Striche gefallen lasse, von denen ich überzeugt bin, dass {\pb}sie
               meine Arbeit nicht nur schädigen, sondern direct umbringen. Spuren einer
               redactionellen Thätigkeit werden sie in dem Manuscript\pwindex{neue Lehrer. Novelle@\emph{Der neue Lehrer. Novelle}|pwv} genügend vorfinden. – – – Ich bin doch nicht
               verpflichtet, für die Moral der Leser der »Wage\pwindex{Wage. Eine Wiener Wochenschrift@\emph{Die Wage. Eine Wiener Wochenschrift}|pw}«
               zu sorgen und ihre Sittlichkeit zu behüten. Die »inciminirten« Stellen der Arbeit\pwindex{neue Lehrer. Novelle@\emph{Der neue Lehrer. Novelle}|pwv} habe ich mir im
               Interesse derselben \uline{abzwingen} müssen, denn Sie können
               es mir glauben, auch ich schreibe so etwas nicht gerne nieder. Aber was ich als
               Mädchen über mich {\pb}gewinnen kann zu schreiben: das ist noch immer zahm
               genug, dass es die »Wage\pwindex{Wage. Eine Wiener Wochenschrift@\emph{Die Wage. Eine Wiener Wochenschrift}|pw}« die doch kein
               Familienblatt ist – ruhig abdrucken kann. – – –\pend
           
\pstart
           Ansonsten bin ich sehr gespannt auf Ihr Urtheil über diese Arbeit\pwindex{neue Lehrer. Novelle@\emph{Der neue Lehrer. Novelle}|pwv}. Es ist die erst, etwas größere,
               und ausgeführtere Novelle\pwindex{neue Lehrer. Novelle@\emph{Der neue Lehrer. Novelle}|pwv} im
               Gegensatz zu meinen früheren Skizzen. –\pend
           
\pstart
           – Nur eine Bitte habe ich: – Lesen Sie sie \uuline{auf einen
                  Zug} und ungestört durch, wenn ich auch länger auf Ihren Ausspruch zappeln
               muss. –\pend
           
\pstart
           In unveränderlicher Verehrung Prosit Neujahr!{\\[\baselineskip]}\spacefill\mbox{Elsa Plessner}\pend
           \leftskip=0em{}\selectlanguage{ngerman}\endnumbering\briefempfaengerindex{Schnitzler, Arthur@\textsc{Schnitzler, Arthur}!zzzPlessner, Elsa@\emph{von Elsa Plessner}!1899-01-021@{2. 1. 1899}|)be}\mylabel{L03718h}
\begin{anhang}
\end{anhang}\normalsize

\doendnotes{C}
\bigskip
\vfill

\clearpage

\footnotesize

\lohead{\textsc{register}}

% Definiere theindex-Environment komplett neu ohne reledmac
\makeatletter
\renewenvironment{theindex}{%
  \section*{\indexname}%
  \setlength{\parindent}{0pt}%
  \setlength{\parskip}{0pt plus 0.3pt}%
  \let\item\@idxitem
}{%
  \clearpage
}
\makeatother

\IfFileExists{\jobname-pw.ind}{\input{\jobname-pw.ind}}{}

\end{document}

      