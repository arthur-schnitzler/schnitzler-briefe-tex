%% latex-korrekturansicht-vorspann.tex
%% Vorspann für die Korrekturansicht.
%% Lädt die gemeinsame Datei latex-vorspann.tex mit gesetztem Schalter.

\newif\ifkorrekturansicht
\korrekturansichttrue

\input{../tex-inputs/latex-vorspann}


\section[ Arthur Schnitzler: Widmungsexemplar Lebendige Stunden für Felix Salten, {[}11.?{]} 1. 1902]{L03605 Arthur Schnitzler: Widmungsexemplar Lebendige Stunden für Felix
               Salten, {[}11.?{]} 1. 1902}
\nopagebreak\mylabel{L03605v}
\rehead{ }\normalsize\beginnumbering\briefempfaengerindex{Salten, Felix@\textsc{Salten, Felix}!zzzSchnitzler, Arthur@\emph{von Arthur Schnitzler}!1902-01-112@{{[}11.?{]} 1. 1902}|(be}
\toendnotes[C]{\smallbreak\pagebreak[2]}\Standort{Wienbibliothek im Rathaus, A-355812, DS-2019-58.}
\physDesc{Widmung am Vorsatzblatt, 49 Zeichen
\newline{}Handschrift: schwarze Tinte, deutsche Kurrent
\newline{}Salten: mit schwarzer Tinte ausgefüllter Stempel: »\noindent{}\textcolor{gray}{\textbf{\textit{Felix Salt{[}en{]}}}}{ / }\textcolor{gray}{\textbf{\textit{Inv. Nr.}}}{ }4473{ / }\textcolor{gray}{\textbf{\textit{Werk Nr.}}}{ }2201{ / }\textcolor{gray}{\textbf{\textit{Schrank}}}{ }XIV A. Z. \textcolor{gray}{\textbf{\textit{Fach}}} b«  }\toendnotes[C]{\smallbreak}
\pstart
           \noindent{}{\pb}Meinem lieben \textsc{Felix Salten}\pend
           \pstart \spacefill\mbox{ArthSch}\pend{}
\pstart
           Wien\oindex{Wien@\textbf{Wien}, \emph{A.ADM2}|pw}{ }\label{K_L03605-1v}\edtext{Jänner 902}{\lemma{\textnormal{\emph{Jänner 902}}}\Cendnote{\textnormal{Am 23. 12. 1901 wurde \emph{Lebendige
                        Stunden}\pwindex{Lebendige Stunden. Vier Einakter@\emph{Lebendige Stunden. Vier Einakter}|pwk} vom \emph{Börsenblatt für den
                           deutschen Buchhandel}\pwindex{Boersenblatt fuer den Deutschen Buchhandel@\emph{Börsenblatt für den Deutschen Buchhandel}|pwk} als Neuerscheinung gemeldet. Am 
                     [12. 1. 1902]
                     bestätigte Salten\pwindex{Salten, Felix 06.09.1869 – 08.10.1945@\textsc{Salten, Felix} (06.09.1869 – 08.10.1945), \emph{Schriftsteller/Schriftstellerin, Journalist/Journalistin, Chefredakteur/Chefredakteurin}|pwk} den Erhalt des
                     Buches, sodass der Versand der Widmungsexemplare am 
                     Vortag erfolgt sein dürfte.}}}\label{K_L03605-1}\pend
           \selectlanguage{ngerman}\vspace{1em}{\vspace{1\baselineskip}}
\pstart
           \centering{}{\pb}\textcolor{gray}{\textbf{Arthur Schnitzler}}\pend
           
\pstart
           \centering{}\textcolor{gray}{\textbf{Lebendige Stunden\pwindex{Lebendige Stunden. Vier Einakter@\emph{Lebendige Stunden. Vier Einakter}|pw}}}\pend
           
\pstart
           \centering{}\textcolor{gray}{\textbf{Vier Einakter}}\pend
           
\pstart
           \centering{}\textcolor{gray}{\textbf{Zweite Auflage}}\pend
           {\vspace{1\baselineskip}}
\pstart
           \centering{}\textcolor{gray}{\textbf{\so{Berlin}\oindex{Berlin@\textbf{Berlin}, \emph{P.PPLC}|pw}{ }1902}}\pend
           
\pstart
           \centering{}\textcolor{gray}{\textbf{\so{S. Fiſcher, Verlag}\orgindex{S. Fischer Verlag@S. Fischer Verlag|pw}}}\pend
           \selectlanguage{ngerman}\endnumbering\briefempfaengerindex{Salten, Felix@\textsc{Salten, Felix}!zzzSchnitzler, Arthur@\emph{von Arthur Schnitzler}!1902-01-112@{{[}11.?{]} 1. 1902}|)be}\mylabel{L03605h}  \normalsize

\doendnotes{C}
\bigskip
\vfill

\clearpage

\footnotesize

\lohead{\textsc{register}}

% Definiere theindex-Environment komplett neu ohne reledmac
\makeatletter
\renewenvironment{theindex}{%
  \section*{\indexname}%
  \setlength{\parindent}{0pt}%
  \setlength{\parskip}{0pt plus 0.3pt}%
  \let\item\@idxitem
}{%
  \clearpage
}
\makeatother

\IfFileExists{\jobname-pw.ind}{\input{\jobname-pw.ind}}{}

\end{document}

      