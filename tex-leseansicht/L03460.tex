%% latex-korrekturansicht-vorspann.tex
%% Vorspann für die Korrekturansicht.
%% Lädt die gemeinsame Datei latex-vorspann.tex mit gesetztem Schalter.

\newif\ifkorrekturansicht
\korrekturansichttrue

\input{../tex-inputs/latex-vorspann}


\section[ Paul Goldmann an Arthur Schnitzler, 16. 1. 1908]{L03460 Paul Goldmann an Arthur Schnitzler, 16. 1. 1908}
\nopagebreak\mylabel{L03460v}
\rehead{ }\normalsize\beginnumbering\briefempfaengerindex{Schnitzler, Arthur@\textsc{Schnitzler, Arthur}!zzzGoldmann, Paul@\emph{von Paul Goldmann}!1908-01-162@{16. 1. 1908}|(be}
\toendnotes[C]{\smallbreak\pagebreak[2]}\Standort{DLA, A:Schnitzler, HS.NZ85.1.3175.}
\physDesc{Bildpostkarte, 259 Zeichen
\newline{}Handschrift: 1) blaue Tinte, deutsche Kurrent\hspace{1em}2) blaue Tinte, lateinische Kurrent (\noindent{}Adresse)\hspace{1em}
\newline{}Versand: Stempel: »\nobreak{}\oindex{Berlin@\textbf{Berlin}, \emph{P.PPLC}|pwk}Berlin SW 11, 16. 1. 08, 5–6N.\nobreak{}«.  
\newline{}Schnitzler: mit Bleistift Unterstreichung der Unterschrift »Goldmann\pwindex{Goldmann, Paul 31.01.1865 – 25.09.1935@\textsc{Goldmann, Paul} (31.01.1865 – 25.09.1935), \emph{Schriftsteller/Schriftstellerin, Journalist/Journalistin}|pw}« }\toendnotes[C]{\smallbreak}\pstart{}{\pb}Herrn\pend{}\pstart{}Dr. Arthur Schnitzler\pend{}\pstart{}Wien\oindex{Wien@\textbf{Wien}, \emph{A.ADM2}|pw}\pend{}\pstart{}XVIII. Spöttelgaſse 7\oindex{Edmund-Weiss-Gasse 7@\textbf{Edmund-Weiß-Gasse 7}, \emph{Wohngebäude (K.WHS)}|pw}.\pend{}{\bigskip}
\pstart
           \noindent{}\centering{}{\pb}\textcolor{gray}{\textbf{Glückliches Neujahr!}}\pend
           \vspace{1em}
\pstart
           {\pb}16. 1. 08.\pend
           
\pstart{}Lieber Freund,\pend\vspace{0.5em}
\pstart
           Daß Dir der \label{K_L03460-1v}\edtext{Grillparzer-Preis\orgindex{Franz-Grillparzer-Preis@Franz-Grillparzer-Preis|pw}}{\lemma{\textnormal{\emph{Grillparzer-Preis}}}\Cendnote{\textnormal{Das Auswahlkomitee hatte am 15. 1. 1908
                  entschieden, Schnitzler für seine
                  Komödie \emph{Zwischenspiel}\pwindex{Zwischenspiel. Komoedie in drei Akten@\emph{Zwischenspiel. Komödie in drei Akten}|pwk} den mit 5000 Kronen
                  dotierten \emph{Grillparzer-Preis}\orgindex{Franz-Grillparzer-Preis@Franz-Grillparzer-Preis|pwk} zu verleihen. In
                  den Jahren zuvor war er zwar immer wieder als Favorit gehandelt worden, doch
                  stellte das Zerwürfnis mit dem \emph{Burgtheater}\orgindex{Burgtheater@Burgtheater|pwk} in
                  Folge der Rückgabe von \emph{Der Schleier der
                     Beatrice}\pwindex{Schleier der Beatrice. Schauspiel in fuenf Akten@\emph{Der Schleier der Beatrice. Schauspiel in fünf Akten}|pwk} (1901) ein Hindernis dar. Seit Sommer 1905 war der Konflikt behoben und Schnitzler konnte wieder bei der Preisvergabe\orgindex{Franz-Grillparzer-Preis@Franz-Grillparzer-Preis|pwkv} berücksichtigt
                  werden. }}}\label{K_L03460-1} verliehen
               worden iſt, hat mich aufrichtig gefreut, u. ich beglückwünſche Dich auf das
               Herzlichſte.\pend
           
\pstart
           Mit vielen Grüßen an Dich u. Deine Frau\pwindex{Schnitzler, Olga 17.01.1882 – 13.01.1970@\textsc{Schnitzler, Olga} (17.01.1882 – 13.01.1970), \emph{Schauspieler/Schauspielerin, Sänger/Sängerin}|pwv}{ }{\\[\baselineskip]}Dein {\\[\baselineskip]}\spacefill\mbox{Paul Goldmann.}\pend
           \leftskip=0em{}\selectlanguage{ngerman}\endnumbering\briefempfaengerindex{Schnitzler, Arthur@\textsc{Schnitzler, Arthur}!zzzGoldmann, Paul@\emph{von Paul Goldmann}!1908-01-162@{16. 1. 1908}|)be}\mylabel{L03460h}  \normalsize

\doendnotes{C}
\bigskip
\vfill

\clearpage

\footnotesize

\lohead{\textsc{register}}

% Definiere theindex-Environment komplett neu ohne reledmac
\makeatletter
\renewenvironment{theindex}{%
  \section*{\indexname}%
  \setlength{\parindent}{0pt}%
  \setlength{\parskip}{0pt plus 0.3pt}%
  \let\item\@idxitem
}{%
  \clearpage
}
\makeatother

\IfFileExists{\jobname-pw.ind}{\input{\jobname-pw.ind}}{}

\end{document}

      