\input{../tex-inputs/latex-pdf-vorspann}
\begin{center}
            \textcolor{red}{ENTWURF. ENTZIFFERUNG NOCH NICHT KORREKTURGELESEN}
                      \end{center}
            
               \section[Oscar Blumenthal an Arthur Schnitzler, 19. 11. 1896]{ Oscar Blumenthal an Arthur Schnitzler, 19. 11. 1896}\nopagebreak\mylabel{v}\rehead{ }\begin{ledgroupsized}[t]{13cm}\normalsize\beginnumbering\briefempfaengerindex{Schnitzler, Arthur@\textsc{Schnitzler, Arthur}!zzzBlumenthal, Oskar@\emph{von Oskar Blumenthal}!1896-11-191@{19. 11. 1896}|(be} \toendnotes[C]{\smallbreak\pagebreak[2]} \Standort{CUL, Schnitzler, B 15.}
\physDesc{Brief, 1 Blatt, 2 Seiten
\newline{}Schreibmaschine
\newline{}Handschrift: schwarze Tinte, lateinische Kurrent (\noindent{}eine
                                        Korrektur, Unterschrift)
\newline{}Schnitzler: mit rotem Buntstift eine Unterstreichung \newline{}Ordnung: mit Bleistift von unbekannter Hand nummeriert:
                                                »8« }\toendnotes[C]{\smallbreak}\pstart
           \noindent{}\centering{}{\pb}\textcolor{gray}{\textbf{\textsc{Lessing-Theater}\orgindex{Lessing-Theater@Lessing-Theater|pw}}}\pend
           \pstart
           \noindent{}\centering{}\textcolor{gray}{\textbf{\textsc{Director}:}}{ }\textcolor{gray}{\textbf{\textsc{Dr.}{ }OSCAR BLUMENTHAL.}}\pend
           \pstart
           \noindent{}\raggedleft{}\textcolor{gray}{\textbf{Berlin N.W. (40)\oindex{Berlin@\textbf{Berlin}|pw}, den}}{ }19. November \textcolor{gray}{\textbf{18}}96.\pend
           \pstart\center{}Sehr geehrter Herr Doctor!\pend\pstart
           Ich sage Ihnen zunächst meinen wärmsten Dank für Ihre prinzipielle Zustimmung zu
                    meinem Vorschlage, von der ich auch Freund MITTERWURZER\pwindex{Mitterwurzer, Friedrich 16.10.1844 – 13.02.1897@\textsc{Mitterwurzer, Friedrich} (16.10.1844 – 13.02.1897), \emph{Schauspieler}|pw} sofort benachrichtige. Die
                    Aussicht, dass Sie durch ein neues Schlussstück den Cyclus\pwindex{Schnitzler, Arthur 15.05.1862 – 21.10.1931@\textsc{Schnitzler, Arthur} (15.05.1862 – 21.10.1931), \emph{Schriftsteller, Mediziner}!Anatol1892-10-29 – 1892-10-29@\strich\emph{Anatol} {[}1892-10-29 – 1892-10-29{]}|pwv} abrunden werden, erfreut mich noch ganz
                    besonders. Jedenfalls werde ich jetzt das Buch noch einmal von Anfang bis zu
                    Ende auf mich wirken lassen, und auch die von Ihnen hervorgehobenen Plaudereien
                        »AGONIE\pwindex{Schnitzler, Arthur 15.05.1862 – 21.10.1931@\textsc{Schnitzler, Arthur} (15.05.1862 – 21.10.1931), \emph{Schriftsteller, Mediziner}!Agonie1892@\strich\emph{Agonie} {[}1892{]}|pw}« und »DENKSTEINE\pwindex{Schnitzler, Arthur 15.05.1862 – 21.10.1931@\textsc{Schnitzler, Arthur} (15.05.1862 – 21.10.1931), \emph{Schriftsteller, Mediziner}!Denksteine15. 05. 1891@\strich\emph{Denksteine} {[}15. 05. 1891{]}|pw}« in’s Auge
                    fassen, damit wir uns zunächst über die Auswahl aus dem Vorhandenem schlüssig
                    machen. \substVorne{}\textsuperscript{Damit}\substDazwischen{}Darin\substHinten{} stimme ich mit Ihnen selbstverständlich überein, dass die Frauenrollen
                    in den verschiedenen Stücken von verschiedenen Darstellerinnen gespielt werden
                    müssen. Das »LESSING-THEATER\orgindex{Lessing-Theater@Lessing-Theater|pw}« hat glücklicherweise eine reiche Auswahl {\pb}von frischen weiblichen Talenten,
                    die für diese Stücke zur Verfügung stehen. Gewiss finden Sie inzwischen auch
                    einmal Gelegenheit mit MITTERWURZER\pwindex{Mitterwurzer, Friedrich 16.10.1844 – 13.02.1897@\textsc{Mitterwurzer, Friedrich} (16.10.1844 – 13.02.1897), \emph{Schauspieler}|pw} persönlich zusammenzutreffen; der lebhafte
                    Eifer, mit welchem er auf den Gedanken eingegangen ist, lässt mich hoffen, dass
                    er aus Ihrem ANATOL\pwindex{Schnitzler, Arthur 15.05.1862 – 21.10.1931@\textsc{Schnitzler, Arthur} (15.05.1862 – 21.10.1931), \emph{Schriftsteller, Mediziner}!Anatol1892-10-29 – 1892-10-29@\strich\emph{Anatol} {[}1892-10-29 – 1892-10-29{]}|pw}
                    ein packendes Characterbild schaffen wird.\pend
           \pstart
           Mit besten Grüssen Ihr{\\[\baselineskip]} aufrichtig ergebener{\\[\baselineskip]}\spacefill\mbox{{[}hs.:{]} Dr. Osc. Blumenthal}\pend
           \leftskip=0em{}\endnumbering\briefempfaengerindex{Schnitzler, Arthur@\textsc{Schnitzler, Arthur}!zzzBlumenthal, Oskar@\emph{von Oskar Blumenthal}!1896-11-191@{19. 11. 1896}|)be}\mylabel{h}\end{ledgroupsized}  \newcommand{\dateiname}{L00623}\newcommand{\titel}{Oscar Blumenthal an Arthur Schnitzler, 19. 11. 1896}\newcommand{\editorInnen}{Martin Anton Müller und Gerd-Hermann Susen}\input{../tex-inputs/latex-pdf-abspann}
      