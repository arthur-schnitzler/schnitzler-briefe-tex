%% latex-korrekturansicht-vorspann.tex
%% Vorspann für die Korrekturansicht.
%% Lädt die gemeinsame Datei latex-vorspann.tex mit gesetztem Schalter.

\newif\ifkorrekturansicht
\korrekturansichttrue

\input{../tex-inputs/latex-vorspann}


\section[Oscar Blumenthal an Arthur Schnitzler, 19. 11. 1896]{L00623 Oscar Blumenthal an Arthur Schnitzler, 19. 11. 1896}
\nopagebreak\mylabel{L00623v}
\rehead{ }\normalsize\beginnumbering\briefempfaengerindex{Schnitzler, Arthur@\textsc{Schnitzler, Arthur}!zzzBlumenthal, Oskar@\emph{von Oskar Blumenthal}!1896-11-191@{19. 11. 1896}|(be}
\toendnotes[C]{\smallbreak\pagebreak[2]}\Standort{CUL, Schnitzler, B 15.}
\physDesc{Brief, 1 Blatt, 2 Seiten, 1199 Zeichen
\newline{}Schreibmaschine
\newline{}Handschrift: schwarze Tinte, lateinische Kurrent (\noindent{}eine Korrektur, Unterschrift)
\newline{}Schnitzler: mit rotem Buntstift eine Unterstreichung 
\newline{}Ordnung: mit Bleistift von unbekannter Hand nummeriert:
                                 »8« }\toendnotes[C]{\smallbreak}
\pstart
           \centering{}{\pb}\textcolor{gray}{\textbf{\textsc{Lessing-Theater}\orgindex{Lessing-Theater@Lessing-Theater|pw}}}\pend
           
\pstart
           \centering{}\textcolor{gray}{\textbf{\textsc{Director}:}}{ }\textcolor{gray}{\textbf{\textsc{Dr.}{ }OSCAR BLUMENTHAL.}}\pend
           
\pstart
           \raggedleft{}\textcolor{gray}{\textbf{Berlin N.W. (40)\oindex{Berlin@\textbf{Berlin}, \emph{P.PPLC}|pw}, den}}{ }19. November \textcolor{gray}{\textbf{18}}96.\pend
           
\pstart\center{}Sehr geehrter Herr Doctor!\pend\vspace{0.5em}
\pstart
           Ich sage Ihnen zunächst meinen wärmsten Dank für Ihre prinzipielle Zustimmung zu
               meinem Vorschlage, von der ich auch Freund MITTERWURZER\pwindex{Mitterwurzer, Friedrich 16.10.1844 – 13.02.1897@\textsc{Mitterwurzer, Friedrich} (16.10.1844 – 13.02.1897), \emph{Schauspieler/Schauspielerin}|pw} sofort benachrichtige. Die Aussicht, dass Sie durch ein neues Schlussstück den
                  Cyclus\pwindex{Anatol@\emph{Anatol}|pwv} abrunden werden,
               erfreut mich noch ganz besonders. Jedenfalls werde ich jetzt das Buch noch einmal von
               Anfang bis zu Ende auf mich wirken lassen, und auch die von Ihnen hervorgehobenen
               Plaudereien »AGONIE\pwindex{Agonie@\emph{Agonie}|pw}« und »DENKSTEINE\pwindex{Denksteine@\emph{Denksteine}|pw}« in’s Auge fassen, damit wir uns zunächst über die Auswahl aus dem Vorhandenem
               schlüssig machen. \substVorne{}\textsuperscript{Damit}\substDazwischen{}Darin\substHinten{} stimme ich mit Ihnen selbstverständlich überein, dass die Frauenrollen in
               den verschiedenen Stücken von verschiedenen Darstellerinnen gespielt werden müssen.
               Das »LESSING-THEATER\orgindex{Lessing-Theater@Lessing-Theater|pw}« hat glücklicherweise eine reiche Auswahl {\pb}von frischen weiblichen Talenten, die für
               diese Stücke zur Verfügung stehen. Gewiss finden Sie inzwischen auch einmal
               Gelegenheit mit MITTERWURZER\pwindex{Mitterwurzer, Friedrich 16.10.1844 – 13.02.1897@\textsc{Mitterwurzer, Friedrich} (16.10.1844 – 13.02.1897), \emph{Schauspieler/Schauspielerin}|pw} persönlich zusammenzutreffen; der lebhafte Eifer, mit welchem er auf den
               Gedanken eingegangen ist, lässt mich hoffen, dass er aus Ihrem ANATOL\pwindex{Anatol@\emph{Anatol}|pw} ein packendes Characterbild schaffen wird.\pend
           
\pstart
           Mit besten Grüssen Ihr{\\[\baselineskip]} aufrichtig ergebener{\\[\baselineskip]}\spacefill\mbox{{[}hs.:{]} Dr. Osc. Blumenthal}\pend
           \leftskip=0em{}\selectlanguage{ngerman}\endnumbering\briefempfaengerindex{Schnitzler, Arthur@\textsc{Schnitzler, Arthur}!zzzBlumenthal, Oskar@\emph{von Oskar Blumenthal}!1896-11-191@{19. 11. 1896}|)be}\mylabel{L00623h}  \normalsize

\doendnotes{C}
\bigskip
\vfill

\clearpage

\footnotesize

\lohead{\textsc{register}}

% Definiere theindex-Environment komplett neu ohne reledmac
\makeatletter
\renewenvironment{theindex}{%
  \section*{\indexname}%
  \setlength{\parindent}{0pt}%
  \setlength{\parskip}{0pt plus 0.3pt}%
  \let\item\@idxitem
}{%
  \clearpage
}
\makeatother

\IfFileExists{\jobname-pw.ind}{\input{\jobname-pw.ind}}{}

\end{document}

      