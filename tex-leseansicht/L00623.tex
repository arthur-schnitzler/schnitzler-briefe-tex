%% latex-leseansicht-vorspann.tex
%% Vorspann für die Leseansicht.
%% Lädt die gemeinsame Datei latex-vorspann.tex mit nicht gesetztem Schalter.

\newif\ifkorrekturansicht
\korrekturansichtfalse

\input{../tex-inputs/latex-vorspann}


\section[Oscar Blumenthal an Arthur Schnitzler, 19. 11. 1896]{L00623 Oscar Blumenthal an Arthur Schnitzler, 19. 11. 1896}
\nopagebreak\mylabel{L00623v}
\rehead{ }\normalsize\beginnumbering\briefempfaengerindex{Schnitzler, Arthur@\textsc{Schnitzler, Arthur}!zzzBlumenthal, Oskar@\emph{von Oskar Blumenthal}!1896-11-191@{19. 11. 1896}|(be}
\toendnotes[C]{\smallbreak\pagebreak[2]}
\correspDesc{Versand  durch Oscar Blumenthal am 19. 11. 1896 in Berlin
\newline{}Erhalt  durch Arthur Schnitzler im Zeitraum [20. 11. 1896 – 24. 11. 1896?] in Wien}\toendnotes[C]{\smallbreak}
\Standort{CUL, Schnitzler, B 15.}
\physDesc{Brief, 1 Blatt, 2 Seiten, 1199 Zeichen
\newline{}Schreibmaschine
\newline{}Handschrift: schwarze Tinte, lateinische Kurrent (\noindent{}eine Korrektur, Unterschrift)
\newline{}Schnitzler: mit rotem Buntstift eine Unterstreichung 
\newline{}Ordnung: mit Bleistift von unbekannter Hand nummeriert:
                                 »8« }\toendnotes[C]{\smallbreak}
\pstart
           \centering{}{\pb}\textcolor{gray}{\textbf{\textsc{Lessing-Theater}\orgindex{Lessing-Theater@Lessing-Theater|pw}}}\pend
           
\pstart
           \centering{}\textcolor{gray}{\textbf{\textsc{Director}:}}{ }\textcolor{gray}{\textbf{\textsc{Dr.}{ }OSCAR BLUMENTHAL.}}\pend
           
\pstart
           \raggedleft{}\textcolor{gray}{\textbf{Berlin N.W. (40)\oindex{Berlin@\textbf{Berlin}, \emph{Hauptstadt}|pw}, den}}{ }19. November \textcolor{gray}{\textbf{18}}96.\pend
           
\pstart\center{}Sehr geehrter Herr Doctor!\pend\vspace{0.5em}
\pstart
           Ich sage Ihnen zunächst meinen wärmsten Dank für Ihre prinzipielle Zustimmung zu
               meinem Vorschlage, von der ich auch Freund MITTERWURZER\pwindex{Mitterwurzer, Friedrich 16.\,10.\,1844 Dresden – 13.\,2.\,1897 Wien@\textsc{Mitterwurzer, Friedrich} (16.\,10.\,1844 Dresden – 13.\,2.\,1897 Wien), \emph{Schauspieler}|pw} sofort benachrichtige. Die Aussicht, dass Sie durch ein neues Schlussstück den
                  Cyclus\pwindex{Schnitzler, Arthur 15.\,5.\,1862 Wien – 21.\,10.\,1931 ebd.@\textsc{Schnitzler, Arthur} (15.\,5.\,1862 Wien – 21.\,10.\,1931 ebd.), \emph{Schriftsteller, Mediziner}!Anatol@\strich\emph{Anatol}|pwv} abrunden werden,
               erfreut mich noch ganz besonders. Jedenfalls werde ich jetzt das Buch noch einmal von
               Anfang bis zu Ende auf mich wirken lassen, und auch die von Ihnen hervorgehobenen
               Plaudereien »AGONIE\pwindex{Schnitzler, Arthur 15.\,5.\,1862 Wien – 21.\,10.\,1931 ebd.@\textsc{Schnitzler, Arthur} (15.\,5.\,1862 Wien – 21.\,10.\,1931 ebd.), \emph{Schriftsteller, Mediziner}!Agonie@\strich\emph{Agonie}|pw}« und »DENKSTEINE\pwindex{Schnitzler, Arthur 15.\,5.\,1862 Wien – 21.\,10.\,1931 ebd.@\textsc{Schnitzler, Arthur} (15.\,5.\,1862 Wien – 21.\,10.\,1931 ebd.), \emph{Schriftsteller, Mediziner}!Denksteine@\strich\emph{Denksteine}|pw}« in’s Auge fassen, damit wir uns zunächst über die Auswahl aus dem Vorhandenem
               schlüssig machen. \substVorne{}\textsuperscript{Damit}\substDazwischen{}Darin\substHinten{} stimme ich mit Ihnen selbstverständlich überein, dass die Frauenrollen in
               den verschiedenen Stücken von verschiedenen Darstellerinnen gespielt werden müssen.
               Das »LESSING-THEATER\orgindex{Lessing-Theater@Lessing-Theater|pw}« hat glücklicherweise eine reiche Auswahl {\pb}von frischen weiblichen Talenten, die für
               diese Stücke zur Verfügung stehen. Gewiss finden Sie inzwischen auch einmal
               Gelegenheit mit MITTERWURZER\pwindex{Mitterwurzer, Friedrich 16.\,10.\,1844 Dresden – 13.\,2.\,1897 Wien@\textsc{Mitterwurzer, Friedrich} (16.\,10.\,1844 Dresden – 13.\,2.\,1897 Wien), \emph{Schauspieler}|pw} persönlich zusammenzutreffen; der lebhafte Eifer, mit welchem er auf den
               Gedanken eingegangen ist, lässt mich hoffen, dass er aus Ihrem ANATOL\pwindex{Schnitzler, Arthur 15.\,5.\,1862 Wien – 21.\,10.\,1931 ebd.@\textsc{Schnitzler, Arthur} (15.\,5.\,1862 Wien – 21.\,10.\,1931 ebd.), \emph{Schriftsteller, Mediziner}!Anatol@\strich\emph{Anatol}|pw} ein packendes Characterbild schaffen wird.\pend
           
\pstart
           Mit besten Grüssen Ihr{\\[\baselineskip]} aufrichtig ergebener{\\[\baselineskip]}\spacefill\mbox{{[}hs.:{]} Dr. Osc. Blumenthal}\pend
           \leftskip=0em{}\selectlanguage{ngerman}\endnumbering\briefempfaengerindex{Schnitzler, Arthur@\textsc{Schnitzler, Arthur}!zzzBlumenthal, Oskar@\emph{von Oskar Blumenthal}!1896-11-191@{19. 11. 1896}|)be}\mylabel{L00623h}  \newcommand{\dateiname}{L00623}\newcommand{\titel}{Oscar Blumenthal an Arthur Schnitzler, 19. 11. 1896}\newcommand{\editorInnen}{Martin Anton Müller und Gerd-Hermann Susen}%% latex-leseansicht-abspann.tex
%% Abspann für die Leseansicht.
%% Der Schalter \ifkorrekturansicht ist bereits durch den Vorspann gesetzt.

%% latex-abspann.tex
%% Gemeinsamer Abspann für Korrekturansicht und Leseansicht.
%% Setzt den Schalter \ifkorrekturansicht voraus (gesetzt in den
%% einbindenden Dateien latex-korrekturansicht-abspann.tex bzw.
%% latex-leseansicht-abspann.tex).
%% ---------------------------------------------------------------

\normalsize

% Das esempio-Environment wird nur in der Leseansicht benötigt
\ifkorrekturansicht\else
\newenvironment{esempio}[3]%
{
    \vspace{1.5ex}
    \rlap{\underline{#1}}
    \par
    \setlength{\parindent}{0cm}
    \nopagebreak
    \leftskip=#2cm
    \rightskip=#3cm
}
{
    \par
}
\fi

\doendnotes{C}
\bigskip
\vfill

\clearpage

\footnotesize

\ifkorrekturansicht
  \lohead{\textsc{register}}
\fi

% theindex-Environment neu definieren ohne reledmac
\makeatletter
\renewenvironment{theindex}{%
  \ifkorrekturansicht
    \section*{\indexname}%
  \else
    \subsubsection*{Index der erwähnten Entitäten}%
  \fi
  \setlength{\parindent}{0pt}%
  \setlength{\parskip}{0pt plus 0.3pt}%
  \let\item\@idxitem
}{%
  \ifkorrekturansicht\clearpage\fi
}
\makeatother

\IfFileExists{\jobname-pw.ind}{\input{\jobname-pw.ind}}{}

% Quellenangabe nur in der Leseansicht
\ifkorrekturansicht\else
% Fallback-Definitionen, falls die .tex-Datei \titel etc. nicht gesetzt hat
\providecommand{\titel}{}
\providecommand{\editorInnen}{}
\providecommand{\dateiname}{\jobname}

\vspace{3cm}

\vfill

\footnotesize
\textsc{Quelle}: \titel. Herausgegeben von {\editorInnen}. In: \emph{Arthur Schnitzler: Briefwechsel mit Autorinnen und Autoren}.
 Digitale Edition, https://schnitzler-briefe.acdh.oeaw.ac.at/{\dateiname}.html (Stand \today)
\fi

\end{document}


