%% latex-leseansicht-vorspann.tex
%% Vorspann für die Leseansicht.
%% Lädt die gemeinsame Datei latex-vorspann.tex mit nicht gesetztem Schalter.

\newif\ifkorrekturansicht
\korrekturansichtfalse

\input{../tex-inputs/latex-vorspann}


\section[Hugo von Hofmannsthal an Arthur Schnitzler, 7. 10. 1903]{L01324 Hugo von Hofmannsthal an Arthur Schnitzler, 7. 10. 1903}
\nopagebreak\mylabel{L01324v}
\rehead{ }\normalsize\beginnumbering\briefempfaengerindex{Schnitzler, Arthur@\textsc{Schnitzler, Arthur}!zzzHofmannsthal, Hugo von@\emph{von Hugo von Hofmannsthal}!1903-10-072@{7. 10. 1903}|(be}
\toendnotes[C]{\smallbreak\pagebreak[2]}
\correspDesc{Versand  durch Hugo von Hofmannsthal am 7. 10. 1903 in Wien
\newline{}Erhalt  durch Arthur Schnitzler am 8. 10. 1903 in Wien}\toendnotes[C]{\smallbreak}
\Standort{CUL, Schnitzler, B 43.}
\physDesc{Postkarte, 202 Zeichen
\newline{}Handschrift: schwarze Tinte, deutsche Kurrent
\newline{}Versand: Stempel: »\nobreak{}\oindex{XVIII., Währing@\textbf{XVIII., Währing}, \emph{Verwaltungsgebiet}|pwk}\textcolor{gray}{18/1} Wien, 8. 10. 03, 8.V, Bestellt\nobreak{}«.  
\newline{}Schnitzler: mit Bleistift datiert: »8/10 903« 
\newline{}Ordnung: 1) mit Bleistift von unbekannter Hand nummeriert: »\strikeout{\textcolor{gray}{221}}«  2) mit Bleistift von unbekannter Hand nummeriert:
                                    »202«}
\buchAbdrucke{\weitereDrucke{Hugo von Hofmannsthal, Arthur Schnitzler: \emph{Briefwechsel}. Herausgegeben von Therese Nickl und Heinrich Schnitzler. Frankfurt am Main: \emph{S. Fischer} 1964, S. 175.} }\toendnotes[C]{\smallbreak}\pstart{}\textsc{{\pb}Herrn D\textsuperscript{r} Arthur Schnitzler}\pend{}\pstart{}\textsc{Wien\oindex{Wien@\textbf{Wien}, \emph{Verwaltungsgebiet}|pw}}\pend{}\pstart{}\textsc{XVIII. Spöttelgasse 7\oindex{Wien@\textbf{Wien}!XVIII., Währing@\textbf{XVIII., Währing}!Edmund-Weiß-Gasse 7@\textbf{Edmund-Weiß-Gasse 7}, \emph{Wohngebäude}|pw}}\pend{}{\bigskip}\vspace{1em}
\pstart
           \noindent{}{\pb}lieber, wenn Sie \uline{nicht} abſagen,
               erwarten wir Sie beide dieſen \label{K_L01324-1v}\edtext{Samstag}{\lemma{\textnormal{\emph{Samstag}}}\Cendnote{\textnormal{Vgl. A. S.: \emph{Tagebuch}, 10. 10. 1903.
               }}}\label{K_L01324-1}{ }\textsc{circa}{ }6 – ½ 7 bei uns.\pend
           
\pstart
           Vielleicht bringen Sie mir Ihre Photogr.\pwindex{\textcolor{red}{\textsuperscript{XXXX indx1}}!Arthur Schnitzler [Halbprofil 1903]@\strich\emph{Arthur Schnitzler [Halbprofil 1903]}|pwv} mit.\pend
           
\pstart
           Von Herzen{\\[\baselineskip]}\spacefill\mbox{Hugo.}\pend
           \leftskip=0em{}
\pstart
           \noindent{}Mittwoch.\pend
           \selectlanguage{ngerman}\endnumbering\briefempfaengerindex{Schnitzler, Arthur@\textsc{Schnitzler, Arthur}!zzzHofmannsthal, Hugo von@\emph{von Hugo von Hofmannsthal}!1903-10-072@{7. 10. 1903}|)be}\mylabel{L01324h}  \newcommand{\dateiname}{L01324}\newcommand{\titel}{Hugo von Hofmannsthal an Arthur Schnitzler, 7. 10. 1903}\newcommand{\editorInnen}{Martin Anton Müller und Gerd-Hermann Susen}%% latex-leseansicht-abspann.tex
%% Abspann für die Leseansicht.
%% Der Schalter \ifkorrekturansicht ist bereits durch den Vorspann gesetzt.

%% latex-abspann.tex
%% Gemeinsamer Abspann für Korrekturansicht und Leseansicht.
%% Setzt den Schalter \ifkorrekturansicht voraus (gesetzt in den
%% einbindenden Dateien latex-korrekturansicht-abspann.tex bzw.
%% latex-leseansicht-abspann.tex).
%% ---------------------------------------------------------------

\normalsize

% Das esempio-Environment wird nur in der Leseansicht benötigt
\ifkorrekturansicht\else
\newenvironment{esempio}[3]%
{
    \vspace{1.5ex}
    \rlap{\underline{#1}}
    \par
    \setlength{\parindent}{0cm}
    \nopagebreak
    \leftskip=#2cm
    \rightskip=#3cm
}
{
    \par
}
\fi

\doendnotes{C}
\bigskip
\vfill

\clearpage

\footnotesize

\ifkorrekturansicht
  \lohead{\textsc{register}}
\fi

% theindex-Environment neu definieren ohne reledmac
\makeatletter
\renewenvironment{theindex}{%
  \ifkorrekturansicht
    \section*{\indexname}%
  \else
    \subsubsection*{Index der erwähnten Entitäten}%
  \fi
  \setlength{\parindent}{0pt}%
  \setlength{\parskip}{0pt plus 0.3pt}%
  \let\item\@idxitem
}{%
  \ifkorrekturansicht\clearpage\fi
}
\makeatother

\IfFileExists{\jobname-pw.ind}{\input{\jobname-pw.ind}}{}

% Quellenangabe nur in der Leseansicht
\ifkorrekturansicht\else
% Fallback-Definitionen, falls die .tex-Datei \titel etc. nicht gesetzt hat
\providecommand{\titel}{}
\providecommand{\editorInnen}{}
\providecommand{\dateiname}{\jobname}

\vspace{3cm}

\vfill

\footnotesize
\textsc{Quelle}: \titel. Herausgegeben von {\editorInnen}. In: \emph{Arthur Schnitzler: Briefwechsel mit Autorinnen und Autoren}.
 Digitale Edition, https://schnitzler-briefe.acdh.oeaw.ac.at/{\dateiname}.html (Stand \today)
\fi

\end{document}


