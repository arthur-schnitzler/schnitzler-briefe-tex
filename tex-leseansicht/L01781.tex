%% latex-leseansicht-vorspann.tex
%% Vorspann für die Leseansicht.
%% Lädt die gemeinsame Datei latex-vorspann.tex mit nicht gesetztem Schalter.

\newif\ifkorrekturansicht
\korrekturansichtfalse

\input{../tex-inputs/latex-vorspann}


\section[Richard Beer-Hofmann an Arthur Schnitzler, 6. [7. 1908]]{L01781 Richard Beer-Hofmann an Arthur Schnitzler, 6. [7. 1908]}
\nopagebreak\mylabel{L01781v}
\rehead{ }\normalsize\beginnumbering\briefempfaengerindex{Schnitzler, Arthur@\textsc{Schnitzler, Arthur}!zzzBeer-Hofmann, Richard@\emph{von Richard Beer-Hofmann}!1908-07-062@{6. [7. 1908]}|(be}
\toendnotes[C]{\smallbreak\pagebreak[2]}
\correspDesc{Versand  durch Richard Beer-Hofmann am 6. [7. 1908] in Wien
\newline{}Erhalt  durch Arthur Schnitzler im Zeitraum [7. 7. 1908
                  – 11. 7. 1908?] in Seis am Schlern}\toendnotes[C]{\smallbreak}
\Standort{CUL, Schnitzler, B 8.}
\physDesc{Bildpostkarte, 163 Zeichen
\newline{}Handschrift: schwarze Tinte, lateinische Kurrent
\newline{}Versand: Stempel: »\nobreak{}\oindex{IX., Alsergrund@\textbf{IX., Alsergrund}, \emph{Verwaltungsgebiet}|pwk}9/4 W{[}ien{]}, 6.V\textcolor{gray}{II.08}, \textcolor{gray}{8}\nobreak{}«.  
\newline{}Schnitzler: mit Bleistift datiert: »6/7 (?) 08« und beschriftet: »\textsc{Beerhfm}« 
\newline{}Ordnung: 1) mit Bleistift von unbekannter Hand nummeriert: »\strikeout{215}«  2) mit Bleistift von unbekannter Hand nummeriert:
                                    »216«}\pstart{}{\pb}Herrn\pend{}\pstart{} D\textsuperscript{r} Arthur Schnitzler\pend{}\pstart{}Seis a. Schlern\oindex{Seis am Schlern@\textbf{Seis am Schlern}|pw}\pend{}\pstart{}Villa Heufler\oindex{Villa Heufler@\textbf{Villa Heufler}, \emph{Beherbergungsgebäude}|pw}\pend{}\pstart{}Tirol\oindex{Südtirol@\textbf{Südtirol}, \emph{Verwaltungsgebiet}|pw}\pend{}{\bigskip}
\pstart
           \noindent{}\centering{}{\pb}\textcolor{gray}{\textbf{Wien\oindex{Wien@\textbf{Wien}, \emph{Verwaltungsgebiet}|pw}. Panorama vom Türkenschanzpark\oindex{Wien@\textbf{Wien}!XVIII., Währing@\textbf{XVIII., Währing}!Türkenschanzpark@\textbf{Türkenschanzpark}, \emph{Park}|pw} aus.}}\pend
           \vspace{1em}
\pstart{}{\pb}Lieber Arthur!\pend\vspace{0.5em}
\pstart
           Wir wollen von 14{ }\introOben{}Juli\introOben{} an in \uline{Strobl\oindex{Strobl@\textbf{Strobl}, \emph{Verwaltungsgebiet}|pw}}{ }\uline{Hôtel a See\oindex{Hotel am See@\textbf{Hotel am See}, \emph{Hotel}|pw}} sein.\pend
           
\pstart
           Herzliche Grüsse Ihnen u. den Ihren.{\\[\baselineskip]}\spacefill\mbox{Richard}\pend
           \leftskip=0em{}\selectlanguage{ngerman}\endnumbering\briefempfaengerindex{Schnitzler, Arthur@\textsc{Schnitzler, Arthur}!zzzBeer-Hofmann, Richard@\emph{von Richard Beer-Hofmann}!1908-07-062@{6. [7. 1908]}|)be}\mylabel{L01781h}  \newcommand{\dateiname}{L01781}\newcommand{\titel}{Richard Beer-Hofmann an Arthur Schnitzler, 6. [7. 1908]}\newcommand{\editorInnen}{Martin Anton Müller und Gerd-Hermann Susen}%% latex-leseansicht-abspann.tex
%% Abspann für die Leseansicht.
%% Der Schalter \ifkorrekturansicht ist bereits durch den Vorspann gesetzt.

%% latex-abspann.tex
%% Gemeinsamer Abspann für Korrekturansicht und Leseansicht.
%% Setzt den Schalter \ifkorrekturansicht voraus (gesetzt in den
%% einbindenden Dateien latex-korrekturansicht-abspann.tex bzw.
%% latex-leseansicht-abspann.tex).
%% ---------------------------------------------------------------

\normalsize

% Das esempio-Environment wird nur in der Leseansicht benötigt
\ifkorrekturansicht\else
\newenvironment{esempio}[3]%
{
    \vspace{1.5ex}
    \rlap{\underline{#1}}
    \par
    \setlength{\parindent}{0cm}
    \nopagebreak
    \leftskip=#2cm
    \rightskip=#3cm
}
{
    \par
}
\fi

\doendnotes{C}
\bigskip
\vfill

\clearpage

\footnotesize

\ifkorrekturansicht
  \lohead{\textsc{register}}
\fi

% theindex-Environment neu definieren ohne reledmac
\makeatletter
\renewenvironment{theindex}{%
  \ifkorrekturansicht
    \section*{\indexname}%
  \else
    \subsubsection*{Index der erwähnten Entitäten}%
  \fi
  \setlength{\parindent}{0pt}%
  \setlength{\parskip}{0pt plus 0.3pt}%
  \let\item\@idxitem
}{%
  \ifkorrekturansicht\clearpage\fi
}
\makeatother

\IfFileExists{\jobname-pw.ind}{\input{\jobname-pw.ind}}{}

% Quellenangabe nur in der Leseansicht
\ifkorrekturansicht\else
% Fallback-Definitionen, falls die .tex-Datei \titel etc. nicht gesetzt hat
\providecommand{\titel}{}
\providecommand{\editorInnen}{}
\providecommand{\dateiname}{\jobname}

\vspace{3cm}

\vfill

\footnotesize
\textsc{Quelle}: \titel. Herausgegeben von {\editorInnen}. In: \emph{Arthur Schnitzler: Briefwechsel mit Autorinnen und Autoren}.
 Digitale Edition, https://schnitzler-briefe.acdh.oeaw.ac.at/{\dateiname}.html (Stand \today)
\fi

\end{document}


