%% latex-korrekturansicht-vorspann.tex
%% Vorspann für die Korrekturansicht.
%% Lädt die gemeinsame Datei latex-vorspann.tex mit gesetztem Schalter.

\newif\ifkorrekturansicht
\korrekturansichttrue

\input{../tex-inputs/latex-vorspann}


\section[Bertha von Suttner an Arthur Schnitzler, 8. 12. 1913]{L02161 Bertha von Suttner an Arthur Schnitzler, 8. 12. 1913}
\nopagebreak\mylabel{L02161v}
\rehead{ }\normalsize\beginnumbering\briefempfaengerindex{Schnitzler, Arthur@\textsc{Schnitzler, Arthur}!zzzSuttner, Bertha von@\emph{von Bertha von Suttner}!1913-12-081@{8. 12. 1913}|(be}
\toendnotes[C]{\smallbreak\pagebreak[2]}\Standort{CUL, Schnitzler, B 104.}
\physDesc{Brief, 1 Blatt, 2 Seiten, 341 Zeichen (aufgeprägte Krone in Golddruck)
\newline{}Handschrift: schwarze Tinte, lateinische Kurrent
\newline{}Schnitzler: mit Bleistift beschriftet: »\textsc{Suttner}« }\Standort{DLA, A:Schnitzler, HS.NZ85.1.4773.}
\physDesc{maschinenschriftliche Abschrift1 Blatt, 1 Seite, 341 Zeichen
\newline{}Schreibmaschine}\toendnotes[C]{\smallbreak}
\pstart
           \centering{}{\pb}8/12 13\pend
           
\pstart{}Hochgeehrter Herr D\textsuperscript{r} Schnitzler\pend\vspace{0.5em}
\pstart
           Leider bin ich am Mittwoch nicht frei; auch am künftigen
                  Donnerstag nicht. Kann ich Sie und Ihre liebenswürdige Gattin\pwindex{Schnitzler, Olga 17.01.1882 – 13.01.1970@\textsc{Schnitzler, Olga} (17.01.1882 – 13.01.1970), \emph{Schauspieler/Schauspielerin, Sänger/Sängerin}|pwv} am \uuline{Freitag} erwarten?\pend
           
\pstart
           Ich sehnte mich schon lange nach dem angesagten Besuch\pend
           
\pstart
           Bitte also nur eine Zeile, ob {\pb}Freitag, oder, wenn es Ihnen bequemer ist, Samstag\pend
           
\pstart
           Mit herzlichsten Grüssen{\\[\baselineskip]}Ihre erg{\\[\baselineskip]}\spacefill\mbox{B Suttner}\pend
           \leftskip=0em{}\selectlanguage{ngerman}\endnumbering\briefempfaengerindex{Schnitzler, Arthur@\textsc{Schnitzler, Arthur}!zzzSuttner, Bertha von@\emph{von Bertha von Suttner}!1913-12-081@{8. 12. 1913}|)be}\mylabel{L02161h}  \normalsize

\doendnotes{C}
\bigskip
\vfill

\clearpage

\footnotesize

\lohead{\textsc{register}}

% Definiere theindex-Environment komplett neu ohne reledmac
\makeatletter
\renewenvironment{theindex}{%
  \section*{\indexname}%
  \setlength{\parindent}{0pt}%
  \setlength{\parskip}{0pt plus 0.3pt}%
  \let\item\@idxitem
}{%
  \clearpage
}
\makeatother

\IfFileExists{\jobname-pw.ind}{\input{\jobname-pw.ind}}{}

\end{document}

      