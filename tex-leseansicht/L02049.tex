%% latex-leseansicht-vorspann.tex
%% Vorspann für die Leseansicht.
%% Lädt die gemeinsame Datei latex-vorspann.tex mit nicht gesetztem Schalter.

\newif\ifkorrekturansicht
\korrekturansichtfalse

\input{../tex-inputs/latex-vorspann}


         
         \renewcommand{\erwaehntePersonen}{Personen: Georg Brandes, Helene Heyman}
         \renewcommand{\erwaehnteOrte}{Orte: Det Kongelige Teater, Deutschland, Dänemark, Kopenhagen, Mainz, Rosbæksvej, Strandvejen, Wien, Österreich}
         \renewcommand{\erwaehnteWerke}{Werke: Das weite Land. Tragikomödie in fünf Akten, Der Weg ins Freie. Roman}
               \section[Georg Brandes an Arthur Schnitzler, vor dem 2. 12. 1911]{ Georg Brandes an Arthur Schnitzler, vor dem 2. 12. 1911}\nopagebreak\mylabel{v}\rehead{ }\begin{ledgroupsized}[t]{13cm}\normalsize\beginnumbering\briefempfaengerindex{Schnitzler, Arthur@\textsc{Schnitzler, Arthur}!zzzBrandes, Georg@\emph{von Georg Brandes}!1911-12-021@{vor dem 2. 12. 1911}|(be} \toendnotes[C]{\smallbreak\pagebreak[2]} \Standort{DLA, A:Schnitzler, HS.NZ85.1.3446,3.}
\physDesc{Visitenkarte, 231 Zeichen
\newline{}Handschrift: schwarze Tinte, lateinische Kurrent}\toendnotes[C]{\smallbreak}\pstart
           \noindent{}\centering{}{\pb}\textcolor{gray}{\textbf{Georg Brandes}}\pend
           \pstart
           \noindent{}Verehrter Freund, lieber Arthur Schnitzler. Frau \label{K_L02049-1v}\edtext{Helene Heyman\pwindex{Heyman, Helene 15.01.1876 – 1950@\textsc{Heyman, Helene} (15.01.1876 – 1950)|pw}}{\lemma{\textnormal{\emph{Helene Heyman}}}\Cendnote{\textnormal{Die undatierte Karte geht dem Brief von
                     Helene Heyman\pwindex{Heyman, Helene 15.01.1876 – 1950@\textsc{Heyman, Helene} (15.01.1876 – 1950)|pwk} voran (\emph{Deutsches Literaturarchiv}, A:Schnitzler,
                     HS.NZ85.1.3446,1): »Kopenhagen\oindex{Kopenhagen@\textbf{Kopenhagen}|pw}, \textcolor{gray}{\textbf{ROSBÆKSVEJ STRANDVEJ}}\oindex{Rosbæksvej@\textbf{Rosbæksvej}|pw}\oindex{Strandvejen@\textbf{Strandvejen}|pw}{ }2./12 1911{ / }Sehr geehrter Herr Schnitzler!{ / }Ich bitte Sie zu entschuldigen, dass ich mich, ehedem ich Ihnen vollständig
                        fremd bin, an Sie wende mit der Vorfrage Ihr neuestes Werk »Das weite Land\pwindex{Schnitzler, Arthur 15.05.1862 – 21.10.1931@\textsc{Schnitzler, Arthur} (15.05.1862 – 21.10.1931), \emph{Schriftsteller, Mediziner}!weite Land. Tragikomoedie in fuenf Akten1910-10-20@\strich\emph{Das weite Land. Tragikomödie in fünf Akten} {[}1910-10-20{]}|pw}« auf dänisch\oindex{Daenemark@\textbf{Dänemark}|pw} übersetzen zu dürfen, und es womöglich hier am
                           königlichen Theater\oindex{Det Kongelige Teater@\textbf{Det Kongelige Teater}|pw} zur Aufführung
                        zu bringen.{ / }Herr Professor Georg Brandes\pwindex{Brandes, Georg 04.02.1842 – 19.02.1927@\textsc{Brandes, Georg} (04.02.1842 – 19.02.1927)|pw} hatte
                        die grosse Freundlichkeit mich bei Ihnen zu introducieren, da ich ja nicht
                        erwarten konnte, dass Sie einer Fremden Ihre Arbeit anvertrauen wollten.
                        {\pb}Ich darf wohl hinzufügen, dass ich eine grosse Verehrerin Ihrer Werke
                        bin, besonders »Der Weg ins Freie\pwindex{Schnitzler, Arthur 15.05.1862 – 21.10.1931@\textsc{Schnitzler, Arthur} (15.05.1862 – 21.10.1931), \emph{Schriftsteller, Mediziner}!Weg ins Freie. Roman1.1.1908 – 1.6.1908@\strich\emph{Der Weg ins Freie. Roman} {[}1.1.1908 – 1.6.1908{]}|pw}«, hat
                        mich aufs höchste interessiert und gefesselt. Alle Menschen, die in dem
                        Roman vorkommen, standen Einem geradezu nahe, man fühlte mit ihnen. Die
                        grösste Lust hatte ich schon damals an Sie zu schreiben und Ihnen zu
                        erzählen wie ganz anders das Verhältnis zwischen Christen und Juden hier
                        ist, es giebt wirklich nur wenig richtigen Antisemitismus hier, man merkt
                        ihn jedenfalls sehr selten. Aber die Juden hier sind auch sehr
                        ver{\pb}nünftig, gemischte Ehen gehören zur Tagesordnung, und sie sind nicht
                        auf pekuniären Vorteil taxiert. Im Verkehr mit Christen wird die
                        Religionsfrage so wenig wie möglich berührt, während gerade die deutschen\oindex{Deutschland@\textbf{Deutschland}|pw} und österreichischen\oindex{Oesterreich@\textbf{Österreich}|pw} Juden immer und ewig auf dies heikle
                        Thema zurückkommen, und dadurch die Kluft zwischen den Rassen nur
                        erweitern.{ / }Ich selbst bin Rheinländerin\oindex{Mainz@\textbf{Mainz}|pwv} und kam als ganz junges Mädchen hierher, und bin nun
                        schon 17 Jahre hier verheiratet, also ich kann den Unterschied nur {\pb}zu
                        gut merken.{ / }Entschuldigen Sie, dass ich so frei war mich an Sie zu wenden, hoffentlich
                        wird Ihre Antwort eine günstige für mich sein.{ / }Mit vorzüglicher Hochachtung zeichnet{ / }Ihre ergebene{ / }Helene Heyman.«}}}\label{K_L02049-1h} wünscht von mir bei Ihnen introducirt zu werden. Sie können der
               Dame vollständig vertrauen. Deutsch geboren versteht und schreibt sie Dänisch\oindex{Daenemark@\textbf{Dänemark}|pw} mit vollkommenster Sicherheit.\pend
           
         
         \endnumbering\mylabel{h}\end{ledgroupsized}  \newcommand{\dateiname}{L02049}\newcommand{\titel}{Georg Brandes an Arthur Schnitzler, vor dem 2. 12. 1911}\newcommand{\editorInnen}{Martin Anton Müller und Gerd-Hermann Susen}%% latex-leseansicht-abspann.tex
%% Abspann für die Leseansicht.
%% Der Schalter \ifkorrekturansicht ist bereits durch den Vorspann gesetzt.

%% latex-abspann.tex
%% Gemeinsamer Abspann für Korrekturansicht und Leseansicht.
%% Setzt den Schalter \ifkorrekturansicht voraus (gesetzt in den
%% einbindenden Dateien latex-korrekturansicht-abspann.tex bzw.
%% latex-leseansicht-abspann.tex).
%% ---------------------------------------------------------------

\normalsize

% Das esempio-Environment wird nur in der Leseansicht benötigt
\ifkorrekturansicht\else
\newenvironment{esempio}[3]%
{
    \vspace{1.5ex}
    \rlap{\underline{#1}}
    \par
    \setlength{\parindent}{0cm}
    \nopagebreak
    \leftskip=#2cm
    \rightskip=#3cm
}
{
    \par
}
\fi

\doendnotes{C}
\bigskip
\vfill

\clearpage

\footnotesize

\ifkorrekturansicht
  \lohead{\textsc{register}}
\fi

% theindex-Environment neu definieren ohne reledmac
\makeatletter
\renewenvironment{theindex}{%
  \ifkorrekturansicht
    \section*{\indexname}%
  \else
    \subsubsection*{Index der erwähnten Entitäten}%
  \fi
  \setlength{\parindent}{0pt}%
  \setlength{\parskip}{0pt plus 0.3pt}%
  \let\item\@idxitem
}{%
  \ifkorrekturansicht\clearpage\fi
}
\makeatother

\IfFileExists{\jobname-pw.ind}{\input{\jobname-pw.ind}}{}

% Quellenangabe nur in der Leseansicht
\ifkorrekturansicht\else
% Fallback-Definitionen, falls die .tex-Datei \titel etc. nicht gesetzt hat
\providecommand{\titel}{}
\providecommand{\editorInnen}{}
\providecommand{\dateiname}{\jobname}

\vspace{3cm}

\vfill

\footnotesize
\textsc{Quelle}: \titel. Herausgegeben von {\editorInnen}. In: \emph{Arthur Schnitzler: Briefwechsel mit Autorinnen und Autoren}.
 Digitale Edition, https://schnitzler-briefe.acdh.oeaw.ac.at/{\dateiname}.html (Stand \today)
\fi

\end{document}


      