%% latex-leseansicht-vorspann.tex
%% Vorspann für die Leseansicht.
%% Lädt die gemeinsame Datei latex-vorspann.tex mit nicht gesetztem Schalter.

\newif\ifkorrekturansicht
\korrekturansichtfalse

\input{../tex-inputs/latex-vorspann}

\begin{center}
            \textcolor{red}{ENTWURF, NICHT FERTIG KORRIGIERT}
                      \end{center}
            
         
         \renewcommand{\erwaehntePersonen}{Personen: Paul Lindau, Paul Marx, Olga Schnitzler, Elisabeth Steinrück}
         \renewcommand{\erwaehnteOrte}{Orte: Berlin, Dessauer Straße, Wien}
         \renewcommand{\erwaehnteWerke}{Werke: Marionetten. Drei Einakter}
               \section[ Paul Goldmann an Olga Gussmann, 9. 3. {[}1901{]}]{ Paul Goldmann an Olga Gussmann, 9. 3. {[}1901{]}}\nopagebreak\mylabel{v}\rehead{ }\begin{ledgroupsized}[t]{13cm}\normalsize\beginnumbering \toendnotes[C]{\smallbreak\pagebreak[2]} \Standort{DLA, A:Schnitzler, HS.NZ85.1.5247.}
\physDesc{Brief, 1 Blatt, 4 Seiten
\newline{}Handschrift: blaue Tinte, deutsche Kurrent\newline{}Ordnung: 1) mit Bleistift von Arthur
                                    Schnitzler\pwindex{Schnitzler, Arthur 15.05.1862 – 21.10.1931@\textsc{Schnitzler, Arthur} (15.05.1862 – 21.10.1931), \emph{Schriftsteller, Mediziner}|pw} das Jahr »1901« vermerkt  2) mit rotem Buntstift von Arthur
                                    Schnitzler\pwindex{Schnitzler, Arthur 15.05.1862 – 21.10.1931@\textsc{Schnitzler, Arthur} (15.05.1862 – 21.10.1931), \emph{Schriftsteller, Mediziner}|pw} den ersten Absatz fast vollständig unterstrichen
                                 und mit »\textsc{Marionetten\pwindex{Schnitzler, Arthur 15.05.1862 – 21.10.1931@\textsc{Schnitzler, Arthur} (15.05.1862 – 21.10.1931), \emph{Schriftsteller, Mediziner}!Marionetten. Drei Einakter1906@\strich\emph{Marionetten. Drei Einakter} {[}1906{]}|pw}}« annotiert sowie eine weitere Unterstreichung}\toendnotes[C]{\smallbreak}\pstart
           \noindent{}\raggedleft{}{\pb}\textcolor{gray}{\textbf{DESSAUERSTRASSE 19\oindex{Dessauer Strasse@\textbf{Dessauer Straße}|pw}}}\pend
           \pstart
           Berlin\oindex{Berlin@\textbf{Berlin}|pw}, 9. März.\pend
           \pstart\center{}Liebes Fräulein \textsc{Olga},\pend\pstart
           \textsc{Dr. Schnitzler\pwindex{Schnitzler, Arthur 15.05.1862 – 21.10.1931@\textsc{Schnitzler, Arthur} (15.05.1862 – 21.10.1931), \emph{Schriftsteller, Mediziner}|pw}s}\label{K_L03526-1v}\edtext{Stück\textcolor{red}{\textsuperscript{\textbf{KEY}}}}{\lemma{\textnormal{\emph{Stück}}}\Cendnote{\textnormal{}}}\label{K_L03526-1h} kam infolge
               unzureichender Darſtellung nicht zur rechten Wirkung. Auch hatte man die
               Unverſchämtheit und Taktloſigkeit, es ganz zuletzt, um \strikeout{\textcolor{gray}{½}}{ }½ 11 Uhr Abends, nachdem das Publikum bereits durch ein \label{K_L03526-2v}\edtext{überlanges Programm}{\lemma{\textnormal{\emph{überlanges Programm}}}\Cendnote{\textnormal{}}}\label{K_L03526-2h} ermüdet war, aufzuführen.\pend
           \pstart
           \textsc{Dr. Schnitzler\pwindex{Schnitzler, Arthur 15.05.1862 – 21.10.1931@\textsc{Schnitzler, Arthur} (15.05.1862 – 21.10.1931), \emph{Schriftsteller, Mediziner}|pw}s}{ }\label{K_L03526-3v}\edtext{Anweſenheit}{\lemma{\textnormal{\emph{Anweſenheit}}}\Cendnote{\textnormal{Schnitzler\pwindex{Schnitzler, Arthur 15.05.1862 – 21.10.1931@\textsc{Schnitzler, Arthur} (15.05.1862 – 21.10.1931), \emph{Schriftsteller, Mediziner}|pwk} war zwischen 3. 3. 1901 und 10. 3. 1901 in Berlin\oindex{Berlin@\textbf{Berlin}|pwk}. Goldmann\pwindex{Goldmann, Paul 31.01.1865 – 25.09.1935@\textsc{Goldmann, Paul} (31.01.1865 – 25.09.1935), \emph{Schriftsteller, Journalist}|pwk} traf er am 6. 3. 1901, 7. 3. 1901, 8. 3. 1901 und 10. 3. 1901.}}}\label{K_L03526-3h}\strikeout{h\textcolor{gray}{×}\-\textcolor{gray}{×}\-\textcolor{gray}{×}} thut mir ſehr wohl, und ich werde mich nachher nur umſo einſamer fühlen.\pend
           \pstart
           Ich gratulire Ihnen zu ihren \label{K_L03526-4v}\edtext{ſchauſpieleriſchen Erfolgen}{\lemma{\textnormal{\emph{ſchauſpieleriſchen Erfolgen}}}\Cendnote{\textnormal{}}}\label{K_L03526-4h}, von denen Sie mir mit ſo überzeugender Beredſamkeit
               berichten. {\pb}Selbſtverſtändlich werde ich bei \label{K_L03526-5v}\edtext{\textsc{Lindau\pwindex{Lindau, Paul 03.06.1839 – 31.01.1919@\textsc{Lindau, Paul} (03.06.1839 – 31.01.1919), \emph{Schriftsteller, Kritiker, Theaterleiter}|pw}}}{\lemma{\textnormal{\emph{Lindau}}}\Cendnote{\textnormal{}}}\label{K_L03526-5h}, ſoweit es in
               meinen ſchwachen Kräften ſteht, Ihnen behilflich ſein.\pend
           \pstart
           Zerbrechen Sie ſich nicht den Kopf über das Künftige. Erſtens nützt es doch nichts,
               und zweitens kommt das Künftige ſchon von ſelbſt, wenn man jung iſt und Talent
               hat.\pend
           \pstart
           Ich würde mich freuen, wenn Sie \label{K_L03526-6v}\edtext{nach
                  Berlin\oindex{Berlin@\textbf{Berlin}|pw}}{\lemma{\textnormal{\emph{nach
                  Berlin}}}\Cendnote{\textnormal{}}}\label{K_L03526-6h} kämen. Dann hätte
               auch ich »doch wenigſtens eine bekannte Seele in der Stadt\oindex{Berlin@\textbf{Berlin}|pwv}« (wie Sie ſich in Bezug auf mich
               ausdrücken).\pend
           \pstart
           {\pb}Hoffentlich ſind Sie wieder in guter \label{K_L03526-7v}\edtext{Stimmung}{\lemma{\textnormal{\emph{Stimmung}}}\Cendnote{\textnormal{}}}\label{K_L03526-7h}, wenn dieſer Brief ankommt. Iſt das
               Leben wirklich ſo bitter? Ich finde aber, alle Bitterkeit macht auch nichts, wenn es
                  \strikeout{\textcolor{gray}{×}\-\textcolor{gray}{×}\-\textcolor{gray}{×}\-\textcolor{gray}{×}\-\textcolor{gray}{×}\-\textcolor{gray}{×}}{ }\introOben{}nur\introOben{} hier und da einen ſüßen Schluck gibt. Nur ganz ohne \strikeout{\textcolor{gray}{feſ}\textcolor{gray}{×}\-\textcolor{gray}{×}\-\textcolor{gray}{×}\-\textcolor{gray}{×}\-\textcolor{gray}{×}\textcolor{gray}{k}} ſüßen Schluck iſt es ſchwer zu tragen.\pend
           \pstart
           Ihr \label{K_L03526-17v}\edtext{Bild}{\lemma{\textnormal{\emph{Bild}}}\Cendnote{\textnormal{}}}\label{K_L03526-17h} ſoll
               willkommen ſein.\pend
           \pstart
           Ich habe Ihnen lange nicht geantwortet, weil ich wenig Zeit zum Schreiben habe und
               weil – weil – weil ich nicht recht wußte, {\pb}was ich
               Ihnen antworten ſollte.\pend
           \pstart
           Grüßen Sie Ihr Schweſterlein\pwindex{Steinrueck, Elisabeth 19.11.1885 – 07.04.1920@\textsc{Steinrück, Elisabeth} (19.11.1885 – 07.04.1920)|pwv} und ſeien Sie ſelbſt recht herzlich gegrüßt von {\\[\baselineskip]}Ihrem
               ergebenen {\\[\baselineskip]}\spacefill\mbox{Dr. Paul Goldmann}\pend
           \leftskip=0em{}\pstart
           \noindent{}Grüße an Herrn Paul\pwindex{Marx, Paul 21.07.1879 – 1956-10-30@\textsc{Marx, Paul} (21.07.1879 – 1956-10-30), \emph{Regisseur, Schauspieler}|pw}!\pend
           
         
         \endnumbering\mylabel{h}\end{ledgroupsized}\begin{anhang}\end{anhang}\newcommand{\dateiname}{L03526}\newcommand{\titel}{Paul Goldmann an Olga Gussmann, 9. 3. [1901]}\newcommand{\editorInnen}{Martin Anton Müller und Laura Untner}%% latex-leseansicht-abspann.tex
%% Abspann für die Leseansicht.
%% Der Schalter \ifkorrekturansicht ist bereits durch den Vorspann gesetzt.

%% latex-abspann.tex
%% Gemeinsamer Abspann für Korrekturansicht und Leseansicht.
%% Setzt den Schalter \ifkorrekturansicht voraus (gesetzt in den
%% einbindenden Dateien latex-korrekturansicht-abspann.tex bzw.
%% latex-leseansicht-abspann.tex).
%% ---------------------------------------------------------------

\normalsize

% Das esempio-Environment wird nur in der Leseansicht benötigt
\ifkorrekturansicht\else
\newenvironment{esempio}[3]%
{
    \vspace{1.5ex}
    \rlap{\underline{#1}}
    \par
    \setlength{\parindent}{0cm}
    \nopagebreak
    \leftskip=#2cm
    \rightskip=#3cm
}
{
    \par
}
\fi

\doendnotes{C}
\bigskip
\vfill

\clearpage

\footnotesize

\ifkorrekturansicht
  \lohead{\textsc{register}}
\fi

% theindex-Environment neu definieren ohne reledmac
\makeatletter
\renewenvironment{theindex}{%
  \ifkorrekturansicht
    \section*{\indexname}%
  \else
    \subsubsection*{Index der erwähnten Entitäten}%
  \fi
  \setlength{\parindent}{0pt}%
  \setlength{\parskip}{0pt plus 0.3pt}%
  \let\item\@idxitem
}{%
  \ifkorrekturansicht\clearpage\fi
}
\makeatother

\IfFileExists{\jobname-pw.ind}{\input{\jobname-pw.ind}}{}

% Quellenangabe nur in der Leseansicht
\ifkorrekturansicht\else
% Fallback-Definitionen, falls die .tex-Datei \titel etc. nicht gesetzt hat
\providecommand{\titel}{}
\providecommand{\editorInnen}{}
\providecommand{\dateiname}{\jobname}

\vspace{3cm}

\vfill

\footnotesize
\textsc{Quelle}: \titel. Herausgegeben von {\editorInnen}. In: \emph{Arthur Schnitzler: Briefwechsel mit Autorinnen und Autoren}.
 Digitale Edition, https://schnitzler-briefe.acdh.oeaw.ac.at/{\dateiname}.html (Stand \today)
\fi

\end{document}


      