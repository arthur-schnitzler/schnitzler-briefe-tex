%% latex-leseansicht-vorspann.tex
%% Vorspann für die Leseansicht.
%% Lädt die gemeinsame Datei latex-vorspann.tex mit nicht gesetztem Schalter.

\newif\ifkorrekturansicht
\korrekturansichtfalse

\input{../tex-inputs/latex-vorspann}


\section[ Paul Goldmann an Olga Gussmann, 9. 3. {[}1901{]}]{L03526 Paul Goldmann an Olga Gussmann,  9. 3. [1901]}
\nopagebreak\mylabel{L03526v}
\rehead{ }\normalsize\beginnumbering\briefempfaengerindex{Schnitzler, Olga@\textsc{Schnitzler, Olga}!zzzGoldmann, Paul@\emph{von Paul Goldmann}!1901-03-091@{9. 3. [1901]}|(be}
\toendnotes[C]{\smallbreak\pagebreak[2]}
\correspDesc{Versand  durch Paul Goldmann am 9. 3. [1901] in Berlin
\newline{}Erhalt  durch Olga Gussmann im Zeitraum [10. 3. 1901
                  – 14. 3. 1901?] in Wien}\toendnotes[C]{\smallbreak}
\Standort{DLA, A:Schnitzler, HS.NZ85.1.5247.}
\physDesc{Brief, 1 Blatt, 4 Seiten, 1536 Zeichen
\newline{}Handschrift: blaue Tinte, deutsche Kurrent
\newline{}Ordnung: 1) mit Bleistift von Arthur Schnitzler das
                                 Jahr »1901« vermerkt  2) mit rotem Buntstift von Arthur Schnitzler den
                                 ersten Absatz fast vollständig unterstrichen und mit »\textsc{Marionetten\pwindex{Schnitzler, Arthur 15.\,5.\,1862 Wien – 21.\,10.\,1931 ebd.@\textsc{Schnitzler, Arthur} (15.\,5.\,1862 Wien – 21.\,10.\,1931 ebd.), \emph{Schriftsteller, Mediziner}!Marionetten. Drei Einakter@\strich\emph{Marionetten. Drei Einakter}|pw}}« annotiert sowie eine weitere Unterstreichung}\toendnotes[C]{\smallbreak}
\pstart
           \raggedleft{}{\pb}\textcolor{gray}{\textbf{DESSAUERSTRASSE 19\oindex{Dessauer Straße@\textbf{Dessauer Straße}, \emph{Straße}|pw}}}\pend
           
\pstart
           Berlin\oindex{Berlin@\textbf{Berlin}, \emph{Hauptstadt}|pw}, 9. März.\pend
           
\pstart\center{}Liebes Fräulein \textsc{Olga},\pend\vspace{0.5em}
\pstart
           \textsc{Dr. Schnitzlers}{ }\label{K_L03526-1v}\edtext{Stück\pwindex{Schnitzler, Arthur 15.\,5.\,1862 Wien – 21.\,10.\,1931 ebd.@\textsc{Schnitzler, Arthur} (15.\,5.\,1862 Wien – 21.\,10.\,1931 ebd.), \emph{Schriftsteller, Mediziner}!Zum großen Wurstel. Burleske in einem Akt@\strich\emph{Zum großen Wurstel. Burleske in einem Akt}|pwv}}{\lemma{\textnormal{\emph{Stück}}}\Cendnote{\textnormal{\emph{Zum großen Wurstel}\pwindex{Schnitzler, Arthur 15.\,5.\,1862 Wien – 21.\,10.\,1931 ebd.@\textsc{Schnitzler, Arthur} (15.\,5.\,1862 Wien – 21.\,10.\,1931 ebd.), \emph{Schriftsteller, Mediziner}!Zum großen Wurstel. Burleske in einem Akt@\strich\emph{Zum großen Wurstel. Burleske in einem Akt}|pwk} aus dem \emph{Marionetten}\pwindex{Schnitzler, Arthur 15.\,5.\,1862 Wien – 21.\,10.\,1931 ebd.@\textsc{Schnitzler, Arthur} (15.\,5.\,1862 Wien – 21.\,10.\,1931 ebd.), \emph{Schriftsteller, Mediziner}!Marionetten. Drei Einakter@\strich\emph{Marionetten. Drei Einakter}|pwk}-Zyklus, am 8. 3. 1901 am Berlin\oindex{Berlin@\textbf{Berlin}, \emph{Hauptstadt}|pwk}er \emph{Überbrettl}\orgindex{Überbrettl@Überbrettl|pwk}
                  uraufgeführt}}}\label{K_L03526-1} kam infolge unzureichender Darſtellung nicht zur rechten
               Wirkung. Auch hatte man die Unverſchämtheit und Taktloſigkeit, es ganz zuletzt, um
                  \strikeout{\textcolor{gray}{½}}{ }½ 11 Uhr Abends, nachdem das Publikum bereits durch ein überlanges
               Programm ermüdet war, aufzuführen.\pend
           
\pstart
           \textsc{Dr. Schnitzlers}{ }\label{K_L03526-2v}\edtext{Anweſenheit}{\lemma{\textnormal{\emph{Anwesenheit}}}\Cendnote{\textnormal{Schnitzler war zwischen 3. 3. 1901 und 10. 3. 1901 in Berlin\oindex{Berlin@\textbf{Berlin}, \emph{Hauptstadt}|pwk}. Goldmann\pwindex{Goldmann, Paul 31.\,1.\,1865 Breslau – 25.\,9.\,1935 Wien@\textsc{Goldmann, Paul} (31.\,1.\,1865 Breslau – 25.\,9.\,1935 Wien), \emph{Schriftsteller, Journalist}|pwk} traf er nachweislich am 6. 3. 1901, 7. 3. 1901, 8. 3. 1901 und 10. 3. 1901.}}}\label{K_L03526-2}{ }\strikeout{h\textcolor{gray}{ier}} thut mir{ }ſehr wohl, und ich werde mich nachher nur umſo einſamer fühlen.\pend
           
\pstart
           Ich gratulire Ihnen zu Ihren \label{K_L03526-3v}\edtext{ſchauſpieleriſchen Erfolgen}{\lemma{\textnormal{\emph{schauspielerischen Erfolgen}}}\Cendnote{\textnormal{Gussmann\pwindex{Schnitzler, Olga 17.\,1.\,1882 Wien – 13.\,1.\,1970 Lugano@\textsc{Schnitzler, Olga} (17.\,1.\,1882 Wien – 13.\,1.\,1970 Lugano), \emph{Schauspielerin, Sängerin}|pwk} studierte Schauspiel am \emph{Konservatorium}\orgindex{Konservatorium der Gesellschaft der Musikfreunde@Konservatorium der Gesellschaft der Musikfreunde|pwk}.}}}\label{K_L03526-3}, von denen Sie mir mit{ }ſo überzeugender Beredſamkeit berichten. {\pb}Selbſtverſtändlich werde ich bei \label{K_L03526-4v}\edtext{\textsc{Lindau\pwindex{Lindau, Paul 3.\,6.\,1839 Magdeburg – 31.\,1.\,1919 Berlin@\textsc{Lindau, Paul} (3.\,6.\,1839 Magdeburg – 31.\,1.\,1919 Berlin), \emph{Schriftsteller, Kritiker, Theaterleiter}|pw}}}{\lemma{\textnormal{\emph{Lindau}}}\Cendnote{\textnormal{Paul Lindau\pwindex{Lindau, Paul 3.\,6.\,1839 Magdeburg – 31.\,1.\,1919 Berlin@\textsc{Lindau, Paul} (3.\,6.\,1839 Magdeburg – 31.\,1.\,1919 Berlin), \emph{Schriftsteller, Kritiker, Theaterleiter}|pwk} leitete das \emph{Berliner Theater}\orgindex{Berliner Theater@Berliner Theater|pwk}. Siehe auch A. S.: \emph{Tagebuch}, 3. 8. 1901 und XXXX Auszeichnungsfehler: Dokument L03059 nicht gefunden.}}}\label{K_L03526-4},{ }ſoweit es in meinen{ }ſchwachen Kräften{ }ſteht, Ihnen behilflich{ }ſein.\pend
           
\pstart
           Zerbrechen Sie{ }ſich nicht den Kopf über das Künftige. Erſtens nützt es doch nichts,
               und zweitens kommt das Künftige{ }ſchon von{ }ſelbſt, wenn man jung iſt und Talent
               hat.\pend
           
\pstart
           Ich würde mich freuen, wenn Sie nach Berlin\oindex{Berlin@\textbf{Berlin}, \emph{Hauptstadt}|pw}
               kämen. Dann hätte auch ich »doch wenigſtens eine bekannte Seele in der Stadt\oindex{Berlin@\textbf{Berlin}, \emph{Hauptstadt}|pwv}« (wie Sie{ }ſich in Bezug
               auf mich ausdrücken).\pend
           
\pstart
           {\pb}Hoffentlich{ }ſind Sie wieder in guter \label{K_L03526-5v}\edtext{Stimmung}{\lemma{\textnormal{\emph{Stimmung}}}\Cendnote{\textnormal{Womöglich handelt es sich an dieser Stelle um eine Anspielung auf Olga Gussmanns\pwindex{Schnitzler, Olga 17.\,1.\,1882 Wien – 13.\,1.\,1970 Lugano@\textsc{Schnitzler, Olga} (17.\,1.\,1882 Wien – 13.\,1.\,1970 Lugano), \emph{Schauspielerin, Sängerin}|pwk} Eifersucht, die sie Schnitzler gegenüber, wie dem \emph{Tagebuch}\pwindex{Schnitzler, Arthur 15.\,5.\,1862 Wien – 21.\,10.\,1931 ebd.@\textsc{Schnitzler, Arthur} (15.\,5.\,1862 Wien – 21.\,10.\,1931 ebd.), \emph{Schriftsteller, Mediziner}!Tagebuch@\strich\emph{Tagebuch}|pwk} zu entnehmen ist, in dieser Zeit mehrfach äußerte.}}}\label{K_L03526-5}, wenn
               dieſer Brief ankommt. Iſt das Leben wirklich{ }ſo bitter? Ich finde aber, alle
               Bitterkeit macht auch nichts, wenn es \strikeout{\textcolor{gray}{richtig}}{ }\introOben{}nur\introOben{} hier und da einen{ }ſüßen Schluck gibt. Nur ganz ohne \strikeout{\textcolor{gray}{Schluck}}{ }ſüßen Schluck iſt es{ }ſchwer zu tragen.\pend
           
\pstart
           Ihr \label{K_L03526-6v}\edtext{Bild\pwindex{\textcolor{red}{\textsuperscript{XXXX indx1}}!Portraitfoto von Olga Gussmann]@\strich\emph{[Portraitfoto von Olga Gussmann]}|pwv}}{\lemma{\textnormal{\emph{Bild}}}\Cendnote{\textnormal{Siehe XXXX Auszeichnungsfehler: Dokument L03527 nicht gefunden.
               }}}\label{K_L03526-6}{ }ſoll willkommen{ }ſein.\pend
           
\pstart
           Ich habe Ihnen lange nicht geantwortet, weil ich wenig Zeit zum Schreiben habe und
               weil – weil – weil ich nicht recht wußte, {\pb}was ich
               Ihnen antworten{ }ſollte.\pend
           
\pstart
           Grüßen Sie Ihr Schweſterlein\pwindex{Steinrück, Elisabeth 19.\,11.\,1885 – 7.\,4.\,1920 Partenkirchen@\textsc{Steinrück, Elisabeth} (19.\,11.\,1885 – 7.\,4.\,1920 Partenkirchen)|pwv} und{ }ſeien Sie{ }ſelbſt recht herzlich gegrüßt von {\\[\baselineskip]}Ihrem
               ergebenen {\\[\baselineskip]}\spacefill\mbox{Dr. Paul Goldmann}\pend
           \leftskip=0em{}
\pstart
           \noindent{}Grüße an Herrn \label{K_L03526-7v}\edtext{\textsc{Paul\pwindex{Marx, Paul 21.\,7.\,1879 Wien – 30.\,10.\,1956 ebd.@\textsc{Marx, Paul} (21.\,7.\,1879 Wien – 30.\,10.\,1956 ebd.), \emph{Regisseur, Schauspieler}|pw}}}{\lemma{\textnormal{\emph{Paul}}}\Cendnote{\textnormal{Paul Marx\pwindex{Marx, Paul 21.\,7.\,1879 Wien – 30.\,10.\,1956 ebd.@\textsc{Marx, Paul} (21.\,7.\,1879 Wien – 30.\,10.\,1956 ebd.), \emph{Regisseur, Schauspieler}|pwk} war zwischen 1900 und 1903 der Partner von Olgas\pwindex{Schnitzler, Olga 17.\,1.\,1882 Wien – 13.\,1.\,1970 Lugano@\textsc{Schnitzler, Olga} (17.\,1.\,1882 Wien – 13.\,1.\,1970 Lugano), \emph{Schauspielerin, Sängerin}|pwk} Schwester Elisabeth Gussmann\pwindex{Steinrück, Elisabeth 19.\,11.\,1885 – 7.\,4.\,1920 Partenkirchen@\textsc{Steinrück, Elisabeth} (19.\,11.\,1885 – 7.\,4.\,1920 Partenkirchen)|pwk} und, wie Olga und Elisabeth\pwindex{Schnitzler, Olga 17.\,1.\,1882 Wien – 13.\,1.\,1970 Lugano@\textsc{Schnitzler, Olga} (17.\,1.\,1882 Wien – 13.\,1.\,1970 Lugano), \emph{Schauspielerin, Sängerin}|pwk}\pwindex{Steinrück, Elisabeth 19.\,11.\,1885 – 7.\,4.\,1920 Partenkirchen@\textsc{Steinrück, Elisabeth} (19.\,11.\,1885 – 7.\,4.\,1920 Partenkirchen)|pwk}, Schüler am \emph{Konservatorium}\orgindex{Konservatorium der Gesellschaft der Musikfreunde@Konservatorium der Gesellschaft der Musikfreunde|pwk}.}}}\label{K_L03526-7}!\pend
           \selectlanguage{ngerman}\endnumbering\briefempfaengerindex{Schnitzler, Olga@\textsc{Schnitzler, Olga}!zzzGoldmann, Paul@\emph{von Paul Goldmann}!1901-03-091@{9. 3. [1901]}|)be}\mylabel{L03526h}  \newcommand{\dateiname}{L03526}\newcommand{\titel}{Paul Goldmann an Olga Gussmann, 9. 3. [1901]}\newcommand{\editorInnen}{Martin Anton Müller und Laura Untner}%% latex-leseansicht-abspann.tex
%% Abspann für die Leseansicht.
%% Der Schalter \ifkorrekturansicht ist bereits durch den Vorspann gesetzt.

%% latex-abspann.tex
%% Gemeinsamer Abspann für Korrekturansicht und Leseansicht.
%% Setzt den Schalter \ifkorrekturansicht voraus (gesetzt in den
%% einbindenden Dateien latex-korrekturansicht-abspann.tex bzw.
%% latex-leseansicht-abspann.tex).
%% ---------------------------------------------------------------

\normalsize

% Das esempio-Environment wird nur in der Leseansicht benötigt
\ifkorrekturansicht\else
\newenvironment{esempio}[3]%
{
    \vspace{1.5ex}
    \rlap{\underline{#1}}
    \par
    \setlength{\parindent}{0cm}
    \nopagebreak
    \leftskip=#2cm
    \rightskip=#3cm
}
{
    \par
}
\fi

\doendnotes{C}
\bigskip
\vfill

\clearpage

\footnotesize

\ifkorrekturansicht
  \lohead{\textsc{register}}
\fi

% theindex-Environment neu definieren ohne reledmac
\makeatletter
\renewenvironment{theindex}{%
  \ifkorrekturansicht
    \section*{\indexname}%
  \else
    \subsubsection*{Index der erwähnten Entitäten}%
  \fi
  \setlength{\parindent}{0pt}%
  \setlength{\parskip}{0pt plus 0.3pt}%
  \let\item\@idxitem
}{%
  \ifkorrekturansicht\clearpage\fi
}
\makeatother

\IfFileExists{\jobname-pw.ind}{\input{\jobname-pw.ind}}{}

% Quellenangabe nur in der Leseansicht
\ifkorrekturansicht\else
% Fallback-Definitionen, falls die .tex-Datei \titel etc. nicht gesetzt hat
\providecommand{\titel}{}
\providecommand{\editorInnen}{}
\providecommand{\dateiname}{\jobname}

\vspace{3cm}

\vfill

\footnotesize
\textsc{Quelle}: \titel. Herausgegeben von {\editorInnen}. In: \emph{Arthur Schnitzler: Briefwechsel mit Autorinnen und Autoren}.
 Digitale Edition, https://schnitzler-briefe.acdh.oeaw.ac.at/{\dateiname}.html (Stand \today)
\fi

\end{document}


