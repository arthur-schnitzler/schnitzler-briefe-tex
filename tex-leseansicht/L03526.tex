%% latex-korrekturansicht-vorspann.tex
%% Vorspann für die Korrekturansicht.
%% Lädt die gemeinsame Datei latex-vorspann.tex mit gesetztem Schalter.

\newif\ifkorrekturansicht
\korrekturansichttrue

\input{../tex-inputs/latex-vorspann}


\section[ Paul Goldmann an Olga Gussmann, 9. 3. {[}1901{]}]{L03526 Paul Goldmann an Olga Gussmann, 9. 3. {[}1901{]}}
\nopagebreak\mylabel{L03526v}
\rehead{ }\normalsize\beginnumbering\briefempfaengerindex{Schnitzler, Olga@\textsc{Schnitzler, Olga}!zzzGoldmann, Paul@\emph{von Paul Goldmann}!1901-03-091@{9. 3. {[}1901{]}}|(be}
\toendnotes[C]{\smallbreak\pagebreak[2]}\Standort{DLA, A:Schnitzler, HS.NZ85.1.5247.}
\physDesc{Brief, 1 Blatt, 4 Seiten, 1536 Zeichen
\newline{}Handschrift: blaue Tinte, deutsche Kurrent
\newline{}Ordnung: 1) mit Bleistift von Arthur Schnitzler das
                                 Jahr »1901« vermerkt  2) mit rotem Buntstift von Arthur Schnitzler den
                                 ersten Absatz fast vollständig unterstrichen und mit »\textsc{Marionetten\pwindex{Marionetten. Drei Einakter@\emph{Marionetten. Drei Einakter}|pw}}« annotiert sowie eine weitere Unterstreichung}\toendnotes[C]{\smallbreak}
\pstart
           \raggedleft{}{\pb}\textcolor{gray}{\textbf{DESSAUERSTRASSE 19\oindex{Dessauer Strasse@\textbf{Dessauer Straße}, \emph{Straße (K.STR)}|pw}}}\pend
           
\pstart
           Berlin\oindex{Berlin@\textbf{Berlin}, \emph{P.PPLC}|pw}, 9. März.\pend
           
\pstart\center{}Liebes Fräulein \textsc{Olga},\pend\vspace{0.5em}
\pstart
           \textsc{Dr. Schnitzlers}{ }\label{K_L03526-1v}\edtext{Stück\pwindex{Zum grossen Wurstel. Burleske in einem Akt@\emph{Zum großen Wurstel. Burleske in einem Akt}|pwv}}{\lemma{\textnormal{\emph{Stück}}}\Cendnote{\textnormal{\emph{Zum großen Wurstel}\pwindex{Zum grossen Wurstel. Burleske in einem Akt@\emph{Zum großen Wurstel. Burleske in einem Akt}|pwk} aus dem \emph{Marionetten}\pwindex{Marionetten. Drei Einakter@\emph{Marionetten. Drei Einakter}|pwk}-Zyklus, am 8. 3. 1901 am Berlin\oindex{Berlin@\textbf{Berlin}, \emph{P.PPLC}|pwk}er \emph{Überbrettl}\orgindex{Ueberbrettl@Überbrettl|pwk}
                  uraufgeführt}}}\label{K_L03526-1} kam infolge unzureichender Darſtellung nicht zur rechten
               Wirkung. Auch hatte man die Unverſchämtheit und Taktloſigkeit, es ganz zuletzt, um
                  \strikeout{\textcolor{gray}{½}}{ }½ 11 Uhr Abends, nachdem das Publikum bereits durch ein überlanges
               Programm ermüdet war, aufzuführen.\pend
           
\pstart
           \textsc{Dr. Schnitzlers}{ }\label{K_L03526-2v}\edtext{Anweſenheit}{\lemma{\textnormal{\emph{Anweſenheit}}}\Cendnote{\textnormal{Schnitzler war zwischen 3. 3. 1901 und 10. 3. 1901 in Berlin\oindex{Berlin@\textbf{Berlin}, \emph{P.PPLC}|pwk}. Goldmann\pwindex{Goldmann, Paul 31.01.1865 – 25.09.1935@\textsc{Goldmann, Paul} (31.01.1865 – 25.09.1935), \emph{Schriftsteller/Schriftstellerin, Journalist/Journalistin}|pwk} traf er nachweislich am 6. 3. 1901, 7. 3. 1901, 8. 3. 1901 und 10. 3. 1901.}}}\label{K_L03526-2}{ }\strikeout{h\textcolor{gray}{ier}} thut mir ſehr wohl, und ich werde mich nachher nur umſo einſamer fühlen.\pend
           
\pstart
           Ich gratulire Ihnen zu Ihren \label{K_L03526-3v}\edtext{ſchauſpieleriſchen Erfolgen}{\lemma{\textnormal{\emph{ſchauſpieleriſchen Erfolgen}}}\Cendnote{\textnormal{Gussmann\pwindex{Schnitzler, Olga 17.01.1882 – 13.01.1970@\textsc{Schnitzler, Olga} (17.01.1882 – 13.01.1970), \emph{Schauspieler/Schauspielerin, Sänger/Sängerin}|pwk} studierte Schauspiel am \emph{Konservatorium}\orgindex{Konservatorium der Gesellschaft der Musikfreunde@Konservatorium der Gesellschaft der Musikfreunde|pwk}.}}}\label{K_L03526-3}, von denen Sie mir mit
               ſo überzeugender Beredſamkeit berichten. {\pb}Selbſtverſtändlich werde ich bei \label{K_L03526-4v}\edtext{\textsc{Lindau\pwindex{Lindau, Paul 03.06.1839 – 31.01.1919@\textsc{Lindau, Paul} (03.06.1839 – 31.01.1919), \emph{Schriftsteller/Schriftstellerin, Kritiker/Kritikerin, Theaterleiter/Theaterleiterin}|pw}}}{\lemma{\textnormal{\emph{Lindau}}}\Cendnote{\textnormal{Paul Lindau\pwindex{Lindau, Paul 03.06.1839 – 31.01.1919@\textsc{Lindau, Paul} (03.06.1839 – 31.01.1919), \emph{Schriftsteller/Schriftstellerin, Kritiker/Kritikerin, Theaterleiter/Theaterleiterin}|pwk} leitete das \emph{Berliner Theater}\orgindex{Berliner Theater@Berliner Theater|pwk}. Siehe auch A. S.: \emph{Tagebuch}, 3. 8. 1901 und Paul Goldmann an Arthur Schnitzler, 18. 2. [1901].}}}\label{K_L03526-4}, ſoweit es in meinen
               ſchwachen Kräften ſteht, Ihnen behilflich ſein.\pend
           
\pstart
           Zerbrechen Sie ſich nicht den Kopf über das Künftige. Erſtens nützt es doch nichts,
               und zweitens kommt das Künftige ſchon von ſelbſt, wenn man jung iſt und Talent
               hat.\pend
           
\pstart
           Ich würde mich freuen, wenn Sie nach Berlin\oindex{Berlin@\textbf{Berlin}, \emph{P.PPLC}|pw}
               kämen. Dann hätte auch ich »doch wenigſtens eine bekannte Seele in der Stadt\oindex{Berlin@\textbf{Berlin}, \emph{P.PPLC}|pwv}« (wie Sie ſich in Bezug
               auf mich ausdrücken).\pend
           
\pstart
           {\pb}Hoffentlich ſind Sie wieder in guter \label{K_L03526-5v}\edtext{Stimmung}{\lemma{\textnormal{\emph{Stimmung}}}\Cendnote{\textnormal{Womöglich handelt es sich an dieser Stelle um eine Anspielung auf Olga Gussmanns\pwindex{Schnitzler, Olga 17.01.1882 – 13.01.1970@\textsc{Schnitzler, Olga} (17.01.1882 – 13.01.1970), \emph{Schauspieler/Schauspielerin, Sänger/Sängerin}|pwk} Eifersucht, die sie Schnitzler gegenüber, wie dem \emph{Tagebuch}\pwindex{Tagebuch@\emph{Tagebuch}|pwk} zu entnehmen ist, in dieser Zeit mehrfach äußerte.}}}\label{K_L03526-5}, wenn
               dieſer Brief ankommt. Iſt das Leben wirklich ſo bitter? Ich finde aber, alle
               Bitterkeit macht auch nichts, wenn es \strikeout{\textcolor{gray}{richtig}}{ }\introOben{}nur\introOben{} hier und da einen ſüßen Schluck gibt. Nur ganz ohne \strikeout{\textcolor{gray}{Schluck}} ſüßen Schluck iſt es ſchwer zu tragen.\pend
           
\pstart
           Ihr \label{K_L03526-6v}\edtext{Bild\pwindex{Portraitfoto von Olga Gussmann]@\emph{[Portraitfoto von Olga Gussmann]}|pwv}}{\lemma{\textnormal{\emph{Bild}}}\Cendnote{\textnormal{Siehe Paul Goldmann an Olga Gussmann, 10. 5. [1901].
               }}}\label{K_L03526-6} ſoll willkommen ſein.\pend
           
\pstart
           Ich habe Ihnen lange nicht geantwortet, weil ich wenig Zeit zum Schreiben habe und
               weil – weil – weil ich nicht recht wußte, {\pb}was ich
               Ihnen antworten ſollte.\pend
           
\pstart
           Grüßen Sie Ihr Schweſterlein\pwindex{Steinrueck, Elisabeth 19.11.1885 – 07.04.1920@\textsc{Steinrück, Elisabeth} (19.11.1885 – 07.04.1920)|pwv} und ſeien Sie ſelbſt recht herzlich gegrüßt von {\\[\baselineskip]}Ihrem
               ergebenen {\\[\baselineskip]}\spacefill\mbox{Dr. Paul Goldmann}\pend
           \leftskip=0em{}
\pstart
           \noindent{}Grüße an Herrn \label{K_L03526-7v}\edtext{\textsc{Paul\pwindex{Marx, Paul 21.07.1879 – 1956-10-30@\textsc{Marx, Paul} (21.07.1879 – 1956-10-30), \emph{Regisseur/Regisseurin, Schauspieler/Schauspielerin}|pw}}}{\lemma{\textnormal{\emph{Paul}}}\Cendnote{\textnormal{Paul Marx\pwindex{Marx, Paul 21.07.1879 – 1956-10-30@\textsc{Marx, Paul} (21.07.1879 – 1956-10-30), \emph{Regisseur/Regisseurin, Schauspieler/Schauspielerin}|pwk} war zwischen 1900 und 1903 der Partner von Olgas\pwindex{Schnitzler, Olga 17.01.1882 – 13.01.1970@\textsc{Schnitzler, Olga} (17.01.1882 – 13.01.1970), \emph{Schauspieler/Schauspielerin, Sänger/Sängerin}|pwk} Schwester Elisabeth Gussmann\pwindex{Steinrueck, Elisabeth 19.11.1885 – 07.04.1920@\textsc{Steinrück, Elisabeth} (19.11.1885 – 07.04.1920)|pwk} und, wie Olga und Elisabeth\pwindex{Schnitzler, Olga 17.01.1882 – 13.01.1970@\textsc{Schnitzler, Olga} (17.01.1882 – 13.01.1970), \emph{Schauspieler/Schauspielerin, Sänger/Sängerin}|pwk}\pwindex{Steinrueck, Elisabeth 19.11.1885 – 07.04.1920@\textsc{Steinrück, Elisabeth} (19.11.1885 – 07.04.1920)|pwk}, Schüler am \emph{Konservatorium}\orgindex{Konservatorium der Gesellschaft der Musikfreunde@Konservatorium der Gesellschaft der Musikfreunde|pwk}.}}}\label{K_L03526-7}!\pend
           \selectlanguage{ngerman}\endnumbering\briefempfaengerindex{Schnitzler, Olga@\textsc{Schnitzler, Olga}!zzzGoldmann, Paul@\emph{von Paul Goldmann}!1901-03-091@{9. 3. {[}1901{]}}|)be}\mylabel{L03526h}  \normalsize

\doendnotes{C}
\bigskip
\vfill

\clearpage

\footnotesize

\lohead{\textsc{register}}

% Definiere theindex-Environment komplett neu ohne reledmac
\makeatletter
\renewenvironment{theindex}{%
  \section*{\indexname}%
  \setlength{\parindent}{0pt}%
  \setlength{\parskip}{0pt plus 0.3pt}%
  \let\item\@idxitem
}{%
  \clearpage
}
\makeatother

\IfFileExists{\jobname-pw.ind}{\input{\jobname-pw.ind}}{}

\end{document}

      