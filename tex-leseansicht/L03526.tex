%% latex-leseansicht-vorspann.tex
%% Vorspann für die Leseansicht.
%% Lädt die gemeinsame Datei latex-vorspann.tex mit nicht gesetztem Schalter.

\newif\ifkorrekturansicht
\korrekturansichtfalse

\input{../tex-inputs/latex-vorspann}

\begin{center}
            \textcolor{red}{ENTWURF, NICHT FERTIG KORRIGIERT}
                      \end{center}
            
         
         \renewcommand{\erwaehntePersonen}{Personen: Olga Schnitzler}
         \renewcommand{\erwaehnteOrte}{Orte: Berlin, Wien}
         \renewcommand{\erwaehnteWerke}{}
               \section[ Paul Goldmann an Olga Gussmann, 9. 3. {[}1901{]}]{ Paul Goldmann an Olga Gussmann, 9. 3. {[}1901{]}}\nopagebreak\mylabel{v}\rehead{ }\begin{ledgroupsized}[t]{13cm}\normalsize\beginnumbering \toendnotes[C]{\smallbreak\pagebreak[2]} \Standort{DLA, A:Schnitzler, HS.1985.1.5247.}
\physDesc{,  Blätter,  Seiten
\newline{}Handschrift: , deutsche Kurrent}\pstart
           \noindent{}\raggedleft{}{\pb}\textcolor{gray}{\textbf{DESSAUERSTRASSE 19\oindex{XXXX Ortsangabe fehlt|pw}}}\pend
           \pstart
           Berlin\oindex{Berlin@\textbf{Berlin}|pw}, 9. März.\pend
           \pstart\center{}Liebes Fräulein \textsc{Olga},\pend\pstart
           \textsc{Dr. Schnitzler\textcolor{red}{\textsuperscript{\textbf{KEY}}}s}Stück\textcolor{red}{\textsuperscript{\textbf{KEY}}} kam infolge unzureichender Darſtellung nicht zur
               rechten Wirkung. Auch hatte man die Unverſchämtheit und Taktloſigkeit, es ganz
               zuletzt, um \strikeout{\textcolor{gray}{1/2}}1/2 11Uhr Abends, nachdem das Publikum bereits durch ein überlanges Programm
               ermüdet war, aufzuführen.\pend
           \pstart
           \textsc{Dr. Schnitzler\textcolor{red}{\textsuperscript{\textbf{KEY}}}s} Anweſenheit
                  \strikeout{hX} thut mir ſehr wohl, und ich werde mich
               nachher nur umſo einſamer fühlen.\pend
           \pstart
           Ich gratulire Ihnen zu ihren ſchauſpieleriſchen Erfolgen, von denen Sie mir mit ſo
               überzeugender Beredſamkeit berichten. {\pb} Selbſtverſtändlich werde ich
               bei \textsc{Lindau\textcolor{red}{\textsuperscript{\textbf{KEY}}}}, ſoweit es in meinen ſchwachen Kräften ſteht, Ihnen behilflich ſein.\pend
           \pstart
           Zerbrechen Sie ſich nicht den Kopf über das Künftige. Erſtens nützt es doch nichts,
               und zweitens kommt das Künftige ſchon von ſelbſt, wenn man jung iſt und Talent hat.\pend
           \pstart
           Ich würde mich freuen, wenn Sie nach Berlin\oindex{Berlin@\textbf{Berlin}|pw}
               kämen. Dann hätte auch ich »doch wenigſtens eine bekannte Seele in der Stadt\textcolor{red}{\textsuperscript{\textbf{KEY}}}« (wie Sie ſich in Bezug auf mich ausdrücken). {\pb}\pend
           \pstart
           Hoffentlich ſind Sie wieder in guter Stimmung, wenn dieſer Brief ankommt. Iſt das
               Leben wirklich ſo bitter? Ich finde aber, alle Bitterkeit macht auch nichts, wenn es
                  \strikeout{X}nur hier und da einen ſüßen Schluck gibt. Nur ganz ohne \strikeout{\textcolor{gray}{feſ}X\textcolor{gray}{k}} ſüßen Schluck iſt es ſchwer zu tragen.\pend
           \pstart
           Ihr Bild ſoll willkommen ſein.\pend
           \pstart
           Ich habe Ihnen lange nicht geantwortet, weil ich wenig Zeit zum Schreiben habe
               und weil – weil – weil ich nicht recht wußte, {\pb} was ich Ihnen antworten ſollte.
               {\\[\baselineskip]}Grüßen Sie Ihr Schweſterlein\textcolor{red}{\textsuperscript{\textbf{KEY}}}\pend
           \leftskip=0em{}\pstart
           {\\[\baselineskip]}und ſeien Sie ſelbſt recht herzlich\pend
           \leftskip=0em{}\pstart
           {\\[\baselineskip]}gegrüßt von\pend
           \leftskip=0em{}\pstart
           {\\[\baselineskip]}Ihrem ergebenen\pend
           \leftskip=0em{}\pstart
           {\\[\baselineskip]}\spacefill\mbox{Dr. Paul Goldmann}\pend
           \leftskip=0em{}\pstart
           \pend
           \pstart
           Grüße an Herrn Paul! \pend
           
         
         \endnumbering\mylabel{h}\end{ledgroupsized}\begin{anhang}\end{anhang}\newcommand{\dateiname}{L03526}\newcommand{\titel}{Paul Goldmann an Olga Gussmann, 9. 3. [1901]}\newcommand{\editorInnen}{Martin Anton Müller und Laura Untner}%% latex-leseansicht-abspann.tex
%% Abspann für die Leseansicht.
%% Der Schalter \ifkorrekturansicht ist bereits durch den Vorspann gesetzt.

%% latex-abspann.tex
%% Gemeinsamer Abspann für Korrekturansicht und Leseansicht.
%% Setzt den Schalter \ifkorrekturansicht voraus (gesetzt in den
%% einbindenden Dateien latex-korrekturansicht-abspann.tex bzw.
%% latex-leseansicht-abspann.tex).
%% ---------------------------------------------------------------

\normalsize

% Das esempio-Environment wird nur in der Leseansicht benötigt
\ifkorrekturansicht\else
\newenvironment{esempio}[3]%
{
    \vspace{1.5ex}
    \rlap{\underline{#1}}
    \par
    \setlength{\parindent}{0cm}
    \nopagebreak
    \leftskip=#2cm
    \rightskip=#3cm
}
{
    \par
}
\fi

\doendnotes{C}
\bigskip
\vfill

\clearpage

\footnotesize

\ifkorrekturansicht
  \lohead{\textsc{register}}
\fi

% theindex-Environment neu definieren ohne reledmac
\makeatletter
\renewenvironment{theindex}{%
  \ifkorrekturansicht
    \section*{\indexname}%
  \else
    \subsubsection*{Index der erwähnten Entitäten}%
  \fi
  \setlength{\parindent}{0pt}%
  \setlength{\parskip}{0pt plus 0.3pt}%
  \let\item\@idxitem
}{%
  \ifkorrekturansicht\clearpage\fi
}
\makeatother

\IfFileExists{\jobname-pw.ind}{\input{\jobname-pw.ind}}{}

% Quellenangabe nur in der Leseansicht
\ifkorrekturansicht\else
% Fallback-Definitionen, falls die .tex-Datei \titel etc. nicht gesetzt hat
\providecommand{\titel}{}
\providecommand{\editorInnen}{}
\providecommand{\dateiname}{\jobname}

\vspace{3cm}

\vfill

\footnotesize
\textsc{Quelle}: \titel. Herausgegeben von {\editorInnen}. In: \emph{Arthur Schnitzler: Briefwechsel mit Autorinnen und Autoren}.
 Digitale Edition, https://schnitzler-briefe.acdh.oeaw.ac.at/{\dateiname}.html (Stand \today)
\fi

\end{document}


      