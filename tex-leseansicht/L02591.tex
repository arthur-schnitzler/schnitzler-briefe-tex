%% latex-leseansicht-vorspann.tex
%% Vorspann für die Leseansicht.
%% Lädt die gemeinsame Datei latex-vorspann.tex mit nicht gesetztem Schalter.

\newif\ifkorrekturansicht
\korrekturansichtfalse

\input{../tex-inputs/latex-vorspann}


\section[Marie Herzfeld an Arthur Schnitzler, 23.\,8.\,1899]{L02591 Marie Herzfeld an Arthur Schnitzler, 23.\,8.\,1899}
\nopagebreak\mylabel{L02591v}
\rehead{ }\normalsize\beginnumbering\briefempfaengerindex{Schnitzler, Arthur@\textsc{Schnitzler, Arthur}!zzzHerzfeld, Marie@\emph{von Marie Herzfeld}!1899-08-232@{23.\,8.\,1899}|(be}
\toendnotes[C]{\smallbreak\pagebreak[2]}
\correspDesc{Versand  durch Marie Herzfeld am 23. 8. 1899 in Hallstätter See
\newline{}Erhalt  durch Arthur Schnitzler im Zeitraum [24. 8. 1899
                  – 28. 8. 1899?] in Wien}\toendnotes[C]{\smallbreak}
\Standort{DLA, A:Schnitzler, HS.1985.1.03436,2.}
\physDesc{Brief, 1 Blatt, 4 Seiten, 1372 Zeichen
\newline{}Handschrift: schwarze Tinte, lateinische Kurrent}\toendnotes[C]{\smallbreak}
\pstart
           \raggedleft{}{\pb}Steg 7 Hallstättersee 4\oindex{Steeg@\textbf{Steeg}|pw}{\\}d. 23. Aug. 1899\pend
           
\pstart\center{}Geehrter Herr Doktor!\pend\vspace{0.5em}
\pstart
           Verzeihen Sie, dass ich mich telegraphisch an Sie wende – ich vermute Sie unter den
               obwaltenden Umständen in \label{K_L02591-1v}\edtext{Ischl\oindex{Bad Ischl@\textbf{Bad Ischl}|pw}}{\lemma{\textnormal{\emph{Ischl}}}\Cendnote{\textnormal{Im August 1899 hielt sich
                     Schnitzler tatsächlich in Bad Ischl\oindex{Bad Ischl@\textbf{Bad Ischl}|pwk} auf, vgl. A. S.: \emph{Tagebuch}, 15. 8. 1899, 19. 8. 1899.
               }}}\label{K_L02591-1} und habe keine Seele dort, die mir sympathisch genug wäre, um sie anzurufen.
               Ich bin seit etwas über 3 Wochen hier\oindex{Steeg@\textbf{Steeg}|pwv}, bin mehreremale gelegen u. war bisher wenig {\pb}wol, dass ich mich zu einem Besuch in Ischl\oindex{Bad Ischl@\textbf{Bad Ischl}|pw}
               nicht aufraffen konnte, ja, eine Ansage bei Freunden daselbst zweimal telegraphisch
               absagen musste. Von unserer verehrten \label{K_L02591-2v}\edtext{Marie Schey\pwindex{Schey, Marie 8.\,5.\,1821 Nagykanizsa – 22.\,8.\,1899 Bad Ischl@\textsc{Schey, Marie} (8.\,5.\,1821 Nagykanizsa – 22.\,8.\,1899 Bad Ischl)|pw}}{\lemma{\textnormal{\emph{Marie Schey}}}\Cendnote{\textnormal{Marie Schey\pwindex{Schey, Marie 8.\,5.\,1821 Nagykanizsa – 22.\,8.\,1899 Bad Ischl@\textsc{Schey, Marie} (8.\,5.\,1821 Nagykanizsa – 22.\,8.\,1899 Bad Ischl)|pwk} war eine angeheiratete
                  Großtante von Schnitzler. Sie starb am 22. 8. 1899.}}}\label{K_L02591-2}
               wusste ich seit Monaten \uline{gar} nichts, hatte sie vor
               ihrer Abreise nicht mehr sehen können, schreibe ihr auch sonst nicht. Da ich aber
               auch etwas von ihr wissen wollte, {\pb}schrieb ich an sie
               vorgestern einen Brief voll von meinen, doch eigentlich nicht \uline{tief}gehenden Leiden u. erhalte als Antwort folgende »\label{K_L02591-3v}\edtext{\begin{otherlanguage}{english}sneering words\end{otherlanguage}}{\lemma{\textnormal{\emph{sneering words}}}\Cendnote{\textnormal{englisch: spöttische Worte}}}\label{K_L02591-3}« von
               Herrn \label{K_L02591-4v}\edtext{Al. Spitzer\pwindex{Spitzer, Alfred @\textsc{Spitzer, Alfred}, \emph{Kaufmann}|pw}}{\lemma{\textnormal{\emph{Al. Spitzer}}}\Cendnote{\textnormal{Die \emph{Ischler Cur-Liste}\pwindex{Ischler Cur-Liste@\emph{Ischler Cur-Liste}|pwk} beschreibt ihn als »Kaufmann, Ungarn\oindex{Ungarn@\textbf{Ungarn}|pw}« (Nr. 33, 8. 8. 1899, S. 8).}}}\label{K_L02591-4}: »Spät
               erkundigen Sie sich um Tante Marie\pwindex{Schey, Marie 8.\,5.\,1821 Nagykanizsa – 22.\,8.\,1899 Bad Ischl@\textsc{Schey, Marie} (8.\,5.\,1821 Nagykanizsa – 22.\,8.\,1899 Bad Ischl)|pw}; sie liegt
               in Agonie.« Stellen Sie sich mein Entsetzen vor, da ich von nichts wusste. Mein
               erster Gedanke war: hinüberfahren. Da ich {\pb}jedoch
               keinesfalls mich einer Beleidigung von Seite der Menschen aussetzen möchte, die sich
               als allein berechtigt ansehen, die Umgebung der mir theuern Frau\pwindex{Schey, Marie 8.\,5.\,1821 Nagykanizsa – 22.\,8.\,1899 Bad Ischl@\textsc{Schey, Marie} (8.\,5.\,1821 Nagykanizsa – 22.\,8.\,1899 Bad Ischl)|pwv} zu bilden u. denen ich seit Jahren
               ausgewichen bin, so bleibt mir nichts übrig als dies Wort an Sie, das, fürchte ich,
               schon zu spät kommt. Mit vielem Dank für jede Auskunft\pend
           
\pstart
           grüße Sie aufs beste {\\[\baselineskip]}\spacefill\mbox{Marie Herzfeld}\pend
           \leftskip=0em{}\selectlanguage{ngerman}\endnumbering\briefempfaengerindex{Schnitzler, Arthur@\textsc{Schnitzler, Arthur}!zzzHerzfeld, Marie@\emph{von Marie Herzfeld}!1899-08-232@{23.\,8.\,1899}|)be}\mylabel{L02591h}  \newcommand{\dateiname}{L02591}\newcommand{\titel}{Marie Herzfeld an Arthur Schnitzler, 23. 8. 1899}\newcommand{\editorInnen}{Martin Anton Müller und Laura Untner}%% latex-leseansicht-abspann.tex
%% Abspann für die Leseansicht.
%% Der Schalter \ifkorrekturansicht ist bereits durch den Vorspann gesetzt.

%% latex-abspann.tex
%% Gemeinsamer Abspann für Korrekturansicht und Leseansicht.
%% Setzt den Schalter \ifkorrekturansicht voraus (gesetzt in den
%% einbindenden Dateien latex-korrekturansicht-abspann.tex bzw.
%% latex-leseansicht-abspann.tex).
%% ---------------------------------------------------------------

\normalsize

% Das esempio-Environment wird nur in der Leseansicht benötigt
\ifkorrekturansicht\else
\newenvironment{esempio}[3]%
{
    \vspace{1.5ex}
    \rlap{\underline{#1}}
    \par
    \setlength{\parindent}{0cm}
    \nopagebreak
    \leftskip=#2cm
    \rightskip=#3cm
}
{
    \par
}
\fi

\doendnotes{C}
\bigskip
\vfill

\clearpage

\footnotesize

\ifkorrekturansicht
  \lohead{\textsc{register}}
\fi

% theindex-Environment neu definieren ohne reledmac
\makeatletter
\renewenvironment{theindex}{%
  \ifkorrekturansicht
    \section*{\indexname}%
  \else
    \subsubsection*{Index der erwähnten Entitäten}%
  \fi
  \setlength{\parindent}{0pt}%
  \setlength{\parskip}{0pt plus 0.3pt}%
  \let\item\@idxitem
}{%
  \ifkorrekturansicht\clearpage\fi
}
\makeatother

\IfFileExists{\jobname-pw.ind}{\input{\jobname-pw.ind}}{}

% Quellenangabe nur in der Leseansicht
\ifkorrekturansicht\else
% Fallback-Definitionen, falls die .tex-Datei \titel etc. nicht gesetzt hat
\providecommand{\titel}{}
\providecommand{\editorInnen}{}
\providecommand{\dateiname}{\jobname}

\vspace{3cm}

\vfill

\footnotesize
\textsc{Quelle}: \titel. Herausgegeben von {\editorInnen}. In: \emph{Arthur Schnitzler: Briefwechsel mit Autorinnen und Autoren}.
 Digitale Edition, https://schnitzler-briefe.acdh.oeaw.ac.at/{\dateiname}.html (Stand \today)
\fi

\end{document}


