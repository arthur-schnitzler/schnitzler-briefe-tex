%% latex-leseansicht-vorspann.tex
%% Vorspann für die Leseansicht.
%% Lädt die gemeinsame Datei latex-vorspann.tex mit nicht gesetztem Schalter.

\newif\ifkorrekturansicht
\korrekturansichtfalse

\input{../tex-inputs/latex-vorspann}


         
         \newcommand{\erwaehntePersonen}{Personen: Marie Schey, Alfred Spitzer}
         \newcommand{\erwaehnteOrte}{Orte: Bad Ischl, Hallstätter See, Steeg, Ungarn, Wien}
         \newcommand{\erwaehnteWerke}{Werke: Ischler Cur-Liste}
               \section[Marie Herzfeld an Arthur Schnitzler, 23. 8. 1899]{ Marie Herzfeld an Arthur Schnitzler, 23. 8. 1899}\nopagebreak\mylabel{v}\rehead{ }\begin{ledgroupsized}[t]{13cm}\normalsize\beginnumbering \toendnotes[C]{\smallbreak\pagebreak[2]} \Standort{DLA, A:Schnitzler, HS.1985.1.03436,2.}
\physDesc{Brief, 1 Blatt, 4 Seiten
\newline{}Handschrift: schwarze Tinte, lateinische Kurrent}\toendnotes[C]{\smallbreak}\pstart
           \raggedleft{}{\pb}Steg 7 Hallstättersee 4\oindex{Steeg@\textbf{Steeg}|pw}{\\}d.
                     23. Aug. 1899\pend
           \pstart\center{}Geehrter Herr Doktor!\pend\pstart
           Verzeihen Sie, dass ich mich telegraphisch an Sie wende – ich vermute Sie unter den
               obwaltenden Umständen in \label{K_L02591-1v}\edtext{Ischl\oindex{Bad Ischl@\textbf{Bad Ischl}|pw}}{\lemma{\textnormal{\emph{Ischl}}}\Cendnote{\textnormal{Im
                     August 1899 hielt sich Schnitzler\pwindex{Schnitzler, Arthur 15.05.1862 – 21.10.1931@\textsc{Schnitzler, Arthur} (15.05.1862 – 21.10.1931), \emph{Schriftsteller, Mediziner}|pwk} tatsächlich in Bad Ischl\oindex{Bad Ischl@\textbf{Bad Ischl}|pwk} auf.
                     vgl. A. S.: \emph{Tagebuch}, 15. 8. 1899, 19. 8. 1899}}}\label{K_L02591-1h} und
               habe keine Seele dort, die mir sympathisch genug wäre, um sie anzurufen. Ich bin seit
               etwas über 3 Wochen hier\oindex{Steeg@\textbf{Steeg}|pwv}, bin
               mehreremale gelegen u. war bisher wenig {\pb}wol, dass ich
               mich zu einem Besuch in Ischl\oindex{Bad Ischl@\textbf{Bad Ischl}|pw} nicht aufraffen
               konnte, ja, eine Ansage bei Freunden daselbst zweimal telegraphisch absagen musste.
               Von unserer verehrten \label{K_L02591-13v}\edtext{Marie Schey\pwindex{Schey, Marie 08.05.1821 – 22.08.1899@\textsc{Schey, Marie} (08.05.1821 – 22.08.1899)|pw}}{\lemma{\textnormal{\emph{Marie Schey}}}\Cendnote{\textnormal{Marie Schey\pwindex{Schey, Marie 08.05.1821 – 22.08.1899@\textsc{Schey, Marie} (08.05.1821 – 22.08.1899)|pwk} war eine
                  angeheiratete Großtante von Schnitzler\pwindex{Schnitzler, Arthur 15.05.1862 – 21.10.1931@\textsc{Schnitzler, Arthur} (15.05.1862 – 21.10.1931), \emph{Schriftsteller, Mediziner}|pwk}. Sie
                  starb am 22. 8. 1899.}}}\label{K_L02591-13h} wusste ich seit Monaten \uline{gar} nichts,
               hatte sie vor ihrer Abreise nicht mehr sehen können, schreibe ihr auch sonst nicht.
               Da ich aber auch etwas von ihr wissen wollte, {\pb}schrieb ich
               an sie vorgestern einen Brief voll von meinen, doch eigentlich nicht \uline{tief}gehenden Leiden u. erhalte als Antwort folgende
                  »\label{K_L02591-44v}\edtext{\begin{otherlanguage}{english}sneering words\end{otherlanguage}}{\lemma{\textnormal{\emph{sneering words}}}\Cendnote{\textnormal{englisch:
                  spöttische Worte}}}\label{K_L02591-44h}« von Herrn \label{K_L02591-5v}\edtext{Al. Spitzer\pwindex{Spitzer, Alfred @\textsc{Spitzer, Alfred}, \emph{Kaufmann}|pw}}{\lemma{\textnormal{\emph{Al. Spitzer}}}\Cendnote{\textnormal{Die \emph{Ischler Cur-Liste}\pwindex{?? Werk@Nicht ermittelte Verfasserinnen und Verfasser!Ischler Cur-Liste1842 – 1938@\emph{Ischler Cur-Liste} {[}1842 – 1938{]}|pwk}
                  beschreibt ihn als »Kaufmann, Ungarn\oindex{Ungarn@\textbf{Ungarn}|pw}«. (Nr. 33, 8. 8. 1899,
                     S. 8.)}}}\label{K_L02591-5h}: »Spät erkundigen Sie sich um Tante Marie\pwindex{Schey, Marie 08.05.1821 – 22.08.1899@\textsc{Schey, Marie} (08.05.1821 – 22.08.1899)|pw}; sie liegt in Agonie.« Stellen Sie sich mein
               Entsetzen vor, da ich von nichts wusste. Mein erster Gedanke war: hinüberfahren. Da
               ich {\pb}jedoch keinesfalls mich einer Beleidigung von Seite
               der Menschen aussetzen möchte, die sich als allein berechtigt ansehen, die Umgebung
               der mir theuern Frau\pwindex{Schey, Marie 08.05.1821 – 22.08.1899@\textsc{Schey, Marie} (08.05.1821 – 22.08.1899)|pwv} zu bilden
               u. denen ich seit Jahren ausgewichen bin, so bleibt mir nichts übrig als dies Wort an
               Sie, das, fürchte ich, schon zu spät kommt. Mit vielem Dank für jede
               Auskunft\pend
           \pstart
           grüße Sie aufs beste {\\[\baselineskip]}\spacefill\mbox{Marie Herzfeld}\pend
           \leftskip=0em{}
         
         \endnumbering\mylabel{h}\end{ledgroupsized}  \newcommand{\dateiname}{L02591}\newcommand{\titel}{Marie Herzfeld an Arthur Schnitzler, 23. 8. 1899}\newcommand{\editorInnen}{Martin Anton Müller und Laura Untner}%% latex-leseansicht-abspann.tex
%% Abspann für die Leseansicht.
%% Der Schalter \ifkorrekturansicht ist bereits durch den Vorspann gesetzt.

%% latex-abspann.tex
%% Gemeinsamer Abspann für Korrekturansicht und Leseansicht.
%% Setzt den Schalter \ifkorrekturansicht voraus (gesetzt in den
%% einbindenden Dateien latex-korrekturansicht-abspann.tex bzw.
%% latex-leseansicht-abspann.tex).
%% ---------------------------------------------------------------

\normalsize

% Das esempio-Environment wird nur in der Leseansicht benötigt
\ifkorrekturansicht\else
\newenvironment{esempio}[3]%
{
    \vspace{1.5ex}
    \rlap{\underline{#1}}
    \par
    \setlength{\parindent}{0cm}
    \nopagebreak
    \leftskip=#2cm
    \rightskip=#3cm
}
{
    \par
}
\fi

\doendnotes{C}
\bigskip
\vfill

\clearpage

\footnotesize

\ifkorrekturansicht
  \lohead{\textsc{register}}
\fi

% theindex-Environment neu definieren ohne reledmac
\makeatletter
\renewenvironment{theindex}{%
  \ifkorrekturansicht
    \section*{\indexname}%
  \else
    \subsubsection*{Index der erwähnten Entitäten}%
  \fi
  \setlength{\parindent}{0pt}%
  \setlength{\parskip}{0pt plus 0.3pt}%
  \let\item\@idxitem
}{%
  \ifkorrekturansicht\clearpage\fi
}
\makeatother

\IfFileExists{\jobname-pw.ind}{\input{\jobname-pw.ind}}{}

% Quellenangabe nur in der Leseansicht
\ifkorrekturansicht\else
% Fallback-Definitionen, falls die .tex-Datei \titel etc. nicht gesetzt hat
\providecommand{\titel}{}
\providecommand{\editorInnen}{}
\providecommand{\dateiname}{\jobname}

\vspace{3cm}

\vfill

\footnotesize
\textsc{Quelle}: \titel. Herausgegeben von {\editorInnen}. In: \emph{Arthur Schnitzler: Briefwechsel mit Autorinnen und Autoren}.
 Digitale Edition, https://schnitzler-briefe.acdh.oeaw.ac.at/{\dateiname}.html (Stand \today)
\fi

\end{document}


      