%% latex-leseansicht-vorspann.tex
%% Vorspann für die Leseansicht.
%% Lädt die gemeinsame Datei latex-vorspann.tex mit nicht gesetztem Schalter.

\newif\ifkorrekturansicht
\korrekturansichtfalse

\input{../tex-inputs/latex-vorspann}


\section[Hermann Bahr an Arthur Schnitzler, 9. 11. 1903]{L01337 Hermann Bahr an Arthur Schnitzler, 9. 11. 1903}
\nopagebreak\mylabel{L01337v}
\rehead{ }\normalsize\beginnumbering\briefempfaengerindex{Schnitzler, Arthur@\textsc{Schnitzler, Arthur}!zzzBahr, Hermann@\emph{von Hermann Bahr}!1903-11-091@{9. 11. 1903}|(be}
\toendnotes[C]{\smallbreak\pagebreak[2]}
\correspDesc{Versand  durch Hermann Bahr am 9. 11. 1903 in Wien
\newline{}Erhalt  durch Arthur Schnitzler im Zeitraum [9. 11. 1903
                  – 13. 11. 1903?] in Wien}\toendnotes[C]{\smallbreak}
\Standort{CUL, Schnitzler, B 5b.}
\physDesc{Brief, 1 Blatt, 2 Seiten, 1336 Zeichen
\newline{}Handschrift: schwarze Tinte, deutsche Kurrent
\newline{}Ordnung: mit Bleistift von unbekannter Hand nummeriert: »102« }
\buchAbdrucke{\weitereDrucke{Hermann Bahr, Arthur Schnitzler: \emph{Briefwechsel, Aufzeichnungen, Dokumente (1891–1931)}. Herausgegeben von Kurt Ifkovits und Martin Anton Müller. Göttingen: \emph{Wallstein} 2018, S. 277–278.} }\toendnotes[C]{\smallbreak}
\pstart
           \raggedleft{}{\pb}9. 11. 03\pend
           
\pstart\center{}Lieber Arthur!\pend\vspace{0.5em}
\pstart
           Ich habe geſtern Dein »Excentric\pwindex{Schnitzler, Arthur 15.\,5.\,1862 Wien – 21.\,10.\,1931 ebd.@\textsc{Schnitzler, Arthur} (15.\,5.\,1862 Wien – 21.\,10.\,1931 ebd.), \emph{Schriftsteller, Mediziner}!Excentric@\strich\emph{Excentric}|pw}« vorgeleſen und
               die Leute haben über das liebenswürdige Fräulein de la Roſière{ }ſo gebrüllt, daß ich
               wirklich bisweilen eine Minute lang warten mußte, bis{ }ſie{ }ſich{ }ſo weit gefaßt hatten,
               mich wieder anzuhören. Die Geſchichte iſt köſtlich und zum Vorleſen ideal. Ich{ }ſchicke Dir das Heft mit derſelben Poſt zurück, ich habe mir die betr. Nummer der
                  \label{K_L01337-1v}\edtext{Jugend\pwindex{Jugend@\emph{Jugend}|pw}}{\lemma{\textnormal{\emph{Jugend}}}\Cendnote{\textnormal{Arthur Schnitzler: \emph{Excentric}\pwindex{Schnitzler, Arthur 15.\,5.\,1862 Wien – 21.\,10.\,1931 ebd.@\textsc{Schnitzler, Arthur} (15.\,5.\,1862 Wien – 21.\,10.\,1931 ebd.), \emph{Schriftsteller, Mediziner}!Excentric@\strich\emph{Excentric}|pwk}. In: \emph{Jugend}\pwindex{Jugend@\emph{Jugend}|pwk},
                     Jg. 7, Nr. 30, [16.] 7. 1902, S. 492–496.}}}\label{K_L01337-1} bereits
               verſchafft.\pend
           
\pstart
           Noch etwas, ganz aufrichtig. Da Du keine Sitze von mir verlangt haſt, habe ich Dir
               keine \substVorne{}\textsuperscript{\textcolor{gray}{×}}\substDazwischen{}g\substHinten{}eſchickt, weil mir das von mir immer so furchtbar aufdringlich vorkommt,
               Jemandem ungebeten Sitze zu{ }ſchicken, der dann am End erſt{ }ſeine Köchin anflehen muß,{ }ſie zu benützen.\pend
           
\pstart
           {\pb}Anbei findeſt Du den \label{K_L01337-2v}\edtext{Rek\textcolor{gray}{o}urs, der am 5. d. der Statthalterei\orgindex{Niederösterreichische Statthalterei@Niederösterreichische Statthalterei|pw} überreicht worden iſt}{\lemma{\textnormal{\emph{Rekours, … ist}}}\Cendnote{\textnormal{Vgl. Schnitzler an Otto P. Schinnerer\pwindex{Schinnerer, Otto Paul 5.\,11.\,1890 Ocheyedan – 7.\,11.\,1942 New York City@\textsc{Schinnerer, Otto Paul} (5.\,11.\,1890 Ocheyedan – 7.\,11.\,1942 New York City), \emph{Literaturwissenschaftler}|pwk}, 6. 2. 1930, in A. S. 
                     \emph{Briefe 1913–1931}, S. 660–664.}}}\label{K_L01337-2}. Er iſt von mir
               mit Burckhard\pwindex{Burckhard, Max Eugen 14.\,7.\,1854 Korneuburg – 16.\,3.\,1912 Wien@\textsc{Burckhard, Max Eugen} (14.\,7.\,1854 Korneuburg – 16.\,3.\,1912 Wien), \emph{Schriftsteller, Rechtswissenschaftler, Theaterleiter}|pw} berathen und dann von dieſem
               verfaßt worden, was aber, nach{ }ſeinem Wunſch, nicht bekannt werden{ }ſoll. Verſuche,
               den Rekurs in irgend eine Wiener\oindex{Wien@\textbf{Wien}, \emph{Verwaltungsgebiet}|pw} Zeitung zu
               bringen,{ }ſind durchaus misglückt. Überlege, ob Du ihn eventuell der nächſten Auflage
               des Reigens\pwindex{Schnitzler, Arthur 15.\,5.\,1862 Wien – 21.\,10.\,1931 ebd.@\textsc{Schnitzler, Arthur} (15.\,5.\,1862 Wien – 21.\,10.\,1931 ebd.), \emph{Schriftsteller, Mediziner}!Reigen. Zehn Dialoge@\strich\emph{Reigen. Zehn Dialoge}|pw} vordrucken würdeſt. Sag aber nur
               offen Nein, wenn es Dir nicht paßt.\pend
           
\pstart
           \label{K_L01337-3v}\edtext{Salten\pwindex{Salten, Felix 6.\,9.\,1869 Budapest – 8.\,10.\,1945 Zürich@\textsc{Salten, Felix} (6.\,9.\,1869 Budapest – 8.\,10.\,1945 Zürich), \emph{Schriftsteller, Journalist, Chefredakteur}|pw} tuſt Du glaub ich unrecht}{\lemma{\textnormal{\emph{Salten … unrecht}}}\Cendnote{\textnormal{Das dürfte auf ein verlorenes
                  Korrespondenzstück hinweisen, in dem Schnitzler seine Verärgerung über Saltens\pwindex{Salten, Felix 6.\,9.\,1869 Budapest – 8.\,10.\,1945 Zürich@\textsc{Salten, Felix} (6.\,9.\,1869 Budapest – 8.\,10.\,1945 Zürich), \emph{Schriftsteller, Journalist, Chefredakteur}|pwk}{ } Feuilleton \emph{Arthur Schnitzler und sein »Reigen«}\pwindex{Salten, Felix 6.\,9.\,1869 Budapest – 8.\,10.\,1945 Zürich@\textsc{Salten, Felix} (6.\,9.\,1869 Budapest – 8.\,10.\,1945 Zürich), \emph{Schriftsteller, Journalist, Chefredakteur}!Arthur Schnitzler und sein »Reigen«@\strich\emph{Arthur Schnitzler und sein »Reigen«}|pwk} zum Ausdruck gebracht
                  hat (Felix Salten\pwindex{Salten, Felix 6.\,9.\,1869 Budapest – 8.\,10.\,1945 Zürich@\textsc{Salten, Felix} (6.\,9.\,1869 Budapest – 8.\,10.\,1945 Zürich), \emph{Schriftsteller, Journalist, Chefredakteur}|pwk}: \emph{Arthur Schnitzler und sein »Reigen«}\pwindex{Salten, Felix 6.\,9.\,1869 Budapest – 8.\,10.\,1945 Zürich@\textsc{Salten, Felix} (6.\,9.\,1869 Budapest – 8.\,10.\,1945 Zürich), \emph{Schriftsteller, Journalist, Chefredakteur}!Arthur Schnitzler und sein »Reigen«@\strich\emph{Arthur Schnitzler und sein »Reigen«}|pwk} (In: \emph{Die Zeit}\pwindex{Zeit@\emph{Die Zeit}|pwk}, Jg. 2, Nr. 398, 7. 11. 1903, Morgenblatt, S. 1–2). Vgl. XXXX Auszeichnungsfehler: Dokument L02988 nicht gefunden.}}}\label{K_L01337-3}. Du mußt nur
               doch die für ihn unglaublich heikle und gefährliche Situation bedenken, in der er
               geſchrieben hat. Aber darüber mündlich.\pend
           
\pstart
           Mit den beſten Grüßen an Deine Frau\pwindex{Schnitzler, Olga 17.\,1.\,1882 Wien – 13.\,1.\,1970 Lugano@\textsc{Schnitzler, Olga} (17.\,1.\,1882 Wien – 13.\,1.\,1970 Lugano), \emph{Schauspielerin, Sängerin}|pwv}{\\[\baselineskip]}herzlichſt Dein{\\[\baselineskip]}\spacefill\mbox{Hermann}\pend
           \leftskip=0em{}\selectlanguage{ngerman}\endnumbering\briefempfaengerindex{Schnitzler, Arthur@\textsc{Schnitzler, Arthur}!zzzBahr, Hermann@\emph{von Hermann Bahr}!1903-11-091@{9. 11. 1903}|)be}\mylabel{L01337h}  \newcommand{\dateiname}{L01337}\newcommand{\titel}{Hermann Bahr an Arthur Schnitzler, 9. 11. 1903}\newcommand{\editorInnen}{Herausgegeben von Martin Anton Müller}%% latex-leseansicht-abspann.tex
%% Abspann für die Leseansicht.
%% Der Schalter \ifkorrekturansicht ist bereits durch den Vorspann gesetzt.

%% latex-abspann.tex
%% Gemeinsamer Abspann für Korrekturansicht und Leseansicht.
%% Setzt den Schalter \ifkorrekturansicht voraus (gesetzt in den
%% einbindenden Dateien latex-korrekturansicht-abspann.tex bzw.
%% latex-leseansicht-abspann.tex).
%% ---------------------------------------------------------------

\normalsize

% Das esempio-Environment wird nur in der Leseansicht benötigt
\ifkorrekturansicht\else
\newenvironment{esempio}[3]%
{
    \vspace{1.5ex}
    \rlap{\underline{#1}}
    \par
    \setlength{\parindent}{0cm}
    \nopagebreak
    \leftskip=#2cm
    \rightskip=#3cm
}
{
    \par
}
\fi

\doendnotes{C}
\bigskip
\vfill

\clearpage

\footnotesize

\ifkorrekturansicht
  \lohead{\textsc{register}}
\fi

% theindex-Environment neu definieren ohne reledmac
\makeatletter
\renewenvironment{theindex}{%
  \ifkorrekturansicht
    \section*{\indexname}%
  \else
    \subsubsection*{Index der erwähnten Entitäten}%
  \fi
  \setlength{\parindent}{0pt}%
  \setlength{\parskip}{0pt plus 0.3pt}%
  \let\item\@idxitem
}{%
  \ifkorrekturansicht\clearpage\fi
}
\makeatother

\IfFileExists{\jobname-pw.ind}{\input{\jobname-pw.ind}}{}

% Quellenangabe nur in der Leseansicht
\ifkorrekturansicht\else
% Fallback-Definitionen, falls die .tex-Datei \titel etc. nicht gesetzt hat
\providecommand{\titel}{}
\providecommand{\editorInnen}{}
\providecommand{\dateiname}{\jobname}

\vspace{3cm}

\vfill

\footnotesize
\textsc{Quelle}: \titel. Herausgegeben von {\editorInnen}. In: \emph{Arthur Schnitzler: Briefwechsel mit Autorinnen und Autoren}.
 Digitale Edition, https://schnitzler-briefe.acdh.oeaw.ac.at/{\dateiname}.html (Stand \today)
\fi

\end{document}


